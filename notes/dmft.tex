\chapter{Dynamical Mean-Field Theory}\label{ch:dynamical-mean-field-theory}

Source: \citeauthor{Georges_Kotliar_Krauth_Rozenberg_1996} -~\citetitle{Georges_Kotliar_Krauth_Rozenberg_1996}~\cite{Georges_Kotliar_Krauth_Rozenberg_1996}

With help from~\cite{Schüler} and~\cite{Werner} to make it more concise.

Most general non-interacting electronic Hamiltonian in second quantization:
\begin{equation}
    H_0 = \sum_{i, j, \sigma}
\end{equation}
with lattice coordinates \(i, j\) and spin \(\sigma\).

One particle Green's function (many-body object, coming from the Hubbard model):
\begin{equation}
    G(\vb{k}, \iu \omega_n) = \frac{1}{\iu \omega_n + \mu - \epsilon_{\vb{k}} - \Sigma(\vb{k}, \iu \omega_n)}
\end{equation}
with the self energy \(\Sigma(\iu \omega_n)\) coming from the solution of the effect on-site problem:

The Dyson equation
\begin{equation}
    G(\vb{k}, \iu \omega_n) = \left( G_0 (\vb{k}, \iu \omega_n) - \Sigma(\vb{k}, \iu \omega_n)\right)^{-1}
\end{equation}
relates the non-interacting Greens function \(G_0 (\vb{k}, \iu \omega_n)\) and the fully-interacting Greens function \(G (\vb{k}, \iu \omega_n)\) (inversion of a matrix!).
