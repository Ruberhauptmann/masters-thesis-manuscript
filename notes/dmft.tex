\documentclass[../main.tex]{subfiles}
\graphicspath{{\subfix{../images/}}, {\subfix{../}}}

\begin{document}
\chapter{Dynamical Mean-Field Theory}\label{ch:dynamical-mean-field-theory}

\section{Green's Function Formalism}

Following~\cite{bruusManyBodyQuantumTheory2004}

Green's functions: method to encode influence of many-body effects on propagation of particles in a system.

Have different kinds of Green's functions, for example the retarded Green's function:
\begin{equation}
	G^R (\vb{r}\sigma t, \vb{r}^{\prime} \sigma^{\prime} t^{\prime}) = -\iu \Theta(t- t^{\prime}) \braket{ \{c_{\vb{r} \sigma} (t), c_{\vb{r} \sigma}^{\dagger} (t^{\prime})\}}
\end{equation}
They give the amplitude of a particle inserted at point \(\vb{r}^{\prime}\) at time \(t^{\prime}\) to propagate to position \(\vb{r}\) at time \(t\).
For time-independent Hamiltonians and systems in equilibrium, the GFs only depend on time differences:
\begin{equation}
	G^R (\vb{r}\sigma t, \vb{r}^{\prime} \sigma^{\prime} t^{\prime}) = G^R (\vb{r} \sigma, \vb{r}^{\prime} \sigma^{\prime}, t - t^{\prime})
\end{equation}
So we can take \(t^{\prime} = 0\) and consider \(t\) as the only free variable:
\begin{equation}
	G^R (\vb{r}\sigma, \vb{r}^{\prime} \sigma^{\prime}, t) = -\iu \Theta(t) \braket{ \{c_{\vb{r} \sigma} (t), c_{\vb{r} \sigma}^{\dagger} (0)\}}
\end{equation}
In a translation invariant system: can use \(\vb{k}\) as a natural basis set:
\begin{equation}
	G^R (\vb{k}, \sigma, \sigma^{\prime} t) = -\iu \Theta(t- t^{\prime}) \braket{ \{c_{\vb{k} \sigma} (t), c_{\vb{k} \sigma^{\prime}}^{\dagger} (0)\}}
\end{equation}
Define Fourier-transform:
\begin{equation}
	G^R (\vb{k}, \sigma, \sigma^{\prime}, \omega) = \int_{-\infty}^{\infty} \odif{t} G^R (\vb{k}, \sigma, \sigma^{\prime} t)
\end{equation}
Can define the spectral function from this:
\begin{equation}
	A(\vb{k} \sigma, \omega) = -2 \Im G^R (\vb{k} \sigma, \omega)
\end{equation}
Looking at the diagonal elements of \(G^R\) here.
The spectral function can be thought of as the energy resolution of a particle with energy \(\omega\).
This mean, for non-interacting systems, the spectral function is a delta-function around the single-particle energies:
\begin{equation}
	A_0 (\vb{k} \sigma, \omega) = 2\pi \delta (\omega - \epsilon_{\vb{k} \sigma})
\end{equation}
For interacting systems this is not true, but \(A\) can still be peaked.

\todo{Show GFs can be related to observables}

Mathematical technique to calculate retarded GFs involves defining GFs on imaginary times \gls{imaginary time}:
\begin{equation}
	t \to -\iu \tau
\end{equation}
where \(\tau\) is real and has the dimension time.
This enables the simultaneous expansion of exponential \(e^{-\beta H}\) coming from the thermodynamic average and \(e^{-\iu H t}\) coming from the time evolution of operators.

Define imaginary time/Matsubara GF \gls{matsubara correlation function}:
\begin{equation}
	\mathcal{C}_{A B} (\tau, 0) = - \Braket{T_{\tau} (A(\tau) B(0))}
\end{equation}
with time-ordering operator in imaginary time:
\begin{equation}
	T_{\tau} (A(\tau) B(\tau^{\prime})) = \Theta(\tau - \tau^{\prime}) A(\tau) B(\tau^{\prime}) \pm \Theta(\tau^{\prime} - \tau) B(\tau^{\prime}) A(\tau)
\end{equation}
so that operators with later `times' go to the left.

Can prove from properties of Matsubara GF, that they are only defined for
\begin{equation}
	-\beta < \tau < \beta
\end{equation}
Due to this, the Fourier transform of the Matsubara GF is defined on discrete values:
\begin{equation}
	\mathcal{C}_{A B} (\iu \omega_n) = \int_{0}^{\beta} \odif{\tau}
\end{equation}
with fermionic/bosonic Matsubara frequencies
\begin{equation}
	\omega_n =
	\begin{cases}
		\frac{2n \pi}{\beta} \, \text{for bosons} \\
		\frac{(2n + 1)\pi}{\beta} \, \text{for fermions}
	\end{cases}
\end{equation}

\todo{How to resolve ambiguity at borders of integral}

It turns out that Matsubara GFs and retarded GFs can be generated from a common function \(\mathcal{C}_{AB} (z)\) that is defined on the entire complex plane except for the real axis.
So we can get the retarded GF \(\mathcal{C}_{AB}^R (\omega)\) by analytic continuation:
\begin{equation}
	\mathcal{C}_{AB}^R (\omega) = \mathcal{C}_{AB} (\iu \omega_n \to \omega + \iu \eta)
\end{equation}
So in particular the extrapolation of the Matsubara GF to zero is proportional to the density of states at the chemical potential.
Gapped: density is zero (Matsubara GF goes to 0), metal: density is finite (Matsubara GF goes to finite value) ~\cite[8.3.4]{bruusManyBodyQuantumTheory2004}.

\todo{single-particle Matsubara GF}

\todo{equations of motion for Matsubara GF}

\section{Perturbation theory, Dyson equation}

\todo{Short introduction to diagrams}

\todo{Self energy}

\todo{Dyson equation}

Dyson equation:
\begin{equation}
	\mathcal{G}_{\sigma} (\vb{k}, \iu \omega_n) = \frac{\mathcal{G}_{\sigma}^0 (\vb{k}, \iu \omega_n)}{1 - \mathcal{G}_{\sigma}^0 (\vb{k}, \iu \omega_n) \Sigma_{\sigma} (\vb{k}, \iu \omega_n)} = \frac{1}{\iu \omega_n - \xi_{\vb{k} - \Sigma_{\sigma} (\vb{k}, \iu \omega_n)}}
\end{equation}


\section{Nambu-Gorkov GF}

Introduction following~\cite[ch. 14.7]{colemanIntroductionManyBodyPhysics2015}

\todo{More general introduction into NG GFs, how they look like, what they describe etc.}

Order parameter can be chosen as the anomalous GF:
\begin{equation}
	\Psi = F^{\mathrm{loc}} (\tau = 0^-)
\end{equation}
or the superconducting gap
\begin{equation}
	\Delta = Z \Sigma^{\mathrm{AN}}
\end{equation}
that can be calculated from the anomalous self-energy \(\Sigma^{\mathrm{AN}}\) and quasiparticle weight \(Z\)
\todo{Sources for these?}

\todo{How to get quasiparticle weight?}

\section{DMFT}

Following \cite{georgesDynamicalMeanfieldTheory1996}.

Most general non-interacting electronic Hamiltonian in second quantization:
\begin{equation}
    H_0 = \sum_{i, j, \sigma}
\end{equation}
with lattice coordinates \(i, j\) and spin \(\sigma\).

One particle Green's function (many-body object, coming from the Hubbard model):
\begin{equation}
    G(\vb{k}, \iu \omega_n) = \frac{1}{\iu \omega_n + \mu - \epsilon_{\vb{k}} - \Sigma(\vb{k}, \iu \omega_n)}
\end{equation}
with the self energy \(\Sigma(\iu \omega_n)\) coming from the solution of the effect on-site problem:

The Dyson equation
\begin{equation}
    G(\vb{k}, \iu \omega_n) = \left( G_0 (\vb{k}, \iu \omega_n) - \Sigma(\vb{k}, \iu \omega_n)\right)^{-1}
\end{equation}
relates the non-interacting Greens function \(G_0 (\vb{k}, \iu \omega_n)\) and the fully-interacting Greens function \(G (\vb{k}, \iu \omega_n)\) (inversion of a matrix!).
\end{document}
