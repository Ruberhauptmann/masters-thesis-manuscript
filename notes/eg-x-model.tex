\chapter{EG-X Model}\label{ch:eg-x-model}

\section{Lattice Structure of Graphene}\label{sec:lattice-structure-of-graphene}

Following~\cite{Yang_Li_Lee_Ng_2018}.

Monolayer graphene forms a hexagonal lattice.
This is formed by two triangular sublattices.
So in the unit cell of the hexagonal actually has two atoms.

Primitive lattice vectors of the hexagonal lattice:
\begin{align}
    \vb{a}_1 &= a \left( 1, \sqrt{3} \right) \\
    \vb{a}_2 &= a \left( 1, -\sqrt{3} \right)
\end{align}
with lattice constant \(a\).

Vectors to the nearest-neighbor \(B_l\) (\(l = 1, 2, 3,\)) from atom \(A_i\):
\begin{align}
    \vb{\delta} =
\end{align}


\section{EG-X Model}\label{sec:eg-x-model}

Graphene lattice and a site X\@.
Without interaction:
\begin{equation}
    H_0 = -t_X \sum_{\langle ij \rangle, \sigma \sigma^{\prime}}
\end{equation}
with:
\begin{itemize}
    \item \(d\) operators on the X atom
    \item \(c\) operators on the graphene
    \item \(t_X\) NN hopping for X
    \item \(t_{Gr}\) NN hopping of Gr
    \item \(V\) hybridization between X and Gr2
\end{itemize}
We can also introduce an onsite Hubbard interaction:
\begin{equation}
    H_{\mathrm{int}}
\end{equation}

\subsection{Band structure of the non-interacting EG-X model}\label{subsec:band-structure-of-the-non-interacting-eg-x-model}

Values used for calculation:
\begin{itemize}
    \item \(t_{\mathrm{Gr}} = 1\)
    \item \(t_{\mathrm{X}} = 0.01\)
\end{itemize}
\(V\) is the control parameter.

