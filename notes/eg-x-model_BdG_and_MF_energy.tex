\documentclass[../main.tex]{subfiles}
\graphicspath{{\subfix{../images/}}, {\subfix{../}}}

\begin{document}

\chapter{EG-X Model with interactions}

\section{BdG}

\subsection{BdG Hamiltonian}

Define sublattice index
\begin{equation}
    \alpha = 1, 2, 3
\end{equation}
with \(1 \hateq \mathrm{Gr}_1, 2 \hateq \mathrm{Gr}_2, 3 \hateq \mathrm{X}\).
Then we can write the non-interacting term as
\begin{equation}
    H_0 = - \sum_{\langle i, j \rangle, \alpha, \beta, \sigma} \left[\mat{t} \right]_{i\alpha,j\beta} c_{i\alpha}^{\dagger} c_{j\beta}
\end{equation}
with the matrix
\begin{equation}
    \mat{t} = \begin{pmatrix}
                  0 & t_{\mathrm{Gr}} & 0 \\
                  t_{\mathrm{Gr}} & 0 & -V \delta_{ij} \\
                  0 & -V \delta_{ij} & t_{\mathrm{X}} \\
    \end{pmatrix}
\end{equation}

Add chemical potential:
\begin{equation}
    -\mu \sum_{i \alpha \sigma} n_{i \alpha \sigma}
\end{equation}

Also write the interaction part with \(\alpha\) (with changed signs compared to Niklas, to keep in line with papers about the attractive Hubbard model):
\begin{equation}
    H_{int} = - \sum_{i \alpha} U_{\alpha} c_{i\alpha \uparrow}^{\dagger} c_{i\alpha \downarrow}^{\dagger} c_{i\alpha \downarrow} c_{i\alpha \uparrow}
\end{equation}
Fourier transformation:
\begin{equation}
    H_{int} = - \frac{1}{N^2} \sum_{\alpha, \vb{k}_{1, 2, 3, 4}} U_{\alpha} e^{\iu (\vb{k}_1 + \vb{k}_4 - \vb{k}_1 - \vb{k}_3) r_{i \alpha}}  c_{\vb{k}_1 \alpha \uparrow}^{\dagger} c_{\vb{k}_3 \alpha \downarrow}^{\dagger} c_{\vb{k}_2 \alpha \downarrow} c_{\vb{k}_4 \alpha \uparrow}
\end{equation}
Impose zero-momentum pairing: \(\vb{k}_1 + \vb{k}_3 = 0\) and \(\vb{k}_2 + \vb{k}_4 = 0\):
\begin{align}
    H_{int} = - \sum_{\alpha, \vb{k}, \vb{k}^{\prime}} U_{\alpha} c_{\vb{k} \alpha \uparrow}^{\dagger} c_{-\vb{k} \alpha \downarrow}^{\dagger} c_{-\vb{k}^{\prime} \alpha \downarrow} c_{\vb{k}^{\prime} \alpha \uparrow}
\end{align}
Mean-field approximation:
\begin{align}
    H_{int} \approx \sum_{\alpha, \vb{k}} (\Delta_{\alpha} c_{\vb{k} \alpha \uparrow}^{\dagger} c_{-\vb{k} \alpha \downarrow}^{\dagger} + \Delta_{\alpha}^* c_{-\vb{k} \alpha \downarrow} c_{\vb{k} \alpha \uparrow})
\end{align}
with
\begin{align}
    \Delta_{\alpha} &= - U_{\alpha} \sum_{\vb{k}^{\prime}} \braket{c_{-\vb{k}^{\prime} \alpha \downarrow} c_{\vb{k}^{\prime} \alpha \uparrow}} \\
    \Delta_{\alpha}^* &= - U_{\alpha} \sum_{\vb{k}^{\prime}} \braket{c_{\vb{k}^{\prime} \alpha \uparrow}^{\dagger} c_{-\vb{k}^{\prime} \alpha \downarrow}^{\dagger}}
\end{align}
This gives the BCS mean field Hamiltonian:
\begin{align}
    H_{BCS} = \sum_{\vb{k} \alpha \beta \sigma} [H_{0, \sigma} (\vb{k})]_{\alpha \beta} c_{\vb{k} \alpha \sigma}^{\dagger} c_{\vb{k} \beta \sigma}
    -\mu \sum_{\vb{k} \alpha \sigma} n_{\vb{k} \alpha \sigma}
    + \sum_{\alpha, \vb{k}} (\Delta_{\alpha} c_{\vb{k} \alpha \uparrow}^{\dagger} c_{-\vb{k} \alpha \downarrow}^{\dagger} + \Delta_{\alpha}^* c_{-\vb{k} \alpha \downarrow} c_{\vb{k} \alpha \uparrow})
\end{align}
with Nambu spinor
\begin{equation}
    \Psi_{\vb{k}} =
    \begin{pmatrix}
        c_{1, \vb{k} \uparrow} \\
        c_{2, \vb{k} \uparrow} \\
        c_{3, \vb{k} \uparrow} \\
        c_{1, -\vb{k} \downarrow}^{\dagger} \\
        c_{2, -\vb{k} \downarrow}^{\dagger} \\
        c_{3, -\vb{k} \downarrow}^{\dagger} \\
    \end{pmatrix}
\end{equation}
we have:
\begin{equation}
    H_{MF} = \sum_{\vb{k}} \Psi_{\vb{k}}^{\dagger} \mathcal{H} (\vb{k}) \Psi_{\vb{k}}
\end{equation}
with
\begin{equation}
    \mathcal{H} (\vb{k}) =
    \begin{pmatrix}
        H_{0, \uparrow} (\vb{k}) - \mu & \Delta \\
        \Delta^{\dagger} & - H_{0, \downarrow}^* (-\vb{k}) + \mu
    \end{pmatrix}
\end{equation}
with \(H_{0, \sigma}\) being the F.T. of the kinetic term and \(\Delta = diag(\Delta_1, \Delta_2, \Delta_3)\).


\subsection{BdG Hamiltonian in band basis}

Use transformation
\begin{equation}
    c_{\vb{k} \alpha \sigma}^{\dagger} = \sum_{n} [\mat{G}]_{\alpha n}^* d_{n \vb{k} \sigma}^{\dagger}
\end{equation}
where the columns are made up of the eigenvectors of \(\mat{H}_{0, \sigma}\) for a given \(\vb{k}\):
\begin{equation}
    \mat{G} = 
    \begin{pmatrix}
        \vb{G}_1 & \vb{G}_2 & \vb{G}_3
    \end{pmatrix}
\end{equation}

with that:
\begin{equation}
    \mat{G}^{\dagger}_{\sigma} (\vb{k}) \mat{H}_{0, \sigma} (\vb{k}) \mat{G}_{\sigma} (\vb{k}) =
    \begin{pmatrix}
        \epsilon_1 & 0 & 0 \\
        0 & \epsilon_2 & 0 \\
        0 & 0 & \epsilon_3
    \end{pmatrix}
\end{equation}
So the kinetic part of the BdG Hamiltonian becomes:
\begin{align}
    &\sum_{\vb{k} \alpha \beta \sigma} [H_{0, \sigma} (\vb{k})]_{\alpha \beta}
    \sum_{n} [\mat{G} (\vb{k})]_{\alpha n}^* d_{n \vb{k} \sigma}^{\dagger}
    \sum_{m} [\mat{G} (\vb{k})]_{\beta m} d_{m \vb{k} \sigma}
    -\mu \sum_{\vb{k} \alpha \sigma} n_{n \vb{k} \sigma} \\
    &= \sum_{m n \vb{k} \sigma} d_{n \vb{k} \sigma}^{\dagger} d_{m \vb{k} \sigma}
   \sum_{\alpha \beta} [\mat{G} (\vb{k})]_{\alpha n}^* [H_{0, \sigma} (\vb{k})]_{\alpha \beta} [\mat{G} (\vb{k})]_{\beta m}
    -\mu \sum_{\vb{k} \alpha \sigma} n_{n \vb{k} \sigma} \\
    &= \sum_{m n \vb{k} \sigma} d_{n \vb{k} \sigma}^{\dagger} d_{m \vb{k} \sigma} \epsilon_{n} \delta_{n m}
    -\mu \sum_{\vb{k} \alpha \sigma} n_{n \vb{k} \sigma} \\
    &= \sum_{n \vb{k} \sigma} \epsilon_{n} d_{n \vb{k} \sigma}^{\dagger} d_{n \vb{k} \sigma}
    -\mu \sum_{\vb{k} \alpha \sigma} n_{n \vb{k} \sigma} \\
    &\eqqcolon \sum_{n \vb{k} \sigma} \xi_{\vb{k}} d_{n \vb{k} \sigma}^{\dagger} d_{n \vb{k} \sigma}
\end{align}
with \(\xi_{\vb{k}} \coloneqq \epsilon_{\vb{k}} - \mu\).
The pairing terms become:
\begin{align}
    \sum_{\vb{k} \alpha} \Delta_{\alpha} c_{\vb{k} \alpha \uparrow}^{\dagger} c_{-\vb{k} \alpha \downarrow}^{\dagger}
    &= \sum_{\vb{k} \alpha} \Delta_{\alpha} \sum_n [\mat{G}_{\uparrow} (\vb{k})]_{\alpha n}^* d_{n \vb{k} \uparrow}^{\dagger} \sum_m [\mat{G}_{\downarrow} (-\vb{k})]_{\beta m}^* d_{m -\vb{k} \downarrow}^{\dagger} \\
    &=
\end{align}

So that:
\begin{equation}
    \mathcal{H} (\vb{k}) =
    \begin{pmatrix}
        \epsilon_{\vb{k}} - \mu & G^{\dagger} \Delta G\\
        G^{\dagger} \Delta^{\dagger} G & -\epsilon_{\vb{k}} + \mu
    \end{pmatrix}
\end{equation}
with
\begin{equation}
    \epsilon_{\vb{k}} =
    \begin{pmatrix}
        \epsilon_1 (\vb{k}) & 0 & 0 \\
        0 & \epsilon_2 (\vb{k}) & 0 \\
        0 & 0 & \epsilon_3 (\vb{k}) \\
    \end{pmatrix}
\end{equation}

Concrete example for transformation of gaps from orbital to band basis at \(\mathrm{K} = \frac{4\pi}{3 a} \begin{pmatrix} 1 \\ 0 \end{pmatrix}\).
There, the non-interacting part becomes simply:
\begin{align}
\mathcal{H}_0 &=
\begin{pmatrix}
    0 & 0 & V \\
    0 & 0 & 0 \\
    V & 0 & 3 t_X
\end{pmatrix}
\end{align}
The eigenvalue problem can be solved e.g.~via sympy:
\begin{equation}
   G =
   \begin{pmatrix}
       \frac{- 3 t_{X} - \sqrt{4 V^{2} + 9 t_{X}^{2}}}{\sqrt{4 V^{2} + \left(3 t_{X} + \sqrt{4 V^{2} + 9 t_{X}^{2}}\right)^{2}}} & 0 & \frac{- 3 t_{X} + \sqrt{4 V^{2} + 9 t_{X}^{2}}}{\sqrt{4 V^{2} + \left(3 t_{X} - \sqrt{4 V^{2} + 9 t_{X}^{2}}\right)^{2}}} \\
       0 & 1 & 0 \\
       \frac{2 V}{\sqrt{4 V^{2} + \left(3 t_{X} + \sqrt{4 V^{2} + 9 t_{X}^{2}}\right)^{2}}} & 0 & \frac{2 V}{\sqrt{4 V^{2} + \left(3 t_{X} - \sqrt{4 V^{2} + 9 t_{X}^{2}}\right)^{2}}}
   \end{pmatrix}
\end{equation}
So for \(V \to 0\):
\begin{equation}
    G =
    \begin{pmatrix}
        -1 & 0 & 0 \\
        0 & 1 & 0 \\
        0 & 0 & 1
    \end{pmatrix}
\end{equation}
but for \(V > 0\), there are off-diagonal elements, e.g.~\(V = 0.1\):
\begin{equation}
    G =
    \begin{pmatrix}
        -0.7578 & 0 & 0.6526 \\
        0 & 1 & 0 \\
        0.6526 & 0 & 0.7578
    \end{pmatrix}
\end{equation}
So the transformation of the gap from orbital to band space reads:
\begin{equation}
    G^{\dagger} \Delta G =
    \begin{pmatrix}
       \frac{3 \Delta_{1} t_{X} - 3 \Delta_{3} t_{X} + \left(\Delta_{1} + \Delta_{3}\right) \sqrt{4 V^{2} + 9 t_{X}^{2}}}{2 \sqrt{4 V^{2} + 9 t_{X}^{2}}} & 0 & \frac{V \left(- \Delta_{1} + \Delta_{3}\right)}{\sqrt{4 V^{2} + 9 t_{X}^{2}}} \\
       0 & \Delta_{2} & 0 \\
       \frac{V \left(- \Delta_{1} + \Delta_{3}\right)}{\sqrt{4 V^{2} + 9 t_{X}^{2}}} & 0 & \frac{- 3 \Delta_{1} t_{X} + 3 \Delta_{3} t_{X} + \left(\Delta_{1} + \Delta_{3}\right) \sqrt{4 V^{2} + 9 t_{X}^{2}}}{2 \sqrt{4 V^{2} + 9 t_{X}^{2}}}
   \end{pmatrix}
\end{equation}
So in particular there is no interband pairing for \(V \to 0\):
\begin{equation}
    G^{\dagger} \Delta G =
    \begin{pmatrix}
        \Delta_1 & 0 & 0 \\
        0 & \Delta_{2} & 0 \\
        0 & 0 & \Delta_3
    \end{pmatrix}
\end{equation}
But for \(V > 0\), there is interband pairing (e.g. \(V = 0.1\)):
\begin{equation}
    G^{\dagger} \Delta G =
    \begin{pmatrix}
        0.5742 \Delta_{1} + 0.4258 \Delta_{3} & 0 & - 0.4945 \Delta_{1} + 0.4945 \Delta_{3} \\
        0 & \Delta_{2} & 0 \\
        - 0.4945 \Delta_{1} + 0.4945 \Delta_{3} & 0 & 0.4258 \Delta_{1} + 0.5742 \Delta_{3}
    \end{pmatrix}
\end{equation}

\section{Grand potential}

See \cite{peottaSuperfluidityTopologicallyNontrivial2015}, especially supplementary material, notes 1 and 3.

Mean-Field Hamiltonian (with the last two terms due to exchange of anticommuting fermion operators and the term quadratic in the expectation value from the mean-field decoupling respectively):
\begin{equation}
	H_{MF} = \sum_{\vb{k}} \Psi_{\vb{k}}^{\dagger} \mathcal{H} (\vb{k}) \Psi_{\vb{k}} + \sum_{\vb{k}} \Tr(H^{\downarrow}_{\vb{k}}) + \sum_{\vb{k} \alpha} \frac{\vert \Delta_{\alpha} \vert^2}{U}
\end{equation}
The second term is the trace of the non-interacting Hamiltonian.

Thermodynamic grand potential (which at zero temperature is equivalent to the mean-field energy):
\begin{align}
	\Omega (T, \Delta) &= -\frac{1}{\beta} \ln{Z_{\Omega}} = -\frac{1}{\beta} \ln{\Tr(e^{-\beta H_{MF}})} \\
	&= \sum_{\vb{k}} \Tr(H^{\downarrow}_{\vb{k}}) + \sum_{\vb{k} \alpha} \frac{\vert \Delta_{\alpha} \vert^2}{U} - \frac{1}{\beta} \ln{\Tr(e^{-\beta \Psi_{\vb{k}}^{\dagger} \mathcal{H} (\vb{k}) \Psi_{\vb{k}}})}
\end{align}
Zero temperature limit:
\begin{align}
	\Omega (\Delta) &= \sum_{\vb{k}} \Tr(H^{\downarrow}_{\vb{k}}) + \sum_{\vb{k} \alpha} \frac{\vert \Delta_{\alpha} \vert^2}{U} - \frac{1}{2} \sum_{\vb{k}} \Tr([\vert \mathcal{H}_{\vb{k}} \vert])
\end{align}
where a function of a matrix \(H\) (such as taking the absolute value of the BdG Hamiltonian \(\mathcal{H}_{\vb{k}}\)) is defined for the diagonal matrix of eigenvalues \(D\) and the unitary matrix \(U\) that diagonalizes \(H\):
\begin{equation}
	f(H) = U f(D) U^{\dagger}
\end{equation}
The route to finding the value of the order parameter for a fixed interaction \(U\) is minimizing the grand potential with respect to \(\Delta\).


\end{document}
