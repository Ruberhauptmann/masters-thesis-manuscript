\documentclass[../notes.tex]{subfiles}
%\graphicspath{{\subfix{../images/}}, {\subfix{../}}}

\begin{document}
\raggedbottom
	
\chapter{Application of the Finite-Momentum Method}\label{ch:results}

\section{Decorated Graphene Model}

\subsection*{Critical Temperatures}


%\begin{SCfigure}[1][t]
\begin{figure}[H]
	\centering
	\import{images/result_plots/dressed_graphene_MF_critical_temps}{OP_vs_T_U_0.0010_V_1.60.pgf}
	\caption{Linear fit for extracting the critical temperature}
	\label{fig:decorated graphene OP vs T}
\end{figure}
%\end{SCfigure}
Critical temperatures via \cref{eq:extract T_C via OP}:
\begin{equation}
	\vert \Psi \vert^2 \propto T_C - T \;.
\end{equation}
Fit and corresponding \(T_C\) are shown in \cref{fig:decorated graphene OP vs T}.
Notable:
\begin{itemize}
	\item the gaps \(\Delta_{\alpha}\) have very different orders of magnitude, but fall in the same way, i.e. have the same critical temperature
\end{itemize}

Can plot critical temperatures against \(V\) and \(U\).

\Cref{fig:decorated Graphene TC and gaps against V} shows \(T_C\) and gaps against \(V\).
\begin{figure}[H]
	\centering
	\begin{subfigure}[b]{0.5\textwidth}
		\centering
		\caption{\hfill\null}\label{sfig:decorated graphene critical temperatures vs V}
		\import{images/result_plots/dressed_graphene_MF_critical_temps}{T_C_vs_V_U_0.0100.pgf}
	\end{subfigure}%
	\begin{subfigure}[b]{0.5\textwidth}
		\centering
		\caption{\hfill\null}\label{sfig:decorated graphene gaps vs V}
		\import{images/result_plots/dressed_graphene_MF_critical_temps}{OP_vs_V_U_0.0100.pgf}
	\end{subfigure}
	\caption{
		\textbf{T} \textbf{(\subref{sfig:decorated graphene critical temperatures vs V})} Critical temperatures vs V. \textbf{(\subref{sfig:decorated graphene gaps vs V})} Gaps vs V .
	} 
	\label{fig:decorated Graphene TC and gaps against V}
\end{figure}
Notable:
\begin{itemize}
	\item Miminum of \(T_C\) and gaps at some \(V = 1.43\), corresponds with metallic \(V\) in repulsive Hubbard model
	\item But no gap closure
	\item TC follows maximal gap
	\item What is happening at \(V = 0\)? New jobs \todo{This}
\end{itemize}

%\Cref{fig:decorated Graphene TC and gaps against U} shows \(T_C\) and gaps against \(U\).
%\begin{figure}[H]
	%\centering
	%\begin{subfigure}[b]{0.5\textwidth}
		%\centering
		%\caption{\hfill\null}\label{sfig:decorated graphene critical temperatures vs U}
		%\import{images/result_plots/dressed_graphene_MF_critical_temps}{T_C_vs_U_V_1.3333.pgf}
	%\end{subfigure}%
	%\begin{subfigure}[b]{0.5\textwidth}
		%\centering
		%\caption{\hfill\null}
		%\label{sfig:decorated graphene gaps vs U}
		%\import{images/result_plots/dressed_graphene_MF_critical_temps}{OP_vs_U_V_1.3333.pgf}
	%\end{subfigure}	
	%\caption{
		%\textbf{T} \textbf{(\subref{sfig:decorated graphene critical temperatures vs U})} Critical temperatures vs U. \textbf{(\subref{sfig:decorated graphene gaps vs U})} Gaps vs U.
	%} 
	%\label{fig:decorated Graphene TC and gaps against U}
%\end{figure}
%Notable:
%\begin{itemize}
	%\item
%\end{itemize}

\Cref{fig:decorated Graphene gaps against U for different V} shows \(T_C\) and gaps against \(U\) for different \(V\).
\begin{figure}[H]
	\centering
	\import{images/result_plots/dressed_graphene_MF_critical_temps}{OP_vs_U_for_different_V.pgf}
	\caption{Gaps vs U. for different V} 
	\label{fig:decorated Graphene gaps against U for different V}
\end{figure}
Notable:
\begin{itemize}
	\item Low V: flat band on X, high V: flat band on Graphene B, medium V: both
	\item Linear with interaction strength, typical for flat band in BCS (reference)
	\item mirrors the switchover seen in the gaps against V and the band structure
\end{itemize}


\subsection*{Extracting the Superconducting Length Scales}

Plot two important gaps against \(q\) in \cref{fig:decorated graphene relevant gaps vs q for medium V} for a medium V
%\begin{SCfigure}
\begin{figure}[H]
	\centering
	\import{images/result_plots/DG_MF_length_scales}{important_gaps_vs_q_medium_V_1.33_U_0.3400.pgf}
	\caption{Gap vs q}
	\label{fig:decorated graphene relevant gaps vs q for medium V}
\end{figure}
%\end{SCfigure}
Notable:
\begin{itemize}
	\item
\end{itemize}
Plot two important gaps against \(q\) in \cref{fig:decorated graphene relevant gaps vs q for low and high V} for a high and low V
\begin{figure}[H]
	\centering
	\import{images/result_plots/DG_MF_length_scales}{important_gaps_vs_q_low_and_high_V_0.27_4.00_U_0.3400.pgf}
	\caption{Gap vs q}
	\label{fig:decorated graphene relevant gaps vs q for low and high V}
\end{figure}
Notable:
\begin{itemize}
	\item
\end{itemize}
Take
\begin{equation}
	\xi(T) = \frac{1}{\sqrt{2} \vert \vb{Q} \vert}
\end{equation}
with \(\vb{Q}\) such that
\begin{equation}
	\vert \frac{\psi_{\vb{Q}}(T)}{\psi_0 (T)} \vert = \frac{1}{\sqrt{2}}
\end{equation}

Plot also current against \(q\) \cref{fig:decorated graphene current vs q}
%\begin{SCfigure}
\begin{figure}[H]
	\centering
	\import{images/result_plots/DG_MF_length_scales}{current_vs_q_medium_V_1.33_U_0.3400.pgf}
	\caption{Current against q}
	\label{fig:decorated graphene current vs q}
\end{figure}
%\end{SCfigure}
Notable:
\begin{itemize}
	\item
\end{itemize}
\todo{Plot also current for low and high V}

Calculate \(\xi (T)\) and \(\lambda_{\mathrm{L}} (T)\).
Plot against \(T\) and fit to get \(\xi_0\), \(\lambda_{L,0}\).
\begin{figure}[H]
	\centering
	\import{images/result_plots/DG_MF_length_scales}{lengths_vs_T_1.33_0.3400.pgf}
	\caption{Temperature fits for xi and lambda}
	\label{fig:decorated graphene temperature fits for xi and lambda}
\end{figure}
\begin{itemize}
	\item
\end{itemize}

\subsection*{Length Scales}

\todo{Plot SC length scales vs V, between different gaps}

\todo{More data points for quantum metric, x axis}
\begin{figure}[H]
	\centering
	\import{images/result_plots/DG_MF_length_scales}{D_S_comparison_U_0.01.pgf}
	\caption{DS}
	\label{fig:decorated graphene compariosn of DS}
\end{figure}

\section{One-Band Hubbard Model on the Square Lattice}

Can extract \(T_C\) same way as above.
Plot it against \(U\).
\begin{figure}[H]
	\centering
	\import{images/result_plots/OBH_DMFT_critical_temps}{T_C_vs_U.pgf}
	\caption{TC against U (also mean field results)}
	\label{fig:DMFT OBH T_C vs U}
\end{figure}
Notable:
\begin{itemize}
	\item
\end{itemize}

\subsection*{Extracting Superconducting Length Scales}

Extraction same as before, plot gap and current against \(q\) in \cref{fig:DMFT OBH gap and current vs q}.
\begin{figure}[H]
	\centering
	\import{images/result_plots/OBH_DMFT_length_scales}{gap_current_vs_q_U_-2.000.pgf}
	\label{fig:DMFT OBH gap and current vs q}
	\caption{Gap and current vs q.}
\end{figure}
Notable:
\begin{itemize}
	\item
\end{itemize}

Calculate \(\xi (T)\) and \(\lambda_{\mathrm{L}} (T)\).
Plot against \(T\) and fit to get \(\xi_0\), \(\lambda_{L,0}\).
\begin{figure}[H]
	\centering
	\import{images/result_plots/OBH_DMFT_length_scales}{lengths_vs_T.pgf}
	\caption{Temperature fits for xi and lambda}
	\label{fig:DMFT OBH temperature fits for xi and lambda}
\end{figure}
\begin{itemize}
	\item
\end{itemize}

\subsection*{BCS-BEC Crossover}

\begin{figure}[H]
	\centering
	\import{images/result_plots/OBH_DMFT_length_scales}{length_scales_vs_U.pgf}
	\caption{Normalized TC/xi/DS against U (to show BCS-BEC crossover)}
	\label{fig:DMFT OBH BCS to BEC crossover}
\end{figure}

\subsection*{Comparison of MF and DMFT Data}

\begin{figure}[H]
	\centering
	\import{images/result_plots/OBH_DMFT_length_scales}{different_D_S_vs_U.pgf}
	\caption{}
	\label{fig:DS comparison MF and DMFT}
\end{figure}

\todo{Plot MF and DMFT xi and lambda against U}

\end{document}