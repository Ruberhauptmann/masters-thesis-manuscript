\documentclass[../notes.tex]{subfiles}
\graphicspath{{\subfix{../images/}}, {\subfix{../}}}

\begin{document}
\raggedbottom
	
\chapter{Superconducting Length Scales}

%\todo{Explain how to get the length scales in the different ways}

%Specifically: take
%\begin{equation}
%	\xi(T) = \frac{1}{\sqrt{2} \vert \vb{Q} \vert}
%\end{equation}
%with \(\vb{Q}\) such that
%\begin{equation}
%	\vert \frac{\psi_{\vb{Q}}(T)}{\psi_0 (T)} \vert = \frac{1}{\sqrt{2}}
%\end{equation}

%Die GL Formel Ψ_q/Ψ_0  = 1 + ξ²q² für spezielles Q auswerten.
%So im Paper gemacht mit Q für das Ψ_Q/Ψ_0 = 1/sqrt(2) mit ξ  = 1/(Q*sqrt(2)). Prinzipiell auch andere Punkte möglich...
%Prinzipiell könnte man auch sagen, "ξ = wo Ψ_q verschwindet". Wird schwierig in deinem Fall, außer dass man dann sagen müsste " ξ < Einheitszelle" (für Graphen also ξ < sqrt(3)*a, wo a der C-C Abstand/Gitterkonstante ist). Aber nicht weiter hilfreich...
%Die GL Formel Ψ_q/Ψ_0  = 1 + ξ²q² anfitten. Ist die Frage, wo man das q_cut setzt für den Fitbereich.
%Die Krümmung von Ψ_q/Ψ_0 für q-> 0 auswerten, da mit f(q) := Ψ_q/Ψ_0 dann gilt ξ  = - 1/2 * d²f/dq² (in GL Theorie). Ist die Frage, wie genau/gut kann man diese Krümmung numerisch bestimmen kann.
%Innerhalb von GL Theorie kann man das q_max, wo j_q maximal wird, auch prinzipiell über ξ = 1/(q_max * sqrt(3) ) mit der Kohärenzlänge in Verbindung bringen.
%Unabhängig von GL Theorie könnte man die volle freie Energie benutzen, aus der man einen allgemeinen Ausdruck für ξ bestimmen kann, der auch außerhalb der Grenzen von niedrigster Ordnung GL Theorie gilt. Das hatte sich Aritas Masterstudent genauer angeschaut, aber der ist jetzt in der Wirtschaft (wollte erst promovieren, aber hat sich beim Schreiben wohl umentschieden :sweat_smile:). Ehrlich gesagt kenne ich die Endergebnisse da auch nicht genau und was damit passieren soll. Muss ich Arita mal fragen... So oder so: In Mean-Field ist es kein Problem, die freie Energie zu berechnen und zu analysieren. In DMFT wird/ist das ecklig :grimacing:

	
\end{document}