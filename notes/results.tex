\documentclass[../notes.tex]{subfiles}
%\graphicspath{{\subfix{../images/}}, {\subfix{../}}}

\begin{document}
	
\chapter{Application of the Finite-Momentum Pairing Method}\label{ch:results}

\Cref{ch:superconductivity} introduced the method of enforcing a finite momentum on the order parameter to gain access to the coherence length \(\xi_0\) and the London penetration depth \(\lambda_{L, 0}\).
In this chapter, it will be applied in two ways: in \cref{sec:results decorated graphene model} on the mean-field level as introduced in \cref{sec:bcs-theory} on the decorated graphene model, see \cref{ch:decorated graphene model}.
Here, the influence of the quantum geometry on superconductivity as explained in \cref{sec:quantum-metric} will be explored.

In \cref{sec:results OBH}, is is applied to the one-band Hubbard model on a square lattice, both on the mean-field level and using \gls{dmft} as introduced in \cref{sec:Dynamical Mean-Field Theory}.
\gls{dmft} gives the opportunity to explore the BCS-BEC crossover phenomenon and the simpler model is an opportunity to compare the results from \gls{dmft} and \gls{bcs} theory. 

\section{BCS: Decorated graphene Model}\label{sec:results decorated graphene model}

In BCS theory, the method involves self-consistently solving the gap equation for a set of external parameters, which in the case of the decorated graphene model are the Hubbard interaction \(U\), the hybridization \(V\) as well as temperature \(T\) and Cooper pair momentum \(\vb{q}\).
This gives gap values \(\Delta_{\alpha}\) for the three orbitals \(\alpha \in \{\mathrm{Gr}_{\mathrm{A}}, \mathrm{Gr}_{\mathrm{B}}, \mathrm{X}\}\).

\subsection*{Critical Temperatures}

The zero-temperature lengths \(\xi_0, \lambda_{L,0}\) are extracted from the temperature dependence \(\xi (T), \lambda_L (T)\), for example
\begin{equation}
	\xi(T) = \xi_0 \left(1 - \frac{T}{T_{\mathrm{C}}}\right)^{-\frac{1}{2}}\;.
\end{equation}
This means the first step in the analysis is to find the critical temperature \(T_{\mathrm{C}}\).
Finding \(T_{\mathrm{C}}\) directly by 

Instead, it can be extracted from the linear behavior of the order parameter near \(T_{\mathrm{C}}\), see \cref{eq:extract TC via OP}:
\begin{equation}
	\vert \Delta_{\alpha} \vert^2 \propto T_{\mathrm{C}} - T \;.
\end{equation}
For a specific \(U\) and \(V\), this is shown in \cref{fig:decorated graphene OP vs T}.
Notable here is that even though \(\Delta_{\mathrm{A}}\) is order of magnitude smaller than \(\Delta_{\mathrm{B}}\) and \(\Delta_{\mathrm{X}}\), \(T_{\mathrm{C}}\) is the same for every orbital.
\begin{SCfigure}[1][tb]
	\centering
	\import{images/result_plots/dressed_graphene_MF_critical_temps}{OP_vs_T_U_0.0100_V_1.60.pgf}
	\caption
		[Extraction of \(T_{\mathrm{C}}\) from the linear behavior of the order parameter.]{
		\textbf{Extraction of \(T_{\mathrm{C}}\) from the linear behavior of the order parameter.}
		Shown is the square of the gap \(\Delta_{\alpha}\) near \(T_{\mathrm{C}}\) for \(U = 0.01 t\) and \(V = 1.6 t\). The linear fit for extracting \(T_{\mathrm{C}}\) is shown in tan, the corresponding \(T_{\mathrm{C}}\) is marked by the dashed gray line.
	}
	\label{fig:decorated graphene OP vs T}
\end{SCfigure}

\Cref{fig:decorated graphene TC and gaps against V} shows the extracted \(T_{\mathrm{C}}\) and gaps against the hybridization \(V\).
\begin{figure}[tb]
	\centering
	\begin{subfigure}[b]{0.5\textwidth}
		\centering
		\caption{\hfill\null}\label{sfig:decorated graphene critical temperatures vs V}
		\import{images/result_plots/dressed_graphene_MF_critical_temps}{T_C_vs_V_U_0.0100.pgf}
	\end{subfigure}%
	\begin{subfigure}[b]{0.5\textwidth}
		\centering
		\caption{\hfill\null}\label{sfig:decorated graphene gaps vs V}
		\import{images/result_plots/dressed_graphene_MF_critical_temps}{OP_vs_V_U_0.0100.pgf}
	\end{subfigure}
	\caption[Critical temperatures and gaps against \(V\).]{
		\textbf{Critical temperatures and gaps against \(V\).}
		\textbf{(\subref{sfig:decorated graphene critical temperatures vs V})} \(T_{\mathrm{C}}\) against hybridization \(V\), the same for all three orbitals. \textbf{(\subref{sfig:decorated graphene gaps vs V})} Gaps \(\Delta_{\alpha}\) for the same values of \(V\). The dashed lines are the orbital weight of the flat band as defined in \cref{sec:decorated graphene quantum metric}. The dashed value \(V = 1.44t\) is taken from the minimum of \(T_{\mathrm{C}} (V)\), coinciding with the switchover of the orbital character. Both plots are for the same \(U = 0.01t\).
	} 
	\label{fig:decorated graphene TC and gaps against V}
\end{figure}
\(T_{\mathrm{C}}\) follows the maximal value of the \(\Delta_{\alpha}\), switching over from \(\mathrm{X}\) to \(\mathrm{Gr}_{\mathrm{B}}\) at \(V = 1.44t\).
The value of \(\Delta_{\alpha}\) exactly follows the orbital weight \(w_{\alpha}\) of the flat band, for the orbitals \(\alpha \in \{\mathrm{Gr}_{\mathrm{A}}, \mathrm{Gr}_{\mathrm{B}}, \mathrm{X}\}\).
In contrast to a repulsive Hubbard interaction \cite{wittQuantumGeometryLocal2025} there is no gap closure for a medium \(V\), instead there is just a minimum of the maximal gap value.

\subsection*{Extracting the Superconducting Length Scales}

The correlation length \(\xi (T)\) is associated with the breakdown of the order parameter:
\begin{equation}
	\vert \Psi_{\vb{q}} \vert^2 = \vert \Psi_{0} \vert^2 \left(1 - \xi(T)^2 q^2\right) \;,
\end{equation}
which means that the \(q_C\) where the order parameter breaks down is related to the correlation length via
\begin{equation}
	\xi = \frac{1}{q_C}\;.
\end{equation}
The momentum \(\vb{q}\) is chosen as \(\vb{q} = x \cdot \vb{b}_1\) with the reciprocal vector \(\vb{b}_1\) and \(x \in [0, 0.5]\).
For \(x > 0.5\), the 
Similar to finding \(T_{\mathrm{C}}\), numerical calculations near the point where the gap goes to zero are hard to converge, so instead the criterion employed here is to choose \(\vb{Q}\) such that
\begin{equation}
	\left\vert \frac{\psi_{\vb{Q}}(T)}{\psi_0 (T)} \right\vert = \frac{1}{\sqrt{2}}\;,
\end{equation}
and then take
\begin{equation}
	\xi = \frac{1}{\sqrt{2} \vb{Q}}\;,
\end{equation}
compare ref. \cite{wittBypassingLatticeBCS2024} for discussion about this method and comparison to other ways to extract the superconducting length scales from the \(\vb{q}\)-dependence of the order parameter.

As shown in \cref{sfig:decorated graphene gaps vs V}, only \(\Delta_{\mathrm{B}}\) and \(\Delta_{\mathrm{X}}\) have a significant contribution in the parameter range of \(U\) here, so for these the \(\vb{q}\)-dependence is shown in \cref{fig:decorated graphene relevant gaps vs q for medium V} for \(V = 1.5\), so in a parameter regime switching over between dominating \(\mathrm{X}\) and \(\mathrm{B}\) contribution.
For higher temperatures \(q_C \to 0\), showing how the correlation length diverges for \(T \to T_{\mathrm{C}}\).
\begin{figure}[tb]
	\centering
	\import{images/result_plots/DG_MF_length_scales}{important_gaps_vs_q_medium_V_1.50_U_0.1000.pgf}
	\caption[Suppression of the order parameter with \(\vb{q}\) for \(V = 1.5t\) and \(U = 0.1t\).]{
		\textbf{Suppression of the order parameter with \(\vb{q}\) for \(V = 1.5t\) and \(U = 0.1t\).} The x-axis is marked in relative lattice units, i.e. \(\vb{q} = q \cdot \vb{b}_1\) for the reciprocal unit vector \(\vb{b}_1\). Marked in gray are the points at which the gaps have fallen off to \(\nicefrac{1}{\sqrt{2}}\) of their value at \(\vb{q} = 0\).}
	\label{fig:decorated graphene relevant gaps vs q for medium V}
\end{figure}

In the case of high and low \(V\) where the superconducting order is dominated by one of \(\Delta_{\mathrm{A}}, \Delta_{\mathrm{X}}\), \cref{fig:decorated graphene relevant gaps vs q for low and high V} shows that the gap does not fully go down to 0 for \(\vb{q} = \nicefrac{1}{2} \cdot \vb{b}_1\), meaning that the correlation length is smaller than 
\begin{equation}
	\xi = \frac{1}{0.5 \cdot \vert \vb{b}_1 \vert} = \frac{\sqrt{3} a}{2 \pi} = \frac{a_0}{2 \pi}\;.
\end{equation}
\todo{Talk about limitation of GL: does not apply for high q}
\begin{figure}[tb]
	\centering
	\import{images/result_plots/DG_MF_length_scales}{important_gaps_vs_q_low_and_high_V_0.50_4.00_U_0.1000.pgf}
	\caption[Suppression of the order parameter with \(\vb{q}\) for \(V = 0.5t\) and \(V = 4t\) (both for \(U = 0.1t\)).]{
		\textbf{Suppression of the order parameter with \(\vb{q}\) for \(V = 0.5t\) and \(V = 4t\) (both for \(U = 0.1t\)).} In contrast to \cref{fig:decorated graphene relevant gaps vs q for medium V}, in this parameter regime the order parameter does not get fully suppressed for the maximal \(q = 0.5\).}
	\label{fig:decorated graphene relevant gaps vs q for low and high V}
\end{figure}

Besides the correlation lengths extracted from the behavior of the order parameter against \(\vb{q}\), also the behavior of the superconducting current \(j (\vb{q})\) is needed to calculate the London penetration depth \(\lambda_{\mathrm{L}}\) via
\begin{equation}
	\lambda_{\mathrm{L}} (T) = \sqrt{\frac{\Phi_0}{3 \sqrt{3} \pi \mu_0 \xi (T) j_{dp} (T)}}\;.
\end{equation}
In particular, the maximum \(j_{dp}\) (the depairing current) is used.
\Cref{fig:decorated graphene current vs q} shows the current \(\vb{j} (\vb{q})\) with the maximum calculated from an interpolation marked for every temperature.
\begin{figure}[tb]
	\centering
	\import{images/result_plots/DG_MF_length_scales}{current_vs_q_U_0.1000.pgf}
	\caption[Superconducting current from a finite \(\vb{q}\).]{
		\textbf{Superconducting current from a finite \(\vb{q}\).} For calculation of the London penetration depth \(\lambda_{\mathrm{L}}\), the maximum \(j_{\mathrm{dp}}\) of the current is needed, marked here in gray.
	}
	\label{fig:decorated graphene current vs q}
\end{figure}
As in the \(\vb{q}\)-dependence of the gaps, for the low and high \(V\)-values, the current is not fully suppressed for the lower temperatures and \(q = 0.5\).
In contrast to the analysis of the gaps, it is still possible to find a maximum of the current in theses cases, but because the validity of Ginzburg-Landau theory is not given for high \(\vb{q}\), it is unclear whether the connection to the superconducting length scales as derived in Ginzburg-Landau theory is given.

\Cref{fig:decorated graphene temperature fits for xi and lambda} shows the temperature dependence for \(\xi (T)\) and \(\lambda_{\mathrm{L}}\).
These can be fit to
\begin{align}
	\xi(T) = \xi_0 \left(1 - \frac{T}{T_{\mathrm{C}}}\right)^{-\frac{1}{2}}
\end{align}
and
\begin{align}
	\lambda_{\mathrm{L}} (T) = 	\xi_0 \left(1 - \frac{T}{T_{\mathrm{C}}}\right)^{-\frac{1}{2}}
\end{align}
to obtain the zero-temperature values \(\xi_0\) and \(\lambda_{\mathrm{L}, 0}\).
\begin{figure}[tb]
	\centering
	\import{images/result_plots/DG_MF_length_scales}{lengths_vs_T_V_1.50_U_0.1000.pgf}
	\caption[Temperature dependence of the correlation length \(\xi\) and London penetration depth \(\lambda_{\mathrm{L}}\) for \(V = 1.50t\) and \(U  = 0.1t\).]{\textbf{Temperature dependence of the correlation length \(\xi\) and London penetration depth \(\lambda_{\mathrm{L}}\) for \(V = 1.50t\) and \(U  = 0.1t\).} The fits for extracting the zero-temperature values \(\xi_0\), \(\lambda_{\mathrm{L}, 0}\) are marked as dash lines.}
	\label{fig:decorated graphene temperature fits for xi and lambda}
\end{figure}
For the low and and high \(V\) (\cref{fig:decorated graphene relevant gaps vs q for low and high V}): extraction of \(xi (T)\) is not properly possible due to the behaviour of the gaps against \(q\).
One can try to extract information from the criterium, but the temperature dependence does not follow the fit. \todo{Make that properly}

%\begin{figure}[tb]
	%\centering
	%\import{images/result_plots/DG_MF_length_scales}{lengths_vs_T_V_0.50_U_1.0000.pgf}
	%\caption{\textbf{Temperature dependence of the correlation length \(\xi\) and London penetration depth \(\lambda_{\mathrm{L}}\) for \(V = 1.50t\) and \(U  = 0.1t\).} The fits for extracting the zero-temperature values \(\xi_0\), \(\lambda_{\mathrm{L}, 0}\) are marked as dash lines.}
	%\label{fig:decorated graphene temperature fits for xi and lambda for problematic V}
%\end{figure}

\subsection*{Length Scales}

\Cref{fig:decorated graphene length scales} shows the extracted length scales for two different values of the attractive interaction \(U\).
In the coherence length, the behavior is similar for between these two values: the orbital with the largest gap value has the shortest coherence length, with a switchover between the small and large \(V\)-values.
\begin{figure}[!tb]
	\centering
	\begin{subfigure}[b]{0.5\textwidth}
		\centering
		\caption{\hfill\null}\label{sfig:decorated graphene length scales U 0.10}
		\import{images/result_plots/DG_MF_length_scales}{length_scales_comparison_U_0.10.pgf}
	\end{subfigure}%
	\begin{subfigure}[b]{0.5\textwidth}
		\centering
		\caption{\hfill\null}\label{sfig:decorated graphene length scales U 4.00}
		\import{images/result_plots/DG_MF_length_scales}{length_scales_comparison_U_4.00.pgf}
	\end{subfigure}
	\caption[Superconducting length scales for \(U = 0.1t\).]{
		\textbf{Superconducting length scales for \(U = 0.1t\).} The coherence length \(\xi_0\) is related to the size of electron pairs, the London penetration depth \(\lambda_{\mathrm{L}, 0}\) is the distance magnetic fields penetrate into the material.
	}
	\label{fig:decorated graphene length scales}
\end{figure}
In general, a higher attractive interaction is associated with smaller Cooper pairs.
Interestingly, the orbital with vanishing gaps in the large \(V\)-limit go to the same value of \(\xi_0\), independent of \(U\).
It should be noted however that for the \(\mathrm{B}\)-orbital, the coherence length becomes smaller than \(\frac{a_0}{2 \pi} \approx 1.6\), the minimum value that can be extracted for \(V \approx 3\), so no statement about the dominating gap can be made in the large \(V\)-limit.
\todo{Talk about the exact moment of the switchover}
\todo{Talk about London penetration depth}


From the London penetration depth \(\lambda_{\mathrm{L}, 0}\), the superfluid weight can be calculated via
\begin{equation}
	D_S \propto \lambda_{\mathrm{L}, 0}^{-}
\end{equation}
Another way to calculate the superfluid weight from linear response theory was shown in \cref{sec:quantum-metric}.
\Cref{fig:decorated graphene comparison of DS normalized} shows the superfluid weight from the \(\vb{q}\)-dependence and the linear response formula, split up between the geometric and the conventional contribution as well as the quantum metric.
\begin{figure}[!tb]
	\centering
	\import{images/result_plots/DG_MF_length_scales}{D_S_comparison_norm.pgf}
	\caption[Comparison of the superfluid weight calculated by different methods.]{
		\textbf{Comparison of the superfluid weight calculated by different methods.} All quantities are normalized to analyze the general trend in comparison to the quantum metric.
		For the calculation coming from linear response theory (see \cref{sec:quantum-metric}), the geometric and conventional contributions are marked separately.
	}
	\label{fig:decorated graphene comparison of DS normalized}
\end{figure}
The calculation from the linear response formula shows that the isolated flat band limit, where the geometric contribution dominates the superfluid weight is only given for low \(U\), when it is on the order of the gap between the flat and dispersive bands.
For \(U = 1.00t\), the conventional contribution becomes bigger, while for \(U = 6.00t\) it dominates up until the gap (which is given by \(V\)) is comparable to \(U\).
\todo{Where is the maximum of qm, compared with maximum of Aalto DS}

The results from the \gls{fmp} method agree with the linear response insofar that they show a peak in the intermediate \(V\)-regime and go to zero for \(V \to 0\) and \(V \to \infty\), but the location of this peak is not the same between the two method.
Especially in the low \(V\)-limit, the \gls{fmp} method (at least for this model) is at the boundary of applicability, so this is possibly a limitation of the method.

The critical temperatures in BCS theory as seen in \cref{sfig:decorated graphene critical temperatures vs V} shows a minimum in the intermediate \(V\) region.
The effect of \(D_S\) that cannot be seen in the critical temperatures from BCS theory, so the expectation would that the critical temperature actually falls off in the low and high \(V\) limit and the intermediate region is actually better for superconductivity \todo{Is that correct?}.

\clearpage
\section{BCS and DMFT: One-Band Hubbard Model on the Square Lattice}\label{sec:results OBH}

\subsection*{BCS-BEC Crossover}

\Acrshort{dmft} gives insight into the phenomenon of the BCS-BEC crossover.
Can extract \(T_{\mathrm{C}}\) same way as above.
\Cref{sfig:DMFT OBH T_C vs U} shows \(T_{\mathrm{C}}\) against \(U\).
The DMFT curve shows the typical dome-shape of the BCS-BEC crossover and the fact that the MF \(T_{\mathrm{C}}\) marks the pairing temperature, which is different to the superconducting \(T_{\mathrm{C}}\). 

The extraction of the superconducting length scales works the same as in \cref{sec:results decorated graphene model}. \todo{Talk about the fact that the method works for all U values}
\Cref{sfig:DMFT OBH BCS to BEC crossover} shows how these length scales characterize the BCS-BEC crossover phenomenon: the coherence length goes to zero when going into the BEC regime, marking how the Cooper pairs become strongly localized. \todo{Minimal value of coherence length?}
The superfluid weight has its maximal value for low \(U\) and also goes to zero for stronger attractive interaction, 
\begin{figure}[tb]
	\centering
	\begin{subfigure}[b]{0.5\textwidth}
		\centering
		\caption{\hfill\null}\label{sfig:DMFT OBH T_C vs U}
		\import{images/result_plots/OBH_DMFT_critical_temps}{T_C_vs_U.pgf}
	\end{subfigure}%
	\begin{subfigure}[b]{0.5\textwidth}
		\centering
		\caption{\hfill\null}\label{sfig:DMFT OBH BCS to BEC crossover}
		\import{images/result_plots/OBH_DMFT_length_scales}{normalized_length_scales_vs_U.pgf}
	\end{subfigure}
	\caption[\(T_{\mathrm{C}}\) and superconducting length scales for the one-band Hubbard model.]{
		\textbf{\(T_{\mathrm{C}}\) and superconducting length scales for the one-band Hubbard model.} \textbf{(\subref{sfig:DMFT OBH T_C vs U})} \(T_{\mathrm{C}}\) calculated from mean-field theory and \gls{dmft} respectively. It shows the characteristic dome of the BCS-BEC crossover that is not captured in mean-field theory. \textbf{(\subref{sfig:DMFT OBH BCS to BEC crossover})}  
	}
	\label{fig:DMFT OBH BCS BEC crossover and T_C vs U}
\end{figure}

\subsection*{Comparison of MF and DMFT}

\Cref{fig:length scales MF and DMFT} shows that in mean-field theory, the \gls{fmp} method in BCS theory underestimates the length scales in comparison to the DMFT method. \todo{Write what goes on in the high U limit between MF and DMFT} \todo{Why? What is captured in DMFT?}

\begin{figure}[tb]
	\centering
	\import{images/result_plots/OBH_DMFT_length_scales}{length_scales_vs_U.pgf}
	\caption[Comparison of superconducting length scales between mean-field and DMFT.]{\textbf{Comparison of superconducting length scales between mean-field and DMFT.}}
	\label{fig:length scales MF and DMFT}
\end{figure}

\end{document}