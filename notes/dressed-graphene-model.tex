\documentclass[../notes.tex]{subfiles}
\graphicspath{{\subfix{../images/}}, {\subfix{../}}}

\begin{document}
\raggedbottom

\chapter{Dressed Graphene Model}

This thesis concerned with a specific model.
Idea: Graphene with an added orbital on one of the lattice site with a low hopping, as to provide a flat band.
I will call this model dressed Graphene from here on.
This chapter reviews the lattice structure in \cref{sec:lattice-structure-of-graphene}.

\todo{Write introduction to the model and what is done in this chapter}

\todo{Connection with Niklas/Siheeon paper on dressed Graphene}

\section{Lattice Structure}\label{sec:lattice-structure-of-graphene}

There exist a few different ways to define the lattice structure of Graphene which are all equivalent, but intermediate steps in calculating tight-binding models look different depending on the definition.
This review on  follows ref.~\cite{yangStructureGrapheneIts2018}.

Monolayer graphene forms a honeycomb lattice, which is a hexagonal Bravais lattice with a two atom basis, as can be seen in \cref{sfig:graphene lattice structure}.
The primitive lattice vectors of the hexagonal lattice are:
\begin{align}
	\vb{a}_1 = \frac{a}{2} \begin{pmatrix} 1 \\ \sqrt{3} \end{pmatrix}, \; \vb{a}_2 = \frac{a}{2} \begin{pmatrix} 1 \\ -\sqrt{3} \end{pmatrix}
\end{align}
\todo{Labels on vectors}
with lattice constant \(a = \sqrt{3} a_0 \approx \SI{2.46}{\angstrom}\), using the nearest-neighbour distance \(a_0\).
The vectors to the nearest-neighbor atoms \(B_i\) (\(i = 1, 2, 3,\)) from atom \(A\) are
\begin{align}
	\vb{\delta}_{AB, 1} = \begin{pmatrix} 0 \\ \frac{a}{\sqrt{3}} \end{pmatrix},\; \vb{\delta}_{AB, 2} = \begin{pmatrix} \frac{a}{2} \\ -\frac{a}{2\sqrt{3}} \end{pmatrix},\; \vb{\delta}_{AB, 3} = \begin{pmatrix} -\frac{a}{2} \\ -\frac{a}{2\sqrt{3}} \end{pmatrix}
\end{align}
and the vectors to the nearest-neighbor atoms \(A_i\) (\(i = 1, 2, 3,\)) from atom \(B\) are
\begin{align}
	\vb{\delta}_{BA, 1} = \begin{pmatrix} 0 \\ -\frac{a}{\sqrt{3}} \end{pmatrix},\; \vb{\delta}_{BA, 2} = \begin{pmatrix} \frac{a}{2} \\ \frac{a}{2\sqrt{3}} \end{pmatrix},\; \vb{\delta}_{BA, 3} = \begin{pmatrix} -\frac{a}{2} \\ \frac{a}{2\sqrt{3}} \end{pmatrix} \;.
\end{align}
The vectors between the Graphene \(\mathrm{A}\) atom and the six neighbours on the same sub lattice are:
\begin{alignat}{3}
	&\vb{\delta}_{AA, 1} = \begin{pmatrix} 1 \\ \sqrt{3} \end{pmatrix}, \;
	&&\vb{\delta}_{AA, 2} = a \begin{pmatrix} 1 \\ 0 \end{pmatrix}, \;
	& &\vb{\delta}_{AA, 3} = a \begin{pmatrix} \frac{1}{2} \\ -\frac{\sqrt{3}}{2} \end{pmatrix}, \\
	&\vb{\delta}_{AA, 4} = a \begin{pmatrix} -\frac{1}{2} \\ -\frac{\sqrt{3}}{2} \end{pmatrix}, \;
	&&\vb{\delta}_{AA, 5} = a \begin{pmatrix} -1 \\ 0 \end{pmatrix}, \;
	& &\vb{\delta}_{AA, 6} = a \begin{pmatrix} -\frac{1}{2} \\ \frac{\sqrt{3}}{2} \end{pmatrix}
\end{alignat}
\begin{figure}[tb]
	\centering
	\begin{subfigure}[t]{0.5\textwidth}
		\centering
		\caption{\hfill\null}\label{sfig:graphene lattice structure}
		%% Creator: Matplotlib, PGF backend
%%
%% To include the figure in your LaTeX document, write
%%   \input{<filename>.pgf}
%%
%% Make sure the required packages are loaded in your preamble
%%   \usepackage{pgf}
%%
%% Also ensure that all the required font packages are loaded; for instance,
%% the lmodern package is sometimes necessary when using math font.
%%   \usepackage{lmodern}
%%
%% Figures using additional raster images can only be included by \input if
%% they are in the same directory as the main LaTeX file. For loading figures
%% from other directories you can use the `import` package
%%   \usepackage{import}
%%
%% and then include the figures with
%%   \import{<path to file>}{<filename>.pgf}
%%
%% Matplotlib used the following preamble
%%   \def\mathdefault#1{#1}
%%   \everymath=\expandafter{\the\everymath\displaystyle}
%%   \IfFileExists{scrextend.sty}{
%%     \usepackage[fontsize=11.000000pt]{scrextend}
%%   }{
%%     \renewcommand{\normalsize}{\fontsize{11.000000}{13.200000}\selectfont}
%%     \normalsize
%%   }
%%   \usepackage{fontspec}\usepackage{unicode-math}\setmathfont{texgyrepagella-math.otf}\setmainfont{texgyrepagella-math}\usepackage{nicefrac}
%%   \makeatletter\@ifpackageloaded{underscore}{}{\usepackage[strings]{underscore}}\makeatother
%%
\begingroup%
\makeatletter%
\begin{pgfpicture}%
\pgfpathrectangle{\pgfpointorigin}{\pgfqpoint{2.800000in}{2.240000in}}%
\pgfusepath{use as bounding box, clip}%
\begin{pgfscope}%
\pgfsetbuttcap%
\pgfsetmiterjoin%
\definecolor{currentfill}{rgb}{1.000000,1.000000,1.000000}%
\pgfsetfillcolor{currentfill}%
\pgfsetlinewidth{0.000000pt}%
\definecolor{currentstroke}{rgb}{1.000000,1.000000,1.000000}%
\pgfsetstrokecolor{currentstroke}%
\pgfsetdash{}{0pt}%
\pgfpathmoveto{\pgfqpoint{0.000000in}{0.000000in}}%
\pgfpathlineto{\pgfqpoint{2.800000in}{0.000000in}}%
\pgfpathlineto{\pgfqpoint{2.800000in}{2.240000in}}%
\pgfpathlineto{\pgfqpoint{0.000000in}{2.240000in}}%
\pgfpathlineto{\pgfqpoint{0.000000in}{0.000000in}}%
\pgfpathclose%
\pgfusepath{fill}%
\end{pgfscope}%
\begin{pgfscope}%
\pgfsetbuttcap%
\pgfsetmiterjoin%
\definecolor{currentfill}{rgb}{1.000000,1.000000,1.000000}%
\pgfsetfillcolor{currentfill}%
\pgfsetlinewidth{0.000000pt}%
\definecolor{currentstroke}{rgb}{0.000000,0.000000,0.000000}%
\pgfsetstrokecolor{currentstroke}%
\pgfsetstrokeopacity{0.000000}%
\pgfsetdash{}{0pt}%
\pgfpathmoveto{\pgfqpoint{0.338326in}{0.377767in}}%
\pgfpathlineto{\pgfqpoint{2.596020in}{0.377767in}}%
\pgfpathlineto{\pgfqpoint{2.596020in}{2.023029in}}%
\pgfpathlineto{\pgfqpoint{0.338326in}{2.023029in}}%
\pgfpathlineto{\pgfqpoint{0.338326in}{0.377767in}}%
\pgfpathclose%
\pgfusepath{fill}%
\end{pgfscope}%
\begin{pgfscope}%
\pgfpathrectangle{\pgfqpoint{0.338326in}{0.377767in}}{\pgfqpoint{2.257694in}{1.645262in}}%
\pgfusepath{clip}%
\pgfsetbuttcap%
\pgfsetroundjoin%
\pgfsetlinewidth{1.505625pt}%
\definecolor{currentstroke}{rgb}{0.000000,0.000000,0.000000}%
\pgfsetstrokecolor{currentstroke}%
\pgfsetdash{}{0pt}%
\pgfpathmoveto{\pgfqpoint{0.550288in}{0.671034in}}%
\pgfpathlineto{\pgfqpoint{0.779510in}{0.538693in}}%
\pgfusepath{stroke}%
\end{pgfscope}%
\begin{pgfscope}%
\pgfpathrectangle{\pgfqpoint{0.338326in}{0.377767in}}{\pgfqpoint{2.257694in}{1.645262in}}%
\pgfusepath{clip}%
\pgfsetbuttcap%
\pgfsetroundjoin%
\pgfsetlinewidth{1.505625pt}%
\definecolor{currentstroke}{rgb}{0.000000,0.000000,0.000000}%
\pgfsetstrokecolor{currentstroke}%
\pgfsetdash{}{0pt}%
\pgfpathmoveto{\pgfqpoint{0.550288in}{0.671034in}}%
\pgfpathlineto{\pgfqpoint{0.550288in}{0.935716in}}%
\pgfusepath{stroke}%
\end{pgfscope}%
\begin{pgfscope}%
\pgfpathrectangle{\pgfqpoint{0.338326in}{0.377767in}}{\pgfqpoint{2.257694in}{1.645262in}}%
\pgfusepath{clip}%
\pgfsetbuttcap%
\pgfsetroundjoin%
\pgfsetlinewidth{1.505625pt}%
\definecolor{currentstroke}{rgb}{0.000000,0.000000,0.000000}%
\pgfsetstrokecolor{currentstroke}%
\pgfsetdash{}{0pt}%
\pgfpathmoveto{\pgfqpoint{0.550288in}{0.935716in}}%
\pgfpathlineto{\pgfqpoint{0.779510in}{1.068057in}}%
\pgfusepath{stroke}%
\end{pgfscope}%
\begin{pgfscope}%
\pgfpathrectangle{\pgfqpoint{0.338326in}{0.377767in}}{\pgfqpoint{2.257694in}{1.645262in}}%
\pgfusepath{clip}%
\pgfsetbuttcap%
\pgfsetroundjoin%
\pgfsetlinewidth{1.505625pt}%
\definecolor{currentstroke}{rgb}{0.000000,0.000000,0.000000}%
\pgfsetstrokecolor{currentstroke}%
\pgfsetdash{}{0pt}%
\pgfpathmoveto{\pgfqpoint{0.779510in}{0.538693in}}%
\pgfpathlineto{\pgfqpoint{1.008731in}{0.671034in}}%
\pgfusepath{stroke}%
\end{pgfscope}%
\begin{pgfscope}%
\pgfpathrectangle{\pgfqpoint{0.338326in}{0.377767in}}{\pgfqpoint{2.257694in}{1.645262in}}%
\pgfusepath{clip}%
\pgfsetbuttcap%
\pgfsetroundjoin%
\pgfsetlinewidth{1.505625pt}%
\definecolor{currentstroke}{rgb}{0.000000,0.000000,0.000000}%
\pgfsetstrokecolor{currentstroke}%
\pgfsetdash{}{0pt}%
\pgfpathmoveto{\pgfqpoint{0.550288in}{1.465079in}}%
\pgfpathlineto{\pgfqpoint{0.779510in}{1.332739in}}%
\pgfusepath{stroke}%
\end{pgfscope}%
\begin{pgfscope}%
\pgfpathrectangle{\pgfqpoint{0.338326in}{0.377767in}}{\pgfqpoint{2.257694in}{1.645262in}}%
\pgfusepath{clip}%
\pgfsetbuttcap%
\pgfsetroundjoin%
\pgfsetlinewidth{1.505625pt}%
\definecolor{currentstroke}{rgb}{0.000000,0.000000,0.000000}%
\pgfsetstrokecolor{currentstroke}%
\pgfsetdash{}{0pt}%
\pgfpathmoveto{\pgfqpoint{0.550288in}{1.465079in}}%
\pgfpathlineto{\pgfqpoint{0.550288in}{1.729761in}}%
\pgfusepath{stroke}%
\end{pgfscope}%
\begin{pgfscope}%
\pgfpathrectangle{\pgfqpoint{0.338326in}{0.377767in}}{\pgfqpoint{2.257694in}{1.645262in}}%
\pgfusepath{clip}%
\pgfsetbuttcap%
\pgfsetroundjoin%
\pgfsetlinewidth{1.505625pt}%
\definecolor{currentstroke}{rgb}{0.000000,0.000000,0.000000}%
\pgfsetstrokecolor{currentstroke}%
\pgfsetdash{}{0pt}%
\pgfpathmoveto{\pgfqpoint{0.550288in}{1.729761in}}%
\pgfpathlineto{\pgfqpoint{0.779510in}{1.862102in}}%
\pgfusepath{stroke}%
\end{pgfscope}%
\begin{pgfscope}%
\pgfpathrectangle{\pgfqpoint{0.338326in}{0.377767in}}{\pgfqpoint{2.257694in}{1.645262in}}%
\pgfusepath{clip}%
\pgfsetbuttcap%
\pgfsetroundjoin%
\pgfsetlinewidth{1.505625pt}%
\definecolor{currentstroke}{rgb}{0.000000,0.000000,0.000000}%
\pgfsetstrokecolor{currentstroke}%
\pgfsetdash{}{0pt}%
\pgfpathmoveto{\pgfqpoint{0.779510in}{1.068057in}}%
\pgfpathlineto{\pgfqpoint{1.008731in}{0.935716in}}%
\pgfusepath{stroke}%
\end{pgfscope}%
\begin{pgfscope}%
\pgfpathrectangle{\pgfqpoint{0.338326in}{0.377767in}}{\pgfqpoint{2.257694in}{1.645262in}}%
\pgfusepath{clip}%
\pgfsetbuttcap%
\pgfsetroundjoin%
\pgfsetlinewidth{1.505625pt}%
\definecolor{currentstroke}{rgb}{0.000000,0.000000,0.000000}%
\pgfsetstrokecolor{currentstroke}%
\pgfsetdash{}{0pt}%
\pgfpathmoveto{\pgfqpoint{0.779510in}{1.068057in}}%
\pgfpathlineto{\pgfqpoint{0.779510in}{1.332739in}}%
\pgfusepath{stroke}%
\end{pgfscope}%
\begin{pgfscope}%
\pgfpathrectangle{\pgfqpoint{0.338326in}{0.377767in}}{\pgfqpoint{2.257694in}{1.645262in}}%
\pgfusepath{clip}%
\pgfsetbuttcap%
\pgfsetroundjoin%
\pgfsetlinewidth{1.505625pt}%
\definecolor{currentstroke}{rgb}{0.000000,0.000000,0.000000}%
\pgfsetstrokecolor{currentstroke}%
\pgfsetdash{}{0pt}%
\pgfpathmoveto{\pgfqpoint{0.779510in}{1.332739in}}%
\pgfpathlineto{\pgfqpoint{1.008731in}{1.465079in}}%
\pgfusepath{stroke}%
\end{pgfscope}%
\begin{pgfscope}%
\pgfpathrectangle{\pgfqpoint{0.338326in}{0.377767in}}{\pgfqpoint{2.257694in}{1.645262in}}%
\pgfusepath{clip}%
\pgfsetbuttcap%
\pgfsetroundjoin%
\pgfsetlinewidth{1.505625pt}%
\definecolor{currentstroke}{rgb}{0.000000,0.000000,0.000000}%
\pgfsetstrokecolor{currentstroke}%
\pgfsetdash{}{0pt}%
\pgfpathmoveto{\pgfqpoint{1.008731in}{0.671034in}}%
\pgfpathlineto{\pgfqpoint{1.237952in}{0.538693in}}%
\pgfusepath{stroke}%
\end{pgfscope}%
\begin{pgfscope}%
\pgfpathrectangle{\pgfqpoint{0.338326in}{0.377767in}}{\pgfqpoint{2.257694in}{1.645262in}}%
\pgfusepath{clip}%
\pgfsetbuttcap%
\pgfsetroundjoin%
\pgfsetlinewidth{1.505625pt}%
\definecolor{currentstroke}{rgb}{0.000000,0.000000,0.000000}%
\pgfsetstrokecolor{currentstroke}%
\pgfsetdash{}{0pt}%
\pgfpathmoveto{\pgfqpoint{1.008731in}{0.671034in}}%
\pgfpathlineto{\pgfqpoint{1.008731in}{0.935716in}}%
\pgfusepath{stroke}%
\end{pgfscope}%
\begin{pgfscope}%
\pgfpathrectangle{\pgfqpoint{0.338326in}{0.377767in}}{\pgfqpoint{2.257694in}{1.645262in}}%
\pgfusepath{clip}%
\pgfsetbuttcap%
\pgfsetroundjoin%
\pgfsetlinewidth{1.505625pt}%
\definecolor{currentstroke}{rgb}{0.000000,0.000000,0.000000}%
\pgfsetstrokecolor{currentstroke}%
\pgfsetdash{}{0pt}%
\pgfpathmoveto{\pgfqpoint{1.008731in}{0.935716in}}%
\pgfpathlineto{\pgfqpoint{1.237952in}{1.068057in}}%
\pgfusepath{stroke}%
\end{pgfscope}%
\begin{pgfscope}%
\pgfpathrectangle{\pgfqpoint{0.338326in}{0.377767in}}{\pgfqpoint{2.257694in}{1.645262in}}%
\pgfusepath{clip}%
\pgfsetbuttcap%
\pgfsetroundjoin%
\pgfsetlinewidth{1.505625pt}%
\definecolor{currentstroke}{rgb}{0.000000,0.000000,0.000000}%
\pgfsetstrokecolor{currentstroke}%
\pgfsetdash{}{0pt}%
\pgfpathmoveto{\pgfqpoint{1.237952in}{0.538693in}}%
\pgfpathlineto{\pgfqpoint{1.467173in}{0.671034in}}%
\pgfusepath{stroke}%
\end{pgfscope}%
\begin{pgfscope}%
\pgfpathrectangle{\pgfqpoint{0.338326in}{0.377767in}}{\pgfqpoint{2.257694in}{1.645262in}}%
\pgfusepath{clip}%
\pgfsetbuttcap%
\pgfsetroundjoin%
\pgfsetlinewidth{1.505625pt}%
\definecolor{currentstroke}{rgb}{0.000000,0.000000,0.000000}%
\pgfsetstrokecolor{currentstroke}%
\pgfsetdash{}{0pt}%
\pgfpathmoveto{\pgfqpoint{0.779510in}{1.862102in}}%
\pgfpathlineto{\pgfqpoint{1.008731in}{1.729761in}}%
\pgfusepath{stroke}%
\end{pgfscope}%
\begin{pgfscope}%
\pgfpathrectangle{\pgfqpoint{0.338326in}{0.377767in}}{\pgfqpoint{2.257694in}{1.645262in}}%
\pgfusepath{clip}%
\pgfsetbuttcap%
\pgfsetroundjoin%
\pgfsetlinewidth{1.505625pt}%
\definecolor{currentstroke}{rgb}{0.000000,0.000000,0.000000}%
\pgfsetstrokecolor{currentstroke}%
\pgfsetdash{}{0pt}%
\pgfpathmoveto{\pgfqpoint{1.008731in}{1.465079in}}%
\pgfpathlineto{\pgfqpoint{1.237952in}{1.332739in}}%
\pgfusepath{stroke}%
\end{pgfscope}%
\begin{pgfscope}%
\pgfpathrectangle{\pgfqpoint{0.338326in}{0.377767in}}{\pgfqpoint{2.257694in}{1.645262in}}%
\pgfusepath{clip}%
\pgfsetbuttcap%
\pgfsetroundjoin%
\pgfsetlinewidth{1.505625pt}%
\definecolor{currentstroke}{rgb}{0.000000,0.000000,0.000000}%
\pgfsetstrokecolor{currentstroke}%
\pgfsetdash{}{0pt}%
\pgfpathmoveto{\pgfqpoint{1.008731in}{1.465079in}}%
\pgfpathlineto{\pgfqpoint{1.008731in}{1.729761in}}%
\pgfusepath{stroke}%
\end{pgfscope}%
\begin{pgfscope}%
\pgfpathrectangle{\pgfqpoint{0.338326in}{0.377767in}}{\pgfqpoint{2.257694in}{1.645262in}}%
\pgfusepath{clip}%
\pgfsetbuttcap%
\pgfsetroundjoin%
\pgfsetlinewidth{1.505625pt}%
\definecolor{currentstroke}{rgb}{0.000000,0.000000,0.000000}%
\pgfsetstrokecolor{currentstroke}%
\pgfsetdash{}{0pt}%
\pgfpathmoveto{\pgfqpoint{1.008731in}{1.729761in}}%
\pgfpathlineto{\pgfqpoint{1.237952in}{1.862102in}}%
\pgfusepath{stroke}%
\end{pgfscope}%
\begin{pgfscope}%
\pgfpathrectangle{\pgfqpoint{0.338326in}{0.377767in}}{\pgfqpoint{2.257694in}{1.645262in}}%
\pgfusepath{clip}%
\pgfsetbuttcap%
\pgfsetroundjoin%
\pgfsetlinewidth{1.505625pt}%
\definecolor{currentstroke}{rgb}{0.000000,0.000000,0.000000}%
\pgfsetstrokecolor{currentstroke}%
\pgfsetdash{}{0pt}%
\pgfpathmoveto{\pgfqpoint{1.237952in}{1.068057in}}%
\pgfpathlineto{\pgfqpoint{1.467173in}{0.935716in}}%
\pgfusepath{stroke}%
\end{pgfscope}%
\begin{pgfscope}%
\pgfpathrectangle{\pgfqpoint{0.338326in}{0.377767in}}{\pgfqpoint{2.257694in}{1.645262in}}%
\pgfusepath{clip}%
\pgfsetbuttcap%
\pgfsetroundjoin%
\pgfsetlinewidth{1.505625pt}%
\definecolor{currentstroke}{rgb}{0.000000,0.000000,0.000000}%
\pgfsetstrokecolor{currentstroke}%
\pgfsetdash{}{0pt}%
\pgfpathmoveto{\pgfqpoint{1.237952in}{1.068057in}}%
\pgfpathlineto{\pgfqpoint{1.237952in}{1.332739in}}%
\pgfusepath{stroke}%
\end{pgfscope}%
\begin{pgfscope}%
\pgfpathrectangle{\pgfqpoint{0.338326in}{0.377767in}}{\pgfqpoint{2.257694in}{1.645262in}}%
\pgfusepath{clip}%
\pgfsetbuttcap%
\pgfsetroundjoin%
\pgfsetlinewidth{1.505625pt}%
\definecolor{currentstroke}{rgb}{0.000000,0.000000,0.000000}%
\pgfsetstrokecolor{currentstroke}%
\pgfsetdash{}{0pt}%
\pgfpathmoveto{\pgfqpoint{1.237952in}{1.332739in}}%
\pgfpathlineto{\pgfqpoint{1.467173in}{1.465079in}}%
\pgfusepath{stroke}%
\end{pgfscope}%
\begin{pgfscope}%
\pgfpathrectangle{\pgfqpoint{0.338326in}{0.377767in}}{\pgfqpoint{2.257694in}{1.645262in}}%
\pgfusepath{clip}%
\pgfsetbuttcap%
\pgfsetroundjoin%
\pgfsetlinewidth{1.505625pt}%
\definecolor{currentstroke}{rgb}{0.000000,0.000000,0.000000}%
\pgfsetstrokecolor{currentstroke}%
\pgfsetdash{}{0pt}%
\pgfpathmoveto{\pgfqpoint{1.467173in}{0.671034in}}%
\pgfpathlineto{\pgfqpoint{1.696394in}{0.538693in}}%
\pgfusepath{stroke}%
\end{pgfscope}%
\begin{pgfscope}%
\pgfpathrectangle{\pgfqpoint{0.338326in}{0.377767in}}{\pgfqpoint{2.257694in}{1.645262in}}%
\pgfusepath{clip}%
\pgfsetbuttcap%
\pgfsetroundjoin%
\pgfsetlinewidth{1.505625pt}%
\definecolor{currentstroke}{rgb}{0.000000,0.000000,0.000000}%
\pgfsetstrokecolor{currentstroke}%
\pgfsetdash{}{0pt}%
\pgfpathmoveto{\pgfqpoint{1.467173in}{0.671034in}}%
\pgfpathlineto{\pgfqpoint{1.467173in}{0.935716in}}%
\pgfusepath{stroke}%
\end{pgfscope}%
\begin{pgfscope}%
\pgfpathrectangle{\pgfqpoint{0.338326in}{0.377767in}}{\pgfqpoint{2.257694in}{1.645262in}}%
\pgfusepath{clip}%
\pgfsetbuttcap%
\pgfsetroundjoin%
\pgfsetlinewidth{1.505625pt}%
\definecolor{currentstroke}{rgb}{0.000000,0.000000,0.000000}%
\pgfsetstrokecolor{currentstroke}%
\pgfsetdash{}{0pt}%
\pgfpathmoveto{\pgfqpoint{1.467173in}{0.935716in}}%
\pgfpathlineto{\pgfqpoint{1.696394in}{1.068057in}}%
\pgfusepath{stroke}%
\end{pgfscope}%
\begin{pgfscope}%
\pgfpathrectangle{\pgfqpoint{0.338326in}{0.377767in}}{\pgfqpoint{2.257694in}{1.645262in}}%
\pgfusepath{clip}%
\pgfsetbuttcap%
\pgfsetroundjoin%
\pgfsetlinewidth{1.505625pt}%
\definecolor{currentstroke}{rgb}{0.000000,0.000000,0.000000}%
\pgfsetstrokecolor{currentstroke}%
\pgfsetdash{}{0pt}%
\pgfpathmoveto{\pgfqpoint{1.696394in}{0.538693in}}%
\pgfpathlineto{\pgfqpoint{1.925615in}{0.671034in}}%
\pgfusepath{stroke}%
\end{pgfscope}%
\begin{pgfscope}%
\pgfpathrectangle{\pgfqpoint{0.338326in}{0.377767in}}{\pgfqpoint{2.257694in}{1.645262in}}%
\pgfusepath{clip}%
\pgfsetbuttcap%
\pgfsetroundjoin%
\pgfsetlinewidth{1.505625pt}%
\definecolor{currentstroke}{rgb}{0.000000,0.000000,0.000000}%
\pgfsetstrokecolor{currentstroke}%
\pgfsetdash{}{0pt}%
\pgfpathmoveto{\pgfqpoint{1.237952in}{1.862102in}}%
\pgfpathlineto{\pgfqpoint{1.467173in}{1.729761in}}%
\pgfusepath{stroke}%
\end{pgfscope}%
\begin{pgfscope}%
\pgfpathrectangle{\pgfqpoint{0.338326in}{0.377767in}}{\pgfqpoint{2.257694in}{1.645262in}}%
\pgfusepath{clip}%
\pgfsetbuttcap%
\pgfsetroundjoin%
\pgfsetlinewidth{1.505625pt}%
\definecolor{currentstroke}{rgb}{0.000000,0.000000,0.000000}%
\pgfsetstrokecolor{currentstroke}%
\pgfsetdash{}{0pt}%
\pgfpathmoveto{\pgfqpoint{1.467173in}{1.465079in}}%
\pgfpathlineto{\pgfqpoint{1.696394in}{1.332739in}}%
\pgfusepath{stroke}%
\end{pgfscope}%
\begin{pgfscope}%
\pgfpathrectangle{\pgfqpoint{0.338326in}{0.377767in}}{\pgfqpoint{2.257694in}{1.645262in}}%
\pgfusepath{clip}%
\pgfsetbuttcap%
\pgfsetroundjoin%
\pgfsetlinewidth{1.505625pt}%
\definecolor{currentstroke}{rgb}{0.000000,0.000000,0.000000}%
\pgfsetstrokecolor{currentstroke}%
\pgfsetdash{}{0pt}%
\pgfpathmoveto{\pgfqpoint{1.467173in}{1.465079in}}%
\pgfpathlineto{\pgfqpoint{1.467173in}{1.729761in}}%
\pgfusepath{stroke}%
\end{pgfscope}%
\begin{pgfscope}%
\pgfpathrectangle{\pgfqpoint{0.338326in}{0.377767in}}{\pgfqpoint{2.257694in}{1.645262in}}%
\pgfusepath{clip}%
\pgfsetbuttcap%
\pgfsetroundjoin%
\pgfsetlinewidth{1.505625pt}%
\definecolor{currentstroke}{rgb}{0.000000,0.000000,0.000000}%
\pgfsetstrokecolor{currentstroke}%
\pgfsetdash{}{0pt}%
\pgfpathmoveto{\pgfqpoint{1.467173in}{1.729761in}}%
\pgfpathlineto{\pgfqpoint{1.696394in}{1.862102in}}%
\pgfusepath{stroke}%
\end{pgfscope}%
\begin{pgfscope}%
\pgfpathrectangle{\pgfqpoint{0.338326in}{0.377767in}}{\pgfqpoint{2.257694in}{1.645262in}}%
\pgfusepath{clip}%
\pgfsetbuttcap%
\pgfsetroundjoin%
\pgfsetlinewidth{1.505625pt}%
\definecolor{currentstroke}{rgb}{0.000000,0.000000,0.000000}%
\pgfsetstrokecolor{currentstroke}%
\pgfsetdash{}{0pt}%
\pgfpathmoveto{\pgfqpoint{1.696394in}{1.068057in}}%
\pgfpathlineto{\pgfqpoint{1.925615in}{0.935716in}}%
\pgfusepath{stroke}%
\end{pgfscope}%
\begin{pgfscope}%
\pgfpathrectangle{\pgfqpoint{0.338326in}{0.377767in}}{\pgfqpoint{2.257694in}{1.645262in}}%
\pgfusepath{clip}%
\pgfsetbuttcap%
\pgfsetroundjoin%
\pgfsetlinewidth{1.505625pt}%
\definecolor{currentstroke}{rgb}{0.000000,0.000000,0.000000}%
\pgfsetstrokecolor{currentstroke}%
\pgfsetdash{}{0pt}%
\pgfpathmoveto{\pgfqpoint{1.696394in}{1.068057in}}%
\pgfpathlineto{\pgfqpoint{1.696394in}{1.332739in}}%
\pgfusepath{stroke}%
\end{pgfscope}%
\begin{pgfscope}%
\pgfpathrectangle{\pgfqpoint{0.338326in}{0.377767in}}{\pgfqpoint{2.257694in}{1.645262in}}%
\pgfusepath{clip}%
\pgfsetbuttcap%
\pgfsetroundjoin%
\pgfsetlinewidth{1.505625pt}%
\definecolor{currentstroke}{rgb}{0.000000,0.000000,0.000000}%
\pgfsetstrokecolor{currentstroke}%
\pgfsetdash{}{0pt}%
\pgfpathmoveto{\pgfqpoint{1.696394in}{1.332739in}}%
\pgfpathlineto{\pgfqpoint{1.925615in}{1.465079in}}%
\pgfusepath{stroke}%
\end{pgfscope}%
\begin{pgfscope}%
\pgfpathrectangle{\pgfqpoint{0.338326in}{0.377767in}}{\pgfqpoint{2.257694in}{1.645262in}}%
\pgfusepath{clip}%
\pgfsetbuttcap%
\pgfsetroundjoin%
\pgfsetlinewidth{1.505625pt}%
\definecolor{currentstroke}{rgb}{0.000000,0.000000,0.000000}%
\pgfsetstrokecolor{currentstroke}%
\pgfsetdash{}{0pt}%
\pgfpathmoveto{\pgfqpoint{1.925615in}{0.671034in}}%
\pgfpathlineto{\pgfqpoint{1.925615in}{0.935716in}}%
\pgfusepath{stroke}%
\end{pgfscope}%
\begin{pgfscope}%
\pgfpathrectangle{\pgfqpoint{0.338326in}{0.377767in}}{\pgfqpoint{2.257694in}{1.645262in}}%
\pgfusepath{clip}%
\pgfsetbuttcap%
\pgfsetroundjoin%
\pgfsetlinewidth{1.505625pt}%
\definecolor{currentstroke}{rgb}{0.000000,0.000000,0.000000}%
\pgfsetstrokecolor{currentstroke}%
\pgfsetdash{}{0pt}%
\pgfpathmoveto{\pgfqpoint{1.925615in}{0.671034in}}%
\pgfpathlineto{\pgfqpoint{2.154837in}{0.538693in}}%
\pgfusepath{stroke}%
\end{pgfscope}%
\begin{pgfscope}%
\pgfpathrectangle{\pgfqpoint{0.338326in}{0.377767in}}{\pgfqpoint{2.257694in}{1.645262in}}%
\pgfusepath{clip}%
\pgfsetbuttcap%
\pgfsetroundjoin%
\pgfsetlinewidth{1.505625pt}%
\definecolor{currentstroke}{rgb}{0.000000,0.000000,0.000000}%
\pgfsetstrokecolor{currentstroke}%
\pgfsetdash{}{0pt}%
\pgfpathmoveto{\pgfqpoint{1.925615in}{0.935716in}}%
\pgfpathlineto{\pgfqpoint{2.154837in}{1.068057in}}%
\pgfusepath{stroke}%
\end{pgfscope}%
\begin{pgfscope}%
\pgfpathrectangle{\pgfqpoint{0.338326in}{0.377767in}}{\pgfqpoint{2.257694in}{1.645262in}}%
\pgfusepath{clip}%
\pgfsetbuttcap%
\pgfsetroundjoin%
\pgfsetlinewidth{1.505625pt}%
\definecolor{currentstroke}{rgb}{0.000000,0.000000,0.000000}%
\pgfsetstrokecolor{currentstroke}%
\pgfsetdash{}{0pt}%
\pgfpathmoveto{\pgfqpoint{2.154837in}{0.538693in}}%
\pgfpathlineto{\pgfqpoint{2.384058in}{0.671034in}}%
\pgfusepath{stroke}%
\end{pgfscope}%
\begin{pgfscope}%
\pgfpathrectangle{\pgfqpoint{0.338326in}{0.377767in}}{\pgfqpoint{2.257694in}{1.645262in}}%
\pgfusepath{clip}%
\pgfsetbuttcap%
\pgfsetroundjoin%
\pgfsetlinewidth{1.505625pt}%
\definecolor{currentstroke}{rgb}{0.000000,0.000000,0.000000}%
\pgfsetstrokecolor{currentstroke}%
\pgfsetdash{}{0pt}%
\pgfpathmoveto{\pgfqpoint{1.696394in}{1.862102in}}%
\pgfpathlineto{\pgfqpoint{1.925615in}{1.729761in}}%
\pgfusepath{stroke}%
\end{pgfscope}%
\begin{pgfscope}%
\pgfpathrectangle{\pgfqpoint{0.338326in}{0.377767in}}{\pgfqpoint{2.257694in}{1.645262in}}%
\pgfusepath{clip}%
\pgfsetbuttcap%
\pgfsetroundjoin%
\pgfsetlinewidth{1.505625pt}%
\definecolor{currentstroke}{rgb}{0.000000,0.000000,0.000000}%
\pgfsetstrokecolor{currentstroke}%
\pgfsetdash{}{0pt}%
\pgfpathmoveto{\pgfqpoint{1.925615in}{1.465079in}}%
\pgfpathlineto{\pgfqpoint{1.925615in}{1.729761in}}%
\pgfusepath{stroke}%
\end{pgfscope}%
\begin{pgfscope}%
\pgfpathrectangle{\pgfqpoint{0.338326in}{0.377767in}}{\pgfqpoint{2.257694in}{1.645262in}}%
\pgfusepath{clip}%
\pgfsetbuttcap%
\pgfsetroundjoin%
\pgfsetlinewidth{1.505625pt}%
\definecolor{currentstroke}{rgb}{0.000000,0.000000,0.000000}%
\pgfsetstrokecolor{currentstroke}%
\pgfsetdash{}{0pt}%
\pgfpathmoveto{\pgfqpoint{1.925615in}{1.465079in}}%
\pgfpathlineto{\pgfqpoint{2.154837in}{1.332739in}}%
\pgfusepath{stroke}%
\end{pgfscope}%
\begin{pgfscope}%
\pgfpathrectangle{\pgfqpoint{0.338326in}{0.377767in}}{\pgfqpoint{2.257694in}{1.645262in}}%
\pgfusepath{clip}%
\pgfsetbuttcap%
\pgfsetroundjoin%
\pgfsetlinewidth{1.505625pt}%
\definecolor{currentstroke}{rgb}{0.000000,0.000000,0.000000}%
\pgfsetstrokecolor{currentstroke}%
\pgfsetdash{}{0pt}%
\pgfpathmoveto{\pgfqpoint{1.925615in}{1.729761in}}%
\pgfpathlineto{\pgfqpoint{2.154837in}{1.862102in}}%
\pgfusepath{stroke}%
\end{pgfscope}%
\begin{pgfscope}%
\pgfpathrectangle{\pgfqpoint{0.338326in}{0.377767in}}{\pgfqpoint{2.257694in}{1.645262in}}%
\pgfusepath{clip}%
\pgfsetbuttcap%
\pgfsetroundjoin%
\pgfsetlinewidth{1.505625pt}%
\definecolor{currentstroke}{rgb}{0.000000,0.000000,0.000000}%
\pgfsetstrokecolor{currentstroke}%
\pgfsetdash{}{0pt}%
\pgfpathmoveto{\pgfqpoint{2.154837in}{1.068057in}}%
\pgfpathlineto{\pgfqpoint{2.384058in}{0.935716in}}%
\pgfusepath{stroke}%
\end{pgfscope}%
\begin{pgfscope}%
\pgfpathrectangle{\pgfqpoint{0.338326in}{0.377767in}}{\pgfqpoint{2.257694in}{1.645262in}}%
\pgfusepath{clip}%
\pgfsetbuttcap%
\pgfsetroundjoin%
\pgfsetlinewidth{1.505625pt}%
\definecolor{currentstroke}{rgb}{0.000000,0.000000,0.000000}%
\pgfsetstrokecolor{currentstroke}%
\pgfsetdash{}{0pt}%
\pgfpathmoveto{\pgfqpoint{2.154837in}{1.068057in}}%
\pgfpathlineto{\pgfqpoint{2.154837in}{1.332739in}}%
\pgfusepath{stroke}%
\end{pgfscope}%
\begin{pgfscope}%
\pgfpathrectangle{\pgfqpoint{0.338326in}{0.377767in}}{\pgfqpoint{2.257694in}{1.645262in}}%
\pgfusepath{clip}%
\pgfsetbuttcap%
\pgfsetroundjoin%
\pgfsetlinewidth{1.505625pt}%
\definecolor{currentstroke}{rgb}{0.000000,0.000000,0.000000}%
\pgfsetstrokecolor{currentstroke}%
\pgfsetdash{}{0pt}%
\pgfpathmoveto{\pgfqpoint{2.154837in}{1.332739in}}%
\pgfpathlineto{\pgfqpoint{2.384058in}{1.465079in}}%
\pgfusepath{stroke}%
\end{pgfscope}%
\begin{pgfscope}%
\pgfpathrectangle{\pgfqpoint{0.338326in}{0.377767in}}{\pgfqpoint{2.257694in}{1.645262in}}%
\pgfusepath{clip}%
\pgfsetbuttcap%
\pgfsetroundjoin%
\pgfsetlinewidth{1.505625pt}%
\definecolor{currentstroke}{rgb}{0.000000,0.000000,0.000000}%
\pgfsetstrokecolor{currentstroke}%
\pgfsetdash{}{0pt}%
\pgfpathmoveto{\pgfqpoint{2.384058in}{0.671034in}}%
\pgfpathlineto{\pgfqpoint{2.384058in}{0.935716in}}%
\pgfusepath{stroke}%
\end{pgfscope}%
\begin{pgfscope}%
\pgfpathrectangle{\pgfqpoint{0.338326in}{0.377767in}}{\pgfqpoint{2.257694in}{1.645262in}}%
\pgfusepath{clip}%
\pgfsetbuttcap%
\pgfsetroundjoin%
\pgfsetlinewidth{1.505625pt}%
\definecolor{currentstroke}{rgb}{0.000000,0.000000,0.000000}%
\pgfsetstrokecolor{currentstroke}%
\pgfsetdash{}{0pt}%
\pgfpathmoveto{\pgfqpoint{2.154837in}{1.862102in}}%
\pgfpathlineto{\pgfqpoint{2.384058in}{1.729761in}}%
\pgfusepath{stroke}%
\end{pgfscope}%
\begin{pgfscope}%
\pgfpathrectangle{\pgfqpoint{0.338326in}{0.377767in}}{\pgfqpoint{2.257694in}{1.645262in}}%
\pgfusepath{clip}%
\pgfsetbuttcap%
\pgfsetroundjoin%
\pgfsetlinewidth{1.505625pt}%
\definecolor{currentstroke}{rgb}{0.000000,0.000000,0.000000}%
\pgfsetstrokecolor{currentstroke}%
\pgfsetdash{}{0pt}%
\pgfpathmoveto{\pgfqpoint{2.384058in}{1.465079in}}%
\pgfpathlineto{\pgfqpoint{2.384058in}{1.729761in}}%
\pgfusepath{stroke}%
\end{pgfscope}%
\begin{pgfscope}%
\pgfpathrectangle{\pgfqpoint{0.338326in}{0.377767in}}{\pgfqpoint{2.257694in}{1.645262in}}%
\pgfusepath{clip}%
\pgfsetbuttcap%
\pgfsetroundjoin%
\definecolor{currentfill}{rgb}{0.247059,0.564706,0.854902}%
\pgfsetfillcolor{currentfill}%
\pgfsetlinewidth{1.003750pt}%
\definecolor{currentstroke}{rgb}{0.247059,0.564706,0.854902}%
\pgfsetstrokecolor{currentstroke}%
\pgfsetdash{}{0pt}%
\pgfsys@defobject{currentmarker}{\pgfqpoint{-0.023821in}{-0.023821in}}{\pgfqpoint{0.023821in}{0.023821in}}{%
\pgfpathmoveto{\pgfqpoint{0.000000in}{-0.023821in}}%
\pgfpathcurveto{\pgfqpoint{0.006317in}{-0.023821in}}{\pgfqpoint{0.012377in}{-0.021311in}}{\pgfqpoint{0.016844in}{-0.016844in}}%
\pgfpathcurveto{\pgfqpoint{0.021311in}{-0.012377in}}{\pgfqpoint{0.023821in}{-0.006317in}}{\pgfqpoint{0.023821in}{0.000000in}}%
\pgfpathcurveto{\pgfqpoint{0.023821in}{0.006317in}}{\pgfqpoint{0.021311in}{0.012377in}}{\pgfqpoint{0.016844in}{0.016844in}}%
\pgfpathcurveto{\pgfqpoint{0.012377in}{0.021311in}}{\pgfqpoint{0.006317in}{0.023821in}}{\pgfqpoint{0.000000in}{0.023821in}}%
\pgfpathcurveto{\pgfqpoint{-0.006317in}{0.023821in}}{\pgfqpoint{-0.012377in}{0.021311in}}{\pgfqpoint{-0.016844in}{0.016844in}}%
\pgfpathcurveto{\pgfqpoint{-0.021311in}{0.012377in}}{\pgfqpoint{-0.023821in}{0.006317in}}{\pgfqpoint{-0.023821in}{0.000000in}}%
\pgfpathcurveto{\pgfqpoint{-0.023821in}{-0.006317in}}{\pgfqpoint{-0.021311in}{-0.012377in}}{\pgfqpoint{-0.016844in}{-0.016844in}}%
\pgfpathcurveto{\pgfqpoint{-0.012377in}{-0.021311in}}{\pgfqpoint{-0.006317in}{-0.023821in}}{\pgfqpoint{0.000000in}{-0.023821in}}%
\pgfpathlineto{\pgfqpoint{0.000000in}{-0.023821in}}%
\pgfpathclose%
\pgfusepath{stroke,fill}%
}%
\begin{pgfscope}%
\pgfsys@transformshift{0.550288in}{0.671034in}%
\pgfsys@useobject{currentmarker}{}%
\end{pgfscope}%
\begin{pgfscope}%
\pgfsys@transformshift{0.550288in}{1.465079in}%
\pgfsys@useobject{currentmarker}{}%
\end{pgfscope}%
\begin{pgfscope}%
\pgfsys@transformshift{0.779510in}{1.068057in}%
\pgfsys@useobject{currentmarker}{}%
\end{pgfscope}%
\begin{pgfscope}%
\pgfsys@transformshift{1.008731in}{0.671034in}%
\pgfsys@useobject{currentmarker}{}%
\end{pgfscope}%
\begin{pgfscope}%
\pgfsys@transformshift{0.779510in}{1.862102in}%
\pgfsys@useobject{currentmarker}{}%
\end{pgfscope}%
\begin{pgfscope}%
\pgfsys@transformshift{1.008731in}{1.465079in}%
\pgfsys@useobject{currentmarker}{}%
\end{pgfscope}%
\begin{pgfscope}%
\pgfsys@transformshift{1.237952in}{1.068057in}%
\pgfsys@useobject{currentmarker}{}%
\end{pgfscope}%
\begin{pgfscope}%
\pgfsys@transformshift{1.467173in}{0.671034in}%
\pgfsys@useobject{currentmarker}{}%
\end{pgfscope}%
\begin{pgfscope}%
\pgfsys@transformshift{1.237952in}{1.862102in}%
\pgfsys@useobject{currentmarker}{}%
\end{pgfscope}%
\begin{pgfscope}%
\pgfsys@transformshift{1.467173in}{1.465079in}%
\pgfsys@useobject{currentmarker}{}%
\end{pgfscope}%
\begin{pgfscope}%
\pgfsys@transformshift{1.696394in}{1.068057in}%
\pgfsys@useobject{currentmarker}{}%
\end{pgfscope}%
\begin{pgfscope}%
\pgfsys@transformshift{1.925615in}{0.671034in}%
\pgfsys@useobject{currentmarker}{}%
\end{pgfscope}%
\begin{pgfscope}%
\pgfsys@transformshift{1.696394in}{1.862102in}%
\pgfsys@useobject{currentmarker}{}%
\end{pgfscope}%
\begin{pgfscope}%
\pgfsys@transformshift{1.925615in}{1.465079in}%
\pgfsys@useobject{currentmarker}{}%
\end{pgfscope}%
\begin{pgfscope}%
\pgfsys@transformshift{2.154837in}{1.068057in}%
\pgfsys@useobject{currentmarker}{}%
\end{pgfscope}%
\begin{pgfscope}%
\pgfsys@transformshift{2.384058in}{0.671034in}%
\pgfsys@useobject{currentmarker}{}%
\end{pgfscope}%
\begin{pgfscope}%
\pgfsys@transformshift{2.154837in}{1.862102in}%
\pgfsys@useobject{currentmarker}{}%
\end{pgfscope}%
\begin{pgfscope}%
\pgfsys@transformshift{2.384058in}{1.465079in}%
\pgfsys@useobject{currentmarker}{}%
\end{pgfscope}%
\end{pgfscope}%
\begin{pgfscope}%
\pgfpathrectangle{\pgfqpoint{0.338326in}{0.377767in}}{\pgfqpoint{2.257694in}{1.645262in}}%
\pgfusepath{clip}%
\pgfsetbuttcap%
\pgfsetroundjoin%
\definecolor{currentfill}{rgb}{1.000000,0.662745,0.054902}%
\pgfsetfillcolor{currentfill}%
\pgfsetlinewidth{1.003750pt}%
\definecolor{currentstroke}{rgb}{1.000000,0.662745,0.054902}%
\pgfsetstrokecolor{currentstroke}%
\pgfsetdash{}{0pt}%
\pgfsys@defobject{currentmarker}{\pgfqpoint{-0.023821in}{-0.023821in}}{\pgfqpoint{0.023821in}{0.023821in}}{%
\pgfpathmoveto{\pgfqpoint{0.000000in}{-0.023821in}}%
\pgfpathcurveto{\pgfqpoint{0.006317in}{-0.023821in}}{\pgfqpoint{0.012377in}{-0.021311in}}{\pgfqpoint{0.016844in}{-0.016844in}}%
\pgfpathcurveto{\pgfqpoint{0.021311in}{-0.012377in}}{\pgfqpoint{0.023821in}{-0.006317in}}{\pgfqpoint{0.023821in}{0.000000in}}%
\pgfpathcurveto{\pgfqpoint{0.023821in}{0.006317in}}{\pgfqpoint{0.021311in}{0.012377in}}{\pgfqpoint{0.016844in}{0.016844in}}%
\pgfpathcurveto{\pgfqpoint{0.012377in}{0.021311in}}{\pgfqpoint{0.006317in}{0.023821in}}{\pgfqpoint{0.000000in}{0.023821in}}%
\pgfpathcurveto{\pgfqpoint{-0.006317in}{0.023821in}}{\pgfqpoint{-0.012377in}{0.021311in}}{\pgfqpoint{-0.016844in}{0.016844in}}%
\pgfpathcurveto{\pgfqpoint{-0.021311in}{0.012377in}}{\pgfqpoint{-0.023821in}{0.006317in}}{\pgfqpoint{-0.023821in}{0.000000in}}%
\pgfpathcurveto{\pgfqpoint{-0.023821in}{-0.006317in}}{\pgfqpoint{-0.021311in}{-0.012377in}}{\pgfqpoint{-0.016844in}{-0.016844in}}%
\pgfpathcurveto{\pgfqpoint{-0.012377in}{-0.021311in}}{\pgfqpoint{-0.006317in}{-0.023821in}}{\pgfqpoint{0.000000in}{-0.023821in}}%
\pgfpathlineto{\pgfqpoint{0.000000in}{-0.023821in}}%
\pgfpathclose%
\pgfusepath{stroke,fill}%
}%
\begin{pgfscope}%
\pgfsys@transformshift{0.550288in}{0.935716in}%
\pgfsys@useobject{currentmarker}{}%
\end{pgfscope}%
\begin{pgfscope}%
\pgfsys@transformshift{0.779510in}{0.538693in}%
\pgfsys@useobject{currentmarker}{}%
\end{pgfscope}%
\begin{pgfscope}%
\pgfsys@transformshift{0.550288in}{1.729761in}%
\pgfsys@useobject{currentmarker}{}%
\end{pgfscope}%
\begin{pgfscope}%
\pgfsys@transformshift{0.779510in}{1.332739in}%
\pgfsys@useobject{currentmarker}{}%
\end{pgfscope}%
\begin{pgfscope}%
\pgfsys@transformshift{1.008731in}{0.935716in}%
\pgfsys@useobject{currentmarker}{}%
\end{pgfscope}%
\begin{pgfscope}%
\pgfsys@transformshift{1.237952in}{0.538693in}%
\pgfsys@useobject{currentmarker}{}%
\end{pgfscope}%
\begin{pgfscope}%
\pgfsys@transformshift{1.008731in}{1.729761in}%
\pgfsys@useobject{currentmarker}{}%
\end{pgfscope}%
\begin{pgfscope}%
\pgfsys@transformshift{1.237952in}{1.332739in}%
\pgfsys@useobject{currentmarker}{}%
\end{pgfscope}%
\begin{pgfscope}%
\pgfsys@transformshift{1.467173in}{0.935716in}%
\pgfsys@useobject{currentmarker}{}%
\end{pgfscope}%
\begin{pgfscope}%
\pgfsys@transformshift{1.696394in}{0.538693in}%
\pgfsys@useobject{currentmarker}{}%
\end{pgfscope}%
\begin{pgfscope}%
\pgfsys@transformshift{1.467173in}{1.729761in}%
\pgfsys@useobject{currentmarker}{}%
\end{pgfscope}%
\begin{pgfscope}%
\pgfsys@transformshift{1.696394in}{1.332739in}%
\pgfsys@useobject{currentmarker}{}%
\end{pgfscope}%
\begin{pgfscope}%
\pgfsys@transformshift{1.925615in}{0.935716in}%
\pgfsys@useobject{currentmarker}{}%
\end{pgfscope}%
\begin{pgfscope}%
\pgfsys@transformshift{2.154837in}{0.538693in}%
\pgfsys@useobject{currentmarker}{}%
\end{pgfscope}%
\begin{pgfscope}%
\pgfsys@transformshift{1.925615in}{1.729761in}%
\pgfsys@useobject{currentmarker}{}%
\end{pgfscope}%
\begin{pgfscope}%
\pgfsys@transformshift{2.154837in}{1.332739in}%
\pgfsys@useobject{currentmarker}{}%
\end{pgfscope}%
\begin{pgfscope}%
\pgfsys@transformshift{2.384058in}{0.935716in}%
\pgfsys@useobject{currentmarker}{}%
\end{pgfscope}%
\begin{pgfscope}%
\pgfsys@transformshift{2.384058in}{1.729761in}%
\pgfsys@useobject{currentmarker}{}%
\end{pgfscope}%
\end{pgfscope}%
\begin{pgfscope}%
\pgfpathrectangle{\pgfqpoint{0.338326in}{0.377767in}}{\pgfqpoint{2.257694in}{1.645262in}}%
\pgfusepath{clip}%
\pgfsetbuttcap%
\pgfsetroundjoin%
\definecolor{currentfill}{rgb}{0.247059,0.564706,0.854902}%
\pgfsetfillcolor{currentfill}%
\pgfsetlinewidth{1.003750pt}%
\definecolor{currentstroke}{rgb}{0.247059,0.564706,0.854902}%
\pgfsetstrokecolor{currentstroke}%
\pgfsetdash{}{0pt}%
\pgfsys@defobject{currentmarker}{\pgfqpoint{0.000000in}{0.000000in}}{\pgfqpoint{0.000000in}{0.000000in}}{%
\pgfpathmoveto{\pgfqpoint{0.000000in}{0.000000in}}%
\pgfpathcurveto{\pgfqpoint{0.000000in}{0.000000in}}{\pgfqpoint{0.000000in}{0.000000in}}{\pgfqpoint{0.000000in}{0.000000in}}%
\pgfpathcurveto{\pgfqpoint{0.000000in}{0.000000in}}{\pgfqpoint{0.000000in}{0.000000in}}{\pgfqpoint{0.000000in}{0.000000in}}%
\pgfpathcurveto{\pgfqpoint{0.000000in}{0.000000in}}{\pgfqpoint{0.000000in}{0.000000in}}{\pgfqpoint{0.000000in}{0.000000in}}%
\pgfpathcurveto{\pgfqpoint{0.000000in}{0.000000in}}{\pgfqpoint{0.000000in}{0.000000in}}{\pgfqpoint{0.000000in}{0.000000in}}%
\pgfpathcurveto{\pgfqpoint{0.000000in}{0.000000in}}{\pgfqpoint{0.000000in}{0.000000in}}{\pgfqpoint{0.000000in}{0.000000in}}%
\pgfpathcurveto{\pgfqpoint{0.000000in}{0.000000in}}{\pgfqpoint{0.000000in}{0.000000in}}{\pgfqpoint{0.000000in}{0.000000in}}%
\pgfpathcurveto{\pgfqpoint{0.000000in}{0.000000in}}{\pgfqpoint{0.000000in}{0.000000in}}{\pgfqpoint{0.000000in}{0.000000in}}%
\pgfpathcurveto{\pgfqpoint{0.000000in}{0.000000in}}{\pgfqpoint{0.000000in}{0.000000in}}{\pgfqpoint{0.000000in}{0.000000in}}%
\pgfpathlineto{\pgfqpoint{0.000000in}{0.000000in}}%
\pgfpathclose%
\pgfusepath{stroke,fill}%
}%
\begin{pgfscope}%
\pgfsys@transformshift{1.696394in}{1.597420in}%
\pgfsys@useobject{currentmarker}{}%
\end{pgfscope}%
\end{pgfscope}%
\begin{pgfscope}%
\pgfpathrectangle{\pgfqpoint{0.338326in}{0.377767in}}{\pgfqpoint{2.257694in}{1.645262in}}%
\pgfusepath{clip}%
\pgfsetbuttcap%
\pgfsetroundjoin%
\definecolor{currentfill}{rgb}{0.000000,0.000000,0.000000}%
\pgfsetfillcolor{currentfill}%
\pgfsetlinewidth{0.000000pt}%
\definecolor{currentstroke}{rgb}{0.000000,0.000000,0.000000}%
\pgfsetstrokecolor{currentstroke}%
\pgfsetdash{}{0pt}%
\pgfpathmoveto{\pgfqpoint{1.459841in}{1.204631in}}%
\pgfpathlineto{\pgfqpoint{1.650964in}{1.535665in}}%
\pgfpathlineto{\pgfqpoint{1.632066in}{1.536799in}}%
\pgfpathlineto{\pgfqpoint{1.696394in}{1.597420in}}%
\pgfpathlineto{\pgfqpoint{1.676059in}{1.511400in}}%
\pgfpathlineto{\pgfqpoint{1.665628in}{1.527198in}}%
\pgfpathlineto{\pgfqpoint{1.474505in}{1.196164in}}%
\pgfpathlineto{\pgfqpoint{1.459841in}{1.204631in}}%
\pgfusepath{fill}%
\end{pgfscope}%
\begin{pgfscope}%
\pgfpathrectangle{\pgfqpoint{0.338326in}{0.377767in}}{\pgfqpoint{2.257694in}{1.645262in}}%
\pgfusepath{clip}%
\pgfsetbuttcap%
\pgfsetroundjoin%
\definecolor{currentfill}{rgb}{1.000000,0.662745,0.054902}%
\pgfsetfillcolor{currentfill}%
\pgfsetlinewidth{1.003750pt}%
\definecolor{currentstroke}{rgb}{1.000000,0.662745,0.054902}%
\pgfsetstrokecolor{currentstroke}%
\pgfsetdash{}{0pt}%
\pgfsys@defobject{currentmarker}{\pgfqpoint{0.000000in}{0.000000in}}{\pgfqpoint{0.000000in}{0.000000in}}{%
\pgfpathmoveto{\pgfqpoint{0.000000in}{0.000000in}}%
\pgfpathcurveto{\pgfqpoint{0.000000in}{0.000000in}}{\pgfqpoint{0.000000in}{0.000000in}}{\pgfqpoint{0.000000in}{0.000000in}}%
\pgfpathcurveto{\pgfqpoint{0.000000in}{0.000000in}}{\pgfqpoint{0.000000in}{0.000000in}}{\pgfqpoint{0.000000in}{0.000000in}}%
\pgfpathcurveto{\pgfqpoint{0.000000in}{0.000000in}}{\pgfqpoint{0.000000in}{0.000000in}}{\pgfqpoint{0.000000in}{0.000000in}}%
\pgfpathcurveto{\pgfqpoint{0.000000in}{0.000000in}}{\pgfqpoint{0.000000in}{0.000000in}}{\pgfqpoint{0.000000in}{0.000000in}}%
\pgfpathcurveto{\pgfqpoint{0.000000in}{0.000000in}}{\pgfqpoint{0.000000in}{0.000000in}}{\pgfqpoint{0.000000in}{0.000000in}}%
\pgfpathcurveto{\pgfqpoint{0.000000in}{0.000000in}}{\pgfqpoint{0.000000in}{0.000000in}}{\pgfqpoint{0.000000in}{0.000000in}}%
\pgfpathcurveto{\pgfqpoint{0.000000in}{0.000000in}}{\pgfqpoint{0.000000in}{0.000000in}}{\pgfqpoint{0.000000in}{0.000000in}}%
\pgfpathcurveto{\pgfqpoint{0.000000in}{0.000000in}}{\pgfqpoint{0.000000in}{0.000000in}}{\pgfqpoint{0.000000in}{0.000000in}}%
\pgfpathlineto{\pgfqpoint{0.000000in}{0.000000in}}%
\pgfpathclose%
\pgfusepath{stroke,fill}%
}%
\begin{pgfscope}%
\pgfsys@transformshift{1.696394in}{0.803375in}%
\pgfsys@useobject{currentmarker}{}%
\end{pgfscope}%
\end{pgfscope}%
\begin{pgfscope}%
\pgfpathrectangle{\pgfqpoint{0.338326in}{0.377767in}}{\pgfqpoint{2.257694in}{1.645262in}}%
\pgfusepath{clip}%
\pgfsetbuttcap%
\pgfsetroundjoin%
\definecolor{currentfill}{rgb}{0.000000,0.000000,0.000000}%
\pgfsetfillcolor{currentfill}%
\pgfsetlinewidth{0.000000pt}%
\definecolor{currentstroke}{rgb}{0.000000,0.000000,0.000000}%
\pgfsetstrokecolor{currentstroke}%
\pgfsetdash{}{0pt}%
\pgfpathmoveto{\pgfqpoint{1.474505in}{1.204631in}}%
\pgfpathlineto{\pgfqpoint{1.665628in}{0.873597in}}%
\pgfpathlineto{\pgfqpoint{1.676059in}{0.889395in}}%
\pgfpathlineto{\pgfqpoint{1.696394in}{0.803375in}}%
\pgfpathlineto{\pgfqpoint{1.632066in}{0.863996in}}%
\pgfpathlineto{\pgfqpoint{1.650964in}{0.865130in}}%
\pgfpathlineto{\pgfqpoint{1.459841in}{1.196164in}}%
\pgfpathlineto{\pgfqpoint{1.474505in}{1.204631in}}%
\pgfusepath{fill}%
\end{pgfscope}%
\begin{pgfscope}%
\pgfpathrectangle{\pgfqpoint{0.338326in}{0.377767in}}{\pgfqpoint{2.257694in}{1.645262in}}%
\pgfusepath{clip}%
\pgfsetbuttcap%
\pgfsetroundjoin%
\definecolor{currentfill}{rgb}{0.741176,0.121569,0.003922}%
\pgfsetfillcolor{currentfill}%
\pgfsetlinewidth{1.003750pt}%
\definecolor{currentstroke}{rgb}{0.741176,0.121569,0.003922}%
\pgfsetstrokecolor{currentstroke}%
\pgfsetdash{}{0pt}%
\pgfsys@defobject{currentmarker}{\pgfqpoint{0.000000in}{0.000000in}}{\pgfqpoint{0.000000in}{0.000000in}}{%
\pgfpathmoveto{\pgfqpoint{0.000000in}{0.000000in}}%
\pgfpathcurveto{\pgfqpoint{0.000000in}{0.000000in}}{\pgfqpoint{0.000000in}{0.000000in}}{\pgfqpoint{0.000000in}{0.000000in}}%
\pgfpathcurveto{\pgfqpoint{0.000000in}{0.000000in}}{\pgfqpoint{0.000000in}{0.000000in}}{\pgfqpoint{0.000000in}{0.000000in}}%
\pgfpathcurveto{\pgfqpoint{0.000000in}{0.000000in}}{\pgfqpoint{0.000000in}{0.000000in}}{\pgfqpoint{0.000000in}{0.000000in}}%
\pgfpathcurveto{\pgfqpoint{0.000000in}{0.000000in}}{\pgfqpoint{0.000000in}{0.000000in}}{\pgfqpoint{0.000000in}{0.000000in}}%
\pgfpathcurveto{\pgfqpoint{0.000000in}{0.000000in}}{\pgfqpoint{0.000000in}{0.000000in}}{\pgfqpoint{0.000000in}{0.000000in}}%
\pgfpathcurveto{\pgfqpoint{0.000000in}{0.000000in}}{\pgfqpoint{0.000000in}{0.000000in}}{\pgfqpoint{0.000000in}{0.000000in}}%
\pgfpathcurveto{\pgfqpoint{0.000000in}{0.000000in}}{\pgfqpoint{0.000000in}{0.000000in}}{\pgfqpoint{0.000000in}{0.000000in}}%
\pgfpathcurveto{\pgfqpoint{0.000000in}{0.000000in}}{\pgfqpoint{0.000000in}{0.000000in}}{\pgfqpoint{0.000000in}{0.000000in}}%
\pgfpathlineto{\pgfqpoint{0.000000in}{0.000000in}}%
\pgfpathclose%
\pgfusepath{stroke,fill}%
}%
\begin{pgfscope}%
\pgfsys@transformshift{1.696394in}{1.332739in}%
\pgfsys@useobject{currentmarker}{}%
\end{pgfscope}%
\end{pgfscope}%
\begin{pgfscope}%
\pgfpathrectangle{\pgfqpoint{0.338326in}{0.377767in}}{\pgfqpoint{2.257694in}{1.645262in}}%
\pgfusepath{clip}%
\pgfsetbuttcap%
\pgfsetroundjoin%
\definecolor{currentfill}{rgb}{0.000000,0.000000,0.000000}%
\pgfsetfillcolor{currentfill}%
\pgfsetlinewidth{0.000000pt}%
\definecolor{currentstroke}{rgb}{0.000000,0.000000,0.000000}%
\pgfsetstrokecolor{currentstroke}%
\pgfsetdash{}{0pt}%
\pgfpathmoveto{\pgfqpoint{1.687928in}{1.068057in}}%
\pgfpathlineto{\pgfqpoint{1.687928in}{1.256541in}}%
\pgfpathlineto{\pgfqpoint{1.670995in}{1.248075in}}%
\pgfpathlineto{\pgfqpoint{1.696394in}{1.332739in}}%
\pgfpathlineto{\pgfqpoint{1.721793in}{1.248075in}}%
\pgfpathlineto{\pgfqpoint{1.704861in}{1.256541in}}%
\pgfpathlineto{\pgfqpoint{1.704861in}{1.068057in}}%
\pgfpathlineto{\pgfqpoint{1.687928in}{1.068057in}}%
\pgfusepath{fill}%
\end{pgfscope}%
\begin{pgfscope}%
\pgfpathrectangle{\pgfqpoint{0.338326in}{0.377767in}}{\pgfqpoint{2.257694in}{1.645262in}}%
\pgfusepath{clip}%
\pgfsetbuttcap%
\pgfsetroundjoin%
\pgfsetlinewidth{1.505625pt}%
\definecolor{currentstroke}{rgb}{0.000000,0.000000,0.000000}%
\pgfsetstrokecolor{currentstroke}%
\pgfsetdash{{5.550000pt}{2.400000pt}}{0.000000pt}%
\pgfpathmoveto{\pgfqpoint{1.696394in}{1.597420in}}%
\pgfpathlineto{\pgfqpoint{1.925615in}{1.200398in}}%
\pgfusepath{stroke}%
\end{pgfscope}%
\begin{pgfscope}%
\pgfpathrectangle{\pgfqpoint{0.338326in}{0.377767in}}{\pgfqpoint{2.257694in}{1.645262in}}%
\pgfusepath{clip}%
\pgfsetbuttcap%
\pgfsetroundjoin%
\pgfsetlinewidth{1.505625pt}%
\definecolor{currentstroke}{rgb}{0.000000,0.000000,0.000000}%
\pgfsetstrokecolor{currentstroke}%
\pgfsetdash{{5.550000pt}{2.400000pt}}{0.000000pt}%
\pgfpathmoveto{\pgfqpoint{1.696394in}{0.803375in}}%
\pgfpathlineto{\pgfqpoint{1.925615in}{1.200398in}}%
\pgfusepath{stroke}%
\end{pgfscope}%
\begin{pgfscope}%
\definecolor{textcolor}{rgb}{0.000000,0.000000,0.000000}%
\pgfsetstrokecolor{textcolor}%
\pgfsetfillcolor{textcolor}%
\pgftext[x=1.503143in,y=1.518016in,left,base]{\color{textcolor}{\sffamily\fontsize{9.163000}{10.995600}\selectfont\catcode`\^=\active\def^{\ifmmode\sp\else\^{}\fi}\catcode`\%=\active\def%{\%}$\mathbf{a}_1$}}%
\end{pgfscope}%
\begin{pgfscope}%
\definecolor{textcolor}{rgb}{0.000000,0.000000,0.000000}%
\pgfsetstrokecolor{textcolor}%
\pgfsetfillcolor{textcolor}%
\pgftext[x=1.503143in,y=0.803375in,left,base]{\color{textcolor}{\sffamily\fontsize{9.163000}{10.995600}\selectfont\catcode`\^=\active\def^{\ifmmode\sp\else\^{}\fi}\catcode`\%=\active\def%{\%}$\mathbf{a}_2$}}%
\end{pgfscope}%
\begin{pgfscope}%
\pgfsetbuttcap%
\pgfsetmiterjoin%
\definecolor{currentfill}{rgb}{1.000000,1.000000,1.000000}%
\pgfsetfillcolor{currentfill}%
\pgfsetfillopacity{0.800000}%
\pgfsetlinewidth{1.003750pt}%
\definecolor{currentstroke}{rgb}{0.800000,0.800000,0.800000}%
\pgfsetstrokecolor{currentstroke}%
\pgfsetstrokeopacity{0.800000}%
\pgfsetdash{}{0pt}%
\pgfpathmoveto{\pgfqpoint{0.445270in}{0.454156in}}%
\pgfpathlineto{\pgfqpoint{1.033465in}{0.454156in}}%
\pgfpathquadraticcurveto{\pgfqpoint{1.064020in}{0.454156in}}{\pgfqpoint{1.064020in}{0.484711in}}%
\pgfpathlineto{\pgfqpoint{1.064020in}{0.895378in}}%
\pgfpathquadraticcurveto{\pgfqpoint{1.064020in}{0.925933in}}{\pgfqpoint{1.033465in}{0.925933in}}%
\pgfpathlineto{\pgfqpoint{0.445270in}{0.925933in}}%
\pgfpathquadraticcurveto{\pgfqpoint{0.414715in}{0.925933in}}{\pgfqpoint{0.414715in}{0.895378in}}%
\pgfpathlineto{\pgfqpoint{0.414715in}{0.484711in}}%
\pgfpathquadraticcurveto{\pgfqpoint{0.414715in}{0.454156in}}{\pgfqpoint{0.445270in}{0.454156in}}%
\pgfpathlineto{\pgfqpoint{0.445270in}{0.454156in}}%
\pgfpathclose%
\pgfusepath{stroke,fill}%
\end{pgfscope}%
\begin{pgfscope}%
\pgfsetrectcap%
\pgfsetroundjoin%
\pgfsetlinewidth{0.000000pt}%
\definecolor{currentstroke}{rgb}{0.247059,0.564706,0.854902}%
\pgfsetstrokecolor{currentstroke}%
\pgfsetdash{}{0pt}%
\pgfpathmoveto{\pgfqpoint{0.475826in}{0.811350in}}%
\pgfpathlineto{\pgfqpoint{0.628604in}{0.811350in}}%
\pgfpathlineto{\pgfqpoint{0.781381in}{0.811350in}}%
\pgfusepath{}%
\end{pgfscope}%
\begin{pgfscope}%
\pgfsetbuttcap%
\pgfsetroundjoin%
\definecolor{currentfill}{rgb}{0.247059,0.564706,0.854902}%
\pgfsetfillcolor{currentfill}%
\pgfsetlinewidth{1.003750pt}%
\definecolor{currentstroke}{rgb}{0.247059,0.564706,0.854902}%
\pgfsetstrokecolor{currentstroke}%
\pgfsetdash{}{0pt}%
\pgfsys@defobject{currentmarker}{\pgfqpoint{-0.069444in}{-0.069444in}}{\pgfqpoint{0.069444in}{0.069444in}}{%
\pgfpathmoveto{\pgfqpoint{0.000000in}{-0.069444in}}%
\pgfpathcurveto{\pgfqpoint{0.018417in}{-0.069444in}}{\pgfqpoint{0.036082in}{-0.062127in}}{\pgfqpoint{0.049105in}{-0.049105in}}%
\pgfpathcurveto{\pgfqpoint{0.062127in}{-0.036082in}}{\pgfqpoint{0.069444in}{-0.018417in}}{\pgfqpoint{0.069444in}{0.000000in}}%
\pgfpathcurveto{\pgfqpoint{0.069444in}{0.018417in}}{\pgfqpoint{0.062127in}{0.036082in}}{\pgfqpoint{0.049105in}{0.049105in}}%
\pgfpathcurveto{\pgfqpoint{0.036082in}{0.062127in}}{\pgfqpoint{0.018417in}{0.069444in}}{\pgfqpoint{0.000000in}{0.069444in}}%
\pgfpathcurveto{\pgfqpoint{-0.018417in}{0.069444in}}{\pgfqpoint{-0.036082in}{0.062127in}}{\pgfqpoint{-0.049105in}{0.049105in}}%
\pgfpathcurveto{\pgfqpoint{-0.062127in}{0.036082in}}{\pgfqpoint{-0.069444in}{0.018417in}}{\pgfqpoint{-0.069444in}{0.000000in}}%
\pgfpathcurveto{\pgfqpoint{-0.069444in}{-0.018417in}}{\pgfqpoint{-0.062127in}{-0.036082in}}{\pgfqpoint{-0.049105in}{-0.049105in}}%
\pgfpathcurveto{\pgfqpoint{-0.036082in}{-0.062127in}}{\pgfqpoint{-0.018417in}{-0.069444in}}{\pgfqpoint{0.000000in}{-0.069444in}}%
\pgfpathlineto{\pgfqpoint{0.000000in}{-0.069444in}}%
\pgfpathclose%
\pgfusepath{stroke,fill}%
}%
\begin{pgfscope}%
\pgfsys@transformshift{0.628604in}{0.811350in}%
\pgfsys@useobject{currentmarker}{}%
\end{pgfscope}%
\end{pgfscope}%
\begin{pgfscope}%
\definecolor{textcolor}{rgb}{0.000000,0.000000,0.000000}%
\pgfsetstrokecolor{textcolor}%
\pgfsetfillcolor{textcolor}%
\pgftext[x=0.903604in,y=0.757878in,left,base]{\color{textcolor}{\sffamily\fontsize{11.000000}{13.200000}\selectfont\catcode`\^=\active\def^{\ifmmode\sp\else\^{}\fi}\catcode`\%=\active\def%{\%}A}}%
\end{pgfscope}%
\begin{pgfscope}%
\pgfsetrectcap%
\pgfsetroundjoin%
\pgfsetlinewidth{0.000000pt}%
\definecolor{currentstroke}{rgb}{1.000000,0.662745,0.054902}%
\pgfsetstrokecolor{currentstroke}%
\pgfsetdash{}{0pt}%
\pgfpathmoveto{\pgfqpoint{0.475826in}{0.598378in}}%
\pgfpathlineto{\pgfqpoint{0.628604in}{0.598378in}}%
\pgfpathlineto{\pgfqpoint{0.781381in}{0.598378in}}%
\pgfusepath{}%
\end{pgfscope}%
\begin{pgfscope}%
\pgfsetbuttcap%
\pgfsetroundjoin%
\definecolor{currentfill}{rgb}{1.000000,0.662745,0.054902}%
\pgfsetfillcolor{currentfill}%
\pgfsetlinewidth{1.003750pt}%
\definecolor{currentstroke}{rgb}{1.000000,0.662745,0.054902}%
\pgfsetstrokecolor{currentstroke}%
\pgfsetdash{}{0pt}%
\pgfsys@defobject{currentmarker}{\pgfqpoint{-0.069444in}{-0.069444in}}{\pgfqpoint{0.069444in}{0.069444in}}{%
\pgfpathmoveto{\pgfqpoint{0.000000in}{-0.069444in}}%
\pgfpathcurveto{\pgfqpoint{0.018417in}{-0.069444in}}{\pgfqpoint{0.036082in}{-0.062127in}}{\pgfqpoint{0.049105in}{-0.049105in}}%
\pgfpathcurveto{\pgfqpoint{0.062127in}{-0.036082in}}{\pgfqpoint{0.069444in}{-0.018417in}}{\pgfqpoint{0.069444in}{0.000000in}}%
\pgfpathcurveto{\pgfqpoint{0.069444in}{0.018417in}}{\pgfqpoint{0.062127in}{0.036082in}}{\pgfqpoint{0.049105in}{0.049105in}}%
\pgfpathcurveto{\pgfqpoint{0.036082in}{0.062127in}}{\pgfqpoint{0.018417in}{0.069444in}}{\pgfqpoint{0.000000in}{0.069444in}}%
\pgfpathcurveto{\pgfqpoint{-0.018417in}{0.069444in}}{\pgfqpoint{-0.036082in}{0.062127in}}{\pgfqpoint{-0.049105in}{0.049105in}}%
\pgfpathcurveto{\pgfqpoint{-0.062127in}{0.036082in}}{\pgfqpoint{-0.069444in}{0.018417in}}{\pgfqpoint{-0.069444in}{0.000000in}}%
\pgfpathcurveto{\pgfqpoint{-0.069444in}{-0.018417in}}{\pgfqpoint{-0.062127in}{-0.036082in}}{\pgfqpoint{-0.049105in}{-0.049105in}}%
\pgfpathcurveto{\pgfqpoint{-0.036082in}{-0.062127in}}{\pgfqpoint{-0.018417in}{-0.069444in}}{\pgfqpoint{0.000000in}{-0.069444in}}%
\pgfpathlineto{\pgfqpoint{0.000000in}{-0.069444in}}%
\pgfpathclose%
\pgfusepath{stroke,fill}%
}%
\begin{pgfscope}%
\pgfsys@transformshift{0.628604in}{0.598378in}%
\pgfsys@useobject{currentmarker}{}%
\end{pgfscope}%
\end{pgfscope}%
\begin{pgfscope}%
\definecolor{textcolor}{rgb}{0.000000,0.000000,0.000000}%
\pgfsetstrokecolor{textcolor}%
\pgfsetfillcolor{textcolor}%
\pgftext[x=0.903604in,y=0.544906in,left,base]{\color{textcolor}{\sffamily\fontsize{11.000000}{13.200000}\selectfont\catcode`\^=\active\def^{\ifmmode\sp\else\^{}\fi}\catcode`\%=\active\def%{\%}B}}%
\end{pgfscope}%
\end{pgfpicture}%
\makeatother%
\endgroup%

	\end{subfigure}%
	\begin{subfigure}[t]{0.5\textwidth}
		\centering
		\caption{\hfill\null}\label{sfig:graphene Brillouin zone}
		%% Creator: Matplotlib, PGF backend
%%
%% To include the figure in your LaTeX document, write
%%   \input{<filename>.pgf}
%%
%% Make sure the required packages are loaded in your preamble
%%   \usepackage{pgf}
%%
%% Also ensure that all the required font packages are loaded; for instance,
%% the lmodern package is sometimes necessary when using math font.
%%   \usepackage{lmodern}
%%
%% Figures using additional raster images can only be included by \input if
%% they are in the same directory as the main LaTeX file. For loading figures
%% from other directories you can use the `import` package
%%   \usepackage{import}
%%
%% and then include the figures with
%%   \import{<path to file>}{<filename>.pgf}
%%
%% Matplotlib used the following preamble
%%   \def\mathdefault#1{#1}
%%   \everymath=\expandafter{\the\everymath\displaystyle}
%%   \IfFileExists{scrextend.sty}{
%%     \usepackage[fontsize=11.000000pt]{scrextend}
%%   }{
%%     \renewcommand{\normalsize}{\fontsize{11.000000}{13.200000}\selectfont}
%%     \normalsize
%%   }
%%   \usepackage{fontspec}\usepackage{unicode-math}\setmathfont{texgyrepagella-math.otf}\setmainfont{texgyrepagella-math}
%%   \makeatletter\@ifpackageloaded{underscore}{}{\usepackage[strings]{underscore}}\makeatother
%%
\begingroup%
\makeatletter%
\begin{pgfpicture}%
\pgfpathrectangle{\pgfpointorigin}{\pgfqpoint{2.584000in}{2.584000in}}%
\pgfusepath{use as bounding box, clip}%
\begin{pgfscope}%
\pgfsetbuttcap%
\pgfsetmiterjoin%
\definecolor{currentfill}{rgb}{1.000000,1.000000,1.000000}%
\pgfsetfillcolor{currentfill}%
\pgfsetlinewidth{0.000000pt}%
\definecolor{currentstroke}{rgb}{1.000000,1.000000,1.000000}%
\pgfsetstrokecolor{currentstroke}%
\pgfsetdash{}{0pt}%
\pgfpathmoveto{\pgfqpoint{0.000000in}{0.000000in}}%
\pgfpathlineto{\pgfqpoint{2.584000in}{0.000000in}}%
\pgfpathlineto{\pgfqpoint{2.584000in}{2.584000in}}%
\pgfpathlineto{\pgfqpoint{0.000000in}{2.584000in}}%
\pgfpathlineto{\pgfqpoint{0.000000in}{0.000000in}}%
\pgfpathclose%
\pgfusepath{fill}%
\end{pgfscope}%
\begin{pgfscope}%
\pgfsetbuttcap%
\pgfsetmiterjoin%
\definecolor{currentfill}{rgb}{1.000000,1.000000,1.000000}%
\pgfsetfillcolor{currentfill}%
\pgfsetlinewidth{0.000000pt}%
\definecolor{currentstroke}{rgb}{0.000000,0.000000,0.000000}%
\pgfsetstrokecolor{currentstroke}%
\pgfsetstrokeopacity{0.000000}%
\pgfsetdash{}{0pt}%
\pgfpathmoveto{\pgfqpoint{0.462743in}{0.284240in}}%
\pgfpathlineto{\pgfqpoint{2.185857in}{0.284240in}}%
\pgfpathlineto{\pgfqpoint{2.185857in}{2.273920in}}%
\pgfpathlineto{\pgfqpoint{0.462743in}{2.273920in}}%
\pgfpathlineto{\pgfqpoint{0.462743in}{0.284240in}}%
\pgfpathclose%
\pgfusepath{fill}%
\end{pgfscope}%
\begin{pgfscope}%
\pgfpathrectangle{\pgfqpoint{0.462743in}{0.284240in}}{\pgfqpoint{1.723113in}{1.989680in}}%
\pgfusepath{clip}%
\pgfsetbuttcap%
\pgfsetroundjoin%
\definecolor{currentfill}{rgb}{0.000000,0.000000,0.000000}%
\pgfsetfillcolor{currentfill}%
\pgfsetlinewidth{1.003750pt}%
\definecolor{currentstroke}{rgb}{0.000000,0.000000,0.000000}%
\pgfsetstrokecolor{currentstroke}%
\pgfsetdash{}{0pt}%
\pgfsys@defobject{currentmarker}{\pgfqpoint{-0.020833in}{-0.020833in}}{\pgfqpoint{0.020833in}{0.020833in}}{%
\pgfpathmoveto{\pgfqpoint{0.000000in}{-0.020833in}}%
\pgfpathcurveto{\pgfqpoint{0.005525in}{-0.020833in}}{\pgfqpoint{0.010825in}{-0.018638in}}{\pgfqpoint{0.014731in}{-0.014731in}}%
\pgfpathcurveto{\pgfqpoint{0.018638in}{-0.010825in}}{\pgfqpoint{0.020833in}{-0.005525in}}{\pgfqpoint{0.020833in}{0.000000in}}%
\pgfpathcurveto{\pgfqpoint{0.020833in}{0.005525in}}{\pgfqpoint{0.018638in}{0.010825in}}{\pgfqpoint{0.014731in}{0.014731in}}%
\pgfpathcurveto{\pgfqpoint{0.010825in}{0.018638in}}{\pgfqpoint{0.005525in}{0.020833in}}{\pgfqpoint{0.000000in}{0.020833in}}%
\pgfpathcurveto{\pgfqpoint{-0.005525in}{0.020833in}}{\pgfqpoint{-0.010825in}{0.018638in}}{\pgfqpoint{-0.014731in}{0.014731in}}%
\pgfpathcurveto{\pgfqpoint{-0.018638in}{0.010825in}}{\pgfqpoint{-0.020833in}{0.005525in}}{\pgfqpoint{-0.020833in}{0.000000in}}%
\pgfpathcurveto{\pgfqpoint{-0.020833in}{-0.005525in}}{\pgfqpoint{-0.018638in}{-0.010825in}}{\pgfqpoint{-0.014731in}{-0.014731in}}%
\pgfpathcurveto{\pgfqpoint{-0.010825in}{-0.018638in}}{\pgfqpoint{-0.005525in}{-0.020833in}}{\pgfqpoint{0.000000in}{-0.020833in}}%
\pgfpathlineto{\pgfqpoint{0.000000in}{-0.020833in}}%
\pgfpathclose%
\pgfusepath{stroke,fill}%
}%
\begin{pgfscope}%
\pgfsys@transformshift{1.324300in}{1.279080in}%
\pgfsys@useobject{currentmarker}{}%
\end{pgfscope}%
\begin{pgfscope}%
\pgfsys@transformshift{0.541067in}{0.826880in}%
\pgfsys@useobject{currentmarker}{}%
\end{pgfscope}%
\begin{pgfscope}%
\pgfsys@transformshift{1.324300in}{0.374680in}%
\pgfsys@useobject{currentmarker}{}%
\end{pgfscope}%
\begin{pgfscope}%
\pgfsys@transformshift{0.541067in}{1.731280in}%
\pgfsys@useobject{currentmarker}{}%
\end{pgfscope}%
\begin{pgfscope}%
\pgfsys@transformshift{2.107533in}{0.826880in}%
\pgfsys@useobject{currentmarker}{}%
\end{pgfscope}%
\begin{pgfscope}%
\pgfsys@transformshift{1.324300in}{2.183480in}%
\pgfsys@useobject{currentmarker}{}%
\end{pgfscope}%
\begin{pgfscope}%
\pgfsys@transformshift{2.107533in}{1.731280in}%
\pgfsys@useobject{currentmarker}{}%
\end{pgfscope}%
\end{pgfscope}%
\begin{pgfscope}%
\pgfpathrectangle{\pgfqpoint{0.462743in}{0.284240in}}{\pgfqpoint{1.723113in}{1.989680in}}%
\pgfusepath{clip}%
\pgfsetbuttcap%
\pgfsetroundjoin%
\definecolor{currentfill}{rgb}{0.000000,0.000000,0.000000}%
\pgfsetfillcolor{currentfill}%
\pgfsetlinewidth{1.003750pt}%
\definecolor{currentstroke}{rgb}{0.000000,0.000000,0.000000}%
\pgfsetstrokecolor{currentstroke}%
\pgfsetdash{}{0pt}%
\pgfsys@defobject{currentmarker}{\pgfqpoint{-0.020833in}{-0.020833in}}{\pgfqpoint{0.020833in}{0.020833in}}{%
\pgfpathmoveto{\pgfqpoint{0.000000in}{-0.020833in}}%
\pgfpathcurveto{\pgfqpoint{0.005525in}{-0.020833in}}{\pgfqpoint{0.010825in}{-0.018638in}}{\pgfqpoint{0.014731in}{-0.014731in}}%
\pgfpathcurveto{\pgfqpoint{0.018638in}{-0.010825in}}{\pgfqpoint{0.020833in}{-0.005525in}}{\pgfqpoint{0.020833in}{0.000000in}}%
\pgfpathcurveto{\pgfqpoint{0.020833in}{0.005525in}}{\pgfqpoint{0.018638in}{0.010825in}}{\pgfqpoint{0.014731in}{0.014731in}}%
\pgfpathcurveto{\pgfqpoint{0.010825in}{0.018638in}}{\pgfqpoint{0.005525in}{0.020833in}}{\pgfqpoint{0.000000in}{0.020833in}}%
\pgfpathcurveto{\pgfqpoint{-0.005525in}{0.020833in}}{\pgfqpoint{-0.010825in}{0.018638in}}{\pgfqpoint{-0.014731in}{0.014731in}}%
\pgfpathcurveto{\pgfqpoint{-0.018638in}{0.010825in}}{\pgfqpoint{-0.020833in}{0.005525in}}{\pgfqpoint{-0.020833in}{0.000000in}}%
\pgfpathcurveto{\pgfqpoint{-0.020833in}{-0.005525in}}{\pgfqpoint{-0.018638in}{-0.010825in}}{\pgfqpoint{-0.014731in}{-0.014731in}}%
\pgfpathcurveto{\pgfqpoint{-0.010825in}{-0.018638in}}{\pgfqpoint{-0.005525in}{-0.020833in}}{\pgfqpoint{0.000000in}{-0.020833in}}%
\pgfpathlineto{\pgfqpoint{0.000000in}{-0.020833in}}%
\pgfpathclose%
\pgfusepath{stroke,fill}%
}%
\begin{pgfscope}%
\pgfsys@transformshift{0.802144in}{1.279080in}%
\pgfsys@useobject{currentmarker}{}%
\end{pgfscope}%
\begin{pgfscope}%
\pgfsys@transformshift{1.063222in}{1.731280in}%
\pgfsys@useobject{currentmarker}{}%
\end{pgfscope}%
\begin{pgfscope}%
\pgfsys@transformshift{1.063222in}{0.826880in}%
\pgfsys@useobject{currentmarker}{}%
\end{pgfscope}%
\begin{pgfscope}%
\pgfsys@transformshift{1.585378in}{0.826880in}%
\pgfsys@useobject{currentmarker}{}%
\end{pgfscope}%
\begin{pgfscope}%
\pgfsys@transformshift{1.846456in}{1.279080in}%
\pgfsys@useobject{currentmarker}{}%
\end{pgfscope}%
\begin{pgfscope}%
\pgfsys@transformshift{1.585378in}{1.731280in}%
\pgfsys@useobject{currentmarker}{}%
\end{pgfscope}%
\end{pgfscope}%
\begin{pgfscope}%
\pgfpathrectangle{\pgfqpoint{0.462743in}{0.284240in}}{\pgfqpoint{1.723113in}{1.989680in}}%
\pgfusepath{clip}%
\pgfsetbuttcap%
\pgfsetroundjoin%
\definecolor{currentfill}{rgb}{0.247059,0.564706,0.854902}%
\pgfsetfillcolor{currentfill}%
\pgfsetlinewidth{1.003750pt}%
\definecolor{currentstroke}{rgb}{0.247059,0.564706,0.854902}%
\pgfsetstrokecolor{currentstroke}%
\pgfsetdash{}{0pt}%
\pgfsys@defobject{currentmarker}{\pgfqpoint{0.000000in}{0.000000in}}{\pgfqpoint{0.000000in}{0.000000in}}{%
\pgfpathmoveto{\pgfqpoint{0.000000in}{0.000000in}}%
\pgfpathcurveto{\pgfqpoint{0.000000in}{0.000000in}}{\pgfqpoint{0.000000in}{0.000000in}}{\pgfqpoint{0.000000in}{0.000000in}}%
\pgfpathcurveto{\pgfqpoint{0.000000in}{0.000000in}}{\pgfqpoint{0.000000in}{0.000000in}}{\pgfqpoint{0.000000in}{0.000000in}}%
\pgfpathcurveto{\pgfqpoint{0.000000in}{0.000000in}}{\pgfqpoint{0.000000in}{0.000000in}}{\pgfqpoint{0.000000in}{0.000000in}}%
\pgfpathcurveto{\pgfqpoint{0.000000in}{0.000000in}}{\pgfqpoint{0.000000in}{0.000000in}}{\pgfqpoint{0.000000in}{0.000000in}}%
\pgfpathcurveto{\pgfqpoint{0.000000in}{0.000000in}}{\pgfqpoint{0.000000in}{0.000000in}}{\pgfqpoint{0.000000in}{0.000000in}}%
\pgfpathcurveto{\pgfqpoint{0.000000in}{0.000000in}}{\pgfqpoint{0.000000in}{0.000000in}}{\pgfqpoint{0.000000in}{0.000000in}}%
\pgfpathcurveto{\pgfqpoint{0.000000in}{0.000000in}}{\pgfqpoint{0.000000in}{0.000000in}}{\pgfqpoint{0.000000in}{0.000000in}}%
\pgfpathcurveto{\pgfqpoint{0.000000in}{0.000000in}}{\pgfqpoint{0.000000in}{0.000000in}}{\pgfqpoint{0.000000in}{0.000000in}}%
\pgfpathlineto{\pgfqpoint{0.000000in}{0.000000in}}%
\pgfpathclose%
\pgfusepath{stroke,fill}%
}%
\begin{pgfscope}%
\pgfsys@transformshift{2.107533in}{1.731280in}%
\pgfsys@useobject{currentmarker}{}%
\end{pgfscope}%
\end{pgfscope}%
\begin{pgfscope}%
\pgfpathrectangle{\pgfqpoint{0.462743in}{0.284240in}}{\pgfqpoint{1.723113in}{1.989680in}}%
\pgfusepath{clip}%
\pgfsetbuttcap%
\pgfsetroundjoin%
\definecolor{currentfill}{rgb}{0.000000,0.000000,0.000000}%
\pgfsetfillcolor{currentfill}%
\pgfsetlinewidth{0.000000pt}%
\definecolor{currentstroke}{rgb}{0.000000,0.000000,0.000000}%
\pgfsetstrokecolor{currentstroke}%
\pgfsetdash{}{0pt}%
\pgfpathmoveto{\pgfqpoint{1.321069in}{1.284676in}}%
\pgfpathlineto{\pgfqpoint{2.053939in}{1.707798in}}%
\pgfpathlineto{\pgfqpoint{2.041881in}{1.715760in}}%
\pgfpathlineto{\pgfqpoint{2.107533in}{1.731280in}}%
\pgfpathlineto{\pgfqpoint{2.061266in}{1.682184in}}%
\pgfpathlineto{\pgfqpoint{2.060400in}{1.696606in}}%
\pgfpathlineto{\pgfqpoint{1.327531in}{1.273484in}}%
\pgfpathlineto{\pgfqpoint{1.321069in}{1.284676in}}%
\pgfusepath{fill}%
\end{pgfscope}%
\begin{pgfscope}%
\pgfpathrectangle{\pgfqpoint{0.462743in}{0.284240in}}{\pgfqpoint{1.723113in}{1.989680in}}%
\pgfusepath{clip}%
\pgfsetbuttcap%
\pgfsetroundjoin%
\definecolor{currentfill}{rgb}{1.000000,0.662745,0.054902}%
\pgfsetfillcolor{currentfill}%
\pgfsetlinewidth{1.003750pt}%
\definecolor{currentstroke}{rgb}{1.000000,0.662745,0.054902}%
\pgfsetstrokecolor{currentstroke}%
\pgfsetdash{}{0pt}%
\pgfsys@defobject{currentmarker}{\pgfqpoint{0.000000in}{0.000000in}}{\pgfqpoint{0.000000in}{0.000000in}}{%
\pgfpathmoveto{\pgfqpoint{0.000000in}{0.000000in}}%
\pgfpathcurveto{\pgfqpoint{0.000000in}{0.000000in}}{\pgfqpoint{0.000000in}{0.000000in}}{\pgfqpoint{0.000000in}{0.000000in}}%
\pgfpathcurveto{\pgfqpoint{0.000000in}{0.000000in}}{\pgfqpoint{0.000000in}{0.000000in}}{\pgfqpoint{0.000000in}{0.000000in}}%
\pgfpathcurveto{\pgfqpoint{0.000000in}{0.000000in}}{\pgfqpoint{0.000000in}{0.000000in}}{\pgfqpoint{0.000000in}{0.000000in}}%
\pgfpathcurveto{\pgfqpoint{0.000000in}{0.000000in}}{\pgfqpoint{0.000000in}{0.000000in}}{\pgfqpoint{0.000000in}{0.000000in}}%
\pgfpathcurveto{\pgfqpoint{0.000000in}{0.000000in}}{\pgfqpoint{0.000000in}{0.000000in}}{\pgfqpoint{0.000000in}{0.000000in}}%
\pgfpathcurveto{\pgfqpoint{0.000000in}{0.000000in}}{\pgfqpoint{0.000000in}{0.000000in}}{\pgfqpoint{0.000000in}{0.000000in}}%
\pgfpathcurveto{\pgfqpoint{0.000000in}{0.000000in}}{\pgfqpoint{0.000000in}{0.000000in}}{\pgfqpoint{0.000000in}{0.000000in}}%
\pgfpathcurveto{\pgfqpoint{0.000000in}{0.000000in}}{\pgfqpoint{0.000000in}{0.000000in}}{\pgfqpoint{0.000000in}{0.000000in}}%
\pgfpathlineto{\pgfqpoint{0.000000in}{0.000000in}}%
\pgfpathclose%
\pgfusepath{stroke,fill}%
}%
\begin{pgfscope}%
\pgfsys@transformshift{2.107533in}{0.826880in}%
\pgfsys@useobject{currentmarker}{}%
\end{pgfscope}%
\end{pgfscope}%
\begin{pgfscope}%
\pgfpathrectangle{\pgfqpoint{0.462743in}{0.284240in}}{\pgfqpoint{1.723113in}{1.989680in}}%
\pgfusepath{clip}%
\pgfsetbuttcap%
\pgfsetroundjoin%
\definecolor{currentfill}{rgb}{0.000000,0.000000,0.000000}%
\pgfsetfillcolor{currentfill}%
\pgfsetlinewidth{0.000000pt}%
\definecolor{currentstroke}{rgb}{0.000000,0.000000,0.000000}%
\pgfsetstrokecolor{currentstroke}%
\pgfsetdash{}{0pt}%
\pgfpathmoveto{\pgfqpoint{1.327531in}{1.284676in}}%
\pgfpathlineto{\pgfqpoint{2.060400in}{0.861554in}}%
\pgfpathlineto{\pgfqpoint{2.061266in}{0.875976in}}%
\pgfpathlineto{\pgfqpoint{2.107533in}{0.826880in}}%
\pgfpathlineto{\pgfqpoint{2.041881in}{0.842400in}}%
\pgfpathlineto{\pgfqpoint{2.053939in}{0.850362in}}%
\pgfpathlineto{\pgfqpoint{1.321069in}{1.273484in}}%
\pgfpathlineto{\pgfqpoint{1.327531in}{1.284676in}}%
\pgfusepath{fill}%
\end{pgfscope}%
\begin{pgfscope}%
\pgfpathrectangle{\pgfqpoint{0.462743in}{0.284240in}}{\pgfqpoint{1.723113in}{1.989680in}}%
\pgfusepath{clip}%
\pgfsetbuttcap%
\pgfsetroundjoin%
\definecolor{currentfill}{rgb}{0.000000,0.000000,0.000000}%
\pgfsetfillcolor{currentfill}%
\pgfsetlinewidth{1.003750pt}%
\definecolor{currentstroke}{rgb}{0.000000,0.000000,0.000000}%
\pgfsetstrokecolor{currentstroke}%
\pgfsetdash{}{0pt}%
\pgfsys@defobject{currentmarker}{\pgfqpoint{-0.069444in}{-0.069444in}}{\pgfqpoint{0.069444in}{0.069444in}}{%
\pgfpathmoveto{\pgfqpoint{0.000000in}{-0.069444in}}%
\pgfpathcurveto{\pgfqpoint{0.018417in}{-0.069444in}}{\pgfqpoint{0.036082in}{-0.062127in}}{\pgfqpoint{0.049105in}{-0.049105in}}%
\pgfpathcurveto{\pgfqpoint{0.062127in}{-0.036082in}}{\pgfqpoint{0.069444in}{-0.018417in}}{\pgfqpoint{0.069444in}{0.000000in}}%
\pgfpathcurveto{\pgfqpoint{0.069444in}{0.018417in}}{\pgfqpoint{0.062127in}{0.036082in}}{\pgfqpoint{0.049105in}{0.049105in}}%
\pgfpathcurveto{\pgfqpoint{0.036082in}{0.062127in}}{\pgfqpoint{0.018417in}{0.069444in}}{\pgfqpoint{0.000000in}{0.069444in}}%
\pgfpathcurveto{\pgfqpoint{-0.018417in}{0.069444in}}{\pgfqpoint{-0.036082in}{0.062127in}}{\pgfqpoint{-0.049105in}{0.049105in}}%
\pgfpathcurveto{\pgfqpoint{-0.062127in}{0.036082in}}{\pgfqpoint{-0.069444in}{0.018417in}}{\pgfqpoint{-0.069444in}{0.000000in}}%
\pgfpathcurveto{\pgfqpoint{-0.069444in}{-0.018417in}}{\pgfqpoint{-0.062127in}{-0.036082in}}{\pgfqpoint{-0.049105in}{-0.049105in}}%
\pgfpathcurveto{\pgfqpoint{-0.036082in}{-0.062127in}}{\pgfqpoint{-0.018417in}{-0.069444in}}{\pgfqpoint{0.000000in}{-0.069444in}}%
\pgfpathlineto{\pgfqpoint{0.000000in}{-0.069444in}}%
\pgfpathclose%
\pgfusepath{stroke,fill}%
}%
\begin{pgfscope}%
\pgfsys@transformshift{1.324300in}{1.279080in}%
\pgfsys@useobject{currentmarker}{}%
\end{pgfscope}%
\end{pgfscope}%
\begin{pgfscope}%
\pgfpathrectangle{\pgfqpoint{0.462743in}{0.284240in}}{\pgfqpoint{1.723113in}{1.989680in}}%
\pgfusepath{clip}%
\pgfsetbuttcap%
\pgfsetroundjoin%
\definecolor{currentfill}{rgb}{0.000000,0.000000,0.000000}%
\pgfsetfillcolor{currentfill}%
\pgfsetlinewidth{1.003750pt}%
\definecolor{currentstroke}{rgb}{0.000000,0.000000,0.000000}%
\pgfsetstrokecolor{currentstroke}%
\pgfsetdash{}{0pt}%
\pgfsys@defobject{currentmarker}{\pgfqpoint{-0.069444in}{-0.069444in}}{\pgfqpoint{0.069444in}{0.069444in}}{%
\pgfpathmoveto{\pgfqpoint{0.000000in}{-0.069444in}}%
\pgfpathcurveto{\pgfqpoint{0.018417in}{-0.069444in}}{\pgfqpoint{0.036082in}{-0.062127in}}{\pgfqpoint{0.049105in}{-0.049105in}}%
\pgfpathcurveto{\pgfqpoint{0.062127in}{-0.036082in}}{\pgfqpoint{0.069444in}{-0.018417in}}{\pgfqpoint{0.069444in}{0.000000in}}%
\pgfpathcurveto{\pgfqpoint{0.069444in}{0.018417in}}{\pgfqpoint{0.062127in}{0.036082in}}{\pgfqpoint{0.049105in}{0.049105in}}%
\pgfpathcurveto{\pgfqpoint{0.036082in}{0.062127in}}{\pgfqpoint{0.018417in}{0.069444in}}{\pgfqpoint{0.000000in}{0.069444in}}%
\pgfpathcurveto{\pgfqpoint{-0.018417in}{0.069444in}}{\pgfqpoint{-0.036082in}{0.062127in}}{\pgfqpoint{-0.049105in}{0.049105in}}%
\pgfpathcurveto{\pgfqpoint{-0.062127in}{0.036082in}}{\pgfqpoint{-0.069444in}{0.018417in}}{\pgfqpoint{-0.069444in}{0.000000in}}%
\pgfpathcurveto{\pgfqpoint{-0.069444in}{-0.018417in}}{\pgfqpoint{-0.062127in}{-0.036082in}}{\pgfqpoint{-0.049105in}{-0.049105in}}%
\pgfpathcurveto{\pgfqpoint{-0.036082in}{-0.062127in}}{\pgfqpoint{-0.018417in}{-0.069444in}}{\pgfqpoint{0.000000in}{-0.069444in}}%
\pgfpathlineto{\pgfqpoint{0.000000in}{-0.069444in}}%
\pgfpathclose%
\pgfusepath{stroke,fill}%
}%
\begin{pgfscope}%
\pgfsys@transformshift{1.715917in}{1.505180in}%
\pgfsys@useobject{currentmarker}{}%
\end{pgfscope}%
\end{pgfscope}%
\begin{pgfscope}%
\pgfpathrectangle{\pgfqpoint{0.462743in}{0.284240in}}{\pgfqpoint{1.723113in}{1.989680in}}%
\pgfusepath{clip}%
\pgfsetbuttcap%
\pgfsetroundjoin%
\definecolor{currentfill}{rgb}{0.000000,0.000000,0.000000}%
\pgfsetfillcolor{currentfill}%
\pgfsetlinewidth{1.003750pt}%
\definecolor{currentstroke}{rgb}{0.000000,0.000000,0.000000}%
\pgfsetstrokecolor{currentstroke}%
\pgfsetdash{}{0pt}%
\pgfsys@defobject{currentmarker}{\pgfqpoint{-0.069444in}{-0.069444in}}{\pgfqpoint{0.069444in}{0.069444in}}{%
\pgfpathmoveto{\pgfqpoint{0.000000in}{-0.069444in}}%
\pgfpathcurveto{\pgfqpoint{0.018417in}{-0.069444in}}{\pgfqpoint{0.036082in}{-0.062127in}}{\pgfqpoint{0.049105in}{-0.049105in}}%
\pgfpathcurveto{\pgfqpoint{0.062127in}{-0.036082in}}{\pgfqpoint{0.069444in}{-0.018417in}}{\pgfqpoint{0.069444in}{0.000000in}}%
\pgfpathcurveto{\pgfqpoint{0.069444in}{0.018417in}}{\pgfqpoint{0.062127in}{0.036082in}}{\pgfqpoint{0.049105in}{0.049105in}}%
\pgfpathcurveto{\pgfqpoint{0.036082in}{0.062127in}}{\pgfqpoint{0.018417in}{0.069444in}}{\pgfqpoint{0.000000in}{0.069444in}}%
\pgfpathcurveto{\pgfqpoint{-0.018417in}{0.069444in}}{\pgfqpoint{-0.036082in}{0.062127in}}{\pgfqpoint{-0.049105in}{0.049105in}}%
\pgfpathcurveto{\pgfqpoint{-0.062127in}{0.036082in}}{\pgfqpoint{-0.069444in}{0.018417in}}{\pgfqpoint{-0.069444in}{0.000000in}}%
\pgfpathcurveto{\pgfqpoint{-0.069444in}{-0.018417in}}{\pgfqpoint{-0.062127in}{-0.036082in}}{\pgfqpoint{-0.049105in}{-0.049105in}}%
\pgfpathcurveto{\pgfqpoint{-0.036082in}{-0.062127in}}{\pgfqpoint{-0.018417in}{-0.069444in}}{\pgfqpoint{0.000000in}{-0.069444in}}%
\pgfpathlineto{\pgfqpoint{0.000000in}{-0.069444in}}%
\pgfpathclose%
\pgfusepath{stroke,fill}%
}%
\begin{pgfscope}%
\pgfsys@transformshift{1.846456in}{1.279080in}%
\pgfsys@useobject{currentmarker}{}%
\end{pgfscope}%
\end{pgfscope}%
\begin{pgfscope}%
\pgfpathrectangle{\pgfqpoint{0.462743in}{0.284240in}}{\pgfqpoint{1.723113in}{1.989680in}}%
\pgfusepath{clip}%
\pgfsetbuttcap%
\pgfsetroundjoin%
\pgfsetlinewidth{0.501875pt}%
\definecolor{currentstroke}{rgb}{0.000000,0.000000,0.000000}%
\pgfsetstrokecolor{currentstroke}%
\pgfsetdash{}{0pt}%
\pgfpathmoveto{\pgfqpoint{1.324300in}{1.279080in}}%
\pgfpathlineto{\pgfqpoint{1.324300in}{1.279080in}}%
\pgfusepath{stroke}%
\end{pgfscope}%
\begin{pgfscope}%
\pgfpathrectangle{\pgfqpoint{0.462743in}{0.284240in}}{\pgfqpoint{1.723113in}{1.989680in}}%
\pgfusepath{clip}%
\pgfsetbuttcap%
\pgfsetroundjoin%
\pgfsetlinewidth{0.501875pt}%
\definecolor{currentstroke}{rgb}{0.000000,0.000000,0.000000}%
\pgfsetstrokecolor{currentstroke}%
\pgfsetdash{}{0pt}%
\pgfpathmoveto{\pgfqpoint{1.324300in}{1.279080in}}%
\pgfpathlineto{\pgfqpoint{0.541067in}{0.826880in}}%
\pgfusepath{stroke}%
\end{pgfscope}%
\begin{pgfscope}%
\pgfpathrectangle{\pgfqpoint{0.462743in}{0.284240in}}{\pgfqpoint{1.723113in}{1.989680in}}%
\pgfusepath{clip}%
\pgfsetbuttcap%
\pgfsetroundjoin%
\pgfsetlinewidth{0.501875pt}%
\definecolor{currentstroke}{rgb}{0.000000,0.000000,0.000000}%
\pgfsetstrokecolor{currentstroke}%
\pgfsetdash{}{0pt}%
\pgfpathmoveto{\pgfqpoint{1.324300in}{1.279080in}}%
\pgfpathlineto{\pgfqpoint{1.324300in}{0.374680in}}%
\pgfusepath{stroke}%
\end{pgfscope}%
\begin{pgfscope}%
\pgfpathrectangle{\pgfqpoint{0.462743in}{0.284240in}}{\pgfqpoint{1.723113in}{1.989680in}}%
\pgfusepath{clip}%
\pgfsetbuttcap%
\pgfsetroundjoin%
\pgfsetlinewidth{0.501875pt}%
\definecolor{currentstroke}{rgb}{0.000000,0.000000,0.000000}%
\pgfsetstrokecolor{currentstroke}%
\pgfsetdash{}{0pt}%
\pgfpathmoveto{\pgfqpoint{1.324300in}{1.279080in}}%
\pgfpathlineto{\pgfqpoint{0.541067in}{1.731280in}}%
\pgfusepath{stroke}%
\end{pgfscope}%
\begin{pgfscope}%
\pgfpathrectangle{\pgfqpoint{0.462743in}{0.284240in}}{\pgfqpoint{1.723113in}{1.989680in}}%
\pgfusepath{clip}%
\pgfsetbuttcap%
\pgfsetroundjoin%
\pgfsetlinewidth{0.501875pt}%
\definecolor{currentstroke}{rgb}{0.000000,0.000000,0.000000}%
\pgfsetstrokecolor{currentstroke}%
\pgfsetdash{}{0pt}%
\pgfpathmoveto{\pgfqpoint{1.324300in}{1.279080in}}%
\pgfpathlineto{\pgfqpoint{2.107533in}{0.826880in}}%
\pgfusepath{stroke}%
\end{pgfscope}%
\begin{pgfscope}%
\pgfpathrectangle{\pgfqpoint{0.462743in}{0.284240in}}{\pgfqpoint{1.723113in}{1.989680in}}%
\pgfusepath{clip}%
\pgfsetbuttcap%
\pgfsetroundjoin%
\pgfsetlinewidth{0.501875pt}%
\definecolor{currentstroke}{rgb}{0.000000,0.000000,0.000000}%
\pgfsetstrokecolor{currentstroke}%
\pgfsetdash{}{0pt}%
\pgfpathmoveto{\pgfqpoint{1.324300in}{1.279080in}}%
\pgfpathlineto{\pgfqpoint{1.324300in}{2.183480in}}%
\pgfusepath{stroke}%
\end{pgfscope}%
\begin{pgfscope}%
\pgfpathrectangle{\pgfqpoint{0.462743in}{0.284240in}}{\pgfqpoint{1.723113in}{1.989680in}}%
\pgfusepath{clip}%
\pgfsetbuttcap%
\pgfsetroundjoin%
\pgfsetlinewidth{0.501875pt}%
\definecolor{currentstroke}{rgb}{0.000000,0.000000,0.000000}%
\pgfsetstrokecolor{currentstroke}%
\pgfsetdash{}{0pt}%
\pgfpathmoveto{\pgfqpoint{1.324300in}{1.279080in}}%
\pgfpathlineto{\pgfqpoint{2.107533in}{1.731280in}}%
\pgfusepath{stroke}%
\end{pgfscope}%
\begin{pgfscope}%
\pgfpathrectangle{\pgfqpoint{0.462743in}{0.284240in}}{\pgfqpoint{1.723113in}{1.989680in}}%
\pgfusepath{clip}%
\pgfsetbuttcap%
\pgfsetroundjoin%
\pgfsetlinewidth{1.003750pt}%
\definecolor{currentstroke}{rgb}{0.000000,0.000000,0.000000}%
\pgfsetstrokecolor{currentstroke}%
\pgfsetdash{}{0pt}%
\pgfpathmoveto{\pgfqpoint{0.802144in}{1.279080in}}%
\pgfpathlineto{\pgfqpoint{1.063222in}{1.731280in}}%
\pgfusepath{stroke}%
\end{pgfscope}%
\begin{pgfscope}%
\pgfpathrectangle{\pgfqpoint{0.462743in}{0.284240in}}{\pgfqpoint{1.723113in}{1.989680in}}%
\pgfusepath{clip}%
\pgfsetbuttcap%
\pgfsetroundjoin%
\pgfsetlinewidth{1.003750pt}%
\definecolor{currentstroke}{rgb}{0.000000,0.000000,0.000000}%
\pgfsetstrokecolor{currentstroke}%
\pgfsetdash{}{0pt}%
\pgfpathmoveto{\pgfqpoint{1.063222in}{1.731280in}}%
\pgfpathlineto{\pgfqpoint{1.585378in}{1.731280in}}%
\pgfusepath{stroke}%
\end{pgfscope}%
\begin{pgfscope}%
\pgfpathrectangle{\pgfqpoint{0.462743in}{0.284240in}}{\pgfqpoint{1.723113in}{1.989680in}}%
\pgfusepath{clip}%
\pgfsetbuttcap%
\pgfsetroundjoin%
\pgfsetlinewidth{1.003750pt}%
\definecolor{currentstroke}{rgb}{0.000000,0.000000,0.000000}%
\pgfsetstrokecolor{currentstroke}%
\pgfsetdash{}{0pt}%
\pgfpathmoveto{\pgfqpoint{1.063222in}{0.826880in}}%
\pgfpathlineto{\pgfqpoint{1.585378in}{0.826880in}}%
\pgfusepath{stroke}%
\end{pgfscope}%
\begin{pgfscope}%
\pgfpathrectangle{\pgfqpoint{0.462743in}{0.284240in}}{\pgfqpoint{1.723113in}{1.989680in}}%
\pgfusepath{clip}%
\pgfsetbuttcap%
\pgfsetroundjoin%
\pgfsetlinewidth{1.003750pt}%
\definecolor{currentstroke}{rgb}{0.000000,0.000000,0.000000}%
\pgfsetstrokecolor{currentstroke}%
\pgfsetdash{}{0pt}%
\pgfpathmoveto{\pgfqpoint{1.585378in}{0.826880in}}%
\pgfpathlineto{\pgfqpoint{1.846456in}{1.279080in}}%
\pgfusepath{stroke}%
\end{pgfscope}%
\begin{pgfscope}%
\pgfpathrectangle{\pgfqpoint{0.462743in}{0.284240in}}{\pgfqpoint{1.723113in}{1.989680in}}%
\pgfusepath{clip}%
\pgfsetbuttcap%
\pgfsetroundjoin%
\pgfsetlinewidth{1.003750pt}%
\definecolor{currentstroke}{rgb}{0.000000,0.000000,0.000000}%
\pgfsetstrokecolor{currentstroke}%
\pgfsetdash{}{0pt}%
\pgfpathmoveto{\pgfqpoint{0.802144in}{1.279080in}}%
\pgfpathlineto{\pgfqpoint{1.063222in}{0.826880in}}%
\pgfusepath{stroke}%
\end{pgfscope}%
\begin{pgfscope}%
\pgfpathrectangle{\pgfqpoint{0.462743in}{0.284240in}}{\pgfqpoint{1.723113in}{1.989680in}}%
\pgfusepath{clip}%
\pgfsetbuttcap%
\pgfsetroundjoin%
\pgfsetlinewidth{1.003750pt}%
\definecolor{currentstroke}{rgb}{0.000000,0.000000,0.000000}%
\pgfsetstrokecolor{currentstroke}%
\pgfsetdash{}{0pt}%
\pgfpathmoveto{\pgfqpoint{1.846456in}{1.279080in}}%
\pgfpathlineto{\pgfqpoint{1.585378in}{1.731280in}}%
\pgfusepath{stroke}%
\end{pgfscope}%
\begin{pgfscope}%
\pgfpathrectangle{\pgfqpoint{0.462743in}{0.284240in}}{\pgfqpoint{1.723113in}{1.989680in}}%
\pgfusepath{clip}%
\pgfsetrectcap%
\pgfsetroundjoin%
\pgfsetlinewidth{1.505625pt}%
\definecolor{currentstroke}{rgb}{0.000000,0.000000,0.000000}%
\pgfsetstrokecolor{currentstroke}%
\pgfsetdash{}{0pt}%
\pgfpathmoveto{\pgfqpoint{1.324300in}{1.279080in}}%
\pgfpathlineto{\pgfqpoint{1.715917in}{1.505180in}}%
\pgfpathlineto{\pgfqpoint{1.846456in}{1.279080in}}%
\pgfpathlineto{\pgfqpoint{1.324300in}{1.279080in}}%
\pgfusepath{stroke}%
\end{pgfscope}%
\begin{pgfscope}%
\definecolor{textcolor}{rgb}{0.000000,0.000000,0.000000}%
\pgfsetstrokecolor{textcolor}%
\pgfsetfillcolor{textcolor}%
\pgftext[x=1.352078in,y=1.417969in,left,base]{\color{textcolor}{\rmfamily\fontsize{13.200000}{15.840000}\selectfont\catcode`\^=\active\def^{\ifmmode\sp\else\^{}\fi}\catcode`\%=\active\def%{\%}$\Gamma$}}%
\end{pgfscope}%
\begin{pgfscope}%
\definecolor{textcolor}{rgb}{0.000000,0.000000,0.000000}%
\pgfsetstrokecolor{textcolor}%
\pgfsetfillcolor{textcolor}%
\pgftext[x=1.715917in,y=1.671847in,left,base]{\color{textcolor}{\rmfamily\fontsize{13.200000}{15.840000}\selectfont\catcode`\^=\active\def^{\ifmmode\sp\else\^{}\fi}\catcode`\%=\active\def%{\%}$\mathrm{M}$}}%
\end{pgfscope}%
\begin{pgfscope}%
\definecolor{textcolor}{rgb}{0.000000,0.000000,0.000000}%
\pgfsetstrokecolor{textcolor}%
\pgfsetfillcolor{textcolor}%
\pgftext[x=1.846456in,y=1.445747in,left,base]{\color{textcolor}{\rmfamily\fontsize{13.200000}{15.840000}\selectfont\catcode`\^=\active\def^{\ifmmode\sp\else\^{}\fi}\catcode`\%=\active\def%{\%}$\mathrm{K}$}}%
\end{pgfscope}%
\end{pgfpicture}%
\makeatother%
\endgroup%

	\end{subfigure}
	\caption{(\subref{sfig:graphene lattice structure}) Graphene lattice structure and (\subref{sfig:graphene Brillouin zone}) Brilluoin zone} 
	\label{fig:Graphene lattice structure and Brilluoin zone}
\end{figure}
The primitive reciprocal lattice vectors \(\vb{b}_1\), \(\vb{b}_2\) fulfill \todo{labels on vectors}
\begin{align}
	\vb{a}_1 \cdot \vb{b}_1 &= \vb{a}_2 \cdot \vb{b}_2 = 2\pi \\
	\vb{a}_1 \cdot \vb{b}_2 &= \vb{a}_2 \cdot \vb{b}_1 = 0\;,
\end{align}
so we have:
\begin{align}
	\vb{b}_1 = \frac{2\pi}{a} \begin{pmatrix} 1 \\ \frac{1}{\sqrt{3}} \end{pmatrix},\;
	\vb{b}_2 = \frac{2\pi}{a} \begin{pmatrix} 1 \\ - \frac{1}{\sqrt{3}} \end{pmatrix}
\end{align}
The first Brilluoin zone of the hexagonal lattice is shown in \cref{sfig:graphene Brillouin zone}, with the points of high symmetry
\begin{align}
	\Gamma = \begin{pmatrix} 0 \\ 0 \end{pmatrix},\;
	\mathrm{M} = \frac{\pi}{a} \begin{pmatrix} 1 \\ \frac{1}{\sqrt{3}} \end{pmatrix},\;
	\mathrm{K} = \frac{4\pi}{3 a} \begin{pmatrix} 1 \\ 0 \end{pmatrix}\;.
\end{align}


\section{Dressed Graphene Model}\label{sec:dressed graphene model}

The model I am concerned with in this thesis consists of a Hubbard Hamiltonian (as introduced in \cref{sec:bcs-theory}) on a Graphene lattice, with one additional atom at one of the two sites in a unit cell, which I will call X\@.
This is shown in \cref{fig:eg-x model}.\todo{Work over image for dressed graphene lattice}
\begin{figure}[tb]
	\centering
	%% Creator: Matplotlib, PGF backend
%%
%% To include the figure in your LaTeX document, write
%%   \input{<filename>.pgf}
%%
%% Make sure the required packages are loaded in your preamble
%%   \usepackage{pgf}
%%
%% Also ensure that all the required font packages are loaded; for instance,
%% the lmodern package is sometimes necessary when using math font.
%%   \usepackage{lmodern}
%%
%% Figures using additional raster images can only be included by \input if
%% they are in the same directory as the main LaTeX file. For loading figures
%% from other directories you can use the `import` package
%%   \usepackage{import}
%%
%% and then include the figures with
%%   \import{<path to file>}{<filename>.pgf}
%%
%% Matplotlib used the following preamble
%%   \def\mathdefault#1{#1}
%%   \everymath=\expandafter{\the\everymath\displaystyle}
%%   \IfFileExists{scrextend.sty}{
%%     \usepackage[fontsize=11.000000pt]{scrextend}
%%   }{
%%     \renewcommand{\normalsize}{\fontsize{11.000000}{13.200000}\selectfont}
%%     \normalsize
%%   }
%%   \usepackage{fontspec}\usepackage{unicode-math}\setmathfont{texgyrepagella-math.otf}\setmainfont{texgyrepagella-math}
%%   \makeatletter\@ifpackageloaded{underscore}{}{\usepackage[strings]{underscore}}\makeatother
%%
\begingroup%
\makeatletter%
\begin{pgfpicture}%
\pgfpathrectangle{\pgfpointorigin}{\pgfqpoint{2.519000in}{2.519000in}}%
\pgfusepath{use as bounding box, clip}%
\begin{pgfscope}%
\pgfsetbuttcap%
\pgfsetmiterjoin%
\definecolor{currentfill}{rgb}{1.000000,1.000000,1.000000}%
\pgfsetfillcolor{currentfill}%
\pgfsetlinewidth{0.000000pt}%
\definecolor{currentstroke}{rgb}{1.000000,1.000000,1.000000}%
\pgfsetstrokecolor{currentstroke}%
\pgfsetdash{}{0pt}%
\pgfpathmoveto{\pgfqpoint{0.000000in}{0.000000in}}%
\pgfpathlineto{\pgfqpoint{2.519000in}{0.000000in}}%
\pgfpathlineto{\pgfqpoint{2.519000in}{2.519000in}}%
\pgfpathlineto{\pgfqpoint{0.000000in}{2.519000in}}%
\pgfpathlineto{\pgfqpoint{0.000000in}{0.000000in}}%
\pgfpathclose%
\pgfusepath{fill}%
\end{pgfscope}%
\begin{pgfscope}%
\pgfsetbuttcap%
\pgfsetmiterjoin%
\definecolor{currentfill}{rgb}{1.000000,1.000000,1.000000}%
\pgfsetfillcolor{currentfill}%
\pgfsetlinewidth{0.000000pt}%
\definecolor{currentstroke}{rgb}{0.000000,0.000000,0.000000}%
\pgfsetstrokecolor{currentstroke}%
\pgfsetstrokeopacity{0.000000}%
\pgfsetdash{}{0pt}%
\pgfpathmoveto{\pgfqpoint{0.050000in}{0.050000in}}%
\pgfpathlineto{\pgfqpoint{2.469000in}{0.050000in}}%
\pgfpathlineto{\pgfqpoint{2.469000in}{2.469000in}}%
\pgfpathlineto{\pgfqpoint{0.050000in}{2.469000in}}%
\pgfpathlineto{\pgfqpoint{0.050000in}{0.050000in}}%
\pgfpathclose%
\pgfusepath{fill}%
\end{pgfscope}%
\begin{pgfscope}%
\pgfpathrectangle{\pgfqpoint{0.050000in}{0.050000in}}{\pgfqpoint{2.419000in}{2.419000in}}%
\pgfusepath{clip}%
\pgfsetbuttcap%
\pgfsetroundjoin%
\pgfsetlinewidth{1.003750pt}%
\definecolor{currentstroke}{rgb}{0.000000,0.000000,0.000000}%
\pgfsetstrokecolor{currentstroke}%
\pgfsetdash{}{0pt}%
\pgfusepath{stroke}%
\end{pgfscope}%
\begin{pgfscope}%
\pgfpathrectangle{\pgfqpoint{0.050000in}{0.050000in}}{\pgfqpoint{2.419000in}{2.419000in}}%
\pgfusepath{clip}%
\pgfsetbuttcap%
\pgfsetroundjoin%
\pgfsetlinewidth{1.003750pt}%
\definecolor{currentstroke}{rgb}{0.000000,0.000000,0.000000}%
\pgfsetstrokecolor{currentstroke}%
\pgfsetdash{}{0pt}%
\pgfusepath{stroke}%
\end{pgfscope}%
\begin{pgfscope}%
\pgfpathrectangle{\pgfqpoint{0.050000in}{0.050000in}}{\pgfqpoint{2.419000in}{2.419000in}}%
\pgfusepath{clip}%
\pgfsetbuttcap%
\pgfsetroundjoin%
\pgfsetlinewidth{1.003750pt}%
\definecolor{currentstroke}{rgb}{0.000000,0.000000,0.000000}%
\pgfsetstrokecolor{currentstroke}%
\pgfsetdash{}{0pt}%
\pgfusepath{stroke}%
\end{pgfscope}%
\begin{pgfscope}%
\pgfpathrectangle{\pgfqpoint{0.050000in}{0.050000in}}{\pgfqpoint{2.419000in}{2.419000in}}%
\pgfusepath{clip}%
\pgfsetbuttcap%
\pgfsetroundjoin%
\pgfsetlinewidth{1.003750pt}%
\definecolor{currentstroke}{rgb}{0.000000,0.000000,0.000000}%
\pgfsetstrokecolor{currentstroke}%
\pgfsetdash{}{0pt}%
\pgfusepath{stroke}%
\end{pgfscope}%
\begin{pgfscope}%
\pgfpathrectangle{\pgfqpoint{0.050000in}{0.050000in}}{\pgfqpoint{2.419000in}{2.419000in}}%
\pgfusepath{clip}%
\pgfsetbuttcap%
\pgfsetroundjoin%
\pgfsetlinewidth{1.003750pt}%
\definecolor{currentstroke}{rgb}{0.000000,0.000000,0.000000}%
\pgfsetstrokecolor{currentstroke}%
\pgfsetdash{}{0pt}%
\pgfusepath{stroke}%
\end{pgfscope}%
\begin{pgfscope}%
\pgfpathrectangle{\pgfqpoint{0.050000in}{0.050000in}}{\pgfqpoint{2.419000in}{2.419000in}}%
\pgfusepath{clip}%
\pgfsetbuttcap%
\pgfsetroundjoin%
\pgfsetlinewidth{1.003750pt}%
\definecolor{currentstroke}{rgb}{0.000000,0.000000,0.000000}%
\pgfsetstrokecolor{currentstroke}%
\pgfsetdash{}{0pt}%
\pgfusepath{stroke}%
\end{pgfscope}%
\begin{pgfscope}%
\pgfpathrectangle{\pgfqpoint{0.050000in}{0.050000in}}{\pgfqpoint{2.419000in}{2.419000in}}%
\pgfusepath{clip}%
\pgfsetbuttcap%
\pgfsetroundjoin%
\pgfsetlinewidth{1.003750pt}%
\definecolor{currentstroke}{rgb}{0.000000,0.000000,0.000000}%
\pgfsetstrokecolor{currentstroke}%
\pgfsetdash{}{0pt}%
\pgfusepath{stroke}%
\end{pgfscope}%
\begin{pgfscope}%
\pgfpathrectangle{\pgfqpoint{0.050000in}{0.050000in}}{\pgfqpoint{2.419000in}{2.419000in}}%
\pgfusepath{clip}%
\pgfsetbuttcap%
\pgfsetroundjoin%
\pgfsetlinewidth{1.003750pt}%
\definecolor{currentstroke}{rgb}{0.000000,0.000000,0.000000}%
\pgfsetstrokecolor{currentstroke}%
\pgfsetdash{}{0pt}%
\pgfusepath{stroke}%
\end{pgfscope}%
\begin{pgfscope}%
\pgfpathrectangle{\pgfqpoint{0.050000in}{0.050000in}}{\pgfqpoint{2.419000in}{2.419000in}}%
\pgfusepath{clip}%
\pgfsetbuttcap%
\pgfsetroundjoin%
\pgfsetlinewidth{1.003750pt}%
\definecolor{currentstroke}{rgb}{0.000000,0.000000,0.000000}%
\pgfsetstrokecolor{currentstroke}%
\pgfsetdash{}{0pt}%
\pgfusepath{stroke}%
\end{pgfscope}%
\begin{pgfscope}%
\pgfpathrectangle{\pgfqpoint{0.050000in}{0.050000in}}{\pgfqpoint{2.419000in}{2.419000in}}%
\pgfusepath{clip}%
\pgfsetbuttcap%
\pgfsetroundjoin%
\pgfsetlinewidth{1.003750pt}%
\definecolor{currentstroke}{rgb}{0.000000,0.000000,0.000000}%
\pgfsetstrokecolor{currentstroke}%
\pgfsetdash{}{0pt}%
\pgfusepath{stroke}%
\end{pgfscope}%
\begin{pgfscope}%
\pgfpathrectangle{\pgfqpoint{0.050000in}{0.050000in}}{\pgfqpoint{2.419000in}{2.419000in}}%
\pgfusepath{clip}%
\pgfsetbuttcap%
\pgfsetroundjoin%
\pgfsetlinewidth{1.003750pt}%
\definecolor{currentstroke}{rgb}{0.000000,0.000000,0.000000}%
\pgfsetstrokecolor{currentstroke}%
\pgfsetdash{}{0pt}%
\pgfusepath{stroke}%
\end{pgfscope}%
\begin{pgfscope}%
\pgfpathrectangle{\pgfqpoint{0.050000in}{0.050000in}}{\pgfqpoint{2.419000in}{2.419000in}}%
\pgfusepath{clip}%
\pgfsetbuttcap%
\pgfsetroundjoin%
\pgfsetlinewidth{1.003750pt}%
\definecolor{currentstroke}{rgb}{0.000000,0.000000,0.000000}%
\pgfsetstrokecolor{currentstroke}%
\pgfsetdash{}{0pt}%
\pgfusepath{stroke}%
\end{pgfscope}%
\begin{pgfscope}%
\pgfpathrectangle{\pgfqpoint{0.050000in}{0.050000in}}{\pgfqpoint{2.419000in}{2.419000in}}%
\pgfusepath{clip}%
\pgfsetbuttcap%
\pgfsetroundjoin%
\pgfsetlinewidth{1.003750pt}%
\definecolor{currentstroke}{rgb}{0.000000,0.000000,0.000000}%
\pgfsetstrokecolor{currentstroke}%
\pgfsetdash{}{0pt}%
\pgfusepath{stroke}%
\end{pgfscope}%
\begin{pgfscope}%
\pgfpathrectangle{\pgfqpoint{0.050000in}{0.050000in}}{\pgfqpoint{2.419000in}{2.419000in}}%
\pgfusepath{clip}%
\pgfsetbuttcap%
\pgfsetroundjoin%
\pgfsetlinewidth{1.003750pt}%
\definecolor{currentstroke}{rgb}{0.000000,0.000000,0.000000}%
\pgfsetstrokecolor{currentstroke}%
\pgfsetdash{}{0pt}%
\pgfusepath{stroke}%
\end{pgfscope}%
\begin{pgfscope}%
\pgfpathrectangle{\pgfqpoint{0.050000in}{0.050000in}}{\pgfqpoint{2.419000in}{2.419000in}}%
\pgfusepath{clip}%
\pgfsetbuttcap%
\pgfsetroundjoin%
\pgfsetlinewidth{1.003750pt}%
\definecolor{currentstroke}{rgb}{0.000000,0.000000,0.000000}%
\pgfsetstrokecolor{currentstroke}%
\pgfsetdash{}{0pt}%
\pgfusepath{stroke}%
\end{pgfscope}%
\begin{pgfscope}%
\pgfpathrectangle{\pgfqpoint{0.050000in}{0.050000in}}{\pgfqpoint{2.419000in}{2.419000in}}%
\pgfusepath{clip}%
\pgfsetbuttcap%
\pgfsetroundjoin%
\pgfsetlinewidth{1.003750pt}%
\definecolor{currentstroke}{rgb}{0.000000,0.000000,0.000000}%
\pgfsetstrokecolor{currentstroke}%
\pgfsetdash{}{0pt}%
\pgfusepath{stroke}%
\end{pgfscope}%
\begin{pgfscope}%
\pgfpathrectangle{\pgfqpoint{0.050000in}{0.050000in}}{\pgfqpoint{2.419000in}{2.419000in}}%
\pgfusepath{clip}%
\pgfsetbuttcap%
\pgfsetroundjoin%
\pgfsetlinewidth{1.003750pt}%
\definecolor{currentstroke}{rgb}{0.000000,0.000000,0.000000}%
\pgfsetstrokecolor{currentstroke}%
\pgfsetdash{}{0pt}%
\pgfusepath{stroke}%
\end{pgfscope}%
\begin{pgfscope}%
\pgfpathrectangle{\pgfqpoint{0.050000in}{0.050000in}}{\pgfqpoint{2.419000in}{2.419000in}}%
\pgfusepath{clip}%
\pgfsetbuttcap%
\pgfsetroundjoin%
\pgfsetlinewidth{1.003750pt}%
\definecolor{currentstroke}{rgb}{0.000000,0.000000,0.000000}%
\pgfsetstrokecolor{currentstroke}%
\pgfsetdash{}{0pt}%
\pgfusepath{stroke}%
\end{pgfscope}%
\begin{pgfscope}%
\pgfpathrectangle{\pgfqpoint{0.050000in}{0.050000in}}{\pgfqpoint{2.419000in}{2.419000in}}%
\pgfusepath{clip}%
\pgfsetbuttcap%
\pgfsetroundjoin%
\pgfsetlinewidth{1.003750pt}%
\definecolor{currentstroke}{rgb}{0.000000,0.000000,0.000000}%
\pgfsetstrokecolor{currentstroke}%
\pgfsetdash{}{0pt}%
\pgfusepath{stroke}%
\end{pgfscope}%
\begin{pgfscope}%
\pgfpathrectangle{\pgfqpoint{0.050000in}{0.050000in}}{\pgfqpoint{2.419000in}{2.419000in}}%
\pgfusepath{clip}%
\pgfsetbuttcap%
\pgfsetroundjoin%
\pgfsetlinewidth{1.003750pt}%
\definecolor{currentstroke}{rgb}{0.000000,0.000000,0.000000}%
\pgfsetstrokecolor{currentstroke}%
\pgfsetdash{}{0pt}%
\pgfusepath{stroke}%
\end{pgfscope}%
\begin{pgfscope}%
\pgfpathrectangle{\pgfqpoint{0.050000in}{0.050000in}}{\pgfqpoint{2.419000in}{2.419000in}}%
\pgfusepath{clip}%
\pgfsetbuttcap%
\pgfsetroundjoin%
\pgfsetlinewidth{1.003750pt}%
\definecolor{currentstroke}{rgb}{0.000000,0.000000,0.000000}%
\pgfsetstrokecolor{currentstroke}%
\pgfsetdash{}{0pt}%
\pgfusepath{stroke}%
\end{pgfscope}%
\begin{pgfscope}%
\pgfpathrectangle{\pgfqpoint{0.050000in}{0.050000in}}{\pgfqpoint{2.419000in}{2.419000in}}%
\pgfusepath{clip}%
\pgfsetbuttcap%
\pgfsetroundjoin%
\pgfsetlinewidth{1.003750pt}%
\definecolor{currentstroke}{rgb}{0.000000,0.000000,0.000000}%
\pgfsetstrokecolor{currentstroke}%
\pgfsetdash{}{0pt}%
\pgfusepath{stroke}%
\end{pgfscope}%
\begin{pgfscope}%
\pgfpathrectangle{\pgfqpoint{0.050000in}{0.050000in}}{\pgfqpoint{2.419000in}{2.419000in}}%
\pgfusepath{clip}%
\pgfsetbuttcap%
\pgfsetroundjoin%
\pgfsetlinewidth{1.003750pt}%
\definecolor{currentstroke}{rgb}{0.000000,0.000000,0.000000}%
\pgfsetstrokecolor{currentstroke}%
\pgfsetdash{}{0pt}%
\pgfusepath{stroke}%
\end{pgfscope}%
\begin{pgfscope}%
\pgfpathrectangle{\pgfqpoint{0.050000in}{0.050000in}}{\pgfqpoint{2.419000in}{2.419000in}}%
\pgfusepath{clip}%
\pgfsetbuttcap%
\pgfsetroundjoin%
\pgfsetlinewidth{1.003750pt}%
\definecolor{currentstroke}{rgb}{0.000000,0.000000,0.000000}%
\pgfsetstrokecolor{currentstroke}%
\pgfsetdash{}{0pt}%
\pgfusepath{stroke}%
\end{pgfscope}%
\begin{pgfscope}%
\pgfpathrectangle{\pgfqpoint{0.050000in}{0.050000in}}{\pgfqpoint{2.419000in}{2.419000in}}%
\pgfusepath{clip}%
\pgfsetbuttcap%
\pgfsetroundjoin%
\pgfsetlinewidth{1.003750pt}%
\definecolor{currentstroke}{rgb}{0.000000,0.000000,0.000000}%
\pgfsetstrokecolor{currentstroke}%
\pgfsetdash{}{0pt}%
\pgfusepath{stroke}%
\end{pgfscope}%
\begin{pgfscope}%
\pgfpathrectangle{\pgfqpoint{0.050000in}{0.050000in}}{\pgfqpoint{2.419000in}{2.419000in}}%
\pgfusepath{clip}%
\pgfsetbuttcap%
\pgfsetroundjoin%
\pgfsetlinewidth{1.003750pt}%
\definecolor{currentstroke}{rgb}{0.000000,0.000000,0.000000}%
\pgfsetstrokecolor{currentstroke}%
\pgfsetdash{}{0pt}%
\pgfusepath{stroke}%
\end{pgfscope}%
\begin{pgfscope}%
\pgfpathrectangle{\pgfqpoint{0.050000in}{0.050000in}}{\pgfqpoint{2.419000in}{2.419000in}}%
\pgfusepath{clip}%
\pgfsetbuttcap%
\pgfsetroundjoin%
\pgfsetlinewidth{1.003750pt}%
\definecolor{currentstroke}{rgb}{0.000000,0.000000,0.000000}%
\pgfsetstrokecolor{currentstroke}%
\pgfsetdash{}{0pt}%
\pgfusepath{stroke}%
\end{pgfscope}%
\begin{pgfscope}%
\pgfpathrectangle{\pgfqpoint{0.050000in}{0.050000in}}{\pgfqpoint{2.419000in}{2.419000in}}%
\pgfusepath{clip}%
\pgfsetbuttcap%
\pgfsetroundjoin%
\pgfsetlinewidth{1.003750pt}%
\definecolor{currentstroke}{rgb}{0.000000,0.000000,0.000000}%
\pgfsetstrokecolor{currentstroke}%
\pgfsetdash{}{0pt}%
\pgfusepath{stroke}%
\end{pgfscope}%
\begin{pgfscope}%
\pgfpathrectangle{\pgfqpoint{0.050000in}{0.050000in}}{\pgfqpoint{2.419000in}{2.419000in}}%
\pgfusepath{clip}%
\pgfsetbuttcap%
\pgfsetroundjoin%
\pgfsetlinewidth{1.003750pt}%
\definecolor{currentstroke}{rgb}{0.000000,0.000000,0.000000}%
\pgfsetstrokecolor{currentstroke}%
\pgfsetdash{}{0pt}%
\pgfusepath{stroke}%
\end{pgfscope}%
\begin{pgfscope}%
\pgfpathrectangle{\pgfqpoint{0.050000in}{0.050000in}}{\pgfqpoint{2.419000in}{2.419000in}}%
\pgfusepath{clip}%
\pgfsetbuttcap%
\pgfsetroundjoin%
\pgfsetlinewidth{1.003750pt}%
\definecolor{currentstroke}{rgb}{0.000000,0.000000,0.000000}%
\pgfsetstrokecolor{currentstroke}%
\pgfsetdash{}{0pt}%
\pgfusepath{stroke}%
\end{pgfscope}%
\begin{pgfscope}%
\pgfpathrectangle{\pgfqpoint{0.050000in}{0.050000in}}{\pgfqpoint{2.419000in}{2.419000in}}%
\pgfusepath{clip}%
\pgfsetbuttcap%
\pgfsetroundjoin%
\pgfsetlinewidth{1.003750pt}%
\definecolor{currentstroke}{rgb}{0.000000,0.000000,0.000000}%
\pgfsetstrokecolor{currentstroke}%
\pgfsetdash{}{0pt}%
\pgfusepath{stroke}%
\end{pgfscope}%
\begin{pgfscope}%
\pgfpathrectangle{\pgfqpoint{0.050000in}{0.050000in}}{\pgfqpoint{2.419000in}{2.419000in}}%
\pgfusepath{clip}%
\pgfsetbuttcap%
\pgfsetroundjoin%
\pgfsetlinewidth{1.003750pt}%
\definecolor{currentstroke}{rgb}{0.000000,0.000000,0.000000}%
\pgfsetstrokecolor{currentstroke}%
\pgfsetdash{}{0pt}%
\pgfusepath{stroke}%
\end{pgfscope}%
\begin{pgfscope}%
\pgfpathrectangle{\pgfqpoint{0.050000in}{0.050000in}}{\pgfqpoint{2.419000in}{2.419000in}}%
\pgfusepath{clip}%
\pgfsetbuttcap%
\pgfsetroundjoin%
\pgfsetlinewidth{1.003750pt}%
\definecolor{currentstroke}{rgb}{0.000000,0.000000,0.000000}%
\pgfsetstrokecolor{currentstroke}%
\pgfsetdash{}{0pt}%
\pgfusepath{stroke}%
\end{pgfscope}%
\begin{pgfscope}%
\pgfpathrectangle{\pgfqpoint{0.050000in}{0.050000in}}{\pgfqpoint{2.419000in}{2.419000in}}%
\pgfusepath{clip}%
\pgfsetbuttcap%
\pgfsetroundjoin%
\pgfsetlinewidth{1.003750pt}%
\definecolor{currentstroke}{rgb}{0.000000,0.000000,0.000000}%
\pgfsetstrokecolor{currentstroke}%
\pgfsetdash{}{0pt}%
\pgfusepath{stroke}%
\end{pgfscope}%
\begin{pgfscope}%
\pgfpathrectangle{\pgfqpoint{0.050000in}{0.050000in}}{\pgfqpoint{2.419000in}{2.419000in}}%
\pgfusepath{clip}%
\pgfsetbuttcap%
\pgfsetroundjoin%
\pgfsetlinewidth{1.003750pt}%
\definecolor{currentstroke}{rgb}{0.000000,0.000000,0.000000}%
\pgfsetstrokecolor{currentstroke}%
\pgfsetdash{}{0pt}%
\pgfusepath{stroke}%
\end{pgfscope}%
\begin{pgfscope}%
\pgfpathrectangle{\pgfqpoint{0.050000in}{0.050000in}}{\pgfqpoint{2.419000in}{2.419000in}}%
\pgfusepath{clip}%
\pgfsetbuttcap%
\pgfsetroundjoin%
\pgfsetlinewidth{1.003750pt}%
\definecolor{currentstroke}{rgb}{0.000000,0.000000,0.000000}%
\pgfsetstrokecolor{currentstroke}%
\pgfsetdash{}{0pt}%
\pgfusepath{stroke}%
\end{pgfscope}%
\begin{pgfscope}%
\pgfpathrectangle{\pgfqpoint{0.050000in}{0.050000in}}{\pgfqpoint{2.419000in}{2.419000in}}%
\pgfusepath{clip}%
\pgfsetbuttcap%
\pgfsetroundjoin%
\pgfsetlinewidth{1.003750pt}%
\definecolor{currentstroke}{rgb}{0.000000,0.000000,0.000000}%
\pgfsetstrokecolor{currentstroke}%
\pgfsetdash{}{0pt}%
\pgfusepath{stroke}%
\end{pgfscope}%
\begin{pgfscope}%
\pgfpathrectangle{\pgfqpoint{0.050000in}{0.050000in}}{\pgfqpoint{2.419000in}{2.419000in}}%
\pgfusepath{clip}%
\pgfsetbuttcap%
\pgfsetroundjoin%
\pgfsetlinewidth{1.003750pt}%
\definecolor{currentstroke}{rgb}{0.000000,0.000000,0.000000}%
\pgfsetstrokecolor{currentstroke}%
\pgfsetdash{}{0pt}%
\pgfusepath{stroke}%
\end{pgfscope}%
\begin{pgfscope}%
\pgfpathrectangle{\pgfqpoint{0.050000in}{0.050000in}}{\pgfqpoint{2.419000in}{2.419000in}}%
\pgfusepath{clip}%
\pgfsetbuttcap%
\pgfsetroundjoin%
\pgfsetlinewidth{1.003750pt}%
\definecolor{currentstroke}{rgb}{0.000000,0.000000,0.000000}%
\pgfsetstrokecolor{currentstroke}%
\pgfsetdash{}{0pt}%
\pgfusepath{stroke}%
\end{pgfscope}%
\begin{pgfscope}%
\pgfpathrectangle{\pgfqpoint{0.050000in}{0.050000in}}{\pgfqpoint{2.419000in}{2.419000in}}%
\pgfusepath{clip}%
\pgfsetbuttcap%
\pgfsetroundjoin%
\pgfsetlinewidth{1.003750pt}%
\definecolor{currentstroke}{rgb}{0.000000,0.000000,0.000000}%
\pgfsetstrokecolor{currentstroke}%
\pgfsetdash{}{0pt}%
\pgfusepath{stroke}%
\end{pgfscope}%
\begin{pgfscope}%
\pgfpathrectangle{\pgfqpoint{0.050000in}{0.050000in}}{\pgfqpoint{2.419000in}{2.419000in}}%
\pgfusepath{clip}%
\pgfsetbuttcap%
\pgfsetroundjoin%
\pgfsetlinewidth{1.003750pt}%
\definecolor{currentstroke}{rgb}{0.000000,0.000000,0.000000}%
\pgfsetstrokecolor{currentstroke}%
\pgfsetdash{}{0pt}%
\pgfusepath{stroke}%
\end{pgfscope}%
\begin{pgfscope}%
\pgfpathrectangle{\pgfqpoint{0.050000in}{0.050000in}}{\pgfqpoint{2.419000in}{2.419000in}}%
\pgfusepath{clip}%
\pgfsetbuttcap%
\pgfsetroundjoin%
\pgfsetlinewidth{1.003750pt}%
\definecolor{currentstroke}{rgb}{0.000000,0.000000,0.000000}%
\pgfsetstrokecolor{currentstroke}%
\pgfsetdash{}{0pt}%
\pgfusepath{stroke}%
\end{pgfscope}%
\begin{pgfscope}%
\pgfpathrectangle{\pgfqpoint{0.050000in}{0.050000in}}{\pgfqpoint{2.419000in}{2.419000in}}%
\pgfusepath{clip}%
\pgfsetbuttcap%
\pgfsetroundjoin%
\pgfsetlinewidth{1.003750pt}%
\definecolor{currentstroke}{rgb}{0.000000,0.000000,0.000000}%
\pgfsetstrokecolor{currentstroke}%
\pgfsetdash{}{0pt}%
\pgfusepath{stroke}%
\end{pgfscope}%
\begin{pgfscope}%
\pgfpathrectangle{\pgfqpoint{0.050000in}{0.050000in}}{\pgfqpoint{2.419000in}{2.419000in}}%
\pgfusepath{clip}%
\pgfsetbuttcap%
\pgfsetroundjoin%
\pgfsetlinewidth{1.003750pt}%
\definecolor{currentstroke}{rgb}{0.000000,0.000000,0.000000}%
\pgfsetstrokecolor{currentstroke}%
\pgfsetdash{}{0pt}%
\pgfusepath{stroke}%
\end{pgfscope}%
\begin{pgfscope}%
\pgfpathrectangle{\pgfqpoint{0.050000in}{0.050000in}}{\pgfqpoint{2.419000in}{2.419000in}}%
\pgfusepath{clip}%
\pgfsetbuttcap%
\pgfsetroundjoin%
\pgfsetlinewidth{1.003750pt}%
\definecolor{currentstroke}{rgb}{0.000000,0.000000,0.000000}%
\pgfsetstrokecolor{currentstroke}%
\pgfsetdash{}{0pt}%
\pgfusepath{stroke}%
\end{pgfscope}%
\begin{pgfscope}%
\pgfpathrectangle{\pgfqpoint{0.050000in}{0.050000in}}{\pgfqpoint{2.419000in}{2.419000in}}%
\pgfusepath{clip}%
\pgfsetbuttcap%
\pgfsetroundjoin%
\pgfsetlinewidth{1.003750pt}%
\definecolor{currentstroke}{rgb}{0.000000,0.000000,0.000000}%
\pgfsetstrokecolor{currentstroke}%
\pgfsetdash{}{0pt}%
\pgfusepath{stroke}%
\end{pgfscope}%
\begin{pgfscope}%
\pgfpathrectangle{\pgfqpoint{0.050000in}{0.050000in}}{\pgfqpoint{2.419000in}{2.419000in}}%
\pgfusepath{clip}%
\pgfsetbuttcap%
\pgfsetroundjoin%
\pgfsetlinewidth{1.003750pt}%
\definecolor{currentstroke}{rgb}{0.000000,0.000000,0.000000}%
\pgfsetstrokecolor{currentstroke}%
\pgfsetdash{}{0pt}%
\pgfusepath{stroke}%
\end{pgfscope}%
\begin{pgfscope}%
\pgfpathrectangle{\pgfqpoint{0.050000in}{0.050000in}}{\pgfqpoint{2.419000in}{2.419000in}}%
\pgfusepath{clip}%
\pgfsetbuttcap%
\pgfsetroundjoin%
\pgfsetlinewidth{1.003750pt}%
\definecolor{currentstroke}{rgb}{0.000000,0.000000,0.000000}%
\pgfsetstrokecolor{currentstroke}%
\pgfsetdash{}{0pt}%
\pgfusepath{stroke}%
\end{pgfscope}%
\begin{pgfscope}%
\pgfpathrectangle{\pgfqpoint{0.050000in}{0.050000in}}{\pgfqpoint{2.419000in}{2.419000in}}%
\pgfusepath{clip}%
\pgfsetbuttcap%
\pgfsetroundjoin%
\pgfsetlinewidth{1.003750pt}%
\definecolor{currentstroke}{rgb}{0.000000,0.000000,0.000000}%
\pgfsetstrokecolor{currentstroke}%
\pgfsetdash{}{0pt}%
\pgfusepath{stroke}%
\end{pgfscope}%
\begin{pgfscope}%
\pgfpathrectangle{\pgfqpoint{0.050000in}{0.050000in}}{\pgfqpoint{2.419000in}{2.419000in}}%
\pgfusepath{clip}%
\pgfsetbuttcap%
\pgfsetroundjoin%
\pgfsetlinewidth{1.003750pt}%
\definecolor{currentstroke}{rgb}{0.000000,0.000000,0.000000}%
\pgfsetstrokecolor{currentstroke}%
\pgfsetdash{}{0pt}%
\pgfusepath{stroke}%
\end{pgfscope}%
\begin{pgfscope}%
\pgfpathrectangle{\pgfqpoint{0.050000in}{0.050000in}}{\pgfqpoint{2.419000in}{2.419000in}}%
\pgfusepath{clip}%
\pgfsetbuttcap%
\pgfsetroundjoin%
\pgfsetlinewidth{1.003750pt}%
\definecolor{currentstroke}{rgb}{0.000000,0.000000,0.000000}%
\pgfsetstrokecolor{currentstroke}%
\pgfsetdash{}{0pt}%
\pgfusepath{stroke}%
\end{pgfscope}%
\begin{pgfscope}%
\pgfpathrectangle{\pgfqpoint{0.050000in}{0.050000in}}{\pgfqpoint{2.419000in}{2.419000in}}%
\pgfusepath{clip}%
\pgfsetbuttcap%
\pgfsetroundjoin%
\pgfsetlinewidth{1.003750pt}%
\definecolor{currentstroke}{rgb}{0.000000,0.000000,0.000000}%
\pgfsetstrokecolor{currentstroke}%
\pgfsetdash{}{0pt}%
\pgfusepath{stroke}%
\end{pgfscope}%
\begin{pgfscope}%
\pgfpathrectangle{\pgfqpoint{0.050000in}{0.050000in}}{\pgfqpoint{2.419000in}{2.419000in}}%
\pgfusepath{clip}%
\pgfsetbuttcap%
\pgfsetroundjoin%
\pgfsetlinewidth{1.003750pt}%
\definecolor{currentstroke}{rgb}{0.000000,0.000000,0.000000}%
\pgfsetstrokecolor{currentstroke}%
\pgfsetdash{}{0pt}%
\pgfusepath{stroke}%
\end{pgfscope}%
\begin{pgfscope}%
\pgfpathrectangle{\pgfqpoint{0.050000in}{0.050000in}}{\pgfqpoint{2.419000in}{2.419000in}}%
\pgfusepath{clip}%
\pgfsetbuttcap%
\pgfsetroundjoin%
\pgfsetlinewidth{1.003750pt}%
\definecolor{currentstroke}{rgb}{0.000000,0.000000,0.000000}%
\pgfsetstrokecolor{currentstroke}%
\pgfsetdash{}{0pt}%
\pgfusepath{stroke}%
\end{pgfscope}%
\begin{pgfscope}%
\pgfpathrectangle{\pgfqpoint{0.050000in}{0.050000in}}{\pgfqpoint{2.419000in}{2.419000in}}%
\pgfusepath{clip}%
\pgfsetbuttcap%
\pgfsetroundjoin%
\pgfsetlinewidth{1.003750pt}%
\definecolor{currentstroke}{rgb}{0.000000,0.000000,0.000000}%
\pgfsetstrokecolor{currentstroke}%
\pgfsetdash{}{0pt}%
\pgfusepath{stroke}%
\end{pgfscope}%
\begin{pgfscope}%
\pgfpathrectangle{\pgfqpoint{0.050000in}{0.050000in}}{\pgfqpoint{2.419000in}{2.419000in}}%
\pgfusepath{clip}%
\pgfsetbuttcap%
\pgfsetroundjoin%
\pgfsetlinewidth{1.003750pt}%
\definecolor{currentstroke}{rgb}{0.000000,0.000000,0.000000}%
\pgfsetstrokecolor{currentstroke}%
\pgfsetdash{}{0pt}%
\pgfusepath{stroke}%
\end{pgfscope}%
\begin{pgfscope}%
\pgfpathrectangle{\pgfqpoint{0.050000in}{0.050000in}}{\pgfqpoint{2.419000in}{2.419000in}}%
\pgfusepath{clip}%
\pgfsetbuttcap%
\pgfsetroundjoin%
\pgfsetlinewidth{1.003750pt}%
\definecolor{currentstroke}{rgb}{0.000000,0.000000,0.000000}%
\pgfsetstrokecolor{currentstroke}%
\pgfsetdash{}{0pt}%
\pgfusepath{stroke}%
\end{pgfscope}%
\begin{pgfscope}%
\pgfpathrectangle{\pgfqpoint{0.050000in}{0.050000in}}{\pgfqpoint{2.419000in}{2.419000in}}%
\pgfusepath{clip}%
\pgfsetbuttcap%
\pgfsetroundjoin%
\pgfsetlinewidth{1.003750pt}%
\definecolor{currentstroke}{rgb}{0.000000,0.000000,0.000000}%
\pgfsetstrokecolor{currentstroke}%
\pgfsetdash{}{0pt}%
\pgfusepath{stroke}%
\end{pgfscope}%
\begin{pgfscope}%
\pgfpathrectangle{\pgfqpoint{0.050000in}{0.050000in}}{\pgfqpoint{2.419000in}{2.419000in}}%
\pgfusepath{clip}%
\pgfsetbuttcap%
\pgfsetroundjoin%
\pgfsetlinewidth{1.003750pt}%
\definecolor{currentstroke}{rgb}{0.000000,0.000000,0.000000}%
\pgfsetstrokecolor{currentstroke}%
\pgfsetdash{}{0pt}%
\pgfusepath{stroke}%
\end{pgfscope}%
\begin{pgfscope}%
\pgfpathrectangle{\pgfqpoint{0.050000in}{0.050000in}}{\pgfqpoint{2.419000in}{2.419000in}}%
\pgfusepath{clip}%
\pgfsetbuttcap%
\pgfsetroundjoin%
\pgfsetlinewidth{1.003750pt}%
\definecolor{currentstroke}{rgb}{0.000000,0.000000,0.000000}%
\pgfsetstrokecolor{currentstroke}%
\pgfsetdash{}{0pt}%
\pgfusepath{stroke}%
\end{pgfscope}%
\begin{pgfscope}%
\pgfpathrectangle{\pgfqpoint{0.050000in}{0.050000in}}{\pgfqpoint{2.419000in}{2.419000in}}%
\pgfusepath{clip}%
\pgfsetbuttcap%
\pgfsetroundjoin%
\pgfsetlinewidth{1.003750pt}%
\definecolor{currentstroke}{rgb}{0.000000,0.000000,0.000000}%
\pgfsetstrokecolor{currentstroke}%
\pgfsetdash{}{0pt}%
\pgfusepath{stroke}%
\end{pgfscope}%
\begin{pgfscope}%
\pgfpathrectangle{\pgfqpoint{0.050000in}{0.050000in}}{\pgfqpoint{2.419000in}{2.419000in}}%
\pgfusepath{clip}%
\pgfsetbuttcap%
\pgfsetroundjoin%
\pgfsetlinewidth{1.003750pt}%
\definecolor{currentstroke}{rgb}{0.000000,0.000000,0.000000}%
\pgfsetstrokecolor{currentstroke}%
\pgfsetdash{}{0pt}%
\pgfusepath{stroke}%
\end{pgfscope}%
\begin{pgfscope}%
\pgfpathrectangle{\pgfqpoint{0.050000in}{0.050000in}}{\pgfqpoint{2.419000in}{2.419000in}}%
\pgfusepath{clip}%
\pgfsetbuttcap%
\pgfsetroundjoin%
\pgfsetlinewidth{1.003750pt}%
\definecolor{currentstroke}{rgb}{0.000000,0.000000,0.000000}%
\pgfsetstrokecolor{currentstroke}%
\pgfsetdash{}{0pt}%
\pgfusepath{stroke}%
\end{pgfscope}%
\begin{pgfscope}%
\pgfpathrectangle{\pgfqpoint{0.050000in}{0.050000in}}{\pgfqpoint{2.419000in}{2.419000in}}%
\pgfusepath{clip}%
\pgfsetbuttcap%
\pgfsetroundjoin%
\pgfsetlinewidth{1.003750pt}%
\definecolor{currentstroke}{rgb}{0.000000,0.000000,0.000000}%
\pgfsetstrokecolor{currentstroke}%
\pgfsetdash{}{0pt}%
\pgfusepath{stroke}%
\end{pgfscope}%
\begin{pgfscope}%
\pgfpathrectangle{\pgfqpoint{0.050000in}{0.050000in}}{\pgfqpoint{2.419000in}{2.419000in}}%
\pgfusepath{clip}%
\pgfsetbuttcap%
\pgfsetroundjoin%
\pgfsetlinewidth{1.003750pt}%
\definecolor{currentstroke}{rgb}{0.000000,0.000000,0.000000}%
\pgfsetstrokecolor{currentstroke}%
\pgfsetdash{}{0pt}%
\pgfusepath{stroke}%
\end{pgfscope}%
\begin{pgfscope}%
\pgfpathrectangle{\pgfqpoint{0.050000in}{0.050000in}}{\pgfqpoint{2.419000in}{2.419000in}}%
\pgfusepath{clip}%
\pgfsetbuttcap%
\pgfsetroundjoin%
\pgfsetlinewidth{1.003750pt}%
\definecolor{currentstroke}{rgb}{0.000000,0.000000,0.000000}%
\pgfsetstrokecolor{currentstroke}%
\pgfsetdash{}{0pt}%
\pgfusepath{stroke}%
\end{pgfscope}%
\begin{pgfscope}%
\pgfpathrectangle{\pgfqpoint{0.050000in}{0.050000in}}{\pgfqpoint{2.419000in}{2.419000in}}%
\pgfusepath{clip}%
\pgfsetbuttcap%
\pgfsetroundjoin%
\pgfsetlinewidth{1.003750pt}%
\definecolor{currentstroke}{rgb}{0.000000,0.000000,0.000000}%
\pgfsetstrokecolor{currentstroke}%
\pgfsetdash{}{0pt}%
\pgfusepath{stroke}%
\end{pgfscope}%
\begin{pgfscope}%
\pgfpathrectangle{\pgfqpoint{0.050000in}{0.050000in}}{\pgfqpoint{2.419000in}{2.419000in}}%
\pgfusepath{clip}%
\pgfsetbuttcap%
\pgfsetroundjoin%
\pgfsetlinewidth{1.003750pt}%
\definecolor{currentstroke}{rgb}{0.000000,0.000000,0.000000}%
\pgfsetstrokecolor{currentstroke}%
\pgfsetdash{}{0pt}%
\pgfusepath{stroke}%
\end{pgfscope}%
\begin{pgfscope}%
\pgfpathrectangle{\pgfqpoint{0.050000in}{0.050000in}}{\pgfqpoint{2.419000in}{2.419000in}}%
\pgfusepath{clip}%
\pgfsetbuttcap%
\pgfsetroundjoin%
\pgfsetlinewidth{1.003750pt}%
\definecolor{currentstroke}{rgb}{0.000000,0.000000,0.000000}%
\pgfsetstrokecolor{currentstroke}%
\pgfsetdash{}{0pt}%
\pgfusepath{stroke}%
\end{pgfscope}%
\begin{pgfscope}%
\pgfpathrectangle{\pgfqpoint{0.050000in}{0.050000in}}{\pgfqpoint{2.419000in}{2.419000in}}%
\pgfusepath{clip}%
\pgfsetbuttcap%
\pgfsetroundjoin%
\pgfsetlinewidth{1.003750pt}%
\definecolor{currentstroke}{rgb}{0.000000,0.000000,0.000000}%
\pgfsetstrokecolor{currentstroke}%
\pgfsetdash{}{0pt}%
\pgfusepath{stroke}%
\end{pgfscope}%
\begin{pgfscope}%
\pgfpathrectangle{\pgfqpoint{0.050000in}{0.050000in}}{\pgfqpoint{2.419000in}{2.419000in}}%
\pgfusepath{clip}%
\pgfsetbuttcap%
\pgfsetroundjoin%
\pgfsetlinewidth{1.003750pt}%
\definecolor{currentstroke}{rgb}{0.000000,0.000000,0.000000}%
\pgfsetstrokecolor{currentstroke}%
\pgfsetdash{}{0pt}%
\pgfusepath{stroke}%
\end{pgfscope}%
\begin{pgfscope}%
\pgfpathrectangle{\pgfqpoint{0.050000in}{0.050000in}}{\pgfqpoint{2.419000in}{2.419000in}}%
\pgfusepath{clip}%
\pgfsetbuttcap%
\pgfsetroundjoin%
\pgfsetlinewidth{1.003750pt}%
\definecolor{currentstroke}{rgb}{0.000000,0.000000,0.000000}%
\pgfsetstrokecolor{currentstroke}%
\pgfsetdash{}{0pt}%
\pgfusepath{stroke}%
\end{pgfscope}%
\begin{pgfscope}%
\pgfpathrectangle{\pgfqpoint{0.050000in}{0.050000in}}{\pgfqpoint{2.419000in}{2.419000in}}%
\pgfusepath{clip}%
\pgfsetbuttcap%
\pgfsetroundjoin%
\pgfsetlinewidth{1.003750pt}%
\definecolor{currentstroke}{rgb}{0.000000,0.000000,0.000000}%
\pgfsetstrokecolor{currentstroke}%
\pgfsetdash{}{0pt}%
\pgfusepath{stroke}%
\end{pgfscope}%
\begin{pgfscope}%
\pgfpathrectangle{\pgfqpoint{0.050000in}{0.050000in}}{\pgfqpoint{2.419000in}{2.419000in}}%
\pgfusepath{clip}%
\pgfsetbuttcap%
\pgfsetroundjoin%
\pgfsetlinewidth{1.003750pt}%
\definecolor{currentstroke}{rgb}{0.000000,0.000000,0.000000}%
\pgfsetstrokecolor{currentstroke}%
\pgfsetdash{}{0pt}%
\pgfusepath{stroke}%
\end{pgfscope}%
\begin{pgfscope}%
\pgfpathrectangle{\pgfqpoint{0.050000in}{0.050000in}}{\pgfqpoint{2.419000in}{2.419000in}}%
\pgfusepath{clip}%
\pgfsetbuttcap%
\pgfsetroundjoin%
\pgfsetlinewidth{1.003750pt}%
\definecolor{currentstroke}{rgb}{0.000000,0.000000,0.000000}%
\pgfsetstrokecolor{currentstroke}%
\pgfsetdash{}{0pt}%
\pgfusepath{stroke}%
\end{pgfscope}%
\begin{pgfscope}%
\pgfpathrectangle{\pgfqpoint{0.050000in}{0.050000in}}{\pgfqpoint{2.419000in}{2.419000in}}%
\pgfusepath{clip}%
\pgfsetbuttcap%
\pgfsetroundjoin%
\pgfsetlinewidth{1.003750pt}%
\definecolor{currentstroke}{rgb}{0.000000,0.000000,0.000000}%
\pgfsetstrokecolor{currentstroke}%
\pgfsetdash{}{0pt}%
\pgfusepath{stroke}%
\end{pgfscope}%
\begin{pgfscope}%
\pgfpathrectangle{\pgfqpoint{0.050000in}{0.050000in}}{\pgfqpoint{2.419000in}{2.419000in}}%
\pgfusepath{clip}%
\pgfsetbuttcap%
\pgfsetroundjoin%
\pgfsetlinewidth{1.003750pt}%
\definecolor{currentstroke}{rgb}{0.000000,0.000000,0.000000}%
\pgfsetstrokecolor{currentstroke}%
\pgfsetdash{}{0pt}%
\pgfusepath{stroke}%
\end{pgfscope}%
\begin{pgfscope}%
\pgfpathrectangle{\pgfqpoint{0.050000in}{0.050000in}}{\pgfqpoint{2.419000in}{2.419000in}}%
\pgfusepath{clip}%
\pgfsetbuttcap%
\pgfsetroundjoin%
\pgfsetlinewidth{1.003750pt}%
\definecolor{currentstroke}{rgb}{0.000000,0.000000,0.000000}%
\pgfsetstrokecolor{currentstroke}%
\pgfsetdash{}{0pt}%
\pgfusepath{stroke}%
\end{pgfscope}%
\begin{pgfscope}%
\pgfpathrectangle{\pgfqpoint{0.050000in}{0.050000in}}{\pgfqpoint{2.419000in}{2.419000in}}%
\pgfusepath{clip}%
\pgfsetbuttcap%
\pgfsetroundjoin%
\pgfsetlinewidth{1.003750pt}%
\definecolor{currentstroke}{rgb}{0.000000,0.000000,0.000000}%
\pgfsetstrokecolor{currentstroke}%
\pgfsetdash{}{0pt}%
\pgfusepath{stroke}%
\end{pgfscope}%
\begin{pgfscope}%
\pgfpathrectangle{\pgfqpoint{0.050000in}{0.050000in}}{\pgfqpoint{2.419000in}{2.419000in}}%
\pgfusepath{clip}%
\pgfsetbuttcap%
\pgfsetroundjoin%
\pgfsetlinewidth{1.003750pt}%
\definecolor{currentstroke}{rgb}{0.000000,0.000000,0.000000}%
\pgfsetstrokecolor{currentstroke}%
\pgfsetdash{}{0pt}%
\pgfusepath{stroke}%
\end{pgfscope}%
\begin{pgfscope}%
\pgfpathrectangle{\pgfqpoint{0.050000in}{0.050000in}}{\pgfqpoint{2.419000in}{2.419000in}}%
\pgfusepath{clip}%
\pgfsetbuttcap%
\pgfsetroundjoin%
\pgfsetlinewidth{1.003750pt}%
\definecolor{currentstroke}{rgb}{0.000000,0.000000,0.000000}%
\pgfsetstrokecolor{currentstroke}%
\pgfsetdash{}{0pt}%
\pgfusepath{stroke}%
\end{pgfscope}%
\begin{pgfscope}%
\pgfpathrectangle{\pgfqpoint{0.050000in}{0.050000in}}{\pgfqpoint{2.419000in}{2.419000in}}%
\pgfusepath{clip}%
\pgfsetbuttcap%
\pgfsetroundjoin%
\pgfsetlinewidth{1.003750pt}%
\definecolor{currentstroke}{rgb}{0.000000,0.000000,0.000000}%
\pgfsetstrokecolor{currentstroke}%
\pgfsetdash{}{0pt}%
\pgfusepath{stroke}%
\end{pgfscope}%
\begin{pgfscope}%
\pgfpathrectangle{\pgfqpoint{0.050000in}{0.050000in}}{\pgfqpoint{2.419000in}{2.419000in}}%
\pgfusepath{clip}%
\pgfsetbuttcap%
\pgfsetroundjoin%
\pgfsetlinewidth{1.003750pt}%
\definecolor{currentstroke}{rgb}{0.000000,0.000000,0.000000}%
\pgfsetstrokecolor{currentstroke}%
\pgfsetdash{}{0pt}%
\pgfusepath{stroke}%
\end{pgfscope}%
\begin{pgfscope}%
\pgfpathrectangle{\pgfqpoint{0.050000in}{0.050000in}}{\pgfqpoint{2.419000in}{2.419000in}}%
\pgfusepath{clip}%
\pgfsetbuttcap%
\pgfsetroundjoin%
\pgfsetlinewidth{1.003750pt}%
\definecolor{currentstroke}{rgb}{0.000000,0.000000,0.000000}%
\pgfsetstrokecolor{currentstroke}%
\pgfsetdash{}{0pt}%
\pgfusepath{stroke}%
\end{pgfscope}%
\begin{pgfscope}%
\pgfpathrectangle{\pgfqpoint{0.050000in}{0.050000in}}{\pgfqpoint{2.419000in}{2.419000in}}%
\pgfusepath{clip}%
\pgfsetbuttcap%
\pgfsetroundjoin%
\pgfsetlinewidth{1.003750pt}%
\definecolor{currentstroke}{rgb}{0.000000,0.000000,0.000000}%
\pgfsetstrokecolor{currentstroke}%
\pgfsetdash{}{0pt}%
\pgfusepath{stroke}%
\end{pgfscope}%
\begin{pgfscope}%
\pgfpathrectangle{\pgfqpoint{0.050000in}{0.050000in}}{\pgfqpoint{2.419000in}{2.419000in}}%
\pgfusepath{clip}%
\pgfsetbuttcap%
\pgfsetroundjoin%
\pgfsetlinewidth{1.003750pt}%
\definecolor{currentstroke}{rgb}{0.000000,0.000000,0.000000}%
\pgfsetstrokecolor{currentstroke}%
\pgfsetdash{}{0pt}%
\pgfusepath{stroke}%
\end{pgfscope}%
\begin{pgfscope}%
\pgfpathrectangle{\pgfqpoint{0.050000in}{0.050000in}}{\pgfqpoint{2.419000in}{2.419000in}}%
\pgfusepath{clip}%
\pgfsetbuttcap%
\pgfsetroundjoin%
\pgfsetlinewidth{1.003750pt}%
\definecolor{currentstroke}{rgb}{0.000000,0.000000,0.000000}%
\pgfsetstrokecolor{currentstroke}%
\pgfsetdash{}{0pt}%
\pgfusepath{stroke}%
\end{pgfscope}%
\begin{pgfscope}%
\pgfpathrectangle{\pgfqpoint{0.050000in}{0.050000in}}{\pgfqpoint{2.419000in}{2.419000in}}%
\pgfusepath{clip}%
\pgfsetbuttcap%
\pgfsetroundjoin%
\pgfsetlinewidth{1.003750pt}%
\definecolor{currentstroke}{rgb}{0.000000,0.000000,0.000000}%
\pgfsetstrokecolor{currentstroke}%
\pgfsetdash{}{0pt}%
\pgfusepath{stroke}%
\end{pgfscope}%
\begin{pgfscope}%
\pgfpathrectangle{\pgfqpoint{0.050000in}{0.050000in}}{\pgfqpoint{2.419000in}{2.419000in}}%
\pgfusepath{clip}%
\pgfsetbuttcap%
\pgfsetroundjoin%
\pgfsetlinewidth{1.003750pt}%
\definecolor{currentstroke}{rgb}{0.000000,0.000000,0.000000}%
\pgfsetstrokecolor{currentstroke}%
\pgfsetdash{}{0pt}%
\pgfusepath{stroke}%
\end{pgfscope}%
\begin{pgfscope}%
\pgfpathrectangle{\pgfqpoint{0.050000in}{0.050000in}}{\pgfqpoint{2.419000in}{2.419000in}}%
\pgfusepath{clip}%
\pgfsetbuttcap%
\pgfsetroundjoin%
\pgfsetlinewidth{1.003750pt}%
\definecolor{currentstroke}{rgb}{0.000000,0.000000,0.000000}%
\pgfsetstrokecolor{currentstroke}%
\pgfsetdash{}{0pt}%
\pgfusepath{stroke}%
\end{pgfscope}%
\begin{pgfscope}%
\pgfpathrectangle{\pgfqpoint{0.050000in}{0.050000in}}{\pgfqpoint{2.419000in}{2.419000in}}%
\pgfusepath{clip}%
\pgfsetbuttcap%
\pgfsetroundjoin%
\pgfsetlinewidth{1.003750pt}%
\definecolor{currentstroke}{rgb}{0.000000,0.000000,0.000000}%
\pgfsetstrokecolor{currentstroke}%
\pgfsetdash{}{0pt}%
\pgfusepath{stroke}%
\end{pgfscope}%
\begin{pgfscope}%
\pgfpathrectangle{\pgfqpoint{0.050000in}{0.050000in}}{\pgfqpoint{2.419000in}{2.419000in}}%
\pgfusepath{clip}%
\pgfsetbuttcap%
\pgfsetroundjoin%
\pgfsetlinewidth{1.003750pt}%
\definecolor{currentstroke}{rgb}{0.000000,0.000000,0.000000}%
\pgfsetstrokecolor{currentstroke}%
\pgfsetdash{}{0pt}%
\pgfusepath{stroke}%
\end{pgfscope}%
\begin{pgfscope}%
\pgfpathrectangle{\pgfqpoint{0.050000in}{0.050000in}}{\pgfqpoint{2.419000in}{2.419000in}}%
\pgfusepath{clip}%
\pgfsetbuttcap%
\pgfsetroundjoin%
\pgfsetlinewidth{1.003750pt}%
\definecolor{currentstroke}{rgb}{0.000000,0.000000,0.000000}%
\pgfsetstrokecolor{currentstroke}%
\pgfsetdash{}{0pt}%
\pgfusepath{stroke}%
\end{pgfscope}%
\begin{pgfscope}%
\pgfpathrectangle{\pgfqpoint{0.050000in}{0.050000in}}{\pgfqpoint{2.419000in}{2.419000in}}%
\pgfusepath{clip}%
\pgfsetbuttcap%
\pgfsetroundjoin%
\pgfsetlinewidth{1.003750pt}%
\definecolor{currentstroke}{rgb}{0.000000,0.000000,0.000000}%
\pgfsetstrokecolor{currentstroke}%
\pgfsetdash{}{0pt}%
\pgfusepath{stroke}%
\end{pgfscope}%
\begin{pgfscope}%
\pgfpathrectangle{\pgfqpoint{0.050000in}{0.050000in}}{\pgfqpoint{2.419000in}{2.419000in}}%
\pgfusepath{clip}%
\pgfsetbuttcap%
\pgfsetroundjoin%
\pgfsetlinewidth{1.003750pt}%
\definecolor{currentstroke}{rgb}{0.000000,0.000000,0.000000}%
\pgfsetstrokecolor{currentstroke}%
\pgfsetdash{}{0pt}%
\pgfusepath{stroke}%
\end{pgfscope}%
\begin{pgfscope}%
\pgfpathrectangle{\pgfqpoint{0.050000in}{0.050000in}}{\pgfqpoint{2.419000in}{2.419000in}}%
\pgfusepath{clip}%
\pgfsetbuttcap%
\pgfsetroundjoin%
\pgfsetlinewidth{1.003750pt}%
\definecolor{currentstroke}{rgb}{0.000000,0.000000,0.000000}%
\pgfsetstrokecolor{currentstroke}%
\pgfsetdash{}{0pt}%
\pgfusepath{stroke}%
\end{pgfscope}%
\begin{pgfscope}%
\pgfpathrectangle{\pgfqpoint{0.050000in}{0.050000in}}{\pgfqpoint{2.419000in}{2.419000in}}%
\pgfusepath{clip}%
\pgfsetbuttcap%
\pgfsetroundjoin%
\pgfsetlinewidth{1.003750pt}%
\definecolor{currentstroke}{rgb}{0.000000,0.000000,0.000000}%
\pgfsetstrokecolor{currentstroke}%
\pgfsetdash{}{0pt}%
\pgfusepath{stroke}%
\end{pgfscope}%
\begin{pgfscope}%
\pgfpathrectangle{\pgfqpoint{0.050000in}{0.050000in}}{\pgfqpoint{2.419000in}{2.419000in}}%
\pgfusepath{clip}%
\pgfsetbuttcap%
\pgfsetroundjoin%
\pgfsetlinewidth{1.003750pt}%
\definecolor{currentstroke}{rgb}{0.000000,0.000000,0.000000}%
\pgfsetstrokecolor{currentstroke}%
\pgfsetdash{}{0pt}%
\pgfusepath{stroke}%
\end{pgfscope}%
\begin{pgfscope}%
\pgfpathrectangle{\pgfqpoint{0.050000in}{0.050000in}}{\pgfqpoint{2.419000in}{2.419000in}}%
\pgfusepath{clip}%
\pgfsetbuttcap%
\pgfsetroundjoin%
\pgfsetlinewidth{1.003750pt}%
\definecolor{currentstroke}{rgb}{0.000000,0.000000,0.000000}%
\pgfsetstrokecolor{currentstroke}%
\pgfsetdash{}{0pt}%
\pgfusepath{stroke}%
\end{pgfscope}%
\begin{pgfscope}%
\pgfpathrectangle{\pgfqpoint{0.050000in}{0.050000in}}{\pgfqpoint{2.419000in}{2.419000in}}%
\pgfusepath{clip}%
\pgfsetbuttcap%
\pgfsetroundjoin%
\pgfsetlinewidth{1.003750pt}%
\definecolor{currentstroke}{rgb}{0.000000,0.000000,0.000000}%
\pgfsetstrokecolor{currentstroke}%
\pgfsetdash{}{0pt}%
\pgfusepath{stroke}%
\end{pgfscope}%
\begin{pgfscope}%
\pgfpathrectangle{\pgfqpoint{0.050000in}{0.050000in}}{\pgfqpoint{2.419000in}{2.419000in}}%
\pgfusepath{clip}%
\pgfsetbuttcap%
\pgfsetroundjoin%
\pgfsetlinewidth{1.003750pt}%
\definecolor{currentstroke}{rgb}{0.000000,0.000000,0.000000}%
\pgfsetstrokecolor{currentstroke}%
\pgfsetdash{}{0pt}%
\pgfusepath{stroke}%
\end{pgfscope}%
\begin{pgfscope}%
\pgfpathrectangle{\pgfqpoint{0.050000in}{0.050000in}}{\pgfqpoint{2.419000in}{2.419000in}}%
\pgfusepath{clip}%
\pgfsetbuttcap%
\pgfsetroundjoin%
\pgfsetlinewidth{1.003750pt}%
\definecolor{currentstroke}{rgb}{0.000000,0.000000,0.000000}%
\pgfsetstrokecolor{currentstroke}%
\pgfsetdash{}{0pt}%
\pgfusepath{stroke}%
\end{pgfscope}%
\begin{pgfscope}%
\pgfpathrectangle{\pgfqpoint{0.050000in}{0.050000in}}{\pgfqpoint{2.419000in}{2.419000in}}%
\pgfusepath{clip}%
\pgfsetbuttcap%
\pgfsetroundjoin%
\pgfsetlinewidth{1.003750pt}%
\definecolor{currentstroke}{rgb}{0.000000,0.000000,0.000000}%
\pgfsetstrokecolor{currentstroke}%
\pgfsetdash{}{0pt}%
\pgfusepath{stroke}%
\end{pgfscope}%
\begin{pgfscope}%
\pgfpathrectangle{\pgfqpoint{0.050000in}{0.050000in}}{\pgfqpoint{2.419000in}{2.419000in}}%
\pgfusepath{clip}%
\pgfsetbuttcap%
\pgfsetroundjoin%
\pgfsetlinewidth{1.003750pt}%
\definecolor{currentstroke}{rgb}{0.000000,0.000000,0.000000}%
\pgfsetstrokecolor{currentstroke}%
\pgfsetdash{}{0pt}%
\pgfusepath{stroke}%
\end{pgfscope}%
\begin{pgfscope}%
\pgfpathrectangle{\pgfqpoint{0.050000in}{0.050000in}}{\pgfqpoint{2.419000in}{2.419000in}}%
\pgfusepath{clip}%
\pgfsetbuttcap%
\pgfsetroundjoin%
\pgfsetlinewidth{1.003750pt}%
\definecolor{currentstroke}{rgb}{0.000000,0.000000,0.000000}%
\pgfsetstrokecolor{currentstroke}%
\pgfsetdash{}{0pt}%
\pgfusepath{stroke}%
\end{pgfscope}%
\begin{pgfscope}%
\pgfpathrectangle{\pgfqpoint{0.050000in}{0.050000in}}{\pgfqpoint{2.419000in}{2.419000in}}%
\pgfusepath{clip}%
\pgfsetbuttcap%
\pgfsetroundjoin%
\pgfsetlinewidth{1.003750pt}%
\definecolor{currentstroke}{rgb}{0.000000,0.000000,0.000000}%
\pgfsetstrokecolor{currentstroke}%
\pgfsetdash{}{0pt}%
\pgfusepath{stroke}%
\end{pgfscope}%
\begin{pgfscope}%
\pgfpathrectangle{\pgfqpoint{0.050000in}{0.050000in}}{\pgfqpoint{2.419000in}{2.419000in}}%
\pgfusepath{clip}%
\pgfsetbuttcap%
\pgfsetroundjoin%
\pgfsetlinewidth{1.003750pt}%
\definecolor{currentstroke}{rgb}{0.000000,0.000000,0.000000}%
\pgfsetstrokecolor{currentstroke}%
\pgfsetdash{}{0pt}%
\pgfusepath{stroke}%
\end{pgfscope}%
\begin{pgfscope}%
\pgfpathrectangle{\pgfqpoint{0.050000in}{0.050000in}}{\pgfqpoint{2.419000in}{2.419000in}}%
\pgfusepath{clip}%
\pgfsetbuttcap%
\pgfsetroundjoin%
\pgfsetlinewidth{1.003750pt}%
\definecolor{currentstroke}{rgb}{0.000000,0.000000,0.000000}%
\pgfsetstrokecolor{currentstroke}%
\pgfsetdash{}{0pt}%
\pgfusepath{stroke}%
\end{pgfscope}%
\begin{pgfscope}%
\pgfpathrectangle{\pgfqpoint{0.050000in}{0.050000in}}{\pgfqpoint{2.419000in}{2.419000in}}%
\pgfusepath{clip}%
\pgfsetbuttcap%
\pgfsetroundjoin%
\pgfsetlinewidth{1.003750pt}%
\definecolor{currentstroke}{rgb}{0.000000,0.000000,0.000000}%
\pgfsetstrokecolor{currentstroke}%
\pgfsetdash{}{0pt}%
\pgfusepath{stroke}%
\end{pgfscope}%
\begin{pgfscope}%
\pgfpathrectangle{\pgfqpoint{0.050000in}{0.050000in}}{\pgfqpoint{2.419000in}{2.419000in}}%
\pgfusepath{clip}%
\pgfsetbuttcap%
\pgfsetroundjoin%
\pgfsetlinewidth{1.003750pt}%
\definecolor{currentstroke}{rgb}{0.000000,0.000000,0.000000}%
\pgfsetstrokecolor{currentstroke}%
\pgfsetdash{}{0pt}%
\pgfusepath{stroke}%
\end{pgfscope}%
\begin{pgfscope}%
\pgfpathrectangle{\pgfqpoint{0.050000in}{0.050000in}}{\pgfqpoint{2.419000in}{2.419000in}}%
\pgfusepath{clip}%
\pgfsetbuttcap%
\pgfsetroundjoin%
\pgfsetlinewidth{1.003750pt}%
\definecolor{currentstroke}{rgb}{0.000000,0.000000,0.000000}%
\pgfsetstrokecolor{currentstroke}%
\pgfsetdash{}{0pt}%
\pgfusepath{stroke}%
\end{pgfscope}%
\begin{pgfscope}%
\pgfpathrectangle{\pgfqpoint{0.050000in}{0.050000in}}{\pgfqpoint{2.419000in}{2.419000in}}%
\pgfusepath{clip}%
\pgfsetbuttcap%
\pgfsetroundjoin%
\pgfsetlinewidth{1.003750pt}%
\definecolor{currentstroke}{rgb}{0.000000,0.000000,0.000000}%
\pgfsetstrokecolor{currentstroke}%
\pgfsetdash{}{0pt}%
\pgfusepath{stroke}%
\end{pgfscope}%
\begin{pgfscope}%
\pgfpathrectangle{\pgfqpoint{0.050000in}{0.050000in}}{\pgfqpoint{2.419000in}{2.419000in}}%
\pgfusepath{clip}%
\pgfsetbuttcap%
\pgfsetroundjoin%
\pgfsetlinewidth{1.003750pt}%
\definecolor{currentstroke}{rgb}{0.000000,0.000000,0.000000}%
\pgfsetstrokecolor{currentstroke}%
\pgfsetdash{}{0pt}%
\pgfusepath{stroke}%
\end{pgfscope}%
\begin{pgfscope}%
\pgfpathrectangle{\pgfqpoint{0.050000in}{0.050000in}}{\pgfqpoint{2.419000in}{2.419000in}}%
\pgfusepath{clip}%
\pgfsetbuttcap%
\pgfsetroundjoin%
\pgfsetlinewidth{1.003750pt}%
\definecolor{currentstroke}{rgb}{0.000000,0.000000,0.000000}%
\pgfsetstrokecolor{currentstroke}%
\pgfsetdash{}{0pt}%
\pgfusepath{stroke}%
\end{pgfscope}%
\begin{pgfscope}%
\pgfpathrectangle{\pgfqpoint{0.050000in}{0.050000in}}{\pgfqpoint{2.419000in}{2.419000in}}%
\pgfusepath{clip}%
\pgfsetbuttcap%
\pgfsetroundjoin%
\pgfsetlinewidth{1.003750pt}%
\definecolor{currentstroke}{rgb}{0.000000,0.000000,0.000000}%
\pgfsetstrokecolor{currentstroke}%
\pgfsetdash{}{0pt}%
\pgfusepath{stroke}%
\end{pgfscope}%
\begin{pgfscope}%
\pgfpathrectangle{\pgfqpoint{0.050000in}{0.050000in}}{\pgfqpoint{2.419000in}{2.419000in}}%
\pgfusepath{clip}%
\pgfsetbuttcap%
\pgfsetroundjoin%
\pgfsetlinewidth{1.003750pt}%
\definecolor{currentstroke}{rgb}{0.000000,0.000000,0.000000}%
\pgfsetstrokecolor{currentstroke}%
\pgfsetdash{}{0pt}%
\pgfusepath{stroke}%
\end{pgfscope}%
\begin{pgfscope}%
\pgfpathrectangle{\pgfqpoint{0.050000in}{0.050000in}}{\pgfqpoint{2.419000in}{2.419000in}}%
\pgfusepath{clip}%
\pgfsetbuttcap%
\pgfsetroundjoin%
\pgfsetlinewidth{1.003750pt}%
\definecolor{currentstroke}{rgb}{0.000000,0.000000,0.000000}%
\pgfsetstrokecolor{currentstroke}%
\pgfsetdash{}{0pt}%
\pgfusepath{stroke}%
\end{pgfscope}%
\begin{pgfscope}%
\pgfpathrectangle{\pgfqpoint{0.050000in}{0.050000in}}{\pgfqpoint{2.419000in}{2.419000in}}%
\pgfusepath{clip}%
\pgfsetbuttcap%
\pgfsetroundjoin%
\pgfsetlinewidth{1.003750pt}%
\definecolor{currentstroke}{rgb}{0.000000,0.000000,0.000000}%
\pgfsetstrokecolor{currentstroke}%
\pgfsetdash{}{0pt}%
\pgfusepath{stroke}%
\end{pgfscope}%
\begin{pgfscope}%
\pgfpathrectangle{\pgfqpoint{0.050000in}{0.050000in}}{\pgfqpoint{2.419000in}{2.419000in}}%
\pgfusepath{clip}%
\pgfsetbuttcap%
\pgfsetroundjoin%
\pgfsetlinewidth{1.003750pt}%
\definecolor{currentstroke}{rgb}{0.000000,0.000000,0.000000}%
\pgfsetstrokecolor{currentstroke}%
\pgfsetdash{}{0pt}%
\pgfusepath{stroke}%
\end{pgfscope}%
\begin{pgfscope}%
\pgfpathrectangle{\pgfqpoint{0.050000in}{0.050000in}}{\pgfqpoint{2.419000in}{2.419000in}}%
\pgfusepath{clip}%
\pgfsetbuttcap%
\pgfsetroundjoin%
\pgfsetlinewidth{1.003750pt}%
\definecolor{currentstroke}{rgb}{0.000000,0.000000,0.000000}%
\pgfsetstrokecolor{currentstroke}%
\pgfsetdash{}{0pt}%
\pgfusepath{stroke}%
\end{pgfscope}%
\begin{pgfscope}%
\pgfpathrectangle{\pgfqpoint{0.050000in}{0.050000in}}{\pgfqpoint{2.419000in}{2.419000in}}%
\pgfusepath{clip}%
\pgfsetbuttcap%
\pgfsetroundjoin%
\pgfsetlinewidth{1.003750pt}%
\definecolor{currentstroke}{rgb}{0.000000,0.000000,0.000000}%
\pgfsetstrokecolor{currentstroke}%
\pgfsetdash{}{0pt}%
\pgfusepath{stroke}%
\end{pgfscope}%
\begin{pgfscope}%
\pgfpathrectangle{\pgfqpoint{0.050000in}{0.050000in}}{\pgfqpoint{2.419000in}{2.419000in}}%
\pgfusepath{clip}%
\pgfsetbuttcap%
\pgfsetroundjoin%
\pgfsetlinewidth{1.003750pt}%
\definecolor{currentstroke}{rgb}{0.000000,0.000000,0.000000}%
\pgfsetstrokecolor{currentstroke}%
\pgfsetdash{}{0pt}%
\pgfusepath{stroke}%
\end{pgfscope}%
\begin{pgfscope}%
\pgfpathrectangle{\pgfqpoint{0.050000in}{0.050000in}}{\pgfqpoint{2.419000in}{2.419000in}}%
\pgfusepath{clip}%
\pgfsetbuttcap%
\pgfsetroundjoin%
\pgfsetlinewidth{1.003750pt}%
\definecolor{currentstroke}{rgb}{0.000000,0.000000,0.000000}%
\pgfsetstrokecolor{currentstroke}%
\pgfsetdash{}{0pt}%
\pgfusepath{stroke}%
\end{pgfscope}%
\begin{pgfscope}%
\pgfpathrectangle{\pgfqpoint{0.050000in}{0.050000in}}{\pgfqpoint{2.419000in}{2.419000in}}%
\pgfusepath{clip}%
\pgfsetbuttcap%
\pgfsetroundjoin%
\pgfsetlinewidth{1.003750pt}%
\definecolor{currentstroke}{rgb}{0.000000,0.000000,0.000000}%
\pgfsetstrokecolor{currentstroke}%
\pgfsetdash{}{0pt}%
\pgfusepath{stroke}%
\end{pgfscope}%
\begin{pgfscope}%
\pgfpathrectangle{\pgfqpoint{0.050000in}{0.050000in}}{\pgfqpoint{2.419000in}{2.419000in}}%
\pgfusepath{clip}%
\pgfsetbuttcap%
\pgfsetroundjoin%
\pgfsetlinewidth{1.003750pt}%
\definecolor{currentstroke}{rgb}{0.000000,0.000000,0.000000}%
\pgfsetstrokecolor{currentstroke}%
\pgfsetdash{}{0pt}%
\pgfusepath{stroke}%
\end{pgfscope}%
\begin{pgfscope}%
\pgfpathrectangle{\pgfqpoint{0.050000in}{0.050000in}}{\pgfqpoint{2.419000in}{2.419000in}}%
\pgfusepath{clip}%
\pgfsetbuttcap%
\pgfsetroundjoin%
\pgfsetlinewidth{1.003750pt}%
\definecolor{currentstroke}{rgb}{0.000000,0.000000,0.000000}%
\pgfsetstrokecolor{currentstroke}%
\pgfsetdash{}{0pt}%
\pgfusepath{stroke}%
\end{pgfscope}%
\begin{pgfscope}%
\pgfpathrectangle{\pgfqpoint{0.050000in}{0.050000in}}{\pgfqpoint{2.419000in}{2.419000in}}%
\pgfusepath{clip}%
\pgfsetbuttcap%
\pgfsetroundjoin%
\pgfsetlinewidth{1.003750pt}%
\definecolor{currentstroke}{rgb}{0.000000,0.000000,0.000000}%
\pgfsetstrokecolor{currentstroke}%
\pgfsetdash{}{0pt}%
\pgfusepath{stroke}%
\end{pgfscope}%
\begin{pgfscope}%
\pgfpathrectangle{\pgfqpoint{0.050000in}{0.050000in}}{\pgfqpoint{2.419000in}{2.419000in}}%
\pgfusepath{clip}%
\pgfsetbuttcap%
\pgfsetroundjoin%
\pgfsetlinewidth{1.003750pt}%
\definecolor{currentstroke}{rgb}{0.000000,0.000000,0.000000}%
\pgfsetstrokecolor{currentstroke}%
\pgfsetdash{}{0pt}%
\pgfusepath{stroke}%
\end{pgfscope}%
\begin{pgfscope}%
\pgfpathrectangle{\pgfqpoint{0.050000in}{0.050000in}}{\pgfqpoint{2.419000in}{2.419000in}}%
\pgfusepath{clip}%
\pgfsetbuttcap%
\pgfsetroundjoin%
\pgfsetlinewidth{1.003750pt}%
\definecolor{currentstroke}{rgb}{0.000000,0.000000,0.000000}%
\pgfsetstrokecolor{currentstroke}%
\pgfsetdash{}{0pt}%
\pgfusepath{stroke}%
\end{pgfscope}%
\begin{pgfscope}%
\pgfpathrectangle{\pgfqpoint{0.050000in}{0.050000in}}{\pgfqpoint{2.419000in}{2.419000in}}%
\pgfusepath{clip}%
\pgfsetbuttcap%
\pgfsetroundjoin%
\pgfsetlinewidth{1.003750pt}%
\definecolor{currentstroke}{rgb}{0.000000,0.000000,0.000000}%
\pgfsetstrokecolor{currentstroke}%
\pgfsetdash{}{0pt}%
\pgfusepath{stroke}%
\end{pgfscope}%
\begin{pgfscope}%
\pgfpathrectangle{\pgfqpoint{0.050000in}{0.050000in}}{\pgfqpoint{2.419000in}{2.419000in}}%
\pgfusepath{clip}%
\pgfsetbuttcap%
\pgfsetroundjoin%
\pgfsetlinewidth{1.003750pt}%
\definecolor{currentstroke}{rgb}{0.000000,0.000000,0.000000}%
\pgfsetstrokecolor{currentstroke}%
\pgfsetdash{}{0pt}%
\pgfusepath{stroke}%
\end{pgfscope}%
\begin{pgfscope}%
\pgfpathrectangle{\pgfqpoint{0.050000in}{0.050000in}}{\pgfqpoint{2.419000in}{2.419000in}}%
\pgfusepath{clip}%
\pgfsetbuttcap%
\pgfsetroundjoin%
\pgfsetlinewidth{1.003750pt}%
\definecolor{currentstroke}{rgb}{0.000000,0.000000,0.000000}%
\pgfsetstrokecolor{currentstroke}%
\pgfsetdash{}{0pt}%
\pgfusepath{stroke}%
\end{pgfscope}%
\begin{pgfscope}%
\pgfpathrectangle{\pgfqpoint{0.050000in}{0.050000in}}{\pgfqpoint{2.419000in}{2.419000in}}%
\pgfusepath{clip}%
\pgfsetbuttcap%
\pgfsetroundjoin%
\pgfsetlinewidth{1.003750pt}%
\definecolor{currentstroke}{rgb}{0.000000,0.000000,0.000000}%
\pgfsetstrokecolor{currentstroke}%
\pgfsetdash{}{0pt}%
\pgfusepath{stroke}%
\end{pgfscope}%
\begin{pgfscope}%
\pgfpathrectangle{\pgfqpoint{0.050000in}{0.050000in}}{\pgfqpoint{2.419000in}{2.419000in}}%
\pgfusepath{clip}%
\pgfsetbuttcap%
\pgfsetroundjoin%
\pgfsetlinewidth{1.003750pt}%
\definecolor{currentstroke}{rgb}{0.000000,0.000000,0.000000}%
\pgfsetstrokecolor{currentstroke}%
\pgfsetdash{}{0pt}%
\pgfusepath{stroke}%
\end{pgfscope}%
\begin{pgfscope}%
\pgfpathrectangle{\pgfqpoint{0.050000in}{0.050000in}}{\pgfqpoint{2.419000in}{2.419000in}}%
\pgfusepath{clip}%
\pgfsetbuttcap%
\pgfsetroundjoin%
\pgfsetlinewidth{1.003750pt}%
\definecolor{currentstroke}{rgb}{0.000000,0.000000,0.000000}%
\pgfsetstrokecolor{currentstroke}%
\pgfsetdash{}{0pt}%
\pgfusepath{stroke}%
\end{pgfscope}%
\begin{pgfscope}%
\pgfpathrectangle{\pgfqpoint{0.050000in}{0.050000in}}{\pgfqpoint{2.419000in}{2.419000in}}%
\pgfusepath{clip}%
\pgfsetbuttcap%
\pgfsetroundjoin%
\pgfsetlinewidth{1.003750pt}%
\definecolor{currentstroke}{rgb}{0.000000,0.000000,0.000000}%
\pgfsetstrokecolor{currentstroke}%
\pgfsetdash{}{0pt}%
\pgfusepath{stroke}%
\end{pgfscope}%
\begin{pgfscope}%
\pgfpathrectangle{\pgfqpoint{0.050000in}{0.050000in}}{\pgfqpoint{2.419000in}{2.419000in}}%
\pgfusepath{clip}%
\pgfsetbuttcap%
\pgfsetroundjoin%
\pgfsetlinewidth{1.003750pt}%
\definecolor{currentstroke}{rgb}{0.000000,0.000000,0.000000}%
\pgfsetstrokecolor{currentstroke}%
\pgfsetdash{}{0pt}%
\pgfusepath{stroke}%
\end{pgfscope}%
\begin{pgfscope}%
\pgfpathrectangle{\pgfqpoint{0.050000in}{0.050000in}}{\pgfqpoint{2.419000in}{2.419000in}}%
\pgfusepath{clip}%
\pgfsetbuttcap%
\pgfsetroundjoin%
\pgfsetlinewidth{1.003750pt}%
\definecolor{currentstroke}{rgb}{0.000000,0.000000,0.000000}%
\pgfsetstrokecolor{currentstroke}%
\pgfsetdash{}{0pt}%
\pgfusepath{stroke}%
\end{pgfscope}%
\begin{pgfscope}%
\pgfpathrectangle{\pgfqpoint{0.050000in}{0.050000in}}{\pgfqpoint{2.419000in}{2.419000in}}%
\pgfusepath{clip}%
\pgfsetbuttcap%
\pgfsetroundjoin%
\pgfsetlinewidth{1.003750pt}%
\definecolor{currentstroke}{rgb}{0.000000,0.000000,0.000000}%
\pgfsetstrokecolor{currentstroke}%
\pgfsetdash{}{0pt}%
\pgfusepath{stroke}%
\end{pgfscope}%
\begin{pgfscope}%
\pgfpathrectangle{\pgfqpoint{0.050000in}{0.050000in}}{\pgfqpoint{2.419000in}{2.419000in}}%
\pgfusepath{clip}%
\pgfsetbuttcap%
\pgfsetroundjoin%
\pgfsetlinewidth{1.003750pt}%
\definecolor{currentstroke}{rgb}{0.000000,0.000000,0.000000}%
\pgfsetstrokecolor{currentstroke}%
\pgfsetdash{}{0pt}%
\pgfusepath{stroke}%
\end{pgfscope}%
\begin{pgfscope}%
\pgfpathrectangle{\pgfqpoint{0.050000in}{0.050000in}}{\pgfqpoint{2.419000in}{2.419000in}}%
\pgfusepath{clip}%
\pgfsetbuttcap%
\pgfsetroundjoin%
\pgfsetlinewidth{1.003750pt}%
\definecolor{currentstroke}{rgb}{0.000000,0.000000,0.000000}%
\pgfsetstrokecolor{currentstroke}%
\pgfsetdash{}{0pt}%
\pgfusepath{stroke}%
\end{pgfscope}%
\begin{pgfscope}%
\pgfpathrectangle{\pgfqpoint{0.050000in}{0.050000in}}{\pgfqpoint{2.419000in}{2.419000in}}%
\pgfusepath{clip}%
\pgfsetbuttcap%
\pgfsetroundjoin%
\pgfsetlinewidth{1.003750pt}%
\definecolor{currentstroke}{rgb}{0.000000,0.000000,0.000000}%
\pgfsetstrokecolor{currentstroke}%
\pgfsetdash{}{0pt}%
\pgfusepath{stroke}%
\end{pgfscope}%
\begin{pgfscope}%
\pgfpathrectangle{\pgfqpoint{0.050000in}{0.050000in}}{\pgfqpoint{2.419000in}{2.419000in}}%
\pgfusepath{clip}%
\pgfsetbuttcap%
\pgfsetroundjoin%
\pgfsetlinewidth{1.003750pt}%
\definecolor{currentstroke}{rgb}{0.000000,0.000000,0.000000}%
\pgfsetstrokecolor{currentstroke}%
\pgfsetdash{}{0pt}%
\pgfusepath{stroke}%
\end{pgfscope}%
\begin{pgfscope}%
\pgfpathrectangle{\pgfqpoint{0.050000in}{0.050000in}}{\pgfqpoint{2.419000in}{2.419000in}}%
\pgfusepath{clip}%
\pgfsetbuttcap%
\pgfsetroundjoin%
\pgfsetlinewidth{1.003750pt}%
\definecolor{currentstroke}{rgb}{0.000000,0.000000,0.000000}%
\pgfsetstrokecolor{currentstroke}%
\pgfsetdash{}{0pt}%
\pgfusepath{stroke}%
\end{pgfscope}%
\begin{pgfscope}%
\pgfpathrectangle{\pgfqpoint{0.050000in}{0.050000in}}{\pgfqpoint{2.419000in}{2.419000in}}%
\pgfusepath{clip}%
\pgfsetbuttcap%
\pgfsetroundjoin%
\pgfsetlinewidth{1.003750pt}%
\definecolor{currentstroke}{rgb}{0.000000,0.000000,0.000000}%
\pgfsetstrokecolor{currentstroke}%
\pgfsetdash{}{0pt}%
\pgfusepath{stroke}%
\end{pgfscope}%
\begin{pgfscope}%
\pgfpathrectangle{\pgfqpoint{0.050000in}{0.050000in}}{\pgfqpoint{2.419000in}{2.419000in}}%
\pgfusepath{clip}%
\pgfsetbuttcap%
\pgfsetroundjoin%
\pgfsetlinewidth{1.003750pt}%
\definecolor{currentstroke}{rgb}{0.000000,0.000000,0.000000}%
\pgfsetstrokecolor{currentstroke}%
\pgfsetdash{}{0pt}%
\pgfusepath{stroke}%
\end{pgfscope}%
\begin{pgfscope}%
\pgfpathrectangle{\pgfqpoint{0.050000in}{0.050000in}}{\pgfqpoint{2.419000in}{2.419000in}}%
\pgfusepath{clip}%
\pgfsetbuttcap%
\pgfsetroundjoin%
\pgfsetlinewidth{1.003750pt}%
\definecolor{currentstroke}{rgb}{0.000000,0.000000,0.000000}%
\pgfsetstrokecolor{currentstroke}%
\pgfsetdash{}{0pt}%
\pgfusepath{stroke}%
\end{pgfscope}%
\begin{pgfscope}%
\pgfpathrectangle{\pgfqpoint{0.050000in}{0.050000in}}{\pgfqpoint{2.419000in}{2.419000in}}%
\pgfusepath{clip}%
\pgfsetbuttcap%
\pgfsetroundjoin%
\pgfsetlinewidth{1.003750pt}%
\definecolor{currentstroke}{rgb}{0.000000,0.000000,0.000000}%
\pgfsetstrokecolor{currentstroke}%
\pgfsetdash{}{0pt}%
\pgfusepath{stroke}%
\end{pgfscope}%
\begin{pgfscope}%
\pgfpathrectangle{\pgfqpoint{0.050000in}{0.050000in}}{\pgfqpoint{2.419000in}{2.419000in}}%
\pgfusepath{clip}%
\pgfsetbuttcap%
\pgfsetroundjoin%
\pgfsetlinewidth{1.003750pt}%
\definecolor{currentstroke}{rgb}{0.000000,0.000000,0.000000}%
\pgfsetstrokecolor{currentstroke}%
\pgfsetdash{}{0pt}%
\pgfusepath{stroke}%
\end{pgfscope}%
\begin{pgfscope}%
\pgfpathrectangle{\pgfqpoint{0.050000in}{0.050000in}}{\pgfqpoint{2.419000in}{2.419000in}}%
\pgfusepath{clip}%
\pgfsetbuttcap%
\pgfsetroundjoin%
\pgfsetlinewidth{1.003750pt}%
\definecolor{currentstroke}{rgb}{0.000000,0.000000,0.000000}%
\pgfsetstrokecolor{currentstroke}%
\pgfsetdash{}{0pt}%
\pgfusepath{stroke}%
\end{pgfscope}%
\begin{pgfscope}%
\pgfpathrectangle{\pgfqpoint{0.050000in}{0.050000in}}{\pgfqpoint{2.419000in}{2.419000in}}%
\pgfusepath{clip}%
\pgfsetbuttcap%
\pgfsetroundjoin%
\pgfsetlinewidth{1.003750pt}%
\definecolor{currentstroke}{rgb}{0.000000,0.000000,0.000000}%
\pgfsetstrokecolor{currentstroke}%
\pgfsetdash{}{0pt}%
\pgfusepath{stroke}%
\end{pgfscope}%
\begin{pgfscope}%
\pgfpathrectangle{\pgfqpoint{0.050000in}{0.050000in}}{\pgfqpoint{2.419000in}{2.419000in}}%
\pgfusepath{clip}%
\pgfsetbuttcap%
\pgfsetroundjoin%
\pgfsetlinewidth{1.003750pt}%
\definecolor{currentstroke}{rgb}{0.000000,0.000000,0.000000}%
\pgfsetstrokecolor{currentstroke}%
\pgfsetdash{}{0pt}%
\pgfusepath{stroke}%
\end{pgfscope}%
\begin{pgfscope}%
\pgfpathrectangle{\pgfqpoint{0.050000in}{0.050000in}}{\pgfqpoint{2.419000in}{2.419000in}}%
\pgfusepath{clip}%
\pgfsetbuttcap%
\pgfsetroundjoin%
\pgfsetlinewidth{1.003750pt}%
\definecolor{currentstroke}{rgb}{0.000000,0.000000,0.000000}%
\pgfsetstrokecolor{currentstroke}%
\pgfsetdash{}{0pt}%
\pgfusepath{stroke}%
\end{pgfscope}%
\begin{pgfscope}%
\pgfpathrectangle{\pgfqpoint{0.050000in}{0.050000in}}{\pgfqpoint{2.419000in}{2.419000in}}%
\pgfusepath{clip}%
\pgfsetbuttcap%
\pgfsetroundjoin%
\pgfsetlinewidth{1.003750pt}%
\definecolor{currentstroke}{rgb}{0.000000,0.000000,0.000000}%
\pgfsetstrokecolor{currentstroke}%
\pgfsetdash{}{0pt}%
\pgfusepath{stroke}%
\end{pgfscope}%
\begin{pgfscope}%
\pgfpathrectangle{\pgfqpoint{0.050000in}{0.050000in}}{\pgfqpoint{2.419000in}{2.419000in}}%
\pgfusepath{clip}%
\pgfsetbuttcap%
\pgfsetroundjoin%
\pgfsetlinewidth{1.003750pt}%
\definecolor{currentstroke}{rgb}{0.000000,0.000000,0.000000}%
\pgfsetstrokecolor{currentstroke}%
\pgfsetdash{}{0pt}%
\pgfusepath{stroke}%
\end{pgfscope}%
\begin{pgfscope}%
\pgfpathrectangle{\pgfqpoint{0.050000in}{0.050000in}}{\pgfqpoint{2.419000in}{2.419000in}}%
\pgfusepath{clip}%
\pgfsetbuttcap%
\pgfsetroundjoin%
\pgfsetlinewidth{1.003750pt}%
\definecolor{currentstroke}{rgb}{0.000000,0.000000,0.000000}%
\pgfsetstrokecolor{currentstroke}%
\pgfsetdash{}{0pt}%
\pgfusepath{stroke}%
\end{pgfscope}%
\begin{pgfscope}%
\pgfpathrectangle{\pgfqpoint{0.050000in}{0.050000in}}{\pgfqpoint{2.419000in}{2.419000in}}%
\pgfusepath{clip}%
\pgfsetbuttcap%
\pgfsetroundjoin%
\pgfsetlinewidth{1.003750pt}%
\definecolor{currentstroke}{rgb}{0.000000,0.000000,0.000000}%
\pgfsetstrokecolor{currentstroke}%
\pgfsetdash{}{0pt}%
\pgfusepath{stroke}%
\end{pgfscope}%
\begin{pgfscope}%
\pgfpathrectangle{\pgfqpoint{0.050000in}{0.050000in}}{\pgfqpoint{2.419000in}{2.419000in}}%
\pgfusepath{clip}%
\pgfsetbuttcap%
\pgfsetroundjoin%
\pgfsetlinewidth{1.003750pt}%
\definecolor{currentstroke}{rgb}{0.000000,0.000000,0.000000}%
\pgfsetstrokecolor{currentstroke}%
\pgfsetdash{}{0pt}%
\pgfusepath{stroke}%
\end{pgfscope}%
\begin{pgfscope}%
\pgfpathrectangle{\pgfqpoint{0.050000in}{0.050000in}}{\pgfqpoint{2.419000in}{2.419000in}}%
\pgfusepath{clip}%
\pgfsetbuttcap%
\pgfsetroundjoin%
\pgfsetlinewidth{1.003750pt}%
\definecolor{currentstroke}{rgb}{0.000000,0.000000,0.000000}%
\pgfsetstrokecolor{currentstroke}%
\pgfsetdash{}{0pt}%
\pgfusepath{stroke}%
\end{pgfscope}%
\begin{pgfscope}%
\pgfpathrectangle{\pgfqpoint{0.050000in}{0.050000in}}{\pgfqpoint{2.419000in}{2.419000in}}%
\pgfusepath{clip}%
\pgfsetbuttcap%
\pgfsetroundjoin%
\pgfsetlinewidth{1.003750pt}%
\definecolor{currentstroke}{rgb}{0.000000,0.000000,0.000000}%
\pgfsetstrokecolor{currentstroke}%
\pgfsetdash{}{0pt}%
\pgfusepath{stroke}%
\end{pgfscope}%
\begin{pgfscope}%
\pgfpathrectangle{\pgfqpoint{0.050000in}{0.050000in}}{\pgfqpoint{2.419000in}{2.419000in}}%
\pgfusepath{clip}%
\pgfsetbuttcap%
\pgfsetroundjoin%
\pgfsetlinewidth{1.003750pt}%
\definecolor{currentstroke}{rgb}{0.000000,0.000000,0.000000}%
\pgfsetstrokecolor{currentstroke}%
\pgfsetdash{}{0pt}%
\pgfusepath{stroke}%
\end{pgfscope}%
\begin{pgfscope}%
\pgfpathrectangle{\pgfqpoint{0.050000in}{0.050000in}}{\pgfqpoint{2.419000in}{2.419000in}}%
\pgfusepath{clip}%
\pgfsetbuttcap%
\pgfsetroundjoin%
\pgfsetlinewidth{1.003750pt}%
\definecolor{currentstroke}{rgb}{0.000000,0.000000,0.000000}%
\pgfsetstrokecolor{currentstroke}%
\pgfsetdash{}{0pt}%
\pgfusepath{stroke}%
\end{pgfscope}%
\begin{pgfscope}%
\pgfpathrectangle{\pgfqpoint{0.050000in}{0.050000in}}{\pgfqpoint{2.419000in}{2.419000in}}%
\pgfusepath{clip}%
\pgfsetbuttcap%
\pgfsetroundjoin%
\pgfsetlinewidth{1.003750pt}%
\definecolor{currentstroke}{rgb}{0.000000,0.000000,0.000000}%
\pgfsetstrokecolor{currentstroke}%
\pgfsetdash{}{0pt}%
\pgfpathmoveto{\pgfqpoint{2.029220in}{2.479000in}}%
\pgfpathlineto{\pgfqpoint{2.093165in}{2.420575in}}%
\pgfusepath{stroke}%
\end{pgfscope}%
\begin{pgfscope}%
\pgfpathrectangle{\pgfqpoint{0.050000in}{0.050000in}}{\pgfqpoint{2.419000in}{2.419000in}}%
\pgfusepath{clip}%
\pgfsetbuttcap%
\pgfsetroundjoin%
\pgfsetlinewidth{1.003750pt}%
\definecolor{currentstroke}{rgb}{0.000000,0.000000,0.000000}%
\pgfsetstrokecolor{currentstroke}%
\pgfsetdash{}{0pt}%
\pgfpathmoveto{\pgfqpoint{2.093165in}{2.420575in}}%
\pgfpathlineto{\pgfqpoint{2.479000in}{2.442593in}}%
\pgfusepath{stroke}%
\end{pgfscope}%
\begin{pgfscope}%
\pgfpathrectangle{\pgfqpoint{0.050000in}{0.050000in}}{\pgfqpoint{2.419000in}{2.419000in}}%
\pgfusepath{clip}%
\pgfsetbuttcap%
\pgfsetroundjoin%
\pgfsetlinewidth{1.003750pt}%
\definecolor{currentstroke}{rgb}{0.000000,0.000000,0.000000}%
\pgfsetstrokecolor{currentstroke}%
\pgfsetdash{}{0pt}%
\pgfusepath{stroke}%
\end{pgfscope}%
\begin{pgfscope}%
\pgfpathrectangle{\pgfqpoint{0.050000in}{0.050000in}}{\pgfqpoint{2.419000in}{2.419000in}}%
\pgfusepath{clip}%
\pgfsetbuttcap%
\pgfsetroundjoin%
\pgfsetlinewidth{1.003750pt}%
\definecolor{currentstroke}{rgb}{0.000000,0.000000,0.000000}%
\pgfsetstrokecolor{currentstroke}%
\pgfsetdash{}{0pt}%
\pgfusepath{stroke}%
\end{pgfscope}%
\begin{pgfscope}%
\pgfpathrectangle{\pgfqpoint{0.050000in}{0.050000in}}{\pgfqpoint{2.419000in}{2.419000in}}%
\pgfusepath{clip}%
\pgfsetbuttcap%
\pgfsetroundjoin%
\pgfsetlinewidth{1.003750pt}%
\definecolor{currentstroke}{rgb}{0.000000,0.000000,0.000000}%
\pgfsetstrokecolor{currentstroke}%
\pgfsetdash{}{0pt}%
\pgfusepath{stroke}%
\end{pgfscope}%
\begin{pgfscope}%
\pgfpathrectangle{\pgfqpoint{0.050000in}{0.050000in}}{\pgfqpoint{2.419000in}{2.419000in}}%
\pgfusepath{clip}%
\pgfsetbuttcap%
\pgfsetroundjoin%
\pgfsetlinewidth{1.003750pt}%
\definecolor{currentstroke}{rgb}{0.000000,0.000000,0.000000}%
\pgfsetstrokecolor{currentstroke}%
\pgfsetdash{}{0pt}%
\pgfusepath{stroke}%
\end{pgfscope}%
\begin{pgfscope}%
\pgfpathrectangle{\pgfqpoint{0.050000in}{0.050000in}}{\pgfqpoint{2.419000in}{2.419000in}}%
\pgfusepath{clip}%
\pgfsetbuttcap%
\pgfsetroundjoin%
\pgfsetlinewidth{1.003750pt}%
\definecolor{currentstroke}{rgb}{0.000000,0.000000,0.000000}%
\pgfsetstrokecolor{currentstroke}%
\pgfsetdash{}{0pt}%
\pgfusepath{stroke}%
\end{pgfscope}%
\begin{pgfscope}%
\pgfpathrectangle{\pgfqpoint{0.050000in}{0.050000in}}{\pgfqpoint{2.419000in}{2.419000in}}%
\pgfusepath{clip}%
\pgfsetbuttcap%
\pgfsetroundjoin%
\pgfsetlinewidth{1.003750pt}%
\definecolor{currentstroke}{rgb}{0.000000,0.000000,0.000000}%
\pgfsetstrokecolor{currentstroke}%
\pgfsetdash{}{0pt}%
\pgfusepath{stroke}%
\end{pgfscope}%
\begin{pgfscope}%
\pgfpathrectangle{\pgfqpoint{0.050000in}{0.050000in}}{\pgfqpoint{2.419000in}{2.419000in}}%
\pgfusepath{clip}%
\pgfsetbuttcap%
\pgfsetroundjoin%
\pgfsetlinewidth{1.003750pt}%
\definecolor{currentstroke}{rgb}{0.000000,0.000000,0.000000}%
\pgfsetstrokecolor{currentstroke}%
\pgfsetdash{}{0pt}%
\pgfpathmoveto{\pgfqpoint{0.992998in}{2.357796in}}%
\pgfpathlineto{\pgfqpoint{1.161547in}{2.479000in}}%
\pgfusepath{stroke}%
\end{pgfscope}%
\begin{pgfscope}%
\pgfpathrectangle{\pgfqpoint{0.050000in}{0.050000in}}{\pgfqpoint{2.419000in}{2.419000in}}%
\pgfusepath{clip}%
\pgfsetbuttcap%
\pgfsetroundjoin%
\pgfsetlinewidth{1.003750pt}%
\definecolor{currentstroke}{rgb}{0.000000,0.000000,0.000000}%
\pgfsetstrokecolor{currentstroke}%
\pgfsetdash{}{0pt}%
\pgfusepath{stroke}%
\end{pgfscope}%
\begin{pgfscope}%
\pgfpathrectangle{\pgfqpoint{0.050000in}{0.050000in}}{\pgfqpoint{2.419000in}{2.419000in}}%
\pgfusepath{clip}%
\pgfsetbuttcap%
\pgfsetroundjoin%
\pgfsetlinewidth{1.003750pt}%
\definecolor{currentstroke}{rgb}{0.000000,0.000000,0.000000}%
\pgfsetstrokecolor{currentstroke}%
\pgfsetdash{}{0pt}%
\pgfusepath{stroke}%
\end{pgfscope}%
\begin{pgfscope}%
\pgfpathrectangle{\pgfqpoint{0.050000in}{0.050000in}}{\pgfqpoint{2.419000in}{2.419000in}}%
\pgfusepath{clip}%
\pgfsetbuttcap%
\pgfsetroundjoin%
\pgfsetlinewidth{1.003750pt}%
\definecolor{currentstroke}{rgb}{0.000000,0.000000,0.000000}%
\pgfsetstrokecolor{currentstroke}%
\pgfsetdash{}{0pt}%
\pgfpathmoveto{\pgfqpoint{0.308680in}{2.479000in}}%
\pgfpathlineto{\pgfqpoint{0.434807in}{2.325944in}}%
\pgfusepath{stroke}%
\end{pgfscope}%
\begin{pgfscope}%
\pgfpathrectangle{\pgfqpoint{0.050000in}{0.050000in}}{\pgfqpoint{2.419000in}{2.419000in}}%
\pgfusepath{clip}%
\pgfsetbuttcap%
\pgfsetroundjoin%
\pgfsetlinewidth{1.003750pt}%
\definecolor{currentstroke}{rgb}{0.000000,0.000000,0.000000}%
\pgfsetstrokecolor{currentstroke}%
\pgfsetdash{}{0pt}%
\pgfpathmoveto{\pgfqpoint{0.434807in}{2.325944in}}%
\pgfpathlineto{\pgfqpoint{0.992998in}{2.357796in}}%
\pgfusepath{stroke}%
\end{pgfscope}%
\begin{pgfscope}%
\pgfpathrectangle{\pgfqpoint{0.050000in}{0.050000in}}{\pgfqpoint{2.419000in}{2.419000in}}%
\pgfusepath{clip}%
\pgfsetbuttcap%
\pgfsetroundjoin%
\pgfsetlinewidth{1.003750pt}%
\definecolor{currentstroke}{rgb}{0.000000,0.000000,0.000000}%
\pgfsetstrokecolor{currentstroke}%
\pgfsetdash{}{0pt}%
\pgfusepath{stroke}%
\end{pgfscope}%
\begin{pgfscope}%
\pgfpathrectangle{\pgfqpoint{0.050000in}{0.050000in}}{\pgfqpoint{2.419000in}{2.419000in}}%
\pgfusepath{clip}%
\pgfsetbuttcap%
\pgfsetroundjoin%
\pgfsetlinewidth{1.003750pt}%
\definecolor{currentstroke}{rgb}{0.000000,0.000000,0.000000}%
\pgfsetstrokecolor{currentstroke}%
\pgfsetdash{}{0pt}%
\pgfusepath{stroke}%
\end{pgfscope}%
\begin{pgfscope}%
\pgfpathrectangle{\pgfqpoint{0.050000in}{0.050000in}}{\pgfqpoint{2.419000in}{2.419000in}}%
\pgfusepath{clip}%
\pgfsetbuttcap%
\pgfsetroundjoin%
\pgfsetlinewidth{1.003750pt}%
\definecolor{currentstroke}{rgb}{0.000000,0.000000,0.000000}%
\pgfsetstrokecolor{currentstroke}%
\pgfsetdash{}{0pt}%
\pgfusepath{stroke}%
\end{pgfscope}%
\begin{pgfscope}%
\pgfpathrectangle{\pgfqpoint{0.050000in}{0.050000in}}{\pgfqpoint{2.419000in}{2.419000in}}%
\pgfusepath{clip}%
\pgfsetbuttcap%
\pgfsetroundjoin%
\pgfsetlinewidth{1.003750pt}%
\definecolor{currentstroke}{rgb}{0.000000,0.000000,0.000000}%
\pgfsetstrokecolor{currentstroke}%
\pgfsetdash{}{0pt}%
\pgfusepath{stroke}%
\end{pgfscope}%
\begin{pgfscope}%
\pgfpathrectangle{\pgfqpoint{0.050000in}{0.050000in}}{\pgfqpoint{2.419000in}{2.419000in}}%
\pgfusepath{clip}%
\pgfsetbuttcap%
\pgfsetroundjoin%
\pgfsetlinewidth{1.003750pt}%
\definecolor{currentstroke}{rgb}{0.000000,0.000000,0.000000}%
\pgfsetstrokecolor{currentstroke}%
\pgfsetdash{}{0pt}%
\pgfpathmoveto{\pgfqpoint{0.992998in}{2.357796in}}%
\pgfpathlineto{\pgfqpoint{0.991469in}{2.479000in}}%
\pgfusepath{stroke}%
\end{pgfscope}%
\begin{pgfscope}%
\pgfpathrectangle{\pgfqpoint{0.050000in}{0.050000in}}{\pgfqpoint{2.419000in}{2.419000in}}%
\pgfusepath{clip}%
\pgfsetbuttcap%
\pgfsetroundjoin%
\pgfsetlinewidth{1.003750pt}%
\definecolor{currentstroke}{rgb}{0.000000,0.000000,0.000000}%
\pgfsetstrokecolor{currentstroke}%
\pgfsetdash{}{0pt}%
\pgfusepath{stroke}%
\end{pgfscope}%
\begin{pgfscope}%
\pgfpathrectangle{\pgfqpoint{0.050000in}{0.050000in}}{\pgfqpoint{2.419000in}{2.419000in}}%
\pgfusepath{clip}%
\pgfsetbuttcap%
\pgfsetroundjoin%
\pgfsetlinewidth{1.003750pt}%
\definecolor{currentstroke}{rgb}{0.000000,0.000000,0.000000}%
\pgfsetstrokecolor{currentstroke}%
\pgfsetdash{}{0pt}%
\pgfusepath{stroke}%
\end{pgfscope}%
\begin{pgfscope}%
\pgfpathrectangle{\pgfqpoint{0.050000in}{0.050000in}}{\pgfqpoint{2.419000in}{2.419000in}}%
\pgfusepath{clip}%
\pgfsetbuttcap%
\pgfsetroundjoin%
\pgfsetlinewidth{1.003750pt}%
\definecolor{currentstroke}{rgb}{0.000000,0.000000,0.000000}%
\pgfsetstrokecolor{currentstroke}%
\pgfsetdash{}{0pt}%
\pgfusepath{stroke}%
\end{pgfscope}%
\begin{pgfscope}%
\pgfpathrectangle{\pgfqpoint{0.050000in}{0.050000in}}{\pgfqpoint{2.419000in}{2.419000in}}%
\pgfusepath{clip}%
\pgfsetbuttcap%
\pgfsetroundjoin%
\pgfsetlinewidth{1.003750pt}%
\definecolor{currentstroke}{rgb}{0.000000,0.000000,0.000000}%
\pgfsetstrokecolor{currentstroke}%
\pgfsetdash{}{0pt}%
\pgfusepath{stroke}%
\end{pgfscope}%
\begin{pgfscope}%
\pgfpathrectangle{\pgfqpoint{0.050000in}{0.050000in}}{\pgfqpoint{2.419000in}{2.419000in}}%
\pgfusepath{clip}%
\pgfsetbuttcap%
\pgfsetroundjoin%
\pgfsetlinewidth{1.003750pt}%
\definecolor{currentstroke}{rgb}{0.000000,0.000000,0.000000}%
\pgfsetstrokecolor{currentstroke}%
\pgfsetdash{}{0pt}%
\pgfusepath{stroke}%
\end{pgfscope}%
\begin{pgfscope}%
\pgfpathrectangle{\pgfqpoint{0.050000in}{0.050000in}}{\pgfqpoint{2.419000in}{2.419000in}}%
\pgfusepath{clip}%
\pgfsetbuttcap%
\pgfsetroundjoin%
\pgfsetlinewidth{1.003750pt}%
\definecolor{currentstroke}{rgb}{0.000000,0.000000,0.000000}%
\pgfsetstrokecolor{currentstroke}%
\pgfsetdash{}{0pt}%
\pgfusepath{stroke}%
\end{pgfscope}%
\begin{pgfscope}%
\pgfpathrectangle{\pgfqpoint{0.050000in}{0.050000in}}{\pgfqpoint{2.419000in}{2.419000in}}%
\pgfusepath{clip}%
\pgfsetbuttcap%
\pgfsetroundjoin%
\pgfsetlinewidth{1.003750pt}%
\definecolor{currentstroke}{rgb}{0.000000,0.000000,0.000000}%
\pgfsetstrokecolor{currentstroke}%
\pgfsetdash{}{0pt}%
\pgfusepath{stroke}%
\end{pgfscope}%
\begin{pgfscope}%
\pgfpathrectangle{\pgfqpoint{0.050000in}{0.050000in}}{\pgfqpoint{2.419000in}{2.419000in}}%
\pgfusepath{clip}%
\pgfsetbuttcap%
\pgfsetroundjoin%
\pgfsetlinewidth{1.003750pt}%
\definecolor{currentstroke}{rgb}{0.000000,0.000000,0.000000}%
\pgfsetstrokecolor{currentstroke}%
\pgfsetdash{}{0pt}%
\pgfusepath{stroke}%
\end{pgfscope}%
\begin{pgfscope}%
\pgfpathrectangle{\pgfqpoint{0.050000in}{0.050000in}}{\pgfqpoint{2.419000in}{2.419000in}}%
\pgfusepath{clip}%
\pgfsetbuttcap%
\pgfsetroundjoin%
\pgfsetlinewidth{1.003750pt}%
\definecolor{currentstroke}{rgb}{0.000000,0.000000,0.000000}%
\pgfsetstrokecolor{currentstroke}%
\pgfsetdash{}{0pt}%
\pgfusepath{stroke}%
\end{pgfscope}%
\begin{pgfscope}%
\pgfpathrectangle{\pgfqpoint{0.050000in}{0.050000in}}{\pgfqpoint{2.419000in}{2.419000in}}%
\pgfusepath{clip}%
\pgfsetbuttcap%
\pgfsetroundjoin%
\pgfsetlinewidth{1.003750pt}%
\definecolor{currentstroke}{rgb}{0.000000,0.000000,0.000000}%
\pgfsetstrokecolor{currentstroke}%
\pgfsetdash{}{0pt}%
\pgfusepath{stroke}%
\end{pgfscope}%
\begin{pgfscope}%
\pgfpathrectangle{\pgfqpoint{0.050000in}{0.050000in}}{\pgfqpoint{2.419000in}{2.419000in}}%
\pgfusepath{clip}%
\pgfsetbuttcap%
\pgfsetroundjoin%
\pgfsetlinewidth{1.003750pt}%
\definecolor{currentstroke}{rgb}{0.000000,0.000000,0.000000}%
\pgfsetstrokecolor{currentstroke}%
\pgfsetdash{}{0pt}%
\pgfusepath{stroke}%
\end{pgfscope}%
\begin{pgfscope}%
\pgfpathrectangle{\pgfqpoint{0.050000in}{0.050000in}}{\pgfqpoint{2.419000in}{2.419000in}}%
\pgfusepath{clip}%
\pgfsetbuttcap%
\pgfsetroundjoin%
\pgfsetlinewidth{1.003750pt}%
\definecolor{currentstroke}{rgb}{0.000000,0.000000,0.000000}%
\pgfsetstrokecolor{currentstroke}%
\pgfsetdash{}{0pt}%
\pgfusepath{stroke}%
\end{pgfscope}%
\begin{pgfscope}%
\pgfpathrectangle{\pgfqpoint{0.050000in}{0.050000in}}{\pgfqpoint{2.419000in}{2.419000in}}%
\pgfusepath{clip}%
\pgfsetbuttcap%
\pgfsetroundjoin%
\pgfsetlinewidth{1.003750pt}%
\definecolor{currentstroke}{rgb}{0.000000,0.000000,0.000000}%
\pgfsetstrokecolor{currentstroke}%
\pgfsetdash{}{0pt}%
\pgfusepath{stroke}%
\end{pgfscope}%
\begin{pgfscope}%
\pgfpathrectangle{\pgfqpoint{0.050000in}{0.050000in}}{\pgfqpoint{2.419000in}{2.419000in}}%
\pgfusepath{clip}%
\pgfsetbuttcap%
\pgfsetroundjoin%
\pgfsetlinewidth{1.003750pt}%
\definecolor{currentstroke}{rgb}{0.000000,0.000000,0.000000}%
\pgfsetstrokecolor{currentstroke}%
\pgfsetdash{}{0pt}%
\pgfusepath{stroke}%
\end{pgfscope}%
\begin{pgfscope}%
\pgfpathrectangle{\pgfqpoint{0.050000in}{0.050000in}}{\pgfqpoint{2.419000in}{2.419000in}}%
\pgfusepath{clip}%
\pgfsetbuttcap%
\pgfsetroundjoin%
\pgfsetlinewidth{1.003750pt}%
\definecolor{currentstroke}{rgb}{0.000000,0.000000,0.000000}%
\pgfsetstrokecolor{currentstroke}%
\pgfsetdash{}{0pt}%
\pgfpathmoveto{\pgfqpoint{1.774212in}{2.161870in}}%
\pgfpathlineto{\pgfqpoint{2.093165in}{2.420575in}}%
\pgfusepath{stroke}%
\end{pgfscope}%
\begin{pgfscope}%
\pgfpathrectangle{\pgfqpoint{0.050000in}{0.050000in}}{\pgfqpoint{2.419000in}{2.419000in}}%
\pgfusepath{clip}%
\pgfsetbuttcap%
\pgfsetroundjoin%
\pgfsetlinewidth{1.003750pt}%
\definecolor{currentstroke}{rgb}{0.000000,0.000000,0.000000}%
\pgfsetstrokecolor{currentstroke}%
\pgfsetdash{}{0pt}%
\pgfusepath{stroke}%
\end{pgfscope}%
\begin{pgfscope}%
\pgfpathrectangle{\pgfqpoint{0.050000in}{0.050000in}}{\pgfqpoint{2.419000in}{2.419000in}}%
\pgfusepath{clip}%
\pgfsetbuttcap%
\pgfsetroundjoin%
\pgfsetlinewidth{1.003750pt}%
\definecolor{currentstroke}{rgb}{0.000000,0.000000,0.000000}%
\pgfsetstrokecolor{currentstroke}%
\pgfsetdash{}{0pt}%
\pgfusepath{stroke}%
\end{pgfscope}%
\begin{pgfscope}%
\pgfpathrectangle{\pgfqpoint{0.050000in}{0.050000in}}{\pgfqpoint{2.419000in}{2.419000in}}%
\pgfusepath{clip}%
\pgfsetbuttcap%
\pgfsetroundjoin%
\pgfsetlinewidth{1.003750pt}%
\definecolor{currentstroke}{rgb}{0.000000,0.000000,0.000000}%
\pgfsetstrokecolor{currentstroke}%
\pgfsetdash{}{0pt}%
\pgfpathmoveto{\pgfqpoint{0.992998in}{2.357796in}}%
\pgfpathlineto{\pgfqpoint{1.203023in}{2.128059in}}%
\pgfusepath{stroke}%
\end{pgfscope}%
\begin{pgfscope}%
\pgfpathrectangle{\pgfqpoint{0.050000in}{0.050000in}}{\pgfqpoint{2.419000in}{2.419000in}}%
\pgfusepath{clip}%
\pgfsetbuttcap%
\pgfsetroundjoin%
\pgfsetlinewidth{1.003750pt}%
\definecolor{currentstroke}{rgb}{0.000000,0.000000,0.000000}%
\pgfsetstrokecolor{currentstroke}%
\pgfsetdash{}{0pt}%
\pgfpathmoveto{\pgfqpoint{1.203023in}{2.128059in}}%
\pgfpathlineto{\pgfqpoint{1.774212in}{2.161870in}}%
\pgfusepath{stroke}%
\end{pgfscope}%
\begin{pgfscope}%
\pgfpathrectangle{\pgfqpoint{0.050000in}{0.050000in}}{\pgfqpoint{2.419000in}{2.419000in}}%
\pgfusepath{clip}%
\pgfsetbuttcap%
\pgfsetroundjoin%
\pgfsetlinewidth{1.003750pt}%
\definecolor{currentstroke}{rgb}{0.000000,0.000000,0.000000}%
\pgfsetstrokecolor{currentstroke}%
\pgfsetdash{}{0pt}%
\pgfusepath{stroke}%
\end{pgfscope}%
\begin{pgfscope}%
\pgfpathrectangle{\pgfqpoint{0.050000in}{0.050000in}}{\pgfqpoint{2.419000in}{2.419000in}}%
\pgfusepath{clip}%
\pgfsetbuttcap%
\pgfsetroundjoin%
\pgfsetlinewidth{1.003750pt}%
\definecolor{currentstroke}{rgb}{0.000000,0.000000,0.000000}%
\pgfsetstrokecolor{currentstroke}%
\pgfsetdash{}{0pt}%
\pgfusepath{stroke}%
\end{pgfscope}%
\begin{pgfscope}%
\pgfpathrectangle{\pgfqpoint{0.050000in}{0.050000in}}{\pgfqpoint{2.419000in}{2.419000in}}%
\pgfusepath{clip}%
\pgfsetbuttcap%
\pgfsetroundjoin%
\pgfsetlinewidth{1.003750pt}%
\definecolor{currentstroke}{rgb}{0.000000,0.000000,0.000000}%
\pgfsetstrokecolor{currentstroke}%
\pgfsetdash{}{0pt}%
\pgfusepath{stroke}%
\end{pgfscope}%
\begin{pgfscope}%
\pgfpathrectangle{\pgfqpoint{0.050000in}{0.050000in}}{\pgfqpoint{2.419000in}{2.419000in}}%
\pgfusepath{clip}%
\pgfsetbuttcap%
\pgfsetroundjoin%
\pgfsetlinewidth{1.003750pt}%
\definecolor{currentstroke}{rgb}{0.000000,0.000000,0.000000}%
\pgfsetstrokecolor{currentstroke}%
\pgfsetdash{}{0pt}%
\pgfusepath{stroke}%
\end{pgfscope}%
\begin{pgfscope}%
\pgfpathrectangle{\pgfqpoint{0.050000in}{0.050000in}}{\pgfqpoint{2.419000in}{2.419000in}}%
\pgfusepath{clip}%
\pgfsetbuttcap%
\pgfsetroundjoin%
\pgfsetlinewidth{1.003750pt}%
\definecolor{currentstroke}{rgb}{0.000000,0.000000,0.000000}%
\pgfsetstrokecolor{currentstroke}%
\pgfsetdash{}{0pt}%
\pgfpathmoveto{\pgfqpoint{1.774212in}{2.161870in}}%
\pgfpathlineto{\pgfqpoint{1.778706in}{2.381113in}}%
\pgfusepath{stroke}%
\end{pgfscope}%
\begin{pgfscope}%
\pgfpathrectangle{\pgfqpoint{0.050000in}{0.050000in}}{\pgfqpoint{2.419000in}{2.419000in}}%
\pgfusepath{clip}%
\pgfsetbuttcap%
\pgfsetroundjoin%
\pgfsetlinewidth{1.003750pt}%
\definecolor{currentstroke}{rgb}{0.000000,0.000000,0.000000}%
\pgfsetstrokecolor{currentstroke}%
\pgfsetdash{}{0pt}%
\pgfusepath{stroke}%
\end{pgfscope}%
\begin{pgfscope}%
\pgfpathrectangle{\pgfqpoint{0.050000in}{0.050000in}}{\pgfqpoint{2.419000in}{2.419000in}}%
\pgfusepath{clip}%
\pgfsetbuttcap%
\pgfsetroundjoin%
\pgfsetlinewidth{1.003750pt}%
\definecolor{currentstroke}{rgb}{0.000000,0.000000,0.000000}%
\pgfsetstrokecolor{currentstroke}%
\pgfsetdash{}{0pt}%
\pgfusepath{stroke}%
\end{pgfscope}%
\begin{pgfscope}%
\pgfpathrectangle{\pgfqpoint{0.050000in}{0.050000in}}{\pgfqpoint{2.419000in}{2.419000in}}%
\pgfusepath{clip}%
\pgfsetbuttcap%
\pgfsetroundjoin%
\pgfsetlinewidth{1.003750pt}%
\definecolor{currentstroke}{rgb}{0.000000,0.000000,0.000000}%
\pgfsetstrokecolor{currentstroke}%
\pgfsetdash{}{0pt}%
\pgfpathmoveto{\pgfqpoint{0.043119in}{2.059399in}}%
\pgfpathlineto{\pgfqpoint{0.434807in}{2.325944in}}%
\pgfusepath{stroke}%
\end{pgfscope}%
\begin{pgfscope}%
\pgfpathrectangle{\pgfqpoint{0.050000in}{0.050000in}}{\pgfqpoint{2.419000in}{2.419000in}}%
\pgfusepath{clip}%
\pgfsetbuttcap%
\pgfsetroundjoin%
\pgfsetlinewidth{1.003750pt}%
\definecolor{currentstroke}{rgb}{0.000000,0.000000,0.000000}%
\pgfsetstrokecolor{currentstroke}%
\pgfsetdash{}{0pt}%
\pgfusepath{stroke}%
\end{pgfscope}%
\begin{pgfscope}%
\pgfpathrectangle{\pgfqpoint{0.050000in}{0.050000in}}{\pgfqpoint{2.419000in}{2.419000in}}%
\pgfusepath{clip}%
\pgfsetbuttcap%
\pgfsetroundjoin%
\pgfsetlinewidth{1.003750pt}%
\definecolor{currentstroke}{rgb}{0.000000,0.000000,0.000000}%
\pgfsetstrokecolor{currentstroke}%
\pgfsetdash{}{0pt}%
\pgfusepath{stroke}%
\end{pgfscope}%
\begin{pgfscope}%
\pgfpathrectangle{\pgfqpoint{0.050000in}{0.050000in}}{\pgfqpoint{2.419000in}{2.419000in}}%
\pgfusepath{clip}%
\pgfsetbuttcap%
\pgfsetroundjoin%
\pgfsetlinewidth{1.003750pt}%
\definecolor{currentstroke}{rgb}{0.000000,0.000000,0.000000}%
\pgfsetstrokecolor{currentstroke}%
\pgfsetdash{}{0pt}%
\pgfusepath{stroke}%
\end{pgfscope}%
\begin{pgfscope}%
\pgfpathrectangle{\pgfqpoint{0.050000in}{0.050000in}}{\pgfqpoint{2.419000in}{2.419000in}}%
\pgfusepath{clip}%
\pgfsetbuttcap%
\pgfsetroundjoin%
\pgfsetlinewidth{1.003750pt}%
\definecolor{currentstroke}{rgb}{0.000000,0.000000,0.000000}%
\pgfsetstrokecolor{currentstroke}%
\pgfsetdash{}{0pt}%
\pgfpathmoveto{\pgfqpoint{0.040000in}{2.059214in}}%
\pgfpathlineto{\pgfqpoint{0.043119in}{2.059399in}}%
\pgfusepath{stroke}%
\end{pgfscope}%
\begin{pgfscope}%
\pgfpathrectangle{\pgfqpoint{0.050000in}{0.050000in}}{\pgfqpoint{2.419000in}{2.419000in}}%
\pgfusepath{clip}%
\pgfsetbuttcap%
\pgfsetroundjoin%
\pgfsetlinewidth{1.003750pt}%
\definecolor{currentstroke}{rgb}{0.000000,0.000000,0.000000}%
\pgfsetstrokecolor{currentstroke}%
\pgfsetdash{}{0pt}%
\pgfusepath{stroke}%
\end{pgfscope}%
\begin{pgfscope}%
\pgfpathrectangle{\pgfqpoint{0.050000in}{0.050000in}}{\pgfqpoint{2.419000in}{2.419000in}}%
\pgfusepath{clip}%
\pgfsetbuttcap%
\pgfsetroundjoin%
\pgfsetlinewidth{1.003750pt}%
\definecolor{currentstroke}{rgb}{0.000000,0.000000,0.000000}%
\pgfsetstrokecolor{currentstroke}%
\pgfsetdash{}{0pt}%
\pgfusepath{stroke}%
\end{pgfscope}%
\begin{pgfscope}%
\pgfpathrectangle{\pgfqpoint{0.050000in}{0.050000in}}{\pgfqpoint{2.419000in}{2.419000in}}%
\pgfusepath{clip}%
\pgfsetbuttcap%
\pgfsetroundjoin%
\pgfsetlinewidth{1.003750pt}%
\definecolor{currentstroke}{rgb}{0.000000,0.000000,0.000000}%
\pgfsetstrokecolor{currentstroke}%
\pgfsetdash{}{0pt}%
\pgfusepath{stroke}%
\end{pgfscope}%
\begin{pgfscope}%
\pgfpathrectangle{\pgfqpoint{0.050000in}{0.050000in}}{\pgfqpoint{2.419000in}{2.419000in}}%
\pgfusepath{clip}%
\pgfsetbuttcap%
\pgfsetroundjoin%
\pgfsetlinewidth{1.003750pt}%
\definecolor{currentstroke}{rgb}{0.000000,0.000000,0.000000}%
\pgfsetstrokecolor{currentstroke}%
\pgfsetdash{}{0pt}%
\pgfusepath{stroke}%
\end{pgfscope}%
\begin{pgfscope}%
\pgfpathrectangle{\pgfqpoint{0.050000in}{0.050000in}}{\pgfqpoint{2.419000in}{2.419000in}}%
\pgfusepath{clip}%
\pgfsetbuttcap%
\pgfsetroundjoin%
\pgfsetlinewidth{1.003750pt}%
\definecolor{currentstroke}{rgb}{0.000000,0.000000,0.000000}%
\pgfsetstrokecolor{currentstroke}%
\pgfsetdash{}{0pt}%
\pgfpathmoveto{\pgfqpoint{0.043119in}{2.059399in}}%
\pgfpathlineto{\pgfqpoint{0.040000in}{2.117868in}}%
\pgfusepath{stroke}%
\end{pgfscope}%
\begin{pgfscope}%
\pgfpathrectangle{\pgfqpoint{0.050000in}{0.050000in}}{\pgfqpoint{2.419000in}{2.419000in}}%
\pgfusepath{clip}%
\pgfsetbuttcap%
\pgfsetroundjoin%
\pgfsetlinewidth{1.003750pt}%
\definecolor{currentstroke}{rgb}{0.000000,0.000000,0.000000}%
\pgfsetstrokecolor{currentstroke}%
\pgfsetdash{}{0pt}%
\pgfusepath{stroke}%
\end{pgfscope}%
\begin{pgfscope}%
\pgfpathrectangle{\pgfqpoint{0.050000in}{0.050000in}}{\pgfqpoint{2.419000in}{2.419000in}}%
\pgfusepath{clip}%
\pgfsetbuttcap%
\pgfsetroundjoin%
\pgfsetlinewidth{1.003750pt}%
\definecolor{currentstroke}{rgb}{0.000000,0.000000,0.000000}%
\pgfsetstrokecolor{currentstroke}%
\pgfsetdash{}{0pt}%
\pgfusepath{stroke}%
\end{pgfscope}%
\begin{pgfscope}%
\pgfpathrectangle{\pgfqpoint{0.050000in}{0.050000in}}{\pgfqpoint{2.419000in}{2.419000in}}%
\pgfusepath{clip}%
\pgfsetbuttcap%
\pgfsetroundjoin%
\pgfsetlinewidth{1.003750pt}%
\definecolor{currentstroke}{rgb}{0.000000,0.000000,0.000000}%
\pgfsetstrokecolor{currentstroke}%
\pgfsetdash{}{0pt}%
\pgfusepath{stroke}%
\end{pgfscope}%
\begin{pgfscope}%
\pgfpathrectangle{\pgfqpoint{0.050000in}{0.050000in}}{\pgfqpoint{2.419000in}{2.419000in}}%
\pgfusepath{clip}%
\pgfsetbuttcap%
\pgfsetroundjoin%
\pgfsetlinewidth{1.003750pt}%
\definecolor{currentstroke}{rgb}{0.000000,0.000000,0.000000}%
\pgfsetstrokecolor{currentstroke}%
\pgfsetdash{}{0pt}%
\pgfusepath{stroke}%
\end{pgfscope}%
\begin{pgfscope}%
\pgfpathrectangle{\pgfqpoint{0.050000in}{0.050000in}}{\pgfqpoint{2.419000in}{2.419000in}}%
\pgfusepath{clip}%
\pgfsetbuttcap%
\pgfsetroundjoin%
\pgfsetlinewidth{1.003750pt}%
\definecolor{currentstroke}{rgb}{0.000000,0.000000,0.000000}%
\pgfsetstrokecolor{currentstroke}%
\pgfsetdash{}{0pt}%
\pgfusepath{stroke}%
\end{pgfscope}%
\begin{pgfscope}%
\pgfpathrectangle{\pgfqpoint{0.050000in}{0.050000in}}{\pgfqpoint{2.419000in}{2.419000in}}%
\pgfusepath{clip}%
\pgfsetbuttcap%
\pgfsetroundjoin%
\pgfsetlinewidth{1.003750pt}%
\definecolor{currentstroke}{rgb}{0.000000,0.000000,0.000000}%
\pgfsetstrokecolor{currentstroke}%
\pgfsetdash{}{0pt}%
\pgfusepath{stroke}%
\end{pgfscope}%
\begin{pgfscope}%
\pgfpathrectangle{\pgfqpoint{0.050000in}{0.050000in}}{\pgfqpoint{2.419000in}{2.419000in}}%
\pgfusepath{clip}%
\pgfsetbuttcap%
\pgfsetroundjoin%
\pgfsetlinewidth{1.003750pt}%
\definecolor{currentstroke}{rgb}{0.000000,0.000000,0.000000}%
\pgfsetstrokecolor{currentstroke}%
\pgfsetdash{}{0pt}%
\pgfpathmoveto{\pgfqpoint{1.774212in}{2.161870in}}%
\pgfpathlineto{\pgfqpoint{2.019364in}{1.917777in}}%
\pgfusepath{stroke}%
\end{pgfscope}%
\begin{pgfscope}%
\pgfpathrectangle{\pgfqpoint{0.050000in}{0.050000in}}{\pgfqpoint{2.419000in}{2.419000in}}%
\pgfusepath{clip}%
\pgfsetbuttcap%
\pgfsetroundjoin%
\pgfsetlinewidth{1.003750pt}%
\definecolor{currentstroke}{rgb}{0.000000,0.000000,0.000000}%
\pgfsetstrokecolor{currentstroke}%
\pgfsetdash{}{0pt}%
\pgfpathmoveto{\pgfqpoint{2.019364in}{1.917777in}}%
\pgfpathlineto{\pgfqpoint{2.479000in}{1.946041in}}%
\pgfusepath{stroke}%
\end{pgfscope}%
\begin{pgfscope}%
\pgfpathrectangle{\pgfqpoint{0.050000in}{0.050000in}}{\pgfqpoint{2.419000in}{2.419000in}}%
\pgfusepath{clip}%
\pgfsetbuttcap%
\pgfsetroundjoin%
\pgfsetlinewidth{1.003750pt}%
\definecolor{currentstroke}{rgb}{0.000000,0.000000,0.000000}%
\pgfsetstrokecolor{currentstroke}%
\pgfsetdash{}{0pt}%
\pgfusepath{stroke}%
\end{pgfscope}%
\begin{pgfscope}%
\pgfpathrectangle{\pgfqpoint{0.050000in}{0.050000in}}{\pgfqpoint{2.419000in}{2.419000in}}%
\pgfusepath{clip}%
\pgfsetbuttcap%
\pgfsetroundjoin%
\pgfsetlinewidth{1.003750pt}%
\definecolor{currentstroke}{rgb}{0.000000,0.000000,0.000000}%
\pgfsetstrokecolor{currentstroke}%
\pgfsetdash{}{0pt}%
\pgfusepath{stroke}%
\end{pgfscope}%
\begin{pgfscope}%
\pgfpathrectangle{\pgfqpoint{0.050000in}{0.050000in}}{\pgfqpoint{2.419000in}{2.419000in}}%
\pgfusepath{clip}%
\pgfsetbuttcap%
\pgfsetroundjoin%
\pgfsetlinewidth{1.003750pt}%
\definecolor{currentstroke}{rgb}{0.000000,0.000000,0.000000}%
\pgfsetstrokecolor{currentstroke}%
\pgfsetdash{}{0pt}%
\pgfusepath{stroke}%
\end{pgfscope}%
\begin{pgfscope}%
\pgfpathrectangle{\pgfqpoint{0.050000in}{0.050000in}}{\pgfqpoint{2.419000in}{2.419000in}}%
\pgfusepath{clip}%
\pgfsetbuttcap%
\pgfsetroundjoin%
\pgfsetlinewidth{1.003750pt}%
\definecolor{currentstroke}{rgb}{0.000000,0.000000,0.000000}%
\pgfsetstrokecolor{currentstroke}%
\pgfsetdash{}{0pt}%
\pgfusepath{stroke}%
\end{pgfscope}%
\begin{pgfscope}%
\pgfpathrectangle{\pgfqpoint{0.050000in}{0.050000in}}{\pgfqpoint{2.419000in}{2.419000in}}%
\pgfusepath{clip}%
\pgfsetbuttcap%
\pgfsetroundjoin%
\pgfsetlinewidth{1.003750pt}%
\definecolor{currentstroke}{rgb}{0.000000,0.000000,0.000000}%
\pgfsetstrokecolor{currentstroke}%
\pgfsetdash{}{0pt}%
\pgfusepath{stroke}%
\end{pgfscope}%
\begin{pgfscope}%
\pgfpathrectangle{\pgfqpoint{0.050000in}{0.050000in}}{\pgfqpoint{2.419000in}{2.419000in}}%
\pgfusepath{clip}%
\pgfsetbuttcap%
\pgfsetroundjoin%
\pgfsetlinewidth{1.003750pt}%
\definecolor{currentstroke}{rgb}{0.000000,0.000000,0.000000}%
\pgfsetstrokecolor{currentstroke}%
\pgfsetdash{}{0pt}%
\pgfpathmoveto{\pgfqpoint{0.831359in}{1.844725in}}%
\pgfpathlineto{\pgfqpoint{1.203023in}{2.128059in}}%
\pgfusepath{stroke}%
\end{pgfscope}%
\begin{pgfscope}%
\pgfpathrectangle{\pgfqpoint{0.050000in}{0.050000in}}{\pgfqpoint{2.419000in}{2.419000in}}%
\pgfusepath{clip}%
\pgfsetbuttcap%
\pgfsetroundjoin%
\pgfsetlinewidth{1.003750pt}%
\definecolor{currentstroke}{rgb}{0.000000,0.000000,0.000000}%
\pgfsetstrokecolor{currentstroke}%
\pgfsetdash{}{0pt}%
\pgfusepath{stroke}%
\end{pgfscope}%
\begin{pgfscope}%
\pgfpathrectangle{\pgfqpoint{0.050000in}{0.050000in}}{\pgfqpoint{2.419000in}{2.419000in}}%
\pgfusepath{clip}%
\pgfsetbuttcap%
\pgfsetroundjoin%
\pgfsetlinewidth{1.003750pt}%
\definecolor{currentstroke}{rgb}{0.000000,0.000000,0.000000}%
\pgfsetstrokecolor{currentstroke}%
\pgfsetdash{}{0pt}%
\pgfusepath{stroke}%
\end{pgfscope}%
\begin{pgfscope}%
\pgfpathrectangle{\pgfqpoint{0.050000in}{0.050000in}}{\pgfqpoint{2.419000in}{2.419000in}}%
\pgfusepath{clip}%
\pgfsetbuttcap%
\pgfsetroundjoin%
\pgfsetlinewidth{1.003750pt}%
\definecolor{currentstroke}{rgb}{0.000000,0.000000,0.000000}%
\pgfsetstrokecolor{currentstroke}%
\pgfsetdash{}{0pt}%
\pgfpathmoveto{\pgfqpoint{0.043119in}{2.059399in}}%
\pgfpathlineto{\pgfqpoint{0.227905in}{1.807617in}}%
\pgfusepath{stroke}%
\end{pgfscope}%
\begin{pgfscope}%
\pgfpathrectangle{\pgfqpoint{0.050000in}{0.050000in}}{\pgfqpoint{2.419000in}{2.419000in}}%
\pgfusepath{clip}%
\pgfsetbuttcap%
\pgfsetroundjoin%
\pgfsetlinewidth{1.003750pt}%
\definecolor{currentstroke}{rgb}{0.000000,0.000000,0.000000}%
\pgfsetstrokecolor{currentstroke}%
\pgfsetdash{}{0pt}%
\pgfpathmoveto{\pgfqpoint{0.227905in}{1.807617in}}%
\pgfpathlineto{\pgfqpoint{0.831359in}{1.844725in}}%
\pgfusepath{stroke}%
\end{pgfscope}%
\begin{pgfscope}%
\pgfpathrectangle{\pgfqpoint{0.050000in}{0.050000in}}{\pgfqpoint{2.419000in}{2.419000in}}%
\pgfusepath{clip}%
\pgfsetbuttcap%
\pgfsetroundjoin%
\pgfsetlinewidth{1.003750pt}%
\definecolor{currentstroke}{rgb}{0.000000,0.000000,0.000000}%
\pgfsetstrokecolor{currentstroke}%
\pgfsetdash{}{0pt}%
\pgfusepath{stroke}%
\end{pgfscope}%
\begin{pgfscope}%
\pgfpathrectangle{\pgfqpoint{0.050000in}{0.050000in}}{\pgfqpoint{2.419000in}{2.419000in}}%
\pgfusepath{clip}%
\pgfsetbuttcap%
\pgfsetroundjoin%
\pgfsetlinewidth{1.003750pt}%
\definecolor{currentstroke}{rgb}{0.000000,0.000000,0.000000}%
\pgfsetstrokecolor{currentstroke}%
\pgfsetdash{}{0pt}%
\pgfusepath{stroke}%
\end{pgfscope}%
\begin{pgfscope}%
\pgfpathrectangle{\pgfqpoint{0.050000in}{0.050000in}}{\pgfqpoint{2.419000in}{2.419000in}}%
\pgfusepath{clip}%
\pgfsetbuttcap%
\pgfsetroundjoin%
\pgfsetlinewidth{1.003750pt}%
\definecolor{currentstroke}{rgb}{0.000000,0.000000,0.000000}%
\pgfsetstrokecolor{currentstroke}%
\pgfsetdash{}{0pt}%
\pgfusepath{stroke}%
\end{pgfscope}%
\begin{pgfscope}%
\pgfpathrectangle{\pgfqpoint{0.050000in}{0.050000in}}{\pgfqpoint{2.419000in}{2.419000in}}%
\pgfusepath{clip}%
\pgfsetbuttcap%
\pgfsetroundjoin%
\pgfsetlinewidth{1.003750pt}%
\definecolor{currentstroke}{rgb}{0.000000,0.000000,0.000000}%
\pgfsetstrokecolor{currentstroke}%
\pgfsetdash{}{0pt}%
\pgfusepath{stroke}%
\end{pgfscope}%
\begin{pgfscope}%
\pgfpathrectangle{\pgfqpoint{0.050000in}{0.050000in}}{\pgfqpoint{2.419000in}{2.419000in}}%
\pgfusepath{clip}%
\pgfsetbuttcap%
\pgfsetroundjoin%
\pgfsetlinewidth{1.003750pt}%
\definecolor{currentstroke}{rgb}{0.000000,0.000000,0.000000}%
\pgfsetstrokecolor{currentstroke}%
\pgfsetdash{}{0pt}%
\pgfpathmoveto{\pgfqpoint{0.831359in}{1.844725in}}%
\pgfpathlineto{\pgfqpoint{0.826857in}{2.071381in}}%
\pgfusepath{stroke}%
\end{pgfscope}%
\begin{pgfscope}%
\pgfpathrectangle{\pgfqpoint{0.050000in}{0.050000in}}{\pgfqpoint{2.419000in}{2.419000in}}%
\pgfusepath{clip}%
\pgfsetbuttcap%
\pgfsetroundjoin%
\pgfsetlinewidth{1.003750pt}%
\definecolor{currentstroke}{rgb}{0.000000,0.000000,0.000000}%
\pgfsetstrokecolor{currentstroke}%
\pgfsetdash{}{0pt}%
\pgfusepath{stroke}%
\end{pgfscope}%
\begin{pgfscope}%
\pgfpathrectangle{\pgfqpoint{0.050000in}{0.050000in}}{\pgfqpoint{2.419000in}{2.419000in}}%
\pgfusepath{clip}%
\pgfsetbuttcap%
\pgfsetroundjoin%
\pgfsetlinewidth{1.003750pt}%
\definecolor{currentstroke}{rgb}{0.000000,0.000000,0.000000}%
\pgfsetstrokecolor{currentstroke}%
\pgfsetdash{}{0pt}%
\pgfusepath{stroke}%
\end{pgfscope}%
\begin{pgfscope}%
\pgfpathrectangle{\pgfqpoint{0.050000in}{0.050000in}}{\pgfqpoint{2.419000in}{2.419000in}}%
\pgfusepath{clip}%
\pgfsetbuttcap%
\pgfsetroundjoin%
\pgfsetlinewidth{1.003750pt}%
\definecolor{currentstroke}{rgb}{0.000000,0.000000,0.000000}%
\pgfsetstrokecolor{currentstroke}%
\pgfsetdash{}{0pt}%
\pgfusepath{stroke}%
\end{pgfscope}%
\begin{pgfscope}%
\pgfpathrectangle{\pgfqpoint{0.050000in}{0.050000in}}{\pgfqpoint{2.419000in}{2.419000in}}%
\pgfusepath{clip}%
\pgfsetbuttcap%
\pgfsetroundjoin%
\pgfsetlinewidth{1.003750pt}%
\definecolor{currentstroke}{rgb}{0.000000,0.000000,0.000000}%
\pgfsetstrokecolor{currentstroke}%
\pgfsetdash{}{0pt}%
\pgfusepath{stroke}%
\end{pgfscope}%
\begin{pgfscope}%
\pgfpathrectangle{\pgfqpoint{0.050000in}{0.050000in}}{\pgfqpoint{2.419000in}{2.419000in}}%
\pgfusepath{clip}%
\pgfsetbuttcap%
\pgfsetroundjoin%
\pgfsetlinewidth{1.003750pt}%
\definecolor{currentstroke}{rgb}{0.000000,0.000000,0.000000}%
\pgfsetstrokecolor{currentstroke}%
\pgfsetdash{}{0pt}%
\pgfusepath{stroke}%
\end{pgfscope}%
\begin{pgfscope}%
\pgfpathrectangle{\pgfqpoint{0.050000in}{0.050000in}}{\pgfqpoint{2.419000in}{2.419000in}}%
\pgfusepath{clip}%
\pgfsetbuttcap%
\pgfsetroundjoin%
\pgfsetlinewidth{1.003750pt}%
\definecolor{currentstroke}{rgb}{0.000000,0.000000,0.000000}%
\pgfsetstrokecolor{currentstroke}%
\pgfsetdash{}{0pt}%
\pgfusepath{stroke}%
\end{pgfscope}%
\begin{pgfscope}%
\pgfpathrectangle{\pgfqpoint{0.050000in}{0.050000in}}{\pgfqpoint{2.419000in}{2.419000in}}%
\pgfusepath{clip}%
\pgfsetbuttcap%
\pgfsetroundjoin%
\pgfsetlinewidth{1.003750pt}%
\definecolor{currentstroke}{rgb}{0.000000,0.000000,0.000000}%
\pgfsetstrokecolor{currentstroke}%
\pgfsetdash{}{0pt}%
\pgfusepath{stroke}%
\end{pgfscope}%
\begin{pgfscope}%
\pgfpathrectangle{\pgfqpoint{0.050000in}{0.050000in}}{\pgfqpoint{2.419000in}{2.419000in}}%
\pgfusepath{clip}%
\pgfsetbuttcap%
\pgfsetroundjoin%
\pgfsetlinewidth{1.003750pt}%
\definecolor{currentstroke}{rgb}{0.000000,0.000000,0.000000}%
\pgfsetstrokecolor{currentstroke}%
\pgfsetdash{}{0pt}%
\pgfusepath{stroke}%
\end{pgfscope}%
\begin{pgfscope}%
\pgfpathrectangle{\pgfqpoint{0.050000in}{0.050000in}}{\pgfqpoint{2.419000in}{2.419000in}}%
\pgfusepath{clip}%
\pgfsetbuttcap%
\pgfsetroundjoin%
\pgfsetlinewidth{1.003750pt}%
\definecolor{currentstroke}{rgb}{0.000000,0.000000,0.000000}%
\pgfsetstrokecolor{currentstroke}%
\pgfsetdash{}{0pt}%
\pgfusepath{stroke}%
\end{pgfscope}%
\begin{pgfscope}%
\pgfpathrectangle{\pgfqpoint{0.050000in}{0.050000in}}{\pgfqpoint{2.419000in}{2.419000in}}%
\pgfusepath{clip}%
\pgfsetbuttcap%
\pgfsetroundjoin%
\pgfsetlinewidth{1.003750pt}%
\definecolor{currentstroke}{rgb}{0.000000,0.000000,0.000000}%
\pgfsetstrokecolor{currentstroke}%
\pgfsetdash{}{0pt}%
\pgfusepath{stroke}%
\end{pgfscope}%
\begin{pgfscope}%
\pgfpathrectangle{\pgfqpoint{0.050000in}{0.050000in}}{\pgfqpoint{2.419000in}{2.419000in}}%
\pgfusepath{clip}%
\pgfsetbuttcap%
\pgfsetroundjoin%
\pgfsetlinewidth{1.003750pt}%
\definecolor{currentstroke}{rgb}{0.000000,0.000000,0.000000}%
\pgfsetstrokecolor{currentstroke}%
\pgfsetdash{}{0pt}%
\pgfusepath{stroke}%
\end{pgfscope}%
\begin{pgfscope}%
\pgfpathrectangle{\pgfqpoint{0.050000in}{0.050000in}}{\pgfqpoint{2.419000in}{2.419000in}}%
\pgfusepath{clip}%
\pgfsetbuttcap%
\pgfsetroundjoin%
\pgfsetlinewidth{1.003750pt}%
\definecolor{currentstroke}{rgb}{0.000000,0.000000,0.000000}%
\pgfsetstrokecolor{currentstroke}%
\pgfsetdash{}{0pt}%
\pgfusepath{stroke}%
\end{pgfscope}%
\begin{pgfscope}%
\pgfpathrectangle{\pgfqpoint{0.050000in}{0.050000in}}{\pgfqpoint{2.419000in}{2.419000in}}%
\pgfusepath{clip}%
\pgfsetbuttcap%
\pgfsetroundjoin%
\pgfsetlinewidth{1.003750pt}%
\definecolor{currentstroke}{rgb}{0.000000,0.000000,0.000000}%
\pgfsetstrokecolor{currentstroke}%
\pgfsetdash{}{0pt}%
\pgfusepath{stroke}%
\end{pgfscope}%
\begin{pgfscope}%
\pgfpathrectangle{\pgfqpoint{0.050000in}{0.050000in}}{\pgfqpoint{2.419000in}{2.419000in}}%
\pgfusepath{clip}%
\pgfsetbuttcap%
\pgfsetroundjoin%
\pgfsetlinewidth{1.003750pt}%
\definecolor{currentstroke}{rgb}{0.000000,0.000000,0.000000}%
\pgfsetstrokecolor{currentstroke}%
\pgfsetdash{}{0pt}%
\pgfusepath{stroke}%
\end{pgfscope}%
\begin{pgfscope}%
\pgfpathrectangle{\pgfqpoint{0.050000in}{0.050000in}}{\pgfqpoint{2.419000in}{2.419000in}}%
\pgfusepath{clip}%
\pgfsetbuttcap%
\pgfsetroundjoin%
\pgfsetlinewidth{1.003750pt}%
\definecolor{currentstroke}{rgb}{0.000000,0.000000,0.000000}%
\pgfsetstrokecolor{currentstroke}%
\pgfsetdash{}{0pt}%
\pgfusepath{stroke}%
\end{pgfscope}%
\begin{pgfscope}%
\pgfpathrectangle{\pgfqpoint{0.050000in}{0.050000in}}{\pgfqpoint{2.419000in}{2.419000in}}%
\pgfusepath{clip}%
\pgfsetbuttcap%
\pgfsetroundjoin%
\pgfsetlinewidth{1.003750pt}%
\definecolor{currentstroke}{rgb}{0.000000,0.000000,0.000000}%
\pgfsetstrokecolor{currentstroke}%
\pgfsetdash{}{0pt}%
\pgfpathmoveto{\pgfqpoint{1.671131in}{1.616016in}}%
\pgfpathlineto{\pgfqpoint{2.019364in}{1.917777in}}%
\pgfusepath{stroke}%
\end{pgfscope}%
\begin{pgfscope}%
\pgfpathrectangle{\pgfqpoint{0.050000in}{0.050000in}}{\pgfqpoint{2.419000in}{2.419000in}}%
\pgfusepath{clip}%
\pgfsetbuttcap%
\pgfsetroundjoin%
\pgfsetlinewidth{1.003750pt}%
\definecolor{currentstroke}{rgb}{0.000000,0.000000,0.000000}%
\pgfsetstrokecolor{currentstroke}%
\pgfsetdash{}{0pt}%
\pgfusepath{stroke}%
\end{pgfscope}%
\begin{pgfscope}%
\pgfpathrectangle{\pgfqpoint{0.050000in}{0.050000in}}{\pgfqpoint{2.419000in}{2.419000in}}%
\pgfusepath{clip}%
\pgfsetbuttcap%
\pgfsetroundjoin%
\pgfsetlinewidth{1.003750pt}%
\definecolor{currentstroke}{rgb}{0.000000,0.000000,0.000000}%
\pgfsetstrokecolor{currentstroke}%
\pgfsetdash{}{0pt}%
\pgfpathmoveto{\pgfqpoint{0.831359in}{1.844725in}}%
\pgfpathlineto{\pgfqpoint{1.052443in}{1.576436in}}%
\pgfusepath{stroke}%
\end{pgfscope}%
\begin{pgfscope}%
\pgfpathrectangle{\pgfqpoint{0.050000in}{0.050000in}}{\pgfqpoint{2.419000in}{2.419000in}}%
\pgfusepath{clip}%
\pgfsetbuttcap%
\pgfsetroundjoin%
\pgfsetlinewidth{1.003750pt}%
\definecolor{currentstroke}{rgb}{0.000000,0.000000,0.000000}%
\pgfsetstrokecolor{currentstroke}%
\pgfsetdash{}{0pt}%
\pgfpathmoveto{\pgfqpoint{1.052443in}{1.576436in}}%
\pgfpathlineto{\pgfqpoint{1.671131in}{1.616016in}}%
\pgfusepath{stroke}%
\end{pgfscope}%
\begin{pgfscope}%
\pgfpathrectangle{\pgfqpoint{0.050000in}{0.050000in}}{\pgfqpoint{2.419000in}{2.419000in}}%
\pgfusepath{clip}%
\pgfsetbuttcap%
\pgfsetroundjoin%
\pgfsetlinewidth{1.003750pt}%
\definecolor{currentstroke}{rgb}{0.000000,0.000000,0.000000}%
\pgfsetstrokecolor{currentstroke}%
\pgfsetdash{}{0pt}%
\pgfusepath{stroke}%
\end{pgfscope}%
\begin{pgfscope}%
\pgfpathrectangle{\pgfqpoint{0.050000in}{0.050000in}}{\pgfqpoint{2.419000in}{2.419000in}}%
\pgfusepath{clip}%
\pgfsetbuttcap%
\pgfsetroundjoin%
\pgfsetlinewidth{1.003750pt}%
\definecolor{currentstroke}{rgb}{0.000000,0.000000,0.000000}%
\pgfsetstrokecolor{currentstroke}%
\pgfsetdash{}{0pt}%
\pgfusepath{stroke}%
\end{pgfscope}%
\begin{pgfscope}%
\pgfpathrectangle{\pgfqpoint{0.050000in}{0.050000in}}{\pgfqpoint{2.419000in}{2.419000in}}%
\pgfusepath{clip}%
\pgfsetbuttcap%
\pgfsetroundjoin%
\pgfsetlinewidth{1.003750pt}%
\definecolor{currentstroke}{rgb}{0.000000,0.000000,0.000000}%
\pgfsetstrokecolor{currentstroke}%
\pgfsetdash{}{0pt}%
\pgfusepath{stroke}%
\end{pgfscope}%
\begin{pgfscope}%
\pgfpathrectangle{\pgfqpoint{0.050000in}{0.050000in}}{\pgfqpoint{2.419000in}{2.419000in}}%
\pgfusepath{clip}%
\pgfsetbuttcap%
\pgfsetroundjoin%
\pgfsetlinewidth{1.003750pt}%
\definecolor{currentstroke}{rgb}{0.000000,0.000000,0.000000}%
\pgfsetstrokecolor{currentstroke}%
\pgfsetdash{}{0pt}%
\pgfusepath{stroke}%
\end{pgfscope}%
\begin{pgfscope}%
\pgfpathrectangle{\pgfqpoint{0.050000in}{0.050000in}}{\pgfqpoint{2.419000in}{2.419000in}}%
\pgfusepath{clip}%
\pgfsetbuttcap%
\pgfsetroundjoin%
\pgfsetlinewidth{1.003750pt}%
\definecolor{currentstroke}{rgb}{0.000000,0.000000,0.000000}%
\pgfsetstrokecolor{currentstroke}%
\pgfsetdash{}{0pt}%
\pgfpathmoveto{\pgfqpoint{1.671131in}{1.616016in}}%
\pgfpathlineto{\pgfqpoint{1.674955in}{1.847848in}}%
\pgfusepath{stroke}%
\end{pgfscope}%
\begin{pgfscope}%
\pgfpathrectangle{\pgfqpoint{0.050000in}{0.050000in}}{\pgfqpoint{2.419000in}{2.419000in}}%
\pgfusepath{clip}%
\pgfsetbuttcap%
\pgfsetroundjoin%
\pgfsetlinewidth{1.003750pt}%
\definecolor{currentstroke}{rgb}{0.000000,0.000000,0.000000}%
\pgfsetstrokecolor{currentstroke}%
\pgfsetdash{}{0pt}%
\pgfusepath{stroke}%
\end{pgfscope}%
\begin{pgfscope}%
\pgfpathrectangle{\pgfqpoint{0.050000in}{0.050000in}}{\pgfqpoint{2.419000in}{2.419000in}}%
\pgfusepath{clip}%
\pgfsetbuttcap%
\pgfsetroundjoin%
\pgfsetlinewidth{1.003750pt}%
\definecolor{currentstroke}{rgb}{0.000000,0.000000,0.000000}%
\pgfsetstrokecolor{currentstroke}%
\pgfsetdash{}{0pt}%
\pgfpathmoveto{\pgfqpoint{0.040000in}{1.672494in}}%
\pgfpathlineto{\pgfqpoint{0.227905in}{1.807617in}}%
\pgfusepath{stroke}%
\end{pgfscope}%
\begin{pgfscope}%
\pgfpathrectangle{\pgfqpoint{0.050000in}{0.050000in}}{\pgfqpoint{2.419000in}{2.419000in}}%
\pgfusepath{clip}%
\pgfsetbuttcap%
\pgfsetroundjoin%
\pgfsetlinewidth{1.003750pt}%
\definecolor{currentstroke}{rgb}{0.000000,0.000000,0.000000}%
\pgfsetstrokecolor{currentstroke}%
\pgfsetdash{}{0pt}%
\pgfusepath{stroke}%
\end{pgfscope}%
\begin{pgfscope}%
\pgfpathrectangle{\pgfqpoint{0.050000in}{0.050000in}}{\pgfqpoint{2.419000in}{2.419000in}}%
\pgfusepath{clip}%
\pgfsetbuttcap%
\pgfsetroundjoin%
\pgfsetlinewidth{1.003750pt}%
\definecolor{currentstroke}{rgb}{0.000000,0.000000,0.000000}%
\pgfsetstrokecolor{currentstroke}%
\pgfsetdash{}{0pt}%
\pgfusepath{stroke}%
\end{pgfscope}%
\begin{pgfscope}%
\pgfpathrectangle{\pgfqpoint{0.050000in}{0.050000in}}{\pgfqpoint{2.419000in}{2.419000in}}%
\pgfusepath{clip}%
\pgfsetbuttcap%
\pgfsetroundjoin%
\pgfsetlinewidth{1.003750pt}%
\definecolor{currentstroke}{rgb}{0.000000,0.000000,0.000000}%
\pgfsetstrokecolor{currentstroke}%
\pgfsetdash{}{0pt}%
\pgfusepath{stroke}%
\end{pgfscope}%
\begin{pgfscope}%
\pgfpathrectangle{\pgfqpoint{0.050000in}{0.050000in}}{\pgfqpoint{2.419000in}{2.419000in}}%
\pgfusepath{clip}%
\pgfsetbuttcap%
\pgfsetroundjoin%
\pgfsetlinewidth{1.003750pt}%
\definecolor{currentstroke}{rgb}{0.000000,0.000000,0.000000}%
\pgfsetstrokecolor{currentstroke}%
\pgfsetdash{}{0pt}%
\pgfusepath{stroke}%
\end{pgfscope}%
\begin{pgfscope}%
\pgfpathrectangle{\pgfqpoint{0.050000in}{0.050000in}}{\pgfqpoint{2.419000in}{2.419000in}}%
\pgfusepath{clip}%
\pgfsetbuttcap%
\pgfsetroundjoin%
\pgfsetlinewidth{1.003750pt}%
\definecolor{currentstroke}{rgb}{0.000000,0.000000,0.000000}%
\pgfsetstrokecolor{currentstroke}%
\pgfsetdash{}{0pt}%
\pgfusepath{stroke}%
\end{pgfscope}%
\begin{pgfscope}%
\pgfpathrectangle{\pgfqpoint{0.050000in}{0.050000in}}{\pgfqpoint{2.419000in}{2.419000in}}%
\pgfusepath{clip}%
\pgfsetbuttcap%
\pgfsetroundjoin%
\pgfsetlinewidth{1.003750pt}%
\definecolor{currentstroke}{rgb}{0.000000,0.000000,0.000000}%
\pgfsetstrokecolor{currentstroke}%
\pgfsetdash{}{0pt}%
\pgfusepath{stroke}%
\end{pgfscope}%
\begin{pgfscope}%
\pgfpathrectangle{\pgfqpoint{0.050000in}{0.050000in}}{\pgfqpoint{2.419000in}{2.419000in}}%
\pgfusepath{clip}%
\pgfsetbuttcap%
\pgfsetroundjoin%
\pgfsetlinewidth{1.003750pt}%
\definecolor{currentstroke}{rgb}{0.000000,0.000000,0.000000}%
\pgfsetstrokecolor{currentstroke}%
\pgfsetdash{}{0pt}%
\pgfusepath{stroke}%
\end{pgfscope}%
\begin{pgfscope}%
\pgfpathrectangle{\pgfqpoint{0.050000in}{0.050000in}}{\pgfqpoint{2.419000in}{2.419000in}}%
\pgfusepath{clip}%
\pgfsetbuttcap%
\pgfsetroundjoin%
\pgfsetlinewidth{1.003750pt}%
\definecolor{currentstroke}{rgb}{0.000000,0.000000,0.000000}%
\pgfsetstrokecolor{currentstroke}%
\pgfsetdash{}{0pt}%
\pgfusepath{stroke}%
\end{pgfscope}%
\begin{pgfscope}%
\pgfpathrectangle{\pgfqpoint{0.050000in}{0.050000in}}{\pgfqpoint{2.419000in}{2.419000in}}%
\pgfusepath{clip}%
\pgfsetbuttcap%
\pgfsetroundjoin%
\pgfsetlinewidth{1.003750pt}%
\definecolor{currentstroke}{rgb}{0.000000,0.000000,0.000000}%
\pgfsetstrokecolor{currentstroke}%
\pgfsetdash{}{0pt}%
\pgfusepath{stroke}%
\end{pgfscope}%
\begin{pgfscope}%
\pgfpathrectangle{\pgfqpoint{0.050000in}{0.050000in}}{\pgfqpoint{2.419000in}{2.419000in}}%
\pgfusepath{clip}%
\pgfsetbuttcap%
\pgfsetroundjoin%
\pgfsetlinewidth{1.003750pt}%
\definecolor{currentstroke}{rgb}{0.000000,0.000000,0.000000}%
\pgfsetstrokecolor{currentstroke}%
\pgfsetdash{}{0pt}%
\pgfusepath{stroke}%
\end{pgfscope}%
\begin{pgfscope}%
\pgfpathrectangle{\pgfqpoint{0.050000in}{0.050000in}}{\pgfqpoint{2.419000in}{2.419000in}}%
\pgfusepath{clip}%
\pgfsetbuttcap%
\pgfsetroundjoin%
\pgfsetlinewidth{1.003750pt}%
\definecolor{currentstroke}{rgb}{0.000000,0.000000,0.000000}%
\pgfsetstrokecolor{currentstroke}%
\pgfsetdash{}{0pt}%
\pgfusepath{stroke}%
\end{pgfscope}%
\begin{pgfscope}%
\pgfpathrectangle{\pgfqpoint{0.050000in}{0.050000in}}{\pgfqpoint{2.419000in}{2.419000in}}%
\pgfusepath{clip}%
\pgfsetbuttcap%
\pgfsetroundjoin%
\pgfsetlinewidth{1.003750pt}%
\definecolor{currentstroke}{rgb}{0.000000,0.000000,0.000000}%
\pgfsetstrokecolor{currentstroke}%
\pgfsetdash{}{0pt}%
\pgfusepath{stroke}%
\end{pgfscope}%
\begin{pgfscope}%
\pgfpathrectangle{\pgfqpoint{0.050000in}{0.050000in}}{\pgfqpoint{2.419000in}{2.419000in}}%
\pgfusepath{clip}%
\pgfsetbuttcap%
\pgfsetroundjoin%
\pgfsetlinewidth{1.003750pt}%
\definecolor{currentstroke}{rgb}{0.000000,0.000000,0.000000}%
\pgfsetstrokecolor{currentstroke}%
\pgfsetdash{}{0pt}%
\pgfusepath{stroke}%
\end{pgfscope}%
\begin{pgfscope}%
\pgfpathrectangle{\pgfqpoint{0.050000in}{0.050000in}}{\pgfqpoint{2.419000in}{2.419000in}}%
\pgfusepath{clip}%
\pgfsetbuttcap%
\pgfsetroundjoin%
\pgfsetlinewidth{1.003750pt}%
\definecolor{currentstroke}{rgb}{0.000000,0.000000,0.000000}%
\pgfsetstrokecolor{currentstroke}%
\pgfsetdash{}{0pt}%
\pgfusepath{stroke}%
\end{pgfscope}%
\begin{pgfscope}%
\pgfpathrectangle{\pgfqpoint{0.050000in}{0.050000in}}{\pgfqpoint{2.419000in}{2.419000in}}%
\pgfusepath{clip}%
\pgfsetbuttcap%
\pgfsetroundjoin%
\pgfsetlinewidth{1.003750pt}%
\definecolor{currentstroke}{rgb}{0.000000,0.000000,0.000000}%
\pgfsetstrokecolor{currentstroke}%
\pgfsetdash{}{0pt}%
\pgfusepath{stroke}%
\end{pgfscope}%
\begin{pgfscope}%
\pgfpathrectangle{\pgfqpoint{0.050000in}{0.050000in}}{\pgfqpoint{2.419000in}{2.419000in}}%
\pgfusepath{clip}%
\pgfsetbuttcap%
\pgfsetroundjoin%
\pgfsetlinewidth{1.003750pt}%
\definecolor{currentstroke}{rgb}{0.000000,0.000000,0.000000}%
\pgfsetstrokecolor{currentstroke}%
\pgfsetdash{}{0pt}%
\pgfusepath{stroke}%
\end{pgfscope}%
\begin{pgfscope}%
\pgfpathrectangle{\pgfqpoint{0.050000in}{0.050000in}}{\pgfqpoint{2.419000in}{2.419000in}}%
\pgfusepath{clip}%
\pgfsetbuttcap%
\pgfsetroundjoin%
\pgfsetlinewidth{1.003750pt}%
\definecolor{currentstroke}{rgb}{0.000000,0.000000,0.000000}%
\pgfsetstrokecolor{currentstroke}%
\pgfsetdash{}{0pt}%
\pgfpathmoveto{\pgfqpoint{1.671131in}{1.616016in}}%
\pgfpathlineto{\pgfqpoint{1.933024in}{1.329543in}}%
\pgfusepath{stroke}%
\end{pgfscope}%
\begin{pgfscope}%
\pgfpathrectangle{\pgfqpoint{0.050000in}{0.050000in}}{\pgfqpoint{2.419000in}{2.419000in}}%
\pgfusepath{clip}%
\pgfsetbuttcap%
\pgfsetroundjoin%
\pgfsetlinewidth{1.003750pt}%
\definecolor{currentstroke}{rgb}{0.000000,0.000000,0.000000}%
\pgfsetstrokecolor{currentstroke}%
\pgfsetdash{}{0pt}%
\pgfpathmoveto{\pgfqpoint{1.933024in}{1.329543in}}%
\pgfpathlineto{\pgfqpoint{2.479000in}{1.365940in}}%
\pgfusepath{stroke}%
\end{pgfscope}%
\begin{pgfscope}%
\pgfpathrectangle{\pgfqpoint{0.050000in}{0.050000in}}{\pgfqpoint{2.419000in}{2.419000in}}%
\pgfusepath{clip}%
\pgfsetbuttcap%
\pgfsetroundjoin%
\pgfsetlinewidth{1.003750pt}%
\definecolor{currentstroke}{rgb}{0.000000,0.000000,0.000000}%
\pgfsetstrokecolor{currentstroke}%
\pgfsetdash{}{0pt}%
\pgfusepath{stroke}%
\end{pgfscope}%
\begin{pgfscope}%
\pgfpathrectangle{\pgfqpoint{0.050000in}{0.050000in}}{\pgfqpoint{2.419000in}{2.419000in}}%
\pgfusepath{clip}%
\pgfsetbuttcap%
\pgfsetroundjoin%
\pgfsetlinewidth{1.003750pt}%
\definecolor{currentstroke}{rgb}{0.000000,0.000000,0.000000}%
\pgfsetstrokecolor{currentstroke}%
\pgfsetdash{}{0pt}%
\pgfusepath{stroke}%
\end{pgfscope}%
\begin{pgfscope}%
\pgfpathrectangle{\pgfqpoint{0.050000in}{0.050000in}}{\pgfqpoint{2.419000in}{2.419000in}}%
\pgfusepath{clip}%
\pgfsetbuttcap%
\pgfsetroundjoin%
\pgfsetlinewidth{1.003750pt}%
\definecolor{currentstroke}{rgb}{0.000000,0.000000,0.000000}%
\pgfsetstrokecolor{currentstroke}%
\pgfsetdash{}{0pt}%
\pgfusepath{stroke}%
\end{pgfscope}%
\begin{pgfscope}%
\pgfpathrectangle{\pgfqpoint{0.050000in}{0.050000in}}{\pgfqpoint{2.419000in}{2.419000in}}%
\pgfusepath{clip}%
\pgfsetbuttcap%
\pgfsetroundjoin%
\pgfsetlinewidth{1.003750pt}%
\definecolor{currentstroke}{rgb}{0.000000,0.000000,0.000000}%
\pgfsetstrokecolor{currentstroke}%
\pgfsetdash{}{0pt}%
\pgfusepath{stroke}%
\end{pgfscope}%
\begin{pgfscope}%
\pgfpathrectangle{\pgfqpoint{0.050000in}{0.050000in}}{\pgfqpoint{2.419000in}{2.419000in}}%
\pgfusepath{clip}%
\pgfsetbuttcap%
\pgfsetroundjoin%
\pgfsetlinewidth{1.003750pt}%
\definecolor{currentstroke}{rgb}{0.000000,0.000000,0.000000}%
\pgfsetstrokecolor{currentstroke}%
\pgfsetdash{}{0pt}%
\pgfusepath{stroke}%
\end{pgfscope}%
\begin{pgfscope}%
\pgfpathrectangle{\pgfqpoint{0.050000in}{0.050000in}}{\pgfqpoint{2.419000in}{2.419000in}}%
\pgfusepath{clip}%
\pgfsetbuttcap%
\pgfsetroundjoin%
\pgfsetlinewidth{1.003750pt}%
\definecolor{currentstroke}{rgb}{0.000000,0.000000,0.000000}%
\pgfsetstrokecolor{currentstroke}%
\pgfsetdash{}{0pt}%
\pgfpathmoveto{\pgfqpoint{0.641940in}{1.243473in}}%
\pgfpathlineto{\pgfqpoint{1.052443in}{1.576436in}}%
\pgfusepath{stroke}%
\end{pgfscope}%
\begin{pgfscope}%
\pgfpathrectangle{\pgfqpoint{0.050000in}{0.050000in}}{\pgfqpoint{2.419000in}{2.419000in}}%
\pgfusepath{clip}%
\pgfsetbuttcap%
\pgfsetroundjoin%
\pgfsetlinewidth{1.003750pt}%
\definecolor{currentstroke}{rgb}{0.000000,0.000000,0.000000}%
\pgfsetstrokecolor{currentstroke}%
\pgfsetdash{}{0pt}%
\pgfusepath{stroke}%
\end{pgfscope}%
\begin{pgfscope}%
\pgfpathrectangle{\pgfqpoint{0.050000in}{0.050000in}}{\pgfqpoint{2.419000in}{2.419000in}}%
\pgfusepath{clip}%
\pgfsetbuttcap%
\pgfsetroundjoin%
\pgfsetlinewidth{1.003750pt}%
\definecolor{currentstroke}{rgb}{0.000000,0.000000,0.000000}%
\pgfsetstrokecolor{currentstroke}%
\pgfsetdash{}{0pt}%
\pgfusepath{stroke}%
\end{pgfscope}%
\begin{pgfscope}%
\pgfpathrectangle{\pgfqpoint{0.050000in}{0.050000in}}{\pgfqpoint{2.419000in}{2.419000in}}%
\pgfusepath{clip}%
\pgfsetbuttcap%
\pgfsetroundjoin%
\pgfsetlinewidth{1.003750pt}%
\definecolor{currentstroke}{rgb}{0.000000,0.000000,0.000000}%
\pgfsetstrokecolor{currentstroke}%
\pgfsetdash{}{0pt}%
\pgfusepath{stroke}%
\end{pgfscope}%
\begin{pgfscope}%
\pgfpathrectangle{\pgfqpoint{0.050000in}{0.050000in}}{\pgfqpoint{2.419000in}{2.419000in}}%
\pgfusepath{clip}%
\pgfsetbuttcap%
\pgfsetroundjoin%
\pgfsetlinewidth{1.003750pt}%
\definecolor{currentstroke}{rgb}{0.000000,0.000000,0.000000}%
\pgfsetstrokecolor{currentstroke}%
\pgfsetdash{}{0pt}%
\pgfpathmoveto{\pgfqpoint{0.040000in}{1.203345in}}%
\pgfpathlineto{\pgfqpoint{0.641940in}{1.243473in}}%
\pgfusepath{stroke}%
\end{pgfscope}%
\begin{pgfscope}%
\pgfpathrectangle{\pgfqpoint{0.050000in}{0.050000in}}{\pgfqpoint{2.419000in}{2.419000in}}%
\pgfusepath{clip}%
\pgfsetbuttcap%
\pgfsetroundjoin%
\pgfsetlinewidth{1.003750pt}%
\definecolor{currentstroke}{rgb}{0.000000,0.000000,0.000000}%
\pgfsetstrokecolor{currentstroke}%
\pgfsetdash{}{0pt}%
\pgfusepath{stroke}%
\end{pgfscope}%
\begin{pgfscope}%
\pgfpathrectangle{\pgfqpoint{0.050000in}{0.050000in}}{\pgfqpoint{2.419000in}{2.419000in}}%
\pgfusepath{clip}%
\pgfsetbuttcap%
\pgfsetroundjoin%
\pgfsetlinewidth{1.003750pt}%
\definecolor{currentstroke}{rgb}{0.000000,0.000000,0.000000}%
\pgfsetstrokecolor{currentstroke}%
\pgfsetdash{}{0pt}%
\pgfusepath{stroke}%
\end{pgfscope}%
\begin{pgfscope}%
\pgfpathrectangle{\pgfqpoint{0.050000in}{0.050000in}}{\pgfqpoint{2.419000in}{2.419000in}}%
\pgfusepath{clip}%
\pgfsetbuttcap%
\pgfsetroundjoin%
\pgfsetlinewidth{1.003750pt}%
\definecolor{currentstroke}{rgb}{0.000000,0.000000,0.000000}%
\pgfsetstrokecolor{currentstroke}%
\pgfsetdash{}{0pt}%
\pgfusepath{stroke}%
\end{pgfscope}%
\begin{pgfscope}%
\pgfpathrectangle{\pgfqpoint{0.050000in}{0.050000in}}{\pgfqpoint{2.419000in}{2.419000in}}%
\pgfusepath{clip}%
\pgfsetbuttcap%
\pgfsetroundjoin%
\pgfsetlinewidth{1.003750pt}%
\definecolor{currentstroke}{rgb}{0.000000,0.000000,0.000000}%
\pgfsetstrokecolor{currentstroke}%
\pgfsetdash{}{0pt}%
\pgfusepath{stroke}%
\end{pgfscope}%
\begin{pgfscope}%
\pgfpathrectangle{\pgfqpoint{0.050000in}{0.050000in}}{\pgfqpoint{2.419000in}{2.419000in}}%
\pgfusepath{clip}%
\pgfsetbuttcap%
\pgfsetroundjoin%
\pgfsetlinewidth{1.003750pt}%
\definecolor{currentstroke}{rgb}{0.000000,0.000000,0.000000}%
\pgfsetstrokecolor{currentstroke}%
\pgfsetdash{}{0pt}%
\pgfpathmoveto{\pgfqpoint{0.641940in}{1.243473in}}%
\pgfpathlineto{\pgfqpoint{0.635035in}{1.483429in}}%
\pgfusepath{stroke}%
\end{pgfscope}%
\begin{pgfscope}%
\pgfpathrectangle{\pgfqpoint{0.050000in}{0.050000in}}{\pgfqpoint{2.419000in}{2.419000in}}%
\pgfusepath{clip}%
\pgfsetbuttcap%
\pgfsetroundjoin%
\pgfsetlinewidth{1.003750pt}%
\definecolor{currentstroke}{rgb}{0.000000,0.000000,0.000000}%
\pgfsetstrokecolor{currentstroke}%
\pgfsetdash{}{0pt}%
\pgfusepath{stroke}%
\end{pgfscope}%
\begin{pgfscope}%
\pgfpathrectangle{\pgfqpoint{0.050000in}{0.050000in}}{\pgfqpoint{2.419000in}{2.419000in}}%
\pgfusepath{clip}%
\pgfsetbuttcap%
\pgfsetroundjoin%
\pgfsetlinewidth{1.003750pt}%
\definecolor{currentstroke}{rgb}{0.000000,0.000000,0.000000}%
\pgfsetstrokecolor{currentstroke}%
\pgfsetdash{}{0pt}%
\pgfusepath{stroke}%
\end{pgfscope}%
\begin{pgfscope}%
\pgfpathrectangle{\pgfqpoint{0.050000in}{0.050000in}}{\pgfqpoint{2.419000in}{2.419000in}}%
\pgfusepath{clip}%
\pgfsetbuttcap%
\pgfsetroundjoin%
\pgfsetlinewidth{1.003750pt}%
\definecolor{currentstroke}{rgb}{0.000000,0.000000,0.000000}%
\pgfsetstrokecolor{currentstroke}%
\pgfsetdash{}{0pt}%
\pgfusepath{stroke}%
\end{pgfscope}%
\begin{pgfscope}%
\pgfpathrectangle{\pgfqpoint{0.050000in}{0.050000in}}{\pgfqpoint{2.419000in}{2.419000in}}%
\pgfusepath{clip}%
\pgfsetbuttcap%
\pgfsetroundjoin%
\pgfsetlinewidth{1.003750pt}%
\definecolor{currentstroke}{rgb}{0.000000,0.000000,0.000000}%
\pgfsetstrokecolor{currentstroke}%
\pgfsetdash{}{0pt}%
\pgfusepath{stroke}%
\end{pgfscope}%
\begin{pgfscope}%
\pgfpathrectangle{\pgfqpoint{0.050000in}{0.050000in}}{\pgfqpoint{2.419000in}{2.419000in}}%
\pgfusepath{clip}%
\pgfsetbuttcap%
\pgfsetroundjoin%
\pgfsetlinewidth{1.003750pt}%
\definecolor{currentstroke}{rgb}{0.000000,0.000000,0.000000}%
\pgfsetstrokecolor{currentstroke}%
\pgfsetdash{}{0pt}%
\pgfusepath{stroke}%
\end{pgfscope}%
\begin{pgfscope}%
\pgfpathrectangle{\pgfqpoint{0.050000in}{0.050000in}}{\pgfqpoint{2.419000in}{2.419000in}}%
\pgfusepath{clip}%
\pgfsetbuttcap%
\pgfsetroundjoin%
\pgfsetlinewidth{1.003750pt}%
\definecolor{currentstroke}{rgb}{0.000000,0.000000,0.000000}%
\pgfsetstrokecolor{currentstroke}%
\pgfsetdash{}{0pt}%
\pgfusepath{stroke}%
\end{pgfscope}%
\begin{pgfscope}%
\pgfpathrectangle{\pgfqpoint{0.050000in}{0.050000in}}{\pgfqpoint{2.419000in}{2.419000in}}%
\pgfusepath{clip}%
\pgfsetbuttcap%
\pgfsetroundjoin%
\pgfsetlinewidth{1.003750pt}%
\definecolor{currentstroke}{rgb}{0.000000,0.000000,0.000000}%
\pgfsetstrokecolor{currentstroke}%
\pgfsetdash{}{0pt}%
\pgfusepath{stroke}%
\end{pgfscope}%
\begin{pgfscope}%
\pgfpathrectangle{\pgfqpoint{0.050000in}{0.050000in}}{\pgfqpoint{2.419000in}{2.419000in}}%
\pgfusepath{clip}%
\pgfsetbuttcap%
\pgfsetroundjoin%
\pgfsetlinewidth{1.003750pt}%
\definecolor{currentstroke}{rgb}{0.000000,0.000000,0.000000}%
\pgfsetstrokecolor{currentstroke}%
\pgfsetdash{}{0pt}%
\pgfusepath{stroke}%
\end{pgfscope}%
\begin{pgfscope}%
\pgfpathrectangle{\pgfqpoint{0.050000in}{0.050000in}}{\pgfqpoint{2.419000in}{2.419000in}}%
\pgfusepath{clip}%
\pgfsetbuttcap%
\pgfsetroundjoin%
\pgfsetlinewidth{1.003750pt}%
\definecolor{currentstroke}{rgb}{0.000000,0.000000,0.000000}%
\pgfsetstrokecolor{currentstroke}%
\pgfsetdash{}{0pt}%
\pgfusepath{stroke}%
\end{pgfscope}%
\begin{pgfscope}%
\pgfpathrectangle{\pgfqpoint{0.050000in}{0.050000in}}{\pgfqpoint{2.419000in}{2.419000in}}%
\pgfusepath{clip}%
\pgfsetbuttcap%
\pgfsetroundjoin%
\pgfsetlinewidth{1.003750pt}%
\definecolor{currentstroke}{rgb}{0.000000,0.000000,0.000000}%
\pgfsetstrokecolor{currentstroke}%
\pgfsetdash{}{0pt}%
\pgfusepath{stroke}%
\end{pgfscope}%
\begin{pgfscope}%
\pgfpathrectangle{\pgfqpoint{0.050000in}{0.050000in}}{\pgfqpoint{2.419000in}{2.419000in}}%
\pgfusepath{clip}%
\pgfsetbuttcap%
\pgfsetroundjoin%
\pgfsetlinewidth{1.003750pt}%
\definecolor{currentstroke}{rgb}{0.000000,0.000000,0.000000}%
\pgfsetstrokecolor{currentstroke}%
\pgfsetdash{}{0pt}%
\pgfusepath{stroke}%
\end{pgfscope}%
\begin{pgfscope}%
\pgfpathrectangle{\pgfqpoint{0.050000in}{0.050000in}}{\pgfqpoint{2.419000in}{2.419000in}}%
\pgfusepath{clip}%
\pgfsetbuttcap%
\pgfsetroundjoin%
\pgfsetlinewidth{1.003750pt}%
\definecolor{currentstroke}{rgb}{0.000000,0.000000,0.000000}%
\pgfsetstrokecolor{currentstroke}%
\pgfsetdash{}{0pt}%
\pgfusepath{stroke}%
\end{pgfscope}%
\begin{pgfscope}%
\pgfpathrectangle{\pgfqpoint{0.050000in}{0.050000in}}{\pgfqpoint{2.419000in}{2.419000in}}%
\pgfusepath{clip}%
\pgfsetbuttcap%
\pgfsetroundjoin%
\pgfsetlinewidth{1.003750pt}%
\definecolor{currentstroke}{rgb}{0.000000,0.000000,0.000000}%
\pgfsetstrokecolor{currentstroke}%
\pgfsetdash{}{0pt}%
\pgfusepath{stroke}%
\end{pgfscope}%
\begin{pgfscope}%
\pgfpathrectangle{\pgfqpoint{0.050000in}{0.050000in}}{\pgfqpoint{2.419000in}{2.419000in}}%
\pgfusepath{clip}%
\pgfsetbuttcap%
\pgfsetroundjoin%
\pgfsetlinewidth{1.003750pt}%
\definecolor{currentstroke}{rgb}{0.000000,0.000000,0.000000}%
\pgfsetstrokecolor{currentstroke}%
\pgfsetdash{}{0pt}%
\pgfusepath{stroke}%
\end{pgfscope}%
\begin{pgfscope}%
\pgfpathrectangle{\pgfqpoint{0.050000in}{0.050000in}}{\pgfqpoint{2.419000in}{2.419000in}}%
\pgfusepath{clip}%
\pgfsetbuttcap%
\pgfsetroundjoin%
\pgfsetlinewidth{1.003750pt}%
\definecolor{currentstroke}{rgb}{0.000000,0.000000,0.000000}%
\pgfsetstrokecolor{currentstroke}%
\pgfsetdash{}{0pt}%
\pgfusepath{stroke}%
\end{pgfscope}%
\begin{pgfscope}%
\pgfpathrectangle{\pgfqpoint{0.050000in}{0.050000in}}{\pgfqpoint{2.419000in}{2.419000in}}%
\pgfusepath{clip}%
\pgfsetbuttcap%
\pgfsetroundjoin%
\pgfsetlinewidth{1.003750pt}%
\definecolor{currentstroke}{rgb}{0.000000,0.000000,0.000000}%
\pgfsetstrokecolor{currentstroke}%
\pgfsetdash{}{0pt}%
\pgfpathmoveto{\pgfqpoint{1.549702in}{0.973008in}}%
\pgfpathlineto{\pgfqpoint{1.933024in}{1.329543in}}%
\pgfusepath{stroke}%
\end{pgfscope}%
\begin{pgfscope}%
\pgfpathrectangle{\pgfqpoint{0.050000in}{0.050000in}}{\pgfqpoint{2.419000in}{2.419000in}}%
\pgfusepath{clip}%
\pgfsetbuttcap%
\pgfsetroundjoin%
\pgfsetlinewidth{1.003750pt}%
\definecolor{currentstroke}{rgb}{0.000000,0.000000,0.000000}%
\pgfsetstrokecolor{currentstroke}%
\pgfsetdash{}{0pt}%
\pgfusepath{stroke}%
\end{pgfscope}%
\begin{pgfscope}%
\pgfpathrectangle{\pgfqpoint{0.050000in}{0.050000in}}{\pgfqpoint{2.419000in}{2.419000in}}%
\pgfusepath{clip}%
\pgfsetbuttcap%
\pgfsetroundjoin%
\pgfsetlinewidth{1.003750pt}%
\definecolor{currentstroke}{rgb}{0.000000,0.000000,0.000000}%
\pgfsetstrokecolor{currentstroke}%
\pgfsetdash{}{0pt}%
\pgfusepath{stroke}%
\end{pgfscope}%
\begin{pgfscope}%
\pgfpathrectangle{\pgfqpoint{0.050000in}{0.050000in}}{\pgfqpoint{2.419000in}{2.419000in}}%
\pgfusepath{clip}%
\pgfsetbuttcap%
\pgfsetroundjoin%
\pgfsetlinewidth{1.003750pt}%
\definecolor{currentstroke}{rgb}{0.000000,0.000000,0.000000}%
\pgfsetstrokecolor{currentstroke}%
\pgfsetdash{}{0pt}%
\pgfpathmoveto{\pgfqpoint{0.641940in}{1.243473in}}%
\pgfpathlineto{\pgfqpoint{0.874903in}{0.926048in}}%
\pgfusepath{stroke}%
\end{pgfscope}%
\begin{pgfscope}%
\pgfpathrectangle{\pgfqpoint{0.050000in}{0.050000in}}{\pgfqpoint{2.419000in}{2.419000in}}%
\pgfusepath{clip}%
\pgfsetbuttcap%
\pgfsetroundjoin%
\pgfsetlinewidth{1.003750pt}%
\definecolor{currentstroke}{rgb}{0.000000,0.000000,0.000000}%
\pgfsetstrokecolor{currentstroke}%
\pgfsetdash{}{0pt}%
\pgfpathmoveto{\pgfqpoint{0.874903in}{0.926048in}}%
\pgfpathlineto{\pgfqpoint{1.549702in}{0.973008in}}%
\pgfusepath{stroke}%
\end{pgfscope}%
\begin{pgfscope}%
\pgfpathrectangle{\pgfqpoint{0.050000in}{0.050000in}}{\pgfqpoint{2.419000in}{2.419000in}}%
\pgfusepath{clip}%
\pgfsetbuttcap%
\pgfsetroundjoin%
\pgfsetlinewidth{1.003750pt}%
\definecolor{currentstroke}{rgb}{0.000000,0.000000,0.000000}%
\pgfsetstrokecolor{currentstroke}%
\pgfsetdash{}{0pt}%
\pgfusepath{stroke}%
\end{pgfscope}%
\begin{pgfscope}%
\pgfpathrectangle{\pgfqpoint{0.050000in}{0.050000in}}{\pgfqpoint{2.419000in}{2.419000in}}%
\pgfusepath{clip}%
\pgfsetbuttcap%
\pgfsetroundjoin%
\pgfsetlinewidth{1.003750pt}%
\definecolor{currentstroke}{rgb}{0.000000,0.000000,0.000000}%
\pgfsetstrokecolor{currentstroke}%
\pgfsetdash{}{0pt}%
\pgfusepath{stroke}%
\end{pgfscope}%
\begin{pgfscope}%
\pgfpathrectangle{\pgfqpoint{0.050000in}{0.050000in}}{\pgfqpoint{2.419000in}{2.419000in}}%
\pgfusepath{clip}%
\pgfsetbuttcap%
\pgfsetroundjoin%
\pgfsetlinewidth{1.003750pt}%
\definecolor{currentstroke}{rgb}{0.000000,0.000000,0.000000}%
\pgfsetstrokecolor{currentstroke}%
\pgfsetdash{}{0pt}%
\pgfusepath{stroke}%
\end{pgfscope}%
\begin{pgfscope}%
\pgfpathrectangle{\pgfqpoint{0.050000in}{0.050000in}}{\pgfqpoint{2.419000in}{2.419000in}}%
\pgfusepath{clip}%
\pgfsetbuttcap%
\pgfsetroundjoin%
\pgfsetlinewidth{1.003750pt}%
\definecolor{currentstroke}{rgb}{0.000000,0.000000,0.000000}%
\pgfsetstrokecolor{currentstroke}%
\pgfsetdash{}{0pt}%
\pgfpathmoveto{\pgfqpoint{1.549702in}{0.973008in}}%
\pgfpathlineto{\pgfqpoint{1.552535in}{1.218625in}}%
\pgfusepath{stroke}%
\end{pgfscope}%
\begin{pgfscope}%
\pgfpathrectangle{\pgfqpoint{0.050000in}{0.050000in}}{\pgfqpoint{2.419000in}{2.419000in}}%
\pgfusepath{clip}%
\pgfsetbuttcap%
\pgfsetroundjoin%
\pgfsetlinewidth{1.003750pt}%
\definecolor{currentstroke}{rgb}{0.000000,0.000000,0.000000}%
\pgfsetstrokecolor{currentstroke}%
\pgfsetdash{}{0pt}%
\pgfusepath{stroke}%
\end{pgfscope}%
\begin{pgfscope}%
\pgfpathrectangle{\pgfqpoint{0.050000in}{0.050000in}}{\pgfqpoint{2.419000in}{2.419000in}}%
\pgfusepath{clip}%
\pgfsetbuttcap%
\pgfsetroundjoin%
\pgfsetlinewidth{1.003750pt}%
\definecolor{currentstroke}{rgb}{0.000000,0.000000,0.000000}%
\pgfsetstrokecolor{currentstroke}%
\pgfsetdash{}{0pt}%
\pgfusepath{stroke}%
\end{pgfscope}%
\begin{pgfscope}%
\pgfpathrectangle{\pgfqpoint{0.050000in}{0.050000in}}{\pgfqpoint{2.419000in}{2.419000in}}%
\pgfusepath{clip}%
\pgfsetbuttcap%
\pgfsetroundjoin%
\pgfsetlinewidth{1.003750pt}%
\definecolor{currentstroke}{rgb}{0.000000,0.000000,0.000000}%
\pgfsetstrokecolor{currentstroke}%
\pgfsetdash{}{0pt}%
\pgfusepath{stroke}%
\end{pgfscope}%
\begin{pgfscope}%
\pgfpathrectangle{\pgfqpoint{0.050000in}{0.050000in}}{\pgfqpoint{2.419000in}{2.419000in}}%
\pgfusepath{clip}%
\pgfsetbuttcap%
\pgfsetroundjoin%
\pgfsetlinewidth{1.003750pt}%
\definecolor{currentstroke}{rgb}{0.000000,0.000000,0.000000}%
\pgfsetstrokecolor{currentstroke}%
\pgfsetdash{}{0pt}%
\pgfusepath{stroke}%
\end{pgfscope}%
\begin{pgfscope}%
\pgfpathrectangle{\pgfqpoint{0.050000in}{0.050000in}}{\pgfqpoint{2.419000in}{2.419000in}}%
\pgfusepath{clip}%
\pgfsetbuttcap%
\pgfsetroundjoin%
\pgfsetlinewidth{1.003750pt}%
\definecolor{currentstroke}{rgb}{0.000000,0.000000,0.000000}%
\pgfsetstrokecolor{currentstroke}%
\pgfsetdash{}{0pt}%
\pgfusepath{stroke}%
\end{pgfscope}%
\begin{pgfscope}%
\pgfpathrectangle{\pgfqpoint{0.050000in}{0.050000in}}{\pgfqpoint{2.419000in}{2.419000in}}%
\pgfusepath{clip}%
\pgfsetbuttcap%
\pgfsetroundjoin%
\pgfsetlinewidth{1.003750pt}%
\definecolor{currentstroke}{rgb}{0.000000,0.000000,0.000000}%
\pgfsetstrokecolor{currentstroke}%
\pgfsetdash{}{0pt}%
\pgfusepath{stroke}%
\end{pgfscope}%
\begin{pgfscope}%
\pgfpathrectangle{\pgfqpoint{0.050000in}{0.050000in}}{\pgfqpoint{2.419000in}{2.419000in}}%
\pgfusepath{clip}%
\pgfsetbuttcap%
\pgfsetroundjoin%
\pgfsetlinewidth{1.003750pt}%
\definecolor{currentstroke}{rgb}{0.000000,0.000000,0.000000}%
\pgfsetstrokecolor{currentstroke}%
\pgfsetdash{}{0pt}%
\pgfusepath{stroke}%
\end{pgfscope}%
\begin{pgfscope}%
\pgfpathrectangle{\pgfqpoint{0.050000in}{0.050000in}}{\pgfqpoint{2.419000in}{2.419000in}}%
\pgfusepath{clip}%
\pgfsetbuttcap%
\pgfsetroundjoin%
\pgfsetlinewidth{1.003750pt}%
\definecolor{currentstroke}{rgb}{0.000000,0.000000,0.000000}%
\pgfsetstrokecolor{currentstroke}%
\pgfsetdash{}{0pt}%
\pgfusepath{stroke}%
\end{pgfscope}%
\begin{pgfscope}%
\pgfpathrectangle{\pgfqpoint{0.050000in}{0.050000in}}{\pgfqpoint{2.419000in}{2.419000in}}%
\pgfusepath{clip}%
\pgfsetbuttcap%
\pgfsetroundjoin%
\pgfsetlinewidth{1.003750pt}%
\definecolor{currentstroke}{rgb}{0.000000,0.000000,0.000000}%
\pgfsetstrokecolor{currentstroke}%
\pgfsetdash{}{0pt}%
\pgfusepath{stroke}%
\end{pgfscope}%
\begin{pgfscope}%
\pgfpathrectangle{\pgfqpoint{0.050000in}{0.050000in}}{\pgfqpoint{2.419000in}{2.419000in}}%
\pgfusepath{clip}%
\pgfsetbuttcap%
\pgfsetroundjoin%
\pgfsetlinewidth{1.003750pt}%
\definecolor{currentstroke}{rgb}{0.000000,0.000000,0.000000}%
\pgfsetstrokecolor{currentstroke}%
\pgfsetdash{}{0pt}%
\pgfusepath{stroke}%
\end{pgfscope}%
\begin{pgfscope}%
\pgfpathrectangle{\pgfqpoint{0.050000in}{0.050000in}}{\pgfqpoint{2.419000in}{2.419000in}}%
\pgfusepath{clip}%
\pgfsetbuttcap%
\pgfsetroundjoin%
\pgfsetlinewidth{1.003750pt}%
\definecolor{currentstroke}{rgb}{0.000000,0.000000,0.000000}%
\pgfsetstrokecolor{currentstroke}%
\pgfsetdash{}{0pt}%
\pgfusepath{stroke}%
\end{pgfscope}%
\begin{pgfscope}%
\pgfpathrectangle{\pgfqpoint{0.050000in}{0.050000in}}{\pgfqpoint{2.419000in}{2.419000in}}%
\pgfusepath{clip}%
\pgfsetbuttcap%
\pgfsetroundjoin%
\pgfsetlinewidth{1.003750pt}%
\definecolor{currentstroke}{rgb}{0.000000,0.000000,0.000000}%
\pgfsetstrokecolor{currentstroke}%
\pgfsetdash{}{0pt}%
\pgfusepath{stroke}%
\end{pgfscope}%
\begin{pgfscope}%
\pgfpathrectangle{\pgfqpoint{0.050000in}{0.050000in}}{\pgfqpoint{2.419000in}{2.419000in}}%
\pgfusepath{clip}%
\pgfsetbuttcap%
\pgfsetroundjoin%
\pgfsetlinewidth{1.003750pt}%
\definecolor{currentstroke}{rgb}{0.000000,0.000000,0.000000}%
\pgfsetstrokecolor{currentstroke}%
\pgfsetdash{}{0pt}%
\pgfusepath{stroke}%
\end{pgfscope}%
\begin{pgfscope}%
\pgfpathrectangle{\pgfqpoint{0.050000in}{0.050000in}}{\pgfqpoint{2.419000in}{2.419000in}}%
\pgfusepath{clip}%
\pgfsetbuttcap%
\pgfsetroundjoin%
\pgfsetlinewidth{1.003750pt}%
\definecolor{currentstroke}{rgb}{0.000000,0.000000,0.000000}%
\pgfsetstrokecolor{currentstroke}%
\pgfsetdash{}{0pt}%
\pgfusepath{stroke}%
\end{pgfscope}%
\begin{pgfscope}%
\pgfpathrectangle{\pgfqpoint{0.050000in}{0.050000in}}{\pgfqpoint{2.419000in}{2.419000in}}%
\pgfusepath{clip}%
\pgfsetbuttcap%
\pgfsetroundjoin%
\pgfsetlinewidth{1.003750pt}%
\definecolor{currentstroke}{rgb}{0.000000,0.000000,0.000000}%
\pgfsetstrokecolor{currentstroke}%
\pgfsetdash{}{0pt}%
\pgfusepath{stroke}%
\end{pgfscope}%
\begin{pgfscope}%
\pgfpathrectangle{\pgfqpoint{0.050000in}{0.050000in}}{\pgfqpoint{2.419000in}{2.419000in}}%
\pgfusepath{clip}%
\pgfsetbuttcap%
\pgfsetroundjoin%
\pgfsetlinewidth{1.003750pt}%
\definecolor{currentstroke}{rgb}{0.000000,0.000000,0.000000}%
\pgfsetstrokecolor{currentstroke}%
\pgfsetdash{}{0pt}%
\pgfusepath{stroke}%
\end{pgfscope}%
\begin{pgfscope}%
\pgfpathrectangle{\pgfqpoint{0.050000in}{0.050000in}}{\pgfqpoint{2.419000in}{2.419000in}}%
\pgfusepath{clip}%
\pgfsetbuttcap%
\pgfsetroundjoin%
\pgfsetlinewidth{1.003750pt}%
\definecolor{currentstroke}{rgb}{0.000000,0.000000,0.000000}%
\pgfsetstrokecolor{currentstroke}%
\pgfsetdash{}{0pt}%
\pgfusepath{stroke}%
\end{pgfscope}%
\begin{pgfscope}%
\pgfpathrectangle{\pgfqpoint{0.050000in}{0.050000in}}{\pgfqpoint{2.419000in}{2.419000in}}%
\pgfusepath{clip}%
\pgfsetbuttcap%
\pgfsetroundjoin%
\pgfsetlinewidth{1.003750pt}%
\definecolor{currentstroke}{rgb}{0.000000,0.000000,0.000000}%
\pgfsetstrokecolor{currentstroke}%
\pgfsetdash{}{0pt}%
\pgfusepath{stroke}%
\end{pgfscope}%
\begin{pgfscope}%
\pgfpathrectangle{\pgfqpoint{0.050000in}{0.050000in}}{\pgfqpoint{2.419000in}{2.419000in}}%
\pgfusepath{clip}%
\pgfsetbuttcap%
\pgfsetroundjoin%
\pgfsetlinewidth{1.003750pt}%
\definecolor{currentstroke}{rgb}{0.000000,0.000000,0.000000}%
\pgfsetstrokecolor{currentstroke}%
\pgfsetdash{}{0pt}%
\pgfpathmoveto{\pgfqpoint{1.549702in}{0.973008in}}%
\pgfpathlineto{\pgfqpoint{1.830650in}{0.632075in}}%
\pgfusepath{stroke}%
\end{pgfscope}%
\begin{pgfscope}%
\pgfpathrectangle{\pgfqpoint{0.050000in}{0.050000in}}{\pgfqpoint{2.419000in}{2.419000in}}%
\pgfusepath{clip}%
\pgfsetbuttcap%
\pgfsetroundjoin%
\pgfsetlinewidth{1.003750pt}%
\definecolor{currentstroke}{rgb}{0.000000,0.000000,0.000000}%
\pgfsetstrokecolor{currentstroke}%
\pgfsetdash{}{0pt}%
\pgfpathmoveto{\pgfqpoint{1.830650in}{0.632075in}}%
\pgfpathlineto{\pgfqpoint{2.479000in}{0.679267in}}%
\pgfusepath{stroke}%
\end{pgfscope}%
\begin{pgfscope}%
\pgfpathrectangle{\pgfqpoint{0.050000in}{0.050000in}}{\pgfqpoint{2.419000in}{2.419000in}}%
\pgfusepath{clip}%
\pgfsetbuttcap%
\pgfsetroundjoin%
\pgfsetlinewidth{1.003750pt}%
\definecolor{currentstroke}{rgb}{0.000000,0.000000,0.000000}%
\pgfsetstrokecolor{currentstroke}%
\pgfsetdash{}{0pt}%
\pgfusepath{stroke}%
\end{pgfscope}%
\begin{pgfscope}%
\pgfpathrectangle{\pgfqpoint{0.050000in}{0.050000in}}{\pgfqpoint{2.419000in}{2.419000in}}%
\pgfusepath{clip}%
\pgfsetbuttcap%
\pgfsetroundjoin%
\pgfsetlinewidth{1.003750pt}%
\definecolor{currentstroke}{rgb}{0.000000,0.000000,0.000000}%
\pgfsetstrokecolor{currentstroke}%
\pgfsetdash{}{0pt}%
\pgfusepath{stroke}%
\end{pgfscope}%
\begin{pgfscope}%
\pgfpathrectangle{\pgfqpoint{0.050000in}{0.050000in}}{\pgfqpoint{2.419000in}{2.419000in}}%
\pgfusepath{clip}%
\pgfsetbuttcap%
\pgfsetroundjoin%
\pgfsetlinewidth{1.003750pt}%
\definecolor{currentstroke}{rgb}{0.000000,0.000000,0.000000}%
\pgfsetstrokecolor{currentstroke}%
\pgfsetdash{}{0pt}%
\pgfusepath{stroke}%
\end{pgfscope}%
\begin{pgfscope}%
\pgfpathrectangle{\pgfqpoint{0.050000in}{0.050000in}}{\pgfqpoint{2.419000in}{2.419000in}}%
\pgfusepath{clip}%
\pgfsetbuttcap%
\pgfsetroundjoin%
\pgfsetlinewidth{1.003750pt}%
\definecolor{currentstroke}{rgb}{0.000000,0.000000,0.000000}%
\pgfsetstrokecolor{currentstroke}%
\pgfsetdash{}{0pt}%
\pgfusepath{stroke}%
\end{pgfscope}%
\begin{pgfscope}%
\pgfpathrectangle{\pgfqpoint{0.050000in}{0.050000in}}{\pgfqpoint{2.419000in}{2.419000in}}%
\pgfusepath{clip}%
\pgfsetbuttcap%
\pgfsetroundjoin%
\pgfsetlinewidth{1.003750pt}%
\definecolor{currentstroke}{rgb}{0.000000,0.000000,0.000000}%
\pgfsetstrokecolor{currentstroke}%
\pgfsetdash{}{0pt}%
\pgfusepath{stroke}%
\end{pgfscope}%
\begin{pgfscope}%
\pgfpathrectangle{\pgfqpoint{0.050000in}{0.050000in}}{\pgfqpoint{2.419000in}{2.419000in}}%
\pgfusepath{clip}%
\pgfsetbuttcap%
\pgfsetroundjoin%
\pgfsetlinewidth{1.003750pt}%
\definecolor{currentstroke}{rgb}{0.000000,0.000000,0.000000}%
\pgfsetstrokecolor{currentstroke}%
\pgfsetdash{}{0pt}%
\pgfusepath{stroke}%
\end{pgfscope}%
\begin{pgfscope}%
\pgfpathrectangle{\pgfqpoint{0.050000in}{0.050000in}}{\pgfqpoint{2.419000in}{2.419000in}}%
\pgfusepath{clip}%
\pgfsetbuttcap%
\pgfsetroundjoin%
\pgfsetlinewidth{1.003750pt}%
\definecolor{currentstroke}{rgb}{0.000000,0.000000,0.000000}%
\pgfsetstrokecolor{currentstroke}%
\pgfsetdash{}{0pt}%
\pgfpathmoveto{\pgfqpoint{0.416905in}{0.529171in}}%
\pgfpathlineto{\pgfqpoint{0.874903in}{0.926048in}}%
\pgfusepath{stroke}%
\end{pgfscope}%
\begin{pgfscope}%
\pgfpathrectangle{\pgfqpoint{0.050000in}{0.050000in}}{\pgfqpoint{2.419000in}{2.419000in}}%
\pgfusepath{clip}%
\pgfsetbuttcap%
\pgfsetroundjoin%
\pgfsetlinewidth{1.003750pt}%
\definecolor{currentstroke}{rgb}{0.000000,0.000000,0.000000}%
\pgfsetstrokecolor{currentstroke}%
\pgfsetdash{}{0pt}%
\pgfusepath{stroke}%
\end{pgfscope}%
\begin{pgfscope}%
\pgfpathrectangle{\pgfqpoint{0.050000in}{0.050000in}}{\pgfqpoint{2.419000in}{2.419000in}}%
\pgfusepath{clip}%
\pgfsetbuttcap%
\pgfsetroundjoin%
\pgfsetlinewidth{1.003750pt}%
\definecolor{currentstroke}{rgb}{0.000000,0.000000,0.000000}%
\pgfsetstrokecolor{currentstroke}%
\pgfsetdash{}{0pt}%
\pgfusepath{stroke}%
\end{pgfscope}%
\begin{pgfscope}%
\pgfpathrectangle{\pgfqpoint{0.050000in}{0.050000in}}{\pgfqpoint{2.419000in}{2.419000in}}%
\pgfusepath{clip}%
\pgfsetbuttcap%
\pgfsetroundjoin%
\pgfsetlinewidth{1.003750pt}%
\definecolor{currentstroke}{rgb}{0.000000,0.000000,0.000000}%
\pgfsetstrokecolor{currentstroke}%
\pgfsetdash{}{0pt}%
\pgfpathmoveto{\pgfqpoint{0.040000in}{0.501737in}}%
\pgfpathlineto{\pgfqpoint{0.416905in}{0.529171in}}%
\pgfusepath{stroke}%
\end{pgfscope}%
\begin{pgfscope}%
\pgfpathrectangle{\pgfqpoint{0.050000in}{0.050000in}}{\pgfqpoint{2.419000in}{2.419000in}}%
\pgfusepath{clip}%
\pgfsetbuttcap%
\pgfsetroundjoin%
\pgfsetlinewidth{1.003750pt}%
\definecolor{currentstroke}{rgb}{0.000000,0.000000,0.000000}%
\pgfsetstrokecolor{currentstroke}%
\pgfsetdash{}{0pt}%
\pgfusepath{stroke}%
\end{pgfscope}%
\begin{pgfscope}%
\pgfpathrectangle{\pgfqpoint{0.050000in}{0.050000in}}{\pgfqpoint{2.419000in}{2.419000in}}%
\pgfusepath{clip}%
\pgfsetbuttcap%
\pgfsetroundjoin%
\pgfsetlinewidth{1.003750pt}%
\definecolor{currentstroke}{rgb}{0.000000,0.000000,0.000000}%
\pgfsetstrokecolor{currentstroke}%
\pgfsetdash{}{0pt}%
\pgfusepath{stroke}%
\end{pgfscope}%
\begin{pgfscope}%
\pgfpathrectangle{\pgfqpoint{0.050000in}{0.050000in}}{\pgfqpoint{2.419000in}{2.419000in}}%
\pgfusepath{clip}%
\pgfsetbuttcap%
\pgfsetroundjoin%
\pgfsetlinewidth{1.003750pt}%
\definecolor{currentstroke}{rgb}{0.000000,0.000000,0.000000}%
\pgfsetstrokecolor{currentstroke}%
\pgfsetdash{}{0pt}%
\pgfusepath{stroke}%
\end{pgfscope}%
\begin{pgfscope}%
\pgfpathrectangle{\pgfqpoint{0.050000in}{0.050000in}}{\pgfqpoint{2.419000in}{2.419000in}}%
\pgfusepath{clip}%
\pgfsetbuttcap%
\pgfsetroundjoin%
\pgfsetlinewidth{1.003750pt}%
\definecolor{currentstroke}{rgb}{0.000000,0.000000,0.000000}%
\pgfsetstrokecolor{currentstroke}%
\pgfsetdash{}{0pt}%
\pgfusepath{stroke}%
\end{pgfscope}%
\begin{pgfscope}%
\pgfpathrectangle{\pgfqpoint{0.050000in}{0.050000in}}{\pgfqpoint{2.419000in}{2.419000in}}%
\pgfusepath{clip}%
\pgfsetbuttcap%
\pgfsetroundjoin%
\pgfsetlinewidth{1.003750pt}%
\definecolor{currentstroke}{rgb}{0.000000,0.000000,0.000000}%
\pgfsetstrokecolor{currentstroke}%
\pgfsetdash{}{0pt}%
\pgfpathmoveto{\pgfqpoint{0.416905in}{0.529171in}}%
\pgfpathlineto{\pgfqpoint{0.406727in}{0.783643in}}%
\pgfusepath{stroke}%
\end{pgfscope}%
\begin{pgfscope}%
\pgfpathrectangle{\pgfqpoint{0.050000in}{0.050000in}}{\pgfqpoint{2.419000in}{2.419000in}}%
\pgfusepath{clip}%
\pgfsetbuttcap%
\pgfsetroundjoin%
\pgfsetlinewidth{1.003750pt}%
\definecolor{currentstroke}{rgb}{0.000000,0.000000,0.000000}%
\pgfsetstrokecolor{currentstroke}%
\pgfsetdash{}{0pt}%
\pgfusepath{stroke}%
\end{pgfscope}%
\begin{pgfscope}%
\pgfpathrectangle{\pgfqpoint{0.050000in}{0.050000in}}{\pgfqpoint{2.419000in}{2.419000in}}%
\pgfusepath{clip}%
\pgfsetbuttcap%
\pgfsetroundjoin%
\pgfsetlinewidth{1.003750pt}%
\definecolor{currentstroke}{rgb}{0.000000,0.000000,0.000000}%
\pgfsetstrokecolor{currentstroke}%
\pgfsetdash{}{0pt}%
\pgfusepath{stroke}%
\end{pgfscope}%
\begin{pgfscope}%
\pgfpathrectangle{\pgfqpoint{0.050000in}{0.050000in}}{\pgfqpoint{2.419000in}{2.419000in}}%
\pgfusepath{clip}%
\pgfsetbuttcap%
\pgfsetroundjoin%
\pgfsetlinewidth{1.003750pt}%
\definecolor{currentstroke}{rgb}{0.000000,0.000000,0.000000}%
\pgfsetstrokecolor{currentstroke}%
\pgfsetdash{}{0pt}%
\pgfusepath{stroke}%
\end{pgfscope}%
\begin{pgfscope}%
\pgfpathrectangle{\pgfqpoint{0.050000in}{0.050000in}}{\pgfqpoint{2.419000in}{2.419000in}}%
\pgfusepath{clip}%
\pgfsetbuttcap%
\pgfsetroundjoin%
\pgfsetlinewidth{1.003750pt}%
\definecolor{currentstroke}{rgb}{0.000000,0.000000,0.000000}%
\pgfsetstrokecolor{currentstroke}%
\pgfsetdash{}{0pt}%
\pgfusepath{stroke}%
\end{pgfscope}%
\begin{pgfscope}%
\pgfpathrectangle{\pgfqpoint{0.050000in}{0.050000in}}{\pgfqpoint{2.419000in}{2.419000in}}%
\pgfusepath{clip}%
\pgfsetbuttcap%
\pgfsetroundjoin%
\pgfsetlinewidth{1.003750pt}%
\definecolor{currentstroke}{rgb}{0.000000,0.000000,0.000000}%
\pgfsetstrokecolor{currentstroke}%
\pgfsetdash{}{0pt}%
\pgfusepath{stroke}%
\end{pgfscope}%
\begin{pgfscope}%
\pgfpathrectangle{\pgfqpoint{0.050000in}{0.050000in}}{\pgfqpoint{2.419000in}{2.419000in}}%
\pgfusepath{clip}%
\pgfsetbuttcap%
\pgfsetroundjoin%
\pgfsetlinewidth{1.003750pt}%
\definecolor{currentstroke}{rgb}{0.000000,0.000000,0.000000}%
\pgfsetstrokecolor{currentstroke}%
\pgfsetdash{}{0pt}%
\pgfusepath{stroke}%
\end{pgfscope}%
\begin{pgfscope}%
\pgfpathrectangle{\pgfqpoint{0.050000in}{0.050000in}}{\pgfqpoint{2.419000in}{2.419000in}}%
\pgfusepath{clip}%
\pgfsetbuttcap%
\pgfsetroundjoin%
\pgfsetlinewidth{1.003750pt}%
\definecolor{currentstroke}{rgb}{0.000000,0.000000,0.000000}%
\pgfsetstrokecolor{currentstroke}%
\pgfsetdash{}{0pt}%
\pgfusepath{stroke}%
\end{pgfscope}%
\begin{pgfscope}%
\pgfpathrectangle{\pgfqpoint{0.050000in}{0.050000in}}{\pgfqpoint{2.419000in}{2.419000in}}%
\pgfusepath{clip}%
\pgfsetbuttcap%
\pgfsetroundjoin%
\pgfsetlinewidth{1.003750pt}%
\definecolor{currentstroke}{rgb}{0.000000,0.000000,0.000000}%
\pgfsetstrokecolor{currentstroke}%
\pgfsetdash{}{0pt}%
\pgfusepath{stroke}%
\end{pgfscope}%
\begin{pgfscope}%
\pgfpathrectangle{\pgfqpoint{0.050000in}{0.050000in}}{\pgfqpoint{2.419000in}{2.419000in}}%
\pgfusepath{clip}%
\pgfsetbuttcap%
\pgfsetroundjoin%
\pgfsetlinewidth{1.003750pt}%
\definecolor{currentstroke}{rgb}{0.000000,0.000000,0.000000}%
\pgfsetstrokecolor{currentstroke}%
\pgfsetdash{}{0pt}%
\pgfusepath{stroke}%
\end{pgfscope}%
\begin{pgfscope}%
\pgfpathrectangle{\pgfqpoint{0.050000in}{0.050000in}}{\pgfqpoint{2.419000in}{2.419000in}}%
\pgfusepath{clip}%
\pgfsetbuttcap%
\pgfsetroundjoin%
\pgfsetlinewidth{1.003750pt}%
\definecolor{currentstroke}{rgb}{0.000000,0.000000,0.000000}%
\pgfsetstrokecolor{currentstroke}%
\pgfsetdash{}{0pt}%
\pgfusepath{stroke}%
\end{pgfscope}%
\begin{pgfscope}%
\pgfpathrectangle{\pgfqpoint{0.050000in}{0.050000in}}{\pgfqpoint{2.419000in}{2.419000in}}%
\pgfusepath{clip}%
\pgfsetbuttcap%
\pgfsetroundjoin%
\pgfsetlinewidth{1.003750pt}%
\definecolor{currentstroke}{rgb}{0.000000,0.000000,0.000000}%
\pgfsetstrokecolor{currentstroke}%
\pgfsetdash{}{0pt}%
\pgfusepath{stroke}%
\end{pgfscope}%
\begin{pgfscope}%
\pgfpathrectangle{\pgfqpoint{0.050000in}{0.050000in}}{\pgfqpoint{2.419000in}{2.419000in}}%
\pgfusepath{clip}%
\pgfsetbuttcap%
\pgfsetroundjoin%
\pgfsetlinewidth{1.003750pt}%
\definecolor{currentstroke}{rgb}{0.000000,0.000000,0.000000}%
\pgfsetstrokecolor{currentstroke}%
\pgfsetdash{}{0pt}%
\pgfusepath{stroke}%
\end{pgfscope}%
\begin{pgfscope}%
\pgfpathrectangle{\pgfqpoint{0.050000in}{0.050000in}}{\pgfqpoint{2.419000in}{2.419000in}}%
\pgfusepath{clip}%
\pgfsetbuttcap%
\pgfsetroundjoin%
\pgfsetlinewidth{1.003750pt}%
\definecolor{currentstroke}{rgb}{0.000000,0.000000,0.000000}%
\pgfsetstrokecolor{currentstroke}%
\pgfsetdash{}{0pt}%
\pgfusepath{stroke}%
\end{pgfscope}%
\begin{pgfscope}%
\pgfpathrectangle{\pgfqpoint{0.050000in}{0.050000in}}{\pgfqpoint{2.419000in}{2.419000in}}%
\pgfusepath{clip}%
\pgfsetbuttcap%
\pgfsetroundjoin%
\pgfsetlinewidth{1.003750pt}%
\definecolor{currentstroke}{rgb}{0.000000,0.000000,0.000000}%
\pgfsetstrokecolor{currentstroke}%
\pgfsetdash{}{0pt}%
\pgfusepath{stroke}%
\end{pgfscope}%
\begin{pgfscope}%
\pgfpathrectangle{\pgfqpoint{0.050000in}{0.050000in}}{\pgfqpoint{2.419000in}{2.419000in}}%
\pgfusepath{clip}%
\pgfsetbuttcap%
\pgfsetroundjoin%
\pgfsetlinewidth{1.003750pt}%
\definecolor{currentstroke}{rgb}{0.000000,0.000000,0.000000}%
\pgfsetstrokecolor{currentstroke}%
\pgfsetdash{}{0pt}%
\pgfpathmoveto{\pgfqpoint{1.404550in}{0.204375in}}%
\pgfpathlineto{\pgfqpoint{1.830650in}{0.632075in}}%
\pgfusepath{stroke}%
\end{pgfscope}%
\begin{pgfscope}%
\pgfpathrectangle{\pgfqpoint{0.050000in}{0.050000in}}{\pgfqpoint{2.419000in}{2.419000in}}%
\pgfusepath{clip}%
\pgfsetbuttcap%
\pgfsetroundjoin%
\pgfsetlinewidth{1.003750pt}%
\definecolor{currentstroke}{rgb}{0.000000,0.000000,0.000000}%
\pgfsetstrokecolor{currentstroke}%
\pgfsetdash{}{0pt}%
\pgfusepath{stroke}%
\end{pgfscope}%
\begin{pgfscope}%
\pgfpathrectangle{\pgfqpoint{0.050000in}{0.050000in}}{\pgfqpoint{2.419000in}{2.419000in}}%
\pgfusepath{clip}%
\pgfsetbuttcap%
\pgfsetroundjoin%
\pgfsetlinewidth{1.003750pt}%
\definecolor{currentstroke}{rgb}{0.000000,0.000000,0.000000}%
\pgfsetstrokecolor{currentstroke}%
\pgfsetdash{}{0pt}%
\pgfusepath{stroke}%
\end{pgfscope}%
\begin{pgfscope}%
\pgfpathrectangle{\pgfqpoint{0.050000in}{0.050000in}}{\pgfqpoint{2.419000in}{2.419000in}}%
\pgfusepath{clip}%
\pgfsetbuttcap%
\pgfsetroundjoin%
\pgfsetlinewidth{1.003750pt}%
\definecolor{currentstroke}{rgb}{0.000000,0.000000,0.000000}%
\pgfsetstrokecolor{currentstroke}%
\pgfsetdash{}{0pt}%
\pgfusepath{stroke}%
\end{pgfscope}%
\begin{pgfscope}%
\pgfpathrectangle{\pgfqpoint{0.050000in}{0.050000in}}{\pgfqpoint{2.419000in}{2.419000in}}%
\pgfusepath{clip}%
\pgfsetbuttcap%
\pgfsetroundjoin%
\pgfsetlinewidth{1.003750pt}%
\definecolor{currentstroke}{rgb}{0.000000,0.000000,0.000000}%
\pgfsetstrokecolor{currentstroke}%
\pgfsetdash{}{0pt}%
\pgfpathmoveto{\pgfqpoint{0.416905in}{0.529171in}}%
\pgfpathlineto{\pgfqpoint{0.662449in}{0.147759in}}%
\pgfusepath{stroke}%
\end{pgfscope}%
\begin{pgfscope}%
\pgfpathrectangle{\pgfqpoint{0.050000in}{0.050000in}}{\pgfqpoint{2.419000in}{2.419000in}}%
\pgfusepath{clip}%
\pgfsetbuttcap%
\pgfsetroundjoin%
\pgfsetlinewidth{1.003750pt}%
\definecolor{currentstroke}{rgb}{0.000000,0.000000,0.000000}%
\pgfsetstrokecolor{currentstroke}%
\pgfsetdash{}{0pt}%
\pgfpathmoveto{\pgfqpoint{0.662449in}{0.147759in}}%
\pgfpathlineto{\pgfqpoint{1.404550in}{0.204375in}}%
\pgfusepath{stroke}%
\end{pgfscope}%
\begin{pgfscope}%
\pgfpathrectangle{\pgfqpoint{0.050000in}{0.050000in}}{\pgfqpoint{2.419000in}{2.419000in}}%
\pgfusepath{clip}%
\pgfsetbuttcap%
\pgfsetroundjoin%
\pgfsetlinewidth{1.003750pt}%
\definecolor{currentstroke}{rgb}{0.000000,0.000000,0.000000}%
\pgfsetstrokecolor{currentstroke}%
\pgfsetdash{}{0pt}%
\pgfusepath{stroke}%
\end{pgfscope}%
\begin{pgfscope}%
\pgfpathrectangle{\pgfqpoint{0.050000in}{0.050000in}}{\pgfqpoint{2.419000in}{2.419000in}}%
\pgfusepath{clip}%
\pgfsetbuttcap%
\pgfsetroundjoin%
\pgfsetlinewidth{1.003750pt}%
\definecolor{currentstroke}{rgb}{0.000000,0.000000,0.000000}%
\pgfsetstrokecolor{currentstroke}%
\pgfsetdash{}{0pt}%
\pgfusepath{stroke}%
\end{pgfscope}%
\begin{pgfscope}%
\pgfpathrectangle{\pgfqpoint{0.050000in}{0.050000in}}{\pgfqpoint{2.419000in}{2.419000in}}%
\pgfusepath{clip}%
\pgfsetbuttcap%
\pgfsetroundjoin%
\pgfsetlinewidth{1.003750pt}%
\definecolor{currentstroke}{rgb}{0.000000,0.000000,0.000000}%
\pgfsetstrokecolor{currentstroke}%
\pgfsetdash{}{0pt}%
\pgfusepath{stroke}%
\end{pgfscope}%
\begin{pgfscope}%
\pgfpathrectangle{\pgfqpoint{0.050000in}{0.050000in}}{\pgfqpoint{2.419000in}{2.419000in}}%
\pgfusepath{clip}%
\pgfsetbuttcap%
\pgfsetroundjoin%
\pgfsetlinewidth{1.003750pt}%
\definecolor{currentstroke}{rgb}{0.000000,0.000000,0.000000}%
\pgfsetstrokecolor{currentstroke}%
\pgfsetdash{}{0pt}%
\pgfpathmoveto{\pgfqpoint{1.404550in}{0.204375in}}%
\pgfpathlineto{\pgfqpoint{1.405908in}{0.464984in}}%
\pgfusepath{stroke}%
\end{pgfscope}%
\begin{pgfscope}%
\pgfpathrectangle{\pgfqpoint{0.050000in}{0.050000in}}{\pgfqpoint{2.419000in}{2.419000in}}%
\pgfusepath{clip}%
\pgfsetbuttcap%
\pgfsetroundjoin%
\pgfsetlinewidth{1.003750pt}%
\definecolor{currentstroke}{rgb}{0.000000,0.000000,0.000000}%
\pgfsetstrokecolor{currentstroke}%
\pgfsetdash{}{0pt}%
\pgfusepath{stroke}%
\end{pgfscope}%
\begin{pgfscope}%
\pgfpathrectangle{\pgfqpoint{0.050000in}{0.050000in}}{\pgfqpoint{2.419000in}{2.419000in}}%
\pgfusepath{clip}%
\pgfsetbuttcap%
\pgfsetroundjoin%
\pgfsetlinewidth{1.003750pt}%
\definecolor{currentstroke}{rgb}{0.000000,0.000000,0.000000}%
\pgfsetstrokecolor{currentstroke}%
\pgfsetdash{}{0pt}%
\pgfusepath{stroke}%
\end{pgfscope}%
\begin{pgfscope}%
\pgfpathrectangle{\pgfqpoint{0.050000in}{0.050000in}}{\pgfqpoint{2.419000in}{2.419000in}}%
\pgfusepath{clip}%
\pgfsetbuttcap%
\pgfsetroundjoin%
\pgfsetlinewidth{1.003750pt}%
\definecolor{currentstroke}{rgb}{0.000000,0.000000,0.000000}%
\pgfsetstrokecolor{currentstroke}%
\pgfsetdash{}{0pt}%
\pgfusepath{stroke}%
\end{pgfscope}%
\begin{pgfscope}%
\pgfpathrectangle{\pgfqpoint{0.050000in}{0.050000in}}{\pgfqpoint{2.419000in}{2.419000in}}%
\pgfusepath{clip}%
\pgfsetbuttcap%
\pgfsetroundjoin%
\pgfsetlinewidth{1.003750pt}%
\definecolor{currentstroke}{rgb}{0.000000,0.000000,0.000000}%
\pgfsetstrokecolor{currentstroke}%
\pgfsetdash{}{0pt}%
\pgfusepath{stroke}%
\end{pgfscope}%
\begin{pgfscope}%
\pgfpathrectangle{\pgfqpoint{0.050000in}{0.050000in}}{\pgfqpoint{2.419000in}{2.419000in}}%
\pgfusepath{clip}%
\pgfsetbuttcap%
\pgfsetroundjoin%
\pgfsetlinewidth{1.003750pt}%
\definecolor{currentstroke}{rgb}{0.000000,0.000000,0.000000}%
\pgfsetstrokecolor{currentstroke}%
\pgfsetdash{}{0pt}%
\pgfusepath{stroke}%
\end{pgfscope}%
\begin{pgfscope}%
\pgfpathrectangle{\pgfqpoint{0.050000in}{0.050000in}}{\pgfqpoint{2.419000in}{2.419000in}}%
\pgfusepath{clip}%
\pgfsetbuttcap%
\pgfsetroundjoin%
\pgfsetlinewidth{1.003750pt}%
\definecolor{currentstroke}{rgb}{0.000000,0.000000,0.000000}%
\pgfsetstrokecolor{currentstroke}%
\pgfsetdash{}{0pt}%
\pgfusepath{stroke}%
\end{pgfscope}%
\begin{pgfscope}%
\pgfpathrectangle{\pgfqpoint{0.050000in}{0.050000in}}{\pgfqpoint{2.419000in}{2.419000in}}%
\pgfusepath{clip}%
\pgfsetbuttcap%
\pgfsetroundjoin%
\pgfsetlinewidth{1.003750pt}%
\definecolor{currentstroke}{rgb}{0.000000,0.000000,0.000000}%
\pgfsetstrokecolor{currentstroke}%
\pgfsetdash{}{0pt}%
\pgfusepath{stroke}%
\end{pgfscope}%
\begin{pgfscope}%
\pgfpathrectangle{\pgfqpoint{0.050000in}{0.050000in}}{\pgfqpoint{2.419000in}{2.419000in}}%
\pgfusepath{clip}%
\pgfsetbuttcap%
\pgfsetroundjoin%
\pgfsetlinewidth{1.003750pt}%
\definecolor{currentstroke}{rgb}{0.000000,0.000000,0.000000}%
\pgfsetstrokecolor{currentstroke}%
\pgfsetdash{}{0pt}%
\pgfusepath{stroke}%
\end{pgfscope}%
\begin{pgfscope}%
\pgfpathrectangle{\pgfqpoint{0.050000in}{0.050000in}}{\pgfqpoint{2.419000in}{2.419000in}}%
\pgfusepath{clip}%
\pgfsetbuttcap%
\pgfsetroundjoin%
\pgfsetlinewidth{1.003750pt}%
\definecolor{currentstroke}{rgb}{0.000000,0.000000,0.000000}%
\pgfsetstrokecolor{currentstroke}%
\pgfsetdash{}{0pt}%
\pgfusepath{stroke}%
\end{pgfscope}%
\begin{pgfscope}%
\pgfpathrectangle{\pgfqpoint{0.050000in}{0.050000in}}{\pgfqpoint{2.419000in}{2.419000in}}%
\pgfusepath{clip}%
\pgfsetbuttcap%
\pgfsetroundjoin%
\pgfsetlinewidth{1.003750pt}%
\definecolor{currentstroke}{rgb}{0.000000,0.000000,0.000000}%
\pgfsetstrokecolor{currentstroke}%
\pgfsetdash{}{0pt}%
\pgfusepath{stroke}%
\end{pgfscope}%
\begin{pgfscope}%
\pgfpathrectangle{\pgfqpoint{0.050000in}{0.050000in}}{\pgfqpoint{2.419000in}{2.419000in}}%
\pgfusepath{clip}%
\pgfsetbuttcap%
\pgfsetroundjoin%
\pgfsetlinewidth{1.003750pt}%
\definecolor{currentstroke}{rgb}{0.000000,0.000000,0.000000}%
\pgfsetstrokecolor{currentstroke}%
\pgfsetdash{}{0pt}%
\pgfusepath{stroke}%
\end{pgfscope}%
\begin{pgfscope}%
\pgfpathrectangle{\pgfqpoint{0.050000in}{0.050000in}}{\pgfqpoint{2.419000in}{2.419000in}}%
\pgfusepath{clip}%
\pgfsetbuttcap%
\pgfsetroundjoin%
\pgfsetlinewidth{1.003750pt}%
\definecolor{currentstroke}{rgb}{0.000000,0.000000,0.000000}%
\pgfsetstrokecolor{currentstroke}%
\pgfsetdash{}{0pt}%
\pgfusepath{stroke}%
\end{pgfscope}%
\begin{pgfscope}%
\pgfpathrectangle{\pgfqpoint{0.050000in}{0.050000in}}{\pgfqpoint{2.419000in}{2.419000in}}%
\pgfusepath{clip}%
\pgfsetbuttcap%
\pgfsetroundjoin%
\pgfsetlinewidth{1.003750pt}%
\definecolor{currentstroke}{rgb}{0.000000,0.000000,0.000000}%
\pgfsetstrokecolor{currentstroke}%
\pgfsetdash{}{0pt}%
\pgfusepath{stroke}%
\end{pgfscope}%
\begin{pgfscope}%
\pgfpathrectangle{\pgfqpoint{0.050000in}{0.050000in}}{\pgfqpoint{2.419000in}{2.419000in}}%
\pgfusepath{clip}%
\pgfsetbuttcap%
\pgfsetroundjoin%
\pgfsetlinewidth{1.003750pt}%
\definecolor{currentstroke}{rgb}{0.000000,0.000000,0.000000}%
\pgfsetstrokecolor{currentstroke}%
\pgfsetdash{}{0pt}%
\pgfusepath{stroke}%
\end{pgfscope}%
\begin{pgfscope}%
\pgfpathrectangle{\pgfqpoint{0.050000in}{0.050000in}}{\pgfqpoint{2.419000in}{2.419000in}}%
\pgfusepath{clip}%
\pgfsetbuttcap%
\pgfsetroundjoin%
\pgfsetlinewidth{1.003750pt}%
\definecolor{currentstroke}{rgb}{0.000000,0.000000,0.000000}%
\pgfsetstrokecolor{currentstroke}%
\pgfsetdash{}{0pt}%
\pgfusepath{stroke}%
\end{pgfscope}%
\begin{pgfscope}%
\pgfpathrectangle{\pgfqpoint{0.050000in}{0.050000in}}{\pgfqpoint{2.419000in}{2.419000in}}%
\pgfusepath{clip}%
\pgfsetbuttcap%
\pgfsetroundjoin%
\pgfsetlinewidth{1.003750pt}%
\definecolor{currentstroke}{rgb}{0.000000,0.000000,0.000000}%
\pgfsetstrokecolor{currentstroke}%
\pgfsetdash{}{0pt}%
\pgfusepath{stroke}%
\end{pgfscope}%
\begin{pgfscope}%
\pgfpathrectangle{\pgfqpoint{0.050000in}{0.050000in}}{\pgfqpoint{2.419000in}{2.419000in}}%
\pgfusepath{clip}%
\pgfsetbuttcap%
\pgfsetroundjoin%
\pgfsetlinewidth{1.003750pt}%
\definecolor{currentstroke}{rgb}{0.000000,0.000000,0.000000}%
\pgfsetstrokecolor{currentstroke}%
\pgfsetdash{}{0pt}%
\pgfusepath{stroke}%
\end{pgfscope}%
\begin{pgfscope}%
\pgfpathrectangle{\pgfqpoint{0.050000in}{0.050000in}}{\pgfqpoint{2.419000in}{2.419000in}}%
\pgfusepath{clip}%
\pgfsetbuttcap%
\pgfsetroundjoin%
\pgfsetlinewidth{1.003750pt}%
\definecolor{currentstroke}{rgb}{0.000000,0.000000,0.000000}%
\pgfsetstrokecolor{currentstroke}%
\pgfsetdash{}{0pt}%
\pgfusepath{stroke}%
\end{pgfscope}%
\begin{pgfscope}%
\pgfpathrectangle{\pgfqpoint{0.050000in}{0.050000in}}{\pgfqpoint{2.419000in}{2.419000in}}%
\pgfusepath{clip}%
\pgfsetbuttcap%
\pgfsetroundjoin%
\pgfsetlinewidth{1.003750pt}%
\definecolor{currentstroke}{rgb}{0.000000,0.000000,0.000000}%
\pgfsetstrokecolor{currentstroke}%
\pgfsetdash{}{0pt}%
\pgfpathmoveto{\pgfqpoint{1.404550in}{0.204375in}}%
\pgfpathlineto{\pgfqpoint{1.525187in}{0.040000in}}%
\pgfusepath{stroke}%
\end{pgfscope}%
\begin{pgfscope}%
\pgfpathrectangle{\pgfqpoint{0.050000in}{0.050000in}}{\pgfqpoint{2.419000in}{2.419000in}}%
\pgfusepath{clip}%
\pgfsetbuttcap%
\pgfsetroundjoin%
\pgfsetlinewidth{1.003750pt}%
\definecolor{currentstroke}{rgb}{0.000000,0.000000,0.000000}%
\pgfsetstrokecolor{currentstroke}%
\pgfsetdash{}{0pt}%
\pgfusepath{stroke}%
\end{pgfscope}%
\begin{pgfscope}%
\pgfpathrectangle{\pgfqpoint{0.050000in}{0.050000in}}{\pgfqpoint{2.419000in}{2.419000in}}%
\pgfusepath{clip}%
\pgfsetbuttcap%
\pgfsetroundjoin%
\pgfsetlinewidth{1.003750pt}%
\definecolor{currentstroke}{rgb}{0.000000,0.000000,0.000000}%
\pgfsetstrokecolor{currentstroke}%
\pgfsetdash{}{0pt}%
\pgfusepath{stroke}%
\end{pgfscope}%
\begin{pgfscope}%
\pgfpathrectangle{\pgfqpoint{0.050000in}{0.050000in}}{\pgfqpoint{2.419000in}{2.419000in}}%
\pgfusepath{clip}%
\pgfsetbuttcap%
\pgfsetroundjoin%
\pgfsetlinewidth{1.003750pt}%
\definecolor{currentstroke}{rgb}{0.000000,0.000000,0.000000}%
\pgfsetstrokecolor{currentstroke}%
\pgfsetdash{}{0pt}%
\pgfusepath{stroke}%
\end{pgfscope}%
\begin{pgfscope}%
\pgfpathrectangle{\pgfqpoint{0.050000in}{0.050000in}}{\pgfqpoint{2.419000in}{2.419000in}}%
\pgfusepath{clip}%
\pgfsetbuttcap%
\pgfsetroundjoin%
\pgfsetlinewidth{1.003750pt}%
\definecolor{currentstroke}{rgb}{0.000000,0.000000,0.000000}%
\pgfsetstrokecolor{currentstroke}%
\pgfsetdash{}{0pt}%
\pgfusepath{stroke}%
\end{pgfscope}%
\begin{pgfscope}%
\pgfpathrectangle{\pgfqpoint{0.050000in}{0.050000in}}{\pgfqpoint{2.419000in}{2.419000in}}%
\pgfusepath{clip}%
\pgfsetbuttcap%
\pgfsetroundjoin%
\pgfsetlinewidth{1.003750pt}%
\definecolor{currentstroke}{rgb}{0.000000,0.000000,0.000000}%
\pgfsetstrokecolor{currentstroke}%
\pgfsetdash{}{0pt}%
\pgfusepath{stroke}%
\end{pgfscope}%
\begin{pgfscope}%
\pgfpathrectangle{\pgfqpoint{0.050000in}{0.050000in}}{\pgfqpoint{2.419000in}{2.419000in}}%
\pgfusepath{clip}%
\pgfsetbuttcap%
\pgfsetroundjoin%
\pgfsetlinewidth{1.003750pt}%
\definecolor{currentstroke}{rgb}{0.000000,0.000000,0.000000}%
\pgfsetstrokecolor{currentstroke}%
\pgfsetdash{}{0pt}%
\pgfusepath{stroke}%
\end{pgfscope}%
\begin{pgfscope}%
\pgfpathrectangle{\pgfqpoint{0.050000in}{0.050000in}}{\pgfqpoint{2.419000in}{2.419000in}}%
\pgfusepath{clip}%
\pgfsetbuttcap%
\pgfsetroundjoin%
\pgfsetlinewidth{1.003750pt}%
\definecolor{currentstroke}{rgb}{0.000000,0.000000,0.000000}%
\pgfsetstrokecolor{currentstroke}%
\pgfsetdash{}{0pt}%
\pgfusepath{stroke}%
\end{pgfscope}%
\begin{pgfscope}%
\pgfpathrectangle{\pgfqpoint{0.050000in}{0.050000in}}{\pgfqpoint{2.419000in}{2.419000in}}%
\pgfusepath{clip}%
\pgfsetbuttcap%
\pgfsetroundjoin%
\pgfsetlinewidth{1.003750pt}%
\definecolor{currentstroke}{rgb}{0.000000,0.000000,0.000000}%
\pgfsetstrokecolor{currentstroke}%
\pgfsetdash{}{0pt}%
\pgfpathmoveto{\pgfqpoint{0.546594in}{0.040000in}}%
\pgfpathlineto{\pgfqpoint{0.662449in}{0.147759in}}%
\pgfusepath{stroke}%
\end{pgfscope}%
\begin{pgfscope}%
\pgfpathrectangle{\pgfqpoint{0.050000in}{0.050000in}}{\pgfqpoint{2.419000in}{2.419000in}}%
\pgfusepath{clip}%
\pgfsetbuttcap%
\pgfsetroundjoin%
\pgfsetlinewidth{1.003750pt}%
\definecolor{currentstroke}{rgb}{0.000000,0.000000,0.000000}%
\pgfsetstrokecolor{currentstroke}%
\pgfsetdash{}{0pt}%
\pgfusepath{stroke}%
\end{pgfscope}%
\begin{pgfscope}%
\pgfpathrectangle{\pgfqpoint{0.050000in}{0.050000in}}{\pgfqpoint{2.419000in}{2.419000in}}%
\pgfusepath{clip}%
\pgfsetbuttcap%
\pgfsetroundjoin%
\pgfsetlinewidth{1.003750pt}%
\definecolor{currentstroke}{rgb}{0.000000,0.000000,0.000000}%
\pgfsetstrokecolor{currentstroke}%
\pgfsetdash{}{0pt}%
\pgfusepath{stroke}%
\end{pgfscope}%
\begin{pgfscope}%
\pgfpathrectangle{\pgfqpoint{0.050000in}{0.050000in}}{\pgfqpoint{2.419000in}{2.419000in}}%
\pgfusepath{clip}%
\pgfsetbuttcap%
\pgfsetroundjoin%
\pgfsetlinewidth{1.003750pt}%
\definecolor{currentstroke}{rgb}{0.000000,0.000000,0.000000}%
\pgfsetstrokecolor{currentstroke}%
\pgfsetdash{}{0pt}%
\pgfusepath{stroke}%
\end{pgfscope}%
\begin{pgfscope}%
\pgfpathrectangle{\pgfqpoint{0.050000in}{0.050000in}}{\pgfqpoint{2.419000in}{2.419000in}}%
\pgfusepath{clip}%
\pgfsetbuttcap%
\pgfsetroundjoin%
\pgfsetlinewidth{1.003750pt}%
\definecolor{currentstroke}{rgb}{0.000000,0.000000,0.000000}%
\pgfsetstrokecolor{currentstroke}%
\pgfsetdash{}{0pt}%
\pgfusepath{stroke}%
\end{pgfscope}%
\begin{pgfscope}%
\pgfpathrectangle{\pgfqpoint{0.050000in}{0.050000in}}{\pgfqpoint{2.419000in}{2.419000in}}%
\pgfusepath{clip}%
\pgfsetbuttcap%
\pgfsetroundjoin%
\pgfsetlinewidth{1.003750pt}%
\definecolor{currentstroke}{rgb}{0.000000,0.000000,0.000000}%
\pgfsetstrokecolor{currentstroke}%
\pgfsetdash{}{0pt}%
\pgfusepath{stroke}%
\end{pgfscope}%
\begin{pgfscope}%
\pgfpathrectangle{\pgfqpoint{0.050000in}{0.050000in}}{\pgfqpoint{2.419000in}{2.419000in}}%
\pgfusepath{clip}%
\pgfsetbuttcap%
\pgfsetroundjoin%
\pgfsetlinewidth{1.003750pt}%
\definecolor{currentstroke}{rgb}{0.000000,0.000000,0.000000}%
\pgfsetstrokecolor{currentstroke}%
\pgfsetdash{}{0pt}%
\pgfusepath{stroke}%
\end{pgfscope}%
\begin{pgfscope}%
\pgfpathrectangle{\pgfqpoint{0.050000in}{0.050000in}}{\pgfqpoint{2.419000in}{2.419000in}}%
\pgfusepath{clip}%
\pgfsetbuttcap%
\pgfsetroundjoin%
\pgfsetlinewidth{1.003750pt}%
\definecolor{currentstroke}{rgb}{0.000000,0.000000,0.000000}%
\pgfsetstrokecolor{currentstroke}%
\pgfsetdash{}{0pt}%
\pgfusepath{stroke}%
\end{pgfscope}%
\begin{pgfscope}%
\pgfpathrectangle{\pgfqpoint{0.050000in}{0.050000in}}{\pgfqpoint{2.419000in}{2.419000in}}%
\pgfusepath{clip}%
\pgfsetbuttcap%
\pgfsetroundjoin%
\pgfsetlinewidth{1.003750pt}%
\definecolor{currentstroke}{rgb}{0.000000,0.000000,0.000000}%
\pgfsetstrokecolor{currentstroke}%
\pgfsetdash{}{0pt}%
\pgfusepath{stroke}%
\end{pgfscope}%
\begin{pgfscope}%
\pgfpathrectangle{\pgfqpoint{0.050000in}{0.050000in}}{\pgfqpoint{2.419000in}{2.419000in}}%
\pgfusepath{clip}%
\pgfsetbuttcap%
\pgfsetroundjoin%
\pgfsetlinewidth{1.003750pt}%
\definecolor{currentstroke}{rgb}{0.000000,0.000000,0.000000}%
\pgfsetstrokecolor{currentstroke}%
\pgfsetdash{}{0pt}%
\pgfusepath{stroke}%
\end{pgfscope}%
\begin{pgfscope}%
\pgfpathrectangle{\pgfqpoint{0.050000in}{0.050000in}}{\pgfqpoint{2.419000in}{2.419000in}}%
\pgfusepath{clip}%
\pgfsetbuttcap%
\pgfsetroundjoin%
\pgfsetlinewidth{1.003750pt}%
\definecolor{currentstroke}{rgb}{0.000000,0.000000,0.000000}%
\pgfsetstrokecolor{currentstroke}%
\pgfsetdash{}{0pt}%
\pgfusepath{stroke}%
\end{pgfscope}%
\begin{pgfscope}%
\pgfpathrectangle{\pgfqpoint{0.050000in}{0.050000in}}{\pgfqpoint{2.419000in}{2.419000in}}%
\pgfusepath{clip}%
\pgfsetbuttcap%
\pgfsetroundjoin%
\pgfsetlinewidth{1.003750pt}%
\definecolor{currentstroke}{rgb}{0.000000,0.000000,0.000000}%
\pgfsetstrokecolor{currentstroke}%
\pgfsetdash{}{0pt}%
\pgfusepath{stroke}%
\end{pgfscope}%
\begin{pgfscope}%
\pgfpathrectangle{\pgfqpoint{0.050000in}{0.050000in}}{\pgfqpoint{2.419000in}{2.419000in}}%
\pgfusepath{clip}%
\pgfsetbuttcap%
\pgfsetroundjoin%
\pgfsetlinewidth{1.003750pt}%
\definecolor{currentstroke}{rgb}{0.000000,0.000000,0.000000}%
\pgfsetstrokecolor{currentstroke}%
\pgfsetdash{}{0pt}%
\pgfusepath{stroke}%
\end{pgfscope}%
\begin{pgfscope}%
\pgfpathrectangle{\pgfqpoint{0.050000in}{0.050000in}}{\pgfqpoint{2.419000in}{2.419000in}}%
\pgfusepath{clip}%
\pgfsetbuttcap%
\pgfsetroundjoin%
\pgfsetlinewidth{1.003750pt}%
\definecolor{currentstroke}{rgb}{0.000000,0.000000,0.000000}%
\pgfsetstrokecolor{currentstroke}%
\pgfsetdash{}{0pt}%
\pgfusepath{stroke}%
\end{pgfscope}%
\begin{pgfscope}%
\pgfpathrectangle{\pgfqpoint{0.050000in}{0.050000in}}{\pgfqpoint{2.419000in}{2.419000in}}%
\pgfusepath{clip}%
\pgfsetbuttcap%
\pgfsetroundjoin%
\pgfsetlinewidth{1.003750pt}%
\definecolor{currentstroke}{rgb}{0.000000,0.000000,0.000000}%
\pgfsetstrokecolor{currentstroke}%
\pgfsetdash{}{0pt}%
\pgfusepath{stroke}%
\end{pgfscope}%
\begin{pgfscope}%
\pgfpathrectangle{\pgfqpoint{0.050000in}{0.050000in}}{\pgfqpoint{2.419000in}{2.419000in}}%
\pgfusepath{clip}%
\pgfsetbuttcap%
\pgfsetroundjoin%
\pgfsetlinewidth{1.003750pt}%
\definecolor{currentstroke}{rgb}{0.000000,0.000000,0.000000}%
\pgfsetstrokecolor{currentstroke}%
\pgfsetdash{}{0pt}%
\pgfusepath{stroke}%
\end{pgfscope}%
\begin{pgfscope}%
\pgfpathrectangle{\pgfqpoint{0.050000in}{0.050000in}}{\pgfqpoint{2.419000in}{2.419000in}}%
\pgfusepath{clip}%
\pgfsetbuttcap%
\pgfsetroundjoin%
\pgfsetlinewidth{1.003750pt}%
\definecolor{currentstroke}{rgb}{0.000000,0.000000,0.000000}%
\pgfsetstrokecolor{currentstroke}%
\pgfsetdash{}{0pt}%
\pgfusepath{stroke}%
\end{pgfscope}%
\begin{pgfscope}%
\pgfpathrectangle{\pgfqpoint{0.050000in}{0.050000in}}{\pgfqpoint{2.419000in}{2.419000in}}%
\pgfusepath{clip}%
\pgfsetbuttcap%
\pgfsetroundjoin%
\pgfsetlinewidth{1.003750pt}%
\definecolor{currentstroke}{rgb}{0.000000,0.000000,0.000000}%
\pgfsetstrokecolor{currentstroke}%
\pgfsetdash{}{0pt}%
\pgfusepath{stroke}%
\end{pgfscope}%
\begin{pgfscope}%
\pgfpathrectangle{\pgfqpoint{0.050000in}{0.050000in}}{\pgfqpoint{2.419000in}{2.419000in}}%
\pgfusepath{clip}%
\pgfsetbuttcap%
\pgfsetroundjoin%
\pgfsetlinewidth{1.003750pt}%
\definecolor{currentstroke}{rgb}{0.000000,0.000000,0.000000}%
\pgfsetstrokecolor{currentstroke}%
\pgfsetdash{}{0pt}%
\pgfusepath{stroke}%
\end{pgfscope}%
\begin{pgfscope}%
\pgfpathrectangle{\pgfqpoint{0.050000in}{0.050000in}}{\pgfqpoint{2.419000in}{2.419000in}}%
\pgfusepath{clip}%
\pgfsetbuttcap%
\pgfsetroundjoin%
\pgfsetlinewidth{1.003750pt}%
\definecolor{currentstroke}{rgb}{0.000000,0.000000,0.000000}%
\pgfsetstrokecolor{currentstroke}%
\pgfsetdash{}{0pt}%
\pgfusepath{stroke}%
\end{pgfscope}%
\begin{pgfscope}%
\pgfpathrectangle{\pgfqpoint{0.050000in}{0.050000in}}{\pgfqpoint{2.419000in}{2.419000in}}%
\pgfusepath{clip}%
\pgfsetbuttcap%
\pgfsetroundjoin%
\pgfsetlinewidth{1.003750pt}%
\definecolor{currentstroke}{rgb}{0.000000,0.000000,0.000000}%
\pgfsetstrokecolor{currentstroke}%
\pgfsetdash{}{0pt}%
\pgfusepath{stroke}%
\end{pgfscope}%
\begin{pgfscope}%
\pgfpathrectangle{\pgfqpoint{0.050000in}{0.050000in}}{\pgfqpoint{2.419000in}{2.419000in}}%
\pgfusepath{clip}%
\pgfsetbuttcap%
\pgfsetroundjoin%
\pgfsetlinewidth{1.003750pt}%
\definecolor{currentstroke}{rgb}{0.000000,0.000000,0.000000}%
\pgfsetstrokecolor{currentstroke}%
\pgfsetdash{}{0pt}%
\pgfusepath{stroke}%
\end{pgfscope}%
\begin{pgfscope}%
\pgfpathrectangle{\pgfqpoint{0.050000in}{0.050000in}}{\pgfqpoint{2.419000in}{2.419000in}}%
\pgfusepath{clip}%
\pgfsetbuttcap%
\pgfsetroundjoin%
\pgfsetlinewidth{1.003750pt}%
\definecolor{currentstroke}{rgb}{0.000000,0.000000,0.000000}%
\pgfsetstrokecolor{currentstroke}%
\pgfsetdash{}{0pt}%
\pgfusepath{stroke}%
\end{pgfscope}%
\begin{pgfscope}%
\pgfpathrectangle{\pgfqpoint{0.050000in}{0.050000in}}{\pgfqpoint{2.419000in}{2.419000in}}%
\pgfusepath{clip}%
\pgfsetbuttcap%
\pgfsetroundjoin%
\pgfsetlinewidth{1.003750pt}%
\definecolor{currentstroke}{rgb}{0.000000,0.000000,0.000000}%
\pgfsetstrokecolor{currentstroke}%
\pgfsetdash{}{0pt}%
\pgfusepath{stroke}%
\end{pgfscope}%
\begin{pgfscope}%
\pgfpathrectangle{\pgfqpoint{0.050000in}{0.050000in}}{\pgfqpoint{2.419000in}{2.419000in}}%
\pgfusepath{clip}%
\pgfsetbuttcap%
\pgfsetroundjoin%
\pgfsetlinewidth{1.003750pt}%
\definecolor{currentstroke}{rgb}{0.000000,0.000000,0.000000}%
\pgfsetstrokecolor{currentstroke}%
\pgfsetdash{}{0pt}%
\pgfusepath{stroke}%
\end{pgfscope}%
\begin{pgfscope}%
\pgfpathrectangle{\pgfqpoint{0.050000in}{0.050000in}}{\pgfqpoint{2.419000in}{2.419000in}}%
\pgfusepath{clip}%
\pgfsetbuttcap%
\pgfsetroundjoin%
\pgfsetlinewidth{1.003750pt}%
\definecolor{currentstroke}{rgb}{0.000000,0.000000,0.000000}%
\pgfsetstrokecolor{currentstroke}%
\pgfsetdash{}{0pt}%
\pgfusepath{stroke}%
\end{pgfscope}%
\begin{pgfscope}%
\pgfpathrectangle{\pgfqpoint{0.050000in}{0.050000in}}{\pgfqpoint{2.419000in}{2.419000in}}%
\pgfusepath{clip}%
\pgfsetbuttcap%
\pgfsetroundjoin%
\pgfsetlinewidth{1.003750pt}%
\definecolor{currentstroke}{rgb}{0.000000,0.000000,0.000000}%
\pgfsetstrokecolor{currentstroke}%
\pgfsetdash{}{0pt}%
\pgfusepath{stroke}%
\end{pgfscope}%
\begin{pgfscope}%
\pgfpathrectangle{\pgfqpoint{0.050000in}{0.050000in}}{\pgfqpoint{2.419000in}{2.419000in}}%
\pgfusepath{clip}%
\pgfsetbuttcap%
\pgfsetroundjoin%
\pgfsetlinewidth{1.003750pt}%
\definecolor{currentstroke}{rgb}{0.000000,0.000000,0.000000}%
\pgfsetstrokecolor{currentstroke}%
\pgfsetdash{}{0pt}%
\pgfusepath{stroke}%
\end{pgfscope}%
\begin{pgfscope}%
\pgfpathrectangle{\pgfqpoint{0.050000in}{0.050000in}}{\pgfqpoint{2.419000in}{2.419000in}}%
\pgfusepath{clip}%
\pgfsetbuttcap%
\pgfsetroundjoin%
\pgfsetlinewidth{1.003750pt}%
\definecolor{currentstroke}{rgb}{0.000000,0.000000,0.000000}%
\pgfsetstrokecolor{currentstroke}%
\pgfsetdash{}{0pt}%
\pgfusepath{stroke}%
\end{pgfscope}%
\begin{pgfscope}%
\pgfpathrectangle{\pgfqpoint{0.050000in}{0.050000in}}{\pgfqpoint{2.419000in}{2.419000in}}%
\pgfusepath{clip}%
\pgfsetbuttcap%
\pgfsetroundjoin%
\pgfsetlinewidth{1.003750pt}%
\definecolor{currentstroke}{rgb}{0.000000,0.000000,0.000000}%
\pgfsetstrokecolor{currentstroke}%
\pgfsetdash{}{0pt}%
\pgfusepath{stroke}%
\end{pgfscope}%
\begin{pgfscope}%
\pgfpathrectangle{\pgfqpoint{0.050000in}{0.050000in}}{\pgfqpoint{2.419000in}{2.419000in}}%
\pgfusepath{clip}%
\pgfsetbuttcap%
\pgfsetroundjoin%
\pgfsetlinewidth{1.003750pt}%
\definecolor{currentstroke}{rgb}{0.000000,0.000000,0.000000}%
\pgfsetstrokecolor{currentstroke}%
\pgfsetdash{}{0pt}%
\pgfusepath{stroke}%
\end{pgfscope}%
\begin{pgfscope}%
\pgfpathrectangle{\pgfqpoint{0.050000in}{0.050000in}}{\pgfqpoint{2.419000in}{2.419000in}}%
\pgfusepath{clip}%
\pgfsetbuttcap%
\pgfsetroundjoin%
\pgfsetlinewidth{1.003750pt}%
\definecolor{currentstroke}{rgb}{0.000000,0.000000,0.000000}%
\pgfsetstrokecolor{currentstroke}%
\pgfsetdash{}{0pt}%
\pgfusepath{stroke}%
\end{pgfscope}%
\begin{pgfscope}%
\pgfpathrectangle{\pgfqpoint{0.050000in}{0.050000in}}{\pgfqpoint{2.419000in}{2.419000in}}%
\pgfusepath{clip}%
\pgfsetbuttcap%
\pgfsetroundjoin%
\pgfsetlinewidth{1.003750pt}%
\definecolor{currentstroke}{rgb}{0.000000,0.000000,0.000000}%
\pgfsetstrokecolor{currentstroke}%
\pgfsetdash{}{0pt}%
\pgfusepath{stroke}%
\end{pgfscope}%
\begin{pgfscope}%
\pgfpathrectangle{\pgfqpoint{0.050000in}{0.050000in}}{\pgfqpoint{2.419000in}{2.419000in}}%
\pgfusepath{clip}%
\pgfsetbuttcap%
\pgfsetroundjoin%
\pgfsetlinewidth{1.003750pt}%
\definecolor{currentstroke}{rgb}{0.000000,0.000000,0.000000}%
\pgfsetstrokecolor{currentstroke}%
\pgfsetdash{}{0pt}%
\pgfusepath{stroke}%
\end{pgfscope}%
\begin{pgfscope}%
\pgfpathrectangle{\pgfqpoint{0.050000in}{0.050000in}}{\pgfqpoint{2.419000in}{2.419000in}}%
\pgfusepath{clip}%
\pgfsetbuttcap%
\pgfsetroundjoin%
\pgfsetlinewidth{1.003750pt}%
\definecolor{currentstroke}{rgb}{0.000000,0.000000,0.000000}%
\pgfsetstrokecolor{currentstroke}%
\pgfsetdash{}{0pt}%
\pgfusepath{stroke}%
\end{pgfscope}%
\begin{pgfscope}%
\pgfpathrectangle{\pgfqpoint{0.050000in}{0.050000in}}{\pgfqpoint{2.419000in}{2.419000in}}%
\pgfusepath{clip}%
\pgfsetbuttcap%
\pgfsetroundjoin%
\pgfsetlinewidth{1.003750pt}%
\definecolor{currentstroke}{rgb}{0.000000,0.000000,0.000000}%
\pgfsetstrokecolor{currentstroke}%
\pgfsetdash{}{0pt}%
\pgfusepath{stroke}%
\end{pgfscope}%
\begin{pgfscope}%
\pgfpathrectangle{\pgfqpoint{0.050000in}{0.050000in}}{\pgfqpoint{2.419000in}{2.419000in}}%
\pgfusepath{clip}%
\pgfsetbuttcap%
\pgfsetroundjoin%
\pgfsetlinewidth{1.003750pt}%
\definecolor{currentstroke}{rgb}{0.000000,0.000000,0.000000}%
\pgfsetstrokecolor{currentstroke}%
\pgfsetdash{}{0pt}%
\pgfusepath{stroke}%
\end{pgfscope}%
\begin{pgfscope}%
\pgfpathrectangle{\pgfqpoint{0.050000in}{0.050000in}}{\pgfqpoint{2.419000in}{2.419000in}}%
\pgfusepath{clip}%
\pgfsetbuttcap%
\pgfsetroundjoin%
\pgfsetlinewidth{1.003750pt}%
\definecolor{currentstroke}{rgb}{0.000000,0.000000,0.000000}%
\pgfsetstrokecolor{currentstroke}%
\pgfsetdash{}{0pt}%
\pgfusepath{stroke}%
\end{pgfscope}%
\begin{pgfscope}%
\pgfpathrectangle{\pgfqpoint{0.050000in}{0.050000in}}{\pgfqpoint{2.419000in}{2.419000in}}%
\pgfusepath{clip}%
\pgfsetbuttcap%
\pgfsetroundjoin%
\pgfsetlinewidth{1.003750pt}%
\definecolor{currentstroke}{rgb}{0.000000,0.000000,0.000000}%
\pgfsetstrokecolor{currentstroke}%
\pgfsetdash{}{0pt}%
\pgfusepath{stroke}%
\end{pgfscope}%
\begin{pgfscope}%
\pgfpathrectangle{\pgfqpoint{0.050000in}{0.050000in}}{\pgfqpoint{2.419000in}{2.419000in}}%
\pgfusepath{clip}%
\pgfsetbuttcap%
\pgfsetroundjoin%
\pgfsetlinewidth{1.003750pt}%
\definecolor{currentstroke}{rgb}{0.000000,0.000000,0.000000}%
\pgfsetstrokecolor{currentstroke}%
\pgfsetdash{}{0pt}%
\pgfusepath{stroke}%
\end{pgfscope}%
\begin{pgfscope}%
\pgfpathrectangle{\pgfqpoint{0.050000in}{0.050000in}}{\pgfqpoint{2.419000in}{2.419000in}}%
\pgfusepath{clip}%
\pgfsetbuttcap%
\pgfsetroundjoin%
\pgfsetlinewidth{1.003750pt}%
\definecolor{currentstroke}{rgb}{0.000000,0.000000,0.000000}%
\pgfsetstrokecolor{currentstroke}%
\pgfsetdash{}{0pt}%
\pgfusepath{stroke}%
\end{pgfscope}%
\begin{pgfscope}%
\pgfpathrectangle{\pgfqpoint{0.050000in}{0.050000in}}{\pgfqpoint{2.419000in}{2.419000in}}%
\pgfusepath{clip}%
\pgfsetbuttcap%
\pgfsetroundjoin%
\pgfsetlinewidth{1.003750pt}%
\definecolor{currentstroke}{rgb}{0.000000,0.000000,0.000000}%
\pgfsetstrokecolor{currentstroke}%
\pgfsetdash{}{0pt}%
\pgfusepath{stroke}%
\end{pgfscope}%
\begin{pgfscope}%
\pgfpathrectangle{\pgfqpoint{0.050000in}{0.050000in}}{\pgfqpoint{2.419000in}{2.419000in}}%
\pgfusepath{clip}%
\pgfsetbuttcap%
\pgfsetroundjoin%
\pgfsetlinewidth{1.003750pt}%
\definecolor{currentstroke}{rgb}{0.000000,0.000000,0.000000}%
\pgfsetstrokecolor{currentstroke}%
\pgfsetdash{}{0pt}%
\pgfusepath{stroke}%
\end{pgfscope}%
\begin{pgfscope}%
\pgfpathrectangle{\pgfqpoint{0.050000in}{0.050000in}}{\pgfqpoint{2.419000in}{2.419000in}}%
\pgfusepath{clip}%
\pgfsetbuttcap%
\pgfsetroundjoin%
\pgfsetlinewidth{1.003750pt}%
\definecolor{currentstroke}{rgb}{0.000000,0.000000,0.000000}%
\pgfsetstrokecolor{currentstroke}%
\pgfsetdash{}{0pt}%
\pgfusepath{stroke}%
\end{pgfscope}%
\begin{pgfscope}%
\pgfpathrectangle{\pgfqpoint{0.050000in}{0.050000in}}{\pgfqpoint{2.419000in}{2.419000in}}%
\pgfusepath{clip}%
\pgfsetbuttcap%
\pgfsetroundjoin%
\pgfsetlinewidth{1.003750pt}%
\definecolor{currentstroke}{rgb}{0.000000,0.000000,0.000000}%
\pgfsetstrokecolor{currentstroke}%
\pgfsetdash{}{0pt}%
\pgfusepath{stroke}%
\end{pgfscope}%
\begin{pgfscope}%
\pgfpathrectangle{\pgfqpoint{0.050000in}{0.050000in}}{\pgfqpoint{2.419000in}{2.419000in}}%
\pgfusepath{clip}%
\pgfsetbuttcap%
\pgfsetroundjoin%
\pgfsetlinewidth{1.003750pt}%
\definecolor{currentstroke}{rgb}{0.000000,0.000000,0.000000}%
\pgfsetstrokecolor{currentstroke}%
\pgfsetdash{}{0pt}%
\pgfusepath{stroke}%
\end{pgfscope}%
\begin{pgfscope}%
\pgfpathrectangle{\pgfqpoint{0.050000in}{0.050000in}}{\pgfqpoint{2.419000in}{2.419000in}}%
\pgfusepath{clip}%
\pgfsetbuttcap%
\pgfsetroundjoin%
\pgfsetlinewidth{1.003750pt}%
\definecolor{currentstroke}{rgb}{0.000000,0.000000,0.000000}%
\pgfsetstrokecolor{currentstroke}%
\pgfsetdash{}{0pt}%
\pgfusepath{stroke}%
\end{pgfscope}%
\begin{pgfscope}%
\pgfpathrectangle{\pgfqpoint{0.050000in}{0.050000in}}{\pgfqpoint{2.419000in}{2.419000in}}%
\pgfusepath{clip}%
\pgfsetbuttcap%
\pgfsetroundjoin%
\pgfsetlinewidth{1.003750pt}%
\definecolor{currentstroke}{rgb}{0.000000,0.000000,0.000000}%
\pgfsetstrokecolor{currentstroke}%
\pgfsetdash{}{0pt}%
\pgfusepath{stroke}%
\end{pgfscope}%
\begin{pgfscope}%
\pgfpathrectangle{\pgfqpoint{0.050000in}{0.050000in}}{\pgfqpoint{2.419000in}{2.419000in}}%
\pgfusepath{clip}%
\pgfsetbuttcap%
\pgfsetroundjoin%
\pgfsetlinewidth{1.003750pt}%
\definecolor{currentstroke}{rgb}{0.000000,0.000000,0.000000}%
\pgfsetstrokecolor{currentstroke}%
\pgfsetdash{}{0pt}%
\pgfusepath{stroke}%
\end{pgfscope}%
\begin{pgfscope}%
\pgfpathrectangle{\pgfqpoint{0.050000in}{0.050000in}}{\pgfqpoint{2.419000in}{2.419000in}}%
\pgfusepath{clip}%
\pgfsetbuttcap%
\pgfsetroundjoin%
\pgfsetlinewidth{1.003750pt}%
\definecolor{currentstroke}{rgb}{0.000000,0.000000,0.000000}%
\pgfsetstrokecolor{currentstroke}%
\pgfsetdash{}{0pt}%
\pgfusepath{stroke}%
\end{pgfscope}%
\begin{pgfscope}%
\pgfpathrectangle{\pgfqpoint{0.050000in}{0.050000in}}{\pgfqpoint{2.419000in}{2.419000in}}%
\pgfusepath{clip}%
\pgfsetbuttcap%
\pgfsetroundjoin%
\pgfsetlinewidth{1.003750pt}%
\definecolor{currentstroke}{rgb}{0.000000,0.000000,0.000000}%
\pgfsetstrokecolor{currentstroke}%
\pgfsetdash{}{0pt}%
\pgfusepath{stroke}%
\end{pgfscope}%
\begin{pgfscope}%
\pgfpathrectangle{\pgfqpoint{0.050000in}{0.050000in}}{\pgfqpoint{2.419000in}{2.419000in}}%
\pgfusepath{clip}%
\pgfsetbuttcap%
\pgfsetroundjoin%
\pgfsetlinewidth{1.003750pt}%
\definecolor{currentstroke}{rgb}{0.000000,0.000000,0.000000}%
\pgfsetstrokecolor{currentstroke}%
\pgfsetdash{}{0pt}%
\pgfusepath{stroke}%
\end{pgfscope}%
\begin{pgfscope}%
\pgfpathrectangle{\pgfqpoint{0.050000in}{0.050000in}}{\pgfqpoint{2.419000in}{2.419000in}}%
\pgfusepath{clip}%
\pgfsetbuttcap%
\pgfsetroundjoin%
\pgfsetlinewidth{1.003750pt}%
\definecolor{currentstroke}{rgb}{0.000000,0.000000,0.000000}%
\pgfsetstrokecolor{currentstroke}%
\pgfsetdash{}{0pt}%
\pgfusepath{stroke}%
\end{pgfscope}%
\begin{pgfscope}%
\pgfpathrectangle{\pgfqpoint{0.050000in}{0.050000in}}{\pgfqpoint{2.419000in}{2.419000in}}%
\pgfusepath{clip}%
\pgfsetbuttcap%
\pgfsetroundjoin%
\pgfsetlinewidth{1.003750pt}%
\definecolor{currentstroke}{rgb}{0.000000,0.000000,0.000000}%
\pgfsetstrokecolor{currentstroke}%
\pgfsetdash{}{0pt}%
\pgfusepath{stroke}%
\end{pgfscope}%
\begin{pgfscope}%
\pgfpathrectangle{\pgfqpoint{0.050000in}{0.050000in}}{\pgfqpoint{2.419000in}{2.419000in}}%
\pgfusepath{clip}%
\pgfsetbuttcap%
\pgfsetroundjoin%
\pgfsetlinewidth{1.003750pt}%
\definecolor{currentstroke}{rgb}{0.000000,0.000000,0.000000}%
\pgfsetstrokecolor{currentstroke}%
\pgfsetdash{}{0pt}%
\pgfusepath{stroke}%
\end{pgfscope}%
\begin{pgfscope}%
\pgfpathrectangle{\pgfqpoint{0.050000in}{0.050000in}}{\pgfqpoint{2.419000in}{2.419000in}}%
\pgfusepath{clip}%
\pgfsetbuttcap%
\pgfsetroundjoin%
\pgfsetlinewidth{1.003750pt}%
\definecolor{currentstroke}{rgb}{0.000000,0.000000,0.000000}%
\pgfsetstrokecolor{currentstroke}%
\pgfsetdash{}{0pt}%
\pgfusepath{stroke}%
\end{pgfscope}%
\begin{pgfscope}%
\pgfpathrectangle{\pgfqpoint{0.050000in}{0.050000in}}{\pgfqpoint{2.419000in}{2.419000in}}%
\pgfusepath{clip}%
\pgfsetbuttcap%
\pgfsetroundjoin%
\pgfsetlinewidth{1.003750pt}%
\definecolor{currentstroke}{rgb}{0.000000,0.000000,0.000000}%
\pgfsetstrokecolor{currentstroke}%
\pgfsetdash{}{0pt}%
\pgfusepath{stroke}%
\end{pgfscope}%
\begin{pgfscope}%
\pgfpathrectangle{\pgfqpoint{0.050000in}{0.050000in}}{\pgfqpoint{2.419000in}{2.419000in}}%
\pgfusepath{clip}%
\pgfsetbuttcap%
\pgfsetroundjoin%
\pgfsetlinewidth{1.003750pt}%
\definecolor{currentstroke}{rgb}{0.000000,0.000000,0.000000}%
\pgfsetstrokecolor{currentstroke}%
\pgfsetdash{}{0pt}%
\pgfusepath{stroke}%
\end{pgfscope}%
\begin{pgfscope}%
\pgfpathrectangle{\pgfqpoint{0.050000in}{0.050000in}}{\pgfqpoint{2.419000in}{2.419000in}}%
\pgfusepath{clip}%
\pgfsetbuttcap%
\pgfsetroundjoin%
\pgfsetlinewidth{1.003750pt}%
\definecolor{currentstroke}{rgb}{0.000000,0.000000,0.000000}%
\pgfsetstrokecolor{currentstroke}%
\pgfsetdash{}{0pt}%
\pgfusepath{stroke}%
\end{pgfscope}%
\begin{pgfscope}%
\pgfpathrectangle{\pgfqpoint{0.050000in}{0.050000in}}{\pgfqpoint{2.419000in}{2.419000in}}%
\pgfusepath{clip}%
\pgfsetbuttcap%
\pgfsetroundjoin%
\pgfsetlinewidth{1.003750pt}%
\definecolor{currentstroke}{rgb}{0.000000,0.000000,0.000000}%
\pgfsetstrokecolor{currentstroke}%
\pgfsetdash{}{0pt}%
\pgfusepath{stroke}%
\end{pgfscope}%
\begin{pgfscope}%
\pgfpathrectangle{\pgfqpoint{0.050000in}{0.050000in}}{\pgfqpoint{2.419000in}{2.419000in}}%
\pgfusepath{clip}%
\pgfsetbuttcap%
\pgfsetroundjoin%
\pgfsetlinewidth{1.003750pt}%
\definecolor{currentstroke}{rgb}{0.000000,0.000000,0.000000}%
\pgfsetstrokecolor{currentstroke}%
\pgfsetdash{}{0pt}%
\pgfusepath{stroke}%
\end{pgfscope}%
\begin{pgfscope}%
\pgfpathrectangle{\pgfqpoint{0.050000in}{0.050000in}}{\pgfqpoint{2.419000in}{2.419000in}}%
\pgfusepath{clip}%
\pgfsetbuttcap%
\pgfsetroundjoin%
\pgfsetlinewidth{1.003750pt}%
\definecolor{currentstroke}{rgb}{0.000000,0.000000,0.000000}%
\pgfsetstrokecolor{currentstroke}%
\pgfsetdash{}{0pt}%
\pgfusepath{stroke}%
\end{pgfscope}%
\begin{pgfscope}%
\pgfpathrectangle{\pgfqpoint{0.050000in}{0.050000in}}{\pgfqpoint{2.419000in}{2.419000in}}%
\pgfusepath{clip}%
\pgfsetbuttcap%
\pgfsetroundjoin%
\pgfsetlinewidth{1.003750pt}%
\definecolor{currentstroke}{rgb}{0.000000,0.000000,0.000000}%
\pgfsetstrokecolor{currentstroke}%
\pgfsetdash{}{0pt}%
\pgfusepath{stroke}%
\end{pgfscope}%
\begin{pgfscope}%
\pgfpathrectangle{\pgfqpoint{0.050000in}{0.050000in}}{\pgfqpoint{2.419000in}{2.419000in}}%
\pgfusepath{clip}%
\pgfsetbuttcap%
\pgfsetroundjoin%
\pgfsetlinewidth{1.003750pt}%
\definecolor{currentstroke}{rgb}{0.000000,0.000000,0.000000}%
\pgfsetstrokecolor{currentstroke}%
\pgfsetdash{}{0pt}%
\pgfusepath{stroke}%
\end{pgfscope}%
\begin{pgfscope}%
\pgfpathrectangle{\pgfqpoint{0.050000in}{0.050000in}}{\pgfqpoint{2.419000in}{2.419000in}}%
\pgfusepath{clip}%
\pgfsetbuttcap%
\pgfsetroundjoin%
\pgfsetlinewidth{1.003750pt}%
\definecolor{currentstroke}{rgb}{0.000000,0.000000,0.000000}%
\pgfsetstrokecolor{currentstroke}%
\pgfsetdash{}{0pt}%
\pgfusepath{stroke}%
\end{pgfscope}%
\begin{pgfscope}%
\pgfpathrectangle{\pgfqpoint{0.050000in}{0.050000in}}{\pgfqpoint{2.419000in}{2.419000in}}%
\pgfusepath{clip}%
\pgfsetbuttcap%
\pgfsetroundjoin%
\pgfsetlinewidth{1.003750pt}%
\definecolor{currentstroke}{rgb}{0.000000,0.000000,0.000000}%
\pgfsetstrokecolor{currentstroke}%
\pgfsetdash{}{0pt}%
\pgfusepath{stroke}%
\end{pgfscope}%
\begin{pgfscope}%
\pgfpathrectangle{\pgfqpoint{0.050000in}{0.050000in}}{\pgfqpoint{2.419000in}{2.419000in}}%
\pgfusepath{clip}%
\pgfsetbuttcap%
\pgfsetroundjoin%
\pgfsetlinewidth{1.003750pt}%
\definecolor{currentstroke}{rgb}{0.000000,0.000000,0.000000}%
\pgfsetstrokecolor{currentstroke}%
\pgfsetdash{}{0pt}%
\pgfusepath{stroke}%
\end{pgfscope}%
\begin{pgfscope}%
\pgfpathrectangle{\pgfqpoint{0.050000in}{0.050000in}}{\pgfqpoint{2.419000in}{2.419000in}}%
\pgfusepath{clip}%
\pgfsetbuttcap%
\pgfsetroundjoin%
\pgfsetlinewidth{1.003750pt}%
\definecolor{currentstroke}{rgb}{0.000000,0.000000,0.000000}%
\pgfsetstrokecolor{currentstroke}%
\pgfsetdash{}{0pt}%
\pgfusepath{stroke}%
\end{pgfscope}%
\begin{pgfscope}%
\pgfpathrectangle{\pgfqpoint{0.050000in}{0.050000in}}{\pgfqpoint{2.419000in}{2.419000in}}%
\pgfusepath{clip}%
\pgfsetbuttcap%
\pgfsetroundjoin%
\pgfsetlinewidth{1.003750pt}%
\definecolor{currentstroke}{rgb}{0.000000,0.000000,0.000000}%
\pgfsetstrokecolor{currentstroke}%
\pgfsetdash{}{0pt}%
\pgfusepath{stroke}%
\end{pgfscope}%
\begin{pgfscope}%
\pgfpathrectangle{\pgfqpoint{0.050000in}{0.050000in}}{\pgfqpoint{2.419000in}{2.419000in}}%
\pgfusepath{clip}%
\pgfsetbuttcap%
\pgfsetroundjoin%
\pgfsetlinewidth{1.003750pt}%
\definecolor{currentstroke}{rgb}{0.000000,0.000000,0.000000}%
\pgfsetstrokecolor{currentstroke}%
\pgfsetdash{}{0pt}%
\pgfusepath{stroke}%
\end{pgfscope}%
\begin{pgfscope}%
\pgfpathrectangle{\pgfqpoint{0.050000in}{0.050000in}}{\pgfqpoint{2.419000in}{2.419000in}}%
\pgfusepath{clip}%
\pgfsetbuttcap%
\pgfsetroundjoin%
\pgfsetlinewidth{1.003750pt}%
\definecolor{currentstroke}{rgb}{0.000000,0.000000,0.000000}%
\pgfsetstrokecolor{currentstroke}%
\pgfsetdash{}{0pt}%
\pgfusepath{stroke}%
\end{pgfscope}%
\begin{pgfscope}%
\pgfpathrectangle{\pgfqpoint{0.050000in}{0.050000in}}{\pgfqpoint{2.419000in}{2.419000in}}%
\pgfusepath{clip}%
\pgfsetbuttcap%
\pgfsetroundjoin%
\pgfsetlinewidth{1.003750pt}%
\definecolor{currentstroke}{rgb}{0.000000,0.000000,0.000000}%
\pgfsetstrokecolor{currentstroke}%
\pgfsetdash{}{0pt}%
\pgfusepath{stroke}%
\end{pgfscope}%
\begin{pgfscope}%
\pgfpathrectangle{\pgfqpoint{0.050000in}{0.050000in}}{\pgfqpoint{2.419000in}{2.419000in}}%
\pgfusepath{clip}%
\pgfsetbuttcap%
\pgfsetroundjoin%
\pgfsetlinewidth{1.003750pt}%
\definecolor{currentstroke}{rgb}{0.000000,0.000000,0.000000}%
\pgfsetstrokecolor{currentstroke}%
\pgfsetdash{}{0pt}%
\pgfusepath{stroke}%
\end{pgfscope}%
\begin{pgfscope}%
\pgfpathrectangle{\pgfqpoint{0.050000in}{0.050000in}}{\pgfqpoint{2.419000in}{2.419000in}}%
\pgfusepath{clip}%
\pgfsetbuttcap%
\pgfsetroundjoin%
\pgfsetlinewidth{1.003750pt}%
\definecolor{currentstroke}{rgb}{0.000000,0.000000,0.000000}%
\pgfsetstrokecolor{currentstroke}%
\pgfsetdash{}{0pt}%
\pgfusepath{stroke}%
\end{pgfscope}%
\begin{pgfscope}%
\pgfpathrectangle{\pgfqpoint{0.050000in}{0.050000in}}{\pgfqpoint{2.419000in}{2.419000in}}%
\pgfusepath{clip}%
\pgfsetbuttcap%
\pgfsetroundjoin%
\pgfsetlinewidth{1.003750pt}%
\definecolor{currentstroke}{rgb}{0.000000,0.000000,0.000000}%
\pgfsetstrokecolor{currentstroke}%
\pgfsetdash{}{0pt}%
\pgfusepath{stroke}%
\end{pgfscope}%
\begin{pgfscope}%
\pgfpathrectangle{\pgfqpoint{0.050000in}{0.050000in}}{\pgfqpoint{2.419000in}{2.419000in}}%
\pgfusepath{clip}%
\pgfsetbuttcap%
\pgfsetroundjoin%
\pgfsetlinewidth{1.003750pt}%
\definecolor{currentstroke}{rgb}{0.000000,0.000000,0.000000}%
\pgfsetstrokecolor{currentstroke}%
\pgfsetdash{}{0pt}%
\pgfusepath{stroke}%
\end{pgfscope}%
\begin{pgfscope}%
\pgfpathrectangle{\pgfqpoint{0.050000in}{0.050000in}}{\pgfqpoint{2.419000in}{2.419000in}}%
\pgfusepath{clip}%
\pgfsetbuttcap%
\pgfsetroundjoin%
\pgfsetlinewidth{1.003750pt}%
\definecolor{currentstroke}{rgb}{0.000000,0.000000,0.000000}%
\pgfsetstrokecolor{currentstroke}%
\pgfsetdash{}{0pt}%
\pgfusepath{stroke}%
\end{pgfscope}%
\begin{pgfscope}%
\pgfpathrectangle{\pgfqpoint{0.050000in}{0.050000in}}{\pgfqpoint{2.419000in}{2.419000in}}%
\pgfusepath{clip}%
\pgfsetbuttcap%
\pgfsetroundjoin%
\pgfsetlinewidth{1.003750pt}%
\definecolor{currentstroke}{rgb}{0.000000,0.000000,0.000000}%
\pgfsetstrokecolor{currentstroke}%
\pgfsetdash{}{0pt}%
\pgfusepath{stroke}%
\end{pgfscope}%
\begin{pgfscope}%
\pgfpathrectangle{\pgfqpoint{0.050000in}{0.050000in}}{\pgfqpoint{2.419000in}{2.419000in}}%
\pgfusepath{clip}%
\pgfsetbuttcap%
\pgfsetroundjoin%
\pgfsetlinewidth{1.003750pt}%
\definecolor{currentstroke}{rgb}{0.000000,0.000000,0.000000}%
\pgfsetstrokecolor{currentstroke}%
\pgfsetdash{}{0pt}%
\pgfusepath{stroke}%
\end{pgfscope}%
\begin{pgfscope}%
\pgfpathrectangle{\pgfqpoint{0.050000in}{0.050000in}}{\pgfqpoint{2.419000in}{2.419000in}}%
\pgfusepath{clip}%
\pgfsetbuttcap%
\pgfsetroundjoin%
\pgfsetlinewidth{1.003750pt}%
\definecolor{currentstroke}{rgb}{0.000000,0.000000,0.000000}%
\pgfsetstrokecolor{currentstroke}%
\pgfsetdash{}{0pt}%
\pgfusepath{stroke}%
\end{pgfscope}%
\begin{pgfscope}%
\pgfpathrectangle{\pgfqpoint{0.050000in}{0.050000in}}{\pgfqpoint{2.419000in}{2.419000in}}%
\pgfusepath{clip}%
\pgfsetbuttcap%
\pgfsetroundjoin%
\pgfsetlinewidth{1.003750pt}%
\definecolor{currentstroke}{rgb}{0.000000,0.000000,0.000000}%
\pgfsetstrokecolor{currentstroke}%
\pgfsetdash{}{0pt}%
\pgfusepath{stroke}%
\end{pgfscope}%
\begin{pgfscope}%
\pgfpathrectangle{\pgfqpoint{0.050000in}{0.050000in}}{\pgfqpoint{2.419000in}{2.419000in}}%
\pgfusepath{clip}%
\pgfsetbuttcap%
\pgfsetroundjoin%
\pgfsetlinewidth{1.003750pt}%
\definecolor{currentstroke}{rgb}{0.000000,0.000000,0.000000}%
\pgfsetstrokecolor{currentstroke}%
\pgfsetdash{}{0pt}%
\pgfusepath{stroke}%
\end{pgfscope}%
\begin{pgfscope}%
\pgfpathrectangle{\pgfqpoint{0.050000in}{0.050000in}}{\pgfqpoint{2.419000in}{2.419000in}}%
\pgfusepath{clip}%
\pgfsetbuttcap%
\pgfsetroundjoin%
\pgfsetlinewidth{1.003750pt}%
\definecolor{currentstroke}{rgb}{0.000000,0.000000,0.000000}%
\pgfsetstrokecolor{currentstroke}%
\pgfsetdash{}{0pt}%
\pgfusepath{stroke}%
\end{pgfscope}%
\begin{pgfscope}%
\pgfpathrectangle{\pgfqpoint{0.050000in}{0.050000in}}{\pgfqpoint{2.419000in}{2.419000in}}%
\pgfusepath{clip}%
\pgfsetbuttcap%
\pgfsetroundjoin%
\pgfsetlinewidth{1.003750pt}%
\definecolor{currentstroke}{rgb}{0.000000,0.000000,0.000000}%
\pgfsetstrokecolor{currentstroke}%
\pgfsetdash{}{0pt}%
\pgfusepath{stroke}%
\end{pgfscope}%
\begin{pgfscope}%
\pgfpathrectangle{\pgfqpoint{0.050000in}{0.050000in}}{\pgfqpoint{2.419000in}{2.419000in}}%
\pgfusepath{clip}%
\pgfsetbuttcap%
\pgfsetroundjoin%
\pgfsetlinewidth{1.003750pt}%
\definecolor{currentstroke}{rgb}{0.000000,0.000000,0.000000}%
\pgfsetstrokecolor{currentstroke}%
\pgfsetdash{}{0pt}%
\pgfusepath{stroke}%
\end{pgfscope}%
\begin{pgfscope}%
\pgfpathrectangle{\pgfqpoint{0.050000in}{0.050000in}}{\pgfqpoint{2.419000in}{2.419000in}}%
\pgfusepath{clip}%
\pgfsetbuttcap%
\pgfsetroundjoin%
\pgfsetlinewidth{1.003750pt}%
\definecolor{currentstroke}{rgb}{0.000000,0.000000,0.000000}%
\pgfsetstrokecolor{currentstroke}%
\pgfsetdash{}{0pt}%
\pgfusepath{stroke}%
\end{pgfscope}%
\begin{pgfscope}%
\pgfpathrectangle{\pgfqpoint{0.050000in}{0.050000in}}{\pgfqpoint{2.419000in}{2.419000in}}%
\pgfusepath{clip}%
\pgfsetbuttcap%
\pgfsetroundjoin%
\pgfsetlinewidth{1.003750pt}%
\definecolor{currentstroke}{rgb}{0.000000,0.000000,0.000000}%
\pgfsetstrokecolor{currentstroke}%
\pgfsetdash{}{0pt}%
\pgfusepath{stroke}%
\end{pgfscope}%
\begin{pgfscope}%
\pgfpathrectangle{\pgfqpoint{0.050000in}{0.050000in}}{\pgfqpoint{2.419000in}{2.419000in}}%
\pgfusepath{clip}%
\pgfsetbuttcap%
\pgfsetroundjoin%
\pgfsetlinewidth{1.003750pt}%
\definecolor{currentstroke}{rgb}{0.000000,0.000000,0.000000}%
\pgfsetstrokecolor{currentstroke}%
\pgfsetdash{}{0pt}%
\pgfusepath{stroke}%
\end{pgfscope}%
\begin{pgfscope}%
\pgfpathrectangle{\pgfqpoint{0.050000in}{0.050000in}}{\pgfqpoint{2.419000in}{2.419000in}}%
\pgfusepath{clip}%
\pgfsetbuttcap%
\pgfsetroundjoin%
\pgfsetlinewidth{1.003750pt}%
\definecolor{currentstroke}{rgb}{0.000000,0.000000,0.000000}%
\pgfsetstrokecolor{currentstroke}%
\pgfsetdash{}{0pt}%
\pgfusepath{stroke}%
\end{pgfscope}%
\begin{pgfscope}%
\pgfpathrectangle{\pgfqpoint{0.050000in}{0.050000in}}{\pgfqpoint{2.419000in}{2.419000in}}%
\pgfusepath{clip}%
\pgfsetbuttcap%
\pgfsetroundjoin%
\pgfsetlinewidth{1.003750pt}%
\definecolor{currentstroke}{rgb}{0.000000,0.000000,0.000000}%
\pgfsetstrokecolor{currentstroke}%
\pgfsetdash{}{0pt}%
\pgfusepath{stroke}%
\end{pgfscope}%
\begin{pgfscope}%
\pgfpathrectangle{\pgfqpoint{0.050000in}{0.050000in}}{\pgfqpoint{2.419000in}{2.419000in}}%
\pgfusepath{clip}%
\pgfsetbuttcap%
\pgfsetroundjoin%
\pgfsetlinewidth{1.003750pt}%
\definecolor{currentstroke}{rgb}{0.000000,0.000000,0.000000}%
\pgfsetstrokecolor{currentstroke}%
\pgfsetdash{}{0pt}%
\pgfusepath{stroke}%
\end{pgfscope}%
\begin{pgfscope}%
\pgfpathrectangle{\pgfqpoint{0.050000in}{0.050000in}}{\pgfqpoint{2.419000in}{2.419000in}}%
\pgfusepath{clip}%
\pgfsetbuttcap%
\pgfsetroundjoin%
\pgfsetlinewidth{1.003750pt}%
\definecolor{currentstroke}{rgb}{0.000000,0.000000,0.000000}%
\pgfsetstrokecolor{currentstroke}%
\pgfsetdash{}{0pt}%
\pgfusepath{stroke}%
\end{pgfscope}%
\begin{pgfscope}%
\pgfpathrectangle{\pgfqpoint{0.050000in}{0.050000in}}{\pgfqpoint{2.419000in}{2.419000in}}%
\pgfusepath{clip}%
\pgfsetbuttcap%
\pgfsetroundjoin%
\pgfsetlinewidth{1.003750pt}%
\definecolor{currentstroke}{rgb}{0.000000,0.000000,0.000000}%
\pgfsetstrokecolor{currentstroke}%
\pgfsetdash{}{0pt}%
\pgfusepath{stroke}%
\end{pgfscope}%
\begin{pgfscope}%
\pgfpathrectangle{\pgfqpoint{0.050000in}{0.050000in}}{\pgfqpoint{2.419000in}{2.419000in}}%
\pgfusepath{clip}%
\pgfsetbuttcap%
\pgfsetroundjoin%
\pgfsetlinewidth{1.003750pt}%
\definecolor{currentstroke}{rgb}{0.000000,0.000000,0.000000}%
\pgfsetstrokecolor{currentstroke}%
\pgfsetdash{}{0pt}%
\pgfusepath{stroke}%
\end{pgfscope}%
\begin{pgfscope}%
\pgfpathrectangle{\pgfqpoint{0.050000in}{0.050000in}}{\pgfqpoint{2.419000in}{2.419000in}}%
\pgfusepath{clip}%
\pgfsetbuttcap%
\pgfsetroundjoin%
\pgfsetlinewidth{1.003750pt}%
\definecolor{currentstroke}{rgb}{0.000000,0.000000,0.000000}%
\pgfsetstrokecolor{currentstroke}%
\pgfsetdash{}{0pt}%
\pgfusepath{stroke}%
\end{pgfscope}%
\begin{pgfscope}%
\pgfpathrectangle{\pgfqpoint{0.050000in}{0.050000in}}{\pgfqpoint{2.419000in}{2.419000in}}%
\pgfusepath{clip}%
\pgfsetbuttcap%
\pgfsetroundjoin%
\pgfsetlinewidth{1.003750pt}%
\definecolor{currentstroke}{rgb}{0.000000,0.000000,0.000000}%
\pgfsetstrokecolor{currentstroke}%
\pgfsetdash{}{0pt}%
\pgfusepath{stroke}%
\end{pgfscope}%
\begin{pgfscope}%
\pgfpathrectangle{\pgfqpoint{0.050000in}{0.050000in}}{\pgfqpoint{2.419000in}{2.419000in}}%
\pgfusepath{clip}%
\pgfsetbuttcap%
\pgfsetroundjoin%
\pgfsetlinewidth{1.003750pt}%
\definecolor{currentstroke}{rgb}{0.000000,0.000000,0.000000}%
\pgfsetstrokecolor{currentstroke}%
\pgfsetdash{}{0pt}%
\pgfusepath{stroke}%
\end{pgfscope}%
\begin{pgfscope}%
\pgfpathrectangle{\pgfqpoint{0.050000in}{0.050000in}}{\pgfqpoint{2.419000in}{2.419000in}}%
\pgfusepath{clip}%
\pgfsetbuttcap%
\pgfsetroundjoin%
\pgfsetlinewidth{1.003750pt}%
\definecolor{currentstroke}{rgb}{0.000000,0.000000,0.000000}%
\pgfsetstrokecolor{currentstroke}%
\pgfsetdash{}{0pt}%
\pgfusepath{stroke}%
\end{pgfscope}%
\begin{pgfscope}%
\pgfpathrectangle{\pgfqpoint{0.050000in}{0.050000in}}{\pgfqpoint{2.419000in}{2.419000in}}%
\pgfusepath{clip}%
\pgfsetbuttcap%
\pgfsetroundjoin%
\pgfsetlinewidth{1.003750pt}%
\definecolor{currentstroke}{rgb}{0.000000,0.000000,0.000000}%
\pgfsetstrokecolor{currentstroke}%
\pgfsetdash{}{0pt}%
\pgfusepath{stroke}%
\end{pgfscope}%
\begin{pgfscope}%
\pgfpathrectangle{\pgfqpoint{0.050000in}{0.050000in}}{\pgfqpoint{2.419000in}{2.419000in}}%
\pgfusepath{clip}%
\pgfsetbuttcap%
\pgfsetroundjoin%
\pgfsetlinewidth{1.003750pt}%
\definecolor{currentstroke}{rgb}{0.000000,0.000000,0.000000}%
\pgfsetstrokecolor{currentstroke}%
\pgfsetdash{}{0pt}%
\pgfusepath{stroke}%
\end{pgfscope}%
\begin{pgfscope}%
\pgfpathrectangle{\pgfqpoint{0.050000in}{0.050000in}}{\pgfqpoint{2.419000in}{2.419000in}}%
\pgfusepath{clip}%
\pgfsetbuttcap%
\pgfsetroundjoin%
\pgfsetlinewidth{1.003750pt}%
\definecolor{currentstroke}{rgb}{0.000000,0.000000,0.000000}%
\pgfsetstrokecolor{currentstroke}%
\pgfsetdash{}{0pt}%
\pgfusepath{stroke}%
\end{pgfscope}%
\begin{pgfscope}%
\pgfpathrectangle{\pgfqpoint{0.050000in}{0.050000in}}{\pgfqpoint{2.419000in}{2.419000in}}%
\pgfusepath{clip}%
\pgfsetbuttcap%
\pgfsetroundjoin%
\pgfsetlinewidth{1.003750pt}%
\definecolor{currentstroke}{rgb}{0.000000,0.000000,0.000000}%
\pgfsetstrokecolor{currentstroke}%
\pgfsetdash{}{0pt}%
\pgfusepath{stroke}%
\end{pgfscope}%
\begin{pgfscope}%
\pgfpathrectangle{\pgfqpoint{0.050000in}{0.050000in}}{\pgfqpoint{2.419000in}{2.419000in}}%
\pgfusepath{clip}%
\pgfsetbuttcap%
\pgfsetroundjoin%
\pgfsetlinewidth{1.003750pt}%
\definecolor{currentstroke}{rgb}{0.000000,0.000000,0.000000}%
\pgfsetstrokecolor{currentstroke}%
\pgfsetdash{}{0pt}%
\pgfusepath{stroke}%
\end{pgfscope}%
\begin{pgfscope}%
\pgfpathrectangle{\pgfqpoint{0.050000in}{0.050000in}}{\pgfqpoint{2.419000in}{2.419000in}}%
\pgfusepath{clip}%
\pgfsetbuttcap%
\pgfsetroundjoin%
\pgfsetlinewidth{1.003750pt}%
\definecolor{currentstroke}{rgb}{0.000000,0.000000,0.000000}%
\pgfsetstrokecolor{currentstroke}%
\pgfsetdash{}{0pt}%
\pgfusepath{stroke}%
\end{pgfscope}%
\begin{pgfscope}%
\pgfpathrectangle{\pgfqpoint{0.050000in}{0.050000in}}{\pgfqpoint{2.419000in}{2.419000in}}%
\pgfusepath{clip}%
\pgfsetbuttcap%
\pgfsetroundjoin%
\pgfsetlinewidth{1.003750pt}%
\definecolor{currentstroke}{rgb}{0.000000,0.000000,0.000000}%
\pgfsetstrokecolor{currentstroke}%
\pgfsetdash{}{0pt}%
\pgfusepath{stroke}%
\end{pgfscope}%
\begin{pgfscope}%
\pgfpathrectangle{\pgfqpoint{0.050000in}{0.050000in}}{\pgfqpoint{2.419000in}{2.419000in}}%
\pgfusepath{clip}%
\pgfsetbuttcap%
\pgfsetroundjoin%
\pgfsetlinewidth{1.003750pt}%
\definecolor{currentstroke}{rgb}{0.000000,0.000000,0.000000}%
\pgfsetstrokecolor{currentstroke}%
\pgfsetdash{}{0pt}%
\pgfusepath{stroke}%
\end{pgfscope}%
\begin{pgfscope}%
\pgfpathrectangle{\pgfqpoint{0.050000in}{0.050000in}}{\pgfqpoint{2.419000in}{2.419000in}}%
\pgfusepath{clip}%
\pgfsetbuttcap%
\pgfsetroundjoin%
\pgfsetlinewidth{1.003750pt}%
\definecolor{currentstroke}{rgb}{0.000000,0.000000,0.000000}%
\pgfsetstrokecolor{currentstroke}%
\pgfsetdash{}{0pt}%
\pgfusepath{stroke}%
\end{pgfscope}%
\begin{pgfscope}%
\pgfpathrectangle{\pgfqpoint{0.050000in}{0.050000in}}{\pgfqpoint{2.419000in}{2.419000in}}%
\pgfusepath{clip}%
\pgfsetbuttcap%
\pgfsetroundjoin%
\pgfsetlinewidth{1.003750pt}%
\definecolor{currentstroke}{rgb}{0.000000,0.000000,0.000000}%
\pgfsetstrokecolor{currentstroke}%
\pgfsetdash{}{0pt}%
\pgfusepath{stroke}%
\end{pgfscope}%
\begin{pgfscope}%
\pgfpathrectangle{\pgfqpoint{0.050000in}{0.050000in}}{\pgfqpoint{2.419000in}{2.419000in}}%
\pgfusepath{clip}%
\pgfsetbuttcap%
\pgfsetroundjoin%
\pgfsetlinewidth{1.003750pt}%
\definecolor{currentstroke}{rgb}{0.000000,0.000000,0.000000}%
\pgfsetstrokecolor{currentstroke}%
\pgfsetdash{}{0pt}%
\pgfusepath{stroke}%
\end{pgfscope}%
\begin{pgfscope}%
\pgfpathrectangle{\pgfqpoint{0.050000in}{0.050000in}}{\pgfqpoint{2.419000in}{2.419000in}}%
\pgfusepath{clip}%
\pgfsetbuttcap%
\pgfsetroundjoin%
\pgfsetlinewidth{1.003750pt}%
\definecolor{currentstroke}{rgb}{0.000000,0.000000,0.000000}%
\pgfsetstrokecolor{currentstroke}%
\pgfsetdash{}{0pt}%
\pgfusepath{stroke}%
\end{pgfscope}%
\begin{pgfscope}%
\pgfpathrectangle{\pgfqpoint{0.050000in}{0.050000in}}{\pgfqpoint{2.419000in}{2.419000in}}%
\pgfusepath{clip}%
\pgfsetbuttcap%
\pgfsetroundjoin%
\pgfsetlinewidth{1.003750pt}%
\definecolor{currentstroke}{rgb}{0.000000,0.000000,0.000000}%
\pgfsetstrokecolor{currentstroke}%
\pgfsetdash{}{0pt}%
\pgfusepath{stroke}%
\end{pgfscope}%
\begin{pgfscope}%
\pgfpathrectangle{\pgfqpoint{0.050000in}{0.050000in}}{\pgfqpoint{2.419000in}{2.419000in}}%
\pgfusepath{clip}%
\pgfsetbuttcap%
\pgfsetroundjoin%
\pgfsetlinewidth{1.003750pt}%
\definecolor{currentstroke}{rgb}{0.000000,0.000000,0.000000}%
\pgfsetstrokecolor{currentstroke}%
\pgfsetdash{}{0pt}%
\pgfusepath{stroke}%
\end{pgfscope}%
\begin{pgfscope}%
\pgfpathrectangle{\pgfqpoint{0.050000in}{0.050000in}}{\pgfqpoint{2.419000in}{2.419000in}}%
\pgfusepath{clip}%
\pgfsetbuttcap%
\pgfsetroundjoin%
\pgfsetlinewidth{1.003750pt}%
\definecolor{currentstroke}{rgb}{0.000000,0.000000,0.000000}%
\pgfsetstrokecolor{currentstroke}%
\pgfsetdash{}{0pt}%
\pgfusepath{stroke}%
\end{pgfscope}%
\begin{pgfscope}%
\pgfpathrectangle{\pgfqpoint{0.050000in}{0.050000in}}{\pgfqpoint{2.419000in}{2.419000in}}%
\pgfusepath{clip}%
\pgfsetbuttcap%
\pgfsetroundjoin%
\pgfsetlinewidth{1.003750pt}%
\definecolor{currentstroke}{rgb}{0.000000,0.000000,0.000000}%
\pgfsetstrokecolor{currentstroke}%
\pgfsetdash{}{0pt}%
\pgfusepath{stroke}%
\end{pgfscope}%
\begin{pgfscope}%
\pgfpathrectangle{\pgfqpoint{0.050000in}{0.050000in}}{\pgfqpoint{2.419000in}{2.419000in}}%
\pgfusepath{clip}%
\pgfsetbuttcap%
\pgfsetroundjoin%
\pgfsetlinewidth{1.003750pt}%
\definecolor{currentstroke}{rgb}{0.000000,0.000000,0.000000}%
\pgfsetstrokecolor{currentstroke}%
\pgfsetdash{}{0pt}%
\pgfusepath{stroke}%
\end{pgfscope}%
\begin{pgfscope}%
\pgfpathrectangle{\pgfqpoint{0.050000in}{0.050000in}}{\pgfqpoint{2.419000in}{2.419000in}}%
\pgfusepath{clip}%
\pgfsetbuttcap%
\pgfsetroundjoin%
\pgfsetlinewidth{1.003750pt}%
\definecolor{currentstroke}{rgb}{0.000000,0.000000,0.000000}%
\pgfsetstrokecolor{currentstroke}%
\pgfsetdash{}{0pt}%
\pgfusepath{stroke}%
\end{pgfscope}%
\begin{pgfscope}%
\pgfpathrectangle{\pgfqpoint{0.050000in}{0.050000in}}{\pgfqpoint{2.419000in}{2.419000in}}%
\pgfusepath{clip}%
\pgfsetbuttcap%
\pgfsetroundjoin%
\pgfsetlinewidth{1.003750pt}%
\definecolor{currentstroke}{rgb}{0.000000,0.000000,0.000000}%
\pgfsetstrokecolor{currentstroke}%
\pgfsetdash{}{0pt}%
\pgfusepath{stroke}%
\end{pgfscope}%
\begin{pgfscope}%
\pgfpathrectangle{\pgfqpoint{0.050000in}{0.050000in}}{\pgfqpoint{2.419000in}{2.419000in}}%
\pgfusepath{clip}%
\pgfsetbuttcap%
\pgfsetroundjoin%
\pgfsetlinewidth{1.003750pt}%
\definecolor{currentstroke}{rgb}{0.000000,0.000000,0.000000}%
\pgfsetstrokecolor{currentstroke}%
\pgfsetdash{}{0pt}%
\pgfusepath{stroke}%
\end{pgfscope}%
\begin{pgfscope}%
\pgfpathrectangle{\pgfqpoint{0.050000in}{0.050000in}}{\pgfqpoint{2.419000in}{2.419000in}}%
\pgfusepath{clip}%
\pgfsetbuttcap%
\pgfsetroundjoin%
\pgfsetlinewidth{1.003750pt}%
\definecolor{currentstroke}{rgb}{0.000000,0.000000,0.000000}%
\pgfsetstrokecolor{currentstroke}%
\pgfsetdash{}{0pt}%
\pgfusepath{stroke}%
\end{pgfscope}%
\begin{pgfscope}%
\pgfpathrectangle{\pgfqpoint{0.050000in}{0.050000in}}{\pgfqpoint{2.419000in}{2.419000in}}%
\pgfusepath{clip}%
\pgfsetbuttcap%
\pgfsetroundjoin%
\pgfsetlinewidth{1.003750pt}%
\definecolor{currentstroke}{rgb}{0.000000,0.000000,0.000000}%
\pgfsetstrokecolor{currentstroke}%
\pgfsetdash{}{0pt}%
\pgfusepath{stroke}%
\end{pgfscope}%
\begin{pgfscope}%
\pgfpathrectangle{\pgfqpoint{0.050000in}{0.050000in}}{\pgfqpoint{2.419000in}{2.419000in}}%
\pgfusepath{clip}%
\pgfsetbuttcap%
\pgfsetroundjoin%
\pgfsetlinewidth{1.003750pt}%
\definecolor{currentstroke}{rgb}{0.000000,0.000000,0.000000}%
\pgfsetstrokecolor{currentstroke}%
\pgfsetdash{}{0pt}%
\pgfusepath{stroke}%
\end{pgfscope}%
\begin{pgfscope}%
\pgfpathrectangle{\pgfqpoint{0.050000in}{0.050000in}}{\pgfqpoint{2.419000in}{2.419000in}}%
\pgfusepath{clip}%
\pgfsetbuttcap%
\pgfsetroundjoin%
\pgfsetlinewidth{1.003750pt}%
\definecolor{currentstroke}{rgb}{0.000000,0.000000,0.000000}%
\pgfsetstrokecolor{currentstroke}%
\pgfsetdash{}{0pt}%
\pgfusepath{stroke}%
\end{pgfscope}%
\begin{pgfscope}%
\pgfpathrectangle{\pgfqpoint{0.050000in}{0.050000in}}{\pgfqpoint{2.419000in}{2.419000in}}%
\pgfusepath{clip}%
\pgfsetbuttcap%
\pgfsetroundjoin%
\pgfsetlinewidth{1.003750pt}%
\definecolor{currentstroke}{rgb}{0.000000,0.000000,0.000000}%
\pgfsetstrokecolor{currentstroke}%
\pgfsetdash{}{0pt}%
\pgfusepath{stroke}%
\end{pgfscope}%
\begin{pgfscope}%
\pgfpathrectangle{\pgfqpoint{0.050000in}{0.050000in}}{\pgfqpoint{2.419000in}{2.419000in}}%
\pgfusepath{clip}%
\pgfsetbuttcap%
\pgfsetroundjoin%
\pgfsetlinewidth{1.003750pt}%
\definecolor{currentstroke}{rgb}{0.000000,0.000000,0.000000}%
\pgfsetstrokecolor{currentstroke}%
\pgfsetdash{}{0pt}%
\pgfusepath{stroke}%
\end{pgfscope}%
\begin{pgfscope}%
\pgfpathrectangle{\pgfqpoint{0.050000in}{0.050000in}}{\pgfqpoint{2.419000in}{2.419000in}}%
\pgfusepath{clip}%
\pgfsetbuttcap%
\pgfsetroundjoin%
\pgfsetlinewidth{1.003750pt}%
\definecolor{currentstroke}{rgb}{0.000000,0.000000,0.000000}%
\pgfsetstrokecolor{currentstroke}%
\pgfsetdash{}{0pt}%
\pgfusepath{stroke}%
\end{pgfscope}%
\begin{pgfscope}%
\pgfpathrectangle{\pgfqpoint{0.050000in}{0.050000in}}{\pgfqpoint{2.419000in}{2.419000in}}%
\pgfusepath{clip}%
\pgfsetbuttcap%
\pgfsetroundjoin%
\pgfsetlinewidth{1.003750pt}%
\definecolor{currentstroke}{rgb}{0.000000,0.000000,0.000000}%
\pgfsetstrokecolor{currentstroke}%
\pgfsetdash{}{0pt}%
\pgfusepath{stroke}%
\end{pgfscope}%
\begin{pgfscope}%
\pgfpathrectangle{\pgfqpoint{0.050000in}{0.050000in}}{\pgfqpoint{2.419000in}{2.419000in}}%
\pgfusepath{clip}%
\pgfsetbuttcap%
\pgfsetroundjoin%
\pgfsetlinewidth{1.003750pt}%
\definecolor{currentstroke}{rgb}{0.000000,0.000000,0.000000}%
\pgfsetstrokecolor{currentstroke}%
\pgfsetdash{}{0pt}%
\pgfusepath{stroke}%
\end{pgfscope}%
\begin{pgfscope}%
\pgfpathrectangle{\pgfqpoint{0.050000in}{0.050000in}}{\pgfqpoint{2.419000in}{2.419000in}}%
\pgfusepath{clip}%
\pgfsetbuttcap%
\pgfsetroundjoin%
\pgfsetlinewidth{1.003750pt}%
\definecolor{currentstroke}{rgb}{0.000000,0.000000,0.000000}%
\pgfsetstrokecolor{currentstroke}%
\pgfsetdash{}{0pt}%
\pgfusepath{stroke}%
\end{pgfscope}%
\begin{pgfscope}%
\pgfpathrectangle{\pgfqpoint{0.050000in}{0.050000in}}{\pgfqpoint{2.419000in}{2.419000in}}%
\pgfusepath{clip}%
\pgfsetbuttcap%
\pgfsetroundjoin%
\pgfsetlinewidth{1.003750pt}%
\definecolor{currentstroke}{rgb}{0.000000,0.000000,0.000000}%
\pgfsetstrokecolor{currentstroke}%
\pgfsetdash{}{0pt}%
\pgfusepath{stroke}%
\end{pgfscope}%
\begin{pgfscope}%
\pgfpathrectangle{\pgfqpoint{0.050000in}{0.050000in}}{\pgfqpoint{2.419000in}{2.419000in}}%
\pgfusepath{clip}%
\pgfsetbuttcap%
\pgfsetroundjoin%
\pgfsetlinewidth{1.003750pt}%
\definecolor{currentstroke}{rgb}{0.000000,0.000000,0.000000}%
\pgfsetstrokecolor{currentstroke}%
\pgfsetdash{}{0pt}%
\pgfusepath{stroke}%
\end{pgfscope}%
\begin{pgfscope}%
\pgfpathrectangle{\pgfqpoint{0.050000in}{0.050000in}}{\pgfqpoint{2.419000in}{2.419000in}}%
\pgfusepath{clip}%
\pgfsetbuttcap%
\pgfsetroundjoin%
\pgfsetlinewidth{1.003750pt}%
\definecolor{currentstroke}{rgb}{0.000000,0.000000,0.000000}%
\pgfsetstrokecolor{currentstroke}%
\pgfsetdash{}{0pt}%
\pgfusepath{stroke}%
\end{pgfscope}%
\begin{pgfscope}%
\pgfpathrectangle{\pgfqpoint{0.050000in}{0.050000in}}{\pgfqpoint{2.419000in}{2.419000in}}%
\pgfusepath{clip}%
\pgfsetbuttcap%
\pgfsetroundjoin%
\pgfsetlinewidth{1.003750pt}%
\definecolor{currentstroke}{rgb}{0.000000,0.000000,0.000000}%
\pgfsetstrokecolor{currentstroke}%
\pgfsetdash{}{0pt}%
\pgfusepath{stroke}%
\end{pgfscope}%
\begin{pgfscope}%
\pgfpathrectangle{\pgfqpoint{0.050000in}{0.050000in}}{\pgfqpoint{2.419000in}{2.419000in}}%
\pgfusepath{clip}%
\pgfsetbuttcap%
\pgfsetroundjoin%
\pgfsetlinewidth{1.003750pt}%
\definecolor{currentstroke}{rgb}{0.000000,0.000000,0.000000}%
\pgfsetstrokecolor{currentstroke}%
\pgfsetdash{}{0pt}%
\pgfusepath{stroke}%
\end{pgfscope}%
\begin{pgfscope}%
\pgfpathrectangle{\pgfqpoint{0.050000in}{0.050000in}}{\pgfqpoint{2.419000in}{2.419000in}}%
\pgfusepath{clip}%
\pgfsetbuttcap%
\pgfsetroundjoin%
\pgfsetlinewidth{1.003750pt}%
\definecolor{currentstroke}{rgb}{0.000000,0.000000,0.000000}%
\pgfsetstrokecolor{currentstroke}%
\pgfsetdash{}{0pt}%
\pgfusepath{stroke}%
\end{pgfscope}%
\begin{pgfscope}%
\pgfpathrectangle{\pgfqpoint{0.050000in}{0.050000in}}{\pgfqpoint{2.419000in}{2.419000in}}%
\pgfusepath{clip}%
\pgfsetbuttcap%
\pgfsetroundjoin%
\pgfsetlinewidth{1.003750pt}%
\definecolor{currentstroke}{rgb}{0.000000,0.000000,0.000000}%
\pgfsetstrokecolor{currentstroke}%
\pgfsetdash{}{0pt}%
\pgfusepath{stroke}%
\end{pgfscope}%
\begin{pgfscope}%
\pgfpathrectangle{\pgfqpoint{0.050000in}{0.050000in}}{\pgfqpoint{2.419000in}{2.419000in}}%
\pgfusepath{clip}%
\pgfsetbuttcap%
\pgfsetroundjoin%
\pgfsetlinewidth{1.003750pt}%
\definecolor{currentstroke}{rgb}{0.000000,0.000000,0.000000}%
\pgfsetstrokecolor{currentstroke}%
\pgfsetdash{}{0pt}%
\pgfusepath{stroke}%
\end{pgfscope}%
\begin{pgfscope}%
\pgfpathrectangle{\pgfqpoint{0.050000in}{0.050000in}}{\pgfqpoint{2.419000in}{2.419000in}}%
\pgfusepath{clip}%
\pgfsetbuttcap%
\pgfsetroundjoin%
\pgfsetlinewidth{1.003750pt}%
\definecolor{currentstroke}{rgb}{0.000000,0.000000,0.000000}%
\pgfsetstrokecolor{currentstroke}%
\pgfsetdash{}{0pt}%
\pgfusepath{stroke}%
\end{pgfscope}%
\begin{pgfscope}%
\pgfpathrectangle{\pgfqpoint{0.050000in}{0.050000in}}{\pgfqpoint{2.419000in}{2.419000in}}%
\pgfusepath{clip}%
\pgfsetbuttcap%
\pgfsetroundjoin%
\pgfsetlinewidth{1.003750pt}%
\definecolor{currentstroke}{rgb}{0.000000,0.000000,0.000000}%
\pgfsetstrokecolor{currentstroke}%
\pgfsetdash{}{0pt}%
\pgfusepath{stroke}%
\end{pgfscope}%
\begin{pgfscope}%
\pgfpathrectangle{\pgfqpoint{0.050000in}{0.050000in}}{\pgfqpoint{2.419000in}{2.419000in}}%
\pgfusepath{clip}%
\pgfsetbuttcap%
\pgfsetroundjoin%
\pgfsetlinewidth{1.003750pt}%
\definecolor{currentstroke}{rgb}{0.000000,0.000000,0.000000}%
\pgfsetstrokecolor{currentstroke}%
\pgfsetdash{}{0pt}%
\pgfusepath{stroke}%
\end{pgfscope}%
\begin{pgfscope}%
\pgfpathrectangle{\pgfqpoint{0.050000in}{0.050000in}}{\pgfqpoint{2.419000in}{2.419000in}}%
\pgfusepath{clip}%
\pgfsetbuttcap%
\pgfsetroundjoin%
\pgfsetlinewidth{1.003750pt}%
\definecolor{currentstroke}{rgb}{0.000000,0.000000,0.000000}%
\pgfsetstrokecolor{currentstroke}%
\pgfsetdash{}{0pt}%
\pgfusepath{stroke}%
\end{pgfscope}%
\begin{pgfscope}%
\pgfpathrectangle{\pgfqpoint{0.050000in}{0.050000in}}{\pgfqpoint{2.419000in}{2.419000in}}%
\pgfusepath{clip}%
\pgfsetbuttcap%
\pgfsetroundjoin%
\pgfsetlinewidth{1.003750pt}%
\definecolor{currentstroke}{rgb}{0.000000,0.000000,0.000000}%
\pgfsetstrokecolor{currentstroke}%
\pgfsetdash{}{0pt}%
\pgfusepath{stroke}%
\end{pgfscope}%
\begin{pgfscope}%
\pgfpathrectangle{\pgfqpoint{0.050000in}{0.050000in}}{\pgfqpoint{2.419000in}{2.419000in}}%
\pgfusepath{clip}%
\pgfsetbuttcap%
\pgfsetroundjoin%
\pgfsetlinewidth{1.003750pt}%
\definecolor{currentstroke}{rgb}{0.000000,0.000000,0.000000}%
\pgfsetstrokecolor{currentstroke}%
\pgfsetdash{}{0pt}%
\pgfusepath{stroke}%
\end{pgfscope}%
\begin{pgfscope}%
\pgfpathrectangle{\pgfqpoint{0.050000in}{0.050000in}}{\pgfqpoint{2.419000in}{2.419000in}}%
\pgfusepath{clip}%
\pgfsetbuttcap%
\pgfsetroundjoin%
\pgfsetlinewidth{1.003750pt}%
\definecolor{currentstroke}{rgb}{0.000000,0.000000,0.000000}%
\pgfsetstrokecolor{currentstroke}%
\pgfsetdash{}{0pt}%
\pgfusepath{stroke}%
\end{pgfscope}%
\begin{pgfscope}%
\pgfpathrectangle{\pgfqpoint{0.050000in}{0.050000in}}{\pgfqpoint{2.419000in}{2.419000in}}%
\pgfusepath{clip}%
\pgfsetbuttcap%
\pgfsetroundjoin%
\pgfsetlinewidth{1.003750pt}%
\definecolor{currentstroke}{rgb}{0.000000,0.000000,0.000000}%
\pgfsetstrokecolor{currentstroke}%
\pgfsetdash{}{0pt}%
\pgfusepath{stroke}%
\end{pgfscope}%
\begin{pgfscope}%
\pgfpathrectangle{\pgfqpoint{0.050000in}{0.050000in}}{\pgfqpoint{2.419000in}{2.419000in}}%
\pgfusepath{clip}%
\pgfsetbuttcap%
\pgfsetroundjoin%
\pgfsetlinewidth{1.003750pt}%
\definecolor{currentstroke}{rgb}{0.000000,0.000000,0.000000}%
\pgfsetstrokecolor{currentstroke}%
\pgfsetdash{}{0pt}%
\pgfusepath{stroke}%
\end{pgfscope}%
\begin{pgfscope}%
\pgfpathrectangle{\pgfqpoint{0.050000in}{0.050000in}}{\pgfqpoint{2.419000in}{2.419000in}}%
\pgfusepath{clip}%
\pgfsetbuttcap%
\pgfsetroundjoin%
\pgfsetlinewidth{1.003750pt}%
\definecolor{currentstroke}{rgb}{0.000000,0.000000,0.000000}%
\pgfsetstrokecolor{currentstroke}%
\pgfsetdash{}{0pt}%
\pgfusepath{stroke}%
\end{pgfscope}%
\begin{pgfscope}%
\pgfpathrectangle{\pgfqpoint{0.050000in}{0.050000in}}{\pgfqpoint{2.419000in}{2.419000in}}%
\pgfusepath{clip}%
\pgfsetbuttcap%
\pgfsetroundjoin%
\pgfsetlinewidth{1.003750pt}%
\definecolor{currentstroke}{rgb}{0.000000,0.000000,0.000000}%
\pgfsetstrokecolor{currentstroke}%
\pgfsetdash{}{0pt}%
\pgfusepath{stroke}%
\end{pgfscope}%
\begin{pgfscope}%
\pgfpathrectangle{\pgfqpoint{0.050000in}{0.050000in}}{\pgfqpoint{2.419000in}{2.419000in}}%
\pgfusepath{clip}%
\pgfsetbuttcap%
\pgfsetroundjoin%
\pgfsetlinewidth{1.003750pt}%
\definecolor{currentstroke}{rgb}{0.000000,0.000000,0.000000}%
\pgfsetstrokecolor{currentstroke}%
\pgfsetdash{}{0pt}%
\pgfusepath{stroke}%
\end{pgfscope}%
\begin{pgfscope}%
\pgfpathrectangle{\pgfqpoint{0.050000in}{0.050000in}}{\pgfqpoint{2.419000in}{2.419000in}}%
\pgfusepath{clip}%
\pgfsetbuttcap%
\pgfsetroundjoin%
\pgfsetlinewidth{1.003750pt}%
\definecolor{currentstroke}{rgb}{0.000000,0.000000,0.000000}%
\pgfsetstrokecolor{currentstroke}%
\pgfsetdash{}{0pt}%
\pgfusepath{stroke}%
\end{pgfscope}%
\begin{pgfscope}%
\pgfpathrectangle{\pgfqpoint{0.050000in}{0.050000in}}{\pgfqpoint{2.419000in}{2.419000in}}%
\pgfusepath{clip}%
\pgfsetbuttcap%
\pgfsetroundjoin%
\pgfsetlinewidth{1.003750pt}%
\definecolor{currentstroke}{rgb}{0.000000,0.000000,0.000000}%
\pgfsetstrokecolor{currentstroke}%
\pgfsetdash{}{0pt}%
\pgfusepath{stroke}%
\end{pgfscope}%
\begin{pgfscope}%
\pgfpathrectangle{\pgfqpoint{0.050000in}{0.050000in}}{\pgfqpoint{2.419000in}{2.419000in}}%
\pgfusepath{clip}%
\pgfsetbuttcap%
\pgfsetroundjoin%
\pgfsetlinewidth{1.003750pt}%
\definecolor{currentstroke}{rgb}{0.000000,0.000000,0.000000}%
\pgfsetstrokecolor{currentstroke}%
\pgfsetdash{}{0pt}%
\pgfusepath{stroke}%
\end{pgfscope}%
\begin{pgfscope}%
\pgfpathrectangle{\pgfqpoint{0.050000in}{0.050000in}}{\pgfqpoint{2.419000in}{2.419000in}}%
\pgfusepath{clip}%
\pgfsetbuttcap%
\pgfsetroundjoin%
\pgfsetlinewidth{1.003750pt}%
\definecolor{currentstroke}{rgb}{0.000000,0.000000,0.000000}%
\pgfsetstrokecolor{currentstroke}%
\pgfsetdash{}{0pt}%
\pgfusepath{stroke}%
\end{pgfscope}%
\begin{pgfscope}%
\pgfpathrectangle{\pgfqpoint{0.050000in}{0.050000in}}{\pgfqpoint{2.419000in}{2.419000in}}%
\pgfusepath{clip}%
\pgfsetbuttcap%
\pgfsetroundjoin%
\pgfsetlinewidth{1.003750pt}%
\definecolor{currentstroke}{rgb}{0.000000,0.000000,0.000000}%
\pgfsetstrokecolor{currentstroke}%
\pgfsetdash{}{0pt}%
\pgfusepath{stroke}%
\end{pgfscope}%
\begin{pgfscope}%
\pgfpathrectangle{\pgfqpoint{0.050000in}{0.050000in}}{\pgfqpoint{2.419000in}{2.419000in}}%
\pgfusepath{clip}%
\pgfsetbuttcap%
\pgfsetroundjoin%
\pgfsetlinewidth{1.003750pt}%
\definecolor{currentstroke}{rgb}{0.000000,0.000000,0.000000}%
\pgfsetstrokecolor{currentstroke}%
\pgfsetdash{}{0pt}%
\pgfusepath{stroke}%
\end{pgfscope}%
\begin{pgfscope}%
\pgfpathrectangle{\pgfqpoint{0.050000in}{0.050000in}}{\pgfqpoint{2.419000in}{2.419000in}}%
\pgfusepath{clip}%
\pgfsetbuttcap%
\pgfsetroundjoin%
\pgfsetlinewidth{1.003750pt}%
\definecolor{currentstroke}{rgb}{0.000000,0.000000,0.000000}%
\pgfsetstrokecolor{currentstroke}%
\pgfsetdash{}{0pt}%
\pgfusepath{stroke}%
\end{pgfscope}%
\begin{pgfscope}%
\pgfpathrectangle{\pgfqpoint{0.050000in}{0.050000in}}{\pgfqpoint{2.419000in}{2.419000in}}%
\pgfusepath{clip}%
\pgfsetbuttcap%
\pgfsetroundjoin%
\pgfsetlinewidth{1.003750pt}%
\definecolor{currentstroke}{rgb}{0.000000,0.000000,0.000000}%
\pgfsetstrokecolor{currentstroke}%
\pgfsetdash{}{0pt}%
\pgfusepath{stroke}%
\end{pgfscope}%
\begin{pgfscope}%
\pgfpathrectangle{\pgfqpoint{0.050000in}{0.050000in}}{\pgfqpoint{2.419000in}{2.419000in}}%
\pgfusepath{clip}%
\pgfsetbuttcap%
\pgfsetroundjoin%
\pgfsetlinewidth{1.003750pt}%
\definecolor{currentstroke}{rgb}{0.000000,0.000000,0.000000}%
\pgfsetstrokecolor{currentstroke}%
\pgfsetdash{}{0pt}%
\pgfusepath{stroke}%
\end{pgfscope}%
\begin{pgfscope}%
\pgfpathrectangle{\pgfqpoint{0.050000in}{0.050000in}}{\pgfqpoint{2.419000in}{2.419000in}}%
\pgfusepath{clip}%
\pgfsetbuttcap%
\pgfsetroundjoin%
\pgfsetlinewidth{1.003750pt}%
\definecolor{currentstroke}{rgb}{0.000000,0.000000,0.000000}%
\pgfsetstrokecolor{currentstroke}%
\pgfsetdash{}{0pt}%
\pgfusepath{stroke}%
\end{pgfscope}%
\begin{pgfscope}%
\pgfpathrectangle{\pgfqpoint{0.050000in}{0.050000in}}{\pgfqpoint{2.419000in}{2.419000in}}%
\pgfusepath{clip}%
\pgfsetbuttcap%
\pgfsetroundjoin%
\pgfsetlinewidth{1.003750pt}%
\definecolor{currentstroke}{rgb}{0.000000,0.000000,0.000000}%
\pgfsetstrokecolor{currentstroke}%
\pgfsetdash{}{0pt}%
\pgfusepath{stroke}%
\end{pgfscope}%
\begin{pgfscope}%
\pgfpathrectangle{\pgfqpoint{0.050000in}{0.050000in}}{\pgfqpoint{2.419000in}{2.419000in}}%
\pgfusepath{clip}%
\pgfsetbuttcap%
\pgfsetroundjoin%
\pgfsetlinewidth{1.003750pt}%
\definecolor{currentstroke}{rgb}{0.000000,0.000000,0.000000}%
\pgfsetstrokecolor{currentstroke}%
\pgfsetdash{}{0pt}%
\pgfusepath{stroke}%
\end{pgfscope}%
\begin{pgfscope}%
\pgfpathrectangle{\pgfqpoint{0.050000in}{0.050000in}}{\pgfqpoint{2.419000in}{2.419000in}}%
\pgfusepath{clip}%
\pgfsetbuttcap%
\pgfsetroundjoin%
\pgfsetlinewidth{1.003750pt}%
\definecolor{currentstroke}{rgb}{0.000000,0.000000,0.000000}%
\pgfsetstrokecolor{currentstroke}%
\pgfsetdash{}{0pt}%
\pgfusepath{stroke}%
\end{pgfscope}%
\begin{pgfscope}%
\pgfpathrectangle{\pgfqpoint{0.050000in}{0.050000in}}{\pgfqpoint{2.419000in}{2.419000in}}%
\pgfusepath{clip}%
\pgfsetbuttcap%
\pgfsetroundjoin%
\pgfsetlinewidth{1.003750pt}%
\definecolor{currentstroke}{rgb}{0.000000,0.000000,0.000000}%
\pgfsetstrokecolor{currentstroke}%
\pgfsetdash{}{0pt}%
\pgfusepath{stroke}%
\end{pgfscope}%
\begin{pgfscope}%
\pgfpathrectangle{\pgfqpoint{0.050000in}{0.050000in}}{\pgfqpoint{2.419000in}{2.419000in}}%
\pgfusepath{clip}%
\pgfsetbuttcap%
\pgfsetroundjoin%
\pgfsetlinewidth{1.003750pt}%
\definecolor{currentstroke}{rgb}{0.000000,0.000000,0.000000}%
\pgfsetstrokecolor{currentstroke}%
\pgfsetdash{}{0pt}%
\pgfusepath{stroke}%
\end{pgfscope}%
\begin{pgfscope}%
\pgfpathrectangle{\pgfqpoint{0.050000in}{0.050000in}}{\pgfqpoint{2.419000in}{2.419000in}}%
\pgfusepath{clip}%
\pgfsetbuttcap%
\pgfsetroundjoin%
\pgfsetlinewidth{1.003750pt}%
\definecolor{currentstroke}{rgb}{0.000000,0.000000,0.000000}%
\pgfsetstrokecolor{currentstroke}%
\pgfsetdash{}{0pt}%
\pgfusepath{stroke}%
\end{pgfscope}%
\begin{pgfscope}%
\pgfpathrectangle{\pgfqpoint{0.050000in}{0.050000in}}{\pgfqpoint{2.419000in}{2.419000in}}%
\pgfusepath{clip}%
\pgfsetbuttcap%
\pgfsetroundjoin%
\pgfsetlinewidth{1.003750pt}%
\definecolor{currentstroke}{rgb}{0.000000,0.000000,0.000000}%
\pgfsetstrokecolor{currentstroke}%
\pgfsetdash{}{0pt}%
\pgfusepath{stroke}%
\end{pgfscope}%
\begin{pgfscope}%
\pgfpathrectangle{\pgfqpoint{0.050000in}{0.050000in}}{\pgfqpoint{2.419000in}{2.419000in}}%
\pgfusepath{clip}%
\pgfsetbuttcap%
\pgfsetroundjoin%
\pgfsetlinewidth{1.003750pt}%
\definecolor{currentstroke}{rgb}{0.000000,0.000000,0.000000}%
\pgfsetstrokecolor{currentstroke}%
\pgfsetdash{}{0pt}%
\pgfusepath{stroke}%
\end{pgfscope}%
\begin{pgfscope}%
\pgfpathrectangle{\pgfqpoint{0.050000in}{0.050000in}}{\pgfqpoint{2.419000in}{2.419000in}}%
\pgfusepath{clip}%
\pgfsetbuttcap%
\pgfsetroundjoin%
\pgfsetlinewidth{1.003750pt}%
\definecolor{currentstroke}{rgb}{0.000000,0.000000,0.000000}%
\pgfsetstrokecolor{currentstroke}%
\pgfsetdash{}{0pt}%
\pgfusepath{stroke}%
\end{pgfscope}%
\begin{pgfscope}%
\pgfpathrectangle{\pgfqpoint{0.050000in}{0.050000in}}{\pgfqpoint{2.419000in}{2.419000in}}%
\pgfusepath{clip}%
\pgfsetbuttcap%
\pgfsetroundjoin%
\pgfsetlinewidth{1.003750pt}%
\definecolor{currentstroke}{rgb}{0.000000,0.000000,0.000000}%
\pgfsetstrokecolor{currentstroke}%
\pgfsetdash{}{0pt}%
\pgfusepath{stroke}%
\end{pgfscope}%
\begin{pgfscope}%
\pgfpathrectangle{\pgfqpoint{0.050000in}{0.050000in}}{\pgfqpoint{2.419000in}{2.419000in}}%
\pgfusepath{clip}%
\pgfsetbuttcap%
\pgfsetroundjoin%
\pgfsetlinewidth{1.003750pt}%
\definecolor{currentstroke}{rgb}{0.000000,0.000000,0.000000}%
\pgfsetstrokecolor{currentstroke}%
\pgfsetdash{}{0pt}%
\pgfusepath{stroke}%
\end{pgfscope}%
\begin{pgfscope}%
\pgfpathrectangle{\pgfqpoint{0.050000in}{0.050000in}}{\pgfqpoint{2.419000in}{2.419000in}}%
\pgfusepath{clip}%
\pgfsetbuttcap%
\pgfsetroundjoin%
\pgfsetlinewidth{1.003750pt}%
\definecolor{currentstroke}{rgb}{0.000000,0.000000,0.000000}%
\pgfsetstrokecolor{currentstroke}%
\pgfsetdash{}{0pt}%
\pgfusepath{stroke}%
\end{pgfscope}%
\begin{pgfscope}%
\pgfpathrectangle{\pgfqpoint{0.050000in}{0.050000in}}{\pgfqpoint{2.419000in}{2.419000in}}%
\pgfusepath{clip}%
\pgfsetbuttcap%
\pgfsetroundjoin%
\pgfsetlinewidth{1.003750pt}%
\definecolor{currentstroke}{rgb}{0.000000,0.000000,0.000000}%
\pgfsetstrokecolor{currentstroke}%
\pgfsetdash{}{0pt}%
\pgfusepath{stroke}%
\end{pgfscope}%
\begin{pgfscope}%
\pgfpathrectangle{\pgfqpoint{0.050000in}{0.050000in}}{\pgfqpoint{2.419000in}{2.419000in}}%
\pgfusepath{clip}%
\pgfsetbuttcap%
\pgfsetroundjoin%
\pgfsetlinewidth{1.003750pt}%
\definecolor{currentstroke}{rgb}{0.000000,0.000000,0.000000}%
\pgfsetstrokecolor{currentstroke}%
\pgfsetdash{}{0pt}%
\pgfusepath{stroke}%
\end{pgfscope}%
\begin{pgfscope}%
\pgfpathrectangle{\pgfqpoint{0.050000in}{0.050000in}}{\pgfqpoint{2.419000in}{2.419000in}}%
\pgfusepath{clip}%
\pgfsetbuttcap%
\pgfsetroundjoin%
\pgfsetlinewidth{1.003750pt}%
\definecolor{currentstroke}{rgb}{0.000000,0.000000,0.000000}%
\pgfsetstrokecolor{currentstroke}%
\pgfsetdash{}{0pt}%
\pgfusepath{stroke}%
\end{pgfscope}%
\begin{pgfscope}%
\pgfpathrectangle{\pgfqpoint{0.050000in}{0.050000in}}{\pgfqpoint{2.419000in}{2.419000in}}%
\pgfusepath{clip}%
\pgfsetbuttcap%
\pgfsetroundjoin%
\pgfsetlinewidth{1.003750pt}%
\definecolor{currentstroke}{rgb}{0.000000,0.000000,0.000000}%
\pgfsetstrokecolor{currentstroke}%
\pgfsetdash{}{0pt}%
\pgfusepath{stroke}%
\end{pgfscope}%
\begin{pgfscope}%
\pgfpathrectangle{\pgfqpoint{0.050000in}{0.050000in}}{\pgfqpoint{2.419000in}{2.419000in}}%
\pgfusepath{clip}%
\pgfsetbuttcap%
\pgfsetroundjoin%
\pgfsetlinewidth{1.003750pt}%
\definecolor{currentstroke}{rgb}{0.000000,0.000000,0.000000}%
\pgfsetstrokecolor{currentstroke}%
\pgfsetdash{}{0pt}%
\pgfusepath{stroke}%
\end{pgfscope}%
\begin{pgfscope}%
\pgfpathrectangle{\pgfqpoint{0.050000in}{0.050000in}}{\pgfqpoint{2.419000in}{2.419000in}}%
\pgfusepath{clip}%
\pgfsetbuttcap%
\pgfsetroundjoin%
\pgfsetlinewidth{1.003750pt}%
\definecolor{currentstroke}{rgb}{0.000000,0.000000,0.000000}%
\pgfsetstrokecolor{currentstroke}%
\pgfsetdash{}{0pt}%
\pgfusepath{stroke}%
\end{pgfscope}%
\begin{pgfscope}%
\pgfpathrectangle{\pgfqpoint{0.050000in}{0.050000in}}{\pgfqpoint{2.419000in}{2.419000in}}%
\pgfusepath{clip}%
\pgfsetbuttcap%
\pgfsetroundjoin%
\pgfsetlinewidth{1.003750pt}%
\definecolor{currentstroke}{rgb}{0.000000,0.000000,0.000000}%
\pgfsetstrokecolor{currentstroke}%
\pgfsetdash{}{0pt}%
\pgfusepath{stroke}%
\end{pgfscope}%
\begin{pgfscope}%
\pgfpathrectangle{\pgfqpoint{0.050000in}{0.050000in}}{\pgfqpoint{2.419000in}{2.419000in}}%
\pgfusepath{clip}%
\pgfsetbuttcap%
\pgfsetroundjoin%
\pgfsetlinewidth{1.003750pt}%
\definecolor{currentstroke}{rgb}{0.000000,0.000000,0.000000}%
\pgfsetstrokecolor{currentstroke}%
\pgfsetdash{}{0pt}%
\pgfusepath{stroke}%
\end{pgfscope}%
\begin{pgfscope}%
\pgfpathrectangle{\pgfqpoint{0.050000in}{0.050000in}}{\pgfqpoint{2.419000in}{2.419000in}}%
\pgfusepath{clip}%
\pgfsetbuttcap%
\pgfsetroundjoin%
\pgfsetlinewidth{1.003750pt}%
\definecolor{currentstroke}{rgb}{0.000000,0.000000,0.000000}%
\pgfsetstrokecolor{currentstroke}%
\pgfsetdash{}{0pt}%
\pgfusepath{stroke}%
\end{pgfscope}%
\begin{pgfscope}%
\pgfpathrectangle{\pgfqpoint{0.050000in}{0.050000in}}{\pgfqpoint{2.419000in}{2.419000in}}%
\pgfusepath{clip}%
\pgfsetbuttcap%
\pgfsetroundjoin%
\pgfsetlinewidth{1.003750pt}%
\definecolor{currentstroke}{rgb}{0.000000,0.000000,0.000000}%
\pgfsetstrokecolor{currentstroke}%
\pgfsetdash{}{0pt}%
\pgfusepath{stroke}%
\end{pgfscope}%
\begin{pgfscope}%
\pgfpathrectangle{\pgfqpoint{0.050000in}{0.050000in}}{\pgfqpoint{2.419000in}{2.419000in}}%
\pgfusepath{clip}%
\pgfsetbuttcap%
\pgfsetroundjoin%
\pgfsetlinewidth{1.003750pt}%
\definecolor{currentstroke}{rgb}{0.000000,0.000000,0.000000}%
\pgfsetstrokecolor{currentstroke}%
\pgfsetdash{}{0pt}%
\pgfusepath{stroke}%
\end{pgfscope}%
\begin{pgfscope}%
\pgfpathrectangle{\pgfqpoint{0.050000in}{0.050000in}}{\pgfqpoint{2.419000in}{2.419000in}}%
\pgfusepath{clip}%
\pgfsetbuttcap%
\pgfsetroundjoin%
\pgfsetlinewidth{1.003750pt}%
\definecolor{currentstroke}{rgb}{0.000000,0.000000,0.000000}%
\pgfsetstrokecolor{currentstroke}%
\pgfsetdash{}{0pt}%
\pgfusepath{stroke}%
\end{pgfscope}%
\begin{pgfscope}%
\pgfpathrectangle{\pgfqpoint{0.050000in}{0.050000in}}{\pgfqpoint{2.419000in}{2.419000in}}%
\pgfusepath{clip}%
\pgfsetbuttcap%
\pgfsetroundjoin%
\pgfsetlinewidth{1.003750pt}%
\definecolor{currentstroke}{rgb}{0.000000,0.000000,0.000000}%
\pgfsetstrokecolor{currentstroke}%
\pgfsetdash{}{0pt}%
\pgfusepath{stroke}%
\end{pgfscope}%
\begin{pgfscope}%
\pgfpathrectangle{\pgfqpoint{0.050000in}{0.050000in}}{\pgfqpoint{2.419000in}{2.419000in}}%
\pgfusepath{clip}%
\pgfsetbuttcap%
\pgfsetroundjoin%
\pgfsetlinewidth{1.003750pt}%
\definecolor{currentstroke}{rgb}{0.000000,0.000000,0.000000}%
\pgfsetstrokecolor{currentstroke}%
\pgfsetdash{}{0pt}%
\pgfusepath{stroke}%
\end{pgfscope}%
\begin{pgfscope}%
\pgfpathrectangle{\pgfqpoint{0.050000in}{0.050000in}}{\pgfqpoint{2.419000in}{2.419000in}}%
\pgfusepath{clip}%
\pgfsetbuttcap%
\pgfsetroundjoin%
\pgfsetlinewidth{1.003750pt}%
\definecolor{currentstroke}{rgb}{0.000000,0.000000,0.000000}%
\pgfsetstrokecolor{currentstroke}%
\pgfsetdash{}{0pt}%
\pgfusepath{stroke}%
\end{pgfscope}%
\begin{pgfscope}%
\pgfpathrectangle{\pgfqpoint{0.050000in}{0.050000in}}{\pgfqpoint{2.419000in}{2.419000in}}%
\pgfusepath{clip}%
\pgfsetbuttcap%
\pgfsetroundjoin%
\pgfsetlinewidth{1.003750pt}%
\definecolor{currentstroke}{rgb}{0.000000,0.000000,0.000000}%
\pgfsetstrokecolor{currentstroke}%
\pgfsetdash{}{0pt}%
\pgfusepath{stroke}%
\end{pgfscope}%
\begin{pgfscope}%
\pgfpathrectangle{\pgfqpoint{0.050000in}{0.050000in}}{\pgfqpoint{2.419000in}{2.419000in}}%
\pgfusepath{clip}%
\pgfsetbuttcap%
\pgfsetroundjoin%
\pgfsetlinewidth{1.003750pt}%
\definecolor{currentstroke}{rgb}{0.000000,0.000000,0.000000}%
\pgfsetstrokecolor{currentstroke}%
\pgfsetdash{}{0pt}%
\pgfusepath{stroke}%
\end{pgfscope}%
\begin{pgfscope}%
\pgfpathrectangle{\pgfqpoint{0.050000in}{0.050000in}}{\pgfqpoint{2.419000in}{2.419000in}}%
\pgfusepath{clip}%
\pgfsetbuttcap%
\pgfsetroundjoin%
\pgfsetlinewidth{1.003750pt}%
\definecolor{currentstroke}{rgb}{0.000000,0.000000,0.000000}%
\pgfsetstrokecolor{currentstroke}%
\pgfsetdash{}{0pt}%
\pgfusepath{stroke}%
\end{pgfscope}%
\begin{pgfscope}%
\pgfpathrectangle{\pgfqpoint{0.050000in}{0.050000in}}{\pgfqpoint{2.419000in}{2.419000in}}%
\pgfusepath{clip}%
\pgfsetbuttcap%
\pgfsetroundjoin%
\pgfsetlinewidth{1.003750pt}%
\definecolor{currentstroke}{rgb}{0.000000,0.000000,0.000000}%
\pgfsetstrokecolor{currentstroke}%
\pgfsetdash{}{0pt}%
\pgfusepath{stroke}%
\end{pgfscope}%
\begin{pgfscope}%
\pgfpathrectangle{\pgfqpoint{0.050000in}{0.050000in}}{\pgfqpoint{2.419000in}{2.419000in}}%
\pgfusepath{clip}%
\pgfsetbuttcap%
\pgfsetroundjoin%
\pgfsetlinewidth{1.003750pt}%
\definecolor{currentstroke}{rgb}{0.000000,0.000000,0.000000}%
\pgfsetstrokecolor{currentstroke}%
\pgfsetdash{}{0pt}%
\pgfusepath{stroke}%
\end{pgfscope}%
\begin{pgfscope}%
\pgfpathrectangle{\pgfqpoint{0.050000in}{0.050000in}}{\pgfqpoint{2.419000in}{2.419000in}}%
\pgfusepath{clip}%
\pgfsetbuttcap%
\pgfsetroundjoin%
\pgfsetlinewidth{1.003750pt}%
\definecolor{currentstroke}{rgb}{0.000000,0.000000,0.000000}%
\pgfsetstrokecolor{currentstroke}%
\pgfsetdash{}{0pt}%
\pgfusepath{stroke}%
\end{pgfscope}%
\begin{pgfscope}%
\pgfpathrectangle{\pgfqpoint{0.050000in}{0.050000in}}{\pgfqpoint{2.419000in}{2.419000in}}%
\pgfusepath{clip}%
\pgfsetbuttcap%
\pgfsetroundjoin%
\pgfsetlinewidth{1.003750pt}%
\definecolor{currentstroke}{rgb}{0.000000,0.000000,0.000000}%
\pgfsetstrokecolor{currentstroke}%
\pgfsetdash{}{0pt}%
\pgfusepath{stroke}%
\end{pgfscope}%
\begin{pgfscope}%
\pgfpathrectangle{\pgfqpoint{0.050000in}{0.050000in}}{\pgfqpoint{2.419000in}{2.419000in}}%
\pgfusepath{clip}%
\pgfsetbuttcap%
\pgfsetroundjoin%
\pgfsetlinewidth{1.003750pt}%
\definecolor{currentstroke}{rgb}{0.000000,0.000000,0.000000}%
\pgfsetstrokecolor{currentstroke}%
\pgfsetdash{}{0pt}%
\pgfusepath{stroke}%
\end{pgfscope}%
\begin{pgfscope}%
\pgfpathrectangle{\pgfqpoint{0.050000in}{0.050000in}}{\pgfqpoint{2.419000in}{2.419000in}}%
\pgfusepath{clip}%
\pgfsetbuttcap%
\pgfsetroundjoin%
\pgfsetlinewidth{1.003750pt}%
\definecolor{currentstroke}{rgb}{0.000000,0.000000,0.000000}%
\pgfsetstrokecolor{currentstroke}%
\pgfsetdash{}{0pt}%
\pgfusepath{stroke}%
\end{pgfscope}%
\begin{pgfscope}%
\pgfpathrectangle{\pgfqpoint{0.050000in}{0.050000in}}{\pgfqpoint{2.419000in}{2.419000in}}%
\pgfusepath{clip}%
\pgfsetbuttcap%
\pgfsetroundjoin%
\pgfsetlinewidth{1.003750pt}%
\definecolor{currentstroke}{rgb}{0.000000,0.000000,0.000000}%
\pgfsetstrokecolor{currentstroke}%
\pgfsetdash{}{0pt}%
\pgfusepath{stroke}%
\end{pgfscope}%
\begin{pgfscope}%
\pgfpathrectangle{\pgfqpoint{0.050000in}{0.050000in}}{\pgfqpoint{2.419000in}{2.419000in}}%
\pgfusepath{clip}%
\pgfsetbuttcap%
\pgfsetroundjoin%
\pgfsetlinewidth{1.003750pt}%
\definecolor{currentstroke}{rgb}{0.000000,0.000000,0.000000}%
\pgfsetstrokecolor{currentstroke}%
\pgfsetdash{}{0pt}%
\pgfusepath{stroke}%
\end{pgfscope}%
\begin{pgfscope}%
\pgfpathrectangle{\pgfqpoint{0.050000in}{0.050000in}}{\pgfqpoint{2.419000in}{2.419000in}}%
\pgfusepath{clip}%
\pgfsetbuttcap%
\pgfsetroundjoin%
\pgfsetlinewidth{1.003750pt}%
\definecolor{currentstroke}{rgb}{0.000000,0.000000,0.000000}%
\pgfsetstrokecolor{currentstroke}%
\pgfsetdash{}{0pt}%
\pgfusepath{stroke}%
\end{pgfscope}%
\begin{pgfscope}%
\pgfpathrectangle{\pgfqpoint{0.050000in}{0.050000in}}{\pgfqpoint{2.419000in}{2.419000in}}%
\pgfusepath{clip}%
\pgfsetbuttcap%
\pgfsetroundjoin%
\pgfsetlinewidth{1.003750pt}%
\definecolor{currentstroke}{rgb}{0.000000,0.000000,0.000000}%
\pgfsetstrokecolor{currentstroke}%
\pgfsetdash{}{0pt}%
\pgfusepath{stroke}%
\end{pgfscope}%
\begin{pgfscope}%
\pgfpathrectangle{\pgfqpoint{0.050000in}{0.050000in}}{\pgfqpoint{2.419000in}{2.419000in}}%
\pgfusepath{clip}%
\pgfsetbuttcap%
\pgfsetroundjoin%
\pgfsetlinewidth{1.003750pt}%
\definecolor{currentstroke}{rgb}{0.000000,0.000000,0.000000}%
\pgfsetstrokecolor{currentstroke}%
\pgfsetdash{}{0pt}%
\pgfusepath{stroke}%
\end{pgfscope}%
\begin{pgfscope}%
\pgfpathrectangle{\pgfqpoint{0.050000in}{0.050000in}}{\pgfqpoint{2.419000in}{2.419000in}}%
\pgfusepath{clip}%
\pgfsetbuttcap%
\pgfsetroundjoin%
\definecolor{currentfill}{rgb}{0.800000,0.400000,0.466667}%
\pgfsetfillcolor{currentfill}%
\pgfsetfillopacity{0.300000}%
\pgfsetlinewidth{1.003750pt}%
\definecolor{currentstroke}{rgb}{0.800000,0.400000,0.466667}%
\pgfsetstrokecolor{currentstroke}%
\pgfsetstrokeopacity{0.300000}%
\pgfsetdash{}{0pt}%
\pgfpathmoveto{\pgfqpoint{1.541000in}{4.062536in}}%
\pgfpathcurveto{\pgfqpoint{1.550209in}{4.062536in}}{\pgfqpoint{1.559041in}{4.066195in}}{\pgfqpoint{1.565553in}{4.072706in}}%
\pgfpathcurveto{\pgfqpoint{1.572064in}{4.079217in}}{\pgfqpoint{1.575723in}{4.088050in}}{\pgfqpoint{1.575723in}{4.097258in}}%
\pgfpathcurveto{\pgfqpoint{1.575723in}{4.106467in}}{\pgfqpoint{1.572064in}{4.115299in}}{\pgfqpoint{1.565553in}{4.121811in}}%
\pgfpathcurveto{\pgfqpoint{1.559041in}{4.128322in}}{\pgfqpoint{1.550209in}{4.131980in}}{\pgfqpoint{1.541000in}{4.131980in}}%
\pgfpathcurveto{\pgfqpoint{1.531792in}{4.131980in}}{\pgfqpoint{1.522959in}{4.128322in}}{\pgfqpoint{1.516448in}{4.121811in}}%
\pgfpathcurveto{\pgfqpoint{1.509937in}{4.115299in}}{\pgfqpoint{1.506278in}{4.106467in}}{\pgfqpoint{1.506278in}{4.097258in}}%
\pgfpathcurveto{\pgfqpoint{1.506278in}{4.088050in}}{\pgfqpoint{1.509937in}{4.079217in}}{\pgfqpoint{1.516448in}{4.072706in}}%
\pgfpathcurveto{\pgfqpoint{1.522959in}{4.066195in}}{\pgfqpoint{1.531792in}{4.062536in}}{\pgfqpoint{1.541000in}{4.062536in}}%
\pgfpathlineto{\pgfqpoint{1.541000in}{4.062536in}}%
\pgfpathclose%
\pgfusepath{stroke,fill}%
\end{pgfscope}%
\begin{pgfscope}%
\pgfpathrectangle{\pgfqpoint{0.050000in}{0.050000in}}{\pgfqpoint{2.419000in}{2.419000in}}%
\pgfusepath{clip}%
\pgfsetbuttcap%
\pgfsetroundjoin%
\definecolor{currentfill}{rgb}{0.800000,0.400000,0.466667}%
\pgfsetfillcolor{currentfill}%
\pgfsetfillopacity{0.305085}%
\pgfsetlinewidth{1.003750pt}%
\definecolor{currentstroke}{rgb}{0.800000,0.400000,0.466667}%
\pgfsetstrokecolor{currentstroke}%
\pgfsetstrokeopacity{0.305085}%
\pgfsetdash{}{0pt}%
\pgfpathmoveto{\pgfqpoint{2.120050in}{3.958479in}}%
\pgfpathcurveto{\pgfqpoint{2.129258in}{3.958479in}}{\pgfqpoint{2.138091in}{3.962138in}}{\pgfqpoint{2.144602in}{3.968649in}}%
\pgfpathcurveto{\pgfqpoint{2.151113in}{3.975161in}}{\pgfqpoint{2.154772in}{3.983993in}}{\pgfqpoint{2.154772in}{3.993202in}}%
\pgfpathcurveto{\pgfqpoint{2.154772in}{4.002410in}}{\pgfqpoint{2.151113in}{4.011243in}}{\pgfqpoint{2.144602in}{4.017754in}}%
\pgfpathcurveto{\pgfqpoint{2.138091in}{4.024265in}}{\pgfqpoint{2.129258in}{4.027924in}}{\pgfqpoint{2.120050in}{4.027924in}}%
\pgfpathcurveto{\pgfqpoint{2.110841in}{4.027924in}}{\pgfqpoint{2.102009in}{4.024265in}}{\pgfqpoint{2.095497in}{4.017754in}}%
\pgfpathcurveto{\pgfqpoint{2.088986in}{4.011243in}}{\pgfqpoint{2.085327in}{4.002410in}}{\pgfqpoint{2.085327in}{3.993202in}}%
\pgfpathcurveto{\pgfqpoint{2.085327in}{3.983993in}}{\pgfqpoint{2.088986in}{3.975161in}}{\pgfqpoint{2.095497in}{3.968649in}}%
\pgfpathcurveto{\pgfqpoint{2.102009in}{3.962138in}}{\pgfqpoint{2.110841in}{3.958479in}}{\pgfqpoint{2.120050in}{3.958479in}}%
\pgfpathlineto{\pgfqpoint{2.120050in}{3.958479in}}%
\pgfpathclose%
\pgfusepath{stroke,fill}%
\end{pgfscope}%
\begin{pgfscope}%
\pgfpathrectangle{\pgfqpoint{0.050000in}{0.050000in}}{\pgfqpoint{2.419000in}{2.419000in}}%
\pgfusepath{clip}%
\pgfsetbuttcap%
\pgfsetroundjoin%
\definecolor{currentfill}{rgb}{0.800000,0.400000,0.466667}%
\pgfsetfillcolor{currentfill}%
\pgfsetfillopacity{0.310398}%
\pgfsetlinewidth{1.003750pt}%
\definecolor{currentstroke}{rgb}{0.800000,0.400000,0.466667}%
\pgfsetstrokecolor{currentstroke}%
\pgfsetstrokeopacity{0.310398}%
\pgfsetdash{}{0pt}%
\pgfpathmoveto{\pgfqpoint{2.725070in}{3.849756in}}%
\pgfpathcurveto{\pgfqpoint{2.734278in}{3.849756in}}{\pgfqpoint{2.743111in}{3.853414in}}{\pgfqpoint{2.749622in}{3.859926in}}%
\pgfpathcurveto{\pgfqpoint{2.756133in}{3.866437in}}{\pgfqpoint{2.759792in}{3.875270in}}{\pgfqpoint{2.759792in}{3.884478in}}%
\pgfpathcurveto{\pgfqpoint{2.759792in}{3.893686in}}{\pgfqpoint{2.756133in}{3.902519in}}{\pgfqpoint{2.749622in}{3.909030in}}%
\pgfpathcurveto{\pgfqpoint{2.743111in}{3.915542in}}{\pgfqpoint{2.734278in}{3.919200in}}{\pgfqpoint{2.725070in}{3.919200in}}%
\pgfpathcurveto{\pgfqpoint{2.715861in}{3.919200in}}{\pgfqpoint{2.707029in}{3.915542in}}{\pgfqpoint{2.700517in}{3.909030in}}%
\pgfpathcurveto{\pgfqpoint{2.694006in}{3.902519in}}{\pgfqpoint{2.690347in}{3.893686in}}{\pgfqpoint{2.690347in}{3.884478in}}%
\pgfpathcurveto{\pgfqpoint{2.690347in}{3.875270in}}{\pgfqpoint{2.694006in}{3.866437in}}{\pgfqpoint{2.700517in}{3.859926in}}%
\pgfpathcurveto{\pgfqpoint{2.707029in}{3.853414in}}{\pgfqpoint{2.715861in}{3.849756in}}{\pgfqpoint{2.725070in}{3.849756in}}%
\pgfpathlineto{\pgfqpoint{2.725070in}{3.849756in}}%
\pgfpathclose%
\pgfusepath{stroke,fill}%
\end{pgfscope}%
\begin{pgfscope}%
\pgfpathrectangle{\pgfqpoint{0.050000in}{0.050000in}}{\pgfqpoint{2.419000in}{2.419000in}}%
\pgfusepath{clip}%
\pgfsetbuttcap%
\pgfsetroundjoin%
\definecolor{currentfill}{rgb}{0.800000,0.400000,0.466667}%
\pgfsetfillcolor{currentfill}%
\pgfsetfillopacity{0.313145}%
\pgfsetlinewidth{1.003750pt}%
\definecolor{currentstroke}{rgb}{0.800000,0.400000,0.466667}%
\pgfsetstrokecolor{currentstroke}%
\pgfsetstrokeopacity{0.313145}%
\pgfsetdash{}{0pt}%
\pgfpathmoveto{\pgfqpoint{1.456257in}{3.793545in}}%
\pgfpathcurveto{\pgfqpoint{1.465465in}{3.793545in}}{\pgfqpoint{1.474298in}{3.797203in}}{\pgfqpoint{1.480809in}{3.803715in}}%
\pgfpathcurveto{\pgfqpoint{1.487321in}{3.810226in}}{\pgfqpoint{1.490979in}{3.819058in}}{\pgfqpoint{1.490979in}{3.828267in}}%
\pgfpathcurveto{\pgfqpoint{1.490979in}{3.837475in}}{\pgfqpoint{1.487321in}{3.846308in}}{\pgfqpoint{1.480809in}{3.852819in}}%
\pgfpathcurveto{\pgfqpoint{1.474298in}{3.859331in}}{\pgfqpoint{1.465465in}{3.862989in}}{\pgfqpoint{1.456257in}{3.862989in}}%
\pgfpathcurveto{\pgfqpoint{1.447049in}{3.862989in}}{\pgfqpoint{1.438216in}{3.859331in}}{\pgfqpoint{1.431705in}{3.852819in}}%
\pgfpathcurveto{\pgfqpoint{1.425193in}{3.846308in}}{\pgfqpoint{1.421535in}{3.837475in}}{\pgfqpoint{1.421535in}{3.828267in}}%
\pgfpathcurveto{\pgfqpoint{1.421535in}{3.819058in}}{\pgfqpoint{1.425193in}{3.810226in}}{\pgfqpoint{1.431705in}{3.803715in}}%
\pgfpathcurveto{\pgfqpoint{1.438216in}{3.797203in}}{\pgfqpoint{1.447049in}{3.793545in}}{\pgfqpoint{1.456257in}{3.793545in}}%
\pgfpathlineto{\pgfqpoint{1.456257in}{3.793545in}}%
\pgfpathclose%
\pgfusepath{stroke,fill}%
\end{pgfscope}%
\begin{pgfscope}%
\pgfpathrectangle{\pgfqpoint{0.050000in}{0.050000in}}{\pgfqpoint{2.419000in}{2.419000in}}%
\pgfusepath{clip}%
\pgfsetbuttcap%
\pgfsetroundjoin%
\definecolor{currentfill}{rgb}{0.800000,0.400000,0.466667}%
\pgfsetfillcolor{currentfill}%
\pgfsetfillopacity{0.315955}%
\pgfsetlinewidth{1.003750pt}%
\definecolor{currentstroke}{rgb}{0.800000,0.400000,0.466667}%
\pgfsetstrokecolor{currentstroke}%
\pgfsetstrokeopacity{0.315955}%
\pgfsetdash{}{0pt}%
\pgfpathmoveto{\pgfqpoint{3.357848in}{3.736044in}}%
\pgfpathcurveto{\pgfqpoint{3.367056in}{3.736044in}}{\pgfqpoint{3.375889in}{3.739703in}}{\pgfqpoint{3.382400in}{3.746214in}}%
\pgfpathcurveto{\pgfqpoint{3.388911in}{3.752725in}}{\pgfqpoint{3.392570in}{3.761558in}}{\pgfqpoint{3.392570in}{3.770766in}}%
\pgfpathcurveto{\pgfqpoint{3.392570in}{3.779975in}}{\pgfqpoint{3.388911in}{3.788807in}}{\pgfqpoint{3.382400in}{3.795319in}}%
\pgfpathcurveto{\pgfqpoint{3.375889in}{3.801830in}}{\pgfqpoint{3.367056in}{3.805488in}}{\pgfqpoint{3.357848in}{3.805488in}}%
\pgfpathcurveto{\pgfqpoint{3.348639in}{3.805488in}}{\pgfqpoint{3.339807in}{3.801830in}}{\pgfqpoint{3.333295in}{3.795319in}}%
\pgfpathcurveto{\pgfqpoint{3.326784in}{3.788807in}}{\pgfqpoint{3.323126in}{3.779975in}}{\pgfqpoint{3.323126in}{3.770766in}}%
\pgfpathcurveto{\pgfqpoint{3.323126in}{3.761558in}}{\pgfqpoint{3.326784in}{3.752725in}}{\pgfqpoint{3.333295in}{3.746214in}}%
\pgfpathcurveto{\pgfqpoint{3.339807in}{3.739703in}}{\pgfqpoint{3.348639in}{3.736044in}}{\pgfqpoint{3.357848in}{3.736044in}}%
\pgfpathlineto{\pgfqpoint{3.357848in}{3.736044in}}%
\pgfpathclose%
\pgfusepath{stroke,fill}%
\end{pgfscope}%
\begin{pgfscope}%
\pgfpathrectangle{\pgfqpoint{0.050000in}{0.050000in}}{\pgfqpoint{2.419000in}{2.419000in}}%
\pgfusepath{clip}%
\pgfsetbuttcap%
\pgfsetroundjoin%
\definecolor{currentfill}{rgb}{0.800000,0.400000,0.466667}%
\pgfsetfillcolor{currentfill}%
\pgfsetfillopacity{0.318831}%
\pgfsetlinewidth{1.003750pt}%
\definecolor{currentstroke}{rgb}{0.800000,0.400000,0.466667}%
\pgfsetstrokecolor{currentstroke}%
\pgfsetstrokeopacity{0.318831}%
\pgfsetdash{}{0pt}%
\pgfpathmoveto{\pgfqpoint{2.066933in}{3.677209in}}%
\pgfpathcurveto{\pgfqpoint{2.076142in}{3.677209in}}{\pgfqpoint{2.084974in}{3.680868in}}{\pgfqpoint{2.091486in}{3.687379in}}%
\pgfpathcurveto{\pgfqpoint{2.097997in}{3.693890in}}{\pgfqpoint{2.101655in}{3.702723in}}{\pgfqpoint{2.101655in}{3.711931in}}%
\pgfpathcurveto{\pgfqpoint{2.101655in}{3.721140in}}{\pgfqpoint{2.097997in}{3.729972in}}{\pgfqpoint{2.091486in}{3.736484in}}%
\pgfpathcurveto{\pgfqpoint{2.084974in}{3.742995in}}{\pgfqpoint{2.076142in}{3.746653in}}{\pgfqpoint{2.066933in}{3.746653in}}%
\pgfpathcurveto{\pgfqpoint{2.057725in}{3.746653in}}{\pgfqpoint{2.048892in}{3.742995in}}{\pgfqpoint{2.042381in}{3.736484in}}%
\pgfpathcurveto{\pgfqpoint{2.035870in}{3.729972in}}{\pgfqpoint{2.032211in}{3.721140in}}{\pgfqpoint{2.032211in}{3.711931in}}%
\pgfpathcurveto{\pgfqpoint{2.032211in}{3.702723in}}{\pgfqpoint{2.035870in}{3.693890in}}{\pgfqpoint{2.042381in}{3.687379in}}%
\pgfpathcurveto{\pgfqpoint{2.048892in}{3.680868in}}{\pgfqpoint{2.057725in}{3.677209in}}{\pgfqpoint{2.066933in}{3.677209in}}%
\pgfpathlineto{\pgfqpoint{2.066933in}{3.677209in}}%
\pgfpathclose%
\pgfusepath{stroke,fill}%
\end{pgfscope}%
\begin{pgfscope}%
\pgfpathrectangle{\pgfqpoint{0.050000in}{0.050000in}}{\pgfqpoint{2.419000in}{2.419000in}}%
\pgfusepath{clip}%
\pgfsetbuttcap%
\pgfsetroundjoin%
\definecolor{currentfill}{rgb}{0.800000,0.400000,0.466667}%
\pgfsetfillcolor{currentfill}%
\pgfsetfillopacity{0.321773}%
\pgfsetlinewidth{1.003750pt}%
\definecolor{currentstroke}{rgb}{0.800000,0.400000,0.466667}%
\pgfsetstrokecolor{currentstroke}%
\pgfsetstrokeopacity{0.321773}%
\pgfsetdash{}{0pt}%
\pgfpathmoveto{\pgfqpoint{0.745710in}{3.616993in}}%
\pgfpathcurveto{\pgfqpoint{0.754918in}{3.616993in}}{\pgfqpoint{0.763751in}{3.620651in}}{\pgfqpoint{0.770262in}{3.627163in}}%
\pgfpathcurveto{\pgfqpoint{0.776774in}{3.633674in}}{\pgfqpoint{0.780432in}{3.642506in}}{\pgfqpoint{0.780432in}{3.651715in}}%
\pgfpathcurveto{\pgfqpoint{0.780432in}{3.660923in}}{\pgfqpoint{0.776774in}{3.669756in}}{\pgfqpoint{0.770262in}{3.676267in}}%
\pgfpathcurveto{\pgfqpoint{0.763751in}{3.682779in}}{\pgfqpoint{0.754918in}{3.686437in}}{\pgfqpoint{0.745710in}{3.686437in}}%
\pgfpathcurveto{\pgfqpoint{0.736501in}{3.686437in}}{\pgfqpoint{0.727669in}{3.682779in}}{\pgfqpoint{0.721158in}{3.676267in}}%
\pgfpathcurveto{\pgfqpoint{0.714646in}{3.669756in}}{\pgfqpoint{0.710988in}{3.660923in}}{\pgfqpoint{0.710988in}{3.651715in}}%
\pgfpathcurveto{\pgfqpoint{0.710988in}{3.642506in}}{\pgfqpoint{0.714646in}{3.633674in}}{\pgfqpoint{0.721158in}{3.627163in}}%
\pgfpathcurveto{\pgfqpoint{0.727669in}{3.620651in}}{\pgfqpoint{0.736501in}{3.616993in}}{\pgfqpoint{0.745710in}{3.616993in}}%
\pgfpathlineto{\pgfqpoint{0.745710in}{3.616993in}}%
\pgfpathclose%
\pgfusepath{stroke,fill}%
\end{pgfscope}%
\begin{pgfscope}%
\pgfpathrectangle{\pgfqpoint{0.050000in}{0.050000in}}{\pgfqpoint{2.419000in}{2.419000in}}%
\pgfusepath{clip}%
\pgfsetbuttcap%
\pgfsetroundjoin%
\definecolor{currentfill}{rgb}{0.800000,0.400000,0.466667}%
\pgfsetfillcolor{currentfill}%
\pgfsetfillopacity{0.321773}%
\pgfsetlinewidth{1.003750pt}%
\definecolor{currentstroke}{rgb}{0.800000,0.400000,0.466667}%
\pgfsetstrokecolor{currentstroke}%
\pgfsetstrokeopacity{0.321773}%
\pgfsetdash{}{0pt}%
\pgfpathmoveto{\pgfqpoint{4.020339in}{3.616993in}}%
\pgfpathcurveto{\pgfqpoint{4.029548in}{3.616993in}}{\pgfqpoint{4.038380in}{3.620651in}}{\pgfqpoint{4.044892in}{3.627163in}}%
\pgfpathcurveto{\pgfqpoint{4.051403in}{3.633674in}}{\pgfqpoint{4.055061in}{3.642506in}}{\pgfqpoint{4.055061in}{3.651715in}}%
\pgfpathcurveto{\pgfqpoint{4.055061in}{3.660923in}}{\pgfqpoint{4.051403in}{3.669756in}}{\pgfqpoint{4.044892in}{3.676267in}}%
\pgfpathcurveto{\pgfqpoint{4.038380in}{3.682779in}}{\pgfqpoint{4.029548in}{3.686437in}}{\pgfqpoint{4.020339in}{3.686437in}}%
\pgfpathcurveto{\pgfqpoint{4.011131in}{3.686437in}}{\pgfqpoint{4.002298in}{3.682779in}}{\pgfqpoint{3.995787in}{3.676267in}}%
\pgfpathcurveto{\pgfqpoint{3.989276in}{3.669756in}}{\pgfqpoint{3.985617in}{3.660923in}}{\pgfqpoint{3.985617in}{3.651715in}}%
\pgfpathcurveto{\pgfqpoint{3.985617in}{3.642506in}}{\pgfqpoint{3.989276in}{3.633674in}}{\pgfqpoint{3.995787in}{3.627163in}}%
\pgfpathcurveto{\pgfqpoint{4.002298in}{3.620651in}}{\pgfqpoint{4.011131in}{3.616993in}}{\pgfqpoint{4.020339in}{3.616993in}}%
\pgfpathlineto{\pgfqpoint{4.020339in}{3.616993in}}%
\pgfpathclose%
\pgfusepath{stroke,fill}%
\end{pgfscope}%
\begin{pgfscope}%
\pgfpathrectangle{\pgfqpoint{0.050000in}{0.050000in}}{\pgfqpoint{2.419000in}{2.419000in}}%
\pgfusepath{clip}%
\pgfsetbuttcap%
\pgfsetroundjoin%
\definecolor{currentfill}{rgb}{0.800000,0.400000,0.466667}%
\pgfsetfillcolor{currentfill}%
\pgfsetfillopacity{0.324786}%
\pgfsetlinewidth{1.003750pt}%
\definecolor{currentstroke}{rgb}{0.800000,0.400000,0.466667}%
\pgfsetstrokecolor{currentstroke}%
\pgfsetstrokeopacity{0.324786}%
\pgfsetdash{}{0pt}%
\pgfpathmoveto{\pgfqpoint{2.706625in}{3.555346in}}%
\pgfpathcurveto{\pgfqpoint{2.715834in}{3.555346in}}{\pgfqpoint{2.724666in}{3.559004in}}{\pgfqpoint{2.731178in}{3.565516in}}%
\pgfpathcurveto{\pgfqpoint{2.737689in}{3.572027in}}{\pgfqpoint{2.741348in}{3.580859in}}{\pgfqpoint{2.741348in}{3.590068in}}%
\pgfpathcurveto{\pgfqpoint{2.741348in}{3.599276in}}{\pgfqpoint{2.737689in}{3.608109in}}{\pgfqpoint{2.731178in}{3.614620in}}%
\pgfpathcurveto{\pgfqpoint{2.724666in}{3.621132in}}{\pgfqpoint{2.715834in}{3.624790in}}{\pgfqpoint{2.706625in}{3.624790in}}%
\pgfpathcurveto{\pgfqpoint{2.697417in}{3.624790in}}{\pgfqpoint{2.688584in}{3.621132in}}{\pgfqpoint{2.682073in}{3.614620in}}%
\pgfpathcurveto{\pgfqpoint{2.675562in}{3.608109in}}{\pgfqpoint{2.671903in}{3.599276in}}{\pgfqpoint{2.671903in}{3.590068in}}%
\pgfpathcurveto{\pgfqpoint{2.671903in}{3.580859in}}{\pgfqpoint{2.675562in}{3.572027in}}{\pgfqpoint{2.682073in}{3.565516in}}%
\pgfpathcurveto{\pgfqpoint{2.688584in}{3.559004in}}{\pgfqpoint{2.697417in}{3.555346in}}{\pgfqpoint{2.706625in}{3.555346in}}%
\pgfpathlineto{\pgfqpoint{2.706625in}{3.555346in}}%
\pgfpathclose%
\pgfusepath{stroke,fill}%
\end{pgfscope}%
\begin{pgfscope}%
\pgfpathrectangle{\pgfqpoint{0.050000in}{0.050000in}}{\pgfqpoint{2.419000in}{2.419000in}}%
\pgfusepath{clip}%
\pgfsetbuttcap%
\pgfsetroundjoin%
\definecolor{currentfill}{rgb}{0.800000,0.400000,0.466667}%
\pgfsetfillcolor{currentfill}%
\pgfsetfillopacity{0.327871}%
\pgfsetlinewidth{1.003750pt}%
\definecolor{currentstroke}{rgb}{0.800000,0.400000,0.466667}%
\pgfsetstrokecolor{currentstroke}%
\pgfsetstrokeopacity{0.327871}%
\pgfsetdash{}{0pt}%
\pgfpathmoveto{\pgfqpoint{1.361326in}{3.492217in}}%
\pgfpathcurveto{\pgfqpoint{1.370535in}{3.492217in}}{\pgfqpoint{1.379367in}{3.495875in}}{\pgfqpoint{1.385878in}{3.502386in}}%
\pgfpathcurveto{\pgfqpoint{1.392390in}{3.508898in}}{\pgfqpoint{1.396048in}{3.517730in}}{\pgfqpoint{1.396048in}{3.526939in}}%
\pgfpathcurveto{\pgfqpoint{1.396048in}{3.536147in}}{\pgfqpoint{1.392390in}{3.544980in}}{\pgfqpoint{1.385878in}{3.551491in}}%
\pgfpathcurveto{\pgfqpoint{1.379367in}{3.558002in}}{\pgfqpoint{1.370535in}{3.561661in}}{\pgfqpoint{1.361326in}{3.561661in}}%
\pgfpathcurveto{\pgfqpoint{1.352118in}{3.561661in}}{\pgfqpoint{1.343285in}{3.558002in}}{\pgfqpoint{1.336774in}{3.551491in}}%
\pgfpathcurveto{\pgfqpoint{1.330262in}{3.544980in}}{\pgfqpoint{1.326604in}{3.536147in}}{\pgfqpoint{1.326604in}{3.526939in}}%
\pgfpathcurveto{\pgfqpoint{1.326604in}{3.517730in}}{\pgfqpoint{1.330262in}{3.508898in}}{\pgfqpoint{1.336774in}{3.502386in}}%
\pgfpathcurveto{\pgfqpoint{1.343285in}{3.495875in}}{\pgfqpoint{1.352118in}{3.492217in}}{\pgfqpoint{1.361326in}{3.492217in}}%
\pgfpathlineto{\pgfqpoint{1.361326in}{3.492217in}}%
\pgfpathclose%
\pgfusepath{stroke,fill}%
\end{pgfscope}%
\begin{pgfscope}%
\pgfpathrectangle{\pgfqpoint{0.050000in}{0.050000in}}{\pgfqpoint{2.419000in}{2.419000in}}%
\pgfusepath{clip}%
\pgfsetbuttcap%
\pgfsetroundjoin%
\definecolor{currentfill}{rgb}{0.800000,0.400000,0.466667}%
\pgfsetfillcolor{currentfill}%
\pgfsetfillopacity{0.327871}%
\pgfsetlinewidth{1.003750pt}%
\definecolor{currentstroke}{rgb}{0.800000,0.400000,0.466667}%
\pgfsetstrokecolor{currentstroke}%
\pgfsetstrokeopacity{0.327871}%
\pgfsetdash{}{0pt}%
\pgfpathmoveto{\pgfqpoint{4.714687in}{3.492217in}}%
\pgfpathcurveto{\pgfqpoint{4.723896in}{3.492217in}}{\pgfqpoint{4.732728in}{3.495875in}}{\pgfqpoint{4.739239in}{3.502386in}}%
\pgfpathcurveto{\pgfqpoint{4.745751in}{3.508898in}}{\pgfqpoint{4.749409in}{3.517730in}}{\pgfqpoint{4.749409in}{3.526939in}}%
\pgfpathcurveto{\pgfqpoint{4.749409in}{3.536147in}}{\pgfqpoint{4.745751in}{3.544980in}}{\pgfqpoint{4.739239in}{3.551491in}}%
\pgfpathcurveto{\pgfqpoint{4.732728in}{3.558002in}}{\pgfqpoint{4.723896in}{3.561661in}}{\pgfqpoint{4.714687in}{3.561661in}}%
\pgfpathcurveto{\pgfqpoint{4.705479in}{3.561661in}}{\pgfqpoint{4.696646in}{3.558002in}}{\pgfqpoint{4.690135in}{3.551491in}}%
\pgfpathcurveto{\pgfqpoint{4.683623in}{3.544980in}}{\pgfqpoint{4.679965in}{3.536147in}}{\pgfqpoint{4.679965in}{3.526939in}}%
\pgfpathcurveto{\pgfqpoint{4.679965in}{3.517730in}}{\pgfqpoint{4.683623in}{3.508898in}}{\pgfqpoint{4.690135in}{3.502386in}}%
\pgfpathcurveto{\pgfqpoint{4.696646in}{3.495875in}}{\pgfqpoint{4.705479in}{3.492217in}}{\pgfqpoint{4.714687in}{3.492217in}}%
\pgfpathlineto{\pgfqpoint{4.714687in}{3.492217in}}%
\pgfpathclose%
\pgfusepath{stroke,fill}%
\end{pgfscope}%
\begin{pgfscope}%
\pgfpathrectangle{\pgfqpoint{0.050000in}{0.050000in}}{\pgfqpoint{2.419000in}{2.419000in}}%
\pgfusepath{clip}%
\pgfsetbuttcap%
\pgfsetroundjoin%
\definecolor{currentfill}{rgb}{0.800000,0.400000,0.466667}%
\pgfsetfillcolor{currentfill}%
\pgfsetfillopacity{0.331031}%
\pgfsetlinewidth{1.003750pt}%
\definecolor{currentstroke}{rgb}{0.800000,0.400000,0.466667}%
\pgfsetstrokecolor{currentstroke}%
\pgfsetstrokeopacity{0.331031}%
\pgfsetdash{}{0pt}%
\pgfpathmoveto{\pgfqpoint{3.377452in}{3.427551in}}%
\pgfpathcurveto{\pgfqpoint{3.386661in}{3.427551in}}{\pgfqpoint{3.395493in}{3.431210in}}{\pgfqpoint{3.402004in}{3.437721in}}%
\pgfpathcurveto{\pgfqpoint{3.408516in}{3.444232in}}{\pgfqpoint{3.412174in}{3.453065in}}{\pgfqpoint{3.412174in}{3.462273in}}%
\pgfpathcurveto{\pgfqpoint{3.412174in}{3.471482in}}{\pgfqpoint{3.408516in}{3.480314in}}{\pgfqpoint{3.402004in}{3.486826in}}%
\pgfpathcurveto{\pgfqpoint{3.395493in}{3.493337in}}{\pgfqpoint{3.386661in}{3.496996in}}{\pgfqpoint{3.377452in}{3.496996in}}%
\pgfpathcurveto{\pgfqpoint{3.368244in}{3.496996in}}{\pgfqpoint{3.359411in}{3.493337in}}{\pgfqpoint{3.352900in}{3.486826in}}%
\pgfpathcurveto{\pgfqpoint{3.346388in}{3.480314in}}{\pgfqpoint{3.342730in}{3.471482in}}{\pgfqpoint{3.342730in}{3.462273in}}%
\pgfpathcurveto{\pgfqpoint{3.342730in}{3.453065in}}{\pgfqpoint{3.346388in}{3.444232in}}{\pgfqpoint{3.352900in}{3.437721in}}%
\pgfpathcurveto{\pgfqpoint{3.359411in}{3.431210in}}{\pgfqpoint{3.368244in}{3.427551in}}{\pgfqpoint{3.377452in}{3.427551in}}%
\pgfpathlineto{\pgfqpoint{3.377452in}{3.427551in}}%
\pgfpathclose%
\pgfusepath{stroke,fill}%
\end{pgfscope}%
\begin{pgfscope}%
\pgfpathrectangle{\pgfqpoint{0.050000in}{0.050000in}}{\pgfqpoint{2.419000in}{2.419000in}}%
\pgfusepath{clip}%
\pgfsetbuttcap%
\pgfsetroundjoin%
\definecolor{currentfill}{rgb}{0.800000,0.400000,0.466667}%
\pgfsetfillcolor{currentfill}%
\pgfsetfillopacity{0.334269}%
\pgfsetlinewidth{1.003750pt}%
\definecolor{currentstroke}{rgb}{0.800000,0.400000,0.466667}%
\pgfsetstrokecolor{currentstroke}%
\pgfsetstrokeopacity{0.334269}%
\pgfsetdash{}{0pt}%
\pgfpathmoveto{\pgfqpoint{2.007274in}{3.361293in}}%
\pgfpathcurveto{\pgfqpoint{2.016483in}{3.361293in}}{\pgfqpoint{2.025315in}{3.364951in}}{\pgfqpoint{2.031826in}{3.371463in}}%
\pgfpathcurveto{\pgfqpoint{2.038338in}{3.377974in}}{\pgfqpoint{2.041996in}{3.386806in}}{\pgfqpoint{2.041996in}{3.396015in}}%
\pgfpathcurveto{\pgfqpoint{2.041996in}{3.405223in}}{\pgfqpoint{2.038338in}{3.414056in}}{\pgfqpoint{2.031826in}{3.420567in}}%
\pgfpathcurveto{\pgfqpoint{2.025315in}{3.427079in}}{\pgfqpoint{2.016483in}{3.430737in}}{\pgfqpoint{2.007274in}{3.430737in}}%
\pgfpathcurveto{\pgfqpoint{1.998066in}{3.430737in}}{\pgfqpoint{1.989233in}{3.427079in}}{\pgfqpoint{1.982722in}{3.420567in}}%
\pgfpathcurveto{\pgfqpoint{1.976210in}{3.414056in}}{\pgfqpoint{1.972552in}{3.405223in}}{\pgfqpoint{1.972552in}{3.396015in}}%
\pgfpathcurveto{\pgfqpoint{1.972552in}{3.386806in}}{\pgfqpoint{1.976210in}{3.377974in}}{\pgfqpoint{1.982722in}{3.371463in}}%
\pgfpathcurveto{\pgfqpoint{1.989233in}{3.364951in}}{\pgfqpoint{1.998066in}{3.361293in}}{\pgfqpoint{2.007274in}{3.361293in}}%
\pgfpathlineto{\pgfqpoint{2.007274in}{3.361293in}}%
\pgfpathclose%
\pgfusepath{stroke,fill}%
\end{pgfscope}%
\begin{pgfscope}%
\pgfpathrectangle{\pgfqpoint{0.050000in}{0.050000in}}{\pgfqpoint{2.419000in}{2.419000in}}%
\pgfusepath{clip}%
\pgfsetbuttcap%
\pgfsetroundjoin%
\definecolor{currentfill}{rgb}{0.800000,0.400000,0.466667}%
\pgfsetfillcolor{currentfill}%
\pgfsetfillopacity{0.334269}%
\pgfsetlinewidth{1.003750pt}%
\definecolor{currentstroke}{rgb}{0.800000,0.400000,0.466667}%
\pgfsetstrokecolor{currentstroke}%
\pgfsetstrokeopacity{0.334269}%
\pgfsetdash{}{0pt}%
\pgfpathmoveto{\pgfqpoint{5.443246in}{3.361293in}}%
\pgfpathcurveto{\pgfqpoint{5.452454in}{3.361293in}}{\pgfqpoint{5.461287in}{3.364951in}}{\pgfqpoint{5.467798in}{3.371463in}}%
\pgfpathcurveto{\pgfqpoint{5.474310in}{3.377974in}}{\pgfqpoint{5.477968in}{3.386806in}}{\pgfqpoint{5.477968in}{3.396015in}}%
\pgfpathcurveto{\pgfqpoint{5.477968in}{3.405223in}}{\pgfqpoint{5.474310in}{3.414056in}}{\pgfqpoint{5.467798in}{3.420567in}}%
\pgfpathcurveto{\pgfqpoint{5.461287in}{3.427079in}}{\pgfqpoint{5.452454in}{3.430737in}}{\pgfqpoint{5.443246in}{3.430737in}}%
\pgfpathcurveto{\pgfqpoint{5.434037in}{3.430737in}}{\pgfqpoint{5.425205in}{3.427079in}}{\pgfqpoint{5.418694in}{3.420567in}}%
\pgfpathcurveto{\pgfqpoint{5.412182in}{3.414056in}}{\pgfqpoint{5.408524in}{3.405223in}}{\pgfqpoint{5.408524in}{3.396015in}}%
\pgfpathcurveto{\pgfqpoint{5.408524in}{3.386806in}}{\pgfqpoint{5.412182in}{3.377974in}}{\pgfqpoint{5.418694in}{3.371463in}}%
\pgfpathcurveto{\pgfqpoint{5.425205in}{3.364951in}}{\pgfqpoint{5.434037in}{3.361293in}}{\pgfqpoint{5.443246in}{3.361293in}}%
\pgfpathlineto{\pgfqpoint{5.443246in}{3.361293in}}%
\pgfpathclose%
\pgfusepath{stroke,fill}%
\end{pgfscope}%
\begin{pgfscope}%
\pgfpathrectangle{\pgfqpoint{0.050000in}{0.050000in}}{\pgfqpoint{2.419000in}{2.419000in}}%
\pgfusepath{clip}%
\pgfsetbuttcap%
\pgfsetroundjoin%
\definecolor{currentfill}{rgb}{0.800000,0.400000,0.466667}%
\pgfsetfillcolor{currentfill}%
\pgfsetfillopacity{0.337588}%
\pgfsetlinewidth{1.003750pt}%
\definecolor{currentstroke}{rgb}{0.800000,0.400000,0.466667}%
\pgfsetstrokecolor{currentstroke}%
\pgfsetstrokeopacity{0.337588}%
\pgfsetdash{}{0pt}%
\pgfpathmoveto{\pgfqpoint{0.602920in}{3.293382in}}%
\pgfpathcurveto{\pgfqpoint{0.612129in}{3.293382in}}{\pgfqpoint{0.620961in}{3.297040in}}{\pgfqpoint{0.627473in}{3.303552in}}%
\pgfpathcurveto{\pgfqpoint{0.633984in}{3.310063in}}{\pgfqpoint{0.637643in}{3.318895in}}{\pgfqpoint{0.637643in}{3.328104in}}%
\pgfpathcurveto{\pgfqpoint{0.637643in}{3.337312in}}{\pgfqpoint{0.633984in}{3.346145in}}{\pgfqpoint{0.627473in}{3.352656in}}%
\pgfpathcurveto{\pgfqpoint{0.620961in}{3.359167in}}{\pgfqpoint{0.612129in}{3.362826in}}{\pgfqpoint{0.602920in}{3.362826in}}%
\pgfpathcurveto{\pgfqpoint{0.593712in}{3.362826in}}{\pgfqpoint{0.584879in}{3.359167in}}{\pgfqpoint{0.578368in}{3.352656in}}%
\pgfpathcurveto{\pgfqpoint{0.571857in}{3.346145in}}{\pgfqpoint{0.568198in}{3.337312in}}{\pgfqpoint{0.568198in}{3.328104in}}%
\pgfpathcurveto{\pgfqpoint{0.568198in}{3.318895in}}{\pgfqpoint{0.571857in}{3.310063in}}{\pgfqpoint{0.578368in}{3.303552in}}%
\pgfpathcurveto{\pgfqpoint{0.584879in}{3.297040in}}{\pgfqpoint{0.593712in}{3.293382in}}{\pgfqpoint{0.602920in}{3.293382in}}%
\pgfpathlineto{\pgfqpoint{0.602920in}{3.293382in}}%
\pgfpathclose%
\pgfusepath{stroke,fill}%
\end{pgfscope}%
\begin{pgfscope}%
\pgfpathrectangle{\pgfqpoint{0.050000in}{0.050000in}}{\pgfqpoint{2.419000in}{2.419000in}}%
\pgfusepath{clip}%
\pgfsetbuttcap%
\pgfsetroundjoin%
\definecolor{currentfill}{rgb}{0.800000,0.400000,0.466667}%
\pgfsetfillcolor{currentfill}%
\pgfsetfillopacity{0.337588}%
\pgfsetlinewidth{1.003750pt}%
\definecolor{currentstroke}{rgb}{0.800000,0.400000,0.466667}%
\pgfsetstrokecolor{currentstroke}%
\pgfsetstrokeopacity{0.337588}%
\pgfsetdash{}{0pt}%
\pgfpathmoveto{\pgfqpoint{4.081743in}{3.293382in}}%
\pgfpathcurveto{\pgfqpoint{4.090951in}{3.293382in}}{\pgfqpoint{4.099784in}{3.297040in}}{\pgfqpoint{4.106295in}{3.303552in}}%
\pgfpathcurveto{\pgfqpoint{4.112807in}{3.310063in}}{\pgfqpoint{4.116465in}{3.318895in}}{\pgfqpoint{4.116465in}{3.328104in}}%
\pgfpathcurveto{\pgfqpoint{4.116465in}{3.337312in}}{\pgfqpoint{4.112807in}{3.346145in}}{\pgfqpoint{4.106295in}{3.352656in}}%
\pgfpathcurveto{\pgfqpoint{4.099784in}{3.359167in}}{\pgfqpoint{4.090951in}{3.362826in}}{\pgfqpoint{4.081743in}{3.362826in}}%
\pgfpathcurveto{\pgfqpoint{4.072535in}{3.362826in}}{\pgfqpoint{4.063702in}{3.359167in}}{\pgfqpoint{4.057191in}{3.352656in}}%
\pgfpathcurveto{\pgfqpoint{4.050679in}{3.346145in}}{\pgfqpoint{4.047021in}{3.337312in}}{\pgfqpoint{4.047021in}{3.328104in}}%
\pgfpathcurveto{\pgfqpoint{4.047021in}{3.318895in}}{\pgfqpoint{4.050679in}{3.310063in}}{\pgfqpoint{4.057191in}{3.303552in}}%
\pgfpathcurveto{\pgfqpoint{4.063702in}{3.297040in}}{\pgfqpoint{4.072535in}{3.293382in}}{\pgfqpoint{4.081743in}{3.293382in}}%
\pgfpathlineto{\pgfqpoint{4.081743in}{3.293382in}}%
\pgfpathclose%
\pgfusepath{stroke,fill}%
\end{pgfscope}%
\begin{pgfscope}%
\pgfpathrectangle{\pgfqpoint{0.050000in}{0.050000in}}{\pgfqpoint{2.419000in}{2.419000in}}%
\pgfusepath{clip}%
\pgfsetbuttcap%
\pgfsetroundjoin%
\definecolor{currentfill}{rgb}{0.800000,0.400000,0.466667}%
\pgfsetfillcolor{currentfill}%
\pgfsetfillopacity{0.340991}%
\pgfsetlinewidth{1.003750pt}%
\definecolor{currentstroke}{rgb}{0.800000,0.400000,0.466667}%
\pgfsetstrokecolor{currentstroke}%
\pgfsetstrokeopacity{0.340991}%
\pgfsetdash{}{0pt}%
\pgfpathmoveto{\pgfqpoint{2.685852in}{3.223755in}}%
\pgfpathcurveto{\pgfqpoint{2.695060in}{3.223755in}}{\pgfqpoint{2.703893in}{3.227414in}}{\pgfqpoint{2.710404in}{3.233925in}}%
\pgfpathcurveto{\pgfqpoint{2.716916in}{3.240437in}}{\pgfqpoint{2.720574in}{3.249269in}}{\pgfqpoint{2.720574in}{3.258477in}}%
\pgfpathcurveto{\pgfqpoint{2.720574in}{3.267686in}}{\pgfqpoint{2.716916in}{3.276518in}}{\pgfqpoint{2.710404in}{3.283030in}}%
\pgfpathcurveto{\pgfqpoint{2.703893in}{3.289541in}}{\pgfqpoint{2.695060in}{3.293200in}}{\pgfqpoint{2.685852in}{3.293200in}}%
\pgfpathcurveto{\pgfqpoint{2.676644in}{3.293200in}}{\pgfqpoint{2.667811in}{3.289541in}}{\pgfqpoint{2.661300in}{3.283030in}}%
\pgfpathcurveto{\pgfqpoint{2.654788in}{3.276518in}}{\pgfqpoint{2.651130in}{3.267686in}}{\pgfqpoint{2.651130in}{3.258477in}}%
\pgfpathcurveto{\pgfqpoint{2.651130in}{3.249269in}}{\pgfqpoint{2.654788in}{3.240437in}}{\pgfqpoint{2.661300in}{3.233925in}}%
\pgfpathcurveto{\pgfqpoint{2.667811in}{3.227414in}}{\pgfqpoint{2.676644in}{3.223755in}}{\pgfqpoint{2.685852in}{3.223755in}}%
\pgfpathlineto{\pgfqpoint{2.685852in}{3.223755in}}%
\pgfpathclose%
\pgfusepath{stroke,fill}%
\end{pgfscope}%
\begin{pgfscope}%
\pgfpathrectangle{\pgfqpoint{0.050000in}{0.050000in}}{\pgfqpoint{2.419000in}{2.419000in}}%
\pgfusepath{clip}%
\pgfsetbuttcap%
\pgfsetroundjoin%
\definecolor{currentfill}{rgb}{0.800000,0.400000,0.466667}%
\pgfsetfillcolor{currentfill}%
\pgfsetfillopacity{0.340991}%
\pgfsetlinewidth{1.003750pt}%
\definecolor{currentstroke}{rgb}{0.800000,0.400000,0.466667}%
\pgfsetstrokecolor{currentstroke}%
\pgfsetstrokeopacity{0.340991}%
\pgfsetdash{}{0pt}%
\pgfpathmoveto{\pgfqpoint{6.208608in}{3.223755in}}%
\pgfpathcurveto{\pgfqpoint{6.217816in}{3.223755in}}{\pgfqpoint{6.226649in}{3.227414in}}{\pgfqpoint{6.233160in}{3.233925in}}%
\pgfpathcurveto{\pgfqpoint{6.239671in}{3.240437in}}{\pgfqpoint{6.243330in}{3.249269in}}{\pgfqpoint{6.243330in}{3.258477in}}%
\pgfpathcurveto{\pgfqpoint{6.243330in}{3.267686in}}{\pgfqpoint{6.239671in}{3.276518in}}{\pgfqpoint{6.233160in}{3.283030in}}%
\pgfpathcurveto{\pgfqpoint{6.226649in}{3.289541in}}{\pgfqpoint{6.217816in}{3.293200in}}{\pgfqpoint{6.208608in}{3.293200in}}%
\pgfpathcurveto{\pgfqpoint{6.199399in}{3.293200in}}{\pgfqpoint{6.190567in}{3.289541in}}{\pgfqpoint{6.184055in}{3.283030in}}%
\pgfpathcurveto{\pgfqpoint{6.177544in}{3.276518in}}{\pgfqpoint{6.173886in}{3.267686in}}{\pgfqpoint{6.173886in}{3.258477in}}%
\pgfpathcurveto{\pgfqpoint{6.173886in}{3.249269in}}{\pgfqpoint{6.177544in}{3.240437in}}{\pgfqpoint{6.184055in}{3.233925in}}%
\pgfpathcurveto{\pgfqpoint{6.190567in}{3.227414in}}{\pgfqpoint{6.199399in}{3.223755in}}{\pgfqpoint{6.208608in}{3.223755in}}%
\pgfpathlineto{\pgfqpoint{6.208608in}{3.223755in}}%
\pgfpathclose%
\pgfusepath{stroke,fill}%
\end{pgfscope}%
\begin{pgfscope}%
\pgfpathrectangle{\pgfqpoint{0.050000in}{0.050000in}}{\pgfqpoint{2.419000in}{2.419000in}}%
\pgfusepath{clip}%
\pgfsetbuttcap%
\pgfsetroundjoin%
\definecolor{currentfill}{rgb}{0.800000,0.400000,0.466667}%
\pgfsetfillcolor{currentfill}%
\pgfsetfillopacity{0.344480}%
\pgfsetlinewidth{1.003750pt}%
\definecolor{currentstroke}{rgb}{0.800000,0.400000,0.466667}%
\pgfsetstrokecolor{currentstroke}%
\pgfsetstrokeopacity{0.344480}%
\pgfsetdash{}{0pt}%
\pgfpathmoveto{\pgfqpoint{4.822066in}{3.152348in}}%
\pgfpathcurveto{\pgfqpoint{4.831275in}{3.152348in}}{\pgfqpoint{4.840107in}{3.156006in}}{\pgfqpoint{4.846618in}{3.162518in}}%
\pgfpathcurveto{\pgfqpoint{4.853130in}{3.169029in}}{\pgfqpoint{4.856788in}{3.177862in}}{\pgfqpoint{4.856788in}{3.187070in}}%
\pgfpathcurveto{\pgfqpoint{4.856788in}{3.196278in}}{\pgfqpoint{4.853130in}{3.205111in}}{\pgfqpoint{4.846618in}{3.211622in}}%
\pgfpathcurveto{\pgfqpoint{4.840107in}{3.218134in}}{\pgfqpoint{4.831275in}{3.221792in}}{\pgfqpoint{4.822066in}{3.221792in}}%
\pgfpathcurveto{\pgfqpoint{4.812858in}{3.221792in}}{\pgfqpoint{4.804025in}{3.218134in}}{\pgfqpoint{4.797514in}{3.211622in}}%
\pgfpathcurveto{\pgfqpoint{4.791002in}{3.205111in}}{\pgfqpoint{4.787344in}{3.196278in}}{\pgfqpoint{4.787344in}{3.187070in}}%
\pgfpathcurveto{\pgfqpoint{4.787344in}{3.177862in}}{\pgfqpoint{4.791002in}{3.169029in}}{\pgfqpoint{4.797514in}{3.162518in}}%
\pgfpathcurveto{\pgfqpoint{4.804025in}{3.156006in}}{\pgfqpoint{4.812858in}{3.152348in}}{\pgfqpoint{4.822066in}{3.152348in}}%
\pgfpathlineto{\pgfqpoint{4.822066in}{3.152348in}}%
\pgfpathclose%
\pgfusepath{stroke,fill}%
\end{pgfscope}%
\begin{pgfscope}%
\pgfpathrectangle{\pgfqpoint{0.050000in}{0.050000in}}{\pgfqpoint{2.419000in}{2.419000in}}%
\pgfusepath{clip}%
\pgfsetbuttcap%
\pgfsetroundjoin%
\definecolor{currentfill}{rgb}{0.800000,0.400000,0.466667}%
\pgfsetfillcolor{currentfill}%
\pgfsetfillopacity{0.344480}%
\pgfsetlinewidth{1.003750pt}%
\definecolor{currentstroke}{rgb}{0.800000,0.400000,0.466667}%
\pgfsetstrokecolor{currentstroke}%
\pgfsetstrokeopacity{0.344480}%
\pgfsetdash{}{0pt}%
\pgfpathmoveto{\pgfqpoint{1.254253in}{3.152348in}}%
\pgfpathcurveto{\pgfqpoint{1.263462in}{3.152348in}}{\pgfqpoint{1.272294in}{3.156006in}}{\pgfqpoint{1.278806in}{3.162518in}}%
\pgfpathcurveto{\pgfqpoint{1.285317in}{3.169029in}}{\pgfqpoint{1.288976in}{3.177862in}}{\pgfqpoint{1.288976in}{3.187070in}}%
\pgfpathcurveto{\pgfqpoint{1.288976in}{3.196278in}}{\pgfqpoint{1.285317in}{3.205111in}}{\pgfqpoint{1.278806in}{3.211622in}}%
\pgfpathcurveto{\pgfqpoint{1.272294in}{3.218134in}}{\pgfqpoint{1.263462in}{3.221792in}}{\pgfqpoint{1.254253in}{3.221792in}}%
\pgfpathcurveto{\pgfqpoint{1.245045in}{3.221792in}}{\pgfqpoint{1.236212in}{3.218134in}}{\pgfqpoint{1.229701in}{3.211622in}}%
\pgfpathcurveto{\pgfqpoint{1.223190in}{3.205111in}}{\pgfqpoint{1.219531in}{3.196278in}}{\pgfqpoint{1.219531in}{3.187070in}}%
\pgfpathcurveto{\pgfqpoint{1.219531in}{3.177862in}}{\pgfqpoint{1.223190in}{3.169029in}}{\pgfqpoint{1.229701in}{3.162518in}}%
\pgfpathcurveto{\pgfqpoint{1.236212in}{3.156006in}}{\pgfqpoint{1.245045in}{3.152348in}}{\pgfqpoint{1.254253in}{3.152348in}}%
\pgfpathlineto{\pgfqpoint{1.254253in}{3.152348in}}%
\pgfpathclose%
\pgfusepath{stroke,fill}%
\end{pgfscope}%
\begin{pgfscope}%
\pgfpathrectangle{\pgfqpoint{0.050000in}{0.050000in}}{\pgfqpoint{2.419000in}{2.419000in}}%
\pgfusepath{clip}%
\pgfsetbuttcap%
\pgfsetroundjoin%
\definecolor{currentfill}{rgb}{0.800000,0.400000,0.466667}%
\pgfsetfillcolor{currentfill}%
\pgfsetfillopacity{0.348060}%
\pgfsetlinewidth{1.003750pt}%
\definecolor{currentstroke}{rgb}{0.800000,0.400000,0.466667}%
\pgfsetstrokecolor{currentstroke}%
\pgfsetstrokeopacity{0.348060}%
\pgfsetdash{}{0pt}%
\pgfpathmoveto{\pgfqpoint{-0.214441in}{3.079090in}}%
\pgfpathcurveto{\pgfqpoint{-0.205232in}{3.079090in}}{\pgfqpoint{-0.196400in}{3.082749in}}{\pgfqpoint{-0.189888in}{3.089260in}}%
\pgfpathcurveto{\pgfqpoint{-0.183377in}{3.095771in}}{\pgfqpoint{-0.179718in}{3.104604in}}{\pgfqpoint{-0.179718in}{3.113812in}}%
\pgfpathcurveto{\pgfqpoint{-0.179718in}{3.123021in}}{\pgfqpoint{-0.183377in}{3.131853in}}{\pgfqpoint{-0.189888in}{3.138365in}}%
\pgfpathcurveto{\pgfqpoint{-0.196400in}{3.144876in}}{\pgfqpoint{-0.205232in}{3.148535in}}{\pgfqpoint{-0.214441in}{3.148535in}}%
\pgfpathcurveto{\pgfqpoint{-0.223649in}{3.148535in}}{\pgfqpoint{-0.232482in}{3.144876in}}{\pgfqpoint{-0.238993in}{3.138365in}}%
\pgfpathcurveto{\pgfqpoint{-0.245504in}{3.131853in}}{\pgfqpoint{-0.249163in}{3.123021in}}{\pgfqpoint{-0.249163in}{3.113812in}}%
\pgfpathcurveto{\pgfqpoint{-0.249163in}{3.104604in}}{\pgfqpoint{-0.245504in}{3.095771in}}{\pgfqpoint{-0.238993in}{3.089260in}}%
\pgfpathcurveto{\pgfqpoint{-0.232482in}{3.082749in}}{\pgfqpoint{-0.223649in}{3.079090in}}{\pgfqpoint{-0.214441in}{3.079090in}}%
\pgfpathlineto{\pgfqpoint{-0.214441in}{3.079090in}}%
\pgfpathclose%
\pgfusepath{stroke,fill}%
\end{pgfscope}%
\begin{pgfscope}%
\pgfpathrectangle{\pgfqpoint{0.050000in}{0.050000in}}{\pgfqpoint{2.419000in}{2.419000in}}%
\pgfusepath{clip}%
\pgfsetbuttcap%
\pgfsetroundjoin%
\definecolor{currentfill}{rgb}{0.800000,0.400000,0.466667}%
\pgfsetfillcolor{currentfill}%
\pgfsetfillopacity{0.348060}%
\pgfsetlinewidth{1.003750pt}%
\definecolor{currentstroke}{rgb}{0.800000,0.400000,0.466667}%
\pgfsetstrokecolor{currentstroke}%
\pgfsetstrokeopacity{0.348060}%
\pgfsetdash{}{0pt}%
\pgfpathmoveto{\pgfqpoint{3.399596in}{3.079090in}}%
\pgfpathcurveto{\pgfqpoint{3.408805in}{3.079090in}}{\pgfqpoint{3.417637in}{3.082749in}}{\pgfqpoint{3.424149in}{3.089260in}}%
\pgfpathcurveto{\pgfqpoint{3.430660in}{3.095771in}}{\pgfqpoint{3.434319in}{3.104604in}}{\pgfqpoint{3.434319in}{3.113812in}}%
\pgfpathcurveto{\pgfqpoint{3.434319in}{3.123021in}}{\pgfqpoint{3.430660in}{3.131853in}}{\pgfqpoint{3.424149in}{3.138365in}}%
\pgfpathcurveto{\pgfqpoint{3.417637in}{3.144876in}}{\pgfqpoint{3.408805in}{3.148535in}}{\pgfqpoint{3.399596in}{3.148535in}}%
\pgfpathcurveto{\pgfqpoint{3.390388in}{3.148535in}}{\pgfqpoint{3.381555in}{3.144876in}}{\pgfqpoint{3.375044in}{3.138365in}}%
\pgfpathcurveto{\pgfqpoint{3.368533in}{3.131853in}}{\pgfqpoint{3.364874in}{3.123021in}}{\pgfqpoint{3.364874in}{3.113812in}}%
\pgfpathcurveto{\pgfqpoint{3.364874in}{3.104604in}}{\pgfqpoint{3.368533in}{3.095771in}}{\pgfqpoint{3.375044in}{3.089260in}}%
\pgfpathcurveto{\pgfqpoint{3.381555in}{3.082749in}}{\pgfqpoint{3.390388in}{3.079090in}}{\pgfqpoint{3.399596in}{3.079090in}}%
\pgfpathlineto{\pgfqpoint{3.399596in}{3.079090in}}%
\pgfpathclose%
\pgfusepath{stroke,fill}%
\end{pgfscope}%
\begin{pgfscope}%
\pgfpathrectangle{\pgfqpoint{0.050000in}{0.050000in}}{\pgfqpoint{2.419000in}{2.419000in}}%
\pgfusepath{clip}%
\pgfsetbuttcap%
\pgfsetroundjoin%
\definecolor{currentfill}{rgb}{0.800000,0.400000,0.466667}%
\pgfsetfillcolor{currentfill}%
\pgfsetfillopacity{0.348060}%
\pgfsetlinewidth{1.003750pt}%
\definecolor{currentstroke}{rgb}{0.800000,0.400000,0.466667}%
\pgfsetstrokecolor{currentstroke}%
\pgfsetstrokeopacity{0.348060}%
\pgfsetdash{}{0pt}%
\pgfpathmoveto{\pgfqpoint{7.013634in}{3.079090in}}%
\pgfpathcurveto{\pgfqpoint{7.022842in}{3.079090in}}{\pgfqpoint{7.031675in}{3.082749in}}{\pgfqpoint{7.038186in}{3.089260in}}%
\pgfpathcurveto{\pgfqpoint{7.044697in}{3.095771in}}{\pgfqpoint{7.048356in}{3.104604in}}{\pgfqpoint{7.048356in}{3.113812in}}%
\pgfpathcurveto{\pgfqpoint{7.048356in}{3.123021in}}{\pgfqpoint{7.044697in}{3.131853in}}{\pgfqpoint{7.038186in}{3.138365in}}%
\pgfpathcurveto{\pgfqpoint{7.031675in}{3.144876in}}{\pgfqpoint{7.022842in}{3.148535in}}{\pgfqpoint{7.013634in}{3.148535in}}%
\pgfpathcurveto{\pgfqpoint{7.004425in}{3.148535in}}{\pgfqpoint{6.995593in}{3.144876in}}{\pgfqpoint{6.989081in}{3.138365in}}%
\pgfpathcurveto{\pgfqpoint{6.982570in}{3.131853in}}{\pgfqpoint{6.978911in}{3.123021in}}{\pgfqpoint{6.978911in}{3.113812in}}%
\pgfpathcurveto{\pgfqpoint{6.978911in}{3.104604in}}{\pgfqpoint{6.982570in}{3.095771in}}{\pgfqpoint{6.989081in}{3.089260in}}%
\pgfpathcurveto{\pgfqpoint{6.995593in}{3.082749in}}{\pgfqpoint{7.004425in}{3.079090in}}{\pgfqpoint{7.013634in}{3.079090in}}%
\pgfpathlineto{\pgfqpoint{7.013634in}{3.079090in}}%
\pgfpathclose%
\pgfusepath{stroke,fill}%
\end{pgfscope}%
\begin{pgfscope}%
\pgfpathrectangle{\pgfqpoint{0.050000in}{0.050000in}}{\pgfqpoint{2.419000in}{2.419000in}}%
\pgfusepath{clip}%
\pgfsetbuttcap%
\pgfsetroundjoin%
\definecolor{currentfill}{rgb}{0.800000,0.400000,0.466667}%
\pgfsetfillcolor{currentfill}%
\pgfsetfillopacity{0.351735}%
\pgfsetlinewidth{1.003750pt}%
\definecolor{currentstroke}{rgb}{0.800000,0.400000,0.466667}%
\pgfsetstrokecolor{currentstroke}%
\pgfsetstrokeopacity{0.351735}%
\pgfsetdash{}{0pt}%
\pgfpathmoveto{\pgfqpoint{5.601259in}{3.003909in}}%
\pgfpathcurveto{\pgfqpoint{5.610468in}{3.003909in}}{\pgfqpoint{5.619300in}{3.007568in}}{\pgfqpoint{5.625811in}{3.014079in}}%
\pgfpathcurveto{\pgfqpoint{5.632323in}{3.020590in}}{\pgfqpoint{5.635981in}{3.029423in}}{\pgfqpoint{5.635981in}{3.038631in}}%
\pgfpathcurveto{\pgfqpoint{5.635981in}{3.047840in}}{\pgfqpoint{5.632323in}{3.056672in}}{\pgfqpoint{5.625811in}{3.063184in}}%
\pgfpathcurveto{\pgfqpoint{5.619300in}{3.069695in}}{\pgfqpoint{5.610468in}{3.073354in}}{\pgfqpoint{5.601259in}{3.073354in}}%
\pgfpathcurveto{\pgfqpoint{5.592051in}{3.073354in}}{\pgfqpoint{5.583218in}{3.069695in}}{\pgfqpoint{5.576707in}{3.063184in}}%
\pgfpathcurveto{\pgfqpoint{5.570195in}{3.056672in}}{\pgfqpoint{5.566537in}{3.047840in}}{\pgfqpoint{5.566537in}{3.038631in}}%
\pgfpathcurveto{\pgfqpoint{5.566537in}{3.029423in}}{\pgfqpoint{5.570195in}{3.020590in}}{\pgfqpoint{5.576707in}{3.014079in}}%
\pgfpathcurveto{\pgfqpoint{5.583218in}{3.007568in}}{\pgfqpoint{5.592051in}{3.003909in}}{\pgfqpoint{5.601259in}{3.003909in}}%
\pgfpathlineto{\pgfqpoint{5.601259in}{3.003909in}}%
\pgfpathclose%
\pgfusepath{stroke,fill}%
\end{pgfscope}%
\begin{pgfscope}%
\pgfpathrectangle{\pgfqpoint{0.050000in}{0.050000in}}{\pgfqpoint{2.419000in}{2.419000in}}%
\pgfusepath{clip}%
\pgfsetbuttcap%
\pgfsetroundjoin%
\definecolor{currentfill}{rgb}{0.800000,0.400000,0.466667}%
\pgfsetfillcolor{currentfill}%
\pgfsetfillopacity{0.351735}%
\pgfsetlinewidth{1.003750pt}%
\definecolor{currentstroke}{rgb}{0.800000,0.400000,0.466667}%
\pgfsetstrokecolor{currentstroke}%
\pgfsetstrokeopacity{0.351735}%
\pgfsetdash{}{0pt}%
\pgfpathmoveto{\pgfqpoint{1.939784in}{3.003909in}}%
\pgfpathcurveto{\pgfqpoint{1.948993in}{3.003909in}}{\pgfqpoint{1.957825in}{3.007568in}}{\pgfqpoint{1.964336in}{3.014079in}}%
\pgfpathcurveto{\pgfqpoint{1.970848in}{3.020590in}}{\pgfqpoint{1.974506in}{3.029423in}}{\pgfqpoint{1.974506in}{3.038631in}}%
\pgfpathcurveto{\pgfqpoint{1.974506in}{3.047840in}}{\pgfqpoint{1.970848in}{3.056672in}}{\pgfqpoint{1.964336in}{3.063184in}}%
\pgfpathcurveto{\pgfqpoint{1.957825in}{3.069695in}}{\pgfqpoint{1.948993in}{3.073354in}}{\pgfqpoint{1.939784in}{3.073354in}}%
\pgfpathcurveto{\pgfqpoint{1.930576in}{3.073354in}}{\pgfqpoint{1.921743in}{3.069695in}}{\pgfqpoint{1.915232in}{3.063184in}}%
\pgfpathcurveto{\pgfqpoint{1.908720in}{3.056672in}}{\pgfqpoint{1.905062in}{3.047840in}}{\pgfqpoint{1.905062in}{3.038631in}}%
\pgfpathcurveto{\pgfqpoint{1.905062in}{3.029423in}}{\pgfqpoint{1.908720in}{3.020590in}}{\pgfqpoint{1.915232in}{3.014079in}}%
\pgfpathcurveto{\pgfqpoint{1.921743in}{3.007568in}}{\pgfqpoint{1.930576in}{3.003909in}}{\pgfqpoint{1.939784in}{3.003909in}}%
\pgfpathlineto{\pgfqpoint{1.939784in}{3.003909in}}%
\pgfpathclose%
\pgfusepath{stroke,fill}%
\end{pgfscope}%
\begin{pgfscope}%
\pgfpathrectangle{\pgfqpoint{0.050000in}{0.050000in}}{\pgfqpoint{2.419000in}{2.419000in}}%
\pgfusepath{clip}%
\pgfsetbuttcap%
\pgfsetroundjoin%
\definecolor{currentfill}{rgb}{0.800000,0.400000,0.466667}%
\pgfsetfillcolor{currentfill}%
\pgfsetfillopacity{0.355506}%
\pgfsetlinewidth{1.003750pt}%
\definecolor{currentstroke}{rgb}{0.800000,0.400000,0.466667}%
\pgfsetstrokecolor{currentstroke}%
\pgfsetstrokeopacity{0.355506}%
\pgfsetdash{}{0pt}%
\pgfpathmoveto{\pgfqpoint{0.441139in}{2.926728in}}%
\pgfpathcurveto{\pgfqpoint{0.450347in}{2.926728in}}{\pgfqpoint{0.459180in}{2.930387in}}{\pgfqpoint{0.465691in}{2.936898in}}%
\pgfpathcurveto{\pgfqpoint{0.472203in}{2.943410in}}{\pgfqpoint{0.475861in}{2.952242in}}{\pgfqpoint{0.475861in}{2.961451in}}%
\pgfpathcurveto{\pgfqpoint{0.475861in}{2.970659in}}{\pgfqpoint{0.472203in}{2.979492in}}{\pgfqpoint{0.465691in}{2.986003in}}%
\pgfpathcurveto{\pgfqpoint{0.459180in}{2.992514in}}{\pgfqpoint{0.450347in}{2.996173in}}{\pgfqpoint{0.441139in}{2.996173in}}%
\pgfpathcurveto{\pgfqpoint{0.431930in}{2.996173in}}{\pgfqpoint{0.423098in}{2.992514in}}{\pgfqpoint{0.416587in}{2.986003in}}%
\pgfpathcurveto{\pgfqpoint{0.410075in}{2.979492in}}{\pgfqpoint{0.406417in}{2.970659in}}{\pgfqpoint{0.406417in}{2.961451in}}%
\pgfpathcurveto{\pgfqpoint{0.406417in}{2.952242in}}{\pgfqpoint{0.410075in}{2.943410in}}{\pgfqpoint{0.416587in}{2.936898in}}%
\pgfpathcurveto{\pgfqpoint{0.423098in}{2.930387in}}{\pgfqpoint{0.431930in}{2.926728in}}{\pgfqpoint{0.441139in}{2.926728in}}%
\pgfpathlineto{\pgfqpoint{0.441139in}{2.926728in}}%
\pgfpathclose%
\pgfusepath{stroke,fill}%
\end{pgfscope}%
\begin{pgfscope}%
\pgfpathrectangle{\pgfqpoint{0.050000in}{0.050000in}}{\pgfqpoint{2.419000in}{2.419000in}}%
\pgfusepath{clip}%
\pgfsetbuttcap%
\pgfsetroundjoin%
\definecolor{currentfill}{rgb}{0.800000,0.400000,0.466667}%
\pgfsetfillcolor{currentfill}%
\pgfsetfillopacity{0.355506}%
\pgfsetlinewidth{1.003750pt}%
\definecolor{currentstroke}{rgb}{0.800000,0.400000,0.466667}%
\pgfsetstrokecolor{currentstroke}%
\pgfsetstrokeopacity{0.355506}%
\pgfsetdash{}{0pt}%
\pgfpathmoveto{\pgfqpoint{4.151314in}{2.926728in}}%
\pgfpathcurveto{\pgfqpoint{4.160522in}{2.926728in}}{\pgfqpoint{4.169355in}{2.930387in}}{\pgfqpoint{4.175866in}{2.936898in}}%
\pgfpathcurveto{\pgfqpoint{4.182378in}{2.943410in}}{\pgfqpoint{4.186036in}{2.952242in}}{\pgfqpoint{4.186036in}{2.961451in}}%
\pgfpathcurveto{\pgfqpoint{4.186036in}{2.970659in}}{\pgfqpoint{4.182378in}{2.979492in}}{\pgfqpoint{4.175866in}{2.986003in}}%
\pgfpathcurveto{\pgfqpoint{4.169355in}{2.992514in}}{\pgfqpoint{4.160522in}{2.996173in}}{\pgfqpoint{4.151314in}{2.996173in}}%
\pgfpathcurveto{\pgfqpoint{4.142105in}{2.996173in}}{\pgfqpoint{4.133273in}{2.992514in}}{\pgfqpoint{4.126762in}{2.986003in}}%
\pgfpathcurveto{\pgfqpoint{4.120250in}{2.979492in}}{\pgfqpoint{4.116592in}{2.970659in}}{\pgfqpoint{4.116592in}{2.961451in}}%
\pgfpathcurveto{\pgfqpoint{4.116592in}{2.952242in}}{\pgfqpoint{4.120250in}{2.943410in}}{\pgfqpoint{4.126762in}{2.936898in}}%
\pgfpathcurveto{\pgfqpoint{4.133273in}{2.930387in}}{\pgfqpoint{4.142105in}{2.926728in}}{\pgfqpoint{4.151314in}{2.926728in}}%
\pgfpathlineto{\pgfqpoint{4.151314in}{2.926728in}}%
\pgfpathclose%
\pgfusepath{stroke,fill}%
\end{pgfscope}%
\begin{pgfscope}%
\pgfpathrectangle{\pgfqpoint{0.050000in}{0.050000in}}{\pgfqpoint{2.419000in}{2.419000in}}%
\pgfusepath{clip}%
\pgfsetbuttcap%
\pgfsetroundjoin%
\definecolor{currentfill}{rgb}{0.800000,0.400000,0.466667}%
\pgfsetfillcolor{currentfill}%
\pgfsetfillopacity{0.355506}%
\pgfsetlinewidth{1.003750pt}%
\definecolor{currentstroke}{rgb}{0.800000,0.400000,0.466667}%
\pgfsetstrokecolor{currentstroke}%
\pgfsetstrokeopacity{0.355506}%
\pgfsetdash{}{0pt}%
\pgfpathmoveto{\pgfqpoint{7.861489in}{2.926728in}}%
\pgfpathcurveto{\pgfqpoint{7.870697in}{2.926728in}}{\pgfqpoint{7.879530in}{2.930387in}}{\pgfqpoint{7.886041in}{2.936898in}}%
\pgfpathcurveto{\pgfqpoint{7.892552in}{2.943410in}}{\pgfqpoint{7.896211in}{2.952242in}}{\pgfqpoint{7.896211in}{2.961451in}}%
\pgfpathcurveto{\pgfqpoint{7.896211in}{2.970659in}}{\pgfqpoint{7.892552in}{2.979492in}}{\pgfqpoint{7.886041in}{2.986003in}}%
\pgfpathcurveto{\pgfqpoint{7.879530in}{2.992514in}}{\pgfqpoint{7.870697in}{2.996173in}}{\pgfqpoint{7.861489in}{2.996173in}}%
\pgfpathcurveto{\pgfqpoint{7.852280in}{2.996173in}}{\pgfqpoint{7.843448in}{2.992514in}}{\pgfqpoint{7.836936in}{2.986003in}}%
\pgfpathcurveto{\pgfqpoint{7.830425in}{2.979492in}}{\pgfqpoint{7.826767in}{2.970659in}}{\pgfqpoint{7.826767in}{2.961451in}}%
\pgfpathcurveto{\pgfqpoint{7.826767in}{2.952242in}}{\pgfqpoint{7.830425in}{2.943410in}}{\pgfqpoint{7.836936in}{2.936898in}}%
\pgfpathcurveto{\pgfqpoint{7.843448in}{2.930387in}}{\pgfqpoint{7.852280in}{2.926728in}}{\pgfqpoint{7.861489in}{2.926728in}}%
\pgfpathlineto{\pgfqpoint{7.861489in}{2.926728in}}%
\pgfpathclose%
\pgfusepath{stroke,fill}%
\end{pgfscope}%
\begin{pgfscope}%
\pgfpathrectangle{\pgfqpoint{0.050000in}{0.050000in}}{\pgfqpoint{2.419000in}{2.419000in}}%
\pgfusepath{clip}%
\pgfsetbuttcap%
\pgfsetroundjoin%
\definecolor{currentfill}{rgb}{0.800000,0.400000,0.466667}%
\pgfsetfillcolor{currentfill}%
\pgfsetfillopacity{0.359380}%
\pgfsetlinewidth{1.003750pt}%
\definecolor{currentstroke}{rgb}{0.800000,0.400000,0.466667}%
\pgfsetstrokecolor{currentstroke}%
\pgfsetstrokeopacity{0.359380}%
\pgfsetdash{}{0pt}%
\pgfpathmoveto{\pgfqpoint{2.662278in}{2.847467in}}%
\pgfpathcurveto{\pgfqpoint{2.671487in}{2.847467in}}{\pgfqpoint{2.680319in}{2.851125in}}{\pgfqpoint{2.686831in}{2.857637in}}%
\pgfpathcurveto{\pgfqpoint{2.693342in}{2.864148in}}{\pgfqpoint{2.697001in}{2.872981in}}{\pgfqpoint{2.697001in}{2.882189in}}%
\pgfpathcurveto{\pgfqpoint{2.697001in}{2.891397in}}{\pgfqpoint{2.693342in}{2.900230in}}{\pgfqpoint{2.686831in}{2.906741in}}%
\pgfpathcurveto{\pgfqpoint{2.680319in}{2.913253in}}{\pgfqpoint{2.671487in}{2.916911in}}{\pgfqpoint{2.662278in}{2.916911in}}%
\pgfpathcurveto{\pgfqpoint{2.653070in}{2.916911in}}{\pgfqpoint{2.644237in}{2.913253in}}{\pgfqpoint{2.637726in}{2.906741in}}%
\pgfpathcurveto{\pgfqpoint{2.631215in}{2.900230in}}{\pgfqpoint{2.627556in}{2.891397in}}{\pgfqpoint{2.627556in}{2.882189in}}%
\pgfpathcurveto{\pgfqpoint{2.627556in}{2.872981in}}{\pgfqpoint{2.631215in}{2.864148in}}{\pgfqpoint{2.637726in}{2.857637in}}%
\pgfpathcurveto{\pgfqpoint{2.644237in}{2.851125in}}{\pgfqpoint{2.653070in}{2.847467in}}{\pgfqpoint{2.662278in}{2.847467in}}%
\pgfpathlineto{\pgfqpoint{2.662278in}{2.847467in}}%
\pgfpathclose%
\pgfusepath{stroke,fill}%
\end{pgfscope}%
\begin{pgfscope}%
\pgfpathrectangle{\pgfqpoint{0.050000in}{0.050000in}}{\pgfqpoint{2.419000in}{2.419000in}}%
\pgfusepath{clip}%
\pgfsetbuttcap%
\pgfsetroundjoin%
\definecolor{currentfill}{rgb}{0.800000,0.400000,0.466667}%
\pgfsetfillcolor{currentfill}%
\pgfsetfillopacity{0.359380}%
\pgfsetlinewidth{1.003750pt}%
\definecolor{currentstroke}{rgb}{0.800000,0.400000,0.466667}%
\pgfsetstrokecolor{currentstroke}%
\pgfsetstrokeopacity{0.359380}%
\pgfsetdash{}{0pt}%
\pgfpathmoveto{\pgfqpoint{6.422466in}{2.847467in}}%
\pgfpathcurveto{\pgfqpoint{6.431674in}{2.847467in}}{\pgfqpoint{6.440507in}{2.851125in}}{\pgfqpoint{6.447018in}{2.857637in}}%
\pgfpathcurveto{\pgfqpoint{6.453530in}{2.864148in}}{\pgfqpoint{6.457188in}{2.872981in}}{\pgfqpoint{6.457188in}{2.882189in}}%
\pgfpathcurveto{\pgfqpoint{6.457188in}{2.891397in}}{\pgfqpoint{6.453530in}{2.900230in}}{\pgfqpoint{6.447018in}{2.906741in}}%
\pgfpathcurveto{\pgfqpoint{6.440507in}{2.913253in}}{\pgfqpoint{6.431674in}{2.916911in}}{\pgfqpoint{6.422466in}{2.916911in}}%
\pgfpathcurveto{\pgfqpoint{6.413258in}{2.916911in}}{\pgfqpoint{6.404425in}{2.913253in}}{\pgfqpoint{6.397914in}{2.906741in}}%
\pgfpathcurveto{\pgfqpoint{6.391402in}{2.900230in}}{\pgfqpoint{6.387744in}{2.891397in}}{\pgfqpoint{6.387744in}{2.882189in}}%
\pgfpathcurveto{\pgfqpoint{6.387744in}{2.872981in}}{\pgfqpoint{6.391402in}{2.864148in}}{\pgfqpoint{6.397914in}{2.857637in}}%
\pgfpathcurveto{\pgfqpoint{6.404425in}{2.851125in}}{\pgfqpoint{6.413258in}{2.847467in}}{\pgfqpoint{6.422466in}{2.847467in}}%
\pgfpathlineto{\pgfqpoint{6.422466in}{2.847467in}}%
\pgfpathclose%
\pgfusepath{stroke,fill}%
\end{pgfscope}%
\begin{pgfscope}%
\pgfpathrectangle{\pgfqpoint{0.050000in}{0.050000in}}{\pgfqpoint{2.419000in}{2.419000in}}%
\pgfusepath{clip}%
\pgfsetbuttcap%
\pgfsetroundjoin%
\definecolor{currentfill}{rgb}{0.800000,0.400000,0.466667}%
\pgfsetfillcolor{currentfill}%
\pgfsetfillopacity{0.363359}%
\pgfsetlinewidth{1.003750pt}%
\definecolor{currentstroke}{rgb}{0.800000,0.400000,0.466667}%
\pgfsetstrokecolor{currentstroke}%
\pgfsetstrokeopacity{0.363359}%
\pgfsetdash{}{0pt}%
\pgfpathmoveto{\pgfqpoint{1.132550in}{2.766039in}}%
\pgfpathcurveto{\pgfqpoint{1.141759in}{2.766039in}}{\pgfqpoint{1.150591in}{2.769698in}}{\pgfqpoint{1.157102in}{2.776209in}}%
\pgfpathcurveto{\pgfqpoint{1.163614in}{2.782720in}}{\pgfqpoint{1.167272in}{2.791553in}}{\pgfqpoint{1.167272in}{2.800761in}}%
\pgfpathcurveto{\pgfqpoint{1.167272in}{2.809970in}}{\pgfqpoint{1.163614in}{2.818802in}}{\pgfqpoint{1.157102in}{2.825314in}}%
\pgfpathcurveto{\pgfqpoint{1.150591in}{2.831825in}}{\pgfqpoint{1.141759in}{2.835484in}}{\pgfqpoint{1.132550in}{2.835484in}}%
\pgfpathcurveto{\pgfqpoint{1.123342in}{2.835484in}}{\pgfqpoint{1.114509in}{2.831825in}}{\pgfqpoint{1.107998in}{2.825314in}}%
\pgfpathcurveto{\pgfqpoint{1.101486in}{2.818802in}}{\pgfqpoint{1.097828in}{2.809970in}}{\pgfqpoint{1.097828in}{2.800761in}}%
\pgfpathcurveto{\pgfqpoint{1.097828in}{2.791553in}}{\pgfqpoint{1.101486in}{2.782720in}}{\pgfqpoint{1.107998in}{2.776209in}}%
\pgfpathcurveto{\pgfqpoint{1.114509in}{2.769698in}}{\pgfqpoint{1.123342in}{2.766039in}}{\pgfqpoint{1.132550in}{2.766039in}}%
\pgfpathlineto{\pgfqpoint{1.132550in}{2.766039in}}%
\pgfpathclose%
\pgfusepath{stroke,fill}%
\end{pgfscope}%
\begin{pgfscope}%
\pgfpathrectangle{\pgfqpoint{0.050000in}{0.050000in}}{\pgfqpoint{2.419000in}{2.419000in}}%
\pgfusepath{clip}%
\pgfsetbuttcap%
\pgfsetroundjoin%
\definecolor{currentfill}{rgb}{0.800000,0.400000,0.466667}%
\pgfsetfillcolor{currentfill}%
\pgfsetfillopacity{0.363359}%
\pgfsetlinewidth{1.003750pt}%
\definecolor{currentstroke}{rgb}{0.800000,0.400000,0.466667}%
\pgfsetstrokecolor{currentstroke}%
\pgfsetstrokeopacity{0.363359}%
\pgfsetdash{}{0pt}%
\pgfpathmoveto{\pgfqpoint{4.944117in}{2.766039in}}%
\pgfpathcurveto{\pgfqpoint{4.953326in}{2.766039in}}{\pgfqpoint{4.962158in}{2.769698in}}{\pgfqpoint{4.968670in}{2.776209in}}%
\pgfpathcurveto{\pgfqpoint{4.975181in}{2.782720in}}{\pgfqpoint{4.978840in}{2.791553in}}{\pgfqpoint{4.978840in}{2.800761in}}%
\pgfpathcurveto{\pgfqpoint{4.978840in}{2.809970in}}{\pgfqpoint{4.975181in}{2.818802in}}{\pgfqpoint{4.968670in}{2.825314in}}%
\pgfpathcurveto{\pgfqpoint{4.962158in}{2.831825in}}{\pgfqpoint{4.953326in}{2.835484in}}{\pgfqpoint{4.944117in}{2.835484in}}%
\pgfpathcurveto{\pgfqpoint{4.934909in}{2.835484in}}{\pgfqpoint{4.926076in}{2.831825in}}{\pgfqpoint{4.919565in}{2.825314in}}%
\pgfpathcurveto{\pgfqpoint{4.913054in}{2.818802in}}{\pgfqpoint{4.909395in}{2.809970in}}{\pgfqpoint{4.909395in}{2.800761in}}%
\pgfpathcurveto{\pgfqpoint{4.909395in}{2.791553in}}{\pgfqpoint{4.913054in}{2.782720in}}{\pgfqpoint{4.919565in}{2.776209in}}%
\pgfpathcurveto{\pgfqpoint{4.926076in}{2.769698in}}{\pgfqpoint{4.934909in}{2.766039in}}{\pgfqpoint{4.944117in}{2.766039in}}%
\pgfpathlineto{\pgfqpoint{4.944117in}{2.766039in}}%
\pgfpathclose%
\pgfusepath{stroke,fill}%
\end{pgfscope}%
\begin{pgfscope}%
\pgfpathrectangle{\pgfqpoint{0.050000in}{0.050000in}}{\pgfqpoint{2.419000in}{2.419000in}}%
\pgfusepath{clip}%
\pgfsetbuttcap%
\pgfsetroundjoin%
\definecolor{currentfill}{rgb}{0.800000,0.400000,0.466667}%
\pgfsetfillcolor{currentfill}%
\pgfsetfillopacity{0.363359}%
\pgfsetlinewidth{1.003750pt}%
\definecolor{currentstroke}{rgb}{0.800000,0.400000,0.466667}%
\pgfsetstrokecolor{currentstroke}%
\pgfsetstrokeopacity{0.363359}%
\pgfsetdash{}{0pt}%
\pgfpathmoveto{\pgfqpoint{8.755685in}{2.766039in}}%
\pgfpathcurveto{\pgfqpoint{8.764893in}{2.766039in}}{\pgfqpoint{8.773726in}{2.769698in}}{\pgfqpoint{8.780237in}{2.776209in}}%
\pgfpathcurveto{\pgfqpoint{8.786748in}{2.782720in}}{\pgfqpoint{8.790407in}{2.791553in}}{\pgfqpoint{8.790407in}{2.800761in}}%
\pgfpathcurveto{\pgfqpoint{8.790407in}{2.809970in}}{\pgfqpoint{8.786748in}{2.818802in}}{\pgfqpoint{8.780237in}{2.825314in}}%
\pgfpathcurveto{\pgfqpoint{8.773726in}{2.831825in}}{\pgfqpoint{8.764893in}{2.835484in}}{\pgfqpoint{8.755685in}{2.835484in}}%
\pgfpathcurveto{\pgfqpoint{8.746476in}{2.835484in}}{\pgfqpoint{8.737644in}{2.831825in}}{\pgfqpoint{8.731132in}{2.825314in}}%
\pgfpathcurveto{\pgfqpoint{8.724621in}{2.818802in}}{\pgfqpoint{8.720962in}{2.809970in}}{\pgfqpoint{8.720962in}{2.800761in}}%
\pgfpathcurveto{\pgfqpoint{8.720962in}{2.791553in}}{\pgfqpoint{8.724621in}{2.782720in}}{\pgfqpoint{8.731132in}{2.776209in}}%
\pgfpathcurveto{\pgfqpoint{8.737644in}{2.769698in}}{\pgfqpoint{8.746476in}{2.766039in}}{\pgfqpoint{8.755685in}{2.766039in}}%
\pgfpathlineto{\pgfqpoint{8.755685in}{2.766039in}}%
\pgfpathclose%
\pgfusepath{stroke,fill}%
\end{pgfscope}%
\begin{pgfscope}%
\pgfpathrectangle{\pgfqpoint{0.050000in}{0.050000in}}{\pgfqpoint{2.419000in}{2.419000in}}%
\pgfusepath{clip}%
\pgfsetbuttcap%
\pgfsetroundjoin%
\definecolor{currentfill}{rgb}{0.800000,0.400000,0.466667}%
\pgfsetfillcolor{currentfill}%
\pgfsetfillopacity{0.367449}%
\pgfsetlinewidth{1.003750pt}%
\definecolor{currentstroke}{rgb}{0.800000,0.400000,0.466667}%
\pgfsetstrokecolor{currentstroke}%
\pgfsetstrokeopacity{0.367449}%
\pgfsetdash{}{0pt}%
\pgfpathmoveto{\pgfqpoint{-0.439562in}{2.682355in}}%
\pgfpathcurveto{\pgfqpoint{-0.430353in}{2.682355in}}{\pgfqpoint{-0.421521in}{2.686014in}}{\pgfqpoint{-0.415010in}{2.692525in}}%
\pgfpathcurveto{\pgfqpoint{-0.408498in}{2.699037in}}{\pgfqpoint{-0.404840in}{2.707869in}}{\pgfqpoint{-0.404840in}{2.717078in}}%
\pgfpathcurveto{\pgfqpoint{-0.404840in}{2.726286in}}{\pgfqpoint{-0.408498in}{2.735119in}}{\pgfqpoint{-0.415010in}{2.741630in}}%
\pgfpathcurveto{\pgfqpoint{-0.421521in}{2.748141in}}{\pgfqpoint{-0.430353in}{2.751800in}}{\pgfqpoint{-0.439562in}{2.751800in}}%
\pgfpathcurveto{\pgfqpoint{-0.448770in}{2.751800in}}{\pgfqpoint{-0.457603in}{2.748141in}}{\pgfqpoint{-0.464114in}{2.741630in}}%
\pgfpathcurveto{\pgfqpoint{-0.470626in}{2.735119in}}{\pgfqpoint{-0.474284in}{2.726286in}}{\pgfqpoint{-0.474284in}{2.717078in}}%
\pgfpathcurveto{\pgfqpoint{-0.474284in}{2.707869in}}{\pgfqpoint{-0.470626in}{2.699037in}}{\pgfqpoint{-0.464114in}{2.692525in}}%
\pgfpathcurveto{\pgfqpoint{-0.457603in}{2.686014in}}{\pgfqpoint{-0.448770in}{2.682355in}}{\pgfqpoint{-0.439562in}{2.682355in}}%
\pgfpathlineto{\pgfqpoint{-0.439562in}{2.682355in}}%
\pgfpathclose%
\pgfusepath{stroke,fill}%
\end{pgfscope}%
\begin{pgfscope}%
\pgfpathrectangle{\pgfqpoint{0.050000in}{0.050000in}}{\pgfqpoint{2.419000in}{2.419000in}}%
\pgfusepath{clip}%
\pgfsetbuttcap%
\pgfsetroundjoin%
\definecolor{currentfill}{rgb}{0.800000,0.400000,0.466667}%
\pgfsetfillcolor{currentfill}%
\pgfsetfillopacity{0.367449}%
\pgfsetlinewidth{1.003750pt}%
\definecolor{currentstroke}{rgb}{0.800000,0.400000,0.466667}%
\pgfsetstrokecolor{currentstroke}%
\pgfsetstrokeopacity{0.367449}%
\pgfsetdash{}{0pt}%
\pgfpathmoveto{\pgfqpoint{3.424808in}{2.682355in}}%
\pgfpathcurveto{\pgfqpoint{3.434017in}{2.682355in}}{\pgfqpoint{3.442849in}{2.686014in}}{\pgfqpoint{3.449361in}{2.692525in}}%
\pgfpathcurveto{\pgfqpoint{3.455872in}{2.699037in}}{\pgfqpoint{3.459531in}{2.707869in}}{\pgfqpoint{3.459531in}{2.717078in}}%
\pgfpathcurveto{\pgfqpoint{3.459531in}{2.726286in}}{\pgfqpoint{3.455872in}{2.735119in}}{\pgfqpoint{3.449361in}{2.741630in}}%
\pgfpathcurveto{\pgfqpoint{3.442849in}{2.748141in}}{\pgfqpoint{3.434017in}{2.751800in}}{\pgfqpoint{3.424808in}{2.751800in}}%
\pgfpathcurveto{\pgfqpoint{3.415600in}{2.751800in}}{\pgfqpoint{3.406768in}{2.748141in}}{\pgfqpoint{3.400256in}{2.741630in}}%
\pgfpathcurveto{\pgfqpoint{3.393745in}{2.735119in}}{\pgfqpoint{3.390086in}{2.726286in}}{\pgfqpoint{3.390086in}{2.717078in}}%
\pgfpathcurveto{\pgfqpoint{3.390086in}{2.707869in}}{\pgfqpoint{3.393745in}{2.699037in}}{\pgfqpoint{3.400256in}{2.692525in}}%
\pgfpathcurveto{\pgfqpoint{3.406768in}{2.686014in}}{\pgfqpoint{3.415600in}{2.682355in}}{\pgfqpoint{3.424808in}{2.682355in}}%
\pgfpathlineto{\pgfqpoint{3.424808in}{2.682355in}}%
\pgfpathclose%
\pgfusepath{stroke,fill}%
\end{pgfscope}%
\begin{pgfscope}%
\pgfpathrectangle{\pgfqpoint{0.050000in}{0.050000in}}{\pgfqpoint{2.419000in}{2.419000in}}%
\pgfusepath{clip}%
\pgfsetbuttcap%
\pgfsetroundjoin%
\definecolor{currentfill}{rgb}{0.800000,0.400000,0.466667}%
\pgfsetfillcolor{currentfill}%
\pgfsetfillopacity{0.367449}%
\pgfsetlinewidth{1.003750pt}%
\definecolor{currentstroke}{rgb}{0.800000,0.400000,0.466667}%
\pgfsetstrokecolor{currentstroke}%
\pgfsetstrokeopacity{0.367449}%
\pgfsetdash{}{0pt}%
\pgfpathmoveto{\pgfqpoint{7.289179in}{2.682355in}}%
\pgfpathcurveto{\pgfqpoint{7.298387in}{2.682355in}}{\pgfqpoint{7.307220in}{2.686014in}}{\pgfqpoint{7.313731in}{2.692525in}}%
\pgfpathcurveto{\pgfqpoint{7.320243in}{2.699037in}}{\pgfqpoint{7.323901in}{2.707869in}}{\pgfqpoint{7.323901in}{2.717078in}}%
\pgfpathcurveto{\pgfqpoint{7.323901in}{2.726286in}}{\pgfqpoint{7.320243in}{2.735119in}}{\pgfqpoint{7.313731in}{2.741630in}}%
\pgfpathcurveto{\pgfqpoint{7.307220in}{2.748141in}}{\pgfqpoint{7.298387in}{2.751800in}}{\pgfqpoint{7.289179in}{2.751800in}}%
\pgfpathcurveto{\pgfqpoint{7.279970in}{2.751800in}}{\pgfqpoint{7.271138in}{2.748141in}}{\pgfqpoint{7.264627in}{2.741630in}}%
\pgfpathcurveto{\pgfqpoint{7.258115in}{2.735119in}}{\pgfqpoint{7.254457in}{2.726286in}}{\pgfqpoint{7.254457in}{2.717078in}}%
\pgfpathcurveto{\pgfqpoint{7.254457in}{2.707869in}}{\pgfqpoint{7.258115in}{2.699037in}}{\pgfqpoint{7.264627in}{2.692525in}}%
\pgfpathcurveto{\pgfqpoint{7.271138in}{2.686014in}}{\pgfqpoint{7.279970in}{2.682355in}}{\pgfqpoint{7.289179in}{2.682355in}}%
\pgfpathlineto{\pgfqpoint{7.289179in}{2.682355in}}%
\pgfpathclose%
\pgfusepath{stroke,fill}%
\end{pgfscope}%
\begin{pgfscope}%
\pgfpathrectangle{\pgfqpoint{0.050000in}{0.050000in}}{\pgfqpoint{2.419000in}{2.419000in}}%
\pgfusepath{clip}%
\pgfsetbuttcap%
\pgfsetroundjoin%
\definecolor{currentfill}{rgb}{0.800000,0.400000,0.466667}%
\pgfsetfillcolor{currentfill}%
\pgfsetfillopacity{0.371653}%
\pgfsetlinewidth{1.003750pt}%
\definecolor{currentstroke}{rgb}{0.800000,0.400000,0.466667}%
\pgfsetstrokecolor{currentstroke}%
\pgfsetstrokeopacity{0.371653}%
\pgfsetdash{}{0pt}%
\pgfpathmoveto{\pgfqpoint{1.862813in}{2.596320in}}%
\pgfpathcurveto{\pgfqpoint{1.872022in}{2.596320in}}{\pgfqpoint{1.880854in}{2.599979in}}{\pgfqpoint{1.887365in}{2.606490in}}%
\pgfpathcurveto{\pgfqpoint{1.893877in}{2.613002in}}{\pgfqpoint{1.897535in}{2.621834in}}{\pgfqpoint{1.897535in}{2.631043in}}%
\pgfpathcurveto{\pgfqpoint{1.897535in}{2.640251in}}{\pgfqpoint{1.893877in}{2.649084in}}{\pgfqpoint{1.887365in}{2.655595in}}%
\pgfpathcurveto{\pgfqpoint{1.880854in}{2.662106in}}{\pgfqpoint{1.872022in}{2.665765in}}{\pgfqpoint{1.862813in}{2.665765in}}%
\pgfpathcurveto{\pgfqpoint{1.853605in}{2.665765in}}{\pgfqpoint{1.844772in}{2.662106in}}{\pgfqpoint{1.838261in}{2.655595in}}%
\pgfpathcurveto{\pgfqpoint{1.831749in}{2.649084in}}{\pgfqpoint{1.828091in}{2.640251in}}{\pgfqpoint{1.828091in}{2.631043in}}%
\pgfpathcurveto{\pgfqpoint{1.828091in}{2.621834in}}{\pgfqpoint{1.831749in}{2.613002in}}{\pgfqpoint{1.838261in}{2.606490in}}%
\pgfpathcurveto{\pgfqpoint{1.844772in}{2.599979in}}{\pgfqpoint{1.853605in}{2.596320in}}{\pgfqpoint{1.862813in}{2.596320in}}%
\pgfpathlineto{\pgfqpoint{1.862813in}{2.596320in}}%
\pgfpathclose%
\pgfusepath{stroke,fill}%
\end{pgfscope}%
\begin{pgfscope}%
\pgfpathrectangle{\pgfqpoint{0.050000in}{0.050000in}}{\pgfqpoint{2.419000in}{2.419000in}}%
\pgfusepath{clip}%
\pgfsetbuttcap%
\pgfsetroundjoin%
\definecolor{currentfill}{rgb}{0.800000,0.400000,0.466667}%
\pgfsetfillcolor{currentfill}%
\pgfsetfillopacity{0.371653}%
\pgfsetlinewidth{1.003750pt}%
\definecolor{currentstroke}{rgb}{0.800000,0.400000,0.466667}%
\pgfsetstrokecolor{currentstroke}%
\pgfsetstrokeopacity{0.371653}%
\pgfsetdash{}{0pt}%
\pgfpathmoveto{\pgfqpoint{5.781470in}{2.596320in}}%
\pgfpathcurveto{\pgfqpoint{5.790679in}{2.596320in}}{\pgfqpoint{5.799511in}{2.599979in}}{\pgfqpoint{5.806022in}{2.606490in}}%
\pgfpathcurveto{\pgfqpoint{5.812534in}{2.613002in}}{\pgfqpoint{5.816192in}{2.621834in}}{\pgfqpoint{5.816192in}{2.631043in}}%
\pgfpathcurveto{\pgfqpoint{5.816192in}{2.640251in}}{\pgfqpoint{5.812534in}{2.649084in}}{\pgfqpoint{5.806022in}{2.655595in}}%
\pgfpathcurveto{\pgfqpoint{5.799511in}{2.662106in}}{\pgfqpoint{5.790679in}{2.665765in}}{\pgfqpoint{5.781470in}{2.665765in}}%
\pgfpathcurveto{\pgfqpoint{5.772262in}{2.665765in}}{\pgfqpoint{5.763429in}{2.662106in}}{\pgfqpoint{5.756918in}{2.655595in}}%
\pgfpathcurveto{\pgfqpoint{5.750406in}{2.649084in}}{\pgfqpoint{5.746748in}{2.640251in}}{\pgfqpoint{5.746748in}{2.631043in}}%
\pgfpathcurveto{\pgfqpoint{5.746748in}{2.621834in}}{\pgfqpoint{5.750406in}{2.613002in}}{\pgfqpoint{5.756918in}{2.606490in}}%
\pgfpathcurveto{\pgfqpoint{5.763429in}{2.599979in}}{\pgfqpoint{5.772262in}{2.596320in}}{\pgfqpoint{5.781470in}{2.596320in}}%
\pgfpathlineto{\pgfqpoint{5.781470in}{2.596320in}}%
\pgfpathclose%
\pgfusepath{stroke,fill}%
\end{pgfscope}%
\begin{pgfscope}%
\pgfpathrectangle{\pgfqpoint{0.050000in}{0.050000in}}{\pgfqpoint{2.419000in}{2.419000in}}%
\pgfusepath{clip}%
\pgfsetbuttcap%
\pgfsetroundjoin%
\definecolor{currentfill}{rgb}{0.800000,0.400000,0.466667}%
\pgfsetfillcolor{currentfill}%
\pgfsetfillopacity{0.371653}%
\pgfsetlinewidth{1.003750pt}%
\definecolor{currentstroke}{rgb}{0.800000,0.400000,0.466667}%
\pgfsetstrokecolor{currentstroke}%
\pgfsetstrokeopacity{0.371653}%
\pgfsetdash{}{0pt}%
\pgfpathmoveto{\pgfqpoint{9.700127in}{2.596320in}}%
\pgfpathcurveto{\pgfqpoint{9.709336in}{2.596320in}}{\pgfqpoint{9.718168in}{2.599979in}}{\pgfqpoint{9.724679in}{2.606490in}}%
\pgfpathcurveto{\pgfqpoint{9.731191in}{2.613002in}}{\pgfqpoint{9.734849in}{2.621834in}}{\pgfqpoint{9.734849in}{2.631043in}}%
\pgfpathcurveto{\pgfqpoint{9.734849in}{2.640251in}}{\pgfqpoint{9.731191in}{2.649084in}}{\pgfqpoint{9.724679in}{2.655595in}}%
\pgfpathcurveto{\pgfqpoint{9.718168in}{2.662106in}}{\pgfqpoint{9.709336in}{2.665765in}}{\pgfqpoint{9.700127in}{2.665765in}}%
\pgfpathcurveto{\pgfqpoint{9.690919in}{2.665765in}}{\pgfqpoint{9.682086in}{2.662106in}}{\pgfqpoint{9.675575in}{2.655595in}}%
\pgfpathcurveto{\pgfqpoint{9.669063in}{2.649084in}}{\pgfqpoint{9.665405in}{2.640251in}}{\pgfqpoint{9.665405in}{2.631043in}}%
\pgfpathcurveto{\pgfqpoint{9.665405in}{2.621834in}}{\pgfqpoint{9.669063in}{2.613002in}}{\pgfqpoint{9.675575in}{2.606490in}}%
\pgfpathcurveto{\pgfqpoint{9.682086in}{2.599979in}}{\pgfqpoint{9.690919in}{2.596320in}}{\pgfqpoint{9.700127in}{2.596320in}}%
\pgfpathlineto{\pgfqpoint{9.700127in}{2.596320in}}%
\pgfpathclose%
\pgfusepath{stroke,fill}%
\end{pgfscope}%
\begin{pgfscope}%
\pgfpathrectangle{\pgfqpoint{0.050000in}{0.050000in}}{\pgfqpoint{2.419000in}{2.419000in}}%
\pgfusepath{clip}%
\pgfsetbuttcap%
\pgfsetroundjoin%
\definecolor{currentfill}{rgb}{0.800000,0.400000,0.466667}%
\pgfsetfillcolor{currentfill}%
\pgfsetfillopacity{0.375977}%
\pgfsetlinewidth{1.003750pt}%
\definecolor{currentstroke}{rgb}{0.800000,0.400000,0.466667}%
\pgfsetstrokecolor{currentstroke}%
\pgfsetstrokeopacity{0.375977}%
\pgfsetdash{}{0pt}%
\pgfpathmoveto{\pgfqpoint{4.230797in}{2.507834in}}%
\pgfpathcurveto{\pgfqpoint{4.240006in}{2.507834in}}{\pgfqpoint{4.248838in}{2.511492in}}{\pgfqpoint{4.255350in}{2.518004in}}%
\pgfpathcurveto{\pgfqpoint{4.261861in}{2.524515in}}{\pgfqpoint{4.265519in}{2.533348in}}{\pgfqpoint{4.265519in}{2.542556in}}%
\pgfpathcurveto{\pgfqpoint{4.265519in}{2.551764in}}{\pgfqpoint{4.261861in}{2.560597in}}{\pgfqpoint{4.255350in}{2.567108in}}%
\pgfpathcurveto{\pgfqpoint{4.248838in}{2.573620in}}{\pgfqpoint{4.240006in}{2.577278in}}{\pgfqpoint{4.230797in}{2.577278in}}%
\pgfpathcurveto{\pgfqpoint{4.221589in}{2.577278in}}{\pgfqpoint{4.212756in}{2.573620in}}{\pgfqpoint{4.206245in}{2.567108in}}%
\pgfpathcurveto{\pgfqpoint{4.199734in}{2.560597in}}{\pgfqpoint{4.196075in}{2.551764in}}{\pgfqpoint{4.196075in}{2.542556in}}%
\pgfpathcurveto{\pgfqpoint{4.196075in}{2.533348in}}{\pgfqpoint{4.199734in}{2.524515in}}{\pgfqpoint{4.206245in}{2.518004in}}%
\pgfpathcurveto{\pgfqpoint{4.212756in}{2.511492in}}{\pgfqpoint{4.221589in}{2.507834in}}{\pgfqpoint{4.230797in}{2.507834in}}%
\pgfpathlineto{\pgfqpoint{4.230797in}{2.507834in}}%
\pgfpathclose%
\pgfusepath{stroke,fill}%
\end{pgfscope}%
\begin{pgfscope}%
\pgfpathrectangle{\pgfqpoint{0.050000in}{0.050000in}}{\pgfqpoint{2.419000in}{2.419000in}}%
\pgfusepath{clip}%
\pgfsetbuttcap%
\pgfsetroundjoin%
\definecolor{currentfill}{rgb}{0.800000,0.400000,0.466667}%
\pgfsetfillcolor{currentfill}%
\pgfsetfillopacity{0.375977}%
\pgfsetlinewidth{1.003750pt}%
\definecolor{currentstroke}{rgb}{0.800000,0.400000,0.466667}%
\pgfsetstrokecolor{currentstroke}%
\pgfsetstrokeopacity{0.375977}%
\pgfsetdash{}{0pt}%
\pgfpathmoveto{\pgfqpoint{0.256307in}{2.507834in}}%
\pgfpathcurveto{\pgfqpoint{0.265515in}{2.507834in}}{\pgfqpoint{0.274348in}{2.511492in}}{\pgfqpoint{0.280859in}{2.518004in}}%
\pgfpathcurveto{\pgfqpoint{0.287370in}{2.524515in}}{\pgfqpoint{0.291029in}{2.533348in}}{\pgfqpoint{0.291029in}{2.542556in}}%
\pgfpathcurveto{\pgfqpoint{0.291029in}{2.551764in}}{\pgfqpoint{0.287370in}{2.560597in}}{\pgfqpoint{0.280859in}{2.567108in}}%
\pgfpathcurveto{\pgfqpoint{0.274348in}{2.573620in}}{\pgfqpoint{0.265515in}{2.577278in}}{\pgfqpoint{0.256307in}{2.577278in}}%
\pgfpathcurveto{\pgfqpoint{0.247098in}{2.577278in}}{\pgfqpoint{0.238266in}{2.573620in}}{\pgfqpoint{0.231754in}{2.567108in}}%
\pgfpathcurveto{\pgfqpoint{0.225243in}{2.560597in}}{\pgfqpoint{0.221584in}{2.551764in}}{\pgfqpoint{0.221584in}{2.542556in}}%
\pgfpathcurveto{\pgfqpoint{0.221584in}{2.533348in}}{\pgfqpoint{0.225243in}{2.524515in}}{\pgfqpoint{0.231754in}{2.518004in}}%
\pgfpathcurveto{\pgfqpoint{0.238266in}{2.511492in}}{\pgfqpoint{0.247098in}{2.507834in}}{\pgfqpoint{0.256307in}{2.507834in}}%
\pgfpathlineto{\pgfqpoint{0.256307in}{2.507834in}}%
\pgfpathclose%
\pgfusepath{stroke,fill}%
\end{pgfscope}%
\begin{pgfscope}%
\pgfpathrectangle{\pgfqpoint{0.050000in}{0.050000in}}{\pgfqpoint{2.419000in}{2.419000in}}%
\pgfusepath{clip}%
\pgfsetbuttcap%
\pgfsetroundjoin%
\definecolor{currentfill}{rgb}{0.800000,0.400000,0.466667}%
\pgfsetfillcolor{currentfill}%
\pgfsetfillopacity{0.375977}%
\pgfsetlinewidth{1.003750pt}%
\definecolor{currentstroke}{rgb}{0.800000,0.400000,0.466667}%
\pgfsetstrokecolor{currentstroke}%
\pgfsetstrokeopacity{0.375977}%
\pgfsetdash{}{0pt}%
\pgfpathmoveto{\pgfqpoint{8.205288in}{2.507834in}}%
\pgfpathcurveto{\pgfqpoint{8.214496in}{2.507834in}}{\pgfqpoint{8.223329in}{2.511492in}}{\pgfqpoint{8.229840in}{2.518004in}}%
\pgfpathcurveto{\pgfqpoint{8.236352in}{2.524515in}}{\pgfqpoint{8.240010in}{2.533348in}}{\pgfqpoint{8.240010in}{2.542556in}}%
\pgfpathcurveto{\pgfqpoint{8.240010in}{2.551764in}}{\pgfqpoint{8.236352in}{2.560597in}}{\pgfqpoint{8.229840in}{2.567108in}}%
\pgfpathcurveto{\pgfqpoint{8.223329in}{2.573620in}}{\pgfqpoint{8.214496in}{2.577278in}}{\pgfqpoint{8.205288in}{2.577278in}}%
\pgfpathcurveto{\pgfqpoint{8.196079in}{2.577278in}}{\pgfqpoint{8.187247in}{2.573620in}}{\pgfqpoint{8.180736in}{2.567108in}}%
\pgfpathcurveto{\pgfqpoint{8.174224in}{2.560597in}}{\pgfqpoint{8.170566in}{2.551764in}}{\pgfqpoint{8.170566in}{2.542556in}}%
\pgfpathcurveto{\pgfqpoint{8.170566in}{2.533348in}}{\pgfqpoint{8.174224in}{2.524515in}}{\pgfqpoint{8.180736in}{2.518004in}}%
\pgfpathcurveto{\pgfqpoint{8.187247in}{2.511492in}}{\pgfqpoint{8.196079in}{2.507834in}}{\pgfqpoint{8.205288in}{2.507834in}}%
\pgfpathlineto{\pgfqpoint{8.205288in}{2.507834in}}%
\pgfpathclose%
\pgfusepath{stroke,fill}%
\end{pgfscope}%
\begin{pgfscope}%
\pgfpathrectangle{\pgfqpoint{0.050000in}{0.050000in}}{\pgfqpoint{2.419000in}{2.419000in}}%
\pgfusepath{clip}%
\pgfsetbuttcap%
\pgfsetroundjoin%
\definecolor{currentfill}{rgb}{0.800000,0.400000,0.466667}%
\pgfsetfillcolor{currentfill}%
\pgfsetfillopacity{0.380427}%
\pgfsetlinewidth{1.003750pt}%
\definecolor{currentstroke}{rgb}{0.800000,0.400000,0.466667}%
\pgfsetstrokecolor{currentstroke}%
\pgfsetstrokeopacity{0.380427}%
\pgfsetdash{}{0pt}%
\pgfpathmoveto{\pgfqpoint{2.635297in}{2.416789in}}%
\pgfpathcurveto{\pgfqpoint{2.644506in}{2.416789in}}{\pgfqpoint{2.653338in}{2.420448in}}{\pgfqpoint{2.659850in}{2.426959in}}%
\pgfpathcurveto{\pgfqpoint{2.666361in}{2.433470in}}{\pgfqpoint{2.670019in}{2.442303in}}{\pgfqpoint{2.670019in}{2.451511in}}%
\pgfpathcurveto{\pgfqpoint{2.670019in}{2.460720in}}{\pgfqpoint{2.666361in}{2.469552in}}{\pgfqpoint{2.659850in}{2.476064in}}%
\pgfpathcurveto{\pgfqpoint{2.653338in}{2.482575in}}{\pgfqpoint{2.644506in}{2.486234in}}{\pgfqpoint{2.635297in}{2.486234in}}%
\pgfpathcurveto{\pgfqpoint{2.626089in}{2.486234in}}{\pgfqpoint{2.617256in}{2.482575in}}{\pgfqpoint{2.610745in}{2.476064in}}%
\pgfpathcurveto{\pgfqpoint{2.604234in}{2.469552in}}{\pgfqpoint{2.600575in}{2.460720in}}{\pgfqpoint{2.600575in}{2.451511in}}%
\pgfpathcurveto{\pgfqpoint{2.600575in}{2.442303in}}{\pgfqpoint{2.604234in}{2.433470in}}{\pgfqpoint{2.610745in}{2.426959in}}%
\pgfpathcurveto{\pgfqpoint{2.617256in}{2.420448in}}{\pgfqpoint{2.626089in}{2.416789in}}{\pgfqpoint{2.635297in}{2.416789in}}%
\pgfpathlineto{\pgfqpoint{2.635297in}{2.416789in}}%
\pgfpathclose%
\pgfusepath{stroke,fill}%
\end{pgfscope}%
\begin{pgfscope}%
\pgfpathrectangle{\pgfqpoint{0.050000in}{0.050000in}}{\pgfqpoint{2.419000in}{2.419000in}}%
\pgfusepath{clip}%
\pgfsetbuttcap%
\pgfsetroundjoin%
\definecolor{currentfill}{rgb}{0.800000,0.400000,0.466667}%
\pgfsetfillcolor{currentfill}%
\pgfsetfillopacity{0.380427}%
\pgfsetlinewidth{1.003750pt}%
\definecolor{currentstroke}{rgb}{0.800000,0.400000,0.466667}%
\pgfsetstrokecolor{currentstroke}%
\pgfsetstrokeopacity{0.380427}%
\pgfsetdash{}{0pt}%
\pgfpathmoveto{\pgfqpoint{-1.396641in}{2.416789in}}%
\pgfpathcurveto{\pgfqpoint{-1.387433in}{2.416789in}}{\pgfqpoint{-1.378600in}{2.420448in}}{\pgfqpoint{-1.372089in}{2.426959in}}%
\pgfpathcurveto{\pgfqpoint{-1.365577in}{2.433470in}}{\pgfqpoint{-1.361919in}{2.442303in}}{\pgfqpoint{-1.361919in}{2.451511in}}%
\pgfpathcurveto{\pgfqpoint{-1.361919in}{2.460720in}}{\pgfqpoint{-1.365577in}{2.469552in}}{\pgfqpoint{-1.372089in}{2.476064in}}%
\pgfpathcurveto{\pgfqpoint{-1.378600in}{2.482575in}}{\pgfqpoint{-1.387433in}{2.486234in}}{\pgfqpoint{-1.396641in}{2.486234in}}%
\pgfpathcurveto{\pgfqpoint{-1.405850in}{2.486234in}}{\pgfqpoint{-1.414682in}{2.482575in}}{\pgfqpoint{-1.421193in}{2.476064in}}%
\pgfpathcurveto{\pgfqpoint{-1.427705in}{2.469552in}}{\pgfqpoint{-1.431363in}{2.460720in}}{\pgfqpoint{-1.431363in}{2.451511in}}%
\pgfpathcurveto{\pgfqpoint{-1.431363in}{2.442303in}}{\pgfqpoint{-1.427705in}{2.433470in}}{\pgfqpoint{-1.421193in}{2.426959in}}%
\pgfpathcurveto{\pgfqpoint{-1.414682in}{2.420448in}}{\pgfqpoint{-1.405850in}{2.416789in}}{\pgfqpoint{-1.396641in}{2.416789in}}%
\pgfpathlineto{\pgfqpoint{-1.396641in}{2.416789in}}%
\pgfpathclose%
\pgfusepath{stroke,fill}%
\end{pgfscope}%
\begin{pgfscope}%
\pgfpathrectangle{\pgfqpoint{0.050000in}{0.050000in}}{\pgfqpoint{2.419000in}{2.419000in}}%
\pgfusepath{clip}%
\pgfsetbuttcap%
\pgfsetroundjoin%
\definecolor{currentfill}{rgb}{0.800000,0.400000,0.466667}%
\pgfsetfillcolor{currentfill}%
\pgfsetfillopacity{0.380427}%
\pgfsetlinewidth{1.003750pt}%
\definecolor{currentstroke}{rgb}{0.800000,0.400000,0.466667}%
\pgfsetstrokecolor{currentstroke}%
\pgfsetstrokeopacity{0.380427}%
\pgfsetdash{}{0pt}%
\pgfpathmoveto{\pgfqpoint{6.667236in}{2.416789in}}%
\pgfpathcurveto{\pgfqpoint{6.676444in}{2.416789in}}{\pgfqpoint{6.685277in}{2.420448in}}{\pgfqpoint{6.691788in}{2.426959in}}%
\pgfpathcurveto{\pgfqpoint{6.698299in}{2.433470in}}{\pgfqpoint{6.701958in}{2.442303in}}{\pgfqpoint{6.701958in}{2.451511in}}%
\pgfpathcurveto{\pgfqpoint{6.701958in}{2.460720in}}{\pgfqpoint{6.698299in}{2.469552in}}{\pgfqpoint{6.691788in}{2.476064in}}%
\pgfpathcurveto{\pgfqpoint{6.685277in}{2.482575in}}{\pgfqpoint{6.676444in}{2.486234in}}{\pgfqpoint{6.667236in}{2.486234in}}%
\pgfpathcurveto{\pgfqpoint{6.658027in}{2.486234in}}{\pgfqpoint{6.649195in}{2.482575in}}{\pgfqpoint{6.642683in}{2.476064in}}%
\pgfpathcurveto{\pgfqpoint{6.636172in}{2.469552in}}{\pgfqpoint{6.632513in}{2.460720in}}{\pgfqpoint{6.632513in}{2.451511in}}%
\pgfpathcurveto{\pgfqpoint{6.632513in}{2.442303in}}{\pgfqpoint{6.636172in}{2.433470in}}{\pgfqpoint{6.642683in}{2.426959in}}%
\pgfpathcurveto{\pgfqpoint{6.649195in}{2.420448in}}{\pgfqpoint{6.658027in}{2.416789in}}{\pgfqpoint{6.667236in}{2.416789in}}%
\pgfpathlineto{\pgfqpoint{6.667236in}{2.416789in}}%
\pgfpathclose%
\pgfusepath{stroke,fill}%
\end{pgfscope}%
\begin{pgfscope}%
\pgfpathrectangle{\pgfqpoint{0.050000in}{0.050000in}}{\pgfqpoint{2.419000in}{2.419000in}}%
\pgfusepath{clip}%
\pgfsetbuttcap%
\pgfsetroundjoin%
\definecolor{currentfill}{rgb}{0.800000,0.400000,0.466667}%
\pgfsetfillcolor{currentfill}%
\pgfsetfillopacity{0.385007}%
\pgfsetlinewidth{1.003750pt}%
\definecolor{currentstroke}{rgb}{0.800000,0.400000,0.466667}%
\pgfsetstrokecolor{currentstroke}%
\pgfsetstrokeopacity{0.385007}%
\pgfsetdash{}{0pt}%
\pgfpathmoveto{\pgfqpoint{0.992998in}{2.323074in}}%
\pgfpathcurveto{\pgfqpoint{1.002206in}{2.323074in}}{\pgfqpoint{1.011039in}{2.326733in}}{\pgfqpoint{1.017550in}{2.333244in}}%
\pgfpathcurveto{\pgfqpoint{1.024061in}{2.339755in}}{\pgfqpoint{1.027720in}{2.348588in}}{\pgfqpoint{1.027720in}{2.357796in}}%
\pgfpathcurveto{\pgfqpoint{1.027720in}{2.367005in}}{\pgfqpoint{1.024061in}{2.375837in}}{\pgfqpoint{1.017550in}{2.382349in}}%
\pgfpathcurveto{\pgfqpoint{1.011039in}{2.388860in}}{\pgfqpoint{1.002206in}{2.392518in}}{\pgfqpoint{0.992998in}{2.392518in}}%
\pgfpathcurveto{\pgfqpoint{0.983789in}{2.392518in}}{\pgfqpoint{0.974957in}{2.388860in}}{\pgfqpoint{0.968445in}{2.382349in}}%
\pgfpathcurveto{\pgfqpoint{0.961934in}{2.375837in}}{\pgfqpoint{0.958276in}{2.367005in}}{\pgfqpoint{0.958276in}{2.357796in}}%
\pgfpathcurveto{\pgfqpoint{0.958276in}{2.348588in}}{\pgfqpoint{0.961934in}{2.339755in}}{\pgfqpoint{0.968445in}{2.333244in}}%
\pgfpathcurveto{\pgfqpoint{0.974957in}{2.326733in}}{\pgfqpoint{0.983789in}{2.323074in}}{\pgfqpoint{0.992998in}{2.323074in}}%
\pgfpathlineto{\pgfqpoint{0.992998in}{2.323074in}}%
\pgfpathclose%
\pgfusepath{stroke,fill}%
\end{pgfscope}%
\begin{pgfscope}%
\pgfpathrectangle{\pgfqpoint{0.050000in}{0.050000in}}{\pgfqpoint{2.419000in}{2.419000in}}%
\pgfusepath{clip}%
\pgfsetbuttcap%
\pgfsetroundjoin%
\definecolor{currentfill}{rgb}{0.800000,0.400000,0.466667}%
\pgfsetfillcolor{currentfill}%
\pgfsetfillopacity{0.385007}%
\pgfsetlinewidth{1.003750pt}%
\definecolor{currentstroke}{rgb}{0.800000,0.400000,0.466667}%
\pgfsetstrokecolor{currentstroke}%
\pgfsetstrokeopacity{0.385007}%
\pgfsetdash{}{0pt}%
\pgfpathmoveto{\pgfqpoint{9.175140in}{2.323074in}}%
\pgfpathcurveto{\pgfqpoint{9.184348in}{2.323074in}}{\pgfqpoint{9.193181in}{2.326733in}}{\pgfqpoint{9.199692in}{2.333244in}}%
\pgfpathcurveto{\pgfqpoint{9.206204in}{2.339755in}}{\pgfqpoint{9.209862in}{2.348588in}}{\pgfqpoint{9.209862in}{2.357796in}}%
\pgfpathcurveto{\pgfqpoint{9.209862in}{2.367005in}}{\pgfqpoint{9.206204in}{2.375837in}}{\pgfqpoint{9.199692in}{2.382349in}}%
\pgfpathcurveto{\pgfqpoint{9.193181in}{2.388860in}}{\pgfqpoint{9.184348in}{2.392518in}}{\pgfqpoint{9.175140in}{2.392518in}}%
\pgfpathcurveto{\pgfqpoint{9.165931in}{2.392518in}}{\pgfqpoint{9.157099in}{2.388860in}}{\pgfqpoint{9.150588in}{2.382349in}}%
\pgfpathcurveto{\pgfqpoint{9.144076in}{2.375837in}}{\pgfqpoint{9.140418in}{2.367005in}}{\pgfqpoint{9.140418in}{2.357796in}}%
\pgfpathcurveto{\pgfqpoint{9.140418in}{2.348588in}}{\pgfqpoint{9.144076in}{2.339755in}}{\pgfqpoint{9.150588in}{2.333244in}}%
\pgfpathcurveto{\pgfqpoint{9.157099in}{2.326733in}}{\pgfqpoint{9.165931in}{2.323074in}}{\pgfqpoint{9.175140in}{2.323074in}}%
\pgfpathlineto{\pgfqpoint{9.175140in}{2.323074in}}%
\pgfpathclose%
\pgfusepath{stroke,fill}%
\end{pgfscope}%
\begin{pgfscope}%
\pgfpathrectangle{\pgfqpoint{0.050000in}{0.050000in}}{\pgfqpoint{2.419000in}{2.419000in}}%
\pgfusepath{clip}%
\pgfsetbuttcap%
\pgfsetroundjoin%
\definecolor{currentfill}{rgb}{0.800000,0.400000,0.466667}%
\pgfsetfillcolor{currentfill}%
\pgfsetfillopacity{0.385007}%
\pgfsetlinewidth{1.003750pt}%
\definecolor{currentstroke}{rgb}{0.800000,0.400000,0.466667}%
\pgfsetstrokecolor{currentstroke}%
\pgfsetstrokeopacity{0.385007}%
\pgfsetdash{}{0pt}%
\pgfpathmoveto{\pgfqpoint{5.084069in}{2.323074in}}%
\pgfpathcurveto{\pgfqpoint{5.093277in}{2.323074in}}{\pgfqpoint{5.102110in}{2.326733in}}{\pgfqpoint{5.108621in}{2.333244in}}%
\pgfpathcurveto{\pgfqpoint{5.115133in}{2.339755in}}{\pgfqpoint{5.118791in}{2.348588in}}{\pgfqpoint{5.118791in}{2.357796in}}%
\pgfpathcurveto{\pgfqpoint{5.118791in}{2.367005in}}{\pgfqpoint{5.115133in}{2.375837in}}{\pgfqpoint{5.108621in}{2.382349in}}%
\pgfpathcurveto{\pgfqpoint{5.102110in}{2.388860in}}{\pgfqpoint{5.093277in}{2.392518in}}{\pgfqpoint{5.084069in}{2.392518in}}%
\pgfpathcurveto{\pgfqpoint{5.074860in}{2.392518in}}{\pgfqpoint{5.066028in}{2.388860in}}{\pgfqpoint{5.059517in}{2.382349in}}%
\pgfpathcurveto{\pgfqpoint{5.053005in}{2.375837in}}{\pgfqpoint{5.049347in}{2.367005in}}{\pgfqpoint{5.049347in}{2.357796in}}%
\pgfpathcurveto{\pgfqpoint{5.049347in}{2.348588in}}{\pgfqpoint{5.053005in}{2.339755in}}{\pgfqpoint{5.059517in}{2.333244in}}%
\pgfpathcurveto{\pgfqpoint{5.066028in}{2.326733in}}{\pgfqpoint{5.074860in}{2.323074in}}{\pgfqpoint{5.084069in}{2.323074in}}%
\pgfpathlineto{\pgfqpoint{5.084069in}{2.323074in}}%
\pgfpathclose%
\pgfusepath{stroke,fill}%
\end{pgfscope}%
\begin{pgfscope}%
\pgfpathrectangle{\pgfqpoint{0.050000in}{0.050000in}}{\pgfqpoint{2.419000in}{2.419000in}}%
\pgfusepath{clip}%
\pgfsetbuttcap%
\pgfsetroundjoin%
\definecolor{currentfill}{rgb}{0.800000,0.400000,0.466667}%
\pgfsetfillcolor{currentfill}%
\pgfsetfillopacity{0.389723}%
\pgfsetlinewidth{1.003750pt}%
\definecolor{currentstroke}{rgb}{0.800000,0.400000,0.466667}%
\pgfsetstrokecolor{currentstroke}%
\pgfsetstrokeopacity{0.389723}%
\pgfsetdash{}{0pt}%
\pgfpathmoveto{\pgfqpoint{3.453773in}{2.226569in}}%
\pgfpathcurveto{\pgfqpoint{3.462982in}{2.226569in}}{\pgfqpoint{3.471814in}{2.230228in}}{\pgfqpoint{3.478326in}{2.236739in}}%
\pgfpathcurveto{\pgfqpoint{3.484837in}{2.243250in}}{\pgfqpoint{3.488495in}{2.252083in}}{\pgfqpoint{3.488495in}{2.261291in}}%
\pgfpathcurveto{\pgfqpoint{3.488495in}{2.270500in}}{\pgfqpoint{3.484837in}{2.279332in}}{\pgfqpoint{3.478326in}{2.285844in}}%
\pgfpathcurveto{\pgfqpoint{3.471814in}{2.292355in}}{\pgfqpoint{3.462982in}{2.296014in}}{\pgfqpoint{3.453773in}{2.296014in}}%
\pgfpathcurveto{\pgfqpoint{3.444565in}{2.296014in}}{\pgfqpoint{3.435732in}{2.292355in}}{\pgfqpoint{3.429221in}{2.285844in}}%
\pgfpathcurveto{\pgfqpoint{3.422710in}{2.279332in}}{\pgfqpoint{3.419051in}{2.270500in}}{\pgfqpoint{3.419051in}{2.261291in}}%
\pgfpathcurveto{\pgfqpoint{3.419051in}{2.252083in}}{\pgfqpoint{3.422710in}{2.243250in}}{\pgfqpoint{3.429221in}{2.236739in}}%
\pgfpathcurveto{\pgfqpoint{3.435732in}{2.230228in}}{\pgfqpoint{3.444565in}{2.226569in}}{\pgfqpoint{3.453773in}{2.226569in}}%
\pgfpathlineto{\pgfqpoint{3.453773in}{2.226569in}}%
\pgfpathclose%
\pgfusepath{stroke,fill}%
\end{pgfscope}%
\begin{pgfscope}%
\pgfpathrectangle{\pgfqpoint{0.050000in}{0.050000in}}{\pgfqpoint{2.419000in}{2.419000in}}%
\pgfusepath{clip}%
\pgfsetbuttcap%
\pgfsetroundjoin%
\definecolor{currentfill}{rgb}{0.800000,0.400000,0.466667}%
\pgfsetfillcolor{currentfill}%
\pgfsetfillopacity{0.389723}%
\pgfsetlinewidth{1.003750pt}%
\definecolor{currentstroke}{rgb}{0.800000,0.400000,0.466667}%
\pgfsetstrokecolor{currentstroke}%
\pgfsetstrokeopacity{0.389723}%
\pgfsetdash{}{0pt}%
\pgfpathmoveto{\pgfqpoint{-0.698191in}{2.226569in}}%
\pgfpathcurveto{\pgfqpoint{-0.688982in}{2.226569in}}{\pgfqpoint{-0.680150in}{2.230228in}}{\pgfqpoint{-0.673639in}{2.236739in}}%
\pgfpathcurveto{\pgfqpoint{-0.667127in}{2.243250in}}{\pgfqpoint{-0.663469in}{2.252083in}}{\pgfqpoint{-0.663469in}{2.261291in}}%
\pgfpathcurveto{\pgfqpoint{-0.663469in}{2.270500in}}{\pgfqpoint{-0.667127in}{2.279332in}}{\pgfqpoint{-0.673639in}{2.285844in}}%
\pgfpathcurveto{\pgfqpoint{-0.680150in}{2.292355in}}{\pgfqpoint{-0.688982in}{2.296014in}}{\pgfqpoint{-0.698191in}{2.296014in}}%
\pgfpathcurveto{\pgfqpoint{-0.707399in}{2.296014in}}{\pgfqpoint{-0.716232in}{2.292355in}}{\pgfqpoint{-0.722743in}{2.285844in}}%
\pgfpathcurveto{\pgfqpoint{-0.729255in}{2.279332in}}{\pgfqpoint{-0.732913in}{2.270500in}}{\pgfqpoint{-0.732913in}{2.261291in}}%
\pgfpathcurveto{\pgfqpoint{-0.732913in}{2.252083in}}{\pgfqpoint{-0.729255in}{2.243250in}}{\pgfqpoint{-0.722743in}{2.236739in}}%
\pgfpathcurveto{\pgfqpoint{-0.716232in}{2.230228in}}{\pgfqpoint{-0.707399in}{2.226569in}}{\pgfqpoint{-0.698191in}{2.226569in}}%
\pgfpathlineto{\pgfqpoint{-0.698191in}{2.226569in}}%
\pgfpathclose%
\pgfusepath{stroke,fill}%
\end{pgfscope}%
\begin{pgfscope}%
\pgfpathrectangle{\pgfqpoint{0.050000in}{0.050000in}}{\pgfqpoint{2.419000in}{2.419000in}}%
\pgfusepath{clip}%
\pgfsetbuttcap%
\pgfsetroundjoin%
\definecolor{currentfill}{rgb}{0.800000,0.400000,0.466667}%
\pgfsetfillcolor{currentfill}%
\pgfsetfillopacity{0.389723}%
\pgfsetlinewidth{1.003750pt}%
\definecolor{currentstroke}{rgb}{0.800000,0.400000,0.466667}%
\pgfsetstrokecolor{currentstroke}%
\pgfsetstrokeopacity{0.389723}%
\pgfsetdash{}{0pt}%
\pgfpathmoveto{\pgfqpoint{7.605737in}{2.226569in}}%
\pgfpathcurveto{\pgfqpoint{7.614946in}{2.226569in}}{\pgfqpoint{7.623778in}{2.230228in}}{\pgfqpoint{7.630290in}{2.236739in}}%
\pgfpathcurveto{\pgfqpoint{7.636801in}{2.243250in}}{\pgfqpoint{7.640460in}{2.252083in}}{\pgfqpoint{7.640460in}{2.261291in}}%
\pgfpathcurveto{\pgfqpoint{7.640460in}{2.270500in}}{\pgfqpoint{7.636801in}{2.279332in}}{\pgfqpoint{7.630290in}{2.285844in}}%
\pgfpathcurveto{\pgfqpoint{7.623778in}{2.292355in}}{\pgfqpoint{7.614946in}{2.296014in}}{\pgfqpoint{7.605737in}{2.296014in}}%
\pgfpathcurveto{\pgfqpoint{7.596529in}{2.296014in}}{\pgfqpoint{7.587696in}{2.292355in}}{\pgfqpoint{7.581185in}{2.285844in}}%
\pgfpathcurveto{\pgfqpoint{7.574674in}{2.279332in}}{\pgfqpoint{7.571015in}{2.270500in}}{\pgfqpoint{7.571015in}{2.261291in}}%
\pgfpathcurveto{\pgfqpoint{7.571015in}{2.252083in}}{\pgfqpoint{7.574674in}{2.243250in}}{\pgfqpoint{7.581185in}{2.236739in}}%
\pgfpathcurveto{\pgfqpoint{7.587696in}{2.230228in}}{\pgfqpoint{7.596529in}{2.226569in}}{\pgfqpoint{7.605737in}{2.226569in}}%
\pgfpathlineto{\pgfqpoint{7.605737in}{2.226569in}}%
\pgfpathclose%
\pgfusepath{stroke,fill}%
\end{pgfscope}%
\begin{pgfscope}%
\pgfpathrectangle{\pgfqpoint{0.050000in}{0.050000in}}{\pgfqpoint{2.419000in}{2.419000in}}%
\pgfusepath{clip}%
\pgfsetbuttcap%
\pgfsetroundjoin%
\definecolor{currentfill}{rgb}{0.800000,0.400000,0.466667}%
\pgfsetfillcolor{currentfill}%
\pgfsetfillopacity{0.394581}%
\pgfsetlinewidth{1.003750pt}%
\definecolor{currentstroke}{rgb}{0.800000,0.400000,0.466667}%
\pgfsetstrokecolor{currentstroke}%
\pgfsetstrokeopacity{0.394581}%
\pgfsetdash{}{0pt}%
\pgfpathmoveto{\pgfqpoint{1.774212in}{2.127148in}}%
\pgfpathcurveto{\pgfqpoint{1.783421in}{2.127148in}}{\pgfqpoint{1.792253in}{2.130807in}}{\pgfqpoint{1.798765in}{2.137318in}}%
\pgfpathcurveto{\pgfqpoint{1.805276in}{2.143829in}}{\pgfqpoint{1.808935in}{2.152662in}}{\pgfqpoint{1.808935in}{2.161870in}}%
\pgfpathcurveto{\pgfqpoint{1.808935in}{2.171079in}}{\pgfqpoint{1.805276in}{2.179911in}}{\pgfqpoint{1.798765in}{2.186422in}}%
\pgfpathcurveto{\pgfqpoint{1.792253in}{2.192934in}}{\pgfqpoint{1.783421in}{2.196592in}}{\pgfqpoint{1.774212in}{2.196592in}}%
\pgfpathcurveto{\pgfqpoint{1.765004in}{2.196592in}}{\pgfqpoint{1.756171in}{2.192934in}}{\pgfqpoint{1.749660in}{2.186422in}}%
\pgfpathcurveto{\pgfqpoint{1.743149in}{2.179911in}}{\pgfqpoint{1.739490in}{2.171079in}}{\pgfqpoint{1.739490in}{2.161870in}}%
\pgfpathcurveto{\pgfqpoint{1.739490in}{2.152662in}}{\pgfqpoint{1.743149in}{2.143829in}}{\pgfqpoint{1.749660in}{2.137318in}}%
\pgfpathcurveto{\pgfqpoint{1.756171in}{2.130807in}}{\pgfqpoint{1.765004in}{2.127148in}}{\pgfqpoint{1.774212in}{2.127148in}}%
\pgfpathlineto{\pgfqpoint{1.774212in}{2.127148in}}%
\pgfpathclose%
\pgfusepath{stroke,fill}%
\end{pgfscope}%
\begin{pgfscope}%
\pgfpathrectangle{\pgfqpoint{0.050000in}{0.050000in}}{\pgfqpoint{2.419000in}{2.419000in}}%
\pgfusepath{clip}%
\pgfsetbuttcap%
\pgfsetroundjoin%
\definecolor{currentfill}{rgb}{0.800000,0.400000,0.466667}%
\pgfsetfillcolor{currentfill}%
\pgfsetfillopacity{0.394581}%
\pgfsetlinewidth{1.003750pt}%
\definecolor{currentstroke}{rgb}{0.800000,0.400000,0.466667}%
\pgfsetstrokecolor{currentstroke}%
\pgfsetstrokeopacity{0.394581}%
\pgfsetdash{}{0pt}%
\pgfpathmoveto{\pgfqpoint{10.203607in}{2.127148in}}%
\pgfpathcurveto{\pgfqpoint{10.212815in}{2.127148in}}{\pgfqpoint{10.221648in}{2.130807in}}{\pgfqpoint{10.228159in}{2.137318in}}%
\pgfpathcurveto{\pgfqpoint{10.234671in}{2.143829in}}{\pgfqpoint{10.238329in}{2.152662in}}{\pgfqpoint{10.238329in}{2.161870in}}%
\pgfpathcurveto{\pgfqpoint{10.238329in}{2.171079in}}{\pgfqpoint{10.234671in}{2.179911in}}{\pgfqpoint{10.228159in}{2.186422in}}%
\pgfpathcurveto{\pgfqpoint{10.221648in}{2.192934in}}{\pgfqpoint{10.212815in}{2.196592in}}{\pgfqpoint{10.203607in}{2.196592in}}%
\pgfpathcurveto{\pgfqpoint{10.194399in}{2.196592in}}{\pgfqpoint{10.185566in}{2.192934in}}{\pgfqpoint{10.179055in}{2.186422in}}%
\pgfpathcurveto{\pgfqpoint{10.172543in}{2.179911in}}{\pgfqpoint{10.168885in}{2.171079in}}{\pgfqpoint{10.168885in}{2.161870in}}%
\pgfpathcurveto{\pgfqpoint{10.168885in}{2.152662in}}{\pgfqpoint{10.172543in}{2.143829in}}{\pgfqpoint{10.179055in}{2.137318in}}%
\pgfpathcurveto{\pgfqpoint{10.185566in}{2.130807in}}{\pgfqpoint{10.194399in}{2.127148in}}{\pgfqpoint{10.203607in}{2.127148in}}%
\pgfpathlineto{\pgfqpoint{10.203607in}{2.127148in}}%
\pgfpathclose%
\pgfusepath{stroke,fill}%
\end{pgfscope}%
\begin{pgfscope}%
\pgfpathrectangle{\pgfqpoint{0.050000in}{0.050000in}}{\pgfqpoint{2.419000in}{2.419000in}}%
\pgfusepath{clip}%
\pgfsetbuttcap%
\pgfsetroundjoin%
\definecolor{currentfill}{rgb}{0.800000,0.400000,0.466667}%
\pgfsetfillcolor{currentfill}%
\pgfsetfillopacity{0.394581}%
\pgfsetlinewidth{1.003750pt}%
\definecolor{currentstroke}{rgb}{0.800000,0.400000,0.466667}%
\pgfsetstrokecolor{currentstroke}%
\pgfsetstrokeopacity{0.394581}%
\pgfsetdash{}{0pt}%
\pgfpathmoveto{\pgfqpoint{5.988910in}{2.127148in}}%
\pgfpathcurveto{\pgfqpoint{5.998118in}{2.127148in}}{\pgfqpoint{6.006951in}{2.130807in}}{\pgfqpoint{6.013462in}{2.137318in}}%
\pgfpathcurveto{\pgfqpoint{6.019973in}{2.143829in}}{\pgfqpoint{6.023632in}{2.152662in}}{\pgfqpoint{6.023632in}{2.161870in}}%
\pgfpathcurveto{\pgfqpoint{6.023632in}{2.171079in}}{\pgfqpoint{6.019973in}{2.179911in}}{\pgfqpoint{6.013462in}{2.186422in}}%
\pgfpathcurveto{\pgfqpoint{6.006951in}{2.192934in}}{\pgfqpoint{5.998118in}{2.196592in}}{\pgfqpoint{5.988910in}{2.196592in}}%
\pgfpathcurveto{\pgfqpoint{5.979701in}{2.196592in}}{\pgfqpoint{5.970869in}{2.192934in}}{\pgfqpoint{5.964357in}{2.186422in}}%
\pgfpathcurveto{\pgfqpoint{5.957846in}{2.179911in}}{\pgfqpoint{5.954187in}{2.171079in}}{\pgfqpoint{5.954187in}{2.161870in}}%
\pgfpathcurveto{\pgfqpoint{5.954187in}{2.152662in}}{\pgfqpoint{5.957846in}{2.143829in}}{\pgfqpoint{5.964357in}{2.137318in}}%
\pgfpathcurveto{\pgfqpoint{5.970869in}{2.130807in}}{\pgfqpoint{5.979701in}{2.127148in}}{\pgfqpoint{5.988910in}{2.127148in}}%
\pgfpathlineto{\pgfqpoint{5.988910in}{2.127148in}}%
\pgfpathclose%
\pgfusepath{stroke,fill}%
\end{pgfscope}%
\begin{pgfscope}%
\pgfpathrectangle{\pgfqpoint{0.050000in}{0.050000in}}{\pgfqpoint{2.419000in}{2.419000in}}%
\pgfusepath{clip}%
\pgfsetbuttcap%
\pgfsetroundjoin%
\definecolor{currentfill}{rgb}{0.800000,0.400000,0.466667}%
\pgfsetfillcolor{currentfill}%
\pgfsetfillopacity{0.399589}%
\pgfsetlinewidth{1.003750pt}%
\definecolor{currentstroke}{rgb}{0.800000,0.400000,0.466667}%
\pgfsetstrokecolor{currentstroke}%
\pgfsetstrokeopacity{0.399589}%
\pgfsetdash{}{0pt}%
\pgfpathmoveto{\pgfqpoint{4.322474in}{2.024676in}}%
\pgfpathcurveto{\pgfqpoint{4.331683in}{2.024676in}}{\pgfqpoint{4.340515in}{2.028335in}}{\pgfqpoint{4.347027in}{2.034846in}}%
\pgfpathcurveto{\pgfqpoint{4.353538in}{2.041358in}}{\pgfqpoint{4.357197in}{2.050190in}}{\pgfqpoint{4.357197in}{2.059399in}}%
\pgfpathcurveto{\pgfqpoint{4.357197in}{2.068607in}}{\pgfqpoint{4.353538in}{2.077440in}}{\pgfqpoint{4.347027in}{2.083951in}}%
\pgfpathcurveto{\pgfqpoint{4.340515in}{2.090462in}}{\pgfqpoint{4.331683in}{2.094121in}}{\pgfqpoint{4.322474in}{2.094121in}}%
\pgfpathcurveto{\pgfqpoint{4.313266in}{2.094121in}}{\pgfqpoint{4.304433in}{2.090462in}}{\pgfqpoint{4.297922in}{2.083951in}}%
\pgfpathcurveto{\pgfqpoint{4.291411in}{2.077440in}}{\pgfqpoint{4.287752in}{2.068607in}}{\pgfqpoint{4.287752in}{2.059399in}}%
\pgfpathcurveto{\pgfqpoint{4.287752in}{2.050190in}}{\pgfqpoint{4.291411in}{2.041358in}}{\pgfqpoint{4.297922in}{2.034846in}}%
\pgfpathcurveto{\pgfqpoint{4.304433in}{2.028335in}}{\pgfqpoint{4.313266in}{2.024676in}}{\pgfqpoint{4.322474in}{2.024676in}}%
\pgfpathlineto{\pgfqpoint{4.322474in}{2.024676in}}%
\pgfpathclose%
\pgfusepath{stroke,fill}%
\end{pgfscope}%
\begin{pgfscope}%
\pgfpathrectangle{\pgfqpoint{0.050000in}{0.050000in}}{\pgfqpoint{2.419000in}{2.419000in}}%
\pgfusepath{clip}%
\pgfsetbuttcap%
\pgfsetroundjoin%
\definecolor{currentfill}{rgb}{0.800000,0.400000,0.466667}%
\pgfsetfillcolor{currentfill}%
\pgfsetfillopacity{0.399589}%
\pgfsetlinewidth{1.003750pt}%
\definecolor{currentstroke}{rgb}{0.800000,0.400000,0.466667}%
\pgfsetstrokecolor{currentstroke}%
\pgfsetstrokeopacity{0.399589}%
\pgfsetdash{}{0pt}%
\pgfpathmoveto{\pgfqpoint{0.043119in}{2.024676in}}%
\pgfpathcurveto{\pgfqpoint{0.052328in}{2.024676in}}{\pgfqpoint{0.061160in}{2.028335in}}{\pgfqpoint{0.067671in}{2.034846in}}%
\pgfpathcurveto{\pgfqpoint{0.074183in}{2.041358in}}{\pgfqpoint{0.077841in}{2.050190in}}{\pgfqpoint{0.077841in}{2.059399in}}%
\pgfpathcurveto{\pgfqpoint{0.077841in}{2.068607in}}{\pgfqpoint{0.074183in}{2.077440in}}{\pgfqpoint{0.067671in}{2.083951in}}%
\pgfpathcurveto{\pgfqpoint{0.061160in}{2.090462in}}{\pgfqpoint{0.052328in}{2.094121in}}{\pgfqpoint{0.043119in}{2.094121in}}%
\pgfpathcurveto{\pgfqpoint{0.033911in}{2.094121in}}{\pgfqpoint{0.025078in}{2.090462in}}{\pgfqpoint{0.018567in}{2.083951in}}%
\pgfpathcurveto{\pgfqpoint{0.012055in}{2.077440in}}{\pgfqpoint{0.008397in}{2.068607in}}{\pgfqpoint{0.008397in}{2.059399in}}%
\pgfpathcurveto{\pgfqpoint{0.008397in}{2.050190in}}{\pgfqpoint{0.012055in}{2.041358in}}{\pgfqpoint{0.018567in}{2.034846in}}%
\pgfpathcurveto{\pgfqpoint{0.025078in}{2.028335in}}{\pgfqpoint{0.033911in}{2.024676in}}{\pgfqpoint{0.043119in}{2.024676in}}%
\pgfpathlineto{\pgfqpoint{0.043119in}{2.024676in}}%
\pgfpathclose%
\pgfusepath{stroke,fill}%
\end{pgfscope}%
\begin{pgfscope}%
\pgfpathrectangle{\pgfqpoint{0.050000in}{0.050000in}}{\pgfqpoint{2.419000in}{2.419000in}}%
\pgfusepath{clip}%
\pgfsetbuttcap%
\pgfsetroundjoin%
\definecolor{currentfill}{rgb}{0.800000,0.400000,0.466667}%
\pgfsetfillcolor{currentfill}%
\pgfsetfillopacity{0.399589}%
\pgfsetlinewidth{1.003750pt}%
\definecolor{currentstroke}{rgb}{0.800000,0.400000,0.466667}%
\pgfsetstrokecolor{currentstroke}%
\pgfsetstrokeopacity{0.399589}%
\pgfsetdash{}{0pt}%
\pgfpathmoveto{\pgfqpoint{8.601830in}{2.024676in}}%
\pgfpathcurveto{\pgfqpoint{8.611038in}{2.024676in}}{\pgfqpoint{8.619870in}{2.028335in}}{\pgfqpoint{8.626382in}{2.034846in}}%
\pgfpathcurveto{\pgfqpoint{8.632893in}{2.041358in}}{\pgfqpoint{8.636552in}{2.050190in}}{\pgfqpoint{8.636552in}{2.059399in}}%
\pgfpathcurveto{\pgfqpoint{8.636552in}{2.068607in}}{\pgfqpoint{8.632893in}{2.077440in}}{\pgfqpoint{8.626382in}{2.083951in}}%
\pgfpathcurveto{\pgfqpoint{8.619870in}{2.090462in}}{\pgfqpoint{8.611038in}{2.094121in}}{\pgfqpoint{8.601830in}{2.094121in}}%
\pgfpathcurveto{\pgfqpoint{8.592621in}{2.094121in}}{\pgfqpoint{8.583789in}{2.090462in}}{\pgfqpoint{8.577277in}{2.083951in}}%
\pgfpathcurveto{\pgfqpoint{8.570766in}{2.077440in}}{\pgfqpoint{8.567107in}{2.068607in}}{\pgfqpoint{8.567107in}{2.059399in}}%
\pgfpathcurveto{\pgfqpoint{8.567107in}{2.050190in}}{\pgfqpoint{8.570766in}{2.041358in}}{\pgfqpoint{8.577277in}{2.034846in}}%
\pgfpathcurveto{\pgfqpoint{8.583789in}{2.028335in}}{\pgfqpoint{8.592621in}{2.024676in}}{\pgfqpoint{8.601830in}{2.024676in}}%
\pgfpathlineto{\pgfqpoint{8.601830in}{2.024676in}}%
\pgfpathclose%
\pgfusepath{stroke,fill}%
\end{pgfscope}%
\begin{pgfscope}%
\pgfpathrectangle{\pgfqpoint{0.050000in}{0.050000in}}{\pgfqpoint{2.419000in}{2.419000in}}%
\pgfusepath{clip}%
\pgfsetbuttcap%
\pgfsetroundjoin%
\definecolor{currentfill}{rgb}{0.800000,0.400000,0.466667}%
\pgfsetfillcolor{currentfill}%
\pgfsetfillopacity{0.404753}%
\pgfsetlinewidth{1.003750pt}%
\definecolor{currentstroke}{rgb}{0.800000,0.400000,0.466667}%
\pgfsetstrokecolor{currentstroke}%
\pgfsetstrokeopacity{0.404753}%
\pgfsetdash{}{0pt}%
\pgfpathmoveto{\pgfqpoint{2.604113in}{1.919012in}}%
\pgfpathcurveto{\pgfqpoint{2.613321in}{1.919012in}}{\pgfqpoint{2.622154in}{1.922670in}}{\pgfqpoint{2.628665in}{1.929182in}}%
\pgfpathcurveto{\pgfqpoint{2.635176in}{1.935693in}}{\pgfqpoint{2.638835in}{1.944525in}}{\pgfqpoint{2.638835in}{1.953734in}}%
\pgfpathcurveto{\pgfqpoint{2.638835in}{1.962942in}}{\pgfqpoint{2.635176in}{1.971775in}}{\pgfqpoint{2.628665in}{1.978286in}}%
\pgfpathcurveto{\pgfqpoint{2.622154in}{1.984798in}}{\pgfqpoint{2.613321in}{1.988456in}}{\pgfqpoint{2.604113in}{1.988456in}}%
\pgfpathcurveto{\pgfqpoint{2.594904in}{1.988456in}}{\pgfqpoint{2.586072in}{1.984798in}}{\pgfqpoint{2.579560in}{1.978286in}}%
\pgfpathcurveto{\pgfqpoint{2.573049in}{1.971775in}}{\pgfqpoint{2.569390in}{1.962942in}}{\pgfqpoint{2.569390in}{1.953734in}}%
\pgfpathcurveto{\pgfqpoint{2.569390in}{1.944525in}}{\pgfqpoint{2.573049in}{1.935693in}}{\pgfqpoint{2.579560in}{1.929182in}}%
\pgfpathcurveto{\pgfqpoint{2.586072in}{1.922670in}}{\pgfqpoint{2.594904in}{1.919012in}}{\pgfqpoint{2.604113in}{1.919012in}}%
\pgfpathlineto{\pgfqpoint{2.604113in}{1.919012in}}%
\pgfpathclose%
\pgfusepath{stroke,fill}%
\end{pgfscope}%
\begin{pgfscope}%
\pgfpathrectangle{\pgfqpoint{0.050000in}{0.050000in}}{\pgfqpoint{2.419000in}{2.419000in}}%
\pgfusepath{clip}%
\pgfsetbuttcap%
\pgfsetroundjoin%
\definecolor{currentfill}{rgb}{0.800000,0.400000,0.466667}%
\pgfsetfillcolor{currentfill}%
\pgfsetfillopacity{0.404753}%
\pgfsetlinewidth{1.003750pt}%
\definecolor{currentstroke}{rgb}{0.800000,0.400000,0.466667}%
\pgfsetstrokecolor{currentstroke}%
\pgfsetstrokeopacity{0.404753}%
\pgfsetdash{}{0pt}%
\pgfpathmoveto{\pgfqpoint{-1.741915in}{1.919012in}}%
\pgfpathcurveto{\pgfqpoint{-1.732707in}{1.919012in}}{\pgfqpoint{-1.723874in}{1.922670in}}{\pgfqpoint{-1.717363in}{1.929182in}}%
\pgfpathcurveto{\pgfqpoint{-1.710852in}{1.935693in}}{\pgfqpoint{-1.707193in}{1.944525in}}{\pgfqpoint{-1.707193in}{1.953734in}}%
\pgfpathcurveto{\pgfqpoint{-1.707193in}{1.962942in}}{\pgfqpoint{-1.710852in}{1.971775in}}{\pgfqpoint{-1.717363in}{1.978286in}}%
\pgfpathcurveto{\pgfqpoint{-1.723874in}{1.984798in}}{\pgfqpoint{-1.732707in}{1.988456in}}{\pgfqpoint{-1.741915in}{1.988456in}}%
\pgfpathcurveto{\pgfqpoint{-1.751124in}{1.988456in}}{\pgfqpoint{-1.759956in}{1.984798in}}{\pgfqpoint{-1.766468in}{1.978286in}}%
\pgfpathcurveto{\pgfqpoint{-1.772979in}{1.971775in}}{\pgfqpoint{-1.776638in}{1.962942in}}{\pgfqpoint{-1.776638in}{1.953734in}}%
\pgfpathcurveto{\pgfqpoint{-1.776638in}{1.944525in}}{\pgfqpoint{-1.772979in}{1.935693in}}{\pgfqpoint{-1.766468in}{1.929182in}}%
\pgfpathcurveto{\pgfqpoint{-1.759956in}{1.922670in}}{\pgfqpoint{-1.751124in}{1.919012in}}{\pgfqpoint{-1.741915in}{1.919012in}}%
\pgfpathlineto{\pgfqpoint{-1.741915in}{1.919012in}}%
\pgfpathclose%
\pgfusepath{stroke,fill}%
\end{pgfscope}%
\begin{pgfscope}%
\pgfpathrectangle{\pgfqpoint{0.050000in}{0.050000in}}{\pgfqpoint{2.419000in}{2.419000in}}%
\pgfusepath{clip}%
\pgfsetbuttcap%
\pgfsetroundjoin%
\definecolor{currentfill}{rgb}{0.800000,0.400000,0.466667}%
\pgfsetfillcolor{currentfill}%
\pgfsetfillopacity{0.404753}%
\pgfsetlinewidth{1.003750pt}%
\definecolor{currentstroke}{rgb}{0.800000,0.400000,0.466667}%
\pgfsetstrokecolor{currentstroke}%
\pgfsetstrokeopacity{0.404753}%
\pgfsetdash{}{0pt}%
\pgfpathmoveto{\pgfqpoint{6.950140in}{1.919012in}}%
\pgfpathcurveto{\pgfqpoint{6.959349in}{1.919012in}}{\pgfqpoint{6.968181in}{1.922670in}}{\pgfqpoint{6.974693in}{1.929182in}}%
\pgfpathcurveto{\pgfqpoint{6.981204in}{1.935693in}}{\pgfqpoint{6.984863in}{1.944525in}}{\pgfqpoint{6.984863in}{1.953734in}}%
\pgfpathcurveto{\pgfqpoint{6.984863in}{1.962942in}}{\pgfqpoint{6.981204in}{1.971775in}}{\pgfqpoint{6.974693in}{1.978286in}}%
\pgfpathcurveto{\pgfqpoint{6.968181in}{1.984798in}}{\pgfqpoint{6.959349in}{1.988456in}}{\pgfqpoint{6.950140in}{1.988456in}}%
\pgfpathcurveto{\pgfqpoint{6.940932in}{1.988456in}}{\pgfqpoint{6.932099in}{1.984798in}}{\pgfqpoint{6.925588in}{1.978286in}}%
\pgfpathcurveto{\pgfqpoint{6.919077in}{1.971775in}}{\pgfqpoint{6.915418in}{1.962942in}}{\pgfqpoint{6.915418in}{1.953734in}}%
\pgfpathcurveto{\pgfqpoint{6.915418in}{1.944525in}}{\pgfqpoint{6.919077in}{1.935693in}}{\pgfqpoint{6.925588in}{1.929182in}}%
\pgfpathcurveto{\pgfqpoint{6.932099in}{1.922670in}}{\pgfqpoint{6.940932in}{1.919012in}}{\pgfqpoint{6.950140in}{1.919012in}}%
\pgfpathlineto{\pgfqpoint{6.950140in}{1.919012in}}%
\pgfpathclose%
\pgfusepath{stroke,fill}%
\end{pgfscope}%
\begin{pgfscope}%
\pgfpathrectangle{\pgfqpoint{0.050000in}{0.050000in}}{\pgfqpoint{2.419000in}{2.419000in}}%
\pgfusepath{clip}%
\pgfsetbuttcap%
\pgfsetroundjoin%
\definecolor{currentfill}{rgb}{0.800000,0.400000,0.466667}%
\pgfsetfillcolor{currentfill}%
\pgfsetfillopacity{0.410080}%
\pgfsetlinewidth{1.003750pt}%
\definecolor{currentstroke}{rgb}{0.800000,0.400000,0.466667}%
\pgfsetstrokecolor{currentstroke}%
\pgfsetstrokeopacity{0.410080}%
\pgfsetdash{}{0pt}%
\pgfpathmoveto{\pgfqpoint{0.831359in}{1.810002in}}%
\pgfpathcurveto{\pgfqpoint{0.840567in}{1.810002in}}{\pgfqpoint{0.849400in}{1.813661in}}{\pgfqpoint{0.855911in}{1.820172in}}%
\pgfpathcurveto{\pgfqpoint{0.862423in}{1.826684in}}{\pgfqpoint{0.866081in}{1.835516in}}{\pgfqpoint{0.866081in}{1.844725in}}%
\pgfpathcurveto{\pgfqpoint{0.866081in}{1.853933in}}{\pgfqpoint{0.862423in}{1.862766in}}{\pgfqpoint{0.855911in}{1.869277in}}%
\pgfpathcurveto{\pgfqpoint{0.849400in}{1.875788in}}{\pgfqpoint{0.840567in}{1.879447in}}{\pgfqpoint{0.831359in}{1.879447in}}%
\pgfpathcurveto{\pgfqpoint{0.822150in}{1.879447in}}{\pgfqpoint{0.813318in}{1.875788in}}{\pgfqpoint{0.806807in}{1.869277in}}%
\pgfpathcurveto{\pgfqpoint{0.800295in}{1.862766in}}{\pgfqpoint{0.796637in}{1.853933in}}{\pgfqpoint{0.796637in}{1.844725in}}%
\pgfpathcurveto{\pgfqpoint{0.796637in}{1.835516in}}{\pgfqpoint{0.800295in}{1.826684in}}{\pgfqpoint{0.806807in}{1.820172in}}%
\pgfpathcurveto{\pgfqpoint{0.813318in}{1.813661in}}{\pgfqpoint{0.822150in}{1.810002in}}{\pgfqpoint{0.831359in}{1.810002in}}%
\pgfpathlineto{\pgfqpoint{0.831359in}{1.810002in}}%
\pgfpathclose%
\pgfusepath{stroke,fill}%
\end{pgfscope}%
\begin{pgfscope}%
\pgfpathrectangle{\pgfqpoint{0.050000in}{0.050000in}}{\pgfqpoint{2.419000in}{2.419000in}}%
\pgfusepath{clip}%
\pgfsetbuttcap%
\pgfsetroundjoin%
\definecolor{currentfill}{rgb}{0.800000,0.400000,0.466667}%
\pgfsetfillcolor{currentfill}%
\pgfsetfillopacity{0.410080}%
\pgfsetlinewidth{1.003750pt}%
\definecolor{currentstroke}{rgb}{0.800000,0.400000,0.466667}%
\pgfsetstrokecolor{currentstroke}%
\pgfsetstrokeopacity{0.410080}%
\pgfsetdash{}{0pt}%
\pgfpathmoveto{\pgfqpoint{5.246170in}{1.810002in}}%
\pgfpathcurveto{\pgfqpoint{5.255378in}{1.810002in}}{\pgfqpoint{5.264211in}{1.813661in}}{\pgfqpoint{5.270722in}{1.820172in}}%
\pgfpathcurveto{\pgfqpoint{5.277234in}{1.826684in}}{\pgfqpoint{5.280892in}{1.835516in}}{\pgfqpoint{5.280892in}{1.844725in}}%
\pgfpathcurveto{\pgfqpoint{5.280892in}{1.853933in}}{\pgfqpoint{5.277234in}{1.862766in}}{\pgfqpoint{5.270722in}{1.869277in}}%
\pgfpathcurveto{\pgfqpoint{5.264211in}{1.875788in}}{\pgfqpoint{5.255378in}{1.879447in}}{\pgfqpoint{5.246170in}{1.879447in}}%
\pgfpathcurveto{\pgfqpoint{5.236962in}{1.879447in}}{\pgfqpoint{5.228129in}{1.875788in}}{\pgfqpoint{5.221618in}{1.869277in}}%
\pgfpathcurveto{\pgfqpoint{5.215106in}{1.862766in}}{\pgfqpoint{5.211448in}{1.853933in}}{\pgfqpoint{5.211448in}{1.844725in}}%
\pgfpathcurveto{\pgfqpoint{5.211448in}{1.835516in}}{\pgfqpoint{5.215106in}{1.826684in}}{\pgfqpoint{5.221618in}{1.820172in}}%
\pgfpathcurveto{\pgfqpoint{5.228129in}{1.813661in}}{\pgfqpoint{5.236962in}{1.810002in}}{\pgfqpoint{5.246170in}{1.810002in}}%
\pgfpathlineto{\pgfqpoint{5.246170in}{1.810002in}}%
\pgfpathclose%
\pgfusepath{stroke,fill}%
\end{pgfscope}%
\begin{pgfscope}%
\pgfpathrectangle{\pgfqpoint{0.050000in}{0.050000in}}{\pgfqpoint{2.419000in}{2.419000in}}%
\pgfusepath{clip}%
\pgfsetbuttcap%
\pgfsetroundjoin%
\definecolor{currentfill}{rgb}{0.800000,0.400000,0.466667}%
\pgfsetfillcolor{currentfill}%
\pgfsetfillopacity{0.410080}%
\pgfsetlinewidth{1.003750pt}%
\definecolor{currentstroke}{rgb}{0.800000,0.400000,0.466667}%
\pgfsetstrokecolor{currentstroke}%
\pgfsetstrokeopacity{0.410080}%
\pgfsetdash{}{0pt}%
\pgfpathmoveto{\pgfqpoint{9.660981in}{1.810002in}}%
\pgfpathcurveto{\pgfqpoint{9.670189in}{1.810002in}}{\pgfqpoint{9.679022in}{1.813661in}}{\pgfqpoint{9.685533in}{1.820172in}}%
\pgfpathcurveto{\pgfqpoint{9.692045in}{1.826684in}}{\pgfqpoint{9.695703in}{1.835516in}}{\pgfqpoint{9.695703in}{1.844725in}}%
\pgfpathcurveto{\pgfqpoint{9.695703in}{1.853933in}}{\pgfqpoint{9.692045in}{1.862766in}}{\pgfqpoint{9.685533in}{1.869277in}}%
\pgfpathcurveto{\pgfqpoint{9.679022in}{1.875788in}}{\pgfqpoint{9.670189in}{1.879447in}}{\pgfqpoint{9.660981in}{1.879447in}}%
\pgfpathcurveto{\pgfqpoint{9.651773in}{1.879447in}}{\pgfqpoint{9.642940in}{1.875788in}}{\pgfqpoint{9.636429in}{1.869277in}}%
\pgfpathcurveto{\pgfqpoint{9.629917in}{1.862766in}}{\pgfqpoint{9.626259in}{1.853933in}}{\pgfqpoint{9.626259in}{1.844725in}}%
\pgfpathcurveto{\pgfqpoint{9.626259in}{1.835516in}}{\pgfqpoint{9.629917in}{1.826684in}}{\pgfqpoint{9.636429in}{1.820172in}}%
\pgfpathcurveto{\pgfqpoint{9.642940in}{1.813661in}}{\pgfqpoint{9.651773in}{1.810002in}}{\pgfqpoint{9.660981in}{1.810002in}}%
\pgfpathlineto{\pgfqpoint{9.660981in}{1.810002in}}%
\pgfpathclose%
\pgfusepath{stroke,fill}%
\end{pgfscope}%
\begin{pgfscope}%
\pgfpathrectangle{\pgfqpoint{0.050000in}{0.050000in}}{\pgfqpoint{2.419000in}{2.419000in}}%
\pgfusepath{clip}%
\pgfsetbuttcap%
\pgfsetroundjoin%
\definecolor{currentfill}{rgb}{0.800000,0.400000,0.466667}%
\pgfsetfillcolor{currentfill}%
\pgfsetfillopacity{0.415579}%
\pgfsetlinewidth{1.003750pt}%
\definecolor{currentstroke}{rgb}{0.800000,0.400000,0.466667}%
\pgfsetstrokecolor{currentstroke}%
\pgfsetstrokeopacity{0.415579}%
\pgfsetdash{}{0pt}%
\pgfpathmoveto{\pgfqpoint{-0.998411in}{1.697487in}}%
\pgfpathcurveto{\pgfqpoint{-0.989202in}{1.697487in}}{\pgfqpoint{-0.980370in}{1.701146in}}{\pgfqpoint{-0.973858in}{1.707657in}}%
\pgfpathcurveto{\pgfqpoint{-0.967347in}{1.714168in}}{\pgfqpoint{-0.963688in}{1.723001in}}{\pgfqpoint{-0.963688in}{1.732209in}}%
\pgfpathcurveto{\pgfqpoint{-0.963688in}{1.741418in}}{\pgfqpoint{-0.967347in}{1.750250in}}{\pgfqpoint{-0.973858in}{1.756762in}}%
\pgfpathcurveto{\pgfqpoint{-0.980370in}{1.763273in}}{\pgfqpoint{-0.989202in}{1.766932in}}{\pgfqpoint{-0.998411in}{1.766932in}}%
\pgfpathcurveto{\pgfqpoint{-1.007619in}{1.766932in}}{\pgfqpoint{-1.016452in}{1.763273in}}{\pgfqpoint{-1.022963in}{1.756762in}}%
\pgfpathcurveto{\pgfqpoint{-1.029474in}{1.750250in}}{\pgfqpoint{-1.033133in}{1.741418in}}{\pgfqpoint{-1.033133in}{1.732209in}}%
\pgfpathcurveto{\pgfqpoint{-1.033133in}{1.723001in}}{\pgfqpoint{-1.029474in}{1.714168in}}{\pgfqpoint{-1.022963in}{1.707657in}}%
\pgfpathcurveto{\pgfqpoint{-1.016452in}{1.701146in}}{\pgfqpoint{-1.007619in}{1.697487in}}{\pgfqpoint{-0.998411in}{1.697487in}}%
\pgfpathlineto{\pgfqpoint{-0.998411in}{1.697487in}}%
\pgfpathclose%
\pgfusepath{stroke,fill}%
\end{pgfscope}%
\begin{pgfscope}%
\pgfpathrectangle{\pgfqpoint{0.050000in}{0.050000in}}{\pgfqpoint{2.419000in}{2.419000in}}%
\pgfusepath{clip}%
\pgfsetbuttcap%
\pgfsetroundjoin%
\definecolor{currentfill}{rgb}{0.800000,0.400000,0.466667}%
\pgfsetfillcolor{currentfill}%
\pgfsetfillopacity{0.415579}%
\pgfsetlinewidth{1.003750pt}%
\definecolor{currentstroke}{rgb}{0.800000,0.400000,0.466667}%
\pgfsetstrokecolor{currentstroke}%
\pgfsetstrokeopacity{0.415579}%
\pgfsetdash{}{0pt}%
\pgfpathmoveto{\pgfqpoint{3.487396in}{1.697487in}}%
\pgfpathcurveto{\pgfqpoint{3.496604in}{1.697487in}}{\pgfqpoint{3.505437in}{1.701146in}}{\pgfqpoint{3.511948in}{1.707657in}}%
\pgfpathcurveto{\pgfqpoint{3.518459in}{1.714168in}}{\pgfqpoint{3.522118in}{1.723001in}}{\pgfqpoint{3.522118in}{1.732209in}}%
\pgfpathcurveto{\pgfqpoint{3.522118in}{1.741418in}}{\pgfqpoint{3.518459in}{1.750250in}}{\pgfqpoint{3.511948in}{1.756762in}}%
\pgfpathcurveto{\pgfqpoint{3.505437in}{1.763273in}}{\pgfqpoint{3.496604in}{1.766932in}}{\pgfqpoint{3.487396in}{1.766932in}}%
\pgfpathcurveto{\pgfqpoint{3.478187in}{1.766932in}}{\pgfqpoint{3.469355in}{1.763273in}}{\pgfqpoint{3.462843in}{1.756762in}}%
\pgfpathcurveto{\pgfqpoint{3.456332in}{1.750250in}}{\pgfqpoint{3.452674in}{1.741418in}}{\pgfqpoint{3.452674in}{1.732209in}}%
\pgfpathcurveto{\pgfqpoint{3.452674in}{1.723001in}}{\pgfqpoint{3.456332in}{1.714168in}}{\pgfqpoint{3.462843in}{1.707657in}}%
\pgfpathcurveto{\pgfqpoint{3.469355in}{1.701146in}}{\pgfqpoint{3.478187in}{1.697487in}}{\pgfqpoint{3.487396in}{1.697487in}}%
\pgfpathlineto{\pgfqpoint{3.487396in}{1.697487in}}%
\pgfpathclose%
\pgfusepath{stroke,fill}%
\end{pgfscope}%
\begin{pgfscope}%
\pgfpathrectangle{\pgfqpoint{0.050000in}{0.050000in}}{\pgfqpoint{2.419000in}{2.419000in}}%
\pgfusepath{clip}%
\pgfsetbuttcap%
\pgfsetroundjoin%
\definecolor{currentfill}{rgb}{0.800000,0.400000,0.466667}%
\pgfsetfillcolor{currentfill}%
\pgfsetfillopacity{0.415579}%
\pgfsetlinewidth{1.003750pt}%
\definecolor{currentstroke}{rgb}{0.800000,0.400000,0.466667}%
\pgfsetstrokecolor{currentstroke}%
\pgfsetstrokeopacity{0.415579}%
\pgfsetdash{}{0pt}%
\pgfpathmoveto{\pgfqpoint{7.973202in}{1.697487in}}%
\pgfpathcurveto{\pgfqpoint{7.982411in}{1.697487in}}{\pgfqpoint{7.991243in}{1.701146in}}{\pgfqpoint{7.997754in}{1.707657in}}%
\pgfpathcurveto{\pgfqpoint{8.004266in}{1.714168in}}{\pgfqpoint{8.007924in}{1.723001in}}{\pgfqpoint{8.007924in}{1.732209in}}%
\pgfpathcurveto{\pgfqpoint{8.007924in}{1.741418in}}{\pgfqpoint{8.004266in}{1.750250in}}{\pgfqpoint{7.997754in}{1.756762in}}%
\pgfpathcurveto{\pgfqpoint{7.991243in}{1.763273in}}{\pgfqpoint{7.982411in}{1.766932in}}{\pgfqpoint{7.973202in}{1.766932in}}%
\pgfpathcurveto{\pgfqpoint{7.963994in}{1.766932in}}{\pgfqpoint{7.955161in}{1.763273in}}{\pgfqpoint{7.948650in}{1.756762in}}%
\pgfpathcurveto{\pgfqpoint{7.942138in}{1.750250in}}{\pgfqpoint{7.938480in}{1.741418in}}{\pgfqpoint{7.938480in}{1.732209in}}%
\pgfpathcurveto{\pgfqpoint{7.938480in}{1.723001in}}{\pgfqpoint{7.942138in}{1.714168in}}{\pgfqpoint{7.948650in}{1.707657in}}%
\pgfpathcurveto{\pgfqpoint{7.955161in}{1.701146in}}{\pgfqpoint{7.963994in}{1.697487in}}{\pgfqpoint{7.973202in}{1.697487in}}%
\pgfpathlineto{\pgfqpoint{7.973202in}{1.697487in}}%
\pgfpathclose%
\pgfusepath{stroke,fill}%
\end{pgfscope}%
\begin{pgfscope}%
\pgfpathrectangle{\pgfqpoint{0.050000in}{0.050000in}}{\pgfqpoint{2.419000in}{2.419000in}}%
\pgfusepath{clip}%
\pgfsetbuttcap%
\pgfsetroundjoin%
\definecolor{currentfill}{rgb}{0.800000,0.400000,0.466667}%
\pgfsetfillcolor{currentfill}%
\pgfsetfillopacity{0.421257}%
\pgfsetlinewidth{1.003750pt}%
\definecolor{currentstroke}{rgb}{0.800000,0.400000,0.466667}%
\pgfsetstrokecolor{currentstroke}%
\pgfsetstrokeopacity{0.421257}%
\pgfsetdash{}{0pt}%
\pgfpathmoveto{\pgfqpoint{-2.887992in}{1.581294in}}%
\pgfpathcurveto{\pgfqpoint{-2.878783in}{1.581294in}}{\pgfqpoint{-2.869951in}{1.584952in}}{\pgfqpoint{-2.863439in}{1.591464in}}%
\pgfpathcurveto{\pgfqpoint{-2.856928in}{1.597975in}}{\pgfqpoint{-2.853269in}{1.606808in}}{\pgfqpoint{-2.853269in}{1.616016in}}%
\pgfpathcurveto{\pgfqpoint{-2.853269in}{1.625225in}}{\pgfqpoint{-2.856928in}{1.634057in}}{\pgfqpoint{-2.863439in}{1.640568in}}%
\pgfpathcurveto{\pgfqpoint{-2.869951in}{1.647080in}}{\pgfqpoint{-2.878783in}{1.650738in}}{\pgfqpoint{-2.887992in}{1.650738in}}%
\pgfpathcurveto{\pgfqpoint{-2.897200in}{1.650738in}}{\pgfqpoint{-2.906033in}{1.647080in}}{\pgfqpoint{-2.912544in}{1.640568in}}%
\pgfpathcurveto{\pgfqpoint{-2.919055in}{1.634057in}}{\pgfqpoint{-2.922714in}{1.625225in}}{\pgfqpoint{-2.922714in}{1.616016in}}%
\pgfpathcurveto{\pgfqpoint{-2.922714in}{1.606808in}}{\pgfqpoint{-2.919055in}{1.597975in}}{\pgfqpoint{-2.912544in}{1.591464in}}%
\pgfpathcurveto{\pgfqpoint{-2.906033in}{1.584952in}}{\pgfqpoint{-2.897200in}{1.581294in}}{\pgfqpoint{-2.887992in}{1.581294in}}%
\pgfpathlineto{\pgfqpoint{-2.887992in}{1.581294in}}%
\pgfpathclose%
\pgfusepath{stroke,fill}%
\end{pgfscope}%
\begin{pgfscope}%
\pgfpathrectangle{\pgfqpoint{0.050000in}{0.050000in}}{\pgfqpoint{2.419000in}{2.419000in}}%
\pgfusepath{clip}%
\pgfsetbuttcap%
\pgfsetroundjoin%
\definecolor{currentfill}{rgb}{0.800000,0.400000,0.466667}%
\pgfsetfillcolor{currentfill}%
\pgfsetfillopacity{0.421257}%
\pgfsetlinewidth{1.003750pt}%
\definecolor{currentstroke}{rgb}{0.800000,0.400000,0.466667}%
\pgfsetstrokecolor{currentstroke}%
\pgfsetstrokeopacity{0.421257}%
\pgfsetdash{}{0pt}%
\pgfpathmoveto{\pgfqpoint{1.671131in}{1.581294in}}%
\pgfpathcurveto{\pgfqpoint{1.680339in}{1.581294in}}{\pgfqpoint{1.689172in}{1.584952in}}{\pgfqpoint{1.695683in}{1.591464in}}%
\pgfpathcurveto{\pgfqpoint{1.702194in}{1.597975in}}{\pgfqpoint{1.705853in}{1.606808in}}{\pgfqpoint{1.705853in}{1.616016in}}%
\pgfpathcurveto{\pgfqpoint{1.705853in}{1.625225in}}{\pgfqpoint{1.702194in}{1.634057in}}{\pgfqpoint{1.695683in}{1.640568in}}%
\pgfpathcurveto{\pgfqpoint{1.689172in}{1.647080in}}{\pgfqpoint{1.680339in}{1.650738in}}{\pgfqpoint{1.671131in}{1.650738in}}%
\pgfpathcurveto{\pgfqpoint{1.661922in}{1.650738in}}{\pgfqpoint{1.653090in}{1.647080in}}{\pgfqpoint{1.646578in}{1.640568in}}%
\pgfpathcurveto{\pgfqpoint{1.640067in}{1.634057in}}{\pgfqpoint{1.636409in}{1.625225in}}{\pgfqpoint{1.636409in}{1.616016in}}%
\pgfpathcurveto{\pgfqpoint{1.636409in}{1.606808in}}{\pgfqpoint{1.640067in}{1.597975in}}{\pgfqpoint{1.646578in}{1.591464in}}%
\pgfpathcurveto{\pgfqpoint{1.653090in}{1.584952in}}{\pgfqpoint{1.661922in}{1.581294in}}{\pgfqpoint{1.671131in}{1.581294in}}%
\pgfpathlineto{\pgfqpoint{1.671131in}{1.581294in}}%
\pgfpathclose%
\pgfusepath{stroke,fill}%
\end{pgfscope}%
\begin{pgfscope}%
\pgfpathrectangle{\pgfqpoint{0.050000in}{0.050000in}}{\pgfqpoint{2.419000in}{2.419000in}}%
\pgfusepath{clip}%
\pgfsetbuttcap%
\pgfsetroundjoin%
\definecolor{currentfill}{rgb}{0.800000,0.400000,0.466667}%
\pgfsetfillcolor{currentfill}%
\pgfsetfillopacity{0.421257}%
\pgfsetlinewidth{1.003750pt}%
\definecolor{currentstroke}{rgb}{0.800000,0.400000,0.466667}%
\pgfsetstrokecolor{currentstroke}%
\pgfsetstrokeopacity{0.421257}%
\pgfsetdash{}{0pt}%
\pgfpathmoveto{\pgfqpoint{6.230253in}{1.581294in}}%
\pgfpathcurveto{\pgfqpoint{6.239462in}{1.581294in}}{\pgfqpoint{6.248294in}{1.584952in}}{\pgfqpoint{6.254805in}{1.591464in}}%
\pgfpathcurveto{\pgfqpoint{6.261317in}{1.597975in}}{\pgfqpoint{6.264975in}{1.606808in}}{\pgfqpoint{6.264975in}{1.616016in}}%
\pgfpathcurveto{\pgfqpoint{6.264975in}{1.625225in}}{\pgfqpoint{6.261317in}{1.634057in}}{\pgfqpoint{6.254805in}{1.640568in}}%
\pgfpathcurveto{\pgfqpoint{6.248294in}{1.647080in}}{\pgfqpoint{6.239462in}{1.650738in}}{\pgfqpoint{6.230253in}{1.650738in}}%
\pgfpathcurveto{\pgfqpoint{6.221045in}{1.650738in}}{\pgfqpoint{6.212212in}{1.647080in}}{\pgfqpoint{6.205701in}{1.640568in}}%
\pgfpathcurveto{\pgfqpoint{6.199189in}{1.634057in}}{\pgfqpoint{6.195531in}{1.625225in}}{\pgfqpoint{6.195531in}{1.616016in}}%
\pgfpathcurveto{\pgfqpoint{6.195531in}{1.606808in}}{\pgfqpoint{6.199189in}{1.597975in}}{\pgfqpoint{6.205701in}{1.591464in}}%
\pgfpathcurveto{\pgfqpoint{6.212212in}{1.584952in}}{\pgfqpoint{6.221045in}{1.581294in}}{\pgfqpoint{6.230253in}{1.581294in}}%
\pgfpathlineto{\pgfqpoint{6.230253in}{1.581294in}}%
\pgfpathclose%
\pgfusepath{stroke,fill}%
\end{pgfscope}%
\begin{pgfscope}%
\pgfpathrectangle{\pgfqpoint{0.050000in}{0.050000in}}{\pgfqpoint{2.419000in}{2.419000in}}%
\pgfusepath{clip}%
\pgfsetbuttcap%
\pgfsetroundjoin%
\definecolor{currentfill}{rgb}{0.800000,0.400000,0.466667}%
\pgfsetfillcolor{currentfill}%
\pgfsetfillopacity{0.427124}%
\pgfsetlinewidth{1.003750pt}%
\definecolor{currentstroke}{rgb}{0.800000,0.400000,0.466667}%
\pgfsetstrokecolor{currentstroke}%
\pgfsetstrokeopacity{0.427124}%
\pgfsetdash{}{0pt}%
\pgfpathmoveto{\pgfqpoint{4.429384in}{1.461240in}}%
\pgfpathcurveto{\pgfqpoint{4.438592in}{1.461240in}}{\pgfqpoint{4.447425in}{1.464898in}}{\pgfqpoint{4.453936in}{1.471409in}}%
\pgfpathcurveto{\pgfqpoint{4.460448in}{1.477921in}}{\pgfqpoint{4.464106in}{1.486753in}}{\pgfqpoint{4.464106in}{1.495962in}}%
\pgfpathcurveto{\pgfqpoint{4.464106in}{1.505170in}}{\pgfqpoint{4.460448in}{1.514003in}}{\pgfqpoint{4.453936in}{1.520514in}}%
\pgfpathcurveto{\pgfqpoint{4.447425in}{1.527025in}}{\pgfqpoint{4.438592in}{1.530684in}}{\pgfqpoint{4.429384in}{1.530684in}}%
\pgfpathcurveto{\pgfqpoint{4.420176in}{1.530684in}}{\pgfqpoint{4.411343in}{1.527025in}}{\pgfqpoint{4.404832in}{1.520514in}}%
\pgfpathcurveto{\pgfqpoint{4.398320in}{1.514003in}}{\pgfqpoint{4.394662in}{1.505170in}}{\pgfqpoint{4.394662in}{1.495962in}}%
\pgfpathcurveto{\pgfqpoint{4.394662in}{1.486753in}}{\pgfqpoint{4.398320in}{1.477921in}}{\pgfqpoint{4.404832in}{1.471409in}}%
\pgfpathcurveto{\pgfqpoint{4.411343in}{1.464898in}}{\pgfqpoint{4.420176in}{1.461240in}}{\pgfqpoint{4.429384in}{1.461240in}}%
\pgfpathlineto{\pgfqpoint{4.429384in}{1.461240in}}%
\pgfpathclose%
\pgfusepath{stroke,fill}%
\end{pgfscope}%
\begin{pgfscope}%
\pgfpathrectangle{\pgfqpoint{0.050000in}{0.050000in}}{\pgfqpoint{2.419000in}{2.419000in}}%
\pgfusepath{clip}%
\pgfsetbuttcap%
\pgfsetroundjoin%
\definecolor{currentfill}{rgb}{0.800000,0.400000,0.466667}%
\pgfsetfillcolor{currentfill}%
\pgfsetfillopacity{0.427124}%
\pgfsetlinewidth{1.003750pt}%
\definecolor{currentstroke}{rgb}{0.800000,0.400000,0.466667}%
\pgfsetstrokecolor{currentstroke}%
\pgfsetstrokeopacity{0.427124}%
\pgfsetdash{}{0pt}%
\pgfpathmoveto{\pgfqpoint{9.064259in}{1.461240in}}%
\pgfpathcurveto{\pgfqpoint{9.073467in}{1.461240in}}{\pgfqpoint{9.082300in}{1.464898in}}{\pgfqpoint{9.088811in}{1.471409in}}%
\pgfpathcurveto{\pgfqpoint{9.095322in}{1.477921in}}{\pgfqpoint{9.098981in}{1.486753in}}{\pgfqpoint{9.098981in}{1.495962in}}%
\pgfpathcurveto{\pgfqpoint{9.098981in}{1.505170in}}{\pgfqpoint{9.095322in}{1.514003in}}{\pgfqpoint{9.088811in}{1.520514in}}%
\pgfpathcurveto{\pgfqpoint{9.082300in}{1.527025in}}{\pgfqpoint{9.073467in}{1.530684in}}{\pgfqpoint{9.064259in}{1.530684in}}%
\pgfpathcurveto{\pgfqpoint{9.055050in}{1.530684in}}{\pgfqpoint{9.046218in}{1.527025in}}{\pgfqpoint{9.039706in}{1.520514in}}%
\pgfpathcurveto{\pgfqpoint{9.033195in}{1.514003in}}{\pgfqpoint{9.029537in}{1.505170in}}{\pgfqpoint{9.029537in}{1.495962in}}%
\pgfpathcurveto{\pgfqpoint{9.029537in}{1.486753in}}{\pgfqpoint{9.033195in}{1.477921in}}{\pgfqpoint{9.039706in}{1.471409in}}%
\pgfpathcurveto{\pgfqpoint{9.046218in}{1.464898in}}{\pgfqpoint{9.055050in}{1.461240in}}{\pgfqpoint{9.064259in}{1.461240in}}%
\pgfpathlineto{\pgfqpoint{9.064259in}{1.461240in}}%
\pgfpathclose%
\pgfusepath{stroke,fill}%
\end{pgfscope}%
\begin{pgfscope}%
\pgfpathrectangle{\pgfqpoint{0.050000in}{0.050000in}}{\pgfqpoint{2.419000in}{2.419000in}}%
\pgfusepath{clip}%
\pgfsetbuttcap%
\pgfsetroundjoin%
\definecolor{currentfill}{rgb}{0.800000,0.400000,0.466667}%
\pgfsetfillcolor{currentfill}%
\pgfsetfillopacity{0.427124}%
\pgfsetlinewidth{1.003750pt}%
\definecolor{currentstroke}{rgb}{0.800000,0.400000,0.466667}%
\pgfsetstrokecolor{currentstroke}%
\pgfsetstrokeopacity{0.427124}%
\pgfsetdash{}{0pt}%
\pgfpathmoveto{\pgfqpoint{-0.205491in}{1.461240in}}%
\pgfpathcurveto{\pgfqpoint{-0.196282in}{1.461240in}}{\pgfqpoint{-0.187450in}{1.464898in}}{\pgfqpoint{-0.180938in}{1.471409in}}%
\pgfpathcurveto{\pgfqpoint{-0.174427in}{1.477921in}}{\pgfqpoint{-0.170769in}{1.486753in}}{\pgfqpoint{-0.170769in}{1.495962in}}%
\pgfpathcurveto{\pgfqpoint{-0.170769in}{1.505170in}}{\pgfqpoint{-0.174427in}{1.514003in}}{\pgfqpoint{-0.180938in}{1.520514in}}%
\pgfpathcurveto{\pgfqpoint{-0.187450in}{1.527025in}}{\pgfqpoint{-0.196282in}{1.530684in}}{\pgfqpoint{-0.205491in}{1.530684in}}%
\pgfpathcurveto{\pgfqpoint{-0.214699in}{1.530684in}}{\pgfqpoint{-0.223532in}{1.527025in}}{\pgfqpoint{-0.230043in}{1.520514in}}%
\pgfpathcurveto{\pgfqpoint{-0.236554in}{1.514003in}}{\pgfqpoint{-0.240213in}{1.505170in}}{\pgfqpoint{-0.240213in}{1.495962in}}%
\pgfpathcurveto{\pgfqpoint{-0.240213in}{1.486753in}}{\pgfqpoint{-0.236554in}{1.477921in}}{\pgfqpoint{-0.230043in}{1.471409in}}%
\pgfpathcurveto{\pgfqpoint{-0.223532in}{1.464898in}}{\pgfqpoint{-0.214699in}{1.461240in}}{\pgfqpoint{-0.205491in}{1.461240in}}%
\pgfpathlineto{\pgfqpoint{-0.205491in}{1.461240in}}%
\pgfpathclose%
\pgfusepath{stroke,fill}%
\end{pgfscope}%
\begin{pgfscope}%
\pgfpathrectangle{\pgfqpoint{0.050000in}{0.050000in}}{\pgfqpoint{2.419000in}{2.419000in}}%
\pgfusepath{clip}%
\pgfsetbuttcap%
\pgfsetroundjoin%
\definecolor{currentfill}{rgb}{0.800000,0.400000,0.466667}%
\pgfsetfillcolor{currentfill}%
\pgfsetfillopacity{0.433189}%
\pgfsetlinewidth{1.003750pt}%
\definecolor{currentstroke}{rgb}{0.800000,0.400000,0.466667}%
\pgfsetstrokecolor{currentstroke}%
\pgfsetstrokeopacity{0.433189}%
\pgfsetdash{}{0pt}%
\pgfpathmoveto{\pgfqpoint{-2.145528in}{1.337128in}}%
\pgfpathcurveto{\pgfqpoint{-2.136320in}{1.337128in}}{\pgfqpoint{-2.127487in}{1.340787in}}{\pgfqpoint{-2.120976in}{1.347298in}}%
\pgfpathcurveto{\pgfqpoint{-2.114465in}{1.353809in}}{\pgfqpoint{-2.110806in}{1.362642in}}{\pgfqpoint{-2.110806in}{1.371850in}}%
\pgfpathcurveto{\pgfqpoint{-2.110806in}{1.381059in}}{\pgfqpoint{-2.114465in}{1.389891in}}{\pgfqpoint{-2.120976in}{1.396403in}}%
\pgfpathcurveto{\pgfqpoint{-2.127487in}{1.402914in}}{\pgfqpoint{-2.136320in}{1.406573in}}{\pgfqpoint{-2.145528in}{1.406573in}}%
\pgfpathcurveto{\pgfqpoint{-2.154737in}{1.406573in}}{\pgfqpoint{-2.163569in}{1.402914in}}{\pgfqpoint{-2.170081in}{1.396403in}}%
\pgfpathcurveto{\pgfqpoint{-2.176592in}{1.389891in}}{\pgfqpoint{-2.180251in}{1.381059in}}{\pgfqpoint{-2.180251in}{1.371850in}}%
\pgfpathcurveto{\pgfqpoint{-2.180251in}{1.362642in}}{\pgfqpoint{-2.176592in}{1.353809in}}{\pgfqpoint{-2.170081in}{1.347298in}}%
\pgfpathcurveto{\pgfqpoint{-2.163569in}{1.340787in}}{\pgfqpoint{-2.154737in}{1.337128in}}{\pgfqpoint{-2.145528in}{1.337128in}}%
\pgfpathlineto{\pgfqpoint{-2.145528in}{1.337128in}}%
\pgfpathclose%
\pgfusepath{stroke,fill}%
\end{pgfscope}%
\begin{pgfscope}%
\pgfpathrectangle{\pgfqpoint{0.050000in}{0.050000in}}{\pgfqpoint{2.419000in}{2.419000in}}%
\pgfusepath{clip}%
\pgfsetbuttcap%
\pgfsetroundjoin%
\definecolor{currentfill}{rgb}{0.800000,0.400000,0.466667}%
\pgfsetfillcolor{currentfill}%
\pgfsetfillopacity{0.433189}%
\pgfsetlinewidth{1.003750pt}%
\definecolor{currentstroke}{rgb}{0.800000,0.400000,0.466667}%
\pgfsetstrokecolor{currentstroke}%
\pgfsetstrokeopacity{0.433189}%
\pgfsetdash{}{0pt}%
\pgfpathmoveto{\pgfqpoint{2.567659in}{1.337128in}}%
\pgfpathcurveto{\pgfqpoint{2.576867in}{1.337128in}}{\pgfqpoint{2.585700in}{1.340787in}}{\pgfqpoint{2.592211in}{1.347298in}}%
\pgfpathcurveto{\pgfqpoint{2.598722in}{1.353809in}}{\pgfqpoint{2.602381in}{1.362642in}}{\pgfqpoint{2.602381in}{1.371850in}}%
\pgfpathcurveto{\pgfqpoint{2.602381in}{1.381059in}}{\pgfqpoint{2.598722in}{1.389891in}}{\pgfqpoint{2.592211in}{1.396403in}}%
\pgfpathcurveto{\pgfqpoint{2.585700in}{1.402914in}}{\pgfqpoint{2.576867in}{1.406573in}}{\pgfqpoint{2.567659in}{1.406573in}}%
\pgfpathcurveto{\pgfqpoint{2.558450in}{1.406573in}}{\pgfqpoint{2.549618in}{1.402914in}}{\pgfqpoint{2.543106in}{1.396403in}}%
\pgfpathcurveto{\pgfqpoint{2.536595in}{1.389891in}}{\pgfqpoint{2.532937in}{1.381059in}}{\pgfqpoint{2.532937in}{1.371850in}}%
\pgfpathcurveto{\pgfqpoint{2.532937in}{1.362642in}}{\pgfqpoint{2.536595in}{1.353809in}}{\pgfqpoint{2.543106in}{1.347298in}}%
\pgfpathcurveto{\pgfqpoint{2.549618in}{1.340787in}}{\pgfqpoint{2.558450in}{1.337128in}}{\pgfqpoint{2.567659in}{1.337128in}}%
\pgfpathlineto{\pgfqpoint{2.567659in}{1.337128in}}%
\pgfpathclose%
\pgfusepath{stroke,fill}%
\end{pgfscope}%
\begin{pgfscope}%
\pgfpathrectangle{\pgfqpoint{0.050000in}{0.050000in}}{\pgfqpoint{2.419000in}{2.419000in}}%
\pgfusepath{clip}%
\pgfsetbuttcap%
\pgfsetroundjoin%
\definecolor{currentfill}{rgb}{0.800000,0.400000,0.466667}%
\pgfsetfillcolor{currentfill}%
\pgfsetfillopacity{0.433189}%
\pgfsetlinewidth{1.003750pt}%
\definecolor{currentstroke}{rgb}{0.800000,0.400000,0.466667}%
\pgfsetstrokecolor{currentstroke}%
\pgfsetstrokeopacity{0.433189}%
\pgfsetdash{}{0pt}%
\pgfpathmoveto{\pgfqpoint{7.280846in}{1.337128in}}%
\pgfpathcurveto{\pgfqpoint{7.290054in}{1.337128in}}{\pgfqpoint{7.298887in}{1.340787in}}{\pgfqpoint{7.305398in}{1.347298in}}%
\pgfpathcurveto{\pgfqpoint{7.311909in}{1.353809in}}{\pgfqpoint{7.315568in}{1.362642in}}{\pgfqpoint{7.315568in}{1.371850in}}%
\pgfpathcurveto{\pgfqpoint{7.315568in}{1.381059in}}{\pgfqpoint{7.311909in}{1.389891in}}{\pgfqpoint{7.305398in}{1.396403in}}%
\pgfpathcurveto{\pgfqpoint{7.298887in}{1.402914in}}{\pgfqpoint{7.290054in}{1.406573in}}{\pgfqpoint{7.280846in}{1.406573in}}%
\pgfpathcurveto{\pgfqpoint{7.271637in}{1.406573in}}{\pgfqpoint{7.262805in}{1.402914in}}{\pgfqpoint{7.256293in}{1.396403in}}%
\pgfpathcurveto{\pgfqpoint{7.249782in}{1.389891in}}{\pgfqpoint{7.246124in}{1.381059in}}{\pgfqpoint{7.246124in}{1.371850in}}%
\pgfpathcurveto{\pgfqpoint{7.246124in}{1.362642in}}{\pgfqpoint{7.249782in}{1.353809in}}{\pgfqpoint{7.256293in}{1.347298in}}%
\pgfpathcurveto{\pgfqpoint{7.262805in}{1.340787in}}{\pgfqpoint{7.271637in}{1.337128in}}{\pgfqpoint{7.280846in}{1.337128in}}%
\pgfpathlineto{\pgfqpoint{7.280846in}{1.337128in}}%
\pgfpathclose%
\pgfusepath{stroke,fill}%
\end{pgfscope}%
\begin{pgfscope}%
\pgfpathrectangle{\pgfqpoint{0.050000in}{0.050000in}}{\pgfqpoint{2.419000in}{2.419000in}}%
\pgfusepath{clip}%
\pgfsetbuttcap%
\pgfsetroundjoin%
\definecolor{currentfill}{rgb}{0.800000,0.400000,0.466667}%
\pgfsetfillcolor{currentfill}%
\pgfsetfillopacity{0.439463}%
\pgfsetlinewidth{1.003750pt}%
\definecolor{currentstroke}{rgb}{0.800000,0.400000,0.466667}%
\pgfsetstrokecolor{currentstroke}%
\pgfsetstrokeopacity{0.439463}%
\pgfsetdash{}{0pt}%
\pgfpathmoveto{\pgfqpoint{10.230322in}{1.208751in}}%
\pgfpathcurveto{\pgfqpoint{10.239530in}{1.208751in}}{\pgfqpoint{10.248363in}{1.212409in}}{\pgfqpoint{10.254874in}{1.218921in}}%
\pgfpathcurveto{\pgfqpoint{10.261386in}{1.225432in}}{\pgfqpoint{10.265044in}{1.234264in}}{\pgfqpoint{10.265044in}{1.243473in}}%
\pgfpathcurveto{\pgfqpoint{10.265044in}{1.252681in}}{\pgfqpoint{10.261386in}{1.261514in}}{\pgfqpoint{10.254874in}{1.268025in}}%
\pgfpathcurveto{\pgfqpoint{10.248363in}{1.274537in}}{\pgfqpoint{10.239530in}{1.278195in}}{\pgfqpoint{10.230322in}{1.278195in}}%
\pgfpathcurveto{\pgfqpoint{10.221114in}{1.278195in}}{\pgfqpoint{10.212281in}{1.274537in}}{\pgfqpoint{10.205770in}{1.268025in}}%
\pgfpathcurveto{\pgfqpoint{10.199258in}{1.261514in}}{\pgfqpoint{10.195600in}{1.252681in}}{\pgfqpoint{10.195600in}{1.243473in}}%
\pgfpathcurveto{\pgfqpoint{10.195600in}{1.234264in}}{\pgfqpoint{10.199258in}{1.225432in}}{\pgfqpoint{10.205770in}{1.218921in}}%
\pgfpathcurveto{\pgfqpoint{10.212281in}{1.212409in}}{\pgfqpoint{10.221114in}{1.208751in}}{\pgfqpoint{10.230322in}{1.208751in}}%
\pgfpathlineto{\pgfqpoint{10.230322in}{1.208751in}}%
\pgfpathclose%
\pgfusepath{stroke,fill}%
\end{pgfscope}%
\begin{pgfscope}%
\pgfpathrectangle{\pgfqpoint{0.050000in}{0.050000in}}{\pgfqpoint{2.419000in}{2.419000in}}%
\pgfusepath{clip}%
\pgfsetbuttcap%
\pgfsetroundjoin%
\definecolor{currentfill}{rgb}{0.800000,0.400000,0.466667}%
\pgfsetfillcolor{currentfill}%
\pgfsetfillopacity{0.439463}%
\pgfsetlinewidth{1.003750pt}%
\definecolor{currentstroke}{rgb}{0.800000,0.400000,0.466667}%
\pgfsetstrokecolor{currentstroke}%
\pgfsetstrokeopacity{0.439463}%
\pgfsetdash{}{0pt}%
\pgfpathmoveto{\pgfqpoint{0.641940in}{1.208751in}}%
\pgfpathcurveto{\pgfqpoint{0.651148in}{1.208751in}}{\pgfqpoint{0.659981in}{1.212409in}}{\pgfqpoint{0.666492in}{1.218921in}}%
\pgfpathcurveto{\pgfqpoint{0.673003in}{1.225432in}}{\pgfqpoint{0.676662in}{1.234264in}}{\pgfqpoint{0.676662in}{1.243473in}}%
\pgfpathcurveto{\pgfqpoint{0.676662in}{1.252681in}}{\pgfqpoint{0.673003in}{1.261514in}}{\pgfqpoint{0.666492in}{1.268025in}}%
\pgfpathcurveto{\pgfqpoint{0.659981in}{1.274537in}}{\pgfqpoint{0.651148in}{1.278195in}}{\pgfqpoint{0.641940in}{1.278195in}}%
\pgfpathcurveto{\pgfqpoint{0.632731in}{1.278195in}}{\pgfqpoint{0.623899in}{1.274537in}}{\pgfqpoint{0.617387in}{1.268025in}}%
\pgfpathcurveto{\pgfqpoint{0.610876in}{1.261514in}}{\pgfqpoint{0.607217in}{1.252681in}}{\pgfqpoint{0.607217in}{1.243473in}}%
\pgfpathcurveto{\pgfqpoint{0.607217in}{1.234264in}}{\pgfqpoint{0.610876in}{1.225432in}}{\pgfqpoint{0.617387in}{1.218921in}}%
\pgfpathcurveto{\pgfqpoint{0.623899in}{1.212409in}}{\pgfqpoint{0.632731in}{1.208751in}}{\pgfqpoint{0.641940in}{1.208751in}}%
\pgfpathlineto{\pgfqpoint{0.641940in}{1.208751in}}%
\pgfpathclose%
\pgfusepath{stroke,fill}%
\end{pgfscope}%
\begin{pgfscope}%
\pgfpathrectangle{\pgfqpoint{0.050000in}{0.050000in}}{\pgfqpoint{2.419000in}{2.419000in}}%
\pgfusepath{clip}%
\pgfsetbuttcap%
\pgfsetroundjoin%
\definecolor{currentfill}{rgb}{0.800000,0.400000,0.466667}%
\pgfsetfillcolor{currentfill}%
\pgfsetfillopacity{0.439463}%
\pgfsetlinewidth{1.003750pt}%
\definecolor{currentstroke}{rgb}{0.800000,0.400000,0.466667}%
\pgfsetstrokecolor{currentstroke}%
\pgfsetstrokeopacity{0.439463}%
\pgfsetdash{}{0pt}%
\pgfpathmoveto{\pgfqpoint{5.436131in}{1.208751in}}%
\pgfpathcurveto{\pgfqpoint{5.445339in}{1.208751in}}{\pgfqpoint{5.454172in}{1.212409in}}{\pgfqpoint{5.460683in}{1.218921in}}%
\pgfpathcurveto{\pgfqpoint{5.467195in}{1.225432in}}{\pgfqpoint{5.470853in}{1.234264in}}{\pgfqpoint{5.470853in}{1.243473in}}%
\pgfpathcurveto{\pgfqpoint{5.470853in}{1.252681in}}{\pgfqpoint{5.467195in}{1.261514in}}{\pgfqpoint{5.460683in}{1.268025in}}%
\pgfpathcurveto{\pgfqpoint{5.454172in}{1.274537in}}{\pgfqpoint{5.445339in}{1.278195in}}{\pgfqpoint{5.436131in}{1.278195in}}%
\pgfpathcurveto{\pgfqpoint{5.426922in}{1.278195in}}{\pgfqpoint{5.418090in}{1.274537in}}{\pgfqpoint{5.411579in}{1.268025in}}%
\pgfpathcurveto{\pgfqpoint{5.405067in}{1.261514in}}{\pgfqpoint{5.401409in}{1.252681in}}{\pgfqpoint{5.401409in}{1.243473in}}%
\pgfpathcurveto{\pgfqpoint{5.401409in}{1.234264in}}{\pgfqpoint{5.405067in}{1.225432in}}{\pgfqpoint{5.411579in}{1.218921in}}%
\pgfpathcurveto{\pgfqpoint{5.418090in}{1.212409in}}{\pgfqpoint{5.426922in}{1.208751in}}{\pgfqpoint{5.436131in}{1.208751in}}%
\pgfpathlineto{\pgfqpoint{5.436131in}{1.208751in}}%
\pgfpathclose%
\pgfusepath{stroke,fill}%
\end{pgfscope}%
\begin{pgfscope}%
\pgfpathrectangle{\pgfqpoint{0.050000in}{0.050000in}}{\pgfqpoint{2.419000in}{2.419000in}}%
\pgfusepath{clip}%
\pgfsetbuttcap%
\pgfsetroundjoin%
\definecolor{currentfill}{rgb}{0.800000,0.400000,0.466667}%
\pgfsetfillcolor{currentfill}%
\pgfsetfillopacity{0.445956}%
\pgfsetlinewidth{1.003750pt}%
\definecolor{currentstroke}{rgb}{0.800000,0.400000,0.466667}%
\pgfsetstrokecolor{currentstroke}%
\pgfsetstrokeopacity{0.445956}%
\pgfsetdash{}{0pt}%
\pgfpathmoveto{\pgfqpoint{8.404926in}{1.075883in}}%
\pgfpathcurveto{\pgfqpoint{8.414135in}{1.075883in}}{\pgfqpoint{8.422967in}{1.079542in}}{\pgfqpoint{8.429479in}{1.086053in}}%
\pgfpathcurveto{\pgfqpoint{8.435990in}{1.092565in}}{\pgfqpoint{8.439649in}{1.101397in}}{\pgfqpoint{8.439649in}{1.110605in}}%
\pgfpathcurveto{\pgfqpoint{8.439649in}{1.119814in}}{\pgfqpoint{8.435990in}{1.128646in}}{\pgfqpoint{8.429479in}{1.135158in}}%
\pgfpathcurveto{\pgfqpoint{8.422967in}{1.141669in}}{\pgfqpoint{8.414135in}{1.145328in}}{\pgfqpoint{8.404926in}{1.145328in}}%
\pgfpathcurveto{\pgfqpoint{8.395718in}{1.145328in}}{\pgfqpoint{8.386885in}{1.141669in}}{\pgfqpoint{8.380374in}{1.135158in}}%
\pgfpathcurveto{\pgfqpoint{8.373863in}{1.128646in}}{\pgfqpoint{8.370204in}{1.119814in}}{\pgfqpoint{8.370204in}{1.110605in}}%
\pgfpathcurveto{\pgfqpoint{8.370204in}{1.101397in}}{\pgfqpoint{8.373863in}{1.092565in}}{\pgfqpoint{8.380374in}{1.086053in}}%
\pgfpathcurveto{\pgfqpoint{8.386885in}{1.079542in}}{\pgfqpoint{8.395718in}{1.075883in}}{\pgfqpoint{8.404926in}{1.075883in}}%
\pgfpathlineto{\pgfqpoint{8.404926in}{1.075883in}}%
\pgfpathclose%
\pgfusepath{stroke,fill}%
\end{pgfscope}%
\begin{pgfscope}%
\pgfpathrectangle{\pgfqpoint{0.050000in}{0.050000in}}{\pgfqpoint{2.419000in}{2.419000in}}%
\pgfusepath{clip}%
\pgfsetbuttcap%
\pgfsetroundjoin%
\definecolor{currentfill}{rgb}{0.800000,0.400000,0.466667}%
\pgfsetfillcolor{currentfill}%
\pgfsetfillopacity{0.445956}%
\pgfsetlinewidth{1.003750pt}%
\definecolor{currentstroke}{rgb}{0.800000,0.400000,0.466667}%
\pgfsetstrokecolor{currentstroke}%
\pgfsetstrokeopacity{0.445956}%
\pgfsetdash{}{0pt}%
\pgfpathmoveto{\pgfqpoint{-1.351130in}{1.075883in}}%
\pgfpathcurveto{\pgfqpoint{-1.341922in}{1.075883in}}{\pgfqpoint{-1.333089in}{1.079542in}}{\pgfqpoint{-1.326578in}{1.086053in}}%
\pgfpathcurveto{\pgfqpoint{-1.320067in}{1.092565in}}{\pgfqpoint{-1.316408in}{1.101397in}}{\pgfqpoint{-1.316408in}{1.110605in}}%
\pgfpathcurveto{\pgfqpoint{-1.316408in}{1.119814in}}{\pgfqpoint{-1.320067in}{1.128646in}}{\pgfqpoint{-1.326578in}{1.135158in}}%
\pgfpathcurveto{\pgfqpoint{-1.333089in}{1.141669in}}{\pgfqpoint{-1.341922in}{1.145328in}}{\pgfqpoint{-1.351130in}{1.145328in}}%
\pgfpathcurveto{\pgfqpoint{-1.360339in}{1.145328in}}{\pgfqpoint{-1.369171in}{1.141669in}}{\pgfqpoint{-1.375683in}{1.135158in}}%
\pgfpathcurveto{\pgfqpoint{-1.382194in}{1.128646in}}{\pgfqpoint{-1.385853in}{1.119814in}}{\pgfqpoint{-1.385853in}{1.110605in}}%
\pgfpathcurveto{\pgfqpoint{-1.385853in}{1.101397in}}{\pgfqpoint{-1.382194in}{1.092565in}}{\pgfqpoint{-1.375683in}{1.086053in}}%
\pgfpathcurveto{\pgfqpoint{-1.369171in}{1.079542in}}{\pgfqpoint{-1.360339in}{1.075883in}}{\pgfqpoint{-1.351130in}{1.075883in}}%
\pgfpathlineto{\pgfqpoint{-1.351130in}{1.075883in}}%
\pgfpathclose%
\pgfusepath{stroke,fill}%
\end{pgfscope}%
\begin{pgfscope}%
\pgfpathrectangle{\pgfqpoint{0.050000in}{0.050000in}}{\pgfqpoint{2.419000in}{2.419000in}}%
\pgfusepath{clip}%
\pgfsetbuttcap%
\pgfsetroundjoin%
\definecolor{currentfill}{rgb}{0.800000,0.400000,0.466667}%
\pgfsetfillcolor{currentfill}%
\pgfsetfillopacity{0.445956}%
\pgfsetlinewidth{1.003750pt}%
\definecolor{currentstroke}{rgb}{0.800000,0.400000,0.466667}%
\pgfsetstrokecolor{currentstroke}%
\pgfsetstrokeopacity{0.445956}%
\pgfsetdash{}{0pt}%
\pgfpathmoveto{\pgfqpoint{3.526898in}{1.075883in}}%
\pgfpathcurveto{\pgfqpoint{3.536106in}{1.075883in}}{\pgfqpoint{3.544939in}{1.079542in}}{\pgfqpoint{3.551450in}{1.086053in}}%
\pgfpathcurveto{\pgfqpoint{3.557962in}{1.092565in}}{\pgfqpoint{3.561620in}{1.101397in}}{\pgfqpoint{3.561620in}{1.110605in}}%
\pgfpathcurveto{\pgfqpoint{3.561620in}{1.119814in}}{\pgfqpoint{3.557962in}{1.128646in}}{\pgfqpoint{3.551450in}{1.135158in}}%
\pgfpathcurveto{\pgfqpoint{3.544939in}{1.141669in}}{\pgfqpoint{3.536106in}{1.145328in}}{\pgfqpoint{3.526898in}{1.145328in}}%
\pgfpathcurveto{\pgfqpoint{3.517690in}{1.145328in}}{\pgfqpoint{3.508857in}{1.141669in}}{\pgfqpoint{3.502346in}{1.135158in}}%
\pgfpathcurveto{\pgfqpoint{3.495834in}{1.128646in}}{\pgfqpoint{3.492176in}{1.119814in}}{\pgfqpoint{3.492176in}{1.110605in}}%
\pgfpathcurveto{\pgfqpoint{3.492176in}{1.101397in}}{\pgfqpoint{3.495834in}{1.092565in}}{\pgfqpoint{3.502346in}{1.086053in}}%
\pgfpathcurveto{\pgfqpoint{3.508857in}{1.079542in}}{\pgfqpoint{3.517690in}{1.075883in}}{\pgfqpoint{3.526898in}{1.075883in}}%
\pgfpathlineto{\pgfqpoint{3.526898in}{1.075883in}}%
\pgfpathclose%
\pgfusepath{stroke,fill}%
\end{pgfscope}%
\begin{pgfscope}%
\pgfpathrectangle{\pgfqpoint{0.050000in}{0.050000in}}{\pgfqpoint{2.419000in}{2.419000in}}%
\pgfusepath{clip}%
\pgfsetbuttcap%
\pgfsetroundjoin%
\definecolor{currentfill}{rgb}{0.800000,0.400000,0.466667}%
\pgfsetfillcolor{currentfill}%
\pgfsetfillopacity{0.452680}%
\pgfsetlinewidth{1.003750pt}%
\definecolor{currentstroke}{rgb}{0.800000,0.400000,0.466667}%
\pgfsetstrokecolor{currentstroke}%
\pgfsetstrokeopacity{0.452680}%
\pgfsetdash{}{0pt}%
\pgfpathmoveto{\pgfqpoint{-3.415148in}{0.938286in}}%
\pgfpathcurveto{\pgfqpoint{-3.405939in}{0.938286in}}{\pgfqpoint{-3.397107in}{0.941945in}}{\pgfqpoint{-3.390595in}{0.948456in}}%
\pgfpathcurveto{\pgfqpoint{-3.384084in}{0.954967in}}{\pgfqpoint{-3.380426in}{0.963800in}}{\pgfqpoint{-3.380426in}{0.973008in}}%
\pgfpathcurveto{\pgfqpoint{-3.380426in}{0.982217in}}{\pgfqpoint{-3.384084in}{0.991049in}}{\pgfqpoint{-3.390595in}{0.997561in}}%
\pgfpathcurveto{\pgfqpoint{-3.397107in}{1.004072in}}{\pgfqpoint{-3.405939in}{1.007731in}}{\pgfqpoint{-3.415148in}{1.007731in}}%
\pgfpathcurveto{\pgfqpoint{-3.424356in}{1.007731in}}{\pgfqpoint{-3.433189in}{1.004072in}}{\pgfqpoint{-3.439700in}{0.997561in}}%
\pgfpathcurveto{\pgfqpoint{-3.446211in}{0.991049in}}{\pgfqpoint{-3.449870in}{0.982217in}}{\pgfqpoint{-3.449870in}{0.973008in}}%
\pgfpathcurveto{\pgfqpoint{-3.449870in}{0.963800in}}{\pgfqpoint{-3.446211in}{0.954967in}}{\pgfqpoint{-3.439700in}{0.948456in}}%
\pgfpathcurveto{\pgfqpoint{-3.433189in}{0.941945in}}{\pgfqpoint{-3.424356in}{0.938286in}}{\pgfqpoint{-3.415148in}{0.938286in}}%
\pgfpathlineto{\pgfqpoint{-3.415148in}{0.938286in}}%
\pgfpathclose%
\pgfusepath{stroke,fill}%
\end{pgfscope}%
\begin{pgfscope}%
\pgfpathrectangle{\pgfqpoint{0.050000in}{0.050000in}}{\pgfqpoint{2.419000in}{2.419000in}}%
\pgfusepath{clip}%
\pgfsetbuttcap%
\pgfsetroundjoin%
\definecolor{currentfill}{rgb}{0.800000,0.400000,0.466667}%
\pgfsetfillcolor{currentfill}%
\pgfsetfillopacity{0.452680}%
\pgfsetlinewidth{1.003750pt}%
\definecolor{currentstroke}{rgb}{0.800000,0.400000,0.466667}%
\pgfsetstrokecolor{currentstroke}%
\pgfsetstrokeopacity{0.452680}%
\pgfsetdash{}{0pt}%
\pgfpathmoveto{\pgfqpoint{1.549702in}{0.938286in}}%
\pgfpathcurveto{\pgfqpoint{1.558911in}{0.938286in}}{\pgfqpoint{1.567743in}{0.941945in}}{\pgfqpoint{1.574254in}{0.948456in}}%
\pgfpathcurveto{\pgfqpoint{1.580766in}{0.954967in}}{\pgfqpoint{1.584424in}{0.963800in}}{\pgfqpoint{1.584424in}{0.973008in}}%
\pgfpathcurveto{\pgfqpoint{1.584424in}{0.982217in}}{\pgfqpoint{1.580766in}{0.991049in}}{\pgfqpoint{1.574254in}{0.997561in}}%
\pgfpathcurveto{\pgfqpoint{1.567743in}{1.004072in}}{\pgfqpoint{1.558911in}{1.007731in}}{\pgfqpoint{1.549702in}{1.007731in}}%
\pgfpathcurveto{\pgfqpoint{1.540494in}{1.007731in}}{\pgfqpoint{1.531661in}{1.004072in}}{\pgfqpoint{1.525150in}{0.997561in}}%
\pgfpathcurveto{\pgfqpoint{1.518638in}{0.991049in}}{\pgfqpoint{1.514980in}{0.982217in}}{\pgfqpoint{1.514980in}{0.973008in}}%
\pgfpathcurveto{\pgfqpoint{1.514980in}{0.963800in}}{\pgfqpoint{1.518638in}{0.954967in}}{\pgfqpoint{1.525150in}{0.948456in}}%
\pgfpathcurveto{\pgfqpoint{1.531661in}{0.941945in}}{\pgfqpoint{1.540494in}{0.938286in}}{\pgfqpoint{1.549702in}{0.938286in}}%
\pgfpathlineto{\pgfqpoint{1.549702in}{0.938286in}}%
\pgfpathclose%
\pgfusepath{stroke,fill}%
\end{pgfscope}%
\begin{pgfscope}%
\pgfpathrectangle{\pgfqpoint{0.050000in}{0.050000in}}{\pgfqpoint{2.419000in}{2.419000in}}%
\pgfusepath{clip}%
\pgfsetbuttcap%
\pgfsetroundjoin%
\definecolor{currentfill}{rgb}{0.800000,0.400000,0.466667}%
\pgfsetfillcolor{currentfill}%
\pgfsetfillopacity{0.452680}%
\pgfsetlinewidth{1.003750pt}%
\definecolor{currentstroke}{rgb}{0.800000,0.400000,0.466667}%
\pgfsetstrokecolor{currentstroke}%
\pgfsetstrokeopacity{0.452680}%
\pgfsetdash{}{0pt}%
\pgfpathmoveto{\pgfqpoint{6.514552in}{0.938286in}}%
\pgfpathcurveto{\pgfqpoint{6.523761in}{0.938286in}}{\pgfqpoint{6.532593in}{0.941945in}}{\pgfqpoint{6.539104in}{0.948456in}}%
\pgfpathcurveto{\pgfqpoint{6.545616in}{0.954967in}}{\pgfqpoint{6.549274in}{0.963800in}}{\pgfqpoint{6.549274in}{0.973008in}}%
\pgfpathcurveto{\pgfqpoint{6.549274in}{0.982217in}}{\pgfqpoint{6.545616in}{0.991049in}}{\pgfqpoint{6.539104in}{0.997561in}}%
\pgfpathcurveto{\pgfqpoint{6.532593in}{1.004072in}}{\pgfqpoint{6.523761in}{1.007731in}}{\pgfqpoint{6.514552in}{1.007731in}}%
\pgfpathcurveto{\pgfqpoint{6.505344in}{1.007731in}}{\pgfqpoint{6.496511in}{1.004072in}}{\pgfqpoint{6.490000in}{0.997561in}}%
\pgfpathcurveto{\pgfqpoint{6.483488in}{0.991049in}}{\pgfqpoint{6.479830in}{0.982217in}}{\pgfqpoint{6.479830in}{0.973008in}}%
\pgfpathcurveto{\pgfqpoint{6.479830in}{0.963800in}}{\pgfqpoint{6.483488in}{0.954967in}}{\pgfqpoint{6.490000in}{0.948456in}}%
\pgfpathcurveto{\pgfqpoint{6.496511in}{0.941945in}}{\pgfqpoint{6.505344in}{0.938286in}}{\pgfqpoint{6.514552in}{0.938286in}}%
\pgfpathlineto{\pgfqpoint{6.514552in}{0.938286in}}%
\pgfpathclose%
\pgfusepath{stroke,fill}%
\end{pgfscope}%
\begin{pgfscope}%
\pgfpathrectangle{\pgfqpoint{0.050000in}{0.050000in}}{\pgfqpoint{2.419000in}{2.419000in}}%
\pgfusepath{clip}%
\pgfsetbuttcap%
\pgfsetroundjoin%
\definecolor{currentfill}{rgb}{0.800000,0.400000,0.466667}%
\pgfsetfillcolor{currentfill}%
\pgfsetfillopacity{0.459648}%
\pgfsetlinewidth{1.003750pt}%
\definecolor{currentstroke}{rgb}{0.800000,0.400000,0.466667}%
\pgfsetstrokecolor{currentstroke}%
\pgfsetstrokeopacity{0.459648}%
\pgfsetdash{}{0pt}%
\pgfpathmoveto{\pgfqpoint{9.610485in}{0.795702in}}%
\pgfpathcurveto{\pgfqpoint{9.619693in}{0.795702in}}{\pgfqpoint{9.628526in}{0.799361in}}{\pgfqpoint{9.635037in}{0.805872in}}%
\pgfpathcurveto{\pgfqpoint{9.641549in}{0.812383in}}{\pgfqpoint{9.645207in}{0.821216in}}{\pgfqpoint{9.645207in}{0.830424in}}%
\pgfpathcurveto{\pgfqpoint{9.645207in}{0.839633in}}{\pgfqpoint{9.641549in}{0.848465in}}{\pgfqpoint{9.635037in}{0.854977in}}%
\pgfpathcurveto{\pgfqpoint{9.628526in}{0.861488in}}{\pgfqpoint{9.619693in}{0.865147in}}{\pgfqpoint{9.610485in}{0.865147in}}%
\pgfpathcurveto{\pgfqpoint{9.601276in}{0.865147in}}{\pgfqpoint{9.592444in}{0.861488in}}{\pgfqpoint{9.585933in}{0.854977in}}%
\pgfpathcurveto{\pgfqpoint{9.579421in}{0.848465in}}{\pgfqpoint{9.575763in}{0.839633in}}{\pgfqpoint{9.575763in}{0.830424in}}%
\pgfpathcurveto{\pgfqpoint{9.575763in}{0.821216in}}{\pgfqpoint{9.579421in}{0.812383in}}{\pgfqpoint{9.585933in}{0.805872in}}%
\pgfpathcurveto{\pgfqpoint{9.592444in}{0.799361in}}{\pgfqpoint{9.601276in}{0.795702in}}{\pgfqpoint{9.610485in}{0.795702in}}%
\pgfpathlineto{\pgfqpoint{9.610485in}{0.795702in}}%
\pgfpathclose%
\pgfusepath{stroke,fill}%
\end{pgfscope}%
\begin{pgfscope}%
\pgfpathrectangle{\pgfqpoint{0.050000in}{0.050000in}}{\pgfqpoint{2.419000in}{2.419000in}}%
\pgfusepath{clip}%
\pgfsetbuttcap%
\pgfsetroundjoin%
\definecolor{currentfill}{rgb}{0.800000,0.400000,0.466667}%
\pgfsetfillcolor{currentfill}%
\pgfsetfillopacity{0.459648}%
\pgfsetlinewidth{1.003750pt}%
\definecolor{currentstroke}{rgb}{0.800000,0.400000,0.466667}%
\pgfsetstrokecolor{currentstroke}%
\pgfsetstrokeopacity{0.459648}%
\pgfsetdash{}{0pt}%
\pgfpathmoveto{\pgfqpoint{-0.499151in}{0.795702in}}%
\pgfpathcurveto{\pgfqpoint{-0.489943in}{0.795702in}}{\pgfqpoint{-0.481110in}{0.799361in}}{\pgfqpoint{-0.474599in}{0.805872in}}%
\pgfpathcurveto{\pgfqpoint{-0.468088in}{0.812383in}}{\pgfqpoint{-0.464429in}{0.821216in}}{\pgfqpoint{-0.464429in}{0.830424in}}%
\pgfpathcurveto{\pgfqpoint{-0.464429in}{0.839633in}}{\pgfqpoint{-0.468088in}{0.848465in}}{\pgfqpoint{-0.474599in}{0.854977in}}%
\pgfpathcurveto{\pgfqpoint{-0.481110in}{0.861488in}}{\pgfqpoint{-0.489943in}{0.865147in}}{\pgfqpoint{-0.499151in}{0.865147in}}%
\pgfpathcurveto{\pgfqpoint{-0.508360in}{0.865147in}}{\pgfqpoint{-0.517192in}{0.861488in}}{\pgfqpoint{-0.523704in}{0.854977in}}%
\pgfpathcurveto{\pgfqpoint{-0.530215in}{0.848465in}}{\pgfqpoint{-0.533874in}{0.839633in}}{\pgfqpoint{-0.533874in}{0.830424in}}%
\pgfpathcurveto{\pgfqpoint{-0.533874in}{0.821216in}}{\pgfqpoint{-0.530215in}{0.812383in}}{\pgfqpoint{-0.523704in}{0.805872in}}%
\pgfpathcurveto{\pgfqpoint{-0.517192in}{0.799361in}}{\pgfqpoint{-0.508360in}{0.795702in}}{\pgfqpoint{-0.499151in}{0.795702in}}%
\pgfpathlineto{\pgfqpoint{-0.499151in}{0.795702in}}%
\pgfpathclose%
\pgfusepath{stroke,fill}%
\end{pgfscope}%
\begin{pgfscope}%
\pgfpathrectangle{\pgfqpoint{0.050000in}{0.050000in}}{\pgfqpoint{2.419000in}{2.419000in}}%
\pgfusepath{clip}%
\pgfsetbuttcap%
\pgfsetroundjoin%
\definecolor{currentfill}{rgb}{0.800000,0.400000,0.466667}%
\pgfsetfillcolor{currentfill}%
\pgfsetfillopacity{0.459648}%
\pgfsetlinewidth{1.003750pt}%
\definecolor{currentstroke}{rgb}{0.800000,0.400000,0.466667}%
\pgfsetstrokecolor{currentstroke}%
\pgfsetstrokeopacity{0.459648}%
\pgfsetdash{}{0pt}%
\pgfpathmoveto{\pgfqpoint{4.555667in}{0.795702in}}%
\pgfpathcurveto{\pgfqpoint{4.564875in}{0.795702in}}{\pgfqpoint{4.573708in}{0.799361in}}{\pgfqpoint{4.580219in}{0.805872in}}%
\pgfpathcurveto{\pgfqpoint{4.586730in}{0.812383in}}{\pgfqpoint{4.590389in}{0.821216in}}{\pgfqpoint{4.590389in}{0.830424in}}%
\pgfpathcurveto{\pgfqpoint{4.590389in}{0.839633in}}{\pgfqpoint{4.586730in}{0.848465in}}{\pgfqpoint{4.580219in}{0.854977in}}%
\pgfpathcurveto{\pgfqpoint{4.573708in}{0.861488in}}{\pgfqpoint{4.564875in}{0.865147in}}{\pgfqpoint{4.555667in}{0.865147in}}%
\pgfpathcurveto{\pgfqpoint{4.546458in}{0.865147in}}{\pgfqpoint{4.537626in}{0.861488in}}{\pgfqpoint{4.531114in}{0.854977in}}%
\pgfpathcurveto{\pgfqpoint{4.524603in}{0.848465in}}{\pgfqpoint{4.520945in}{0.839633in}}{\pgfqpoint{4.520945in}{0.830424in}}%
\pgfpathcurveto{\pgfqpoint{4.520945in}{0.821216in}}{\pgfqpoint{4.524603in}{0.812383in}}{\pgfqpoint{4.531114in}{0.805872in}}%
\pgfpathcurveto{\pgfqpoint{4.537626in}{0.799361in}}{\pgfqpoint{4.546458in}{0.795702in}}{\pgfqpoint{4.555667in}{0.795702in}}%
\pgfpathlineto{\pgfqpoint{4.555667in}{0.795702in}}%
\pgfpathclose%
\pgfusepath{stroke,fill}%
\end{pgfscope}%
\begin{pgfscope}%
\pgfpathrectangle{\pgfqpoint{0.050000in}{0.050000in}}{\pgfqpoint{2.419000in}{2.419000in}}%
\pgfusepath{clip}%
\pgfsetbuttcap%
\pgfsetroundjoin%
\definecolor{currentfill}{rgb}{0.800000,0.400000,0.466667}%
\pgfsetfillcolor{currentfill}%
\pgfsetfillopacity{0.466874}%
\pgfsetlinewidth{1.003750pt}%
\definecolor{currentstroke}{rgb}{0.800000,0.400000,0.466667}%
\pgfsetstrokecolor{currentstroke}%
\pgfsetstrokeopacity{0.466874}%
\pgfsetdash{}{0pt}%
\pgfpathmoveto{\pgfqpoint{-2.623630in}{0.647855in}}%
\pgfpathcurveto{\pgfqpoint{-2.614421in}{0.647855in}}{\pgfqpoint{-2.605589in}{0.651514in}}{\pgfqpoint{-2.599077in}{0.658025in}}%
\pgfpathcurveto{\pgfqpoint{-2.592566in}{0.664537in}}{\pgfqpoint{-2.588908in}{0.673369in}}{\pgfqpoint{-2.588908in}{0.682578in}}%
\pgfpathcurveto{\pgfqpoint{-2.588908in}{0.691786in}}{\pgfqpoint{-2.592566in}{0.700619in}}{\pgfqpoint{-2.599077in}{0.707130in}}%
\pgfpathcurveto{\pgfqpoint{-2.605589in}{0.713641in}}{\pgfqpoint{-2.614421in}{0.717300in}}{\pgfqpoint{-2.623630in}{0.717300in}}%
\pgfpathcurveto{\pgfqpoint{-2.632838in}{0.717300in}}{\pgfqpoint{-2.641671in}{0.713641in}}{\pgfqpoint{-2.648182in}{0.707130in}}%
\pgfpathcurveto{\pgfqpoint{-2.654693in}{0.700619in}}{\pgfqpoint{-2.658352in}{0.691786in}}{\pgfqpoint{-2.658352in}{0.682578in}}%
\pgfpathcurveto{\pgfqpoint{-2.658352in}{0.673369in}}{\pgfqpoint{-2.654693in}{0.664537in}}{\pgfqpoint{-2.648182in}{0.658025in}}%
\pgfpathcurveto{\pgfqpoint{-2.641671in}{0.651514in}}{\pgfqpoint{-2.632838in}{0.647855in}}{\pgfqpoint{-2.623630in}{0.647855in}}%
\pgfpathlineto{\pgfqpoint{-2.623630in}{0.647855in}}%
\pgfpathclose%
\pgfusepath{stroke,fill}%
\end{pgfscope}%
\begin{pgfscope}%
\pgfpathrectangle{\pgfqpoint{0.050000in}{0.050000in}}{\pgfqpoint{2.419000in}{2.419000in}}%
\pgfusepath{clip}%
\pgfsetbuttcap%
\pgfsetroundjoin%
\definecolor{currentfill}{rgb}{0.800000,0.400000,0.466667}%
\pgfsetfillcolor{currentfill}%
\pgfsetfillopacity{0.466874}%
\pgfsetlinewidth{1.003750pt}%
\definecolor{currentstroke}{rgb}{0.800000,0.400000,0.466667}%
\pgfsetstrokecolor{currentstroke}%
\pgfsetstrokeopacity{0.466874}%
\pgfsetdash{}{0pt}%
\pgfpathmoveto{\pgfqpoint{2.524477in}{0.647855in}}%
\pgfpathcurveto{\pgfqpoint{2.533686in}{0.647855in}}{\pgfqpoint{2.542518in}{0.651514in}}{\pgfqpoint{2.549030in}{0.658025in}}%
\pgfpathcurveto{\pgfqpoint{2.555541in}{0.664537in}}{\pgfqpoint{2.559199in}{0.673369in}}{\pgfqpoint{2.559199in}{0.682578in}}%
\pgfpathcurveto{\pgfqpoint{2.559199in}{0.691786in}}{\pgfqpoint{2.555541in}{0.700619in}}{\pgfqpoint{2.549030in}{0.707130in}}%
\pgfpathcurveto{\pgfqpoint{2.542518in}{0.713641in}}{\pgfqpoint{2.533686in}{0.717300in}}{\pgfqpoint{2.524477in}{0.717300in}}%
\pgfpathcurveto{\pgfqpoint{2.515269in}{0.717300in}}{\pgfqpoint{2.506436in}{0.713641in}}{\pgfqpoint{2.499925in}{0.707130in}}%
\pgfpathcurveto{\pgfqpoint{2.493414in}{0.700619in}}{\pgfqpoint{2.489755in}{0.691786in}}{\pgfqpoint{2.489755in}{0.682578in}}%
\pgfpathcurveto{\pgfqpoint{2.489755in}{0.673369in}}{\pgfqpoint{2.493414in}{0.664537in}}{\pgfqpoint{2.499925in}{0.658025in}}%
\pgfpathcurveto{\pgfqpoint{2.506436in}{0.651514in}}{\pgfqpoint{2.515269in}{0.647855in}}{\pgfqpoint{2.524477in}{0.647855in}}%
\pgfpathlineto{\pgfqpoint{2.524477in}{0.647855in}}%
\pgfpathclose%
\pgfusepath{stroke,fill}%
\end{pgfscope}%
\begin{pgfscope}%
\pgfpathrectangle{\pgfqpoint{0.050000in}{0.050000in}}{\pgfqpoint{2.419000in}{2.419000in}}%
\pgfusepath{clip}%
\pgfsetbuttcap%
\pgfsetroundjoin%
\definecolor{currentfill}{rgb}{0.800000,0.400000,0.466667}%
\pgfsetfillcolor{currentfill}%
\pgfsetfillopacity{0.466874}%
\pgfsetlinewidth{1.003750pt}%
\definecolor{currentstroke}{rgb}{0.800000,0.400000,0.466667}%
\pgfsetstrokecolor{currentstroke}%
\pgfsetstrokeopacity{0.466874}%
\pgfsetdash{}{0pt}%
\pgfpathmoveto{\pgfqpoint{7.672584in}{0.647855in}}%
\pgfpathcurveto{\pgfqpoint{7.681793in}{0.647855in}}{\pgfqpoint{7.690625in}{0.651514in}}{\pgfqpoint{7.697137in}{0.658025in}}%
\pgfpathcurveto{\pgfqpoint{7.703648in}{0.664537in}}{\pgfqpoint{7.707307in}{0.673369in}}{\pgfqpoint{7.707307in}{0.682578in}}%
\pgfpathcurveto{\pgfqpoint{7.707307in}{0.691786in}}{\pgfqpoint{7.703648in}{0.700619in}}{\pgfqpoint{7.697137in}{0.707130in}}%
\pgfpathcurveto{\pgfqpoint{7.690625in}{0.713641in}}{\pgfqpoint{7.681793in}{0.717300in}}{\pgfqpoint{7.672584in}{0.717300in}}%
\pgfpathcurveto{\pgfqpoint{7.663376in}{0.717300in}}{\pgfqpoint{7.654543in}{0.713641in}}{\pgfqpoint{7.648032in}{0.707130in}}%
\pgfpathcurveto{\pgfqpoint{7.641521in}{0.700619in}}{\pgfqpoint{7.637862in}{0.691786in}}{\pgfqpoint{7.637862in}{0.682578in}}%
\pgfpathcurveto{\pgfqpoint{7.637862in}{0.673369in}}{\pgfqpoint{7.641521in}{0.664537in}}{\pgfqpoint{7.648032in}{0.658025in}}%
\pgfpathcurveto{\pgfqpoint{7.654543in}{0.651514in}}{\pgfqpoint{7.663376in}{0.647855in}}{\pgfqpoint{7.672584in}{0.647855in}}%
\pgfpathlineto{\pgfqpoint{7.672584in}{0.647855in}}%
\pgfpathclose%
\pgfusepath{stroke,fill}%
\end{pgfscope}%
\begin{pgfscope}%
\pgfpathrectangle{\pgfqpoint{0.050000in}{0.050000in}}{\pgfqpoint{2.419000in}{2.419000in}}%
\pgfusepath{clip}%
\pgfsetbuttcap%
\pgfsetroundjoin%
\definecolor{currentfill}{rgb}{0.800000,0.400000,0.466667}%
\pgfsetfillcolor{currentfill}%
\pgfsetfillopacity{0.474370}%
\pgfsetlinewidth{1.003750pt}%
\definecolor{currentstroke}{rgb}{0.800000,0.400000,0.466667}%
\pgfsetstrokecolor{currentstroke}%
\pgfsetstrokeopacity{0.474370}%
\pgfsetdash{}{0pt}%
\pgfpathmoveto{\pgfqpoint{0.416905in}{0.494449in}}%
\pgfpathcurveto{\pgfqpoint{0.426113in}{0.494449in}}{\pgfqpoint{0.434946in}{0.498107in}}{\pgfqpoint{0.441457in}{0.504619in}}%
\pgfpathcurveto{\pgfqpoint{0.447969in}{0.511130in}}{\pgfqpoint{0.451627in}{0.519963in}}{\pgfqpoint{0.451627in}{0.529171in}}%
\pgfpathcurveto{\pgfqpoint{0.451627in}{0.538380in}}{\pgfqpoint{0.447969in}{0.547212in}}{\pgfqpoint{0.441457in}{0.553723in}}%
\pgfpathcurveto{\pgfqpoint{0.434946in}{0.560235in}}{\pgfqpoint{0.426113in}{0.563893in}}{\pgfqpoint{0.416905in}{0.563893in}}%
\pgfpathcurveto{\pgfqpoint{0.407697in}{0.563893in}}{\pgfqpoint{0.398864in}{0.560235in}}{\pgfqpoint{0.392353in}{0.553723in}}%
\pgfpathcurveto{\pgfqpoint{0.385841in}{0.547212in}}{\pgfqpoint{0.382183in}{0.538380in}}{\pgfqpoint{0.382183in}{0.529171in}}%
\pgfpathcurveto{\pgfqpoint{0.382183in}{0.519963in}}{\pgfqpoint{0.385841in}{0.511130in}}{\pgfqpoint{0.392353in}{0.504619in}}%
\pgfpathcurveto{\pgfqpoint{0.398864in}{0.498107in}}{\pgfqpoint{0.407697in}{0.494449in}}{\pgfqpoint{0.416905in}{0.494449in}}%
\pgfpathlineto{\pgfqpoint{0.416905in}{0.494449in}}%
\pgfpathclose%
\pgfusepath{stroke,fill}%
\end{pgfscope}%
\begin{pgfscope}%
\pgfpathrectangle{\pgfqpoint{0.050000in}{0.050000in}}{\pgfqpoint{2.419000in}{2.419000in}}%
\pgfusepath{clip}%
\pgfsetbuttcap%
\pgfsetroundjoin%
\definecolor{currentfill}{rgb}{0.800000,0.400000,0.466667}%
\pgfsetfillcolor{currentfill}%
\pgfsetfillopacity{0.474370}%
\pgfsetlinewidth{1.003750pt}%
\definecolor{currentstroke}{rgb}{0.800000,0.400000,0.466667}%
\pgfsetstrokecolor{currentstroke}%
\pgfsetstrokeopacity{0.474370}%
\pgfsetdash{}{0pt}%
\pgfpathmoveto{\pgfqpoint{-4.827999in}{0.494449in}}%
\pgfpathcurveto{\pgfqpoint{-4.818791in}{0.494449in}}{\pgfqpoint{-4.809958in}{0.498107in}}{\pgfqpoint{-4.803447in}{0.504619in}}%
\pgfpathcurveto{\pgfqpoint{-4.796935in}{0.511130in}}{\pgfqpoint{-4.793277in}{0.519963in}}{\pgfqpoint{-4.793277in}{0.529171in}}%
\pgfpathcurveto{\pgfqpoint{-4.793277in}{0.538380in}}{\pgfqpoint{-4.796935in}{0.547212in}}{\pgfqpoint{-4.803447in}{0.553723in}}%
\pgfpathcurveto{\pgfqpoint{-4.809958in}{0.560235in}}{\pgfqpoint{-4.818791in}{0.563893in}}{\pgfqpoint{-4.827999in}{0.563893in}}%
\pgfpathcurveto{\pgfqpoint{-4.837208in}{0.563893in}}{\pgfqpoint{-4.846040in}{0.560235in}}{\pgfqpoint{-4.852551in}{0.553723in}}%
\pgfpathcurveto{\pgfqpoint{-4.859063in}{0.547212in}}{\pgfqpoint{-4.862721in}{0.538380in}}{\pgfqpoint{-4.862721in}{0.529171in}}%
\pgfpathcurveto{\pgfqpoint{-4.862721in}{0.519963in}}{\pgfqpoint{-4.859063in}{0.511130in}}{\pgfqpoint{-4.852551in}{0.504619in}}%
\pgfpathcurveto{\pgfqpoint{-4.846040in}{0.498107in}}{\pgfqpoint{-4.837208in}{0.494449in}}{\pgfqpoint{-4.827999in}{0.494449in}}%
\pgfpathlineto{\pgfqpoint{-4.827999in}{0.494449in}}%
\pgfpathclose%
\pgfusepath{stroke,fill}%
\end{pgfscope}%
\begin{pgfscope}%
\pgfpathrectangle{\pgfqpoint{0.050000in}{0.050000in}}{\pgfqpoint{2.419000in}{2.419000in}}%
\pgfusepath{clip}%
\pgfsetbuttcap%
\pgfsetroundjoin%
\definecolor{currentfill}{rgb}{0.800000,0.400000,0.466667}%
\pgfsetfillcolor{currentfill}%
\pgfsetfillopacity{0.474370}%
\pgfsetlinewidth{1.003750pt}%
\definecolor{currentstroke}{rgb}{0.800000,0.400000,0.466667}%
\pgfsetstrokecolor{currentstroke}%
\pgfsetstrokeopacity{0.474370}%
\pgfsetdash{}{0pt}%
\pgfpathmoveto{\pgfqpoint{5.661809in}{0.494449in}}%
\pgfpathcurveto{\pgfqpoint{5.671018in}{0.494449in}}{\pgfqpoint{5.679850in}{0.498107in}}{\pgfqpoint{5.686361in}{0.504619in}}%
\pgfpathcurveto{\pgfqpoint{5.692873in}{0.511130in}}{\pgfqpoint{5.696531in}{0.519963in}}{\pgfqpoint{5.696531in}{0.529171in}}%
\pgfpathcurveto{\pgfqpoint{5.696531in}{0.538380in}}{\pgfqpoint{5.692873in}{0.547212in}}{\pgfqpoint{5.686361in}{0.553723in}}%
\pgfpathcurveto{\pgfqpoint{5.679850in}{0.560235in}}{\pgfqpoint{5.671018in}{0.563893in}}{\pgfqpoint{5.661809in}{0.563893in}}%
\pgfpathcurveto{\pgfqpoint{5.652601in}{0.563893in}}{\pgfqpoint{5.643768in}{0.560235in}}{\pgfqpoint{5.637257in}{0.553723in}}%
\pgfpathcurveto{\pgfqpoint{5.630745in}{0.547212in}}{\pgfqpoint{5.627087in}{0.538380in}}{\pgfqpoint{5.627087in}{0.529171in}}%
\pgfpathcurveto{\pgfqpoint{5.627087in}{0.519963in}}{\pgfqpoint{5.630745in}{0.511130in}}{\pgfqpoint{5.637257in}{0.504619in}}%
\pgfpathcurveto{\pgfqpoint{5.643768in}{0.498107in}}{\pgfqpoint{5.652601in}{0.494449in}}{\pgfqpoint{5.661809in}{0.494449in}}%
\pgfpathlineto{\pgfqpoint{5.661809in}{0.494449in}}%
\pgfpathclose%
\pgfusepath{stroke,fill}%
\end{pgfscope}%
\begin{pgfscope}%
\pgfpathrectangle{\pgfqpoint{0.050000in}{0.050000in}}{\pgfqpoint{2.419000in}{2.419000in}}%
\pgfusepath{clip}%
\pgfsetbuttcap%
\pgfsetroundjoin%
\definecolor{currentfill}{rgb}{0.800000,0.400000,0.466667}%
\pgfsetfillcolor{currentfill}%
\pgfsetfillopacity{0.482155}%
\pgfsetlinewidth{1.003750pt}%
\definecolor{currentstroke}{rgb}{0.800000,0.400000,0.466667}%
\pgfsetstrokecolor{currentstroke}%
\pgfsetstrokeopacity{0.482155}%
\pgfsetdash{}{0pt}%
\pgfpathmoveto{\pgfqpoint{-1.771441in}{0.335163in}}%
\pgfpathcurveto{\pgfqpoint{-1.762233in}{0.335163in}}{\pgfqpoint{-1.753400in}{0.338821in}}{\pgfqpoint{-1.746889in}{0.345333in}}%
\pgfpathcurveto{\pgfqpoint{-1.740377in}{0.351844in}}{\pgfqpoint{-1.736719in}{0.360677in}}{\pgfqpoint{-1.736719in}{0.369885in}}%
\pgfpathcurveto{\pgfqpoint{-1.736719in}{0.379094in}}{\pgfqpoint{-1.740377in}{0.387926in}}{\pgfqpoint{-1.746889in}{0.394437in}}%
\pgfpathcurveto{\pgfqpoint{-1.753400in}{0.400949in}}{\pgfqpoint{-1.762233in}{0.404607in}}{\pgfqpoint{-1.771441in}{0.404607in}}%
\pgfpathcurveto{\pgfqpoint{-1.780650in}{0.404607in}}{\pgfqpoint{-1.789482in}{0.400949in}}{\pgfqpoint{-1.795993in}{0.394437in}}%
\pgfpathcurveto{\pgfqpoint{-1.802505in}{0.387926in}}{\pgfqpoint{-1.806163in}{0.379094in}}{\pgfqpoint{-1.806163in}{0.369885in}}%
\pgfpathcurveto{\pgfqpoint{-1.806163in}{0.360677in}}{\pgfqpoint{-1.802505in}{0.351844in}}{\pgfqpoint{-1.795993in}{0.345333in}}%
\pgfpathcurveto{\pgfqpoint{-1.789482in}{0.338821in}}{\pgfqpoint{-1.780650in}{0.335163in}}{\pgfqpoint{-1.771441in}{0.335163in}}%
\pgfpathlineto{\pgfqpoint{-1.771441in}{0.335163in}}%
\pgfpathclose%
\pgfusepath{stroke,fill}%
\end{pgfscope}%
\begin{pgfscope}%
\pgfpathrectangle{\pgfqpoint{0.050000in}{0.050000in}}{\pgfqpoint{2.419000in}{2.419000in}}%
\pgfusepath{clip}%
\pgfsetbuttcap%
\pgfsetroundjoin%
\definecolor{currentfill}{rgb}{0.800000,0.400000,0.466667}%
\pgfsetfillcolor{currentfill}%
\pgfsetfillopacity{0.482155}%
\pgfsetlinewidth{1.003750pt}%
\definecolor{currentstroke}{rgb}{0.800000,0.400000,0.466667}%
\pgfsetstrokecolor{currentstroke}%
\pgfsetstrokeopacity{0.482155}%
\pgfsetdash{}{0pt}%
\pgfpathmoveto{\pgfqpoint{8.919381in}{0.335163in}}%
\pgfpathcurveto{\pgfqpoint{8.928589in}{0.335163in}}{\pgfqpoint{8.937422in}{0.338821in}}{\pgfqpoint{8.943933in}{0.345333in}}%
\pgfpathcurveto{\pgfqpoint{8.950445in}{0.351844in}}{\pgfqpoint{8.954103in}{0.360677in}}{\pgfqpoint{8.954103in}{0.369885in}}%
\pgfpathcurveto{\pgfqpoint{8.954103in}{0.379094in}}{\pgfqpoint{8.950445in}{0.387926in}}{\pgfqpoint{8.943933in}{0.394437in}}%
\pgfpathcurveto{\pgfqpoint{8.937422in}{0.400949in}}{\pgfqpoint{8.928589in}{0.404607in}}{\pgfqpoint{8.919381in}{0.404607in}}%
\pgfpathcurveto{\pgfqpoint{8.910172in}{0.404607in}}{\pgfqpoint{8.901340in}{0.400949in}}{\pgfqpoint{8.894829in}{0.394437in}}%
\pgfpathcurveto{\pgfqpoint{8.888317in}{0.387926in}}{\pgfqpoint{8.884659in}{0.379094in}}{\pgfqpoint{8.884659in}{0.369885in}}%
\pgfpathcurveto{\pgfqpoint{8.884659in}{0.360677in}}{\pgfqpoint{8.888317in}{0.351844in}}{\pgfqpoint{8.894829in}{0.345333in}}%
\pgfpathcurveto{\pgfqpoint{8.901340in}{0.338821in}}{\pgfqpoint{8.910172in}{0.335163in}}{\pgfqpoint{8.919381in}{0.335163in}}%
\pgfpathlineto{\pgfqpoint{8.919381in}{0.335163in}}%
\pgfpathclose%
\pgfusepath{stroke,fill}%
\end{pgfscope}%
\begin{pgfscope}%
\pgfpathrectangle{\pgfqpoint{0.050000in}{0.050000in}}{\pgfqpoint{2.419000in}{2.419000in}}%
\pgfusepath{clip}%
\pgfsetbuttcap%
\pgfsetroundjoin%
\definecolor{currentfill}{rgb}{0.800000,0.400000,0.466667}%
\pgfsetfillcolor{currentfill}%
\pgfsetfillopacity{0.482155}%
\pgfsetlinewidth{1.003750pt}%
\definecolor{currentstroke}{rgb}{0.800000,0.400000,0.466667}%
\pgfsetstrokecolor{currentstroke}%
\pgfsetstrokeopacity{0.482155}%
\pgfsetdash{}{0pt}%
\pgfpathmoveto{\pgfqpoint{3.573970in}{0.335163in}}%
\pgfpathcurveto{\pgfqpoint{3.583178in}{0.335163in}}{\pgfqpoint{3.592011in}{0.338821in}}{\pgfqpoint{3.598522in}{0.345333in}}%
\pgfpathcurveto{\pgfqpoint{3.605034in}{0.351844in}}{\pgfqpoint{3.608692in}{0.360677in}}{\pgfqpoint{3.608692in}{0.369885in}}%
\pgfpathcurveto{\pgfqpoint{3.608692in}{0.379094in}}{\pgfqpoint{3.605034in}{0.387926in}}{\pgfqpoint{3.598522in}{0.394437in}}%
\pgfpathcurveto{\pgfqpoint{3.592011in}{0.400949in}}{\pgfqpoint{3.583178in}{0.404607in}}{\pgfqpoint{3.573970in}{0.404607in}}%
\pgfpathcurveto{\pgfqpoint{3.564761in}{0.404607in}}{\pgfqpoint{3.555929in}{0.400949in}}{\pgfqpoint{3.549418in}{0.394437in}}%
\pgfpathcurveto{\pgfqpoint{3.542906in}{0.387926in}}{\pgfqpoint{3.539248in}{0.379094in}}{\pgfqpoint{3.539248in}{0.369885in}}%
\pgfpathcurveto{\pgfqpoint{3.539248in}{0.360677in}}{\pgfqpoint{3.542906in}{0.351844in}}{\pgfqpoint{3.549418in}{0.345333in}}%
\pgfpathcurveto{\pgfqpoint{3.555929in}{0.338821in}}{\pgfqpoint{3.564761in}{0.335163in}}{\pgfqpoint{3.573970in}{0.335163in}}%
\pgfpathlineto{\pgfqpoint{3.573970in}{0.335163in}}%
\pgfpathclose%
\pgfusepath{stroke,fill}%
\end{pgfscope}%
\begin{pgfscope}%
\pgfpathrectangle{\pgfqpoint{0.050000in}{0.050000in}}{\pgfqpoint{2.419000in}{2.419000in}}%
\pgfusepath{clip}%
\pgfsetbuttcap%
\pgfsetroundjoin%
\definecolor{currentfill}{rgb}{0.800000,0.400000,0.466667}%
\pgfsetfillcolor{currentfill}%
\pgfsetfillopacity{0.490243}%
\pgfsetlinewidth{1.003750pt}%
\definecolor{currentstroke}{rgb}{0.800000,0.400000,0.466667}%
\pgfsetstrokecolor{currentstroke}%
\pgfsetstrokeopacity{0.490243}%
\pgfsetdash{}{0pt}%
\pgfpathmoveto{\pgfqpoint{-4.045295in}{0.169653in}}%
\pgfpathcurveto{\pgfqpoint{-4.036087in}{0.169653in}}{\pgfqpoint{-4.027254in}{0.173311in}}{\pgfqpoint{-4.020743in}{0.179823in}}%
\pgfpathcurveto{\pgfqpoint{-4.014232in}{0.186334in}}{\pgfqpoint{-4.010573in}{0.195167in}}{\pgfqpoint{-4.010573in}{0.204375in}}%
\pgfpathcurveto{\pgfqpoint{-4.010573in}{0.213584in}}{\pgfqpoint{-4.014232in}{0.222416in}}{\pgfqpoint{-4.020743in}{0.228927in}}%
\pgfpathcurveto{\pgfqpoint{-4.027254in}{0.235439in}}{\pgfqpoint{-4.036087in}{0.239097in}}{\pgfqpoint{-4.045295in}{0.239097in}}%
\pgfpathcurveto{\pgfqpoint{-4.054504in}{0.239097in}}{\pgfqpoint{-4.063336in}{0.235439in}}{\pgfqpoint{-4.069848in}{0.228927in}}%
\pgfpathcurveto{\pgfqpoint{-4.076359in}{0.222416in}}{\pgfqpoint{-4.080018in}{0.213584in}}{\pgfqpoint{-4.080018in}{0.204375in}}%
\pgfpathcurveto{\pgfqpoint{-4.080018in}{0.195167in}}{\pgfqpoint{-4.076359in}{0.186334in}}{\pgfqpoint{-4.069848in}{0.179823in}}%
\pgfpathcurveto{\pgfqpoint{-4.063336in}{0.173311in}}{\pgfqpoint{-4.054504in}{0.169653in}}{\pgfqpoint{-4.045295in}{0.169653in}}%
\pgfpathlineto{\pgfqpoint{-4.045295in}{0.169653in}}%
\pgfpathclose%
\pgfusepath{stroke,fill}%
\end{pgfscope}%
\begin{pgfscope}%
\pgfpathrectangle{\pgfqpoint{0.050000in}{0.050000in}}{\pgfqpoint{2.419000in}{2.419000in}}%
\pgfusepath{clip}%
\pgfsetbuttcap%
\pgfsetroundjoin%
\definecolor{currentfill}{rgb}{0.800000,0.400000,0.466667}%
\pgfsetfillcolor{currentfill}%
\pgfsetfillopacity{0.490243}%
\pgfsetlinewidth{1.003750pt}%
\definecolor{currentstroke}{rgb}{0.800000,0.400000,0.466667}%
\pgfsetstrokecolor{currentstroke}%
\pgfsetstrokeopacity{0.490243}%
\pgfsetdash{}{0pt}%
\pgfpathmoveto{\pgfqpoint{1.404550in}{0.169653in}}%
\pgfpathcurveto{\pgfqpoint{1.413758in}{0.169653in}}{\pgfqpoint{1.422591in}{0.173311in}}{\pgfqpoint{1.429102in}{0.179823in}}%
\pgfpathcurveto{\pgfqpoint{1.435613in}{0.186334in}}{\pgfqpoint{1.439272in}{0.195167in}}{\pgfqpoint{1.439272in}{0.204375in}}%
\pgfpathcurveto{\pgfqpoint{1.439272in}{0.213584in}}{\pgfqpoint{1.435613in}{0.222416in}}{\pgfqpoint{1.429102in}{0.228927in}}%
\pgfpathcurveto{\pgfqpoint{1.422591in}{0.235439in}}{\pgfqpoint{1.413758in}{0.239097in}}{\pgfqpoint{1.404550in}{0.239097in}}%
\pgfpathcurveto{\pgfqpoint{1.395341in}{0.239097in}}{\pgfqpoint{1.386509in}{0.235439in}}{\pgfqpoint{1.379998in}{0.228927in}}%
\pgfpathcurveto{\pgfqpoint{1.373486in}{0.222416in}}{\pgfqpoint{1.369828in}{0.213584in}}{\pgfqpoint{1.369828in}{0.204375in}}%
\pgfpathcurveto{\pgfqpoint{1.369828in}{0.195167in}}{\pgfqpoint{1.373486in}{0.186334in}}{\pgfqpoint{1.379998in}{0.179823in}}%
\pgfpathcurveto{\pgfqpoint{1.386509in}{0.173311in}}{\pgfqpoint{1.395341in}{0.169653in}}{\pgfqpoint{1.404550in}{0.169653in}}%
\pgfpathlineto{\pgfqpoint{1.404550in}{0.169653in}}%
\pgfpathclose%
\pgfusepath{stroke,fill}%
\end{pgfscope}%
\begin{pgfscope}%
\pgfpathrectangle{\pgfqpoint{0.050000in}{0.050000in}}{\pgfqpoint{2.419000in}{2.419000in}}%
\pgfusepath{clip}%
\pgfsetbuttcap%
\pgfsetroundjoin%
\definecolor{currentfill}{rgb}{0.800000,0.400000,0.466667}%
\pgfsetfillcolor{currentfill}%
\pgfsetfillopacity{0.490243}%
\pgfsetlinewidth{1.003750pt}%
\definecolor{currentstroke}{rgb}{0.800000,0.400000,0.466667}%
\pgfsetstrokecolor{currentstroke}%
\pgfsetstrokeopacity{0.490243}%
\pgfsetdash{}{0pt}%
\pgfpathmoveto{\pgfqpoint{6.854395in}{0.169653in}}%
\pgfpathcurveto{\pgfqpoint{6.863603in}{0.169653in}}{\pgfqpoint{6.872436in}{0.173311in}}{\pgfqpoint{6.878947in}{0.179823in}}%
\pgfpathcurveto{\pgfqpoint{6.885459in}{0.186334in}}{\pgfqpoint{6.889117in}{0.195167in}}{\pgfqpoint{6.889117in}{0.204375in}}%
\pgfpathcurveto{\pgfqpoint{6.889117in}{0.213584in}}{\pgfqpoint{6.885459in}{0.222416in}}{\pgfqpoint{6.878947in}{0.228927in}}%
\pgfpathcurveto{\pgfqpoint{6.872436in}{0.235439in}}{\pgfqpoint{6.863603in}{0.239097in}}{\pgfqpoint{6.854395in}{0.239097in}}%
\pgfpathcurveto{\pgfqpoint{6.845187in}{0.239097in}}{\pgfqpoint{6.836354in}{0.235439in}}{\pgfqpoint{6.829843in}{0.228927in}}%
\pgfpathcurveto{\pgfqpoint{6.823331in}{0.222416in}}{\pgfqpoint{6.819673in}{0.213584in}}{\pgfqpoint{6.819673in}{0.204375in}}%
\pgfpathcurveto{\pgfqpoint{6.819673in}{0.195167in}}{\pgfqpoint{6.823331in}{0.186334in}}{\pgfqpoint{6.829843in}{0.179823in}}%
\pgfpathcurveto{\pgfqpoint{6.836354in}{0.173311in}}{\pgfqpoint{6.845187in}{0.169653in}}{\pgfqpoint{6.854395in}{0.169653in}}%
\pgfpathlineto{\pgfqpoint{6.854395in}{0.169653in}}%
\pgfpathclose%
\pgfusepath{stroke,fill}%
\end{pgfscope}%
\begin{pgfscope}%
\pgfpathrectangle{\pgfqpoint{0.050000in}{0.050000in}}{\pgfqpoint{2.419000in}{2.419000in}}%
\pgfusepath{clip}%
\pgfsetbuttcap%
\pgfsetroundjoin%
\definecolor{currentfill}{rgb}{0.800000,0.400000,0.466667}%
\pgfsetfillcolor{currentfill}%
\pgfsetfillopacity{0.498654}%
\pgfsetlinewidth{1.003750pt}%
\definecolor{currentstroke}{rgb}{0.800000,0.400000,0.466667}%
\pgfsetstrokecolor{currentstroke}%
\pgfsetstrokeopacity{0.498654}%
\pgfsetdash{}{0pt}%
\pgfpathmoveto{\pgfqpoint{-0.851328in}{-0.002453in}}%
\pgfpathcurveto{\pgfqpoint{-0.842120in}{-0.002453in}}{\pgfqpoint{-0.833287in}{0.001205in}}{\pgfqpoint{-0.826776in}{0.007717in}}%
\pgfpathcurveto{\pgfqpoint{-0.820264in}{0.014228in}}{\pgfqpoint{-0.816606in}{0.023061in}}{\pgfqpoint{-0.816606in}{0.032269in}}%
\pgfpathcurveto{\pgfqpoint{-0.816606in}{0.041478in}}{\pgfqpoint{-0.820264in}{0.050310in}}{\pgfqpoint{-0.826776in}{0.056821in}}%
\pgfpathcurveto{\pgfqpoint{-0.833287in}{0.063333in}}{\pgfqpoint{-0.842120in}{0.066991in}}{\pgfqpoint{-0.851328in}{0.066991in}}%
\pgfpathcurveto{\pgfqpoint{-0.860536in}{0.066991in}}{\pgfqpoint{-0.869369in}{0.063333in}}{\pgfqpoint{-0.875880in}{0.056821in}}%
\pgfpathcurveto{\pgfqpoint{-0.882392in}{0.050310in}}{\pgfqpoint{-0.886050in}{0.041478in}}{\pgfqpoint{-0.886050in}{0.032269in}}%
\pgfpathcurveto{\pgfqpoint{-0.886050in}{0.023061in}}{\pgfqpoint{-0.882392in}{0.014228in}}{\pgfqpoint{-0.875880in}{0.007717in}}%
\pgfpathcurveto{\pgfqpoint{-0.869369in}{0.001205in}}{\pgfqpoint{-0.860536in}{-0.002453in}}{\pgfqpoint{-0.851328in}{-0.002453in}}%
\pgfpathlineto{\pgfqpoint{-0.851328in}{-0.002453in}}%
\pgfpathclose%
\pgfusepath{stroke,fill}%
\end{pgfscope}%
\begin{pgfscope}%
\pgfpathrectangle{\pgfqpoint{0.050000in}{0.050000in}}{\pgfqpoint{2.419000in}{2.419000in}}%
\pgfusepath{clip}%
\pgfsetbuttcap%
\pgfsetroundjoin%
\definecolor{currentfill}{rgb}{0.800000,0.400000,0.466667}%
\pgfsetfillcolor{currentfill}%
\pgfsetfillopacity{0.498654}%
\pgfsetlinewidth{1.003750pt}%
\definecolor{currentstroke}{rgb}{0.800000,0.400000,0.466667}%
\pgfsetstrokecolor{currentstroke}%
\pgfsetstrokeopacity{0.498654}%
\pgfsetdash{}{0pt}%
\pgfpathmoveto{\pgfqpoint{4.707113in}{-0.002453in}}%
\pgfpathcurveto{\pgfqpoint{4.716322in}{-0.002453in}}{\pgfqpoint{4.725154in}{0.001205in}}{\pgfqpoint{4.731666in}{0.007717in}}%
\pgfpathcurveto{\pgfqpoint{4.738177in}{0.014228in}}{\pgfqpoint{4.741835in}{0.023061in}}{\pgfqpoint{4.741835in}{0.032269in}}%
\pgfpathcurveto{\pgfqpoint{4.741835in}{0.041478in}}{\pgfqpoint{4.738177in}{0.050310in}}{\pgfqpoint{4.731666in}{0.056821in}}%
\pgfpathcurveto{\pgfqpoint{4.725154in}{0.063333in}}{\pgfqpoint{4.716322in}{0.066991in}}{\pgfqpoint{4.707113in}{0.066991in}}%
\pgfpathcurveto{\pgfqpoint{4.697905in}{0.066991in}}{\pgfqpoint{4.689072in}{0.063333in}}{\pgfqpoint{4.682561in}{0.056821in}}%
\pgfpathcurveto{\pgfqpoint{4.676050in}{0.050310in}}{\pgfqpoint{4.672391in}{0.041478in}}{\pgfqpoint{4.672391in}{0.032269in}}%
\pgfpathcurveto{\pgfqpoint{4.672391in}{0.023061in}}{\pgfqpoint{4.676050in}{0.014228in}}{\pgfqpoint{4.682561in}{0.007717in}}%
\pgfpathcurveto{\pgfqpoint{4.689072in}{0.001205in}}{\pgfqpoint{4.697905in}{-0.002453in}}{\pgfqpoint{4.707113in}{-0.002453in}}%
\pgfpathlineto{\pgfqpoint{4.707113in}{-0.002453in}}%
\pgfpathclose%
\pgfusepath{stroke,fill}%
\end{pgfscope}%
\begin{pgfscope}%
\pgfpathrectangle{\pgfqpoint{0.050000in}{0.050000in}}{\pgfqpoint{2.419000in}{2.419000in}}%
\pgfusepath{clip}%
\pgfsetbuttcap%
\pgfsetroundjoin%
\definecolor{currentfill}{rgb}{0.800000,0.400000,0.466667}%
\pgfsetfillcolor{currentfill}%
\pgfsetfillopacity{0.498654}%
\pgfsetlinewidth{1.003750pt}%
\definecolor{currentstroke}{rgb}{0.800000,0.400000,0.466667}%
\pgfsetstrokecolor{currentstroke}%
\pgfsetstrokeopacity{0.498654}%
\pgfsetdash{}{0pt}%
\pgfpathmoveto{\pgfqpoint{10.265555in}{-0.002453in}}%
\pgfpathcurveto{\pgfqpoint{10.274763in}{-0.002453in}}{\pgfqpoint{10.283596in}{0.001205in}}{\pgfqpoint{10.290107in}{0.007717in}}%
\pgfpathcurveto{\pgfqpoint{10.296618in}{0.014228in}}{\pgfqpoint{10.300277in}{0.023061in}}{\pgfqpoint{10.300277in}{0.032269in}}%
\pgfpathcurveto{\pgfqpoint{10.300277in}{0.041478in}}{\pgfqpoint{10.296618in}{0.050310in}}{\pgfqpoint{10.290107in}{0.056821in}}%
\pgfpathcurveto{\pgfqpoint{10.283596in}{0.063333in}}{\pgfqpoint{10.274763in}{0.066991in}}{\pgfqpoint{10.265555in}{0.066991in}}%
\pgfpathcurveto{\pgfqpoint{10.256346in}{0.066991in}}{\pgfqpoint{10.247514in}{0.063333in}}{\pgfqpoint{10.241002in}{0.056821in}}%
\pgfpathcurveto{\pgfqpoint{10.234491in}{0.050310in}}{\pgfqpoint{10.230832in}{0.041478in}}{\pgfqpoint{10.230832in}{0.032269in}}%
\pgfpathcurveto{\pgfqpoint{10.230832in}{0.023061in}}{\pgfqpoint{10.234491in}{0.014228in}}{\pgfqpoint{10.241002in}{0.007717in}}%
\pgfpathcurveto{\pgfqpoint{10.247514in}{0.001205in}}{\pgfqpoint{10.256346in}{-0.002453in}}{\pgfqpoint{10.265555in}{-0.002453in}}%
\pgfpathlineto{\pgfqpoint{10.265555in}{-0.002453in}}%
\pgfpathclose%
\pgfusepath{stroke,fill}%
\end{pgfscope}%
\begin{pgfscope}%
\pgfpathrectangle{\pgfqpoint{0.050000in}{0.050000in}}{\pgfqpoint{2.419000in}{2.419000in}}%
\pgfusepath{clip}%
\pgfsetbuttcap%
\pgfsetroundjoin%
\definecolor{currentfill}{rgb}{0.800000,0.400000,0.466667}%
\pgfsetfillcolor{currentfill}%
\pgfsetfillopacity{0.507407}%
\pgfsetlinewidth{1.003750pt}%
\definecolor{currentstroke}{rgb}{0.800000,0.400000,0.466667}%
\pgfsetstrokecolor{currentstroke}%
\pgfsetstrokeopacity{0.507407}%
\pgfsetdash{}{0pt}%
\pgfpathmoveto{\pgfqpoint{-3.198937in}{-0.181557in}}%
\pgfpathcurveto{\pgfqpoint{-3.189729in}{-0.181557in}}{\pgfqpoint{-3.180896in}{-0.177899in}}{\pgfqpoint{-3.174385in}{-0.171388in}}%
\pgfpathcurveto{\pgfqpoint{-3.167873in}{-0.164876in}}{\pgfqpoint{-3.164215in}{-0.156044in}}{\pgfqpoint{-3.164215in}{-0.146835in}}%
\pgfpathcurveto{\pgfqpoint{-3.164215in}{-0.137627in}}{\pgfqpoint{-3.167873in}{-0.128794in}}{\pgfqpoint{-3.174385in}{-0.122283in}}%
\pgfpathcurveto{\pgfqpoint{-3.180896in}{-0.115772in}}{\pgfqpoint{-3.189729in}{-0.112113in}}{\pgfqpoint{-3.198937in}{-0.112113in}}%
\pgfpathcurveto{\pgfqpoint{-3.208145in}{-0.112113in}}{\pgfqpoint{-3.216978in}{-0.115772in}}{\pgfqpoint{-3.223489in}{-0.122283in}}%
\pgfpathcurveto{\pgfqpoint{-3.230001in}{-0.128794in}}{\pgfqpoint{-3.233659in}{-0.137627in}}{\pgfqpoint{-3.233659in}{-0.146835in}}%
\pgfpathcurveto{\pgfqpoint{-3.233659in}{-0.156044in}}{\pgfqpoint{-3.230001in}{-0.164876in}}{\pgfqpoint{-3.223489in}{-0.171388in}}%
\pgfpathcurveto{\pgfqpoint{-3.216978in}{-0.177899in}}{\pgfqpoint{-3.208145in}{-0.181557in}}{\pgfqpoint{-3.198937in}{-0.181557in}}%
\pgfpathlineto{\pgfqpoint{-3.198937in}{-0.181557in}}%
\pgfpathclose%
\pgfusepath{stroke,fill}%
\end{pgfscope}%
\begin{pgfscope}%
\pgfpathrectangle{\pgfqpoint{0.050000in}{0.050000in}}{\pgfqpoint{2.419000in}{2.419000in}}%
\pgfusepath{clip}%
\pgfsetbuttcap%
\pgfsetroundjoin%
\definecolor{currentfill}{rgb}{0.800000,0.400000,0.466667}%
\pgfsetfillcolor{currentfill}%
\pgfsetfillopacity{0.507407}%
\pgfsetlinewidth{1.003750pt}%
\definecolor{currentstroke}{rgb}{0.800000,0.400000,0.466667}%
\pgfsetstrokecolor{currentstroke}%
\pgfsetstrokeopacity{0.507407}%
\pgfsetdash{}{0pt}%
\pgfpathmoveto{\pgfqpoint{2.472516in}{-0.181557in}}%
\pgfpathcurveto{\pgfqpoint{2.481725in}{-0.181557in}}{\pgfqpoint{2.490557in}{-0.177899in}}{\pgfqpoint{2.497069in}{-0.171388in}}%
\pgfpathcurveto{\pgfqpoint{2.503580in}{-0.164876in}}{\pgfqpoint{2.507238in}{-0.156044in}}{\pgfqpoint{2.507238in}{-0.146835in}}%
\pgfpathcurveto{\pgfqpoint{2.507238in}{-0.137627in}}{\pgfqpoint{2.503580in}{-0.128794in}}{\pgfqpoint{2.497069in}{-0.122283in}}%
\pgfpathcurveto{\pgfqpoint{2.490557in}{-0.115772in}}{\pgfqpoint{2.481725in}{-0.112113in}}{\pgfqpoint{2.472516in}{-0.112113in}}%
\pgfpathcurveto{\pgfqpoint{2.463308in}{-0.112113in}}{\pgfqpoint{2.454475in}{-0.115772in}}{\pgfqpoint{2.447964in}{-0.122283in}}%
\pgfpathcurveto{\pgfqpoint{2.441453in}{-0.128794in}}{\pgfqpoint{2.437794in}{-0.137627in}}{\pgfqpoint{2.437794in}{-0.146835in}}%
\pgfpathcurveto{\pgfqpoint{2.437794in}{-0.156044in}}{\pgfqpoint{2.441453in}{-0.164876in}}{\pgfqpoint{2.447964in}{-0.171388in}}%
\pgfpathcurveto{\pgfqpoint{2.454475in}{-0.177899in}}{\pgfqpoint{2.463308in}{-0.181557in}}{\pgfqpoint{2.472516in}{-0.181557in}}%
\pgfpathlineto{\pgfqpoint{2.472516in}{-0.181557in}}%
\pgfpathclose%
\pgfusepath{stroke,fill}%
\end{pgfscope}%
\begin{pgfscope}%
\pgfpathrectangle{\pgfqpoint{0.050000in}{0.050000in}}{\pgfqpoint{2.419000in}{2.419000in}}%
\pgfusepath{clip}%
\pgfsetbuttcap%
\pgfsetroundjoin%
\definecolor{currentfill}{rgb}{0.800000,0.400000,0.466667}%
\pgfsetfillcolor{currentfill}%
\pgfsetfillopacity{0.507407}%
\pgfsetlinewidth{1.003750pt}%
\definecolor{currentstroke}{rgb}{0.800000,0.400000,0.466667}%
\pgfsetstrokecolor{currentstroke}%
\pgfsetstrokeopacity{0.507407}%
\pgfsetdash{}{0pt}%
\pgfpathmoveto{\pgfqpoint{8.143970in}{-0.181557in}}%
\pgfpathcurveto{\pgfqpoint{8.153178in}{-0.181557in}}{\pgfqpoint{8.162010in}{-0.177899in}}{\pgfqpoint{8.168522in}{-0.171388in}}%
\pgfpathcurveto{\pgfqpoint{8.175033in}{-0.164876in}}{\pgfqpoint{8.178692in}{-0.156044in}}{\pgfqpoint{8.178692in}{-0.146835in}}%
\pgfpathcurveto{\pgfqpoint{8.178692in}{-0.137627in}}{\pgfqpoint{8.175033in}{-0.128794in}}{\pgfqpoint{8.168522in}{-0.122283in}}%
\pgfpathcurveto{\pgfqpoint{8.162010in}{-0.115772in}}{\pgfqpoint{8.153178in}{-0.112113in}}{\pgfqpoint{8.143970in}{-0.112113in}}%
\pgfpathcurveto{\pgfqpoint{8.134761in}{-0.112113in}}{\pgfqpoint{8.125929in}{-0.115772in}}{\pgfqpoint{8.119417in}{-0.122283in}}%
\pgfpathcurveto{\pgfqpoint{8.112906in}{-0.128794in}}{\pgfqpoint{8.109247in}{-0.137627in}}{\pgfqpoint{8.109247in}{-0.146835in}}%
\pgfpathcurveto{\pgfqpoint{8.109247in}{-0.156044in}}{\pgfqpoint{8.112906in}{-0.164876in}}{\pgfqpoint{8.119417in}{-0.171388in}}%
\pgfpathcurveto{\pgfqpoint{8.125929in}{-0.177899in}}{\pgfqpoint{8.134761in}{-0.181557in}}{\pgfqpoint{8.143970in}{-0.181557in}}%
\pgfpathlineto{\pgfqpoint{8.143970in}{-0.181557in}}%
\pgfpathclose%
\pgfusepath{stroke,fill}%
\end{pgfscope}%
\begin{pgfscope}%
\pgfpathrectangle{\pgfqpoint{0.050000in}{0.050000in}}{\pgfqpoint{2.419000in}{2.419000in}}%
\pgfusepath{clip}%
\pgfsetbuttcap%
\pgfsetroundjoin%
\definecolor{currentfill}{rgb}{0.800000,0.400000,0.466667}%
\pgfsetfillcolor{currentfill}%
\pgfsetfillopacity{0.516523}%
\pgfsetlinewidth{1.003750pt}%
\definecolor{currentstroke}{rgb}{0.800000,0.400000,0.466667}%
\pgfsetstrokecolor{currentstroke}%
\pgfsetstrokeopacity{0.516523}%
\pgfsetdash{}{0pt}%
\pgfpathmoveto{\pgfqpoint{0.145168in}{-0.368096in}}%
\pgfpathcurveto{\pgfqpoint{0.154376in}{-0.368096in}}{\pgfqpoint{0.163209in}{-0.364437in}}{\pgfqpoint{0.169720in}{-0.357926in}}%
\pgfpathcurveto{\pgfqpoint{0.176231in}{-0.351415in}}{\pgfqpoint{0.179890in}{-0.342582in}}{\pgfqpoint{0.179890in}{-0.333374in}}%
\pgfpathcurveto{\pgfqpoint{0.179890in}{-0.324165in}}{\pgfqpoint{0.176231in}{-0.315333in}}{\pgfqpoint{0.169720in}{-0.308821in}}%
\pgfpathcurveto{\pgfqpoint{0.163209in}{-0.302310in}}{\pgfqpoint{0.154376in}{-0.298652in}}{\pgfqpoint{0.145168in}{-0.298652in}}%
\pgfpathcurveto{\pgfqpoint{0.135959in}{-0.298652in}}{\pgfqpoint{0.127127in}{-0.302310in}}{\pgfqpoint{0.120615in}{-0.308821in}}%
\pgfpathcurveto{\pgfqpoint{0.114104in}{-0.315333in}}{\pgfqpoint{0.110445in}{-0.324165in}}{\pgfqpoint{0.110445in}{-0.333374in}}%
\pgfpathcurveto{\pgfqpoint{0.110445in}{-0.342582in}}{\pgfqpoint{0.114104in}{-0.351415in}}{\pgfqpoint{0.120615in}{-0.357926in}}%
\pgfpathcurveto{\pgfqpoint{0.127127in}{-0.364437in}}{\pgfqpoint{0.135959in}{-0.368096in}}{\pgfqpoint{0.145168in}{-0.368096in}}%
\pgfpathlineto{\pgfqpoint{0.145168in}{-0.368096in}}%
\pgfpathclose%
\pgfusepath{stroke,fill}%
\end{pgfscope}%
\begin{pgfscope}%
\pgfpathrectangle{\pgfqpoint{0.050000in}{0.050000in}}{\pgfqpoint{2.419000in}{2.419000in}}%
\pgfusepath{clip}%
\pgfsetbuttcap%
\pgfsetroundjoin%
\definecolor{currentfill}{rgb}{0.800000,0.400000,0.466667}%
\pgfsetfillcolor{currentfill}%
\pgfsetfillopacity{0.516523}%
\pgfsetlinewidth{1.003750pt}%
\definecolor{currentstroke}{rgb}{0.800000,0.400000,0.466667}%
\pgfsetstrokecolor{currentstroke}%
\pgfsetstrokeopacity{0.516523}%
\pgfsetdash{}{0pt}%
\pgfpathmoveto{\pgfqpoint{-5.643988in}{-0.368096in}}%
\pgfpathcurveto{\pgfqpoint{-5.634780in}{-0.368096in}}{\pgfqpoint{-5.625947in}{-0.364437in}}{\pgfqpoint{-5.619436in}{-0.357926in}}%
\pgfpathcurveto{\pgfqpoint{-5.612925in}{-0.351415in}}{\pgfqpoint{-5.609266in}{-0.342582in}}{\pgfqpoint{-5.609266in}{-0.333374in}}%
\pgfpathcurveto{\pgfqpoint{-5.609266in}{-0.324165in}}{\pgfqpoint{-5.612925in}{-0.315333in}}{\pgfqpoint{-5.619436in}{-0.308821in}}%
\pgfpathcurveto{\pgfqpoint{-5.625947in}{-0.302310in}}{\pgfqpoint{-5.634780in}{-0.298652in}}{\pgfqpoint{-5.643988in}{-0.298652in}}%
\pgfpathcurveto{\pgfqpoint{-5.653197in}{-0.298652in}}{\pgfqpoint{-5.662029in}{-0.302310in}}{\pgfqpoint{-5.668541in}{-0.308821in}}%
\pgfpathcurveto{\pgfqpoint{-5.675052in}{-0.315333in}}{\pgfqpoint{-5.678711in}{-0.324165in}}{\pgfqpoint{-5.678711in}{-0.333374in}}%
\pgfpathcurveto{\pgfqpoint{-5.678711in}{-0.342582in}}{\pgfqpoint{-5.675052in}{-0.351415in}}{\pgfqpoint{-5.668541in}{-0.357926in}}%
\pgfpathcurveto{\pgfqpoint{-5.662029in}{-0.364437in}}{\pgfqpoint{-5.653197in}{-0.368096in}}{\pgfqpoint{-5.643988in}{-0.368096in}}%
\pgfpathlineto{\pgfqpoint{-5.643988in}{-0.368096in}}%
\pgfpathclose%
\pgfusepath{stroke,fill}%
\end{pgfscope}%
\begin{pgfscope}%
\pgfpathrectangle{\pgfqpoint{0.050000in}{0.050000in}}{\pgfqpoint{2.419000in}{2.419000in}}%
\pgfusepath{clip}%
\pgfsetbuttcap%
\pgfsetroundjoin%
\definecolor{currentfill}{rgb}{0.800000,0.400000,0.466667}%
\pgfsetfillcolor{currentfill}%
\pgfsetfillopacity{0.516523}%
\pgfsetlinewidth{1.003750pt}%
\definecolor{currentstroke}{rgb}{0.800000,0.400000,0.466667}%
\pgfsetstrokecolor{currentstroke}%
\pgfsetstrokeopacity{0.516523}%
\pgfsetdash{}{0pt}%
\pgfpathmoveto{\pgfqpoint{5.934324in}{-0.368096in}}%
\pgfpathcurveto{\pgfqpoint{5.943532in}{-0.368096in}}{\pgfqpoint{5.952365in}{-0.364437in}}{\pgfqpoint{5.958876in}{-0.357926in}}%
\pgfpathcurveto{\pgfqpoint{5.965387in}{-0.351415in}}{\pgfqpoint{5.969046in}{-0.342582in}}{\pgfqpoint{5.969046in}{-0.333374in}}%
\pgfpathcurveto{\pgfqpoint{5.969046in}{-0.324165in}}{\pgfqpoint{5.965387in}{-0.315333in}}{\pgfqpoint{5.958876in}{-0.308821in}}%
\pgfpathcurveto{\pgfqpoint{5.952365in}{-0.302310in}}{\pgfqpoint{5.943532in}{-0.298652in}}{\pgfqpoint{5.934324in}{-0.298652in}}%
\pgfpathcurveto{\pgfqpoint{5.925115in}{-0.298652in}}{\pgfqpoint{5.916283in}{-0.302310in}}{\pgfqpoint{5.909771in}{-0.308821in}}%
\pgfpathcurveto{\pgfqpoint{5.903260in}{-0.315333in}}{\pgfqpoint{5.899601in}{-0.324165in}}{\pgfqpoint{5.899601in}{-0.333374in}}%
\pgfpathcurveto{\pgfqpoint{5.899601in}{-0.342582in}}{\pgfqpoint{5.903260in}{-0.351415in}}{\pgfqpoint{5.909771in}{-0.357926in}}%
\pgfpathcurveto{\pgfqpoint{5.916283in}{-0.364437in}}{\pgfqpoint{5.925115in}{-0.368096in}}{\pgfqpoint{5.934324in}{-0.368096in}}%
\pgfpathlineto{\pgfqpoint{5.934324in}{-0.368096in}}%
\pgfpathclose%
\pgfusepath{stroke,fill}%
\end{pgfscope}%
\begin{pgfscope}%
\pgfpathrectangle{\pgfqpoint{0.050000in}{0.050000in}}{\pgfqpoint{2.419000in}{2.419000in}}%
\pgfusepath{clip}%
\pgfsetbuttcap%
\pgfsetroundjoin%
\definecolor{currentfill}{rgb}{0.800000,0.400000,0.466667}%
\pgfsetfillcolor{currentfill}%
\pgfsetfillopacity{0.526025}%
\pgfsetlinewidth{1.003750pt}%
\definecolor{currentstroke}{rgb}{0.800000,0.400000,0.466667}%
\pgfsetstrokecolor{currentstroke}%
\pgfsetstrokeopacity{0.526025}%
\pgfsetdash{}{0pt}%
\pgfpathmoveto{\pgfqpoint{-2.280830in}{-0.562541in}}%
\pgfpathcurveto{\pgfqpoint{-2.271621in}{-0.562541in}}{\pgfqpoint{-2.262789in}{-0.558883in}}{\pgfqpoint{-2.256278in}{-0.552371in}}%
\pgfpathcurveto{\pgfqpoint{-2.249766in}{-0.545860in}}{\pgfqpoint{-2.246108in}{-0.537027in}}{\pgfqpoint{-2.246108in}{-0.527819in}}%
\pgfpathcurveto{\pgfqpoint{-2.246108in}{-0.518611in}}{\pgfqpoint{-2.249766in}{-0.509778in}}{\pgfqpoint{-2.256278in}{-0.503267in}}%
\pgfpathcurveto{\pgfqpoint{-2.262789in}{-0.496755in}}{\pgfqpoint{-2.271621in}{-0.493097in}}{\pgfqpoint{-2.280830in}{-0.493097in}}%
\pgfpathcurveto{\pgfqpoint{-2.290038in}{-0.493097in}}{\pgfqpoint{-2.298871in}{-0.496755in}}{\pgfqpoint{-2.305382in}{-0.503267in}}%
\pgfpathcurveto{\pgfqpoint{-2.311894in}{-0.509778in}}{\pgfqpoint{-2.315552in}{-0.518611in}}{\pgfqpoint{-2.315552in}{-0.527819in}}%
\pgfpathcurveto{\pgfqpoint{-2.315552in}{-0.537027in}}{\pgfqpoint{-2.311894in}{-0.545860in}}{\pgfqpoint{-2.305382in}{-0.552371in}}%
\pgfpathcurveto{\pgfqpoint{-2.298871in}{-0.558883in}}{\pgfqpoint{-2.290038in}{-0.562541in}}{\pgfqpoint{-2.280830in}{-0.562541in}}%
\pgfpathlineto{\pgfqpoint{-2.280830in}{-0.562541in}}%
\pgfpathclose%
\pgfusepath{stroke,fill}%
\end{pgfscope}%
\begin{pgfscope}%
\pgfpathrectangle{\pgfqpoint{0.050000in}{0.050000in}}{\pgfqpoint{2.419000in}{2.419000in}}%
\pgfusepath{clip}%
\pgfsetbuttcap%
\pgfsetroundjoin%
\definecolor{currentfill}{rgb}{0.800000,0.400000,0.466667}%
\pgfsetfillcolor{currentfill}%
\pgfsetfillopacity{0.526025}%
\pgfsetlinewidth{1.003750pt}%
\definecolor{currentstroke}{rgb}{0.800000,0.400000,0.466667}%
\pgfsetstrokecolor{currentstroke}%
\pgfsetstrokeopacity{0.526025}%
\pgfsetdash{}{0pt}%
\pgfpathmoveto{\pgfqpoint{3.631018in}{-0.562541in}}%
\pgfpathcurveto{\pgfqpoint{3.640226in}{-0.562541in}}{\pgfqpoint{3.649059in}{-0.558883in}}{\pgfqpoint{3.655570in}{-0.552371in}}%
\pgfpathcurveto{\pgfqpoint{3.662082in}{-0.545860in}}{\pgfqpoint{3.665740in}{-0.537027in}}{\pgfqpoint{3.665740in}{-0.527819in}}%
\pgfpathcurveto{\pgfqpoint{3.665740in}{-0.518611in}}{\pgfqpoint{3.662082in}{-0.509778in}}{\pgfqpoint{3.655570in}{-0.503267in}}%
\pgfpathcurveto{\pgfqpoint{3.649059in}{-0.496755in}}{\pgfqpoint{3.640226in}{-0.493097in}}{\pgfqpoint{3.631018in}{-0.493097in}}%
\pgfpathcurveto{\pgfqpoint{3.621810in}{-0.493097in}}{\pgfqpoint{3.612977in}{-0.496755in}}{\pgfqpoint{3.606466in}{-0.503267in}}%
\pgfpathcurveto{\pgfqpoint{3.599954in}{-0.509778in}}{\pgfqpoint{3.596296in}{-0.518611in}}{\pgfqpoint{3.596296in}{-0.527819in}}%
\pgfpathcurveto{\pgfqpoint{3.596296in}{-0.537027in}}{\pgfqpoint{3.599954in}{-0.545860in}}{\pgfqpoint{3.606466in}{-0.552371in}}%
\pgfpathcurveto{\pgfqpoint{3.612977in}{-0.558883in}}{\pgfqpoint{3.621810in}{-0.562541in}}{\pgfqpoint{3.631018in}{-0.562541in}}%
\pgfpathlineto{\pgfqpoint{3.631018in}{-0.562541in}}%
\pgfpathclose%
\pgfusepath{stroke,fill}%
\end{pgfscope}%
\begin{pgfscope}%
\pgfpathrectangle{\pgfqpoint{0.050000in}{0.050000in}}{\pgfqpoint{2.419000in}{2.419000in}}%
\pgfusepath{clip}%
\pgfsetbuttcap%
\pgfsetroundjoin%
\definecolor{currentfill}{rgb}{0.800000,0.400000,0.466667}%
\pgfsetfillcolor{currentfill}%
\pgfsetfillopacity{0.526025}%
\pgfsetlinewidth{1.003750pt}%
\definecolor{currentstroke}{rgb}{0.800000,0.400000,0.466667}%
\pgfsetstrokecolor{currentstroke}%
\pgfsetstrokeopacity{0.526025}%
\pgfsetdash{}{0pt}%
\pgfpathmoveto{\pgfqpoint{9.542866in}{-0.562541in}}%
\pgfpathcurveto{\pgfqpoint{9.552074in}{-0.562541in}}{\pgfqpoint{9.560907in}{-0.558883in}}{\pgfqpoint{9.567418in}{-0.552371in}}%
\pgfpathcurveto{\pgfqpoint{9.573930in}{-0.545860in}}{\pgfqpoint{9.577588in}{-0.537027in}}{\pgfqpoint{9.577588in}{-0.527819in}}%
\pgfpathcurveto{\pgfqpoint{9.577588in}{-0.518611in}}{\pgfqpoint{9.573930in}{-0.509778in}}{\pgfqpoint{9.567418in}{-0.503267in}}%
\pgfpathcurveto{\pgfqpoint{9.560907in}{-0.496755in}}{\pgfqpoint{9.552074in}{-0.493097in}}{\pgfqpoint{9.542866in}{-0.493097in}}%
\pgfpathcurveto{\pgfqpoint{9.533657in}{-0.493097in}}{\pgfqpoint{9.524825in}{-0.496755in}}{\pgfqpoint{9.518314in}{-0.503267in}}%
\pgfpathcurveto{\pgfqpoint{9.511802in}{-0.509778in}}{\pgfqpoint{9.508144in}{-0.518611in}}{\pgfqpoint{9.508144in}{-0.527819in}}%
\pgfpathcurveto{\pgfqpoint{9.508144in}{-0.537027in}}{\pgfqpoint{9.511802in}{-0.545860in}}{\pgfqpoint{9.518314in}{-0.552371in}}%
\pgfpathcurveto{\pgfqpoint{9.524825in}{-0.558883in}}{\pgfqpoint{9.533657in}{-0.562541in}}{\pgfqpoint{9.542866in}{-0.562541in}}%
\pgfpathlineto{\pgfqpoint{9.542866in}{-0.562541in}}%
\pgfpathclose%
\pgfusepath{stroke,fill}%
\end{pgfscope}%
\begin{pgfscope}%
\pgfpathrectangle{\pgfqpoint{0.050000in}{0.050000in}}{\pgfqpoint{2.419000in}{2.419000in}}%
\pgfusepath{clip}%
\pgfsetbuttcap%
\pgfsetroundjoin%
\definecolor{currentfill}{rgb}{0.800000,0.400000,0.466667}%
\pgfsetfillcolor{currentfill}%
\pgfsetfillopacity{0.535939}%
\pgfsetlinewidth{1.003750pt}%
\definecolor{currentstroke}{rgb}{0.800000,0.400000,0.466667}%
\pgfsetstrokecolor{currentstroke}%
\pgfsetstrokeopacity{0.535939}%
\pgfsetdash{}{0pt}%
\pgfpathmoveto{\pgfqpoint{1.227969in}{-0.765407in}}%
\pgfpathcurveto{\pgfqpoint{1.237177in}{-0.765407in}}{\pgfqpoint{1.246010in}{-0.761748in}}{\pgfqpoint{1.252521in}{-0.755237in}}%
\pgfpathcurveto{\pgfqpoint{1.259032in}{-0.748726in}}{\pgfqpoint{1.262691in}{-0.739893in}}{\pgfqpoint{1.262691in}{-0.730685in}}%
\pgfpathcurveto{\pgfqpoint{1.262691in}{-0.721476in}}{\pgfqpoint{1.259032in}{-0.712644in}}{\pgfqpoint{1.252521in}{-0.706132in}}%
\pgfpathcurveto{\pgfqpoint{1.246010in}{-0.699621in}}{\pgfqpoint{1.237177in}{-0.695962in}}{\pgfqpoint{1.227969in}{-0.695962in}}%
\pgfpathcurveto{\pgfqpoint{1.218760in}{-0.695962in}}{\pgfqpoint{1.209928in}{-0.699621in}}{\pgfqpoint{1.203416in}{-0.706132in}}%
\pgfpathcurveto{\pgfqpoint{1.196905in}{-0.712644in}}{\pgfqpoint{1.193247in}{-0.721476in}}{\pgfqpoint{1.193247in}{-0.730685in}}%
\pgfpathcurveto{\pgfqpoint{1.193247in}{-0.739893in}}{\pgfqpoint{1.196905in}{-0.748726in}}{\pgfqpoint{1.203416in}{-0.755237in}}%
\pgfpathcurveto{\pgfqpoint{1.209928in}{-0.761748in}}{\pgfqpoint{1.218760in}{-0.765407in}}{\pgfqpoint{1.227969in}{-0.765407in}}%
\pgfpathlineto{\pgfqpoint{1.227969in}{-0.765407in}}%
\pgfpathclose%
\pgfusepath{stroke,fill}%
\end{pgfscope}%
\begin{pgfscope}%
\pgfpathrectangle{\pgfqpoint{0.050000in}{0.050000in}}{\pgfqpoint{2.419000in}{2.419000in}}%
\pgfusepath{clip}%
\pgfsetbuttcap%
\pgfsetroundjoin%
\definecolor{currentfill}{rgb}{0.800000,0.400000,0.466667}%
\pgfsetfillcolor{currentfill}%
\pgfsetfillopacity{0.535939}%
\pgfsetlinewidth{1.003750pt}%
\definecolor{currentstroke}{rgb}{0.800000,0.400000,0.466667}%
\pgfsetstrokecolor{currentstroke}%
\pgfsetstrokeopacity{0.535939}%
\pgfsetdash{}{0pt}%
\pgfpathmoveto{\pgfqpoint{-4.811884in}{-0.765407in}}%
\pgfpathcurveto{\pgfqpoint{-4.802676in}{-0.765407in}}{\pgfqpoint{-4.793843in}{-0.761748in}}{\pgfqpoint{-4.787332in}{-0.755237in}}%
\pgfpathcurveto{\pgfqpoint{-4.780820in}{-0.748726in}}{\pgfqpoint{-4.777162in}{-0.739893in}}{\pgfqpoint{-4.777162in}{-0.730685in}}%
\pgfpathcurveto{\pgfqpoint{-4.777162in}{-0.721476in}}{\pgfqpoint{-4.780820in}{-0.712644in}}{\pgfqpoint{-4.787332in}{-0.706132in}}%
\pgfpathcurveto{\pgfqpoint{-4.793843in}{-0.699621in}}{\pgfqpoint{-4.802676in}{-0.695962in}}{\pgfqpoint{-4.811884in}{-0.695962in}}%
\pgfpathcurveto{\pgfqpoint{-4.821093in}{-0.695962in}}{\pgfqpoint{-4.829925in}{-0.699621in}}{\pgfqpoint{-4.836436in}{-0.706132in}}%
\pgfpathcurveto{\pgfqpoint{-4.842948in}{-0.712644in}}{\pgfqpoint{-4.846606in}{-0.721476in}}{\pgfqpoint{-4.846606in}{-0.730685in}}%
\pgfpathcurveto{\pgfqpoint{-4.846606in}{-0.739893in}}{\pgfqpoint{-4.842948in}{-0.748726in}}{\pgfqpoint{-4.836436in}{-0.755237in}}%
\pgfpathcurveto{\pgfqpoint{-4.829925in}{-0.761748in}}{\pgfqpoint{-4.821093in}{-0.765407in}}{\pgfqpoint{-4.811884in}{-0.765407in}}%
\pgfpathlineto{\pgfqpoint{-4.811884in}{-0.765407in}}%
\pgfpathclose%
\pgfusepath{stroke,fill}%
\end{pgfscope}%
\begin{pgfscope}%
\pgfpathrectangle{\pgfqpoint{0.050000in}{0.050000in}}{\pgfqpoint{2.419000in}{2.419000in}}%
\pgfusepath{clip}%
\pgfsetbuttcap%
\pgfsetroundjoin%
\definecolor{currentfill}{rgb}{0.800000,0.400000,0.466667}%
\pgfsetfillcolor{currentfill}%
\pgfsetfillopacity{0.535939}%
\pgfsetlinewidth{1.003750pt}%
\definecolor{currentstroke}{rgb}{0.800000,0.400000,0.466667}%
\pgfsetstrokecolor{currentstroke}%
\pgfsetstrokeopacity{0.535939}%
\pgfsetdash{}{0pt}%
\pgfpathmoveto{\pgfqpoint{7.267822in}{-0.765407in}}%
\pgfpathcurveto{\pgfqpoint{7.277030in}{-0.765407in}}{\pgfqpoint{7.285863in}{-0.761748in}}{\pgfqpoint{7.292374in}{-0.755237in}}%
\pgfpathcurveto{\pgfqpoint{7.298885in}{-0.748726in}}{\pgfqpoint{7.302544in}{-0.739893in}}{\pgfqpoint{7.302544in}{-0.730685in}}%
\pgfpathcurveto{\pgfqpoint{7.302544in}{-0.721476in}}{\pgfqpoint{7.298885in}{-0.712644in}}{\pgfqpoint{7.292374in}{-0.706132in}}%
\pgfpathcurveto{\pgfqpoint{7.285863in}{-0.699621in}}{\pgfqpoint{7.277030in}{-0.695962in}}{\pgfqpoint{7.267822in}{-0.695962in}}%
\pgfpathcurveto{\pgfqpoint{7.258613in}{-0.695962in}}{\pgfqpoint{7.249781in}{-0.699621in}}{\pgfqpoint{7.243269in}{-0.706132in}}%
\pgfpathcurveto{\pgfqpoint{7.236758in}{-0.712644in}}{\pgfqpoint{7.233099in}{-0.721476in}}{\pgfqpoint{7.233099in}{-0.730685in}}%
\pgfpathcurveto{\pgfqpoint{7.233099in}{-0.739893in}}{\pgfqpoint{7.236758in}{-0.748726in}}{\pgfqpoint{7.243269in}{-0.755237in}}%
\pgfpathcurveto{\pgfqpoint{7.249781in}{-0.761748in}}{\pgfqpoint{7.258613in}{-0.765407in}}{\pgfqpoint{7.267822in}{-0.765407in}}%
\pgfpathlineto{\pgfqpoint{7.267822in}{-0.765407in}}%
\pgfpathclose%
\pgfusepath{stroke,fill}%
\end{pgfscope}%
\begin{pgfscope}%
\pgfpathrectangle{\pgfqpoint{0.050000in}{0.050000in}}{\pgfqpoint{2.419000in}{2.419000in}}%
\pgfusepath{clip}%
\pgfsetbuttcap%
\pgfsetroundjoin%
\definecolor{currentfill}{rgb}{0.800000,0.400000,0.466667}%
\pgfsetfillcolor{currentfill}%
\pgfsetfillopacity{0.546292}%
\pgfsetlinewidth{1.003750pt}%
\definecolor{currentstroke}{rgb}{0.800000,0.400000,0.466667}%
\pgfsetstrokecolor{currentstroke}%
\pgfsetstrokeopacity{0.546292}%
\pgfsetdash{}{0pt}%
\pgfpathmoveto{\pgfqpoint{-1.281447in}{-0.977252in}}%
\pgfpathcurveto{\pgfqpoint{-1.272238in}{-0.977252in}}{\pgfqpoint{-1.263406in}{-0.973593in}}{\pgfqpoint{-1.256894in}{-0.967082in}}%
\pgfpathcurveto{\pgfqpoint{-1.250383in}{-0.960571in}}{\pgfqpoint{-1.246724in}{-0.951738in}}{\pgfqpoint{-1.246724in}{-0.942530in}}%
\pgfpathcurveto{\pgfqpoint{-1.246724in}{-0.933321in}}{\pgfqpoint{-1.250383in}{-0.924489in}}{\pgfqpoint{-1.256894in}{-0.917977in}}%
\pgfpathcurveto{\pgfqpoint{-1.263406in}{-0.911466in}}{\pgfqpoint{-1.272238in}{-0.907807in}}{\pgfqpoint{-1.281447in}{-0.907807in}}%
\pgfpathcurveto{\pgfqpoint{-1.290655in}{-0.907807in}}{\pgfqpoint{-1.299488in}{-0.911466in}}{\pgfqpoint{-1.305999in}{-0.917977in}}%
\pgfpathcurveto{\pgfqpoint{-1.312510in}{-0.924489in}}{\pgfqpoint{-1.316169in}{-0.933321in}}{\pgfqpoint{-1.316169in}{-0.942530in}}%
\pgfpathcurveto{\pgfqpoint{-1.316169in}{-0.951738in}}{\pgfqpoint{-1.312510in}{-0.960571in}}{\pgfqpoint{-1.305999in}{-0.967082in}}%
\pgfpathcurveto{\pgfqpoint{-1.299488in}{-0.973593in}}{\pgfqpoint{-1.290655in}{-0.977252in}}{\pgfqpoint{-1.281447in}{-0.977252in}}%
\pgfpathlineto{\pgfqpoint{-1.281447in}{-0.977252in}}%
\pgfpathclose%
\pgfusepath{stroke,fill}%
\end{pgfscope}%
\begin{pgfscope}%
\pgfpathrectangle{\pgfqpoint{0.050000in}{0.050000in}}{\pgfqpoint{2.419000in}{2.419000in}}%
\pgfusepath{clip}%
\pgfsetbuttcap%
\pgfsetroundjoin%
\definecolor{currentfill}{rgb}{0.800000,0.400000,0.466667}%
\pgfsetfillcolor{currentfill}%
\pgfsetfillopacity{0.546292}%
\pgfsetlinewidth{1.003750pt}%
\definecolor{currentstroke}{rgb}{0.800000,0.400000,0.466667}%
\pgfsetstrokecolor{currentstroke}%
\pgfsetstrokeopacity{0.546292}%
\pgfsetdash{}{0pt}%
\pgfpathmoveto{\pgfqpoint{-7.454970in}{-0.977252in}}%
\pgfpathcurveto{\pgfqpoint{-7.445762in}{-0.977252in}}{\pgfqpoint{-7.436929in}{-0.973593in}}{\pgfqpoint{-7.430418in}{-0.967082in}}%
\pgfpathcurveto{\pgfqpoint{-7.423907in}{-0.960571in}}{\pgfqpoint{-7.420248in}{-0.951738in}}{\pgfqpoint{-7.420248in}{-0.942530in}}%
\pgfpathcurveto{\pgfqpoint{-7.420248in}{-0.933321in}}{\pgfqpoint{-7.423907in}{-0.924489in}}{\pgfqpoint{-7.430418in}{-0.917977in}}%
\pgfpathcurveto{\pgfqpoint{-7.436929in}{-0.911466in}}{\pgfqpoint{-7.445762in}{-0.907807in}}{\pgfqpoint{-7.454970in}{-0.907807in}}%
\pgfpathcurveto{\pgfqpoint{-7.464179in}{-0.907807in}}{\pgfqpoint{-7.473011in}{-0.911466in}}{\pgfqpoint{-7.479522in}{-0.917977in}}%
\pgfpathcurveto{\pgfqpoint{-7.486034in}{-0.924489in}}{\pgfqpoint{-7.489692in}{-0.933321in}}{\pgfqpoint{-7.489692in}{-0.942530in}}%
\pgfpathcurveto{\pgfqpoint{-7.489692in}{-0.951738in}}{\pgfqpoint{-7.486034in}{-0.960571in}}{\pgfqpoint{-7.479522in}{-0.967082in}}%
\pgfpathcurveto{\pgfqpoint{-7.473011in}{-0.973593in}}{\pgfqpoint{-7.464179in}{-0.977252in}}{\pgfqpoint{-7.454970in}{-0.977252in}}%
\pgfpathlineto{\pgfqpoint{-7.454970in}{-0.977252in}}%
\pgfpathclose%
\pgfusepath{stroke,fill}%
\end{pgfscope}%
\begin{pgfscope}%
\pgfpathrectangle{\pgfqpoint{0.050000in}{0.050000in}}{\pgfqpoint{2.419000in}{2.419000in}}%
\pgfusepath{clip}%
\pgfsetbuttcap%
\pgfsetroundjoin%
\definecolor{currentfill}{rgb}{0.800000,0.400000,0.466667}%
\pgfsetfillcolor{currentfill}%
\pgfsetfillopacity{0.546292}%
\pgfsetlinewidth{1.003750pt}%
\definecolor{currentstroke}{rgb}{0.800000,0.400000,0.466667}%
\pgfsetstrokecolor{currentstroke}%
\pgfsetstrokeopacity{0.546292}%
\pgfsetdash{}{0pt}%
\pgfpathmoveto{\pgfqpoint{4.892077in}{-0.977252in}}%
\pgfpathcurveto{\pgfqpoint{4.901286in}{-0.977252in}}{\pgfqpoint{4.910118in}{-0.973593in}}{\pgfqpoint{4.916629in}{-0.967082in}}%
\pgfpathcurveto{\pgfqpoint{4.923141in}{-0.960571in}}{\pgfqpoint{4.926799in}{-0.951738in}}{\pgfqpoint{4.926799in}{-0.942530in}}%
\pgfpathcurveto{\pgfqpoint{4.926799in}{-0.933321in}}{\pgfqpoint{4.923141in}{-0.924489in}}{\pgfqpoint{4.916629in}{-0.917977in}}%
\pgfpathcurveto{\pgfqpoint{4.910118in}{-0.911466in}}{\pgfqpoint{4.901286in}{-0.907807in}}{\pgfqpoint{4.892077in}{-0.907807in}}%
\pgfpathcurveto{\pgfqpoint{4.882869in}{-0.907807in}}{\pgfqpoint{4.874036in}{-0.911466in}}{\pgfqpoint{4.867525in}{-0.917977in}}%
\pgfpathcurveto{\pgfqpoint{4.861013in}{-0.924489in}}{\pgfqpoint{4.857355in}{-0.933321in}}{\pgfqpoint{4.857355in}{-0.942530in}}%
\pgfpathcurveto{\pgfqpoint{4.857355in}{-0.951738in}}{\pgfqpoint{4.861013in}{-0.960571in}}{\pgfqpoint{4.867525in}{-0.967082in}}%
\pgfpathcurveto{\pgfqpoint{4.874036in}{-0.973593in}}{\pgfqpoint{4.882869in}{-0.977252in}}{\pgfqpoint{4.892077in}{-0.977252in}}%
\pgfpathlineto{\pgfqpoint{4.892077in}{-0.977252in}}%
\pgfpathclose%
\pgfusepath{stroke,fill}%
\end{pgfscope}%
\begin{pgfscope}%
\pgfpathrectangle{\pgfqpoint{0.050000in}{0.050000in}}{\pgfqpoint{2.419000in}{2.419000in}}%
\pgfusepath{clip}%
\pgfsetbuttcap%
\pgfsetroundjoin%
\definecolor{currentfill}{rgb}{0.800000,0.400000,0.466667}%
\pgfsetfillcolor{currentfill}%
\pgfsetfillopacity{0.557113}%
\pgfsetlinewidth{1.003750pt}%
\definecolor{currentstroke}{rgb}{0.800000,0.400000,0.466667}%
\pgfsetstrokecolor{currentstroke}%
\pgfsetstrokeopacity{0.557113}%
\pgfsetdash{}{0pt}%
\pgfpathmoveto{\pgfqpoint{-3.904450in}{-1.198686in}}%
\pgfpathcurveto{\pgfqpoint{-3.895241in}{-1.198686in}}{\pgfqpoint{-3.886409in}{-1.195028in}}{\pgfqpoint{-3.879897in}{-1.188516in}}%
\pgfpathcurveto{\pgfqpoint{-3.873386in}{-1.182005in}}{\pgfqpoint{-3.869728in}{-1.173172in}}{\pgfqpoint{-3.869728in}{-1.163964in}}%
\pgfpathcurveto{\pgfqpoint{-3.869728in}{-1.154755in}}{\pgfqpoint{-3.873386in}{-1.145923in}}{\pgfqpoint{-3.879897in}{-1.139412in}}%
\pgfpathcurveto{\pgfqpoint{-3.886409in}{-1.132900in}}{\pgfqpoint{-3.895241in}{-1.129242in}}{\pgfqpoint{-3.904450in}{-1.129242in}}%
\pgfpathcurveto{\pgfqpoint{-3.913658in}{-1.129242in}}{\pgfqpoint{-3.922491in}{-1.132900in}}{\pgfqpoint{-3.929002in}{-1.139412in}}%
\pgfpathcurveto{\pgfqpoint{-3.935513in}{-1.145923in}}{\pgfqpoint{-3.939172in}{-1.154755in}}{\pgfqpoint{-3.939172in}{-1.163964in}}%
\pgfpathcurveto{\pgfqpoint{-3.939172in}{-1.173172in}}{\pgfqpoint{-3.935513in}{-1.182005in}}{\pgfqpoint{-3.929002in}{-1.188516in}}%
\pgfpathcurveto{\pgfqpoint{-3.922491in}{-1.195028in}}{\pgfqpoint{-3.913658in}{-1.198686in}}{\pgfqpoint{-3.904450in}{-1.198686in}}%
\pgfpathlineto{\pgfqpoint{-3.904450in}{-1.198686in}}%
\pgfpathclose%
\pgfusepath{stroke,fill}%
\end{pgfscope}%
\begin{pgfscope}%
\pgfpathrectangle{\pgfqpoint{0.050000in}{0.050000in}}{\pgfqpoint{2.419000in}{2.419000in}}%
\pgfusepath{clip}%
\pgfsetbuttcap%
\pgfsetroundjoin%
\definecolor{currentfill}{rgb}{0.800000,0.400000,0.466667}%
\pgfsetfillcolor{currentfill}%
\pgfsetfillopacity{0.557113}%
\pgfsetlinewidth{1.003750pt}%
\definecolor{currentstroke}{rgb}{0.800000,0.400000,0.466667}%
\pgfsetstrokecolor{currentstroke}%
\pgfsetstrokeopacity{0.557113}%
\pgfsetdash{}{0pt}%
\pgfpathmoveto{\pgfqpoint{2.408795in}{-1.198686in}}%
\pgfpathcurveto{\pgfqpoint{2.418004in}{-1.198686in}}{\pgfqpoint{2.426836in}{-1.195028in}}{\pgfqpoint{2.433348in}{-1.188516in}}%
\pgfpathcurveto{\pgfqpoint{2.439859in}{-1.182005in}}{\pgfqpoint{2.443517in}{-1.173172in}}{\pgfqpoint{2.443517in}{-1.163964in}}%
\pgfpathcurveto{\pgfqpoint{2.443517in}{-1.154755in}}{\pgfqpoint{2.439859in}{-1.145923in}}{\pgfqpoint{2.433348in}{-1.139412in}}%
\pgfpathcurveto{\pgfqpoint{2.426836in}{-1.132900in}}{\pgfqpoint{2.418004in}{-1.129242in}}{\pgfqpoint{2.408795in}{-1.129242in}}%
\pgfpathcurveto{\pgfqpoint{2.399587in}{-1.129242in}}{\pgfqpoint{2.390754in}{-1.132900in}}{\pgfqpoint{2.384243in}{-1.139412in}}%
\pgfpathcurveto{\pgfqpoint{2.377732in}{-1.145923in}}{\pgfqpoint{2.374073in}{-1.154755in}}{\pgfqpoint{2.374073in}{-1.163964in}}%
\pgfpathcurveto{\pgfqpoint{2.374073in}{-1.173172in}}{\pgfqpoint{2.377732in}{-1.182005in}}{\pgfqpoint{2.384243in}{-1.188516in}}%
\pgfpathcurveto{\pgfqpoint{2.390754in}{-1.195028in}}{\pgfqpoint{2.399587in}{-1.198686in}}{\pgfqpoint{2.408795in}{-1.198686in}}%
\pgfpathlineto{\pgfqpoint{2.408795in}{-1.198686in}}%
\pgfpathclose%
\pgfusepath{stroke,fill}%
\end{pgfscope}%
\begin{pgfscope}%
\pgfpathrectangle{\pgfqpoint{0.050000in}{0.050000in}}{\pgfqpoint{2.419000in}{2.419000in}}%
\pgfusepath{clip}%
\pgfsetbuttcap%
\pgfsetroundjoin%
\definecolor{currentfill}{rgb}{0.800000,0.400000,0.466667}%
\pgfsetfillcolor{currentfill}%
\pgfsetfillopacity{0.557113}%
\pgfsetlinewidth{1.003750pt}%
\definecolor{currentstroke}{rgb}{0.800000,0.400000,0.466667}%
\pgfsetstrokecolor{currentstroke}%
\pgfsetstrokeopacity{0.557113}%
\pgfsetdash{}{0pt}%
\pgfpathmoveto{\pgfqpoint{8.722040in}{-1.198686in}}%
\pgfpathcurveto{\pgfqpoint{8.731249in}{-1.198686in}}{\pgfqpoint{8.740081in}{-1.195028in}}{\pgfqpoint{8.746593in}{-1.188516in}}%
\pgfpathcurveto{\pgfqpoint{8.753104in}{-1.182005in}}{\pgfqpoint{8.756763in}{-1.173172in}}{\pgfqpoint{8.756763in}{-1.163964in}}%
\pgfpathcurveto{\pgfqpoint{8.756763in}{-1.154755in}}{\pgfqpoint{8.753104in}{-1.145923in}}{\pgfqpoint{8.746593in}{-1.139412in}}%
\pgfpathcurveto{\pgfqpoint{8.740081in}{-1.132900in}}{\pgfqpoint{8.731249in}{-1.129242in}}{\pgfqpoint{8.722040in}{-1.129242in}}%
\pgfpathcurveto{\pgfqpoint{8.712832in}{-1.129242in}}{\pgfqpoint{8.703999in}{-1.132900in}}{\pgfqpoint{8.697488in}{-1.139412in}}%
\pgfpathcurveto{\pgfqpoint{8.690977in}{-1.145923in}}{\pgfqpoint{8.687318in}{-1.154755in}}{\pgfqpoint{8.687318in}{-1.163964in}}%
\pgfpathcurveto{\pgfqpoint{8.687318in}{-1.173172in}}{\pgfqpoint{8.690977in}{-1.182005in}}{\pgfqpoint{8.697488in}{-1.188516in}}%
\pgfpathcurveto{\pgfqpoint{8.703999in}{-1.195028in}}{\pgfqpoint{8.712832in}{-1.198686in}}{\pgfqpoint{8.722040in}{-1.198686in}}%
\pgfpathlineto{\pgfqpoint{8.722040in}{-1.198686in}}%
\pgfpathclose%
\pgfusepath{stroke,fill}%
\end{pgfscope}%
\begin{pgfscope}%
\pgfpathrectangle{\pgfqpoint{0.050000in}{0.050000in}}{\pgfqpoint{2.419000in}{2.419000in}}%
\pgfusepath{clip}%
\pgfsetbuttcap%
\pgfsetroundjoin%
\definecolor{currentfill}{rgb}{0.800000,0.400000,0.466667}%
\pgfsetfillcolor{currentfill}%
\pgfsetfillopacity{0.568436}%
\pgfsetlinewidth{1.003750pt}%
\definecolor{currentstroke}{rgb}{0.800000,0.400000,0.466667}%
\pgfsetstrokecolor{currentstroke}%
\pgfsetstrokeopacity{0.568436}%
\pgfsetdash{}{0pt}%
\pgfpathmoveto{\pgfqpoint{-6.648932in}{-1.430376in}}%
\pgfpathcurveto{\pgfqpoint{-6.639723in}{-1.430376in}}{\pgfqpoint{-6.630891in}{-1.426717in}}{\pgfqpoint{-6.624380in}{-1.420206in}}%
\pgfpathcurveto{\pgfqpoint{-6.617868in}{-1.413694in}}{\pgfqpoint{-6.614210in}{-1.404862in}}{\pgfqpoint{-6.614210in}{-1.395653in}}%
\pgfpathcurveto{\pgfqpoint{-6.614210in}{-1.386445in}}{\pgfqpoint{-6.617868in}{-1.377612in}}{\pgfqpoint{-6.624380in}{-1.371101in}}%
\pgfpathcurveto{\pgfqpoint{-6.630891in}{-1.364590in}}{\pgfqpoint{-6.639723in}{-1.360931in}}{\pgfqpoint{-6.648932in}{-1.360931in}}%
\pgfpathcurveto{\pgfqpoint{-6.658140in}{-1.360931in}}{\pgfqpoint{-6.666973in}{-1.364590in}}{\pgfqpoint{-6.673484in}{-1.371101in}}%
\pgfpathcurveto{\pgfqpoint{-6.679996in}{-1.377612in}}{\pgfqpoint{-6.683654in}{-1.386445in}}{\pgfqpoint{-6.683654in}{-1.395653in}}%
\pgfpathcurveto{\pgfqpoint{-6.683654in}{-1.404862in}}{\pgfqpoint{-6.679996in}{-1.413694in}}{\pgfqpoint{-6.673484in}{-1.420206in}}%
\pgfpathcurveto{\pgfqpoint{-6.666973in}{-1.426717in}}{\pgfqpoint{-6.658140in}{-1.430376in}}{\pgfqpoint{-6.648932in}{-1.430376in}}%
\pgfpathlineto{\pgfqpoint{-6.648932in}{-1.430376in}}%
\pgfpathclose%
\pgfusepath{stroke,fill}%
\end{pgfscope}%
\begin{pgfscope}%
\pgfpathrectangle{\pgfqpoint{0.050000in}{0.050000in}}{\pgfqpoint{2.419000in}{2.419000in}}%
\pgfusepath{clip}%
\pgfsetbuttcap%
\pgfsetroundjoin%
\definecolor{currentfill}{rgb}{0.800000,0.400000,0.466667}%
\pgfsetfillcolor{currentfill}%
\pgfsetfillopacity{0.568436}%
\pgfsetlinewidth{1.003750pt}%
\definecolor{currentstroke}{rgb}{0.800000,0.400000,0.466667}%
\pgfsetstrokecolor{currentstroke}%
\pgfsetstrokeopacity{0.568436}%
\pgfsetdash{}{0pt}%
\pgfpathmoveto{\pgfqpoint{-0.189495in}{-1.430376in}}%
\pgfpathcurveto{\pgfqpoint{-0.180286in}{-1.430376in}}{\pgfqpoint{-0.171454in}{-1.426717in}}{\pgfqpoint{-0.164942in}{-1.420206in}}%
\pgfpathcurveto{\pgfqpoint{-0.158431in}{-1.413694in}}{\pgfqpoint{-0.154772in}{-1.404862in}}{\pgfqpoint{-0.154772in}{-1.395653in}}%
\pgfpathcurveto{\pgfqpoint{-0.154772in}{-1.386445in}}{\pgfqpoint{-0.158431in}{-1.377612in}}{\pgfqpoint{-0.164942in}{-1.371101in}}%
\pgfpathcurveto{\pgfqpoint{-0.171454in}{-1.364590in}}{\pgfqpoint{-0.180286in}{-1.360931in}}{\pgfqpoint{-0.189495in}{-1.360931in}}%
\pgfpathcurveto{\pgfqpoint{-0.198703in}{-1.360931in}}{\pgfqpoint{-0.207535in}{-1.364590in}}{\pgfqpoint{-0.214047in}{-1.371101in}}%
\pgfpathcurveto{\pgfqpoint{-0.220558in}{-1.377612in}}{\pgfqpoint{-0.224217in}{-1.386445in}}{\pgfqpoint{-0.224217in}{-1.395653in}}%
\pgfpathcurveto{\pgfqpoint{-0.224217in}{-1.404862in}}{\pgfqpoint{-0.220558in}{-1.413694in}}{\pgfqpoint{-0.214047in}{-1.420206in}}%
\pgfpathcurveto{\pgfqpoint{-0.207535in}{-1.426717in}}{\pgfqpoint{-0.198703in}{-1.430376in}}{\pgfqpoint{-0.189495in}{-1.430376in}}%
\pgfpathlineto{\pgfqpoint{-0.189495in}{-1.430376in}}%
\pgfpathclose%
\pgfusepath{stroke,fill}%
\end{pgfscope}%
\begin{pgfscope}%
\pgfpathrectangle{\pgfqpoint{0.050000in}{0.050000in}}{\pgfqpoint{2.419000in}{2.419000in}}%
\pgfusepath{clip}%
\pgfsetbuttcap%
\pgfsetroundjoin%
\definecolor{currentfill}{rgb}{0.800000,0.400000,0.466667}%
\pgfsetfillcolor{currentfill}%
\pgfsetfillopacity{0.568436}%
\pgfsetlinewidth{1.003750pt}%
\definecolor{currentstroke}{rgb}{0.800000,0.400000,0.466667}%
\pgfsetstrokecolor{currentstroke}%
\pgfsetstrokeopacity{0.568436}%
\pgfsetdash{}{0pt}%
\pgfpathmoveto{\pgfqpoint{6.269943in}{-1.430376in}}%
\pgfpathcurveto{\pgfqpoint{6.279151in}{-1.430376in}}{\pgfqpoint{6.287984in}{-1.426717in}}{\pgfqpoint{6.294495in}{-1.420206in}}%
\pgfpathcurveto{\pgfqpoint{6.301006in}{-1.413694in}}{\pgfqpoint{6.304665in}{-1.404862in}}{\pgfqpoint{6.304665in}{-1.395653in}}%
\pgfpathcurveto{\pgfqpoint{6.304665in}{-1.386445in}}{\pgfqpoint{6.301006in}{-1.377612in}}{\pgfqpoint{6.294495in}{-1.371101in}}%
\pgfpathcurveto{\pgfqpoint{6.287984in}{-1.364590in}}{\pgfqpoint{6.279151in}{-1.360931in}}{\pgfqpoint{6.269943in}{-1.360931in}}%
\pgfpathcurveto{\pgfqpoint{6.260734in}{-1.360931in}}{\pgfqpoint{6.251902in}{-1.364590in}}{\pgfqpoint{6.245390in}{-1.371101in}}%
\pgfpathcurveto{\pgfqpoint{6.238879in}{-1.377612in}}{\pgfqpoint{6.235221in}{-1.386445in}}{\pgfqpoint{6.235221in}{-1.395653in}}%
\pgfpathcurveto{\pgfqpoint{6.235221in}{-1.404862in}}{\pgfqpoint{6.238879in}{-1.413694in}}{\pgfqpoint{6.245390in}{-1.420206in}}%
\pgfpathcurveto{\pgfqpoint{6.251902in}{-1.426717in}}{\pgfqpoint{6.260734in}{-1.430376in}}{\pgfqpoint{6.269943in}{-1.430376in}}%
\pgfpathlineto{\pgfqpoint{6.269943in}{-1.430376in}}%
\pgfpathclose%
\pgfusepath{stroke,fill}%
\end{pgfscope}%
\begin{pgfscope}%
\pgfpathrectangle{\pgfqpoint{0.050000in}{0.050000in}}{\pgfqpoint{2.419000in}{2.419000in}}%
\pgfusepath{clip}%
\pgfsetbuttcap%
\pgfsetroundjoin%
\definecolor{currentfill}{rgb}{0.800000,0.400000,0.466667}%
\pgfsetfillcolor{currentfill}%
\pgfsetfillopacity{0.580295}%
\pgfsetlinewidth{1.003750pt}%
\definecolor{currentstroke}{rgb}{0.800000,0.400000,0.466667}%
\pgfsetstrokecolor{currentstroke}%
\pgfsetstrokeopacity{0.580295}%
\pgfsetdash{}{0pt}%
\pgfpathmoveto{\pgfqpoint{3.701590in}{-1.673049in}}%
\pgfpathcurveto{\pgfqpoint{3.710798in}{-1.673049in}}{\pgfqpoint{3.719630in}{-1.669391in}}{\pgfqpoint{3.726142in}{-1.662880in}}%
\pgfpathcurveto{\pgfqpoint{3.732653in}{-1.656368in}}{\pgfqpoint{3.736312in}{-1.647536in}}{\pgfqpoint{3.736312in}{-1.638327in}}%
\pgfpathcurveto{\pgfqpoint{3.736312in}{-1.629119in}}{\pgfqpoint{3.732653in}{-1.620286in}}{\pgfqpoint{3.726142in}{-1.613775in}}%
\pgfpathcurveto{\pgfqpoint{3.719630in}{-1.607264in}}{\pgfqpoint{3.710798in}{-1.603605in}}{\pgfqpoint{3.701590in}{-1.603605in}}%
\pgfpathcurveto{\pgfqpoint{3.692381in}{-1.603605in}}{\pgfqpoint{3.683549in}{-1.607264in}}{\pgfqpoint{3.677037in}{-1.613775in}}%
\pgfpathcurveto{\pgfqpoint{3.670526in}{-1.620286in}}{\pgfqpoint{3.666867in}{-1.629119in}}{\pgfqpoint{3.666867in}{-1.638327in}}%
\pgfpathcurveto{\pgfqpoint{3.666867in}{-1.647536in}}{\pgfqpoint{3.670526in}{-1.656368in}}{\pgfqpoint{3.677037in}{-1.662880in}}%
\pgfpathcurveto{\pgfqpoint{3.683549in}{-1.669391in}}{\pgfqpoint{3.692381in}{-1.673049in}}{\pgfqpoint{3.701590in}{-1.673049in}}%
\pgfpathlineto{\pgfqpoint{3.701590in}{-1.673049in}}%
\pgfpathclose%
\pgfusepath{stroke,fill}%
\end{pgfscope}%
\begin{pgfscope}%
\pgfpathrectangle{\pgfqpoint{0.050000in}{0.050000in}}{\pgfqpoint{2.419000in}{2.419000in}}%
\pgfusepath{clip}%
\pgfsetbuttcap%
\pgfsetroundjoin%
\definecolor{currentfill}{rgb}{0.800000,0.400000,0.466667}%
\pgfsetfillcolor{currentfill}%
\pgfsetfillopacity{0.580295}%
\pgfsetlinewidth{1.003750pt}%
\definecolor{currentstroke}{rgb}{0.800000,0.400000,0.466667}%
\pgfsetstrokecolor{currentstroke}%
\pgfsetstrokeopacity{0.580295}%
\pgfsetdash{}{0pt}%
\pgfpathmoveto{\pgfqpoint{-2.910971in}{-1.673049in}}%
\pgfpathcurveto{\pgfqpoint{-2.901763in}{-1.673049in}}{\pgfqpoint{-2.892930in}{-1.669391in}}{\pgfqpoint{-2.886419in}{-1.662880in}}%
\pgfpathcurveto{\pgfqpoint{-2.879908in}{-1.656368in}}{\pgfqpoint{-2.876249in}{-1.647536in}}{\pgfqpoint{-2.876249in}{-1.638327in}}%
\pgfpathcurveto{\pgfqpoint{-2.876249in}{-1.629119in}}{\pgfqpoint{-2.879908in}{-1.620286in}}{\pgfqpoint{-2.886419in}{-1.613775in}}%
\pgfpathcurveto{\pgfqpoint{-2.892930in}{-1.607264in}}{\pgfqpoint{-2.901763in}{-1.603605in}}{\pgfqpoint{-2.910971in}{-1.603605in}}%
\pgfpathcurveto{\pgfqpoint{-2.920180in}{-1.603605in}}{\pgfqpoint{-2.929012in}{-1.607264in}}{\pgfqpoint{-2.935524in}{-1.613775in}}%
\pgfpathcurveto{\pgfqpoint{-2.942035in}{-1.620286in}}{\pgfqpoint{-2.945693in}{-1.629119in}}{\pgfqpoint{-2.945693in}{-1.638327in}}%
\pgfpathcurveto{\pgfqpoint{-2.945693in}{-1.647536in}}{\pgfqpoint{-2.942035in}{-1.656368in}}{\pgfqpoint{-2.935524in}{-1.662880in}}%
\pgfpathcurveto{\pgfqpoint{-2.929012in}{-1.669391in}}{\pgfqpoint{-2.920180in}{-1.673049in}}{\pgfqpoint{-2.910971in}{-1.673049in}}%
\pgfpathlineto{\pgfqpoint{-2.910971in}{-1.673049in}}%
\pgfpathclose%
\pgfusepath{stroke,fill}%
\end{pgfscope}%
\begin{pgfscope}%
\pgfpathrectangle{\pgfqpoint{0.050000in}{0.050000in}}{\pgfqpoint{2.419000in}{2.419000in}}%
\pgfusepath{clip}%
\pgfsetbuttcap%
\pgfsetroundjoin%
\definecolor{currentfill}{rgb}{0.800000,0.400000,0.466667}%
\pgfsetfillcolor{currentfill}%
\pgfsetfillopacity{0.580295}%
\pgfsetlinewidth{1.003750pt}%
\definecolor{currentstroke}{rgb}{0.800000,0.400000,0.466667}%
\pgfsetstrokecolor{currentstroke}%
\pgfsetstrokeopacity{0.580295}%
\pgfsetdash{}{0pt}%
\pgfpathmoveto{\pgfqpoint{10.314150in}{-1.673049in}}%
\pgfpathcurveto{\pgfqpoint{10.323359in}{-1.673049in}}{\pgfqpoint{10.332191in}{-1.669391in}}{\pgfqpoint{10.338703in}{-1.662880in}}%
\pgfpathcurveto{\pgfqpoint{10.345214in}{-1.656368in}}{\pgfqpoint{10.348872in}{-1.647536in}}{\pgfqpoint{10.348872in}{-1.638327in}}%
\pgfpathcurveto{\pgfqpoint{10.348872in}{-1.629119in}}{\pgfqpoint{10.345214in}{-1.620286in}}{\pgfqpoint{10.338703in}{-1.613775in}}%
\pgfpathcurveto{\pgfqpoint{10.332191in}{-1.607264in}}{\pgfqpoint{10.323359in}{-1.603605in}}{\pgfqpoint{10.314150in}{-1.603605in}}%
\pgfpathcurveto{\pgfqpoint{10.304942in}{-1.603605in}}{\pgfqpoint{10.296109in}{-1.607264in}}{\pgfqpoint{10.289598in}{-1.613775in}}%
\pgfpathcurveto{\pgfqpoint{10.283087in}{-1.620286in}}{\pgfqpoint{10.279428in}{-1.629119in}}{\pgfqpoint{10.279428in}{-1.638327in}}%
\pgfpathcurveto{\pgfqpoint{10.279428in}{-1.647536in}}{\pgfqpoint{10.283087in}{-1.656368in}}{\pgfqpoint{10.289598in}{-1.662880in}}%
\pgfpathcurveto{\pgfqpoint{10.296109in}{-1.669391in}}{\pgfqpoint{10.304942in}{-1.673049in}}{\pgfqpoint{10.314150in}{-1.673049in}}%
\pgfpathlineto{\pgfqpoint{10.314150in}{-1.673049in}}%
\pgfpathclose%
\pgfusepath{stroke,fill}%
\end{pgfscope}%
\begin{pgfscope}%
\pgfpathrectangle{\pgfqpoint{0.050000in}{0.050000in}}{\pgfqpoint{2.419000in}{2.419000in}}%
\pgfusepath{clip}%
\pgfsetbuttcap%
\pgfsetroundjoin%
\definecolor{currentfill}{rgb}{0.800000,0.400000,0.466667}%
\pgfsetfillcolor{currentfill}%
\pgfsetfillopacity{0.592730}%
\pgfsetlinewidth{1.003750pt}%
\definecolor{currentstroke}{rgb}{0.800000,0.400000,0.466667}%
\pgfsetstrokecolor{currentstroke}%
\pgfsetstrokeopacity{0.592730}%
\pgfsetdash{}{0pt}%
\pgfpathmoveto{\pgfqpoint{-5.764608in}{-1.927508in}}%
\pgfpathcurveto{\pgfqpoint{-5.755400in}{-1.927508in}}{\pgfqpoint{-5.746567in}{-1.923850in}}{\pgfqpoint{-5.740056in}{-1.917338in}}%
\pgfpathcurveto{\pgfqpoint{-5.733544in}{-1.910827in}}{\pgfqpoint{-5.729886in}{-1.901994in}}{\pgfqpoint{-5.729886in}{-1.892786in}}%
\pgfpathcurveto{\pgfqpoint{-5.729886in}{-1.883578in}}{\pgfqpoint{-5.733544in}{-1.874745in}}{\pgfqpoint{-5.740056in}{-1.868234in}}%
\pgfpathcurveto{\pgfqpoint{-5.746567in}{-1.861722in}}{\pgfqpoint{-5.755400in}{-1.858064in}}{\pgfqpoint{-5.764608in}{-1.858064in}}%
\pgfpathcurveto{\pgfqpoint{-5.773816in}{-1.858064in}}{\pgfqpoint{-5.782649in}{-1.861722in}}{\pgfqpoint{-5.789160in}{-1.868234in}}%
\pgfpathcurveto{\pgfqpoint{-5.795672in}{-1.874745in}}{\pgfqpoint{-5.799330in}{-1.883578in}}{\pgfqpoint{-5.799330in}{-1.892786in}}%
\pgfpathcurveto{\pgfqpoint{-5.799330in}{-1.901994in}}{\pgfqpoint{-5.795672in}{-1.910827in}}{\pgfqpoint{-5.789160in}{-1.917338in}}%
\pgfpathcurveto{\pgfqpoint{-5.782649in}{-1.923850in}}{\pgfqpoint{-5.773816in}{-1.927508in}}{\pgfqpoint{-5.764608in}{-1.927508in}}%
\pgfpathlineto{\pgfqpoint{-5.764608in}{-1.927508in}}%
\pgfpathclose%
\pgfusepath{stroke,fill}%
\end{pgfscope}%
\begin{pgfscope}%
\pgfpathrectangle{\pgfqpoint{0.050000in}{0.050000in}}{\pgfqpoint{2.419000in}{2.419000in}}%
\pgfusepath{clip}%
\pgfsetbuttcap%
\pgfsetroundjoin%
\definecolor{currentfill}{rgb}{0.800000,0.400000,0.466667}%
\pgfsetfillcolor{currentfill}%
\pgfsetfillopacity{0.592730}%
\pgfsetlinewidth{1.003750pt}%
\definecolor{currentstroke}{rgb}{0.800000,0.400000,0.466667}%
\pgfsetstrokecolor{currentstroke}%
\pgfsetstrokeopacity{0.592730}%
\pgfsetdash{}{0pt}%
\pgfpathmoveto{\pgfqpoint{1.008512in}{-1.927508in}}%
\pgfpathcurveto{\pgfqpoint{1.017721in}{-1.927508in}}{\pgfqpoint{1.026553in}{-1.923850in}}{\pgfqpoint{1.033064in}{-1.917338in}}%
\pgfpathcurveto{\pgfqpoint{1.039576in}{-1.910827in}}{\pgfqpoint{1.043234in}{-1.901994in}}{\pgfqpoint{1.043234in}{-1.892786in}}%
\pgfpathcurveto{\pgfqpoint{1.043234in}{-1.883578in}}{\pgfqpoint{1.039576in}{-1.874745in}}{\pgfqpoint{1.033064in}{-1.868234in}}%
\pgfpathcurveto{\pgfqpoint{1.026553in}{-1.861722in}}{\pgfqpoint{1.017721in}{-1.858064in}}{\pgfqpoint{1.008512in}{-1.858064in}}%
\pgfpathcurveto{\pgfqpoint{0.999304in}{-1.858064in}}{\pgfqpoint{0.990471in}{-1.861722in}}{\pgfqpoint{0.983960in}{-1.868234in}}%
\pgfpathcurveto{\pgfqpoint{0.977448in}{-1.874745in}}{\pgfqpoint{0.973790in}{-1.883578in}}{\pgfqpoint{0.973790in}{-1.892786in}}%
\pgfpathcurveto{\pgfqpoint{0.973790in}{-1.901994in}}{\pgfqpoint{0.977448in}{-1.910827in}}{\pgfqpoint{0.983960in}{-1.917338in}}%
\pgfpathcurveto{\pgfqpoint{0.990471in}{-1.923850in}}{\pgfqpoint{0.999304in}{-1.927508in}}{\pgfqpoint{1.008512in}{-1.927508in}}%
\pgfpathlineto{\pgfqpoint{1.008512in}{-1.927508in}}%
\pgfpathclose%
\pgfusepath{stroke,fill}%
\end{pgfscope}%
\begin{pgfscope}%
\pgfpathrectangle{\pgfqpoint{0.050000in}{0.050000in}}{\pgfqpoint{2.419000in}{2.419000in}}%
\pgfusepath{clip}%
\pgfsetbuttcap%
\pgfsetroundjoin%
\definecolor{currentfill}{rgb}{0.800000,0.400000,0.466667}%
\pgfsetfillcolor{currentfill}%
\pgfsetfillopacity{0.592730}%
\pgfsetlinewidth{1.003750pt}%
\definecolor{currentstroke}{rgb}{0.800000,0.400000,0.466667}%
\pgfsetstrokecolor{currentstroke}%
\pgfsetstrokeopacity{0.592730}%
\pgfsetdash{}{0pt}%
\pgfpathmoveto{\pgfqpoint{7.781632in}{-1.927508in}}%
\pgfpathcurveto{\pgfqpoint{7.790841in}{-1.927508in}}{\pgfqpoint{7.799673in}{-1.923850in}}{\pgfqpoint{7.806184in}{-1.917338in}}%
\pgfpathcurveto{\pgfqpoint{7.812696in}{-1.910827in}}{\pgfqpoint{7.816354in}{-1.901994in}}{\pgfqpoint{7.816354in}{-1.892786in}}%
\pgfpathcurveto{\pgfqpoint{7.816354in}{-1.883578in}}{\pgfqpoint{7.812696in}{-1.874745in}}{\pgfqpoint{7.806184in}{-1.868234in}}%
\pgfpathcurveto{\pgfqpoint{7.799673in}{-1.861722in}}{\pgfqpoint{7.790841in}{-1.858064in}}{\pgfqpoint{7.781632in}{-1.858064in}}%
\pgfpathcurveto{\pgfqpoint{7.772424in}{-1.858064in}}{\pgfqpoint{7.763591in}{-1.861722in}}{\pgfqpoint{7.757080in}{-1.868234in}}%
\pgfpathcurveto{\pgfqpoint{7.750568in}{-1.874745in}}{\pgfqpoint{7.746910in}{-1.883578in}}{\pgfqpoint{7.746910in}{-1.892786in}}%
\pgfpathcurveto{\pgfqpoint{7.746910in}{-1.901994in}}{\pgfqpoint{7.750568in}{-1.910827in}}{\pgfqpoint{7.757080in}{-1.917338in}}%
\pgfpathcurveto{\pgfqpoint{7.763591in}{-1.923850in}}{\pgfqpoint{7.772424in}{-1.927508in}}{\pgfqpoint{7.781632in}{-1.927508in}}%
\pgfpathlineto{\pgfqpoint{7.781632in}{-1.927508in}}%
\pgfpathclose%
\pgfusepath{stroke,fill}%
\end{pgfscope}%
\begin{pgfscope}%
\pgfpathrectangle{\pgfqpoint{0.050000in}{0.050000in}}{\pgfqpoint{2.419000in}{2.419000in}}%
\pgfusepath{clip}%
\pgfsetbuttcap%
\pgfsetroundjoin%
\definecolor{currentfill}{rgb}{0.800000,0.400000,0.466667}%
\pgfsetfillcolor{currentfill}%
\pgfsetfillopacity{0.605784}%
\pgfsetlinewidth{1.003750pt}%
\definecolor{currentstroke}{rgb}{0.800000,0.400000,0.466667}%
\pgfsetstrokecolor{currentstroke}%
\pgfsetstrokeopacity{0.605784}%
\pgfsetdash{}{0pt}%
\pgfpathmoveto{\pgfqpoint{-1.818601in}{-2.194631in}}%
\pgfpathcurveto{\pgfqpoint{-1.809392in}{-2.194631in}}{\pgfqpoint{-1.800560in}{-2.190973in}}{\pgfqpoint{-1.794049in}{-2.184462in}}%
\pgfpathcurveto{\pgfqpoint{-1.787537in}{-2.177950in}}{\pgfqpoint{-1.783879in}{-2.169118in}}{\pgfqpoint{-1.783879in}{-2.159909in}}%
\pgfpathcurveto{\pgfqpoint{-1.783879in}{-2.150701in}}{\pgfqpoint{-1.787537in}{-2.141868in}}{\pgfqpoint{-1.794049in}{-2.135357in}}%
\pgfpathcurveto{\pgfqpoint{-1.800560in}{-2.128846in}}{\pgfqpoint{-1.809392in}{-2.125187in}}{\pgfqpoint{-1.818601in}{-2.125187in}}%
\pgfpathcurveto{\pgfqpoint{-1.827809in}{-2.125187in}}{\pgfqpoint{-1.836642in}{-2.128846in}}{\pgfqpoint{-1.843153in}{-2.135357in}}%
\pgfpathcurveto{\pgfqpoint{-1.849665in}{-2.141868in}}{\pgfqpoint{-1.853323in}{-2.150701in}}{\pgfqpoint{-1.853323in}{-2.159909in}}%
\pgfpathcurveto{\pgfqpoint{-1.853323in}{-2.169118in}}{\pgfqpoint{-1.849665in}{-2.177950in}}{\pgfqpoint{-1.843153in}{-2.184462in}}%
\pgfpathcurveto{\pgfqpoint{-1.836642in}{-2.190973in}}{\pgfqpoint{-1.827809in}{-2.194631in}}{\pgfqpoint{-1.818601in}{-2.194631in}}%
\pgfpathlineto{\pgfqpoint{-1.818601in}{-2.194631in}}%
\pgfpathclose%
\pgfusepath{stroke,fill}%
\end{pgfscope}%
\begin{pgfscope}%
\pgfpathrectangle{\pgfqpoint{0.050000in}{0.050000in}}{\pgfqpoint{2.419000in}{2.419000in}}%
\pgfusepath{clip}%
\pgfsetbuttcap%
\pgfsetroundjoin%
\definecolor{currentfill}{rgb}{0.800000,0.400000,0.466667}%
\pgfsetfillcolor{currentfill}%
\pgfsetfillopacity{0.605784}%
\pgfsetlinewidth{1.003750pt}%
\definecolor{currentstroke}{rgb}{0.800000,0.400000,0.466667}%
\pgfsetstrokecolor{currentstroke}%
\pgfsetstrokeopacity{0.605784}%
\pgfsetdash{}{0pt}%
\pgfpathmoveto{\pgfqpoint{5.123070in}{-2.194631in}}%
\pgfpathcurveto{\pgfqpoint{5.132278in}{-2.194631in}}{\pgfqpoint{5.141111in}{-2.190973in}}{\pgfqpoint{5.147622in}{-2.184462in}}%
\pgfpathcurveto{\pgfqpoint{5.154133in}{-2.177950in}}{\pgfqpoint{5.157792in}{-2.169118in}}{\pgfqpoint{5.157792in}{-2.159909in}}%
\pgfpathcurveto{\pgfqpoint{5.157792in}{-2.150701in}}{\pgfqpoint{5.154133in}{-2.141868in}}{\pgfqpoint{5.147622in}{-2.135357in}}%
\pgfpathcurveto{\pgfqpoint{5.141111in}{-2.128846in}}{\pgfqpoint{5.132278in}{-2.125187in}}{\pgfqpoint{5.123070in}{-2.125187in}}%
\pgfpathcurveto{\pgfqpoint{5.113861in}{-2.125187in}}{\pgfqpoint{5.105029in}{-2.128846in}}{\pgfqpoint{5.098517in}{-2.135357in}}%
\pgfpathcurveto{\pgfqpoint{5.092006in}{-2.141868in}}{\pgfqpoint{5.088347in}{-2.150701in}}{\pgfqpoint{5.088347in}{-2.159909in}}%
\pgfpathcurveto{\pgfqpoint{5.088347in}{-2.169118in}}{\pgfqpoint{5.092006in}{-2.177950in}}{\pgfqpoint{5.098517in}{-2.184462in}}%
\pgfpathcurveto{\pgfqpoint{5.105029in}{-2.190973in}}{\pgfqpoint{5.113861in}{-2.194631in}}{\pgfqpoint{5.123070in}{-2.194631in}}%
\pgfpathlineto{\pgfqpoint{5.123070in}{-2.194631in}}%
\pgfpathclose%
\pgfusepath{stroke,fill}%
\end{pgfscope}%
\begin{pgfscope}%
\pgfpathrectangle{\pgfqpoint{0.050000in}{0.050000in}}{\pgfqpoint{2.419000in}{2.419000in}}%
\pgfusepath{clip}%
\pgfsetbuttcap%
\pgfsetroundjoin%
\definecolor{currentfill}{rgb}{0.800000,0.400000,0.466667}%
\pgfsetfillcolor{currentfill}%
\pgfsetfillopacity{0.619505}%
\pgfsetlinewidth{1.003750pt}%
\definecolor{currentstroke}{rgb}{0.800000,0.400000,0.466667}%
\pgfsetstrokecolor{currentstroke}%
\pgfsetstrokeopacity{0.619505}%
\pgfsetdash{}{0pt}%
\pgfpathmoveto{\pgfqpoint{2.328812in}{-2.475389in}}%
\pgfpathcurveto{\pgfqpoint{2.338021in}{-2.475389in}}{\pgfqpoint{2.346853in}{-2.471730in}}{\pgfqpoint{2.353365in}{-2.465219in}}%
\pgfpathcurveto{\pgfqpoint{2.359876in}{-2.458707in}}{\pgfqpoint{2.363535in}{-2.449875in}}{\pgfqpoint{2.363535in}{-2.440667in}}%
\pgfpathcurveto{\pgfqpoint{2.363535in}{-2.431458in}}{\pgfqpoint{2.359876in}{-2.422626in}}{\pgfqpoint{2.353365in}{-2.416114in}}%
\pgfpathcurveto{\pgfqpoint{2.346853in}{-2.409603in}}{\pgfqpoint{2.338021in}{-2.405944in}}{\pgfqpoint{2.328812in}{-2.405944in}}%
\pgfpathcurveto{\pgfqpoint{2.319604in}{-2.405944in}}{\pgfqpoint{2.310772in}{-2.409603in}}{\pgfqpoint{2.304260in}{-2.416114in}}%
\pgfpathcurveto{\pgfqpoint{2.297749in}{-2.422626in}}{\pgfqpoint{2.294090in}{-2.431458in}}{\pgfqpoint{2.294090in}{-2.440667in}}%
\pgfpathcurveto{\pgfqpoint{2.294090in}{-2.449875in}}{\pgfqpoint{2.297749in}{-2.458707in}}{\pgfqpoint{2.304260in}{-2.465219in}}%
\pgfpathcurveto{\pgfqpoint{2.310772in}{-2.471730in}}{\pgfqpoint{2.319604in}{-2.475389in}}{\pgfqpoint{2.328812in}{-2.475389in}}%
\pgfpathlineto{\pgfqpoint{2.328812in}{-2.475389in}}%
\pgfpathclose%
\pgfusepath{stroke,fill}%
\end{pgfscope}%
\begin{pgfscope}%
\pgfpathrectangle{\pgfqpoint{0.050000in}{0.050000in}}{\pgfqpoint{2.419000in}{2.419000in}}%
\pgfusepath{clip}%
\pgfsetbuttcap%
\pgfsetroundjoin%
\definecolor{currentfill}{rgb}{0.800000,0.400000,0.466667}%
\pgfsetfillcolor{currentfill}%
\pgfsetfillopacity{0.619505}%
\pgfsetlinewidth{1.003750pt}%
\definecolor{currentstroke}{rgb}{0.800000,0.400000,0.466667}%
\pgfsetstrokecolor{currentstroke}%
\pgfsetstrokeopacity{0.619505}%
\pgfsetdash{}{0pt}%
\pgfpathmoveto{\pgfqpoint{-4.790011in}{-2.475389in}}%
\pgfpathcurveto{\pgfqpoint{-4.780803in}{-2.475389in}}{\pgfqpoint{-4.771970in}{-2.471730in}}{\pgfqpoint{-4.765459in}{-2.465219in}}%
\pgfpathcurveto{\pgfqpoint{-4.758948in}{-2.458707in}}{\pgfqpoint{-4.755289in}{-2.449875in}}{\pgfqpoint{-4.755289in}{-2.440667in}}%
\pgfpathcurveto{\pgfqpoint{-4.755289in}{-2.431458in}}{\pgfqpoint{-4.758948in}{-2.422626in}}{\pgfqpoint{-4.765459in}{-2.416114in}}%
\pgfpathcurveto{\pgfqpoint{-4.771970in}{-2.409603in}}{\pgfqpoint{-4.780803in}{-2.405944in}}{\pgfqpoint{-4.790011in}{-2.405944in}}%
\pgfpathcurveto{\pgfqpoint{-4.799220in}{-2.405944in}}{\pgfqpoint{-4.808052in}{-2.409603in}}{\pgfqpoint{-4.814564in}{-2.416114in}}%
\pgfpathcurveto{\pgfqpoint{-4.821075in}{-2.422626in}}{\pgfqpoint{-4.824734in}{-2.431458in}}{\pgfqpoint{-4.824734in}{-2.440667in}}%
\pgfpathcurveto{\pgfqpoint{-4.824734in}{-2.449875in}}{\pgfqpoint{-4.821075in}{-2.458707in}}{\pgfqpoint{-4.814564in}{-2.465219in}}%
\pgfpathcurveto{\pgfqpoint{-4.808052in}{-2.471730in}}{\pgfqpoint{-4.799220in}{-2.475389in}}{\pgfqpoint{-4.790011in}{-2.475389in}}%
\pgfpathlineto{\pgfqpoint{-4.790011in}{-2.475389in}}%
\pgfpathclose%
\pgfusepath{stroke,fill}%
\end{pgfscope}%
\begin{pgfscope}%
\pgfpathrectangle{\pgfqpoint{0.050000in}{0.050000in}}{\pgfqpoint{2.419000in}{2.419000in}}%
\pgfusepath{clip}%
\pgfsetbuttcap%
\pgfsetroundjoin%
\definecolor{currentfill}{rgb}{0.800000,0.400000,0.466667}%
\pgfsetfillcolor{currentfill}%
\pgfsetfillopacity{0.619505}%
\pgfsetlinewidth{1.003750pt}%
\definecolor{currentstroke}{rgb}{0.800000,0.400000,0.466667}%
\pgfsetstrokecolor{currentstroke}%
\pgfsetstrokeopacity{0.619505}%
\pgfsetdash{}{0pt}%
\pgfpathmoveto{\pgfqpoint{9.447636in}{-2.475389in}}%
\pgfpathcurveto{\pgfqpoint{9.456845in}{-2.475389in}}{\pgfqpoint{9.465677in}{-2.471730in}}{\pgfqpoint{9.472189in}{-2.465219in}}%
\pgfpathcurveto{\pgfqpoint{9.478700in}{-2.458707in}}{\pgfqpoint{9.482359in}{-2.449875in}}{\pgfqpoint{9.482359in}{-2.440667in}}%
\pgfpathcurveto{\pgfqpoint{9.482359in}{-2.431458in}}{\pgfqpoint{9.478700in}{-2.422626in}}{\pgfqpoint{9.472189in}{-2.416114in}}%
\pgfpathcurveto{\pgfqpoint{9.465677in}{-2.409603in}}{\pgfqpoint{9.456845in}{-2.405944in}}{\pgfqpoint{9.447636in}{-2.405944in}}%
\pgfpathcurveto{\pgfqpoint{9.438428in}{-2.405944in}}{\pgfqpoint{9.429595in}{-2.409603in}}{\pgfqpoint{9.423084in}{-2.416114in}}%
\pgfpathcurveto{\pgfqpoint{9.416573in}{-2.422626in}}{\pgfqpoint{9.412914in}{-2.431458in}}{\pgfqpoint{9.412914in}{-2.440667in}}%
\pgfpathcurveto{\pgfqpoint{9.412914in}{-2.449875in}}{\pgfqpoint{9.416573in}{-2.458707in}}{\pgfqpoint{9.423084in}{-2.465219in}}%
\pgfpathcurveto{\pgfqpoint{9.429595in}{-2.471730in}}{\pgfqpoint{9.438428in}{-2.475389in}}{\pgfqpoint{9.447636in}{-2.475389in}}%
\pgfpathlineto{\pgfqpoint{9.447636in}{-2.475389in}}%
\pgfpathclose%
\pgfusepath{stroke,fill}%
\end{pgfscope}%
\begin{pgfscope}%
\pgfpathrectangle{\pgfqpoint{0.050000in}{0.050000in}}{\pgfqpoint{2.419000in}{2.419000in}}%
\pgfusepath{clip}%
\pgfsetbuttcap%
\pgfsetroundjoin%
\definecolor{currentfill}{rgb}{0.800000,0.400000,0.466667}%
\pgfsetfillcolor{currentfill}%
\pgfsetfillopacity{0.633944}%
\pgfsetlinewidth{1.003750pt}%
\definecolor{currentstroke}{rgb}{0.800000,0.400000,0.466667}%
\pgfsetstrokecolor{currentstroke}%
\pgfsetstrokeopacity{0.633944}%
\pgfsetdash{}{0pt}%
\pgfpathmoveto{\pgfqpoint{-0.611800in}{-2.770851in}}%
\pgfpathcurveto{\pgfqpoint{-0.602592in}{-2.770851in}}{\pgfqpoint{-0.593759in}{-2.767193in}}{\pgfqpoint{-0.587248in}{-2.760681in}}%
\pgfpathcurveto{\pgfqpoint{-0.580736in}{-2.754170in}}{\pgfqpoint{-0.577078in}{-2.745338in}}{\pgfqpoint{-0.577078in}{-2.736129in}}%
\pgfpathcurveto{\pgfqpoint{-0.577078in}{-2.726921in}}{\pgfqpoint{-0.580736in}{-2.718088in}}{\pgfqpoint{-0.587248in}{-2.711577in}}%
\pgfpathcurveto{\pgfqpoint{-0.593759in}{-2.705065in}}{\pgfqpoint{-0.602592in}{-2.701407in}}{\pgfqpoint{-0.611800in}{-2.701407in}}%
\pgfpathcurveto{\pgfqpoint{-0.621008in}{-2.701407in}}{\pgfqpoint{-0.629841in}{-2.705065in}}{\pgfqpoint{-0.636352in}{-2.711577in}}%
\pgfpathcurveto{\pgfqpoint{-0.642864in}{-2.718088in}}{\pgfqpoint{-0.646522in}{-2.726921in}}{\pgfqpoint{-0.646522in}{-2.736129in}}%
\pgfpathcurveto{\pgfqpoint{-0.646522in}{-2.745338in}}{\pgfqpoint{-0.642864in}{-2.754170in}}{\pgfqpoint{-0.636352in}{-2.760681in}}%
\pgfpathcurveto{\pgfqpoint{-0.629841in}{-2.767193in}}{\pgfqpoint{-0.621008in}{-2.770851in}}{\pgfqpoint{-0.611800in}{-2.770851in}}%
\pgfpathlineto{\pgfqpoint{-0.611800in}{-2.770851in}}%
\pgfpathclose%
\pgfusepath{stroke,fill}%
\end{pgfscope}%
\begin{pgfscope}%
\pgfpathrectangle{\pgfqpoint{0.050000in}{0.050000in}}{\pgfqpoint{2.419000in}{2.419000in}}%
\pgfusepath{clip}%
\pgfsetbuttcap%
\pgfsetroundjoin%
\definecolor{currentfill}{rgb}{0.800000,0.400000,0.466667}%
\pgfsetfillcolor{currentfill}%
\pgfsetfillopacity{0.633944}%
\pgfsetlinewidth{1.003750pt}%
\definecolor{currentstroke}{rgb}{0.800000,0.400000,0.466667}%
\pgfsetstrokecolor{currentstroke}%
\pgfsetstrokeopacity{0.633944}%
\pgfsetdash{}{0pt}%
\pgfpathmoveto{\pgfqpoint{6.693456in}{-2.770851in}}%
\pgfpathcurveto{\pgfqpoint{6.702664in}{-2.770851in}}{\pgfqpoint{6.711497in}{-2.767193in}}{\pgfqpoint{6.718008in}{-2.760681in}}%
\pgfpathcurveto{\pgfqpoint{6.724520in}{-2.754170in}}{\pgfqpoint{6.728178in}{-2.745338in}}{\pgfqpoint{6.728178in}{-2.736129in}}%
\pgfpathcurveto{\pgfqpoint{6.728178in}{-2.726921in}}{\pgfqpoint{6.724520in}{-2.718088in}}{\pgfqpoint{6.718008in}{-2.711577in}}%
\pgfpathcurveto{\pgfqpoint{6.711497in}{-2.705065in}}{\pgfqpoint{6.702664in}{-2.701407in}}{\pgfqpoint{6.693456in}{-2.701407in}}%
\pgfpathcurveto{\pgfqpoint{6.684248in}{-2.701407in}}{\pgfqpoint{6.675415in}{-2.705065in}}{\pgfqpoint{6.668904in}{-2.711577in}}%
\pgfpathcurveto{\pgfqpoint{6.662392in}{-2.718088in}}{\pgfqpoint{6.658734in}{-2.726921in}}{\pgfqpoint{6.658734in}{-2.736129in}}%
\pgfpathcurveto{\pgfqpoint{6.658734in}{-2.745338in}}{\pgfqpoint{6.662392in}{-2.754170in}}{\pgfqpoint{6.668904in}{-2.760681in}}%
\pgfpathcurveto{\pgfqpoint{6.675415in}{-2.767193in}}{\pgfqpoint{6.684248in}{-2.770851in}}{\pgfqpoint{6.693456in}{-2.770851in}}%
\pgfpathlineto{\pgfqpoint{6.693456in}{-2.770851in}}%
\pgfpathclose%
\pgfusepath{stroke,fill}%
\end{pgfscope}%
\begin{pgfscope}%
\pgfpathrectangle{\pgfqpoint{0.050000in}{0.050000in}}{\pgfqpoint{2.419000in}{2.419000in}}%
\pgfusepath{clip}%
\pgfsetbuttcap%
\pgfsetroundjoin%
\definecolor{currentfill}{rgb}{0.800000,0.400000,0.466667}%
\pgfsetfillcolor{currentfill}%
\pgfsetfillopacity{0.649160}%
\pgfsetlinewidth{1.003750pt}%
\definecolor{currentstroke}{rgb}{0.800000,0.400000,0.466667}%
\pgfsetstrokecolor{currentstroke}%
\pgfsetstrokeopacity{0.649160}%
\pgfsetdash{}{0pt}%
\pgfpathmoveto{\pgfqpoint{-3.710576in}{-3.082206in}}%
\pgfpathcurveto{\pgfqpoint{-3.701367in}{-3.082206in}}{\pgfqpoint{-3.692535in}{-3.078547in}}{\pgfqpoint{-3.686023in}{-3.072036in}}%
\pgfpathcurveto{\pgfqpoint{-3.679512in}{-3.065524in}}{\pgfqpoint{-3.675854in}{-3.056692in}}{\pgfqpoint{-3.675854in}{-3.047483in}}%
\pgfpathcurveto{\pgfqpoint{-3.675854in}{-3.038275in}}{\pgfqpoint{-3.679512in}{-3.029442in}}{\pgfqpoint{-3.686023in}{-3.022931in}}%
\pgfpathcurveto{\pgfqpoint{-3.692535in}{-3.016420in}}{\pgfqpoint{-3.701367in}{-3.012761in}}{\pgfqpoint{-3.710576in}{-3.012761in}}%
\pgfpathcurveto{\pgfqpoint{-3.719784in}{-3.012761in}}{\pgfqpoint{-3.728617in}{-3.016420in}}{\pgfqpoint{-3.735128in}{-3.022931in}}%
\pgfpathcurveto{\pgfqpoint{-3.741639in}{-3.029442in}}{\pgfqpoint{-3.745298in}{-3.038275in}}{\pgfqpoint{-3.745298in}{-3.047483in}}%
\pgfpathcurveto{\pgfqpoint{-3.745298in}{-3.056692in}}{\pgfqpoint{-3.741639in}{-3.065524in}}{\pgfqpoint{-3.735128in}{-3.072036in}}%
\pgfpathcurveto{\pgfqpoint{-3.728617in}{-3.078547in}}{\pgfqpoint{-3.719784in}{-3.082206in}}{\pgfqpoint{-3.710576in}{-3.082206in}}%
\pgfpathlineto{\pgfqpoint{-3.710576in}{-3.082206in}}%
\pgfpathclose%
\pgfusepath{stroke,fill}%
\end{pgfscope}%
\begin{pgfscope}%
\pgfpathrectangle{\pgfqpoint{0.050000in}{0.050000in}}{\pgfqpoint{2.419000in}{2.419000in}}%
\pgfusepath{clip}%
\pgfsetbuttcap%
\pgfsetroundjoin%
\definecolor{currentfill}{rgb}{0.800000,0.400000,0.466667}%
\pgfsetfillcolor{currentfill}%
\pgfsetfillopacity{0.649160}%
\pgfsetlinewidth{1.003750pt}%
\definecolor{currentstroke}{rgb}{0.800000,0.400000,0.466667}%
\pgfsetstrokecolor{currentstroke}%
\pgfsetstrokeopacity{0.649160}%
\pgfsetdash{}{0pt}%
\pgfpathmoveto{\pgfqpoint{3.791140in}{-3.082206in}}%
\pgfpathcurveto{\pgfqpoint{3.800348in}{-3.082206in}}{\pgfqpoint{3.809181in}{-3.078547in}}{\pgfqpoint{3.815692in}{-3.072036in}}%
\pgfpathcurveto{\pgfqpoint{3.822203in}{-3.065524in}}{\pgfqpoint{3.825862in}{-3.056692in}}{\pgfqpoint{3.825862in}{-3.047483in}}%
\pgfpathcurveto{\pgfqpoint{3.825862in}{-3.038275in}}{\pgfqpoint{3.822203in}{-3.029442in}}{\pgfqpoint{3.815692in}{-3.022931in}}%
\pgfpathcurveto{\pgfqpoint{3.809181in}{-3.016420in}}{\pgfqpoint{3.800348in}{-3.012761in}}{\pgfqpoint{3.791140in}{-3.012761in}}%
\pgfpathcurveto{\pgfqpoint{3.781931in}{-3.012761in}}{\pgfqpoint{3.773099in}{-3.016420in}}{\pgfqpoint{3.766587in}{-3.022931in}}%
\pgfpathcurveto{\pgfqpoint{3.760076in}{-3.029442in}}{\pgfqpoint{3.756418in}{-3.038275in}}{\pgfqpoint{3.756418in}{-3.047483in}}%
\pgfpathcurveto{\pgfqpoint{3.756418in}{-3.056692in}}{\pgfqpoint{3.760076in}{-3.065524in}}{\pgfqpoint{3.766587in}{-3.072036in}}%
\pgfpathcurveto{\pgfqpoint{3.773099in}{-3.078547in}}{\pgfqpoint{3.781931in}{-3.082206in}}{\pgfqpoint{3.791140in}{-3.082206in}}%
\pgfpathlineto{\pgfqpoint{3.791140in}{-3.082206in}}%
\pgfpathclose%
\pgfusepath{stroke,fill}%
\end{pgfscope}%
\begin{pgfscope}%
\pgfpathrectangle{\pgfqpoint{0.050000in}{0.050000in}}{\pgfqpoint{2.419000in}{2.419000in}}%
\pgfusepath{clip}%
\pgfsetbuttcap%
\pgfsetroundjoin%
\definecolor{currentfill}{rgb}{0.800000,0.400000,0.466667}%
\pgfsetfillcolor{currentfill}%
\pgfsetfillopacity{0.665216}%
\pgfsetlinewidth{1.003750pt}%
\definecolor{currentstroke}{rgb}{0.800000,0.400000,0.466667}%
\pgfsetstrokecolor{currentstroke}%
\pgfsetstrokeopacity{0.665216}%
\pgfsetdash{}{0pt}%
\pgfpathmoveto{\pgfqpoint{0.728406in}{-3.410769in}}%
\pgfpathcurveto{\pgfqpoint{0.737615in}{-3.410769in}}{\pgfqpoint{0.746447in}{-3.407111in}}{\pgfqpoint{0.752958in}{-3.400599in}}%
\pgfpathcurveto{\pgfqpoint{0.759470in}{-3.394088in}}{\pgfqpoint{0.763128in}{-3.385255in}}{\pgfqpoint{0.763128in}{-3.376047in}}%
\pgfpathcurveto{\pgfqpoint{0.763128in}{-3.366839in}}{\pgfqpoint{0.759470in}{-3.358006in}}{\pgfqpoint{0.752958in}{-3.351495in}}%
\pgfpathcurveto{\pgfqpoint{0.746447in}{-3.344983in}}{\pgfqpoint{0.737615in}{-3.341325in}}{\pgfqpoint{0.728406in}{-3.341325in}}%
\pgfpathcurveto{\pgfqpoint{0.719198in}{-3.341325in}}{\pgfqpoint{0.710365in}{-3.344983in}}{\pgfqpoint{0.703854in}{-3.351495in}}%
\pgfpathcurveto{\pgfqpoint{0.697342in}{-3.358006in}}{\pgfqpoint{0.693684in}{-3.366839in}}{\pgfqpoint{0.693684in}{-3.376047in}}%
\pgfpathcurveto{\pgfqpoint{0.693684in}{-3.385255in}}{\pgfqpoint{0.697342in}{-3.394088in}}{\pgfqpoint{0.703854in}{-3.400599in}}%
\pgfpathcurveto{\pgfqpoint{0.710365in}{-3.407111in}}{\pgfqpoint{0.719198in}{-3.410769in}}{\pgfqpoint{0.728406in}{-3.410769in}}%
\pgfpathlineto{\pgfqpoint{0.728406in}{-3.410769in}}%
\pgfpathclose%
\pgfusepath{stroke,fill}%
\end{pgfscope}%
\begin{pgfscope}%
\pgfpathrectangle{\pgfqpoint{0.050000in}{0.050000in}}{\pgfqpoint{2.419000in}{2.419000in}}%
\pgfusepath{clip}%
\pgfsetbuttcap%
\pgfsetroundjoin%
\definecolor{currentfill}{rgb}{0.800000,0.400000,0.466667}%
\pgfsetfillcolor{currentfill}%
\pgfsetfillopacity{0.665216}%
\pgfsetlinewidth{1.003750pt}%
\definecolor{currentstroke}{rgb}{0.800000,0.400000,0.466667}%
\pgfsetstrokecolor{currentstroke}%
\pgfsetstrokeopacity{0.665216}%
\pgfsetdash{}{0pt}%
\pgfpathmoveto{\pgfqpoint{8.437440in}{-3.410769in}}%
\pgfpathcurveto{\pgfqpoint{8.446648in}{-3.410769in}}{\pgfqpoint{8.455481in}{-3.407111in}}{\pgfqpoint{8.461992in}{-3.400599in}}%
\pgfpathcurveto{\pgfqpoint{8.468504in}{-3.394088in}}{\pgfqpoint{8.472162in}{-3.385255in}}{\pgfqpoint{8.472162in}{-3.376047in}}%
\pgfpathcurveto{\pgfqpoint{8.472162in}{-3.366839in}}{\pgfqpoint{8.468504in}{-3.358006in}}{\pgfqpoint{8.461992in}{-3.351495in}}%
\pgfpathcurveto{\pgfqpoint{8.455481in}{-3.344983in}}{\pgfqpoint{8.446648in}{-3.341325in}}{\pgfqpoint{8.437440in}{-3.341325in}}%
\pgfpathcurveto{\pgfqpoint{8.428232in}{-3.341325in}}{\pgfqpoint{8.419399in}{-3.344983in}}{\pgfqpoint{8.412888in}{-3.351495in}}%
\pgfpathcurveto{\pgfqpoint{8.406376in}{-3.358006in}}{\pgfqpoint{8.402718in}{-3.366839in}}{\pgfqpoint{8.402718in}{-3.376047in}}%
\pgfpathcurveto{\pgfqpoint{8.402718in}{-3.385255in}}{\pgfqpoint{8.406376in}{-3.394088in}}{\pgfqpoint{8.412888in}{-3.400599in}}%
\pgfpathcurveto{\pgfqpoint{8.419399in}{-3.407111in}}{\pgfqpoint{8.428232in}{-3.410769in}}{\pgfqpoint{8.437440in}{-3.410769in}}%
\pgfpathlineto{\pgfqpoint{8.437440in}{-3.410769in}}%
\pgfpathclose%
\pgfusepath{stroke,fill}%
\end{pgfscope}%
\begin{pgfscope}%
\pgfpathrectangle{\pgfqpoint{0.050000in}{0.050000in}}{\pgfqpoint{2.419000in}{2.419000in}}%
\pgfusepath{clip}%
\pgfsetbuttcap%
\pgfsetroundjoin%
\definecolor{currentfill}{rgb}{0.800000,0.400000,0.466667}%
\pgfsetfillcolor{currentfill}%
\pgfsetfillopacity{0.682186}%
\pgfsetlinewidth{1.003750pt}%
\definecolor{currentstroke}{rgb}{0.800000,0.400000,0.466667}%
\pgfsetstrokecolor{currentstroke}%
\pgfsetstrokeopacity{0.682186}%
\pgfsetdash{}{0pt}%
\pgfpathmoveto{\pgfqpoint{5.419714in}{-3.758009in}}%
\pgfpathcurveto{\pgfqpoint{5.428922in}{-3.758009in}}{\pgfqpoint{5.437755in}{-3.754351in}}{\pgfqpoint{5.444266in}{-3.747840in}}%
\pgfpathcurveto{\pgfqpoint{5.450777in}{-3.741328in}}{\pgfqpoint{5.454436in}{-3.732496in}}{\pgfqpoint{5.454436in}{-3.723287in}}%
\pgfpathcurveto{\pgfqpoint{5.454436in}{-3.714079in}}{\pgfqpoint{5.450777in}{-3.705246in}}{\pgfqpoint{5.444266in}{-3.698735in}}%
\pgfpathcurveto{\pgfqpoint{5.437755in}{-3.692224in}}{\pgfqpoint{5.428922in}{-3.688565in}}{\pgfqpoint{5.419714in}{-3.688565in}}%
\pgfpathcurveto{\pgfqpoint{5.410505in}{-3.688565in}}{\pgfqpoint{5.401673in}{-3.692224in}}{\pgfqpoint{5.395161in}{-3.698735in}}%
\pgfpathcurveto{\pgfqpoint{5.388650in}{-3.705246in}}{\pgfqpoint{5.384991in}{-3.714079in}}{\pgfqpoint{5.384991in}{-3.723287in}}%
\pgfpathcurveto{\pgfqpoint{5.384991in}{-3.732496in}}{\pgfqpoint{5.388650in}{-3.741328in}}{\pgfqpoint{5.395161in}{-3.747840in}}%
\pgfpathcurveto{\pgfqpoint{5.401673in}{-3.754351in}}{\pgfqpoint{5.410505in}{-3.758009in}}{\pgfqpoint{5.419714in}{-3.758009in}}%
\pgfpathlineto{\pgfqpoint{5.419714in}{-3.758009in}}%
\pgfpathclose%
\pgfusepath{stroke,fill}%
\end{pgfscope}%
\begin{pgfscope}%
\pgfpathrectangle{\pgfqpoint{0.050000in}{0.050000in}}{\pgfqpoint{2.419000in}{2.419000in}}%
\pgfusepath{clip}%
\pgfsetbuttcap%
\pgfsetroundjoin%
\definecolor{currentfill}{rgb}{0.800000,0.400000,0.466667}%
\pgfsetfillcolor{currentfill}%
\pgfsetfillopacity{0.682186}%
\pgfsetlinewidth{1.003750pt}%
\definecolor{currentstroke}{rgb}{0.800000,0.400000,0.466667}%
\pgfsetstrokecolor{currentstroke}%
\pgfsetstrokeopacity{0.682186}%
\pgfsetdash{}{0pt}%
\pgfpathmoveto{\pgfqpoint{-2.508423in}{-3.758009in}}%
\pgfpathcurveto{\pgfqpoint{-2.499215in}{-3.758009in}}{\pgfqpoint{-2.490382in}{-3.754351in}}{\pgfqpoint{-2.483871in}{-3.747840in}}%
\pgfpathcurveto{\pgfqpoint{-2.477359in}{-3.741328in}}{\pgfqpoint{-2.473701in}{-3.732496in}}{\pgfqpoint{-2.473701in}{-3.723287in}}%
\pgfpathcurveto{\pgfqpoint{-2.473701in}{-3.714079in}}{\pgfqpoint{-2.477359in}{-3.705246in}}{\pgfqpoint{-2.483871in}{-3.698735in}}%
\pgfpathcurveto{\pgfqpoint{-2.490382in}{-3.692224in}}{\pgfqpoint{-2.499215in}{-3.688565in}}{\pgfqpoint{-2.508423in}{-3.688565in}}%
\pgfpathcurveto{\pgfqpoint{-2.517632in}{-3.688565in}}{\pgfqpoint{-2.526464in}{-3.692224in}}{\pgfqpoint{-2.532975in}{-3.698735in}}%
\pgfpathcurveto{\pgfqpoint{-2.539487in}{-3.705246in}}{\pgfqpoint{-2.543145in}{-3.714079in}}{\pgfqpoint{-2.543145in}{-3.723287in}}%
\pgfpathcurveto{\pgfqpoint{-2.543145in}{-3.732496in}}{\pgfqpoint{-2.539487in}{-3.741328in}}{\pgfqpoint{-2.532975in}{-3.747840in}}%
\pgfpathcurveto{\pgfqpoint{-2.526464in}{-3.754351in}}{\pgfqpoint{-2.517632in}{-3.758009in}}{\pgfqpoint{-2.508423in}{-3.758009in}}%
\pgfpathlineto{\pgfqpoint{-2.508423in}{-3.758009in}}%
\pgfpathclose%
\pgfusepath{stroke,fill}%
\end{pgfscope}%
\begin{pgfscope}%
\pgfpathrectangle{\pgfqpoint{0.050000in}{0.050000in}}{\pgfqpoint{2.419000in}{2.419000in}}%
\pgfusepath{clip}%
\pgfsetbuttcap%
\pgfsetroundjoin%
\definecolor{currentfill}{rgb}{0.800000,0.400000,0.466667}%
\pgfsetfillcolor{currentfill}%
\pgfsetfillopacity{0.700148}%
\pgfsetlinewidth{1.003750pt}%
\definecolor{currentstroke}{rgb}{0.800000,0.400000,0.466667}%
\pgfsetstrokecolor{currentstroke}%
\pgfsetstrokeopacity{0.700148}%
\pgfsetdash{}{0pt}%
\pgfpathmoveto{\pgfqpoint{2.225432in}{-4.125565in}}%
\pgfpathcurveto{\pgfqpoint{2.234641in}{-4.125565in}}{\pgfqpoint{2.243473in}{-4.121907in}}{\pgfqpoint{2.249985in}{-4.115395in}}%
\pgfpathcurveto{\pgfqpoint{2.256496in}{-4.108884in}}{\pgfqpoint{2.260155in}{-4.100051in}}{\pgfqpoint{2.260155in}{-4.090843in}}%
\pgfpathcurveto{\pgfqpoint{2.260155in}{-4.081635in}}{\pgfqpoint{2.256496in}{-4.072802in}}{\pgfqpoint{2.249985in}{-4.066291in}}%
\pgfpathcurveto{\pgfqpoint{2.243473in}{-4.059779in}}{\pgfqpoint{2.234641in}{-4.056121in}}{\pgfqpoint{2.225432in}{-4.056121in}}%
\pgfpathcurveto{\pgfqpoint{2.216224in}{-4.056121in}}{\pgfqpoint{2.207391in}{-4.059779in}}{\pgfqpoint{2.200880in}{-4.066291in}}%
\pgfpathcurveto{\pgfqpoint{2.194369in}{-4.072802in}}{\pgfqpoint{2.190710in}{-4.081635in}}{\pgfqpoint{2.190710in}{-4.090843in}}%
\pgfpathcurveto{\pgfqpoint{2.190710in}{-4.100051in}}{\pgfqpoint{2.194369in}{-4.108884in}}{\pgfqpoint{2.200880in}{-4.115395in}}%
\pgfpathcurveto{\pgfqpoint{2.207391in}{-4.121907in}}{\pgfqpoint{2.216224in}{-4.125565in}}{\pgfqpoint{2.225432in}{-4.125565in}}%
\pgfpathlineto{\pgfqpoint{2.225432in}{-4.125565in}}%
\pgfpathclose%
\pgfusepath{stroke,fill}%
\end{pgfscope}%
\begin{pgfscope}%
\pgfpathrectangle{\pgfqpoint{0.050000in}{0.050000in}}{\pgfqpoint{2.419000in}{2.419000in}}%
\pgfusepath{clip}%
\pgfsetbuttcap%
\pgfsetroundjoin%
\definecolor{currentfill}{rgb}{0.800000,0.400000,0.466667}%
\pgfsetfillcolor{currentfill}%
\pgfsetfillopacity{0.700148}%
\pgfsetlinewidth{1.003750pt}%
\definecolor{currentstroke}{rgb}{0.800000,0.400000,0.466667}%
\pgfsetstrokecolor{currentstroke}%
\pgfsetstrokeopacity{0.700148}%
\pgfsetdash{}{0pt}%
\pgfpathmoveto{\pgfqpoint{10.385491in}{-4.125565in}}%
\pgfpathcurveto{\pgfqpoint{10.394699in}{-4.125565in}}{\pgfqpoint{10.403532in}{-4.121907in}}{\pgfqpoint{10.410043in}{-4.115395in}}%
\pgfpathcurveto{\pgfqpoint{10.416555in}{-4.108884in}}{\pgfqpoint{10.420213in}{-4.100051in}}{\pgfqpoint{10.420213in}{-4.090843in}}%
\pgfpathcurveto{\pgfqpoint{10.420213in}{-4.081635in}}{\pgfqpoint{10.416555in}{-4.072802in}}{\pgfqpoint{10.410043in}{-4.066291in}}%
\pgfpathcurveto{\pgfqpoint{10.403532in}{-4.059779in}}{\pgfqpoint{10.394699in}{-4.056121in}}{\pgfqpoint{10.385491in}{-4.056121in}}%
\pgfpathcurveto{\pgfqpoint{10.376283in}{-4.056121in}}{\pgfqpoint{10.367450in}{-4.059779in}}{\pgfqpoint{10.360939in}{-4.066291in}}%
\pgfpathcurveto{\pgfqpoint{10.354427in}{-4.072802in}}{\pgfqpoint{10.350769in}{-4.081635in}}{\pgfqpoint{10.350769in}{-4.090843in}}%
\pgfpathcurveto{\pgfqpoint{10.350769in}{-4.100051in}}{\pgfqpoint{10.354427in}{-4.108884in}}{\pgfqpoint{10.360939in}{-4.115395in}}%
\pgfpathcurveto{\pgfqpoint{10.367450in}{-4.121907in}}{\pgfqpoint{10.376283in}{-4.125565in}}{\pgfqpoint{10.385491in}{-4.125565in}}%
\pgfpathlineto{\pgfqpoint{10.385491in}{-4.125565in}}%
\pgfpathclose%
\pgfusepath{stroke,fill}%
\end{pgfscope}%
\begin{pgfscope}%
\pgfpathrectangle{\pgfqpoint{0.050000in}{0.050000in}}{\pgfqpoint{2.419000in}{2.419000in}}%
\pgfusepath{clip}%
\pgfsetbuttcap%
\pgfsetroundjoin%
\definecolor{currentfill}{rgb}{0.800000,0.400000,0.466667}%
\pgfsetfillcolor{currentfill}%
\pgfsetfillopacity{0.719193}%
\pgfsetlinewidth{1.003750pt}%
\definecolor{currentstroke}{rgb}{0.800000,0.400000,0.466667}%
\pgfsetstrokecolor{currentstroke}%
\pgfsetstrokeopacity{0.719193}%
\pgfsetdash{}{0pt}%
\pgfpathmoveto{\pgfqpoint{-1.161365in}{-4.515273in}}%
\pgfpathcurveto{\pgfqpoint{-1.152157in}{-4.515273in}}{\pgfqpoint{-1.143324in}{-4.511615in}}{\pgfqpoint{-1.136813in}{-4.505103in}}%
\pgfpathcurveto{\pgfqpoint{-1.130302in}{-4.498592in}}{\pgfqpoint{-1.126643in}{-4.489759in}}{\pgfqpoint{-1.126643in}{-4.480551in}}%
\pgfpathcurveto{\pgfqpoint{-1.126643in}{-4.471343in}}{\pgfqpoint{-1.130302in}{-4.462510in}}{\pgfqpoint{-1.136813in}{-4.455999in}}%
\pgfpathcurveto{\pgfqpoint{-1.143324in}{-4.449487in}}{\pgfqpoint{-1.152157in}{-4.445829in}}{\pgfqpoint{-1.161365in}{-4.445829in}}%
\pgfpathcurveto{\pgfqpoint{-1.170574in}{-4.445829in}}{\pgfqpoint{-1.179406in}{-4.449487in}}{\pgfqpoint{-1.185918in}{-4.455999in}}%
\pgfpathcurveto{\pgfqpoint{-1.192429in}{-4.462510in}}{\pgfqpoint{-1.196088in}{-4.471343in}}{\pgfqpoint{-1.196088in}{-4.480551in}}%
\pgfpathcurveto{\pgfqpoint{-1.196088in}{-4.489759in}}{\pgfqpoint{-1.192429in}{-4.498592in}}{\pgfqpoint{-1.185918in}{-4.505103in}}%
\pgfpathcurveto{\pgfqpoint{-1.179406in}{-4.511615in}}{\pgfqpoint{-1.170574in}{-4.515273in}}{\pgfqpoint{-1.161365in}{-4.515273in}}%
\pgfpathlineto{\pgfqpoint{-1.161365in}{-4.515273in}}%
\pgfpathclose%
\pgfusepath{stroke,fill}%
\end{pgfscope}%
\begin{pgfscope}%
\pgfpathrectangle{\pgfqpoint{0.050000in}{0.050000in}}{\pgfqpoint{2.419000in}{2.419000in}}%
\pgfusepath{clip}%
\pgfsetbuttcap%
\pgfsetroundjoin%
\definecolor{currentfill}{rgb}{0.800000,0.400000,0.466667}%
\pgfsetfillcolor{currentfill}%
\pgfsetfillopacity{0.719193}%
\pgfsetlinewidth{1.003750pt}%
\definecolor{currentstroke}{rgb}{0.800000,0.400000,0.466667}%
\pgfsetstrokecolor{currentstroke}%
\pgfsetstrokeopacity{0.719193}%
\pgfsetdash{}{0pt}%
\pgfpathmoveto{\pgfqpoint{7.244593in}{-4.515273in}}%
\pgfpathcurveto{\pgfqpoint{7.253801in}{-4.515273in}}{\pgfqpoint{7.262634in}{-4.511615in}}{\pgfqpoint{7.269145in}{-4.505103in}}%
\pgfpathcurveto{\pgfqpoint{7.275657in}{-4.498592in}}{\pgfqpoint{7.279315in}{-4.489759in}}{\pgfqpoint{7.279315in}{-4.480551in}}%
\pgfpathcurveto{\pgfqpoint{7.279315in}{-4.471343in}}{\pgfqpoint{7.275657in}{-4.462510in}}{\pgfqpoint{7.269145in}{-4.455999in}}%
\pgfpathcurveto{\pgfqpoint{7.262634in}{-4.449487in}}{\pgfqpoint{7.253801in}{-4.445829in}}{\pgfqpoint{7.244593in}{-4.445829in}}%
\pgfpathcurveto{\pgfqpoint{7.235384in}{-4.445829in}}{\pgfqpoint{7.226552in}{-4.449487in}}{\pgfqpoint{7.220041in}{-4.455999in}}%
\pgfpathcurveto{\pgfqpoint{7.213529in}{-4.462510in}}{\pgfqpoint{7.209871in}{-4.471343in}}{\pgfqpoint{7.209871in}{-4.480551in}}%
\pgfpathcurveto{\pgfqpoint{7.209871in}{-4.489759in}}{\pgfqpoint{7.213529in}{-4.498592in}}{\pgfqpoint{7.220041in}{-4.505103in}}%
\pgfpathcurveto{\pgfqpoint{7.226552in}{-4.511615in}}{\pgfqpoint{7.235384in}{-4.515273in}}{\pgfqpoint{7.244593in}{-4.515273in}}%
\pgfpathlineto{\pgfqpoint{7.244593in}{-4.515273in}}%
\pgfpathclose%
\pgfusepath{stroke,fill}%
\end{pgfscope}%
\begin{pgfscope}%
\pgfpathrectangle{\pgfqpoint{0.050000in}{0.050000in}}{\pgfqpoint{2.419000in}{2.419000in}}%
\pgfusepath{clip}%
\pgfsetbuttcap%
\pgfsetroundjoin%
\definecolor{currentfill}{rgb}{0.800000,0.400000,0.466667}%
\pgfsetfillcolor{currentfill}%
\pgfsetfillopacity{0.739421}%
\pgfsetlinewidth{1.003750pt}%
\definecolor{currentstroke}{rgb}{0.800000,0.400000,0.466667}%
\pgfsetstrokecolor{currentstroke}%
\pgfsetstrokeopacity{0.739421}%
\pgfsetdash{}{0pt}%
\pgfpathmoveto{\pgfqpoint{3.908514in}{-4.929198in}}%
\pgfpathcurveto{\pgfqpoint{3.917723in}{-4.929198in}}{\pgfqpoint{3.926555in}{-4.925540in}}{\pgfqpoint{3.933066in}{-4.919028in}}%
\pgfpathcurveto{\pgfqpoint{3.939578in}{-4.912517in}}{\pgfqpoint{3.943236in}{-4.903685in}}{\pgfqpoint{3.943236in}{-4.894476in}}%
\pgfpathcurveto{\pgfqpoint{3.943236in}{-4.885268in}}{\pgfqpoint{3.939578in}{-4.876435in}}{\pgfqpoint{3.933066in}{-4.869924in}}%
\pgfpathcurveto{\pgfqpoint{3.926555in}{-4.863413in}}{\pgfqpoint{3.917723in}{-4.859754in}}{\pgfqpoint{3.908514in}{-4.859754in}}%
\pgfpathcurveto{\pgfqpoint{3.899306in}{-4.859754in}}{\pgfqpoint{3.890473in}{-4.863413in}}{\pgfqpoint{3.883962in}{-4.869924in}}%
\pgfpathcurveto{\pgfqpoint{3.877450in}{-4.876435in}}{\pgfqpoint{3.873792in}{-4.885268in}}{\pgfqpoint{3.873792in}{-4.894476in}}%
\pgfpathcurveto{\pgfqpoint{3.873792in}{-4.903685in}}{\pgfqpoint{3.877450in}{-4.912517in}}{\pgfqpoint{3.883962in}{-4.919028in}}%
\pgfpathcurveto{\pgfqpoint{3.890473in}{-4.925540in}}{\pgfqpoint{3.899306in}{-4.929198in}}{\pgfqpoint{3.908514in}{-4.929198in}}%
\pgfpathlineto{\pgfqpoint{3.908514in}{-4.929198in}}%
\pgfpathclose%
\pgfusepath{stroke,fill}%
\end{pgfscope}%
\begin{pgfscope}%
\pgfpathrectangle{\pgfqpoint{0.050000in}{0.050000in}}{\pgfqpoint{2.419000in}{2.419000in}}%
\pgfusepath{clip}%
\pgfsetbuttcap%
\pgfsetroundjoin%
\definecolor{currentfill}{rgb}{0.800000,0.400000,0.466667}%
\pgfsetfillcolor{currentfill}%
\pgfsetfillopacity{0.760947}%
\pgfsetlinewidth{1.003750pt}%
\definecolor{currentstroke}{rgb}{0.800000,0.400000,0.466667}%
\pgfsetstrokecolor{currentstroke}%
\pgfsetstrokeopacity{0.760947}%
\pgfsetdash{}{0pt}%
\pgfpathmoveto{\pgfqpoint{0.358478in}{-5.369670in}}%
\pgfpathcurveto{\pgfqpoint{0.367686in}{-5.369670in}}{\pgfqpoint{0.376519in}{-5.366012in}}{\pgfqpoint{0.383030in}{-5.359500in}}%
\pgfpathcurveto{\pgfqpoint{0.389542in}{-5.352989in}}{\pgfqpoint{0.393200in}{-5.344157in}}{\pgfqpoint{0.393200in}{-5.334948in}}%
\pgfpathcurveto{\pgfqpoint{0.393200in}{-5.325740in}}{\pgfqpoint{0.389542in}{-5.316907in}}{\pgfqpoint{0.383030in}{-5.310396in}}%
\pgfpathcurveto{\pgfqpoint{0.376519in}{-5.303884in}}{\pgfqpoint{0.367686in}{-5.300226in}}{\pgfqpoint{0.358478in}{-5.300226in}}%
\pgfpathcurveto{\pgfqpoint{0.349270in}{-5.300226in}}{\pgfqpoint{0.340437in}{-5.303884in}}{\pgfqpoint{0.333926in}{-5.310396in}}%
\pgfpathcurveto{\pgfqpoint{0.327414in}{-5.316907in}}{\pgfqpoint{0.323756in}{-5.325740in}}{\pgfqpoint{0.323756in}{-5.334948in}}%
\pgfpathcurveto{\pgfqpoint{0.323756in}{-5.344157in}}{\pgfqpoint{0.327414in}{-5.352989in}}{\pgfqpoint{0.333926in}{-5.359500in}}%
\pgfpathcurveto{\pgfqpoint{0.340437in}{-5.366012in}}{\pgfqpoint{0.349270in}{-5.369670in}}{\pgfqpoint{0.358478in}{-5.369670in}}%
\pgfpathlineto{\pgfqpoint{0.358478in}{-5.369670in}}%
\pgfpathclose%
\pgfusepath{stroke,fill}%
\end{pgfscope}%
\begin{pgfscope}%
\pgfpathrectangle{\pgfqpoint{0.050000in}{0.050000in}}{\pgfqpoint{2.419000in}{2.419000in}}%
\pgfusepath{clip}%
\pgfsetbuttcap%
\pgfsetroundjoin%
\definecolor{currentfill}{rgb}{0.800000,0.400000,0.466667}%
\pgfsetfillcolor{currentfill}%
\pgfsetfillopacity{0.760947}%
\pgfsetlinewidth{1.003750pt}%
\definecolor{currentstroke}{rgb}{0.800000,0.400000,0.466667}%
\pgfsetstrokecolor{currentstroke}%
\pgfsetstrokeopacity{0.760947}%
\pgfsetdash{}{0pt}%
\pgfpathmoveto{\pgfqpoint{9.303547in}{-5.369670in}}%
\pgfpathcurveto{\pgfqpoint{9.312755in}{-5.369670in}}{\pgfqpoint{9.321588in}{-5.366012in}}{\pgfqpoint{9.328099in}{-5.359500in}}%
\pgfpathcurveto{\pgfqpoint{9.334611in}{-5.352989in}}{\pgfqpoint{9.338269in}{-5.344157in}}{\pgfqpoint{9.338269in}{-5.334948in}}%
\pgfpathcurveto{\pgfqpoint{9.338269in}{-5.325740in}}{\pgfqpoint{9.334611in}{-5.316907in}}{\pgfqpoint{9.328099in}{-5.310396in}}%
\pgfpathcurveto{\pgfqpoint{9.321588in}{-5.303884in}}{\pgfqpoint{9.312755in}{-5.300226in}}{\pgfqpoint{9.303547in}{-5.300226in}}%
\pgfpathcurveto{\pgfqpoint{9.294339in}{-5.300226in}}{\pgfqpoint{9.285506in}{-5.303884in}}{\pgfqpoint{9.278995in}{-5.310396in}}%
\pgfpathcurveto{\pgfqpoint{9.272483in}{-5.316907in}}{\pgfqpoint{9.268825in}{-5.325740in}}{\pgfqpoint{9.268825in}{-5.334948in}}%
\pgfpathcurveto{\pgfqpoint{9.268825in}{-5.344157in}}{\pgfqpoint{9.272483in}{-5.352989in}}{\pgfqpoint{9.278995in}{-5.359500in}}%
\pgfpathcurveto{\pgfqpoint{9.285506in}{-5.366012in}}{\pgfqpoint{9.294339in}{-5.369670in}}{\pgfqpoint{9.303547in}{-5.369670in}}%
\pgfpathlineto{\pgfqpoint{9.303547in}{-5.369670in}}%
\pgfpathclose%
\pgfusepath{stroke,fill}%
\end{pgfscope}%
\begin{pgfscope}%
\pgfpathrectangle{\pgfqpoint{0.050000in}{0.050000in}}{\pgfqpoint{2.419000in}{2.419000in}}%
\pgfusepath{clip}%
\pgfsetbuttcap%
\pgfsetroundjoin%
\definecolor{currentfill}{rgb}{0.800000,0.400000,0.466667}%
\pgfsetfillcolor{currentfill}%
\pgfsetfillopacity{0.783899}%
\pgfsetlinewidth{1.003750pt}%
\definecolor{currentstroke}{rgb}{0.800000,0.400000,0.466667}%
\pgfsetstrokecolor{currentstroke}%
\pgfsetstrokeopacity{0.783899}%
\pgfsetdash{}{0pt}%
\pgfpathmoveto{\pgfqpoint{5.814635in}{-5.839328in}}%
\pgfpathcurveto{\pgfqpoint{5.823843in}{-5.839328in}}{\pgfqpoint{5.832676in}{-5.835669in}}{\pgfqpoint{5.839187in}{-5.829158in}}%
\pgfpathcurveto{\pgfqpoint{5.845698in}{-5.822646in}}{\pgfqpoint{5.849357in}{-5.813814in}}{\pgfqpoint{5.849357in}{-5.804605in}}%
\pgfpathcurveto{\pgfqpoint{5.849357in}{-5.795397in}}{\pgfqpoint{5.845698in}{-5.786564in}}{\pgfqpoint{5.839187in}{-5.780053in}}%
\pgfpathcurveto{\pgfqpoint{5.832676in}{-5.773542in}}{\pgfqpoint{5.823843in}{-5.769883in}}{\pgfqpoint{5.814635in}{-5.769883in}}%
\pgfpathcurveto{\pgfqpoint{5.805426in}{-5.769883in}}{\pgfqpoint{5.796594in}{-5.773542in}}{\pgfqpoint{5.790082in}{-5.780053in}}%
\pgfpathcurveto{\pgfqpoint{5.783571in}{-5.786564in}}{\pgfqpoint{5.779913in}{-5.795397in}}{\pgfqpoint{5.779913in}{-5.804605in}}%
\pgfpathcurveto{\pgfqpoint{5.779913in}{-5.813814in}}{\pgfqpoint{5.783571in}{-5.822646in}}{\pgfqpoint{5.790082in}{-5.829158in}}%
\pgfpathcurveto{\pgfqpoint{5.796594in}{-5.835669in}}{\pgfqpoint{5.805426in}{-5.839328in}}{\pgfqpoint{5.814635in}{-5.839328in}}%
\pgfpathlineto{\pgfqpoint{5.814635in}{-5.839328in}}%
\pgfpathclose%
\pgfusepath{stroke,fill}%
\end{pgfscope}%
\begin{pgfscope}%
\pgfpathrectangle{\pgfqpoint{0.050000in}{0.050000in}}{\pgfqpoint{2.419000in}{2.419000in}}%
\pgfusepath{clip}%
\pgfsetbuttcap%
\pgfsetroundjoin%
\definecolor{currentfill}{rgb}{0.800000,0.400000,0.466667}%
\pgfsetfillcolor{currentfill}%
\pgfsetfillopacity{0.808423}%
\pgfsetlinewidth{1.003750pt}%
\definecolor{currentstroke}{rgb}{0.800000,0.400000,0.466667}%
\pgfsetstrokecolor{currentstroke}%
\pgfsetstrokeopacity{0.808423}%
\pgfsetdash{}{0pt}%
\pgfpathmoveto{\pgfqpoint{2.086629in}{-6.341170in}}%
\pgfpathcurveto{\pgfqpoint{2.095838in}{-6.341170in}}{\pgfqpoint{2.104670in}{-6.337512in}}{\pgfqpoint{2.111182in}{-6.331000in}}%
\pgfpathcurveto{\pgfqpoint{2.117693in}{-6.324489in}}{\pgfqpoint{2.121352in}{-6.315656in}}{\pgfqpoint{2.121352in}{-6.306448in}}%
\pgfpathcurveto{\pgfqpoint{2.121352in}{-6.297240in}}{\pgfqpoint{2.117693in}{-6.288407in}}{\pgfqpoint{2.111182in}{-6.281896in}}%
\pgfpathcurveto{\pgfqpoint{2.104670in}{-6.275384in}}{\pgfqpoint{2.095838in}{-6.271726in}}{\pgfqpoint{2.086629in}{-6.271726in}}%
\pgfpathcurveto{\pgfqpoint{2.077421in}{-6.271726in}}{\pgfqpoint{2.068588in}{-6.275384in}}{\pgfqpoint{2.062077in}{-6.281896in}}%
\pgfpathcurveto{\pgfqpoint{2.055566in}{-6.288407in}}{\pgfqpoint{2.051907in}{-6.297240in}}{\pgfqpoint{2.051907in}{-6.306448in}}%
\pgfpathcurveto{\pgfqpoint{2.051907in}{-6.315656in}}{\pgfqpoint{2.055566in}{-6.324489in}}{\pgfqpoint{2.062077in}{-6.331000in}}%
\pgfpathcurveto{\pgfqpoint{2.068588in}{-6.337512in}}{\pgfqpoint{2.077421in}{-6.341170in}}{\pgfqpoint{2.086629in}{-6.341170in}}%
\pgfpathlineto{\pgfqpoint{2.086629in}{-6.341170in}}%
\pgfpathclose%
\pgfusepath{stroke,fill}%
\end{pgfscope}%
\begin{pgfscope}%
\pgfpathrectangle{\pgfqpoint{0.050000in}{0.050000in}}{\pgfqpoint{2.419000in}{2.419000in}}%
\pgfusepath{clip}%
\pgfsetbuttcap%
\pgfsetroundjoin%
\definecolor{currentfill}{rgb}{0.800000,0.400000,0.466667}%
\pgfsetfillcolor{currentfill}%
\pgfsetfillopacity{0.834689}%
\pgfsetlinewidth{1.003750pt}%
\definecolor{currentstroke}{rgb}{0.800000,0.400000,0.466667}%
\pgfsetstrokecolor{currentstroke}%
\pgfsetstrokeopacity{0.834689}%
\pgfsetdash{}{0pt}%
\pgfpathmoveto{\pgfqpoint{7.991276in}{-6.878624in}}%
\pgfpathcurveto{\pgfqpoint{8.000484in}{-6.878624in}}{\pgfqpoint{8.009317in}{-6.874966in}}{\pgfqpoint{8.015828in}{-6.868454in}}%
\pgfpathcurveto{\pgfqpoint{8.022339in}{-6.861943in}}{\pgfqpoint{8.025998in}{-6.853110in}}{\pgfqpoint{8.025998in}{-6.843902in}}%
\pgfpathcurveto{\pgfqpoint{8.025998in}{-6.834693in}}{\pgfqpoint{8.022339in}{-6.825861in}}{\pgfqpoint{8.015828in}{-6.819350in}}%
\pgfpathcurveto{\pgfqpoint{8.009317in}{-6.812838in}}{\pgfqpoint{8.000484in}{-6.809180in}}{\pgfqpoint{7.991276in}{-6.809180in}}%
\pgfpathcurveto{\pgfqpoint{7.982067in}{-6.809180in}}{\pgfqpoint{7.973235in}{-6.812838in}}{\pgfqpoint{7.966723in}{-6.819350in}}%
\pgfpathcurveto{\pgfqpoint{7.960212in}{-6.825861in}}{\pgfqpoint{7.956553in}{-6.834693in}}{\pgfqpoint{7.956553in}{-6.843902in}}%
\pgfpathcurveto{\pgfqpoint{7.956553in}{-6.853110in}}{\pgfqpoint{7.960212in}{-6.861943in}}{\pgfqpoint{7.966723in}{-6.868454in}}%
\pgfpathcurveto{\pgfqpoint{7.973235in}{-6.874966in}}{\pgfqpoint{7.982067in}{-6.878624in}}{\pgfqpoint{7.991276in}{-6.878624in}}%
\pgfpathlineto{\pgfqpoint{7.991276in}{-6.878624in}}%
\pgfpathclose%
\pgfusepath{stroke,fill}%
\end{pgfscope}%
\begin{pgfscope}%
\pgfpathrectangle{\pgfqpoint{0.050000in}{0.050000in}}{\pgfqpoint{2.419000in}{2.419000in}}%
\pgfusepath{clip}%
\pgfsetbuttcap%
\pgfsetroundjoin%
\definecolor{currentfill}{rgb}{0.800000,0.400000,0.466667}%
\pgfsetfillcolor{currentfill}%
\pgfsetfillopacity{0.862886}%
\pgfsetlinewidth{1.003750pt}%
\definecolor{currentstroke}{rgb}{0.800000,0.400000,0.466667}%
\pgfsetstrokecolor{currentstroke}%
\pgfsetstrokeopacity{0.862886}%
\pgfsetdash{}{0pt}%
\pgfpathmoveto{\pgfqpoint{4.069065in}{-7.455619in}}%
\pgfpathcurveto{\pgfqpoint{4.078274in}{-7.455619in}}{\pgfqpoint{4.087106in}{-7.451960in}}{\pgfqpoint{4.093618in}{-7.445449in}}%
\pgfpathcurveto{\pgfqpoint{4.100129in}{-7.438938in}}{\pgfqpoint{4.103787in}{-7.430105in}}{\pgfqpoint{4.103787in}{-7.420897in}}%
\pgfpathcurveto{\pgfqpoint{4.103787in}{-7.411688in}}{\pgfqpoint{4.100129in}{-7.402856in}}{\pgfqpoint{4.093618in}{-7.396345in}}%
\pgfpathcurveto{\pgfqpoint{4.087106in}{-7.389833in}}{\pgfqpoint{4.078274in}{-7.386175in}}{\pgfqpoint{4.069065in}{-7.386175in}}%
\pgfpathcurveto{\pgfqpoint{4.059857in}{-7.386175in}}{\pgfqpoint{4.051024in}{-7.389833in}}{\pgfqpoint{4.044513in}{-7.396345in}}%
\pgfpathcurveto{\pgfqpoint{4.038002in}{-7.402856in}}{\pgfqpoint{4.034343in}{-7.411688in}}{\pgfqpoint{4.034343in}{-7.420897in}}%
\pgfpathcurveto{\pgfqpoint{4.034343in}{-7.430105in}}{\pgfqpoint{4.038002in}{-7.438938in}}{\pgfqpoint{4.044513in}{-7.445449in}}%
\pgfpathcurveto{\pgfqpoint{4.051024in}{-7.451960in}}{\pgfqpoint{4.059857in}{-7.455619in}}{\pgfqpoint{4.069065in}{-7.455619in}}%
\pgfpathlineto{\pgfqpoint{4.069065in}{-7.455619in}}%
\pgfpathclose%
\pgfusepath{stroke,fill}%
\end{pgfscope}%
\begin{pgfscope}%
\pgfpathrectangle{\pgfqpoint{0.050000in}{0.050000in}}{\pgfqpoint{2.419000in}{2.419000in}}%
\pgfusepath{clip}%
\pgfsetbuttcap%
\pgfsetroundjoin%
\definecolor{currentfill}{rgb}{0.800000,0.400000,0.466667}%
\pgfsetfillcolor{currentfill}%
\pgfsetfillopacity{0.893237}%
\pgfsetlinewidth{1.003750pt}%
\definecolor{currentstroke}{rgb}{0.800000,0.400000,0.466667}%
\pgfsetstrokecolor{currentstroke}%
\pgfsetstrokeopacity{0.893237}%
\pgfsetdash{}{0pt}%
\pgfpathmoveto{\pgfqpoint{10.500424in}{-8.076685in}}%
\pgfpathcurveto{\pgfqpoint{10.509633in}{-8.076685in}}{\pgfqpoint{10.518465in}{-8.073027in}}{\pgfqpoint{10.524977in}{-8.066516in}}%
\pgfpathcurveto{\pgfqpoint{10.531488in}{-8.060004in}}{\pgfqpoint{10.535147in}{-8.051172in}}{\pgfqpoint{10.535147in}{-8.041963in}}%
\pgfpathcurveto{\pgfqpoint{10.535147in}{-8.032755in}}{\pgfqpoint{10.531488in}{-8.023922in}}{\pgfqpoint{10.524977in}{-8.017411in}}%
\pgfpathcurveto{\pgfqpoint{10.518465in}{-8.010900in}}{\pgfqpoint{10.509633in}{-8.007241in}}{\pgfqpoint{10.500424in}{-8.007241in}}%
\pgfpathcurveto{\pgfqpoint{10.491216in}{-8.007241in}}{\pgfqpoint{10.482383in}{-8.010900in}}{\pgfqpoint{10.475872in}{-8.017411in}}%
\pgfpathcurveto{\pgfqpoint{10.469361in}{-8.023922in}}{\pgfqpoint{10.465702in}{-8.032755in}}{\pgfqpoint{10.465702in}{-8.041963in}}%
\pgfpathcurveto{\pgfqpoint{10.465702in}{-8.051172in}}{\pgfqpoint{10.469361in}{-8.060004in}}{\pgfqpoint{10.475872in}{-8.066516in}}%
\pgfpathcurveto{\pgfqpoint{10.482383in}{-8.073027in}}{\pgfqpoint{10.491216in}{-8.076685in}}{\pgfqpoint{10.500424in}{-8.076685in}}%
\pgfpathlineto{\pgfqpoint{10.500424in}{-8.076685in}}%
\pgfpathclose%
\pgfusepath{stroke,fill}%
\end{pgfscope}%
\begin{pgfscope}%
\pgfpathrectangle{\pgfqpoint{0.050000in}{0.050000in}}{\pgfqpoint{2.419000in}{2.419000in}}%
\pgfusepath{clip}%
\pgfsetbuttcap%
\pgfsetroundjoin%
\definecolor{currentfill}{rgb}{0.800000,0.400000,0.466667}%
\pgfsetfillcolor{currentfill}%
\pgfsetfillopacity{0.925999}%
\pgfsetlinewidth{1.003750pt}%
\definecolor{currentstroke}{rgb}{0.800000,0.400000,0.466667}%
\pgfsetstrokecolor{currentstroke}%
\pgfsetstrokeopacity{0.925999}%
\pgfsetdash{}{0pt}%
\pgfpathmoveto{\pgfqpoint{6.366367in}{-8.747073in}}%
\pgfpathcurveto{\pgfqpoint{6.375575in}{-8.747073in}}{\pgfqpoint{6.384408in}{-8.743415in}}{\pgfqpoint{6.390919in}{-8.736903in}}%
\pgfpathcurveto{\pgfqpoint{6.397430in}{-8.730392in}}{\pgfqpoint{6.401089in}{-8.721559in}}{\pgfqpoint{6.401089in}{-8.712351in}}%
\pgfpathcurveto{\pgfqpoint{6.401089in}{-8.703143in}}{\pgfqpoint{6.397430in}{-8.694310in}}{\pgfqpoint{6.390919in}{-8.687799in}}%
\pgfpathcurveto{\pgfqpoint{6.384408in}{-8.681287in}}{\pgfqpoint{6.375575in}{-8.677629in}}{\pgfqpoint{6.366367in}{-8.677629in}}%
\pgfpathcurveto{\pgfqpoint{6.357158in}{-8.677629in}}{\pgfqpoint{6.348326in}{-8.681287in}}{\pgfqpoint{6.341815in}{-8.687799in}}%
\pgfpathcurveto{\pgfqpoint{6.335303in}{-8.694310in}}{\pgfqpoint{6.331645in}{-8.703143in}}{\pgfqpoint{6.331645in}{-8.712351in}}%
\pgfpathcurveto{\pgfqpoint{6.331645in}{-8.721559in}}{\pgfqpoint{6.335303in}{-8.730392in}}{\pgfqpoint{6.341815in}{-8.736903in}}%
\pgfpathcurveto{\pgfqpoint{6.348326in}{-8.743415in}}{\pgfqpoint{6.357158in}{-8.747073in}}{\pgfqpoint{6.366367in}{-8.747073in}}%
\pgfpathlineto{\pgfqpoint{6.366367in}{-8.747073in}}%
\pgfpathclose%
\pgfusepath{stroke,fill}%
\end{pgfscope}%
\begin{pgfscope}%
\pgfpathrectangle{\pgfqpoint{0.050000in}{0.050000in}}{\pgfqpoint{2.419000in}{2.419000in}}%
\pgfusepath{clip}%
\pgfsetbuttcap%
\pgfsetroundjoin%
\definecolor{currentfill}{rgb}{0.800000,0.400000,0.466667}%
\pgfsetfillcolor{currentfill}%
\pgfsetlinewidth{1.003750pt}%
\definecolor{currentstroke}{rgb}{0.800000,0.400000,0.466667}%
\pgfsetstrokecolor{currentstroke}%
\pgfsetdash{}{0pt}%
\pgfpathmoveto{\pgfqpoint{9.060019in}{-10.261340in}}%
\pgfpathcurveto{\pgfqpoint{9.069228in}{-10.261340in}}{\pgfqpoint{9.078060in}{-10.257682in}}{\pgfqpoint{9.084572in}{-10.251170in}}%
\pgfpathcurveto{\pgfqpoint{9.091083in}{-10.244659in}}{\pgfqpoint{9.094741in}{-10.235827in}}{\pgfqpoint{9.094741in}{-10.226618in}}%
\pgfpathcurveto{\pgfqpoint{9.094741in}{-10.217410in}}{\pgfqpoint{9.091083in}{-10.208577in}}{\pgfqpoint{9.084572in}{-10.202066in}}%
\pgfpathcurveto{\pgfqpoint{9.078060in}{-10.195555in}}{\pgfqpoint{9.069228in}{-10.191896in}}{\pgfqpoint{9.060019in}{-10.191896in}}%
\pgfpathcurveto{\pgfqpoint{9.050811in}{-10.191896in}}{\pgfqpoint{9.041978in}{-10.195555in}}{\pgfqpoint{9.035467in}{-10.202066in}}%
\pgfpathcurveto{\pgfqpoint{9.028956in}{-10.208577in}}{\pgfqpoint{9.025297in}{-10.217410in}}{\pgfqpoint{9.025297in}{-10.226618in}}%
\pgfpathcurveto{\pgfqpoint{9.025297in}{-10.235827in}}{\pgfqpoint{9.028956in}{-10.244659in}}{\pgfqpoint{9.035467in}{-10.251170in}}%
\pgfpathcurveto{\pgfqpoint{9.041978in}{-10.257682in}}{\pgfqpoint{9.050811in}{-10.261340in}}{\pgfqpoint{9.060019in}{-10.261340in}}%
\pgfpathlineto{\pgfqpoint{9.060019in}{-10.261340in}}%
\pgfpathclose%
\pgfusepath{stroke,fill}%
\end{pgfscope}%
\begin{pgfscope}%
\pgfpathrectangle{\pgfqpoint{0.050000in}{0.050000in}}{\pgfqpoint{2.419000in}{2.419000in}}%
\pgfusepath{clip}%
\pgfsetbuttcap%
\pgfsetroundjoin%
\pgfsetlinewidth{1.003750pt}%
\definecolor{currentstroke}{rgb}{0.000000,0.000000,0.000000}%
\pgfsetstrokecolor{currentstroke}%
\pgfsetdash{}{0pt}%
\pgfusepath{stroke}%
\end{pgfscope}%
\begin{pgfscope}%
\pgfpathrectangle{\pgfqpoint{0.050000in}{0.050000in}}{\pgfqpoint{2.419000in}{2.419000in}}%
\pgfusepath{clip}%
\pgfsetbuttcap%
\pgfsetroundjoin%
\pgfsetlinewidth{1.003750pt}%
\definecolor{currentstroke}{rgb}{0.000000,0.000000,0.000000}%
\pgfsetstrokecolor{currentstroke}%
\pgfsetdash{}{0pt}%
\pgfusepath{stroke}%
\end{pgfscope}%
\begin{pgfscope}%
\pgfpathrectangle{\pgfqpoint{0.050000in}{0.050000in}}{\pgfqpoint{2.419000in}{2.419000in}}%
\pgfusepath{clip}%
\pgfsetbuttcap%
\pgfsetroundjoin%
\pgfsetlinewidth{1.003750pt}%
\definecolor{currentstroke}{rgb}{0.000000,0.000000,0.000000}%
\pgfsetstrokecolor{currentstroke}%
\pgfsetdash{}{0pt}%
\pgfusepath{stroke}%
\end{pgfscope}%
\begin{pgfscope}%
\pgfpathrectangle{\pgfqpoint{0.050000in}{0.050000in}}{\pgfqpoint{2.419000in}{2.419000in}}%
\pgfusepath{clip}%
\pgfsetbuttcap%
\pgfsetroundjoin%
\definecolor{currentfill}{rgb}{0.200000,0.133333,0.533333}%
\pgfsetfillcolor{currentfill}%
\pgfsetfillopacity{0.300000}%
\pgfsetlinewidth{1.003750pt}%
\definecolor{currentstroke}{rgb}{0.200000,0.133333,0.533333}%
\pgfsetstrokecolor{currentstroke}%
\pgfsetstrokeopacity{0.300000}%
\pgfsetdash{}{0pt}%
\pgfpathmoveto{\pgfqpoint{1.776596in}{4.194523in}}%
\pgfpathcurveto{\pgfqpoint{1.785805in}{4.194523in}}{\pgfqpoint{1.794637in}{4.198182in}}{\pgfqpoint{1.801149in}{4.204693in}}%
\pgfpathcurveto{\pgfqpoint{1.807660in}{4.211205in}}{\pgfqpoint{1.811319in}{4.220037in}}{\pgfqpoint{1.811319in}{4.229246in}}%
\pgfpathcurveto{\pgfqpoint{1.811319in}{4.238454in}}{\pgfqpoint{1.807660in}{4.247286in}}{\pgfqpoint{1.801149in}{4.253798in}}%
\pgfpathcurveto{\pgfqpoint{1.794637in}{4.260309in}}{\pgfqpoint{1.785805in}{4.263968in}}{\pgfqpoint{1.776596in}{4.263968in}}%
\pgfpathcurveto{\pgfqpoint{1.767388in}{4.263968in}}{\pgfqpoint{1.758555in}{4.260309in}}{\pgfqpoint{1.752044in}{4.253798in}}%
\pgfpathcurveto{\pgfqpoint{1.745533in}{4.247286in}}{\pgfqpoint{1.741874in}{4.238454in}}{\pgfqpoint{1.741874in}{4.229246in}}%
\pgfpathcurveto{\pgfqpoint{1.741874in}{4.220037in}}{\pgfqpoint{1.745533in}{4.211205in}}{\pgfqpoint{1.752044in}{4.204693in}}%
\pgfpathcurveto{\pgfqpoint{1.758555in}{4.198182in}}{\pgfqpoint{1.767388in}{4.194523in}}{\pgfqpoint{1.776596in}{4.194523in}}%
\pgfpathlineto{\pgfqpoint{1.776596in}{4.194523in}}%
\pgfpathclose%
\pgfusepath{stroke,fill}%
\end{pgfscope}%
\begin{pgfscope}%
\pgfpathrectangle{\pgfqpoint{0.050000in}{0.050000in}}{\pgfqpoint{2.419000in}{2.419000in}}%
\pgfusepath{clip}%
\pgfsetbuttcap%
\pgfsetroundjoin%
\definecolor{currentfill}{rgb}{0.200000,0.133333,0.533333}%
\pgfsetfillcolor{currentfill}%
\pgfsetfillopacity{0.304670}%
\pgfsetlinewidth{1.003750pt}%
\definecolor{currentstroke}{rgb}{0.200000,0.133333,0.533333}%
\pgfsetstrokecolor{currentstroke}%
\pgfsetstrokeopacity{0.304670}%
\pgfsetdash{}{0pt}%
\pgfpathmoveto{\pgfqpoint{2.344213in}{4.096236in}}%
\pgfpathcurveto{\pgfqpoint{2.353422in}{4.096236in}}{\pgfqpoint{2.362254in}{4.099895in}}{\pgfqpoint{2.368766in}{4.106406in}}%
\pgfpathcurveto{\pgfqpoint{2.375277in}{4.112917in}}{\pgfqpoint{2.378935in}{4.121750in}}{\pgfqpoint{2.378935in}{4.130958in}}%
\pgfpathcurveto{\pgfqpoint{2.378935in}{4.140167in}}{\pgfqpoint{2.375277in}{4.148999in}}{\pgfqpoint{2.368766in}{4.155511in}}%
\pgfpathcurveto{\pgfqpoint{2.362254in}{4.162022in}}{\pgfqpoint{2.353422in}{4.165680in}}{\pgfqpoint{2.344213in}{4.165680in}}%
\pgfpathcurveto{\pgfqpoint{2.335005in}{4.165680in}}{\pgfqpoint{2.326172in}{4.162022in}}{\pgfqpoint{2.319661in}{4.155511in}}%
\pgfpathcurveto{\pgfqpoint{2.313150in}{4.148999in}}{\pgfqpoint{2.309491in}{4.140167in}}{\pgfqpoint{2.309491in}{4.130958in}}%
\pgfpathcurveto{\pgfqpoint{2.309491in}{4.121750in}}{\pgfqpoint{2.313150in}{4.112917in}}{\pgfqpoint{2.319661in}{4.106406in}}%
\pgfpathcurveto{\pgfqpoint{2.326172in}{4.099895in}}{\pgfqpoint{2.335005in}{4.096236in}}{\pgfqpoint{2.344213in}{4.096236in}}%
\pgfpathlineto{\pgfqpoint{2.344213in}{4.096236in}}%
\pgfpathclose%
\pgfusepath{stroke,fill}%
\end{pgfscope}%
\begin{pgfscope}%
\pgfpathrectangle{\pgfqpoint{0.050000in}{0.050000in}}{\pgfqpoint{2.419000in}{2.419000in}}%
\pgfusepath{clip}%
\pgfsetbuttcap%
\pgfsetroundjoin%
\definecolor{currentfill}{rgb}{0.200000,0.133333,0.533333}%
\pgfsetfillcolor{currentfill}%
\pgfsetfillopacity{0.309542}%
\pgfsetlinewidth{1.003750pt}%
\definecolor{currentstroke}{rgb}{0.200000,0.133333,0.533333}%
\pgfsetstrokecolor{currentstroke}%
\pgfsetstrokeopacity{0.309542}%
\pgfsetdash{}{0pt}%
\pgfpathmoveto{\pgfqpoint{2.936549in}{3.993668in}}%
\pgfpathcurveto{\pgfqpoint{2.945757in}{3.993668in}}{\pgfqpoint{2.954590in}{3.997327in}}{\pgfqpoint{2.961101in}{4.003838in}}%
\pgfpathcurveto{\pgfqpoint{2.967613in}{4.010350in}}{\pgfqpoint{2.971271in}{4.019182in}}{\pgfqpoint{2.971271in}{4.028391in}}%
\pgfpathcurveto{\pgfqpoint{2.971271in}{4.037599in}}{\pgfqpoint{2.967613in}{4.046432in}}{\pgfqpoint{2.961101in}{4.052943in}}%
\pgfpathcurveto{\pgfqpoint{2.954590in}{4.059454in}}{\pgfqpoint{2.945757in}{4.063113in}}{\pgfqpoint{2.936549in}{4.063113in}}%
\pgfpathcurveto{\pgfqpoint{2.927341in}{4.063113in}}{\pgfqpoint{2.918508in}{4.059454in}}{\pgfqpoint{2.911997in}{4.052943in}}%
\pgfpathcurveto{\pgfqpoint{2.905485in}{4.046432in}}{\pgfqpoint{2.901827in}{4.037599in}}{\pgfqpoint{2.901827in}{4.028391in}}%
\pgfpathcurveto{\pgfqpoint{2.901827in}{4.019182in}}{\pgfqpoint{2.905485in}{4.010350in}}{\pgfqpoint{2.911997in}{4.003838in}}%
\pgfpathcurveto{\pgfqpoint{2.918508in}{3.997327in}}{\pgfqpoint{2.927341in}{3.993668in}}{\pgfqpoint{2.936549in}{3.993668in}}%
\pgfpathlineto{\pgfqpoint{2.936549in}{3.993668in}}%
\pgfpathclose%
\pgfusepath{stroke,fill}%
\end{pgfscope}%
\begin{pgfscope}%
\pgfpathrectangle{\pgfqpoint{0.050000in}{0.050000in}}{\pgfqpoint{2.419000in}{2.419000in}}%
\pgfusepath{clip}%
\pgfsetbuttcap%
\pgfsetroundjoin%
\definecolor{currentfill}{rgb}{0.200000,0.133333,0.533333}%
\pgfsetfillcolor{currentfill}%
\pgfsetfillopacity{0.312059}%
\pgfsetlinewidth{1.003750pt}%
\definecolor{currentstroke}{rgb}{0.200000,0.133333,0.533333}%
\pgfsetstrokecolor{currentstroke}%
\pgfsetstrokeopacity{0.312059}%
\pgfsetdash{}{0pt}%
\pgfpathmoveto{\pgfqpoint{1.707306in}{3.940691in}}%
\pgfpathcurveto{\pgfqpoint{1.716515in}{3.940691in}}{\pgfqpoint{1.725347in}{3.944350in}}{\pgfqpoint{1.731859in}{3.950861in}}%
\pgfpathcurveto{\pgfqpoint{1.738370in}{3.957373in}}{\pgfqpoint{1.742029in}{3.966205in}}{\pgfqpoint{1.742029in}{3.975413in}}%
\pgfpathcurveto{\pgfqpoint{1.742029in}{3.984622in}}{\pgfqpoint{1.738370in}{3.993454in}}{\pgfqpoint{1.731859in}{3.999966in}}%
\pgfpathcurveto{\pgfqpoint{1.725347in}{4.006477in}}{\pgfqpoint{1.716515in}{4.010136in}}{\pgfqpoint{1.707306in}{4.010136in}}%
\pgfpathcurveto{\pgfqpoint{1.698098in}{4.010136in}}{\pgfqpoint{1.689265in}{4.006477in}}{\pgfqpoint{1.682754in}{3.999966in}}%
\pgfpathcurveto{\pgfqpoint{1.676243in}{3.993454in}}{\pgfqpoint{1.672584in}{3.984622in}}{\pgfqpoint{1.672584in}{3.975413in}}%
\pgfpathcurveto{\pgfqpoint{1.672584in}{3.966205in}}{\pgfqpoint{1.676243in}{3.957373in}}{\pgfqpoint{1.682754in}{3.950861in}}%
\pgfpathcurveto{\pgfqpoint{1.689265in}{3.944350in}}{\pgfqpoint{1.698098in}{3.940691in}}{\pgfqpoint{1.707306in}{3.940691in}}%
\pgfpathlineto{\pgfqpoint{1.707306in}{3.940691in}}%
\pgfpathclose%
\pgfusepath{stroke,fill}%
\end{pgfscope}%
\begin{pgfscope}%
\pgfpathrectangle{\pgfqpoint{0.050000in}{0.050000in}}{\pgfqpoint{2.419000in}{2.419000in}}%
\pgfusepath{clip}%
\pgfsetbuttcap%
\pgfsetroundjoin%
\definecolor{currentfill}{rgb}{0.200000,0.133333,0.533333}%
\pgfsetfillcolor{currentfill}%
\pgfsetfillopacity{0.314632}%
\pgfsetlinewidth{1.003750pt}%
\definecolor{currentstroke}{rgb}{0.200000,0.133333,0.533333}%
\pgfsetstrokecolor{currentstroke}%
\pgfsetstrokeopacity{0.314632}%
\pgfsetdash{}{0pt}%
\pgfpathmoveto{\pgfqpoint{3.555254in}{3.886535in}}%
\pgfpathcurveto{\pgfqpoint{3.564463in}{3.886535in}}{\pgfqpoint{3.573295in}{3.890193in}}{\pgfqpoint{3.579807in}{3.896705in}}%
\pgfpathcurveto{\pgfqpoint{3.586318in}{3.903216in}}{\pgfqpoint{3.589977in}{3.912049in}}{\pgfqpoint{3.589977in}{3.921257in}}%
\pgfpathcurveto{\pgfqpoint{3.589977in}{3.930465in}}{\pgfqpoint{3.586318in}{3.939298in}}{\pgfqpoint{3.579807in}{3.945809in}}%
\pgfpathcurveto{\pgfqpoint{3.573295in}{3.952321in}}{\pgfqpoint{3.564463in}{3.955979in}}{\pgfqpoint{3.555254in}{3.955979in}}%
\pgfpathcurveto{\pgfqpoint{3.546046in}{3.955979in}}{\pgfqpoint{3.537213in}{3.952321in}}{\pgfqpoint{3.530702in}{3.945809in}}%
\pgfpathcurveto{\pgfqpoint{3.524191in}{3.939298in}}{\pgfqpoint{3.520532in}{3.930465in}}{\pgfqpoint{3.520532in}{3.921257in}}%
\pgfpathcurveto{\pgfqpoint{3.520532in}{3.912049in}}{\pgfqpoint{3.524191in}{3.903216in}}{\pgfqpoint{3.530702in}{3.896705in}}%
\pgfpathcurveto{\pgfqpoint{3.537213in}{3.890193in}}{\pgfqpoint{3.546046in}{3.886535in}}{\pgfqpoint{3.555254in}{3.886535in}}%
\pgfpathlineto{\pgfqpoint{3.555254in}{3.886535in}}%
\pgfpathclose%
\pgfusepath{stroke,fill}%
\end{pgfscope}%
\begin{pgfscope}%
\pgfpathrectangle{\pgfqpoint{0.050000in}{0.050000in}}{\pgfqpoint{2.419000in}{2.419000in}}%
\pgfusepath{clip}%
\pgfsetbuttcap%
\pgfsetroundjoin%
\definecolor{currentfill}{rgb}{0.200000,0.133333,0.533333}%
\pgfsetfillcolor{currentfill}%
\pgfsetfillopacity{0.317263}%
\pgfsetlinewidth{1.003750pt}%
\definecolor{currentstroke}{rgb}{0.200000,0.133333,0.533333}%
\pgfsetstrokecolor{currentstroke}%
\pgfsetstrokeopacity{0.317263}%
\pgfsetdash{}{0pt}%
\pgfpathmoveto{\pgfqpoint{2.305305in}{3.831159in}}%
\pgfpathcurveto{\pgfqpoint{2.314514in}{3.831159in}}{\pgfqpoint{2.323346in}{3.834818in}}{\pgfqpoint{2.329858in}{3.841329in}}%
\pgfpathcurveto{\pgfqpoint{2.336369in}{3.847841in}}{\pgfqpoint{2.340028in}{3.856673in}}{\pgfqpoint{2.340028in}{3.865882in}}%
\pgfpathcurveto{\pgfqpoint{2.340028in}{3.875090in}}{\pgfqpoint{2.336369in}{3.883923in}}{\pgfqpoint{2.329858in}{3.890434in}}%
\pgfpathcurveto{\pgfqpoint{2.323346in}{3.896945in}}{\pgfqpoint{2.314514in}{3.900604in}}{\pgfqpoint{2.305305in}{3.900604in}}%
\pgfpathcurveto{\pgfqpoint{2.296097in}{3.900604in}}{\pgfqpoint{2.287265in}{3.896945in}}{\pgfqpoint{2.280753in}{3.890434in}}%
\pgfpathcurveto{\pgfqpoint{2.274242in}{3.883923in}}{\pgfqpoint{2.270583in}{3.875090in}}{\pgfqpoint{2.270583in}{3.865882in}}%
\pgfpathcurveto{\pgfqpoint{2.270583in}{3.856673in}}{\pgfqpoint{2.274242in}{3.847841in}}{\pgfqpoint{2.280753in}{3.841329in}}%
\pgfpathcurveto{\pgfqpoint{2.287265in}{3.834818in}}{\pgfqpoint{2.296097in}{3.831159in}}{\pgfqpoint{2.305305in}{3.831159in}}%
\pgfpathlineto{\pgfqpoint{2.305305in}{3.831159in}}%
\pgfpathclose%
\pgfusepath{stroke,fill}%
\end{pgfscope}%
\begin{pgfscope}%
\pgfpathrectangle{\pgfqpoint{0.050000in}{0.050000in}}{\pgfqpoint{2.419000in}{2.419000in}}%
\pgfusepath{clip}%
\pgfsetbuttcap%
\pgfsetroundjoin%
\definecolor{currentfill}{rgb}{0.200000,0.133333,0.533333}%
\pgfsetfillcolor{currentfill}%
\pgfsetfillopacity{0.319954}%
\pgfsetlinewidth{1.003750pt}%
\definecolor{currentstroke}{rgb}{0.200000,0.133333,0.533333}%
\pgfsetstrokecolor{currentstroke}%
\pgfsetstrokeopacity{0.319954}%
\pgfsetdash{}{0pt}%
\pgfpathmoveto{\pgfqpoint{4.202130in}{3.774523in}}%
\pgfpathcurveto{\pgfqpoint{4.211339in}{3.774523in}}{\pgfqpoint{4.220171in}{3.778182in}}{\pgfqpoint{4.226683in}{3.784693in}}%
\pgfpathcurveto{\pgfqpoint{4.233194in}{3.791204in}}{\pgfqpoint{4.236853in}{3.800037in}}{\pgfqpoint{4.236853in}{3.809245in}}%
\pgfpathcurveto{\pgfqpoint{4.236853in}{3.818454in}}{\pgfqpoint{4.233194in}{3.827286in}}{\pgfqpoint{4.226683in}{3.833798in}}%
\pgfpathcurveto{\pgfqpoint{4.220171in}{3.840309in}}{\pgfqpoint{4.211339in}{3.843968in}}{\pgfqpoint{4.202130in}{3.843968in}}%
\pgfpathcurveto{\pgfqpoint{4.192922in}{3.843968in}}{\pgfqpoint{4.184090in}{3.840309in}}{\pgfqpoint{4.177578in}{3.833798in}}%
\pgfpathcurveto{\pgfqpoint{4.171067in}{3.827286in}}{\pgfqpoint{4.167408in}{3.818454in}}{\pgfqpoint{4.167408in}{3.809245in}}%
\pgfpathcurveto{\pgfqpoint{4.167408in}{3.800037in}}{\pgfqpoint{4.171067in}{3.791204in}}{\pgfqpoint{4.177578in}{3.784693in}}%
\pgfpathcurveto{\pgfqpoint{4.184090in}{3.778182in}}{\pgfqpoint{4.192922in}{3.774523in}}{\pgfqpoint{4.202130in}{3.774523in}}%
\pgfpathlineto{\pgfqpoint{4.202130in}{3.774523in}}%
\pgfpathclose%
\pgfusepath{stroke,fill}%
\end{pgfscope}%
\begin{pgfscope}%
\pgfpathrectangle{\pgfqpoint{0.050000in}{0.050000in}}{\pgfqpoint{2.419000in}{2.419000in}}%
\pgfusepath{clip}%
\pgfsetbuttcap%
\pgfsetroundjoin%
\definecolor{currentfill}{rgb}{0.200000,0.133333,0.533333}%
\pgfsetfillcolor{currentfill}%
\pgfsetfillopacity{0.319954}%
\pgfsetlinewidth{1.003750pt}%
\definecolor{currentstroke}{rgb}{0.200000,0.133333,0.533333}%
\pgfsetstrokecolor{currentstroke}%
\pgfsetstrokeopacity{0.319954}%
\pgfsetdash{}{0pt}%
\pgfpathmoveto{\pgfqpoint{1.026900in}{3.774523in}}%
\pgfpathcurveto{\pgfqpoint{1.036109in}{3.774523in}}{\pgfqpoint{1.044941in}{3.778182in}}{\pgfqpoint{1.051453in}{3.784693in}}%
\pgfpathcurveto{\pgfqpoint{1.057964in}{3.791204in}}{\pgfqpoint{1.061623in}{3.800037in}}{\pgfqpoint{1.061623in}{3.809245in}}%
\pgfpathcurveto{\pgfqpoint{1.061623in}{3.818454in}}{\pgfqpoint{1.057964in}{3.827286in}}{\pgfqpoint{1.051453in}{3.833798in}}%
\pgfpathcurveto{\pgfqpoint{1.044941in}{3.840309in}}{\pgfqpoint{1.036109in}{3.843968in}}{\pgfqpoint{1.026900in}{3.843968in}}%
\pgfpathcurveto{\pgfqpoint{1.017692in}{3.843968in}}{\pgfqpoint{1.008859in}{3.840309in}}{\pgfqpoint{1.002348in}{3.833798in}}%
\pgfpathcurveto{\pgfqpoint{0.995837in}{3.827286in}}{\pgfqpoint{0.992178in}{3.818454in}}{\pgfqpoint{0.992178in}{3.809245in}}%
\pgfpathcurveto{\pgfqpoint{0.992178in}{3.800037in}}{\pgfqpoint{0.995837in}{3.791204in}}{\pgfqpoint{1.002348in}{3.784693in}}%
\pgfpathcurveto{\pgfqpoint{1.008859in}{3.778182in}}{\pgfqpoint{1.017692in}{3.774523in}}{\pgfqpoint{1.026900in}{3.774523in}}%
\pgfpathlineto{\pgfqpoint{1.026900in}{3.774523in}}%
\pgfpathclose%
\pgfusepath{stroke,fill}%
\end{pgfscope}%
\begin{pgfscope}%
\pgfpathrectangle{\pgfqpoint{0.050000in}{0.050000in}}{\pgfqpoint{2.419000in}{2.419000in}}%
\pgfusepath{clip}%
\pgfsetbuttcap%
\pgfsetroundjoin%
\definecolor{currentfill}{rgb}{0.200000,0.133333,0.533333}%
\pgfsetfillcolor{currentfill}%
\pgfsetfillopacity{0.322707}%
\pgfsetlinewidth{1.003750pt}%
\definecolor{currentstroke}{rgb}{0.200000,0.133333,0.533333}%
\pgfsetstrokecolor{currentstroke}%
\pgfsetstrokeopacity{0.322707}%
\pgfsetdash{}{0pt}%
\pgfpathmoveto{\pgfqpoint{2.930846in}{3.716583in}}%
\pgfpathcurveto{\pgfqpoint{2.940054in}{3.716583in}}{\pgfqpoint{2.948887in}{3.720241in}}{\pgfqpoint{2.955398in}{3.726753in}}%
\pgfpathcurveto{\pgfqpoint{2.961910in}{3.733264in}}{\pgfqpoint{2.965568in}{3.742097in}}{\pgfqpoint{2.965568in}{3.751305in}}%
\pgfpathcurveto{\pgfqpoint{2.965568in}{3.760514in}}{\pgfqpoint{2.961910in}{3.769346in}}{\pgfqpoint{2.955398in}{3.775857in}}%
\pgfpathcurveto{\pgfqpoint{2.948887in}{3.782369in}}{\pgfqpoint{2.940054in}{3.786027in}}{\pgfqpoint{2.930846in}{3.786027in}}%
\pgfpathcurveto{\pgfqpoint{2.921637in}{3.786027in}}{\pgfqpoint{2.912805in}{3.782369in}}{\pgfqpoint{2.906294in}{3.775857in}}%
\pgfpathcurveto{\pgfqpoint{2.899782in}{3.769346in}}{\pgfqpoint{2.896124in}{3.760514in}}{\pgfqpoint{2.896124in}{3.751305in}}%
\pgfpathcurveto{\pgfqpoint{2.896124in}{3.742097in}}{\pgfqpoint{2.899782in}{3.733264in}}{\pgfqpoint{2.906294in}{3.726753in}}%
\pgfpathcurveto{\pgfqpoint{2.912805in}{3.720241in}}{\pgfqpoint{2.921637in}{3.716583in}}{\pgfqpoint{2.930846in}{3.716583in}}%
\pgfpathlineto{\pgfqpoint{2.930846in}{3.716583in}}%
\pgfpathclose%
\pgfusepath{stroke,fill}%
\end{pgfscope}%
\begin{pgfscope}%
\pgfpathrectangle{\pgfqpoint{0.050000in}{0.050000in}}{\pgfqpoint{2.419000in}{2.419000in}}%
\pgfusepath{clip}%
\pgfsetbuttcap%
\pgfsetroundjoin%
\definecolor{currentfill}{rgb}{0.200000,0.133333,0.533333}%
\pgfsetfillcolor{currentfill}%
\pgfsetfillopacity{0.325523}%
\pgfsetlinewidth{1.003750pt}%
\definecolor{currentstroke}{rgb}{0.200000,0.133333,0.533333}%
\pgfsetstrokecolor{currentstroke}%
\pgfsetstrokeopacity{0.325523}%
\pgfsetdash{}{0pt}%
\pgfpathmoveto{\pgfqpoint{1.629945in}{3.657293in}}%
\pgfpathcurveto{\pgfqpoint{1.639154in}{3.657293in}}{\pgfqpoint{1.647986in}{3.660951in}}{\pgfqpoint{1.654498in}{3.667463in}}%
\pgfpathcurveto{\pgfqpoint{1.661009in}{3.673974in}}{\pgfqpoint{1.664668in}{3.682807in}}{\pgfqpoint{1.664668in}{3.692015in}}%
\pgfpathcurveto{\pgfqpoint{1.664668in}{3.701223in}}{\pgfqpoint{1.661009in}{3.710056in}}{\pgfqpoint{1.654498in}{3.716567in}}%
\pgfpathcurveto{\pgfqpoint{1.647986in}{3.723079in}}{\pgfqpoint{1.639154in}{3.726737in}}{\pgfqpoint{1.629945in}{3.726737in}}%
\pgfpathcurveto{\pgfqpoint{1.620737in}{3.726737in}}{\pgfqpoint{1.611904in}{3.723079in}}{\pgfqpoint{1.605393in}{3.716567in}}%
\pgfpathcurveto{\pgfqpoint{1.598882in}{3.710056in}}{\pgfqpoint{1.595223in}{3.701223in}}{\pgfqpoint{1.595223in}{3.692015in}}%
\pgfpathcurveto{\pgfqpoint{1.595223in}{3.682807in}}{\pgfqpoint{1.598882in}{3.673974in}}{\pgfqpoint{1.605393in}{3.667463in}}%
\pgfpathcurveto{\pgfqpoint{1.611904in}{3.660951in}}{\pgfqpoint{1.620737in}{3.657293in}}{\pgfqpoint{1.629945in}{3.657293in}}%
\pgfpathlineto{\pgfqpoint{1.629945in}{3.657293in}}%
\pgfpathclose%
\pgfusepath{stroke,fill}%
\end{pgfscope}%
\begin{pgfscope}%
\pgfpathrectangle{\pgfqpoint{0.050000in}{0.050000in}}{\pgfqpoint{2.419000in}{2.419000in}}%
\pgfusepath{clip}%
\pgfsetbuttcap%
\pgfsetroundjoin%
\definecolor{currentfill}{rgb}{0.200000,0.133333,0.533333}%
\pgfsetfillcolor{currentfill}%
\pgfsetfillopacity{0.325523}%
\pgfsetlinewidth{1.003750pt}%
\definecolor{currentstroke}{rgb}{0.200000,0.133333,0.533333}%
\pgfsetstrokecolor{currentstroke}%
\pgfsetstrokeopacity{0.325523}%
\pgfsetdash{}{0pt}%
\pgfpathmoveto{\pgfqpoint{4.879146in}{3.657293in}}%
\pgfpathcurveto{\pgfqpoint{4.888354in}{3.657293in}}{\pgfqpoint{4.897187in}{3.660951in}}{\pgfqpoint{4.903698in}{3.667463in}}%
\pgfpathcurveto{\pgfqpoint{4.910210in}{3.673974in}}{\pgfqpoint{4.913868in}{3.682807in}}{\pgfqpoint{4.913868in}{3.692015in}}%
\pgfpathcurveto{\pgfqpoint{4.913868in}{3.701223in}}{\pgfqpoint{4.910210in}{3.710056in}}{\pgfqpoint{4.903698in}{3.716567in}}%
\pgfpathcurveto{\pgfqpoint{4.897187in}{3.723079in}}{\pgfqpoint{4.888354in}{3.726737in}}{\pgfqpoint{4.879146in}{3.726737in}}%
\pgfpathcurveto{\pgfqpoint{4.869937in}{3.726737in}}{\pgfqpoint{4.861105in}{3.723079in}}{\pgfqpoint{4.854594in}{3.716567in}}%
\pgfpathcurveto{\pgfqpoint{4.848082in}{3.710056in}}{\pgfqpoint{4.844424in}{3.701223in}}{\pgfqpoint{4.844424in}{3.692015in}}%
\pgfpathcurveto{\pgfqpoint{4.844424in}{3.682807in}}{\pgfqpoint{4.848082in}{3.673974in}}{\pgfqpoint{4.854594in}{3.667463in}}%
\pgfpathcurveto{\pgfqpoint{4.861105in}{3.660951in}}{\pgfqpoint{4.869937in}{3.657293in}}{\pgfqpoint{4.879146in}{3.657293in}}%
\pgfpathlineto{\pgfqpoint{4.879146in}{3.657293in}}%
\pgfpathclose%
\pgfusepath{stroke,fill}%
\end{pgfscope}%
\begin{pgfscope}%
\pgfpathrectangle{\pgfqpoint{0.050000in}{0.050000in}}{\pgfqpoint{2.419000in}{2.419000in}}%
\pgfusepath{clip}%
\pgfsetbuttcap%
\pgfsetroundjoin%
\definecolor{currentfill}{rgb}{0.200000,0.133333,0.533333}%
\pgfsetfillcolor{currentfill}%
\pgfsetfillopacity{0.328407}%
\pgfsetlinewidth{1.003750pt}%
\definecolor{currentstroke}{rgb}{0.200000,0.133333,0.533333}%
\pgfsetstrokecolor{currentstroke}%
\pgfsetstrokeopacity{0.328407}%
\pgfsetdash{}{0pt}%
\pgfpathmoveto{\pgfqpoint{3.585875in}{3.596605in}}%
\pgfpathcurveto{\pgfqpoint{3.595084in}{3.596605in}}{\pgfqpoint{3.603916in}{3.600264in}}{\pgfqpoint{3.610428in}{3.606775in}}%
\pgfpathcurveto{\pgfqpoint{3.616939in}{3.613286in}}{\pgfqpoint{3.620597in}{3.622119in}}{\pgfqpoint{3.620597in}{3.631327in}}%
\pgfpathcurveto{\pgfqpoint{3.620597in}{3.640536in}}{\pgfqpoint{3.616939in}{3.649368in}}{\pgfqpoint{3.610428in}{3.655880in}}%
\pgfpathcurveto{\pgfqpoint{3.603916in}{3.662391in}}{\pgfqpoint{3.595084in}{3.666050in}}{\pgfqpoint{3.585875in}{3.666050in}}%
\pgfpathcurveto{\pgfqpoint{3.576667in}{3.666050in}}{\pgfqpoint{3.567834in}{3.662391in}}{\pgfqpoint{3.561323in}{3.655880in}}%
\pgfpathcurveto{\pgfqpoint{3.554812in}{3.649368in}}{\pgfqpoint{3.551153in}{3.640536in}}{\pgfqpoint{3.551153in}{3.631327in}}%
\pgfpathcurveto{\pgfqpoint{3.551153in}{3.622119in}}{\pgfqpoint{3.554812in}{3.613286in}}{\pgfqpoint{3.561323in}{3.606775in}}%
\pgfpathcurveto{\pgfqpoint{3.567834in}{3.600264in}}{\pgfqpoint{3.576667in}{3.596605in}}{\pgfqpoint{3.585875in}{3.596605in}}%
\pgfpathlineto{\pgfqpoint{3.585875in}{3.596605in}}%
\pgfpathclose%
\pgfusepath{stroke,fill}%
\end{pgfscope}%
\begin{pgfscope}%
\pgfpathrectangle{\pgfqpoint{0.050000in}{0.050000in}}{\pgfqpoint{2.419000in}{2.419000in}}%
\pgfusepath{clip}%
\pgfsetbuttcap%
\pgfsetroundjoin%
\definecolor{currentfill}{rgb}{0.200000,0.133333,0.533333}%
\pgfsetfillcolor{currentfill}%
\pgfsetfillopacity{0.331359}%
\pgfsetlinewidth{1.003750pt}%
\definecolor{currentstroke}{rgb}{0.200000,0.133333,0.533333}%
\pgfsetstrokecolor{currentstroke}%
\pgfsetstrokeopacity{0.331359}%
\pgfsetdash{}{0pt}%
\pgfpathmoveto{\pgfqpoint{2.261758in}{3.534470in}}%
\pgfpathcurveto{\pgfqpoint{2.270966in}{3.534470in}}{\pgfqpoint{2.279799in}{3.538128in}}{\pgfqpoint{2.286310in}{3.544640in}}%
\pgfpathcurveto{\pgfqpoint{2.292821in}{3.551151in}}{\pgfqpoint{2.296480in}{3.559984in}}{\pgfqpoint{2.296480in}{3.569192in}}%
\pgfpathcurveto{\pgfqpoint{2.296480in}{3.578401in}}{\pgfqpoint{2.292821in}{3.587233in}}{\pgfqpoint{2.286310in}{3.593744in}}%
\pgfpathcurveto{\pgfqpoint{2.279799in}{3.600256in}}{\pgfqpoint{2.270966in}{3.603914in}}{\pgfqpoint{2.261758in}{3.603914in}}%
\pgfpathcurveto{\pgfqpoint{2.252549in}{3.603914in}}{\pgfqpoint{2.243717in}{3.600256in}}{\pgfqpoint{2.237205in}{3.593744in}}%
\pgfpathcurveto{\pgfqpoint{2.230694in}{3.587233in}}{\pgfqpoint{2.227035in}{3.578401in}}{\pgfqpoint{2.227035in}{3.569192in}}%
\pgfpathcurveto{\pgfqpoint{2.227035in}{3.559984in}}{\pgfqpoint{2.230694in}{3.551151in}}{\pgfqpoint{2.237205in}{3.544640in}}%
\pgfpathcurveto{\pgfqpoint{2.243717in}{3.538128in}}{\pgfqpoint{2.252549in}{3.534470in}}{\pgfqpoint{2.261758in}{3.534470in}}%
\pgfpathlineto{\pgfqpoint{2.261758in}{3.534470in}}%
\pgfpathclose%
\pgfusepath{stroke,fill}%
\end{pgfscope}%
\begin{pgfscope}%
\pgfpathrectangle{\pgfqpoint{0.050000in}{0.050000in}}{\pgfqpoint{2.419000in}{2.419000in}}%
\pgfusepath{clip}%
\pgfsetbuttcap%
\pgfsetroundjoin%
\definecolor{currentfill}{rgb}{0.200000,0.133333,0.533333}%
\pgfsetfillcolor{currentfill}%
\pgfsetfillopacity{0.331359}%
\pgfsetlinewidth{1.003750pt}%
\definecolor{currentstroke}{rgb}{0.200000,0.133333,0.533333}%
\pgfsetstrokecolor{currentstroke}%
\pgfsetstrokeopacity{0.331359}%
\pgfsetdash{}{0pt}%
\pgfpathmoveto{\pgfqpoint{5.588457in}{3.534470in}}%
\pgfpathcurveto{\pgfqpoint{5.597666in}{3.534470in}}{\pgfqpoint{5.606498in}{3.538128in}}{\pgfqpoint{5.613010in}{3.544640in}}%
\pgfpathcurveto{\pgfqpoint{5.619521in}{3.551151in}}{\pgfqpoint{5.623180in}{3.559984in}}{\pgfqpoint{5.623180in}{3.569192in}}%
\pgfpathcurveto{\pgfqpoint{5.623180in}{3.578401in}}{\pgfqpoint{5.619521in}{3.587233in}}{\pgfqpoint{5.613010in}{3.593744in}}%
\pgfpathcurveto{\pgfqpoint{5.606498in}{3.600256in}}{\pgfqpoint{5.597666in}{3.603914in}}{\pgfqpoint{5.588457in}{3.603914in}}%
\pgfpathcurveto{\pgfqpoint{5.579249in}{3.603914in}}{\pgfqpoint{5.570416in}{3.600256in}}{\pgfqpoint{5.563905in}{3.593744in}}%
\pgfpathcurveto{\pgfqpoint{5.557394in}{3.587233in}}{\pgfqpoint{5.553735in}{3.578401in}}{\pgfqpoint{5.553735in}{3.569192in}}%
\pgfpathcurveto{\pgfqpoint{5.553735in}{3.559984in}}{\pgfqpoint{5.557394in}{3.551151in}}{\pgfqpoint{5.563905in}{3.544640in}}%
\pgfpathcurveto{\pgfqpoint{5.570416in}{3.538128in}}{\pgfqpoint{5.579249in}{3.534470in}}{\pgfqpoint{5.588457in}{3.534470in}}%
\pgfpathlineto{\pgfqpoint{5.588457in}{3.534470in}}%
\pgfpathclose%
\pgfusepath{stroke,fill}%
\end{pgfscope}%
\begin{pgfscope}%
\pgfpathrectangle{\pgfqpoint{0.050000in}{0.050000in}}{\pgfqpoint{2.419000in}{2.419000in}}%
\pgfusepath{clip}%
\pgfsetbuttcap%
\pgfsetroundjoin%
\definecolor{currentfill}{rgb}{0.200000,0.133333,0.533333}%
\pgfsetfillcolor{currentfill}%
\pgfsetfillopacity{0.334382}%
\pgfsetlinewidth{1.003750pt}%
\definecolor{currentstroke}{rgb}{0.200000,0.133333,0.533333}%
\pgfsetstrokecolor{currentstroke}%
\pgfsetstrokeopacity{0.334382}%
\pgfsetdash{}{0pt}%
\pgfpathmoveto{\pgfqpoint{0.905676in}{3.470835in}}%
\pgfpathcurveto{\pgfqpoint{0.914885in}{3.470835in}}{\pgfqpoint{0.923717in}{3.474493in}}{\pgfqpoint{0.930229in}{3.481005in}}%
\pgfpathcurveto{\pgfqpoint{0.936740in}{3.487516in}}{\pgfqpoint{0.940399in}{3.496349in}}{\pgfqpoint{0.940399in}{3.505557in}}%
\pgfpathcurveto{\pgfqpoint{0.940399in}{3.514766in}}{\pgfqpoint{0.936740in}{3.523598in}}{\pgfqpoint{0.930229in}{3.530109in}}%
\pgfpathcurveto{\pgfqpoint{0.923717in}{3.536621in}}{\pgfqpoint{0.914885in}{3.540279in}}{\pgfqpoint{0.905676in}{3.540279in}}%
\pgfpathcurveto{\pgfqpoint{0.896468in}{3.540279in}}{\pgfqpoint{0.887635in}{3.536621in}}{\pgfqpoint{0.881124in}{3.530109in}}%
\pgfpathcurveto{\pgfqpoint{0.874613in}{3.523598in}}{\pgfqpoint{0.870954in}{3.514766in}}{\pgfqpoint{0.870954in}{3.505557in}}%
\pgfpathcurveto{\pgfqpoint{0.870954in}{3.496349in}}{\pgfqpoint{0.874613in}{3.487516in}}{\pgfqpoint{0.881124in}{3.481005in}}%
\pgfpathcurveto{\pgfqpoint{0.887635in}{3.474493in}}{\pgfqpoint{0.896468in}{3.470835in}}{\pgfqpoint{0.905676in}{3.470835in}}%
\pgfpathlineto{\pgfqpoint{0.905676in}{3.470835in}}%
\pgfpathclose%
\pgfusepath{stroke,fill}%
\end{pgfscope}%
\begin{pgfscope}%
\pgfpathrectangle{\pgfqpoint{0.050000in}{0.050000in}}{\pgfqpoint{2.419000in}{2.419000in}}%
\pgfusepath{clip}%
\pgfsetbuttcap%
\pgfsetroundjoin%
\definecolor{currentfill}{rgb}{0.200000,0.133333,0.533333}%
\pgfsetfillcolor{currentfill}%
\pgfsetfillopacity{0.334382}%
\pgfsetlinewidth{1.003750pt}%
\definecolor{currentstroke}{rgb}{0.200000,0.133333,0.533333}%
\pgfsetstrokecolor{currentstroke}%
\pgfsetstrokeopacity{0.334382}%
\pgfsetdash{}{0pt}%
\pgfpathmoveto{\pgfqpoint{4.272529in}{3.470835in}}%
\pgfpathcurveto{\pgfqpoint{4.281737in}{3.470835in}}{\pgfqpoint{4.290570in}{3.474493in}}{\pgfqpoint{4.297081in}{3.481005in}}%
\pgfpathcurveto{\pgfqpoint{4.303593in}{3.487516in}}{\pgfqpoint{4.307251in}{3.496349in}}{\pgfqpoint{4.307251in}{3.505557in}}%
\pgfpathcurveto{\pgfqpoint{4.307251in}{3.514766in}}{\pgfqpoint{4.303593in}{3.523598in}}{\pgfqpoint{4.297081in}{3.530109in}}%
\pgfpathcurveto{\pgfqpoint{4.290570in}{3.536621in}}{\pgfqpoint{4.281737in}{3.540279in}}{\pgfqpoint{4.272529in}{3.540279in}}%
\pgfpathcurveto{\pgfqpoint{4.263320in}{3.540279in}}{\pgfqpoint{4.254488in}{3.536621in}}{\pgfqpoint{4.247977in}{3.530109in}}%
\pgfpathcurveto{\pgfqpoint{4.241465in}{3.523598in}}{\pgfqpoint{4.237807in}{3.514766in}}{\pgfqpoint{4.237807in}{3.505557in}}%
\pgfpathcurveto{\pgfqpoint{4.237807in}{3.496349in}}{\pgfqpoint{4.241465in}{3.487516in}}{\pgfqpoint{4.247977in}{3.481005in}}%
\pgfpathcurveto{\pgfqpoint{4.254488in}{3.474493in}}{\pgfqpoint{4.263320in}{3.470835in}}{\pgfqpoint{4.272529in}{3.470835in}}%
\pgfpathlineto{\pgfqpoint{4.272529in}{3.470835in}}%
\pgfpathclose%
\pgfusepath{stroke,fill}%
\end{pgfscope}%
\begin{pgfscope}%
\pgfpathrectangle{\pgfqpoint{0.050000in}{0.050000in}}{\pgfqpoint{2.419000in}{2.419000in}}%
\pgfusepath{clip}%
\pgfsetbuttcap%
\pgfsetroundjoin%
\definecolor{currentfill}{rgb}{0.200000,0.133333,0.533333}%
\pgfsetfillcolor{currentfill}%
\pgfsetfillopacity{0.337479}%
\pgfsetlinewidth{1.003750pt}%
\definecolor{currentstroke}{rgb}{0.200000,0.133333,0.533333}%
\pgfsetstrokecolor{currentstroke}%
\pgfsetstrokeopacity{0.337479}%
\pgfsetdash{}{0pt}%
\pgfpathmoveto{\pgfqpoint{2.924446in}{3.405645in}}%
\pgfpathcurveto{\pgfqpoint{2.933655in}{3.405645in}}{\pgfqpoint{2.942487in}{3.409303in}}{\pgfqpoint{2.948998in}{3.415815in}}%
\pgfpathcurveto{\pgfqpoint{2.955510in}{3.422326in}}{\pgfqpoint{2.959168in}{3.431159in}}{\pgfqpoint{2.959168in}{3.440367in}}%
\pgfpathcurveto{\pgfqpoint{2.959168in}{3.449576in}}{\pgfqpoint{2.955510in}{3.458408in}}{\pgfqpoint{2.948998in}{3.464919in}}%
\pgfpathcurveto{\pgfqpoint{2.942487in}{3.471431in}}{\pgfqpoint{2.933655in}{3.475089in}}{\pgfqpoint{2.924446in}{3.475089in}}%
\pgfpathcurveto{\pgfqpoint{2.915238in}{3.475089in}}{\pgfqpoint{2.906405in}{3.471431in}}{\pgfqpoint{2.899894in}{3.464919in}}%
\pgfpathcurveto{\pgfqpoint{2.893382in}{3.458408in}}{\pgfqpoint{2.889724in}{3.449576in}}{\pgfqpoint{2.889724in}{3.440367in}}%
\pgfpathcurveto{\pgfqpoint{2.889724in}{3.431159in}}{\pgfqpoint{2.893382in}{3.422326in}}{\pgfqpoint{2.899894in}{3.415815in}}%
\pgfpathcurveto{\pgfqpoint{2.906405in}{3.409303in}}{\pgfqpoint{2.915238in}{3.405645in}}{\pgfqpoint{2.924446in}{3.405645in}}%
\pgfpathlineto{\pgfqpoint{2.924446in}{3.405645in}}%
\pgfpathclose%
\pgfusepath{stroke,fill}%
\end{pgfscope}%
\begin{pgfscope}%
\pgfpathrectangle{\pgfqpoint{0.050000in}{0.050000in}}{\pgfqpoint{2.419000in}{2.419000in}}%
\pgfusepath{clip}%
\pgfsetbuttcap%
\pgfsetroundjoin%
\definecolor{currentfill}{rgb}{0.200000,0.133333,0.533333}%
\pgfsetfillcolor{currentfill}%
\pgfsetfillopacity{0.337479}%
\pgfsetlinewidth{1.003750pt}%
\definecolor{currentstroke}{rgb}{0.200000,0.133333,0.533333}%
\pgfsetstrokecolor{currentstroke}%
\pgfsetstrokeopacity{0.337479}%
\pgfsetdash{}{0pt}%
\pgfpathmoveto{\pgfqpoint{6.332432in}{3.405645in}}%
\pgfpathcurveto{\pgfqpoint{6.341641in}{3.405645in}}{\pgfqpoint{6.350473in}{3.409303in}}{\pgfqpoint{6.356985in}{3.415815in}}%
\pgfpathcurveto{\pgfqpoint{6.363496in}{3.422326in}}{\pgfqpoint{6.367155in}{3.431159in}}{\pgfqpoint{6.367155in}{3.440367in}}%
\pgfpathcurveto{\pgfqpoint{6.367155in}{3.449576in}}{\pgfqpoint{6.363496in}{3.458408in}}{\pgfqpoint{6.356985in}{3.464919in}}%
\pgfpathcurveto{\pgfqpoint{6.350473in}{3.471431in}}{\pgfqpoint{6.341641in}{3.475089in}}{\pgfqpoint{6.332432in}{3.475089in}}%
\pgfpathcurveto{\pgfqpoint{6.323224in}{3.475089in}}{\pgfqpoint{6.314391in}{3.471431in}}{\pgfqpoint{6.307880in}{3.464919in}}%
\pgfpathcurveto{\pgfqpoint{6.301369in}{3.458408in}}{\pgfqpoint{6.297710in}{3.449576in}}{\pgfqpoint{6.297710in}{3.440367in}}%
\pgfpathcurveto{\pgfqpoint{6.297710in}{3.431159in}}{\pgfqpoint{6.301369in}{3.422326in}}{\pgfqpoint{6.307880in}{3.415815in}}%
\pgfpathcurveto{\pgfqpoint{6.314391in}{3.409303in}}{\pgfqpoint{6.323224in}{3.405645in}}{\pgfqpoint{6.332432in}{3.405645in}}%
\pgfpathlineto{\pgfqpoint{6.332432in}{3.405645in}}%
\pgfpathclose%
\pgfusepath{stroke,fill}%
\end{pgfscope}%
\begin{pgfscope}%
\pgfpathrectangle{\pgfqpoint{0.050000in}{0.050000in}}{\pgfqpoint{2.419000in}{2.419000in}}%
\pgfusepath{clip}%
\pgfsetbuttcap%
\pgfsetroundjoin%
\definecolor{currentfill}{rgb}{0.200000,0.133333,0.533333}%
\pgfsetfillcolor{currentfill}%
\pgfsetfillopacity{0.340653}%
\pgfsetlinewidth{1.003750pt}%
\definecolor{currentstroke}{rgb}{0.200000,0.133333,0.533333}%
\pgfsetstrokecolor{currentstroke}%
\pgfsetstrokeopacity{0.340653}%
\pgfsetdash{}{0pt}%
\pgfpathmoveto{\pgfqpoint{4.993154in}{3.338842in}}%
\pgfpathcurveto{\pgfqpoint{5.002362in}{3.338842in}}{\pgfqpoint{5.011195in}{3.342501in}}{\pgfqpoint{5.017706in}{3.349012in}}%
\pgfpathcurveto{\pgfqpoint{5.024217in}{3.355524in}}{\pgfqpoint{5.027876in}{3.364356in}}{\pgfqpoint{5.027876in}{3.373565in}}%
\pgfpathcurveto{\pgfqpoint{5.027876in}{3.382773in}}{\pgfqpoint{5.024217in}{3.391606in}}{\pgfqpoint{5.017706in}{3.398117in}}%
\pgfpathcurveto{\pgfqpoint{5.011195in}{3.404628in}}{\pgfqpoint{5.002362in}{3.408287in}}{\pgfqpoint{4.993154in}{3.408287in}}%
\pgfpathcurveto{\pgfqpoint{4.983945in}{3.408287in}}{\pgfqpoint{4.975113in}{3.404628in}}{\pgfqpoint{4.968601in}{3.398117in}}%
\pgfpathcurveto{\pgfqpoint{4.962090in}{3.391606in}}{\pgfqpoint{4.958432in}{3.382773in}}{\pgfqpoint{4.958432in}{3.373565in}}%
\pgfpathcurveto{\pgfqpoint{4.958432in}{3.364356in}}{\pgfqpoint{4.962090in}{3.355524in}}{\pgfqpoint{4.968601in}{3.349012in}}%
\pgfpathcurveto{\pgfqpoint{4.975113in}{3.342501in}}{\pgfqpoint{4.983945in}{3.338842in}}{\pgfqpoint{4.993154in}{3.338842in}}%
\pgfpathlineto{\pgfqpoint{4.993154in}{3.338842in}}%
\pgfpathclose%
\pgfusepath{stroke,fill}%
\end{pgfscope}%
\begin{pgfscope}%
\pgfpathrectangle{\pgfqpoint{0.050000in}{0.050000in}}{\pgfqpoint{2.419000in}{2.419000in}}%
\pgfusepath{clip}%
\pgfsetbuttcap%
\pgfsetroundjoin%
\definecolor{currentfill}{rgb}{0.200000,0.133333,0.533333}%
\pgfsetfillcolor{currentfill}%
\pgfsetfillopacity{0.340653}%
\pgfsetlinewidth{1.003750pt}%
\definecolor{currentstroke}{rgb}{0.200000,0.133333,0.533333}%
\pgfsetstrokecolor{currentstroke}%
\pgfsetstrokeopacity{0.340653}%
\pgfsetdash{}{0pt}%
\pgfpathmoveto{\pgfqpoint{1.543016in}{3.338842in}}%
\pgfpathcurveto{\pgfqpoint{1.552225in}{3.338842in}}{\pgfqpoint{1.561057in}{3.342501in}}{\pgfqpoint{1.567568in}{3.349012in}}%
\pgfpathcurveto{\pgfqpoint{1.574080in}{3.355524in}}{\pgfqpoint{1.577738in}{3.364356in}}{\pgfqpoint{1.577738in}{3.373565in}}%
\pgfpathcurveto{\pgfqpoint{1.577738in}{3.382773in}}{\pgfqpoint{1.574080in}{3.391606in}}{\pgfqpoint{1.567568in}{3.398117in}}%
\pgfpathcurveto{\pgfqpoint{1.561057in}{3.404628in}}{\pgfqpoint{1.552225in}{3.408287in}}{\pgfqpoint{1.543016in}{3.408287in}}%
\pgfpathcurveto{\pgfqpoint{1.533808in}{3.408287in}}{\pgfqpoint{1.524975in}{3.404628in}}{\pgfqpoint{1.518464in}{3.398117in}}%
\pgfpathcurveto{\pgfqpoint{1.511952in}{3.391606in}}{\pgfqpoint{1.508294in}{3.382773in}}{\pgfqpoint{1.508294in}{3.373565in}}%
\pgfpathcurveto{\pgfqpoint{1.508294in}{3.364356in}}{\pgfqpoint{1.511952in}{3.355524in}}{\pgfqpoint{1.518464in}{3.349012in}}%
\pgfpathcurveto{\pgfqpoint{1.524975in}{3.342501in}}{\pgfqpoint{1.533808in}{3.338842in}}{\pgfqpoint{1.543016in}{3.338842in}}%
\pgfpathlineto{\pgfqpoint{1.543016in}{3.338842in}}%
\pgfpathclose%
\pgfusepath{stroke,fill}%
\end{pgfscope}%
\begin{pgfscope}%
\pgfpathrectangle{\pgfqpoint{0.050000in}{0.050000in}}{\pgfqpoint{2.419000in}{2.419000in}}%
\pgfusepath{clip}%
\pgfsetbuttcap%
\pgfsetroundjoin%
\definecolor{currentfill}{rgb}{0.200000,0.133333,0.533333}%
\pgfsetfillcolor{currentfill}%
\pgfsetfillopacity{0.343906}%
\pgfsetlinewidth{1.003750pt}%
\definecolor{currentstroke}{rgb}{0.200000,0.133333,0.533333}%
\pgfsetstrokecolor{currentstroke}%
\pgfsetstrokeopacity{0.343906}%
\pgfsetdash{}{0pt}%
\pgfpathmoveto{\pgfqpoint{0.126986in}{3.270367in}}%
\pgfpathcurveto{\pgfqpoint{0.136194in}{3.270367in}}{\pgfqpoint{0.145027in}{3.274025in}}{\pgfqpoint{0.151538in}{3.280536in}}%
\pgfpathcurveto{\pgfqpoint{0.158050in}{3.287048in}}{\pgfqpoint{0.161708in}{3.295880in}}{\pgfqpoint{0.161708in}{3.305089in}}%
\pgfpathcurveto{\pgfqpoint{0.161708in}{3.314297in}}{\pgfqpoint{0.158050in}{3.323130in}}{\pgfqpoint{0.151538in}{3.329641in}}%
\pgfpathcurveto{\pgfqpoint{0.145027in}{3.336152in}}{\pgfqpoint{0.136194in}{3.339811in}}{\pgfqpoint{0.126986in}{3.339811in}}%
\pgfpathcurveto{\pgfqpoint{0.117778in}{3.339811in}}{\pgfqpoint{0.108945in}{3.336152in}}{\pgfqpoint{0.102434in}{3.329641in}}%
\pgfpathcurveto{\pgfqpoint{0.095922in}{3.323130in}}{\pgfqpoint{0.092264in}{3.314297in}}{\pgfqpoint{0.092264in}{3.305089in}}%
\pgfpathcurveto{\pgfqpoint{0.092264in}{3.295880in}}{\pgfqpoint{0.095922in}{3.287048in}}{\pgfqpoint{0.102434in}{3.280536in}}%
\pgfpathcurveto{\pgfqpoint{0.108945in}{3.274025in}}{\pgfqpoint{0.117778in}{3.270367in}}{\pgfqpoint{0.126986in}{3.270367in}}%
\pgfpathlineto{\pgfqpoint{0.126986in}{3.270367in}}%
\pgfpathclose%
\pgfusepath{stroke,fill}%
\end{pgfscope}%
\begin{pgfscope}%
\pgfpathrectangle{\pgfqpoint{0.050000in}{0.050000in}}{\pgfqpoint{2.419000in}{2.419000in}}%
\pgfusepath{clip}%
\pgfsetbuttcap%
\pgfsetroundjoin%
\definecolor{currentfill}{rgb}{0.200000,0.133333,0.533333}%
\pgfsetfillcolor{currentfill}%
\pgfsetfillopacity{0.343906}%
\pgfsetlinewidth{1.003750pt}%
\definecolor{currentstroke}{rgb}{0.200000,0.133333,0.533333}%
\pgfsetstrokecolor{currentstroke}%
\pgfsetstrokeopacity{0.343906}%
\pgfsetdash{}{0pt}%
\pgfpathmoveto{\pgfqpoint{3.620331in}{3.270367in}}%
\pgfpathcurveto{\pgfqpoint{3.629539in}{3.270367in}}{\pgfqpoint{3.638372in}{3.274025in}}{\pgfqpoint{3.644883in}{3.280536in}}%
\pgfpathcurveto{\pgfqpoint{3.651394in}{3.287048in}}{\pgfqpoint{3.655053in}{3.295880in}}{\pgfqpoint{3.655053in}{3.305089in}}%
\pgfpathcurveto{\pgfqpoint{3.655053in}{3.314297in}}{\pgfqpoint{3.651394in}{3.323130in}}{\pgfqpoint{3.644883in}{3.329641in}}%
\pgfpathcurveto{\pgfqpoint{3.638372in}{3.336152in}}{\pgfqpoint{3.629539in}{3.339811in}}{\pgfqpoint{3.620331in}{3.339811in}}%
\pgfpathcurveto{\pgfqpoint{3.611122in}{3.339811in}}{\pgfqpoint{3.602290in}{3.336152in}}{\pgfqpoint{3.595778in}{3.329641in}}%
\pgfpathcurveto{\pgfqpoint{3.589267in}{3.323130in}}{\pgfqpoint{3.585608in}{3.314297in}}{\pgfqpoint{3.585608in}{3.305089in}}%
\pgfpathcurveto{\pgfqpoint{3.585608in}{3.295880in}}{\pgfqpoint{3.589267in}{3.287048in}}{\pgfqpoint{3.595778in}{3.280536in}}%
\pgfpathcurveto{\pgfqpoint{3.602290in}{3.274025in}}{\pgfqpoint{3.611122in}{3.270367in}}{\pgfqpoint{3.620331in}{3.270367in}}%
\pgfpathlineto{\pgfqpoint{3.620331in}{3.270367in}}%
\pgfpathclose%
\pgfusepath{stroke,fill}%
\end{pgfscope}%
\begin{pgfscope}%
\pgfpathrectangle{\pgfqpoint{0.050000in}{0.050000in}}{\pgfqpoint{2.419000in}{2.419000in}}%
\pgfusepath{clip}%
\pgfsetbuttcap%
\pgfsetroundjoin%
\definecolor{currentfill}{rgb}{0.200000,0.133333,0.533333}%
\pgfsetfillcolor{currentfill}%
\pgfsetfillopacity{0.343906}%
\pgfsetlinewidth{1.003750pt}%
\definecolor{currentstroke}{rgb}{0.200000,0.133333,0.533333}%
\pgfsetstrokecolor{currentstroke}%
\pgfsetstrokeopacity{0.343906}%
\pgfsetdash{}{0pt}%
\pgfpathmoveto{\pgfqpoint{7.113675in}{3.270367in}}%
\pgfpathcurveto{\pgfqpoint{7.122884in}{3.270367in}}{\pgfqpoint{7.131716in}{3.274025in}}{\pgfqpoint{7.138228in}{3.280536in}}%
\pgfpathcurveto{\pgfqpoint{7.144739in}{3.287048in}}{\pgfqpoint{7.148398in}{3.295880in}}{\pgfqpoint{7.148398in}{3.305089in}}%
\pgfpathcurveto{\pgfqpoint{7.148398in}{3.314297in}}{\pgfqpoint{7.144739in}{3.323130in}}{\pgfqpoint{7.138228in}{3.329641in}}%
\pgfpathcurveto{\pgfqpoint{7.131716in}{3.336152in}}{\pgfqpoint{7.122884in}{3.339811in}}{\pgfqpoint{7.113675in}{3.339811in}}%
\pgfpathcurveto{\pgfqpoint{7.104467in}{3.339811in}}{\pgfqpoint{7.095634in}{3.336152in}}{\pgfqpoint{7.089123in}{3.329641in}}%
\pgfpathcurveto{\pgfqpoint{7.082612in}{3.323130in}}{\pgfqpoint{7.078953in}{3.314297in}}{\pgfqpoint{7.078953in}{3.305089in}}%
\pgfpathcurveto{\pgfqpoint{7.078953in}{3.295880in}}{\pgfqpoint{7.082612in}{3.287048in}}{\pgfqpoint{7.089123in}{3.280536in}}%
\pgfpathcurveto{\pgfqpoint{7.095634in}{3.274025in}}{\pgfqpoint{7.104467in}{3.270367in}}{\pgfqpoint{7.113675in}{3.270367in}}%
\pgfpathlineto{\pgfqpoint{7.113675in}{3.270367in}}%
\pgfpathclose%
\pgfusepath{stroke,fill}%
\end{pgfscope}%
\begin{pgfscope}%
\pgfpathrectangle{\pgfqpoint{0.050000in}{0.050000in}}{\pgfqpoint{2.419000in}{2.419000in}}%
\pgfusepath{clip}%
\pgfsetbuttcap%
\pgfsetroundjoin%
\definecolor{currentfill}{rgb}{0.200000,0.133333,0.533333}%
\pgfsetfillcolor{currentfill}%
\pgfsetfillopacity{0.347242}%
\pgfsetlinewidth{1.003750pt}%
\definecolor{currentstroke}{rgb}{0.200000,0.133333,0.533333}%
\pgfsetstrokecolor{currentstroke}%
\pgfsetstrokeopacity{0.347242}%
\pgfsetdash{}{0pt}%
\pgfpathmoveto{\pgfqpoint{2.212687in}{3.200154in}}%
\pgfpathcurveto{\pgfqpoint{2.221896in}{3.200154in}}{\pgfqpoint{2.230728in}{3.203813in}}{\pgfqpoint{2.237239in}{3.210324in}}%
\pgfpathcurveto{\pgfqpoint{2.243751in}{3.216835in}}{\pgfqpoint{2.247409in}{3.225668in}}{\pgfqpoint{2.247409in}{3.234876in}}%
\pgfpathcurveto{\pgfqpoint{2.247409in}{3.244085in}}{\pgfqpoint{2.243751in}{3.252917in}}{\pgfqpoint{2.237239in}{3.259429in}}%
\pgfpathcurveto{\pgfqpoint{2.230728in}{3.265940in}}{\pgfqpoint{2.221896in}{3.269598in}}{\pgfqpoint{2.212687in}{3.269598in}}%
\pgfpathcurveto{\pgfqpoint{2.203479in}{3.269598in}}{\pgfqpoint{2.194646in}{3.265940in}}{\pgfqpoint{2.188135in}{3.259429in}}%
\pgfpathcurveto{\pgfqpoint{2.181623in}{3.252917in}}{\pgfqpoint{2.177965in}{3.244085in}}{\pgfqpoint{2.177965in}{3.234876in}}%
\pgfpathcurveto{\pgfqpoint{2.177965in}{3.225668in}}{\pgfqpoint{2.181623in}{3.216835in}}{\pgfqpoint{2.188135in}{3.210324in}}%
\pgfpathcurveto{\pgfqpoint{2.194646in}{3.203813in}}{\pgfqpoint{2.203479in}{3.200154in}}{\pgfqpoint{2.212687in}{3.200154in}}%
\pgfpathlineto{\pgfqpoint{2.212687in}{3.200154in}}%
\pgfpathclose%
\pgfusepath{stroke,fill}%
\end{pgfscope}%
\begin{pgfscope}%
\pgfpathrectangle{\pgfqpoint{0.050000in}{0.050000in}}{\pgfqpoint{2.419000in}{2.419000in}}%
\pgfusepath{clip}%
\pgfsetbuttcap%
\pgfsetroundjoin%
\definecolor{currentfill}{rgb}{0.200000,0.133333,0.533333}%
\pgfsetfillcolor{currentfill}%
\pgfsetfillopacity{0.347242}%
\pgfsetlinewidth{1.003750pt}%
\definecolor{currentstroke}{rgb}{0.200000,0.133333,0.533333}%
\pgfsetstrokecolor{currentstroke}%
\pgfsetstrokeopacity{0.347242}%
\pgfsetdash{}{0pt}%
\pgfpathmoveto{\pgfqpoint{5.750335in}{3.200154in}}%
\pgfpathcurveto{\pgfqpoint{5.759543in}{3.200154in}}{\pgfqpoint{5.768376in}{3.203813in}}{\pgfqpoint{5.774887in}{3.210324in}}%
\pgfpathcurveto{\pgfqpoint{5.781398in}{3.216835in}}{\pgfqpoint{5.785057in}{3.225668in}}{\pgfqpoint{5.785057in}{3.234876in}}%
\pgfpathcurveto{\pgfqpoint{5.785057in}{3.244085in}}{\pgfqpoint{5.781398in}{3.252917in}}{\pgfqpoint{5.774887in}{3.259429in}}%
\pgfpathcurveto{\pgfqpoint{5.768376in}{3.265940in}}{\pgfqpoint{5.759543in}{3.269598in}}{\pgfqpoint{5.750335in}{3.269598in}}%
\pgfpathcurveto{\pgfqpoint{5.741126in}{3.269598in}}{\pgfqpoint{5.732294in}{3.265940in}}{\pgfqpoint{5.725782in}{3.259429in}}%
\pgfpathcurveto{\pgfqpoint{5.719271in}{3.252917in}}{\pgfqpoint{5.715613in}{3.244085in}}{\pgfqpoint{5.715613in}{3.234876in}}%
\pgfpathcurveto{\pgfqpoint{5.715613in}{3.225668in}}{\pgfqpoint{5.719271in}{3.216835in}}{\pgfqpoint{5.725782in}{3.210324in}}%
\pgfpathcurveto{\pgfqpoint{5.732294in}{3.203813in}}{\pgfqpoint{5.741126in}{3.200154in}}{\pgfqpoint{5.750335in}{3.200154in}}%
\pgfpathlineto{\pgfqpoint{5.750335in}{3.200154in}}%
\pgfpathclose%
\pgfusepath{stroke,fill}%
\end{pgfscope}%
\begin{pgfscope}%
\pgfpathrectangle{\pgfqpoint{0.050000in}{0.050000in}}{\pgfqpoint{2.419000in}{2.419000in}}%
\pgfusepath{clip}%
\pgfsetbuttcap%
\pgfsetroundjoin%
\definecolor{currentfill}{rgb}{0.200000,0.133333,0.533333}%
\pgfsetfillcolor{currentfill}%
\pgfsetfillopacity{0.350663}%
\pgfsetlinewidth{1.003750pt}%
\definecolor{currentstroke}{rgb}{0.200000,0.133333,0.533333}%
\pgfsetstrokecolor{currentstroke}%
\pgfsetstrokeopacity{0.350663}%
\pgfsetdash{}{0pt}%
\pgfpathmoveto{\pgfqpoint{0.768881in}{3.128138in}}%
\pgfpathcurveto{\pgfqpoint{0.778089in}{3.128138in}}{\pgfqpoint{0.786922in}{3.131796in}}{\pgfqpoint{0.793433in}{3.138308in}}%
\pgfpathcurveto{\pgfqpoint{0.799945in}{3.144819in}}{\pgfqpoint{0.803603in}{3.153651in}}{\pgfqpoint{0.803603in}{3.162860in}}%
\pgfpathcurveto{\pgfqpoint{0.803603in}{3.172068in}}{\pgfqpoint{0.799945in}{3.180901in}}{\pgfqpoint{0.793433in}{3.187412in}}%
\pgfpathcurveto{\pgfqpoint{0.786922in}{3.193924in}}{\pgfqpoint{0.778089in}{3.197582in}}{\pgfqpoint{0.768881in}{3.197582in}}%
\pgfpathcurveto{\pgfqpoint{0.759673in}{3.197582in}}{\pgfqpoint{0.750840in}{3.193924in}}{\pgfqpoint{0.744329in}{3.187412in}}%
\pgfpathcurveto{\pgfqpoint{0.737817in}{3.180901in}}{\pgfqpoint{0.734159in}{3.172068in}}{\pgfqpoint{0.734159in}{3.162860in}}%
\pgfpathcurveto{\pgfqpoint{0.734159in}{3.153651in}}{\pgfqpoint{0.737817in}{3.144819in}}{\pgfqpoint{0.744329in}{3.138308in}}%
\pgfpathcurveto{\pgfqpoint{0.750840in}{3.131796in}}{\pgfqpoint{0.759673in}{3.128138in}}{\pgfqpoint{0.768881in}{3.128138in}}%
\pgfpathlineto{\pgfqpoint{0.768881in}{3.128138in}}%
\pgfpathclose%
\pgfusepath{stroke,fill}%
\end{pgfscope}%
\begin{pgfscope}%
\pgfpathrectangle{\pgfqpoint{0.050000in}{0.050000in}}{\pgfqpoint{2.419000in}{2.419000in}}%
\pgfusepath{clip}%
\pgfsetbuttcap%
\pgfsetroundjoin%
\definecolor{currentfill}{rgb}{0.200000,0.133333,0.533333}%
\pgfsetfillcolor{currentfill}%
\pgfsetfillopacity{0.350663}%
\pgfsetlinewidth{1.003750pt}%
\definecolor{currentstroke}{rgb}{0.200000,0.133333,0.533333}%
\pgfsetstrokecolor{currentstroke}%
\pgfsetstrokeopacity{0.350663}%
\pgfsetdash{}{0pt}%
\pgfpathmoveto{\pgfqpoint{4.351970in}{3.128138in}}%
\pgfpathcurveto{\pgfqpoint{4.361178in}{3.128138in}}{\pgfqpoint{4.370011in}{3.131796in}}{\pgfqpoint{4.376522in}{3.138308in}}%
\pgfpathcurveto{\pgfqpoint{4.383034in}{3.144819in}}{\pgfqpoint{4.386692in}{3.153651in}}{\pgfqpoint{4.386692in}{3.162860in}}%
\pgfpathcurveto{\pgfqpoint{4.386692in}{3.172068in}}{\pgfqpoint{4.383034in}{3.180901in}}{\pgfqpoint{4.376522in}{3.187412in}}%
\pgfpathcurveto{\pgfqpoint{4.370011in}{3.193924in}}{\pgfqpoint{4.361178in}{3.197582in}}{\pgfqpoint{4.351970in}{3.197582in}}%
\pgfpathcurveto{\pgfqpoint{4.342761in}{3.197582in}}{\pgfqpoint{4.333929in}{3.193924in}}{\pgfqpoint{4.327418in}{3.187412in}}%
\pgfpathcurveto{\pgfqpoint{4.320906in}{3.180901in}}{\pgfqpoint{4.317248in}{3.172068in}}{\pgfqpoint{4.317248in}{3.162860in}}%
\pgfpathcurveto{\pgfqpoint{4.317248in}{3.153651in}}{\pgfqpoint{4.320906in}{3.144819in}}{\pgfqpoint{4.327418in}{3.138308in}}%
\pgfpathcurveto{\pgfqpoint{4.333929in}{3.131796in}}{\pgfqpoint{4.342761in}{3.128138in}}{\pgfqpoint{4.351970in}{3.128138in}}%
\pgfpathlineto{\pgfqpoint{4.351970in}{3.128138in}}%
\pgfpathclose%
\pgfusepath{stroke,fill}%
\end{pgfscope}%
\begin{pgfscope}%
\pgfpathrectangle{\pgfqpoint{0.050000in}{0.050000in}}{\pgfqpoint{2.419000in}{2.419000in}}%
\pgfusepath{clip}%
\pgfsetbuttcap%
\pgfsetroundjoin%
\definecolor{currentfill}{rgb}{0.200000,0.133333,0.533333}%
\pgfsetfillcolor{currentfill}%
\pgfsetfillopacity{0.350663}%
\pgfsetlinewidth{1.003750pt}%
\definecolor{currentstroke}{rgb}{0.200000,0.133333,0.533333}%
\pgfsetstrokecolor{currentstroke}%
\pgfsetstrokeopacity{0.350663}%
\pgfsetdash{}{0pt}%
\pgfpathmoveto{\pgfqpoint{7.935059in}{3.128138in}}%
\pgfpathcurveto{\pgfqpoint{7.944267in}{3.128138in}}{\pgfqpoint{7.953100in}{3.131796in}}{\pgfqpoint{7.959611in}{3.138308in}}%
\pgfpathcurveto{\pgfqpoint{7.966122in}{3.144819in}}{\pgfqpoint{7.969781in}{3.153651in}}{\pgfqpoint{7.969781in}{3.162860in}}%
\pgfpathcurveto{\pgfqpoint{7.969781in}{3.172068in}}{\pgfqpoint{7.966122in}{3.180901in}}{\pgfqpoint{7.959611in}{3.187412in}}%
\pgfpathcurveto{\pgfqpoint{7.953100in}{3.193924in}}{\pgfqpoint{7.944267in}{3.197582in}}{\pgfqpoint{7.935059in}{3.197582in}}%
\pgfpathcurveto{\pgfqpoint{7.925850in}{3.197582in}}{\pgfqpoint{7.917018in}{3.193924in}}{\pgfqpoint{7.910506in}{3.187412in}}%
\pgfpathcurveto{\pgfqpoint{7.903995in}{3.180901in}}{\pgfqpoint{7.900337in}{3.172068in}}{\pgfqpoint{7.900337in}{3.162860in}}%
\pgfpathcurveto{\pgfqpoint{7.900337in}{3.153651in}}{\pgfqpoint{7.903995in}{3.144819in}}{\pgfqpoint{7.910506in}{3.138308in}}%
\pgfpathcurveto{\pgfqpoint{7.917018in}{3.131796in}}{\pgfqpoint{7.925850in}{3.128138in}}{\pgfqpoint{7.935059in}{3.128138in}}%
\pgfpathlineto{\pgfqpoint{7.935059in}{3.128138in}}%
\pgfpathclose%
\pgfusepath{stroke,fill}%
\end{pgfscope}%
\begin{pgfscope}%
\pgfpathrectangle{\pgfqpoint{0.050000in}{0.050000in}}{\pgfqpoint{2.419000in}{2.419000in}}%
\pgfusepath{clip}%
\pgfsetbuttcap%
\pgfsetroundjoin%
\definecolor{currentfill}{rgb}{0.200000,0.133333,0.533333}%
\pgfsetfillcolor{currentfill}%
\pgfsetfillopacity{0.354174}%
\pgfsetlinewidth{1.003750pt}%
\definecolor{currentstroke}{rgb}{0.200000,0.133333,0.533333}%
\pgfsetstrokecolor{currentstroke}%
\pgfsetstrokeopacity{0.354174}%
\pgfsetdash{}{0pt}%
\pgfpathmoveto{\pgfqpoint{2.917214in}{3.054247in}}%
\pgfpathcurveto{\pgfqpoint{2.926422in}{3.054247in}}{\pgfqpoint{2.935254in}{3.057906in}}{\pgfqpoint{2.941766in}{3.064417in}}%
\pgfpathcurveto{\pgfqpoint{2.948277in}{3.070928in}}{\pgfqpoint{2.951936in}{3.079761in}}{\pgfqpoint{2.951936in}{3.088969in}}%
\pgfpathcurveto{\pgfqpoint{2.951936in}{3.098178in}}{\pgfqpoint{2.948277in}{3.107010in}}{\pgfqpoint{2.941766in}{3.113522in}}%
\pgfpathcurveto{\pgfqpoint{2.935254in}{3.120033in}}{\pgfqpoint{2.926422in}{3.123692in}}{\pgfqpoint{2.917214in}{3.123692in}}%
\pgfpathcurveto{\pgfqpoint{2.908005in}{3.123692in}}{\pgfqpoint{2.899173in}{3.120033in}}{\pgfqpoint{2.892661in}{3.113522in}}%
\pgfpathcurveto{\pgfqpoint{2.886150in}{3.107010in}}{\pgfqpoint{2.882491in}{3.098178in}}{\pgfqpoint{2.882491in}{3.088969in}}%
\pgfpathcurveto{\pgfqpoint{2.882491in}{3.079761in}}{\pgfqpoint{2.886150in}{3.070928in}}{\pgfqpoint{2.892661in}{3.064417in}}%
\pgfpathcurveto{\pgfqpoint{2.899173in}{3.057906in}}{\pgfqpoint{2.908005in}{3.054247in}}{\pgfqpoint{2.917214in}{3.054247in}}%
\pgfpathlineto{\pgfqpoint{2.917214in}{3.054247in}}%
\pgfpathclose%
\pgfusepath{stroke,fill}%
\end{pgfscope}%
\begin{pgfscope}%
\pgfpathrectangle{\pgfqpoint{0.050000in}{0.050000in}}{\pgfqpoint{2.419000in}{2.419000in}}%
\pgfusepath{clip}%
\pgfsetbuttcap%
\pgfsetroundjoin%
\definecolor{currentfill}{rgb}{0.200000,0.133333,0.533333}%
\pgfsetfillcolor{currentfill}%
\pgfsetfillopacity{0.354174}%
\pgfsetlinewidth{1.003750pt}%
\definecolor{currentstroke}{rgb}{0.200000,0.133333,0.533333}%
\pgfsetstrokecolor{currentstroke}%
\pgfsetstrokeopacity{0.354174}%
\pgfsetdash{}{0pt}%
\pgfpathmoveto{\pgfqpoint{6.546926in}{3.054247in}}%
\pgfpathcurveto{\pgfqpoint{6.556135in}{3.054247in}}{\pgfqpoint{6.564967in}{3.057906in}}{\pgfqpoint{6.571478in}{3.064417in}}%
\pgfpathcurveto{\pgfqpoint{6.577990in}{3.070928in}}{\pgfqpoint{6.581648in}{3.079761in}}{\pgfqpoint{6.581648in}{3.088969in}}%
\pgfpathcurveto{\pgfqpoint{6.581648in}{3.098178in}}{\pgfqpoint{6.577990in}{3.107010in}}{\pgfqpoint{6.571478in}{3.113522in}}%
\pgfpathcurveto{\pgfqpoint{6.564967in}{3.120033in}}{\pgfqpoint{6.556135in}{3.123692in}}{\pgfqpoint{6.546926in}{3.123692in}}%
\pgfpathcurveto{\pgfqpoint{6.537718in}{3.123692in}}{\pgfqpoint{6.528885in}{3.120033in}}{\pgfqpoint{6.522374in}{3.113522in}}%
\pgfpathcurveto{\pgfqpoint{6.515862in}{3.107010in}}{\pgfqpoint{6.512204in}{3.098178in}}{\pgfqpoint{6.512204in}{3.088969in}}%
\pgfpathcurveto{\pgfqpoint{6.512204in}{3.079761in}}{\pgfqpoint{6.515862in}{3.070928in}}{\pgfqpoint{6.522374in}{3.064417in}}%
\pgfpathcurveto{\pgfqpoint{6.528885in}{3.057906in}}{\pgfqpoint{6.537718in}{3.054247in}}{\pgfqpoint{6.546926in}{3.054247in}}%
\pgfpathlineto{\pgfqpoint{6.546926in}{3.054247in}}%
\pgfpathclose%
\pgfusepath{stroke,fill}%
\end{pgfscope}%
\begin{pgfscope}%
\pgfpathrectangle{\pgfqpoint{0.050000in}{0.050000in}}{\pgfqpoint{2.419000in}{2.419000in}}%
\pgfusepath{clip}%
\pgfsetbuttcap%
\pgfsetroundjoin%
\definecolor{currentfill}{rgb}{0.200000,0.133333,0.533333}%
\pgfsetfillcolor{currentfill}%
\pgfsetfillopacity{0.357777}%
\pgfsetlinewidth{1.003750pt}%
\definecolor{currentstroke}{rgb}{0.200000,0.133333,0.533333}%
\pgfsetstrokecolor{currentstroke}%
\pgfsetstrokeopacity{0.357777}%
\pgfsetdash{}{0pt}%
\pgfpathmoveto{\pgfqpoint{1.444626in}{2.978408in}}%
\pgfpathcurveto{\pgfqpoint{1.453835in}{2.978408in}}{\pgfqpoint{1.462667in}{2.982067in}}{\pgfqpoint{1.469179in}{2.988578in}}%
\pgfpathcurveto{\pgfqpoint{1.475690in}{2.995090in}}{\pgfqpoint{1.479349in}{3.003922in}}{\pgfqpoint{1.479349in}{3.013131in}}%
\pgfpathcurveto{\pgfqpoint{1.479349in}{3.022339in}}{\pgfqpoint{1.475690in}{3.031172in}}{\pgfqpoint{1.469179in}{3.037683in}}%
\pgfpathcurveto{\pgfqpoint{1.462667in}{3.044194in}}{\pgfqpoint{1.453835in}{3.047853in}}{\pgfqpoint{1.444626in}{3.047853in}}%
\pgfpathcurveto{\pgfqpoint{1.435418in}{3.047853in}}{\pgfqpoint{1.426585in}{3.044194in}}{\pgfqpoint{1.420074in}{3.037683in}}%
\pgfpathcurveto{\pgfqpoint{1.413563in}{3.031172in}}{\pgfqpoint{1.409904in}{3.022339in}}{\pgfqpoint{1.409904in}{3.013131in}}%
\pgfpathcurveto{\pgfqpoint{1.409904in}{3.003922in}}{\pgfqpoint{1.413563in}{2.995090in}}{\pgfqpoint{1.420074in}{2.988578in}}%
\pgfpathcurveto{\pgfqpoint{1.426585in}{2.982067in}}{\pgfqpoint{1.435418in}{2.978408in}}{\pgfqpoint{1.444626in}{2.978408in}}%
\pgfpathlineto{\pgfqpoint{1.444626in}{2.978408in}}%
\pgfpathclose%
\pgfusepath{stroke,fill}%
\end{pgfscope}%
\begin{pgfscope}%
\pgfpathrectangle{\pgfqpoint{0.050000in}{0.050000in}}{\pgfqpoint{2.419000in}{2.419000in}}%
\pgfusepath{clip}%
\pgfsetbuttcap%
\pgfsetroundjoin%
\definecolor{currentfill}{rgb}{0.200000,0.133333,0.533333}%
\pgfsetfillcolor{currentfill}%
\pgfsetfillopacity{0.357777}%
\pgfsetlinewidth{1.003750pt}%
\definecolor{currentstroke}{rgb}{0.200000,0.133333,0.533333}%
\pgfsetstrokecolor{currentstroke}%
\pgfsetstrokeopacity{0.357777}%
\pgfsetdash{}{0pt}%
\pgfpathmoveto{\pgfqpoint{5.122192in}{2.978408in}}%
\pgfpathcurveto{\pgfqpoint{5.131400in}{2.978408in}}{\pgfqpoint{5.140233in}{2.982067in}}{\pgfqpoint{5.146744in}{2.988578in}}%
\pgfpathcurveto{\pgfqpoint{5.153256in}{2.995090in}}{\pgfqpoint{5.156914in}{3.003922in}}{\pgfqpoint{5.156914in}{3.013131in}}%
\pgfpathcurveto{\pgfqpoint{5.156914in}{3.022339in}}{\pgfqpoint{5.153256in}{3.031172in}}{\pgfqpoint{5.146744in}{3.037683in}}%
\pgfpathcurveto{\pgfqpoint{5.140233in}{3.044194in}}{\pgfqpoint{5.131400in}{3.047853in}}{\pgfqpoint{5.122192in}{3.047853in}}%
\pgfpathcurveto{\pgfqpoint{5.112984in}{3.047853in}}{\pgfqpoint{5.104151in}{3.044194in}}{\pgfqpoint{5.097640in}{3.037683in}}%
\pgfpathcurveto{\pgfqpoint{5.091128in}{3.031172in}}{\pgfqpoint{5.087470in}{3.022339in}}{\pgfqpoint{5.087470in}{3.013131in}}%
\pgfpathcurveto{\pgfqpoint{5.087470in}{3.003922in}}{\pgfqpoint{5.091128in}{2.995090in}}{\pgfqpoint{5.097640in}{2.988578in}}%
\pgfpathcurveto{\pgfqpoint{5.104151in}{2.982067in}}{\pgfqpoint{5.112984in}{2.978408in}}{\pgfqpoint{5.122192in}{2.978408in}}%
\pgfpathlineto{\pgfqpoint{5.122192in}{2.978408in}}%
\pgfpathclose%
\pgfusepath{stroke,fill}%
\end{pgfscope}%
\begin{pgfscope}%
\pgfpathrectangle{\pgfqpoint{0.050000in}{0.050000in}}{\pgfqpoint{2.419000in}{2.419000in}}%
\pgfusepath{clip}%
\pgfsetbuttcap%
\pgfsetroundjoin%
\definecolor{currentfill}{rgb}{0.200000,0.133333,0.533333}%
\pgfsetfillcolor{currentfill}%
\pgfsetfillopacity{0.357777}%
\pgfsetlinewidth{1.003750pt}%
\definecolor{currentstroke}{rgb}{0.200000,0.133333,0.533333}%
\pgfsetstrokecolor{currentstroke}%
\pgfsetstrokeopacity{0.357777}%
\pgfsetdash{}{0pt}%
\pgfpathmoveto{\pgfqpoint{8.799758in}{2.978408in}}%
\pgfpathcurveto{\pgfqpoint{8.808966in}{2.978408in}}{\pgfqpoint{8.817799in}{2.982067in}}{\pgfqpoint{8.824310in}{2.988578in}}%
\pgfpathcurveto{\pgfqpoint{8.830821in}{2.995090in}}{\pgfqpoint{8.834480in}{3.003922in}}{\pgfqpoint{8.834480in}{3.013131in}}%
\pgfpathcurveto{\pgfqpoint{8.834480in}{3.022339in}}{\pgfqpoint{8.830821in}{3.031172in}}{\pgfqpoint{8.824310in}{3.037683in}}%
\pgfpathcurveto{\pgfqpoint{8.817799in}{3.044194in}}{\pgfqpoint{8.808966in}{3.047853in}}{\pgfqpoint{8.799758in}{3.047853in}}%
\pgfpathcurveto{\pgfqpoint{8.790549in}{3.047853in}}{\pgfqpoint{8.781717in}{3.044194in}}{\pgfqpoint{8.775205in}{3.037683in}}%
\pgfpathcurveto{\pgfqpoint{8.768694in}{3.031172in}}{\pgfqpoint{8.765035in}{3.022339in}}{\pgfqpoint{8.765035in}{3.013131in}}%
\pgfpathcurveto{\pgfqpoint{8.765035in}{3.003922in}}{\pgfqpoint{8.768694in}{2.995090in}}{\pgfqpoint{8.775205in}{2.988578in}}%
\pgfpathcurveto{\pgfqpoint{8.781717in}{2.982067in}}{\pgfqpoint{8.790549in}{2.978408in}}{\pgfqpoint{8.799758in}{2.978408in}}%
\pgfpathlineto{\pgfqpoint{8.799758in}{2.978408in}}%
\pgfpathclose%
\pgfusepath{stroke,fill}%
\end{pgfscope}%
\begin{pgfscope}%
\pgfpathrectangle{\pgfqpoint{0.050000in}{0.050000in}}{\pgfqpoint{2.419000in}{2.419000in}}%
\pgfusepath{clip}%
\pgfsetbuttcap%
\pgfsetroundjoin%
\definecolor{currentfill}{rgb}{0.200000,0.133333,0.533333}%
\pgfsetfillcolor{currentfill}%
\pgfsetfillopacity{0.361476}%
\pgfsetlinewidth{1.003750pt}%
\definecolor{currentstroke}{rgb}{0.200000,0.133333,0.533333}%
\pgfsetstrokecolor{currentstroke}%
\pgfsetstrokeopacity{0.361476}%
\pgfsetdash{}{0pt}%
\pgfpathmoveto{\pgfqpoint{-0.067308in}{2.900543in}}%
\pgfpathcurveto{\pgfqpoint{-0.058100in}{2.900543in}}{\pgfqpoint{-0.049267in}{2.904202in}}{\pgfqpoint{-0.042756in}{2.910713in}}%
\pgfpathcurveto{\pgfqpoint{-0.036244in}{2.917224in}}{\pgfqpoint{-0.032586in}{2.926057in}}{\pgfqpoint{-0.032586in}{2.935265in}}%
\pgfpathcurveto{\pgfqpoint{-0.032586in}{2.944474in}}{\pgfqpoint{-0.036244in}{2.953306in}}{\pgfqpoint{-0.042756in}{2.959818in}}%
\pgfpathcurveto{\pgfqpoint{-0.049267in}{2.966329in}}{\pgfqpoint{-0.058100in}{2.969988in}}{\pgfqpoint{-0.067308in}{2.969988in}}%
\pgfpathcurveto{\pgfqpoint{-0.076516in}{2.969988in}}{\pgfqpoint{-0.085349in}{2.966329in}}{\pgfqpoint{-0.091860in}{2.959818in}}%
\pgfpathcurveto{\pgfqpoint{-0.098372in}{2.953306in}}{\pgfqpoint{-0.102030in}{2.944474in}}{\pgfqpoint{-0.102030in}{2.935265in}}%
\pgfpathcurveto{\pgfqpoint{-0.102030in}{2.926057in}}{\pgfqpoint{-0.098372in}{2.917224in}}{\pgfqpoint{-0.091860in}{2.910713in}}%
\pgfpathcurveto{\pgfqpoint{-0.085349in}{2.904202in}}{\pgfqpoint{-0.076516in}{2.900543in}}{\pgfqpoint{-0.067308in}{2.900543in}}%
\pgfpathlineto{\pgfqpoint{-0.067308in}{2.900543in}}%
\pgfpathclose%
\pgfusepath{stroke,fill}%
\end{pgfscope}%
\begin{pgfscope}%
\pgfpathrectangle{\pgfqpoint{0.050000in}{0.050000in}}{\pgfqpoint{2.419000in}{2.419000in}}%
\pgfusepath{clip}%
\pgfsetbuttcap%
\pgfsetroundjoin%
\definecolor{currentfill}{rgb}{0.200000,0.133333,0.533333}%
\pgfsetfillcolor{currentfill}%
\pgfsetfillopacity{0.361476}%
\pgfsetlinewidth{1.003750pt}%
\definecolor{currentstroke}{rgb}{0.200000,0.133333,0.533333}%
\pgfsetstrokecolor{currentstroke}%
\pgfsetstrokeopacity{0.361476}%
\pgfsetdash{}{0pt}%
\pgfpathmoveto{\pgfqpoint{3.659389in}{2.900543in}}%
\pgfpathcurveto{\pgfqpoint{3.668598in}{2.900543in}}{\pgfqpoint{3.677430in}{2.904202in}}{\pgfqpoint{3.683942in}{2.910713in}}%
\pgfpathcurveto{\pgfqpoint{3.690453in}{2.917224in}}{\pgfqpoint{3.694112in}{2.926057in}}{\pgfqpoint{3.694112in}{2.935265in}}%
\pgfpathcurveto{\pgfqpoint{3.694112in}{2.944474in}}{\pgfqpoint{3.690453in}{2.953306in}}{\pgfqpoint{3.683942in}{2.959818in}}%
\pgfpathcurveto{\pgfqpoint{3.677430in}{2.966329in}}{\pgfqpoint{3.668598in}{2.969988in}}{\pgfqpoint{3.659389in}{2.969988in}}%
\pgfpathcurveto{\pgfqpoint{3.650181in}{2.969988in}}{\pgfqpoint{3.641348in}{2.966329in}}{\pgfqpoint{3.634837in}{2.959818in}}%
\pgfpathcurveto{\pgfqpoint{3.628326in}{2.953306in}}{\pgfqpoint{3.624667in}{2.944474in}}{\pgfqpoint{3.624667in}{2.935265in}}%
\pgfpathcurveto{\pgfqpoint{3.624667in}{2.926057in}}{\pgfqpoint{3.628326in}{2.917224in}}{\pgfqpoint{3.634837in}{2.910713in}}%
\pgfpathcurveto{\pgfqpoint{3.641348in}{2.904202in}}{\pgfqpoint{3.650181in}{2.900543in}}{\pgfqpoint{3.659389in}{2.900543in}}%
\pgfpathlineto{\pgfqpoint{3.659389in}{2.900543in}}%
\pgfpathclose%
\pgfusepath{stroke,fill}%
\end{pgfscope}%
\begin{pgfscope}%
\pgfpathrectangle{\pgfqpoint{0.050000in}{0.050000in}}{\pgfqpoint{2.419000in}{2.419000in}}%
\pgfusepath{clip}%
\pgfsetbuttcap%
\pgfsetroundjoin%
\definecolor{currentfill}{rgb}{0.200000,0.133333,0.533333}%
\pgfsetfillcolor{currentfill}%
\pgfsetfillopacity{0.361476}%
\pgfsetlinewidth{1.003750pt}%
\definecolor{currentstroke}{rgb}{0.200000,0.133333,0.533333}%
\pgfsetstrokecolor{currentstroke}%
\pgfsetstrokeopacity{0.361476}%
\pgfsetdash{}{0pt}%
\pgfpathmoveto{\pgfqpoint{7.386087in}{2.900543in}}%
\pgfpathcurveto{\pgfqpoint{7.395295in}{2.900543in}}{\pgfqpoint{7.404128in}{2.904202in}}{\pgfqpoint{7.410639in}{2.910713in}}%
\pgfpathcurveto{\pgfqpoint{7.417150in}{2.917224in}}{\pgfqpoint{7.420809in}{2.926057in}}{\pgfqpoint{7.420809in}{2.935265in}}%
\pgfpathcurveto{\pgfqpoint{7.420809in}{2.944474in}}{\pgfqpoint{7.417150in}{2.953306in}}{\pgfqpoint{7.410639in}{2.959818in}}%
\pgfpathcurveto{\pgfqpoint{7.404128in}{2.966329in}}{\pgfqpoint{7.395295in}{2.969988in}}{\pgfqpoint{7.386087in}{2.969988in}}%
\pgfpathcurveto{\pgfqpoint{7.376878in}{2.969988in}}{\pgfqpoint{7.368046in}{2.966329in}}{\pgfqpoint{7.361534in}{2.959818in}}%
\pgfpathcurveto{\pgfqpoint{7.355023in}{2.953306in}}{\pgfqpoint{7.351365in}{2.944474in}}{\pgfqpoint{7.351365in}{2.935265in}}%
\pgfpathcurveto{\pgfqpoint{7.351365in}{2.926057in}}{\pgfqpoint{7.355023in}{2.917224in}}{\pgfqpoint{7.361534in}{2.910713in}}%
\pgfpathcurveto{\pgfqpoint{7.368046in}{2.904202in}}{\pgfqpoint{7.376878in}{2.900543in}}{\pgfqpoint{7.386087in}{2.900543in}}%
\pgfpathlineto{\pgfqpoint{7.386087in}{2.900543in}}%
\pgfpathclose%
\pgfusepath{stroke,fill}%
\end{pgfscope}%
\begin{pgfscope}%
\pgfpathrectangle{\pgfqpoint{0.050000in}{0.050000in}}{\pgfqpoint{2.419000in}{2.419000in}}%
\pgfusepath{clip}%
\pgfsetbuttcap%
\pgfsetroundjoin%
\definecolor{currentfill}{rgb}{0.200000,0.133333,0.533333}%
\pgfsetfillcolor{currentfill}%
\pgfsetfillopacity{0.365275}%
\pgfsetlinewidth{1.003750pt}%
\definecolor{currentstroke}{rgb}{0.200000,0.133333,0.533333}%
\pgfsetstrokecolor{currentstroke}%
\pgfsetstrokeopacity{0.365275}%
\pgfsetdash{}{0pt}%
\pgfpathmoveto{\pgfqpoint{2.156972in}{2.820569in}}%
\pgfpathcurveto{\pgfqpoint{2.166180in}{2.820569in}}{\pgfqpoint{2.175013in}{2.824228in}}{\pgfqpoint{2.181524in}{2.830739in}}%
\pgfpathcurveto{\pgfqpoint{2.188036in}{2.837250in}}{\pgfqpoint{2.191694in}{2.846083in}}{\pgfqpoint{2.191694in}{2.855291in}}%
\pgfpathcurveto{\pgfqpoint{2.191694in}{2.864500in}}{\pgfqpoint{2.188036in}{2.873332in}}{\pgfqpoint{2.181524in}{2.879844in}}%
\pgfpathcurveto{\pgfqpoint{2.175013in}{2.886355in}}{\pgfqpoint{2.166180in}{2.890014in}}{\pgfqpoint{2.156972in}{2.890014in}}%
\pgfpathcurveto{\pgfqpoint{2.147764in}{2.890014in}}{\pgfqpoint{2.138931in}{2.886355in}}{\pgfqpoint{2.132420in}{2.879844in}}%
\pgfpathcurveto{\pgfqpoint{2.125908in}{2.873332in}}{\pgfqpoint{2.122250in}{2.864500in}}{\pgfqpoint{2.122250in}{2.855291in}}%
\pgfpathcurveto{\pgfqpoint{2.122250in}{2.846083in}}{\pgfqpoint{2.125908in}{2.837250in}}{\pgfqpoint{2.132420in}{2.830739in}}%
\pgfpathcurveto{\pgfqpoint{2.138931in}{2.824228in}}{\pgfqpoint{2.147764in}{2.820569in}}{\pgfqpoint{2.156972in}{2.820569in}}%
\pgfpathlineto{\pgfqpoint{2.156972in}{2.820569in}}%
\pgfpathclose%
\pgfusepath{stroke,fill}%
\end{pgfscope}%
\begin{pgfscope}%
\pgfpathrectangle{\pgfqpoint{0.050000in}{0.050000in}}{\pgfqpoint{2.419000in}{2.419000in}}%
\pgfusepath{clip}%
\pgfsetbuttcap%
\pgfsetroundjoin%
\definecolor{currentfill}{rgb}{0.200000,0.133333,0.533333}%
\pgfsetfillcolor{currentfill}%
\pgfsetfillopacity{0.365275}%
\pgfsetlinewidth{1.003750pt}%
\definecolor{currentstroke}{rgb}{0.200000,0.133333,0.533333}%
\pgfsetstrokecolor{currentstroke}%
\pgfsetstrokeopacity{0.365275}%
\pgfsetdash{}{0pt}%
\pgfpathmoveto{\pgfqpoint{5.934132in}{2.820569in}}%
\pgfpathcurveto{\pgfqpoint{5.943340in}{2.820569in}}{\pgfqpoint{5.952173in}{2.824228in}}{\pgfqpoint{5.958684in}{2.830739in}}%
\pgfpathcurveto{\pgfqpoint{5.965195in}{2.837250in}}{\pgfqpoint{5.968854in}{2.846083in}}{\pgfqpoint{5.968854in}{2.855291in}}%
\pgfpathcurveto{\pgfqpoint{5.968854in}{2.864500in}}{\pgfqpoint{5.965195in}{2.873332in}}{\pgfqpoint{5.958684in}{2.879844in}}%
\pgfpathcurveto{\pgfqpoint{5.952173in}{2.886355in}}{\pgfqpoint{5.943340in}{2.890014in}}{\pgfqpoint{5.934132in}{2.890014in}}%
\pgfpathcurveto{\pgfqpoint{5.924923in}{2.890014in}}{\pgfqpoint{5.916091in}{2.886355in}}{\pgfqpoint{5.909579in}{2.879844in}}%
\pgfpathcurveto{\pgfqpoint{5.903068in}{2.873332in}}{\pgfqpoint{5.899409in}{2.864500in}}{\pgfqpoint{5.899409in}{2.855291in}}%
\pgfpathcurveto{\pgfqpoint{5.899409in}{2.846083in}}{\pgfqpoint{5.903068in}{2.837250in}}{\pgfqpoint{5.909579in}{2.830739in}}%
\pgfpathcurveto{\pgfqpoint{5.916091in}{2.824228in}}{\pgfqpoint{5.924923in}{2.820569in}}{\pgfqpoint{5.934132in}{2.820569in}}%
\pgfpathlineto{\pgfqpoint{5.934132in}{2.820569in}}%
\pgfpathclose%
\pgfusepath{stroke,fill}%
\end{pgfscope}%
\begin{pgfscope}%
\pgfpathrectangle{\pgfqpoint{0.050000in}{0.050000in}}{\pgfqpoint{2.419000in}{2.419000in}}%
\pgfusepath{clip}%
\pgfsetbuttcap%
\pgfsetroundjoin%
\definecolor{currentfill}{rgb}{0.200000,0.133333,0.533333}%
\pgfsetfillcolor{currentfill}%
\pgfsetfillopacity{0.365275}%
\pgfsetlinewidth{1.003750pt}%
\definecolor{currentstroke}{rgb}{0.200000,0.133333,0.533333}%
\pgfsetstrokecolor{currentstroke}%
\pgfsetstrokeopacity{0.365275}%
\pgfsetdash{}{0pt}%
\pgfpathmoveto{\pgfqpoint{9.711291in}{2.820569in}}%
\pgfpathcurveto{\pgfqpoint{9.720500in}{2.820569in}}{\pgfqpoint{9.729332in}{2.824228in}}{\pgfqpoint{9.735844in}{2.830739in}}%
\pgfpathcurveto{\pgfqpoint{9.742355in}{2.837250in}}{\pgfqpoint{9.746013in}{2.846083in}}{\pgfqpoint{9.746013in}{2.855291in}}%
\pgfpathcurveto{\pgfqpoint{9.746013in}{2.864500in}}{\pgfqpoint{9.742355in}{2.873332in}}{\pgfqpoint{9.735844in}{2.879844in}}%
\pgfpathcurveto{\pgfqpoint{9.729332in}{2.886355in}}{\pgfqpoint{9.720500in}{2.890014in}}{\pgfqpoint{9.711291in}{2.890014in}}%
\pgfpathcurveto{\pgfqpoint{9.702083in}{2.890014in}}{\pgfqpoint{9.693250in}{2.886355in}}{\pgfqpoint{9.686739in}{2.879844in}}%
\pgfpathcurveto{\pgfqpoint{9.680228in}{2.873332in}}{\pgfqpoint{9.676569in}{2.864500in}}{\pgfqpoint{9.676569in}{2.855291in}}%
\pgfpathcurveto{\pgfqpoint{9.676569in}{2.846083in}}{\pgfqpoint{9.680228in}{2.837250in}}{\pgfqpoint{9.686739in}{2.830739in}}%
\pgfpathcurveto{\pgfqpoint{9.693250in}{2.824228in}}{\pgfqpoint{9.702083in}{2.820569in}}{\pgfqpoint{9.711291in}{2.820569in}}%
\pgfpathlineto{\pgfqpoint{9.711291in}{2.820569in}}%
\pgfpathclose%
\pgfusepath{stroke,fill}%
\end{pgfscope}%
\begin{pgfscope}%
\pgfpathrectangle{\pgfqpoint{0.050000in}{0.050000in}}{\pgfqpoint{2.419000in}{2.419000in}}%
\pgfusepath{clip}%
\pgfsetbuttcap%
\pgfsetroundjoin%
\definecolor{currentfill}{rgb}{0.200000,0.133333,0.533333}%
\pgfsetfillcolor{currentfill}%
\pgfsetfillopacity{0.369179}%
\pgfsetlinewidth{1.003750pt}%
\definecolor{currentstroke}{rgb}{0.200000,0.133333,0.533333}%
\pgfsetstrokecolor{currentstroke}%
\pgfsetstrokeopacity{0.369179}%
\pgfsetdash{}{0pt}%
\pgfpathmoveto{\pgfqpoint{4.442316in}{2.738400in}}%
\pgfpathcurveto{\pgfqpoint{4.451524in}{2.738400in}}{\pgfqpoint{4.460356in}{2.742058in}}{\pgfqpoint{4.466868in}{2.748570in}}%
\pgfpathcurveto{\pgfqpoint{4.473379in}{2.755081in}}{\pgfqpoint{4.477038in}{2.763914in}}{\pgfqpoint{4.477038in}{2.773122in}}%
\pgfpathcurveto{\pgfqpoint{4.477038in}{2.782330in}}{\pgfqpoint{4.473379in}{2.791163in}}{\pgfqpoint{4.466868in}{2.797674in}}%
\pgfpathcurveto{\pgfqpoint{4.460356in}{2.804186in}}{\pgfqpoint{4.451524in}{2.807844in}}{\pgfqpoint{4.442316in}{2.807844in}}%
\pgfpathcurveto{\pgfqpoint{4.433107in}{2.807844in}}{\pgfqpoint{4.424275in}{2.804186in}}{\pgfqpoint{4.417763in}{2.797674in}}%
\pgfpathcurveto{\pgfqpoint{4.411252in}{2.791163in}}{\pgfqpoint{4.407593in}{2.782330in}}{\pgfqpoint{4.407593in}{2.773122in}}%
\pgfpathcurveto{\pgfqpoint{4.407593in}{2.763914in}}{\pgfqpoint{4.411252in}{2.755081in}}{\pgfqpoint{4.417763in}{2.748570in}}%
\pgfpathcurveto{\pgfqpoint{4.424275in}{2.742058in}}{\pgfqpoint{4.433107in}{2.738400in}}{\pgfqpoint{4.442316in}{2.738400in}}%
\pgfpathlineto{\pgfqpoint{4.442316in}{2.738400in}}%
\pgfpathclose%
\pgfusepath{stroke,fill}%
\end{pgfscope}%
\begin{pgfscope}%
\pgfpathrectangle{\pgfqpoint{0.050000in}{0.050000in}}{\pgfqpoint{2.419000in}{2.419000in}}%
\pgfusepath{clip}%
\pgfsetbuttcap%
\pgfsetroundjoin%
\definecolor{currentfill}{rgb}{0.200000,0.133333,0.533333}%
\pgfsetfillcolor{currentfill}%
\pgfsetfillopacity{0.369179}%
\pgfsetlinewidth{1.003750pt}%
\definecolor{currentstroke}{rgb}{0.200000,0.133333,0.533333}%
\pgfsetstrokecolor{currentstroke}%
\pgfsetstrokeopacity{0.369179}%
\pgfsetdash{}{0pt}%
\pgfpathmoveto{\pgfqpoint{0.613308in}{2.738400in}}%
\pgfpathcurveto{\pgfqpoint{0.622517in}{2.738400in}}{\pgfqpoint{0.631349in}{2.742058in}}{\pgfqpoint{0.637861in}{2.748570in}}%
\pgfpathcurveto{\pgfqpoint{0.644372in}{2.755081in}}{\pgfqpoint{0.648031in}{2.763914in}}{\pgfqpoint{0.648031in}{2.773122in}}%
\pgfpathcurveto{\pgfqpoint{0.648031in}{2.782330in}}{\pgfqpoint{0.644372in}{2.791163in}}{\pgfqpoint{0.637861in}{2.797674in}}%
\pgfpathcurveto{\pgfqpoint{0.631349in}{2.804186in}}{\pgfqpoint{0.622517in}{2.807844in}}{\pgfqpoint{0.613308in}{2.807844in}}%
\pgfpathcurveto{\pgfqpoint{0.604100in}{2.807844in}}{\pgfqpoint{0.595267in}{2.804186in}}{\pgfqpoint{0.588756in}{2.797674in}}%
\pgfpathcurveto{\pgfqpoint{0.582245in}{2.791163in}}{\pgfqpoint{0.578586in}{2.782330in}}{\pgfqpoint{0.578586in}{2.773122in}}%
\pgfpathcurveto{\pgfqpoint{0.578586in}{2.763914in}}{\pgfqpoint{0.582245in}{2.755081in}}{\pgfqpoint{0.588756in}{2.748570in}}%
\pgfpathcurveto{\pgfqpoint{0.595267in}{2.742058in}}{\pgfqpoint{0.604100in}{2.738400in}}{\pgfqpoint{0.613308in}{2.738400in}}%
\pgfpathlineto{\pgfqpoint{0.613308in}{2.738400in}}%
\pgfpathclose%
\pgfusepath{stroke,fill}%
\end{pgfscope}%
\begin{pgfscope}%
\pgfpathrectangle{\pgfqpoint{0.050000in}{0.050000in}}{\pgfqpoint{2.419000in}{2.419000in}}%
\pgfusepath{clip}%
\pgfsetbuttcap%
\pgfsetroundjoin%
\definecolor{currentfill}{rgb}{0.200000,0.133333,0.533333}%
\pgfsetfillcolor{currentfill}%
\pgfsetfillopacity{0.369179}%
\pgfsetlinewidth{1.003750pt}%
\definecolor{currentstroke}{rgb}{0.200000,0.133333,0.533333}%
\pgfsetstrokecolor{currentstroke}%
\pgfsetstrokeopacity{0.369179}%
\pgfsetdash{}{0pt}%
\pgfpathmoveto{\pgfqpoint{8.271323in}{2.738400in}}%
\pgfpathcurveto{\pgfqpoint{8.280531in}{2.738400in}}{\pgfqpoint{8.289364in}{2.742058in}}{\pgfqpoint{8.295875in}{2.748570in}}%
\pgfpathcurveto{\pgfqpoint{8.302386in}{2.755081in}}{\pgfqpoint{8.306045in}{2.763914in}}{\pgfqpoint{8.306045in}{2.773122in}}%
\pgfpathcurveto{\pgfqpoint{8.306045in}{2.782330in}}{\pgfqpoint{8.302386in}{2.791163in}}{\pgfqpoint{8.295875in}{2.797674in}}%
\pgfpathcurveto{\pgfqpoint{8.289364in}{2.804186in}}{\pgfqpoint{8.280531in}{2.807844in}}{\pgfqpoint{8.271323in}{2.807844in}}%
\pgfpathcurveto{\pgfqpoint{8.262114in}{2.807844in}}{\pgfqpoint{8.253282in}{2.804186in}}{\pgfqpoint{8.246770in}{2.797674in}}%
\pgfpathcurveto{\pgfqpoint{8.240259in}{2.791163in}}{\pgfqpoint{8.236601in}{2.782330in}}{\pgfqpoint{8.236601in}{2.773122in}}%
\pgfpathcurveto{\pgfqpoint{8.236601in}{2.763914in}}{\pgfqpoint{8.240259in}{2.755081in}}{\pgfqpoint{8.246770in}{2.748570in}}%
\pgfpathcurveto{\pgfqpoint{8.253282in}{2.742058in}}{\pgfqpoint{8.262114in}{2.738400in}}{\pgfqpoint{8.271323in}{2.738400in}}%
\pgfpathlineto{\pgfqpoint{8.271323in}{2.738400in}}%
\pgfpathclose%
\pgfusepath{stroke,fill}%
\end{pgfscope}%
\begin{pgfscope}%
\pgfpathrectangle{\pgfqpoint{0.050000in}{0.050000in}}{\pgfqpoint{2.419000in}{2.419000in}}%
\pgfusepath{clip}%
\pgfsetbuttcap%
\pgfsetroundjoin%
\definecolor{currentfill}{rgb}{0.200000,0.133333,0.533333}%
\pgfsetfillcolor{currentfill}%
\pgfsetfillopacity{0.373192}%
\pgfsetlinewidth{1.003750pt}%
\definecolor{currentstroke}{rgb}{0.200000,0.133333,0.533333}%
\pgfsetstrokecolor{currentstroke}%
\pgfsetstrokeopacity{0.373192}%
\pgfsetdash{}{0pt}%
\pgfpathmoveto{\pgfqpoint{-0.973324in}{2.653943in}}%
\pgfpathcurveto{\pgfqpoint{-0.964115in}{2.653943in}}{\pgfqpoint{-0.955283in}{2.657602in}}{\pgfqpoint{-0.948771in}{2.664113in}}%
\pgfpathcurveto{\pgfqpoint{-0.942260in}{2.670624in}}{\pgfqpoint{-0.938601in}{2.679457in}}{\pgfqpoint{-0.938601in}{2.688665in}}%
\pgfpathcurveto{\pgfqpoint{-0.938601in}{2.697874in}}{\pgfqpoint{-0.942260in}{2.706706in}}{\pgfqpoint{-0.948771in}{2.713218in}}%
\pgfpathcurveto{\pgfqpoint{-0.955283in}{2.719729in}}{\pgfqpoint{-0.964115in}{2.723388in}}{\pgfqpoint{-0.973324in}{2.723388in}}%
\pgfpathcurveto{\pgfqpoint{-0.982532in}{2.723388in}}{\pgfqpoint{-0.991365in}{2.719729in}}{\pgfqpoint{-0.997876in}{2.713218in}}%
\pgfpathcurveto{\pgfqpoint{-1.004387in}{2.706706in}}{\pgfqpoint{-1.008046in}{2.697874in}}{\pgfqpoint{-1.008046in}{2.688665in}}%
\pgfpathcurveto{\pgfqpoint{-1.008046in}{2.679457in}}{\pgfqpoint{-1.004387in}{2.670624in}}{\pgfqpoint{-0.997876in}{2.664113in}}%
\pgfpathcurveto{\pgfqpoint{-0.991365in}{2.657602in}}{\pgfqpoint{-0.982532in}{2.653943in}}{\pgfqpoint{-0.973324in}{2.653943in}}%
\pgfpathlineto{\pgfqpoint{-0.973324in}{2.653943in}}%
\pgfpathclose%
\pgfusepath{stroke,fill}%
\end{pgfscope}%
\begin{pgfscope}%
\pgfpathrectangle{\pgfqpoint{0.050000in}{0.050000in}}{\pgfqpoint{2.419000in}{2.419000in}}%
\pgfusepath{clip}%
\pgfsetbuttcap%
\pgfsetroundjoin%
\definecolor{currentfill}{rgb}{0.200000,0.133333,0.533333}%
\pgfsetfillcolor{currentfill}%
\pgfsetfillopacity{0.373192}%
\pgfsetlinewidth{1.003750pt}%
\definecolor{currentstroke}{rgb}{0.200000,0.133333,0.533333}%
\pgfsetstrokecolor{currentstroke}%
\pgfsetstrokeopacity{0.373192}%
\pgfsetdash{}{0pt}%
\pgfpathmoveto{\pgfqpoint{2.908974in}{2.653943in}}%
\pgfpathcurveto{\pgfqpoint{2.918183in}{2.653943in}}{\pgfqpoint{2.927015in}{2.657602in}}{\pgfqpoint{2.933527in}{2.664113in}}%
\pgfpathcurveto{\pgfqpoint{2.940038in}{2.670624in}}{\pgfqpoint{2.943697in}{2.679457in}}{\pgfqpoint{2.943697in}{2.688665in}}%
\pgfpathcurveto{\pgfqpoint{2.943697in}{2.697874in}}{\pgfqpoint{2.940038in}{2.706706in}}{\pgfqpoint{2.933527in}{2.713218in}}%
\pgfpathcurveto{\pgfqpoint{2.927015in}{2.719729in}}{\pgfqpoint{2.918183in}{2.723388in}}{\pgfqpoint{2.908974in}{2.723388in}}%
\pgfpathcurveto{\pgfqpoint{2.899766in}{2.723388in}}{\pgfqpoint{2.890933in}{2.719729in}}{\pgfqpoint{2.884422in}{2.713218in}}%
\pgfpathcurveto{\pgfqpoint{2.877911in}{2.706706in}}{\pgfqpoint{2.874252in}{2.697874in}}{\pgfqpoint{2.874252in}{2.688665in}}%
\pgfpathcurveto{\pgfqpoint{2.874252in}{2.679457in}}{\pgfqpoint{2.877911in}{2.670624in}}{\pgfqpoint{2.884422in}{2.664113in}}%
\pgfpathcurveto{\pgfqpoint{2.890933in}{2.657602in}}{\pgfqpoint{2.899766in}{2.653943in}}{\pgfqpoint{2.908974in}{2.653943in}}%
\pgfpathlineto{\pgfqpoint{2.908974in}{2.653943in}}%
\pgfpathclose%
\pgfusepath{stroke,fill}%
\end{pgfscope}%
\begin{pgfscope}%
\pgfpathrectangle{\pgfqpoint{0.050000in}{0.050000in}}{\pgfqpoint{2.419000in}{2.419000in}}%
\pgfusepath{clip}%
\pgfsetbuttcap%
\pgfsetroundjoin%
\definecolor{currentfill}{rgb}{0.200000,0.133333,0.533333}%
\pgfsetfillcolor{currentfill}%
\pgfsetfillopacity{0.373192}%
\pgfsetlinewidth{1.003750pt}%
\definecolor{currentstroke}{rgb}{0.200000,0.133333,0.533333}%
\pgfsetstrokecolor{currentstroke}%
\pgfsetstrokeopacity{0.373192}%
\pgfsetdash{}{0pt}%
\pgfpathmoveto{\pgfqpoint{6.791272in}{2.653943in}}%
\pgfpathcurveto{\pgfqpoint{6.800481in}{2.653943in}}{\pgfqpoint{6.809313in}{2.657602in}}{\pgfqpoint{6.815825in}{2.664113in}}%
\pgfpathcurveto{\pgfqpoint{6.822336in}{2.670624in}}{\pgfqpoint{6.825995in}{2.679457in}}{\pgfqpoint{6.825995in}{2.688665in}}%
\pgfpathcurveto{\pgfqpoint{6.825995in}{2.697874in}}{\pgfqpoint{6.822336in}{2.706706in}}{\pgfqpoint{6.815825in}{2.713218in}}%
\pgfpathcurveto{\pgfqpoint{6.809313in}{2.719729in}}{\pgfqpoint{6.800481in}{2.723388in}}{\pgfqpoint{6.791272in}{2.723388in}}%
\pgfpathcurveto{\pgfqpoint{6.782064in}{2.723388in}}{\pgfqpoint{6.773231in}{2.719729in}}{\pgfqpoint{6.766720in}{2.713218in}}%
\pgfpathcurveto{\pgfqpoint{6.760209in}{2.706706in}}{\pgfqpoint{6.756550in}{2.697874in}}{\pgfqpoint{6.756550in}{2.688665in}}%
\pgfpathcurveto{\pgfqpoint{6.756550in}{2.679457in}}{\pgfqpoint{6.760209in}{2.670624in}}{\pgfqpoint{6.766720in}{2.664113in}}%
\pgfpathcurveto{\pgfqpoint{6.773231in}{2.657602in}}{\pgfqpoint{6.782064in}{2.653943in}}{\pgfqpoint{6.791272in}{2.653943in}}%
\pgfpathlineto{\pgfqpoint{6.791272in}{2.653943in}}%
\pgfpathclose%
\pgfusepath{stroke,fill}%
\end{pgfscope}%
\begin{pgfscope}%
\pgfpathrectangle{\pgfqpoint{0.050000in}{0.050000in}}{\pgfqpoint{2.419000in}{2.419000in}}%
\pgfusepath{clip}%
\pgfsetbuttcap%
\pgfsetroundjoin%
\definecolor{currentfill}{rgb}{0.200000,0.133333,0.533333}%
\pgfsetfillcolor{currentfill}%
\pgfsetfillopacity{0.377317}%
\pgfsetlinewidth{1.003750pt}%
\definecolor{currentstroke}{rgb}{0.200000,0.133333,0.533333}%
\pgfsetstrokecolor{currentstroke}%
\pgfsetstrokeopacity{0.377317}%
\pgfsetdash{}{0pt}%
\pgfpathmoveto{\pgfqpoint{1.332350in}{2.567102in}}%
\pgfpathcurveto{\pgfqpoint{1.341558in}{2.567102in}}{\pgfqpoint{1.350391in}{2.570761in}}{\pgfqpoint{1.356902in}{2.577272in}}%
\pgfpathcurveto{\pgfqpoint{1.363413in}{2.583784in}}{\pgfqpoint{1.367072in}{2.592616in}}{\pgfqpoint{1.367072in}{2.601825in}}%
\pgfpathcurveto{\pgfqpoint{1.367072in}{2.611033in}}{\pgfqpoint{1.363413in}{2.619866in}}{\pgfqpoint{1.356902in}{2.626377in}}%
\pgfpathcurveto{\pgfqpoint{1.350391in}{2.632888in}}{\pgfqpoint{1.341558in}{2.636547in}}{\pgfqpoint{1.332350in}{2.636547in}}%
\pgfpathcurveto{\pgfqpoint{1.323141in}{2.636547in}}{\pgfqpoint{1.314309in}{2.632888in}}{\pgfqpoint{1.307797in}{2.626377in}}%
\pgfpathcurveto{\pgfqpoint{1.301286in}{2.619866in}}{\pgfqpoint{1.297627in}{2.611033in}}{\pgfqpoint{1.297627in}{2.601825in}}%
\pgfpathcurveto{\pgfqpoint{1.297627in}{2.592616in}}{\pgfqpoint{1.301286in}{2.583784in}}{\pgfqpoint{1.307797in}{2.577272in}}%
\pgfpathcurveto{\pgfqpoint{1.314309in}{2.570761in}}{\pgfqpoint{1.323141in}{2.567102in}}{\pgfqpoint{1.332350in}{2.567102in}}%
\pgfpathlineto{\pgfqpoint{1.332350in}{2.567102in}}%
\pgfpathclose%
\pgfusepath{stroke,fill}%
\end{pgfscope}%
\begin{pgfscope}%
\pgfpathrectangle{\pgfqpoint{0.050000in}{0.050000in}}{\pgfqpoint{2.419000in}{2.419000in}}%
\pgfusepath{clip}%
\pgfsetbuttcap%
\pgfsetroundjoin%
\definecolor{currentfill}{rgb}{0.200000,0.133333,0.533333}%
\pgfsetfillcolor{currentfill}%
\pgfsetfillopacity{0.377317}%
\pgfsetlinewidth{1.003750pt}%
\definecolor{currentstroke}{rgb}{0.200000,0.133333,0.533333}%
\pgfsetstrokecolor{currentstroke}%
\pgfsetstrokeopacity{0.377317}%
\pgfsetdash{}{0pt}%
\pgfpathmoveto{\pgfqpoint{5.269443in}{2.567102in}}%
\pgfpathcurveto{\pgfqpoint{5.278651in}{2.567102in}}{\pgfqpoint{5.287484in}{2.570761in}}{\pgfqpoint{5.293995in}{2.577272in}}%
\pgfpathcurveto{\pgfqpoint{5.300506in}{2.583784in}}{\pgfqpoint{5.304165in}{2.592616in}}{\pgfqpoint{5.304165in}{2.601825in}}%
\pgfpathcurveto{\pgfqpoint{5.304165in}{2.611033in}}{\pgfqpoint{5.300506in}{2.619866in}}{\pgfqpoint{5.293995in}{2.626377in}}%
\pgfpathcurveto{\pgfqpoint{5.287484in}{2.632888in}}{\pgfqpoint{5.278651in}{2.636547in}}{\pgfqpoint{5.269443in}{2.636547in}}%
\pgfpathcurveto{\pgfqpoint{5.260234in}{2.636547in}}{\pgfqpoint{5.251402in}{2.632888in}}{\pgfqpoint{5.244890in}{2.626377in}}%
\pgfpathcurveto{\pgfqpoint{5.238379in}{2.619866in}}{\pgfqpoint{5.234721in}{2.611033in}}{\pgfqpoint{5.234721in}{2.601825in}}%
\pgfpathcurveto{\pgfqpoint{5.234721in}{2.592616in}}{\pgfqpoint{5.238379in}{2.583784in}}{\pgfqpoint{5.244890in}{2.577272in}}%
\pgfpathcurveto{\pgfqpoint{5.251402in}{2.570761in}}{\pgfqpoint{5.260234in}{2.567102in}}{\pgfqpoint{5.269443in}{2.567102in}}%
\pgfpathlineto{\pgfqpoint{5.269443in}{2.567102in}}%
\pgfpathclose%
\pgfusepath{stroke,fill}%
\end{pgfscope}%
\begin{pgfscope}%
\pgfpathrectangle{\pgfqpoint{0.050000in}{0.050000in}}{\pgfqpoint{2.419000in}{2.419000in}}%
\pgfusepath{clip}%
\pgfsetbuttcap%
\pgfsetroundjoin%
\definecolor{currentfill}{rgb}{0.200000,0.133333,0.533333}%
\pgfsetfillcolor{currentfill}%
\pgfsetfillopacity{0.377317}%
\pgfsetlinewidth{1.003750pt}%
\definecolor{currentstroke}{rgb}{0.200000,0.133333,0.533333}%
\pgfsetstrokecolor{currentstroke}%
\pgfsetstrokeopacity{0.377317}%
\pgfsetdash{}{0pt}%
\pgfpathmoveto{\pgfqpoint{9.206536in}{2.567102in}}%
\pgfpathcurveto{\pgfqpoint{9.215744in}{2.567102in}}{\pgfqpoint{9.224577in}{2.570761in}}{\pgfqpoint{9.231088in}{2.577272in}}%
\pgfpathcurveto{\pgfqpoint{9.237600in}{2.583784in}}{\pgfqpoint{9.241258in}{2.592616in}}{\pgfqpoint{9.241258in}{2.601825in}}%
\pgfpathcurveto{\pgfqpoint{9.241258in}{2.611033in}}{\pgfqpoint{9.237600in}{2.619866in}}{\pgfqpoint{9.231088in}{2.626377in}}%
\pgfpathcurveto{\pgfqpoint{9.224577in}{2.632888in}}{\pgfqpoint{9.215744in}{2.636547in}}{\pgfqpoint{9.206536in}{2.636547in}}%
\pgfpathcurveto{\pgfqpoint{9.197327in}{2.636547in}}{\pgfqpoint{9.188495in}{2.632888in}}{\pgfqpoint{9.181984in}{2.626377in}}%
\pgfpathcurveto{\pgfqpoint{9.175472in}{2.619866in}}{\pgfqpoint{9.171814in}{2.611033in}}{\pgfqpoint{9.171814in}{2.601825in}}%
\pgfpathcurveto{\pgfqpoint{9.171814in}{2.592616in}}{\pgfqpoint{9.175472in}{2.583784in}}{\pgfqpoint{9.181984in}{2.577272in}}%
\pgfpathcurveto{\pgfqpoint{9.188495in}{2.570761in}}{\pgfqpoint{9.197327in}{2.567102in}}{\pgfqpoint{9.206536in}{2.567102in}}%
\pgfpathlineto{\pgfqpoint{9.206536in}{2.567102in}}%
\pgfpathclose%
\pgfusepath{stroke,fill}%
\end{pgfscope}%
\begin{pgfscope}%
\pgfpathrectangle{\pgfqpoint{0.050000in}{0.050000in}}{\pgfqpoint{2.419000in}{2.419000in}}%
\pgfusepath{clip}%
\pgfsetbuttcap%
\pgfsetroundjoin%
\definecolor{currentfill}{rgb}{0.200000,0.133333,0.533333}%
\pgfsetfillcolor{currentfill}%
\pgfsetfillopacity{0.381561}%
\pgfsetlinewidth{1.003750pt}%
\definecolor{currentstroke}{rgb}{0.200000,0.133333,0.533333}%
\pgfsetstrokecolor{currentstroke}%
\pgfsetstrokeopacity{0.381561}%
\pgfsetdash{}{0pt}%
\pgfpathmoveto{\pgfqpoint{3.704040in}{2.477775in}}%
\pgfpathcurveto{\pgfqpoint{3.713248in}{2.477775in}}{\pgfqpoint{3.722081in}{2.481434in}}{\pgfqpoint{3.728592in}{2.487945in}}%
\pgfpathcurveto{\pgfqpoint{3.735103in}{2.494456in}}{\pgfqpoint{3.738762in}{2.503289in}}{\pgfqpoint{3.738762in}{2.512497in}}%
\pgfpathcurveto{\pgfqpoint{3.738762in}{2.521706in}}{\pgfqpoint{3.735103in}{2.530538in}}{\pgfqpoint{3.728592in}{2.537050in}}%
\pgfpathcurveto{\pgfqpoint{3.722081in}{2.543561in}}{\pgfqpoint{3.713248in}{2.547220in}}{\pgfqpoint{3.704040in}{2.547220in}}%
\pgfpathcurveto{\pgfqpoint{3.694831in}{2.547220in}}{\pgfqpoint{3.685999in}{2.543561in}}{\pgfqpoint{3.679487in}{2.537050in}}%
\pgfpathcurveto{\pgfqpoint{3.672976in}{2.530538in}}{\pgfqpoint{3.669318in}{2.521706in}}{\pgfqpoint{3.669318in}{2.512497in}}%
\pgfpathcurveto{\pgfqpoint{3.669318in}{2.503289in}}{\pgfqpoint{3.672976in}{2.494456in}}{\pgfqpoint{3.679487in}{2.487945in}}%
\pgfpathcurveto{\pgfqpoint{3.685999in}{2.481434in}}{\pgfqpoint{3.694831in}{2.477775in}}{\pgfqpoint{3.704040in}{2.477775in}}%
\pgfpathlineto{\pgfqpoint{3.704040in}{2.477775in}}%
\pgfpathclose%
\pgfusepath{stroke,fill}%
\end{pgfscope}%
\begin{pgfscope}%
\pgfpathrectangle{\pgfqpoint{0.050000in}{0.050000in}}{\pgfqpoint{2.419000in}{2.419000in}}%
\pgfusepath{clip}%
\pgfsetbuttcap%
\pgfsetroundjoin%
\definecolor{currentfill}{rgb}{0.200000,0.133333,0.533333}%
\pgfsetfillcolor{currentfill}%
\pgfsetfillopacity{0.381561}%
\pgfsetlinewidth{1.003750pt}%
\definecolor{currentstroke}{rgb}{0.200000,0.133333,0.533333}%
\pgfsetstrokecolor{currentstroke}%
\pgfsetstrokeopacity{0.381561}%
\pgfsetdash{}{0pt}%
\pgfpathmoveto{\pgfqpoint{-0.289417in}{2.477775in}}%
\pgfpathcurveto{\pgfqpoint{-0.280209in}{2.477775in}}{\pgfqpoint{-0.271376in}{2.481434in}}{\pgfqpoint{-0.264865in}{2.487945in}}%
\pgfpathcurveto{\pgfqpoint{-0.258354in}{2.494456in}}{\pgfqpoint{-0.254695in}{2.503289in}}{\pgfqpoint{-0.254695in}{2.512497in}}%
\pgfpathcurveto{\pgfqpoint{-0.254695in}{2.521706in}}{\pgfqpoint{-0.258354in}{2.530538in}}{\pgfqpoint{-0.264865in}{2.537050in}}%
\pgfpathcurveto{\pgfqpoint{-0.271376in}{2.543561in}}{\pgfqpoint{-0.280209in}{2.547220in}}{\pgfqpoint{-0.289417in}{2.547220in}}%
\pgfpathcurveto{\pgfqpoint{-0.298626in}{2.547220in}}{\pgfqpoint{-0.307458in}{2.543561in}}{\pgfqpoint{-0.313970in}{2.537050in}}%
\pgfpathcurveto{\pgfqpoint{-0.320481in}{2.530538in}}{\pgfqpoint{-0.324140in}{2.521706in}}{\pgfqpoint{-0.324140in}{2.512497in}}%
\pgfpathcurveto{\pgfqpoint{-0.324140in}{2.503289in}}{\pgfqpoint{-0.320481in}{2.494456in}}{\pgfqpoint{-0.313970in}{2.487945in}}%
\pgfpathcurveto{\pgfqpoint{-0.307458in}{2.481434in}}{\pgfqpoint{-0.298626in}{2.477775in}}{\pgfqpoint{-0.289417in}{2.477775in}}%
\pgfpathlineto{\pgfqpoint{-0.289417in}{2.477775in}}%
\pgfpathclose%
\pgfusepath{stroke,fill}%
\end{pgfscope}%
\begin{pgfscope}%
\pgfpathrectangle{\pgfqpoint{0.050000in}{0.050000in}}{\pgfqpoint{2.419000in}{2.419000in}}%
\pgfusepath{clip}%
\pgfsetbuttcap%
\pgfsetroundjoin%
\definecolor{currentfill}{rgb}{0.200000,0.133333,0.533333}%
\pgfsetfillcolor{currentfill}%
\pgfsetfillopacity{0.381561}%
\pgfsetlinewidth{1.003750pt}%
\definecolor{currentstroke}{rgb}{0.200000,0.133333,0.533333}%
\pgfsetstrokecolor{currentstroke}%
\pgfsetstrokeopacity{0.381561}%
\pgfsetdash{}{0pt}%
\pgfpathmoveto{\pgfqpoint{7.697497in}{2.477775in}}%
\pgfpathcurveto{\pgfqpoint{7.706705in}{2.477775in}}{\pgfqpoint{7.715538in}{2.481434in}}{\pgfqpoint{7.722049in}{2.487945in}}%
\pgfpathcurveto{\pgfqpoint{7.728561in}{2.494456in}}{\pgfqpoint{7.732219in}{2.503289in}}{\pgfqpoint{7.732219in}{2.512497in}}%
\pgfpathcurveto{\pgfqpoint{7.732219in}{2.521706in}}{\pgfqpoint{7.728561in}{2.530538in}}{\pgfqpoint{7.722049in}{2.537050in}}%
\pgfpathcurveto{\pgfqpoint{7.715538in}{2.543561in}}{\pgfqpoint{7.706705in}{2.547220in}}{\pgfqpoint{7.697497in}{2.547220in}}%
\pgfpathcurveto{\pgfqpoint{7.688288in}{2.547220in}}{\pgfqpoint{7.679456in}{2.543561in}}{\pgfqpoint{7.672945in}{2.537050in}}%
\pgfpathcurveto{\pgfqpoint{7.666433in}{2.530538in}}{\pgfqpoint{7.662775in}{2.521706in}}{\pgfqpoint{7.662775in}{2.512497in}}%
\pgfpathcurveto{\pgfqpoint{7.662775in}{2.503289in}}{\pgfqpoint{7.666433in}{2.494456in}}{\pgfqpoint{7.672945in}{2.487945in}}%
\pgfpathcurveto{\pgfqpoint{7.679456in}{2.481434in}}{\pgfqpoint{7.688288in}{2.477775in}}{\pgfqpoint{7.697497in}{2.477775in}}%
\pgfpathlineto{\pgfqpoint{7.697497in}{2.477775in}}%
\pgfpathclose%
\pgfusepath{stroke,fill}%
\end{pgfscope}%
\begin{pgfscope}%
\pgfpathrectangle{\pgfqpoint{0.050000in}{0.050000in}}{\pgfqpoint{2.419000in}{2.419000in}}%
\pgfusepath{clip}%
\pgfsetbuttcap%
\pgfsetroundjoin%
\definecolor{currentfill}{rgb}{0.200000,0.133333,0.533333}%
\pgfsetfillcolor{currentfill}%
\pgfsetfillopacity{0.385928}%
\pgfsetlinewidth{1.003750pt}%
\definecolor{currentstroke}{rgb}{0.200000,0.133333,0.533333}%
\pgfsetstrokecolor{currentstroke}%
\pgfsetstrokeopacity{0.385928}%
\pgfsetdash{}{0pt}%
\pgfpathmoveto{\pgfqpoint{6.144623in}{2.385853in}}%
\pgfpathcurveto{\pgfqpoint{6.153832in}{2.385853in}}{\pgfqpoint{6.162664in}{2.389512in}}{\pgfqpoint{6.169175in}{2.396023in}}%
\pgfpathcurveto{\pgfqpoint{6.175687in}{2.402535in}}{\pgfqpoint{6.179345in}{2.411367in}}{\pgfqpoint{6.179345in}{2.420575in}}%
\pgfpathcurveto{\pgfqpoint{6.179345in}{2.429784in}}{\pgfqpoint{6.175687in}{2.438616in}}{\pgfqpoint{6.169175in}{2.445128in}}%
\pgfpathcurveto{\pgfqpoint{6.162664in}{2.451639in}}{\pgfqpoint{6.153832in}{2.455298in}}{\pgfqpoint{6.144623in}{2.455298in}}%
\pgfpathcurveto{\pgfqpoint{6.135415in}{2.455298in}}{\pgfqpoint{6.126582in}{2.451639in}}{\pgfqpoint{6.120071in}{2.445128in}}%
\pgfpathcurveto{\pgfqpoint{6.113559in}{2.438616in}}{\pgfqpoint{6.109901in}{2.429784in}}{\pgfqpoint{6.109901in}{2.420575in}}%
\pgfpathcurveto{\pgfqpoint{6.109901in}{2.411367in}}{\pgfqpoint{6.113559in}{2.402535in}}{\pgfqpoint{6.120071in}{2.396023in}}%
\pgfpathcurveto{\pgfqpoint{6.126582in}{2.389512in}}{\pgfqpoint{6.135415in}{2.385853in}}{\pgfqpoint{6.144623in}{2.385853in}}%
\pgfpathlineto{\pgfqpoint{6.144623in}{2.385853in}}%
\pgfpathclose%
\pgfusepath{stroke,fill}%
\end{pgfscope}%
\begin{pgfscope}%
\pgfpathrectangle{\pgfqpoint{0.050000in}{0.050000in}}{\pgfqpoint{2.419000in}{2.419000in}}%
\pgfusepath{clip}%
\pgfsetbuttcap%
\pgfsetroundjoin%
\definecolor{currentfill}{rgb}{0.200000,0.133333,0.533333}%
\pgfsetfillcolor{currentfill}%
\pgfsetfillopacity{0.385928}%
\pgfsetlinewidth{1.003750pt}%
\definecolor{currentstroke}{rgb}{0.200000,0.133333,0.533333}%
\pgfsetstrokecolor{currentstroke}%
\pgfsetstrokeopacity{0.385928}%
\pgfsetdash{}{0pt}%
\pgfpathmoveto{\pgfqpoint{2.093165in}{2.385853in}}%
\pgfpathcurveto{\pgfqpoint{2.102373in}{2.385853in}}{\pgfqpoint{2.111206in}{2.389512in}}{\pgfqpoint{2.117717in}{2.396023in}}%
\pgfpathcurveto{\pgfqpoint{2.124228in}{2.402535in}}{\pgfqpoint{2.127887in}{2.411367in}}{\pgfqpoint{2.127887in}{2.420575in}}%
\pgfpathcurveto{\pgfqpoint{2.127887in}{2.429784in}}{\pgfqpoint{2.124228in}{2.438616in}}{\pgfqpoint{2.117717in}{2.445128in}}%
\pgfpathcurveto{\pgfqpoint{2.111206in}{2.451639in}}{\pgfqpoint{2.102373in}{2.455298in}}{\pgfqpoint{2.093165in}{2.455298in}}%
\pgfpathcurveto{\pgfqpoint{2.083956in}{2.455298in}}{\pgfqpoint{2.075124in}{2.451639in}}{\pgfqpoint{2.068612in}{2.445128in}}%
\pgfpathcurveto{\pgfqpoint{2.062101in}{2.438616in}}{\pgfqpoint{2.058443in}{2.429784in}}{\pgfqpoint{2.058443in}{2.420575in}}%
\pgfpathcurveto{\pgfqpoint{2.058443in}{2.411367in}}{\pgfqpoint{2.062101in}{2.402535in}}{\pgfqpoint{2.068612in}{2.396023in}}%
\pgfpathcurveto{\pgfqpoint{2.075124in}{2.389512in}}{\pgfqpoint{2.083956in}{2.385853in}}{\pgfqpoint{2.093165in}{2.385853in}}%
\pgfpathlineto{\pgfqpoint{2.093165in}{2.385853in}}%
\pgfpathclose%
\pgfusepath{stroke,fill}%
\end{pgfscope}%
\begin{pgfscope}%
\pgfpathrectangle{\pgfqpoint{0.050000in}{0.050000in}}{\pgfqpoint{2.419000in}{2.419000in}}%
\pgfusepath{clip}%
\pgfsetbuttcap%
\pgfsetroundjoin%
\definecolor{currentfill}{rgb}{0.200000,0.133333,0.533333}%
\pgfsetfillcolor{currentfill}%
\pgfsetfillopacity{0.385928}%
\pgfsetlinewidth{1.003750pt}%
\definecolor{currentstroke}{rgb}{0.200000,0.133333,0.533333}%
\pgfsetstrokecolor{currentstroke}%
\pgfsetstrokeopacity{0.385928}%
\pgfsetdash{}{0pt}%
\pgfpathmoveto{\pgfqpoint{10.196082in}{2.385853in}}%
\pgfpathcurveto{\pgfqpoint{10.205290in}{2.385853in}}{\pgfqpoint{10.214122in}{2.389512in}}{\pgfqpoint{10.220634in}{2.396023in}}%
\pgfpathcurveto{\pgfqpoint{10.227145in}{2.402535in}}{\pgfqpoint{10.230804in}{2.411367in}}{\pgfqpoint{10.230804in}{2.420575in}}%
\pgfpathcurveto{\pgfqpoint{10.230804in}{2.429784in}}{\pgfqpoint{10.227145in}{2.438616in}}{\pgfqpoint{10.220634in}{2.445128in}}%
\pgfpathcurveto{\pgfqpoint{10.214122in}{2.451639in}}{\pgfqpoint{10.205290in}{2.455298in}}{\pgfqpoint{10.196082in}{2.455298in}}%
\pgfpathcurveto{\pgfqpoint{10.186873in}{2.455298in}}{\pgfqpoint{10.178041in}{2.451639in}}{\pgfqpoint{10.171529in}{2.445128in}}%
\pgfpathcurveto{\pgfqpoint{10.165018in}{2.438616in}}{\pgfqpoint{10.161359in}{2.429784in}}{\pgfqpoint{10.161359in}{2.420575in}}%
\pgfpathcurveto{\pgfqpoint{10.161359in}{2.411367in}}{\pgfqpoint{10.165018in}{2.402535in}}{\pgfqpoint{10.171529in}{2.396023in}}%
\pgfpathcurveto{\pgfqpoint{10.178041in}{2.389512in}}{\pgfqpoint{10.186873in}{2.385853in}}{\pgfqpoint{10.196082in}{2.385853in}}%
\pgfpathlineto{\pgfqpoint{10.196082in}{2.385853in}}%
\pgfpathclose%
\pgfusepath{stroke,fill}%
\end{pgfscope}%
\begin{pgfscope}%
\pgfpathrectangle{\pgfqpoint{0.050000in}{0.050000in}}{\pgfqpoint{2.419000in}{2.419000in}}%
\pgfusepath{clip}%
\pgfsetbuttcap%
\pgfsetroundjoin%
\definecolor{currentfill}{rgb}{0.200000,0.133333,0.533333}%
\pgfsetfillcolor{currentfill}%
\pgfsetfillopacity{0.390424}%
\pgfsetlinewidth{1.003750pt}%
\definecolor{currentstroke}{rgb}{0.200000,0.133333,0.533333}%
\pgfsetstrokecolor{currentstroke}%
\pgfsetstrokeopacity{0.390424}%
\pgfsetdash{}{0pt}%
\pgfpathmoveto{\pgfqpoint{0.434807in}{2.291222in}}%
\pgfpathcurveto{\pgfqpoint{0.444016in}{2.291222in}}{\pgfqpoint{0.452848in}{2.294880in}}{\pgfqpoint{0.459359in}{2.301392in}}%
\pgfpathcurveto{\pgfqpoint{0.465871in}{2.307903in}}{\pgfqpoint{0.469529in}{2.316736in}}{\pgfqpoint{0.469529in}{2.325944in}}%
\pgfpathcurveto{\pgfqpoint{0.469529in}{2.335152in}}{\pgfqpoint{0.465871in}{2.343985in}}{\pgfqpoint{0.459359in}{2.350496in}}%
\pgfpathcurveto{\pgfqpoint{0.452848in}{2.357008in}}{\pgfqpoint{0.444016in}{2.360666in}}{\pgfqpoint{0.434807in}{2.360666in}}%
\pgfpathcurveto{\pgfqpoint{0.425599in}{2.360666in}}{\pgfqpoint{0.416766in}{2.357008in}}{\pgfqpoint{0.410255in}{2.350496in}}%
\pgfpathcurveto{\pgfqpoint{0.403743in}{2.343985in}}{\pgfqpoint{0.400085in}{2.335152in}}{\pgfqpoint{0.400085in}{2.325944in}}%
\pgfpathcurveto{\pgfqpoint{0.400085in}{2.316736in}}{\pgfqpoint{0.403743in}{2.307903in}}{\pgfqpoint{0.410255in}{2.301392in}}%
\pgfpathcurveto{\pgfqpoint{0.416766in}{2.294880in}}{\pgfqpoint{0.425599in}{2.291222in}}{\pgfqpoint{0.434807in}{2.291222in}}%
\pgfpathlineto{\pgfqpoint{0.434807in}{2.291222in}}%
\pgfpathclose%
\pgfusepath{stroke,fill}%
\end{pgfscope}%
\begin{pgfscope}%
\pgfpathrectangle{\pgfqpoint{0.050000in}{0.050000in}}{\pgfqpoint{2.419000in}{2.419000in}}%
\pgfusepath{clip}%
\pgfsetbuttcap%
\pgfsetroundjoin%
\definecolor{currentfill}{rgb}{0.200000,0.133333,0.533333}%
\pgfsetfillcolor{currentfill}%
\pgfsetfillopacity{0.390424}%
\pgfsetlinewidth{1.003750pt}%
\definecolor{currentstroke}{rgb}{0.200000,0.133333,0.533333}%
\pgfsetstrokecolor{currentstroke}%
\pgfsetstrokeopacity{0.390424}%
\pgfsetdash{}{0pt}%
\pgfpathmoveto{\pgfqpoint{4.545976in}{2.291222in}}%
\pgfpathcurveto{\pgfqpoint{4.555185in}{2.291222in}}{\pgfqpoint{4.564017in}{2.294880in}}{\pgfqpoint{4.570529in}{2.301392in}}%
\pgfpathcurveto{\pgfqpoint{4.577040in}{2.307903in}}{\pgfqpoint{4.580699in}{2.316736in}}{\pgfqpoint{4.580699in}{2.325944in}}%
\pgfpathcurveto{\pgfqpoint{4.580699in}{2.335152in}}{\pgfqpoint{4.577040in}{2.343985in}}{\pgfqpoint{4.570529in}{2.350496in}}%
\pgfpathcurveto{\pgfqpoint{4.564017in}{2.357008in}}{\pgfqpoint{4.555185in}{2.360666in}}{\pgfqpoint{4.545976in}{2.360666in}}%
\pgfpathcurveto{\pgfqpoint{4.536768in}{2.360666in}}{\pgfqpoint{4.527935in}{2.357008in}}{\pgfqpoint{4.521424in}{2.350496in}}%
\pgfpathcurveto{\pgfqpoint{4.514913in}{2.343985in}}{\pgfqpoint{4.511254in}{2.335152in}}{\pgfqpoint{4.511254in}{2.325944in}}%
\pgfpathcurveto{\pgfqpoint{4.511254in}{2.316736in}}{\pgfqpoint{4.514913in}{2.307903in}}{\pgfqpoint{4.521424in}{2.301392in}}%
\pgfpathcurveto{\pgfqpoint{4.527935in}{2.294880in}}{\pgfqpoint{4.536768in}{2.291222in}}{\pgfqpoint{4.545976in}{2.291222in}}%
\pgfpathlineto{\pgfqpoint{4.545976in}{2.291222in}}%
\pgfpathclose%
\pgfusepath{stroke,fill}%
\end{pgfscope}%
\begin{pgfscope}%
\pgfpathrectangle{\pgfqpoint{0.050000in}{0.050000in}}{\pgfqpoint{2.419000in}{2.419000in}}%
\pgfusepath{clip}%
\pgfsetbuttcap%
\pgfsetroundjoin%
\definecolor{currentfill}{rgb}{0.200000,0.133333,0.533333}%
\pgfsetfillcolor{currentfill}%
\pgfsetfillopacity{0.390424}%
\pgfsetlinewidth{1.003750pt}%
\definecolor{currentstroke}{rgb}{0.200000,0.133333,0.533333}%
\pgfsetstrokecolor{currentstroke}%
\pgfsetstrokeopacity{0.390424}%
\pgfsetdash{}{0pt}%
\pgfpathmoveto{\pgfqpoint{8.657146in}{2.291222in}}%
\pgfpathcurveto{\pgfqpoint{8.666354in}{2.291222in}}{\pgfqpoint{8.675187in}{2.294880in}}{\pgfqpoint{8.681698in}{2.301392in}}%
\pgfpathcurveto{\pgfqpoint{8.688209in}{2.307903in}}{\pgfqpoint{8.691868in}{2.316736in}}{\pgfqpoint{8.691868in}{2.325944in}}%
\pgfpathcurveto{\pgfqpoint{8.691868in}{2.335152in}}{\pgfqpoint{8.688209in}{2.343985in}}{\pgfqpoint{8.681698in}{2.350496in}}%
\pgfpathcurveto{\pgfqpoint{8.675187in}{2.357008in}}{\pgfqpoint{8.666354in}{2.360666in}}{\pgfqpoint{8.657146in}{2.360666in}}%
\pgfpathcurveto{\pgfqpoint{8.647937in}{2.360666in}}{\pgfqpoint{8.639105in}{2.357008in}}{\pgfqpoint{8.632593in}{2.350496in}}%
\pgfpathcurveto{\pgfqpoint{8.626082in}{2.343985in}}{\pgfqpoint{8.622424in}{2.335152in}}{\pgfqpoint{8.622424in}{2.325944in}}%
\pgfpathcurveto{\pgfqpoint{8.622424in}{2.316736in}}{\pgfqpoint{8.626082in}{2.307903in}}{\pgfqpoint{8.632593in}{2.301392in}}%
\pgfpathcurveto{\pgfqpoint{8.639105in}{2.294880in}}{\pgfqpoint{8.647937in}{2.291222in}}{\pgfqpoint{8.657146in}{2.291222in}}%
\pgfpathlineto{\pgfqpoint{8.657146in}{2.291222in}}%
\pgfpathclose%
\pgfusepath{stroke,fill}%
\end{pgfscope}%
\begin{pgfscope}%
\pgfpathrectangle{\pgfqpoint{0.050000in}{0.050000in}}{\pgfqpoint{2.419000in}{2.419000in}}%
\pgfusepath{clip}%
\pgfsetbuttcap%
\pgfsetroundjoin%
\definecolor{currentfill}{rgb}{0.200000,0.133333,0.533333}%
\pgfsetfillcolor{currentfill}%
\pgfsetfillopacity{0.395055}%
\pgfsetlinewidth{1.003750pt}%
\definecolor{currentstroke}{rgb}{0.200000,0.133333,0.533333}%
\pgfsetstrokecolor{currentstroke}%
\pgfsetstrokeopacity{0.395055}%
\pgfsetdash{}{0pt}%
\pgfpathmoveto{\pgfqpoint{2.899503in}{2.193759in}}%
\pgfpathcurveto{\pgfqpoint{2.908711in}{2.193759in}}{\pgfqpoint{2.917544in}{2.197418in}}{\pgfqpoint{2.924055in}{2.203929in}}%
\pgfpathcurveto{\pgfqpoint{2.930566in}{2.210440in}}{\pgfqpoint{2.934225in}{2.219273in}}{\pgfqpoint{2.934225in}{2.228481in}}%
\pgfpathcurveto{\pgfqpoint{2.934225in}{2.237690in}}{\pgfqpoint{2.930566in}{2.246522in}}{\pgfqpoint{2.924055in}{2.253034in}}%
\pgfpathcurveto{\pgfqpoint{2.917544in}{2.259545in}}{\pgfqpoint{2.908711in}{2.263204in}}{\pgfqpoint{2.899503in}{2.263204in}}%
\pgfpathcurveto{\pgfqpoint{2.890294in}{2.263204in}}{\pgfqpoint{2.881462in}{2.259545in}}{\pgfqpoint{2.874950in}{2.253034in}}%
\pgfpathcurveto{\pgfqpoint{2.868439in}{2.246522in}}{\pgfqpoint{2.864780in}{2.237690in}}{\pgfqpoint{2.864780in}{2.228481in}}%
\pgfpathcurveto{\pgfqpoint{2.864780in}{2.219273in}}{\pgfqpoint{2.868439in}{2.210440in}}{\pgfqpoint{2.874950in}{2.203929in}}%
\pgfpathcurveto{\pgfqpoint{2.881462in}{2.197418in}}{\pgfqpoint{2.890294in}{2.193759in}}{\pgfqpoint{2.899503in}{2.193759in}}%
\pgfpathlineto{\pgfqpoint{2.899503in}{2.193759in}}%
\pgfpathclose%
\pgfusepath{stroke,fill}%
\end{pgfscope}%
\begin{pgfscope}%
\pgfpathrectangle{\pgfqpoint{0.050000in}{0.050000in}}{\pgfqpoint{2.419000in}{2.419000in}}%
\pgfusepath{clip}%
\pgfsetbuttcap%
\pgfsetroundjoin%
\definecolor{currentfill}{rgb}{0.200000,0.133333,0.533333}%
\pgfsetfillcolor{currentfill}%
\pgfsetfillopacity{0.395055}%
\pgfsetlinewidth{1.003750pt}%
\definecolor{currentstroke}{rgb}{0.200000,0.133333,0.533333}%
\pgfsetstrokecolor{currentstroke}%
\pgfsetstrokeopacity{0.395055}%
\pgfsetdash{}{0pt}%
\pgfpathmoveto{\pgfqpoint{-1.273164in}{2.193759in}}%
\pgfpathcurveto{\pgfqpoint{-1.263956in}{2.193759in}}{\pgfqpoint{-1.255123in}{2.197418in}}{\pgfqpoint{-1.248612in}{2.203929in}}%
\pgfpathcurveto{\pgfqpoint{-1.242100in}{2.210440in}}{\pgfqpoint{-1.238442in}{2.219273in}}{\pgfqpoint{-1.238442in}{2.228481in}}%
\pgfpathcurveto{\pgfqpoint{-1.238442in}{2.237690in}}{\pgfqpoint{-1.242100in}{2.246522in}}{\pgfqpoint{-1.248612in}{2.253034in}}%
\pgfpathcurveto{\pgfqpoint{-1.255123in}{2.259545in}}{\pgfqpoint{-1.263956in}{2.263204in}}{\pgfqpoint{-1.273164in}{2.263204in}}%
\pgfpathcurveto{\pgfqpoint{-1.282372in}{2.263204in}}{\pgfqpoint{-1.291205in}{2.259545in}}{\pgfqpoint{-1.297716in}{2.253034in}}%
\pgfpathcurveto{\pgfqpoint{-1.304228in}{2.246522in}}{\pgfqpoint{-1.307886in}{2.237690in}}{\pgfqpoint{-1.307886in}{2.228481in}}%
\pgfpathcurveto{\pgfqpoint{-1.307886in}{2.219273in}}{\pgfqpoint{-1.304228in}{2.210440in}}{\pgfqpoint{-1.297716in}{2.203929in}}%
\pgfpathcurveto{\pgfqpoint{-1.291205in}{2.197418in}}{\pgfqpoint{-1.282372in}{2.193759in}}{\pgfqpoint{-1.273164in}{2.193759in}}%
\pgfpathlineto{\pgfqpoint{-1.273164in}{2.193759in}}%
\pgfpathclose%
\pgfusepath{stroke,fill}%
\end{pgfscope}%
\begin{pgfscope}%
\pgfpathrectangle{\pgfqpoint{0.050000in}{0.050000in}}{\pgfqpoint{2.419000in}{2.419000in}}%
\pgfusepath{clip}%
\pgfsetbuttcap%
\pgfsetroundjoin%
\definecolor{currentfill}{rgb}{0.200000,0.133333,0.533333}%
\pgfsetfillcolor{currentfill}%
\pgfsetfillopacity{0.395055}%
\pgfsetlinewidth{1.003750pt}%
\definecolor{currentstroke}{rgb}{0.200000,0.133333,0.533333}%
\pgfsetstrokecolor{currentstroke}%
\pgfsetstrokeopacity{0.395055}%
\pgfsetdash{}{0pt}%
\pgfpathmoveto{\pgfqpoint{7.072169in}{2.193759in}}%
\pgfpathcurveto{\pgfqpoint{7.081378in}{2.193759in}}{\pgfqpoint{7.090210in}{2.197418in}}{\pgfqpoint{7.096722in}{2.203929in}}%
\pgfpathcurveto{\pgfqpoint{7.103233in}{2.210440in}}{\pgfqpoint{7.106892in}{2.219273in}}{\pgfqpoint{7.106892in}{2.228481in}}%
\pgfpathcurveto{\pgfqpoint{7.106892in}{2.237690in}}{\pgfqpoint{7.103233in}{2.246522in}}{\pgfqpoint{7.096722in}{2.253034in}}%
\pgfpathcurveto{\pgfqpoint{7.090210in}{2.259545in}}{\pgfqpoint{7.081378in}{2.263204in}}{\pgfqpoint{7.072169in}{2.263204in}}%
\pgfpathcurveto{\pgfqpoint{7.062961in}{2.263204in}}{\pgfqpoint{7.054128in}{2.259545in}}{\pgfqpoint{7.047617in}{2.253034in}}%
\pgfpathcurveto{\pgfqpoint{7.041106in}{2.246522in}}{\pgfqpoint{7.037447in}{2.237690in}}{\pgfqpoint{7.037447in}{2.228481in}}%
\pgfpathcurveto{\pgfqpoint{7.037447in}{2.219273in}}{\pgfqpoint{7.041106in}{2.210440in}}{\pgfqpoint{7.047617in}{2.203929in}}%
\pgfpathcurveto{\pgfqpoint{7.054128in}{2.197418in}}{\pgfqpoint{7.062961in}{2.193759in}}{\pgfqpoint{7.072169in}{2.193759in}}%
\pgfpathlineto{\pgfqpoint{7.072169in}{2.193759in}}%
\pgfpathclose%
\pgfusepath{stroke,fill}%
\end{pgfscope}%
\begin{pgfscope}%
\pgfpathrectangle{\pgfqpoint{0.050000in}{0.050000in}}{\pgfqpoint{2.419000in}{2.419000in}}%
\pgfusepath{clip}%
\pgfsetbuttcap%
\pgfsetroundjoin%
\definecolor{currentfill}{rgb}{0.200000,0.133333,0.533333}%
\pgfsetfillcolor{currentfill}%
\pgfsetfillopacity{0.399826}%
\pgfsetlinewidth{1.003750pt}%
\definecolor{currentstroke}{rgb}{0.200000,0.133333,0.533333}%
\pgfsetstrokecolor{currentstroke}%
\pgfsetstrokeopacity{0.399826}%
\pgfsetdash{}{0pt}%
\pgfpathmoveto{\pgfqpoint{1.203023in}{2.093337in}}%
\pgfpathcurveto{\pgfqpoint{1.212231in}{2.093337in}}{\pgfqpoint{1.221064in}{2.096995in}}{\pgfqpoint{1.227575in}{2.103506in}}%
\pgfpathcurveto{\pgfqpoint{1.234087in}{2.110018in}}{\pgfqpoint{1.237745in}{2.118850in}}{\pgfqpoint{1.237745in}{2.128059in}}%
\pgfpathcurveto{\pgfqpoint{1.237745in}{2.137267in}}{\pgfqpoint{1.234087in}{2.146100in}}{\pgfqpoint{1.227575in}{2.152611in}}%
\pgfpathcurveto{\pgfqpoint{1.221064in}{2.159122in}}{\pgfqpoint{1.212231in}{2.162781in}}{\pgfqpoint{1.203023in}{2.162781in}}%
\pgfpathcurveto{\pgfqpoint{1.193815in}{2.162781in}}{\pgfqpoint{1.184982in}{2.159122in}}{\pgfqpoint{1.178471in}{2.152611in}}%
\pgfpathcurveto{\pgfqpoint{1.171959in}{2.146100in}}{\pgfqpoint{1.168301in}{2.137267in}}{\pgfqpoint{1.168301in}{2.128059in}}%
\pgfpathcurveto{\pgfqpoint{1.168301in}{2.118850in}}{\pgfqpoint{1.171959in}{2.110018in}}{\pgfqpoint{1.178471in}{2.103506in}}%
\pgfpathcurveto{\pgfqpoint{1.184982in}{2.096995in}}{\pgfqpoint{1.193815in}{2.093337in}}{\pgfqpoint{1.203023in}{2.093337in}}%
\pgfpathlineto{\pgfqpoint{1.203023in}{2.093337in}}%
\pgfpathclose%
\pgfusepath{stroke,fill}%
\end{pgfscope}%
\begin{pgfscope}%
\pgfpathrectangle{\pgfqpoint{0.050000in}{0.050000in}}{\pgfqpoint{2.419000in}{2.419000in}}%
\pgfusepath{clip}%
\pgfsetbuttcap%
\pgfsetroundjoin%
\definecolor{currentfill}{rgb}{0.200000,0.133333,0.533333}%
\pgfsetfillcolor{currentfill}%
\pgfsetfillopacity{0.399826}%
\pgfsetlinewidth{1.003750pt}%
\definecolor{currentstroke}{rgb}{0.200000,0.133333,0.533333}%
\pgfsetstrokecolor{currentstroke}%
\pgfsetstrokeopacity{0.399826}%
\pgfsetdash{}{0pt}%
\pgfpathmoveto{\pgfqpoint{5.439055in}{2.093337in}}%
\pgfpathcurveto{\pgfqpoint{5.448263in}{2.093337in}}{\pgfqpoint{5.457096in}{2.096995in}}{\pgfqpoint{5.463607in}{2.103506in}}%
\pgfpathcurveto{\pgfqpoint{5.470118in}{2.110018in}}{\pgfqpoint{5.473777in}{2.118850in}}{\pgfqpoint{5.473777in}{2.128059in}}%
\pgfpathcurveto{\pgfqpoint{5.473777in}{2.137267in}}{\pgfqpoint{5.470118in}{2.146100in}}{\pgfqpoint{5.463607in}{2.152611in}}%
\pgfpathcurveto{\pgfqpoint{5.457096in}{2.159122in}}{\pgfqpoint{5.448263in}{2.162781in}}{\pgfqpoint{5.439055in}{2.162781in}}%
\pgfpathcurveto{\pgfqpoint{5.429846in}{2.162781in}}{\pgfqpoint{5.421014in}{2.159122in}}{\pgfqpoint{5.414502in}{2.152611in}}%
\pgfpathcurveto{\pgfqpoint{5.407991in}{2.146100in}}{\pgfqpoint{5.404333in}{2.137267in}}{\pgfqpoint{5.404333in}{2.128059in}}%
\pgfpathcurveto{\pgfqpoint{5.404333in}{2.118850in}}{\pgfqpoint{5.407991in}{2.110018in}}{\pgfqpoint{5.414502in}{2.103506in}}%
\pgfpathcurveto{\pgfqpoint{5.421014in}{2.096995in}}{\pgfqpoint{5.429846in}{2.093337in}}{\pgfqpoint{5.439055in}{2.093337in}}%
\pgfpathlineto{\pgfqpoint{5.439055in}{2.093337in}}%
\pgfpathclose%
\pgfusepath{stroke,fill}%
\end{pgfscope}%
\begin{pgfscope}%
\pgfpathrectangle{\pgfqpoint{0.050000in}{0.050000in}}{\pgfqpoint{2.419000in}{2.419000in}}%
\pgfusepath{clip}%
\pgfsetbuttcap%
\pgfsetroundjoin%
\definecolor{currentfill}{rgb}{0.200000,0.133333,0.533333}%
\pgfsetfillcolor{currentfill}%
\pgfsetfillopacity{0.399826}%
\pgfsetlinewidth{1.003750pt}%
\definecolor{currentstroke}{rgb}{0.200000,0.133333,0.533333}%
\pgfsetstrokecolor{currentstroke}%
\pgfsetstrokeopacity{0.399826}%
\pgfsetdash{}{0pt}%
\pgfpathmoveto{\pgfqpoint{9.675087in}{2.093337in}}%
\pgfpathcurveto{\pgfqpoint{9.684295in}{2.093337in}}{\pgfqpoint{9.693127in}{2.096995in}}{\pgfqpoint{9.699639in}{2.103506in}}%
\pgfpathcurveto{\pgfqpoint{9.706150in}{2.110018in}}{\pgfqpoint{9.709809in}{2.118850in}}{\pgfqpoint{9.709809in}{2.128059in}}%
\pgfpathcurveto{\pgfqpoint{9.709809in}{2.137267in}}{\pgfqpoint{9.706150in}{2.146100in}}{\pgfqpoint{9.699639in}{2.152611in}}%
\pgfpathcurveto{\pgfqpoint{9.693127in}{2.159122in}}{\pgfqpoint{9.684295in}{2.162781in}}{\pgfqpoint{9.675087in}{2.162781in}}%
\pgfpathcurveto{\pgfqpoint{9.665878in}{2.162781in}}{\pgfqpoint{9.657046in}{2.159122in}}{\pgfqpoint{9.650534in}{2.152611in}}%
\pgfpathcurveto{\pgfqpoint{9.644023in}{2.146100in}}{\pgfqpoint{9.640364in}{2.137267in}}{\pgfqpoint{9.640364in}{2.128059in}}%
\pgfpathcurveto{\pgfqpoint{9.640364in}{2.118850in}}{\pgfqpoint{9.644023in}{2.110018in}}{\pgfqpoint{9.650534in}{2.103506in}}%
\pgfpathcurveto{\pgfqpoint{9.657046in}{2.096995in}}{\pgfqpoint{9.665878in}{2.093337in}}{\pgfqpoint{9.675087in}{2.093337in}}%
\pgfpathlineto{\pgfqpoint{9.675087in}{2.093337in}}%
\pgfpathclose%
\pgfusepath{stroke,fill}%
\end{pgfscope}%
\begin{pgfscope}%
\pgfpathrectangle{\pgfqpoint{0.050000in}{0.050000in}}{\pgfqpoint{2.419000in}{2.419000in}}%
\pgfusepath{clip}%
\pgfsetbuttcap%
\pgfsetroundjoin%
\definecolor{currentfill}{rgb}{0.200000,0.133333,0.533333}%
\pgfsetfillcolor{currentfill}%
\pgfsetfillopacity{0.404744}%
\pgfsetlinewidth{1.003750pt}%
\definecolor{currentstroke}{rgb}{0.200000,0.133333,0.533333}%
\pgfsetstrokecolor{currentstroke}%
\pgfsetstrokeopacity{0.404744}%
\pgfsetdash{}{0pt}%
\pgfpathmoveto{\pgfqpoint{-0.545776in}{1.989817in}}%
\pgfpathcurveto{\pgfqpoint{-0.536567in}{1.989817in}}{\pgfqpoint{-0.527735in}{1.993475in}}{\pgfqpoint{-0.521223in}{1.999987in}}%
\pgfpathcurveto{\pgfqpoint{-0.514712in}{2.006498in}}{\pgfqpoint{-0.511054in}{2.015331in}}{\pgfqpoint{-0.511054in}{2.024539in}}%
\pgfpathcurveto{\pgfqpoint{-0.511054in}{2.033748in}}{\pgfqpoint{-0.514712in}{2.042580in}}{\pgfqpoint{-0.521223in}{2.049091in}}%
\pgfpathcurveto{\pgfqpoint{-0.527735in}{2.055603in}}{\pgfqpoint{-0.536567in}{2.059261in}}{\pgfqpoint{-0.545776in}{2.059261in}}%
\pgfpathcurveto{\pgfqpoint{-0.554984in}{2.059261in}}{\pgfqpoint{-0.563817in}{2.055603in}}{\pgfqpoint{-0.570328in}{2.049091in}}%
\pgfpathcurveto{\pgfqpoint{-0.576839in}{2.042580in}}{\pgfqpoint{-0.580498in}{2.033748in}}{\pgfqpoint{-0.580498in}{2.024539in}}%
\pgfpathcurveto{\pgfqpoint{-0.580498in}{2.015331in}}{\pgfqpoint{-0.576839in}{2.006498in}}{\pgfqpoint{-0.570328in}{1.999987in}}%
\pgfpathcurveto{\pgfqpoint{-0.563817in}{1.993475in}}{\pgfqpoint{-0.554984in}{1.989817in}}{\pgfqpoint{-0.545776in}{1.989817in}}%
\pgfpathlineto{\pgfqpoint{-0.545776in}{1.989817in}}%
\pgfpathclose%
\pgfusepath{stroke,fill}%
\end{pgfscope}%
\begin{pgfscope}%
\pgfpathrectangle{\pgfqpoint{0.050000in}{0.050000in}}{\pgfqpoint{2.419000in}{2.419000in}}%
\pgfusepath{clip}%
\pgfsetbuttcap%
\pgfsetroundjoin%
\definecolor{currentfill}{rgb}{0.200000,0.133333,0.533333}%
\pgfsetfillcolor{currentfill}%
\pgfsetfillopacity{0.404744}%
\pgfsetlinewidth{1.003750pt}%
\definecolor{currentstroke}{rgb}{0.200000,0.133333,0.533333}%
\pgfsetstrokecolor{currentstroke}%
\pgfsetstrokeopacity{0.404744}%
\pgfsetdash{}{0pt}%
\pgfpathmoveto{\pgfqpoint{8.056926in}{1.989817in}}%
\pgfpathcurveto{\pgfqpoint{8.066135in}{1.989817in}}{\pgfqpoint{8.074967in}{1.993475in}}{\pgfqpoint{8.081479in}{1.999987in}}%
\pgfpathcurveto{\pgfqpoint{8.087990in}{2.006498in}}{\pgfqpoint{8.091648in}{2.015331in}}{\pgfqpoint{8.091648in}{2.024539in}}%
\pgfpathcurveto{\pgfqpoint{8.091648in}{2.033748in}}{\pgfqpoint{8.087990in}{2.042580in}}{\pgfqpoint{8.081479in}{2.049091in}}%
\pgfpathcurveto{\pgfqpoint{8.074967in}{2.055603in}}{\pgfqpoint{8.066135in}{2.059261in}}{\pgfqpoint{8.056926in}{2.059261in}}%
\pgfpathcurveto{\pgfqpoint{8.047718in}{2.059261in}}{\pgfqpoint{8.038885in}{2.055603in}}{\pgfqpoint{8.032374in}{2.049091in}}%
\pgfpathcurveto{\pgfqpoint{8.025863in}{2.042580in}}{\pgfqpoint{8.022204in}{2.033748in}}{\pgfqpoint{8.022204in}{2.024539in}}%
\pgfpathcurveto{\pgfqpoint{8.022204in}{2.015331in}}{\pgfqpoint{8.025863in}{2.006498in}}{\pgfqpoint{8.032374in}{1.999987in}}%
\pgfpathcurveto{\pgfqpoint{8.038885in}{1.993475in}}{\pgfqpoint{8.047718in}{1.989817in}}{\pgfqpoint{8.056926in}{1.989817in}}%
\pgfpathlineto{\pgfqpoint{8.056926in}{1.989817in}}%
\pgfpathclose%
\pgfusepath{stroke,fill}%
\end{pgfscope}%
\begin{pgfscope}%
\pgfpathrectangle{\pgfqpoint{0.050000in}{0.050000in}}{\pgfqpoint{2.419000in}{2.419000in}}%
\pgfusepath{clip}%
\pgfsetbuttcap%
\pgfsetroundjoin%
\definecolor{currentfill}{rgb}{0.200000,0.133333,0.533333}%
\pgfsetfillcolor{currentfill}%
\pgfsetfillopacity{0.404744}%
\pgfsetlinewidth{1.003750pt}%
\definecolor{currentstroke}{rgb}{0.200000,0.133333,0.533333}%
\pgfsetstrokecolor{currentstroke}%
\pgfsetstrokeopacity{0.404744}%
\pgfsetdash{}{0pt}%
\pgfpathmoveto{\pgfqpoint{3.755575in}{1.989817in}}%
\pgfpathcurveto{\pgfqpoint{3.764784in}{1.989817in}}{\pgfqpoint{3.773616in}{1.993475in}}{\pgfqpoint{3.780128in}{1.999987in}}%
\pgfpathcurveto{\pgfqpoint{3.786639in}{2.006498in}}{\pgfqpoint{3.790297in}{2.015331in}}{\pgfqpoint{3.790297in}{2.024539in}}%
\pgfpathcurveto{\pgfqpoint{3.790297in}{2.033748in}}{\pgfqpoint{3.786639in}{2.042580in}}{\pgfqpoint{3.780128in}{2.049091in}}%
\pgfpathcurveto{\pgfqpoint{3.773616in}{2.055603in}}{\pgfqpoint{3.764784in}{2.059261in}}{\pgfqpoint{3.755575in}{2.059261in}}%
\pgfpathcurveto{\pgfqpoint{3.746367in}{2.059261in}}{\pgfqpoint{3.737534in}{2.055603in}}{\pgfqpoint{3.731023in}{2.049091in}}%
\pgfpathcurveto{\pgfqpoint{3.724512in}{2.042580in}}{\pgfqpoint{3.720853in}{2.033748in}}{\pgfqpoint{3.720853in}{2.024539in}}%
\pgfpathcurveto{\pgfqpoint{3.720853in}{2.015331in}}{\pgfqpoint{3.724512in}{2.006498in}}{\pgfqpoint{3.731023in}{1.999987in}}%
\pgfpathcurveto{\pgfqpoint{3.737534in}{1.993475in}}{\pgfqpoint{3.746367in}{1.989817in}}{\pgfqpoint{3.755575in}{1.989817in}}%
\pgfpathlineto{\pgfqpoint{3.755575in}{1.989817in}}%
\pgfpathclose%
\pgfusepath{stroke,fill}%
\end{pgfscope}%
\begin{pgfscope}%
\pgfpathrectangle{\pgfqpoint{0.050000in}{0.050000in}}{\pgfqpoint{2.419000in}{2.419000in}}%
\pgfusepath{clip}%
\pgfsetbuttcap%
\pgfsetroundjoin%
\definecolor{currentfill}{rgb}{0.200000,0.133333,0.533333}%
\pgfsetfillcolor{currentfill}%
\pgfsetfillopacity{0.409816}%
\pgfsetlinewidth{1.003750pt}%
\definecolor{currentstroke}{rgb}{0.200000,0.133333,0.533333}%
\pgfsetstrokecolor{currentstroke}%
\pgfsetstrokeopacity{0.409816}%
\pgfsetdash{}{0pt}%
\pgfpathmoveto{\pgfqpoint{-2.349352in}{1.883055in}}%
\pgfpathcurveto{\pgfqpoint{-2.340143in}{1.883055in}}{\pgfqpoint{-2.331311in}{1.886713in}}{\pgfqpoint{-2.324800in}{1.893225in}}%
\pgfpathcurveto{\pgfqpoint{-2.318288in}{1.899736in}}{\pgfqpoint{-2.314630in}{1.908568in}}{\pgfqpoint{-2.314630in}{1.917777in}}%
\pgfpathcurveto{\pgfqpoint{-2.314630in}{1.926985in}}{\pgfqpoint{-2.318288in}{1.935818in}}{\pgfqpoint{-2.324800in}{1.942329in}}%
\pgfpathcurveto{\pgfqpoint{-2.331311in}{1.948841in}}{\pgfqpoint{-2.340143in}{1.952499in}}{\pgfqpoint{-2.349352in}{1.952499in}}%
\pgfpathcurveto{\pgfqpoint{-2.358560in}{1.952499in}}{\pgfqpoint{-2.367393in}{1.948841in}}{\pgfqpoint{-2.373904in}{1.942329in}}%
\pgfpathcurveto{\pgfqpoint{-2.380415in}{1.935818in}}{\pgfqpoint{-2.384074in}{1.926985in}}{\pgfqpoint{-2.384074in}{1.917777in}}%
\pgfpathcurveto{\pgfqpoint{-2.384074in}{1.908568in}}{\pgfqpoint{-2.380415in}{1.899736in}}{\pgfqpoint{-2.373904in}{1.893225in}}%
\pgfpathcurveto{\pgfqpoint{-2.367393in}{1.886713in}}{\pgfqpoint{-2.358560in}{1.883055in}}{\pgfqpoint{-2.349352in}{1.883055in}}%
\pgfpathlineto{\pgfqpoint{-2.349352in}{1.883055in}}%
\pgfpathclose%
\pgfusepath{stroke,fill}%
\end{pgfscope}%
\begin{pgfscope}%
\pgfpathrectangle{\pgfqpoint{0.050000in}{0.050000in}}{\pgfqpoint{2.419000in}{2.419000in}}%
\pgfusepath{clip}%
\pgfsetbuttcap%
\pgfsetroundjoin%
\definecolor{currentfill}{rgb}{0.200000,0.133333,0.533333}%
\pgfsetfillcolor{currentfill}%
\pgfsetfillopacity{0.409816}%
\pgfsetlinewidth{1.003750pt}%
\definecolor{currentstroke}{rgb}{0.200000,0.133333,0.533333}%
\pgfsetstrokecolor{currentstroke}%
\pgfsetstrokeopacity{0.409816}%
\pgfsetdash{}{0pt}%
\pgfpathmoveto{\pgfqpoint{2.019364in}{1.883055in}}%
\pgfpathcurveto{\pgfqpoint{2.028573in}{1.883055in}}{\pgfqpoint{2.037405in}{1.886713in}}{\pgfqpoint{2.043917in}{1.893225in}}%
\pgfpathcurveto{\pgfqpoint{2.050428in}{1.899736in}}{\pgfqpoint{2.054087in}{1.908568in}}{\pgfqpoint{2.054087in}{1.917777in}}%
\pgfpathcurveto{\pgfqpoint{2.054087in}{1.926985in}}{\pgfqpoint{2.050428in}{1.935818in}}{\pgfqpoint{2.043917in}{1.942329in}}%
\pgfpathcurveto{\pgfqpoint{2.037405in}{1.948841in}}{\pgfqpoint{2.028573in}{1.952499in}}{\pgfqpoint{2.019364in}{1.952499in}}%
\pgfpathcurveto{\pgfqpoint{2.010156in}{1.952499in}}{\pgfqpoint{2.001323in}{1.948841in}}{\pgfqpoint{1.994812in}{1.942329in}}%
\pgfpathcurveto{\pgfqpoint{1.988301in}{1.935818in}}{\pgfqpoint{1.984642in}{1.926985in}}{\pgfqpoint{1.984642in}{1.917777in}}%
\pgfpathcurveto{\pgfqpoint{1.984642in}{1.908568in}}{\pgfqpoint{1.988301in}{1.899736in}}{\pgfqpoint{1.994812in}{1.893225in}}%
\pgfpathcurveto{\pgfqpoint{2.001323in}{1.886713in}}{\pgfqpoint{2.010156in}{1.883055in}}{\pgfqpoint{2.019364in}{1.883055in}}%
\pgfpathlineto{\pgfqpoint{2.019364in}{1.883055in}}%
\pgfpathclose%
\pgfusepath{stroke,fill}%
\end{pgfscope}%
\begin{pgfscope}%
\pgfpathrectangle{\pgfqpoint{0.050000in}{0.050000in}}{\pgfqpoint{2.419000in}{2.419000in}}%
\pgfusepath{clip}%
\pgfsetbuttcap%
\pgfsetroundjoin%
\definecolor{currentfill}{rgb}{0.200000,0.133333,0.533333}%
\pgfsetfillcolor{currentfill}%
\pgfsetfillopacity{0.409816}%
\pgfsetlinewidth{1.003750pt}%
\definecolor{currentstroke}{rgb}{0.200000,0.133333,0.533333}%
\pgfsetstrokecolor{currentstroke}%
\pgfsetstrokeopacity{0.409816}%
\pgfsetdash{}{0pt}%
\pgfpathmoveto{\pgfqpoint{6.388081in}{1.883055in}}%
\pgfpathcurveto{\pgfqpoint{6.397289in}{1.883055in}}{\pgfqpoint{6.406122in}{1.886713in}}{\pgfqpoint{6.412633in}{1.893225in}}%
\pgfpathcurveto{\pgfqpoint{6.419144in}{1.899736in}}{\pgfqpoint{6.422803in}{1.908568in}}{\pgfqpoint{6.422803in}{1.917777in}}%
\pgfpathcurveto{\pgfqpoint{6.422803in}{1.926985in}}{\pgfqpoint{6.419144in}{1.935818in}}{\pgfqpoint{6.412633in}{1.942329in}}%
\pgfpathcurveto{\pgfqpoint{6.406122in}{1.948841in}}{\pgfqpoint{6.397289in}{1.952499in}}{\pgfqpoint{6.388081in}{1.952499in}}%
\pgfpathcurveto{\pgfqpoint{6.378872in}{1.952499in}}{\pgfqpoint{6.370040in}{1.948841in}}{\pgfqpoint{6.363528in}{1.942329in}}%
\pgfpathcurveto{\pgfqpoint{6.357017in}{1.935818in}}{\pgfqpoint{6.353358in}{1.926985in}}{\pgfqpoint{6.353358in}{1.917777in}}%
\pgfpathcurveto{\pgfqpoint{6.353358in}{1.908568in}}{\pgfqpoint{6.357017in}{1.899736in}}{\pgfqpoint{6.363528in}{1.893225in}}%
\pgfpathcurveto{\pgfqpoint{6.370040in}{1.886713in}}{\pgfqpoint{6.378872in}{1.883055in}}{\pgfqpoint{6.388081in}{1.883055in}}%
\pgfpathlineto{\pgfqpoint{6.388081in}{1.883055in}}%
\pgfpathclose%
\pgfusepath{stroke,fill}%
\end{pgfscope}%
\begin{pgfscope}%
\pgfpathrectangle{\pgfqpoint{0.050000in}{0.050000in}}{\pgfqpoint{2.419000in}{2.419000in}}%
\pgfusepath{clip}%
\pgfsetbuttcap%
\pgfsetroundjoin%
\definecolor{currentfill}{rgb}{0.200000,0.133333,0.533333}%
\pgfsetfillcolor{currentfill}%
\pgfsetfillopacity{0.415049}%
\pgfsetlinewidth{1.003750pt}%
\definecolor{currentstroke}{rgb}{0.200000,0.133333,0.533333}%
\pgfsetstrokecolor{currentstroke}%
\pgfsetstrokeopacity{0.415049}%
\pgfsetdash{}{0pt}%
\pgfpathmoveto{\pgfqpoint{4.666130in}{1.772895in}}%
\pgfpathcurveto{\pgfqpoint{4.675339in}{1.772895in}}{\pgfqpoint{4.684171in}{1.776554in}}{\pgfqpoint{4.690683in}{1.783065in}}%
\pgfpathcurveto{\pgfqpoint{4.697194in}{1.789576in}}{\pgfqpoint{4.700853in}{1.798409in}}{\pgfqpoint{4.700853in}{1.807617in}}%
\pgfpathcurveto{\pgfqpoint{4.700853in}{1.816826in}}{\pgfqpoint{4.697194in}{1.825658in}}{\pgfqpoint{4.690683in}{1.832170in}}%
\pgfpathcurveto{\pgfqpoint{4.684171in}{1.838681in}}{\pgfqpoint{4.675339in}{1.842340in}}{\pgfqpoint{4.666130in}{1.842340in}}%
\pgfpathcurveto{\pgfqpoint{4.656922in}{1.842340in}}{\pgfqpoint{4.648089in}{1.838681in}}{\pgfqpoint{4.641578in}{1.832170in}}%
\pgfpathcurveto{\pgfqpoint{4.635067in}{1.825658in}}{\pgfqpoint{4.631408in}{1.816826in}}{\pgfqpoint{4.631408in}{1.807617in}}%
\pgfpathcurveto{\pgfqpoint{4.631408in}{1.798409in}}{\pgfqpoint{4.635067in}{1.789576in}}{\pgfqpoint{4.641578in}{1.783065in}}%
\pgfpathcurveto{\pgfqpoint{4.648089in}{1.776554in}}{\pgfqpoint{4.656922in}{1.772895in}}{\pgfqpoint{4.666130in}{1.772895in}}%
\pgfpathlineto{\pgfqpoint{4.666130in}{1.772895in}}%
\pgfpathclose%
\pgfusepath{stroke,fill}%
\end{pgfscope}%
\begin{pgfscope}%
\pgfpathrectangle{\pgfqpoint{0.050000in}{0.050000in}}{\pgfqpoint{2.419000in}{2.419000in}}%
\pgfusepath{clip}%
\pgfsetbuttcap%
\pgfsetroundjoin%
\definecolor{currentfill}{rgb}{0.200000,0.133333,0.533333}%
\pgfsetfillcolor{currentfill}%
\pgfsetfillopacity{0.415049}%
\pgfsetlinewidth{1.003750pt}%
\definecolor{currentstroke}{rgb}{0.200000,0.133333,0.533333}%
\pgfsetstrokecolor{currentstroke}%
\pgfsetstrokeopacity{0.415049}%
\pgfsetdash{}{0pt}%
\pgfpathmoveto{\pgfqpoint{9.104355in}{1.772895in}}%
\pgfpathcurveto{\pgfqpoint{9.113564in}{1.772895in}}{\pgfqpoint{9.122396in}{1.776554in}}{\pgfqpoint{9.128908in}{1.783065in}}%
\pgfpathcurveto{\pgfqpoint{9.135419in}{1.789576in}}{\pgfqpoint{9.139078in}{1.798409in}}{\pgfqpoint{9.139078in}{1.807617in}}%
\pgfpathcurveto{\pgfqpoint{9.139078in}{1.816826in}}{\pgfqpoint{9.135419in}{1.825658in}}{\pgfqpoint{9.128908in}{1.832170in}}%
\pgfpathcurveto{\pgfqpoint{9.122396in}{1.838681in}}{\pgfqpoint{9.113564in}{1.842340in}}{\pgfqpoint{9.104355in}{1.842340in}}%
\pgfpathcurveto{\pgfqpoint{9.095147in}{1.842340in}}{\pgfqpoint{9.086315in}{1.838681in}}{\pgfqpoint{9.079803in}{1.832170in}}%
\pgfpathcurveto{\pgfqpoint{9.073292in}{1.825658in}}{\pgfqpoint{9.069633in}{1.816826in}}{\pgfqpoint{9.069633in}{1.807617in}}%
\pgfpathcurveto{\pgfqpoint{9.069633in}{1.798409in}}{\pgfqpoint{9.073292in}{1.789576in}}{\pgfqpoint{9.079803in}{1.783065in}}%
\pgfpathcurveto{\pgfqpoint{9.086315in}{1.776554in}}{\pgfqpoint{9.095147in}{1.772895in}}{\pgfqpoint{9.104355in}{1.772895in}}%
\pgfpathlineto{\pgfqpoint{9.104355in}{1.772895in}}%
\pgfpathclose%
\pgfusepath{stroke,fill}%
\end{pgfscope}%
\begin{pgfscope}%
\pgfpathrectangle{\pgfqpoint{0.050000in}{0.050000in}}{\pgfqpoint{2.419000in}{2.419000in}}%
\pgfusepath{clip}%
\pgfsetbuttcap%
\pgfsetroundjoin%
\definecolor{currentfill}{rgb}{0.200000,0.133333,0.533333}%
\pgfsetfillcolor{currentfill}%
\pgfsetfillopacity{0.415049}%
\pgfsetlinewidth{1.003750pt}%
\definecolor{currentstroke}{rgb}{0.200000,0.133333,0.533333}%
\pgfsetstrokecolor{currentstroke}%
\pgfsetstrokeopacity{0.415049}%
\pgfsetdash{}{0pt}%
\pgfpathmoveto{\pgfqpoint{0.227905in}{1.772895in}}%
\pgfpathcurveto{\pgfqpoint{0.237114in}{1.772895in}}{\pgfqpoint{0.245946in}{1.776554in}}{\pgfqpoint{0.252458in}{1.783065in}}%
\pgfpathcurveto{\pgfqpoint{0.258969in}{1.789576in}}{\pgfqpoint{0.262627in}{1.798409in}}{\pgfqpoint{0.262627in}{1.807617in}}%
\pgfpathcurveto{\pgfqpoint{0.262627in}{1.816826in}}{\pgfqpoint{0.258969in}{1.825658in}}{\pgfqpoint{0.252458in}{1.832170in}}%
\pgfpathcurveto{\pgfqpoint{0.245946in}{1.838681in}}{\pgfqpoint{0.237114in}{1.842340in}}{\pgfqpoint{0.227905in}{1.842340in}}%
\pgfpathcurveto{\pgfqpoint{0.218697in}{1.842340in}}{\pgfqpoint{0.209864in}{1.838681in}}{\pgfqpoint{0.203353in}{1.832170in}}%
\pgfpathcurveto{\pgfqpoint{0.196842in}{1.825658in}}{\pgfqpoint{0.193183in}{1.816826in}}{\pgfqpoint{0.193183in}{1.807617in}}%
\pgfpathcurveto{\pgfqpoint{0.193183in}{1.798409in}}{\pgfqpoint{0.196842in}{1.789576in}}{\pgfqpoint{0.203353in}{1.783065in}}%
\pgfpathcurveto{\pgfqpoint{0.209864in}{1.776554in}}{\pgfqpoint{0.218697in}{1.772895in}}{\pgfqpoint{0.227905in}{1.772895in}}%
\pgfpathlineto{\pgfqpoint{0.227905in}{1.772895in}}%
\pgfpathclose%
\pgfusepath{stroke,fill}%
\end{pgfscope}%
\begin{pgfscope}%
\pgfpathrectangle{\pgfqpoint{0.050000in}{0.050000in}}{\pgfqpoint{2.419000in}{2.419000in}}%
\pgfusepath{clip}%
\pgfsetbuttcap%
\pgfsetroundjoin%
\definecolor{currentfill}{rgb}{0.200000,0.133333,0.533333}%
\pgfsetfillcolor{currentfill}%
\pgfsetfillopacity{0.420452}%
\pgfsetlinewidth{1.003750pt}%
\definecolor{currentstroke}{rgb}{0.200000,0.133333,0.533333}%
\pgfsetstrokecolor{currentstroke}%
\pgfsetstrokeopacity{0.420452}%
\pgfsetdash{}{0pt}%
\pgfpathmoveto{\pgfqpoint{-1.621482in}{1.659174in}}%
\pgfpathcurveto{\pgfqpoint{-1.612274in}{1.659174in}}{\pgfqpoint{-1.603441in}{1.662832in}}{\pgfqpoint{-1.596930in}{1.669343in}}%
\pgfpathcurveto{\pgfqpoint{-1.590418in}{1.675855in}}{\pgfqpoint{-1.586760in}{1.684687in}}{\pgfqpoint{-1.586760in}{1.693896in}}%
\pgfpathcurveto{\pgfqpoint{-1.586760in}{1.703104in}}{\pgfqpoint{-1.590418in}{1.711937in}}{\pgfqpoint{-1.596930in}{1.718448in}}%
\pgfpathcurveto{\pgfqpoint{-1.603441in}{1.724959in}}{\pgfqpoint{-1.612274in}{1.728618in}}{\pgfqpoint{-1.621482in}{1.728618in}}%
\pgfpathcurveto{\pgfqpoint{-1.630690in}{1.728618in}}{\pgfqpoint{-1.639523in}{1.724959in}}{\pgfqpoint{-1.646034in}{1.718448in}}%
\pgfpathcurveto{\pgfqpoint{-1.652546in}{1.711937in}}{\pgfqpoint{-1.656204in}{1.703104in}}{\pgfqpoint{-1.656204in}{1.693896in}}%
\pgfpathcurveto{\pgfqpoint{-1.656204in}{1.684687in}}{\pgfqpoint{-1.652546in}{1.675855in}}{\pgfqpoint{-1.646034in}{1.669343in}}%
\pgfpathcurveto{\pgfqpoint{-1.639523in}{1.662832in}}{\pgfqpoint{-1.630690in}{1.659174in}}{\pgfqpoint{-1.621482in}{1.659174in}}%
\pgfpathlineto{\pgfqpoint{-1.621482in}{1.659174in}}%
\pgfpathclose%
\pgfusepath{stroke,fill}%
\end{pgfscope}%
\begin{pgfscope}%
\pgfpathrectangle{\pgfqpoint{0.050000in}{0.050000in}}{\pgfqpoint{2.419000in}{2.419000in}}%
\pgfusepath{clip}%
\pgfsetbuttcap%
\pgfsetroundjoin%
\definecolor{currentfill}{rgb}{0.200000,0.133333,0.533333}%
\pgfsetfillcolor{currentfill}%
\pgfsetfillopacity{0.420452}%
\pgfsetlinewidth{1.003750pt}%
\definecolor{currentstroke}{rgb}{0.200000,0.133333,0.533333}%
\pgfsetstrokecolor{currentstroke}%
\pgfsetstrokeopacity{0.420452}%
\pgfsetdash{}{0pt}%
\pgfpathmoveto{\pgfqpoint{2.888500in}{1.659174in}}%
\pgfpathcurveto{\pgfqpoint{2.897708in}{1.659174in}}{\pgfqpoint{2.906541in}{1.662832in}}{\pgfqpoint{2.913052in}{1.669343in}}%
\pgfpathcurveto{\pgfqpoint{2.919563in}{1.675855in}}{\pgfqpoint{2.923222in}{1.684687in}}{\pgfqpoint{2.923222in}{1.693896in}}%
\pgfpathcurveto{\pgfqpoint{2.923222in}{1.703104in}}{\pgfqpoint{2.919563in}{1.711937in}}{\pgfqpoint{2.913052in}{1.718448in}}%
\pgfpathcurveto{\pgfqpoint{2.906541in}{1.724959in}}{\pgfqpoint{2.897708in}{1.728618in}}{\pgfqpoint{2.888500in}{1.728618in}}%
\pgfpathcurveto{\pgfqpoint{2.879291in}{1.728618in}}{\pgfqpoint{2.870459in}{1.724959in}}{\pgfqpoint{2.863947in}{1.718448in}}%
\pgfpathcurveto{\pgfqpoint{2.857436in}{1.711937in}}{\pgfqpoint{2.853777in}{1.703104in}}{\pgfqpoint{2.853777in}{1.693896in}}%
\pgfpathcurveto{\pgfqpoint{2.853777in}{1.684687in}}{\pgfqpoint{2.857436in}{1.675855in}}{\pgfqpoint{2.863947in}{1.669343in}}%
\pgfpathcurveto{\pgfqpoint{2.870459in}{1.662832in}}{\pgfqpoint{2.879291in}{1.659174in}}{\pgfqpoint{2.888500in}{1.659174in}}%
\pgfpathlineto{\pgfqpoint{2.888500in}{1.659174in}}%
\pgfpathclose%
\pgfusepath{stroke,fill}%
\end{pgfscope}%
\begin{pgfscope}%
\pgfpathrectangle{\pgfqpoint{0.050000in}{0.050000in}}{\pgfqpoint{2.419000in}{2.419000in}}%
\pgfusepath{clip}%
\pgfsetbuttcap%
\pgfsetroundjoin%
\definecolor{currentfill}{rgb}{0.200000,0.133333,0.533333}%
\pgfsetfillcolor{currentfill}%
\pgfsetfillopacity{0.420452}%
\pgfsetlinewidth{1.003750pt}%
\definecolor{currentstroke}{rgb}{0.200000,0.133333,0.533333}%
\pgfsetstrokecolor{currentstroke}%
\pgfsetstrokeopacity{0.420452}%
\pgfsetdash{}{0pt}%
\pgfpathmoveto{\pgfqpoint{7.398481in}{1.659174in}}%
\pgfpathcurveto{\pgfqpoint{7.407690in}{1.659174in}}{\pgfqpoint{7.416522in}{1.662832in}}{\pgfqpoint{7.423034in}{1.669343in}}%
\pgfpathcurveto{\pgfqpoint{7.429545in}{1.675855in}}{\pgfqpoint{7.433203in}{1.684687in}}{\pgfqpoint{7.433203in}{1.693896in}}%
\pgfpathcurveto{\pgfqpoint{7.433203in}{1.703104in}}{\pgfqpoint{7.429545in}{1.711937in}}{\pgfqpoint{7.423034in}{1.718448in}}%
\pgfpathcurveto{\pgfqpoint{7.416522in}{1.724959in}}{\pgfqpoint{7.407690in}{1.728618in}}{\pgfqpoint{7.398481in}{1.728618in}}%
\pgfpathcurveto{\pgfqpoint{7.389273in}{1.728618in}}{\pgfqpoint{7.380440in}{1.724959in}}{\pgfqpoint{7.373929in}{1.718448in}}%
\pgfpathcurveto{\pgfqpoint{7.367418in}{1.711937in}}{\pgfqpoint{7.363759in}{1.703104in}}{\pgfqpoint{7.363759in}{1.693896in}}%
\pgfpathcurveto{\pgfqpoint{7.363759in}{1.684687in}}{\pgfqpoint{7.367418in}{1.675855in}}{\pgfqpoint{7.373929in}{1.669343in}}%
\pgfpathcurveto{\pgfqpoint{7.380440in}{1.662832in}}{\pgfqpoint{7.389273in}{1.659174in}}{\pgfqpoint{7.398481in}{1.659174in}}%
\pgfpathlineto{\pgfqpoint{7.398481in}{1.659174in}}%
\pgfpathclose%
\pgfusepath{stroke,fill}%
\end{pgfscope}%
\begin{pgfscope}%
\pgfpathrectangle{\pgfqpoint{0.050000in}{0.050000in}}{\pgfqpoint{2.419000in}{2.419000in}}%
\pgfusepath{clip}%
\pgfsetbuttcap%
\pgfsetroundjoin%
\definecolor{currentfill}{rgb}{0.200000,0.133333,0.533333}%
\pgfsetfillcolor{currentfill}%
\pgfsetfillopacity{0.426033}%
\pgfsetlinewidth{1.003750pt}%
\definecolor{currentstroke}{rgb}{0.200000,0.133333,0.533333}%
\pgfsetstrokecolor{currentstroke}%
\pgfsetstrokeopacity{0.426033}%
\pgfsetdash{}{0pt}%
\pgfpathmoveto{\pgfqpoint{1.052443in}{1.541714in}}%
\pgfpathcurveto{\pgfqpoint{1.061652in}{1.541714in}}{\pgfqpoint{1.070484in}{1.545373in}}{\pgfqpoint{1.076996in}{1.551884in}}%
\pgfpathcurveto{\pgfqpoint{1.083507in}{1.558395in}}{\pgfqpoint{1.087166in}{1.567228in}}{\pgfqpoint{1.087166in}{1.576436in}}%
\pgfpathcurveto{\pgfqpoint{1.087166in}{1.585645in}}{\pgfqpoint{1.083507in}{1.594477in}}{\pgfqpoint{1.076996in}{1.600989in}}%
\pgfpathcurveto{\pgfqpoint{1.070484in}{1.607500in}}{\pgfqpoint{1.061652in}{1.611159in}}{\pgfqpoint{1.052443in}{1.611159in}}%
\pgfpathcurveto{\pgfqpoint{1.043235in}{1.611159in}}{\pgfqpoint{1.034402in}{1.607500in}}{\pgfqpoint{1.027891in}{1.600989in}}%
\pgfpathcurveto{\pgfqpoint{1.021380in}{1.594477in}}{\pgfqpoint{1.017721in}{1.585645in}}{\pgfqpoint{1.017721in}{1.576436in}}%
\pgfpathcurveto{\pgfqpoint{1.017721in}{1.567228in}}{\pgfqpoint{1.021380in}{1.558395in}}{\pgfqpoint{1.027891in}{1.551884in}}%
\pgfpathcurveto{\pgfqpoint{1.034402in}{1.545373in}}{\pgfqpoint{1.043235in}{1.541714in}}{\pgfqpoint{1.052443in}{1.541714in}}%
\pgfpathlineto{\pgfqpoint{1.052443in}{1.541714in}}%
\pgfpathclose%
\pgfusepath{stroke,fill}%
\end{pgfscope}%
\begin{pgfscope}%
\pgfpathrectangle{\pgfqpoint{0.050000in}{0.050000in}}{\pgfqpoint{2.419000in}{2.419000in}}%
\pgfusepath{clip}%
\pgfsetbuttcap%
\pgfsetroundjoin%
\definecolor{currentfill}{rgb}{0.200000,0.133333,0.533333}%
\pgfsetfillcolor{currentfill}%
\pgfsetfillopacity{0.426033}%
\pgfsetlinewidth{1.003750pt}%
\definecolor{currentstroke}{rgb}{0.200000,0.133333,0.533333}%
\pgfsetstrokecolor{currentstroke}%
\pgfsetstrokeopacity{0.426033}%
\pgfsetdash{}{0pt}%
\pgfpathmoveto{\pgfqpoint{10.220637in}{1.541714in}}%
\pgfpathcurveto{\pgfqpoint{10.229845in}{1.541714in}}{\pgfqpoint{10.238677in}{1.545373in}}{\pgfqpoint{10.245189in}{1.551884in}}%
\pgfpathcurveto{\pgfqpoint{10.251700in}{1.558395in}}{\pgfqpoint{10.255359in}{1.567228in}}{\pgfqpoint{10.255359in}{1.576436in}}%
\pgfpathcurveto{\pgfqpoint{10.255359in}{1.585645in}}{\pgfqpoint{10.251700in}{1.594477in}}{\pgfqpoint{10.245189in}{1.600989in}}%
\pgfpathcurveto{\pgfqpoint{10.238677in}{1.607500in}}{\pgfqpoint{10.229845in}{1.611159in}}{\pgfqpoint{10.220637in}{1.611159in}}%
\pgfpathcurveto{\pgfqpoint{10.211428in}{1.611159in}}{\pgfqpoint{10.202596in}{1.607500in}}{\pgfqpoint{10.196084in}{1.600989in}}%
\pgfpathcurveto{\pgfqpoint{10.189573in}{1.594477in}}{\pgfqpoint{10.185914in}{1.585645in}}{\pgfqpoint{10.185914in}{1.576436in}}%
\pgfpathcurveto{\pgfqpoint{10.185914in}{1.567228in}}{\pgfqpoint{10.189573in}{1.558395in}}{\pgfqpoint{10.196084in}{1.551884in}}%
\pgfpathcurveto{\pgfqpoint{10.202596in}{1.545373in}}{\pgfqpoint{10.211428in}{1.541714in}}{\pgfqpoint{10.220637in}{1.541714in}}%
\pgfpathlineto{\pgfqpoint{10.220637in}{1.541714in}}%
\pgfpathclose%
\pgfusepath{stroke,fill}%
\end{pgfscope}%
\begin{pgfscope}%
\pgfpathrectangle{\pgfqpoint{0.050000in}{0.050000in}}{\pgfqpoint{2.419000in}{2.419000in}}%
\pgfusepath{clip}%
\pgfsetbuttcap%
\pgfsetroundjoin%
\definecolor{currentfill}{rgb}{0.200000,0.133333,0.533333}%
\pgfsetfillcolor{currentfill}%
\pgfsetfillopacity{0.426033}%
\pgfsetlinewidth{1.003750pt}%
\definecolor{currentstroke}{rgb}{0.200000,0.133333,0.533333}%
\pgfsetstrokecolor{currentstroke}%
\pgfsetstrokeopacity{0.426033}%
\pgfsetdash{}{0pt}%
\pgfpathmoveto{\pgfqpoint{5.636540in}{1.541714in}}%
\pgfpathcurveto{\pgfqpoint{5.645748in}{1.541714in}}{\pgfqpoint{5.654581in}{1.545373in}}{\pgfqpoint{5.661092in}{1.551884in}}%
\pgfpathcurveto{\pgfqpoint{5.667604in}{1.558395in}}{\pgfqpoint{5.671262in}{1.567228in}}{\pgfqpoint{5.671262in}{1.576436in}}%
\pgfpathcurveto{\pgfqpoint{5.671262in}{1.585645in}}{\pgfqpoint{5.667604in}{1.594477in}}{\pgfqpoint{5.661092in}{1.600989in}}%
\pgfpathcurveto{\pgfqpoint{5.654581in}{1.607500in}}{\pgfqpoint{5.645748in}{1.611159in}}{\pgfqpoint{5.636540in}{1.611159in}}%
\pgfpathcurveto{\pgfqpoint{5.627332in}{1.611159in}}{\pgfqpoint{5.618499in}{1.607500in}}{\pgfqpoint{5.611988in}{1.600989in}}%
\pgfpathcurveto{\pgfqpoint{5.605476in}{1.594477in}}{\pgfqpoint{5.601818in}{1.585645in}}{\pgfqpoint{5.601818in}{1.576436in}}%
\pgfpathcurveto{\pgfqpoint{5.601818in}{1.567228in}}{\pgfqpoint{5.605476in}{1.558395in}}{\pgfqpoint{5.611988in}{1.551884in}}%
\pgfpathcurveto{\pgfqpoint{5.618499in}{1.545373in}}{\pgfqpoint{5.627332in}{1.541714in}}{\pgfqpoint{5.636540in}{1.541714in}}%
\pgfpathlineto{\pgfqpoint{5.636540in}{1.541714in}}%
\pgfpathclose%
\pgfusepath{stroke,fill}%
\end{pgfscope}%
\begin{pgfscope}%
\pgfpathrectangle{\pgfqpoint{0.050000in}{0.050000in}}{\pgfqpoint{2.419000in}{2.419000in}}%
\pgfusepath{clip}%
\pgfsetbuttcap%
\pgfsetroundjoin%
\definecolor{currentfill}{rgb}{0.200000,0.133333,0.533333}%
\pgfsetfillcolor{currentfill}%
\pgfsetfillopacity{0.431799}%
\pgfsetlinewidth{1.003750pt}%
\definecolor{currentstroke}{rgb}{0.200000,0.133333,0.533333}%
\pgfsetstrokecolor{currentstroke}%
\pgfsetstrokeopacity{0.431799}%
\pgfsetdash{}{0pt}%
\pgfpathmoveto{\pgfqpoint{-0.844967in}{1.420330in}}%
\pgfpathcurveto{\pgfqpoint{-0.835758in}{1.420330in}}{\pgfqpoint{-0.826926in}{1.423988in}}{\pgfqpoint{-0.820414in}{1.430500in}}%
\pgfpathcurveto{\pgfqpoint{-0.813903in}{1.437011in}}{\pgfqpoint{-0.810245in}{1.445844in}}{\pgfqpoint{-0.810245in}{1.455052in}}%
\pgfpathcurveto{\pgfqpoint{-0.810245in}{1.464261in}}{\pgfqpoint{-0.813903in}{1.473093in}}{\pgfqpoint{-0.820414in}{1.479604in}}%
\pgfpathcurveto{\pgfqpoint{-0.826926in}{1.486116in}}{\pgfqpoint{-0.835758in}{1.489774in}}{\pgfqpoint{-0.844967in}{1.489774in}}%
\pgfpathcurveto{\pgfqpoint{-0.854175in}{1.489774in}}{\pgfqpoint{-0.863008in}{1.486116in}}{\pgfqpoint{-0.869519in}{1.479604in}}%
\pgfpathcurveto{\pgfqpoint{-0.876030in}{1.473093in}}{\pgfqpoint{-0.879689in}{1.464261in}}{\pgfqpoint{-0.879689in}{1.455052in}}%
\pgfpathcurveto{\pgfqpoint{-0.879689in}{1.445844in}}{\pgfqpoint{-0.876030in}{1.437011in}}{\pgfqpoint{-0.869519in}{1.430500in}}%
\pgfpathcurveto{\pgfqpoint{-0.863008in}{1.423988in}}{\pgfqpoint{-0.854175in}{1.420330in}}{\pgfqpoint{-0.844967in}{1.420330in}}%
\pgfpathlineto{\pgfqpoint{-0.844967in}{1.420330in}}%
\pgfpathclose%
\pgfusepath{stroke,fill}%
\end{pgfscope}%
\begin{pgfscope}%
\pgfpathrectangle{\pgfqpoint{0.050000in}{0.050000in}}{\pgfqpoint{2.419000in}{2.419000in}}%
\pgfusepath{clip}%
\pgfsetbuttcap%
\pgfsetroundjoin%
\definecolor{currentfill}{rgb}{0.200000,0.133333,0.533333}%
\pgfsetfillcolor{currentfill}%
\pgfsetfillopacity{0.431799}%
\pgfsetlinewidth{1.003750pt}%
\definecolor{currentstroke}{rgb}{0.200000,0.133333,0.533333}%
\pgfsetstrokecolor{currentstroke}%
\pgfsetstrokeopacity{0.431799}%
\pgfsetdash{}{0pt}%
\pgfpathmoveto{\pgfqpoint{3.815721in}{1.420330in}}%
\pgfpathcurveto{\pgfqpoint{3.824930in}{1.420330in}}{\pgfqpoint{3.833762in}{1.423988in}}{\pgfqpoint{3.840274in}{1.430500in}}%
\pgfpathcurveto{\pgfqpoint{3.846785in}{1.437011in}}{\pgfqpoint{3.850444in}{1.445844in}}{\pgfqpoint{3.850444in}{1.455052in}}%
\pgfpathcurveto{\pgfqpoint{3.850444in}{1.464261in}}{\pgfqpoint{3.846785in}{1.473093in}}{\pgfqpoint{3.840274in}{1.479604in}}%
\pgfpathcurveto{\pgfqpoint{3.833762in}{1.486116in}}{\pgfqpoint{3.824930in}{1.489774in}}{\pgfqpoint{3.815721in}{1.489774in}}%
\pgfpathcurveto{\pgfqpoint{3.806513in}{1.489774in}}{\pgfqpoint{3.797680in}{1.486116in}}{\pgfqpoint{3.791169in}{1.479604in}}%
\pgfpathcurveto{\pgfqpoint{3.784658in}{1.473093in}}{\pgfqpoint{3.780999in}{1.464261in}}{\pgfqpoint{3.780999in}{1.455052in}}%
\pgfpathcurveto{\pgfqpoint{3.780999in}{1.445844in}}{\pgfqpoint{3.784658in}{1.437011in}}{\pgfqpoint{3.791169in}{1.430500in}}%
\pgfpathcurveto{\pgfqpoint{3.797680in}{1.423988in}}{\pgfqpoint{3.806513in}{1.420330in}}{\pgfqpoint{3.815721in}{1.420330in}}%
\pgfpathlineto{\pgfqpoint{3.815721in}{1.420330in}}%
\pgfpathclose%
\pgfusepath{stroke,fill}%
\end{pgfscope}%
\begin{pgfscope}%
\pgfpathrectangle{\pgfqpoint{0.050000in}{0.050000in}}{\pgfqpoint{2.419000in}{2.419000in}}%
\pgfusepath{clip}%
\pgfsetbuttcap%
\pgfsetroundjoin%
\definecolor{currentfill}{rgb}{0.200000,0.133333,0.533333}%
\pgfsetfillcolor{currentfill}%
\pgfsetfillopacity{0.431799}%
\pgfsetlinewidth{1.003750pt}%
\definecolor{currentstroke}{rgb}{0.200000,0.133333,0.533333}%
\pgfsetstrokecolor{currentstroke}%
\pgfsetstrokeopacity{0.431799}%
\pgfsetdash{}{0pt}%
\pgfpathmoveto{\pgfqpoint{8.476409in}{1.420330in}}%
\pgfpathcurveto{\pgfqpoint{8.485618in}{1.420330in}}{\pgfqpoint{8.494450in}{1.423988in}}{\pgfqpoint{8.500962in}{1.430500in}}%
\pgfpathcurveto{\pgfqpoint{8.507473in}{1.437011in}}{\pgfqpoint{8.511132in}{1.445844in}}{\pgfqpoint{8.511132in}{1.455052in}}%
\pgfpathcurveto{\pgfqpoint{8.511132in}{1.464261in}}{\pgfqpoint{8.507473in}{1.473093in}}{\pgfqpoint{8.500962in}{1.479604in}}%
\pgfpathcurveto{\pgfqpoint{8.494450in}{1.486116in}}{\pgfqpoint{8.485618in}{1.489774in}}{\pgfqpoint{8.476409in}{1.489774in}}%
\pgfpathcurveto{\pgfqpoint{8.467201in}{1.489774in}}{\pgfqpoint{8.458368in}{1.486116in}}{\pgfqpoint{8.451857in}{1.479604in}}%
\pgfpathcurveto{\pgfqpoint{8.445346in}{1.473093in}}{\pgfqpoint{8.441687in}{1.464261in}}{\pgfqpoint{8.441687in}{1.455052in}}%
\pgfpathcurveto{\pgfqpoint{8.441687in}{1.445844in}}{\pgfqpoint{8.445346in}{1.437011in}}{\pgfqpoint{8.451857in}{1.430500in}}%
\pgfpathcurveto{\pgfqpoint{8.458368in}{1.423988in}}{\pgfqpoint{8.467201in}{1.420330in}}{\pgfqpoint{8.476409in}{1.420330in}}%
\pgfpathlineto{\pgfqpoint{8.476409in}{1.420330in}}%
\pgfpathclose%
\pgfusepath{stroke,fill}%
\end{pgfscope}%
\begin{pgfscope}%
\pgfpathrectangle{\pgfqpoint{0.050000in}{0.050000in}}{\pgfqpoint{2.419000in}{2.419000in}}%
\pgfusepath{clip}%
\pgfsetbuttcap%
\pgfsetroundjoin%
\definecolor{currentfill}{rgb}{0.200000,0.133333,0.533333}%
\pgfsetfillcolor{currentfill}%
\pgfsetfillopacity{0.437762}%
\pgfsetlinewidth{1.003750pt}%
\definecolor{currentstroke}{rgb}{0.200000,0.133333,0.533333}%
\pgfsetstrokecolor{currentstroke}%
\pgfsetstrokeopacity{0.437762}%
\pgfsetdash{}{0pt}%
\pgfpathmoveto{\pgfqpoint{-2.806859in}{1.294820in}}%
\pgfpathcurveto{\pgfqpoint{-2.797650in}{1.294820in}}{\pgfqpoint{-2.788818in}{1.298479in}}{\pgfqpoint{-2.782306in}{1.304990in}}%
\pgfpathcurveto{\pgfqpoint{-2.775795in}{1.311502in}}{\pgfqpoint{-2.772136in}{1.320334in}}{\pgfqpoint{-2.772136in}{1.329543in}}%
\pgfpathcurveto{\pgfqpoint{-2.772136in}{1.338751in}}{\pgfqpoint{-2.775795in}{1.347584in}}{\pgfqpoint{-2.782306in}{1.354095in}}%
\pgfpathcurveto{\pgfqpoint{-2.788818in}{1.360606in}}{\pgfqpoint{-2.797650in}{1.364265in}}{\pgfqpoint{-2.806859in}{1.364265in}}%
\pgfpathcurveto{\pgfqpoint{-2.816067in}{1.364265in}}{\pgfqpoint{-2.824900in}{1.360606in}}{\pgfqpoint{-2.831411in}{1.354095in}}%
\pgfpathcurveto{\pgfqpoint{-2.837922in}{1.347584in}}{\pgfqpoint{-2.841581in}{1.338751in}}{\pgfqpoint{-2.841581in}{1.329543in}}%
\pgfpathcurveto{\pgfqpoint{-2.841581in}{1.320334in}}{\pgfqpoint{-2.837922in}{1.311502in}}{\pgfqpoint{-2.831411in}{1.304990in}}%
\pgfpathcurveto{\pgfqpoint{-2.824900in}{1.298479in}}{\pgfqpoint{-2.816067in}{1.294820in}}{\pgfqpoint{-2.806859in}{1.294820in}}%
\pgfpathlineto{\pgfqpoint{-2.806859in}{1.294820in}}%
\pgfpathclose%
\pgfusepath{stroke,fill}%
\end{pgfscope}%
\begin{pgfscope}%
\pgfpathrectangle{\pgfqpoint{0.050000in}{0.050000in}}{\pgfqpoint{2.419000in}{2.419000in}}%
\pgfusepath{clip}%
\pgfsetbuttcap%
\pgfsetroundjoin%
\definecolor{currentfill}{rgb}{0.200000,0.133333,0.533333}%
\pgfsetfillcolor{currentfill}%
\pgfsetfillopacity{0.437762}%
\pgfsetlinewidth{1.003750pt}%
\definecolor{currentstroke}{rgb}{0.200000,0.133333,0.533333}%
\pgfsetstrokecolor{currentstroke}%
\pgfsetstrokeopacity{0.437762}%
\pgfsetdash{}{0pt}%
\pgfpathmoveto{\pgfqpoint{1.933024in}{1.294820in}}%
\pgfpathcurveto{\pgfqpoint{1.942232in}{1.294820in}}{\pgfqpoint{1.951065in}{1.298479in}}{\pgfqpoint{1.957576in}{1.304990in}}%
\pgfpathcurveto{\pgfqpoint{1.964088in}{1.311502in}}{\pgfqpoint{1.967746in}{1.320334in}}{\pgfqpoint{1.967746in}{1.329543in}}%
\pgfpathcurveto{\pgfqpoint{1.967746in}{1.338751in}}{\pgfqpoint{1.964088in}{1.347584in}}{\pgfqpoint{1.957576in}{1.354095in}}%
\pgfpathcurveto{\pgfqpoint{1.951065in}{1.360606in}}{\pgfqpoint{1.942232in}{1.364265in}}{\pgfqpoint{1.933024in}{1.364265in}}%
\pgfpathcurveto{\pgfqpoint{1.923815in}{1.364265in}}{\pgfqpoint{1.914983in}{1.360606in}}{\pgfqpoint{1.908472in}{1.354095in}}%
\pgfpathcurveto{\pgfqpoint{1.901960in}{1.347584in}}{\pgfqpoint{1.898302in}{1.338751in}}{\pgfqpoint{1.898302in}{1.329543in}}%
\pgfpathcurveto{\pgfqpoint{1.898302in}{1.320334in}}{\pgfqpoint{1.901960in}{1.311502in}}{\pgfqpoint{1.908472in}{1.304990in}}%
\pgfpathcurveto{\pgfqpoint{1.914983in}{1.298479in}}{\pgfqpoint{1.923815in}{1.294820in}}{\pgfqpoint{1.933024in}{1.294820in}}%
\pgfpathlineto{\pgfqpoint{1.933024in}{1.294820in}}%
\pgfpathclose%
\pgfusepath{stroke,fill}%
\end{pgfscope}%
\begin{pgfscope}%
\pgfpathrectangle{\pgfqpoint{0.050000in}{0.050000in}}{\pgfqpoint{2.419000in}{2.419000in}}%
\pgfusepath{clip}%
\pgfsetbuttcap%
\pgfsetroundjoin%
\definecolor{currentfill}{rgb}{0.200000,0.133333,0.533333}%
\pgfsetfillcolor{currentfill}%
\pgfsetfillopacity{0.437762}%
\pgfsetlinewidth{1.003750pt}%
\definecolor{currentstroke}{rgb}{0.200000,0.133333,0.533333}%
\pgfsetstrokecolor{currentstroke}%
\pgfsetstrokeopacity{0.437762}%
\pgfsetdash{}{0pt}%
\pgfpathmoveto{\pgfqpoint{6.672906in}{1.294820in}}%
\pgfpathcurveto{\pgfqpoint{6.682115in}{1.294820in}}{\pgfqpoint{6.690947in}{1.298479in}}{\pgfqpoint{6.697459in}{1.304990in}}%
\pgfpathcurveto{\pgfqpoint{6.703970in}{1.311502in}}{\pgfqpoint{6.707629in}{1.320334in}}{\pgfqpoint{6.707629in}{1.329543in}}%
\pgfpathcurveto{\pgfqpoint{6.707629in}{1.338751in}}{\pgfqpoint{6.703970in}{1.347584in}}{\pgfqpoint{6.697459in}{1.354095in}}%
\pgfpathcurveto{\pgfqpoint{6.690947in}{1.360606in}}{\pgfqpoint{6.682115in}{1.364265in}}{\pgfqpoint{6.672906in}{1.364265in}}%
\pgfpathcurveto{\pgfqpoint{6.663698in}{1.364265in}}{\pgfqpoint{6.654865in}{1.360606in}}{\pgfqpoint{6.648354in}{1.354095in}}%
\pgfpathcurveto{\pgfqpoint{6.641843in}{1.347584in}}{\pgfqpoint{6.638184in}{1.338751in}}{\pgfqpoint{6.638184in}{1.329543in}}%
\pgfpathcurveto{\pgfqpoint{6.638184in}{1.320334in}}{\pgfqpoint{6.641843in}{1.311502in}}{\pgfqpoint{6.648354in}{1.304990in}}%
\pgfpathcurveto{\pgfqpoint{6.654865in}{1.298479in}}{\pgfqpoint{6.663698in}{1.294820in}}{\pgfqpoint{6.672906in}{1.294820in}}%
\pgfpathlineto{\pgfqpoint{6.672906in}{1.294820in}}%
\pgfpathclose%
\pgfusepath{stroke,fill}%
\end{pgfscope}%
\begin{pgfscope}%
\pgfpathrectangle{\pgfqpoint{0.050000in}{0.050000in}}{\pgfqpoint{2.419000in}{2.419000in}}%
\pgfusepath{clip}%
\pgfsetbuttcap%
\pgfsetroundjoin%
\definecolor{currentfill}{rgb}{0.200000,0.133333,0.533333}%
\pgfsetfillcolor{currentfill}%
\pgfsetfillopacity{0.443931}%
\pgfsetlinewidth{1.003750pt}%
\definecolor{currentstroke}{rgb}{0.200000,0.133333,0.533333}%
\pgfsetstrokecolor{currentstroke}%
\pgfsetstrokeopacity{0.443931}%
\pgfsetdash{}{0pt}%
\pgfpathmoveto{\pgfqpoint{9.628869in}{1.164972in}}%
\pgfpathcurveto{\pgfqpoint{9.638077in}{1.164972in}}{\pgfqpoint{9.646910in}{1.168630in}}{\pgfqpoint{9.653421in}{1.175142in}}%
\pgfpathcurveto{\pgfqpoint{9.659932in}{1.181653in}}{\pgfqpoint{9.663591in}{1.190486in}}{\pgfqpoint{9.663591in}{1.199694in}}%
\pgfpathcurveto{\pgfqpoint{9.663591in}{1.208903in}}{\pgfqpoint{9.659932in}{1.217735in}}{\pgfqpoint{9.653421in}{1.224246in}}%
\pgfpathcurveto{\pgfqpoint{9.646910in}{1.230758in}}{\pgfqpoint{9.638077in}{1.234416in}}{\pgfqpoint{9.628869in}{1.234416in}}%
\pgfpathcurveto{\pgfqpoint{9.619660in}{1.234416in}}{\pgfqpoint{9.610828in}{1.230758in}}{\pgfqpoint{9.604316in}{1.224246in}}%
\pgfpathcurveto{\pgfqpoint{9.597805in}{1.217735in}}{\pgfqpoint{9.594146in}{1.208903in}}{\pgfqpoint{9.594146in}{1.199694in}}%
\pgfpathcurveto{\pgfqpoint{9.594146in}{1.190486in}}{\pgfqpoint{9.597805in}{1.181653in}}{\pgfqpoint{9.604316in}{1.175142in}}%
\pgfpathcurveto{\pgfqpoint{9.610828in}{1.168630in}}{\pgfqpoint{9.619660in}{1.164972in}}{\pgfqpoint{9.628869in}{1.164972in}}%
\pgfpathlineto{\pgfqpoint{9.628869in}{1.164972in}}%
\pgfpathclose%
\pgfusepath{stroke,fill}%
\end{pgfscope}%
\begin{pgfscope}%
\pgfpathrectangle{\pgfqpoint{0.050000in}{0.050000in}}{\pgfqpoint{2.419000in}{2.419000in}}%
\pgfusepath{clip}%
\pgfsetbuttcap%
\pgfsetroundjoin%
\definecolor{currentfill}{rgb}{0.200000,0.133333,0.533333}%
\pgfsetfillcolor{currentfill}%
\pgfsetfillopacity{0.443931}%
\pgfsetlinewidth{1.003750pt}%
\definecolor{currentstroke}{rgb}{0.200000,0.133333,0.533333}%
\pgfsetstrokecolor{currentstroke}%
\pgfsetstrokeopacity{0.443931}%
\pgfsetdash{}{0pt}%
\pgfpathmoveto{\pgfqpoint{4.807054in}{1.164972in}}%
\pgfpathcurveto{\pgfqpoint{4.816262in}{1.164972in}}{\pgfqpoint{4.825095in}{1.168630in}}{\pgfqpoint{4.831606in}{1.175142in}}%
\pgfpathcurveto{\pgfqpoint{4.838117in}{1.181653in}}{\pgfqpoint{4.841776in}{1.190486in}}{\pgfqpoint{4.841776in}{1.199694in}}%
\pgfpathcurveto{\pgfqpoint{4.841776in}{1.208903in}}{\pgfqpoint{4.838117in}{1.217735in}}{\pgfqpoint{4.831606in}{1.224246in}}%
\pgfpathcurveto{\pgfqpoint{4.825095in}{1.230758in}}{\pgfqpoint{4.816262in}{1.234416in}}{\pgfqpoint{4.807054in}{1.234416in}}%
\pgfpathcurveto{\pgfqpoint{4.797845in}{1.234416in}}{\pgfqpoint{4.789013in}{1.230758in}}{\pgfqpoint{4.782501in}{1.224246in}}%
\pgfpathcurveto{\pgfqpoint{4.775990in}{1.217735in}}{\pgfqpoint{4.772332in}{1.208903in}}{\pgfqpoint{4.772332in}{1.199694in}}%
\pgfpathcurveto{\pgfqpoint{4.772332in}{1.190486in}}{\pgfqpoint{4.775990in}{1.181653in}}{\pgfqpoint{4.782501in}{1.175142in}}%
\pgfpathcurveto{\pgfqpoint{4.789013in}{1.168630in}}{\pgfqpoint{4.797845in}{1.164972in}}{\pgfqpoint{4.807054in}{1.164972in}}%
\pgfpathlineto{\pgfqpoint{4.807054in}{1.164972in}}%
\pgfpathclose%
\pgfusepath{stroke,fill}%
\end{pgfscope}%
\begin{pgfscope}%
\pgfpathrectangle{\pgfqpoint{0.050000in}{0.050000in}}{\pgfqpoint{2.419000in}{2.419000in}}%
\pgfusepath{clip}%
\pgfsetbuttcap%
\pgfsetroundjoin%
\definecolor{currentfill}{rgb}{0.200000,0.133333,0.533333}%
\pgfsetfillcolor{currentfill}%
\pgfsetfillopacity{0.443931}%
\pgfsetlinewidth{1.003750pt}%
\definecolor{currentstroke}{rgb}{0.200000,0.133333,0.533333}%
\pgfsetstrokecolor{currentstroke}%
\pgfsetstrokeopacity{0.443931}%
\pgfsetdash{}{0pt}%
\pgfpathmoveto{\pgfqpoint{-0.014761in}{1.164972in}}%
\pgfpathcurveto{\pgfqpoint{-0.005553in}{1.164972in}}{\pgfqpoint{0.003280in}{1.168630in}}{\pgfqpoint{0.009791in}{1.175142in}}%
\pgfpathcurveto{\pgfqpoint{0.016303in}{1.181653in}}{\pgfqpoint{0.019961in}{1.190486in}}{\pgfqpoint{0.019961in}{1.199694in}}%
\pgfpathcurveto{\pgfqpoint{0.019961in}{1.208903in}}{\pgfqpoint{0.016303in}{1.217735in}}{\pgfqpoint{0.009791in}{1.224246in}}%
\pgfpathcurveto{\pgfqpoint{0.003280in}{1.230758in}}{\pgfqpoint{-0.005553in}{1.234416in}}{\pgfqpoint{-0.014761in}{1.234416in}}%
\pgfpathcurveto{\pgfqpoint{-0.023970in}{1.234416in}}{\pgfqpoint{-0.032802in}{1.230758in}}{\pgfqpoint{-0.039313in}{1.224246in}}%
\pgfpathcurveto{\pgfqpoint{-0.045825in}{1.217735in}}{\pgfqpoint{-0.049483in}{1.208903in}}{\pgfqpoint{-0.049483in}{1.199694in}}%
\pgfpathcurveto{\pgfqpoint{-0.049483in}{1.190486in}}{\pgfqpoint{-0.045825in}{1.181653in}}{\pgfqpoint{-0.039313in}{1.175142in}}%
\pgfpathcurveto{\pgfqpoint{-0.032802in}{1.168630in}}{\pgfqpoint{-0.023970in}{1.164972in}}{\pgfqpoint{-0.014761in}{1.164972in}}%
\pgfpathlineto{\pgfqpoint{-0.014761in}{1.164972in}}%
\pgfpathclose%
\pgfusepath{stroke,fill}%
\end{pgfscope}%
\begin{pgfscope}%
\pgfpathrectangle{\pgfqpoint{0.050000in}{0.050000in}}{\pgfqpoint{2.419000in}{2.419000in}}%
\pgfusepath{clip}%
\pgfsetbuttcap%
\pgfsetroundjoin%
\definecolor{currentfill}{rgb}{0.200000,0.133333,0.533333}%
\pgfsetfillcolor{currentfill}%
\pgfsetfillopacity{0.450317}%
\pgfsetlinewidth{1.003750pt}%
\definecolor{currentstroke}{rgb}{0.200000,0.133333,0.533333}%
\pgfsetstrokecolor{currentstroke}%
\pgfsetstrokeopacity{0.450317}%
\pgfsetdash{}{0pt}%
\pgfpathmoveto{\pgfqpoint{-2.031068in}{1.030555in}}%
\pgfpathcurveto{\pgfqpoint{-2.021860in}{1.030555in}}{\pgfqpoint{-2.013027in}{1.034214in}}{\pgfqpoint{-2.006516in}{1.040725in}}%
\pgfpathcurveto{\pgfqpoint{-2.000005in}{1.047237in}}{\pgfqpoint{-1.996346in}{1.056069in}}{\pgfqpoint{-1.996346in}{1.065278in}}%
\pgfpathcurveto{\pgfqpoint{-1.996346in}{1.074486in}}{\pgfqpoint{-2.000005in}{1.083319in}}{\pgfqpoint{-2.006516in}{1.089830in}}%
\pgfpathcurveto{\pgfqpoint{-2.013027in}{1.096341in}}{\pgfqpoint{-2.021860in}{1.100000in}}{\pgfqpoint{-2.031068in}{1.100000in}}%
\pgfpathcurveto{\pgfqpoint{-2.040277in}{1.100000in}}{\pgfqpoint{-2.049109in}{1.096341in}}{\pgfqpoint{-2.055621in}{1.089830in}}%
\pgfpathcurveto{\pgfqpoint{-2.062132in}{1.083319in}}{\pgfqpoint{-2.065791in}{1.074486in}}{\pgfqpoint{-2.065791in}{1.065278in}}%
\pgfpathcurveto{\pgfqpoint{-2.065791in}{1.056069in}}{\pgfqpoint{-2.062132in}{1.047237in}}{\pgfqpoint{-2.055621in}{1.040725in}}%
\pgfpathcurveto{\pgfqpoint{-2.049109in}{1.034214in}}{\pgfqpoint{-2.040277in}{1.030555in}}{\pgfqpoint{-2.031068in}{1.030555in}}%
\pgfpathlineto{\pgfqpoint{-2.031068in}{1.030555in}}%
\pgfpathclose%
\pgfusepath{stroke,fill}%
\end{pgfscope}%
\begin{pgfscope}%
\pgfpathrectangle{\pgfqpoint{0.050000in}{0.050000in}}{\pgfqpoint{2.419000in}{2.419000in}}%
\pgfusepath{clip}%
\pgfsetbuttcap%
\pgfsetroundjoin%
\definecolor{currentfill}{rgb}{0.200000,0.133333,0.533333}%
\pgfsetfillcolor{currentfill}%
\pgfsetfillopacity{0.450317}%
\pgfsetlinewidth{1.003750pt}%
\definecolor{currentstroke}{rgb}{0.200000,0.133333,0.533333}%
\pgfsetstrokecolor{currentstroke}%
\pgfsetstrokeopacity{0.450317}%
\pgfsetdash{}{0pt}%
\pgfpathmoveto{\pgfqpoint{2.875561in}{1.030555in}}%
\pgfpathcurveto{\pgfqpoint{2.884770in}{1.030555in}}{\pgfqpoint{2.893602in}{1.034214in}}{\pgfqpoint{2.900114in}{1.040725in}}%
\pgfpathcurveto{\pgfqpoint{2.906625in}{1.047237in}}{\pgfqpoint{2.910283in}{1.056069in}}{\pgfqpoint{2.910283in}{1.065278in}}%
\pgfpathcurveto{\pgfqpoint{2.910283in}{1.074486in}}{\pgfqpoint{2.906625in}{1.083319in}}{\pgfqpoint{2.900114in}{1.089830in}}%
\pgfpathcurveto{\pgfqpoint{2.893602in}{1.096341in}}{\pgfqpoint{2.884770in}{1.100000in}}{\pgfqpoint{2.875561in}{1.100000in}}%
\pgfpathcurveto{\pgfqpoint{2.866353in}{1.100000in}}{\pgfqpoint{2.857520in}{1.096341in}}{\pgfqpoint{2.851009in}{1.089830in}}%
\pgfpathcurveto{\pgfqpoint{2.844498in}{1.083319in}}{\pgfqpoint{2.840839in}{1.074486in}}{\pgfqpoint{2.840839in}{1.065278in}}%
\pgfpathcurveto{\pgfqpoint{2.840839in}{1.056069in}}{\pgfqpoint{2.844498in}{1.047237in}}{\pgfqpoint{2.851009in}{1.040725in}}%
\pgfpathcurveto{\pgfqpoint{2.857520in}{1.034214in}}{\pgfqpoint{2.866353in}{1.030555in}}{\pgfqpoint{2.875561in}{1.030555in}}%
\pgfpathlineto{\pgfqpoint{2.875561in}{1.030555in}}%
\pgfpathclose%
\pgfusepath{stroke,fill}%
\end{pgfscope}%
\begin{pgfscope}%
\pgfpathrectangle{\pgfqpoint{0.050000in}{0.050000in}}{\pgfqpoint{2.419000in}{2.419000in}}%
\pgfusepath{clip}%
\pgfsetbuttcap%
\pgfsetroundjoin%
\definecolor{currentfill}{rgb}{0.200000,0.133333,0.533333}%
\pgfsetfillcolor{currentfill}%
\pgfsetfillopacity{0.450317}%
\pgfsetlinewidth{1.003750pt}%
\definecolor{currentstroke}{rgb}{0.200000,0.133333,0.533333}%
\pgfsetstrokecolor{currentstroke}%
\pgfsetstrokeopacity{0.450317}%
\pgfsetdash{}{0pt}%
\pgfpathmoveto{\pgfqpoint{7.782191in}{1.030555in}}%
\pgfpathcurveto{\pgfqpoint{7.791399in}{1.030555in}}{\pgfqpoint{7.800232in}{1.034214in}}{\pgfqpoint{7.806743in}{1.040725in}}%
\pgfpathcurveto{\pgfqpoint{7.813254in}{1.047237in}}{\pgfqpoint{7.816913in}{1.056069in}}{\pgfqpoint{7.816913in}{1.065278in}}%
\pgfpathcurveto{\pgfqpoint{7.816913in}{1.074486in}}{\pgfqpoint{7.813254in}{1.083319in}}{\pgfqpoint{7.806743in}{1.089830in}}%
\pgfpathcurveto{\pgfqpoint{7.800232in}{1.096341in}}{\pgfqpoint{7.791399in}{1.100000in}}{\pgfqpoint{7.782191in}{1.100000in}}%
\pgfpathcurveto{\pgfqpoint{7.772982in}{1.100000in}}{\pgfqpoint{7.764150in}{1.096341in}}{\pgfqpoint{7.757638in}{1.089830in}}%
\pgfpathcurveto{\pgfqpoint{7.751127in}{1.083319in}}{\pgfqpoint{7.747469in}{1.074486in}}{\pgfqpoint{7.747469in}{1.065278in}}%
\pgfpathcurveto{\pgfqpoint{7.747469in}{1.056069in}}{\pgfqpoint{7.751127in}{1.047237in}}{\pgfqpoint{7.757638in}{1.040725in}}%
\pgfpathcurveto{\pgfqpoint{7.764150in}{1.034214in}}{\pgfqpoint{7.772982in}{1.030555in}}{\pgfqpoint{7.782191in}{1.030555in}}%
\pgfpathlineto{\pgfqpoint{7.782191in}{1.030555in}}%
\pgfpathclose%
\pgfusepath{stroke,fill}%
\end{pgfscope}%
\begin{pgfscope}%
\pgfpathrectangle{\pgfqpoint{0.050000in}{0.050000in}}{\pgfqpoint{2.419000in}{2.419000in}}%
\pgfusepath{clip}%
\pgfsetbuttcap%
\pgfsetroundjoin%
\definecolor{currentfill}{rgb}{0.200000,0.133333,0.533333}%
\pgfsetfillcolor{currentfill}%
\pgfsetfillopacity{0.456932}%
\pgfsetlinewidth{1.003750pt}%
\definecolor{currentstroke}{rgb}{0.200000,0.133333,0.533333}%
\pgfsetstrokecolor{currentstroke}%
\pgfsetstrokeopacity{0.456932}%
\pgfsetdash{}{0pt}%
\pgfpathmoveto{\pgfqpoint{-4.119578in}{0.891325in}}%
\pgfpathcurveto{\pgfqpoint{-4.110370in}{0.891325in}}{\pgfqpoint{-4.101537in}{0.894984in}}{\pgfqpoint{-4.095026in}{0.901495in}}%
\pgfpathcurveto{\pgfqpoint{-4.088515in}{0.908007in}}{\pgfqpoint{-4.084856in}{0.916839in}}{\pgfqpoint{-4.084856in}{0.926048in}}%
\pgfpathcurveto{\pgfqpoint{-4.084856in}{0.935256in}}{\pgfqpoint{-4.088515in}{0.944089in}}{\pgfqpoint{-4.095026in}{0.950600in}}%
\pgfpathcurveto{\pgfqpoint{-4.101537in}{0.957111in}}{\pgfqpoint{-4.110370in}{0.960770in}}{\pgfqpoint{-4.119578in}{0.960770in}}%
\pgfpathcurveto{\pgfqpoint{-4.128787in}{0.960770in}}{\pgfqpoint{-4.137619in}{0.957111in}}{\pgfqpoint{-4.144131in}{0.950600in}}%
\pgfpathcurveto{\pgfqpoint{-4.150642in}{0.944089in}}{\pgfqpoint{-4.154301in}{0.935256in}}{\pgfqpoint{-4.154301in}{0.926048in}}%
\pgfpathcurveto{\pgfqpoint{-4.154301in}{0.916839in}}{\pgfqpoint{-4.150642in}{0.908007in}}{\pgfqpoint{-4.144131in}{0.901495in}}%
\pgfpathcurveto{\pgfqpoint{-4.137619in}{0.894984in}}{\pgfqpoint{-4.128787in}{0.891325in}}{\pgfqpoint{-4.119578in}{0.891325in}}%
\pgfpathlineto{\pgfqpoint{-4.119578in}{0.891325in}}%
\pgfpathclose%
\pgfusepath{stroke,fill}%
\end{pgfscope}%
\begin{pgfscope}%
\pgfpathrectangle{\pgfqpoint{0.050000in}{0.050000in}}{\pgfqpoint{2.419000in}{2.419000in}}%
\pgfusepath{clip}%
\pgfsetbuttcap%
\pgfsetroundjoin%
\definecolor{currentfill}{rgb}{0.200000,0.133333,0.533333}%
\pgfsetfillcolor{currentfill}%
\pgfsetfillopacity{0.456932}%
\pgfsetlinewidth{1.003750pt}%
\definecolor{currentstroke}{rgb}{0.200000,0.133333,0.533333}%
\pgfsetstrokecolor{currentstroke}%
\pgfsetstrokeopacity{0.456932}%
\pgfsetdash{}{0pt}%
\pgfpathmoveto{\pgfqpoint{0.874903in}{0.891325in}}%
\pgfpathcurveto{\pgfqpoint{0.884111in}{0.891325in}}{\pgfqpoint{0.892944in}{0.894984in}}{\pgfqpoint{0.899455in}{0.901495in}}%
\pgfpathcurveto{\pgfqpoint{0.905967in}{0.908007in}}{\pgfqpoint{0.909625in}{0.916839in}}{\pgfqpoint{0.909625in}{0.926048in}}%
\pgfpathcurveto{\pgfqpoint{0.909625in}{0.935256in}}{\pgfqpoint{0.905967in}{0.944089in}}{\pgfqpoint{0.899455in}{0.950600in}}%
\pgfpathcurveto{\pgfqpoint{0.892944in}{0.957111in}}{\pgfqpoint{0.884111in}{0.960770in}}{\pgfqpoint{0.874903in}{0.960770in}}%
\pgfpathcurveto{\pgfqpoint{0.865695in}{0.960770in}}{\pgfqpoint{0.856862in}{0.957111in}}{\pgfqpoint{0.850351in}{0.950600in}}%
\pgfpathcurveto{\pgfqpoint{0.843839in}{0.944089in}}{\pgfqpoint{0.840181in}{0.935256in}}{\pgfqpoint{0.840181in}{0.926048in}}%
\pgfpathcurveto{\pgfqpoint{0.840181in}{0.916839in}}{\pgfqpoint{0.843839in}{0.908007in}}{\pgfqpoint{0.850351in}{0.901495in}}%
\pgfpathcurveto{\pgfqpoint{0.856862in}{0.894984in}}{\pgfqpoint{0.865695in}{0.891325in}}{\pgfqpoint{0.874903in}{0.891325in}}%
\pgfpathlineto{\pgfqpoint{0.874903in}{0.891325in}}%
\pgfpathclose%
\pgfusepath{stroke,fill}%
\end{pgfscope}%
\begin{pgfscope}%
\pgfpathrectangle{\pgfqpoint{0.050000in}{0.050000in}}{\pgfqpoint{2.419000in}{2.419000in}}%
\pgfusepath{clip}%
\pgfsetbuttcap%
\pgfsetroundjoin%
\definecolor{currentfill}{rgb}{0.200000,0.133333,0.533333}%
\pgfsetfillcolor{currentfill}%
\pgfsetfillopacity{0.456932}%
\pgfsetlinewidth{1.003750pt}%
\definecolor{currentstroke}{rgb}{0.200000,0.133333,0.533333}%
\pgfsetstrokecolor{currentstroke}%
\pgfsetstrokeopacity{0.456932}%
\pgfsetdash{}{0pt}%
\pgfpathmoveto{\pgfqpoint{5.869384in}{0.891325in}}%
\pgfpathcurveto{\pgfqpoint{5.878593in}{0.891325in}}{\pgfqpoint{5.887425in}{0.894984in}}{\pgfqpoint{5.893937in}{0.901495in}}%
\pgfpathcurveto{\pgfqpoint{5.900448in}{0.908007in}}{\pgfqpoint{5.904107in}{0.916839in}}{\pgfqpoint{5.904107in}{0.926048in}}%
\pgfpathcurveto{\pgfqpoint{5.904107in}{0.935256in}}{\pgfqpoint{5.900448in}{0.944089in}}{\pgfqpoint{5.893937in}{0.950600in}}%
\pgfpathcurveto{\pgfqpoint{5.887425in}{0.957111in}}{\pgfqpoint{5.878593in}{0.960770in}}{\pgfqpoint{5.869384in}{0.960770in}}%
\pgfpathcurveto{\pgfqpoint{5.860176in}{0.960770in}}{\pgfqpoint{5.851343in}{0.957111in}}{\pgfqpoint{5.844832in}{0.950600in}}%
\pgfpathcurveto{\pgfqpoint{5.838321in}{0.944089in}}{\pgfqpoint{5.834662in}{0.935256in}}{\pgfqpoint{5.834662in}{0.926048in}}%
\pgfpathcurveto{\pgfqpoint{5.834662in}{0.916839in}}{\pgfqpoint{5.838321in}{0.908007in}}{\pgfqpoint{5.844832in}{0.901495in}}%
\pgfpathcurveto{\pgfqpoint{5.851343in}{0.894984in}}{\pgfqpoint{5.860176in}{0.891325in}}{\pgfqpoint{5.869384in}{0.891325in}}%
\pgfpathlineto{\pgfqpoint{5.869384in}{0.891325in}}%
\pgfpathclose%
\pgfusepath{stroke,fill}%
\end{pgfscope}%
\begin{pgfscope}%
\pgfpathrectangle{\pgfqpoint{0.050000in}{0.050000in}}{\pgfqpoint{2.419000in}{2.419000in}}%
\pgfusepath{clip}%
\pgfsetbuttcap%
\pgfsetroundjoin%
\definecolor{currentfill}{rgb}{0.200000,0.133333,0.533333}%
\pgfsetfillcolor{currentfill}%
\pgfsetfillopacity{0.463788}%
\pgfsetlinewidth{1.003750pt}%
\definecolor{currentstroke}{rgb}{0.200000,0.133333,0.533333}%
\pgfsetstrokecolor{currentstroke}%
\pgfsetstrokeopacity{0.463788}%
\pgfsetdash{}{0pt}%
\pgfpathmoveto{\pgfqpoint{-1.198704in}{0.747019in}}%
\pgfpathcurveto{\pgfqpoint{-1.189495in}{0.747019in}}{\pgfqpoint{-1.180663in}{0.750677in}}{\pgfqpoint{-1.174151in}{0.757189in}}%
\pgfpathcurveto{\pgfqpoint{-1.167640in}{0.763700in}}{\pgfqpoint{-1.163982in}{0.772533in}}{\pgfqpoint{-1.163982in}{0.781741in}}%
\pgfpathcurveto{\pgfqpoint{-1.163982in}{0.790950in}}{\pgfqpoint{-1.167640in}{0.799782in}}{\pgfqpoint{-1.174151in}{0.806293in}}%
\pgfpathcurveto{\pgfqpoint{-1.180663in}{0.812805in}}{\pgfqpoint{-1.189495in}{0.816463in}}{\pgfqpoint{-1.198704in}{0.816463in}}%
\pgfpathcurveto{\pgfqpoint{-1.207912in}{0.816463in}}{\pgfqpoint{-1.216745in}{0.812805in}}{\pgfqpoint{-1.223256in}{0.806293in}}%
\pgfpathcurveto{\pgfqpoint{-1.229767in}{0.799782in}}{\pgfqpoint{-1.233426in}{0.790950in}}{\pgfqpoint{-1.233426in}{0.781741in}}%
\pgfpathcurveto{\pgfqpoint{-1.233426in}{0.772533in}}{\pgfqpoint{-1.229767in}{0.763700in}}{\pgfqpoint{-1.223256in}{0.757189in}}%
\pgfpathcurveto{\pgfqpoint{-1.216745in}{0.750677in}}{\pgfqpoint{-1.207912in}{0.747019in}}{\pgfqpoint{-1.198704in}{0.747019in}}%
\pgfpathlineto{\pgfqpoint{-1.198704in}{0.747019in}}%
\pgfpathclose%
\pgfusepath{stroke,fill}%
\end{pgfscope}%
\begin{pgfscope}%
\pgfpathrectangle{\pgfqpoint{0.050000in}{0.050000in}}{\pgfqpoint{2.419000in}{2.419000in}}%
\pgfusepath{clip}%
\pgfsetbuttcap%
\pgfsetroundjoin%
\definecolor{currentfill}{rgb}{0.200000,0.133333,0.533333}%
\pgfsetfillcolor{currentfill}%
\pgfsetfillopacity{0.463788}%
\pgfsetlinewidth{1.003750pt}%
\definecolor{currentstroke}{rgb}{0.200000,0.133333,0.533333}%
\pgfsetstrokecolor{currentstroke}%
\pgfsetstrokeopacity{0.463788}%
\pgfsetdash{}{0pt}%
\pgfpathmoveto{\pgfqpoint{8.972369in}{0.747019in}}%
\pgfpathcurveto{\pgfqpoint{8.981578in}{0.747019in}}{\pgfqpoint{8.990410in}{0.750677in}}{\pgfqpoint{8.996921in}{0.757189in}}%
\pgfpathcurveto{\pgfqpoint{9.003433in}{0.763700in}}{\pgfqpoint{9.007091in}{0.772533in}}{\pgfqpoint{9.007091in}{0.781741in}}%
\pgfpathcurveto{\pgfqpoint{9.007091in}{0.790950in}}{\pgfqpoint{9.003433in}{0.799782in}}{\pgfqpoint{8.996921in}{0.806293in}}%
\pgfpathcurveto{\pgfqpoint{8.990410in}{0.812805in}}{\pgfqpoint{8.981578in}{0.816463in}}{\pgfqpoint{8.972369in}{0.816463in}}%
\pgfpathcurveto{\pgfqpoint{8.963161in}{0.816463in}}{\pgfqpoint{8.954328in}{0.812805in}}{\pgfqpoint{8.947817in}{0.806293in}}%
\pgfpathcurveto{\pgfqpoint{8.941305in}{0.799782in}}{\pgfqpoint{8.937647in}{0.790950in}}{\pgfqpoint{8.937647in}{0.781741in}}%
\pgfpathcurveto{\pgfqpoint{8.937647in}{0.772533in}}{\pgfqpoint{8.941305in}{0.763700in}}{\pgfqpoint{8.947817in}{0.757189in}}%
\pgfpathcurveto{\pgfqpoint{8.954328in}{0.750677in}}{\pgfqpoint{8.963161in}{0.747019in}}{\pgfqpoint{8.972369in}{0.747019in}}%
\pgfpathlineto{\pgfqpoint{8.972369in}{0.747019in}}%
\pgfpathclose%
\pgfusepath{stroke,fill}%
\end{pgfscope}%
\begin{pgfscope}%
\pgfpathrectangle{\pgfqpoint{0.050000in}{0.050000in}}{\pgfqpoint{2.419000in}{2.419000in}}%
\pgfusepath{clip}%
\pgfsetbuttcap%
\pgfsetroundjoin%
\definecolor{currentfill}{rgb}{0.200000,0.133333,0.533333}%
\pgfsetfillcolor{currentfill}%
\pgfsetfillopacity{0.463788}%
\pgfsetlinewidth{1.003750pt}%
\definecolor{currentstroke}{rgb}{0.200000,0.133333,0.533333}%
\pgfsetstrokecolor{currentstroke}%
\pgfsetstrokeopacity{0.463788}%
\pgfsetdash{}{0pt}%
\pgfpathmoveto{\pgfqpoint{3.886833in}{0.747019in}}%
\pgfpathcurveto{\pgfqpoint{3.896041in}{0.747019in}}{\pgfqpoint{3.904874in}{0.750677in}}{\pgfqpoint{3.911385in}{0.757189in}}%
\pgfpathcurveto{\pgfqpoint{3.917896in}{0.763700in}}{\pgfqpoint{3.921555in}{0.772533in}}{\pgfqpoint{3.921555in}{0.781741in}}%
\pgfpathcurveto{\pgfqpoint{3.921555in}{0.790950in}}{\pgfqpoint{3.917896in}{0.799782in}}{\pgfqpoint{3.911385in}{0.806293in}}%
\pgfpathcurveto{\pgfqpoint{3.904874in}{0.812805in}}{\pgfqpoint{3.896041in}{0.816463in}}{\pgfqpoint{3.886833in}{0.816463in}}%
\pgfpathcurveto{\pgfqpoint{3.877624in}{0.816463in}}{\pgfqpoint{3.868792in}{0.812805in}}{\pgfqpoint{3.862280in}{0.806293in}}%
\pgfpathcurveto{\pgfqpoint{3.855769in}{0.799782in}}{\pgfqpoint{3.852110in}{0.790950in}}{\pgfqpoint{3.852110in}{0.781741in}}%
\pgfpathcurveto{\pgfqpoint{3.852110in}{0.772533in}}{\pgfqpoint{3.855769in}{0.763700in}}{\pgfqpoint{3.862280in}{0.757189in}}%
\pgfpathcurveto{\pgfqpoint{3.868792in}{0.750677in}}{\pgfqpoint{3.877624in}{0.747019in}}{\pgfqpoint{3.886833in}{0.747019in}}%
\pgfpathlineto{\pgfqpoint{3.886833in}{0.747019in}}%
\pgfpathclose%
\pgfusepath{stroke,fill}%
\end{pgfscope}%
\begin{pgfscope}%
\pgfpathrectangle{\pgfqpoint{0.050000in}{0.050000in}}{\pgfqpoint{2.419000in}{2.419000in}}%
\pgfusepath{clip}%
\pgfsetbuttcap%
\pgfsetroundjoin%
\definecolor{currentfill}{rgb}{0.200000,0.133333,0.533333}%
\pgfsetfillcolor{currentfill}%
\pgfsetfillopacity{0.470898}%
\pgfsetlinewidth{1.003750pt}%
\definecolor{currentstroke}{rgb}{0.200000,0.133333,0.533333}%
\pgfsetstrokecolor{currentstroke}%
\pgfsetstrokeopacity{0.470898}%
\pgfsetdash{}{0pt}%
\pgfpathmoveto{\pgfqpoint{-3.349323in}{0.597353in}}%
\pgfpathcurveto{\pgfqpoint{-3.340115in}{0.597353in}}{\pgfqpoint{-3.331282in}{0.601011in}}{\pgfqpoint{-3.324771in}{0.607523in}}%
\pgfpathcurveto{\pgfqpoint{-3.318259in}{0.614034in}}{\pgfqpoint{-3.314601in}{0.622867in}}{\pgfqpoint{-3.314601in}{0.632075in}}%
\pgfpathcurveto{\pgfqpoint{-3.314601in}{0.641284in}}{\pgfqpoint{-3.318259in}{0.650116in}}{\pgfqpoint{-3.324771in}{0.656627in}}%
\pgfpathcurveto{\pgfqpoint{-3.331282in}{0.663139in}}{\pgfqpoint{-3.340115in}{0.666797in}}{\pgfqpoint{-3.349323in}{0.666797in}}%
\pgfpathcurveto{\pgfqpoint{-3.358531in}{0.666797in}}{\pgfqpoint{-3.367364in}{0.663139in}}{\pgfqpoint{-3.373875in}{0.656627in}}%
\pgfpathcurveto{\pgfqpoint{-3.380387in}{0.650116in}}{\pgfqpoint{-3.384045in}{0.641284in}}{\pgfqpoint{-3.384045in}{0.632075in}}%
\pgfpathcurveto{\pgfqpoint{-3.384045in}{0.622867in}}{\pgfqpoint{-3.380387in}{0.614034in}}{\pgfqpoint{-3.373875in}{0.607523in}}%
\pgfpathcurveto{\pgfqpoint{-3.367364in}{0.601011in}}{\pgfqpoint{-3.358531in}{0.597353in}}{\pgfqpoint{-3.349323in}{0.597353in}}%
\pgfpathlineto{\pgfqpoint{-3.349323in}{0.597353in}}%
\pgfpathclose%
\pgfusepath{stroke,fill}%
\end{pgfscope}%
\begin{pgfscope}%
\pgfpathrectangle{\pgfqpoint{0.050000in}{0.050000in}}{\pgfqpoint{2.419000in}{2.419000in}}%
\pgfusepath{clip}%
\pgfsetbuttcap%
\pgfsetroundjoin%
\definecolor{currentfill}{rgb}{0.200000,0.133333,0.533333}%
\pgfsetfillcolor{currentfill}%
\pgfsetfillopacity{0.470898}%
\pgfsetlinewidth{1.003750pt}%
\definecolor{currentstroke}{rgb}{0.200000,0.133333,0.533333}%
\pgfsetstrokecolor{currentstroke}%
\pgfsetstrokeopacity{0.470898}%
\pgfsetdash{}{0pt}%
\pgfpathmoveto{\pgfqpoint{1.830650in}{0.597353in}}%
\pgfpathcurveto{\pgfqpoint{1.839859in}{0.597353in}}{\pgfqpoint{1.848691in}{0.601011in}}{\pgfqpoint{1.855203in}{0.607523in}}%
\pgfpathcurveto{\pgfqpoint{1.861714in}{0.614034in}}{\pgfqpoint{1.865372in}{0.622867in}}{\pgfqpoint{1.865372in}{0.632075in}}%
\pgfpathcurveto{\pgfqpoint{1.865372in}{0.641284in}}{\pgfqpoint{1.861714in}{0.650116in}}{\pgfqpoint{1.855203in}{0.656627in}}%
\pgfpathcurveto{\pgfqpoint{1.848691in}{0.663139in}}{\pgfqpoint{1.839859in}{0.666797in}}{\pgfqpoint{1.830650in}{0.666797in}}%
\pgfpathcurveto{\pgfqpoint{1.821442in}{0.666797in}}{\pgfqpoint{1.812609in}{0.663139in}}{\pgfqpoint{1.806098in}{0.656627in}}%
\pgfpathcurveto{\pgfqpoint{1.799587in}{0.650116in}}{\pgfqpoint{1.795928in}{0.641284in}}{\pgfqpoint{1.795928in}{0.632075in}}%
\pgfpathcurveto{\pgfqpoint{1.795928in}{0.622867in}}{\pgfqpoint{1.799587in}{0.614034in}}{\pgfqpoint{1.806098in}{0.607523in}}%
\pgfpathcurveto{\pgfqpoint{1.812609in}{0.601011in}}{\pgfqpoint{1.821442in}{0.597353in}}{\pgfqpoint{1.830650in}{0.597353in}}%
\pgfpathlineto{\pgfqpoint{1.830650in}{0.597353in}}%
\pgfpathclose%
\pgfusepath{stroke,fill}%
\end{pgfscope}%
\begin{pgfscope}%
\pgfpathrectangle{\pgfqpoint{0.050000in}{0.050000in}}{\pgfqpoint{2.419000in}{2.419000in}}%
\pgfusepath{clip}%
\pgfsetbuttcap%
\pgfsetroundjoin%
\definecolor{currentfill}{rgb}{0.200000,0.133333,0.533333}%
\pgfsetfillcolor{currentfill}%
\pgfsetfillopacity{0.470898}%
\pgfsetlinewidth{1.003750pt}%
\definecolor{currentstroke}{rgb}{0.200000,0.133333,0.533333}%
\pgfsetstrokecolor{currentstroke}%
\pgfsetstrokeopacity{0.470898}%
\pgfsetdash{}{0pt}%
\pgfpathmoveto{\pgfqpoint{7.010624in}{0.597353in}}%
\pgfpathcurveto{\pgfqpoint{7.019832in}{0.597353in}}{\pgfqpoint{7.028665in}{0.601011in}}{\pgfqpoint{7.035176in}{0.607523in}}%
\pgfpathcurveto{\pgfqpoint{7.041687in}{0.614034in}}{\pgfqpoint{7.045346in}{0.622867in}}{\pgfqpoint{7.045346in}{0.632075in}}%
\pgfpathcurveto{\pgfqpoint{7.045346in}{0.641284in}}{\pgfqpoint{7.041687in}{0.650116in}}{\pgfqpoint{7.035176in}{0.656627in}}%
\pgfpathcurveto{\pgfqpoint{7.028665in}{0.663139in}}{\pgfqpoint{7.019832in}{0.666797in}}{\pgfqpoint{7.010624in}{0.666797in}}%
\pgfpathcurveto{\pgfqpoint{7.001415in}{0.666797in}}{\pgfqpoint{6.992583in}{0.663139in}}{\pgfqpoint{6.986071in}{0.656627in}}%
\pgfpathcurveto{\pgfqpoint{6.979560in}{0.650116in}}{\pgfqpoint{6.975901in}{0.641284in}}{\pgfqpoint{6.975901in}{0.632075in}}%
\pgfpathcurveto{\pgfqpoint{6.975901in}{0.622867in}}{\pgfqpoint{6.979560in}{0.614034in}}{\pgfqpoint{6.986071in}{0.607523in}}%
\pgfpathcurveto{\pgfqpoint{6.992583in}{0.601011in}}{\pgfqpoint{7.001415in}{0.597353in}}{\pgfqpoint{7.010624in}{0.597353in}}%
\pgfpathlineto{\pgfqpoint{7.010624in}{0.597353in}}%
\pgfpathclose%
\pgfusepath{stroke,fill}%
\end{pgfscope}%
\begin{pgfscope}%
\pgfpathrectangle{\pgfqpoint{0.050000in}{0.050000in}}{\pgfqpoint{2.419000in}{2.419000in}}%
\pgfusepath{clip}%
\pgfsetbuttcap%
\pgfsetroundjoin%
\definecolor{currentfill}{rgb}{0.200000,0.133333,0.533333}%
\pgfsetfillcolor{currentfill}%
\pgfsetfillopacity{0.478278}%
\pgfsetlinewidth{1.003750pt}%
\definecolor{currentstroke}{rgb}{0.200000,0.133333,0.533333}%
\pgfsetstrokecolor{currentstroke}%
\pgfsetstrokeopacity{0.478278}%
\pgfsetdash{}{0pt}%
\pgfpathmoveto{\pgfqpoint{-0.303342in}{0.442023in}}%
\pgfpathcurveto{\pgfqpoint{-0.294134in}{0.442023in}}{\pgfqpoint{-0.285301in}{0.445682in}}{\pgfqpoint{-0.278790in}{0.452193in}}%
\pgfpathcurveto{\pgfqpoint{-0.272279in}{0.458705in}}{\pgfqpoint{-0.268620in}{0.467537in}}{\pgfqpoint{-0.268620in}{0.476745in}}%
\pgfpathcurveto{\pgfqpoint{-0.268620in}{0.485954in}}{\pgfqpoint{-0.272279in}{0.494786in}}{\pgfqpoint{-0.278790in}{0.501298in}}%
\pgfpathcurveto{\pgfqpoint{-0.285301in}{0.507809in}}{\pgfqpoint{-0.294134in}{0.511468in}}{\pgfqpoint{-0.303342in}{0.511468in}}%
\pgfpathcurveto{\pgfqpoint{-0.312551in}{0.511468in}}{\pgfqpoint{-0.321383in}{0.507809in}}{\pgfqpoint{-0.327895in}{0.501298in}}%
\pgfpathcurveto{\pgfqpoint{-0.334406in}{0.494786in}}{\pgfqpoint{-0.338065in}{0.485954in}}{\pgfqpoint{-0.338065in}{0.476745in}}%
\pgfpathcurveto{\pgfqpoint{-0.338065in}{0.467537in}}{\pgfqpoint{-0.334406in}{0.458705in}}{\pgfqpoint{-0.327895in}{0.452193in}}%
\pgfpathcurveto{\pgfqpoint{-0.321383in}{0.445682in}}{\pgfqpoint{-0.312551in}{0.442023in}}{\pgfqpoint{-0.303342in}{0.442023in}}%
\pgfpathlineto{\pgfqpoint{-0.303342in}{0.442023in}}%
\pgfpathclose%
\pgfusepath{stroke,fill}%
\end{pgfscope}%
\begin{pgfscope}%
\pgfpathrectangle{\pgfqpoint{0.050000in}{0.050000in}}{\pgfqpoint{2.419000in}{2.419000in}}%
\pgfusepath{clip}%
\pgfsetbuttcap%
\pgfsetroundjoin%
\definecolor{currentfill}{rgb}{0.200000,0.133333,0.533333}%
\pgfsetfillcolor{currentfill}%
\pgfsetfillopacity{0.478278}%
\pgfsetlinewidth{1.003750pt}%
\definecolor{currentstroke}{rgb}{0.200000,0.133333,0.533333}%
\pgfsetstrokecolor{currentstroke}%
\pgfsetstrokeopacity{0.478278}%
\pgfsetdash{}{0pt}%
\pgfpathmoveto{\pgfqpoint{4.974641in}{0.442023in}}%
\pgfpathcurveto{\pgfqpoint{4.983850in}{0.442023in}}{\pgfqpoint{4.992682in}{0.445682in}}{\pgfqpoint{4.999194in}{0.452193in}}%
\pgfpathcurveto{\pgfqpoint{5.005705in}{0.458705in}}{\pgfqpoint{5.009364in}{0.467537in}}{\pgfqpoint{5.009364in}{0.476745in}}%
\pgfpathcurveto{\pgfqpoint{5.009364in}{0.485954in}}{\pgfqpoint{5.005705in}{0.494786in}}{\pgfqpoint{4.999194in}{0.501298in}}%
\pgfpathcurveto{\pgfqpoint{4.992682in}{0.507809in}}{\pgfqpoint{4.983850in}{0.511468in}}{\pgfqpoint{4.974641in}{0.511468in}}%
\pgfpathcurveto{\pgfqpoint{4.965433in}{0.511468in}}{\pgfqpoint{4.956600in}{0.507809in}}{\pgfqpoint{4.950089in}{0.501298in}}%
\pgfpathcurveto{\pgfqpoint{4.943578in}{0.494786in}}{\pgfqpoint{4.939919in}{0.485954in}}{\pgfqpoint{4.939919in}{0.476745in}}%
\pgfpathcurveto{\pgfqpoint{4.939919in}{0.467537in}}{\pgfqpoint{4.943578in}{0.458705in}}{\pgfqpoint{4.950089in}{0.452193in}}%
\pgfpathcurveto{\pgfqpoint{4.956600in}{0.445682in}}{\pgfqpoint{4.965433in}{0.442023in}}{\pgfqpoint{4.974641in}{0.442023in}}%
\pgfpathlineto{\pgfqpoint{4.974641in}{0.442023in}}%
\pgfpathclose%
\pgfusepath{stroke,fill}%
\end{pgfscope}%
\begin{pgfscope}%
\pgfpathrectangle{\pgfqpoint{0.050000in}{0.050000in}}{\pgfqpoint{2.419000in}{2.419000in}}%
\pgfusepath{clip}%
\pgfsetbuttcap%
\pgfsetroundjoin%
\definecolor{currentfill}{rgb}{0.200000,0.133333,0.533333}%
\pgfsetfillcolor{currentfill}%
\pgfsetfillopacity{0.478278}%
\pgfsetlinewidth{1.003750pt}%
\definecolor{currentstroke}{rgb}{0.200000,0.133333,0.533333}%
\pgfsetstrokecolor{currentstroke}%
\pgfsetstrokeopacity{0.478278}%
\pgfsetdash{}{0pt}%
\pgfpathmoveto{\pgfqpoint{10.252625in}{0.442023in}}%
\pgfpathcurveto{\pgfqpoint{10.261834in}{0.442023in}}{\pgfqpoint{10.270666in}{0.445682in}}{\pgfqpoint{10.277178in}{0.452193in}}%
\pgfpathcurveto{\pgfqpoint{10.283689in}{0.458705in}}{\pgfqpoint{10.287347in}{0.467537in}}{\pgfqpoint{10.287347in}{0.476745in}}%
\pgfpathcurveto{\pgfqpoint{10.287347in}{0.485954in}}{\pgfqpoint{10.283689in}{0.494786in}}{\pgfqpoint{10.277178in}{0.501298in}}%
\pgfpathcurveto{\pgfqpoint{10.270666in}{0.507809in}}{\pgfqpoint{10.261834in}{0.511468in}}{\pgfqpoint{10.252625in}{0.511468in}}%
\pgfpathcurveto{\pgfqpoint{10.243417in}{0.511468in}}{\pgfqpoint{10.234584in}{0.507809in}}{\pgfqpoint{10.228073in}{0.501298in}}%
\pgfpathcurveto{\pgfqpoint{10.221562in}{0.494786in}}{\pgfqpoint{10.217903in}{0.485954in}}{\pgfqpoint{10.217903in}{0.476745in}}%
\pgfpathcurveto{\pgfqpoint{10.217903in}{0.467537in}}{\pgfqpoint{10.221562in}{0.458705in}}{\pgfqpoint{10.228073in}{0.452193in}}%
\pgfpathcurveto{\pgfqpoint{10.234584in}{0.445682in}}{\pgfqpoint{10.243417in}{0.442023in}}{\pgfqpoint{10.252625in}{0.442023in}}%
\pgfpathlineto{\pgfqpoint{10.252625in}{0.442023in}}%
\pgfpathclose%
\pgfusepath{stroke,fill}%
\end{pgfscope}%
\begin{pgfscope}%
\pgfpathrectangle{\pgfqpoint{0.050000in}{0.050000in}}{\pgfqpoint{2.419000in}{2.419000in}}%
\pgfusepath{clip}%
\pgfsetbuttcap%
\pgfsetroundjoin%
\definecolor{currentfill}{rgb}{0.200000,0.133333,0.533333}%
\pgfsetfillcolor{currentfill}%
\pgfsetfillopacity{0.485942}%
\pgfsetlinewidth{1.003750pt}%
\definecolor{currentstroke}{rgb}{0.200000,0.133333,0.533333}%
\pgfsetstrokecolor{currentstroke}%
\pgfsetstrokeopacity{0.485942}%
\pgfsetdash{}{0pt}%
\pgfpathmoveto{\pgfqpoint{-2.519647in}{0.280702in}}%
\pgfpathcurveto{\pgfqpoint{-2.510439in}{0.280702in}}{\pgfqpoint{-2.501606in}{0.284361in}}{\pgfqpoint{-2.495095in}{0.290872in}}%
\pgfpathcurveto{\pgfqpoint{-2.488584in}{0.297383in}}{\pgfqpoint{-2.484925in}{0.306216in}}{\pgfqpoint{-2.484925in}{0.315424in}}%
\pgfpathcurveto{\pgfqpoint{-2.484925in}{0.324633in}}{\pgfqpoint{-2.488584in}{0.333465in}}{\pgfqpoint{-2.495095in}{0.339977in}}%
\pgfpathcurveto{\pgfqpoint{-2.501606in}{0.346488in}}{\pgfqpoint{-2.510439in}{0.350147in}}{\pgfqpoint{-2.519647in}{0.350147in}}%
\pgfpathcurveto{\pgfqpoint{-2.528856in}{0.350147in}}{\pgfqpoint{-2.537688in}{0.346488in}}{\pgfqpoint{-2.544200in}{0.339977in}}%
\pgfpathcurveto{\pgfqpoint{-2.550711in}{0.333465in}}{\pgfqpoint{-2.554370in}{0.324633in}}{\pgfqpoint{-2.554370in}{0.315424in}}%
\pgfpathcurveto{\pgfqpoint{-2.554370in}{0.306216in}}{\pgfqpoint{-2.550711in}{0.297383in}}{\pgfqpoint{-2.544200in}{0.290872in}}%
\pgfpathcurveto{\pgfqpoint{-2.537688in}{0.284361in}}{\pgfqpoint{-2.528856in}{0.280702in}}{\pgfqpoint{-2.519647in}{0.280702in}}%
\pgfpathlineto{\pgfqpoint{-2.519647in}{0.280702in}}%
\pgfpathclose%
\pgfusepath{stroke,fill}%
\end{pgfscope}%
\begin{pgfscope}%
\pgfpathrectangle{\pgfqpoint{0.050000in}{0.050000in}}{\pgfqpoint{2.419000in}{2.419000in}}%
\pgfusepath{clip}%
\pgfsetbuttcap%
\pgfsetroundjoin%
\definecolor{currentfill}{rgb}{0.200000,0.133333,0.533333}%
\pgfsetfillcolor{currentfill}%
\pgfsetfillopacity{0.485942}%
\pgfsetlinewidth{1.003750pt}%
\definecolor{currentstroke}{rgb}{0.200000,0.133333,0.533333}%
\pgfsetstrokecolor{currentstroke}%
\pgfsetstrokeopacity{0.485942}%
\pgfsetdash{}{0pt}%
\pgfpathmoveto{\pgfqpoint{8.239902in}{0.280702in}}%
\pgfpathcurveto{\pgfqpoint{8.249111in}{0.280702in}}{\pgfqpoint{8.257943in}{0.284361in}}{\pgfqpoint{8.264455in}{0.290872in}}%
\pgfpathcurveto{\pgfqpoint{8.270966in}{0.297383in}}{\pgfqpoint{8.274625in}{0.306216in}}{\pgfqpoint{8.274625in}{0.315424in}}%
\pgfpathcurveto{\pgfqpoint{8.274625in}{0.324633in}}{\pgfqpoint{8.270966in}{0.333465in}}{\pgfqpoint{8.264455in}{0.339977in}}%
\pgfpathcurveto{\pgfqpoint{8.257943in}{0.346488in}}{\pgfqpoint{8.249111in}{0.350147in}}{\pgfqpoint{8.239902in}{0.350147in}}%
\pgfpathcurveto{\pgfqpoint{8.230694in}{0.350147in}}{\pgfqpoint{8.221861in}{0.346488in}}{\pgfqpoint{8.215350in}{0.339977in}}%
\pgfpathcurveto{\pgfqpoint{8.208839in}{0.333465in}}{\pgfqpoint{8.205180in}{0.324633in}}{\pgfqpoint{8.205180in}{0.315424in}}%
\pgfpathcurveto{\pgfqpoint{8.205180in}{0.306216in}}{\pgfqpoint{8.208839in}{0.297383in}}{\pgfqpoint{8.215350in}{0.290872in}}%
\pgfpathcurveto{\pgfqpoint{8.221861in}{0.284361in}}{\pgfqpoint{8.230694in}{0.280702in}}{\pgfqpoint{8.239902in}{0.280702in}}%
\pgfpathlineto{\pgfqpoint{8.239902in}{0.280702in}}%
\pgfpathclose%
\pgfusepath{stroke,fill}%
\end{pgfscope}%
\begin{pgfscope}%
\pgfpathrectangle{\pgfqpoint{0.050000in}{0.050000in}}{\pgfqpoint{2.419000in}{2.419000in}}%
\pgfusepath{clip}%
\pgfsetbuttcap%
\pgfsetroundjoin%
\definecolor{currentfill}{rgb}{0.200000,0.133333,0.533333}%
\pgfsetfillcolor{currentfill}%
\pgfsetfillopacity{0.485942}%
\pgfsetlinewidth{1.003750pt}%
\definecolor{currentstroke}{rgb}{0.200000,0.133333,0.533333}%
\pgfsetstrokecolor{currentstroke}%
\pgfsetstrokeopacity{0.485942}%
\pgfsetdash{}{0pt}%
\pgfpathmoveto{\pgfqpoint{2.860128in}{0.280702in}}%
\pgfpathcurveto{\pgfqpoint{2.869336in}{0.280702in}}{\pgfqpoint{2.878168in}{0.284361in}}{\pgfqpoint{2.884680in}{0.290872in}}%
\pgfpathcurveto{\pgfqpoint{2.891191in}{0.297383in}}{\pgfqpoint{2.894850in}{0.306216in}}{\pgfqpoint{2.894850in}{0.315424in}}%
\pgfpathcurveto{\pgfqpoint{2.894850in}{0.324633in}}{\pgfqpoint{2.891191in}{0.333465in}}{\pgfqpoint{2.884680in}{0.339977in}}%
\pgfpathcurveto{\pgfqpoint{2.878168in}{0.346488in}}{\pgfqpoint{2.869336in}{0.350147in}}{\pgfqpoint{2.860128in}{0.350147in}}%
\pgfpathcurveto{\pgfqpoint{2.850919in}{0.350147in}}{\pgfqpoint{2.842087in}{0.346488in}}{\pgfqpoint{2.835575in}{0.339977in}}%
\pgfpathcurveto{\pgfqpoint{2.829064in}{0.333465in}}{\pgfqpoint{2.825405in}{0.324633in}}{\pgfqpoint{2.825405in}{0.315424in}}%
\pgfpathcurveto{\pgfqpoint{2.825405in}{0.306216in}}{\pgfqpoint{2.829064in}{0.297383in}}{\pgfqpoint{2.835575in}{0.290872in}}%
\pgfpathcurveto{\pgfqpoint{2.842087in}{0.284361in}}{\pgfqpoint{2.850919in}{0.280702in}}{\pgfqpoint{2.860128in}{0.280702in}}%
\pgfpathlineto{\pgfqpoint{2.860128in}{0.280702in}}%
\pgfpathclose%
\pgfusepath{stroke,fill}%
\end{pgfscope}%
\begin{pgfscope}%
\pgfpathrectangle{\pgfqpoint{0.050000in}{0.050000in}}{\pgfqpoint{2.419000in}{2.419000in}}%
\pgfusepath{clip}%
\pgfsetbuttcap%
\pgfsetroundjoin%
\definecolor{currentfill}{rgb}{0.200000,0.133333,0.533333}%
\pgfsetfillcolor{currentfill}%
\pgfsetfillopacity{0.493908}%
\pgfsetlinewidth{1.003750pt}%
\definecolor{currentstroke}{rgb}{0.200000,0.133333,0.533333}%
\pgfsetstrokecolor{currentstroke}%
\pgfsetstrokeopacity{0.493908}%
\pgfsetdash{}{0pt}%
\pgfpathmoveto{\pgfqpoint{-4.823120in}{0.113036in}}%
\pgfpathcurveto{\pgfqpoint{-4.813912in}{0.113036in}}{\pgfqpoint{-4.805079in}{0.116695in}}{\pgfqpoint{-4.798568in}{0.123206in}}%
\pgfpathcurveto{\pgfqpoint{-4.792057in}{0.129718in}}{\pgfqpoint{-4.788398in}{0.138550in}}{\pgfqpoint{-4.788398in}{0.147759in}}%
\pgfpathcurveto{\pgfqpoint{-4.788398in}{0.156967in}}{\pgfqpoint{-4.792057in}{0.165800in}}{\pgfqpoint{-4.798568in}{0.172311in}}%
\pgfpathcurveto{\pgfqpoint{-4.805079in}{0.178822in}}{\pgfqpoint{-4.813912in}{0.182481in}}{\pgfqpoint{-4.823120in}{0.182481in}}%
\pgfpathcurveto{\pgfqpoint{-4.832329in}{0.182481in}}{\pgfqpoint{-4.841161in}{0.178822in}}{\pgfqpoint{-4.847673in}{0.172311in}}%
\pgfpathcurveto{\pgfqpoint{-4.854184in}{0.165800in}}{\pgfqpoint{-4.857843in}{0.156967in}}{\pgfqpoint{-4.857843in}{0.147759in}}%
\pgfpathcurveto{\pgfqpoint{-4.857843in}{0.138550in}}{\pgfqpoint{-4.854184in}{0.129718in}}{\pgfqpoint{-4.847673in}{0.123206in}}%
\pgfpathcurveto{\pgfqpoint{-4.841161in}{0.116695in}}{\pgfqpoint{-4.832329in}{0.113036in}}{\pgfqpoint{-4.823120in}{0.113036in}}%
\pgfpathlineto{\pgfqpoint{-4.823120in}{0.113036in}}%
\pgfpathclose%
\pgfusepath{stroke,fill}%
\end{pgfscope}%
\begin{pgfscope}%
\pgfpathrectangle{\pgfqpoint{0.050000in}{0.050000in}}{\pgfqpoint{2.419000in}{2.419000in}}%
\pgfusepath{clip}%
\pgfsetbuttcap%
\pgfsetroundjoin%
\definecolor{currentfill}{rgb}{0.200000,0.133333,0.533333}%
\pgfsetfillcolor{currentfill}%
\pgfsetfillopacity{0.493908}%
\pgfsetlinewidth{1.003750pt}%
\definecolor{currentstroke}{rgb}{0.200000,0.133333,0.533333}%
\pgfsetstrokecolor{currentstroke}%
\pgfsetstrokeopacity{0.493908}%
\pgfsetdash{}{0pt}%
\pgfpathmoveto{\pgfqpoint{0.662449in}{0.113036in}}%
\pgfpathcurveto{\pgfqpoint{0.671657in}{0.113036in}}{\pgfqpoint{0.680490in}{0.116695in}}{\pgfqpoint{0.687001in}{0.123206in}}%
\pgfpathcurveto{\pgfqpoint{0.693513in}{0.129718in}}{\pgfqpoint{0.697171in}{0.138550in}}{\pgfqpoint{0.697171in}{0.147759in}}%
\pgfpathcurveto{\pgfqpoint{0.697171in}{0.156967in}}{\pgfqpoint{0.693513in}{0.165800in}}{\pgfqpoint{0.687001in}{0.172311in}}%
\pgfpathcurveto{\pgfqpoint{0.680490in}{0.178822in}}{\pgfqpoint{0.671657in}{0.182481in}}{\pgfqpoint{0.662449in}{0.182481in}}%
\pgfpathcurveto{\pgfqpoint{0.653240in}{0.182481in}}{\pgfqpoint{0.644408in}{0.178822in}}{\pgfqpoint{0.637897in}{0.172311in}}%
\pgfpathcurveto{\pgfqpoint{0.631385in}{0.165800in}}{\pgfqpoint{0.627727in}{0.156967in}}{\pgfqpoint{0.627727in}{0.147759in}}%
\pgfpathcurveto{\pgfqpoint{0.627727in}{0.138550in}}{\pgfqpoint{0.631385in}{0.129718in}}{\pgfqpoint{0.637897in}{0.123206in}}%
\pgfpathcurveto{\pgfqpoint{0.644408in}{0.116695in}}{\pgfqpoint{0.653240in}{0.113036in}}{\pgfqpoint{0.662449in}{0.113036in}}%
\pgfpathlineto{\pgfqpoint{0.662449in}{0.113036in}}%
\pgfpathclose%
\pgfusepath{stroke,fill}%
\end{pgfscope}%
\begin{pgfscope}%
\pgfpathrectangle{\pgfqpoint{0.050000in}{0.050000in}}{\pgfqpoint{2.419000in}{2.419000in}}%
\pgfusepath{clip}%
\pgfsetbuttcap%
\pgfsetroundjoin%
\definecolor{currentfill}{rgb}{0.200000,0.133333,0.533333}%
\pgfsetfillcolor{currentfill}%
\pgfsetfillopacity{0.493908}%
\pgfsetlinewidth{1.003750pt}%
\definecolor{currentstroke}{rgb}{0.200000,0.133333,0.533333}%
\pgfsetstrokecolor{currentstroke}%
\pgfsetstrokeopacity{0.493908}%
\pgfsetdash{}{0pt}%
\pgfpathmoveto{\pgfqpoint{6.148018in}{0.113036in}}%
\pgfpathcurveto{\pgfqpoint{6.157227in}{0.113036in}}{\pgfqpoint{6.166059in}{0.116695in}}{\pgfqpoint{6.172570in}{0.123206in}}%
\pgfpathcurveto{\pgfqpoint{6.179082in}{0.129718in}}{\pgfqpoint{6.182740in}{0.138550in}}{\pgfqpoint{6.182740in}{0.147759in}}%
\pgfpathcurveto{\pgfqpoint{6.182740in}{0.156967in}}{\pgfqpoint{6.179082in}{0.165800in}}{\pgfqpoint{6.172570in}{0.172311in}}%
\pgfpathcurveto{\pgfqpoint{6.166059in}{0.178822in}}{\pgfqpoint{6.157227in}{0.182481in}}{\pgfqpoint{6.148018in}{0.182481in}}%
\pgfpathcurveto{\pgfqpoint{6.138810in}{0.182481in}}{\pgfqpoint{6.129977in}{0.178822in}}{\pgfqpoint{6.123466in}{0.172311in}}%
\pgfpathcurveto{\pgfqpoint{6.116954in}{0.165800in}}{\pgfqpoint{6.113296in}{0.156967in}}{\pgfqpoint{6.113296in}{0.147759in}}%
\pgfpathcurveto{\pgfqpoint{6.113296in}{0.138550in}}{\pgfqpoint{6.116954in}{0.129718in}}{\pgfqpoint{6.123466in}{0.123206in}}%
\pgfpathcurveto{\pgfqpoint{6.129977in}{0.116695in}}{\pgfqpoint{6.138810in}{0.113036in}}{\pgfqpoint{6.148018in}{0.113036in}}%
\pgfpathlineto{\pgfqpoint{6.148018in}{0.113036in}}%
\pgfpathclose%
\pgfusepath{stroke,fill}%
\end{pgfscope}%
\begin{pgfscope}%
\pgfpathrectangle{\pgfqpoint{0.050000in}{0.050000in}}{\pgfqpoint{2.419000in}{2.419000in}}%
\pgfusepath{clip}%
\pgfsetbuttcap%
\pgfsetroundjoin%
\definecolor{currentfill}{rgb}{0.200000,0.133333,0.533333}%
\pgfsetfillcolor{currentfill}%
\pgfsetfillopacity{0.502193}%
\pgfsetlinewidth{1.003750pt}%
\definecolor{currentstroke}{rgb}{0.200000,0.133333,0.533333}%
\pgfsetstrokecolor{currentstroke}%
\pgfsetstrokeopacity{0.502193}%
\pgfsetdash{}{0pt}%
\pgfpathmoveto{\pgfqpoint{-1.623399in}{-0.061356in}}%
\pgfpathcurveto{\pgfqpoint{-1.614191in}{-0.061356in}}{\pgfqpoint{-1.605358in}{-0.057698in}}{\pgfqpoint{-1.598847in}{-0.051186in}}%
\pgfpathcurveto{\pgfqpoint{-1.592336in}{-0.044675in}}{\pgfqpoint{-1.588677in}{-0.035842in}}{\pgfqpoint{-1.588677in}{-0.026634in}}%
\pgfpathcurveto{\pgfqpoint{-1.588677in}{-0.017426in}}{\pgfqpoint{-1.592336in}{-0.008593in}}{\pgfqpoint{-1.598847in}{-0.002082in}}%
\pgfpathcurveto{\pgfqpoint{-1.605358in}{0.004430in}}{\pgfqpoint{-1.614191in}{0.008088in}}{\pgfqpoint{-1.623399in}{0.008088in}}%
\pgfpathcurveto{\pgfqpoint{-1.632608in}{0.008088in}}{\pgfqpoint{-1.641440in}{0.004430in}}{\pgfqpoint{-1.647952in}{-0.002082in}}%
\pgfpathcurveto{\pgfqpoint{-1.654463in}{-0.008593in}}{\pgfqpoint{-1.658122in}{-0.017426in}}{\pgfqpoint{-1.658122in}{-0.026634in}}%
\pgfpathcurveto{\pgfqpoint{-1.658122in}{-0.035842in}}{\pgfqpoint{-1.654463in}{-0.044675in}}{\pgfqpoint{-1.647952in}{-0.051186in}}%
\pgfpathcurveto{\pgfqpoint{-1.641440in}{-0.057698in}}{\pgfqpoint{-1.632608in}{-0.061356in}}{\pgfqpoint{-1.623399in}{-0.061356in}}%
\pgfpathlineto{\pgfqpoint{-1.623399in}{-0.061356in}}%
\pgfpathclose%
\pgfusepath{stroke,fill}%
\end{pgfscope}%
\begin{pgfscope}%
\pgfpathrectangle{\pgfqpoint{0.050000in}{0.050000in}}{\pgfqpoint{2.419000in}{2.419000in}}%
\pgfusepath{clip}%
\pgfsetbuttcap%
\pgfsetroundjoin%
\definecolor{currentfill}{rgb}{0.200000,0.133333,0.533333}%
\pgfsetfillcolor{currentfill}%
\pgfsetfillopacity{0.502193}%
\pgfsetlinewidth{1.003750pt}%
\definecolor{currentstroke}{rgb}{0.200000,0.133333,0.533333}%
\pgfsetstrokecolor{currentstroke}%
\pgfsetstrokeopacity{0.502193}%
\pgfsetdash{}{0pt}%
\pgfpathmoveto{\pgfqpoint{9.567817in}{-0.061356in}}%
\pgfpathcurveto{\pgfqpoint{9.577025in}{-0.061356in}}{\pgfqpoint{9.585858in}{-0.057698in}}{\pgfqpoint{9.592369in}{-0.051186in}}%
\pgfpathcurveto{\pgfqpoint{9.598881in}{-0.044675in}}{\pgfqpoint{9.602539in}{-0.035842in}}{\pgfqpoint{9.602539in}{-0.026634in}}%
\pgfpathcurveto{\pgfqpoint{9.602539in}{-0.017426in}}{\pgfqpoint{9.598881in}{-0.008593in}}{\pgfqpoint{9.592369in}{-0.002082in}}%
\pgfpathcurveto{\pgfqpoint{9.585858in}{0.004430in}}{\pgfqpoint{9.577025in}{0.008088in}}{\pgfqpoint{9.567817in}{0.008088in}}%
\pgfpathcurveto{\pgfqpoint{9.558608in}{0.008088in}}{\pgfqpoint{9.549776in}{0.004430in}}{\pgfqpoint{9.543265in}{-0.002082in}}%
\pgfpathcurveto{\pgfqpoint{9.536753in}{-0.008593in}}{\pgfqpoint{9.533095in}{-0.017426in}}{\pgfqpoint{9.533095in}{-0.026634in}}%
\pgfpathcurveto{\pgfqpoint{9.533095in}{-0.035842in}}{\pgfqpoint{9.536753in}{-0.044675in}}{\pgfqpoint{9.543265in}{-0.051186in}}%
\pgfpathcurveto{\pgfqpoint{9.549776in}{-0.057698in}}{\pgfqpoint{9.558608in}{-0.061356in}}{\pgfqpoint{9.567817in}{-0.061356in}}%
\pgfpathlineto{\pgfqpoint{9.567817in}{-0.061356in}}%
\pgfpathclose%
\pgfusepath{stroke,fill}%
\end{pgfscope}%
\begin{pgfscope}%
\pgfpathrectangle{\pgfqpoint{0.050000in}{0.050000in}}{\pgfqpoint{2.419000in}{2.419000in}}%
\pgfusepath{clip}%
\pgfsetbuttcap%
\pgfsetroundjoin%
\definecolor{currentfill}{rgb}{0.200000,0.133333,0.533333}%
\pgfsetfillcolor{currentfill}%
\pgfsetfillopacity{0.502193}%
\pgfsetlinewidth{1.003750pt}%
\definecolor{currentstroke}{rgb}{0.200000,0.133333,0.533333}%
\pgfsetstrokecolor{currentstroke}%
\pgfsetstrokeopacity{0.502193}%
\pgfsetdash{}{0pt}%
\pgfpathmoveto{\pgfqpoint{3.972209in}{-0.061356in}}%
\pgfpathcurveto{\pgfqpoint{3.981417in}{-0.061356in}}{\pgfqpoint{3.990250in}{-0.057698in}}{\pgfqpoint{3.996761in}{-0.051186in}}%
\pgfpathcurveto{\pgfqpoint{4.003272in}{-0.044675in}}{\pgfqpoint{4.006931in}{-0.035842in}}{\pgfqpoint{4.006931in}{-0.026634in}}%
\pgfpathcurveto{\pgfqpoint{4.006931in}{-0.017426in}}{\pgfqpoint{4.003272in}{-0.008593in}}{\pgfqpoint{3.996761in}{-0.002082in}}%
\pgfpathcurveto{\pgfqpoint{3.990250in}{0.004430in}}{\pgfqpoint{3.981417in}{0.008088in}}{\pgfqpoint{3.972209in}{0.008088in}}%
\pgfpathcurveto{\pgfqpoint{3.963000in}{0.008088in}}{\pgfqpoint{3.954168in}{0.004430in}}{\pgfqpoint{3.947656in}{-0.002082in}}%
\pgfpathcurveto{\pgfqpoint{3.941145in}{-0.008593in}}{\pgfqpoint{3.937487in}{-0.017426in}}{\pgfqpoint{3.937487in}{-0.026634in}}%
\pgfpathcurveto{\pgfqpoint{3.937487in}{-0.035842in}}{\pgfqpoint{3.941145in}{-0.044675in}}{\pgfqpoint{3.947656in}{-0.051186in}}%
\pgfpathcurveto{\pgfqpoint{3.954168in}{-0.057698in}}{\pgfqpoint{3.963000in}{-0.061356in}}{\pgfqpoint{3.972209in}{-0.061356in}}%
\pgfpathlineto{\pgfqpoint{3.972209in}{-0.061356in}}%
\pgfpathclose%
\pgfusepath{stroke,fill}%
\end{pgfscope}%
\begin{pgfscope}%
\pgfpathrectangle{\pgfqpoint{0.050000in}{0.050000in}}{\pgfqpoint{2.419000in}{2.419000in}}%
\pgfusepath{clip}%
\pgfsetbuttcap%
\pgfsetroundjoin%
\definecolor{currentfill}{rgb}{0.200000,0.133333,0.533333}%
\pgfsetfillcolor{currentfill}%
\pgfsetfillopacity{0.510818}%
\pgfsetlinewidth{1.003750pt}%
\definecolor{currentstroke}{rgb}{0.200000,0.133333,0.533333}%
\pgfsetstrokecolor{currentstroke}%
\pgfsetstrokeopacity{0.510818}%
\pgfsetdash{}{0pt}%
\pgfpathmoveto{\pgfqpoint{-4.002832in}{-0.242888in}}%
\pgfpathcurveto{\pgfqpoint{-3.993623in}{-0.242888in}}{\pgfqpoint{-3.984791in}{-0.239230in}}{\pgfqpoint{-3.978279in}{-0.232719in}}%
\pgfpathcurveto{\pgfqpoint{-3.971768in}{-0.226207in}}{\pgfqpoint{-3.968109in}{-0.217375in}}{\pgfqpoint{-3.968109in}{-0.208166in}}%
\pgfpathcurveto{\pgfqpoint{-3.968109in}{-0.198958in}}{\pgfqpoint{-3.971768in}{-0.190125in}}{\pgfqpoint{-3.978279in}{-0.183614in}}%
\pgfpathcurveto{\pgfqpoint{-3.984791in}{-0.177103in}}{\pgfqpoint{-3.993623in}{-0.173444in}}{\pgfqpoint{-4.002832in}{-0.173444in}}%
\pgfpathcurveto{\pgfqpoint{-4.012040in}{-0.173444in}}{\pgfqpoint{-4.020873in}{-0.177103in}}{\pgfqpoint{-4.027384in}{-0.183614in}}%
\pgfpathcurveto{\pgfqpoint{-4.033895in}{-0.190125in}}{\pgfqpoint{-4.037554in}{-0.198958in}}{\pgfqpoint{-4.037554in}{-0.208166in}}%
\pgfpathcurveto{\pgfqpoint{-4.037554in}{-0.217375in}}{\pgfqpoint{-4.033895in}{-0.226207in}}{\pgfqpoint{-4.027384in}{-0.232719in}}%
\pgfpathcurveto{\pgfqpoint{-4.020873in}{-0.239230in}}{\pgfqpoint{-4.012040in}{-0.242888in}}{\pgfqpoint{-4.002832in}{-0.242888in}}%
\pgfpathlineto{\pgfqpoint{-4.002832in}{-0.242888in}}%
\pgfpathclose%
\pgfusepath{stroke,fill}%
\end{pgfscope}%
\begin{pgfscope}%
\pgfpathrectangle{\pgfqpoint{0.050000in}{0.050000in}}{\pgfqpoint{2.419000in}{2.419000in}}%
\pgfusepath{clip}%
\pgfsetbuttcap%
\pgfsetroundjoin%
\definecolor{currentfill}{rgb}{0.200000,0.133333,0.533333}%
\pgfsetfillcolor{currentfill}%
\pgfsetfillopacity{0.510818}%
\pgfsetlinewidth{1.003750pt}%
\definecolor{currentstroke}{rgb}{0.200000,0.133333,0.533333}%
\pgfsetstrokecolor{currentstroke}%
\pgfsetstrokeopacity{0.510818}%
\pgfsetdash{}{0pt}%
\pgfpathmoveto{\pgfqpoint{1.707320in}{-0.242888in}}%
\pgfpathcurveto{\pgfqpoint{1.716529in}{-0.242888in}}{\pgfqpoint{1.725361in}{-0.239230in}}{\pgfqpoint{1.731873in}{-0.232719in}}%
\pgfpathcurveto{\pgfqpoint{1.738384in}{-0.226207in}}{\pgfqpoint{1.742043in}{-0.217375in}}{\pgfqpoint{1.742043in}{-0.208166in}}%
\pgfpathcurveto{\pgfqpoint{1.742043in}{-0.198958in}}{\pgfqpoint{1.738384in}{-0.190125in}}{\pgfqpoint{1.731873in}{-0.183614in}}%
\pgfpathcurveto{\pgfqpoint{1.725361in}{-0.177103in}}{\pgfqpoint{1.716529in}{-0.173444in}}{\pgfqpoint{1.707320in}{-0.173444in}}%
\pgfpathcurveto{\pgfqpoint{1.698112in}{-0.173444in}}{\pgfqpoint{1.689279in}{-0.177103in}}{\pgfqpoint{1.682768in}{-0.183614in}}%
\pgfpathcurveto{\pgfqpoint{1.676257in}{-0.190125in}}{\pgfqpoint{1.672598in}{-0.198958in}}{\pgfqpoint{1.672598in}{-0.208166in}}%
\pgfpathcurveto{\pgfqpoint{1.672598in}{-0.217375in}}{\pgfqpoint{1.676257in}{-0.226207in}}{\pgfqpoint{1.682768in}{-0.232719in}}%
\pgfpathcurveto{\pgfqpoint{1.689279in}{-0.239230in}}{\pgfqpoint{1.698112in}{-0.242888in}}{\pgfqpoint{1.707320in}{-0.242888in}}%
\pgfpathlineto{\pgfqpoint{1.707320in}{-0.242888in}}%
\pgfpathclose%
\pgfusepath{stroke,fill}%
\end{pgfscope}%
\begin{pgfscope}%
\pgfpathrectangle{\pgfqpoint{0.050000in}{0.050000in}}{\pgfqpoint{2.419000in}{2.419000in}}%
\pgfusepath{clip}%
\pgfsetbuttcap%
\pgfsetroundjoin%
\definecolor{currentfill}{rgb}{0.200000,0.133333,0.533333}%
\pgfsetfillcolor{currentfill}%
\pgfsetfillopacity{0.510818}%
\pgfsetlinewidth{1.003750pt}%
\definecolor{currentstroke}{rgb}{0.200000,0.133333,0.533333}%
\pgfsetstrokecolor{currentstroke}%
\pgfsetstrokeopacity{0.510818}%
\pgfsetdash{}{0pt}%
\pgfpathmoveto{\pgfqpoint{7.417472in}{-0.242888in}}%
\pgfpathcurveto{\pgfqpoint{7.426681in}{-0.242888in}}{\pgfqpoint{7.435513in}{-0.239230in}}{\pgfqpoint{7.442025in}{-0.232719in}}%
\pgfpathcurveto{\pgfqpoint{7.448536in}{-0.226207in}}{\pgfqpoint{7.452195in}{-0.217375in}}{\pgfqpoint{7.452195in}{-0.208166in}}%
\pgfpathcurveto{\pgfqpoint{7.452195in}{-0.198958in}}{\pgfqpoint{7.448536in}{-0.190125in}}{\pgfqpoint{7.442025in}{-0.183614in}}%
\pgfpathcurveto{\pgfqpoint{7.435513in}{-0.177103in}}{\pgfqpoint{7.426681in}{-0.173444in}}{\pgfqpoint{7.417472in}{-0.173444in}}%
\pgfpathcurveto{\pgfqpoint{7.408264in}{-0.173444in}}{\pgfqpoint{7.399432in}{-0.177103in}}{\pgfqpoint{7.392920in}{-0.183614in}}%
\pgfpathcurveto{\pgfqpoint{7.386409in}{-0.190125in}}{\pgfqpoint{7.382750in}{-0.198958in}}{\pgfqpoint{7.382750in}{-0.208166in}}%
\pgfpathcurveto{\pgfqpoint{7.382750in}{-0.217375in}}{\pgfqpoint{7.386409in}{-0.226207in}}{\pgfqpoint{7.392920in}{-0.232719in}}%
\pgfpathcurveto{\pgfqpoint{7.399432in}{-0.239230in}}{\pgfqpoint{7.408264in}{-0.242888in}}{\pgfqpoint{7.417472in}{-0.242888in}}%
\pgfpathlineto{\pgfqpoint{7.417472in}{-0.242888in}}%
\pgfpathclose%
\pgfusepath{stroke,fill}%
\end{pgfscope}%
\begin{pgfscope}%
\pgfpathrectangle{\pgfqpoint{0.050000in}{0.050000in}}{\pgfqpoint{2.419000in}{2.419000in}}%
\pgfusepath{clip}%
\pgfsetbuttcap%
\pgfsetroundjoin%
\definecolor{currentfill}{rgb}{0.200000,0.133333,0.533333}%
\pgfsetfillcolor{currentfill}%
\pgfsetfillopacity{0.519802}%
\pgfsetlinewidth{1.003750pt}%
\definecolor{currentstroke}{rgb}{0.200000,0.133333,0.533333}%
\pgfsetstrokecolor{currentstroke}%
\pgfsetstrokeopacity{0.519802}%
\pgfsetdash{}{0pt}%
\pgfpathmoveto{\pgfqpoint{-0.652232in}{-0.432008in}}%
\pgfpathcurveto{\pgfqpoint{-0.643023in}{-0.432008in}}{\pgfqpoint{-0.634191in}{-0.428350in}}{\pgfqpoint{-0.627680in}{-0.421838in}}%
\pgfpathcurveto{\pgfqpoint{-0.621168in}{-0.415327in}}{\pgfqpoint{-0.617510in}{-0.406494in}}{\pgfqpoint{-0.617510in}{-0.397286in}}%
\pgfpathcurveto{\pgfqpoint{-0.617510in}{-0.388077in}}{\pgfqpoint{-0.621168in}{-0.379245in}}{\pgfqpoint{-0.627680in}{-0.372734in}}%
\pgfpathcurveto{\pgfqpoint{-0.634191in}{-0.366222in}}{\pgfqpoint{-0.643023in}{-0.362564in}}{\pgfqpoint{-0.652232in}{-0.362564in}}%
\pgfpathcurveto{\pgfqpoint{-0.661440in}{-0.362564in}}{\pgfqpoint{-0.670273in}{-0.366222in}}{\pgfqpoint{-0.676784in}{-0.372734in}}%
\pgfpathcurveto{\pgfqpoint{-0.683296in}{-0.379245in}}{\pgfqpoint{-0.686954in}{-0.388077in}}{\pgfqpoint{-0.686954in}{-0.397286in}}%
\pgfpathcurveto{\pgfqpoint{-0.686954in}{-0.406494in}}{\pgfqpoint{-0.683296in}{-0.415327in}}{\pgfqpoint{-0.676784in}{-0.421838in}}%
\pgfpathcurveto{\pgfqpoint{-0.670273in}{-0.428350in}}{\pgfqpoint{-0.661440in}{-0.432008in}}{\pgfqpoint{-0.652232in}{-0.432008in}}%
\pgfpathlineto{\pgfqpoint{-0.652232in}{-0.432008in}}%
\pgfpathclose%
\pgfusepath{stroke,fill}%
\end{pgfscope}%
\begin{pgfscope}%
\pgfpathrectangle{\pgfqpoint{0.050000in}{0.050000in}}{\pgfqpoint{2.419000in}{2.419000in}}%
\pgfusepath{clip}%
\pgfsetbuttcap%
\pgfsetroundjoin%
\definecolor{currentfill}{rgb}{0.200000,0.133333,0.533333}%
\pgfsetfillcolor{currentfill}%
\pgfsetfillopacity{0.519802}%
\pgfsetlinewidth{1.003750pt}%
\definecolor{currentstroke}{rgb}{0.200000,0.133333,0.533333}%
\pgfsetstrokecolor{currentstroke}%
\pgfsetstrokeopacity{0.519802}%
\pgfsetdash{}{0pt}%
\pgfpathmoveto{\pgfqpoint{-6.481715in}{-0.432008in}}%
\pgfpathcurveto{\pgfqpoint{-6.472507in}{-0.432008in}}{\pgfqpoint{-6.463674in}{-0.428350in}}{\pgfqpoint{-6.457163in}{-0.421838in}}%
\pgfpathcurveto{\pgfqpoint{-6.450652in}{-0.415327in}}{\pgfqpoint{-6.446993in}{-0.406494in}}{\pgfqpoint{-6.446993in}{-0.397286in}}%
\pgfpathcurveto{\pgfqpoint{-6.446993in}{-0.388077in}}{\pgfqpoint{-6.450652in}{-0.379245in}}{\pgfqpoint{-6.457163in}{-0.372734in}}%
\pgfpathcurveto{\pgfqpoint{-6.463674in}{-0.366222in}}{\pgfqpoint{-6.472507in}{-0.362564in}}{\pgfqpoint{-6.481715in}{-0.362564in}}%
\pgfpathcurveto{\pgfqpoint{-6.490924in}{-0.362564in}}{\pgfqpoint{-6.499756in}{-0.366222in}}{\pgfqpoint{-6.506268in}{-0.372734in}}%
\pgfpathcurveto{\pgfqpoint{-6.512779in}{-0.379245in}}{\pgfqpoint{-6.516438in}{-0.388077in}}{\pgfqpoint{-6.516438in}{-0.397286in}}%
\pgfpathcurveto{\pgfqpoint{-6.516438in}{-0.406494in}}{\pgfqpoint{-6.512779in}{-0.415327in}}{\pgfqpoint{-6.506268in}{-0.421838in}}%
\pgfpathcurveto{\pgfqpoint{-6.499756in}{-0.428350in}}{\pgfqpoint{-6.490924in}{-0.432008in}}{\pgfqpoint{-6.481715in}{-0.432008in}}%
\pgfpathlineto{\pgfqpoint{-6.481715in}{-0.432008in}}%
\pgfpathclose%
\pgfusepath{stroke,fill}%
\end{pgfscope}%
\begin{pgfscope}%
\pgfpathrectangle{\pgfqpoint{0.050000in}{0.050000in}}{\pgfqpoint{2.419000in}{2.419000in}}%
\pgfusepath{clip}%
\pgfsetbuttcap%
\pgfsetroundjoin%
\definecolor{currentfill}{rgb}{0.200000,0.133333,0.533333}%
\pgfsetfillcolor{currentfill}%
\pgfsetfillopacity{0.519802}%
\pgfsetlinewidth{1.003750pt}%
\definecolor{currentstroke}{rgb}{0.200000,0.133333,0.533333}%
\pgfsetstrokecolor{currentstroke}%
\pgfsetstrokeopacity{0.519802}%
\pgfsetdash{}{0pt}%
\pgfpathmoveto{\pgfqpoint{5.177252in}{-0.432008in}}%
\pgfpathcurveto{\pgfqpoint{5.186460in}{-0.432008in}}{\pgfqpoint{5.195293in}{-0.428350in}}{\pgfqpoint{5.201804in}{-0.421838in}}%
\pgfpathcurveto{\pgfqpoint{5.208315in}{-0.415327in}}{\pgfqpoint{5.211974in}{-0.406494in}}{\pgfqpoint{5.211974in}{-0.397286in}}%
\pgfpathcurveto{\pgfqpoint{5.211974in}{-0.388077in}}{\pgfqpoint{5.208315in}{-0.379245in}}{\pgfqpoint{5.201804in}{-0.372734in}}%
\pgfpathcurveto{\pgfqpoint{5.195293in}{-0.366222in}}{\pgfqpoint{5.186460in}{-0.362564in}}{\pgfqpoint{5.177252in}{-0.362564in}}%
\pgfpathcurveto{\pgfqpoint{5.168043in}{-0.362564in}}{\pgfqpoint{5.159211in}{-0.366222in}}{\pgfqpoint{5.152699in}{-0.372734in}}%
\pgfpathcurveto{\pgfqpoint{5.146188in}{-0.379245in}}{\pgfqpoint{5.142529in}{-0.388077in}}{\pgfqpoint{5.142529in}{-0.397286in}}%
\pgfpathcurveto{\pgfqpoint{5.142529in}{-0.406494in}}{\pgfqpoint{5.146188in}{-0.415327in}}{\pgfqpoint{5.152699in}{-0.421838in}}%
\pgfpathcurveto{\pgfqpoint{5.159211in}{-0.428350in}}{\pgfqpoint{5.168043in}{-0.432008in}}{\pgfqpoint{5.177252in}{-0.432008in}}%
\pgfpathlineto{\pgfqpoint{5.177252in}{-0.432008in}}%
\pgfpathclose%
\pgfusepath{stroke,fill}%
\end{pgfscope}%
\begin{pgfscope}%
\pgfpathrectangle{\pgfqpoint{0.050000in}{0.050000in}}{\pgfqpoint{2.419000in}{2.419000in}}%
\pgfusepath{clip}%
\pgfsetbuttcap%
\pgfsetroundjoin%
\definecolor{currentfill}{rgb}{0.200000,0.133333,0.533333}%
\pgfsetfillcolor{currentfill}%
\pgfsetfillopacity{0.529171}%
\pgfsetlinewidth{1.003750pt}%
\definecolor{currentstroke}{rgb}{0.200000,0.133333,0.533333}%
\pgfsetstrokecolor{currentstroke}%
\pgfsetstrokeopacity{0.529171}%
\pgfsetdash{}{0pt}%
\pgfpathmoveto{\pgfqpoint{-3.112509in}{-0.629201in}}%
\pgfpathcurveto{\pgfqpoint{-3.103301in}{-0.629201in}}{\pgfqpoint{-3.094468in}{-0.625542in}}{\pgfqpoint{-3.087957in}{-0.619031in}}%
\pgfpathcurveto{\pgfqpoint{-3.081446in}{-0.612520in}}{\pgfqpoint{-3.077787in}{-0.603687in}}{\pgfqpoint{-3.077787in}{-0.594479in}}%
\pgfpathcurveto{\pgfqpoint{-3.077787in}{-0.585270in}}{\pgfqpoint{-3.081446in}{-0.576438in}}{\pgfqpoint{-3.087957in}{-0.569926in}}%
\pgfpathcurveto{\pgfqpoint{-3.094468in}{-0.563415in}}{\pgfqpoint{-3.103301in}{-0.559756in}}{\pgfqpoint{-3.112509in}{-0.559756in}}%
\pgfpathcurveto{\pgfqpoint{-3.121718in}{-0.559756in}}{\pgfqpoint{-3.130550in}{-0.563415in}}{\pgfqpoint{-3.137062in}{-0.569926in}}%
\pgfpathcurveto{\pgfqpoint{-3.143573in}{-0.576438in}}{\pgfqpoint{-3.147232in}{-0.585270in}}{\pgfqpoint{-3.147232in}{-0.594479in}}%
\pgfpathcurveto{\pgfqpoint{-3.147232in}{-0.603687in}}{\pgfqpoint{-3.143573in}{-0.612520in}}{\pgfqpoint{-3.137062in}{-0.619031in}}%
\pgfpathcurveto{\pgfqpoint{-3.130550in}{-0.625542in}}{\pgfqpoint{-3.121718in}{-0.629201in}}{\pgfqpoint{-3.112509in}{-0.629201in}}%
\pgfpathlineto{\pgfqpoint{-3.112509in}{-0.629201in}}%
\pgfpathclose%
\pgfusepath{stroke,fill}%
\end{pgfscope}%
\begin{pgfscope}%
\pgfpathrectangle{\pgfqpoint{0.050000in}{0.050000in}}{\pgfqpoint{2.419000in}{2.419000in}}%
\pgfusepath{clip}%
\pgfsetbuttcap%
\pgfsetroundjoin%
\definecolor{currentfill}{rgb}{0.200000,0.133333,0.533333}%
\pgfsetfillcolor{currentfill}%
\pgfsetfillopacity{0.529171}%
\pgfsetlinewidth{1.003750pt}%
\definecolor{currentstroke}{rgb}{0.200000,0.133333,0.533333}%
\pgfsetstrokecolor{currentstroke}%
\pgfsetstrokeopacity{0.529171}%
\pgfsetdash{}{0pt}%
\pgfpathmoveto{\pgfqpoint{2.841400in}{-0.629201in}}%
\pgfpathcurveto{\pgfqpoint{2.850608in}{-0.629201in}}{\pgfqpoint{2.859441in}{-0.625542in}}{\pgfqpoint{2.865952in}{-0.619031in}}%
\pgfpathcurveto{\pgfqpoint{2.872463in}{-0.612520in}}{\pgfqpoint{2.876122in}{-0.603687in}}{\pgfqpoint{2.876122in}{-0.594479in}}%
\pgfpathcurveto{\pgfqpoint{2.876122in}{-0.585270in}}{\pgfqpoint{2.872463in}{-0.576438in}}{\pgfqpoint{2.865952in}{-0.569926in}}%
\pgfpathcurveto{\pgfqpoint{2.859441in}{-0.563415in}}{\pgfqpoint{2.850608in}{-0.559756in}}{\pgfqpoint{2.841400in}{-0.559756in}}%
\pgfpathcurveto{\pgfqpoint{2.832191in}{-0.559756in}}{\pgfqpoint{2.823359in}{-0.563415in}}{\pgfqpoint{2.816847in}{-0.569926in}}%
\pgfpathcurveto{\pgfqpoint{2.810336in}{-0.576438in}}{\pgfqpoint{2.806677in}{-0.585270in}}{\pgfqpoint{2.806677in}{-0.594479in}}%
\pgfpathcurveto{\pgfqpoint{2.806677in}{-0.603687in}}{\pgfqpoint{2.810336in}{-0.612520in}}{\pgfqpoint{2.816847in}{-0.619031in}}%
\pgfpathcurveto{\pgfqpoint{2.823359in}{-0.625542in}}{\pgfqpoint{2.832191in}{-0.629201in}}{\pgfqpoint{2.841400in}{-0.629201in}}%
\pgfpathlineto{\pgfqpoint{2.841400in}{-0.629201in}}%
\pgfpathclose%
\pgfusepath{stroke,fill}%
\end{pgfscope}%
\begin{pgfscope}%
\pgfpathrectangle{\pgfqpoint{0.050000in}{0.050000in}}{\pgfqpoint{2.419000in}{2.419000in}}%
\pgfusepath{clip}%
\pgfsetbuttcap%
\pgfsetroundjoin%
\definecolor{currentfill}{rgb}{0.200000,0.133333,0.533333}%
\pgfsetfillcolor{currentfill}%
\pgfsetfillopacity{0.529171}%
\pgfsetlinewidth{1.003750pt}%
\definecolor{currentstroke}{rgb}{0.200000,0.133333,0.533333}%
\pgfsetstrokecolor{currentstroke}%
\pgfsetstrokeopacity{0.529171}%
\pgfsetdash{}{0pt}%
\pgfpathmoveto{\pgfqpoint{8.795309in}{-0.629201in}}%
\pgfpathcurveto{\pgfqpoint{8.804517in}{-0.629201in}}{\pgfqpoint{8.813350in}{-0.625542in}}{\pgfqpoint{8.819861in}{-0.619031in}}%
\pgfpathcurveto{\pgfqpoint{8.826372in}{-0.612520in}}{\pgfqpoint{8.830031in}{-0.603687in}}{\pgfqpoint{8.830031in}{-0.594479in}}%
\pgfpathcurveto{\pgfqpoint{8.830031in}{-0.585270in}}{\pgfqpoint{8.826372in}{-0.576438in}}{\pgfqpoint{8.819861in}{-0.569926in}}%
\pgfpathcurveto{\pgfqpoint{8.813350in}{-0.563415in}}{\pgfqpoint{8.804517in}{-0.559756in}}{\pgfqpoint{8.795309in}{-0.559756in}}%
\pgfpathcurveto{\pgfqpoint{8.786100in}{-0.559756in}}{\pgfqpoint{8.777268in}{-0.563415in}}{\pgfqpoint{8.770756in}{-0.569926in}}%
\pgfpathcurveto{\pgfqpoint{8.764245in}{-0.576438in}}{\pgfqpoint{8.760586in}{-0.585270in}}{\pgfqpoint{8.760586in}{-0.594479in}}%
\pgfpathcurveto{\pgfqpoint{8.760586in}{-0.603687in}}{\pgfqpoint{8.764245in}{-0.612520in}}{\pgfqpoint{8.770756in}{-0.619031in}}%
\pgfpathcurveto{\pgfqpoint{8.777268in}{-0.625542in}}{\pgfqpoint{8.786100in}{-0.629201in}}{\pgfqpoint{8.795309in}{-0.629201in}}%
\pgfpathlineto{\pgfqpoint{8.795309in}{-0.629201in}}%
\pgfpathclose%
\pgfusepath{stroke,fill}%
\end{pgfscope}%
\begin{pgfscope}%
\pgfpathrectangle{\pgfqpoint{0.050000in}{0.050000in}}{\pgfqpoint{2.419000in}{2.419000in}}%
\pgfusepath{clip}%
\pgfsetbuttcap%
\pgfsetroundjoin%
\definecolor{currentfill}{rgb}{0.200000,0.133333,0.533333}%
\pgfsetfillcolor{currentfill}%
\pgfsetfillopacity{0.538948}%
\pgfsetlinewidth{1.003750pt}%
\definecolor{currentstroke}{rgb}{0.200000,0.133333,0.533333}%
\pgfsetstrokecolor{currentstroke}%
\pgfsetstrokeopacity{0.538948}%
\pgfsetdash{}{0pt}%
\pgfpathmoveto{\pgfqpoint{-5.680103in}{-0.834995in}}%
\pgfpathcurveto{\pgfqpoint{-5.670894in}{-0.834995in}}{\pgfqpoint{-5.662062in}{-0.831337in}}{\pgfqpoint{-5.655550in}{-0.824825in}}%
\pgfpathcurveto{\pgfqpoint{-5.649039in}{-0.818314in}}{\pgfqpoint{-5.645381in}{-0.809481in}}{\pgfqpoint{-5.645381in}{-0.800273in}}%
\pgfpathcurveto{\pgfqpoint{-5.645381in}{-0.791065in}}{\pgfqpoint{-5.649039in}{-0.782232in}}{\pgfqpoint{-5.655550in}{-0.775721in}}%
\pgfpathcurveto{\pgfqpoint{-5.662062in}{-0.769209in}}{\pgfqpoint{-5.670894in}{-0.765551in}}{\pgfqpoint{-5.680103in}{-0.765551in}}%
\pgfpathcurveto{\pgfqpoint{-5.689311in}{-0.765551in}}{\pgfqpoint{-5.698144in}{-0.769209in}}{\pgfqpoint{-5.704655in}{-0.775721in}}%
\pgfpathcurveto{\pgfqpoint{-5.711166in}{-0.782232in}}{\pgfqpoint{-5.714825in}{-0.791065in}}{\pgfqpoint{-5.714825in}{-0.800273in}}%
\pgfpathcurveto{\pgfqpoint{-5.714825in}{-0.809481in}}{\pgfqpoint{-5.711166in}{-0.818314in}}{\pgfqpoint{-5.704655in}{-0.824825in}}%
\pgfpathcurveto{\pgfqpoint{-5.698144in}{-0.831337in}}{\pgfqpoint{-5.689311in}{-0.834995in}}{\pgfqpoint{-5.680103in}{-0.834995in}}%
\pgfpathlineto{\pgfqpoint{-5.680103in}{-0.834995in}}%
\pgfpathclose%
\pgfusepath{stroke,fill}%
\end{pgfscope}%
\begin{pgfscope}%
\pgfpathrectangle{\pgfqpoint{0.050000in}{0.050000in}}{\pgfqpoint{2.419000in}{2.419000in}}%
\pgfusepath{clip}%
\pgfsetbuttcap%
\pgfsetroundjoin%
\definecolor{currentfill}{rgb}{0.200000,0.133333,0.533333}%
\pgfsetfillcolor{currentfill}%
\pgfsetfillopacity{0.538948}%
\pgfsetlinewidth{1.003750pt}%
\definecolor{currentstroke}{rgb}{0.200000,0.133333,0.533333}%
\pgfsetstrokecolor{currentstroke}%
\pgfsetstrokeopacity{0.538948}%
\pgfsetdash{}{0pt}%
\pgfpathmoveto{\pgfqpoint{0.403659in}{-0.834995in}}%
\pgfpathcurveto{\pgfqpoint{0.412868in}{-0.834995in}}{\pgfqpoint{0.421700in}{-0.831337in}}{\pgfqpoint{0.428211in}{-0.824825in}}%
\pgfpathcurveto{\pgfqpoint{0.434723in}{-0.818314in}}{\pgfqpoint{0.438381in}{-0.809481in}}{\pgfqpoint{0.438381in}{-0.800273in}}%
\pgfpathcurveto{\pgfqpoint{0.438381in}{-0.791065in}}{\pgfqpoint{0.434723in}{-0.782232in}}{\pgfqpoint{0.428211in}{-0.775721in}}%
\pgfpathcurveto{\pgfqpoint{0.421700in}{-0.769209in}}{\pgfqpoint{0.412868in}{-0.765551in}}{\pgfqpoint{0.403659in}{-0.765551in}}%
\pgfpathcurveto{\pgfqpoint{0.394451in}{-0.765551in}}{\pgfqpoint{0.385618in}{-0.769209in}}{\pgfqpoint{0.379107in}{-0.775721in}}%
\pgfpathcurveto{\pgfqpoint{0.372595in}{-0.782232in}}{\pgfqpoint{0.368937in}{-0.791065in}}{\pgfqpoint{0.368937in}{-0.800273in}}%
\pgfpathcurveto{\pgfqpoint{0.368937in}{-0.809481in}}{\pgfqpoint{0.372595in}{-0.818314in}}{\pgfqpoint{0.379107in}{-0.824825in}}%
\pgfpathcurveto{\pgfqpoint{0.385618in}{-0.831337in}}{\pgfqpoint{0.394451in}{-0.834995in}}{\pgfqpoint{0.403659in}{-0.834995in}}%
\pgfpathlineto{\pgfqpoint{0.403659in}{-0.834995in}}%
\pgfpathclose%
\pgfusepath{stroke,fill}%
\end{pgfscope}%
\begin{pgfscope}%
\pgfpathrectangle{\pgfqpoint{0.050000in}{0.050000in}}{\pgfqpoint{2.419000in}{2.419000in}}%
\pgfusepath{clip}%
\pgfsetbuttcap%
\pgfsetroundjoin%
\definecolor{currentfill}{rgb}{0.200000,0.133333,0.533333}%
\pgfsetfillcolor{currentfill}%
\pgfsetfillopacity{0.538948}%
\pgfsetlinewidth{1.003750pt}%
\definecolor{currentstroke}{rgb}{0.200000,0.133333,0.533333}%
\pgfsetstrokecolor{currentstroke}%
\pgfsetstrokeopacity{0.538948}%
\pgfsetdash{}{0pt}%
\pgfpathmoveto{\pgfqpoint{6.487421in}{-0.834995in}}%
\pgfpathcurveto{\pgfqpoint{6.496629in}{-0.834995in}}{\pgfqpoint{6.505462in}{-0.831337in}}{\pgfqpoint{6.511973in}{-0.824825in}}%
\pgfpathcurveto{\pgfqpoint{6.518485in}{-0.818314in}}{\pgfqpoint{6.522143in}{-0.809481in}}{\pgfqpoint{6.522143in}{-0.800273in}}%
\pgfpathcurveto{\pgfqpoint{6.522143in}{-0.791065in}}{\pgfqpoint{6.518485in}{-0.782232in}}{\pgfqpoint{6.511973in}{-0.775721in}}%
\pgfpathcurveto{\pgfqpoint{6.505462in}{-0.769209in}}{\pgfqpoint{6.496629in}{-0.765551in}}{\pgfqpoint{6.487421in}{-0.765551in}}%
\pgfpathcurveto{\pgfqpoint{6.478213in}{-0.765551in}}{\pgfqpoint{6.469380in}{-0.769209in}}{\pgfqpoint{6.462869in}{-0.775721in}}%
\pgfpathcurveto{\pgfqpoint{6.456357in}{-0.782232in}}{\pgfqpoint{6.452699in}{-0.791065in}}{\pgfqpoint{6.452699in}{-0.800273in}}%
\pgfpathcurveto{\pgfqpoint{6.452699in}{-0.809481in}}{\pgfqpoint{6.456357in}{-0.818314in}}{\pgfqpoint{6.462869in}{-0.824825in}}%
\pgfpathcurveto{\pgfqpoint{6.469380in}{-0.831337in}}{\pgfqpoint{6.478213in}{-0.834995in}}{\pgfqpoint{6.487421in}{-0.834995in}}%
\pgfpathlineto{\pgfqpoint{6.487421in}{-0.834995in}}%
\pgfpathclose%
\pgfusepath{stroke,fill}%
\end{pgfscope}%
\begin{pgfscope}%
\pgfpathrectangle{\pgfqpoint{0.050000in}{0.050000in}}{\pgfqpoint{2.419000in}{2.419000in}}%
\pgfusepath{clip}%
\pgfsetbuttcap%
\pgfsetroundjoin%
\definecolor{currentfill}{rgb}{0.200000,0.133333,0.533333}%
\pgfsetfillcolor{currentfill}%
\pgfsetfillopacity{0.549161}%
\pgfsetlinewidth{1.003750pt}%
\definecolor{currentstroke}{rgb}{0.200000,0.133333,0.533333}%
\pgfsetstrokecolor{currentstroke}%
\pgfsetstrokeopacity{0.549161}%
\pgfsetdash{}{0pt}%
\pgfpathmoveto{\pgfqpoint{-2.142785in}{-1.049966in}}%
\pgfpathcurveto{\pgfqpoint{-2.133576in}{-1.049966in}}{\pgfqpoint{-2.124744in}{-1.046308in}}{\pgfqpoint{-2.118233in}{-1.039796in}}%
\pgfpathcurveto{\pgfqpoint{-2.111721in}{-1.033285in}}{\pgfqpoint{-2.108063in}{-1.024452in}}{\pgfqpoint{-2.108063in}{-1.015244in}}%
\pgfpathcurveto{\pgfqpoint{-2.108063in}{-1.006036in}}{\pgfqpoint{-2.111721in}{-0.997203in}}{\pgfqpoint{-2.118233in}{-0.990692in}}%
\pgfpathcurveto{\pgfqpoint{-2.124744in}{-0.984180in}}{\pgfqpoint{-2.133576in}{-0.980522in}}{\pgfqpoint{-2.142785in}{-0.980522in}}%
\pgfpathcurveto{\pgfqpoint{-2.151993in}{-0.980522in}}{\pgfqpoint{-2.160826in}{-0.984180in}}{\pgfqpoint{-2.167337in}{-0.990692in}}%
\pgfpathcurveto{\pgfqpoint{-2.173849in}{-0.997203in}}{\pgfqpoint{-2.177507in}{-1.006036in}}{\pgfqpoint{-2.177507in}{-1.015244in}}%
\pgfpathcurveto{\pgfqpoint{-2.177507in}{-1.024452in}}{\pgfqpoint{-2.173849in}{-1.033285in}}{\pgfqpoint{-2.167337in}{-1.039796in}}%
\pgfpathcurveto{\pgfqpoint{-2.160826in}{-1.046308in}}{\pgfqpoint{-2.151993in}{-1.049966in}}{\pgfqpoint{-2.142785in}{-1.049966in}}%
\pgfpathlineto{\pgfqpoint{-2.142785in}{-1.049966in}}%
\pgfpathclose%
\pgfusepath{stroke,fill}%
\end{pgfscope}%
\begin{pgfscope}%
\pgfpathrectangle{\pgfqpoint{0.050000in}{0.050000in}}{\pgfqpoint{2.419000in}{2.419000in}}%
\pgfusepath{clip}%
\pgfsetbuttcap%
\pgfsetroundjoin%
\definecolor{currentfill}{rgb}{0.200000,0.133333,0.533333}%
\pgfsetfillcolor{currentfill}%
\pgfsetfillopacity{0.549161}%
\pgfsetlinewidth{1.003750pt}%
\definecolor{currentstroke}{rgb}{0.200000,0.133333,0.533333}%
\pgfsetstrokecolor{currentstroke}%
\pgfsetstrokeopacity{0.549161}%
\pgfsetdash{}{0pt}%
\pgfpathmoveto{\pgfqpoint{4.076620in}{-1.049966in}}%
\pgfpathcurveto{\pgfqpoint{4.085829in}{-1.049966in}}{\pgfqpoint{4.094661in}{-1.046308in}}{\pgfqpoint{4.101173in}{-1.039796in}}%
\pgfpathcurveto{\pgfqpoint{4.107684in}{-1.033285in}}{\pgfqpoint{4.111343in}{-1.024452in}}{\pgfqpoint{4.111343in}{-1.015244in}}%
\pgfpathcurveto{\pgfqpoint{4.111343in}{-1.006036in}}{\pgfqpoint{4.107684in}{-0.997203in}}{\pgfqpoint{4.101173in}{-0.990692in}}%
\pgfpathcurveto{\pgfqpoint{4.094661in}{-0.984180in}}{\pgfqpoint{4.085829in}{-0.980522in}}{\pgfqpoint{4.076620in}{-0.980522in}}%
\pgfpathcurveto{\pgfqpoint{4.067412in}{-0.980522in}}{\pgfqpoint{4.058579in}{-0.984180in}}{\pgfqpoint{4.052068in}{-0.990692in}}%
\pgfpathcurveto{\pgfqpoint{4.045557in}{-0.997203in}}{\pgfqpoint{4.041898in}{-1.006036in}}{\pgfqpoint{4.041898in}{-1.015244in}}%
\pgfpathcurveto{\pgfqpoint{4.041898in}{-1.024452in}}{\pgfqpoint{4.045557in}{-1.033285in}}{\pgfqpoint{4.052068in}{-1.039796in}}%
\pgfpathcurveto{\pgfqpoint{4.058579in}{-1.046308in}}{\pgfqpoint{4.067412in}{-1.049966in}}{\pgfqpoint{4.076620in}{-1.049966in}}%
\pgfpathlineto{\pgfqpoint{4.076620in}{-1.049966in}}%
\pgfpathclose%
\pgfusepath{stroke,fill}%
\end{pgfscope}%
\begin{pgfscope}%
\pgfpathrectangle{\pgfqpoint{0.050000in}{0.050000in}}{\pgfqpoint{2.419000in}{2.419000in}}%
\pgfusepath{clip}%
\pgfsetbuttcap%
\pgfsetroundjoin%
\definecolor{currentfill}{rgb}{0.200000,0.133333,0.533333}%
\pgfsetfillcolor{currentfill}%
\pgfsetfillopacity{0.549161}%
\pgfsetlinewidth{1.003750pt}%
\definecolor{currentstroke}{rgb}{0.200000,0.133333,0.533333}%
\pgfsetstrokecolor{currentstroke}%
\pgfsetstrokeopacity{0.549161}%
\pgfsetdash{}{0pt}%
\pgfpathmoveto{\pgfqpoint{10.296025in}{-1.049966in}}%
\pgfpathcurveto{\pgfqpoint{10.305234in}{-1.049966in}}{\pgfqpoint{10.314066in}{-1.046308in}}{\pgfqpoint{10.320578in}{-1.039796in}}%
\pgfpathcurveto{\pgfqpoint{10.327089in}{-1.033285in}}{\pgfqpoint{10.330748in}{-1.024452in}}{\pgfqpoint{10.330748in}{-1.015244in}}%
\pgfpathcurveto{\pgfqpoint{10.330748in}{-1.006036in}}{\pgfqpoint{10.327089in}{-0.997203in}}{\pgfqpoint{10.320578in}{-0.990692in}}%
\pgfpathcurveto{\pgfqpoint{10.314066in}{-0.984180in}}{\pgfqpoint{10.305234in}{-0.980522in}}{\pgfqpoint{10.296025in}{-0.980522in}}%
\pgfpathcurveto{\pgfqpoint{10.286817in}{-0.980522in}}{\pgfqpoint{10.277984in}{-0.984180in}}{\pgfqpoint{10.271473in}{-0.990692in}}%
\pgfpathcurveto{\pgfqpoint{10.264962in}{-0.997203in}}{\pgfqpoint{10.261303in}{-1.006036in}}{\pgfqpoint{10.261303in}{-1.015244in}}%
\pgfpathcurveto{\pgfqpoint{10.261303in}{-1.024452in}}{\pgfqpoint{10.264962in}{-1.033285in}}{\pgfqpoint{10.271473in}{-1.039796in}}%
\pgfpathcurveto{\pgfqpoint{10.277984in}{-1.046308in}}{\pgfqpoint{10.286817in}{-1.049966in}}{\pgfqpoint{10.296025in}{-1.049966in}}%
\pgfpathlineto{\pgfqpoint{10.296025in}{-1.049966in}}%
\pgfpathclose%
\pgfusepath{stroke,fill}%
\end{pgfscope}%
\begin{pgfscope}%
\pgfpathrectangle{\pgfqpoint{0.050000in}{0.050000in}}{\pgfqpoint{2.419000in}{2.419000in}}%
\pgfusepath{clip}%
\pgfsetbuttcap%
\pgfsetroundjoin%
\definecolor{currentfill}{rgb}{0.200000,0.133333,0.533333}%
\pgfsetfillcolor{currentfill}%
\pgfsetfillopacity{0.559840}%
\pgfsetlinewidth{1.003750pt}%
\definecolor{currentstroke}{rgb}{0.200000,0.133333,0.533333}%
\pgfsetstrokecolor{currentstroke}%
\pgfsetstrokeopacity{0.559840}%
\pgfsetdash{}{0pt}%
\pgfpathmoveto{\pgfqpoint{-4.805369in}{-1.274742in}}%
\pgfpathcurveto{\pgfqpoint{-4.796161in}{-1.274742in}}{\pgfqpoint{-4.787328in}{-1.271083in}}{\pgfqpoint{-4.780817in}{-1.264572in}}%
\pgfpathcurveto{\pgfqpoint{-4.774305in}{-1.258061in}}{\pgfqpoint{-4.770647in}{-1.249228in}}{\pgfqpoint{-4.770647in}{-1.240020in}}%
\pgfpathcurveto{\pgfqpoint{-4.770647in}{-1.230811in}}{\pgfqpoint{-4.774305in}{-1.221979in}}{\pgfqpoint{-4.780817in}{-1.215467in}}%
\pgfpathcurveto{\pgfqpoint{-4.787328in}{-1.208956in}}{\pgfqpoint{-4.796161in}{-1.205297in}}{\pgfqpoint{-4.805369in}{-1.205297in}}%
\pgfpathcurveto{\pgfqpoint{-4.814578in}{-1.205297in}}{\pgfqpoint{-4.823410in}{-1.208956in}}{\pgfqpoint{-4.829921in}{-1.215467in}}%
\pgfpathcurveto{\pgfqpoint{-4.836433in}{-1.221979in}}{\pgfqpoint{-4.840091in}{-1.230811in}}{\pgfqpoint{-4.840091in}{-1.240020in}}%
\pgfpathcurveto{\pgfqpoint{-4.840091in}{-1.249228in}}{\pgfqpoint{-4.836433in}{-1.258061in}}{\pgfqpoint{-4.829921in}{-1.264572in}}%
\pgfpathcurveto{\pgfqpoint{-4.823410in}{-1.271083in}}{\pgfqpoint{-4.814578in}{-1.274742in}}{\pgfqpoint{-4.805369in}{-1.274742in}}%
\pgfpathlineto{\pgfqpoint{-4.805369in}{-1.274742in}}%
\pgfpathclose%
\pgfusepath{stroke,fill}%
\end{pgfscope}%
\begin{pgfscope}%
\pgfpathrectangle{\pgfqpoint{0.050000in}{0.050000in}}{\pgfqpoint{2.419000in}{2.419000in}}%
\pgfusepath{clip}%
\pgfsetbuttcap%
\pgfsetroundjoin%
\definecolor{currentfill}{rgb}{0.200000,0.133333,0.533333}%
\pgfsetfillcolor{currentfill}%
\pgfsetfillopacity{0.559840}%
\pgfsetlinewidth{1.003750pt}%
\definecolor{currentstroke}{rgb}{0.200000,0.133333,0.533333}%
\pgfsetstrokecolor{currentstroke}%
\pgfsetstrokeopacity{0.559840}%
\pgfsetdash{}{0pt}%
\pgfpathmoveto{\pgfqpoint{1.555866in}{-1.274742in}}%
\pgfpathcurveto{\pgfqpoint{1.565074in}{-1.274742in}}{\pgfqpoint{1.573907in}{-1.271083in}}{\pgfqpoint{1.580418in}{-1.264572in}}%
\pgfpathcurveto{\pgfqpoint{1.586930in}{-1.258061in}}{\pgfqpoint{1.590588in}{-1.249228in}}{\pgfqpoint{1.590588in}{-1.240020in}}%
\pgfpathcurveto{\pgfqpoint{1.590588in}{-1.230811in}}{\pgfqpoint{1.586930in}{-1.221979in}}{\pgfqpoint{1.580418in}{-1.215467in}}%
\pgfpathcurveto{\pgfqpoint{1.573907in}{-1.208956in}}{\pgfqpoint{1.565074in}{-1.205297in}}{\pgfqpoint{1.555866in}{-1.205297in}}%
\pgfpathcurveto{\pgfqpoint{1.546657in}{-1.205297in}}{\pgfqpoint{1.537825in}{-1.208956in}}{\pgfqpoint{1.531314in}{-1.215467in}}%
\pgfpathcurveto{\pgfqpoint{1.524802in}{-1.221979in}}{\pgfqpoint{1.521144in}{-1.230811in}}{\pgfqpoint{1.521144in}{-1.240020in}}%
\pgfpathcurveto{\pgfqpoint{1.521144in}{-1.249228in}}{\pgfqpoint{1.524802in}{-1.258061in}}{\pgfqpoint{1.531314in}{-1.264572in}}%
\pgfpathcurveto{\pgfqpoint{1.537825in}{-1.271083in}}{\pgfqpoint{1.546657in}{-1.274742in}}{\pgfqpoint{1.555866in}{-1.274742in}}%
\pgfpathlineto{\pgfqpoint{1.555866in}{-1.274742in}}%
\pgfpathclose%
\pgfusepath{stroke,fill}%
\end{pgfscope}%
\begin{pgfscope}%
\pgfpathrectangle{\pgfqpoint{0.050000in}{0.050000in}}{\pgfqpoint{2.419000in}{2.419000in}}%
\pgfusepath{clip}%
\pgfsetbuttcap%
\pgfsetroundjoin%
\definecolor{currentfill}{rgb}{0.200000,0.133333,0.533333}%
\pgfsetfillcolor{currentfill}%
\pgfsetfillopacity{0.559840}%
\pgfsetlinewidth{1.003750pt}%
\definecolor{currentstroke}{rgb}{0.200000,0.133333,0.533333}%
\pgfsetstrokecolor{currentstroke}%
\pgfsetstrokeopacity{0.559840}%
\pgfsetdash{}{0pt}%
\pgfpathmoveto{\pgfqpoint{7.917101in}{-1.274742in}}%
\pgfpathcurveto{\pgfqpoint{7.926309in}{-1.274742in}}{\pgfqpoint{7.935142in}{-1.271083in}}{\pgfqpoint{7.941653in}{-1.264572in}}%
\pgfpathcurveto{\pgfqpoint{7.948164in}{-1.258061in}}{\pgfqpoint{7.951823in}{-1.249228in}}{\pgfqpoint{7.951823in}{-1.240020in}}%
\pgfpathcurveto{\pgfqpoint{7.951823in}{-1.230811in}}{\pgfqpoint{7.948164in}{-1.221979in}}{\pgfqpoint{7.941653in}{-1.215467in}}%
\pgfpathcurveto{\pgfqpoint{7.935142in}{-1.208956in}}{\pgfqpoint{7.926309in}{-1.205297in}}{\pgfqpoint{7.917101in}{-1.205297in}}%
\pgfpathcurveto{\pgfqpoint{7.907892in}{-1.205297in}}{\pgfqpoint{7.899060in}{-1.208956in}}{\pgfqpoint{7.892548in}{-1.215467in}}%
\pgfpathcurveto{\pgfqpoint{7.886037in}{-1.221979in}}{\pgfqpoint{7.882379in}{-1.230811in}}{\pgfqpoint{7.882379in}{-1.240020in}}%
\pgfpathcurveto{\pgfqpoint{7.882379in}{-1.249228in}}{\pgfqpoint{7.886037in}{-1.258061in}}{\pgfqpoint{7.892548in}{-1.264572in}}%
\pgfpathcurveto{\pgfqpoint{7.899060in}{-1.271083in}}{\pgfqpoint{7.907892in}{-1.274742in}}{\pgfqpoint{7.917101in}{-1.274742in}}%
\pgfpathlineto{\pgfqpoint{7.917101in}{-1.274742in}}%
\pgfpathclose%
\pgfusepath{stroke,fill}%
\end{pgfscope}%
\begin{pgfscope}%
\pgfpathrectangle{\pgfqpoint{0.050000in}{0.050000in}}{\pgfqpoint{2.419000in}{2.419000in}}%
\pgfusepath{clip}%
\pgfsetbuttcap%
\pgfsetroundjoin%
\definecolor{currentfill}{rgb}{0.200000,0.133333,0.533333}%
\pgfsetfillcolor{currentfill}%
\pgfsetfillopacity{0.571017}%
\pgfsetlinewidth{1.003750pt}%
\definecolor{currentstroke}{rgb}{0.200000,0.133333,0.533333}%
\pgfsetstrokecolor{currentstroke}%
\pgfsetstrokeopacity{0.571017}%
\pgfsetdash{}{0pt}%
\pgfpathmoveto{\pgfqpoint{-1.082540in}{-1.510008in}}%
\pgfpathcurveto{\pgfqpoint{-1.073332in}{-1.510008in}}{\pgfqpoint{-1.064499in}{-1.506350in}}{\pgfqpoint{-1.057988in}{-1.499838in}}%
\pgfpathcurveto{\pgfqpoint{-1.051476in}{-1.493327in}}{\pgfqpoint{-1.047818in}{-1.484495in}}{\pgfqpoint{-1.047818in}{-1.475286in}}%
\pgfpathcurveto{\pgfqpoint{-1.047818in}{-1.466078in}}{\pgfqpoint{-1.051476in}{-1.457245in}}{\pgfqpoint{-1.057988in}{-1.450734in}}%
\pgfpathcurveto{\pgfqpoint{-1.064499in}{-1.444222in}}{\pgfqpoint{-1.073332in}{-1.440564in}}{\pgfqpoint{-1.082540in}{-1.440564in}}%
\pgfpathcurveto{\pgfqpoint{-1.091749in}{-1.440564in}}{\pgfqpoint{-1.100581in}{-1.444222in}}{\pgfqpoint{-1.107092in}{-1.450734in}}%
\pgfpathcurveto{\pgfqpoint{-1.113604in}{-1.457245in}}{\pgfqpoint{-1.117262in}{-1.466078in}}{\pgfqpoint{-1.117262in}{-1.475286in}}%
\pgfpathcurveto{\pgfqpoint{-1.117262in}{-1.484495in}}{\pgfqpoint{-1.113604in}{-1.493327in}}{\pgfqpoint{-1.107092in}{-1.499838in}}%
\pgfpathcurveto{\pgfqpoint{-1.100581in}{-1.506350in}}{\pgfqpoint{-1.091749in}{-1.510008in}}{\pgfqpoint{-1.082540in}{-1.510008in}}%
\pgfpathlineto{\pgfqpoint{-1.082540in}{-1.510008in}}%
\pgfpathclose%
\pgfusepath{stroke,fill}%
\end{pgfscope}%
\begin{pgfscope}%
\pgfpathrectangle{\pgfqpoint{0.050000in}{0.050000in}}{\pgfqpoint{2.419000in}{2.419000in}}%
\pgfusepath{clip}%
\pgfsetbuttcap%
\pgfsetroundjoin%
\definecolor{currentfill}{rgb}{0.200000,0.133333,0.533333}%
\pgfsetfillcolor{currentfill}%
\pgfsetfillopacity{0.571017}%
\pgfsetlinewidth{1.003750pt}%
\definecolor{currentstroke}{rgb}{0.200000,0.133333,0.533333}%
\pgfsetstrokecolor{currentstroke}%
\pgfsetstrokeopacity{0.571017}%
\pgfsetdash{}{0pt}%
\pgfpathmoveto{\pgfqpoint{-7.592224in}{-1.510008in}}%
\pgfpathcurveto{\pgfqpoint{-7.583016in}{-1.510008in}}{\pgfqpoint{-7.574183in}{-1.506350in}}{\pgfqpoint{-7.567672in}{-1.499838in}}%
\pgfpathcurveto{\pgfqpoint{-7.561161in}{-1.493327in}}{\pgfqpoint{-7.557502in}{-1.484495in}}{\pgfqpoint{-7.557502in}{-1.475286in}}%
\pgfpathcurveto{\pgfqpoint{-7.557502in}{-1.466078in}}{\pgfqpoint{-7.561161in}{-1.457245in}}{\pgfqpoint{-7.567672in}{-1.450734in}}%
\pgfpathcurveto{\pgfqpoint{-7.574183in}{-1.444222in}}{\pgfqpoint{-7.583016in}{-1.440564in}}{\pgfqpoint{-7.592224in}{-1.440564in}}%
\pgfpathcurveto{\pgfqpoint{-7.601433in}{-1.440564in}}{\pgfqpoint{-7.610265in}{-1.444222in}}{\pgfqpoint{-7.616777in}{-1.450734in}}%
\pgfpathcurveto{\pgfqpoint{-7.623288in}{-1.457245in}}{\pgfqpoint{-7.626947in}{-1.466078in}}{\pgfqpoint{-7.626947in}{-1.475286in}}%
\pgfpathcurveto{\pgfqpoint{-7.626947in}{-1.484495in}}{\pgfqpoint{-7.623288in}{-1.493327in}}{\pgfqpoint{-7.616777in}{-1.499838in}}%
\pgfpathcurveto{\pgfqpoint{-7.610265in}{-1.506350in}}{\pgfqpoint{-7.601433in}{-1.510008in}}{\pgfqpoint{-7.592224in}{-1.510008in}}%
\pgfpathlineto{\pgfqpoint{-7.592224in}{-1.510008in}}%
\pgfpathclose%
\pgfusepath{stroke,fill}%
\end{pgfscope}%
\begin{pgfscope}%
\pgfpathrectangle{\pgfqpoint{0.050000in}{0.050000in}}{\pgfqpoint{2.419000in}{2.419000in}}%
\pgfusepath{clip}%
\pgfsetbuttcap%
\pgfsetroundjoin%
\definecolor{currentfill}{rgb}{0.200000,0.133333,0.533333}%
\pgfsetfillcolor{currentfill}%
\pgfsetfillopacity{0.571017}%
\pgfsetlinewidth{1.003750pt}%
\definecolor{currentstroke}{rgb}{0.200000,0.133333,0.533333}%
\pgfsetstrokecolor{currentstroke}%
\pgfsetstrokeopacity{0.571017}%
\pgfsetdash{}{0pt}%
\pgfpathmoveto{\pgfqpoint{5.427144in}{-1.510008in}}%
\pgfpathcurveto{\pgfqpoint{5.436353in}{-1.510008in}}{\pgfqpoint{5.445185in}{-1.506350in}}{\pgfqpoint{5.451697in}{-1.499838in}}%
\pgfpathcurveto{\pgfqpoint{5.458208in}{-1.493327in}}{\pgfqpoint{5.461866in}{-1.484495in}}{\pgfqpoint{5.461866in}{-1.475286in}}%
\pgfpathcurveto{\pgfqpoint{5.461866in}{-1.466078in}}{\pgfqpoint{5.458208in}{-1.457245in}}{\pgfqpoint{5.451697in}{-1.450734in}}%
\pgfpathcurveto{\pgfqpoint{5.445185in}{-1.444222in}}{\pgfqpoint{5.436353in}{-1.440564in}}{\pgfqpoint{5.427144in}{-1.440564in}}%
\pgfpathcurveto{\pgfqpoint{5.417936in}{-1.440564in}}{\pgfqpoint{5.409103in}{-1.444222in}}{\pgfqpoint{5.402592in}{-1.450734in}}%
\pgfpathcurveto{\pgfqpoint{5.396081in}{-1.457245in}}{\pgfqpoint{5.392422in}{-1.466078in}}{\pgfqpoint{5.392422in}{-1.475286in}}%
\pgfpathcurveto{\pgfqpoint{5.392422in}{-1.484495in}}{\pgfqpoint{5.396081in}{-1.493327in}}{\pgfqpoint{5.402592in}{-1.499838in}}%
\pgfpathcurveto{\pgfqpoint{5.409103in}{-1.506350in}}{\pgfqpoint{5.417936in}{-1.510008in}}{\pgfqpoint{5.427144in}{-1.510008in}}%
\pgfpathlineto{\pgfqpoint{5.427144in}{-1.510008in}}%
\pgfpathclose%
\pgfusepath{stroke,fill}%
\end{pgfscope}%
\begin{pgfscope}%
\pgfpathrectangle{\pgfqpoint{0.050000in}{0.050000in}}{\pgfqpoint{2.419000in}{2.419000in}}%
\pgfusepath{clip}%
\pgfsetbuttcap%
\pgfsetroundjoin%
\definecolor{currentfill}{rgb}{0.200000,0.133333,0.533333}%
\pgfsetfillcolor{currentfill}%
\pgfsetfillopacity{0.582729}%
\pgfsetlinewidth{1.003750pt}%
\definecolor{currentstroke}{rgb}{0.200000,0.133333,0.533333}%
\pgfsetstrokecolor{currentstroke}%
\pgfsetstrokeopacity{0.582729}%
\pgfsetdash{}{0pt}%
\pgfpathmoveto{\pgfqpoint{2.818197in}{-1.756518in}}%
\pgfpathcurveto{\pgfqpoint{2.827405in}{-1.756518in}}{\pgfqpoint{2.836238in}{-1.752859in}}{\pgfqpoint{2.842749in}{-1.746348in}}%
\pgfpathcurveto{\pgfqpoint{2.849260in}{-1.739837in}}{\pgfqpoint{2.852919in}{-1.731004in}}{\pgfqpoint{2.852919in}{-1.721796in}}%
\pgfpathcurveto{\pgfqpoint{2.852919in}{-1.712587in}}{\pgfqpoint{2.849260in}{-1.703755in}}{\pgfqpoint{2.842749in}{-1.697243in}}%
\pgfpathcurveto{\pgfqpoint{2.836238in}{-1.690732in}}{\pgfqpoint{2.827405in}{-1.687073in}}{\pgfqpoint{2.818197in}{-1.687073in}}%
\pgfpathcurveto{\pgfqpoint{2.808988in}{-1.687073in}}{\pgfqpoint{2.800156in}{-1.690732in}}{\pgfqpoint{2.793644in}{-1.697243in}}%
\pgfpathcurveto{\pgfqpoint{2.787133in}{-1.703755in}}{\pgfqpoint{2.783475in}{-1.712587in}}{\pgfqpoint{2.783475in}{-1.721796in}}%
\pgfpathcurveto{\pgfqpoint{2.783475in}{-1.731004in}}{\pgfqpoint{2.787133in}{-1.739837in}}{\pgfqpoint{2.793644in}{-1.746348in}}%
\pgfpathcurveto{\pgfqpoint{2.800156in}{-1.752859in}}{\pgfqpoint{2.808988in}{-1.756518in}}{\pgfqpoint{2.818197in}{-1.756518in}}%
\pgfpathlineto{\pgfqpoint{2.818197in}{-1.756518in}}%
\pgfpathclose%
\pgfusepath{stroke,fill}%
\end{pgfscope}%
\begin{pgfscope}%
\pgfpathrectangle{\pgfqpoint{0.050000in}{0.050000in}}{\pgfqpoint{2.419000in}{2.419000in}}%
\pgfusepath{clip}%
\pgfsetbuttcap%
\pgfsetroundjoin%
\definecolor{currentfill}{rgb}{0.200000,0.133333,0.533333}%
\pgfsetfillcolor{currentfill}%
\pgfsetfillopacity{0.582729}%
\pgfsetlinewidth{1.003750pt}%
\definecolor{currentstroke}{rgb}{0.200000,0.133333,0.533333}%
\pgfsetstrokecolor{currentstroke}%
\pgfsetstrokeopacity{0.582729}%
\pgfsetdash{}{0pt}%
\pgfpathmoveto{\pgfqpoint{-3.847031in}{-1.756518in}}%
\pgfpathcurveto{\pgfqpoint{-3.837823in}{-1.756518in}}{\pgfqpoint{-3.828990in}{-1.752859in}}{\pgfqpoint{-3.822479in}{-1.746348in}}%
\pgfpathcurveto{\pgfqpoint{-3.815967in}{-1.739837in}}{\pgfqpoint{-3.812309in}{-1.731004in}}{\pgfqpoint{-3.812309in}{-1.721796in}}%
\pgfpathcurveto{\pgfqpoint{-3.812309in}{-1.712587in}}{\pgfqpoint{-3.815967in}{-1.703755in}}{\pgfqpoint{-3.822479in}{-1.697243in}}%
\pgfpathcurveto{\pgfqpoint{-3.828990in}{-1.690732in}}{\pgfqpoint{-3.837823in}{-1.687073in}}{\pgfqpoint{-3.847031in}{-1.687073in}}%
\pgfpathcurveto{\pgfqpoint{-3.856240in}{-1.687073in}}{\pgfqpoint{-3.865072in}{-1.690732in}}{\pgfqpoint{-3.871583in}{-1.697243in}}%
\pgfpathcurveto{\pgfqpoint{-3.878095in}{-1.703755in}}{\pgfqpoint{-3.881753in}{-1.712587in}}{\pgfqpoint{-3.881753in}{-1.721796in}}%
\pgfpathcurveto{\pgfqpoint{-3.881753in}{-1.731004in}}{\pgfqpoint{-3.878095in}{-1.739837in}}{\pgfqpoint{-3.871583in}{-1.746348in}}%
\pgfpathcurveto{\pgfqpoint{-3.865072in}{-1.752859in}}{\pgfqpoint{-3.856240in}{-1.756518in}}{\pgfqpoint{-3.847031in}{-1.756518in}}%
\pgfpathlineto{\pgfqpoint{-3.847031in}{-1.756518in}}%
\pgfpathclose%
\pgfusepath{stroke,fill}%
\end{pgfscope}%
\begin{pgfscope}%
\pgfpathrectangle{\pgfqpoint{0.050000in}{0.050000in}}{\pgfqpoint{2.419000in}{2.419000in}}%
\pgfusepath{clip}%
\pgfsetbuttcap%
\pgfsetroundjoin%
\definecolor{currentfill}{rgb}{0.200000,0.133333,0.533333}%
\pgfsetfillcolor{currentfill}%
\pgfsetfillopacity{0.582729}%
\pgfsetlinewidth{1.003750pt}%
\definecolor{currentstroke}{rgb}{0.200000,0.133333,0.533333}%
\pgfsetstrokecolor{currentstroke}%
\pgfsetstrokeopacity{0.582729}%
\pgfsetdash{}{0pt}%
\pgfpathmoveto{\pgfqpoint{9.483425in}{-1.756518in}}%
\pgfpathcurveto{\pgfqpoint{9.492633in}{-1.756518in}}{\pgfqpoint{9.501466in}{-1.752859in}}{\pgfqpoint{9.507977in}{-1.746348in}}%
\pgfpathcurveto{\pgfqpoint{9.514488in}{-1.739837in}}{\pgfqpoint{9.518147in}{-1.731004in}}{\pgfqpoint{9.518147in}{-1.721796in}}%
\pgfpathcurveto{\pgfqpoint{9.518147in}{-1.712587in}}{\pgfqpoint{9.514488in}{-1.703755in}}{\pgfqpoint{9.507977in}{-1.697243in}}%
\pgfpathcurveto{\pgfqpoint{9.501466in}{-1.690732in}}{\pgfqpoint{9.492633in}{-1.687073in}}{\pgfqpoint{9.483425in}{-1.687073in}}%
\pgfpathcurveto{\pgfqpoint{9.474216in}{-1.687073in}}{\pgfqpoint{9.465384in}{-1.690732in}}{\pgfqpoint{9.458872in}{-1.697243in}}%
\pgfpathcurveto{\pgfqpoint{9.452361in}{-1.703755in}}{\pgfqpoint{9.448703in}{-1.712587in}}{\pgfqpoint{9.448703in}{-1.721796in}}%
\pgfpathcurveto{\pgfqpoint{9.448703in}{-1.731004in}}{\pgfqpoint{9.452361in}{-1.739837in}}{\pgfqpoint{9.458872in}{-1.746348in}}%
\pgfpathcurveto{\pgfqpoint{9.465384in}{-1.752859in}}{\pgfqpoint{9.474216in}{-1.756518in}}{\pgfqpoint{9.483425in}{-1.756518in}}%
\pgfpathlineto{\pgfqpoint{9.483425in}{-1.756518in}}%
\pgfpathclose%
\pgfusepath{stroke,fill}%
\end{pgfscope}%
\begin{pgfscope}%
\pgfpathrectangle{\pgfqpoint{0.050000in}{0.050000in}}{\pgfqpoint{2.419000in}{2.419000in}}%
\pgfusepath{clip}%
\pgfsetbuttcap%
\pgfsetroundjoin%
\definecolor{currentfill}{rgb}{0.200000,0.133333,0.533333}%
\pgfsetfillcolor{currentfill}%
\pgfsetfillopacity{0.595014}%
\pgfsetlinewidth{1.003750pt}%
\definecolor{currentstroke}{rgb}{0.200000,0.133333,0.533333}%
\pgfsetstrokecolor{currentstroke}%
\pgfsetstrokeopacity{0.595014}%
\pgfsetdash{}{0pt}%
\pgfpathmoveto{\pgfqpoint{-6.746867in}{-2.015096in}}%
\pgfpathcurveto{\pgfqpoint{-6.737658in}{-2.015096in}}{\pgfqpoint{-6.728826in}{-2.011438in}}{\pgfqpoint{-6.722314in}{-2.004926in}}%
\pgfpathcurveto{\pgfqpoint{-6.715803in}{-1.998415in}}{\pgfqpoint{-6.712144in}{-1.989582in}}{\pgfqpoint{-6.712144in}{-1.980374in}}%
\pgfpathcurveto{\pgfqpoint{-6.712144in}{-1.971166in}}{\pgfqpoint{-6.715803in}{-1.962333in}}{\pgfqpoint{-6.722314in}{-1.955822in}}%
\pgfpathcurveto{\pgfqpoint{-6.728826in}{-1.949310in}}{\pgfqpoint{-6.737658in}{-1.945652in}}{\pgfqpoint{-6.746867in}{-1.945652in}}%
\pgfpathcurveto{\pgfqpoint{-6.756075in}{-1.945652in}}{\pgfqpoint{-6.764908in}{-1.949310in}}{\pgfqpoint{-6.771419in}{-1.955822in}}%
\pgfpathcurveto{\pgfqpoint{-6.777930in}{-1.962333in}}{\pgfqpoint{-6.781589in}{-1.971166in}}{\pgfqpoint{-6.781589in}{-1.980374in}}%
\pgfpathcurveto{\pgfqpoint{-6.781589in}{-1.989582in}}{\pgfqpoint{-6.777930in}{-1.998415in}}{\pgfqpoint{-6.771419in}{-2.004926in}}%
\pgfpathcurveto{\pgfqpoint{-6.764908in}{-2.011438in}}{\pgfqpoint{-6.756075in}{-2.015096in}}{\pgfqpoint{-6.746867in}{-2.015096in}}%
\pgfpathlineto{\pgfqpoint{-6.746867in}{-2.015096in}}%
\pgfpathclose%
\pgfusepath{stroke,fill}%
\end{pgfscope}%
\begin{pgfscope}%
\pgfpathrectangle{\pgfqpoint{0.050000in}{0.050000in}}{\pgfqpoint{2.419000in}{2.419000in}}%
\pgfusepath{clip}%
\pgfsetbuttcap%
\pgfsetroundjoin%
\definecolor{currentfill}{rgb}{0.200000,0.133333,0.533333}%
\pgfsetfillcolor{currentfill}%
\pgfsetfillopacity{0.595014}%
\pgfsetlinewidth{1.003750pt}%
\definecolor{currentstroke}{rgb}{0.200000,0.133333,0.533333}%
\pgfsetstrokecolor{currentstroke}%
\pgfsetstrokeopacity{0.595014}%
\pgfsetdash{}{0pt}%
\pgfpathmoveto{\pgfqpoint{0.081520in}{-2.015096in}}%
\pgfpathcurveto{\pgfqpoint{0.090728in}{-2.015096in}}{\pgfqpoint{0.099561in}{-2.011438in}}{\pgfqpoint{0.106072in}{-2.004926in}}%
\pgfpathcurveto{\pgfqpoint{0.112584in}{-1.998415in}}{\pgfqpoint{0.116242in}{-1.989582in}}{\pgfqpoint{0.116242in}{-1.980374in}}%
\pgfpathcurveto{\pgfqpoint{0.116242in}{-1.971166in}}{\pgfqpoint{0.112584in}{-1.962333in}}{\pgfqpoint{0.106072in}{-1.955822in}}%
\pgfpathcurveto{\pgfqpoint{0.099561in}{-1.949310in}}{\pgfqpoint{0.090728in}{-1.945652in}}{\pgfqpoint{0.081520in}{-1.945652in}}%
\pgfpathcurveto{\pgfqpoint{0.072312in}{-1.945652in}}{\pgfqpoint{0.063479in}{-1.949310in}}{\pgfqpoint{0.056968in}{-1.955822in}}%
\pgfpathcurveto{\pgfqpoint{0.050456in}{-1.962333in}}{\pgfqpoint{0.046798in}{-1.971166in}}{\pgfqpoint{0.046798in}{-1.980374in}}%
\pgfpathcurveto{\pgfqpoint{0.046798in}{-1.989582in}}{\pgfqpoint{0.050456in}{-1.998415in}}{\pgfqpoint{0.056968in}{-2.004926in}}%
\pgfpathcurveto{\pgfqpoint{0.063479in}{-2.011438in}}{\pgfqpoint{0.072312in}{-2.015096in}}{\pgfqpoint{0.081520in}{-2.015096in}}%
\pgfpathlineto{\pgfqpoint{0.081520in}{-2.015096in}}%
\pgfpathclose%
\pgfusepath{stroke,fill}%
\end{pgfscope}%
\begin{pgfscope}%
\pgfpathrectangle{\pgfqpoint{0.050000in}{0.050000in}}{\pgfqpoint{2.419000in}{2.419000in}}%
\pgfusepath{clip}%
\pgfsetbuttcap%
\pgfsetroundjoin%
\definecolor{currentfill}{rgb}{0.200000,0.133333,0.533333}%
\pgfsetfillcolor{currentfill}%
\pgfsetfillopacity{0.595014}%
\pgfsetlinewidth{1.003750pt}%
\definecolor{currentstroke}{rgb}{0.200000,0.133333,0.533333}%
\pgfsetstrokecolor{currentstroke}%
\pgfsetstrokeopacity{0.595014}%
\pgfsetdash{}{0pt}%
\pgfpathmoveto{\pgfqpoint{6.909907in}{-2.015096in}}%
\pgfpathcurveto{\pgfqpoint{6.919115in}{-2.015096in}}{\pgfqpoint{6.927948in}{-2.011438in}}{\pgfqpoint{6.934459in}{-2.004926in}}%
\pgfpathcurveto{\pgfqpoint{6.940970in}{-1.998415in}}{\pgfqpoint{6.944629in}{-1.989582in}}{\pgfqpoint{6.944629in}{-1.980374in}}%
\pgfpathcurveto{\pgfqpoint{6.944629in}{-1.971166in}}{\pgfqpoint{6.940970in}{-1.962333in}}{\pgfqpoint{6.934459in}{-1.955822in}}%
\pgfpathcurveto{\pgfqpoint{6.927948in}{-1.949310in}}{\pgfqpoint{6.919115in}{-1.945652in}}{\pgfqpoint{6.909907in}{-1.945652in}}%
\pgfpathcurveto{\pgfqpoint{6.900698in}{-1.945652in}}{\pgfqpoint{6.891866in}{-1.949310in}}{\pgfqpoint{6.885354in}{-1.955822in}}%
\pgfpathcurveto{\pgfqpoint{6.878843in}{-1.962333in}}{\pgfqpoint{6.875184in}{-1.971166in}}{\pgfqpoint{6.875184in}{-1.980374in}}%
\pgfpathcurveto{\pgfqpoint{6.875184in}{-1.989582in}}{\pgfqpoint{6.878843in}{-1.998415in}}{\pgfqpoint{6.885354in}{-2.004926in}}%
\pgfpathcurveto{\pgfqpoint{6.891866in}{-2.011438in}}{\pgfqpoint{6.900698in}{-2.015096in}}{\pgfqpoint{6.909907in}{-2.015096in}}%
\pgfpathlineto{\pgfqpoint{6.909907in}{-2.015096in}}%
\pgfpathclose%
\pgfusepath{stroke,fill}%
\end{pgfscope}%
\begin{pgfscope}%
\pgfpathrectangle{\pgfqpoint{0.050000in}{0.050000in}}{\pgfqpoint{2.419000in}{2.419000in}}%
\pgfusepath{clip}%
\pgfsetbuttcap%
\pgfsetroundjoin%
\definecolor{currentfill}{rgb}{0.200000,0.133333,0.533333}%
\pgfsetfillcolor{currentfill}%
\pgfsetfillopacity{0.607915}%
\pgfsetlinewidth{1.003750pt}%
\definecolor{currentstroke}{rgb}{0.200000,0.133333,0.533333}%
\pgfsetstrokecolor{currentstroke}%
\pgfsetstrokeopacity{0.607915}%
\pgfsetdash{}{0pt}%
\pgfpathmoveto{\pgfqpoint{4.207232in}{-2.286652in}}%
\pgfpathcurveto{\pgfqpoint{4.216441in}{-2.286652in}}{\pgfqpoint{4.225273in}{-2.282993in}}{\pgfqpoint{4.231784in}{-2.276482in}}%
\pgfpathcurveto{\pgfqpoint{4.238296in}{-2.269970in}}{\pgfqpoint{4.241954in}{-2.261138in}}{\pgfqpoint{4.241954in}{-2.251929in}}%
\pgfpathcurveto{\pgfqpoint{4.241954in}{-2.242721in}}{\pgfqpoint{4.238296in}{-2.233888in}}{\pgfqpoint{4.231784in}{-2.227377in}}%
\pgfpathcurveto{\pgfqpoint{4.225273in}{-2.220866in}}{\pgfqpoint{4.216441in}{-2.217207in}}{\pgfqpoint{4.207232in}{-2.217207in}}%
\pgfpathcurveto{\pgfqpoint{4.198024in}{-2.217207in}}{\pgfqpoint{4.189191in}{-2.220866in}}{\pgfqpoint{4.182680in}{-2.227377in}}%
\pgfpathcurveto{\pgfqpoint{4.176168in}{-2.233888in}}{\pgfqpoint{4.172510in}{-2.242721in}}{\pgfqpoint{4.172510in}{-2.251929in}}%
\pgfpathcurveto{\pgfqpoint{4.172510in}{-2.261138in}}{\pgfqpoint{4.176168in}{-2.269970in}}{\pgfqpoint{4.182680in}{-2.276482in}}%
\pgfpathcurveto{\pgfqpoint{4.189191in}{-2.282993in}}{\pgfqpoint{4.198024in}{-2.286652in}}{\pgfqpoint{4.207232in}{-2.286652in}}%
\pgfpathlineto{\pgfqpoint{4.207232in}{-2.286652in}}%
\pgfpathclose%
\pgfusepath{stroke,fill}%
\end{pgfscope}%
\begin{pgfscope}%
\pgfpathrectangle{\pgfqpoint{0.050000in}{0.050000in}}{\pgfqpoint{2.419000in}{2.419000in}}%
\pgfusepath{clip}%
\pgfsetbuttcap%
\pgfsetroundjoin%
\definecolor{currentfill}{rgb}{0.200000,0.133333,0.533333}%
\pgfsetfillcolor{currentfill}%
\pgfsetfillopacity{0.607915}%
\pgfsetlinewidth{1.003750pt}%
\definecolor{currentstroke}{rgb}{0.200000,0.133333,0.533333}%
\pgfsetstrokecolor{currentstroke}%
\pgfsetstrokeopacity{0.607915}%
\pgfsetdash{}{0pt}%
\pgfpathmoveto{\pgfqpoint{-2.792502in}{-2.286652in}}%
\pgfpathcurveto{\pgfqpoint{-2.783293in}{-2.286652in}}{\pgfqpoint{-2.774461in}{-2.282993in}}{\pgfqpoint{-2.767949in}{-2.276482in}}%
\pgfpathcurveto{\pgfqpoint{-2.761438in}{-2.269970in}}{\pgfqpoint{-2.757779in}{-2.261138in}}{\pgfqpoint{-2.757779in}{-2.251929in}}%
\pgfpathcurveto{\pgfqpoint{-2.757779in}{-2.242721in}}{\pgfqpoint{-2.761438in}{-2.233888in}}{\pgfqpoint{-2.767949in}{-2.227377in}}%
\pgfpathcurveto{\pgfqpoint{-2.774461in}{-2.220866in}}{\pgfqpoint{-2.783293in}{-2.217207in}}{\pgfqpoint{-2.792502in}{-2.217207in}}%
\pgfpathcurveto{\pgfqpoint{-2.801710in}{-2.217207in}}{\pgfqpoint{-2.810543in}{-2.220866in}}{\pgfqpoint{-2.817054in}{-2.227377in}}%
\pgfpathcurveto{\pgfqpoint{-2.823565in}{-2.233888in}}{\pgfqpoint{-2.827224in}{-2.242721in}}{\pgfqpoint{-2.827224in}{-2.251929in}}%
\pgfpathcurveto{\pgfqpoint{-2.827224in}{-2.261138in}}{\pgfqpoint{-2.823565in}{-2.269970in}}{\pgfqpoint{-2.817054in}{-2.276482in}}%
\pgfpathcurveto{\pgfqpoint{-2.810543in}{-2.282993in}}{\pgfqpoint{-2.801710in}{-2.286652in}}{\pgfqpoint{-2.792502in}{-2.286652in}}%
\pgfpathlineto{\pgfqpoint{-2.792502in}{-2.286652in}}%
\pgfpathclose%
\pgfusepath{stroke,fill}%
\end{pgfscope}%
\begin{pgfscope}%
\pgfpathrectangle{\pgfqpoint{0.050000in}{0.050000in}}{\pgfqpoint{2.419000in}{2.419000in}}%
\pgfusepath{clip}%
\pgfsetbuttcap%
\pgfsetroundjoin%
\definecolor{currentfill}{rgb}{0.200000,0.133333,0.533333}%
\pgfsetfillcolor{currentfill}%
\pgfsetfillopacity{0.621480}%
\pgfsetlinewidth{1.003750pt}%
\definecolor{currentstroke}{rgb}{0.200000,0.133333,0.533333}%
\pgfsetstrokecolor{currentstroke}%
\pgfsetstrokeopacity{0.621480}%
\pgfsetdash{}{0pt}%
\pgfpathmoveto{\pgfqpoint{-5.814473in}{-2.572186in}}%
\pgfpathcurveto{\pgfqpoint{-5.805265in}{-2.572186in}}{\pgfqpoint{-5.796432in}{-2.568528in}}{\pgfqpoint{-5.789921in}{-2.562016in}}%
\pgfpathcurveto{\pgfqpoint{-5.783410in}{-2.555505in}}{\pgfqpoint{-5.779751in}{-2.546673in}}{\pgfqpoint{-5.779751in}{-2.537464in}}%
\pgfpathcurveto{\pgfqpoint{-5.779751in}{-2.528256in}}{\pgfqpoint{-5.783410in}{-2.519423in}}{\pgfqpoint{-5.789921in}{-2.512912in}}%
\pgfpathcurveto{\pgfqpoint{-5.796432in}{-2.506400in}}{\pgfqpoint{-5.805265in}{-2.502742in}}{\pgfqpoint{-5.814473in}{-2.502742in}}%
\pgfpathcurveto{\pgfqpoint{-5.823682in}{-2.502742in}}{\pgfqpoint{-5.832514in}{-2.506400in}}{\pgfqpoint{-5.839026in}{-2.512912in}}%
\pgfpathcurveto{\pgfqpoint{-5.845537in}{-2.519423in}}{\pgfqpoint{-5.849196in}{-2.528256in}}{\pgfqpoint{-5.849196in}{-2.537464in}}%
\pgfpathcurveto{\pgfqpoint{-5.849196in}{-2.546673in}}{\pgfqpoint{-5.845537in}{-2.555505in}}{\pgfqpoint{-5.839026in}{-2.562016in}}%
\pgfpathcurveto{\pgfqpoint{-5.832514in}{-2.568528in}}{\pgfqpoint{-5.823682in}{-2.572186in}}{\pgfqpoint{-5.814473in}{-2.572186in}}%
\pgfpathlineto{\pgfqpoint{-5.814473in}{-2.572186in}}%
\pgfpathclose%
\pgfusepath{stroke,fill}%
\end{pgfscope}%
\begin{pgfscope}%
\pgfpathrectangle{\pgfqpoint{0.050000in}{0.050000in}}{\pgfqpoint{2.419000in}{2.419000in}}%
\pgfusepath{clip}%
\pgfsetbuttcap%
\pgfsetroundjoin%
\definecolor{currentfill}{rgb}{0.200000,0.133333,0.533333}%
\pgfsetfillcolor{currentfill}%
\pgfsetfillopacity{0.621480}%
\pgfsetlinewidth{1.003750pt}%
\definecolor{currentstroke}{rgb}{0.200000,0.133333,0.533333}%
\pgfsetstrokecolor{currentstroke}%
\pgfsetstrokeopacity{0.621480}%
\pgfsetdash{}{0pt}%
\pgfpathmoveto{\pgfqpoint{1.365428in}{-2.572186in}}%
\pgfpathcurveto{\pgfqpoint{1.374637in}{-2.572186in}}{\pgfqpoint{1.383469in}{-2.568528in}}{\pgfqpoint{1.389980in}{-2.562016in}}%
\pgfpathcurveto{\pgfqpoint{1.396492in}{-2.555505in}}{\pgfqpoint{1.400150in}{-2.546673in}}{\pgfqpoint{1.400150in}{-2.537464in}}%
\pgfpathcurveto{\pgfqpoint{1.400150in}{-2.528256in}}{\pgfqpoint{1.396492in}{-2.519423in}}{\pgfqpoint{1.389980in}{-2.512912in}}%
\pgfpathcurveto{\pgfqpoint{1.383469in}{-2.506400in}}{\pgfqpoint{1.374637in}{-2.502742in}}{\pgfqpoint{1.365428in}{-2.502742in}}%
\pgfpathcurveto{\pgfqpoint{1.356220in}{-2.502742in}}{\pgfqpoint{1.347387in}{-2.506400in}}{\pgfqpoint{1.340876in}{-2.512912in}}%
\pgfpathcurveto{\pgfqpoint{1.334364in}{-2.519423in}}{\pgfqpoint{1.330706in}{-2.528256in}}{\pgfqpoint{1.330706in}{-2.537464in}}%
\pgfpathcurveto{\pgfqpoint{1.330706in}{-2.546673in}}{\pgfqpoint{1.334364in}{-2.555505in}}{\pgfqpoint{1.340876in}{-2.562016in}}%
\pgfpathcurveto{\pgfqpoint{1.347387in}{-2.568528in}}{\pgfqpoint{1.356220in}{-2.572186in}}{\pgfqpoint{1.365428in}{-2.572186in}}%
\pgfpathlineto{\pgfqpoint{1.365428in}{-2.572186in}}%
\pgfpathclose%
\pgfusepath{stroke,fill}%
\end{pgfscope}%
\begin{pgfscope}%
\pgfpathrectangle{\pgfqpoint{0.050000in}{0.050000in}}{\pgfqpoint{2.419000in}{2.419000in}}%
\pgfusepath{clip}%
\pgfsetbuttcap%
\pgfsetroundjoin%
\definecolor{currentfill}{rgb}{0.200000,0.133333,0.533333}%
\pgfsetfillcolor{currentfill}%
\pgfsetfillopacity{0.621480}%
\pgfsetlinewidth{1.003750pt}%
\definecolor{currentstroke}{rgb}{0.200000,0.133333,0.533333}%
\pgfsetstrokecolor{currentstroke}%
\pgfsetstrokeopacity{0.621480}%
\pgfsetdash{}{0pt}%
\pgfpathmoveto{\pgfqpoint{8.545330in}{-2.572186in}}%
\pgfpathcurveto{\pgfqpoint{8.554538in}{-2.572186in}}{\pgfqpoint{8.563371in}{-2.568528in}}{\pgfqpoint{8.569882in}{-2.562016in}}%
\pgfpathcurveto{\pgfqpoint{8.576393in}{-2.555505in}}{\pgfqpoint{8.580052in}{-2.546673in}}{\pgfqpoint{8.580052in}{-2.537464in}}%
\pgfpathcurveto{\pgfqpoint{8.580052in}{-2.528256in}}{\pgfqpoint{8.576393in}{-2.519423in}}{\pgfqpoint{8.569882in}{-2.512912in}}%
\pgfpathcurveto{\pgfqpoint{8.563371in}{-2.506400in}}{\pgfqpoint{8.554538in}{-2.502742in}}{\pgfqpoint{8.545330in}{-2.502742in}}%
\pgfpathcurveto{\pgfqpoint{8.536121in}{-2.502742in}}{\pgfqpoint{8.527289in}{-2.506400in}}{\pgfqpoint{8.520777in}{-2.512912in}}%
\pgfpathcurveto{\pgfqpoint{8.514266in}{-2.519423in}}{\pgfqpoint{8.510607in}{-2.528256in}}{\pgfqpoint{8.510607in}{-2.537464in}}%
\pgfpathcurveto{\pgfqpoint{8.510607in}{-2.546673in}}{\pgfqpoint{8.514266in}{-2.555505in}}{\pgfqpoint{8.520777in}{-2.562016in}}%
\pgfpathcurveto{\pgfqpoint{8.527289in}{-2.568528in}}{\pgfqpoint{8.536121in}{-2.572186in}}{\pgfqpoint{8.545330in}{-2.572186in}}%
\pgfpathlineto{\pgfqpoint{8.545330in}{-2.572186in}}%
\pgfpathclose%
\pgfusepath{stroke,fill}%
\end{pgfscope}%
\begin{pgfscope}%
\pgfpathrectangle{\pgfqpoint{0.050000in}{0.050000in}}{\pgfqpoint{2.419000in}{2.419000in}}%
\pgfusepath{clip}%
\pgfsetbuttcap%
\pgfsetroundjoin%
\definecolor{currentfill}{rgb}{0.200000,0.133333,0.533333}%
\pgfsetfillcolor{currentfill}%
\pgfsetfillopacity{0.635763}%
\pgfsetlinewidth{1.003750pt}%
\definecolor{currentstroke}{rgb}{0.200000,0.133333,0.533333}%
\pgfsetstrokecolor{currentstroke}%
\pgfsetstrokeopacity{0.635763}%
\pgfsetdash{}{0pt}%
\pgfpathmoveto{\pgfqpoint{5.743057in}{-2.872808in}}%
\pgfpathcurveto{\pgfqpoint{5.752265in}{-2.872808in}}{\pgfqpoint{5.761098in}{-2.869150in}}{\pgfqpoint{5.767609in}{-2.862638in}}%
\pgfpathcurveto{\pgfqpoint{5.774120in}{-2.856127in}}{\pgfqpoint{5.777779in}{-2.847294in}}{\pgfqpoint{5.777779in}{-2.838086in}}%
\pgfpathcurveto{\pgfqpoint{5.777779in}{-2.828878in}}{\pgfqpoint{5.774120in}{-2.820045in}}{\pgfqpoint{5.767609in}{-2.813534in}}%
\pgfpathcurveto{\pgfqpoint{5.761098in}{-2.807022in}}{\pgfqpoint{5.752265in}{-2.803364in}}{\pgfqpoint{5.743057in}{-2.803364in}}%
\pgfpathcurveto{\pgfqpoint{5.733848in}{-2.803364in}}{\pgfqpoint{5.725016in}{-2.807022in}}{\pgfqpoint{5.718504in}{-2.813534in}}%
\pgfpathcurveto{\pgfqpoint{5.711993in}{-2.820045in}}{\pgfqpoint{5.708334in}{-2.828878in}}{\pgfqpoint{5.708334in}{-2.838086in}}%
\pgfpathcurveto{\pgfqpoint{5.708334in}{-2.847294in}}{\pgfqpoint{5.711993in}{-2.856127in}}{\pgfqpoint{5.718504in}{-2.862638in}}%
\pgfpathcurveto{\pgfqpoint{5.725016in}{-2.869150in}}{\pgfqpoint{5.733848in}{-2.872808in}}{\pgfqpoint{5.743057in}{-2.872808in}}%
\pgfpathlineto{\pgfqpoint{5.743057in}{-2.872808in}}%
\pgfpathclose%
\pgfusepath{stroke,fill}%
\end{pgfscope}%
\begin{pgfscope}%
\pgfpathrectangle{\pgfqpoint{0.050000in}{0.050000in}}{\pgfqpoint{2.419000in}{2.419000in}}%
\pgfusepath{clip}%
\pgfsetbuttcap%
\pgfsetroundjoin%
\definecolor{currentfill}{rgb}{0.200000,0.133333,0.533333}%
\pgfsetfillcolor{currentfill}%
\pgfsetfillopacity{0.635763}%
\pgfsetlinewidth{1.003750pt}%
\definecolor{currentstroke}{rgb}{0.200000,0.133333,0.533333}%
\pgfsetstrokecolor{currentstroke}%
\pgfsetstrokeopacity{0.635763}%
\pgfsetdash{}{0pt}%
\pgfpathmoveto{\pgfqpoint{-1.626533in}{-2.872808in}}%
\pgfpathcurveto{\pgfqpoint{-1.617324in}{-2.872808in}}{\pgfqpoint{-1.608492in}{-2.869150in}}{\pgfqpoint{-1.601980in}{-2.862638in}}%
\pgfpathcurveto{\pgfqpoint{-1.595469in}{-2.856127in}}{\pgfqpoint{-1.591810in}{-2.847294in}}{\pgfqpoint{-1.591810in}{-2.838086in}}%
\pgfpathcurveto{\pgfqpoint{-1.591810in}{-2.828878in}}{\pgfqpoint{-1.595469in}{-2.820045in}}{\pgfqpoint{-1.601980in}{-2.813534in}}%
\pgfpathcurveto{\pgfqpoint{-1.608492in}{-2.807022in}}{\pgfqpoint{-1.617324in}{-2.803364in}}{\pgfqpoint{-1.626533in}{-2.803364in}}%
\pgfpathcurveto{\pgfqpoint{-1.635741in}{-2.803364in}}{\pgfqpoint{-1.644574in}{-2.807022in}}{\pgfqpoint{-1.651085in}{-2.813534in}}%
\pgfpathcurveto{\pgfqpoint{-1.657596in}{-2.820045in}}{\pgfqpoint{-1.661255in}{-2.828878in}}{\pgfqpoint{-1.661255in}{-2.838086in}}%
\pgfpathcurveto{\pgfqpoint{-1.661255in}{-2.847294in}}{\pgfqpoint{-1.657596in}{-2.856127in}}{\pgfqpoint{-1.651085in}{-2.862638in}}%
\pgfpathcurveto{\pgfqpoint{-1.644574in}{-2.869150in}}{\pgfqpoint{-1.635741in}{-2.872808in}}{\pgfqpoint{-1.626533in}{-2.872808in}}%
\pgfpathlineto{\pgfqpoint{-1.626533in}{-2.872808in}}%
\pgfpathclose%
\pgfusepath{stroke,fill}%
\end{pgfscope}%
\begin{pgfscope}%
\pgfpathrectangle{\pgfqpoint{0.050000in}{0.050000in}}{\pgfqpoint{2.419000in}{2.419000in}}%
\pgfusepath{clip}%
\pgfsetbuttcap%
\pgfsetroundjoin%
\definecolor{currentfill}{rgb}{0.200000,0.133333,0.533333}%
\pgfsetfillcolor{currentfill}%
\pgfsetfillopacity{0.650820}%
\pgfsetlinewidth{1.003750pt}%
\definecolor{currentstroke}{rgb}{0.200000,0.133333,0.533333}%
\pgfsetstrokecolor{currentstroke}%
\pgfsetstrokeopacity{0.650820}%
\pgfsetdash{}{0pt}%
\pgfpathmoveto{\pgfqpoint{2.788698in}{-3.189746in}}%
\pgfpathcurveto{\pgfqpoint{2.797906in}{-3.189746in}}{\pgfqpoint{2.806739in}{-3.186087in}}{\pgfqpoint{2.813250in}{-3.179576in}}%
\pgfpathcurveto{\pgfqpoint{2.819761in}{-3.173064in}}{\pgfqpoint{2.823420in}{-3.164232in}}{\pgfqpoint{2.823420in}{-3.155023in}}%
\pgfpathcurveto{\pgfqpoint{2.823420in}{-3.145815in}}{\pgfqpoint{2.819761in}{-3.136982in}}{\pgfqpoint{2.813250in}{-3.130471in}}%
\pgfpathcurveto{\pgfqpoint{2.806739in}{-3.123960in}}{\pgfqpoint{2.797906in}{-3.120301in}}{\pgfqpoint{2.788698in}{-3.120301in}}%
\pgfpathcurveto{\pgfqpoint{2.779489in}{-3.120301in}}{\pgfqpoint{2.770657in}{-3.123960in}}{\pgfqpoint{2.764145in}{-3.130471in}}%
\pgfpathcurveto{\pgfqpoint{2.757634in}{-3.136982in}}{\pgfqpoint{2.753975in}{-3.145815in}}{\pgfqpoint{2.753975in}{-3.155023in}}%
\pgfpathcurveto{\pgfqpoint{2.753975in}{-3.164232in}}{\pgfqpoint{2.757634in}{-3.173064in}}{\pgfqpoint{2.764145in}{-3.179576in}}%
\pgfpathcurveto{\pgfqpoint{2.770657in}{-3.186087in}}{\pgfqpoint{2.779489in}{-3.189746in}}{\pgfqpoint{2.788698in}{-3.189746in}}%
\pgfpathlineto{\pgfqpoint{2.788698in}{-3.189746in}}%
\pgfpathclose%
\pgfusepath{stroke,fill}%
\end{pgfscope}%
\begin{pgfscope}%
\pgfpathrectangle{\pgfqpoint{0.050000in}{0.050000in}}{\pgfqpoint{2.419000in}{2.419000in}}%
\pgfusepath{clip}%
\pgfsetbuttcap%
\pgfsetroundjoin%
\definecolor{currentfill}{rgb}{0.200000,0.133333,0.533333}%
\pgfsetfillcolor{currentfill}%
\pgfsetfillopacity{0.650820}%
\pgfsetlinewidth{1.003750pt}%
\definecolor{currentstroke}{rgb}{0.200000,0.133333,0.533333}%
\pgfsetstrokecolor{currentstroke}%
\pgfsetstrokeopacity{0.650820}%
\pgfsetdash{}{0pt}%
\pgfpathmoveto{\pgfqpoint{-4.780874in}{-3.189746in}}%
\pgfpathcurveto{\pgfqpoint{-4.771665in}{-3.189746in}}{\pgfqpoint{-4.762833in}{-3.186087in}}{\pgfqpoint{-4.756322in}{-3.179576in}}%
\pgfpathcurveto{\pgfqpoint{-4.749810in}{-3.173064in}}{\pgfqpoint{-4.746152in}{-3.164232in}}{\pgfqpoint{-4.746152in}{-3.155023in}}%
\pgfpathcurveto{\pgfqpoint{-4.746152in}{-3.145815in}}{\pgfqpoint{-4.749810in}{-3.136982in}}{\pgfqpoint{-4.756322in}{-3.130471in}}%
\pgfpathcurveto{\pgfqpoint{-4.762833in}{-3.123960in}}{\pgfqpoint{-4.771665in}{-3.120301in}}{\pgfqpoint{-4.780874in}{-3.120301in}}%
\pgfpathcurveto{\pgfqpoint{-4.790082in}{-3.120301in}}{\pgfqpoint{-4.798915in}{-3.123960in}}{\pgfqpoint{-4.805426in}{-3.130471in}}%
\pgfpathcurveto{\pgfqpoint{-4.811938in}{-3.136982in}}{\pgfqpoint{-4.815596in}{-3.145815in}}{\pgfqpoint{-4.815596in}{-3.155023in}}%
\pgfpathcurveto{\pgfqpoint{-4.815596in}{-3.164232in}}{\pgfqpoint{-4.811938in}{-3.173064in}}{\pgfqpoint{-4.805426in}{-3.179576in}}%
\pgfpathcurveto{\pgfqpoint{-4.798915in}{-3.186087in}}{\pgfqpoint{-4.790082in}{-3.189746in}}{\pgfqpoint{-4.780874in}{-3.189746in}}%
\pgfpathlineto{\pgfqpoint{-4.780874in}{-3.189746in}}%
\pgfpathclose%
\pgfusepath{stroke,fill}%
\end{pgfscope}%
\begin{pgfscope}%
\pgfpathrectangle{\pgfqpoint{0.050000in}{0.050000in}}{\pgfqpoint{2.419000in}{2.419000in}}%
\pgfusepath{clip}%
\pgfsetbuttcap%
\pgfsetroundjoin%
\definecolor{currentfill}{rgb}{0.200000,0.133333,0.533333}%
\pgfsetfillcolor{currentfill}%
\pgfsetfillopacity{0.650820}%
\pgfsetlinewidth{1.003750pt}%
\definecolor{currentstroke}{rgb}{0.200000,0.133333,0.533333}%
\pgfsetstrokecolor{currentstroke}%
\pgfsetstrokeopacity{0.650820}%
\pgfsetdash{}{0pt}%
\pgfpathmoveto{\pgfqpoint{10.358269in}{-3.189746in}}%
\pgfpathcurveto{\pgfqpoint{10.367478in}{-3.189746in}}{\pgfqpoint{10.376310in}{-3.186087in}}{\pgfqpoint{10.382821in}{-3.179576in}}%
\pgfpathcurveto{\pgfqpoint{10.389333in}{-3.173064in}}{\pgfqpoint{10.392991in}{-3.164232in}}{\pgfqpoint{10.392991in}{-3.155023in}}%
\pgfpathcurveto{\pgfqpoint{10.392991in}{-3.145815in}}{\pgfqpoint{10.389333in}{-3.136982in}}{\pgfqpoint{10.382821in}{-3.130471in}}%
\pgfpathcurveto{\pgfqpoint{10.376310in}{-3.123960in}}{\pgfqpoint{10.367478in}{-3.120301in}}{\pgfqpoint{10.358269in}{-3.120301in}}%
\pgfpathcurveto{\pgfqpoint{10.349061in}{-3.120301in}}{\pgfqpoint{10.340228in}{-3.123960in}}{\pgfqpoint{10.333717in}{-3.130471in}}%
\pgfpathcurveto{\pgfqpoint{10.327205in}{-3.136982in}}{\pgfqpoint{10.323547in}{-3.145815in}}{\pgfqpoint{10.323547in}{-3.155023in}}%
\pgfpathcurveto{\pgfqpoint{10.323547in}{-3.164232in}}{\pgfqpoint{10.327205in}{-3.173064in}}{\pgfqpoint{10.333717in}{-3.179576in}}%
\pgfpathcurveto{\pgfqpoint{10.340228in}{-3.186087in}}{\pgfqpoint{10.349061in}{-3.189746in}}{\pgfqpoint{10.358269in}{-3.189746in}}%
\pgfpathlineto{\pgfqpoint{10.358269in}{-3.189746in}}%
\pgfpathclose%
\pgfusepath{stroke,fill}%
\end{pgfscope}%
\begin{pgfscope}%
\pgfpathrectangle{\pgfqpoint{0.050000in}{0.050000in}}{\pgfqpoint{2.419000in}{2.419000in}}%
\pgfusepath{clip}%
\pgfsetbuttcap%
\pgfsetroundjoin%
\definecolor{currentfill}{rgb}{0.200000,0.133333,0.533333}%
\pgfsetfillcolor{currentfill}%
\pgfsetfillopacity{0.666718}%
\pgfsetlinewidth{1.003750pt}%
\definecolor{currentstroke}{rgb}{0.200000,0.133333,0.533333}%
\pgfsetstrokecolor{currentstroke}%
\pgfsetstrokeopacity{0.666718}%
\pgfsetdash{}{0pt}%
\pgfpathmoveto{\pgfqpoint{-0.330474in}{-3.524364in}}%
\pgfpathcurveto{\pgfqpoint{-0.321265in}{-3.524364in}}{\pgfqpoint{-0.312433in}{-3.520705in}}{\pgfqpoint{-0.305921in}{-3.514194in}}%
\pgfpathcurveto{\pgfqpoint{-0.299410in}{-3.507682in}}{\pgfqpoint{-0.295751in}{-3.498850in}}{\pgfqpoint{-0.295751in}{-3.489641in}}%
\pgfpathcurveto{\pgfqpoint{-0.295751in}{-3.480433in}}{\pgfqpoint{-0.299410in}{-3.471600in}}{\pgfqpoint{-0.305921in}{-3.465089in}}%
\pgfpathcurveto{\pgfqpoint{-0.312433in}{-3.458578in}}{\pgfqpoint{-0.321265in}{-3.454919in}}{\pgfqpoint{-0.330474in}{-3.454919in}}%
\pgfpathcurveto{\pgfqpoint{-0.339682in}{-3.454919in}}{\pgfqpoint{-0.348515in}{-3.458578in}}{\pgfqpoint{-0.355026in}{-3.465089in}}%
\pgfpathcurveto{\pgfqpoint{-0.361537in}{-3.471600in}}{\pgfqpoint{-0.365196in}{-3.480433in}}{\pgfqpoint{-0.365196in}{-3.489641in}}%
\pgfpathcurveto{\pgfqpoint{-0.365196in}{-3.498850in}}{\pgfqpoint{-0.361537in}{-3.507682in}}{\pgfqpoint{-0.355026in}{-3.514194in}}%
\pgfpathcurveto{\pgfqpoint{-0.348515in}{-3.520705in}}{\pgfqpoint{-0.339682in}{-3.524364in}}{\pgfqpoint{-0.330474in}{-3.524364in}}%
\pgfpathlineto{\pgfqpoint{-0.330474in}{-3.524364in}}%
\pgfpathclose%
\pgfusepath{stroke,fill}%
\end{pgfscope}%
\begin{pgfscope}%
\pgfpathrectangle{\pgfqpoint{0.050000in}{0.050000in}}{\pgfqpoint{2.419000in}{2.419000in}}%
\pgfusepath{clip}%
\pgfsetbuttcap%
\pgfsetroundjoin%
\definecolor{currentfill}{rgb}{0.200000,0.133333,0.533333}%
\pgfsetfillcolor{currentfill}%
\pgfsetfillopacity{0.666718}%
\pgfsetlinewidth{1.003750pt}%
\definecolor{currentstroke}{rgb}{0.200000,0.133333,0.533333}%
\pgfsetstrokecolor{currentstroke}%
\pgfsetstrokeopacity{0.666718}%
\pgfsetdash{}{0pt}%
\pgfpathmoveto{\pgfqpoint{7.450236in}{-3.524364in}}%
\pgfpathcurveto{\pgfqpoint{7.459445in}{-3.524364in}}{\pgfqpoint{7.468277in}{-3.520705in}}{\pgfqpoint{7.474789in}{-3.514194in}}%
\pgfpathcurveto{\pgfqpoint{7.481300in}{-3.507682in}}{\pgfqpoint{7.484959in}{-3.498850in}}{\pgfqpoint{7.484959in}{-3.489641in}}%
\pgfpathcurveto{\pgfqpoint{7.484959in}{-3.480433in}}{\pgfqpoint{7.481300in}{-3.471600in}}{\pgfqpoint{7.474789in}{-3.465089in}}%
\pgfpathcurveto{\pgfqpoint{7.468277in}{-3.458578in}}{\pgfqpoint{7.459445in}{-3.454919in}}{\pgfqpoint{7.450236in}{-3.454919in}}%
\pgfpathcurveto{\pgfqpoint{7.441028in}{-3.454919in}}{\pgfqpoint{7.432196in}{-3.458578in}}{\pgfqpoint{7.425684in}{-3.465089in}}%
\pgfpathcurveto{\pgfqpoint{7.419173in}{-3.471600in}}{\pgfqpoint{7.415514in}{-3.480433in}}{\pgfqpoint{7.415514in}{-3.489641in}}%
\pgfpathcurveto{\pgfqpoint{7.415514in}{-3.498850in}}{\pgfqpoint{7.419173in}{-3.507682in}}{\pgfqpoint{7.425684in}{-3.514194in}}%
\pgfpathcurveto{\pgfqpoint{7.432196in}{-3.520705in}}{\pgfqpoint{7.441028in}{-3.524364in}}{\pgfqpoint{7.450236in}{-3.524364in}}%
\pgfpathlineto{\pgfqpoint{7.450236in}{-3.524364in}}%
\pgfpathclose%
\pgfusepath{stroke,fill}%
\end{pgfscope}%
\begin{pgfscope}%
\pgfpathrectangle{\pgfqpoint{0.050000in}{0.050000in}}{\pgfqpoint{2.419000in}{2.419000in}}%
\pgfusepath{clip}%
\pgfsetbuttcap%
\pgfsetroundjoin%
\definecolor{currentfill}{rgb}{0.200000,0.133333,0.533333}%
\pgfsetfillcolor{currentfill}%
\pgfsetfillopacity{0.683527}%
\pgfsetlinewidth{1.003750pt}%
\definecolor{currentstroke}{rgb}{0.200000,0.133333,0.533333}%
\pgfsetstrokecolor{currentstroke}%
\pgfsetstrokeopacity{0.683527}%
\pgfsetdash{}{0pt}%
\pgfpathmoveto{\pgfqpoint{-3.628644in}{-3.878184in}}%
\pgfpathcurveto{\pgfqpoint{-3.619436in}{-3.878184in}}{\pgfqpoint{-3.610603in}{-3.874526in}}{\pgfqpoint{-3.604092in}{-3.868014in}}%
\pgfpathcurveto{\pgfqpoint{-3.597581in}{-3.861503in}}{\pgfqpoint{-3.593922in}{-3.852671in}}{\pgfqpoint{-3.593922in}{-3.843462in}}%
\pgfpathcurveto{\pgfqpoint{-3.593922in}{-3.834254in}}{\pgfqpoint{-3.597581in}{-3.825421in}}{\pgfqpoint{-3.604092in}{-3.818910in}}%
\pgfpathcurveto{\pgfqpoint{-3.610603in}{-3.812398in}}{\pgfqpoint{-3.619436in}{-3.808740in}}{\pgfqpoint{-3.628644in}{-3.808740in}}%
\pgfpathcurveto{\pgfqpoint{-3.637853in}{-3.808740in}}{\pgfqpoint{-3.646685in}{-3.812398in}}{\pgfqpoint{-3.653197in}{-3.818910in}}%
\pgfpathcurveto{\pgfqpoint{-3.659708in}{-3.825421in}}{\pgfqpoint{-3.663367in}{-3.834254in}}{\pgfqpoint{-3.663367in}{-3.843462in}}%
\pgfpathcurveto{\pgfqpoint{-3.663367in}{-3.852671in}}{\pgfqpoint{-3.659708in}{-3.861503in}}{\pgfqpoint{-3.653197in}{-3.868014in}}%
\pgfpathcurveto{\pgfqpoint{-3.646685in}{-3.874526in}}{\pgfqpoint{-3.637853in}{-3.878184in}}{\pgfqpoint{-3.628644in}{-3.878184in}}%
\pgfpathlineto{\pgfqpoint{-3.628644in}{-3.878184in}}%
\pgfpathclose%
\pgfusepath{stroke,fill}%
\end{pgfscope}%
\begin{pgfscope}%
\pgfpathrectangle{\pgfqpoint{0.050000in}{0.050000in}}{\pgfqpoint{2.419000in}{2.419000in}}%
\pgfusepath{clip}%
\pgfsetbuttcap%
\pgfsetroundjoin%
\definecolor{currentfill}{rgb}{0.200000,0.133333,0.533333}%
\pgfsetfillcolor{currentfill}%
\pgfsetfillopacity{0.683527}%
\pgfsetlinewidth{1.003750pt}%
\definecolor{currentstroke}{rgb}{0.200000,0.133333,0.533333}%
\pgfsetstrokecolor{currentstroke}%
\pgfsetstrokeopacity{0.683527}%
\pgfsetdash{}{0pt}%
\pgfpathmoveto{\pgfqpoint{4.375321in}{-3.878184in}}%
\pgfpathcurveto{\pgfqpoint{4.384529in}{-3.878184in}}{\pgfqpoint{4.393362in}{-3.874526in}}{\pgfqpoint{4.399873in}{-3.868014in}}%
\pgfpathcurveto{\pgfqpoint{4.406385in}{-3.861503in}}{\pgfqpoint{4.410043in}{-3.852671in}}{\pgfqpoint{4.410043in}{-3.843462in}}%
\pgfpathcurveto{\pgfqpoint{4.410043in}{-3.834254in}}{\pgfqpoint{4.406385in}{-3.825421in}}{\pgfqpoint{4.399873in}{-3.818910in}}%
\pgfpathcurveto{\pgfqpoint{4.393362in}{-3.812398in}}{\pgfqpoint{4.384529in}{-3.808740in}}{\pgfqpoint{4.375321in}{-3.808740in}}%
\pgfpathcurveto{\pgfqpoint{4.366113in}{-3.808740in}}{\pgfqpoint{4.357280in}{-3.812398in}}{\pgfqpoint{4.350769in}{-3.818910in}}%
\pgfpathcurveto{\pgfqpoint{4.344257in}{-3.825421in}}{\pgfqpoint{4.340599in}{-3.834254in}}{\pgfqpoint{4.340599in}{-3.843462in}}%
\pgfpathcurveto{\pgfqpoint{4.340599in}{-3.852671in}}{\pgfqpoint{4.344257in}{-3.861503in}}{\pgfqpoint{4.350769in}{-3.868014in}}%
\pgfpathcurveto{\pgfqpoint{4.357280in}{-3.874526in}}{\pgfqpoint{4.366113in}{-3.878184in}}{\pgfqpoint{4.375321in}{-3.878184in}}%
\pgfpathlineto{\pgfqpoint{4.375321in}{-3.878184in}}%
\pgfpathclose%
\pgfusepath{stroke,fill}%
\end{pgfscope}%
\begin{pgfscope}%
\pgfpathrectangle{\pgfqpoint{0.050000in}{0.050000in}}{\pgfqpoint{2.419000in}{2.419000in}}%
\pgfusepath{clip}%
\pgfsetbuttcap%
\pgfsetroundjoin%
\definecolor{currentfill}{rgb}{0.200000,0.133333,0.533333}%
\pgfsetfillcolor{currentfill}%
\pgfsetfillopacity{0.701330}%
\pgfsetlinewidth{1.003750pt}%
\definecolor{currentstroke}{rgb}{0.200000,0.133333,0.533333}%
\pgfsetstrokecolor{currentstroke}%
\pgfsetstrokeopacity{0.701330}%
\pgfsetdash{}{0pt}%
\pgfpathmoveto{\pgfqpoint{1.118733in}{-4.252910in}}%
\pgfpathcurveto{\pgfqpoint{1.127942in}{-4.252910in}}{\pgfqpoint{1.136774in}{-4.249251in}}{\pgfqpoint{1.143285in}{-4.242740in}}%
\pgfpathcurveto{\pgfqpoint{1.149797in}{-4.236228in}}{\pgfqpoint{1.153455in}{-4.227396in}}{\pgfqpoint{1.153455in}{-4.218187in}}%
\pgfpathcurveto{\pgfqpoint{1.153455in}{-4.208979in}}{\pgfqpoint{1.149797in}{-4.200146in}}{\pgfqpoint{1.143285in}{-4.193635in}}%
\pgfpathcurveto{\pgfqpoint{1.136774in}{-4.187124in}}{\pgfqpoint{1.127942in}{-4.183465in}}{\pgfqpoint{1.118733in}{-4.183465in}}%
\pgfpathcurveto{\pgfqpoint{1.109525in}{-4.183465in}}{\pgfqpoint{1.100692in}{-4.187124in}}{\pgfqpoint{1.094181in}{-4.193635in}}%
\pgfpathcurveto{\pgfqpoint{1.087669in}{-4.200146in}}{\pgfqpoint{1.084011in}{-4.208979in}}{\pgfqpoint{1.084011in}{-4.218187in}}%
\pgfpathcurveto{\pgfqpoint{1.084011in}{-4.227396in}}{\pgfqpoint{1.087669in}{-4.236228in}}{\pgfqpoint{1.094181in}{-4.242740in}}%
\pgfpathcurveto{\pgfqpoint{1.100692in}{-4.249251in}}{\pgfqpoint{1.109525in}{-4.252910in}}{\pgfqpoint{1.118733in}{-4.252910in}}%
\pgfpathlineto{\pgfqpoint{1.118733in}{-4.252910in}}%
\pgfpathclose%
\pgfusepath{stroke,fill}%
\end{pgfscope}%
\begin{pgfscope}%
\pgfpathrectangle{\pgfqpoint{0.050000in}{0.050000in}}{\pgfqpoint{2.419000in}{2.419000in}}%
\pgfusepath{clip}%
\pgfsetbuttcap%
\pgfsetroundjoin%
\definecolor{currentfill}{rgb}{0.200000,0.133333,0.533333}%
\pgfsetfillcolor{currentfill}%
\pgfsetfillopacity{0.701330}%
\pgfsetlinewidth{1.003750pt}%
\definecolor{currentstroke}{rgb}{0.200000,0.133333,0.533333}%
\pgfsetstrokecolor{currentstroke}%
\pgfsetstrokeopacity{0.701330}%
\pgfsetdash{}{0pt}%
\pgfpathmoveto{\pgfqpoint{9.359144in}{-4.252910in}}%
\pgfpathcurveto{\pgfqpoint{9.368352in}{-4.252910in}}{\pgfqpoint{9.377185in}{-4.249251in}}{\pgfqpoint{9.383696in}{-4.242740in}}%
\pgfpathcurveto{\pgfqpoint{9.390208in}{-4.236228in}}{\pgfqpoint{9.393866in}{-4.227396in}}{\pgfqpoint{9.393866in}{-4.218187in}}%
\pgfpathcurveto{\pgfqpoint{9.393866in}{-4.208979in}}{\pgfqpoint{9.390208in}{-4.200146in}}{\pgfqpoint{9.383696in}{-4.193635in}}%
\pgfpathcurveto{\pgfqpoint{9.377185in}{-4.187124in}}{\pgfqpoint{9.368352in}{-4.183465in}}{\pgfqpoint{9.359144in}{-4.183465in}}%
\pgfpathcurveto{\pgfqpoint{9.349936in}{-4.183465in}}{\pgfqpoint{9.341103in}{-4.187124in}}{\pgfqpoint{9.334592in}{-4.193635in}}%
\pgfpathcurveto{\pgfqpoint{9.328080in}{-4.200146in}}{\pgfqpoint{9.324422in}{-4.208979in}}{\pgfqpoint{9.324422in}{-4.218187in}}%
\pgfpathcurveto{\pgfqpoint{9.324422in}{-4.227396in}}{\pgfqpoint{9.328080in}{-4.236228in}}{\pgfqpoint{9.334592in}{-4.242740in}}%
\pgfpathcurveto{\pgfqpoint{9.341103in}{-4.249251in}}{\pgfqpoint{9.349936in}{-4.252910in}}{\pgfqpoint{9.359144in}{-4.252910in}}%
\pgfpathlineto{\pgfqpoint{9.359144in}{-4.252910in}}%
\pgfpathclose%
\pgfusepath{stroke,fill}%
\end{pgfscope}%
\begin{pgfscope}%
\pgfpathrectangle{\pgfqpoint{0.050000in}{0.050000in}}{\pgfqpoint{2.419000in}{2.419000in}}%
\pgfusepath{clip}%
\pgfsetbuttcap%
\pgfsetroundjoin%
\definecolor{currentfill}{rgb}{0.200000,0.133333,0.533333}%
\pgfsetfillcolor{currentfill}%
\pgfsetfillopacity{0.720217}%
\pgfsetlinewidth{1.003750pt}%
\definecolor{currentstroke}{rgb}{0.200000,0.133333,0.533333}%
\pgfsetstrokecolor{currentstroke}%
\pgfsetstrokeopacity{0.720217}%
\pgfsetdash{}{0pt}%
\pgfpathmoveto{\pgfqpoint{-2.336118in}{-4.650448in}}%
\pgfpathcurveto{\pgfqpoint{-2.326909in}{-4.650448in}}{\pgfqpoint{-2.318077in}{-4.646790in}}{\pgfqpoint{-2.311566in}{-4.640278in}}%
\pgfpathcurveto{\pgfqpoint{-2.305054in}{-4.633767in}}{\pgfqpoint{-2.301396in}{-4.624935in}}{\pgfqpoint{-2.301396in}{-4.615726in}}%
\pgfpathcurveto{\pgfqpoint{-2.301396in}{-4.606518in}}{\pgfqpoint{-2.305054in}{-4.597685in}}{\pgfqpoint{-2.311566in}{-4.591174in}}%
\pgfpathcurveto{\pgfqpoint{-2.318077in}{-4.584662in}}{\pgfqpoint{-2.326909in}{-4.581004in}}{\pgfqpoint{-2.336118in}{-4.581004in}}%
\pgfpathcurveto{\pgfqpoint{-2.345326in}{-4.581004in}}{\pgfqpoint{-2.354159in}{-4.584662in}}{\pgfqpoint{-2.360670in}{-4.591174in}}%
\pgfpathcurveto{\pgfqpoint{-2.367182in}{-4.597685in}}{\pgfqpoint{-2.370840in}{-4.606518in}}{\pgfqpoint{-2.370840in}{-4.615726in}}%
\pgfpathcurveto{\pgfqpoint{-2.370840in}{-4.624935in}}{\pgfqpoint{-2.367182in}{-4.633767in}}{\pgfqpoint{-2.360670in}{-4.640278in}}%
\pgfpathcurveto{\pgfqpoint{-2.354159in}{-4.646790in}}{\pgfqpoint{-2.345326in}{-4.650448in}}{\pgfqpoint{-2.336118in}{-4.650448in}}%
\pgfpathlineto{\pgfqpoint{-2.336118in}{-4.650448in}}%
\pgfpathclose%
\pgfusepath{stroke,fill}%
\end{pgfscope}%
\begin{pgfscope}%
\pgfpathrectangle{\pgfqpoint{0.050000in}{0.050000in}}{\pgfqpoint{2.419000in}{2.419000in}}%
\pgfusepath{clip}%
\pgfsetbuttcap%
\pgfsetroundjoin%
\definecolor{currentfill}{rgb}{0.200000,0.133333,0.533333}%
\pgfsetfillcolor{currentfill}%
\pgfsetfillopacity{0.720217}%
\pgfsetlinewidth{1.003750pt}%
\definecolor{currentstroke}{rgb}{0.200000,0.133333,0.533333}%
\pgfsetstrokecolor{currentstroke}%
\pgfsetstrokeopacity{0.720217}%
\pgfsetdash{}{0pt}%
\pgfpathmoveto{\pgfqpoint{6.155134in}{-4.650448in}}%
\pgfpathcurveto{\pgfqpoint{6.164342in}{-4.650448in}}{\pgfqpoint{6.173175in}{-4.646790in}}{\pgfqpoint{6.179686in}{-4.640278in}}%
\pgfpathcurveto{\pgfqpoint{6.186197in}{-4.633767in}}{\pgfqpoint{6.189856in}{-4.624935in}}{\pgfqpoint{6.189856in}{-4.615726in}}%
\pgfpathcurveto{\pgfqpoint{6.189856in}{-4.606518in}}{\pgfqpoint{6.186197in}{-4.597685in}}{\pgfqpoint{6.179686in}{-4.591174in}}%
\pgfpathcurveto{\pgfqpoint{6.173175in}{-4.584662in}}{\pgfqpoint{6.164342in}{-4.581004in}}{\pgfqpoint{6.155134in}{-4.581004in}}%
\pgfpathcurveto{\pgfqpoint{6.145925in}{-4.581004in}}{\pgfqpoint{6.137093in}{-4.584662in}}{\pgfqpoint{6.130581in}{-4.591174in}}%
\pgfpathcurveto{\pgfqpoint{6.124070in}{-4.597685in}}{\pgfqpoint{6.120411in}{-4.606518in}}{\pgfqpoint{6.120411in}{-4.615726in}}%
\pgfpathcurveto{\pgfqpoint{6.120411in}{-4.624935in}}{\pgfqpoint{6.124070in}{-4.633767in}}{\pgfqpoint{6.130581in}{-4.640278in}}%
\pgfpathcurveto{\pgfqpoint{6.137093in}{-4.646790in}}{\pgfqpoint{6.145925in}{-4.650448in}}{\pgfqpoint{6.155134in}{-4.650448in}}%
\pgfpathlineto{\pgfqpoint{6.155134in}{-4.650448in}}%
\pgfpathclose%
\pgfusepath{stroke,fill}%
\end{pgfscope}%
\begin{pgfscope}%
\pgfpathrectangle{\pgfqpoint{0.050000in}{0.050000in}}{\pgfqpoint{2.419000in}{2.419000in}}%
\pgfusepath{clip}%
\pgfsetbuttcap%
\pgfsetroundjoin%
\definecolor{currentfill}{rgb}{0.200000,0.133333,0.533333}%
\pgfsetfillcolor{currentfill}%
\pgfsetfillopacity{0.740290}%
\pgfsetlinewidth{1.003750pt}%
\definecolor{currentstroke}{rgb}{0.200000,0.133333,0.533333}%
\pgfsetstrokecolor{currentstroke}%
\pgfsetstrokeopacity{0.740290}%
\pgfsetdash{}{0pt}%
\pgfpathmoveto{\pgfqpoint{2.749937in}{-5.072949in}}%
\pgfpathcurveto{\pgfqpoint{2.759145in}{-5.072949in}}{\pgfqpoint{2.767978in}{-5.069291in}}{\pgfqpoint{2.774489in}{-5.062779in}}%
\pgfpathcurveto{\pgfqpoint{2.781001in}{-5.056268in}}{\pgfqpoint{2.784659in}{-5.047435in}}{\pgfqpoint{2.784659in}{-5.038227in}}%
\pgfpathcurveto{\pgfqpoint{2.784659in}{-5.029019in}}{\pgfqpoint{2.781001in}{-5.020186in}}{\pgfqpoint{2.774489in}{-5.013675in}}%
\pgfpathcurveto{\pgfqpoint{2.767978in}{-5.007163in}}{\pgfqpoint{2.759145in}{-5.003505in}}{\pgfqpoint{2.749937in}{-5.003505in}}%
\pgfpathcurveto{\pgfqpoint{2.740728in}{-5.003505in}}{\pgfqpoint{2.731896in}{-5.007163in}}{\pgfqpoint{2.725385in}{-5.013675in}}%
\pgfpathcurveto{\pgfqpoint{2.718873in}{-5.020186in}}{\pgfqpoint{2.715215in}{-5.029019in}}{\pgfqpoint{2.715215in}{-5.038227in}}%
\pgfpathcurveto{\pgfqpoint{2.715215in}{-5.047435in}}{\pgfqpoint{2.718873in}{-5.056268in}}{\pgfqpoint{2.725385in}{-5.062779in}}%
\pgfpathcurveto{\pgfqpoint{2.731896in}{-5.069291in}}{\pgfqpoint{2.740728in}{-5.072949in}}{\pgfqpoint{2.749937in}{-5.072949in}}%
\pgfpathlineto{\pgfqpoint{2.749937in}{-5.072949in}}%
\pgfpathclose%
\pgfusepath{stroke,fill}%
\end{pgfscope}%
\begin{pgfscope}%
\pgfpathrectangle{\pgfqpoint{0.050000in}{0.050000in}}{\pgfqpoint{2.419000in}{2.419000in}}%
\pgfusepath{clip}%
\pgfsetbuttcap%
\pgfsetroundjoin%
\definecolor{currentfill}{rgb}{0.200000,0.133333,0.533333}%
\pgfsetfillcolor{currentfill}%
\pgfsetfillopacity{0.761663}%
\pgfsetlinewidth{1.003750pt}%
\definecolor{currentstroke}{rgb}{0.200000,0.133333,0.533333}%
\pgfsetstrokecolor{currentstroke}%
\pgfsetstrokeopacity{0.761663}%
\pgfsetdash{}{0pt}%
\pgfpathmoveto{\pgfqpoint{-0.876009in}{-5.522840in}}%
\pgfpathcurveto{\pgfqpoint{-0.866801in}{-5.522840in}}{\pgfqpoint{-0.857968in}{-5.519181in}}{\pgfqpoint{-0.851457in}{-5.512670in}}%
\pgfpathcurveto{\pgfqpoint{-0.844946in}{-5.506159in}}{\pgfqpoint{-0.841287in}{-5.497326in}}{\pgfqpoint{-0.841287in}{-5.488118in}}%
\pgfpathcurveto{\pgfqpoint{-0.841287in}{-5.478909in}}{\pgfqpoint{-0.844946in}{-5.470077in}}{\pgfqpoint{-0.851457in}{-5.463565in}}%
\pgfpathcurveto{\pgfqpoint{-0.857968in}{-5.457054in}}{\pgfqpoint{-0.866801in}{-5.453395in}}{\pgfqpoint{-0.876009in}{-5.453395in}}%
\pgfpathcurveto{\pgfqpoint{-0.885218in}{-5.453395in}}{\pgfqpoint{-0.894050in}{-5.457054in}}{\pgfqpoint{-0.900562in}{-5.463565in}}%
\pgfpathcurveto{\pgfqpoint{-0.907073in}{-5.470077in}}{\pgfqpoint{-0.910732in}{-5.478909in}}{\pgfqpoint{-0.910732in}{-5.488118in}}%
\pgfpathcurveto{\pgfqpoint{-0.910732in}{-5.497326in}}{\pgfqpoint{-0.907073in}{-5.506159in}}{\pgfqpoint{-0.900562in}{-5.512670in}}%
\pgfpathcurveto{\pgfqpoint{-0.894050in}{-5.519181in}}{\pgfqpoint{-0.885218in}{-5.522840in}}{\pgfqpoint{-0.876009in}{-5.522840in}}%
\pgfpathlineto{\pgfqpoint{-0.876009in}{-5.522840in}}%
\pgfpathclose%
\pgfusepath{stroke,fill}%
\end{pgfscope}%
\begin{pgfscope}%
\pgfpathrectangle{\pgfqpoint{0.050000in}{0.050000in}}{\pgfqpoint{2.419000in}{2.419000in}}%
\pgfusepath{clip}%
\pgfsetbuttcap%
\pgfsetroundjoin%
\definecolor{currentfill}{rgb}{0.200000,0.133333,0.533333}%
\pgfsetfillcolor{currentfill}%
\pgfsetfillopacity{0.761663}%
\pgfsetlinewidth{1.003750pt}%
\definecolor{currentstroke}{rgb}{0.200000,0.133333,0.533333}%
\pgfsetstrokecolor{currentstroke}%
\pgfsetstrokeopacity{0.761663}%
\pgfsetdash{}{0pt}%
\pgfpathmoveto{\pgfqpoint{8.165707in}{-5.522840in}}%
\pgfpathcurveto{\pgfqpoint{8.174915in}{-5.522840in}}{\pgfqpoint{8.183748in}{-5.519181in}}{\pgfqpoint{8.190259in}{-5.512670in}}%
\pgfpathcurveto{\pgfqpoint{8.196771in}{-5.506159in}}{\pgfqpoint{8.200429in}{-5.497326in}}{\pgfqpoint{8.200429in}{-5.488118in}}%
\pgfpathcurveto{\pgfqpoint{8.200429in}{-5.478909in}}{\pgfqpoint{8.196771in}{-5.470077in}}{\pgfqpoint{8.190259in}{-5.463565in}}%
\pgfpathcurveto{\pgfqpoint{8.183748in}{-5.457054in}}{\pgfqpoint{8.174915in}{-5.453395in}}{\pgfqpoint{8.165707in}{-5.453395in}}%
\pgfpathcurveto{\pgfqpoint{8.156499in}{-5.453395in}}{\pgfqpoint{8.147666in}{-5.457054in}}{\pgfqpoint{8.141155in}{-5.463565in}}%
\pgfpathcurveto{\pgfqpoint{8.134643in}{-5.470077in}}{\pgfqpoint{8.130985in}{-5.478909in}}{\pgfqpoint{8.130985in}{-5.488118in}}%
\pgfpathcurveto{\pgfqpoint{8.130985in}{-5.497326in}}{\pgfqpoint{8.134643in}{-5.506159in}}{\pgfqpoint{8.141155in}{-5.512670in}}%
\pgfpathcurveto{\pgfqpoint{8.147666in}{-5.519181in}}{\pgfqpoint{8.156499in}{-5.522840in}}{\pgfqpoint{8.165707in}{-5.522840in}}%
\pgfpathlineto{\pgfqpoint{8.165707in}{-5.522840in}}%
\pgfpathclose%
\pgfusepath{stroke,fill}%
\end{pgfscope}%
\begin{pgfscope}%
\pgfpathrectangle{\pgfqpoint{0.050000in}{0.050000in}}{\pgfqpoint{2.419000in}{2.419000in}}%
\pgfusepath{clip}%
\pgfsetbuttcap%
\pgfsetroundjoin%
\definecolor{currentfill}{rgb}{0.200000,0.133333,0.533333}%
\pgfsetfillcolor{currentfill}%
\pgfsetfillopacity{0.784469}%
\pgfsetlinewidth{1.003750pt}%
\definecolor{currentstroke}{rgb}{0.200000,0.133333,0.533333}%
\pgfsetstrokecolor{currentstroke}%
\pgfsetstrokeopacity{0.784469}%
\pgfsetdash{}{0pt}%
\pgfpathmoveto{\pgfqpoint{4.599719in}{-6.002873in}}%
\pgfpathcurveto{\pgfqpoint{4.608927in}{-6.002873in}}{\pgfqpoint{4.617760in}{-5.999214in}}{\pgfqpoint{4.624271in}{-5.992703in}}%
\pgfpathcurveto{\pgfqpoint{4.630782in}{-5.986191in}}{\pgfqpoint{4.634441in}{-5.977359in}}{\pgfqpoint{4.634441in}{-5.968150in}}%
\pgfpathcurveto{\pgfqpoint{4.634441in}{-5.958942in}}{\pgfqpoint{4.630782in}{-5.950109in}}{\pgfqpoint{4.624271in}{-5.943598in}}%
\pgfpathcurveto{\pgfqpoint{4.617760in}{-5.937087in}}{\pgfqpoint{4.608927in}{-5.933428in}}{\pgfqpoint{4.599719in}{-5.933428in}}%
\pgfpathcurveto{\pgfqpoint{4.590510in}{-5.933428in}}{\pgfqpoint{4.581678in}{-5.937087in}}{\pgfqpoint{4.575166in}{-5.943598in}}%
\pgfpathcurveto{\pgfqpoint{4.568655in}{-5.950109in}}{\pgfqpoint{4.564997in}{-5.958942in}}{\pgfqpoint{4.564997in}{-5.968150in}}%
\pgfpathcurveto{\pgfqpoint{4.564997in}{-5.977359in}}{\pgfqpoint{4.568655in}{-5.986191in}}{\pgfqpoint{4.575166in}{-5.992703in}}%
\pgfpathcurveto{\pgfqpoint{4.581678in}{-5.999214in}}{\pgfqpoint{4.590510in}{-6.002873in}}{\pgfqpoint{4.599719in}{-6.002873in}}%
\pgfpathlineto{\pgfqpoint{4.599719in}{-6.002873in}}%
\pgfpathclose%
\pgfusepath{stroke,fill}%
\end{pgfscope}%
\begin{pgfscope}%
\pgfpathrectangle{\pgfqpoint{0.050000in}{0.050000in}}{\pgfqpoint{2.419000in}{2.419000in}}%
\pgfusepath{clip}%
\pgfsetbuttcap%
\pgfsetroundjoin%
\definecolor{currentfill}{rgb}{0.200000,0.133333,0.533333}%
\pgfsetfillcolor{currentfill}%
\pgfsetfillopacity{0.808856}%
\pgfsetlinewidth{1.003750pt}%
\definecolor{currentstroke}{rgb}{0.200000,0.133333,0.533333}%
\pgfsetstrokecolor{currentstroke}%
\pgfsetstrokeopacity{0.808856}%
\pgfsetdash{}{0pt}%
\pgfpathmoveto{\pgfqpoint{0.786532in}{-6.516182in}}%
\pgfpathcurveto{\pgfqpoint{0.795740in}{-6.516182in}}{\pgfqpoint{0.804573in}{-6.512523in}}{\pgfqpoint{0.811084in}{-6.506012in}}%
\pgfpathcurveto{\pgfqpoint{0.817596in}{-6.499501in}}{\pgfqpoint{0.821254in}{-6.490668in}}{\pgfqpoint{0.821254in}{-6.481460in}}%
\pgfpathcurveto{\pgfqpoint{0.821254in}{-6.472251in}}{\pgfqpoint{0.817596in}{-6.463419in}}{\pgfqpoint{0.811084in}{-6.456907in}}%
\pgfpathcurveto{\pgfqpoint{0.804573in}{-6.450396in}}{\pgfqpoint{0.795740in}{-6.446737in}}{\pgfqpoint{0.786532in}{-6.446737in}}%
\pgfpathcurveto{\pgfqpoint{0.777324in}{-6.446737in}}{\pgfqpoint{0.768491in}{-6.450396in}}{\pgfqpoint{0.761980in}{-6.456907in}}%
\pgfpathcurveto{\pgfqpoint{0.755468in}{-6.463419in}}{\pgfqpoint{0.751810in}{-6.472251in}}{\pgfqpoint{0.751810in}{-6.481460in}}%
\pgfpathcurveto{\pgfqpoint{0.751810in}{-6.490668in}}{\pgfqpoint{0.755468in}{-6.499501in}}{\pgfqpoint{0.761980in}{-6.506012in}}%
\pgfpathcurveto{\pgfqpoint{0.768491in}{-6.512523in}}{\pgfqpoint{0.777324in}{-6.516182in}}{\pgfqpoint{0.786532in}{-6.516182in}}%
\pgfpathlineto{\pgfqpoint{0.786532in}{-6.516182in}}%
\pgfpathclose%
\pgfusepath{stroke,fill}%
\end{pgfscope}%
\begin{pgfscope}%
\pgfpathrectangle{\pgfqpoint{0.050000in}{0.050000in}}{\pgfqpoint{2.419000in}{2.419000in}}%
\pgfusepath{clip}%
\pgfsetbuttcap%
\pgfsetroundjoin%
\definecolor{currentfill}{rgb}{0.200000,0.133333,0.533333}%
\pgfsetfillcolor{currentfill}%
\pgfsetfillopacity{0.808856}%
\pgfsetlinewidth{1.003750pt}%
\definecolor{currentstroke}{rgb}{0.200000,0.133333,0.533333}%
\pgfsetstrokecolor{currentstroke}%
\pgfsetstrokeopacity{0.808856}%
\pgfsetdash{}{0pt}%
\pgfpathmoveto{\pgfqpoint{10.455031in}{-6.516182in}}%
\pgfpathcurveto{\pgfqpoint{10.464240in}{-6.516182in}}{\pgfqpoint{10.473072in}{-6.512523in}}{\pgfqpoint{10.479584in}{-6.506012in}}%
\pgfpathcurveto{\pgfqpoint{10.486095in}{-6.499501in}}{\pgfqpoint{10.489753in}{-6.490668in}}{\pgfqpoint{10.489753in}{-6.481460in}}%
\pgfpathcurveto{\pgfqpoint{10.489753in}{-6.472251in}}{\pgfqpoint{10.486095in}{-6.463419in}}{\pgfqpoint{10.479584in}{-6.456907in}}%
\pgfpathcurveto{\pgfqpoint{10.473072in}{-6.450396in}}{\pgfqpoint{10.464240in}{-6.446737in}}{\pgfqpoint{10.455031in}{-6.446737in}}%
\pgfpathcurveto{\pgfqpoint{10.445823in}{-6.446737in}}{\pgfqpoint{10.436990in}{-6.450396in}}{\pgfqpoint{10.430479in}{-6.456907in}}%
\pgfpathcurveto{\pgfqpoint{10.423968in}{-6.463419in}}{\pgfqpoint{10.420309in}{-6.472251in}}{\pgfqpoint{10.420309in}{-6.481460in}}%
\pgfpathcurveto{\pgfqpoint{10.420309in}{-6.490668in}}{\pgfqpoint{10.423968in}{-6.499501in}}{\pgfqpoint{10.430479in}{-6.506012in}}%
\pgfpathcurveto{\pgfqpoint{10.436990in}{-6.512523in}}{\pgfqpoint{10.445823in}{-6.516182in}}{\pgfqpoint{10.455031in}{-6.516182in}}%
\pgfpathlineto{\pgfqpoint{10.455031in}{-6.516182in}}%
\pgfpathclose%
\pgfusepath{stroke,fill}%
\end{pgfscope}%
\begin{pgfscope}%
\pgfpathrectangle{\pgfqpoint{0.050000in}{0.050000in}}{\pgfqpoint{2.419000in}{2.419000in}}%
\pgfusepath{clip}%
\pgfsetbuttcap%
\pgfsetroundjoin%
\definecolor{currentfill}{rgb}{0.200000,0.133333,0.533333}%
\pgfsetfillcolor{currentfill}%
\pgfsetfillopacity{0.834994}%
\pgfsetlinewidth{1.003750pt}%
\definecolor{currentstroke}{rgb}{0.200000,0.133333,0.533333}%
\pgfsetstrokecolor{currentstroke}%
\pgfsetstrokeopacity{0.834994}%
\pgfsetdash{}{0pt}%
\pgfpathmoveto{\pgfqpoint{6.715167in}{-7.066352in}}%
\pgfpathcurveto{\pgfqpoint{6.724376in}{-7.066352in}}{\pgfqpoint{6.733208in}{-7.062693in}}{\pgfqpoint{6.739719in}{-7.056182in}}%
\pgfpathcurveto{\pgfqpoint{6.746231in}{-7.049671in}}{\pgfqpoint{6.749889in}{-7.040838in}}{\pgfqpoint{6.749889in}{-7.031630in}}%
\pgfpathcurveto{\pgfqpoint{6.749889in}{-7.022421in}}{\pgfqpoint{6.746231in}{-7.013589in}}{\pgfqpoint{6.739719in}{-7.007077in}}%
\pgfpathcurveto{\pgfqpoint{6.733208in}{-7.000566in}}{\pgfqpoint{6.724376in}{-6.996907in}}{\pgfqpoint{6.715167in}{-6.996907in}}%
\pgfpathcurveto{\pgfqpoint{6.705959in}{-6.996907in}}{\pgfqpoint{6.697126in}{-7.000566in}}{\pgfqpoint{6.690615in}{-7.007077in}}%
\pgfpathcurveto{\pgfqpoint{6.684103in}{-7.013589in}}{\pgfqpoint{6.680445in}{-7.022421in}}{\pgfqpoint{6.680445in}{-7.031630in}}%
\pgfpathcurveto{\pgfqpoint{6.680445in}{-7.040838in}}{\pgfqpoint{6.684103in}{-7.049671in}}{\pgfqpoint{6.690615in}{-7.056182in}}%
\pgfpathcurveto{\pgfqpoint{6.697126in}{-7.062693in}}{\pgfqpoint{6.705959in}{-7.066352in}}{\pgfqpoint{6.715167in}{-7.066352in}}%
\pgfpathlineto{\pgfqpoint{6.715167in}{-7.066352in}}%
\pgfpathclose%
\pgfusepath{stroke,fill}%
\end{pgfscope}%
\begin{pgfscope}%
\pgfpathrectangle{\pgfqpoint{0.050000in}{0.050000in}}{\pgfqpoint{2.419000in}{2.419000in}}%
\pgfusepath{clip}%
\pgfsetbuttcap%
\pgfsetroundjoin%
\definecolor{currentfill}{rgb}{0.200000,0.133333,0.533333}%
\pgfsetfillcolor{currentfill}%
\pgfsetfillopacity{0.863079}%
\pgfsetlinewidth{1.003750pt}%
\definecolor{currentstroke}{rgb}{0.200000,0.133333,0.533333}%
\pgfsetstrokecolor{currentstroke}%
\pgfsetstrokeopacity{0.863079}%
\pgfsetdash{}{0pt}%
\pgfpathmoveto{\pgfqpoint{2.696741in}{-7.657501in}}%
\pgfpathcurveto{\pgfqpoint{2.705949in}{-7.657501in}}{\pgfqpoint{2.714782in}{-7.653843in}}{\pgfqpoint{2.721293in}{-7.647331in}}%
\pgfpathcurveto{\pgfqpoint{2.727804in}{-7.640820in}}{\pgfqpoint{2.731463in}{-7.631987in}}{\pgfqpoint{2.731463in}{-7.622779in}}%
\pgfpathcurveto{\pgfqpoint{2.731463in}{-7.613571in}}{\pgfqpoint{2.727804in}{-7.604738in}}{\pgfqpoint{2.721293in}{-7.598227in}}%
\pgfpathcurveto{\pgfqpoint{2.714782in}{-7.591715in}}{\pgfqpoint{2.705949in}{-7.588057in}}{\pgfqpoint{2.696741in}{-7.588057in}}%
\pgfpathcurveto{\pgfqpoint{2.687532in}{-7.588057in}}{\pgfqpoint{2.678700in}{-7.591715in}}{\pgfqpoint{2.672189in}{-7.598227in}}%
\pgfpathcurveto{\pgfqpoint{2.665677in}{-7.604738in}}{\pgfqpoint{2.662019in}{-7.613571in}}{\pgfqpoint{2.662019in}{-7.622779in}}%
\pgfpathcurveto{\pgfqpoint{2.662019in}{-7.631987in}}{\pgfqpoint{2.665677in}{-7.640820in}}{\pgfqpoint{2.672189in}{-7.647331in}}%
\pgfpathcurveto{\pgfqpoint{2.678700in}{-7.653843in}}{\pgfqpoint{2.687532in}{-7.657501in}}{\pgfqpoint{2.696741in}{-7.657501in}}%
\pgfpathlineto{\pgfqpoint{2.696741in}{-7.657501in}}%
\pgfpathclose%
\pgfusepath{stroke,fill}%
\end{pgfscope}%
\begin{pgfscope}%
\pgfpathrectangle{\pgfqpoint{0.050000in}{0.050000in}}{\pgfqpoint{2.419000in}{2.419000in}}%
\pgfusepath{clip}%
\pgfsetbuttcap%
\pgfsetroundjoin%
\definecolor{currentfill}{rgb}{0.200000,0.133333,0.533333}%
\pgfsetfillcolor{currentfill}%
\pgfsetfillopacity{0.893337}%
\pgfsetlinewidth{1.003750pt}%
\definecolor{currentstroke}{rgb}{0.200000,0.133333,0.533333}%
\pgfsetstrokecolor{currentstroke}%
\pgfsetstrokeopacity{0.893337}%
\pgfsetdash{}{0pt}%
\pgfpathmoveto{\pgfqpoint{9.157942in}{-8.294385in}}%
\pgfpathcurveto{\pgfqpoint{9.167151in}{-8.294385in}}{\pgfqpoint{9.175983in}{-8.290727in}}{\pgfqpoint{9.182495in}{-8.284215in}}%
\pgfpathcurveto{\pgfqpoint{9.189006in}{-8.277704in}}{\pgfqpoint{9.192665in}{-8.268871in}}{\pgfqpoint{9.192665in}{-8.259663in}}%
\pgfpathcurveto{\pgfqpoint{9.192665in}{-8.250455in}}{\pgfqpoint{9.189006in}{-8.241622in}}{\pgfqpoint{9.182495in}{-8.235111in}}%
\pgfpathcurveto{\pgfqpoint{9.175983in}{-8.228599in}}{\pgfqpoint{9.167151in}{-8.224941in}}{\pgfqpoint{9.157942in}{-8.224941in}}%
\pgfpathcurveto{\pgfqpoint{9.148734in}{-8.224941in}}{\pgfqpoint{9.139901in}{-8.228599in}}{\pgfqpoint{9.133390in}{-8.235111in}}%
\pgfpathcurveto{\pgfqpoint{9.126879in}{-8.241622in}}{\pgfqpoint{9.123220in}{-8.250455in}}{\pgfqpoint{9.123220in}{-8.259663in}}%
\pgfpathcurveto{\pgfqpoint{9.123220in}{-8.268871in}}{\pgfqpoint{9.126879in}{-8.277704in}}{\pgfqpoint{9.133390in}{-8.284215in}}%
\pgfpathcurveto{\pgfqpoint{9.139901in}{-8.290727in}}{\pgfqpoint{9.148734in}{-8.294385in}}{\pgfqpoint{9.157942in}{-8.294385in}}%
\pgfpathlineto{\pgfqpoint{9.157942in}{-8.294385in}}%
\pgfpathclose%
\pgfusepath{stroke,fill}%
\end{pgfscope}%
\begin{pgfscope}%
\pgfpathrectangle{\pgfqpoint{0.050000in}{0.050000in}}{\pgfqpoint{2.419000in}{2.419000in}}%
\pgfusepath{clip}%
\pgfsetbuttcap%
\pgfsetroundjoin%
\definecolor{currentfill}{rgb}{0.200000,0.133333,0.533333}%
\pgfsetfillcolor{currentfill}%
\pgfsetfillopacity{0.926030}%
\pgfsetlinewidth{1.003750pt}%
\definecolor{currentstroke}{rgb}{0.200000,0.133333,0.533333}%
\pgfsetstrokecolor{currentstroke}%
\pgfsetstrokeopacity{0.926030}%
\pgfsetdash{}{0pt}%
\pgfpathmoveto{\pgfqpoint{4.914413in}{-8.982525in}}%
\pgfpathcurveto{\pgfqpoint{4.923622in}{-8.982525in}}{\pgfqpoint{4.932454in}{-8.978867in}}{\pgfqpoint{4.938966in}{-8.972355in}}%
\pgfpathcurveto{\pgfqpoint{4.945477in}{-8.965844in}}{\pgfqpoint{4.949135in}{-8.957011in}}{\pgfqpoint{4.949135in}{-8.947803in}}%
\pgfpathcurveto{\pgfqpoint{4.949135in}{-8.938594in}}{\pgfqpoint{4.945477in}{-8.929762in}}{\pgfqpoint{4.938966in}{-8.923251in}}%
\pgfpathcurveto{\pgfqpoint{4.932454in}{-8.916739in}}{\pgfqpoint{4.923622in}{-8.913081in}}{\pgfqpoint{4.914413in}{-8.913081in}}%
\pgfpathcurveto{\pgfqpoint{4.905205in}{-8.913081in}}{\pgfqpoint{4.896372in}{-8.916739in}}{\pgfqpoint{4.889861in}{-8.923251in}}%
\pgfpathcurveto{\pgfqpoint{4.883350in}{-8.929762in}}{\pgfqpoint{4.879691in}{-8.938594in}}{\pgfqpoint{4.879691in}{-8.947803in}}%
\pgfpathcurveto{\pgfqpoint{4.879691in}{-8.957011in}}{\pgfqpoint{4.883350in}{-8.965844in}}{\pgfqpoint{4.889861in}{-8.972355in}}%
\pgfpathcurveto{\pgfqpoint{4.896372in}{-8.978867in}}{\pgfqpoint{4.905205in}{-8.982525in}}{\pgfqpoint{4.914413in}{-8.982525in}}%
\pgfpathlineto{\pgfqpoint{4.914413in}{-8.982525in}}%
\pgfpathclose%
\pgfusepath{stroke,fill}%
\end{pgfscope}%
\begin{pgfscope}%
\pgfpathrectangle{\pgfqpoint{0.050000in}{0.050000in}}{\pgfqpoint{2.419000in}{2.419000in}}%
\pgfusepath{clip}%
\pgfsetbuttcap%
\pgfsetroundjoin%
\definecolor{currentfill}{rgb}{0.200000,0.133333,0.533333}%
\pgfsetfillcolor{currentfill}%
\pgfsetlinewidth{1.003750pt}%
\definecolor{currentstroke}{rgb}{0.200000,0.133333,0.533333}%
\pgfsetstrokecolor{currentstroke}%
\pgfsetdash{}{0pt}%
\pgfpathmoveto{\pgfqpoint{7.520279in}{-10.539489in}}%
\pgfpathcurveto{\pgfqpoint{7.529488in}{-10.539489in}}{\pgfqpoint{7.538320in}{-10.535830in}}{\pgfqpoint{7.544832in}{-10.529319in}}%
\pgfpathcurveto{\pgfqpoint{7.551343in}{-10.522807in}}{\pgfqpoint{7.555001in}{-10.513975in}}{\pgfqpoint{7.555001in}{-10.504766in}}%
\pgfpathcurveto{\pgfqpoint{7.555001in}{-10.495558in}}{\pgfqpoint{7.551343in}{-10.486725in}}{\pgfqpoint{7.544832in}{-10.480214in}}%
\pgfpathcurveto{\pgfqpoint{7.538320in}{-10.473703in}}{\pgfqpoint{7.529488in}{-10.470044in}}{\pgfqpoint{7.520279in}{-10.470044in}}%
\pgfpathcurveto{\pgfqpoint{7.511071in}{-10.470044in}}{\pgfqpoint{7.502238in}{-10.473703in}}{\pgfqpoint{7.495727in}{-10.480214in}}%
\pgfpathcurveto{\pgfqpoint{7.489216in}{-10.486725in}}{\pgfqpoint{7.485557in}{-10.495558in}}{\pgfqpoint{7.485557in}{-10.504766in}}%
\pgfpathcurveto{\pgfqpoint{7.485557in}{-10.513975in}}{\pgfqpoint{7.489216in}{-10.522807in}}{\pgfqpoint{7.495727in}{-10.529319in}}%
\pgfpathcurveto{\pgfqpoint{7.502238in}{-10.535830in}}{\pgfqpoint{7.511071in}{-10.539489in}}{\pgfqpoint{7.520279in}{-10.539489in}}%
\pgfpathlineto{\pgfqpoint{7.520279in}{-10.539489in}}%
\pgfpathclose%
\pgfusepath{stroke,fill}%
\end{pgfscope}%
\begin{pgfscope}%
\pgfpathrectangle{\pgfqpoint{0.050000in}{0.050000in}}{\pgfqpoint{2.419000in}{2.419000in}}%
\pgfusepath{clip}%
\pgfsetbuttcap%
\pgfsetroundjoin%
\definecolor{currentfill}{rgb}{0.866667,0.800000,0.466667}%
\pgfsetfillcolor{currentfill}%
\pgfsetfillopacity{0.300000}%
\pgfsetlinewidth{1.003750pt}%
\definecolor{currentstroke}{rgb}{0.866667,0.800000,0.466667}%
\pgfsetstrokecolor{currentstroke}%
\pgfsetstrokeopacity{0.300000}%
\pgfsetdash{}{0pt}%
\pgfpathmoveto{\pgfqpoint{1.542643in}{4.230615in}}%
\pgfpathcurveto{\pgfqpoint{1.551852in}{4.230615in}}{\pgfqpoint{1.560684in}{4.234274in}}{\pgfqpoint{1.567196in}{4.240785in}}%
\pgfpathcurveto{\pgfqpoint{1.573707in}{4.247296in}}{\pgfqpoint{1.577366in}{4.256129in}}{\pgfqpoint{1.577366in}{4.265337in}}%
\pgfpathcurveto{\pgfqpoint{1.577366in}{4.274546in}}{\pgfqpoint{1.573707in}{4.283378in}}{\pgfqpoint{1.567196in}{4.289890in}}%
\pgfpathcurveto{\pgfqpoint{1.560684in}{4.296401in}}{\pgfqpoint{1.551852in}{4.300059in}}{\pgfqpoint{1.542643in}{4.300059in}}%
\pgfpathcurveto{\pgfqpoint{1.533435in}{4.300059in}}{\pgfqpoint{1.524603in}{4.296401in}}{\pgfqpoint{1.518091in}{4.289890in}}%
\pgfpathcurveto{\pgfqpoint{1.511580in}{4.283378in}}{\pgfqpoint{1.507921in}{4.274546in}}{\pgfqpoint{1.507921in}{4.265337in}}%
\pgfpathcurveto{\pgfqpoint{1.507921in}{4.256129in}}{\pgfqpoint{1.511580in}{4.247296in}}{\pgfqpoint{1.518091in}{4.240785in}}%
\pgfpathcurveto{\pgfqpoint{1.524603in}{4.234274in}}{\pgfqpoint{1.533435in}{4.230615in}}{\pgfqpoint{1.542643in}{4.230615in}}%
\pgfpathlineto{\pgfqpoint{1.542643in}{4.230615in}}%
\pgfpathclose%
\pgfusepath{stroke,fill}%
\end{pgfscope}%
\begin{pgfscope}%
\pgfpathrectangle{\pgfqpoint{0.050000in}{0.050000in}}{\pgfqpoint{2.419000in}{2.419000in}}%
\pgfusepath{clip}%
\pgfsetbuttcap%
\pgfsetroundjoin%
\definecolor{currentfill}{rgb}{0.866667,0.800000,0.466667}%
\pgfsetfillcolor{currentfill}%
\pgfsetfillopacity{0.304985}%
\pgfsetlinewidth{1.003750pt}%
\definecolor{currentstroke}{rgb}{0.866667,0.800000,0.466667}%
\pgfsetstrokecolor{currentstroke}%
\pgfsetstrokeopacity{0.304985}%
\pgfsetdash{}{0pt}%
\pgfpathmoveto{\pgfqpoint{2.125637in}{4.129567in}}%
\pgfpathcurveto{\pgfqpoint{2.134846in}{4.129567in}}{\pgfqpoint{2.143678in}{4.133226in}}{\pgfqpoint{2.150189in}{4.139737in}}%
\pgfpathcurveto{\pgfqpoint{2.156701in}{4.146249in}}{\pgfqpoint{2.160359in}{4.155081in}}{\pgfqpoint{2.160359in}{4.164290in}}%
\pgfpathcurveto{\pgfqpoint{2.160359in}{4.173498in}}{\pgfqpoint{2.156701in}{4.182331in}}{\pgfqpoint{2.150189in}{4.188842in}}%
\pgfpathcurveto{\pgfqpoint{2.143678in}{4.195353in}}{\pgfqpoint{2.134846in}{4.199012in}}{\pgfqpoint{2.125637in}{4.199012in}}%
\pgfpathcurveto{\pgfqpoint{2.116429in}{4.199012in}}{\pgfqpoint{2.107596in}{4.195353in}}{\pgfqpoint{2.101085in}{4.188842in}}%
\pgfpathcurveto{\pgfqpoint{2.094573in}{4.182331in}}{\pgfqpoint{2.090915in}{4.173498in}}{\pgfqpoint{2.090915in}{4.164290in}}%
\pgfpathcurveto{\pgfqpoint{2.090915in}{4.155081in}}{\pgfqpoint{2.094573in}{4.146249in}}{\pgfqpoint{2.101085in}{4.139737in}}%
\pgfpathcurveto{\pgfqpoint{2.107596in}{4.133226in}}{\pgfqpoint{2.116429in}{4.129567in}}{\pgfqpoint{2.125637in}{4.129567in}}%
\pgfpathlineto{\pgfqpoint{2.125637in}{4.129567in}}%
\pgfpathclose%
\pgfusepath{stroke,fill}%
\end{pgfscope}%
\begin{pgfscope}%
\pgfpathrectangle{\pgfqpoint{0.050000in}{0.050000in}}{\pgfqpoint{2.419000in}{2.419000in}}%
\pgfusepath{clip}%
\pgfsetbuttcap%
\pgfsetroundjoin%
\definecolor{currentfill}{rgb}{0.866667,0.800000,0.466667}%
\pgfsetfillcolor{currentfill}%
\pgfsetfillopacity{0.310194}%
\pgfsetlinewidth{1.003750pt}%
\definecolor{currentstroke}{rgb}{0.866667,0.800000,0.466667}%
\pgfsetstrokecolor{currentstroke}%
\pgfsetstrokeopacity{0.310194}%
\pgfsetdash{}{0pt}%
\pgfpathmoveto{\pgfqpoint{2.734959in}{4.023957in}}%
\pgfpathcurveto{\pgfqpoint{2.744167in}{4.023957in}}{\pgfqpoint{2.753000in}{4.027615in}}{\pgfqpoint{2.759511in}{4.034127in}}%
\pgfpathcurveto{\pgfqpoint{2.766023in}{4.040638in}}{\pgfqpoint{2.769681in}{4.049470in}}{\pgfqpoint{2.769681in}{4.058679in}}%
\pgfpathcurveto{\pgfqpoint{2.769681in}{4.067887in}}{\pgfqpoint{2.766023in}{4.076720in}}{\pgfqpoint{2.759511in}{4.083231in}}%
\pgfpathcurveto{\pgfqpoint{2.753000in}{4.089743in}}{\pgfqpoint{2.744167in}{4.093401in}}{\pgfqpoint{2.734959in}{4.093401in}}%
\pgfpathcurveto{\pgfqpoint{2.725751in}{4.093401in}}{\pgfqpoint{2.716918in}{4.089743in}}{\pgfqpoint{2.710407in}{4.083231in}}%
\pgfpathcurveto{\pgfqpoint{2.703895in}{4.076720in}}{\pgfqpoint{2.700237in}{4.067887in}}{\pgfqpoint{2.700237in}{4.058679in}}%
\pgfpathcurveto{\pgfqpoint{2.700237in}{4.049470in}}{\pgfqpoint{2.703895in}{4.040638in}}{\pgfqpoint{2.710407in}{4.034127in}}%
\pgfpathcurveto{\pgfqpoint{2.716918in}{4.027615in}}{\pgfqpoint{2.725751in}{4.023957in}}{\pgfqpoint{2.734959in}{4.023957in}}%
\pgfpathlineto{\pgfqpoint{2.734959in}{4.023957in}}%
\pgfpathclose%
\pgfusepath{stroke,fill}%
\end{pgfscope}%
\begin{pgfscope}%
\pgfpathrectangle{\pgfqpoint{0.050000in}{0.050000in}}{\pgfqpoint{2.419000in}{2.419000in}}%
\pgfusepath{clip}%
\pgfsetbuttcap%
\pgfsetroundjoin%
\definecolor{currentfill}{rgb}{0.866667,0.800000,0.466667}%
\pgfsetfillcolor{currentfill}%
\pgfsetfillopacity{0.312888}%
\pgfsetlinewidth{1.003750pt}%
\definecolor{currentstroke}{rgb}{0.866667,0.800000,0.466667}%
\pgfsetstrokecolor{currentstroke}%
\pgfsetstrokeopacity{0.312888}%
\pgfsetdash{}{0pt}%
\pgfpathmoveto{\pgfqpoint{1.457402in}{3.969342in}}%
\pgfpathcurveto{\pgfqpoint{1.466611in}{3.969342in}}{\pgfqpoint{1.475443in}{3.973001in}}{\pgfqpoint{1.481955in}{3.979512in}}%
\pgfpathcurveto{\pgfqpoint{1.488466in}{3.986024in}}{\pgfqpoint{1.492125in}{3.994856in}}{\pgfqpoint{1.492125in}{4.004065in}}%
\pgfpathcurveto{\pgfqpoint{1.492125in}{4.013273in}}{\pgfqpoint{1.488466in}{4.022106in}}{\pgfqpoint{1.481955in}{4.028617in}}%
\pgfpathcurveto{\pgfqpoint{1.475443in}{4.035128in}}{\pgfqpoint{1.466611in}{4.038787in}}{\pgfqpoint{1.457402in}{4.038787in}}%
\pgfpathcurveto{\pgfqpoint{1.448194in}{4.038787in}}{\pgfqpoint{1.439361in}{4.035128in}}{\pgfqpoint{1.432850in}{4.028617in}}%
\pgfpathcurveto{\pgfqpoint{1.426339in}{4.022106in}}{\pgfqpoint{1.422680in}{4.013273in}}{\pgfqpoint{1.422680in}{4.004065in}}%
\pgfpathcurveto{\pgfqpoint{1.422680in}{3.994856in}}{\pgfqpoint{1.426339in}{3.986024in}}{\pgfqpoint{1.432850in}{3.979512in}}%
\pgfpathcurveto{\pgfqpoint{1.439361in}{3.973001in}}{\pgfqpoint{1.448194in}{3.969342in}}{\pgfqpoint{1.457402in}{3.969342in}}%
\pgfpathlineto{\pgfqpoint{1.457402in}{3.969342in}}%
\pgfpathclose%
\pgfusepath{stroke,fill}%
\end{pgfscope}%
\begin{pgfscope}%
\pgfpathrectangle{\pgfqpoint{0.050000in}{0.050000in}}{\pgfqpoint{2.419000in}{2.419000in}}%
\pgfusepath{clip}%
\pgfsetbuttcap%
\pgfsetroundjoin%
\definecolor{currentfill}{rgb}{0.866667,0.800000,0.466667}%
\pgfsetfillcolor{currentfill}%
\pgfsetfillopacity{0.315644}%
\pgfsetlinewidth{1.003750pt}%
\definecolor{currentstroke}{rgb}{0.866667,0.800000,0.466667}%
\pgfsetstrokecolor{currentstroke}%
\pgfsetstrokeopacity{0.315644}%
\pgfsetdash{}{0pt}%
\pgfpathmoveto{\pgfqpoint{3.372434in}{3.913466in}}%
\pgfpathcurveto{\pgfqpoint{3.381642in}{3.913466in}}{\pgfqpoint{3.390475in}{3.917125in}}{\pgfqpoint{3.396986in}{3.923636in}}%
\pgfpathcurveto{\pgfqpoint{3.403498in}{3.930148in}}{\pgfqpoint{3.407156in}{3.938980in}}{\pgfqpoint{3.407156in}{3.948189in}}%
\pgfpathcurveto{\pgfqpoint{3.407156in}{3.957397in}}{\pgfqpoint{3.403498in}{3.966229in}}{\pgfqpoint{3.396986in}{3.972741in}}%
\pgfpathcurveto{\pgfqpoint{3.390475in}{3.979252in}}{\pgfqpoint{3.381642in}{3.982911in}}{\pgfqpoint{3.372434in}{3.982911in}}%
\pgfpathcurveto{\pgfqpoint{3.363225in}{3.982911in}}{\pgfqpoint{3.354393in}{3.979252in}}{\pgfqpoint{3.347882in}{3.972741in}}%
\pgfpathcurveto{\pgfqpoint{3.341370in}{3.966229in}}{\pgfqpoint{3.337712in}{3.957397in}}{\pgfqpoint{3.337712in}{3.948189in}}%
\pgfpathcurveto{\pgfqpoint{3.337712in}{3.938980in}}{\pgfqpoint{3.341370in}{3.930148in}}{\pgfqpoint{3.347882in}{3.923636in}}%
\pgfpathcurveto{\pgfqpoint{3.354393in}{3.917125in}}{\pgfqpoint{3.363225in}{3.913466in}}{\pgfqpoint{3.372434in}{3.913466in}}%
\pgfpathlineto{\pgfqpoint{3.372434in}{3.913466in}}%
\pgfpathclose%
\pgfusepath{stroke,fill}%
\end{pgfscope}%
\begin{pgfscope}%
\pgfpathrectangle{\pgfqpoint{0.050000in}{0.050000in}}{\pgfqpoint{2.419000in}{2.419000in}}%
\pgfusepath{clip}%
\pgfsetbuttcap%
\pgfsetroundjoin%
\definecolor{currentfill}{rgb}{0.866667,0.800000,0.466667}%
\pgfsetfillcolor{currentfill}%
\pgfsetfillopacity{0.318465}%
\pgfsetlinewidth{1.003750pt}%
\definecolor{currentstroke}{rgb}{0.866667,0.800000,0.466667}%
\pgfsetstrokecolor{currentstroke}%
\pgfsetstrokeopacity{0.318465}%
\pgfsetdash{}{0pt}%
\pgfpathmoveto{\pgfqpoint{2.072468in}{3.856284in}}%
\pgfpathcurveto{\pgfqpoint{2.081676in}{3.856284in}}{\pgfqpoint{2.090509in}{3.859943in}}{\pgfqpoint{2.097020in}{3.866454in}}%
\pgfpathcurveto{\pgfqpoint{2.103531in}{3.872966in}}{\pgfqpoint{2.107190in}{3.881798in}}{\pgfqpoint{2.107190in}{3.891007in}}%
\pgfpathcurveto{\pgfqpoint{2.107190in}{3.900215in}}{\pgfqpoint{2.103531in}{3.909047in}}{\pgfqpoint{2.097020in}{3.915559in}}%
\pgfpathcurveto{\pgfqpoint{2.090509in}{3.922070in}}{\pgfqpoint{2.081676in}{3.925729in}}{\pgfqpoint{2.072468in}{3.925729in}}%
\pgfpathcurveto{\pgfqpoint{2.063259in}{3.925729in}}{\pgfqpoint{2.054427in}{3.922070in}}{\pgfqpoint{2.047915in}{3.915559in}}%
\pgfpathcurveto{\pgfqpoint{2.041404in}{3.909047in}}{\pgfqpoint{2.037746in}{3.900215in}}{\pgfqpoint{2.037746in}{3.891007in}}%
\pgfpathcurveto{\pgfqpoint{2.037746in}{3.881798in}}{\pgfqpoint{2.041404in}{3.872966in}}{\pgfqpoint{2.047915in}{3.866454in}}%
\pgfpathcurveto{\pgfqpoint{2.054427in}{3.859943in}}{\pgfqpoint{2.063259in}{3.856284in}}{\pgfqpoint{2.072468in}{3.856284in}}%
\pgfpathlineto{\pgfqpoint{2.072468in}{3.856284in}}%
\pgfpathclose%
\pgfusepath{stroke,fill}%
\end{pgfscope}%
\begin{pgfscope}%
\pgfpathrectangle{\pgfqpoint{0.050000in}{0.050000in}}{\pgfqpoint{2.419000in}{2.419000in}}%
\pgfusepath{clip}%
\pgfsetbuttcap%
\pgfsetroundjoin%
\definecolor{currentfill}{rgb}{0.866667,0.800000,0.466667}%
\pgfsetfillcolor{currentfill}%
\pgfsetfillopacity{0.321353}%
\pgfsetlinewidth{1.003750pt}%
\definecolor{currentstroke}{rgb}{0.866667,0.800000,0.466667}%
\pgfsetstrokecolor{currentstroke}%
\pgfsetstrokeopacity{0.321353}%
\pgfsetdash{}{0pt}%
\pgfpathmoveto{\pgfqpoint{4.040059in}{3.797750in}}%
\pgfpathcurveto{\pgfqpoint{4.049267in}{3.797750in}}{\pgfqpoint{4.058100in}{3.801409in}}{\pgfqpoint{4.064611in}{3.807920in}}%
\pgfpathcurveto{\pgfqpoint{4.071123in}{3.814431in}}{\pgfqpoint{4.074781in}{3.823264in}}{\pgfqpoint{4.074781in}{3.832472in}}%
\pgfpathcurveto{\pgfqpoint{4.074781in}{3.841681in}}{\pgfqpoint{4.071123in}{3.850513in}}{\pgfqpoint{4.064611in}{3.857025in}}%
\pgfpathcurveto{\pgfqpoint{4.058100in}{3.863536in}}{\pgfqpoint{4.049267in}{3.867195in}}{\pgfqpoint{4.040059in}{3.867195in}}%
\pgfpathcurveto{\pgfqpoint{4.030850in}{3.867195in}}{\pgfqpoint{4.022018in}{3.863536in}}{\pgfqpoint{4.015507in}{3.857025in}}%
\pgfpathcurveto{\pgfqpoint{4.008995in}{3.850513in}}{\pgfqpoint{4.005337in}{3.841681in}}{\pgfqpoint{4.005337in}{3.832472in}}%
\pgfpathcurveto{\pgfqpoint{4.005337in}{3.823264in}}{\pgfqpoint{4.008995in}{3.814431in}}{\pgfqpoint{4.015507in}{3.807920in}}%
\pgfpathcurveto{\pgfqpoint{4.022018in}{3.801409in}}{\pgfqpoint{4.030850in}{3.797750in}}{\pgfqpoint{4.040059in}{3.797750in}}%
\pgfpathlineto{\pgfqpoint{4.040059in}{3.797750in}}%
\pgfpathclose%
\pgfusepath{stroke,fill}%
\end{pgfscope}%
\begin{pgfscope}%
\pgfpathrectangle{\pgfqpoint{0.050000in}{0.050000in}}{\pgfqpoint{2.419000in}{2.419000in}}%
\pgfusepath{clip}%
\pgfsetbuttcap%
\pgfsetroundjoin%
\definecolor{currentfill}{rgb}{0.866667,0.800000,0.466667}%
\pgfsetfillcolor{currentfill}%
\pgfsetfillopacity{0.321353}%
\pgfsetlinewidth{1.003750pt}%
\definecolor{currentstroke}{rgb}{0.866667,0.800000,0.466667}%
\pgfsetstrokecolor{currentstroke}%
\pgfsetstrokeopacity{0.321353}%
\pgfsetdash{}{0pt}%
\pgfpathmoveto{\pgfqpoint{0.741760in}{3.797750in}}%
\pgfpathcurveto{\pgfqpoint{0.750968in}{3.797750in}}{\pgfqpoint{0.759801in}{3.801409in}}{\pgfqpoint{0.766312in}{3.807920in}}%
\pgfpathcurveto{\pgfqpoint{0.772824in}{3.814431in}}{\pgfqpoint{0.776482in}{3.823264in}}{\pgfqpoint{0.776482in}{3.832472in}}%
\pgfpathcurveto{\pgfqpoint{0.776482in}{3.841681in}}{\pgfqpoint{0.772824in}{3.850513in}}{\pgfqpoint{0.766312in}{3.857025in}}%
\pgfpathcurveto{\pgfqpoint{0.759801in}{3.863536in}}{\pgfqpoint{0.750968in}{3.867195in}}{\pgfqpoint{0.741760in}{3.867195in}}%
\pgfpathcurveto{\pgfqpoint{0.732551in}{3.867195in}}{\pgfqpoint{0.723719in}{3.863536in}}{\pgfqpoint{0.717208in}{3.857025in}}%
\pgfpathcurveto{\pgfqpoint{0.710696in}{3.850513in}}{\pgfqpoint{0.707038in}{3.841681in}}{\pgfqpoint{0.707038in}{3.832472in}}%
\pgfpathcurveto{\pgfqpoint{0.707038in}{3.823264in}}{\pgfqpoint{0.710696in}{3.814431in}}{\pgfqpoint{0.717208in}{3.807920in}}%
\pgfpathcurveto{\pgfqpoint{0.723719in}{3.801409in}}{\pgfqpoint{0.732551in}{3.797750in}}{\pgfqpoint{0.741760in}{3.797750in}}%
\pgfpathlineto{\pgfqpoint{0.741760in}{3.797750in}}%
\pgfpathclose%
\pgfusepath{stroke,fill}%
\end{pgfscope}%
\begin{pgfscope}%
\pgfpathrectangle{\pgfqpoint{0.050000in}{0.050000in}}{\pgfqpoint{2.419000in}{2.419000in}}%
\pgfusepath{clip}%
\pgfsetbuttcap%
\pgfsetroundjoin%
\definecolor{currentfill}{rgb}{0.866667,0.800000,0.466667}%
\pgfsetfillcolor{currentfill}%
\pgfsetfillopacity{0.324309}%
\pgfsetlinewidth{1.003750pt}%
\definecolor{currentstroke}{rgb}{0.866667,0.800000,0.466667}%
\pgfsetstrokecolor{currentstroke}%
\pgfsetstrokeopacity{0.324309}%
\pgfsetdash{}{0pt}%
\pgfpathmoveto{\pgfqpoint{2.716972in}{3.737815in}}%
\pgfpathcurveto{\pgfqpoint{2.726180in}{3.737815in}}{\pgfqpoint{2.735013in}{3.741474in}}{\pgfqpoint{2.741524in}{3.747985in}}%
\pgfpathcurveto{\pgfqpoint{2.748035in}{3.754496in}}{\pgfqpoint{2.751694in}{3.763329in}}{\pgfqpoint{2.751694in}{3.772537in}}%
\pgfpathcurveto{\pgfqpoint{2.751694in}{3.781746in}}{\pgfqpoint{2.748035in}{3.790578in}}{\pgfqpoint{2.741524in}{3.797090in}}%
\pgfpathcurveto{\pgfqpoint{2.735013in}{3.803601in}}{\pgfqpoint{2.726180in}{3.807260in}}{\pgfqpoint{2.716972in}{3.807260in}}%
\pgfpathcurveto{\pgfqpoint{2.707763in}{3.807260in}}{\pgfqpoint{2.698931in}{3.803601in}}{\pgfqpoint{2.692419in}{3.797090in}}%
\pgfpathcurveto{\pgfqpoint{2.685908in}{3.790578in}}{\pgfqpoint{2.682249in}{3.781746in}}{\pgfqpoint{2.682249in}{3.772537in}}%
\pgfpathcurveto{\pgfqpoint{2.682249in}{3.763329in}}{\pgfqpoint{2.685908in}{3.754496in}}{\pgfqpoint{2.692419in}{3.747985in}}%
\pgfpathcurveto{\pgfqpoint{2.698931in}{3.741474in}}{\pgfqpoint{2.707763in}{3.737815in}}{\pgfqpoint{2.716972in}{3.737815in}}%
\pgfpathlineto{\pgfqpoint{2.716972in}{3.737815in}}%
\pgfpathclose%
\pgfusepath{stroke,fill}%
\end{pgfscope}%
\begin{pgfscope}%
\pgfpathrectangle{\pgfqpoint{0.050000in}{0.050000in}}{\pgfqpoint{2.419000in}{2.419000in}}%
\pgfusepath{clip}%
\pgfsetbuttcap%
\pgfsetroundjoin%
\definecolor{currentfill}{rgb}{0.866667,0.800000,0.466667}%
\pgfsetfillcolor{currentfill}%
\pgfsetfillopacity{0.327337}%
\pgfsetlinewidth{1.003750pt}%
\definecolor{currentstroke}{rgb}{0.866667,0.800000,0.466667}%
\pgfsetstrokecolor{currentstroke}%
\pgfsetstrokeopacity{0.327337}%
\pgfsetdash{}{0pt}%
\pgfpathmoveto{\pgfqpoint{1.361838in}{3.676428in}}%
\pgfpathcurveto{\pgfqpoint{1.371046in}{3.676428in}}{\pgfqpoint{1.379879in}{3.680087in}}{\pgfqpoint{1.386390in}{3.686598in}}%
\pgfpathcurveto{\pgfqpoint{1.392902in}{3.693110in}}{\pgfqpoint{1.396560in}{3.701942in}}{\pgfqpoint{1.396560in}{3.711151in}}%
\pgfpathcurveto{\pgfqpoint{1.396560in}{3.720359in}}{\pgfqpoint{1.392902in}{3.729192in}}{\pgfqpoint{1.386390in}{3.735703in}}%
\pgfpathcurveto{\pgfqpoint{1.379879in}{3.742214in}}{\pgfqpoint{1.371046in}{3.745873in}}{\pgfqpoint{1.361838in}{3.745873in}}%
\pgfpathcurveto{\pgfqpoint{1.352630in}{3.745873in}}{\pgfqpoint{1.343797in}{3.742214in}}{\pgfqpoint{1.337286in}{3.735703in}}%
\pgfpathcurveto{\pgfqpoint{1.330774in}{3.729192in}}{\pgfqpoint{1.327116in}{3.720359in}}{\pgfqpoint{1.327116in}{3.711151in}}%
\pgfpathcurveto{\pgfqpoint{1.327116in}{3.701942in}}{\pgfqpoint{1.330774in}{3.693110in}}{\pgfqpoint{1.337286in}{3.686598in}}%
\pgfpathcurveto{\pgfqpoint{1.343797in}{3.680087in}}{\pgfqpoint{1.352630in}{3.676428in}}{\pgfqpoint{1.361838in}{3.676428in}}%
\pgfpathlineto{\pgfqpoint{1.361838in}{3.676428in}}%
\pgfpathclose%
\pgfusepath{stroke,fill}%
\end{pgfscope}%
\begin{pgfscope}%
\pgfpathrectangle{\pgfqpoint{0.050000in}{0.050000in}}{\pgfqpoint{2.419000in}{2.419000in}}%
\pgfusepath{clip}%
\pgfsetbuttcap%
\pgfsetroundjoin%
\definecolor{currentfill}{rgb}{0.866667,0.800000,0.466667}%
\pgfsetfillcolor{currentfill}%
\pgfsetfillopacity{0.327337}%
\pgfsetlinewidth{1.003750pt}%
\definecolor{currentstroke}{rgb}{0.866667,0.800000,0.466667}%
\pgfsetstrokecolor{currentstroke}%
\pgfsetstrokeopacity{0.327337}%
\pgfsetdash{}{0pt}%
\pgfpathmoveto{\pgfqpoint{4.740025in}{3.676428in}}%
\pgfpathcurveto{\pgfqpoint{4.749233in}{3.676428in}}{\pgfqpoint{4.758066in}{3.680087in}}{\pgfqpoint{4.764577in}{3.686598in}}%
\pgfpathcurveto{\pgfqpoint{4.771088in}{3.693110in}}{\pgfqpoint{4.774747in}{3.701942in}}{\pgfqpoint{4.774747in}{3.711151in}}%
\pgfpathcurveto{\pgfqpoint{4.774747in}{3.720359in}}{\pgfqpoint{4.771088in}{3.729192in}}{\pgfqpoint{4.764577in}{3.735703in}}%
\pgfpathcurveto{\pgfqpoint{4.758066in}{3.742214in}}{\pgfqpoint{4.749233in}{3.745873in}}{\pgfqpoint{4.740025in}{3.745873in}}%
\pgfpathcurveto{\pgfqpoint{4.730816in}{3.745873in}}{\pgfqpoint{4.721984in}{3.742214in}}{\pgfqpoint{4.715472in}{3.735703in}}%
\pgfpathcurveto{\pgfqpoint{4.708961in}{3.729192in}}{\pgfqpoint{4.705303in}{3.720359in}}{\pgfqpoint{4.705303in}{3.711151in}}%
\pgfpathcurveto{\pgfqpoint{4.705303in}{3.701942in}}{\pgfqpoint{4.708961in}{3.693110in}}{\pgfqpoint{4.715472in}{3.686598in}}%
\pgfpathcurveto{\pgfqpoint{4.721984in}{3.680087in}}{\pgfqpoint{4.730816in}{3.676428in}}{\pgfqpoint{4.740025in}{3.676428in}}%
\pgfpathlineto{\pgfqpoint{4.740025in}{3.676428in}}%
\pgfpathclose%
\pgfusepath{stroke,fill}%
\end{pgfscope}%
\begin{pgfscope}%
\pgfpathrectangle{\pgfqpoint{0.050000in}{0.050000in}}{\pgfqpoint{2.419000in}{2.419000in}}%
\pgfusepath{clip}%
\pgfsetbuttcap%
\pgfsetroundjoin%
\definecolor{currentfill}{rgb}{0.866667,0.800000,0.466667}%
\pgfsetfillcolor{currentfill}%
\pgfsetfillopacity{0.330440}%
\pgfsetlinewidth{1.003750pt}%
\definecolor{currentstroke}{rgb}{0.866667,0.800000,0.466667}%
\pgfsetstrokecolor{currentstroke}%
\pgfsetstrokeopacity{0.330440}%
\pgfsetdash{}{0pt}%
\pgfpathmoveto{\pgfqpoint{3.393079in}{3.613537in}}%
\pgfpathcurveto{\pgfqpoint{3.402288in}{3.613537in}}{\pgfqpoint{3.411120in}{3.617195in}}{\pgfqpoint{3.417631in}{3.623707in}}%
\pgfpathcurveto{\pgfqpoint{3.424143in}{3.630218in}}{\pgfqpoint{3.427801in}{3.639050in}}{\pgfqpoint{3.427801in}{3.648259in}}%
\pgfpathcurveto{\pgfqpoint{3.427801in}{3.657467in}}{\pgfqpoint{3.424143in}{3.666300in}}{\pgfqpoint{3.417631in}{3.672811in}}%
\pgfpathcurveto{\pgfqpoint{3.411120in}{3.679322in}}{\pgfqpoint{3.402288in}{3.682981in}}{\pgfqpoint{3.393079in}{3.682981in}}%
\pgfpathcurveto{\pgfqpoint{3.383871in}{3.682981in}}{\pgfqpoint{3.375038in}{3.679322in}}{\pgfqpoint{3.368527in}{3.672811in}}%
\pgfpathcurveto{\pgfqpoint{3.362016in}{3.666300in}}{\pgfqpoint{3.358357in}{3.657467in}}{\pgfqpoint{3.358357in}{3.648259in}}%
\pgfpathcurveto{\pgfqpoint{3.358357in}{3.639050in}}{\pgfqpoint{3.362016in}{3.630218in}}{\pgfqpoint{3.368527in}{3.623707in}}%
\pgfpathcurveto{\pgfqpoint{3.375038in}{3.617195in}}{\pgfqpoint{3.383871in}{3.613537in}}{\pgfqpoint{3.393079in}{3.613537in}}%
\pgfpathlineto{\pgfqpoint{3.393079in}{3.613537in}}%
\pgfpathclose%
\pgfusepath{stroke,fill}%
\end{pgfscope}%
\begin{pgfscope}%
\pgfpathrectangle{\pgfqpoint{0.050000in}{0.050000in}}{\pgfqpoint{2.419000in}{2.419000in}}%
\pgfusepath{clip}%
\pgfsetbuttcap%
\pgfsetroundjoin%
\definecolor{currentfill}{rgb}{0.866667,0.800000,0.466667}%
\pgfsetfillcolor{currentfill}%
\pgfsetfillopacity{0.333619}%
\pgfsetlinewidth{1.003750pt}%
\definecolor{currentstroke}{rgb}{0.866667,0.800000,0.466667}%
\pgfsetstrokecolor{currentstroke}%
\pgfsetstrokeopacity{0.333619}%
\pgfsetdash{}{0pt}%
\pgfpathmoveto{\pgfqpoint{2.012699in}{3.549084in}}%
\pgfpathcurveto{\pgfqpoint{2.021908in}{3.549084in}}{\pgfqpoint{2.030740in}{3.552742in}}{\pgfqpoint{2.037252in}{3.559254in}}%
\pgfpathcurveto{\pgfqpoint{2.043763in}{3.565765in}}{\pgfqpoint{2.047422in}{3.574598in}}{\pgfqpoint{2.047422in}{3.583806in}}%
\pgfpathcurveto{\pgfqpoint{2.047422in}{3.593014in}}{\pgfqpoint{2.043763in}{3.601847in}}{\pgfqpoint{2.037252in}{3.608358in}}%
\pgfpathcurveto{\pgfqpoint{2.030740in}{3.614870in}}{\pgfqpoint{2.021908in}{3.618528in}}{\pgfqpoint{2.012699in}{3.618528in}}%
\pgfpathcurveto{\pgfqpoint{2.003491in}{3.618528in}}{\pgfqpoint{1.994658in}{3.614870in}}{\pgfqpoint{1.988147in}{3.608358in}}%
\pgfpathcurveto{\pgfqpoint{1.981636in}{3.601847in}}{\pgfqpoint{1.977977in}{3.593014in}}{\pgfqpoint{1.977977in}{3.583806in}}%
\pgfpathcurveto{\pgfqpoint{1.977977in}{3.574598in}}{\pgfqpoint{1.981636in}{3.565765in}}{\pgfqpoint{1.988147in}{3.559254in}}%
\pgfpathcurveto{\pgfqpoint{1.994658in}{3.552742in}}{\pgfqpoint{2.003491in}{3.549084in}}{\pgfqpoint{2.012699in}{3.549084in}}%
\pgfpathlineto{\pgfqpoint{2.012699in}{3.549084in}}%
\pgfpathclose%
\pgfusepath{stroke,fill}%
\end{pgfscope}%
\begin{pgfscope}%
\pgfpathrectangle{\pgfqpoint{0.050000in}{0.050000in}}{\pgfqpoint{2.419000in}{2.419000in}}%
\pgfusepath{clip}%
\pgfsetbuttcap%
\pgfsetroundjoin%
\definecolor{currentfill}{rgb}{0.866667,0.800000,0.466667}%
\pgfsetfillcolor{currentfill}%
\pgfsetfillopacity{0.333619}%
\pgfsetlinewidth{1.003750pt}%
\definecolor{currentstroke}{rgb}{0.866667,0.800000,0.466667}%
\pgfsetstrokecolor{currentstroke}%
\pgfsetstrokeopacity{0.333619}%
\pgfsetdash{}{0pt}%
\pgfpathmoveto{\pgfqpoint{5.474740in}{3.549084in}}%
\pgfpathcurveto{\pgfqpoint{5.483949in}{3.549084in}}{\pgfqpoint{5.492781in}{3.552742in}}{\pgfqpoint{5.499292in}{3.559254in}}%
\pgfpathcurveto{\pgfqpoint{5.505804in}{3.565765in}}{\pgfqpoint{5.509462in}{3.574598in}}{\pgfqpoint{5.509462in}{3.583806in}}%
\pgfpathcurveto{\pgfqpoint{5.509462in}{3.593014in}}{\pgfqpoint{5.505804in}{3.601847in}}{\pgfqpoint{5.499292in}{3.608358in}}%
\pgfpathcurveto{\pgfqpoint{5.492781in}{3.614870in}}{\pgfqpoint{5.483949in}{3.618528in}}{\pgfqpoint{5.474740in}{3.618528in}}%
\pgfpathcurveto{\pgfqpoint{5.465532in}{3.618528in}}{\pgfqpoint{5.456699in}{3.614870in}}{\pgfqpoint{5.450188in}{3.608358in}}%
\pgfpathcurveto{\pgfqpoint{5.443676in}{3.601847in}}{\pgfqpoint{5.440018in}{3.593014in}}{\pgfqpoint{5.440018in}{3.583806in}}%
\pgfpathcurveto{\pgfqpoint{5.440018in}{3.574598in}}{\pgfqpoint{5.443676in}{3.565765in}}{\pgfqpoint{5.450188in}{3.559254in}}%
\pgfpathcurveto{\pgfqpoint{5.456699in}{3.552742in}}{\pgfqpoint{5.465532in}{3.549084in}}{\pgfqpoint{5.474740in}{3.549084in}}%
\pgfpathlineto{\pgfqpoint{5.474740in}{3.549084in}}%
\pgfpathclose%
\pgfusepath{stroke,fill}%
\end{pgfscope}%
\begin{pgfscope}%
\pgfpathrectangle{\pgfqpoint{0.050000in}{0.050000in}}{\pgfqpoint{2.419000in}{2.419000in}}%
\pgfusepath{clip}%
\pgfsetbuttcap%
\pgfsetroundjoin%
\definecolor{currentfill}{rgb}{0.866667,0.800000,0.466667}%
\pgfsetfillcolor{currentfill}%
\pgfsetfillopacity{0.336878}%
\pgfsetlinewidth{1.003750pt}%
\definecolor{currentstroke}{rgb}{0.866667,0.800000,0.466667}%
\pgfsetstrokecolor{currentstroke}%
\pgfsetstrokeopacity{0.336878}%
\pgfsetdash{}{0pt}%
\pgfpathmoveto{\pgfqpoint{0.597625in}{3.483011in}}%
\pgfpathcurveto{\pgfqpoint{0.606834in}{3.483011in}}{\pgfqpoint{0.615666in}{3.486669in}}{\pgfqpoint{0.622177in}{3.493181in}}%
\pgfpathcurveto{\pgfqpoint{0.628689in}{3.499692in}}{\pgfqpoint{0.632347in}{3.508525in}}{\pgfqpoint{0.632347in}{3.517733in}}%
\pgfpathcurveto{\pgfqpoint{0.632347in}{3.526942in}}{\pgfqpoint{0.628689in}{3.535774in}}{\pgfqpoint{0.622177in}{3.542285in}}%
\pgfpathcurveto{\pgfqpoint{0.615666in}{3.548797in}}{\pgfqpoint{0.606834in}{3.552455in}}{\pgfqpoint{0.597625in}{3.552455in}}%
\pgfpathcurveto{\pgfqpoint{0.588417in}{3.552455in}}{\pgfqpoint{0.579584in}{3.548797in}}{\pgfqpoint{0.573073in}{3.542285in}}%
\pgfpathcurveto{\pgfqpoint{0.566561in}{3.535774in}}{\pgfqpoint{0.562903in}{3.526942in}}{\pgfqpoint{0.562903in}{3.517733in}}%
\pgfpathcurveto{\pgfqpoint{0.562903in}{3.508525in}}{\pgfqpoint{0.566561in}{3.499692in}}{\pgfqpoint{0.573073in}{3.493181in}}%
\pgfpathcurveto{\pgfqpoint{0.579584in}{3.486669in}}{\pgfqpoint{0.588417in}{3.483011in}}{\pgfqpoint{0.597625in}{3.483011in}}%
\pgfpathlineto{\pgfqpoint{0.597625in}{3.483011in}}%
\pgfpathclose%
\pgfusepath{stroke,fill}%
\end{pgfscope}%
\begin{pgfscope}%
\pgfpathrectangle{\pgfqpoint{0.050000in}{0.050000in}}{\pgfqpoint{2.419000in}{2.419000in}}%
\pgfusepath{clip}%
\pgfsetbuttcap%
\pgfsetroundjoin%
\definecolor{currentfill}{rgb}{0.866667,0.800000,0.466667}%
\pgfsetfillcolor{currentfill}%
\pgfsetfillopacity{0.336878}%
\pgfsetlinewidth{1.003750pt}%
\definecolor{currentstroke}{rgb}{0.866667,0.800000,0.466667}%
\pgfsetstrokecolor{currentstroke}%
\pgfsetstrokeopacity{0.336878}%
\pgfsetdash{}{0pt}%
\pgfpathmoveto{\pgfqpoint{4.103173in}{3.483011in}}%
\pgfpathcurveto{\pgfqpoint{4.112382in}{3.483011in}}{\pgfqpoint{4.121214in}{3.486669in}}{\pgfqpoint{4.127726in}{3.493181in}}%
\pgfpathcurveto{\pgfqpoint{4.134237in}{3.499692in}}{\pgfqpoint{4.137896in}{3.508525in}}{\pgfqpoint{4.137896in}{3.517733in}}%
\pgfpathcurveto{\pgfqpoint{4.137896in}{3.526942in}}{\pgfqpoint{4.134237in}{3.535774in}}{\pgfqpoint{4.127726in}{3.542285in}}%
\pgfpathcurveto{\pgfqpoint{4.121214in}{3.548797in}}{\pgfqpoint{4.112382in}{3.552455in}}{\pgfqpoint{4.103173in}{3.552455in}}%
\pgfpathcurveto{\pgfqpoint{4.093965in}{3.552455in}}{\pgfqpoint{4.085132in}{3.548797in}}{\pgfqpoint{4.078621in}{3.542285in}}%
\pgfpathcurveto{\pgfqpoint{4.072110in}{3.535774in}}{\pgfqpoint{4.068451in}{3.526942in}}{\pgfqpoint{4.068451in}{3.517733in}}%
\pgfpathcurveto{\pgfqpoint{4.068451in}{3.508525in}}{\pgfqpoint{4.072110in}{3.499692in}}{\pgfqpoint{4.078621in}{3.493181in}}%
\pgfpathcurveto{\pgfqpoint{4.085132in}{3.486669in}}{\pgfqpoint{4.093965in}{3.483011in}}{\pgfqpoint{4.103173in}{3.483011in}}%
\pgfpathlineto{\pgfqpoint{4.103173in}{3.483011in}}%
\pgfpathclose%
\pgfusepath{stroke,fill}%
\end{pgfscope}%
\begin{pgfscope}%
\pgfpathrectangle{\pgfqpoint{0.050000in}{0.050000in}}{\pgfqpoint{2.419000in}{2.419000in}}%
\pgfusepath{clip}%
\pgfsetbuttcap%
\pgfsetroundjoin%
\definecolor{currentfill}{rgb}{0.866667,0.800000,0.466667}%
\pgfsetfillcolor{currentfill}%
\pgfsetfillopacity{0.340220}%
\pgfsetlinewidth{1.003750pt}%
\definecolor{currentstroke}{rgb}{0.866667,0.800000,0.466667}%
\pgfsetstrokecolor{currentstroke}%
\pgfsetstrokeopacity{0.340220}%
\pgfsetdash{}{0pt}%
\pgfpathmoveto{\pgfqpoint{2.696695in}{3.415256in}}%
\pgfpathcurveto{\pgfqpoint{2.705903in}{3.415256in}}{\pgfqpoint{2.714736in}{3.418915in}}{\pgfqpoint{2.721247in}{3.425426in}}%
\pgfpathcurveto{\pgfqpoint{2.727759in}{3.431937in}}{\pgfqpoint{2.731417in}{3.440770in}}{\pgfqpoint{2.731417in}{3.449978in}}%
\pgfpathcurveto{\pgfqpoint{2.731417in}{3.459187in}}{\pgfqpoint{2.727759in}{3.468019in}}{\pgfqpoint{2.721247in}{3.474531in}}%
\pgfpathcurveto{\pgfqpoint{2.714736in}{3.481042in}}{\pgfqpoint{2.705903in}{3.484701in}}{\pgfqpoint{2.696695in}{3.484701in}}%
\pgfpathcurveto{\pgfqpoint{2.687486in}{3.484701in}}{\pgfqpoint{2.678654in}{3.481042in}}{\pgfqpoint{2.672143in}{3.474531in}}%
\pgfpathcurveto{\pgfqpoint{2.665631in}{3.468019in}}{\pgfqpoint{2.661973in}{3.459187in}}{\pgfqpoint{2.661973in}{3.449978in}}%
\pgfpathcurveto{\pgfqpoint{2.661973in}{3.440770in}}{\pgfqpoint{2.665631in}{3.431937in}}{\pgfqpoint{2.672143in}{3.425426in}}%
\pgfpathcurveto{\pgfqpoint{2.678654in}{3.418915in}}{\pgfqpoint{2.687486in}{3.415256in}}{\pgfqpoint{2.696695in}{3.415256in}}%
\pgfpathlineto{\pgfqpoint{2.696695in}{3.415256in}}%
\pgfpathclose%
\pgfusepath{stroke,fill}%
\end{pgfscope}%
\begin{pgfscope}%
\pgfpathrectangle{\pgfqpoint{0.050000in}{0.050000in}}{\pgfqpoint{2.419000in}{2.419000in}}%
\pgfusepath{clip}%
\pgfsetbuttcap%
\pgfsetroundjoin%
\definecolor{currentfill}{rgb}{0.866667,0.800000,0.466667}%
\pgfsetfillcolor{currentfill}%
\pgfsetfillopacity{0.340220}%
\pgfsetlinewidth{1.003750pt}%
\definecolor{currentstroke}{rgb}{0.866667,0.800000,0.466667}%
\pgfsetstrokecolor{currentstroke}%
\pgfsetstrokeopacity{0.340220}%
\pgfsetdash{}{0pt}%
\pgfpathmoveto{\pgfqpoint{6.246858in}{3.415256in}}%
\pgfpathcurveto{\pgfqpoint{6.256067in}{3.415256in}}{\pgfqpoint{6.264899in}{3.418915in}}{\pgfqpoint{6.271411in}{3.425426in}}%
\pgfpathcurveto{\pgfqpoint{6.277922in}{3.431937in}}{\pgfqpoint{6.281580in}{3.440770in}}{\pgfqpoint{6.281580in}{3.449978in}}%
\pgfpathcurveto{\pgfqpoint{6.281580in}{3.459187in}}{\pgfqpoint{6.277922in}{3.468019in}}{\pgfqpoint{6.271411in}{3.474531in}}%
\pgfpathcurveto{\pgfqpoint{6.264899in}{3.481042in}}{\pgfqpoint{6.256067in}{3.484701in}}{\pgfqpoint{6.246858in}{3.484701in}}%
\pgfpathcurveto{\pgfqpoint{6.237650in}{3.484701in}}{\pgfqpoint{6.228817in}{3.481042in}}{\pgfqpoint{6.222306in}{3.474531in}}%
\pgfpathcurveto{\pgfqpoint{6.215795in}{3.468019in}}{\pgfqpoint{6.212136in}{3.459187in}}{\pgfqpoint{6.212136in}{3.449978in}}%
\pgfpathcurveto{\pgfqpoint{6.212136in}{3.440770in}}{\pgfqpoint{6.215795in}{3.431937in}}{\pgfqpoint{6.222306in}{3.425426in}}%
\pgfpathcurveto{\pgfqpoint{6.228817in}{3.418915in}}{\pgfqpoint{6.237650in}{3.415256in}}{\pgfqpoint{6.246858in}{3.415256in}}%
\pgfpathlineto{\pgfqpoint{6.246858in}{3.415256in}}%
\pgfpathclose%
\pgfusepath{stroke,fill}%
\end{pgfscope}%
\begin{pgfscope}%
\pgfpathrectangle{\pgfqpoint{0.050000in}{0.050000in}}{\pgfqpoint{2.419000in}{2.419000in}}%
\pgfusepath{clip}%
\pgfsetbuttcap%
\pgfsetroundjoin%
\definecolor{currentfill}{rgb}{0.866667,0.800000,0.466667}%
\pgfsetfillcolor{currentfill}%
\pgfsetfillopacity{0.343649}%
\pgfsetlinewidth{1.003750pt}%
\definecolor{currentstroke}{rgb}{0.866667,0.800000,0.466667}%
\pgfsetstrokecolor{currentstroke}%
\pgfsetstrokeopacity{0.343649}%
\pgfsetdash{}{0pt}%
\pgfpathmoveto{\pgfqpoint{4.849883in}{3.345755in}}%
\pgfpathcurveto{\pgfqpoint{4.859092in}{3.345755in}}{\pgfqpoint{4.867924in}{3.349413in}}{\pgfqpoint{4.874435in}{3.355925in}}%
\pgfpathcurveto{\pgfqpoint{4.880947in}{3.362436in}}{\pgfqpoint{4.884605in}{3.371269in}}{\pgfqpoint{4.884605in}{3.380477in}}%
\pgfpathcurveto{\pgfqpoint{4.884605in}{3.389685in}}{\pgfqpoint{4.880947in}{3.398518in}}{\pgfqpoint{4.874435in}{3.405029in}}%
\pgfpathcurveto{\pgfqpoint{4.867924in}{3.411541in}}{\pgfqpoint{4.859092in}{3.415199in}}{\pgfqpoint{4.849883in}{3.415199in}}%
\pgfpathcurveto{\pgfqpoint{4.840675in}{3.415199in}}{\pgfqpoint{4.831842in}{3.411541in}}{\pgfqpoint{4.825331in}{3.405029in}}%
\pgfpathcurveto{\pgfqpoint{4.818819in}{3.398518in}}{\pgfqpoint{4.815161in}{3.389685in}}{\pgfqpoint{4.815161in}{3.380477in}}%
\pgfpathcurveto{\pgfqpoint{4.815161in}{3.371269in}}{\pgfqpoint{4.818819in}{3.362436in}}{\pgfqpoint{4.825331in}{3.355925in}}%
\pgfpathcurveto{\pgfqpoint{4.831842in}{3.349413in}}{\pgfqpoint{4.840675in}{3.345755in}}{\pgfqpoint{4.849883in}{3.345755in}}%
\pgfpathlineto{\pgfqpoint{4.849883in}{3.345755in}}%
\pgfpathclose%
\pgfusepath{stroke,fill}%
\end{pgfscope}%
\begin{pgfscope}%
\pgfpathrectangle{\pgfqpoint{0.050000in}{0.050000in}}{\pgfqpoint{2.419000in}{2.419000in}}%
\pgfusepath{clip}%
\pgfsetbuttcap%
\pgfsetroundjoin%
\definecolor{currentfill}{rgb}{0.866667,0.800000,0.466667}%
\pgfsetfillcolor{currentfill}%
\pgfsetfillopacity{0.343649}%
\pgfsetlinewidth{1.003750pt}%
\definecolor{currentstroke}{rgb}{0.866667,0.800000,0.466667}%
\pgfsetstrokecolor{currentstroke}%
\pgfsetstrokeopacity{0.343649}%
\pgfsetdash{}{0pt}%
\pgfpathmoveto{\pgfqpoint{1.253954in}{3.345755in}}%
\pgfpathcurveto{\pgfqpoint{1.263163in}{3.345755in}}{\pgfqpoint{1.271995in}{3.349413in}}{\pgfqpoint{1.278507in}{3.355925in}}%
\pgfpathcurveto{\pgfqpoint{1.285018in}{3.362436in}}{\pgfqpoint{1.288677in}{3.371269in}}{\pgfqpoint{1.288677in}{3.380477in}}%
\pgfpathcurveto{\pgfqpoint{1.288677in}{3.389685in}}{\pgfqpoint{1.285018in}{3.398518in}}{\pgfqpoint{1.278507in}{3.405029in}}%
\pgfpathcurveto{\pgfqpoint{1.271995in}{3.411541in}}{\pgfqpoint{1.263163in}{3.415199in}}{\pgfqpoint{1.253954in}{3.415199in}}%
\pgfpathcurveto{\pgfqpoint{1.244746in}{3.415199in}}{\pgfqpoint{1.235913in}{3.411541in}}{\pgfqpoint{1.229402in}{3.405029in}}%
\pgfpathcurveto{\pgfqpoint{1.222891in}{3.398518in}}{\pgfqpoint{1.219232in}{3.389685in}}{\pgfqpoint{1.219232in}{3.380477in}}%
\pgfpathcurveto{\pgfqpoint{1.219232in}{3.371269in}}{\pgfqpoint{1.222891in}{3.362436in}}{\pgfqpoint{1.229402in}{3.355925in}}%
\pgfpathcurveto{\pgfqpoint{1.235913in}{3.349413in}}{\pgfqpoint{1.244746in}{3.345755in}}{\pgfqpoint{1.253954in}{3.345755in}}%
\pgfpathlineto{\pgfqpoint{1.253954in}{3.345755in}}%
\pgfpathclose%
\pgfusepath{stroke,fill}%
\end{pgfscope}%
\begin{pgfscope}%
\pgfpathrectangle{\pgfqpoint{0.050000in}{0.050000in}}{\pgfqpoint{2.419000in}{2.419000in}}%
\pgfusepath{clip}%
\pgfsetbuttcap%
\pgfsetroundjoin%
\definecolor{currentfill}{rgb}{0.866667,0.800000,0.466667}%
\pgfsetfillcolor{currentfill}%
\pgfsetfillopacity{0.347167}%
\pgfsetlinewidth{1.003750pt}%
\definecolor{currentstroke}{rgb}{0.866667,0.800000,0.466667}%
\pgfsetstrokecolor{currentstroke}%
\pgfsetstrokeopacity{0.347167}%
\pgfsetdash{}{0pt}%
\pgfpathmoveto{\pgfqpoint{-0.226469in}{3.274438in}}%
\pgfpathcurveto{\pgfqpoint{-0.217260in}{3.274438in}}{\pgfqpoint{-0.208428in}{3.278096in}}{\pgfqpoint{-0.201916in}{3.284608in}}%
\pgfpathcurveto{\pgfqpoint{-0.195405in}{3.291119in}}{\pgfqpoint{-0.191746in}{3.299952in}}{\pgfqpoint{-0.191746in}{3.309160in}}%
\pgfpathcurveto{\pgfqpoint{-0.191746in}{3.318369in}}{\pgfqpoint{-0.195405in}{3.327201in}}{\pgfqpoint{-0.201916in}{3.333712in}}%
\pgfpathcurveto{\pgfqpoint{-0.208428in}{3.340224in}}{\pgfqpoint{-0.217260in}{3.343882in}}{\pgfqpoint{-0.226469in}{3.343882in}}%
\pgfpathcurveto{\pgfqpoint{-0.235677in}{3.343882in}}{\pgfqpoint{-0.244510in}{3.340224in}}{\pgfqpoint{-0.251021in}{3.333712in}}%
\pgfpathcurveto{\pgfqpoint{-0.257532in}{3.327201in}}{\pgfqpoint{-0.261191in}{3.318369in}}{\pgfqpoint{-0.261191in}{3.309160in}}%
\pgfpathcurveto{\pgfqpoint{-0.261191in}{3.299952in}}{\pgfqpoint{-0.257532in}{3.291119in}}{\pgfqpoint{-0.251021in}{3.284608in}}%
\pgfpathcurveto{\pgfqpoint{-0.244510in}{3.278096in}}{\pgfqpoint{-0.235677in}{3.274438in}}{\pgfqpoint{-0.226469in}{3.274438in}}%
\pgfpathlineto{\pgfqpoint{-0.226469in}{3.274438in}}%
\pgfpathclose%
\pgfusepath{stroke,fill}%
\end{pgfscope}%
\begin{pgfscope}%
\pgfpathrectangle{\pgfqpoint{0.050000in}{0.050000in}}{\pgfqpoint{2.419000in}{2.419000in}}%
\pgfusepath{clip}%
\pgfsetbuttcap%
\pgfsetroundjoin%
\definecolor{currentfill}{rgb}{0.866667,0.800000,0.466667}%
\pgfsetfillcolor{currentfill}%
\pgfsetfillopacity{0.347167}%
\pgfsetlinewidth{1.003750pt}%
\definecolor{currentstroke}{rgb}{0.866667,0.800000,0.466667}%
\pgfsetstrokecolor{currentstroke}%
\pgfsetstrokeopacity{0.347167}%
\pgfsetdash{}{0pt}%
\pgfpathmoveto{\pgfqpoint{3.416421in}{3.274438in}}%
\pgfpathcurveto{\pgfqpoint{3.425629in}{3.274438in}}{\pgfqpoint{3.434462in}{3.278096in}}{\pgfqpoint{3.440973in}{3.284608in}}%
\pgfpathcurveto{\pgfqpoint{3.447484in}{3.291119in}}{\pgfqpoint{3.451143in}{3.299952in}}{\pgfqpoint{3.451143in}{3.309160in}}%
\pgfpathcurveto{\pgfqpoint{3.451143in}{3.318369in}}{\pgfqpoint{3.447484in}{3.327201in}}{\pgfqpoint{3.440973in}{3.333712in}}%
\pgfpathcurveto{\pgfqpoint{3.434462in}{3.340224in}}{\pgfqpoint{3.425629in}{3.343882in}}{\pgfqpoint{3.416421in}{3.343882in}}%
\pgfpathcurveto{\pgfqpoint{3.407212in}{3.343882in}}{\pgfqpoint{3.398380in}{3.340224in}}{\pgfqpoint{3.391868in}{3.333712in}}%
\pgfpathcurveto{\pgfqpoint{3.385357in}{3.327201in}}{\pgfqpoint{3.381698in}{3.318369in}}{\pgfqpoint{3.381698in}{3.309160in}}%
\pgfpathcurveto{\pgfqpoint{3.381698in}{3.299952in}}{\pgfqpoint{3.385357in}{3.291119in}}{\pgfqpoint{3.391868in}{3.284608in}}%
\pgfpathcurveto{\pgfqpoint{3.398380in}{3.278096in}}{\pgfqpoint{3.407212in}{3.274438in}}{\pgfqpoint{3.416421in}{3.274438in}}%
\pgfpathlineto{\pgfqpoint{3.416421in}{3.274438in}}%
\pgfpathclose%
\pgfusepath{stroke,fill}%
\end{pgfscope}%
\begin{pgfscope}%
\pgfpathrectangle{\pgfqpoint{0.050000in}{0.050000in}}{\pgfqpoint{2.419000in}{2.419000in}}%
\pgfusepath{clip}%
\pgfsetbuttcap%
\pgfsetroundjoin%
\definecolor{currentfill}{rgb}{0.866667,0.800000,0.466667}%
\pgfsetfillcolor{currentfill}%
\pgfsetfillopacity{0.347167}%
\pgfsetlinewidth{1.003750pt}%
\definecolor{currentstroke}{rgb}{0.866667,0.800000,0.466667}%
\pgfsetstrokecolor{currentstroke}%
\pgfsetstrokeopacity{0.347167}%
\pgfsetdash{}{0pt}%
\pgfpathmoveto{\pgfqpoint{7.059310in}{3.274438in}}%
\pgfpathcurveto{\pgfqpoint{7.068518in}{3.274438in}}{\pgfqpoint{7.077351in}{3.278096in}}{\pgfqpoint{7.083862in}{3.284608in}}%
\pgfpathcurveto{\pgfqpoint{7.090374in}{3.291119in}}{\pgfqpoint{7.094032in}{3.299952in}}{\pgfqpoint{7.094032in}{3.309160in}}%
\pgfpathcurveto{\pgfqpoint{7.094032in}{3.318369in}}{\pgfqpoint{7.090374in}{3.327201in}}{\pgfqpoint{7.083862in}{3.333712in}}%
\pgfpathcurveto{\pgfqpoint{7.077351in}{3.340224in}}{\pgfqpoint{7.068518in}{3.343882in}}{\pgfqpoint{7.059310in}{3.343882in}}%
\pgfpathcurveto{\pgfqpoint{7.050102in}{3.343882in}}{\pgfqpoint{7.041269in}{3.340224in}}{\pgfqpoint{7.034758in}{3.333712in}}%
\pgfpathcurveto{\pgfqpoint{7.028246in}{3.327201in}}{\pgfqpoint{7.024588in}{3.318369in}}{\pgfqpoint{7.024588in}{3.309160in}}%
\pgfpathcurveto{\pgfqpoint{7.024588in}{3.299952in}}{\pgfqpoint{7.028246in}{3.291119in}}{\pgfqpoint{7.034758in}{3.284608in}}%
\pgfpathcurveto{\pgfqpoint{7.041269in}{3.278096in}}{\pgfqpoint{7.050102in}{3.274438in}}{\pgfqpoint{7.059310in}{3.274438in}}%
\pgfpathlineto{\pgfqpoint{7.059310in}{3.274438in}}%
\pgfpathclose%
\pgfusepath{stroke,fill}%
\end{pgfscope}%
\begin{pgfscope}%
\pgfpathrectangle{\pgfqpoint{0.050000in}{0.050000in}}{\pgfqpoint{2.419000in}{2.419000in}}%
\pgfusepath{clip}%
\pgfsetbuttcap%
\pgfsetroundjoin%
\definecolor{currentfill}{rgb}{0.866667,0.800000,0.466667}%
\pgfsetfillcolor{currentfill}%
\pgfsetfillopacity{0.350778}%
\pgfsetlinewidth{1.003750pt}%
\definecolor{currentstroke}{rgb}{0.866667,0.800000,0.466667}%
\pgfsetstrokecolor{currentstroke}%
\pgfsetstrokeopacity{0.350778}%
\pgfsetdash{}{0pt}%
\pgfpathmoveto{\pgfqpoint{1.945022in}{3.201234in}}%
\pgfpathcurveto{\pgfqpoint{1.954231in}{3.201234in}}{\pgfqpoint{1.963063in}{3.204892in}}{\pgfqpoint{1.969575in}{3.211404in}}%
\pgfpathcurveto{\pgfqpoint{1.976086in}{3.217915in}}{\pgfqpoint{1.979745in}{3.226748in}}{\pgfqpoint{1.979745in}{3.235956in}}%
\pgfpathcurveto{\pgfqpoint{1.979745in}{3.245164in}}{\pgfqpoint{1.976086in}{3.253997in}}{\pgfqpoint{1.969575in}{3.260508in}}%
\pgfpathcurveto{\pgfqpoint{1.963063in}{3.267020in}}{\pgfqpoint{1.954231in}{3.270678in}}{\pgfqpoint{1.945022in}{3.270678in}}%
\pgfpathcurveto{\pgfqpoint{1.935814in}{3.270678in}}{\pgfqpoint{1.926982in}{3.267020in}}{\pgfqpoint{1.920470in}{3.260508in}}%
\pgfpathcurveto{\pgfqpoint{1.913959in}{3.253997in}}{\pgfqpoint{1.910300in}{3.245164in}}{\pgfqpoint{1.910300in}{3.235956in}}%
\pgfpathcurveto{\pgfqpoint{1.910300in}{3.226748in}}{\pgfqpoint{1.913959in}{3.217915in}}{\pgfqpoint{1.920470in}{3.211404in}}%
\pgfpathcurveto{\pgfqpoint{1.926982in}{3.204892in}}{\pgfqpoint{1.935814in}{3.201234in}}{\pgfqpoint{1.945022in}{3.201234in}}%
\pgfpathlineto{\pgfqpoint{1.945022in}{3.201234in}}%
\pgfpathclose%
\pgfusepath{stroke,fill}%
\end{pgfscope}%
\begin{pgfscope}%
\pgfpathrectangle{\pgfqpoint{0.050000in}{0.050000in}}{\pgfqpoint{2.419000in}{2.419000in}}%
\pgfusepath{clip}%
\pgfsetbuttcap%
\pgfsetroundjoin%
\definecolor{currentfill}{rgb}{0.866667,0.800000,0.466667}%
\pgfsetfillcolor{currentfill}%
\pgfsetfillopacity{0.350778}%
\pgfsetlinewidth{1.003750pt}%
\definecolor{currentstroke}{rgb}{0.866667,0.800000,0.466667}%
\pgfsetstrokecolor{currentstroke}%
\pgfsetstrokeopacity{0.350778}%
\pgfsetdash{}{0pt}%
\pgfpathmoveto{\pgfqpoint{5.636115in}{3.201234in}}%
\pgfpathcurveto{\pgfqpoint{5.645324in}{3.201234in}}{\pgfqpoint{5.654156in}{3.204892in}}{\pgfqpoint{5.660668in}{3.211404in}}%
\pgfpathcurveto{\pgfqpoint{5.667179in}{3.217915in}}{\pgfqpoint{5.670837in}{3.226748in}}{\pgfqpoint{5.670837in}{3.235956in}}%
\pgfpathcurveto{\pgfqpoint{5.670837in}{3.245164in}}{\pgfqpoint{5.667179in}{3.253997in}}{\pgfqpoint{5.660668in}{3.260508in}}%
\pgfpathcurveto{\pgfqpoint{5.654156in}{3.267020in}}{\pgfqpoint{5.645324in}{3.270678in}}{\pgfqpoint{5.636115in}{3.270678in}}%
\pgfpathcurveto{\pgfqpoint{5.626907in}{3.270678in}}{\pgfqpoint{5.618074in}{3.267020in}}{\pgfqpoint{5.611563in}{3.260508in}}%
\pgfpathcurveto{\pgfqpoint{5.605052in}{3.253997in}}{\pgfqpoint{5.601393in}{3.245164in}}{\pgfqpoint{5.601393in}{3.235956in}}%
\pgfpathcurveto{\pgfqpoint{5.601393in}{3.226748in}}{\pgfqpoint{5.605052in}{3.217915in}}{\pgfqpoint{5.611563in}{3.211404in}}%
\pgfpathcurveto{\pgfqpoint{5.618074in}{3.204892in}}{\pgfqpoint{5.626907in}{3.201234in}}{\pgfqpoint{5.636115in}{3.201234in}}%
\pgfpathlineto{\pgfqpoint{5.636115in}{3.201234in}}%
\pgfpathclose%
\pgfusepath{stroke,fill}%
\end{pgfscope}%
\begin{pgfscope}%
\pgfpathrectangle{\pgfqpoint{0.050000in}{0.050000in}}{\pgfqpoint{2.419000in}{2.419000in}}%
\pgfusepath{clip}%
\pgfsetbuttcap%
\pgfsetroundjoin%
\definecolor{currentfill}{rgb}{0.866667,0.800000,0.466667}%
\pgfsetfillcolor{currentfill}%
\pgfsetfillopacity{0.354486}%
\pgfsetlinewidth{1.003750pt}%
\definecolor{currentstroke}{rgb}{0.866667,0.800000,0.466667}%
\pgfsetstrokecolor{currentstroke}%
\pgfsetstrokeopacity{0.354486}%
\pgfsetdash{}{0pt}%
\pgfpathmoveto{\pgfqpoint{0.434162in}{3.126066in}}%
\pgfpathcurveto{\pgfqpoint{0.443371in}{3.126066in}}{\pgfqpoint{0.452203in}{3.129725in}}{\pgfqpoint{0.458715in}{3.136236in}}%
\pgfpathcurveto{\pgfqpoint{0.465226in}{3.142748in}}{\pgfqpoint{0.468885in}{3.151580in}}{\pgfqpoint{0.468885in}{3.160789in}}%
\pgfpathcurveto{\pgfqpoint{0.468885in}{3.169997in}}{\pgfqpoint{0.465226in}{3.178830in}}{\pgfqpoint{0.458715in}{3.185341in}}%
\pgfpathcurveto{\pgfqpoint{0.452203in}{3.191852in}}{\pgfqpoint{0.443371in}{3.195511in}}{\pgfqpoint{0.434162in}{3.195511in}}%
\pgfpathcurveto{\pgfqpoint{0.424954in}{3.195511in}}{\pgfqpoint{0.416122in}{3.191852in}}{\pgfqpoint{0.409610in}{3.185341in}}%
\pgfpathcurveto{\pgfqpoint{0.403099in}{3.178830in}}{\pgfqpoint{0.399440in}{3.169997in}}{\pgfqpoint{0.399440in}{3.160789in}}%
\pgfpathcurveto{\pgfqpoint{0.399440in}{3.151580in}}{\pgfqpoint{0.403099in}{3.142748in}}{\pgfqpoint{0.409610in}{3.136236in}}%
\pgfpathcurveto{\pgfqpoint{0.416122in}{3.129725in}}{\pgfqpoint{0.424954in}{3.126066in}}{\pgfqpoint{0.434162in}{3.126066in}}%
\pgfpathlineto{\pgfqpoint{0.434162in}{3.126066in}}%
\pgfpathclose%
\pgfusepath{stroke,fill}%
\end{pgfscope}%
\begin{pgfscope}%
\pgfpathrectangle{\pgfqpoint{0.050000in}{0.050000in}}{\pgfqpoint{2.419000in}{2.419000in}}%
\pgfusepath{clip}%
\pgfsetbuttcap%
\pgfsetroundjoin%
\definecolor{currentfill}{rgb}{0.866667,0.800000,0.466667}%
\pgfsetfillcolor{currentfill}%
\pgfsetfillopacity{0.354486}%
\pgfsetlinewidth{1.003750pt}%
\definecolor{currentstroke}{rgb}{0.866667,0.800000,0.466667}%
\pgfsetstrokecolor{currentstroke}%
\pgfsetstrokeopacity{0.354486}%
\pgfsetdash{}{0pt}%
\pgfpathmoveto{\pgfqpoint{4.174751in}{3.126066in}}%
\pgfpathcurveto{\pgfqpoint{4.183960in}{3.126066in}}{\pgfqpoint{4.192792in}{3.129725in}}{\pgfqpoint{4.199304in}{3.136236in}}%
\pgfpathcurveto{\pgfqpoint{4.205815in}{3.142748in}}{\pgfqpoint{4.209474in}{3.151580in}}{\pgfqpoint{4.209474in}{3.160789in}}%
\pgfpathcurveto{\pgfqpoint{4.209474in}{3.169997in}}{\pgfqpoint{4.205815in}{3.178830in}}{\pgfqpoint{4.199304in}{3.185341in}}%
\pgfpathcurveto{\pgfqpoint{4.192792in}{3.191852in}}{\pgfqpoint{4.183960in}{3.195511in}}{\pgfqpoint{4.174751in}{3.195511in}}%
\pgfpathcurveto{\pgfqpoint{4.165543in}{3.195511in}}{\pgfqpoint{4.156710in}{3.191852in}}{\pgfqpoint{4.150199in}{3.185341in}}%
\pgfpathcurveto{\pgfqpoint{4.143688in}{3.178830in}}{\pgfqpoint{4.140029in}{3.169997in}}{\pgfqpoint{4.140029in}{3.160789in}}%
\pgfpathcurveto{\pgfqpoint{4.140029in}{3.151580in}}{\pgfqpoint{4.143688in}{3.142748in}}{\pgfqpoint{4.150199in}{3.136236in}}%
\pgfpathcurveto{\pgfqpoint{4.156710in}{3.129725in}}{\pgfqpoint{4.165543in}{3.126066in}}{\pgfqpoint{4.174751in}{3.126066in}}%
\pgfpathlineto{\pgfqpoint{4.174751in}{3.126066in}}%
\pgfpathclose%
\pgfusepath{stroke,fill}%
\end{pgfscope}%
\begin{pgfscope}%
\pgfpathrectangle{\pgfqpoint{0.050000in}{0.050000in}}{\pgfqpoint{2.419000in}{2.419000in}}%
\pgfusepath{clip}%
\pgfsetbuttcap%
\pgfsetroundjoin%
\definecolor{currentfill}{rgb}{0.866667,0.800000,0.466667}%
\pgfsetfillcolor{currentfill}%
\pgfsetfillopacity{0.354486}%
\pgfsetlinewidth{1.003750pt}%
\definecolor{currentstroke}{rgb}{0.866667,0.800000,0.466667}%
\pgfsetstrokecolor{currentstroke}%
\pgfsetstrokeopacity{0.354486}%
\pgfsetdash{}{0pt}%
\pgfpathmoveto{\pgfqpoint{7.915340in}{3.126066in}}%
\pgfpathcurveto{\pgfqpoint{7.924549in}{3.126066in}}{\pgfqpoint{7.933381in}{3.129725in}}{\pgfqpoint{7.939893in}{3.136236in}}%
\pgfpathcurveto{\pgfqpoint{7.946404in}{3.142748in}}{\pgfqpoint{7.950063in}{3.151580in}}{\pgfqpoint{7.950063in}{3.160789in}}%
\pgfpathcurveto{\pgfqpoint{7.950063in}{3.169997in}}{\pgfqpoint{7.946404in}{3.178830in}}{\pgfqpoint{7.939893in}{3.185341in}}%
\pgfpathcurveto{\pgfqpoint{7.933381in}{3.191852in}}{\pgfqpoint{7.924549in}{3.195511in}}{\pgfqpoint{7.915340in}{3.195511in}}%
\pgfpathcurveto{\pgfqpoint{7.906132in}{3.195511in}}{\pgfqpoint{7.897299in}{3.191852in}}{\pgfqpoint{7.890788in}{3.185341in}}%
\pgfpathcurveto{\pgfqpoint{7.884277in}{3.178830in}}{\pgfqpoint{7.880618in}{3.169997in}}{\pgfqpoint{7.880618in}{3.160789in}}%
\pgfpathcurveto{\pgfqpoint{7.880618in}{3.151580in}}{\pgfqpoint{7.884277in}{3.142748in}}{\pgfqpoint{7.890788in}{3.136236in}}%
\pgfpathcurveto{\pgfqpoint{7.897299in}{3.129725in}}{\pgfqpoint{7.906132in}{3.126066in}}{\pgfqpoint{7.915340in}{3.126066in}}%
\pgfpathlineto{\pgfqpoint{7.915340in}{3.126066in}}%
\pgfpathclose%
\pgfusepath{stroke,fill}%
\end{pgfscope}%
\begin{pgfscope}%
\pgfpathrectangle{\pgfqpoint{0.050000in}{0.050000in}}{\pgfqpoint{2.419000in}{2.419000in}}%
\pgfusepath{clip}%
\pgfsetbuttcap%
\pgfsetroundjoin%
\definecolor{currentfill}{rgb}{0.866667,0.800000,0.466667}%
\pgfsetfillcolor{currentfill}%
\pgfsetfillopacity{0.358294}%
\pgfsetlinewidth{1.003750pt}%
\definecolor{currentstroke}{rgb}{0.866667,0.800000,0.466667}%
\pgfsetstrokecolor{currentstroke}%
\pgfsetstrokeopacity{0.358294}%
\pgfsetdash{}{0pt}%
\pgfpathmoveto{\pgfqpoint{2.673662in}{3.048856in}}%
\pgfpathcurveto{\pgfqpoint{2.682871in}{3.048856in}}{\pgfqpoint{2.691703in}{3.052514in}}{\pgfqpoint{2.698215in}{3.059025in}}%
\pgfpathcurveto{\pgfqpoint{2.704726in}{3.065537in}}{\pgfqpoint{2.708384in}{3.074369in}}{\pgfqpoint{2.708384in}{3.083578in}}%
\pgfpathcurveto{\pgfqpoint{2.708384in}{3.092786in}}{\pgfqpoint{2.704726in}{3.101619in}}{\pgfqpoint{2.698215in}{3.108130in}}%
\pgfpathcurveto{\pgfqpoint{2.691703in}{3.114641in}}{\pgfqpoint{2.682871in}{3.118300in}}{\pgfqpoint{2.673662in}{3.118300in}}%
\pgfpathcurveto{\pgfqpoint{2.664454in}{3.118300in}}{\pgfqpoint{2.655621in}{3.114641in}}{\pgfqpoint{2.649110in}{3.108130in}}%
\pgfpathcurveto{\pgfqpoint{2.642599in}{3.101619in}}{\pgfqpoint{2.638940in}{3.092786in}}{\pgfqpoint{2.638940in}{3.083578in}}%
\pgfpathcurveto{\pgfqpoint{2.638940in}{3.074369in}}{\pgfqpoint{2.642599in}{3.065537in}}{\pgfqpoint{2.649110in}{3.059025in}}%
\pgfpathcurveto{\pgfqpoint{2.655621in}{3.052514in}}{\pgfqpoint{2.664454in}{3.048856in}}{\pgfqpoint{2.673662in}{3.048856in}}%
\pgfpathlineto{\pgfqpoint{2.673662in}{3.048856in}}%
\pgfpathclose%
\pgfusepath{stroke,fill}%
\end{pgfscope}%
\begin{pgfscope}%
\pgfpathrectangle{\pgfqpoint{0.050000in}{0.050000in}}{\pgfqpoint{2.419000in}{2.419000in}}%
\pgfusepath{clip}%
\pgfsetbuttcap%
\pgfsetroundjoin%
\definecolor{currentfill}{rgb}{0.866667,0.800000,0.466667}%
\pgfsetfillcolor{currentfill}%
\pgfsetfillopacity{0.358294}%
\pgfsetlinewidth{1.003750pt}%
\definecolor{currentstroke}{rgb}{0.866667,0.800000,0.466667}%
\pgfsetstrokecolor{currentstroke}%
\pgfsetstrokeopacity{0.358294}%
\pgfsetdash{}{0pt}%
\pgfpathmoveto{\pgfqpoint{6.465093in}{3.048856in}}%
\pgfpathcurveto{\pgfqpoint{6.474301in}{3.048856in}}{\pgfqpoint{6.483134in}{3.052514in}}{\pgfqpoint{6.489645in}{3.059025in}}%
\pgfpathcurveto{\pgfqpoint{6.496157in}{3.065537in}}{\pgfqpoint{6.499815in}{3.074369in}}{\pgfqpoint{6.499815in}{3.083578in}}%
\pgfpathcurveto{\pgfqpoint{6.499815in}{3.092786in}}{\pgfqpoint{6.496157in}{3.101619in}}{\pgfqpoint{6.489645in}{3.108130in}}%
\pgfpathcurveto{\pgfqpoint{6.483134in}{3.114641in}}{\pgfqpoint{6.474301in}{3.118300in}}{\pgfqpoint{6.465093in}{3.118300in}}%
\pgfpathcurveto{\pgfqpoint{6.455885in}{3.118300in}}{\pgfqpoint{6.447052in}{3.114641in}}{\pgfqpoint{6.440541in}{3.108130in}}%
\pgfpathcurveto{\pgfqpoint{6.434029in}{3.101619in}}{\pgfqpoint{6.430371in}{3.092786in}}{\pgfqpoint{6.430371in}{3.083578in}}%
\pgfpathcurveto{\pgfqpoint{6.430371in}{3.074369in}}{\pgfqpoint{6.434029in}{3.065537in}}{\pgfqpoint{6.440541in}{3.059025in}}%
\pgfpathcurveto{\pgfqpoint{6.447052in}{3.052514in}}{\pgfqpoint{6.455885in}{3.048856in}}{\pgfqpoint{6.465093in}{3.048856in}}%
\pgfpathlineto{\pgfqpoint{6.465093in}{3.048856in}}%
\pgfpathclose%
\pgfusepath{stroke,fill}%
\end{pgfscope}%
\begin{pgfscope}%
\pgfpathrectangle{\pgfqpoint{0.050000in}{0.050000in}}{\pgfqpoint{2.419000in}{2.419000in}}%
\pgfusepath{clip}%
\pgfsetbuttcap%
\pgfsetroundjoin%
\definecolor{currentfill}{rgb}{0.866667,0.800000,0.466667}%
\pgfsetfillcolor{currentfill}%
\pgfsetfillopacity{0.362208}%
\pgfsetlinewidth{1.003750pt}%
\definecolor{currentstroke}{rgb}{0.866667,0.800000,0.466667}%
\pgfsetstrokecolor{currentstroke}%
\pgfsetstrokeopacity{0.362208}%
\pgfsetdash{}{0pt}%
\pgfpathmoveto{\pgfqpoint{4.974879in}{2.969517in}}%
\pgfpathcurveto{\pgfqpoint{4.984087in}{2.969517in}}{\pgfqpoint{4.992920in}{2.973176in}}{\pgfqpoint{4.999431in}{2.979687in}}%
\pgfpathcurveto{\pgfqpoint{5.005943in}{2.986198in}}{\pgfqpoint{5.009601in}{2.995031in}}{\pgfqpoint{5.009601in}{3.004239in}}%
\pgfpathcurveto{\pgfqpoint{5.009601in}{3.013448in}}{\pgfqpoint{5.005943in}{3.022280in}}{\pgfqpoint{4.999431in}{3.028791in}}%
\pgfpathcurveto{\pgfqpoint{4.992920in}{3.035303in}}{\pgfqpoint{4.984087in}{3.038961in}}{\pgfqpoint{4.974879in}{3.038961in}}%
\pgfpathcurveto{\pgfqpoint{4.965671in}{3.038961in}}{\pgfqpoint{4.956838in}{3.035303in}}{\pgfqpoint{4.950327in}{3.028791in}}%
\pgfpathcurveto{\pgfqpoint{4.943815in}{3.022280in}}{\pgfqpoint{4.940157in}{3.013448in}}{\pgfqpoint{4.940157in}{3.004239in}}%
\pgfpathcurveto{\pgfqpoint{4.940157in}{2.995031in}}{\pgfqpoint{4.943815in}{2.986198in}}{\pgfqpoint{4.950327in}{2.979687in}}%
\pgfpathcurveto{\pgfqpoint{4.956838in}{2.973176in}}{\pgfqpoint{4.965671in}{2.969517in}}{\pgfqpoint{4.974879in}{2.969517in}}%
\pgfpathlineto{\pgfqpoint{4.974879in}{2.969517in}}%
\pgfpathclose%
\pgfusepath{stroke,fill}%
\end{pgfscope}%
\begin{pgfscope}%
\pgfpathrectangle{\pgfqpoint{0.050000in}{0.050000in}}{\pgfqpoint{2.419000in}{2.419000in}}%
\pgfusepath{clip}%
\pgfsetbuttcap%
\pgfsetroundjoin%
\definecolor{currentfill}{rgb}{0.866667,0.800000,0.466667}%
\pgfsetfillcolor{currentfill}%
\pgfsetfillopacity{0.362208}%
\pgfsetlinewidth{1.003750pt}%
\definecolor{currentstroke}{rgb}{0.866667,0.800000,0.466667}%
\pgfsetstrokecolor{currentstroke}%
\pgfsetstrokeopacity{0.362208}%
\pgfsetdash{}{0pt}%
\pgfpathmoveto{\pgfqpoint{1.131205in}{2.969517in}}%
\pgfpathcurveto{\pgfqpoint{1.140414in}{2.969517in}}{\pgfqpoint{1.149246in}{2.973176in}}{\pgfqpoint{1.155758in}{2.979687in}}%
\pgfpathcurveto{\pgfqpoint{1.162269in}{2.986198in}}{\pgfqpoint{1.165928in}{2.995031in}}{\pgfqpoint{1.165928in}{3.004239in}}%
\pgfpathcurveto{\pgfqpoint{1.165928in}{3.013448in}}{\pgfqpoint{1.162269in}{3.022280in}}{\pgfqpoint{1.155758in}{3.028791in}}%
\pgfpathcurveto{\pgfqpoint{1.149246in}{3.035303in}}{\pgfqpoint{1.140414in}{3.038961in}}{\pgfqpoint{1.131205in}{3.038961in}}%
\pgfpathcurveto{\pgfqpoint{1.121997in}{3.038961in}}{\pgfqpoint{1.113164in}{3.035303in}}{\pgfqpoint{1.106653in}{3.028791in}}%
\pgfpathcurveto{\pgfqpoint{1.100142in}{3.022280in}}{\pgfqpoint{1.096483in}{3.013448in}}{\pgfqpoint{1.096483in}{3.004239in}}%
\pgfpathcurveto{\pgfqpoint{1.096483in}{2.995031in}}{\pgfqpoint{1.100142in}{2.986198in}}{\pgfqpoint{1.106653in}{2.979687in}}%
\pgfpathcurveto{\pgfqpoint{1.113164in}{2.973176in}}{\pgfqpoint{1.121997in}{2.969517in}}{\pgfqpoint{1.131205in}{2.969517in}}%
\pgfpathlineto{\pgfqpoint{1.131205in}{2.969517in}}%
\pgfpathclose%
\pgfusepath{stroke,fill}%
\end{pgfscope}%
\begin{pgfscope}%
\pgfpathrectangle{\pgfqpoint{0.050000in}{0.050000in}}{\pgfqpoint{2.419000in}{2.419000in}}%
\pgfusepath{clip}%
\pgfsetbuttcap%
\pgfsetroundjoin%
\definecolor{currentfill}{rgb}{0.866667,0.800000,0.466667}%
\pgfsetfillcolor{currentfill}%
\pgfsetfillopacity{0.362208}%
\pgfsetlinewidth{1.003750pt}%
\definecolor{currentstroke}{rgb}{0.866667,0.800000,0.466667}%
\pgfsetstrokecolor{currentstroke}%
\pgfsetstrokeopacity{0.362208}%
\pgfsetdash{}{0pt}%
\pgfpathmoveto{\pgfqpoint{8.818553in}{2.969517in}}%
\pgfpathcurveto{\pgfqpoint{8.827761in}{2.969517in}}{\pgfqpoint{8.836593in}{2.973176in}}{\pgfqpoint{8.843105in}{2.979687in}}%
\pgfpathcurveto{\pgfqpoint{8.849616in}{2.986198in}}{\pgfqpoint{8.853275in}{2.995031in}}{\pgfqpoint{8.853275in}{3.004239in}}%
\pgfpathcurveto{\pgfqpoint{8.853275in}{3.013448in}}{\pgfqpoint{8.849616in}{3.022280in}}{\pgfqpoint{8.843105in}{3.028791in}}%
\pgfpathcurveto{\pgfqpoint{8.836593in}{3.035303in}}{\pgfqpoint{8.827761in}{3.038961in}}{\pgfqpoint{8.818553in}{3.038961in}}%
\pgfpathcurveto{\pgfqpoint{8.809344in}{3.038961in}}{\pgfqpoint{8.800512in}{3.035303in}}{\pgfqpoint{8.794000in}{3.028791in}}%
\pgfpathcurveto{\pgfqpoint{8.787489in}{3.022280in}}{\pgfqpoint{8.783830in}{3.013448in}}{\pgfqpoint{8.783830in}{3.004239in}}%
\pgfpathcurveto{\pgfqpoint{8.783830in}{2.995031in}}{\pgfqpoint{8.787489in}{2.986198in}}{\pgfqpoint{8.794000in}{2.979687in}}%
\pgfpathcurveto{\pgfqpoint{8.800512in}{2.973176in}}{\pgfqpoint{8.809344in}{2.969517in}}{\pgfqpoint{8.818553in}{2.969517in}}%
\pgfpathlineto{\pgfqpoint{8.818553in}{2.969517in}}%
\pgfpathclose%
\pgfusepath{stroke,fill}%
\end{pgfscope}%
\begin{pgfscope}%
\pgfpathrectangle{\pgfqpoint{0.050000in}{0.050000in}}{\pgfqpoint{2.419000in}{2.419000in}}%
\pgfusepath{clip}%
\pgfsetbuttcap%
\pgfsetroundjoin%
\definecolor{currentfill}{rgb}{0.866667,0.800000,0.466667}%
\pgfsetfillcolor{currentfill}%
\pgfsetfillopacity{0.366231}%
\pgfsetlinewidth{1.003750pt}%
\definecolor{currentstroke}{rgb}{0.866667,0.800000,0.466667}%
\pgfsetstrokecolor{currentstroke}%
\pgfsetstrokeopacity{0.366231}%
\pgfsetdash{}{0pt}%
\pgfpathmoveto{\pgfqpoint{-0.454353in}{2.887961in}}%
\pgfpathcurveto{\pgfqpoint{-0.445144in}{2.887961in}}{\pgfqpoint{-0.436312in}{2.891620in}}{\pgfqpoint{-0.429801in}{2.898131in}}%
\pgfpathcurveto{\pgfqpoint{-0.423289in}{2.904643in}}{\pgfqpoint{-0.419631in}{2.913475in}}{\pgfqpoint{-0.419631in}{2.922684in}}%
\pgfpathcurveto{\pgfqpoint{-0.419631in}{2.931892in}}{\pgfqpoint{-0.423289in}{2.940725in}}{\pgfqpoint{-0.429801in}{2.947236in}}%
\pgfpathcurveto{\pgfqpoint{-0.436312in}{2.953747in}}{\pgfqpoint{-0.445144in}{2.957406in}}{\pgfqpoint{-0.454353in}{2.957406in}}%
\pgfpathcurveto{\pgfqpoint{-0.463561in}{2.957406in}}{\pgfqpoint{-0.472394in}{2.953747in}}{\pgfqpoint{-0.478905in}{2.947236in}}%
\pgfpathcurveto{\pgfqpoint{-0.485417in}{2.940725in}}{\pgfqpoint{-0.489075in}{2.931892in}}{\pgfqpoint{-0.489075in}{2.922684in}}%
\pgfpathcurveto{\pgfqpoint{-0.489075in}{2.913475in}}{\pgfqpoint{-0.485417in}{2.904643in}}{\pgfqpoint{-0.478905in}{2.898131in}}%
\pgfpathcurveto{\pgfqpoint{-0.472394in}{2.891620in}}{\pgfqpoint{-0.463561in}{2.887961in}}{\pgfqpoint{-0.454353in}{2.887961in}}%
\pgfpathlineto{\pgfqpoint{-0.454353in}{2.887961in}}%
\pgfpathclose%
\pgfusepath{stroke,fill}%
\end{pgfscope}%
\begin{pgfscope}%
\pgfpathrectangle{\pgfqpoint{0.050000in}{0.050000in}}{\pgfqpoint{2.419000in}{2.419000in}}%
\pgfusepath{clip}%
\pgfsetbuttcap%
\pgfsetroundjoin%
\definecolor{currentfill}{rgb}{0.866667,0.800000,0.466667}%
\pgfsetfillcolor{currentfill}%
\pgfsetfillopacity{0.366231}%
\pgfsetlinewidth{1.003750pt}%
\definecolor{currentstroke}{rgb}{0.866667,0.800000,0.466667}%
\pgfsetstrokecolor{currentstroke}%
\pgfsetstrokeopacity{0.366231}%
\pgfsetdash{}{0pt}%
\pgfpathmoveto{\pgfqpoint{3.443023in}{2.887961in}}%
\pgfpathcurveto{\pgfqpoint{3.452232in}{2.887961in}}{\pgfqpoint{3.461064in}{2.891620in}}{\pgfqpoint{3.467576in}{2.898131in}}%
\pgfpathcurveto{\pgfqpoint{3.474087in}{2.904643in}}{\pgfqpoint{3.477746in}{2.913475in}}{\pgfqpoint{3.477746in}{2.922684in}}%
\pgfpathcurveto{\pgfqpoint{3.477746in}{2.931892in}}{\pgfqpoint{3.474087in}{2.940725in}}{\pgfqpoint{3.467576in}{2.947236in}}%
\pgfpathcurveto{\pgfqpoint{3.461064in}{2.953747in}}{\pgfqpoint{3.452232in}{2.957406in}}{\pgfqpoint{3.443023in}{2.957406in}}%
\pgfpathcurveto{\pgfqpoint{3.433815in}{2.957406in}}{\pgfqpoint{3.424982in}{2.953747in}}{\pgfqpoint{3.418471in}{2.947236in}}%
\pgfpathcurveto{\pgfqpoint{3.411960in}{2.940725in}}{\pgfqpoint{3.408301in}{2.931892in}}{\pgfqpoint{3.408301in}{2.922684in}}%
\pgfpathcurveto{\pgfqpoint{3.408301in}{2.913475in}}{\pgfqpoint{3.411960in}{2.904643in}}{\pgfqpoint{3.418471in}{2.898131in}}%
\pgfpathcurveto{\pgfqpoint{3.424982in}{2.891620in}}{\pgfqpoint{3.433815in}{2.887961in}}{\pgfqpoint{3.443023in}{2.887961in}}%
\pgfpathlineto{\pgfqpoint{3.443023in}{2.887961in}}%
\pgfpathclose%
\pgfusepath{stroke,fill}%
\end{pgfscope}%
\begin{pgfscope}%
\pgfpathrectangle{\pgfqpoint{0.050000in}{0.050000in}}{\pgfqpoint{2.419000in}{2.419000in}}%
\pgfusepath{clip}%
\pgfsetbuttcap%
\pgfsetroundjoin%
\definecolor{currentfill}{rgb}{0.866667,0.800000,0.466667}%
\pgfsetfillcolor{currentfill}%
\pgfsetfillopacity{0.366231}%
\pgfsetlinewidth{1.003750pt}%
\definecolor{currentstroke}{rgb}{0.866667,0.800000,0.466667}%
\pgfsetstrokecolor{currentstroke}%
\pgfsetstrokeopacity{0.366231}%
\pgfsetdash{}{0pt}%
\pgfpathmoveto{\pgfqpoint{7.340400in}{2.887961in}}%
\pgfpathcurveto{\pgfqpoint{7.349608in}{2.887961in}}{\pgfqpoint{7.358441in}{2.891620in}}{\pgfqpoint{7.364952in}{2.898131in}}%
\pgfpathcurveto{\pgfqpoint{7.371463in}{2.904643in}}{\pgfqpoint{7.375122in}{2.913475in}}{\pgfqpoint{7.375122in}{2.922684in}}%
\pgfpathcurveto{\pgfqpoint{7.375122in}{2.931892in}}{\pgfqpoint{7.371463in}{2.940725in}}{\pgfqpoint{7.364952in}{2.947236in}}%
\pgfpathcurveto{\pgfqpoint{7.358441in}{2.953747in}}{\pgfqpoint{7.349608in}{2.957406in}}{\pgfqpoint{7.340400in}{2.957406in}}%
\pgfpathcurveto{\pgfqpoint{7.331191in}{2.957406in}}{\pgfqpoint{7.322359in}{2.953747in}}{\pgfqpoint{7.315847in}{2.947236in}}%
\pgfpathcurveto{\pgfqpoint{7.309336in}{2.940725in}}{\pgfqpoint{7.305677in}{2.931892in}}{\pgfqpoint{7.305677in}{2.922684in}}%
\pgfpathcurveto{\pgfqpoint{7.305677in}{2.913475in}}{\pgfqpoint{7.309336in}{2.904643in}}{\pgfqpoint{7.315847in}{2.898131in}}%
\pgfpathcurveto{\pgfqpoint{7.322359in}{2.891620in}}{\pgfqpoint{7.331191in}{2.887961in}}{\pgfqpoint{7.340400in}{2.887961in}}%
\pgfpathlineto{\pgfqpoint{7.340400in}{2.887961in}}%
\pgfpathclose%
\pgfusepath{stroke,fill}%
\end{pgfscope}%
\begin{pgfscope}%
\pgfpathrectangle{\pgfqpoint{0.050000in}{0.050000in}}{\pgfqpoint{2.419000in}{2.419000in}}%
\pgfusepath{clip}%
\pgfsetbuttcap%
\pgfsetroundjoin%
\definecolor{currentfill}{rgb}{0.866667,0.800000,0.466667}%
\pgfsetfillcolor{currentfill}%
\pgfsetfillopacity{0.370368}%
\pgfsetlinewidth{1.003750pt}%
\definecolor{currentstroke}{rgb}{0.866667,0.800000,0.466667}%
\pgfsetstrokecolor{currentstroke}%
\pgfsetstrokeopacity{0.370368}%
\pgfsetdash{}{0pt}%
\pgfpathmoveto{\pgfqpoint{1.867756in}{2.804095in}}%
\pgfpathcurveto{\pgfqpoint{1.876964in}{2.804095in}}{\pgfqpoint{1.885797in}{2.807753in}}{\pgfqpoint{1.892308in}{2.814264in}}%
\pgfpathcurveto{\pgfqpoint{1.898820in}{2.820776in}}{\pgfqpoint{1.902478in}{2.829608in}}{\pgfqpoint{1.902478in}{2.838817in}}%
\pgfpathcurveto{\pgfqpoint{1.902478in}{2.848025in}}{\pgfqpoint{1.898820in}{2.856858in}}{\pgfqpoint{1.892308in}{2.863369in}}%
\pgfpathcurveto{\pgfqpoint{1.885797in}{2.869880in}}{\pgfqpoint{1.876964in}{2.873539in}}{\pgfqpoint{1.867756in}{2.873539in}}%
\pgfpathcurveto{\pgfqpoint{1.858547in}{2.873539in}}{\pgfqpoint{1.849715in}{2.869880in}}{\pgfqpoint{1.843204in}{2.863369in}}%
\pgfpathcurveto{\pgfqpoint{1.836692in}{2.856858in}}{\pgfqpoint{1.833034in}{2.848025in}}{\pgfqpoint{1.833034in}{2.838817in}}%
\pgfpathcurveto{\pgfqpoint{1.833034in}{2.829608in}}{\pgfqpoint{1.836692in}{2.820776in}}{\pgfqpoint{1.843204in}{2.814264in}}%
\pgfpathcurveto{\pgfqpoint{1.849715in}{2.807753in}}{\pgfqpoint{1.858547in}{2.804095in}}{\pgfqpoint{1.867756in}{2.804095in}}%
\pgfpathlineto{\pgfqpoint{1.867756in}{2.804095in}}%
\pgfpathclose%
\pgfusepath{stroke,fill}%
\end{pgfscope}%
\begin{pgfscope}%
\pgfpathrectangle{\pgfqpoint{0.050000in}{0.050000in}}{\pgfqpoint{2.419000in}{2.419000in}}%
\pgfusepath{clip}%
\pgfsetbuttcap%
\pgfsetroundjoin%
\definecolor{currentfill}{rgb}{0.866667,0.800000,0.466667}%
\pgfsetfillcolor{currentfill}%
\pgfsetfillopacity{0.370368}%
\pgfsetlinewidth{1.003750pt}%
\definecolor{currentstroke}{rgb}{0.866667,0.800000,0.466667}%
\pgfsetstrokecolor{currentstroke}%
\pgfsetstrokeopacity{0.370368}%
\pgfsetdash{}{0pt}%
\pgfpathmoveto{\pgfqpoint{5.820357in}{2.804095in}}%
\pgfpathcurveto{\pgfqpoint{5.829565in}{2.804095in}}{\pgfqpoint{5.838398in}{2.807753in}}{\pgfqpoint{5.844909in}{2.814264in}}%
\pgfpathcurveto{\pgfqpoint{5.851420in}{2.820776in}}{\pgfqpoint{5.855079in}{2.829608in}}{\pgfqpoint{5.855079in}{2.838817in}}%
\pgfpathcurveto{\pgfqpoint{5.855079in}{2.848025in}}{\pgfqpoint{5.851420in}{2.856858in}}{\pgfqpoint{5.844909in}{2.863369in}}%
\pgfpathcurveto{\pgfqpoint{5.838398in}{2.869880in}}{\pgfqpoint{5.829565in}{2.873539in}}{\pgfqpoint{5.820357in}{2.873539in}}%
\pgfpathcurveto{\pgfqpoint{5.811148in}{2.873539in}}{\pgfqpoint{5.802316in}{2.869880in}}{\pgfqpoint{5.795804in}{2.863369in}}%
\pgfpathcurveto{\pgfqpoint{5.789293in}{2.856858in}}{\pgfqpoint{5.785634in}{2.848025in}}{\pgfqpoint{5.785634in}{2.838817in}}%
\pgfpathcurveto{\pgfqpoint{5.785634in}{2.829608in}}{\pgfqpoint{5.789293in}{2.820776in}}{\pgfqpoint{5.795804in}{2.814264in}}%
\pgfpathcurveto{\pgfqpoint{5.802316in}{2.807753in}}{\pgfqpoint{5.811148in}{2.804095in}}{\pgfqpoint{5.820357in}{2.804095in}}%
\pgfpathlineto{\pgfqpoint{5.820357in}{2.804095in}}%
\pgfpathclose%
\pgfusepath{stroke,fill}%
\end{pgfscope}%
\begin{pgfscope}%
\pgfpathrectangle{\pgfqpoint{0.050000in}{0.050000in}}{\pgfqpoint{2.419000in}{2.419000in}}%
\pgfusepath{clip}%
\pgfsetbuttcap%
\pgfsetroundjoin%
\definecolor{currentfill}{rgb}{0.866667,0.800000,0.466667}%
\pgfsetfillcolor{currentfill}%
\pgfsetfillopacity{0.370368}%
\pgfsetlinewidth{1.003750pt}%
\definecolor{currentstroke}{rgb}{0.866667,0.800000,0.466667}%
\pgfsetstrokecolor{currentstroke}%
\pgfsetstrokeopacity{0.370368}%
\pgfsetdash{}{0pt}%
\pgfpathmoveto{\pgfqpoint{9.772958in}{2.804095in}}%
\pgfpathcurveto{\pgfqpoint{9.782166in}{2.804095in}}{\pgfqpoint{9.790998in}{2.807753in}}{\pgfqpoint{9.797510in}{2.814264in}}%
\pgfpathcurveto{\pgfqpoint{9.804021in}{2.820776in}}{\pgfqpoint{9.807680in}{2.829608in}}{\pgfqpoint{9.807680in}{2.838817in}}%
\pgfpathcurveto{\pgfqpoint{9.807680in}{2.848025in}}{\pgfqpoint{9.804021in}{2.856858in}}{\pgfqpoint{9.797510in}{2.863369in}}%
\pgfpathcurveto{\pgfqpoint{9.790998in}{2.869880in}}{\pgfqpoint{9.782166in}{2.873539in}}{\pgfqpoint{9.772958in}{2.873539in}}%
\pgfpathcurveto{\pgfqpoint{9.763749in}{2.873539in}}{\pgfqpoint{9.754917in}{2.869880in}}{\pgfqpoint{9.748405in}{2.863369in}}%
\pgfpathcurveto{\pgfqpoint{9.741894in}{2.856858in}}{\pgfqpoint{9.738235in}{2.848025in}}{\pgfqpoint{9.738235in}{2.838817in}}%
\pgfpathcurveto{\pgfqpoint{9.738235in}{2.829608in}}{\pgfqpoint{9.741894in}{2.820776in}}{\pgfqpoint{9.748405in}{2.814264in}}%
\pgfpathcurveto{\pgfqpoint{9.754917in}{2.807753in}}{\pgfqpoint{9.763749in}{2.804095in}}{\pgfqpoint{9.772958in}{2.804095in}}%
\pgfpathlineto{\pgfqpoint{9.772958in}{2.804095in}}%
\pgfpathclose%
\pgfusepath{stroke,fill}%
\end{pgfscope}%
\begin{pgfscope}%
\pgfpathrectangle{\pgfqpoint{0.050000in}{0.050000in}}{\pgfqpoint{2.419000in}{2.419000in}}%
\pgfusepath{clip}%
\pgfsetbuttcap%
\pgfsetroundjoin%
\definecolor{currentfill}{rgb}{0.866667,0.800000,0.466667}%
\pgfsetfillcolor{currentfill}%
\pgfsetfillopacity{0.374624}%
\pgfsetlinewidth{1.003750pt}%
\definecolor{currentstroke}{rgb}{0.866667,0.800000,0.466667}%
\pgfsetstrokecolor{currentstroke}%
\pgfsetstrokeopacity{0.374624}%
\pgfsetdash{}{0pt}%
\pgfpathmoveto{\pgfqpoint{0.247205in}{2.717817in}}%
\pgfpathcurveto{\pgfqpoint{0.256413in}{2.717817in}}{\pgfqpoint{0.265246in}{2.721475in}}{\pgfqpoint{0.271757in}{2.727987in}}%
\pgfpathcurveto{\pgfqpoint{0.278268in}{2.734498in}}{\pgfqpoint{0.281927in}{2.743331in}}{\pgfqpoint{0.281927in}{2.752539in}}%
\pgfpathcurveto{\pgfqpoint{0.281927in}{2.761748in}}{\pgfqpoint{0.278268in}{2.770580in}}{\pgfqpoint{0.271757in}{2.777091in}}%
\pgfpathcurveto{\pgfqpoint{0.265246in}{2.783603in}}{\pgfqpoint{0.256413in}{2.787261in}}{\pgfqpoint{0.247205in}{2.787261in}}%
\pgfpathcurveto{\pgfqpoint{0.237996in}{2.787261in}}{\pgfqpoint{0.229164in}{2.783603in}}{\pgfqpoint{0.222652in}{2.777091in}}%
\pgfpathcurveto{\pgfqpoint{0.216141in}{2.770580in}}{\pgfqpoint{0.212483in}{2.761748in}}{\pgfqpoint{0.212483in}{2.752539in}}%
\pgfpathcurveto{\pgfqpoint{0.212483in}{2.743331in}}{\pgfqpoint{0.216141in}{2.734498in}}{\pgfqpoint{0.222652in}{2.727987in}}%
\pgfpathcurveto{\pgfqpoint{0.229164in}{2.721475in}}{\pgfqpoint{0.237996in}{2.717817in}}{\pgfqpoint{0.247205in}{2.717817in}}%
\pgfpathlineto{\pgfqpoint{0.247205in}{2.717817in}}%
\pgfpathclose%
\pgfusepath{stroke,fill}%
\end{pgfscope}%
\begin{pgfscope}%
\pgfpathrectangle{\pgfqpoint{0.050000in}{0.050000in}}{\pgfqpoint{2.419000in}{2.419000in}}%
\pgfusepath{clip}%
\pgfsetbuttcap%
\pgfsetroundjoin%
\definecolor{currentfill}{rgb}{0.866667,0.800000,0.466667}%
\pgfsetfillcolor{currentfill}%
\pgfsetfillopacity{0.374624}%
\pgfsetlinewidth{1.003750pt}%
\definecolor{currentstroke}{rgb}{0.866667,0.800000,0.466667}%
\pgfsetstrokecolor{currentstroke}%
\pgfsetstrokeopacity{0.374624}%
\pgfsetdash{}{0pt}%
\pgfpathmoveto{\pgfqpoint{4.256618in}{2.717817in}}%
\pgfpathcurveto{\pgfqpoint{4.265826in}{2.717817in}}{\pgfqpoint{4.274659in}{2.721475in}}{\pgfqpoint{4.281170in}{2.727987in}}%
\pgfpathcurveto{\pgfqpoint{4.287681in}{2.734498in}}{\pgfqpoint{4.291340in}{2.743331in}}{\pgfqpoint{4.291340in}{2.752539in}}%
\pgfpathcurveto{\pgfqpoint{4.291340in}{2.761748in}}{\pgfqpoint{4.287681in}{2.770580in}}{\pgfqpoint{4.281170in}{2.777091in}}%
\pgfpathcurveto{\pgfqpoint{4.274659in}{2.783603in}}{\pgfqpoint{4.265826in}{2.787261in}}{\pgfqpoint{4.256618in}{2.787261in}}%
\pgfpathcurveto{\pgfqpoint{4.247409in}{2.787261in}}{\pgfqpoint{4.238577in}{2.783603in}}{\pgfqpoint{4.232065in}{2.777091in}}%
\pgfpathcurveto{\pgfqpoint{4.225554in}{2.770580in}}{\pgfqpoint{4.221895in}{2.761748in}}{\pgfqpoint{4.221895in}{2.752539in}}%
\pgfpathcurveto{\pgfqpoint{4.221895in}{2.743331in}}{\pgfqpoint{4.225554in}{2.734498in}}{\pgfqpoint{4.232065in}{2.727987in}}%
\pgfpathcurveto{\pgfqpoint{4.238577in}{2.721475in}}{\pgfqpoint{4.247409in}{2.717817in}}{\pgfqpoint{4.256618in}{2.717817in}}%
\pgfpathlineto{\pgfqpoint{4.256618in}{2.717817in}}%
\pgfpathclose%
\pgfusepath{stroke,fill}%
\end{pgfscope}%
\begin{pgfscope}%
\pgfpathrectangle{\pgfqpoint{0.050000in}{0.050000in}}{\pgfqpoint{2.419000in}{2.419000in}}%
\pgfusepath{clip}%
\pgfsetbuttcap%
\pgfsetroundjoin%
\definecolor{currentfill}{rgb}{0.866667,0.800000,0.466667}%
\pgfsetfillcolor{currentfill}%
\pgfsetfillopacity{0.374624}%
\pgfsetlinewidth{1.003750pt}%
\definecolor{currentstroke}{rgb}{0.866667,0.800000,0.466667}%
\pgfsetstrokecolor{currentstroke}%
\pgfsetstrokeopacity{0.374624}%
\pgfsetdash{}{0pt}%
\pgfpathmoveto{\pgfqpoint{8.266031in}{2.717817in}}%
\pgfpathcurveto{\pgfqpoint{8.275239in}{2.717817in}}{\pgfqpoint{8.284072in}{2.721475in}}{\pgfqpoint{8.290583in}{2.727987in}}%
\pgfpathcurveto{\pgfqpoint{8.297094in}{2.734498in}}{\pgfqpoint{8.300753in}{2.743331in}}{\pgfqpoint{8.300753in}{2.752539in}}%
\pgfpathcurveto{\pgfqpoint{8.300753in}{2.761748in}}{\pgfqpoint{8.297094in}{2.770580in}}{\pgfqpoint{8.290583in}{2.777091in}}%
\pgfpathcurveto{\pgfqpoint{8.284072in}{2.783603in}}{\pgfqpoint{8.275239in}{2.787261in}}{\pgfqpoint{8.266031in}{2.787261in}}%
\pgfpathcurveto{\pgfqpoint{8.256822in}{2.787261in}}{\pgfqpoint{8.247990in}{2.783603in}}{\pgfqpoint{8.241478in}{2.777091in}}%
\pgfpathcurveto{\pgfqpoint{8.234967in}{2.770580in}}{\pgfqpoint{8.231308in}{2.761748in}}{\pgfqpoint{8.231308in}{2.752539in}}%
\pgfpathcurveto{\pgfqpoint{8.231308in}{2.743331in}}{\pgfqpoint{8.234967in}{2.734498in}}{\pgfqpoint{8.241478in}{2.727987in}}%
\pgfpathcurveto{\pgfqpoint{8.247990in}{2.721475in}}{\pgfqpoint{8.256822in}{2.717817in}}{\pgfqpoint{8.266031in}{2.717817in}}%
\pgfpathlineto{\pgfqpoint{8.266031in}{2.717817in}}%
\pgfpathclose%
\pgfusepath{stroke,fill}%
\end{pgfscope}%
\begin{pgfscope}%
\pgfpathrectangle{\pgfqpoint{0.050000in}{0.050000in}}{\pgfqpoint{2.419000in}{2.419000in}}%
\pgfusepath{clip}%
\pgfsetbuttcap%
\pgfsetroundjoin%
\definecolor{currentfill}{rgb}{0.866667,0.800000,0.466667}%
\pgfsetfillcolor{currentfill}%
\pgfsetfillopacity{0.379004}%
\pgfsetlinewidth{1.003750pt}%
\definecolor{currentstroke}{rgb}{0.866667,0.800000,0.466667}%
\pgfsetstrokecolor{currentstroke}%
\pgfsetstrokeopacity{0.379004}%
\pgfsetdash{}{0pt}%
\pgfpathmoveto{\pgfqpoint{2.647271in}{2.629023in}}%
\pgfpathcurveto{\pgfqpoint{2.656479in}{2.629023in}}{\pgfqpoint{2.665312in}{2.632681in}}{\pgfqpoint{2.671823in}{2.639193in}}%
\pgfpathcurveto{\pgfqpoint{2.678334in}{2.645704in}}{\pgfqpoint{2.681993in}{2.654537in}}{\pgfqpoint{2.681993in}{2.663745in}}%
\pgfpathcurveto{\pgfqpoint{2.681993in}{2.672953in}}{\pgfqpoint{2.678334in}{2.681786in}}{\pgfqpoint{2.671823in}{2.688297in}}%
\pgfpathcurveto{\pgfqpoint{2.665312in}{2.694809in}}{\pgfqpoint{2.656479in}{2.698467in}}{\pgfqpoint{2.647271in}{2.698467in}}%
\pgfpathcurveto{\pgfqpoint{2.638062in}{2.698467in}}{\pgfqpoint{2.629230in}{2.694809in}}{\pgfqpoint{2.622718in}{2.688297in}}%
\pgfpathcurveto{\pgfqpoint{2.616207in}{2.681786in}}{\pgfqpoint{2.612548in}{2.672953in}}{\pgfqpoint{2.612548in}{2.663745in}}%
\pgfpathcurveto{\pgfqpoint{2.612548in}{2.654537in}}{\pgfqpoint{2.616207in}{2.645704in}}{\pgfqpoint{2.622718in}{2.639193in}}%
\pgfpathcurveto{\pgfqpoint{2.629230in}{2.632681in}}{\pgfqpoint{2.638062in}{2.629023in}}{\pgfqpoint{2.647271in}{2.629023in}}%
\pgfpathlineto{\pgfqpoint{2.647271in}{2.629023in}}%
\pgfpathclose%
\pgfusepath{stroke,fill}%
\end{pgfscope}%
\begin{pgfscope}%
\pgfpathrectangle{\pgfqpoint{0.050000in}{0.050000in}}{\pgfqpoint{2.419000in}{2.419000in}}%
\pgfusepath{clip}%
\pgfsetbuttcap%
\pgfsetroundjoin%
\definecolor{currentfill}{rgb}{0.866667,0.800000,0.466667}%
\pgfsetfillcolor{currentfill}%
\pgfsetfillopacity{0.379004}%
\pgfsetlinewidth{1.003750pt}%
\definecolor{currentstroke}{rgb}{0.866667,0.800000,0.466667}%
\pgfsetstrokecolor{currentstroke}%
\pgfsetstrokeopacity{0.379004}%
\pgfsetdash{}{0pt}%
\pgfpathmoveto{\pgfqpoint{-1.420611in}{2.629023in}}%
\pgfpathcurveto{\pgfqpoint{-1.411403in}{2.629023in}}{\pgfqpoint{-1.402570in}{2.632681in}}{\pgfqpoint{-1.396059in}{2.639193in}}%
\pgfpathcurveto{\pgfqpoint{-1.389548in}{2.645704in}}{\pgfqpoint{-1.385889in}{2.654537in}}{\pgfqpoint{-1.385889in}{2.663745in}}%
\pgfpathcurveto{\pgfqpoint{-1.385889in}{2.672953in}}{\pgfqpoint{-1.389548in}{2.681786in}}{\pgfqpoint{-1.396059in}{2.688297in}}%
\pgfpathcurveto{\pgfqpoint{-1.402570in}{2.694809in}}{\pgfqpoint{-1.411403in}{2.698467in}}{\pgfqpoint{-1.420611in}{2.698467in}}%
\pgfpathcurveto{\pgfqpoint{-1.429820in}{2.698467in}}{\pgfqpoint{-1.438652in}{2.694809in}}{\pgfqpoint{-1.445164in}{2.688297in}}%
\pgfpathcurveto{\pgfqpoint{-1.451675in}{2.681786in}}{\pgfqpoint{-1.455334in}{2.672953in}}{\pgfqpoint{-1.455334in}{2.663745in}}%
\pgfpathcurveto{\pgfqpoint{-1.455334in}{2.654537in}}{\pgfqpoint{-1.451675in}{2.645704in}}{\pgfqpoint{-1.445164in}{2.639193in}}%
\pgfpathcurveto{\pgfqpoint{-1.438652in}{2.632681in}}{\pgfqpoint{-1.429820in}{2.629023in}}{\pgfqpoint{-1.420611in}{2.629023in}}%
\pgfpathlineto{\pgfqpoint{-1.420611in}{2.629023in}}%
\pgfpathclose%
\pgfusepath{stroke,fill}%
\end{pgfscope}%
\begin{pgfscope}%
\pgfpathrectangle{\pgfqpoint{0.050000in}{0.050000in}}{\pgfqpoint{2.419000in}{2.419000in}}%
\pgfusepath{clip}%
\pgfsetbuttcap%
\pgfsetroundjoin%
\definecolor{currentfill}{rgb}{0.866667,0.800000,0.466667}%
\pgfsetfillcolor{currentfill}%
\pgfsetfillopacity{0.379004}%
\pgfsetlinewidth{1.003750pt}%
\definecolor{currentstroke}{rgb}{0.866667,0.800000,0.466667}%
\pgfsetstrokecolor{currentstroke}%
\pgfsetstrokeopacity{0.379004}%
\pgfsetdash{}{0pt}%
\pgfpathmoveto{\pgfqpoint{6.715153in}{2.629023in}}%
\pgfpathcurveto{\pgfqpoint{6.724361in}{2.629023in}}{\pgfqpoint{6.733194in}{2.632681in}}{\pgfqpoint{6.739705in}{2.639193in}}%
\pgfpathcurveto{\pgfqpoint{6.746216in}{2.645704in}}{\pgfqpoint{6.749875in}{2.654537in}}{\pgfqpoint{6.749875in}{2.663745in}}%
\pgfpathcurveto{\pgfqpoint{6.749875in}{2.672953in}}{\pgfqpoint{6.746216in}{2.681786in}}{\pgfqpoint{6.739705in}{2.688297in}}%
\pgfpathcurveto{\pgfqpoint{6.733194in}{2.694809in}}{\pgfqpoint{6.724361in}{2.698467in}}{\pgfqpoint{6.715153in}{2.698467in}}%
\pgfpathcurveto{\pgfqpoint{6.705944in}{2.698467in}}{\pgfqpoint{6.697112in}{2.694809in}}{\pgfqpoint{6.690600in}{2.688297in}}%
\pgfpathcurveto{\pgfqpoint{6.684089in}{2.681786in}}{\pgfqpoint{6.680430in}{2.672953in}}{\pgfqpoint{6.680430in}{2.663745in}}%
\pgfpathcurveto{\pgfqpoint{6.680430in}{2.654537in}}{\pgfqpoint{6.684089in}{2.645704in}}{\pgfqpoint{6.690600in}{2.639193in}}%
\pgfpathcurveto{\pgfqpoint{6.697112in}{2.632681in}}{\pgfqpoint{6.705944in}{2.629023in}}{\pgfqpoint{6.715153in}{2.629023in}}%
\pgfpathlineto{\pgfqpoint{6.715153in}{2.629023in}}%
\pgfpathclose%
\pgfusepath{stroke,fill}%
\end{pgfscope}%
\begin{pgfscope}%
\pgfpathrectangle{\pgfqpoint{0.050000in}{0.050000in}}{\pgfqpoint{2.419000in}{2.419000in}}%
\pgfusepath{clip}%
\pgfsetbuttcap%
\pgfsetroundjoin%
\definecolor{currentfill}{rgb}{0.866667,0.800000,0.466667}%
\pgfsetfillcolor{currentfill}%
\pgfsetfillopacity{0.383514}%
\pgfsetlinewidth{1.003750pt}%
\definecolor{currentstroke}{rgb}{0.866667,0.800000,0.466667}%
\pgfsetstrokecolor{currentstroke}%
\pgfsetstrokeopacity{0.383514}%
\pgfsetdash{}{0pt}%
\pgfpathmoveto{\pgfqpoint{0.990291in}{2.537601in}}%
\pgfpathcurveto{\pgfqpoint{0.999500in}{2.537601in}}{\pgfqpoint{1.008332in}{2.541259in}}{\pgfqpoint{1.014843in}{2.547771in}}%
\pgfpathcurveto{\pgfqpoint{1.021355in}{2.554282in}}{\pgfqpoint{1.025013in}{2.563114in}}{\pgfqpoint{1.025013in}{2.572323in}}%
\pgfpathcurveto{\pgfqpoint{1.025013in}{2.581531in}}{\pgfqpoint{1.021355in}{2.590364in}}{\pgfqpoint{1.014843in}{2.596875in}}%
\pgfpathcurveto{\pgfqpoint{1.008332in}{2.603387in}}{\pgfqpoint{0.999500in}{2.607045in}}{\pgfqpoint{0.990291in}{2.607045in}}%
\pgfpathcurveto{\pgfqpoint{0.981083in}{2.607045in}}{\pgfqpoint{0.972250in}{2.603387in}}{\pgfqpoint{0.965739in}{2.596875in}}%
\pgfpathcurveto{\pgfqpoint{0.959227in}{2.590364in}}{\pgfqpoint{0.955569in}{2.581531in}}{\pgfqpoint{0.955569in}{2.572323in}}%
\pgfpathcurveto{\pgfqpoint{0.955569in}{2.563114in}}{\pgfqpoint{0.959227in}{2.554282in}}{\pgfqpoint{0.965739in}{2.547771in}}%
\pgfpathcurveto{\pgfqpoint{0.972250in}{2.541259in}}{\pgfqpoint{0.981083in}{2.537601in}}{\pgfqpoint{0.990291in}{2.537601in}}%
\pgfpathlineto{\pgfqpoint{0.990291in}{2.537601in}}%
\pgfpathclose%
\pgfusepath{stroke,fill}%
\end{pgfscope}%
\begin{pgfscope}%
\pgfpathrectangle{\pgfqpoint{0.050000in}{0.050000in}}{\pgfqpoint{2.419000in}{2.419000in}}%
\pgfusepath{clip}%
\pgfsetbuttcap%
\pgfsetroundjoin%
\definecolor{currentfill}{rgb}{0.866667,0.800000,0.466667}%
\pgfsetfillcolor{currentfill}%
\pgfsetfillopacity{0.383514}%
\pgfsetlinewidth{1.003750pt}%
\definecolor{currentstroke}{rgb}{0.866667,0.800000,0.466667}%
\pgfsetstrokecolor{currentstroke}%
\pgfsetstrokeopacity{0.383514}%
\pgfsetdash{}{0pt}%
\pgfpathmoveto{\pgfqpoint{9.246454in}{2.537601in}}%
\pgfpathcurveto{\pgfqpoint{9.255663in}{2.537601in}}{\pgfqpoint{9.264495in}{2.541259in}}{\pgfqpoint{9.271007in}{2.547771in}}%
\pgfpathcurveto{\pgfqpoint{9.277518in}{2.554282in}}{\pgfqpoint{9.281177in}{2.563114in}}{\pgfqpoint{9.281177in}{2.572323in}}%
\pgfpathcurveto{\pgfqpoint{9.281177in}{2.581531in}}{\pgfqpoint{9.277518in}{2.590364in}}{\pgfqpoint{9.271007in}{2.596875in}}%
\pgfpathcurveto{\pgfqpoint{9.264495in}{2.603387in}}{\pgfqpoint{9.255663in}{2.607045in}}{\pgfqpoint{9.246454in}{2.607045in}}%
\pgfpathcurveto{\pgfqpoint{9.237246in}{2.607045in}}{\pgfqpoint{9.228413in}{2.603387in}}{\pgfqpoint{9.221902in}{2.596875in}}%
\pgfpathcurveto{\pgfqpoint{9.215391in}{2.590364in}}{\pgfqpoint{9.211732in}{2.581531in}}{\pgfqpoint{9.211732in}{2.572323in}}%
\pgfpathcurveto{\pgfqpoint{9.211732in}{2.563114in}}{\pgfqpoint{9.215391in}{2.554282in}}{\pgfqpoint{9.221902in}{2.547771in}}%
\pgfpathcurveto{\pgfqpoint{9.228413in}{2.541259in}}{\pgfqpoint{9.237246in}{2.537601in}}{\pgfqpoint{9.246454in}{2.537601in}}%
\pgfpathlineto{\pgfqpoint{9.246454in}{2.537601in}}%
\pgfpathclose%
\pgfusepath{stroke,fill}%
\end{pgfscope}%
\begin{pgfscope}%
\pgfpathrectangle{\pgfqpoint{0.050000in}{0.050000in}}{\pgfqpoint{2.419000in}{2.419000in}}%
\pgfusepath{clip}%
\pgfsetbuttcap%
\pgfsetroundjoin%
\definecolor{currentfill}{rgb}{0.866667,0.800000,0.466667}%
\pgfsetfillcolor{currentfill}%
\pgfsetfillopacity{0.383514}%
\pgfsetlinewidth{1.003750pt}%
\definecolor{currentstroke}{rgb}{0.866667,0.800000,0.466667}%
\pgfsetstrokecolor{currentstroke}%
\pgfsetstrokeopacity{0.383514}%
\pgfsetdash{}{0pt}%
\pgfpathmoveto{\pgfqpoint{5.118373in}{2.537601in}}%
\pgfpathcurveto{\pgfqpoint{5.127581in}{2.537601in}}{\pgfqpoint{5.136414in}{2.541259in}}{\pgfqpoint{5.142925in}{2.547771in}}%
\pgfpathcurveto{\pgfqpoint{5.149436in}{2.554282in}}{\pgfqpoint{5.153095in}{2.563114in}}{\pgfqpoint{5.153095in}{2.572323in}}%
\pgfpathcurveto{\pgfqpoint{5.153095in}{2.581531in}}{\pgfqpoint{5.149436in}{2.590364in}}{\pgfqpoint{5.142925in}{2.596875in}}%
\pgfpathcurveto{\pgfqpoint{5.136414in}{2.603387in}}{\pgfqpoint{5.127581in}{2.607045in}}{\pgfqpoint{5.118373in}{2.607045in}}%
\pgfpathcurveto{\pgfqpoint{5.109164in}{2.607045in}}{\pgfqpoint{5.100332in}{2.603387in}}{\pgfqpoint{5.093820in}{2.596875in}}%
\pgfpathcurveto{\pgfqpoint{5.087309in}{2.590364in}}{\pgfqpoint{5.083650in}{2.581531in}}{\pgfqpoint{5.083650in}{2.572323in}}%
\pgfpathcurveto{\pgfqpoint{5.083650in}{2.563114in}}{\pgfqpoint{5.087309in}{2.554282in}}{\pgfqpoint{5.093820in}{2.547771in}}%
\pgfpathcurveto{\pgfqpoint{5.100332in}{2.541259in}}{\pgfqpoint{5.109164in}{2.537601in}}{\pgfqpoint{5.118373in}{2.537601in}}%
\pgfpathlineto{\pgfqpoint{5.118373in}{2.537601in}}%
\pgfpathclose%
\pgfusepath{stroke,fill}%
\end{pgfscope}%
\begin{pgfscope}%
\pgfpathrectangle{\pgfqpoint{0.050000in}{0.050000in}}{\pgfqpoint{2.419000in}{2.419000in}}%
\pgfusepath{clip}%
\pgfsetbuttcap%
\pgfsetroundjoin%
\definecolor{currentfill}{rgb}{0.866667,0.800000,0.466667}%
\pgfsetfillcolor{currentfill}%
\pgfsetfillopacity{0.388159}%
\pgfsetlinewidth{1.003750pt}%
\definecolor{currentstroke}{rgb}{0.866667,0.800000,0.466667}%
\pgfsetstrokecolor{currentstroke}%
\pgfsetstrokeopacity{0.388159}%
\pgfsetdash{}{0pt}%
\pgfpathmoveto{\pgfqpoint{3.473622in}{2.443432in}}%
\pgfpathcurveto{\pgfqpoint{3.482830in}{2.443432in}}{\pgfqpoint{3.491663in}{2.447091in}}{\pgfqpoint{3.498174in}{2.453602in}}%
\pgfpathcurveto{\pgfqpoint{3.504686in}{2.460113in}}{\pgfqpoint{3.508344in}{2.468946in}}{\pgfqpoint{3.508344in}{2.478154in}}%
\pgfpathcurveto{\pgfqpoint{3.508344in}{2.487363in}}{\pgfqpoint{3.504686in}{2.496195in}}{\pgfqpoint{3.498174in}{2.502707in}}%
\pgfpathcurveto{\pgfqpoint{3.491663in}{2.509218in}}{\pgfqpoint{3.482830in}{2.512876in}}{\pgfqpoint{3.473622in}{2.512876in}}%
\pgfpathcurveto{\pgfqpoint{3.464414in}{2.512876in}}{\pgfqpoint{3.455581in}{2.509218in}}{\pgfqpoint{3.449070in}{2.502707in}}%
\pgfpathcurveto{\pgfqpoint{3.442558in}{2.496195in}}{\pgfqpoint{3.438900in}{2.487363in}}{\pgfqpoint{3.438900in}{2.478154in}}%
\pgfpathcurveto{\pgfqpoint{3.438900in}{2.468946in}}{\pgfqpoint{3.442558in}{2.460113in}}{\pgfqpoint{3.449070in}{2.453602in}}%
\pgfpathcurveto{\pgfqpoint{3.455581in}{2.447091in}}{\pgfqpoint{3.464414in}{2.443432in}}{\pgfqpoint{3.473622in}{2.443432in}}%
\pgfpathlineto{\pgfqpoint{3.473622in}{2.443432in}}%
\pgfpathclose%
\pgfusepath{stroke,fill}%
\end{pgfscope}%
\begin{pgfscope}%
\pgfpathrectangle{\pgfqpoint{0.050000in}{0.050000in}}{\pgfqpoint{2.419000in}{2.419000in}}%
\pgfusepath{clip}%
\pgfsetbuttcap%
\pgfsetroundjoin%
\definecolor{currentfill}{rgb}{0.866667,0.800000,0.466667}%
\pgfsetfillcolor{currentfill}%
\pgfsetfillopacity{0.388159}%
\pgfsetlinewidth{1.003750pt}%
\definecolor{currentstroke}{rgb}{0.866667,0.800000,0.466667}%
\pgfsetstrokecolor{currentstroke}%
\pgfsetstrokeopacity{0.388159}%
\pgfsetdash{}{0pt}%
\pgfpathmoveto{\pgfqpoint{-0.716468in}{2.443432in}}%
\pgfpathcurveto{\pgfqpoint{-0.707259in}{2.443432in}}{\pgfqpoint{-0.698427in}{2.447091in}}{\pgfqpoint{-0.691915in}{2.453602in}}%
\pgfpathcurveto{\pgfqpoint{-0.685404in}{2.460113in}}{\pgfqpoint{-0.681745in}{2.468946in}}{\pgfqpoint{-0.681745in}{2.478154in}}%
\pgfpathcurveto{\pgfqpoint{-0.681745in}{2.487363in}}{\pgfqpoint{-0.685404in}{2.496195in}}{\pgfqpoint{-0.691915in}{2.502707in}}%
\pgfpathcurveto{\pgfqpoint{-0.698427in}{2.509218in}}{\pgfqpoint{-0.707259in}{2.512876in}}{\pgfqpoint{-0.716468in}{2.512876in}}%
\pgfpathcurveto{\pgfqpoint{-0.725676in}{2.512876in}}{\pgfqpoint{-0.734509in}{2.509218in}}{\pgfqpoint{-0.741020in}{2.502707in}}%
\pgfpathcurveto{\pgfqpoint{-0.747531in}{2.496195in}}{\pgfqpoint{-0.751190in}{2.487363in}}{\pgfqpoint{-0.751190in}{2.478154in}}%
\pgfpathcurveto{\pgfqpoint{-0.751190in}{2.468946in}}{\pgfqpoint{-0.747531in}{2.460113in}}{\pgfqpoint{-0.741020in}{2.453602in}}%
\pgfpathcurveto{\pgfqpoint{-0.734509in}{2.447091in}}{\pgfqpoint{-0.725676in}{2.443432in}}{\pgfqpoint{-0.716468in}{2.443432in}}%
\pgfpathlineto{\pgfqpoint{-0.716468in}{2.443432in}}%
\pgfpathclose%
\pgfusepath{stroke,fill}%
\end{pgfscope}%
\begin{pgfscope}%
\pgfpathrectangle{\pgfqpoint{0.050000in}{0.050000in}}{\pgfqpoint{2.419000in}{2.419000in}}%
\pgfusepath{clip}%
\pgfsetbuttcap%
\pgfsetroundjoin%
\definecolor{currentfill}{rgb}{0.866667,0.800000,0.466667}%
\pgfsetfillcolor{currentfill}%
\pgfsetfillopacity{0.388159}%
\pgfsetlinewidth{1.003750pt}%
\definecolor{currentstroke}{rgb}{0.866667,0.800000,0.466667}%
\pgfsetstrokecolor{currentstroke}%
\pgfsetstrokeopacity{0.388159}%
\pgfsetdash{}{0pt}%
\pgfpathmoveto{\pgfqpoint{7.663712in}{2.443432in}}%
\pgfpathcurveto{\pgfqpoint{7.672920in}{2.443432in}}{\pgfqpoint{7.681753in}{2.447091in}}{\pgfqpoint{7.688264in}{2.453602in}}%
\pgfpathcurveto{\pgfqpoint{7.694775in}{2.460113in}}{\pgfqpoint{7.698434in}{2.468946in}}{\pgfqpoint{7.698434in}{2.478154in}}%
\pgfpathcurveto{\pgfqpoint{7.698434in}{2.487363in}}{\pgfqpoint{7.694775in}{2.496195in}}{\pgfqpoint{7.688264in}{2.502707in}}%
\pgfpathcurveto{\pgfqpoint{7.681753in}{2.509218in}}{\pgfqpoint{7.672920in}{2.512876in}}{\pgfqpoint{7.663712in}{2.512876in}}%
\pgfpathcurveto{\pgfqpoint{7.654503in}{2.512876in}}{\pgfqpoint{7.645671in}{2.509218in}}{\pgfqpoint{7.639159in}{2.502707in}}%
\pgfpathcurveto{\pgfqpoint{7.632648in}{2.496195in}}{\pgfqpoint{7.628990in}{2.487363in}}{\pgfqpoint{7.628990in}{2.478154in}}%
\pgfpathcurveto{\pgfqpoint{7.628990in}{2.468946in}}{\pgfqpoint{7.632648in}{2.460113in}}{\pgfqpoint{7.639159in}{2.453602in}}%
\pgfpathcurveto{\pgfqpoint{7.645671in}{2.447091in}}{\pgfqpoint{7.654503in}{2.443432in}}{\pgfqpoint{7.663712in}{2.443432in}}%
\pgfpathlineto{\pgfqpoint{7.663712in}{2.443432in}}%
\pgfpathclose%
\pgfusepath{stroke,fill}%
\end{pgfscope}%
\begin{pgfscope}%
\pgfpathrectangle{\pgfqpoint{0.050000in}{0.050000in}}{\pgfqpoint{2.419000in}{2.419000in}}%
\pgfusepath{clip}%
\pgfsetbuttcap%
\pgfsetroundjoin%
\definecolor{currentfill}{rgb}{0.866667,0.800000,0.466667}%
\pgfsetfillcolor{currentfill}%
\pgfsetfillopacity{0.392946}%
\pgfsetlinewidth{1.003750pt}%
\definecolor{currentstroke}{rgb}{0.866667,0.800000,0.466667}%
\pgfsetstrokecolor{currentstroke}%
\pgfsetstrokeopacity{0.392946}%
\pgfsetdash{}{0pt}%
\pgfpathmoveto{\pgfqpoint{1.778706in}{2.346391in}}%
\pgfpathcurveto{\pgfqpoint{1.787915in}{2.346391in}}{\pgfqpoint{1.796747in}{2.350050in}}{\pgfqpoint{1.803258in}{2.356561in}}%
\pgfpathcurveto{\pgfqpoint{1.809770in}{2.363072in}}{\pgfqpoint{1.813428in}{2.371905in}}{\pgfqpoint{1.813428in}{2.381113in}}%
\pgfpathcurveto{\pgfqpoint{1.813428in}{2.390322in}}{\pgfqpoint{1.809770in}{2.399154in}}{\pgfqpoint{1.803258in}{2.405666in}}%
\pgfpathcurveto{\pgfqpoint{1.796747in}{2.412177in}}{\pgfqpoint{1.787915in}{2.415836in}}{\pgfqpoint{1.778706in}{2.415836in}}%
\pgfpathcurveto{\pgfqpoint{1.769498in}{2.415836in}}{\pgfqpoint{1.760665in}{2.412177in}}{\pgfqpoint{1.754154in}{2.405666in}}%
\pgfpathcurveto{\pgfqpoint{1.747642in}{2.399154in}}{\pgfqpoint{1.743984in}{2.390322in}}{\pgfqpoint{1.743984in}{2.381113in}}%
\pgfpathcurveto{\pgfqpoint{1.743984in}{2.371905in}}{\pgfqpoint{1.747642in}{2.363072in}}{\pgfqpoint{1.754154in}{2.356561in}}%
\pgfpathcurveto{\pgfqpoint{1.760665in}{2.350050in}}{\pgfqpoint{1.769498in}{2.346391in}}{\pgfqpoint{1.778706in}{2.346391in}}%
\pgfpathlineto{\pgfqpoint{1.778706in}{2.346391in}}%
\pgfpathclose%
\pgfusepath{stroke,fill}%
\end{pgfscope}%
\begin{pgfscope}%
\pgfpathrectangle{\pgfqpoint{0.050000in}{0.050000in}}{\pgfqpoint{2.419000in}{2.419000in}}%
\pgfusepath{clip}%
\pgfsetbuttcap%
\pgfsetroundjoin%
\definecolor{currentfill}{rgb}{0.866667,0.800000,0.466667}%
\pgfsetfillcolor{currentfill}%
\pgfsetfillopacity{0.392946}%
\pgfsetlinewidth{1.003750pt}%
\definecolor{currentstroke}{rgb}{0.866667,0.800000,0.466667}%
\pgfsetstrokecolor{currentstroke}%
\pgfsetstrokeopacity{0.392946}%
\pgfsetdash{}{0pt}%
\pgfpathmoveto{\pgfqpoint{10.286684in}{2.346391in}}%
\pgfpathcurveto{\pgfqpoint{10.295893in}{2.346391in}}{\pgfqpoint{10.304725in}{2.350050in}}{\pgfqpoint{10.311237in}{2.356561in}}%
\pgfpathcurveto{\pgfqpoint{10.317748in}{2.363072in}}{\pgfqpoint{10.321407in}{2.371905in}}{\pgfqpoint{10.321407in}{2.381113in}}%
\pgfpathcurveto{\pgfqpoint{10.321407in}{2.390322in}}{\pgfqpoint{10.317748in}{2.399154in}}{\pgfqpoint{10.311237in}{2.405666in}}%
\pgfpathcurveto{\pgfqpoint{10.304725in}{2.412177in}}{\pgfqpoint{10.295893in}{2.415836in}}{\pgfqpoint{10.286684in}{2.415836in}}%
\pgfpathcurveto{\pgfqpoint{10.277476in}{2.415836in}}{\pgfqpoint{10.268643in}{2.412177in}}{\pgfqpoint{10.262132in}{2.405666in}}%
\pgfpathcurveto{\pgfqpoint{10.255621in}{2.399154in}}{\pgfqpoint{10.251962in}{2.390322in}}{\pgfqpoint{10.251962in}{2.381113in}}%
\pgfpathcurveto{\pgfqpoint{10.251962in}{2.371905in}}{\pgfqpoint{10.255621in}{2.363072in}}{\pgfqpoint{10.262132in}{2.356561in}}%
\pgfpathcurveto{\pgfqpoint{10.268643in}{2.350050in}}{\pgfqpoint{10.277476in}{2.346391in}}{\pgfqpoint{10.286684in}{2.346391in}}%
\pgfpathlineto{\pgfqpoint{10.286684in}{2.346391in}}%
\pgfpathclose%
\pgfusepath{stroke,fill}%
\end{pgfscope}%
\begin{pgfscope}%
\pgfpathrectangle{\pgfqpoint{0.050000in}{0.050000in}}{\pgfqpoint{2.419000in}{2.419000in}}%
\pgfusepath{clip}%
\pgfsetbuttcap%
\pgfsetroundjoin%
\definecolor{currentfill}{rgb}{0.866667,0.800000,0.466667}%
\pgfsetfillcolor{currentfill}%
\pgfsetfillopacity{0.392946}%
\pgfsetlinewidth{1.003750pt}%
\definecolor{currentstroke}{rgb}{0.866667,0.800000,0.466667}%
\pgfsetstrokecolor{currentstroke}%
\pgfsetstrokeopacity{0.392946}%
\pgfsetdash{}{0pt}%
\pgfpathmoveto{\pgfqpoint{6.032695in}{2.346391in}}%
\pgfpathcurveto{\pgfqpoint{6.041904in}{2.346391in}}{\pgfqpoint{6.050736in}{2.350050in}}{\pgfqpoint{6.057248in}{2.356561in}}%
\pgfpathcurveto{\pgfqpoint{6.063759in}{2.363072in}}{\pgfqpoint{6.067417in}{2.371905in}}{\pgfqpoint{6.067417in}{2.381113in}}%
\pgfpathcurveto{\pgfqpoint{6.067417in}{2.390322in}}{\pgfqpoint{6.063759in}{2.399154in}}{\pgfqpoint{6.057248in}{2.405666in}}%
\pgfpathcurveto{\pgfqpoint{6.050736in}{2.412177in}}{\pgfqpoint{6.041904in}{2.415836in}}{\pgfqpoint{6.032695in}{2.415836in}}%
\pgfpathcurveto{\pgfqpoint{6.023487in}{2.415836in}}{\pgfqpoint{6.014654in}{2.412177in}}{\pgfqpoint{6.008143in}{2.405666in}}%
\pgfpathcurveto{\pgfqpoint{6.001632in}{2.399154in}}{\pgfqpoint{5.997973in}{2.390322in}}{\pgfqpoint{5.997973in}{2.381113in}}%
\pgfpathcurveto{\pgfqpoint{5.997973in}{2.371905in}}{\pgfqpoint{6.001632in}{2.363072in}}{\pgfqpoint{6.008143in}{2.356561in}}%
\pgfpathcurveto{\pgfqpoint{6.014654in}{2.350050in}}{\pgfqpoint{6.023487in}{2.346391in}}{\pgfqpoint{6.032695in}{2.346391in}}%
\pgfpathlineto{\pgfqpoint{6.032695in}{2.346391in}}%
\pgfpathclose%
\pgfusepath{stroke,fill}%
\end{pgfscope}%
\begin{pgfscope}%
\pgfpathrectangle{\pgfqpoint{0.050000in}{0.050000in}}{\pgfqpoint{2.419000in}{2.419000in}}%
\pgfusepath{clip}%
\pgfsetbuttcap%
\pgfsetroundjoin%
\definecolor{currentfill}{rgb}{0.866667,0.800000,0.466667}%
\pgfsetfillcolor{currentfill}%
\pgfsetfillopacity{0.397881}%
\pgfsetlinewidth{1.003750pt}%
\definecolor{currentstroke}{rgb}{0.866667,0.800000,0.466667}%
\pgfsetstrokecolor{currentstroke}%
\pgfsetstrokeopacity{0.397881}%
\pgfsetdash{}{0pt}%
\pgfpathmoveto{\pgfqpoint{4.351162in}{2.246345in}}%
\pgfpathcurveto{\pgfqpoint{4.360370in}{2.246345in}}{\pgfqpoint{4.369203in}{2.250003in}}{\pgfqpoint{4.375714in}{2.256515in}}%
\pgfpathcurveto{\pgfqpoint{4.382226in}{2.263026in}}{\pgfqpoint{4.385884in}{2.271858in}}{\pgfqpoint{4.385884in}{2.281067in}}%
\pgfpathcurveto{\pgfqpoint{4.385884in}{2.290275in}}{\pgfqpoint{4.382226in}{2.299108in}}{\pgfqpoint{4.375714in}{2.305619in}}%
\pgfpathcurveto{\pgfqpoint{4.369203in}{2.312131in}}{\pgfqpoint{4.360370in}{2.315789in}}{\pgfqpoint{4.351162in}{2.315789in}}%
\pgfpathcurveto{\pgfqpoint{4.341953in}{2.315789in}}{\pgfqpoint{4.333121in}{2.312131in}}{\pgfqpoint{4.326610in}{2.305619in}}%
\pgfpathcurveto{\pgfqpoint{4.320098in}{2.299108in}}{\pgfqpoint{4.316440in}{2.290275in}}{\pgfqpoint{4.316440in}{2.281067in}}%
\pgfpathcurveto{\pgfqpoint{4.316440in}{2.271858in}}{\pgfqpoint{4.320098in}{2.263026in}}{\pgfqpoint{4.326610in}{2.256515in}}%
\pgfpathcurveto{\pgfqpoint{4.333121in}{2.250003in}}{\pgfqpoint{4.341953in}{2.246345in}}{\pgfqpoint{4.351162in}{2.246345in}}%
\pgfpathlineto{\pgfqpoint{4.351162in}{2.246345in}}%
\pgfpathclose%
\pgfusepath{stroke,fill}%
\end{pgfscope}%
\begin{pgfscope}%
\pgfpathrectangle{\pgfqpoint{0.050000in}{0.050000in}}{\pgfqpoint{2.419000in}{2.419000in}}%
\pgfusepath{clip}%
\pgfsetbuttcap%
\pgfsetroundjoin%
\definecolor{currentfill}{rgb}{0.866667,0.800000,0.466667}%
\pgfsetfillcolor{currentfill}%
\pgfsetfillopacity{0.397881}%
\pgfsetlinewidth{1.003750pt}%
\definecolor{currentstroke}{rgb}{0.866667,0.800000,0.466667}%
\pgfsetstrokecolor{currentstroke}%
\pgfsetstrokeopacity{0.397881}%
\pgfsetdash{}{0pt}%
\pgfpathmoveto{\pgfqpoint{0.031294in}{2.246345in}}%
\pgfpathcurveto{\pgfqpoint{0.040503in}{2.246345in}}{\pgfqpoint{0.049335in}{2.250003in}}{\pgfqpoint{0.055846in}{2.256515in}}%
\pgfpathcurveto{\pgfqpoint{0.062358in}{2.263026in}}{\pgfqpoint{0.066016in}{2.271858in}}{\pgfqpoint{0.066016in}{2.281067in}}%
\pgfpathcurveto{\pgfqpoint{0.066016in}{2.290275in}}{\pgfqpoint{0.062358in}{2.299108in}}{\pgfqpoint{0.055846in}{2.305619in}}%
\pgfpathcurveto{\pgfqpoint{0.049335in}{2.312131in}}{\pgfqpoint{0.040503in}{2.315789in}}{\pgfqpoint{0.031294in}{2.315789in}}%
\pgfpathcurveto{\pgfqpoint{0.022086in}{2.315789in}}{\pgfqpoint{0.013253in}{2.312131in}}{\pgfqpoint{0.006742in}{2.305619in}}%
\pgfpathcurveto{\pgfqpoint{0.000231in}{2.299108in}}{\pgfqpoint{-0.003428in}{2.290275in}}{\pgfqpoint{-0.003428in}{2.281067in}}%
\pgfpathcurveto{\pgfqpoint{-0.003428in}{2.271858in}}{\pgfqpoint{0.000231in}{2.263026in}}{\pgfqpoint{0.006742in}{2.256515in}}%
\pgfpathcurveto{\pgfqpoint{0.013253in}{2.250003in}}{\pgfqpoint{0.022086in}{2.246345in}}{\pgfqpoint{0.031294in}{2.246345in}}%
\pgfpathlineto{\pgfqpoint{0.031294in}{2.246345in}}%
\pgfpathclose%
\pgfusepath{stroke,fill}%
\end{pgfscope}%
\begin{pgfscope}%
\pgfpathrectangle{\pgfqpoint{0.050000in}{0.050000in}}{\pgfqpoint{2.419000in}{2.419000in}}%
\pgfusepath{clip}%
\pgfsetbuttcap%
\pgfsetroundjoin%
\definecolor{currentfill}{rgb}{0.866667,0.800000,0.466667}%
\pgfsetfillcolor{currentfill}%
\pgfsetfillopacity{0.397881}%
\pgfsetlinewidth{1.003750pt}%
\definecolor{currentstroke}{rgb}{0.866667,0.800000,0.466667}%
\pgfsetstrokecolor{currentstroke}%
\pgfsetstrokeopacity{0.397881}%
\pgfsetdash{}{0pt}%
\pgfpathmoveto{\pgfqpoint{8.671030in}{2.246345in}}%
\pgfpathcurveto{\pgfqpoint{8.680238in}{2.246345in}}{\pgfqpoint{8.689071in}{2.250003in}}{\pgfqpoint{8.695582in}{2.256515in}}%
\pgfpathcurveto{\pgfqpoint{8.702093in}{2.263026in}}{\pgfqpoint{8.705752in}{2.271858in}}{\pgfqpoint{8.705752in}{2.281067in}}%
\pgfpathcurveto{\pgfqpoint{8.705752in}{2.290275in}}{\pgfqpoint{8.702093in}{2.299108in}}{\pgfqpoint{8.695582in}{2.305619in}}%
\pgfpathcurveto{\pgfqpoint{8.689071in}{2.312131in}}{\pgfqpoint{8.680238in}{2.315789in}}{\pgfqpoint{8.671030in}{2.315789in}}%
\pgfpathcurveto{\pgfqpoint{8.661821in}{2.315789in}}{\pgfqpoint{8.652989in}{2.312131in}}{\pgfqpoint{8.646477in}{2.305619in}}%
\pgfpathcurveto{\pgfqpoint{8.639966in}{2.299108in}}{\pgfqpoint{8.636307in}{2.290275in}}{\pgfqpoint{8.636307in}{2.281067in}}%
\pgfpathcurveto{\pgfqpoint{8.636307in}{2.271858in}}{\pgfqpoint{8.639966in}{2.263026in}}{\pgfqpoint{8.646477in}{2.256515in}}%
\pgfpathcurveto{\pgfqpoint{8.652989in}{2.250003in}}{\pgfqpoint{8.661821in}{2.246345in}}{\pgfqpoint{8.671030in}{2.246345in}}%
\pgfpathlineto{\pgfqpoint{8.671030in}{2.246345in}}%
\pgfpathclose%
\pgfusepath{stroke,fill}%
\end{pgfscope}%
\begin{pgfscope}%
\pgfpathrectangle{\pgfqpoint{0.050000in}{0.050000in}}{\pgfqpoint{2.419000in}{2.419000in}}%
\pgfusepath{clip}%
\pgfsetbuttcap%
\pgfsetroundjoin%
\definecolor{currentfill}{rgb}{0.866667,0.800000,0.466667}%
\pgfsetfillcolor{currentfill}%
\pgfsetfillopacity{0.402972}%
\pgfsetlinewidth{1.003750pt}%
\definecolor{currentstroke}{rgb}{0.866667,0.800000,0.466667}%
\pgfsetstrokecolor{currentstroke}%
\pgfsetstrokeopacity{0.402972}%
\pgfsetdash{}{0pt}%
\pgfpathmoveto{\pgfqpoint{-1.771091in}{2.143151in}}%
\pgfpathcurveto{\pgfqpoint{-1.761883in}{2.143151in}}{\pgfqpoint{-1.753050in}{2.146809in}}{\pgfqpoint{-1.746539in}{2.153321in}}%
\pgfpathcurveto{\pgfqpoint{-1.740027in}{2.159832in}}{\pgfqpoint{-1.736369in}{2.168665in}}{\pgfqpoint{-1.736369in}{2.177873in}}%
\pgfpathcurveto{\pgfqpoint{-1.736369in}{2.187081in}}{\pgfqpoint{-1.740027in}{2.195914in}}{\pgfqpoint{-1.746539in}{2.202425in}}%
\pgfpathcurveto{\pgfqpoint{-1.753050in}{2.208937in}}{\pgfqpoint{-1.761883in}{2.212595in}}{\pgfqpoint{-1.771091in}{2.212595in}}%
\pgfpathcurveto{\pgfqpoint{-1.780299in}{2.212595in}}{\pgfqpoint{-1.789132in}{2.208937in}}{\pgfqpoint{-1.795643in}{2.202425in}}%
\pgfpathcurveto{\pgfqpoint{-1.802155in}{2.195914in}}{\pgfqpoint{-1.805813in}{2.187081in}}{\pgfqpoint{-1.805813in}{2.177873in}}%
\pgfpathcurveto{\pgfqpoint{-1.805813in}{2.168665in}}{\pgfqpoint{-1.802155in}{2.159832in}}{\pgfqpoint{-1.795643in}{2.153321in}}%
\pgfpathcurveto{\pgfqpoint{-1.789132in}{2.146809in}}{\pgfqpoint{-1.780299in}{2.143151in}}{\pgfqpoint{-1.771091in}{2.143151in}}%
\pgfpathlineto{\pgfqpoint{-1.771091in}{2.143151in}}%
\pgfpathclose%
\pgfusepath{stroke,fill}%
\end{pgfscope}%
\begin{pgfscope}%
\pgfpathrectangle{\pgfqpoint{0.050000in}{0.050000in}}{\pgfqpoint{2.419000in}{2.419000in}}%
\pgfusepath{clip}%
\pgfsetbuttcap%
\pgfsetroundjoin%
\definecolor{currentfill}{rgb}{0.866667,0.800000,0.466667}%
\pgfsetfillcolor{currentfill}%
\pgfsetfillopacity{0.402972}%
\pgfsetlinewidth{1.003750pt}%
\definecolor{currentstroke}{rgb}{0.866667,0.800000,0.466667}%
\pgfsetstrokecolor{currentstroke}%
\pgfsetstrokeopacity{0.402972}%
\pgfsetdash{}{0pt}%
\pgfpathmoveto{\pgfqpoint{2.616728in}{2.143151in}}%
\pgfpathcurveto{\pgfqpoint{2.625936in}{2.143151in}}{\pgfqpoint{2.634769in}{2.146809in}}{\pgfqpoint{2.641280in}{2.153321in}}%
\pgfpathcurveto{\pgfqpoint{2.647791in}{2.159832in}}{\pgfqpoint{2.651450in}{2.168665in}}{\pgfqpoint{2.651450in}{2.177873in}}%
\pgfpathcurveto{\pgfqpoint{2.651450in}{2.187081in}}{\pgfqpoint{2.647791in}{2.195914in}}{\pgfqpoint{2.641280in}{2.202425in}}%
\pgfpathcurveto{\pgfqpoint{2.634769in}{2.208937in}}{\pgfqpoint{2.625936in}{2.212595in}}{\pgfqpoint{2.616728in}{2.212595in}}%
\pgfpathcurveto{\pgfqpoint{2.607519in}{2.212595in}}{\pgfqpoint{2.598687in}{2.208937in}}{\pgfqpoint{2.592176in}{2.202425in}}%
\pgfpathcurveto{\pgfqpoint{2.585664in}{2.195914in}}{\pgfqpoint{2.582006in}{2.187081in}}{\pgfqpoint{2.582006in}{2.177873in}}%
\pgfpathcurveto{\pgfqpoint{2.582006in}{2.168665in}}{\pgfqpoint{2.585664in}{2.159832in}}{\pgfqpoint{2.592176in}{2.153321in}}%
\pgfpathcurveto{\pgfqpoint{2.598687in}{2.146809in}}{\pgfqpoint{2.607519in}{2.143151in}}{\pgfqpoint{2.616728in}{2.143151in}}%
\pgfpathlineto{\pgfqpoint{2.616728in}{2.143151in}}%
\pgfpathclose%
\pgfusepath{stroke,fill}%
\end{pgfscope}%
\begin{pgfscope}%
\pgfpathrectangle{\pgfqpoint{0.050000in}{0.050000in}}{\pgfqpoint{2.419000in}{2.419000in}}%
\pgfusepath{clip}%
\pgfsetbuttcap%
\pgfsetroundjoin%
\definecolor{currentfill}{rgb}{0.866667,0.800000,0.466667}%
\pgfsetfillcolor{currentfill}%
\pgfsetfillopacity{0.402972}%
\pgfsetlinewidth{1.003750pt}%
\definecolor{currentstroke}{rgb}{0.866667,0.800000,0.466667}%
\pgfsetstrokecolor{currentstroke}%
\pgfsetstrokeopacity{0.402972}%
\pgfsetdash{}{0pt}%
\pgfpathmoveto{\pgfqpoint{7.004547in}{2.143151in}}%
\pgfpathcurveto{\pgfqpoint{7.013755in}{2.143151in}}{\pgfqpoint{7.022588in}{2.146809in}}{\pgfqpoint{7.029099in}{2.153321in}}%
\pgfpathcurveto{\pgfqpoint{7.035610in}{2.159832in}}{\pgfqpoint{7.039269in}{2.168665in}}{\pgfqpoint{7.039269in}{2.177873in}}%
\pgfpathcurveto{\pgfqpoint{7.039269in}{2.187081in}}{\pgfqpoint{7.035610in}{2.195914in}}{\pgfqpoint{7.029099in}{2.202425in}}%
\pgfpathcurveto{\pgfqpoint{7.022588in}{2.208937in}}{\pgfqpoint{7.013755in}{2.212595in}}{\pgfqpoint{7.004547in}{2.212595in}}%
\pgfpathcurveto{\pgfqpoint{6.995338in}{2.212595in}}{\pgfqpoint{6.986506in}{2.208937in}}{\pgfqpoint{6.979994in}{2.202425in}}%
\pgfpathcurveto{\pgfqpoint{6.973483in}{2.195914in}}{\pgfqpoint{6.969824in}{2.187081in}}{\pgfqpoint{6.969824in}{2.177873in}}%
\pgfpathcurveto{\pgfqpoint{6.969824in}{2.168665in}}{\pgfqpoint{6.973483in}{2.159832in}}{\pgfqpoint{6.979994in}{2.153321in}}%
\pgfpathcurveto{\pgfqpoint{6.986506in}{2.146809in}}{\pgfqpoint{6.995338in}{2.143151in}}{\pgfqpoint{7.004547in}{2.143151in}}%
\pgfpathlineto{\pgfqpoint{7.004547in}{2.143151in}}%
\pgfpathclose%
\pgfusepath{stroke,fill}%
\end{pgfscope}%
\begin{pgfscope}%
\pgfpathrectangle{\pgfqpoint{0.050000in}{0.050000in}}{\pgfqpoint{2.419000in}{2.419000in}}%
\pgfusepath{clip}%
\pgfsetbuttcap%
\pgfsetroundjoin%
\definecolor{currentfill}{rgb}{0.866667,0.800000,0.466667}%
\pgfsetfillcolor{currentfill}%
\pgfsetfillopacity{0.408225}%
\pgfsetlinewidth{1.003750pt}%
\definecolor{currentstroke}{rgb}{0.866667,0.800000,0.466667}%
\pgfsetstrokecolor{currentstroke}%
\pgfsetstrokeopacity{0.408225}%
\pgfsetdash{}{0pt}%
\pgfpathmoveto{\pgfqpoint{0.826857in}{2.036659in}}%
\pgfpathcurveto{\pgfqpoint{0.836065in}{2.036659in}}{\pgfqpoint{0.844898in}{2.040317in}}{\pgfqpoint{0.851409in}{2.046828in}}%
\pgfpathcurveto{\pgfqpoint{0.857920in}{2.053340in}}{\pgfqpoint{0.861579in}{2.062172in}}{\pgfqpoint{0.861579in}{2.071381in}}%
\pgfpathcurveto{\pgfqpoint{0.861579in}{2.080589in}}{\pgfqpoint{0.857920in}{2.089422in}}{\pgfqpoint{0.851409in}{2.095933in}}%
\pgfpathcurveto{\pgfqpoint{0.844898in}{2.102444in}}{\pgfqpoint{0.836065in}{2.106103in}}{\pgfqpoint{0.826857in}{2.106103in}}%
\pgfpathcurveto{\pgfqpoint{0.817648in}{2.106103in}}{\pgfqpoint{0.808816in}{2.102444in}}{\pgfqpoint{0.802304in}{2.095933in}}%
\pgfpathcurveto{\pgfqpoint{0.795793in}{2.089422in}}{\pgfqpoint{0.792135in}{2.080589in}}{\pgfqpoint{0.792135in}{2.071381in}}%
\pgfpathcurveto{\pgfqpoint{0.792135in}{2.062172in}}{\pgfqpoint{0.795793in}{2.053340in}}{\pgfqpoint{0.802304in}{2.046828in}}%
\pgfpathcurveto{\pgfqpoint{0.808816in}{2.040317in}}{\pgfqpoint{0.817648in}{2.036659in}}{\pgfqpoint{0.826857in}{2.036659in}}%
\pgfpathlineto{\pgfqpoint{0.826857in}{2.036659in}}%
\pgfpathclose%
\pgfusepath{stroke,fill}%
\end{pgfscope}%
\begin{pgfscope}%
\pgfpathrectangle{\pgfqpoint{0.050000in}{0.050000in}}{\pgfqpoint{2.419000in}{2.419000in}}%
\pgfusepath{clip}%
\pgfsetbuttcap%
\pgfsetroundjoin%
\definecolor{currentfill}{rgb}{0.866667,0.800000,0.466667}%
\pgfsetfillcolor{currentfill}%
\pgfsetfillopacity{0.408225}%
\pgfsetlinewidth{1.003750pt}%
\definecolor{currentstroke}{rgb}{0.866667,0.800000,0.466667}%
\pgfsetstrokecolor{currentstroke}%
\pgfsetstrokeopacity{0.408225}%
\pgfsetdash{}{0pt}%
\pgfpathmoveto{\pgfqpoint{5.284799in}{2.036659in}}%
\pgfpathcurveto{\pgfqpoint{5.294007in}{2.036659in}}{\pgfqpoint{5.302840in}{2.040317in}}{\pgfqpoint{5.309351in}{2.046828in}}%
\pgfpathcurveto{\pgfqpoint{5.315862in}{2.053340in}}{\pgfqpoint{5.319521in}{2.062172in}}{\pgfqpoint{5.319521in}{2.071381in}}%
\pgfpathcurveto{\pgfqpoint{5.319521in}{2.080589in}}{\pgfqpoint{5.315862in}{2.089422in}}{\pgfqpoint{5.309351in}{2.095933in}}%
\pgfpathcurveto{\pgfqpoint{5.302840in}{2.102444in}}{\pgfqpoint{5.294007in}{2.106103in}}{\pgfqpoint{5.284799in}{2.106103in}}%
\pgfpathcurveto{\pgfqpoint{5.275590in}{2.106103in}}{\pgfqpoint{5.266758in}{2.102444in}}{\pgfqpoint{5.260246in}{2.095933in}}%
\pgfpathcurveto{\pgfqpoint{5.253735in}{2.089422in}}{\pgfqpoint{5.250076in}{2.080589in}}{\pgfqpoint{5.250076in}{2.071381in}}%
\pgfpathcurveto{\pgfqpoint{5.250076in}{2.062172in}}{\pgfqpoint{5.253735in}{2.053340in}}{\pgfqpoint{5.260246in}{2.046828in}}%
\pgfpathcurveto{\pgfqpoint{5.266758in}{2.040317in}}{\pgfqpoint{5.275590in}{2.036659in}}{\pgfqpoint{5.284799in}{2.036659in}}%
\pgfpathlineto{\pgfqpoint{5.284799in}{2.036659in}}%
\pgfpathclose%
\pgfusepath{stroke,fill}%
\end{pgfscope}%
\begin{pgfscope}%
\pgfpathrectangle{\pgfqpoint{0.050000in}{0.050000in}}{\pgfqpoint{2.419000in}{2.419000in}}%
\pgfusepath{clip}%
\pgfsetbuttcap%
\pgfsetroundjoin%
\definecolor{currentfill}{rgb}{0.866667,0.800000,0.466667}%
\pgfsetfillcolor{currentfill}%
\pgfsetfillopacity{0.408225}%
\pgfsetlinewidth{1.003750pt}%
\definecolor{currentstroke}{rgb}{0.866667,0.800000,0.466667}%
\pgfsetstrokecolor{currentstroke}%
\pgfsetstrokeopacity{0.408225}%
\pgfsetdash{}{0pt}%
\pgfpathmoveto{\pgfqpoint{9.742740in}{2.036659in}}%
\pgfpathcurveto{\pgfqpoint{9.751949in}{2.036659in}}{\pgfqpoint{9.760781in}{2.040317in}}{\pgfqpoint{9.767293in}{2.046828in}}%
\pgfpathcurveto{\pgfqpoint{9.773804in}{2.053340in}}{\pgfqpoint{9.777463in}{2.062172in}}{\pgfqpoint{9.777463in}{2.071381in}}%
\pgfpathcurveto{\pgfqpoint{9.777463in}{2.080589in}}{\pgfqpoint{9.773804in}{2.089422in}}{\pgfqpoint{9.767293in}{2.095933in}}%
\pgfpathcurveto{\pgfqpoint{9.760781in}{2.102444in}}{\pgfqpoint{9.751949in}{2.106103in}}{\pgfqpoint{9.742740in}{2.106103in}}%
\pgfpathcurveto{\pgfqpoint{9.733532in}{2.106103in}}{\pgfqpoint{9.724699in}{2.102444in}}{\pgfqpoint{9.718188in}{2.095933in}}%
\pgfpathcurveto{\pgfqpoint{9.711677in}{2.089422in}}{\pgfqpoint{9.708018in}{2.080589in}}{\pgfqpoint{9.708018in}{2.071381in}}%
\pgfpathcurveto{\pgfqpoint{9.708018in}{2.062172in}}{\pgfqpoint{9.711677in}{2.053340in}}{\pgfqpoint{9.718188in}{2.046828in}}%
\pgfpathcurveto{\pgfqpoint{9.724699in}{2.040317in}}{\pgfqpoint{9.733532in}{2.036659in}}{\pgfqpoint{9.742740in}{2.036659in}}%
\pgfpathlineto{\pgfqpoint{9.742740in}{2.036659in}}%
\pgfpathclose%
\pgfusepath{stroke,fill}%
\end{pgfscope}%
\begin{pgfscope}%
\pgfpathrectangle{\pgfqpoint{0.050000in}{0.050000in}}{\pgfqpoint{2.419000in}{2.419000in}}%
\pgfusepath{clip}%
\pgfsetbuttcap%
\pgfsetroundjoin%
\definecolor{currentfill}{rgb}{0.866667,0.800000,0.466667}%
\pgfsetfillcolor{currentfill}%
\pgfsetfillopacity{0.413648}%
\pgfsetlinewidth{1.003750pt}%
\definecolor{currentstroke}{rgb}{0.866667,0.800000,0.466667}%
\pgfsetstrokecolor{currentstroke}%
\pgfsetstrokeopacity{0.413648}%
\pgfsetdash{}{0pt}%
\pgfpathmoveto{\pgfqpoint{3.509190in}{1.926707in}}%
\pgfpathcurveto{\pgfqpoint{3.518399in}{1.926707in}}{\pgfqpoint{3.527231in}{1.930366in}}{\pgfqpoint{3.533743in}{1.936877in}}%
\pgfpathcurveto{\pgfqpoint{3.540254in}{1.943389in}}{\pgfqpoint{3.543912in}{1.952221in}}{\pgfqpoint{3.543912in}{1.961430in}}%
\pgfpathcurveto{\pgfqpoint{3.543912in}{1.970638in}}{\pgfqpoint{3.540254in}{1.979470in}}{\pgfqpoint{3.533743in}{1.985982in}}%
\pgfpathcurveto{\pgfqpoint{3.527231in}{1.992493in}}{\pgfqpoint{3.518399in}{1.996152in}}{\pgfqpoint{3.509190in}{1.996152in}}%
\pgfpathcurveto{\pgfqpoint{3.499982in}{1.996152in}}{\pgfqpoint{3.491149in}{1.992493in}}{\pgfqpoint{3.484638in}{1.985982in}}%
\pgfpathcurveto{\pgfqpoint{3.478127in}{1.979470in}}{\pgfqpoint{3.474468in}{1.970638in}}{\pgfqpoint{3.474468in}{1.961430in}}%
\pgfpathcurveto{\pgfqpoint{3.474468in}{1.952221in}}{\pgfqpoint{3.478127in}{1.943389in}}{\pgfqpoint{3.484638in}{1.936877in}}%
\pgfpathcurveto{\pgfqpoint{3.491149in}{1.930366in}}{\pgfqpoint{3.499982in}{1.926707in}}{\pgfqpoint{3.509190in}{1.926707in}}%
\pgfpathlineto{\pgfqpoint{3.509190in}{1.926707in}}%
\pgfpathclose%
\pgfusepath{stroke,fill}%
\end{pgfscope}%
\begin{pgfscope}%
\pgfpathrectangle{\pgfqpoint{0.050000in}{0.050000in}}{\pgfqpoint{2.419000in}{2.419000in}}%
\pgfusepath{clip}%
\pgfsetbuttcap%
\pgfsetroundjoin%
\definecolor{currentfill}{rgb}{0.866667,0.800000,0.466667}%
\pgfsetfillcolor{currentfill}%
\pgfsetfillopacity{0.413648}%
\pgfsetlinewidth{1.003750pt}%
\definecolor{currentstroke}{rgb}{0.866667,0.800000,0.466667}%
\pgfsetstrokecolor{currentstroke}%
\pgfsetstrokeopacity{0.413648}%
\pgfsetdash{}{0pt}%
\pgfpathmoveto{\pgfqpoint{-1.021152in}{1.926707in}}%
\pgfpathcurveto{\pgfqpoint{-1.011944in}{1.926707in}}{\pgfqpoint{-1.003111in}{1.930366in}}{\pgfqpoint{-0.996600in}{1.936877in}}%
\pgfpathcurveto{\pgfqpoint{-0.990088in}{1.943389in}}{\pgfqpoint{-0.986430in}{1.952221in}}{\pgfqpoint{-0.986430in}{1.961430in}}%
\pgfpathcurveto{\pgfqpoint{-0.986430in}{1.970638in}}{\pgfqpoint{-0.990088in}{1.979470in}}{\pgfqpoint{-0.996600in}{1.985982in}}%
\pgfpathcurveto{\pgfqpoint{-1.003111in}{1.992493in}}{\pgfqpoint{-1.011944in}{1.996152in}}{\pgfqpoint{-1.021152in}{1.996152in}}%
\pgfpathcurveto{\pgfqpoint{-1.030361in}{1.996152in}}{\pgfqpoint{-1.039193in}{1.992493in}}{\pgfqpoint{-1.045704in}{1.985982in}}%
\pgfpathcurveto{\pgfqpoint{-1.052216in}{1.979470in}}{\pgfqpoint{-1.055874in}{1.970638in}}{\pgfqpoint{-1.055874in}{1.961430in}}%
\pgfpathcurveto{\pgfqpoint{-1.055874in}{1.952221in}}{\pgfqpoint{-1.052216in}{1.943389in}}{\pgfqpoint{-1.045704in}{1.936877in}}%
\pgfpathcurveto{\pgfqpoint{-1.039193in}{1.930366in}}{\pgfqpoint{-1.030361in}{1.926707in}}{\pgfqpoint{-1.021152in}{1.926707in}}%
\pgfpathlineto{\pgfqpoint{-1.021152in}{1.926707in}}%
\pgfpathclose%
\pgfusepath{stroke,fill}%
\end{pgfscope}%
\begin{pgfscope}%
\pgfpathrectangle{\pgfqpoint{0.050000in}{0.050000in}}{\pgfqpoint{2.419000in}{2.419000in}}%
\pgfusepath{clip}%
\pgfsetbuttcap%
\pgfsetroundjoin%
\definecolor{currentfill}{rgb}{0.866667,0.800000,0.466667}%
\pgfsetfillcolor{currentfill}%
\pgfsetfillopacity{0.413648}%
\pgfsetlinewidth{1.003750pt}%
\definecolor{currentstroke}{rgb}{0.866667,0.800000,0.466667}%
\pgfsetstrokecolor{currentstroke}%
\pgfsetstrokeopacity{0.413648}%
\pgfsetdash{}{0pt}%
\pgfpathmoveto{\pgfqpoint{8.039533in}{1.926707in}}%
\pgfpathcurveto{\pgfqpoint{8.048741in}{1.926707in}}{\pgfqpoint{8.057574in}{1.930366in}}{\pgfqpoint{8.064085in}{1.936877in}}%
\pgfpathcurveto{\pgfqpoint{8.070596in}{1.943389in}}{\pgfqpoint{8.074255in}{1.952221in}}{\pgfqpoint{8.074255in}{1.961430in}}%
\pgfpathcurveto{\pgfqpoint{8.074255in}{1.970638in}}{\pgfqpoint{8.070596in}{1.979470in}}{\pgfqpoint{8.064085in}{1.985982in}}%
\pgfpathcurveto{\pgfqpoint{8.057574in}{1.992493in}}{\pgfqpoint{8.048741in}{1.996152in}}{\pgfqpoint{8.039533in}{1.996152in}}%
\pgfpathcurveto{\pgfqpoint{8.030324in}{1.996152in}}{\pgfqpoint{8.021492in}{1.992493in}}{\pgfqpoint{8.014980in}{1.985982in}}%
\pgfpathcurveto{\pgfqpoint{8.008469in}{1.979470in}}{\pgfqpoint{8.004810in}{1.970638in}}{\pgfqpoint{8.004810in}{1.961430in}}%
\pgfpathcurveto{\pgfqpoint{8.004810in}{1.952221in}}{\pgfqpoint{8.008469in}{1.943389in}}{\pgfqpoint{8.014980in}{1.936877in}}%
\pgfpathcurveto{\pgfqpoint{8.021492in}{1.930366in}}{\pgfqpoint{8.030324in}{1.926707in}}{\pgfqpoint{8.039533in}{1.926707in}}%
\pgfpathlineto{\pgfqpoint{8.039533in}{1.926707in}}%
\pgfpathclose%
\pgfusepath{stroke,fill}%
\end{pgfscope}%
\begin{pgfscope}%
\pgfpathrectangle{\pgfqpoint{0.050000in}{0.050000in}}{\pgfqpoint{2.419000in}{2.419000in}}%
\pgfusepath{clip}%
\pgfsetbuttcap%
\pgfsetroundjoin%
\definecolor{currentfill}{rgb}{0.866667,0.800000,0.466667}%
\pgfsetfillcolor{currentfill}%
\pgfsetfillopacity{0.419251}%
\pgfsetlinewidth{1.003750pt}%
\definecolor{currentstroke}{rgb}{0.866667,0.800000,0.466667}%
\pgfsetstrokecolor{currentstroke}%
\pgfsetstrokeopacity{0.419251}%
\pgfsetdash{}{0pt}%
\pgfpathmoveto{\pgfqpoint{-2.930179in}{1.813126in}}%
\pgfpathcurveto{\pgfqpoint{-2.920970in}{1.813126in}}{\pgfqpoint{-2.912138in}{1.816784in}}{\pgfqpoint{-2.905626in}{1.823296in}}%
\pgfpathcurveto{\pgfqpoint{-2.899115in}{1.829807in}}{\pgfqpoint{-2.895456in}{1.838639in}}{\pgfqpoint{-2.895456in}{1.847848in}}%
\pgfpathcurveto{\pgfqpoint{-2.895456in}{1.857056in}}{\pgfqpoint{-2.899115in}{1.865889in}}{\pgfqpoint{-2.905626in}{1.872400in}}%
\pgfpathcurveto{\pgfqpoint{-2.912138in}{1.878912in}}{\pgfqpoint{-2.920970in}{1.882570in}}{\pgfqpoint{-2.930179in}{1.882570in}}%
\pgfpathcurveto{\pgfqpoint{-2.939387in}{1.882570in}}{\pgfqpoint{-2.948219in}{1.878912in}}{\pgfqpoint{-2.954731in}{1.872400in}}%
\pgfpathcurveto{\pgfqpoint{-2.961242in}{1.865889in}}{\pgfqpoint{-2.964901in}{1.857056in}}{\pgfqpoint{-2.964901in}{1.847848in}}%
\pgfpathcurveto{\pgfqpoint{-2.964901in}{1.838639in}}{\pgfqpoint{-2.961242in}{1.829807in}}{\pgfqpoint{-2.954731in}{1.823296in}}%
\pgfpathcurveto{\pgfqpoint{-2.948219in}{1.816784in}}{\pgfqpoint{-2.939387in}{1.813126in}}{\pgfqpoint{-2.930179in}{1.813126in}}%
\pgfpathlineto{\pgfqpoint{-2.930179in}{1.813126in}}%
\pgfpathclose%
\pgfusepath{stroke,fill}%
\end{pgfscope}%
\begin{pgfscope}%
\pgfpathrectangle{\pgfqpoint{0.050000in}{0.050000in}}{\pgfqpoint{2.419000in}{2.419000in}}%
\pgfusepath{clip}%
\pgfsetbuttcap%
\pgfsetroundjoin%
\definecolor{currentfill}{rgb}{0.866667,0.800000,0.466667}%
\pgfsetfillcolor{currentfill}%
\pgfsetfillopacity{0.419251}%
\pgfsetlinewidth{1.003750pt}%
\definecolor{currentstroke}{rgb}{0.866667,0.800000,0.466667}%
\pgfsetstrokecolor{currentstroke}%
\pgfsetstrokeopacity{0.419251}%
\pgfsetdash{}{0pt}%
\pgfpathmoveto{\pgfqpoint{1.674955in}{1.813126in}}%
\pgfpathcurveto{\pgfqpoint{1.684164in}{1.813126in}}{\pgfqpoint{1.692996in}{1.816784in}}{\pgfqpoint{1.699507in}{1.823296in}}%
\pgfpathcurveto{\pgfqpoint{1.706019in}{1.829807in}}{\pgfqpoint{1.709677in}{1.838639in}}{\pgfqpoint{1.709677in}{1.847848in}}%
\pgfpathcurveto{\pgfqpoint{1.709677in}{1.857056in}}{\pgfqpoint{1.706019in}{1.865889in}}{\pgfqpoint{1.699507in}{1.872400in}}%
\pgfpathcurveto{\pgfqpoint{1.692996in}{1.878912in}}{\pgfqpoint{1.684164in}{1.882570in}}{\pgfqpoint{1.674955in}{1.882570in}}%
\pgfpathcurveto{\pgfqpoint{1.665747in}{1.882570in}}{\pgfqpoint{1.656914in}{1.878912in}}{\pgfqpoint{1.650403in}{1.872400in}}%
\pgfpathcurveto{\pgfqpoint{1.643891in}{1.865889in}}{\pgfqpoint{1.640233in}{1.857056in}}{\pgfqpoint{1.640233in}{1.847848in}}%
\pgfpathcurveto{\pgfqpoint{1.640233in}{1.838639in}}{\pgfqpoint{1.643891in}{1.829807in}}{\pgfqpoint{1.650403in}{1.823296in}}%
\pgfpathcurveto{\pgfqpoint{1.656914in}{1.816784in}}{\pgfqpoint{1.665747in}{1.813126in}}{\pgfqpoint{1.674955in}{1.813126in}}%
\pgfpathlineto{\pgfqpoint{1.674955in}{1.813126in}}%
\pgfpathclose%
\pgfusepath{stroke,fill}%
\end{pgfscope}%
\begin{pgfscope}%
\pgfpathrectangle{\pgfqpoint{0.050000in}{0.050000in}}{\pgfqpoint{2.419000in}{2.419000in}}%
\pgfusepath{clip}%
\pgfsetbuttcap%
\pgfsetroundjoin%
\definecolor{currentfill}{rgb}{0.866667,0.800000,0.466667}%
\pgfsetfillcolor{currentfill}%
\pgfsetfillopacity{0.419251}%
\pgfsetlinewidth{1.003750pt}%
\definecolor{currentstroke}{rgb}{0.866667,0.800000,0.466667}%
\pgfsetstrokecolor{currentstroke}%
\pgfsetstrokeopacity{0.419251}%
\pgfsetdash{}{0pt}%
\pgfpathmoveto{\pgfqpoint{6.280089in}{1.813126in}}%
\pgfpathcurveto{\pgfqpoint{6.289297in}{1.813126in}}{\pgfqpoint{6.298130in}{1.816784in}}{\pgfqpoint{6.304641in}{1.823296in}}%
\pgfpathcurveto{\pgfqpoint{6.311152in}{1.829807in}}{\pgfqpoint{6.314811in}{1.838639in}}{\pgfqpoint{6.314811in}{1.847848in}}%
\pgfpathcurveto{\pgfqpoint{6.314811in}{1.857056in}}{\pgfqpoint{6.311152in}{1.865889in}}{\pgfqpoint{6.304641in}{1.872400in}}%
\pgfpathcurveto{\pgfqpoint{6.298130in}{1.878912in}}{\pgfqpoint{6.289297in}{1.882570in}}{\pgfqpoint{6.280089in}{1.882570in}}%
\pgfpathcurveto{\pgfqpoint{6.270880in}{1.882570in}}{\pgfqpoint{6.262048in}{1.878912in}}{\pgfqpoint{6.255536in}{1.872400in}}%
\pgfpathcurveto{\pgfqpoint{6.249025in}{1.865889in}}{\pgfqpoint{6.245366in}{1.857056in}}{\pgfqpoint{6.245366in}{1.847848in}}%
\pgfpathcurveto{\pgfqpoint{6.245366in}{1.838639in}}{\pgfqpoint{6.249025in}{1.829807in}}{\pgfqpoint{6.255536in}{1.823296in}}%
\pgfpathcurveto{\pgfqpoint{6.262048in}{1.816784in}}{\pgfqpoint{6.270880in}{1.813126in}}{\pgfqpoint{6.280089in}{1.813126in}}%
\pgfpathlineto{\pgfqpoint{6.280089in}{1.813126in}}%
\pgfpathclose%
\pgfusepath{stroke,fill}%
\end{pgfscope}%
\begin{pgfscope}%
\pgfpathrectangle{\pgfqpoint{0.050000in}{0.050000in}}{\pgfqpoint{2.419000in}{2.419000in}}%
\pgfusepath{clip}%
\pgfsetbuttcap%
\pgfsetroundjoin%
\definecolor{currentfill}{rgb}{0.866667,0.800000,0.466667}%
\pgfsetfillcolor{currentfill}%
\pgfsetfillopacity{0.425042}%
\pgfsetlinewidth{1.003750pt}%
\definecolor{currentstroke}{rgb}{0.866667,0.800000,0.466667}%
\pgfsetstrokecolor{currentstroke}%
\pgfsetstrokeopacity{0.425042}%
\pgfsetdash{}{0pt}%
\pgfpathmoveto{\pgfqpoint{4.461576in}{1.695731in}}%
\pgfpathcurveto{\pgfqpoint{4.470785in}{1.695731in}}{\pgfqpoint{4.479617in}{1.699389in}}{\pgfqpoint{4.486129in}{1.705901in}}%
\pgfpathcurveto{\pgfqpoint{4.492640in}{1.712412in}}{\pgfqpoint{4.496299in}{1.721245in}}{\pgfqpoint{4.496299in}{1.730453in}}%
\pgfpathcurveto{\pgfqpoint{4.496299in}{1.739662in}}{\pgfqpoint{4.492640in}{1.748494in}}{\pgfqpoint{4.486129in}{1.755005in}}%
\pgfpathcurveto{\pgfqpoint{4.479617in}{1.761517in}}{\pgfqpoint{4.470785in}{1.765175in}}{\pgfqpoint{4.461576in}{1.765175in}}%
\pgfpathcurveto{\pgfqpoint{4.452368in}{1.765175in}}{\pgfqpoint{4.443535in}{1.761517in}}{\pgfqpoint{4.437024in}{1.755005in}}%
\pgfpathcurveto{\pgfqpoint{4.430513in}{1.748494in}}{\pgfqpoint{4.426854in}{1.739662in}}{\pgfqpoint{4.426854in}{1.730453in}}%
\pgfpathcurveto{\pgfqpoint{4.426854in}{1.721245in}}{\pgfqpoint{4.430513in}{1.712412in}}{\pgfqpoint{4.437024in}{1.705901in}}%
\pgfpathcurveto{\pgfqpoint{4.443535in}{1.699389in}}{\pgfqpoint{4.452368in}{1.695731in}}{\pgfqpoint{4.461576in}{1.695731in}}%
\pgfpathlineto{\pgfqpoint{4.461576in}{1.695731in}}%
\pgfpathclose%
\pgfusepath{stroke,fill}%
\end{pgfscope}%
\begin{pgfscope}%
\pgfpathrectangle{\pgfqpoint{0.050000in}{0.050000in}}{\pgfqpoint{2.419000in}{2.419000in}}%
\pgfusepath{clip}%
\pgfsetbuttcap%
\pgfsetroundjoin%
\definecolor{currentfill}{rgb}{0.866667,0.800000,0.466667}%
\pgfsetfillcolor{currentfill}%
\pgfsetfillopacity{0.425042}%
\pgfsetlinewidth{1.003750pt}%
\definecolor{currentstroke}{rgb}{0.866667,0.800000,0.466667}%
\pgfsetstrokecolor{currentstroke}%
\pgfsetstrokeopacity{0.425042}%
\pgfsetdash{}{0pt}%
\pgfpathmoveto{\pgfqpoint{9.144012in}{1.695731in}}%
\pgfpathcurveto{\pgfqpoint{9.153220in}{1.695731in}}{\pgfqpoint{9.162053in}{1.699389in}}{\pgfqpoint{9.168564in}{1.705901in}}%
\pgfpathcurveto{\pgfqpoint{9.175076in}{1.712412in}}{\pgfqpoint{9.178734in}{1.721245in}}{\pgfqpoint{9.178734in}{1.730453in}}%
\pgfpathcurveto{\pgfqpoint{9.178734in}{1.739662in}}{\pgfqpoint{9.175076in}{1.748494in}}{\pgfqpoint{9.168564in}{1.755005in}}%
\pgfpathcurveto{\pgfqpoint{9.162053in}{1.761517in}}{\pgfqpoint{9.153220in}{1.765175in}}{\pgfqpoint{9.144012in}{1.765175in}}%
\pgfpathcurveto{\pgfqpoint{9.134804in}{1.765175in}}{\pgfqpoint{9.125971in}{1.761517in}}{\pgfqpoint{9.119460in}{1.755005in}}%
\pgfpathcurveto{\pgfqpoint{9.112948in}{1.748494in}}{\pgfqpoint{9.109290in}{1.739662in}}{\pgfqpoint{9.109290in}{1.730453in}}%
\pgfpathcurveto{\pgfqpoint{9.109290in}{1.721245in}}{\pgfqpoint{9.112948in}{1.712412in}}{\pgfqpoint{9.119460in}{1.705901in}}%
\pgfpathcurveto{\pgfqpoint{9.125971in}{1.699389in}}{\pgfqpoint{9.134804in}{1.695731in}}{\pgfqpoint{9.144012in}{1.695731in}}%
\pgfpathlineto{\pgfqpoint{9.144012in}{1.695731in}}%
\pgfpathclose%
\pgfusepath{stroke,fill}%
\end{pgfscope}%
\begin{pgfscope}%
\pgfpathrectangle{\pgfqpoint{0.050000in}{0.050000in}}{\pgfqpoint{2.419000in}{2.419000in}}%
\pgfusepath{clip}%
\pgfsetbuttcap%
\pgfsetroundjoin%
\definecolor{currentfill}{rgb}{0.866667,0.800000,0.466667}%
\pgfsetfillcolor{currentfill}%
\pgfsetfillopacity{0.425042}%
\pgfsetlinewidth{1.003750pt}%
\definecolor{currentstroke}{rgb}{0.866667,0.800000,0.466667}%
\pgfsetstrokecolor{currentstroke}%
\pgfsetstrokeopacity{0.425042}%
\pgfsetdash{}{0pt}%
\pgfpathmoveto{\pgfqpoint{-0.220859in}{1.695731in}}%
\pgfpathcurveto{\pgfqpoint{-0.211651in}{1.695731in}}{\pgfqpoint{-0.202818in}{1.699389in}}{\pgfqpoint{-0.196307in}{1.705901in}}%
\pgfpathcurveto{\pgfqpoint{-0.189796in}{1.712412in}}{\pgfqpoint{-0.186137in}{1.721245in}}{\pgfqpoint{-0.186137in}{1.730453in}}%
\pgfpathcurveto{\pgfqpoint{-0.186137in}{1.739662in}}{\pgfqpoint{-0.189796in}{1.748494in}}{\pgfqpoint{-0.196307in}{1.755005in}}%
\pgfpathcurveto{\pgfqpoint{-0.202818in}{1.761517in}}{\pgfqpoint{-0.211651in}{1.765175in}}{\pgfqpoint{-0.220859in}{1.765175in}}%
\pgfpathcurveto{\pgfqpoint{-0.230068in}{1.765175in}}{\pgfqpoint{-0.238900in}{1.761517in}}{\pgfqpoint{-0.245412in}{1.755005in}}%
\pgfpathcurveto{\pgfqpoint{-0.251923in}{1.748494in}}{\pgfqpoint{-0.255582in}{1.739662in}}{\pgfqpoint{-0.255582in}{1.730453in}}%
\pgfpathcurveto{\pgfqpoint{-0.255582in}{1.721245in}}{\pgfqpoint{-0.251923in}{1.712412in}}{\pgfqpoint{-0.245412in}{1.705901in}}%
\pgfpathcurveto{\pgfqpoint{-0.238900in}{1.699389in}}{\pgfqpoint{-0.230068in}{1.695731in}}{\pgfqpoint{-0.220859in}{1.695731in}}%
\pgfpathlineto{\pgfqpoint{-0.220859in}{1.695731in}}%
\pgfpathclose%
\pgfusepath{stroke,fill}%
\end{pgfscope}%
\begin{pgfscope}%
\pgfpathrectangle{\pgfqpoint{0.050000in}{0.050000in}}{\pgfqpoint{2.419000in}{2.419000in}}%
\pgfusepath{clip}%
\pgfsetbuttcap%
\pgfsetroundjoin%
\definecolor{currentfill}{rgb}{0.866667,0.800000,0.466667}%
\pgfsetfillcolor{currentfill}%
\pgfsetfillopacity{0.431031}%
\pgfsetlinewidth{1.003750pt}%
\definecolor{currentstroke}{rgb}{0.866667,0.800000,0.466667}%
\pgfsetstrokecolor{currentstroke}%
\pgfsetstrokeopacity{0.431031}%
\pgfsetdash{}{0pt}%
\pgfpathmoveto{\pgfqpoint{-2.181407in}{1.574328in}}%
\pgfpathcurveto{\pgfqpoint{-2.172198in}{1.574328in}}{\pgfqpoint{-2.163366in}{1.577986in}}{\pgfqpoint{-2.156854in}{1.584497in}}%
\pgfpathcurveto{\pgfqpoint{-2.150343in}{1.591009in}}{\pgfqpoint{-2.146685in}{1.599841in}}{\pgfqpoint{-2.146685in}{1.609050in}}%
\pgfpathcurveto{\pgfqpoint{-2.146685in}{1.618258in}}{\pgfqpoint{-2.150343in}{1.627091in}}{\pgfqpoint{-2.156854in}{1.633602in}}%
\pgfpathcurveto{\pgfqpoint{-2.163366in}{1.640113in}}{\pgfqpoint{-2.172198in}{1.643772in}}{\pgfqpoint{-2.181407in}{1.643772in}}%
\pgfpathcurveto{\pgfqpoint{-2.190615in}{1.643772in}}{\pgfqpoint{-2.199448in}{1.640113in}}{\pgfqpoint{-2.205959in}{1.633602in}}%
\pgfpathcurveto{\pgfqpoint{-2.212470in}{1.627091in}}{\pgfqpoint{-2.216129in}{1.618258in}}{\pgfqpoint{-2.216129in}{1.609050in}}%
\pgfpathcurveto{\pgfqpoint{-2.216129in}{1.599841in}}{\pgfqpoint{-2.212470in}{1.591009in}}{\pgfqpoint{-2.205959in}{1.584497in}}%
\pgfpathcurveto{\pgfqpoint{-2.199448in}{1.577986in}}{\pgfqpoint{-2.190615in}{1.574328in}}{\pgfqpoint{-2.181407in}{1.574328in}}%
\pgfpathlineto{\pgfqpoint{-2.181407in}{1.574328in}}%
\pgfpathclose%
\pgfusepath{stroke,fill}%
\end{pgfscope}%
\begin{pgfscope}%
\pgfpathrectangle{\pgfqpoint{0.050000in}{0.050000in}}{\pgfqpoint{2.419000in}{2.419000in}}%
\pgfusepath{clip}%
\pgfsetbuttcap%
\pgfsetroundjoin%
\definecolor{currentfill}{rgb}{0.866667,0.800000,0.466667}%
\pgfsetfillcolor{currentfill}%
\pgfsetfillopacity{0.431031}%
\pgfsetlinewidth{1.003750pt}%
\definecolor{currentstroke}{rgb}{0.866667,0.800000,0.466667}%
\pgfsetstrokecolor{currentstroke}%
\pgfsetstrokeopacity{0.431031}%
\pgfsetdash{}{0pt}%
\pgfpathmoveto{\pgfqpoint{2.580970in}{1.574328in}}%
\pgfpathcurveto{\pgfqpoint{2.590179in}{1.574328in}}{\pgfqpoint{2.599011in}{1.577986in}}{\pgfqpoint{2.605523in}{1.584497in}}%
\pgfpathcurveto{\pgfqpoint{2.612034in}{1.591009in}}{\pgfqpoint{2.615693in}{1.599841in}}{\pgfqpoint{2.615693in}{1.609050in}}%
\pgfpathcurveto{\pgfqpoint{2.615693in}{1.618258in}}{\pgfqpoint{2.612034in}{1.627091in}}{\pgfqpoint{2.605523in}{1.633602in}}%
\pgfpathcurveto{\pgfqpoint{2.599011in}{1.640113in}}{\pgfqpoint{2.590179in}{1.643772in}}{\pgfqpoint{2.580970in}{1.643772in}}%
\pgfpathcurveto{\pgfqpoint{2.571762in}{1.643772in}}{\pgfqpoint{2.562929in}{1.640113in}}{\pgfqpoint{2.556418in}{1.633602in}}%
\pgfpathcurveto{\pgfqpoint{2.549907in}{1.627091in}}{\pgfqpoint{2.546248in}{1.618258in}}{\pgfqpoint{2.546248in}{1.609050in}}%
\pgfpathcurveto{\pgfqpoint{2.546248in}{1.599841in}}{\pgfqpoint{2.549907in}{1.591009in}}{\pgfqpoint{2.556418in}{1.584497in}}%
\pgfpathcurveto{\pgfqpoint{2.562929in}{1.577986in}}{\pgfqpoint{2.571762in}{1.574328in}}{\pgfqpoint{2.580970in}{1.574328in}}%
\pgfpathlineto{\pgfqpoint{2.580970in}{1.574328in}}%
\pgfpathclose%
\pgfusepath{stroke,fill}%
\end{pgfscope}%
\begin{pgfscope}%
\pgfpathrectangle{\pgfqpoint{0.050000in}{0.050000in}}{\pgfqpoint{2.419000in}{2.419000in}}%
\pgfusepath{clip}%
\pgfsetbuttcap%
\pgfsetroundjoin%
\definecolor{currentfill}{rgb}{0.866667,0.800000,0.466667}%
\pgfsetfillcolor{currentfill}%
\pgfsetfillopacity{0.431031}%
\pgfsetlinewidth{1.003750pt}%
\definecolor{currentstroke}{rgb}{0.866667,0.800000,0.466667}%
\pgfsetstrokecolor{currentstroke}%
\pgfsetstrokeopacity{0.431031}%
\pgfsetdash{}{0pt}%
\pgfpathmoveto{\pgfqpoint{7.343348in}{1.574328in}}%
\pgfpathcurveto{\pgfqpoint{7.352556in}{1.574328in}}{\pgfqpoint{7.361389in}{1.577986in}}{\pgfqpoint{7.367900in}{1.584497in}}%
\pgfpathcurveto{\pgfqpoint{7.374411in}{1.591009in}}{\pgfqpoint{7.378070in}{1.599841in}}{\pgfqpoint{7.378070in}{1.609050in}}%
\pgfpathcurveto{\pgfqpoint{7.378070in}{1.618258in}}{\pgfqpoint{7.374411in}{1.627091in}}{\pgfqpoint{7.367900in}{1.633602in}}%
\pgfpathcurveto{\pgfqpoint{7.361389in}{1.640113in}}{\pgfqpoint{7.352556in}{1.643772in}}{\pgfqpoint{7.343348in}{1.643772in}}%
\pgfpathcurveto{\pgfqpoint{7.334139in}{1.643772in}}{\pgfqpoint{7.325307in}{1.640113in}}{\pgfqpoint{7.318795in}{1.633602in}}%
\pgfpathcurveto{\pgfqpoint{7.312284in}{1.627091in}}{\pgfqpoint{7.308625in}{1.618258in}}{\pgfqpoint{7.308625in}{1.609050in}}%
\pgfpathcurveto{\pgfqpoint{7.308625in}{1.599841in}}{\pgfqpoint{7.312284in}{1.591009in}}{\pgfqpoint{7.318795in}{1.584497in}}%
\pgfpathcurveto{\pgfqpoint{7.325307in}{1.577986in}}{\pgfqpoint{7.334139in}{1.574328in}}{\pgfqpoint{7.343348in}{1.574328in}}%
\pgfpathlineto{\pgfqpoint{7.343348in}{1.574328in}}%
\pgfpathclose%
\pgfusepath{stroke,fill}%
\end{pgfscope}%
\begin{pgfscope}%
\pgfpathrectangle{\pgfqpoint{0.050000in}{0.050000in}}{\pgfqpoint{2.419000in}{2.419000in}}%
\pgfusepath{clip}%
\pgfsetbuttcap%
\pgfsetroundjoin%
\definecolor{currentfill}{rgb}{0.866667,0.800000,0.466667}%
\pgfsetfillcolor{currentfill}%
\pgfsetfillopacity{0.437227}%
\pgfsetlinewidth{1.003750pt}%
\definecolor{currentstroke}{rgb}{0.866667,0.800000,0.466667}%
\pgfsetstrokecolor{currentstroke}%
\pgfsetstrokeopacity{0.437227}%
\pgfsetdash{}{0pt}%
\pgfpathmoveto{\pgfqpoint{0.635035in}{1.448707in}}%
\pgfpathcurveto{\pgfqpoint{0.644244in}{1.448707in}}{\pgfqpoint{0.653076in}{1.452365in}}{\pgfqpoint{0.659588in}{1.458877in}}%
\pgfpathcurveto{\pgfqpoint{0.666099in}{1.465388in}}{\pgfqpoint{0.669758in}{1.474221in}}{\pgfqpoint{0.669758in}{1.483429in}}%
\pgfpathcurveto{\pgfqpoint{0.669758in}{1.492638in}}{\pgfqpoint{0.666099in}{1.501470in}}{\pgfqpoint{0.659588in}{1.507981in}}%
\pgfpathcurveto{\pgfqpoint{0.653076in}{1.514493in}}{\pgfqpoint{0.644244in}{1.518151in}}{\pgfqpoint{0.635035in}{1.518151in}}%
\pgfpathcurveto{\pgfqpoint{0.625827in}{1.518151in}}{\pgfqpoint{0.616994in}{1.514493in}}{\pgfqpoint{0.610483in}{1.507981in}}%
\pgfpathcurveto{\pgfqpoint{0.603972in}{1.501470in}}{\pgfqpoint{0.600313in}{1.492638in}}{\pgfqpoint{0.600313in}{1.483429in}}%
\pgfpathcurveto{\pgfqpoint{0.600313in}{1.474221in}}{\pgfqpoint{0.603972in}{1.465388in}}{\pgfqpoint{0.610483in}{1.458877in}}%
\pgfpathcurveto{\pgfqpoint{0.616994in}{1.452365in}}{\pgfqpoint{0.625827in}{1.448707in}}{\pgfqpoint{0.635035in}{1.448707in}}%
\pgfpathlineto{\pgfqpoint{0.635035in}{1.448707in}}%
\pgfpathclose%
\pgfusepath{stroke,fill}%
\end{pgfscope}%
\begin{pgfscope}%
\pgfpathrectangle{\pgfqpoint{0.050000in}{0.050000in}}{\pgfqpoint{2.419000in}{2.419000in}}%
\pgfusepath{clip}%
\pgfsetbuttcap%
\pgfsetroundjoin%
\definecolor{currentfill}{rgb}{0.866667,0.800000,0.466667}%
\pgfsetfillcolor{currentfill}%
\pgfsetfillopacity{0.437227}%
\pgfsetlinewidth{1.003750pt}%
\definecolor{currentstroke}{rgb}{0.866667,0.800000,0.466667}%
\pgfsetstrokecolor{currentstroke}%
\pgfsetstrokeopacity{0.437227}%
\pgfsetdash{}{0pt}%
\pgfpathmoveto{\pgfqpoint{10.325227in}{1.448707in}}%
\pgfpathcurveto{\pgfqpoint{10.334436in}{1.448707in}}{\pgfqpoint{10.343268in}{1.452365in}}{\pgfqpoint{10.349779in}{1.458877in}}%
\pgfpathcurveto{\pgfqpoint{10.356291in}{1.465388in}}{\pgfqpoint{10.359949in}{1.474221in}}{\pgfqpoint{10.359949in}{1.483429in}}%
\pgfpathcurveto{\pgfqpoint{10.359949in}{1.492638in}}{\pgfqpoint{10.356291in}{1.501470in}}{\pgfqpoint{10.349779in}{1.507981in}}%
\pgfpathcurveto{\pgfqpoint{10.343268in}{1.514493in}}{\pgfqpoint{10.334436in}{1.518151in}}{\pgfqpoint{10.325227in}{1.518151in}}%
\pgfpathcurveto{\pgfqpoint{10.316019in}{1.518151in}}{\pgfqpoint{10.307186in}{1.514493in}}{\pgfqpoint{10.300675in}{1.507981in}}%
\pgfpathcurveto{\pgfqpoint{10.294163in}{1.501470in}}{\pgfqpoint{10.290505in}{1.492638in}}{\pgfqpoint{10.290505in}{1.483429in}}%
\pgfpathcurveto{\pgfqpoint{10.290505in}{1.474221in}}{\pgfqpoint{10.294163in}{1.465388in}}{\pgfqpoint{10.300675in}{1.458877in}}%
\pgfpathcurveto{\pgfqpoint{10.307186in}{1.452365in}}{\pgfqpoint{10.316019in}{1.448707in}}{\pgfqpoint{10.325227in}{1.448707in}}%
\pgfpathlineto{\pgfqpoint{10.325227in}{1.448707in}}%
\pgfpathclose%
\pgfusepath{stroke,fill}%
\end{pgfscope}%
\begin{pgfscope}%
\pgfpathrectangle{\pgfqpoint{0.050000in}{0.050000in}}{\pgfqpoint{2.419000in}{2.419000in}}%
\pgfusepath{clip}%
\pgfsetbuttcap%
\pgfsetroundjoin%
\definecolor{currentfill}{rgb}{0.866667,0.800000,0.466667}%
\pgfsetfillcolor{currentfill}%
\pgfsetfillopacity{0.437227}%
\pgfsetlinewidth{1.003750pt}%
\definecolor{currentstroke}{rgb}{0.866667,0.800000,0.466667}%
\pgfsetstrokecolor{currentstroke}%
\pgfsetstrokeopacity{0.437227}%
\pgfsetdash{}{0pt}%
\pgfpathmoveto{\pgfqpoint{5.480131in}{1.448707in}}%
\pgfpathcurveto{\pgfqpoint{5.489340in}{1.448707in}}{\pgfqpoint{5.498172in}{1.452365in}}{\pgfqpoint{5.504684in}{1.458877in}}%
\pgfpathcurveto{\pgfqpoint{5.511195in}{1.465388in}}{\pgfqpoint{5.514853in}{1.474221in}}{\pgfqpoint{5.514853in}{1.483429in}}%
\pgfpathcurveto{\pgfqpoint{5.514853in}{1.492638in}}{\pgfqpoint{5.511195in}{1.501470in}}{\pgfqpoint{5.504684in}{1.507981in}}%
\pgfpathcurveto{\pgfqpoint{5.498172in}{1.514493in}}{\pgfqpoint{5.489340in}{1.518151in}}{\pgfqpoint{5.480131in}{1.518151in}}%
\pgfpathcurveto{\pgfqpoint{5.470923in}{1.518151in}}{\pgfqpoint{5.462090in}{1.514493in}}{\pgfqpoint{5.455579in}{1.507981in}}%
\pgfpathcurveto{\pgfqpoint{5.449068in}{1.501470in}}{\pgfqpoint{5.445409in}{1.492638in}}{\pgfqpoint{5.445409in}{1.483429in}}%
\pgfpathcurveto{\pgfqpoint{5.445409in}{1.474221in}}{\pgfqpoint{5.449068in}{1.465388in}}{\pgfqpoint{5.455579in}{1.458877in}}%
\pgfpathcurveto{\pgfqpoint{5.462090in}{1.452365in}}{\pgfqpoint{5.470923in}{1.448707in}}{\pgfqpoint{5.480131in}{1.448707in}}%
\pgfpathlineto{\pgfqpoint{5.480131in}{1.448707in}}%
\pgfpathclose%
\pgfusepath{stroke,fill}%
\end{pgfscope}%
\begin{pgfscope}%
\pgfpathrectangle{\pgfqpoint{0.050000in}{0.050000in}}{\pgfqpoint{2.419000in}{2.419000in}}%
\pgfusepath{clip}%
\pgfsetbuttcap%
\pgfsetroundjoin%
\definecolor{currentfill}{rgb}{0.866667,0.800000,0.466667}%
\pgfsetfillcolor{currentfill}%
\pgfsetfillopacity{0.443643}%
\pgfsetlinewidth{1.003750pt}%
\definecolor{currentstroke}{rgb}{0.866667,0.800000,0.466667}%
\pgfsetstrokecolor{currentstroke}%
\pgfsetstrokeopacity{0.443643}%
\pgfsetdash{}{0pt}%
\pgfpathmoveto{\pgfqpoint{8.481784in}{1.318645in}}%
\pgfpathcurveto{\pgfqpoint{8.490993in}{1.318645in}}{\pgfqpoint{8.499825in}{1.322304in}}{\pgfqpoint{8.506337in}{1.328815in}}%
\pgfpathcurveto{\pgfqpoint{8.512848in}{1.335327in}}{\pgfqpoint{8.516507in}{1.344159in}}{\pgfqpoint{8.516507in}{1.353368in}}%
\pgfpathcurveto{\pgfqpoint{8.516507in}{1.362576in}}{\pgfqpoint{8.512848in}{1.371408in}}{\pgfqpoint{8.506337in}{1.377920in}}%
\pgfpathcurveto{\pgfqpoint{8.499825in}{1.384431in}}{\pgfqpoint{8.490993in}{1.388090in}}{\pgfqpoint{8.481784in}{1.388090in}}%
\pgfpathcurveto{\pgfqpoint{8.472576in}{1.388090in}}{\pgfqpoint{8.463743in}{1.384431in}}{\pgfqpoint{8.457232in}{1.377920in}}%
\pgfpathcurveto{\pgfqpoint{8.450721in}{1.371408in}}{\pgfqpoint{8.447062in}{1.362576in}}{\pgfqpoint{8.447062in}{1.353368in}}%
\pgfpathcurveto{\pgfqpoint{8.447062in}{1.344159in}}{\pgfqpoint{8.450721in}{1.335327in}}{\pgfqpoint{8.457232in}{1.328815in}}%
\pgfpathcurveto{\pgfqpoint{8.463743in}{1.322304in}}{\pgfqpoint{8.472576in}{1.318645in}}{\pgfqpoint{8.481784in}{1.318645in}}%
\pgfpathlineto{\pgfqpoint{8.481784in}{1.318645in}}%
\pgfpathclose%
\pgfusepath{stroke,fill}%
\end{pgfscope}%
\begin{pgfscope}%
\pgfpathrectangle{\pgfqpoint{0.050000in}{0.050000in}}{\pgfqpoint{2.419000in}{2.419000in}}%
\pgfusepath{clip}%
\pgfsetbuttcap%
\pgfsetroundjoin%
\definecolor{currentfill}{rgb}{0.866667,0.800000,0.466667}%
\pgfsetfillcolor{currentfill}%
\pgfsetfillopacity{0.443643}%
\pgfsetlinewidth{1.003750pt}%
\definecolor{currentstroke}{rgb}{0.866667,0.800000,0.466667}%
\pgfsetstrokecolor{currentstroke}%
\pgfsetstrokeopacity{0.443643}%
\pgfsetdash{}{0pt}%
\pgfpathmoveto{\pgfqpoint{-1.379693in}{1.318645in}}%
\pgfpathcurveto{\pgfqpoint{-1.370485in}{1.318645in}}{\pgfqpoint{-1.361652in}{1.322304in}}{\pgfqpoint{-1.355141in}{1.328815in}}%
\pgfpathcurveto{\pgfqpoint{-1.348630in}{1.335327in}}{\pgfqpoint{-1.344971in}{1.344159in}}{\pgfqpoint{-1.344971in}{1.353368in}}%
\pgfpathcurveto{\pgfqpoint{-1.344971in}{1.362576in}}{\pgfqpoint{-1.348630in}{1.371408in}}{\pgfqpoint{-1.355141in}{1.377920in}}%
\pgfpathcurveto{\pgfqpoint{-1.361652in}{1.384431in}}{\pgfqpoint{-1.370485in}{1.388090in}}{\pgfqpoint{-1.379693in}{1.388090in}}%
\pgfpathcurveto{\pgfqpoint{-1.388902in}{1.388090in}}{\pgfqpoint{-1.397734in}{1.384431in}}{\pgfqpoint{-1.404246in}{1.377920in}}%
\pgfpathcurveto{\pgfqpoint{-1.410757in}{1.371408in}}{\pgfqpoint{-1.414416in}{1.362576in}}{\pgfqpoint{-1.414416in}{1.353368in}}%
\pgfpathcurveto{\pgfqpoint{-1.414416in}{1.344159in}}{\pgfqpoint{-1.410757in}{1.335327in}}{\pgfqpoint{-1.404246in}{1.328815in}}%
\pgfpathcurveto{\pgfqpoint{-1.397734in}{1.322304in}}{\pgfqpoint{-1.388902in}{1.318645in}}{\pgfqpoint{-1.379693in}{1.318645in}}%
\pgfpathlineto{\pgfqpoint{-1.379693in}{1.318645in}}%
\pgfpathclose%
\pgfusepath{stroke,fill}%
\end{pgfscope}%
\begin{pgfscope}%
\pgfpathrectangle{\pgfqpoint{0.050000in}{0.050000in}}{\pgfqpoint{2.419000in}{2.419000in}}%
\pgfusepath{clip}%
\pgfsetbuttcap%
\pgfsetroundjoin%
\definecolor{currentfill}{rgb}{0.866667,0.800000,0.466667}%
\pgfsetfillcolor{currentfill}%
\pgfsetfillopacity{0.443643}%
\pgfsetlinewidth{1.003750pt}%
\definecolor{currentstroke}{rgb}{0.866667,0.800000,0.466667}%
\pgfsetstrokecolor{currentstroke}%
\pgfsetstrokeopacity{0.443643}%
\pgfsetdash{}{0pt}%
\pgfpathmoveto{\pgfqpoint{3.551046in}{1.318645in}}%
\pgfpathcurveto{\pgfqpoint{3.560254in}{1.318645in}}{\pgfqpoint{3.569086in}{1.322304in}}{\pgfqpoint{3.575598in}{1.328815in}}%
\pgfpathcurveto{\pgfqpoint{3.582109in}{1.335327in}}{\pgfqpoint{3.585768in}{1.344159in}}{\pgfqpoint{3.585768in}{1.353368in}}%
\pgfpathcurveto{\pgfqpoint{3.585768in}{1.362576in}}{\pgfqpoint{3.582109in}{1.371408in}}{\pgfqpoint{3.575598in}{1.377920in}}%
\pgfpathcurveto{\pgfqpoint{3.569086in}{1.384431in}}{\pgfqpoint{3.560254in}{1.388090in}}{\pgfqpoint{3.551046in}{1.388090in}}%
\pgfpathcurveto{\pgfqpoint{3.541837in}{1.388090in}}{\pgfqpoint{3.533005in}{1.384431in}}{\pgfqpoint{3.526493in}{1.377920in}}%
\pgfpathcurveto{\pgfqpoint{3.519982in}{1.371408in}}{\pgfqpoint{3.516323in}{1.362576in}}{\pgfqpoint{3.516323in}{1.353368in}}%
\pgfpathcurveto{\pgfqpoint{3.516323in}{1.344159in}}{\pgfqpoint{3.519982in}{1.335327in}}{\pgfqpoint{3.526493in}{1.328815in}}%
\pgfpathcurveto{\pgfqpoint{3.533005in}{1.322304in}}{\pgfqpoint{3.541837in}{1.318645in}}{\pgfqpoint{3.551046in}{1.318645in}}%
\pgfpathlineto{\pgfqpoint{3.551046in}{1.318645in}}%
\pgfpathclose%
\pgfusepath{stroke,fill}%
\end{pgfscope}%
\begin{pgfscope}%
\pgfpathrectangle{\pgfqpoint{0.050000in}{0.050000in}}{\pgfqpoint{2.419000in}{2.419000in}}%
\pgfusepath{clip}%
\pgfsetbuttcap%
\pgfsetroundjoin%
\definecolor{currentfill}{rgb}{0.866667,0.800000,0.466667}%
\pgfsetfillcolor{currentfill}%
\pgfsetfillopacity{0.450290}%
\pgfsetlinewidth{1.003750pt}%
\definecolor{currentstroke}{rgb}{0.866667,0.800000,0.466667}%
\pgfsetstrokecolor{currentstroke}%
\pgfsetstrokeopacity{0.450290}%
\pgfsetdash{}{0pt}%
\pgfpathmoveto{\pgfqpoint{-3.466929in}{1.183903in}}%
\pgfpathcurveto{\pgfqpoint{-3.457721in}{1.183903in}}{\pgfqpoint{-3.448888in}{1.187561in}}{\pgfqpoint{-3.442377in}{1.194073in}}%
\pgfpathcurveto{\pgfqpoint{-3.435865in}{1.200584in}}{\pgfqpoint{-3.432207in}{1.209417in}}{\pgfqpoint{-3.432207in}{1.218625in}}%
\pgfpathcurveto{\pgfqpoint{-3.432207in}{1.227834in}}{\pgfqpoint{-3.435865in}{1.236666in}}{\pgfqpoint{-3.442377in}{1.243177in}}%
\pgfpathcurveto{\pgfqpoint{-3.448888in}{1.249689in}}{\pgfqpoint{-3.457721in}{1.253347in}}{\pgfqpoint{-3.466929in}{1.253347in}}%
\pgfpathcurveto{\pgfqpoint{-3.476138in}{1.253347in}}{\pgfqpoint{-3.484970in}{1.249689in}}{\pgfqpoint{-3.491481in}{1.243177in}}%
\pgfpathcurveto{\pgfqpoint{-3.497993in}{1.236666in}}{\pgfqpoint{-3.501651in}{1.227834in}}{\pgfqpoint{-3.501651in}{1.218625in}}%
\pgfpathcurveto{\pgfqpoint{-3.501651in}{1.209417in}}{\pgfqpoint{-3.497993in}{1.200584in}}{\pgfqpoint{-3.491481in}{1.194073in}}%
\pgfpathcurveto{\pgfqpoint{-3.484970in}{1.187561in}}{\pgfqpoint{-3.476138in}{1.183903in}}{\pgfqpoint{-3.466929in}{1.183903in}}%
\pgfpathlineto{\pgfqpoint{-3.466929in}{1.183903in}}%
\pgfpathclose%
\pgfusepath{stroke,fill}%
\end{pgfscope}%
\begin{pgfscope}%
\pgfpathrectangle{\pgfqpoint{0.050000in}{0.050000in}}{\pgfqpoint{2.419000in}{2.419000in}}%
\pgfusepath{clip}%
\pgfsetbuttcap%
\pgfsetroundjoin%
\definecolor{currentfill}{rgb}{0.866667,0.800000,0.466667}%
\pgfsetfillcolor{currentfill}%
\pgfsetfillopacity{0.450290}%
\pgfsetlinewidth{1.003750pt}%
\definecolor{currentstroke}{rgb}{0.866667,0.800000,0.466667}%
\pgfsetstrokecolor{currentstroke}%
\pgfsetstrokeopacity{0.450290}%
\pgfsetdash{}{0pt}%
\pgfpathmoveto{\pgfqpoint{1.552535in}{1.183903in}}%
\pgfpathcurveto{\pgfqpoint{1.561743in}{1.183903in}}{\pgfqpoint{1.570576in}{1.187561in}}{\pgfqpoint{1.577087in}{1.194073in}}%
\pgfpathcurveto{\pgfqpoint{1.583598in}{1.200584in}}{\pgfqpoint{1.587257in}{1.209417in}}{\pgfqpoint{1.587257in}{1.218625in}}%
\pgfpathcurveto{\pgfqpoint{1.587257in}{1.227834in}}{\pgfqpoint{1.583598in}{1.236666in}}{\pgfqpoint{1.577087in}{1.243177in}}%
\pgfpathcurveto{\pgfqpoint{1.570576in}{1.249689in}}{\pgfqpoint{1.561743in}{1.253347in}}{\pgfqpoint{1.552535in}{1.253347in}}%
\pgfpathcurveto{\pgfqpoint{1.543326in}{1.253347in}}{\pgfqpoint{1.534494in}{1.249689in}}{\pgfqpoint{1.527982in}{1.243177in}}%
\pgfpathcurveto{\pgfqpoint{1.521471in}{1.236666in}}{\pgfqpoint{1.517813in}{1.227834in}}{\pgfqpoint{1.517813in}{1.218625in}}%
\pgfpathcurveto{\pgfqpoint{1.517813in}{1.209417in}}{\pgfqpoint{1.521471in}{1.200584in}}{\pgfqpoint{1.527982in}{1.194073in}}%
\pgfpathcurveto{\pgfqpoint{1.534494in}{1.187561in}}{\pgfqpoint{1.543326in}{1.183903in}}{\pgfqpoint{1.552535in}{1.183903in}}%
\pgfpathlineto{\pgfqpoint{1.552535in}{1.183903in}}%
\pgfpathclose%
\pgfusepath{stroke,fill}%
\end{pgfscope}%
\begin{pgfscope}%
\pgfpathrectangle{\pgfqpoint{0.050000in}{0.050000in}}{\pgfqpoint{2.419000in}{2.419000in}}%
\pgfusepath{clip}%
\pgfsetbuttcap%
\pgfsetroundjoin%
\definecolor{currentfill}{rgb}{0.866667,0.800000,0.466667}%
\pgfsetfillcolor{currentfill}%
\pgfsetfillopacity{0.450290}%
\pgfsetlinewidth{1.003750pt}%
\definecolor{currentstroke}{rgb}{0.866667,0.800000,0.466667}%
\pgfsetstrokecolor{currentstroke}%
\pgfsetstrokeopacity{0.450290}%
\pgfsetdash{}{0pt}%
\pgfpathmoveto{\pgfqpoint{6.571999in}{1.183903in}}%
\pgfpathcurveto{\pgfqpoint{6.581207in}{1.183903in}}{\pgfqpoint{6.590040in}{1.187561in}}{\pgfqpoint{6.596551in}{1.194073in}}%
\pgfpathcurveto{\pgfqpoint{6.603062in}{1.200584in}}{\pgfqpoint{6.606721in}{1.209417in}}{\pgfqpoint{6.606721in}{1.218625in}}%
\pgfpathcurveto{\pgfqpoint{6.606721in}{1.227834in}}{\pgfqpoint{6.603062in}{1.236666in}}{\pgfqpoint{6.596551in}{1.243177in}}%
\pgfpathcurveto{\pgfqpoint{6.590040in}{1.249689in}}{\pgfqpoint{6.581207in}{1.253347in}}{\pgfqpoint{6.571999in}{1.253347in}}%
\pgfpathcurveto{\pgfqpoint{6.562790in}{1.253347in}}{\pgfqpoint{6.553958in}{1.249689in}}{\pgfqpoint{6.547446in}{1.243177in}}%
\pgfpathcurveto{\pgfqpoint{6.540935in}{1.236666in}}{\pgfqpoint{6.537276in}{1.227834in}}{\pgfqpoint{6.537276in}{1.218625in}}%
\pgfpathcurveto{\pgfqpoint{6.537276in}{1.209417in}}{\pgfqpoint{6.540935in}{1.200584in}}{\pgfqpoint{6.547446in}{1.194073in}}%
\pgfpathcurveto{\pgfqpoint{6.553958in}{1.187561in}}{\pgfqpoint{6.562790in}{1.183903in}}{\pgfqpoint{6.571999in}{1.183903in}}%
\pgfpathlineto{\pgfqpoint{6.571999in}{1.183903in}}%
\pgfpathclose%
\pgfusepath{stroke,fill}%
\end{pgfscope}%
\begin{pgfscope}%
\pgfpathrectangle{\pgfqpoint{0.050000in}{0.050000in}}{\pgfqpoint{2.419000in}{2.419000in}}%
\pgfusepath{clip}%
\pgfsetbuttcap%
\pgfsetroundjoin%
\definecolor{currentfill}{rgb}{0.866667,0.800000,0.466667}%
\pgfsetfillcolor{currentfill}%
\pgfsetfillopacity{0.457180}%
\pgfsetlinewidth{1.003750pt}%
\definecolor{currentstroke}{rgb}{0.866667,0.800000,0.466667}%
\pgfsetstrokecolor{currentstroke}%
\pgfsetstrokeopacity{0.457180}%
\pgfsetdash{}{0pt}%
\pgfpathmoveto{\pgfqpoint{-0.519217in}{1.044223in}}%
\pgfpathcurveto{\pgfqpoint{-0.510009in}{1.044223in}}{\pgfqpoint{-0.501176in}{1.047881in}}{\pgfqpoint{-0.494665in}{1.054392in}}%
\pgfpathcurveto{\pgfqpoint{-0.488154in}{1.060904in}}{\pgfqpoint{-0.484495in}{1.069736in}}{\pgfqpoint{-0.484495in}{1.078945in}}%
\pgfpathcurveto{\pgfqpoint{-0.484495in}{1.088153in}}{\pgfqpoint{-0.488154in}{1.096986in}}{\pgfqpoint{-0.494665in}{1.103497in}}%
\pgfpathcurveto{\pgfqpoint{-0.501176in}{1.110008in}}{\pgfqpoint{-0.510009in}{1.113667in}}{\pgfqpoint{-0.519217in}{1.113667in}}%
\pgfpathcurveto{\pgfqpoint{-0.528426in}{1.113667in}}{\pgfqpoint{-0.537258in}{1.110008in}}{\pgfqpoint{-0.543770in}{1.103497in}}%
\pgfpathcurveto{\pgfqpoint{-0.550281in}{1.096986in}}{\pgfqpoint{-0.553940in}{1.088153in}}{\pgfqpoint{-0.553940in}{1.078945in}}%
\pgfpathcurveto{\pgfqpoint{-0.553940in}{1.069736in}}{\pgfqpoint{-0.550281in}{1.060904in}}{\pgfqpoint{-0.543770in}{1.054392in}}%
\pgfpathcurveto{\pgfqpoint{-0.537258in}{1.047881in}}{\pgfqpoint{-0.528426in}{1.044223in}}{\pgfqpoint{-0.519217in}{1.044223in}}%
\pgfpathlineto{\pgfqpoint{-0.519217in}{1.044223in}}%
\pgfpathclose%
\pgfusepath{stroke,fill}%
\end{pgfscope}%
\begin{pgfscope}%
\pgfpathrectangle{\pgfqpoint{0.050000in}{0.050000in}}{\pgfqpoint{2.419000in}{2.419000in}}%
\pgfusepath{clip}%
\pgfsetbuttcap%
\pgfsetroundjoin%
\definecolor{currentfill}{rgb}{0.866667,0.800000,0.466667}%
\pgfsetfillcolor{currentfill}%
\pgfsetfillopacity{0.457180}%
\pgfsetlinewidth{1.003750pt}%
\definecolor{currentstroke}{rgb}{0.866667,0.800000,0.466667}%
\pgfsetstrokecolor{currentstroke}%
\pgfsetstrokeopacity{0.457180}%
\pgfsetdash{}{0pt}%
\pgfpathmoveto{\pgfqpoint{9.703664in}{1.044223in}}%
\pgfpathcurveto{\pgfqpoint{9.712872in}{1.044223in}}{\pgfqpoint{9.721705in}{1.047881in}}{\pgfqpoint{9.728216in}{1.054392in}}%
\pgfpathcurveto{\pgfqpoint{9.734727in}{1.060904in}}{\pgfqpoint{9.738386in}{1.069736in}}{\pgfqpoint{9.738386in}{1.078945in}}%
\pgfpathcurveto{\pgfqpoint{9.738386in}{1.088153in}}{\pgfqpoint{9.734727in}{1.096986in}}{\pgfqpoint{9.728216in}{1.103497in}}%
\pgfpathcurveto{\pgfqpoint{9.721705in}{1.110008in}}{\pgfqpoint{9.712872in}{1.113667in}}{\pgfqpoint{9.703664in}{1.113667in}}%
\pgfpathcurveto{\pgfqpoint{9.694455in}{1.113667in}}{\pgfqpoint{9.685623in}{1.110008in}}{\pgfqpoint{9.679111in}{1.103497in}}%
\pgfpathcurveto{\pgfqpoint{9.672600in}{1.096986in}}{\pgfqpoint{9.668942in}{1.088153in}}{\pgfqpoint{9.668942in}{1.078945in}}%
\pgfpathcurveto{\pgfqpoint{9.668942in}{1.069736in}}{\pgfqpoint{9.672600in}{1.060904in}}{\pgfqpoint{9.679111in}{1.054392in}}%
\pgfpathcurveto{\pgfqpoint{9.685623in}{1.047881in}}{\pgfqpoint{9.694455in}{1.044223in}}{\pgfqpoint{9.703664in}{1.044223in}}%
\pgfpathlineto{\pgfqpoint{9.703664in}{1.044223in}}%
\pgfpathclose%
\pgfusepath{stroke,fill}%
\end{pgfscope}%
\begin{pgfscope}%
\pgfpathrectangle{\pgfqpoint{0.050000in}{0.050000in}}{\pgfqpoint{2.419000in}{2.419000in}}%
\pgfusepath{clip}%
\pgfsetbuttcap%
\pgfsetroundjoin%
\definecolor{currentfill}{rgb}{0.866667,0.800000,0.466667}%
\pgfsetfillcolor{currentfill}%
\pgfsetfillopacity{0.457180}%
\pgfsetlinewidth{1.003750pt}%
\definecolor{currentstroke}{rgb}{0.866667,0.800000,0.466667}%
\pgfsetstrokecolor{currentstroke}%
\pgfsetstrokeopacity{0.457180}%
\pgfsetdash{}{0pt}%
\pgfpathmoveto{\pgfqpoint{4.592223in}{1.044223in}}%
\pgfpathcurveto{\pgfqpoint{4.601432in}{1.044223in}}{\pgfqpoint{4.610264in}{1.047881in}}{\pgfqpoint{4.616776in}{1.054392in}}%
\pgfpathcurveto{\pgfqpoint{4.623287in}{1.060904in}}{\pgfqpoint{4.626945in}{1.069736in}}{\pgfqpoint{4.626945in}{1.078945in}}%
\pgfpathcurveto{\pgfqpoint{4.626945in}{1.088153in}}{\pgfqpoint{4.623287in}{1.096986in}}{\pgfqpoint{4.616776in}{1.103497in}}%
\pgfpathcurveto{\pgfqpoint{4.610264in}{1.110008in}}{\pgfqpoint{4.601432in}{1.113667in}}{\pgfqpoint{4.592223in}{1.113667in}}%
\pgfpathcurveto{\pgfqpoint{4.583015in}{1.113667in}}{\pgfqpoint{4.574182in}{1.110008in}}{\pgfqpoint{4.567671in}{1.103497in}}%
\pgfpathcurveto{\pgfqpoint{4.561160in}{1.096986in}}{\pgfqpoint{4.557501in}{1.088153in}}{\pgfqpoint{4.557501in}{1.078945in}}%
\pgfpathcurveto{\pgfqpoint{4.557501in}{1.069736in}}{\pgfqpoint{4.561160in}{1.060904in}}{\pgfqpoint{4.567671in}{1.054392in}}%
\pgfpathcurveto{\pgfqpoint{4.574182in}{1.047881in}}{\pgfqpoint{4.583015in}{1.044223in}}{\pgfqpoint{4.592223in}{1.044223in}}%
\pgfpathlineto{\pgfqpoint{4.592223in}{1.044223in}}%
\pgfpathclose%
\pgfusepath{stroke,fill}%
\end{pgfscope}%
\begin{pgfscope}%
\pgfpathrectangle{\pgfqpoint{0.050000in}{0.050000in}}{\pgfqpoint{2.419000in}{2.419000in}}%
\pgfusepath{clip}%
\pgfsetbuttcap%
\pgfsetroundjoin%
\definecolor{currentfill}{rgb}{0.866667,0.800000,0.466667}%
\pgfsetfillcolor{currentfill}%
\pgfsetfillopacity{0.464328}%
\pgfsetlinewidth{1.003750pt}%
\definecolor{currentstroke}{rgb}{0.866667,0.800000,0.466667}%
\pgfsetstrokecolor{currentstroke}%
\pgfsetstrokeopacity{0.464328}%
\pgfsetdash{}{0pt}%
\pgfpathmoveto{\pgfqpoint{-2.668312in}{0.899328in}}%
\pgfpathcurveto{\pgfqpoint{-2.659104in}{0.899328in}}{\pgfqpoint{-2.650271in}{0.902986in}}{\pgfqpoint{-2.643760in}{0.909497in}}%
\pgfpathcurveto{\pgfqpoint{-2.637249in}{0.916009in}}{\pgfqpoint{-2.633590in}{0.924841in}}{\pgfqpoint{-2.633590in}{0.934050in}}%
\pgfpathcurveto{\pgfqpoint{-2.633590in}{0.943258in}}{\pgfqpoint{-2.637249in}{0.952091in}}{\pgfqpoint{-2.643760in}{0.958602in}}%
\pgfpathcurveto{\pgfqpoint{-2.650271in}{0.965113in}}{\pgfqpoint{-2.659104in}{0.968772in}}{\pgfqpoint{-2.668312in}{0.968772in}}%
\pgfpathcurveto{\pgfqpoint{-2.677521in}{0.968772in}}{\pgfqpoint{-2.686353in}{0.965113in}}{\pgfqpoint{-2.692865in}{0.958602in}}%
\pgfpathcurveto{\pgfqpoint{-2.699376in}{0.952091in}}{\pgfqpoint{-2.703035in}{0.943258in}}{\pgfqpoint{-2.703035in}{0.934050in}}%
\pgfpathcurveto{\pgfqpoint{-2.703035in}{0.924841in}}{\pgfqpoint{-2.699376in}{0.916009in}}{\pgfqpoint{-2.692865in}{0.909497in}}%
\pgfpathcurveto{\pgfqpoint{-2.686353in}{0.902986in}}{\pgfqpoint{-2.677521in}{0.899328in}}{\pgfqpoint{-2.668312in}{0.899328in}}%
\pgfpathlineto{\pgfqpoint{-2.668312in}{0.899328in}}%
\pgfpathclose%
\pgfusepath{stroke,fill}%
\end{pgfscope}%
\begin{pgfscope}%
\pgfpathrectangle{\pgfqpoint{0.050000in}{0.050000in}}{\pgfqpoint{2.419000in}{2.419000in}}%
\pgfusepath{clip}%
\pgfsetbuttcap%
\pgfsetroundjoin%
\definecolor{currentfill}{rgb}{0.866667,0.800000,0.466667}%
\pgfsetfillcolor{currentfill}%
\pgfsetfillopacity{0.464328}%
\pgfsetlinewidth{1.003750pt}%
\definecolor{currentstroke}{rgb}{0.866667,0.800000,0.466667}%
\pgfsetstrokecolor{currentstroke}%
\pgfsetstrokeopacity{0.464328}%
\pgfsetdash{}{0pt}%
\pgfpathmoveto{\pgfqpoint{7.745390in}{0.899328in}}%
\pgfpathcurveto{\pgfqpoint{7.754598in}{0.899328in}}{\pgfqpoint{7.763431in}{0.902986in}}{\pgfqpoint{7.769942in}{0.909497in}}%
\pgfpathcurveto{\pgfqpoint{7.776453in}{0.916009in}}{\pgfqpoint{7.780112in}{0.924841in}}{\pgfqpoint{7.780112in}{0.934050in}}%
\pgfpathcurveto{\pgfqpoint{7.780112in}{0.943258in}}{\pgfqpoint{7.776453in}{0.952091in}}{\pgfqpoint{7.769942in}{0.958602in}}%
\pgfpathcurveto{\pgfqpoint{7.763431in}{0.965113in}}{\pgfqpoint{7.754598in}{0.968772in}}{\pgfqpoint{7.745390in}{0.968772in}}%
\pgfpathcurveto{\pgfqpoint{7.736181in}{0.968772in}}{\pgfqpoint{7.727349in}{0.965113in}}{\pgfqpoint{7.720837in}{0.958602in}}%
\pgfpathcurveto{\pgfqpoint{7.714326in}{0.952091in}}{\pgfqpoint{7.710667in}{0.943258in}}{\pgfqpoint{7.710667in}{0.934050in}}%
\pgfpathcurveto{\pgfqpoint{7.710667in}{0.924841in}}{\pgfqpoint{7.714326in}{0.916009in}}{\pgfqpoint{7.720837in}{0.909497in}}%
\pgfpathcurveto{\pgfqpoint{7.727349in}{0.902986in}}{\pgfqpoint{7.736181in}{0.899328in}}{\pgfqpoint{7.745390in}{0.899328in}}%
\pgfpathlineto{\pgfqpoint{7.745390in}{0.899328in}}%
\pgfpathclose%
\pgfusepath{stroke,fill}%
\end{pgfscope}%
\begin{pgfscope}%
\pgfpathrectangle{\pgfqpoint{0.050000in}{0.050000in}}{\pgfqpoint{2.419000in}{2.419000in}}%
\pgfusepath{clip}%
\pgfsetbuttcap%
\pgfsetroundjoin%
\definecolor{currentfill}{rgb}{0.866667,0.800000,0.466667}%
\pgfsetfillcolor{currentfill}%
\pgfsetfillopacity{0.464328}%
\pgfsetlinewidth{1.003750pt}%
\definecolor{currentstroke}{rgb}{0.866667,0.800000,0.466667}%
\pgfsetstrokecolor{currentstroke}%
\pgfsetstrokeopacity{0.464328}%
\pgfsetdash{}{0pt}%
\pgfpathmoveto{\pgfqpoint{2.538539in}{0.899328in}}%
\pgfpathcurveto{\pgfqpoint{2.547747in}{0.899328in}}{\pgfqpoint{2.556580in}{0.902986in}}{\pgfqpoint{2.563091in}{0.909497in}}%
\pgfpathcurveto{\pgfqpoint{2.569602in}{0.916009in}}{\pgfqpoint{2.573261in}{0.924841in}}{\pgfqpoint{2.573261in}{0.934050in}}%
\pgfpathcurveto{\pgfqpoint{2.573261in}{0.943258in}}{\pgfqpoint{2.569602in}{0.952091in}}{\pgfqpoint{2.563091in}{0.958602in}}%
\pgfpathcurveto{\pgfqpoint{2.556580in}{0.965113in}}{\pgfqpoint{2.547747in}{0.968772in}}{\pgfqpoint{2.538539in}{0.968772in}}%
\pgfpathcurveto{\pgfqpoint{2.529330in}{0.968772in}}{\pgfqpoint{2.520498in}{0.965113in}}{\pgfqpoint{2.513986in}{0.958602in}}%
\pgfpathcurveto{\pgfqpoint{2.507475in}{0.952091in}}{\pgfqpoint{2.503816in}{0.943258in}}{\pgfqpoint{2.503816in}{0.934050in}}%
\pgfpathcurveto{\pgfqpoint{2.503816in}{0.924841in}}{\pgfqpoint{2.507475in}{0.916009in}}{\pgfqpoint{2.513986in}{0.909497in}}%
\pgfpathcurveto{\pgfqpoint{2.520498in}{0.902986in}}{\pgfqpoint{2.529330in}{0.899328in}}{\pgfqpoint{2.538539in}{0.899328in}}%
\pgfpathlineto{\pgfqpoint{2.538539in}{0.899328in}}%
\pgfpathclose%
\pgfusepath{stroke,fill}%
\end{pgfscope}%
\begin{pgfscope}%
\pgfpathrectangle{\pgfqpoint{0.050000in}{0.050000in}}{\pgfqpoint{2.419000in}{2.419000in}}%
\pgfusepath{clip}%
\pgfsetbuttcap%
\pgfsetroundjoin%
\definecolor{currentfill}{rgb}{0.866667,0.800000,0.466667}%
\pgfsetfillcolor{currentfill}%
\pgfsetfillopacity{0.471747}%
\pgfsetlinewidth{1.003750pt}%
\definecolor{currentstroke}{rgb}{0.866667,0.800000,0.466667}%
\pgfsetstrokecolor{currentstroke}%
\pgfsetstrokeopacity{0.471747}%
\pgfsetdash{}{0pt}%
\pgfpathmoveto{\pgfqpoint{-4.899164in}{0.748920in}}%
\pgfpathcurveto{\pgfqpoint{-4.889955in}{0.748920in}}{\pgfqpoint{-4.881123in}{0.752579in}}{\pgfqpoint{-4.874611in}{0.759090in}}%
\pgfpathcurveto{\pgfqpoint{-4.868100in}{0.765602in}}{\pgfqpoint{-4.864441in}{0.774434in}}{\pgfqpoint{-4.864441in}{0.783643in}}%
\pgfpathcurveto{\pgfqpoint{-4.864441in}{0.792851in}}{\pgfqpoint{-4.868100in}{0.801684in}}{\pgfqpoint{-4.874611in}{0.808195in}}%
\pgfpathcurveto{\pgfqpoint{-4.881123in}{0.814706in}}{\pgfqpoint{-4.889955in}{0.818365in}}{\pgfqpoint{-4.899164in}{0.818365in}}%
\pgfpathcurveto{\pgfqpoint{-4.908372in}{0.818365in}}{\pgfqpoint{-4.917205in}{0.814706in}}{\pgfqpoint{-4.923716in}{0.808195in}}%
\pgfpathcurveto{\pgfqpoint{-4.930227in}{0.801684in}}{\pgfqpoint{-4.933886in}{0.792851in}}{\pgfqpoint{-4.933886in}{0.783643in}}%
\pgfpathcurveto{\pgfqpoint{-4.933886in}{0.774434in}}{\pgfqpoint{-4.930227in}{0.765602in}}{\pgfqpoint{-4.923716in}{0.759090in}}%
\pgfpathcurveto{\pgfqpoint{-4.917205in}{0.752579in}}{\pgfqpoint{-4.908372in}{0.748920in}}{\pgfqpoint{-4.899164in}{0.748920in}}%
\pgfpathlineto{\pgfqpoint{-4.899164in}{0.748920in}}%
\pgfpathclose%
\pgfusepath{stroke,fill}%
\end{pgfscope}%
\begin{pgfscope}%
\pgfpathrectangle{\pgfqpoint{0.050000in}{0.050000in}}{\pgfqpoint{2.419000in}{2.419000in}}%
\pgfusepath{clip}%
\pgfsetbuttcap%
\pgfsetroundjoin%
\definecolor{currentfill}{rgb}{0.866667,0.800000,0.466667}%
\pgfsetfillcolor{currentfill}%
\pgfsetfillopacity{0.471747}%
\pgfsetlinewidth{1.003750pt}%
\definecolor{currentstroke}{rgb}{0.866667,0.800000,0.466667}%
\pgfsetstrokecolor{currentstroke}%
\pgfsetstrokeopacity{0.471747}%
\pgfsetdash{}{0pt}%
\pgfpathmoveto{\pgfqpoint{0.406727in}{0.748920in}}%
\pgfpathcurveto{\pgfqpoint{0.415936in}{0.748920in}}{\pgfqpoint{0.424768in}{0.752579in}}{\pgfqpoint{0.431280in}{0.759090in}}%
\pgfpathcurveto{\pgfqpoint{0.437791in}{0.765602in}}{\pgfqpoint{0.441450in}{0.774434in}}{\pgfqpoint{0.441450in}{0.783643in}}%
\pgfpathcurveto{\pgfqpoint{0.441450in}{0.792851in}}{\pgfqpoint{0.437791in}{0.801684in}}{\pgfqpoint{0.431280in}{0.808195in}}%
\pgfpathcurveto{\pgfqpoint{0.424768in}{0.814706in}}{\pgfqpoint{0.415936in}{0.818365in}}{\pgfqpoint{0.406727in}{0.818365in}}%
\pgfpathcurveto{\pgfqpoint{0.397519in}{0.818365in}}{\pgfqpoint{0.388686in}{0.814706in}}{\pgfqpoint{0.382175in}{0.808195in}}%
\pgfpathcurveto{\pgfqpoint{0.375664in}{0.801684in}}{\pgfqpoint{0.372005in}{0.792851in}}{\pgfqpoint{0.372005in}{0.783643in}}%
\pgfpathcurveto{\pgfqpoint{0.372005in}{0.774434in}}{\pgfqpoint{0.375664in}{0.765602in}}{\pgfqpoint{0.382175in}{0.759090in}}%
\pgfpathcurveto{\pgfqpoint{0.388686in}{0.752579in}}{\pgfqpoint{0.397519in}{0.748920in}}{\pgfqpoint{0.406727in}{0.748920in}}%
\pgfpathlineto{\pgfqpoint{0.406727in}{0.748920in}}%
\pgfpathclose%
\pgfusepath{stroke,fill}%
\end{pgfscope}%
\begin{pgfscope}%
\pgfpathrectangle{\pgfqpoint{0.050000in}{0.050000in}}{\pgfqpoint{2.419000in}{2.419000in}}%
\pgfusepath{clip}%
\pgfsetbuttcap%
\pgfsetroundjoin%
\definecolor{currentfill}{rgb}{0.866667,0.800000,0.466667}%
\pgfsetfillcolor{currentfill}%
\pgfsetfillopacity{0.471747}%
\pgfsetlinewidth{1.003750pt}%
\definecolor{currentstroke}{rgb}{0.866667,0.800000,0.466667}%
\pgfsetstrokecolor{currentstroke}%
\pgfsetstrokeopacity{0.471747}%
\pgfsetdash{}{0pt}%
\pgfpathmoveto{\pgfqpoint{5.712618in}{0.748920in}}%
\pgfpathcurveto{\pgfqpoint{5.721827in}{0.748920in}}{\pgfqpoint{5.730659in}{0.752579in}}{\pgfqpoint{5.737171in}{0.759090in}}%
\pgfpathcurveto{\pgfqpoint{5.743682in}{0.765602in}}{\pgfqpoint{5.747341in}{0.774434in}}{\pgfqpoint{5.747341in}{0.783643in}}%
\pgfpathcurveto{\pgfqpoint{5.747341in}{0.792851in}}{\pgfqpoint{5.743682in}{0.801684in}}{\pgfqpoint{5.737171in}{0.808195in}}%
\pgfpathcurveto{\pgfqpoint{5.730659in}{0.814706in}}{\pgfqpoint{5.721827in}{0.818365in}}{\pgfqpoint{5.712618in}{0.818365in}}%
\pgfpathcurveto{\pgfqpoint{5.703410in}{0.818365in}}{\pgfqpoint{5.694577in}{0.814706in}}{\pgfqpoint{5.688066in}{0.808195in}}%
\pgfpathcurveto{\pgfqpoint{5.681555in}{0.801684in}}{\pgfqpoint{5.677896in}{0.792851in}}{\pgfqpoint{5.677896in}{0.783643in}}%
\pgfpathcurveto{\pgfqpoint{5.677896in}{0.774434in}}{\pgfqpoint{5.681555in}{0.765602in}}{\pgfqpoint{5.688066in}{0.759090in}}%
\pgfpathcurveto{\pgfqpoint{5.694577in}{0.752579in}}{\pgfqpoint{5.703410in}{0.748920in}}{\pgfqpoint{5.712618in}{0.748920in}}%
\pgfpathlineto{\pgfqpoint{5.712618in}{0.748920in}}%
\pgfpathclose%
\pgfusepath{stroke,fill}%
\end{pgfscope}%
\begin{pgfscope}%
\pgfpathrectangle{\pgfqpoint{0.050000in}{0.050000in}}{\pgfqpoint{2.419000in}{2.419000in}}%
\pgfusepath{clip}%
\pgfsetbuttcap%
\pgfsetroundjoin%
\definecolor{currentfill}{rgb}{0.866667,0.800000,0.466667}%
\pgfsetfillcolor{currentfill}%
\pgfsetfillopacity{0.479454}%
\pgfsetlinewidth{1.003750pt}%
\definecolor{currentstroke}{rgb}{0.866667,0.800000,0.466667}%
\pgfsetstrokecolor{currentstroke}%
\pgfsetstrokeopacity{0.479454}%
\pgfsetdash{}{0pt}%
\pgfpathmoveto{\pgfqpoint{-1.807755in}{0.592681in}}%
\pgfpathcurveto{\pgfqpoint{-1.798547in}{0.592681in}}{\pgfqpoint{-1.789714in}{0.596339in}}{\pgfqpoint{-1.783203in}{0.602851in}}%
\pgfpathcurveto{\pgfqpoint{-1.776692in}{0.609362in}}{\pgfqpoint{-1.773033in}{0.618194in}}{\pgfqpoint{-1.773033in}{0.627403in}}%
\pgfpathcurveto{\pgfqpoint{-1.773033in}{0.636611in}}{\pgfqpoint{-1.776692in}{0.645444in}}{\pgfqpoint{-1.783203in}{0.651955in}}%
\pgfpathcurveto{\pgfqpoint{-1.789714in}{0.658466in}}{\pgfqpoint{-1.798547in}{0.662125in}}{\pgfqpoint{-1.807755in}{0.662125in}}%
\pgfpathcurveto{\pgfqpoint{-1.816964in}{0.662125in}}{\pgfqpoint{-1.825796in}{0.658466in}}{\pgfqpoint{-1.832308in}{0.651955in}}%
\pgfpathcurveto{\pgfqpoint{-1.838819in}{0.645444in}}{\pgfqpoint{-1.842477in}{0.636611in}}{\pgfqpoint{-1.842477in}{0.627403in}}%
\pgfpathcurveto{\pgfqpoint{-1.842477in}{0.618194in}}{\pgfqpoint{-1.838819in}{0.609362in}}{\pgfqpoint{-1.832308in}{0.602851in}}%
\pgfpathcurveto{\pgfqpoint{-1.825796in}{0.596339in}}{\pgfqpoint{-1.816964in}{0.592681in}}{\pgfqpoint{-1.807755in}{0.592681in}}%
\pgfpathlineto{\pgfqpoint{-1.807755in}{0.592681in}}%
\pgfpathclose%
\pgfusepath{stroke,fill}%
\end{pgfscope}%
\begin{pgfscope}%
\pgfpathrectangle{\pgfqpoint{0.050000in}{0.050000in}}{\pgfqpoint{2.419000in}{2.419000in}}%
\pgfusepath{clip}%
\pgfsetbuttcap%
\pgfsetroundjoin%
\definecolor{currentfill}{rgb}{0.866667,0.800000,0.466667}%
\pgfsetfillcolor{currentfill}%
\pgfsetfillopacity{0.479454}%
\pgfsetlinewidth{1.003750pt}%
\definecolor{currentstroke}{rgb}{0.866667,0.800000,0.466667}%
\pgfsetstrokecolor{currentstroke}%
\pgfsetstrokeopacity{0.479454}%
\pgfsetdash{}{0pt}%
\pgfpathmoveto{\pgfqpoint{9.009788in}{0.592681in}}%
\pgfpathcurveto{\pgfqpoint{9.018997in}{0.592681in}}{\pgfqpoint{9.027829in}{0.596339in}}{\pgfqpoint{9.034341in}{0.602851in}}%
\pgfpathcurveto{\pgfqpoint{9.040852in}{0.609362in}}{\pgfqpoint{9.044510in}{0.618194in}}{\pgfqpoint{9.044510in}{0.627403in}}%
\pgfpathcurveto{\pgfqpoint{9.044510in}{0.636611in}}{\pgfqpoint{9.040852in}{0.645444in}}{\pgfqpoint{9.034341in}{0.651955in}}%
\pgfpathcurveto{\pgfqpoint{9.027829in}{0.658466in}}{\pgfqpoint{9.018997in}{0.662125in}}{\pgfqpoint{9.009788in}{0.662125in}}%
\pgfpathcurveto{\pgfqpoint{9.000580in}{0.662125in}}{\pgfqpoint{8.991747in}{0.658466in}}{\pgfqpoint{8.985236in}{0.651955in}}%
\pgfpathcurveto{\pgfqpoint{8.978725in}{0.645444in}}{\pgfqpoint{8.975066in}{0.636611in}}{\pgfqpoint{8.975066in}{0.627403in}}%
\pgfpathcurveto{\pgfqpoint{8.975066in}{0.618194in}}{\pgfqpoint{8.978725in}{0.609362in}}{\pgfqpoint{8.985236in}{0.602851in}}%
\pgfpathcurveto{\pgfqpoint{8.991747in}{0.596339in}}{\pgfqpoint{9.000580in}{0.592681in}}{\pgfqpoint{9.009788in}{0.592681in}}%
\pgfpathlineto{\pgfqpoint{9.009788in}{0.592681in}}%
\pgfpathclose%
\pgfusepath{stroke,fill}%
\end{pgfscope}%
\begin{pgfscope}%
\pgfpathrectangle{\pgfqpoint{0.050000in}{0.050000in}}{\pgfqpoint{2.419000in}{2.419000in}}%
\pgfusepath{clip}%
\pgfsetbuttcap%
\pgfsetroundjoin%
\definecolor{currentfill}{rgb}{0.866667,0.800000,0.466667}%
\pgfsetfillcolor{currentfill}%
\pgfsetfillopacity{0.479454}%
\pgfsetlinewidth{1.003750pt}%
\definecolor{currentstroke}{rgb}{0.866667,0.800000,0.466667}%
\pgfsetstrokecolor{currentstroke}%
\pgfsetstrokeopacity{0.479454}%
\pgfsetdash{}{0pt}%
\pgfpathmoveto{\pgfqpoint{3.601017in}{0.592681in}}%
\pgfpathcurveto{\pgfqpoint{3.610225in}{0.592681in}}{\pgfqpoint{3.619057in}{0.596339in}}{\pgfqpoint{3.625569in}{0.602851in}}%
\pgfpathcurveto{\pgfqpoint{3.632080in}{0.609362in}}{\pgfqpoint{3.635739in}{0.618194in}}{\pgfqpoint{3.635739in}{0.627403in}}%
\pgfpathcurveto{\pgfqpoint{3.635739in}{0.636611in}}{\pgfqpoint{3.632080in}{0.645444in}}{\pgfqpoint{3.625569in}{0.651955in}}%
\pgfpathcurveto{\pgfqpoint{3.619057in}{0.658466in}}{\pgfqpoint{3.610225in}{0.662125in}}{\pgfqpoint{3.601017in}{0.662125in}}%
\pgfpathcurveto{\pgfqpoint{3.591808in}{0.662125in}}{\pgfqpoint{3.582976in}{0.658466in}}{\pgfqpoint{3.576464in}{0.651955in}}%
\pgfpathcurveto{\pgfqpoint{3.569953in}{0.645444in}}{\pgfqpoint{3.566294in}{0.636611in}}{\pgfqpoint{3.566294in}{0.627403in}}%
\pgfpathcurveto{\pgfqpoint{3.566294in}{0.618194in}}{\pgfqpoint{3.569953in}{0.609362in}}{\pgfqpoint{3.576464in}{0.602851in}}%
\pgfpathcurveto{\pgfqpoint{3.582976in}{0.596339in}}{\pgfqpoint{3.591808in}{0.592681in}}{\pgfqpoint{3.601017in}{0.592681in}}%
\pgfpathlineto{\pgfqpoint{3.601017in}{0.592681in}}%
\pgfpathclose%
\pgfusepath{stroke,fill}%
\end{pgfscope}%
\begin{pgfscope}%
\pgfpathrectangle{\pgfqpoint{0.050000in}{0.050000in}}{\pgfqpoint{2.419000in}{2.419000in}}%
\pgfusepath{clip}%
\pgfsetbuttcap%
\pgfsetroundjoin%
\definecolor{currentfill}{rgb}{0.866667,0.800000,0.466667}%
\pgfsetfillcolor{currentfill}%
\pgfsetfillopacity{0.487466}%
\pgfsetlinewidth{1.003750pt}%
\definecolor{currentstroke}{rgb}{0.866667,0.800000,0.466667}%
\pgfsetstrokecolor{currentstroke}%
\pgfsetstrokeopacity{0.487466}%
\pgfsetdash{}{0pt}%
\pgfpathmoveto{\pgfqpoint{-4.109813in}{0.430262in}}%
\pgfpathcurveto{\pgfqpoint{-4.100605in}{0.430262in}}{\pgfqpoint{-4.091772in}{0.433921in}}{\pgfqpoint{-4.085261in}{0.440432in}}%
\pgfpathcurveto{\pgfqpoint{-4.078749in}{0.446943in}}{\pgfqpoint{-4.075091in}{0.455776in}}{\pgfqpoint{-4.075091in}{0.464984in}}%
\pgfpathcurveto{\pgfqpoint{-4.075091in}{0.474193in}}{\pgfqpoint{-4.078749in}{0.483025in}}{\pgfqpoint{-4.085261in}{0.489537in}}%
\pgfpathcurveto{\pgfqpoint{-4.091772in}{0.496048in}}{\pgfqpoint{-4.100605in}{0.499706in}}{\pgfqpoint{-4.109813in}{0.499706in}}%
\pgfpathcurveto{\pgfqpoint{-4.119021in}{0.499706in}}{\pgfqpoint{-4.127854in}{0.496048in}}{\pgfqpoint{-4.134365in}{0.489537in}}%
\pgfpathcurveto{\pgfqpoint{-4.140877in}{0.483025in}}{\pgfqpoint{-4.144535in}{0.474193in}}{\pgfqpoint{-4.144535in}{0.464984in}}%
\pgfpathcurveto{\pgfqpoint{-4.144535in}{0.455776in}}{\pgfqpoint{-4.140877in}{0.446943in}}{\pgfqpoint{-4.134365in}{0.440432in}}%
\pgfpathcurveto{\pgfqpoint{-4.127854in}{0.433921in}}{\pgfqpoint{-4.119021in}{0.430262in}}{\pgfqpoint{-4.109813in}{0.430262in}}%
\pgfpathlineto{\pgfqpoint{-4.109813in}{0.430262in}}%
\pgfpathclose%
\pgfusepath{stroke,fill}%
\end{pgfscope}%
\begin{pgfscope}%
\pgfpathrectangle{\pgfqpoint{0.050000in}{0.050000in}}{\pgfqpoint{2.419000in}{2.419000in}}%
\pgfusepath{clip}%
\pgfsetbuttcap%
\pgfsetroundjoin%
\definecolor{currentfill}{rgb}{0.866667,0.800000,0.466667}%
\pgfsetfillcolor{currentfill}%
\pgfsetfillopacity{0.487466}%
\pgfsetlinewidth{1.003750pt}%
\definecolor{currentstroke}{rgb}{0.866667,0.800000,0.466667}%
\pgfsetstrokecolor{currentstroke}%
\pgfsetstrokeopacity{0.487466}%
\pgfsetdash{}{0pt}%
\pgfpathmoveto{\pgfqpoint{1.405908in}{0.430262in}}%
\pgfpathcurveto{\pgfqpoint{1.415116in}{0.430262in}}{\pgfqpoint{1.423949in}{0.433921in}}{\pgfqpoint{1.430460in}{0.440432in}}%
\pgfpathcurveto{\pgfqpoint{1.436972in}{0.446943in}}{\pgfqpoint{1.440630in}{0.455776in}}{\pgfqpoint{1.440630in}{0.464984in}}%
\pgfpathcurveto{\pgfqpoint{1.440630in}{0.474193in}}{\pgfqpoint{1.436972in}{0.483025in}}{\pgfqpoint{1.430460in}{0.489537in}}%
\pgfpathcurveto{\pgfqpoint{1.423949in}{0.496048in}}{\pgfqpoint{1.415116in}{0.499706in}}{\pgfqpoint{1.405908in}{0.499706in}}%
\pgfpathcurveto{\pgfqpoint{1.396700in}{0.499706in}}{\pgfqpoint{1.387867in}{0.496048in}}{\pgfqpoint{1.381356in}{0.489537in}}%
\pgfpathcurveto{\pgfqpoint{1.374844in}{0.483025in}}{\pgfqpoint{1.371186in}{0.474193in}}{\pgfqpoint{1.371186in}{0.464984in}}%
\pgfpathcurveto{\pgfqpoint{1.371186in}{0.455776in}}{\pgfqpoint{1.374844in}{0.446943in}}{\pgfqpoint{1.381356in}{0.440432in}}%
\pgfpathcurveto{\pgfqpoint{1.387867in}{0.433921in}}{\pgfqpoint{1.396700in}{0.430262in}}{\pgfqpoint{1.405908in}{0.430262in}}%
\pgfpathlineto{\pgfqpoint{1.405908in}{0.430262in}}%
\pgfpathclose%
\pgfusepath{stroke,fill}%
\end{pgfscope}%
\begin{pgfscope}%
\pgfpathrectangle{\pgfqpoint{0.050000in}{0.050000in}}{\pgfqpoint{2.419000in}{2.419000in}}%
\pgfusepath{clip}%
\pgfsetbuttcap%
\pgfsetroundjoin%
\definecolor{currentfill}{rgb}{0.866667,0.800000,0.466667}%
\pgfsetfillcolor{currentfill}%
\pgfsetfillopacity{0.487466}%
\pgfsetlinewidth{1.003750pt}%
\definecolor{currentstroke}{rgb}{0.866667,0.800000,0.466667}%
\pgfsetstrokecolor{currentstroke}%
\pgfsetstrokeopacity{0.487466}%
\pgfsetdash{}{0pt}%
\pgfpathmoveto{\pgfqpoint{6.921629in}{0.430262in}}%
\pgfpathcurveto{\pgfqpoint{6.930837in}{0.430262in}}{\pgfqpoint{6.939670in}{0.433921in}}{\pgfqpoint{6.946181in}{0.440432in}}%
\pgfpathcurveto{\pgfqpoint{6.952693in}{0.446943in}}{\pgfqpoint{6.956351in}{0.455776in}}{\pgfqpoint{6.956351in}{0.464984in}}%
\pgfpathcurveto{\pgfqpoint{6.956351in}{0.474193in}}{\pgfqpoint{6.952693in}{0.483025in}}{\pgfqpoint{6.946181in}{0.489537in}}%
\pgfpathcurveto{\pgfqpoint{6.939670in}{0.496048in}}{\pgfqpoint{6.930837in}{0.499706in}}{\pgfqpoint{6.921629in}{0.499706in}}%
\pgfpathcurveto{\pgfqpoint{6.912421in}{0.499706in}}{\pgfqpoint{6.903588in}{0.496048in}}{\pgfqpoint{6.897077in}{0.489537in}}%
\pgfpathcurveto{\pgfqpoint{6.890565in}{0.483025in}}{\pgfqpoint{6.886907in}{0.474193in}}{\pgfqpoint{6.886907in}{0.464984in}}%
\pgfpathcurveto{\pgfqpoint{6.886907in}{0.455776in}}{\pgfqpoint{6.890565in}{0.446943in}}{\pgfqpoint{6.897077in}{0.440432in}}%
\pgfpathcurveto{\pgfqpoint{6.903588in}{0.433921in}}{\pgfqpoint{6.912421in}{0.430262in}}{\pgfqpoint{6.921629in}{0.430262in}}%
\pgfpathlineto{\pgfqpoint{6.921629in}{0.430262in}}%
\pgfpathclose%
\pgfusepath{stroke,fill}%
\end{pgfscope}%
\begin{pgfscope}%
\pgfpathrectangle{\pgfqpoint{0.050000in}{0.050000in}}{\pgfqpoint{2.419000in}{2.419000in}}%
\pgfusepath{clip}%
\pgfsetbuttcap%
\pgfsetroundjoin%
\definecolor{currentfill}{rgb}{0.866667,0.800000,0.466667}%
\pgfsetfillcolor{currentfill}%
\pgfsetfillopacity{0.495801}%
\pgfsetlinewidth{1.003750pt}%
\definecolor{currentstroke}{rgb}{0.866667,0.800000,0.466667}%
\pgfsetstrokecolor{currentstroke}%
\pgfsetstrokeopacity{0.495801}%
\pgfsetdash{}{0pt}%
\pgfpathmoveto{\pgfqpoint{-0.877761in}{0.261291in}}%
\pgfpathcurveto{\pgfqpoint{-0.868552in}{0.261291in}}{\pgfqpoint{-0.859720in}{0.264949in}}{\pgfqpoint{-0.853208in}{0.271461in}}%
\pgfpathcurveto{\pgfqpoint{-0.846697in}{0.277972in}}{\pgfqpoint{-0.843039in}{0.286804in}}{\pgfqpoint{-0.843039in}{0.296013in}}%
\pgfpathcurveto{\pgfqpoint{-0.843039in}{0.305221in}}{\pgfqpoint{-0.846697in}{0.314054in}}{\pgfqpoint{-0.853208in}{0.320565in}}%
\pgfpathcurveto{\pgfqpoint{-0.859720in}{0.327077in}}{\pgfqpoint{-0.868552in}{0.330735in}}{\pgfqpoint{-0.877761in}{0.330735in}}%
\pgfpathcurveto{\pgfqpoint{-0.886969in}{0.330735in}}{\pgfqpoint{-0.895802in}{0.327077in}}{\pgfqpoint{-0.902313in}{0.320565in}}%
\pgfpathcurveto{\pgfqpoint{-0.908824in}{0.314054in}}{\pgfqpoint{-0.912483in}{0.305221in}}{\pgfqpoint{-0.912483in}{0.296013in}}%
\pgfpathcurveto{\pgfqpoint{-0.912483in}{0.286804in}}{\pgfqpoint{-0.908824in}{0.277972in}}{\pgfqpoint{-0.902313in}{0.271461in}}%
\pgfpathcurveto{\pgfqpoint{-0.895802in}{0.264949in}}{\pgfqpoint{-0.886969in}{0.261291in}}{\pgfqpoint{-0.877761in}{0.261291in}}%
\pgfpathlineto{\pgfqpoint{-0.877761in}{0.261291in}}%
\pgfpathclose%
\pgfusepath{stroke,fill}%
\end{pgfscope}%
\begin{pgfscope}%
\pgfpathrectangle{\pgfqpoint{0.050000in}{0.050000in}}{\pgfqpoint{2.419000in}{2.419000in}}%
\pgfusepath{clip}%
\pgfsetbuttcap%
\pgfsetroundjoin%
\definecolor{currentfill}{rgb}{0.866667,0.800000,0.466667}%
\pgfsetfillcolor{currentfill}%
\pgfsetfillopacity{0.495801}%
\pgfsetlinewidth{1.003750pt}%
\definecolor{currentstroke}{rgb}{0.866667,0.800000,0.466667}%
\pgfsetstrokecolor{currentstroke}%
\pgfsetstrokeopacity{0.495801}%
\pgfsetdash{}{0pt}%
\pgfpathmoveto{\pgfqpoint{4.749224in}{0.261291in}}%
\pgfpathcurveto{\pgfqpoint{4.758433in}{0.261291in}}{\pgfqpoint{4.767265in}{0.264949in}}{\pgfqpoint{4.773777in}{0.271461in}}%
\pgfpathcurveto{\pgfqpoint{4.780288in}{0.277972in}}{\pgfqpoint{4.783947in}{0.286804in}}{\pgfqpoint{4.783947in}{0.296013in}}%
\pgfpathcurveto{\pgfqpoint{4.783947in}{0.305221in}}{\pgfqpoint{4.780288in}{0.314054in}}{\pgfqpoint{4.773777in}{0.320565in}}%
\pgfpathcurveto{\pgfqpoint{4.767265in}{0.327077in}}{\pgfqpoint{4.758433in}{0.330735in}}{\pgfqpoint{4.749224in}{0.330735in}}%
\pgfpathcurveto{\pgfqpoint{4.740016in}{0.330735in}}{\pgfqpoint{4.731183in}{0.327077in}}{\pgfqpoint{4.724672in}{0.320565in}}%
\pgfpathcurveto{\pgfqpoint{4.718161in}{0.314054in}}{\pgfqpoint{4.714502in}{0.305221in}}{\pgfqpoint{4.714502in}{0.296013in}}%
\pgfpathcurveto{\pgfqpoint{4.714502in}{0.286804in}}{\pgfqpoint{4.718161in}{0.277972in}}{\pgfqpoint{4.724672in}{0.271461in}}%
\pgfpathcurveto{\pgfqpoint{4.731183in}{0.264949in}}{\pgfqpoint{4.740016in}{0.261291in}}{\pgfqpoint{4.749224in}{0.261291in}}%
\pgfpathlineto{\pgfqpoint{4.749224in}{0.261291in}}%
\pgfpathclose%
\pgfusepath{stroke,fill}%
\end{pgfscope}%
\begin{pgfscope}%
\pgfpathrectangle{\pgfqpoint{0.050000in}{0.050000in}}{\pgfqpoint{2.419000in}{2.419000in}}%
\pgfusepath{clip}%
\pgfsetbuttcap%
\pgfsetroundjoin%
\definecolor{currentfill}{rgb}{0.866667,0.800000,0.466667}%
\pgfsetfillcolor{currentfill}%
\pgfsetfillopacity{0.495801}%
\pgfsetlinewidth{1.003750pt}%
\definecolor{currentstroke}{rgb}{0.866667,0.800000,0.466667}%
\pgfsetstrokecolor{currentstroke}%
\pgfsetstrokeopacity{0.495801}%
\pgfsetdash{}{0pt}%
\pgfpathmoveto{\pgfqpoint{10.376210in}{0.261291in}}%
\pgfpathcurveto{\pgfqpoint{10.385418in}{0.261291in}}{\pgfqpoint{10.394251in}{0.264949in}}{\pgfqpoint{10.400762in}{0.271461in}}%
\pgfpathcurveto{\pgfqpoint{10.407273in}{0.277972in}}{\pgfqpoint{10.410932in}{0.286804in}}{\pgfqpoint{10.410932in}{0.296013in}}%
\pgfpathcurveto{\pgfqpoint{10.410932in}{0.305221in}}{\pgfqpoint{10.407273in}{0.314054in}}{\pgfqpoint{10.400762in}{0.320565in}}%
\pgfpathcurveto{\pgfqpoint{10.394251in}{0.327077in}}{\pgfqpoint{10.385418in}{0.330735in}}{\pgfqpoint{10.376210in}{0.330735in}}%
\pgfpathcurveto{\pgfqpoint{10.367001in}{0.330735in}}{\pgfqpoint{10.358169in}{0.327077in}}{\pgfqpoint{10.351657in}{0.320565in}}%
\pgfpathcurveto{\pgfqpoint{10.345146in}{0.314054in}}{\pgfqpoint{10.341487in}{0.305221in}}{\pgfqpoint{10.341487in}{0.296013in}}%
\pgfpathcurveto{\pgfqpoint{10.341487in}{0.286804in}}{\pgfqpoint{10.345146in}{0.277972in}}{\pgfqpoint{10.351657in}{0.271461in}}%
\pgfpathcurveto{\pgfqpoint{10.358169in}{0.264949in}}{\pgfqpoint{10.367001in}{0.261291in}}{\pgfqpoint{10.376210in}{0.261291in}}%
\pgfpathlineto{\pgfqpoint{10.376210in}{0.261291in}}%
\pgfpathclose%
\pgfusepath{stroke,fill}%
\end{pgfscope}%
\begin{pgfscope}%
\pgfpathrectangle{\pgfqpoint{0.050000in}{0.050000in}}{\pgfqpoint{2.419000in}{2.419000in}}%
\pgfusepath{clip}%
\pgfsetbuttcap%
\pgfsetroundjoin%
\definecolor{currentfill}{rgb}{0.866667,0.800000,0.466667}%
\pgfsetfillcolor{currentfill}%
\pgfsetfillopacity{0.504479}%
\pgfsetlinewidth{1.003750pt}%
\definecolor{currentstroke}{rgb}{0.866667,0.800000,0.466667}%
\pgfsetstrokecolor{currentstroke}%
\pgfsetstrokeopacity{0.504479}%
\pgfsetdash{}{0pt}%
\pgfpathmoveto{\pgfqpoint{-3.255460in}{0.085362in}}%
\pgfpathcurveto{\pgfqpoint{-3.246251in}{0.085362in}}{\pgfqpoint{-3.237419in}{0.089021in}}{\pgfqpoint{-3.230907in}{0.095532in}}%
\pgfpathcurveto{\pgfqpoint{-3.224396in}{0.102043in}}{\pgfqpoint{-3.220737in}{0.110876in}}{\pgfqpoint{-3.220737in}{0.120084in}}%
\pgfpathcurveto{\pgfqpoint{-3.220737in}{0.129293in}}{\pgfqpoint{-3.224396in}{0.138125in}}{\pgfqpoint{-3.230907in}{0.144637in}}%
\pgfpathcurveto{\pgfqpoint{-3.237419in}{0.151148in}}{\pgfqpoint{-3.246251in}{0.154806in}}{\pgfqpoint{-3.255460in}{0.154806in}}%
\pgfpathcurveto{\pgfqpoint{-3.264668in}{0.154806in}}{\pgfqpoint{-3.273500in}{0.151148in}}{\pgfqpoint{-3.280012in}{0.144637in}}%
\pgfpathcurveto{\pgfqpoint{-3.286523in}{0.138125in}}{\pgfqpoint{-3.290182in}{0.129293in}}{\pgfqpoint{-3.290182in}{0.120084in}}%
\pgfpathcurveto{\pgfqpoint{-3.290182in}{0.110876in}}{\pgfqpoint{-3.286523in}{0.102043in}}{\pgfqpoint{-3.280012in}{0.095532in}}%
\pgfpathcurveto{\pgfqpoint{-3.273500in}{0.089021in}}{\pgfqpoint{-3.264668in}{0.085362in}}{\pgfqpoint{-3.255460in}{0.085362in}}%
\pgfpathlineto{\pgfqpoint{-3.255460in}{0.085362in}}%
\pgfpathclose%
\pgfusepath{stroke,fill}%
\end{pgfscope}%
\begin{pgfscope}%
\pgfpathrectangle{\pgfqpoint{0.050000in}{0.050000in}}{\pgfqpoint{2.419000in}{2.419000in}}%
\pgfusepath{clip}%
\pgfsetbuttcap%
\pgfsetroundjoin%
\definecolor{currentfill}{rgb}{0.866667,0.800000,0.466667}%
\pgfsetfillcolor{currentfill}%
\pgfsetfillopacity{0.504479}%
\pgfsetlinewidth{1.003750pt}%
\definecolor{currentstroke}{rgb}{0.866667,0.800000,0.466667}%
\pgfsetstrokecolor{currentstroke}%
\pgfsetstrokeopacity{0.504479}%
\pgfsetdash{}{0pt}%
\pgfpathmoveto{\pgfqpoint{2.487371in}{0.085362in}}%
\pgfpathcurveto{\pgfqpoint{2.496580in}{0.085362in}}{\pgfqpoint{2.505412in}{0.089021in}}{\pgfqpoint{2.511923in}{0.095532in}}%
\pgfpathcurveto{\pgfqpoint{2.518435in}{0.102043in}}{\pgfqpoint{2.522093in}{0.110876in}}{\pgfqpoint{2.522093in}{0.120084in}}%
\pgfpathcurveto{\pgfqpoint{2.522093in}{0.129293in}}{\pgfqpoint{2.518435in}{0.138125in}}{\pgfqpoint{2.511923in}{0.144637in}}%
\pgfpathcurveto{\pgfqpoint{2.505412in}{0.151148in}}{\pgfqpoint{2.496580in}{0.154806in}}{\pgfqpoint{2.487371in}{0.154806in}}%
\pgfpathcurveto{\pgfqpoint{2.478163in}{0.154806in}}{\pgfqpoint{2.469330in}{0.151148in}}{\pgfqpoint{2.462819in}{0.144637in}}%
\pgfpathcurveto{\pgfqpoint{2.456307in}{0.138125in}}{\pgfqpoint{2.452649in}{0.129293in}}{\pgfqpoint{2.452649in}{0.120084in}}%
\pgfpathcurveto{\pgfqpoint{2.452649in}{0.110876in}}{\pgfqpoint{2.456307in}{0.102043in}}{\pgfqpoint{2.462819in}{0.095532in}}%
\pgfpathcurveto{\pgfqpoint{2.469330in}{0.089021in}}{\pgfqpoint{2.478163in}{0.085362in}}{\pgfqpoint{2.487371in}{0.085362in}}%
\pgfpathlineto{\pgfqpoint{2.487371in}{0.085362in}}%
\pgfpathclose%
\pgfusepath{stroke,fill}%
\end{pgfscope}%
\begin{pgfscope}%
\pgfpathrectangle{\pgfqpoint{0.050000in}{0.050000in}}{\pgfqpoint{2.419000in}{2.419000in}}%
\pgfusepath{clip}%
\pgfsetbuttcap%
\pgfsetroundjoin%
\definecolor{currentfill}{rgb}{0.866667,0.800000,0.466667}%
\pgfsetfillcolor{currentfill}%
\pgfsetfillopacity{0.504479}%
\pgfsetlinewidth{1.003750pt}%
\definecolor{currentstroke}{rgb}{0.866667,0.800000,0.466667}%
\pgfsetstrokecolor{currentstroke}%
\pgfsetstrokeopacity{0.504479}%
\pgfsetdash{}{0pt}%
\pgfpathmoveto{\pgfqpoint{8.230202in}{0.085362in}}%
\pgfpathcurveto{\pgfqpoint{8.239410in}{0.085362in}}{\pgfqpoint{8.248243in}{0.089021in}}{\pgfqpoint{8.254754in}{0.095532in}}%
\pgfpathcurveto{\pgfqpoint{8.261265in}{0.102043in}}{\pgfqpoint{8.264924in}{0.110876in}}{\pgfqpoint{8.264924in}{0.120084in}}%
\pgfpathcurveto{\pgfqpoint{8.264924in}{0.129293in}}{\pgfqpoint{8.261265in}{0.138125in}}{\pgfqpoint{8.254754in}{0.144637in}}%
\pgfpathcurveto{\pgfqpoint{8.248243in}{0.151148in}}{\pgfqpoint{8.239410in}{0.154806in}}{\pgfqpoint{8.230202in}{0.154806in}}%
\pgfpathcurveto{\pgfqpoint{8.220993in}{0.154806in}}{\pgfqpoint{8.212161in}{0.151148in}}{\pgfqpoint{8.205649in}{0.144637in}}%
\pgfpathcurveto{\pgfqpoint{8.199138in}{0.138125in}}{\pgfqpoint{8.195479in}{0.129293in}}{\pgfqpoint{8.195479in}{0.120084in}}%
\pgfpathcurveto{\pgfqpoint{8.195479in}{0.110876in}}{\pgfqpoint{8.199138in}{0.102043in}}{\pgfqpoint{8.205649in}{0.095532in}}%
\pgfpathcurveto{\pgfqpoint{8.212161in}{0.089021in}}{\pgfqpoint{8.220993in}{0.085362in}}{\pgfqpoint{8.230202in}{0.085362in}}%
\pgfpathlineto{\pgfqpoint{8.230202in}{0.085362in}}%
\pgfpathclose%
\pgfusepath{stroke,fill}%
\end{pgfscope}%
\begin{pgfscope}%
\pgfpathrectangle{\pgfqpoint{0.050000in}{0.050000in}}{\pgfqpoint{2.419000in}{2.419000in}}%
\pgfusepath{clip}%
\pgfsetbuttcap%
\pgfsetroundjoin%
\definecolor{currentfill}{rgb}{0.866667,0.800000,0.466667}%
\pgfsetfillcolor{currentfill}%
\pgfsetfillopacity{0.513522}%
\pgfsetlinewidth{1.003750pt}%
\definecolor{currentstroke}{rgb}{0.866667,0.800000,0.466667}%
\pgfsetstrokecolor{currentstroke}%
\pgfsetstrokeopacity{0.513522}%
\pgfsetdash{}{0pt}%
\pgfpathmoveto{\pgfqpoint{-5.733118in}{-0.097963in}}%
\pgfpathcurveto{\pgfqpoint{-5.723909in}{-0.097963in}}{\pgfqpoint{-5.715077in}{-0.094304in}}{\pgfqpoint{-5.708565in}{-0.087793in}}%
\pgfpathcurveto{\pgfqpoint{-5.702054in}{-0.081281in}}{\pgfqpoint{-5.698396in}{-0.072449in}}{\pgfqpoint{-5.698396in}{-0.063241in}}%
\pgfpathcurveto{\pgfqpoint{-5.698396in}{-0.054032in}}{\pgfqpoint{-5.702054in}{-0.045200in}}{\pgfqpoint{-5.708565in}{-0.038688in}}%
\pgfpathcurveto{\pgfqpoint{-5.715077in}{-0.032177in}}{\pgfqpoint{-5.723909in}{-0.028518in}}{\pgfqpoint{-5.733118in}{-0.028518in}}%
\pgfpathcurveto{\pgfqpoint{-5.742326in}{-0.028518in}}{\pgfqpoint{-5.751159in}{-0.032177in}}{\pgfqpoint{-5.757670in}{-0.038688in}}%
\pgfpathcurveto{\pgfqpoint{-5.764181in}{-0.045200in}}{\pgfqpoint{-5.767840in}{-0.054032in}}{\pgfqpoint{-5.767840in}{-0.063241in}}%
\pgfpathcurveto{\pgfqpoint{-5.767840in}{-0.072449in}}{\pgfqpoint{-5.764181in}{-0.081281in}}{\pgfqpoint{-5.757670in}{-0.087793in}}%
\pgfpathcurveto{\pgfqpoint{-5.751159in}{-0.094304in}}{\pgfqpoint{-5.742326in}{-0.097963in}}{\pgfqpoint{-5.733118in}{-0.097963in}}%
\pgfpathlineto{\pgfqpoint{-5.733118in}{-0.097963in}}%
\pgfpathclose%
\pgfusepath{stroke,fill}%
\end{pgfscope}%
\begin{pgfscope}%
\pgfpathrectangle{\pgfqpoint{0.050000in}{0.050000in}}{\pgfqpoint{2.419000in}{2.419000in}}%
\pgfusepath{clip}%
\pgfsetbuttcap%
\pgfsetroundjoin%
\definecolor{currentfill}{rgb}{0.866667,0.800000,0.466667}%
\pgfsetfillcolor{currentfill}%
\pgfsetfillopacity{0.513522}%
\pgfsetlinewidth{1.003750pt}%
\definecolor{currentstroke}{rgb}{0.866667,0.800000,0.466667}%
\pgfsetstrokecolor{currentstroke}%
\pgfsetstrokeopacity{0.513522}%
\pgfsetdash{}{0pt}%
\pgfpathmoveto{\pgfqpoint{0.130428in}{-0.097963in}}%
\pgfpathcurveto{\pgfqpoint{0.139637in}{-0.097963in}}{\pgfqpoint{0.148469in}{-0.094304in}}{\pgfqpoint{0.154981in}{-0.087793in}}%
\pgfpathcurveto{\pgfqpoint{0.161492in}{-0.081281in}}{\pgfqpoint{0.165151in}{-0.072449in}}{\pgfqpoint{0.165151in}{-0.063241in}}%
\pgfpathcurveto{\pgfqpoint{0.165151in}{-0.054032in}}{\pgfqpoint{0.161492in}{-0.045200in}}{\pgfqpoint{0.154981in}{-0.038688in}}%
\pgfpathcurveto{\pgfqpoint{0.148469in}{-0.032177in}}{\pgfqpoint{0.139637in}{-0.028518in}}{\pgfqpoint{0.130428in}{-0.028518in}}%
\pgfpathcurveto{\pgfqpoint{0.121220in}{-0.028518in}}{\pgfqpoint{0.112387in}{-0.032177in}}{\pgfqpoint{0.105876in}{-0.038688in}}%
\pgfpathcurveto{\pgfqpoint{0.099365in}{-0.045200in}}{\pgfqpoint{0.095706in}{-0.054032in}}{\pgfqpoint{0.095706in}{-0.063241in}}%
\pgfpathcurveto{\pgfqpoint{0.095706in}{-0.072449in}}{\pgfqpoint{0.099365in}{-0.081281in}}{\pgfqpoint{0.105876in}{-0.087793in}}%
\pgfpathcurveto{\pgfqpoint{0.112387in}{-0.094304in}}{\pgfqpoint{0.121220in}{-0.097963in}}{\pgfqpoint{0.130428in}{-0.097963in}}%
\pgfpathlineto{\pgfqpoint{0.130428in}{-0.097963in}}%
\pgfpathclose%
\pgfusepath{stroke,fill}%
\end{pgfscope}%
\begin{pgfscope}%
\pgfpathrectangle{\pgfqpoint{0.050000in}{0.050000in}}{\pgfqpoint{2.419000in}{2.419000in}}%
\pgfusepath{clip}%
\pgfsetbuttcap%
\pgfsetroundjoin%
\definecolor{currentfill}{rgb}{0.866667,0.800000,0.466667}%
\pgfsetfillcolor{currentfill}%
\pgfsetfillopacity{0.513522}%
\pgfsetlinewidth{1.003750pt}%
\definecolor{currentstroke}{rgb}{0.866667,0.800000,0.466667}%
\pgfsetstrokecolor{currentstroke}%
\pgfsetstrokeopacity{0.513522}%
\pgfsetdash{}{0pt}%
\pgfpathmoveto{\pgfqpoint{5.993975in}{-0.097963in}}%
\pgfpathcurveto{\pgfqpoint{6.003183in}{-0.097963in}}{\pgfqpoint{6.012016in}{-0.094304in}}{\pgfqpoint{6.018527in}{-0.087793in}}%
\pgfpathcurveto{\pgfqpoint{6.025038in}{-0.081281in}}{\pgfqpoint{6.028697in}{-0.072449in}}{\pgfqpoint{6.028697in}{-0.063241in}}%
\pgfpathcurveto{\pgfqpoint{6.028697in}{-0.054032in}}{\pgfqpoint{6.025038in}{-0.045200in}}{\pgfqpoint{6.018527in}{-0.038688in}}%
\pgfpathcurveto{\pgfqpoint{6.012016in}{-0.032177in}}{\pgfqpoint{6.003183in}{-0.028518in}}{\pgfqpoint{5.993975in}{-0.028518in}}%
\pgfpathcurveto{\pgfqpoint{5.984766in}{-0.028518in}}{\pgfqpoint{5.975934in}{-0.032177in}}{\pgfqpoint{5.969422in}{-0.038688in}}%
\pgfpathcurveto{\pgfqpoint{5.962911in}{-0.045200in}}{\pgfqpoint{5.959252in}{-0.054032in}}{\pgfqpoint{5.959252in}{-0.063241in}}%
\pgfpathcurveto{\pgfqpoint{5.959252in}{-0.072449in}}{\pgfqpoint{5.962911in}{-0.081281in}}{\pgfqpoint{5.969422in}{-0.087793in}}%
\pgfpathcurveto{\pgfqpoint{5.975934in}{-0.094304in}}{\pgfqpoint{5.984766in}{-0.097963in}}{\pgfqpoint{5.993975in}{-0.097963in}}%
\pgfpathlineto{\pgfqpoint{5.993975in}{-0.097963in}}%
\pgfpathclose%
\pgfusepath{stroke,fill}%
\end{pgfscope}%
\begin{pgfscope}%
\pgfpathrectangle{\pgfqpoint{0.050000in}{0.050000in}}{\pgfqpoint{2.419000in}{2.419000in}}%
\pgfusepath{clip}%
\pgfsetbuttcap%
\pgfsetroundjoin%
\definecolor{currentfill}{rgb}{0.866667,0.800000,0.466667}%
\pgfsetfillcolor{currentfill}%
\pgfsetfillopacity{0.522954}%
\pgfsetlinewidth{1.003750pt}%
\definecolor{currentstroke}{rgb}{0.866667,0.800000,0.466667}%
\pgfsetstrokecolor{currentstroke}%
\pgfsetstrokeopacity{0.522954}%
\pgfsetdash{}{0pt}%
\pgfpathmoveto{\pgfqpoint{-2.327729in}{-0.289160in}}%
\pgfpathcurveto{\pgfqpoint{-2.318520in}{-0.289160in}}{\pgfqpoint{-2.309688in}{-0.285502in}}{\pgfqpoint{-2.303176in}{-0.278990in}}%
\pgfpathcurveto{\pgfqpoint{-2.296665in}{-0.272479in}}{\pgfqpoint{-2.293007in}{-0.263646in}}{\pgfqpoint{-2.293007in}{-0.254438in}}%
\pgfpathcurveto{\pgfqpoint{-2.293007in}{-0.245229in}}{\pgfqpoint{-2.296665in}{-0.236397in}}{\pgfqpoint{-2.303176in}{-0.229886in}}%
\pgfpathcurveto{\pgfqpoint{-2.309688in}{-0.223374in}}{\pgfqpoint{-2.318520in}{-0.219716in}}{\pgfqpoint{-2.327729in}{-0.219716in}}%
\pgfpathcurveto{\pgfqpoint{-2.336937in}{-0.219716in}}{\pgfqpoint{-2.345770in}{-0.223374in}}{\pgfqpoint{-2.352281in}{-0.229886in}}%
\pgfpathcurveto{\pgfqpoint{-2.358792in}{-0.236397in}}{\pgfqpoint{-2.362451in}{-0.245229in}}{\pgfqpoint{-2.362451in}{-0.254438in}}%
\pgfpathcurveto{\pgfqpoint{-2.362451in}{-0.263646in}}{\pgfqpoint{-2.358792in}{-0.272479in}}{\pgfqpoint{-2.352281in}{-0.278990in}}%
\pgfpathcurveto{\pgfqpoint{-2.345770in}{-0.285502in}}{\pgfqpoint{-2.336937in}{-0.289160in}}{\pgfqpoint{-2.327729in}{-0.289160in}}%
\pgfpathlineto{\pgfqpoint{-2.327729in}{-0.289160in}}%
\pgfpathclose%
\pgfusepath{stroke,fill}%
\end{pgfscope}%
\begin{pgfscope}%
\pgfpathrectangle{\pgfqpoint{0.050000in}{0.050000in}}{\pgfqpoint{2.419000in}{2.419000in}}%
\pgfusepath{clip}%
\pgfsetbuttcap%
\pgfsetroundjoin%
\definecolor{currentfill}{rgb}{0.866667,0.800000,0.466667}%
\pgfsetfillcolor{currentfill}%
\pgfsetfillopacity{0.522954}%
\pgfsetlinewidth{1.003750pt}%
\definecolor{currentstroke}{rgb}{0.866667,0.800000,0.466667}%
\pgfsetstrokecolor{currentstroke}%
\pgfsetstrokeopacity{0.522954}%
\pgfsetdash{}{0pt}%
\pgfpathmoveto{\pgfqpoint{3.661717in}{-0.289160in}}%
\pgfpathcurveto{\pgfqpoint{3.670925in}{-0.289160in}}{\pgfqpoint{3.679758in}{-0.285502in}}{\pgfqpoint{3.686269in}{-0.278990in}}%
\pgfpathcurveto{\pgfqpoint{3.692781in}{-0.272479in}}{\pgfqpoint{3.696439in}{-0.263646in}}{\pgfqpoint{3.696439in}{-0.254438in}}%
\pgfpathcurveto{\pgfqpoint{3.696439in}{-0.245229in}}{\pgfqpoint{3.692781in}{-0.236397in}}{\pgfqpoint{3.686269in}{-0.229886in}}%
\pgfpathcurveto{\pgfqpoint{3.679758in}{-0.223374in}}{\pgfqpoint{3.670925in}{-0.219716in}}{\pgfqpoint{3.661717in}{-0.219716in}}%
\pgfpathcurveto{\pgfqpoint{3.652509in}{-0.219716in}}{\pgfqpoint{3.643676in}{-0.223374in}}{\pgfqpoint{3.637165in}{-0.229886in}}%
\pgfpathcurveto{\pgfqpoint{3.630653in}{-0.236397in}}{\pgfqpoint{3.626995in}{-0.245229in}}{\pgfqpoint{3.626995in}{-0.254438in}}%
\pgfpathcurveto{\pgfqpoint{3.626995in}{-0.263646in}}{\pgfqpoint{3.630653in}{-0.272479in}}{\pgfqpoint{3.637165in}{-0.278990in}}%
\pgfpathcurveto{\pgfqpoint{3.643676in}{-0.285502in}}{\pgfqpoint{3.652509in}{-0.289160in}}{\pgfqpoint{3.661717in}{-0.289160in}}%
\pgfpathlineto{\pgfqpoint{3.661717in}{-0.289160in}}%
\pgfpathclose%
\pgfusepath{stroke,fill}%
\end{pgfscope}%
\begin{pgfscope}%
\pgfpathrectangle{\pgfqpoint{0.050000in}{0.050000in}}{\pgfqpoint{2.419000in}{2.419000in}}%
\pgfusepath{clip}%
\pgfsetbuttcap%
\pgfsetroundjoin%
\definecolor{currentfill}{rgb}{0.866667,0.800000,0.466667}%
\pgfsetfillcolor{currentfill}%
\pgfsetfillopacity{0.522954}%
\pgfsetlinewidth{1.003750pt}%
\definecolor{currentstroke}{rgb}{0.866667,0.800000,0.466667}%
\pgfsetstrokecolor{currentstroke}%
\pgfsetstrokeopacity{0.522954}%
\pgfsetdash{}{0pt}%
\pgfpathmoveto{\pgfqpoint{9.651163in}{-0.289160in}}%
\pgfpathcurveto{\pgfqpoint{9.660371in}{-0.289160in}}{\pgfqpoint{9.669204in}{-0.285502in}}{\pgfqpoint{9.675715in}{-0.278990in}}%
\pgfpathcurveto{\pgfqpoint{9.682226in}{-0.272479in}}{\pgfqpoint{9.685885in}{-0.263646in}}{\pgfqpoint{9.685885in}{-0.254438in}}%
\pgfpathcurveto{\pgfqpoint{9.685885in}{-0.245229in}}{\pgfqpoint{9.682226in}{-0.236397in}}{\pgfqpoint{9.675715in}{-0.229886in}}%
\pgfpathcurveto{\pgfqpoint{9.669204in}{-0.223374in}}{\pgfqpoint{9.660371in}{-0.219716in}}{\pgfqpoint{9.651163in}{-0.219716in}}%
\pgfpathcurveto{\pgfqpoint{9.641954in}{-0.219716in}}{\pgfqpoint{9.633122in}{-0.223374in}}{\pgfqpoint{9.626611in}{-0.229886in}}%
\pgfpathcurveto{\pgfqpoint{9.620099in}{-0.236397in}}{\pgfqpoint{9.616441in}{-0.245229in}}{\pgfqpoint{9.616441in}{-0.254438in}}%
\pgfpathcurveto{\pgfqpoint{9.616441in}{-0.263646in}}{\pgfqpoint{9.620099in}{-0.272479in}}{\pgfqpoint{9.626611in}{-0.278990in}}%
\pgfpathcurveto{\pgfqpoint{9.633122in}{-0.285502in}}{\pgfqpoint{9.641954in}{-0.289160in}}{\pgfqpoint{9.651163in}{-0.289160in}}%
\pgfpathlineto{\pgfqpoint{9.651163in}{-0.289160in}}%
\pgfpathclose%
\pgfusepath{stroke,fill}%
\end{pgfscope}%
\begin{pgfscope}%
\pgfpathrectangle{\pgfqpoint{0.050000in}{0.050000in}}{\pgfqpoint{2.419000in}{2.419000in}}%
\pgfusepath{clip}%
\pgfsetbuttcap%
\pgfsetroundjoin%
\definecolor{currentfill}{rgb}{0.866667,0.800000,0.466667}%
\pgfsetfillcolor{currentfill}%
\pgfsetfillopacity{0.532799}%
\pgfsetlinewidth{1.003750pt}%
\definecolor{currentstroke}{rgb}{0.866667,0.800000,0.466667}%
\pgfsetstrokecolor{currentstroke}%
\pgfsetstrokeopacity{0.532799}%
\pgfsetdash{}{0pt}%
\pgfpathmoveto{\pgfqpoint{1.227107in}{-0.488748in}}%
\pgfpathcurveto{\pgfqpoint{1.236316in}{-0.488748in}}{\pgfqpoint{1.245148in}{-0.485090in}}{\pgfqpoint{1.251660in}{-0.478578in}}%
\pgfpathcurveto{\pgfqpoint{1.258171in}{-0.472067in}}{\pgfqpoint{1.261830in}{-0.463234in}}{\pgfqpoint{1.261830in}{-0.454026in}}%
\pgfpathcurveto{\pgfqpoint{1.261830in}{-0.444818in}}{\pgfqpoint{1.258171in}{-0.435985in}}{\pgfqpoint{1.251660in}{-0.429474in}}%
\pgfpathcurveto{\pgfqpoint{1.245148in}{-0.422962in}}{\pgfqpoint{1.236316in}{-0.419304in}}{\pgfqpoint{1.227107in}{-0.419304in}}%
\pgfpathcurveto{\pgfqpoint{1.217899in}{-0.419304in}}{\pgfqpoint{1.209066in}{-0.422962in}}{\pgfqpoint{1.202555in}{-0.429474in}}%
\pgfpathcurveto{\pgfqpoint{1.196044in}{-0.435985in}}{\pgfqpoint{1.192385in}{-0.444818in}}{\pgfqpoint{1.192385in}{-0.454026in}}%
\pgfpathcurveto{\pgfqpoint{1.192385in}{-0.463234in}}{\pgfqpoint{1.196044in}{-0.472067in}}{\pgfqpoint{1.202555in}{-0.478578in}}%
\pgfpathcurveto{\pgfqpoint{1.209066in}{-0.485090in}}{\pgfqpoint{1.217899in}{-0.488748in}}{\pgfqpoint{1.227107in}{-0.488748in}}%
\pgfpathlineto{\pgfqpoint{1.227107in}{-0.488748in}}%
\pgfpathclose%
\pgfusepath{stroke,fill}%
\end{pgfscope}%
\begin{pgfscope}%
\pgfpathrectangle{\pgfqpoint{0.050000in}{0.050000in}}{\pgfqpoint{2.419000in}{2.419000in}}%
\pgfusepath{clip}%
\pgfsetbuttcap%
\pgfsetroundjoin%
\definecolor{currentfill}{rgb}{0.866667,0.800000,0.466667}%
\pgfsetfillcolor{currentfill}%
\pgfsetfillopacity{0.532799}%
\pgfsetlinewidth{1.003750pt}%
\definecolor{currentstroke}{rgb}{0.866667,0.800000,0.466667}%
\pgfsetstrokecolor{currentstroke}%
\pgfsetstrokeopacity{0.532799}%
\pgfsetdash{}{0pt}%
\pgfpathmoveto{\pgfqpoint{-4.893763in}{-0.488748in}}%
\pgfpathcurveto{\pgfqpoint{-4.884555in}{-0.488748in}}{\pgfqpoint{-4.875722in}{-0.485090in}}{\pgfqpoint{-4.869211in}{-0.478578in}}%
\pgfpathcurveto{\pgfqpoint{-4.862700in}{-0.472067in}}{\pgfqpoint{-4.859041in}{-0.463234in}}{\pgfqpoint{-4.859041in}{-0.454026in}}%
\pgfpathcurveto{\pgfqpoint{-4.859041in}{-0.444818in}}{\pgfqpoint{-4.862700in}{-0.435985in}}{\pgfqpoint{-4.869211in}{-0.429474in}}%
\pgfpathcurveto{\pgfqpoint{-4.875722in}{-0.422962in}}{\pgfqpoint{-4.884555in}{-0.419304in}}{\pgfqpoint{-4.893763in}{-0.419304in}}%
\pgfpathcurveto{\pgfqpoint{-4.902972in}{-0.419304in}}{\pgfqpoint{-4.911804in}{-0.422962in}}{\pgfqpoint{-4.918315in}{-0.429474in}}%
\pgfpathcurveto{\pgfqpoint{-4.924827in}{-0.435985in}}{\pgfqpoint{-4.928485in}{-0.444818in}}{\pgfqpoint{-4.928485in}{-0.454026in}}%
\pgfpathcurveto{\pgfqpoint{-4.928485in}{-0.463234in}}{\pgfqpoint{-4.924827in}{-0.472067in}}{\pgfqpoint{-4.918315in}{-0.478578in}}%
\pgfpathcurveto{\pgfqpoint{-4.911804in}{-0.485090in}}{\pgfqpoint{-4.902972in}{-0.488748in}}{\pgfqpoint{-4.893763in}{-0.488748in}}%
\pgfpathlineto{\pgfqpoint{-4.893763in}{-0.488748in}}%
\pgfpathclose%
\pgfusepath{stroke,fill}%
\end{pgfscope}%
\begin{pgfscope}%
\pgfpathrectangle{\pgfqpoint{0.050000in}{0.050000in}}{\pgfqpoint{2.419000in}{2.419000in}}%
\pgfusepath{clip}%
\pgfsetbuttcap%
\pgfsetroundjoin%
\definecolor{currentfill}{rgb}{0.866667,0.800000,0.466667}%
\pgfsetfillcolor{currentfill}%
\pgfsetfillopacity{0.532799}%
\pgfsetlinewidth{1.003750pt}%
\definecolor{currentstroke}{rgb}{0.866667,0.800000,0.466667}%
\pgfsetstrokecolor{currentstroke}%
\pgfsetstrokeopacity{0.532799}%
\pgfsetdash{}{0pt}%
\pgfpathmoveto{\pgfqpoint{7.347978in}{-0.488748in}}%
\pgfpathcurveto{\pgfqpoint{7.357186in}{-0.488748in}}{\pgfqpoint{7.366019in}{-0.485090in}}{\pgfqpoint{7.372530in}{-0.478578in}}%
\pgfpathcurveto{\pgfqpoint{7.379041in}{-0.472067in}}{\pgfqpoint{7.382700in}{-0.463234in}}{\pgfqpoint{7.382700in}{-0.454026in}}%
\pgfpathcurveto{\pgfqpoint{7.382700in}{-0.444818in}}{\pgfqpoint{7.379041in}{-0.435985in}}{\pgfqpoint{7.372530in}{-0.429474in}}%
\pgfpathcurveto{\pgfqpoint{7.366019in}{-0.422962in}}{\pgfqpoint{7.357186in}{-0.419304in}}{\pgfqpoint{7.347978in}{-0.419304in}}%
\pgfpathcurveto{\pgfqpoint{7.338769in}{-0.419304in}}{\pgfqpoint{7.329937in}{-0.422962in}}{\pgfqpoint{7.323425in}{-0.429474in}}%
\pgfpathcurveto{\pgfqpoint{7.316914in}{-0.435985in}}{\pgfqpoint{7.313256in}{-0.444818in}}{\pgfqpoint{7.313256in}{-0.454026in}}%
\pgfpathcurveto{\pgfqpoint{7.313256in}{-0.463234in}}{\pgfqpoint{7.316914in}{-0.472067in}}{\pgfqpoint{7.323425in}{-0.478578in}}%
\pgfpathcurveto{\pgfqpoint{7.329937in}{-0.485090in}}{\pgfqpoint{7.338769in}{-0.488748in}}{\pgfqpoint{7.347978in}{-0.488748in}}%
\pgfpathlineto{\pgfqpoint{7.347978in}{-0.488748in}}%
\pgfpathclose%
\pgfusepath{stroke,fill}%
\end{pgfscope}%
\begin{pgfscope}%
\pgfpathrectangle{\pgfqpoint{0.050000in}{0.050000in}}{\pgfqpoint{2.419000in}{2.419000in}}%
\pgfusepath{clip}%
\pgfsetbuttcap%
\pgfsetroundjoin%
\definecolor{currentfill}{rgb}{0.866667,0.800000,0.466667}%
\pgfsetfillcolor{currentfill}%
\pgfsetfillopacity{0.543086}%
\pgfsetlinewidth{1.003750pt}%
\definecolor{currentstroke}{rgb}{0.866667,0.800000,0.466667}%
\pgfsetstrokecolor{currentstroke}%
\pgfsetstrokeopacity{0.543086}%
\pgfsetdash{}{0pt}%
\pgfpathmoveto{\pgfqpoint{-1.316743in}{-0.697292in}}%
\pgfpathcurveto{\pgfqpoint{-1.307535in}{-0.697292in}}{\pgfqpoint{-1.298702in}{-0.693633in}}{\pgfqpoint{-1.292191in}{-0.687122in}}%
\pgfpathcurveto{\pgfqpoint{-1.285680in}{-0.680611in}}{\pgfqpoint{-1.282021in}{-0.671778in}}{\pgfqpoint{-1.282021in}{-0.662570in}}%
\pgfpathcurveto{\pgfqpoint{-1.282021in}{-0.653361in}}{\pgfqpoint{-1.285680in}{-0.644529in}}{\pgfqpoint{-1.292191in}{-0.638017in}}%
\pgfpathcurveto{\pgfqpoint{-1.298702in}{-0.631506in}}{\pgfqpoint{-1.307535in}{-0.627847in}}{\pgfqpoint{-1.316743in}{-0.627847in}}%
\pgfpathcurveto{\pgfqpoint{-1.325952in}{-0.627847in}}{\pgfqpoint{-1.334784in}{-0.631506in}}{\pgfqpoint{-1.341296in}{-0.638017in}}%
\pgfpathcurveto{\pgfqpoint{-1.347807in}{-0.644529in}}{\pgfqpoint{-1.351466in}{-0.653361in}}{\pgfqpoint{-1.351466in}{-0.662570in}}%
\pgfpathcurveto{\pgfqpoint{-1.351466in}{-0.671778in}}{\pgfqpoint{-1.347807in}{-0.680611in}}{\pgfqpoint{-1.341296in}{-0.687122in}}%
\pgfpathcurveto{\pgfqpoint{-1.334784in}{-0.693633in}}{\pgfqpoint{-1.325952in}{-0.697292in}}{\pgfqpoint{-1.316743in}{-0.697292in}}%
\pgfpathlineto{\pgfqpoint{-1.316743in}{-0.697292in}}%
\pgfpathclose%
\pgfusepath{stroke,fill}%
\end{pgfscope}%
\begin{pgfscope}%
\pgfpathrectangle{\pgfqpoint{0.050000in}{0.050000in}}{\pgfqpoint{2.419000in}{2.419000in}}%
\pgfusepath{clip}%
\pgfsetbuttcap%
\pgfsetroundjoin%
\definecolor{currentfill}{rgb}{0.866667,0.800000,0.466667}%
\pgfsetfillcolor{currentfill}%
\pgfsetfillopacity{0.543086}%
\pgfsetlinewidth{1.003750pt}%
\definecolor{currentstroke}{rgb}{0.866667,0.800000,0.466667}%
\pgfsetstrokecolor{currentstroke}%
\pgfsetstrokeopacity{0.543086}%
\pgfsetdash{}{0pt}%
\pgfpathmoveto{\pgfqpoint{-7.574936in}{-0.697292in}}%
\pgfpathcurveto{\pgfqpoint{-7.565727in}{-0.697292in}}{\pgfqpoint{-7.556895in}{-0.693633in}}{\pgfqpoint{-7.550383in}{-0.687122in}}%
\pgfpathcurveto{\pgfqpoint{-7.543872in}{-0.680611in}}{\pgfqpoint{-7.540213in}{-0.671778in}}{\pgfqpoint{-7.540213in}{-0.662570in}}%
\pgfpathcurveto{\pgfqpoint{-7.540213in}{-0.653361in}}{\pgfqpoint{-7.543872in}{-0.644529in}}{\pgfqpoint{-7.550383in}{-0.638017in}}%
\pgfpathcurveto{\pgfqpoint{-7.556895in}{-0.631506in}}{\pgfqpoint{-7.565727in}{-0.627847in}}{\pgfqpoint{-7.574936in}{-0.627847in}}%
\pgfpathcurveto{\pgfqpoint{-7.584144in}{-0.627847in}}{\pgfqpoint{-7.592977in}{-0.631506in}}{\pgfqpoint{-7.599488in}{-0.638017in}}%
\pgfpathcurveto{\pgfqpoint{-7.605999in}{-0.644529in}}{\pgfqpoint{-7.609658in}{-0.653361in}}{\pgfqpoint{-7.609658in}{-0.662570in}}%
\pgfpathcurveto{\pgfqpoint{-7.609658in}{-0.671778in}}{\pgfqpoint{-7.605999in}{-0.680611in}}{\pgfqpoint{-7.599488in}{-0.687122in}}%
\pgfpathcurveto{\pgfqpoint{-7.592977in}{-0.693633in}}{\pgfqpoint{-7.584144in}{-0.697292in}}{\pgfqpoint{-7.574936in}{-0.697292in}}%
\pgfpathlineto{\pgfqpoint{-7.574936in}{-0.697292in}}%
\pgfpathclose%
\pgfusepath{stroke,fill}%
\end{pgfscope}%
\begin{pgfscope}%
\pgfpathrectangle{\pgfqpoint{0.050000in}{0.050000in}}{\pgfqpoint{2.419000in}{2.419000in}}%
\pgfusepath{clip}%
\pgfsetbuttcap%
\pgfsetroundjoin%
\definecolor{currentfill}{rgb}{0.866667,0.800000,0.466667}%
\pgfsetfillcolor{currentfill}%
\pgfsetfillopacity{0.543086}%
\pgfsetlinewidth{1.003750pt}%
\definecolor{currentstroke}{rgb}{0.866667,0.800000,0.466667}%
\pgfsetstrokecolor{currentstroke}%
\pgfsetstrokeopacity{0.543086}%
\pgfsetdash{}{0pt}%
\pgfpathmoveto{\pgfqpoint{4.941449in}{-0.697292in}}%
\pgfpathcurveto{\pgfqpoint{4.950657in}{-0.697292in}}{\pgfqpoint{4.959490in}{-0.693633in}}{\pgfqpoint{4.966001in}{-0.687122in}}%
\pgfpathcurveto{\pgfqpoint{4.972512in}{-0.680611in}}{\pgfqpoint{4.976171in}{-0.671778in}}{\pgfqpoint{4.976171in}{-0.662570in}}%
\pgfpathcurveto{\pgfqpoint{4.976171in}{-0.653361in}}{\pgfqpoint{4.972512in}{-0.644529in}}{\pgfqpoint{4.966001in}{-0.638017in}}%
\pgfpathcurveto{\pgfqpoint{4.959490in}{-0.631506in}}{\pgfqpoint{4.950657in}{-0.627847in}}{\pgfqpoint{4.941449in}{-0.627847in}}%
\pgfpathcurveto{\pgfqpoint{4.932240in}{-0.627847in}}{\pgfqpoint{4.923408in}{-0.631506in}}{\pgfqpoint{4.916896in}{-0.638017in}}%
\pgfpathcurveto{\pgfqpoint{4.910385in}{-0.644529in}}{\pgfqpoint{4.906727in}{-0.653361in}}{\pgfqpoint{4.906727in}{-0.662570in}}%
\pgfpathcurveto{\pgfqpoint{4.906727in}{-0.671778in}}{\pgfqpoint{4.910385in}{-0.680611in}}{\pgfqpoint{4.916896in}{-0.687122in}}%
\pgfpathcurveto{\pgfqpoint{4.923408in}{-0.693633in}}{\pgfqpoint{4.932240in}{-0.697292in}}{\pgfqpoint{4.941449in}{-0.697292in}}%
\pgfpathlineto{\pgfqpoint{4.941449in}{-0.697292in}}%
\pgfpathclose%
\pgfusepath{stroke,fill}%
\end{pgfscope}%
\begin{pgfscope}%
\pgfpathrectangle{\pgfqpoint{0.050000in}{0.050000in}}{\pgfqpoint{2.419000in}{2.419000in}}%
\pgfusepath{clip}%
\pgfsetbuttcap%
\pgfsetroundjoin%
\definecolor{currentfill}{rgb}{0.866667,0.800000,0.466667}%
\pgfsetfillcolor{currentfill}%
\pgfsetfillopacity{0.553846}%
\pgfsetlinewidth{1.003750pt}%
\definecolor{currentstroke}{rgb}{0.866667,0.800000,0.466667}%
\pgfsetstrokecolor{currentstroke}%
\pgfsetstrokeopacity{0.553846}%
\pgfsetdash{}{0pt}%
\pgfpathmoveto{\pgfqpoint{-3.977356in}{-0.915408in}}%
\pgfpathcurveto{\pgfqpoint{-3.968148in}{-0.915408in}}{\pgfqpoint{-3.959315in}{-0.911749in}}{\pgfqpoint{-3.952804in}{-0.905238in}}%
\pgfpathcurveto{\pgfqpoint{-3.946293in}{-0.898726in}}{\pgfqpoint{-3.942634in}{-0.889894in}}{\pgfqpoint{-3.942634in}{-0.880685in}}%
\pgfpathcurveto{\pgfqpoint{-3.942634in}{-0.871477in}}{\pgfqpoint{-3.946293in}{-0.862644in}}{\pgfqpoint{-3.952804in}{-0.856133in}}%
\pgfpathcurveto{\pgfqpoint{-3.959315in}{-0.849622in}}{\pgfqpoint{-3.968148in}{-0.845963in}}{\pgfqpoint{-3.977356in}{-0.845963in}}%
\pgfpathcurveto{\pgfqpoint{-3.986565in}{-0.845963in}}{\pgfqpoint{-3.995397in}{-0.849622in}}{\pgfqpoint{-4.001909in}{-0.856133in}}%
\pgfpathcurveto{\pgfqpoint{-4.008420in}{-0.862644in}}{\pgfqpoint{-4.012078in}{-0.871477in}}{\pgfqpoint{-4.012078in}{-0.880685in}}%
\pgfpathcurveto{\pgfqpoint{-4.012078in}{-0.889894in}}{\pgfqpoint{-4.008420in}{-0.898726in}}{\pgfqpoint{-4.001909in}{-0.905238in}}%
\pgfpathcurveto{\pgfqpoint{-3.995397in}{-0.911749in}}{\pgfqpoint{-3.986565in}{-0.915408in}}{\pgfqpoint{-3.977356in}{-0.915408in}}%
\pgfpathlineto{\pgfqpoint{-3.977356in}{-0.915408in}}%
\pgfpathclose%
\pgfusepath{stroke,fill}%
\end{pgfscope}%
\begin{pgfscope}%
\pgfpathrectangle{\pgfqpoint{0.050000in}{0.050000in}}{\pgfqpoint{2.419000in}{2.419000in}}%
\pgfusepath{clip}%
\pgfsetbuttcap%
\pgfsetroundjoin%
\definecolor{currentfill}{rgb}{0.866667,0.800000,0.466667}%
\pgfsetfillcolor{currentfill}%
\pgfsetfillopacity{0.553846}%
\pgfsetlinewidth{1.003750pt}%
\definecolor{currentstroke}{rgb}{0.866667,0.800000,0.466667}%
\pgfsetstrokecolor{currentstroke}%
\pgfsetstrokeopacity{0.553846}%
\pgfsetdash{}{0pt}%
\pgfpathmoveto{\pgfqpoint{2.424461in}{-0.915408in}}%
\pgfpathcurveto{\pgfqpoint{2.433669in}{-0.915408in}}{\pgfqpoint{2.442502in}{-0.911749in}}{\pgfqpoint{2.449013in}{-0.905238in}}%
\pgfpathcurveto{\pgfqpoint{2.455524in}{-0.898726in}}{\pgfqpoint{2.459183in}{-0.889894in}}{\pgfqpoint{2.459183in}{-0.880685in}}%
\pgfpathcurveto{\pgfqpoint{2.459183in}{-0.871477in}}{\pgfqpoint{2.455524in}{-0.862644in}}{\pgfqpoint{2.449013in}{-0.856133in}}%
\pgfpathcurveto{\pgfqpoint{2.442502in}{-0.849622in}}{\pgfqpoint{2.433669in}{-0.845963in}}{\pgfqpoint{2.424461in}{-0.845963in}}%
\pgfpathcurveto{\pgfqpoint{2.415252in}{-0.845963in}}{\pgfqpoint{2.406420in}{-0.849622in}}{\pgfqpoint{2.399908in}{-0.856133in}}%
\pgfpathcurveto{\pgfqpoint{2.393397in}{-0.862644in}}{\pgfqpoint{2.389739in}{-0.871477in}}{\pgfqpoint{2.389739in}{-0.880685in}}%
\pgfpathcurveto{\pgfqpoint{2.389739in}{-0.889894in}}{\pgfqpoint{2.393397in}{-0.898726in}}{\pgfqpoint{2.399908in}{-0.905238in}}%
\pgfpathcurveto{\pgfqpoint{2.406420in}{-0.911749in}}{\pgfqpoint{2.415252in}{-0.915408in}}{\pgfqpoint{2.424461in}{-0.915408in}}%
\pgfpathlineto{\pgfqpoint{2.424461in}{-0.915408in}}%
\pgfpathclose%
\pgfusepath{stroke,fill}%
\end{pgfscope}%
\begin{pgfscope}%
\pgfpathrectangle{\pgfqpoint{0.050000in}{0.050000in}}{\pgfqpoint{2.419000in}{2.419000in}}%
\pgfusepath{clip}%
\pgfsetbuttcap%
\pgfsetroundjoin%
\definecolor{currentfill}{rgb}{0.866667,0.800000,0.466667}%
\pgfsetfillcolor{currentfill}%
\pgfsetfillopacity{0.553846}%
\pgfsetlinewidth{1.003750pt}%
\definecolor{currentstroke}{rgb}{0.866667,0.800000,0.466667}%
\pgfsetstrokecolor{currentstroke}%
\pgfsetstrokeopacity{0.553846}%
\pgfsetdash{}{0pt}%
\pgfpathmoveto{\pgfqpoint{8.826278in}{-0.915408in}}%
\pgfpathcurveto{\pgfqpoint{8.835486in}{-0.915408in}}{\pgfqpoint{8.844319in}{-0.911749in}}{\pgfqpoint{8.850830in}{-0.905238in}}%
\pgfpathcurveto{\pgfqpoint{8.857341in}{-0.898726in}}{\pgfqpoint{8.861000in}{-0.889894in}}{\pgfqpoint{8.861000in}{-0.880685in}}%
\pgfpathcurveto{\pgfqpoint{8.861000in}{-0.871477in}}{\pgfqpoint{8.857341in}{-0.862644in}}{\pgfqpoint{8.850830in}{-0.856133in}}%
\pgfpathcurveto{\pgfqpoint{8.844319in}{-0.849622in}}{\pgfqpoint{8.835486in}{-0.845963in}}{\pgfqpoint{8.826278in}{-0.845963in}}%
\pgfpathcurveto{\pgfqpoint{8.817069in}{-0.845963in}}{\pgfqpoint{8.808237in}{-0.849622in}}{\pgfqpoint{8.801725in}{-0.856133in}}%
\pgfpathcurveto{\pgfqpoint{8.795214in}{-0.862644in}}{\pgfqpoint{8.791555in}{-0.871477in}}{\pgfqpoint{8.791555in}{-0.880685in}}%
\pgfpathcurveto{\pgfqpoint{8.791555in}{-0.889894in}}{\pgfqpoint{8.795214in}{-0.898726in}}{\pgfqpoint{8.801725in}{-0.905238in}}%
\pgfpathcurveto{\pgfqpoint{8.808237in}{-0.911749in}}{\pgfqpoint{8.817069in}{-0.915408in}}{\pgfqpoint{8.826278in}{-0.915408in}}%
\pgfpathlineto{\pgfqpoint{8.826278in}{-0.915408in}}%
\pgfpathclose%
\pgfusepath{stroke,fill}%
\end{pgfscope}%
\begin{pgfscope}%
\pgfpathrectangle{\pgfqpoint{0.050000in}{0.050000in}}{\pgfqpoint{2.419000in}{2.419000in}}%
\pgfusepath{clip}%
\pgfsetbuttcap%
\pgfsetroundjoin%
\definecolor{currentfill}{rgb}{0.866667,0.800000,0.466667}%
\pgfsetfillcolor{currentfill}%
\pgfsetfillopacity{0.565111}%
\pgfsetlinewidth{1.003750pt}%
\definecolor{currentstroke}{rgb}{0.866667,0.800000,0.466667}%
\pgfsetstrokecolor{currentstroke}%
\pgfsetstrokeopacity{0.565111}%
\pgfsetdash{}{0pt}%
\pgfpathmoveto{\pgfqpoint{-6.762959in}{-1.143770in}}%
\pgfpathcurveto{\pgfqpoint{-6.753751in}{-1.143770in}}{\pgfqpoint{-6.744918in}{-1.140111in}}{\pgfqpoint{-6.738407in}{-1.133600in}}%
\pgfpathcurveto{\pgfqpoint{-6.731895in}{-1.127089in}}{\pgfqpoint{-6.728237in}{-1.118256in}}{\pgfqpoint{-6.728237in}{-1.109048in}}%
\pgfpathcurveto{\pgfqpoint{-6.728237in}{-1.099839in}}{\pgfqpoint{-6.731895in}{-1.091007in}}{\pgfqpoint{-6.738407in}{-1.084495in}}%
\pgfpathcurveto{\pgfqpoint{-6.744918in}{-1.077984in}}{\pgfqpoint{-6.753751in}{-1.074325in}}{\pgfqpoint{-6.762959in}{-1.074325in}}%
\pgfpathcurveto{\pgfqpoint{-6.772167in}{-1.074325in}}{\pgfqpoint{-6.781000in}{-1.077984in}}{\pgfqpoint{-6.787511in}{-1.084495in}}%
\pgfpathcurveto{\pgfqpoint{-6.794023in}{-1.091007in}}{\pgfqpoint{-6.797681in}{-1.099839in}}{\pgfqpoint{-6.797681in}{-1.109048in}}%
\pgfpathcurveto{\pgfqpoint{-6.797681in}{-1.118256in}}{\pgfqpoint{-6.794023in}{-1.127089in}}{\pgfqpoint{-6.787511in}{-1.133600in}}%
\pgfpathcurveto{\pgfqpoint{-6.781000in}{-1.140111in}}{\pgfqpoint{-6.772167in}{-1.143770in}}{\pgfqpoint{-6.762959in}{-1.143770in}}%
\pgfpathlineto{\pgfqpoint{-6.762959in}{-1.143770in}}%
\pgfpathclose%
\pgfusepath{stroke,fill}%
\end{pgfscope}%
\begin{pgfscope}%
\pgfpathrectangle{\pgfqpoint{0.050000in}{0.050000in}}{\pgfqpoint{2.419000in}{2.419000in}}%
\pgfusepath{clip}%
\pgfsetbuttcap%
\pgfsetroundjoin%
\definecolor{currentfill}{rgb}{0.866667,0.800000,0.466667}%
\pgfsetfillcolor{currentfill}%
\pgfsetfillopacity{0.565111}%
\pgfsetlinewidth{1.003750pt}%
\definecolor{currentstroke}{rgb}{0.866667,0.800000,0.466667}%
\pgfsetstrokecolor{currentstroke}%
\pgfsetstrokeopacity{0.565111}%
\pgfsetdash{}{0pt}%
\pgfpathmoveto{\pgfqpoint{-0.210770in}{-1.143770in}}%
\pgfpathcurveto{\pgfqpoint{-0.201562in}{-1.143770in}}{\pgfqpoint{-0.192729in}{-1.140111in}}{\pgfqpoint{-0.186218in}{-1.133600in}}%
\pgfpathcurveto{\pgfqpoint{-0.179706in}{-1.127089in}}{\pgfqpoint{-0.176048in}{-1.118256in}}{\pgfqpoint{-0.176048in}{-1.109048in}}%
\pgfpathcurveto{\pgfqpoint{-0.176048in}{-1.099839in}}{\pgfqpoint{-0.179706in}{-1.091007in}}{\pgfqpoint{-0.186218in}{-1.084495in}}%
\pgfpathcurveto{\pgfqpoint{-0.192729in}{-1.077984in}}{\pgfqpoint{-0.201562in}{-1.074325in}}{\pgfqpoint{-0.210770in}{-1.074325in}}%
\pgfpathcurveto{\pgfqpoint{-0.219979in}{-1.074325in}}{\pgfqpoint{-0.228811in}{-1.077984in}}{\pgfqpoint{-0.235322in}{-1.084495in}}%
\pgfpathcurveto{\pgfqpoint{-0.241834in}{-1.091007in}}{\pgfqpoint{-0.245492in}{-1.099839in}}{\pgfqpoint{-0.245492in}{-1.109048in}}%
\pgfpathcurveto{\pgfqpoint{-0.245492in}{-1.118256in}}{\pgfqpoint{-0.241834in}{-1.127089in}}{\pgfqpoint{-0.235322in}{-1.133600in}}%
\pgfpathcurveto{\pgfqpoint{-0.228811in}{-1.140111in}}{\pgfqpoint{-0.219979in}{-1.143770in}}{\pgfqpoint{-0.210770in}{-1.143770in}}%
\pgfpathlineto{\pgfqpoint{-0.210770in}{-1.143770in}}%
\pgfpathclose%
\pgfusepath{stroke,fill}%
\end{pgfscope}%
\begin{pgfscope}%
\pgfpathrectangle{\pgfqpoint{0.050000in}{0.050000in}}{\pgfqpoint{2.419000in}{2.419000in}}%
\pgfusepath{clip}%
\pgfsetbuttcap%
\pgfsetroundjoin%
\definecolor{currentfill}{rgb}{0.866667,0.800000,0.466667}%
\pgfsetfillcolor{currentfill}%
\pgfsetfillopacity{0.565111}%
\pgfsetlinewidth{1.003750pt}%
\definecolor{currentstroke}{rgb}{0.866667,0.800000,0.466667}%
\pgfsetstrokecolor{currentstroke}%
\pgfsetstrokeopacity{0.565111}%
\pgfsetdash{}{0pt}%
\pgfpathmoveto{\pgfqpoint{6.341419in}{-1.143770in}}%
\pgfpathcurveto{\pgfqpoint{6.350627in}{-1.143770in}}{\pgfqpoint{6.359460in}{-1.140111in}}{\pgfqpoint{6.365971in}{-1.133600in}}%
\pgfpathcurveto{\pgfqpoint{6.372482in}{-1.127089in}}{\pgfqpoint{6.376141in}{-1.118256in}}{\pgfqpoint{6.376141in}{-1.109048in}}%
\pgfpathcurveto{\pgfqpoint{6.376141in}{-1.099839in}}{\pgfqpoint{6.372482in}{-1.091007in}}{\pgfqpoint{6.365971in}{-1.084495in}}%
\pgfpathcurveto{\pgfqpoint{6.359460in}{-1.077984in}}{\pgfqpoint{6.350627in}{-1.074325in}}{\pgfqpoint{6.341419in}{-1.074325in}}%
\pgfpathcurveto{\pgfqpoint{6.332210in}{-1.074325in}}{\pgfqpoint{6.323378in}{-1.077984in}}{\pgfqpoint{6.316866in}{-1.084495in}}%
\pgfpathcurveto{\pgfqpoint{6.310355in}{-1.091007in}}{\pgfqpoint{6.306697in}{-1.099839in}}{\pgfqpoint{6.306697in}{-1.109048in}}%
\pgfpathcurveto{\pgfqpoint{6.306697in}{-1.118256in}}{\pgfqpoint{6.310355in}{-1.127089in}}{\pgfqpoint{6.316866in}{-1.133600in}}%
\pgfpathcurveto{\pgfqpoint{6.323378in}{-1.140111in}}{\pgfqpoint{6.332210in}{-1.143770in}}{\pgfqpoint{6.341419in}{-1.143770in}}%
\pgfpathlineto{\pgfqpoint{6.341419in}{-1.143770in}}%
\pgfpathclose%
\pgfusepath{stroke,fill}%
\end{pgfscope}%
\begin{pgfscope}%
\pgfpathrectangle{\pgfqpoint{0.050000in}{0.050000in}}{\pgfqpoint{2.419000in}{2.419000in}}%
\pgfusepath{clip}%
\pgfsetbuttcap%
\pgfsetroundjoin%
\definecolor{currentfill}{rgb}{0.866667,0.800000,0.466667}%
\pgfsetfillcolor{currentfill}%
\pgfsetfillopacity{0.576917}%
\pgfsetlinewidth{1.003750pt}%
\definecolor{currentstroke}{rgb}{0.866667,0.800000,0.466667}%
\pgfsetstrokecolor{currentstroke}%
\pgfsetstrokeopacity{0.576917}%
\pgfsetdash{}{0pt}%
\pgfpathmoveto{\pgfqpoint{3.737018in}{-1.383118in}}%
\pgfpathcurveto{\pgfqpoint{3.746227in}{-1.383118in}}{\pgfqpoint{3.755059in}{-1.379460in}}{\pgfqpoint{3.761571in}{-1.372948in}}%
\pgfpathcurveto{\pgfqpoint{3.768082in}{-1.366437in}}{\pgfqpoint{3.771741in}{-1.357604in}}{\pgfqpoint{3.771741in}{-1.348396in}}%
\pgfpathcurveto{\pgfqpoint{3.771741in}{-1.339188in}}{\pgfqpoint{3.768082in}{-1.330355in}}{\pgfqpoint{3.761571in}{-1.323844in}}%
\pgfpathcurveto{\pgfqpoint{3.755059in}{-1.317332in}}{\pgfqpoint{3.746227in}{-1.313674in}}{\pgfqpoint{3.737018in}{-1.313674in}}%
\pgfpathcurveto{\pgfqpoint{3.727810in}{-1.313674in}}{\pgfqpoint{3.718978in}{-1.317332in}}{\pgfqpoint{3.712466in}{-1.323844in}}%
\pgfpathcurveto{\pgfqpoint{3.705955in}{-1.330355in}}{\pgfqpoint{3.702296in}{-1.339188in}}{\pgfqpoint{3.702296in}{-1.348396in}}%
\pgfpathcurveto{\pgfqpoint{3.702296in}{-1.357604in}}{\pgfqpoint{3.705955in}{-1.366437in}}{\pgfqpoint{3.712466in}{-1.372948in}}%
\pgfpathcurveto{\pgfqpoint{3.718978in}{-1.379460in}}{\pgfqpoint{3.727810in}{-1.383118in}}{\pgfqpoint{3.737018in}{-1.383118in}}%
\pgfpathlineto{\pgfqpoint{3.737018in}{-1.383118in}}%
\pgfpathclose%
\pgfusepath{stroke,fill}%
\end{pgfscope}%
\begin{pgfscope}%
\pgfpathrectangle{\pgfqpoint{0.050000in}{0.050000in}}{\pgfqpoint{2.419000in}{2.419000in}}%
\pgfusepath{clip}%
\pgfsetbuttcap%
\pgfsetroundjoin%
\definecolor{currentfill}{rgb}{0.866667,0.800000,0.466667}%
\pgfsetfillcolor{currentfill}%
\pgfsetfillopacity{0.576917}%
\pgfsetlinewidth{1.003750pt}%
\definecolor{currentstroke}{rgb}{0.866667,0.800000,0.466667}%
\pgfsetstrokecolor{currentstroke}%
\pgfsetstrokeopacity{0.576917}%
\pgfsetdash{}{0pt}%
\pgfpathmoveto{\pgfqpoint{-2.972776in}{-1.383118in}}%
\pgfpathcurveto{\pgfqpoint{-2.963568in}{-1.383118in}}{\pgfqpoint{-2.954735in}{-1.379460in}}{\pgfqpoint{-2.948224in}{-1.372948in}}%
\pgfpathcurveto{\pgfqpoint{-2.941713in}{-1.366437in}}{\pgfqpoint{-2.938054in}{-1.357604in}}{\pgfqpoint{-2.938054in}{-1.348396in}}%
\pgfpathcurveto{\pgfqpoint{-2.938054in}{-1.339188in}}{\pgfqpoint{-2.941713in}{-1.330355in}}{\pgfqpoint{-2.948224in}{-1.323844in}}%
\pgfpathcurveto{\pgfqpoint{-2.954735in}{-1.317332in}}{\pgfqpoint{-2.963568in}{-1.313674in}}{\pgfqpoint{-2.972776in}{-1.313674in}}%
\pgfpathcurveto{\pgfqpoint{-2.981985in}{-1.313674in}}{\pgfqpoint{-2.990817in}{-1.317332in}}{\pgfqpoint{-2.997329in}{-1.323844in}}%
\pgfpathcurveto{\pgfqpoint{-3.003840in}{-1.330355in}}{\pgfqpoint{-3.007499in}{-1.339188in}}{\pgfqpoint{-3.007499in}{-1.348396in}}%
\pgfpathcurveto{\pgfqpoint{-3.007499in}{-1.357604in}}{\pgfqpoint{-3.003840in}{-1.366437in}}{\pgfqpoint{-2.997329in}{-1.372948in}}%
\pgfpathcurveto{\pgfqpoint{-2.990817in}{-1.379460in}}{\pgfqpoint{-2.981985in}{-1.383118in}}{\pgfqpoint{-2.972776in}{-1.383118in}}%
\pgfpathlineto{\pgfqpoint{-2.972776in}{-1.383118in}}%
\pgfpathclose%
\pgfusepath{stroke,fill}%
\end{pgfscope}%
\begin{pgfscope}%
\pgfpathrectangle{\pgfqpoint{0.050000in}{0.050000in}}{\pgfqpoint{2.419000in}{2.419000in}}%
\pgfusepath{clip}%
\pgfsetbuttcap%
\pgfsetroundjoin%
\definecolor{currentfill}{rgb}{0.866667,0.800000,0.466667}%
\pgfsetfillcolor{currentfill}%
\pgfsetfillopacity{0.576917}%
\pgfsetlinewidth{1.003750pt}%
\definecolor{currentstroke}{rgb}{0.866667,0.800000,0.466667}%
\pgfsetstrokecolor{currentstroke}%
\pgfsetstrokeopacity{0.576917}%
\pgfsetdash{}{0pt}%
\pgfpathmoveto{\pgfqpoint{10.446813in}{-1.383118in}}%
\pgfpathcurveto{\pgfqpoint{10.456022in}{-1.383118in}}{\pgfqpoint{10.464854in}{-1.379460in}}{\pgfqpoint{10.471366in}{-1.372948in}}%
\pgfpathcurveto{\pgfqpoint{10.477877in}{-1.366437in}}{\pgfqpoint{10.481536in}{-1.357604in}}{\pgfqpoint{10.481536in}{-1.348396in}}%
\pgfpathcurveto{\pgfqpoint{10.481536in}{-1.339188in}}{\pgfqpoint{10.477877in}{-1.330355in}}{\pgfqpoint{10.471366in}{-1.323844in}}%
\pgfpathcurveto{\pgfqpoint{10.464854in}{-1.317332in}}{\pgfqpoint{10.456022in}{-1.313674in}}{\pgfqpoint{10.446813in}{-1.313674in}}%
\pgfpathcurveto{\pgfqpoint{10.437605in}{-1.313674in}}{\pgfqpoint{10.428772in}{-1.317332in}}{\pgfqpoint{10.422261in}{-1.323844in}}%
\pgfpathcurveto{\pgfqpoint{10.415750in}{-1.330355in}}{\pgfqpoint{10.412091in}{-1.339188in}}{\pgfqpoint{10.412091in}{-1.348396in}}%
\pgfpathcurveto{\pgfqpoint{10.412091in}{-1.357604in}}{\pgfqpoint{10.415750in}{-1.366437in}}{\pgfqpoint{10.422261in}{-1.372948in}}%
\pgfpathcurveto{\pgfqpoint{10.428772in}{-1.379460in}}{\pgfqpoint{10.437605in}{-1.383118in}}{\pgfqpoint{10.446813in}{-1.383118in}}%
\pgfpathlineto{\pgfqpoint{10.446813in}{-1.383118in}}%
\pgfpathclose%
\pgfusepath{stroke,fill}%
\end{pgfscope}%
\begin{pgfscope}%
\pgfpathrectangle{\pgfqpoint{0.050000in}{0.050000in}}{\pgfqpoint{2.419000in}{2.419000in}}%
\pgfusepath{clip}%
\pgfsetbuttcap%
\pgfsetroundjoin%
\definecolor{currentfill}{rgb}{0.866667,0.800000,0.466667}%
\pgfsetfillcolor{currentfill}%
\pgfsetfillopacity{0.589306}%
\pgfsetlinewidth{1.003750pt}%
\definecolor{currentstroke}{rgb}{0.866667,0.800000,0.466667}%
\pgfsetstrokecolor{currentstroke}%
\pgfsetstrokeopacity{0.589306}%
\pgfsetdash{}{0pt}%
\pgfpathmoveto{\pgfqpoint{-5.870932in}{-1.634265in}}%
\pgfpathcurveto{\pgfqpoint{-5.861724in}{-1.634265in}}{\pgfqpoint{-5.852891in}{-1.630606in}}{\pgfqpoint{-5.846380in}{-1.624095in}}%
\pgfpathcurveto{\pgfqpoint{-5.839868in}{-1.617584in}}{\pgfqpoint{-5.836210in}{-1.608751in}}{\pgfqpoint{-5.836210in}{-1.599543in}}%
\pgfpathcurveto{\pgfqpoint{-5.836210in}{-1.590334in}}{\pgfqpoint{-5.839868in}{-1.581502in}}{\pgfqpoint{-5.846380in}{-1.574990in}}%
\pgfpathcurveto{\pgfqpoint{-5.852891in}{-1.568479in}}{\pgfqpoint{-5.861724in}{-1.564821in}}{\pgfqpoint{-5.870932in}{-1.564821in}}%
\pgfpathcurveto{\pgfqpoint{-5.880140in}{-1.564821in}}{\pgfqpoint{-5.888973in}{-1.568479in}}{\pgfqpoint{-5.895484in}{-1.574990in}}%
\pgfpathcurveto{\pgfqpoint{-5.901996in}{-1.581502in}}{\pgfqpoint{-5.905654in}{-1.590334in}}{\pgfqpoint{-5.905654in}{-1.599543in}}%
\pgfpathcurveto{\pgfqpoint{-5.905654in}{-1.608751in}}{\pgfqpoint{-5.901996in}{-1.617584in}}{\pgfqpoint{-5.895484in}{-1.624095in}}%
\pgfpathcurveto{\pgfqpoint{-5.888973in}{-1.630606in}}{\pgfqpoint{-5.880140in}{-1.634265in}}{\pgfqpoint{-5.870932in}{-1.634265in}}%
\pgfpathlineto{\pgfqpoint{-5.870932in}{-1.634265in}}%
\pgfpathclose%
\pgfusepath{stroke,fill}%
\end{pgfscope}%
\begin{pgfscope}%
\pgfpathrectangle{\pgfqpoint{0.050000in}{0.050000in}}{\pgfqpoint{2.419000in}{2.419000in}}%
\pgfusepath{clip}%
\pgfsetbuttcap%
\pgfsetroundjoin%
\definecolor{currentfill}{rgb}{0.866667,0.800000,0.466667}%
\pgfsetfillcolor{currentfill}%
\pgfsetfillopacity{0.589306}%
\pgfsetlinewidth{1.003750pt}%
\definecolor{currentstroke}{rgb}{0.866667,0.800000,0.466667}%
\pgfsetstrokecolor{currentstroke}%
\pgfsetstrokeopacity{0.589306}%
\pgfsetdash{}{0pt}%
\pgfpathmoveto{\pgfqpoint{1.004238in}{-1.634265in}}%
\pgfpathcurveto{\pgfqpoint{1.013446in}{-1.634265in}}{\pgfqpoint{1.022279in}{-1.630606in}}{\pgfqpoint{1.028790in}{-1.624095in}}%
\pgfpathcurveto{\pgfqpoint{1.035302in}{-1.617584in}}{\pgfqpoint{1.038960in}{-1.608751in}}{\pgfqpoint{1.038960in}{-1.599543in}}%
\pgfpathcurveto{\pgfqpoint{1.038960in}{-1.590334in}}{\pgfqpoint{1.035302in}{-1.581502in}}{\pgfqpoint{1.028790in}{-1.574990in}}%
\pgfpathcurveto{\pgfqpoint{1.022279in}{-1.568479in}}{\pgfqpoint{1.013446in}{-1.564821in}}{\pgfqpoint{1.004238in}{-1.564821in}}%
\pgfpathcurveto{\pgfqpoint{0.995029in}{-1.564821in}}{\pgfqpoint{0.986197in}{-1.568479in}}{\pgfqpoint{0.979686in}{-1.574990in}}%
\pgfpathcurveto{\pgfqpoint{0.973174in}{-1.581502in}}{\pgfqpoint{0.969516in}{-1.590334in}}{\pgfqpoint{0.969516in}{-1.599543in}}%
\pgfpathcurveto{\pgfqpoint{0.969516in}{-1.608751in}}{\pgfqpoint{0.973174in}{-1.617584in}}{\pgfqpoint{0.979686in}{-1.624095in}}%
\pgfpathcurveto{\pgfqpoint{0.986197in}{-1.630606in}}{\pgfqpoint{0.995029in}{-1.634265in}}{\pgfqpoint{1.004238in}{-1.634265in}}%
\pgfpathlineto{\pgfqpoint{1.004238in}{-1.634265in}}%
\pgfpathclose%
\pgfusepath{stroke,fill}%
\end{pgfscope}%
\begin{pgfscope}%
\pgfpathrectangle{\pgfqpoint{0.050000in}{0.050000in}}{\pgfqpoint{2.419000in}{2.419000in}}%
\pgfusepath{clip}%
\pgfsetbuttcap%
\pgfsetroundjoin%
\definecolor{currentfill}{rgb}{0.866667,0.800000,0.466667}%
\pgfsetfillcolor{currentfill}%
\pgfsetfillopacity{0.589306}%
\pgfsetlinewidth{1.003750pt}%
\definecolor{currentstroke}{rgb}{0.866667,0.800000,0.466667}%
\pgfsetstrokecolor{currentstroke}%
\pgfsetstrokeopacity{0.589306}%
\pgfsetdash{}{0pt}%
\pgfpathmoveto{\pgfqpoint{7.879408in}{-1.634265in}}%
\pgfpathcurveto{\pgfqpoint{7.888616in}{-1.634265in}}{\pgfqpoint{7.897449in}{-1.630606in}}{\pgfqpoint{7.903960in}{-1.624095in}}%
\pgfpathcurveto{\pgfqpoint{7.910471in}{-1.617584in}}{\pgfqpoint{7.914130in}{-1.608751in}}{\pgfqpoint{7.914130in}{-1.599543in}}%
\pgfpathcurveto{\pgfqpoint{7.914130in}{-1.590334in}}{\pgfqpoint{7.910471in}{-1.581502in}}{\pgfqpoint{7.903960in}{-1.574990in}}%
\pgfpathcurveto{\pgfqpoint{7.897449in}{-1.568479in}}{\pgfqpoint{7.888616in}{-1.564821in}}{\pgfqpoint{7.879408in}{-1.564821in}}%
\pgfpathcurveto{\pgfqpoint{7.870199in}{-1.564821in}}{\pgfqpoint{7.861367in}{-1.568479in}}{\pgfqpoint{7.854855in}{-1.574990in}}%
\pgfpathcurveto{\pgfqpoint{7.848344in}{-1.581502in}}{\pgfqpoint{7.844686in}{-1.590334in}}{\pgfqpoint{7.844686in}{-1.599543in}}%
\pgfpathcurveto{\pgfqpoint{7.844686in}{-1.608751in}}{\pgfqpoint{7.848344in}{-1.617584in}}{\pgfqpoint{7.854855in}{-1.624095in}}%
\pgfpathcurveto{\pgfqpoint{7.861367in}{-1.630606in}}{\pgfqpoint{7.870199in}{-1.634265in}}{\pgfqpoint{7.879408in}{-1.634265in}}%
\pgfpathlineto{\pgfqpoint{7.879408in}{-1.634265in}}%
\pgfpathclose%
\pgfusepath{stroke,fill}%
\end{pgfscope}%
\begin{pgfscope}%
\pgfpathrectangle{\pgfqpoint{0.050000in}{0.050000in}}{\pgfqpoint{2.419000in}{2.419000in}}%
\pgfusepath{clip}%
\pgfsetbuttcap%
\pgfsetroundjoin%
\definecolor{currentfill}{rgb}{0.866667,0.800000,0.466667}%
\pgfsetfillcolor{currentfill}%
\pgfsetfillopacity{0.602321}%
\pgfsetlinewidth{1.003750pt}%
\definecolor{currentstroke}{rgb}{0.866667,0.800000,0.466667}%
\pgfsetstrokecolor{currentstroke}%
\pgfsetstrokeopacity{0.602321}%
\pgfsetdash{}{0pt}%
\pgfpathmoveto{\pgfqpoint{-1.866655in}{-1.898104in}}%
\pgfpathcurveto{\pgfqpoint{-1.857447in}{-1.898104in}}{\pgfqpoint{-1.848614in}{-1.894446in}}{\pgfqpoint{-1.842103in}{-1.887935in}}%
\pgfpathcurveto{\pgfqpoint{-1.835592in}{-1.881423in}}{\pgfqpoint{-1.831933in}{-1.872591in}}{\pgfqpoint{-1.831933in}{-1.863382in}}%
\pgfpathcurveto{\pgfqpoint{-1.831933in}{-1.854174in}}{\pgfqpoint{-1.835592in}{-1.845341in}}{\pgfqpoint{-1.842103in}{-1.838830in}}%
\pgfpathcurveto{\pgfqpoint{-1.848614in}{-1.832319in}}{\pgfqpoint{-1.857447in}{-1.828660in}}{\pgfqpoint{-1.866655in}{-1.828660in}}%
\pgfpathcurveto{\pgfqpoint{-1.875864in}{-1.828660in}}{\pgfqpoint{-1.884696in}{-1.832319in}}{\pgfqpoint{-1.891208in}{-1.838830in}}%
\pgfpathcurveto{\pgfqpoint{-1.897719in}{-1.845341in}}{\pgfqpoint{-1.901377in}{-1.854174in}}{\pgfqpoint{-1.901377in}{-1.863382in}}%
\pgfpathcurveto{\pgfqpoint{-1.901377in}{-1.872591in}}{\pgfqpoint{-1.897719in}{-1.881423in}}{\pgfqpoint{-1.891208in}{-1.887935in}}%
\pgfpathcurveto{\pgfqpoint{-1.884696in}{-1.894446in}}{\pgfqpoint{-1.875864in}{-1.898104in}}{\pgfqpoint{-1.866655in}{-1.898104in}}%
\pgfpathlineto{\pgfqpoint{-1.866655in}{-1.898104in}}%
\pgfpathclose%
\pgfusepath{stroke,fill}%
\end{pgfscope}%
\begin{pgfscope}%
\pgfpathrectangle{\pgfqpoint{0.050000in}{0.050000in}}{\pgfqpoint{2.419000in}{2.419000in}}%
\pgfusepath{clip}%
\pgfsetbuttcap%
\pgfsetroundjoin%
\definecolor{currentfill}{rgb}{0.866667,0.800000,0.466667}%
\pgfsetfillcolor{currentfill}%
\pgfsetfillopacity{0.602321}%
\pgfsetlinewidth{1.003750pt}%
\definecolor{currentstroke}{rgb}{0.866667,0.800000,0.466667}%
\pgfsetstrokecolor{currentstroke}%
\pgfsetstrokeopacity{0.602321}%
\pgfsetdash{}{0pt}%
\pgfpathmoveto{\pgfqpoint{5.182248in}{-1.898104in}}%
\pgfpathcurveto{\pgfqpoint{5.191456in}{-1.898104in}}{\pgfqpoint{5.200289in}{-1.894446in}}{\pgfqpoint{5.206800in}{-1.887935in}}%
\pgfpathcurveto{\pgfqpoint{5.213311in}{-1.881423in}}{\pgfqpoint{5.216970in}{-1.872591in}}{\pgfqpoint{5.216970in}{-1.863382in}}%
\pgfpathcurveto{\pgfqpoint{5.216970in}{-1.854174in}}{\pgfqpoint{5.213311in}{-1.845341in}}{\pgfqpoint{5.206800in}{-1.838830in}}%
\pgfpathcurveto{\pgfqpoint{5.200289in}{-1.832319in}}{\pgfqpoint{5.191456in}{-1.828660in}}{\pgfqpoint{5.182248in}{-1.828660in}}%
\pgfpathcurveto{\pgfqpoint{5.173039in}{-1.828660in}}{\pgfqpoint{5.164207in}{-1.832319in}}{\pgfqpoint{5.157695in}{-1.838830in}}%
\pgfpathcurveto{\pgfqpoint{5.151184in}{-1.845341in}}{\pgfqpoint{5.147525in}{-1.854174in}}{\pgfqpoint{5.147525in}{-1.863382in}}%
\pgfpathcurveto{\pgfqpoint{5.147525in}{-1.872591in}}{\pgfqpoint{5.151184in}{-1.881423in}}{\pgfqpoint{5.157695in}{-1.887935in}}%
\pgfpathcurveto{\pgfqpoint{5.164207in}{-1.894446in}}{\pgfqpoint{5.173039in}{-1.898104in}}{\pgfqpoint{5.182248in}{-1.898104in}}%
\pgfpathlineto{\pgfqpoint{5.182248in}{-1.898104in}}%
\pgfpathclose%
\pgfusepath{stroke,fill}%
\end{pgfscope}%
\begin{pgfscope}%
\pgfpathrectangle{\pgfqpoint{0.050000in}{0.050000in}}{\pgfqpoint{2.419000in}{2.419000in}}%
\pgfusepath{clip}%
\pgfsetbuttcap%
\pgfsetroundjoin%
\definecolor{currentfill}{rgb}{0.866667,0.800000,0.466667}%
\pgfsetfillcolor{currentfill}%
\pgfsetfillopacity{0.616010}%
\pgfsetlinewidth{1.003750pt}%
\definecolor{currentstroke}{rgb}{0.866667,0.800000,0.466667}%
\pgfsetstrokecolor{currentstroke}%
\pgfsetstrokeopacity{0.616010}%
\pgfsetdash{}{0pt}%
\pgfpathmoveto{\pgfqpoint{2.345241in}{-2.175624in}}%
\pgfpathcurveto{\pgfqpoint{2.354449in}{-2.175624in}}{\pgfqpoint{2.363282in}{-2.171965in}}{\pgfqpoint{2.369793in}{-2.165454in}}%
\pgfpathcurveto{\pgfqpoint{2.376305in}{-2.158943in}}{\pgfqpoint{2.379963in}{-2.150110in}}{\pgfqpoint{2.379963in}{-2.140902in}}%
\pgfpathcurveto{\pgfqpoint{2.379963in}{-2.131693in}}{\pgfqpoint{2.376305in}{-2.122861in}}{\pgfqpoint{2.369793in}{-2.116349in}}%
\pgfpathcurveto{\pgfqpoint{2.363282in}{-2.109838in}}{\pgfqpoint{2.354449in}{-2.106179in}}{\pgfqpoint{2.345241in}{-2.106179in}}%
\pgfpathcurveto{\pgfqpoint{2.336033in}{-2.106179in}}{\pgfqpoint{2.327200in}{-2.109838in}}{\pgfqpoint{2.320689in}{-2.116349in}}%
\pgfpathcurveto{\pgfqpoint{2.314177in}{-2.122861in}}{\pgfqpoint{2.310519in}{-2.131693in}}{\pgfqpoint{2.310519in}{-2.140902in}}%
\pgfpathcurveto{\pgfqpoint{2.310519in}{-2.150110in}}{\pgfqpoint{2.314177in}{-2.158943in}}{\pgfqpoint{2.320689in}{-2.165454in}}%
\pgfpathcurveto{\pgfqpoint{2.327200in}{-2.171965in}}{\pgfqpoint{2.336033in}{-2.175624in}}{\pgfqpoint{2.345241in}{-2.175624in}}%
\pgfpathlineto{\pgfqpoint{2.345241in}{-2.175624in}}%
\pgfpathclose%
\pgfusepath{stroke,fill}%
\end{pgfscope}%
\begin{pgfscope}%
\pgfpathrectangle{\pgfqpoint{0.050000in}{0.050000in}}{\pgfqpoint{2.419000in}{2.419000in}}%
\pgfusepath{clip}%
\pgfsetbuttcap%
\pgfsetroundjoin%
\definecolor{currentfill}{rgb}{0.866667,0.800000,0.466667}%
\pgfsetfillcolor{currentfill}%
\pgfsetfillopacity{0.616010}%
\pgfsetlinewidth{1.003750pt}%
\definecolor{currentstroke}{rgb}{0.866667,0.800000,0.466667}%
\pgfsetstrokecolor{currentstroke}%
\pgfsetstrokeopacity{0.616010}%
\pgfsetdash{}{0pt}%
\pgfpathmoveto{\pgfqpoint{-4.886403in}{-2.175624in}}%
\pgfpathcurveto{\pgfqpoint{-4.877194in}{-2.175624in}}{\pgfqpoint{-4.868362in}{-2.171965in}}{\pgfqpoint{-4.861850in}{-2.165454in}}%
\pgfpathcurveto{\pgfqpoint{-4.855339in}{-2.158943in}}{\pgfqpoint{-4.851680in}{-2.150110in}}{\pgfqpoint{-4.851680in}{-2.140902in}}%
\pgfpathcurveto{\pgfqpoint{-4.851680in}{-2.131693in}}{\pgfqpoint{-4.855339in}{-2.122861in}}{\pgfqpoint{-4.861850in}{-2.116349in}}%
\pgfpathcurveto{\pgfqpoint{-4.868362in}{-2.109838in}}{\pgfqpoint{-4.877194in}{-2.106179in}}{\pgfqpoint{-4.886403in}{-2.106179in}}%
\pgfpathcurveto{\pgfqpoint{-4.895611in}{-2.106179in}}{\pgfqpoint{-4.904444in}{-2.109838in}}{\pgfqpoint{-4.910955in}{-2.116349in}}%
\pgfpathcurveto{\pgfqpoint{-4.917466in}{-2.122861in}}{\pgfqpoint{-4.921125in}{-2.131693in}}{\pgfqpoint{-4.921125in}{-2.140902in}}%
\pgfpathcurveto{\pgfqpoint{-4.921125in}{-2.150110in}}{\pgfqpoint{-4.917466in}{-2.158943in}}{\pgfqpoint{-4.910955in}{-2.165454in}}%
\pgfpathcurveto{\pgfqpoint{-4.904444in}{-2.171965in}}{\pgfqpoint{-4.895611in}{-2.175624in}}{\pgfqpoint{-4.886403in}{-2.175624in}}%
\pgfpathlineto{\pgfqpoint{-4.886403in}{-2.175624in}}%
\pgfpathclose%
\pgfusepath{stroke,fill}%
\end{pgfscope}%
\begin{pgfscope}%
\pgfpathrectangle{\pgfqpoint{0.050000in}{0.050000in}}{\pgfqpoint{2.419000in}{2.419000in}}%
\pgfusepath{clip}%
\pgfsetbuttcap%
\pgfsetroundjoin%
\definecolor{currentfill}{rgb}{0.866667,0.800000,0.466667}%
\pgfsetfillcolor{currentfill}%
\pgfsetfillopacity{0.616010}%
\pgfsetlinewidth{1.003750pt}%
\definecolor{currentstroke}{rgb}{0.866667,0.800000,0.466667}%
\pgfsetstrokecolor{currentstroke}%
\pgfsetstrokeopacity{0.616010}%
\pgfsetdash{}{0pt}%
\pgfpathmoveto{\pgfqpoint{9.576885in}{-2.175624in}}%
\pgfpathcurveto{\pgfqpoint{9.586093in}{-2.175624in}}{\pgfqpoint{9.594926in}{-2.171965in}}{\pgfqpoint{9.601437in}{-2.165454in}}%
\pgfpathcurveto{\pgfqpoint{9.607948in}{-2.158943in}}{\pgfqpoint{9.611607in}{-2.150110in}}{\pgfqpoint{9.611607in}{-2.140902in}}%
\pgfpathcurveto{\pgfqpoint{9.611607in}{-2.131693in}}{\pgfqpoint{9.607948in}{-2.122861in}}{\pgfqpoint{9.601437in}{-2.116349in}}%
\pgfpathcurveto{\pgfqpoint{9.594926in}{-2.109838in}}{\pgfqpoint{9.586093in}{-2.106179in}}{\pgfqpoint{9.576885in}{-2.106179in}}%
\pgfpathcurveto{\pgfqpoint{9.567676in}{-2.106179in}}{\pgfqpoint{9.558844in}{-2.109838in}}{\pgfqpoint{9.552332in}{-2.116349in}}%
\pgfpathcurveto{\pgfqpoint{9.545821in}{-2.122861in}}{\pgfqpoint{9.542162in}{-2.131693in}}{\pgfqpoint{9.542162in}{-2.140902in}}%
\pgfpathcurveto{\pgfqpoint{9.542162in}{-2.150110in}}{\pgfqpoint{9.545821in}{-2.158943in}}{\pgfqpoint{9.552332in}{-2.165454in}}%
\pgfpathcurveto{\pgfqpoint{9.558844in}{-2.171965in}}{\pgfqpoint{9.567676in}{-2.175624in}}{\pgfqpoint{9.576885in}{-2.175624in}}%
\pgfpathlineto{\pgfqpoint{9.576885in}{-2.175624in}}%
\pgfpathclose%
\pgfusepath{stroke,fill}%
\end{pgfscope}%
\begin{pgfscope}%
\pgfpathrectangle{\pgfqpoint{0.050000in}{0.050000in}}{\pgfqpoint{2.419000in}{2.419000in}}%
\pgfusepath{clip}%
\pgfsetbuttcap%
\pgfsetroundjoin%
\definecolor{currentfill}{rgb}{0.866667,0.800000,0.466667}%
\pgfsetfillcolor{currentfill}%
\pgfsetfillopacity{0.630429}%
\pgfsetlinewidth{1.003750pt}%
\definecolor{currentstroke}{rgb}{0.866667,0.800000,0.466667}%
\pgfsetstrokecolor{currentstroke}%
\pgfsetstrokeopacity{0.630429}%
\pgfsetdash{}{0pt}%
\pgfpathmoveto{\pgfqpoint{6.781334in}{-2.467916in}}%
\pgfpathcurveto{\pgfqpoint{6.790542in}{-2.467916in}}{\pgfqpoint{6.799375in}{-2.464257in}}{\pgfqpoint{6.805886in}{-2.457746in}}%
\pgfpathcurveto{\pgfqpoint{6.812398in}{-2.451234in}}{\pgfqpoint{6.816056in}{-2.442402in}}{\pgfqpoint{6.816056in}{-2.433193in}}%
\pgfpathcurveto{\pgfqpoint{6.816056in}{-2.423985in}}{\pgfqpoint{6.812398in}{-2.415152in}}{\pgfqpoint{6.805886in}{-2.408641in}}%
\pgfpathcurveto{\pgfqpoint{6.799375in}{-2.402130in}}{\pgfqpoint{6.790542in}{-2.398471in}}{\pgfqpoint{6.781334in}{-2.398471in}}%
\pgfpathcurveto{\pgfqpoint{6.772126in}{-2.398471in}}{\pgfqpoint{6.763293in}{-2.402130in}}{\pgfqpoint{6.756782in}{-2.408641in}}%
\pgfpathcurveto{\pgfqpoint{6.750270in}{-2.415152in}}{\pgfqpoint{6.746612in}{-2.423985in}}{\pgfqpoint{6.746612in}{-2.433193in}}%
\pgfpathcurveto{\pgfqpoint{6.746612in}{-2.442402in}}{\pgfqpoint{6.750270in}{-2.451234in}}{\pgfqpoint{6.756782in}{-2.457746in}}%
\pgfpathcurveto{\pgfqpoint{6.763293in}{-2.464257in}}{\pgfqpoint{6.772126in}{-2.467916in}}{\pgfqpoint{6.781334in}{-2.467916in}}%
\pgfpathlineto{\pgfqpoint{6.781334in}{-2.467916in}}%
\pgfpathclose%
\pgfusepath{stroke,fill}%
\end{pgfscope}%
\begin{pgfscope}%
\pgfpathrectangle{\pgfqpoint{0.050000in}{0.050000in}}{\pgfqpoint{2.419000in}{2.419000in}}%
\pgfusepath{clip}%
\pgfsetbuttcap%
\pgfsetroundjoin%
\definecolor{currentfill}{rgb}{0.866667,0.800000,0.466667}%
\pgfsetfillcolor{currentfill}%
\pgfsetfillopacity{0.630429}%
\pgfsetlinewidth{1.003750pt}%
\definecolor{currentstroke}{rgb}{0.866667,0.800000,0.466667}%
\pgfsetstrokecolor{currentstroke}%
\pgfsetstrokeopacity{0.630429}%
\pgfsetdash{}{0pt}%
\pgfpathmoveto{\pgfqpoint{-0.642778in}{-2.467916in}}%
\pgfpathcurveto{\pgfqpoint{-0.633569in}{-2.467916in}}{\pgfqpoint{-0.624737in}{-2.464257in}}{\pgfqpoint{-0.618225in}{-2.457746in}}%
\pgfpathcurveto{\pgfqpoint{-0.611714in}{-2.451234in}}{\pgfqpoint{-0.608055in}{-2.442402in}}{\pgfqpoint{-0.608055in}{-2.433193in}}%
\pgfpathcurveto{\pgfqpoint{-0.608055in}{-2.423985in}}{\pgfqpoint{-0.611714in}{-2.415152in}}{\pgfqpoint{-0.618225in}{-2.408641in}}%
\pgfpathcurveto{\pgfqpoint{-0.624737in}{-2.402130in}}{\pgfqpoint{-0.633569in}{-2.398471in}}{\pgfqpoint{-0.642778in}{-2.398471in}}%
\pgfpathcurveto{\pgfqpoint{-0.651986in}{-2.398471in}}{\pgfqpoint{-0.660819in}{-2.402130in}}{\pgfqpoint{-0.667330in}{-2.408641in}}%
\pgfpathcurveto{\pgfqpoint{-0.673841in}{-2.415152in}}{\pgfqpoint{-0.677500in}{-2.423985in}}{\pgfqpoint{-0.677500in}{-2.433193in}}%
\pgfpathcurveto{\pgfqpoint{-0.677500in}{-2.442402in}}{\pgfqpoint{-0.673841in}{-2.451234in}}{\pgfqpoint{-0.667330in}{-2.457746in}}%
\pgfpathcurveto{\pgfqpoint{-0.660819in}{-2.464257in}}{\pgfqpoint{-0.651986in}{-2.467916in}}{\pgfqpoint{-0.642778in}{-2.467916in}}%
\pgfpathlineto{\pgfqpoint{-0.642778in}{-2.467916in}}%
\pgfpathclose%
\pgfusepath{stroke,fill}%
\end{pgfscope}%
\begin{pgfscope}%
\pgfpathrectangle{\pgfqpoint{0.050000in}{0.050000in}}{\pgfqpoint{2.419000in}{2.419000in}}%
\pgfusepath{clip}%
\pgfsetbuttcap%
\pgfsetroundjoin%
\definecolor{currentfill}{rgb}{0.866667,0.800000,0.466667}%
\pgfsetfillcolor{currentfill}%
\pgfsetfillopacity{0.645635}%
\pgfsetlinewidth{1.003750pt}%
\definecolor{currentstroke}{rgb}{0.866667,0.800000,0.466667}%
\pgfsetstrokecolor{currentstroke}%
\pgfsetstrokeopacity{0.645635}%
\pgfsetdash{}{0pt}%
\pgfpathmoveto{\pgfqpoint{-3.794196in}{-2.776191in}}%
\pgfpathcurveto{\pgfqpoint{-3.784987in}{-2.776191in}}{\pgfqpoint{-3.776155in}{-2.772533in}}{\pgfqpoint{-3.769643in}{-2.766021in}}%
\pgfpathcurveto{\pgfqpoint{-3.763132in}{-2.759510in}}{\pgfqpoint{-3.759474in}{-2.750677in}}{\pgfqpoint{-3.759474in}{-2.741469in}}%
\pgfpathcurveto{\pgfqpoint{-3.759474in}{-2.732260in}}{\pgfqpoint{-3.763132in}{-2.723428in}}{\pgfqpoint{-3.769643in}{-2.716917in}}%
\pgfpathcurveto{\pgfqpoint{-3.776155in}{-2.710405in}}{\pgfqpoint{-3.784987in}{-2.706747in}}{\pgfqpoint{-3.794196in}{-2.706747in}}%
\pgfpathcurveto{\pgfqpoint{-3.803404in}{-2.706747in}}{\pgfqpoint{-3.812237in}{-2.710405in}}{\pgfqpoint{-3.818748in}{-2.716917in}}%
\pgfpathcurveto{\pgfqpoint{-3.825259in}{-2.723428in}}{\pgfqpoint{-3.828918in}{-2.732260in}}{\pgfqpoint{-3.828918in}{-2.741469in}}%
\pgfpathcurveto{\pgfqpoint{-3.828918in}{-2.750677in}}{\pgfqpoint{-3.825259in}{-2.759510in}}{\pgfqpoint{-3.818748in}{-2.766021in}}%
\pgfpathcurveto{\pgfqpoint{-3.812237in}{-2.772533in}}{\pgfqpoint{-3.803404in}{-2.776191in}}{\pgfqpoint{-3.794196in}{-2.776191in}}%
\pgfpathlineto{\pgfqpoint{-3.794196in}{-2.776191in}}%
\pgfpathclose%
\pgfusepath{stroke,fill}%
\end{pgfscope}%
\begin{pgfscope}%
\pgfpathrectangle{\pgfqpoint{0.050000in}{0.050000in}}{\pgfqpoint{2.419000in}{2.419000in}}%
\pgfusepath{clip}%
\pgfsetbuttcap%
\pgfsetroundjoin%
\definecolor{currentfill}{rgb}{0.866667,0.800000,0.466667}%
\pgfsetfillcolor{currentfill}%
\pgfsetfillopacity{0.645635}%
\pgfsetlinewidth{1.003750pt}%
\definecolor{currentstroke}{rgb}{0.866667,0.800000,0.466667}%
\pgfsetstrokecolor{currentstroke}%
\pgfsetstrokeopacity{0.645635}%
\pgfsetdash{}{0pt}%
\pgfpathmoveto{\pgfqpoint{3.832909in}{-2.776191in}}%
\pgfpathcurveto{\pgfqpoint{3.842118in}{-2.776191in}}{\pgfqpoint{3.850950in}{-2.772533in}}{\pgfqpoint{3.857461in}{-2.766021in}}%
\pgfpathcurveto{\pgfqpoint{3.863973in}{-2.759510in}}{\pgfqpoint{3.867631in}{-2.750677in}}{\pgfqpoint{3.867631in}{-2.741469in}}%
\pgfpathcurveto{\pgfqpoint{3.867631in}{-2.732260in}}{\pgfqpoint{3.863973in}{-2.723428in}}{\pgfqpoint{3.857461in}{-2.716917in}}%
\pgfpathcurveto{\pgfqpoint{3.850950in}{-2.710405in}}{\pgfqpoint{3.842118in}{-2.706747in}}{\pgfqpoint{3.832909in}{-2.706747in}}%
\pgfpathcurveto{\pgfqpoint{3.823701in}{-2.706747in}}{\pgfqpoint{3.814868in}{-2.710405in}}{\pgfqpoint{3.808357in}{-2.716917in}}%
\pgfpathcurveto{\pgfqpoint{3.801845in}{-2.723428in}}{\pgfqpoint{3.798187in}{-2.732260in}}{\pgfqpoint{3.798187in}{-2.741469in}}%
\pgfpathcurveto{\pgfqpoint{3.798187in}{-2.750677in}}{\pgfqpoint{3.801845in}{-2.759510in}}{\pgfqpoint{3.808357in}{-2.766021in}}%
\pgfpathcurveto{\pgfqpoint{3.814868in}{-2.772533in}}{\pgfqpoint{3.823701in}{-2.776191in}}{\pgfqpoint{3.832909in}{-2.776191in}}%
\pgfpathlineto{\pgfqpoint{3.832909in}{-2.776191in}}%
\pgfpathclose%
\pgfusepath{stroke,fill}%
\end{pgfscope}%
\begin{pgfscope}%
\pgfpathrectangle{\pgfqpoint{0.050000in}{0.050000in}}{\pgfqpoint{2.419000in}{2.419000in}}%
\pgfusepath{clip}%
\pgfsetbuttcap%
\pgfsetroundjoin%
\definecolor{currentfill}{rgb}{0.866667,0.800000,0.466667}%
\pgfsetfillcolor{currentfill}%
\pgfsetfillopacity{0.661697}%
\pgfsetlinewidth{1.003750pt}%
\definecolor{currentstroke}{rgb}{0.866667,0.800000,0.466667}%
\pgfsetstrokecolor{currentstroke}%
\pgfsetstrokeopacity{0.661697}%
\pgfsetdash{}{0pt}%
\pgfpathmoveto{\pgfqpoint{0.718718in}{-3.101798in}}%
\pgfpathcurveto{\pgfqpoint{0.727926in}{-3.101798in}}{\pgfqpoint{0.736759in}{-3.098140in}}{\pgfqpoint{0.743270in}{-3.091629in}}%
\pgfpathcurveto{\pgfqpoint{0.749781in}{-3.085117in}}{\pgfqpoint{0.753440in}{-3.076285in}}{\pgfqpoint{0.753440in}{-3.067076in}}%
\pgfpathcurveto{\pgfqpoint{0.753440in}{-3.057868in}}{\pgfqpoint{0.749781in}{-3.049035in}}{\pgfqpoint{0.743270in}{-3.042524in}}%
\pgfpathcurveto{\pgfqpoint{0.736759in}{-3.036013in}}{\pgfqpoint{0.727926in}{-3.032354in}}{\pgfqpoint{0.718718in}{-3.032354in}}%
\pgfpathcurveto{\pgfqpoint{0.709509in}{-3.032354in}}{\pgfqpoint{0.700677in}{-3.036013in}}{\pgfqpoint{0.694165in}{-3.042524in}}%
\pgfpathcurveto{\pgfqpoint{0.687654in}{-3.049035in}}{\pgfqpoint{0.683995in}{-3.057868in}}{\pgfqpoint{0.683995in}{-3.067076in}}%
\pgfpathcurveto{\pgfqpoint{0.683995in}{-3.076285in}}{\pgfqpoint{0.687654in}{-3.085117in}}{\pgfqpoint{0.694165in}{-3.091629in}}%
\pgfpathcurveto{\pgfqpoint{0.700677in}{-3.098140in}}{\pgfqpoint{0.709509in}{-3.101798in}}{\pgfqpoint{0.718718in}{-3.101798in}}%
\pgfpathlineto{\pgfqpoint{0.718718in}{-3.101798in}}%
\pgfpathclose%
\pgfusepath{stroke,fill}%
\end{pgfscope}%
\begin{pgfscope}%
\pgfpathrectangle{\pgfqpoint{0.050000in}{0.050000in}}{\pgfqpoint{2.419000in}{2.419000in}}%
\pgfusepath{clip}%
\pgfsetbuttcap%
\pgfsetroundjoin%
\definecolor{currentfill}{rgb}{0.866667,0.800000,0.466667}%
\pgfsetfillcolor{currentfill}%
\pgfsetfillopacity{0.661697}%
\pgfsetlinewidth{1.003750pt}%
\definecolor{currentstroke}{rgb}{0.866667,0.800000,0.466667}%
\pgfsetstrokecolor{currentstroke}%
\pgfsetstrokeopacity{0.661697}%
\pgfsetdash{}{0pt}%
\pgfpathmoveto{\pgfqpoint{8.560228in}{-3.101798in}}%
\pgfpathcurveto{\pgfqpoint{8.569437in}{-3.101798in}}{\pgfqpoint{8.578269in}{-3.098140in}}{\pgfqpoint{8.584781in}{-3.091629in}}%
\pgfpathcurveto{\pgfqpoint{8.591292in}{-3.085117in}}{\pgfqpoint{8.594951in}{-3.076285in}}{\pgfqpoint{8.594951in}{-3.067076in}}%
\pgfpathcurveto{\pgfqpoint{8.594951in}{-3.057868in}}{\pgfqpoint{8.591292in}{-3.049035in}}{\pgfqpoint{8.584781in}{-3.042524in}}%
\pgfpathcurveto{\pgfqpoint{8.578269in}{-3.036013in}}{\pgfqpoint{8.569437in}{-3.032354in}}{\pgfqpoint{8.560228in}{-3.032354in}}%
\pgfpathcurveto{\pgfqpoint{8.551020in}{-3.032354in}}{\pgfqpoint{8.542187in}{-3.036013in}}{\pgfqpoint{8.535676in}{-3.042524in}}%
\pgfpathcurveto{\pgfqpoint{8.529165in}{-3.049035in}}{\pgfqpoint{8.525506in}{-3.057868in}}{\pgfqpoint{8.525506in}{-3.067076in}}%
\pgfpathcurveto{\pgfqpoint{8.525506in}{-3.076285in}}{\pgfqpoint{8.529165in}{-3.085117in}}{\pgfqpoint{8.535676in}{-3.091629in}}%
\pgfpathcurveto{\pgfqpoint{8.542187in}{-3.098140in}}{\pgfqpoint{8.551020in}{-3.101798in}}{\pgfqpoint{8.560228in}{-3.101798in}}%
\pgfpathlineto{\pgfqpoint{8.560228in}{-3.101798in}}%
\pgfpathclose%
\pgfusepath{stroke,fill}%
\end{pgfscope}%
\begin{pgfscope}%
\pgfpathrectangle{\pgfqpoint{0.050000in}{0.050000in}}{\pgfqpoint{2.419000in}{2.419000in}}%
\pgfusepath{clip}%
\pgfsetbuttcap%
\pgfsetroundjoin%
\definecolor{currentfill}{rgb}{0.866667,0.800000,0.466667}%
\pgfsetfillcolor{currentfill}%
\pgfsetfillopacity{0.678688}%
\pgfsetlinewidth{1.003750pt}%
\definecolor{currentstroke}{rgb}{0.866667,0.800000,0.466667}%
\pgfsetstrokecolor{currentstroke}%
\pgfsetstrokeopacity{0.678688}%
\pgfsetdash{}{0pt}%
\pgfpathmoveto{\pgfqpoint{5.492695in}{-3.446242in}}%
\pgfpathcurveto{\pgfqpoint{5.501904in}{-3.446242in}}{\pgfqpoint{5.510736in}{-3.442583in}}{\pgfqpoint{5.517248in}{-3.436072in}}%
\pgfpathcurveto{\pgfqpoint{5.523759in}{-3.429560in}}{\pgfqpoint{5.527417in}{-3.420728in}}{\pgfqpoint{5.527417in}{-3.411519in}}%
\pgfpathcurveto{\pgfqpoint{5.527417in}{-3.402311in}}{\pgfqpoint{5.523759in}{-3.393479in}}{\pgfqpoint{5.517248in}{-3.386967in}}%
\pgfpathcurveto{\pgfqpoint{5.510736in}{-3.380456in}}{\pgfqpoint{5.501904in}{-3.376797in}}{\pgfqpoint{5.492695in}{-3.376797in}}%
\pgfpathcurveto{\pgfqpoint{5.483487in}{-3.376797in}}{\pgfqpoint{5.474654in}{-3.380456in}}{\pgfqpoint{5.468143in}{-3.386967in}}%
\pgfpathcurveto{\pgfqpoint{5.461632in}{-3.393479in}}{\pgfqpoint{5.457973in}{-3.402311in}}{\pgfqpoint{5.457973in}{-3.411519in}}%
\pgfpathcurveto{\pgfqpoint{5.457973in}{-3.420728in}}{\pgfqpoint{5.461632in}{-3.429560in}}{\pgfqpoint{5.468143in}{-3.436072in}}%
\pgfpathcurveto{\pgfqpoint{5.474654in}{-3.442583in}}{\pgfqpoint{5.483487in}{-3.446242in}}{\pgfqpoint{5.492695in}{-3.446242in}}%
\pgfpathlineto{\pgfqpoint{5.492695in}{-3.446242in}}%
\pgfpathclose%
\pgfusepath{stroke,fill}%
\end{pgfscope}%
\begin{pgfscope}%
\pgfpathrectangle{\pgfqpoint{0.050000in}{0.050000in}}{\pgfqpoint{2.419000in}{2.419000in}}%
\pgfusepath{clip}%
\pgfsetbuttcap%
\pgfsetroundjoin%
\definecolor{currentfill}{rgb}{0.866667,0.800000,0.466667}%
\pgfsetfillcolor{currentfill}%
\pgfsetfillopacity{0.678688}%
\pgfsetlinewidth{1.003750pt}%
\definecolor{currentstroke}{rgb}{0.866667,0.800000,0.466667}%
\pgfsetstrokecolor{currentstroke}%
\pgfsetstrokeopacity{0.678688}%
\pgfsetdash{}{0pt}%
\pgfpathmoveto{\pgfqpoint{-2.575624in}{-3.446242in}}%
\pgfpathcurveto{\pgfqpoint{-2.566416in}{-3.446242in}}{\pgfqpoint{-2.557583in}{-3.442583in}}{\pgfqpoint{-2.551072in}{-3.436072in}}%
\pgfpathcurveto{\pgfqpoint{-2.544561in}{-3.429560in}}{\pgfqpoint{-2.540902in}{-3.420728in}}{\pgfqpoint{-2.540902in}{-3.411519in}}%
\pgfpathcurveto{\pgfqpoint{-2.540902in}{-3.402311in}}{\pgfqpoint{-2.544561in}{-3.393479in}}{\pgfqpoint{-2.551072in}{-3.386967in}}%
\pgfpathcurveto{\pgfqpoint{-2.557583in}{-3.380456in}}{\pgfqpoint{-2.566416in}{-3.376797in}}{\pgfqpoint{-2.575624in}{-3.376797in}}%
\pgfpathcurveto{\pgfqpoint{-2.584833in}{-3.376797in}}{\pgfqpoint{-2.593665in}{-3.380456in}}{\pgfqpoint{-2.600177in}{-3.386967in}}%
\pgfpathcurveto{\pgfqpoint{-2.606688in}{-3.393479in}}{\pgfqpoint{-2.610347in}{-3.402311in}}{\pgfqpoint{-2.610347in}{-3.411519in}}%
\pgfpathcurveto{\pgfqpoint{-2.610347in}{-3.420728in}}{\pgfqpoint{-2.606688in}{-3.429560in}}{\pgfqpoint{-2.600177in}{-3.436072in}}%
\pgfpathcurveto{\pgfqpoint{-2.593665in}{-3.442583in}}{\pgfqpoint{-2.584833in}{-3.446242in}}{\pgfqpoint{-2.575624in}{-3.446242in}}%
\pgfpathlineto{\pgfqpoint{-2.575624in}{-3.446242in}}%
\pgfpathclose%
\pgfusepath{stroke,fill}%
\end{pgfscope}%
\begin{pgfscope}%
\pgfpathrectangle{\pgfqpoint{0.050000in}{0.050000in}}{\pgfqpoint{2.419000in}{2.419000in}}%
\pgfusepath{clip}%
\pgfsetbuttcap%
\pgfsetroundjoin%
\definecolor{currentfill}{rgb}{0.866667,0.800000,0.466667}%
\pgfsetfillcolor{currentfill}%
\pgfsetfillopacity{0.696691}%
\pgfsetlinewidth{1.003750pt}%
\definecolor{currentstroke}{rgb}{0.866667,0.800000,0.466667}%
\pgfsetstrokecolor{currentstroke}%
\pgfsetstrokeopacity{0.696691}%
\pgfsetdash{}{0pt}%
\pgfpathmoveto{\pgfqpoint{2.242425in}{-3.811204in}}%
\pgfpathcurveto{\pgfqpoint{2.251634in}{-3.811204in}}{\pgfqpoint{2.260466in}{-3.807545in}}{\pgfqpoint{2.266977in}{-3.801034in}}%
\pgfpathcurveto{\pgfqpoint{2.273489in}{-3.794523in}}{\pgfqpoint{2.277147in}{-3.785690in}}{\pgfqpoint{2.277147in}{-3.776482in}}%
\pgfpathcurveto{\pgfqpoint{2.277147in}{-3.767273in}}{\pgfqpoint{2.273489in}{-3.758441in}}{\pgfqpoint{2.266977in}{-3.751929in}}%
\pgfpathcurveto{\pgfqpoint{2.260466in}{-3.745418in}}{\pgfqpoint{2.251634in}{-3.741759in}}{\pgfqpoint{2.242425in}{-3.741759in}}%
\pgfpathcurveto{\pgfqpoint{2.233217in}{-3.741759in}}{\pgfqpoint{2.224384in}{-3.745418in}}{\pgfqpoint{2.217873in}{-3.751929in}}%
\pgfpathcurveto{\pgfqpoint{2.211362in}{-3.758441in}}{\pgfqpoint{2.207703in}{-3.767273in}}{\pgfqpoint{2.207703in}{-3.776482in}}%
\pgfpathcurveto{\pgfqpoint{2.207703in}{-3.785690in}}{\pgfqpoint{2.211362in}{-3.794523in}}{\pgfqpoint{2.217873in}{-3.801034in}}%
\pgfpathcurveto{\pgfqpoint{2.224384in}{-3.807545in}}{\pgfqpoint{2.233217in}{-3.811204in}}{\pgfqpoint{2.242425in}{-3.811204in}}%
\pgfpathlineto{\pgfqpoint{2.242425in}{-3.811204in}}%
\pgfpathclose%
\pgfusepath{stroke,fill}%
\end{pgfscope}%
\begin{pgfscope}%
\pgfpathrectangle{\pgfqpoint{0.050000in}{0.050000in}}{\pgfqpoint{2.419000in}{2.419000in}}%
\pgfusepath{clip}%
\pgfsetbuttcap%
\pgfsetroundjoin%
\definecolor{currentfill}{rgb}{0.866667,0.800000,0.466667}%
\pgfsetfillcolor{currentfill}%
\pgfsetfillopacity{0.696691}%
\pgfsetlinewidth{1.003750pt}%
\definecolor{currentstroke}{rgb}{0.866667,0.800000,0.466667}%
\pgfsetstrokecolor{currentstroke}%
\pgfsetstrokeopacity{0.696691}%
\pgfsetdash{}{0pt}%
\pgfpathmoveto{\pgfqpoint{10.551065in}{-3.811204in}}%
\pgfpathcurveto{\pgfqpoint{10.560273in}{-3.811204in}}{\pgfqpoint{10.569106in}{-3.807545in}}{\pgfqpoint{10.575617in}{-3.801034in}}%
\pgfpathcurveto{\pgfqpoint{10.582128in}{-3.794523in}}{\pgfqpoint{10.585787in}{-3.785690in}}{\pgfqpoint{10.585787in}{-3.776482in}}%
\pgfpathcurveto{\pgfqpoint{10.585787in}{-3.767273in}}{\pgfqpoint{10.582128in}{-3.758441in}}{\pgfqpoint{10.575617in}{-3.751929in}}%
\pgfpathcurveto{\pgfqpoint{10.569106in}{-3.745418in}}{\pgfqpoint{10.560273in}{-3.741759in}}{\pgfqpoint{10.551065in}{-3.741759in}}%
\pgfpathcurveto{\pgfqpoint{10.541856in}{-3.741759in}}{\pgfqpoint{10.533024in}{-3.745418in}}{\pgfqpoint{10.526512in}{-3.751929in}}%
\pgfpathcurveto{\pgfqpoint{10.520001in}{-3.758441in}}{\pgfqpoint{10.516343in}{-3.767273in}}{\pgfqpoint{10.516343in}{-3.776482in}}%
\pgfpathcurveto{\pgfqpoint{10.516343in}{-3.785690in}}{\pgfqpoint{10.520001in}{-3.794523in}}{\pgfqpoint{10.526512in}{-3.801034in}}%
\pgfpathcurveto{\pgfqpoint{10.533024in}{-3.807545in}}{\pgfqpoint{10.541856in}{-3.811204in}}{\pgfqpoint{10.551065in}{-3.811204in}}%
\pgfpathlineto{\pgfqpoint{10.551065in}{-3.811204in}}%
\pgfpathclose%
\pgfusepath{stroke,fill}%
\end{pgfscope}%
\begin{pgfscope}%
\pgfpathrectangle{\pgfqpoint{0.050000in}{0.050000in}}{\pgfqpoint{2.419000in}{2.419000in}}%
\pgfusepath{clip}%
\pgfsetbuttcap%
\pgfsetroundjoin%
\definecolor{currentfill}{rgb}{0.866667,0.800000,0.466667}%
\pgfsetfillcolor{currentfill}%
\pgfsetfillopacity{0.715800}%
\pgfsetlinewidth{1.003750pt}%
\definecolor{currentstroke}{rgb}{0.866667,0.800000,0.466667}%
\pgfsetstrokecolor{currentstroke}%
\pgfsetstrokeopacity{0.715800}%
\pgfsetdash{}{0pt}%
\pgfpathmoveto{\pgfqpoint{-1.207412in}{-4.198575in}}%
\pgfpathcurveto{\pgfqpoint{-1.198203in}{-4.198575in}}{\pgfqpoint{-1.189371in}{-4.194916in}}{\pgfqpoint{-1.182860in}{-4.188405in}}%
\pgfpathcurveto{\pgfqpoint{-1.176348in}{-4.181894in}}{\pgfqpoint{-1.172690in}{-4.173061in}}{\pgfqpoint{-1.172690in}{-4.163853in}}%
\pgfpathcurveto{\pgfqpoint{-1.172690in}{-4.154644in}}{\pgfqpoint{-1.176348in}{-4.145812in}}{\pgfqpoint{-1.182860in}{-4.139300in}}%
\pgfpathcurveto{\pgfqpoint{-1.189371in}{-4.132789in}}{\pgfqpoint{-1.198203in}{-4.129130in}}{\pgfqpoint{-1.207412in}{-4.129130in}}%
\pgfpathcurveto{\pgfqpoint{-1.216620in}{-4.129130in}}{\pgfqpoint{-1.225453in}{-4.132789in}}{\pgfqpoint{-1.231964in}{-4.139300in}}%
\pgfpathcurveto{\pgfqpoint{-1.238476in}{-4.145812in}}{\pgfqpoint{-1.242134in}{-4.154644in}}{\pgfqpoint{-1.242134in}{-4.163853in}}%
\pgfpathcurveto{\pgfqpoint{-1.242134in}{-4.173061in}}{\pgfqpoint{-1.238476in}{-4.181894in}}{\pgfqpoint{-1.231964in}{-4.188405in}}%
\pgfpathcurveto{\pgfqpoint{-1.225453in}{-4.194916in}}{\pgfqpoint{-1.216620in}{-4.198575in}}{\pgfqpoint{-1.207412in}{-4.198575in}}%
\pgfpathlineto{\pgfqpoint{-1.207412in}{-4.198575in}}%
\pgfpathclose%
\pgfusepath{stroke,fill}%
\end{pgfscope}%
\begin{pgfscope}%
\pgfpathrectangle{\pgfqpoint{0.050000in}{0.050000in}}{\pgfqpoint{2.419000in}{2.419000in}}%
\pgfusepath{clip}%
\pgfsetbuttcap%
\pgfsetroundjoin%
\definecolor{currentfill}{rgb}{0.866667,0.800000,0.466667}%
\pgfsetfillcolor{currentfill}%
\pgfsetfillopacity{0.715800}%
\pgfsetlinewidth{1.003750pt}%
\definecolor{currentstroke}{rgb}{0.866667,0.800000,0.466667}%
\pgfsetstrokecolor{currentstroke}%
\pgfsetstrokeopacity{0.715800}%
\pgfsetdash{}{0pt}%
\pgfpathmoveto{\pgfqpoint{7.356304in}{-4.198575in}}%
\pgfpathcurveto{\pgfqpoint{7.365512in}{-4.198575in}}{\pgfqpoint{7.374345in}{-4.194916in}}{\pgfqpoint{7.380856in}{-4.188405in}}%
\pgfpathcurveto{\pgfqpoint{7.387367in}{-4.181894in}}{\pgfqpoint{7.391026in}{-4.173061in}}{\pgfqpoint{7.391026in}{-4.163853in}}%
\pgfpathcurveto{\pgfqpoint{7.391026in}{-4.154644in}}{\pgfqpoint{7.387367in}{-4.145812in}}{\pgfqpoint{7.380856in}{-4.139300in}}%
\pgfpathcurveto{\pgfqpoint{7.374345in}{-4.132789in}}{\pgfqpoint{7.365512in}{-4.129130in}}{\pgfqpoint{7.356304in}{-4.129130in}}%
\pgfpathcurveto{\pgfqpoint{7.347095in}{-4.129130in}}{\pgfqpoint{7.338263in}{-4.132789in}}{\pgfqpoint{7.331751in}{-4.139300in}}%
\pgfpathcurveto{\pgfqpoint{7.325240in}{-4.145812in}}{\pgfqpoint{7.321581in}{-4.154644in}}{\pgfqpoint{7.321581in}{-4.163853in}}%
\pgfpathcurveto{\pgfqpoint{7.321581in}{-4.173061in}}{\pgfqpoint{7.325240in}{-4.181894in}}{\pgfqpoint{7.331751in}{-4.188405in}}%
\pgfpathcurveto{\pgfqpoint{7.338263in}{-4.194916in}}{\pgfqpoint{7.347095in}{-4.198575in}}{\pgfqpoint{7.356304in}{-4.198575in}}%
\pgfpathlineto{\pgfqpoint{7.356304in}{-4.198575in}}%
\pgfpathclose%
\pgfusepath{stroke,fill}%
\end{pgfscope}%
\begin{pgfscope}%
\pgfpathrectangle{\pgfqpoint{0.050000in}{0.050000in}}{\pgfqpoint{2.419000in}{2.419000in}}%
\pgfusepath{clip}%
\pgfsetbuttcap%
\pgfsetroundjoin%
\definecolor{currentfill}{rgb}{0.866667,0.800000,0.466667}%
\pgfsetfillcolor{currentfill}%
\pgfsetfillopacity{0.736118}%
\pgfsetlinewidth{1.003750pt}%
\definecolor{currentstroke}{rgb}{0.866667,0.800000,0.466667}%
\pgfsetstrokecolor{currentstroke}%
\pgfsetstrokeopacity{0.736118}%
\pgfsetdash{}{0pt}%
\pgfpathmoveto{\pgfqpoint{3.959171in}{-4.610484in}}%
\pgfpathcurveto{\pgfqpoint{3.968379in}{-4.610484in}}{\pgfqpoint{3.977212in}{-4.606825in}}{\pgfqpoint{3.983723in}{-4.600314in}}%
\pgfpathcurveto{\pgfqpoint{3.990234in}{-4.593802in}}{\pgfqpoint{3.993893in}{-4.584970in}}{\pgfqpoint{3.993893in}{-4.575761in}}%
\pgfpathcurveto{\pgfqpoint{3.993893in}{-4.566553in}}{\pgfqpoint{3.990234in}{-4.557720in}}{\pgfqpoint{3.983723in}{-4.551209in}}%
\pgfpathcurveto{\pgfqpoint{3.977212in}{-4.544698in}}{\pgfqpoint{3.968379in}{-4.541039in}}{\pgfqpoint{3.959171in}{-4.541039in}}%
\pgfpathcurveto{\pgfqpoint{3.949962in}{-4.541039in}}{\pgfqpoint{3.941130in}{-4.544698in}}{\pgfqpoint{3.934618in}{-4.551209in}}%
\pgfpathcurveto{\pgfqpoint{3.928107in}{-4.557720in}}{\pgfqpoint{3.924448in}{-4.566553in}}{\pgfqpoint{3.924448in}{-4.575761in}}%
\pgfpathcurveto{\pgfqpoint{3.924448in}{-4.584970in}}{\pgfqpoint{3.928107in}{-4.593802in}}{\pgfqpoint{3.934618in}{-4.600314in}}%
\pgfpathcurveto{\pgfqpoint{3.941130in}{-4.606825in}}{\pgfqpoint{3.949962in}{-4.610484in}}{\pgfqpoint{3.959171in}{-4.610484in}}%
\pgfpathlineto{\pgfqpoint{3.959171in}{-4.610484in}}%
\pgfpathclose%
\pgfusepath{stroke,fill}%
\end{pgfscope}%
\begin{pgfscope}%
\pgfpathrectangle{\pgfqpoint{0.050000in}{0.050000in}}{\pgfqpoint{2.419000in}{2.419000in}}%
\pgfusepath{clip}%
\pgfsetbuttcap%
\pgfsetroundjoin%
\definecolor{currentfill}{rgb}{0.866667,0.800000,0.466667}%
\pgfsetfillcolor{currentfill}%
\pgfsetfillopacity{0.757766}%
\pgfsetlinewidth{1.003750pt}%
\definecolor{currentstroke}{rgb}{0.866667,0.800000,0.466667}%
\pgfsetstrokecolor{currentstroke}%
\pgfsetstrokeopacity{0.757766}%
\pgfsetdash{}{0pt}%
\pgfpathmoveto{\pgfqpoint{0.339808in}{-5.049338in}}%
\pgfpathcurveto{\pgfqpoint{0.349017in}{-5.049338in}}{\pgfqpoint{0.357849in}{-5.045680in}}{\pgfqpoint{0.364361in}{-5.039168in}}%
\pgfpathcurveto{\pgfqpoint{0.370872in}{-5.032657in}}{\pgfqpoint{0.374531in}{-5.023824in}}{\pgfqpoint{0.374531in}{-5.014616in}}%
\pgfpathcurveto{\pgfqpoint{0.374531in}{-5.005408in}}{\pgfqpoint{0.370872in}{-4.996575in}}{\pgfqpoint{0.364361in}{-4.990064in}}%
\pgfpathcurveto{\pgfqpoint{0.357849in}{-4.983552in}}{\pgfqpoint{0.349017in}{-4.979894in}}{\pgfqpoint{0.339808in}{-4.979894in}}%
\pgfpathcurveto{\pgfqpoint{0.330600in}{-4.979894in}}{\pgfqpoint{0.321767in}{-4.983552in}}{\pgfqpoint{0.315256in}{-4.990064in}}%
\pgfpathcurveto{\pgfqpoint{0.308745in}{-4.996575in}}{\pgfqpoint{0.305086in}{-5.005408in}}{\pgfqpoint{0.305086in}{-5.014616in}}%
\pgfpathcurveto{\pgfqpoint{0.305086in}{-5.023824in}}{\pgfqpoint{0.308745in}{-5.032657in}}{\pgfqpoint{0.315256in}{-5.039168in}}%
\pgfpathcurveto{\pgfqpoint{0.321767in}{-5.045680in}}{\pgfqpoint{0.330600in}{-5.049338in}}{\pgfqpoint{0.339808in}{-5.049338in}}%
\pgfpathlineto{\pgfqpoint{0.339808in}{-5.049338in}}%
\pgfpathclose%
\pgfusepath{stroke,fill}%
\end{pgfscope}%
\begin{pgfscope}%
\pgfpathrectangle{\pgfqpoint{0.050000in}{0.050000in}}{\pgfqpoint{2.419000in}{2.419000in}}%
\pgfusepath{clip}%
\pgfsetbuttcap%
\pgfsetroundjoin%
\definecolor{currentfill}{rgb}{0.866667,0.800000,0.466667}%
\pgfsetfillcolor{currentfill}%
\pgfsetfillopacity{0.757766}%
\pgfsetlinewidth{1.003750pt}%
\definecolor{currentstroke}{rgb}{0.866667,0.800000,0.466667}%
\pgfsetstrokecolor{currentstroke}%
\pgfsetstrokeopacity{0.757766}%
\pgfsetdash{}{0pt}%
\pgfpathmoveto{\pgfqpoint{9.463734in}{-5.049338in}}%
\pgfpathcurveto{\pgfqpoint{9.472943in}{-5.049338in}}{\pgfqpoint{9.481775in}{-5.045680in}}{\pgfqpoint{9.488287in}{-5.039168in}}%
\pgfpathcurveto{\pgfqpoint{9.494798in}{-5.032657in}}{\pgfqpoint{9.498456in}{-5.023824in}}{\pgfqpoint{9.498456in}{-5.014616in}}%
\pgfpathcurveto{\pgfqpoint{9.498456in}{-5.005408in}}{\pgfqpoint{9.494798in}{-4.996575in}}{\pgfqpoint{9.488287in}{-4.990064in}}%
\pgfpathcurveto{\pgfqpoint{9.481775in}{-4.983552in}}{\pgfqpoint{9.472943in}{-4.979894in}}{\pgfqpoint{9.463734in}{-4.979894in}}%
\pgfpathcurveto{\pgfqpoint{9.454526in}{-4.979894in}}{\pgfqpoint{9.445693in}{-4.983552in}}{\pgfqpoint{9.439182in}{-4.990064in}}%
\pgfpathcurveto{\pgfqpoint{9.432671in}{-4.996575in}}{\pgfqpoint{9.429012in}{-5.005408in}}{\pgfqpoint{9.429012in}{-5.014616in}}%
\pgfpathcurveto{\pgfqpoint{9.429012in}{-5.023824in}}{\pgfqpoint{9.432671in}{-5.032657in}}{\pgfqpoint{9.439182in}{-5.039168in}}%
\pgfpathcurveto{\pgfqpoint{9.445693in}{-5.045680in}}{\pgfqpoint{9.454526in}{-5.049338in}}{\pgfqpoint{9.463734in}{-5.049338in}}%
\pgfpathlineto{\pgfqpoint{9.463734in}{-5.049338in}}%
\pgfpathclose%
\pgfusepath{stroke,fill}%
\end{pgfscope}%
\begin{pgfscope}%
\pgfpathrectangle{\pgfqpoint{0.050000in}{0.050000in}}{\pgfqpoint{2.419000in}{2.419000in}}%
\pgfusepath{clip}%
\pgfsetbuttcap%
\pgfsetroundjoin%
\definecolor{currentfill}{rgb}{0.866667,0.800000,0.466667}%
\pgfsetfillcolor{currentfill}%
\pgfsetfillopacity{0.780879}%
\pgfsetlinewidth{1.003750pt}%
\definecolor{currentstroke}{rgb}{0.866667,0.800000,0.466667}%
\pgfsetstrokecolor{currentstroke}%
\pgfsetstrokeopacity{0.780879}%
\pgfsetdash{}{0pt}%
\pgfpathmoveto{\pgfqpoint{5.908119in}{-5.517872in}}%
\pgfpathcurveto{\pgfqpoint{5.917327in}{-5.517872in}}{\pgfqpoint{5.926160in}{-5.514213in}}{\pgfqpoint{5.932671in}{-5.507702in}}%
\pgfpathcurveto{\pgfqpoint{5.939183in}{-5.501191in}}{\pgfqpoint{5.942841in}{-5.492358in}}{\pgfqpoint{5.942841in}{-5.483150in}}%
\pgfpathcurveto{\pgfqpoint{5.942841in}{-5.473941in}}{\pgfqpoint{5.939183in}{-5.465109in}}{\pgfqpoint{5.932671in}{-5.458597in}}%
\pgfpathcurveto{\pgfqpoint{5.926160in}{-5.452086in}}{\pgfqpoint{5.917327in}{-5.448428in}}{\pgfqpoint{5.908119in}{-5.448428in}}%
\pgfpathcurveto{\pgfqpoint{5.898910in}{-5.448428in}}{\pgfqpoint{5.890078in}{-5.452086in}}{\pgfqpoint{5.883567in}{-5.458597in}}%
\pgfpathcurveto{\pgfqpoint{5.877055in}{-5.465109in}}{\pgfqpoint{5.873397in}{-5.473941in}}{\pgfqpoint{5.873397in}{-5.483150in}}%
\pgfpathcurveto{\pgfqpoint{5.873397in}{-5.492358in}}{\pgfqpoint{5.877055in}{-5.501191in}}{\pgfqpoint{5.883567in}{-5.507702in}}%
\pgfpathcurveto{\pgfqpoint{5.890078in}{-5.514213in}}{\pgfqpoint{5.898910in}{-5.517872in}}{\pgfqpoint{5.908119in}{-5.517872in}}%
\pgfpathlineto{\pgfqpoint{5.908119in}{-5.517872in}}%
\pgfpathclose%
\pgfusepath{stroke,fill}%
\end{pgfscope}%
\begin{pgfscope}%
\pgfpathrectangle{\pgfqpoint{0.050000in}{0.050000in}}{\pgfqpoint{2.419000in}{2.419000in}}%
\pgfusepath{clip}%
\pgfsetbuttcap%
\pgfsetroundjoin%
\definecolor{currentfill}{rgb}{0.866667,0.800000,0.466667}%
\pgfsetfillcolor{currentfill}%
\pgfsetfillopacity{0.805608}%
\pgfsetlinewidth{1.003750pt}%
\definecolor{currentstroke}{rgb}{0.866667,0.800000,0.466667}%
\pgfsetstrokecolor{currentstroke}%
\pgfsetstrokeopacity{0.805608}%
\pgfsetdash{}{0pt}%
\pgfpathmoveto{\pgfqpoint{2.103626in}{-6.019201in}}%
\pgfpathcurveto{\pgfqpoint{2.112834in}{-6.019201in}}{\pgfqpoint{2.121667in}{-6.015543in}}{\pgfqpoint{2.128178in}{-6.009031in}}%
\pgfpathcurveto{\pgfqpoint{2.134690in}{-6.002520in}}{\pgfqpoint{2.138348in}{-5.993687in}}{\pgfqpoint{2.138348in}{-5.984479in}}%
\pgfpathcurveto{\pgfqpoint{2.138348in}{-5.975270in}}{\pgfqpoint{2.134690in}{-5.966438in}}{\pgfqpoint{2.128178in}{-5.959927in}}%
\pgfpathcurveto{\pgfqpoint{2.121667in}{-5.953415in}}{\pgfqpoint{2.112834in}{-5.949757in}}{\pgfqpoint{2.103626in}{-5.949757in}}%
\pgfpathcurveto{\pgfqpoint{2.094418in}{-5.949757in}}{\pgfqpoint{2.085585in}{-5.953415in}}{\pgfqpoint{2.079074in}{-5.959927in}}%
\pgfpathcurveto{\pgfqpoint{2.072562in}{-5.966438in}}{\pgfqpoint{2.068904in}{-5.975270in}}{\pgfqpoint{2.068904in}{-5.984479in}}%
\pgfpathcurveto{\pgfqpoint{2.068904in}{-5.993687in}}{\pgfqpoint{2.072562in}{-6.002520in}}{\pgfqpoint{2.079074in}{-6.009031in}}%
\pgfpathcurveto{\pgfqpoint{2.085585in}{-6.015543in}}{\pgfqpoint{2.094418in}{-6.019201in}}{\pgfqpoint{2.103626in}{-6.019201in}}%
\pgfpathlineto{\pgfqpoint{2.103626in}{-6.019201in}}%
\pgfpathclose%
\pgfusepath{stroke,fill}%
\end{pgfscope}%
\begin{pgfscope}%
\pgfpathrectangle{\pgfqpoint{0.050000in}{0.050000in}}{\pgfqpoint{2.419000in}{2.419000in}}%
\pgfusepath{clip}%
\pgfsetbuttcap%
\pgfsetroundjoin%
\definecolor{currentfill}{rgb}{0.866667,0.800000,0.466667}%
\pgfsetfillcolor{currentfill}%
\pgfsetfillopacity{0.832132}%
\pgfsetlinewidth{1.003750pt}%
\definecolor{currentstroke}{rgb}{0.866667,0.800000,0.466667}%
\pgfsetstrokecolor{currentstroke}%
\pgfsetstrokeopacity{0.832132}%
\pgfsetdash{}{0pt}%
\pgfpathmoveto{\pgfqpoint{8.139798in}{-6.556894in}}%
\pgfpathcurveto{\pgfqpoint{8.149006in}{-6.556894in}}{\pgfqpoint{8.157839in}{-6.553235in}}{\pgfqpoint{8.164350in}{-6.546724in}}%
\pgfpathcurveto{\pgfqpoint{8.170861in}{-6.540212in}}{\pgfqpoint{8.174520in}{-6.531380in}}{\pgfqpoint{8.174520in}{-6.522171in}}%
\pgfpathcurveto{\pgfqpoint{8.174520in}{-6.512963in}}{\pgfqpoint{8.170861in}{-6.504130in}}{\pgfqpoint{8.164350in}{-6.497619in}}%
\pgfpathcurveto{\pgfqpoint{8.157839in}{-6.491108in}}{\pgfqpoint{8.149006in}{-6.487449in}}{\pgfqpoint{8.139798in}{-6.487449in}}%
\pgfpathcurveto{\pgfqpoint{8.130589in}{-6.487449in}}{\pgfqpoint{8.121757in}{-6.491108in}}{\pgfqpoint{8.115245in}{-6.497619in}}%
\pgfpathcurveto{\pgfqpoint{8.108734in}{-6.504130in}}{\pgfqpoint{8.105076in}{-6.512963in}}{\pgfqpoint{8.105076in}{-6.522171in}}%
\pgfpathcurveto{\pgfqpoint{8.105076in}{-6.531380in}}{\pgfqpoint{8.108734in}{-6.540212in}}{\pgfqpoint{8.115245in}{-6.546724in}}%
\pgfpathcurveto{\pgfqpoint{8.121757in}{-6.553235in}}{\pgfqpoint{8.130589in}{-6.556894in}}{\pgfqpoint{8.139798in}{-6.556894in}}%
\pgfpathlineto{\pgfqpoint{8.139798in}{-6.556894in}}%
\pgfpathclose%
\pgfusepath{stroke,fill}%
\end{pgfscope}%
\begin{pgfscope}%
\pgfpathrectangle{\pgfqpoint{0.050000in}{0.050000in}}{\pgfqpoint{2.419000in}{2.419000in}}%
\pgfusepath{clip}%
\pgfsetbuttcap%
\pgfsetroundjoin%
\definecolor{currentfill}{rgb}{0.866667,0.800000,0.466667}%
\pgfsetfillcolor{currentfill}%
\pgfsetfillopacity{0.860652}%
\pgfsetlinewidth{1.003750pt}%
\definecolor{currentstroke}{rgb}{0.866667,0.800000,0.466667}%
\pgfsetstrokecolor{currentstroke}%
\pgfsetstrokeopacity{0.860652}%
\pgfsetdash{}{0pt}%
\pgfpathmoveto{\pgfqpoint{4.132947in}{-7.135055in}}%
\pgfpathcurveto{\pgfqpoint{4.142155in}{-7.135055in}}{\pgfqpoint{4.150988in}{-7.131396in}}{\pgfqpoint{4.157499in}{-7.124885in}}%
\pgfpathcurveto{\pgfqpoint{4.164010in}{-7.118374in}}{\pgfqpoint{4.167669in}{-7.109541in}}{\pgfqpoint{4.167669in}{-7.100333in}}%
\pgfpathcurveto{\pgfqpoint{4.167669in}{-7.091124in}}{\pgfqpoint{4.164010in}{-7.082292in}}{\pgfqpoint{4.157499in}{-7.075780in}}%
\pgfpathcurveto{\pgfqpoint{4.150988in}{-7.069269in}}{\pgfqpoint{4.142155in}{-7.065611in}}{\pgfqpoint{4.132947in}{-7.065611in}}%
\pgfpathcurveto{\pgfqpoint{4.123738in}{-7.065611in}}{\pgfqpoint{4.114906in}{-7.069269in}}{\pgfqpoint{4.108394in}{-7.075780in}}%
\pgfpathcurveto{\pgfqpoint{4.101883in}{-7.082292in}}{\pgfqpoint{4.098225in}{-7.091124in}}{\pgfqpoint{4.098225in}{-7.100333in}}%
\pgfpathcurveto{\pgfqpoint{4.098225in}{-7.109541in}}{\pgfqpoint{4.101883in}{-7.118374in}}{\pgfqpoint{4.108394in}{-7.124885in}}%
\pgfpathcurveto{\pgfqpoint{4.114906in}{-7.131396in}}{\pgfqpoint{4.123738in}{-7.135055in}}{\pgfqpoint{4.132947in}{-7.135055in}}%
\pgfpathlineto{\pgfqpoint{4.132947in}{-7.135055in}}%
\pgfpathclose%
\pgfusepath{stroke,fill}%
\end{pgfscope}%
\begin{pgfscope}%
\pgfpathrectangle{\pgfqpoint{0.050000in}{0.050000in}}{\pgfqpoint{2.419000in}{2.419000in}}%
\pgfusepath{clip}%
\pgfsetbuttcap%
\pgfsetroundjoin%
\definecolor{currentfill}{rgb}{0.866667,0.800000,0.466667}%
\pgfsetfillcolor{currentfill}%
\pgfsetfillopacity{0.891402}%
\pgfsetlinewidth{1.003750pt}%
\definecolor{currentstroke}{rgb}{0.866667,0.800000,0.466667}%
\pgfsetstrokecolor{currentstroke}%
\pgfsetstrokeopacity{0.891402}%
\pgfsetdash{}{0pt}%
\pgfpathmoveto{\pgfqpoint{10.720542in}{-7.758432in}}%
\pgfpathcurveto{\pgfqpoint{10.729750in}{-7.758432in}}{\pgfqpoint{10.738583in}{-7.754774in}}{\pgfqpoint{10.745094in}{-7.748263in}}%
\pgfpathcurveto{\pgfqpoint{10.751605in}{-7.741751in}}{\pgfqpoint{10.755264in}{-7.732919in}}{\pgfqpoint{10.755264in}{-7.723710in}}%
\pgfpathcurveto{\pgfqpoint{10.755264in}{-7.714502in}}{\pgfqpoint{10.751605in}{-7.705669in}}{\pgfqpoint{10.745094in}{-7.699158in}}%
\pgfpathcurveto{\pgfqpoint{10.738583in}{-7.692647in}}{\pgfqpoint{10.729750in}{-7.688988in}}{\pgfqpoint{10.720542in}{-7.688988in}}%
\pgfpathcurveto{\pgfqpoint{10.711333in}{-7.688988in}}{\pgfqpoint{10.702501in}{-7.692647in}}{\pgfqpoint{10.695989in}{-7.699158in}}%
\pgfpathcurveto{\pgfqpoint{10.689478in}{-7.705669in}}{\pgfqpoint{10.685819in}{-7.714502in}}{\pgfqpoint{10.685819in}{-7.723710in}}%
\pgfpathcurveto{\pgfqpoint{10.685819in}{-7.732919in}}{\pgfqpoint{10.689478in}{-7.741751in}}{\pgfqpoint{10.695989in}{-7.748263in}}%
\pgfpathcurveto{\pgfqpoint{10.702501in}{-7.754774in}}{\pgfqpoint{10.711333in}{-7.758432in}}{\pgfqpoint{10.720542in}{-7.758432in}}%
\pgfpathlineto{\pgfqpoint{10.720542in}{-7.758432in}}%
\pgfpathclose%
\pgfusepath{stroke,fill}%
\end{pgfscope}%
\begin{pgfscope}%
\pgfpathrectangle{\pgfqpoint{0.050000in}{0.050000in}}{\pgfqpoint{2.419000in}{2.419000in}}%
\pgfusepath{clip}%
\pgfsetbuttcap%
\pgfsetroundjoin%
\definecolor{currentfill}{rgb}{0.866667,0.800000,0.466667}%
\pgfsetfillcolor{currentfill}%
\pgfsetfillopacity{0.924655}%
\pgfsetlinewidth{1.003750pt}%
\definecolor{currentstroke}{rgb}{0.866667,0.800000,0.466667}%
\pgfsetstrokecolor{currentstroke}%
\pgfsetstrokeopacity{0.924655}%
\pgfsetdash{}{0pt}%
\pgfpathmoveto{\pgfqpoint{6.492598in}{-8.432546in}}%
\pgfpathcurveto{\pgfqpoint{6.501806in}{-8.432546in}}{\pgfqpoint{6.510639in}{-8.428888in}}{\pgfqpoint{6.517150in}{-8.422376in}}%
\pgfpathcurveto{\pgfqpoint{6.523662in}{-8.415865in}}{\pgfqpoint{6.527320in}{-8.407032in}}{\pgfqpoint{6.527320in}{-8.397824in}}%
\pgfpathcurveto{\pgfqpoint{6.527320in}{-8.388616in}}{\pgfqpoint{6.523662in}{-8.379783in}}{\pgfqpoint{6.517150in}{-8.373272in}}%
\pgfpathcurveto{\pgfqpoint{6.510639in}{-8.366760in}}{\pgfqpoint{6.501806in}{-8.363102in}}{\pgfqpoint{6.492598in}{-8.363102in}}%
\pgfpathcurveto{\pgfqpoint{6.483390in}{-8.363102in}}{\pgfqpoint{6.474557in}{-8.366760in}}{\pgfqpoint{6.468046in}{-8.373272in}}%
\pgfpathcurveto{\pgfqpoint{6.461534in}{-8.379783in}}{\pgfqpoint{6.457876in}{-8.388616in}}{\pgfqpoint{6.457876in}{-8.397824in}}%
\pgfpathcurveto{\pgfqpoint{6.457876in}{-8.407032in}}{\pgfqpoint{6.461534in}{-8.415865in}}{\pgfqpoint{6.468046in}{-8.422376in}}%
\pgfpathcurveto{\pgfqpoint{6.474557in}{-8.428888in}}{\pgfqpoint{6.483390in}{-8.432546in}}{\pgfqpoint{6.492598in}{-8.432546in}}%
\pgfpathlineto{\pgfqpoint{6.492598in}{-8.432546in}}%
\pgfpathclose%
\pgfusepath{stroke,fill}%
\end{pgfscope}%
\begin{pgfscope}%
\pgfpathrectangle{\pgfqpoint{0.050000in}{0.050000in}}{\pgfqpoint{2.419000in}{2.419000in}}%
\pgfusepath{clip}%
\pgfsetbuttcap%
\pgfsetroundjoin%
\definecolor{currentfill}{rgb}{0.866667,0.800000,0.466667}%
\pgfsetfillcolor{currentfill}%
\pgfsetlinewidth{1.003750pt}%
\definecolor{currentstroke}{rgb}{0.866667,0.800000,0.466667}%
\pgfsetstrokecolor{currentstroke}%
\pgfsetdash{}{0pt}%
\pgfpathmoveto{\pgfqpoint{9.270382in}{-9.959955in}}%
\pgfpathcurveto{\pgfqpoint{9.279591in}{-9.959955in}}{\pgfqpoint{9.288423in}{-9.956296in}}{\pgfqpoint{9.294935in}{-9.949785in}}%
\pgfpathcurveto{\pgfqpoint{9.301446in}{-9.943273in}}{\pgfqpoint{9.305104in}{-9.934441in}}{\pgfqpoint{9.305104in}{-9.925232in}}%
\pgfpathcurveto{\pgfqpoint{9.305104in}{-9.916024in}}{\pgfqpoint{9.301446in}{-9.907191in}}{\pgfqpoint{9.294935in}{-9.900680in}}%
\pgfpathcurveto{\pgfqpoint{9.288423in}{-9.894169in}}{\pgfqpoint{9.279591in}{-9.890510in}}{\pgfqpoint{9.270382in}{-9.890510in}}%
\pgfpathcurveto{\pgfqpoint{9.261174in}{-9.890510in}}{\pgfqpoint{9.252341in}{-9.894169in}}{\pgfqpoint{9.245830in}{-9.900680in}}%
\pgfpathcurveto{\pgfqpoint{9.239319in}{-9.907191in}}{\pgfqpoint{9.235660in}{-9.916024in}}{\pgfqpoint{9.235660in}{-9.925232in}}%
\pgfpathcurveto{\pgfqpoint{9.235660in}{-9.934441in}}{\pgfqpoint{9.239319in}{-9.943273in}}{\pgfqpoint{9.245830in}{-9.949785in}}%
\pgfpathcurveto{\pgfqpoint{9.252341in}{-9.956296in}}{\pgfqpoint{9.261174in}{-9.959955in}}{\pgfqpoint{9.270382in}{-9.959955in}}%
\pgfpathlineto{\pgfqpoint{9.270382in}{-9.959955in}}%
\pgfpathclose%
\pgfusepath{stroke,fill}%
\end{pgfscope}%
\begin{pgfscope}%
\pgfpathrectangle{\pgfqpoint{0.050000in}{0.050000in}}{\pgfqpoint{2.419000in}{2.419000in}}%
\pgfusepath{clip}%
\pgfsetbuttcap%
\pgfsetroundjoin%
\pgfsetlinewidth{1.003750pt}%
\definecolor{currentstroke}{rgb}{0.000000,0.000000,0.000000}%
\pgfsetstrokecolor{currentstroke}%
\pgfsetdash{}{0pt}%
\pgfusepath{stroke}%
\end{pgfscope}%
\begin{pgfscope}%
\pgfpathrectangle{\pgfqpoint{0.050000in}{0.050000in}}{\pgfqpoint{2.419000in}{2.419000in}}%
\pgfusepath{clip}%
\pgfsetbuttcap%
\pgfsetroundjoin%
\pgfsetlinewidth{1.003750pt}%
\definecolor{currentstroke}{rgb}{0.000000,0.000000,0.000000}%
\pgfsetstrokecolor{currentstroke}%
\pgfsetdash{}{0pt}%
\pgfusepath{stroke}%
\end{pgfscope}%
\end{pgfpicture}%
\makeatother%
\endgroup%

	\caption{Dressed Graphene model}
	\label{fig:eg-x model}
\end{figure}
The kinetic term is
\begin{align}
	H_0 &= -t_{\mathrm{X}} \sum_{\langle ij \rangle, \sigma} d_{i, \sigma}^{\dagger} d_{j, \sigma}
	-t_{\mathrm{Gr}} \sum_{\langle ij \rangle, \sigma}
	c_{i, \sigma}^{(\mathrm{A}), \dagger} c_{j, \sigma}^{(\mathrm{B})}
	+ V \sum_{i, \sigma \sigma^{\prime}}
	d_{i, \sigma}^{\dagger} c_{i, \sigma^{\prime}}^{(\mathrm{A})} + \mathrm{h.c.}
	\label{eq:EG-X model Hamiltonian non-interacting}
\end{align}
with
\begin{itemize}
	\item \(d\) - operators on the X atom
	\item \(c^{(\epsilon)}\) - operators on the graphene sites (\(\epsilon = \mathrm{A}, \mathrm{B}\))
	\item \(t_{\mathrm{X}}\) - nearest neighbour hopping for X
	\item \(t_{\mathrm{Gr}}\) - nearest neighbour hopping between Graphene sites
	\item \(V\) - hopping between \(\mathrm{X}\) and Graphene \(\mathrm{A}\) sites.
\end{itemize}
The (attractive) Hubbard interaction has the following form:
\begin{equation}
	H_{\mathrm{int}} = -U_{\mathrm{X}} \sum_{i} d_{i, \uparrow}^{\dagger} d_{i, \downarrow}^{\dagger} d_{i, \downarrow} d_{i, \uparrow}
	- U_{\mathrm{Gr}} \sum_{i, \epsilon=A, B} c_{i, \uparrow}^{(\epsilon) \dagger} c_{i, \downarrow}^{(\epsilon) \dagger} c_{i, \downarrow}^{\epsilon} c_{i, \uparrow}^{\epsilon}
\end{equation}
The notation using different letters for the sites connects intuitively to the physical picture, but it is more economical and in line with the notation for mean field-theory established in \cref{sec:bcs-theory} to write the Hamiltonian using a sublattice index
\begin{equation}
	\alpha = 1, 2, 3
\end{equation}
with \(1 \hateq \mathrm{Gr}_{\mathrm{A}}\), \(2 \hateq \mathrm{Gr}_{\mathrm{B}}\), \(3 \hateq \mathrm{X}\).
Then we can write the non-interacting term as
\begin{equation}
	H_0 = \sum_{\langle i, j \rangle, \alpha, \beta, \sigma} \left[\mat{t} \right]_{i\alpha,j\beta} c_{i\alpha}^{\dagger} c_{j\beta}
\end{equation}
with the matrix in the sublattice indices
\begin{equation}
	\mat{t} = \begin{pmatrix}
		0 & -t_{\mathrm{Gr}} & V \delta_{ij} \\
		-t_{\mathrm{Gr}} & 0 & 0 \\
		V \delta_{ij} & 0 & -t_{\mathrm{X}} \\
	\end{pmatrix}\;
\end{equation}
Also write the interaction part as
\begin{equation}
	H_{\mathrm{int}} = - \sum_{i \alpha} U_{\alpha} c_{i\alpha \uparrow}^{\dagger} c_{i\alpha \downarrow}^{\dagger} c_{i\alpha \downarrow} c_{i\alpha \uparrow}\;.
\end{equation}
\todo{Clean up the section from here}
Using the Fourier transformation \cref{ch:dressed graphene reciprocal space}
\begin{align}
	H_0 &= \sum_{\vb{k}, \sigma, \sigma^{\prime}} \begin{pmatrix} c_{k, \sigma}^{A, \dagger} & c_{k, \sigma}^{B, \dagger} & d_{k, \sigma}^{\dagger} \end{pmatrix}
	\begin{pmatrix}
		0 & f_{Gr} & V \\
		f_{Gr}^* & 0 & 0 \\
		V & 0 & f_{X}
	\end{pmatrix} \begin{pmatrix} c_{k, \sigma}^{A} \\ c_{k, \sigma}^{B} \\ d_{k, \sigma} \end{pmatrix}
	\label{eq:dressed graphene Hamiltonian non-interacting matrix}
\end{align}
The band structure for the non-interacting dressed graphene model is easily obtained by diagonalising the matrix in \cref{eq:dressed graphene Hamiltonian non-interacting matrix}.
This was done in \cref{fig:dressed graphene model non-interacting bands}.
\begin{figure}[t]
	\centering
	 \import{images}{dressed graphene bands.pgf}
	\caption{Bands of the non-interacting dressed Graphene model, with parameters \(t_{\mathrm{X}} = 0 \cdot t_{\mathrm{Gr}}\)}
	\label{fig:dressed graphene model non-interacting bands}
\end{figure}

\end{document}
