\documentclass[../notes.tex]{subfiles}
\graphicspath{{\subfix{../images/}}, {\subfix{../}}}

\begin{document}
\raggedbottom

\chapter{Decorated Graphene Model}\label{ch:decorated graphene model}

Following the 2018 discovery of superconductivity in twisted bilayer Graphene \cite{caoUnconventionalSuperconductivityMagicangle2018}, graphene-based systems gained a renewed interest as a platform for strongly correlated physics.
Two methods to engineer strong electron correlations emerged: twisted multilayer systems  \cite{caoUnconventionalSuperconductivityMagicangle2018, tanakaSuperfluidStiffnessMagicangle2025, tormaSuperconductivitySuperfluidityQuantum2022, andreiGrapheneBilayersTwist2020, xieTopologyBoundedSuperfluidWeight2020} and multilayer systems without twisting, such as Bernal bilayer, ABC or ABCA layered systems \cite{pantaleonSuperconductivityCorrelatedPhases2023}.
Through different means, electrons in these systems become localized so that interaction effects get more strongly pronounced.
Connecting both kind of systems is the strong quantum geometry coming from the Graphene Dirac cones \cite{wehlingDiracMaterials2014}, which plays a role in stabilizing superconducting \cite{liangBandGeometryBerry2017, tanakaSuperfluidStiffnessMagicangle2025} and magnetic order \cite{abouelkomsanQuantumMetricInduced2023, liuOrbitalMagneticStates2021}.

\citeauthor{wittQuantumGeometryLocal2025} suggested another platform for strongly correlated physics based on Graphene with the same strong quantum geometry, but higher intrinsic energy scales and thus also higher critical temperatures for strong correlation phenomena \cite{wittQuantumGeometryLocal2025}.
The model is inspired by an earlier experiment \cite{ghosalElectronicCorrelationsEpitaxial2024} of a \ce{SiC}(0001) substrate with a single layer of Graphene on top and \ce{Sn} as an intercalant\footnote{An intercalant is an atom or molecule inserted between the layers of layered system.} between the substrate and the Graphene layer.
The system shows signs of Mott-Hubbard bands, a hallmark of strong correlation physics.
\citeauthor{wittQuantumGeometryLocal2025} suggested that by using different group-IV intercalants (\ce{C}, \ce{Si}, \ce{Ge}, \ce{Sn}, \ce{Pb}) between the graphene sheet and the semiconducting \ce{SiC}(0001) substrate, different distances to the Graphene sheet occur in the ground state.
Band structures obtained from \acrfull{dft} show a relatively flat band at the Fermi level from the intercalant’s \(p_z\) orbitals hybridized to the Dirac bands of graphene for all intercalants, with the hybridization strength being tuned by the equilibrium distance of the Graphene sheet and the intercalants.

In this thesis I will be treating an elemental model introduced in the work by \citeauthor{wittQuantumGeometryLocal2025} capturing the essential flat band character of the system.
The lattice structure can be seen in \cref{fig:decorated graphene model}.
It consists of the usual hexagonal Graphene lattice, with an additional atom at one of the sublattice sites providing the flat band.
Here, the hopping \(V\) models the hybridization 
\begin{SCfigure}[50][t]
	\centering
	\import{images/}{dressed graphene lattice.pdf_tex}
	\caption{\textbf{Lattice structure of decorated graphene honeycomb lattice.} with impurity X hybridized to sublattice site A. Only hopping t between sublattices A and B as well as V between X and A exist. Created using VESTA \cite{mommaVESTA3Threedimensional2011}.}
	\label{fig:decorated graphene model}
\end{SCfigure}

This elemental model shows two symmetry distinct Mott states for the small and large \(V\) regimes: in the low \(V\) regime, the \(\mathrm{X}\) are responsible for the development of local moments and Mottness occurs at, where in the high \(V\) limit, the \(\mathrm{B}\) atom are responsible.
Between these Mott states emerges a metallic state, similar to the topological phase transition of non-interacting bands in the Su-Schrieffer-Heger model \cite{suSolitonExcitationsPolyacetylene1980}.

In twisted or untwisted multilayer Graphene systems, the energy scale for the emergence of ordered phases is \(\mathcal{O} (\unit{\milli\electronvolt})\), corresponding to temperatures of a few \unit{\kelvin} \cite{nuckollsMicroscopicPerspectiveMoire2024, pantaleonSuperconductivityCorrelatedPhases2023a}.
In contrast, the energy scale in this decorated Graphene model is set by the hopping \(t\), i.e. \(\mathcal{O} (\unit{\electronvolt})\) for Graphene, so that the correlated flat band physics might persist to higher temperatures.

\section{Lattice Structure}\label{sec:lattice and band structure}

Monolayer graphene forms a honeycomb lattice \cite{yangStructureGrapheneIts2018}, which is a hexagonal Bravais lattice with a two-atom basis, as can be seen in \cref{sfig:graphene lattice structure}.
The primitive lattice vectors of the hexagonal lattice are:
\begin{align}
	\vb{a}_1 = \frac{a}{2} \begin{pmatrix} 1 \\ \sqrt{3} \end{pmatrix}, \; \vb{a}_2 = \frac{a}{2} \begin{pmatrix} 1 \\ -\sqrt{3} \end{pmatrix}
\end{align}
with lattice constant \(a = \sqrt{3} a_0 \approx \SI{2.46}{\angstrom}\), using the nearest-neighbor distance \(a_0\).
The vectors from atom \(A\) to the nearest-neighbor atoms \(B_i\) (\(i = 1, 2, 3,\)) are
\begin{align}
	\vb{\delta}_{AB, 1} = \begin{pmatrix} 0 \\ \frac{a}{\sqrt{3}} \end{pmatrix},\; \vb{\delta}_{AB, 2} = \begin{pmatrix} \frac{a}{2} \\ -\frac{a}{2\sqrt{3}} \end{pmatrix},\; \vb{\delta}_{AB, 3} = \begin{pmatrix} -\frac{a}{2} \\ -\frac{a}{2\sqrt{3}} \end{pmatrix}
\end{align}
and the vectors from atom \(B\) to the nearest-neighbor atoms \(A_i\) (\(i = 1, 2, 3,\)) are
\begin{align}
	\vb{\delta}_{BA, 1} = \begin{pmatrix} 0 \\ -\frac{a}{\sqrt{3}} \end{pmatrix},\; \vb{\delta}_{BA, 2} = \begin{pmatrix} -\frac{a}{2} \\ \frac{a}{2\sqrt{3}} \end{pmatrix},\; \vb{\delta}_{BA, 3} = \begin{pmatrix} \frac{a}{2} \\ \frac{a}{2\sqrt{3}} \end{pmatrix} \;.
\end{align}
\begin{figure}[tb]
	\centering
	\begin{subfigure}[t]{0.5\textwidth}
		\centering
		\caption{\hfill\null}\label{sfig:graphene lattice structure}
		%% Creator: Matplotlib, PGF backend
%%
%% To include the figure in your LaTeX document, write
%%   \input{<filename>.pgf}
%%
%% Make sure the required packages are loaded in your preamble
%%   \usepackage{pgf}
%%
%% Also ensure that all the required font packages are loaded; for instance,
%% the lmodern package is sometimes necessary when using math font.
%%   \usepackage{lmodern}
%%
%% Figures using additional raster images can only be included by \input if
%% they are in the same directory as the main LaTeX file. For loading figures
%% from other directories you can use the `import` package
%%   \usepackage{import}
%%
%% and then include the figures with
%%   \import{<path to file>}{<filename>.pgf}
%%
%% Matplotlib used the following preamble
%%   \def\mathdefault#1{#1}
%%   \everymath=\expandafter{\the\everymath\displaystyle}
%%   \IfFileExists{scrextend.sty}{
%%     \usepackage[fontsize=11.000000pt]{scrextend}
%%   }{
%%     \renewcommand{\normalsize}{\fontsize{11.000000}{13.200000}\selectfont}
%%     \normalsize
%%   }
%%   \usepackage{fontspec}\usepackage{unicode-math}\setmathfont{texgyrepagella-math.otf}\setmainfont{texgyrepagella-math}\usepackage{nicefrac}
%%   \makeatletter\@ifpackageloaded{underscore}{}{\usepackage[strings]{underscore}}\makeatother
%%
\begingroup%
\makeatletter%
\begin{pgfpicture}%
\pgfpathrectangle{\pgfpointorigin}{\pgfqpoint{2.800000in}{2.240000in}}%
\pgfusepath{use as bounding box, clip}%
\begin{pgfscope}%
\pgfsetbuttcap%
\pgfsetmiterjoin%
\definecolor{currentfill}{rgb}{1.000000,1.000000,1.000000}%
\pgfsetfillcolor{currentfill}%
\pgfsetlinewidth{0.000000pt}%
\definecolor{currentstroke}{rgb}{1.000000,1.000000,1.000000}%
\pgfsetstrokecolor{currentstroke}%
\pgfsetdash{}{0pt}%
\pgfpathmoveto{\pgfqpoint{0.000000in}{0.000000in}}%
\pgfpathlineto{\pgfqpoint{2.800000in}{0.000000in}}%
\pgfpathlineto{\pgfqpoint{2.800000in}{2.240000in}}%
\pgfpathlineto{\pgfqpoint{0.000000in}{2.240000in}}%
\pgfpathlineto{\pgfqpoint{0.000000in}{0.000000in}}%
\pgfpathclose%
\pgfusepath{fill}%
\end{pgfscope}%
\begin{pgfscope}%
\pgfsetbuttcap%
\pgfsetmiterjoin%
\definecolor{currentfill}{rgb}{1.000000,1.000000,1.000000}%
\pgfsetfillcolor{currentfill}%
\pgfsetlinewidth{0.000000pt}%
\definecolor{currentstroke}{rgb}{0.000000,0.000000,0.000000}%
\pgfsetstrokecolor{currentstroke}%
\pgfsetstrokeopacity{0.000000}%
\pgfsetdash{}{0pt}%
\pgfpathmoveto{\pgfqpoint{0.338326in}{0.377767in}}%
\pgfpathlineto{\pgfqpoint{2.596020in}{0.377767in}}%
\pgfpathlineto{\pgfqpoint{2.596020in}{2.023029in}}%
\pgfpathlineto{\pgfqpoint{0.338326in}{2.023029in}}%
\pgfpathlineto{\pgfqpoint{0.338326in}{0.377767in}}%
\pgfpathclose%
\pgfusepath{fill}%
\end{pgfscope}%
\begin{pgfscope}%
\pgfpathrectangle{\pgfqpoint{0.338326in}{0.377767in}}{\pgfqpoint{2.257694in}{1.645262in}}%
\pgfusepath{clip}%
\pgfsetbuttcap%
\pgfsetroundjoin%
\pgfsetlinewidth{1.505625pt}%
\definecolor{currentstroke}{rgb}{0.000000,0.000000,0.000000}%
\pgfsetstrokecolor{currentstroke}%
\pgfsetdash{}{0pt}%
\pgfpathmoveto{\pgfqpoint{0.550288in}{0.671034in}}%
\pgfpathlineto{\pgfqpoint{0.779510in}{0.538693in}}%
\pgfusepath{stroke}%
\end{pgfscope}%
\begin{pgfscope}%
\pgfpathrectangle{\pgfqpoint{0.338326in}{0.377767in}}{\pgfqpoint{2.257694in}{1.645262in}}%
\pgfusepath{clip}%
\pgfsetbuttcap%
\pgfsetroundjoin%
\pgfsetlinewidth{1.505625pt}%
\definecolor{currentstroke}{rgb}{0.000000,0.000000,0.000000}%
\pgfsetstrokecolor{currentstroke}%
\pgfsetdash{}{0pt}%
\pgfpathmoveto{\pgfqpoint{0.550288in}{0.671034in}}%
\pgfpathlineto{\pgfqpoint{0.550288in}{0.935716in}}%
\pgfusepath{stroke}%
\end{pgfscope}%
\begin{pgfscope}%
\pgfpathrectangle{\pgfqpoint{0.338326in}{0.377767in}}{\pgfqpoint{2.257694in}{1.645262in}}%
\pgfusepath{clip}%
\pgfsetbuttcap%
\pgfsetroundjoin%
\pgfsetlinewidth{1.505625pt}%
\definecolor{currentstroke}{rgb}{0.000000,0.000000,0.000000}%
\pgfsetstrokecolor{currentstroke}%
\pgfsetdash{}{0pt}%
\pgfpathmoveto{\pgfqpoint{0.550288in}{0.935716in}}%
\pgfpathlineto{\pgfqpoint{0.779510in}{1.068057in}}%
\pgfusepath{stroke}%
\end{pgfscope}%
\begin{pgfscope}%
\pgfpathrectangle{\pgfqpoint{0.338326in}{0.377767in}}{\pgfqpoint{2.257694in}{1.645262in}}%
\pgfusepath{clip}%
\pgfsetbuttcap%
\pgfsetroundjoin%
\pgfsetlinewidth{1.505625pt}%
\definecolor{currentstroke}{rgb}{0.000000,0.000000,0.000000}%
\pgfsetstrokecolor{currentstroke}%
\pgfsetdash{}{0pt}%
\pgfpathmoveto{\pgfqpoint{0.779510in}{0.538693in}}%
\pgfpathlineto{\pgfqpoint{1.008731in}{0.671034in}}%
\pgfusepath{stroke}%
\end{pgfscope}%
\begin{pgfscope}%
\pgfpathrectangle{\pgfqpoint{0.338326in}{0.377767in}}{\pgfqpoint{2.257694in}{1.645262in}}%
\pgfusepath{clip}%
\pgfsetbuttcap%
\pgfsetroundjoin%
\pgfsetlinewidth{1.505625pt}%
\definecolor{currentstroke}{rgb}{0.000000,0.000000,0.000000}%
\pgfsetstrokecolor{currentstroke}%
\pgfsetdash{}{0pt}%
\pgfpathmoveto{\pgfqpoint{0.550288in}{1.465079in}}%
\pgfpathlineto{\pgfqpoint{0.779510in}{1.332739in}}%
\pgfusepath{stroke}%
\end{pgfscope}%
\begin{pgfscope}%
\pgfpathrectangle{\pgfqpoint{0.338326in}{0.377767in}}{\pgfqpoint{2.257694in}{1.645262in}}%
\pgfusepath{clip}%
\pgfsetbuttcap%
\pgfsetroundjoin%
\pgfsetlinewidth{1.505625pt}%
\definecolor{currentstroke}{rgb}{0.000000,0.000000,0.000000}%
\pgfsetstrokecolor{currentstroke}%
\pgfsetdash{}{0pt}%
\pgfpathmoveto{\pgfqpoint{0.550288in}{1.465079in}}%
\pgfpathlineto{\pgfqpoint{0.550288in}{1.729761in}}%
\pgfusepath{stroke}%
\end{pgfscope}%
\begin{pgfscope}%
\pgfpathrectangle{\pgfqpoint{0.338326in}{0.377767in}}{\pgfqpoint{2.257694in}{1.645262in}}%
\pgfusepath{clip}%
\pgfsetbuttcap%
\pgfsetroundjoin%
\pgfsetlinewidth{1.505625pt}%
\definecolor{currentstroke}{rgb}{0.000000,0.000000,0.000000}%
\pgfsetstrokecolor{currentstroke}%
\pgfsetdash{}{0pt}%
\pgfpathmoveto{\pgfqpoint{0.550288in}{1.729761in}}%
\pgfpathlineto{\pgfqpoint{0.779510in}{1.862102in}}%
\pgfusepath{stroke}%
\end{pgfscope}%
\begin{pgfscope}%
\pgfpathrectangle{\pgfqpoint{0.338326in}{0.377767in}}{\pgfqpoint{2.257694in}{1.645262in}}%
\pgfusepath{clip}%
\pgfsetbuttcap%
\pgfsetroundjoin%
\pgfsetlinewidth{1.505625pt}%
\definecolor{currentstroke}{rgb}{0.000000,0.000000,0.000000}%
\pgfsetstrokecolor{currentstroke}%
\pgfsetdash{}{0pt}%
\pgfpathmoveto{\pgfqpoint{0.779510in}{1.068057in}}%
\pgfpathlineto{\pgfqpoint{1.008731in}{0.935716in}}%
\pgfusepath{stroke}%
\end{pgfscope}%
\begin{pgfscope}%
\pgfpathrectangle{\pgfqpoint{0.338326in}{0.377767in}}{\pgfqpoint{2.257694in}{1.645262in}}%
\pgfusepath{clip}%
\pgfsetbuttcap%
\pgfsetroundjoin%
\pgfsetlinewidth{1.505625pt}%
\definecolor{currentstroke}{rgb}{0.000000,0.000000,0.000000}%
\pgfsetstrokecolor{currentstroke}%
\pgfsetdash{}{0pt}%
\pgfpathmoveto{\pgfqpoint{0.779510in}{1.068057in}}%
\pgfpathlineto{\pgfqpoint{0.779510in}{1.332739in}}%
\pgfusepath{stroke}%
\end{pgfscope}%
\begin{pgfscope}%
\pgfpathrectangle{\pgfqpoint{0.338326in}{0.377767in}}{\pgfqpoint{2.257694in}{1.645262in}}%
\pgfusepath{clip}%
\pgfsetbuttcap%
\pgfsetroundjoin%
\pgfsetlinewidth{1.505625pt}%
\definecolor{currentstroke}{rgb}{0.000000,0.000000,0.000000}%
\pgfsetstrokecolor{currentstroke}%
\pgfsetdash{}{0pt}%
\pgfpathmoveto{\pgfqpoint{0.779510in}{1.332739in}}%
\pgfpathlineto{\pgfqpoint{1.008731in}{1.465079in}}%
\pgfusepath{stroke}%
\end{pgfscope}%
\begin{pgfscope}%
\pgfpathrectangle{\pgfqpoint{0.338326in}{0.377767in}}{\pgfqpoint{2.257694in}{1.645262in}}%
\pgfusepath{clip}%
\pgfsetbuttcap%
\pgfsetroundjoin%
\pgfsetlinewidth{1.505625pt}%
\definecolor{currentstroke}{rgb}{0.000000,0.000000,0.000000}%
\pgfsetstrokecolor{currentstroke}%
\pgfsetdash{}{0pt}%
\pgfpathmoveto{\pgfqpoint{1.008731in}{0.671034in}}%
\pgfpathlineto{\pgfqpoint{1.237952in}{0.538693in}}%
\pgfusepath{stroke}%
\end{pgfscope}%
\begin{pgfscope}%
\pgfpathrectangle{\pgfqpoint{0.338326in}{0.377767in}}{\pgfqpoint{2.257694in}{1.645262in}}%
\pgfusepath{clip}%
\pgfsetbuttcap%
\pgfsetroundjoin%
\pgfsetlinewidth{1.505625pt}%
\definecolor{currentstroke}{rgb}{0.000000,0.000000,0.000000}%
\pgfsetstrokecolor{currentstroke}%
\pgfsetdash{}{0pt}%
\pgfpathmoveto{\pgfqpoint{1.008731in}{0.671034in}}%
\pgfpathlineto{\pgfqpoint{1.008731in}{0.935716in}}%
\pgfusepath{stroke}%
\end{pgfscope}%
\begin{pgfscope}%
\pgfpathrectangle{\pgfqpoint{0.338326in}{0.377767in}}{\pgfqpoint{2.257694in}{1.645262in}}%
\pgfusepath{clip}%
\pgfsetbuttcap%
\pgfsetroundjoin%
\pgfsetlinewidth{1.505625pt}%
\definecolor{currentstroke}{rgb}{0.000000,0.000000,0.000000}%
\pgfsetstrokecolor{currentstroke}%
\pgfsetdash{}{0pt}%
\pgfpathmoveto{\pgfqpoint{1.008731in}{0.935716in}}%
\pgfpathlineto{\pgfqpoint{1.237952in}{1.068057in}}%
\pgfusepath{stroke}%
\end{pgfscope}%
\begin{pgfscope}%
\pgfpathrectangle{\pgfqpoint{0.338326in}{0.377767in}}{\pgfqpoint{2.257694in}{1.645262in}}%
\pgfusepath{clip}%
\pgfsetbuttcap%
\pgfsetroundjoin%
\pgfsetlinewidth{1.505625pt}%
\definecolor{currentstroke}{rgb}{0.000000,0.000000,0.000000}%
\pgfsetstrokecolor{currentstroke}%
\pgfsetdash{}{0pt}%
\pgfpathmoveto{\pgfqpoint{1.237952in}{0.538693in}}%
\pgfpathlineto{\pgfqpoint{1.467173in}{0.671034in}}%
\pgfusepath{stroke}%
\end{pgfscope}%
\begin{pgfscope}%
\pgfpathrectangle{\pgfqpoint{0.338326in}{0.377767in}}{\pgfqpoint{2.257694in}{1.645262in}}%
\pgfusepath{clip}%
\pgfsetbuttcap%
\pgfsetroundjoin%
\pgfsetlinewidth{1.505625pt}%
\definecolor{currentstroke}{rgb}{0.000000,0.000000,0.000000}%
\pgfsetstrokecolor{currentstroke}%
\pgfsetdash{}{0pt}%
\pgfpathmoveto{\pgfqpoint{0.779510in}{1.862102in}}%
\pgfpathlineto{\pgfqpoint{1.008731in}{1.729761in}}%
\pgfusepath{stroke}%
\end{pgfscope}%
\begin{pgfscope}%
\pgfpathrectangle{\pgfqpoint{0.338326in}{0.377767in}}{\pgfqpoint{2.257694in}{1.645262in}}%
\pgfusepath{clip}%
\pgfsetbuttcap%
\pgfsetroundjoin%
\pgfsetlinewidth{1.505625pt}%
\definecolor{currentstroke}{rgb}{0.000000,0.000000,0.000000}%
\pgfsetstrokecolor{currentstroke}%
\pgfsetdash{}{0pt}%
\pgfpathmoveto{\pgfqpoint{1.008731in}{1.465079in}}%
\pgfpathlineto{\pgfqpoint{1.237952in}{1.332739in}}%
\pgfusepath{stroke}%
\end{pgfscope}%
\begin{pgfscope}%
\pgfpathrectangle{\pgfqpoint{0.338326in}{0.377767in}}{\pgfqpoint{2.257694in}{1.645262in}}%
\pgfusepath{clip}%
\pgfsetbuttcap%
\pgfsetroundjoin%
\pgfsetlinewidth{1.505625pt}%
\definecolor{currentstroke}{rgb}{0.000000,0.000000,0.000000}%
\pgfsetstrokecolor{currentstroke}%
\pgfsetdash{}{0pt}%
\pgfpathmoveto{\pgfqpoint{1.008731in}{1.465079in}}%
\pgfpathlineto{\pgfqpoint{1.008731in}{1.729761in}}%
\pgfusepath{stroke}%
\end{pgfscope}%
\begin{pgfscope}%
\pgfpathrectangle{\pgfqpoint{0.338326in}{0.377767in}}{\pgfqpoint{2.257694in}{1.645262in}}%
\pgfusepath{clip}%
\pgfsetbuttcap%
\pgfsetroundjoin%
\pgfsetlinewidth{1.505625pt}%
\definecolor{currentstroke}{rgb}{0.000000,0.000000,0.000000}%
\pgfsetstrokecolor{currentstroke}%
\pgfsetdash{}{0pt}%
\pgfpathmoveto{\pgfqpoint{1.008731in}{1.729761in}}%
\pgfpathlineto{\pgfqpoint{1.237952in}{1.862102in}}%
\pgfusepath{stroke}%
\end{pgfscope}%
\begin{pgfscope}%
\pgfpathrectangle{\pgfqpoint{0.338326in}{0.377767in}}{\pgfqpoint{2.257694in}{1.645262in}}%
\pgfusepath{clip}%
\pgfsetbuttcap%
\pgfsetroundjoin%
\pgfsetlinewidth{1.505625pt}%
\definecolor{currentstroke}{rgb}{0.000000,0.000000,0.000000}%
\pgfsetstrokecolor{currentstroke}%
\pgfsetdash{}{0pt}%
\pgfpathmoveto{\pgfqpoint{1.237952in}{1.068057in}}%
\pgfpathlineto{\pgfqpoint{1.467173in}{0.935716in}}%
\pgfusepath{stroke}%
\end{pgfscope}%
\begin{pgfscope}%
\pgfpathrectangle{\pgfqpoint{0.338326in}{0.377767in}}{\pgfqpoint{2.257694in}{1.645262in}}%
\pgfusepath{clip}%
\pgfsetbuttcap%
\pgfsetroundjoin%
\pgfsetlinewidth{1.505625pt}%
\definecolor{currentstroke}{rgb}{0.000000,0.000000,0.000000}%
\pgfsetstrokecolor{currentstroke}%
\pgfsetdash{}{0pt}%
\pgfpathmoveto{\pgfqpoint{1.237952in}{1.068057in}}%
\pgfpathlineto{\pgfqpoint{1.237952in}{1.332739in}}%
\pgfusepath{stroke}%
\end{pgfscope}%
\begin{pgfscope}%
\pgfpathrectangle{\pgfqpoint{0.338326in}{0.377767in}}{\pgfqpoint{2.257694in}{1.645262in}}%
\pgfusepath{clip}%
\pgfsetbuttcap%
\pgfsetroundjoin%
\pgfsetlinewidth{1.505625pt}%
\definecolor{currentstroke}{rgb}{0.000000,0.000000,0.000000}%
\pgfsetstrokecolor{currentstroke}%
\pgfsetdash{}{0pt}%
\pgfpathmoveto{\pgfqpoint{1.237952in}{1.332739in}}%
\pgfpathlineto{\pgfqpoint{1.467173in}{1.465079in}}%
\pgfusepath{stroke}%
\end{pgfscope}%
\begin{pgfscope}%
\pgfpathrectangle{\pgfqpoint{0.338326in}{0.377767in}}{\pgfqpoint{2.257694in}{1.645262in}}%
\pgfusepath{clip}%
\pgfsetbuttcap%
\pgfsetroundjoin%
\pgfsetlinewidth{1.505625pt}%
\definecolor{currentstroke}{rgb}{0.000000,0.000000,0.000000}%
\pgfsetstrokecolor{currentstroke}%
\pgfsetdash{}{0pt}%
\pgfpathmoveto{\pgfqpoint{1.467173in}{0.671034in}}%
\pgfpathlineto{\pgfqpoint{1.696394in}{0.538693in}}%
\pgfusepath{stroke}%
\end{pgfscope}%
\begin{pgfscope}%
\pgfpathrectangle{\pgfqpoint{0.338326in}{0.377767in}}{\pgfqpoint{2.257694in}{1.645262in}}%
\pgfusepath{clip}%
\pgfsetbuttcap%
\pgfsetroundjoin%
\pgfsetlinewidth{1.505625pt}%
\definecolor{currentstroke}{rgb}{0.000000,0.000000,0.000000}%
\pgfsetstrokecolor{currentstroke}%
\pgfsetdash{}{0pt}%
\pgfpathmoveto{\pgfqpoint{1.467173in}{0.671034in}}%
\pgfpathlineto{\pgfqpoint{1.467173in}{0.935716in}}%
\pgfusepath{stroke}%
\end{pgfscope}%
\begin{pgfscope}%
\pgfpathrectangle{\pgfqpoint{0.338326in}{0.377767in}}{\pgfqpoint{2.257694in}{1.645262in}}%
\pgfusepath{clip}%
\pgfsetbuttcap%
\pgfsetroundjoin%
\pgfsetlinewidth{1.505625pt}%
\definecolor{currentstroke}{rgb}{0.000000,0.000000,0.000000}%
\pgfsetstrokecolor{currentstroke}%
\pgfsetdash{}{0pt}%
\pgfpathmoveto{\pgfqpoint{1.467173in}{0.935716in}}%
\pgfpathlineto{\pgfqpoint{1.696394in}{1.068057in}}%
\pgfusepath{stroke}%
\end{pgfscope}%
\begin{pgfscope}%
\pgfpathrectangle{\pgfqpoint{0.338326in}{0.377767in}}{\pgfqpoint{2.257694in}{1.645262in}}%
\pgfusepath{clip}%
\pgfsetbuttcap%
\pgfsetroundjoin%
\pgfsetlinewidth{1.505625pt}%
\definecolor{currentstroke}{rgb}{0.000000,0.000000,0.000000}%
\pgfsetstrokecolor{currentstroke}%
\pgfsetdash{}{0pt}%
\pgfpathmoveto{\pgfqpoint{1.696394in}{0.538693in}}%
\pgfpathlineto{\pgfqpoint{1.925615in}{0.671034in}}%
\pgfusepath{stroke}%
\end{pgfscope}%
\begin{pgfscope}%
\pgfpathrectangle{\pgfqpoint{0.338326in}{0.377767in}}{\pgfqpoint{2.257694in}{1.645262in}}%
\pgfusepath{clip}%
\pgfsetbuttcap%
\pgfsetroundjoin%
\pgfsetlinewidth{1.505625pt}%
\definecolor{currentstroke}{rgb}{0.000000,0.000000,0.000000}%
\pgfsetstrokecolor{currentstroke}%
\pgfsetdash{}{0pt}%
\pgfpathmoveto{\pgfqpoint{1.237952in}{1.862102in}}%
\pgfpathlineto{\pgfqpoint{1.467173in}{1.729761in}}%
\pgfusepath{stroke}%
\end{pgfscope}%
\begin{pgfscope}%
\pgfpathrectangle{\pgfqpoint{0.338326in}{0.377767in}}{\pgfqpoint{2.257694in}{1.645262in}}%
\pgfusepath{clip}%
\pgfsetbuttcap%
\pgfsetroundjoin%
\pgfsetlinewidth{1.505625pt}%
\definecolor{currentstroke}{rgb}{0.000000,0.000000,0.000000}%
\pgfsetstrokecolor{currentstroke}%
\pgfsetdash{}{0pt}%
\pgfpathmoveto{\pgfqpoint{1.467173in}{1.465079in}}%
\pgfpathlineto{\pgfqpoint{1.696394in}{1.332739in}}%
\pgfusepath{stroke}%
\end{pgfscope}%
\begin{pgfscope}%
\pgfpathrectangle{\pgfqpoint{0.338326in}{0.377767in}}{\pgfqpoint{2.257694in}{1.645262in}}%
\pgfusepath{clip}%
\pgfsetbuttcap%
\pgfsetroundjoin%
\pgfsetlinewidth{1.505625pt}%
\definecolor{currentstroke}{rgb}{0.000000,0.000000,0.000000}%
\pgfsetstrokecolor{currentstroke}%
\pgfsetdash{}{0pt}%
\pgfpathmoveto{\pgfqpoint{1.467173in}{1.465079in}}%
\pgfpathlineto{\pgfqpoint{1.467173in}{1.729761in}}%
\pgfusepath{stroke}%
\end{pgfscope}%
\begin{pgfscope}%
\pgfpathrectangle{\pgfqpoint{0.338326in}{0.377767in}}{\pgfqpoint{2.257694in}{1.645262in}}%
\pgfusepath{clip}%
\pgfsetbuttcap%
\pgfsetroundjoin%
\pgfsetlinewidth{1.505625pt}%
\definecolor{currentstroke}{rgb}{0.000000,0.000000,0.000000}%
\pgfsetstrokecolor{currentstroke}%
\pgfsetdash{}{0pt}%
\pgfpathmoveto{\pgfqpoint{1.467173in}{1.729761in}}%
\pgfpathlineto{\pgfqpoint{1.696394in}{1.862102in}}%
\pgfusepath{stroke}%
\end{pgfscope}%
\begin{pgfscope}%
\pgfpathrectangle{\pgfqpoint{0.338326in}{0.377767in}}{\pgfqpoint{2.257694in}{1.645262in}}%
\pgfusepath{clip}%
\pgfsetbuttcap%
\pgfsetroundjoin%
\pgfsetlinewidth{1.505625pt}%
\definecolor{currentstroke}{rgb}{0.000000,0.000000,0.000000}%
\pgfsetstrokecolor{currentstroke}%
\pgfsetdash{}{0pt}%
\pgfpathmoveto{\pgfqpoint{1.696394in}{1.068057in}}%
\pgfpathlineto{\pgfqpoint{1.925615in}{0.935716in}}%
\pgfusepath{stroke}%
\end{pgfscope}%
\begin{pgfscope}%
\pgfpathrectangle{\pgfqpoint{0.338326in}{0.377767in}}{\pgfqpoint{2.257694in}{1.645262in}}%
\pgfusepath{clip}%
\pgfsetbuttcap%
\pgfsetroundjoin%
\pgfsetlinewidth{1.505625pt}%
\definecolor{currentstroke}{rgb}{0.000000,0.000000,0.000000}%
\pgfsetstrokecolor{currentstroke}%
\pgfsetdash{}{0pt}%
\pgfpathmoveto{\pgfqpoint{1.696394in}{1.068057in}}%
\pgfpathlineto{\pgfqpoint{1.696394in}{1.332739in}}%
\pgfusepath{stroke}%
\end{pgfscope}%
\begin{pgfscope}%
\pgfpathrectangle{\pgfqpoint{0.338326in}{0.377767in}}{\pgfqpoint{2.257694in}{1.645262in}}%
\pgfusepath{clip}%
\pgfsetbuttcap%
\pgfsetroundjoin%
\pgfsetlinewidth{1.505625pt}%
\definecolor{currentstroke}{rgb}{0.000000,0.000000,0.000000}%
\pgfsetstrokecolor{currentstroke}%
\pgfsetdash{}{0pt}%
\pgfpathmoveto{\pgfqpoint{1.696394in}{1.332739in}}%
\pgfpathlineto{\pgfqpoint{1.925615in}{1.465079in}}%
\pgfusepath{stroke}%
\end{pgfscope}%
\begin{pgfscope}%
\pgfpathrectangle{\pgfqpoint{0.338326in}{0.377767in}}{\pgfqpoint{2.257694in}{1.645262in}}%
\pgfusepath{clip}%
\pgfsetbuttcap%
\pgfsetroundjoin%
\pgfsetlinewidth{1.505625pt}%
\definecolor{currentstroke}{rgb}{0.000000,0.000000,0.000000}%
\pgfsetstrokecolor{currentstroke}%
\pgfsetdash{}{0pt}%
\pgfpathmoveto{\pgfqpoint{1.925615in}{0.671034in}}%
\pgfpathlineto{\pgfqpoint{1.925615in}{0.935716in}}%
\pgfusepath{stroke}%
\end{pgfscope}%
\begin{pgfscope}%
\pgfpathrectangle{\pgfqpoint{0.338326in}{0.377767in}}{\pgfqpoint{2.257694in}{1.645262in}}%
\pgfusepath{clip}%
\pgfsetbuttcap%
\pgfsetroundjoin%
\pgfsetlinewidth{1.505625pt}%
\definecolor{currentstroke}{rgb}{0.000000,0.000000,0.000000}%
\pgfsetstrokecolor{currentstroke}%
\pgfsetdash{}{0pt}%
\pgfpathmoveto{\pgfqpoint{1.925615in}{0.671034in}}%
\pgfpathlineto{\pgfqpoint{2.154837in}{0.538693in}}%
\pgfusepath{stroke}%
\end{pgfscope}%
\begin{pgfscope}%
\pgfpathrectangle{\pgfqpoint{0.338326in}{0.377767in}}{\pgfqpoint{2.257694in}{1.645262in}}%
\pgfusepath{clip}%
\pgfsetbuttcap%
\pgfsetroundjoin%
\pgfsetlinewidth{1.505625pt}%
\definecolor{currentstroke}{rgb}{0.000000,0.000000,0.000000}%
\pgfsetstrokecolor{currentstroke}%
\pgfsetdash{}{0pt}%
\pgfpathmoveto{\pgfqpoint{1.925615in}{0.935716in}}%
\pgfpathlineto{\pgfqpoint{2.154837in}{1.068057in}}%
\pgfusepath{stroke}%
\end{pgfscope}%
\begin{pgfscope}%
\pgfpathrectangle{\pgfqpoint{0.338326in}{0.377767in}}{\pgfqpoint{2.257694in}{1.645262in}}%
\pgfusepath{clip}%
\pgfsetbuttcap%
\pgfsetroundjoin%
\pgfsetlinewidth{1.505625pt}%
\definecolor{currentstroke}{rgb}{0.000000,0.000000,0.000000}%
\pgfsetstrokecolor{currentstroke}%
\pgfsetdash{}{0pt}%
\pgfpathmoveto{\pgfqpoint{2.154837in}{0.538693in}}%
\pgfpathlineto{\pgfqpoint{2.384058in}{0.671034in}}%
\pgfusepath{stroke}%
\end{pgfscope}%
\begin{pgfscope}%
\pgfpathrectangle{\pgfqpoint{0.338326in}{0.377767in}}{\pgfqpoint{2.257694in}{1.645262in}}%
\pgfusepath{clip}%
\pgfsetbuttcap%
\pgfsetroundjoin%
\pgfsetlinewidth{1.505625pt}%
\definecolor{currentstroke}{rgb}{0.000000,0.000000,0.000000}%
\pgfsetstrokecolor{currentstroke}%
\pgfsetdash{}{0pt}%
\pgfpathmoveto{\pgfqpoint{1.696394in}{1.862102in}}%
\pgfpathlineto{\pgfqpoint{1.925615in}{1.729761in}}%
\pgfusepath{stroke}%
\end{pgfscope}%
\begin{pgfscope}%
\pgfpathrectangle{\pgfqpoint{0.338326in}{0.377767in}}{\pgfqpoint{2.257694in}{1.645262in}}%
\pgfusepath{clip}%
\pgfsetbuttcap%
\pgfsetroundjoin%
\pgfsetlinewidth{1.505625pt}%
\definecolor{currentstroke}{rgb}{0.000000,0.000000,0.000000}%
\pgfsetstrokecolor{currentstroke}%
\pgfsetdash{}{0pt}%
\pgfpathmoveto{\pgfqpoint{1.925615in}{1.465079in}}%
\pgfpathlineto{\pgfqpoint{1.925615in}{1.729761in}}%
\pgfusepath{stroke}%
\end{pgfscope}%
\begin{pgfscope}%
\pgfpathrectangle{\pgfqpoint{0.338326in}{0.377767in}}{\pgfqpoint{2.257694in}{1.645262in}}%
\pgfusepath{clip}%
\pgfsetbuttcap%
\pgfsetroundjoin%
\pgfsetlinewidth{1.505625pt}%
\definecolor{currentstroke}{rgb}{0.000000,0.000000,0.000000}%
\pgfsetstrokecolor{currentstroke}%
\pgfsetdash{}{0pt}%
\pgfpathmoveto{\pgfqpoint{1.925615in}{1.465079in}}%
\pgfpathlineto{\pgfqpoint{2.154837in}{1.332739in}}%
\pgfusepath{stroke}%
\end{pgfscope}%
\begin{pgfscope}%
\pgfpathrectangle{\pgfqpoint{0.338326in}{0.377767in}}{\pgfqpoint{2.257694in}{1.645262in}}%
\pgfusepath{clip}%
\pgfsetbuttcap%
\pgfsetroundjoin%
\pgfsetlinewidth{1.505625pt}%
\definecolor{currentstroke}{rgb}{0.000000,0.000000,0.000000}%
\pgfsetstrokecolor{currentstroke}%
\pgfsetdash{}{0pt}%
\pgfpathmoveto{\pgfqpoint{1.925615in}{1.729761in}}%
\pgfpathlineto{\pgfqpoint{2.154837in}{1.862102in}}%
\pgfusepath{stroke}%
\end{pgfscope}%
\begin{pgfscope}%
\pgfpathrectangle{\pgfqpoint{0.338326in}{0.377767in}}{\pgfqpoint{2.257694in}{1.645262in}}%
\pgfusepath{clip}%
\pgfsetbuttcap%
\pgfsetroundjoin%
\pgfsetlinewidth{1.505625pt}%
\definecolor{currentstroke}{rgb}{0.000000,0.000000,0.000000}%
\pgfsetstrokecolor{currentstroke}%
\pgfsetdash{}{0pt}%
\pgfpathmoveto{\pgfqpoint{2.154837in}{1.068057in}}%
\pgfpathlineto{\pgfqpoint{2.384058in}{0.935716in}}%
\pgfusepath{stroke}%
\end{pgfscope}%
\begin{pgfscope}%
\pgfpathrectangle{\pgfqpoint{0.338326in}{0.377767in}}{\pgfqpoint{2.257694in}{1.645262in}}%
\pgfusepath{clip}%
\pgfsetbuttcap%
\pgfsetroundjoin%
\pgfsetlinewidth{1.505625pt}%
\definecolor{currentstroke}{rgb}{0.000000,0.000000,0.000000}%
\pgfsetstrokecolor{currentstroke}%
\pgfsetdash{}{0pt}%
\pgfpathmoveto{\pgfqpoint{2.154837in}{1.068057in}}%
\pgfpathlineto{\pgfqpoint{2.154837in}{1.332739in}}%
\pgfusepath{stroke}%
\end{pgfscope}%
\begin{pgfscope}%
\pgfpathrectangle{\pgfqpoint{0.338326in}{0.377767in}}{\pgfqpoint{2.257694in}{1.645262in}}%
\pgfusepath{clip}%
\pgfsetbuttcap%
\pgfsetroundjoin%
\pgfsetlinewidth{1.505625pt}%
\definecolor{currentstroke}{rgb}{0.000000,0.000000,0.000000}%
\pgfsetstrokecolor{currentstroke}%
\pgfsetdash{}{0pt}%
\pgfpathmoveto{\pgfqpoint{2.154837in}{1.332739in}}%
\pgfpathlineto{\pgfqpoint{2.384058in}{1.465079in}}%
\pgfusepath{stroke}%
\end{pgfscope}%
\begin{pgfscope}%
\pgfpathrectangle{\pgfqpoint{0.338326in}{0.377767in}}{\pgfqpoint{2.257694in}{1.645262in}}%
\pgfusepath{clip}%
\pgfsetbuttcap%
\pgfsetroundjoin%
\pgfsetlinewidth{1.505625pt}%
\definecolor{currentstroke}{rgb}{0.000000,0.000000,0.000000}%
\pgfsetstrokecolor{currentstroke}%
\pgfsetdash{}{0pt}%
\pgfpathmoveto{\pgfqpoint{2.384058in}{0.671034in}}%
\pgfpathlineto{\pgfqpoint{2.384058in}{0.935716in}}%
\pgfusepath{stroke}%
\end{pgfscope}%
\begin{pgfscope}%
\pgfpathrectangle{\pgfqpoint{0.338326in}{0.377767in}}{\pgfqpoint{2.257694in}{1.645262in}}%
\pgfusepath{clip}%
\pgfsetbuttcap%
\pgfsetroundjoin%
\pgfsetlinewidth{1.505625pt}%
\definecolor{currentstroke}{rgb}{0.000000,0.000000,0.000000}%
\pgfsetstrokecolor{currentstroke}%
\pgfsetdash{}{0pt}%
\pgfpathmoveto{\pgfqpoint{2.154837in}{1.862102in}}%
\pgfpathlineto{\pgfqpoint{2.384058in}{1.729761in}}%
\pgfusepath{stroke}%
\end{pgfscope}%
\begin{pgfscope}%
\pgfpathrectangle{\pgfqpoint{0.338326in}{0.377767in}}{\pgfqpoint{2.257694in}{1.645262in}}%
\pgfusepath{clip}%
\pgfsetbuttcap%
\pgfsetroundjoin%
\pgfsetlinewidth{1.505625pt}%
\definecolor{currentstroke}{rgb}{0.000000,0.000000,0.000000}%
\pgfsetstrokecolor{currentstroke}%
\pgfsetdash{}{0pt}%
\pgfpathmoveto{\pgfqpoint{2.384058in}{1.465079in}}%
\pgfpathlineto{\pgfqpoint{2.384058in}{1.729761in}}%
\pgfusepath{stroke}%
\end{pgfscope}%
\begin{pgfscope}%
\pgfpathrectangle{\pgfqpoint{0.338326in}{0.377767in}}{\pgfqpoint{2.257694in}{1.645262in}}%
\pgfusepath{clip}%
\pgfsetbuttcap%
\pgfsetroundjoin%
\definecolor{currentfill}{rgb}{0.247059,0.564706,0.854902}%
\pgfsetfillcolor{currentfill}%
\pgfsetlinewidth{1.003750pt}%
\definecolor{currentstroke}{rgb}{0.247059,0.564706,0.854902}%
\pgfsetstrokecolor{currentstroke}%
\pgfsetdash{}{0pt}%
\pgfsys@defobject{currentmarker}{\pgfqpoint{-0.023821in}{-0.023821in}}{\pgfqpoint{0.023821in}{0.023821in}}{%
\pgfpathmoveto{\pgfqpoint{0.000000in}{-0.023821in}}%
\pgfpathcurveto{\pgfqpoint{0.006317in}{-0.023821in}}{\pgfqpoint{0.012377in}{-0.021311in}}{\pgfqpoint{0.016844in}{-0.016844in}}%
\pgfpathcurveto{\pgfqpoint{0.021311in}{-0.012377in}}{\pgfqpoint{0.023821in}{-0.006317in}}{\pgfqpoint{0.023821in}{0.000000in}}%
\pgfpathcurveto{\pgfqpoint{0.023821in}{0.006317in}}{\pgfqpoint{0.021311in}{0.012377in}}{\pgfqpoint{0.016844in}{0.016844in}}%
\pgfpathcurveto{\pgfqpoint{0.012377in}{0.021311in}}{\pgfqpoint{0.006317in}{0.023821in}}{\pgfqpoint{0.000000in}{0.023821in}}%
\pgfpathcurveto{\pgfqpoint{-0.006317in}{0.023821in}}{\pgfqpoint{-0.012377in}{0.021311in}}{\pgfqpoint{-0.016844in}{0.016844in}}%
\pgfpathcurveto{\pgfqpoint{-0.021311in}{0.012377in}}{\pgfqpoint{-0.023821in}{0.006317in}}{\pgfqpoint{-0.023821in}{0.000000in}}%
\pgfpathcurveto{\pgfqpoint{-0.023821in}{-0.006317in}}{\pgfqpoint{-0.021311in}{-0.012377in}}{\pgfqpoint{-0.016844in}{-0.016844in}}%
\pgfpathcurveto{\pgfqpoint{-0.012377in}{-0.021311in}}{\pgfqpoint{-0.006317in}{-0.023821in}}{\pgfqpoint{0.000000in}{-0.023821in}}%
\pgfpathlineto{\pgfqpoint{0.000000in}{-0.023821in}}%
\pgfpathclose%
\pgfusepath{stroke,fill}%
}%
\begin{pgfscope}%
\pgfsys@transformshift{0.550288in}{0.671034in}%
\pgfsys@useobject{currentmarker}{}%
\end{pgfscope}%
\begin{pgfscope}%
\pgfsys@transformshift{0.550288in}{1.465079in}%
\pgfsys@useobject{currentmarker}{}%
\end{pgfscope}%
\begin{pgfscope}%
\pgfsys@transformshift{0.779510in}{1.068057in}%
\pgfsys@useobject{currentmarker}{}%
\end{pgfscope}%
\begin{pgfscope}%
\pgfsys@transformshift{1.008731in}{0.671034in}%
\pgfsys@useobject{currentmarker}{}%
\end{pgfscope}%
\begin{pgfscope}%
\pgfsys@transformshift{0.779510in}{1.862102in}%
\pgfsys@useobject{currentmarker}{}%
\end{pgfscope}%
\begin{pgfscope}%
\pgfsys@transformshift{1.008731in}{1.465079in}%
\pgfsys@useobject{currentmarker}{}%
\end{pgfscope}%
\begin{pgfscope}%
\pgfsys@transformshift{1.237952in}{1.068057in}%
\pgfsys@useobject{currentmarker}{}%
\end{pgfscope}%
\begin{pgfscope}%
\pgfsys@transformshift{1.467173in}{0.671034in}%
\pgfsys@useobject{currentmarker}{}%
\end{pgfscope}%
\begin{pgfscope}%
\pgfsys@transformshift{1.237952in}{1.862102in}%
\pgfsys@useobject{currentmarker}{}%
\end{pgfscope}%
\begin{pgfscope}%
\pgfsys@transformshift{1.467173in}{1.465079in}%
\pgfsys@useobject{currentmarker}{}%
\end{pgfscope}%
\begin{pgfscope}%
\pgfsys@transformshift{1.696394in}{1.068057in}%
\pgfsys@useobject{currentmarker}{}%
\end{pgfscope}%
\begin{pgfscope}%
\pgfsys@transformshift{1.925615in}{0.671034in}%
\pgfsys@useobject{currentmarker}{}%
\end{pgfscope}%
\begin{pgfscope}%
\pgfsys@transformshift{1.696394in}{1.862102in}%
\pgfsys@useobject{currentmarker}{}%
\end{pgfscope}%
\begin{pgfscope}%
\pgfsys@transformshift{1.925615in}{1.465079in}%
\pgfsys@useobject{currentmarker}{}%
\end{pgfscope}%
\begin{pgfscope}%
\pgfsys@transformshift{2.154837in}{1.068057in}%
\pgfsys@useobject{currentmarker}{}%
\end{pgfscope}%
\begin{pgfscope}%
\pgfsys@transformshift{2.384058in}{0.671034in}%
\pgfsys@useobject{currentmarker}{}%
\end{pgfscope}%
\begin{pgfscope}%
\pgfsys@transformshift{2.154837in}{1.862102in}%
\pgfsys@useobject{currentmarker}{}%
\end{pgfscope}%
\begin{pgfscope}%
\pgfsys@transformshift{2.384058in}{1.465079in}%
\pgfsys@useobject{currentmarker}{}%
\end{pgfscope}%
\end{pgfscope}%
\begin{pgfscope}%
\pgfpathrectangle{\pgfqpoint{0.338326in}{0.377767in}}{\pgfqpoint{2.257694in}{1.645262in}}%
\pgfusepath{clip}%
\pgfsetbuttcap%
\pgfsetroundjoin%
\definecolor{currentfill}{rgb}{1.000000,0.662745,0.054902}%
\pgfsetfillcolor{currentfill}%
\pgfsetlinewidth{1.003750pt}%
\definecolor{currentstroke}{rgb}{1.000000,0.662745,0.054902}%
\pgfsetstrokecolor{currentstroke}%
\pgfsetdash{}{0pt}%
\pgfsys@defobject{currentmarker}{\pgfqpoint{-0.023821in}{-0.023821in}}{\pgfqpoint{0.023821in}{0.023821in}}{%
\pgfpathmoveto{\pgfqpoint{0.000000in}{-0.023821in}}%
\pgfpathcurveto{\pgfqpoint{0.006317in}{-0.023821in}}{\pgfqpoint{0.012377in}{-0.021311in}}{\pgfqpoint{0.016844in}{-0.016844in}}%
\pgfpathcurveto{\pgfqpoint{0.021311in}{-0.012377in}}{\pgfqpoint{0.023821in}{-0.006317in}}{\pgfqpoint{0.023821in}{0.000000in}}%
\pgfpathcurveto{\pgfqpoint{0.023821in}{0.006317in}}{\pgfqpoint{0.021311in}{0.012377in}}{\pgfqpoint{0.016844in}{0.016844in}}%
\pgfpathcurveto{\pgfqpoint{0.012377in}{0.021311in}}{\pgfqpoint{0.006317in}{0.023821in}}{\pgfqpoint{0.000000in}{0.023821in}}%
\pgfpathcurveto{\pgfqpoint{-0.006317in}{0.023821in}}{\pgfqpoint{-0.012377in}{0.021311in}}{\pgfqpoint{-0.016844in}{0.016844in}}%
\pgfpathcurveto{\pgfqpoint{-0.021311in}{0.012377in}}{\pgfqpoint{-0.023821in}{0.006317in}}{\pgfqpoint{-0.023821in}{0.000000in}}%
\pgfpathcurveto{\pgfqpoint{-0.023821in}{-0.006317in}}{\pgfqpoint{-0.021311in}{-0.012377in}}{\pgfqpoint{-0.016844in}{-0.016844in}}%
\pgfpathcurveto{\pgfqpoint{-0.012377in}{-0.021311in}}{\pgfqpoint{-0.006317in}{-0.023821in}}{\pgfqpoint{0.000000in}{-0.023821in}}%
\pgfpathlineto{\pgfqpoint{0.000000in}{-0.023821in}}%
\pgfpathclose%
\pgfusepath{stroke,fill}%
}%
\begin{pgfscope}%
\pgfsys@transformshift{0.550288in}{0.935716in}%
\pgfsys@useobject{currentmarker}{}%
\end{pgfscope}%
\begin{pgfscope}%
\pgfsys@transformshift{0.779510in}{0.538693in}%
\pgfsys@useobject{currentmarker}{}%
\end{pgfscope}%
\begin{pgfscope}%
\pgfsys@transformshift{0.550288in}{1.729761in}%
\pgfsys@useobject{currentmarker}{}%
\end{pgfscope}%
\begin{pgfscope}%
\pgfsys@transformshift{0.779510in}{1.332739in}%
\pgfsys@useobject{currentmarker}{}%
\end{pgfscope}%
\begin{pgfscope}%
\pgfsys@transformshift{1.008731in}{0.935716in}%
\pgfsys@useobject{currentmarker}{}%
\end{pgfscope}%
\begin{pgfscope}%
\pgfsys@transformshift{1.237952in}{0.538693in}%
\pgfsys@useobject{currentmarker}{}%
\end{pgfscope}%
\begin{pgfscope}%
\pgfsys@transformshift{1.008731in}{1.729761in}%
\pgfsys@useobject{currentmarker}{}%
\end{pgfscope}%
\begin{pgfscope}%
\pgfsys@transformshift{1.237952in}{1.332739in}%
\pgfsys@useobject{currentmarker}{}%
\end{pgfscope}%
\begin{pgfscope}%
\pgfsys@transformshift{1.467173in}{0.935716in}%
\pgfsys@useobject{currentmarker}{}%
\end{pgfscope}%
\begin{pgfscope}%
\pgfsys@transformshift{1.696394in}{0.538693in}%
\pgfsys@useobject{currentmarker}{}%
\end{pgfscope}%
\begin{pgfscope}%
\pgfsys@transformshift{1.467173in}{1.729761in}%
\pgfsys@useobject{currentmarker}{}%
\end{pgfscope}%
\begin{pgfscope}%
\pgfsys@transformshift{1.696394in}{1.332739in}%
\pgfsys@useobject{currentmarker}{}%
\end{pgfscope}%
\begin{pgfscope}%
\pgfsys@transformshift{1.925615in}{0.935716in}%
\pgfsys@useobject{currentmarker}{}%
\end{pgfscope}%
\begin{pgfscope}%
\pgfsys@transformshift{2.154837in}{0.538693in}%
\pgfsys@useobject{currentmarker}{}%
\end{pgfscope}%
\begin{pgfscope}%
\pgfsys@transformshift{1.925615in}{1.729761in}%
\pgfsys@useobject{currentmarker}{}%
\end{pgfscope}%
\begin{pgfscope}%
\pgfsys@transformshift{2.154837in}{1.332739in}%
\pgfsys@useobject{currentmarker}{}%
\end{pgfscope}%
\begin{pgfscope}%
\pgfsys@transformshift{2.384058in}{0.935716in}%
\pgfsys@useobject{currentmarker}{}%
\end{pgfscope}%
\begin{pgfscope}%
\pgfsys@transformshift{2.384058in}{1.729761in}%
\pgfsys@useobject{currentmarker}{}%
\end{pgfscope}%
\end{pgfscope}%
\begin{pgfscope}%
\pgfpathrectangle{\pgfqpoint{0.338326in}{0.377767in}}{\pgfqpoint{2.257694in}{1.645262in}}%
\pgfusepath{clip}%
\pgfsetbuttcap%
\pgfsetroundjoin%
\definecolor{currentfill}{rgb}{0.247059,0.564706,0.854902}%
\pgfsetfillcolor{currentfill}%
\pgfsetlinewidth{1.003750pt}%
\definecolor{currentstroke}{rgb}{0.247059,0.564706,0.854902}%
\pgfsetstrokecolor{currentstroke}%
\pgfsetdash{}{0pt}%
\pgfsys@defobject{currentmarker}{\pgfqpoint{0.000000in}{0.000000in}}{\pgfqpoint{0.000000in}{0.000000in}}{%
\pgfpathmoveto{\pgfqpoint{0.000000in}{0.000000in}}%
\pgfpathcurveto{\pgfqpoint{0.000000in}{0.000000in}}{\pgfqpoint{0.000000in}{0.000000in}}{\pgfqpoint{0.000000in}{0.000000in}}%
\pgfpathcurveto{\pgfqpoint{0.000000in}{0.000000in}}{\pgfqpoint{0.000000in}{0.000000in}}{\pgfqpoint{0.000000in}{0.000000in}}%
\pgfpathcurveto{\pgfqpoint{0.000000in}{0.000000in}}{\pgfqpoint{0.000000in}{0.000000in}}{\pgfqpoint{0.000000in}{0.000000in}}%
\pgfpathcurveto{\pgfqpoint{0.000000in}{0.000000in}}{\pgfqpoint{0.000000in}{0.000000in}}{\pgfqpoint{0.000000in}{0.000000in}}%
\pgfpathcurveto{\pgfqpoint{0.000000in}{0.000000in}}{\pgfqpoint{0.000000in}{0.000000in}}{\pgfqpoint{0.000000in}{0.000000in}}%
\pgfpathcurveto{\pgfqpoint{0.000000in}{0.000000in}}{\pgfqpoint{0.000000in}{0.000000in}}{\pgfqpoint{0.000000in}{0.000000in}}%
\pgfpathcurveto{\pgfqpoint{0.000000in}{0.000000in}}{\pgfqpoint{0.000000in}{0.000000in}}{\pgfqpoint{0.000000in}{0.000000in}}%
\pgfpathcurveto{\pgfqpoint{0.000000in}{0.000000in}}{\pgfqpoint{0.000000in}{0.000000in}}{\pgfqpoint{0.000000in}{0.000000in}}%
\pgfpathlineto{\pgfqpoint{0.000000in}{0.000000in}}%
\pgfpathclose%
\pgfusepath{stroke,fill}%
}%
\begin{pgfscope}%
\pgfsys@transformshift{1.696394in}{1.597420in}%
\pgfsys@useobject{currentmarker}{}%
\end{pgfscope}%
\end{pgfscope}%
\begin{pgfscope}%
\pgfpathrectangle{\pgfqpoint{0.338326in}{0.377767in}}{\pgfqpoint{2.257694in}{1.645262in}}%
\pgfusepath{clip}%
\pgfsetbuttcap%
\pgfsetroundjoin%
\definecolor{currentfill}{rgb}{0.000000,0.000000,0.000000}%
\pgfsetfillcolor{currentfill}%
\pgfsetlinewidth{0.000000pt}%
\definecolor{currentstroke}{rgb}{0.000000,0.000000,0.000000}%
\pgfsetstrokecolor{currentstroke}%
\pgfsetdash{}{0pt}%
\pgfpathmoveto{\pgfqpoint{1.459841in}{1.204631in}}%
\pgfpathlineto{\pgfqpoint{1.650964in}{1.535665in}}%
\pgfpathlineto{\pgfqpoint{1.632066in}{1.536799in}}%
\pgfpathlineto{\pgfqpoint{1.696394in}{1.597420in}}%
\pgfpathlineto{\pgfqpoint{1.676059in}{1.511400in}}%
\pgfpathlineto{\pgfqpoint{1.665628in}{1.527198in}}%
\pgfpathlineto{\pgfqpoint{1.474505in}{1.196164in}}%
\pgfpathlineto{\pgfqpoint{1.459841in}{1.204631in}}%
\pgfusepath{fill}%
\end{pgfscope}%
\begin{pgfscope}%
\pgfpathrectangle{\pgfqpoint{0.338326in}{0.377767in}}{\pgfqpoint{2.257694in}{1.645262in}}%
\pgfusepath{clip}%
\pgfsetbuttcap%
\pgfsetroundjoin%
\definecolor{currentfill}{rgb}{1.000000,0.662745,0.054902}%
\pgfsetfillcolor{currentfill}%
\pgfsetlinewidth{1.003750pt}%
\definecolor{currentstroke}{rgb}{1.000000,0.662745,0.054902}%
\pgfsetstrokecolor{currentstroke}%
\pgfsetdash{}{0pt}%
\pgfsys@defobject{currentmarker}{\pgfqpoint{0.000000in}{0.000000in}}{\pgfqpoint{0.000000in}{0.000000in}}{%
\pgfpathmoveto{\pgfqpoint{0.000000in}{0.000000in}}%
\pgfpathcurveto{\pgfqpoint{0.000000in}{0.000000in}}{\pgfqpoint{0.000000in}{0.000000in}}{\pgfqpoint{0.000000in}{0.000000in}}%
\pgfpathcurveto{\pgfqpoint{0.000000in}{0.000000in}}{\pgfqpoint{0.000000in}{0.000000in}}{\pgfqpoint{0.000000in}{0.000000in}}%
\pgfpathcurveto{\pgfqpoint{0.000000in}{0.000000in}}{\pgfqpoint{0.000000in}{0.000000in}}{\pgfqpoint{0.000000in}{0.000000in}}%
\pgfpathcurveto{\pgfqpoint{0.000000in}{0.000000in}}{\pgfqpoint{0.000000in}{0.000000in}}{\pgfqpoint{0.000000in}{0.000000in}}%
\pgfpathcurveto{\pgfqpoint{0.000000in}{0.000000in}}{\pgfqpoint{0.000000in}{0.000000in}}{\pgfqpoint{0.000000in}{0.000000in}}%
\pgfpathcurveto{\pgfqpoint{0.000000in}{0.000000in}}{\pgfqpoint{0.000000in}{0.000000in}}{\pgfqpoint{0.000000in}{0.000000in}}%
\pgfpathcurveto{\pgfqpoint{0.000000in}{0.000000in}}{\pgfqpoint{0.000000in}{0.000000in}}{\pgfqpoint{0.000000in}{0.000000in}}%
\pgfpathcurveto{\pgfqpoint{0.000000in}{0.000000in}}{\pgfqpoint{0.000000in}{0.000000in}}{\pgfqpoint{0.000000in}{0.000000in}}%
\pgfpathlineto{\pgfqpoint{0.000000in}{0.000000in}}%
\pgfpathclose%
\pgfusepath{stroke,fill}%
}%
\begin{pgfscope}%
\pgfsys@transformshift{1.696394in}{0.803375in}%
\pgfsys@useobject{currentmarker}{}%
\end{pgfscope}%
\end{pgfscope}%
\begin{pgfscope}%
\pgfpathrectangle{\pgfqpoint{0.338326in}{0.377767in}}{\pgfqpoint{2.257694in}{1.645262in}}%
\pgfusepath{clip}%
\pgfsetbuttcap%
\pgfsetroundjoin%
\definecolor{currentfill}{rgb}{0.000000,0.000000,0.000000}%
\pgfsetfillcolor{currentfill}%
\pgfsetlinewidth{0.000000pt}%
\definecolor{currentstroke}{rgb}{0.000000,0.000000,0.000000}%
\pgfsetstrokecolor{currentstroke}%
\pgfsetdash{}{0pt}%
\pgfpathmoveto{\pgfqpoint{1.474505in}{1.204631in}}%
\pgfpathlineto{\pgfqpoint{1.665628in}{0.873597in}}%
\pgfpathlineto{\pgfqpoint{1.676059in}{0.889395in}}%
\pgfpathlineto{\pgfqpoint{1.696394in}{0.803375in}}%
\pgfpathlineto{\pgfqpoint{1.632066in}{0.863996in}}%
\pgfpathlineto{\pgfqpoint{1.650964in}{0.865130in}}%
\pgfpathlineto{\pgfqpoint{1.459841in}{1.196164in}}%
\pgfpathlineto{\pgfqpoint{1.474505in}{1.204631in}}%
\pgfusepath{fill}%
\end{pgfscope}%
\begin{pgfscope}%
\pgfpathrectangle{\pgfqpoint{0.338326in}{0.377767in}}{\pgfqpoint{2.257694in}{1.645262in}}%
\pgfusepath{clip}%
\pgfsetbuttcap%
\pgfsetroundjoin%
\definecolor{currentfill}{rgb}{0.741176,0.121569,0.003922}%
\pgfsetfillcolor{currentfill}%
\pgfsetlinewidth{1.003750pt}%
\definecolor{currentstroke}{rgb}{0.741176,0.121569,0.003922}%
\pgfsetstrokecolor{currentstroke}%
\pgfsetdash{}{0pt}%
\pgfsys@defobject{currentmarker}{\pgfqpoint{0.000000in}{0.000000in}}{\pgfqpoint{0.000000in}{0.000000in}}{%
\pgfpathmoveto{\pgfqpoint{0.000000in}{0.000000in}}%
\pgfpathcurveto{\pgfqpoint{0.000000in}{0.000000in}}{\pgfqpoint{0.000000in}{0.000000in}}{\pgfqpoint{0.000000in}{0.000000in}}%
\pgfpathcurveto{\pgfqpoint{0.000000in}{0.000000in}}{\pgfqpoint{0.000000in}{0.000000in}}{\pgfqpoint{0.000000in}{0.000000in}}%
\pgfpathcurveto{\pgfqpoint{0.000000in}{0.000000in}}{\pgfqpoint{0.000000in}{0.000000in}}{\pgfqpoint{0.000000in}{0.000000in}}%
\pgfpathcurveto{\pgfqpoint{0.000000in}{0.000000in}}{\pgfqpoint{0.000000in}{0.000000in}}{\pgfqpoint{0.000000in}{0.000000in}}%
\pgfpathcurveto{\pgfqpoint{0.000000in}{0.000000in}}{\pgfqpoint{0.000000in}{0.000000in}}{\pgfqpoint{0.000000in}{0.000000in}}%
\pgfpathcurveto{\pgfqpoint{0.000000in}{0.000000in}}{\pgfqpoint{0.000000in}{0.000000in}}{\pgfqpoint{0.000000in}{0.000000in}}%
\pgfpathcurveto{\pgfqpoint{0.000000in}{0.000000in}}{\pgfqpoint{0.000000in}{0.000000in}}{\pgfqpoint{0.000000in}{0.000000in}}%
\pgfpathcurveto{\pgfqpoint{0.000000in}{0.000000in}}{\pgfqpoint{0.000000in}{0.000000in}}{\pgfqpoint{0.000000in}{0.000000in}}%
\pgfpathlineto{\pgfqpoint{0.000000in}{0.000000in}}%
\pgfpathclose%
\pgfusepath{stroke,fill}%
}%
\begin{pgfscope}%
\pgfsys@transformshift{1.696394in}{1.332739in}%
\pgfsys@useobject{currentmarker}{}%
\end{pgfscope}%
\end{pgfscope}%
\begin{pgfscope}%
\pgfpathrectangle{\pgfqpoint{0.338326in}{0.377767in}}{\pgfqpoint{2.257694in}{1.645262in}}%
\pgfusepath{clip}%
\pgfsetbuttcap%
\pgfsetroundjoin%
\definecolor{currentfill}{rgb}{0.000000,0.000000,0.000000}%
\pgfsetfillcolor{currentfill}%
\pgfsetlinewidth{0.000000pt}%
\definecolor{currentstroke}{rgb}{0.000000,0.000000,0.000000}%
\pgfsetstrokecolor{currentstroke}%
\pgfsetdash{}{0pt}%
\pgfpathmoveto{\pgfqpoint{1.687928in}{1.068057in}}%
\pgfpathlineto{\pgfqpoint{1.687928in}{1.256541in}}%
\pgfpathlineto{\pgfqpoint{1.670995in}{1.248075in}}%
\pgfpathlineto{\pgfqpoint{1.696394in}{1.332739in}}%
\pgfpathlineto{\pgfqpoint{1.721793in}{1.248075in}}%
\pgfpathlineto{\pgfqpoint{1.704861in}{1.256541in}}%
\pgfpathlineto{\pgfqpoint{1.704861in}{1.068057in}}%
\pgfpathlineto{\pgfqpoint{1.687928in}{1.068057in}}%
\pgfusepath{fill}%
\end{pgfscope}%
\begin{pgfscope}%
\pgfpathrectangle{\pgfqpoint{0.338326in}{0.377767in}}{\pgfqpoint{2.257694in}{1.645262in}}%
\pgfusepath{clip}%
\pgfsetbuttcap%
\pgfsetroundjoin%
\pgfsetlinewidth{1.505625pt}%
\definecolor{currentstroke}{rgb}{0.000000,0.000000,0.000000}%
\pgfsetstrokecolor{currentstroke}%
\pgfsetdash{{5.550000pt}{2.400000pt}}{0.000000pt}%
\pgfpathmoveto{\pgfqpoint{1.696394in}{1.597420in}}%
\pgfpathlineto{\pgfqpoint{1.925615in}{1.200398in}}%
\pgfusepath{stroke}%
\end{pgfscope}%
\begin{pgfscope}%
\pgfpathrectangle{\pgfqpoint{0.338326in}{0.377767in}}{\pgfqpoint{2.257694in}{1.645262in}}%
\pgfusepath{clip}%
\pgfsetbuttcap%
\pgfsetroundjoin%
\pgfsetlinewidth{1.505625pt}%
\definecolor{currentstroke}{rgb}{0.000000,0.000000,0.000000}%
\pgfsetstrokecolor{currentstroke}%
\pgfsetdash{{5.550000pt}{2.400000pt}}{0.000000pt}%
\pgfpathmoveto{\pgfqpoint{1.696394in}{0.803375in}}%
\pgfpathlineto{\pgfqpoint{1.925615in}{1.200398in}}%
\pgfusepath{stroke}%
\end{pgfscope}%
\begin{pgfscope}%
\definecolor{textcolor}{rgb}{0.000000,0.000000,0.000000}%
\pgfsetstrokecolor{textcolor}%
\pgfsetfillcolor{textcolor}%
\pgftext[x=1.503143in,y=1.518016in,left,base]{\color{textcolor}{\sffamily\fontsize{9.163000}{10.995600}\selectfont\catcode`\^=\active\def^{\ifmmode\sp\else\^{}\fi}\catcode`\%=\active\def%{\%}$\mathbf{a}_1$}}%
\end{pgfscope}%
\begin{pgfscope}%
\definecolor{textcolor}{rgb}{0.000000,0.000000,0.000000}%
\pgfsetstrokecolor{textcolor}%
\pgfsetfillcolor{textcolor}%
\pgftext[x=1.503143in,y=0.803375in,left,base]{\color{textcolor}{\sffamily\fontsize{9.163000}{10.995600}\selectfont\catcode`\^=\active\def^{\ifmmode\sp\else\^{}\fi}\catcode`\%=\active\def%{\%}$\mathbf{a}_2$}}%
\end{pgfscope}%
\begin{pgfscope}%
\pgfsetbuttcap%
\pgfsetmiterjoin%
\definecolor{currentfill}{rgb}{1.000000,1.000000,1.000000}%
\pgfsetfillcolor{currentfill}%
\pgfsetfillopacity{0.800000}%
\pgfsetlinewidth{1.003750pt}%
\definecolor{currentstroke}{rgb}{0.800000,0.800000,0.800000}%
\pgfsetstrokecolor{currentstroke}%
\pgfsetstrokeopacity{0.800000}%
\pgfsetdash{}{0pt}%
\pgfpathmoveto{\pgfqpoint{0.445270in}{0.454156in}}%
\pgfpathlineto{\pgfqpoint{1.033465in}{0.454156in}}%
\pgfpathquadraticcurveto{\pgfqpoint{1.064020in}{0.454156in}}{\pgfqpoint{1.064020in}{0.484711in}}%
\pgfpathlineto{\pgfqpoint{1.064020in}{0.895378in}}%
\pgfpathquadraticcurveto{\pgfqpoint{1.064020in}{0.925933in}}{\pgfqpoint{1.033465in}{0.925933in}}%
\pgfpathlineto{\pgfqpoint{0.445270in}{0.925933in}}%
\pgfpathquadraticcurveto{\pgfqpoint{0.414715in}{0.925933in}}{\pgfqpoint{0.414715in}{0.895378in}}%
\pgfpathlineto{\pgfqpoint{0.414715in}{0.484711in}}%
\pgfpathquadraticcurveto{\pgfqpoint{0.414715in}{0.454156in}}{\pgfqpoint{0.445270in}{0.454156in}}%
\pgfpathlineto{\pgfqpoint{0.445270in}{0.454156in}}%
\pgfpathclose%
\pgfusepath{stroke,fill}%
\end{pgfscope}%
\begin{pgfscope}%
\pgfsetrectcap%
\pgfsetroundjoin%
\pgfsetlinewidth{0.000000pt}%
\definecolor{currentstroke}{rgb}{0.247059,0.564706,0.854902}%
\pgfsetstrokecolor{currentstroke}%
\pgfsetdash{}{0pt}%
\pgfpathmoveto{\pgfqpoint{0.475826in}{0.811350in}}%
\pgfpathlineto{\pgfqpoint{0.628604in}{0.811350in}}%
\pgfpathlineto{\pgfqpoint{0.781381in}{0.811350in}}%
\pgfusepath{}%
\end{pgfscope}%
\begin{pgfscope}%
\pgfsetbuttcap%
\pgfsetroundjoin%
\definecolor{currentfill}{rgb}{0.247059,0.564706,0.854902}%
\pgfsetfillcolor{currentfill}%
\pgfsetlinewidth{1.003750pt}%
\definecolor{currentstroke}{rgb}{0.247059,0.564706,0.854902}%
\pgfsetstrokecolor{currentstroke}%
\pgfsetdash{}{0pt}%
\pgfsys@defobject{currentmarker}{\pgfqpoint{-0.069444in}{-0.069444in}}{\pgfqpoint{0.069444in}{0.069444in}}{%
\pgfpathmoveto{\pgfqpoint{0.000000in}{-0.069444in}}%
\pgfpathcurveto{\pgfqpoint{0.018417in}{-0.069444in}}{\pgfqpoint{0.036082in}{-0.062127in}}{\pgfqpoint{0.049105in}{-0.049105in}}%
\pgfpathcurveto{\pgfqpoint{0.062127in}{-0.036082in}}{\pgfqpoint{0.069444in}{-0.018417in}}{\pgfqpoint{0.069444in}{0.000000in}}%
\pgfpathcurveto{\pgfqpoint{0.069444in}{0.018417in}}{\pgfqpoint{0.062127in}{0.036082in}}{\pgfqpoint{0.049105in}{0.049105in}}%
\pgfpathcurveto{\pgfqpoint{0.036082in}{0.062127in}}{\pgfqpoint{0.018417in}{0.069444in}}{\pgfqpoint{0.000000in}{0.069444in}}%
\pgfpathcurveto{\pgfqpoint{-0.018417in}{0.069444in}}{\pgfqpoint{-0.036082in}{0.062127in}}{\pgfqpoint{-0.049105in}{0.049105in}}%
\pgfpathcurveto{\pgfqpoint{-0.062127in}{0.036082in}}{\pgfqpoint{-0.069444in}{0.018417in}}{\pgfqpoint{-0.069444in}{0.000000in}}%
\pgfpathcurveto{\pgfqpoint{-0.069444in}{-0.018417in}}{\pgfqpoint{-0.062127in}{-0.036082in}}{\pgfqpoint{-0.049105in}{-0.049105in}}%
\pgfpathcurveto{\pgfqpoint{-0.036082in}{-0.062127in}}{\pgfqpoint{-0.018417in}{-0.069444in}}{\pgfqpoint{0.000000in}{-0.069444in}}%
\pgfpathlineto{\pgfqpoint{0.000000in}{-0.069444in}}%
\pgfpathclose%
\pgfusepath{stroke,fill}%
}%
\begin{pgfscope}%
\pgfsys@transformshift{0.628604in}{0.811350in}%
\pgfsys@useobject{currentmarker}{}%
\end{pgfscope}%
\end{pgfscope}%
\begin{pgfscope}%
\definecolor{textcolor}{rgb}{0.000000,0.000000,0.000000}%
\pgfsetstrokecolor{textcolor}%
\pgfsetfillcolor{textcolor}%
\pgftext[x=0.903604in,y=0.757878in,left,base]{\color{textcolor}{\sffamily\fontsize{11.000000}{13.200000}\selectfont\catcode`\^=\active\def^{\ifmmode\sp\else\^{}\fi}\catcode`\%=\active\def%{\%}A}}%
\end{pgfscope}%
\begin{pgfscope}%
\pgfsetrectcap%
\pgfsetroundjoin%
\pgfsetlinewidth{0.000000pt}%
\definecolor{currentstroke}{rgb}{1.000000,0.662745,0.054902}%
\pgfsetstrokecolor{currentstroke}%
\pgfsetdash{}{0pt}%
\pgfpathmoveto{\pgfqpoint{0.475826in}{0.598378in}}%
\pgfpathlineto{\pgfqpoint{0.628604in}{0.598378in}}%
\pgfpathlineto{\pgfqpoint{0.781381in}{0.598378in}}%
\pgfusepath{}%
\end{pgfscope}%
\begin{pgfscope}%
\pgfsetbuttcap%
\pgfsetroundjoin%
\definecolor{currentfill}{rgb}{1.000000,0.662745,0.054902}%
\pgfsetfillcolor{currentfill}%
\pgfsetlinewidth{1.003750pt}%
\definecolor{currentstroke}{rgb}{1.000000,0.662745,0.054902}%
\pgfsetstrokecolor{currentstroke}%
\pgfsetdash{}{0pt}%
\pgfsys@defobject{currentmarker}{\pgfqpoint{-0.069444in}{-0.069444in}}{\pgfqpoint{0.069444in}{0.069444in}}{%
\pgfpathmoveto{\pgfqpoint{0.000000in}{-0.069444in}}%
\pgfpathcurveto{\pgfqpoint{0.018417in}{-0.069444in}}{\pgfqpoint{0.036082in}{-0.062127in}}{\pgfqpoint{0.049105in}{-0.049105in}}%
\pgfpathcurveto{\pgfqpoint{0.062127in}{-0.036082in}}{\pgfqpoint{0.069444in}{-0.018417in}}{\pgfqpoint{0.069444in}{0.000000in}}%
\pgfpathcurveto{\pgfqpoint{0.069444in}{0.018417in}}{\pgfqpoint{0.062127in}{0.036082in}}{\pgfqpoint{0.049105in}{0.049105in}}%
\pgfpathcurveto{\pgfqpoint{0.036082in}{0.062127in}}{\pgfqpoint{0.018417in}{0.069444in}}{\pgfqpoint{0.000000in}{0.069444in}}%
\pgfpathcurveto{\pgfqpoint{-0.018417in}{0.069444in}}{\pgfqpoint{-0.036082in}{0.062127in}}{\pgfqpoint{-0.049105in}{0.049105in}}%
\pgfpathcurveto{\pgfqpoint{-0.062127in}{0.036082in}}{\pgfqpoint{-0.069444in}{0.018417in}}{\pgfqpoint{-0.069444in}{0.000000in}}%
\pgfpathcurveto{\pgfqpoint{-0.069444in}{-0.018417in}}{\pgfqpoint{-0.062127in}{-0.036082in}}{\pgfqpoint{-0.049105in}{-0.049105in}}%
\pgfpathcurveto{\pgfqpoint{-0.036082in}{-0.062127in}}{\pgfqpoint{-0.018417in}{-0.069444in}}{\pgfqpoint{0.000000in}{-0.069444in}}%
\pgfpathlineto{\pgfqpoint{0.000000in}{-0.069444in}}%
\pgfpathclose%
\pgfusepath{stroke,fill}%
}%
\begin{pgfscope}%
\pgfsys@transformshift{0.628604in}{0.598378in}%
\pgfsys@useobject{currentmarker}{}%
\end{pgfscope}%
\end{pgfscope}%
\begin{pgfscope}%
\definecolor{textcolor}{rgb}{0.000000,0.000000,0.000000}%
\pgfsetstrokecolor{textcolor}%
\pgfsetfillcolor{textcolor}%
\pgftext[x=0.903604in,y=0.544906in,left,base]{\color{textcolor}{\sffamily\fontsize{11.000000}{13.200000}\selectfont\catcode`\^=\active\def^{\ifmmode\sp\else\^{}\fi}\catcode`\%=\active\def%{\%}B}}%
\end{pgfscope}%
\end{pgfpicture}%
\makeatother%
\endgroup%

	\end{subfigure}%
	\begin{subfigure}[t]{0.5\textwidth}
		\centering
		\caption{\hfill\null}\label{sfig:graphene Brillouin zone}
		%% Creator: Matplotlib, PGF backend
%%
%% To include the figure in your LaTeX document, write
%%   \input{<filename>.pgf}
%%
%% Make sure the required packages are loaded in your preamble
%%   \usepackage{pgf}
%%
%% Also ensure that all the required font packages are loaded; for instance,
%% the lmodern package is sometimes necessary when using math font.
%%   \usepackage{lmodern}
%%
%% Figures using additional raster images can only be included by \input if
%% they are in the same directory as the main LaTeX file. For loading figures
%% from other directories you can use the `import` package
%%   \usepackage{import}
%%
%% and then include the figures with
%%   \import{<path to file>}{<filename>.pgf}
%%
%% Matplotlib used the following preamble
%%   \def\mathdefault#1{#1}
%%   \everymath=\expandafter{\the\everymath\displaystyle}
%%   \IfFileExists{scrextend.sty}{
%%     \usepackage[fontsize=11.000000pt]{scrextend}
%%   }{
%%     \renewcommand{\normalsize}{\fontsize{11.000000}{13.200000}\selectfont}
%%     \normalsize
%%   }
%%   \usepackage{fontspec}\usepackage{unicode-math}\setmathfont{texgyrepagella-math.otf}\setmainfont{texgyrepagella-math}
%%   \makeatletter\@ifpackageloaded{underscore}{}{\usepackage[strings]{underscore}}\makeatother
%%
\begingroup%
\makeatletter%
\begin{pgfpicture}%
\pgfpathrectangle{\pgfpointorigin}{\pgfqpoint{2.584000in}{2.584000in}}%
\pgfusepath{use as bounding box, clip}%
\begin{pgfscope}%
\pgfsetbuttcap%
\pgfsetmiterjoin%
\definecolor{currentfill}{rgb}{1.000000,1.000000,1.000000}%
\pgfsetfillcolor{currentfill}%
\pgfsetlinewidth{0.000000pt}%
\definecolor{currentstroke}{rgb}{1.000000,1.000000,1.000000}%
\pgfsetstrokecolor{currentstroke}%
\pgfsetdash{}{0pt}%
\pgfpathmoveto{\pgfqpoint{0.000000in}{0.000000in}}%
\pgfpathlineto{\pgfqpoint{2.584000in}{0.000000in}}%
\pgfpathlineto{\pgfqpoint{2.584000in}{2.584000in}}%
\pgfpathlineto{\pgfqpoint{0.000000in}{2.584000in}}%
\pgfpathlineto{\pgfqpoint{0.000000in}{0.000000in}}%
\pgfpathclose%
\pgfusepath{fill}%
\end{pgfscope}%
\begin{pgfscope}%
\pgfsetbuttcap%
\pgfsetmiterjoin%
\definecolor{currentfill}{rgb}{1.000000,1.000000,1.000000}%
\pgfsetfillcolor{currentfill}%
\pgfsetlinewidth{0.000000pt}%
\definecolor{currentstroke}{rgb}{0.000000,0.000000,0.000000}%
\pgfsetstrokecolor{currentstroke}%
\pgfsetstrokeopacity{0.000000}%
\pgfsetdash{}{0pt}%
\pgfpathmoveto{\pgfqpoint{0.462743in}{0.284240in}}%
\pgfpathlineto{\pgfqpoint{2.185857in}{0.284240in}}%
\pgfpathlineto{\pgfqpoint{2.185857in}{2.273920in}}%
\pgfpathlineto{\pgfqpoint{0.462743in}{2.273920in}}%
\pgfpathlineto{\pgfqpoint{0.462743in}{0.284240in}}%
\pgfpathclose%
\pgfusepath{fill}%
\end{pgfscope}%
\begin{pgfscope}%
\pgfpathrectangle{\pgfqpoint{0.462743in}{0.284240in}}{\pgfqpoint{1.723113in}{1.989680in}}%
\pgfusepath{clip}%
\pgfsetbuttcap%
\pgfsetroundjoin%
\definecolor{currentfill}{rgb}{0.000000,0.000000,0.000000}%
\pgfsetfillcolor{currentfill}%
\pgfsetlinewidth{1.003750pt}%
\definecolor{currentstroke}{rgb}{0.000000,0.000000,0.000000}%
\pgfsetstrokecolor{currentstroke}%
\pgfsetdash{}{0pt}%
\pgfsys@defobject{currentmarker}{\pgfqpoint{-0.020833in}{-0.020833in}}{\pgfqpoint{0.020833in}{0.020833in}}{%
\pgfpathmoveto{\pgfqpoint{0.000000in}{-0.020833in}}%
\pgfpathcurveto{\pgfqpoint{0.005525in}{-0.020833in}}{\pgfqpoint{0.010825in}{-0.018638in}}{\pgfqpoint{0.014731in}{-0.014731in}}%
\pgfpathcurveto{\pgfqpoint{0.018638in}{-0.010825in}}{\pgfqpoint{0.020833in}{-0.005525in}}{\pgfqpoint{0.020833in}{0.000000in}}%
\pgfpathcurveto{\pgfqpoint{0.020833in}{0.005525in}}{\pgfqpoint{0.018638in}{0.010825in}}{\pgfqpoint{0.014731in}{0.014731in}}%
\pgfpathcurveto{\pgfqpoint{0.010825in}{0.018638in}}{\pgfqpoint{0.005525in}{0.020833in}}{\pgfqpoint{0.000000in}{0.020833in}}%
\pgfpathcurveto{\pgfqpoint{-0.005525in}{0.020833in}}{\pgfqpoint{-0.010825in}{0.018638in}}{\pgfqpoint{-0.014731in}{0.014731in}}%
\pgfpathcurveto{\pgfqpoint{-0.018638in}{0.010825in}}{\pgfqpoint{-0.020833in}{0.005525in}}{\pgfqpoint{-0.020833in}{0.000000in}}%
\pgfpathcurveto{\pgfqpoint{-0.020833in}{-0.005525in}}{\pgfqpoint{-0.018638in}{-0.010825in}}{\pgfqpoint{-0.014731in}{-0.014731in}}%
\pgfpathcurveto{\pgfqpoint{-0.010825in}{-0.018638in}}{\pgfqpoint{-0.005525in}{-0.020833in}}{\pgfqpoint{0.000000in}{-0.020833in}}%
\pgfpathlineto{\pgfqpoint{0.000000in}{-0.020833in}}%
\pgfpathclose%
\pgfusepath{stroke,fill}%
}%
\begin{pgfscope}%
\pgfsys@transformshift{1.324300in}{1.279080in}%
\pgfsys@useobject{currentmarker}{}%
\end{pgfscope}%
\begin{pgfscope}%
\pgfsys@transformshift{0.541067in}{0.826880in}%
\pgfsys@useobject{currentmarker}{}%
\end{pgfscope}%
\begin{pgfscope}%
\pgfsys@transformshift{1.324300in}{0.374680in}%
\pgfsys@useobject{currentmarker}{}%
\end{pgfscope}%
\begin{pgfscope}%
\pgfsys@transformshift{0.541067in}{1.731280in}%
\pgfsys@useobject{currentmarker}{}%
\end{pgfscope}%
\begin{pgfscope}%
\pgfsys@transformshift{2.107533in}{0.826880in}%
\pgfsys@useobject{currentmarker}{}%
\end{pgfscope}%
\begin{pgfscope}%
\pgfsys@transformshift{1.324300in}{2.183480in}%
\pgfsys@useobject{currentmarker}{}%
\end{pgfscope}%
\begin{pgfscope}%
\pgfsys@transformshift{2.107533in}{1.731280in}%
\pgfsys@useobject{currentmarker}{}%
\end{pgfscope}%
\end{pgfscope}%
\begin{pgfscope}%
\pgfpathrectangle{\pgfqpoint{0.462743in}{0.284240in}}{\pgfqpoint{1.723113in}{1.989680in}}%
\pgfusepath{clip}%
\pgfsetbuttcap%
\pgfsetroundjoin%
\definecolor{currentfill}{rgb}{0.000000,0.000000,0.000000}%
\pgfsetfillcolor{currentfill}%
\pgfsetlinewidth{1.003750pt}%
\definecolor{currentstroke}{rgb}{0.000000,0.000000,0.000000}%
\pgfsetstrokecolor{currentstroke}%
\pgfsetdash{}{0pt}%
\pgfsys@defobject{currentmarker}{\pgfqpoint{-0.020833in}{-0.020833in}}{\pgfqpoint{0.020833in}{0.020833in}}{%
\pgfpathmoveto{\pgfqpoint{0.000000in}{-0.020833in}}%
\pgfpathcurveto{\pgfqpoint{0.005525in}{-0.020833in}}{\pgfqpoint{0.010825in}{-0.018638in}}{\pgfqpoint{0.014731in}{-0.014731in}}%
\pgfpathcurveto{\pgfqpoint{0.018638in}{-0.010825in}}{\pgfqpoint{0.020833in}{-0.005525in}}{\pgfqpoint{0.020833in}{0.000000in}}%
\pgfpathcurveto{\pgfqpoint{0.020833in}{0.005525in}}{\pgfqpoint{0.018638in}{0.010825in}}{\pgfqpoint{0.014731in}{0.014731in}}%
\pgfpathcurveto{\pgfqpoint{0.010825in}{0.018638in}}{\pgfqpoint{0.005525in}{0.020833in}}{\pgfqpoint{0.000000in}{0.020833in}}%
\pgfpathcurveto{\pgfqpoint{-0.005525in}{0.020833in}}{\pgfqpoint{-0.010825in}{0.018638in}}{\pgfqpoint{-0.014731in}{0.014731in}}%
\pgfpathcurveto{\pgfqpoint{-0.018638in}{0.010825in}}{\pgfqpoint{-0.020833in}{0.005525in}}{\pgfqpoint{-0.020833in}{0.000000in}}%
\pgfpathcurveto{\pgfqpoint{-0.020833in}{-0.005525in}}{\pgfqpoint{-0.018638in}{-0.010825in}}{\pgfqpoint{-0.014731in}{-0.014731in}}%
\pgfpathcurveto{\pgfqpoint{-0.010825in}{-0.018638in}}{\pgfqpoint{-0.005525in}{-0.020833in}}{\pgfqpoint{0.000000in}{-0.020833in}}%
\pgfpathlineto{\pgfqpoint{0.000000in}{-0.020833in}}%
\pgfpathclose%
\pgfusepath{stroke,fill}%
}%
\begin{pgfscope}%
\pgfsys@transformshift{0.802144in}{1.279080in}%
\pgfsys@useobject{currentmarker}{}%
\end{pgfscope}%
\begin{pgfscope}%
\pgfsys@transformshift{1.063222in}{1.731280in}%
\pgfsys@useobject{currentmarker}{}%
\end{pgfscope}%
\begin{pgfscope}%
\pgfsys@transformshift{1.063222in}{0.826880in}%
\pgfsys@useobject{currentmarker}{}%
\end{pgfscope}%
\begin{pgfscope}%
\pgfsys@transformshift{1.585378in}{0.826880in}%
\pgfsys@useobject{currentmarker}{}%
\end{pgfscope}%
\begin{pgfscope}%
\pgfsys@transformshift{1.846456in}{1.279080in}%
\pgfsys@useobject{currentmarker}{}%
\end{pgfscope}%
\begin{pgfscope}%
\pgfsys@transformshift{1.585378in}{1.731280in}%
\pgfsys@useobject{currentmarker}{}%
\end{pgfscope}%
\end{pgfscope}%
\begin{pgfscope}%
\pgfpathrectangle{\pgfqpoint{0.462743in}{0.284240in}}{\pgfqpoint{1.723113in}{1.989680in}}%
\pgfusepath{clip}%
\pgfsetbuttcap%
\pgfsetroundjoin%
\definecolor{currentfill}{rgb}{0.247059,0.564706,0.854902}%
\pgfsetfillcolor{currentfill}%
\pgfsetlinewidth{1.003750pt}%
\definecolor{currentstroke}{rgb}{0.247059,0.564706,0.854902}%
\pgfsetstrokecolor{currentstroke}%
\pgfsetdash{}{0pt}%
\pgfsys@defobject{currentmarker}{\pgfqpoint{0.000000in}{0.000000in}}{\pgfqpoint{0.000000in}{0.000000in}}{%
\pgfpathmoveto{\pgfqpoint{0.000000in}{0.000000in}}%
\pgfpathcurveto{\pgfqpoint{0.000000in}{0.000000in}}{\pgfqpoint{0.000000in}{0.000000in}}{\pgfqpoint{0.000000in}{0.000000in}}%
\pgfpathcurveto{\pgfqpoint{0.000000in}{0.000000in}}{\pgfqpoint{0.000000in}{0.000000in}}{\pgfqpoint{0.000000in}{0.000000in}}%
\pgfpathcurveto{\pgfqpoint{0.000000in}{0.000000in}}{\pgfqpoint{0.000000in}{0.000000in}}{\pgfqpoint{0.000000in}{0.000000in}}%
\pgfpathcurveto{\pgfqpoint{0.000000in}{0.000000in}}{\pgfqpoint{0.000000in}{0.000000in}}{\pgfqpoint{0.000000in}{0.000000in}}%
\pgfpathcurveto{\pgfqpoint{0.000000in}{0.000000in}}{\pgfqpoint{0.000000in}{0.000000in}}{\pgfqpoint{0.000000in}{0.000000in}}%
\pgfpathcurveto{\pgfqpoint{0.000000in}{0.000000in}}{\pgfqpoint{0.000000in}{0.000000in}}{\pgfqpoint{0.000000in}{0.000000in}}%
\pgfpathcurveto{\pgfqpoint{0.000000in}{0.000000in}}{\pgfqpoint{0.000000in}{0.000000in}}{\pgfqpoint{0.000000in}{0.000000in}}%
\pgfpathcurveto{\pgfqpoint{0.000000in}{0.000000in}}{\pgfqpoint{0.000000in}{0.000000in}}{\pgfqpoint{0.000000in}{0.000000in}}%
\pgfpathlineto{\pgfqpoint{0.000000in}{0.000000in}}%
\pgfpathclose%
\pgfusepath{stroke,fill}%
}%
\begin{pgfscope}%
\pgfsys@transformshift{2.107533in}{1.731280in}%
\pgfsys@useobject{currentmarker}{}%
\end{pgfscope}%
\end{pgfscope}%
\begin{pgfscope}%
\pgfpathrectangle{\pgfqpoint{0.462743in}{0.284240in}}{\pgfqpoint{1.723113in}{1.989680in}}%
\pgfusepath{clip}%
\pgfsetbuttcap%
\pgfsetroundjoin%
\definecolor{currentfill}{rgb}{0.000000,0.000000,0.000000}%
\pgfsetfillcolor{currentfill}%
\pgfsetlinewidth{0.000000pt}%
\definecolor{currentstroke}{rgb}{0.000000,0.000000,0.000000}%
\pgfsetstrokecolor{currentstroke}%
\pgfsetdash{}{0pt}%
\pgfpathmoveto{\pgfqpoint{1.321069in}{1.284676in}}%
\pgfpathlineto{\pgfqpoint{2.053939in}{1.707798in}}%
\pgfpathlineto{\pgfqpoint{2.041881in}{1.715760in}}%
\pgfpathlineto{\pgfqpoint{2.107533in}{1.731280in}}%
\pgfpathlineto{\pgfqpoint{2.061266in}{1.682184in}}%
\pgfpathlineto{\pgfqpoint{2.060400in}{1.696606in}}%
\pgfpathlineto{\pgfqpoint{1.327531in}{1.273484in}}%
\pgfpathlineto{\pgfqpoint{1.321069in}{1.284676in}}%
\pgfusepath{fill}%
\end{pgfscope}%
\begin{pgfscope}%
\pgfpathrectangle{\pgfqpoint{0.462743in}{0.284240in}}{\pgfqpoint{1.723113in}{1.989680in}}%
\pgfusepath{clip}%
\pgfsetbuttcap%
\pgfsetroundjoin%
\definecolor{currentfill}{rgb}{1.000000,0.662745,0.054902}%
\pgfsetfillcolor{currentfill}%
\pgfsetlinewidth{1.003750pt}%
\definecolor{currentstroke}{rgb}{1.000000,0.662745,0.054902}%
\pgfsetstrokecolor{currentstroke}%
\pgfsetdash{}{0pt}%
\pgfsys@defobject{currentmarker}{\pgfqpoint{0.000000in}{0.000000in}}{\pgfqpoint{0.000000in}{0.000000in}}{%
\pgfpathmoveto{\pgfqpoint{0.000000in}{0.000000in}}%
\pgfpathcurveto{\pgfqpoint{0.000000in}{0.000000in}}{\pgfqpoint{0.000000in}{0.000000in}}{\pgfqpoint{0.000000in}{0.000000in}}%
\pgfpathcurveto{\pgfqpoint{0.000000in}{0.000000in}}{\pgfqpoint{0.000000in}{0.000000in}}{\pgfqpoint{0.000000in}{0.000000in}}%
\pgfpathcurveto{\pgfqpoint{0.000000in}{0.000000in}}{\pgfqpoint{0.000000in}{0.000000in}}{\pgfqpoint{0.000000in}{0.000000in}}%
\pgfpathcurveto{\pgfqpoint{0.000000in}{0.000000in}}{\pgfqpoint{0.000000in}{0.000000in}}{\pgfqpoint{0.000000in}{0.000000in}}%
\pgfpathcurveto{\pgfqpoint{0.000000in}{0.000000in}}{\pgfqpoint{0.000000in}{0.000000in}}{\pgfqpoint{0.000000in}{0.000000in}}%
\pgfpathcurveto{\pgfqpoint{0.000000in}{0.000000in}}{\pgfqpoint{0.000000in}{0.000000in}}{\pgfqpoint{0.000000in}{0.000000in}}%
\pgfpathcurveto{\pgfqpoint{0.000000in}{0.000000in}}{\pgfqpoint{0.000000in}{0.000000in}}{\pgfqpoint{0.000000in}{0.000000in}}%
\pgfpathcurveto{\pgfqpoint{0.000000in}{0.000000in}}{\pgfqpoint{0.000000in}{0.000000in}}{\pgfqpoint{0.000000in}{0.000000in}}%
\pgfpathlineto{\pgfqpoint{0.000000in}{0.000000in}}%
\pgfpathclose%
\pgfusepath{stroke,fill}%
}%
\begin{pgfscope}%
\pgfsys@transformshift{2.107533in}{0.826880in}%
\pgfsys@useobject{currentmarker}{}%
\end{pgfscope}%
\end{pgfscope}%
\begin{pgfscope}%
\pgfpathrectangle{\pgfqpoint{0.462743in}{0.284240in}}{\pgfqpoint{1.723113in}{1.989680in}}%
\pgfusepath{clip}%
\pgfsetbuttcap%
\pgfsetroundjoin%
\definecolor{currentfill}{rgb}{0.000000,0.000000,0.000000}%
\pgfsetfillcolor{currentfill}%
\pgfsetlinewidth{0.000000pt}%
\definecolor{currentstroke}{rgb}{0.000000,0.000000,0.000000}%
\pgfsetstrokecolor{currentstroke}%
\pgfsetdash{}{0pt}%
\pgfpathmoveto{\pgfqpoint{1.327531in}{1.284676in}}%
\pgfpathlineto{\pgfqpoint{2.060400in}{0.861554in}}%
\pgfpathlineto{\pgfqpoint{2.061266in}{0.875976in}}%
\pgfpathlineto{\pgfqpoint{2.107533in}{0.826880in}}%
\pgfpathlineto{\pgfqpoint{2.041881in}{0.842400in}}%
\pgfpathlineto{\pgfqpoint{2.053939in}{0.850362in}}%
\pgfpathlineto{\pgfqpoint{1.321069in}{1.273484in}}%
\pgfpathlineto{\pgfqpoint{1.327531in}{1.284676in}}%
\pgfusepath{fill}%
\end{pgfscope}%
\begin{pgfscope}%
\pgfpathrectangle{\pgfqpoint{0.462743in}{0.284240in}}{\pgfqpoint{1.723113in}{1.989680in}}%
\pgfusepath{clip}%
\pgfsetbuttcap%
\pgfsetroundjoin%
\definecolor{currentfill}{rgb}{0.000000,0.000000,0.000000}%
\pgfsetfillcolor{currentfill}%
\pgfsetlinewidth{1.003750pt}%
\definecolor{currentstroke}{rgb}{0.000000,0.000000,0.000000}%
\pgfsetstrokecolor{currentstroke}%
\pgfsetdash{}{0pt}%
\pgfsys@defobject{currentmarker}{\pgfqpoint{-0.069444in}{-0.069444in}}{\pgfqpoint{0.069444in}{0.069444in}}{%
\pgfpathmoveto{\pgfqpoint{0.000000in}{-0.069444in}}%
\pgfpathcurveto{\pgfqpoint{0.018417in}{-0.069444in}}{\pgfqpoint{0.036082in}{-0.062127in}}{\pgfqpoint{0.049105in}{-0.049105in}}%
\pgfpathcurveto{\pgfqpoint{0.062127in}{-0.036082in}}{\pgfqpoint{0.069444in}{-0.018417in}}{\pgfqpoint{0.069444in}{0.000000in}}%
\pgfpathcurveto{\pgfqpoint{0.069444in}{0.018417in}}{\pgfqpoint{0.062127in}{0.036082in}}{\pgfqpoint{0.049105in}{0.049105in}}%
\pgfpathcurveto{\pgfqpoint{0.036082in}{0.062127in}}{\pgfqpoint{0.018417in}{0.069444in}}{\pgfqpoint{0.000000in}{0.069444in}}%
\pgfpathcurveto{\pgfqpoint{-0.018417in}{0.069444in}}{\pgfqpoint{-0.036082in}{0.062127in}}{\pgfqpoint{-0.049105in}{0.049105in}}%
\pgfpathcurveto{\pgfqpoint{-0.062127in}{0.036082in}}{\pgfqpoint{-0.069444in}{0.018417in}}{\pgfqpoint{-0.069444in}{0.000000in}}%
\pgfpathcurveto{\pgfqpoint{-0.069444in}{-0.018417in}}{\pgfqpoint{-0.062127in}{-0.036082in}}{\pgfqpoint{-0.049105in}{-0.049105in}}%
\pgfpathcurveto{\pgfqpoint{-0.036082in}{-0.062127in}}{\pgfqpoint{-0.018417in}{-0.069444in}}{\pgfqpoint{0.000000in}{-0.069444in}}%
\pgfpathlineto{\pgfqpoint{0.000000in}{-0.069444in}}%
\pgfpathclose%
\pgfusepath{stroke,fill}%
}%
\begin{pgfscope}%
\pgfsys@transformshift{1.324300in}{1.279080in}%
\pgfsys@useobject{currentmarker}{}%
\end{pgfscope}%
\end{pgfscope}%
\begin{pgfscope}%
\pgfpathrectangle{\pgfqpoint{0.462743in}{0.284240in}}{\pgfqpoint{1.723113in}{1.989680in}}%
\pgfusepath{clip}%
\pgfsetbuttcap%
\pgfsetroundjoin%
\definecolor{currentfill}{rgb}{0.000000,0.000000,0.000000}%
\pgfsetfillcolor{currentfill}%
\pgfsetlinewidth{1.003750pt}%
\definecolor{currentstroke}{rgb}{0.000000,0.000000,0.000000}%
\pgfsetstrokecolor{currentstroke}%
\pgfsetdash{}{0pt}%
\pgfsys@defobject{currentmarker}{\pgfqpoint{-0.069444in}{-0.069444in}}{\pgfqpoint{0.069444in}{0.069444in}}{%
\pgfpathmoveto{\pgfqpoint{0.000000in}{-0.069444in}}%
\pgfpathcurveto{\pgfqpoint{0.018417in}{-0.069444in}}{\pgfqpoint{0.036082in}{-0.062127in}}{\pgfqpoint{0.049105in}{-0.049105in}}%
\pgfpathcurveto{\pgfqpoint{0.062127in}{-0.036082in}}{\pgfqpoint{0.069444in}{-0.018417in}}{\pgfqpoint{0.069444in}{0.000000in}}%
\pgfpathcurveto{\pgfqpoint{0.069444in}{0.018417in}}{\pgfqpoint{0.062127in}{0.036082in}}{\pgfqpoint{0.049105in}{0.049105in}}%
\pgfpathcurveto{\pgfqpoint{0.036082in}{0.062127in}}{\pgfqpoint{0.018417in}{0.069444in}}{\pgfqpoint{0.000000in}{0.069444in}}%
\pgfpathcurveto{\pgfqpoint{-0.018417in}{0.069444in}}{\pgfqpoint{-0.036082in}{0.062127in}}{\pgfqpoint{-0.049105in}{0.049105in}}%
\pgfpathcurveto{\pgfqpoint{-0.062127in}{0.036082in}}{\pgfqpoint{-0.069444in}{0.018417in}}{\pgfqpoint{-0.069444in}{0.000000in}}%
\pgfpathcurveto{\pgfqpoint{-0.069444in}{-0.018417in}}{\pgfqpoint{-0.062127in}{-0.036082in}}{\pgfqpoint{-0.049105in}{-0.049105in}}%
\pgfpathcurveto{\pgfqpoint{-0.036082in}{-0.062127in}}{\pgfqpoint{-0.018417in}{-0.069444in}}{\pgfqpoint{0.000000in}{-0.069444in}}%
\pgfpathlineto{\pgfqpoint{0.000000in}{-0.069444in}}%
\pgfpathclose%
\pgfusepath{stroke,fill}%
}%
\begin{pgfscope}%
\pgfsys@transformshift{1.715917in}{1.505180in}%
\pgfsys@useobject{currentmarker}{}%
\end{pgfscope}%
\end{pgfscope}%
\begin{pgfscope}%
\pgfpathrectangle{\pgfqpoint{0.462743in}{0.284240in}}{\pgfqpoint{1.723113in}{1.989680in}}%
\pgfusepath{clip}%
\pgfsetbuttcap%
\pgfsetroundjoin%
\definecolor{currentfill}{rgb}{0.000000,0.000000,0.000000}%
\pgfsetfillcolor{currentfill}%
\pgfsetlinewidth{1.003750pt}%
\definecolor{currentstroke}{rgb}{0.000000,0.000000,0.000000}%
\pgfsetstrokecolor{currentstroke}%
\pgfsetdash{}{0pt}%
\pgfsys@defobject{currentmarker}{\pgfqpoint{-0.069444in}{-0.069444in}}{\pgfqpoint{0.069444in}{0.069444in}}{%
\pgfpathmoveto{\pgfqpoint{0.000000in}{-0.069444in}}%
\pgfpathcurveto{\pgfqpoint{0.018417in}{-0.069444in}}{\pgfqpoint{0.036082in}{-0.062127in}}{\pgfqpoint{0.049105in}{-0.049105in}}%
\pgfpathcurveto{\pgfqpoint{0.062127in}{-0.036082in}}{\pgfqpoint{0.069444in}{-0.018417in}}{\pgfqpoint{0.069444in}{0.000000in}}%
\pgfpathcurveto{\pgfqpoint{0.069444in}{0.018417in}}{\pgfqpoint{0.062127in}{0.036082in}}{\pgfqpoint{0.049105in}{0.049105in}}%
\pgfpathcurveto{\pgfqpoint{0.036082in}{0.062127in}}{\pgfqpoint{0.018417in}{0.069444in}}{\pgfqpoint{0.000000in}{0.069444in}}%
\pgfpathcurveto{\pgfqpoint{-0.018417in}{0.069444in}}{\pgfqpoint{-0.036082in}{0.062127in}}{\pgfqpoint{-0.049105in}{0.049105in}}%
\pgfpathcurveto{\pgfqpoint{-0.062127in}{0.036082in}}{\pgfqpoint{-0.069444in}{0.018417in}}{\pgfqpoint{-0.069444in}{0.000000in}}%
\pgfpathcurveto{\pgfqpoint{-0.069444in}{-0.018417in}}{\pgfqpoint{-0.062127in}{-0.036082in}}{\pgfqpoint{-0.049105in}{-0.049105in}}%
\pgfpathcurveto{\pgfqpoint{-0.036082in}{-0.062127in}}{\pgfqpoint{-0.018417in}{-0.069444in}}{\pgfqpoint{0.000000in}{-0.069444in}}%
\pgfpathlineto{\pgfqpoint{0.000000in}{-0.069444in}}%
\pgfpathclose%
\pgfusepath{stroke,fill}%
}%
\begin{pgfscope}%
\pgfsys@transformshift{1.846456in}{1.279080in}%
\pgfsys@useobject{currentmarker}{}%
\end{pgfscope}%
\end{pgfscope}%
\begin{pgfscope}%
\pgfpathrectangle{\pgfqpoint{0.462743in}{0.284240in}}{\pgfqpoint{1.723113in}{1.989680in}}%
\pgfusepath{clip}%
\pgfsetbuttcap%
\pgfsetroundjoin%
\pgfsetlinewidth{0.501875pt}%
\definecolor{currentstroke}{rgb}{0.000000,0.000000,0.000000}%
\pgfsetstrokecolor{currentstroke}%
\pgfsetdash{}{0pt}%
\pgfpathmoveto{\pgfqpoint{1.324300in}{1.279080in}}%
\pgfpathlineto{\pgfqpoint{1.324300in}{1.279080in}}%
\pgfusepath{stroke}%
\end{pgfscope}%
\begin{pgfscope}%
\pgfpathrectangle{\pgfqpoint{0.462743in}{0.284240in}}{\pgfqpoint{1.723113in}{1.989680in}}%
\pgfusepath{clip}%
\pgfsetbuttcap%
\pgfsetroundjoin%
\pgfsetlinewidth{0.501875pt}%
\definecolor{currentstroke}{rgb}{0.000000,0.000000,0.000000}%
\pgfsetstrokecolor{currentstroke}%
\pgfsetdash{}{0pt}%
\pgfpathmoveto{\pgfqpoint{1.324300in}{1.279080in}}%
\pgfpathlineto{\pgfqpoint{0.541067in}{0.826880in}}%
\pgfusepath{stroke}%
\end{pgfscope}%
\begin{pgfscope}%
\pgfpathrectangle{\pgfqpoint{0.462743in}{0.284240in}}{\pgfqpoint{1.723113in}{1.989680in}}%
\pgfusepath{clip}%
\pgfsetbuttcap%
\pgfsetroundjoin%
\pgfsetlinewidth{0.501875pt}%
\definecolor{currentstroke}{rgb}{0.000000,0.000000,0.000000}%
\pgfsetstrokecolor{currentstroke}%
\pgfsetdash{}{0pt}%
\pgfpathmoveto{\pgfqpoint{1.324300in}{1.279080in}}%
\pgfpathlineto{\pgfqpoint{1.324300in}{0.374680in}}%
\pgfusepath{stroke}%
\end{pgfscope}%
\begin{pgfscope}%
\pgfpathrectangle{\pgfqpoint{0.462743in}{0.284240in}}{\pgfqpoint{1.723113in}{1.989680in}}%
\pgfusepath{clip}%
\pgfsetbuttcap%
\pgfsetroundjoin%
\pgfsetlinewidth{0.501875pt}%
\definecolor{currentstroke}{rgb}{0.000000,0.000000,0.000000}%
\pgfsetstrokecolor{currentstroke}%
\pgfsetdash{}{0pt}%
\pgfpathmoveto{\pgfqpoint{1.324300in}{1.279080in}}%
\pgfpathlineto{\pgfqpoint{0.541067in}{1.731280in}}%
\pgfusepath{stroke}%
\end{pgfscope}%
\begin{pgfscope}%
\pgfpathrectangle{\pgfqpoint{0.462743in}{0.284240in}}{\pgfqpoint{1.723113in}{1.989680in}}%
\pgfusepath{clip}%
\pgfsetbuttcap%
\pgfsetroundjoin%
\pgfsetlinewidth{0.501875pt}%
\definecolor{currentstroke}{rgb}{0.000000,0.000000,0.000000}%
\pgfsetstrokecolor{currentstroke}%
\pgfsetdash{}{0pt}%
\pgfpathmoveto{\pgfqpoint{1.324300in}{1.279080in}}%
\pgfpathlineto{\pgfqpoint{2.107533in}{0.826880in}}%
\pgfusepath{stroke}%
\end{pgfscope}%
\begin{pgfscope}%
\pgfpathrectangle{\pgfqpoint{0.462743in}{0.284240in}}{\pgfqpoint{1.723113in}{1.989680in}}%
\pgfusepath{clip}%
\pgfsetbuttcap%
\pgfsetroundjoin%
\pgfsetlinewidth{0.501875pt}%
\definecolor{currentstroke}{rgb}{0.000000,0.000000,0.000000}%
\pgfsetstrokecolor{currentstroke}%
\pgfsetdash{}{0pt}%
\pgfpathmoveto{\pgfqpoint{1.324300in}{1.279080in}}%
\pgfpathlineto{\pgfqpoint{1.324300in}{2.183480in}}%
\pgfusepath{stroke}%
\end{pgfscope}%
\begin{pgfscope}%
\pgfpathrectangle{\pgfqpoint{0.462743in}{0.284240in}}{\pgfqpoint{1.723113in}{1.989680in}}%
\pgfusepath{clip}%
\pgfsetbuttcap%
\pgfsetroundjoin%
\pgfsetlinewidth{0.501875pt}%
\definecolor{currentstroke}{rgb}{0.000000,0.000000,0.000000}%
\pgfsetstrokecolor{currentstroke}%
\pgfsetdash{}{0pt}%
\pgfpathmoveto{\pgfqpoint{1.324300in}{1.279080in}}%
\pgfpathlineto{\pgfqpoint{2.107533in}{1.731280in}}%
\pgfusepath{stroke}%
\end{pgfscope}%
\begin{pgfscope}%
\pgfpathrectangle{\pgfqpoint{0.462743in}{0.284240in}}{\pgfqpoint{1.723113in}{1.989680in}}%
\pgfusepath{clip}%
\pgfsetbuttcap%
\pgfsetroundjoin%
\pgfsetlinewidth{1.003750pt}%
\definecolor{currentstroke}{rgb}{0.000000,0.000000,0.000000}%
\pgfsetstrokecolor{currentstroke}%
\pgfsetdash{}{0pt}%
\pgfpathmoveto{\pgfqpoint{0.802144in}{1.279080in}}%
\pgfpathlineto{\pgfqpoint{1.063222in}{1.731280in}}%
\pgfusepath{stroke}%
\end{pgfscope}%
\begin{pgfscope}%
\pgfpathrectangle{\pgfqpoint{0.462743in}{0.284240in}}{\pgfqpoint{1.723113in}{1.989680in}}%
\pgfusepath{clip}%
\pgfsetbuttcap%
\pgfsetroundjoin%
\pgfsetlinewidth{1.003750pt}%
\definecolor{currentstroke}{rgb}{0.000000,0.000000,0.000000}%
\pgfsetstrokecolor{currentstroke}%
\pgfsetdash{}{0pt}%
\pgfpathmoveto{\pgfqpoint{1.063222in}{1.731280in}}%
\pgfpathlineto{\pgfqpoint{1.585378in}{1.731280in}}%
\pgfusepath{stroke}%
\end{pgfscope}%
\begin{pgfscope}%
\pgfpathrectangle{\pgfqpoint{0.462743in}{0.284240in}}{\pgfqpoint{1.723113in}{1.989680in}}%
\pgfusepath{clip}%
\pgfsetbuttcap%
\pgfsetroundjoin%
\pgfsetlinewidth{1.003750pt}%
\definecolor{currentstroke}{rgb}{0.000000,0.000000,0.000000}%
\pgfsetstrokecolor{currentstroke}%
\pgfsetdash{}{0pt}%
\pgfpathmoveto{\pgfqpoint{1.063222in}{0.826880in}}%
\pgfpathlineto{\pgfqpoint{1.585378in}{0.826880in}}%
\pgfusepath{stroke}%
\end{pgfscope}%
\begin{pgfscope}%
\pgfpathrectangle{\pgfqpoint{0.462743in}{0.284240in}}{\pgfqpoint{1.723113in}{1.989680in}}%
\pgfusepath{clip}%
\pgfsetbuttcap%
\pgfsetroundjoin%
\pgfsetlinewidth{1.003750pt}%
\definecolor{currentstroke}{rgb}{0.000000,0.000000,0.000000}%
\pgfsetstrokecolor{currentstroke}%
\pgfsetdash{}{0pt}%
\pgfpathmoveto{\pgfqpoint{1.585378in}{0.826880in}}%
\pgfpathlineto{\pgfqpoint{1.846456in}{1.279080in}}%
\pgfusepath{stroke}%
\end{pgfscope}%
\begin{pgfscope}%
\pgfpathrectangle{\pgfqpoint{0.462743in}{0.284240in}}{\pgfqpoint{1.723113in}{1.989680in}}%
\pgfusepath{clip}%
\pgfsetbuttcap%
\pgfsetroundjoin%
\pgfsetlinewidth{1.003750pt}%
\definecolor{currentstroke}{rgb}{0.000000,0.000000,0.000000}%
\pgfsetstrokecolor{currentstroke}%
\pgfsetdash{}{0pt}%
\pgfpathmoveto{\pgfqpoint{0.802144in}{1.279080in}}%
\pgfpathlineto{\pgfqpoint{1.063222in}{0.826880in}}%
\pgfusepath{stroke}%
\end{pgfscope}%
\begin{pgfscope}%
\pgfpathrectangle{\pgfqpoint{0.462743in}{0.284240in}}{\pgfqpoint{1.723113in}{1.989680in}}%
\pgfusepath{clip}%
\pgfsetbuttcap%
\pgfsetroundjoin%
\pgfsetlinewidth{1.003750pt}%
\definecolor{currentstroke}{rgb}{0.000000,0.000000,0.000000}%
\pgfsetstrokecolor{currentstroke}%
\pgfsetdash{}{0pt}%
\pgfpathmoveto{\pgfqpoint{1.846456in}{1.279080in}}%
\pgfpathlineto{\pgfqpoint{1.585378in}{1.731280in}}%
\pgfusepath{stroke}%
\end{pgfscope}%
\begin{pgfscope}%
\pgfpathrectangle{\pgfqpoint{0.462743in}{0.284240in}}{\pgfqpoint{1.723113in}{1.989680in}}%
\pgfusepath{clip}%
\pgfsetrectcap%
\pgfsetroundjoin%
\pgfsetlinewidth{1.505625pt}%
\definecolor{currentstroke}{rgb}{0.000000,0.000000,0.000000}%
\pgfsetstrokecolor{currentstroke}%
\pgfsetdash{}{0pt}%
\pgfpathmoveto{\pgfqpoint{1.324300in}{1.279080in}}%
\pgfpathlineto{\pgfqpoint{1.715917in}{1.505180in}}%
\pgfpathlineto{\pgfqpoint{1.846456in}{1.279080in}}%
\pgfpathlineto{\pgfqpoint{1.324300in}{1.279080in}}%
\pgfusepath{stroke}%
\end{pgfscope}%
\begin{pgfscope}%
\definecolor{textcolor}{rgb}{0.000000,0.000000,0.000000}%
\pgfsetstrokecolor{textcolor}%
\pgfsetfillcolor{textcolor}%
\pgftext[x=1.352078in,y=1.417969in,left,base]{\color{textcolor}{\rmfamily\fontsize{13.200000}{15.840000}\selectfont\catcode`\^=\active\def^{\ifmmode\sp\else\^{}\fi}\catcode`\%=\active\def%{\%}$\Gamma$}}%
\end{pgfscope}%
\begin{pgfscope}%
\definecolor{textcolor}{rgb}{0.000000,0.000000,0.000000}%
\pgfsetstrokecolor{textcolor}%
\pgfsetfillcolor{textcolor}%
\pgftext[x=1.715917in,y=1.671847in,left,base]{\color{textcolor}{\rmfamily\fontsize{13.200000}{15.840000}\selectfont\catcode`\^=\active\def^{\ifmmode\sp\else\^{}\fi}\catcode`\%=\active\def%{\%}$\mathrm{M}$}}%
\end{pgfscope}%
\begin{pgfscope}%
\definecolor{textcolor}{rgb}{0.000000,0.000000,0.000000}%
\pgfsetstrokecolor{textcolor}%
\pgfsetfillcolor{textcolor}%
\pgftext[x=1.846456in,y=1.445747in,left,base]{\color{textcolor}{\rmfamily\fontsize{13.200000}{15.840000}\selectfont\catcode`\^=\active\def^{\ifmmode\sp\else\^{}\fi}\catcode`\%=\active\def%{\%}$\mathrm{K}$}}%
\end{pgfscope}%
\end{pgfpicture}%
\makeatother%
\endgroup%

	\end{subfigure}
	\caption{(\subref{sfig:graphene lattice structure}) Graphene lattice structure with primitive lattice vectors \(\vb{a}_1\), \(\vb{a}_2\) and (\subref{sfig:graphene Brillouin zone}) Brillouin zone with reciprocal vectors \(\vb{b}_1\), \(\vb{b}_2\). Both images created with lattpy \cite{Jones_lattpy_2022}}
	\label{fig:Graphene lattice structure and Brilluoin zone}
\end{figure}
The primitive reciprocal lattice vectors \(\vb{b}_1\), \(\vb{b}_2\) fulfill:
\begin{align}
	\vb{a}_1 \cdot \vb{b}_1 &= \vb{a}_2 \cdot \vb{b}_2 = 2\pi \\
	\vb{a}_1 \cdot \vb{b}_2 &= \vb{a}_2 \cdot \vb{b}_1 = 0\;,
\end{align}
so that
\begin{align}
	\vb{b}_1 = \frac{2\pi}{a} \begin{pmatrix} 1 \\ \frac{1}{\sqrt{3}} \end{pmatrix},\;
	\vb{b}_2 = \frac{2\pi}{a} \begin{pmatrix} 1 \\ - \frac{1}{\sqrt{3}} \end{pmatrix} \;.
\end{align}
The first Brillouin zone of the hexagonal lattice is shown in \cref{sfig:graphene Brillouin zone}, with the points of high symmetry
\begin{align}
	\Gamma = \begin{pmatrix} 0 \\ 0 \end{pmatrix},\;
	\mathrm{M} = \frac{\pi}{a} \begin{pmatrix} 1 \\ \frac{1}{\sqrt{3}} \end{pmatrix},\;
	\mathrm{K} = \frac{4\pi}{3 a} \begin{pmatrix} 1 \\ 0 \end{pmatrix}\;.
\end{align}

The elemental model as shown in \cref{fig:decorated graphene model} has the following kinetic terms:
\begin{align}
	H_0 &= -t \sum_{\langle ij \rangle, \sigma}
	c_{i, \sigma}^{(\mathrm{A}), \dagger} c_{j, \sigma}^{(\mathrm{B})}
	+ V \sum_{i, \sigma \sigma^{\prime}}
	d_{i, \sigma}^{\dagger} c_{i, \sigma^{\prime}}^{(\mathrm{A})} + \mathrm{h.c.}
	\label{eq:EG-X model Hamiltonian non-interacting}
\end{align}
with
\begin{itemize}
	\item \(d\) - operators on the X atom
	\item \(c^{(\epsilon)}\) - operators on the graphene sites (\(\epsilon = \mathrm{A}, \mathrm{B}\))
	\item \(t\) - nearest neighbor hopping between Graphene sites
	\item \(V\) - hopping between \(\mathrm{X}\) and Graphene \(\mathrm{A}\) sites.
\end{itemize}

Using the Fourier transformation
\begin{equation}
	c_{i \alpha \sigma} = \frac{1}{\sqrt{N}} \sum_{\vb{k}} e^{\iu \vb{k} \vb{r}_{i \alpha}} c_{\vb{k} \alpha \sigma} \;,
\end{equation}
the hopping term becomes
\begin{align}
	&-t \sum_{\langle ij \rangle, \sigma} c_{i, \sigma}^{(\mathrm{A}), \dagger} c_{j, \sigma}^{(\mathrm{B})} \\
	&= -t \sum_{i,\delta_{\mathrm{AB}},\sigma} c_{i, \sigma}^{(\mathrm{A}) \dagger} c_{i + \delta_{\mathrm{AB}}, \sigma}^{(\mathrm{B})} \\
	&= -\frac{t}{N^2} \sum_{i,\sigma} \sum_{\vb{k}, \vb{k}^{\prime}, \delta_{\mathrm{AB}}} \left(e^{-\iu \vb{k} \vb{r}_{i \alpha}} c_{\vb{k}, \sigma}^{(\mathrm{A}) \dagger}\right) \left(e^{\iu \vb{k}^{\prime} \vb{r}_{i \alpha} +\delta_{AB}} c_{\vb{k}^{\prime}, \sigma}^{(B)} \right) \\
	&= -\frac{t}{N^2} \sum_{\vb{k}, \vb{k^{\prime}}, \delta_{\mathrm{AB}}, \sigma} c_{\vb{k}, \sigma}^{(\mathrm{A}) \dagger} c_{\vb{k}^{\prime}, \sigma}^{(\mathrm{B})} e^{\iu \vb{k}^{\prime} \delta_{\mathrm{AB}}} e^{\iu (\vb{k} (\delta_A - \delta_B) + \vb{k}^{\prime} (\delta_A - \delta_B))} \sum_{i} e^{-\iu \vb{k} \vb{R}_i} e^{\iu \vb{k}^{\prime} \vb{R}_i} \\
	&= -\frac{t}{N^2} \sum_{\vb{k}, \vb{k^{\prime}}, \sigma}  c_{\vb{k}, \sigma}^{(\mathrm{A}) \dagger} c_{\vb{k}^{\prime}, \sigma}^{(\mathrm{B})} \sum_{\delta_{\mathrm{AB}}} e^{\iu \vb{k}^{\prime} \delta_{\mathrm{AB}}} e^{\iu (\vb{k} (\delta_A - \delta_B) + \vb{k}^{\prime} (\delta_A - \delta_B))} \left(N^2 \delta_{\vb{k}, \vb{k}^{\prime}} \right)\\
	&= -t \sum_{\vb{k}, \sigma}  c_{\vb{k}, \sigma}^{(\mathrm{A}) \dagger} c_{\vb{k}, \sigma}^{(\mathrm{B})} \sum_{\delta_{\mathrm{AB}}} e^{\iu (\vb{k} \delta_{AB} + 2 k_y a)} = \sum_{\vb{k}, \sigma} f_{\vb{k}} c_{\vb{k}, \sigma}^{(\mathrm{A}) \dagger} c_{\vb{k}, \sigma}^{(\mathrm{B})} \;.
\end{align}
The factor \(f_{\vb{k}}\) can be written out explicitly using the nearest-neighbor vectors, for example
\begin{align}
	\vb{k} \cdot \vb{\delta_{AB, 1}} = \begin{pmatrix} k_x \\ k_y \end{pmatrix} \cdot \begin{pmatrix} 0 \\ \frac{a}{\sqrt{3}} \end{pmatrix} = \frac{1}{\sqrt{3}} k_y \;.
\end{align}
This gives:
\begin{align}
	f_{\vb{k}} &= -t \sum_{\delta_{AB}} e^{\iu (\vb{k} \delta_{AB} + 2 k_y a)} \\
	&= -t_{\mathrm{Gr}} e^{2 \iu k_y a} \left(
	e^{\iu \frac{a}{\sqrt{3}} k_y} +
	e^{\iu \frac{a}{2\sqrt{3}} (\sqrt{3} k_x - k_y)} +
	e^{\iu \frac{a}{2\sqrt{3}} (-\sqrt{3} k_x - k_y)} \right) \\
	&= -t_{\mathrm{Gr}} e^{2 \iu k_y a} \left(
	e^{\iu \frac{a}{\sqrt{3}} k_y} +
	2 e^{-\iu \frac{a}{2\sqrt{3}} k_y}
	\cos{(\frac{a}{2} k_x)} \right) \;.
\end{align}
Using the fact that \(\delta_{\mathrm{BA}, i} = -\delta_{\mathrm{AB}, i}\), it follows
\begin{align}
	-t \sum_{\delta_{BA}} e^{\iu \vb{k} \delta_{BA}} = -t \sum_{\delta_{AB}} e^{-\iu \vb{k} \delta_{AB}} = \left(-t \sum_{\delta_{AB}} e^{\iu \vb{k} \delta_{AB}}\right)^* = f_{\vb{k}}^* \;,
\end{align}
which then gives
\begin{align}
	H_0 &= \sum_{\vb{k}, \sigma} C_{\vb{k}, \sigma}^{\dagger}
	\begin{pmatrix}
		0 & f_{\vb{k}} & V \\
		f_{\vb{k}}^* & 0 & 0 \\
		V & 0 & 0
	\end{pmatrix} C_{\vb{k}, \sigma}
	\label{eq:decorated graphene Hamiltonian non-interacting matrix} \\
	 C_{\vb{k}, \sigma} &= \begin{pmatrix} c_{\vb{k}, \sigma}^{A, \dagger} & c_{\vb{k}, \sigma}^{B, \dagger} & d_{\vb{k}, \sigma}^{\dagger} \end{pmatrix}^T
\end{align}
The band structure for the non-interacting decorated graphene model is obtained by diagonalizing the matrix in \cref{eq:decorated graphene Hamiltonian non-interacting matrix}.
\todo{Write more about the bands}
\todo{Derirative of h(k)}
%This was done in \cref{fig:decorated graphene model non-interacting bands}.
\begin{figure}[t]
	\centering
	 \import{images}{dressed graphene bands.pgf}
	\caption{Bands of the non-interacting  decorated Graphene model}
	\label{fig:decorated graphene model non-interacting bands}
\end{figure}

\section{Quantum Geometry}

\todo{Section about quantum geometry, maybe with lattice site local quantum metric?}

\end{document}
