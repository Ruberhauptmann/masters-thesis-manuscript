\documentclass[../notes.tex]{subfiles}
\graphicspath{{\subfix{../images/}}, {\subfix{../}}}

\begin{document}
\raggedbottom

\chapter{Dressed Graphene Model}

This thesis concerned with a specific model.
Idea: Graphene with an added orbital on one of the lattice site with a low hopping, as to provide a flat band.
I will call this model dressed Graphene from here on.
This chapter reviews the lattice structure in \cref{sec:lattice-structure-of-graphene}.

\todo{Write introduction to the model and what is done in this chapter}

\todo{Connection with Niklas/Siheeon paper on dressed Graphene}

\section{Lattice Structure}\label{sec:lattice-structure-of-graphene}

There exist a few different ways to define the lattice structure of Graphene which are all equivalent, but intermediate steps in calculating tight-binding models look different depending on the definition.
This review on  follows ref.~\cite{yangStructureGrapheneIts2018}.

Monolayer graphene forms a honeycomb lattice, which is a hexagonal Bravais lattice with a two atom basis, as can be seen in \cref{sfig:graphene lattice structure}.
The primitive lattice vectors of the hexagonal lattice are:
\begin{align}
	\vb{a}_1 = \frac{a}{2} \begin{pmatrix} 1 \\ \sqrt{3} \end{pmatrix}, \; \vb{a}_2 = \frac{a}{2} \begin{pmatrix} 1 \\ -\sqrt{3} \end{pmatrix}
\end{align}
\todo{Labels on vectors}
with lattice constant \(a = \sqrt{3} a_0 \approx \SI{2.46}{\angstrom}\), using the nearest-neighbour distance \(a_0\).
The vectors to the nearest-neighbor atoms \(B_i\) (\(i = 1, 2, 3,\)) from atom \(A\) are
\begin{align}
	\vb{\delta}_{AB, 1} = \begin{pmatrix} 0 \\ \frac{a}{\sqrt{3}} \end{pmatrix},\; \vb{\delta}_{AB, 2} = \begin{pmatrix} \frac{a}{2} \\ -\frac{a}{2\sqrt{3}} \end{pmatrix},\; \vb{\delta}_{AB, 3} = \begin{pmatrix} -\frac{a}{2} \\ -\frac{a}{2\sqrt{3}} \end{pmatrix}
\end{align}
and the vectors to the nearest-neighbor atoms \(A_i\) (\(i = 1, 2, 3,\)) from atom \(B\) are
\begin{align}
	\vb{\delta}_{BA, 1} = \begin{pmatrix} 0 \\ -\frac{a}{\sqrt{3}} \end{pmatrix},\; \vb{\delta}_{BA, 2} = \begin{pmatrix} \frac{a}{2} \\ \frac{a}{2\sqrt{3}} \end{pmatrix},\; \vb{\delta}_{BA, 3} = \begin{pmatrix} -\frac{a}{2} \\ \frac{a}{2\sqrt{3}} \end{pmatrix} \;.
\end{align}
The vectors between the Graphene \(\mathrm{A}\) atom and the six neighbours on the same sub lattice are:
\begin{alignat}{3}
	&\vb{\delta}_{AA, 1} = \begin{pmatrix} 1 \\ \sqrt{3} \end{pmatrix}, \;
	&&\vb{\delta}_{AA, 2} = a \begin{pmatrix} 1 \\ 0 \end{pmatrix}, \;
	& &\vb{\delta}_{AA, 3} = a \begin{pmatrix} \frac{1}{2} \\ -\frac{\sqrt{3}}{2} \end{pmatrix}, \\
	&\vb{\delta}_{AA, 4} = a \begin{pmatrix} -\frac{1}{2} \\ -\frac{\sqrt{3}}{2} \end{pmatrix}, \;
	&&\vb{\delta}_{AA, 5} = a \begin{pmatrix} -1 \\ 0 \end{pmatrix}, \;
	& &\vb{\delta}_{AA, 6} = a \begin{pmatrix} -\frac{1}{2} \\ \frac{\sqrt{3}}{2} \end{pmatrix}
\end{alignat}
\begin{figure}[tb]
	\centering
	\begin{subfigure}[t]{0.5\textwidth}
		\centering
		\caption{\hfill\null}\label{sfig:graphene lattice structure}
		%% Creator: Matplotlib, PGF backend
%%
%% To include the figure in your LaTeX document, write
%%   \input{<filename>.pgf}
%%
%% Make sure the required packages are loaded in your preamble
%%   \usepackage{pgf}
%%
%% Also ensure that all the required font packages are loaded; for instance,
%% the lmodern package is sometimes necessary when using math font.
%%   \usepackage{lmodern}
%%
%% Figures using additional raster images can only be included by \input if
%% they are in the same directory as the main LaTeX file. For loading figures
%% from other directories you can use the `import` package
%%   \usepackage{import}
%%
%% and then include the figures with
%%   \import{<path to file>}{<filename>.pgf}
%%
%% Matplotlib used the following preamble
%%   \def\mathdefault#1{#1}
%%   \everymath=\expandafter{\the\everymath\displaystyle}
%%   \IfFileExists{scrextend.sty}{
%%     \usepackage[fontsize=11.000000pt]{scrextend}
%%   }{
%%     \renewcommand{\normalsize}{\fontsize{11.000000}{13.200000}\selectfont}
%%     \normalsize
%%   }
%%   \usepackage{fontspec}\usepackage{unicode-math}\setmathfont{texgyrepagella-math.otf}\setmainfont{texgyrepagella-math}
%%   \makeatletter\@ifpackageloaded{underscore}{}{\usepackage[strings]{underscore}}\makeatother
%%
\begingroup%
\makeatletter%
\begin{pgfpicture}%
\pgfpathrectangle{\pgfpointorigin}{\pgfqpoint{2.185811in}{1.620004in}}%
\pgfusepath{use as bounding box, clip}%
\begin{pgfscope}%
\pgfsetbuttcap%
\pgfsetmiterjoin%
\definecolor{currentfill}{rgb}{1.000000,1.000000,1.000000}%
\pgfsetfillcolor{currentfill}%
\pgfsetlinewidth{0.000000pt}%
\definecolor{currentstroke}{rgb}{1.000000,1.000000,1.000000}%
\pgfsetstrokecolor{currentstroke}%
\pgfsetdash{}{0pt}%
\pgfpathmoveto{\pgfqpoint{0.000000in}{0.000000in}}%
\pgfpathlineto{\pgfqpoint{2.185811in}{0.000000in}}%
\pgfpathlineto{\pgfqpoint{2.185811in}{1.620004in}}%
\pgfpathlineto{\pgfqpoint{0.000000in}{1.620004in}}%
\pgfpathlineto{\pgfqpoint{0.000000in}{0.000000in}}%
\pgfpathclose%
\pgfusepath{fill}%
\end{pgfscope}%
\begin{pgfscope}%
\pgfsetbuttcap%
\pgfsetmiterjoin%
\definecolor{currentfill}{rgb}{1.000000,1.000000,1.000000}%
\pgfsetfillcolor{currentfill}%
\pgfsetlinewidth{0.000000pt}%
\definecolor{currentstroke}{rgb}{0.000000,0.000000,0.000000}%
\pgfsetstrokecolor{currentstroke}%
\pgfsetstrokeopacity{0.000000}%
\pgfsetdash{}{0pt}%
\pgfpathmoveto{\pgfqpoint{0.050000in}{0.050000in}}%
\pgfpathlineto{\pgfqpoint{2.135811in}{0.050000in}}%
\pgfpathlineto{\pgfqpoint{2.135811in}{1.570004in}}%
\pgfpathlineto{\pgfqpoint{0.050000in}{1.570004in}}%
\pgfpathlineto{\pgfqpoint{0.050000in}{0.050000in}}%
\pgfpathclose%
\pgfusepath{fill}%
\end{pgfscope}%
\begin{pgfscope}%
\pgfpathrectangle{\pgfqpoint{0.050000in}{0.050000in}}{\pgfqpoint{2.085811in}{1.520004in}}%
\pgfusepath{clip}%
\pgfsetbuttcap%
\pgfsetroundjoin%
\pgfsetlinewidth{1.003750pt}%
\definecolor{currentstroke}{rgb}{0.000000,0.000000,0.000000}%
\pgfsetstrokecolor{currentstroke}%
\pgfsetdash{}{0pt}%
\pgfpathmoveto{\pgfqpoint{0.245825in}{0.320940in}}%
\pgfpathlineto{\pgfqpoint{0.457595in}{0.198675in}}%
\pgfusepath{stroke}%
\end{pgfscope}%
\begin{pgfscope}%
\pgfpathrectangle{\pgfqpoint{0.050000in}{0.050000in}}{\pgfqpoint{2.085811in}{1.520004in}}%
\pgfusepath{clip}%
\pgfsetbuttcap%
\pgfsetroundjoin%
\pgfsetlinewidth{1.003750pt}%
\definecolor{currentstroke}{rgb}{0.000000,0.000000,0.000000}%
\pgfsetstrokecolor{currentstroke}%
\pgfsetdash{}{0pt}%
\pgfpathmoveto{\pgfqpoint{0.245825in}{0.320940in}}%
\pgfpathlineto{\pgfqpoint{0.245825in}{0.565471in}}%
\pgfusepath{stroke}%
\end{pgfscope}%
\begin{pgfscope}%
\pgfpathrectangle{\pgfqpoint{0.050000in}{0.050000in}}{\pgfqpoint{2.085811in}{1.520004in}}%
\pgfusepath{clip}%
\pgfsetbuttcap%
\pgfsetroundjoin%
\pgfsetlinewidth{1.003750pt}%
\definecolor{currentstroke}{rgb}{0.000000,0.000000,0.000000}%
\pgfsetstrokecolor{currentstroke}%
\pgfsetdash{}{0pt}%
\pgfpathmoveto{\pgfqpoint{0.245825in}{0.565471in}}%
\pgfpathlineto{\pgfqpoint{0.457595in}{0.687737in}}%
\pgfusepath{stroke}%
\end{pgfscope}%
\begin{pgfscope}%
\pgfpathrectangle{\pgfqpoint{0.050000in}{0.050000in}}{\pgfqpoint{2.085811in}{1.520004in}}%
\pgfusepath{clip}%
\pgfsetbuttcap%
\pgfsetroundjoin%
\pgfsetlinewidth{1.003750pt}%
\definecolor{currentstroke}{rgb}{0.000000,0.000000,0.000000}%
\pgfsetstrokecolor{currentstroke}%
\pgfsetdash{}{0pt}%
\pgfpathmoveto{\pgfqpoint{0.457595in}{0.198675in}}%
\pgfpathlineto{\pgfqpoint{0.669365in}{0.320940in}}%
\pgfusepath{stroke}%
\end{pgfscope}%
\begin{pgfscope}%
\pgfpathrectangle{\pgfqpoint{0.050000in}{0.050000in}}{\pgfqpoint{2.085811in}{1.520004in}}%
\pgfusepath{clip}%
\pgfsetbuttcap%
\pgfsetroundjoin%
\pgfsetlinewidth{1.003750pt}%
\definecolor{currentstroke}{rgb}{0.000000,0.000000,0.000000}%
\pgfsetstrokecolor{currentstroke}%
\pgfsetdash{}{0pt}%
\pgfpathmoveto{\pgfqpoint{0.245825in}{1.054533in}}%
\pgfpathlineto{\pgfqpoint{0.457595in}{0.932268in}}%
\pgfusepath{stroke}%
\end{pgfscope}%
\begin{pgfscope}%
\pgfpathrectangle{\pgfqpoint{0.050000in}{0.050000in}}{\pgfqpoint{2.085811in}{1.520004in}}%
\pgfusepath{clip}%
\pgfsetbuttcap%
\pgfsetroundjoin%
\pgfsetlinewidth{1.003750pt}%
\definecolor{currentstroke}{rgb}{0.000000,0.000000,0.000000}%
\pgfsetstrokecolor{currentstroke}%
\pgfsetdash{}{0pt}%
\pgfpathmoveto{\pgfqpoint{0.245825in}{1.054533in}}%
\pgfpathlineto{\pgfqpoint{0.245825in}{1.299064in}}%
\pgfusepath{stroke}%
\end{pgfscope}%
\begin{pgfscope}%
\pgfpathrectangle{\pgfqpoint{0.050000in}{0.050000in}}{\pgfqpoint{2.085811in}{1.520004in}}%
\pgfusepath{clip}%
\pgfsetbuttcap%
\pgfsetroundjoin%
\pgfsetlinewidth{1.003750pt}%
\definecolor{currentstroke}{rgb}{0.000000,0.000000,0.000000}%
\pgfsetstrokecolor{currentstroke}%
\pgfsetdash{}{0pt}%
\pgfpathmoveto{\pgfqpoint{0.245825in}{1.299064in}}%
\pgfpathlineto{\pgfqpoint{0.457595in}{1.421330in}}%
\pgfusepath{stroke}%
\end{pgfscope}%
\begin{pgfscope}%
\pgfpathrectangle{\pgfqpoint{0.050000in}{0.050000in}}{\pgfqpoint{2.085811in}{1.520004in}}%
\pgfusepath{clip}%
\pgfsetbuttcap%
\pgfsetroundjoin%
\pgfsetlinewidth{1.003750pt}%
\definecolor{currentstroke}{rgb}{0.000000,0.000000,0.000000}%
\pgfsetstrokecolor{currentstroke}%
\pgfsetdash{}{0pt}%
\pgfpathmoveto{\pgfqpoint{0.457595in}{0.687737in}}%
\pgfpathlineto{\pgfqpoint{0.669365in}{0.565471in}}%
\pgfusepath{stroke}%
\end{pgfscope}%
\begin{pgfscope}%
\pgfpathrectangle{\pgfqpoint{0.050000in}{0.050000in}}{\pgfqpoint{2.085811in}{1.520004in}}%
\pgfusepath{clip}%
\pgfsetbuttcap%
\pgfsetroundjoin%
\pgfsetlinewidth{1.003750pt}%
\definecolor{currentstroke}{rgb}{0.000000,0.000000,0.000000}%
\pgfsetstrokecolor{currentstroke}%
\pgfsetdash{}{0pt}%
\pgfpathmoveto{\pgfqpoint{0.457595in}{0.687737in}}%
\pgfpathlineto{\pgfqpoint{0.457595in}{0.932268in}}%
\pgfusepath{stroke}%
\end{pgfscope}%
\begin{pgfscope}%
\pgfpathrectangle{\pgfqpoint{0.050000in}{0.050000in}}{\pgfqpoint{2.085811in}{1.520004in}}%
\pgfusepath{clip}%
\pgfsetbuttcap%
\pgfsetroundjoin%
\pgfsetlinewidth{1.003750pt}%
\definecolor{currentstroke}{rgb}{0.000000,0.000000,0.000000}%
\pgfsetstrokecolor{currentstroke}%
\pgfsetdash{}{0pt}%
\pgfpathmoveto{\pgfqpoint{0.457595in}{0.932268in}}%
\pgfpathlineto{\pgfqpoint{0.669365in}{1.054533in}}%
\pgfusepath{stroke}%
\end{pgfscope}%
\begin{pgfscope}%
\pgfpathrectangle{\pgfqpoint{0.050000in}{0.050000in}}{\pgfqpoint{2.085811in}{1.520004in}}%
\pgfusepath{clip}%
\pgfsetbuttcap%
\pgfsetroundjoin%
\pgfsetlinewidth{1.003750pt}%
\definecolor{currentstroke}{rgb}{0.000000,0.000000,0.000000}%
\pgfsetstrokecolor{currentstroke}%
\pgfsetdash{}{0pt}%
\pgfpathmoveto{\pgfqpoint{0.669365in}{0.320940in}}%
\pgfpathlineto{\pgfqpoint{0.669365in}{0.565471in}}%
\pgfusepath{stroke}%
\end{pgfscope}%
\begin{pgfscope}%
\pgfpathrectangle{\pgfqpoint{0.050000in}{0.050000in}}{\pgfqpoint{2.085811in}{1.520004in}}%
\pgfusepath{clip}%
\pgfsetbuttcap%
\pgfsetroundjoin%
\pgfsetlinewidth{1.003750pt}%
\definecolor{currentstroke}{rgb}{0.000000,0.000000,0.000000}%
\pgfsetstrokecolor{currentstroke}%
\pgfsetdash{}{0pt}%
\pgfpathmoveto{\pgfqpoint{0.669365in}{0.320940in}}%
\pgfpathlineto{\pgfqpoint{0.881135in}{0.198675in}}%
\pgfusepath{stroke}%
\end{pgfscope}%
\begin{pgfscope}%
\pgfpathrectangle{\pgfqpoint{0.050000in}{0.050000in}}{\pgfqpoint{2.085811in}{1.520004in}}%
\pgfusepath{clip}%
\pgfsetbuttcap%
\pgfsetroundjoin%
\pgfsetlinewidth{1.003750pt}%
\definecolor{currentstroke}{rgb}{0.000000,0.000000,0.000000}%
\pgfsetstrokecolor{currentstroke}%
\pgfsetdash{}{0pt}%
\pgfpathmoveto{\pgfqpoint{0.669365in}{0.565471in}}%
\pgfpathlineto{\pgfqpoint{0.881135in}{0.687737in}}%
\pgfusepath{stroke}%
\end{pgfscope}%
\begin{pgfscope}%
\pgfpathrectangle{\pgfqpoint{0.050000in}{0.050000in}}{\pgfqpoint{2.085811in}{1.520004in}}%
\pgfusepath{clip}%
\pgfsetbuttcap%
\pgfsetroundjoin%
\pgfsetlinewidth{1.003750pt}%
\definecolor{currentstroke}{rgb}{0.000000,0.000000,0.000000}%
\pgfsetstrokecolor{currentstroke}%
\pgfsetdash{}{0pt}%
\pgfpathmoveto{\pgfqpoint{0.881135in}{0.198675in}}%
\pgfpathlineto{\pgfqpoint{1.092905in}{0.320940in}}%
\pgfusepath{stroke}%
\end{pgfscope}%
\begin{pgfscope}%
\pgfpathrectangle{\pgfqpoint{0.050000in}{0.050000in}}{\pgfqpoint{2.085811in}{1.520004in}}%
\pgfusepath{clip}%
\pgfsetbuttcap%
\pgfsetroundjoin%
\pgfsetlinewidth{1.003750pt}%
\definecolor{currentstroke}{rgb}{0.000000,0.000000,0.000000}%
\pgfsetstrokecolor{currentstroke}%
\pgfsetdash{}{0pt}%
\pgfpathmoveto{\pgfqpoint{0.457595in}{1.421330in}}%
\pgfpathlineto{\pgfqpoint{0.669365in}{1.299064in}}%
\pgfusepath{stroke}%
\end{pgfscope}%
\begin{pgfscope}%
\pgfpathrectangle{\pgfqpoint{0.050000in}{0.050000in}}{\pgfqpoint{2.085811in}{1.520004in}}%
\pgfusepath{clip}%
\pgfsetbuttcap%
\pgfsetroundjoin%
\pgfsetlinewidth{1.003750pt}%
\definecolor{currentstroke}{rgb}{0.000000,0.000000,0.000000}%
\pgfsetstrokecolor{currentstroke}%
\pgfsetdash{}{0pt}%
\pgfpathmoveto{\pgfqpoint{0.669365in}{1.054533in}}%
\pgfpathlineto{\pgfqpoint{0.881135in}{0.932268in}}%
\pgfusepath{stroke}%
\end{pgfscope}%
\begin{pgfscope}%
\pgfpathrectangle{\pgfqpoint{0.050000in}{0.050000in}}{\pgfqpoint{2.085811in}{1.520004in}}%
\pgfusepath{clip}%
\pgfsetbuttcap%
\pgfsetroundjoin%
\pgfsetlinewidth{1.003750pt}%
\definecolor{currentstroke}{rgb}{0.000000,0.000000,0.000000}%
\pgfsetstrokecolor{currentstroke}%
\pgfsetdash{}{0pt}%
\pgfpathmoveto{\pgfqpoint{0.669365in}{1.054533in}}%
\pgfpathlineto{\pgfqpoint{0.669365in}{1.299064in}}%
\pgfusepath{stroke}%
\end{pgfscope}%
\begin{pgfscope}%
\pgfpathrectangle{\pgfqpoint{0.050000in}{0.050000in}}{\pgfqpoint{2.085811in}{1.520004in}}%
\pgfusepath{clip}%
\pgfsetbuttcap%
\pgfsetroundjoin%
\pgfsetlinewidth{1.003750pt}%
\definecolor{currentstroke}{rgb}{0.000000,0.000000,0.000000}%
\pgfsetstrokecolor{currentstroke}%
\pgfsetdash{}{0pt}%
\pgfpathmoveto{\pgfqpoint{0.669365in}{1.299064in}}%
\pgfpathlineto{\pgfqpoint{0.881135in}{1.421330in}}%
\pgfusepath{stroke}%
\end{pgfscope}%
\begin{pgfscope}%
\pgfpathrectangle{\pgfqpoint{0.050000in}{0.050000in}}{\pgfqpoint{2.085811in}{1.520004in}}%
\pgfusepath{clip}%
\pgfsetbuttcap%
\pgfsetroundjoin%
\pgfsetlinewidth{1.003750pt}%
\definecolor{currentstroke}{rgb}{0.000000,0.000000,0.000000}%
\pgfsetstrokecolor{currentstroke}%
\pgfsetdash{}{0pt}%
\pgfpathmoveto{\pgfqpoint{0.881135in}{0.687737in}}%
\pgfpathlineto{\pgfqpoint{1.092905in}{0.565471in}}%
\pgfusepath{stroke}%
\end{pgfscope}%
\begin{pgfscope}%
\pgfpathrectangle{\pgfqpoint{0.050000in}{0.050000in}}{\pgfqpoint{2.085811in}{1.520004in}}%
\pgfusepath{clip}%
\pgfsetbuttcap%
\pgfsetroundjoin%
\pgfsetlinewidth{1.003750pt}%
\definecolor{currentstroke}{rgb}{0.000000,0.000000,0.000000}%
\pgfsetstrokecolor{currentstroke}%
\pgfsetdash{}{0pt}%
\pgfpathmoveto{\pgfqpoint{0.881135in}{0.687737in}}%
\pgfpathlineto{\pgfqpoint{0.881135in}{0.932268in}}%
\pgfusepath{stroke}%
\end{pgfscope}%
\begin{pgfscope}%
\pgfpathrectangle{\pgfqpoint{0.050000in}{0.050000in}}{\pgfqpoint{2.085811in}{1.520004in}}%
\pgfusepath{clip}%
\pgfsetbuttcap%
\pgfsetroundjoin%
\pgfsetlinewidth{1.003750pt}%
\definecolor{currentstroke}{rgb}{0.000000,0.000000,0.000000}%
\pgfsetstrokecolor{currentstroke}%
\pgfsetdash{}{0pt}%
\pgfpathmoveto{\pgfqpoint{0.881135in}{0.932268in}}%
\pgfpathlineto{\pgfqpoint{1.092905in}{1.054533in}}%
\pgfusepath{stroke}%
\end{pgfscope}%
\begin{pgfscope}%
\pgfpathrectangle{\pgfqpoint{0.050000in}{0.050000in}}{\pgfqpoint{2.085811in}{1.520004in}}%
\pgfusepath{clip}%
\pgfsetbuttcap%
\pgfsetroundjoin%
\pgfsetlinewidth{1.003750pt}%
\definecolor{currentstroke}{rgb}{0.000000,0.000000,0.000000}%
\pgfsetstrokecolor{currentstroke}%
\pgfsetdash{}{0pt}%
\pgfpathmoveto{\pgfqpoint{1.092905in}{0.320940in}}%
\pgfpathlineto{\pgfqpoint{1.304675in}{0.198675in}}%
\pgfusepath{stroke}%
\end{pgfscope}%
\begin{pgfscope}%
\pgfpathrectangle{\pgfqpoint{0.050000in}{0.050000in}}{\pgfqpoint{2.085811in}{1.520004in}}%
\pgfusepath{clip}%
\pgfsetbuttcap%
\pgfsetroundjoin%
\pgfsetlinewidth{1.003750pt}%
\definecolor{currentstroke}{rgb}{0.000000,0.000000,0.000000}%
\pgfsetstrokecolor{currentstroke}%
\pgfsetdash{}{0pt}%
\pgfpathmoveto{\pgfqpoint{1.092905in}{0.320940in}}%
\pgfpathlineto{\pgfqpoint{1.092905in}{0.565471in}}%
\pgfusepath{stroke}%
\end{pgfscope}%
\begin{pgfscope}%
\pgfpathrectangle{\pgfqpoint{0.050000in}{0.050000in}}{\pgfqpoint{2.085811in}{1.520004in}}%
\pgfusepath{clip}%
\pgfsetbuttcap%
\pgfsetroundjoin%
\pgfsetlinewidth{1.003750pt}%
\definecolor{currentstroke}{rgb}{0.000000,0.000000,0.000000}%
\pgfsetstrokecolor{currentstroke}%
\pgfsetdash{}{0pt}%
\pgfpathmoveto{\pgfqpoint{1.092905in}{0.565471in}}%
\pgfpathlineto{\pgfqpoint{1.304675in}{0.687737in}}%
\pgfusepath{stroke}%
\end{pgfscope}%
\begin{pgfscope}%
\pgfpathrectangle{\pgfqpoint{0.050000in}{0.050000in}}{\pgfqpoint{2.085811in}{1.520004in}}%
\pgfusepath{clip}%
\pgfsetbuttcap%
\pgfsetroundjoin%
\pgfsetlinewidth{1.003750pt}%
\definecolor{currentstroke}{rgb}{0.000000,0.000000,0.000000}%
\pgfsetstrokecolor{currentstroke}%
\pgfsetdash{}{0pt}%
\pgfpathmoveto{\pgfqpoint{1.304675in}{0.198675in}}%
\pgfpathlineto{\pgfqpoint{1.516445in}{0.320940in}}%
\pgfusepath{stroke}%
\end{pgfscope}%
\begin{pgfscope}%
\pgfpathrectangle{\pgfqpoint{0.050000in}{0.050000in}}{\pgfqpoint{2.085811in}{1.520004in}}%
\pgfusepath{clip}%
\pgfsetbuttcap%
\pgfsetroundjoin%
\pgfsetlinewidth{1.003750pt}%
\definecolor{currentstroke}{rgb}{0.000000,0.000000,0.000000}%
\pgfsetstrokecolor{currentstroke}%
\pgfsetdash{}{0pt}%
\pgfpathmoveto{\pgfqpoint{0.881135in}{1.421330in}}%
\pgfpathlineto{\pgfqpoint{1.092905in}{1.299064in}}%
\pgfusepath{stroke}%
\end{pgfscope}%
\begin{pgfscope}%
\pgfpathrectangle{\pgfqpoint{0.050000in}{0.050000in}}{\pgfqpoint{2.085811in}{1.520004in}}%
\pgfusepath{clip}%
\pgfsetbuttcap%
\pgfsetroundjoin%
\pgfsetlinewidth{1.003750pt}%
\definecolor{currentstroke}{rgb}{0.000000,0.000000,0.000000}%
\pgfsetstrokecolor{currentstroke}%
\pgfsetdash{}{0pt}%
\pgfpathmoveto{\pgfqpoint{1.092905in}{1.054533in}}%
\pgfpathlineto{\pgfqpoint{1.304675in}{0.932268in}}%
\pgfusepath{stroke}%
\end{pgfscope}%
\begin{pgfscope}%
\pgfpathrectangle{\pgfqpoint{0.050000in}{0.050000in}}{\pgfqpoint{2.085811in}{1.520004in}}%
\pgfusepath{clip}%
\pgfsetbuttcap%
\pgfsetroundjoin%
\pgfsetlinewidth{1.003750pt}%
\definecolor{currentstroke}{rgb}{0.000000,0.000000,0.000000}%
\pgfsetstrokecolor{currentstroke}%
\pgfsetdash{}{0pt}%
\pgfpathmoveto{\pgfqpoint{1.092905in}{1.054533in}}%
\pgfpathlineto{\pgfqpoint{1.092905in}{1.299064in}}%
\pgfusepath{stroke}%
\end{pgfscope}%
\begin{pgfscope}%
\pgfpathrectangle{\pgfqpoint{0.050000in}{0.050000in}}{\pgfqpoint{2.085811in}{1.520004in}}%
\pgfusepath{clip}%
\pgfsetbuttcap%
\pgfsetroundjoin%
\pgfsetlinewidth{1.003750pt}%
\definecolor{currentstroke}{rgb}{0.000000,0.000000,0.000000}%
\pgfsetstrokecolor{currentstroke}%
\pgfsetdash{}{0pt}%
\pgfpathmoveto{\pgfqpoint{1.092905in}{1.299064in}}%
\pgfpathlineto{\pgfqpoint{1.304675in}{1.421330in}}%
\pgfusepath{stroke}%
\end{pgfscope}%
\begin{pgfscope}%
\pgfpathrectangle{\pgfqpoint{0.050000in}{0.050000in}}{\pgfqpoint{2.085811in}{1.520004in}}%
\pgfusepath{clip}%
\pgfsetbuttcap%
\pgfsetroundjoin%
\pgfsetlinewidth{1.003750pt}%
\definecolor{currentstroke}{rgb}{0.000000,0.000000,0.000000}%
\pgfsetstrokecolor{currentstroke}%
\pgfsetdash{}{0pt}%
\pgfpathmoveto{\pgfqpoint{1.304675in}{0.687737in}}%
\pgfpathlineto{\pgfqpoint{1.516445in}{0.565471in}}%
\pgfusepath{stroke}%
\end{pgfscope}%
\begin{pgfscope}%
\pgfpathrectangle{\pgfqpoint{0.050000in}{0.050000in}}{\pgfqpoint{2.085811in}{1.520004in}}%
\pgfusepath{clip}%
\pgfsetbuttcap%
\pgfsetroundjoin%
\pgfsetlinewidth{1.003750pt}%
\definecolor{currentstroke}{rgb}{0.000000,0.000000,0.000000}%
\pgfsetstrokecolor{currentstroke}%
\pgfsetdash{}{0pt}%
\pgfpathmoveto{\pgfqpoint{1.304675in}{0.687737in}}%
\pgfpathlineto{\pgfqpoint{1.304675in}{0.932268in}}%
\pgfusepath{stroke}%
\end{pgfscope}%
\begin{pgfscope}%
\pgfpathrectangle{\pgfqpoint{0.050000in}{0.050000in}}{\pgfqpoint{2.085811in}{1.520004in}}%
\pgfusepath{clip}%
\pgfsetbuttcap%
\pgfsetroundjoin%
\pgfsetlinewidth{1.003750pt}%
\definecolor{currentstroke}{rgb}{0.000000,0.000000,0.000000}%
\pgfsetstrokecolor{currentstroke}%
\pgfsetdash{}{0pt}%
\pgfpathmoveto{\pgfqpoint{1.304675in}{0.932268in}}%
\pgfpathlineto{\pgfqpoint{1.516445in}{1.054533in}}%
\pgfusepath{stroke}%
\end{pgfscope}%
\begin{pgfscope}%
\pgfpathrectangle{\pgfqpoint{0.050000in}{0.050000in}}{\pgfqpoint{2.085811in}{1.520004in}}%
\pgfusepath{clip}%
\pgfsetbuttcap%
\pgfsetroundjoin%
\pgfsetlinewidth{1.003750pt}%
\definecolor{currentstroke}{rgb}{0.000000,0.000000,0.000000}%
\pgfsetstrokecolor{currentstroke}%
\pgfsetdash{}{0pt}%
\pgfpathmoveto{\pgfqpoint{1.516445in}{0.320940in}}%
\pgfpathlineto{\pgfqpoint{1.516445in}{0.565471in}}%
\pgfusepath{stroke}%
\end{pgfscope}%
\begin{pgfscope}%
\pgfpathrectangle{\pgfqpoint{0.050000in}{0.050000in}}{\pgfqpoint{2.085811in}{1.520004in}}%
\pgfusepath{clip}%
\pgfsetbuttcap%
\pgfsetroundjoin%
\pgfsetlinewidth{1.003750pt}%
\definecolor{currentstroke}{rgb}{0.000000,0.000000,0.000000}%
\pgfsetstrokecolor{currentstroke}%
\pgfsetdash{}{0pt}%
\pgfpathmoveto{\pgfqpoint{1.516445in}{0.320940in}}%
\pgfpathlineto{\pgfqpoint{1.728215in}{0.198675in}}%
\pgfusepath{stroke}%
\end{pgfscope}%
\begin{pgfscope}%
\pgfpathrectangle{\pgfqpoint{0.050000in}{0.050000in}}{\pgfqpoint{2.085811in}{1.520004in}}%
\pgfusepath{clip}%
\pgfsetbuttcap%
\pgfsetroundjoin%
\pgfsetlinewidth{1.003750pt}%
\definecolor{currentstroke}{rgb}{0.000000,0.000000,0.000000}%
\pgfsetstrokecolor{currentstroke}%
\pgfsetdash{}{0pt}%
\pgfpathmoveto{\pgfqpoint{1.516445in}{0.565471in}}%
\pgfpathlineto{\pgfqpoint{1.728215in}{0.687737in}}%
\pgfusepath{stroke}%
\end{pgfscope}%
\begin{pgfscope}%
\pgfpathrectangle{\pgfqpoint{0.050000in}{0.050000in}}{\pgfqpoint{2.085811in}{1.520004in}}%
\pgfusepath{clip}%
\pgfsetbuttcap%
\pgfsetroundjoin%
\pgfsetlinewidth{1.003750pt}%
\definecolor{currentstroke}{rgb}{0.000000,0.000000,0.000000}%
\pgfsetstrokecolor{currentstroke}%
\pgfsetdash{}{0pt}%
\pgfpathmoveto{\pgfqpoint{1.728215in}{0.198675in}}%
\pgfpathlineto{\pgfqpoint{1.939985in}{0.320940in}}%
\pgfusepath{stroke}%
\end{pgfscope}%
\begin{pgfscope}%
\pgfpathrectangle{\pgfqpoint{0.050000in}{0.050000in}}{\pgfqpoint{2.085811in}{1.520004in}}%
\pgfusepath{clip}%
\pgfsetbuttcap%
\pgfsetroundjoin%
\pgfsetlinewidth{1.003750pt}%
\definecolor{currentstroke}{rgb}{0.000000,0.000000,0.000000}%
\pgfsetstrokecolor{currentstroke}%
\pgfsetdash{}{0pt}%
\pgfpathmoveto{\pgfqpoint{1.304675in}{1.421330in}}%
\pgfpathlineto{\pgfqpoint{1.516445in}{1.299064in}}%
\pgfusepath{stroke}%
\end{pgfscope}%
\begin{pgfscope}%
\pgfpathrectangle{\pgfqpoint{0.050000in}{0.050000in}}{\pgfqpoint{2.085811in}{1.520004in}}%
\pgfusepath{clip}%
\pgfsetbuttcap%
\pgfsetroundjoin%
\pgfsetlinewidth{1.003750pt}%
\definecolor{currentstroke}{rgb}{0.000000,0.000000,0.000000}%
\pgfsetstrokecolor{currentstroke}%
\pgfsetdash{}{0pt}%
\pgfpathmoveto{\pgfqpoint{1.516445in}{1.054533in}}%
\pgfpathlineto{\pgfqpoint{1.516445in}{1.299064in}}%
\pgfusepath{stroke}%
\end{pgfscope}%
\begin{pgfscope}%
\pgfpathrectangle{\pgfqpoint{0.050000in}{0.050000in}}{\pgfqpoint{2.085811in}{1.520004in}}%
\pgfusepath{clip}%
\pgfsetbuttcap%
\pgfsetroundjoin%
\pgfsetlinewidth{1.003750pt}%
\definecolor{currentstroke}{rgb}{0.000000,0.000000,0.000000}%
\pgfsetstrokecolor{currentstroke}%
\pgfsetdash{}{0pt}%
\pgfpathmoveto{\pgfqpoint{1.516445in}{1.054533in}}%
\pgfpathlineto{\pgfqpoint{1.728215in}{0.932268in}}%
\pgfusepath{stroke}%
\end{pgfscope}%
\begin{pgfscope}%
\pgfpathrectangle{\pgfqpoint{0.050000in}{0.050000in}}{\pgfqpoint{2.085811in}{1.520004in}}%
\pgfusepath{clip}%
\pgfsetbuttcap%
\pgfsetroundjoin%
\pgfsetlinewidth{1.003750pt}%
\definecolor{currentstroke}{rgb}{0.000000,0.000000,0.000000}%
\pgfsetstrokecolor{currentstroke}%
\pgfsetdash{}{0pt}%
\pgfpathmoveto{\pgfqpoint{1.516445in}{1.299064in}}%
\pgfpathlineto{\pgfqpoint{1.728215in}{1.421330in}}%
\pgfusepath{stroke}%
\end{pgfscope}%
\begin{pgfscope}%
\pgfpathrectangle{\pgfqpoint{0.050000in}{0.050000in}}{\pgfqpoint{2.085811in}{1.520004in}}%
\pgfusepath{clip}%
\pgfsetbuttcap%
\pgfsetroundjoin%
\pgfsetlinewidth{1.003750pt}%
\definecolor{currentstroke}{rgb}{0.000000,0.000000,0.000000}%
\pgfsetstrokecolor{currentstroke}%
\pgfsetdash{}{0pt}%
\pgfpathmoveto{\pgfqpoint{1.728215in}{0.687737in}}%
\pgfpathlineto{\pgfqpoint{1.939985in}{0.565471in}}%
\pgfusepath{stroke}%
\end{pgfscope}%
\begin{pgfscope}%
\pgfpathrectangle{\pgfqpoint{0.050000in}{0.050000in}}{\pgfqpoint{2.085811in}{1.520004in}}%
\pgfusepath{clip}%
\pgfsetbuttcap%
\pgfsetroundjoin%
\pgfsetlinewidth{1.003750pt}%
\definecolor{currentstroke}{rgb}{0.000000,0.000000,0.000000}%
\pgfsetstrokecolor{currentstroke}%
\pgfsetdash{}{0pt}%
\pgfpathmoveto{\pgfqpoint{1.728215in}{0.687737in}}%
\pgfpathlineto{\pgfqpoint{1.728215in}{0.932268in}}%
\pgfusepath{stroke}%
\end{pgfscope}%
\begin{pgfscope}%
\pgfpathrectangle{\pgfqpoint{0.050000in}{0.050000in}}{\pgfqpoint{2.085811in}{1.520004in}}%
\pgfusepath{clip}%
\pgfsetbuttcap%
\pgfsetroundjoin%
\pgfsetlinewidth{1.003750pt}%
\definecolor{currentstroke}{rgb}{0.000000,0.000000,0.000000}%
\pgfsetstrokecolor{currentstroke}%
\pgfsetdash{}{0pt}%
\pgfpathmoveto{\pgfqpoint{1.728215in}{0.932268in}}%
\pgfpathlineto{\pgfqpoint{1.939985in}{1.054533in}}%
\pgfusepath{stroke}%
\end{pgfscope}%
\begin{pgfscope}%
\pgfpathrectangle{\pgfqpoint{0.050000in}{0.050000in}}{\pgfqpoint{2.085811in}{1.520004in}}%
\pgfusepath{clip}%
\pgfsetbuttcap%
\pgfsetroundjoin%
\pgfsetlinewidth{1.003750pt}%
\definecolor{currentstroke}{rgb}{0.000000,0.000000,0.000000}%
\pgfsetstrokecolor{currentstroke}%
\pgfsetdash{}{0pt}%
\pgfpathmoveto{\pgfqpoint{1.939985in}{0.320940in}}%
\pgfpathlineto{\pgfqpoint{1.939985in}{0.565471in}}%
\pgfusepath{stroke}%
\end{pgfscope}%
\begin{pgfscope}%
\pgfpathrectangle{\pgfqpoint{0.050000in}{0.050000in}}{\pgfqpoint{2.085811in}{1.520004in}}%
\pgfusepath{clip}%
\pgfsetbuttcap%
\pgfsetroundjoin%
\pgfsetlinewidth{1.003750pt}%
\definecolor{currentstroke}{rgb}{0.000000,0.000000,0.000000}%
\pgfsetstrokecolor{currentstroke}%
\pgfsetdash{}{0pt}%
\pgfpathmoveto{\pgfqpoint{1.728215in}{1.421330in}}%
\pgfpathlineto{\pgfqpoint{1.939985in}{1.299064in}}%
\pgfusepath{stroke}%
\end{pgfscope}%
\begin{pgfscope}%
\pgfpathrectangle{\pgfqpoint{0.050000in}{0.050000in}}{\pgfqpoint{2.085811in}{1.520004in}}%
\pgfusepath{clip}%
\pgfsetbuttcap%
\pgfsetroundjoin%
\pgfsetlinewidth{1.003750pt}%
\definecolor{currentstroke}{rgb}{0.000000,0.000000,0.000000}%
\pgfsetstrokecolor{currentstroke}%
\pgfsetdash{}{0pt}%
\pgfpathmoveto{\pgfqpoint{1.939985in}{1.054533in}}%
\pgfpathlineto{\pgfqpoint{1.939985in}{1.299064in}}%
\pgfusepath{stroke}%
\end{pgfscope}%
\begin{pgfscope}%
\pgfpathrectangle{\pgfqpoint{0.050000in}{0.050000in}}{\pgfqpoint{2.085811in}{1.520004in}}%
\pgfusepath{clip}%
\pgfsetbuttcap%
\pgfsetroundjoin%
\definecolor{currentfill}{rgb}{0.800000,0.400000,0.466667}%
\pgfsetfillcolor{currentfill}%
\pgfsetlinewidth{1.003750pt}%
\definecolor{currentstroke}{rgb}{0.800000,0.400000,0.466667}%
\pgfsetstrokecolor{currentstroke}%
\pgfsetdash{}{0pt}%
\pgfsys@defobject{currentmarker}{\pgfqpoint{-0.022008in}{-0.022008in}}{\pgfqpoint{0.022008in}{0.022008in}}{%
\pgfpathmoveto{\pgfqpoint{0.000000in}{-0.022008in}}%
\pgfpathcurveto{\pgfqpoint{0.005837in}{-0.022008in}}{\pgfqpoint{0.011435in}{-0.019689in}}{\pgfqpoint{0.015562in}{-0.015562in}}%
\pgfpathcurveto{\pgfqpoint{0.019689in}{-0.011435in}}{\pgfqpoint{0.022008in}{-0.005837in}}{\pgfqpoint{0.022008in}{0.000000in}}%
\pgfpathcurveto{\pgfqpoint{0.022008in}{0.005837in}}{\pgfqpoint{0.019689in}{0.011435in}}{\pgfqpoint{0.015562in}{0.015562in}}%
\pgfpathcurveto{\pgfqpoint{0.011435in}{0.019689in}}{\pgfqpoint{0.005837in}{0.022008in}}{\pgfqpoint{0.000000in}{0.022008in}}%
\pgfpathcurveto{\pgfqpoint{-0.005837in}{0.022008in}}{\pgfqpoint{-0.011435in}{0.019689in}}{\pgfqpoint{-0.015562in}{0.015562in}}%
\pgfpathcurveto{\pgfqpoint{-0.019689in}{0.011435in}}{\pgfqpoint{-0.022008in}{0.005837in}}{\pgfqpoint{-0.022008in}{0.000000in}}%
\pgfpathcurveto{\pgfqpoint{-0.022008in}{-0.005837in}}{\pgfqpoint{-0.019689in}{-0.011435in}}{\pgfqpoint{-0.015562in}{-0.015562in}}%
\pgfpathcurveto{\pgfqpoint{-0.011435in}{-0.019689in}}{\pgfqpoint{-0.005837in}{-0.022008in}}{\pgfqpoint{0.000000in}{-0.022008in}}%
\pgfpathlineto{\pgfqpoint{0.000000in}{-0.022008in}}%
\pgfpathclose%
\pgfusepath{stroke,fill}%
}%
\begin{pgfscope}%
\pgfsys@transformshift{0.245825in}{0.320940in}%
\pgfsys@useobject{currentmarker}{}%
\end{pgfscope}%
\begin{pgfscope}%
\pgfsys@transformshift{0.245825in}{1.054533in}%
\pgfsys@useobject{currentmarker}{}%
\end{pgfscope}%
\begin{pgfscope}%
\pgfsys@transformshift{0.457595in}{0.687737in}%
\pgfsys@useobject{currentmarker}{}%
\end{pgfscope}%
\begin{pgfscope}%
\pgfsys@transformshift{0.669365in}{0.320940in}%
\pgfsys@useobject{currentmarker}{}%
\end{pgfscope}%
\begin{pgfscope}%
\pgfsys@transformshift{0.457595in}{1.421330in}%
\pgfsys@useobject{currentmarker}{}%
\end{pgfscope}%
\begin{pgfscope}%
\pgfsys@transformshift{0.669365in}{1.054533in}%
\pgfsys@useobject{currentmarker}{}%
\end{pgfscope}%
\begin{pgfscope}%
\pgfsys@transformshift{0.881135in}{0.687737in}%
\pgfsys@useobject{currentmarker}{}%
\end{pgfscope}%
\begin{pgfscope}%
\pgfsys@transformshift{1.092905in}{0.320940in}%
\pgfsys@useobject{currentmarker}{}%
\end{pgfscope}%
\begin{pgfscope}%
\pgfsys@transformshift{0.881135in}{1.421330in}%
\pgfsys@useobject{currentmarker}{}%
\end{pgfscope}%
\begin{pgfscope}%
\pgfsys@transformshift{1.092905in}{1.054533in}%
\pgfsys@useobject{currentmarker}{}%
\end{pgfscope}%
\begin{pgfscope}%
\pgfsys@transformshift{1.304675in}{0.687737in}%
\pgfsys@useobject{currentmarker}{}%
\end{pgfscope}%
\begin{pgfscope}%
\pgfsys@transformshift{1.516445in}{0.320940in}%
\pgfsys@useobject{currentmarker}{}%
\end{pgfscope}%
\begin{pgfscope}%
\pgfsys@transformshift{1.304675in}{1.421330in}%
\pgfsys@useobject{currentmarker}{}%
\end{pgfscope}%
\begin{pgfscope}%
\pgfsys@transformshift{1.516445in}{1.054533in}%
\pgfsys@useobject{currentmarker}{}%
\end{pgfscope}%
\begin{pgfscope}%
\pgfsys@transformshift{1.728215in}{0.687737in}%
\pgfsys@useobject{currentmarker}{}%
\end{pgfscope}%
\begin{pgfscope}%
\pgfsys@transformshift{1.939985in}{0.320940in}%
\pgfsys@useobject{currentmarker}{}%
\end{pgfscope}%
\begin{pgfscope}%
\pgfsys@transformshift{1.728215in}{1.421330in}%
\pgfsys@useobject{currentmarker}{}%
\end{pgfscope}%
\begin{pgfscope}%
\pgfsys@transformshift{1.939985in}{1.054533in}%
\pgfsys@useobject{currentmarker}{}%
\end{pgfscope}%
\end{pgfscope}%
\begin{pgfscope}%
\pgfpathrectangle{\pgfqpoint{0.050000in}{0.050000in}}{\pgfqpoint{2.085811in}{1.520004in}}%
\pgfusepath{clip}%
\pgfsetbuttcap%
\pgfsetroundjoin%
\definecolor{currentfill}{rgb}{0.200000,0.133333,0.533333}%
\pgfsetfillcolor{currentfill}%
\pgfsetlinewidth{1.003750pt}%
\definecolor{currentstroke}{rgb}{0.200000,0.133333,0.533333}%
\pgfsetstrokecolor{currentstroke}%
\pgfsetdash{}{0pt}%
\pgfsys@defobject{currentmarker}{\pgfqpoint{-0.022008in}{-0.022008in}}{\pgfqpoint{0.022008in}{0.022008in}}{%
\pgfpathmoveto{\pgfqpoint{0.000000in}{-0.022008in}}%
\pgfpathcurveto{\pgfqpoint{0.005837in}{-0.022008in}}{\pgfqpoint{0.011435in}{-0.019689in}}{\pgfqpoint{0.015562in}{-0.015562in}}%
\pgfpathcurveto{\pgfqpoint{0.019689in}{-0.011435in}}{\pgfqpoint{0.022008in}{-0.005837in}}{\pgfqpoint{0.022008in}{0.000000in}}%
\pgfpathcurveto{\pgfqpoint{0.022008in}{0.005837in}}{\pgfqpoint{0.019689in}{0.011435in}}{\pgfqpoint{0.015562in}{0.015562in}}%
\pgfpathcurveto{\pgfqpoint{0.011435in}{0.019689in}}{\pgfqpoint{0.005837in}{0.022008in}}{\pgfqpoint{0.000000in}{0.022008in}}%
\pgfpathcurveto{\pgfqpoint{-0.005837in}{0.022008in}}{\pgfqpoint{-0.011435in}{0.019689in}}{\pgfqpoint{-0.015562in}{0.015562in}}%
\pgfpathcurveto{\pgfqpoint{-0.019689in}{0.011435in}}{\pgfqpoint{-0.022008in}{0.005837in}}{\pgfqpoint{-0.022008in}{0.000000in}}%
\pgfpathcurveto{\pgfqpoint{-0.022008in}{-0.005837in}}{\pgfqpoint{-0.019689in}{-0.011435in}}{\pgfqpoint{-0.015562in}{-0.015562in}}%
\pgfpathcurveto{\pgfqpoint{-0.011435in}{-0.019689in}}{\pgfqpoint{-0.005837in}{-0.022008in}}{\pgfqpoint{0.000000in}{-0.022008in}}%
\pgfpathlineto{\pgfqpoint{0.000000in}{-0.022008in}}%
\pgfpathclose%
\pgfusepath{stroke,fill}%
}%
\begin{pgfscope}%
\pgfsys@transformshift{0.245825in}{0.565471in}%
\pgfsys@useobject{currentmarker}{}%
\end{pgfscope}%
\begin{pgfscope}%
\pgfsys@transformshift{0.457595in}{0.198675in}%
\pgfsys@useobject{currentmarker}{}%
\end{pgfscope}%
\begin{pgfscope}%
\pgfsys@transformshift{0.245825in}{1.299064in}%
\pgfsys@useobject{currentmarker}{}%
\end{pgfscope}%
\begin{pgfscope}%
\pgfsys@transformshift{0.457595in}{0.932268in}%
\pgfsys@useobject{currentmarker}{}%
\end{pgfscope}%
\begin{pgfscope}%
\pgfsys@transformshift{0.669365in}{0.565471in}%
\pgfsys@useobject{currentmarker}{}%
\end{pgfscope}%
\begin{pgfscope}%
\pgfsys@transformshift{0.881135in}{0.198675in}%
\pgfsys@useobject{currentmarker}{}%
\end{pgfscope}%
\begin{pgfscope}%
\pgfsys@transformshift{0.669365in}{1.299064in}%
\pgfsys@useobject{currentmarker}{}%
\end{pgfscope}%
\begin{pgfscope}%
\pgfsys@transformshift{0.881135in}{0.932268in}%
\pgfsys@useobject{currentmarker}{}%
\end{pgfscope}%
\begin{pgfscope}%
\pgfsys@transformshift{1.092905in}{0.565471in}%
\pgfsys@useobject{currentmarker}{}%
\end{pgfscope}%
\begin{pgfscope}%
\pgfsys@transformshift{1.304675in}{0.198675in}%
\pgfsys@useobject{currentmarker}{}%
\end{pgfscope}%
\begin{pgfscope}%
\pgfsys@transformshift{1.092905in}{1.299064in}%
\pgfsys@useobject{currentmarker}{}%
\end{pgfscope}%
\begin{pgfscope}%
\pgfsys@transformshift{1.304675in}{0.932268in}%
\pgfsys@useobject{currentmarker}{}%
\end{pgfscope}%
\begin{pgfscope}%
\pgfsys@transformshift{1.516445in}{0.565471in}%
\pgfsys@useobject{currentmarker}{}%
\end{pgfscope}%
\begin{pgfscope}%
\pgfsys@transformshift{1.728215in}{0.198675in}%
\pgfsys@useobject{currentmarker}{}%
\end{pgfscope}%
\begin{pgfscope}%
\pgfsys@transformshift{1.516445in}{1.299064in}%
\pgfsys@useobject{currentmarker}{}%
\end{pgfscope}%
\begin{pgfscope}%
\pgfsys@transformshift{1.728215in}{0.932268in}%
\pgfsys@useobject{currentmarker}{}%
\end{pgfscope}%
\begin{pgfscope}%
\pgfsys@transformshift{1.939985in}{0.565471in}%
\pgfsys@useobject{currentmarker}{}%
\end{pgfscope}%
\begin{pgfscope}%
\pgfsys@transformshift{1.939985in}{1.299064in}%
\pgfsys@useobject{currentmarker}{}%
\end{pgfscope}%
\end{pgfscope}%
\begin{pgfscope}%
\pgfpathrectangle{\pgfqpoint{0.050000in}{0.050000in}}{\pgfqpoint{2.085811in}{1.520004in}}%
\pgfusepath{clip}%
\pgfsetbuttcap%
\pgfsetroundjoin%
\definecolor{currentfill}{rgb}{0.800000,0.400000,0.466667}%
\pgfsetfillcolor{currentfill}%
\pgfsetlinewidth{1.003750pt}%
\definecolor{currentstroke}{rgb}{0.800000,0.400000,0.466667}%
\pgfsetstrokecolor{currentstroke}%
\pgfsetdash{}{0pt}%
\pgfsys@defobject{currentmarker}{\pgfqpoint{0.000000in}{0.000000in}}{\pgfqpoint{0.000000in}{0.000000in}}{%
\pgfpathmoveto{\pgfqpoint{0.000000in}{0.000000in}}%
\pgfpathcurveto{\pgfqpoint{0.000000in}{0.000000in}}{\pgfqpoint{0.000000in}{0.000000in}}{\pgfqpoint{0.000000in}{0.000000in}}%
\pgfpathcurveto{\pgfqpoint{0.000000in}{0.000000in}}{\pgfqpoint{0.000000in}{0.000000in}}{\pgfqpoint{0.000000in}{0.000000in}}%
\pgfpathcurveto{\pgfqpoint{0.000000in}{0.000000in}}{\pgfqpoint{0.000000in}{0.000000in}}{\pgfqpoint{0.000000in}{0.000000in}}%
\pgfpathcurveto{\pgfqpoint{0.000000in}{0.000000in}}{\pgfqpoint{0.000000in}{0.000000in}}{\pgfqpoint{0.000000in}{0.000000in}}%
\pgfpathcurveto{\pgfqpoint{0.000000in}{0.000000in}}{\pgfqpoint{0.000000in}{0.000000in}}{\pgfqpoint{0.000000in}{0.000000in}}%
\pgfpathcurveto{\pgfqpoint{0.000000in}{0.000000in}}{\pgfqpoint{0.000000in}{0.000000in}}{\pgfqpoint{0.000000in}{0.000000in}}%
\pgfpathcurveto{\pgfqpoint{0.000000in}{0.000000in}}{\pgfqpoint{0.000000in}{0.000000in}}{\pgfqpoint{0.000000in}{0.000000in}}%
\pgfpathcurveto{\pgfqpoint{0.000000in}{0.000000in}}{\pgfqpoint{0.000000in}{0.000000in}}{\pgfqpoint{0.000000in}{0.000000in}}%
\pgfpathlineto{\pgfqpoint{0.000000in}{0.000000in}}%
\pgfpathclose%
\pgfusepath{stroke,fill}%
}%
\begin{pgfscope}%
\pgfsys@transformshift{1.304675in}{1.176799in}%
\pgfsys@useobject{currentmarker}{}%
\end{pgfscope}%
\end{pgfscope}%
\begin{pgfscope}%
\pgfpathrectangle{\pgfqpoint{0.050000in}{0.050000in}}{\pgfqpoint{2.085811in}{1.520004in}}%
\pgfusepath{clip}%
\pgfsetbuttcap%
\pgfsetroundjoin%
\definecolor{currentfill}{rgb}{0.000000,0.000000,0.000000}%
\pgfsetfillcolor{currentfill}%
\pgfsetlinewidth{0.000000pt}%
\definecolor{currentstroke}{rgb}{0.000000,0.000000,0.000000}%
\pgfsetstrokecolor{currentstroke}%
\pgfsetdash{}{0pt}%
\pgfpathmoveto{\pgfqpoint{1.086132in}{0.813913in}}%
\pgfpathlineto{\pgfqpoint{1.262703in}{1.119745in}}%
\pgfpathlineto{\pgfqpoint{1.245245in}{1.120793in}}%
\pgfpathlineto{\pgfqpoint{1.304675in}{1.176799in}}%
\pgfpathlineto{\pgfqpoint{1.285888in}{1.097327in}}%
\pgfpathlineto{\pgfqpoint{1.276251in}{1.111923in}}%
\pgfpathlineto{\pgfqpoint{1.099679in}{0.806091in}}%
\pgfpathlineto{\pgfqpoint{1.086132in}{0.813913in}}%
\pgfusepath{fill}%
\end{pgfscope}%
\begin{pgfscope}%
\pgfpathrectangle{\pgfqpoint{0.050000in}{0.050000in}}{\pgfqpoint{2.085811in}{1.520004in}}%
\pgfusepath{clip}%
\pgfsetbuttcap%
\pgfsetroundjoin%
\definecolor{currentfill}{rgb}{0.200000,0.133333,0.533333}%
\pgfsetfillcolor{currentfill}%
\pgfsetlinewidth{1.003750pt}%
\definecolor{currentstroke}{rgb}{0.200000,0.133333,0.533333}%
\pgfsetstrokecolor{currentstroke}%
\pgfsetdash{}{0pt}%
\pgfsys@defobject{currentmarker}{\pgfqpoint{0.000000in}{0.000000in}}{\pgfqpoint{0.000000in}{0.000000in}}{%
\pgfpathmoveto{\pgfqpoint{0.000000in}{0.000000in}}%
\pgfpathcurveto{\pgfqpoint{0.000000in}{0.000000in}}{\pgfqpoint{0.000000in}{0.000000in}}{\pgfqpoint{0.000000in}{0.000000in}}%
\pgfpathcurveto{\pgfqpoint{0.000000in}{0.000000in}}{\pgfqpoint{0.000000in}{0.000000in}}{\pgfqpoint{0.000000in}{0.000000in}}%
\pgfpathcurveto{\pgfqpoint{0.000000in}{0.000000in}}{\pgfqpoint{0.000000in}{0.000000in}}{\pgfqpoint{0.000000in}{0.000000in}}%
\pgfpathcurveto{\pgfqpoint{0.000000in}{0.000000in}}{\pgfqpoint{0.000000in}{0.000000in}}{\pgfqpoint{0.000000in}{0.000000in}}%
\pgfpathcurveto{\pgfqpoint{0.000000in}{0.000000in}}{\pgfqpoint{0.000000in}{0.000000in}}{\pgfqpoint{0.000000in}{0.000000in}}%
\pgfpathcurveto{\pgfqpoint{0.000000in}{0.000000in}}{\pgfqpoint{0.000000in}{0.000000in}}{\pgfqpoint{0.000000in}{0.000000in}}%
\pgfpathcurveto{\pgfqpoint{0.000000in}{0.000000in}}{\pgfqpoint{0.000000in}{0.000000in}}{\pgfqpoint{0.000000in}{0.000000in}}%
\pgfpathcurveto{\pgfqpoint{0.000000in}{0.000000in}}{\pgfqpoint{0.000000in}{0.000000in}}{\pgfqpoint{0.000000in}{0.000000in}}%
\pgfpathlineto{\pgfqpoint{0.000000in}{0.000000in}}%
\pgfpathclose%
\pgfusepath{stroke,fill}%
}%
\begin{pgfscope}%
\pgfsys@transformshift{1.304675in}{0.443206in}%
\pgfsys@useobject{currentmarker}{}%
\end{pgfscope}%
\end{pgfscope}%
\begin{pgfscope}%
\pgfpathrectangle{\pgfqpoint{0.050000in}{0.050000in}}{\pgfqpoint{2.085811in}{1.520004in}}%
\pgfusepath{clip}%
\pgfsetbuttcap%
\pgfsetroundjoin%
\definecolor{currentfill}{rgb}{0.000000,0.000000,0.000000}%
\pgfsetfillcolor{currentfill}%
\pgfsetlinewidth{0.000000pt}%
\definecolor{currentstroke}{rgb}{0.000000,0.000000,0.000000}%
\pgfsetstrokecolor{currentstroke}%
\pgfsetdash{}{0pt}%
\pgfpathmoveto{\pgfqpoint{1.099679in}{0.813913in}}%
\pgfpathlineto{\pgfqpoint{1.276251in}{0.508081in}}%
\pgfpathlineto{\pgfqpoint{1.285888in}{0.522677in}}%
\pgfpathlineto{\pgfqpoint{1.304675in}{0.443206in}}%
\pgfpathlineto{\pgfqpoint{1.245245in}{0.499212in}}%
\pgfpathlineto{\pgfqpoint{1.262703in}{0.500260in}}%
\pgfpathlineto{\pgfqpoint{1.086132in}{0.806091in}}%
\pgfpathlineto{\pgfqpoint{1.099679in}{0.813913in}}%
\pgfusepath{fill}%
\end{pgfscope}%
\begin{pgfscope}%
\pgfpathrectangle{\pgfqpoint{0.050000in}{0.050000in}}{\pgfqpoint{2.085811in}{1.520004in}}%
\pgfusepath{clip}%
\pgfsetbuttcap%
\pgfsetroundjoin%
\definecolor{currentfill}{rgb}{0.866667,0.800000,0.466667}%
\pgfsetfillcolor{currentfill}%
\pgfsetlinewidth{1.003750pt}%
\definecolor{currentstroke}{rgb}{0.866667,0.800000,0.466667}%
\pgfsetstrokecolor{currentstroke}%
\pgfsetdash{}{0pt}%
\pgfsys@defobject{currentmarker}{\pgfqpoint{0.000000in}{0.000000in}}{\pgfqpoint{0.000000in}{0.000000in}}{%
\pgfpathmoveto{\pgfqpoint{0.000000in}{0.000000in}}%
\pgfpathcurveto{\pgfqpoint{0.000000in}{0.000000in}}{\pgfqpoint{0.000000in}{0.000000in}}{\pgfqpoint{0.000000in}{0.000000in}}%
\pgfpathcurveto{\pgfqpoint{0.000000in}{0.000000in}}{\pgfqpoint{0.000000in}{0.000000in}}{\pgfqpoint{0.000000in}{0.000000in}}%
\pgfpathcurveto{\pgfqpoint{0.000000in}{0.000000in}}{\pgfqpoint{0.000000in}{0.000000in}}{\pgfqpoint{0.000000in}{0.000000in}}%
\pgfpathcurveto{\pgfqpoint{0.000000in}{0.000000in}}{\pgfqpoint{0.000000in}{0.000000in}}{\pgfqpoint{0.000000in}{0.000000in}}%
\pgfpathcurveto{\pgfqpoint{0.000000in}{0.000000in}}{\pgfqpoint{0.000000in}{0.000000in}}{\pgfqpoint{0.000000in}{0.000000in}}%
\pgfpathcurveto{\pgfqpoint{0.000000in}{0.000000in}}{\pgfqpoint{0.000000in}{0.000000in}}{\pgfqpoint{0.000000in}{0.000000in}}%
\pgfpathcurveto{\pgfqpoint{0.000000in}{0.000000in}}{\pgfqpoint{0.000000in}{0.000000in}}{\pgfqpoint{0.000000in}{0.000000in}}%
\pgfpathcurveto{\pgfqpoint{0.000000in}{0.000000in}}{\pgfqpoint{0.000000in}{0.000000in}}{\pgfqpoint{0.000000in}{0.000000in}}%
\pgfpathlineto{\pgfqpoint{0.000000in}{0.000000in}}%
\pgfpathclose%
\pgfusepath{stroke,fill}%
}%
\begin{pgfscope}%
\pgfsys@transformshift{1.304675in}{0.932268in}%
\pgfsys@useobject{currentmarker}{}%
\end{pgfscope}%
\end{pgfscope}%
\begin{pgfscope}%
\pgfpathrectangle{\pgfqpoint{0.050000in}{0.050000in}}{\pgfqpoint{2.085811in}{1.520004in}}%
\pgfusepath{clip}%
\pgfsetbuttcap%
\pgfsetroundjoin%
\definecolor{currentfill}{rgb}{0.000000,0.000000,0.000000}%
\pgfsetfillcolor{currentfill}%
\pgfsetlinewidth{0.000000pt}%
\definecolor{currentstroke}{rgb}{0.000000,0.000000,0.000000}%
\pgfsetstrokecolor{currentstroke}%
\pgfsetdash{}{0pt}%
\pgfpathmoveto{\pgfqpoint{1.296854in}{0.687737in}}%
\pgfpathlineto{\pgfqpoint{1.296854in}{0.861872in}}%
\pgfpathlineto{\pgfqpoint{1.281210in}{0.854050in}}%
\pgfpathlineto{\pgfqpoint{1.304675in}{0.932268in}}%
\pgfpathlineto{\pgfqpoint{1.328141in}{0.854050in}}%
\pgfpathlineto{\pgfqpoint{1.312497in}{0.861872in}}%
\pgfpathlineto{\pgfqpoint{1.312497in}{0.687737in}}%
\pgfpathlineto{\pgfqpoint{1.296854in}{0.687737in}}%
\pgfusepath{fill}%
\end{pgfscope}%
\begin{pgfscope}%
\pgfpathrectangle{\pgfqpoint{0.050000in}{0.050000in}}{\pgfqpoint{2.085811in}{1.520004in}}%
\pgfusepath{clip}%
\pgfsetbuttcap%
\pgfsetroundjoin%
\pgfsetlinewidth{1.003750pt}%
\definecolor{currentstroke}{rgb}{0.000000,0.000000,0.000000}%
\pgfsetstrokecolor{currentstroke}%
\pgfsetdash{{3.700000pt}{1.600000pt}}{0.000000pt}%
\pgfpathmoveto{\pgfqpoint{1.304675in}{1.176799in}}%
\pgfpathlineto{\pgfqpoint{1.516445in}{0.810002in}}%
\pgfusepath{stroke}%
\end{pgfscope}%
\begin{pgfscope}%
\pgfpathrectangle{\pgfqpoint{0.050000in}{0.050000in}}{\pgfqpoint{2.085811in}{1.520004in}}%
\pgfusepath{clip}%
\pgfsetbuttcap%
\pgfsetroundjoin%
\pgfsetlinewidth{1.003750pt}%
\definecolor{currentstroke}{rgb}{0.000000,0.000000,0.000000}%
\pgfsetstrokecolor{currentstroke}%
\pgfsetdash{{3.700000pt}{1.600000pt}}{0.000000pt}%
\pgfpathmoveto{\pgfqpoint{1.304675in}{0.443206in}}%
\pgfpathlineto{\pgfqpoint{1.516445in}{0.810002in}}%
\pgfusepath{stroke}%
\end{pgfscope}%
\begin{pgfscope}%
\pgfsetbuttcap%
\pgfsetmiterjoin%
\definecolor{currentfill}{rgb}{1.000000,1.000000,1.000000}%
\pgfsetfillcolor{currentfill}%
\pgfsetlinewidth{1.003750pt}%
\definecolor{currentstroke}{rgb}{0.800000,0.800000,0.800000}%
\pgfsetstrokecolor{currentstroke}%
\pgfsetdash{}{0pt}%
\pgfpathmoveto{\pgfqpoint{0.156944in}{0.126389in}}%
\pgfpathlineto{\pgfqpoint{0.764694in}{0.126389in}}%
\pgfpathquadraticcurveto{\pgfqpoint{0.795250in}{0.126389in}}{\pgfqpoint{0.795250in}{0.156944in}}%
\pgfpathlineto{\pgfqpoint{0.795250in}{0.602139in}}%
\pgfpathquadraticcurveto{\pgfqpoint{0.795250in}{0.632694in}}{\pgfqpoint{0.764694in}{0.632694in}}%
\pgfpathlineto{\pgfqpoint{0.156944in}{0.632694in}}%
\pgfpathquadraticcurveto{\pgfqpoint{0.126389in}{0.632694in}}{\pgfqpoint{0.126389in}{0.602139in}}%
\pgfpathlineto{\pgfqpoint{0.126389in}{0.156944in}}%
\pgfpathquadraticcurveto{\pgfqpoint{0.126389in}{0.126389in}}{\pgfqpoint{0.156944in}{0.126389in}}%
\pgfpathlineto{\pgfqpoint{0.156944in}{0.126389in}}%
\pgfpathclose%
\pgfusepath{stroke,fill}%
\end{pgfscope}%
\begin{pgfscope}%
\pgfsetrectcap%
\pgfsetroundjoin%
\pgfsetlinewidth{0.000000pt}%
\definecolor{currentstroke}{rgb}{0.800000,0.400000,0.466667}%
\pgfsetstrokecolor{currentstroke}%
\pgfsetdash{}{0pt}%
\pgfpathmoveto{\pgfqpoint{0.187500in}{0.514139in}}%
\pgfpathlineto{\pgfqpoint{0.340278in}{0.514139in}}%
\pgfpathlineto{\pgfqpoint{0.493056in}{0.514139in}}%
\pgfusepath{}%
\end{pgfscope}%
\begin{pgfscope}%
\pgfsetbuttcap%
\pgfsetroundjoin%
\definecolor{currentfill}{rgb}{0.800000,0.400000,0.466667}%
\pgfsetfillcolor{currentfill}%
\pgfsetlinewidth{1.003750pt}%
\definecolor{currentstroke}{rgb}{0.800000,0.400000,0.466667}%
\pgfsetstrokecolor{currentstroke}%
\pgfsetdash{}{0pt}%
\pgfsys@defobject{currentmarker}{\pgfqpoint{-0.069444in}{-0.069444in}}{\pgfqpoint{0.069444in}{0.069444in}}{%
\pgfpathmoveto{\pgfqpoint{0.000000in}{-0.069444in}}%
\pgfpathcurveto{\pgfqpoint{0.018417in}{-0.069444in}}{\pgfqpoint{0.036082in}{-0.062127in}}{\pgfqpoint{0.049105in}{-0.049105in}}%
\pgfpathcurveto{\pgfqpoint{0.062127in}{-0.036082in}}{\pgfqpoint{0.069444in}{-0.018417in}}{\pgfqpoint{0.069444in}{0.000000in}}%
\pgfpathcurveto{\pgfqpoint{0.069444in}{0.018417in}}{\pgfqpoint{0.062127in}{0.036082in}}{\pgfqpoint{0.049105in}{0.049105in}}%
\pgfpathcurveto{\pgfqpoint{0.036082in}{0.062127in}}{\pgfqpoint{0.018417in}{0.069444in}}{\pgfqpoint{0.000000in}{0.069444in}}%
\pgfpathcurveto{\pgfqpoint{-0.018417in}{0.069444in}}{\pgfqpoint{-0.036082in}{0.062127in}}{\pgfqpoint{-0.049105in}{0.049105in}}%
\pgfpathcurveto{\pgfqpoint{-0.062127in}{0.036082in}}{\pgfqpoint{-0.069444in}{0.018417in}}{\pgfqpoint{-0.069444in}{0.000000in}}%
\pgfpathcurveto{\pgfqpoint{-0.069444in}{-0.018417in}}{\pgfqpoint{-0.062127in}{-0.036082in}}{\pgfqpoint{-0.049105in}{-0.049105in}}%
\pgfpathcurveto{\pgfqpoint{-0.036082in}{-0.062127in}}{\pgfqpoint{-0.018417in}{-0.069444in}}{\pgfqpoint{0.000000in}{-0.069444in}}%
\pgfpathlineto{\pgfqpoint{0.000000in}{-0.069444in}}%
\pgfpathclose%
\pgfusepath{stroke,fill}%
}%
\begin{pgfscope}%
\pgfsys@transformshift{0.340278in}{0.514139in}%
\pgfsys@useobject{currentmarker}{}%
\end{pgfscope}%
\end{pgfscope}%
\begin{pgfscope}%
\definecolor{textcolor}{rgb}{0.000000,0.000000,0.000000}%
\pgfsetstrokecolor{textcolor}%
\pgfsetfillcolor{textcolor}%
\pgftext[x=0.615278in,y=0.460667in,left,base]{\color{textcolor}{\rmfamily\fontsize{11.000000}{13.200000}\selectfont\catcode`\^=\active\def^{\ifmmode\sp\else\^{}\fi}\catcode`\%=\active\def%{\%}A}}%
\end{pgfscope}%
\begin{pgfscope}%
\pgfsetrectcap%
\pgfsetroundjoin%
\pgfsetlinewidth{0.000000pt}%
\definecolor{currentstroke}{rgb}{0.200000,0.133333,0.533333}%
\pgfsetstrokecolor{currentstroke}%
\pgfsetdash{}{0pt}%
\pgfpathmoveto{\pgfqpoint{0.187500in}{0.283903in}}%
\pgfpathlineto{\pgfqpoint{0.340278in}{0.283903in}}%
\pgfpathlineto{\pgfqpoint{0.493056in}{0.283903in}}%
\pgfusepath{}%
\end{pgfscope}%
\begin{pgfscope}%
\pgfsetbuttcap%
\pgfsetroundjoin%
\definecolor{currentfill}{rgb}{0.200000,0.133333,0.533333}%
\pgfsetfillcolor{currentfill}%
\pgfsetlinewidth{1.003750pt}%
\definecolor{currentstroke}{rgb}{0.200000,0.133333,0.533333}%
\pgfsetstrokecolor{currentstroke}%
\pgfsetdash{}{0pt}%
\pgfsys@defobject{currentmarker}{\pgfqpoint{-0.069444in}{-0.069444in}}{\pgfqpoint{0.069444in}{0.069444in}}{%
\pgfpathmoveto{\pgfqpoint{0.000000in}{-0.069444in}}%
\pgfpathcurveto{\pgfqpoint{0.018417in}{-0.069444in}}{\pgfqpoint{0.036082in}{-0.062127in}}{\pgfqpoint{0.049105in}{-0.049105in}}%
\pgfpathcurveto{\pgfqpoint{0.062127in}{-0.036082in}}{\pgfqpoint{0.069444in}{-0.018417in}}{\pgfqpoint{0.069444in}{0.000000in}}%
\pgfpathcurveto{\pgfqpoint{0.069444in}{0.018417in}}{\pgfqpoint{0.062127in}{0.036082in}}{\pgfqpoint{0.049105in}{0.049105in}}%
\pgfpathcurveto{\pgfqpoint{0.036082in}{0.062127in}}{\pgfqpoint{0.018417in}{0.069444in}}{\pgfqpoint{0.000000in}{0.069444in}}%
\pgfpathcurveto{\pgfqpoint{-0.018417in}{0.069444in}}{\pgfqpoint{-0.036082in}{0.062127in}}{\pgfqpoint{-0.049105in}{0.049105in}}%
\pgfpathcurveto{\pgfqpoint{-0.062127in}{0.036082in}}{\pgfqpoint{-0.069444in}{0.018417in}}{\pgfqpoint{-0.069444in}{0.000000in}}%
\pgfpathcurveto{\pgfqpoint{-0.069444in}{-0.018417in}}{\pgfqpoint{-0.062127in}{-0.036082in}}{\pgfqpoint{-0.049105in}{-0.049105in}}%
\pgfpathcurveto{\pgfqpoint{-0.036082in}{-0.062127in}}{\pgfqpoint{-0.018417in}{-0.069444in}}{\pgfqpoint{0.000000in}{-0.069444in}}%
\pgfpathlineto{\pgfqpoint{0.000000in}{-0.069444in}}%
\pgfpathclose%
\pgfusepath{stroke,fill}%
}%
\begin{pgfscope}%
\pgfsys@transformshift{0.340278in}{0.283903in}%
\pgfsys@useobject{currentmarker}{}%
\end{pgfscope}%
\end{pgfscope}%
\begin{pgfscope}%
\definecolor{textcolor}{rgb}{0.000000,0.000000,0.000000}%
\pgfsetstrokecolor{textcolor}%
\pgfsetfillcolor{textcolor}%
\pgftext[x=0.615278in,y=0.230431in,left,base]{\color{textcolor}{\rmfamily\fontsize{11.000000}{13.200000}\selectfont\catcode`\^=\active\def^{\ifmmode\sp\else\^{}\fi}\catcode`\%=\active\def%{\%}B}}%
\end{pgfscope}%
\end{pgfpicture}%
\makeatother%
\endgroup%

	\end{subfigure}%
	\begin{subfigure}[t]{0.5\textwidth}
		\centering
		\caption{\hfill\null}\label{sfig:graphene Brillouin zone}
		%% Creator: Matplotlib, PGF backend
%%
%% To include the figure in your LaTeX document, write
%%   \input{<filename>.pgf}
%%
%% Make sure the required packages are loaded in your preamble
%%   \usepackage{pgf}
%%
%% Also ensure that all the required font packages are loaded; for instance,
%% the lmodern package is sometimes necessary when using math font.
%%   \usepackage{lmodern}
%%
%% Figures using additional raster images can only be included by \input if
%% they are in the same directory as the main LaTeX file. For loading figures
%% from other directories you can use the `import` package
%%   \usepackage{import}
%%
%% and then include the figures with
%%   \import{<path to file>}{<filename>.pgf}
%%
%% Matplotlib used the following preamble
%%   \def\mathdefault#1{#1}
%%   \everymath=\expandafter{\the\everymath\displaystyle}
%%   \IfFileExists{scrextend.sty}{
%%     \usepackage[fontsize=11.000000pt]{scrextend}
%%   }{
%%     \renewcommand{\normalsize}{\fontsize{11.000000}{13.200000}\selectfont}
%%     \normalsize
%%   }
%%   \usepackage{fontspec}\usepackage{unicode-math}\setmathfont{texgyrepagella-math.otf}\setmainfont{texgyrepagella-math}\usepackage{nicefrac}
%%   \makeatletter\@ifpackageloaded{underscore}{}{\usepackage[strings]{underscore}}\makeatother
%%
\begingroup%
\makeatletter%
\begin{pgfpicture}%
\pgfpathrectangle{\pgfpointorigin}{\pgfqpoint{2.800000in}{2.240000in}}%
\pgfusepath{use as bounding box, clip}%
\begin{pgfscope}%
\pgfsetbuttcap%
\pgfsetmiterjoin%
\definecolor{currentfill}{rgb}{1.000000,1.000000,1.000000}%
\pgfsetfillcolor{currentfill}%
\pgfsetlinewidth{0.000000pt}%
\definecolor{currentstroke}{rgb}{1.000000,1.000000,1.000000}%
\pgfsetstrokecolor{currentstroke}%
\pgfsetdash{}{0pt}%
\pgfpathmoveto{\pgfqpoint{0.000000in}{0.000000in}}%
\pgfpathlineto{\pgfqpoint{2.800000in}{0.000000in}}%
\pgfpathlineto{\pgfqpoint{2.800000in}{2.240000in}}%
\pgfpathlineto{\pgfqpoint{0.000000in}{2.240000in}}%
\pgfpathlineto{\pgfqpoint{0.000000in}{0.000000in}}%
\pgfpathclose%
\pgfusepath{fill}%
\end{pgfscope}%
\begin{pgfscope}%
\pgfsetbuttcap%
\pgfsetmiterjoin%
\definecolor{currentfill}{rgb}{1.000000,1.000000,1.000000}%
\pgfsetfillcolor{currentfill}%
\pgfsetlinewidth{0.000000pt}%
\definecolor{currentstroke}{rgb}{0.000000,0.000000,0.000000}%
\pgfsetstrokecolor{currentstroke}%
\pgfsetstrokeopacity{0.000000}%
\pgfsetdash{}{0pt}%
\pgfpathmoveto{\pgfqpoint{0.688140in}{0.246400in}}%
\pgfpathlineto{\pgfqpoint{2.181860in}{0.246400in}}%
\pgfpathlineto{\pgfqpoint{2.181860in}{1.971200in}}%
\pgfpathlineto{\pgfqpoint{0.688140in}{1.971200in}}%
\pgfpathlineto{\pgfqpoint{0.688140in}{0.246400in}}%
\pgfpathclose%
\pgfusepath{fill}%
\end{pgfscope}%
\begin{pgfscope}%
\pgfpathrectangle{\pgfqpoint{0.688140in}{0.246400in}}{\pgfqpoint{1.493721in}{1.724800in}}%
\pgfusepath{clip}%
\pgfsetbuttcap%
\pgfsetroundjoin%
\definecolor{currentfill}{rgb}{0.000000,0.000000,0.000000}%
\pgfsetfillcolor{currentfill}%
\pgfsetlinewidth{1.003750pt}%
\definecolor{currentstroke}{rgb}{0.000000,0.000000,0.000000}%
\pgfsetstrokecolor{currentstroke}%
\pgfsetdash{}{0pt}%
\pgfsys@defobject{currentmarker}{\pgfqpoint{-0.020833in}{-0.020833in}}{\pgfqpoint{0.020833in}{0.020833in}}{%
\pgfpathmoveto{\pgfqpoint{0.000000in}{-0.020833in}}%
\pgfpathcurveto{\pgfqpoint{0.005525in}{-0.020833in}}{\pgfqpoint{0.010825in}{-0.018638in}}{\pgfqpoint{0.014731in}{-0.014731in}}%
\pgfpathcurveto{\pgfqpoint{0.018638in}{-0.010825in}}{\pgfqpoint{0.020833in}{-0.005525in}}{\pgfqpoint{0.020833in}{0.000000in}}%
\pgfpathcurveto{\pgfqpoint{0.020833in}{0.005525in}}{\pgfqpoint{0.018638in}{0.010825in}}{\pgfqpoint{0.014731in}{0.014731in}}%
\pgfpathcurveto{\pgfqpoint{0.010825in}{0.018638in}}{\pgfqpoint{0.005525in}{0.020833in}}{\pgfqpoint{0.000000in}{0.020833in}}%
\pgfpathcurveto{\pgfqpoint{-0.005525in}{0.020833in}}{\pgfqpoint{-0.010825in}{0.018638in}}{\pgfqpoint{-0.014731in}{0.014731in}}%
\pgfpathcurveto{\pgfqpoint{-0.018638in}{0.010825in}}{\pgfqpoint{-0.020833in}{0.005525in}}{\pgfqpoint{-0.020833in}{0.000000in}}%
\pgfpathcurveto{\pgfqpoint{-0.020833in}{-0.005525in}}{\pgfqpoint{-0.018638in}{-0.010825in}}{\pgfqpoint{-0.014731in}{-0.014731in}}%
\pgfpathcurveto{\pgfqpoint{-0.010825in}{-0.018638in}}{\pgfqpoint{-0.005525in}{-0.020833in}}{\pgfqpoint{0.000000in}{-0.020833in}}%
\pgfpathlineto{\pgfqpoint{0.000000in}{-0.020833in}}%
\pgfpathclose%
\pgfusepath{stroke,fill}%
}%
\begin{pgfscope}%
\pgfsys@transformshift{1.435000in}{1.108800in}%
\pgfsys@useobject{currentmarker}{}%
\end{pgfscope}%
\begin{pgfscope}%
\pgfsys@transformshift{0.756036in}{0.716800in}%
\pgfsys@useobject{currentmarker}{}%
\end{pgfscope}%
\begin{pgfscope}%
\pgfsys@transformshift{1.435000in}{0.324800in}%
\pgfsys@useobject{currentmarker}{}%
\end{pgfscope}%
\begin{pgfscope}%
\pgfsys@transformshift{0.756036in}{1.500800in}%
\pgfsys@useobject{currentmarker}{}%
\end{pgfscope}%
\begin{pgfscope}%
\pgfsys@transformshift{2.113964in}{0.716800in}%
\pgfsys@useobject{currentmarker}{}%
\end{pgfscope}%
\begin{pgfscope}%
\pgfsys@transformshift{1.435000in}{1.892800in}%
\pgfsys@useobject{currentmarker}{}%
\end{pgfscope}%
\begin{pgfscope}%
\pgfsys@transformshift{2.113964in}{1.500800in}%
\pgfsys@useobject{currentmarker}{}%
\end{pgfscope}%
\end{pgfscope}%
\begin{pgfscope}%
\pgfpathrectangle{\pgfqpoint{0.688140in}{0.246400in}}{\pgfqpoint{1.493721in}{1.724800in}}%
\pgfusepath{clip}%
\pgfsetbuttcap%
\pgfsetroundjoin%
\definecolor{currentfill}{rgb}{0.000000,0.000000,0.000000}%
\pgfsetfillcolor{currentfill}%
\pgfsetlinewidth{1.003750pt}%
\definecolor{currentstroke}{rgb}{0.000000,0.000000,0.000000}%
\pgfsetstrokecolor{currentstroke}%
\pgfsetdash{}{0pt}%
\pgfsys@defobject{currentmarker}{\pgfqpoint{-0.020833in}{-0.020833in}}{\pgfqpoint{0.020833in}{0.020833in}}{%
\pgfpathmoveto{\pgfqpoint{0.000000in}{-0.020833in}}%
\pgfpathcurveto{\pgfqpoint{0.005525in}{-0.020833in}}{\pgfqpoint{0.010825in}{-0.018638in}}{\pgfqpoint{0.014731in}{-0.014731in}}%
\pgfpathcurveto{\pgfqpoint{0.018638in}{-0.010825in}}{\pgfqpoint{0.020833in}{-0.005525in}}{\pgfqpoint{0.020833in}{0.000000in}}%
\pgfpathcurveto{\pgfqpoint{0.020833in}{0.005525in}}{\pgfqpoint{0.018638in}{0.010825in}}{\pgfqpoint{0.014731in}{0.014731in}}%
\pgfpathcurveto{\pgfqpoint{0.010825in}{0.018638in}}{\pgfqpoint{0.005525in}{0.020833in}}{\pgfqpoint{0.000000in}{0.020833in}}%
\pgfpathcurveto{\pgfqpoint{-0.005525in}{0.020833in}}{\pgfqpoint{-0.010825in}{0.018638in}}{\pgfqpoint{-0.014731in}{0.014731in}}%
\pgfpathcurveto{\pgfqpoint{-0.018638in}{0.010825in}}{\pgfqpoint{-0.020833in}{0.005525in}}{\pgfqpoint{-0.020833in}{0.000000in}}%
\pgfpathcurveto{\pgfqpoint{-0.020833in}{-0.005525in}}{\pgfqpoint{-0.018638in}{-0.010825in}}{\pgfqpoint{-0.014731in}{-0.014731in}}%
\pgfpathcurveto{\pgfqpoint{-0.010825in}{-0.018638in}}{\pgfqpoint{-0.005525in}{-0.020833in}}{\pgfqpoint{0.000000in}{-0.020833in}}%
\pgfpathlineto{\pgfqpoint{0.000000in}{-0.020833in}}%
\pgfpathclose%
\pgfusepath{stroke,fill}%
}%
\begin{pgfscope}%
\pgfsys@transformshift{0.982357in}{1.108800in}%
\pgfsys@useobject{currentmarker}{}%
\end{pgfscope}%
\begin{pgfscope}%
\pgfsys@transformshift{1.208679in}{1.500800in}%
\pgfsys@useobject{currentmarker}{}%
\end{pgfscope}%
\begin{pgfscope}%
\pgfsys@transformshift{1.208679in}{0.716800in}%
\pgfsys@useobject{currentmarker}{}%
\end{pgfscope}%
\begin{pgfscope}%
\pgfsys@transformshift{1.661321in}{0.716800in}%
\pgfsys@useobject{currentmarker}{}%
\end{pgfscope}%
\begin{pgfscope}%
\pgfsys@transformshift{1.887643in}{1.108800in}%
\pgfsys@useobject{currentmarker}{}%
\end{pgfscope}%
\begin{pgfscope}%
\pgfsys@transformshift{1.661321in}{1.500800in}%
\pgfsys@useobject{currentmarker}{}%
\end{pgfscope}%
\end{pgfscope}%
\begin{pgfscope}%
\pgfpathrectangle{\pgfqpoint{0.688140in}{0.246400in}}{\pgfqpoint{1.493721in}{1.724800in}}%
\pgfusepath{clip}%
\pgfsetbuttcap%
\pgfsetroundjoin%
\definecolor{currentfill}{rgb}{0.247059,0.564706,0.854902}%
\pgfsetfillcolor{currentfill}%
\pgfsetlinewidth{1.003750pt}%
\definecolor{currentstroke}{rgb}{0.247059,0.564706,0.854902}%
\pgfsetstrokecolor{currentstroke}%
\pgfsetdash{}{0pt}%
\pgfsys@defobject{currentmarker}{\pgfqpoint{0.000000in}{0.000000in}}{\pgfqpoint{0.000000in}{0.000000in}}{%
\pgfpathmoveto{\pgfqpoint{0.000000in}{0.000000in}}%
\pgfpathcurveto{\pgfqpoint{0.000000in}{0.000000in}}{\pgfqpoint{0.000000in}{0.000000in}}{\pgfqpoint{0.000000in}{0.000000in}}%
\pgfpathcurveto{\pgfqpoint{0.000000in}{0.000000in}}{\pgfqpoint{0.000000in}{0.000000in}}{\pgfqpoint{0.000000in}{0.000000in}}%
\pgfpathcurveto{\pgfqpoint{0.000000in}{0.000000in}}{\pgfqpoint{0.000000in}{0.000000in}}{\pgfqpoint{0.000000in}{0.000000in}}%
\pgfpathcurveto{\pgfqpoint{0.000000in}{0.000000in}}{\pgfqpoint{0.000000in}{0.000000in}}{\pgfqpoint{0.000000in}{0.000000in}}%
\pgfpathcurveto{\pgfqpoint{0.000000in}{0.000000in}}{\pgfqpoint{0.000000in}{0.000000in}}{\pgfqpoint{0.000000in}{0.000000in}}%
\pgfpathcurveto{\pgfqpoint{0.000000in}{0.000000in}}{\pgfqpoint{0.000000in}{0.000000in}}{\pgfqpoint{0.000000in}{0.000000in}}%
\pgfpathcurveto{\pgfqpoint{0.000000in}{0.000000in}}{\pgfqpoint{0.000000in}{0.000000in}}{\pgfqpoint{0.000000in}{0.000000in}}%
\pgfpathcurveto{\pgfqpoint{0.000000in}{0.000000in}}{\pgfqpoint{0.000000in}{0.000000in}}{\pgfqpoint{0.000000in}{0.000000in}}%
\pgfpathlineto{\pgfqpoint{0.000000in}{0.000000in}}%
\pgfpathclose%
\pgfusepath{stroke,fill}%
}%
\begin{pgfscope}%
\pgfsys@transformshift{2.113964in}{1.500800in}%
\pgfsys@useobject{currentmarker}{}%
\end{pgfscope}%
\end{pgfscope}%
\begin{pgfscope}%
\pgfpathrectangle{\pgfqpoint{0.688140in}{0.246400in}}{\pgfqpoint{1.493721in}{1.724800in}}%
\pgfusepath{clip}%
\pgfsetbuttcap%
\pgfsetroundjoin%
\definecolor{currentfill}{rgb}{0.000000,0.000000,0.000000}%
\pgfsetfillcolor{currentfill}%
\pgfsetlinewidth{0.000000pt}%
\definecolor{currentstroke}{rgb}{0.000000,0.000000,0.000000}%
\pgfsetstrokecolor{currentstroke}%
\pgfsetdash{}{0pt}%
\pgfpathmoveto{\pgfqpoint{1.432199in}{1.113651in}}%
\pgfpathlineto{\pgfqpoint{2.067504in}{1.480444in}}%
\pgfpathlineto{\pgfqpoint{2.057052in}{1.487346in}}%
\pgfpathlineto{\pgfqpoint{2.113964in}{1.500800in}}%
\pgfpathlineto{\pgfqpoint{2.073856in}{1.458240in}}%
\pgfpathlineto{\pgfqpoint{2.073106in}{1.470742in}}%
\pgfpathlineto{\pgfqpoint{1.437801in}{1.103949in}}%
\pgfpathlineto{\pgfqpoint{1.432199in}{1.113651in}}%
\pgfusepath{fill}%
\end{pgfscope}%
\begin{pgfscope}%
\pgfpathrectangle{\pgfqpoint{0.688140in}{0.246400in}}{\pgfqpoint{1.493721in}{1.724800in}}%
\pgfusepath{clip}%
\pgfsetbuttcap%
\pgfsetroundjoin%
\definecolor{currentfill}{rgb}{1.000000,0.662745,0.054902}%
\pgfsetfillcolor{currentfill}%
\pgfsetlinewidth{1.003750pt}%
\definecolor{currentstroke}{rgb}{1.000000,0.662745,0.054902}%
\pgfsetstrokecolor{currentstroke}%
\pgfsetdash{}{0pt}%
\pgfsys@defobject{currentmarker}{\pgfqpoint{0.000000in}{0.000000in}}{\pgfqpoint{0.000000in}{0.000000in}}{%
\pgfpathmoveto{\pgfqpoint{0.000000in}{0.000000in}}%
\pgfpathcurveto{\pgfqpoint{0.000000in}{0.000000in}}{\pgfqpoint{0.000000in}{0.000000in}}{\pgfqpoint{0.000000in}{0.000000in}}%
\pgfpathcurveto{\pgfqpoint{0.000000in}{0.000000in}}{\pgfqpoint{0.000000in}{0.000000in}}{\pgfqpoint{0.000000in}{0.000000in}}%
\pgfpathcurveto{\pgfqpoint{0.000000in}{0.000000in}}{\pgfqpoint{0.000000in}{0.000000in}}{\pgfqpoint{0.000000in}{0.000000in}}%
\pgfpathcurveto{\pgfqpoint{0.000000in}{0.000000in}}{\pgfqpoint{0.000000in}{0.000000in}}{\pgfqpoint{0.000000in}{0.000000in}}%
\pgfpathcurveto{\pgfqpoint{0.000000in}{0.000000in}}{\pgfqpoint{0.000000in}{0.000000in}}{\pgfqpoint{0.000000in}{0.000000in}}%
\pgfpathcurveto{\pgfqpoint{0.000000in}{0.000000in}}{\pgfqpoint{0.000000in}{0.000000in}}{\pgfqpoint{0.000000in}{0.000000in}}%
\pgfpathcurveto{\pgfqpoint{0.000000in}{0.000000in}}{\pgfqpoint{0.000000in}{0.000000in}}{\pgfqpoint{0.000000in}{0.000000in}}%
\pgfpathcurveto{\pgfqpoint{0.000000in}{0.000000in}}{\pgfqpoint{0.000000in}{0.000000in}}{\pgfqpoint{0.000000in}{0.000000in}}%
\pgfpathlineto{\pgfqpoint{0.000000in}{0.000000in}}%
\pgfpathclose%
\pgfusepath{stroke,fill}%
}%
\begin{pgfscope}%
\pgfsys@transformshift{2.113964in}{0.716800in}%
\pgfsys@useobject{currentmarker}{}%
\end{pgfscope}%
\end{pgfscope}%
\begin{pgfscope}%
\pgfpathrectangle{\pgfqpoint{0.688140in}{0.246400in}}{\pgfqpoint{1.493721in}{1.724800in}}%
\pgfusepath{clip}%
\pgfsetbuttcap%
\pgfsetroundjoin%
\definecolor{currentfill}{rgb}{0.000000,0.000000,0.000000}%
\pgfsetfillcolor{currentfill}%
\pgfsetlinewidth{0.000000pt}%
\definecolor{currentstroke}{rgb}{0.000000,0.000000,0.000000}%
\pgfsetstrokecolor{currentstroke}%
\pgfsetdash{}{0pt}%
\pgfpathmoveto{\pgfqpoint{1.437801in}{1.113651in}}%
\pgfpathlineto{\pgfqpoint{2.073106in}{0.746858in}}%
\pgfpathlineto{\pgfqpoint{2.073856in}{0.759360in}}%
\pgfpathlineto{\pgfqpoint{2.113964in}{0.716800in}}%
\pgfpathlineto{\pgfqpoint{2.057052in}{0.730254in}}%
\pgfpathlineto{\pgfqpoint{2.067504in}{0.737156in}}%
\pgfpathlineto{\pgfqpoint{1.432199in}{1.103949in}}%
\pgfpathlineto{\pgfqpoint{1.437801in}{1.113651in}}%
\pgfusepath{fill}%
\end{pgfscope}%
\begin{pgfscope}%
\pgfpathrectangle{\pgfqpoint{0.688140in}{0.246400in}}{\pgfqpoint{1.493721in}{1.724800in}}%
\pgfusepath{clip}%
\pgfsetbuttcap%
\pgfsetroundjoin%
\definecolor{currentfill}{rgb}{0.000000,0.000000,0.000000}%
\pgfsetfillcolor{currentfill}%
\pgfsetlinewidth{1.003750pt}%
\definecolor{currentstroke}{rgb}{0.000000,0.000000,0.000000}%
\pgfsetstrokecolor{currentstroke}%
\pgfsetdash{}{0pt}%
\pgfsys@defobject{currentmarker}{\pgfqpoint{-0.041667in}{-0.041667in}}{\pgfqpoint{0.041667in}{0.041667in}}{%
\pgfpathmoveto{\pgfqpoint{0.000000in}{-0.041667in}}%
\pgfpathcurveto{\pgfqpoint{0.011050in}{-0.041667in}}{\pgfqpoint{0.021649in}{-0.037276in}}{\pgfqpoint{0.029463in}{-0.029463in}}%
\pgfpathcurveto{\pgfqpoint{0.037276in}{-0.021649in}}{\pgfqpoint{0.041667in}{-0.011050in}}{\pgfqpoint{0.041667in}{0.000000in}}%
\pgfpathcurveto{\pgfqpoint{0.041667in}{0.011050in}}{\pgfqpoint{0.037276in}{0.021649in}}{\pgfqpoint{0.029463in}{0.029463in}}%
\pgfpathcurveto{\pgfqpoint{0.021649in}{0.037276in}}{\pgfqpoint{0.011050in}{0.041667in}}{\pgfqpoint{0.000000in}{0.041667in}}%
\pgfpathcurveto{\pgfqpoint{-0.011050in}{0.041667in}}{\pgfqpoint{-0.021649in}{0.037276in}}{\pgfqpoint{-0.029463in}{0.029463in}}%
\pgfpathcurveto{\pgfqpoint{-0.037276in}{0.021649in}}{\pgfqpoint{-0.041667in}{0.011050in}}{\pgfqpoint{-0.041667in}{0.000000in}}%
\pgfpathcurveto{\pgfqpoint{-0.041667in}{-0.011050in}}{\pgfqpoint{-0.037276in}{-0.021649in}}{\pgfqpoint{-0.029463in}{-0.029463in}}%
\pgfpathcurveto{\pgfqpoint{-0.021649in}{-0.037276in}}{\pgfqpoint{-0.011050in}{-0.041667in}}{\pgfqpoint{0.000000in}{-0.041667in}}%
\pgfpathlineto{\pgfqpoint{0.000000in}{-0.041667in}}%
\pgfpathclose%
\pgfusepath{stroke,fill}%
}%
\begin{pgfscope}%
\pgfsys@transformshift{1.435000in}{1.108800in}%
\pgfsys@useobject{currentmarker}{}%
\end{pgfscope}%
\end{pgfscope}%
\begin{pgfscope}%
\pgfpathrectangle{\pgfqpoint{0.688140in}{0.246400in}}{\pgfqpoint{1.493721in}{1.724800in}}%
\pgfusepath{clip}%
\pgfsetbuttcap%
\pgfsetroundjoin%
\definecolor{currentfill}{rgb}{0.000000,0.000000,0.000000}%
\pgfsetfillcolor{currentfill}%
\pgfsetlinewidth{1.003750pt}%
\definecolor{currentstroke}{rgb}{0.000000,0.000000,0.000000}%
\pgfsetstrokecolor{currentstroke}%
\pgfsetdash{}{0pt}%
\pgfsys@defobject{currentmarker}{\pgfqpoint{-0.041667in}{-0.041667in}}{\pgfqpoint{0.041667in}{0.041667in}}{%
\pgfpathmoveto{\pgfqpoint{0.000000in}{-0.041667in}}%
\pgfpathcurveto{\pgfqpoint{0.011050in}{-0.041667in}}{\pgfqpoint{0.021649in}{-0.037276in}}{\pgfqpoint{0.029463in}{-0.029463in}}%
\pgfpathcurveto{\pgfqpoint{0.037276in}{-0.021649in}}{\pgfqpoint{0.041667in}{-0.011050in}}{\pgfqpoint{0.041667in}{0.000000in}}%
\pgfpathcurveto{\pgfqpoint{0.041667in}{0.011050in}}{\pgfqpoint{0.037276in}{0.021649in}}{\pgfqpoint{0.029463in}{0.029463in}}%
\pgfpathcurveto{\pgfqpoint{0.021649in}{0.037276in}}{\pgfqpoint{0.011050in}{0.041667in}}{\pgfqpoint{0.000000in}{0.041667in}}%
\pgfpathcurveto{\pgfqpoint{-0.011050in}{0.041667in}}{\pgfqpoint{-0.021649in}{0.037276in}}{\pgfqpoint{-0.029463in}{0.029463in}}%
\pgfpathcurveto{\pgfqpoint{-0.037276in}{0.021649in}}{\pgfqpoint{-0.041667in}{0.011050in}}{\pgfqpoint{-0.041667in}{0.000000in}}%
\pgfpathcurveto{\pgfqpoint{-0.041667in}{-0.011050in}}{\pgfqpoint{-0.037276in}{-0.021649in}}{\pgfqpoint{-0.029463in}{-0.029463in}}%
\pgfpathcurveto{\pgfqpoint{-0.021649in}{-0.037276in}}{\pgfqpoint{-0.011050in}{-0.041667in}}{\pgfqpoint{0.000000in}{-0.041667in}}%
\pgfpathlineto{\pgfqpoint{0.000000in}{-0.041667in}}%
\pgfpathclose%
\pgfusepath{stroke,fill}%
}%
\begin{pgfscope}%
\pgfsys@transformshift{1.774482in}{1.304800in}%
\pgfsys@useobject{currentmarker}{}%
\end{pgfscope}%
\end{pgfscope}%
\begin{pgfscope}%
\pgfpathrectangle{\pgfqpoint{0.688140in}{0.246400in}}{\pgfqpoint{1.493721in}{1.724800in}}%
\pgfusepath{clip}%
\pgfsetbuttcap%
\pgfsetroundjoin%
\definecolor{currentfill}{rgb}{0.000000,0.000000,0.000000}%
\pgfsetfillcolor{currentfill}%
\pgfsetlinewidth{1.003750pt}%
\definecolor{currentstroke}{rgb}{0.000000,0.000000,0.000000}%
\pgfsetstrokecolor{currentstroke}%
\pgfsetdash{}{0pt}%
\pgfsys@defobject{currentmarker}{\pgfqpoint{-0.041667in}{-0.041667in}}{\pgfqpoint{0.041667in}{0.041667in}}{%
\pgfpathmoveto{\pgfqpoint{0.000000in}{-0.041667in}}%
\pgfpathcurveto{\pgfqpoint{0.011050in}{-0.041667in}}{\pgfqpoint{0.021649in}{-0.037276in}}{\pgfqpoint{0.029463in}{-0.029463in}}%
\pgfpathcurveto{\pgfqpoint{0.037276in}{-0.021649in}}{\pgfqpoint{0.041667in}{-0.011050in}}{\pgfqpoint{0.041667in}{0.000000in}}%
\pgfpathcurveto{\pgfqpoint{0.041667in}{0.011050in}}{\pgfqpoint{0.037276in}{0.021649in}}{\pgfqpoint{0.029463in}{0.029463in}}%
\pgfpathcurveto{\pgfqpoint{0.021649in}{0.037276in}}{\pgfqpoint{0.011050in}{0.041667in}}{\pgfqpoint{0.000000in}{0.041667in}}%
\pgfpathcurveto{\pgfqpoint{-0.011050in}{0.041667in}}{\pgfqpoint{-0.021649in}{0.037276in}}{\pgfqpoint{-0.029463in}{0.029463in}}%
\pgfpathcurveto{\pgfqpoint{-0.037276in}{0.021649in}}{\pgfqpoint{-0.041667in}{0.011050in}}{\pgfqpoint{-0.041667in}{0.000000in}}%
\pgfpathcurveto{\pgfqpoint{-0.041667in}{-0.011050in}}{\pgfqpoint{-0.037276in}{-0.021649in}}{\pgfqpoint{-0.029463in}{-0.029463in}}%
\pgfpathcurveto{\pgfqpoint{-0.021649in}{-0.037276in}}{\pgfqpoint{-0.011050in}{-0.041667in}}{\pgfqpoint{0.000000in}{-0.041667in}}%
\pgfpathlineto{\pgfqpoint{0.000000in}{-0.041667in}}%
\pgfpathclose%
\pgfusepath{stroke,fill}%
}%
\begin{pgfscope}%
\pgfsys@transformshift{1.887643in}{1.108800in}%
\pgfsys@useobject{currentmarker}{}%
\end{pgfscope}%
\end{pgfscope}%
\begin{pgfscope}%
\pgfpathrectangle{\pgfqpoint{0.688140in}{0.246400in}}{\pgfqpoint{1.493721in}{1.724800in}}%
\pgfusepath{clip}%
\pgfsetbuttcap%
\pgfsetroundjoin%
\pgfsetlinewidth{0.501875pt}%
\definecolor{currentstroke}{rgb}{0.000000,0.000000,0.000000}%
\pgfsetstrokecolor{currentstroke}%
\pgfsetdash{}{0pt}%
\pgfpathmoveto{\pgfqpoint{1.435000in}{1.108800in}}%
\pgfpathlineto{\pgfqpoint{1.435000in}{1.108800in}}%
\pgfusepath{stroke}%
\end{pgfscope}%
\begin{pgfscope}%
\pgfpathrectangle{\pgfqpoint{0.688140in}{0.246400in}}{\pgfqpoint{1.493721in}{1.724800in}}%
\pgfusepath{clip}%
\pgfsetbuttcap%
\pgfsetroundjoin%
\pgfsetlinewidth{0.501875pt}%
\definecolor{currentstroke}{rgb}{0.000000,0.000000,0.000000}%
\pgfsetstrokecolor{currentstroke}%
\pgfsetdash{}{0pt}%
\pgfpathmoveto{\pgfqpoint{1.435000in}{1.108800in}}%
\pgfpathlineto{\pgfqpoint{0.756036in}{0.716800in}}%
\pgfusepath{stroke}%
\end{pgfscope}%
\begin{pgfscope}%
\pgfpathrectangle{\pgfqpoint{0.688140in}{0.246400in}}{\pgfqpoint{1.493721in}{1.724800in}}%
\pgfusepath{clip}%
\pgfsetbuttcap%
\pgfsetroundjoin%
\pgfsetlinewidth{0.501875pt}%
\definecolor{currentstroke}{rgb}{0.000000,0.000000,0.000000}%
\pgfsetstrokecolor{currentstroke}%
\pgfsetdash{}{0pt}%
\pgfpathmoveto{\pgfqpoint{1.435000in}{1.108800in}}%
\pgfpathlineto{\pgfqpoint{1.435000in}{0.324800in}}%
\pgfusepath{stroke}%
\end{pgfscope}%
\begin{pgfscope}%
\pgfpathrectangle{\pgfqpoint{0.688140in}{0.246400in}}{\pgfqpoint{1.493721in}{1.724800in}}%
\pgfusepath{clip}%
\pgfsetbuttcap%
\pgfsetroundjoin%
\pgfsetlinewidth{0.501875pt}%
\definecolor{currentstroke}{rgb}{0.000000,0.000000,0.000000}%
\pgfsetstrokecolor{currentstroke}%
\pgfsetdash{}{0pt}%
\pgfpathmoveto{\pgfqpoint{1.435000in}{1.108800in}}%
\pgfpathlineto{\pgfqpoint{0.756036in}{1.500800in}}%
\pgfusepath{stroke}%
\end{pgfscope}%
\begin{pgfscope}%
\pgfpathrectangle{\pgfqpoint{0.688140in}{0.246400in}}{\pgfqpoint{1.493721in}{1.724800in}}%
\pgfusepath{clip}%
\pgfsetbuttcap%
\pgfsetroundjoin%
\pgfsetlinewidth{0.501875pt}%
\definecolor{currentstroke}{rgb}{0.000000,0.000000,0.000000}%
\pgfsetstrokecolor{currentstroke}%
\pgfsetdash{}{0pt}%
\pgfpathmoveto{\pgfqpoint{1.435000in}{1.108800in}}%
\pgfpathlineto{\pgfqpoint{2.113964in}{0.716800in}}%
\pgfusepath{stroke}%
\end{pgfscope}%
\begin{pgfscope}%
\pgfpathrectangle{\pgfqpoint{0.688140in}{0.246400in}}{\pgfqpoint{1.493721in}{1.724800in}}%
\pgfusepath{clip}%
\pgfsetbuttcap%
\pgfsetroundjoin%
\pgfsetlinewidth{0.501875pt}%
\definecolor{currentstroke}{rgb}{0.000000,0.000000,0.000000}%
\pgfsetstrokecolor{currentstroke}%
\pgfsetdash{}{0pt}%
\pgfpathmoveto{\pgfqpoint{1.435000in}{1.108800in}}%
\pgfpathlineto{\pgfqpoint{1.435000in}{1.892800in}}%
\pgfusepath{stroke}%
\end{pgfscope}%
\begin{pgfscope}%
\pgfpathrectangle{\pgfqpoint{0.688140in}{0.246400in}}{\pgfqpoint{1.493721in}{1.724800in}}%
\pgfusepath{clip}%
\pgfsetbuttcap%
\pgfsetroundjoin%
\pgfsetlinewidth{0.501875pt}%
\definecolor{currentstroke}{rgb}{0.000000,0.000000,0.000000}%
\pgfsetstrokecolor{currentstroke}%
\pgfsetdash{}{0pt}%
\pgfpathmoveto{\pgfqpoint{1.435000in}{1.108800in}}%
\pgfpathlineto{\pgfqpoint{2.113964in}{1.500800in}}%
\pgfusepath{stroke}%
\end{pgfscope}%
\begin{pgfscope}%
\pgfpathrectangle{\pgfqpoint{0.688140in}{0.246400in}}{\pgfqpoint{1.493721in}{1.724800in}}%
\pgfusepath{clip}%
\pgfsetbuttcap%
\pgfsetroundjoin%
\pgfsetlinewidth{1.003750pt}%
\definecolor{currentstroke}{rgb}{0.000000,0.000000,0.000000}%
\pgfsetstrokecolor{currentstroke}%
\pgfsetdash{}{0pt}%
\pgfpathmoveto{\pgfqpoint{0.982357in}{1.108800in}}%
\pgfpathlineto{\pgfqpoint{1.208679in}{1.500800in}}%
\pgfusepath{stroke}%
\end{pgfscope}%
\begin{pgfscope}%
\pgfpathrectangle{\pgfqpoint{0.688140in}{0.246400in}}{\pgfqpoint{1.493721in}{1.724800in}}%
\pgfusepath{clip}%
\pgfsetbuttcap%
\pgfsetroundjoin%
\pgfsetlinewidth{1.003750pt}%
\definecolor{currentstroke}{rgb}{0.000000,0.000000,0.000000}%
\pgfsetstrokecolor{currentstroke}%
\pgfsetdash{}{0pt}%
\pgfpathmoveto{\pgfqpoint{1.208679in}{1.500800in}}%
\pgfpathlineto{\pgfqpoint{1.661321in}{1.500800in}}%
\pgfusepath{stroke}%
\end{pgfscope}%
\begin{pgfscope}%
\pgfpathrectangle{\pgfqpoint{0.688140in}{0.246400in}}{\pgfqpoint{1.493721in}{1.724800in}}%
\pgfusepath{clip}%
\pgfsetbuttcap%
\pgfsetroundjoin%
\pgfsetlinewidth{1.003750pt}%
\definecolor{currentstroke}{rgb}{0.000000,0.000000,0.000000}%
\pgfsetstrokecolor{currentstroke}%
\pgfsetdash{}{0pt}%
\pgfpathmoveto{\pgfqpoint{1.208679in}{0.716800in}}%
\pgfpathlineto{\pgfqpoint{1.661321in}{0.716800in}}%
\pgfusepath{stroke}%
\end{pgfscope}%
\begin{pgfscope}%
\pgfpathrectangle{\pgfqpoint{0.688140in}{0.246400in}}{\pgfqpoint{1.493721in}{1.724800in}}%
\pgfusepath{clip}%
\pgfsetbuttcap%
\pgfsetroundjoin%
\pgfsetlinewidth{1.003750pt}%
\definecolor{currentstroke}{rgb}{0.000000,0.000000,0.000000}%
\pgfsetstrokecolor{currentstroke}%
\pgfsetdash{}{0pt}%
\pgfpathmoveto{\pgfqpoint{1.661321in}{0.716800in}}%
\pgfpathlineto{\pgfqpoint{1.887643in}{1.108800in}}%
\pgfusepath{stroke}%
\end{pgfscope}%
\begin{pgfscope}%
\pgfpathrectangle{\pgfqpoint{0.688140in}{0.246400in}}{\pgfqpoint{1.493721in}{1.724800in}}%
\pgfusepath{clip}%
\pgfsetbuttcap%
\pgfsetroundjoin%
\pgfsetlinewidth{1.003750pt}%
\definecolor{currentstroke}{rgb}{0.000000,0.000000,0.000000}%
\pgfsetstrokecolor{currentstroke}%
\pgfsetdash{}{0pt}%
\pgfpathmoveto{\pgfqpoint{0.982357in}{1.108800in}}%
\pgfpathlineto{\pgfqpoint{1.208679in}{0.716800in}}%
\pgfusepath{stroke}%
\end{pgfscope}%
\begin{pgfscope}%
\pgfpathrectangle{\pgfqpoint{0.688140in}{0.246400in}}{\pgfqpoint{1.493721in}{1.724800in}}%
\pgfusepath{clip}%
\pgfsetbuttcap%
\pgfsetroundjoin%
\pgfsetlinewidth{1.003750pt}%
\definecolor{currentstroke}{rgb}{0.000000,0.000000,0.000000}%
\pgfsetstrokecolor{currentstroke}%
\pgfsetdash{}{0pt}%
\pgfpathmoveto{\pgfqpoint{1.887643in}{1.108800in}}%
\pgfpathlineto{\pgfqpoint{1.661321in}{1.500800in}}%
\pgfusepath{stroke}%
\end{pgfscope}%
\begin{pgfscope}%
\pgfpathrectangle{\pgfqpoint{0.688140in}{0.246400in}}{\pgfqpoint{1.493721in}{1.724800in}}%
\pgfusepath{clip}%
\pgfsetrectcap%
\pgfsetroundjoin%
\pgfsetlinewidth{1.505625pt}%
\definecolor{currentstroke}{rgb}{0.000000,0.000000,0.000000}%
\pgfsetstrokecolor{currentstroke}%
\pgfsetdash{}{0pt}%
\pgfpathmoveto{\pgfqpoint{1.435000in}{1.108800in}}%
\pgfpathlineto{\pgfqpoint{1.774482in}{1.304800in}}%
\pgfpathlineto{\pgfqpoint{1.887643in}{1.108800in}}%
\pgfpathlineto{\pgfqpoint{1.435000in}{1.108800in}}%
\pgfusepath{stroke}%
\end{pgfscope}%
\begin{pgfscope}%
\definecolor{textcolor}{rgb}{0.000000,0.000000,0.000000}%
\pgfsetstrokecolor{textcolor}%
\pgfsetfillcolor{textcolor}%
\pgftext[x=1.462778in,y=1.192133in,left,base]{\color{textcolor}{\sffamily\fontsize{11.000000}{13.200000}\selectfont\catcode`\^=\active\def^{\ifmmode\sp\else\^{}\fi}\catcode`\%=\active\def%{\%}$\Gamma$}}%
\end{pgfscope}%
\begin{pgfscope}%
\definecolor{textcolor}{rgb}{0.000000,0.000000,0.000000}%
\pgfsetstrokecolor{textcolor}%
\pgfsetfillcolor{textcolor}%
\pgftext[x=1.760593in,y=1.388133in,left,base]{\color{textcolor}{\sffamily\fontsize{11.000000}{13.200000}\selectfont\catcode`\^=\active\def^{\ifmmode\sp\else\^{}\fi}\catcode`\%=\active\def%{\%}$\mathrm{M}$}}%
\end{pgfscope}%
\begin{pgfscope}%
\definecolor{textcolor}{rgb}{0.000000,0.000000,0.000000}%
\pgfsetstrokecolor{textcolor}%
\pgfsetfillcolor{textcolor}%
\pgftext[x=1.887643in,y=1.178244in,left,base]{\color{textcolor}{\sffamily\fontsize{11.000000}{13.200000}\selectfont\catcode`\^=\active\def^{\ifmmode\sp\else\^{}\fi}\catcode`\%=\active\def%{\%}$\mathrm{K}$}}%
\end{pgfscope}%
\begin{pgfscope}%
\definecolor{textcolor}{rgb}{0.000000,0.000000,0.000000}%
\pgfsetstrokecolor{textcolor}%
\pgfsetfillcolor{textcolor}%
\pgftext[x=1.949707in,y=1.539282in,left,base]{\color{textcolor}{\sffamily\fontsize{9.163000}{10.995600}\selectfont\catcode`\^=\active\def^{\ifmmode\sp\else\^{}\fi}\catcode`\%=\active\def%{\%}$\mathbf{b}_1$}}%
\end{pgfscope}%
\begin{pgfscope}%
\definecolor{textcolor}{rgb}{0.000000,0.000000,0.000000}%
\pgfsetstrokecolor{textcolor}%
\pgfsetfillcolor{textcolor}%
\pgftext[x=1.949707in,y=0.640884in,left,base]{\color{textcolor}{\sffamily\fontsize{9.163000}{10.995600}\selectfont\catcode`\^=\active\def^{\ifmmode\sp\else\^{}\fi}\catcode`\%=\active\def%{\%}$\mathbf{b}_2$}}%
\end{pgfscope}%
\end{pgfpicture}%
\makeatother%
\endgroup%

	\end{subfigure}
	\caption{(\subref{sfig:graphene lattice structure}) Graphene lattice structure and (\subref{sfig:graphene Brillouin zone}) Brilluoin zone} 
	\label{fig:Graphene lattice structure and Brilluoin zone}
\end{figure}
The primitive reciprocal lattice vectors \(\vb{b}_1\), \(\vb{b}_2\) fulfill \todo{labels on vectors}
\begin{align}
	\vb{a}_1 \cdot \vb{b}_1 &= \vb{a}_2 \cdot \vb{b}_2 = 2\pi \\
	\vb{a}_1 \cdot \vb{b}_2 &= \vb{a}_2 \cdot \vb{b}_1 = 0\;,
\end{align}
so we have:
\begin{align}
	\vb{b}_1 = \frac{2\pi}{a} \begin{pmatrix} 1 \\ \frac{1}{\sqrt{3}} \end{pmatrix},\;
	\vb{b}_2 = \frac{2\pi}{a} \begin{pmatrix} 1 \\ - \frac{1}{\sqrt{3}} \end{pmatrix}
\end{align}
The first Brilluoin zone of the hexagonal lattice is shown in \cref{sfig:graphene Brillouin zone}, with the points of high symmetry
\begin{align}
	\Gamma = \begin{pmatrix} 0 \\ 0 \end{pmatrix},\;
	\mathrm{M} = \frac{\pi}{a} \begin{pmatrix} 1 \\ \frac{1}{\sqrt{3}} \end{pmatrix},\;
	\mathrm{K} = \frac{4\pi}{3 a} \begin{pmatrix} 1 \\ 0 \end{pmatrix}\;.
\end{align}


\section{Dressed Graphene Model}\label{sec:dressed graphene model}

The model I am concerned with in this thesis consists of a Hubbard Hamiltonian (as introduced in \cref{sec:bcs-theory}) on a Graphene lattice, with one additional atom at one of the two sites in a unit cell, which I will call X\@.
This is shown in \cref{fig:eg-x model}.\todo{Work over image for dressed graphene lattice}
\begin{figure}[tb]
	\centering
	%% Creator: Matplotlib, PGF backend
%%
%% To include the figure in your LaTeX document, write
%%   \input{<filename>.pgf}
%%
%% Make sure the required packages are loaded in your preamble
%%   \usepackage{pgf}
%%
%% Also ensure that all the required font packages are loaded; for instance,
%% the lmodern package is sometimes necessary when using math font.
%%   \usepackage{lmodern}
%%
%% Figures using additional raster images can only be included by \input if
%% they are in the same directory as the main LaTeX file. For loading figures
%% from other directories you can use the `import` package
%%   \usepackage{import}
%%
%% and then include the figures with
%%   \import{<path to file>}{<filename>.pgf}
%%
%% Matplotlib used the following preamble
%%   \def\mathdefault#1{#1}
%%   \everymath=\expandafter{\the\everymath\displaystyle}
%%   \IfFileExists{scrextend.sty}{
%%     \usepackage[fontsize=11.000000pt]{scrextend}
%%   }{
%%     \renewcommand{\normalsize}{\fontsize{11.000000}{13.200000}\selectfont}
%%     \normalsize
%%   }
%%   \usepackage{fontspec}\usepackage{unicode-math}\setmathfont{texgyrepagella-math.otf}\setmainfont{texgyrepagella-math}
%%   \makeatletter\@ifpackageloaded{underscore}{}{\usepackage[strings]{underscore}}\makeatother
%%
\begingroup%
\makeatletter%
\begin{pgfpicture}%
\pgfpathrectangle{\pgfpointorigin}{\pgfqpoint{5.168000in}{2.584000in}}%
\pgfusepath{use as bounding box, clip}%
\begin{pgfscope}%
\pgfsetbuttcap%
\pgfsetmiterjoin%
\definecolor{currentfill}{rgb}{1.000000,1.000000,1.000000}%
\pgfsetfillcolor{currentfill}%
\pgfsetlinewidth{0.000000pt}%
\definecolor{currentstroke}{rgb}{1.000000,1.000000,1.000000}%
\pgfsetstrokecolor{currentstroke}%
\pgfsetdash{}{0pt}%
\pgfpathmoveto{\pgfqpoint{0.000000in}{0.000000in}}%
\pgfpathlineto{\pgfqpoint{5.168000in}{0.000000in}}%
\pgfpathlineto{\pgfqpoint{5.168000in}{2.584000in}}%
\pgfpathlineto{\pgfqpoint{0.000000in}{2.584000in}}%
\pgfpathlineto{\pgfqpoint{0.000000in}{0.000000in}}%
\pgfpathclose%
\pgfusepath{fill}%
\end{pgfscope}%
\begin{pgfscope}%
\pgfsetbuttcap%
\pgfsetmiterjoin%
\definecolor{currentfill}{rgb}{1.000000,1.000000,1.000000}%
\pgfsetfillcolor{currentfill}%
\pgfsetlinewidth{0.000000pt}%
\definecolor{currentstroke}{rgb}{0.000000,0.000000,0.000000}%
\pgfsetstrokecolor{currentstroke}%
\pgfsetstrokeopacity{0.000000}%
\pgfsetdash{}{0pt}%
\pgfpathmoveto{\pgfqpoint{1.374500in}{0.082500in}}%
\pgfpathlineto{\pgfqpoint{3.793500in}{0.082500in}}%
\pgfpathlineto{\pgfqpoint{3.793500in}{2.501500in}}%
\pgfpathlineto{\pgfqpoint{1.374500in}{2.501500in}}%
\pgfpathlineto{\pgfqpoint{1.374500in}{0.082500in}}%
\pgfpathclose%
\pgfusepath{fill}%
\end{pgfscope}%
\begin{pgfscope}%
\pgfpathrectangle{\pgfqpoint{1.374500in}{0.082500in}}{\pgfqpoint{2.419000in}{2.419000in}}%
\pgfusepath{clip}%
\pgfsetbuttcap%
\pgfsetroundjoin%
\pgfsetlinewidth{1.505625pt}%
\definecolor{currentstroke}{rgb}{0.000000,0.000000,0.000000}%
\pgfsetstrokecolor{currentstroke}%
\pgfsetdash{}{0pt}%
\pgfusepath{stroke}%
\end{pgfscope}%
\begin{pgfscope}%
\pgfpathrectangle{\pgfqpoint{1.374500in}{0.082500in}}{\pgfqpoint{2.419000in}{2.419000in}}%
\pgfusepath{clip}%
\pgfsetbuttcap%
\pgfsetroundjoin%
\pgfsetlinewidth{1.505625pt}%
\definecolor{currentstroke}{rgb}{0.000000,0.000000,0.000000}%
\pgfsetstrokecolor{currentstroke}%
\pgfsetdash{}{0pt}%
\pgfusepath{stroke}%
\end{pgfscope}%
\begin{pgfscope}%
\pgfpathrectangle{\pgfqpoint{1.374500in}{0.082500in}}{\pgfqpoint{2.419000in}{2.419000in}}%
\pgfusepath{clip}%
\pgfsetbuttcap%
\pgfsetroundjoin%
\pgfsetlinewidth{1.505625pt}%
\definecolor{currentstroke}{rgb}{0.000000,0.000000,0.000000}%
\pgfsetstrokecolor{currentstroke}%
\pgfsetdash{}{0pt}%
\pgfusepath{stroke}%
\end{pgfscope}%
\begin{pgfscope}%
\pgfpathrectangle{\pgfqpoint{1.374500in}{0.082500in}}{\pgfqpoint{2.419000in}{2.419000in}}%
\pgfusepath{clip}%
\pgfsetbuttcap%
\pgfsetroundjoin%
\pgfsetlinewidth{1.505625pt}%
\definecolor{currentstroke}{rgb}{0.000000,0.000000,0.000000}%
\pgfsetstrokecolor{currentstroke}%
\pgfsetdash{}{0pt}%
\pgfusepath{stroke}%
\end{pgfscope}%
\begin{pgfscope}%
\pgfpathrectangle{\pgfqpoint{1.374500in}{0.082500in}}{\pgfqpoint{2.419000in}{2.419000in}}%
\pgfusepath{clip}%
\pgfsetbuttcap%
\pgfsetroundjoin%
\pgfsetlinewidth{1.505625pt}%
\definecolor{currentstroke}{rgb}{0.000000,0.000000,0.000000}%
\pgfsetstrokecolor{currentstroke}%
\pgfsetdash{}{0pt}%
\pgfusepath{stroke}%
\end{pgfscope}%
\begin{pgfscope}%
\pgfpathrectangle{\pgfqpoint{1.374500in}{0.082500in}}{\pgfqpoint{2.419000in}{2.419000in}}%
\pgfusepath{clip}%
\pgfsetbuttcap%
\pgfsetroundjoin%
\pgfsetlinewidth{1.505625pt}%
\definecolor{currentstroke}{rgb}{0.000000,0.000000,0.000000}%
\pgfsetstrokecolor{currentstroke}%
\pgfsetdash{}{0pt}%
\pgfusepath{stroke}%
\end{pgfscope}%
\begin{pgfscope}%
\pgfpathrectangle{\pgfqpoint{1.374500in}{0.082500in}}{\pgfqpoint{2.419000in}{2.419000in}}%
\pgfusepath{clip}%
\pgfsetbuttcap%
\pgfsetroundjoin%
\pgfsetlinewidth{1.505625pt}%
\definecolor{currentstroke}{rgb}{0.000000,0.000000,0.000000}%
\pgfsetstrokecolor{currentstroke}%
\pgfsetdash{}{0pt}%
\pgfusepath{stroke}%
\end{pgfscope}%
\begin{pgfscope}%
\pgfpathrectangle{\pgfqpoint{1.374500in}{0.082500in}}{\pgfqpoint{2.419000in}{2.419000in}}%
\pgfusepath{clip}%
\pgfsetbuttcap%
\pgfsetroundjoin%
\pgfsetlinewidth{1.505625pt}%
\definecolor{currentstroke}{rgb}{0.000000,0.000000,0.000000}%
\pgfsetstrokecolor{currentstroke}%
\pgfsetdash{}{0pt}%
\pgfusepath{stroke}%
\end{pgfscope}%
\begin{pgfscope}%
\pgfpathrectangle{\pgfqpoint{1.374500in}{0.082500in}}{\pgfqpoint{2.419000in}{2.419000in}}%
\pgfusepath{clip}%
\pgfsetbuttcap%
\pgfsetroundjoin%
\pgfsetlinewidth{1.505625pt}%
\definecolor{currentstroke}{rgb}{0.000000,0.000000,0.000000}%
\pgfsetstrokecolor{currentstroke}%
\pgfsetdash{}{0pt}%
\pgfusepath{stroke}%
\end{pgfscope}%
\begin{pgfscope}%
\pgfpathrectangle{\pgfqpoint{1.374500in}{0.082500in}}{\pgfqpoint{2.419000in}{2.419000in}}%
\pgfusepath{clip}%
\pgfsetbuttcap%
\pgfsetroundjoin%
\pgfsetlinewidth{1.505625pt}%
\definecolor{currentstroke}{rgb}{0.000000,0.000000,0.000000}%
\pgfsetstrokecolor{currentstroke}%
\pgfsetdash{}{0pt}%
\pgfusepath{stroke}%
\end{pgfscope}%
\begin{pgfscope}%
\pgfpathrectangle{\pgfqpoint{1.374500in}{0.082500in}}{\pgfqpoint{2.419000in}{2.419000in}}%
\pgfusepath{clip}%
\pgfsetbuttcap%
\pgfsetroundjoin%
\pgfsetlinewidth{1.505625pt}%
\definecolor{currentstroke}{rgb}{0.000000,0.000000,0.000000}%
\pgfsetstrokecolor{currentstroke}%
\pgfsetdash{}{0pt}%
\pgfusepath{stroke}%
\end{pgfscope}%
\begin{pgfscope}%
\pgfpathrectangle{\pgfqpoint{1.374500in}{0.082500in}}{\pgfqpoint{2.419000in}{2.419000in}}%
\pgfusepath{clip}%
\pgfsetbuttcap%
\pgfsetroundjoin%
\pgfsetlinewidth{1.505625pt}%
\definecolor{currentstroke}{rgb}{0.000000,0.000000,0.000000}%
\pgfsetstrokecolor{currentstroke}%
\pgfsetdash{}{0pt}%
\pgfusepath{stroke}%
\end{pgfscope}%
\begin{pgfscope}%
\pgfpathrectangle{\pgfqpoint{1.374500in}{0.082500in}}{\pgfqpoint{2.419000in}{2.419000in}}%
\pgfusepath{clip}%
\pgfsetbuttcap%
\pgfsetroundjoin%
\pgfsetlinewidth{1.505625pt}%
\definecolor{currentstroke}{rgb}{0.000000,0.000000,0.000000}%
\pgfsetstrokecolor{currentstroke}%
\pgfsetdash{}{0pt}%
\pgfusepath{stroke}%
\end{pgfscope}%
\begin{pgfscope}%
\pgfpathrectangle{\pgfqpoint{1.374500in}{0.082500in}}{\pgfqpoint{2.419000in}{2.419000in}}%
\pgfusepath{clip}%
\pgfsetbuttcap%
\pgfsetroundjoin%
\pgfsetlinewidth{1.505625pt}%
\definecolor{currentstroke}{rgb}{0.000000,0.000000,0.000000}%
\pgfsetstrokecolor{currentstroke}%
\pgfsetdash{}{0pt}%
\pgfusepath{stroke}%
\end{pgfscope}%
\begin{pgfscope}%
\pgfpathrectangle{\pgfqpoint{1.374500in}{0.082500in}}{\pgfqpoint{2.419000in}{2.419000in}}%
\pgfusepath{clip}%
\pgfsetbuttcap%
\pgfsetroundjoin%
\pgfsetlinewidth{1.505625pt}%
\definecolor{currentstroke}{rgb}{0.000000,0.000000,0.000000}%
\pgfsetstrokecolor{currentstroke}%
\pgfsetdash{}{0pt}%
\pgfusepath{stroke}%
\end{pgfscope}%
\begin{pgfscope}%
\pgfpathrectangle{\pgfqpoint{1.374500in}{0.082500in}}{\pgfqpoint{2.419000in}{2.419000in}}%
\pgfusepath{clip}%
\pgfsetbuttcap%
\pgfsetroundjoin%
\pgfsetlinewidth{1.505625pt}%
\definecolor{currentstroke}{rgb}{0.000000,0.000000,0.000000}%
\pgfsetstrokecolor{currentstroke}%
\pgfsetdash{}{0pt}%
\pgfusepath{stroke}%
\end{pgfscope}%
\begin{pgfscope}%
\pgfpathrectangle{\pgfqpoint{1.374500in}{0.082500in}}{\pgfqpoint{2.419000in}{2.419000in}}%
\pgfusepath{clip}%
\pgfsetbuttcap%
\pgfsetroundjoin%
\pgfsetlinewidth{1.505625pt}%
\definecolor{currentstroke}{rgb}{0.000000,0.000000,0.000000}%
\pgfsetstrokecolor{currentstroke}%
\pgfsetdash{}{0pt}%
\pgfusepath{stroke}%
\end{pgfscope}%
\begin{pgfscope}%
\pgfpathrectangle{\pgfqpoint{1.374500in}{0.082500in}}{\pgfqpoint{2.419000in}{2.419000in}}%
\pgfusepath{clip}%
\pgfsetbuttcap%
\pgfsetroundjoin%
\pgfsetlinewidth{1.505625pt}%
\definecolor{currentstroke}{rgb}{0.000000,0.000000,0.000000}%
\pgfsetstrokecolor{currentstroke}%
\pgfsetdash{}{0pt}%
\pgfusepath{stroke}%
\end{pgfscope}%
\begin{pgfscope}%
\pgfpathrectangle{\pgfqpoint{1.374500in}{0.082500in}}{\pgfqpoint{2.419000in}{2.419000in}}%
\pgfusepath{clip}%
\pgfsetbuttcap%
\pgfsetroundjoin%
\pgfsetlinewidth{1.505625pt}%
\definecolor{currentstroke}{rgb}{0.000000,0.000000,0.000000}%
\pgfsetstrokecolor{currentstroke}%
\pgfsetdash{}{0pt}%
\pgfusepath{stroke}%
\end{pgfscope}%
\begin{pgfscope}%
\pgfpathrectangle{\pgfqpoint{1.374500in}{0.082500in}}{\pgfqpoint{2.419000in}{2.419000in}}%
\pgfusepath{clip}%
\pgfsetbuttcap%
\pgfsetroundjoin%
\pgfsetlinewidth{1.505625pt}%
\definecolor{currentstroke}{rgb}{0.000000,0.000000,0.000000}%
\pgfsetstrokecolor{currentstroke}%
\pgfsetdash{}{0pt}%
\pgfusepath{stroke}%
\end{pgfscope}%
\begin{pgfscope}%
\pgfpathrectangle{\pgfqpoint{1.374500in}{0.082500in}}{\pgfqpoint{2.419000in}{2.419000in}}%
\pgfusepath{clip}%
\pgfsetbuttcap%
\pgfsetroundjoin%
\pgfsetlinewidth{1.505625pt}%
\definecolor{currentstroke}{rgb}{0.000000,0.000000,0.000000}%
\pgfsetstrokecolor{currentstroke}%
\pgfsetdash{}{0pt}%
\pgfusepath{stroke}%
\end{pgfscope}%
\begin{pgfscope}%
\pgfpathrectangle{\pgfqpoint{1.374500in}{0.082500in}}{\pgfqpoint{2.419000in}{2.419000in}}%
\pgfusepath{clip}%
\pgfsetbuttcap%
\pgfsetroundjoin%
\pgfsetlinewidth{1.505625pt}%
\definecolor{currentstroke}{rgb}{0.000000,0.000000,0.000000}%
\pgfsetstrokecolor{currentstroke}%
\pgfsetdash{}{0pt}%
\pgfusepath{stroke}%
\end{pgfscope}%
\begin{pgfscope}%
\pgfpathrectangle{\pgfqpoint{1.374500in}{0.082500in}}{\pgfqpoint{2.419000in}{2.419000in}}%
\pgfusepath{clip}%
\pgfsetbuttcap%
\pgfsetroundjoin%
\pgfsetlinewidth{1.505625pt}%
\definecolor{currentstroke}{rgb}{0.000000,0.000000,0.000000}%
\pgfsetstrokecolor{currentstroke}%
\pgfsetdash{}{0pt}%
\pgfusepath{stroke}%
\end{pgfscope}%
\begin{pgfscope}%
\pgfpathrectangle{\pgfqpoint{1.374500in}{0.082500in}}{\pgfqpoint{2.419000in}{2.419000in}}%
\pgfusepath{clip}%
\pgfsetbuttcap%
\pgfsetroundjoin%
\pgfsetlinewidth{1.505625pt}%
\definecolor{currentstroke}{rgb}{0.000000,0.000000,0.000000}%
\pgfsetstrokecolor{currentstroke}%
\pgfsetdash{}{0pt}%
\pgfusepath{stroke}%
\end{pgfscope}%
\begin{pgfscope}%
\pgfpathrectangle{\pgfqpoint{1.374500in}{0.082500in}}{\pgfqpoint{2.419000in}{2.419000in}}%
\pgfusepath{clip}%
\pgfsetbuttcap%
\pgfsetroundjoin%
\pgfsetlinewidth{1.505625pt}%
\definecolor{currentstroke}{rgb}{0.000000,0.000000,0.000000}%
\pgfsetstrokecolor{currentstroke}%
\pgfsetdash{}{0pt}%
\pgfusepath{stroke}%
\end{pgfscope}%
\begin{pgfscope}%
\pgfpathrectangle{\pgfqpoint{1.374500in}{0.082500in}}{\pgfqpoint{2.419000in}{2.419000in}}%
\pgfusepath{clip}%
\pgfsetbuttcap%
\pgfsetroundjoin%
\pgfsetlinewidth{1.505625pt}%
\definecolor{currentstroke}{rgb}{0.000000,0.000000,0.000000}%
\pgfsetstrokecolor{currentstroke}%
\pgfsetdash{}{0pt}%
\pgfusepath{stroke}%
\end{pgfscope}%
\begin{pgfscope}%
\pgfpathrectangle{\pgfqpoint{1.374500in}{0.082500in}}{\pgfqpoint{2.419000in}{2.419000in}}%
\pgfusepath{clip}%
\pgfsetbuttcap%
\pgfsetroundjoin%
\pgfsetlinewidth{1.505625pt}%
\definecolor{currentstroke}{rgb}{0.000000,0.000000,0.000000}%
\pgfsetstrokecolor{currentstroke}%
\pgfsetdash{}{0pt}%
\pgfusepath{stroke}%
\end{pgfscope}%
\begin{pgfscope}%
\pgfpathrectangle{\pgfqpoint{1.374500in}{0.082500in}}{\pgfqpoint{2.419000in}{2.419000in}}%
\pgfusepath{clip}%
\pgfsetbuttcap%
\pgfsetroundjoin%
\pgfsetlinewidth{1.505625pt}%
\definecolor{currentstroke}{rgb}{0.000000,0.000000,0.000000}%
\pgfsetstrokecolor{currentstroke}%
\pgfsetdash{}{0pt}%
\pgfusepath{stroke}%
\end{pgfscope}%
\begin{pgfscope}%
\pgfpathrectangle{\pgfqpoint{1.374500in}{0.082500in}}{\pgfqpoint{2.419000in}{2.419000in}}%
\pgfusepath{clip}%
\pgfsetbuttcap%
\pgfsetroundjoin%
\pgfsetlinewidth{1.505625pt}%
\definecolor{currentstroke}{rgb}{0.000000,0.000000,0.000000}%
\pgfsetstrokecolor{currentstroke}%
\pgfsetdash{}{0pt}%
\pgfusepath{stroke}%
\end{pgfscope}%
\begin{pgfscope}%
\pgfpathrectangle{\pgfqpoint{1.374500in}{0.082500in}}{\pgfqpoint{2.419000in}{2.419000in}}%
\pgfusepath{clip}%
\pgfsetbuttcap%
\pgfsetroundjoin%
\pgfsetlinewidth{1.505625pt}%
\definecolor{currentstroke}{rgb}{0.000000,0.000000,0.000000}%
\pgfsetstrokecolor{currentstroke}%
\pgfsetdash{}{0pt}%
\pgfusepath{stroke}%
\end{pgfscope}%
\begin{pgfscope}%
\pgfpathrectangle{\pgfqpoint{1.374500in}{0.082500in}}{\pgfqpoint{2.419000in}{2.419000in}}%
\pgfusepath{clip}%
\pgfsetbuttcap%
\pgfsetroundjoin%
\pgfsetlinewidth{1.505625pt}%
\definecolor{currentstroke}{rgb}{0.000000,0.000000,0.000000}%
\pgfsetstrokecolor{currentstroke}%
\pgfsetdash{}{0pt}%
\pgfusepath{stroke}%
\end{pgfscope}%
\begin{pgfscope}%
\pgfpathrectangle{\pgfqpoint{1.374500in}{0.082500in}}{\pgfqpoint{2.419000in}{2.419000in}}%
\pgfusepath{clip}%
\pgfsetbuttcap%
\pgfsetroundjoin%
\pgfsetlinewidth{1.505625pt}%
\definecolor{currentstroke}{rgb}{0.000000,0.000000,0.000000}%
\pgfsetstrokecolor{currentstroke}%
\pgfsetdash{}{0pt}%
\pgfusepath{stroke}%
\end{pgfscope}%
\begin{pgfscope}%
\pgfpathrectangle{\pgfqpoint{1.374500in}{0.082500in}}{\pgfqpoint{2.419000in}{2.419000in}}%
\pgfusepath{clip}%
\pgfsetbuttcap%
\pgfsetroundjoin%
\pgfsetlinewidth{1.505625pt}%
\definecolor{currentstroke}{rgb}{0.000000,0.000000,0.000000}%
\pgfsetstrokecolor{currentstroke}%
\pgfsetdash{}{0pt}%
\pgfusepath{stroke}%
\end{pgfscope}%
\begin{pgfscope}%
\pgfpathrectangle{\pgfqpoint{1.374500in}{0.082500in}}{\pgfqpoint{2.419000in}{2.419000in}}%
\pgfusepath{clip}%
\pgfsetbuttcap%
\pgfsetroundjoin%
\pgfsetlinewidth{1.505625pt}%
\definecolor{currentstroke}{rgb}{0.000000,0.000000,0.000000}%
\pgfsetstrokecolor{currentstroke}%
\pgfsetdash{}{0pt}%
\pgfusepath{stroke}%
\end{pgfscope}%
\begin{pgfscope}%
\pgfpathrectangle{\pgfqpoint{1.374500in}{0.082500in}}{\pgfqpoint{2.419000in}{2.419000in}}%
\pgfusepath{clip}%
\pgfsetbuttcap%
\pgfsetroundjoin%
\pgfsetlinewidth{1.505625pt}%
\definecolor{currentstroke}{rgb}{0.000000,0.000000,0.000000}%
\pgfsetstrokecolor{currentstroke}%
\pgfsetdash{}{0pt}%
\pgfusepath{stroke}%
\end{pgfscope}%
\begin{pgfscope}%
\pgfpathrectangle{\pgfqpoint{1.374500in}{0.082500in}}{\pgfqpoint{2.419000in}{2.419000in}}%
\pgfusepath{clip}%
\pgfsetbuttcap%
\pgfsetroundjoin%
\pgfsetlinewidth{1.505625pt}%
\definecolor{currentstroke}{rgb}{0.000000,0.000000,0.000000}%
\pgfsetstrokecolor{currentstroke}%
\pgfsetdash{}{0pt}%
\pgfusepath{stroke}%
\end{pgfscope}%
\begin{pgfscope}%
\pgfpathrectangle{\pgfqpoint{1.374500in}{0.082500in}}{\pgfqpoint{2.419000in}{2.419000in}}%
\pgfusepath{clip}%
\pgfsetbuttcap%
\pgfsetroundjoin%
\pgfsetlinewidth{1.505625pt}%
\definecolor{currentstroke}{rgb}{0.000000,0.000000,0.000000}%
\pgfsetstrokecolor{currentstroke}%
\pgfsetdash{}{0pt}%
\pgfusepath{stroke}%
\end{pgfscope}%
\begin{pgfscope}%
\pgfpathrectangle{\pgfqpoint{1.374500in}{0.082500in}}{\pgfqpoint{2.419000in}{2.419000in}}%
\pgfusepath{clip}%
\pgfsetbuttcap%
\pgfsetroundjoin%
\pgfsetlinewidth{1.505625pt}%
\definecolor{currentstroke}{rgb}{0.000000,0.000000,0.000000}%
\pgfsetstrokecolor{currentstroke}%
\pgfsetdash{}{0pt}%
\pgfusepath{stroke}%
\end{pgfscope}%
\begin{pgfscope}%
\pgfpathrectangle{\pgfqpoint{1.374500in}{0.082500in}}{\pgfqpoint{2.419000in}{2.419000in}}%
\pgfusepath{clip}%
\pgfsetbuttcap%
\pgfsetroundjoin%
\pgfsetlinewidth{1.505625pt}%
\definecolor{currentstroke}{rgb}{0.000000,0.000000,0.000000}%
\pgfsetstrokecolor{currentstroke}%
\pgfsetdash{}{0pt}%
\pgfusepath{stroke}%
\end{pgfscope}%
\begin{pgfscope}%
\pgfpathrectangle{\pgfqpoint{1.374500in}{0.082500in}}{\pgfqpoint{2.419000in}{2.419000in}}%
\pgfusepath{clip}%
\pgfsetbuttcap%
\pgfsetroundjoin%
\pgfsetlinewidth{1.505625pt}%
\definecolor{currentstroke}{rgb}{0.000000,0.000000,0.000000}%
\pgfsetstrokecolor{currentstroke}%
\pgfsetdash{}{0pt}%
\pgfusepath{stroke}%
\end{pgfscope}%
\begin{pgfscope}%
\pgfpathrectangle{\pgfqpoint{1.374500in}{0.082500in}}{\pgfqpoint{2.419000in}{2.419000in}}%
\pgfusepath{clip}%
\pgfsetbuttcap%
\pgfsetroundjoin%
\pgfsetlinewidth{1.505625pt}%
\definecolor{currentstroke}{rgb}{0.000000,0.000000,0.000000}%
\pgfsetstrokecolor{currentstroke}%
\pgfsetdash{}{0pt}%
\pgfusepath{stroke}%
\end{pgfscope}%
\begin{pgfscope}%
\pgfpathrectangle{\pgfqpoint{1.374500in}{0.082500in}}{\pgfqpoint{2.419000in}{2.419000in}}%
\pgfusepath{clip}%
\pgfsetbuttcap%
\pgfsetroundjoin%
\pgfsetlinewidth{1.505625pt}%
\definecolor{currentstroke}{rgb}{0.000000,0.000000,0.000000}%
\pgfsetstrokecolor{currentstroke}%
\pgfsetdash{}{0pt}%
\pgfusepath{stroke}%
\end{pgfscope}%
\begin{pgfscope}%
\pgfpathrectangle{\pgfqpoint{1.374500in}{0.082500in}}{\pgfqpoint{2.419000in}{2.419000in}}%
\pgfusepath{clip}%
\pgfsetbuttcap%
\pgfsetroundjoin%
\pgfsetlinewidth{1.505625pt}%
\definecolor{currentstroke}{rgb}{0.000000,0.000000,0.000000}%
\pgfsetstrokecolor{currentstroke}%
\pgfsetdash{}{0pt}%
\pgfusepath{stroke}%
\end{pgfscope}%
\begin{pgfscope}%
\pgfpathrectangle{\pgfqpoint{1.374500in}{0.082500in}}{\pgfqpoint{2.419000in}{2.419000in}}%
\pgfusepath{clip}%
\pgfsetbuttcap%
\pgfsetroundjoin%
\pgfsetlinewidth{1.505625pt}%
\definecolor{currentstroke}{rgb}{0.000000,0.000000,0.000000}%
\pgfsetstrokecolor{currentstroke}%
\pgfsetdash{}{0pt}%
\pgfusepath{stroke}%
\end{pgfscope}%
\begin{pgfscope}%
\pgfpathrectangle{\pgfqpoint{1.374500in}{0.082500in}}{\pgfqpoint{2.419000in}{2.419000in}}%
\pgfusepath{clip}%
\pgfsetbuttcap%
\pgfsetroundjoin%
\pgfsetlinewidth{1.505625pt}%
\definecolor{currentstroke}{rgb}{0.000000,0.000000,0.000000}%
\pgfsetstrokecolor{currentstroke}%
\pgfsetdash{}{0pt}%
\pgfusepath{stroke}%
\end{pgfscope}%
\begin{pgfscope}%
\pgfpathrectangle{\pgfqpoint{1.374500in}{0.082500in}}{\pgfqpoint{2.419000in}{2.419000in}}%
\pgfusepath{clip}%
\pgfsetbuttcap%
\pgfsetroundjoin%
\pgfsetlinewidth{1.505625pt}%
\definecolor{currentstroke}{rgb}{0.000000,0.000000,0.000000}%
\pgfsetstrokecolor{currentstroke}%
\pgfsetdash{}{0pt}%
\pgfusepath{stroke}%
\end{pgfscope}%
\begin{pgfscope}%
\pgfpathrectangle{\pgfqpoint{1.374500in}{0.082500in}}{\pgfqpoint{2.419000in}{2.419000in}}%
\pgfusepath{clip}%
\pgfsetbuttcap%
\pgfsetroundjoin%
\pgfsetlinewidth{1.505625pt}%
\definecolor{currentstroke}{rgb}{0.000000,0.000000,0.000000}%
\pgfsetstrokecolor{currentstroke}%
\pgfsetdash{}{0pt}%
\pgfusepath{stroke}%
\end{pgfscope}%
\begin{pgfscope}%
\pgfpathrectangle{\pgfqpoint{1.374500in}{0.082500in}}{\pgfqpoint{2.419000in}{2.419000in}}%
\pgfusepath{clip}%
\pgfsetbuttcap%
\pgfsetroundjoin%
\pgfsetlinewidth{1.505625pt}%
\definecolor{currentstroke}{rgb}{0.000000,0.000000,0.000000}%
\pgfsetstrokecolor{currentstroke}%
\pgfsetdash{}{0pt}%
\pgfusepath{stroke}%
\end{pgfscope}%
\begin{pgfscope}%
\pgfpathrectangle{\pgfqpoint{1.374500in}{0.082500in}}{\pgfqpoint{2.419000in}{2.419000in}}%
\pgfusepath{clip}%
\pgfsetbuttcap%
\pgfsetroundjoin%
\pgfsetlinewidth{1.505625pt}%
\definecolor{currentstroke}{rgb}{0.000000,0.000000,0.000000}%
\pgfsetstrokecolor{currentstroke}%
\pgfsetdash{}{0pt}%
\pgfusepath{stroke}%
\end{pgfscope}%
\begin{pgfscope}%
\pgfpathrectangle{\pgfqpoint{1.374500in}{0.082500in}}{\pgfqpoint{2.419000in}{2.419000in}}%
\pgfusepath{clip}%
\pgfsetbuttcap%
\pgfsetroundjoin%
\pgfsetlinewidth{1.505625pt}%
\definecolor{currentstroke}{rgb}{0.000000,0.000000,0.000000}%
\pgfsetstrokecolor{currentstroke}%
\pgfsetdash{}{0pt}%
\pgfusepath{stroke}%
\end{pgfscope}%
\begin{pgfscope}%
\pgfpathrectangle{\pgfqpoint{1.374500in}{0.082500in}}{\pgfqpoint{2.419000in}{2.419000in}}%
\pgfusepath{clip}%
\pgfsetbuttcap%
\pgfsetroundjoin%
\pgfsetlinewidth{1.505625pt}%
\definecolor{currentstroke}{rgb}{0.000000,0.000000,0.000000}%
\pgfsetstrokecolor{currentstroke}%
\pgfsetdash{}{0pt}%
\pgfusepath{stroke}%
\end{pgfscope}%
\begin{pgfscope}%
\pgfpathrectangle{\pgfqpoint{1.374500in}{0.082500in}}{\pgfqpoint{2.419000in}{2.419000in}}%
\pgfusepath{clip}%
\pgfsetbuttcap%
\pgfsetroundjoin%
\pgfsetlinewidth{1.505625pt}%
\definecolor{currentstroke}{rgb}{0.000000,0.000000,0.000000}%
\pgfsetstrokecolor{currentstroke}%
\pgfsetdash{}{0pt}%
\pgfusepath{stroke}%
\end{pgfscope}%
\begin{pgfscope}%
\pgfpathrectangle{\pgfqpoint{1.374500in}{0.082500in}}{\pgfqpoint{2.419000in}{2.419000in}}%
\pgfusepath{clip}%
\pgfsetbuttcap%
\pgfsetroundjoin%
\pgfsetlinewidth{1.505625pt}%
\definecolor{currentstroke}{rgb}{0.000000,0.000000,0.000000}%
\pgfsetstrokecolor{currentstroke}%
\pgfsetdash{}{0pt}%
\pgfusepath{stroke}%
\end{pgfscope}%
\begin{pgfscope}%
\pgfpathrectangle{\pgfqpoint{1.374500in}{0.082500in}}{\pgfqpoint{2.419000in}{2.419000in}}%
\pgfusepath{clip}%
\pgfsetbuttcap%
\pgfsetroundjoin%
\pgfsetlinewidth{1.505625pt}%
\definecolor{currentstroke}{rgb}{0.000000,0.000000,0.000000}%
\pgfsetstrokecolor{currentstroke}%
\pgfsetdash{}{0pt}%
\pgfusepath{stroke}%
\end{pgfscope}%
\begin{pgfscope}%
\pgfpathrectangle{\pgfqpoint{1.374500in}{0.082500in}}{\pgfqpoint{2.419000in}{2.419000in}}%
\pgfusepath{clip}%
\pgfsetbuttcap%
\pgfsetroundjoin%
\pgfsetlinewidth{1.505625pt}%
\definecolor{currentstroke}{rgb}{0.000000,0.000000,0.000000}%
\pgfsetstrokecolor{currentstroke}%
\pgfsetdash{}{0pt}%
\pgfusepath{stroke}%
\end{pgfscope}%
\begin{pgfscope}%
\pgfpathrectangle{\pgfqpoint{1.374500in}{0.082500in}}{\pgfqpoint{2.419000in}{2.419000in}}%
\pgfusepath{clip}%
\pgfsetbuttcap%
\pgfsetroundjoin%
\pgfsetlinewidth{1.505625pt}%
\definecolor{currentstroke}{rgb}{0.000000,0.000000,0.000000}%
\pgfsetstrokecolor{currentstroke}%
\pgfsetdash{}{0pt}%
\pgfusepath{stroke}%
\end{pgfscope}%
\begin{pgfscope}%
\pgfpathrectangle{\pgfqpoint{1.374500in}{0.082500in}}{\pgfqpoint{2.419000in}{2.419000in}}%
\pgfusepath{clip}%
\pgfsetbuttcap%
\pgfsetroundjoin%
\pgfsetlinewidth{1.505625pt}%
\definecolor{currentstroke}{rgb}{0.000000,0.000000,0.000000}%
\pgfsetstrokecolor{currentstroke}%
\pgfsetdash{}{0pt}%
\pgfusepath{stroke}%
\end{pgfscope}%
\begin{pgfscope}%
\pgfpathrectangle{\pgfqpoint{1.374500in}{0.082500in}}{\pgfqpoint{2.419000in}{2.419000in}}%
\pgfusepath{clip}%
\pgfsetbuttcap%
\pgfsetroundjoin%
\pgfsetlinewidth{1.505625pt}%
\definecolor{currentstroke}{rgb}{0.000000,0.000000,0.000000}%
\pgfsetstrokecolor{currentstroke}%
\pgfsetdash{}{0pt}%
\pgfusepath{stroke}%
\end{pgfscope}%
\begin{pgfscope}%
\pgfpathrectangle{\pgfqpoint{1.374500in}{0.082500in}}{\pgfqpoint{2.419000in}{2.419000in}}%
\pgfusepath{clip}%
\pgfsetbuttcap%
\pgfsetroundjoin%
\pgfsetlinewidth{1.505625pt}%
\definecolor{currentstroke}{rgb}{0.000000,0.000000,0.000000}%
\pgfsetstrokecolor{currentstroke}%
\pgfsetdash{}{0pt}%
\pgfusepath{stroke}%
\end{pgfscope}%
\begin{pgfscope}%
\pgfpathrectangle{\pgfqpoint{1.374500in}{0.082500in}}{\pgfqpoint{2.419000in}{2.419000in}}%
\pgfusepath{clip}%
\pgfsetbuttcap%
\pgfsetroundjoin%
\pgfsetlinewidth{1.505625pt}%
\definecolor{currentstroke}{rgb}{0.000000,0.000000,0.000000}%
\pgfsetstrokecolor{currentstroke}%
\pgfsetdash{}{0pt}%
\pgfusepath{stroke}%
\end{pgfscope}%
\begin{pgfscope}%
\pgfpathrectangle{\pgfqpoint{1.374500in}{0.082500in}}{\pgfqpoint{2.419000in}{2.419000in}}%
\pgfusepath{clip}%
\pgfsetbuttcap%
\pgfsetroundjoin%
\pgfsetlinewidth{1.505625pt}%
\definecolor{currentstroke}{rgb}{0.000000,0.000000,0.000000}%
\pgfsetstrokecolor{currentstroke}%
\pgfsetdash{}{0pt}%
\pgfusepath{stroke}%
\end{pgfscope}%
\begin{pgfscope}%
\pgfpathrectangle{\pgfqpoint{1.374500in}{0.082500in}}{\pgfqpoint{2.419000in}{2.419000in}}%
\pgfusepath{clip}%
\pgfsetbuttcap%
\pgfsetroundjoin%
\pgfsetlinewidth{1.505625pt}%
\definecolor{currentstroke}{rgb}{0.000000,0.000000,0.000000}%
\pgfsetstrokecolor{currentstroke}%
\pgfsetdash{}{0pt}%
\pgfusepath{stroke}%
\end{pgfscope}%
\begin{pgfscope}%
\pgfpathrectangle{\pgfqpoint{1.374500in}{0.082500in}}{\pgfqpoint{2.419000in}{2.419000in}}%
\pgfusepath{clip}%
\pgfsetbuttcap%
\pgfsetroundjoin%
\pgfsetlinewidth{1.505625pt}%
\definecolor{currentstroke}{rgb}{0.000000,0.000000,0.000000}%
\pgfsetstrokecolor{currentstroke}%
\pgfsetdash{}{0pt}%
\pgfusepath{stroke}%
\end{pgfscope}%
\begin{pgfscope}%
\pgfpathrectangle{\pgfqpoint{1.374500in}{0.082500in}}{\pgfqpoint{2.419000in}{2.419000in}}%
\pgfusepath{clip}%
\pgfsetbuttcap%
\pgfsetroundjoin%
\pgfsetlinewidth{1.505625pt}%
\definecolor{currentstroke}{rgb}{0.000000,0.000000,0.000000}%
\pgfsetstrokecolor{currentstroke}%
\pgfsetdash{}{0pt}%
\pgfusepath{stroke}%
\end{pgfscope}%
\begin{pgfscope}%
\pgfpathrectangle{\pgfqpoint{1.374500in}{0.082500in}}{\pgfqpoint{2.419000in}{2.419000in}}%
\pgfusepath{clip}%
\pgfsetbuttcap%
\pgfsetroundjoin%
\pgfsetlinewidth{1.505625pt}%
\definecolor{currentstroke}{rgb}{0.000000,0.000000,0.000000}%
\pgfsetstrokecolor{currentstroke}%
\pgfsetdash{}{0pt}%
\pgfusepath{stroke}%
\end{pgfscope}%
\begin{pgfscope}%
\pgfpathrectangle{\pgfqpoint{1.374500in}{0.082500in}}{\pgfqpoint{2.419000in}{2.419000in}}%
\pgfusepath{clip}%
\pgfsetbuttcap%
\pgfsetroundjoin%
\pgfsetlinewidth{1.505625pt}%
\definecolor{currentstroke}{rgb}{0.000000,0.000000,0.000000}%
\pgfsetstrokecolor{currentstroke}%
\pgfsetdash{}{0pt}%
\pgfusepath{stroke}%
\end{pgfscope}%
\begin{pgfscope}%
\pgfpathrectangle{\pgfqpoint{1.374500in}{0.082500in}}{\pgfqpoint{2.419000in}{2.419000in}}%
\pgfusepath{clip}%
\pgfsetbuttcap%
\pgfsetroundjoin%
\pgfsetlinewidth{1.505625pt}%
\definecolor{currentstroke}{rgb}{0.000000,0.000000,0.000000}%
\pgfsetstrokecolor{currentstroke}%
\pgfsetdash{}{0pt}%
\pgfusepath{stroke}%
\end{pgfscope}%
\begin{pgfscope}%
\pgfpathrectangle{\pgfqpoint{1.374500in}{0.082500in}}{\pgfqpoint{2.419000in}{2.419000in}}%
\pgfusepath{clip}%
\pgfsetbuttcap%
\pgfsetroundjoin%
\pgfsetlinewidth{1.505625pt}%
\definecolor{currentstroke}{rgb}{0.000000,0.000000,0.000000}%
\pgfsetstrokecolor{currentstroke}%
\pgfsetdash{}{0pt}%
\pgfusepath{stroke}%
\end{pgfscope}%
\begin{pgfscope}%
\pgfpathrectangle{\pgfqpoint{1.374500in}{0.082500in}}{\pgfqpoint{2.419000in}{2.419000in}}%
\pgfusepath{clip}%
\pgfsetbuttcap%
\pgfsetroundjoin%
\pgfsetlinewidth{1.505625pt}%
\definecolor{currentstroke}{rgb}{0.000000,0.000000,0.000000}%
\pgfsetstrokecolor{currentstroke}%
\pgfsetdash{}{0pt}%
\pgfusepath{stroke}%
\end{pgfscope}%
\begin{pgfscope}%
\pgfpathrectangle{\pgfqpoint{1.374500in}{0.082500in}}{\pgfqpoint{2.419000in}{2.419000in}}%
\pgfusepath{clip}%
\pgfsetbuttcap%
\pgfsetroundjoin%
\pgfsetlinewidth{1.505625pt}%
\definecolor{currentstroke}{rgb}{0.000000,0.000000,0.000000}%
\pgfsetstrokecolor{currentstroke}%
\pgfsetdash{}{0pt}%
\pgfusepath{stroke}%
\end{pgfscope}%
\begin{pgfscope}%
\pgfpathrectangle{\pgfqpoint{1.374500in}{0.082500in}}{\pgfqpoint{2.419000in}{2.419000in}}%
\pgfusepath{clip}%
\pgfsetbuttcap%
\pgfsetroundjoin%
\pgfsetlinewidth{1.505625pt}%
\definecolor{currentstroke}{rgb}{0.000000,0.000000,0.000000}%
\pgfsetstrokecolor{currentstroke}%
\pgfsetdash{}{0pt}%
\pgfusepath{stroke}%
\end{pgfscope}%
\begin{pgfscope}%
\pgfpathrectangle{\pgfqpoint{1.374500in}{0.082500in}}{\pgfqpoint{2.419000in}{2.419000in}}%
\pgfusepath{clip}%
\pgfsetbuttcap%
\pgfsetroundjoin%
\pgfsetlinewidth{1.505625pt}%
\definecolor{currentstroke}{rgb}{0.000000,0.000000,0.000000}%
\pgfsetstrokecolor{currentstroke}%
\pgfsetdash{}{0pt}%
\pgfusepath{stroke}%
\end{pgfscope}%
\begin{pgfscope}%
\pgfpathrectangle{\pgfqpoint{1.374500in}{0.082500in}}{\pgfqpoint{2.419000in}{2.419000in}}%
\pgfusepath{clip}%
\pgfsetbuttcap%
\pgfsetroundjoin%
\pgfsetlinewidth{1.505625pt}%
\definecolor{currentstroke}{rgb}{0.000000,0.000000,0.000000}%
\pgfsetstrokecolor{currentstroke}%
\pgfsetdash{}{0pt}%
\pgfusepath{stroke}%
\end{pgfscope}%
\begin{pgfscope}%
\pgfpathrectangle{\pgfqpoint{1.374500in}{0.082500in}}{\pgfqpoint{2.419000in}{2.419000in}}%
\pgfusepath{clip}%
\pgfsetbuttcap%
\pgfsetroundjoin%
\pgfsetlinewidth{1.505625pt}%
\definecolor{currentstroke}{rgb}{0.000000,0.000000,0.000000}%
\pgfsetstrokecolor{currentstroke}%
\pgfsetdash{}{0pt}%
\pgfusepath{stroke}%
\end{pgfscope}%
\begin{pgfscope}%
\pgfpathrectangle{\pgfqpoint{1.374500in}{0.082500in}}{\pgfqpoint{2.419000in}{2.419000in}}%
\pgfusepath{clip}%
\pgfsetbuttcap%
\pgfsetroundjoin%
\pgfsetlinewidth{1.505625pt}%
\definecolor{currentstroke}{rgb}{0.000000,0.000000,0.000000}%
\pgfsetstrokecolor{currentstroke}%
\pgfsetdash{}{0pt}%
\pgfusepath{stroke}%
\end{pgfscope}%
\begin{pgfscope}%
\pgfpathrectangle{\pgfqpoint{1.374500in}{0.082500in}}{\pgfqpoint{2.419000in}{2.419000in}}%
\pgfusepath{clip}%
\pgfsetbuttcap%
\pgfsetroundjoin%
\pgfsetlinewidth{1.505625pt}%
\definecolor{currentstroke}{rgb}{0.000000,0.000000,0.000000}%
\pgfsetstrokecolor{currentstroke}%
\pgfsetdash{}{0pt}%
\pgfusepath{stroke}%
\end{pgfscope}%
\begin{pgfscope}%
\pgfpathrectangle{\pgfqpoint{1.374500in}{0.082500in}}{\pgfqpoint{2.419000in}{2.419000in}}%
\pgfusepath{clip}%
\pgfsetbuttcap%
\pgfsetroundjoin%
\pgfsetlinewidth{1.505625pt}%
\definecolor{currentstroke}{rgb}{0.000000,0.000000,0.000000}%
\pgfsetstrokecolor{currentstroke}%
\pgfsetdash{}{0pt}%
\pgfusepath{stroke}%
\end{pgfscope}%
\begin{pgfscope}%
\pgfpathrectangle{\pgfqpoint{1.374500in}{0.082500in}}{\pgfqpoint{2.419000in}{2.419000in}}%
\pgfusepath{clip}%
\pgfsetbuttcap%
\pgfsetroundjoin%
\pgfsetlinewidth{1.505625pt}%
\definecolor{currentstroke}{rgb}{0.000000,0.000000,0.000000}%
\pgfsetstrokecolor{currentstroke}%
\pgfsetdash{}{0pt}%
\pgfusepath{stroke}%
\end{pgfscope}%
\begin{pgfscope}%
\pgfpathrectangle{\pgfqpoint{1.374500in}{0.082500in}}{\pgfqpoint{2.419000in}{2.419000in}}%
\pgfusepath{clip}%
\pgfsetbuttcap%
\pgfsetroundjoin%
\pgfsetlinewidth{1.505625pt}%
\definecolor{currentstroke}{rgb}{0.000000,0.000000,0.000000}%
\pgfsetstrokecolor{currentstroke}%
\pgfsetdash{}{0pt}%
\pgfusepath{stroke}%
\end{pgfscope}%
\begin{pgfscope}%
\pgfpathrectangle{\pgfqpoint{1.374500in}{0.082500in}}{\pgfqpoint{2.419000in}{2.419000in}}%
\pgfusepath{clip}%
\pgfsetbuttcap%
\pgfsetroundjoin%
\pgfsetlinewidth{1.505625pt}%
\definecolor{currentstroke}{rgb}{0.000000,0.000000,0.000000}%
\pgfsetstrokecolor{currentstroke}%
\pgfsetdash{}{0pt}%
\pgfusepath{stroke}%
\end{pgfscope}%
\begin{pgfscope}%
\pgfpathrectangle{\pgfqpoint{1.374500in}{0.082500in}}{\pgfqpoint{2.419000in}{2.419000in}}%
\pgfusepath{clip}%
\pgfsetbuttcap%
\pgfsetroundjoin%
\pgfsetlinewidth{1.505625pt}%
\definecolor{currentstroke}{rgb}{0.000000,0.000000,0.000000}%
\pgfsetstrokecolor{currentstroke}%
\pgfsetdash{}{0pt}%
\pgfusepath{stroke}%
\end{pgfscope}%
\begin{pgfscope}%
\pgfpathrectangle{\pgfqpoint{1.374500in}{0.082500in}}{\pgfqpoint{2.419000in}{2.419000in}}%
\pgfusepath{clip}%
\pgfsetbuttcap%
\pgfsetroundjoin%
\pgfsetlinewidth{1.505625pt}%
\definecolor{currentstroke}{rgb}{0.000000,0.000000,0.000000}%
\pgfsetstrokecolor{currentstroke}%
\pgfsetdash{}{0pt}%
\pgfusepath{stroke}%
\end{pgfscope}%
\begin{pgfscope}%
\pgfpathrectangle{\pgfqpoint{1.374500in}{0.082500in}}{\pgfqpoint{2.419000in}{2.419000in}}%
\pgfusepath{clip}%
\pgfsetbuttcap%
\pgfsetroundjoin%
\pgfsetlinewidth{1.505625pt}%
\definecolor{currentstroke}{rgb}{0.000000,0.000000,0.000000}%
\pgfsetstrokecolor{currentstroke}%
\pgfsetdash{}{0pt}%
\pgfusepath{stroke}%
\end{pgfscope}%
\begin{pgfscope}%
\pgfpathrectangle{\pgfqpoint{1.374500in}{0.082500in}}{\pgfqpoint{2.419000in}{2.419000in}}%
\pgfusepath{clip}%
\pgfsetbuttcap%
\pgfsetroundjoin%
\pgfsetlinewidth{1.505625pt}%
\definecolor{currentstroke}{rgb}{0.000000,0.000000,0.000000}%
\pgfsetstrokecolor{currentstroke}%
\pgfsetdash{}{0pt}%
\pgfusepath{stroke}%
\end{pgfscope}%
\begin{pgfscope}%
\pgfpathrectangle{\pgfqpoint{1.374500in}{0.082500in}}{\pgfqpoint{2.419000in}{2.419000in}}%
\pgfusepath{clip}%
\pgfsetbuttcap%
\pgfsetroundjoin%
\pgfsetlinewidth{1.505625pt}%
\definecolor{currentstroke}{rgb}{0.000000,0.000000,0.000000}%
\pgfsetstrokecolor{currentstroke}%
\pgfsetdash{}{0pt}%
\pgfusepath{stroke}%
\end{pgfscope}%
\begin{pgfscope}%
\pgfpathrectangle{\pgfqpoint{1.374500in}{0.082500in}}{\pgfqpoint{2.419000in}{2.419000in}}%
\pgfusepath{clip}%
\pgfsetbuttcap%
\pgfsetroundjoin%
\pgfsetlinewidth{1.505625pt}%
\definecolor{currentstroke}{rgb}{0.000000,0.000000,0.000000}%
\pgfsetstrokecolor{currentstroke}%
\pgfsetdash{}{0pt}%
\pgfusepath{stroke}%
\end{pgfscope}%
\begin{pgfscope}%
\pgfpathrectangle{\pgfqpoint{1.374500in}{0.082500in}}{\pgfqpoint{2.419000in}{2.419000in}}%
\pgfusepath{clip}%
\pgfsetbuttcap%
\pgfsetroundjoin%
\pgfsetlinewidth{1.505625pt}%
\definecolor{currentstroke}{rgb}{0.000000,0.000000,0.000000}%
\pgfsetstrokecolor{currentstroke}%
\pgfsetdash{}{0pt}%
\pgfusepath{stroke}%
\end{pgfscope}%
\begin{pgfscope}%
\pgfpathrectangle{\pgfqpoint{1.374500in}{0.082500in}}{\pgfqpoint{2.419000in}{2.419000in}}%
\pgfusepath{clip}%
\pgfsetbuttcap%
\pgfsetroundjoin%
\pgfsetlinewidth{1.505625pt}%
\definecolor{currentstroke}{rgb}{0.000000,0.000000,0.000000}%
\pgfsetstrokecolor{currentstroke}%
\pgfsetdash{}{0pt}%
\pgfusepath{stroke}%
\end{pgfscope}%
\begin{pgfscope}%
\pgfpathrectangle{\pgfqpoint{1.374500in}{0.082500in}}{\pgfqpoint{2.419000in}{2.419000in}}%
\pgfusepath{clip}%
\pgfsetbuttcap%
\pgfsetroundjoin%
\pgfsetlinewidth{1.505625pt}%
\definecolor{currentstroke}{rgb}{0.000000,0.000000,0.000000}%
\pgfsetstrokecolor{currentstroke}%
\pgfsetdash{}{0pt}%
\pgfusepath{stroke}%
\end{pgfscope}%
\begin{pgfscope}%
\pgfpathrectangle{\pgfqpoint{1.374500in}{0.082500in}}{\pgfqpoint{2.419000in}{2.419000in}}%
\pgfusepath{clip}%
\pgfsetbuttcap%
\pgfsetroundjoin%
\pgfsetlinewidth{1.505625pt}%
\definecolor{currentstroke}{rgb}{0.000000,0.000000,0.000000}%
\pgfsetstrokecolor{currentstroke}%
\pgfsetdash{}{0pt}%
\pgfusepath{stroke}%
\end{pgfscope}%
\begin{pgfscope}%
\pgfpathrectangle{\pgfqpoint{1.374500in}{0.082500in}}{\pgfqpoint{2.419000in}{2.419000in}}%
\pgfusepath{clip}%
\pgfsetbuttcap%
\pgfsetroundjoin%
\pgfsetlinewidth{1.505625pt}%
\definecolor{currentstroke}{rgb}{0.000000,0.000000,0.000000}%
\pgfsetstrokecolor{currentstroke}%
\pgfsetdash{}{0pt}%
\pgfusepath{stroke}%
\end{pgfscope}%
\begin{pgfscope}%
\pgfpathrectangle{\pgfqpoint{1.374500in}{0.082500in}}{\pgfqpoint{2.419000in}{2.419000in}}%
\pgfusepath{clip}%
\pgfsetbuttcap%
\pgfsetroundjoin%
\pgfsetlinewidth{1.505625pt}%
\definecolor{currentstroke}{rgb}{0.000000,0.000000,0.000000}%
\pgfsetstrokecolor{currentstroke}%
\pgfsetdash{}{0pt}%
\pgfusepath{stroke}%
\end{pgfscope}%
\begin{pgfscope}%
\pgfpathrectangle{\pgfqpoint{1.374500in}{0.082500in}}{\pgfqpoint{2.419000in}{2.419000in}}%
\pgfusepath{clip}%
\pgfsetbuttcap%
\pgfsetroundjoin%
\pgfsetlinewidth{1.505625pt}%
\definecolor{currentstroke}{rgb}{0.000000,0.000000,0.000000}%
\pgfsetstrokecolor{currentstroke}%
\pgfsetdash{}{0pt}%
\pgfusepath{stroke}%
\end{pgfscope}%
\begin{pgfscope}%
\pgfpathrectangle{\pgfqpoint{1.374500in}{0.082500in}}{\pgfqpoint{2.419000in}{2.419000in}}%
\pgfusepath{clip}%
\pgfsetbuttcap%
\pgfsetroundjoin%
\pgfsetlinewidth{1.505625pt}%
\definecolor{currentstroke}{rgb}{0.000000,0.000000,0.000000}%
\pgfsetstrokecolor{currentstroke}%
\pgfsetdash{}{0pt}%
\pgfusepath{stroke}%
\end{pgfscope}%
\begin{pgfscope}%
\pgfpathrectangle{\pgfqpoint{1.374500in}{0.082500in}}{\pgfqpoint{2.419000in}{2.419000in}}%
\pgfusepath{clip}%
\pgfsetbuttcap%
\pgfsetroundjoin%
\pgfsetlinewidth{1.505625pt}%
\definecolor{currentstroke}{rgb}{0.000000,0.000000,0.000000}%
\pgfsetstrokecolor{currentstroke}%
\pgfsetdash{}{0pt}%
\pgfusepath{stroke}%
\end{pgfscope}%
\begin{pgfscope}%
\pgfpathrectangle{\pgfqpoint{1.374500in}{0.082500in}}{\pgfqpoint{2.419000in}{2.419000in}}%
\pgfusepath{clip}%
\pgfsetbuttcap%
\pgfsetroundjoin%
\pgfsetlinewidth{1.505625pt}%
\definecolor{currentstroke}{rgb}{0.000000,0.000000,0.000000}%
\pgfsetstrokecolor{currentstroke}%
\pgfsetdash{}{0pt}%
\pgfusepath{stroke}%
\end{pgfscope}%
\begin{pgfscope}%
\pgfpathrectangle{\pgfqpoint{1.374500in}{0.082500in}}{\pgfqpoint{2.419000in}{2.419000in}}%
\pgfusepath{clip}%
\pgfsetbuttcap%
\pgfsetroundjoin%
\pgfsetlinewidth{1.505625pt}%
\definecolor{currentstroke}{rgb}{0.000000,0.000000,0.000000}%
\pgfsetstrokecolor{currentstroke}%
\pgfsetdash{}{0pt}%
\pgfusepath{stroke}%
\end{pgfscope}%
\begin{pgfscope}%
\pgfpathrectangle{\pgfqpoint{1.374500in}{0.082500in}}{\pgfqpoint{2.419000in}{2.419000in}}%
\pgfusepath{clip}%
\pgfsetbuttcap%
\pgfsetroundjoin%
\pgfsetlinewidth{1.505625pt}%
\definecolor{currentstroke}{rgb}{0.000000,0.000000,0.000000}%
\pgfsetstrokecolor{currentstroke}%
\pgfsetdash{}{0pt}%
\pgfusepath{stroke}%
\end{pgfscope}%
\begin{pgfscope}%
\pgfpathrectangle{\pgfqpoint{1.374500in}{0.082500in}}{\pgfqpoint{2.419000in}{2.419000in}}%
\pgfusepath{clip}%
\pgfsetbuttcap%
\pgfsetroundjoin%
\pgfsetlinewidth{1.505625pt}%
\definecolor{currentstroke}{rgb}{0.000000,0.000000,0.000000}%
\pgfsetstrokecolor{currentstroke}%
\pgfsetdash{}{0pt}%
\pgfusepath{stroke}%
\end{pgfscope}%
\begin{pgfscope}%
\pgfpathrectangle{\pgfqpoint{1.374500in}{0.082500in}}{\pgfqpoint{2.419000in}{2.419000in}}%
\pgfusepath{clip}%
\pgfsetbuttcap%
\pgfsetroundjoin%
\pgfsetlinewidth{1.505625pt}%
\definecolor{currentstroke}{rgb}{0.000000,0.000000,0.000000}%
\pgfsetstrokecolor{currentstroke}%
\pgfsetdash{}{0pt}%
\pgfusepath{stroke}%
\end{pgfscope}%
\begin{pgfscope}%
\pgfpathrectangle{\pgfqpoint{1.374500in}{0.082500in}}{\pgfqpoint{2.419000in}{2.419000in}}%
\pgfusepath{clip}%
\pgfsetbuttcap%
\pgfsetroundjoin%
\pgfsetlinewidth{1.505625pt}%
\definecolor{currentstroke}{rgb}{0.000000,0.000000,0.000000}%
\pgfsetstrokecolor{currentstroke}%
\pgfsetdash{}{0pt}%
\pgfusepath{stroke}%
\end{pgfscope}%
\begin{pgfscope}%
\pgfpathrectangle{\pgfqpoint{1.374500in}{0.082500in}}{\pgfqpoint{2.419000in}{2.419000in}}%
\pgfusepath{clip}%
\pgfsetbuttcap%
\pgfsetroundjoin%
\pgfsetlinewidth{1.505625pt}%
\definecolor{currentstroke}{rgb}{0.000000,0.000000,0.000000}%
\pgfsetstrokecolor{currentstroke}%
\pgfsetdash{}{0pt}%
\pgfusepath{stroke}%
\end{pgfscope}%
\begin{pgfscope}%
\pgfpathrectangle{\pgfqpoint{1.374500in}{0.082500in}}{\pgfqpoint{2.419000in}{2.419000in}}%
\pgfusepath{clip}%
\pgfsetbuttcap%
\pgfsetroundjoin%
\pgfsetlinewidth{1.505625pt}%
\definecolor{currentstroke}{rgb}{0.000000,0.000000,0.000000}%
\pgfsetstrokecolor{currentstroke}%
\pgfsetdash{}{0pt}%
\pgfusepath{stroke}%
\end{pgfscope}%
\begin{pgfscope}%
\pgfpathrectangle{\pgfqpoint{1.374500in}{0.082500in}}{\pgfqpoint{2.419000in}{2.419000in}}%
\pgfusepath{clip}%
\pgfsetbuttcap%
\pgfsetroundjoin%
\pgfsetlinewidth{1.505625pt}%
\definecolor{currentstroke}{rgb}{0.000000,0.000000,0.000000}%
\pgfsetstrokecolor{currentstroke}%
\pgfsetdash{}{0pt}%
\pgfusepath{stroke}%
\end{pgfscope}%
\begin{pgfscope}%
\pgfpathrectangle{\pgfqpoint{1.374500in}{0.082500in}}{\pgfqpoint{2.419000in}{2.419000in}}%
\pgfusepath{clip}%
\pgfsetbuttcap%
\pgfsetroundjoin%
\pgfsetlinewidth{1.505625pt}%
\definecolor{currentstroke}{rgb}{0.000000,0.000000,0.000000}%
\pgfsetstrokecolor{currentstroke}%
\pgfsetdash{}{0pt}%
\pgfusepath{stroke}%
\end{pgfscope}%
\begin{pgfscope}%
\pgfpathrectangle{\pgfqpoint{1.374500in}{0.082500in}}{\pgfqpoint{2.419000in}{2.419000in}}%
\pgfusepath{clip}%
\pgfsetbuttcap%
\pgfsetroundjoin%
\pgfsetlinewidth{1.505625pt}%
\definecolor{currentstroke}{rgb}{0.000000,0.000000,0.000000}%
\pgfsetstrokecolor{currentstroke}%
\pgfsetdash{}{0pt}%
\pgfusepath{stroke}%
\end{pgfscope}%
\begin{pgfscope}%
\pgfpathrectangle{\pgfqpoint{1.374500in}{0.082500in}}{\pgfqpoint{2.419000in}{2.419000in}}%
\pgfusepath{clip}%
\pgfsetbuttcap%
\pgfsetroundjoin%
\pgfsetlinewidth{1.505625pt}%
\definecolor{currentstroke}{rgb}{0.000000,0.000000,0.000000}%
\pgfsetstrokecolor{currentstroke}%
\pgfsetdash{}{0pt}%
\pgfusepath{stroke}%
\end{pgfscope}%
\begin{pgfscope}%
\pgfpathrectangle{\pgfqpoint{1.374500in}{0.082500in}}{\pgfqpoint{2.419000in}{2.419000in}}%
\pgfusepath{clip}%
\pgfsetbuttcap%
\pgfsetroundjoin%
\pgfsetlinewidth{1.505625pt}%
\definecolor{currentstroke}{rgb}{0.000000,0.000000,0.000000}%
\pgfsetstrokecolor{currentstroke}%
\pgfsetdash{}{0pt}%
\pgfusepath{stroke}%
\end{pgfscope}%
\begin{pgfscope}%
\pgfpathrectangle{\pgfqpoint{1.374500in}{0.082500in}}{\pgfqpoint{2.419000in}{2.419000in}}%
\pgfusepath{clip}%
\pgfsetbuttcap%
\pgfsetroundjoin%
\pgfsetlinewidth{1.505625pt}%
\definecolor{currentstroke}{rgb}{0.000000,0.000000,0.000000}%
\pgfsetstrokecolor{currentstroke}%
\pgfsetdash{}{0pt}%
\pgfusepath{stroke}%
\end{pgfscope}%
\begin{pgfscope}%
\pgfpathrectangle{\pgfqpoint{1.374500in}{0.082500in}}{\pgfqpoint{2.419000in}{2.419000in}}%
\pgfusepath{clip}%
\pgfsetbuttcap%
\pgfsetroundjoin%
\pgfsetlinewidth{1.505625pt}%
\definecolor{currentstroke}{rgb}{0.000000,0.000000,0.000000}%
\pgfsetstrokecolor{currentstroke}%
\pgfsetdash{}{0pt}%
\pgfusepath{stroke}%
\end{pgfscope}%
\begin{pgfscope}%
\pgfpathrectangle{\pgfqpoint{1.374500in}{0.082500in}}{\pgfqpoint{2.419000in}{2.419000in}}%
\pgfusepath{clip}%
\pgfsetbuttcap%
\pgfsetroundjoin%
\pgfsetlinewidth{1.505625pt}%
\definecolor{currentstroke}{rgb}{0.000000,0.000000,0.000000}%
\pgfsetstrokecolor{currentstroke}%
\pgfsetdash{}{0pt}%
\pgfusepath{stroke}%
\end{pgfscope}%
\begin{pgfscope}%
\pgfpathrectangle{\pgfqpoint{1.374500in}{0.082500in}}{\pgfqpoint{2.419000in}{2.419000in}}%
\pgfusepath{clip}%
\pgfsetbuttcap%
\pgfsetroundjoin%
\pgfsetlinewidth{1.505625pt}%
\definecolor{currentstroke}{rgb}{0.000000,0.000000,0.000000}%
\pgfsetstrokecolor{currentstroke}%
\pgfsetdash{}{0pt}%
\pgfusepath{stroke}%
\end{pgfscope}%
\begin{pgfscope}%
\pgfpathrectangle{\pgfqpoint{1.374500in}{0.082500in}}{\pgfqpoint{2.419000in}{2.419000in}}%
\pgfusepath{clip}%
\pgfsetbuttcap%
\pgfsetroundjoin%
\pgfsetlinewidth{1.505625pt}%
\definecolor{currentstroke}{rgb}{0.000000,0.000000,0.000000}%
\pgfsetstrokecolor{currentstroke}%
\pgfsetdash{}{0pt}%
\pgfusepath{stroke}%
\end{pgfscope}%
\begin{pgfscope}%
\pgfpathrectangle{\pgfqpoint{1.374500in}{0.082500in}}{\pgfqpoint{2.419000in}{2.419000in}}%
\pgfusepath{clip}%
\pgfsetbuttcap%
\pgfsetroundjoin%
\pgfsetlinewidth{1.505625pt}%
\definecolor{currentstroke}{rgb}{0.000000,0.000000,0.000000}%
\pgfsetstrokecolor{currentstroke}%
\pgfsetdash{}{0pt}%
\pgfusepath{stroke}%
\end{pgfscope}%
\begin{pgfscope}%
\pgfpathrectangle{\pgfqpoint{1.374500in}{0.082500in}}{\pgfqpoint{2.419000in}{2.419000in}}%
\pgfusepath{clip}%
\pgfsetbuttcap%
\pgfsetroundjoin%
\pgfsetlinewidth{1.505625pt}%
\definecolor{currentstroke}{rgb}{0.000000,0.000000,0.000000}%
\pgfsetstrokecolor{currentstroke}%
\pgfsetdash{}{0pt}%
\pgfusepath{stroke}%
\end{pgfscope}%
\begin{pgfscope}%
\pgfpathrectangle{\pgfqpoint{1.374500in}{0.082500in}}{\pgfqpoint{2.419000in}{2.419000in}}%
\pgfusepath{clip}%
\pgfsetbuttcap%
\pgfsetroundjoin%
\pgfsetlinewidth{1.505625pt}%
\definecolor{currentstroke}{rgb}{0.000000,0.000000,0.000000}%
\pgfsetstrokecolor{currentstroke}%
\pgfsetdash{}{0pt}%
\pgfusepath{stroke}%
\end{pgfscope}%
\begin{pgfscope}%
\pgfpathrectangle{\pgfqpoint{1.374500in}{0.082500in}}{\pgfqpoint{2.419000in}{2.419000in}}%
\pgfusepath{clip}%
\pgfsetbuttcap%
\pgfsetroundjoin%
\pgfsetlinewidth{1.505625pt}%
\definecolor{currentstroke}{rgb}{0.000000,0.000000,0.000000}%
\pgfsetstrokecolor{currentstroke}%
\pgfsetdash{}{0pt}%
\pgfusepath{stroke}%
\end{pgfscope}%
\begin{pgfscope}%
\pgfpathrectangle{\pgfqpoint{1.374500in}{0.082500in}}{\pgfqpoint{2.419000in}{2.419000in}}%
\pgfusepath{clip}%
\pgfsetbuttcap%
\pgfsetroundjoin%
\pgfsetlinewidth{1.505625pt}%
\definecolor{currentstroke}{rgb}{0.000000,0.000000,0.000000}%
\pgfsetstrokecolor{currentstroke}%
\pgfsetdash{}{0pt}%
\pgfusepath{stroke}%
\end{pgfscope}%
\begin{pgfscope}%
\pgfpathrectangle{\pgfqpoint{1.374500in}{0.082500in}}{\pgfqpoint{2.419000in}{2.419000in}}%
\pgfusepath{clip}%
\pgfsetbuttcap%
\pgfsetroundjoin%
\pgfsetlinewidth{1.505625pt}%
\definecolor{currentstroke}{rgb}{0.000000,0.000000,0.000000}%
\pgfsetstrokecolor{currentstroke}%
\pgfsetdash{}{0pt}%
\pgfusepath{stroke}%
\end{pgfscope}%
\begin{pgfscope}%
\pgfpathrectangle{\pgfqpoint{1.374500in}{0.082500in}}{\pgfqpoint{2.419000in}{2.419000in}}%
\pgfusepath{clip}%
\pgfsetbuttcap%
\pgfsetroundjoin%
\pgfsetlinewidth{1.505625pt}%
\definecolor{currentstroke}{rgb}{0.000000,0.000000,0.000000}%
\pgfsetstrokecolor{currentstroke}%
\pgfsetdash{}{0pt}%
\pgfusepath{stroke}%
\end{pgfscope}%
\begin{pgfscope}%
\pgfpathrectangle{\pgfqpoint{1.374500in}{0.082500in}}{\pgfqpoint{2.419000in}{2.419000in}}%
\pgfusepath{clip}%
\pgfsetbuttcap%
\pgfsetroundjoin%
\pgfsetlinewidth{1.505625pt}%
\definecolor{currentstroke}{rgb}{0.000000,0.000000,0.000000}%
\pgfsetstrokecolor{currentstroke}%
\pgfsetdash{}{0pt}%
\pgfusepath{stroke}%
\end{pgfscope}%
\begin{pgfscope}%
\pgfpathrectangle{\pgfqpoint{1.374500in}{0.082500in}}{\pgfqpoint{2.419000in}{2.419000in}}%
\pgfusepath{clip}%
\pgfsetbuttcap%
\pgfsetroundjoin%
\pgfsetlinewidth{1.505625pt}%
\definecolor{currentstroke}{rgb}{0.000000,0.000000,0.000000}%
\pgfsetstrokecolor{currentstroke}%
\pgfsetdash{}{0pt}%
\pgfusepath{stroke}%
\end{pgfscope}%
\begin{pgfscope}%
\pgfpathrectangle{\pgfqpoint{1.374500in}{0.082500in}}{\pgfqpoint{2.419000in}{2.419000in}}%
\pgfusepath{clip}%
\pgfsetbuttcap%
\pgfsetroundjoin%
\pgfsetlinewidth{1.505625pt}%
\definecolor{currentstroke}{rgb}{0.000000,0.000000,0.000000}%
\pgfsetstrokecolor{currentstroke}%
\pgfsetdash{}{0pt}%
\pgfusepath{stroke}%
\end{pgfscope}%
\begin{pgfscope}%
\pgfpathrectangle{\pgfqpoint{1.374500in}{0.082500in}}{\pgfqpoint{2.419000in}{2.419000in}}%
\pgfusepath{clip}%
\pgfsetbuttcap%
\pgfsetroundjoin%
\pgfsetlinewidth{1.505625pt}%
\definecolor{currentstroke}{rgb}{0.000000,0.000000,0.000000}%
\pgfsetstrokecolor{currentstroke}%
\pgfsetdash{}{0pt}%
\pgfusepath{stroke}%
\end{pgfscope}%
\begin{pgfscope}%
\pgfpathrectangle{\pgfqpoint{1.374500in}{0.082500in}}{\pgfqpoint{2.419000in}{2.419000in}}%
\pgfusepath{clip}%
\pgfsetbuttcap%
\pgfsetroundjoin%
\pgfsetlinewidth{1.505625pt}%
\definecolor{currentstroke}{rgb}{0.000000,0.000000,0.000000}%
\pgfsetstrokecolor{currentstroke}%
\pgfsetdash{}{0pt}%
\pgfusepath{stroke}%
\end{pgfscope}%
\begin{pgfscope}%
\pgfpathrectangle{\pgfqpoint{1.374500in}{0.082500in}}{\pgfqpoint{2.419000in}{2.419000in}}%
\pgfusepath{clip}%
\pgfsetbuttcap%
\pgfsetroundjoin%
\pgfsetlinewidth{1.505625pt}%
\definecolor{currentstroke}{rgb}{0.000000,0.000000,0.000000}%
\pgfsetstrokecolor{currentstroke}%
\pgfsetdash{}{0pt}%
\pgfusepath{stroke}%
\end{pgfscope}%
\begin{pgfscope}%
\pgfpathrectangle{\pgfqpoint{1.374500in}{0.082500in}}{\pgfqpoint{2.419000in}{2.419000in}}%
\pgfusepath{clip}%
\pgfsetbuttcap%
\pgfsetroundjoin%
\pgfsetlinewidth{1.505625pt}%
\definecolor{currentstroke}{rgb}{0.000000,0.000000,0.000000}%
\pgfsetstrokecolor{currentstroke}%
\pgfsetdash{}{0pt}%
\pgfusepath{stroke}%
\end{pgfscope}%
\begin{pgfscope}%
\pgfpathrectangle{\pgfqpoint{1.374500in}{0.082500in}}{\pgfqpoint{2.419000in}{2.419000in}}%
\pgfusepath{clip}%
\pgfsetbuttcap%
\pgfsetroundjoin%
\pgfsetlinewidth{1.505625pt}%
\definecolor{currentstroke}{rgb}{0.000000,0.000000,0.000000}%
\pgfsetstrokecolor{currentstroke}%
\pgfsetdash{}{0pt}%
\pgfusepath{stroke}%
\end{pgfscope}%
\begin{pgfscope}%
\pgfpathrectangle{\pgfqpoint{1.374500in}{0.082500in}}{\pgfqpoint{2.419000in}{2.419000in}}%
\pgfusepath{clip}%
\pgfsetbuttcap%
\pgfsetroundjoin%
\pgfsetlinewidth{1.505625pt}%
\definecolor{currentstroke}{rgb}{0.000000,0.000000,0.000000}%
\pgfsetstrokecolor{currentstroke}%
\pgfsetdash{}{0pt}%
\pgfusepath{stroke}%
\end{pgfscope}%
\begin{pgfscope}%
\pgfpathrectangle{\pgfqpoint{1.374500in}{0.082500in}}{\pgfqpoint{2.419000in}{2.419000in}}%
\pgfusepath{clip}%
\pgfsetbuttcap%
\pgfsetroundjoin%
\pgfsetlinewidth{1.505625pt}%
\definecolor{currentstroke}{rgb}{0.000000,0.000000,0.000000}%
\pgfsetstrokecolor{currentstroke}%
\pgfsetdash{}{0pt}%
\pgfusepath{stroke}%
\end{pgfscope}%
\begin{pgfscope}%
\pgfpathrectangle{\pgfqpoint{1.374500in}{0.082500in}}{\pgfqpoint{2.419000in}{2.419000in}}%
\pgfusepath{clip}%
\pgfsetbuttcap%
\pgfsetroundjoin%
\pgfsetlinewidth{1.505625pt}%
\definecolor{currentstroke}{rgb}{0.000000,0.000000,0.000000}%
\pgfsetstrokecolor{currentstroke}%
\pgfsetdash{}{0pt}%
\pgfusepath{stroke}%
\end{pgfscope}%
\begin{pgfscope}%
\pgfpathrectangle{\pgfqpoint{1.374500in}{0.082500in}}{\pgfqpoint{2.419000in}{2.419000in}}%
\pgfusepath{clip}%
\pgfsetbuttcap%
\pgfsetroundjoin%
\pgfsetlinewidth{1.505625pt}%
\definecolor{currentstroke}{rgb}{0.000000,0.000000,0.000000}%
\pgfsetstrokecolor{currentstroke}%
\pgfsetdash{}{0pt}%
\pgfusepath{stroke}%
\end{pgfscope}%
\begin{pgfscope}%
\pgfpathrectangle{\pgfqpoint{1.374500in}{0.082500in}}{\pgfqpoint{2.419000in}{2.419000in}}%
\pgfusepath{clip}%
\pgfsetbuttcap%
\pgfsetroundjoin%
\pgfsetlinewidth{1.505625pt}%
\definecolor{currentstroke}{rgb}{0.000000,0.000000,0.000000}%
\pgfsetstrokecolor{currentstroke}%
\pgfsetdash{}{0pt}%
\pgfusepath{stroke}%
\end{pgfscope}%
\begin{pgfscope}%
\pgfpathrectangle{\pgfqpoint{1.374500in}{0.082500in}}{\pgfqpoint{2.419000in}{2.419000in}}%
\pgfusepath{clip}%
\pgfsetbuttcap%
\pgfsetroundjoin%
\pgfsetlinewidth{1.505625pt}%
\definecolor{currentstroke}{rgb}{0.000000,0.000000,0.000000}%
\pgfsetstrokecolor{currentstroke}%
\pgfsetdash{}{0pt}%
\pgfusepath{stroke}%
\end{pgfscope}%
\begin{pgfscope}%
\pgfpathrectangle{\pgfqpoint{1.374500in}{0.082500in}}{\pgfqpoint{2.419000in}{2.419000in}}%
\pgfusepath{clip}%
\pgfsetbuttcap%
\pgfsetroundjoin%
\pgfsetlinewidth{1.505625pt}%
\definecolor{currentstroke}{rgb}{0.000000,0.000000,0.000000}%
\pgfsetstrokecolor{currentstroke}%
\pgfsetdash{}{0pt}%
\pgfusepath{stroke}%
\end{pgfscope}%
\begin{pgfscope}%
\pgfpathrectangle{\pgfqpoint{1.374500in}{0.082500in}}{\pgfqpoint{2.419000in}{2.419000in}}%
\pgfusepath{clip}%
\pgfsetbuttcap%
\pgfsetroundjoin%
\pgfsetlinewidth{1.505625pt}%
\definecolor{currentstroke}{rgb}{0.000000,0.000000,0.000000}%
\pgfsetstrokecolor{currentstroke}%
\pgfsetdash{}{0pt}%
\pgfusepath{stroke}%
\end{pgfscope}%
\begin{pgfscope}%
\pgfpathrectangle{\pgfqpoint{1.374500in}{0.082500in}}{\pgfqpoint{2.419000in}{2.419000in}}%
\pgfusepath{clip}%
\pgfsetbuttcap%
\pgfsetroundjoin%
\pgfsetlinewidth{1.505625pt}%
\definecolor{currentstroke}{rgb}{0.000000,0.000000,0.000000}%
\pgfsetstrokecolor{currentstroke}%
\pgfsetdash{}{0pt}%
\pgfusepath{stroke}%
\end{pgfscope}%
\begin{pgfscope}%
\pgfpathrectangle{\pgfqpoint{1.374500in}{0.082500in}}{\pgfqpoint{2.419000in}{2.419000in}}%
\pgfusepath{clip}%
\pgfsetbuttcap%
\pgfsetroundjoin%
\pgfsetlinewidth{1.505625pt}%
\definecolor{currentstroke}{rgb}{0.000000,0.000000,0.000000}%
\pgfsetstrokecolor{currentstroke}%
\pgfsetdash{}{0pt}%
\pgfusepath{stroke}%
\end{pgfscope}%
\begin{pgfscope}%
\pgfpathrectangle{\pgfqpoint{1.374500in}{0.082500in}}{\pgfqpoint{2.419000in}{2.419000in}}%
\pgfusepath{clip}%
\pgfsetbuttcap%
\pgfsetroundjoin%
\pgfsetlinewidth{1.505625pt}%
\definecolor{currentstroke}{rgb}{0.000000,0.000000,0.000000}%
\pgfsetstrokecolor{currentstroke}%
\pgfsetdash{}{0pt}%
\pgfusepath{stroke}%
\end{pgfscope}%
\begin{pgfscope}%
\pgfpathrectangle{\pgfqpoint{1.374500in}{0.082500in}}{\pgfqpoint{2.419000in}{2.419000in}}%
\pgfusepath{clip}%
\pgfsetbuttcap%
\pgfsetroundjoin%
\pgfsetlinewidth{1.505625pt}%
\definecolor{currentstroke}{rgb}{0.000000,0.000000,0.000000}%
\pgfsetstrokecolor{currentstroke}%
\pgfsetdash{}{0pt}%
\pgfusepath{stroke}%
\end{pgfscope}%
\begin{pgfscope}%
\pgfpathrectangle{\pgfqpoint{1.374500in}{0.082500in}}{\pgfqpoint{2.419000in}{2.419000in}}%
\pgfusepath{clip}%
\pgfsetbuttcap%
\pgfsetroundjoin%
\pgfsetlinewidth{1.505625pt}%
\definecolor{currentstroke}{rgb}{0.000000,0.000000,0.000000}%
\pgfsetstrokecolor{currentstroke}%
\pgfsetdash{}{0pt}%
\pgfusepath{stroke}%
\end{pgfscope}%
\begin{pgfscope}%
\pgfpathrectangle{\pgfqpoint{1.374500in}{0.082500in}}{\pgfqpoint{2.419000in}{2.419000in}}%
\pgfusepath{clip}%
\pgfsetbuttcap%
\pgfsetroundjoin%
\pgfsetlinewidth{1.505625pt}%
\definecolor{currentstroke}{rgb}{0.000000,0.000000,0.000000}%
\pgfsetstrokecolor{currentstroke}%
\pgfsetdash{}{0pt}%
\pgfusepath{stroke}%
\end{pgfscope}%
\begin{pgfscope}%
\pgfpathrectangle{\pgfqpoint{1.374500in}{0.082500in}}{\pgfqpoint{2.419000in}{2.419000in}}%
\pgfusepath{clip}%
\pgfsetbuttcap%
\pgfsetroundjoin%
\pgfsetlinewidth{1.505625pt}%
\definecolor{currentstroke}{rgb}{0.000000,0.000000,0.000000}%
\pgfsetstrokecolor{currentstroke}%
\pgfsetdash{}{0pt}%
\pgfusepath{stroke}%
\end{pgfscope}%
\begin{pgfscope}%
\pgfpathrectangle{\pgfqpoint{1.374500in}{0.082500in}}{\pgfqpoint{2.419000in}{2.419000in}}%
\pgfusepath{clip}%
\pgfsetbuttcap%
\pgfsetroundjoin%
\pgfsetlinewidth{1.505625pt}%
\definecolor{currentstroke}{rgb}{0.000000,0.000000,0.000000}%
\pgfsetstrokecolor{currentstroke}%
\pgfsetdash{}{0pt}%
\pgfusepath{stroke}%
\end{pgfscope}%
\begin{pgfscope}%
\pgfpathrectangle{\pgfqpoint{1.374500in}{0.082500in}}{\pgfqpoint{2.419000in}{2.419000in}}%
\pgfusepath{clip}%
\pgfsetbuttcap%
\pgfsetroundjoin%
\pgfsetlinewidth{1.505625pt}%
\definecolor{currentstroke}{rgb}{0.000000,0.000000,0.000000}%
\pgfsetstrokecolor{currentstroke}%
\pgfsetdash{}{0pt}%
\pgfusepath{stroke}%
\end{pgfscope}%
\begin{pgfscope}%
\pgfpathrectangle{\pgfqpoint{1.374500in}{0.082500in}}{\pgfqpoint{2.419000in}{2.419000in}}%
\pgfusepath{clip}%
\pgfsetbuttcap%
\pgfsetroundjoin%
\pgfsetlinewidth{1.505625pt}%
\definecolor{currentstroke}{rgb}{0.000000,0.000000,0.000000}%
\pgfsetstrokecolor{currentstroke}%
\pgfsetdash{}{0pt}%
\pgfusepath{stroke}%
\end{pgfscope}%
\begin{pgfscope}%
\pgfpathrectangle{\pgfqpoint{1.374500in}{0.082500in}}{\pgfqpoint{2.419000in}{2.419000in}}%
\pgfusepath{clip}%
\pgfsetbuttcap%
\pgfsetroundjoin%
\pgfsetlinewidth{1.505625pt}%
\definecolor{currentstroke}{rgb}{0.000000,0.000000,0.000000}%
\pgfsetstrokecolor{currentstroke}%
\pgfsetdash{}{0pt}%
\pgfusepath{stroke}%
\end{pgfscope}%
\begin{pgfscope}%
\pgfpathrectangle{\pgfqpoint{1.374500in}{0.082500in}}{\pgfqpoint{2.419000in}{2.419000in}}%
\pgfusepath{clip}%
\pgfsetbuttcap%
\pgfsetroundjoin%
\pgfsetlinewidth{1.505625pt}%
\definecolor{currentstroke}{rgb}{0.000000,0.000000,0.000000}%
\pgfsetstrokecolor{currentstroke}%
\pgfsetdash{}{0pt}%
\pgfusepath{stroke}%
\end{pgfscope}%
\begin{pgfscope}%
\pgfpathrectangle{\pgfqpoint{1.374500in}{0.082500in}}{\pgfqpoint{2.419000in}{2.419000in}}%
\pgfusepath{clip}%
\pgfsetbuttcap%
\pgfsetroundjoin%
\pgfsetlinewidth{1.505625pt}%
\definecolor{currentstroke}{rgb}{0.000000,0.000000,0.000000}%
\pgfsetstrokecolor{currentstroke}%
\pgfsetdash{}{0pt}%
\pgfusepath{stroke}%
\end{pgfscope}%
\begin{pgfscope}%
\pgfpathrectangle{\pgfqpoint{1.374500in}{0.082500in}}{\pgfqpoint{2.419000in}{2.419000in}}%
\pgfusepath{clip}%
\pgfsetbuttcap%
\pgfsetroundjoin%
\pgfsetlinewidth{1.505625pt}%
\definecolor{currentstroke}{rgb}{0.000000,0.000000,0.000000}%
\pgfsetstrokecolor{currentstroke}%
\pgfsetdash{}{0pt}%
\pgfusepath{stroke}%
\end{pgfscope}%
\begin{pgfscope}%
\pgfpathrectangle{\pgfqpoint{1.374500in}{0.082500in}}{\pgfqpoint{2.419000in}{2.419000in}}%
\pgfusepath{clip}%
\pgfsetbuttcap%
\pgfsetroundjoin%
\pgfsetlinewidth{1.505625pt}%
\definecolor{currentstroke}{rgb}{0.000000,0.000000,0.000000}%
\pgfsetstrokecolor{currentstroke}%
\pgfsetdash{}{0pt}%
\pgfusepath{stroke}%
\end{pgfscope}%
\begin{pgfscope}%
\pgfpathrectangle{\pgfqpoint{1.374500in}{0.082500in}}{\pgfqpoint{2.419000in}{2.419000in}}%
\pgfusepath{clip}%
\pgfsetbuttcap%
\pgfsetroundjoin%
\pgfsetlinewidth{1.505625pt}%
\definecolor{currentstroke}{rgb}{0.000000,0.000000,0.000000}%
\pgfsetstrokecolor{currentstroke}%
\pgfsetdash{}{0pt}%
\pgfusepath{stroke}%
\end{pgfscope}%
\begin{pgfscope}%
\pgfpathrectangle{\pgfqpoint{1.374500in}{0.082500in}}{\pgfqpoint{2.419000in}{2.419000in}}%
\pgfusepath{clip}%
\pgfsetbuttcap%
\pgfsetroundjoin%
\pgfsetlinewidth{1.505625pt}%
\definecolor{currentstroke}{rgb}{0.000000,0.000000,0.000000}%
\pgfsetstrokecolor{currentstroke}%
\pgfsetdash{}{0pt}%
\pgfusepath{stroke}%
\end{pgfscope}%
\begin{pgfscope}%
\pgfpathrectangle{\pgfqpoint{1.374500in}{0.082500in}}{\pgfqpoint{2.419000in}{2.419000in}}%
\pgfusepath{clip}%
\pgfsetbuttcap%
\pgfsetroundjoin%
\pgfsetlinewidth{1.505625pt}%
\definecolor{currentstroke}{rgb}{0.000000,0.000000,0.000000}%
\pgfsetstrokecolor{currentstroke}%
\pgfsetdash{}{0pt}%
\pgfusepath{stroke}%
\end{pgfscope}%
\begin{pgfscope}%
\pgfpathrectangle{\pgfqpoint{1.374500in}{0.082500in}}{\pgfqpoint{2.419000in}{2.419000in}}%
\pgfusepath{clip}%
\pgfsetbuttcap%
\pgfsetroundjoin%
\pgfsetlinewidth{1.505625pt}%
\definecolor{currentstroke}{rgb}{0.000000,0.000000,0.000000}%
\pgfsetstrokecolor{currentstroke}%
\pgfsetdash{}{0pt}%
\pgfusepath{stroke}%
\end{pgfscope}%
\begin{pgfscope}%
\pgfpathrectangle{\pgfqpoint{1.374500in}{0.082500in}}{\pgfqpoint{2.419000in}{2.419000in}}%
\pgfusepath{clip}%
\pgfsetbuttcap%
\pgfsetroundjoin%
\pgfsetlinewidth{1.505625pt}%
\definecolor{currentstroke}{rgb}{0.000000,0.000000,0.000000}%
\pgfsetstrokecolor{currentstroke}%
\pgfsetdash{}{0pt}%
\pgfusepath{stroke}%
\end{pgfscope}%
\begin{pgfscope}%
\pgfpathrectangle{\pgfqpoint{1.374500in}{0.082500in}}{\pgfqpoint{2.419000in}{2.419000in}}%
\pgfusepath{clip}%
\pgfsetbuttcap%
\pgfsetroundjoin%
\pgfsetlinewidth{1.505625pt}%
\definecolor{currentstroke}{rgb}{0.000000,0.000000,0.000000}%
\pgfsetstrokecolor{currentstroke}%
\pgfsetdash{}{0pt}%
\pgfusepath{stroke}%
\end{pgfscope}%
\begin{pgfscope}%
\pgfpathrectangle{\pgfqpoint{1.374500in}{0.082500in}}{\pgfqpoint{2.419000in}{2.419000in}}%
\pgfusepath{clip}%
\pgfsetbuttcap%
\pgfsetroundjoin%
\pgfsetlinewidth{1.505625pt}%
\definecolor{currentstroke}{rgb}{0.000000,0.000000,0.000000}%
\pgfsetstrokecolor{currentstroke}%
\pgfsetdash{}{0pt}%
\pgfusepath{stroke}%
\end{pgfscope}%
\begin{pgfscope}%
\pgfpathrectangle{\pgfqpoint{1.374500in}{0.082500in}}{\pgfqpoint{2.419000in}{2.419000in}}%
\pgfusepath{clip}%
\pgfsetbuttcap%
\pgfsetroundjoin%
\pgfsetlinewidth{1.505625pt}%
\definecolor{currentstroke}{rgb}{0.000000,0.000000,0.000000}%
\pgfsetstrokecolor{currentstroke}%
\pgfsetdash{}{0pt}%
\pgfusepath{stroke}%
\end{pgfscope}%
\begin{pgfscope}%
\pgfpathrectangle{\pgfqpoint{1.374500in}{0.082500in}}{\pgfqpoint{2.419000in}{2.419000in}}%
\pgfusepath{clip}%
\pgfsetbuttcap%
\pgfsetroundjoin%
\pgfsetlinewidth{1.505625pt}%
\definecolor{currentstroke}{rgb}{0.000000,0.000000,0.000000}%
\pgfsetstrokecolor{currentstroke}%
\pgfsetdash{}{0pt}%
\pgfusepath{stroke}%
\end{pgfscope}%
\begin{pgfscope}%
\pgfpathrectangle{\pgfqpoint{1.374500in}{0.082500in}}{\pgfqpoint{2.419000in}{2.419000in}}%
\pgfusepath{clip}%
\pgfsetbuttcap%
\pgfsetroundjoin%
\pgfsetlinewidth{1.505625pt}%
\definecolor{currentstroke}{rgb}{0.000000,0.000000,0.000000}%
\pgfsetstrokecolor{currentstroke}%
\pgfsetdash{}{0pt}%
\pgfusepath{stroke}%
\end{pgfscope}%
\begin{pgfscope}%
\pgfpathrectangle{\pgfqpoint{1.374500in}{0.082500in}}{\pgfqpoint{2.419000in}{2.419000in}}%
\pgfusepath{clip}%
\pgfsetbuttcap%
\pgfsetroundjoin%
\pgfsetlinewidth{1.505625pt}%
\definecolor{currentstroke}{rgb}{0.000000,0.000000,0.000000}%
\pgfsetstrokecolor{currentstroke}%
\pgfsetdash{}{0pt}%
\pgfusepath{stroke}%
\end{pgfscope}%
\begin{pgfscope}%
\pgfpathrectangle{\pgfqpoint{1.374500in}{0.082500in}}{\pgfqpoint{2.419000in}{2.419000in}}%
\pgfusepath{clip}%
\pgfsetbuttcap%
\pgfsetroundjoin%
\pgfsetlinewidth{1.505625pt}%
\definecolor{currentstroke}{rgb}{0.000000,0.000000,0.000000}%
\pgfsetstrokecolor{currentstroke}%
\pgfsetdash{}{0pt}%
\pgfpathmoveto{\pgfqpoint{3.353720in}{2.511500in}}%
\pgfpathlineto{\pgfqpoint{3.417665in}{2.453075in}}%
\pgfusepath{stroke}%
\end{pgfscope}%
\begin{pgfscope}%
\pgfpathrectangle{\pgfqpoint{1.374500in}{0.082500in}}{\pgfqpoint{2.419000in}{2.419000in}}%
\pgfusepath{clip}%
\pgfsetbuttcap%
\pgfsetroundjoin%
\pgfsetlinewidth{1.505625pt}%
\definecolor{currentstroke}{rgb}{0.000000,0.000000,0.000000}%
\pgfsetstrokecolor{currentstroke}%
\pgfsetdash{}{0pt}%
\pgfpathmoveto{\pgfqpoint{3.417665in}{2.453075in}}%
\pgfpathlineto{\pgfqpoint{3.803500in}{2.475093in}}%
\pgfusepath{stroke}%
\end{pgfscope}%
\begin{pgfscope}%
\pgfpathrectangle{\pgfqpoint{1.374500in}{0.082500in}}{\pgfqpoint{2.419000in}{2.419000in}}%
\pgfusepath{clip}%
\pgfsetbuttcap%
\pgfsetroundjoin%
\pgfsetlinewidth{1.505625pt}%
\definecolor{currentstroke}{rgb}{0.000000,0.000000,0.000000}%
\pgfsetstrokecolor{currentstroke}%
\pgfsetdash{}{0pt}%
\pgfusepath{stroke}%
\end{pgfscope}%
\begin{pgfscope}%
\pgfpathrectangle{\pgfqpoint{1.374500in}{0.082500in}}{\pgfqpoint{2.419000in}{2.419000in}}%
\pgfusepath{clip}%
\pgfsetbuttcap%
\pgfsetroundjoin%
\pgfsetlinewidth{1.505625pt}%
\definecolor{currentstroke}{rgb}{0.000000,0.000000,0.000000}%
\pgfsetstrokecolor{currentstroke}%
\pgfsetdash{}{0pt}%
\pgfusepath{stroke}%
\end{pgfscope}%
\begin{pgfscope}%
\pgfpathrectangle{\pgfqpoint{1.374500in}{0.082500in}}{\pgfqpoint{2.419000in}{2.419000in}}%
\pgfusepath{clip}%
\pgfsetbuttcap%
\pgfsetroundjoin%
\pgfsetlinewidth{1.505625pt}%
\definecolor{currentstroke}{rgb}{0.000000,0.000000,0.000000}%
\pgfsetstrokecolor{currentstroke}%
\pgfsetdash{}{0pt}%
\pgfusepath{stroke}%
\end{pgfscope}%
\begin{pgfscope}%
\pgfpathrectangle{\pgfqpoint{1.374500in}{0.082500in}}{\pgfqpoint{2.419000in}{2.419000in}}%
\pgfusepath{clip}%
\pgfsetbuttcap%
\pgfsetroundjoin%
\pgfsetlinewidth{1.505625pt}%
\definecolor{currentstroke}{rgb}{0.000000,0.000000,0.000000}%
\pgfsetstrokecolor{currentstroke}%
\pgfsetdash{}{0pt}%
\pgfusepath{stroke}%
\end{pgfscope}%
\begin{pgfscope}%
\pgfpathrectangle{\pgfqpoint{1.374500in}{0.082500in}}{\pgfqpoint{2.419000in}{2.419000in}}%
\pgfusepath{clip}%
\pgfsetbuttcap%
\pgfsetroundjoin%
\pgfsetlinewidth{1.505625pt}%
\definecolor{currentstroke}{rgb}{0.000000,0.000000,0.000000}%
\pgfsetstrokecolor{currentstroke}%
\pgfsetdash{}{0pt}%
\pgfusepath{stroke}%
\end{pgfscope}%
\begin{pgfscope}%
\pgfpathrectangle{\pgfqpoint{1.374500in}{0.082500in}}{\pgfqpoint{2.419000in}{2.419000in}}%
\pgfusepath{clip}%
\pgfsetbuttcap%
\pgfsetroundjoin%
\pgfsetlinewidth{1.505625pt}%
\definecolor{currentstroke}{rgb}{0.000000,0.000000,0.000000}%
\pgfsetstrokecolor{currentstroke}%
\pgfsetdash{}{0pt}%
\pgfusepath{stroke}%
\end{pgfscope}%
\begin{pgfscope}%
\pgfpathrectangle{\pgfqpoint{1.374500in}{0.082500in}}{\pgfqpoint{2.419000in}{2.419000in}}%
\pgfusepath{clip}%
\pgfsetbuttcap%
\pgfsetroundjoin%
\pgfsetlinewidth{1.505625pt}%
\definecolor{currentstroke}{rgb}{0.000000,0.000000,0.000000}%
\pgfsetstrokecolor{currentstroke}%
\pgfsetdash{}{0pt}%
\pgfpathmoveto{\pgfqpoint{2.317498in}{2.390296in}}%
\pgfpathlineto{\pgfqpoint{2.486047in}{2.511500in}}%
\pgfusepath{stroke}%
\end{pgfscope}%
\begin{pgfscope}%
\pgfpathrectangle{\pgfqpoint{1.374500in}{0.082500in}}{\pgfqpoint{2.419000in}{2.419000in}}%
\pgfusepath{clip}%
\pgfsetbuttcap%
\pgfsetroundjoin%
\pgfsetlinewidth{1.505625pt}%
\definecolor{currentstroke}{rgb}{0.000000,0.000000,0.000000}%
\pgfsetstrokecolor{currentstroke}%
\pgfsetdash{}{0pt}%
\pgfusepath{stroke}%
\end{pgfscope}%
\begin{pgfscope}%
\pgfpathrectangle{\pgfqpoint{1.374500in}{0.082500in}}{\pgfqpoint{2.419000in}{2.419000in}}%
\pgfusepath{clip}%
\pgfsetbuttcap%
\pgfsetroundjoin%
\pgfsetlinewidth{1.505625pt}%
\definecolor{currentstroke}{rgb}{0.000000,0.000000,0.000000}%
\pgfsetstrokecolor{currentstroke}%
\pgfsetdash{}{0pt}%
\pgfusepath{stroke}%
\end{pgfscope}%
\begin{pgfscope}%
\pgfpathrectangle{\pgfqpoint{1.374500in}{0.082500in}}{\pgfqpoint{2.419000in}{2.419000in}}%
\pgfusepath{clip}%
\pgfsetbuttcap%
\pgfsetroundjoin%
\pgfsetlinewidth{1.505625pt}%
\definecolor{currentstroke}{rgb}{0.000000,0.000000,0.000000}%
\pgfsetstrokecolor{currentstroke}%
\pgfsetdash{}{0pt}%
\pgfpathmoveto{\pgfqpoint{1.633180in}{2.511500in}}%
\pgfpathlineto{\pgfqpoint{1.759307in}{2.358444in}}%
\pgfusepath{stroke}%
\end{pgfscope}%
\begin{pgfscope}%
\pgfpathrectangle{\pgfqpoint{1.374500in}{0.082500in}}{\pgfqpoint{2.419000in}{2.419000in}}%
\pgfusepath{clip}%
\pgfsetbuttcap%
\pgfsetroundjoin%
\pgfsetlinewidth{1.505625pt}%
\definecolor{currentstroke}{rgb}{0.000000,0.000000,0.000000}%
\pgfsetstrokecolor{currentstroke}%
\pgfsetdash{}{0pt}%
\pgfpathmoveto{\pgfqpoint{1.759307in}{2.358444in}}%
\pgfpathlineto{\pgfqpoint{2.317498in}{2.390296in}}%
\pgfusepath{stroke}%
\end{pgfscope}%
\begin{pgfscope}%
\pgfpathrectangle{\pgfqpoint{1.374500in}{0.082500in}}{\pgfqpoint{2.419000in}{2.419000in}}%
\pgfusepath{clip}%
\pgfsetbuttcap%
\pgfsetroundjoin%
\pgfsetlinewidth{1.505625pt}%
\definecolor{currentstroke}{rgb}{0.000000,0.000000,0.000000}%
\pgfsetstrokecolor{currentstroke}%
\pgfsetdash{}{0pt}%
\pgfusepath{stroke}%
\end{pgfscope}%
\begin{pgfscope}%
\pgfpathrectangle{\pgfqpoint{1.374500in}{0.082500in}}{\pgfqpoint{2.419000in}{2.419000in}}%
\pgfusepath{clip}%
\pgfsetbuttcap%
\pgfsetroundjoin%
\pgfsetlinewidth{1.505625pt}%
\definecolor{currentstroke}{rgb}{0.000000,0.000000,0.000000}%
\pgfsetstrokecolor{currentstroke}%
\pgfsetdash{}{0pt}%
\pgfusepath{stroke}%
\end{pgfscope}%
\begin{pgfscope}%
\pgfpathrectangle{\pgfqpoint{1.374500in}{0.082500in}}{\pgfqpoint{2.419000in}{2.419000in}}%
\pgfusepath{clip}%
\pgfsetbuttcap%
\pgfsetroundjoin%
\pgfsetlinewidth{1.505625pt}%
\definecolor{currentstroke}{rgb}{0.000000,0.000000,0.000000}%
\pgfsetstrokecolor{currentstroke}%
\pgfsetdash{}{0pt}%
\pgfusepath{stroke}%
\end{pgfscope}%
\begin{pgfscope}%
\pgfpathrectangle{\pgfqpoint{1.374500in}{0.082500in}}{\pgfqpoint{2.419000in}{2.419000in}}%
\pgfusepath{clip}%
\pgfsetbuttcap%
\pgfsetroundjoin%
\pgfsetlinewidth{1.505625pt}%
\definecolor{currentstroke}{rgb}{0.000000,0.000000,0.000000}%
\pgfsetstrokecolor{currentstroke}%
\pgfsetdash{}{0pt}%
\pgfusepath{stroke}%
\end{pgfscope}%
\begin{pgfscope}%
\pgfpathrectangle{\pgfqpoint{1.374500in}{0.082500in}}{\pgfqpoint{2.419000in}{2.419000in}}%
\pgfusepath{clip}%
\pgfsetbuttcap%
\pgfsetroundjoin%
\pgfsetlinewidth{1.505625pt}%
\definecolor{currentstroke}{rgb}{0.000000,0.000000,0.000000}%
\pgfsetstrokecolor{currentstroke}%
\pgfsetdash{}{0pt}%
\pgfpathmoveto{\pgfqpoint{2.317498in}{2.390296in}}%
\pgfpathlineto{\pgfqpoint{2.315969in}{2.511500in}}%
\pgfusepath{stroke}%
\end{pgfscope}%
\begin{pgfscope}%
\pgfpathrectangle{\pgfqpoint{1.374500in}{0.082500in}}{\pgfqpoint{2.419000in}{2.419000in}}%
\pgfusepath{clip}%
\pgfsetbuttcap%
\pgfsetroundjoin%
\pgfsetlinewidth{1.505625pt}%
\definecolor{currentstroke}{rgb}{0.000000,0.000000,0.000000}%
\pgfsetstrokecolor{currentstroke}%
\pgfsetdash{}{0pt}%
\pgfusepath{stroke}%
\end{pgfscope}%
\begin{pgfscope}%
\pgfpathrectangle{\pgfqpoint{1.374500in}{0.082500in}}{\pgfqpoint{2.419000in}{2.419000in}}%
\pgfusepath{clip}%
\pgfsetbuttcap%
\pgfsetroundjoin%
\pgfsetlinewidth{1.505625pt}%
\definecolor{currentstroke}{rgb}{0.000000,0.000000,0.000000}%
\pgfsetstrokecolor{currentstroke}%
\pgfsetdash{}{0pt}%
\pgfusepath{stroke}%
\end{pgfscope}%
\begin{pgfscope}%
\pgfpathrectangle{\pgfqpoint{1.374500in}{0.082500in}}{\pgfqpoint{2.419000in}{2.419000in}}%
\pgfusepath{clip}%
\pgfsetbuttcap%
\pgfsetroundjoin%
\pgfsetlinewidth{1.505625pt}%
\definecolor{currentstroke}{rgb}{0.000000,0.000000,0.000000}%
\pgfsetstrokecolor{currentstroke}%
\pgfsetdash{}{0pt}%
\pgfusepath{stroke}%
\end{pgfscope}%
\begin{pgfscope}%
\pgfpathrectangle{\pgfqpoint{1.374500in}{0.082500in}}{\pgfqpoint{2.419000in}{2.419000in}}%
\pgfusepath{clip}%
\pgfsetbuttcap%
\pgfsetroundjoin%
\pgfsetlinewidth{1.505625pt}%
\definecolor{currentstroke}{rgb}{0.000000,0.000000,0.000000}%
\pgfsetstrokecolor{currentstroke}%
\pgfsetdash{}{0pt}%
\pgfusepath{stroke}%
\end{pgfscope}%
\begin{pgfscope}%
\pgfpathrectangle{\pgfqpoint{1.374500in}{0.082500in}}{\pgfqpoint{2.419000in}{2.419000in}}%
\pgfusepath{clip}%
\pgfsetbuttcap%
\pgfsetroundjoin%
\pgfsetlinewidth{1.505625pt}%
\definecolor{currentstroke}{rgb}{0.000000,0.000000,0.000000}%
\pgfsetstrokecolor{currentstroke}%
\pgfsetdash{}{0pt}%
\pgfusepath{stroke}%
\end{pgfscope}%
\begin{pgfscope}%
\pgfpathrectangle{\pgfqpoint{1.374500in}{0.082500in}}{\pgfqpoint{2.419000in}{2.419000in}}%
\pgfusepath{clip}%
\pgfsetbuttcap%
\pgfsetroundjoin%
\pgfsetlinewidth{1.505625pt}%
\definecolor{currentstroke}{rgb}{0.000000,0.000000,0.000000}%
\pgfsetstrokecolor{currentstroke}%
\pgfsetdash{}{0pt}%
\pgfusepath{stroke}%
\end{pgfscope}%
\begin{pgfscope}%
\pgfpathrectangle{\pgfqpoint{1.374500in}{0.082500in}}{\pgfqpoint{2.419000in}{2.419000in}}%
\pgfusepath{clip}%
\pgfsetbuttcap%
\pgfsetroundjoin%
\pgfsetlinewidth{1.505625pt}%
\definecolor{currentstroke}{rgb}{0.000000,0.000000,0.000000}%
\pgfsetstrokecolor{currentstroke}%
\pgfsetdash{}{0pt}%
\pgfusepath{stroke}%
\end{pgfscope}%
\begin{pgfscope}%
\pgfpathrectangle{\pgfqpoint{1.374500in}{0.082500in}}{\pgfqpoint{2.419000in}{2.419000in}}%
\pgfusepath{clip}%
\pgfsetbuttcap%
\pgfsetroundjoin%
\pgfsetlinewidth{1.505625pt}%
\definecolor{currentstroke}{rgb}{0.000000,0.000000,0.000000}%
\pgfsetstrokecolor{currentstroke}%
\pgfsetdash{}{0pt}%
\pgfusepath{stroke}%
\end{pgfscope}%
\begin{pgfscope}%
\pgfpathrectangle{\pgfqpoint{1.374500in}{0.082500in}}{\pgfqpoint{2.419000in}{2.419000in}}%
\pgfusepath{clip}%
\pgfsetbuttcap%
\pgfsetroundjoin%
\pgfsetlinewidth{1.505625pt}%
\definecolor{currentstroke}{rgb}{0.000000,0.000000,0.000000}%
\pgfsetstrokecolor{currentstroke}%
\pgfsetdash{}{0pt}%
\pgfusepath{stroke}%
\end{pgfscope}%
\begin{pgfscope}%
\pgfpathrectangle{\pgfqpoint{1.374500in}{0.082500in}}{\pgfqpoint{2.419000in}{2.419000in}}%
\pgfusepath{clip}%
\pgfsetbuttcap%
\pgfsetroundjoin%
\pgfsetlinewidth{1.505625pt}%
\definecolor{currentstroke}{rgb}{0.000000,0.000000,0.000000}%
\pgfsetstrokecolor{currentstroke}%
\pgfsetdash{}{0pt}%
\pgfusepath{stroke}%
\end{pgfscope}%
\begin{pgfscope}%
\pgfpathrectangle{\pgfqpoint{1.374500in}{0.082500in}}{\pgfqpoint{2.419000in}{2.419000in}}%
\pgfusepath{clip}%
\pgfsetbuttcap%
\pgfsetroundjoin%
\pgfsetlinewidth{1.505625pt}%
\definecolor{currentstroke}{rgb}{0.000000,0.000000,0.000000}%
\pgfsetstrokecolor{currentstroke}%
\pgfsetdash{}{0pt}%
\pgfusepath{stroke}%
\end{pgfscope}%
\begin{pgfscope}%
\pgfpathrectangle{\pgfqpoint{1.374500in}{0.082500in}}{\pgfqpoint{2.419000in}{2.419000in}}%
\pgfusepath{clip}%
\pgfsetbuttcap%
\pgfsetroundjoin%
\pgfsetlinewidth{1.505625pt}%
\definecolor{currentstroke}{rgb}{0.000000,0.000000,0.000000}%
\pgfsetstrokecolor{currentstroke}%
\pgfsetdash{}{0pt}%
\pgfusepath{stroke}%
\end{pgfscope}%
\begin{pgfscope}%
\pgfpathrectangle{\pgfqpoint{1.374500in}{0.082500in}}{\pgfqpoint{2.419000in}{2.419000in}}%
\pgfusepath{clip}%
\pgfsetbuttcap%
\pgfsetroundjoin%
\pgfsetlinewidth{1.505625pt}%
\definecolor{currentstroke}{rgb}{0.000000,0.000000,0.000000}%
\pgfsetstrokecolor{currentstroke}%
\pgfsetdash{}{0pt}%
\pgfusepath{stroke}%
\end{pgfscope}%
\begin{pgfscope}%
\pgfpathrectangle{\pgfqpoint{1.374500in}{0.082500in}}{\pgfqpoint{2.419000in}{2.419000in}}%
\pgfusepath{clip}%
\pgfsetbuttcap%
\pgfsetroundjoin%
\pgfsetlinewidth{1.505625pt}%
\definecolor{currentstroke}{rgb}{0.000000,0.000000,0.000000}%
\pgfsetstrokecolor{currentstroke}%
\pgfsetdash{}{0pt}%
\pgfusepath{stroke}%
\end{pgfscope}%
\begin{pgfscope}%
\pgfpathrectangle{\pgfqpoint{1.374500in}{0.082500in}}{\pgfqpoint{2.419000in}{2.419000in}}%
\pgfusepath{clip}%
\pgfsetbuttcap%
\pgfsetroundjoin%
\pgfsetlinewidth{1.505625pt}%
\definecolor{currentstroke}{rgb}{0.000000,0.000000,0.000000}%
\pgfsetstrokecolor{currentstroke}%
\pgfsetdash{}{0pt}%
\pgfpathmoveto{\pgfqpoint{3.098712in}{2.194370in}}%
\pgfpathlineto{\pgfqpoint{3.417665in}{2.453075in}}%
\pgfusepath{stroke}%
\end{pgfscope}%
\begin{pgfscope}%
\pgfpathrectangle{\pgfqpoint{1.374500in}{0.082500in}}{\pgfqpoint{2.419000in}{2.419000in}}%
\pgfusepath{clip}%
\pgfsetbuttcap%
\pgfsetroundjoin%
\pgfsetlinewidth{1.505625pt}%
\definecolor{currentstroke}{rgb}{0.000000,0.000000,0.000000}%
\pgfsetstrokecolor{currentstroke}%
\pgfsetdash{}{0pt}%
\pgfusepath{stroke}%
\end{pgfscope}%
\begin{pgfscope}%
\pgfpathrectangle{\pgfqpoint{1.374500in}{0.082500in}}{\pgfqpoint{2.419000in}{2.419000in}}%
\pgfusepath{clip}%
\pgfsetbuttcap%
\pgfsetroundjoin%
\pgfsetlinewidth{1.505625pt}%
\definecolor{currentstroke}{rgb}{0.000000,0.000000,0.000000}%
\pgfsetstrokecolor{currentstroke}%
\pgfsetdash{}{0pt}%
\pgfusepath{stroke}%
\end{pgfscope}%
\begin{pgfscope}%
\pgfpathrectangle{\pgfqpoint{1.374500in}{0.082500in}}{\pgfqpoint{2.419000in}{2.419000in}}%
\pgfusepath{clip}%
\pgfsetbuttcap%
\pgfsetroundjoin%
\pgfsetlinewidth{1.505625pt}%
\definecolor{currentstroke}{rgb}{0.000000,0.000000,0.000000}%
\pgfsetstrokecolor{currentstroke}%
\pgfsetdash{}{0pt}%
\pgfpathmoveto{\pgfqpoint{2.317498in}{2.390296in}}%
\pgfpathlineto{\pgfqpoint{2.527523in}{2.160559in}}%
\pgfusepath{stroke}%
\end{pgfscope}%
\begin{pgfscope}%
\pgfpathrectangle{\pgfqpoint{1.374500in}{0.082500in}}{\pgfqpoint{2.419000in}{2.419000in}}%
\pgfusepath{clip}%
\pgfsetbuttcap%
\pgfsetroundjoin%
\pgfsetlinewidth{1.505625pt}%
\definecolor{currentstroke}{rgb}{0.000000,0.000000,0.000000}%
\pgfsetstrokecolor{currentstroke}%
\pgfsetdash{}{0pt}%
\pgfpathmoveto{\pgfqpoint{2.527523in}{2.160559in}}%
\pgfpathlineto{\pgfqpoint{3.098712in}{2.194370in}}%
\pgfusepath{stroke}%
\end{pgfscope}%
\begin{pgfscope}%
\pgfpathrectangle{\pgfqpoint{1.374500in}{0.082500in}}{\pgfqpoint{2.419000in}{2.419000in}}%
\pgfusepath{clip}%
\pgfsetbuttcap%
\pgfsetroundjoin%
\pgfsetlinewidth{1.505625pt}%
\definecolor{currentstroke}{rgb}{0.000000,0.000000,0.000000}%
\pgfsetstrokecolor{currentstroke}%
\pgfsetdash{}{0pt}%
\pgfusepath{stroke}%
\end{pgfscope}%
\begin{pgfscope}%
\pgfpathrectangle{\pgfqpoint{1.374500in}{0.082500in}}{\pgfqpoint{2.419000in}{2.419000in}}%
\pgfusepath{clip}%
\pgfsetbuttcap%
\pgfsetroundjoin%
\pgfsetlinewidth{1.505625pt}%
\definecolor{currentstroke}{rgb}{0.000000,0.000000,0.000000}%
\pgfsetstrokecolor{currentstroke}%
\pgfsetdash{}{0pt}%
\pgfusepath{stroke}%
\end{pgfscope}%
\begin{pgfscope}%
\pgfpathrectangle{\pgfqpoint{1.374500in}{0.082500in}}{\pgfqpoint{2.419000in}{2.419000in}}%
\pgfusepath{clip}%
\pgfsetbuttcap%
\pgfsetroundjoin%
\pgfsetlinewidth{1.505625pt}%
\definecolor{currentstroke}{rgb}{0.000000,0.000000,0.000000}%
\pgfsetstrokecolor{currentstroke}%
\pgfsetdash{}{0pt}%
\pgfusepath{stroke}%
\end{pgfscope}%
\begin{pgfscope}%
\pgfpathrectangle{\pgfqpoint{1.374500in}{0.082500in}}{\pgfqpoint{2.419000in}{2.419000in}}%
\pgfusepath{clip}%
\pgfsetbuttcap%
\pgfsetroundjoin%
\pgfsetlinewidth{1.505625pt}%
\definecolor{currentstroke}{rgb}{0.000000,0.000000,0.000000}%
\pgfsetstrokecolor{currentstroke}%
\pgfsetdash{}{0pt}%
\pgfusepath{stroke}%
\end{pgfscope}%
\begin{pgfscope}%
\pgfpathrectangle{\pgfqpoint{1.374500in}{0.082500in}}{\pgfqpoint{2.419000in}{2.419000in}}%
\pgfusepath{clip}%
\pgfsetbuttcap%
\pgfsetroundjoin%
\pgfsetlinewidth{1.505625pt}%
\definecolor{currentstroke}{rgb}{0.000000,0.000000,0.000000}%
\pgfsetstrokecolor{currentstroke}%
\pgfsetdash{}{0pt}%
\pgfpathmoveto{\pgfqpoint{3.098712in}{2.194370in}}%
\pgfpathlineto{\pgfqpoint{3.103206in}{2.413613in}}%
\pgfusepath{stroke}%
\end{pgfscope}%
\begin{pgfscope}%
\pgfpathrectangle{\pgfqpoint{1.374500in}{0.082500in}}{\pgfqpoint{2.419000in}{2.419000in}}%
\pgfusepath{clip}%
\pgfsetbuttcap%
\pgfsetroundjoin%
\pgfsetlinewidth{1.505625pt}%
\definecolor{currentstroke}{rgb}{0.000000,0.000000,0.000000}%
\pgfsetstrokecolor{currentstroke}%
\pgfsetdash{}{0pt}%
\pgfusepath{stroke}%
\end{pgfscope}%
\begin{pgfscope}%
\pgfpathrectangle{\pgfqpoint{1.374500in}{0.082500in}}{\pgfqpoint{2.419000in}{2.419000in}}%
\pgfusepath{clip}%
\pgfsetbuttcap%
\pgfsetroundjoin%
\pgfsetlinewidth{1.505625pt}%
\definecolor{currentstroke}{rgb}{0.000000,0.000000,0.000000}%
\pgfsetstrokecolor{currentstroke}%
\pgfsetdash{}{0pt}%
\pgfusepath{stroke}%
\end{pgfscope}%
\begin{pgfscope}%
\pgfpathrectangle{\pgfqpoint{1.374500in}{0.082500in}}{\pgfqpoint{2.419000in}{2.419000in}}%
\pgfusepath{clip}%
\pgfsetbuttcap%
\pgfsetroundjoin%
\pgfsetlinewidth{1.505625pt}%
\definecolor{currentstroke}{rgb}{0.000000,0.000000,0.000000}%
\pgfsetstrokecolor{currentstroke}%
\pgfsetdash{}{0pt}%
\pgfpathmoveto{\pgfqpoint{1.367619in}{2.091899in}}%
\pgfpathlineto{\pgfqpoint{1.759307in}{2.358444in}}%
\pgfusepath{stroke}%
\end{pgfscope}%
\begin{pgfscope}%
\pgfpathrectangle{\pgfqpoint{1.374500in}{0.082500in}}{\pgfqpoint{2.419000in}{2.419000in}}%
\pgfusepath{clip}%
\pgfsetbuttcap%
\pgfsetroundjoin%
\pgfsetlinewidth{1.505625pt}%
\definecolor{currentstroke}{rgb}{0.000000,0.000000,0.000000}%
\pgfsetstrokecolor{currentstroke}%
\pgfsetdash{}{0pt}%
\pgfusepath{stroke}%
\end{pgfscope}%
\begin{pgfscope}%
\pgfpathrectangle{\pgfqpoint{1.374500in}{0.082500in}}{\pgfqpoint{2.419000in}{2.419000in}}%
\pgfusepath{clip}%
\pgfsetbuttcap%
\pgfsetroundjoin%
\pgfsetlinewidth{1.505625pt}%
\definecolor{currentstroke}{rgb}{0.000000,0.000000,0.000000}%
\pgfsetstrokecolor{currentstroke}%
\pgfsetdash{}{0pt}%
\pgfusepath{stroke}%
\end{pgfscope}%
\begin{pgfscope}%
\pgfpathrectangle{\pgfqpoint{1.374500in}{0.082500in}}{\pgfqpoint{2.419000in}{2.419000in}}%
\pgfusepath{clip}%
\pgfsetbuttcap%
\pgfsetroundjoin%
\pgfsetlinewidth{1.505625pt}%
\definecolor{currentstroke}{rgb}{0.000000,0.000000,0.000000}%
\pgfsetstrokecolor{currentstroke}%
\pgfsetdash{}{0pt}%
\pgfusepath{stroke}%
\end{pgfscope}%
\begin{pgfscope}%
\pgfpathrectangle{\pgfqpoint{1.374500in}{0.082500in}}{\pgfqpoint{2.419000in}{2.419000in}}%
\pgfusepath{clip}%
\pgfsetbuttcap%
\pgfsetroundjoin%
\pgfsetlinewidth{1.505625pt}%
\definecolor{currentstroke}{rgb}{0.000000,0.000000,0.000000}%
\pgfsetstrokecolor{currentstroke}%
\pgfsetdash{}{0pt}%
\pgfpathmoveto{\pgfqpoint{1.364500in}{2.091714in}}%
\pgfpathlineto{\pgfqpoint{1.367619in}{2.091899in}}%
\pgfusepath{stroke}%
\end{pgfscope}%
\begin{pgfscope}%
\pgfpathrectangle{\pgfqpoint{1.374500in}{0.082500in}}{\pgfqpoint{2.419000in}{2.419000in}}%
\pgfusepath{clip}%
\pgfsetbuttcap%
\pgfsetroundjoin%
\pgfsetlinewidth{1.505625pt}%
\definecolor{currentstroke}{rgb}{0.000000,0.000000,0.000000}%
\pgfsetstrokecolor{currentstroke}%
\pgfsetdash{}{0pt}%
\pgfusepath{stroke}%
\end{pgfscope}%
\begin{pgfscope}%
\pgfpathrectangle{\pgfqpoint{1.374500in}{0.082500in}}{\pgfqpoint{2.419000in}{2.419000in}}%
\pgfusepath{clip}%
\pgfsetbuttcap%
\pgfsetroundjoin%
\pgfsetlinewidth{1.505625pt}%
\definecolor{currentstroke}{rgb}{0.000000,0.000000,0.000000}%
\pgfsetstrokecolor{currentstroke}%
\pgfsetdash{}{0pt}%
\pgfusepath{stroke}%
\end{pgfscope}%
\begin{pgfscope}%
\pgfpathrectangle{\pgfqpoint{1.374500in}{0.082500in}}{\pgfqpoint{2.419000in}{2.419000in}}%
\pgfusepath{clip}%
\pgfsetbuttcap%
\pgfsetroundjoin%
\pgfsetlinewidth{1.505625pt}%
\definecolor{currentstroke}{rgb}{0.000000,0.000000,0.000000}%
\pgfsetstrokecolor{currentstroke}%
\pgfsetdash{}{0pt}%
\pgfusepath{stroke}%
\end{pgfscope}%
\begin{pgfscope}%
\pgfpathrectangle{\pgfqpoint{1.374500in}{0.082500in}}{\pgfqpoint{2.419000in}{2.419000in}}%
\pgfusepath{clip}%
\pgfsetbuttcap%
\pgfsetroundjoin%
\pgfsetlinewidth{1.505625pt}%
\definecolor{currentstroke}{rgb}{0.000000,0.000000,0.000000}%
\pgfsetstrokecolor{currentstroke}%
\pgfsetdash{}{0pt}%
\pgfusepath{stroke}%
\end{pgfscope}%
\begin{pgfscope}%
\pgfpathrectangle{\pgfqpoint{1.374500in}{0.082500in}}{\pgfqpoint{2.419000in}{2.419000in}}%
\pgfusepath{clip}%
\pgfsetbuttcap%
\pgfsetroundjoin%
\pgfsetlinewidth{1.505625pt}%
\definecolor{currentstroke}{rgb}{0.000000,0.000000,0.000000}%
\pgfsetstrokecolor{currentstroke}%
\pgfsetdash{}{0pt}%
\pgfpathmoveto{\pgfqpoint{1.367619in}{2.091899in}}%
\pgfpathlineto{\pgfqpoint{1.364500in}{2.150368in}}%
\pgfusepath{stroke}%
\end{pgfscope}%
\begin{pgfscope}%
\pgfpathrectangle{\pgfqpoint{1.374500in}{0.082500in}}{\pgfqpoint{2.419000in}{2.419000in}}%
\pgfusepath{clip}%
\pgfsetbuttcap%
\pgfsetroundjoin%
\pgfsetlinewidth{1.505625pt}%
\definecolor{currentstroke}{rgb}{0.000000,0.000000,0.000000}%
\pgfsetstrokecolor{currentstroke}%
\pgfsetdash{}{0pt}%
\pgfusepath{stroke}%
\end{pgfscope}%
\begin{pgfscope}%
\pgfpathrectangle{\pgfqpoint{1.374500in}{0.082500in}}{\pgfqpoint{2.419000in}{2.419000in}}%
\pgfusepath{clip}%
\pgfsetbuttcap%
\pgfsetroundjoin%
\pgfsetlinewidth{1.505625pt}%
\definecolor{currentstroke}{rgb}{0.000000,0.000000,0.000000}%
\pgfsetstrokecolor{currentstroke}%
\pgfsetdash{}{0pt}%
\pgfusepath{stroke}%
\end{pgfscope}%
\begin{pgfscope}%
\pgfpathrectangle{\pgfqpoint{1.374500in}{0.082500in}}{\pgfqpoint{2.419000in}{2.419000in}}%
\pgfusepath{clip}%
\pgfsetbuttcap%
\pgfsetroundjoin%
\pgfsetlinewidth{1.505625pt}%
\definecolor{currentstroke}{rgb}{0.000000,0.000000,0.000000}%
\pgfsetstrokecolor{currentstroke}%
\pgfsetdash{}{0pt}%
\pgfusepath{stroke}%
\end{pgfscope}%
\begin{pgfscope}%
\pgfpathrectangle{\pgfqpoint{1.374500in}{0.082500in}}{\pgfqpoint{2.419000in}{2.419000in}}%
\pgfusepath{clip}%
\pgfsetbuttcap%
\pgfsetroundjoin%
\pgfsetlinewidth{1.505625pt}%
\definecolor{currentstroke}{rgb}{0.000000,0.000000,0.000000}%
\pgfsetstrokecolor{currentstroke}%
\pgfsetdash{}{0pt}%
\pgfusepath{stroke}%
\end{pgfscope}%
\begin{pgfscope}%
\pgfpathrectangle{\pgfqpoint{1.374500in}{0.082500in}}{\pgfqpoint{2.419000in}{2.419000in}}%
\pgfusepath{clip}%
\pgfsetbuttcap%
\pgfsetroundjoin%
\pgfsetlinewidth{1.505625pt}%
\definecolor{currentstroke}{rgb}{0.000000,0.000000,0.000000}%
\pgfsetstrokecolor{currentstroke}%
\pgfsetdash{}{0pt}%
\pgfusepath{stroke}%
\end{pgfscope}%
\begin{pgfscope}%
\pgfpathrectangle{\pgfqpoint{1.374500in}{0.082500in}}{\pgfqpoint{2.419000in}{2.419000in}}%
\pgfusepath{clip}%
\pgfsetbuttcap%
\pgfsetroundjoin%
\pgfsetlinewidth{1.505625pt}%
\definecolor{currentstroke}{rgb}{0.000000,0.000000,0.000000}%
\pgfsetstrokecolor{currentstroke}%
\pgfsetdash{}{0pt}%
\pgfusepath{stroke}%
\end{pgfscope}%
\begin{pgfscope}%
\pgfpathrectangle{\pgfqpoint{1.374500in}{0.082500in}}{\pgfqpoint{2.419000in}{2.419000in}}%
\pgfusepath{clip}%
\pgfsetbuttcap%
\pgfsetroundjoin%
\pgfsetlinewidth{1.505625pt}%
\definecolor{currentstroke}{rgb}{0.000000,0.000000,0.000000}%
\pgfsetstrokecolor{currentstroke}%
\pgfsetdash{}{0pt}%
\pgfpathmoveto{\pgfqpoint{3.098712in}{2.194370in}}%
\pgfpathlineto{\pgfqpoint{3.343864in}{1.950277in}}%
\pgfusepath{stroke}%
\end{pgfscope}%
\begin{pgfscope}%
\pgfpathrectangle{\pgfqpoint{1.374500in}{0.082500in}}{\pgfqpoint{2.419000in}{2.419000in}}%
\pgfusepath{clip}%
\pgfsetbuttcap%
\pgfsetroundjoin%
\pgfsetlinewidth{1.505625pt}%
\definecolor{currentstroke}{rgb}{0.000000,0.000000,0.000000}%
\pgfsetstrokecolor{currentstroke}%
\pgfsetdash{}{0pt}%
\pgfpathmoveto{\pgfqpoint{3.343864in}{1.950277in}}%
\pgfpathlineto{\pgfqpoint{3.803500in}{1.978541in}}%
\pgfusepath{stroke}%
\end{pgfscope}%
\begin{pgfscope}%
\pgfpathrectangle{\pgfqpoint{1.374500in}{0.082500in}}{\pgfqpoint{2.419000in}{2.419000in}}%
\pgfusepath{clip}%
\pgfsetbuttcap%
\pgfsetroundjoin%
\pgfsetlinewidth{1.505625pt}%
\definecolor{currentstroke}{rgb}{0.000000,0.000000,0.000000}%
\pgfsetstrokecolor{currentstroke}%
\pgfsetdash{}{0pt}%
\pgfusepath{stroke}%
\end{pgfscope}%
\begin{pgfscope}%
\pgfpathrectangle{\pgfqpoint{1.374500in}{0.082500in}}{\pgfqpoint{2.419000in}{2.419000in}}%
\pgfusepath{clip}%
\pgfsetbuttcap%
\pgfsetroundjoin%
\pgfsetlinewidth{1.505625pt}%
\definecolor{currentstroke}{rgb}{0.000000,0.000000,0.000000}%
\pgfsetstrokecolor{currentstroke}%
\pgfsetdash{}{0pt}%
\pgfusepath{stroke}%
\end{pgfscope}%
\begin{pgfscope}%
\pgfpathrectangle{\pgfqpoint{1.374500in}{0.082500in}}{\pgfqpoint{2.419000in}{2.419000in}}%
\pgfusepath{clip}%
\pgfsetbuttcap%
\pgfsetroundjoin%
\pgfsetlinewidth{1.505625pt}%
\definecolor{currentstroke}{rgb}{0.000000,0.000000,0.000000}%
\pgfsetstrokecolor{currentstroke}%
\pgfsetdash{}{0pt}%
\pgfusepath{stroke}%
\end{pgfscope}%
\begin{pgfscope}%
\pgfpathrectangle{\pgfqpoint{1.374500in}{0.082500in}}{\pgfqpoint{2.419000in}{2.419000in}}%
\pgfusepath{clip}%
\pgfsetbuttcap%
\pgfsetroundjoin%
\pgfsetlinewidth{1.505625pt}%
\definecolor{currentstroke}{rgb}{0.000000,0.000000,0.000000}%
\pgfsetstrokecolor{currentstroke}%
\pgfsetdash{}{0pt}%
\pgfusepath{stroke}%
\end{pgfscope}%
\begin{pgfscope}%
\pgfpathrectangle{\pgfqpoint{1.374500in}{0.082500in}}{\pgfqpoint{2.419000in}{2.419000in}}%
\pgfusepath{clip}%
\pgfsetbuttcap%
\pgfsetroundjoin%
\pgfsetlinewidth{1.505625pt}%
\definecolor{currentstroke}{rgb}{0.000000,0.000000,0.000000}%
\pgfsetstrokecolor{currentstroke}%
\pgfsetdash{}{0pt}%
\pgfusepath{stroke}%
\end{pgfscope}%
\begin{pgfscope}%
\pgfpathrectangle{\pgfqpoint{1.374500in}{0.082500in}}{\pgfqpoint{2.419000in}{2.419000in}}%
\pgfusepath{clip}%
\pgfsetbuttcap%
\pgfsetroundjoin%
\pgfsetlinewidth{1.505625pt}%
\definecolor{currentstroke}{rgb}{0.000000,0.000000,0.000000}%
\pgfsetstrokecolor{currentstroke}%
\pgfsetdash{}{0pt}%
\pgfpathmoveto{\pgfqpoint{2.155859in}{1.877225in}}%
\pgfpathlineto{\pgfqpoint{2.527523in}{2.160559in}}%
\pgfusepath{stroke}%
\end{pgfscope}%
\begin{pgfscope}%
\pgfpathrectangle{\pgfqpoint{1.374500in}{0.082500in}}{\pgfqpoint{2.419000in}{2.419000in}}%
\pgfusepath{clip}%
\pgfsetbuttcap%
\pgfsetroundjoin%
\pgfsetlinewidth{1.505625pt}%
\definecolor{currentstroke}{rgb}{0.000000,0.000000,0.000000}%
\pgfsetstrokecolor{currentstroke}%
\pgfsetdash{}{0pt}%
\pgfusepath{stroke}%
\end{pgfscope}%
\begin{pgfscope}%
\pgfpathrectangle{\pgfqpoint{1.374500in}{0.082500in}}{\pgfqpoint{2.419000in}{2.419000in}}%
\pgfusepath{clip}%
\pgfsetbuttcap%
\pgfsetroundjoin%
\pgfsetlinewidth{1.505625pt}%
\definecolor{currentstroke}{rgb}{0.000000,0.000000,0.000000}%
\pgfsetstrokecolor{currentstroke}%
\pgfsetdash{}{0pt}%
\pgfusepath{stroke}%
\end{pgfscope}%
\begin{pgfscope}%
\pgfpathrectangle{\pgfqpoint{1.374500in}{0.082500in}}{\pgfqpoint{2.419000in}{2.419000in}}%
\pgfusepath{clip}%
\pgfsetbuttcap%
\pgfsetroundjoin%
\pgfsetlinewidth{1.505625pt}%
\definecolor{currentstroke}{rgb}{0.000000,0.000000,0.000000}%
\pgfsetstrokecolor{currentstroke}%
\pgfsetdash{}{0pt}%
\pgfpathmoveto{\pgfqpoint{1.367619in}{2.091899in}}%
\pgfpathlineto{\pgfqpoint{1.552405in}{1.840117in}}%
\pgfusepath{stroke}%
\end{pgfscope}%
\begin{pgfscope}%
\pgfpathrectangle{\pgfqpoint{1.374500in}{0.082500in}}{\pgfqpoint{2.419000in}{2.419000in}}%
\pgfusepath{clip}%
\pgfsetbuttcap%
\pgfsetroundjoin%
\pgfsetlinewidth{1.505625pt}%
\definecolor{currentstroke}{rgb}{0.000000,0.000000,0.000000}%
\pgfsetstrokecolor{currentstroke}%
\pgfsetdash{}{0pt}%
\pgfpathmoveto{\pgfqpoint{1.552405in}{1.840117in}}%
\pgfpathlineto{\pgfqpoint{2.155859in}{1.877225in}}%
\pgfusepath{stroke}%
\end{pgfscope}%
\begin{pgfscope}%
\pgfpathrectangle{\pgfqpoint{1.374500in}{0.082500in}}{\pgfqpoint{2.419000in}{2.419000in}}%
\pgfusepath{clip}%
\pgfsetbuttcap%
\pgfsetroundjoin%
\pgfsetlinewidth{1.505625pt}%
\definecolor{currentstroke}{rgb}{0.000000,0.000000,0.000000}%
\pgfsetstrokecolor{currentstroke}%
\pgfsetdash{}{0pt}%
\pgfusepath{stroke}%
\end{pgfscope}%
\begin{pgfscope}%
\pgfpathrectangle{\pgfqpoint{1.374500in}{0.082500in}}{\pgfqpoint{2.419000in}{2.419000in}}%
\pgfusepath{clip}%
\pgfsetbuttcap%
\pgfsetroundjoin%
\pgfsetlinewidth{1.505625pt}%
\definecolor{currentstroke}{rgb}{0.000000,0.000000,0.000000}%
\pgfsetstrokecolor{currentstroke}%
\pgfsetdash{}{0pt}%
\pgfusepath{stroke}%
\end{pgfscope}%
\begin{pgfscope}%
\pgfpathrectangle{\pgfqpoint{1.374500in}{0.082500in}}{\pgfqpoint{2.419000in}{2.419000in}}%
\pgfusepath{clip}%
\pgfsetbuttcap%
\pgfsetroundjoin%
\pgfsetlinewidth{1.505625pt}%
\definecolor{currentstroke}{rgb}{0.000000,0.000000,0.000000}%
\pgfsetstrokecolor{currentstroke}%
\pgfsetdash{}{0pt}%
\pgfusepath{stroke}%
\end{pgfscope}%
\begin{pgfscope}%
\pgfpathrectangle{\pgfqpoint{1.374500in}{0.082500in}}{\pgfqpoint{2.419000in}{2.419000in}}%
\pgfusepath{clip}%
\pgfsetbuttcap%
\pgfsetroundjoin%
\pgfsetlinewidth{1.505625pt}%
\definecolor{currentstroke}{rgb}{0.000000,0.000000,0.000000}%
\pgfsetstrokecolor{currentstroke}%
\pgfsetdash{}{0pt}%
\pgfusepath{stroke}%
\end{pgfscope}%
\begin{pgfscope}%
\pgfpathrectangle{\pgfqpoint{1.374500in}{0.082500in}}{\pgfqpoint{2.419000in}{2.419000in}}%
\pgfusepath{clip}%
\pgfsetbuttcap%
\pgfsetroundjoin%
\pgfsetlinewidth{1.505625pt}%
\definecolor{currentstroke}{rgb}{0.000000,0.000000,0.000000}%
\pgfsetstrokecolor{currentstroke}%
\pgfsetdash{}{0pt}%
\pgfpathmoveto{\pgfqpoint{2.155859in}{1.877225in}}%
\pgfpathlineto{\pgfqpoint{2.151357in}{2.103881in}}%
\pgfusepath{stroke}%
\end{pgfscope}%
\begin{pgfscope}%
\pgfpathrectangle{\pgfqpoint{1.374500in}{0.082500in}}{\pgfqpoint{2.419000in}{2.419000in}}%
\pgfusepath{clip}%
\pgfsetbuttcap%
\pgfsetroundjoin%
\pgfsetlinewidth{1.505625pt}%
\definecolor{currentstroke}{rgb}{0.000000,0.000000,0.000000}%
\pgfsetstrokecolor{currentstroke}%
\pgfsetdash{}{0pt}%
\pgfusepath{stroke}%
\end{pgfscope}%
\begin{pgfscope}%
\pgfpathrectangle{\pgfqpoint{1.374500in}{0.082500in}}{\pgfqpoint{2.419000in}{2.419000in}}%
\pgfusepath{clip}%
\pgfsetbuttcap%
\pgfsetroundjoin%
\pgfsetlinewidth{1.505625pt}%
\definecolor{currentstroke}{rgb}{0.000000,0.000000,0.000000}%
\pgfsetstrokecolor{currentstroke}%
\pgfsetdash{}{0pt}%
\pgfusepath{stroke}%
\end{pgfscope}%
\begin{pgfscope}%
\pgfpathrectangle{\pgfqpoint{1.374500in}{0.082500in}}{\pgfqpoint{2.419000in}{2.419000in}}%
\pgfusepath{clip}%
\pgfsetbuttcap%
\pgfsetroundjoin%
\pgfsetlinewidth{1.505625pt}%
\definecolor{currentstroke}{rgb}{0.000000,0.000000,0.000000}%
\pgfsetstrokecolor{currentstroke}%
\pgfsetdash{}{0pt}%
\pgfusepath{stroke}%
\end{pgfscope}%
\begin{pgfscope}%
\pgfpathrectangle{\pgfqpoint{1.374500in}{0.082500in}}{\pgfqpoint{2.419000in}{2.419000in}}%
\pgfusepath{clip}%
\pgfsetbuttcap%
\pgfsetroundjoin%
\pgfsetlinewidth{1.505625pt}%
\definecolor{currentstroke}{rgb}{0.000000,0.000000,0.000000}%
\pgfsetstrokecolor{currentstroke}%
\pgfsetdash{}{0pt}%
\pgfusepath{stroke}%
\end{pgfscope}%
\begin{pgfscope}%
\pgfpathrectangle{\pgfqpoint{1.374500in}{0.082500in}}{\pgfqpoint{2.419000in}{2.419000in}}%
\pgfusepath{clip}%
\pgfsetbuttcap%
\pgfsetroundjoin%
\pgfsetlinewidth{1.505625pt}%
\definecolor{currentstroke}{rgb}{0.000000,0.000000,0.000000}%
\pgfsetstrokecolor{currentstroke}%
\pgfsetdash{}{0pt}%
\pgfusepath{stroke}%
\end{pgfscope}%
\begin{pgfscope}%
\pgfpathrectangle{\pgfqpoint{1.374500in}{0.082500in}}{\pgfqpoint{2.419000in}{2.419000in}}%
\pgfusepath{clip}%
\pgfsetbuttcap%
\pgfsetroundjoin%
\pgfsetlinewidth{1.505625pt}%
\definecolor{currentstroke}{rgb}{0.000000,0.000000,0.000000}%
\pgfsetstrokecolor{currentstroke}%
\pgfsetdash{}{0pt}%
\pgfusepath{stroke}%
\end{pgfscope}%
\begin{pgfscope}%
\pgfpathrectangle{\pgfqpoint{1.374500in}{0.082500in}}{\pgfqpoint{2.419000in}{2.419000in}}%
\pgfusepath{clip}%
\pgfsetbuttcap%
\pgfsetroundjoin%
\pgfsetlinewidth{1.505625pt}%
\definecolor{currentstroke}{rgb}{0.000000,0.000000,0.000000}%
\pgfsetstrokecolor{currentstroke}%
\pgfsetdash{}{0pt}%
\pgfusepath{stroke}%
\end{pgfscope}%
\begin{pgfscope}%
\pgfpathrectangle{\pgfqpoint{1.374500in}{0.082500in}}{\pgfqpoint{2.419000in}{2.419000in}}%
\pgfusepath{clip}%
\pgfsetbuttcap%
\pgfsetroundjoin%
\pgfsetlinewidth{1.505625pt}%
\definecolor{currentstroke}{rgb}{0.000000,0.000000,0.000000}%
\pgfsetstrokecolor{currentstroke}%
\pgfsetdash{}{0pt}%
\pgfusepath{stroke}%
\end{pgfscope}%
\begin{pgfscope}%
\pgfpathrectangle{\pgfqpoint{1.374500in}{0.082500in}}{\pgfqpoint{2.419000in}{2.419000in}}%
\pgfusepath{clip}%
\pgfsetbuttcap%
\pgfsetroundjoin%
\pgfsetlinewidth{1.505625pt}%
\definecolor{currentstroke}{rgb}{0.000000,0.000000,0.000000}%
\pgfsetstrokecolor{currentstroke}%
\pgfsetdash{}{0pt}%
\pgfusepath{stroke}%
\end{pgfscope}%
\begin{pgfscope}%
\pgfpathrectangle{\pgfqpoint{1.374500in}{0.082500in}}{\pgfqpoint{2.419000in}{2.419000in}}%
\pgfusepath{clip}%
\pgfsetbuttcap%
\pgfsetroundjoin%
\pgfsetlinewidth{1.505625pt}%
\definecolor{currentstroke}{rgb}{0.000000,0.000000,0.000000}%
\pgfsetstrokecolor{currentstroke}%
\pgfsetdash{}{0pt}%
\pgfusepath{stroke}%
\end{pgfscope}%
\begin{pgfscope}%
\pgfpathrectangle{\pgfqpoint{1.374500in}{0.082500in}}{\pgfqpoint{2.419000in}{2.419000in}}%
\pgfusepath{clip}%
\pgfsetbuttcap%
\pgfsetroundjoin%
\pgfsetlinewidth{1.505625pt}%
\definecolor{currentstroke}{rgb}{0.000000,0.000000,0.000000}%
\pgfsetstrokecolor{currentstroke}%
\pgfsetdash{}{0pt}%
\pgfusepath{stroke}%
\end{pgfscope}%
\begin{pgfscope}%
\pgfpathrectangle{\pgfqpoint{1.374500in}{0.082500in}}{\pgfqpoint{2.419000in}{2.419000in}}%
\pgfusepath{clip}%
\pgfsetbuttcap%
\pgfsetroundjoin%
\pgfsetlinewidth{1.505625pt}%
\definecolor{currentstroke}{rgb}{0.000000,0.000000,0.000000}%
\pgfsetstrokecolor{currentstroke}%
\pgfsetdash{}{0pt}%
\pgfusepath{stroke}%
\end{pgfscope}%
\begin{pgfscope}%
\pgfpathrectangle{\pgfqpoint{1.374500in}{0.082500in}}{\pgfqpoint{2.419000in}{2.419000in}}%
\pgfusepath{clip}%
\pgfsetbuttcap%
\pgfsetroundjoin%
\pgfsetlinewidth{1.505625pt}%
\definecolor{currentstroke}{rgb}{0.000000,0.000000,0.000000}%
\pgfsetstrokecolor{currentstroke}%
\pgfsetdash{}{0pt}%
\pgfusepath{stroke}%
\end{pgfscope}%
\begin{pgfscope}%
\pgfpathrectangle{\pgfqpoint{1.374500in}{0.082500in}}{\pgfqpoint{2.419000in}{2.419000in}}%
\pgfusepath{clip}%
\pgfsetbuttcap%
\pgfsetroundjoin%
\pgfsetlinewidth{1.505625pt}%
\definecolor{currentstroke}{rgb}{0.000000,0.000000,0.000000}%
\pgfsetstrokecolor{currentstroke}%
\pgfsetdash{}{0pt}%
\pgfusepath{stroke}%
\end{pgfscope}%
\begin{pgfscope}%
\pgfpathrectangle{\pgfqpoint{1.374500in}{0.082500in}}{\pgfqpoint{2.419000in}{2.419000in}}%
\pgfusepath{clip}%
\pgfsetbuttcap%
\pgfsetroundjoin%
\pgfsetlinewidth{1.505625pt}%
\definecolor{currentstroke}{rgb}{0.000000,0.000000,0.000000}%
\pgfsetstrokecolor{currentstroke}%
\pgfsetdash{}{0pt}%
\pgfusepath{stroke}%
\end{pgfscope}%
\begin{pgfscope}%
\pgfpathrectangle{\pgfqpoint{1.374500in}{0.082500in}}{\pgfqpoint{2.419000in}{2.419000in}}%
\pgfusepath{clip}%
\pgfsetbuttcap%
\pgfsetroundjoin%
\pgfsetlinewidth{1.505625pt}%
\definecolor{currentstroke}{rgb}{0.000000,0.000000,0.000000}%
\pgfsetstrokecolor{currentstroke}%
\pgfsetdash{}{0pt}%
\pgfpathmoveto{\pgfqpoint{2.995631in}{1.648516in}}%
\pgfpathlineto{\pgfqpoint{3.343864in}{1.950277in}}%
\pgfusepath{stroke}%
\end{pgfscope}%
\begin{pgfscope}%
\pgfpathrectangle{\pgfqpoint{1.374500in}{0.082500in}}{\pgfqpoint{2.419000in}{2.419000in}}%
\pgfusepath{clip}%
\pgfsetbuttcap%
\pgfsetroundjoin%
\pgfsetlinewidth{1.505625pt}%
\definecolor{currentstroke}{rgb}{0.000000,0.000000,0.000000}%
\pgfsetstrokecolor{currentstroke}%
\pgfsetdash{}{0pt}%
\pgfusepath{stroke}%
\end{pgfscope}%
\begin{pgfscope}%
\pgfpathrectangle{\pgfqpoint{1.374500in}{0.082500in}}{\pgfqpoint{2.419000in}{2.419000in}}%
\pgfusepath{clip}%
\pgfsetbuttcap%
\pgfsetroundjoin%
\pgfsetlinewidth{1.505625pt}%
\definecolor{currentstroke}{rgb}{0.000000,0.000000,0.000000}%
\pgfsetstrokecolor{currentstroke}%
\pgfsetdash{}{0pt}%
\pgfpathmoveto{\pgfqpoint{2.155859in}{1.877225in}}%
\pgfpathlineto{\pgfqpoint{2.376943in}{1.608936in}}%
\pgfusepath{stroke}%
\end{pgfscope}%
\begin{pgfscope}%
\pgfpathrectangle{\pgfqpoint{1.374500in}{0.082500in}}{\pgfqpoint{2.419000in}{2.419000in}}%
\pgfusepath{clip}%
\pgfsetbuttcap%
\pgfsetroundjoin%
\pgfsetlinewidth{1.505625pt}%
\definecolor{currentstroke}{rgb}{0.000000,0.000000,0.000000}%
\pgfsetstrokecolor{currentstroke}%
\pgfsetdash{}{0pt}%
\pgfpathmoveto{\pgfqpoint{2.376943in}{1.608936in}}%
\pgfpathlineto{\pgfqpoint{2.995631in}{1.648516in}}%
\pgfusepath{stroke}%
\end{pgfscope}%
\begin{pgfscope}%
\pgfpathrectangle{\pgfqpoint{1.374500in}{0.082500in}}{\pgfqpoint{2.419000in}{2.419000in}}%
\pgfusepath{clip}%
\pgfsetbuttcap%
\pgfsetroundjoin%
\pgfsetlinewidth{1.505625pt}%
\definecolor{currentstroke}{rgb}{0.000000,0.000000,0.000000}%
\pgfsetstrokecolor{currentstroke}%
\pgfsetdash{}{0pt}%
\pgfusepath{stroke}%
\end{pgfscope}%
\begin{pgfscope}%
\pgfpathrectangle{\pgfqpoint{1.374500in}{0.082500in}}{\pgfqpoint{2.419000in}{2.419000in}}%
\pgfusepath{clip}%
\pgfsetbuttcap%
\pgfsetroundjoin%
\pgfsetlinewidth{1.505625pt}%
\definecolor{currentstroke}{rgb}{0.000000,0.000000,0.000000}%
\pgfsetstrokecolor{currentstroke}%
\pgfsetdash{}{0pt}%
\pgfusepath{stroke}%
\end{pgfscope}%
\begin{pgfscope}%
\pgfpathrectangle{\pgfqpoint{1.374500in}{0.082500in}}{\pgfqpoint{2.419000in}{2.419000in}}%
\pgfusepath{clip}%
\pgfsetbuttcap%
\pgfsetroundjoin%
\pgfsetlinewidth{1.505625pt}%
\definecolor{currentstroke}{rgb}{0.000000,0.000000,0.000000}%
\pgfsetstrokecolor{currentstroke}%
\pgfsetdash{}{0pt}%
\pgfusepath{stroke}%
\end{pgfscope}%
\begin{pgfscope}%
\pgfpathrectangle{\pgfqpoint{1.374500in}{0.082500in}}{\pgfqpoint{2.419000in}{2.419000in}}%
\pgfusepath{clip}%
\pgfsetbuttcap%
\pgfsetroundjoin%
\pgfsetlinewidth{1.505625pt}%
\definecolor{currentstroke}{rgb}{0.000000,0.000000,0.000000}%
\pgfsetstrokecolor{currentstroke}%
\pgfsetdash{}{0pt}%
\pgfusepath{stroke}%
\end{pgfscope}%
\begin{pgfscope}%
\pgfpathrectangle{\pgfqpoint{1.374500in}{0.082500in}}{\pgfqpoint{2.419000in}{2.419000in}}%
\pgfusepath{clip}%
\pgfsetbuttcap%
\pgfsetroundjoin%
\pgfsetlinewidth{1.505625pt}%
\definecolor{currentstroke}{rgb}{0.000000,0.000000,0.000000}%
\pgfsetstrokecolor{currentstroke}%
\pgfsetdash{}{0pt}%
\pgfpathmoveto{\pgfqpoint{2.995631in}{1.648516in}}%
\pgfpathlineto{\pgfqpoint{2.999455in}{1.880348in}}%
\pgfusepath{stroke}%
\end{pgfscope}%
\begin{pgfscope}%
\pgfpathrectangle{\pgfqpoint{1.374500in}{0.082500in}}{\pgfqpoint{2.419000in}{2.419000in}}%
\pgfusepath{clip}%
\pgfsetbuttcap%
\pgfsetroundjoin%
\pgfsetlinewidth{1.505625pt}%
\definecolor{currentstroke}{rgb}{0.000000,0.000000,0.000000}%
\pgfsetstrokecolor{currentstroke}%
\pgfsetdash{}{0pt}%
\pgfusepath{stroke}%
\end{pgfscope}%
\begin{pgfscope}%
\pgfpathrectangle{\pgfqpoint{1.374500in}{0.082500in}}{\pgfqpoint{2.419000in}{2.419000in}}%
\pgfusepath{clip}%
\pgfsetbuttcap%
\pgfsetroundjoin%
\pgfsetlinewidth{1.505625pt}%
\definecolor{currentstroke}{rgb}{0.000000,0.000000,0.000000}%
\pgfsetstrokecolor{currentstroke}%
\pgfsetdash{}{0pt}%
\pgfpathmoveto{\pgfqpoint{1.364500in}{1.704994in}}%
\pgfpathlineto{\pgfqpoint{1.552405in}{1.840117in}}%
\pgfusepath{stroke}%
\end{pgfscope}%
\begin{pgfscope}%
\pgfpathrectangle{\pgfqpoint{1.374500in}{0.082500in}}{\pgfqpoint{2.419000in}{2.419000in}}%
\pgfusepath{clip}%
\pgfsetbuttcap%
\pgfsetroundjoin%
\pgfsetlinewidth{1.505625pt}%
\definecolor{currentstroke}{rgb}{0.000000,0.000000,0.000000}%
\pgfsetstrokecolor{currentstroke}%
\pgfsetdash{}{0pt}%
\pgfusepath{stroke}%
\end{pgfscope}%
\begin{pgfscope}%
\pgfpathrectangle{\pgfqpoint{1.374500in}{0.082500in}}{\pgfqpoint{2.419000in}{2.419000in}}%
\pgfusepath{clip}%
\pgfsetbuttcap%
\pgfsetroundjoin%
\pgfsetlinewidth{1.505625pt}%
\definecolor{currentstroke}{rgb}{0.000000,0.000000,0.000000}%
\pgfsetstrokecolor{currentstroke}%
\pgfsetdash{}{0pt}%
\pgfusepath{stroke}%
\end{pgfscope}%
\begin{pgfscope}%
\pgfpathrectangle{\pgfqpoint{1.374500in}{0.082500in}}{\pgfqpoint{2.419000in}{2.419000in}}%
\pgfusepath{clip}%
\pgfsetbuttcap%
\pgfsetroundjoin%
\pgfsetlinewidth{1.505625pt}%
\definecolor{currentstroke}{rgb}{0.000000,0.000000,0.000000}%
\pgfsetstrokecolor{currentstroke}%
\pgfsetdash{}{0pt}%
\pgfusepath{stroke}%
\end{pgfscope}%
\begin{pgfscope}%
\pgfpathrectangle{\pgfqpoint{1.374500in}{0.082500in}}{\pgfqpoint{2.419000in}{2.419000in}}%
\pgfusepath{clip}%
\pgfsetbuttcap%
\pgfsetroundjoin%
\pgfsetlinewidth{1.505625pt}%
\definecolor{currentstroke}{rgb}{0.000000,0.000000,0.000000}%
\pgfsetstrokecolor{currentstroke}%
\pgfsetdash{}{0pt}%
\pgfusepath{stroke}%
\end{pgfscope}%
\begin{pgfscope}%
\pgfpathrectangle{\pgfqpoint{1.374500in}{0.082500in}}{\pgfqpoint{2.419000in}{2.419000in}}%
\pgfusepath{clip}%
\pgfsetbuttcap%
\pgfsetroundjoin%
\pgfsetlinewidth{1.505625pt}%
\definecolor{currentstroke}{rgb}{0.000000,0.000000,0.000000}%
\pgfsetstrokecolor{currentstroke}%
\pgfsetdash{}{0pt}%
\pgfusepath{stroke}%
\end{pgfscope}%
\begin{pgfscope}%
\pgfpathrectangle{\pgfqpoint{1.374500in}{0.082500in}}{\pgfqpoint{2.419000in}{2.419000in}}%
\pgfusepath{clip}%
\pgfsetbuttcap%
\pgfsetroundjoin%
\pgfsetlinewidth{1.505625pt}%
\definecolor{currentstroke}{rgb}{0.000000,0.000000,0.000000}%
\pgfsetstrokecolor{currentstroke}%
\pgfsetdash{}{0pt}%
\pgfusepath{stroke}%
\end{pgfscope}%
\begin{pgfscope}%
\pgfpathrectangle{\pgfqpoint{1.374500in}{0.082500in}}{\pgfqpoint{2.419000in}{2.419000in}}%
\pgfusepath{clip}%
\pgfsetbuttcap%
\pgfsetroundjoin%
\pgfsetlinewidth{1.505625pt}%
\definecolor{currentstroke}{rgb}{0.000000,0.000000,0.000000}%
\pgfsetstrokecolor{currentstroke}%
\pgfsetdash{}{0pt}%
\pgfusepath{stroke}%
\end{pgfscope}%
\begin{pgfscope}%
\pgfpathrectangle{\pgfqpoint{1.374500in}{0.082500in}}{\pgfqpoint{2.419000in}{2.419000in}}%
\pgfusepath{clip}%
\pgfsetbuttcap%
\pgfsetroundjoin%
\pgfsetlinewidth{1.505625pt}%
\definecolor{currentstroke}{rgb}{0.000000,0.000000,0.000000}%
\pgfsetstrokecolor{currentstroke}%
\pgfsetdash{}{0pt}%
\pgfusepath{stroke}%
\end{pgfscope}%
\begin{pgfscope}%
\pgfpathrectangle{\pgfqpoint{1.374500in}{0.082500in}}{\pgfqpoint{2.419000in}{2.419000in}}%
\pgfusepath{clip}%
\pgfsetbuttcap%
\pgfsetroundjoin%
\pgfsetlinewidth{1.505625pt}%
\definecolor{currentstroke}{rgb}{0.000000,0.000000,0.000000}%
\pgfsetstrokecolor{currentstroke}%
\pgfsetdash{}{0pt}%
\pgfusepath{stroke}%
\end{pgfscope}%
\begin{pgfscope}%
\pgfpathrectangle{\pgfqpoint{1.374500in}{0.082500in}}{\pgfqpoint{2.419000in}{2.419000in}}%
\pgfusepath{clip}%
\pgfsetbuttcap%
\pgfsetroundjoin%
\pgfsetlinewidth{1.505625pt}%
\definecolor{currentstroke}{rgb}{0.000000,0.000000,0.000000}%
\pgfsetstrokecolor{currentstroke}%
\pgfsetdash{}{0pt}%
\pgfusepath{stroke}%
\end{pgfscope}%
\begin{pgfscope}%
\pgfpathrectangle{\pgfqpoint{1.374500in}{0.082500in}}{\pgfqpoint{2.419000in}{2.419000in}}%
\pgfusepath{clip}%
\pgfsetbuttcap%
\pgfsetroundjoin%
\pgfsetlinewidth{1.505625pt}%
\definecolor{currentstroke}{rgb}{0.000000,0.000000,0.000000}%
\pgfsetstrokecolor{currentstroke}%
\pgfsetdash{}{0pt}%
\pgfusepath{stroke}%
\end{pgfscope}%
\begin{pgfscope}%
\pgfpathrectangle{\pgfqpoint{1.374500in}{0.082500in}}{\pgfqpoint{2.419000in}{2.419000in}}%
\pgfusepath{clip}%
\pgfsetbuttcap%
\pgfsetroundjoin%
\pgfsetlinewidth{1.505625pt}%
\definecolor{currentstroke}{rgb}{0.000000,0.000000,0.000000}%
\pgfsetstrokecolor{currentstroke}%
\pgfsetdash{}{0pt}%
\pgfusepath{stroke}%
\end{pgfscope}%
\begin{pgfscope}%
\pgfpathrectangle{\pgfqpoint{1.374500in}{0.082500in}}{\pgfqpoint{2.419000in}{2.419000in}}%
\pgfusepath{clip}%
\pgfsetbuttcap%
\pgfsetroundjoin%
\pgfsetlinewidth{1.505625pt}%
\definecolor{currentstroke}{rgb}{0.000000,0.000000,0.000000}%
\pgfsetstrokecolor{currentstroke}%
\pgfsetdash{}{0pt}%
\pgfusepath{stroke}%
\end{pgfscope}%
\begin{pgfscope}%
\pgfpathrectangle{\pgfqpoint{1.374500in}{0.082500in}}{\pgfqpoint{2.419000in}{2.419000in}}%
\pgfusepath{clip}%
\pgfsetbuttcap%
\pgfsetroundjoin%
\pgfsetlinewidth{1.505625pt}%
\definecolor{currentstroke}{rgb}{0.000000,0.000000,0.000000}%
\pgfsetstrokecolor{currentstroke}%
\pgfsetdash{}{0pt}%
\pgfusepath{stroke}%
\end{pgfscope}%
\begin{pgfscope}%
\pgfpathrectangle{\pgfqpoint{1.374500in}{0.082500in}}{\pgfqpoint{2.419000in}{2.419000in}}%
\pgfusepath{clip}%
\pgfsetbuttcap%
\pgfsetroundjoin%
\pgfsetlinewidth{1.505625pt}%
\definecolor{currentstroke}{rgb}{0.000000,0.000000,0.000000}%
\pgfsetstrokecolor{currentstroke}%
\pgfsetdash{}{0pt}%
\pgfusepath{stroke}%
\end{pgfscope}%
\begin{pgfscope}%
\pgfpathrectangle{\pgfqpoint{1.374500in}{0.082500in}}{\pgfqpoint{2.419000in}{2.419000in}}%
\pgfusepath{clip}%
\pgfsetbuttcap%
\pgfsetroundjoin%
\pgfsetlinewidth{1.505625pt}%
\definecolor{currentstroke}{rgb}{0.000000,0.000000,0.000000}%
\pgfsetstrokecolor{currentstroke}%
\pgfsetdash{}{0pt}%
\pgfusepath{stroke}%
\end{pgfscope}%
\begin{pgfscope}%
\pgfpathrectangle{\pgfqpoint{1.374500in}{0.082500in}}{\pgfqpoint{2.419000in}{2.419000in}}%
\pgfusepath{clip}%
\pgfsetbuttcap%
\pgfsetroundjoin%
\pgfsetlinewidth{1.505625pt}%
\definecolor{currentstroke}{rgb}{0.000000,0.000000,0.000000}%
\pgfsetstrokecolor{currentstroke}%
\pgfsetdash{}{0pt}%
\pgfpathmoveto{\pgfqpoint{2.995631in}{1.648516in}}%
\pgfpathlineto{\pgfqpoint{3.257524in}{1.362043in}}%
\pgfusepath{stroke}%
\end{pgfscope}%
\begin{pgfscope}%
\pgfpathrectangle{\pgfqpoint{1.374500in}{0.082500in}}{\pgfqpoint{2.419000in}{2.419000in}}%
\pgfusepath{clip}%
\pgfsetbuttcap%
\pgfsetroundjoin%
\pgfsetlinewidth{1.505625pt}%
\definecolor{currentstroke}{rgb}{0.000000,0.000000,0.000000}%
\pgfsetstrokecolor{currentstroke}%
\pgfsetdash{}{0pt}%
\pgfpathmoveto{\pgfqpoint{3.257524in}{1.362043in}}%
\pgfpathlineto{\pgfqpoint{3.803500in}{1.398440in}}%
\pgfusepath{stroke}%
\end{pgfscope}%
\begin{pgfscope}%
\pgfpathrectangle{\pgfqpoint{1.374500in}{0.082500in}}{\pgfqpoint{2.419000in}{2.419000in}}%
\pgfusepath{clip}%
\pgfsetbuttcap%
\pgfsetroundjoin%
\pgfsetlinewidth{1.505625pt}%
\definecolor{currentstroke}{rgb}{0.000000,0.000000,0.000000}%
\pgfsetstrokecolor{currentstroke}%
\pgfsetdash{}{0pt}%
\pgfusepath{stroke}%
\end{pgfscope}%
\begin{pgfscope}%
\pgfpathrectangle{\pgfqpoint{1.374500in}{0.082500in}}{\pgfqpoint{2.419000in}{2.419000in}}%
\pgfusepath{clip}%
\pgfsetbuttcap%
\pgfsetroundjoin%
\pgfsetlinewidth{1.505625pt}%
\definecolor{currentstroke}{rgb}{0.000000,0.000000,0.000000}%
\pgfsetstrokecolor{currentstroke}%
\pgfsetdash{}{0pt}%
\pgfusepath{stroke}%
\end{pgfscope}%
\begin{pgfscope}%
\pgfpathrectangle{\pgfqpoint{1.374500in}{0.082500in}}{\pgfqpoint{2.419000in}{2.419000in}}%
\pgfusepath{clip}%
\pgfsetbuttcap%
\pgfsetroundjoin%
\pgfsetlinewidth{1.505625pt}%
\definecolor{currentstroke}{rgb}{0.000000,0.000000,0.000000}%
\pgfsetstrokecolor{currentstroke}%
\pgfsetdash{}{0pt}%
\pgfusepath{stroke}%
\end{pgfscope}%
\begin{pgfscope}%
\pgfpathrectangle{\pgfqpoint{1.374500in}{0.082500in}}{\pgfqpoint{2.419000in}{2.419000in}}%
\pgfusepath{clip}%
\pgfsetbuttcap%
\pgfsetroundjoin%
\pgfsetlinewidth{1.505625pt}%
\definecolor{currentstroke}{rgb}{0.000000,0.000000,0.000000}%
\pgfsetstrokecolor{currentstroke}%
\pgfsetdash{}{0pt}%
\pgfusepath{stroke}%
\end{pgfscope}%
\begin{pgfscope}%
\pgfpathrectangle{\pgfqpoint{1.374500in}{0.082500in}}{\pgfqpoint{2.419000in}{2.419000in}}%
\pgfusepath{clip}%
\pgfsetbuttcap%
\pgfsetroundjoin%
\pgfsetlinewidth{1.505625pt}%
\definecolor{currentstroke}{rgb}{0.000000,0.000000,0.000000}%
\pgfsetstrokecolor{currentstroke}%
\pgfsetdash{}{0pt}%
\pgfusepath{stroke}%
\end{pgfscope}%
\begin{pgfscope}%
\pgfpathrectangle{\pgfqpoint{1.374500in}{0.082500in}}{\pgfqpoint{2.419000in}{2.419000in}}%
\pgfusepath{clip}%
\pgfsetbuttcap%
\pgfsetroundjoin%
\pgfsetlinewidth{1.505625pt}%
\definecolor{currentstroke}{rgb}{0.000000,0.000000,0.000000}%
\pgfsetstrokecolor{currentstroke}%
\pgfsetdash{}{0pt}%
\pgfpathmoveto{\pgfqpoint{1.966440in}{1.275973in}}%
\pgfpathlineto{\pgfqpoint{2.376943in}{1.608936in}}%
\pgfusepath{stroke}%
\end{pgfscope}%
\begin{pgfscope}%
\pgfpathrectangle{\pgfqpoint{1.374500in}{0.082500in}}{\pgfqpoint{2.419000in}{2.419000in}}%
\pgfusepath{clip}%
\pgfsetbuttcap%
\pgfsetroundjoin%
\pgfsetlinewidth{1.505625pt}%
\definecolor{currentstroke}{rgb}{0.000000,0.000000,0.000000}%
\pgfsetstrokecolor{currentstroke}%
\pgfsetdash{}{0pt}%
\pgfusepath{stroke}%
\end{pgfscope}%
\begin{pgfscope}%
\pgfpathrectangle{\pgfqpoint{1.374500in}{0.082500in}}{\pgfqpoint{2.419000in}{2.419000in}}%
\pgfusepath{clip}%
\pgfsetbuttcap%
\pgfsetroundjoin%
\pgfsetlinewidth{1.505625pt}%
\definecolor{currentstroke}{rgb}{0.000000,0.000000,0.000000}%
\pgfsetstrokecolor{currentstroke}%
\pgfsetdash{}{0pt}%
\pgfusepath{stroke}%
\end{pgfscope}%
\begin{pgfscope}%
\pgfpathrectangle{\pgfqpoint{1.374500in}{0.082500in}}{\pgfqpoint{2.419000in}{2.419000in}}%
\pgfusepath{clip}%
\pgfsetbuttcap%
\pgfsetroundjoin%
\pgfsetlinewidth{1.505625pt}%
\definecolor{currentstroke}{rgb}{0.000000,0.000000,0.000000}%
\pgfsetstrokecolor{currentstroke}%
\pgfsetdash{}{0pt}%
\pgfusepath{stroke}%
\end{pgfscope}%
\begin{pgfscope}%
\pgfpathrectangle{\pgfqpoint{1.374500in}{0.082500in}}{\pgfqpoint{2.419000in}{2.419000in}}%
\pgfusepath{clip}%
\pgfsetbuttcap%
\pgfsetroundjoin%
\pgfsetlinewidth{1.505625pt}%
\definecolor{currentstroke}{rgb}{0.000000,0.000000,0.000000}%
\pgfsetstrokecolor{currentstroke}%
\pgfsetdash{}{0pt}%
\pgfpathmoveto{\pgfqpoint{1.364500in}{1.235845in}}%
\pgfpathlineto{\pgfqpoint{1.966440in}{1.275973in}}%
\pgfusepath{stroke}%
\end{pgfscope}%
\begin{pgfscope}%
\pgfpathrectangle{\pgfqpoint{1.374500in}{0.082500in}}{\pgfqpoint{2.419000in}{2.419000in}}%
\pgfusepath{clip}%
\pgfsetbuttcap%
\pgfsetroundjoin%
\pgfsetlinewidth{1.505625pt}%
\definecolor{currentstroke}{rgb}{0.000000,0.000000,0.000000}%
\pgfsetstrokecolor{currentstroke}%
\pgfsetdash{}{0pt}%
\pgfusepath{stroke}%
\end{pgfscope}%
\begin{pgfscope}%
\pgfpathrectangle{\pgfqpoint{1.374500in}{0.082500in}}{\pgfqpoint{2.419000in}{2.419000in}}%
\pgfusepath{clip}%
\pgfsetbuttcap%
\pgfsetroundjoin%
\pgfsetlinewidth{1.505625pt}%
\definecolor{currentstroke}{rgb}{0.000000,0.000000,0.000000}%
\pgfsetstrokecolor{currentstroke}%
\pgfsetdash{}{0pt}%
\pgfusepath{stroke}%
\end{pgfscope}%
\begin{pgfscope}%
\pgfpathrectangle{\pgfqpoint{1.374500in}{0.082500in}}{\pgfqpoint{2.419000in}{2.419000in}}%
\pgfusepath{clip}%
\pgfsetbuttcap%
\pgfsetroundjoin%
\pgfsetlinewidth{1.505625pt}%
\definecolor{currentstroke}{rgb}{0.000000,0.000000,0.000000}%
\pgfsetstrokecolor{currentstroke}%
\pgfsetdash{}{0pt}%
\pgfusepath{stroke}%
\end{pgfscope}%
\begin{pgfscope}%
\pgfpathrectangle{\pgfqpoint{1.374500in}{0.082500in}}{\pgfqpoint{2.419000in}{2.419000in}}%
\pgfusepath{clip}%
\pgfsetbuttcap%
\pgfsetroundjoin%
\pgfsetlinewidth{1.505625pt}%
\definecolor{currentstroke}{rgb}{0.000000,0.000000,0.000000}%
\pgfsetstrokecolor{currentstroke}%
\pgfsetdash{}{0pt}%
\pgfusepath{stroke}%
\end{pgfscope}%
\begin{pgfscope}%
\pgfpathrectangle{\pgfqpoint{1.374500in}{0.082500in}}{\pgfqpoint{2.419000in}{2.419000in}}%
\pgfusepath{clip}%
\pgfsetbuttcap%
\pgfsetroundjoin%
\pgfsetlinewidth{1.505625pt}%
\definecolor{currentstroke}{rgb}{0.000000,0.000000,0.000000}%
\pgfsetstrokecolor{currentstroke}%
\pgfsetdash{}{0pt}%
\pgfpathmoveto{\pgfqpoint{1.966440in}{1.275973in}}%
\pgfpathlineto{\pgfqpoint{1.959535in}{1.515929in}}%
\pgfusepath{stroke}%
\end{pgfscope}%
\begin{pgfscope}%
\pgfpathrectangle{\pgfqpoint{1.374500in}{0.082500in}}{\pgfqpoint{2.419000in}{2.419000in}}%
\pgfusepath{clip}%
\pgfsetbuttcap%
\pgfsetroundjoin%
\pgfsetlinewidth{1.505625pt}%
\definecolor{currentstroke}{rgb}{0.000000,0.000000,0.000000}%
\pgfsetstrokecolor{currentstroke}%
\pgfsetdash{}{0pt}%
\pgfusepath{stroke}%
\end{pgfscope}%
\begin{pgfscope}%
\pgfpathrectangle{\pgfqpoint{1.374500in}{0.082500in}}{\pgfqpoint{2.419000in}{2.419000in}}%
\pgfusepath{clip}%
\pgfsetbuttcap%
\pgfsetroundjoin%
\pgfsetlinewidth{1.505625pt}%
\definecolor{currentstroke}{rgb}{0.000000,0.000000,0.000000}%
\pgfsetstrokecolor{currentstroke}%
\pgfsetdash{}{0pt}%
\pgfusepath{stroke}%
\end{pgfscope}%
\begin{pgfscope}%
\pgfpathrectangle{\pgfqpoint{1.374500in}{0.082500in}}{\pgfqpoint{2.419000in}{2.419000in}}%
\pgfusepath{clip}%
\pgfsetbuttcap%
\pgfsetroundjoin%
\pgfsetlinewidth{1.505625pt}%
\definecolor{currentstroke}{rgb}{0.000000,0.000000,0.000000}%
\pgfsetstrokecolor{currentstroke}%
\pgfsetdash{}{0pt}%
\pgfusepath{stroke}%
\end{pgfscope}%
\begin{pgfscope}%
\pgfpathrectangle{\pgfqpoint{1.374500in}{0.082500in}}{\pgfqpoint{2.419000in}{2.419000in}}%
\pgfusepath{clip}%
\pgfsetbuttcap%
\pgfsetroundjoin%
\pgfsetlinewidth{1.505625pt}%
\definecolor{currentstroke}{rgb}{0.000000,0.000000,0.000000}%
\pgfsetstrokecolor{currentstroke}%
\pgfsetdash{}{0pt}%
\pgfusepath{stroke}%
\end{pgfscope}%
\begin{pgfscope}%
\pgfpathrectangle{\pgfqpoint{1.374500in}{0.082500in}}{\pgfqpoint{2.419000in}{2.419000in}}%
\pgfusepath{clip}%
\pgfsetbuttcap%
\pgfsetroundjoin%
\pgfsetlinewidth{1.505625pt}%
\definecolor{currentstroke}{rgb}{0.000000,0.000000,0.000000}%
\pgfsetstrokecolor{currentstroke}%
\pgfsetdash{}{0pt}%
\pgfusepath{stroke}%
\end{pgfscope}%
\begin{pgfscope}%
\pgfpathrectangle{\pgfqpoint{1.374500in}{0.082500in}}{\pgfqpoint{2.419000in}{2.419000in}}%
\pgfusepath{clip}%
\pgfsetbuttcap%
\pgfsetroundjoin%
\pgfsetlinewidth{1.505625pt}%
\definecolor{currentstroke}{rgb}{0.000000,0.000000,0.000000}%
\pgfsetstrokecolor{currentstroke}%
\pgfsetdash{}{0pt}%
\pgfusepath{stroke}%
\end{pgfscope}%
\begin{pgfscope}%
\pgfpathrectangle{\pgfqpoint{1.374500in}{0.082500in}}{\pgfqpoint{2.419000in}{2.419000in}}%
\pgfusepath{clip}%
\pgfsetbuttcap%
\pgfsetroundjoin%
\pgfsetlinewidth{1.505625pt}%
\definecolor{currentstroke}{rgb}{0.000000,0.000000,0.000000}%
\pgfsetstrokecolor{currentstroke}%
\pgfsetdash{}{0pt}%
\pgfusepath{stroke}%
\end{pgfscope}%
\begin{pgfscope}%
\pgfpathrectangle{\pgfqpoint{1.374500in}{0.082500in}}{\pgfqpoint{2.419000in}{2.419000in}}%
\pgfusepath{clip}%
\pgfsetbuttcap%
\pgfsetroundjoin%
\pgfsetlinewidth{1.505625pt}%
\definecolor{currentstroke}{rgb}{0.000000,0.000000,0.000000}%
\pgfsetstrokecolor{currentstroke}%
\pgfsetdash{}{0pt}%
\pgfusepath{stroke}%
\end{pgfscope}%
\begin{pgfscope}%
\pgfpathrectangle{\pgfqpoint{1.374500in}{0.082500in}}{\pgfqpoint{2.419000in}{2.419000in}}%
\pgfusepath{clip}%
\pgfsetbuttcap%
\pgfsetroundjoin%
\pgfsetlinewidth{1.505625pt}%
\definecolor{currentstroke}{rgb}{0.000000,0.000000,0.000000}%
\pgfsetstrokecolor{currentstroke}%
\pgfsetdash{}{0pt}%
\pgfusepath{stroke}%
\end{pgfscope}%
\begin{pgfscope}%
\pgfpathrectangle{\pgfqpoint{1.374500in}{0.082500in}}{\pgfqpoint{2.419000in}{2.419000in}}%
\pgfusepath{clip}%
\pgfsetbuttcap%
\pgfsetroundjoin%
\pgfsetlinewidth{1.505625pt}%
\definecolor{currentstroke}{rgb}{0.000000,0.000000,0.000000}%
\pgfsetstrokecolor{currentstroke}%
\pgfsetdash{}{0pt}%
\pgfusepath{stroke}%
\end{pgfscope}%
\begin{pgfscope}%
\pgfpathrectangle{\pgfqpoint{1.374500in}{0.082500in}}{\pgfqpoint{2.419000in}{2.419000in}}%
\pgfusepath{clip}%
\pgfsetbuttcap%
\pgfsetroundjoin%
\pgfsetlinewidth{1.505625pt}%
\definecolor{currentstroke}{rgb}{0.000000,0.000000,0.000000}%
\pgfsetstrokecolor{currentstroke}%
\pgfsetdash{}{0pt}%
\pgfusepath{stroke}%
\end{pgfscope}%
\begin{pgfscope}%
\pgfpathrectangle{\pgfqpoint{1.374500in}{0.082500in}}{\pgfqpoint{2.419000in}{2.419000in}}%
\pgfusepath{clip}%
\pgfsetbuttcap%
\pgfsetroundjoin%
\pgfsetlinewidth{1.505625pt}%
\definecolor{currentstroke}{rgb}{0.000000,0.000000,0.000000}%
\pgfsetstrokecolor{currentstroke}%
\pgfsetdash{}{0pt}%
\pgfusepath{stroke}%
\end{pgfscope}%
\begin{pgfscope}%
\pgfpathrectangle{\pgfqpoint{1.374500in}{0.082500in}}{\pgfqpoint{2.419000in}{2.419000in}}%
\pgfusepath{clip}%
\pgfsetbuttcap%
\pgfsetroundjoin%
\pgfsetlinewidth{1.505625pt}%
\definecolor{currentstroke}{rgb}{0.000000,0.000000,0.000000}%
\pgfsetstrokecolor{currentstroke}%
\pgfsetdash{}{0pt}%
\pgfusepath{stroke}%
\end{pgfscope}%
\begin{pgfscope}%
\pgfpathrectangle{\pgfqpoint{1.374500in}{0.082500in}}{\pgfqpoint{2.419000in}{2.419000in}}%
\pgfusepath{clip}%
\pgfsetbuttcap%
\pgfsetroundjoin%
\pgfsetlinewidth{1.505625pt}%
\definecolor{currentstroke}{rgb}{0.000000,0.000000,0.000000}%
\pgfsetstrokecolor{currentstroke}%
\pgfsetdash{}{0pt}%
\pgfusepath{stroke}%
\end{pgfscope}%
\begin{pgfscope}%
\pgfpathrectangle{\pgfqpoint{1.374500in}{0.082500in}}{\pgfqpoint{2.419000in}{2.419000in}}%
\pgfusepath{clip}%
\pgfsetbuttcap%
\pgfsetroundjoin%
\pgfsetlinewidth{1.505625pt}%
\definecolor{currentstroke}{rgb}{0.000000,0.000000,0.000000}%
\pgfsetstrokecolor{currentstroke}%
\pgfsetdash{}{0pt}%
\pgfusepath{stroke}%
\end{pgfscope}%
\begin{pgfscope}%
\pgfpathrectangle{\pgfqpoint{1.374500in}{0.082500in}}{\pgfqpoint{2.419000in}{2.419000in}}%
\pgfusepath{clip}%
\pgfsetbuttcap%
\pgfsetroundjoin%
\pgfsetlinewidth{1.505625pt}%
\definecolor{currentstroke}{rgb}{0.000000,0.000000,0.000000}%
\pgfsetstrokecolor{currentstroke}%
\pgfsetdash{}{0pt}%
\pgfpathmoveto{\pgfqpoint{2.874202in}{1.005508in}}%
\pgfpathlineto{\pgfqpoint{3.257524in}{1.362043in}}%
\pgfusepath{stroke}%
\end{pgfscope}%
\begin{pgfscope}%
\pgfpathrectangle{\pgfqpoint{1.374500in}{0.082500in}}{\pgfqpoint{2.419000in}{2.419000in}}%
\pgfusepath{clip}%
\pgfsetbuttcap%
\pgfsetroundjoin%
\pgfsetlinewidth{1.505625pt}%
\definecolor{currentstroke}{rgb}{0.000000,0.000000,0.000000}%
\pgfsetstrokecolor{currentstroke}%
\pgfsetdash{}{0pt}%
\pgfusepath{stroke}%
\end{pgfscope}%
\begin{pgfscope}%
\pgfpathrectangle{\pgfqpoint{1.374500in}{0.082500in}}{\pgfqpoint{2.419000in}{2.419000in}}%
\pgfusepath{clip}%
\pgfsetbuttcap%
\pgfsetroundjoin%
\pgfsetlinewidth{1.505625pt}%
\definecolor{currentstroke}{rgb}{0.000000,0.000000,0.000000}%
\pgfsetstrokecolor{currentstroke}%
\pgfsetdash{}{0pt}%
\pgfusepath{stroke}%
\end{pgfscope}%
\begin{pgfscope}%
\pgfpathrectangle{\pgfqpoint{1.374500in}{0.082500in}}{\pgfqpoint{2.419000in}{2.419000in}}%
\pgfusepath{clip}%
\pgfsetbuttcap%
\pgfsetroundjoin%
\pgfsetlinewidth{1.505625pt}%
\definecolor{currentstroke}{rgb}{0.000000,0.000000,0.000000}%
\pgfsetstrokecolor{currentstroke}%
\pgfsetdash{}{0pt}%
\pgfpathmoveto{\pgfqpoint{1.966440in}{1.275973in}}%
\pgfpathlineto{\pgfqpoint{2.199403in}{0.958548in}}%
\pgfusepath{stroke}%
\end{pgfscope}%
\begin{pgfscope}%
\pgfpathrectangle{\pgfqpoint{1.374500in}{0.082500in}}{\pgfqpoint{2.419000in}{2.419000in}}%
\pgfusepath{clip}%
\pgfsetbuttcap%
\pgfsetroundjoin%
\pgfsetlinewidth{1.505625pt}%
\definecolor{currentstroke}{rgb}{0.000000,0.000000,0.000000}%
\pgfsetstrokecolor{currentstroke}%
\pgfsetdash{}{0pt}%
\pgfpathmoveto{\pgfqpoint{2.199403in}{0.958548in}}%
\pgfpathlineto{\pgfqpoint{2.874202in}{1.005508in}}%
\pgfusepath{stroke}%
\end{pgfscope}%
\begin{pgfscope}%
\pgfpathrectangle{\pgfqpoint{1.374500in}{0.082500in}}{\pgfqpoint{2.419000in}{2.419000in}}%
\pgfusepath{clip}%
\pgfsetbuttcap%
\pgfsetroundjoin%
\pgfsetlinewidth{1.505625pt}%
\definecolor{currentstroke}{rgb}{0.000000,0.000000,0.000000}%
\pgfsetstrokecolor{currentstroke}%
\pgfsetdash{}{0pt}%
\pgfusepath{stroke}%
\end{pgfscope}%
\begin{pgfscope}%
\pgfpathrectangle{\pgfqpoint{1.374500in}{0.082500in}}{\pgfqpoint{2.419000in}{2.419000in}}%
\pgfusepath{clip}%
\pgfsetbuttcap%
\pgfsetroundjoin%
\pgfsetlinewidth{1.505625pt}%
\definecolor{currentstroke}{rgb}{0.000000,0.000000,0.000000}%
\pgfsetstrokecolor{currentstroke}%
\pgfsetdash{}{0pt}%
\pgfusepath{stroke}%
\end{pgfscope}%
\begin{pgfscope}%
\pgfpathrectangle{\pgfqpoint{1.374500in}{0.082500in}}{\pgfqpoint{2.419000in}{2.419000in}}%
\pgfusepath{clip}%
\pgfsetbuttcap%
\pgfsetroundjoin%
\pgfsetlinewidth{1.505625pt}%
\definecolor{currentstroke}{rgb}{0.000000,0.000000,0.000000}%
\pgfsetstrokecolor{currentstroke}%
\pgfsetdash{}{0pt}%
\pgfusepath{stroke}%
\end{pgfscope}%
\begin{pgfscope}%
\pgfpathrectangle{\pgfqpoint{1.374500in}{0.082500in}}{\pgfqpoint{2.419000in}{2.419000in}}%
\pgfusepath{clip}%
\pgfsetbuttcap%
\pgfsetroundjoin%
\pgfsetlinewidth{1.505625pt}%
\definecolor{currentstroke}{rgb}{0.000000,0.000000,0.000000}%
\pgfsetstrokecolor{currentstroke}%
\pgfsetdash{}{0pt}%
\pgfpathmoveto{\pgfqpoint{2.874202in}{1.005508in}}%
\pgfpathlineto{\pgfqpoint{2.877035in}{1.251125in}}%
\pgfusepath{stroke}%
\end{pgfscope}%
\begin{pgfscope}%
\pgfpathrectangle{\pgfqpoint{1.374500in}{0.082500in}}{\pgfqpoint{2.419000in}{2.419000in}}%
\pgfusepath{clip}%
\pgfsetbuttcap%
\pgfsetroundjoin%
\pgfsetlinewidth{1.505625pt}%
\definecolor{currentstroke}{rgb}{0.000000,0.000000,0.000000}%
\pgfsetstrokecolor{currentstroke}%
\pgfsetdash{}{0pt}%
\pgfusepath{stroke}%
\end{pgfscope}%
\begin{pgfscope}%
\pgfpathrectangle{\pgfqpoint{1.374500in}{0.082500in}}{\pgfqpoint{2.419000in}{2.419000in}}%
\pgfusepath{clip}%
\pgfsetbuttcap%
\pgfsetroundjoin%
\pgfsetlinewidth{1.505625pt}%
\definecolor{currentstroke}{rgb}{0.000000,0.000000,0.000000}%
\pgfsetstrokecolor{currentstroke}%
\pgfsetdash{}{0pt}%
\pgfusepath{stroke}%
\end{pgfscope}%
\begin{pgfscope}%
\pgfpathrectangle{\pgfqpoint{1.374500in}{0.082500in}}{\pgfqpoint{2.419000in}{2.419000in}}%
\pgfusepath{clip}%
\pgfsetbuttcap%
\pgfsetroundjoin%
\pgfsetlinewidth{1.505625pt}%
\definecolor{currentstroke}{rgb}{0.000000,0.000000,0.000000}%
\pgfsetstrokecolor{currentstroke}%
\pgfsetdash{}{0pt}%
\pgfusepath{stroke}%
\end{pgfscope}%
\begin{pgfscope}%
\pgfpathrectangle{\pgfqpoint{1.374500in}{0.082500in}}{\pgfqpoint{2.419000in}{2.419000in}}%
\pgfusepath{clip}%
\pgfsetbuttcap%
\pgfsetroundjoin%
\pgfsetlinewidth{1.505625pt}%
\definecolor{currentstroke}{rgb}{0.000000,0.000000,0.000000}%
\pgfsetstrokecolor{currentstroke}%
\pgfsetdash{}{0pt}%
\pgfusepath{stroke}%
\end{pgfscope}%
\begin{pgfscope}%
\pgfpathrectangle{\pgfqpoint{1.374500in}{0.082500in}}{\pgfqpoint{2.419000in}{2.419000in}}%
\pgfusepath{clip}%
\pgfsetbuttcap%
\pgfsetroundjoin%
\pgfsetlinewidth{1.505625pt}%
\definecolor{currentstroke}{rgb}{0.000000,0.000000,0.000000}%
\pgfsetstrokecolor{currentstroke}%
\pgfsetdash{}{0pt}%
\pgfusepath{stroke}%
\end{pgfscope}%
\begin{pgfscope}%
\pgfpathrectangle{\pgfqpoint{1.374500in}{0.082500in}}{\pgfqpoint{2.419000in}{2.419000in}}%
\pgfusepath{clip}%
\pgfsetbuttcap%
\pgfsetroundjoin%
\pgfsetlinewidth{1.505625pt}%
\definecolor{currentstroke}{rgb}{0.000000,0.000000,0.000000}%
\pgfsetstrokecolor{currentstroke}%
\pgfsetdash{}{0pt}%
\pgfusepath{stroke}%
\end{pgfscope}%
\begin{pgfscope}%
\pgfpathrectangle{\pgfqpoint{1.374500in}{0.082500in}}{\pgfqpoint{2.419000in}{2.419000in}}%
\pgfusepath{clip}%
\pgfsetbuttcap%
\pgfsetroundjoin%
\pgfsetlinewidth{1.505625pt}%
\definecolor{currentstroke}{rgb}{0.000000,0.000000,0.000000}%
\pgfsetstrokecolor{currentstroke}%
\pgfsetdash{}{0pt}%
\pgfusepath{stroke}%
\end{pgfscope}%
\begin{pgfscope}%
\pgfpathrectangle{\pgfqpoint{1.374500in}{0.082500in}}{\pgfqpoint{2.419000in}{2.419000in}}%
\pgfusepath{clip}%
\pgfsetbuttcap%
\pgfsetroundjoin%
\pgfsetlinewidth{1.505625pt}%
\definecolor{currentstroke}{rgb}{0.000000,0.000000,0.000000}%
\pgfsetstrokecolor{currentstroke}%
\pgfsetdash{}{0pt}%
\pgfusepath{stroke}%
\end{pgfscope}%
\begin{pgfscope}%
\pgfpathrectangle{\pgfqpoint{1.374500in}{0.082500in}}{\pgfqpoint{2.419000in}{2.419000in}}%
\pgfusepath{clip}%
\pgfsetbuttcap%
\pgfsetroundjoin%
\pgfsetlinewidth{1.505625pt}%
\definecolor{currentstroke}{rgb}{0.000000,0.000000,0.000000}%
\pgfsetstrokecolor{currentstroke}%
\pgfsetdash{}{0pt}%
\pgfusepath{stroke}%
\end{pgfscope}%
\begin{pgfscope}%
\pgfpathrectangle{\pgfqpoint{1.374500in}{0.082500in}}{\pgfqpoint{2.419000in}{2.419000in}}%
\pgfusepath{clip}%
\pgfsetbuttcap%
\pgfsetroundjoin%
\pgfsetlinewidth{1.505625pt}%
\definecolor{currentstroke}{rgb}{0.000000,0.000000,0.000000}%
\pgfsetstrokecolor{currentstroke}%
\pgfsetdash{}{0pt}%
\pgfusepath{stroke}%
\end{pgfscope}%
\begin{pgfscope}%
\pgfpathrectangle{\pgfqpoint{1.374500in}{0.082500in}}{\pgfqpoint{2.419000in}{2.419000in}}%
\pgfusepath{clip}%
\pgfsetbuttcap%
\pgfsetroundjoin%
\pgfsetlinewidth{1.505625pt}%
\definecolor{currentstroke}{rgb}{0.000000,0.000000,0.000000}%
\pgfsetstrokecolor{currentstroke}%
\pgfsetdash{}{0pt}%
\pgfusepath{stroke}%
\end{pgfscope}%
\begin{pgfscope}%
\pgfpathrectangle{\pgfqpoint{1.374500in}{0.082500in}}{\pgfqpoint{2.419000in}{2.419000in}}%
\pgfusepath{clip}%
\pgfsetbuttcap%
\pgfsetroundjoin%
\pgfsetlinewidth{1.505625pt}%
\definecolor{currentstroke}{rgb}{0.000000,0.000000,0.000000}%
\pgfsetstrokecolor{currentstroke}%
\pgfsetdash{}{0pt}%
\pgfusepath{stroke}%
\end{pgfscope}%
\begin{pgfscope}%
\pgfpathrectangle{\pgfqpoint{1.374500in}{0.082500in}}{\pgfqpoint{2.419000in}{2.419000in}}%
\pgfusepath{clip}%
\pgfsetbuttcap%
\pgfsetroundjoin%
\pgfsetlinewidth{1.505625pt}%
\definecolor{currentstroke}{rgb}{0.000000,0.000000,0.000000}%
\pgfsetstrokecolor{currentstroke}%
\pgfsetdash{}{0pt}%
\pgfusepath{stroke}%
\end{pgfscope}%
\begin{pgfscope}%
\pgfpathrectangle{\pgfqpoint{1.374500in}{0.082500in}}{\pgfqpoint{2.419000in}{2.419000in}}%
\pgfusepath{clip}%
\pgfsetbuttcap%
\pgfsetroundjoin%
\pgfsetlinewidth{1.505625pt}%
\definecolor{currentstroke}{rgb}{0.000000,0.000000,0.000000}%
\pgfsetstrokecolor{currentstroke}%
\pgfsetdash{}{0pt}%
\pgfusepath{stroke}%
\end{pgfscope}%
\begin{pgfscope}%
\pgfpathrectangle{\pgfqpoint{1.374500in}{0.082500in}}{\pgfqpoint{2.419000in}{2.419000in}}%
\pgfusepath{clip}%
\pgfsetbuttcap%
\pgfsetroundjoin%
\pgfsetlinewidth{1.505625pt}%
\definecolor{currentstroke}{rgb}{0.000000,0.000000,0.000000}%
\pgfsetstrokecolor{currentstroke}%
\pgfsetdash{}{0pt}%
\pgfusepath{stroke}%
\end{pgfscope}%
\begin{pgfscope}%
\pgfpathrectangle{\pgfqpoint{1.374500in}{0.082500in}}{\pgfqpoint{2.419000in}{2.419000in}}%
\pgfusepath{clip}%
\pgfsetbuttcap%
\pgfsetroundjoin%
\pgfsetlinewidth{1.505625pt}%
\definecolor{currentstroke}{rgb}{0.000000,0.000000,0.000000}%
\pgfsetstrokecolor{currentstroke}%
\pgfsetdash{}{0pt}%
\pgfusepath{stroke}%
\end{pgfscope}%
\begin{pgfscope}%
\pgfpathrectangle{\pgfqpoint{1.374500in}{0.082500in}}{\pgfqpoint{2.419000in}{2.419000in}}%
\pgfusepath{clip}%
\pgfsetbuttcap%
\pgfsetroundjoin%
\pgfsetlinewidth{1.505625pt}%
\definecolor{currentstroke}{rgb}{0.000000,0.000000,0.000000}%
\pgfsetstrokecolor{currentstroke}%
\pgfsetdash{}{0pt}%
\pgfusepath{stroke}%
\end{pgfscope}%
\begin{pgfscope}%
\pgfpathrectangle{\pgfqpoint{1.374500in}{0.082500in}}{\pgfqpoint{2.419000in}{2.419000in}}%
\pgfusepath{clip}%
\pgfsetbuttcap%
\pgfsetroundjoin%
\pgfsetlinewidth{1.505625pt}%
\definecolor{currentstroke}{rgb}{0.000000,0.000000,0.000000}%
\pgfsetstrokecolor{currentstroke}%
\pgfsetdash{}{0pt}%
\pgfusepath{stroke}%
\end{pgfscope}%
\begin{pgfscope}%
\pgfpathrectangle{\pgfqpoint{1.374500in}{0.082500in}}{\pgfqpoint{2.419000in}{2.419000in}}%
\pgfusepath{clip}%
\pgfsetbuttcap%
\pgfsetroundjoin%
\pgfsetlinewidth{1.505625pt}%
\definecolor{currentstroke}{rgb}{0.000000,0.000000,0.000000}%
\pgfsetstrokecolor{currentstroke}%
\pgfsetdash{}{0pt}%
\pgfpathmoveto{\pgfqpoint{2.874202in}{1.005508in}}%
\pgfpathlineto{\pgfqpoint{3.155150in}{0.664575in}}%
\pgfusepath{stroke}%
\end{pgfscope}%
\begin{pgfscope}%
\pgfpathrectangle{\pgfqpoint{1.374500in}{0.082500in}}{\pgfqpoint{2.419000in}{2.419000in}}%
\pgfusepath{clip}%
\pgfsetbuttcap%
\pgfsetroundjoin%
\pgfsetlinewidth{1.505625pt}%
\definecolor{currentstroke}{rgb}{0.000000,0.000000,0.000000}%
\pgfsetstrokecolor{currentstroke}%
\pgfsetdash{}{0pt}%
\pgfpathmoveto{\pgfqpoint{3.155150in}{0.664575in}}%
\pgfpathlineto{\pgfqpoint{3.803500in}{0.711767in}}%
\pgfusepath{stroke}%
\end{pgfscope}%
\begin{pgfscope}%
\pgfpathrectangle{\pgfqpoint{1.374500in}{0.082500in}}{\pgfqpoint{2.419000in}{2.419000in}}%
\pgfusepath{clip}%
\pgfsetbuttcap%
\pgfsetroundjoin%
\pgfsetlinewidth{1.505625pt}%
\definecolor{currentstroke}{rgb}{0.000000,0.000000,0.000000}%
\pgfsetstrokecolor{currentstroke}%
\pgfsetdash{}{0pt}%
\pgfusepath{stroke}%
\end{pgfscope}%
\begin{pgfscope}%
\pgfpathrectangle{\pgfqpoint{1.374500in}{0.082500in}}{\pgfqpoint{2.419000in}{2.419000in}}%
\pgfusepath{clip}%
\pgfsetbuttcap%
\pgfsetroundjoin%
\pgfsetlinewidth{1.505625pt}%
\definecolor{currentstroke}{rgb}{0.000000,0.000000,0.000000}%
\pgfsetstrokecolor{currentstroke}%
\pgfsetdash{}{0pt}%
\pgfusepath{stroke}%
\end{pgfscope}%
\begin{pgfscope}%
\pgfpathrectangle{\pgfqpoint{1.374500in}{0.082500in}}{\pgfqpoint{2.419000in}{2.419000in}}%
\pgfusepath{clip}%
\pgfsetbuttcap%
\pgfsetroundjoin%
\pgfsetlinewidth{1.505625pt}%
\definecolor{currentstroke}{rgb}{0.000000,0.000000,0.000000}%
\pgfsetstrokecolor{currentstroke}%
\pgfsetdash{}{0pt}%
\pgfusepath{stroke}%
\end{pgfscope}%
\begin{pgfscope}%
\pgfpathrectangle{\pgfqpoint{1.374500in}{0.082500in}}{\pgfqpoint{2.419000in}{2.419000in}}%
\pgfusepath{clip}%
\pgfsetbuttcap%
\pgfsetroundjoin%
\pgfsetlinewidth{1.505625pt}%
\definecolor{currentstroke}{rgb}{0.000000,0.000000,0.000000}%
\pgfsetstrokecolor{currentstroke}%
\pgfsetdash{}{0pt}%
\pgfusepath{stroke}%
\end{pgfscope}%
\begin{pgfscope}%
\pgfpathrectangle{\pgfqpoint{1.374500in}{0.082500in}}{\pgfqpoint{2.419000in}{2.419000in}}%
\pgfusepath{clip}%
\pgfsetbuttcap%
\pgfsetroundjoin%
\pgfsetlinewidth{1.505625pt}%
\definecolor{currentstroke}{rgb}{0.000000,0.000000,0.000000}%
\pgfsetstrokecolor{currentstroke}%
\pgfsetdash{}{0pt}%
\pgfusepath{stroke}%
\end{pgfscope}%
\begin{pgfscope}%
\pgfpathrectangle{\pgfqpoint{1.374500in}{0.082500in}}{\pgfqpoint{2.419000in}{2.419000in}}%
\pgfusepath{clip}%
\pgfsetbuttcap%
\pgfsetroundjoin%
\pgfsetlinewidth{1.505625pt}%
\definecolor{currentstroke}{rgb}{0.000000,0.000000,0.000000}%
\pgfsetstrokecolor{currentstroke}%
\pgfsetdash{}{0pt}%
\pgfusepath{stroke}%
\end{pgfscope}%
\begin{pgfscope}%
\pgfpathrectangle{\pgfqpoint{1.374500in}{0.082500in}}{\pgfqpoint{2.419000in}{2.419000in}}%
\pgfusepath{clip}%
\pgfsetbuttcap%
\pgfsetroundjoin%
\pgfsetlinewidth{1.505625pt}%
\definecolor{currentstroke}{rgb}{0.000000,0.000000,0.000000}%
\pgfsetstrokecolor{currentstroke}%
\pgfsetdash{}{0pt}%
\pgfpathmoveto{\pgfqpoint{1.741405in}{0.561671in}}%
\pgfpathlineto{\pgfqpoint{2.199403in}{0.958548in}}%
\pgfusepath{stroke}%
\end{pgfscope}%
\begin{pgfscope}%
\pgfpathrectangle{\pgfqpoint{1.374500in}{0.082500in}}{\pgfqpoint{2.419000in}{2.419000in}}%
\pgfusepath{clip}%
\pgfsetbuttcap%
\pgfsetroundjoin%
\pgfsetlinewidth{1.505625pt}%
\definecolor{currentstroke}{rgb}{0.000000,0.000000,0.000000}%
\pgfsetstrokecolor{currentstroke}%
\pgfsetdash{}{0pt}%
\pgfusepath{stroke}%
\end{pgfscope}%
\begin{pgfscope}%
\pgfpathrectangle{\pgfqpoint{1.374500in}{0.082500in}}{\pgfqpoint{2.419000in}{2.419000in}}%
\pgfusepath{clip}%
\pgfsetbuttcap%
\pgfsetroundjoin%
\pgfsetlinewidth{1.505625pt}%
\definecolor{currentstroke}{rgb}{0.000000,0.000000,0.000000}%
\pgfsetstrokecolor{currentstroke}%
\pgfsetdash{}{0pt}%
\pgfusepath{stroke}%
\end{pgfscope}%
\begin{pgfscope}%
\pgfpathrectangle{\pgfqpoint{1.374500in}{0.082500in}}{\pgfqpoint{2.419000in}{2.419000in}}%
\pgfusepath{clip}%
\pgfsetbuttcap%
\pgfsetroundjoin%
\pgfsetlinewidth{1.505625pt}%
\definecolor{currentstroke}{rgb}{0.000000,0.000000,0.000000}%
\pgfsetstrokecolor{currentstroke}%
\pgfsetdash{}{0pt}%
\pgfpathmoveto{\pgfqpoint{1.364500in}{0.534237in}}%
\pgfpathlineto{\pgfqpoint{1.741405in}{0.561671in}}%
\pgfusepath{stroke}%
\end{pgfscope}%
\begin{pgfscope}%
\pgfpathrectangle{\pgfqpoint{1.374500in}{0.082500in}}{\pgfqpoint{2.419000in}{2.419000in}}%
\pgfusepath{clip}%
\pgfsetbuttcap%
\pgfsetroundjoin%
\pgfsetlinewidth{1.505625pt}%
\definecolor{currentstroke}{rgb}{0.000000,0.000000,0.000000}%
\pgfsetstrokecolor{currentstroke}%
\pgfsetdash{}{0pt}%
\pgfusepath{stroke}%
\end{pgfscope}%
\begin{pgfscope}%
\pgfpathrectangle{\pgfqpoint{1.374500in}{0.082500in}}{\pgfqpoint{2.419000in}{2.419000in}}%
\pgfusepath{clip}%
\pgfsetbuttcap%
\pgfsetroundjoin%
\pgfsetlinewidth{1.505625pt}%
\definecolor{currentstroke}{rgb}{0.000000,0.000000,0.000000}%
\pgfsetstrokecolor{currentstroke}%
\pgfsetdash{}{0pt}%
\pgfusepath{stroke}%
\end{pgfscope}%
\begin{pgfscope}%
\pgfpathrectangle{\pgfqpoint{1.374500in}{0.082500in}}{\pgfqpoint{2.419000in}{2.419000in}}%
\pgfusepath{clip}%
\pgfsetbuttcap%
\pgfsetroundjoin%
\pgfsetlinewidth{1.505625pt}%
\definecolor{currentstroke}{rgb}{0.000000,0.000000,0.000000}%
\pgfsetstrokecolor{currentstroke}%
\pgfsetdash{}{0pt}%
\pgfusepath{stroke}%
\end{pgfscope}%
\begin{pgfscope}%
\pgfpathrectangle{\pgfqpoint{1.374500in}{0.082500in}}{\pgfqpoint{2.419000in}{2.419000in}}%
\pgfusepath{clip}%
\pgfsetbuttcap%
\pgfsetroundjoin%
\pgfsetlinewidth{1.505625pt}%
\definecolor{currentstroke}{rgb}{0.000000,0.000000,0.000000}%
\pgfsetstrokecolor{currentstroke}%
\pgfsetdash{}{0pt}%
\pgfusepath{stroke}%
\end{pgfscope}%
\begin{pgfscope}%
\pgfpathrectangle{\pgfqpoint{1.374500in}{0.082500in}}{\pgfqpoint{2.419000in}{2.419000in}}%
\pgfusepath{clip}%
\pgfsetbuttcap%
\pgfsetroundjoin%
\pgfsetlinewidth{1.505625pt}%
\definecolor{currentstroke}{rgb}{0.000000,0.000000,0.000000}%
\pgfsetstrokecolor{currentstroke}%
\pgfsetdash{}{0pt}%
\pgfpathmoveto{\pgfqpoint{1.741405in}{0.561671in}}%
\pgfpathlineto{\pgfqpoint{1.731227in}{0.816143in}}%
\pgfusepath{stroke}%
\end{pgfscope}%
\begin{pgfscope}%
\pgfpathrectangle{\pgfqpoint{1.374500in}{0.082500in}}{\pgfqpoint{2.419000in}{2.419000in}}%
\pgfusepath{clip}%
\pgfsetbuttcap%
\pgfsetroundjoin%
\pgfsetlinewidth{1.505625pt}%
\definecolor{currentstroke}{rgb}{0.000000,0.000000,0.000000}%
\pgfsetstrokecolor{currentstroke}%
\pgfsetdash{}{0pt}%
\pgfusepath{stroke}%
\end{pgfscope}%
\begin{pgfscope}%
\pgfpathrectangle{\pgfqpoint{1.374500in}{0.082500in}}{\pgfqpoint{2.419000in}{2.419000in}}%
\pgfusepath{clip}%
\pgfsetbuttcap%
\pgfsetroundjoin%
\pgfsetlinewidth{1.505625pt}%
\definecolor{currentstroke}{rgb}{0.000000,0.000000,0.000000}%
\pgfsetstrokecolor{currentstroke}%
\pgfsetdash{}{0pt}%
\pgfusepath{stroke}%
\end{pgfscope}%
\begin{pgfscope}%
\pgfpathrectangle{\pgfqpoint{1.374500in}{0.082500in}}{\pgfqpoint{2.419000in}{2.419000in}}%
\pgfusepath{clip}%
\pgfsetbuttcap%
\pgfsetroundjoin%
\pgfsetlinewidth{1.505625pt}%
\definecolor{currentstroke}{rgb}{0.000000,0.000000,0.000000}%
\pgfsetstrokecolor{currentstroke}%
\pgfsetdash{}{0pt}%
\pgfusepath{stroke}%
\end{pgfscope}%
\begin{pgfscope}%
\pgfpathrectangle{\pgfqpoint{1.374500in}{0.082500in}}{\pgfqpoint{2.419000in}{2.419000in}}%
\pgfusepath{clip}%
\pgfsetbuttcap%
\pgfsetroundjoin%
\pgfsetlinewidth{1.505625pt}%
\definecolor{currentstroke}{rgb}{0.000000,0.000000,0.000000}%
\pgfsetstrokecolor{currentstroke}%
\pgfsetdash{}{0pt}%
\pgfusepath{stroke}%
\end{pgfscope}%
\begin{pgfscope}%
\pgfpathrectangle{\pgfqpoint{1.374500in}{0.082500in}}{\pgfqpoint{2.419000in}{2.419000in}}%
\pgfusepath{clip}%
\pgfsetbuttcap%
\pgfsetroundjoin%
\pgfsetlinewidth{1.505625pt}%
\definecolor{currentstroke}{rgb}{0.000000,0.000000,0.000000}%
\pgfsetstrokecolor{currentstroke}%
\pgfsetdash{}{0pt}%
\pgfusepath{stroke}%
\end{pgfscope}%
\begin{pgfscope}%
\pgfpathrectangle{\pgfqpoint{1.374500in}{0.082500in}}{\pgfqpoint{2.419000in}{2.419000in}}%
\pgfusepath{clip}%
\pgfsetbuttcap%
\pgfsetroundjoin%
\pgfsetlinewidth{1.505625pt}%
\definecolor{currentstroke}{rgb}{0.000000,0.000000,0.000000}%
\pgfsetstrokecolor{currentstroke}%
\pgfsetdash{}{0pt}%
\pgfusepath{stroke}%
\end{pgfscope}%
\begin{pgfscope}%
\pgfpathrectangle{\pgfqpoint{1.374500in}{0.082500in}}{\pgfqpoint{2.419000in}{2.419000in}}%
\pgfusepath{clip}%
\pgfsetbuttcap%
\pgfsetroundjoin%
\pgfsetlinewidth{1.505625pt}%
\definecolor{currentstroke}{rgb}{0.000000,0.000000,0.000000}%
\pgfsetstrokecolor{currentstroke}%
\pgfsetdash{}{0pt}%
\pgfusepath{stroke}%
\end{pgfscope}%
\begin{pgfscope}%
\pgfpathrectangle{\pgfqpoint{1.374500in}{0.082500in}}{\pgfqpoint{2.419000in}{2.419000in}}%
\pgfusepath{clip}%
\pgfsetbuttcap%
\pgfsetroundjoin%
\pgfsetlinewidth{1.505625pt}%
\definecolor{currentstroke}{rgb}{0.000000,0.000000,0.000000}%
\pgfsetstrokecolor{currentstroke}%
\pgfsetdash{}{0pt}%
\pgfusepath{stroke}%
\end{pgfscope}%
\begin{pgfscope}%
\pgfpathrectangle{\pgfqpoint{1.374500in}{0.082500in}}{\pgfqpoint{2.419000in}{2.419000in}}%
\pgfusepath{clip}%
\pgfsetbuttcap%
\pgfsetroundjoin%
\pgfsetlinewidth{1.505625pt}%
\definecolor{currentstroke}{rgb}{0.000000,0.000000,0.000000}%
\pgfsetstrokecolor{currentstroke}%
\pgfsetdash{}{0pt}%
\pgfusepath{stroke}%
\end{pgfscope}%
\begin{pgfscope}%
\pgfpathrectangle{\pgfqpoint{1.374500in}{0.082500in}}{\pgfqpoint{2.419000in}{2.419000in}}%
\pgfusepath{clip}%
\pgfsetbuttcap%
\pgfsetroundjoin%
\pgfsetlinewidth{1.505625pt}%
\definecolor{currentstroke}{rgb}{0.000000,0.000000,0.000000}%
\pgfsetstrokecolor{currentstroke}%
\pgfsetdash{}{0pt}%
\pgfusepath{stroke}%
\end{pgfscope}%
\begin{pgfscope}%
\pgfpathrectangle{\pgfqpoint{1.374500in}{0.082500in}}{\pgfqpoint{2.419000in}{2.419000in}}%
\pgfusepath{clip}%
\pgfsetbuttcap%
\pgfsetroundjoin%
\pgfsetlinewidth{1.505625pt}%
\definecolor{currentstroke}{rgb}{0.000000,0.000000,0.000000}%
\pgfsetstrokecolor{currentstroke}%
\pgfsetdash{}{0pt}%
\pgfusepath{stroke}%
\end{pgfscope}%
\begin{pgfscope}%
\pgfpathrectangle{\pgfqpoint{1.374500in}{0.082500in}}{\pgfqpoint{2.419000in}{2.419000in}}%
\pgfusepath{clip}%
\pgfsetbuttcap%
\pgfsetroundjoin%
\pgfsetlinewidth{1.505625pt}%
\definecolor{currentstroke}{rgb}{0.000000,0.000000,0.000000}%
\pgfsetstrokecolor{currentstroke}%
\pgfsetdash{}{0pt}%
\pgfusepath{stroke}%
\end{pgfscope}%
\begin{pgfscope}%
\pgfpathrectangle{\pgfqpoint{1.374500in}{0.082500in}}{\pgfqpoint{2.419000in}{2.419000in}}%
\pgfusepath{clip}%
\pgfsetbuttcap%
\pgfsetroundjoin%
\pgfsetlinewidth{1.505625pt}%
\definecolor{currentstroke}{rgb}{0.000000,0.000000,0.000000}%
\pgfsetstrokecolor{currentstroke}%
\pgfsetdash{}{0pt}%
\pgfusepath{stroke}%
\end{pgfscope}%
\begin{pgfscope}%
\pgfpathrectangle{\pgfqpoint{1.374500in}{0.082500in}}{\pgfqpoint{2.419000in}{2.419000in}}%
\pgfusepath{clip}%
\pgfsetbuttcap%
\pgfsetroundjoin%
\pgfsetlinewidth{1.505625pt}%
\definecolor{currentstroke}{rgb}{0.000000,0.000000,0.000000}%
\pgfsetstrokecolor{currentstroke}%
\pgfsetdash{}{0pt}%
\pgfusepath{stroke}%
\end{pgfscope}%
\begin{pgfscope}%
\pgfpathrectangle{\pgfqpoint{1.374500in}{0.082500in}}{\pgfqpoint{2.419000in}{2.419000in}}%
\pgfusepath{clip}%
\pgfsetbuttcap%
\pgfsetroundjoin%
\pgfsetlinewidth{1.505625pt}%
\definecolor{currentstroke}{rgb}{0.000000,0.000000,0.000000}%
\pgfsetstrokecolor{currentstroke}%
\pgfsetdash{}{0pt}%
\pgfpathmoveto{\pgfqpoint{2.729050in}{0.236875in}}%
\pgfpathlineto{\pgfqpoint{3.155150in}{0.664575in}}%
\pgfusepath{stroke}%
\end{pgfscope}%
\begin{pgfscope}%
\pgfpathrectangle{\pgfqpoint{1.374500in}{0.082500in}}{\pgfqpoint{2.419000in}{2.419000in}}%
\pgfusepath{clip}%
\pgfsetbuttcap%
\pgfsetroundjoin%
\pgfsetlinewidth{1.505625pt}%
\definecolor{currentstroke}{rgb}{0.000000,0.000000,0.000000}%
\pgfsetstrokecolor{currentstroke}%
\pgfsetdash{}{0pt}%
\pgfusepath{stroke}%
\end{pgfscope}%
\begin{pgfscope}%
\pgfpathrectangle{\pgfqpoint{1.374500in}{0.082500in}}{\pgfqpoint{2.419000in}{2.419000in}}%
\pgfusepath{clip}%
\pgfsetbuttcap%
\pgfsetroundjoin%
\pgfsetlinewidth{1.505625pt}%
\definecolor{currentstroke}{rgb}{0.000000,0.000000,0.000000}%
\pgfsetstrokecolor{currentstroke}%
\pgfsetdash{}{0pt}%
\pgfusepath{stroke}%
\end{pgfscope}%
\begin{pgfscope}%
\pgfpathrectangle{\pgfqpoint{1.374500in}{0.082500in}}{\pgfqpoint{2.419000in}{2.419000in}}%
\pgfusepath{clip}%
\pgfsetbuttcap%
\pgfsetroundjoin%
\pgfsetlinewidth{1.505625pt}%
\definecolor{currentstroke}{rgb}{0.000000,0.000000,0.000000}%
\pgfsetstrokecolor{currentstroke}%
\pgfsetdash{}{0pt}%
\pgfusepath{stroke}%
\end{pgfscope}%
\begin{pgfscope}%
\pgfpathrectangle{\pgfqpoint{1.374500in}{0.082500in}}{\pgfqpoint{2.419000in}{2.419000in}}%
\pgfusepath{clip}%
\pgfsetbuttcap%
\pgfsetroundjoin%
\pgfsetlinewidth{1.505625pt}%
\definecolor{currentstroke}{rgb}{0.000000,0.000000,0.000000}%
\pgfsetstrokecolor{currentstroke}%
\pgfsetdash{}{0pt}%
\pgfpathmoveto{\pgfqpoint{1.741405in}{0.561671in}}%
\pgfpathlineto{\pgfqpoint{1.986949in}{0.180259in}}%
\pgfusepath{stroke}%
\end{pgfscope}%
\begin{pgfscope}%
\pgfpathrectangle{\pgfqpoint{1.374500in}{0.082500in}}{\pgfqpoint{2.419000in}{2.419000in}}%
\pgfusepath{clip}%
\pgfsetbuttcap%
\pgfsetroundjoin%
\pgfsetlinewidth{1.505625pt}%
\definecolor{currentstroke}{rgb}{0.000000,0.000000,0.000000}%
\pgfsetstrokecolor{currentstroke}%
\pgfsetdash{}{0pt}%
\pgfpathmoveto{\pgfqpoint{1.986949in}{0.180259in}}%
\pgfpathlineto{\pgfqpoint{2.729050in}{0.236875in}}%
\pgfusepath{stroke}%
\end{pgfscope}%
\begin{pgfscope}%
\pgfpathrectangle{\pgfqpoint{1.374500in}{0.082500in}}{\pgfqpoint{2.419000in}{2.419000in}}%
\pgfusepath{clip}%
\pgfsetbuttcap%
\pgfsetroundjoin%
\pgfsetlinewidth{1.505625pt}%
\definecolor{currentstroke}{rgb}{0.000000,0.000000,0.000000}%
\pgfsetstrokecolor{currentstroke}%
\pgfsetdash{}{0pt}%
\pgfusepath{stroke}%
\end{pgfscope}%
\begin{pgfscope}%
\pgfpathrectangle{\pgfqpoint{1.374500in}{0.082500in}}{\pgfqpoint{2.419000in}{2.419000in}}%
\pgfusepath{clip}%
\pgfsetbuttcap%
\pgfsetroundjoin%
\pgfsetlinewidth{1.505625pt}%
\definecolor{currentstroke}{rgb}{0.000000,0.000000,0.000000}%
\pgfsetstrokecolor{currentstroke}%
\pgfsetdash{}{0pt}%
\pgfusepath{stroke}%
\end{pgfscope}%
\begin{pgfscope}%
\pgfpathrectangle{\pgfqpoint{1.374500in}{0.082500in}}{\pgfqpoint{2.419000in}{2.419000in}}%
\pgfusepath{clip}%
\pgfsetbuttcap%
\pgfsetroundjoin%
\pgfsetlinewidth{1.505625pt}%
\definecolor{currentstroke}{rgb}{0.000000,0.000000,0.000000}%
\pgfsetstrokecolor{currentstroke}%
\pgfsetdash{}{0pt}%
\pgfusepath{stroke}%
\end{pgfscope}%
\begin{pgfscope}%
\pgfpathrectangle{\pgfqpoint{1.374500in}{0.082500in}}{\pgfqpoint{2.419000in}{2.419000in}}%
\pgfusepath{clip}%
\pgfsetbuttcap%
\pgfsetroundjoin%
\pgfsetlinewidth{1.505625pt}%
\definecolor{currentstroke}{rgb}{0.000000,0.000000,0.000000}%
\pgfsetstrokecolor{currentstroke}%
\pgfsetdash{}{0pt}%
\pgfpathmoveto{\pgfqpoint{2.729050in}{0.236875in}}%
\pgfpathlineto{\pgfqpoint{2.730408in}{0.497484in}}%
\pgfusepath{stroke}%
\end{pgfscope}%
\begin{pgfscope}%
\pgfpathrectangle{\pgfqpoint{1.374500in}{0.082500in}}{\pgfqpoint{2.419000in}{2.419000in}}%
\pgfusepath{clip}%
\pgfsetbuttcap%
\pgfsetroundjoin%
\pgfsetlinewidth{1.505625pt}%
\definecolor{currentstroke}{rgb}{0.000000,0.000000,0.000000}%
\pgfsetstrokecolor{currentstroke}%
\pgfsetdash{}{0pt}%
\pgfusepath{stroke}%
\end{pgfscope}%
\begin{pgfscope}%
\pgfpathrectangle{\pgfqpoint{1.374500in}{0.082500in}}{\pgfqpoint{2.419000in}{2.419000in}}%
\pgfusepath{clip}%
\pgfsetbuttcap%
\pgfsetroundjoin%
\pgfsetlinewidth{1.505625pt}%
\definecolor{currentstroke}{rgb}{0.000000,0.000000,0.000000}%
\pgfsetstrokecolor{currentstroke}%
\pgfsetdash{}{0pt}%
\pgfusepath{stroke}%
\end{pgfscope}%
\begin{pgfscope}%
\pgfpathrectangle{\pgfqpoint{1.374500in}{0.082500in}}{\pgfqpoint{2.419000in}{2.419000in}}%
\pgfusepath{clip}%
\pgfsetbuttcap%
\pgfsetroundjoin%
\pgfsetlinewidth{1.505625pt}%
\definecolor{currentstroke}{rgb}{0.000000,0.000000,0.000000}%
\pgfsetstrokecolor{currentstroke}%
\pgfsetdash{}{0pt}%
\pgfusepath{stroke}%
\end{pgfscope}%
\begin{pgfscope}%
\pgfpathrectangle{\pgfqpoint{1.374500in}{0.082500in}}{\pgfqpoint{2.419000in}{2.419000in}}%
\pgfusepath{clip}%
\pgfsetbuttcap%
\pgfsetroundjoin%
\pgfsetlinewidth{1.505625pt}%
\definecolor{currentstroke}{rgb}{0.000000,0.000000,0.000000}%
\pgfsetstrokecolor{currentstroke}%
\pgfsetdash{}{0pt}%
\pgfusepath{stroke}%
\end{pgfscope}%
\begin{pgfscope}%
\pgfpathrectangle{\pgfqpoint{1.374500in}{0.082500in}}{\pgfqpoint{2.419000in}{2.419000in}}%
\pgfusepath{clip}%
\pgfsetbuttcap%
\pgfsetroundjoin%
\pgfsetlinewidth{1.505625pt}%
\definecolor{currentstroke}{rgb}{0.000000,0.000000,0.000000}%
\pgfsetstrokecolor{currentstroke}%
\pgfsetdash{}{0pt}%
\pgfusepath{stroke}%
\end{pgfscope}%
\begin{pgfscope}%
\pgfpathrectangle{\pgfqpoint{1.374500in}{0.082500in}}{\pgfqpoint{2.419000in}{2.419000in}}%
\pgfusepath{clip}%
\pgfsetbuttcap%
\pgfsetroundjoin%
\pgfsetlinewidth{1.505625pt}%
\definecolor{currentstroke}{rgb}{0.000000,0.000000,0.000000}%
\pgfsetstrokecolor{currentstroke}%
\pgfsetdash{}{0pt}%
\pgfusepath{stroke}%
\end{pgfscope}%
\begin{pgfscope}%
\pgfpathrectangle{\pgfqpoint{1.374500in}{0.082500in}}{\pgfqpoint{2.419000in}{2.419000in}}%
\pgfusepath{clip}%
\pgfsetbuttcap%
\pgfsetroundjoin%
\pgfsetlinewidth{1.505625pt}%
\definecolor{currentstroke}{rgb}{0.000000,0.000000,0.000000}%
\pgfsetstrokecolor{currentstroke}%
\pgfsetdash{}{0pt}%
\pgfusepath{stroke}%
\end{pgfscope}%
\begin{pgfscope}%
\pgfpathrectangle{\pgfqpoint{1.374500in}{0.082500in}}{\pgfqpoint{2.419000in}{2.419000in}}%
\pgfusepath{clip}%
\pgfsetbuttcap%
\pgfsetroundjoin%
\pgfsetlinewidth{1.505625pt}%
\definecolor{currentstroke}{rgb}{0.000000,0.000000,0.000000}%
\pgfsetstrokecolor{currentstroke}%
\pgfsetdash{}{0pt}%
\pgfusepath{stroke}%
\end{pgfscope}%
\begin{pgfscope}%
\pgfpathrectangle{\pgfqpoint{1.374500in}{0.082500in}}{\pgfqpoint{2.419000in}{2.419000in}}%
\pgfusepath{clip}%
\pgfsetbuttcap%
\pgfsetroundjoin%
\pgfsetlinewidth{1.505625pt}%
\definecolor{currentstroke}{rgb}{0.000000,0.000000,0.000000}%
\pgfsetstrokecolor{currentstroke}%
\pgfsetdash{}{0pt}%
\pgfusepath{stroke}%
\end{pgfscope}%
\begin{pgfscope}%
\pgfpathrectangle{\pgfqpoint{1.374500in}{0.082500in}}{\pgfqpoint{2.419000in}{2.419000in}}%
\pgfusepath{clip}%
\pgfsetbuttcap%
\pgfsetroundjoin%
\pgfsetlinewidth{1.505625pt}%
\definecolor{currentstroke}{rgb}{0.000000,0.000000,0.000000}%
\pgfsetstrokecolor{currentstroke}%
\pgfsetdash{}{0pt}%
\pgfusepath{stroke}%
\end{pgfscope}%
\begin{pgfscope}%
\pgfpathrectangle{\pgfqpoint{1.374500in}{0.082500in}}{\pgfqpoint{2.419000in}{2.419000in}}%
\pgfusepath{clip}%
\pgfsetbuttcap%
\pgfsetroundjoin%
\pgfsetlinewidth{1.505625pt}%
\definecolor{currentstroke}{rgb}{0.000000,0.000000,0.000000}%
\pgfsetstrokecolor{currentstroke}%
\pgfsetdash{}{0pt}%
\pgfusepath{stroke}%
\end{pgfscope}%
\begin{pgfscope}%
\pgfpathrectangle{\pgfqpoint{1.374500in}{0.082500in}}{\pgfqpoint{2.419000in}{2.419000in}}%
\pgfusepath{clip}%
\pgfsetbuttcap%
\pgfsetroundjoin%
\pgfsetlinewidth{1.505625pt}%
\definecolor{currentstroke}{rgb}{0.000000,0.000000,0.000000}%
\pgfsetstrokecolor{currentstroke}%
\pgfsetdash{}{0pt}%
\pgfusepath{stroke}%
\end{pgfscope}%
\begin{pgfscope}%
\pgfpathrectangle{\pgfqpoint{1.374500in}{0.082500in}}{\pgfqpoint{2.419000in}{2.419000in}}%
\pgfusepath{clip}%
\pgfsetbuttcap%
\pgfsetroundjoin%
\pgfsetlinewidth{1.505625pt}%
\definecolor{currentstroke}{rgb}{0.000000,0.000000,0.000000}%
\pgfsetstrokecolor{currentstroke}%
\pgfsetdash{}{0pt}%
\pgfusepath{stroke}%
\end{pgfscope}%
\begin{pgfscope}%
\pgfpathrectangle{\pgfqpoint{1.374500in}{0.082500in}}{\pgfqpoint{2.419000in}{2.419000in}}%
\pgfusepath{clip}%
\pgfsetbuttcap%
\pgfsetroundjoin%
\pgfsetlinewidth{1.505625pt}%
\definecolor{currentstroke}{rgb}{0.000000,0.000000,0.000000}%
\pgfsetstrokecolor{currentstroke}%
\pgfsetdash{}{0pt}%
\pgfusepath{stroke}%
\end{pgfscope}%
\begin{pgfscope}%
\pgfpathrectangle{\pgfqpoint{1.374500in}{0.082500in}}{\pgfqpoint{2.419000in}{2.419000in}}%
\pgfusepath{clip}%
\pgfsetbuttcap%
\pgfsetroundjoin%
\pgfsetlinewidth{1.505625pt}%
\definecolor{currentstroke}{rgb}{0.000000,0.000000,0.000000}%
\pgfsetstrokecolor{currentstroke}%
\pgfsetdash{}{0pt}%
\pgfusepath{stroke}%
\end{pgfscope}%
\begin{pgfscope}%
\pgfpathrectangle{\pgfqpoint{1.374500in}{0.082500in}}{\pgfqpoint{2.419000in}{2.419000in}}%
\pgfusepath{clip}%
\pgfsetbuttcap%
\pgfsetroundjoin%
\pgfsetlinewidth{1.505625pt}%
\definecolor{currentstroke}{rgb}{0.000000,0.000000,0.000000}%
\pgfsetstrokecolor{currentstroke}%
\pgfsetdash{}{0pt}%
\pgfusepath{stroke}%
\end{pgfscope}%
\begin{pgfscope}%
\pgfpathrectangle{\pgfqpoint{1.374500in}{0.082500in}}{\pgfqpoint{2.419000in}{2.419000in}}%
\pgfusepath{clip}%
\pgfsetbuttcap%
\pgfsetroundjoin%
\pgfsetlinewidth{1.505625pt}%
\definecolor{currentstroke}{rgb}{0.000000,0.000000,0.000000}%
\pgfsetstrokecolor{currentstroke}%
\pgfsetdash{}{0pt}%
\pgfusepath{stroke}%
\end{pgfscope}%
\begin{pgfscope}%
\pgfpathrectangle{\pgfqpoint{1.374500in}{0.082500in}}{\pgfqpoint{2.419000in}{2.419000in}}%
\pgfusepath{clip}%
\pgfsetbuttcap%
\pgfsetroundjoin%
\pgfsetlinewidth{1.505625pt}%
\definecolor{currentstroke}{rgb}{0.000000,0.000000,0.000000}%
\pgfsetstrokecolor{currentstroke}%
\pgfsetdash{}{0pt}%
\pgfusepath{stroke}%
\end{pgfscope}%
\begin{pgfscope}%
\pgfpathrectangle{\pgfqpoint{1.374500in}{0.082500in}}{\pgfqpoint{2.419000in}{2.419000in}}%
\pgfusepath{clip}%
\pgfsetbuttcap%
\pgfsetroundjoin%
\pgfsetlinewidth{1.505625pt}%
\definecolor{currentstroke}{rgb}{0.000000,0.000000,0.000000}%
\pgfsetstrokecolor{currentstroke}%
\pgfsetdash{}{0pt}%
\pgfpathmoveto{\pgfqpoint{2.729050in}{0.236875in}}%
\pgfpathlineto{\pgfqpoint{2.849687in}{0.072500in}}%
\pgfusepath{stroke}%
\end{pgfscope}%
\begin{pgfscope}%
\pgfpathrectangle{\pgfqpoint{1.374500in}{0.082500in}}{\pgfqpoint{2.419000in}{2.419000in}}%
\pgfusepath{clip}%
\pgfsetbuttcap%
\pgfsetroundjoin%
\pgfsetlinewidth{1.505625pt}%
\definecolor{currentstroke}{rgb}{0.000000,0.000000,0.000000}%
\pgfsetstrokecolor{currentstroke}%
\pgfsetdash{}{0pt}%
\pgfusepath{stroke}%
\end{pgfscope}%
\begin{pgfscope}%
\pgfpathrectangle{\pgfqpoint{1.374500in}{0.082500in}}{\pgfqpoint{2.419000in}{2.419000in}}%
\pgfusepath{clip}%
\pgfsetbuttcap%
\pgfsetroundjoin%
\pgfsetlinewidth{1.505625pt}%
\definecolor{currentstroke}{rgb}{0.000000,0.000000,0.000000}%
\pgfsetstrokecolor{currentstroke}%
\pgfsetdash{}{0pt}%
\pgfusepath{stroke}%
\end{pgfscope}%
\begin{pgfscope}%
\pgfpathrectangle{\pgfqpoint{1.374500in}{0.082500in}}{\pgfqpoint{2.419000in}{2.419000in}}%
\pgfusepath{clip}%
\pgfsetbuttcap%
\pgfsetroundjoin%
\pgfsetlinewidth{1.505625pt}%
\definecolor{currentstroke}{rgb}{0.000000,0.000000,0.000000}%
\pgfsetstrokecolor{currentstroke}%
\pgfsetdash{}{0pt}%
\pgfusepath{stroke}%
\end{pgfscope}%
\begin{pgfscope}%
\pgfpathrectangle{\pgfqpoint{1.374500in}{0.082500in}}{\pgfqpoint{2.419000in}{2.419000in}}%
\pgfusepath{clip}%
\pgfsetbuttcap%
\pgfsetroundjoin%
\pgfsetlinewidth{1.505625pt}%
\definecolor{currentstroke}{rgb}{0.000000,0.000000,0.000000}%
\pgfsetstrokecolor{currentstroke}%
\pgfsetdash{}{0pt}%
\pgfusepath{stroke}%
\end{pgfscope}%
\begin{pgfscope}%
\pgfpathrectangle{\pgfqpoint{1.374500in}{0.082500in}}{\pgfqpoint{2.419000in}{2.419000in}}%
\pgfusepath{clip}%
\pgfsetbuttcap%
\pgfsetroundjoin%
\pgfsetlinewidth{1.505625pt}%
\definecolor{currentstroke}{rgb}{0.000000,0.000000,0.000000}%
\pgfsetstrokecolor{currentstroke}%
\pgfsetdash{}{0pt}%
\pgfusepath{stroke}%
\end{pgfscope}%
\begin{pgfscope}%
\pgfpathrectangle{\pgfqpoint{1.374500in}{0.082500in}}{\pgfqpoint{2.419000in}{2.419000in}}%
\pgfusepath{clip}%
\pgfsetbuttcap%
\pgfsetroundjoin%
\pgfsetlinewidth{1.505625pt}%
\definecolor{currentstroke}{rgb}{0.000000,0.000000,0.000000}%
\pgfsetstrokecolor{currentstroke}%
\pgfsetdash{}{0pt}%
\pgfusepath{stroke}%
\end{pgfscope}%
\begin{pgfscope}%
\pgfpathrectangle{\pgfqpoint{1.374500in}{0.082500in}}{\pgfqpoint{2.419000in}{2.419000in}}%
\pgfusepath{clip}%
\pgfsetbuttcap%
\pgfsetroundjoin%
\pgfsetlinewidth{1.505625pt}%
\definecolor{currentstroke}{rgb}{0.000000,0.000000,0.000000}%
\pgfsetstrokecolor{currentstroke}%
\pgfsetdash{}{0pt}%
\pgfusepath{stroke}%
\end{pgfscope}%
\begin{pgfscope}%
\pgfpathrectangle{\pgfqpoint{1.374500in}{0.082500in}}{\pgfqpoint{2.419000in}{2.419000in}}%
\pgfusepath{clip}%
\pgfsetbuttcap%
\pgfsetroundjoin%
\pgfsetlinewidth{1.505625pt}%
\definecolor{currentstroke}{rgb}{0.000000,0.000000,0.000000}%
\pgfsetstrokecolor{currentstroke}%
\pgfsetdash{}{0pt}%
\pgfpathmoveto{\pgfqpoint{1.871094in}{0.072500in}}%
\pgfpathlineto{\pgfqpoint{1.986949in}{0.180259in}}%
\pgfusepath{stroke}%
\end{pgfscope}%
\begin{pgfscope}%
\pgfpathrectangle{\pgfqpoint{1.374500in}{0.082500in}}{\pgfqpoint{2.419000in}{2.419000in}}%
\pgfusepath{clip}%
\pgfsetbuttcap%
\pgfsetroundjoin%
\pgfsetlinewidth{1.505625pt}%
\definecolor{currentstroke}{rgb}{0.000000,0.000000,0.000000}%
\pgfsetstrokecolor{currentstroke}%
\pgfsetdash{}{0pt}%
\pgfusepath{stroke}%
\end{pgfscope}%
\begin{pgfscope}%
\pgfpathrectangle{\pgfqpoint{1.374500in}{0.082500in}}{\pgfqpoint{2.419000in}{2.419000in}}%
\pgfusepath{clip}%
\pgfsetbuttcap%
\pgfsetroundjoin%
\pgfsetlinewidth{1.505625pt}%
\definecolor{currentstroke}{rgb}{0.000000,0.000000,0.000000}%
\pgfsetstrokecolor{currentstroke}%
\pgfsetdash{}{0pt}%
\pgfusepath{stroke}%
\end{pgfscope}%
\begin{pgfscope}%
\pgfpathrectangle{\pgfqpoint{1.374500in}{0.082500in}}{\pgfqpoint{2.419000in}{2.419000in}}%
\pgfusepath{clip}%
\pgfsetbuttcap%
\pgfsetroundjoin%
\pgfsetlinewidth{1.505625pt}%
\definecolor{currentstroke}{rgb}{0.000000,0.000000,0.000000}%
\pgfsetstrokecolor{currentstroke}%
\pgfsetdash{}{0pt}%
\pgfusepath{stroke}%
\end{pgfscope}%
\begin{pgfscope}%
\pgfpathrectangle{\pgfqpoint{1.374500in}{0.082500in}}{\pgfqpoint{2.419000in}{2.419000in}}%
\pgfusepath{clip}%
\pgfsetbuttcap%
\pgfsetroundjoin%
\pgfsetlinewidth{1.505625pt}%
\definecolor{currentstroke}{rgb}{0.000000,0.000000,0.000000}%
\pgfsetstrokecolor{currentstroke}%
\pgfsetdash{}{0pt}%
\pgfusepath{stroke}%
\end{pgfscope}%
\begin{pgfscope}%
\pgfpathrectangle{\pgfqpoint{1.374500in}{0.082500in}}{\pgfqpoint{2.419000in}{2.419000in}}%
\pgfusepath{clip}%
\pgfsetbuttcap%
\pgfsetroundjoin%
\pgfsetlinewidth{1.505625pt}%
\definecolor{currentstroke}{rgb}{0.000000,0.000000,0.000000}%
\pgfsetstrokecolor{currentstroke}%
\pgfsetdash{}{0pt}%
\pgfusepath{stroke}%
\end{pgfscope}%
\begin{pgfscope}%
\pgfpathrectangle{\pgfqpoint{1.374500in}{0.082500in}}{\pgfqpoint{2.419000in}{2.419000in}}%
\pgfusepath{clip}%
\pgfsetbuttcap%
\pgfsetroundjoin%
\pgfsetlinewidth{1.505625pt}%
\definecolor{currentstroke}{rgb}{0.000000,0.000000,0.000000}%
\pgfsetstrokecolor{currentstroke}%
\pgfsetdash{}{0pt}%
\pgfusepath{stroke}%
\end{pgfscope}%
\begin{pgfscope}%
\pgfpathrectangle{\pgfqpoint{1.374500in}{0.082500in}}{\pgfqpoint{2.419000in}{2.419000in}}%
\pgfusepath{clip}%
\pgfsetbuttcap%
\pgfsetroundjoin%
\pgfsetlinewidth{1.505625pt}%
\definecolor{currentstroke}{rgb}{0.000000,0.000000,0.000000}%
\pgfsetstrokecolor{currentstroke}%
\pgfsetdash{}{0pt}%
\pgfusepath{stroke}%
\end{pgfscope}%
\begin{pgfscope}%
\pgfpathrectangle{\pgfqpoint{1.374500in}{0.082500in}}{\pgfqpoint{2.419000in}{2.419000in}}%
\pgfusepath{clip}%
\pgfsetbuttcap%
\pgfsetroundjoin%
\pgfsetlinewidth{1.505625pt}%
\definecolor{currentstroke}{rgb}{0.000000,0.000000,0.000000}%
\pgfsetstrokecolor{currentstroke}%
\pgfsetdash{}{0pt}%
\pgfusepath{stroke}%
\end{pgfscope}%
\begin{pgfscope}%
\pgfpathrectangle{\pgfqpoint{1.374500in}{0.082500in}}{\pgfqpoint{2.419000in}{2.419000in}}%
\pgfusepath{clip}%
\pgfsetbuttcap%
\pgfsetroundjoin%
\pgfsetlinewidth{1.505625pt}%
\definecolor{currentstroke}{rgb}{0.000000,0.000000,0.000000}%
\pgfsetstrokecolor{currentstroke}%
\pgfsetdash{}{0pt}%
\pgfusepath{stroke}%
\end{pgfscope}%
\begin{pgfscope}%
\pgfpathrectangle{\pgfqpoint{1.374500in}{0.082500in}}{\pgfqpoint{2.419000in}{2.419000in}}%
\pgfusepath{clip}%
\pgfsetbuttcap%
\pgfsetroundjoin%
\pgfsetlinewidth{1.505625pt}%
\definecolor{currentstroke}{rgb}{0.000000,0.000000,0.000000}%
\pgfsetstrokecolor{currentstroke}%
\pgfsetdash{}{0pt}%
\pgfusepath{stroke}%
\end{pgfscope}%
\begin{pgfscope}%
\pgfpathrectangle{\pgfqpoint{1.374500in}{0.082500in}}{\pgfqpoint{2.419000in}{2.419000in}}%
\pgfusepath{clip}%
\pgfsetbuttcap%
\pgfsetroundjoin%
\pgfsetlinewidth{1.505625pt}%
\definecolor{currentstroke}{rgb}{0.000000,0.000000,0.000000}%
\pgfsetstrokecolor{currentstroke}%
\pgfsetdash{}{0pt}%
\pgfusepath{stroke}%
\end{pgfscope}%
\begin{pgfscope}%
\pgfpathrectangle{\pgfqpoint{1.374500in}{0.082500in}}{\pgfqpoint{2.419000in}{2.419000in}}%
\pgfusepath{clip}%
\pgfsetbuttcap%
\pgfsetroundjoin%
\pgfsetlinewidth{1.505625pt}%
\definecolor{currentstroke}{rgb}{0.000000,0.000000,0.000000}%
\pgfsetstrokecolor{currentstroke}%
\pgfsetdash{}{0pt}%
\pgfusepath{stroke}%
\end{pgfscope}%
\begin{pgfscope}%
\pgfpathrectangle{\pgfqpoint{1.374500in}{0.082500in}}{\pgfqpoint{2.419000in}{2.419000in}}%
\pgfusepath{clip}%
\pgfsetbuttcap%
\pgfsetroundjoin%
\pgfsetlinewidth{1.505625pt}%
\definecolor{currentstroke}{rgb}{0.000000,0.000000,0.000000}%
\pgfsetstrokecolor{currentstroke}%
\pgfsetdash{}{0pt}%
\pgfusepath{stroke}%
\end{pgfscope}%
\begin{pgfscope}%
\pgfpathrectangle{\pgfqpoint{1.374500in}{0.082500in}}{\pgfqpoint{2.419000in}{2.419000in}}%
\pgfusepath{clip}%
\pgfsetbuttcap%
\pgfsetroundjoin%
\pgfsetlinewidth{1.505625pt}%
\definecolor{currentstroke}{rgb}{0.000000,0.000000,0.000000}%
\pgfsetstrokecolor{currentstroke}%
\pgfsetdash{}{0pt}%
\pgfusepath{stroke}%
\end{pgfscope}%
\begin{pgfscope}%
\pgfpathrectangle{\pgfqpoint{1.374500in}{0.082500in}}{\pgfqpoint{2.419000in}{2.419000in}}%
\pgfusepath{clip}%
\pgfsetbuttcap%
\pgfsetroundjoin%
\pgfsetlinewidth{1.505625pt}%
\definecolor{currentstroke}{rgb}{0.000000,0.000000,0.000000}%
\pgfsetstrokecolor{currentstroke}%
\pgfsetdash{}{0pt}%
\pgfusepath{stroke}%
\end{pgfscope}%
\begin{pgfscope}%
\pgfpathrectangle{\pgfqpoint{1.374500in}{0.082500in}}{\pgfqpoint{2.419000in}{2.419000in}}%
\pgfusepath{clip}%
\pgfsetbuttcap%
\pgfsetroundjoin%
\pgfsetlinewidth{1.505625pt}%
\definecolor{currentstroke}{rgb}{0.000000,0.000000,0.000000}%
\pgfsetstrokecolor{currentstroke}%
\pgfsetdash{}{0pt}%
\pgfusepath{stroke}%
\end{pgfscope}%
\begin{pgfscope}%
\pgfpathrectangle{\pgfqpoint{1.374500in}{0.082500in}}{\pgfqpoint{2.419000in}{2.419000in}}%
\pgfusepath{clip}%
\pgfsetbuttcap%
\pgfsetroundjoin%
\pgfsetlinewidth{1.505625pt}%
\definecolor{currentstroke}{rgb}{0.000000,0.000000,0.000000}%
\pgfsetstrokecolor{currentstroke}%
\pgfsetdash{}{0pt}%
\pgfusepath{stroke}%
\end{pgfscope}%
\begin{pgfscope}%
\pgfpathrectangle{\pgfqpoint{1.374500in}{0.082500in}}{\pgfqpoint{2.419000in}{2.419000in}}%
\pgfusepath{clip}%
\pgfsetbuttcap%
\pgfsetroundjoin%
\pgfsetlinewidth{1.505625pt}%
\definecolor{currentstroke}{rgb}{0.000000,0.000000,0.000000}%
\pgfsetstrokecolor{currentstroke}%
\pgfsetdash{}{0pt}%
\pgfusepath{stroke}%
\end{pgfscope}%
\begin{pgfscope}%
\pgfpathrectangle{\pgfqpoint{1.374500in}{0.082500in}}{\pgfqpoint{2.419000in}{2.419000in}}%
\pgfusepath{clip}%
\pgfsetbuttcap%
\pgfsetroundjoin%
\pgfsetlinewidth{1.505625pt}%
\definecolor{currentstroke}{rgb}{0.000000,0.000000,0.000000}%
\pgfsetstrokecolor{currentstroke}%
\pgfsetdash{}{0pt}%
\pgfusepath{stroke}%
\end{pgfscope}%
\begin{pgfscope}%
\pgfpathrectangle{\pgfqpoint{1.374500in}{0.082500in}}{\pgfqpoint{2.419000in}{2.419000in}}%
\pgfusepath{clip}%
\pgfsetbuttcap%
\pgfsetroundjoin%
\pgfsetlinewidth{1.505625pt}%
\definecolor{currentstroke}{rgb}{0.000000,0.000000,0.000000}%
\pgfsetstrokecolor{currentstroke}%
\pgfsetdash{}{0pt}%
\pgfusepath{stroke}%
\end{pgfscope}%
\begin{pgfscope}%
\pgfpathrectangle{\pgfqpoint{1.374500in}{0.082500in}}{\pgfqpoint{2.419000in}{2.419000in}}%
\pgfusepath{clip}%
\pgfsetbuttcap%
\pgfsetroundjoin%
\pgfsetlinewidth{1.505625pt}%
\definecolor{currentstroke}{rgb}{0.000000,0.000000,0.000000}%
\pgfsetstrokecolor{currentstroke}%
\pgfsetdash{}{0pt}%
\pgfusepath{stroke}%
\end{pgfscope}%
\begin{pgfscope}%
\pgfpathrectangle{\pgfqpoint{1.374500in}{0.082500in}}{\pgfqpoint{2.419000in}{2.419000in}}%
\pgfusepath{clip}%
\pgfsetbuttcap%
\pgfsetroundjoin%
\pgfsetlinewidth{1.505625pt}%
\definecolor{currentstroke}{rgb}{0.000000,0.000000,0.000000}%
\pgfsetstrokecolor{currentstroke}%
\pgfsetdash{}{0pt}%
\pgfusepath{stroke}%
\end{pgfscope}%
\begin{pgfscope}%
\pgfpathrectangle{\pgfqpoint{1.374500in}{0.082500in}}{\pgfqpoint{2.419000in}{2.419000in}}%
\pgfusepath{clip}%
\pgfsetbuttcap%
\pgfsetroundjoin%
\pgfsetlinewidth{1.505625pt}%
\definecolor{currentstroke}{rgb}{0.000000,0.000000,0.000000}%
\pgfsetstrokecolor{currentstroke}%
\pgfsetdash{}{0pt}%
\pgfusepath{stroke}%
\end{pgfscope}%
\begin{pgfscope}%
\pgfpathrectangle{\pgfqpoint{1.374500in}{0.082500in}}{\pgfqpoint{2.419000in}{2.419000in}}%
\pgfusepath{clip}%
\pgfsetbuttcap%
\pgfsetroundjoin%
\pgfsetlinewidth{1.505625pt}%
\definecolor{currentstroke}{rgb}{0.000000,0.000000,0.000000}%
\pgfsetstrokecolor{currentstroke}%
\pgfsetdash{}{0pt}%
\pgfusepath{stroke}%
\end{pgfscope}%
\begin{pgfscope}%
\pgfpathrectangle{\pgfqpoint{1.374500in}{0.082500in}}{\pgfqpoint{2.419000in}{2.419000in}}%
\pgfusepath{clip}%
\pgfsetbuttcap%
\pgfsetroundjoin%
\pgfsetlinewidth{1.505625pt}%
\definecolor{currentstroke}{rgb}{0.000000,0.000000,0.000000}%
\pgfsetstrokecolor{currentstroke}%
\pgfsetdash{}{0pt}%
\pgfusepath{stroke}%
\end{pgfscope}%
\begin{pgfscope}%
\pgfpathrectangle{\pgfqpoint{1.374500in}{0.082500in}}{\pgfqpoint{2.419000in}{2.419000in}}%
\pgfusepath{clip}%
\pgfsetbuttcap%
\pgfsetroundjoin%
\pgfsetlinewidth{1.505625pt}%
\definecolor{currentstroke}{rgb}{0.000000,0.000000,0.000000}%
\pgfsetstrokecolor{currentstroke}%
\pgfsetdash{}{0pt}%
\pgfusepath{stroke}%
\end{pgfscope}%
\begin{pgfscope}%
\pgfpathrectangle{\pgfqpoint{1.374500in}{0.082500in}}{\pgfqpoint{2.419000in}{2.419000in}}%
\pgfusepath{clip}%
\pgfsetbuttcap%
\pgfsetroundjoin%
\pgfsetlinewidth{1.505625pt}%
\definecolor{currentstroke}{rgb}{0.000000,0.000000,0.000000}%
\pgfsetstrokecolor{currentstroke}%
\pgfsetdash{}{0pt}%
\pgfusepath{stroke}%
\end{pgfscope}%
\begin{pgfscope}%
\pgfpathrectangle{\pgfqpoint{1.374500in}{0.082500in}}{\pgfqpoint{2.419000in}{2.419000in}}%
\pgfusepath{clip}%
\pgfsetbuttcap%
\pgfsetroundjoin%
\pgfsetlinewidth{1.505625pt}%
\definecolor{currentstroke}{rgb}{0.000000,0.000000,0.000000}%
\pgfsetstrokecolor{currentstroke}%
\pgfsetdash{}{0pt}%
\pgfusepath{stroke}%
\end{pgfscope}%
\begin{pgfscope}%
\pgfpathrectangle{\pgfqpoint{1.374500in}{0.082500in}}{\pgfqpoint{2.419000in}{2.419000in}}%
\pgfusepath{clip}%
\pgfsetbuttcap%
\pgfsetroundjoin%
\pgfsetlinewidth{1.505625pt}%
\definecolor{currentstroke}{rgb}{0.000000,0.000000,0.000000}%
\pgfsetstrokecolor{currentstroke}%
\pgfsetdash{}{0pt}%
\pgfusepath{stroke}%
\end{pgfscope}%
\begin{pgfscope}%
\pgfpathrectangle{\pgfqpoint{1.374500in}{0.082500in}}{\pgfqpoint{2.419000in}{2.419000in}}%
\pgfusepath{clip}%
\pgfsetbuttcap%
\pgfsetroundjoin%
\pgfsetlinewidth{1.505625pt}%
\definecolor{currentstroke}{rgb}{0.000000,0.000000,0.000000}%
\pgfsetstrokecolor{currentstroke}%
\pgfsetdash{}{0pt}%
\pgfusepath{stroke}%
\end{pgfscope}%
\begin{pgfscope}%
\pgfpathrectangle{\pgfqpoint{1.374500in}{0.082500in}}{\pgfqpoint{2.419000in}{2.419000in}}%
\pgfusepath{clip}%
\pgfsetbuttcap%
\pgfsetroundjoin%
\pgfsetlinewidth{1.505625pt}%
\definecolor{currentstroke}{rgb}{0.000000,0.000000,0.000000}%
\pgfsetstrokecolor{currentstroke}%
\pgfsetdash{}{0pt}%
\pgfusepath{stroke}%
\end{pgfscope}%
\begin{pgfscope}%
\pgfpathrectangle{\pgfqpoint{1.374500in}{0.082500in}}{\pgfqpoint{2.419000in}{2.419000in}}%
\pgfusepath{clip}%
\pgfsetbuttcap%
\pgfsetroundjoin%
\pgfsetlinewidth{1.505625pt}%
\definecolor{currentstroke}{rgb}{0.000000,0.000000,0.000000}%
\pgfsetstrokecolor{currentstroke}%
\pgfsetdash{}{0pt}%
\pgfusepath{stroke}%
\end{pgfscope}%
\begin{pgfscope}%
\pgfpathrectangle{\pgfqpoint{1.374500in}{0.082500in}}{\pgfqpoint{2.419000in}{2.419000in}}%
\pgfusepath{clip}%
\pgfsetbuttcap%
\pgfsetroundjoin%
\pgfsetlinewidth{1.505625pt}%
\definecolor{currentstroke}{rgb}{0.000000,0.000000,0.000000}%
\pgfsetstrokecolor{currentstroke}%
\pgfsetdash{}{0pt}%
\pgfusepath{stroke}%
\end{pgfscope}%
\begin{pgfscope}%
\pgfpathrectangle{\pgfqpoint{1.374500in}{0.082500in}}{\pgfqpoint{2.419000in}{2.419000in}}%
\pgfusepath{clip}%
\pgfsetbuttcap%
\pgfsetroundjoin%
\pgfsetlinewidth{1.505625pt}%
\definecolor{currentstroke}{rgb}{0.000000,0.000000,0.000000}%
\pgfsetstrokecolor{currentstroke}%
\pgfsetdash{}{0pt}%
\pgfusepath{stroke}%
\end{pgfscope}%
\begin{pgfscope}%
\pgfpathrectangle{\pgfqpoint{1.374500in}{0.082500in}}{\pgfqpoint{2.419000in}{2.419000in}}%
\pgfusepath{clip}%
\pgfsetbuttcap%
\pgfsetroundjoin%
\pgfsetlinewidth{1.505625pt}%
\definecolor{currentstroke}{rgb}{0.000000,0.000000,0.000000}%
\pgfsetstrokecolor{currentstroke}%
\pgfsetdash{}{0pt}%
\pgfusepath{stroke}%
\end{pgfscope}%
\begin{pgfscope}%
\pgfpathrectangle{\pgfqpoint{1.374500in}{0.082500in}}{\pgfqpoint{2.419000in}{2.419000in}}%
\pgfusepath{clip}%
\pgfsetbuttcap%
\pgfsetroundjoin%
\pgfsetlinewidth{1.505625pt}%
\definecolor{currentstroke}{rgb}{0.000000,0.000000,0.000000}%
\pgfsetstrokecolor{currentstroke}%
\pgfsetdash{}{0pt}%
\pgfusepath{stroke}%
\end{pgfscope}%
\begin{pgfscope}%
\pgfpathrectangle{\pgfqpoint{1.374500in}{0.082500in}}{\pgfqpoint{2.419000in}{2.419000in}}%
\pgfusepath{clip}%
\pgfsetbuttcap%
\pgfsetroundjoin%
\pgfsetlinewidth{1.505625pt}%
\definecolor{currentstroke}{rgb}{0.000000,0.000000,0.000000}%
\pgfsetstrokecolor{currentstroke}%
\pgfsetdash{}{0pt}%
\pgfusepath{stroke}%
\end{pgfscope}%
\begin{pgfscope}%
\pgfpathrectangle{\pgfqpoint{1.374500in}{0.082500in}}{\pgfqpoint{2.419000in}{2.419000in}}%
\pgfusepath{clip}%
\pgfsetbuttcap%
\pgfsetroundjoin%
\pgfsetlinewidth{1.505625pt}%
\definecolor{currentstroke}{rgb}{0.000000,0.000000,0.000000}%
\pgfsetstrokecolor{currentstroke}%
\pgfsetdash{}{0pt}%
\pgfusepath{stroke}%
\end{pgfscope}%
\begin{pgfscope}%
\pgfpathrectangle{\pgfqpoint{1.374500in}{0.082500in}}{\pgfqpoint{2.419000in}{2.419000in}}%
\pgfusepath{clip}%
\pgfsetbuttcap%
\pgfsetroundjoin%
\pgfsetlinewidth{1.505625pt}%
\definecolor{currentstroke}{rgb}{0.000000,0.000000,0.000000}%
\pgfsetstrokecolor{currentstroke}%
\pgfsetdash{}{0pt}%
\pgfusepath{stroke}%
\end{pgfscope}%
\begin{pgfscope}%
\pgfpathrectangle{\pgfqpoint{1.374500in}{0.082500in}}{\pgfqpoint{2.419000in}{2.419000in}}%
\pgfusepath{clip}%
\pgfsetbuttcap%
\pgfsetroundjoin%
\pgfsetlinewidth{1.505625pt}%
\definecolor{currentstroke}{rgb}{0.000000,0.000000,0.000000}%
\pgfsetstrokecolor{currentstroke}%
\pgfsetdash{}{0pt}%
\pgfusepath{stroke}%
\end{pgfscope}%
\begin{pgfscope}%
\pgfpathrectangle{\pgfqpoint{1.374500in}{0.082500in}}{\pgfqpoint{2.419000in}{2.419000in}}%
\pgfusepath{clip}%
\pgfsetbuttcap%
\pgfsetroundjoin%
\pgfsetlinewidth{1.505625pt}%
\definecolor{currentstroke}{rgb}{0.000000,0.000000,0.000000}%
\pgfsetstrokecolor{currentstroke}%
\pgfsetdash{}{0pt}%
\pgfusepath{stroke}%
\end{pgfscope}%
\begin{pgfscope}%
\pgfpathrectangle{\pgfqpoint{1.374500in}{0.082500in}}{\pgfqpoint{2.419000in}{2.419000in}}%
\pgfusepath{clip}%
\pgfsetbuttcap%
\pgfsetroundjoin%
\pgfsetlinewidth{1.505625pt}%
\definecolor{currentstroke}{rgb}{0.000000,0.000000,0.000000}%
\pgfsetstrokecolor{currentstroke}%
\pgfsetdash{}{0pt}%
\pgfusepath{stroke}%
\end{pgfscope}%
\begin{pgfscope}%
\pgfpathrectangle{\pgfqpoint{1.374500in}{0.082500in}}{\pgfqpoint{2.419000in}{2.419000in}}%
\pgfusepath{clip}%
\pgfsetbuttcap%
\pgfsetroundjoin%
\pgfsetlinewidth{1.505625pt}%
\definecolor{currentstroke}{rgb}{0.000000,0.000000,0.000000}%
\pgfsetstrokecolor{currentstroke}%
\pgfsetdash{}{0pt}%
\pgfusepath{stroke}%
\end{pgfscope}%
\begin{pgfscope}%
\pgfpathrectangle{\pgfqpoint{1.374500in}{0.082500in}}{\pgfqpoint{2.419000in}{2.419000in}}%
\pgfusepath{clip}%
\pgfsetbuttcap%
\pgfsetroundjoin%
\pgfsetlinewidth{1.505625pt}%
\definecolor{currentstroke}{rgb}{0.000000,0.000000,0.000000}%
\pgfsetstrokecolor{currentstroke}%
\pgfsetdash{}{0pt}%
\pgfusepath{stroke}%
\end{pgfscope}%
\begin{pgfscope}%
\pgfpathrectangle{\pgfqpoint{1.374500in}{0.082500in}}{\pgfqpoint{2.419000in}{2.419000in}}%
\pgfusepath{clip}%
\pgfsetbuttcap%
\pgfsetroundjoin%
\pgfsetlinewidth{1.505625pt}%
\definecolor{currentstroke}{rgb}{0.000000,0.000000,0.000000}%
\pgfsetstrokecolor{currentstroke}%
\pgfsetdash{}{0pt}%
\pgfusepath{stroke}%
\end{pgfscope}%
\begin{pgfscope}%
\pgfpathrectangle{\pgfqpoint{1.374500in}{0.082500in}}{\pgfqpoint{2.419000in}{2.419000in}}%
\pgfusepath{clip}%
\pgfsetbuttcap%
\pgfsetroundjoin%
\pgfsetlinewidth{1.505625pt}%
\definecolor{currentstroke}{rgb}{0.000000,0.000000,0.000000}%
\pgfsetstrokecolor{currentstroke}%
\pgfsetdash{}{0pt}%
\pgfusepath{stroke}%
\end{pgfscope}%
\begin{pgfscope}%
\pgfpathrectangle{\pgfqpoint{1.374500in}{0.082500in}}{\pgfqpoint{2.419000in}{2.419000in}}%
\pgfusepath{clip}%
\pgfsetbuttcap%
\pgfsetroundjoin%
\pgfsetlinewidth{1.505625pt}%
\definecolor{currentstroke}{rgb}{0.000000,0.000000,0.000000}%
\pgfsetstrokecolor{currentstroke}%
\pgfsetdash{}{0pt}%
\pgfusepath{stroke}%
\end{pgfscope}%
\begin{pgfscope}%
\pgfpathrectangle{\pgfqpoint{1.374500in}{0.082500in}}{\pgfqpoint{2.419000in}{2.419000in}}%
\pgfusepath{clip}%
\pgfsetbuttcap%
\pgfsetroundjoin%
\pgfsetlinewidth{1.505625pt}%
\definecolor{currentstroke}{rgb}{0.000000,0.000000,0.000000}%
\pgfsetstrokecolor{currentstroke}%
\pgfsetdash{}{0pt}%
\pgfusepath{stroke}%
\end{pgfscope}%
\begin{pgfscope}%
\pgfpathrectangle{\pgfqpoint{1.374500in}{0.082500in}}{\pgfqpoint{2.419000in}{2.419000in}}%
\pgfusepath{clip}%
\pgfsetbuttcap%
\pgfsetroundjoin%
\pgfsetlinewidth{1.505625pt}%
\definecolor{currentstroke}{rgb}{0.000000,0.000000,0.000000}%
\pgfsetstrokecolor{currentstroke}%
\pgfsetdash{}{0pt}%
\pgfusepath{stroke}%
\end{pgfscope}%
\begin{pgfscope}%
\pgfpathrectangle{\pgfqpoint{1.374500in}{0.082500in}}{\pgfqpoint{2.419000in}{2.419000in}}%
\pgfusepath{clip}%
\pgfsetbuttcap%
\pgfsetroundjoin%
\pgfsetlinewidth{1.505625pt}%
\definecolor{currentstroke}{rgb}{0.000000,0.000000,0.000000}%
\pgfsetstrokecolor{currentstroke}%
\pgfsetdash{}{0pt}%
\pgfusepath{stroke}%
\end{pgfscope}%
\begin{pgfscope}%
\pgfpathrectangle{\pgfqpoint{1.374500in}{0.082500in}}{\pgfqpoint{2.419000in}{2.419000in}}%
\pgfusepath{clip}%
\pgfsetbuttcap%
\pgfsetroundjoin%
\pgfsetlinewidth{1.505625pt}%
\definecolor{currentstroke}{rgb}{0.000000,0.000000,0.000000}%
\pgfsetstrokecolor{currentstroke}%
\pgfsetdash{}{0pt}%
\pgfusepath{stroke}%
\end{pgfscope}%
\begin{pgfscope}%
\pgfpathrectangle{\pgfqpoint{1.374500in}{0.082500in}}{\pgfqpoint{2.419000in}{2.419000in}}%
\pgfusepath{clip}%
\pgfsetbuttcap%
\pgfsetroundjoin%
\pgfsetlinewidth{1.505625pt}%
\definecolor{currentstroke}{rgb}{0.000000,0.000000,0.000000}%
\pgfsetstrokecolor{currentstroke}%
\pgfsetdash{}{0pt}%
\pgfusepath{stroke}%
\end{pgfscope}%
\begin{pgfscope}%
\pgfpathrectangle{\pgfqpoint{1.374500in}{0.082500in}}{\pgfqpoint{2.419000in}{2.419000in}}%
\pgfusepath{clip}%
\pgfsetbuttcap%
\pgfsetroundjoin%
\pgfsetlinewidth{1.505625pt}%
\definecolor{currentstroke}{rgb}{0.000000,0.000000,0.000000}%
\pgfsetstrokecolor{currentstroke}%
\pgfsetdash{}{0pt}%
\pgfusepath{stroke}%
\end{pgfscope}%
\begin{pgfscope}%
\pgfpathrectangle{\pgfqpoint{1.374500in}{0.082500in}}{\pgfqpoint{2.419000in}{2.419000in}}%
\pgfusepath{clip}%
\pgfsetbuttcap%
\pgfsetroundjoin%
\pgfsetlinewidth{1.505625pt}%
\definecolor{currentstroke}{rgb}{0.000000,0.000000,0.000000}%
\pgfsetstrokecolor{currentstroke}%
\pgfsetdash{}{0pt}%
\pgfusepath{stroke}%
\end{pgfscope}%
\begin{pgfscope}%
\pgfpathrectangle{\pgfqpoint{1.374500in}{0.082500in}}{\pgfqpoint{2.419000in}{2.419000in}}%
\pgfusepath{clip}%
\pgfsetbuttcap%
\pgfsetroundjoin%
\pgfsetlinewidth{1.505625pt}%
\definecolor{currentstroke}{rgb}{0.000000,0.000000,0.000000}%
\pgfsetstrokecolor{currentstroke}%
\pgfsetdash{}{0pt}%
\pgfusepath{stroke}%
\end{pgfscope}%
\begin{pgfscope}%
\pgfpathrectangle{\pgfqpoint{1.374500in}{0.082500in}}{\pgfqpoint{2.419000in}{2.419000in}}%
\pgfusepath{clip}%
\pgfsetbuttcap%
\pgfsetroundjoin%
\pgfsetlinewidth{1.505625pt}%
\definecolor{currentstroke}{rgb}{0.000000,0.000000,0.000000}%
\pgfsetstrokecolor{currentstroke}%
\pgfsetdash{}{0pt}%
\pgfusepath{stroke}%
\end{pgfscope}%
\begin{pgfscope}%
\pgfpathrectangle{\pgfqpoint{1.374500in}{0.082500in}}{\pgfqpoint{2.419000in}{2.419000in}}%
\pgfusepath{clip}%
\pgfsetbuttcap%
\pgfsetroundjoin%
\pgfsetlinewidth{1.505625pt}%
\definecolor{currentstroke}{rgb}{0.000000,0.000000,0.000000}%
\pgfsetstrokecolor{currentstroke}%
\pgfsetdash{}{0pt}%
\pgfusepath{stroke}%
\end{pgfscope}%
\begin{pgfscope}%
\pgfpathrectangle{\pgfqpoint{1.374500in}{0.082500in}}{\pgfqpoint{2.419000in}{2.419000in}}%
\pgfusepath{clip}%
\pgfsetbuttcap%
\pgfsetroundjoin%
\pgfsetlinewidth{1.505625pt}%
\definecolor{currentstroke}{rgb}{0.000000,0.000000,0.000000}%
\pgfsetstrokecolor{currentstroke}%
\pgfsetdash{}{0pt}%
\pgfusepath{stroke}%
\end{pgfscope}%
\begin{pgfscope}%
\pgfpathrectangle{\pgfqpoint{1.374500in}{0.082500in}}{\pgfqpoint{2.419000in}{2.419000in}}%
\pgfusepath{clip}%
\pgfsetbuttcap%
\pgfsetroundjoin%
\pgfsetlinewidth{1.505625pt}%
\definecolor{currentstroke}{rgb}{0.000000,0.000000,0.000000}%
\pgfsetstrokecolor{currentstroke}%
\pgfsetdash{}{0pt}%
\pgfusepath{stroke}%
\end{pgfscope}%
\begin{pgfscope}%
\pgfpathrectangle{\pgfqpoint{1.374500in}{0.082500in}}{\pgfqpoint{2.419000in}{2.419000in}}%
\pgfusepath{clip}%
\pgfsetbuttcap%
\pgfsetroundjoin%
\pgfsetlinewidth{1.505625pt}%
\definecolor{currentstroke}{rgb}{0.000000,0.000000,0.000000}%
\pgfsetstrokecolor{currentstroke}%
\pgfsetdash{}{0pt}%
\pgfusepath{stroke}%
\end{pgfscope}%
\begin{pgfscope}%
\pgfpathrectangle{\pgfqpoint{1.374500in}{0.082500in}}{\pgfqpoint{2.419000in}{2.419000in}}%
\pgfusepath{clip}%
\pgfsetbuttcap%
\pgfsetroundjoin%
\pgfsetlinewidth{1.505625pt}%
\definecolor{currentstroke}{rgb}{0.000000,0.000000,0.000000}%
\pgfsetstrokecolor{currentstroke}%
\pgfsetdash{}{0pt}%
\pgfusepath{stroke}%
\end{pgfscope}%
\begin{pgfscope}%
\pgfpathrectangle{\pgfqpoint{1.374500in}{0.082500in}}{\pgfqpoint{2.419000in}{2.419000in}}%
\pgfusepath{clip}%
\pgfsetbuttcap%
\pgfsetroundjoin%
\pgfsetlinewidth{1.505625pt}%
\definecolor{currentstroke}{rgb}{0.000000,0.000000,0.000000}%
\pgfsetstrokecolor{currentstroke}%
\pgfsetdash{}{0pt}%
\pgfusepath{stroke}%
\end{pgfscope}%
\begin{pgfscope}%
\pgfpathrectangle{\pgfqpoint{1.374500in}{0.082500in}}{\pgfqpoint{2.419000in}{2.419000in}}%
\pgfusepath{clip}%
\pgfsetbuttcap%
\pgfsetroundjoin%
\pgfsetlinewidth{1.505625pt}%
\definecolor{currentstroke}{rgb}{0.000000,0.000000,0.000000}%
\pgfsetstrokecolor{currentstroke}%
\pgfsetdash{}{0pt}%
\pgfusepath{stroke}%
\end{pgfscope}%
\begin{pgfscope}%
\pgfpathrectangle{\pgfqpoint{1.374500in}{0.082500in}}{\pgfqpoint{2.419000in}{2.419000in}}%
\pgfusepath{clip}%
\pgfsetbuttcap%
\pgfsetroundjoin%
\pgfsetlinewidth{1.505625pt}%
\definecolor{currentstroke}{rgb}{0.000000,0.000000,0.000000}%
\pgfsetstrokecolor{currentstroke}%
\pgfsetdash{}{0pt}%
\pgfusepath{stroke}%
\end{pgfscope}%
\begin{pgfscope}%
\pgfpathrectangle{\pgfqpoint{1.374500in}{0.082500in}}{\pgfqpoint{2.419000in}{2.419000in}}%
\pgfusepath{clip}%
\pgfsetbuttcap%
\pgfsetroundjoin%
\pgfsetlinewidth{1.505625pt}%
\definecolor{currentstroke}{rgb}{0.000000,0.000000,0.000000}%
\pgfsetstrokecolor{currentstroke}%
\pgfsetdash{}{0pt}%
\pgfusepath{stroke}%
\end{pgfscope}%
\begin{pgfscope}%
\pgfpathrectangle{\pgfqpoint{1.374500in}{0.082500in}}{\pgfqpoint{2.419000in}{2.419000in}}%
\pgfusepath{clip}%
\pgfsetbuttcap%
\pgfsetroundjoin%
\pgfsetlinewidth{1.505625pt}%
\definecolor{currentstroke}{rgb}{0.000000,0.000000,0.000000}%
\pgfsetstrokecolor{currentstroke}%
\pgfsetdash{}{0pt}%
\pgfusepath{stroke}%
\end{pgfscope}%
\begin{pgfscope}%
\pgfpathrectangle{\pgfqpoint{1.374500in}{0.082500in}}{\pgfqpoint{2.419000in}{2.419000in}}%
\pgfusepath{clip}%
\pgfsetbuttcap%
\pgfsetroundjoin%
\pgfsetlinewidth{1.505625pt}%
\definecolor{currentstroke}{rgb}{0.000000,0.000000,0.000000}%
\pgfsetstrokecolor{currentstroke}%
\pgfsetdash{}{0pt}%
\pgfusepath{stroke}%
\end{pgfscope}%
\begin{pgfscope}%
\pgfpathrectangle{\pgfqpoint{1.374500in}{0.082500in}}{\pgfqpoint{2.419000in}{2.419000in}}%
\pgfusepath{clip}%
\pgfsetbuttcap%
\pgfsetroundjoin%
\pgfsetlinewidth{1.505625pt}%
\definecolor{currentstroke}{rgb}{0.000000,0.000000,0.000000}%
\pgfsetstrokecolor{currentstroke}%
\pgfsetdash{}{0pt}%
\pgfusepath{stroke}%
\end{pgfscope}%
\begin{pgfscope}%
\pgfpathrectangle{\pgfqpoint{1.374500in}{0.082500in}}{\pgfqpoint{2.419000in}{2.419000in}}%
\pgfusepath{clip}%
\pgfsetbuttcap%
\pgfsetroundjoin%
\pgfsetlinewidth{1.505625pt}%
\definecolor{currentstroke}{rgb}{0.000000,0.000000,0.000000}%
\pgfsetstrokecolor{currentstroke}%
\pgfsetdash{}{0pt}%
\pgfusepath{stroke}%
\end{pgfscope}%
\begin{pgfscope}%
\pgfpathrectangle{\pgfqpoint{1.374500in}{0.082500in}}{\pgfqpoint{2.419000in}{2.419000in}}%
\pgfusepath{clip}%
\pgfsetbuttcap%
\pgfsetroundjoin%
\pgfsetlinewidth{1.505625pt}%
\definecolor{currentstroke}{rgb}{0.000000,0.000000,0.000000}%
\pgfsetstrokecolor{currentstroke}%
\pgfsetdash{}{0pt}%
\pgfusepath{stroke}%
\end{pgfscope}%
\begin{pgfscope}%
\pgfpathrectangle{\pgfqpoint{1.374500in}{0.082500in}}{\pgfqpoint{2.419000in}{2.419000in}}%
\pgfusepath{clip}%
\pgfsetbuttcap%
\pgfsetroundjoin%
\pgfsetlinewidth{1.505625pt}%
\definecolor{currentstroke}{rgb}{0.000000,0.000000,0.000000}%
\pgfsetstrokecolor{currentstroke}%
\pgfsetdash{}{0pt}%
\pgfusepath{stroke}%
\end{pgfscope}%
\begin{pgfscope}%
\pgfpathrectangle{\pgfqpoint{1.374500in}{0.082500in}}{\pgfqpoint{2.419000in}{2.419000in}}%
\pgfusepath{clip}%
\pgfsetbuttcap%
\pgfsetroundjoin%
\pgfsetlinewidth{1.505625pt}%
\definecolor{currentstroke}{rgb}{0.000000,0.000000,0.000000}%
\pgfsetstrokecolor{currentstroke}%
\pgfsetdash{}{0pt}%
\pgfusepath{stroke}%
\end{pgfscope}%
\begin{pgfscope}%
\pgfpathrectangle{\pgfqpoint{1.374500in}{0.082500in}}{\pgfqpoint{2.419000in}{2.419000in}}%
\pgfusepath{clip}%
\pgfsetbuttcap%
\pgfsetroundjoin%
\pgfsetlinewidth{1.505625pt}%
\definecolor{currentstroke}{rgb}{0.000000,0.000000,0.000000}%
\pgfsetstrokecolor{currentstroke}%
\pgfsetdash{}{0pt}%
\pgfusepath{stroke}%
\end{pgfscope}%
\begin{pgfscope}%
\pgfpathrectangle{\pgfqpoint{1.374500in}{0.082500in}}{\pgfqpoint{2.419000in}{2.419000in}}%
\pgfusepath{clip}%
\pgfsetbuttcap%
\pgfsetroundjoin%
\pgfsetlinewidth{1.505625pt}%
\definecolor{currentstroke}{rgb}{0.000000,0.000000,0.000000}%
\pgfsetstrokecolor{currentstroke}%
\pgfsetdash{}{0pt}%
\pgfusepath{stroke}%
\end{pgfscope}%
\begin{pgfscope}%
\pgfpathrectangle{\pgfqpoint{1.374500in}{0.082500in}}{\pgfqpoint{2.419000in}{2.419000in}}%
\pgfusepath{clip}%
\pgfsetbuttcap%
\pgfsetroundjoin%
\pgfsetlinewidth{1.505625pt}%
\definecolor{currentstroke}{rgb}{0.000000,0.000000,0.000000}%
\pgfsetstrokecolor{currentstroke}%
\pgfsetdash{}{0pt}%
\pgfusepath{stroke}%
\end{pgfscope}%
\begin{pgfscope}%
\pgfpathrectangle{\pgfqpoint{1.374500in}{0.082500in}}{\pgfqpoint{2.419000in}{2.419000in}}%
\pgfusepath{clip}%
\pgfsetbuttcap%
\pgfsetroundjoin%
\pgfsetlinewidth{1.505625pt}%
\definecolor{currentstroke}{rgb}{0.000000,0.000000,0.000000}%
\pgfsetstrokecolor{currentstroke}%
\pgfsetdash{}{0pt}%
\pgfusepath{stroke}%
\end{pgfscope}%
\begin{pgfscope}%
\pgfpathrectangle{\pgfqpoint{1.374500in}{0.082500in}}{\pgfqpoint{2.419000in}{2.419000in}}%
\pgfusepath{clip}%
\pgfsetbuttcap%
\pgfsetroundjoin%
\pgfsetlinewidth{1.505625pt}%
\definecolor{currentstroke}{rgb}{0.000000,0.000000,0.000000}%
\pgfsetstrokecolor{currentstroke}%
\pgfsetdash{}{0pt}%
\pgfusepath{stroke}%
\end{pgfscope}%
\begin{pgfscope}%
\pgfpathrectangle{\pgfqpoint{1.374500in}{0.082500in}}{\pgfqpoint{2.419000in}{2.419000in}}%
\pgfusepath{clip}%
\pgfsetbuttcap%
\pgfsetroundjoin%
\pgfsetlinewidth{1.505625pt}%
\definecolor{currentstroke}{rgb}{0.000000,0.000000,0.000000}%
\pgfsetstrokecolor{currentstroke}%
\pgfsetdash{}{0pt}%
\pgfusepath{stroke}%
\end{pgfscope}%
\begin{pgfscope}%
\pgfpathrectangle{\pgfqpoint{1.374500in}{0.082500in}}{\pgfqpoint{2.419000in}{2.419000in}}%
\pgfusepath{clip}%
\pgfsetbuttcap%
\pgfsetroundjoin%
\pgfsetlinewidth{1.505625pt}%
\definecolor{currentstroke}{rgb}{0.000000,0.000000,0.000000}%
\pgfsetstrokecolor{currentstroke}%
\pgfsetdash{}{0pt}%
\pgfusepath{stroke}%
\end{pgfscope}%
\begin{pgfscope}%
\pgfpathrectangle{\pgfqpoint{1.374500in}{0.082500in}}{\pgfqpoint{2.419000in}{2.419000in}}%
\pgfusepath{clip}%
\pgfsetbuttcap%
\pgfsetroundjoin%
\pgfsetlinewidth{1.505625pt}%
\definecolor{currentstroke}{rgb}{0.000000,0.000000,0.000000}%
\pgfsetstrokecolor{currentstroke}%
\pgfsetdash{}{0pt}%
\pgfusepath{stroke}%
\end{pgfscope}%
\begin{pgfscope}%
\pgfpathrectangle{\pgfqpoint{1.374500in}{0.082500in}}{\pgfqpoint{2.419000in}{2.419000in}}%
\pgfusepath{clip}%
\pgfsetbuttcap%
\pgfsetroundjoin%
\pgfsetlinewidth{1.505625pt}%
\definecolor{currentstroke}{rgb}{0.000000,0.000000,0.000000}%
\pgfsetstrokecolor{currentstroke}%
\pgfsetdash{}{0pt}%
\pgfusepath{stroke}%
\end{pgfscope}%
\begin{pgfscope}%
\pgfpathrectangle{\pgfqpoint{1.374500in}{0.082500in}}{\pgfqpoint{2.419000in}{2.419000in}}%
\pgfusepath{clip}%
\pgfsetbuttcap%
\pgfsetroundjoin%
\pgfsetlinewidth{1.505625pt}%
\definecolor{currentstroke}{rgb}{0.000000,0.000000,0.000000}%
\pgfsetstrokecolor{currentstroke}%
\pgfsetdash{}{0pt}%
\pgfusepath{stroke}%
\end{pgfscope}%
\begin{pgfscope}%
\pgfpathrectangle{\pgfqpoint{1.374500in}{0.082500in}}{\pgfqpoint{2.419000in}{2.419000in}}%
\pgfusepath{clip}%
\pgfsetbuttcap%
\pgfsetroundjoin%
\pgfsetlinewidth{1.505625pt}%
\definecolor{currentstroke}{rgb}{0.000000,0.000000,0.000000}%
\pgfsetstrokecolor{currentstroke}%
\pgfsetdash{}{0pt}%
\pgfusepath{stroke}%
\end{pgfscope}%
\begin{pgfscope}%
\pgfpathrectangle{\pgfqpoint{1.374500in}{0.082500in}}{\pgfqpoint{2.419000in}{2.419000in}}%
\pgfusepath{clip}%
\pgfsetbuttcap%
\pgfsetroundjoin%
\pgfsetlinewidth{1.505625pt}%
\definecolor{currentstroke}{rgb}{0.000000,0.000000,0.000000}%
\pgfsetstrokecolor{currentstroke}%
\pgfsetdash{}{0pt}%
\pgfusepath{stroke}%
\end{pgfscope}%
\begin{pgfscope}%
\pgfpathrectangle{\pgfqpoint{1.374500in}{0.082500in}}{\pgfqpoint{2.419000in}{2.419000in}}%
\pgfusepath{clip}%
\pgfsetbuttcap%
\pgfsetroundjoin%
\pgfsetlinewidth{1.505625pt}%
\definecolor{currentstroke}{rgb}{0.000000,0.000000,0.000000}%
\pgfsetstrokecolor{currentstroke}%
\pgfsetdash{}{0pt}%
\pgfusepath{stroke}%
\end{pgfscope}%
\begin{pgfscope}%
\pgfpathrectangle{\pgfqpoint{1.374500in}{0.082500in}}{\pgfqpoint{2.419000in}{2.419000in}}%
\pgfusepath{clip}%
\pgfsetbuttcap%
\pgfsetroundjoin%
\pgfsetlinewidth{1.505625pt}%
\definecolor{currentstroke}{rgb}{0.000000,0.000000,0.000000}%
\pgfsetstrokecolor{currentstroke}%
\pgfsetdash{}{0pt}%
\pgfusepath{stroke}%
\end{pgfscope}%
\begin{pgfscope}%
\pgfpathrectangle{\pgfqpoint{1.374500in}{0.082500in}}{\pgfqpoint{2.419000in}{2.419000in}}%
\pgfusepath{clip}%
\pgfsetbuttcap%
\pgfsetroundjoin%
\pgfsetlinewidth{1.505625pt}%
\definecolor{currentstroke}{rgb}{0.000000,0.000000,0.000000}%
\pgfsetstrokecolor{currentstroke}%
\pgfsetdash{}{0pt}%
\pgfusepath{stroke}%
\end{pgfscope}%
\begin{pgfscope}%
\pgfpathrectangle{\pgfqpoint{1.374500in}{0.082500in}}{\pgfqpoint{2.419000in}{2.419000in}}%
\pgfusepath{clip}%
\pgfsetbuttcap%
\pgfsetroundjoin%
\pgfsetlinewidth{1.505625pt}%
\definecolor{currentstroke}{rgb}{0.000000,0.000000,0.000000}%
\pgfsetstrokecolor{currentstroke}%
\pgfsetdash{}{0pt}%
\pgfusepath{stroke}%
\end{pgfscope}%
\begin{pgfscope}%
\pgfpathrectangle{\pgfqpoint{1.374500in}{0.082500in}}{\pgfqpoint{2.419000in}{2.419000in}}%
\pgfusepath{clip}%
\pgfsetbuttcap%
\pgfsetroundjoin%
\pgfsetlinewidth{1.505625pt}%
\definecolor{currentstroke}{rgb}{0.000000,0.000000,0.000000}%
\pgfsetstrokecolor{currentstroke}%
\pgfsetdash{}{0pt}%
\pgfusepath{stroke}%
\end{pgfscope}%
\begin{pgfscope}%
\pgfpathrectangle{\pgfqpoint{1.374500in}{0.082500in}}{\pgfqpoint{2.419000in}{2.419000in}}%
\pgfusepath{clip}%
\pgfsetbuttcap%
\pgfsetroundjoin%
\pgfsetlinewidth{1.505625pt}%
\definecolor{currentstroke}{rgb}{0.000000,0.000000,0.000000}%
\pgfsetstrokecolor{currentstroke}%
\pgfsetdash{}{0pt}%
\pgfusepath{stroke}%
\end{pgfscope}%
\begin{pgfscope}%
\pgfpathrectangle{\pgfqpoint{1.374500in}{0.082500in}}{\pgfqpoint{2.419000in}{2.419000in}}%
\pgfusepath{clip}%
\pgfsetbuttcap%
\pgfsetroundjoin%
\pgfsetlinewidth{1.505625pt}%
\definecolor{currentstroke}{rgb}{0.000000,0.000000,0.000000}%
\pgfsetstrokecolor{currentstroke}%
\pgfsetdash{}{0pt}%
\pgfusepath{stroke}%
\end{pgfscope}%
\begin{pgfscope}%
\pgfpathrectangle{\pgfqpoint{1.374500in}{0.082500in}}{\pgfqpoint{2.419000in}{2.419000in}}%
\pgfusepath{clip}%
\pgfsetbuttcap%
\pgfsetroundjoin%
\pgfsetlinewidth{1.505625pt}%
\definecolor{currentstroke}{rgb}{0.000000,0.000000,0.000000}%
\pgfsetstrokecolor{currentstroke}%
\pgfsetdash{}{0pt}%
\pgfusepath{stroke}%
\end{pgfscope}%
\begin{pgfscope}%
\pgfpathrectangle{\pgfqpoint{1.374500in}{0.082500in}}{\pgfqpoint{2.419000in}{2.419000in}}%
\pgfusepath{clip}%
\pgfsetbuttcap%
\pgfsetroundjoin%
\pgfsetlinewidth{1.505625pt}%
\definecolor{currentstroke}{rgb}{0.000000,0.000000,0.000000}%
\pgfsetstrokecolor{currentstroke}%
\pgfsetdash{}{0pt}%
\pgfusepath{stroke}%
\end{pgfscope}%
\begin{pgfscope}%
\pgfpathrectangle{\pgfqpoint{1.374500in}{0.082500in}}{\pgfqpoint{2.419000in}{2.419000in}}%
\pgfusepath{clip}%
\pgfsetbuttcap%
\pgfsetroundjoin%
\pgfsetlinewidth{1.505625pt}%
\definecolor{currentstroke}{rgb}{0.000000,0.000000,0.000000}%
\pgfsetstrokecolor{currentstroke}%
\pgfsetdash{}{0pt}%
\pgfusepath{stroke}%
\end{pgfscope}%
\begin{pgfscope}%
\pgfpathrectangle{\pgfqpoint{1.374500in}{0.082500in}}{\pgfqpoint{2.419000in}{2.419000in}}%
\pgfusepath{clip}%
\pgfsetbuttcap%
\pgfsetroundjoin%
\pgfsetlinewidth{1.505625pt}%
\definecolor{currentstroke}{rgb}{0.000000,0.000000,0.000000}%
\pgfsetstrokecolor{currentstroke}%
\pgfsetdash{}{0pt}%
\pgfusepath{stroke}%
\end{pgfscope}%
\begin{pgfscope}%
\pgfpathrectangle{\pgfqpoint{1.374500in}{0.082500in}}{\pgfqpoint{2.419000in}{2.419000in}}%
\pgfusepath{clip}%
\pgfsetbuttcap%
\pgfsetroundjoin%
\pgfsetlinewidth{1.505625pt}%
\definecolor{currentstroke}{rgb}{0.000000,0.000000,0.000000}%
\pgfsetstrokecolor{currentstroke}%
\pgfsetdash{}{0pt}%
\pgfusepath{stroke}%
\end{pgfscope}%
\begin{pgfscope}%
\pgfpathrectangle{\pgfqpoint{1.374500in}{0.082500in}}{\pgfqpoint{2.419000in}{2.419000in}}%
\pgfusepath{clip}%
\pgfsetbuttcap%
\pgfsetroundjoin%
\pgfsetlinewidth{1.505625pt}%
\definecolor{currentstroke}{rgb}{0.000000,0.000000,0.000000}%
\pgfsetstrokecolor{currentstroke}%
\pgfsetdash{}{0pt}%
\pgfusepath{stroke}%
\end{pgfscope}%
\begin{pgfscope}%
\pgfpathrectangle{\pgfqpoint{1.374500in}{0.082500in}}{\pgfqpoint{2.419000in}{2.419000in}}%
\pgfusepath{clip}%
\pgfsetbuttcap%
\pgfsetroundjoin%
\pgfsetlinewidth{1.505625pt}%
\definecolor{currentstroke}{rgb}{0.000000,0.000000,0.000000}%
\pgfsetstrokecolor{currentstroke}%
\pgfsetdash{}{0pt}%
\pgfusepath{stroke}%
\end{pgfscope}%
\begin{pgfscope}%
\pgfpathrectangle{\pgfqpoint{1.374500in}{0.082500in}}{\pgfqpoint{2.419000in}{2.419000in}}%
\pgfusepath{clip}%
\pgfsetbuttcap%
\pgfsetroundjoin%
\pgfsetlinewidth{1.505625pt}%
\definecolor{currentstroke}{rgb}{0.000000,0.000000,0.000000}%
\pgfsetstrokecolor{currentstroke}%
\pgfsetdash{}{0pt}%
\pgfusepath{stroke}%
\end{pgfscope}%
\begin{pgfscope}%
\pgfpathrectangle{\pgfqpoint{1.374500in}{0.082500in}}{\pgfqpoint{2.419000in}{2.419000in}}%
\pgfusepath{clip}%
\pgfsetbuttcap%
\pgfsetroundjoin%
\pgfsetlinewidth{1.505625pt}%
\definecolor{currentstroke}{rgb}{0.000000,0.000000,0.000000}%
\pgfsetstrokecolor{currentstroke}%
\pgfsetdash{}{0pt}%
\pgfusepath{stroke}%
\end{pgfscope}%
\begin{pgfscope}%
\pgfpathrectangle{\pgfqpoint{1.374500in}{0.082500in}}{\pgfqpoint{2.419000in}{2.419000in}}%
\pgfusepath{clip}%
\pgfsetbuttcap%
\pgfsetroundjoin%
\pgfsetlinewidth{1.505625pt}%
\definecolor{currentstroke}{rgb}{0.000000,0.000000,0.000000}%
\pgfsetstrokecolor{currentstroke}%
\pgfsetdash{}{0pt}%
\pgfusepath{stroke}%
\end{pgfscope}%
\begin{pgfscope}%
\pgfpathrectangle{\pgfqpoint{1.374500in}{0.082500in}}{\pgfqpoint{2.419000in}{2.419000in}}%
\pgfusepath{clip}%
\pgfsetbuttcap%
\pgfsetroundjoin%
\pgfsetlinewidth{1.505625pt}%
\definecolor{currentstroke}{rgb}{0.000000,0.000000,0.000000}%
\pgfsetstrokecolor{currentstroke}%
\pgfsetdash{}{0pt}%
\pgfusepath{stroke}%
\end{pgfscope}%
\begin{pgfscope}%
\pgfpathrectangle{\pgfqpoint{1.374500in}{0.082500in}}{\pgfqpoint{2.419000in}{2.419000in}}%
\pgfusepath{clip}%
\pgfsetbuttcap%
\pgfsetroundjoin%
\pgfsetlinewidth{1.505625pt}%
\definecolor{currentstroke}{rgb}{0.000000,0.000000,0.000000}%
\pgfsetstrokecolor{currentstroke}%
\pgfsetdash{}{0pt}%
\pgfusepath{stroke}%
\end{pgfscope}%
\begin{pgfscope}%
\pgfpathrectangle{\pgfqpoint{1.374500in}{0.082500in}}{\pgfqpoint{2.419000in}{2.419000in}}%
\pgfusepath{clip}%
\pgfsetbuttcap%
\pgfsetroundjoin%
\pgfsetlinewidth{1.505625pt}%
\definecolor{currentstroke}{rgb}{0.000000,0.000000,0.000000}%
\pgfsetstrokecolor{currentstroke}%
\pgfsetdash{}{0pt}%
\pgfusepath{stroke}%
\end{pgfscope}%
\begin{pgfscope}%
\pgfpathrectangle{\pgfqpoint{1.374500in}{0.082500in}}{\pgfqpoint{2.419000in}{2.419000in}}%
\pgfusepath{clip}%
\pgfsetbuttcap%
\pgfsetroundjoin%
\pgfsetlinewidth{1.505625pt}%
\definecolor{currentstroke}{rgb}{0.000000,0.000000,0.000000}%
\pgfsetstrokecolor{currentstroke}%
\pgfsetdash{}{0pt}%
\pgfusepath{stroke}%
\end{pgfscope}%
\begin{pgfscope}%
\pgfpathrectangle{\pgfqpoint{1.374500in}{0.082500in}}{\pgfqpoint{2.419000in}{2.419000in}}%
\pgfusepath{clip}%
\pgfsetbuttcap%
\pgfsetroundjoin%
\pgfsetlinewidth{1.505625pt}%
\definecolor{currentstroke}{rgb}{0.000000,0.000000,0.000000}%
\pgfsetstrokecolor{currentstroke}%
\pgfsetdash{}{0pt}%
\pgfusepath{stroke}%
\end{pgfscope}%
\begin{pgfscope}%
\pgfpathrectangle{\pgfqpoint{1.374500in}{0.082500in}}{\pgfqpoint{2.419000in}{2.419000in}}%
\pgfusepath{clip}%
\pgfsetbuttcap%
\pgfsetroundjoin%
\pgfsetlinewidth{1.505625pt}%
\definecolor{currentstroke}{rgb}{0.000000,0.000000,0.000000}%
\pgfsetstrokecolor{currentstroke}%
\pgfsetdash{}{0pt}%
\pgfusepath{stroke}%
\end{pgfscope}%
\begin{pgfscope}%
\pgfpathrectangle{\pgfqpoint{1.374500in}{0.082500in}}{\pgfqpoint{2.419000in}{2.419000in}}%
\pgfusepath{clip}%
\pgfsetbuttcap%
\pgfsetroundjoin%
\pgfsetlinewidth{1.505625pt}%
\definecolor{currentstroke}{rgb}{0.000000,0.000000,0.000000}%
\pgfsetstrokecolor{currentstroke}%
\pgfsetdash{}{0pt}%
\pgfusepath{stroke}%
\end{pgfscope}%
\begin{pgfscope}%
\pgfpathrectangle{\pgfqpoint{1.374500in}{0.082500in}}{\pgfqpoint{2.419000in}{2.419000in}}%
\pgfusepath{clip}%
\pgfsetbuttcap%
\pgfsetroundjoin%
\pgfsetlinewidth{1.505625pt}%
\definecolor{currentstroke}{rgb}{0.000000,0.000000,0.000000}%
\pgfsetstrokecolor{currentstroke}%
\pgfsetdash{}{0pt}%
\pgfusepath{stroke}%
\end{pgfscope}%
\begin{pgfscope}%
\pgfpathrectangle{\pgfqpoint{1.374500in}{0.082500in}}{\pgfqpoint{2.419000in}{2.419000in}}%
\pgfusepath{clip}%
\pgfsetbuttcap%
\pgfsetroundjoin%
\pgfsetlinewidth{1.505625pt}%
\definecolor{currentstroke}{rgb}{0.000000,0.000000,0.000000}%
\pgfsetstrokecolor{currentstroke}%
\pgfsetdash{}{0pt}%
\pgfusepath{stroke}%
\end{pgfscope}%
\begin{pgfscope}%
\pgfpathrectangle{\pgfqpoint{1.374500in}{0.082500in}}{\pgfqpoint{2.419000in}{2.419000in}}%
\pgfusepath{clip}%
\pgfsetbuttcap%
\pgfsetroundjoin%
\pgfsetlinewidth{1.505625pt}%
\definecolor{currentstroke}{rgb}{0.000000,0.000000,0.000000}%
\pgfsetstrokecolor{currentstroke}%
\pgfsetdash{}{0pt}%
\pgfusepath{stroke}%
\end{pgfscope}%
\begin{pgfscope}%
\pgfpathrectangle{\pgfqpoint{1.374500in}{0.082500in}}{\pgfqpoint{2.419000in}{2.419000in}}%
\pgfusepath{clip}%
\pgfsetbuttcap%
\pgfsetroundjoin%
\pgfsetlinewidth{1.505625pt}%
\definecolor{currentstroke}{rgb}{0.000000,0.000000,0.000000}%
\pgfsetstrokecolor{currentstroke}%
\pgfsetdash{}{0pt}%
\pgfusepath{stroke}%
\end{pgfscope}%
\begin{pgfscope}%
\pgfpathrectangle{\pgfqpoint{1.374500in}{0.082500in}}{\pgfqpoint{2.419000in}{2.419000in}}%
\pgfusepath{clip}%
\pgfsetbuttcap%
\pgfsetroundjoin%
\pgfsetlinewidth{1.505625pt}%
\definecolor{currentstroke}{rgb}{0.000000,0.000000,0.000000}%
\pgfsetstrokecolor{currentstroke}%
\pgfsetdash{}{0pt}%
\pgfusepath{stroke}%
\end{pgfscope}%
\begin{pgfscope}%
\pgfpathrectangle{\pgfqpoint{1.374500in}{0.082500in}}{\pgfqpoint{2.419000in}{2.419000in}}%
\pgfusepath{clip}%
\pgfsetbuttcap%
\pgfsetroundjoin%
\pgfsetlinewidth{1.505625pt}%
\definecolor{currentstroke}{rgb}{0.000000,0.000000,0.000000}%
\pgfsetstrokecolor{currentstroke}%
\pgfsetdash{}{0pt}%
\pgfusepath{stroke}%
\end{pgfscope}%
\begin{pgfscope}%
\pgfpathrectangle{\pgfqpoint{1.374500in}{0.082500in}}{\pgfqpoint{2.419000in}{2.419000in}}%
\pgfusepath{clip}%
\pgfsetbuttcap%
\pgfsetroundjoin%
\pgfsetlinewidth{1.505625pt}%
\definecolor{currentstroke}{rgb}{0.000000,0.000000,0.000000}%
\pgfsetstrokecolor{currentstroke}%
\pgfsetdash{}{0pt}%
\pgfusepath{stroke}%
\end{pgfscope}%
\begin{pgfscope}%
\pgfpathrectangle{\pgfqpoint{1.374500in}{0.082500in}}{\pgfqpoint{2.419000in}{2.419000in}}%
\pgfusepath{clip}%
\pgfsetbuttcap%
\pgfsetroundjoin%
\pgfsetlinewidth{1.505625pt}%
\definecolor{currentstroke}{rgb}{0.000000,0.000000,0.000000}%
\pgfsetstrokecolor{currentstroke}%
\pgfsetdash{}{0pt}%
\pgfusepath{stroke}%
\end{pgfscope}%
\begin{pgfscope}%
\pgfpathrectangle{\pgfqpoint{1.374500in}{0.082500in}}{\pgfqpoint{2.419000in}{2.419000in}}%
\pgfusepath{clip}%
\pgfsetbuttcap%
\pgfsetroundjoin%
\pgfsetlinewidth{1.505625pt}%
\definecolor{currentstroke}{rgb}{0.000000,0.000000,0.000000}%
\pgfsetstrokecolor{currentstroke}%
\pgfsetdash{}{0pt}%
\pgfusepath{stroke}%
\end{pgfscope}%
\begin{pgfscope}%
\pgfpathrectangle{\pgfqpoint{1.374500in}{0.082500in}}{\pgfqpoint{2.419000in}{2.419000in}}%
\pgfusepath{clip}%
\pgfsetbuttcap%
\pgfsetroundjoin%
\pgfsetlinewidth{1.505625pt}%
\definecolor{currentstroke}{rgb}{0.000000,0.000000,0.000000}%
\pgfsetstrokecolor{currentstroke}%
\pgfsetdash{}{0pt}%
\pgfusepath{stroke}%
\end{pgfscope}%
\begin{pgfscope}%
\pgfpathrectangle{\pgfqpoint{1.374500in}{0.082500in}}{\pgfqpoint{2.419000in}{2.419000in}}%
\pgfusepath{clip}%
\pgfsetbuttcap%
\pgfsetroundjoin%
\pgfsetlinewidth{1.505625pt}%
\definecolor{currentstroke}{rgb}{0.000000,0.000000,0.000000}%
\pgfsetstrokecolor{currentstroke}%
\pgfsetdash{}{0pt}%
\pgfusepath{stroke}%
\end{pgfscope}%
\begin{pgfscope}%
\pgfpathrectangle{\pgfqpoint{1.374500in}{0.082500in}}{\pgfqpoint{2.419000in}{2.419000in}}%
\pgfusepath{clip}%
\pgfsetbuttcap%
\pgfsetroundjoin%
\pgfsetlinewidth{1.505625pt}%
\definecolor{currentstroke}{rgb}{0.000000,0.000000,0.000000}%
\pgfsetstrokecolor{currentstroke}%
\pgfsetdash{}{0pt}%
\pgfusepath{stroke}%
\end{pgfscope}%
\begin{pgfscope}%
\pgfpathrectangle{\pgfqpoint{1.374500in}{0.082500in}}{\pgfqpoint{2.419000in}{2.419000in}}%
\pgfusepath{clip}%
\pgfsetbuttcap%
\pgfsetroundjoin%
\pgfsetlinewidth{1.505625pt}%
\definecolor{currentstroke}{rgb}{0.000000,0.000000,0.000000}%
\pgfsetstrokecolor{currentstroke}%
\pgfsetdash{}{0pt}%
\pgfusepath{stroke}%
\end{pgfscope}%
\begin{pgfscope}%
\pgfpathrectangle{\pgfqpoint{1.374500in}{0.082500in}}{\pgfqpoint{2.419000in}{2.419000in}}%
\pgfusepath{clip}%
\pgfsetbuttcap%
\pgfsetroundjoin%
\pgfsetlinewidth{1.505625pt}%
\definecolor{currentstroke}{rgb}{0.000000,0.000000,0.000000}%
\pgfsetstrokecolor{currentstroke}%
\pgfsetdash{}{0pt}%
\pgfusepath{stroke}%
\end{pgfscope}%
\begin{pgfscope}%
\pgfpathrectangle{\pgfqpoint{1.374500in}{0.082500in}}{\pgfqpoint{2.419000in}{2.419000in}}%
\pgfusepath{clip}%
\pgfsetbuttcap%
\pgfsetroundjoin%
\pgfsetlinewidth{1.505625pt}%
\definecolor{currentstroke}{rgb}{0.000000,0.000000,0.000000}%
\pgfsetstrokecolor{currentstroke}%
\pgfsetdash{}{0pt}%
\pgfusepath{stroke}%
\end{pgfscope}%
\begin{pgfscope}%
\pgfpathrectangle{\pgfqpoint{1.374500in}{0.082500in}}{\pgfqpoint{2.419000in}{2.419000in}}%
\pgfusepath{clip}%
\pgfsetbuttcap%
\pgfsetroundjoin%
\pgfsetlinewidth{1.505625pt}%
\definecolor{currentstroke}{rgb}{0.000000,0.000000,0.000000}%
\pgfsetstrokecolor{currentstroke}%
\pgfsetdash{}{0pt}%
\pgfusepath{stroke}%
\end{pgfscope}%
\begin{pgfscope}%
\pgfpathrectangle{\pgfqpoint{1.374500in}{0.082500in}}{\pgfqpoint{2.419000in}{2.419000in}}%
\pgfusepath{clip}%
\pgfsetbuttcap%
\pgfsetroundjoin%
\pgfsetlinewidth{1.505625pt}%
\definecolor{currentstroke}{rgb}{0.000000,0.000000,0.000000}%
\pgfsetstrokecolor{currentstroke}%
\pgfsetdash{}{0pt}%
\pgfusepath{stroke}%
\end{pgfscope}%
\begin{pgfscope}%
\pgfpathrectangle{\pgfqpoint{1.374500in}{0.082500in}}{\pgfqpoint{2.419000in}{2.419000in}}%
\pgfusepath{clip}%
\pgfsetbuttcap%
\pgfsetroundjoin%
\pgfsetlinewidth{1.505625pt}%
\definecolor{currentstroke}{rgb}{0.000000,0.000000,0.000000}%
\pgfsetstrokecolor{currentstroke}%
\pgfsetdash{}{0pt}%
\pgfusepath{stroke}%
\end{pgfscope}%
\begin{pgfscope}%
\pgfpathrectangle{\pgfqpoint{1.374500in}{0.082500in}}{\pgfqpoint{2.419000in}{2.419000in}}%
\pgfusepath{clip}%
\pgfsetbuttcap%
\pgfsetroundjoin%
\pgfsetlinewidth{1.505625pt}%
\definecolor{currentstroke}{rgb}{0.000000,0.000000,0.000000}%
\pgfsetstrokecolor{currentstroke}%
\pgfsetdash{}{0pt}%
\pgfusepath{stroke}%
\end{pgfscope}%
\begin{pgfscope}%
\pgfpathrectangle{\pgfqpoint{1.374500in}{0.082500in}}{\pgfqpoint{2.419000in}{2.419000in}}%
\pgfusepath{clip}%
\pgfsetbuttcap%
\pgfsetroundjoin%
\pgfsetlinewidth{1.505625pt}%
\definecolor{currentstroke}{rgb}{0.000000,0.000000,0.000000}%
\pgfsetstrokecolor{currentstroke}%
\pgfsetdash{}{0pt}%
\pgfusepath{stroke}%
\end{pgfscope}%
\begin{pgfscope}%
\pgfpathrectangle{\pgfqpoint{1.374500in}{0.082500in}}{\pgfqpoint{2.419000in}{2.419000in}}%
\pgfusepath{clip}%
\pgfsetbuttcap%
\pgfsetroundjoin%
\pgfsetlinewidth{1.505625pt}%
\definecolor{currentstroke}{rgb}{0.000000,0.000000,0.000000}%
\pgfsetstrokecolor{currentstroke}%
\pgfsetdash{}{0pt}%
\pgfusepath{stroke}%
\end{pgfscope}%
\begin{pgfscope}%
\pgfpathrectangle{\pgfqpoint{1.374500in}{0.082500in}}{\pgfqpoint{2.419000in}{2.419000in}}%
\pgfusepath{clip}%
\pgfsetbuttcap%
\pgfsetroundjoin%
\pgfsetlinewidth{1.505625pt}%
\definecolor{currentstroke}{rgb}{0.000000,0.000000,0.000000}%
\pgfsetstrokecolor{currentstroke}%
\pgfsetdash{}{0pt}%
\pgfusepath{stroke}%
\end{pgfscope}%
\begin{pgfscope}%
\pgfpathrectangle{\pgfqpoint{1.374500in}{0.082500in}}{\pgfqpoint{2.419000in}{2.419000in}}%
\pgfusepath{clip}%
\pgfsetbuttcap%
\pgfsetroundjoin%
\pgfsetlinewidth{1.505625pt}%
\definecolor{currentstroke}{rgb}{0.000000,0.000000,0.000000}%
\pgfsetstrokecolor{currentstroke}%
\pgfsetdash{}{0pt}%
\pgfusepath{stroke}%
\end{pgfscope}%
\begin{pgfscope}%
\pgfpathrectangle{\pgfqpoint{1.374500in}{0.082500in}}{\pgfqpoint{2.419000in}{2.419000in}}%
\pgfusepath{clip}%
\pgfsetbuttcap%
\pgfsetroundjoin%
\pgfsetlinewidth{1.505625pt}%
\definecolor{currentstroke}{rgb}{0.000000,0.000000,0.000000}%
\pgfsetstrokecolor{currentstroke}%
\pgfsetdash{}{0pt}%
\pgfusepath{stroke}%
\end{pgfscope}%
\begin{pgfscope}%
\pgfpathrectangle{\pgfqpoint{1.374500in}{0.082500in}}{\pgfqpoint{2.419000in}{2.419000in}}%
\pgfusepath{clip}%
\pgfsetbuttcap%
\pgfsetroundjoin%
\pgfsetlinewidth{1.505625pt}%
\definecolor{currentstroke}{rgb}{0.000000,0.000000,0.000000}%
\pgfsetstrokecolor{currentstroke}%
\pgfsetdash{}{0pt}%
\pgfusepath{stroke}%
\end{pgfscope}%
\begin{pgfscope}%
\pgfpathrectangle{\pgfqpoint{1.374500in}{0.082500in}}{\pgfqpoint{2.419000in}{2.419000in}}%
\pgfusepath{clip}%
\pgfsetbuttcap%
\pgfsetroundjoin%
\pgfsetlinewidth{1.505625pt}%
\definecolor{currentstroke}{rgb}{0.000000,0.000000,0.000000}%
\pgfsetstrokecolor{currentstroke}%
\pgfsetdash{}{0pt}%
\pgfusepath{stroke}%
\end{pgfscope}%
\begin{pgfscope}%
\pgfpathrectangle{\pgfqpoint{1.374500in}{0.082500in}}{\pgfqpoint{2.419000in}{2.419000in}}%
\pgfusepath{clip}%
\pgfsetbuttcap%
\pgfsetroundjoin%
\pgfsetlinewidth{1.505625pt}%
\definecolor{currentstroke}{rgb}{0.000000,0.000000,0.000000}%
\pgfsetstrokecolor{currentstroke}%
\pgfsetdash{}{0pt}%
\pgfusepath{stroke}%
\end{pgfscope}%
\begin{pgfscope}%
\pgfpathrectangle{\pgfqpoint{1.374500in}{0.082500in}}{\pgfqpoint{2.419000in}{2.419000in}}%
\pgfusepath{clip}%
\pgfsetbuttcap%
\pgfsetroundjoin%
\pgfsetlinewidth{1.505625pt}%
\definecolor{currentstroke}{rgb}{0.000000,0.000000,0.000000}%
\pgfsetstrokecolor{currentstroke}%
\pgfsetdash{}{0pt}%
\pgfusepath{stroke}%
\end{pgfscope}%
\begin{pgfscope}%
\pgfpathrectangle{\pgfqpoint{1.374500in}{0.082500in}}{\pgfqpoint{2.419000in}{2.419000in}}%
\pgfusepath{clip}%
\pgfsetbuttcap%
\pgfsetroundjoin%
\pgfsetlinewidth{1.505625pt}%
\definecolor{currentstroke}{rgb}{0.000000,0.000000,0.000000}%
\pgfsetstrokecolor{currentstroke}%
\pgfsetdash{}{0pt}%
\pgfusepath{stroke}%
\end{pgfscope}%
\begin{pgfscope}%
\pgfpathrectangle{\pgfqpoint{1.374500in}{0.082500in}}{\pgfqpoint{2.419000in}{2.419000in}}%
\pgfusepath{clip}%
\pgfsetbuttcap%
\pgfsetroundjoin%
\pgfsetlinewidth{1.505625pt}%
\definecolor{currentstroke}{rgb}{0.000000,0.000000,0.000000}%
\pgfsetstrokecolor{currentstroke}%
\pgfsetdash{}{0pt}%
\pgfusepath{stroke}%
\end{pgfscope}%
\begin{pgfscope}%
\pgfpathrectangle{\pgfqpoint{1.374500in}{0.082500in}}{\pgfqpoint{2.419000in}{2.419000in}}%
\pgfusepath{clip}%
\pgfsetbuttcap%
\pgfsetroundjoin%
\pgfsetlinewidth{1.505625pt}%
\definecolor{currentstroke}{rgb}{0.000000,0.000000,0.000000}%
\pgfsetstrokecolor{currentstroke}%
\pgfsetdash{}{0pt}%
\pgfusepath{stroke}%
\end{pgfscope}%
\begin{pgfscope}%
\pgfpathrectangle{\pgfqpoint{1.374500in}{0.082500in}}{\pgfqpoint{2.419000in}{2.419000in}}%
\pgfusepath{clip}%
\pgfsetbuttcap%
\pgfsetroundjoin%
\pgfsetlinewidth{1.505625pt}%
\definecolor{currentstroke}{rgb}{0.000000,0.000000,0.000000}%
\pgfsetstrokecolor{currentstroke}%
\pgfsetdash{}{0pt}%
\pgfusepath{stroke}%
\end{pgfscope}%
\begin{pgfscope}%
\pgfpathrectangle{\pgfqpoint{1.374500in}{0.082500in}}{\pgfqpoint{2.419000in}{2.419000in}}%
\pgfusepath{clip}%
\pgfsetbuttcap%
\pgfsetroundjoin%
\pgfsetlinewidth{1.505625pt}%
\definecolor{currentstroke}{rgb}{0.000000,0.000000,0.000000}%
\pgfsetstrokecolor{currentstroke}%
\pgfsetdash{}{0pt}%
\pgfusepath{stroke}%
\end{pgfscope}%
\begin{pgfscope}%
\pgfpathrectangle{\pgfqpoint{1.374500in}{0.082500in}}{\pgfqpoint{2.419000in}{2.419000in}}%
\pgfusepath{clip}%
\pgfsetbuttcap%
\pgfsetroundjoin%
\pgfsetlinewidth{1.505625pt}%
\definecolor{currentstroke}{rgb}{0.000000,0.000000,0.000000}%
\pgfsetstrokecolor{currentstroke}%
\pgfsetdash{}{0pt}%
\pgfusepath{stroke}%
\end{pgfscope}%
\begin{pgfscope}%
\pgfpathrectangle{\pgfqpoint{1.374500in}{0.082500in}}{\pgfqpoint{2.419000in}{2.419000in}}%
\pgfusepath{clip}%
\pgfsetbuttcap%
\pgfsetroundjoin%
\pgfsetlinewidth{1.505625pt}%
\definecolor{currentstroke}{rgb}{0.000000,0.000000,0.000000}%
\pgfsetstrokecolor{currentstroke}%
\pgfsetdash{}{0pt}%
\pgfusepath{stroke}%
\end{pgfscope}%
\begin{pgfscope}%
\pgfpathrectangle{\pgfqpoint{1.374500in}{0.082500in}}{\pgfqpoint{2.419000in}{2.419000in}}%
\pgfusepath{clip}%
\pgfsetbuttcap%
\pgfsetroundjoin%
\pgfsetlinewidth{1.505625pt}%
\definecolor{currentstroke}{rgb}{0.000000,0.000000,0.000000}%
\pgfsetstrokecolor{currentstroke}%
\pgfsetdash{}{0pt}%
\pgfusepath{stroke}%
\end{pgfscope}%
\begin{pgfscope}%
\pgfpathrectangle{\pgfqpoint{1.374500in}{0.082500in}}{\pgfqpoint{2.419000in}{2.419000in}}%
\pgfusepath{clip}%
\pgfsetbuttcap%
\pgfsetroundjoin%
\pgfsetlinewidth{1.505625pt}%
\definecolor{currentstroke}{rgb}{0.000000,0.000000,0.000000}%
\pgfsetstrokecolor{currentstroke}%
\pgfsetdash{}{0pt}%
\pgfusepath{stroke}%
\end{pgfscope}%
\begin{pgfscope}%
\pgfpathrectangle{\pgfqpoint{1.374500in}{0.082500in}}{\pgfqpoint{2.419000in}{2.419000in}}%
\pgfusepath{clip}%
\pgfsetbuttcap%
\pgfsetroundjoin%
\pgfsetlinewidth{1.505625pt}%
\definecolor{currentstroke}{rgb}{0.000000,0.000000,0.000000}%
\pgfsetstrokecolor{currentstroke}%
\pgfsetdash{}{0pt}%
\pgfusepath{stroke}%
\end{pgfscope}%
\begin{pgfscope}%
\pgfpathrectangle{\pgfqpoint{1.374500in}{0.082500in}}{\pgfqpoint{2.419000in}{2.419000in}}%
\pgfusepath{clip}%
\pgfsetbuttcap%
\pgfsetroundjoin%
\pgfsetlinewidth{1.505625pt}%
\definecolor{currentstroke}{rgb}{0.000000,0.000000,0.000000}%
\pgfsetstrokecolor{currentstroke}%
\pgfsetdash{}{0pt}%
\pgfusepath{stroke}%
\end{pgfscope}%
\begin{pgfscope}%
\pgfpathrectangle{\pgfqpoint{1.374500in}{0.082500in}}{\pgfqpoint{2.419000in}{2.419000in}}%
\pgfusepath{clip}%
\pgfsetbuttcap%
\pgfsetroundjoin%
\pgfsetlinewidth{1.505625pt}%
\definecolor{currentstroke}{rgb}{0.000000,0.000000,0.000000}%
\pgfsetstrokecolor{currentstroke}%
\pgfsetdash{}{0pt}%
\pgfusepath{stroke}%
\end{pgfscope}%
\begin{pgfscope}%
\pgfpathrectangle{\pgfqpoint{1.374500in}{0.082500in}}{\pgfqpoint{2.419000in}{2.419000in}}%
\pgfusepath{clip}%
\pgfsetbuttcap%
\pgfsetroundjoin%
\pgfsetlinewidth{1.505625pt}%
\definecolor{currentstroke}{rgb}{0.000000,0.000000,0.000000}%
\pgfsetstrokecolor{currentstroke}%
\pgfsetdash{}{0pt}%
\pgfusepath{stroke}%
\end{pgfscope}%
\begin{pgfscope}%
\pgfpathrectangle{\pgfqpoint{1.374500in}{0.082500in}}{\pgfqpoint{2.419000in}{2.419000in}}%
\pgfusepath{clip}%
\pgfsetbuttcap%
\pgfsetroundjoin%
\pgfsetlinewidth{1.505625pt}%
\definecolor{currentstroke}{rgb}{0.000000,0.000000,0.000000}%
\pgfsetstrokecolor{currentstroke}%
\pgfsetdash{}{0pt}%
\pgfusepath{stroke}%
\end{pgfscope}%
\begin{pgfscope}%
\pgfpathrectangle{\pgfqpoint{1.374500in}{0.082500in}}{\pgfqpoint{2.419000in}{2.419000in}}%
\pgfusepath{clip}%
\pgfsetbuttcap%
\pgfsetroundjoin%
\pgfsetlinewidth{1.505625pt}%
\definecolor{currentstroke}{rgb}{0.000000,0.000000,0.000000}%
\pgfsetstrokecolor{currentstroke}%
\pgfsetdash{}{0pt}%
\pgfusepath{stroke}%
\end{pgfscope}%
\begin{pgfscope}%
\pgfpathrectangle{\pgfqpoint{1.374500in}{0.082500in}}{\pgfqpoint{2.419000in}{2.419000in}}%
\pgfusepath{clip}%
\pgfsetbuttcap%
\pgfsetroundjoin%
\pgfsetlinewidth{1.505625pt}%
\definecolor{currentstroke}{rgb}{0.000000,0.000000,0.000000}%
\pgfsetstrokecolor{currentstroke}%
\pgfsetdash{}{0pt}%
\pgfusepath{stroke}%
\end{pgfscope}%
\begin{pgfscope}%
\pgfpathrectangle{\pgfqpoint{1.374500in}{0.082500in}}{\pgfqpoint{2.419000in}{2.419000in}}%
\pgfusepath{clip}%
\pgfsetbuttcap%
\pgfsetroundjoin%
\pgfsetlinewidth{1.505625pt}%
\definecolor{currentstroke}{rgb}{0.000000,0.000000,0.000000}%
\pgfsetstrokecolor{currentstroke}%
\pgfsetdash{}{0pt}%
\pgfusepath{stroke}%
\end{pgfscope}%
\begin{pgfscope}%
\pgfpathrectangle{\pgfqpoint{1.374500in}{0.082500in}}{\pgfqpoint{2.419000in}{2.419000in}}%
\pgfusepath{clip}%
\pgfsetbuttcap%
\pgfsetroundjoin%
\pgfsetlinewidth{1.505625pt}%
\definecolor{currentstroke}{rgb}{0.000000,0.000000,0.000000}%
\pgfsetstrokecolor{currentstroke}%
\pgfsetdash{}{0pt}%
\pgfusepath{stroke}%
\end{pgfscope}%
\begin{pgfscope}%
\pgfpathrectangle{\pgfqpoint{1.374500in}{0.082500in}}{\pgfqpoint{2.419000in}{2.419000in}}%
\pgfusepath{clip}%
\pgfsetbuttcap%
\pgfsetroundjoin%
\pgfsetlinewidth{1.505625pt}%
\definecolor{currentstroke}{rgb}{0.000000,0.000000,0.000000}%
\pgfsetstrokecolor{currentstroke}%
\pgfsetdash{}{0pt}%
\pgfusepath{stroke}%
\end{pgfscope}%
\begin{pgfscope}%
\pgfpathrectangle{\pgfqpoint{1.374500in}{0.082500in}}{\pgfqpoint{2.419000in}{2.419000in}}%
\pgfusepath{clip}%
\pgfsetbuttcap%
\pgfsetroundjoin%
\pgfsetlinewidth{1.505625pt}%
\definecolor{currentstroke}{rgb}{0.000000,0.000000,0.000000}%
\pgfsetstrokecolor{currentstroke}%
\pgfsetdash{}{0pt}%
\pgfusepath{stroke}%
\end{pgfscope}%
\begin{pgfscope}%
\pgfpathrectangle{\pgfqpoint{1.374500in}{0.082500in}}{\pgfqpoint{2.419000in}{2.419000in}}%
\pgfusepath{clip}%
\pgfsetbuttcap%
\pgfsetroundjoin%
\pgfsetlinewidth{1.505625pt}%
\definecolor{currentstroke}{rgb}{0.000000,0.000000,0.000000}%
\pgfsetstrokecolor{currentstroke}%
\pgfsetdash{}{0pt}%
\pgfusepath{stroke}%
\end{pgfscope}%
\begin{pgfscope}%
\pgfpathrectangle{\pgfqpoint{1.374500in}{0.082500in}}{\pgfqpoint{2.419000in}{2.419000in}}%
\pgfusepath{clip}%
\pgfsetbuttcap%
\pgfsetroundjoin%
\pgfsetlinewidth{1.505625pt}%
\definecolor{currentstroke}{rgb}{0.000000,0.000000,0.000000}%
\pgfsetstrokecolor{currentstroke}%
\pgfsetdash{}{0pt}%
\pgfusepath{stroke}%
\end{pgfscope}%
\begin{pgfscope}%
\pgfpathrectangle{\pgfqpoint{1.374500in}{0.082500in}}{\pgfqpoint{2.419000in}{2.419000in}}%
\pgfusepath{clip}%
\pgfsetbuttcap%
\pgfsetroundjoin%
\pgfsetlinewidth{1.505625pt}%
\definecolor{currentstroke}{rgb}{0.000000,0.000000,0.000000}%
\pgfsetstrokecolor{currentstroke}%
\pgfsetdash{}{0pt}%
\pgfusepath{stroke}%
\end{pgfscope}%
\begin{pgfscope}%
\pgfpathrectangle{\pgfqpoint{1.374500in}{0.082500in}}{\pgfqpoint{2.419000in}{2.419000in}}%
\pgfusepath{clip}%
\pgfsetbuttcap%
\pgfsetroundjoin%
\pgfsetlinewidth{1.505625pt}%
\definecolor{currentstroke}{rgb}{0.000000,0.000000,0.000000}%
\pgfsetstrokecolor{currentstroke}%
\pgfsetdash{}{0pt}%
\pgfusepath{stroke}%
\end{pgfscope}%
\begin{pgfscope}%
\pgfpathrectangle{\pgfqpoint{1.374500in}{0.082500in}}{\pgfqpoint{2.419000in}{2.419000in}}%
\pgfusepath{clip}%
\pgfsetbuttcap%
\pgfsetroundjoin%
\pgfsetlinewidth{1.505625pt}%
\definecolor{currentstroke}{rgb}{0.000000,0.000000,0.000000}%
\pgfsetstrokecolor{currentstroke}%
\pgfsetdash{}{0pt}%
\pgfusepath{stroke}%
\end{pgfscope}%
\begin{pgfscope}%
\pgfpathrectangle{\pgfqpoint{1.374500in}{0.082500in}}{\pgfqpoint{2.419000in}{2.419000in}}%
\pgfusepath{clip}%
\pgfsetbuttcap%
\pgfsetroundjoin%
\pgfsetlinewidth{1.505625pt}%
\definecolor{currentstroke}{rgb}{0.000000,0.000000,0.000000}%
\pgfsetstrokecolor{currentstroke}%
\pgfsetdash{}{0pt}%
\pgfusepath{stroke}%
\end{pgfscope}%
\begin{pgfscope}%
\pgfpathrectangle{\pgfqpoint{1.374500in}{0.082500in}}{\pgfqpoint{2.419000in}{2.419000in}}%
\pgfusepath{clip}%
\pgfsetbuttcap%
\pgfsetroundjoin%
\pgfsetlinewidth{1.505625pt}%
\definecolor{currentstroke}{rgb}{0.000000,0.000000,0.000000}%
\pgfsetstrokecolor{currentstroke}%
\pgfsetdash{}{0pt}%
\pgfusepath{stroke}%
\end{pgfscope}%
\begin{pgfscope}%
\pgfpathrectangle{\pgfqpoint{1.374500in}{0.082500in}}{\pgfqpoint{2.419000in}{2.419000in}}%
\pgfusepath{clip}%
\pgfsetbuttcap%
\pgfsetroundjoin%
\pgfsetlinewidth{1.505625pt}%
\definecolor{currentstroke}{rgb}{0.000000,0.000000,0.000000}%
\pgfsetstrokecolor{currentstroke}%
\pgfsetdash{}{0pt}%
\pgfusepath{stroke}%
\end{pgfscope}%
\begin{pgfscope}%
\pgfpathrectangle{\pgfqpoint{1.374500in}{0.082500in}}{\pgfqpoint{2.419000in}{2.419000in}}%
\pgfusepath{clip}%
\pgfsetbuttcap%
\pgfsetroundjoin%
\pgfsetlinewidth{1.505625pt}%
\definecolor{currentstroke}{rgb}{0.000000,0.000000,0.000000}%
\pgfsetstrokecolor{currentstroke}%
\pgfsetdash{}{0pt}%
\pgfusepath{stroke}%
\end{pgfscope}%
\begin{pgfscope}%
\pgfpathrectangle{\pgfqpoint{1.374500in}{0.082500in}}{\pgfqpoint{2.419000in}{2.419000in}}%
\pgfusepath{clip}%
\pgfsetbuttcap%
\pgfsetroundjoin%
\pgfsetlinewidth{1.505625pt}%
\definecolor{currentstroke}{rgb}{0.000000,0.000000,0.000000}%
\pgfsetstrokecolor{currentstroke}%
\pgfsetdash{}{0pt}%
\pgfusepath{stroke}%
\end{pgfscope}%
\begin{pgfscope}%
\pgfpathrectangle{\pgfqpoint{1.374500in}{0.082500in}}{\pgfqpoint{2.419000in}{2.419000in}}%
\pgfusepath{clip}%
\pgfsetbuttcap%
\pgfsetroundjoin%
\pgfsetlinewidth{1.505625pt}%
\definecolor{currentstroke}{rgb}{0.000000,0.000000,0.000000}%
\pgfsetstrokecolor{currentstroke}%
\pgfsetdash{}{0pt}%
\pgfusepath{stroke}%
\end{pgfscope}%
\begin{pgfscope}%
\pgfpathrectangle{\pgfqpoint{1.374500in}{0.082500in}}{\pgfqpoint{2.419000in}{2.419000in}}%
\pgfusepath{clip}%
\pgfsetbuttcap%
\pgfsetroundjoin%
\pgfsetlinewidth{1.505625pt}%
\definecolor{currentstroke}{rgb}{0.000000,0.000000,0.000000}%
\pgfsetstrokecolor{currentstroke}%
\pgfsetdash{}{0pt}%
\pgfusepath{stroke}%
\end{pgfscope}%
\begin{pgfscope}%
\pgfpathrectangle{\pgfqpoint{1.374500in}{0.082500in}}{\pgfqpoint{2.419000in}{2.419000in}}%
\pgfusepath{clip}%
\pgfsetbuttcap%
\pgfsetroundjoin%
\pgfsetlinewidth{1.505625pt}%
\definecolor{currentstroke}{rgb}{0.000000,0.000000,0.000000}%
\pgfsetstrokecolor{currentstroke}%
\pgfsetdash{}{0pt}%
\pgfusepath{stroke}%
\end{pgfscope}%
\begin{pgfscope}%
\pgfpathrectangle{\pgfqpoint{1.374500in}{0.082500in}}{\pgfqpoint{2.419000in}{2.419000in}}%
\pgfusepath{clip}%
\pgfsetbuttcap%
\pgfsetroundjoin%
\pgfsetlinewidth{1.505625pt}%
\definecolor{currentstroke}{rgb}{0.000000,0.000000,0.000000}%
\pgfsetstrokecolor{currentstroke}%
\pgfsetdash{}{0pt}%
\pgfusepath{stroke}%
\end{pgfscope}%
\begin{pgfscope}%
\pgfpathrectangle{\pgfqpoint{1.374500in}{0.082500in}}{\pgfqpoint{2.419000in}{2.419000in}}%
\pgfusepath{clip}%
\pgfsetbuttcap%
\pgfsetroundjoin%
\pgfsetlinewidth{1.505625pt}%
\definecolor{currentstroke}{rgb}{0.000000,0.000000,0.000000}%
\pgfsetstrokecolor{currentstroke}%
\pgfsetdash{}{0pt}%
\pgfusepath{stroke}%
\end{pgfscope}%
\begin{pgfscope}%
\pgfpathrectangle{\pgfqpoint{1.374500in}{0.082500in}}{\pgfqpoint{2.419000in}{2.419000in}}%
\pgfusepath{clip}%
\pgfsetbuttcap%
\pgfsetroundjoin%
\pgfsetlinewidth{1.505625pt}%
\definecolor{currentstroke}{rgb}{0.000000,0.000000,0.000000}%
\pgfsetstrokecolor{currentstroke}%
\pgfsetdash{}{0pt}%
\pgfusepath{stroke}%
\end{pgfscope}%
\begin{pgfscope}%
\pgfpathrectangle{\pgfqpoint{1.374500in}{0.082500in}}{\pgfqpoint{2.419000in}{2.419000in}}%
\pgfusepath{clip}%
\pgfsetbuttcap%
\pgfsetroundjoin%
\pgfsetlinewidth{1.505625pt}%
\definecolor{currentstroke}{rgb}{0.000000,0.000000,0.000000}%
\pgfsetstrokecolor{currentstroke}%
\pgfsetdash{}{0pt}%
\pgfusepath{stroke}%
\end{pgfscope}%
\begin{pgfscope}%
\pgfpathrectangle{\pgfqpoint{1.374500in}{0.082500in}}{\pgfqpoint{2.419000in}{2.419000in}}%
\pgfusepath{clip}%
\pgfsetbuttcap%
\pgfsetroundjoin%
\pgfsetlinewidth{1.505625pt}%
\definecolor{currentstroke}{rgb}{0.000000,0.000000,0.000000}%
\pgfsetstrokecolor{currentstroke}%
\pgfsetdash{}{0pt}%
\pgfusepath{stroke}%
\end{pgfscope}%
\begin{pgfscope}%
\pgfpathrectangle{\pgfqpoint{1.374500in}{0.082500in}}{\pgfqpoint{2.419000in}{2.419000in}}%
\pgfusepath{clip}%
\pgfsetbuttcap%
\pgfsetroundjoin%
\pgfsetlinewidth{1.505625pt}%
\definecolor{currentstroke}{rgb}{0.000000,0.000000,0.000000}%
\pgfsetstrokecolor{currentstroke}%
\pgfsetdash{}{0pt}%
\pgfusepath{stroke}%
\end{pgfscope}%
\begin{pgfscope}%
\pgfpathrectangle{\pgfqpoint{1.374500in}{0.082500in}}{\pgfqpoint{2.419000in}{2.419000in}}%
\pgfusepath{clip}%
\pgfsetbuttcap%
\pgfsetroundjoin%
\pgfsetlinewidth{1.505625pt}%
\definecolor{currentstroke}{rgb}{0.000000,0.000000,0.000000}%
\pgfsetstrokecolor{currentstroke}%
\pgfsetdash{}{0pt}%
\pgfusepath{stroke}%
\end{pgfscope}%
\begin{pgfscope}%
\pgfpathrectangle{\pgfqpoint{1.374500in}{0.082500in}}{\pgfqpoint{2.419000in}{2.419000in}}%
\pgfusepath{clip}%
\pgfsetbuttcap%
\pgfsetroundjoin%
\pgfsetlinewidth{1.505625pt}%
\definecolor{currentstroke}{rgb}{0.000000,0.000000,0.000000}%
\pgfsetstrokecolor{currentstroke}%
\pgfsetdash{}{0pt}%
\pgfusepath{stroke}%
\end{pgfscope}%
\begin{pgfscope}%
\pgfpathrectangle{\pgfqpoint{1.374500in}{0.082500in}}{\pgfqpoint{2.419000in}{2.419000in}}%
\pgfusepath{clip}%
\pgfsetbuttcap%
\pgfsetroundjoin%
\pgfsetlinewidth{1.505625pt}%
\definecolor{currentstroke}{rgb}{0.000000,0.000000,0.000000}%
\pgfsetstrokecolor{currentstroke}%
\pgfsetdash{}{0pt}%
\pgfusepath{stroke}%
\end{pgfscope}%
\begin{pgfscope}%
\pgfpathrectangle{\pgfqpoint{1.374500in}{0.082500in}}{\pgfqpoint{2.419000in}{2.419000in}}%
\pgfusepath{clip}%
\pgfsetbuttcap%
\pgfsetroundjoin%
\pgfsetlinewidth{1.505625pt}%
\definecolor{currentstroke}{rgb}{0.000000,0.000000,0.000000}%
\pgfsetstrokecolor{currentstroke}%
\pgfsetdash{}{0pt}%
\pgfusepath{stroke}%
\end{pgfscope}%
\begin{pgfscope}%
\pgfpathrectangle{\pgfqpoint{1.374500in}{0.082500in}}{\pgfqpoint{2.419000in}{2.419000in}}%
\pgfusepath{clip}%
\pgfsetbuttcap%
\pgfsetroundjoin%
\pgfsetlinewidth{1.505625pt}%
\definecolor{currentstroke}{rgb}{0.000000,0.000000,0.000000}%
\pgfsetstrokecolor{currentstroke}%
\pgfsetdash{}{0pt}%
\pgfusepath{stroke}%
\end{pgfscope}%
\begin{pgfscope}%
\pgfpathrectangle{\pgfqpoint{1.374500in}{0.082500in}}{\pgfqpoint{2.419000in}{2.419000in}}%
\pgfusepath{clip}%
\pgfsetbuttcap%
\pgfsetroundjoin%
\pgfsetlinewidth{1.505625pt}%
\definecolor{currentstroke}{rgb}{0.000000,0.000000,0.000000}%
\pgfsetstrokecolor{currentstroke}%
\pgfsetdash{}{0pt}%
\pgfusepath{stroke}%
\end{pgfscope}%
\begin{pgfscope}%
\pgfpathrectangle{\pgfqpoint{1.374500in}{0.082500in}}{\pgfqpoint{2.419000in}{2.419000in}}%
\pgfusepath{clip}%
\pgfsetbuttcap%
\pgfsetroundjoin%
\pgfsetlinewidth{1.505625pt}%
\definecolor{currentstroke}{rgb}{0.000000,0.000000,0.000000}%
\pgfsetstrokecolor{currentstroke}%
\pgfsetdash{}{0pt}%
\pgfusepath{stroke}%
\end{pgfscope}%
\begin{pgfscope}%
\pgfpathrectangle{\pgfqpoint{1.374500in}{0.082500in}}{\pgfqpoint{2.419000in}{2.419000in}}%
\pgfusepath{clip}%
\pgfsetbuttcap%
\pgfsetroundjoin%
\pgfsetlinewidth{1.505625pt}%
\definecolor{currentstroke}{rgb}{0.000000,0.000000,0.000000}%
\pgfsetstrokecolor{currentstroke}%
\pgfsetdash{}{0pt}%
\pgfusepath{stroke}%
\end{pgfscope}%
\begin{pgfscope}%
\pgfpathrectangle{\pgfqpoint{1.374500in}{0.082500in}}{\pgfqpoint{2.419000in}{2.419000in}}%
\pgfusepath{clip}%
\pgfsetbuttcap%
\pgfsetroundjoin%
\pgfsetlinewidth{1.505625pt}%
\definecolor{currentstroke}{rgb}{0.000000,0.000000,0.000000}%
\pgfsetstrokecolor{currentstroke}%
\pgfsetdash{}{0pt}%
\pgfusepath{stroke}%
\end{pgfscope}%
\begin{pgfscope}%
\pgfpathrectangle{\pgfqpoint{1.374500in}{0.082500in}}{\pgfqpoint{2.419000in}{2.419000in}}%
\pgfusepath{clip}%
\pgfsetbuttcap%
\pgfsetroundjoin%
\pgfsetlinewidth{1.505625pt}%
\definecolor{currentstroke}{rgb}{0.000000,0.000000,0.000000}%
\pgfsetstrokecolor{currentstroke}%
\pgfsetdash{}{0pt}%
\pgfusepath{stroke}%
\end{pgfscope}%
\begin{pgfscope}%
\pgfpathrectangle{\pgfqpoint{1.374500in}{0.082500in}}{\pgfqpoint{2.419000in}{2.419000in}}%
\pgfusepath{clip}%
\pgfsetbuttcap%
\pgfsetroundjoin%
\pgfsetlinewidth{1.505625pt}%
\definecolor{currentstroke}{rgb}{0.000000,0.000000,0.000000}%
\pgfsetstrokecolor{currentstroke}%
\pgfsetdash{}{0pt}%
\pgfusepath{stroke}%
\end{pgfscope}%
\begin{pgfscope}%
\pgfpathrectangle{\pgfqpoint{1.374500in}{0.082500in}}{\pgfqpoint{2.419000in}{2.419000in}}%
\pgfusepath{clip}%
\pgfsetbuttcap%
\pgfsetroundjoin%
\pgfsetlinewidth{1.505625pt}%
\definecolor{currentstroke}{rgb}{0.000000,0.000000,0.000000}%
\pgfsetstrokecolor{currentstroke}%
\pgfsetdash{}{0pt}%
\pgfusepath{stroke}%
\end{pgfscope}%
\begin{pgfscope}%
\pgfpathrectangle{\pgfqpoint{1.374500in}{0.082500in}}{\pgfqpoint{2.419000in}{2.419000in}}%
\pgfusepath{clip}%
\pgfsetbuttcap%
\pgfsetroundjoin%
\pgfsetlinewidth{1.505625pt}%
\definecolor{currentstroke}{rgb}{0.000000,0.000000,0.000000}%
\pgfsetstrokecolor{currentstroke}%
\pgfsetdash{}{0pt}%
\pgfusepath{stroke}%
\end{pgfscope}%
\begin{pgfscope}%
\pgfpathrectangle{\pgfqpoint{1.374500in}{0.082500in}}{\pgfqpoint{2.419000in}{2.419000in}}%
\pgfusepath{clip}%
\pgfsetbuttcap%
\pgfsetroundjoin%
\pgfsetlinewidth{1.505625pt}%
\definecolor{currentstroke}{rgb}{0.000000,0.000000,0.000000}%
\pgfsetstrokecolor{currentstroke}%
\pgfsetdash{}{0pt}%
\pgfusepath{stroke}%
\end{pgfscope}%
\begin{pgfscope}%
\pgfpathrectangle{\pgfqpoint{1.374500in}{0.082500in}}{\pgfqpoint{2.419000in}{2.419000in}}%
\pgfusepath{clip}%
\pgfsetbuttcap%
\pgfsetroundjoin%
\pgfsetlinewidth{1.505625pt}%
\definecolor{currentstroke}{rgb}{0.000000,0.000000,0.000000}%
\pgfsetstrokecolor{currentstroke}%
\pgfsetdash{}{0pt}%
\pgfusepath{stroke}%
\end{pgfscope}%
\begin{pgfscope}%
\pgfpathrectangle{\pgfqpoint{1.374500in}{0.082500in}}{\pgfqpoint{2.419000in}{2.419000in}}%
\pgfusepath{clip}%
\pgfsetbuttcap%
\pgfsetroundjoin%
\pgfsetlinewidth{1.505625pt}%
\definecolor{currentstroke}{rgb}{0.000000,0.000000,0.000000}%
\pgfsetstrokecolor{currentstroke}%
\pgfsetdash{}{0pt}%
\pgfusepath{stroke}%
\end{pgfscope}%
\begin{pgfscope}%
\pgfpathrectangle{\pgfqpoint{1.374500in}{0.082500in}}{\pgfqpoint{2.419000in}{2.419000in}}%
\pgfusepath{clip}%
\pgfsetbuttcap%
\pgfsetroundjoin%
\pgfsetlinewidth{1.505625pt}%
\definecolor{currentstroke}{rgb}{0.000000,0.000000,0.000000}%
\pgfsetstrokecolor{currentstroke}%
\pgfsetdash{}{0pt}%
\pgfusepath{stroke}%
\end{pgfscope}%
\begin{pgfscope}%
\pgfpathrectangle{\pgfqpoint{1.374500in}{0.082500in}}{\pgfqpoint{2.419000in}{2.419000in}}%
\pgfusepath{clip}%
\pgfsetbuttcap%
\pgfsetroundjoin%
\pgfsetlinewidth{1.505625pt}%
\definecolor{currentstroke}{rgb}{0.000000,0.000000,0.000000}%
\pgfsetstrokecolor{currentstroke}%
\pgfsetdash{}{0pt}%
\pgfusepath{stroke}%
\end{pgfscope}%
\begin{pgfscope}%
\pgfpathrectangle{\pgfqpoint{1.374500in}{0.082500in}}{\pgfqpoint{2.419000in}{2.419000in}}%
\pgfusepath{clip}%
\pgfsetbuttcap%
\pgfsetroundjoin%
\pgfsetlinewidth{1.505625pt}%
\definecolor{currentstroke}{rgb}{0.000000,0.000000,0.000000}%
\pgfsetstrokecolor{currentstroke}%
\pgfsetdash{}{0pt}%
\pgfusepath{stroke}%
\end{pgfscope}%
\begin{pgfscope}%
\pgfpathrectangle{\pgfqpoint{1.374500in}{0.082500in}}{\pgfqpoint{2.419000in}{2.419000in}}%
\pgfusepath{clip}%
\pgfsetbuttcap%
\pgfsetroundjoin%
\pgfsetlinewidth{1.505625pt}%
\definecolor{currentstroke}{rgb}{0.000000,0.000000,0.000000}%
\pgfsetstrokecolor{currentstroke}%
\pgfsetdash{}{0pt}%
\pgfusepath{stroke}%
\end{pgfscope}%
\begin{pgfscope}%
\pgfpathrectangle{\pgfqpoint{1.374500in}{0.082500in}}{\pgfqpoint{2.419000in}{2.419000in}}%
\pgfusepath{clip}%
\pgfsetbuttcap%
\pgfsetroundjoin%
\pgfsetlinewidth{1.505625pt}%
\definecolor{currentstroke}{rgb}{0.000000,0.000000,0.000000}%
\pgfsetstrokecolor{currentstroke}%
\pgfsetdash{}{0pt}%
\pgfusepath{stroke}%
\end{pgfscope}%
\begin{pgfscope}%
\pgfpathrectangle{\pgfqpoint{1.374500in}{0.082500in}}{\pgfqpoint{2.419000in}{2.419000in}}%
\pgfusepath{clip}%
\pgfsetbuttcap%
\pgfsetroundjoin%
\pgfsetlinewidth{1.505625pt}%
\definecolor{currentstroke}{rgb}{0.000000,0.000000,0.000000}%
\pgfsetstrokecolor{currentstroke}%
\pgfsetdash{}{0pt}%
\pgfusepath{stroke}%
\end{pgfscope}%
\begin{pgfscope}%
\pgfpathrectangle{\pgfqpoint{1.374500in}{0.082500in}}{\pgfqpoint{2.419000in}{2.419000in}}%
\pgfusepath{clip}%
\pgfsetbuttcap%
\pgfsetroundjoin%
\pgfsetlinewidth{1.505625pt}%
\definecolor{currentstroke}{rgb}{0.000000,0.000000,0.000000}%
\pgfsetstrokecolor{currentstroke}%
\pgfsetdash{}{0pt}%
\pgfusepath{stroke}%
\end{pgfscope}%
\begin{pgfscope}%
\pgfpathrectangle{\pgfqpoint{1.374500in}{0.082500in}}{\pgfqpoint{2.419000in}{2.419000in}}%
\pgfusepath{clip}%
\pgfsetbuttcap%
\pgfsetroundjoin%
\definecolor{currentfill}{rgb}{0.247059,0.564706,0.854902}%
\pgfsetfillcolor{currentfill}%
\pgfsetfillopacity{0.300000}%
\pgfsetlinewidth{1.003750pt}%
\definecolor{currentstroke}{rgb}{0.247059,0.564706,0.854902}%
\pgfsetstrokecolor{currentstroke}%
\pgfsetstrokeopacity{0.300000}%
\pgfsetdash{}{0pt}%
\pgfpathmoveto{\pgfqpoint{2.865500in}{4.095036in}}%
\pgfpathcurveto{\pgfqpoint{2.874709in}{4.095036in}}{\pgfqpoint{2.883541in}{4.098695in}}{\pgfqpoint{2.890053in}{4.105206in}}%
\pgfpathcurveto{\pgfqpoint{2.896564in}{4.111717in}}{\pgfqpoint{2.900223in}{4.120550in}}{\pgfqpoint{2.900223in}{4.129758in}}%
\pgfpathcurveto{\pgfqpoint{2.900223in}{4.138967in}}{\pgfqpoint{2.896564in}{4.147799in}}{\pgfqpoint{2.890053in}{4.154311in}}%
\pgfpathcurveto{\pgfqpoint{2.883541in}{4.160822in}}{\pgfqpoint{2.874709in}{4.164480in}}{\pgfqpoint{2.865500in}{4.164480in}}%
\pgfpathcurveto{\pgfqpoint{2.856292in}{4.164480in}}{\pgfqpoint{2.847459in}{4.160822in}}{\pgfqpoint{2.840948in}{4.154311in}}%
\pgfpathcurveto{\pgfqpoint{2.834437in}{4.147799in}}{\pgfqpoint{2.830778in}{4.138967in}}{\pgfqpoint{2.830778in}{4.129758in}}%
\pgfpathcurveto{\pgfqpoint{2.830778in}{4.120550in}}{\pgfqpoint{2.834437in}{4.111717in}}{\pgfqpoint{2.840948in}{4.105206in}}%
\pgfpathcurveto{\pgfqpoint{2.847459in}{4.098695in}}{\pgfqpoint{2.856292in}{4.095036in}}{\pgfqpoint{2.865500in}{4.095036in}}%
\pgfpathlineto{\pgfqpoint{2.865500in}{4.095036in}}%
\pgfpathclose%
\pgfusepath{stroke,fill}%
\end{pgfscope}%
\begin{pgfscope}%
\pgfpathrectangle{\pgfqpoint{1.374500in}{0.082500in}}{\pgfqpoint{2.419000in}{2.419000in}}%
\pgfusepath{clip}%
\pgfsetbuttcap%
\pgfsetroundjoin%
\definecolor{currentfill}{rgb}{0.247059,0.564706,0.854902}%
\pgfsetfillcolor{currentfill}%
\pgfsetfillopacity{0.305085}%
\pgfsetlinewidth{1.003750pt}%
\definecolor{currentstroke}{rgb}{0.247059,0.564706,0.854902}%
\pgfsetstrokecolor{currentstroke}%
\pgfsetstrokeopacity{0.305085}%
\pgfsetdash{}{0pt}%
\pgfpathmoveto{\pgfqpoint{3.444550in}{3.990979in}}%
\pgfpathcurveto{\pgfqpoint{3.453758in}{3.990979in}}{\pgfqpoint{3.462591in}{3.994638in}}{\pgfqpoint{3.469102in}{4.001149in}}%
\pgfpathcurveto{\pgfqpoint{3.475613in}{4.007661in}}{\pgfqpoint{3.479272in}{4.016493in}}{\pgfqpoint{3.479272in}{4.025702in}}%
\pgfpathcurveto{\pgfqpoint{3.479272in}{4.034910in}}{\pgfqpoint{3.475613in}{4.043743in}}{\pgfqpoint{3.469102in}{4.050254in}}%
\pgfpathcurveto{\pgfqpoint{3.462591in}{4.056765in}}{\pgfqpoint{3.453758in}{4.060424in}}{\pgfqpoint{3.444550in}{4.060424in}}%
\pgfpathcurveto{\pgfqpoint{3.435341in}{4.060424in}}{\pgfqpoint{3.426509in}{4.056765in}}{\pgfqpoint{3.419997in}{4.050254in}}%
\pgfpathcurveto{\pgfqpoint{3.413486in}{4.043743in}}{\pgfqpoint{3.409827in}{4.034910in}}{\pgfqpoint{3.409827in}{4.025702in}}%
\pgfpathcurveto{\pgfqpoint{3.409827in}{4.016493in}}{\pgfqpoint{3.413486in}{4.007661in}}{\pgfqpoint{3.419997in}{4.001149in}}%
\pgfpathcurveto{\pgfqpoint{3.426509in}{3.994638in}}{\pgfqpoint{3.435341in}{3.990979in}}{\pgfqpoint{3.444550in}{3.990979in}}%
\pgfpathlineto{\pgfqpoint{3.444550in}{3.990979in}}%
\pgfpathclose%
\pgfusepath{stroke,fill}%
\end{pgfscope}%
\begin{pgfscope}%
\pgfpathrectangle{\pgfqpoint{1.374500in}{0.082500in}}{\pgfqpoint{2.419000in}{2.419000in}}%
\pgfusepath{clip}%
\pgfsetbuttcap%
\pgfsetroundjoin%
\definecolor{currentfill}{rgb}{0.247059,0.564706,0.854902}%
\pgfsetfillcolor{currentfill}%
\pgfsetfillopacity{0.310398}%
\pgfsetlinewidth{1.003750pt}%
\definecolor{currentstroke}{rgb}{0.247059,0.564706,0.854902}%
\pgfsetstrokecolor{currentstroke}%
\pgfsetstrokeopacity{0.310398}%
\pgfsetdash{}{0pt}%
\pgfpathmoveto{\pgfqpoint{4.049570in}{3.882256in}}%
\pgfpathcurveto{\pgfqpoint{4.058778in}{3.882256in}}{\pgfqpoint{4.067611in}{3.885914in}}{\pgfqpoint{4.074122in}{3.892426in}}%
\pgfpathcurveto{\pgfqpoint{4.080633in}{3.898937in}}{\pgfqpoint{4.084292in}{3.907770in}}{\pgfqpoint{4.084292in}{3.916978in}}%
\pgfpathcurveto{\pgfqpoint{4.084292in}{3.926186in}}{\pgfqpoint{4.080633in}{3.935019in}}{\pgfqpoint{4.074122in}{3.941530in}}%
\pgfpathcurveto{\pgfqpoint{4.067611in}{3.948042in}}{\pgfqpoint{4.058778in}{3.951700in}}{\pgfqpoint{4.049570in}{3.951700in}}%
\pgfpathcurveto{\pgfqpoint{4.040361in}{3.951700in}}{\pgfqpoint{4.031529in}{3.948042in}}{\pgfqpoint{4.025017in}{3.941530in}}%
\pgfpathcurveto{\pgfqpoint{4.018506in}{3.935019in}}{\pgfqpoint{4.014847in}{3.926186in}}{\pgfqpoint{4.014847in}{3.916978in}}%
\pgfpathcurveto{\pgfqpoint{4.014847in}{3.907770in}}{\pgfqpoint{4.018506in}{3.898937in}}{\pgfqpoint{4.025017in}{3.892426in}}%
\pgfpathcurveto{\pgfqpoint{4.031529in}{3.885914in}}{\pgfqpoint{4.040361in}{3.882256in}}{\pgfqpoint{4.049570in}{3.882256in}}%
\pgfpathlineto{\pgfqpoint{4.049570in}{3.882256in}}%
\pgfpathclose%
\pgfusepath{stroke,fill}%
\end{pgfscope}%
\begin{pgfscope}%
\pgfpathrectangle{\pgfqpoint{1.374500in}{0.082500in}}{\pgfqpoint{2.419000in}{2.419000in}}%
\pgfusepath{clip}%
\pgfsetbuttcap%
\pgfsetroundjoin%
\definecolor{currentfill}{rgb}{0.247059,0.564706,0.854902}%
\pgfsetfillcolor{currentfill}%
\pgfsetfillopacity{0.313145}%
\pgfsetlinewidth{1.003750pt}%
\definecolor{currentstroke}{rgb}{0.247059,0.564706,0.854902}%
\pgfsetstrokecolor{currentstroke}%
\pgfsetstrokeopacity{0.313145}%
\pgfsetdash{}{0pt}%
\pgfpathmoveto{\pgfqpoint{2.780757in}{3.826045in}}%
\pgfpathcurveto{\pgfqpoint{2.789965in}{3.826045in}}{\pgfqpoint{2.798798in}{3.829703in}}{\pgfqpoint{2.805309in}{3.836215in}}%
\pgfpathcurveto{\pgfqpoint{2.811821in}{3.842726in}}{\pgfqpoint{2.815479in}{3.851558in}}{\pgfqpoint{2.815479in}{3.860767in}}%
\pgfpathcurveto{\pgfqpoint{2.815479in}{3.869975in}}{\pgfqpoint{2.811821in}{3.878808in}}{\pgfqpoint{2.805309in}{3.885319in}}%
\pgfpathcurveto{\pgfqpoint{2.798798in}{3.891831in}}{\pgfqpoint{2.789965in}{3.895489in}}{\pgfqpoint{2.780757in}{3.895489in}}%
\pgfpathcurveto{\pgfqpoint{2.771549in}{3.895489in}}{\pgfqpoint{2.762716in}{3.891831in}}{\pgfqpoint{2.756205in}{3.885319in}}%
\pgfpathcurveto{\pgfqpoint{2.749693in}{3.878808in}}{\pgfqpoint{2.746035in}{3.869975in}}{\pgfqpoint{2.746035in}{3.860767in}}%
\pgfpathcurveto{\pgfqpoint{2.746035in}{3.851558in}}{\pgfqpoint{2.749693in}{3.842726in}}{\pgfqpoint{2.756205in}{3.836215in}}%
\pgfpathcurveto{\pgfqpoint{2.762716in}{3.829703in}}{\pgfqpoint{2.771549in}{3.826045in}}{\pgfqpoint{2.780757in}{3.826045in}}%
\pgfpathlineto{\pgfqpoint{2.780757in}{3.826045in}}%
\pgfpathclose%
\pgfusepath{stroke,fill}%
\end{pgfscope}%
\begin{pgfscope}%
\pgfpathrectangle{\pgfqpoint{1.374500in}{0.082500in}}{\pgfqpoint{2.419000in}{2.419000in}}%
\pgfusepath{clip}%
\pgfsetbuttcap%
\pgfsetroundjoin%
\definecolor{currentfill}{rgb}{0.247059,0.564706,0.854902}%
\pgfsetfillcolor{currentfill}%
\pgfsetfillopacity{0.315955}%
\pgfsetlinewidth{1.003750pt}%
\definecolor{currentstroke}{rgb}{0.247059,0.564706,0.854902}%
\pgfsetstrokecolor{currentstroke}%
\pgfsetstrokeopacity{0.315955}%
\pgfsetdash{}{0pt}%
\pgfpathmoveto{\pgfqpoint{4.682348in}{3.768544in}}%
\pgfpathcurveto{\pgfqpoint{4.691556in}{3.768544in}}{\pgfqpoint{4.700389in}{3.772203in}}{\pgfqpoint{4.706900in}{3.778714in}}%
\pgfpathcurveto{\pgfqpoint{4.713411in}{3.785225in}}{\pgfqpoint{4.717070in}{3.794058in}}{\pgfqpoint{4.717070in}{3.803266in}}%
\pgfpathcurveto{\pgfqpoint{4.717070in}{3.812475in}}{\pgfqpoint{4.713411in}{3.821307in}}{\pgfqpoint{4.706900in}{3.827819in}}%
\pgfpathcurveto{\pgfqpoint{4.700389in}{3.834330in}}{\pgfqpoint{4.691556in}{3.837988in}}{\pgfqpoint{4.682348in}{3.837988in}}%
\pgfpathcurveto{\pgfqpoint{4.673139in}{3.837988in}}{\pgfqpoint{4.664307in}{3.834330in}}{\pgfqpoint{4.657795in}{3.827819in}}%
\pgfpathcurveto{\pgfqpoint{4.651284in}{3.821307in}}{\pgfqpoint{4.647626in}{3.812475in}}{\pgfqpoint{4.647626in}{3.803266in}}%
\pgfpathcurveto{\pgfqpoint{4.647626in}{3.794058in}}{\pgfqpoint{4.651284in}{3.785225in}}{\pgfqpoint{4.657795in}{3.778714in}}%
\pgfpathcurveto{\pgfqpoint{4.664307in}{3.772203in}}{\pgfqpoint{4.673139in}{3.768544in}}{\pgfqpoint{4.682348in}{3.768544in}}%
\pgfpathlineto{\pgfqpoint{4.682348in}{3.768544in}}%
\pgfpathclose%
\pgfusepath{stroke,fill}%
\end{pgfscope}%
\begin{pgfscope}%
\pgfpathrectangle{\pgfqpoint{1.374500in}{0.082500in}}{\pgfqpoint{2.419000in}{2.419000in}}%
\pgfusepath{clip}%
\pgfsetbuttcap%
\pgfsetroundjoin%
\definecolor{currentfill}{rgb}{0.247059,0.564706,0.854902}%
\pgfsetfillcolor{currentfill}%
\pgfsetfillopacity{0.318831}%
\pgfsetlinewidth{1.003750pt}%
\definecolor{currentstroke}{rgb}{0.247059,0.564706,0.854902}%
\pgfsetstrokecolor{currentstroke}%
\pgfsetstrokeopacity{0.318831}%
\pgfsetdash{}{0pt}%
\pgfpathmoveto{\pgfqpoint{3.391433in}{3.709709in}}%
\pgfpathcurveto{\pgfqpoint{3.400642in}{3.709709in}}{\pgfqpoint{3.409474in}{3.713368in}}{\pgfqpoint{3.415986in}{3.719879in}}%
\pgfpathcurveto{\pgfqpoint{3.422497in}{3.726390in}}{\pgfqpoint{3.426155in}{3.735223in}}{\pgfqpoint{3.426155in}{3.744431in}}%
\pgfpathcurveto{\pgfqpoint{3.426155in}{3.753640in}}{\pgfqpoint{3.422497in}{3.762472in}}{\pgfqpoint{3.415986in}{3.768984in}}%
\pgfpathcurveto{\pgfqpoint{3.409474in}{3.775495in}}{\pgfqpoint{3.400642in}{3.779153in}}{\pgfqpoint{3.391433in}{3.779153in}}%
\pgfpathcurveto{\pgfqpoint{3.382225in}{3.779153in}}{\pgfqpoint{3.373392in}{3.775495in}}{\pgfqpoint{3.366881in}{3.768984in}}%
\pgfpathcurveto{\pgfqpoint{3.360370in}{3.762472in}}{\pgfqpoint{3.356711in}{3.753640in}}{\pgfqpoint{3.356711in}{3.744431in}}%
\pgfpathcurveto{\pgfqpoint{3.356711in}{3.735223in}}{\pgfqpoint{3.360370in}{3.726390in}}{\pgfqpoint{3.366881in}{3.719879in}}%
\pgfpathcurveto{\pgfqpoint{3.373392in}{3.713368in}}{\pgfqpoint{3.382225in}{3.709709in}}{\pgfqpoint{3.391433in}{3.709709in}}%
\pgfpathlineto{\pgfqpoint{3.391433in}{3.709709in}}%
\pgfpathclose%
\pgfusepath{stroke,fill}%
\end{pgfscope}%
\begin{pgfscope}%
\pgfpathrectangle{\pgfqpoint{1.374500in}{0.082500in}}{\pgfqpoint{2.419000in}{2.419000in}}%
\pgfusepath{clip}%
\pgfsetbuttcap%
\pgfsetroundjoin%
\definecolor{currentfill}{rgb}{0.247059,0.564706,0.854902}%
\pgfsetfillcolor{currentfill}%
\pgfsetfillopacity{0.321773}%
\pgfsetlinewidth{1.003750pt}%
\definecolor{currentstroke}{rgb}{0.247059,0.564706,0.854902}%
\pgfsetstrokecolor{currentstroke}%
\pgfsetstrokeopacity{0.321773}%
\pgfsetdash{}{0pt}%
\pgfpathmoveto{\pgfqpoint{2.070210in}{3.649493in}}%
\pgfpathcurveto{\pgfqpoint{2.079418in}{3.649493in}}{\pgfqpoint{2.088251in}{3.653151in}}{\pgfqpoint{2.094762in}{3.659663in}}%
\pgfpathcurveto{\pgfqpoint{2.101274in}{3.666174in}}{\pgfqpoint{2.104932in}{3.675006in}}{\pgfqpoint{2.104932in}{3.684215in}}%
\pgfpathcurveto{\pgfqpoint{2.104932in}{3.693423in}}{\pgfqpoint{2.101274in}{3.702256in}}{\pgfqpoint{2.094762in}{3.708767in}}%
\pgfpathcurveto{\pgfqpoint{2.088251in}{3.715279in}}{\pgfqpoint{2.079418in}{3.718937in}}{\pgfqpoint{2.070210in}{3.718937in}}%
\pgfpathcurveto{\pgfqpoint{2.061001in}{3.718937in}}{\pgfqpoint{2.052169in}{3.715279in}}{\pgfqpoint{2.045658in}{3.708767in}}%
\pgfpathcurveto{\pgfqpoint{2.039146in}{3.702256in}}{\pgfqpoint{2.035488in}{3.693423in}}{\pgfqpoint{2.035488in}{3.684215in}}%
\pgfpathcurveto{\pgfqpoint{2.035488in}{3.675006in}}{\pgfqpoint{2.039146in}{3.666174in}}{\pgfqpoint{2.045658in}{3.659663in}}%
\pgfpathcurveto{\pgfqpoint{2.052169in}{3.653151in}}{\pgfqpoint{2.061001in}{3.649493in}}{\pgfqpoint{2.070210in}{3.649493in}}%
\pgfpathlineto{\pgfqpoint{2.070210in}{3.649493in}}%
\pgfpathclose%
\pgfusepath{stroke,fill}%
\end{pgfscope}%
\begin{pgfscope}%
\pgfpathrectangle{\pgfqpoint{1.374500in}{0.082500in}}{\pgfqpoint{2.419000in}{2.419000in}}%
\pgfusepath{clip}%
\pgfsetbuttcap%
\pgfsetroundjoin%
\definecolor{currentfill}{rgb}{0.247059,0.564706,0.854902}%
\pgfsetfillcolor{currentfill}%
\pgfsetfillopacity{0.321773}%
\pgfsetlinewidth{1.003750pt}%
\definecolor{currentstroke}{rgb}{0.247059,0.564706,0.854902}%
\pgfsetstrokecolor{currentstroke}%
\pgfsetstrokeopacity{0.321773}%
\pgfsetdash{}{0pt}%
\pgfpathmoveto{\pgfqpoint{5.344839in}{3.649493in}}%
\pgfpathcurveto{\pgfqpoint{5.354048in}{3.649493in}}{\pgfqpoint{5.362880in}{3.653151in}}{\pgfqpoint{5.369392in}{3.659663in}}%
\pgfpathcurveto{\pgfqpoint{5.375903in}{3.666174in}}{\pgfqpoint{5.379561in}{3.675006in}}{\pgfqpoint{5.379561in}{3.684215in}}%
\pgfpathcurveto{\pgfqpoint{5.379561in}{3.693423in}}{\pgfqpoint{5.375903in}{3.702256in}}{\pgfqpoint{5.369392in}{3.708767in}}%
\pgfpathcurveto{\pgfqpoint{5.362880in}{3.715279in}}{\pgfqpoint{5.354048in}{3.718937in}}{\pgfqpoint{5.344839in}{3.718937in}}%
\pgfpathcurveto{\pgfqpoint{5.335631in}{3.718937in}}{\pgfqpoint{5.326798in}{3.715279in}}{\pgfqpoint{5.320287in}{3.708767in}}%
\pgfpathcurveto{\pgfqpoint{5.313776in}{3.702256in}}{\pgfqpoint{5.310117in}{3.693423in}}{\pgfqpoint{5.310117in}{3.684215in}}%
\pgfpathcurveto{\pgfqpoint{5.310117in}{3.675006in}}{\pgfqpoint{5.313776in}{3.666174in}}{\pgfqpoint{5.320287in}{3.659663in}}%
\pgfpathcurveto{\pgfqpoint{5.326798in}{3.653151in}}{\pgfqpoint{5.335631in}{3.649493in}}{\pgfqpoint{5.344839in}{3.649493in}}%
\pgfpathlineto{\pgfqpoint{5.344839in}{3.649493in}}%
\pgfpathclose%
\pgfusepath{stroke,fill}%
\end{pgfscope}%
\begin{pgfscope}%
\pgfpathrectangle{\pgfqpoint{1.374500in}{0.082500in}}{\pgfqpoint{2.419000in}{2.419000in}}%
\pgfusepath{clip}%
\pgfsetbuttcap%
\pgfsetroundjoin%
\definecolor{currentfill}{rgb}{0.247059,0.564706,0.854902}%
\pgfsetfillcolor{currentfill}%
\pgfsetfillopacity{0.324786}%
\pgfsetlinewidth{1.003750pt}%
\definecolor{currentstroke}{rgb}{0.247059,0.564706,0.854902}%
\pgfsetstrokecolor{currentstroke}%
\pgfsetstrokeopacity{0.324786}%
\pgfsetdash{}{0pt}%
\pgfpathmoveto{\pgfqpoint{4.031125in}{3.587846in}}%
\pgfpathcurveto{\pgfqpoint{4.040334in}{3.587846in}}{\pgfqpoint{4.049166in}{3.591504in}}{\pgfqpoint{4.055678in}{3.598016in}}%
\pgfpathcurveto{\pgfqpoint{4.062189in}{3.604527in}}{\pgfqpoint{4.065848in}{3.613359in}}{\pgfqpoint{4.065848in}{3.622568in}}%
\pgfpathcurveto{\pgfqpoint{4.065848in}{3.631776in}}{\pgfqpoint{4.062189in}{3.640609in}}{\pgfqpoint{4.055678in}{3.647120in}}%
\pgfpathcurveto{\pgfqpoint{4.049166in}{3.653632in}}{\pgfqpoint{4.040334in}{3.657290in}}{\pgfqpoint{4.031125in}{3.657290in}}%
\pgfpathcurveto{\pgfqpoint{4.021917in}{3.657290in}}{\pgfqpoint{4.013084in}{3.653632in}}{\pgfqpoint{4.006573in}{3.647120in}}%
\pgfpathcurveto{\pgfqpoint{4.000062in}{3.640609in}}{\pgfqpoint{3.996403in}{3.631776in}}{\pgfqpoint{3.996403in}{3.622568in}}%
\pgfpathcurveto{\pgfqpoint{3.996403in}{3.613359in}}{\pgfqpoint{4.000062in}{3.604527in}}{\pgfqpoint{4.006573in}{3.598016in}}%
\pgfpathcurveto{\pgfqpoint{4.013084in}{3.591504in}}{\pgfqpoint{4.021917in}{3.587846in}}{\pgfqpoint{4.031125in}{3.587846in}}%
\pgfpathlineto{\pgfqpoint{4.031125in}{3.587846in}}%
\pgfpathclose%
\pgfusepath{stroke,fill}%
\end{pgfscope}%
\begin{pgfscope}%
\pgfpathrectangle{\pgfqpoint{1.374500in}{0.082500in}}{\pgfqpoint{2.419000in}{2.419000in}}%
\pgfusepath{clip}%
\pgfsetbuttcap%
\pgfsetroundjoin%
\definecolor{currentfill}{rgb}{0.247059,0.564706,0.854902}%
\pgfsetfillcolor{currentfill}%
\pgfsetfillopacity{0.327871}%
\pgfsetlinewidth{1.003750pt}%
\definecolor{currentstroke}{rgb}{0.247059,0.564706,0.854902}%
\pgfsetstrokecolor{currentstroke}%
\pgfsetstrokeopacity{0.327871}%
\pgfsetdash{}{0pt}%
\pgfpathmoveto{\pgfqpoint{2.685826in}{3.524717in}}%
\pgfpathcurveto{\pgfqpoint{2.695035in}{3.524717in}}{\pgfqpoint{2.703867in}{3.528375in}}{\pgfqpoint{2.710378in}{3.534886in}}%
\pgfpathcurveto{\pgfqpoint{2.716890in}{3.541398in}}{\pgfqpoint{2.720548in}{3.550230in}}{\pgfqpoint{2.720548in}{3.559439in}}%
\pgfpathcurveto{\pgfqpoint{2.720548in}{3.568647in}}{\pgfqpoint{2.716890in}{3.577480in}}{\pgfqpoint{2.710378in}{3.583991in}}%
\pgfpathcurveto{\pgfqpoint{2.703867in}{3.590502in}}{\pgfqpoint{2.695035in}{3.594161in}}{\pgfqpoint{2.685826in}{3.594161in}}%
\pgfpathcurveto{\pgfqpoint{2.676618in}{3.594161in}}{\pgfqpoint{2.667785in}{3.590502in}}{\pgfqpoint{2.661274in}{3.583991in}}%
\pgfpathcurveto{\pgfqpoint{2.654762in}{3.577480in}}{\pgfqpoint{2.651104in}{3.568647in}}{\pgfqpoint{2.651104in}{3.559439in}}%
\pgfpathcurveto{\pgfqpoint{2.651104in}{3.550230in}}{\pgfqpoint{2.654762in}{3.541398in}}{\pgfqpoint{2.661274in}{3.534886in}}%
\pgfpathcurveto{\pgfqpoint{2.667785in}{3.528375in}}{\pgfqpoint{2.676618in}{3.524717in}}{\pgfqpoint{2.685826in}{3.524717in}}%
\pgfpathlineto{\pgfqpoint{2.685826in}{3.524717in}}%
\pgfpathclose%
\pgfusepath{stroke,fill}%
\end{pgfscope}%
\begin{pgfscope}%
\pgfpathrectangle{\pgfqpoint{1.374500in}{0.082500in}}{\pgfqpoint{2.419000in}{2.419000in}}%
\pgfusepath{clip}%
\pgfsetbuttcap%
\pgfsetroundjoin%
\definecolor{currentfill}{rgb}{0.247059,0.564706,0.854902}%
\pgfsetfillcolor{currentfill}%
\pgfsetfillopacity{0.327871}%
\pgfsetlinewidth{1.003750pt}%
\definecolor{currentstroke}{rgb}{0.247059,0.564706,0.854902}%
\pgfsetstrokecolor{currentstroke}%
\pgfsetstrokeopacity{0.327871}%
\pgfsetdash{}{0pt}%
\pgfpathmoveto{\pgfqpoint{6.039187in}{3.524717in}}%
\pgfpathcurveto{\pgfqpoint{6.048396in}{3.524717in}}{\pgfqpoint{6.057228in}{3.528375in}}{\pgfqpoint{6.063739in}{3.534886in}}%
\pgfpathcurveto{\pgfqpoint{6.070251in}{3.541398in}}{\pgfqpoint{6.073909in}{3.550230in}}{\pgfqpoint{6.073909in}{3.559439in}}%
\pgfpathcurveto{\pgfqpoint{6.073909in}{3.568647in}}{\pgfqpoint{6.070251in}{3.577480in}}{\pgfqpoint{6.063739in}{3.583991in}}%
\pgfpathcurveto{\pgfqpoint{6.057228in}{3.590502in}}{\pgfqpoint{6.048396in}{3.594161in}}{\pgfqpoint{6.039187in}{3.594161in}}%
\pgfpathcurveto{\pgfqpoint{6.029979in}{3.594161in}}{\pgfqpoint{6.021146in}{3.590502in}}{\pgfqpoint{6.014635in}{3.583991in}}%
\pgfpathcurveto{\pgfqpoint{6.008123in}{3.577480in}}{\pgfqpoint{6.004465in}{3.568647in}}{\pgfqpoint{6.004465in}{3.559439in}}%
\pgfpathcurveto{\pgfqpoint{6.004465in}{3.550230in}}{\pgfqpoint{6.008123in}{3.541398in}}{\pgfqpoint{6.014635in}{3.534886in}}%
\pgfpathcurveto{\pgfqpoint{6.021146in}{3.528375in}}{\pgfqpoint{6.029979in}{3.524717in}}{\pgfqpoint{6.039187in}{3.524717in}}%
\pgfpathlineto{\pgfqpoint{6.039187in}{3.524717in}}%
\pgfpathclose%
\pgfusepath{stroke,fill}%
\end{pgfscope}%
\begin{pgfscope}%
\pgfpathrectangle{\pgfqpoint{1.374500in}{0.082500in}}{\pgfqpoint{2.419000in}{2.419000in}}%
\pgfusepath{clip}%
\pgfsetbuttcap%
\pgfsetroundjoin%
\definecolor{currentfill}{rgb}{0.247059,0.564706,0.854902}%
\pgfsetfillcolor{currentfill}%
\pgfsetfillopacity{0.331031}%
\pgfsetlinewidth{1.003750pt}%
\definecolor{currentstroke}{rgb}{0.247059,0.564706,0.854902}%
\pgfsetstrokecolor{currentstroke}%
\pgfsetstrokeopacity{0.331031}%
\pgfsetdash{}{0pt}%
\pgfpathmoveto{\pgfqpoint{4.701952in}{3.460051in}}%
\pgfpathcurveto{\pgfqpoint{4.711161in}{3.460051in}}{\pgfqpoint{4.719993in}{3.463710in}}{\pgfqpoint{4.726504in}{3.470221in}}%
\pgfpathcurveto{\pgfqpoint{4.733016in}{3.476732in}}{\pgfqpoint{4.736674in}{3.485565in}}{\pgfqpoint{4.736674in}{3.494773in}}%
\pgfpathcurveto{\pgfqpoint{4.736674in}{3.503982in}}{\pgfqpoint{4.733016in}{3.512814in}}{\pgfqpoint{4.726504in}{3.519326in}}%
\pgfpathcurveto{\pgfqpoint{4.719993in}{3.525837in}}{\pgfqpoint{4.711161in}{3.529496in}}{\pgfqpoint{4.701952in}{3.529496in}}%
\pgfpathcurveto{\pgfqpoint{4.692744in}{3.529496in}}{\pgfqpoint{4.683911in}{3.525837in}}{\pgfqpoint{4.677400in}{3.519326in}}%
\pgfpathcurveto{\pgfqpoint{4.670888in}{3.512814in}}{\pgfqpoint{4.667230in}{3.503982in}}{\pgfqpoint{4.667230in}{3.494773in}}%
\pgfpathcurveto{\pgfqpoint{4.667230in}{3.485565in}}{\pgfqpoint{4.670888in}{3.476732in}}{\pgfqpoint{4.677400in}{3.470221in}}%
\pgfpathcurveto{\pgfqpoint{4.683911in}{3.463710in}}{\pgfqpoint{4.692744in}{3.460051in}}{\pgfqpoint{4.701952in}{3.460051in}}%
\pgfpathlineto{\pgfqpoint{4.701952in}{3.460051in}}%
\pgfpathclose%
\pgfusepath{stroke,fill}%
\end{pgfscope}%
\begin{pgfscope}%
\pgfpathrectangle{\pgfqpoint{1.374500in}{0.082500in}}{\pgfqpoint{2.419000in}{2.419000in}}%
\pgfusepath{clip}%
\pgfsetbuttcap%
\pgfsetroundjoin%
\definecolor{currentfill}{rgb}{0.247059,0.564706,0.854902}%
\pgfsetfillcolor{currentfill}%
\pgfsetfillopacity{0.334269}%
\pgfsetlinewidth{1.003750pt}%
\definecolor{currentstroke}{rgb}{0.247059,0.564706,0.854902}%
\pgfsetstrokecolor{currentstroke}%
\pgfsetstrokeopacity{0.334269}%
\pgfsetdash{}{0pt}%
\pgfpathmoveto{\pgfqpoint{3.331774in}{3.393793in}}%
\pgfpathcurveto{\pgfqpoint{3.340983in}{3.393793in}}{\pgfqpoint{3.349815in}{3.397451in}}{\pgfqpoint{3.356326in}{3.403963in}}%
\pgfpathcurveto{\pgfqpoint{3.362838in}{3.410474in}}{\pgfqpoint{3.366496in}{3.419306in}}{\pgfqpoint{3.366496in}{3.428515in}}%
\pgfpathcurveto{\pgfqpoint{3.366496in}{3.437723in}}{\pgfqpoint{3.362838in}{3.446556in}}{\pgfqpoint{3.356326in}{3.453067in}}%
\pgfpathcurveto{\pgfqpoint{3.349815in}{3.459579in}}{\pgfqpoint{3.340983in}{3.463237in}}{\pgfqpoint{3.331774in}{3.463237in}}%
\pgfpathcurveto{\pgfqpoint{3.322566in}{3.463237in}}{\pgfqpoint{3.313733in}{3.459579in}}{\pgfqpoint{3.307222in}{3.453067in}}%
\pgfpathcurveto{\pgfqpoint{3.300710in}{3.446556in}}{\pgfqpoint{3.297052in}{3.437723in}}{\pgfqpoint{3.297052in}{3.428515in}}%
\pgfpathcurveto{\pgfqpoint{3.297052in}{3.419306in}}{\pgfqpoint{3.300710in}{3.410474in}}{\pgfqpoint{3.307222in}{3.403963in}}%
\pgfpathcurveto{\pgfqpoint{3.313733in}{3.397451in}}{\pgfqpoint{3.322566in}{3.393793in}}{\pgfqpoint{3.331774in}{3.393793in}}%
\pgfpathlineto{\pgfqpoint{3.331774in}{3.393793in}}%
\pgfpathclose%
\pgfusepath{stroke,fill}%
\end{pgfscope}%
\begin{pgfscope}%
\pgfpathrectangle{\pgfqpoint{1.374500in}{0.082500in}}{\pgfqpoint{2.419000in}{2.419000in}}%
\pgfusepath{clip}%
\pgfsetbuttcap%
\pgfsetroundjoin%
\definecolor{currentfill}{rgb}{0.247059,0.564706,0.854902}%
\pgfsetfillcolor{currentfill}%
\pgfsetfillopacity{0.334269}%
\pgfsetlinewidth{1.003750pt}%
\definecolor{currentstroke}{rgb}{0.247059,0.564706,0.854902}%
\pgfsetstrokecolor{currentstroke}%
\pgfsetstrokeopacity{0.334269}%
\pgfsetdash{}{0pt}%
\pgfpathmoveto{\pgfqpoint{6.767746in}{3.393793in}}%
\pgfpathcurveto{\pgfqpoint{6.776954in}{3.393793in}}{\pgfqpoint{6.785787in}{3.397451in}}{\pgfqpoint{6.792298in}{3.403963in}}%
\pgfpathcurveto{\pgfqpoint{6.798810in}{3.410474in}}{\pgfqpoint{6.802468in}{3.419306in}}{\pgfqpoint{6.802468in}{3.428515in}}%
\pgfpathcurveto{\pgfqpoint{6.802468in}{3.437723in}}{\pgfqpoint{6.798810in}{3.446556in}}{\pgfqpoint{6.792298in}{3.453067in}}%
\pgfpathcurveto{\pgfqpoint{6.785787in}{3.459579in}}{\pgfqpoint{6.776954in}{3.463237in}}{\pgfqpoint{6.767746in}{3.463237in}}%
\pgfpathcurveto{\pgfqpoint{6.758537in}{3.463237in}}{\pgfqpoint{6.749705in}{3.459579in}}{\pgfqpoint{6.743194in}{3.453067in}}%
\pgfpathcurveto{\pgfqpoint{6.736682in}{3.446556in}}{\pgfqpoint{6.733024in}{3.437723in}}{\pgfqpoint{6.733024in}{3.428515in}}%
\pgfpathcurveto{\pgfqpoint{6.733024in}{3.419306in}}{\pgfqpoint{6.736682in}{3.410474in}}{\pgfqpoint{6.743194in}{3.403963in}}%
\pgfpathcurveto{\pgfqpoint{6.749705in}{3.397451in}}{\pgfqpoint{6.758537in}{3.393793in}}{\pgfqpoint{6.767746in}{3.393793in}}%
\pgfpathlineto{\pgfqpoint{6.767746in}{3.393793in}}%
\pgfpathclose%
\pgfusepath{stroke,fill}%
\end{pgfscope}%
\begin{pgfscope}%
\pgfpathrectangle{\pgfqpoint{1.374500in}{0.082500in}}{\pgfqpoint{2.419000in}{2.419000in}}%
\pgfusepath{clip}%
\pgfsetbuttcap%
\pgfsetroundjoin%
\definecolor{currentfill}{rgb}{0.247059,0.564706,0.854902}%
\pgfsetfillcolor{currentfill}%
\pgfsetfillopacity{0.337588}%
\pgfsetlinewidth{1.003750pt}%
\definecolor{currentstroke}{rgb}{0.247059,0.564706,0.854902}%
\pgfsetstrokecolor{currentstroke}%
\pgfsetstrokeopacity{0.337588}%
\pgfsetdash{}{0pt}%
\pgfpathmoveto{\pgfqpoint{1.927420in}{3.325882in}}%
\pgfpathcurveto{\pgfqpoint{1.936629in}{3.325882in}}{\pgfqpoint{1.945461in}{3.329540in}}{\pgfqpoint{1.951973in}{3.336052in}}%
\pgfpathcurveto{\pgfqpoint{1.958484in}{3.342563in}}{\pgfqpoint{1.962143in}{3.351395in}}{\pgfqpoint{1.962143in}{3.360604in}}%
\pgfpathcurveto{\pgfqpoint{1.962143in}{3.369812in}}{\pgfqpoint{1.958484in}{3.378645in}}{\pgfqpoint{1.951973in}{3.385156in}}%
\pgfpathcurveto{\pgfqpoint{1.945461in}{3.391667in}}{\pgfqpoint{1.936629in}{3.395326in}}{\pgfqpoint{1.927420in}{3.395326in}}%
\pgfpathcurveto{\pgfqpoint{1.918212in}{3.395326in}}{\pgfqpoint{1.909379in}{3.391667in}}{\pgfqpoint{1.902868in}{3.385156in}}%
\pgfpathcurveto{\pgfqpoint{1.896357in}{3.378645in}}{\pgfqpoint{1.892698in}{3.369812in}}{\pgfqpoint{1.892698in}{3.360604in}}%
\pgfpathcurveto{\pgfqpoint{1.892698in}{3.351395in}}{\pgfqpoint{1.896357in}{3.342563in}}{\pgfqpoint{1.902868in}{3.336052in}}%
\pgfpathcurveto{\pgfqpoint{1.909379in}{3.329540in}}{\pgfqpoint{1.918212in}{3.325882in}}{\pgfqpoint{1.927420in}{3.325882in}}%
\pgfpathlineto{\pgfqpoint{1.927420in}{3.325882in}}%
\pgfpathclose%
\pgfusepath{stroke,fill}%
\end{pgfscope}%
\begin{pgfscope}%
\pgfpathrectangle{\pgfqpoint{1.374500in}{0.082500in}}{\pgfqpoint{2.419000in}{2.419000in}}%
\pgfusepath{clip}%
\pgfsetbuttcap%
\pgfsetroundjoin%
\definecolor{currentfill}{rgb}{0.247059,0.564706,0.854902}%
\pgfsetfillcolor{currentfill}%
\pgfsetfillopacity{0.337588}%
\pgfsetlinewidth{1.003750pt}%
\definecolor{currentstroke}{rgb}{0.247059,0.564706,0.854902}%
\pgfsetstrokecolor{currentstroke}%
\pgfsetstrokeopacity{0.337588}%
\pgfsetdash{}{0pt}%
\pgfpathmoveto{\pgfqpoint{5.406243in}{3.325882in}}%
\pgfpathcurveto{\pgfqpoint{5.415451in}{3.325882in}}{\pgfqpoint{5.424284in}{3.329540in}}{\pgfqpoint{5.430795in}{3.336052in}}%
\pgfpathcurveto{\pgfqpoint{5.437307in}{3.342563in}}{\pgfqpoint{5.440965in}{3.351395in}}{\pgfqpoint{5.440965in}{3.360604in}}%
\pgfpathcurveto{\pgfqpoint{5.440965in}{3.369812in}}{\pgfqpoint{5.437307in}{3.378645in}}{\pgfqpoint{5.430795in}{3.385156in}}%
\pgfpathcurveto{\pgfqpoint{5.424284in}{3.391667in}}{\pgfqpoint{5.415451in}{3.395326in}}{\pgfqpoint{5.406243in}{3.395326in}}%
\pgfpathcurveto{\pgfqpoint{5.397035in}{3.395326in}}{\pgfqpoint{5.388202in}{3.391667in}}{\pgfqpoint{5.381691in}{3.385156in}}%
\pgfpathcurveto{\pgfqpoint{5.375179in}{3.378645in}}{\pgfqpoint{5.371521in}{3.369812in}}{\pgfqpoint{5.371521in}{3.360604in}}%
\pgfpathcurveto{\pgfqpoint{5.371521in}{3.351395in}}{\pgfqpoint{5.375179in}{3.342563in}}{\pgfqpoint{5.381691in}{3.336052in}}%
\pgfpathcurveto{\pgfqpoint{5.388202in}{3.329540in}}{\pgfqpoint{5.397035in}{3.325882in}}{\pgfqpoint{5.406243in}{3.325882in}}%
\pgfpathlineto{\pgfqpoint{5.406243in}{3.325882in}}%
\pgfpathclose%
\pgfusepath{stroke,fill}%
\end{pgfscope}%
\begin{pgfscope}%
\pgfpathrectangle{\pgfqpoint{1.374500in}{0.082500in}}{\pgfqpoint{2.419000in}{2.419000in}}%
\pgfusepath{clip}%
\pgfsetbuttcap%
\pgfsetroundjoin%
\definecolor{currentfill}{rgb}{0.247059,0.564706,0.854902}%
\pgfsetfillcolor{currentfill}%
\pgfsetfillopacity{0.340991}%
\pgfsetlinewidth{1.003750pt}%
\definecolor{currentstroke}{rgb}{0.247059,0.564706,0.854902}%
\pgfsetstrokecolor{currentstroke}%
\pgfsetstrokeopacity{0.340991}%
\pgfsetdash{}{0pt}%
\pgfpathmoveto{\pgfqpoint{4.010352in}{3.256255in}}%
\pgfpathcurveto{\pgfqpoint{4.019560in}{3.256255in}}{\pgfqpoint{4.028393in}{3.259914in}}{\pgfqpoint{4.034904in}{3.266425in}}%
\pgfpathcurveto{\pgfqpoint{4.041416in}{3.272937in}}{\pgfqpoint{4.045074in}{3.281769in}}{\pgfqpoint{4.045074in}{3.290977in}}%
\pgfpathcurveto{\pgfqpoint{4.045074in}{3.300186in}}{\pgfqpoint{4.041416in}{3.309018in}}{\pgfqpoint{4.034904in}{3.315530in}}%
\pgfpathcurveto{\pgfqpoint{4.028393in}{3.322041in}}{\pgfqpoint{4.019560in}{3.325700in}}{\pgfqpoint{4.010352in}{3.325700in}}%
\pgfpathcurveto{\pgfqpoint{4.001144in}{3.325700in}}{\pgfqpoint{3.992311in}{3.322041in}}{\pgfqpoint{3.985800in}{3.315530in}}%
\pgfpathcurveto{\pgfqpoint{3.979288in}{3.309018in}}{\pgfqpoint{3.975630in}{3.300186in}}{\pgfqpoint{3.975630in}{3.290977in}}%
\pgfpathcurveto{\pgfqpoint{3.975630in}{3.281769in}}{\pgfqpoint{3.979288in}{3.272937in}}{\pgfqpoint{3.985800in}{3.266425in}}%
\pgfpathcurveto{\pgfqpoint{3.992311in}{3.259914in}}{\pgfqpoint{4.001144in}{3.256255in}}{\pgfqpoint{4.010352in}{3.256255in}}%
\pgfpathlineto{\pgfqpoint{4.010352in}{3.256255in}}%
\pgfpathclose%
\pgfusepath{stroke,fill}%
\end{pgfscope}%
\begin{pgfscope}%
\pgfpathrectangle{\pgfqpoint{1.374500in}{0.082500in}}{\pgfqpoint{2.419000in}{2.419000in}}%
\pgfusepath{clip}%
\pgfsetbuttcap%
\pgfsetroundjoin%
\definecolor{currentfill}{rgb}{0.247059,0.564706,0.854902}%
\pgfsetfillcolor{currentfill}%
\pgfsetfillopacity{0.340991}%
\pgfsetlinewidth{1.003750pt}%
\definecolor{currentstroke}{rgb}{0.247059,0.564706,0.854902}%
\pgfsetstrokecolor{currentstroke}%
\pgfsetstrokeopacity{0.340991}%
\pgfsetdash{}{0pt}%
\pgfpathmoveto{\pgfqpoint{7.533108in}{3.256255in}}%
\pgfpathcurveto{\pgfqpoint{7.542316in}{3.256255in}}{\pgfqpoint{7.551149in}{3.259914in}}{\pgfqpoint{7.557660in}{3.266425in}}%
\pgfpathcurveto{\pgfqpoint{7.564171in}{3.272937in}}{\pgfqpoint{7.567830in}{3.281769in}}{\pgfqpoint{7.567830in}{3.290977in}}%
\pgfpathcurveto{\pgfqpoint{7.567830in}{3.300186in}}{\pgfqpoint{7.564171in}{3.309018in}}{\pgfqpoint{7.557660in}{3.315530in}}%
\pgfpathcurveto{\pgfqpoint{7.551149in}{3.322041in}}{\pgfqpoint{7.542316in}{3.325700in}}{\pgfqpoint{7.533108in}{3.325700in}}%
\pgfpathcurveto{\pgfqpoint{7.523899in}{3.325700in}}{\pgfqpoint{7.515067in}{3.322041in}}{\pgfqpoint{7.508555in}{3.315530in}}%
\pgfpathcurveto{\pgfqpoint{7.502044in}{3.309018in}}{\pgfqpoint{7.498386in}{3.300186in}}{\pgfqpoint{7.498386in}{3.290977in}}%
\pgfpathcurveto{\pgfqpoint{7.498386in}{3.281769in}}{\pgfqpoint{7.502044in}{3.272937in}}{\pgfqpoint{7.508555in}{3.266425in}}%
\pgfpathcurveto{\pgfqpoint{7.515067in}{3.259914in}}{\pgfqpoint{7.523899in}{3.256255in}}{\pgfqpoint{7.533108in}{3.256255in}}%
\pgfpathlineto{\pgfqpoint{7.533108in}{3.256255in}}%
\pgfpathclose%
\pgfusepath{stroke,fill}%
\end{pgfscope}%
\begin{pgfscope}%
\pgfpathrectangle{\pgfqpoint{1.374500in}{0.082500in}}{\pgfqpoint{2.419000in}{2.419000in}}%
\pgfusepath{clip}%
\pgfsetbuttcap%
\pgfsetroundjoin%
\definecolor{currentfill}{rgb}{0.247059,0.564706,0.854902}%
\pgfsetfillcolor{currentfill}%
\pgfsetfillopacity{0.344480}%
\pgfsetlinewidth{1.003750pt}%
\definecolor{currentstroke}{rgb}{0.247059,0.564706,0.854902}%
\pgfsetstrokecolor{currentstroke}%
\pgfsetstrokeopacity{0.344480}%
\pgfsetdash{}{0pt}%
\pgfpathmoveto{\pgfqpoint{6.146566in}{3.184848in}}%
\pgfpathcurveto{\pgfqpoint{6.155775in}{3.184848in}}{\pgfqpoint{6.164607in}{3.188506in}}{\pgfqpoint{6.171118in}{3.195018in}}%
\pgfpathcurveto{\pgfqpoint{6.177630in}{3.201529in}}{\pgfqpoint{6.181288in}{3.210362in}}{\pgfqpoint{6.181288in}{3.219570in}}%
\pgfpathcurveto{\pgfqpoint{6.181288in}{3.228778in}}{\pgfqpoint{6.177630in}{3.237611in}}{\pgfqpoint{6.171118in}{3.244122in}}%
\pgfpathcurveto{\pgfqpoint{6.164607in}{3.250634in}}{\pgfqpoint{6.155775in}{3.254292in}}{\pgfqpoint{6.146566in}{3.254292in}}%
\pgfpathcurveto{\pgfqpoint{6.137358in}{3.254292in}}{\pgfqpoint{6.128525in}{3.250634in}}{\pgfqpoint{6.122014in}{3.244122in}}%
\pgfpathcurveto{\pgfqpoint{6.115502in}{3.237611in}}{\pgfqpoint{6.111844in}{3.228778in}}{\pgfqpoint{6.111844in}{3.219570in}}%
\pgfpathcurveto{\pgfqpoint{6.111844in}{3.210362in}}{\pgfqpoint{6.115502in}{3.201529in}}{\pgfqpoint{6.122014in}{3.195018in}}%
\pgfpathcurveto{\pgfqpoint{6.128525in}{3.188506in}}{\pgfqpoint{6.137358in}{3.184848in}}{\pgfqpoint{6.146566in}{3.184848in}}%
\pgfpathlineto{\pgfqpoint{6.146566in}{3.184848in}}%
\pgfpathclose%
\pgfusepath{stroke,fill}%
\end{pgfscope}%
\begin{pgfscope}%
\pgfpathrectangle{\pgfqpoint{1.374500in}{0.082500in}}{\pgfqpoint{2.419000in}{2.419000in}}%
\pgfusepath{clip}%
\pgfsetbuttcap%
\pgfsetroundjoin%
\definecolor{currentfill}{rgb}{0.247059,0.564706,0.854902}%
\pgfsetfillcolor{currentfill}%
\pgfsetfillopacity{0.344480}%
\pgfsetlinewidth{1.003750pt}%
\definecolor{currentstroke}{rgb}{0.247059,0.564706,0.854902}%
\pgfsetstrokecolor{currentstroke}%
\pgfsetstrokeopacity{0.344480}%
\pgfsetdash{}{0pt}%
\pgfpathmoveto{\pgfqpoint{2.578753in}{3.184848in}}%
\pgfpathcurveto{\pgfqpoint{2.587962in}{3.184848in}}{\pgfqpoint{2.596794in}{3.188506in}}{\pgfqpoint{2.603306in}{3.195018in}}%
\pgfpathcurveto{\pgfqpoint{2.609817in}{3.201529in}}{\pgfqpoint{2.613476in}{3.210362in}}{\pgfqpoint{2.613476in}{3.219570in}}%
\pgfpathcurveto{\pgfqpoint{2.613476in}{3.228778in}}{\pgfqpoint{2.609817in}{3.237611in}}{\pgfqpoint{2.603306in}{3.244122in}}%
\pgfpathcurveto{\pgfqpoint{2.596794in}{3.250634in}}{\pgfqpoint{2.587962in}{3.254292in}}{\pgfqpoint{2.578753in}{3.254292in}}%
\pgfpathcurveto{\pgfqpoint{2.569545in}{3.254292in}}{\pgfqpoint{2.560712in}{3.250634in}}{\pgfqpoint{2.554201in}{3.244122in}}%
\pgfpathcurveto{\pgfqpoint{2.547690in}{3.237611in}}{\pgfqpoint{2.544031in}{3.228778in}}{\pgfqpoint{2.544031in}{3.219570in}}%
\pgfpathcurveto{\pgfqpoint{2.544031in}{3.210362in}}{\pgfqpoint{2.547690in}{3.201529in}}{\pgfqpoint{2.554201in}{3.195018in}}%
\pgfpathcurveto{\pgfqpoint{2.560712in}{3.188506in}}{\pgfqpoint{2.569545in}{3.184848in}}{\pgfqpoint{2.578753in}{3.184848in}}%
\pgfpathlineto{\pgfqpoint{2.578753in}{3.184848in}}%
\pgfpathclose%
\pgfusepath{stroke,fill}%
\end{pgfscope}%
\begin{pgfscope}%
\pgfpathrectangle{\pgfqpoint{1.374500in}{0.082500in}}{\pgfqpoint{2.419000in}{2.419000in}}%
\pgfusepath{clip}%
\pgfsetbuttcap%
\pgfsetroundjoin%
\definecolor{currentfill}{rgb}{0.247059,0.564706,0.854902}%
\pgfsetfillcolor{currentfill}%
\pgfsetfillopacity{0.348060}%
\pgfsetlinewidth{1.003750pt}%
\definecolor{currentstroke}{rgb}{0.247059,0.564706,0.854902}%
\pgfsetstrokecolor{currentstroke}%
\pgfsetstrokeopacity{0.348060}%
\pgfsetdash{}{0pt}%
\pgfpathmoveto{\pgfqpoint{1.110059in}{3.111590in}}%
\pgfpathcurveto{\pgfqpoint{1.119268in}{3.111590in}}{\pgfqpoint{1.128100in}{3.115249in}}{\pgfqpoint{1.134612in}{3.121760in}}%
\pgfpathcurveto{\pgfqpoint{1.141123in}{3.128271in}}{\pgfqpoint{1.144782in}{3.137104in}}{\pgfqpoint{1.144782in}{3.146312in}}%
\pgfpathcurveto{\pgfqpoint{1.144782in}{3.155521in}}{\pgfqpoint{1.141123in}{3.164353in}}{\pgfqpoint{1.134612in}{3.170865in}}%
\pgfpathcurveto{\pgfqpoint{1.128100in}{3.177376in}}{\pgfqpoint{1.119268in}{3.181035in}}{\pgfqpoint{1.110059in}{3.181035in}}%
\pgfpathcurveto{\pgfqpoint{1.100851in}{3.181035in}}{\pgfqpoint{1.092018in}{3.177376in}}{\pgfqpoint{1.085507in}{3.170865in}}%
\pgfpathcurveto{\pgfqpoint{1.078996in}{3.164353in}}{\pgfqpoint{1.075337in}{3.155521in}}{\pgfqpoint{1.075337in}{3.146312in}}%
\pgfpathcurveto{\pgfqpoint{1.075337in}{3.137104in}}{\pgfqpoint{1.078996in}{3.128271in}}{\pgfqpoint{1.085507in}{3.121760in}}%
\pgfpathcurveto{\pgfqpoint{1.092018in}{3.115249in}}{\pgfqpoint{1.100851in}{3.111590in}}{\pgfqpoint{1.110059in}{3.111590in}}%
\pgfpathlineto{\pgfqpoint{1.110059in}{3.111590in}}%
\pgfpathclose%
\pgfusepath{stroke,fill}%
\end{pgfscope}%
\begin{pgfscope}%
\pgfpathrectangle{\pgfqpoint{1.374500in}{0.082500in}}{\pgfqpoint{2.419000in}{2.419000in}}%
\pgfusepath{clip}%
\pgfsetbuttcap%
\pgfsetroundjoin%
\definecolor{currentfill}{rgb}{0.247059,0.564706,0.854902}%
\pgfsetfillcolor{currentfill}%
\pgfsetfillopacity{0.348060}%
\pgfsetlinewidth{1.003750pt}%
\definecolor{currentstroke}{rgb}{0.247059,0.564706,0.854902}%
\pgfsetstrokecolor{currentstroke}%
\pgfsetstrokeopacity{0.348060}%
\pgfsetdash{}{0pt}%
\pgfpathmoveto{\pgfqpoint{4.724096in}{3.111590in}}%
\pgfpathcurveto{\pgfqpoint{4.733305in}{3.111590in}}{\pgfqpoint{4.742137in}{3.115249in}}{\pgfqpoint{4.748649in}{3.121760in}}%
\pgfpathcurveto{\pgfqpoint{4.755160in}{3.128271in}}{\pgfqpoint{4.758819in}{3.137104in}}{\pgfqpoint{4.758819in}{3.146312in}}%
\pgfpathcurveto{\pgfqpoint{4.758819in}{3.155521in}}{\pgfqpoint{4.755160in}{3.164353in}}{\pgfqpoint{4.748649in}{3.170865in}}%
\pgfpathcurveto{\pgfqpoint{4.742137in}{3.177376in}}{\pgfqpoint{4.733305in}{3.181035in}}{\pgfqpoint{4.724096in}{3.181035in}}%
\pgfpathcurveto{\pgfqpoint{4.714888in}{3.181035in}}{\pgfqpoint{4.706055in}{3.177376in}}{\pgfqpoint{4.699544in}{3.170865in}}%
\pgfpathcurveto{\pgfqpoint{4.693033in}{3.164353in}}{\pgfqpoint{4.689374in}{3.155521in}}{\pgfqpoint{4.689374in}{3.146312in}}%
\pgfpathcurveto{\pgfqpoint{4.689374in}{3.137104in}}{\pgfqpoint{4.693033in}{3.128271in}}{\pgfqpoint{4.699544in}{3.121760in}}%
\pgfpathcurveto{\pgfqpoint{4.706055in}{3.115249in}}{\pgfqpoint{4.714888in}{3.111590in}}{\pgfqpoint{4.724096in}{3.111590in}}%
\pgfpathlineto{\pgfqpoint{4.724096in}{3.111590in}}%
\pgfpathclose%
\pgfusepath{stroke,fill}%
\end{pgfscope}%
\begin{pgfscope}%
\pgfpathrectangle{\pgfqpoint{1.374500in}{0.082500in}}{\pgfqpoint{2.419000in}{2.419000in}}%
\pgfusepath{clip}%
\pgfsetbuttcap%
\pgfsetroundjoin%
\definecolor{currentfill}{rgb}{0.247059,0.564706,0.854902}%
\pgfsetfillcolor{currentfill}%
\pgfsetfillopacity{0.348060}%
\pgfsetlinewidth{1.003750pt}%
\definecolor{currentstroke}{rgb}{0.247059,0.564706,0.854902}%
\pgfsetstrokecolor{currentstroke}%
\pgfsetstrokeopacity{0.348060}%
\pgfsetdash{}{0pt}%
\pgfpathmoveto{\pgfqpoint{8.338134in}{3.111590in}}%
\pgfpathcurveto{\pgfqpoint{8.347342in}{3.111590in}}{\pgfqpoint{8.356175in}{3.115249in}}{\pgfqpoint{8.362686in}{3.121760in}}%
\pgfpathcurveto{\pgfqpoint{8.369197in}{3.128271in}}{\pgfqpoint{8.372856in}{3.137104in}}{\pgfqpoint{8.372856in}{3.146312in}}%
\pgfpathcurveto{\pgfqpoint{8.372856in}{3.155521in}}{\pgfqpoint{8.369197in}{3.164353in}}{\pgfqpoint{8.362686in}{3.170865in}}%
\pgfpathcurveto{\pgfqpoint{8.356175in}{3.177376in}}{\pgfqpoint{8.347342in}{3.181035in}}{\pgfqpoint{8.338134in}{3.181035in}}%
\pgfpathcurveto{\pgfqpoint{8.328925in}{3.181035in}}{\pgfqpoint{8.320093in}{3.177376in}}{\pgfqpoint{8.313581in}{3.170865in}}%
\pgfpathcurveto{\pgfqpoint{8.307070in}{3.164353in}}{\pgfqpoint{8.303411in}{3.155521in}}{\pgfqpoint{8.303411in}{3.146312in}}%
\pgfpathcurveto{\pgfqpoint{8.303411in}{3.137104in}}{\pgfqpoint{8.307070in}{3.128271in}}{\pgfqpoint{8.313581in}{3.121760in}}%
\pgfpathcurveto{\pgfqpoint{8.320093in}{3.115249in}}{\pgfqpoint{8.328925in}{3.111590in}}{\pgfqpoint{8.338134in}{3.111590in}}%
\pgfpathlineto{\pgfqpoint{8.338134in}{3.111590in}}%
\pgfpathclose%
\pgfusepath{stroke,fill}%
\end{pgfscope}%
\begin{pgfscope}%
\pgfpathrectangle{\pgfqpoint{1.374500in}{0.082500in}}{\pgfqpoint{2.419000in}{2.419000in}}%
\pgfusepath{clip}%
\pgfsetbuttcap%
\pgfsetroundjoin%
\definecolor{currentfill}{rgb}{0.247059,0.564706,0.854902}%
\pgfsetfillcolor{currentfill}%
\pgfsetfillopacity{0.351735}%
\pgfsetlinewidth{1.003750pt}%
\definecolor{currentstroke}{rgb}{0.247059,0.564706,0.854902}%
\pgfsetstrokecolor{currentstroke}%
\pgfsetstrokeopacity{0.351735}%
\pgfsetdash{}{0pt}%
\pgfpathmoveto{\pgfqpoint{6.925759in}{3.036409in}}%
\pgfpathcurveto{\pgfqpoint{6.934968in}{3.036409in}}{\pgfqpoint{6.943800in}{3.040068in}}{\pgfqpoint{6.950311in}{3.046579in}}%
\pgfpathcurveto{\pgfqpoint{6.956823in}{3.053090in}}{\pgfqpoint{6.960481in}{3.061923in}}{\pgfqpoint{6.960481in}{3.071131in}}%
\pgfpathcurveto{\pgfqpoint{6.960481in}{3.080340in}}{\pgfqpoint{6.956823in}{3.089172in}}{\pgfqpoint{6.950311in}{3.095684in}}%
\pgfpathcurveto{\pgfqpoint{6.943800in}{3.102195in}}{\pgfqpoint{6.934968in}{3.105854in}}{\pgfqpoint{6.925759in}{3.105854in}}%
\pgfpathcurveto{\pgfqpoint{6.916551in}{3.105854in}}{\pgfqpoint{6.907718in}{3.102195in}}{\pgfqpoint{6.901207in}{3.095684in}}%
\pgfpathcurveto{\pgfqpoint{6.894695in}{3.089172in}}{\pgfqpoint{6.891037in}{3.080340in}}{\pgfqpoint{6.891037in}{3.071131in}}%
\pgfpathcurveto{\pgfqpoint{6.891037in}{3.061923in}}{\pgfqpoint{6.894695in}{3.053090in}}{\pgfqpoint{6.901207in}{3.046579in}}%
\pgfpathcurveto{\pgfqpoint{6.907718in}{3.040068in}}{\pgfqpoint{6.916551in}{3.036409in}}{\pgfqpoint{6.925759in}{3.036409in}}%
\pgfpathlineto{\pgfqpoint{6.925759in}{3.036409in}}%
\pgfpathclose%
\pgfusepath{stroke,fill}%
\end{pgfscope}%
\begin{pgfscope}%
\pgfpathrectangle{\pgfqpoint{1.374500in}{0.082500in}}{\pgfqpoint{2.419000in}{2.419000in}}%
\pgfusepath{clip}%
\pgfsetbuttcap%
\pgfsetroundjoin%
\definecolor{currentfill}{rgb}{0.247059,0.564706,0.854902}%
\pgfsetfillcolor{currentfill}%
\pgfsetfillopacity{0.351735}%
\pgfsetlinewidth{1.003750pt}%
\definecolor{currentstroke}{rgb}{0.247059,0.564706,0.854902}%
\pgfsetstrokecolor{currentstroke}%
\pgfsetstrokeopacity{0.351735}%
\pgfsetdash{}{0pt}%
\pgfpathmoveto{\pgfqpoint{3.264284in}{3.036409in}}%
\pgfpathcurveto{\pgfqpoint{3.273493in}{3.036409in}}{\pgfqpoint{3.282325in}{3.040068in}}{\pgfqpoint{3.288836in}{3.046579in}}%
\pgfpathcurveto{\pgfqpoint{3.295348in}{3.053090in}}{\pgfqpoint{3.299006in}{3.061923in}}{\pgfqpoint{3.299006in}{3.071131in}}%
\pgfpathcurveto{\pgfqpoint{3.299006in}{3.080340in}}{\pgfqpoint{3.295348in}{3.089172in}}{\pgfqpoint{3.288836in}{3.095684in}}%
\pgfpathcurveto{\pgfqpoint{3.282325in}{3.102195in}}{\pgfqpoint{3.273493in}{3.105854in}}{\pgfqpoint{3.264284in}{3.105854in}}%
\pgfpathcurveto{\pgfqpoint{3.255076in}{3.105854in}}{\pgfqpoint{3.246243in}{3.102195in}}{\pgfqpoint{3.239732in}{3.095684in}}%
\pgfpathcurveto{\pgfqpoint{3.233220in}{3.089172in}}{\pgfqpoint{3.229562in}{3.080340in}}{\pgfqpoint{3.229562in}{3.071131in}}%
\pgfpathcurveto{\pgfqpoint{3.229562in}{3.061923in}}{\pgfqpoint{3.233220in}{3.053090in}}{\pgfqpoint{3.239732in}{3.046579in}}%
\pgfpathcurveto{\pgfqpoint{3.246243in}{3.040068in}}{\pgfqpoint{3.255076in}{3.036409in}}{\pgfqpoint{3.264284in}{3.036409in}}%
\pgfpathlineto{\pgfqpoint{3.264284in}{3.036409in}}%
\pgfpathclose%
\pgfusepath{stroke,fill}%
\end{pgfscope}%
\begin{pgfscope}%
\pgfpathrectangle{\pgfqpoint{1.374500in}{0.082500in}}{\pgfqpoint{2.419000in}{2.419000in}}%
\pgfusepath{clip}%
\pgfsetbuttcap%
\pgfsetroundjoin%
\definecolor{currentfill}{rgb}{0.247059,0.564706,0.854902}%
\pgfsetfillcolor{currentfill}%
\pgfsetfillopacity{0.355506}%
\pgfsetlinewidth{1.003750pt}%
\definecolor{currentstroke}{rgb}{0.247059,0.564706,0.854902}%
\pgfsetstrokecolor{currentstroke}%
\pgfsetstrokeopacity{0.355506}%
\pgfsetdash{}{0pt}%
\pgfpathmoveto{\pgfqpoint{1.765639in}{2.959228in}}%
\pgfpathcurveto{\pgfqpoint{1.774847in}{2.959228in}}{\pgfqpoint{1.783680in}{2.962887in}}{\pgfqpoint{1.790191in}{2.969398in}}%
\pgfpathcurveto{\pgfqpoint{1.796703in}{2.975910in}}{\pgfqpoint{1.800361in}{2.984742in}}{\pgfqpoint{1.800361in}{2.993951in}}%
\pgfpathcurveto{\pgfqpoint{1.800361in}{3.003159in}}{\pgfqpoint{1.796703in}{3.011992in}}{\pgfqpoint{1.790191in}{3.018503in}}%
\pgfpathcurveto{\pgfqpoint{1.783680in}{3.025014in}}{\pgfqpoint{1.774847in}{3.028673in}}{\pgfqpoint{1.765639in}{3.028673in}}%
\pgfpathcurveto{\pgfqpoint{1.756430in}{3.028673in}}{\pgfqpoint{1.747598in}{3.025014in}}{\pgfqpoint{1.741087in}{3.018503in}}%
\pgfpathcurveto{\pgfqpoint{1.734575in}{3.011992in}}{\pgfqpoint{1.730917in}{3.003159in}}{\pgfqpoint{1.730917in}{2.993951in}}%
\pgfpathcurveto{\pgfqpoint{1.730917in}{2.984742in}}{\pgfqpoint{1.734575in}{2.975910in}}{\pgfqpoint{1.741087in}{2.969398in}}%
\pgfpathcurveto{\pgfqpoint{1.747598in}{2.962887in}}{\pgfqpoint{1.756430in}{2.959228in}}{\pgfqpoint{1.765639in}{2.959228in}}%
\pgfpathlineto{\pgfqpoint{1.765639in}{2.959228in}}%
\pgfpathclose%
\pgfusepath{stroke,fill}%
\end{pgfscope}%
\begin{pgfscope}%
\pgfpathrectangle{\pgfqpoint{1.374500in}{0.082500in}}{\pgfqpoint{2.419000in}{2.419000in}}%
\pgfusepath{clip}%
\pgfsetbuttcap%
\pgfsetroundjoin%
\definecolor{currentfill}{rgb}{0.247059,0.564706,0.854902}%
\pgfsetfillcolor{currentfill}%
\pgfsetfillopacity{0.355506}%
\pgfsetlinewidth{1.003750pt}%
\definecolor{currentstroke}{rgb}{0.247059,0.564706,0.854902}%
\pgfsetstrokecolor{currentstroke}%
\pgfsetstrokeopacity{0.355506}%
\pgfsetdash{}{0pt}%
\pgfpathmoveto{\pgfqpoint{5.475814in}{2.959228in}}%
\pgfpathcurveto{\pgfqpoint{5.485022in}{2.959228in}}{\pgfqpoint{5.493855in}{2.962887in}}{\pgfqpoint{5.500366in}{2.969398in}}%
\pgfpathcurveto{\pgfqpoint{5.506878in}{2.975910in}}{\pgfqpoint{5.510536in}{2.984742in}}{\pgfqpoint{5.510536in}{2.993951in}}%
\pgfpathcurveto{\pgfqpoint{5.510536in}{3.003159in}}{\pgfqpoint{5.506878in}{3.011992in}}{\pgfqpoint{5.500366in}{3.018503in}}%
\pgfpathcurveto{\pgfqpoint{5.493855in}{3.025014in}}{\pgfqpoint{5.485022in}{3.028673in}}{\pgfqpoint{5.475814in}{3.028673in}}%
\pgfpathcurveto{\pgfqpoint{5.466605in}{3.028673in}}{\pgfqpoint{5.457773in}{3.025014in}}{\pgfqpoint{5.451262in}{3.018503in}}%
\pgfpathcurveto{\pgfqpoint{5.444750in}{3.011992in}}{\pgfqpoint{5.441092in}{3.003159in}}{\pgfqpoint{5.441092in}{2.993951in}}%
\pgfpathcurveto{\pgfqpoint{5.441092in}{2.984742in}}{\pgfqpoint{5.444750in}{2.975910in}}{\pgfqpoint{5.451262in}{2.969398in}}%
\pgfpathcurveto{\pgfqpoint{5.457773in}{2.962887in}}{\pgfqpoint{5.466605in}{2.959228in}}{\pgfqpoint{5.475814in}{2.959228in}}%
\pgfpathlineto{\pgfqpoint{5.475814in}{2.959228in}}%
\pgfpathclose%
\pgfusepath{stroke,fill}%
\end{pgfscope}%
\begin{pgfscope}%
\pgfpathrectangle{\pgfqpoint{1.374500in}{0.082500in}}{\pgfqpoint{2.419000in}{2.419000in}}%
\pgfusepath{clip}%
\pgfsetbuttcap%
\pgfsetroundjoin%
\definecolor{currentfill}{rgb}{0.247059,0.564706,0.854902}%
\pgfsetfillcolor{currentfill}%
\pgfsetfillopacity{0.355506}%
\pgfsetlinewidth{1.003750pt}%
\definecolor{currentstroke}{rgb}{0.247059,0.564706,0.854902}%
\pgfsetstrokecolor{currentstroke}%
\pgfsetstrokeopacity{0.355506}%
\pgfsetdash{}{0pt}%
\pgfpathmoveto{\pgfqpoint{9.185989in}{2.959228in}}%
\pgfpathcurveto{\pgfqpoint{9.195197in}{2.959228in}}{\pgfqpoint{9.204030in}{2.962887in}}{\pgfqpoint{9.210541in}{2.969398in}}%
\pgfpathcurveto{\pgfqpoint{9.217052in}{2.975910in}}{\pgfqpoint{9.220711in}{2.984742in}}{\pgfqpoint{9.220711in}{2.993951in}}%
\pgfpathcurveto{\pgfqpoint{9.220711in}{3.003159in}}{\pgfqpoint{9.217052in}{3.011992in}}{\pgfqpoint{9.210541in}{3.018503in}}%
\pgfpathcurveto{\pgfqpoint{9.204030in}{3.025014in}}{\pgfqpoint{9.195197in}{3.028673in}}{\pgfqpoint{9.185989in}{3.028673in}}%
\pgfpathcurveto{\pgfqpoint{9.176780in}{3.028673in}}{\pgfqpoint{9.167948in}{3.025014in}}{\pgfqpoint{9.161436in}{3.018503in}}%
\pgfpathcurveto{\pgfqpoint{9.154925in}{3.011992in}}{\pgfqpoint{9.151267in}{3.003159in}}{\pgfqpoint{9.151267in}{2.993951in}}%
\pgfpathcurveto{\pgfqpoint{9.151267in}{2.984742in}}{\pgfqpoint{9.154925in}{2.975910in}}{\pgfqpoint{9.161436in}{2.969398in}}%
\pgfpathcurveto{\pgfqpoint{9.167948in}{2.962887in}}{\pgfqpoint{9.176780in}{2.959228in}}{\pgfqpoint{9.185989in}{2.959228in}}%
\pgfpathlineto{\pgfqpoint{9.185989in}{2.959228in}}%
\pgfpathclose%
\pgfusepath{stroke,fill}%
\end{pgfscope}%
\begin{pgfscope}%
\pgfpathrectangle{\pgfqpoint{1.374500in}{0.082500in}}{\pgfqpoint{2.419000in}{2.419000in}}%
\pgfusepath{clip}%
\pgfsetbuttcap%
\pgfsetroundjoin%
\definecolor{currentfill}{rgb}{0.247059,0.564706,0.854902}%
\pgfsetfillcolor{currentfill}%
\pgfsetfillopacity{0.359380}%
\pgfsetlinewidth{1.003750pt}%
\definecolor{currentstroke}{rgb}{0.247059,0.564706,0.854902}%
\pgfsetstrokecolor{currentstroke}%
\pgfsetstrokeopacity{0.359380}%
\pgfsetdash{}{0pt}%
\pgfpathmoveto{\pgfqpoint{3.986778in}{2.879967in}}%
\pgfpathcurveto{\pgfqpoint{3.995987in}{2.879967in}}{\pgfqpoint{4.004819in}{2.883625in}}{\pgfqpoint{4.011331in}{2.890137in}}%
\pgfpathcurveto{\pgfqpoint{4.017842in}{2.896648in}}{\pgfqpoint{4.021501in}{2.905481in}}{\pgfqpoint{4.021501in}{2.914689in}}%
\pgfpathcurveto{\pgfqpoint{4.021501in}{2.923897in}}{\pgfqpoint{4.017842in}{2.932730in}}{\pgfqpoint{4.011331in}{2.939241in}}%
\pgfpathcurveto{\pgfqpoint{4.004819in}{2.945753in}}{\pgfqpoint{3.995987in}{2.949411in}}{\pgfqpoint{3.986778in}{2.949411in}}%
\pgfpathcurveto{\pgfqpoint{3.977570in}{2.949411in}}{\pgfqpoint{3.968737in}{2.945753in}}{\pgfqpoint{3.962226in}{2.939241in}}%
\pgfpathcurveto{\pgfqpoint{3.955715in}{2.932730in}}{\pgfqpoint{3.952056in}{2.923897in}}{\pgfqpoint{3.952056in}{2.914689in}}%
\pgfpathcurveto{\pgfqpoint{3.952056in}{2.905481in}}{\pgfqpoint{3.955715in}{2.896648in}}{\pgfqpoint{3.962226in}{2.890137in}}%
\pgfpathcurveto{\pgfqpoint{3.968737in}{2.883625in}}{\pgfqpoint{3.977570in}{2.879967in}}{\pgfqpoint{3.986778in}{2.879967in}}%
\pgfpathlineto{\pgfqpoint{3.986778in}{2.879967in}}%
\pgfpathclose%
\pgfusepath{stroke,fill}%
\end{pgfscope}%
\begin{pgfscope}%
\pgfpathrectangle{\pgfqpoint{1.374500in}{0.082500in}}{\pgfqpoint{2.419000in}{2.419000in}}%
\pgfusepath{clip}%
\pgfsetbuttcap%
\pgfsetroundjoin%
\definecolor{currentfill}{rgb}{0.247059,0.564706,0.854902}%
\pgfsetfillcolor{currentfill}%
\pgfsetfillopacity{0.359380}%
\pgfsetlinewidth{1.003750pt}%
\definecolor{currentstroke}{rgb}{0.247059,0.564706,0.854902}%
\pgfsetstrokecolor{currentstroke}%
\pgfsetstrokeopacity{0.359380}%
\pgfsetdash{}{0pt}%
\pgfpathmoveto{\pgfqpoint{7.746966in}{2.879967in}}%
\pgfpathcurveto{\pgfqpoint{7.756174in}{2.879967in}}{\pgfqpoint{7.765007in}{2.883625in}}{\pgfqpoint{7.771518in}{2.890137in}}%
\pgfpathcurveto{\pgfqpoint{7.778030in}{2.896648in}}{\pgfqpoint{7.781688in}{2.905481in}}{\pgfqpoint{7.781688in}{2.914689in}}%
\pgfpathcurveto{\pgfqpoint{7.781688in}{2.923897in}}{\pgfqpoint{7.778030in}{2.932730in}}{\pgfqpoint{7.771518in}{2.939241in}}%
\pgfpathcurveto{\pgfqpoint{7.765007in}{2.945753in}}{\pgfqpoint{7.756174in}{2.949411in}}{\pgfqpoint{7.746966in}{2.949411in}}%
\pgfpathcurveto{\pgfqpoint{7.737758in}{2.949411in}}{\pgfqpoint{7.728925in}{2.945753in}}{\pgfqpoint{7.722414in}{2.939241in}}%
\pgfpathcurveto{\pgfqpoint{7.715902in}{2.932730in}}{\pgfqpoint{7.712244in}{2.923897in}}{\pgfqpoint{7.712244in}{2.914689in}}%
\pgfpathcurveto{\pgfqpoint{7.712244in}{2.905481in}}{\pgfqpoint{7.715902in}{2.896648in}}{\pgfqpoint{7.722414in}{2.890137in}}%
\pgfpathcurveto{\pgfqpoint{7.728925in}{2.883625in}}{\pgfqpoint{7.737758in}{2.879967in}}{\pgfqpoint{7.746966in}{2.879967in}}%
\pgfpathlineto{\pgfqpoint{7.746966in}{2.879967in}}%
\pgfpathclose%
\pgfusepath{stroke,fill}%
\end{pgfscope}%
\begin{pgfscope}%
\pgfpathrectangle{\pgfqpoint{1.374500in}{0.082500in}}{\pgfqpoint{2.419000in}{2.419000in}}%
\pgfusepath{clip}%
\pgfsetbuttcap%
\pgfsetroundjoin%
\definecolor{currentfill}{rgb}{0.247059,0.564706,0.854902}%
\pgfsetfillcolor{currentfill}%
\pgfsetfillopacity{0.363359}%
\pgfsetlinewidth{1.003750pt}%
\definecolor{currentstroke}{rgb}{0.247059,0.564706,0.854902}%
\pgfsetstrokecolor{currentstroke}%
\pgfsetstrokeopacity{0.363359}%
\pgfsetdash{}{0pt}%
\pgfpathmoveto{\pgfqpoint{2.457050in}{2.798539in}}%
\pgfpathcurveto{\pgfqpoint{2.466259in}{2.798539in}}{\pgfqpoint{2.475091in}{2.802198in}}{\pgfqpoint{2.481602in}{2.808709in}}%
\pgfpathcurveto{\pgfqpoint{2.488114in}{2.815220in}}{\pgfqpoint{2.491772in}{2.824053in}}{\pgfqpoint{2.491772in}{2.833261in}}%
\pgfpathcurveto{\pgfqpoint{2.491772in}{2.842470in}}{\pgfqpoint{2.488114in}{2.851302in}}{\pgfqpoint{2.481602in}{2.857814in}}%
\pgfpathcurveto{\pgfqpoint{2.475091in}{2.864325in}}{\pgfqpoint{2.466259in}{2.867984in}}{\pgfqpoint{2.457050in}{2.867984in}}%
\pgfpathcurveto{\pgfqpoint{2.447842in}{2.867984in}}{\pgfqpoint{2.439009in}{2.864325in}}{\pgfqpoint{2.432498in}{2.857814in}}%
\pgfpathcurveto{\pgfqpoint{2.425986in}{2.851302in}}{\pgfqpoint{2.422328in}{2.842470in}}{\pgfqpoint{2.422328in}{2.833261in}}%
\pgfpathcurveto{\pgfqpoint{2.422328in}{2.824053in}}{\pgfqpoint{2.425986in}{2.815220in}}{\pgfqpoint{2.432498in}{2.808709in}}%
\pgfpathcurveto{\pgfqpoint{2.439009in}{2.802198in}}{\pgfqpoint{2.447842in}{2.798539in}}{\pgfqpoint{2.457050in}{2.798539in}}%
\pgfpathlineto{\pgfqpoint{2.457050in}{2.798539in}}%
\pgfpathclose%
\pgfusepath{stroke,fill}%
\end{pgfscope}%
\begin{pgfscope}%
\pgfpathrectangle{\pgfqpoint{1.374500in}{0.082500in}}{\pgfqpoint{2.419000in}{2.419000in}}%
\pgfusepath{clip}%
\pgfsetbuttcap%
\pgfsetroundjoin%
\definecolor{currentfill}{rgb}{0.247059,0.564706,0.854902}%
\pgfsetfillcolor{currentfill}%
\pgfsetfillopacity{0.363359}%
\pgfsetlinewidth{1.003750pt}%
\definecolor{currentstroke}{rgb}{0.247059,0.564706,0.854902}%
\pgfsetstrokecolor{currentstroke}%
\pgfsetstrokeopacity{0.363359}%
\pgfsetdash{}{0pt}%
\pgfpathmoveto{\pgfqpoint{6.268617in}{2.798539in}}%
\pgfpathcurveto{\pgfqpoint{6.277826in}{2.798539in}}{\pgfqpoint{6.286658in}{2.802198in}}{\pgfqpoint{6.293170in}{2.808709in}}%
\pgfpathcurveto{\pgfqpoint{6.299681in}{2.815220in}}{\pgfqpoint{6.303340in}{2.824053in}}{\pgfqpoint{6.303340in}{2.833261in}}%
\pgfpathcurveto{\pgfqpoint{6.303340in}{2.842470in}}{\pgfqpoint{6.299681in}{2.851302in}}{\pgfqpoint{6.293170in}{2.857814in}}%
\pgfpathcurveto{\pgfqpoint{6.286658in}{2.864325in}}{\pgfqpoint{6.277826in}{2.867984in}}{\pgfqpoint{6.268617in}{2.867984in}}%
\pgfpathcurveto{\pgfqpoint{6.259409in}{2.867984in}}{\pgfqpoint{6.250576in}{2.864325in}}{\pgfqpoint{6.244065in}{2.857814in}}%
\pgfpathcurveto{\pgfqpoint{6.237554in}{2.851302in}}{\pgfqpoint{6.233895in}{2.842470in}}{\pgfqpoint{6.233895in}{2.833261in}}%
\pgfpathcurveto{\pgfqpoint{6.233895in}{2.824053in}}{\pgfqpoint{6.237554in}{2.815220in}}{\pgfqpoint{6.244065in}{2.808709in}}%
\pgfpathcurveto{\pgfqpoint{6.250576in}{2.802198in}}{\pgfqpoint{6.259409in}{2.798539in}}{\pgfqpoint{6.268617in}{2.798539in}}%
\pgfpathlineto{\pgfqpoint{6.268617in}{2.798539in}}%
\pgfpathclose%
\pgfusepath{stroke,fill}%
\end{pgfscope}%
\begin{pgfscope}%
\pgfpathrectangle{\pgfqpoint{1.374500in}{0.082500in}}{\pgfqpoint{2.419000in}{2.419000in}}%
\pgfusepath{clip}%
\pgfsetbuttcap%
\pgfsetroundjoin%
\definecolor{currentfill}{rgb}{0.247059,0.564706,0.854902}%
\pgfsetfillcolor{currentfill}%
\pgfsetfillopacity{0.363359}%
\pgfsetlinewidth{1.003750pt}%
\definecolor{currentstroke}{rgb}{0.247059,0.564706,0.854902}%
\pgfsetstrokecolor{currentstroke}%
\pgfsetstrokeopacity{0.363359}%
\pgfsetdash{}{0pt}%
\pgfpathmoveto{\pgfqpoint{10.080185in}{2.798539in}}%
\pgfpathcurveto{\pgfqpoint{10.089393in}{2.798539in}}{\pgfqpoint{10.098226in}{2.802198in}}{\pgfqpoint{10.104737in}{2.808709in}}%
\pgfpathcurveto{\pgfqpoint{10.111248in}{2.815220in}}{\pgfqpoint{10.114907in}{2.824053in}}{\pgfqpoint{10.114907in}{2.833261in}}%
\pgfpathcurveto{\pgfqpoint{10.114907in}{2.842470in}}{\pgfqpoint{10.111248in}{2.851302in}}{\pgfqpoint{10.104737in}{2.857814in}}%
\pgfpathcurveto{\pgfqpoint{10.098226in}{2.864325in}}{\pgfqpoint{10.089393in}{2.867984in}}{\pgfqpoint{10.080185in}{2.867984in}}%
\pgfpathcurveto{\pgfqpoint{10.070976in}{2.867984in}}{\pgfqpoint{10.062144in}{2.864325in}}{\pgfqpoint{10.055632in}{2.857814in}}%
\pgfpathcurveto{\pgfqpoint{10.049121in}{2.851302in}}{\pgfqpoint{10.045462in}{2.842470in}}{\pgfqpoint{10.045462in}{2.833261in}}%
\pgfpathcurveto{\pgfqpoint{10.045462in}{2.824053in}}{\pgfqpoint{10.049121in}{2.815220in}}{\pgfqpoint{10.055632in}{2.808709in}}%
\pgfpathcurveto{\pgfqpoint{10.062144in}{2.802198in}}{\pgfqpoint{10.070976in}{2.798539in}}{\pgfqpoint{10.080185in}{2.798539in}}%
\pgfpathlineto{\pgfqpoint{10.080185in}{2.798539in}}%
\pgfpathclose%
\pgfusepath{stroke,fill}%
\end{pgfscope}%
\begin{pgfscope}%
\pgfpathrectangle{\pgfqpoint{1.374500in}{0.082500in}}{\pgfqpoint{2.419000in}{2.419000in}}%
\pgfusepath{clip}%
\pgfsetbuttcap%
\pgfsetroundjoin%
\definecolor{currentfill}{rgb}{0.247059,0.564706,0.854902}%
\pgfsetfillcolor{currentfill}%
\pgfsetfillopacity{0.367449}%
\pgfsetlinewidth{1.003750pt}%
\definecolor{currentstroke}{rgb}{0.247059,0.564706,0.854902}%
\pgfsetstrokecolor{currentstroke}%
\pgfsetstrokeopacity{0.367449}%
\pgfsetdash{}{0pt}%
\pgfpathmoveto{\pgfqpoint{0.884938in}{2.714855in}}%
\pgfpathcurveto{\pgfqpoint{0.894147in}{2.714855in}}{\pgfqpoint{0.902979in}{2.718514in}}{\pgfqpoint{0.909490in}{2.725025in}}%
\pgfpathcurveto{\pgfqpoint{0.916002in}{2.731537in}}{\pgfqpoint{0.919660in}{2.740369in}}{\pgfqpoint{0.919660in}{2.749578in}}%
\pgfpathcurveto{\pgfqpoint{0.919660in}{2.758786in}}{\pgfqpoint{0.916002in}{2.767619in}}{\pgfqpoint{0.909490in}{2.774130in}}%
\pgfpathcurveto{\pgfqpoint{0.902979in}{2.780641in}}{\pgfqpoint{0.894147in}{2.784300in}}{\pgfqpoint{0.884938in}{2.784300in}}%
\pgfpathcurveto{\pgfqpoint{0.875730in}{2.784300in}}{\pgfqpoint{0.866897in}{2.780641in}}{\pgfqpoint{0.860386in}{2.774130in}}%
\pgfpathcurveto{\pgfqpoint{0.853874in}{2.767619in}}{\pgfqpoint{0.850216in}{2.758786in}}{\pgfqpoint{0.850216in}{2.749578in}}%
\pgfpathcurveto{\pgfqpoint{0.850216in}{2.740369in}}{\pgfqpoint{0.853874in}{2.731537in}}{\pgfqpoint{0.860386in}{2.725025in}}%
\pgfpathcurveto{\pgfqpoint{0.866897in}{2.718514in}}{\pgfqpoint{0.875730in}{2.714855in}}{\pgfqpoint{0.884938in}{2.714855in}}%
\pgfpathlineto{\pgfqpoint{0.884938in}{2.714855in}}%
\pgfpathclose%
\pgfusepath{stroke,fill}%
\end{pgfscope}%
\begin{pgfscope}%
\pgfpathrectangle{\pgfqpoint{1.374500in}{0.082500in}}{\pgfqpoint{2.419000in}{2.419000in}}%
\pgfusepath{clip}%
\pgfsetbuttcap%
\pgfsetroundjoin%
\definecolor{currentfill}{rgb}{0.247059,0.564706,0.854902}%
\pgfsetfillcolor{currentfill}%
\pgfsetfillopacity{0.367449}%
\pgfsetlinewidth{1.003750pt}%
\definecolor{currentstroke}{rgb}{0.247059,0.564706,0.854902}%
\pgfsetstrokecolor{currentstroke}%
\pgfsetstrokeopacity{0.367449}%
\pgfsetdash{}{0pt}%
\pgfpathmoveto{\pgfqpoint{4.749308in}{2.714855in}}%
\pgfpathcurveto{\pgfqpoint{4.758517in}{2.714855in}}{\pgfqpoint{4.767349in}{2.718514in}}{\pgfqpoint{4.773861in}{2.725025in}}%
\pgfpathcurveto{\pgfqpoint{4.780372in}{2.731537in}}{\pgfqpoint{4.784031in}{2.740369in}}{\pgfqpoint{4.784031in}{2.749578in}}%
\pgfpathcurveto{\pgfqpoint{4.784031in}{2.758786in}}{\pgfqpoint{4.780372in}{2.767619in}}{\pgfqpoint{4.773861in}{2.774130in}}%
\pgfpathcurveto{\pgfqpoint{4.767349in}{2.780641in}}{\pgfqpoint{4.758517in}{2.784300in}}{\pgfqpoint{4.749308in}{2.784300in}}%
\pgfpathcurveto{\pgfqpoint{4.740100in}{2.784300in}}{\pgfqpoint{4.731268in}{2.780641in}}{\pgfqpoint{4.724756in}{2.774130in}}%
\pgfpathcurveto{\pgfqpoint{4.718245in}{2.767619in}}{\pgfqpoint{4.714586in}{2.758786in}}{\pgfqpoint{4.714586in}{2.749578in}}%
\pgfpathcurveto{\pgfqpoint{4.714586in}{2.740369in}}{\pgfqpoint{4.718245in}{2.731537in}}{\pgfqpoint{4.724756in}{2.725025in}}%
\pgfpathcurveto{\pgfqpoint{4.731268in}{2.718514in}}{\pgfqpoint{4.740100in}{2.714855in}}{\pgfqpoint{4.749308in}{2.714855in}}%
\pgfpathlineto{\pgfqpoint{4.749308in}{2.714855in}}%
\pgfpathclose%
\pgfusepath{stroke,fill}%
\end{pgfscope}%
\begin{pgfscope}%
\pgfpathrectangle{\pgfqpoint{1.374500in}{0.082500in}}{\pgfqpoint{2.419000in}{2.419000in}}%
\pgfusepath{clip}%
\pgfsetbuttcap%
\pgfsetroundjoin%
\definecolor{currentfill}{rgb}{0.247059,0.564706,0.854902}%
\pgfsetfillcolor{currentfill}%
\pgfsetfillopacity{0.367449}%
\pgfsetlinewidth{1.003750pt}%
\definecolor{currentstroke}{rgb}{0.247059,0.564706,0.854902}%
\pgfsetstrokecolor{currentstroke}%
\pgfsetstrokeopacity{0.367449}%
\pgfsetdash{}{0pt}%
\pgfpathmoveto{\pgfqpoint{8.613679in}{2.714855in}}%
\pgfpathcurveto{\pgfqpoint{8.622887in}{2.714855in}}{\pgfqpoint{8.631720in}{2.718514in}}{\pgfqpoint{8.638231in}{2.725025in}}%
\pgfpathcurveto{\pgfqpoint{8.644743in}{2.731537in}}{\pgfqpoint{8.648401in}{2.740369in}}{\pgfqpoint{8.648401in}{2.749578in}}%
\pgfpathcurveto{\pgfqpoint{8.648401in}{2.758786in}}{\pgfqpoint{8.644743in}{2.767619in}}{\pgfqpoint{8.638231in}{2.774130in}}%
\pgfpathcurveto{\pgfqpoint{8.631720in}{2.780641in}}{\pgfqpoint{8.622887in}{2.784300in}}{\pgfqpoint{8.613679in}{2.784300in}}%
\pgfpathcurveto{\pgfqpoint{8.604470in}{2.784300in}}{\pgfqpoint{8.595638in}{2.780641in}}{\pgfqpoint{8.589127in}{2.774130in}}%
\pgfpathcurveto{\pgfqpoint{8.582615in}{2.767619in}}{\pgfqpoint{8.578957in}{2.758786in}}{\pgfqpoint{8.578957in}{2.749578in}}%
\pgfpathcurveto{\pgfqpoint{8.578957in}{2.740369in}}{\pgfqpoint{8.582615in}{2.731537in}}{\pgfqpoint{8.589127in}{2.725025in}}%
\pgfpathcurveto{\pgfqpoint{8.595638in}{2.718514in}}{\pgfqpoint{8.604470in}{2.714855in}}{\pgfqpoint{8.613679in}{2.714855in}}%
\pgfpathlineto{\pgfqpoint{8.613679in}{2.714855in}}%
\pgfpathclose%
\pgfusepath{stroke,fill}%
\end{pgfscope}%
\begin{pgfscope}%
\pgfpathrectangle{\pgfqpoint{1.374500in}{0.082500in}}{\pgfqpoint{2.419000in}{2.419000in}}%
\pgfusepath{clip}%
\pgfsetbuttcap%
\pgfsetroundjoin%
\definecolor{currentfill}{rgb}{0.247059,0.564706,0.854902}%
\pgfsetfillcolor{currentfill}%
\pgfsetfillopacity{0.371653}%
\pgfsetlinewidth{1.003750pt}%
\definecolor{currentstroke}{rgb}{0.247059,0.564706,0.854902}%
\pgfsetstrokecolor{currentstroke}%
\pgfsetstrokeopacity{0.371653}%
\pgfsetdash{}{0pt}%
\pgfpathmoveto{\pgfqpoint{3.187313in}{2.628820in}}%
\pgfpathcurveto{\pgfqpoint{3.196522in}{2.628820in}}{\pgfqpoint{3.205354in}{2.632479in}}{\pgfqpoint{3.211865in}{2.638990in}}%
\pgfpathcurveto{\pgfqpoint{3.218377in}{2.645502in}}{\pgfqpoint{3.222035in}{2.654334in}}{\pgfqpoint{3.222035in}{2.663543in}}%
\pgfpathcurveto{\pgfqpoint{3.222035in}{2.672751in}}{\pgfqpoint{3.218377in}{2.681584in}}{\pgfqpoint{3.211865in}{2.688095in}}%
\pgfpathcurveto{\pgfqpoint{3.205354in}{2.694606in}}{\pgfqpoint{3.196522in}{2.698265in}}{\pgfqpoint{3.187313in}{2.698265in}}%
\pgfpathcurveto{\pgfqpoint{3.178105in}{2.698265in}}{\pgfqpoint{3.169272in}{2.694606in}}{\pgfqpoint{3.162761in}{2.688095in}}%
\pgfpathcurveto{\pgfqpoint{3.156249in}{2.681584in}}{\pgfqpoint{3.152591in}{2.672751in}}{\pgfqpoint{3.152591in}{2.663543in}}%
\pgfpathcurveto{\pgfqpoint{3.152591in}{2.654334in}}{\pgfqpoint{3.156249in}{2.645502in}}{\pgfqpoint{3.162761in}{2.638990in}}%
\pgfpathcurveto{\pgfqpoint{3.169272in}{2.632479in}}{\pgfqpoint{3.178105in}{2.628820in}}{\pgfqpoint{3.187313in}{2.628820in}}%
\pgfpathlineto{\pgfqpoint{3.187313in}{2.628820in}}%
\pgfpathclose%
\pgfusepath{stroke,fill}%
\end{pgfscope}%
\begin{pgfscope}%
\pgfpathrectangle{\pgfqpoint{1.374500in}{0.082500in}}{\pgfqpoint{2.419000in}{2.419000in}}%
\pgfusepath{clip}%
\pgfsetbuttcap%
\pgfsetroundjoin%
\definecolor{currentfill}{rgb}{0.247059,0.564706,0.854902}%
\pgfsetfillcolor{currentfill}%
\pgfsetfillopacity{0.371653}%
\pgfsetlinewidth{1.003750pt}%
\definecolor{currentstroke}{rgb}{0.247059,0.564706,0.854902}%
\pgfsetstrokecolor{currentstroke}%
\pgfsetstrokeopacity{0.371653}%
\pgfsetdash{}{0pt}%
\pgfpathmoveto{\pgfqpoint{7.105970in}{2.628820in}}%
\pgfpathcurveto{\pgfqpoint{7.115179in}{2.628820in}}{\pgfqpoint{7.124011in}{2.632479in}}{\pgfqpoint{7.130522in}{2.638990in}}%
\pgfpathcurveto{\pgfqpoint{7.137034in}{2.645502in}}{\pgfqpoint{7.140692in}{2.654334in}}{\pgfqpoint{7.140692in}{2.663543in}}%
\pgfpathcurveto{\pgfqpoint{7.140692in}{2.672751in}}{\pgfqpoint{7.137034in}{2.681584in}}{\pgfqpoint{7.130522in}{2.688095in}}%
\pgfpathcurveto{\pgfqpoint{7.124011in}{2.694606in}}{\pgfqpoint{7.115179in}{2.698265in}}{\pgfqpoint{7.105970in}{2.698265in}}%
\pgfpathcurveto{\pgfqpoint{7.096762in}{2.698265in}}{\pgfqpoint{7.087929in}{2.694606in}}{\pgfqpoint{7.081418in}{2.688095in}}%
\pgfpathcurveto{\pgfqpoint{7.074906in}{2.681584in}}{\pgfqpoint{7.071248in}{2.672751in}}{\pgfqpoint{7.071248in}{2.663543in}}%
\pgfpathcurveto{\pgfqpoint{7.071248in}{2.654334in}}{\pgfqpoint{7.074906in}{2.645502in}}{\pgfqpoint{7.081418in}{2.638990in}}%
\pgfpathcurveto{\pgfqpoint{7.087929in}{2.632479in}}{\pgfqpoint{7.096762in}{2.628820in}}{\pgfqpoint{7.105970in}{2.628820in}}%
\pgfpathlineto{\pgfqpoint{7.105970in}{2.628820in}}%
\pgfpathclose%
\pgfusepath{stroke,fill}%
\end{pgfscope}%
\begin{pgfscope}%
\pgfpathrectangle{\pgfqpoint{1.374500in}{0.082500in}}{\pgfqpoint{2.419000in}{2.419000in}}%
\pgfusepath{clip}%
\pgfsetbuttcap%
\pgfsetroundjoin%
\definecolor{currentfill}{rgb}{0.247059,0.564706,0.854902}%
\pgfsetfillcolor{currentfill}%
\pgfsetfillopacity{0.371653}%
\pgfsetlinewidth{1.003750pt}%
\definecolor{currentstroke}{rgb}{0.247059,0.564706,0.854902}%
\pgfsetstrokecolor{currentstroke}%
\pgfsetstrokeopacity{0.371653}%
\pgfsetdash{}{0pt}%
\pgfpathmoveto{\pgfqpoint{11.024627in}{2.628820in}}%
\pgfpathcurveto{\pgfqpoint{11.033836in}{2.628820in}}{\pgfqpoint{11.042668in}{2.632479in}}{\pgfqpoint{11.049179in}{2.638990in}}%
\pgfpathcurveto{\pgfqpoint{11.055691in}{2.645502in}}{\pgfqpoint{11.059349in}{2.654334in}}{\pgfqpoint{11.059349in}{2.663543in}}%
\pgfpathcurveto{\pgfqpoint{11.059349in}{2.672751in}}{\pgfqpoint{11.055691in}{2.681584in}}{\pgfqpoint{11.049179in}{2.688095in}}%
\pgfpathcurveto{\pgfqpoint{11.042668in}{2.694606in}}{\pgfqpoint{11.033836in}{2.698265in}}{\pgfqpoint{11.024627in}{2.698265in}}%
\pgfpathcurveto{\pgfqpoint{11.015419in}{2.698265in}}{\pgfqpoint{11.006586in}{2.694606in}}{\pgfqpoint{11.000075in}{2.688095in}}%
\pgfpathcurveto{\pgfqpoint{10.993563in}{2.681584in}}{\pgfqpoint{10.989905in}{2.672751in}}{\pgfqpoint{10.989905in}{2.663543in}}%
\pgfpathcurveto{\pgfqpoint{10.989905in}{2.654334in}}{\pgfqpoint{10.993563in}{2.645502in}}{\pgfqpoint{11.000075in}{2.638990in}}%
\pgfpathcurveto{\pgfqpoint{11.006586in}{2.632479in}}{\pgfqpoint{11.015419in}{2.628820in}}{\pgfqpoint{11.024627in}{2.628820in}}%
\pgfpathlineto{\pgfqpoint{11.024627in}{2.628820in}}%
\pgfpathclose%
\pgfusepath{stroke,fill}%
\end{pgfscope}%
\begin{pgfscope}%
\pgfpathrectangle{\pgfqpoint{1.374500in}{0.082500in}}{\pgfqpoint{2.419000in}{2.419000in}}%
\pgfusepath{clip}%
\pgfsetbuttcap%
\pgfsetroundjoin%
\definecolor{currentfill}{rgb}{0.247059,0.564706,0.854902}%
\pgfsetfillcolor{currentfill}%
\pgfsetfillopacity{0.375977}%
\pgfsetlinewidth{1.003750pt}%
\definecolor{currentstroke}{rgb}{0.247059,0.564706,0.854902}%
\pgfsetstrokecolor{currentstroke}%
\pgfsetstrokeopacity{0.375977}%
\pgfsetdash{}{0pt}%
\pgfpathmoveto{\pgfqpoint{5.555297in}{2.540334in}}%
\pgfpathcurveto{\pgfqpoint{5.564506in}{2.540334in}}{\pgfqpoint{5.573338in}{2.543992in}}{\pgfqpoint{5.579850in}{2.550504in}}%
\pgfpathcurveto{\pgfqpoint{5.586361in}{2.557015in}}{\pgfqpoint{5.590019in}{2.565848in}}{\pgfqpoint{5.590019in}{2.575056in}}%
\pgfpathcurveto{\pgfqpoint{5.590019in}{2.584264in}}{\pgfqpoint{5.586361in}{2.593097in}}{\pgfqpoint{5.579850in}{2.599608in}}%
\pgfpathcurveto{\pgfqpoint{5.573338in}{2.606120in}}{\pgfqpoint{5.564506in}{2.609778in}}{\pgfqpoint{5.555297in}{2.609778in}}%
\pgfpathcurveto{\pgfqpoint{5.546089in}{2.609778in}}{\pgfqpoint{5.537256in}{2.606120in}}{\pgfqpoint{5.530745in}{2.599608in}}%
\pgfpathcurveto{\pgfqpoint{5.524234in}{2.593097in}}{\pgfqpoint{5.520575in}{2.584264in}}{\pgfqpoint{5.520575in}{2.575056in}}%
\pgfpathcurveto{\pgfqpoint{5.520575in}{2.565848in}}{\pgfqpoint{5.524234in}{2.557015in}}{\pgfqpoint{5.530745in}{2.550504in}}%
\pgfpathcurveto{\pgfqpoint{5.537256in}{2.543992in}}{\pgfqpoint{5.546089in}{2.540334in}}{\pgfqpoint{5.555297in}{2.540334in}}%
\pgfpathlineto{\pgfqpoint{5.555297in}{2.540334in}}%
\pgfpathclose%
\pgfusepath{stroke,fill}%
\end{pgfscope}%
\begin{pgfscope}%
\pgfpathrectangle{\pgfqpoint{1.374500in}{0.082500in}}{\pgfqpoint{2.419000in}{2.419000in}}%
\pgfusepath{clip}%
\pgfsetbuttcap%
\pgfsetroundjoin%
\definecolor{currentfill}{rgb}{0.247059,0.564706,0.854902}%
\pgfsetfillcolor{currentfill}%
\pgfsetfillopacity{0.375977}%
\pgfsetlinewidth{1.003750pt}%
\definecolor{currentstroke}{rgb}{0.247059,0.564706,0.854902}%
\pgfsetstrokecolor{currentstroke}%
\pgfsetstrokeopacity{0.375977}%
\pgfsetdash{}{0pt}%
\pgfpathmoveto{\pgfqpoint{1.580807in}{2.540334in}}%
\pgfpathcurveto{\pgfqpoint{1.590015in}{2.540334in}}{\pgfqpoint{1.598848in}{2.543992in}}{\pgfqpoint{1.605359in}{2.550504in}}%
\pgfpathcurveto{\pgfqpoint{1.611870in}{2.557015in}}{\pgfqpoint{1.615529in}{2.565848in}}{\pgfqpoint{1.615529in}{2.575056in}}%
\pgfpathcurveto{\pgfqpoint{1.615529in}{2.584264in}}{\pgfqpoint{1.611870in}{2.593097in}}{\pgfqpoint{1.605359in}{2.599608in}}%
\pgfpathcurveto{\pgfqpoint{1.598848in}{2.606120in}}{\pgfqpoint{1.590015in}{2.609778in}}{\pgfqpoint{1.580807in}{2.609778in}}%
\pgfpathcurveto{\pgfqpoint{1.571598in}{2.609778in}}{\pgfqpoint{1.562766in}{2.606120in}}{\pgfqpoint{1.556254in}{2.599608in}}%
\pgfpathcurveto{\pgfqpoint{1.549743in}{2.593097in}}{\pgfqpoint{1.546084in}{2.584264in}}{\pgfqpoint{1.546084in}{2.575056in}}%
\pgfpathcurveto{\pgfqpoint{1.546084in}{2.565848in}}{\pgfqpoint{1.549743in}{2.557015in}}{\pgfqpoint{1.556254in}{2.550504in}}%
\pgfpathcurveto{\pgfqpoint{1.562766in}{2.543992in}}{\pgfqpoint{1.571598in}{2.540334in}}{\pgfqpoint{1.580807in}{2.540334in}}%
\pgfpathlineto{\pgfqpoint{1.580807in}{2.540334in}}%
\pgfpathclose%
\pgfusepath{stroke,fill}%
\end{pgfscope}%
\begin{pgfscope}%
\pgfpathrectangle{\pgfqpoint{1.374500in}{0.082500in}}{\pgfqpoint{2.419000in}{2.419000in}}%
\pgfusepath{clip}%
\pgfsetbuttcap%
\pgfsetroundjoin%
\definecolor{currentfill}{rgb}{0.247059,0.564706,0.854902}%
\pgfsetfillcolor{currentfill}%
\pgfsetfillopacity{0.375977}%
\pgfsetlinewidth{1.003750pt}%
\definecolor{currentstroke}{rgb}{0.247059,0.564706,0.854902}%
\pgfsetstrokecolor{currentstroke}%
\pgfsetstrokeopacity{0.375977}%
\pgfsetdash{}{0pt}%
\pgfpathmoveto{\pgfqpoint{9.529788in}{2.540334in}}%
\pgfpathcurveto{\pgfqpoint{9.538996in}{2.540334in}}{\pgfqpoint{9.547829in}{2.543992in}}{\pgfqpoint{9.554340in}{2.550504in}}%
\pgfpathcurveto{\pgfqpoint{9.560852in}{2.557015in}}{\pgfqpoint{9.564510in}{2.565848in}}{\pgfqpoint{9.564510in}{2.575056in}}%
\pgfpathcurveto{\pgfqpoint{9.564510in}{2.584264in}}{\pgfqpoint{9.560852in}{2.593097in}}{\pgfqpoint{9.554340in}{2.599608in}}%
\pgfpathcurveto{\pgfqpoint{9.547829in}{2.606120in}}{\pgfqpoint{9.538996in}{2.609778in}}{\pgfqpoint{9.529788in}{2.609778in}}%
\pgfpathcurveto{\pgfqpoint{9.520579in}{2.609778in}}{\pgfqpoint{9.511747in}{2.606120in}}{\pgfqpoint{9.505236in}{2.599608in}}%
\pgfpathcurveto{\pgfqpoint{9.498724in}{2.593097in}}{\pgfqpoint{9.495066in}{2.584264in}}{\pgfqpoint{9.495066in}{2.575056in}}%
\pgfpathcurveto{\pgfqpoint{9.495066in}{2.565848in}}{\pgfqpoint{9.498724in}{2.557015in}}{\pgfqpoint{9.505236in}{2.550504in}}%
\pgfpathcurveto{\pgfqpoint{9.511747in}{2.543992in}}{\pgfqpoint{9.520579in}{2.540334in}}{\pgfqpoint{9.529788in}{2.540334in}}%
\pgfpathlineto{\pgfqpoint{9.529788in}{2.540334in}}%
\pgfpathclose%
\pgfusepath{stroke,fill}%
\end{pgfscope}%
\begin{pgfscope}%
\pgfpathrectangle{\pgfqpoint{1.374500in}{0.082500in}}{\pgfqpoint{2.419000in}{2.419000in}}%
\pgfusepath{clip}%
\pgfsetbuttcap%
\pgfsetroundjoin%
\definecolor{currentfill}{rgb}{0.247059,0.564706,0.854902}%
\pgfsetfillcolor{currentfill}%
\pgfsetfillopacity{0.380427}%
\pgfsetlinewidth{1.003750pt}%
\definecolor{currentstroke}{rgb}{0.247059,0.564706,0.854902}%
\pgfsetstrokecolor{currentstroke}%
\pgfsetstrokeopacity{0.380427}%
\pgfsetdash{}{0pt}%
\pgfpathmoveto{\pgfqpoint{3.959797in}{2.449289in}}%
\pgfpathcurveto{\pgfqpoint{3.969006in}{2.449289in}}{\pgfqpoint{3.977838in}{2.452948in}}{\pgfqpoint{3.984350in}{2.459459in}}%
\pgfpathcurveto{\pgfqpoint{3.990861in}{2.465970in}}{\pgfqpoint{3.994519in}{2.474803in}}{\pgfqpoint{3.994519in}{2.484011in}}%
\pgfpathcurveto{\pgfqpoint{3.994519in}{2.493220in}}{\pgfqpoint{3.990861in}{2.502052in}}{\pgfqpoint{3.984350in}{2.508564in}}%
\pgfpathcurveto{\pgfqpoint{3.977838in}{2.515075in}}{\pgfqpoint{3.969006in}{2.518734in}}{\pgfqpoint{3.959797in}{2.518734in}}%
\pgfpathcurveto{\pgfqpoint{3.950589in}{2.518734in}}{\pgfqpoint{3.941756in}{2.515075in}}{\pgfqpoint{3.935245in}{2.508564in}}%
\pgfpathcurveto{\pgfqpoint{3.928734in}{2.502052in}}{\pgfqpoint{3.925075in}{2.493220in}}{\pgfqpoint{3.925075in}{2.484011in}}%
\pgfpathcurveto{\pgfqpoint{3.925075in}{2.474803in}}{\pgfqpoint{3.928734in}{2.465970in}}{\pgfqpoint{3.935245in}{2.459459in}}%
\pgfpathcurveto{\pgfqpoint{3.941756in}{2.452948in}}{\pgfqpoint{3.950589in}{2.449289in}}{\pgfqpoint{3.959797in}{2.449289in}}%
\pgfpathlineto{\pgfqpoint{3.959797in}{2.449289in}}%
\pgfpathclose%
\pgfusepath{stroke,fill}%
\end{pgfscope}%
\begin{pgfscope}%
\pgfpathrectangle{\pgfqpoint{1.374500in}{0.082500in}}{\pgfqpoint{2.419000in}{2.419000in}}%
\pgfusepath{clip}%
\pgfsetbuttcap%
\pgfsetroundjoin%
\definecolor{currentfill}{rgb}{0.247059,0.564706,0.854902}%
\pgfsetfillcolor{currentfill}%
\pgfsetfillopacity{0.380427}%
\pgfsetlinewidth{1.003750pt}%
\definecolor{currentstroke}{rgb}{0.247059,0.564706,0.854902}%
\pgfsetstrokecolor{currentstroke}%
\pgfsetstrokeopacity{0.380427}%
\pgfsetdash{}{0pt}%
\pgfpathmoveto{\pgfqpoint{-0.072141in}{2.449289in}}%
\pgfpathcurveto{\pgfqpoint{-0.062933in}{2.449289in}}{\pgfqpoint{-0.054100in}{2.452948in}}{\pgfqpoint{-0.047589in}{2.459459in}}%
\pgfpathcurveto{\pgfqpoint{-0.041077in}{2.465970in}}{\pgfqpoint{-0.037419in}{2.474803in}}{\pgfqpoint{-0.037419in}{2.484011in}}%
\pgfpathcurveto{\pgfqpoint{-0.037419in}{2.493220in}}{\pgfqpoint{-0.041077in}{2.502052in}}{\pgfqpoint{-0.047589in}{2.508564in}}%
\pgfpathcurveto{\pgfqpoint{-0.054100in}{2.515075in}}{\pgfqpoint{-0.062933in}{2.518734in}}{\pgfqpoint{-0.072141in}{2.518734in}}%
\pgfpathcurveto{\pgfqpoint{-0.081350in}{2.518734in}}{\pgfqpoint{-0.090182in}{2.515075in}}{\pgfqpoint{-0.096693in}{2.508564in}}%
\pgfpathcurveto{\pgfqpoint{-0.103205in}{2.502052in}}{\pgfqpoint{-0.106863in}{2.493220in}}{\pgfqpoint{-0.106863in}{2.484011in}}%
\pgfpathcurveto{\pgfqpoint{-0.106863in}{2.474803in}}{\pgfqpoint{-0.103205in}{2.465970in}}{\pgfqpoint{-0.096693in}{2.459459in}}%
\pgfpathcurveto{\pgfqpoint{-0.090182in}{2.452948in}}{\pgfqpoint{-0.081350in}{2.449289in}}{\pgfqpoint{-0.072141in}{2.449289in}}%
\pgfpathlineto{\pgfqpoint{-0.072141in}{2.449289in}}%
\pgfpathclose%
\pgfusepath{stroke,fill}%
\end{pgfscope}%
\begin{pgfscope}%
\pgfpathrectangle{\pgfqpoint{1.374500in}{0.082500in}}{\pgfqpoint{2.419000in}{2.419000in}}%
\pgfusepath{clip}%
\pgfsetbuttcap%
\pgfsetroundjoin%
\definecolor{currentfill}{rgb}{0.247059,0.564706,0.854902}%
\pgfsetfillcolor{currentfill}%
\pgfsetfillopacity{0.380427}%
\pgfsetlinewidth{1.003750pt}%
\definecolor{currentstroke}{rgb}{0.247059,0.564706,0.854902}%
\pgfsetstrokecolor{currentstroke}%
\pgfsetstrokeopacity{0.380427}%
\pgfsetdash{}{0pt}%
\pgfpathmoveto{\pgfqpoint{7.991736in}{2.449289in}}%
\pgfpathcurveto{\pgfqpoint{8.000944in}{2.449289in}}{\pgfqpoint{8.009777in}{2.452948in}}{\pgfqpoint{8.016288in}{2.459459in}}%
\pgfpathcurveto{\pgfqpoint{8.022799in}{2.465970in}}{\pgfqpoint{8.026458in}{2.474803in}}{\pgfqpoint{8.026458in}{2.484011in}}%
\pgfpathcurveto{\pgfqpoint{8.026458in}{2.493220in}}{\pgfqpoint{8.022799in}{2.502052in}}{\pgfqpoint{8.016288in}{2.508564in}}%
\pgfpathcurveto{\pgfqpoint{8.009777in}{2.515075in}}{\pgfqpoint{8.000944in}{2.518734in}}{\pgfqpoint{7.991736in}{2.518734in}}%
\pgfpathcurveto{\pgfqpoint{7.982527in}{2.518734in}}{\pgfqpoint{7.973695in}{2.515075in}}{\pgfqpoint{7.967183in}{2.508564in}}%
\pgfpathcurveto{\pgfqpoint{7.960672in}{2.502052in}}{\pgfqpoint{7.957013in}{2.493220in}}{\pgfqpoint{7.957013in}{2.484011in}}%
\pgfpathcurveto{\pgfqpoint{7.957013in}{2.474803in}}{\pgfqpoint{7.960672in}{2.465970in}}{\pgfqpoint{7.967183in}{2.459459in}}%
\pgfpathcurveto{\pgfqpoint{7.973695in}{2.452948in}}{\pgfqpoint{7.982527in}{2.449289in}}{\pgfqpoint{7.991736in}{2.449289in}}%
\pgfpathlineto{\pgfqpoint{7.991736in}{2.449289in}}%
\pgfpathclose%
\pgfusepath{stroke,fill}%
\end{pgfscope}%
\begin{pgfscope}%
\pgfpathrectangle{\pgfqpoint{1.374500in}{0.082500in}}{\pgfqpoint{2.419000in}{2.419000in}}%
\pgfusepath{clip}%
\pgfsetbuttcap%
\pgfsetroundjoin%
\definecolor{currentfill}{rgb}{0.247059,0.564706,0.854902}%
\pgfsetfillcolor{currentfill}%
\pgfsetfillopacity{0.385007}%
\pgfsetlinewidth{1.003750pt}%
\definecolor{currentstroke}{rgb}{0.247059,0.564706,0.854902}%
\pgfsetstrokecolor{currentstroke}%
\pgfsetstrokeopacity{0.385007}%
\pgfsetdash{}{0pt}%
\pgfpathmoveto{\pgfqpoint{2.317498in}{2.355574in}}%
\pgfpathcurveto{\pgfqpoint{2.326706in}{2.355574in}}{\pgfqpoint{2.335539in}{2.359233in}}{\pgfqpoint{2.342050in}{2.365744in}}%
\pgfpathcurveto{\pgfqpoint{2.348561in}{2.372255in}}{\pgfqpoint{2.352220in}{2.381088in}}{\pgfqpoint{2.352220in}{2.390296in}}%
\pgfpathcurveto{\pgfqpoint{2.352220in}{2.399505in}}{\pgfqpoint{2.348561in}{2.408337in}}{\pgfqpoint{2.342050in}{2.414849in}}%
\pgfpathcurveto{\pgfqpoint{2.335539in}{2.421360in}}{\pgfqpoint{2.326706in}{2.425018in}}{\pgfqpoint{2.317498in}{2.425018in}}%
\pgfpathcurveto{\pgfqpoint{2.308289in}{2.425018in}}{\pgfqpoint{2.299457in}{2.421360in}}{\pgfqpoint{2.292945in}{2.414849in}}%
\pgfpathcurveto{\pgfqpoint{2.286434in}{2.408337in}}{\pgfqpoint{2.282776in}{2.399505in}}{\pgfqpoint{2.282776in}{2.390296in}}%
\pgfpathcurveto{\pgfqpoint{2.282776in}{2.381088in}}{\pgfqpoint{2.286434in}{2.372255in}}{\pgfqpoint{2.292945in}{2.365744in}}%
\pgfpathcurveto{\pgfqpoint{2.299457in}{2.359233in}}{\pgfqpoint{2.308289in}{2.355574in}}{\pgfqpoint{2.317498in}{2.355574in}}%
\pgfpathlineto{\pgfqpoint{2.317498in}{2.355574in}}%
\pgfpathclose%
\pgfusepath{stroke,fill}%
\end{pgfscope}%
\begin{pgfscope}%
\pgfpathrectangle{\pgfqpoint{1.374500in}{0.082500in}}{\pgfqpoint{2.419000in}{2.419000in}}%
\pgfusepath{clip}%
\pgfsetbuttcap%
\pgfsetroundjoin%
\definecolor{currentfill}{rgb}{0.247059,0.564706,0.854902}%
\pgfsetfillcolor{currentfill}%
\pgfsetfillopacity{0.385007}%
\pgfsetlinewidth{1.003750pt}%
\definecolor{currentstroke}{rgb}{0.247059,0.564706,0.854902}%
\pgfsetstrokecolor{currentstroke}%
\pgfsetstrokeopacity{0.385007}%
\pgfsetdash{}{0pt}%
\pgfpathmoveto{\pgfqpoint{10.499640in}{2.355574in}}%
\pgfpathcurveto{\pgfqpoint{10.508848in}{2.355574in}}{\pgfqpoint{10.517681in}{2.359233in}}{\pgfqpoint{10.524192in}{2.365744in}}%
\pgfpathcurveto{\pgfqpoint{10.530704in}{2.372255in}}{\pgfqpoint{10.534362in}{2.381088in}}{\pgfqpoint{10.534362in}{2.390296in}}%
\pgfpathcurveto{\pgfqpoint{10.534362in}{2.399505in}}{\pgfqpoint{10.530704in}{2.408337in}}{\pgfqpoint{10.524192in}{2.414849in}}%
\pgfpathcurveto{\pgfqpoint{10.517681in}{2.421360in}}{\pgfqpoint{10.508848in}{2.425018in}}{\pgfqpoint{10.499640in}{2.425018in}}%
\pgfpathcurveto{\pgfqpoint{10.490431in}{2.425018in}}{\pgfqpoint{10.481599in}{2.421360in}}{\pgfqpoint{10.475088in}{2.414849in}}%
\pgfpathcurveto{\pgfqpoint{10.468576in}{2.408337in}}{\pgfqpoint{10.464918in}{2.399505in}}{\pgfqpoint{10.464918in}{2.390296in}}%
\pgfpathcurveto{\pgfqpoint{10.464918in}{2.381088in}}{\pgfqpoint{10.468576in}{2.372255in}}{\pgfqpoint{10.475088in}{2.365744in}}%
\pgfpathcurveto{\pgfqpoint{10.481599in}{2.359233in}}{\pgfqpoint{10.490431in}{2.355574in}}{\pgfqpoint{10.499640in}{2.355574in}}%
\pgfpathlineto{\pgfqpoint{10.499640in}{2.355574in}}%
\pgfpathclose%
\pgfusepath{stroke,fill}%
\end{pgfscope}%
\begin{pgfscope}%
\pgfpathrectangle{\pgfqpoint{1.374500in}{0.082500in}}{\pgfqpoint{2.419000in}{2.419000in}}%
\pgfusepath{clip}%
\pgfsetbuttcap%
\pgfsetroundjoin%
\definecolor{currentfill}{rgb}{0.247059,0.564706,0.854902}%
\pgfsetfillcolor{currentfill}%
\pgfsetfillopacity{0.385007}%
\pgfsetlinewidth{1.003750pt}%
\definecolor{currentstroke}{rgb}{0.247059,0.564706,0.854902}%
\pgfsetstrokecolor{currentstroke}%
\pgfsetstrokeopacity{0.385007}%
\pgfsetdash{}{0pt}%
\pgfpathmoveto{\pgfqpoint{6.408569in}{2.355574in}}%
\pgfpathcurveto{\pgfqpoint{6.417777in}{2.355574in}}{\pgfqpoint{6.426610in}{2.359233in}}{\pgfqpoint{6.433121in}{2.365744in}}%
\pgfpathcurveto{\pgfqpoint{6.439633in}{2.372255in}}{\pgfqpoint{6.443291in}{2.381088in}}{\pgfqpoint{6.443291in}{2.390296in}}%
\pgfpathcurveto{\pgfqpoint{6.443291in}{2.399505in}}{\pgfqpoint{6.439633in}{2.408337in}}{\pgfqpoint{6.433121in}{2.414849in}}%
\pgfpathcurveto{\pgfqpoint{6.426610in}{2.421360in}}{\pgfqpoint{6.417777in}{2.425018in}}{\pgfqpoint{6.408569in}{2.425018in}}%
\pgfpathcurveto{\pgfqpoint{6.399360in}{2.425018in}}{\pgfqpoint{6.390528in}{2.421360in}}{\pgfqpoint{6.384017in}{2.414849in}}%
\pgfpathcurveto{\pgfqpoint{6.377505in}{2.408337in}}{\pgfqpoint{6.373847in}{2.399505in}}{\pgfqpoint{6.373847in}{2.390296in}}%
\pgfpathcurveto{\pgfqpoint{6.373847in}{2.381088in}}{\pgfqpoint{6.377505in}{2.372255in}}{\pgfqpoint{6.384017in}{2.365744in}}%
\pgfpathcurveto{\pgfqpoint{6.390528in}{2.359233in}}{\pgfqpoint{6.399360in}{2.355574in}}{\pgfqpoint{6.408569in}{2.355574in}}%
\pgfpathlineto{\pgfqpoint{6.408569in}{2.355574in}}%
\pgfpathclose%
\pgfusepath{stroke,fill}%
\end{pgfscope}%
\begin{pgfscope}%
\pgfpathrectangle{\pgfqpoint{1.374500in}{0.082500in}}{\pgfqpoint{2.419000in}{2.419000in}}%
\pgfusepath{clip}%
\pgfsetbuttcap%
\pgfsetroundjoin%
\definecolor{currentfill}{rgb}{0.247059,0.564706,0.854902}%
\pgfsetfillcolor{currentfill}%
\pgfsetfillopacity{0.389723}%
\pgfsetlinewidth{1.003750pt}%
\definecolor{currentstroke}{rgb}{0.247059,0.564706,0.854902}%
\pgfsetstrokecolor{currentstroke}%
\pgfsetstrokeopacity{0.389723}%
\pgfsetdash{}{0pt}%
\pgfpathmoveto{\pgfqpoint{4.778273in}{2.259069in}}%
\pgfpathcurveto{\pgfqpoint{4.787482in}{2.259069in}}{\pgfqpoint{4.796314in}{2.262728in}}{\pgfqpoint{4.802826in}{2.269239in}}%
\pgfpathcurveto{\pgfqpoint{4.809337in}{2.275750in}}{\pgfqpoint{4.812995in}{2.284583in}}{\pgfqpoint{4.812995in}{2.293791in}}%
\pgfpathcurveto{\pgfqpoint{4.812995in}{2.303000in}}{\pgfqpoint{4.809337in}{2.311832in}}{\pgfqpoint{4.802826in}{2.318344in}}%
\pgfpathcurveto{\pgfqpoint{4.796314in}{2.324855in}}{\pgfqpoint{4.787482in}{2.328514in}}{\pgfqpoint{4.778273in}{2.328514in}}%
\pgfpathcurveto{\pgfqpoint{4.769065in}{2.328514in}}{\pgfqpoint{4.760232in}{2.324855in}}{\pgfqpoint{4.753721in}{2.318344in}}%
\pgfpathcurveto{\pgfqpoint{4.747210in}{2.311832in}}{\pgfqpoint{4.743551in}{2.303000in}}{\pgfqpoint{4.743551in}{2.293791in}}%
\pgfpathcurveto{\pgfqpoint{4.743551in}{2.284583in}}{\pgfqpoint{4.747210in}{2.275750in}}{\pgfqpoint{4.753721in}{2.269239in}}%
\pgfpathcurveto{\pgfqpoint{4.760232in}{2.262728in}}{\pgfqpoint{4.769065in}{2.259069in}}{\pgfqpoint{4.778273in}{2.259069in}}%
\pgfpathlineto{\pgfqpoint{4.778273in}{2.259069in}}%
\pgfpathclose%
\pgfusepath{stroke,fill}%
\end{pgfscope}%
\begin{pgfscope}%
\pgfpathrectangle{\pgfqpoint{1.374500in}{0.082500in}}{\pgfqpoint{2.419000in}{2.419000in}}%
\pgfusepath{clip}%
\pgfsetbuttcap%
\pgfsetroundjoin%
\definecolor{currentfill}{rgb}{0.247059,0.564706,0.854902}%
\pgfsetfillcolor{currentfill}%
\pgfsetfillopacity{0.389723}%
\pgfsetlinewidth{1.003750pt}%
\definecolor{currentstroke}{rgb}{0.247059,0.564706,0.854902}%
\pgfsetstrokecolor{currentstroke}%
\pgfsetstrokeopacity{0.389723}%
\pgfsetdash{}{0pt}%
\pgfpathmoveto{\pgfqpoint{0.626309in}{2.259069in}}%
\pgfpathcurveto{\pgfqpoint{0.635518in}{2.259069in}}{\pgfqpoint{0.644350in}{2.262728in}}{\pgfqpoint{0.650861in}{2.269239in}}%
\pgfpathcurveto{\pgfqpoint{0.657373in}{2.275750in}}{\pgfqpoint{0.661031in}{2.284583in}}{\pgfqpoint{0.661031in}{2.293791in}}%
\pgfpathcurveto{\pgfqpoint{0.661031in}{2.303000in}}{\pgfqpoint{0.657373in}{2.311832in}}{\pgfqpoint{0.650861in}{2.318344in}}%
\pgfpathcurveto{\pgfqpoint{0.644350in}{2.324855in}}{\pgfqpoint{0.635518in}{2.328514in}}{\pgfqpoint{0.626309in}{2.328514in}}%
\pgfpathcurveto{\pgfqpoint{0.617101in}{2.328514in}}{\pgfqpoint{0.608268in}{2.324855in}}{\pgfqpoint{0.601757in}{2.318344in}}%
\pgfpathcurveto{\pgfqpoint{0.595245in}{2.311832in}}{\pgfqpoint{0.591587in}{2.303000in}}{\pgfqpoint{0.591587in}{2.293791in}}%
\pgfpathcurveto{\pgfqpoint{0.591587in}{2.284583in}}{\pgfqpoint{0.595245in}{2.275750in}}{\pgfqpoint{0.601757in}{2.269239in}}%
\pgfpathcurveto{\pgfqpoint{0.608268in}{2.262728in}}{\pgfqpoint{0.617101in}{2.259069in}}{\pgfqpoint{0.626309in}{2.259069in}}%
\pgfpathlineto{\pgfqpoint{0.626309in}{2.259069in}}%
\pgfpathclose%
\pgfusepath{stroke,fill}%
\end{pgfscope}%
\begin{pgfscope}%
\pgfpathrectangle{\pgfqpoint{1.374500in}{0.082500in}}{\pgfqpoint{2.419000in}{2.419000in}}%
\pgfusepath{clip}%
\pgfsetbuttcap%
\pgfsetroundjoin%
\definecolor{currentfill}{rgb}{0.247059,0.564706,0.854902}%
\pgfsetfillcolor{currentfill}%
\pgfsetfillopacity{0.389723}%
\pgfsetlinewidth{1.003750pt}%
\definecolor{currentstroke}{rgb}{0.247059,0.564706,0.854902}%
\pgfsetstrokecolor{currentstroke}%
\pgfsetstrokeopacity{0.389723}%
\pgfsetdash{}{0pt}%
\pgfpathmoveto{\pgfqpoint{8.930237in}{2.259069in}}%
\pgfpathcurveto{\pgfqpoint{8.939446in}{2.259069in}}{\pgfqpoint{8.948278in}{2.262728in}}{\pgfqpoint{8.954790in}{2.269239in}}%
\pgfpathcurveto{\pgfqpoint{8.961301in}{2.275750in}}{\pgfqpoint{8.964960in}{2.284583in}}{\pgfqpoint{8.964960in}{2.293791in}}%
\pgfpathcurveto{\pgfqpoint{8.964960in}{2.303000in}}{\pgfqpoint{8.961301in}{2.311832in}}{\pgfqpoint{8.954790in}{2.318344in}}%
\pgfpathcurveto{\pgfqpoint{8.948278in}{2.324855in}}{\pgfqpoint{8.939446in}{2.328514in}}{\pgfqpoint{8.930237in}{2.328514in}}%
\pgfpathcurveto{\pgfqpoint{8.921029in}{2.328514in}}{\pgfqpoint{8.912196in}{2.324855in}}{\pgfqpoint{8.905685in}{2.318344in}}%
\pgfpathcurveto{\pgfqpoint{8.899174in}{2.311832in}}{\pgfqpoint{8.895515in}{2.303000in}}{\pgfqpoint{8.895515in}{2.293791in}}%
\pgfpathcurveto{\pgfqpoint{8.895515in}{2.284583in}}{\pgfqpoint{8.899174in}{2.275750in}}{\pgfqpoint{8.905685in}{2.269239in}}%
\pgfpathcurveto{\pgfqpoint{8.912196in}{2.262728in}}{\pgfqpoint{8.921029in}{2.259069in}}{\pgfqpoint{8.930237in}{2.259069in}}%
\pgfpathlineto{\pgfqpoint{8.930237in}{2.259069in}}%
\pgfpathclose%
\pgfusepath{stroke,fill}%
\end{pgfscope}%
\begin{pgfscope}%
\pgfpathrectangle{\pgfqpoint{1.374500in}{0.082500in}}{\pgfqpoint{2.419000in}{2.419000in}}%
\pgfusepath{clip}%
\pgfsetbuttcap%
\pgfsetroundjoin%
\definecolor{currentfill}{rgb}{0.247059,0.564706,0.854902}%
\pgfsetfillcolor{currentfill}%
\pgfsetfillopacity{0.394581}%
\pgfsetlinewidth{1.003750pt}%
\definecolor{currentstroke}{rgb}{0.247059,0.564706,0.854902}%
\pgfsetstrokecolor{currentstroke}%
\pgfsetstrokeopacity{0.394581}%
\pgfsetdash{}{0pt}%
\pgfpathmoveto{\pgfqpoint{3.098712in}{2.159648in}}%
\pgfpathcurveto{\pgfqpoint{3.107921in}{2.159648in}}{\pgfqpoint{3.116753in}{2.163307in}}{\pgfqpoint{3.123265in}{2.169818in}}%
\pgfpathcurveto{\pgfqpoint{3.129776in}{2.176329in}}{\pgfqpoint{3.133435in}{2.185162in}}{\pgfqpoint{3.133435in}{2.194370in}}%
\pgfpathcurveto{\pgfqpoint{3.133435in}{2.203579in}}{\pgfqpoint{3.129776in}{2.212411in}}{\pgfqpoint{3.123265in}{2.218922in}}%
\pgfpathcurveto{\pgfqpoint{3.116753in}{2.225434in}}{\pgfqpoint{3.107921in}{2.229092in}}{\pgfqpoint{3.098712in}{2.229092in}}%
\pgfpathcurveto{\pgfqpoint{3.089504in}{2.229092in}}{\pgfqpoint{3.080671in}{2.225434in}}{\pgfqpoint{3.074160in}{2.218922in}}%
\pgfpathcurveto{\pgfqpoint{3.067649in}{2.212411in}}{\pgfqpoint{3.063990in}{2.203579in}}{\pgfqpoint{3.063990in}{2.194370in}}%
\pgfpathcurveto{\pgfqpoint{3.063990in}{2.185162in}}{\pgfqpoint{3.067649in}{2.176329in}}{\pgfqpoint{3.074160in}{2.169818in}}%
\pgfpathcurveto{\pgfqpoint{3.080671in}{2.163307in}}{\pgfqpoint{3.089504in}{2.159648in}}{\pgfqpoint{3.098712in}{2.159648in}}%
\pgfpathlineto{\pgfqpoint{3.098712in}{2.159648in}}%
\pgfpathclose%
\pgfusepath{stroke,fill}%
\end{pgfscope}%
\begin{pgfscope}%
\pgfpathrectangle{\pgfqpoint{1.374500in}{0.082500in}}{\pgfqpoint{2.419000in}{2.419000in}}%
\pgfusepath{clip}%
\pgfsetbuttcap%
\pgfsetroundjoin%
\definecolor{currentfill}{rgb}{0.247059,0.564706,0.854902}%
\pgfsetfillcolor{currentfill}%
\pgfsetfillopacity{0.394581}%
\pgfsetlinewidth{1.003750pt}%
\definecolor{currentstroke}{rgb}{0.247059,0.564706,0.854902}%
\pgfsetstrokecolor{currentstroke}%
\pgfsetstrokeopacity{0.394581}%
\pgfsetdash{}{0pt}%
\pgfpathmoveto{\pgfqpoint{11.528107in}{2.159648in}}%
\pgfpathcurveto{\pgfqpoint{11.537315in}{2.159648in}}{\pgfqpoint{11.546148in}{2.163307in}}{\pgfqpoint{11.552659in}{2.169818in}}%
\pgfpathcurveto{\pgfqpoint{11.559171in}{2.176329in}}{\pgfqpoint{11.562829in}{2.185162in}}{\pgfqpoint{11.562829in}{2.194370in}}%
\pgfpathcurveto{\pgfqpoint{11.562829in}{2.203579in}}{\pgfqpoint{11.559171in}{2.212411in}}{\pgfqpoint{11.552659in}{2.218922in}}%
\pgfpathcurveto{\pgfqpoint{11.546148in}{2.225434in}}{\pgfqpoint{11.537315in}{2.229092in}}{\pgfqpoint{11.528107in}{2.229092in}}%
\pgfpathcurveto{\pgfqpoint{11.518899in}{2.229092in}}{\pgfqpoint{11.510066in}{2.225434in}}{\pgfqpoint{11.503555in}{2.218922in}}%
\pgfpathcurveto{\pgfqpoint{11.497043in}{2.212411in}}{\pgfqpoint{11.493385in}{2.203579in}}{\pgfqpoint{11.493385in}{2.194370in}}%
\pgfpathcurveto{\pgfqpoint{11.493385in}{2.185162in}}{\pgfqpoint{11.497043in}{2.176329in}}{\pgfqpoint{11.503555in}{2.169818in}}%
\pgfpathcurveto{\pgfqpoint{11.510066in}{2.163307in}}{\pgfqpoint{11.518899in}{2.159648in}}{\pgfqpoint{11.528107in}{2.159648in}}%
\pgfpathlineto{\pgfqpoint{11.528107in}{2.159648in}}%
\pgfpathclose%
\pgfusepath{stroke,fill}%
\end{pgfscope}%
\begin{pgfscope}%
\pgfpathrectangle{\pgfqpoint{1.374500in}{0.082500in}}{\pgfqpoint{2.419000in}{2.419000in}}%
\pgfusepath{clip}%
\pgfsetbuttcap%
\pgfsetroundjoin%
\definecolor{currentfill}{rgb}{0.247059,0.564706,0.854902}%
\pgfsetfillcolor{currentfill}%
\pgfsetfillopacity{0.394581}%
\pgfsetlinewidth{1.003750pt}%
\definecolor{currentstroke}{rgb}{0.247059,0.564706,0.854902}%
\pgfsetstrokecolor{currentstroke}%
\pgfsetstrokeopacity{0.394581}%
\pgfsetdash{}{0pt}%
\pgfpathmoveto{\pgfqpoint{7.313410in}{2.159648in}}%
\pgfpathcurveto{\pgfqpoint{7.322618in}{2.159648in}}{\pgfqpoint{7.331451in}{2.163307in}}{\pgfqpoint{7.337962in}{2.169818in}}%
\pgfpathcurveto{\pgfqpoint{7.344473in}{2.176329in}}{\pgfqpoint{7.348132in}{2.185162in}}{\pgfqpoint{7.348132in}{2.194370in}}%
\pgfpathcurveto{\pgfqpoint{7.348132in}{2.203579in}}{\pgfqpoint{7.344473in}{2.212411in}}{\pgfqpoint{7.337962in}{2.218922in}}%
\pgfpathcurveto{\pgfqpoint{7.331451in}{2.225434in}}{\pgfqpoint{7.322618in}{2.229092in}}{\pgfqpoint{7.313410in}{2.229092in}}%
\pgfpathcurveto{\pgfqpoint{7.304201in}{2.229092in}}{\pgfqpoint{7.295369in}{2.225434in}}{\pgfqpoint{7.288857in}{2.218922in}}%
\pgfpathcurveto{\pgfqpoint{7.282346in}{2.212411in}}{\pgfqpoint{7.278687in}{2.203579in}}{\pgfqpoint{7.278687in}{2.194370in}}%
\pgfpathcurveto{\pgfqpoint{7.278687in}{2.185162in}}{\pgfqpoint{7.282346in}{2.176329in}}{\pgfqpoint{7.288857in}{2.169818in}}%
\pgfpathcurveto{\pgfqpoint{7.295369in}{2.163307in}}{\pgfqpoint{7.304201in}{2.159648in}}{\pgfqpoint{7.313410in}{2.159648in}}%
\pgfpathlineto{\pgfqpoint{7.313410in}{2.159648in}}%
\pgfpathclose%
\pgfusepath{stroke,fill}%
\end{pgfscope}%
\begin{pgfscope}%
\pgfpathrectangle{\pgfqpoint{1.374500in}{0.082500in}}{\pgfqpoint{2.419000in}{2.419000in}}%
\pgfusepath{clip}%
\pgfsetbuttcap%
\pgfsetroundjoin%
\definecolor{currentfill}{rgb}{0.247059,0.564706,0.854902}%
\pgfsetfillcolor{currentfill}%
\pgfsetfillopacity{0.399589}%
\pgfsetlinewidth{1.003750pt}%
\definecolor{currentstroke}{rgb}{0.247059,0.564706,0.854902}%
\pgfsetstrokecolor{currentstroke}%
\pgfsetstrokeopacity{0.399589}%
\pgfsetdash{}{0pt}%
\pgfpathmoveto{\pgfqpoint{5.646974in}{2.057176in}}%
\pgfpathcurveto{\pgfqpoint{5.656183in}{2.057176in}}{\pgfqpoint{5.665015in}{2.060835in}}{\pgfqpoint{5.671527in}{2.067346in}}%
\pgfpathcurveto{\pgfqpoint{5.678038in}{2.073858in}}{\pgfqpoint{5.681697in}{2.082690in}}{\pgfqpoint{5.681697in}{2.091899in}}%
\pgfpathcurveto{\pgfqpoint{5.681697in}{2.101107in}}{\pgfqpoint{5.678038in}{2.109940in}}{\pgfqpoint{5.671527in}{2.116451in}}%
\pgfpathcurveto{\pgfqpoint{5.665015in}{2.122962in}}{\pgfqpoint{5.656183in}{2.126621in}}{\pgfqpoint{5.646974in}{2.126621in}}%
\pgfpathcurveto{\pgfqpoint{5.637766in}{2.126621in}}{\pgfqpoint{5.628933in}{2.122962in}}{\pgfqpoint{5.622422in}{2.116451in}}%
\pgfpathcurveto{\pgfqpoint{5.615911in}{2.109940in}}{\pgfqpoint{5.612252in}{2.101107in}}{\pgfqpoint{5.612252in}{2.091899in}}%
\pgfpathcurveto{\pgfqpoint{5.612252in}{2.082690in}}{\pgfqpoint{5.615911in}{2.073858in}}{\pgfqpoint{5.622422in}{2.067346in}}%
\pgfpathcurveto{\pgfqpoint{5.628933in}{2.060835in}}{\pgfqpoint{5.637766in}{2.057176in}}{\pgfqpoint{5.646974in}{2.057176in}}%
\pgfpathlineto{\pgfqpoint{5.646974in}{2.057176in}}%
\pgfpathclose%
\pgfusepath{stroke,fill}%
\end{pgfscope}%
\begin{pgfscope}%
\pgfpathrectangle{\pgfqpoint{1.374500in}{0.082500in}}{\pgfqpoint{2.419000in}{2.419000in}}%
\pgfusepath{clip}%
\pgfsetbuttcap%
\pgfsetroundjoin%
\definecolor{currentfill}{rgb}{0.247059,0.564706,0.854902}%
\pgfsetfillcolor{currentfill}%
\pgfsetfillopacity{0.399589}%
\pgfsetlinewidth{1.003750pt}%
\definecolor{currentstroke}{rgb}{0.247059,0.564706,0.854902}%
\pgfsetstrokecolor{currentstroke}%
\pgfsetstrokeopacity{0.399589}%
\pgfsetdash{}{0pt}%
\pgfpathmoveto{\pgfqpoint{1.367619in}{2.057176in}}%
\pgfpathcurveto{\pgfqpoint{1.376828in}{2.057176in}}{\pgfqpoint{1.385660in}{2.060835in}}{\pgfqpoint{1.392171in}{2.067346in}}%
\pgfpathcurveto{\pgfqpoint{1.398683in}{2.073858in}}{\pgfqpoint{1.402341in}{2.082690in}}{\pgfqpoint{1.402341in}{2.091899in}}%
\pgfpathcurveto{\pgfqpoint{1.402341in}{2.101107in}}{\pgfqpoint{1.398683in}{2.109940in}}{\pgfqpoint{1.392171in}{2.116451in}}%
\pgfpathcurveto{\pgfqpoint{1.385660in}{2.122962in}}{\pgfqpoint{1.376828in}{2.126621in}}{\pgfqpoint{1.367619in}{2.126621in}}%
\pgfpathcurveto{\pgfqpoint{1.358411in}{2.126621in}}{\pgfqpoint{1.349578in}{2.122962in}}{\pgfqpoint{1.343067in}{2.116451in}}%
\pgfpathcurveto{\pgfqpoint{1.336555in}{2.109940in}}{\pgfqpoint{1.332897in}{2.101107in}}{\pgfqpoint{1.332897in}{2.091899in}}%
\pgfpathcurveto{\pgfqpoint{1.332897in}{2.082690in}}{\pgfqpoint{1.336555in}{2.073858in}}{\pgfqpoint{1.343067in}{2.067346in}}%
\pgfpathcurveto{\pgfqpoint{1.349578in}{2.060835in}}{\pgfqpoint{1.358411in}{2.057176in}}{\pgfqpoint{1.367619in}{2.057176in}}%
\pgfpathlineto{\pgfqpoint{1.367619in}{2.057176in}}%
\pgfpathclose%
\pgfusepath{stroke,fill}%
\end{pgfscope}%
\begin{pgfscope}%
\pgfpathrectangle{\pgfqpoint{1.374500in}{0.082500in}}{\pgfqpoint{2.419000in}{2.419000in}}%
\pgfusepath{clip}%
\pgfsetbuttcap%
\pgfsetroundjoin%
\definecolor{currentfill}{rgb}{0.247059,0.564706,0.854902}%
\pgfsetfillcolor{currentfill}%
\pgfsetfillopacity{0.399589}%
\pgfsetlinewidth{1.003750pt}%
\definecolor{currentstroke}{rgb}{0.247059,0.564706,0.854902}%
\pgfsetstrokecolor{currentstroke}%
\pgfsetstrokeopacity{0.399589}%
\pgfsetdash{}{0pt}%
\pgfpathmoveto{\pgfqpoint{9.926330in}{2.057176in}}%
\pgfpathcurveto{\pgfqpoint{9.935538in}{2.057176in}}{\pgfqpoint{9.944370in}{2.060835in}}{\pgfqpoint{9.950882in}{2.067346in}}%
\pgfpathcurveto{\pgfqpoint{9.957393in}{2.073858in}}{\pgfqpoint{9.961052in}{2.082690in}}{\pgfqpoint{9.961052in}{2.091899in}}%
\pgfpathcurveto{\pgfqpoint{9.961052in}{2.101107in}}{\pgfqpoint{9.957393in}{2.109940in}}{\pgfqpoint{9.950882in}{2.116451in}}%
\pgfpathcurveto{\pgfqpoint{9.944370in}{2.122962in}}{\pgfqpoint{9.935538in}{2.126621in}}{\pgfqpoint{9.926330in}{2.126621in}}%
\pgfpathcurveto{\pgfqpoint{9.917121in}{2.126621in}}{\pgfqpoint{9.908289in}{2.122962in}}{\pgfqpoint{9.901777in}{2.116451in}}%
\pgfpathcurveto{\pgfqpoint{9.895266in}{2.109940in}}{\pgfqpoint{9.891607in}{2.101107in}}{\pgfqpoint{9.891607in}{2.091899in}}%
\pgfpathcurveto{\pgfqpoint{9.891607in}{2.082690in}}{\pgfqpoint{9.895266in}{2.073858in}}{\pgfqpoint{9.901777in}{2.067346in}}%
\pgfpathcurveto{\pgfqpoint{9.908289in}{2.060835in}}{\pgfqpoint{9.917121in}{2.057176in}}{\pgfqpoint{9.926330in}{2.057176in}}%
\pgfpathlineto{\pgfqpoint{9.926330in}{2.057176in}}%
\pgfpathclose%
\pgfusepath{stroke,fill}%
\end{pgfscope}%
\begin{pgfscope}%
\pgfpathrectangle{\pgfqpoint{1.374500in}{0.082500in}}{\pgfqpoint{2.419000in}{2.419000in}}%
\pgfusepath{clip}%
\pgfsetbuttcap%
\pgfsetroundjoin%
\definecolor{currentfill}{rgb}{0.247059,0.564706,0.854902}%
\pgfsetfillcolor{currentfill}%
\pgfsetfillopacity{0.404753}%
\pgfsetlinewidth{1.003750pt}%
\definecolor{currentstroke}{rgb}{0.247059,0.564706,0.854902}%
\pgfsetstrokecolor{currentstroke}%
\pgfsetstrokeopacity{0.404753}%
\pgfsetdash{}{0pt}%
\pgfpathmoveto{\pgfqpoint{3.928613in}{1.951512in}}%
\pgfpathcurveto{\pgfqpoint{3.937821in}{1.951512in}}{\pgfqpoint{3.946654in}{1.955170in}}{\pgfqpoint{3.953165in}{1.961682in}}%
\pgfpathcurveto{\pgfqpoint{3.959676in}{1.968193in}}{\pgfqpoint{3.963335in}{1.977025in}}{\pgfqpoint{3.963335in}{1.986234in}}%
\pgfpathcurveto{\pgfqpoint{3.963335in}{1.995442in}}{\pgfqpoint{3.959676in}{2.004275in}}{\pgfqpoint{3.953165in}{2.010786in}}%
\pgfpathcurveto{\pgfqpoint{3.946654in}{2.017298in}}{\pgfqpoint{3.937821in}{2.020956in}}{\pgfqpoint{3.928613in}{2.020956in}}%
\pgfpathcurveto{\pgfqpoint{3.919404in}{2.020956in}}{\pgfqpoint{3.910572in}{2.017298in}}{\pgfqpoint{3.904060in}{2.010786in}}%
\pgfpathcurveto{\pgfqpoint{3.897549in}{2.004275in}}{\pgfqpoint{3.893890in}{1.995442in}}{\pgfqpoint{3.893890in}{1.986234in}}%
\pgfpathcurveto{\pgfqpoint{3.893890in}{1.977025in}}{\pgfqpoint{3.897549in}{1.968193in}}{\pgfqpoint{3.904060in}{1.961682in}}%
\pgfpathcurveto{\pgfqpoint{3.910572in}{1.955170in}}{\pgfqpoint{3.919404in}{1.951512in}}{\pgfqpoint{3.928613in}{1.951512in}}%
\pgfpathlineto{\pgfqpoint{3.928613in}{1.951512in}}%
\pgfpathclose%
\pgfusepath{stroke,fill}%
\end{pgfscope}%
\begin{pgfscope}%
\pgfpathrectangle{\pgfqpoint{1.374500in}{0.082500in}}{\pgfqpoint{2.419000in}{2.419000in}}%
\pgfusepath{clip}%
\pgfsetbuttcap%
\pgfsetroundjoin%
\definecolor{currentfill}{rgb}{0.247059,0.564706,0.854902}%
\pgfsetfillcolor{currentfill}%
\pgfsetfillopacity{0.404753}%
\pgfsetlinewidth{1.003750pt}%
\definecolor{currentstroke}{rgb}{0.247059,0.564706,0.854902}%
\pgfsetstrokecolor{currentstroke}%
\pgfsetstrokeopacity{0.404753}%
\pgfsetdash{}{0pt}%
\pgfpathmoveto{\pgfqpoint{-0.417415in}{1.951512in}}%
\pgfpathcurveto{\pgfqpoint{-0.408207in}{1.951512in}}{\pgfqpoint{-0.399374in}{1.955170in}}{\pgfqpoint{-0.392863in}{1.961682in}}%
\pgfpathcurveto{\pgfqpoint{-0.386352in}{1.968193in}}{\pgfqpoint{-0.382693in}{1.977025in}}{\pgfqpoint{-0.382693in}{1.986234in}}%
\pgfpathcurveto{\pgfqpoint{-0.382693in}{1.995442in}}{\pgfqpoint{-0.386352in}{2.004275in}}{\pgfqpoint{-0.392863in}{2.010786in}}%
\pgfpathcurveto{\pgfqpoint{-0.399374in}{2.017298in}}{\pgfqpoint{-0.408207in}{2.020956in}}{\pgfqpoint{-0.417415in}{2.020956in}}%
\pgfpathcurveto{\pgfqpoint{-0.426624in}{2.020956in}}{\pgfqpoint{-0.435456in}{2.017298in}}{\pgfqpoint{-0.441968in}{2.010786in}}%
\pgfpathcurveto{\pgfqpoint{-0.448479in}{2.004275in}}{\pgfqpoint{-0.452138in}{1.995442in}}{\pgfqpoint{-0.452138in}{1.986234in}}%
\pgfpathcurveto{\pgfqpoint{-0.452138in}{1.977025in}}{\pgfqpoint{-0.448479in}{1.968193in}}{\pgfqpoint{-0.441968in}{1.961682in}}%
\pgfpathcurveto{\pgfqpoint{-0.435456in}{1.955170in}}{\pgfqpoint{-0.426624in}{1.951512in}}{\pgfqpoint{-0.417415in}{1.951512in}}%
\pgfpathlineto{\pgfqpoint{-0.417415in}{1.951512in}}%
\pgfpathclose%
\pgfusepath{stroke,fill}%
\end{pgfscope}%
\begin{pgfscope}%
\pgfpathrectangle{\pgfqpoint{1.374500in}{0.082500in}}{\pgfqpoint{2.419000in}{2.419000in}}%
\pgfusepath{clip}%
\pgfsetbuttcap%
\pgfsetroundjoin%
\definecolor{currentfill}{rgb}{0.247059,0.564706,0.854902}%
\pgfsetfillcolor{currentfill}%
\pgfsetfillopacity{0.404753}%
\pgfsetlinewidth{1.003750pt}%
\definecolor{currentstroke}{rgb}{0.247059,0.564706,0.854902}%
\pgfsetstrokecolor{currentstroke}%
\pgfsetstrokeopacity{0.404753}%
\pgfsetdash{}{0pt}%
\pgfpathmoveto{\pgfqpoint{8.274640in}{1.951512in}}%
\pgfpathcurveto{\pgfqpoint{8.283849in}{1.951512in}}{\pgfqpoint{8.292681in}{1.955170in}}{\pgfqpoint{8.299193in}{1.961682in}}%
\pgfpathcurveto{\pgfqpoint{8.305704in}{1.968193in}}{\pgfqpoint{8.309363in}{1.977025in}}{\pgfqpoint{8.309363in}{1.986234in}}%
\pgfpathcurveto{\pgfqpoint{8.309363in}{1.995442in}}{\pgfqpoint{8.305704in}{2.004275in}}{\pgfqpoint{8.299193in}{2.010786in}}%
\pgfpathcurveto{\pgfqpoint{8.292681in}{2.017298in}}{\pgfqpoint{8.283849in}{2.020956in}}{\pgfqpoint{8.274640in}{2.020956in}}%
\pgfpathcurveto{\pgfqpoint{8.265432in}{2.020956in}}{\pgfqpoint{8.256599in}{2.017298in}}{\pgfqpoint{8.250088in}{2.010786in}}%
\pgfpathcurveto{\pgfqpoint{8.243577in}{2.004275in}}{\pgfqpoint{8.239918in}{1.995442in}}{\pgfqpoint{8.239918in}{1.986234in}}%
\pgfpathcurveto{\pgfqpoint{8.239918in}{1.977025in}}{\pgfqpoint{8.243577in}{1.968193in}}{\pgfqpoint{8.250088in}{1.961682in}}%
\pgfpathcurveto{\pgfqpoint{8.256599in}{1.955170in}}{\pgfqpoint{8.265432in}{1.951512in}}{\pgfqpoint{8.274640in}{1.951512in}}%
\pgfpathlineto{\pgfqpoint{8.274640in}{1.951512in}}%
\pgfpathclose%
\pgfusepath{stroke,fill}%
\end{pgfscope}%
\begin{pgfscope}%
\pgfpathrectangle{\pgfqpoint{1.374500in}{0.082500in}}{\pgfqpoint{2.419000in}{2.419000in}}%
\pgfusepath{clip}%
\pgfsetbuttcap%
\pgfsetroundjoin%
\definecolor{currentfill}{rgb}{0.247059,0.564706,0.854902}%
\pgfsetfillcolor{currentfill}%
\pgfsetfillopacity{0.410080}%
\pgfsetlinewidth{1.003750pt}%
\definecolor{currentstroke}{rgb}{0.247059,0.564706,0.854902}%
\pgfsetstrokecolor{currentstroke}%
\pgfsetstrokeopacity{0.410080}%
\pgfsetdash{}{0pt}%
\pgfpathmoveto{\pgfqpoint{2.155859in}{1.842502in}}%
\pgfpathcurveto{\pgfqpoint{2.165067in}{1.842502in}}{\pgfqpoint{2.173900in}{1.846161in}}{\pgfqpoint{2.180411in}{1.852672in}}%
\pgfpathcurveto{\pgfqpoint{2.186923in}{1.859184in}}{\pgfqpoint{2.190581in}{1.868016in}}{\pgfqpoint{2.190581in}{1.877225in}}%
\pgfpathcurveto{\pgfqpoint{2.190581in}{1.886433in}}{\pgfqpoint{2.186923in}{1.895266in}}{\pgfqpoint{2.180411in}{1.901777in}}%
\pgfpathcurveto{\pgfqpoint{2.173900in}{1.908288in}}{\pgfqpoint{2.165067in}{1.911947in}}{\pgfqpoint{2.155859in}{1.911947in}}%
\pgfpathcurveto{\pgfqpoint{2.146650in}{1.911947in}}{\pgfqpoint{2.137818in}{1.908288in}}{\pgfqpoint{2.131307in}{1.901777in}}%
\pgfpathcurveto{\pgfqpoint{2.124795in}{1.895266in}}{\pgfqpoint{2.121137in}{1.886433in}}{\pgfqpoint{2.121137in}{1.877225in}}%
\pgfpathcurveto{\pgfqpoint{2.121137in}{1.868016in}}{\pgfqpoint{2.124795in}{1.859184in}}{\pgfqpoint{2.131307in}{1.852672in}}%
\pgfpathcurveto{\pgfqpoint{2.137818in}{1.846161in}}{\pgfqpoint{2.146650in}{1.842502in}}{\pgfqpoint{2.155859in}{1.842502in}}%
\pgfpathlineto{\pgfqpoint{2.155859in}{1.842502in}}%
\pgfpathclose%
\pgfusepath{stroke,fill}%
\end{pgfscope}%
\begin{pgfscope}%
\pgfpathrectangle{\pgfqpoint{1.374500in}{0.082500in}}{\pgfqpoint{2.419000in}{2.419000in}}%
\pgfusepath{clip}%
\pgfsetbuttcap%
\pgfsetroundjoin%
\definecolor{currentfill}{rgb}{0.247059,0.564706,0.854902}%
\pgfsetfillcolor{currentfill}%
\pgfsetfillopacity{0.410080}%
\pgfsetlinewidth{1.003750pt}%
\definecolor{currentstroke}{rgb}{0.247059,0.564706,0.854902}%
\pgfsetstrokecolor{currentstroke}%
\pgfsetstrokeopacity{0.410080}%
\pgfsetdash{}{0pt}%
\pgfpathmoveto{\pgfqpoint{6.570670in}{1.842502in}}%
\pgfpathcurveto{\pgfqpoint{6.579878in}{1.842502in}}{\pgfqpoint{6.588711in}{1.846161in}}{\pgfqpoint{6.595222in}{1.852672in}}%
\pgfpathcurveto{\pgfqpoint{6.601734in}{1.859184in}}{\pgfqpoint{6.605392in}{1.868016in}}{\pgfqpoint{6.605392in}{1.877225in}}%
\pgfpathcurveto{\pgfqpoint{6.605392in}{1.886433in}}{\pgfqpoint{6.601734in}{1.895266in}}{\pgfqpoint{6.595222in}{1.901777in}}%
\pgfpathcurveto{\pgfqpoint{6.588711in}{1.908288in}}{\pgfqpoint{6.579878in}{1.911947in}}{\pgfqpoint{6.570670in}{1.911947in}}%
\pgfpathcurveto{\pgfqpoint{6.561462in}{1.911947in}}{\pgfqpoint{6.552629in}{1.908288in}}{\pgfqpoint{6.546118in}{1.901777in}}%
\pgfpathcurveto{\pgfqpoint{6.539606in}{1.895266in}}{\pgfqpoint{6.535948in}{1.886433in}}{\pgfqpoint{6.535948in}{1.877225in}}%
\pgfpathcurveto{\pgfqpoint{6.535948in}{1.868016in}}{\pgfqpoint{6.539606in}{1.859184in}}{\pgfqpoint{6.546118in}{1.852672in}}%
\pgfpathcurveto{\pgfqpoint{6.552629in}{1.846161in}}{\pgfqpoint{6.561462in}{1.842502in}}{\pgfqpoint{6.570670in}{1.842502in}}%
\pgfpathlineto{\pgfqpoint{6.570670in}{1.842502in}}%
\pgfpathclose%
\pgfusepath{stroke,fill}%
\end{pgfscope}%
\begin{pgfscope}%
\pgfpathrectangle{\pgfqpoint{1.374500in}{0.082500in}}{\pgfqpoint{2.419000in}{2.419000in}}%
\pgfusepath{clip}%
\pgfsetbuttcap%
\pgfsetroundjoin%
\definecolor{currentfill}{rgb}{0.247059,0.564706,0.854902}%
\pgfsetfillcolor{currentfill}%
\pgfsetfillopacity{0.410080}%
\pgfsetlinewidth{1.003750pt}%
\definecolor{currentstroke}{rgb}{0.247059,0.564706,0.854902}%
\pgfsetstrokecolor{currentstroke}%
\pgfsetstrokeopacity{0.410080}%
\pgfsetdash{}{0pt}%
\pgfpathmoveto{\pgfqpoint{10.985481in}{1.842502in}}%
\pgfpathcurveto{\pgfqpoint{10.994689in}{1.842502in}}{\pgfqpoint{11.003522in}{1.846161in}}{\pgfqpoint{11.010033in}{1.852672in}}%
\pgfpathcurveto{\pgfqpoint{11.016545in}{1.859184in}}{\pgfqpoint{11.020203in}{1.868016in}}{\pgfqpoint{11.020203in}{1.877225in}}%
\pgfpathcurveto{\pgfqpoint{11.020203in}{1.886433in}}{\pgfqpoint{11.016545in}{1.895266in}}{\pgfqpoint{11.010033in}{1.901777in}}%
\pgfpathcurveto{\pgfqpoint{11.003522in}{1.908288in}}{\pgfqpoint{10.994689in}{1.911947in}}{\pgfqpoint{10.985481in}{1.911947in}}%
\pgfpathcurveto{\pgfqpoint{10.976273in}{1.911947in}}{\pgfqpoint{10.967440in}{1.908288in}}{\pgfqpoint{10.960929in}{1.901777in}}%
\pgfpathcurveto{\pgfqpoint{10.954417in}{1.895266in}}{\pgfqpoint{10.950759in}{1.886433in}}{\pgfqpoint{10.950759in}{1.877225in}}%
\pgfpathcurveto{\pgfqpoint{10.950759in}{1.868016in}}{\pgfqpoint{10.954417in}{1.859184in}}{\pgfqpoint{10.960929in}{1.852672in}}%
\pgfpathcurveto{\pgfqpoint{10.967440in}{1.846161in}}{\pgfqpoint{10.976273in}{1.842502in}}{\pgfqpoint{10.985481in}{1.842502in}}%
\pgfpathlineto{\pgfqpoint{10.985481in}{1.842502in}}%
\pgfpathclose%
\pgfusepath{stroke,fill}%
\end{pgfscope}%
\begin{pgfscope}%
\pgfpathrectangle{\pgfqpoint{1.374500in}{0.082500in}}{\pgfqpoint{2.419000in}{2.419000in}}%
\pgfusepath{clip}%
\pgfsetbuttcap%
\pgfsetroundjoin%
\definecolor{currentfill}{rgb}{0.247059,0.564706,0.854902}%
\pgfsetfillcolor{currentfill}%
\pgfsetfillopacity{0.415579}%
\pgfsetlinewidth{1.003750pt}%
\definecolor{currentstroke}{rgb}{0.247059,0.564706,0.854902}%
\pgfsetstrokecolor{currentstroke}%
\pgfsetstrokeopacity{0.415579}%
\pgfsetdash{}{0pt}%
\pgfpathmoveto{\pgfqpoint{0.326089in}{1.729987in}}%
\pgfpathcurveto{\pgfqpoint{0.335298in}{1.729987in}}{\pgfqpoint{0.344130in}{1.733646in}}{\pgfqpoint{0.350642in}{1.740157in}}%
\pgfpathcurveto{\pgfqpoint{0.357153in}{1.746668in}}{\pgfqpoint{0.360812in}{1.755501in}}{\pgfqpoint{0.360812in}{1.764709in}}%
\pgfpathcurveto{\pgfqpoint{0.360812in}{1.773918in}}{\pgfqpoint{0.357153in}{1.782750in}}{\pgfqpoint{0.350642in}{1.789262in}}%
\pgfpathcurveto{\pgfqpoint{0.344130in}{1.795773in}}{\pgfqpoint{0.335298in}{1.799432in}}{\pgfqpoint{0.326089in}{1.799432in}}%
\pgfpathcurveto{\pgfqpoint{0.316881in}{1.799432in}}{\pgfqpoint{0.308048in}{1.795773in}}{\pgfqpoint{0.301537in}{1.789262in}}%
\pgfpathcurveto{\pgfqpoint{0.295026in}{1.782750in}}{\pgfqpoint{0.291367in}{1.773918in}}{\pgfqpoint{0.291367in}{1.764709in}}%
\pgfpathcurveto{\pgfqpoint{0.291367in}{1.755501in}}{\pgfqpoint{0.295026in}{1.746668in}}{\pgfqpoint{0.301537in}{1.740157in}}%
\pgfpathcurveto{\pgfqpoint{0.308048in}{1.733646in}}{\pgfqpoint{0.316881in}{1.729987in}}{\pgfqpoint{0.326089in}{1.729987in}}%
\pgfpathlineto{\pgfqpoint{0.326089in}{1.729987in}}%
\pgfpathclose%
\pgfusepath{stroke,fill}%
\end{pgfscope}%
\begin{pgfscope}%
\pgfpathrectangle{\pgfqpoint{1.374500in}{0.082500in}}{\pgfqpoint{2.419000in}{2.419000in}}%
\pgfusepath{clip}%
\pgfsetbuttcap%
\pgfsetroundjoin%
\definecolor{currentfill}{rgb}{0.247059,0.564706,0.854902}%
\pgfsetfillcolor{currentfill}%
\pgfsetfillopacity{0.415579}%
\pgfsetlinewidth{1.003750pt}%
\definecolor{currentstroke}{rgb}{0.247059,0.564706,0.854902}%
\pgfsetstrokecolor{currentstroke}%
\pgfsetstrokeopacity{0.415579}%
\pgfsetdash{}{0pt}%
\pgfpathmoveto{\pgfqpoint{4.811896in}{1.729987in}}%
\pgfpathcurveto{\pgfqpoint{4.821104in}{1.729987in}}{\pgfqpoint{4.829937in}{1.733646in}}{\pgfqpoint{4.836448in}{1.740157in}}%
\pgfpathcurveto{\pgfqpoint{4.842959in}{1.746668in}}{\pgfqpoint{4.846618in}{1.755501in}}{\pgfqpoint{4.846618in}{1.764709in}}%
\pgfpathcurveto{\pgfqpoint{4.846618in}{1.773918in}}{\pgfqpoint{4.842959in}{1.782750in}}{\pgfqpoint{4.836448in}{1.789262in}}%
\pgfpathcurveto{\pgfqpoint{4.829937in}{1.795773in}}{\pgfqpoint{4.821104in}{1.799432in}}{\pgfqpoint{4.811896in}{1.799432in}}%
\pgfpathcurveto{\pgfqpoint{4.802687in}{1.799432in}}{\pgfqpoint{4.793855in}{1.795773in}}{\pgfqpoint{4.787343in}{1.789262in}}%
\pgfpathcurveto{\pgfqpoint{4.780832in}{1.782750in}}{\pgfqpoint{4.777174in}{1.773918in}}{\pgfqpoint{4.777174in}{1.764709in}}%
\pgfpathcurveto{\pgfqpoint{4.777174in}{1.755501in}}{\pgfqpoint{4.780832in}{1.746668in}}{\pgfqpoint{4.787343in}{1.740157in}}%
\pgfpathcurveto{\pgfqpoint{4.793855in}{1.733646in}}{\pgfqpoint{4.802687in}{1.729987in}}{\pgfqpoint{4.811896in}{1.729987in}}%
\pgfpathlineto{\pgfqpoint{4.811896in}{1.729987in}}%
\pgfpathclose%
\pgfusepath{stroke,fill}%
\end{pgfscope}%
\begin{pgfscope}%
\pgfpathrectangle{\pgfqpoint{1.374500in}{0.082500in}}{\pgfqpoint{2.419000in}{2.419000in}}%
\pgfusepath{clip}%
\pgfsetbuttcap%
\pgfsetroundjoin%
\definecolor{currentfill}{rgb}{0.247059,0.564706,0.854902}%
\pgfsetfillcolor{currentfill}%
\pgfsetfillopacity{0.415579}%
\pgfsetlinewidth{1.003750pt}%
\definecolor{currentstroke}{rgb}{0.247059,0.564706,0.854902}%
\pgfsetstrokecolor{currentstroke}%
\pgfsetstrokeopacity{0.415579}%
\pgfsetdash{}{0pt}%
\pgfpathmoveto{\pgfqpoint{9.297702in}{1.729987in}}%
\pgfpathcurveto{\pgfqpoint{9.306911in}{1.729987in}}{\pgfqpoint{9.315743in}{1.733646in}}{\pgfqpoint{9.322254in}{1.740157in}}%
\pgfpathcurveto{\pgfqpoint{9.328766in}{1.746668in}}{\pgfqpoint{9.332424in}{1.755501in}}{\pgfqpoint{9.332424in}{1.764709in}}%
\pgfpathcurveto{\pgfqpoint{9.332424in}{1.773918in}}{\pgfqpoint{9.328766in}{1.782750in}}{\pgfqpoint{9.322254in}{1.789262in}}%
\pgfpathcurveto{\pgfqpoint{9.315743in}{1.795773in}}{\pgfqpoint{9.306911in}{1.799432in}}{\pgfqpoint{9.297702in}{1.799432in}}%
\pgfpathcurveto{\pgfqpoint{9.288494in}{1.799432in}}{\pgfqpoint{9.279661in}{1.795773in}}{\pgfqpoint{9.273150in}{1.789262in}}%
\pgfpathcurveto{\pgfqpoint{9.266638in}{1.782750in}}{\pgfqpoint{9.262980in}{1.773918in}}{\pgfqpoint{9.262980in}{1.764709in}}%
\pgfpathcurveto{\pgfqpoint{9.262980in}{1.755501in}}{\pgfqpoint{9.266638in}{1.746668in}}{\pgfqpoint{9.273150in}{1.740157in}}%
\pgfpathcurveto{\pgfqpoint{9.279661in}{1.733646in}}{\pgfqpoint{9.288494in}{1.729987in}}{\pgfqpoint{9.297702in}{1.729987in}}%
\pgfpathlineto{\pgfqpoint{9.297702in}{1.729987in}}%
\pgfpathclose%
\pgfusepath{stroke,fill}%
\end{pgfscope}%
\begin{pgfscope}%
\pgfpathrectangle{\pgfqpoint{1.374500in}{0.082500in}}{\pgfqpoint{2.419000in}{2.419000in}}%
\pgfusepath{clip}%
\pgfsetbuttcap%
\pgfsetroundjoin%
\definecolor{currentfill}{rgb}{0.247059,0.564706,0.854902}%
\pgfsetfillcolor{currentfill}%
\pgfsetfillopacity{0.421257}%
\pgfsetlinewidth{1.003750pt}%
\definecolor{currentstroke}{rgb}{0.247059,0.564706,0.854902}%
\pgfsetstrokecolor{currentstroke}%
\pgfsetstrokeopacity{0.421257}%
\pgfsetdash{}{0pt}%
\pgfpathmoveto{\pgfqpoint{-1.563492in}{1.613794in}}%
\pgfpathcurveto{\pgfqpoint{-1.554283in}{1.613794in}}{\pgfqpoint{-1.545451in}{1.617452in}}{\pgfqpoint{-1.538939in}{1.623964in}}%
\pgfpathcurveto{\pgfqpoint{-1.532428in}{1.630475in}}{\pgfqpoint{-1.528769in}{1.639308in}}{\pgfqpoint{-1.528769in}{1.648516in}}%
\pgfpathcurveto{\pgfqpoint{-1.528769in}{1.657725in}}{\pgfqpoint{-1.532428in}{1.666557in}}{\pgfqpoint{-1.538939in}{1.673068in}}%
\pgfpathcurveto{\pgfqpoint{-1.545451in}{1.679580in}}{\pgfqpoint{-1.554283in}{1.683238in}}{\pgfqpoint{-1.563492in}{1.683238in}}%
\pgfpathcurveto{\pgfqpoint{-1.572700in}{1.683238in}}{\pgfqpoint{-1.581533in}{1.679580in}}{\pgfqpoint{-1.588044in}{1.673068in}}%
\pgfpathcurveto{\pgfqpoint{-1.594555in}{1.666557in}}{\pgfqpoint{-1.598214in}{1.657725in}}{\pgfqpoint{-1.598214in}{1.648516in}}%
\pgfpathcurveto{\pgfqpoint{-1.598214in}{1.639308in}}{\pgfqpoint{-1.594555in}{1.630475in}}{\pgfqpoint{-1.588044in}{1.623964in}}%
\pgfpathcurveto{\pgfqpoint{-1.581533in}{1.617452in}}{\pgfqpoint{-1.572700in}{1.613794in}}{\pgfqpoint{-1.563492in}{1.613794in}}%
\pgfpathlineto{\pgfqpoint{-1.563492in}{1.613794in}}%
\pgfpathclose%
\pgfusepath{stroke,fill}%
\end{pgfscope}%
\begin{pgfscope}%
\pgfpathrectangle{\pgfqpoint{1.374500in}{0.082500in}}{\pgfqpoint{2.419000in}{2.419000in}}%
\pgfusepath{clip}%
\pgfsetbuttcap%
\pgfsetroundjoin%
\definecolor{currentfill}{rgb}{0.247059,0.564706,0.854902}%
\pgfsetfillcolor{currentfill}%
\pgfsetfillopacity{0.421257}%
\pgfsetlinewidth{1.003750pt}%
\definecolor{currentstroke}{rgb}{0.247059,0.564706,0.854902}%
\pgfsetstrokecolor{currentstroke}%
\pgfsetstrokeopacity{0.421257}%
\pgfsetdash{}{0pt}%
\pgfpathmoveto{\pgfqpoint{2.995631in}{1.613794in}}%
\pgfpathcurveto{\pgfqpoint{3.004839in}{1.613794in}}{\pgfqpoint{3.013672in}{1.617452in}}{\pgfqpoint{3.020183in}{1.623964in}}%
\pgfpathcurveto{\pgfqpoint{3.026694in}{1.630475in}}{\pgfqpoint{3.030353in}{1.639308in}}{\pgfqpoint{3.030353in}{1.648516in}}%
\pgfpathcurveto{\pgfqpoint{3.030353in}{1.657725in}}{\pgfqpoint{3.026694in}{1.666557in}}{\pgfqpoint{3.020183in}{1.673068in}}%
\pgfpathcurveto{\pgfqpoint{3.013672in}{1.679580in}}{\pgfqpoint{3.004839in}{1.683238in}}{\pgfqpoint{2.995631in}{1.683238in}}%
\pgfpathcurveto{\pgfqpoint{2.986422in}{1.683238in}}{\pgfqpoint{2.977590in}{1.679580in}}{\pgfqpoint{2.971078in}{1.673068in}}%
\pgfpathcurveto{\pgfqpoint{2.964567in}{1.666557in}}{\pgfqpoint{2.960909in}{1.657725in}}{\pgfqpoint{2.960909in}{1.648516in}}%
\pgfpathcurveto{\pgfqpoint{2.960909in}{1.639308in}}{\pgfqpoint{2.964567in}{1.630475in}}{\pgfqpoint{2.971078in}{1.623964in}}%
\pgfpathcurveto{\pgfqpoint{2.977590in}{1.617452in}}{\pgfqpoint{2.986422in}{1.613794in}}{\pgfqpoint{2.995631in}{1.613794in}}%
\pgfpathlineto{\pgfqpoint{2.995631in}{1.613794in}}%
\pgfpathclose%
\pgfusepath{stroke,fill}%
\end{pgfscope}%
\begin{pgfscope}%
\pgfpathrectangle{\pgfqpoint{1.374500in}{0.082500in}}{\pgfqpoint{2.419000in}{2.419000in}}%
\pgfusepath{clip}%
\pgfsetbuttcap%
\pgfsetroundjoin%
\definecolor{currentfill}{rgb}{0.247059,0.564706,0.854902}%
\pgfsetfillcolor{currentfill}%
\pgfsetfillopacity{0.421257}%
\pgfsetlinewidth{1.003750pt}%
\definecolor{currentstroke}{rgb}{0.247059,0.564706,0.854902}%
\pgfsetstrokecolor{currentstroke}%
\pgfsetstrokeopacity{0.421257}%
\pgfsetdash{}{0pt}%
\pgfpathmoveto{\pgfqpoint{7.554753in}{1.613794in}}%
\pgfpathcurveto{\pgfqpoint{7.563962in}{1.613794in}}{\pgfqpoint{7.572794in}{1.617452in}}{\pgfqpoint{7.579305in}{1.623964in}}%
\pgfpathcurveto{\pgfqpoint{7.585817in}{1.630475in}}{\pgfqpoint{7.589475in}{1.639308in}}{\pgfqpoint{7.589475in}{1.648516in}}%
\pgfpathcurveto{\pgfqpoint{7.589475in}{1.657725in}}{\pgfqpoint{7.585817in}{1.666557in}}{\pgfqpoint{7.579305in}{1.673068in}}%
\pgfpathcurveto{\pgfqpoint{7.572794in}{1.679580in}}{\pgfqpoint{7.563962in}{1.683238in}}{\pgfqpoint{7.554753in}{1.683238in}}%
\pgfpathcurveto{\pgfqpoint{7.545545in}{1.683238in}}{\pgfqpoint{7.536712in}{1.679580in}}{\pgfqpoint{7.530201in}{1.673068in}}%
\pgfpathcurveto{\pgfqpoint{7.523689in}{1.666557in}}{\pgfqpoint{7.520031in}{1.657725in}}{\pgfqpoint{7.520031in}{1.648516in}}%
\pgfpathcurveto{\pgfqpoint{7.520031in}{1.639308in}}{\pgfqpoint{7.523689in}{1.630475in}}{\pgfqpoint{7.530201in}{1.623964in}}%
\pgfpathcurveto{\pgfqpoint{7.536712in}{1.617452in}}{\pgfqpoint{7.545545in}{1.613794in}}{\pgfqpoint{7.554753in}{1.613794in}}%
\pgfpathlineto{\pgfqpoint{7.554753in}{1.613794in}}%
\pgfpathclose%
\pgfusepath{stroke,fill}%
\end{pgfscope}%
\begin{pgfscope}%
\pgfpathrectangle{\pgfqpoint{1.374500in}{0.082500in}}{\pgfqpoint{2.419000in}{2.419000in}}%
\pgfusepath{clip}%
\pgfsetbuttcap%
\pgfsetroundjoin%
\definecolor{currentfill}{rgb}{0.247059,0.564706,0.854902}%
\pgfsetfillcolor{currentfill}%
\pgfsetfillopacity{0.427124}%
\pgfsetlinewidth{1.003750pt}%
\definecolor{currentstroke}{rgb}{0.247059,0.564706,0.854902}%
\pgfsetstrokecolor{currentstroke}%
\pgfsetstrokeopacity{0.427124}%
\pgfsetdash{}{0pt}%
\pgfpathmoveto{\pgfqpoint{5.753884in}{1.493740in}}%
\pgfpathcurveto{\pgfqpoint{5.763092in}{1.493740in}}{\pgfqpoint{5.771925in}{1.497398in}}{\pgfqpoint{5.778436in}{1.503909in}}%
\pgfpathcurveto{\pgfqpoint{5.784948in}{1.510421in}}{\pgfqpoint{5.788606in}{1.519253in}}{\pgfqpoint{5.788606in}{1.528462in}}%
\pgfpathcurveto{\pgfqpoint{5.788606in}{1.537670in}}{\pgfqpoint{5.784948in}{1.546503in}}{\pgfqpoint{5.778436in}{1.553014in}}%
\pgfpathcurveto{\pgfqpoint{5.771925in}{1.559525in}}{\pgfqpoint{5.763092in}{1.563184in}}{\pgfqpoint{5.753884in}{1.563184in}}%
\pgfpathcurveto{\pgfqpoint{5.744676in}{1.563184in}}{\pgfqpoint{5.735843in}{1.559525in}}{\pgfqpoint{5.729332in}{1.553014in}}%
\pgfpathcurveto{\pgfqpoint{5.722820in}{1.546503in}}{\pgfqpoint{5.719162in}{1.537670in}}{\pgfqpoint{5.719162in}{1.528462in}}%
\pgfpathcurveto{\pgfqpoint{5.719162in}{1.519253in}}{\pgfqpoint{5.722820in}{1.510421in}}{\pgfqpoint{5.729332in}{1.503909in}}%
\pgfpathcurveto{\pgfqpoint{5.735843in}{1.497398in}}{\pgfqpoint{5.744676in}{1.493740in}}{\pgfqpoint{5.753884in}{1.493740in}}%
\pgfpathlineto{\pgfqpoint{5.753884in}{1.493740in}}%
\pgfpathclose%
\pgfusepath{stroke,fill}%
\end{pgfscope}%
\begin{pgfscope}%
\pgfpathrectangle{\pgfqpoint{1.374500in}{0.082500in}}{\pgfqpoint{2.419000in}{2.419000in}}%
\pgfusepath{clip}%
\pgfsetbuttcap%
\pgfsetroundjoin%
\definecolor{currentfill}{rgb}{0.247059,0.564706,0.854902}%
\pgfsetfillcolor{currentfill}%
\pgfsetfillopacity{0.427124}%
\pgfsetlinewidth{1.003750pt}%
\definecolor{currentstroke}{rgb}{0.247059,0.564706,0.854902}%
\pgfsetstrokecolor{currentstroke}%
\pgfsetstrokeopacity{0.427124}%
\pgfsetdash{}{0pt}%
\pgfpathmoveto{\pgfqpoint{10.388759in}{1.493740in}}%
\pgfpathcurveto{\pgfqpoint{10.397967in}{1.493740in}}{\pgfqpoint{10.406800in}{1.497398in}}{\pgfqpoint{10.413311in}{1.503909in}}%
\pgfpathcurveto{\pgfqpoint{10.419822in}{1.510421in}}{\pgfqpoint{10.423481in}{1.519253in}}{\pgfqpoint{10.423481in}{1.528462in}}%
\pgfpathcurveto{\pgfqpoint{10.423481in}{1.537670in}}{\pgfqpoint{10.419822in}{1.546503in}}{\pgfqpoint{10.413311in}{1.553014in}}%
\pgfpathcurveto{\pgfqpoint{10.406800in}{1.559525in}}{\pgfqpoint{10.397967in}{1.563184in}}{\pgfqpoint{10.388759in}{1.563184in}}%
\pgfpathcurveto{\pgfqpoint{10.379550in}{1.563184in}}{\pgfqpoint{10.370718in}{1.559525in}}{\pgfqpoint{10.364206in}{1.553014in}}%
\pgfpathcurveto{\pgfqpoint{10.357695in}{1.546503in}}{\pgfqpoint{10.354037in}{1.537670in}}{\pgfqpoint{10.354037in}{1.528462in}}%
\pgfpathcurveto{\pgfqpoint{10.354037in}{1.519253in}}{\pgfqpoint{10.357695in}{1.510421in}}{\pgfqpoint{10.364206in}{1.503909in}}%
\pgfpathcurveto{\pgfqpoint{10.370718in}{1.497398in}}{\pgfqpoint{10.379550in}{1.493740in}}{\pgfqpoint{10.388759in}{1.493740in}}%
\pgfpathlineto{\pgfqpoint{10.388759in}{1.493740in}}%
\pgfpathclose%
\pgfusepath{stroke,fill}%
\end{pgfscope}%
\begin{pgfscope}%
\pgfpathrectangle{\pgfqpoint{1.374500in}{0.082500in}}{\pgfqpoint{2.419000in}{2.419000in}}%
\pgfusepath{clip}%
\pgfsetbuttcap%
\pgfsetroundjoin%
\definecolor{currentfill}{rgb}{0.247059,0.564706,0.854902}%
\pgfsetfillcolor{currentfill}%
\pgfsetfillopacity{0.427124}%
\pgfsetlinewidth{1.003750pt}%
\definecolor{currentstroke}{rgb}{0.247059,0.564706,0.854902}%
\pgfsetstrokecolor{currentstroke}%
\pgfsetstrokeopacity{0.427124}%
\pgfsetdash{}{0pt}%
\pgfpathmoveto{\pgfqpoint{1.119009in}{1.493740in}}%
\pgfpathcurveto{\pgfqpoint{1.128218in}{1.493740in}}{\pgfqpoint{1.137050in}{1.497398in}}{\pgfqpoint{1.143562in}{1.503909in}}%
\pgfpathcurveto{\pgfqpoint{1.150073in}{1.510421in}}{\pgfqpoint{1.153731in}{1.519253in}}{\pgfqpoint{1.153731in}{1.528462in}}%
\pgfpathcurveto{\pgfqpoint{1.153731in}{1.537670in}}{\pgfqpoint{1.150073in}{1.546503in}}{\pgfqpoint{1.143562in}{1.553014in}}%
\pgfpathcurveto{\pgfqpoint{1.137050in}{1.559525in}}{\pgfqpoint{1.128218in}{1.563184in}}{\pgfqpoint{1.119009in}{1.563184in}}%
\pgfpathcurveto{\pgfqpoint{1.109801in}{1.563184in}}{\pgfqpoint{1.100968in}{1.559525in}}{\pgfqpoint{1.094457in}{1.553014in}}%
\pgfpathcurveto{\pgfqpoint{1.087946in}{1.546503in}}{\pgfqpoint{1.084287in}{1.537670in}}{\pgfqpoint{1.084287in}{1.528462in}}%
\pgfpathcurveto{\pgfqpoint{1.084287in}{1.519253in}}{\pgfqpoint{1.087946in}{1.510421in}}{\pgfqpoint{1.094457in}{1.503909in}}%
\pgfpathcurveto{\pgfqpoint{1.100968in}{1.497398in}}{\pgfqpoint{1.109801in}{1.493740in}}{\pgfqpoint{1.119009in}{1.493740in}}%
\pgfpathlineto{\pgfqpoint{1.119009in}{1.493740in}}%
\pgfpathclose%
\pgfusepath{stroke,fill}%
\end{pgfscope}%
\begin{pgfscope}%
\pgfpathrectangle{\pgfqpoint{1.374500in}{0.082500in}}{\pgfqpoint{2.419000in}{2.419000in}}%
\pgfusepath{clip}%
\pgfsetbuttcap%
\pgfsetroundjoin%
\definecolor{currentfill}{rgb}{0.247059,0.564706,0.854902}%
\pgfsetfillcolor{currentfill}%
\pgfsetfillopacity{0.433189}%
\pgfsetlinewidth{1.003750pt}%
\definecolor{currentstroke}{rgb}{0.247059,0.564706,0.854902}%
\pgfsetstrokecolor{currentstroke}%
\pgfsetstrokeopacity{0.433189}%
\pgfsetdash{}{0pt}%
\pgfpathmoveto{\pgfqpoint{-0.821028in}{1.369628in}}%
\pgfpathcurveto{\pgfqpoint{-0.811820in}{1.369628in}}{\pgfqpoint{-0.802987in}{1.373287in}}{\pgfqpoint{-0.796476in}{1.379798in}}%
\pgfpathcurveto{\pgfqpoint{-0.789965in}{1.386309in}}{\pgfqpoint{-0.786306in}{1.395142in}}{\pgfqpoint{-0.786306in}{1.404350in}}%
\pgfpathcurveto{\pgfqpoint{-0.786306in}{1.413559in}}{\pgfqpoint{-0.789965in}{1.422391in}}{\pgfqpoint{-0.796476in}{1.428903in}}%
\pgfpathcurveto{\pgfqpoint{-0.802987in}{1.435414in}}{\pgfqpoint{-0.811820in}{1.439073in}}{\pgfqpoint{-0.821028in}{1.439073in}}%
\pgfpathcurveto{\pgfqpoint{-0.830237in}{1.439073in}}{\pgfqpoint{-0.839069in}{1.435414in}}{\pgfqpoint{-0.845581in}{1.428903in}}%
\pgfpathcurveto{\pgfqpoint{-0.852092in}{1.422391in}}{\pgfqpoint{-0.855751in}{1.413559in}}{\pgfqpoint{-0.855751in}{1.404350in}}%
\pgfpathcurveto{\pgfqpoint{-0.855751in}{1.395142in}}{\pgfqpoint{-0.852092in}{1.386309in}}{\pgfqpoint{-0.845581in}{1.379798in}}%
\pgfpathcurveto{\pgfqpoint{-0.839069in}{1.373287in}}{\pgfqpoint{-0.830237in}{1.369628in}}{\pgfqpoint{-0.821028in}{1.369628in}}%
\pgfpathlineto{\pgfqpoint{-0.821028in}{1.369628in}}%
\pgfpathclose%
\pgfusepath{stroke,fill}%
\end{pgfscope}%
\begin{pgfscope}%
\pgfpathrectangle{\pgfqpoint{1.374500in}{0.082500in}}{\pgfqpoint{2.419000in}{2.419000in}}%
\pgfusepath{clip}%
\pgfsetbuttcap%
\pgfsetroundjoin%
\definecolor{currentfill}{rgb}{0.247059,0.564706,0.854902}%
\pgfsetfillcolor{currentfill}%
\pgfsetfillopacity{0.433189}%
\pgfsetlinewidth{1.003750pt}%
\definecolor{currentstroke}{rgb}{0.247059,0.564706,0.854902}%
\pgfsetstrokecolor{currentstroke}%
\pgfsetstrokeopacity{0.433189}%
\pgfsetdash{}{0pt}%
\pgfpathmoveto{\pgfqpoint{3.892159in}{1.369628in}}%
\pgfpathcurveto{\pgfqpoint{3.901367in}{1.369628in}}{\pgfqpoint{3.910200in}{1.373287in}}{\pgfqpoint{3.916711in}{1.379798in}}%
\pgfpathcurveto{\pgfqpoint{3.923222in}{1.386309in}}{\pgfqpoint{3.926881in}{1.395142in}}{\pgfqpoint{3.926881in}{1.404350in}}%
\pgfpathcurveto{\pgfqpoint{3.926881in}{1.413559in}}{\pgfqpoint{3.923222in}{1.422391in}}{\pgfqpoint{3.916711in}{1.428903in}}%
\pgfpathcurveto{\pgfqpoint{3.910200in}{1.435414in}}{\pgfqpoint{3.901367in}{1.439073in}}{\pgfqpoint{3.892159in}{1.439073in}}%
\pgfpathcurveto{\pgfqpoint{3.882950in}{1.439073in}}{\pgfqpoint{3.874118in}{1.435414in}}{\pgfqpoint{3.867606in}{1.428903in}}%
\pgfpathcurveto{\pgfqpoint{3.861095in}{1.422391in}}{\pgfqpoint{3.857437in}{1.413559in}}{\pgfqpoint{3.857437in}{1.404350in}}%
\pgfpathcurveto{\pgfqpoint{3.857437in}{1.395142in}}{\pgfqpoint{3.861095in}{1.386309in}}{\pgfqpoint{3.867606in}{1.379798in}}%
\pgfpathcurveto{\pgfqpoint{3.874118in}{1.373287in}}{\pgfqpoint{3.882950in}{1.369628in}}{\pgfqpoint{3.892159in}{1.369628in}}%
\pgfpathlineto{\pgfqpoint{3.892159in}{1.369628in}}%
\pgfpathclose%
\pgfusepath{stroke,fill}%
\end{pgfscope}%
\begin{pgfscope}%
\pgfpathrectangle{\pgfqpoint{1.374500in}{0.082500in}}{\pgfqpoint{2.419000in}{2.419000in}}%
\pgfusepath{clip}%
\pgfsetbuttcap%
\pgfsetroundjoin%
\definecolor{currentfill}{rgb}{0.247059,0.564706,0.854902}%
\pgfsetfillcolor{currentfill}%
\pgfsetfillopacity{0.433189}%
\pgfsetlinewidth{1.003750pt}%
\definecolor{currentstroke}{rgb}{0.247059,0.564706,0.854902}%
\pgfsetstrokecolor{currentstroke}%
\pgfsetstrokeopacity{0.433189}%
\pgfsetdash{}{0pt}%
\pgfpathmoveto{\pgfqpoint{8.605346in}{1.369628in}}%
\pgfpathcurveto{\pgfqpoint{8.614554in}{1.369628in}}{\pgfqpoint{8.623387in}{1.373287in}}{\pgfqpoint{8.629898in}{1.379798in}}%
\pgfpathcurveto{\pgfqpoint{8.636409in}{1.386309in}}{\pgfqpoint{8.640068in}{1.395142in}}{\pgfqpoint{8.640068in}{1.404350in}}%
\pgfpathcurveto{\pgfqpoint{8.640068in}{1.413559in}}{\pgfqpoint{8.636409in}{1.422391in}}{\pgfqpoint{8.629898in}{1.428903in}}%
\pgfpathcurveto{\pgfqpoint{8.623387in}{1.435414in}}{\pgfqpoint{8.614554in}{1.439073in}}{\pgfqpoint{8.605346in}{1.439073in}}%
\pgfpathcurveto{\pgfqpoint{8.596137in}{1.439073in}}{\pgfqpoint{8.587305in}{1.435414in}}{\pgfqpoint{8.580793in}{1.428903in}}%
\pgfpathcurveto{\pgfqpoint{8.574282in}{1.422391in}}{\pgfqpoint{8.570624in}{1.413559in}}{\pgfqpoint{8.570624in}{1.404350in}}%
\pgfpathcurveto{\pgfqpoint{8.570624in}{1.395142in}}{\pgfqpoint{8.574282in}{1.386309in}}{\pgfqpoint{8.580793in}{1.379798in}}%
\pgfpathcurveto{\pgfqpoint{8.587305in}{1.373287in}}{\pgfqpoint{8.596137in}{1.369628in}}{\pgfqpoint{8.605346in}{1.369628in}}%
\pgfpathlineto{\pgfqpoint{8.605346in}{1.369628in}}%
\pgfpathclose%
\pgfusepath{stroke,fill}%
\end{pgfscope}%
\begin{pgfscope}%
\pgfpathrectangle{\pgfqpoint{1.374500in}{0.082500in}}{\pgfqpoint{2.419000in}{2.419000in}}%
\pgfusepath{clip}%
\pgfsetbuttcap%
\pgfsetroundjoin%
\definecolor{currentfill}{rgb}{0.247059,0.564706,0.854902}%
\pgfsetfillcolor{currentfill}%
\pgfsetfillopacity{0.439463}%
\pgfsetlinewidth{1.003750pt}%
\definecolor{currentstroke}{rgb}{0.247059,0.564706,0.854902}%
\pgfsetstrokecolor{currentstroke}%
\pgfsetstrokeopacity{0.439463}%
\pgfsetdash{}{0pt}%
\pgfpathmoveto{\pgfqpoint{11.554822in}{1.241251in}}%
\pgfpathcurveto{\pgfqpoint{11.564030in}{1.241251in}}{\pgfqpoint{11.572863in}{1.244909in}}{\pgfqpoint{11.579374in}{1.251421in}}%
\pgfpathcurveto{\pgfqpoint{11.585886in}{1.257932in}}{\pgfqpoint{11.589544in}{1.266764in}}{\pgfqpoint{11.589544in}{1.275973in}}%
\pgfpathcurveto{\pgfqpoint{11.589544in}{1.285181in}}{\pgfqpoint{11.585886in}{1.294014in}}{\pgfqpoint{11.579374in}{1.300525in}}%
\pgfpathcurveto{\pgfqpoint{11.572863in}{1.307037in}}{\pgfqpoint{11.564030in}{1.310695in}}{\pgfqpoint{11.554822in}{1.310695in}}%
\pgfpathcurveto{\pgfqpoint{11.545614in}{1.310695in}}{\pgfqpoint{11.536781in}{1.307037in}}{\pgfqpoint{11.530270in}{1.300525in}}%
\pgfpathcurveto{\pgfqpoint{11.523758in}{1.294014in}}{\pgfqpoint{11.520100in}{1.285181in}}{\pgfqpoint{11.520100in}{1.275973in}}%
\pgfpathcurveto{\pgfqpoint{11.520100in}{1.266764in}}{\pgfqpoint{11.523758in}{1.257932in}}{\pgfqpoint{11.530270in}{1.251421in}}%
\pgfpathcurveto{\pgfqpoint{11.536781in}{1.244909in}}{\pgfqpoint{11.545614in}{1.241251in}}{\pgfqpoint{11.554822in}{1.241251in}}%
\pgfpathlineto{\pgfqpoint{11.554822in}{1.241251in}}%
\pgfpathclose%
\pgfusepath{stroke,fill}%
\end{pgfscope}%
\begin{pgfscope}%
\pgfpathrectangle{\pgfqpoint{1.374500in}{0.082500in}}{\pgfqpoint{2.419000in}{2.419000in}}%
\pgfusepath{clip}%
\pgfsetbuttcap%
\pgfsetroundjoin%
\definecolor{currentfill}{rgb}{0.247059,0.564706,0.854902}%
\pgfsetfillcolor{currentfill}%
\pgfsetfillopacity{0.439463}%
\pgfsetlinewidth{1.003750pt}%
\definecolor{currentstroke}{rgb}{0.247059,0.564706,0.854902}%
\pgfsetstrokecolor{currentstroke}%
\pgfsetstrokeopacity{0.439463}%
\pgfsetdash{}{0pt}%
\pgfpathmoveto{\pgfqpoint{1.966440in}{1.241251in}}%
\pgfpathcurveto{\pgfqpoint{1.975648in}{1.241251in}}{\pgfqpoint{1.984481in}{1.244909in}}{\pgfqpoint{1.990992in}{1.251421in}}%
\pgfpathcurveto{\pgfqpoint{1.997503in}{1.257932in}}{\pgfqpoint{2.001162in}{1.266764in}}{\pgfqpoint{2.001162in}{1.275973in}}%
\pgfpathcurveto{\pgfqpoint{2.001162in}{1.285181in}}{\pgfqpoint{1.997503in}{1.294014in}}{\pgfqpoint{1.990992in}{1.300525in}}%
\pgfpathcurveto{\pgfqpoint{1.984481in}{1.307037in}}{\pgfqpoint{1.975648in}{1.310695in}}{\pgfqpoint{1.966440in}{1.310695in}}%
\pgfpathcurveto{\pgfqpoint{1.957231in}{1.310695in}}{\pgfqpoint{1.948399in}{1.307037in}}{\pgfqpoint{1.941887in}{1.300525in}}%
\pgfpathcurveto{\pgfqpoint{1.935376in}{1.294014in}}{\pgfqpoint{1.931717in}{1.285181in}}{\pgfqpoint{1.931717in}{1.275973in}}%
\pgfpathcurveto{\pgfqpoint{1.931717in}{1.266764in}}{\pgfqpoint{1.935376in}{1.257932in}}{\pgfqpoint{1.941887in}{1.251421in}}%
\pgfpathcurveto{\pgfqpoint{1.948399in}{1.244909in}}{\pgfqpoint{1.957231in}{1.241251in}}{\pgfqpoint{1.966440in}{1.241251in}}%
\pgfpathlineto{\pgfqpoint{1.966440in}{1.241251in}}%
\pgfpathclose%
\pgfusepath{stroke,fill}%
\end{pgfscope}%
\begin{pgfscope}%
\pgfpathrectangle{\pgfqpoint{1.374500in}{0.082500in}}{\pgfqpoint{2.419000in}{2.419000in}}%
\pgfusepath{clip}%
\pgfsetbuttcap%
\pgfsetroundjoin%
\definecolor{currentfill}{rgb}{0.247059,0.564706,0.854902}%
\pgfsetfillcolor{currentfill}%
\pgfsetfillopacity{0.439463}%
\pgfsetlinewidth{1.003750pt}%
\definecolor{currentstroke}{rgb}{0.247059,0.564706,0.854902}%
\pgfsetstrokecolor{currentstroke}%
\pgfsetstrokeopacity{0.439463}%
\pgfsetdash{}{0pt}%
\pgfpathmoveto{\pgfqpoint{6.760631in}{1.241251in}}%
\pgfpathcurveto{\pgfqpoint{6.769839in}{1.241251in}}{\pgfqpoint{6.778672in}{1.244909in}}{\pgfqpoint{6.785183in}{1.251421in}}%
\pgfpathcurveto{\pgfqpoint{6.791695in}{1.257932in}}{\pgfqpoint{6.795353in}{1.266764in}}{\pgfqpoint{6.795353in}{1.275973in}}%
\pgfpathcurveto{\pgfqpoint{6.795353in}{1.285181in}}{\pgfqpoint{6.791695in}{1.294014in}}{\pgfqpoint{6.785183in}{1.300525in}}%
\pgfpathcurveto{\pgfqpoint{6.778672in}{1.307037in}}{\pgfqpoint{6.769839in}{1.310695in}}{\pgfqpoint{6.760631in}{1.310695in}}%
\pgfpathcurveto{\pgfqpoint{6.751422in}{1.310695in}}{\pgfqpoint{6.742590in}{1.307037in}}{\pgfqpoint{6.736079in}{1.300525in}}%
\pgfpathcurveto{\pgfqpoint{6.729567in}{1.294014in}}{\pgfqpoint{6.725909in}{1.285181in}}{\pgfqpoint{6.725909in}{1.275973in}}%
\pgfpathcurveto{\pgfqpoint{6.725909in}{1.266764in}}{\pgfqpoint{6.729567in}{1.257932in}}{\pgfqpoint{6.736079in}{1.251421in}}%
\pgfpathcurveto{\pgfqpoint{6.742590in}{1.244909in}}{\pgfqpoint{6.751422in}{1.241251in}}{\pgfqpoint{6.760631in}{1.241251in}}%
\pgfpathlineto{\pgfqpoint{6.760631in}{1.241251in}}%
\pgfpathclose%
\pgfusepath{stroke,fill}%
\end{pgfscope}%
\begin{pgfscope}%
\pgfpathrectangle{\pgfqpoint{1.374500in}{0.082500in}}{\pgfqpoint{2.419000in}{2.419000in}}%
\pgfusepath{clip}%
\pgfsetbuttcap%
\pgfsetroundjoin%
\definecolor{currentfill}{rgb}{0.247059,0.564706,0.854902}%
\pgfsetfillcolor{currentfill}%
\pgfsetfillopacity{0.445956}%
\pgfsetlinewidth{1.003750pt}%
\definecolor{currentstroke}{rgb}{0.247059,0.564706,0.854902}%
\pgfsetstrokecolor{currentstroke}%
\pgfsetstrokeopacity{0.445956}%
\pgfsetdash{}{0pt}%
\pgfpathmoveto{\pgfqpoint{9.729426in}{1.108383in}}%
\pgfpathcurveto{\pgfqpoint{9.738635in}{1.108383in}}{\pgfqpoint{9.747467in}{1.112042in}}{\pgfqpoint{9.753979in}{1.118553in}}%
\pgfpathcurveto{\pgfqpoint{9.760490in}{1.125065in}}{\pgfqpoint{9.764149in}{1.133897in}}{\pgfqpoint{9.764149in}{1.143105in}}%
\pgfpathcurveto{\pgfqpoint{9.764149in}{1.152314in}}{\pgfqpoint{9.760490in}{1.161146in}}{\pgfqpoint{9.753979in}{1.167658in}}%
\pgfpathcurveto{\pgfqpoint{9.747467in}{1.174169in}}{\pgfqpoint{9.738635in}{1.177828in}}{\pgfqpoint{9.729426in}{1.177828in}}%
\pgfpathcurveto{\pgfqpoint{9.720218in}{1.177828in}}{\pgfqpoint{9.711385in}{1.174169in}}{\pgfqpoint{9.704874in}{1.167658in}}%
\pgfpathcurveto{\pgfqpoint{9.698363in}{1.161146in}}{\pgfqpoint{9.694704in}{1.152314in}}{\pgfqpoint{9.694704in}{1.143105in}}%
\pgfpathcurveto{\pgfqpoint{9.694704in}{1.133897in}}{\pgfqpoint{9.698363in}{1.125065in}}{\pgfqpoint{9.704874in}{1.118553in}}%
\pgfpathcurveto{\pgfqpoint{9.711385in}{1.112042in}}{\pgfqpoint{9.720218in}{1.108383in}}{\pgfqpoint{9.729426in}{1.108383in}}%
\pgfpathlineto{\pgfqpoint{9.729426in}{1.108383in}}%
\pgfpathclose%
\pgfusepath{stroke,fill}%
\end{pgfscope}%
\begin{pgfscope}%
\pgfpathrectangle{\pgfqpoint{1.374500in}{0.082500in}}{\pgfqpoint{2.419000in}{2.419000in}}%
\pgfusepath{clip}%
\pgfsetbuttcap%
\pgfsetroundjoin%
\definecolor{currentfill}{rgb}{0.247059,0.564706,0.854902}%
\pgfsetfillcolor{currentfill}%
\pgfsetfillopacity{0.445956}%
\pgfsetlinewidth{1.003750pt}%
\definecolor{currentstroke}{rgb}{0.247059,0.564706,0.854902}%
\pgfsetstrokecolor{currentstroke}%
\pgfsetstrokeopacity{0.445956}%
\pgfsetdash{}{0pt}%
\pgfpathmoveto{\pgfqpoint{-0.026630in}{1.108383in}}%
\pgfpathcurveto{\pgfqpoint{-0.017422in}{1.108383in}}{\pgfqpoint{-0.008589in}{1.112042in}}{\pgfqpoint{-0.002078in}{1.118553in}}%
\pgfpathcurveto{\pgfqpoint{0.004433in}{1.125065in}}{\pgfqpoint{0.008092in}{1.133897in}}{\pgfqpoint{0.008092in}{1.143105in}}%
\pgfpathcurveto{\pgfqpoint{0.008092in}{1.152314in}}{\pgfqpoint{0.004433in}{1.161146in}}{\pgfqpoint{-0.002078in}{1.167658in}}%
\pgfpathcurveto{\pgfqpoint{-0.008589in}{1.174169in}}{\pgfqpoint{-0.017422in}{1.177828in}}{\pgfqpoint{-0.026630in}{1.177828in}}%
\pgfpathcurveto{\pgfqpoint{-0.035839in}{1.177828in}}{\pgfqpoint{-0.044671in}{1.174169in}}{\pgfqpoint{-0.051183in}{1.167658in}}%
\pgfpathcurveto{\pgfqpoint{-0.057694in}{1.161146in}}{\pgfqpoint{-0.061353in}{1.152314in}}{\pgfqpoint{-0.061353in}{1.143105in}}%
\pgfpathcurveto{\pgfqpoint{-0.061353in}{1.133897in}}{\pgfqpoint{-0.057694in}{1.125065in}}{\pgfqpoint{-0.051183in}{1.118553in}}%
\pgfpathcurveto{\pgfqpoint{-0.044671in}{1.112042in}}{\pgfqpoint{-0.035839in}{1.108383in}}{\pgfqpoint{-0.026630in}{1.108383in}}%
\pgfpathlineto{\pgfqpoint{-0.026630in}{1.108383in}}%
\pgfpathclose%
\pgfusepath{stroke,fill}%
\end{pgfscope}%
\begin{pgfscope}%
\pgfpathrectangle{\pgfqpoint{1.374500in}{0.082500in}}{\pgfqpoint{2.419000in}{2.419000in}}%
\pgfusepath{clip}%
\pgfsetbuttcap%
\pgfsetroundjoin%
\definecolor{currentfill}{rgb}{0.247059,0.564706,0.854902}%
\pgfsetfillcolor{currentfill}%
\pgfsetfillopacity{0.445956}%
\pgfsetlinewidth{1.003750pt}%
\definecolor{currentstroke}{rgb}{0.247059,0.564706,0.854902}%
\pgfsetstrokecolor{currentstroke}%
\pgfsetstrokeopacity{0.445956}%
\pgfsetdash{}{0pt}%
\pgfpathmoveto{\pgfqpoint{4.851398in}{1.108383in}}%
\pgfpathcurveto{\pgfqpoint{4.860606in}{1.108383in}}{\pgfqpoint{4.869439in}{1.112042in}}{\pgfqpoint{4.875950in}{1.118553in}}%
\pgfpathcurveto{\pgfqpoint{4.882462in}{1.125065in}}{\pgfqpoint{4.886120in}{1.133897in}}{\pgfqpoint{4.886120in}{1.143105in}}%
\pgfpathcurveto{\pgfqpoint{4.886120in}{1.152314in}}{\pgfqpoint{4.882462in}{1.161146in}}{\pgfqpoint{4.875950in}{1.167658in}}%
\pgfpathcurveto{\pgfqpoint{4.869439in}{1.174169in}}{\pgfqpoint{4.860606in}{1.177828in}}{\pgfqpoint{4.851398in}{1.177828in}}%
\pgfpathcurveto{\pgfqpoint{4.842190in}{1.177828in}}{\pgfqpoint{4.833357in}{1.174169in}}{\pgfqpoint{4.826846in}{1.167658in}}%
\pgfpathcurveto{\pgfqpoint{4.820334in}{1.161146in}}{\pgfqpoint{4.816676in}{1.152314in}}{\pgfqpoint{4.816676in}{1.143105in}}%
\pgfpathcurveto{\pgfqpoint{4.816676in}{1.133897in}}{\pgfqpoint{4.820334in}{1.125065in}}{\pgfqpoint{4.826846in}{1.118553in}}%
\pgfpathcurveto{\pgfqpoint{4.833357in}{1.112042in}}{\pgfqpoint{4.842190in}{1.108383in}}{\pgfqpoint{4.851398in}{1.108383in}}%
\pgfpathlineto{\pgfqpoint{4.851398in}{1.108383in}}%
\pgfpathclose%
\pgfusepath{stroke,fill}%
\end{pgfscope}%
\begin{pgfscope}%
\pgfpathrectangle{\pgfqpoint{1.374500in}{0.082500in}}{\pgfqpoint{2.419000in}{2.419000in}}%
\pgfusepath{clip}%
\pgfsetbuttcap%
\pgfsetroundjoin%
\definecolor{currentfill}{rgb}{0.247059,0.564706,0.854902}%
\pgfsetfillcolor{currentfill}%
\pgfsetfillopacity{0.452680}%
\pgfsetlinewidth{1.003750pt}%
\definecolor{currentstroke}{rgb}{0.247059,0.564706,0.854902}%
\pgfsetstrokecolor{currentstroke}%
\pgfsetstrokeopacity{0.452680}%
\pgfsetdash{}{0pt}%
\pgfpathmoveto{\pgfqpoint{-2.090648in}{0.970786in}}%
\pgfpathcurveto{\pgfqpoint{-2.081439in}{0.970786in}}{\pgfqpoint{-2.072607in}{0.974445in}}{\pgfqpoint{-2.066095in}{0.980956in}}%
\pgfpathcurveto{\pgfqpoint{-2.059584in}{0.987467in}}{\pgfqpoint{-2.055926in}{0.996300in}}{\pgfqpoint{-2.055926in}{1.005508in}}%
\pgfpathcurveto{\pgfqpoint{-2.055926in}{1.014717in}}{\pgfqpoint{-2.059584in}{1.023549in}}{\pgfqpoint{-2.066095in}{1.030061in}}%
\pgfpathcurveto{\pgfqpoint{-2.072607in}{1.036572in}}{\pgfqpoint{-2.081439in}{1.040231in}}{\pgfqpoint{-2.090648in}{1.040231in}}%
\pgfpathcurveto{\pgfqpoint{-2.099856in}{1.040231in}}{\pgfqpoint{-2.108689in}{1.036572in}}{\pgfqpoint{-2.115200in}{1.030061in}}%
\pgfpathcurveto{\pgfqpoint{-2.121711in}{1.023549in}}{\pgfqpoint{-2.125370in}{1.014717in}}{\pgfqpoint{-2.125370in}{1.005508in}}%
\pgfpathcurveto{\pgfqpoint{-2.125370in}{0.996300in}}{\pgfqpoint{-2.121711in}{0.987467in}}{\pgfqpoint{-2.115200in}{0.980956in}}%
\pgfpathcurveto{\pgfqpoint{-2.108689in}{0.974445in}}{\pgfqpoint{-2.099856in}{0.970786in}}{\pgfqpoint{-2.090648in}{0.970786in}}%
\pgfpathlineto{\pgfqpoint{-2.090648in}{0.970786in}}%
\pgfpathclose%
\pgfusepath{stroke,fill}%
\end{pgfscope}%
\begin{pgfscope}%
\pgfpathrectangle{\pgfqpoint{1.374500in}{0.082500in}}{\pgfqpoint{2.419000in}{2.419000in}}%
\pgfusepath{clip}%
\pgfsetbuttcap%
\pgfsetroundjoin%
\definecolor{currentfill}{rgb}{0.247059,0.564706,0.854902}%
\pgfsetfillcolor{currentfill}%
\pgfsetfillopacity{0.452680}%
\pgfsetlinewidth{1.003750pt}%
\definecolor{currentstroke}{rgb}{0.247059,0.564706,0.854902}%
\pgfsetstrokecolor{currentstroke}%
\pgfsetstrokeopacity{0.452680}%
\pgfsetdash{}{0pt}%
\pgfpathmoveto{\pgfqpoint{2.874202in}{0.970786in}}%
\pgfpathcurveto{\pgfqpoint{2.883411in}{0.970786in}}{\pgfqpoint{2.892243in}{0.974445in}}{\pgfqpoint{2.898754in}{0.980956in}}%
\pgfpathcurveto{\pgfqpoint{2.905266in}{0.987467in}}{\pgfqpoint{2.908924in}{0.996300in}}{\pgfqpoint{2.908924in}{1.005508in}}%
\pgfpathcurveto{\pgfqpoint{2.908924in}{1.014717in}}{\pgfqpoint{2.905266in}{1.023549in}}{\pgfqpoint{2.898754in}{1.030061in}}%
\pgfpathcurveto{\pgfqpoint{2.892243in}{1.036572in}}{\pgfqpoint{2.883411in}{1.040231in}}{\pgfqpoint{2.874202in}{1.040231in}}%
\pgfpathcurveto{\pgfqpoint{2.864994in}{1.040231in}}{\pgfqpoint{2.856161in}{1.036572in}}{\pgfqpoint{2.849650in}{1.030061in}}%
\pgfpathcurveto{\pgfqpoint{2.843138in}{1.023549in}}{\pgfqpoint{2.839480in}{1.014717in}}{\pgfqpoint{2.839480in}{1.005508in}}%
\pgfpathcurveto{\pgfqpoint{2.839480in}{0.996300in}}{\pgfqpoint{2.843138in}{0.987467in}}{\pgfqpoint{2.849650in}{0.980956in}}%
\pgfpathcurveto{\pgfqpoint{2.856161in}{0.974445in}}{\pgfqpoint{2.864994in}{0.970786in}}{\pgfqpoint{2.874202in}{0.970786in}}%
\pgfpathlineto{\pgfqpoint{2.874202in}{0.970786in}}%
\pgfpathclose%
\pgfusepath{stroke,fill}%
\end{pgfscope}%
\begin{pgfscope}%
\pgfpathrectangle{\pgfqpoint{1.374500in}{0.082500in}}{\pgfqpoint{2.419000in}{2.419000in}}%
\pgfusepath{clip}%
\pgfsetbuttcap%
\pgfsetroundjoin%
\definecolor{currentfill}{rgb}{0.247059,0.564706,0.854902}%
\pgfsetfillcolor{currentfill}%
\pgfsetfillopacity{0.452680}%
\pgfsetlinewidth{1.003750pt}%
\definecolor{currentstroke}{rgb}{0.247059,0.564706,0.854902}%
\pgfsetstrokecolor{currentstroke}%
\pgfsetstrokeopacity{0.452680}%
\pgfsetdash{}{0pt}%
\pgfpathmoveto{\pgfqpoint{7.839052in}{0.970786in}}%
\pgfpathcurveto{\pgfqpoint{7.848261in}{0.970786in}}{\pgfqpoint{7.857093in}{0.974445in}}{\pgfqpoint{7.863604in}{0.980956in}}%
\pgfpathcurveto{\pgfqpoint{7.870116in}{0.987467in}}{\pgfqpoint{7.873774in}{0.996300in}}{\pgfqpoint{7.873774in}{1.005508in}}%
\pgfpathcurveto{\pgfqpoint{7.873774in}{1.014717in}}{\pgfqpoint{7.870116in}{1.023549in}}{\pgfqpoint{7.863604in}{1.030061in}}%
\pgfpathcurveto{\pgfqpoint{7.857093in}{1.036572in}}{\pgfqpoint{7.848261in}{1.040231in}}{\pgfqpoint{7.839052in}{1.040231in}}%
\pgfpathcurveto{\pgfqpoint{7.829844in}{1.040231in}}{\pgfqpoint{7.821011in}{1.036572in}}{\pgfqpoint{7.814500in}{1.030061in}}%
\pgfpathcurveto{\pgfqpoint{7.807988in}{1.023549in}}{\pgfqpoint{7.804330in}{1.014717in}}{\pgfqpoint{7.804330in}{1.005508in}}%
\pgfpathcurveto{\pgfqpoint{7.804330in}{0.996300in}}{\pgfqpoint{7.807988in}{0.987467in}}{\pgfqpoint{7.814500in}{0.980956in}}%
\pgfpathcurveto{\pgfqpoint{7.821011in}{0.974445in}}{\pgfqpoint{7.829844in}{0.970786in}}{\pgfqpoint{7.839052in}{0.970786in}}%
\pgfpathlineto{\pgfqpoint{7.839052in}{0.970786in}}%
\pgfpathclose%
\pgfusepath{stroke,fill}%
\end{pgfscope}%
\begin{pgfscope}%
\pgfpathrectangle{\pgfqpoint{1.374500in}{0.082500in}}{\pgfqpoint{2.419000in}{2.419000in}}%
\pgfusepath{clip}%
\pgfsetbuttcap%
\pgfsetroundjoin%
\definecolor{currentfill}{rgb}{0.247059,0.564706,0.854902}%
\pgfsetfillcolor{currentfill}%
\pgfsetfillopacity{0.459648}%
\pgfsetlinewidth{1.003750pt}%
\definecolor{currentstroke}{rgb}{0.247059,0.564706,0.854902}%
\pgfsetstrokecolor{currentstroke}%
\pgfsetstrokeopacity{0.459648}%
\pgfsetdash{}{0pt}%
\pgfpathmoveto{\pgfqpoint{10.934985in}{0.828202in}}%
\pgfpathcurveto{\pgfqpoint{10.944193in}{0.828202in}}{\pgfqpoint{10.953026in}{0.831861in}}{\pgfqpoint{10.959537in}{0.838372in}}%
\pgfpathcurveto{\pgfqpoint{10.966049in}{0.844883in}}{\pgfqpoint{10.969707in}{0.853716in}}{\pgfqpoint{10.969707in}{0.862924in}}%
\pgfpathcurveto{\pgfqpoint{10.969707in}{0.872133in}}{\pgfqpoint{10.966049in}{0.880965in}}{\pgfqpoint{10.959537in}{0.887477in}}%
\pgfpathcurveto{\pgfqpoint{10.953026in}{0.893988in}}{\pgfqpoint{10.944193in}{0.897647in}}{\pgfqpoint{10.934985in}{0.897647in}}%
\pgfpathcurveto{\pgfqpoint{10.925776in}{0.897647in}}{\pgfqpoint{10.916944in}{0.893988in}}{\pgfqpoint{10.910433in}{0.887477in}}%
\pgfpathcurveto{\pgfqpoint{10.903921in}{0.880965in}}{\pgfqpoint{10.900263in}{0.872133in}}{\pgfqpoint{10.900263in}{0.862924in}}%
\pgfpathcurveto{\pgfqpoint{10.900263in}{0.853716in}}{\pgfqpoint{10.903921in}{0.844883in}}{\pgfqpoint{10.910433in}{0.838372in}}%
\pgfpathcurveto{\pgfqpoint{10.916944in}{0.831861in}}{\pgfqpoint{10.925776in}{0.828202in}}{\pgfqpoint{10.934985in}{0.828202in}}%
\pgfpathlineto{\pgfqpoint{10.934985in}{0.828202in}}%
\pgfpathclose%
\pgfusepath{stroke,fill}%
\end{pgfscope}%
\begin{pgfscope}%
\pgfpathrectangle{\pgfqpoint{1.374500in}{0.082500in}}{\pgfqpoint{2.419000in}{2.419000in}}%
\pgfusepath{clip}%
\pgfsetbuttcap%
\pgfsetroundjoin%
\definecolor{currentfill}{rgb}{0.247059,0.564706,0.854902}%
\pgfsetfillcolor{currentfill}%
\pgfsetfillopacity{0.459648}%
\pgfsetlinewidth{1.003750pt}%
\definecolor{currentstroke}{rgb}{0.247059,0.564706,0.854902}%
\pgfsetstrokecolor{currentstroke}%
\pgfsetstrokeopacity{0.459648}%
\pgfsetdash{}{0pt}%
\pgfpathmoveto{\pgfqpoint{0.825349in}{0.828202in}}%
\pgfpathcurveto{\pgfqpoint{0.834557in}{0.828202in}}{\pgfqpoint{0.843390in}{0.831861in}}{\pgfqpoint{0.849901in}{0.838372in}}%
\pgfpathcurveto{\pgfqpoint{0.856412in}{0.844883in}}{\pgfqpoint{0.860071in}{0.853716in}}{\pgfqpoint{0.860071in}{0.862924in}}%
\pgfpathcurveto{\pgfqpoint{0.860071in}{0.872133in}}{\pgfqpoint{0.856412in}{0.880965in}}{\pgfqpoint{0.849901in}{0.887477in}}%
\pgfpathcurveto{\pgfqpoint{0.843390in}{0.893988in}}{\pgfqpoint{0.834557in}{0.897647in}}{\pgfqpoint{0.825349in}{0.897647in}}%
\pgfpathcurveto{\pgfqpoint{0.816140in}{0.897647in}}{\pgfqpoint{0.807308in}{0.893988in}}{\pgfqpoint{0.800796in}{0.887477in}}%
\pgfpathcurveto{\pgfqpoint{0.794285in}{0.880965in}}{\pgfqpoint{0.790626in}{0.872133in}}{\pgfqpoint{0.790626in}{0.862924in}}%
\pgfpathcurveto{\pgfqpoint{0.790626in}{0.853716in}}{\pgfqpoint{0.794285in}{0.844883in}}{\pgfqpoint{0.800796in}{0.838372in}}%
\pgfpathcurveto{\pgfqpoint{0.807308in}{0.831861in}}{\pgfqpoint{0.816140in}{0.828202in}}{\pgfqpoint{0.825349in}{0.828202in}}%
\pgfpathlineto{\pgfqpoint{0.825349in}{0.828202in}}%
\pgfpathclose%
\pgfusepath{stroke,fill}%
\end{pgfscope}%
\begin{pgfscope}%
\pgfpathrectangle{\pgfqpoint{1.374500in}{0.082500in}}{\pgfqpoint{2.419000in}{2.419000in}}%
\pgfusepath{clip}%
\pgfsetbuttcap%
\pgfsetroundjoin%
\definecolor{currentfill}{rgb}{0.247059,0.564706,0.854902}%
\pgfsetfillcolor{currentfill}%
\pgfsetfillopacity{0.459648}%
\pgfsetlinewidth{1.003750pt}%
\definecolor{currentstroke}{rgb}{0.247059,0.564706,0.854902}%
\pgfsetstrokecolor{currentstroke}%
\pgfsetstrokeopacity{0.459648}%
\pgfsetdash{}{0pt}%
\pgfpathmoveto{\pgfqpoint{5.880167in}{0.828202in}}%
\pgfpathcurveto{\pgfqpoint{5.889375in}{0.828202in}}{\pgfqpoint{5.898208in}{0.831861in}}{\pgfqpoint{5.904719in}{0.838372in}}%
\pgfpathcurveto{\pgfqpoint{5.911230in}{0.844883in}}{\pgfqpoint{5.914889in}{0.853716in}}{\pgfqpoint{5.914889in}{0.862924in}}%
\pgfpathcurveto{\pgfqpoint{5.914889in}{0.872133in}}{\pgfqpoint{5.911230in}{0.880965in}}{\pgfqpoint{5.904719in}{0.887477in}}%
\pgfpathcurveto{\pgfqpoint{5.898208in}{0.893988in}}{\pgfqpoint{5.889375in}{0.897647in}}{\pgfqpoint{5.880167in}{0.897647in}}%
\pgfpathcurveto{\pgfqpoint{5.870958in}{0.897647in}}{\pgfqpoint{5.862126in}{0.893988in}}{\pgfqpoint{5.855614in}{0.887477in}}%
\pgfpathcurveto{\pgfqpoint{5.849103in}{0.880965in}}{\pgfqpoint{5.845445in}{0.872133in}}{\pgfqpoint{5.845445in}{0.862924in}}%
\pgfpathcurveto{\pgfqpoint{5.845445in}{0.853716in}}{\pgfqpoint{5.849103in}{0.844883in}}{\pgfqpoint{5.855614in}{0.838372in}}%
\pgfpathcurveto{\pgfqpoint{5.862126in}{0.831861in}}{\pgfqpoint{5.870958in}{0.828202in}}{\pgfqpoint{5.880167in}{0.828202in}}%
\pgfpathlineto{\pgfqpoint{5.880167in}{0.828202in}}%
\pgfpathclose%
\pgfusepath{stroke,fill}%
\end{pgfscope}%
\begin{pgfscope}%
\pgfpathrectangle{\pgfqpoint{1.374500in}{0.082500in}}{\pgfqpoint{2.419000in}{2.419000in}}%
\pgfusepath{clip}%
\pgfsetbuttcap%
\pgfsetroundjoin%
\definecolor{currentfill}{rgb}{0.247059,0.564706,0.854902}%
\pgfsetfillcolor{currentfill}%
\pgfsetfillopacity{0.466874}%
\pgfsetlinewidth{1.003750pt}%
\definecolor{currentstroke}{rgb}{0.247059,0.564706,0.854902}%
\pgfsetstrokecolor{currentstroke}%
\pgfsetstrokeopacity{0.466874}%
\pgfsetdash{}{0pt}%
\pgfpathmoveto{\pgfqpoint{-1.299130in}{0.680355in}}%
\pgfpathcurveto{\pgfqpoint{-1.289921in}{0.680355in}}{\pgfqpoint{-1.281089in}{0.684014in}}{\pgfqpoint{-1.274577in}{0.690525in}}%
\pgfpathcurveto{\pgfqpoint{-1.268066in}{0.697037in}}{\pgfqpoint{-1.264408in}{0.705869in}}{\pgfqpoint{-1.264408in}{0.715078in}}%
\pgfpathcurveto{\pgfqpoint{-1.264408in}{0.724286in}}{\pgfqpoint{-1.268066in}{0.733119in}}{\pgfqpoint{-1.274577in}{0.739630in}}%
\pgfpathcurveto{\pgfqpoint{-1.281089in}{0.746141in}}{\pgfqpoint{-1.289921in}{0.749800in}}{\pgfqpoint{-1.299130in}{0.749800in}}%
\pgfpathcurveto{\pgfqpoint{-1.308338in}{0.749800in}}{\pgfqpoint{-1.317171in}{0.746141in}}{\pgfqpoint{-1.323682in}{0.739630in}}%
\pgfpathcurveto{\pgfqpoint{-1.330193in}{0.733119in}}{\pgfqpoint{-1.333852in}{0.724286in}}{\pgfqpoint{-1.333852in}{0.715078in}}%
\pgfpathcurveto{\pgfqpoint{-1.333852in}{0.705869in}}{\pgfqpoint{-1.330193in}{0.697037in}}{\pgfqpoint{-1.323682in}{0.690525in}}%
\pgfpathcurveto{\pgfqpoint{-1.317171in}{0.684014in}}{\pgfqpoint{-1.308338in}{0.680355in}}{\pgfqpoint{-1.299130in}{0.680355in}}%
\pgfpathlineto{\pgfqpoint{-1.299130in}{0.680355in}}%
\pgfpathclose%
\pgfusepath{stroke,fill}%
\end{pgfscope}%
\begin{pgfscope}%
\pgfpathrectangle{\pgfqpoint{1.374500in}{0.082500in}}{\pgfqpoint{2.419000in}{2.419000in}}%
\pgfusepath{clip}%
\pgfsetbuttcap%
\pgfsetroundjoin%
\definecolor{currentfill}{rgb}{0.247059,0.564706,0.854902}%
\pgfsetfillcolor{currentfill}%
\pgfsetfillopacity{0.466874}%
\pgfsetlinewidth{1.003750pt}%
\definecolor{currentstroke}{rgb}{0.247059,0.564706,0.854902}%
\pgfsetstrokecolor{currentstroke}%
\pgfsetstrokeopacity{0.466874}%
\pgfsetdash{}{0pt}%
\pgfpathmoveto{\pgfqpoint{3.848977in}{0.680355in}}%
\pgfpathcurveto{\pgfqpoint{3.858186in}{0.680355in}}{\pgfqpoint{3.867018in}{0.684014in}}{\pgfqpoint{3.873530in}{0.690525in}}%
\pgfpathcurveto{\pgfqpoint{3.880041in}{0.697037in}}{\pgfqpoint{3.883699in}{0.705869in}}{\pgfqpoint{3.883699in}{0.715078in}}%
\pgfpathcurveto{\pgfqpoint{3.883699in}{0.724286in}}{\pgfqpoint{3.880041in}{0.733119in}}{\pgfqpoint{3.873530in}{0.739630in}}%
\pgfpathcurveto{\pgfqpoint{3.867018in}{0.746141in}}{\pgfqpoint{3.858186in}{0.749800in}}{\pgfqpoint{3.848977in}{0.749800in}}%
\pgfpathcurveto{\pgfqpoint{3.839769in}{0.749800in}}{\pgfqpoint{3.830936in}{0.746141in}}{\pgfqpoint{3.824425in}{0.739630in}}%
\pgfpathcurveto{\pgfqpoint{3.817914in}{0.733119in}}{\pgfqpoint{3.814255in}{0.724286in}}{\pgfqpoint{3.814255in}{0.715078in}}%
\pgfpathcurveto{\pgfqpoint{3.814255in}{0.705869in}}{\pgfqpoint{3.817914in}{0.697037in}}{\pgfqpoint{3.824425in}{0.690525in}}%
\pgfpathcurveto{\pgfqpoint{3.830936in}{0.684014in}}{\pgfqpoint{3.839769in}{0.680355in}}{\pgfqpoint{3.848977in}{0.680355in}}%
\pgfpathlineto{\pgfqpoint{3.848977in}{0.680355in}}%
\pgfpathclose%
\pgfusepath{stroke,fill}%
\end{pgfscope}%
\begin{pgfscope}%
\pgfpathrectangle{\pgfqpoint{1.374500in}{0.082500in}}{\pgfqpoint{2.419000in}{2.419000in}}%
\pgfusepath{clip}%
\pgfsetbuttcap%
\pgfsetroundjoin%
\definecolor{currentfill}{rgb}{0.247059,0.564706,0.854902}%
\pgfsetfillcolor{currentfill}%
\pgfsetfillopacity{0.466874}%
\pgfsetlinewidth{1.003750pt}%
\definecolor{currentstroke}{rgb}{0.247059,0.564706,0.854902}%
\pgfsetstrokecolor{currentstroke}%
\pgfsetstrokeopacity{0.466874}%
\pgfsetdash{}{0pt}%
\pgfpathmoveto{\pgfqpoint{8.997084in}{0.680355in}}%
\pgfpathcurveto{\pgfqpoint{9.006293in}{0.680355in}}{\pgfqpoint{9.015125in}{0.684014in}}{\pgfqpoint{9.021637in}{0.690525in}}%
\pgfpathcurveto{\pgfqpoint{9.028148in}{0.697037in}}{\pgfqpoint{9.031807in}{0.705869in}}{\pgfqpoint{9.031807in}{0.715078in}}%
\pgfpathcurveto{\pgfqpoint{9.031807in}{0.724286in}}{\pgfqpoint{9.028148in}{0.733119in}}{\pgfqpoint{9.021637in}{0.739630in}}%
\pgfpathcurveto{\pgfqpoint{9.015125in}{0.746141in}}{\pgfqpoint{9.006293in}{0.749800in}}{\pgfqpoint{8.997084in}{0.749800in}}%
\pgfpathcurveto{\pgfqpoint{8.987876in}{0.749800in}}{\pgfqpoint{8.979043in}{0.746141in}}{\pgfqpoint{8.972532in}{0.739630in}}%
\pgfpathcurveto{\pgfqpoint{8.966021in}{0.733119in}}{\pgfqpoint{8.962362in}{0.724286in}}{\pgfqpoint{8.962362in}{0.715078in}}%
\pgfpathcurveto{\pgfqpoint{8.962362in}{0.705869in}}{\pgfqpoint{8.966021in}{0.697037in}}{\pgfqpoint{8.972532in}{0.690525in}}%
\pgfpathcurveto{\pgfqpoint{8.979043in}{0.684014in}}{\pgfqpoint{8.987876in}{0.680355in}}{\pgfqpoint{8.997084in}{0.680355in}}%
\pgfpathlineto{\pgfqpoint{8.997084in}{0.680355in}}%
\pgfpathclose%
\pgfusepath{stroke,fill}%
\end{pgfscope}%
\begin{pgfscope}%
\pgfpathrectangle{\pgfqpoint{1.374500in}{0.082500in}}{\pgfqpoint{2.419000in}{2.419000in}}%
\pgfusepath{clip}%
\pgfsetbuttcap%
\pgfsetroundjoin%
\definecolor{currentfill}{rgb}{0.247059,0.564706,0.854902}%
\pgfsetfillcolor{currentfill}%
\pgfsetfillopacity{0.474370}%
\pgfsetlinewidth{1.003750pt}%
\definecolor{currentstroke}{rgb}{0.247059,0.564706,0.854902}%
\pgfsetstrokecolor{currentstroke}%
\pgfsetstrokeopacity{0.474370}%
\pgfsetdash{}{0pt}%
\pgfpathmoveto{\pgfqpoint{1.741405in}{0.526949in}}%
\pgfpathcurveto{\pgfqpoint{1.750613in}{0.526949in}}{\pgfqpoint{1.759446in}{0.530607in}}{\pgfqpoint{1.765957in}{0.537119in}}%
\pgfpathcurveto{\pgfqpoint{1.772469in}{0.543630in}}{\pgfqpoint{1.776127in}{0.552463in}}{\pgfqpoint{1.776127in}{0.561671in}}%
\pgfpathcurveto{\pgfqpoint{1.776127in}{0.570880in}}{\pgfqpoint{1.772469in}{0.579712in}}{\pgfqpoint{1.765957in}{0.586223in}}%
\pgfpathcurveto{\pgfqpoint{1.759446in}{0.592735in}}{\pgfqpoint{1.750613in}{0.596393in}}{\pgfqpoint{1.741405in}{0.596393in}}%
\pgfpathcurveto{\pgfqpoint{1.732197in}{0.596393in}}{\pgfqpoint{1.723364in}{0.592735in}}{\pgfqpoint{1.716853in}{0.586223in}}%
\pgfpathcurveto{\pgfqpoint{1.710341in}{0.579712in}}{\pgfqpoint{1.706683in}{0.570880in}}{\pgfqpoint{1.706683in}{0.561671in}}%
\pgfpathcurveto{\pgfqpoint{1.706683in}{0.552463in}}{\pgfqpoint{1.710341in}{0.543630in}}{\pgfqpoint{1.716853in}{0.537119in}}%
\pgfpathcurveto{\pgfqpoint{1.723364in}{0.530607in}}{\pgfqpoint{1.732197in}{0.526949in}}{\pgfqpoint{1.741405in}{0.526949in}}%
\pgfpathlineto{\pgfqpoint{1.741405in}{0.526949in}}%
\pgfpathclose%
\pgfusepath{stroke,fill}%
\end{pgfscope}%
\begin{pgfscope}%
\pgfpathrectangle{\pgfqpoint{1.374500in}{0.082500in}}{\pgfqpoint{2.419000in}{2.419000in}}%
\pgfusepath{clip}%
\pgfsetbuttcap%
\pgfsetroundjoin%
\definecolor{currentfill}{rgb}{0.247059,0.564706,0.854902}%
\pgfsetfillcolor{currentfill}%
\pgfsetfillopacity{0.474370}%
\pgfsetlinewidth{1.003750pt}%
\definecolor{currentstroke}{rgb}{0.247059,0.564706,0.854902}%
\pgfsetstrokecolor{currentstroke}%
\pgfsetstrokeopacity{0.474370}%
\pgfsetdash{}{0pt}%
\pgfpathmoveto{\pgfqpoint{-3.503499in}{0.526949in}}%
\pgfpathcurveto{\pgfqpoint{-3.494291in}{0.526949in}}{\pgfqpoint{-3.485458in}{0.530607in}}{\pgfqpoint{-3.478947in}{0.537119in}}%
\pgfpathcurveto{\pgfqpoint{-3.472435in}{0.543630in}}{\pgfqpoint{-3.468777in}{0.552463in}}{\pgfqpoint{-3.468777in}{0.561671in}}%
\pgfpathcurveto{\pgfqpoint{-3.468777in}{0.570880in}}{\pgfqpoint{-3.472435in}{0.579712in}}{\pgfqpoint{-3.478947in}{0.586223in}}%
\pgfpathcurveto{\pgfqpoint{-3.485458in}{0.592735in}}{\pgfqpoint{-3.494291in}{0.596393in}}{\pgfqpoint{-3.503499in}{0.596393in}}%
\pgfpathcurveto{\pgfqpoint{-3.512708in}{0.596393in}}{\pgfqpoint{-3.521540in}{0.592735in}}{\pgfqpoint{-3.528051in}{0.586223in}}%
\pgfpathcurveto{\pgfqpoint{-3.534563in}{0.579712in}}{\pgfqpoint{-3.538221in}{0.570880in}}{\pgfqpoint{-3.538221in}{0.561671in}}%
\pgfpathcurveto{\pgfqpoint{-3.538221in}{0.552463in}}{\pgfqpoint{-3.534563in}{0.543630in}}{\pgfqpoint{-3.528051in}{0.537119in}}%
\pgfpathcurveto{\pgfqpoint{-3.521540in}{0.530607in}}{\pgfqpoint{-3.512708in}{0.526949in}}{\pgfqpoint{-3.503499in}{0.526949in}}%
\pgfpathlineto{\pgfqpoint{-3.503499in}{0.526949in}}%
\pgfpathclose%
\pgfusepath{stroke,fill}%
\end{pgfscope}%
\begin{pgfscope}%
\pgfpathrectangle{\pgfqpoint{1.374500in}{0.082500in}}{\pgfqpoint{2.419000in}{2.419000in}}%
\pgfusepath{clip}%
\pgfsetbuttcap%
\pgfsetroundjoin%
\definecolor{currentfill}{rgb}{0.247059,0.564706,0.854902}%
\pgfsetfillcolor{currentfill}%
\pgfsetfillopacity{0.474370}%
\pgfsetlinewidth{1.003750pt}%
\definecolor{currentstroke}{rgb}{0.247059,0.564706,0.854902}%
\pgfsetstrokecolor{currentstroke}%
\pgfsetstrokeopacity{0.474370}%
\pgfsetdash{}{0pt}%
\pgfpathmoveto{\pgfqpoint{6.986309in}{0.526949in}}%
\pgfpathcurveto{\pgfqpoint{6.995518in}{0.526949in}}{\pgfqpoint{7.004350in}{0.530607in}}{\pgfqpoint{7.010861in}{0.537119in}}%
\pgfpathcurveto{\pgfqpoint{7.017373in}{0.543630in}}{\pgfqpoint{7.021031in}{0.552463in}}{\pgfqpoint{7.021031in}{0.561671in}}%
\pgfpathcurveto{\pgfqpoint{7.021031in}{0.570880in}}{\pgfqpoint{7.017373in}{0.579712in}}{\pgfqpoint{7.010861in}{0.586223in}}%
\pgfpathcurveto{\pgfqpoint{7.004350in}{0.592735in}}{\pgfqpoint{6.995518in}{0.596393in}}{\pgfqpoint{6.986309in}{0.596393in}}%
\pgfpathcurveto{\pgfqpoint{6.977101in}{0.596393in}}{\pgfqpoint{6.968268in}{0.592735in}}{\pgfqpoint{6.961757in}{0.586223in}}%
\pgfpathcurveto{\pgfqpoint{6.955245in}{0.579712in}}{\pgfqpoint{6.951587in}{0.570880in}}{\pgfqpoint{6.951587in}{0.561671in}}%
\pgfpathcurveto{\pgfqpoint{6.951587in}{0.552463in}}{\pgfqpoint{6.955245in}{0.543630in}}{\pgfqpoint{6.961757in}{0.537119in}}%
\pgfpathcurveto{\pgfqpoint{6.968268in}{0.530607in}}{\pgfqpoint{6.977101in}{0.526949in}}{\pgfqpoint{6.986309in}{0.526949in}}%
\pgfpathlineto{\pgfqpoint{6.986309in}{0.526949in}}%
\pgfpathclose%
\pgfusepath{stroke,fill}%
\end{pgfscope}%
\begin{pgfscope}%
\pgfpathrectangle{\pgfqpoint{1.374500in}{0.082500in}}{\pgfqpoint{2.419000in}{2.419000in}}%
\pgfusepath{clip}%
\pgfsetbuttcap%
\pgfsetroundjoin%
\definecolor{currentfill}{rgb}{0.247059,0.564706,0.854902}%
\pgfsetfillcolor{currentfill}%
\pgfsetfillopacity{0.482155}%
\pgfsetlinewidth{1.003750pt}%
\definecolor{currentstroke}{rgb}{0.247059,0.564706,0.854902}%
\pgfsetstrokecolor{currentstroke}%
\pgfsetstrokeopacity{0.482155}%
\pgfsetdash{}{0pt}%
\pgfpathmoveto{\pgfqpoint{-0.446941in}{0.367663in}}%
\pgfpathcurveto{\pgfqpoint{-0.437733in}{0.367663in}}{\pgfqpoint{-0.428900in}{0.371321in}}{\pgfqpoint{-0.422389in}{0.377833in}}%
\pgfpathcurveto{\pgfqpoint{-0.415877in}{0.384344in}}{\pgfqpoint{-0.412219in}{0.393177in}}{\pgfqpoint{-0.412219in}{0.402385in}}%
\pgfpathcurveto{\pgfqpoint{-0.412219in}{0.411594in}}{\pgfqpoint{-0.415877in}{0.420426in}}{\pgfqpoint{-0.422389in}{0.426937in}}%
\pgfpathcurveto{\pgfqpoint{-0.428900in}{0.433449in}}{\pgfqpoint{-0.437733in}{0.437107in}}{\pgfqpoint{-0.446941in}{0.437107in}}%
\pgfpathcurveto{\pgfqpoint{-0.456150in}{0.437107in}}{\pgfqpoint{-0.464982in}{0.433449in}}{\pgfqpoint{-0.471493in}{0.426937in}}%
\pgfpathcurveto{\pgfqpoint{-0.478005in}{0.420426in}}{\pgfqpoint{-0.481663in}{0.411594in}}{\pgfqpoint{-0.481663in}{0.402385in}}%
\pgfpathcurveto{\pgfqpoint{-0.481663in}{0.393177in}}{\pgfqpoint{-0.478005in}{0.384344in}}{\pgfqpoint{-0.471493in}{0.377833in}}%
\pgfpathcurveto{\pgfqpoint{-0.464982in}{0.371321in}}{\pgfqpoint{-0.456150in}{0.367663in}}{\pgfqpoint{-0.446941in}{0.367663in}}%
\pgfpathlineto{\pgfqpoint{-0.446941in}{0.367663in}}%
\pgfpathclose%
\pgfusepath{stroke,fill}%
\end{pgfscope}%
\begin{pgfscope}%
\pgfpathrectangle{\pgfqpoint{1.374500in}{0.082500in}}{\pgfqpoint{2.419000in}{2.419000in}}%
\pgfusepath{clip}%
\pgfsetbuttcap%
\pgfsetroundjoin%
\definecolor{currentfill}{rgb}{0.247059,0.564706,0.854902}%
\pgfsetfillcolor{currentfill}%
\pgfsetfillopacity{0.482155}%
\pgfsetlinewidth{1.003750pt}%
\definecolor{currentstroke}{rgb}{0.247059,0.564706,0.854902}%
\pgfsetstrokecolor{currentstroke}%
\pgfsetstrokeopacity{0.482155}%
\pgfsetdash{}{0pt}%
\pgfpathmoveto{\pgfqpoint{10.243881in}{0.367663in}}%
\pgfpathcurveto{\pgfqpoint{10.253089in}{0.367663in}}{\pgfqpoint{10.261922in}{0.371321in}}{\pgfqpoint{10.268433in}{0.377833in}}%
\pgfpathcurveto{\pgfqpoint{10.274945in}{0.384344in}}{\pgfqpoint{10.278603in}{0.393177in}}{\pgfqpoint{10.278603in}{0.402385in}}%
\pgfpathcurveto{\pgfqpoint{10.278603in}{0.411594in}}{\pgfqpoint{10.274945in}{0.420426in}}{\pgfqpoint{10.268433in}{0.426937in}}%
\pgfpathcurveto{\pgfqpoint{10.261922in}{0.433449in}}{\pgfqpoint{10.253089in}{0.437107in}}{\pgfqpoint{10.243881in}{0.437107in}}%
\pgfpathcurveto{\pgfqpoint{10.234672in}{0.437107in}}{\pgfqpoint{10.225840in}{0.433449in}}{\pgfqpoint{10.219329in}{0.426937in}}%
\pgfpathcurveto{\pgfqpoint{10.212817in}{0.420426in}}{\pgfqpoint{10.209159in}{0.411594in}}{\pgfqpoint{10.209159in}{0.402385in}}%
\pgfpathcurveto{\pgfqpoint{10.209159in}{0.393177in}}{\pgfqpoint{10.212817in}{0.384344in}}{\pgfqpoint{10.219329in}{0.377833in}}%
\pgfpathcurveto{\pgfqpoint{10.225840in}{0.371321in}}{\pgfqpoint{10.234672in}{0.367663in}}{\pgfqpoint{10.243881in}{0.367663in}}%
\pgfpathlineto{\pgfqpoint{10.243881in}{0.367663in}}%
\pgfpathclose%
\pgfusepath{stroke,fill}%
\end{pgfscope}%
\begin{pgfscope}%
\pgfpathrectangle{\pgfqpoint{1.374500in}{0.082500in}}{\pgfqpoint{2.419000in}{2.419000in}}%
\pgfusepath{clip}%
\pgfsetbuttcap%
\pgfsetroundjoin%
\definecolor{currentfill}{rgb}{0.247059,0.564706,0.854902}%
\pgfsetfillcolor{currentfill}%
\pgfsetfillopacity{0.482155}%
\pgfsetlinewidth{1.003750pt}%
\definecolor{currentstroke}{rgb}{0.247059,0.564706,0.854902}%
\pgfsetstrokecolor{currentstroke}%
\pgfsetstrokeopacity{0.482155}%
\pgfsetdash{}{0pt}%
\pgfpathmoveto{\pgfqpoint{4.898470in}{0.367663in}}%
\pgfpathcurveto{\pgfqpoint{4.907678in}{0.367663in}}{\pgfqpoint{4.916511in}{0.371321in}}{\pgfqpoint{4.923022in}{0.377833in}}%
\pgfpathcurveto{\pgfqpoint{4.929534in}{0.384344in}}{\pgfqpoint{4.933192in}{0.393177in}}{\pgfqpoint{4.933192in}{0.402385in}}%
\pgfpathcurveto{\pgfqpoint{4.933192in}{0.411594in}}{\pgfqpoint{4.929534in}{0.420426in}}{\pgfqpoint{4.923022in}{0.426937in}}%
\pgfpathcurveto{\pgfqpoint{4.916511in}{0.433449in}}{\pgfqpoint{4.907678in}{0.437107in}}{\pgfqpoint{4.898470in}{0.437107in}}%
\pgfpathcurveto{\pgfqpoint{4.889261in}{0.437107in}}{\pgfqpoint{4.880429in}{0.433449in}}{\pgfqpoint{4.873918in}{0.426937in}}%
\pgfpathcurveto{\pgfqpoint{4.867406in}{0.420426in}}{\pgfqpoint{4.863748in}{0.411594in}}{\pgfqpoint{4.863748in}{0.402385in}}%
\pgfpathcurveto{\pgfqpoint{4.863748in}{0.393177in}}{\pgfqpoint{4.867406in}{0.384344in}}{\pgfqpoint{4.873918in}{0.377833in}}%
\pgfpathcurveto{\pgfqpoint{4.880429in}{0.371321in}}{\pgfqpoint{4.889261in}{0.367663in}}{\pgfqpoint{4.898470in}{0.367663in}}%
\pgfpathlineto{\pgfqpoint{4.898470in}{0.367663in}}%
\pgfpathclose%
\pgfusepath{stroke,fill}%
\end{pgfscope}%
\begin{pgfscope}%
\pgfpathrectangle{\pgfqpoint{1.374500in}{0.082500in}}{\pgfqpoint{2.419000in}{2.419000in}}%
\pgfusepath{clip}%
\pgfsetbuttcap%
\pgfsetroundjoin%
\definecolor{currentfill}{rgb}{0.247059,0.564706,0.854902}%
\pgfsetfillcolor{currentfill}%
\pgfsetfillopacity{0.490243}%
\pgfsetlinewidth{1.003750pt}%
\definecolor{currentstroke}{rgb}{0.247059,0.564706,0.854902}%
\pgfsetstrokecolor{currentstroke}%
\pgfsetstrokeopacity{0.490243}%
\pgfsetdash{}{0pt}%
\pgfpathmoveto{\pgfqpoint{-2.720795in}{0.202153in}}%
\pgfpathcurveto{\pgfqpoint{-2.711587in}{0.202153in}}{\pgfqpoint{-2.702754in}{0.205811in}}{\pgfqpoint{-2.696243in}{0.212323in}}%
\pgfpathcurveto{\pgfqpoint{-2.689732in}{0.218834in}}{\pgfqpoint{-2.686073in}{0.227667in}}{\pgfqpoint{-2.686073in}{0.236875in}}%
\pgfpathcurveto{\pgfqpoint{-2.686073in}{0.246084in}}{\pgfqpoint{-2.689732in}{0.254916in}}{\pgfqpoint{-2.696243in}{0.261427in}}%
\pgfpathcurveto{\pgfqpoint{-2.702754in}{0.267939in}}{\pgfqpoint{-2.711587in}{0.271597in}}{\pgfqpoint{-2.720795in}{0.271597in}}%
\pgfpathcurveto{\pgfqpoint{-2.730004in}{0.271597in}}{\pgfqpoint{-2.738836in}{0.267939in}}{\pgfqpoint{-2.745348in}{0.261427in}}%
\pgfpathcurveto{\pgfqpoint{-2.751859in}{0.254916in}}{\pgfqpoint{-2.755518in}{0.246084in}}{\pgfqpoint{-2.755518in}{0.236875in}}%
\pgfpathcurveto{\pgfqpoint{-2.755518in}{0.227667in}}{\pgfqpoint{-2.751859in}{0.218834in}}{\pgfqpoint{-2.745348in}{0.212323in}}%
\pgfpathcurveto{\pgfqpoint{-2.738836in}{0.205811in}}{\pgfqpoint{-2.730004in}{0.202153in}}{\pgfqpoint{-2.720795in}{0.202153in}}%
\pgfpathlineto{\pgfqpoint{-2.720795in}{0.202153in}}%
\pgfpathclose%
\pgfusepath{stroke,fill}%
\end{pgfscope}%
\begin{pgfscope}%
\pgfpathrectangle{\pgfqpoint{1.374500in}{0.082500in}}{\pgfqpoint{2.419000in}{2.419000in}}%
\pgfusepath{clip}%
\pgfsetbuttcap%
\pgfsetroundjoin%
\definecolor{currentfill}{rgb}{0.247059,0.564706,0.854902}%
\pgfsetfillcolor{currentfill}%
\pgfsetfillopacity{0.490243}%
\pgfsetlinewidth{1.003750pt}%
\definecolor{currentstroke}{rgb}{0.247059,0.564706,0.854902}%
\pgfsetstrokecolor{currentstroke}%
\pgfsetstrokeopacity{0.490243}%
\pgfsetdash{}{0pt}%
\pgfpathmoveto{\pgfqpoint{2.729050in}{0.202153in}}%
\pgfpathcurveto{\pgfqpoint{2.738258in}{0.202153in}}{\pgfqpoint{2.747091in}{0.205811in}}{\pgfqpoint{2.753602in}{0.212323in}}%
\pgfpathcurveto{\pgfqpoint{2.760113in}{0.218834in}}{\pgfqpoint{2.763772in}{0.227667in}}{\pgfqpoint{2.763772in}{0.236875in}}%
\pgfpathcurveto{\pgfqpoint{2.763772in}{0.246084in}}{\pgfqpoint{2.760113in}{0.254916in}}{\pgfqpoint{2.753602in}{0.261427in}}%
\pgfpathcurveto{\pgfqpoint{2.747091in}{0.267939in}}{\pgfqpoint{2.738258in}{0.271597in}}{\pgfqpoint{2.729050in}{0.271597in}}%
\pgfpathcurveto{\pgfqpoint{2.719841in}{0.271597in}}{\pgfqpoint{2.711009in}{0.267939in}}{\pgfqpoint{2.704498in}{0.261427in}}%
\pgfpathcurveto{\pgfqpoint{2.697986in}{0.254916in}}{\pgfqpoint{2.694328in}{0.246084in}}{\pgfqpoint{2.694328in}{0.236875in}}%
\pgfpathcurveto{\pgfqpoint{2.694328in}{0.227667in}}{\pgfqpoint{2.697986in}{0.218834in}}{\pgfqpoint{2.704498in}{0.212323in}}%
\pgfpathcurveto{\pgfqpoint{2.711009in}{0.205811in}}{\pgfqpoint{2.719841in}{0.202153in}}{\pgfqpoint{2.729050in}{0.202153in}}%
\pgfpathlineto{\pgfqpoint{2.729050in}{0.202153in}}%
\pgfpathclose%
\pgfusepath{stroke,fill}%
\end{pgfscope}%
\begin{pgfscope}%
\pgfpathrectangle{\pgfqpoint{1.374500in}{0.082500in}}{\pgfqpoint{2.419000in}{2.419000in}}%
\pgfusepath{clip}%
\pgfsetbuttcap%
\pgfsetroundjoin%
\definecolor{currentfill}{rgb}{0.247059,0.564706,0.854902}%
\pgfsetfillcolor{currentfill}%
\pgfsetfillopacity{0.490243}%
\pgfsetlinewidth{1.003750pt}%
\definecolor{currentstroke}{rgb}{0.247059,0.564706,0.854902}%
\pgfsetstrokecolor{currentstroke}%
\pgfsetstrokeopacity{0.490243}%
\pgfsetdash{}{0pt}%
\pgfpathmoveto{\pgfqpoint{8.178895in}{0.202153in}}%
\pgfpathcurveto{\pgfqpoint{8.188103in}{0.202153in}}{\pgfqpoint{8.196936in}{0.205811in}}{\pgfqpoint{8.203447in}{0.212323in}}%
\pgfpathcurveto{\pgfqpoint{8.209959in}{0.218834in}}{\pgfqpoint{8.213617in}{0.227667in}}{\pgfqpoint{8.213617in}{0.236875in}}%
\pgfpathcurveto{\pgfqpoint{8.213617in}{0.246084in}}{\pgfqpoint{8.209959in}{0.254916in}}{\pgfqpoint{8.203447in}{0.261427in}}%
\pgfpathcurveto{\pgfqpoint{8.196936in}{0.267939in}}{\pgfqpoint{8.188103in}{0.271597in}}{\pgfqpoint{8.178895in}{0.271597in}}%
\pgfpathcurveto{\pgfqpoint{8.169687in}{0.271597in}}{\pgfqpoint{8.160854in}{0.267939in}}{\pgfqpoint{8.154343in}{0.261427in}}%
\pgfpathcurveto{\pgfqpoint{8.147831in}{0.254916in}}{\pgfqpoint{8.144173in}{0.246084in}}{\pgfqpoint{8.144173in}{0.236875in}}%
\pgfpathcurveto{\pgfqpoint{8.144173in}{0.227667in}}{\pgfqpoint{8.147831in}{0.218834in}}{\pgfqpoint{8.154343in}{0.212323in}}%
\pgfpathcurveto{\pgfqpoint{8.160854in}{0.205811in}}{\pgfqpoint{8.169687in}{0.202153in}}{\pgfqpoint{8.178895in}{0.202153in}}%
\pgfpathlineto{\pgfqpoint{8.178895in}{0.202153in}}%
\pgfpathclose%
\pgfusepath{stroke,fill}%
\end{pgfscope}%
\begin{pgfscope}%
\pgfpathrectangle{\pgfqpoint{1.374500in}{0.082500in}}{\pgfqpoint{2.419000in}{2.419000in}}%
\pgfusepath{clip}%
\pgfsetbuttcap%
\pgfsetroundjoin%
\definecolor{currentfill}{rgb}{0.247059,0.564706,0.854902}%
\pgfsetfillcolor{currentfill}%
\pgfsetfillopacity{0.498654}%
\pgfsetlinewidth{1.003750pt}%
\definecolor{currentstroke}{rgb}{0.247059,0.564706,0.854902}%
\pgfsetstrokecolor{currentstroke}%
\pgfsetstrokeopacity{0.498654}%
\pgfsetdash{}{0pt}%
\pgfpathmoveto{\pgfqpoint{0.473172in}{0.030047in}}%
\pgfpathcurveto{\pgfqpoint{0.482380in}{0.030047in}}{\pgfqpoint{0.491213in}{0.033705in}}{\pgfqpoint{0.497724in}{0.040217in}}%
\pgfpathcurveto{\pgfqpoint{0.504236in}{0.046728in}}{\pgfqpoint{0.507894in}{0.055561in}}{\pgfqpoint{0.507894in}{0.064769in}}%
\pgfpathcurveto{\pgfqpoint{0.507894in}{0.073978in}}{\pgfqpoint{0.504236in}{0.082810in}}{\pgfqpoint{0.497724in}{0.089321in}}%
\pgfpathcurveto{\pgfqpoint{0.491213in}{0.095833in}}{\pgfqpoint{0.482380in}{0.099491in}}{\pgfqpoint{0.473172in}{0.099491in}}%
\pgfpathcurveto{\pgfqpoint{0.463964in}{0.099491in}}{\pgfqpoint{0.455131in}{0.095833in}}{\pgfqpoint{0.448620in}{0.089321in}}%
\pgfpathcurveto{\pgfqpoint{0.442108in}{0.082810in}}{\pgfqpoint{0.438450in}{0.073978in}}{\pgfqpoint{0.438450in}{0.064769in}}%
\pgfpathcurveto{\pgfqpoint{0.438450in}{0.055561in}}{\pgfqpoint{0.442108in}{0.046728in}}{\pgfqpoint{0.448620in}{0.040217in}}%
\pgfpathcurveto{\pgfqpoint{0.455131in}{0.033705in}}{\pgfqpoint{0.463964in}{0.030047in}}{\pgfqpoint{0.473172in}{0.030047in}}%
\pgfpathlineto{\pgfqpoint{0.473172in}{0.030047in}}%
\pgfpathclose%
\pgfusepath{stroke,fill}%
\end{pgfscope}%
\begin{pgfscope}%
\pgfpathrectangle{\pgfqpoint{1.374500in}{0.082500in}}{\pgfqpoint{2.419000in}{2.419000in}}%
\pgfusepath{clip}%
\pgfsetbuttcap%
\pgfsetroundjoin%
\definecolor{currentfill}{rgb}{0.247059,0.564706,0.854902}%
\pgfsetfillcolor{currentfill}%
\pgfsetfillopacity{0.498654}%
\pgfsetlinewidth{1.003750pt}%
\definecolor{currentstroke}{rgb}{0.247059,0.564706,0.854902}%
\pgfsetstrokecolor{currentstroke}%
\pgfsetstrokeopacity{0.498654}%
\pgfsetdash{}{0pt}%
\pgfpathmoveto{\pgfqpoint{6.031613in}{0.030047in}}%
\pgfpathcurveto{\pgfqpoint{6.040822in}{0.030047in}}{\pgfqpoint{6.049654in}{0.033705in}}{\pgfqpoint{6.056166in}{0.040217in}}%
\pgfpathcurveto{\pgfqpoint{6.062677in}{0.046728in}}{\pgfqpoint{6.066335in}{0.055561in}}{\pgfqpoint{6.066335in}{0.064769in}}%
\pgfpathcurveto{\pgfqpoint{6.066335in}{0.073978in}}{\pgfqpoint{6.062677in}{0.082810in}}{\pgfqpoint{6.056166in}{0.089321in}}%
\pgfpathcurveto{\pgfqpoint{6.049654in}{0.095833in}}{\pgfqpoint{6.040822in}{0.099491in}}{\pgfqpoint{6.031613in}{0.099491in}}%
\pgfpathcurveto{\pgfqpoint{6.022405in}{0.099491in}}{\pgfqpoint{6.013572in}{0.095833in}}{\pgfqpoint{6.007061in}{0.089321in}}%
\pgfpathcurveto{\pgfqpoint{6.000550in}{0.082810in}}{\pgfqpoint{5.996891in}{0.073978in}}{\pgfqpoint{5.996891in}{0.064769in}}%
\pgfpathcurveto{\pgfqpoint{5.996891in}{0.055561in}}{\pgfqpoint{6.000550in}{0.046728in}}{\pgfqpoint{6.007061in}{0.040217in}}%
\pgfpathcurveto{\pgfqpoint{6.013572in}{0.033705in}}{\pgfqpoint{6.022405in}{0.030047in}}{\pgfqpoint{6.031613in}{0.030047in}}%
\pgfpathlineto{\pgfqpoint{6.031613in}{0.030047in}}%
\pgfpathclose%
\pgfusepath{stroke,fill}%
\end{pgfscope}%
\begin{pgfscope}%
\pgfpathrectangle{\pgfqpoint{1.374500in}{0.082500in}}{\pgfqpoint{2.419000in}{2.419000in}}%
\pgfusepath{clip}%
\pgfsetbuttcap%
\pgfsetroundjoin%
\definecolor{currentfill}{rgb}{0.247059,0.564706,0.854902}%
\pgfsetfillcolor{currentfill}%
\pgfsetfillopacity{0.498654}%
\pgfsetlinewidth{1.003750pt}%
\definecolor{currentstroke}{rgb}{0.247059,0.564706,0.854902}%
\pgfsetstrokecolor{currentstroke}%
\pgfsetstrokeopacity{0.498654}%
\pgfsetdash{}{0pt}%
\pgfpathmoveto{\pgfqpoint{11.590055in}{0.030047in}}%
\pgfpathcurveto{\pgfqpoint{11.599263in}{0.030047in}}{\pgfqpoint{11.608096in}{0.033705in}}{\pgfqpoint{11.614607in}{0.040217in}}%
\pgfpathcurveto{\pgfqpoint{11.621118in}{0.046728in}}{\pgfqpoint{11.624777in}{0.055561in}}{\pgfqpoint{11.624777in}{0.064769in}}%
\pgfpathcurveto{\pgfqpoint{11.624777in}{0.073978in}}{\pgfqpoint{11.621118in}{0.082810in}}{\pgfqpoint{11.614607in}{0.089321in}}%
\pgfpathcurveto{\pgfqpoint{11.608096in}{0.095833in}}{\pgfqpoint{11.599263in}{0.099491in}}{\pgfqpoint{11.590055in}{0.099491in}}%
\pgfpathcurveto{\pgfqpoint{11.580846in}{0.099491in}}{\pgfqpoint{11.572014in}{0.095833in}}{\pgfqpoint{11.565502in}{0.089321in}}%
\pgfpathcurveto{\pgfqpoint{11.558991in}{0.082810in}}{\pgfqpoint{11.555332in}{0.073978in}}{\pgfqpoint{11.555332in}{0.064769in}}%
\pgfpathcurveto{\pgfqpoint{11.555332in}{0.055561in}}{\pgfqpoint{11.558991in}{0.046728in}}{\pgfqpoint{11.565502in}{0.040217in}}%
\pgfpathcurveto{\pgfqpoint{11.572014in}{0.033705in}}{\pgfqpoint{11.580846in}{0.030047in}}{\pgfqpoint{11.590055in}{0.030047in}}%
\pgfpathlineto{\pgfqpoint{11.590055in}{0.030047in}}%
\pgfpathclose%
\pgfusepath{stroke,fill}%
\end{pgfscope}%
\begin{pgfscope}%
\pgfpathrectangle{\pgfqpoint{1.374500in}{0.082500in}}{\pgfqpoint{2.419000in}{2.419000in}}%
\pgfusepath{clip}%
\pgfsetbuttcap%
\pgfsetroundjoin%
\definecolor{currentfill}{rgb}{0.247059,0.564706,0.854902}%
\pgfsetfillcolor{currentfill}%
\pgfsetfillopacity{0.507407}%
\pgfsetlinewidth{1.003750pt}%
\definecolor{currentstroke}{rgb}{0.247059,0.564706,0.854902}%
\pgfsetstrokecolor{currentstroke}%
\pgfsetstrokeopacity{0.507407}%
\pgfsetdash{}{0pt}%
\pgfpathmoveto{\pgfqpoint{-1.874437in}{-0.149057in}}%
\pgfpathcurveto{\pgfqpoint{-1.865229in}{-0.149057in}}{\pgfqpoint{-1.856396in}{-0.145399in}}{\pgfqpoint{-1.849885in}{-0.138888in}}%
\pgfpathcurveto{\pgfqpoint{-1.843373in}{-0.132376in}}{\pgfqpoint{-1.839715in}{-0.123544in}}{\pgfqpoint{-1.839715in}{-0.114335in}}%
\pgfpathcurveto{\pgfqpoint{-1.839715in}{-0.105127in}}{\pgfqpoint{-1.843373in}{-0.096294in}}{\pgfqpoint{-1.849885in}{-0.089783in}}%
\pgfpathcurveto{\pgfqpoint{-1.856396in}{-0.083272in}}{\pgfqpoint{-1.865229in}{-0.079613in}}{\pgfqpoint{-1.874437in}{-0.079613in}}%
\pgfpathcurveto{\pgfqpoint{-1.883645in}{-0.079613in}}{\pgfqpoint{-1.892478in}{-0.083272in}}{\pgfqpoint{-1.898989in}{-0.089783in}}%
\pgfpathcurveto{\pgfqpoint{-1.905501in}{-0.096294in}}{\pgfqpoint{-1.909159in}{-0.105127in}}{\pgfqpoint{-1.909159in}{-0.114335in}}%
\pgfpathcurveto{\pgfqpoint{-1.909159in}{-0.123544in}}{\pgfqpoint{-1.905501in}{-0.132376in}}{\pgfqpoint{-1.898989in}{-0.138888in}}%
\pgfpathcurveto{\pgfqpoint{-1.892478in}{-0.145399in}}{\pgfqpoint{-1.883645in}{-0.149057in}}{\pgfqpoint{-1.874437in}{-0.149057in}}%
\pgfpathlineto{\pgfqpoint{-1.874437in}{-0.149057in}}%
\pgfpathclose%
\pgfusepath{stroke,fill}%
\end{pgfscope}%
\begin{pgfscope}%
\pgfpathrectangle{\pgfqpoint{1.374500in}{0.082500in}}{\pgfqpoint{2.419000in}{2.419000in}}%
\pgfusepath{clip}%
\pgfsetbuttcap%
\pgfsetroundjoin%
\definecolor{currentfill}{rgb}{0.247059,0.564706,0.854902}%
\pgfsetfillcolor{currentfill}%
\pgfsetfillopacity{0.507407}%
\pgfsetlinewidth{1.003750pt}%
\definecolor{currentstroke}{rgb}{0.247059,0.564706,0.854902}%
\pgfsetstrokecolor{currentstroke}%
\pgfsetstrokeopacity{0.507407}%
\pgfsetdash{}{0pt}%
\pgfpathmoveto{\pgfqpoint{3.797016in}{-0.149057in}}%
\pgfpathcurveto{\pgfqpoint{3.806225in}{-0.149057in}}{\pgfqpoint{3.815057in}{-0.145399in}}{\pgfqpoint{3.821569in}{-0.138888in}}%
\pgfpathcurveto{\pgfqpoint{3.828080in}{-0.132376in}}{\pgfqpoint{3.831738in}{-0.123544in}}{\pgfqpoint{3.831738in}{-0.114335in}}%
\pgfpathcurveto{\pgfqpoint{3.831738in}{-0.105127in}}{\pgfqpoint{3.828080in}{-0.096294in}}{\pgfqpoint{3.821569in}{-0.089783in}}%
\pgfpathcurveto{\pgfqpoint{3.815057in}{-0.083272in}}{\pgfqpoint{3.806225in}{-0.079613in}}{\pgfqpoint{3.797016in}{-0.079613in}}%
\pgfpathcurveto{\pgfqpoint{3.787808in}{-0.079613in}}{\pgfqpoint{3.778975in}{-0.083272in}}{\pgfqpoint{3.772464in}{-0.089783in}}%
\pgfpathcurveto{\pgfqpoint{3.765953in}{-0.096294in}}{\pgfqpoint{3.762294in}{-0.105127in}}{\pgfqpoint{3.762294in}{-0.114335in}}%
\pgfpathcurveto{\pgfqpoint{3.762294in}{-0.123544in}}{\pgfqpoint{3.765953in}{-0.132376in}}{\pgfqpoint{3.772464in}{-0.138888in}}%
\pgfpathcurveto{\pgfqpoint{3.778975in}{-0.145399in}}{\pgfqpoint{3.787808in}{-0.149057in}}{\pgfqpoint{3.797016in}{-0.149057in}}%
\pgfpathlineto{\pgfqpoint{3.797016in}{-0.149057in}}%
\pgfpathclose%
\pgfusepath{stroke,fill}%
\end{pgfscope}%
\begin{pgfscope}%
\pgfpathrectangle{\pgfqpoint{1.374500in}{0.082500in}}{\pgfqpoint{2.419000in}{2.419000in}}%
\pgfusepath{clip}%
\pgfsetbuttcap%
\pgfsetroundjoin%
\definecolor{currentfill}{rgb}{0.247059,0.564706,0.854902}%
\pgfsetfillcolor{currentfill}%
\pgfsetfillopacity{0.507407}%
\pgfsetlinewidth{1.003750pt}%
\definecolor{currentstroke}{rgb}{0.247059,0.564706,0.854902}%
\pgfsetstrokecolor{currentstroke}%
\pgfsetstrokeopacity{0.507407}%
\pgfsetdash{}{0pt}%
\pgfpathmoveto{\pgfqpoint{9.468470in}{-0.149057in}}%
\pgfpathcurveto{\pgfqpoint{9.477678in}{-0.149057in}}{\pgfqpoint{9.486510in}{-0.145399in}}{\pgfqpoint{9.493022in}{-0.138888in}}%
\pgfpathcurveto{\pgfqpoint{9.499533in}{-0.132376in}}{\pgfqpoint{9.503192in}{-0.123544in}}{\pgfqpoint{9.503192in}{-0.114335in}}%
\pgfpathcurveto{\pgfqpoint{9.503192in}{-0.105127in}}{\pgfqpoint{9.499533in}{-0.096294in}}{\pgfqpoint{9.493022in}{-0.089783in}}%
\pgfpathcurveto{\pgfqpoint{9.486510in}{-0.083272in}}{\pgfqpoint{9.477678in}{-0.079613in}}{\pgfqpoint{9.468470in}{-0.079613in}}%
\pgfpathcurveto{\pgfqpoint{9.459261in}{-0.079613in}}{\pgfqpoint{9.450429in}{-0.083272in}}{\pgfqpoint{9.443917in}{-0.089783in}}%
\pgfpathcurveto{\pgfqpoint{9.437406in}{-0.096294in}}{\pgfqpoint{9.433747in}{-0.105127in}}{\pgfqpoint{9.433747in}{-0.114335in}}%
\pgfpathcurveto{\pgfqpoint{9.433747in}{-0.123544in}}{\pgfqpoint{9.437406in}{-0.132376in}}{\pgfqpoint{9.443917in}{-0.138888in}}%
\pgfpathcurveto{\pgfqpoint{9.450429in}{-0.145399in}}{\pgfqpoint{9.459261in}{-0.149057in}}{\pgfqpoint{9.468470in}{-0.149057in}}%
\pgfpathlineto{\pgfqpoint{9.468470in}{-0.149057in}}%
\pgfpathclose%
\pgfusepath{stroke,fill}%
\end{pgfscope}%
\begin{pgfscope}%
\pgfpathrectangle{\pgfqpoint{1.374500in}{0.082500in}}{\pgfqpoint{2.419000in}{2.419000in}}%
\pgfusepath{clip}%
\pgfsetbuttcap%
\pgfsetroundjoin%
\definecolor{currentfill}{rgb}{0.247059,0.564706,0.854902}%
\pgfsetfillcolor{currentfill}%
\pgfsetfillopacity{0.516523}%
\pgfsetlinewidth{1.003750pt}%
\definecolor{currentstroke}{rgb}{0.247059,0.564706,0.854902}%
\pgfsetstrokecolor{currentstroke}%
\pgfsetstrokeopacity{0.516523}%
\pgfsetdash{}{0pt}%
\pgfpathmoveto{\pgfqpoint{1.469668in}{-0.335596in}}%
\pgfpathcurveto{\pgfqpoint{1.478876in}{-0.335596in}}{\pgfqpoint{1.487709in}{-0.331937in}}{\pgfqpoint{1.494220in}{-0.325426in}}%
\pgfpathcurveto{\pgfqpoint{1.500731in}{-0.318915in}}{\pgfqpoint{1.504390in}{-0.310082in}}{\pgfqpoint{1.504390in}{-0.300874in}}%
\pgfpathcurveto{\pgfqpoint{1.504390in}{-0.291665in}}{\pgfqpoint{1.500731in}{-0.282833in}}{\pgfqpoint{1.494220in}{-0.276321in}}%
\pgfpathcurveto{\pgfqpoint{1.487709in}{-0.269810in}}{\pgfqpoint{1.478876in}{-0.266152in}}{\pgfqpoint{1.469668in}{-0.266152in}}%
\pgfpathcurveto{\pgfqpoint{1.460459in}{-0.266152in}}{\pgfqpoint{1.451627in}{-0.269810in}}{\pgfqpoint{1.445115in}{-0.276321in}}%
\pgfpathcurveto{\pgfqpoint{1.438604in}{-0.282833in}}{\pgfqpoint{1.434945in}{-0.291665in}}{\pgfqpoint{1.434945in}{-0.300874in}}%
\pgfpathcurveto{\pgfqpoint{1.434945in}{-0.310082in}}{\pgfqpoint{1.438604in}{-0.318915in}}{\pgfqpoint{1.445115in}{-0.325426in}}%
\pgfpathcurveto{\pgfqpoint{1.451627in}{-0.331937in}}{\pgfqpoint{1.460459in}{-0.335596in}}{\pgfqpoint{1.469668in}{-0.335596in}}%
\pgfpathlineto{\pgfqpoint{1.469668in}{-0.335596in}}%
\pgfpathclose%
\pgfusepath{stroke,fill}%
\end{pgfscope}%
\begin{pgfscope}%
\pgfpathrectangle{\pgfqpoint{1.374500in}{0.082500in}}{\pgfqpoint{2.419000in}{2.419000in}}%
\pgfusepath{clip}%
\pgfsetbuttcap%
\pgfsetroundjoin%
\definecolor{currentfill}{rgb}{0.247059,0.564706,0.854902}%
\pgfsetfillcolor{currentfill}%
\pgfsetfillopacity{0.516523}%
\pgfsetlinewidth{1.003750pt}%
\definecolor{currentstroke}{rgb}{0.247059,0.564706,0.854902}%
\pgfsetstrokecolor{currentstroke}%
\pgfsetstrokeopacity{0.516523}%
\pgfsetdash{}{0pt}%
\pgfpathmoveto{\pgfqpoint{-4.319488in}{-0.335596in}}%
\pgfpathcurveto{\pgfqpoint{-4.310280in}{-0.335596in}}{\pgfqpoint{-4.301447in}{-0.331937in}}{\pgfqpoint{-4.294936in}{-0.325426in}}%
\pgfpathcurveto{\pgfqpoint{-4.288425in}{-0.318915in}}{\pgfqpoint{-4.284766in}{-0.310082in}}{\pgfqpoint{-4.284766in}{-0.300874in}}%
\pgfpathcurveto{\pgfqpoint{-4.284766in}{-0.291665in}}{\pgfqpoint{-4.288425in}{-0.282833in}}{\pgfqpoint{-4.294936in}{-0.276321in}}%
\pgfpathcurveto{\pgfqpoint{-4.301447in}{-0.269810in}}{\pgfqpoint{-4.310280in}{-0.266152in}}{\pgfqpoint{-4.319488in}{-0.266152in}}%
\pgfpathcurveto{\pgfqpoint{-4.328697in}{-0.266152in}}{\pgfqpoint{-4.337529in}{-0.269810in}}{\pgfqpoint{-4.344041in}{-0.276321in}}%
\pgfpathcurveto{\pgfqpoint{-4.350552in}{-0.282833in}}{\pgfqpoint{-4.354211in}{-0.291665in}}{\pgfqpoint{-4.354211in}{-0.300874in}}%
\pgfpathcurveto{\pgfqpoint{-4.354211in}{-0.310082in}}{\pgfqpoint{-4.350552in}{-0.318915in}}{\pgfqpoint{-4.344041in}{-0.325426in}}%
\pgfpathcurveto{\pgfqpoint{-4.337529in}{-0.331937in}}{\pgfqpoint{-4.328697in}{-0.335596in}}{\pgfqpoint{-4.319488in}{-0.335596in}}%
\pgfpathlineto{\pgfqpoint{-4.319488in}{-0.335596in}}%
\pgfpathclose%
\pgfusepath{stroke,fill}%
\end{pgfscope}%
\begin{pgfscope}%
\pgfpathrectangle{\pgfqpoint{1.374500in}{0.082500in}}{\pgfqpoint{2.419000in}{2.419000in}}%
\pgfusepath{clip}%
\pgfsetbuttcap%
\pgfsetroundjoin%
\definecolor{currentfill}{rgb}{0.247059,0.564706,0.854902}%
\pgfsetfillcolor{currentfill}%
\pgfsetfillopacity{0.516523}%
\pgfsetlinewidth{1.003750pt}%
\definecolor{currentstroke}{rgb}{0.247059,0.564706,0.854902}%
\pgfsetstrokecolor{currentstroke}%
\pgfsetstrokeopacity{0.516523}%
\pgfsetdash{}{0pt}%
\pgfpathmoveto{\pgfqpoint{7.258824in}{-0.335596in}}%
\pgfpathcurveto{\pgfqpoint{7.268032in}{-0.335596in}}{\pgfqpoint{7.276865in}{-0.331937in}}{\pgfqpoint{7.283376in}{-0.325426in}}%
\pgfpathcurveto{\pgfqpoint{7.289887in}{-0.318915in}}{\pgfqpoint{7.293546in}{-0.310082in}}{\pgfqpoint{7.293546in}{-0.300874in}}%
\pgfpathcurveto{\pgfqpoint{7.293546in}{-0.291665in}}{\pgfqpoint{7.289887in}{-0.282833in}}{\pgfqpoint{7.283376in}{-0.276321in}}%
\pgfpathcurveto{\pgfqpoint{7.276865in}{-0.269810in}}{\pgfqpoint{7.268032in}{-0.266152in}}{\pgfqpoint{7.258824in}{-0.266152in}}%
\pgfpathcurveto{\pgfqpoint{7.249615in}{-0.266152in}}{\pgfqpoint{7.240783in}{-0.269810in}}{\pgfqpoint{7.234271in}{-0.276321in}}%
\pgfpathcurveto{\pgfqpoint{7.227760in}{-0.282833in}}{\pgfqpoint{7.224101in}{-0.291665in}}{\pgfqpoint{7.224101in}{-0.300874in}}%
\pgfpathcurveto{\pgfqpoint{7.224101in}{-0.310082in}}{\pgfqpoint{7.227760in}{-0.318915in}}{\pgfqpoint{7.234271in}{-0.325426in}}%
\pgfpathcurveto{\pgfqpoint{7.240783in}{-0.331937in}}{\pgfqpoint{7.249615in}{-0.335596in}}{\pgfqpoint{7.258824in}{-0.335596in}}%
\pgfpathlineto{\pgfqpoint{7.258824in}{-0.335596in}}%
\pgfpathclose%
\pgfusepath{stroke,fill}%
\end{pgfscope}%
\begin{pgfscope}%
\pgfpathrectangle{\pgfqpoint{1.374500in}{0.082500in}}{\pgfqpoint{2.419000in}{2.419000in}}%
\pgfusepath{clip}%
\pgfsetbuttcap%
\pgfsetroundjoin%
\definecolor{currentfill}{rgb}{0.247059,0.564706,0.854902}%
\pgfsetfillcolor{currentfill}%
\pgfsetfillopacity{0.526025}%
\pgfsetlinewidth{1.003750pt}%
\definecolor{currentstroke}{rgb}{0.247059,0.564706,0.854902}%
\pgfsetstrokecolor{currentstroke}%
\pgfsetstrokeopacity{0.526025}%
\pgfsetdash{}{0pt}%
\pgfpathmoveto{\pgfqpoint{-0.956330in}{-0.530041in}}%
\pgfpathcurveto{\pgfqpoint{-0.947121in}{-0.530041in}}{\pgfqpoint{-0.938289in}{-0.526383in}}{\pgfqpoint{-0.931778in}{-0.519871in}}%
\pgfpathcurveto{\pgfqpoint{-0.925266in}{-0.513360in}}{\pgfqpoint{-0.921608in}{-0.504527in}}{\pgfqpoint{-0.921608in}{-0.495319in}}%
\pgfpathcurveto{\pgfqpoint{-0.921608in}{-0.486111in}}{\pgfqpoint{-0.925266in}{-0.477278in}}{\pgfqpoint{-0.931778in}{-0.470767in}}%
\pgfpathcurveto{\pgfqpoint{-0.938289in}{-0.464255in}}{\pgfqpoint{-0.947121in}{-0.460597in}}{\pgfqpoint{-0.956330in}{-0.460597in}}%
\pgfpathcurveto{\pgfqpoint{-0.965538in}{-0.460597in}}{\pgfqpoint{-0.974371in}{-0.464255in}}{\pgfqpoint{-0.980882in}{-0.470767in}}%
\pgfpathcurveto{\pgfqpoint{-0.987394in}{-0.477278in}}{\pgfqpoint{-0.991052in}{-0.486111in}}{\pgfqpoint{-0.991052in}{-0.495319in}}%
\pgfpathcurveto{\pgfqpoint{-0.991052in}{-0.504527in}}{\pgfqpoint{-0.987394in}{-0.513360in}}{\pgfqpoint{-0.980882in}{-0.519871in}}%
\pgfpathcurveto{\pgfqpoint{-0.974371in}{-0.526383in}}{\pgfqpoint{-0.965538in}{-0.530041in}}{\pgfqpoint{-0.956330in}{-0.530041in}}%
\pgfpathlineto{\pgfqpoint{-0.956330in}{-0.530041in}}%
\pgfpathclose%
\pgfusepath{stroke,fill}%
\end{pgfscope}%
\begin{pgfscope}%
\pgfpathrectangle{\pgfqpoint{1.374500in}{0.082500in}}{\pgfqpoint{2.419000in}{2.419000in}}%
\pgfusepath{clip}%
\pgfsetbuttcap%
\pgfsetroundjoin%
\definecolor{currentfill}{rgb}{0.247059,0.564706,0.854902}%
\pgfsetfillcolor{currentfill}%
\pgfsetfillopacity{0.526025}%
\pgfsetlinewidth{1.003750pt}%
\definecolor{currentstroke}{rgb}{0.247059,0.564706,0.854902}%
\pgfsetstrokecolor{currentstroke}%
\pgfsetstrokeopacity{0.526025}%
\pgfsetdash{}{0pt}%
\pgfpathmoveto{\pgfqpoint{4.955518in}{-0.530041in}}%
\pgfpathcurveto{\pgfqpoint{4.964726in}{-0.530041in}}{\pgfqpoint{4.973559in}{-0.526383in}}{\pgfqpoint{4.980070in}{-0.519871in}}%
\pgfpathcurveto{\pgfqpoint{4.986582in}{-0.513360in}}{\pgfqpoint{4.990240in}{-0.504527in}}{\pgfqpoint{4.990240in}{-0.495319in}}%
\pgfpathcurveto{\pgfqpoint{4.990240in}{-0.486111in}}{\pgfqpoint{4.986582in}{-0.477278in}}{\pgfqpoint{4.980070in}{-0.470767in}}%
\pgfpathcurveto{\pgfqpoint{4.973559in}{-0.464255in}}{\pgfqpoint{4.964726in}{-0.460597in}}{\pgfqpoint{4.955518in}{-0.460597in}}%
\pgfpathcurveto{\pgfqpoint{4.946310in}{-0.460597in}}{\pgfqpoint{4.937477in}{-0.464255in}}{\pgfqpoint{4.930966in}{-0.470767in}}%
\pgfpathcurveto{\pgfqpoint{4.924454in}{-0.477278in}}{\pgfqpoint{4.920796in}{-0.486111in}}{\pgfqpoint{4.920796in}{-0.495319in}}%
\pgfpathcurveto{\pgfqpoint{4.920796in}{-0.504527in}}{\pgfqpoint{4.924454in}{-0.513360in}}{\pgfqpoint{4.930966in}{-0.519871in}}%
\pgfpathcurveto{\pgfqpoint{4.937477in}{-0.526383in}}{\pgfqpoint{4.946310in}{-0.530041in}}{\pgfqpoint{4.955518in}{-0.530041in}}%
\pgfpathlineto{\pgfqpoint{4.955518in}{-0.530041in}}%
\pgfpathclose%
\pgfusepath{stroke,fill}%
\end{pgfscope}%
\begin{pgfscope}%
\pgfpathrectangle{\pgfqpoint{1.374500in}{0.082500in}}{\pgfqpoint{2.419000in}{2.419000in}}%
\pgfusepath{clip}%
\pgfsetbuttcap%
\pgfsetroundjoin%
\definecolor{currentfill}{rgb}{0.247059,0.564706,0.854902}%
\pgfsetfillcolor{currentfill}%
\pgfsetfillopacity{0.526025}%
\pgfsetlinewidth{1.003750pt}%
\definecolor{currentstroke}{rgb}{0.247059,0.564706,0.854902}%
\pgfsetstrokecolor{currentstroke}%
\pgfsetstrokeopacity{0.526025}%
\pgfsetdash{}{0pt}%
\pgfpathmoveto{\pgfqpoint{10.867366in}{-0.530041in}}%
\pgfpathcurveto{\pgfqpoint{10.876574in}{-0.530041in}}{\pgfqpoint{10.885407in}{-0.526383in}}{\pgfqpoint{10.891918in}{-0.519871in}}%
\pgfpathcurveto{\pgfqpoint{10.898430in}{-0.513360in}}{\pgfqpoint{10.902088in}{-0.504527in}}{\pgfqpoint{10.902088in}{-0.495319in}}%
\pgfpathcurveto{\pgfqpoint{10.902088in}{-0.486111in}}{\pgfqpoint{10.898430in}{-0.477278in}}{\pgfqpoint{10.891918in}{-0.470767in}}%
\pgfpathcurveto{\pgfqpoint{10.885407in}{-0.464255in}}{\pgfqpoint{10.876574in}{-0.460597in}}{\pgfqpoint{10.867366in}{-0.460597in}}%
\pgfpathcurveto{\pgfqpoint{10.858157in}{-0.460597in}}{\pgfqpoint{10.849325in}{-0.464255in}}{\pgfqpoint{10.842814in}{-0.470767in}}%
\pgfpathcurveto{\pgfqpoint{10.836302in}{-0.477278in}}{\pgfqpoint{10.832644in}{-0.486111in}}{\pgfqpoint{10.832644in}{-0.495319in}}%
\pgfpathcurveto{\pgfqpoint{10.832644in}{-0.504527in}}{\pgfqpoint{10.836302in}{-0.513360in}}{\pgfqpoint{10.842814in}{-0.519871in}}%
\pgfpathcurveto{\pgfqpoint{10.849325in}{-0.526383in}}{\pgfqpoint{10.858157in}{-0.530041in}}{\pgfqpoint{10.867366in}{-0.530041in}}%
\pgfpathlineto{\pgfqpoint{10.867366in}{-0.530041in}}%
\pgfpathclose%
\pgfusepath{stroke,fill}%
\end{pgfscope}%
\begin{pgfscope}%
\pgfpathrectangle{\pgfqpoint{1.374500in}{0.082500in}}{\pgfqpoint{2.419000in}{2.419000in}}%
\pgfusepath{clip}%
\pgfsetbuttcap%
\pgfsetroundjoin%
\definecolor{currentfill}{rgb}{0.247059,0.564706,0.854902}%
\pgfsetfillcolor{currentfill}%
\pgfsetfillopacity{0.535939}%
\pgfsetlinewidth{1.003750pt}%
\definecolor{currentstroke}{rgb}{0.247059,0.564706,0.854902}%
\pgfsetstrokecolor{currentstroke}%
\pgfsetstrokeopacity{0.535939}%
\pgfsetdash{}{0pt}%
\pgfpathmoveto{\pgfqpoint{2.552469in}{-0.732907in}}%
\pgfpathcurveto{\pgfqpoint{2.561677in}{-0.732907in}}{\pgfqpoint{2.570510in}{-0.729248in}}{\pgfqpoint{2.577021in}{-0.722737in}}%
\pgfpathcurveto{\pgfqpoint{2.583532in}{-0.716226in}}{\pgfqpoint{2.587191in}{-0.707393in}}{\pgfqpoint{2.587191in}{-0.698185in}}%
\pgfpathcurveto{\pgfqpoint{2.587191in}{-0.688976in}}{\pgfqpoint{2.583532in}{-0.680144in}}{\pgfqpoint{2.577021in}{-0.673632in}}%
\pgfpathcurveto{\pgfqpoint{2.570510in}{-0.667121in}}{\pgfqpoint{2.561677in}{-0.663462in}}{\pgfqpoint{2.552469in}{-0.663462in}}%
\pgfpathcurveto{\pgfqpoint{2.543260in}{-0.663462in}}{\pgfqpoint{2.534428in}{-0.667121in}}{\pgfqpoint{2.527916in}{-0.673632in}}%
\pgfpathcurveto{\pgfqpoint{2.521405in}{-0.680144in}}{\pgfqpoint{2.517747in}{-0.688976in}}{\pgfqpoint{2.517747in}{-0.698185in}}%
\pgfpathcurveto{\pgfqpoint{2.517747in}{-0.707393in}}{\pgfqpoint{2.521405in}{-0.716226in}}{\pgfqpoint{2.527916in}{-0.722737in}}%
\pgfpathcurveto{\pgfqpoint{2.534428in}{-0.729248in}}{\pgfqpoint{2.543260in}{-0.732907in}}{\pgfqpoint{2.552469in}{-0.732907in}}%
\pgfpathlineto{\pgfqpoint{2.552469in}{-0.732907in}}%
\pgfpathclose%
\pgfusepath{stroke,fill}%
\end{pgfscope}%
\begin{pgfscope}%
\pgfpathrectangle{\pgfqpoint{1.374500in}{0.082500in}}{\pgfqpoint{2.419000in}{2.419000in}}%
\pgfusepath{clip}%
\pgfsetbuttcap%
\pgfsetroundjoin%
\definecolor{currentfill}{rgb}{0.247059,0.564706,0.854902}%
\pgfsetfillcolor{currentfill}%
\pgfsetfillopacity{0.535939}%
\pgfsetlinewidth{1.003750pt}%
\definecolor{currentstroke}{rgb}{0.247059,0.564706,0.854902}%
\pgfsetstrokecolor{currentstroke}%
\pgfsetstrokeopacity{0.535939}%
\pgfsetdash{}{0pt}%
\pgfpathmoveto{\pgfqpoint{-3.487384in}{-0.732907in}}%
\pgfpathcurveto{\pgfqpoint{-3.478176in}{-0.732907in}}{\pgfqpoint{-3.469343in}{-0.729248in}}{\pgfqpoint{-3.462832in}{-0.722737in}}%
\pgfpathcurveto{\pgfqpoint{-3.456320in}{-0.716226in}}{\pgfqpoint{-3.452662in}{-0.707393in}}{\pgfqpoint{-3.452662in}{-0.698185in}}%
\pgfpathcurveto{\pgfqpoint{-3.452662in}{-0.688976in}}{\pgfqpoint{-3.456320in}{-0.680144in}}{\pgfqpoint{-3.462832in}{-0.673632in}}%
\pgfpathcurveto{\pgfqpoint{-3.469343in}{-0.667121in}}{\pgfqpoint{-3.478176in}{-0.663462in}}{\pgfqpoint{-3.487384in}{-0.663462in}}%
\pgfpathcurveto{\pgfqpoint{-3.496593in}{-0.663462in}}{\pgfqpoint{-3.505425in}{-0.667121in}}{\pgfqpoint{-3.511936in}{-0.673632in}}%
\pgfpathcurveto{\pgfqpoint{-3.518448in}{-0.680144in}}{\pgfqpoint{-3.522106in}{-0.688976in}}{\pgfqpoint{-3.522106in}{-0.698185in}}%
\pgfpathcurveto{\pgfqpoint{-3.522106in}{-0.707393in}}{\pgfqpoint{-3.518448in}{-0.716226in}}{\pgfqpoint{-3.511936in}{-0.722737in}}%
\pgfpathcurveto{\pgfqpoint{-3.505425in}{-0.729248in}}{\pgfqpoint{-3.496593in}{-0.732907in}}{\pgfqpoint{-3.487384in}{-0.732907in}}%
\pgfpathlineto{\pgfqpoint{-3.487384in}{-0.732907in}}%
\pgfpathclose%
\pgfusepath{stroke,fill}%
\end{pgfscope}%
\begin{pgfscope}%
\pgfpathrectangle{\pgfqpoint{1.374500in}{0.082500in}}{\pgfqpoint{2.419000in}{2.419000in}}%
\pgfusepath{clip}%
\pgfsetbuttcap%
\pgfsetroundjoin%
\definecolor{currentfill}{rgb}{0.247059,0.564706,0.854902}%
\pgfsetfillcolor{currentfill}%
\pgfsetfillopacity{0.535939}%
\pgfsetlinewidth{1.003750pt}%
\definecolor{currentstroke}{rgb}{0.247059,0.564706,0.854902}%
\pgfsetstrokecolor{currentstroke}%
\pgfsetstrokeopacity{0.535939}%
\pgfsetdash{}{0pt}%
\pgfpathmoveto{\pgfqpoint{8.592322in}{-0.732907in}}%
\pgfpathcurveto{\pgfqpoint{8.601530in}{-0.732907in}}{\pgfqpoint{8.610363in}{-0.729248in}}{\pgfqpoint{8.616874in}{-0.722737in}}%
\pgfpathcurveto{\pgfqpoint{8.623385in}{-0.716226in}}{\pgfqpoint{8.627044in}{-0.707393in}}{\pgfqpoint{8.627044in}{-0.698185in}}%
\pgfpathcurveto{\pgfqpoint{8.627044in}{-0.688976in}}{\pgfqpoint{8.623385in}{-0.680144in}}{\pgfqpoint{8.616874in}{-0.673632in}}%
\pgfpathcurveto{\pgfqpoint{8.610363in}{-0.667121in}}{\pgfqpoint{8.601530in}{-0.663462in}}{\pgfqpoint{8.592322in}{-0.663462in}}%
\pgfpathcurveto{\pgfqpoint{8.583113in}{-0.663462in}}{\pgfqpoint{8.574281in}{-0.667121in}}{\pgfqpoint{8.567769in}{-0.673632in}}%
\pgfpathcurveto{\pgfqpoint{8.561258in}{-0.680144in}}{\pgfqpoint{8.557599in}{-0.688976in}}{\pgfqpoint{8.557599in}{-0.698185in}}%
\pgfpathcurveto{\pgfqpoint{8.557599in}{-0.707393in}}{\pgfqpoint{8.561258in}{-0.716226in}}{\pgfqpoint{8.567769in}{-0.722737in}}%
\pgfpathcurveto{\pgfqpoint{8.574281in}{-0.729248in}}{\pgfqpoint{8.583113in}{-0.732907in}}{\pgfqpoint{8.592322in}{-0.732907in}}%
\pgfpathlineto{\pgfqpoint{8.592322in}{-0.732907in}}%
\pgfpathclose%
\pgfusepath{stroke,fill}%
\end{pgfscope}%
\begin{pgfscope}%
\pgfpathrectangle{\pgfqpoint{1.374500in}{0.082500in}}{\pgfqpoint{2.419000in}{2.419000in}}%
\pgfusepath{clip}%
\pgfsetbuttcap%
\pgfsetroundjoin%
\definecolor{currentfill}{rgb}{0.247059,0.564706,0.854902}%
\pgfsetfillcolor{currentfill}%
\pgfsetfillopacity{0.546292}%
\pgfsetlinewidth{1.003750pt}%
\definecolor{currentstroke}{rgb}{0.247059,0.564706,0.854902}%
\pgfsetstrokecolor{currentstroke}%
\pgfsetstrokeopacity{0.546292}%
\pgfsetdash{}{0pt}%
\pgfpathmoveto{\pgfqpoint{0.043053in}{-0.944752in}}%
\pgfpathcurveto{\pgfqpoint{0.052262in}{-0.944752in}}{\pgfqpoint{0.061094in}{-0.941093in}}{\pgfqpoint{0.067606in}{-0.934582in}}%
\pgfpathcurveto{\pgfqpoint{0.074117in}{-0.928071in}}{\pgfqpoint{0.077776in}{-0.919238in}}{\pgfqpoint{0.077776in}{-0.910030in}}%
\pgfpathcurveto{\pgfqpoint{0.077776in}{-0.900821in}}{\pgfqpoint{0.074117in}{-0.891989in}}{\pgfqpoint{0.067606in}{-0.885477in}}%
\pgfpathcurveto{\pgfqpoint{0.061094in}{-0.878966in}}{\pgfqpoint{0.052262in}{-0.875307in}}{\pgfqpoint{0.043053in}{-0.875307in}}%
\pgfpathcurveto{\pgfqpoint{0.033845in}{-0.875307in}}{\pgfqpoint{0.025012in}{-0.878966in}}{\pgfqpoint{0.018501in}{-0.885477in}}%
\pgfpathcurveto{\pgfqpoint{0.011990in}{-0.891989in}}{\pgfqpoint{0.008331in}{-0.900821in}}{\pgfqpoint{0.008331in}{-0.910030in}}%
\pgfpathcurveto{\pgfqpoint{0.008331in}{-0.919238in}}{\pgfqpoint{0.011990in}{-0.928071in}}{\pgfqpoint{0.018501in}{-0.934582in}}%
\pgfpathcurveto{\pgfqpoint{0.025012in}{-0.941093in}}{\pgfqpoint{0.033845in}{-0.944752in}}{\pgfqpoint{0.043053in}{-0.944752in}}%
\pgfpathlineto{\pgfqpoint{0.043053in}{-0.944752in}}%
\pgfpathclose%
\pgfusepath{stroke,fill}%
\end{pgfscope}%
\begin{pgfscope}%
\pgfpathrectangle{\pgfqpoint{1.374500in}{0.082500in}}{\pgfqpoint{2.419000in}{2.419000in}}%
\pgfusepath{clip}%
\pgfsetbuttcap%
\pgfsetroundjoin%
\definecolor{currentfill}{rgb}{0.247059,0.564706,0.854902}%
\pgfsetfillcolor{currentfill}%
\pgfsetfillopacity{0.546292}%
\pgfsetlinewidth{1.003750pt}%
\definecolor{currentstroke}{rgb}{0.247059,0.564706,0.854902}%
\pgfsetstrokecolor{currentstroke}%
\pgfsetstrokeopacity{0.546292}%
\pgfsetdash{}{0pt}%
\pgfpathmoveto{\pgfqpoint{-6.130470in}{-0.944752in}}%
\pgfpathcurveto{\pgfqpoint{-6.121262in}{-0.944752in}}{\pgfqpoint{-6.112429in}{-0.941093in}}{\pgfqpoint{-6.105918in}{-0.934582in}}%
\pgfpathcurveto{\pgfqpoint{-6.099407in}{-0.928071in}}{\pgfqpoint{-6.095748in}{-0.919238in}}{\pgfqpoint{-6.095748in}{-0.910030in}}%
\pgfpathcurveto{\pgfqpoint{-6.095748in}{-0.900821in}}{\pgfqpoint{-6.099407in}{-0.891989in}}{\pgfqpoint{-6.105918in}{-0.885477in}}%
\pgfpathcurveto{\pgfqpoint{-6.112429in}{-0.878966in}}{\pgfqpoint{-6.121262in}{-0.875307in}}{\pgfqpoint{-6.130470in}{-0.875307in}}%
\pgfpathcurveto{\pgfqpoint{-6.139679in}{-0.875307in}}{\pgfqpoint{-6.148511in}{-0.878966in}}{\pgfqpoint{-6.155022in}{-0.885477in}}%
\pgfpathcurveto{\pgfqpoint{-6.161534in}{-0.891989in}}{\pgfqpoint{-6.165192in}{-0.900821in}}{\pgfqpoint{-6.165192in}{-0.910030in}}%
\pgfpathcurveto{\pgfqpoint{-6.165192in}{-0.919238in}}{\pgfqpoint{-6.161534in}{-0.928071in}}{\pgfqpoint{-6.155022in}{-0.934582in}}%
\pgfpathcurveto{\pgfqpoint{-6.148511in}{-0.941093in}}{\pgfqpoint{-6.139679in}{-0.944752in}}{\pgfqpoint{-6.130470in}{-0.944752in}}%
\pgfpathlineto{\pgfqpoint{-6.130470in}{-0.944752in}}%
\pgfpathclose%
\pgfusepath{stroke,fill}%
\end{pgfscope}%
\begin{pgfscope}%
\pgfpathrectangle{\pgfqpoint{1.374500in}{0.082500in}}{\pgfqpoint{2.419000in}{2.419000in}}%
\pgfusepath{clip}%
\pgfsetbuttcap%
\pgfsetroundjoin%
\definecolor{currentfill}{rgb}{0.247059,0.564706,0.854902}%
\pgfsetfillcolor{currentfill}%
\pgfsetfillopacity{0.546292}%
\pgfsetlinewidth{1.003750pt}%
\definecolor{currentstroke}{rgb}{0.247059,0.564706,0.854902}%
\pgfsetstrokecolor{currentstroke}%
\pgfsetstrokeopacity{0.546292}%
\pgfsetdash{}{0pt}%
\pgfpathmoveto{\pgfqpoint{6.216577in}{-0.944752in}}%
\pgfpathcurveto{\pgfqpoint{6.225786in}{-0.944752in}}{\pgfqpoint{6.234618in}{-0.941093in}}{\pgfqpoint{6.241129in}{-0.934582in}}%
\pgfpathcurveto{\pgfqpoint{6.247641in}{-0.928071in}}{\pgfqpoint{6.251299in}{-0.919238in}}{\pgfqpoint{6.251299in}{-0.910030in}}%
\pgfpathcurveto{\pgfqpoint{6.251299in}{-0.900821in}}{\pgfqpoint{6.247641in}{-0.891989in}}{\pgfqpoint{6.241129in}{-0.885477in}}%
\pgfpathcurveto{\pgfqpoint{6.234618in}{-0.878966in}}{\pgfqpoint{6.225786in}{-0.875307in}}{\pgfqpoint{6.216577in}{-0.875307in}}%
\pgfpathcurveto{\pgfqpoint{6.207369in}{-0.875307in}}{\pgfqpoint{6.198536in}{-0.878966in}}{\pgfqpoint{6.192025in}{-0.885477in}}%
\pgfpathcurveto{\pgfqpoint{6.185513in}{-0.891989in}}{\pgfqpoint{6.181855in}{-0.900821in}}{\pgfqpoint{6.181855in}{-0.910030in}}%
\pgfpathcurveto{\pgfqpoint{6.181855in}{-0.919238in}}{\pgfqpoint{6.185513in}{-0.928071in}}{\pgfqpoint{6.192025in}{-0.934582in}}%
\pgfpathcurveto{\pgfqpoint{6.198536in}{-0.941093in}}{\pgfqpoint{6.207369in}{-0.944752in}}{\pgfqpoint{6.216577in}{-0.944752in}}%
\pgfpathlineto{\pgfqpoint{6.216577in}{-0.944752in}}%
\pgfpathclose%
\pgfusepath{stroke,fill}%
\end{pgfscope}%
\begin{pgfscope}%
\pgfpathrectangle{\pgfqpoint{1.374500in}{0.082500in}}{\pgfqpoint{2.419000in}{2.419000in}}%
\pgfusepath{clip}%
\pgfsetbuttcap%
\pgfsetroundjoin%
\definecolor{currentfill}{rgb}{0.247059,0.564706,0.854902}%
\pgfsetfillcolor{currentfill}%
\pgfsetfillopacity{0.557113}%
\pgfsetlinewidth{1.003750pt}%
\definecolor{currentstroke}{rgb}{0.247059,0.564706,0.854902}%
\pgfsetstrokecolor{currentstroke}%
\pgfsetstrokeopacity{0.557113}%
\pgfsetdash{}{0pt}%
\pgfpathmoveto{\pgfqpoint{-2.579950in}{-1.166186in}}%
\pgfpathcurveto{\pgfqpoint{-2.570741in}{-1.166186in}}{\pgfqpoint{-2.561909in}{-1.162528in}}{\pgfqpoint{-2.555397in}{-1.156016in}}%
\pgfpathcurveto{\pgfqpoint{-2.548886in}{-1.149505in}}{\pgfqpoint{-2.545228in}{-1.140672in}}{\pgfqpoint{-2.545228in}{-1.131464in}}%
\pgfpathcurveto{\pgfqpoint{-2.545228in}{-1.122255in}}{\pgfqpoint{-2.548886in}{-1.113423in}}{\pgfqpoint{-2.555397in}{-1.106912in}}%
\pgfpathcurveto{\pgfqpoint{-2.561909in}{-1.100400in}}{\pgfqpoint{-2.570741in}{-1.096742in}}{\pgfqpoint{-2.579950in}{-1.096742in}}%
\pgfpathcurveto{\pgfqpoint{-2.589158in}{-1.096742in}}{\pgfqpoint{-2.597991in}{-1.100400in}}{\pgfqpoint{-2.604502in}{-1.106912in}}%
\pgfpathcurveto{\pgfqpoint{-2.611013in}{-1.113423in}}{\pgfqpoint{-2.614672in}{-1.122255in}}{\pgfqpoint{-2.614672in}{-1.131464in}}%
\pgfpathcurveto{\pgfqpoint{-2.614672in}{-1.140672in}}{\pgfqpoint{-2.611013in}{-1.149505in}}{\pgfqpoint{-2.604502in}{-1.156016in}}%
\pgfpathcurveto{\pgfqpoint{-2.597991in}{-1.162528in}}{\pgfqpoint{-2.589158in}{-1.166186in}}{\pgfqpoint{-2.579950in}{-1.166186in}}%
\pgfpathlineto{\pgfqpoint{-2.579950in}{-1.166186in}}%
\pgfpathclose%
\pgfusepath{stroke,fill}%
\end{pgfscope}%
\begin{pgfscope}%
\pgfpathrectangle{\pgfqpoint{1.374500in}{0.082500in}}{\pgfqpoint{2.419000in}{2.419000in}}%
\pgfusepath{clip}%
\pgfsetbuttcap%
\pgfsetroundjoin%
\definecolor{currentfill}{rgb}{0.247059,0.564706,0.854902}%
\pgfsetfillcolor{currentfill}%
\pgfsetfillopacity{0.557113}%
\pgfsetlinewidth{1.003750pt}%
\definecolor{currentstroke}{rgb}{0.247059,0.564706,0.854902}%
\pgfsetstrokecolor{currentstroke}%
\pgfsetstrokeopacity{0.557113}%
\pgfsetdash{}{0pt}%
\pgfpathmoveto{\pgfqpoint{3.733295in}{-1.166186in}}%
\pgfpathcurveto{\pgfqpoint{3.742504in}{-1.166186in}}{\pgfqpoint{3.751336in}{-1.162528in}}{\pgfqpoint{3.757848in}{-1.156016in}}%
\pgfpathcurveto{\pgfqpoint{3.764359in}{-1.149505in}}{\pgfqpoint{3.768017in}{-1.140672in}}{\pgfqpoint{3.768017in}{-1.131464in}}%
\pgfpathcurveto{\pgfqpoint{3.768017in}{-1.122255in}}{\pgfqpoint{3.764359in}{-1.113423in}}{\pgfqpoint{3.757848in}{-1.106912in}}%
\pgfpathcurveto{\pgfqpoint{3.751336in}{-1.100400in}}{\pgfqpoint{3.742504in}{-1.096742in}}{\pgfqpoint{3.733295in}{-1.096742in}}%
\pgfpathcurveto{\pgfqpoint{3.724087in}{-1.096742in}}{\pgfqpoint{3.715254in}{-1.100400in}}{\pgfqpoint{3.708743in}{-1.106912in}}%
\pgfpathcurveto{\pgfqpoint{3.702232in}{-1.113423in}}{\pgfqpoint{3.698573in}{-1.122255in}}{\pgfqpoint{3.698573in}{-1.131464in}}%
\pgfpathcurveto{\pgfqpoint{3.698573in}{-1.140672in}}{\pgfqpoint{3.702232in}{-1.149505in}}{\pgfqpoint{3.708743in}{-1.156016in}}%
\pgfpathcurveto{\pgfqpoint{3.715254in}{-1.162528in}}{\pgfqpoint{3.724087in}{-1.166186in}}{\pgfqpoint{3.733295in}{-1.166186in}}%
\pgfpathlineto{\pgfqpoint{3.733295in}{-1.166186in}}%
\pgfpathclose%
\pgfusepath{stroke,fill}%
\end{pgfscope}%
\begin{pgfscope}%
\pgfpathrectangle{\pgfqpoint{1.374500in}{0.082500in}}{\pgfqpoint{2.419000in}{2.419000in}}%
\pgfusepath{clip}%
\pgfsetbuttcap%
\pgfsetroundjoin%
\definecolor{currentfill}{rgb}{0.247059,0.564706,0.854902}%
\pgfsetfillcolor{currentfill}%
\pgfsetfillopacity{0.557113}%
\pgfsetlinewidth{1.003750pt}%
\definecolor{currentstroke}{rgb}{0.247059,0.564706,0.854902}%
\pgfsetstrokecolor{currentstroke}%
\pgfsetstrokeopacity{0.557113}%
\pgfsetdash{}{0pt}%
\pgfpathmoveto{\pgfqpoint{10.046540in}{-1.166186in}}%
\pgfpathcurveto{\pgfqpoint{10.055749in}{-1.166186in}}{\pgfqpoint{10.064581in}{-1.162528in}}{\pgfqpoint{10.071093in}{-1.156016in}}%
\pgfpathcurveto{\pgfqpoint{10.077604in}{-1.149505in}}{\pgfqpoint{10.081263in}{-1.140672in}}{\pgfqpoint{10.081263in}{-1.131464in}}%
\pgfpathcurveto{\pgfqpoint{10.081263in}{-1.122255in}}{\pgfqpoint{10.077604in}{-1.113423in}}{\pgfqpoint{10.071093in}{-1.106912in}}%
\pgfpathcurveto{\pgfqpoint{10.064581in}{-1.100400in}}{\pgfqpoint{10.055749in}{-1.096742in}}{\pgfqpoint{10.046540in}{-1.096742in}}%
\pgfpathcurveto{\pgfqpoint{10.037332in}{-1.096742in}}{\pgfqpoint{10.028499in}{-1.100400in}}{\pgfqpoint{10.021988in}{-1.106912in}}%
\pgfpathcurveto{\pgfqpoint{10.015477in}{-1.113423in}}{\pgfqpoint{10.011818in}{-1.122255in}}{\pgfqpoint{10.011818in}{-1.131464in}}%
\pgfpathcurveto{\pgfqpoint{10.011818in}{-1.140672in}}{\pgfqpoint{10.015477in}{-1.149505in}}{\pgfqpoint{10.021988in}{-1.156016in}}%
\pgfpathcurveto{\pgfqpoint{10.028499in}{-1.162528in}}{\pgfqpoint{10.037332in}{-1.166186in}}{\pgfqpoint{10.046540in}{-1.166186in}}%
\pgfpathlineto{\pgfqpoint{10.046540in}{-1.166186in}}%
\pgfpathclose%
\pgfusepath{stroke,fill}%
\end{pgfscope}%
\begin{pgfscope}%
\pgfpathrectangle{\pgfqpoint{1.374500in}{0.082500in}}{\pgfqpoint{2.419000in}{2.419000in}}%
\pgfusepath{clip}%
\pgfsetbuttcap%
\pgfsetroundjoin%
\definecolor{currentfill}{rgb}{0.247059,0.564706,0.854902}%
\pgfsetfillcolor{currentfill}%
\pgfsetfillopacity{0.568436}%
\pgfsetlinewidth{1.003750pt}%
\definecolor{currentstroke}{rgb}{0.247059,0.564706,0.854902}%
\pgfsetstrokecolor{currentstroke}%
\pgfsetstrokeopacity{0.568436}%
\pgfsetdash{}{0pt}%
\pgfpathmoveto{\pgfqpoint{-5.324432in}{-1.397876in}}%
\pgfpathcurveto{\pgfqpoint{-5.315223in}{-1.397876in}}{\pgfqpoint{-5.306391in}{-1.394217in}}{\pgfqpoint{-5.299880in}{-1.387706in}}%
\pgfpathcurveto{\pgfqpoint{-5.293368in}{-1.381194in}}{\pgfqpoint{-5.289710in}{-1.372362in}}{\pgfqpoint{-5.289710in}{-1.363153in}}%
\pgfpathcurveto{\pgfqpoint{-5.289710in}{-1.353945in}}{\pgfqpoint{-5.293368in}{-1.345112in}}{\pgfqpoint{-5.299880in}{-1.338601in}}%
\pgfpathcurveto{\pgfqpoint{-5.306391in}{-1.332090in}}{\pgfqpoint{-5.315223in}{-1.328431in}}{\pgfqpoint{-5.324432in}{-1.328431in}}%
\pgfpathcurveto{\pgfqpoint{-5.333640in}{-1.328431in}}{\pgfqpoint{-5.342473in}{-1.332090in}}{\pgfqpoint{-5.348984in}{-1.338601in}}%
\pgfpathcurveto{\pgfqpoint{-5.355496in}{-1.345112in}}{\pgfqpoint{-5.359154in}{-1.353945in}}{\pgfqpoint{-5.359154in}{-1.363153in}}%
\pgfpathcurveto{\pgfqpoint{-5.359154in}{-1.372362in}}{\pgfqpoint{-5.355496in}{-1.381194in}}{\pgfqpoint{-5.348984in}{-1.387706in}}%
\pgfpathcurveto{\pgfqpoint{-5.342473in}{-1.394217in}}{\pgfqpoint{-5.333640in}{-1.397876in}}{\pgfqpoint{-5.324432in}{-1.397876in}}%
\pgfpathlineto{\pgfqpoint{-5.324432in}{-1.397876in}}%
\pgfpathclose%
\pgfusepath{stroke,fill}%
\end{pgfscope}%
\begin{pgfscope}%
\pgfpathrectangle{\pgfqpoint{1.374500in}{0.082500in}}{\pgfqpoint{2.419000in}{2.419000in}}%
\pgfusepath{clip}%
\pgfsetbuttcap%
\pgfsetroundjoin%
\definecolor{currentfill}{rgb}{0.247059,0.564706,0.854902}%
\pgfsetfillcolor{currentfill}%
\pgfsetfillopacity{0.568436}%
\pgfsetlinewidth{1.003750pt}%
\definecolor{currentstroke}{rgb}{0.247059,0.564706,0.854902}%
\pgfsetstrokecolor{currentstroke}%
\pgfsetstrokeopacity{0.568436}%
\pgfsetdash{}{0pt}%
\pgfpathmoveto{\pgfqpoint{1.135005in}{-1.397876in}}%
\pgfpathcurveto{\pgfqpoint{1.144214in}{-1.397876in}}{\pgfqpoint{1.153046in}{-1.394217in}}{\pgfqpoint{1.159558in}{-1.387706in}}%
\pgfpathcurveto{\pgfqpoint{1.166069in}{-1.381194in}}{\pgfqpoint{1.169728in}{-1.372362in}}{\pgfqpoint{1.169728in}{-1.363153in}}%
\pgfpathcurveto{\pgfqpoint{1.169728in}{-1.353945in}}{\pgfqpoint{1.166069in}{-1.345112in}}{\pgfqpoint{1.159558in}{-1.338601in}}%
\pgfpathcurveto{\pgfqpoint{1.153046in}{-1.332090in}}{\pgfqpoint{1.144214in}{-1.328431in}}{\pgfqpoint{1.135005in}{-1.328431in}}%
\pgfpathcurveto{\pgfqpoint{1.125797in}{-1.328431in}}{\pgfqpoint{1.116965in}{-1.332090in}}{\pgfqpoint{1.110453in}{-1.338601in}}%
\pgfpathcurveto{\pgfqpoint{1.103942in}{-1.345112in}}{\pgfqpoint{1.100283in}{-1.353945in}}{\pgfqpoint{1.100283in}{-1.363153in}}%
\pgfpathcurveto{\pgfqpoint{1.100283in}{-1.372362in}}{\pgfqpoint{1.103942in}{-1.381194in}}{\pgfqpoint{1.110453in}{-1.387706in}}%
\pgfpathcurveto{\pgfqpoint{1.116965in}{-1.394217in}}{\pgfqpoint{1.125797in}{-1.397876in}}{\pgfqpoint{1.135005in}{-1.397876in}}%
\pgfpathlineto{\pgfqpoint{1.135005in}{-1.397876in}}%
\pgfpathclose%
\pgfusepath{stroke,fill}%
\end{pgfscope}%
\begin{pgfscope}%
\pgfpathrectangle{\pgfqpoint{1.374500in}{0.082500in}}{\pgfqpoint{2.419000in}{2.419000in}}%
\pgfusepath{clip}%
\pgfsetbuttcap%
\pgfsetroundjoin%
\definecolor{currentfill}{rgb}{0.247059,0.564706,0.854902}%
\pgfsetfillcolor{currentfill}%
\pgfsetfillopacity{0.568436}%
\pgfsetlinewidth{1.003750pt}%
\definecolor{currentstroke}{rgb}{0.247059,0.564706,0.854902}%
\pgfsetstrokecolor{currentstroke}%
\pgfsetstrokeopacity{0.568436}%
\pgfsetdash{}{0pt}%
\pgfpathmoveto{\pgfqpoint{7.594443in}{-1.397876in}}%
\pgfpathcurveto{\pgfqpoint{7.603651in}{-1.397876in}}{\pgfqpoint{7.612484in}{-1.394217in}}{\pgfqpoint{7.618995in}{-1.387706in}}%
\pgfpathcurveto{\pgfqpoint{7.625506in}{-1.381194in}}{\pgfqpoint{7.629165in}{-1.372362in}}{\pgfqpoint{7.629165in}{-1.363153in}}%
\pgfpathcurveto{\pgfqpoint{7.629165in}{-1.353945in}}{\pgfqpoint{7.625506in}{-1.345112in}}{\pgfqpoint{7.618995in}{-1.338601in}}%
\pgfpathcurveto{\pgfqpoint{7.612484in}{-1.332090in}}{\pgfqpoint{7.603651in}{-1.328431in}}{\pgfqpoint{7.594443in}{-1.328431in}}%
\pgfpathcurveto{\pgfqpoint{7.585234in}{-1.328431in}}{\pgfqpoint{7.576402in}{-1.332090in}}{\pgfqpoint{7.569890in}{-1.338601in}}%
\pgfpathcurveto{\pgfqpoint{7.563379in}{-1.345112in}}{\pgfqpoint{7.559721in}{-1.353945in}}{\pgfqpoint{7.559721in}{-1.363153in}}%
\pgfpathcurveto{\pgfqpoint{7.559721in}{-1.372362in}}{\pgfqpoint{7.563379in}{-1.381194in}}{\pgfqpoint{7.569890in}{-1.387706in}}%
\pgfpathcurveto{\pgfqpoint{7.576402in}{-1.394217in}}{\pgfqpoint{7.585234in}{-1.397876in}}{\pgfqpoint{7.594443in}{-1.397876in}}%
\pgfpathlineto{\pgfqpoint{7.594443in}{-1.397876in}}%
\pgfpathclose%
\pgfusepath{stroke,fill}%
\end{pgfscope}%
\begin{pgfscope}%
\pgfpathrectangle{\pgfqpoint{1.374500in}{0.082500in}}{\pgfqpoint{2.419000in}{2.419000in}}%
\pgfusepath{clip}%
\pgfsetbuttcap%
\pgfsetroundjoin%
\definecolor{currentfill}{rgb}{0.247059,0.564706,0.854902}%
\pgfsetfillcolor{currentfill}%
\pgfsetfillopacity{0.580295}%
\pgfsetlinewidth{1.003750pt}%
\definecolor{currentstroke}{rgb}{0.247059,0.564706,0.854902}%
\pgfsetstrokecolor{currentstroke}%
\pgfsetstrokeopacity{0.580295}%
\pgfsetdash{}{0pt}%
\pgfpathmoveto{\pgfqpoint{5.026090in}{-1.640549in}}%
\pgfpathcurveto{\pgfqpoint{5.035298in}{-1.640549in}}{\pgfqpoint{5.044130in}{-1.636891in}}{\pgfqpoint{5.050642in}{-1.630380in}}%
\pgfpathcurveto{\pgfqpoint{5.057153in}{-1.623868in}}{\pgfqpoint{5.060812in}{-1.615036in}}{\pgfqpoint{5.060812in}{-1.605827in}}%
\pgfpathcurveto{\pgfqpoint{5.060812in}{-1.596619in}}{\pgfqpoint{5.057153in}{-1.587786in}}{\pgfqpoint{5.050642in}{-1.581275in}}%
\pgfpathcurveto{\pgfqpoint{5.044130in}{-1.574764in}}{\pgfqpoint{5.035298in}{-1.571105in}}{\pgfqpoint{5.026090in}{-1.571105in}}%
\pgfpathcurveto{\pgfqpoint{5.016881in}{-1.571105in}}{\pgfqpoint{5.008049in}{-1.574764in}}{\pgfqpoint{5.001537in}{-1.581275in}}%
\pgfpathcurveto{\pgfqpoint{4.995026in}{-1.587786in}}{\pgfqpoint{4.991367in}{-1.596619in}}{\pgfqpoint{4.991367in}{-1.605827in}}%
\pgfpathcurveto{\pgfqpoint{4.991367in}{-1.615036in}}{\pgfqpoint{4.995026in}{-1.623868in}}{\pgfqpoint{5.001537in}{-1.630380in}}%
\pgfpathcurveto{\pgfqpoint{5.008049in}{-1.636891in}}{\pgfqpoint{5.016881in}{-1.640549in}}{\pgfqpoint{5.026090in}{-1.640549in}}%
\pgfpathlineto{\pgfqpoint{5.026090in}{-1.640549in}}%
\pgfpathclose%
\pgfusepath{stroke,fill}%
\end{pgfscope}%
\begin{pgfscope}%
\pgfpathrectangle{\pgfqpoint{1.374500in}{0.082500in}}{\pgfqpoint{2.419000in}{2.419000in}}%
\pgfusepath{clip}%
\pgfsetbuttcap%
\pgfsetroundjoin%
\definecolor{currentfill}{rgb}{0.247059,0.564706,0.854902}%
\pgfsetfillcolor{currentfill}%
\pgfsetfillopacity{0.580295}%
\pgfsetlinewidth{1.003750pt}%
\definecolor{currentstroke}{rgb}{0.247059,0.564706,0.854902}%
\pgfsetstrokecolor{currentstroke}%
\pgfsetstrokeopacity{0.580295}%
\pgfsetdash{}{0pt}%
\pgfpathmoveto{\pgfqpoint{-1.586471in}{-1.640549in}}%
\pgfpathcurveto{\pgfqpoint{-1.577263in}{-1.640549in}}{\pgfqpoint{-1.568430in}{-1.636891in}}{\pgfqpoint{-1.561919in}{-1.630380in}}%
\pgfpathcurveto{\pgfqpoint{-1.555408in}{-1.623868in}}{\pgfqpoint{-1.551749in}{-1.615036in}}{\pgfqpoint{-1.551749in}{-1.605827in}}%
\pgfpathcurveto{\pgfqpoint{-1.551749in}{-1.596619in}}{\pgfqpoint{-1.555408in}{-1.587786in}}{\pgfqpoint{-1.561919in}{-1.581275in}}%
\pgfpathcurveto{\pgfqpoint{-1.568430in}{-1.574764in}}{\pgfqpoint{-1.577263in}{-1.571105in}}{\pgfqpoint{-1.586471in}{-1.571105in}}%
\pgfpathcurveto{\pgfqpoint{-1.595680in}{-1.571105in}}{\pgfqpoint{-1.604512in}{-1.574764in}}{\pgfqpoint{-1.611024in}{-1.581275in}}%
\pgfpathcurveto{\pgfqpoint{-1.617535in}{-1.587786in}}{\pgfqpoint{-1.621193in}{-1.596619in}}{\pgfqpoint{-1.621193in}{-1.605827in}}%
\pgfpathcurveto{\pgfqpoint{-1.621193in}{-1.615036in}}{\pgfqpoint{-1.617535in}{-1.623868in}}{\pgfqpoint{-1.611024in}{-1.630380in}}%
\pgfpathcurveto{\pgfqpoint{-1.604512in}{-1.636891in}}{\pgfqpoint{-1.595680in}{-1.640549in}}{\pgfqpoint{-1.586471in}{-1.640549in}}%
\pgfpathlineto{\pgfqpoint{-1.586471in}{-1.640549in}}%
\pgfpathclose%
\pgfusepath{stroke,fill}%
\end{pgfscope}%
\begin{pgfscope}%
\pgfpathrectangle{\pgfqpoint{1.374500in}{0.082500in}}{\pgfqpoint{2.419000in}{2.419000in}}%
\pgfusepath{clip}%
\pgfsetbuttcap%
\pgfsetroundjoin%
\definecolor{currentfill}{rgb}{0.247059,0.564706,0.854902}%
\pgfsetfillcolor{currentfill}%
\pgfsetfillopacity{0.580295}%
\pgfsetlinewidth{1.003750pt}%
\definecolor{currentstroke}{rgb}{0.247059,0.564706,0.854902}%
\pgfsetstrokecolor{currentstroke}%
\pgfsetstrokeopacity{0.580295}%
\pgfsetdash{}{0pt}%
\pgfpathmoveto{\pgfqpoint{11.638650in}{-1.640549in}}%
\pgfpathcurveto{\pgfqpoint{11.647859in}{-1.640549in}}{\pgfqpoint{11.656691in}{-1.636891in}}{\pgfqpoint{11.663203in}{-1.630380in}}%
\pgfpathcurveto{\pgfqpoint{11.669714in}{-1.623868in}}{\pgfqpoint{11.673372in}{-1.615036in}}{\pgfqpoint{11.673372in}{-1.605827in}}%
\pgfpathcurveto{\pgfqpoint{11.673372in}{-1.596619in}}{\pgfqpoint{11.669714in}{-1.587786in}}{\pgfqpoint{11.663203in}{-1.581275in}}%
\pgfpathcurveto{\pgfqpoint{11.656691in}{-1.574764in}}{\pgfqpoint{11.647859in}{-1.571105in}}{\pgfqpoint{11.638650in}{-1.571105in}}%
\pgfpathcurveto{\pgfqpoint{11.629442in}{-1.571105in}}{\pgfqpoint{11.620609in}{-1.574764in}}{\pgfqpoint{11.614098in}{-1.581275in}}%
\pgfpathcurveto{\pgfqpoint{11.607587in}{-1.587786in}}{\pgfqpoint{11.603928in}{-1.596619in}}{\pgfqpoint{11.603928in}{-1.605827in}}%
\pgfpathcurveto{\pgfqpoint{11.603928in}{-1.615036in}}{\pgfqpoint{11.607587in}{-1.623868in}}{\pgfqpoint{11.614098in}{-1.630380in}}%
\pgfpathcurveto{\pgfqpoint{11.620609in}{-1.636891in}}{\pgfqpoint{11.629442in}{-1.640549in}}{\pgfqpoint{11.638650in}{-1.640549in}}%
\pgfpathlineto{\pgfqpoint{11.638650in}{-1.640549in}}%
\pgfpathclose%
\pgfusepath{stroke,fill}%
\end{pgfscope}%
\begin{pgfscope}%
\pgfpathrectangle{\pgfqpoint{1.374500in}{0.082500in}}{\pgfqpoint{2.419000in}{2.419000in}}%
\pgfusepath{clip}%
\pgfsetbuttcap%
\pgfsetroundjoin%
\definecolor{currentfill}{rgb}{0.247059,0.564706,0.854902}%
\pgfsetfillcolor{currentfill}%
\pgfsetfillopacity{0.592730}%
\pgfsetlinewidth{1.003750pt}%
\definecolor{currentstroke}{rgb}{0.247059,0.564706,0.854902}%
\pgfsetstrokecolor{currentstroke}%
\pgfsetstrokeopacity{0.592730}%
\pgfsetdash{}{0pt}%
\pgfpathmoveto{\pgfqpoint{-4.440108in}{-1.895008in}}%
\pgfpathcurveto{\pgfqpoint{-4.430900in}{-1.895008in}}{\pgfqpoint{-4.422067in}{-1.891350in}}{\pgfqpoint{-4.415556in}{-1.884838in}}%
\pgfpathcurveto{\pgfqpoint{-4.409044in}{-1.878327in}}{\pgfqpoint{-4.405386in}{-1.869494in}}{\pgfqpoint{-4.405386in}{-1.860286in}}%
\pgfpathcurveto{\pgfqpoint{-4.405386in}{-1.851078in}}{\pgfqpoint{-4.409044in}{-1.842245in}}{\pgfqpoint{-4.415556in}{-1.835734in}}%
\pgfpathcurveto{\pgfqpoint{-4.422067in}{-1.829222in}}{\pgfqpoint{-4.430900in}{-1.825564in}}{\pgfqpoint{-4.440108in}{-1.825564in}}%
\pgfpathcurveto{\pgfqpoint{-4.449316in}{-1.825564in}}{\pgfqpoint{-4.458149in}{-1.829222in}}{\pgfqpoint{-4.464660in}{-1.835734in}}%
\pgfpathcurveto{\pgfqpoint{-4.471172in}{-1.842245in}}{\pgfqpoint{-4.474830in}{-1.851078in}}{\pgfqpoint{-4.474830in}{-1.860286in}}%
\pgfpathcurveto{\pgfqpoint{-4.474830in}{-1.869494in}}{\pgfqpoint{-4.471172in}{-1.878327in}}{\pgfqpoint{-4.464660in}{-1.884838in}}%
\pgfpathcurveto{\pgfqpoint{-4.458149in}{-1.891350in}}{\pgfqpoint{-4.449316in}{-1.895008in}}{\pgfqpoint{-4.440108in}{-1.895008in}}%
\pgfpathlineto{\pgfqpoint{-4.440108in}{-1.895008in}}%
\pgfpathclose%
\pgfusepath{stroke,fill}%
\end{pgfscope}%
\begin{pgfscope}%
\pgfpathrectangle{\pgfqpoint{1.374500in}{0.082500in}}{\pgfqpoint{2.419000in}{2.419000in}}%
\pgfusepath{clip}%
\pgfsetbuttcap%
\pgfsetroundjoin%
\definecolor{currentfill}{rgb}{0.247059,0.564706,0.854902}%
\pgfsetfillcolor{currentfill}%
\pgfsetfillopacity{0.592730}%
\pgfsetlinewidth{1.003750pt}%
\definecolor{currentstroke}{rgb}{0.247059,0.564706,0.854902}%
\pgfsetstrokecolor{currentstroke}%
\pgfsetstrokeopacity{0.592730}%
\pgfsetdash{}{0pt}%
\pgfpathmoveto{\pgfqpoint{2.333012in}{-1.895008in}}%
\pgfpathcurveto{\pgfqpoint{2.342221in}{-1.895008in}}{\pgfqpoint{2.351053in}{-1.891350in}}{\pgfqpoint{2.357564in}{-1.884838in}}%
\pgfpathcurveto{\pgfqpoint{2.364076in}{-1.878327in}}{\pgfqpoint{2.367734in}{-1.869494in}}{\pgfqpoint{2.367734in}{-1.860286in}}%
\pgfpathcurveto{\pgfqpoint{2.367734in}{-1.851078in}}{\pgfqpoint{2.364076in}{-1.842245in}}{\pgfqpoint{2.357564in}{-1.835734in}}%
\pgfpathcurveto{\pgfqpoint{2.351053in}{-1.829222in}}{\pgfqpoint{2.342221in}{-1.825564in}}{\pgfqpoint{2.333012in}{-1.825564in}}%
\pgfpathcurveto{\pgfqpoint{2.323804in}{-1.825564in}}{\pgfqpoint{2.314971in}{-1.829222in}}{\pgfqpoint{2.308460in}{-1.835734in}}%
\pgfpathcurveto{\pgfqpoint{2.301948in}{-1.842245in}}{\pgfqpoint{2.298290in}{-1.851078in}}{\pgfqpoint{2.298290in}{-1.860286in}}%
\pgfpathcurveto{\pgfqpoint{2.298290in}{-1.869494in}}{\pgfqpoint{2.301948in}{-1.878327in}}{\pgfqpoint{2.308460in}{-1.884838in}}%
\pgfpathcurveto{\pgfqpoint{2.314971in}{-1.891350in}}{\pgfqpoint{2.323804in}{-1.895008in}}{\pgfqpoint{2.333012in}{-1.895008in}}%
\pgfpathlineto{\pgfqpoint{2.333012in}{-1.895008in}}%
\pgfpathclose%
\pgfusepath{stroke,fill}%
\end{pgfscope}%
\begin{pgfscope}%
\pgfpathrectangle{\pgfqpoint{1.374500in}{0.082500in}}{\pgfqpoint{2.419000in}{2.419000in}}%
\pgfusepath{clip}%
\pgfsetbuttcap%
\pgfsetroundjoin%
\definecolor{currentfill}{rgb}{0.247059,0.564706,0.854902}%
\pgfsetfillcolor{currentfill}%
\pgfsetfillopacity{0.592730}%
\pgfsetlinewidth{1.003750pt}%
\definecolor{currentstroke}{rgb}{0.247059,0.564706,0.854902}%
\pgfsetstrokecolor{currentstroke}%
\pgfsetstrokeopacity{0.592730}%
\pgfsetdash{}{0pt}%
\pgfpathmoveto{\pgfqpoint{9.106132in}{-1.895008in}}%
\pgfpathcurveto{\pgfqpoint{9.115341in}{-1.895008in}}{\pgfqpoint{9.124173in}{-1.891350in}}{\pgfqpoint{9.130684in}{-1.884838in}}%
\pgfpathcurveto{\pgfqpoint{9.137196in}{-1.878327in}}{\pgfqpoint{9.140854in}{-1.869494in}}{\pgfqpoint{9.140854in}{-1.860286in}}%
\pgfpathcurveto{\pgfqpoint{9.140854in}{-1.851078in}}{\pgfqpoint{9.137196in}{-1.842245in}}{\pgfqpoint{9.130684in}{-1.835734in}}%
\pgfpathcurveto{\pgfqpoint{9.124173in}{-1.829222in}}{\pgfqpoint{9.115341in}{-1.825564in}}{\pgfqpoint{9.106132in}{-1.825564in}}%
\pgfpathcurveto{\pgfqpoint{9.096924in}{-1.825564in}}{\pgfqpoint{9.088091in}{-1.829222in}}{\pgfqpoint{9.081580in}{-1.835734in}}%
\pgfpathcurveto{\pgfqpoint{9.075068in}{-1.842245in}}{\pgfqpoint{9.071410in}{-1.851078in}}{\pgfqpoint{9.071410in}{-1.860286in}}%
\pgfpathcurveto{\pgfqpoint{9.071410in}{-1.869494in}}{\pgfqpoint{9.075068in}{-1.878327in}}{\pgfqpoint{9.081580in}{-1.884838in}}%
\pgfpathcurveto{\pgfqpoint{9.088091in}{-1.891350in}}{\pgfqpoint{9.096924in}{-1.895008in}}{\pgfqpoint{9.106132in}{-1.895008in}}%
\pgfpathlineto{\pgfqpoint{9.106132in}{-1.895008in}}%
\pgfpathclose%
\pgfusepath{stroke,fill}%
\end{pgfscope}%
\begin{pgfscope}%
\pgfpathrectangle{\pgfqpoint{1.374500in}{0.082500in}}{\pgfqpoint{2.419000in}{2.419000in}}%
\pgfusepath{clip}%
\pgfsetbuttcap%
\pgfsetroundjoin%
\definecolor{currentfill}{rgb}{0.247059,0.564706,0.854902}%
\pgfsetfillcolor{currentfill}%
\pgfsetfillopacity{0.605784}%
\pgfsetlinewidth{1.003750pt}%
\definecolor{currentstroke}{rgb}{0.247059,0.564706,0.854902}%
\pgfsetstrokecolor{currentstroke}%
\pgfsetstrokeopacity{0.605784}%
\pgfsetdash{}{0pt}%
\pgfpathmoveto{\pgfqpoint{-0.494101in}{-2.162131in}}%
\pgfpathcurveto{\pgfqpoint{-0.484892in}{-2.162131in}}{\pgfqpoint{-0.476060in}{-2.158473in}}{\pgfqpoint{-0.469549in}{-2.151962in}}%
\pgfpathcurveto{\pgfqpoint{-0.463037in}{-2.145450in}}{\pgfqpoint{-0.459379in}{-2.136618in}}{\pgfqpoint{-0.459379in}{-2.127409in}}%
\pgfpathcurveto{\pgfqpoint{-0.459379in}{-2.118201in}}{\pgfqpoint{-0.463037in}{-2.109368in}}{\pgfqpoint{-0.469549in}{-2.102857in}}%
\pgfpathcurveto{\pgfqpoint{-0.476060in}{-2.096346in}}{\pgfqpoint{-0.484892in}{-2.092687in}}{\pgfqpoint{-0.494101in}{-2.092687in}}%
\pgfpathcurveto{\pgfqpoint{-0.503309in}{-2.092687in}}{\pgfqpoint{-0.512142in}{-2.096346in}}{\pgfqpoint{-0.518653in}{-2.102857in}}%
\pgfpathcurveto{\pgfqpoint{-0.525165in}{-2.109368in}}{\pgfqpoint{-0.528823in}{-2.118201in}}{\pgfqpoint{-0.528823in}{-2.127409in}}%
\pgfpathcurveto{\pgfqpoint{-0.528823in}{-2.136618in}}{\pgfqpoint{-0.525165in}{-2.145450in}}{\pgfqpoint{-0.518653in}{-2.151962in}}%
\pgfpathcurveto{\pgfqpoint{-0.512142in}{-2.158473in}}{\pgfqpoint{-0.503309in}{-2.162131in}}{\pgfqpoint{-0.494101in}{-2.162131in}}%
\pgfpathlineto{\pgfqpoint{-0.494101in}{-2.162131in}}%
\pgfpathclose%
\pgfusepath{stroke,fill}%
\end{pgfscope}%
\begin{pgfscope}%
\pgfpathrectangle{\pgfqpoint{1.374500in}{0.082500in}}{\pgfqpoint{2.419000in}{2.419000in}}%
\pgfusepath{clip}%
\pgfsetbuttcap%
\pgfsetroundjoin%
\definecolor{currentfill}{rgb}{0.247059,0.564706,0.854902}%
\pgfsetfillcolor{currentfill}%
\pgfsetfillopacity{0.605784}%
\pgfsetlinewidth{1.003750pt}%
\definecolor{currentstroke}{rgb}{0.247059,0.564706,0.854902}%
\pgfsetstrokecolor{currentstroke}%
\pgfsetstrokeopacity{0.605784}%
\pgfsetdash{}{0pt}%
\pgfpathmoveto{\pgfqpoint{6.447570in}{-2.162131in}}%
\pgfpathcurveto{\pgfqpoint{6.456778in}{-2.162131in}}{\pgfqpoint{6.465611in}{-2.158473in}}{\pgfqpoint{6.472122in}{-2.151962in}}%
\pgfpathcurveto{\pgfqpoint{6.478633in}{-2.145450in}}{\pgfqpoint{6.482292in}{-2.136618in}}{\pgfqpoint{6.482292in}{-2.127409in}}%
\pgfpathcurveto{\pgfqpoint{6.482292in}{-2.118201in}}{\pgfqpoint{6.478633in}{-2.109368in}}{\pgfqpoint{6.472122in}{-2.102857in}}%
\pgfpathcurveto{\pgfqpoint{6.465611in}{-2.096346in}}{\pgfqpoint{6.456778in}{-2.092687in}}{\pgfqpoint{6.447570in}{-2.092687in}}%
\pgfpathcurveto{\pgfqpoint{6.438361in}{-2.092687in}}{\pgfqpoint{6.429529in}{-2.096346in}}{\pgfqpoint{6.423017in}{-2.102857in}}%
\pgfpathcurveto{\pgfqpoint{6.416506in}{-2.109368in}}{\pgfqpoint{6.412847in}{-2.118201in}}{\pgfqpoint{6.412847in}{-2.127409in}}%
\pgfpathcurveto{\pgfqpoint{6.412847in}{-2.136618in}}{\pgfqpoint{6.416506in}{-2.145450in}}{\pgfqpoint{6.423017in}{-2.151962in}}%
\pgfpathcurveto{\pgfqpoint{6.429529in}{-2.158473in}}{\pgfqpoint{6.438361in}{-2.162131in}}{\pgfqpoint{6.447570in}{-2.162131in}}%
\pgfpathlineto{\pgfqpoint{6.447570in}{-2.162131in}}%
\pgfpathclose%
\pgfusepath{stroke,fill}%
\end{pgfscope}%
\begin{pgfscope}%
\pgfpathrectangle{\pgfqpoint{1.374500in}{0.082500in}}{\pgfqpoint{2.419000in}{2.419000in}}%
\pgfusepath{clip}%
\pgfsetbuttcap%
\pgfsetroundjoin%
\definecolor{currentfill}{rgb}{0.247059,0.564706,0.854902}%
\pgfsetfillcolor{currentfill}%
\pgfsetfillopacity{0.619505}%
\pgfsetlinewidth{1.003750pt}%
\definecolor{currentstroke}{rgb}{0.247059,0.564706,0.854902}%
\pgfsetstrokecolor{currentstroke}%
\pgfsetstrokeopacity{0.619505}%
\pgfsetdash{}{0pt}%
\pgfpathmoveto{\pgfqpoint{3.653312in}{-2.442889in}}%
\pgfpathcurveto{\pgfqpoint{3.662521in}{-2.442889in}}{\pgfqpoint{3.671353in}{-2.439230in}}{\pgfqpoint{3.677865in}{-2.432719in}}%
\pgfpathcurveto{\pgfqpoint{3.684376in}{-2.426207in}}{\pgfqpoint{3.688035in}{-2.417375in}}{\pgfqpoint{3.688035in}{-2.408167in}}%
\pgfpathcurveto{\pgfqpoint{3.688035in}{-2.398958in}}{\pgfqpoint{3.684376in}{-2.390126in}}{\pgfqpoint{3.677865in}{-2.383614in}}%
\pgfpathcurveto{\pgfqpoint{3.671353in}{-2.377103in}}{\pgfqpoint{3.662521in}{-2.373444in}}{\pgfqpoint{3.653312in}{-2.373444in}}%
\pgfpathcurveto{\pgfqpoint{3.644104in}{-2.373444in}}{\pgfqpoint{3.635272in}{-2.377103in}}{\pgfqpoint{3.628760in}{-2.383614in}}%
\pgfpathcurveto{\pgfqpoint{3.622249in}{-2.390126in}}{\pgfqpoint{3.618590in}{-2.398958in}}{\pgfqpoint{3.618590in}{-2.408167in}}%
\pgfpathcurveto{\pgfqpoint{3.618590in}{-2.417375in}}{\pgfqpoint{3.622249in}{-2.426207in}}{\pgfqpoint{3.628760in}{-2.432719in}}%
\pgfpathcurveto{\pgfqpoint{3.635272in}{-2.439230in}}{\pgfqpoint{3.644104in}{-2.442889in}}{\pgfqpoint{3.653312in}{-2.442889in}}%
\pgfpathlineto{\pgfqpoint{3.653312in}{-2.442889in}}%
\pgfpathclose%
\pgfusepath{stroke,fill}%
\end{pgfscope}%
\begin{pgfscope}%
\pgfpathrectangle{\pgfqpoint{1.374500in}{0.082500in}}{\pgfqpoint{2.419000in}{2.419000in}}%
\pgfusepath{clip}%
\pgfsetbuttcap%
\pgfsetroundjoin%
\definecolor{currentfill}{rgb}{0.247059,0.564706,0.854902}%
\pgfsetfillcolor{currentfill}%
\pgfsetfillopacity{0.619505}%
\pgfsetlinewidth{1.003750pt}%
\definecolor{currentstroke}{rgb}{0.247059,0.564706,0.854902}%
\pgfsetstrokecolor{currentstroke}%
\pgfsetstrokeopacity{0.619505}%
\pgfsetdash{}{0pt}%
\pgfpathmoveto{\pgfqpoint{-3.465511in}{-2.442889in}}%
\pgfpathcurveto{\pgfqpoint{-3.456303in}{-2.442889in}}{\pgfqpoint{-3.447470in}{-2.439230in}}{\pgfqpoint{-3.440959in}{-2.432719in}}%
\pgfpathcurveto{\pgfqpoint{-3.434448in}{-2.426207in}}{\pgfqpoint{-3.430789in}{-2.417375in}}{\pgfqpoint{-3.430789in}{-2.408167in}}%
\pgfpathcurveto{\pgfqpoint{-3.430789in}{-2.398958in}}{\pgfqpoint{-3.434448in}{-2.390126in}}{\pgfqpoint{-3.440959in}{-2.383614in}}%
\pgfpathcurveto{\pgfqpoint{-3.447470in}{-2.377103in}}{\pgfqpoint{-3.456303in}{-2.373444in}}{\pgfqpoint{-3.465511in}{-2.373444in}}%
\pgfpathcurveto{\pgfqpoint{-3.474720in}{-2.373444in}}{\pgfqpoint{-3.483552in}{-2.377103in}}{\pgfqpoint{-3.490064in}{-2.383614in}}%
\pgfpathcurveto{\pgfqpoint{-3.496575in}{-2.390126in}}{\pgfqpoint{-3.500234in}{-2.398958in}}{\pgfqpoint{-3.500234in}{-2.408167in}}%
\pgfpathcurveto{\pgfqpoint{-3.500234in}{-2.417375in}}{\pgfqpoint{-3.496575in}{-2.426207in}}{\pgfqpoint{-3.490064in}{-2.432719in}}%
\pgfpathcurveto{\pgfqpoint{-3.483552in}{-2.439230in}}{\pgfqpoint{-3.474720in}{-2.442889in}}{\pgfqpoint{-3.465511in}{-2.442889in}}%
\pgfpathlineto{\pgfqpoint{-3.465511in}{-2.442889in}}%
\pgfpathclose%
\pgfusepath{stroke,fill}%
\end{pgfscope}%
\begin{pgfscope}%
\pgfpathrectangle{\pgfqpoint{1.374500in}{0.082500in}}{\pgfqpoint{2.419000in}{2.419000in}}%
\pgfusepath{clip}%
\pgfsetbuttcap%
\pgfsetroundjoin%
\definecolor{currentfill}{rgb}{0.247059,0.564706,0.854902}%
\pgfsetfillcolor{currentfill}%
\pgfsetfillopacity{0.619505}%
\pgfsetlinewidth{1.003750pt}%
\definecolor{currentstroke}{rgb}{0.247059,0.564706,0.854902}%
\pgfsetstrokecolor{currentstroke}%
\pgfsetstrokeopacity{0.619505}%
\pgfsetdash{}{0pt}%
\pgfpathmoveto{\pgfqpoint{10.772136in}{-2.442889in}}%
\pgfpathcurveto{\pgfqpoint{10.781345in}{-2.442889in}}{\pgfqpoint{10.790177in}{-2.439230in}}{\pgfqpoint{10.796689in}{-2.432719in}}%
\pgfpathcurveto{\pgfqpoint{10.803200in}{-2.426207in}}{\pgfqpoint{10.806859in}{-2.417375in}}{\pgfqpoint{10.806859in}{-2.408167in}}%
\pgfpathcurveto{\pgfqpoint{10.806859in}{-2.398958in}}{\pgfqpoint{10.803200in}{-2.390126in}}{\pgfqpoint{10.796689in}{-2.383614in}}%
\pgfpathcurveto{\pgfqpoint{10.790177in}{-2.377103in}}{\pgfqpoint{10.781345in}{-2.373444in}}{\pgfqpoint{10.772136in}{-2.373444in}}%
\pgfpathcurveto{\pgfqpoint{10.762928in}{-2.373444in}}{\pgfqpoint{10.754095in}{-2.377103in}}{\pgfqpoint{10.747584in}{-2.383614in}}%
\pgfpathcurveto{\pgfqpoint{10.741073in}{-2.390126in}}{\pgfqpoint{10.737414in}{-2.398958in}}{\pgfqpoint{10.737414in}{-2.408167in}}%
\pgfpathcurveto{\pgfqpoint{10.737414in}{-2.417375in}}{\pgfqpoint{10.741073in}{-2.426207in}}{\pgfqpoint{10.747584in}{-2.432719in}}%
\pgfpathcurveto{\pgfqpoint{10.754095in}{-2.439230in}}{\pgfqpoint{10.762928in}{-2.442889in}}{\pgfqpoint{10.772136in}{-2.442889in}}%
\pgfpathlineto{\pgfqpoint{10.772136in}{-2.442889in}}%
\pgfpathclose%
\pgfusepath{stroke,fill}%
\end{pgfscope}%
\begin{pgfscope}%
\pgfpathrectangle{\pgfqpoint{1.374500in}{0.082500in}}{\pgfqpoint{2.419000in}{2.419000in}}%
\pgfusepath{clip}%
\pgfsetbuttcap%
\pgfsetroundjoin%
\definecolor{currentfill}{rgb}{0.247059,0.564706,0.854902}%
\pgfsetfillcolor{currentfill}%
\pgfsetfillopacity{0.633944}%
\pgfsetlinewidth{1.003750pt}%
\definecolor{currentstroke}{rgb}{0.247059,0.564706,0.854902}%
\pgfsetstrokecolor{currentstroke}%
\pgfsetstrokeopacity{0.633944}%
\pgfsetdash{}{0pt}%
\pgfpathmoveto{\pgfqpoint{0.712700in}{-2.738351in}}%
\pgfpathcurveto{\pgfqpoint{0.721908in}{-2.738351in}}{\pgfqpoint{0.730741in}{-2.734693in}}{\pgfqpoint{0.737252in}{-2.728181in}}%
\pgfpathcurveto{\pgfqpoint{0.743764in}{-2.721670in}}{\pgfqpoint{0.747422in}{-2.712838in}}{\pgfqpoint{0.747422in}{-2.703629in}}%
\pgfpathcurveto{\pgfqpoint{0.747422in}{-2.694421in}}{\pgfqpoint{0.743764in}{-2.685588in}}{\pgfqpoint{0.737252in}{-2.679077in}}%
\pgfpathcurveto{\pgfqpoint{0.730741in}{-2.672565in}}{\pgfqpoint{0.721908in}{-2.668907in}}{\pgfqpoint{0.712700in}{-2.668907in}}%
\pgfpathcurveto{\pgfqpoint{0.703492in}{-2.668907in}}{\pgfqpoint{0.694659in}{-2.672565in}}{\pgfqpoint{0.688148in}{-2.679077in}}%
\pgfpathcurveto{\pgfqpoint{0.681636in}{-2.685588in}}{\pgfqpoint{0.677978in}{-2.694421in}}{\pgfqpoint{0.677978in}{-2.703629in}}%
\pgfpathcurveto{\pgfqpoint{0.677978in}{-2.712838in}}{\pgfqpoint{0.681636in}{-2.721670in}}{\pgfqpoint{0.688148in}{-2.728181in}}%
\pgfpathcurveto{\pgfqpoint{0.694659in}{-2.734693in}}{\pgfqpoint{0.703492in}{-2.738351in}}{\pgfqpoint{0.712700in}{-2.738351in}}%
\pgfpathlineto{\pgfqpoint{0.712700in}{-2.738351in}}%
\pgfpathclose%
\pgfusepath{stroke,fill}%
\end{pgfscope}%
\begin{pgfscope}%
\pgfpathrectangle{\pgfqpoint{1.374500in}{0.082500in}}{\pgfqpoint{2.419000in}{2.419000in}}%
\pgfusepath{clip}%
\pgfsetbuttcap%
\pgfsetroundjoin%
\definecolor{currentfill}{rgb}{0.247059,0.564706,0.854902}%
\pgfsetfillcolor{currentfill}%
\pgfsetfillopacity{0.633944}%
\pgfsetlinewidth{1.003750pt}%
\definecolor{currentstroke}{rgb}{0.247059,0.564706,0.854902}%
\pgfsetstrokecolor{currentstroke}%
\pgfsetstrokeopacity{0.633944}%
\pgfsetdash{}{0pt}%
\pgfpathmoveto{\pgfqpoint{8.017956in}{-2.738351in}}%
\pgfpathcurveto{\pgfqpoint{8.027164in}{-2.738351in}}{\pgfqpoint{8.035997in}{-2.734693in}}{\pgfqpoint{8.042508in}{-2.728181in}}%
\pgfpathcurveto{\pgfqpoint{8.049020in}{-2.721670in}}{\pgfqpoint{8.052678in}{-2.712838in}}{\pgfqpoint{8.052678in}{-2.703629in}}%
\pgfpathcurveto{\pgfqpoint{8.052678in}{-2.694421in}}{\pgfqpoint{8.049020in}{-2.685588in}}{\pgfqpoint{8.042508in}{-2.679077in}}%
\pgfpathcurveto{\pgfqpoint{8.035997in}{-2.672565in}}{\pgfqpoint{8.027164in}{-2.668907in}}{\pgfqpoint{8.017956in}{-2.668907in}}%
\pgfpathcurveto{\pgfqpoint{8.008748in}{-2.668907in}}{\pgfqpoint{7.999915in}{-2.672565in}}{\pgfqpoint{7.993404in}{-2.679077in}}%
\pgfpathcurveto{\pgfqpoint{7.986892in}{-2.685588in}}{\pgfqpoint{7.983234in}{-2.694421in}}{\pgfqpoint{7.983234in}{-2.703629in}}%
\pgfpathcurveto{\pgfqpoint{7.983234in}{-2.712838in}}{\pgfqpoint{7.986892in}{-2.721670in}}{\pgfqpoint{7.993404in}{-2.728181in}}%
\pgfpathcurveto{\pgfqpoint{7.999915in}{-2.734693in}}{\pgfqpoint{8.008748in}{-2.738351in}}{\pgfqpoint{8.017956in}{-2.738351in}}%
\pgfpathlineto{\pgfqpoint{8.017956in}{-2.738351in}}%
\pgfpathclose%
\pgfusepath{stroke,fill}%
\end{pgfscope}%
\begin{pgfscope}%
\pgfpathrectangle{\pgfqpoint{1.374500in}{0.082500in}}{\pgfqpoint{2.419000in}{2.419000in}}%
\pgfusepath{clip}%
\pgfsetbuttcap%
\pgfsetroundjoin%
\definecolor{currentfill}{rgb}{0.247059,0.564706,0.854902}%
\pgfsetfillcolor{currentfill}%
\pgfsetfillopacity{0.649160}%
\pgfsetlinewidth{1.003750pt}%
\definecolor{currentstroke}{rgb}{0.247059,0.564706,0.854902}%
\pgfsetstrokecolor{currentstroke}%
\pgfsetstrokeopacity{0.649160}%
\pgfsetdash{}{0pt}%
\pgfpathmoveto{\pgfqpoint{-2.386076in}{-3.049706in}}%
\pgfpathcurveto{\pgfqpoint{-2.376867in}{-3.049706in}}{\pgfqpoint{-2.368035in}{-3.046047in}}{\pgfqpoint{-2.361523in}{-3.039536in}}%
\pgfpathcurveto{\pgfqpoint{-2.355012in}{-3.033024in}}{\pgfqpoint{-2.351354in}{-3.024192in}}{\pgfqpoint{-2.351354in}{-3.014983in}}%
\pgfpathcurveto{\pgfqpoint{-2.351354in}{-3.005775in}}{\pgfqpoint{-2.355012in}{-2.996942in}}{\pgfqpoint{-2.361523in}{-2.990431in}}%
\pgfpathcurveto{\pgfqpoint{-2.368035in}{-2.983920in}}{\pgfqpoint{-2.376867in}{-2.980261in}}{\pgfqpoint{-2.386076in}{-2.980261in}}%
\pgfpathcurveto{\pgfqpoint{-2.395284in}{-2.980261in}}{\pgfqpoint{-2.404117in}{-2.983920in}}{\pgfqpoint{-2.410628in}{-2.990431in}}%
\pgfpathcurveto{\pgfqpoint{-2.417139in}{-2.996942in}}{\pgfqpoint{-2.420798in}{-3.005775in}}{\pgfqpoint{-2.420798in}{-3.014983in}}%
\pgfpathcurveto{\pgfqpoint{-2.420798in}{-3.024192in}}{\pgfqpoint{-2.417139in}{-3.033024in}}{\pgfqpoint{-2.410628in}{-3.039536in}}%
\pgfpathcurveto{\pgfqpoint{-2.404117in}{-3.046047in}}{\pgfqpoint{-2.395284in}{-3.049706in}}{\pgfqpoint{-2.386076in}{-3.049706in}}%
\pgfpathlineto{\pgfqpoint{-2.386076in}{-3.049706in}}%
\pgfpathclose%
\pgfusepath{stroke,fill}%
\end{pgfscope}%
\begin{pgfscope}%
\pgfpathrectangle{\pgfqpoint{1.374500in}{0.082500in}}{\pgfqpoint{2.419000in}{2.419000in}}%
\pgfusepath{clip}%
\pgfsetbuttcap%
\pgfsetroundjoin%
\definecolor{currentfill}{rgb}{0.247059,0.564706,0.854902}%
\pgfsetfillcolor{currentfill}%
\pgfsetfillopacity{0.649160}%
\pgfsetlinewidth{1.003750pt}%
\definecolor{currentstroke}{rgb}{0.247059,0.564706,0.854902}%
\pgfsetstrokecolor{currentstroke}%
\pgfsetstrokeopacity{0.649160}%
\pgfsetdash{}{0pt}%
\pgfpathmoveto{\pgfqpoint{5.115640in}{-3.049706in}}%
\pgfpathcurveto{\pgfqpoint{5.124848in}{-3.049706in}}{\pgfqpoint{5.133681in}{-3.046047in}}{\pgfqpoint{5.140192in}{-3.039536in}}%
\pgfpathcurveto{\pgfqpoint{5.146703in}{-3.033024in}}{\pgfqpoint{5.150362in}{-3.024192in}}{\pgfqpoint{5.150362in}{-3.014983in}}%
\pgfpathcurveto{\pgfqpoint{5.150362in}{-3.005775in}}{\pgfqpoint{5.146703in}{-2.996942in}}{\pgfqpoint{5.140192in}{-2.990431in}}%
\pgfpathcurveto{\pgfqpoint{5.133681in}{-2.983920in}}{\pgfqpoint{5.124848in}{-2.980261in}}{\pgfqpoint{5.115640in}{-2.980261in}}%
\pgfpathcurveto{\pgfqpoint{5.106431in}{-2.980261in}}{\pgfqpoint{5.097599in}{-2.983920in}}{\pgfqpoint{5.091087in}{-2.990431in}}%
\pgfpathcurveto{\pgfqpoint{5.084576in}{-2.996942in}}{\pgfqpoint{5.080918in}{-3.005775in}}{\pgfqpoint{5.080918in}{-3.014983in}}%
\pgfpathcurveto{\pgfqpoint{5.080918in}{-3.024192in}}{\pgfqpoint{5.084576in}{-3.033024in}}{\pgfqpoint{5.091087in}{-3.039536in}}%
\pgfpathcurveto{\pgfqpoint{5.097599in}{-3.046047in}}{\pgfqpoint{5.106431in}{-3.049706in}}{\pgfqpoint{5.115640in}{-3.049706in}}%
\pgfpathlineto{\pgfqpoint{5.115640in}{-3.049706in}}%
\pgfpathclose%
\pgfusepath{stroke,fill}%
\end{pgfscope}%
\begin{pgfscope}%
\pgfpathrectangle{\pgfqpoint{1.374500in}{0.082500in}}{\pgfqpoint{2.419000in}{2.419000in}}%
\pgfusepath{clip}%
\pgfsetbuttcap%
\pgfsetroundjoin%
\definecolor{currentfill}{rgb}{0.247059,0.564706,0.854902}%
\pgfsetfillcolor{currentfill}%
\pgfsetfillopacity{0.665216}%
\pgfsetlinewidth{1.003750pt}%
\definecolor{currentstroke}{rgb}{0.247059,0.564706,0.854902}%
\pgfsetstrokecolor{currentstroke}%
\pgfsetstrokeopacity{0.665216}%
\pgfsetdash{}{0pt}%
\pgfpathmoveto{\pgfqpoint{2.052906in}{-3.378269in}}%
\pgfpathcurveto{\pgfqpoint{2.062115in}{-3.378269in}}{\pgfqpoint{2.070947in}{-3.374611in}}{\pgfqpoint{2.077458in}{-3.368099in}}%
\pgfpathcurveto{\pgfqpoint{2.083970in}{-3.361588in}}{\pgfqpoint{2.087628in}{-3.352755in}}{\pgfqpoint{2.087628in}{-3.343547in}}%
\pgfpathcurveto{\pgfqpoint{2.087628in}{-3.334339in}}{\pgfqpoint{2.083970in}{-3.325506in}}{\pgfqpoint{2.077458in}{-3.318995in}}%
\pgfpathcurveto{\pgfqpoint{2.070947in}{-3.312483in}}{\pgfqpoint{2.062115in}{-3.308825in}}{\pgfqpoint{2.052906in}{-3.308825in}}%
\pgfpathcurveto{\pgfqpoint{2.043698in}{-3.308825in}}{\pgfqpoint{2.034865in}{-3.312483in}}{\pgfqpoint{2.028354in}{-3.318995in}}%
\pgfpathcurveto{\pgfqpoint{2.021842in}{-3.325506in}}{\pgfqpoint{2.018184in}{-3.334339in}}{\pgfqpoint{2.018184in}{-3.343547in}}%
\pgfpathcurveto{\pgfqpoint{2.018184in}{-3.352755in}}{\pgfqpoint{2.021842in}{-3.361588in}}{\pgfqpoint{2.028354in}{-3.368099in}}%
\pgfpathcurveto{\pgfqpoint{2.034865in}{-3.374611in}}{\pgfqpoint{2.043698in}{-3.378269in}}{\pgfqpoint{2.052906in}{-3.378269in}}%
\pgfpathlineto{\pgfqpoint{2.052906in}{-3.378269in}}%
\pgfpathclose%
\pgfusepath{stroke,fill}%
\end{pgfscope}%
\begin{pgfscope}%
\pgfpathrectangle{\pgfqpoint{1.374500in}{0.082500in}}{\pgfqpoint{2.419000in}{2.419000in}}%
\pgfusepath{clip}%
\pgfsetbuttcap%
\pgfsetroundjoin%
\definecolor{currentfill}{rgb}{0.247059,0.564706,0.854902}%
\pgfsetfillcolor{currentfill}%
\pgfsetfillopacity{0.665216}%
\pgfsetlinewidth{1.003750pt}%
\definecolor{currentstroke}{rgb}{0.247059,0.564706,0.854902}%
\pgfsetstrokecolor{currentstroke}%
\pgfsetstrokeopacity{0.665216}%
\pgfsetdash{}{0pt}%
\pgfpathmoveto{\pgfqpoint{9.761940in}{-3.378269in}}%
\pgfpathcurveto{\pgfqpoint{9.771148in}{-3.378269in}}{\pgfqpoint{9.779981in}{-3.374611in}}{\pgfqpoint{9.786492in}{-3.368099in}}%
\pgfpathcurveto{\pgfqpoint{9.793004in}{-3.361588in}}{\pgfqpoint{9.796662in}{-3.352755in}}{\pgfqpoint{9.796662in}{-3.343547in}}%
\pgfpathcurveto{\pgfqpoint{9.796662in}{-3.334339in}}{\pgfqpoint{9.793004in}{-3.325506in}}{\pgfqpoint{9.786492in}{-3.318995in}}%
\pgfpathcurveto{\pgfqpoint{9.779981in}{-3.312483in}}{\pgfqpoint{9.771148in}{-3.308825in}}{\pgfqpoint{9.761940in}{-3.308825in}}%
\pgfpathcurveto{\pgfqpoint{9.752732in}{-3.308825in}}{\pgfqpoint{9.743899in}{-3.312483in}}{\pgfqpoint{9.737388in}{-3.318995in}}%
\pgfpathcurveto{\pgfqpoint{9.730876in}{-3.325506in}}{\pgfqpoint{9.727218in}{-3.334339in}}{\pgfqpoint{9.727218in}{-3.343547in}}%
\pgfpathcurveto{\pgfqpoint{9.727218in}{-3.352755in}}{\pgfqpoint{9.730876in}{-3.361588in}}{\pgfqpoint{9.737388in}{-3.368099in}}%
\pgfpathcurveto{\pgfqpoint{9.743899in}{-3.374611in}}{\pgfqpoint{9.752732in}{-3.378269in}}{\pgfqpoint{9.761940in}{-3.378269in}}%
\pgfpathlineto{\pgfqpoint{9.761940in}{-3.378269in}}%
\pgfpathclose%
\pgfusepath{stroke,fill}%
\end{pgfscope}%
\begin{pgfscope}%
\pgfpathrectangle{\pgfqpoint{1.374500in}{0.082500in}}{\pgfqpoint{2.419000in}{2.419000in}}%
\pgfusepath{clip}%
\pgfsetbuttcap%
\pgfsetroundjoin%
\definecolor{currentfill}{rgb}{0.247059,0.564706,0.854902}%
\pgfsetfillcolor{currentfill}%
\pgfsetfillopacity{0.682186}%
\pgfsetlinewidth{1.003750pt}%
\definecolor{currentstroke}{rgb}{0.247059,0.564706,0.854902}%
\pgfsetstrokecolor{currentstroke}%
\pgfsetstrokeopacity{0.682186}%
\pgfsetdash{}{0pt}%
\pgfpathmoveto{\pgfqpoint{6.744214in}{-3.725509in}}%
\pgfpathcurveto{\pgfqpoint{6.753422in}{-3.725509in}}{\pgfqpoint{6.762255in}{-3.721851in}}{\pgfqpoint{6.768766in}{-3.715340in}}%
\pgfpathcurveto{\pgfqpoint{6.775277in}{-3.708828in}}{\pgfqpoint{6.778936in}{-3.699996in}}{\pgfqpoint{6.778936in}{-3.690787in}}%
\pgfpathcurveto{\pgfqpoint{6.778936in}{-3.681579in}}{\pgfqpoint{6.775277in}{-3.672746in}}{\pgfqpoint{6.768766in}{-3.666235in}}%
\pgfpathcurveto{\pgfqpoint{6.762255in}{-3.659724in}}{\pgfqpoint{6.753422in}{-3.656065in}}{\pgfqpoint{6.744214in}{-3.656065in}}%
\pgfpathcurveto{\pgfqpoint{6.735005in}{-3.656065in}}{\pgfqpoint{6.726173in}{-3.659724in}}{\pgfqpoint{6.719661in}{-3.666235in}}%
\pgfpathcurveto{\pgfqpoint{6.713150in}{-3.672746in}}{\pgfqpoint{6.709491in}{-3.681579in}}{\pgfqpoint{6.709491in}{-3.690787in}}%
\pgfpathcurveto{\pgfqpoint{6.709491in}{-3.699996in}}{\pgfqpoint{6.713150in}{-3.708828in}}{\pgfqpoint{6.719661in}{-3.715340in}}%
\pgfpathcurveto{\pgfqpoint{6.726173in}{-3.721851in}}{\pgfqpoint{6.735005in}{-3.725509in}}{\pgfqpoint{6.744214in}{-3.725509in}}%
\pgfpathlineto{\pgfqpoint{6.744214in}{-3.725509in}}%
\pgfpathclose%
\pgfusepath{stroke,fill}%
\end{pgfscope}%
\begin{pgfscope}%
\pgfpathrectangle{\pgfqpoint{1.374500in}{0.082500in}}{\pgfqpoint{2.419000in}{2.419000in}}%
\pgfusepath{clip}%
\pgfsetbuttcap%
\pgfsetroundjoin%
\definecolor{currentfill}{rgb}{0.247059,0.564706,0.854902}%
\pgfsetfillcolor{currentfill}%
\pgfsetfillopacity{0.682186}%
\pgfsetlinewidth{1.003750pt}%
\definecolor{currentstroke}{rgb}{0.247059,0.564706,0.854902}%
\pgfsetstrokecolor{currentstroke}%
\pgfsetstrokeopacity{0.682186}%
\pgfsetdash{}{0pt}%
\pgfpathmoveto{\pgfqpoint{-1.183923in}{-3.725509in}}%
\pgfpathcurveto{\pgfqpoint{-1.174715in}{-3.725509in}}{\pgfqpoint{-1.165882in}{-3.721851in}}{\pgfqpoint{-1.159371in}{-3.715340in}}%
\pgfpathcurveto{\pgfqpoint{-1.152859in}{-3.708828in}}{\pgfqpoint{-1.149201in}{-3.699996in}}{\pgfqpoint{-1.149201in}{-3.690787in}}%
\pgfpathcurveto{\pgfqpoint{-1.149201in}{-3.681579in}}{\pgfqpoint{-1.152859in}{-3.672746in}}{\pgfqpoint{-1.159371in}{-3.666235in}}%
\pgfpathcurveto{\pgfqpoint{-1.165882in}{-3.659724in}}{\pgfqpoint{-1.174715in}{-3.656065in}}{\pgfqpoint{-1.183923in}{-3.656065in}}%
\pgfpathcurveto{\pgfqpoint{-1.193132in}{-3.656065in}}{\pgfqpoint{-1.201964in}{-3.659724in}}{\pgfqpoint{-1.208475in}{-3.666235in}}%
\pgfpathcurveto{\pgfqpoint{-1.214987in}{-3.672746in}}{\pgfqpoint{-1.218645in}{-3.681579in}}{\pgfqpoint{-1.218645in}{-3.690787in}}%
\pgfpathcurveto{\pgfqpoint{-1.218645in}{-3.699996in}}{\pgfqpoint{-1.214987in}{-3.708828in}}{\pgfqpoint{-1.208475in}{-3.715340in}}%
\pgfpathcurveto{\pgfqpoint{-1.201964in}{-3.721851in}}{\pgfqpoint{-1.193132in}{-3.725509in}}{\pgfqpoint{-1.183923in}{-3.725509in}}%
\pgfpathlineto{\pgfqpoint{-1.183923in}{-3.725509in}}%
\pgfpathclose%
\pgfusepath{stroke,fill}%
\end{pgfscope}%
\begin{pgfscope}%
\pgfpathrectangle{\pgfqpoint{1.374500in}{0.082500in}}{\pgfqpoint{2.419000in}{2.419000in}}%
\pgfusepath{clip}%
\pgfsetbuttcap%
\pgfsetroundjoin%
\definecolor{currentfill}{rgb}{0.247059,0.564706,0.854902}%
\pgfsetfillcolor{currentfill}%
\pgfsetfillopacity{0.700148}%
\pgfsetlinewidth{1.003750pt}%
\definecolor{currentstroke}{rgb}{0.247059,0.564706,0.854902}%
\pgfsetstrokecolor{currentstroke}%
\pgfsetstrokeopacity{0.700148}%
\pgfsetdash{}{0pt}%
\pgfpathmoveto{\pgfqpoint{3.549932in}{-4.093065in}}%
\pgfpathcurveto{\pgfqpoint{3.559141in}{-4.093065in}}{\pgfqpoint{3.567973in}{-4.089407in}}{\pgfqpoint{3.574485in}{-4.082895in}}%
\pgfpathcurveto{\pgfqpoint{3.580996in}{-4.076384in}}{\pgfqpoint{3.584655in}{-4.067551in}}{\pgfqpoint{3.584655in}{-4.058343in}}%
\pgfpathcurveto{\pgfqpoint{3.584655in}{-4.049135in}}{\pgfqpoint{3.580996in}{-4.040302in}}{\pgfqpoint{3.574485in}{-4.033791in}}%
\pgfpathcurveto{\pgfqpoint{3.567973in}{-4.027279in}}{\pgfqpoint{3.559141in}{-4.023621in}}{\pgfqpoint{3.549932in}{-4.023621in}}%
\pgfpathcurveto{\pgfqpoint{3.540724in}{-4.023621in}}{\pgfqpoint{3.531891in}{-4.027279in}}{\pgfqpoint{3.525380in}{-4.033791in}}%
\pgfpathcurveto{\pgfqpoint{3.518869in}{-4.040302in}}{\pgfqpoint{3.515210in}{-4.049135in}}{\pgfqpoint{3.515210in}{-4.058343in}}%
\pgfpathcurveto{\pgfqpoint{3.515210in}{-4.067551in}}{\pgfqpoint{3.518869in}{-4.076384in}}{\pgfqpoint{3.525380in}{-4.082895in}}%
\pgfpathcurveto{\pgfqpoint{3.531891in}{-4.089407in}}{\pgfqpoint{3.540724in}{-4.093065in}}{\pgfqpoint{3.549932in}{-4.093065in}}%
\pgfpathlineto{\pgfqpoint{3.549932in}{-4.093065in}}%
\pgfpathclose%
\pgfusepath{stroke,fill}%
\end{pgfscope}%
\begin{pgfscope}%
\pgfpathrectangle{\pgfqpoint{1.374500in}{0.082500in}}{\pgfqpoint{2.419000in}{2.419000in}}%
\pgfusepath{clip}%
\pgfsetbuttcap%
\pgfsetroundjoin%
\definecolor{currentfill}{rgb}{0.247059,0.564706,0.854902}%
\pgfsetfillcolor{currentfill}%
\pgfsetfillopacity{0.700148}%
\pgfsetlinewidth{1.003750pt}%
\definecolor{currentstroke}{rgb}{0.247059,0.564706,0.854902}%
\pgfsetstrokecolor{currentstroke}%
\pgfsetstrokeopacity{0.700148}%
\pgfsetdash{}{0pt}%
\pgfpathmoveto{\pgfqpoint{11.709991in}{-4.093065in}}%
\pgfpathcurveto{\pgfqpoint{11.719199in}{-4.093065in}}{\pgfqpoint{11.728032in}{-4.089407in}}{\pgfqpoint{11.734543in}{-4.082895in}}%
\pgfpathcurveto{\pgfqpoint{11.741055in}{-4.076384in}}{\pgfqpoint{11.744713in}{-4.067551in}}{\pgfqpoint{11.744713in}{-4.058343in}}%
\pgfpathcurveto{\pgfqpoint{11.744713in}{-4.049135in}}{\pgfqpoint{11.741055in}{-4.040302in}}{\pgfqpoint{11.734543in}{-4.033791in}}%
\pgfpathcurveto{\pgfqpoint{11.728032in}{-4.027279in}}{\pgfqpoint{11.719199in}{-4.023621in}}{\pgfqpoint{11.709991in}{-4.023621in}}%
\pgfpathcurveto{\pgfqpoint{11.700783in}{-4.023621in}}{\pgfqpoint{11.691950in}{-4.027279in}}{\pgfqpoint{11.685439in}{-4.033791in}}%
\pgfpathcurveto{\pgfqpoint{11.678927in}{-4.040302in}}{\pgfqpoint{11.675269in}{-4.049135in}}{\pgfqpoint{11.675269in}{-4.058343in}}%
\pgfpathcurveto{\pgfqpoint{11.675269in}{-4.067551in}}{\pgfqpoint{11.678927in}{-4.076384in}}{\pgfqpoint{11.685439in}{-4.082895in}}%
\pgfpathcurveto{\pgfqpoint{11.691950in}{-4.089407in}}{\pgfqpoint{11.700783in}{-4.093065in}}{\pgfqpoint{11.709991in}{-4.093065in}}%
\pgfpathlineto{\pgfqpoint{11.709991in}{-4.093065in}}%
\pgfpathclose%
\pgfusepath{stroke,fill}%
\end{pgfscope}%
\begin{pgfscope}%
\pgfpathrectangle{\pgfqpoint{1.374500in}{0.082500in}}{\pgfqpoint{2.419000in}{2.419000in}}%
\pgfusepath{clip}%
\pgfsetbuttcap%
\pgfsetroundjoin%
\definecolor{currentfill}{rgb}{0.247059,0.564706,0.854902}%
\pgfsetfillcolor{currentfill}%
\pgfsetfillopacity{0.719193}%
\pgfsetlinewidth{1.003750pt}%
\definecolor{currentstroke}{rgb}{0.247059,0.564706,0.854902}%
\pgfsetstrokecolor{currentstroke}%
\pgfsetstrokeopacity{0.719193}%
\pgfsetdash{}{0pt}%
\pgfpathmoveto{\pgfqpoint{0.163135in}{-4.482773in}}%
\pgfpathcurveto{\pgfqpoint{0.172343in}{-4.482773in}}{\pgfqpoint{0.181176in}{-4.479115in}}{\pgfqpoint{0.187687in}{-4.472603in}}%
\pgfpathcurveto{\pgfqpoint{0.194198in}{-4.466092in}}{\pgfqpoint{0.197857in}{-4.457259in}}{\pgfqpoint{0.197857in}{-4.448051in}}%
\pgfpathcurveto{\pgfqpoint{0.197857in}{-4.438843in}}{\pgfqpoint{0.194198in}{-4.430010in}}{\pgfqpoint{0.187687in}{-4.423499in}}%
\pgfpathcurveto{\pgfqpoint{0.181176in}{-4.416987in}}{\pgfqpoint{0.172343in}{-4.413329in}}{\pgfqpoint{0.163135in}{-4.413329in}}%
\pgfpathcurveto{\pgfqpoint{0.153926in}{-4.413329in}}{\pgfqpoint{0.145094in}{-4.416987in}}{\pgfqpoint{0.138582in}{-4.423499in}}%
\pgfpathcurveto{\pgfqpoint{0.132071in}{-4.430010in}}{\pgfqpoint{0.128412in}{-4.438843in}}{\pgfqpoint{0.128412in}{-4.448051in}}%
\pgfpathcurveto{\pgfqpoint{0.128412in}{-4.457259in}}{\pgfqpoint{0.132071in}{-4.466092in}}{\pgfqpoint{0.138582in}{-4.472603in}}%
\pgfpathcurveto{\pgfqpoint{0.145094in}{-4.479115in}}{\pgfqpoint{0.153926in}{-4.482773in}}{\pgfqpoint{0.163135in}{-4.482773in}}%
\pgfpathlineto{\pgfqpoint{0.163135in}{-4.482773in}}%
\pgfpathclose%
\pgfusepath{stroke,fill}%
\end{pgfscope}%
\begin{pgfscope}%
\pgfpathrectangle{\pgfqpoint{1.374500in}{0.082500in}}{\pgfqpoint{2.419000in}{2.419000in}}%
\pgfusepath{clip}%
\pgfsetbuttcap%
\pgfsetroundjoin%
\definecolor{currentfill}{rgb}{0.247059,0.564706,0.854902}%
\pgfsetfillcolor{currentfill}%
\pgfsetfillopacity{0.719193}%
\pgfsetlinewidth{1.003750pt}%
\definecolor{currentstroke}{rgb}{0.247059,0.564706,0.854902}%
\pgfsetstrokecolor{currentstroke}%
\pgfsetstrokeopacity{0.719193}%
\pgfsetdash{}{0pt}%
\pgfpathmoveto{\pgfqpoint{8.569093in}{-4.482773in}}%
\pgfpathcurveto{\pgfqpoint{8.578301in}{-4.482773in}}{\pgfqpoint{8.587134in}{-4.479115in}}{\pgfqpoint{8.593645in}{-4.472603in}}%
\pgfpathcurveto{\pgfqpoint{8.600157in}{-4.466092in}}{\pgfqpoint{8.603815in}{-4.457259in}}{\pgfqpoint{8.603815in}{-4.448051in}}%
\pgfpathcurveto{\pgfqpoint{8.603815in}{-4.438843in}}{\pgfqpoint{8.600157in}{-4.430010in}}{\pgfqpoint{8.593645in}{-4.423499in}}%
\pgfpathcurveto{\pgfqpoint{8.587134in}{-4.416987in}}{\pgfqpoint{8.578301in}{-4.413329in}}{\pgfqpoint{8.569093in}{-4.413329in}}%
\pgfpathcurveto{\pgfqpoint{8.559884in}{-4.413329in}}{\pgfqpoint{8.551052in}{-4.416987in}}{\pgfqpoint{8.544541in}{-4.423499in}}%
\pgfpathcurveto{\pgfqpoint{8.538029in}{-4.430010in}}{\pgfqpoint{8.534371in}{-4.438843in}}{\pgfqpoint{8.534371in}{-4.448051in}}%
\pgfpathcurveto{\pgfqpoint{8.534371in}{-4.457259in}}{\pgfqpoint{8.538029in}{-4.466092in}}{\pgfqpoint{8.544541in}{-4.472603in}}%
\pgfpathcurveto{\pgfqpoint{8.551052in}{-4.479115in}}{\pgfqpoint{8.559884in}{-4.482773in}}{\pgfqpoint{8.569093in}{-4.482773in}}%
\pgfpathlineto{\pgfqpoint{8.569093in}{-4.482773in}}%
\pgfpathclose%
\pgfusepath{stroke,fill}%
\end{pgfscope}%
\begin{pgfscope}%
\pgfpathrectangle{\pgfqpoint{1.374500in}{0.082500in}}{\pgfqpoint{2.419000in}{2.419000in}}%
\pgfusepath{clip}%
\pgfsetbuttcap%
\pgfsetroundjoin%
\definecolor{currentfill}{rgb}{0.247059,0.564706,0.854902}%
\pgfsetfillcolor{currentfill}%
\pgfsetfillopacity{0.739421}%
\pgfsetlinewidth{1.003750pt}%
\definecolor{currentstroke}{rgb}{0.247059,0.564706,0.854902}%
\pgfsetstrokecolor{currentstroke}%
\pgfsetstrokeopacity{0.739421}%
\pgfsetdash{}{0pt}%
\pgfpathmoveto{\pgfqpoint{5.233014in}{-4.896698in}}%
\pgfpathcurveto{\pgfqpoint{5.242223in}{-4.896698in}}{\pgfqpoint{5.251055in}{-4.893040in}}{\pgfqpoint{5.257566in}{-4.886528in}}%
\pgfpathcurveto{\pgfqpoint{5.264078in}{-4.880017in}}{\pgfqpoint{5.267736in}{-4.871185in}}{\pgfqpoint{5.267736in}{-4.861976in}}%
\pgfpathcurveto{\pgfqpoint{5.267736in}{-4.852768in}}{\pgfqpoint{5.264078in}{-4.843935in}}{\pgfqpoint{5.257566in}{-4.837424in}}%
\pgfpathcurveto{\pgfqpoint{5.251055in}{-4.830913in}}{\pgfqpoint{5.242223in}{-4.827254in}}{\pgfqpoint{5.233014in}{-4.827254in}}%
\pgfpathcurveto{\pgfqpoint{5.223806in}{-4.827254in}}{\pgfqpoint{5.214973in}{-4.830913in}}{\pgfqpoint{5.208462in}{-4.837424in}}%
\pgfpathcurveto{\pgfqpoint{5.201950in}{-4.843935in}}{\pgfqpoint{5.198292in}{-4.852768in}}{\pgfqpoint{5.198292in}{-4.861976in}}%
\pgfpathcurveto{\pgfqpoint{5.198292in}{-4.871185in}}{\pgfqpoint{5.201950in}{-4.880017in}}{\pgfqpoint{5.208462in}{-4.886528in}}%
\pgfpathcurveto{\pgfqpoint{5.214973in}{-4.893040in}}{\pgfqpoint{5.223806in}{-4.896698in}}{\pgfqpoint{5.233014in}{-4.896698in}}%
\pgfpathlineto{\pgfqpoint{5.233014in}{-4.896698in}}%
\pgfpathclose%
\pgfusepath{stroke,fill}%
\end{pgfscope}%
\begin{pgfscope}%
\pgfpathrectangle{\pgfqpoint{1.374500in}{0.082500in}}{\pgfqpoint{2.419000in}{2.419000in}}%
\pgfusepath{clip}%
\pgfsetbuttcap%
\pgfsetroundjoin%
\definecolor{currentfill}{rgb}{0.247059,0.564706,0.854902}%
\pgfsetfillcolor{currentfill}%
\pgfsetfillopacity{0.760947}%
\pgfsetlinewidth{1.003750pt}%
\definecolor{currentstroke}{rgb}{0.247059,0.564706,0.854902}%
\pgfsetstrokecolor{currentstroke}%
\pgfsetstrokeopacity{0.760947}%
\pgfsetdash{}{0pt}%
\pgfpathmoveto{\pgfqpoint{1.682978in}{-5.337170in}}%
\pgfpathcurveto{\pgfqpoint{1.692186in}{-5.337170in}}{\pgfqpoint{1.701019in}{-5.333512in}}{\pgfqpoint{1.707530in}{-5.327000in}}%
\pgfpathcurveto{\pgfqpoint{1.714042in}{-5.320489in}}{\pgfqpoint{1.717700in}{-5.311657in}}{\pgfqpoint{1.717700in}{-5.302448in}}%
\pgfpathcurveto{\pgfqpoint{1.717700in}{-5.293240in}}{\pgfqpoint{1.714042in}{-5.284407in}}{\pgfqpoint{1.707530in}{-5.277896in}}%
\pgfpathcurveto{\pgfqpoint{1.701019in}{-5.271384in}}{\pgfqpoint{1.692186in}{-5.267726in}}{\pgfqpoint{1.682978in}{-5.267726in}}%
\pgfpathcurveto{\pgfqpoint{1.673770in}{-5.267726in}}{\pgfqpoint{1.664937in}{-5.271384in}}{\pgfqpoint{1.658426in}{-5.277896in}}%
\pgfpathcurveto{\pgfqpoint{1.651914in}{-5.284407in}}{\pgfqpoint{1.648256in}{-5.293240in}}{\pgfqpoint{1.648256in}{-5.302448in}}%
\pgfpathcurveto{\pgfqpoint{1.648256in}{-5.311657in}}{\pgfqpoint{1.651914in}{-5.320489in}}{\pgfqpoint{1.658426in}{-5.327000in}}%
\pgfpathcurveto{\pgfqpoint{1.664937in}{-5.333512in}}{\pgfqpoint{1.673770in}{-5.337170in}}{\pgfqpoint{1.682978in}{-5.337170in}}%
\pgfpathlineto{\pgfqpoint{1.682978in}{-5.337170in}}%
\pgfpathclose%
\pgfusepath{stroke,fill}%
\end{pgfscope}%
\begin{pgfscope}%
\pgfpathrectangle{\pgfqpoint{1.374500in}{0.082500in}}{\pgfqpoint{2.419000in}{2.419000in}}%
\pgfusepath{clip}%
\pgfsetbuttcap%
\pgfsetroundjoin%
\definecolor{currentfill}{rgb}{0.247059,0.564706,0.854902}%
\pgfsetfillcolor{currentfill}%
\pgfsetfillopacity{0.760947}%
\pgfsetlinewidth{1.003750pt}%
\definecolor{currentstroke}{rgb}{0.247059,0.564706,0.854902}%
\pgfsetstrokecolor{currentstroke}%
\pgfsetstrokeopacity{0.760947}%
\pgfsetdash{}{0pt}%
\pgfpathmoveto{\pgfqpoint{10.628047in}{-5.337170in}}%
\pgfpathcurveto{\pgfqpoint{10.637255in}{-5.337170in}}{\pgfqpoint{10.646088in}{-5.333512in}}{\pgfqpoint{10.652599in}{-5.327000in}}%
\pgfpathcurveto{\pgfqpoint{10.659111in}{-5.320489in}}{\pgfqpoint{10.662769in}{-5.311657in}}{\pgfqpoint{10.662769in}{-5.302448in}}%
\pgfpathcurveto{\pgfqpoint{10.662769in}{-5.293240in}}{\pgfqpoint{10.659111in}{-5.284407in}}{\pgfqpoint{10.652599in}{-5.277896in}}%
\pgfpathcurveto{\pgfqpoint{10.646088in}{-5.271384in}}{\pgfqpoint{10.637255in}{-5.267726in}}{\pgfqpoint{10.628047in}{-5.267726in}}%
\pgfpathcurveto{\pgfqpoint{10.618839in}{-5.267726in}}{\pgfqpoint{10.610006in}{-5.271384in}}{\pgfqpoint{10.603495in}{-5.277896in}}%
\pgfpathcurveto{\pgfqpoint{10.596983in}{-5.284407in}}{\pgfqpoint{10.593325in}{-5.293240in}}{\pgfqpoint{10.593325in}{-5.302448in}}%
\pgfpathcurveto{\pgfqpoint{10.593325in}{-5.311657in}}{\pgfqpoint{10.596983in}{-5.320489in}}{\pgfqpoint{10.603495in}{-5.327000in}}%
\pgfpathcurveto{\pgfqpoint{10.610006in}{-5.333512in}}{\pgfqpoint{10.618839in}{-5.337170in}}{\pgfqpoint{10.628047in}{-5.337170in}}%
\pgfpathlineto{\pgfqpoint{10.628047in}{-5.337170in}}%
\pgfpathclose%
\pgfusepath{stroke,fill}%
\end{pgfscope}%
\begin{pgfscope}%
\pgfpathrectangle{\pgfqpoint{1.374500in}{0.082500in}}{\pgfqpoint{2.419000in}{2.419000in}}%
\pgfusepath{clip}%
\pgfsetbuttcap%
\pgfsetroundjoin%
\definecolor{currentfill}{rgb}{0.247059,0.564706,0.854902}%
\pgfsetfillcolor{currentfill}%
\pgfsetfillopacity{0.783899}%
\pgfsetlinewidth{1.003750pt}%
\definecolor{currentstroke}{rgb}{0.247059,0.564706,0.854902}%
\pgfsetstrokecolor{currentstroke}%
\pgfsetstrokeopacity{0.783899}%
\pgfsetdash{}{0pt}%
\pgfpathmoveto{\pgfqpoint{7.139135in}{-5.806828in}}%
\pgfpathcurveto{\pgfqpoint{7.148343in}{-5.806828in}}{\pgfqpoint{7.157176in}{-5.803169in}}{\pgfqpoint{7.163687in}{-5.796658in}}%
\pgfpathcurveto{\pgfqpoint{7.170198in}{-5.790146in}}{\pgfqpoint{7.173857in}{-5.781314in}}{\pgfqpoint{7.173857in}{-5.772105in}}%
\pgfpathcurveto{\pgfqpoint{7.173857in}{-5.762897in}}{\pgfqpoint{7.170198in}{-5.754064in}}{\pgfqpoint{7.163687in}{-5.747553in}}%
\pgfpathcurveto{\pgfqpoint{7.157176in}{-5.741042in}}{\pgfqpoint{7.148343in}{-5.737383in}}{\pgfqpoint{7.139135in}{-5.737383in}}%
\pgfpathcurveto{\pgfqpoint{7.129926in}{-5.737383in}}{\pgfqpoint{7.121094in}{-5.741042in}}{\pgfqpoint{7.114582in}{-5.747553in}}%
\pgfpathcurveto{\pgfqpoint{7.108071in}{-5.754064in}}{\pgfqpoint{7.104413in}{-5.762897in}}{\pgfqpoint{7.104413in}{-5.772105in}}%
\pgfpathcurveto{\pgfqpoint{7.104413in}{-5.781314in}}{\pgfqpoint{7.108071in}{-5.790146in}}{\pgfqpoint{7.114582in}{-5.796658in}}%
\pgfpathcurveto{\pgfqpoint{7.121094in}{-5.803169in}}{\pgfqpoint{7.129926in}{-5.806828in}}{\pgfqpoint{7.139135in}{-5.806828in}}%
\pgfpathlineto{\pgfqpoint{7.139135in}{-5.806828in}}%
\pgfpathclose%
\pgfusepath{stroke,fill}%
\end{pgfscope}%
\begin{pgfscope}%
\pgfpathrectangle{\pgfqpoint{1.374500in}{0.082500in}}{\pgfqpoint{2.419000in}{2.419000in}}%
\pgfusepath{clip}%
\pgfsetbuttcap%
\pgfsetroundjoin%
\definecolor{currentfill}{rgb}{0.247059,0.564706,0.854902}%
\pgfsetfillcolor{currentfill}%
\pgfsetfillopacity{0.808423}%
\pgfsetlinewidth{1.003750pt}%
\definecolor{currentstroke}{rgb}{0.247059,0.564706,0.854902}%
\pgfsetstrokecolor{currentstroke}%
\pgfsetstrokeopacity{0.808423}%
\pgfsetdash{}{0pt}%
\pgfpathmoveto{\pgfqpoint{3.411129in}{-6.308670in}}%
\pgfpathcurveto{\pgfqpoint{3.420338in}{-6.308670in}}{\pgfqpoint{3.429170in}{-6.305012in}}{\pgfqpoint{3.435682in}{-6.298500in}}%
\pgfpathcurveto{\pgfqpoint{3.442193in}{-6.291989in}}{\pgfqpoint{3.445852in}{-6.283156in}}{\pgfqpoint{3.445852in}{-6.273948in}}%
\pgfpathcurveto{\pgfqpoint{3.445852in}{-6.264740in}}{\pgfqpoint{3.442193in}{-6.255907in}}{\pgfqpoint{3.435682in}{-6.249396in}}%
\pgfpathcurveto{\pgfqpoint{3.429170in}{-6.242884in}}{\pgfqpoint{3.420338in}{-6.239226in}}{\pgfqpoint{3.411129in}{-6.239226in}}%
\pgfpathcurveto{\pgfqpoint{3.401921in}{-6.239226in}}{\pgfqpoint{3.393088in}{-6.242884in}}{\pgfqpoint{3.386577in}{-6.249396in}}%
\pgfpathcurveto{\pgfqpoint{3.380066in}{-6.255907in}}{\pgfqpoint{3.376407in}{-6.264740in}}{\pgfqpoint{3.376407in}{-6.273948in}}%
\pgfpathcurveto{\pgfqpoint{3.376407in}{-6.283156in}}{\pgfqpoint{3.380066in}{-6.291989in}}{\pgfqpoint{3.386577in}{-6.298500in}}%
\pgfpathcurveto{\pgfqpoint{3.393088in}{-6.305012in}}{\pgfqpoint{3.401921in}{-6.308670in}}{\pgfqpoint{3.411129in}{-6.308670in}}%
\pgfpathlineto{\pgfqpoint{3.411129in}{-6.308670in}}%
\pgfpathclose%
\pgfusepath{stroke,fill}%
\end{pgfscope}%
\begin{pgfscope}%
\pgfpathrectangle{\pgfqpoint{1.374500in}{0.082500in}}{\pgfqpoint{2.419000in}{2.419000in}}%
\pgfusepath{clip}%
\pgfsetbuttcap%
\pgfsetroundjoin%
\definecolor{currentfill}{rgb}{0.247059,0.564706,0.854902}%
\pgfsetfillcolor{currentfill}%
\pgfsetfillopacity{0.834689}%
\pgfsetlinewidth{1.003750pt}%
\definecolor{currentstroke}{rgb}{0.247059,0.564706,0.854902}%
\pgfsetstrokecolor{currentstroke}%
\pgfsetstrokeopacity{0.834689}%
\pgfsetdash{}{0pt}%
\pgfpathmoveto{\pgfqpoint{9.315776in}{-6.846124in}}%
\pgfpathcurveto{\pgfqpoint{9.324984in}{-6.846124in}}{\pgfqpoint{9.333817in}{-6.842466in}}{\pgfqpoint{9.340328in}{-6.835954in}}%
\pgfpathcurveto{\pgfqpoint{9.346839in}{-6.829443in}}{\pgfqpoint{9.350498in}{-6.820610in}}{\pgfqpoint{9.350498in}{-6.811402in}}%
\pgfpathcurveto{\pgfqpoint{9.350498in}{-6.802193in}}{\pgfqpoint{9.346839in}{-6.793361in}}{\pgfqpoint{9.340328in}{-6.786850in}}%
\pgfpathcurveto{\pgfqpoint{9.333817in}{-6.780338in}}{\pgfqpoint{9.324984in}{-6.776680in}}{\pgfqpoint{9.315776in}{-6.776680in}}%
\pgfpathcurveto{\pgfqpoint{9.306567in}{-6.776680in}}{\pgfqpoint{9.297735in}{-6.780338in}}{\pgfqpoint{9.291223in}{-6.786850in}}%
\pgfpathcurveto{\pgfqpoint{9.284712in}{-6.793361in}}{\pgfqpoint{9.281053in}{-6.802193in}}{\pgfqpoint{9.281053in}{-6.811402in}}%
\pgfpathcurveto{\pgfqpoint{9.281053in}{-6.820610in}}{\pgfqpoint{9.284712in}{-6.829443in}}{\pgfqpoint{9.291223in}{-6.835954in}}%
\pgfpathcurveto{\pgfqpoint{9.297735in}{-6.842466in}}{\pgfqpoint{9.306567in}{-6.846124in}}{\pgfqpoint{9.315776in}{-6.846124in}}%
\pgfpathlineto{\pgfqpoint{9.315776in}{-6.846124in}}%
\pgfpathclose%
\pgfusepath{stroke,fill}%
\end{pgfscope}%
\begin{pgfscope}%
\pgfpathrectangle{\pgfqpoint{1.374500in}{0.082500in}}{\pgfqpoint{2.419000in}{2.419000in}}%
\pgfusepath{clip}%
\pgfsetbuttcap%
\pgfsetroundjoin%
\definecolor{currentfill}{rgb}{0.247059,0.564706,0.854902}%
\pgfsetfillcolor{currentfill}%
\pgfsetfillopacity{0.862886}%
\pgfsetlinewidth{1.003750pt}%
\definecolor{currentstroke}{rgb}{0.247059,0.564706,0.854902}%
\pgfsetstrokecolor{currentstroke}%
\pgfsetstrokeopacity{0.862886}%
\pgfsetdash{}{0pt}%
\pgfpathmoveto{\pgfqpoint{5.393565in}{-7.423119in}}%
\pgfpathcurveto{\pgfqpoint{5.402774in}{-7.423119in}}{\pgfqpoint{5.411606in}{-7.419460in}}{\pgfqpoint{5.418118in}{-7.412949in}}%
\pgfpathcurveto{\pgfqpoint{5.424629in}{-7.406438in}}{\pgfqpoint{5.428287in}{-7.397605in}}{\pgfqpoint{5.428287in}{-7.388397in}}%
\pgfpathcurveto{\pgfqpoint{5.428287in}{-7.379188in}}{\pgfqpoint{5.424629in}{-7.370356in}}{\pgfqpoint{5.418118in}{-7.363845in}}%
\pgfpathcurveto{\pgfqpoint{5.411606in}{-7.357333in}}{\pgfqpoint{5.402774in}{-7.353675in}}{\pgfqpoint{5.393565in}{-7.353675in}}%
\pgfpathcurveto{\pgfqpoint{5.384357in}{-7.353675in}}{\pgfqpoint{5.375524in}{-7.357333in}}{\pgfqpoint{5.369013in}{-7.363845in}}%
\pgfpathcurveto{\pgfqpoint{5.362502in}{-7.370356in}}{\pgfqpoint{5.358843in}{-7.379188in}}{\pgfqpoint{5.358843in}{-7.388397in}}%
\pgfpathcurveto{\pgfqpoint{5.358843in}{-7.397605in}}{\pgfqpoint{5.362502in}{-7.406438in}}{\pgfqpoint{5.369013in}{-7.412949in}}%
\pgfpathcurveto{\pgfqpoint{5.375524in}{-7.419460in}}{\pgfqpoint{5.384357in}{-7.423119in}}{\pgfqpoint{5.393565in}{-7.423119in}}%
\pgfpathlineto{\pgfqpoint{5.393565in}{-7.423119in}}%
\pgfpathclose%
\pgfusepath{stroke,fill}%
\end{pgfscope}%
\begin{pgfscope}%
\pgfpathrectangle{\pgfqpoint{1.374500in}{0.082500in}}{\pgfqpoint{2.419000in}{2.419000in}}%
\pgfusepath{clip}%
\pgfsetbuttcap%
\pgfsetroundjoin%
\definecolor{currentfill}{rgb}{0.247059,0.564706,0.854902}%
\pgfsetfillcolor{currentfill}%
\pgfsetfillopacity{0.893237}%
\pgfsetlinewidth{1.003750pt}%
\definecolor{currentstroke}{rgb}{0.247059,0.564706,0.854902}%
\pgfsetstrokecolor{currentstroke}%
\pgfsetstrokeopacity{0.893237}%
\pgfsetdash{}{0pt}%
\pgfpathmoveto{\pgfqpoint{11.824924in}{-8.044185in}}%
\pgfpathcurveto{\pgfqpoint{11.834133in}{-8.044185in}}{\pgfqpoint{11.842965in}{-8.040527in}}{\pgfqpoint{11.849477in}{-8.034016in}}%
\pgfpathcurveto{\pgfqpoint{11.855988in}{-8.027504in}}{\pgfqpoint{11.859647in}{-8.018672in}}{\pgfqpoint{11.859647in}{-8.009463in}}%
\pgfpathcurveto{\pgfqpoint{11.859647in}{-8.000255in}}{\pgfqpoint{11.855988in}{-7.991422in}}{\pgfqpoint{11.849477in}{-7.984911in}}%
\pgfpathcurveto{\pgfqpoint{11.842965in}{-7.978400in}}{\pgfqpoint{11.834133in}{-7.974741in}}{\pgfqpoint{11.824924in}{-7.974741in}}%
\pgfpathcurveto{\pgfqpoint{11.815716in}{-7.974741in}}{\pgfqpoint{11.806883in}{-7.978400in}}{\pgfqpoint{11.800372in}{-7.984911in}}%
\pgfpathcurveto{\pgfqpoint{11.793861in}{-7.991422in}}{\pgfqpoint{11.790202in}{-8.000255in}}{\pgfqpoint{11.790202in}{-8.009463in}}%
\pgfpathcurveto{\pgfqpoint{11.790202in}{-8.018672in}}{\pgfqpoint{11.793861in}{-8.027504in}}{\pgfqpoint{11.800372in}{-8.034016in}}%
\pgfpathcurveto{\pgfqpoint{11.806883in}{-8.040527in}}{\pgfqpoint{11.815716in}{-8.044185in}}{\pgfqpoint{11.824924in}{-8.044185in}}%
\pgfpathlineto{\pgfqpoint{11.824924in}{-8.044185in}}%
\pgfpathclose%
\pgfusepath{stroke,fill}%
\end{pgfscope}%
\begin{pgfscope}%
\pgfpathrectangle{\pgfqpoint{1.374500in}{0.082500in}}{\pgfqpoint{2.419000in}{2.419000in}}%
\pgfusepath{clip}%
\pgfsetbuttcap%
\pgfsetroundjoin%
\definecolor{currentfill}{rgb}{0.247059,0.564706,0.854902}%
\pgfsetfillcolor{currentfill}%
\pgfsetfillopacity{0.925999}%
\pgfsetlinewidth{1.003750pt}%
\definecolor{currentstroke}{rgb}{0.247059,0.564706,0.854902}%
\pgfsetstrokecolor{currentstroke}%
\pgfsetstrokeopacity{0.925999}%
\pgfsetdash{}{0pt}%
\pgfpathmoveto{\pgfqpoint{7.690867in}{-8.714573in}}%
\pgfpathcurveto{\pgfqpoint{7.700075in}{-8.714573in}}{\pgfqpoint{7.708908in}{-8.710915in}}{\pgfqpoint{7.715419in}{-8.704403in}}%
\pgfpathcurveto{\pgfqpoint{7.721930in}{-8.697892in}}{\pgfqpoint{7.725589in}{-8.689059in}}{\pgfqpoint{7.725589in}{-8.679851in}}%
\pgfpathcurveto{\pgfqpoint{7.725589in}{-8.670643in}}{\pgfqpoint{7.721930in}{-8.661810in}}{\pgfqpoint{7.715419in}{-8.655299in}}%
\pgfpathcurveto{\pgfqpoint{7.708908in}{-8.648787in}}{\pgfqpoint{7.700075in}{-8.645129in}}{\pgfqpoint{7.690867in}{-8.645129in}}%
\pgfpathcurveto{\pgfqpoint{7.681658in}{-8.645129in}}{\pgfqpoint{7.672826in}{-8.648787in}}{\pgfqpoint{7.666315in}{-8.655299in}}%
\pgfpathcurveto{\pgfqpoint{7.659803in}{-8.661810in}}{\pgfqpoint{7.656145in}{-8.670643in}}{\pgfqpoint{7.656145in}{-8.679851in}}%
\pgfpathcurveto{\pgfqpoint{7.656145in}{-8.689059in}}{\pgfqpoint{7.659803in}{-8.697892in}}{\pgfqpoint{7.666315in}{-8.704403in}}%
\pgfpathcurveto{\pgfqpoint{7.672826in}{-8.710915in}}{\pgfqpoint{7.681658in}{-8.714573in}}{\pgfqpoint{7.690867in}{-8.714573in}}%
\pgfpathlineto{\pgfqpoint{7.690867in}{-8.714573in}}%
\pgfpathclose%
\pgfusepath{stroke,fill}%
\end{pgfscope}%
\begin{pgfscope}%
\pgfpathrectangle{\pgfqpoint{1.374500in}{0.082500in}}{\pgfqpoint{2.419000in}{2.419000in}}%
\pgfusepath{clip}%
\pgfsetbuttcap%
\pgfsetroundjoin%
\definecolor{currentfill}{rgb}{0.247059,0.564706,0.854902}%
\pgfsetfillcolor{currentfill}%
\pgfsetlinewidth{1.003750pt}%
\definecolor{currentstroke}{rgb}{0.247059,0.564706,0.854902}%
\pgfsetstrokecolor{currentstroke}%
\pgfsetdash{}{0pt}%
\pgfpathmoveto{\pgfqpoint{10.384519in}{-10.228840in}}%
\pgfpathcurveto{\pgfqpoint{10.393728in}{-10.228840in}}{\pgfqpoint{10.402560in}{-10.225182in}}{\pgfqpoint{10.409072in}{-10.218670in}}%
\pgfpathcurveto{\pgfqpoint{10.415583in}{-10.212159in}}{\pgfqpoint{10.419241in}{-10.203327in}}{\pgfqpoint{10.419241in}{-10.194118in}}%
\pgfpathcurveto{\pgfqpoint{10.419241in}{-10.184910in}}{\pgfqpoint{10.415583in}{-10.176077in}}{\pgfqpoint{10.409072in}{-10.169566in}}%
\pgfpathcurveto{\pgfqpoint{10.402560in}{-10.163055in}}{\pgfqpoint{10.393728in}{-10.159396in}}{\pgfqpoint{10.384519in}{-10.159396in}}%
\pgfpathcurveto{\pgfqpoint{10.375311in}{-10.159396in}}{\pgfqpoint{10.366478in}{-10.163055in}}{\pgfqpoint{10.359967in}{-10.169566in}}%
\pgfpathcurveto{\pgfqpoint{10.353456in}{-10.176077in}}{\pgfqpoint{10.349797in}{-10.184910in}}{\pgfqpoint{10.349797in}{-10.194118in}}%
\pgfpathcurveto{\pgfqpoint{10.349797in}{-10.203327in}}{\pgfqpoint{10.353456in}{-10.212159in}}{\pgfqpoint{10.359967in}{-10.218670in}}%
\pgfpathcurveto{\pgfqpoint{10.366478in}{-10.225182in}}{\pgfqpoint{10.375311in}{-10.228840in}}{\pgfqpoint{10.384519in}{-10.228840in}}%
\pgfpathlineto{\pgfqpoint{10.384519in}{-10.228840in}}%
\pgfpathclose%
\pgfusepath{stroke,fill}%
\end{pgfscope}%
\begin{pgfscope}%
\pgfpathrectangle{\pgfqpoint{1.374500in}{0.082500in}}{\pgfqpoint{2.419000in}{2.419000in}}%
\pgfusepath{clip}%
\pgfsetbuttcap%
\pgfsetroundjoin%
\pgfsetlinewidth{1.505625pt}%
\definecolor{currentstroke}{rgb}{0.000000,0.000000,0.000000}%
\pgfsetstrokecolor{currentstroke}%
\pgfsetdash{}{0pt}%
\pgfusepath{stroke}%
\end{pgfscope}%
\begin{pgfscope}%
\pgfpathrectangle{\pgfqpoint{1.374500in}{0.082500in}}{\pgfqpoint{2.419000in}{2.419000in}}%
\pgfusepath{clip}%
\pgfsetbuttcap%
\pgfsetroundjoin%
\pgfsetlinewidth{1.505625pt}%
\definecolor{currentstroke}{rgb}{0.000000,0.000000,0.000000}%
\pgfsetstrokecolor{currentstroke}%
\pgfsetdash{}{0pt}%
\pgfusepath{stroke}%
\end{pgfscope}%
\begin{pgfscope}%
\pgfpathrectangle{\pgfqpoint{1.374500in}{0.082500in}}{\pgfqpoint{2.419000in}{2.419000in}}%
\pgfusepath{clip}%
\pgfsetbuttcap%
\pgfsetroundjoin%
\pgfsetlinewidth{1.505625pt}%
\definecolor{currentstroke}{rgb}{0.000000,0.000000,0.000000}%
\pgfsetstrokecolor{currentstroke}%
\pgfsetdash{}{0pt}%
\pgfusepath{stroke}%
\end{pgfscope}%
\begin{pgfscope}%
\pgfpathrectangle{\pgfqpoint{1.374500in}{0.082500in}}{\pgfqpoint{2.419000in}{2.419000in}}%
\pgfusepath{clip}%
\pgfsetbuttcap%
\pgfsetroundjoin%
\definecolor{currentfill}{rgb}{1.000000,0.662745,0.054902}%
\pgfsetfillcolor{currentfill}%
\pgfsetfillopacity{0.300000}%
\pgfsetlinewidth{1.003750pt}%
\definecolor{currentstroke}{rgb}{1.000000,0.662745,0.054902}%
\pgfsetstrokecolor{currentstroke}%
\pgfsetstrokeopacity{0.300000}%
\pgfsetdash{}{0pt}%
\pgfpathmoveto{\pgfqpoint{3.101096in}{4.227023in}}%
\pgfpathcurveto{\pgfqpoint{3.110305in}{4.227023in}}{\pgfqpoint{3.119137in}{4.230682in}}{\pgfqpoint{3.125649in}{4.237193in}}%
\pgfpathcurveto{\pgfqpoint{3.132160in}{4.243705in}}{\pgfqpoint{3.135819in}{4.252537in}}{\pgfqpoint{3.135819in}{4.261746in}}%
\pgfpathcurveto{\pgfqpoint{3.135819in}{4.270954in}}{\pgfqpoint{3.132160in}{4.279786in}}{\pgfqpoint{3.125649in}{4.286298in}}%
\pgfpathcurveto{\pgfqpoint{3.119137in}{4.292809in}}{\pgfqpoint{3.110305in}{4.296468in}}{\pgfqpoint{3.101096in}{4.296468in}}%
\pgfpathcurveto{\pgfqpoint{3.091888in}{4.296468in}}{\pgfqpoint{3.083055in}{4.292809in}}{\pgfqpoint{3.076544in}{4.286298in}}%
\pgfpathcurveto{\pgfqpoint{3.070033in}{4.279786in}}{\pgfqpoint{3.066374in}{4.270954in}}{\pgfqpoint{3.066374in}{4.261746in}}%
\pgfpathcurveto{\pgfqpoint{3.066374in}{4.252537in}}{\pgfqpoint{3.070033in}{4.243705in}}{\pgfqpoint{3.076544in}{4.237193in}}%
\pgfpathcurveto{\pgfqpoint{3.083055in}{4.230682in}}{\pgfqpoint{3.091888in}{4.227023in}}{\pgfqpoint{3.101096in}{4.227023in}}%
\pgfpathlineto{\pgfqpoint{3.101096in}{4.227023in}}%
\pgfpathclose%
\pgfusepath{stroke,fill}%
\end{pgfscope}%
\begin{pgfscope}%
\pgfpathrectangle{\pgfqpoint{1.374500in}{0.082500in}}{\pgfqpoint{2.419000in}{2.419000in}}%
\pgfusepath{clip}%
\pgfsetbuttcap%
\pgfsetroundjoin%
\definecolor{currentfill}{rgb}{1.000000,0.662745,0.054902}%
\pgfsetfillcolor{currentfill}%
\pgfsetfillopacity{0.304670}%
\pgfsetlinewidth{1.003750pt}%
\definecolor{currentstroke}{rgb}{1.000000,0.662745,0.054902}%
\pgfsetstrokecolor{currentstroke}%
\pgfsetstrokeopacity{0.304670}%
\pgfsetdash{}{0pt}%
\pgfpathmoveto{\pgfqpoint{3.668713in}{4.128736in}}%
\pgfpathcurveto{\pgfqpoint{3.677922in}{4.128736in}}{\pgfqpoint{3.686754in}{4.132395in}}{\pgfqpoint{3.693266in}{4.138906in}}%
\pgfpathcurveto{\pgfqpoint{3.699777in}{4.145417in}}{\pgfqpoint{3.703435in}{4.154250in}}{\pgfqpoint{3.703435in}{4.163458in}}%
\pgfpathcurveto{\pgfqpoint{3.703435in}{4.172667in}}{\pgfqpoint{3.699777in}{4.181499in}}{\pgfqpoint{3.693266in}{4.188011in}}%
\pgfpathcurveto{\pgfqpoint{3.686754in}{4.194522in}}{\pgfqpoint{3.677922in}{4.198180in}}{\pgfqpoint{3.668713in}{4.198180in}}%
\pgfpathcurveto{\pgfqpoint{3.659505in}{4.198180in}}{\pgfqpoint{3.650672in}{4.194522in}}{\pgfqpoint{3.644161in}{4.188011in}}%
\pgfpathcurveto{\pgfqpoint{3.637650in}{4.181499in}}{\pgfqpoint{3.633991in}{4.172667in}}{\pgfqpoint{3.633991in}{4.163458in}}%
\pgfpathcurveto{\pgfqpoint{3.633991in}{4.154250in}}{\pgfqpoint{3.637650in}{4.145417in}}{\pgfqpoint{3.644161in}{4.138906in}}%
\pgfpathcurveto{\pgfqpoint{3.650672in}{4.132395in}}{\pgfqpoint{3.659505in}{4.128736in}}{\pgfqpoint{3.668713in}{4.128736in}}%
\pgfpathlineto{\pgfqpoint{3.668713in}{4.128736in}}%
\pgfpathclose%
\pgfusepath{stroke,fill}%
\end{pgfscope}%
\begin{pgfscope}%
\pgfpathrectangle{\pgfqpoint{1.374500in}{0.082500in}}{\pgfqpoint{2.419000in}{2.419000in}}%
\pgfusepath{clip}%
\pgfsetbuttcap%
\pgfsetroundjoin%
\definecolor{currentfill}{rgb}{1.000000,0.662745,0.054902}%
\pgfsetfillcolor{currentfill}%
\pgfsetfillopacity{0.309542}%
\pgfsetlinewidth{1.003750pt}%
\definecolor{currentstroke}{rgb}{1.000000,0.662745,0.054902}%
\pgfsetstrokecolor{currentstroke}%
\pgfsetstrokeopacity{0.309542}%
\pgfsetdash{}{0pt}%
\pgfpathmoveto{\pgfqpoint{4.261049in}{4.026168in}}%
\pgfpathcurveto{\pgfqpoint{4.270257in}{4.026168in}}{\pgfqpoint{4.279090in}{4.029827in}}{\pgfqpoint{4.285601in}{4.036338in}}%
\pgfpathcurveto{\pgfqpoint{4.292113in}{4.042850in}}{\pgfqpoint{4.295771in}{4.051682in}}{\pgfqpoint{4.295771in}{4.060891in}}%
\pgfpathcurveto{\pgfqpoint{4.295771in}{4.070099in}}{\pgfqpoint{4.292113in}{4.078932in}}{\pgfqpoint{4.285601in}{4.085443in}}%
\pgfpathcurveto{\pgfqpoint{4.279090in}{4.091954in}}{\pgfqpoint{4.270257in}{4.095613in}}{\pgfqpoint{4.261049in}{4.095613in}}%
\pgfpathcurveto{\pgfqpoint{4.251841in}{4.095613in}}{\pgfqpoint{4.243008in}{4.091954in}}{\pgfqpoint{4.236497in}{4.085443in}}%
\pgfpathcurveto{\pgfqpoint{4.229985in}{4.078932in}}{\pgfqpoint{4.226327in}{4.070099in}}{\pgfqpoint{4.226327in}{4.060891in}}%
\pgfpathcurveto{\pgfqpoint{4.226327in}{4.051682in}}{\pgfqpoint{4.229985in}{4.042850in}}{\pgfqpoint{4.236497in}{4.036338in}}%
\pgfpathcurveto{\pgfqpoint{4.243008in}{4.029827in}}{\pgfqpoint{4.251841in}{4.026168in}}{\pgfqpoint{4.261049in}{4.026168in}}%
\pgfpathlineto{\pgfqpoint{4.261049in}{4.026168in}}%
\pgfpathclose%
\pgfusepath{stroke,fill}%
\end{pgfscope}%
\begin{pgfscope}%
\pgfpathrectangle{\pgfqpoint{1.374500in}{0.082500in}}{\pgfqpoint{2.419000in}{2.419000in}}%
\pgfusepath{clip}%
\pgfsetbuttcap%
\pgfsetroundjoin%
\definecolor{currentfill}{rgb}{1.000000,0.662745,0.054902}%
\pgfsetfillcolor{currentfill}%
\pgfsetfillopacity{0.312059}%
\pgfsetlinewidth{1.003750pt}%
\definecolor{currentstroke}{rgb}{1.000000,0.662745,0.054902}%
\pgfsetstrokecolor{currentstroke}%
\pgfsetstrokeopacity{0.312059}%
\pgfsetdash{}{0pt}%
\pgfpathmoveto{\pgfqpoint{3.031806in}{3.973191in}}%
\pgfpathcurveto{\pgfqpoint{3.041015in}{3.973191in}}{\pgfqpoint{3.049847in}{3.976850in}}{\pgfqpoint{3.056359in}{3.983361in}}%
\pgfpathcurveto{\pgfqpoint{3.062870in}{3.989873in}}{\pgfqpoint{3.066529in}{3.998705in}}{\pgfqpoint{3.066529in}{4.007913in}}%
\pgfpathcurveto{\pgfqpoint{3.066529in}{4.017122in}}{\pgfqpoint{3.062870in}{4.025954in}}{\pgfqpoint{3.056359in}{4.032466in}}%
\pgfpathcurveto{\pgfqpoint{3.049847in}{4.038977in}}{\pgfqpoint{3.041015in}{4.042636in}}{\pgfqpoint{3.031806in}{4.042636in}}%
\pgfpathcurveto{\pgfqpoint{3.022598in}{4.042636in}}{\pgfqpoint{3.013765in}{4.038977in}}{\pgfqpoint{3.007254in}{4.032466in}}%
\pgfpathcurveto{\pgfqpoint{3.000743in}{4.025954in}}{\pgfqpoint{2.997084in}{4.017122in}}{\pgfqpoint{2.997084in}{4.007913in}}%
\pgfpathcurveto{\pgfqpoint{2.997084in}{3.998705in}}{\pgfqpoint{3.000743in}{3.989873in}}{\pgfqpoint{3.007254in}{3.983361in}}%
\pgfpathcurveto{\pgfqpoint{3.013765in}{3.976850in}}{\pgfqpoint{3.022598in}{3.973191in}}{\pgfqpoint{3.031806in}{3.973191in}}%
\pgfpathlineto{\pgfqpoint{3.031806in}{3.973191in}}%
\pgfpathclose%
\pgfusepath{stroke,fill}%
\end{pgfscope}%
\begin{pgfscope}%
\pgfpathrectangle{\pgfqpoint{1.374500in}{0.082500in}}{\pgfqpoint{2.419000in}{2.419000in}}%
\pgfusepath{clip}%
\pgfsetbuttcap%
\pgfsetroundjoin%
\definecolor{currentfill}{rgb}{1.000000,0.662745,0.054902}%
\pgfsetfillcolor{currentfill}%
\pgfsetfillopacity{0.314632}%
\pgfsetlinewidth{1.003750pt}%
\definecolor{currentstroke}{rgb}{1.000000,0.662745,0.054902}%
\pgfsetstrokecolor{currentstroke}%
\pgfsetstrokeopacity{0.314632}%
\pgfsetdash{}{0pt}%
\pgfpathmoveto{\pgfqpoint{4.879754in}{3.919035in}}%
\pgfpathcurveto{\pgfqpoint{4.888963in}{3.919035in}}{\pgfqpoint{4.897795in}{3.922693in}}{\pgfqpoint{4.904307in}{3.929205in}}%
\pgfpathcurveto{\pgfqpoint{4.910818in}{3.935716in}}{\pgfqpoint{4.914477in}{3.944549in}}{\pgfqpoint{4.914477in}{3.953757in}}%
\pgfpathcurveto{\pgfqpoint{4.914477in}{3.962965in}}{\pgfqpoint{4.910818in}{3.971798in}}{\pgfqpoint{4.904307in}{3.978309in}}%
\pgfpathcurveto{\pgfqpoint{4.897795in}{3.984821in}}{\pgfqpoint{4.888963in}{3.988479in}}{\pgfqpoint{4.879754in}{3.988479in}}%
\pgfpathcurveto{\pgfqpoint{4.870546in}{3.988479in}}{\pgfqpoint{4.861713in}{3.984821in}}{\pgfqpoint{4.855202in}{3.978309in}}%
\pgfpathcurveto{\pgfqpoint{4.848691in}{3.971798in}}{\pgfqpoint{4.845032in}{3.962965in}}{\pgfqpoint{4.845032in}{3.953757in}}%
\pgfpathcurveto{\pgfqpoint{4.845032in}{3.944549in}}{\pgfqpoint{4.848691in}{3.935716in}}{\pgfqpoint{4.855202in}{3.929205in}}%
\pgfpathcurveto{\pgfqpoint{4.861713in}{3.922693in}}{\pgfqpoint{4.870546in}{3.919035in}}{\pgfqpoint{4.879754in}{3.919035in}}%
\pgfpathlineto{\pgfqpoint{4.879754in}{3.919035in}}%
\pgfpathclose%
\pgfusepath{stroke,fill}%
\end{pgfscope}%
\begin{pgfscope}%
\pgfpathrectangle{\pgfqpoint{1.374500in}{0.082500in}}{\pgfqpoint{2.419000in}{2.419000in}}%
\pgfusepath{clip}%
\pgfsetbuttcap%
\pgfsetroundjoin%
\definecolor{currentfill}{rgb}{1.000000,0.662745,0.054902}%
\pgfsetfillcolor{currentfill}%
\pgfsetfillopacity{0.317263}%
\pgfsetlinewidth{1.003750pt}%
\definecolor{currentstroke}{rgb}{1.000000,0.662745,0.054902}%
\pgfsetstrokecolor{currentstroke}%
\pgfsetstrokeopacity{0.317263}%
\pgfsetdash{}{0pt}%
\pgfpathmoveto{\pgfqpoint{3.629805in}{3.863659in}}%
\pgfpathcurveto{\pgfqpoint{3.639014in}{3.863659in}}{\pgfqpoint{3.647846in}{3.867318in}}{\pgfqpoint{3.654358in}{3.873829in}}%
\pgfpathcurveto{\pgfqpoint{3.660869in}{3.880341in}}{\pgfqpoint{3.664528in}{3.889173in}}{\pgfqpoint{3.664528in}{3.898382in}}%
\pgfpathcurveto{\pgfqpoint{3.664528in}{3.907590in}}{\pgfqpoint{3.660869in}{3.916423in}}{\pgfqpoint{3.654358in}{3.922934in}}%
\pgfpathcurveto{\pgfqpoint{3.647846in}{3.929445in}}{\pgfqpoint{3.639014in}{3.933104in}}{\pgfqpoint{3.629805in}{3.933104in}}%
\pgfpathcurveto{\pgfqpoint{3.620597in}{3.933104in}}{\pgfqpoint{3.611765in}{3.929445in}}{\pgfqpoint{3.605253in}{3.922934in}}%
\pgfpathcurveto{\pgfqpoint{3.598742in}{3.916423in}}{\pgfqpoint{3.595083in}{3.907590in}}{\pgfqpoint{3.595083in}{3.898382in}}%
\pgfpathcurveto{\pgfqpoint{3.595083in}{3.889173in}}{\pgfqpoint{3.598742in}{3.880341in}}{\pgfqpoint{3.605253in}{3.873829in}}%
\pgfpathcurveto{\pgfqpoint{3.611765in}{3.867318in}}{\pgfqpoint{3.620597in}{3.863659in}}{\pgfqpoint{3.629805in}{3.863659in}}%
\pgfpathlineto{\pgfqpoint{3.629805in}{3.863659in}}%
\pgfpathclose%
\pgfusepath{stroke,fill}%
\end{pgfscope}%
\begin{pgfscope}%
\pgfpathrectangle{\pgfqpoint{1.374500in}{0.082500in}}{\pgfqpoint{2.419000in}{2.419000in}}%
\pgfusepath{clip}%
\pgfsetbuttcap%
\pgfsetroundjoin%
\definecolor{currentfill}{rgb}{1.000000,0.662745,0.054902}%
\pgfsetfillcolor{currentfill}%
\pgfsetfillopacity{0.319954}%
\pgfsetlinewidth{1.003750pt}%
\definecolor{currentstroke}{rgb}{1.000000,0.662745,0.054902}%
\pgfsetstrokecolor{currentstroke}%
\pgfsetstrokeopacity{0.319954}%
\pgfsetdash{}{0pt}%
\pgfpathmoveto{\pgfqpoint{5.526630in}{3.807023in}}%
\pgfpathcurveto{\pgfqpoint{5.535839in}{3.807023in}}{\pgfqpoint{5.544671in}{3.810682in}}{\pgfqpoint{5.551183in}{3.817193in}}%
\pgfpathcurveto{\pgfqpoint{5.557694in}{3.823704in}}{\pgfqpoint{5.561353in}{3.832537in}}{\pgfqpoint{5.561353in}{3.841745in}}%
\pgfpathcurveto{\pgfqpoint{5.561353in}{3.850954in}}{\pgfqpoint{5.557694in}{3.859786in}}{\pgfqpoint{5.551183in}{3.866298in}}%
\pgfpathcurveto{\pgfqpoint{5.544671in}{3.872809in}}{\pgfqpoint{5.535839in}{3.876468in}}{\pgfqpoint{5.526630in}{3.876468in}}%
\pgfpathcurveto{\pgfqpoint{5.517422in}{3.876468in}}{\pgfqpoint{5.508590in}{3.872809in}}{\pgfqpoint{5.502078in}{3.866298in}}%
\pgfpathcurveto{\pgfqpoint{5.495567in}{3.859786in}}{\pgfqpoint{5.491908in}{3.850954in}}{\pgfqpoint{5.491908in}{3.841745in}}%
\pgfpathcurveto{\pgfqpoint{5.491908in}{3.832537in}}{\pgfqpoint{5.495567in}{3.823704in}}{\pgfqpoint{5.502078in}{3.817193in}}%
\pgfpathcurveto{\pgfqpoint{5.508590in}{3.810682in}}{\pgfqpoint{5.517422in}{3.807023in}}{\pgfqpoint{5.526630in}{3.807023in}}%
\pgfpathlineto{\pgfqpoint{5.526630in}{3.807023in}}%
\pgfpathclose%
\pgfusepath{stroke,fill}%
\end{pgfscope}%
\begin{pgfscope}%
\pgfpathrectangle{\pgfqpoint{1.374500in}{0.082500in}}{\pgfqpoint{2.419000in}{2.419000in}}%
\pgfusepath{clip}%
\pgfsetbuttcap%
\pgfsetroundjoin%
\definecolor{currentfill}{rgb}{1.000000,0.662745,0.054902}%
\pgfsetfillcolor{currentfill}%
\pgfsetfillopacity{0.319954}%
\pgfsetlinewidth{1.003750pt}%
\definecolor{currentstroke}{rgb}{1.000000,0.662745,0.054902}%
\pgfsetstrokecolor{currentstroke}%
\pgfsetstrokeopacity{0.319954}%
\pgfsetdash{}{0pt}%
\pgfpathmoveto{\pgfqpoint{2.351400in}{3.807023in}}%
\pgfpathcurveto{\pgfqpoint{2.360609in}{3.807023in}}{\pgfqpoint{2.369441in}{3.810682in}}{\pgfqpoint{2.375953in}{3.817193in}}%
\pgfpathcurveto{\pgfqpoint{2.382464in}{3.823704in}}{\pgfqpoint{2.386123in}{3.832537in}}{\pgfqpoint{2.386123in}{3.841745in}}%
\pgfpathcurveto{\pgfqpoint{2.386123in}{3.850954in}}{\pgfqpoint{2.382464in}{3.859786in}}{\pgfqpoint{2.375953in}{3.866298in}}%
\pgfpathcurveto{\pgfqpoint{2.369441in}{3.872809in}}{\pgfqpoint{2.360609in}{3.876468in}}{\pgfqpoint{2.351400in}{3.876468in}}%
\pgfpathcurveto{\pgfqpoint{2.342192in}{3.876468in}}{\pgfqpoint{2.333359in}{3.872809in}}{\pgfqpoint{2.326848in}{3.866298in}}%
\pgfpathcurveto{\pgfqpoint{2.320337in}{3.859786in}}{\pgfqpoint{2.316678in}{3.850954in}}{\pgfqpoint{2.316678in}{3.841745in}}%
\pgfpathcurveto{\pgfqpoint{2.316678in}{3.832537in}}{\pgfqpoint{2.320337in}{3.823704in}}{\pgfqpoint{2.326848in}{3.817193in}}%
\pgfpathcurveto{\pgfqpoint{2.333359in}{3.810682in}}{\pgfqpoint{2.342192in}{3.807023in}}{\pgfqpoint{2.351400in}{3.807023in}}%
\pgfpathlineto{\pgfqpoint{2.351400in}{3.807023in}}%
\pgfpathclose%
\pgfusepath{stroke,fill}%
\end{pgfscope}%
\begin{pgfscope}%
\pgfpathrectangle{\pgfqpoint{1.374500in}{0.082500in}}{\pgfqpoint{2.419000in}{2.419000in}}%
\pgfusepath{clip}%
\pgfsetbuttcap%
\pgfsetroundjoin%
\definecolor{currentfill}{rgb}{1.000000,0.662745,0.054902}%
\pgfsetfillcolor{currentfill}%
\pgfsetfillopacity{0.322707}%
\pgfsetlinewidth{1.003750pt}%
\definecolor{currentstroke}{rgb}{1.000000,0.662745,0.054902}%
\pgfsetstrokecolor{currentstroke}%
\pgfsetstrokeopacity{0.322707}%
\pgfsetdash{}{0pt}%
\pgfpathmoveto{\pgfqpoint{4.255346in}{3.749083in}}%
\pgfpathcurveto{\pgfqpoint{4.264554in}{3.749083in}}{\pgfqpoint{4.273387in}{3.752741in}}{\pgfqpoint{4.279898in}{3.759253in}}%
\pgfpathcurveto{\pgfqpoint{4.286410in}{3.765764in}}{\pgfqpoint{4.290068in}{3.774597in}}{\pgfqpoint{4.290068in}{3.783805in}}%
\pgfpathcurveto{\pgfqpoint{4.290068in}{3.793014in}}{\pgfqpoint{4.286410in}{3.801846in}}{\pgfqpoint{4.279898in}{3.808357in}}%
\pgfpathcurveto{\pgfqpoint{4.273387in}{3.814869in}}{\pgfqpoint{4.264554in}{3.818527in}}{\pgfqpoint{4.255346in}{3.818527in}}%
\pgfpathcurveto{\pgfqpoint{4.246137in}{3.818527in}}{\pgfqpoint{4.237305in}{3.814869in}}{\pgfqpoint{4.230794in}{3.808357in}}%
\pgfpathcurveto{\pgfqpoint{4.224282in}{3.801846in}}{\pgfqpoint{4.220624in}{3.793014in}}{\pgfqpoint{4.220624in}{3.783805in}}%
\pgfpathcurveto{\pgfqpoint{4.220624in}{3.774597in}}{\pgfqpoint{4.224282in}{3.765764in}}{\pgfqpoint{4.230794in}{3.759253in}}%
\pgfpathcurveto{\pgfqpoint{4.237305in}{3.752741in}}{\pgfqpoint{4.246137in}{3.749083in}}{\pgfqpoint{4.255346in}{3.749083in}}%
\pgfpathlineto{\pgfqpoint{4.255346in}{3.749083in}}%
\pgfpathclose%
\pgfusepath{stroke,fill}%
\end{pgfscope}%
\begin{pgfscope}%
\pgfpathrectangle{\pgfqpoint{1.374500in}{0.082500in}}{\pgfqpoint{2.419000in}{2.419000in}}%
\pgfusepath{clip}%
\pgfsetbuttcap%
\pgfsetroundjoin%
\definecolor{currentfill}{rgb}{1.000000,0.662745,0.054902}%
\pgfsetfillcolor{currentfill}%
\pgfsetfillopacity{0.325523}%
\pgfsetlinewidth{1.003750pt}%
\definecolor{currentstroke}{rgb}{1.000000,0.662745,0.054902}%
\pgfsetstrokecolor{currentstroke}%
\pgfsetstrokeopacity{0.325523}%
\pgfsetdash{}{0pt}%
\pgfpathmoveto{\pgfqpoint{2.954445in}{3.689793in}}%
\pgfpathcurveto{\pgfqpoint{2.963654in}{3.689793in}}{\pgfqpoint{2.972486in}{3.693451in}}{\pgfqpoint{2.978998in}{3.699963in}}%
\pgfpathcurveto{\pgfqpoint{2.985509in}{3.706474in}}{\pgfqpoint{2.989168in}{3.715307in}}{\pgfqpoint{2.989168in}{3.724515in}}%
\pgfpathcurveto{\pgfqpoint{2.989168in}{3.733723in}}{\pgfqpoint{2.985509in}{3.742556in}}{\pgfqpoint{2.978998in}{3.749067in}}%
\pgfpathcurveto{\pgfqpoint{2.972486in}{3.755579in}}{\pgfqpoint{2.963654in}{3.759237in}}{\pgfqpoint{2.954445in}{3.759237in}}%
\pgfpathcurveto{\pgfqpoint{2.945237in}{3.759237in}}{\pgfqpoint{2.936404in}{3.755579in}}{\pgfqpoint{2.929893in}{3.749067in}}%
\pgfpathcurveto{\pgfqpoint{2.923382in}{3.742556in}}{\pgfqpoint{2.919723in}{3.733723in}}{\pgfqpoint{2.919723in}{3.724515in}}%
\pgfpathcurveto{\pgfqpoint{2.919723in}{3.715307in}}{\pgfqpoint{2.923382in}{3.706474in}}{\pgfqpoint{2.929893in}{3.699963in}}%
\pgfpathcurveto{\pgfqpoint{2.936404in}{3.693451in}}{\pgfqpoint{2.945237in}{3.689793in}}{\pgfqpoint{2.954445in}{3.689793in}}%
\pgfpathlineto{\pgfqpoint{2.954445in}{3.689793in}}%
\pgfpathclose%
\pgfusepath{stroke,fill}%
\end{pgfscope}%
\begin{pgfscope}%
\pgfpathrectangle{\pgfqpoint{1.374500in}{0.082500in}}{\pgfqpoint{2.419000in}{2.419000in}}%
\pgfusepath{clip}%
\pgfsetbuttcap%
\pgfsetroundjoin%
\definecolor{currentfill}{rgb}{1.000000,0.662745,0.054902}%
\pgfsetfillcolor{currentfill}%
\pgfsetfillopacity{0.325523}%
\pgfsetlinewidth{1.003750pt}%
\definecolor{currentstroke}{rgb}{1.000000,0.662745,0.054902}%
\pgfsetstrokecolor{currentstroke}%
\pgfsetstrokeopacity{0.325523}%
\pgfsetdash{}{0pt}%
\pgfpathmoveto{\pgfqpoint{6.203646in}{3.689793in}}%
\pgfpathcurveto{\pgfqpoint{6.212854in}{3.689793in}}{\pgfqpoint{6.221687in}{3.693451in}}{\pgfqpoint{6.228198in}{3.699963in}}%
\pgfpathcurveto{\pgfqpoint{6.234710in}{3.706474in}}{\pgfqpoint{6.238368in}{3.715307in}}{\pgfqpoint{6.238368in}{3.724515in}}%
\pgfpathcurveto{\pgfqpoint{6.238368in}{3.733723in}}{\pgfqpoint{6.234710in}{3.742556in}}{\pgfqpoint{6.228198in}{3.749067in}}%
\pgfpathcurveto{\pgfqpoint{6.221687in}{3.755579in}}{\pgfqpoint{6.212854in}{3.759237in}}{\pgfqpoint{6.203646in}{3.759237in}}%
\pgfpathcurveto{\pgfqpoint{6.194437in}{3.759237in}}{\pgfqpoint{6.185605in}{3.755579in}}{\pgfqpoint{6.179094in}{3.749067in}}%
\pgfpathcurveto{\pgfqpoint{6.172582in}{3.742556in}}{\pgfqpoint{6.168924in}{3.733723in}}{\pgfqpoint{6.168924in}{3.724515in}}%
\pgfpathcurveto{\pgfqpoint{6.168924in}{3.715307in}}{\pgfqpoint{6.172582in}{3.706474in}}{\pgfqpoint{6.179094in}{3.699963in}}%
\pgfpathcurveto{\pgfqpoint{6.185605in}{3.693451in}}{\pgfqpoint{6.194437in}{3.689793in}}{\pgfqpoint{6.203646in}{3.689793in}}%
\pgfpathlineto{\pgfqpoint{6.203646in}{3.689793in}}%
\pgfpathclose%
\pgfusepath{stroke,fill}%
\end{pgfscope}%
\begin{pgfscope}%
\pgfpathrectangle{\pgfqpoint{1.374500in}{0.082500in}}{\pgfqpoint{2.419000in}{2.419000in}}%
\pgfusepath{clip}%
\pgfsetbuttcap%
\pgfsetroundjoin%
\definecolor{currentfill}{rgb}{1.000000,0.662745,0.054902}%
\pgfsetfillcolor{currentfill}%
\pgfsetfillopacity{0.328407}%
\pgfsetlinewidth{1.003750pt}%
\definecolor{currentstroke}{rgb}{1.000000,0.662745,0.054902}%
\pgfsetstrokecolor{currentstroke}%
\pgfsetstrokeopacity{0.328407}%
\pgfsetdash{}{0pt}%
\pgfpathmoveto{\pgfqpoint{4.910375in}{3.629105in}}%
\pgfpathcurveto{\pgfqpoint{4.919584in}{3.629105in}}{\pgfqpoint{4.928416in}{3.632764in}}{\pgfqpoint{4.934928in}{3.639275in}}%
\pgfpathcurveto{\pgfqpoint{4.941439in}{3.645786in}}{\pgfqpoint{4.945097in}{3.654619in}}{\pgfqpoint{4.945097in}{3.663827in}}%
\pgfpathcurveto{\pgfqpoint{4.945097in}{3.673036in}}{\pgfqpoint{4.941439in}{3.681868in}}{\pgfqpoint{4.934928in}{3.688380in}}%
\pgfpathcurveto{\pgfqpoint{4.928416in}{3.694891in}}{\pgfqpoint{4.919584in}{3.698550in}}{\pgfqpoint{4.910375in}{3.698550in}}%
\pgfpathcurveto{\pgfqpoint{4.901167in}{3.698550in}}{\pgfqpoint{4.892334in}{3.694891in}}{\pgfqpoint{4.885823in}{3.688380in}}%
\pgfpathcurveto{\pgfqpoint{4.879312in}{3.681868in}}{\pgfqpoint{4.875653in}{3.673036in}}{\pgfqpoint{4.875653in}{3.663827in}}%
\pgfpathcurveto{\pgfqpoint{4.875653in}{3.654619in}}{\pgfqpoint{4.879312in}{3.645786in}}{\pgfqpoint{4.885823in}{3.639275in}}%
\pgfpathcurveto{\pgfqpoint{4.892334in}{3.632764in}}{\pgfqpoint{4.901167in}{3.629105in}}{\pgfqpoint{4.910375in}{3.629105in}}%
\pgfpathlineto{\pgfqpoint{4.910375in}{3.629105in}}%
\pgfpathclose%
\pgfusepath{stroke,fill}%
\end{pgfscope}%
\begin{pgfscope}%
\pgfpathrectangle{\pgfqpoint{1.374500in}{0.082500in}}{\pgfqpoint{2.419000in}{2.419000in}}%
\pgfusepath{clip}%
\pgfsetbuttcap%
\pgfsetroundjoin%
\definecolor{currentfill}{rgb}{1.000000,0.662745,0.054902}%
\pgfsetfillcolor{currentfill}%
\pgfsetfillopacity{0.331359}%
\pgfsetlinewidth{1.003750pt}%
\definecolor{currentstroke}{rgb}{1.000000,0.662745,0.054902}%
\pgfsetstrokecolor{currentstroke}%
\pgfsetstrokeopacity{0.331359}%
\pgfsetdash{}{0pt}%
\pgfpathmoveto{\pgfqpoint{3.586258in}{3.566970in}}%
\pgfpathcurveto{\pgfqpoint{3.595466in}{3.566970in}}{\pgfqpoint{3.604299in}{3.570628in}}{\pgfqpoint{3.610810in}{3.577140in}}%
\pgfpathcurveto{\pgfqpoint{3.617321in}{3.583651in}}{\pgfqpoint{3.620980in}{3.592484in}}{\pgfqpoint{3.620980in}{3.601692in}}%
\pgfpathcurveto{\pgfqpoint{3.620980in}{3.610901in}}{\pgfqpoint{3.617321in}{3.619733in}}{\pgfqpoint{3.610810in}{3.626244in}}%
\pgfpathcurveto{\pgfqpoint{3.604299in}{3.632756in}}{\pgfqpoint{3.595466in}{3.636414in}}{\pgfqpoint{3.586258in}{3.636414in}}%
\pgfpathcurveto{\pgfqpoint{3.577049in}{3.636414in}}{\pgfqpoint{3.568217in}{3.632756in}}{\pgfqpoint{3.561705in}{3.626244in}}%
\pgfpathcurveto{\pgfqpoint{3.555194in}{3.619733in}}{\pgfqpoint{3.551535in}{3.610901in}}{\pgfqpoint{3.551535in}{3.601692in}}%
\pgfpathcurveto{\pgfqpoint{3.551535in}{3.592484in}}{\pgfqpoint{3.555194in}{3.583651in}}{\pgfqpoint{3.561705in}{3.577140in}}%
\pgfpathcurveto{\pgfqpoint{3.568217in}{3.570628in}}{\pgfqpoint{3.577049in}{3.566970in}}{\pgfqpoint{3.586258in}{3.566970in}}%
\pgfpathlineto{\pgfqpoint{3.586258in}{3.566970in}}%
\pgfpathclose%
\pgfusepath{stroke,fill}%
\end{pgfscope}%
\begin{pgfscope}%
\pgfpathrectangle{\pgfqpoint{1.374500in}{0.082500in}}{\pgfqpoint{2.419000in}{2.419000in}}%
\pgfusepath{clip}%
\pgfsetbuttcap%
\pgfsetroundjoin%
\definecolor{currentfill}{rgb}{1.000000,0.662745,0.054902}%
\pgfsetfillcolor{currentfill}%
\pgfsetfillopacity{0.331359}%
\pgfsetlinewidth{1.003750pt}%
\definecolor{currentstroke}{rgb}{1.000000,0.662745,0.054902}%
\pgfsetstrokecolor{currentstroke}%
\pgfsetstrokeopacity{0.331359}%
\pgfsetdash{}{0pt}%
\pgfpathmoveto{\pgfqpoint{6.912957in}{3.566970in}}%
\pgfpathcurveto{\pgfqpoint{6.922166in}{3.566970in}}{\pgfqpoint{6.930998in}{3.570628in}}{\pgfqpoint{6.937510in}{3.577140in}}%
\pgfpathcurveto{\pgfqpoint{6.944021in}{3.583651in}}{\pgfqpoint{6.947680in}{3.592484in}}{\pgfqpoint{6.947680in}{3.601692in}}%
\pgfpathcurveto{\pgfqpoint{6.947680in}{3.610901in}}{\pgfqpoint{6.944021in}{3.619733in}}{\pgfqpoint{6.937510in}{3.626244in}}%
\pgfpathcurveto{\pgfqpoint{6.930998in}{3.632756in}}{\pgfqpoint{6.922166in}{3.636414in}}{\pgfqpoint{6.912957in}{3.636414in}}%
\pgfpathcurveto{\pgfqpoint{6.903749in}{3.636414in}}{\pgfqpoint{6.894916in}{3.632756in}}{\pgfqpoint{6.888405in}{3.626244in}}%
\pgfpathcurveto{\pgfqpoint{6.881894in}{3.619733in}}{\pgfqpoint{6.878235in}{3.610901in}}{\pgfqpoint{6.878235in}{3.601692in}}%
\pgfpathcurveto{\pgfqpoint{6.878235in}{3.592484in}}{\pgfqpoint{6.881894in}{3.583651in}}{\pgfqpoint{6.888405in}{3.577140in}}%
\pgfpathcurveto{\pgfqpoint{6.894916in}{3.570628in}}{\pgfqpoint{6.903749in}{3.566970in}}{\pgfqpoint{6.912957in}{3.566970in}}%
\pgfpathlineto{\pgfqpoint{6.912957in}{3.566970in}}%
\pgfpathclose%
\pgfusepath{stroke,fill}%
\end{pgfscope}%
\begin{pgfscope}%
\pgfpathrectangle{\pgfqpoint{1.374500in}{0.082500in}}{\pgfqpoint{2.419000in}{2.419000in}}%
\pgfusepath{clip}%
\pgfsetbuttcap%
\pgfsetroundjoin%
\definecolor{currentfill}{rgb}{1.000000,0.662745,0.054902}%
\pgfsetfillcolor{currentfill}%
\pgfsetfillopacity{0.334382}%
\pgfsetlinewidth{1.003750pt}%
\definecolor{currentstroke}{rgb}{1.000000,0.662745,0.054902}%
\pgfsetstrokecolor{currentstroke}%
\pgfsetstrokeopacity{0.334382}%
\pgfsetdash{}{0pt}%
\pgfpathmoveto{\pgfqpoint{2.230176in}{3.503335in}}%
\pgfpathcurveto{\pgfqpoint{2.239385in}{3.503335in}}{\pgfqpoint{2.248217in}{3.506993in}}{\pgfqpoint{2.254729in}{3.513505in}}%
\pgfpathcurveto{\pgfqpoint{2.261240in}{3.520016in}}{\pgfqpoint{2.264899in}{3.528849in}}{\pgfqpoint{2.264899in}{3.538057in}}%
\pgfpathcurveto{\pgfqpoint{2.264899in}{3.547266in}}{\pgfqpoint{2.261240in}{3.556098in}}{\pgfqpoint{2.254729in}{3.562609in}}%
\pgfpathcurveto{\pgfqpoint{2.248217in}{3.569121in}}{\pgfqpoint{2.239385in}{3.572779in}}{\pgfqpoint{2.230176in}{3.572779in}}%
\pgfpathcurveto{\pgfqpoint{2.220968in}{3.572779in}}{\pgfqpoint{2.212135in}{3.569121in}}{\pgfqpoint{2.205624in}{3.562609in}}%
\pgfpathcurveto{\pgfqpoint{2.199113in}{3.556098in}}{\pgfqpoint{2.195454in}{3.547266in}}{\pgfqpoint{2.195454in}{3.538057in}}%
\pgfpathcurveto{\pgfqpoint{2.195454in}{3.528849in}}{\pgfqpoint{2.199113in}{3.520016in}}{\pgfqpoint{2.205624in}{3.513505in}}%
\pgfpathcurveto{\pgfqpoint{2.212135in}{3.506993in}}{\pgfqpoint{2.220968in}{3.503335in}}{\pgfqpoint{2.230176in}{3.503335in}}%
\pgfpathlineto{\pgfqpoint{2.230176in}{3.503335in}}%
\pgfpathclose%
\pgfusepath{stroke,fill}%
\end{pgfscope}%
\begin{pgfscope}%
\pgfpathrectangle{\pgfqpoint{1.374500in}{0.082500in}}{\pgfqpoint{2.419000in}{2.419000in}}%
\pgfusepath{clip}%
\pgfsetbuttcap%
\pgfsetroundjoin%
\definecolor{currentfill}{rgb}{1.000000,0.662745,0.054902}%
\pgfsetfillcolor{currentfill}%
\pgfsetfillopacity{0.334382}%
\pgfsetlinewidth{1.003750pt}%
\definecolor{currentstroke}{rgb}{1.000000,0.662745,0.054902}%
\pgfsetstrokecolor{currentstroke}%
\pgfsetstrokeopacity{0.334382}%
\pgfsetdash{}{0pt}%
\pgfpathmoveto{\pgfqpoint{5.597029in}{3.503335in}}%
\pgfpathcurveto{\pgfqpoint{5.606237in}{3.503335in}}{\pgfqpoint{5.615070in}{3.506993in}}{\pgfqpoint{5.621581in}{3.513505in}}%
\pgfpathcurveto{\pgfqpoint{5.628093in}{3.520016in}}{\pgfqpoint{5.631751in}{3.528849in}}{\pgfqpoint{5.631751in}{3.538057in}}%
\pgfpathcurveto{\pgfqpoint{5.631751in}{3.547266in}}{\pgfqpoint{5.628093in}{3.556098in}}{\pgfqpoint{5.621581in}{3.562609in}}%
\pgfpathcurveto{\pgfqpoint{5.615070in}{3.569121in}}{\pgfqpoint{5.606237in}{3.572779in}}{\pgfqpoint{5.597029in}{3.572779in}}%
\pgfpathcurveto{\pgfqpoint{5.587820in}{3.572779in}}{\pgfqpoint{5.578988in}{3.569121in}}{\pgfqpoint{5.572477in}{3.562609in}}%
\pgfpathcurveto{\pgfqpoint{5.565965in}{3.556098in}}{\pgfqpoint{5.562307in}{3.547266in}}{\pgfqpoint{5.562307in}{3.538057in}}%
\pgfpathcurveto{\pgfqpoint{5.562307in}{3.528849in}}{\pgfqpoint{5.565965in}{3.520016in}}{\pgfqpoint{5.572477in}{3.513505in}}%
\pgfpathcurveto{\pgfqpoint{5.578988in}{3.506993in}}{\pgfqpoint{5.587820in}{3.503335in}}{\pgfqpoint{5.597029in}{3.503335in}}%
\pgfpathlineto{\pgfqpoint{5.597029in}{3.503335in}}%
\pgfpathclose%
\pgfusepath{stroke,fill}%
\end{pgfscope}%
\begin{pgfscope}%
\pgfpathrectangle{\pgfqpoint{1.374500in}{0.082500in}}{\pgfqpoint{2.419000in}{2.419000in}}%
\pgfusepath{clip}%
\pgfsetbuttcap%
\pgfsetroundjoin%
\definecolor{currentfill}{rgb}{1.000000,0.662745,0.054902}%
\pgfsetfillcolor{currentfill}%
\pgfsetfillopacity{0.337479}%
\pgfsetlinewidth{1.003750pt}%
\definecolor{currentstroke}{rgb}{1.000000,0.662745,0.054902}%
\pgfsetstrokecolor{currentstroke}%
\pgfsetstrokeopacity{0.337479}%
\pgfsetdash{}{0pt}%
\pgfpathmoveto{\pgfqpoint{4.248946in}{3.438145in}}%
\pgfpathcurveto{\pgfqpoint{4.258155in}{3.438145in}}{\pgfqpoint{4.266987in}{3.441803in}}{\pgfqpoint{4.273498in}{3.448315in}}%
\pgfpathcurveto{\pgfqpoint{4.280010in}{3.454826in}}{\pgfqpoint{4.283668in}{3.463659in}}{\pgfqpoint{4.283668in}{3.472867in}}%
\pgfpathcurveto{\pgfqpoint{4.283668in}{3.482076in}}{\pgfqpoint{4.280010in}{3.490908in}}{\pgfqpoint{4.273498in}{3.497419in}}%
\pgfpathcurveto{\pgfqpoint{4.266987in}{3.503931in}}{\pgfqpoint{4.258155in}{3.507589in}}{\pgfqpoint{4.248946in}{3.507589in}}%
\pgfpathcurveto{\pgfqpoint{4.239738in}{3.507589in}}{\pgfqpoint{4.230905in}{3.503931in}}{\pgfqpoint{4.224394in}{3.497419in}}%
\pgfpathcurveto{\pgfqpoint{4.217882in}{3.490908in}}{\pgfqpoint{4.214224in}{3.482076in}}{\pgfqpoint{4.214224in}{3.472867in}}%
\pgfpathcurveto{\pgfqpoint{4.214224in}{3.463659in}}{\pgfqpoint{4.217882in}{3.454826in}}{\pgfqpoint{4.224394in}{3.448315in}}%
\pgfpathcurveto{\pgfqpoint{4.230905in}{3.441803in}}{\pgfqpoint{4.239738in}{3.438145in}}{\pgfqpoint{4.248946in}{3.438145in}}%
\pgfpathlineto{\pgfqpoint{4.248946in}{3.438145in}}%
\pgfpathclose%
\pgfusepath{stroke,fill}%
\end{pgfscope}%
\begin{pgfscope}%
\pgfpathrectangle{\pgfqpoint{1.374500in}{0.082500in}}{\pgfqpoint{2.419000in}{2.419000in}}%
\pgfusepath{clip}%
\pgfsetbuttcap%
\pgfsetroundjoin%
\definecolor{currentfill}{rgb}{1.000000,0.662745,0.054902}%
\pgfsetfillcolor{currentfill}%
\pgfsetfillopacity{0.337479}%
\pgfsetlinewidth{1.003750pt}%
\definecolor{currentstroke}{rgb}{1.000000,0.662745,0.054902}%
\pgfsetstrokecolor{currentstroke}%
\pgfsetstrokeopacity{0.337479}%
\pgfsetdash{}{0pt}%
\pgfpathmoveto{\pgfqpoint{7.656932in}{3.438145in}}%
\pgfpathcurveto{\pgfqpoint{7.666141in}{3.438145in}}{\pgfqpoint{7.674973in}{3.441803in}}{\pgfqpoint{7.681485in}{3.448315in}}%
\pgfpathcurveto{\pgfqpoint{7.687996in}{3.454826in}}{\pgfqpoint{7.691655in}{3.463659in}}{\pgfqpoint{7.691655in}{3.472867in}}%
\pgfpathcurveto{\pgfqpoint{7.691655in}{3.482076in}}{\pgfqpoint{7.687996in}{3.490908in}}{\pgfqpoint{7.681485in}{3.497419in}}%
\pgfpathcurveto{\pgfqpoint{7.674973in}{3.503931in}}{\pgfqpoint{7.666141in}{3.507589in}}{\pgfqpoint{7.656932in}{3.507589in}}%
\pgfpathcurveto{\pgfqpoint{7.647724in}{3.507589in}}{\pgfqpoint{7.638891in}{3.503931in}}{\pgfqpoint{7.632380in}{3.497419in}}%
\pgfpathcurveto{\pgfqpoint{7.625869in}{3.490908in}}{\pgfqpoint{7.622210in}{3.482076in}}{\pgfqpoint{7.622210in}{3.472867in}}%
\pgfpathcurveto{\pgfqpoint{7.622210in}{3.463659in}}{\pgfqpoint{7.625869in}{3.454826in}}{\pgfqpoint{7.632380in}{3.448315in}}%
\pgfpathcurveto{\pgfqpoint{7.638891in}{3.441803in}}{\pgfqpoint{7.647724in}{3.438145in}}{\pgfqpoint{7.656932in}{3.438145in}}%
\pgfpathlineto{\pgfqpoint{7.656932in}{3.438145in}}%
\pgfpathclose%
\pgfusepath{stroke,fill}%
\end{pgfscope}%
\begin{pgfscope}%
\pgfpathrectangle{\pgfqpoint{1.374500in}{0.082500in}}{\pgfqpoint{2.419000in}{2.419000in}}%
\pgfusepath{clip}%
\pgfsetbuttcap%
\pgfsetroundjoin%
\definecolor{currentfill}{rgb}{1.000000,0.662745,0.054902}%
\pgfsetfillcolor{currentfill}%
\pgfsetfillopacity{0.340653}%
\pgfsetlinewidth{1.003750pt}%
\definecolor{currentstroke}{rgb}{1.000000,0.662745,0.054902}%
\pgfsetstrokecolor{currentstroke}%
\pgfsetstrokeopacity{0.340653}%
\pgfsetdash{}{0pt}%
\pgfpathmoveto{\pgfqpoint{6.317654in}{3.371342in}}%
\pgfpathcurveto{\pgfqpoint{6.326862in}{3.371342in}}{\pgfqpoint{6.335695in}{3.375001in}}{\pgfqpoint{6.342206in}{3.381512in}}%
\pgfpathcurveto{\pgfqpoint{6.348717in}{3.388024in}}{\pgfqpoint{6.352376in}{3.396856in}}{\pgfqpoint{6.352376in}{3.406065in}}%
\pgfpathcurveto{\pgfqpoint{6.352376in}{3.415273in}}{\pgfqpoint{6.348717in}{3.424106in}}{\pgfqpoint{6.342206in}{3.430617in}}%
\pgfpathcurveto{\pgfqpoint{6.335695in}{3.437128in}}{\pgfqpoint{6.326862in}{3.440787in}}{\pgfqpoint{6.317654in}{3.440787in}}%
\pgfpathcurveto{\pgfqpoint{6.308445in}{3.440787in}}{\pgfqpoint{6.299613in}{3.437128in}}{\pgfqpoint{6.293101in}{3.430617in}}%
\pgfpathcurveto{\pgfqpoint{6.286590in}{3.424106in}}{\pgfqpoint{6.282932in}{3.415273in}}{\pgfqpoint{6.282932in}{3.406065in}}%
\pgfpathcurveto{\pgfqpoint{6.282932in}{3.396856in}}{\pgfqpoint{6.286590in}{3.388024in}}{\pgfqpoint{6.293101in}{3.381512in}}%
\pgfpathcurveto{\pgfqpoint{6.299613in}{3.375001in}}{\pgfqpoint{6.308445in}{3.371342in}}{\pgfqpoint{6.317654in}{3.371342in}}%
\pgfpathlineto{\pgfqpoint{6.317654in}{3.371342in}}%
\pgfpathclose%
\pgfusepath{stroke,fill}%
\end{pgfscope}%
\begin{pgfscope}%
\pgfpathrectangle{\pgfqpoint{1.374500in}{0.082500in}}{\pgfqpoint{2.419000in}{2.419000in}}%
\pgfusepath{clip}%
\pgfsetbuttcap%
\pgfsetroundjoin%
\definecolor{currentfill}{rgb}{1.000000,0.662745,0.054902}%
\pgfsetfillcolor{currentfill}%
\pgfsetfillopacity{0.340653}%
\pgfsetlinewidth{1.003750pt}%
\definecolor{currentstroke}{rgb}{1.000000,0.662745,0.054902}%
\pgfsetstrokecolor{currentstroke}%
\pgfsetstrokeopacity{0.340653}%
\pgfsetdash{}{0pt}%
\pgfpathmoveto{\pgfqpoint{2.867516in}{3.371342in}}%
\pgfpathcurveto{\pgfqpoint{2.876725in}{3.371342in}}{\pgfqpoint{2.885557in}{3.375001in}}{\pgfqpoint{2.892068in}{3.381512in}}%
\pgfpathcurveto{\pgfqpoint{2.898580in}{3.388024in}}{\pgfqpoint{2.902238in}{3.396856in}}{\pgfqpoint{2.902238in}{3.406065in}}%
\pgfpathcurveto{\pgfqpoint{2.902238in}{3.415273in}}{\pgfqpoint{2.898580in}{3.424106in}}{\pgfqpoint{2.892068in}{3.430617in}}%
\pgfpathcurveto{\pgfqpoint{2.885557in}{3.437128in}}{\pgfqpoint{2.876725in}{3.440787in}}{\pgfqpoint{2.867516in}{3.440787in}}%
\pgfpathcurveto{\pgfqpoint{2.858308in}{3.440787in}}{\pgfqpoint{2.849475in}{3.437128in}}{\pgfqpoint{2.842964in}{3.430617in}}%
\pgfpathcurveto{\pgfqpoint{2.836452in}{3.424106in}}{\pgfqpoint{2.832794in}{3.415273in}}{\pgfqpoint{2.832794in}{3.406065in}}%
\pgfpathcurveto{\pgfqpoint{2.832794in}{3.396856in}}{\pgfqpoint{2.836452in}{3.388024in}}{\pgfqpoint{2.842964in}{3.381512in}}%
\pgfpathcurveto{\pgfqpoint{2.849475in}{3.375001in}}{\pgfqpoint{2.858308in}{3.371342in}}{\pgfqpoint{2.867516in}{3.371342in}}%
\pgfpathlineto{\pgfqpoint{2.867516in}{3.371342in}}%
\pgfpathclose%
\pgfusepath{stroke,fill}%
\end{pgfscope}%
\begin{pgfscope}%
\pgfpathrectangle{\pgfqpoint{1.374500in}{0.082500in}}{\pgfqpoint{2.419000in}{2.419000in}}%
\pgfusepath{clip}%
\pgfsetbuttcap%
\pgfsetroundjoin%
\definecolor{currentfill}{rgb}{1.000000,0.662745,0.054902}%
\pgfsetfillcolor{currentfill}%
\pgfsetfillopacity{0.343906}%
\pgfsetlinewidth{1.003750pt}%
\definecolor{currentstroke}{rgb}{1.000000,0.662745,0.054902}%
\pgfsetstrokecolor{currentstroke}%
\pgfsetstrokeopacity{0.343906}%
\pgfsetdash{}{0pt}%
\pgfpathmoveto{\pgfqpoint{1.451486in}{3.302867in}}%
\pgfpathcurveto{\pgfqpoint{1.460694in}{3.302867in}}{\pgfqpoint{1.469527in}{3.306525in}}{\pgfqpoint{1.476038in}{3.313036in}}%
\pgfpathcurveto{\pgfqpoint{1.482550in}{3.319548in}}{\pgfqpoint{1.486208in}{3.328380in}}{\pgfqpoint{1.486208in}{3.337589in}}%
\pgfpathcurveto{\pgfqpoint{1.486208in}{3.346797in}}{\pgfqpoint{1.482550in}{3.355630in}}{\pgfqpoint{1.476038in}{3.362141in}}%
\pgfpathcurveto{\pgfqpoint{1.469527in}{3.368652in}}{\pgfqpoint{1.460694in}{3.372311in}}{\pgfqpoint{1.451486in}{3.372311in}}%
\pgfpathcurveto{\pgfqpoint{1.442278in}{3.372311in}}{\pgfqpoint{1.433445in}{3.368652in}}{\pgfqpoint{1.426934in}{3.362141in}}%
\pgfpathcurveto{\pgfqpoint{1.420422in}{3.355630in}}{\pgfqpoint{1.416764in}{3.346797in}}{\pgfqpoint{1.416764in}{3.337589in}}%
\pgfpathcurveto{\pgfqpoint{1.416764in}{3.328380in}}{\pgfqpoint{1.420422in}{3.319548in}}{\pgfqpoint{1.426934in}{3.313036in}}%
\pgfpathcurveto{\pgfqpoint{1.433445in}{3.306525in}}{\pgfqpoint{1.442278in}{3.302867in}}{\pgfqpoint{1.451486in}{3.302867in}}%
\pgfpathlineto{\pgfqpoint{1.451486in}{3.302867in}}%
\pgfpathclose%
\pgfusepath{stroke,fill}%
\end{pgfscope}%
\begin{pgfscope}%
\pgfpathrectangle{\pgfqpoint{1.374500in}{0.082500in}}{\pgfqpoint{2.419000in}{2.419000in}}%
\pgfusepath{clip}%
\pgfsetbuttcap%
\pgfsetroundjoin%
\definecolor{currentfill}{rgb}{1.000000,0.662745,0.054902}%
\pgfsetfillcolor{currentfill}%
\pgfsetfillopacity{0.343906}%
\pgfsetlinewidth{1.003750pt}%
\definecolor{currentstroke}{rgb}{1.000000,0.662745,0.054902}%
\pgfsetstrokecolor{currentstroke}%
\pgfsetstrokeopacity{0.343906}%
\pgfsetdash{}{0pt}%
\pgfpathmoveto{\pgfqpoint{4.944831in}{3.302867in}}%
\pgfpathcurveto{\pgfqpoint{4.954039in}{3.302867in}}{\pgfqpoint{4.962872in}{3.306525in}}{\pgfqpoint{4.969383in}{3.313036in}}%
\pgfpathcurveto{\pgfqpoint{4.975894in}{3.319548in}}{\pgfqpoint{4.979553in}{3.328380in}}{\pgfqpoint{4.979553in}{3.337589in}}%
\pgfpathcurveto{\pgfqpoint{4.979553in}{3.346797in}}{\pgfqpoint{4.975894in}{3.355630in}}{\pgfqpoint{4.969383in}{3.362141in}}%
\pgfpathcurveto{\pgfqpoint{4.962872in}{3.368652in}}{\pgfqpoint{4.954039in}{3.372311in}}{\pgfqpoint{4.944831in}{3.372311in}}%
\pgfpathcurveto{\pgfqpoint{4.935622in}{3.372311in}}{\pgfqpoint{4.926790in}{3.368652in}}{\pgfqpoint{4.920278in}{3.362141in}}%
\pgfpathcurveto{\pgfqpoint{4.913767in}{3.355630in}}{\pgfqpoint{4.910108in}{3.346797in}}{\pgfqpoint{4.910108in}{3.337589in}}%
\pgfpathcurveto{\pgfqpoint{4.910108in}{3.328380in}}{\pgfqpoint{4.913767in}{3.319548in}}{\pgfqpoint{4.920278in}{3.313036in}}%
\pgfpathcurveto{\pgfqpoint{4.926790in}{3.306525in}}{\pgfqpoint{4.935622in}{3.302867in}}{\pgfqpoint{4.944831in}{3.302867in}}%
\pgfpathlineto{\pgfqpoint{4.944831in}{3.302867in}}%
\pgfpathclose%
\pgfusepath{stroke,fill}%
\end{pgfscope}%
\begin{pgfscope}%
\pgfpathrectangle{\pgfqpoint{1.374500in}{0.082500in}}{\pgfqpoint{2.419000in}{2.419000in}}%
\pgfusepath{clip}%
\pgfsetbuttcap%
\pgfsetroundjoin%
\definecolor{currentfill}{rgb}{1.000000,0.662745,0.054902}%
\pgfsetfillcolor{currentfill}%
\pgfsetfillopacity{0.343906}%
\pgfsetlinewidth{1.003750pt}%
\definecolor{currentstroke}{rgb}{1.000000,0.662745,0.054902}%
\pgfsetstrokecolor{currentstroke}%
\pgfsetstrokeopacity{0.343906}%
\pgfsetdash{}{0pt}%
\pgfpathmoveto{\pgfqpoint{8.438175in}{3.302867in}}%
\pgfpathcurveto{\pgfqpoint{8.447384in}{3.302867in}}{\pgfqpoint{8.456216in}{3.306525in}}{\pgfqpoint{8.462728in}{3.313036in}}%
\pgfpathcurveto{\pgfqpoint{8.469239in}{3.319548in}}{\pgfqpoint{8.472898in}{3.328380in}}{\pgfqpoint{8.472898in}{3.337589in}}%
\pgfpathcurveto{\pgfqpoint{8.472898in}{3.346797in}}{\pgfqpoint{8.469239in}{3.355630in}}{\pgfqpoint{8.462728in}{3.362141in}}%
\pgfpathcurveto{\pgfqpoint{8.456216in}{3.368652in}}{\pgfqpoint{8.447384in}{3.372311in}}{\pgfqpoint{8.438175in}{3.372311in}}%
\pgfpathcurveto{\pgfqpoint{8.428967in}{3.372311in}}{\pgfqpoint{8.420134in}{3.368652in}}{\pgfqpoint{8.413623in}{3.362141in}}%
\pgfpathcurveto{\pgfqpoint{8.407112in}{3.355630in}}{\pgfqpoint{8.403453in}{3.346797in}}{\pgfqpoint{8.403453in}{3.337589in}}%
\pgfpathcurveto{\pgfqpoint{8.403453in}{3.328380in}}{\pgfqpoint{8.407112in}{3.319548in}}{\pgfqpoint{8.413623in}{3.313036in}}%
\pgfpathcurveto{\pgfqpoint{8.420134in}{3.306525in}}{\pgfqpoint{8.428967in}{3.302867in}}{\pgfqpoint{8.438175in}{3.302867in}}%
\pgfpathlineto{\pgfqpoint{8.438175in}{3.302867in}}%
\pgfpathclose%
\pgfusepath{stroke,fill}%
\end{pgfscope}%
\begin{pgfscope}%
\pgfpathrectangle{\pgfqpoint{1.374500in}{0.082500in}}{\pgfqpoint{2.419000in}{2.419000in}}%
\pgfusepath{clip}%
\pgfsetbuttcap%
\pgfsetroundjoin%
\definecolor{currentfill}{rgb}{1.000000,0.662745,0.054902}%
\pgfsetfillcolor{currentfill}%
\pgfsetfillopacity{0.347242}%
\pgfsetlinewidth{1.003750pt}%
\definecolor{currentstroke}{rgb}{1.000000,0.662745,0.054902}%
\pgfsetstrokecolor{currentstroke}%
\pgfsetstrokeopacity{0.347242}%
\pgfsetdash{}{0pt}%
\pgfpathmoveto{\pgfqpoint{3.537187in}{3.232654in}}%
\pgfpathcurveto{\pgfqpoint{3.546396in}{3.232654in}}{\pgfqpoint{3.555228in}{3.236313in}}{\pgfqpoint{3.561739in}{3.242824in}}%
\pgfpathcurveto{\pgfqpoint{3.568251in}{3.249335in}}{\pgfqpoint{3.571909in}{3.258168in}}{\pgfqpoint{3.571909in}{3.267376in}}%
\pgfpathcurveto{\pgfqpoint{3.571909in}{3.276585in}}{\pgfqpoint{3.568251in}{3.285417in}}{\pgfqpoint{3.561739in}{3.291929in}}%
\pgfpathcurveto{\pgfqpoint{3.555228in}{3.298440in}}{\pgfqpoint{3.546396in}{3.302098in}}{\pgfqpoint{3.537187in}{3.302098in}}%
\pgfpathcurveto{\pgfqpoint{3.527979in}{3.302098in}}{\pgfqpoint{3.519146in}{3.298440in}}{\pgfqpoint{3.512635in}{3.291929in}}%
\pgfpathcurveto{\pgfqpoint{3.506123in}{3.285417in}}{\pgfqpoint{3.502465in}{3.276585in}}{\pgfqpoint{3.502465in}{3.267376in}}%
\pgfpathcurveto{\pgfqpoint{3.502465in}{3.258168in}}{\pgfqpoint{3.506123in}{3.249335in}}{\pgfqpoint{3.512635in}{3.242824in}}%
\pgfpathcurveto{\pgfqpoint{3.519146in}{3.236313in}}{\pgfqpoint{3.527979in}{3.232654in}}{\pgfqpoint{3.537187in}{3.232654in}}%
\pgfpathlineto{\pgfqpoint{3.537187in}{3.232654in}}%
\pgfpathclose%
\pgfusepath{stroke,fill}%
\end{pgfscope}%
\begin{pgfscope}%
\pgfpathrectangle{\pgfqpoint{1.374500in}{0.082500in}}{\pgfqpoint{2.419000in}{2.419000in}}%
\pgfusepath{clip}%
\pgfsetbuttcap%
\pgfsetroundjoin%
\definecolor{currentfill}{rgb}{1.000000,0.662745,0.054902}%
\pgfsetfillcolor{currentfill}%
\pgfsetfillopacity{0.347242}%
\pgfsetlinewidth{1.003750pt}%
\definecolor{currentstroke}{rgb}{1.000000,0.662745,0.054902}%
\pgfsetstrokecolor{currentstroke}%
\pgfsetstrokeopacity{0.347242}%
\pgfsetdash{}{0pt}%
\pgfpathmoveto{\pgfqpoint{7.074835in}{3.232654in}}%
\pgfpathcurveto{\pgfqpoint{7.084043in}{3.232654in}}{\pgfqpoint{7.092876in}{3.236313in}}{\pgfqpoint{7.099387in}{3.242824in}}%
\pgfpathcurveto{\pgfqpoint{7.105898in}{3.249335in}}{\pgfqpoint{7.109557in}{3.258168in}}{\pgfqpoint{7.109557in}{3.267376in}}%
\pgfpathcurveto{\pgfqpoint{7.109557in}{3.276585in}}{\pgfqpoint{7.105898in}{3.285417in}}{\pgfqpoint{7.099387in}{3.291929in}}%
\pgfpathcurveto{\pgfqpoint{7.092876in}{3.298440in}}{\pgfqpoint{7.084043in}{3.302098in}}{\pgfqpoint{7.074835in}{3.302098in}}%
\pgfpathcurveto{\pgfqpoint{7.065626in}{3.302098in}}{\pgfqpoint{7.056794in}{3.298440in}}{\pgfqpoint{7.050282in}{3.291929in}}%
\pgfpathcurveto{\pgfqpoint{7.043771in}{3.285417in}}{\pgfqpoint{7.040113in}{3.276585in}}{\pgfqpoint{7.040113in}{3.267376in}}%
\pgfpathcurveto{\pgfqpoint{7.040113in}{3.258168in}}{\pgfqpoint{7.043771in}{3.249335in}}{\pgfqpoint{7.050282in}{3.242824in}}%
\pgfpathcurveto{\pgfqpoint{7.056794in}{3.236313in}}{\pgfqpoint{7.065626in}{3.232654in}}{\pgfqpoint{7.074835in}{3.232654in}}%
\pgfpathlineto{\pgfqpoint{7.074835in}{3.232654in}}%
\pgfpathclose%
\pgfusepath{stroke,fill}%
\end{pgfscope}%
\begin{pgfscope}%
\pgfpathrectangle{\pgfqpoint{1.374500in}{0.082500in}}{\pgfqpoint{2.419000in}{2.419000in}}%
\pgfusepath{clip}%
\pgfsetbuttcap%
\pgfsetroundjoin%
\definecolor{currentfill}{rgb}{1.000000,0.662745,0.054902}%
\pgfsetfillcolor{currentfill}%
\pgfsetfillopacity{0.350663}%
\pgfsetlinewidth{1.003750pt}%
\definecolor{currentstroke}{rgb}{1.000000,0.662745,0.054902}%
\pgfsetstrokecolor{currentstroke}%
\pgfsetstrokeopacity{0.350663}%
\pgfsetdash{}{0pt}%
\pgfpathmoveto{\pgfqpoint{2.093381in}{3.160638in}}%
\pgfpathcurveto{\pgfqpoint{2.102589in}{3.160638in}}{\pgfqpoint{2.111422in}{3.164296in}}{\pgfqpoint{2.117933in}{3.170808in}}%
\pgfpathcurveto{\pgfqpoint{2.124445in}{3.177319in}}{\pgfqpoint{2.128103in}{3.186151in}}{\pgfqpoint{2.128103in}{3.195360in}}%
\pgfpathcurveto{\pgfqpoint{2.128103in}{3.204568in}}{\pgfqpoint{2.124445in}{3.213401in}}{\pgfqpoint{2.117933in}{3.219912in}}%
\pgfpathcurveto{\pgfqpoint{2.111422in}{3.226424in}}{\pgfqpoint{2.102589in}{3.230082in}}{\pgfqpoint{2.093381in}{3.230082in}}%
\pgfpathcurveto{\pgfqpoint{2.084173in}{3.230082in}}{\pgfqpoint{2.075340in}{3.226424in}}{\pgfqpoint{2.068829in}{3.219912in}}%
\pgfpathcurveto{\pgfqpoint{2.062317in}{3.213401in}}{\pgfqpoint{2.058659in}{3.204568in}}{\pgfqpoint{2.058659in}{3.195360in}}%
\pgfpathcurveto{\pgfqpoint{2.058659in}{3.186151in}}{\pgfqpoint{2.062317in}{3.177319in}}{\pgfqpoint{2.068829in}{3.170808in}}%
\pgfpathcurveto{\pgfqpoint{2.075340in}{3.164296in}}{\pgfqpoint{2.084173in}{3.160638in}}{\pgfqpoint{2.093381in}{3.160638in}}%
\pgfpathlineto{\pgfqpoint{2.093381in}{3.160638in}}%
\pgfpathclose%
\pgfusepath{stroke,fill}%
\end{pgfscope}%
\begin{pgfscope}%
\pgfpathrectangle{\pgfqpoint{1.374500in}{0.082500in}}{\pgfqpoint{2.419000in}{2.419000in}}%
\pgfusepath{clip}%
\pgfsetbuttcap%
\pgfsetroundjoin%
\definecolor{currentfill}{rgb}{1.000000,0.662745,0.054902}%
\pgfsetfillcolor{currentfill}%
\pgfsetfillopacity{0.350663}%
\pgfsetlinewidth{1.003750pt}%
\definecolor{currentstroke}{rgb}{1.000000,0.662745,0.054902}%
\pgfsetstrokecolor{currentstroke}%
\pgfsetstrokeopacity{0.350663}%
\pgfsetdash{}{0pt}%
\pgfpathmoveto{\pgfqpoint{5.676470in}{3.160638in}}%
\pgfpathcurveto{\pgfqpoint{5.685678in}{3.160638in}}{\pgfqpoint{5.694511in}{3.164296in}}{\pgfqpoint{5.701022in}{3.170808in}}%
\pgfpathcurveto{\pgfqpoint{5.707534in}{3.177319in}}{\pgfqpoint{5.711192in}{3.186151in}}{\pgfqpoint{5.711192in}{3.195360in}}%
\pgfpathcurveto{\pgfqpoint{5.711192in}{3.204568in}}{\pgfqpoint{5.707534in}{3.213401in}}{\pgfqpoint{5.701022in}{3.219912in}}%
\pgfpathcurveto{\pgfqpoint{5.694511in}{3.226424in}}{\pgfqpoint{5.685678in}{3.230082in}}{\pgfqpoint{5.676470in}{3.230082in}}%
\pgfpathcurveto{\pgfqpoint{5.667261in}{3.230082in}}{\pgfqpoint{5.658429in}{3.226424in}}{\pgfqpoint{5.651918in}{3.219912in}}%
\pgfpathcurveto{\pgfqpoint{5.645406in}{3.213401in}}{\pgfqpoint{5.641748in}{3.204568in}}{\pgfqpoint{5.641748in}{3.195360in}}%
\pgfpathcurveto{\pgfqpoint{5.641748in}{3.186151in}}{\pgfqpoint{5.645406in}{3.177319in}}{\pgfqpoint{5.651918in}{3.170808in}}%
\pgfpathcurveto{\pgfqpoint{5.658429in}{3.164296in}}{\pgfqpoint{5.667261in}{3.160638in}}{\pgfqpoint{5.676470in}{3.160638in}}%
\pgfpathlineto{\pgfqpoint{5.676470in}{3.160638in}}%
\pgfpathclose%
\pgfusepath{stroke,fill}%
\end{pgfscope}%
\begin{pgfscope}%
\pgfpathrectangle{\pgfqpoint{1.374500in}{0.082500in}}{\pgfqpoint{2.419000in}{2.419000in}}%
\pgfusepath{clip}%
\pgfsetbuttcap%
\pgfsetroundjoin%
\definecolor{currentfill}{rgb}{1.000000,0.662745,0.054902}%
\pgfsetfillcolor{currentfill}%
\pgfsetfillopacity{0.350663}%
\pgfsetlinewidth{1.003750pt}%
\definecolor{currentstroke}{rgb}{1.000000,0.662745,0.054902}%
\pgfsetstrokecolor{currentstroke}%
\pgfsetstrokeopacity{0.350663}%
\pgfsetdash{}{0pt}%
\pgfpathmoveto{\pgfqpoint{9.259559in}{3.160638in}}%
\pgfpathcurveto{\pgfqpoint{9.268767in}{3.160638in}}{\pgfqpoint{9.277600in}{3.164296in}}{\pgfqpoint{9.284111in}{3.170808in}}%
\pgfpathcurveto{\pgfqpoint{9.290622in}{3.177319in}}{\pgfqpoint{9.294281in}{3.186151in}}{\pgfqpoint{9.294281in}{3.195360in}}%
\pgfpathcurveto{\pgfqpoint{9.294281in}{3.204568in}}{\pgfqpoint{9.290622in}{3.213401in}}{\pgfqpoint{9.284111in}{3.219912in}}%
\pgfpathcurveto{\pgfqpoint{9.277600in}{3.226424in}}{\pgfqpoint{9.268767in}{3.230082in}}{\pgfqpoint{9.259559in}{3.230082in}}%
\pgfpathcurveto{\pgfqpoint{9.250350in}{3.230082in}}{\pgfqpoint{9.241518in}{3.226424in}}{\pgfqpoint{9.235006in}{3.219912in}}%
\pgfpathcurveto{\pgfqpoint{9.228495in}{3.213401in}}{\pgfqpoint{9.224837in}{3.204568in}}{\pgfqpoint{9.224837in}{3.195360in}}%
\pgfpathcurveto{\pgfqpoint{9.224837in}{3.186151in}}{\pgfqpoint{9.228495in}{3.177319in}}{\pgfqpoint{9.235006in}{3.170808in}}%
\pgfpathcurveto{\pgfqpoint{9.241518in}{3.164296in}}{\pgfqpoint{9.250350in}{3.160638in}}{\pgfqpoint{9.259559in}{3.160638in}}%
\pgfpathlineto{\pgfqpoint{9.259559in}{3.160638in}}%
\pgfpathclose%
\pgfusepath{stroke,fill}%
\end{pgfscope}%
\begin{pgfscope}%
\pgfpathrectangle{\pgfqpoint{1.374500in}{0.082500in}}{\pgfqpoint{2.419000in}{2.419000in}}%
\pgfusepath{clip}%
\pgfsetbuttcap%
\pgfsetroundjoin%
\definecolor{currentfill}{rgb}{1.000000,0.662745,0.054902}%
\pgfsetfillcolor{currentfill}%
\pgfsetfillopacity{0.354174}%
\pgfsetlinewidth{1.003750pt}%
\definecolor{currentstroke}{rgb}{1.000000,0.662745,0.054902}%
\pgfsetstrokecolor{currentstroke}%
\pgfsetstrokeopacity{0.354174}%
\pgfsetdash{}{0pt}%
\pgfpathmoveto{\pgfqpoint{4.241714in}{3.086747in}}%
\pgfpathcurveto{\pgfqpoint{4.250922in}{3.086747in}}{\pgfqpoint{4.259754in}{3.090406in}}{\pgfqpoint{4.266266in}{3.096917in}}%
\pgfpathcurveto{\pgfqpoint{4.272777in}{3.103428in}}{\pgfqpoint{4.276436in}{3.112261in}}{\pgfqpoint{4.276436in}{3.121469in}}%
\pgfpathcurveto{\pgfqpoint{4.276436in}{3.130678in}}{\pgfqpoint{4.272777in}{3.139510in}}{\pgfqpoint{4.266266in}{3.146022in}}%
\pgfpathcurveto{\pgfqpoint{4.259754in}{3.152533in}}{\pgfqpoint{4.250922in}{3.156192in}}{\pgfqpoint{4.241714in}{3.156192in}}%
\pgfpathcurveto{\pgfqpoint{4.232505in}{3.156192in}}{\pgfqpoint{4.223673in}{3.152533in}}{\pgfqpoint{4.217161in}{3.146022in}}%
\pgfpathcurveto{\pgfqpoint{4.210650in}{3.139510in}}{\pgfqpoint{4.206991in}{3.130678in}}{\pgfqpoint{4.206991in}{3.121469in}}%
\pgfpathcurveto{\pgfqpoint{4.206991in}{3.112261in}}{\pgfqpoint{4.210650in}{3.103428in}}{\pgfqpoint{4.217161in}{3.096917in}}%
\pgfpathcurveto{\pgfqpoint{4.223673in}{3.090406in}}{\pgfqpoint{4.232505in}{3.086747in}}{\pgfqpoint{4.241714in}{3.086747in}}%
\pgfpathlineto{\pgfqpoint{4.241714in}{3.086747in}}%
\pgfpathclose%
\pgfusepath{stroke,fill}%
\end{pgfscope}%
\begin{pgfscope}%
\pgfpathrectangle{\pgfqpoint{1.374500in}{0.082500in}}{\pgfqpoint{2.419000in}{2.419000in}}%
\pgfusepath{clip}%
\pgfsetbuttcap%
\pgfsetroundjoin%
\definecolor{currentfill}{rgb}{1.000000,0.662745,0.054902}%
\pgfsetfillcolor{currentfill}%
\pgfsetfillopacity{0.354174}%
\pgfsetlinewidth{1.003750pt}%
\definecolor{currentstroke}{rgb}{1.000000,0.662745,0.054902}%
\pgfsetstrokecolor{currentstroke}%
\pgfsetstrokeopacity{0.354174}%
\pgfsetdash{}{0pt}%
\pgfpathmoveto{\pgfqpoint{7.871426in}{3.086747in}}%
\pgfpathcurveto{\pgfqpoint{7.880635in}{3.086747in}}{\pgfqpoint{7.889467in}{3.090406in}}{\pgfqpoint{7.895978in}{3.096917in}}%
\pgfpathcurveto{\pgfqpoint{7.902490in}{3.103428in}}{\pgfqpoint{7.906148in}{3.112261in}}{\pgfqpoint{7.906148in}{3.121469in}}%
\pgfpathcurveto{\pgfqpoint{7.906148in}{3.130678in}}{\pgfqpoint{7.902490in}{3.139510in}}{\pgfqpoint{7.895978in}{3.146022in}}%
\pgfpathcurveto{\pgfqpoint{7.889467in}{3.152533in}}{\pgfqpoint{7.880635in}{3.156192in}}{\pgfqpoint{7.871426in}{3.156192in}}%
\pgfpathcurveto{\pgfqpoint{7.862218in}{3.156192in}}{\pgfqpoint{7.853385in}{3.152533in}}{\pgfqpoint{7.846874in}{3.146022in}}%
\pgfpathcurveto{\pgfqpoint{7.840362in}{3.139510in}}{\pgfqpoint{7.836704in}{3.130678in}}{\pgfqpoint{7.836704in}{3.121469in}}%
\pgfpathcurveto{\pgfqpoint{7.836704in}{3.112261in}}{\pgfqpoint{7.840362in}{3.103428in}}{\pgfqpoint{7.846874in}{3.096917in}}%
\pgfpathcurveto{\pgfqpoint{7.853385in}{3.090406in}}{\pgfqpoint{7.862218in}{3.086747in}}{\pgfqpoint{7.871426in}{3.086747in}}%
\pgfpathlineto{\pgfqpoint{7.871426in}{3.086747in}}%
\pgfpathclose%
\pgfusepath{stroke,fill}%
\end{pgfscope}%
\begin{pgfscope}%
\pgfpathrectangle{\pgfqpoint{1.374500in}{0.082500in}}{\pgfqpoint{2.419000in}{2.419000in}}%
\pgfusepath{clip}%
\pgfsetbuttcap%
\pgfsetroundjoin%
\definecolor{currentfill}{rgb}{1.000000,0.662745,0.054902}%
\pgfsetfillcolor{currentfill}%
\pgfsetfillopacity{0.357777}%
\pgfsetlinewidth{1.003750pt}%
\definecolor{currentstroke}{rgb}{1.000000,0.662745,0.054902}%
\pgfsetstrokecolor{currentstroke}%
\pgfsetstrokeopacity{0.357777}%
\pgfsetdash{}{0pt}%
\pgfpathmoveto{\pgfqpoint{2.769126in}{3.010908in}}%
\pgfpathcurveto{\pgfqpoint{2.778335in}{3.010908in}}{\pgfqpoint{2.787167in}{3.014567in}}{\pgfqpoint{2.793679in}{3.021078in}}%
\pgfpathcurveto{\pgfqpoint{2.800190in}{3.027590in}}{\pgfqpoint{2.803849in}{3.036422in}}{\pgfqpoint{2.803849in}{3.045631in}}%
\pgfpathcurveto{\pgfqpoint{2.803849in}{3.054839in}}{\pgfqpoint{2.800190in}{3.063672in}}{\pgfqpoint{2.793679in}{3.070183in}}%
\pgfpathcurveto{\pgfqpoint{2.787167in}{3.076694in}}{\pgfqpoint{2.778335in}{3.080353in}}{\pgfqpoint{2.769126in}{3.080353in}}%
\pgfpathcurveto{\pgfqpoint{2.759918in}{3.080353in}}{\pgfqpoint{2.751085in}{3.076694in}}{\pgfqpoint{2.744574in}{3.070183in}}%
\pgfpathcurveto{\pgfqpoint{2.738063in}{3.063672in}}{\pgfqpoint{2.734404in}{3.054839in}}{\pgfqpoint{2.734404in}{3.045631in}}%
\pgfpathcurveto{\pgfqpoint{2.734404in}{3.036422in}}{\pgfqpoint{2.738063in}{3.027590in}}{\pgfqpoint{2.744574in}{3.021078in}}%
\pgfpathcurveto{\pgfqpoint{2.751085in}{3.014567in}}{\pgfqpoint{2.759918in}{3.010908in}}{\pgfqpoint{2.769126in}{3.010908in}}%
\pgfpathlineto{\pgfqpoint{2.769126in}{3.010908in}}%
\pgfpathclose%
\pgfusepath{stroke,fill}%
\end{pgfscope}%
\begin{pgfscope}%
\pgfpathrectangle{\pgfqpoint{1.374500in}{0.082500in}}{\pgfqpoint{2.419000in}{2.419000in}}%
\pgfusepath{clip}%
\pgfsetbuttcap%
\pgfsetroundjoin%
\definecolor{currentfill}{rgb}{1.000000,0.662745,0.054902}%
\pgfsetfillcolor{currentfill}%
\pgfsetfillopacity{0.357777}%
\pgfsetlinewidth{1.003750pt}%
\definecolor{currentstroke}{rgb}{1.000000,0.662745,0.054902}%
\pgfsetstrokecolor{currentstroke}%
\pgfsetstrokeopacity{0.357777}%
\pgfsetdash{}{0pt}%
\pgfpathmoveto{\pgfqpoint{6.446692in}{3.010908in}}%
\pgfpathcurveto{\pgfqpoint{6.455900in}{3.010908in}}{\pgfqpoint{6.464733in}{3.014567in}}{\pgfqpoint{6.471244in}{3.021078in}}%
\pgfpathcurveto{\pgfqpoint{6.477756in}{3.027590in}}{\pgfqpoint{6.481414in}{3.036422in}}{\pgfqpoint{6.481414in}{3.045631in}}%
\pgfpathcurveto{\pgfqpoint{6.481414in}{3.054839in}}{\pgfqpoint{6.477756in}{3.063672in}}{\pgfqpoint{6.471244in}{3.070183in}}%
\pgfpathcurveto{\pgfqpoint{6.464733in}{3.076694in}}{\pgfqpoint{6.455900in}{3.080353in}}{\pgfqpoint{6.446692in}{3.080353in}}%
\pgfpathcurveto{\pgfqpoint{6.437484in}{3.080353in}}{\pgfqpoint{6.428651in}{3.076694in}}{\pgfqpoint{6.422140in}{3.070183in}}%
\pgfpathcurveto{\pgfqpoint{6.415628in}{3.063672in}}{\pgfqpoint{6.411970in}{3.054839in}}{\pgfqpoint{6.411970in}{3.045631in}}%
\pgfpathcurveto{\pgfqpoint{6.411970in}{3.036422in}}{\pgfqpoint{6.415628in}{3.027590in}}{\pgfqpoint{6.422140in}{3.021078in}}%
\pgfpathcurveto{\pgfqpoint{6.428651in}{3.014567in}}{\pgfqpoint{6.437484in}{3.010908in}}{\pgfqpoint{6.446692in}{3.010908in}}%
\pgfpathlineto{\pgfqpoint{6.446692in}{3.010908in}}%
\pgfpathclose%
\pgfusepath{stroke,fill}%
\end{pgfscope}%
\begin{pgfscope}%
\pgfpathrectangle{\pgfqpoint{1.374500in}{0.082500in}}{\pgfqpoint{2.419000in}{2.419000in}}%
\pgfusepath{clip}%
\pgfsetbuttcap%
\pgfsetroundjoin%
\definecolor{currentfill}{rgb}{1.000000,0.662745,0.054902}%
\pgfsetfillcolor{currentfill}%
\pgfsetfillopacity{0.357777}%
\pgfsetlinewidth{1.003750pt}%
\definecolor{currentstroke}{rgb}{1.000000,0.662745,0.054902}%
\pgfsetstrokecolor{currentstroke}%
\pgfsetstrokeopacity{0.357777}%
\pgfsetdash{}{0pt}%
\pgfpathmoveto{\pgfqpoint{10.124258in}{3.010908in}}%
\pgfpathcurveto{\pgfqpoint{10.133466in}{3.010908in}}{\pgfqpoint{10.142299in}{3.014567in}}{\pgfqpoint{10.148810in}{3.021078in}}%
\pgfpathcurveto{\pgfqpoint{10.155321in}{3.027590in}}{\pgfqpoint{10.158980in}{3.036422in}}{\pgfqpoint{10.158980in}{3.045631in}}%
\pgfpathcurveto{\pgfqpoint{10.158980in}{3.054839in}}{\pgfqpoint{10.155321in}{3.063672in}}{\pgfqpoint{10.148810in}{3.070183in}}%
\pgfpathcurveto{\pgfqpoint{10.142299in}{3.076694in}}{\pgfqpoint{10.133466in}{3.080353in}}{\pgfqpoint{10.124258in}{3.080353in}}%
\pgfpathcurveto{\pgfqpoint{10.115049in}{3.080353in}}{\pgfqpoint{10.106217in}{3.076694in}}{\pgfqpoint{10.099705in}{3.070183in}}%
\pgfpathcurveto{\pgfqpoint{10.093194in}{3.063672in}}{\pgfqpoint{10.089535in}{3.054839in}}{\pgfqpoint{10.089535in}{3.045631in}}%
\pgfpathcurveto{\pgfqpoint{10.089535in}{3.036422in}}{\pgfqpoint{10.093194in}{3.027590in}}{\pgfqpoint{10.099705in}{3.021078in}}%
\pgfpathcurveto{\pgfqpoint{10.106217in}{3.014567in}}{\pgfqpoint{10.115049in}{3.010908in}}{\pgfqpoint{10.124258in}{3.010908in}}%
\pgfpathlineto{\pgfqpoint{10.124258in}{3.010908in}}%
\pgfpathclose%
\pgfusepath{stroke,fill}%
\end{pgfscope}%
\begin{pgfscope}%
\pgfpathrectangle{\pgfqpoint{1.374500in}{0.082500in}}{\pgfqpoint{2.419000in}{2.419000in}}%
\pgfusepath{clip}%
\pgfsetbuttcap%
\pgfsetroundjoin%
\definecolor{currentfill}{rgb}{1.000000,0.662745,0.054902}%
\pgfsetfillcolor{currentfill}%
\pgfsetfillopacity{0.361476}%
\pgfsetlinewidth{1.003750pt}%
\definecolor{currentstroke}{rgb}{1.000000,0.662745,0.054902}%
\pgfsetstrokecolor{currentstroke}%
\pgfsetstrokeopacity{0.361476}%
\pgfsetdash{}{0pt}%
\pgfpathmoveto{\pgfqpoint{1.257192in}{2.933043in}}%
\pgfpathcurveto{\pgfqpoint{1.266400in}{2.933043in}}{\pgfqpoint{1.275233in}{2.936702in}}{\pgfqpoint{1.281744in}{2.943213in}}%
\pgfpathcurveto{\pgfqpoint{1.288256in}{2.949724in}}{\pgfqpoint{1.291914in}{2.958557in}}{\pgfqpoint{1.291914in}{2.967765in}}%
\pgfpathcurveto{\pgfqpoint{1.291914in}{2.976974in}}{\pgfqpoint{1.288256in}{2.985806in}}{\pgfqpoint{1.281744in}{2.992318in}}%
\pgfpathcurveto{\pgfqpoint{1.275233in}{2.998829in}}{\pgfqpoint{1.266400in}{3.002488in}}{\pgfqpoint{1.257192in}{3.002488in}}%
\pgfpathcurveto{\pgfqpoint{1.247984in}{3.002488in}}{\pgfqpoint{1.239151in}{2.998829in}}{\pgfqpoint{1.232640in}{2.992318in}}%
\pgfpathcurveto{\pgfqpoint{1.226128in}{2.985806in}}{\pgfqpoint{1.222470in}{2.976974in}}{\pgfqpoint{1.222470in}{2.967765in}}%
\pgfpathcurveto{\pgfqpoint{1.222470in}{2.958557in}}{\pgfqpoint{1.226128in}{2.949724in}}{\pgfqpoint{1.232640in}{2.943213in}}%
\pgfpathcurveto{\pgfqpoint{1.239151in}{2.936702in}}{\pgfqpoint{1.247984in}{2.933043in}}{\pgfqpoint{1.257192in}{2.933043in}}%
\pgfpathlineto{\pgfqpoint{1.257192in}{2.933043in}}%
\pgfpathclose%
\pgfusepath{stroke,fill}%
\end{pgfscope}%
\begin{pgfscope}%
\pgfpathrectangle{\pgfqpoint{1.374500in}{0.082500in}}{\pgfqpoint{2.419000in}{2.419000in}}%
\pgfusepath{clip}%
\pgfsetbuttcap%
\pgfsetroundjoin%
\definecolor{currentfill}{rgb}{1.000000,0.662745,0.054902}%
\pgfsetfillcolor{currentfill}%
\pgfsetfillopacity{0.361476}%
\pgfsetlinewidth{1.003750pt}%
\definecolor{currentstroke}{rgb}{1.000000,0.662745,0.054902}%
\pgfsetstrokecolor{currentstroke}%
\pgfsetstrokeopacity{0.361476}%
\pgfsetdash{}{0pt}%
\pgfpathmoveto{\pgfqpoint{4.983889in}{2.933043in}}%
\pgfpathcurveto{\pgfqpoint{4.993098in}{2.933043in}}{\pgfqpoint{5.001930in}{2.936702in}}{\pgfqpoint{5.008442in}{2.943213in}}%
\pgfpathcurveto{\pgfqpoint{5.014953in}{2.949724in}}{\pgfqpoint{5.018612in}{2.958557in}}{\pgfqpoint{5.018612in}{2.967765in}}%
\pgfpathcurveto{\pgfqpoint{5.018612in}{2.976974in}}{\pgfqpoint{5.014953in}{2.985806in}}{\pgfqpoint{5.008442in}{2.992318in}}%
\pgfpathcurveto{\pgfqpoint{5.001930in}{2.998829in}}{\pgfqpoint{4.993098in}{3.002488in}}{\pgfqpoint{4.983889in}{3.002488in}}%
\pgfpathcurveto{\pgfqpoint{4.974681in}{3.002488in}}{\pgfqpoint{4.965848in}{2.998829in}}{\pgfqpoint{4.959337in}{2.992318in}}%
\pgfpathcurveto{\pgfqpoint{4.952826in}{2.985806in}}{\pgfqpoint{4.949167in}{2.976974in}}{\pgfqpoint{4.949167in}{2.967765in}}%
\pgfpathcurveto{\pgfqpoint{4.949167in}{2.958557in}}{\pgfqpoint{4.952826in}{2.949724in}}{\pgfqpoint{4.959337in}{2.943213in}}%
\pgfpathcurveto{\pgfqpoint{4.965848in}{2.936702in}}{\pgfqpoint{4.974681in}{2.933043in}}{\pgfqpoint{4.983889in}{2.933043in}}%
\pgfpathlineto{\pgfqpoint{4.983889in}{2.933043in}}%
\pgfpathclose%
\pgfusepath{stroke,fill}%
\end{pgfscope}%
\begin{pgfscope}%
\pgfpathrectangle{\pgfqpoint{1.374500in}{0.082500in}}{\pgfqpoint{2.419000in}{2.419000in}}%
\pgfusepath{clip}%
\pgfsetbuttcap%
\pgfsetroundjoin%
\definecolor{currentfill}{rgb}{1.000000,0.662745,0.054902}%
\pgfsetfillcolor{currentfill}%
\pgfsetfillopacity{0.361476}%
\pgfsetlinewidth{1.003750pt}%
\definecolor{currentstroke}{rgb}{1.000000,0.662745,0.054902}%
\pgfsetstrokecolor{currentstroke}%
\pgfsetstrokeopacity{0.361476}%
\pgfsetdash{}{0pt}%
\pgfpathmoveto{\pgfqpoint{8.710587in}{2.933043in}}%
\pgfpathcurveto{\pgfqpoint{8.719795in}{2.933043in}}{\pgfqpoint{8.728628in}{2.936702in}}{\pgfqpoint{8.735139in}{2.943213in}}%
\pgfpathcurveto{\pgfqpoint{8.741650in}{2.949724in}}{\pgfqpoint{8.745309in}{2.958557in}}{\pgfqpoint{8.745309in}{2.967765in}}%
\pgfpathcurveto{\pgfqpoint{8.745309in}{2.976974in}}{\pgfqpoint{8.741650in}{2.985806in}}{\pgfqpoint{8.735139in}{2.992318in}}%
\pgfpathcurveto{\pgfqpoint{8.728628in}{2.998829in}}{\pgfqpoint{8.719795in}{3.002488in}}{\pgfqpoint{8.710587in}{3.002488in}}%
\pgfpathcurveto{\pgfqpoint{8.701378in}{3.002488in}}{\pgfqpoint{8.692546in}{2.998829in}}{\pgfqpoint{8.686034in}{2.992318in}}%
\pgfpathcurveto{\pgfqpoint{8.679523in}{2.985806in}}{\pgfqpoint{8.675865in}{2.976974in}}{\pgfqpoint{8.675865in}{2.967765in}}%
\pgfpathcurveto{\pgfqpoint{8.675865in}{2.958557in}}{\pgfqpoint{8.679523in}{2.949724in}}{\pgfqpoint{8.686034in}{2.943213in}}%
\pgfpathcurveto{\pgfqpoint{8.692546in}{2.936702in}}{\pgfqpoint{8.701378in}{2.933043in}}{\pgfqpoint{8.710587in}{2.933043in}}%
\pgfpathlineto{\pgfqpoint{8.710587in}{2.933043in}}%
\pgfpathclose%
\pgfusepath{stroke,fill}%
\end{pgfscope}%
\begin{pgfscope}%
\pgfpathrectangle{\pgfqpoint{1.374500in}{0.082500in}}{\pgfqpoint{2.419000in}{2.419000in}}%
\pgfusepath{clip}%
\pgfsetbuttcap%
\pgfsetroundjoin%
\definecolor{currentfill}{rgb}{1.000000,0.662745,0.054902}%
\pgfsetfillcolor{currentfill}%
\pgfsetfillopacity{0.365275}%
\pgfsetlinewidth{1.003750pt}%
\definecolor{currentstroke}{rgb}{1.000000,0.662745,0.054902}%
\pgfsetstrokecolor{currentstroke}%
\pgfsetstrokeopacity{0.365275}%
\pgfsetdash{}{0pt}%
\pgfpathmoveto{\pgfqpoint{3.481472in}{2.853069in}}%
\pgfpathcurveto{\pgfqpoint{3.490680in}{2.853069in}}{\pgfqpoint{3.499513in}{2.856728in}}{\pgfqpoint{3.506024in}{2.863239in}}%
\pgfpathcurveto{\pgfqpoint{3.512536in}{2.869750in}}{\pgfqpoint{3.516194in}{2.878583in}}{\pgfqpoint{3.516194in}{2.887791in}}%
\pgfpathcurveto{\pgfqpoint{3.516194in}{2.897000in}}{\pgfqpoint{3.512536in}{2.905832in}}{\pgfqpoint{3.506024in}{2.912344in}}%
\pgfpathcurveto{\pgfqpoint{3.499513in}{2.918855in}}{\pgfqpoint{3.490680in}{2.922514in}}{\pgfqpoint{3.481472in}{2.922514in}}%
\pgfpathcurveto{\pgfqpoint{3.472264in}{2.922514in}}{\pgfqpoint{3.463431in}{2.918855in}}{\pgfqpoint{3.456920in}{2.912344in}}%
\pgfpathcurveto{\pgfqpoint{3.450408in}{2.905832in}}{\pgfqpoint{3.446750in}{2.897000in}}{\pgfqpoint{3.446750in}{2.887791in}}%
\pgfpathcurveto{\pgfqpoint{3.446750in}{2.878583in}}{\pgfqpoint{3.450408in}{2.869750in}}{\pgfqpoint{3.456920in}{2.863239in}}%
\pgfpathcurveto{\pgfqpoint{3.463431in}{2.856728in}}{\pgfqpoint{3.472264in}{2.853069in}}{\pgfqpoint{3.481472in}{2.853069in}}%
\pgfpathlineto{\pgfqpoint{3.481472in}{2.853069in}}%
\pgfpathclose%
\pgfusepath{stroke,fill}%
\end{pgfscope}%
\begin{pgfscope}%
\pgfpathrectangle{\pgfqpoint{1.374500in}{0.082500in}}{\pgfqpoint{2.419000in}{2.419000in}}%
\pgfusepath{clip}%
\pgfsetbuttcap%
\pgfsetroundjoin%
\definecolor{currentfill}{rgb}{1.000000,0.662745,0.054902}%
\pgfsetfillcolor{currentfill}%
\pgfsetfillopacity{0.365275}%
\pgfsetlinewidth{1.003750pt}%
\definecolor{currentstroke}{rgb}{1.000000,0.662745,0.054902}%
\pgfsetstrokecolor{currentstroke}%
\pgfsetstrokeopacity{0.365275}%
\pgfsetdash{}{0pt}%
\pgfpathmoveto{\pgfqpoint{7.258632in}{2.853069in}}%
\pgfpathcurveto{\pgfqpoint{7.267840in}{2.853069in}}{\pgfqpoint{7.276673in}{2.856728in}}{\pgfqpoint{7.283184in}{2.863239in}}%
\pgfpathcurveto{\pgfqpoint{7.289695in}{2.869750in}}{\pgfqpoint{7.293354in}{2.878583in}}{\pgfqpoint{7.293354in}{2.887791in}}%
\pgfpathcurveto{\pgfqpoint{7.293354in}{2.897000in}}{\pgfqpoint{7.289695in}{2.905832in}}{\pgfqpoint{7.283184in}{2.912344in}}%
\pgfpathcurveto{\pgfqpoint{7.276673in}{2.918855in}}{\pgfqpoint{7.267840in}{2.922514in}}{\pgfqpoint{7.258632in}{2.922514in}}%
\pgfpathcurveto{\pgfqpoint{7.249423in}{2.922514in}}{\pgfqpoint{7.240591in}{2.918855in}}{\pgfqpoint{7.234079in}{2.912344in}}%
\pgfpathcurveto{\pgfqpoint{7.227568in}{2.905832in}}{\pgfqpoint{7.223909in}{2.897000in}}{\pgfqpoint{7.223909in}{2.887791in}}%
\pgfpathcurveto{\pgfqpoint{7.223909in}{2.878583in}}{\pgfqpoint{7.227568in}{2.869750in}}{\pgfqpoint{7.234079in}{2.863239in}}%
\pgfpathcurveto{\pgfqpoint{7.240591in}{2.856728in}}{\pgfqpoint{7.249423in}{2.853069in}}{\pgfqpoint{7.258632in}{2.853069in}}%
\pgfpathlineto{\pgfqpoint{7.258632in}{2.853069in}}%
\pgfpathclose%
\pgfusepath{stroke,fill}%
\end{pgfscope}%
\begin{pgfscope}%
\pgfpathrectangle{\pgfqpoint{1.374500in}{0.082500in}}{\pgfqpoint{2.419000in}{2.419000in}}%
\pgfusepath{clip}%
\pgfsetbuttcap%
\pgfsetroundjoin%
\definecolor{currentfill}{rgb}{1.000000,0.662745,0.054902}%
\pgfsetfillcolor{currentfill}%
\pgfsetfillopacity{0.365275}%
\pgfsetlinewidth{1.003750pt}%
\definecolor{currentstroke}{rgb}{1.000000,0.662745,0.054902}%
\pgfsetstrokecolor{currentstroke}%
\pgfsetstrokeopacity{0.365275}%
\pgfsetdash{}{0pt}%
\pgfpathmoveto{\pgfqpoint{11.035791in}{2.853069in}}%
\pgfpathcurveto{\pgfqpoint{11.045000in}{2.853069in}}{\pgfqpoint{11.053832in}{2.856728in}}{\pgfqpoint{11.060344in}{2.863239in}}%
\pgfpathcurveto{\pgfqpoint{11.066855in}{2.869750in}}{\pgfqpoint{11.070513in}{2.878583in}}{\pgfqpoint{11.070513in}{2.887791in}}%
\pgfpathcurveto{\pgfqpoint{11.070513in}{2.897000in}}{\pgfqpoint{11.066855in}{2.905832in}}{\pgfqpoint{11.060344in}{2.912344in}}%
\pgfpathcurveto{\pgfqpoint{11.053832in}{2.918855in}}{\pgfqpoint{11.045000in}{2.922514in}}{\pgfqpoint{11.035791in}{2.922514in}}%
\pgfpathcurveto{\pgfqpoint{11.026583in}{2.922514in}}{\pgfqpoint{11.017750in}{2.918855in}}{\pgfqpoint{11.011239in}{2.912344in}}%
\pgfpathcurveto{\pgfqpoint{11.004728in}{2.905832in}}{\pgfqpoint{11.001069in}{2.897000in}}{\pgfqpoint{11.001069in}{2.887791in}}%
\pgfpathcurveto{\pgfqpoint{11.001069in}{2.878583in}}{\pgfqpoint{11.004728in}{2.869750in}}{\pgfqpoint{11.011239in}{2.863239in}}%
\pgfpathcurveto{\pgfqpoint{11.017750in}{2.856728in}}{\pgfqpoint{11.026583in}{2.853069in}}{\pgfqpoint{11.035791in}{2.853069in}}%
\pgfpathlineto{\pgfqpoint{11.035791in}{2.853069in}}%
\pgfpathclose%
\pgfusepath{stroke,fill}%
\end{pgfscope}%
\begin{pgfscope}%
\pgfpathrectangle{\pgfqpoint{1.374500in}{0.082500in}}{\pgfqpoint{2.419000in}{2.419000in}}%
\pgfusepath{clip}%
\pgfsetbuttcap%
\pgfsetroundjoin%
\definecolor{currentfill}{rgb}{1.000000,0.662745,0.054902}%
\pgfsetfillcolor{currentfill}%
\pgfsetfillopacity{0.369179}%
\pgfsetlinewidth{1.003750pt}%
\definecolor{currentstroke}{rgb}{1.000000,0.662745,0.054902}%
\pgfsetstrokecolor{currentstroke}%
\pgfsetstrokeopacity{0.369179}%
\pgfsetdash{}{0pt}%
\pgfpathmoveto{\pgfqpoint{5.766816in}{2.770900in}}%
\pgfpathcurveto{\pgfqpoint{5.776024in}{2.770900in}}{\pgfqpoint{5.784856in}{2.774558in}}{\pgfqpoint{5.791368in}{2.781070in}}%
\pgfpathcurveto{\pgfqpoint{5.797879in}{2.787581in}}{\pgfqpoint{5.801538in}{2.796414in}}{\pgfqpoint{5.801538in}{2.805622in}}%
\pgfpathcurveto{\pgfqpoint{5.801538in}{2.814830in}}{\pgfqpoint{5.797879in}{2.823663in}}{\pgfqpoint{5.791368in}{2.830174in}}%
\pgfpathcurveto{\pgfqpoint{5.784856in}{2.836686in}}{\pgfqpoint{5.776024in}{2.840344in}}{\pgfqpoint{5.766816in}{2.840344in}}%
\pgfpathcurveto{\pgfqpoint{5.757607in}{2.840344in}}{\pgfqpoint{5.748775in}{2.836686in}}{\pgfqpoint{5.742263in}{2.830174in}}%
\pgfpathcurveto{\pgfqpoint{5.735752in}{2.823663in}}{\pgfqpoint{5.732093in}{2.814830in}}{\pgfqpoint{5.732093in}{2.805622in}}%
\pgfpathcurveto{\pgfqpoint{5.732093in}{2.796414in}}{\pgfqpoint{5.735752in}{2.787581in}}{\pgfqpoint{5.742263in}{2.781070in}}%
\pgfpathcurveto{\pgfqpoint{5.748775in}{2.774558in}}{\pgfqpoint{5.757607in}{2.770900in}}{\pgfqpoint{5.766816in}{2.770900in}}%
\pgfpathlineto{\pgfqpoint{5.766816in}{2.770900in}}%
\pgfpathclose%
\pgfusepath{stroke,fill}%
\end{pgfscope}%
\begin{pgfscope}%
\pgfpathrectangle{\pgfqpoint{1.374500in}{0.082500in}}{\pgfqpoint{2.419000in}{2.419000in}}%
\pgfusepath{clip}%
\pgfsetbuttcap%
\pgfsetroundjoin%
\definecolor{currentfill}{rgb}{1.000000,0.662745,0.054902}%
\pgfsetfillcolor{currentfill}%
\pgfsetfillopacity{0.369179}%
\pgfsetlinewidth{1.003750pt}%
\definecolor{currentstroke}{rgb}{1.000000,0.662745,0.054902}%
\pgfsetstrokecolor{currentstroke}%
\pgfsetstrokeopacity{0.369179}%
\pgfsetdash{}{0pt}%
\pgfpathmoveto{\pgfqpoint{1.937808in}{2.770900in}}%
\pgfpathcurveto{\pgfqpoint{1.947017in}{2.770900in}}{\pgfqpoint{1.955849in}{2.774558in}}{\pgfqpoint{1.962361in}{2.781070in}}%
\pgfpathcurveto{\pgfqpoint{1.968872in}{2.787581in}}{\pgfqpoint{1.972531in}{2.796414in}}{\pgfqpoint{1.972531in}{2.805622in}}%
\pgfpathcurveto{\pgfqpoint{1.972531in}{2.814830in}}{\pgfqpoint{1.968872in}{2.823663in}}{\pgfqpoint{1.962361in}{2.830174in}}%
\pgfpathcurveto{\pgfqpoint{1.955849in}{2.836686in}}{\pgfqpoint{1.947017in}{2.840344in}}{\pgfqpoint{1.937808in}{2.840344in}}%
\pgfpathcurveto{\pgfqpoint{1.928600in}{2.840344in}}{\pgfqpoint{1.919767in}{2.836686in}}{\pgfqpoint{1.913256in}{2.830174in}}%
\pgfpathcurveto{\pgfqpoint{1.906745in}{2.823663in}}{\pgfqpoint{1.903086in}{2.814830in}}{\pgfqpoint{1.903086in}{2.805622in}}%
\pgfpathcurveto{\pgfqpoint{1.903086in}{2.796414in}}{\pgfqpoint{1.906745in}{2.787581in}}{\pgfqpoint{1.913256in}{2.781070in}}%
\pgfpathcurveto{\pgfqpoint{1.919767in}{2.774558in}}{\pgfqpoint{1.928600in}{2.770900in}}{\pgfqpoint{1.937808in}{2.770900in}}%
\pgfpathlineto{\pgfqpoint{1.937808in}{2.770900in}}%
\pgfpathclose%
\pgfusepath{stroke,fill}%
\end{pgfscope}%
\begin{pgfscope}%
\pgfpathrectangle{\pgfqpoint{1.374500in}{0.082500in}}{\pgfqpoint{2.419000in}{2.419000in}}%
\pgfusepath{clip}%
\pgfsetbuttcap%
\pgfsetroundjoin%
\definecolor{currentfill}{rgb}{1.000000,0.662745,0.054902}%
\pgfsetfillcolor{currentfill}%
\pgfsetfillopacity{0.369179}%
\pgfsetlinewidth{1.003750pt}%
\definecolor{currentstroke}{rgb}{1.000000,0.662745,0.054902}%
\pgfsetstrokecolor{currentstroke}%
\pgfsetstrokeopacity{0.369179}%
\pgfsetdash{}{0pt}%
\pgfpathmoveto{\pgfqpoint{9.595823in}{2.770900in}}%
\pgfpathcurveto{\pgfqpoint{9.605031in}{2.770900in}}{\pgfqpoint{9.613864in}{2.774558in}}{\pgfqpoint{9.620375in}{2.781070in}}%
\pgfpathcurveto{\pgfqpoint{9.626886in}{2.787581in}}{\pgfqpoint{9.630545in}{2.796414in}}{\pgfqpoint{9.630545in}{2.805622in}}%
\pgfpathcurveto{\pgfqpoint{9.630545in}{2.814830in}}{\pgfqpoint{9.626886in}{2.823663in}}{\pgfqpoint{9.620375in}{2.830174in}}%
\pgfpathcurveto{\pgfqpoint{9.613864in}{2.836686in}}{\pgfqpoint{9.605031in}{2.840344in}}{\pgfqpoint{9.595823in}{2.840344in}}%
\pgfpathcurveto{\pgfqpoint{9.586614in}{2.840344in}}{\pgfqpoint{9.577782in}{2.836686in}}{\pgfqpoint{9.571270in}{2.830174in}}%
\pgfpathcurveto{\pgfqpoint{9.564759in}{2.823663in}}{\pgfqpoint{9.561101in}{2.814830in}}{\pgfqpoint{9.561101in}{2.805622in}}%
\pgfpathcurveto{\pgfqpoint{9.561101in}{2.796414in}}{\pgfqpoint{9.564759in}{2.787581in}}{\pgfqpoint{9.571270in}{2.781070in}}%
\pgfpathcurveto{\pgfqpoint{9.577782in}{2.774558in}}{\pgfqpoint{9.586614in}{2.770900in}}{\pgfqpoint{9.595823in}{2.770900in}}%
\pgfpathlineto{\pgfqpoint{9.595823in}{2.770900in}}%
\pgfpathclose%
\pgfusepath{stroke,fill}%
\end{pgfscope}%
\begin{pgfscope}%
\pgfpathrectangle{\pgfqpoint{1.374500in}{0.082500in}}{\pgfqpoint{2.419000in}{2.419000in}}%
\pgfusepath{clip}%
\pgfsetbuttcap%
\pgfsetroundjoin%
\definecolor{currentfill}{rgb}{1.000000,0.662745,0.054902}%
\pgfsetfillcolor{currentfill}%
\pgfsetfillopacity{0.373192}%
\pgfsetlinewidth{1.003750pt}%
\definecolor{currentstroke}{rgb}{1.000000,0.662745,0.054902}%
\pgfsetstrokecolor{currentstroke}%
\pgfsetstrokeopacity{0.373192}%
\pgfsetdash{}{0pt}%
\pgfpathmoveto{\pgfqpoint{0.351176in}{2.686443in}}%
\pgfpathcurveto{\pgfqpoint{0.360385in}{2.686443in}}{\pgfqpoint{0.369217in}{2.690102in}}{\pgfqpoint{0.375729in}{2.696613in}}%
\pgfpathcurveto{\pgfqpoint{0.382240in}{2.703124in}}{\pgfqpoint{0.385899in}{2.711957in}}{\pgfqpoint{0.385899in}{2.721165in}}%
\pgfpathcurveto{\pgfqpoint{0.385899in}{2.730374in}}{\pgfqpoint{0.382240in}{2.739206in}}{\pgfqpoint{0.375729in}{2.745718in}}%
\pgfpathcurveto{\pgfqpoint{0.369217in}{2.752229in}}{\pgfqpoint{0.360385in}{2.755888in}}{\pgfqpoint{0.351176in}{2.755888in}}%
\pgfpathcurveto{\pgfqpoint{0.341968in}{2.755888in}}{\pgfqpoint{0.333135in}{2.752229in}}{\pgfqpoint{0.326624in}{2.745718in}}%
\pgfpathcurveto{\pgfqpoint{0.320113in}{2.739206in}}{\pgfqpoint{0.316454in}{2.730374in}}{\pgfqpoint{0.316454in}{2.721165in}}%
\pgfpathcurveto{\pgfqpoint{0.316454in}{2.711957in}}{\pgfqpoint{0.320113in}{2.703124in}}{\pgfqpoint{0.326624in}{2.696613in}}%
\pgfpathcurveto{\pgfqpoint{0.333135in}{2.690102in}}{\pgfqpoint{0.341968in}{2.686443in}}{\pgfqpoint{0.351176in}{2.686443in}}%
\pgfpathlineto{\pgfqpoint{0.351176in}{2.686443in}}%
\pgfpathclose%
\pgfusepath{stroke,fill}%
\end{pgfscope}%
\begin{pgfscope}%
\pgfpathrectangle{\pgfqpoint{1.374500in}{0.082500in}}{\pgfqpoint{2.419000in}{2.419000in}}%
\pgfusepath{clip}%
\pgfsetbuttcap%
\pgfsetroundjoin%
\definecolor{currentfill}{rgb}{1.000000,0.662745,0.054902}%
\pgfsetfillcolor{currentfill}%
\pgfsetfillopacity{0.373192}%
\pgfsetlinewidth{1.003750pt}%
\definecolor{currentstroke}{rgb}{1.000000,0.662745,0.054902}%
\pgfsetstrokecolor{currentstroke}%
\pgfsetstrokeopacity{0.373192}%
\pgfsetdash{}{0pt}%
\pgfpathmoveto{\pgfqpoint{4.233474in}{2.686443in}}%
\pgfpathcurveto{\pgfqpoint{4.242683in}{2.686443in}}{\pgfqpoint{4.251515in}{2.690102in}}{\pgfqpoint{4.258027in}{2.696613in}}%
\pgfpathcurveto{\pgfqpoint{4.264538in}{2.703124in}}{\pgfqpoint{4.268197in}{2.711957in}}{\pgfqpoint{4.268197in}{2.721165in}}%
\pgfpathcurveto{\pgfqpoint{4.268197in}{2.730374in}}{\pgfqpoint{4.264538in}{2.739206in}}{\pgfqpoint{4.258027in}{2.745718in}}%
\pgfpathcurveto{\pgfqpoint{4.251515in}{2.752229in}}{\pgfqpoint{4.242683in}{2.755888in}}{\pgfqpoint{4.233474in}{2.755888in}}%
\pgfpathcurveto{\pgfqpoint{4.224266in}{2.755888in}}{\pgfqpoint{4.215433in}{2.752229in}}{\pgfqpoint{4.208922in}{2.745718in}}%
\pgfpathcurveto{\pgfqpoint{4.202411in}{2.739206in}}{\pgfqpoint{4.198752in}{2.730374in}}{\pgfqpoint{4.198752in}{2.721165in}}%
\pgfpathcurveto{\pgfqpoint{4.198752in}{2.711957in}}{\pgfqpoint{4.202411in}{2.703124in}}{\pgfqpoint{4.208922in}{2.696613in}}%
\pgfpathcurveto{\pgfqpoint{4.215433in}{2.690102in}}{\pgfqpoint{4.224266in}{2.686443in}}{\pgfqpoint{4.233474in}{2.686443in}}%
\pgfpathlineto{\pgfqpoint{4.233474in}{2.686443in}}%
\pgfpathclose%
\pgfusepath{stroke,fill}%
\end{pgfscope}%
\begin{pgfscope}%
\pgfpathrectangle{\pgfqpoint{1.374500in}{0.082500in}}{\pgfqpoint{2.419000in}{2.419000in}}%
\pgfusepath{clip}%
\pgfsetbuttcap%
\pgfsetroundjoin%
\definecolor{currentfill}{rgb}{1.000000,0.662745,0.054902}%
\pgfsetfillcolor{currentfill}%
\pgfsetfillopacity{0.373192}%
\pgfsetlinewidth{1.003750pt}%
\definecolor{currentstroke}{rgb}{1.000000,0.662745,0.054902}%
\pgfsetstrokecolor{currentstroke}%
\pgfsetstrokeopacity{0.373192}%
\pgfsetdash{}{0pt}%
\pgfpathmoveto{\pgfqpoint{8.115772in}{2.686443in}}%
\pgfpathcurveto{\pgfqpoint{8.124981in}{2.686443in}}{\pgfqpoint{8.133813in}{2.690102in}}{\pgfqpoint{8.140325in}{2.696613in}}%
\pgfpathcurveto{\pgfqpoint{8.146836in}{2.703124in}}{\pgfqpoint{8.150495in}{2.711957in}}{\pgfqpoint{8.150495in}{2.721165in}}%
\pgfpathcurveto{\pgfqpoint{8.150495in}{2.730374in}}{\pgfqpoint{8.146836in}{2.739206in}}{\pgfqpoint{8.140325in}{2.745718in}}%
\pgfpathcurveto{\pgfqpoint{8.133813in}{2.752229in}}{\pgfqpoint{8.124981in}{2.755888in}}{\pgfqpoint{8.115772in}{2.755888in}}%
\pgfpathcurveto{\pgfqpoint{8.106564in}{2.755888in}}{\pgfqpoint{8.097731in}{2.752229in}}{\pgfqpoint{8.091220in}{2.745718in}}%
\pgfpathcurveto{\pgfqpoint{8.084709in}{2.739206in}}{\pgfqpoint{8.081050in}{2.730374in}}{\pgfqpoint{8.081050in}{2.721165in}}%
\pgfpathcurveto{\pgfqpoint{8.081050in}{2.711957in}}{\pgfqpoint{8.084709in}{2.703124in}}{\pgfqpoint{8.091220in}{2.696613in}}%
\pgfpathcurveto{\pgfqpoint{8.097731in}{2.690102in}}{\pgfqpoint{8.106564in}{2.686443in}}{\pgfqpoint{8.115772in}{2.686443in}}%
\pgfpathlineto{\pgfqpoint{8.115772in}{2.686443in}}%
\pgfpathclose%
\pgfusepath{stroke,fill}%
\end{pgfscope}%
\begin{pgfscope}%
\pgfpathrectangle{\pgfqpoint{1.374500in}{0.082500in}}{\pgfqpoint{2.419000in}{2.419000in}}%
\pgfusepath{clip}%
\pgfsetbuttcap%
\pgfsetroundjoin%
\definecolor{currentfill}{rgb}{1.000000,0.662745,0.054902}%
\pgfsetfillcolor{currentfill}%
\pgfsetfillopacity{0.377317}%
\pgfsetlinewidth{1.003750pt}%
\definecolor{currentstroke}{rgb}{1.000000,0.662745,0.054902}%
\pgfsetstrokecolor{currentstroke}%
\pgfsetstrokeopacity{0.377317}%
\pgfsetdash{}{0pt}%
\pgfpathmoveto{\pgfqpoint{2.656850in}{2.599602in}}%
\pgfpathcurveto{\pgfqpoint{2.666058in}{2.599602in}}{\pgfqpoint{2.674891in}{2.603261in}}{\pgfqpoint{2.681402in}{2.609772in}}%
\pgfpathcurveto{\pgfqpoint{2.687913in}{2.616284in}}{\pgfqpoint{2.691572in}{2.625116in}}{\pgfqpoint{2.691572in}{2.634325in}}%
\pgfpathcurveto{\pgfqpoint{2.691572in}{2.643533in}}{\pgfqpoint{2.687913in}{2.652366in}}{\pgfqpoint{2.681402in}{2.658877in}}%
\pgfpathcurveto{\pgfqpoint{2.674891in}{2.665388in}}{\pgfqpoint{2.666058in}{2.669047in}}{\pgfqpoint{2.656850in}{2.669047in}}%
\pgfpathcurveto{\pgfqpoint{2.647641in}{2.669047in}}{\pgfqpoint{2.638809in}{2.665388in}}{\pgfqpoint{2.632297in}{2.658877in}}%
\pgfpathcurveto{\pgfqpoint{2.625786in}{2.652366in}}{\pgfqpoint{2.622127in}{2.643533in}}{\pgfqpoint{2.622127in}{2.634325in}}%
\pgfpathcurveto{\pgfqpoint{2.622127in}{2.625116in}}{\pgfqpoint{2.625786in}{2.616284in}}{\pgfqpoint{2.632297in}{2.609772in}}%
\pgfpathcurveto{\pgfqpoint{2.638809in}{2.603261in}}{\pgfqpoint{2.647641in}{2.599602in}}{\pgfqpoint{2.656850in}{2.599602in}}%
\pgfpathlineto{\pgfqpoint{2.656850in}{2.599602in}}%
\pgfpathclose%
\pgfusepath{stroke,fill}%
\end{pgfscope}%
\begin{pgfscope}%
\pgfpathrectangle{\pgfqpoint{1.374500in}{0.082500in}}{\pgfqpoint{2.419000in}{2.419000in}}%
\pgfusepath{clip}%
\pgfsetbuttcap%
\pgfsetroundjoin%
\definecolor{currentfill}{rgb}{1.000000,0.662745,0.054902}%
\pgfsetfillcolor{currentfill}%
\pgfsetfillopacity{0.377317}%
\pgfsetlinewidth{1.003750pt}%
\definecolor{currentstroke}{rgb}{1.000000,0.662745,0.054902}%
\pgfsetstrokecolor{currentstroke}%
\pgfsetstrokeopacity{0.377317}%
\pgfsetdash{}{0pt}%
\pgfpathmoveto{\pgfqpoint{6.593943in}{2.599602in}}%
\pgfpathcurveto{\pgfqpoint{6.603151in}{2.599602in}}{\pgfqpoint{6.611984in}{2.603261in}}{\pgfqpoint{6.618495in}{2.609772in}}%
\pgfpathcurveto{\pgfqpoint{6.625006in}{2.616284in}}{\pgfqpoint{6.628665in}{2.625116in}}{\pgfqpoint{6.628665in}{2.634325in}}%
\pgfpathcurveto{\pgfqpoint{6.628665in}{2.643533in}}{\pgfqpoint{6.625006in}{2.652366in}}{\pgfqpoint{6.618495in}{2.658877in}}%
\pgfpathcurveto{\pgfqpoint{6.611984in}{2.665388in}}{\pgfqpoint{6.603151in}{2.669047in}}{\pgfqpoint{6.593943in}{2.669047in}}%
\pgfpathcurveto{\pgfqpoint{6.584734in}{2.669047in}}{\pgfqpoint{6.575902in}{2.665388in}}{\pgfqpoint{6.569390in}{2.658877in}}%
\pgfpathcurveto{\pgfqpoint{6.562879in}{2.652366in}}{\pgfqpoint{6.559221in}{2.643533in}}{\pgfqpoint{6.559221in}{2.634325in}}%
\pgfpathcurveto{\pgfqpoint{6.559221in}{2.625116in}}{\pgfqpoint{6.562879in}{2.616284in}}{\pgfqpoint{6.569390in}{2.609772in}}%
\pgfpathcurveto{\pgfqpoint{6.575902in}{2.603261in}}{\pgfqpoint{6.584734in}{2.599602in}}{\pgfqpoint{6.593943in}{2.599602in}}%
\pgfpathlineto{\pgfqpoint{6.593943in}{2.599602in}}%
\pgfpathclose%
\pgfusepath{stroke,fill}%
\end{pgfscope}%
\begin{pgfscope}%
\pgfpathrectangle{\pgfqpoint{1.374500in}{0.082500in}}{\pgfqpoint{2.419000in}{2.419000in}}%
\pgfusepath{clip}%
\pgfsetbuttcap%
\pgfsetroundjoin%
\definecolor{currentfill}{rgb}{1.000000,0.662745,0.054902}%
\pgfsetfillcolor{currentfill}%
\pgfsetfillopacity{0.377317}%
\pgfsetlinewidth{1.003750pt}%
\definecolor{currentstroke}{rgb}{1.000000,0.662745,0.054902}%
\pgfsetstrokecolor{currentstroke}%
\pgfsetstrokeopacity{0.377317}%
\pgfsetdash{}{0pt}%
\pgfpathmoveto{\pgfqpoint{10.531036in}{2.599602in}}%
\pgfpathcurveto{\pgfqpoint{10.540244in}{2.599602in}}{\pgfqpoint{10.549077in}{2.603261in}}{\pgfqpoint{10.555588in}{2.609772in}}%
\pgfpathcurveto{\pgfqpoint{10.562100in}{2.616284in}}{\pgfqpoint{10.565758in}{2.625116in}}{\pgfqpoint{10.565758in}{2.634325in}}%
\pgfpathcurveto{\pgfqpoint{10.565758in}{2.643533in}}{\pgfqpoint{10.562100in}{2.652366in}}{\pgfqpoint{10.555588in}{2.658877in}}%
\pgfpathcurveto{\pgfqpoint{10.549077in}{2.665388in}}{\pgfqpoint{10.540244in}{2.669047in}}{\pgfqpoint{10.531036in}{2.669047in}}%
\pgfpathcurveto{\pgfqpoint{10.521827in}{2.669047in}}{\pgfqpoint{10.512995in}{2.665388in}}{\pgfqpoint{10.506484in}{2.658877in}}%
\pgfpathcurveto{\pgfqpoint{10.499972in}{2.652366in}}{\pgfqpoint{10.496314in}{2.643533in}}{\pgfqpoint{10.496314in}{2.634325in}}%
\pgfpathcurveto{\pgfqpoint{10.496314in}{2.625116in}}{\pgfqpoint{10.499972in}{2.616284in}}{\pgfqpoint{10.506484in}{2.609772in}}%
\pgfpathcurveto{\pgfqpoint{10.512995in}{2.603261in}}{\pgfqpoint{10.521827in}{2.599602in}}{\pgfqpoint{10.531036in}{2.599602in}}%
\pgfpathlineto{\pgfqpoint{10.531036in}{2.599602in}}%
\pgfpathclose%
\pgfusepath{stroke,fill}%
\end{pgfscope}%
\begin{pgfscope}%
\pgfpathrectangle{\pgfqpoint{1.374500in}{0.082500in}}{\pgfqpoint{2.419000in}{2.419000in}}%
\pgfusepath{clip}%
\pgfsetbuttcap%
\pgfsetroundjoin%
\definecolor{currentfill}{rgb}{1.000000,0.662745,0.054902}%
\pgfsetfillcolor{currentfill}%
\pgfsetfillopacity{0.381561}%
\pgfsetlinewidth{1.003750pt}%
\definecolor{currentstroke}{rgb}{1.000000,0.662745,0.054902}%
\pgfsetstrokecolor{currentstroke}%
\pgfsetstrokeopacity{0.381561}%
\pgfsetdash{}{0pt}%
\pgfpathmoveto{\pgfqpoint{5.028540in}{2.510275in}}%
\pgfpathcurveto{\pgfqpoint{5.037748in}{2.510275in}}{\pgfqpoint{5.046581in}{2.513934in}}{\pgfqpoint{5.053092in}{2.520445in}}%
\pgfpathcurveto{\pgfqpoint{5.059603in}{2.526956in}}{\pgfqpoint{5.063262in}{2.535789in}}{\pgfqpoint{5.063262in}{2.544997in}}%
\pgfpathcurveto{\pgfqpoint{5.063262in}{2.554206in}}{\pgfqpoint{5.059603in}{2.563038in}}{\pgfqpoint{5.053092in}{2.569550in}}%
\pgfpathcurveto{\pgfqpoint{5.046581in}{2.576061in}}{\pgfqpoint{5.037748in}{2.579720in}}{\pgfqpoint{5.028540in}{2.579720in}}%
\pgfpathcurveto{\pgfqpoint{5.019331in}{2.579720in}}{\pgfqpoint{5.010499in}{2.576061in}}{\pgfqpoint{5.003987in}{2.569550in}}%
\pgfpathcurveto{\pgfqpoint{4.997476in}{2.563038in}}{\pgfqpoint{4.993818in}{2.554206in}}{\pgfqpoint{4.993818in}{2.544997in}}%
\pgfpathcurveto{\pgfqpoint{4.993818in}{2.535789in}}{\pgfqpoint{4.997476in}{2.526956in}}{\pgfqpoint{5.003987in}{2.520445in}}%
\pgfpathcurveto{\pgfqpoint{5.010499in}{2.513934in}}{\pgfqpoint{5.019331in}{2.510275in}}{\pgfqpoint{5.028540in}{2.510275in}}%
\pgfpathlineto{\pgfqpoint{5.028540in}{2.510275in}}%
\pgfpathclose%
\pgfusepath{stroke,fill}%
\end{pgfscope}%
\begin{pgfscope}%
\pgfpathrectangle{\pgfqpoint{1.374500in}{0.082500in}}{\pgfqpoint{2.419000in}{2.419000in}}%
\pgfusepath{clip}%
\pgfsetbuttcap%
\pgfsetroundjoin%
\definecolor{currentfill}{rgb}{1.000000,0.662745,0.054902}%
\pgfsetfillcolor{currentfill}%
\pgfsetfillopacity{0.381561}%
\pgfsetlinewidth{1.003750pt}%
\definecolor{currentstroke}{rgb}{1.000000,0.662745,0.054902}%
\pgfsetstrokecolor{currentstroke}%
\pgfsetstrokeopacity{0.381561}%
\pgfsetdash{}{0pt}%
\pgfpathmoveto{\pgfqpoint{1.035083in}{2.510275in}}%
\pgfpathcurveto{\pgfqpoint{1.044291in}{2.510275in}}{\pgfqpoint{1.053124in}{2.513934in}}{\pgfqpoint{1.059635in}{2.520445in}}%
\pgfpathcurveto{\pgfqpoint{1.066146in}{2.526956in}}{\pgfqpoint{1.069805in}{2.535789in}}{\pgfqpoint{1.069805in}{2.544997in}}%
\pgfpathcurveto{\pgfqpoint{1.069805in}{2.554206in}}{\pgfqpoint{1.066146in}{2.563038in}}{\pgfqpoint{1.059635in}{2.569550in}}%
\pgfpathcurveto{\pgfqpoint{1.053124in}{2.576061in}}{\pgfqpoint{1.044291in}{2.579720in}}{\pgfqpoint{1.035083in}{2.579720in}}%
\pgfpathcurveto{\pgfqpoint{1.025874in}{2.579720in}}{\pgfqpoint{1.017042in}{2.576061in}}{\pgfqpoint{1.010530in}{2.569550in}}%
\pgfpathcurveto{\pgfqpoint{1.004019in}{2.563038in}}{\pgfqpoint{1.000360in}{2.554206in}}{\pgfqpoint{1.000360in}{2.544997in}}%
\pgfpathcurveto{\pgfqpoint{1.000360in}{2.535789in}}{\pgfqpoint{1.004019in}{2.526956in}}{\pgfqpoint{1.010530in}{2.520445in}}%
\pgfpathcurveto{\pgfqpoint{1.017042in}{2.513934in}}{\pgfqpoint{1.025874in}{2.510275in}}{\pgfqpoint{1.035083in}{2.510275in}}%
\pgfpathlineto{\pgfqpoint{1.035083in}{2.510275in}}%
\pgfpathclose%
\pgfusepath{stroke,fill}%
\end{pgfscope}%
\begin{pgfscope}%
\pgfpathrectangle{\pgfqpoint{1.374500in}{0.082500in}}{\pgfqpoint{2.419000in}{2.419000in}}%
\pgfusepath{clip}%
\pgfsetbuttcap%
\pgfsetroundjoin%
\definecolor{currentfill}{rgb}{1.000000,0.662745,0.054902}%
\pgfsetfillcolor{currentfill}%
\pgfsetfillopacity{0.381561}%
\pgfsetlinewidth{1.003750pt}%
\definecolor{currentstroke}{rgb}{1.000000,0.662745,0.054902}%
\pgfsetstrokecolor{currentstroke}%
\pgfsetstrokeopacity{0.381561}%
\pgfsetdash{}{0pt}%
\pgfpathmoveto{\pgfqpoint{9.021997in}{2.510275in}}%
\pgfpathcurveto{\pgfqpoint{9.031205in}{2.510275in}}{\pgfqpoint{9.040038in}{2.513934in}}{\pgfqpoint{9.046549in}{2.520445in}}%
\pgfpathcurveto{\pgfqpoint{9.053061in}{2.526956in}}{\pgfqpoint{9.056719in}{2.535789in}}{\pgfqpoint{9.056719in}{2.544997in}}%
\pgfpathcurveto{\pgfqpoint{9.056719in}{2.554206in}}{\pgfqpoint{9.053061in}{2.563038in}}{\pgfqpoint{9.046549in}{2.569550in}}%
\pgfpathcurveto{\pgfqpoint{9.040038in}{2.576061in}}{\pgfqpoint{9.031205in}{2.579720in}}{\pgfqpoint{9.021997in}{2.579720in}}%
\pgfpathcurveto{\pgfqpoint{9.012788in}{2.579720in}}{\pgfqpoint{9.003956in}{2.576061in}}{\pgfqpoint{8.997445in}{2.569550in}}%
\pgfpathcurveto{\pgfqpoint{8.990933in}{2.563038in}}{\pgfqpoint{8.987275in}{2.554206in}}{\pgfqpoint{8.987275in}{2.544997in}}%
\pgfpathcurveto{\pgfqpoint{8.987275in}{2.535789in}}{\pgfqpoint{8.990933in}{2.526956in}}{\pgfqpoint{8.997445in}{2.520445in}}%
\pgfpathcurveto{\pgfqpoint{9.003956in}{2.513934in}}{\pgfqpoint{9.012788in}{2.510275in}}{\pgfqpoint{9.021997in}{2.510275in}}%
\pgfpathlineto{\pgfqpoint{9.021997in}{2.510275in}}%
\pgfpathclose%
\pgfusepath{stroke,fill}%
\end{pgfscope}%
\begin{pgfscope}%
\pgfpathrectangle{\pgfqpoint{1.374500in}{0.082500in}}{\pgfqpoint{2.419000in}{2.419000in}}%
\pgfusepath{clip}%
\pgfsetbuttcap%
\pgfsetroundjoin%
\definecolor{currentfill}{rgb}{1.000000,0.662745,0.054902}%
\pgfsetfillcolor{currentfill}%
\pgfsetfillopacity{0.385928}%
\pgfsetlinewidth{1.003750pt}%
\definecolor{currentstroke}{rgb}{1.000000,0.662745,0.054902}%
\pgfsetstrokecolor{currentstroke}%
\pgfsetstrokeopacity{0.385928}%
\pgfsetdash{}{0pt}%
\pgfpathmoveto{\pgfqpoint{7.469123in}{2.418353in}}%
\pgfpathcurveto{\pgfqpoint{7.478332in}{2.418353in}}{\pgfqpoint{7.487164in}{2.422012in}}{\pgfqpoint{7.493675in}{2.428523in}}%
\pgfpathcurveto{\pgfqpoint{7.500187in}{2.435035in}}{\pgfqpoint{7.503845in}{2.443867in}}{\pgfqpoint{7.503845in}{2.453075in}}%
\pgfpathcurveto{\pgfqpoint{7.503845in}{2.462284in}}{\pgfqpoint{7.500187in}{2.471116in}}{\pgfqpoint{7.493675in}{2.477628in}}%
\pgfpathcurveto{\pgfqpoint{7.487164in}{2.484139in}}{\pgfqpoint{7.478332in}{2.487798in}}{\pgfqpoint{7.469123in}{2.487798in}}%
\pgfpathcurveto{\pgfqpoint{7.459915in}{2.487798in}}{\pgfqpoint{7.451082in}{2.484139in}}{\pgfqpoint{7.444571in}{2.477628in}}%
\pgfpathcurveto{\pgfqpoint{7.438059in}{2.471116in}}{\pgfqpoint{7.434401in}{2.462284in}}{\pgfqpoint{7.434401in}{2.453075in}}%
\pgfpathcurveto{\pgfqpoint{7.434401in}{2.443867in}}{\pgfqpoint{7.438059in}{2.435035in}}{\pgfqpoint{7.444571in}{2.428523in}}%
\pgfpathcurveto{\pgfqpoint{7.451082in}{2.422012in}}{\pgfqpoint{7.459915in}{2.418353in}}{\pgfqpoint{7.469123in}{2.418353in}}%
\pgfpathlineto{\pgfqpoint{7.469123in}{2.418353in}}%
\pgfpathclose%
\pgfusepath{stroke,fill}%
\end{pgfscope}%
\begin{pgfscope}%
\pgfpathrectangle{\pgfqpoint{1.374500in}{0.082500in}}{\pgfqpoint{2.419000in}{2.419000in}}%
\pgfusepath{clip}%
\pgfsetbuttcap%
\pgfsetroundjoin%
\definecolor{currentfill}{rgb}{1.000000,0.662745,0.054902}%
\pgfsetfillcolor{currentfill}%
\pgfsetfillopacity{0.385928}%
\pgfsetlinewidth{1.003750pt}%
\definecolor{currentstroke}{rgb}{1.000000,0.662745,0.054902}%
\pgfsetstrokecolor{currentstroke}%
\pgfsetstrokeopacity{0.385928}%
\pgfsetdash{}{0pt}%
\pgfpathmoveto{\pgfqpoint{3.417665in}{2.418353in}}%
\pgfpathcurveto{\pgfqpoint{3.426873in}{2.418353in}}{\pgfqpoint{3.435706in}{2.422012in}}{\pgfqpoint{3.442217in}{2.428523in}}%
\pgfpathcurveto{\pgfqpoint{3.448728in}{2.435035in}}{\pgfqpoint{3.452387in}{2.443867in}}{\pgfqpoint{3.452387in}{2.453075in}}%
\pgfpathcurveto{\pgfqpoint{3.452387in}{2.462284in}}{\pgfqpoint{3.448728in}{2.471116in}}{\pgfqpoint{3.442217in}{2.477628in}}%
\pgfpathcurveto{\pgfqpoint{3.435706in}{2.484139in}}{\pgfqpoint{3.426873in}{2.487798in}}{\pgfqpoint{3.417665in}{2.487798in}}%
\pgfpathcurveto{\pgfqpoint{3.408456in}{2.487798in}}{\pgfqpoint{3.399624in}{2.484139in}}{\pgfqpoint{3.393112in}{2.477628in}}%
\pgfpathcurveto{\pgfqpoint{3.386601in}{2.471116in}}{\pgfqpoint{3.382943in}{2.462284in}}{\pgfqpoint{3.382943in}{2.453075in}}%
\pgfpathcurveto{\pgfqpoint{3.382943in}{2.443867in}}{\pgfqpoint{3.386601in}{2.435035in}}{\pgfqpoint{3.393112in}{2.428523in}}%
\pgfpathcurveto{\pgfqpoint{3.399624in}{2.422012in}}{\pgfqpoint{3.408456in}{2.418353in}}{\pgfqpoint{3.417665in}{2.418353in}}%
\pgfpathlineto{\pgfqpoint{3.417665in}{2.418353in}}%
\pgfpathclose%
\pgfusepath{stroke,fill}%
\end{pgfscope}%
\begin{pgfscope}%
\pgfpathrectangle{\pgfqpoint{1.374500in}{0.082500in}}{\pgfqpoint{2.419000in}{2.419000in}}%
\pgfusepath{clip}%
\pgfsetbuttcap%
\pgfsetroundjoin%
\definecolor{currentfill}{rgb}{1.000000,0.662745,0.054902}%
\pgfsetfillcolor{currentfill}%
\pgfsetfillopacity{0.385928}%
\pgfsetlinewidth{1.003750pt}%
\definecolor{currentstroke}{rgb}{1.000000,0.662745,0.054902}%
\pgfsetstrokecolor{currentstroke}%
\pgfsetstrokeopacity{0.385928}%
\pgfsetdash{}{0pt}%
\pgfpathmoveto{\pgfqpoint{11.520582in}{2.418353in}}%
\pgfpathcurveto{\pgfqpoint{11.529790in}{2.418353in}}{\pgfqpoint{11.538622in}{2.422012in}}{\pgfqpoint{11.545134in}{2.428523in}}%
\pgfpathcurveto{\pgfqpoint{11.551645in}{2.435035in}}{\pgfqpoint{11.555304in}{2.443867in}}{\pgfqpoint{11.555304in}{2.453075in}}%
\pgfpathcurveto{\pgfqpoint{11.555304in}{2.462284in}}{\pgfqpoint{11.551645in}{2.471116in}}{\pgfqpoint{11.545134in}{2.477628in}}%
\pgfpathcurveto{\pgfqpoint{11.538622in}{2.484139in}}{\pgfqpoint{11.529790in}{2.487798in}}{\pgfqpoint{11.520582in}{2.487798in}}%
\pgfpathcurveto{\pgfqpoint{11.511373in}{2.487798in}}{\pgfqpoint{11.502541in}{2.484139in}}{\pgfqpoint{11.496029in}{2.477628in}}%
\pgfpathcurveto{\pgfqpoint{11.489518in}{2.471116in}}{\pgfqpoint{11.485859in}{2.462284in}}{\pgfqpoint{11.485859in}{2.453075in}}%
\pgfpathcurveto{\pgfqpoint{11.485859in}{2.443867in}}{\pgfqpoint{11.489518in}{2.435035in}}{\pgfqpoint{11.496029in}{2.428523in}}%
\pgfpathcurveto{\pgfqpoint{11.502541in}{2.422012in}}{\pgfqpoint{11.511373in}{2.418353in}}{\pgfqpoint{11.520582in}{2.418353in}}%
\pgfpathlineto{\pgfqpoint{11.520582in}{2.418353in}}%
\pgfpathclose%
\pgfusepath{stroke,fill}%
\end{pgfscope}%
\begin{pgfscope}%
\pgfpathrectangle{\pgfqpoint{1.374500in}{0.082500in}}{\pgfqpoint{2.419000in}{2.419000in}}%
\pgfusepath{clip}%
\pgfsetbuttcap%
\pgfsetroundjoin%
\definecolor{currentfill}{rgb}{1.000000,0.662745,0.054902}%
\pgfsetfillcolor{currentfill}%
\pgfsetfillopacity{0.390424}%
\pgfsetlinewidth{1.003750pt}%
\definecolor{currentstroke}{rgb}{1.000000,0.662745,0.054902}%
\pgfsetstrokecolor{currentstroke}%
\pgfsetstrokeopacity{0.390424}%
\pgfsetdash{}{0pt}%
\pgfpathmoveto{\pgfqpoint{1.759307in}{2.323722in}}%
\pgfpathcurveto{\pgfqpoint{1.768516in}{2.323722in}}{\pgfqpoint{1.777348in}{2.327380in}}{\pgfqpoint{1.783859in}{2.333892in}}%
\pgfpathcurveto{\pgfqpoint{1.790371in}{2.340403in}}{\pgfqpoint{1.794029in}{2.349236in}}{\pgfqpoint{1.794029in}{2.358444in}}%
\pgfpathcurveto{\pgfqpoint{1.794029in}{2.367652in}}{\pgfqpoint{1.790371in}{2.376485in}}{\pgfqpoint{1.783859in}{2.382996in}}%
\pgfpathcurveto{\pgfqpoint{1.777348in}{2.389508in}}{\pgfqpoint{1.768516in}{2.393166in}}{\pgfqpoint{1.759307in}{2.393166in}}%
\pgfpathcurveto{\pgfqpoint{1.750099in}{2.393166in}}{\pgfqpoint{1.741266in}{2.389508in}}{\pgfqpoint{1.734755in}{2.382996in}}%
\pgfpathcurveto{\pgfqpoint{1.728243in}{2.376485in}}{\pgfqpoint{1.724585in}{2.367652in}}{\pgfqpoint{1.724585in}{2.358444in}}%
\pgfpathcurveto{\pgfqpoint{1.724585in}{2.349236in}}{\pgfqpoint{1.728243in}{2.340403in}}{\pgfqpoint{1.734755in}{2.333892in}}%
\pgfpathcurveto{\pgfqpoint{1.741266in}{2.327380in}}{\pgfqpoint{1.750099in}{2.323722in}}{\pgfqpoint{1.759307in}{2.323722in}}%
\pgfpathlineto{\pgfqpoint{1.759307in}{2.323722in}}%
\pgfpathclose%
\pgfusepath{stroke,fill}%
\end{pgfscope}%
\begin{pgfscope}%
\pgfpathrectangle{\pgfqpoint{1.374500in}{0.082500in}}{\pgfqpoint{2.419000in}{2.419000in}}%
\pgfusepath{clip}%
\pgfsetbuttcap%
\pgfsetroundjoin%
\definecolor{currentfill}{rgb}{1.000000,0.662745,0.054902}%
\pgfsetfillcolor{currentfill}%
\pgfsetfillopacity{0.390424}%
\pgfsetlinewidth{1.003750pt}%
\definecolor{currentstroke}{rgb}{1.000000,0.662745,0.054902}%
\pgfsetstrokecolor{currentstroke}%
\pgfsetstrokeopacity{0.390424}%
\pgfsetdash{}{0pt}%
\pgfpathmoveto{\pgfqpoint{5.870476in}{2.323722in}}%
\pgfpathcurveto{\pgfqpoint{5.879685in}{2.323722in}}{\pgfqpoint{5.888517in}{2.327380in}}{\pgfqpoint{5.895029in}{2.333892in}}%
\pgfpathcurveto{\pgfqpoint{5.901540in}{2.340403in}}{\pgfqpoint{5.905199in}{2.349236in}}{\pgfqpoint{5.905199in}{2.358444in}}%
\pgfpathcurveto{\pgfqpoint{5.905199in}{2.367652in}}{\pgfqpoint{5.901540in}{2.376485in}}{\pgfqpoint{5.895029in}{2.382996in}}%
\pgfpathcurveto{\pgfqpoint{5.888517in}{2.389508in}}{\pgfqpoint{5.879685in}{2.393166in}}{\pgfqpoint{5.870476in}{2.393166in}}%
\pgfpathcurveto{\pgfqpoint{5.861268in}{2.393166in}}{\pgfqpoint{5.852435in}{2.389508in}}{\pgfqpoint{5.845924in}{2.382996in}}%
\pgfpathcurveto{\pgfqpoint{5.839413in}{2.376485in}}{\pgfqpoint{5.835754in}{2.367652in}}{\pgfqpoint{5.835754in}{2.358444in}}%
\pgfpathcurveto{\pgfqpoint{5.835754in}{2.349236in}}{\pgfqpoint{5.839413in}{2.340403in}}{\pgfqpoint{5.845924in}{2.333892in}}%
\pgfpathcurveto{\pgfqpoint{5.852435in}{2.327380in}}{\pgfqpoint{5.861268in}{2.323722in}}{\pgfqpoint{5.870476in}{2.323722in}}%
\pgfpathlineto{\pgfqpoint{5.870476in}{2.323722in}}%
\pgfpathclose%
\pgfusepath{stroke,fill}%
\end{pgfscope}%
\begin{pgfscope}%
\pgfpathrectangle{\pgfqpoint{1.374500in}{0.082500in}}{\pgfqpoint{2.419000in}{2.419000in}}%
\pgfusepath{clip}%
\pgfsetbuttcap%
\pgfsetroundjoin%
\definecolor{currentfill}{rgb}{1.000000,0.662745,0.054902}%
\pgfsetfillcolor{currentfill}%
\pgfsetfillopacity{0.390424}%
\pgfsetlinewidth{1.003750pt}%
\definecolor{currentstroke}{rgb}{1.000000,0.662745,0.054902}%
\pgfsetstrokecolor{currentstroke}%
\pgfsetstrokeopacity{0.390424}%
\pgfsetdash{}{0pt}%
\pgfpathmoveto{\pgfqpoint{9.981646in}{2.323722in}}%
\pgfpathcurveto{\pgfqpoint{9.990854in}{2.323722in}}{\pgfqpoint{9.999687in}{2.327380in}}{\pgfqpoint{10.006198in}{2.333892in}}%
\pgfpathcurveto{\pgfqpoint{10.012709in}{2.340403in}}{\pgfqpoint{10.016368in}{2.349236in}}{\pgfqpoint{10.016368in}{2.358444in}}%
\pgfpathcurveto{\pgfqpoint{10.016368in}{2.367652in}}{\pgfqpoint{10.012709in}{2.376485in}}{\pgfqpoint{10.006198in}{2.382996in}}%
\pgfpathcurveto{\pgfqpoint{9.999687in}{2.389508in}}{\pgfqpoint{9.990854in}{2.393166in}}{\pgfqpoint{9.981646in}{2.393166in}}%
\pgfpathcurveto{\pgfqpoint{9.972437in}{2.393166in}}{\pgfqpoint{9.963605in}{2.389508in}}{\pgfqpoint{9.957093in}{2.382996in}}%
\pgfpathcurveto{\pgfqpoint{9.950582in}{2.376485in}}{\pgfqpoint{9.946924in}{2.367652in}}{\pgfqpoint{9.946924in}{2.358444in}}%
\pgfpathcurveto{\pgfqpoint{9.946924in}{2.349236in}}{\pgfqpoint{9.950582in}{2.340403in}}{\pgfqpoint{9.957093in}{2.333892in}}%
\pgfpathcurveto{\pgfqpoint{9.963605in}{2.327380in}}{\pgfqpoint{9.972437in}{2.323722in}}{\pgfqpoint{9.981646in}{2.323722in}}%
\pgfpathlineto{\pgfqpoint{9.981646in}{2.323722in}}%
\pgfpathclose%
\pgfusepath{stroke,fill}%
\end{pgfscope}%
\begin{pgfscope}%
\pgfpathrectangle{\pgfqpoint{1.374500in}{0.082500in}}{\pgfqpoint{2.419000in}{2.419000in}}%
\pgfusepath{clip}%
\pgfsetbuttcap%
\pgfsetroundjoin%
\definecolor{currentfill}{rgb}{1.000000,0.662745,0.054902}%
\pgfsetfillcolor{currentfill}%
\pgfsetfillopacity{0.395055}%
\pgfsetlinewidth{1.003750pt}%
\definecolor{currentstroke}{rgb}{1.000000,0.662745,0.054902}%
\pgfsetstrokecolor{currentstroke}%
\pgfsetstrokeopacity{0.395055}%
\pgfsetdash{}{0pt}%
\pgfpathmoveto{\pgfqpoint{4.224003in}{2.226259in}}%
\pgfpathcurveto{\pgfqpoint{4.233211in}{2.226259in}}{\pgfqpoint{4.242044in}{2.229918in}}{\pgfqpoint{4.248555in}{2.236429in}}%
\pgfpathcurveto{\pgfqpoint{4.255066in}{2.242940in}}{\pgfqpoint{4.258725in}{2.251773in}}{\pgfqpoint{4.258725in}{2.260981in}}%
\pgfpathcurveto{\pgfqpoint{4.258725in}{2.270190in}}{\pgfqpoint{4.255066in}{2.279022in}}{\pgfqpoint{4.248555in}{2.285534in}}%
\pgfpathcurveto{\pgfqpoint{4.242044in}{2.292045in}}{\pgfqpoint{4.233211in}{2.295704in}}{\pgfqpoint{4.224003in}{2.295704in}}%
\pgfpathcurveto{\pgfqpoint{4.214794in}{2.295704in}}{\pgfqpoint{4.205962in}{2.292045in}}{\pgfqpoint{4.199450in}{2.285534in}}%
\pgfpathcurveto{\pgfqpoint{4.192939in}{2.279022in}}{\pgfqpoint{4.189280in}{2.270190in}}{\pgfqpoint{4.189280in}{2.260981in}}%
\pgfpathcurveto{\pgfqpoint{4.189280in}{2.251773in}}{\pgfqpoint{4.192939in}{2.242940in}}{\pgfqpoint{4.199450in}{2.236429in}}%
\pgfpathcurveto{\pgfqpoint{4.205962in}{2.229918in}}{\pgfqpoint{4.214794in}{2.226259in}}{\pgfqpoint{4.224003in}{2.226259in}}%
\pgfpathlineto{\pgfqpoint{4.224003in}{2.226259in}}%
\pgfpathclose%
\pgfusepath{stroke,fill}%
\end{pgfscope}%
\begin{pgfscope}%
\pgfpathrectangle{\pgfqpoint{1.374500in}{0.082500in}}{\pgfqpoint{2.419000in}{2.419000in}}%
\pgfusepath{clip}%
\pgfsetbuttcap%
\pgfsetroundjoin%
\definecolor{currentfill}{rgb}{1.000000,0.662745,0.054902}%
\pgfsetfillcolor{currentfill}%
\pgfsetfillopacity{0.395055}%
\pgfsetlinewidth{1.003750pt}%
\definecolor{currentstroke}{rgb}{1.000000,0.662745,0.054902}%
\pgfsetstrokecolor{currentstroke}%
\pgfsetstrokeopacity{0.395055}%
\pgfsetdash{}{0pt}%
\pgfpathmoveto{\pgfqpoint{0.051336in}{2.226259in}}%
\pgfpathcurveto{\pgfqpoint{0.060544in}{2.226259in}}{\pgfqpoint{0.069377in}{2.229918in}}{\pgfqpoint{0.075888in}{2.236429in}}%
\pgfpathcurveto{\pgfqpoint{0.082400in}{2.242940in}}{\pgfqpoint{0.086058in}{2.251773in}}{\pgfqpoint{0.086058in}{2.260981in}}%
\pgfpathcurveto{\pgfqpoint{0.086058in}{2.270190in}}{\pgfqpoint{0.082400in}{2.279022in}}{\pgfqpoint{0.075888in}{2.285534in}}%
\pgfpathcurveto{\pgfqpoint{0.069377in}{2.292045in}}{\pgfqpoint{0.060544in}{2.295704in}}{\pgfqpoint{0.051336in}{2.295704in}}%
\pgfpathcurveto{\pgfqpoint{0.042128in}{2.295704in}}{\pgfqpoint{0.033295in}{2.292045in}}{\pgfqpoint{0.026784in}{2.285534in}}%
\pgfpathcurveto{\pgfqpoint{0.020272in}{2.279022in}}{\pgfqpoint{0.016614in}{2.270190in}}{\pgfqpoint{0.016614in}{2.260981in}}%
\pgfpathcurveto{\pgfqpoint{0.016614in}{2.251773in}}{\pgfqpoint{0.020272in}{2.242940in}}{\pgfqpoint{0.026784in}{2.236429in}}%
\pgfpathcurveto{\pgfqpoint{0.033295in}{2.229918in}}{\pgfqpoint{0.042128in}{2.226259in}}{\pgfqpoint{0.051336in}{2.226259in}}%
\pgfpathlineto{\pgfqpoint{0.051336in}{2.226259in}}%
\pgfpathclose%
\pgfusepath{stroke,fill}%
\end{pgfscope}%
\begin{pgfscope}%
\pgfpathrectangle{\pgfqpoint{1.374500in}{0.082500in}}{\pgfqpoint{2.419000in}{2.419000in}}%
\pgfusepath{clip}%
\pgfsetbuttcap%
\pgfsetroundjoin%
\definecolor{currentfill}{rgb}{1.000000,0.662745,0.054902}%
\pgfsetfillcolor{currentfill}%
\pgfsetfillopacity{0.395055}%
\pgfsetlinewidth{1.003750pt}%
\definecolor{currentstroke}{rgb}{1.000000,0.662745,0.054902}%
\pgfsetstrokecolor{currentstroke}%
\pgfsetstrokeopacity{0.395055}%
\pgfsetdash{}{0pt}%
\pgfpathmoveto{\pgfqpoint{8.396669in}{2.226259in}}%
\pgfpathcurveto{\pgfqpoint{8.405878in}{2.226259in}}{\pgfqpoint{8.414710in}{2.229918in}}{\pgfqpoint{8.421222in}{2.236429in}}%
\pgfpathcurveto{\pgfqpoint{8.427733in}{2.242940in}}{\pgfqpoint{8.431392in}{2.251773in}}{\pgfqpoint{8.431392in}{2.260981in}}%
\pgfpathcurveto{\pgfqpoint{8.431392in}{2.270190in}}{\pgfqpoint{8.427733in}{2.279022in}}{\pgfqpoint{8.421222in}{2.285534in}}%
\pgfpathcurveto{\pgfqpoint{8.414710in}{2.292045in}}{\pgfqpoint{8.405878in}{2.295704in}}{\pgfqpoint{8.396669in}{2.295704in}}%
\pgfpathcurveto{\pgfqpoint{8.387461in}{2.295704in}}{\pgfqpoint{8.378628in}{2.292045in}}{\pgfqpoint{8.372117in}{2.285534in}}%
\pgfpathcurveto{\pgfqpoint{8.365606in}{2.279022in}}{\pgfqpoint{8.361947in}{2.270190in}}{\pgfqpoint{8.361947in}{2.260981in}}%
\pgfpathcurveto{\pgfqpoint{8.361947in}{2.251773in}}{\pgfqpoint{8.365606in}{2.242940in}}{\pgfqpoint{8.372117in}{2.236429in}}%
\pgfpathcurveto{\pgfqpoint{8.378628in}{2.229918in}}{\pgfqpoint{8.387461in}{2.226259in}}{\pgfqpoint{8.396669in}{2.226259in}}%
\pgfpathlineto{\pgfqpoint{8.396669in}{2.226259in}}%
\pgfpathclose%
\pgfusepath{stroke,fill}%
\end{pgfscope}%
\begin{pgfscope}%
\pgfpathrectangle{\pgfqpoint{1.374500in}{0.082500in}}{\pgfqpoint{2.419000in}{2.419000in}}%
\pgfusepath{clip}%
\pgfsetbuttcap%
\pgfsetroundjoin%
\definecolor{currentfill}{rgb}{1.000000,0.662745,0.054902}%
\pgfsetfillcolor{currentfill}%
\pgfsetfillopacity{0.399826}%
\pgfsetlinewidth{1.003750pt}%
\definecolor{currentstroke}{rgb}{1.000000,0.662745,0.054902}%
\pgfsetstrokecolor{currentstroke}%
\pgfsetstrokeopacity{0.399826}%
\pgfsetdash{}{0pt}%
\pgfpathmoveto{\pgfqpoint{2.527523in}{2.125837in}}%
\pgfpathcurveto{\pgfqpoint{2.536731in}{2.125837in}}{\pgfqpoint{2.545564in}{2.129495in}}{\pgfqpoint{2.552075in}{2.136006in}}%
\pgfpathcurveto{\pgfqpoint{2.558587in}{2.142518in}}{\pgfqpoint{2.562245in}{2.151350in}}{\pgfqpoint{2.562245in}{2.160559in}}%
\pgfpathcurveto{\pgfqpoint{2.562245in}{2.169767in}}{\pgfqpoint{2.558587in}{2.178600in}}{\pgfqpoint{2.552075in}{2.185111in}}%
\pgfpathcurveto{\pgfqpoint{2.545564in}{2.191622in}}{\pgfqpoint{2.536731in}{2.195281in}}{\pgfqpoint{2.527523in}{2.195281in}}%
\pgfpathcurveto{\pgfqpoint{2.518315in}{2.195281in}}{\pgfqpoint{2.509482in}{2.191622in}}{\pgfqpoint{2.502971in}{2.185111in}}%
\pgfpathcurveto{\pgfqpoint{2.496459in}{2.178600in}}{\pgfqpoint{2.492801in}{2.169767in}}{\pgfqpoint{2.492801in}{2.160559in}}%
\pgfpathcurveto{\pgfqpoint{2.492801in}{2.151350in}}{\pgfqpoint{2.496459in}{2.142518in}}{\pgfqpoint{2.502971in}{2.136006in}}%
\pgfpathcurveto{\pgfqpoint{2.509482in}{2.129495in}}{\pgfqpoint{2.518315in}{2.125837in}}{\pgfqpoint{2.527523in}{2.125837in}}%
\pgfpathlineto{\pgfqpoint{2.527523in}{2.125837in}}%
\pgfpathclose%
\pgfusepath{stroke,fill}%
\end{pgfscope}%
\begin{pgfscope}%
\pgfpathrectangle{\pgfqpoint{1.374500in}{0.082500in}}{\pgfqpoint{2.419000in}{2.419000in}}%
\pgfusepath{clip}%
\pgfsetbuttcap%
\pgfsetroundjoin%
\definecolor{currentfill}{rgb}{1.000000,0.662745,0.054902}%
\pgfsetfillcolor{currentfill}%
\pgfsetfillopacity{0.399826}%
\pgfsetlinewidth{1.003750pt}%
\definecolor{currentstroke}{rgb}{1.000000,0.662745,0.054902}%
\pgfsetstrokecolor{currentstroke}%
\pgfsetstrokeopacity{0.399826}%
\pgfsetdash{}{0pt}%
\pgfpathmoveto{\pgfqpoint{6.763555in}{2.125837in}}%
\pgfpathcurveto{\pgfqpoint{6.772763in}{2.125837in}}{\pgfqpoint{6.781596in}{2.129495in}}{\pgfqpoint{6.788107in}{2.136006in}}%
\pgfpathcurveto{\pgfqpoint{6.794618in}{2.142518in}}{\pgfqpoint{6.798277in}{2.151350in}}{\pgfqpoint{6.798277in}{2.160559in}}%
\pgfpathcurveto{\pgfqpoint{6.798277in}{2.169767in}}{\pgfqpoint{6.794618in}{2.178600in}}{\pgfqpoint{6.788107in}{2.185111in}}%
\pgfpathcurveto{\pgfqpoint{6.781596in}{2.191622in}}{\pgfqpoint{6.772763in}{2.195281in}}{\pgfqpoint{6.763555in}{2.195281in}}%
\pgfpathcurveto{\pgfqpoint{6.754346in}{2.195281in}}{\pgfqpoint{6.745514in}{2.191622in}}{\pgfqpoint{6.739002in}{2.185111in}}%
\pgfpathcurveto{\pgfqpoint{6.732491in}{2.178600in}}{\pgfqpoint{6.728833in}{2.169767in}}{\pgfqpoint{6.728833in}{2.160559in}}%
\pgfpathcurveto{\pgfqpoint{6.728833in}{2.151350in}}{\pgfqpoint{6.732491in}{2.142518in}}{\pgfqpoint{6.739002in}{2.136006in}}%
\pgfpathcurveto{\pgfqpoint{6.745514in}{2.129495in}}{\pgfqpoint{6.754346in}{2.125837in}}{\pgfqpoint{6.763555in}{2.125837in}}%
\pgfpathlineto{\pgfqpoint{6.763555in}{2.125837in}}%
\pgfpathclose%
\pgfusepath{stroke,fill}%
\end{pgfscope}%
\begin{pgfscope}%
\pgfpathrectangle{\pgfqpoint{1.374500in}{0.082500in}}{\pgfqpoint{2.419000in}{2.419000in}}%
\pgfusepath{clip}%
\pgfsetbuttcap%
\pgfsetroundjoin%
\definecolor{currentfill}{rgb}{1.000000,0.662745,0.054902}%
\pgfsetfillcolor{currentfill}%
\pgfsetfillopacity{0.399826}%
\pgfsetlinewidth{1.003750pt}%
\definecolor{currentstroke}{rgb}{1.000000,0.662745,0.054902}%
\pgfsetstrokecolor{currentstroke}%
\pgfsetstrokeopacity{0.399826}%
\pgfsetdash{}{0pt}%
\pgfpathmoveto{\pgfqpoint{10.999587in}{2.125837in}}%
\pgfpathcurveto{\pgfqpoint{11.008795in}{2.125837in}}{\pgfqpoint{11.017627in}{2.129495in}}{\pgfqpoint{11.024139in}{2.136006in}}%
\pgfpathcurveto{\pgfqpoint{11.030650in}{2.142518in}}{\pgfqpoint{11.034309in}{2.151350in}}{\pgfqpoint{11.034309in}{2.160559in}}%
\pgfpathcurveto{\pgfqpoint{11.034309in}{2.169767in}}{\pgfqpoint{11.030650in}{2.178600in}}{\pgfqpoint{11.024139in}{2.185111in}}%
\pgfpathcurveto{\pgfqpoint{11.017627in}{2.191622in}}{\pgfqpoint{11.008795in}{2.195281in}}{\pgfqpoint{10.999587in}{2.195281in}}%
\pgfpathcurveto{\pgfqpoint{10.990378in}{2.195281in}}{\pgfqpoint{10.981546in}{2.191622in}}{\pgfqpoint{10.975034in}{2.185111in}}%
\pgfpathcurveto{\pgfqpoint{10.968523in}{2.178600in}}{\pgfqpoint{10.964864in}{2.169767in}}{\pgfqpoint{10.964864in}{2.160559in}}%
\pgfpathcurveto{\pgfqpoint{10.964864in}{2.151350in}}{\pgfqpoint{10.968523in}{2.142518in}}{\pgfqpoint{10.975034in}{2.136006in}}%
\pgfpathcurveto{\pgfqpoint{10.981546in}{2.129495in}}{\pgfqpoint{10.990378in}{2.125837in}}{\pgfqpoint{10.999587in}{2.125837in}}%
\pgfpathlineto{\pgfqpoint{10.999587in}{2.125837in}}%
\pgfpathclose%
\pgfusepath{stroke,fill}%
\end{pgfscope}%
\begin{pgfscope}%
\pgfpathrectangle{\pgfqpoint{1.374500in}{0.082500in}}{\pgfqpoint{2.419000in}{2.419000in}}%
\pgfusepath{clip}%
\pgfsetbuttcap%
\pgfsetroundjoin%
\definecolor{currentfill}{rgb}{1.000000,0.662745,0.054902}%
\pgfsetfillcolor{currentfill}%
\pgfsetfillopacity{0.404744}%
\pgfsetlinewidth{1.003750pt}%
\definecolor{currentstroke}{rgb}{1.000000,0.662745,0.054902}%
\pgfsetstrokecolor{currentstroke}%
\pgfsetstrokeopacity{0.404744}%
\pgfsetdash{}{0pt}%
\pgfpathmoveto{\pgfqpoint{0.778724in}{2.022317in}}%
\pgfpathcurveto{\pgfqpoint{0.787933in}{2.022317in}}{\pgfqpoint{0.796765in}{2.025975in}}{\pgfqpoint{0.803277in}{2.032487in}}%
\pgfpathcurveto{\pgfqpoint{0.809788in}{2.038998in}}{\pgfqpoint{0.813446in}{2.047831in}}{\pgfqpoint{0.813446in}{2.057039in}}%
\pgfpathcurveto{\pgfqpoint{0.813446in}{2.066248in}}{\pgfqpoint{0.809788in}{2.075080in}}{\pgfqpoint{0.803277in}{2.081591in}}%
\pgfpathcurveto{\pgfqpoint{0.796765in}{2.088103in}}{\pgfqpoint{0.787933in}{2.091761in}}{\pgfqpoint{0.778724in}{2.091761in}}%
\pgfpathcurveto{\pgfqpoint{0.769516in}{2.091761in}}{\pgfqpoint{0.760683in}{2.088103in}}{\pgfqpoint{0.754172in}{2.081591in}}%
\pgfpathcurveto{\pgfqpoint{0.747661in}{2.075080in}}{\pgfqpoint{0.744002in}{2.066248in}}{\pgfqpoint{0.744002in}{2.057039in}}%
\pgfpathcurveto{\pgfqpoint{0.744002in}{2.047831in}}{\pgfqpoint{0.747661in}{2.038998in}}{\pgfqpoint{0.754172in}{2.032487in}}%
\pgfpathcurveto{\pgfqpoint{0.760683in}{2.025975in}}{\pgfqpoint{0.769516in}{2.022317in}}{\pgfqpoint{0.778724in}{2.022317in}}%
\pgfpathlineto{\pgfqpoint{0.778724in}{2.022317in}}%
\pgfpathclose%
\pgfusepath{stroke,fill}%
\end{pgfscope}%
\begin{pgfscope}%
\pgfpathrectangle{\pgfqpoint{1.374500in}{0.082500in}}{\pgfqpoint{2.419000in}{2.419000in}}%
\pgfusepath{clip}%
\pgfsetbuttcap%
\pgfsetroundjoin%
\definecolor{currentfill}{rgb}{1.000000,0.662745,0.054902}%
\pgfsetfillcolor{currentfill}%
\pgfsetfillopacity{0.404744}%
\pgfsetlinewidth{1.003750pt}%
\definecolor{currentstroke}{rgb}{1.000000,0.662745,0.054902}%
\pgfsetstrokecolor{currentstroke}%
\pgfsetstrokeopacity{0.404744}%
\pgfsetdash{}{0pt}%
\pgfpathmoveto{\pgfqpoint{9.381426in}{2.022317in}}%
\pgfpathcurveto{\pgfqpoint{9.390635in}{2.022317in}}{\pgfqpoint{9.399467in}{2.025975in}}{\pgfqpoint{9.405979in}{2.032487in}}%
\pgfpathcurveto{\pgfqpoint{9.412490in}{2.038998in}}{\pgfqpoint{9.416148in}{2.047831in}}{\pgfqpoint{9.416148in}{2.057039in}}%
\pgfpathcurveto{\pgfqpoint{9.416148in}{2.066248in}}{\pgfqpoint{9.412490in}{2.075080in}}{\pgfqpoint{9.405979in}{2.081591in}}%
\pgfpathcurveto{\pgfqpoint{9.399467in}{2.088103in}}{\pgfqpoint{9.390635in}{2.091761in}}{\pgfqpoint{9.381426in}{2.091761in}}%
\pgfpathcurveto{\pgfqpoint{9.372218in}{2.091761in}}{\pgfqpoint{9.363385in}{2.088103in}}{\pgfqpoint{9.356874in}{2.081591in}}%
\pgfpathcurveto{\pgfqpoint{9.350363in}{2.075080in}}{\pgfqpoint{9.346704in}{2.066248in}}{\pgfqpoint{9.346704in}{2.057039in}}%
\pgfpathcurveto{\pgfqpoint{9.346704in}{2.047831in}}{\pgfqpoint{9.350363in}{2.038998in}}{\pgfqpoint{9.356874in}{2.032487in}}%
\pgfpathcurveto{\pgfqpoint{9.363385in}{2.025975in}}{\pgfqpoint{9.372218in}{2.022317in}}{\pgfqpoint{9.381426in}{2.022317in}}%
\pgfpathlineto{\pgfqpoint{9.381426in}{2.022317in}}%
\pgfpathclose%
\pgfusepath{stroke,fill}%
\end{pgfscope}%
\begin{pgfscope}%
\pgfpathrectangle{\pgfqpoint{1.374500in}{0.082500in}}{\pgfqpoint{2.419000in}{2.419000in}}%
\pgfusepath{clip}%
\pgfsetbuttcap%
\pgfsetroundjoin%
\definecolor{currentfill}{rgb}{1.000000,0.662745,0.054902}%
\pgfsetfillcolor{currentfill}%
\pgfsetfillopacity{0.404744}%
\pgfsetlinewidth{1.003750pt}%
\definecolor{currentstroke}{rgb}{1.000000,0.662745,0.054902}%
\pgfsetstrokecolor{currentstroke}%
\pgfsetstrokeopacity{0.404744}%
\pgfsetdash{}{0pt}%
\pgfpathmoveto{\pgfqpoint{5.080075in}{2.022317in}}%
\pgfpathcurveto{\pgfqpoint{5.089284in}{2.022317in}}{\pgfqpoint{5.098116in}{2.025975in}}{\pgfqpoint{5.104628in}{2.032487in}}%
\pgfpathcurveto{\pgfqpoint{5.111139in}{2.038998in}}{\pgfqpoint{5.114797in}{2.047831in}}{\pgfqpoint{5.114797in}{2.057039in}}%
\pgfpathcurveto{\pgfqpoint{5.114797in}{2.066248in}}{\pgfqpoint{5.111139in}{2.075080in}}{\pgfqpoint{5.104628in}{2.081591in}}%
\pgfpathcurveto{\pgfqpoint{5.098116in}{2.088103in}}{\pgfqpoint{5.089284in}{2.091761in}}{\pgfqpoint{5.080075in}{2.091761in}}%
\pgfpathcurveto{\pgfqpoint{5.070867in}{2.091761in}}{\pgfqpoint{5.062034in}{2.088103in}}{\pgfqpoint{5.055523in}{2.081591in}}%
\pgfpathcurveto{\pgfqpoint{5.049012in}{2.075080in}}{\pgfqpoint{5.045353in}{2.066248in}}{\pgfqpoint{5.045353in}{2.057039in}}%
\pgfpathcurveto{\pgfqpoint{5.045353in}{2.047831in}}{\pgfqpoint{5.049012in}{2.038998in}}{\pgfqpoint{5.055523in}{2.032487in}}%
\pgfpathcurveto{\pgfqpoint{5.062034in}{2.025975in}}{\pgfqpoint{5.070867in}{2.022317in}}{\pgfqpoint{5.080075in}{2.022317in}}%
\pgfpathlineto{\pgfqpoint{5.080075in}{2.022317in}}%
\pgfpathclose%
\pgfusepath{stroke,fill}%
\end{pgfscope}%
\begin{pgfscope}%
\pgfpathrectangle{\pgfqpoint{1.374500in}{0.082500in}}{\pgfqpoint{2.419000in}{2.419000in}}%
\pgfusepath{clip}%
\pgfsetbuttcap%
\pgfsetroundjoin%
\definecolor{currentfill}{rgb}{1.000000,0.662745,0.054902}%
\pgfsetfillcolor{currentfill}%
\pgfsetfillopacity{0.409816}%
\pgfsetlinewidth{1.003750pt}%
\definecolor{currentstroke}{rgb}{1.000000,0.662745,0.054902}%
\pgfsetstrokecolor{currentstroke}%
\pgfsetstrokeopacity{0.409816}%
\pgfsetdash{}{0pt}%
\pgfpathmoveto{\pgfqpoint{-1.024852in}{1.915555in}}%
\pgfpathcurveto{\pgfqpoint{-1.015643in}{1.915555in}}{\pgfqpoint{-1.006811in}{1.919213in}}{\pgfqpoint{-1.000300in}{1.925725in}}%
\pgfpathcurveto{\pgfqpoint{-0.993788in}{1.932236in}}{\pgfqpoint{-0.990130in}{1.941068in}}{\pgfqpoint{-0.990130in}{1.950277in}}%
\pgfpathcurveto{\pgfqpoint{-0.990130in}{1.959485in}}{\pgfqpoint{-0.993788in}{1.968318in}}{\pgfqpoint{-1.000300in}{1.974829in}}%
\pgfpathcurveto{\pgfqpoint{-1.006811in}{1.981341in}}{\pgfqpoint{-1.015643in}{1.984999in}}{\pgfqpoint{-1.024852in}{1.984999in}}%
\pgfpathcurveto{\pgfqpoint{-1.034060in}{1.984999in}}{\pgfqpoint{-1.042893in}{1.981341in}}{\pgfqpoint{-1.049404in}{1.974829in}}%
\pgfpathcurveto{\pgfqpoint{-1.055915in}{1.968318in}}{\pgfqpoint{-1.059574in}{1.959485in}}{\pgfqpoint{-1.059574in}{1.950277in}}%
\pgfpathcurveto{\pgfqpoint{-1.059574in}{1.941068in}}{\pgfqpoint{-1.055915in}{1.932236in}}{\pgfqpoint{-1.049404in}{1.925725in}}%
\pgfpathcurveto{\pgfqpoint{-1.042893in}{1.919213in}}{\pgfqpoint{-1.034060in}{1.915555in}}{\pgfqpoint{-1.024852in}{1.915555in}}%
\pgfpathlineto{\pgfqpoint{-1.024852in}{1.915555in}}%
\pgfpathclose%
\pgfusepath{stroke,fill}%
\end{pgfscope}%
\begin{pgfscope}%
\pgfpathrectangle{\pgfqpoint{1.374500in}{0.082500in}}{\pgfqpoint{2.419000in}{2.419000in}}%
\pgfusepath{clip}%
\pgfsetbuttcap%
\pgfsetroundjoin%
\definecolor{currentfill}{rgb}{1.000000,0.662745,0.054902}%
\pgfsetfillcolor{currentfill}%
\pgfsetfillopacity{0.409816}%
\pgfsetlinewidth{1.003750pt}%
\definecolor{currentstroke}{rgb}{1.000000,0.662745,0.054902}%
\pgfsetstrokecolor{currentstroke}%
\pgfsetstrokeopacity{0.409816}%
\pgfsetdash{}{0pt}%
\pgfpathmoveto{\pgfqpoint{3.343864in}{1.915555in}}%
\pgfpathcurveto{\pgfqpoint{3.353073in}{1.915555in}}{\pgfqpoint{3.361905in}{1.919213in}}{\pgfqpoint{3.368417in}{1.925725in}}%
\pgfpathcurveto{\pgfqpoint{3.374928in}{1.932236in}}{\pgfqpoint{3.378587in}{1.941068in}}{\pgfqpoint{3.378587in}{1.950277in}}%
\pgfpathcurveto{\pgfqpoint{3.378587in}{1.959485in}}{\pgfqpoint{3.374928in}{1.968318in}}{\pgfqpoint{3.368417in}{1.974829in}}%
\pgfpathcurveto{\pgfqpoint{3.361905in}{1.981341in}}{\pgfqpoint{3.353073in}{1.984999in}}{\pgfqpoint{3.343864in}{1.984999in}}%
\pgfpathcurveto{\pgfqpoint{3.334656in}{1.984999in}}{\pgfqpoint{3.325823in}{1.981341in}}{\pgfqpoint{3.319312in}{1.974829in}}%
\pgfpathcurveto{\pgfqpoint{3.312801in}{1.968318in}}{\pgfqpoint{3.309142in}{1.959485in}}{\pgfqpoint{3.309142in}{1.950277in}}%
\pgfpathcurveto{\pgfqpoint{3.309142in}{1.941068in}}{\pgfqpoint{3.312801in}{1.932236in}}{\pgfqpoint{3.319312in}{1.925725in}}%
\pgfpathcurveto{\pgfqpoint{3.325823in}{1.919213in}}{\pgfqpoint{3.334656in}{1.915555in}}{\pgfqpoint{3.343864in}{1.915555in}}%
\pgfpathlineto{\pgfqpoint{3.343864in}{1.915555in}}%
\pgfpathclose%
\pgfusepath{stroke,fill}%
\end{pgfscope}%
\begin{pgfscope}%
\pgfpathrectangle{\pgfqpoint{1.374500in}{0.082500in}}{\pgfqpoint{2.419000in}{2.419000in}}%
\pgfusepath{clip}%
\pgfsetbuttcap%
\pgfsetroundjoin%
\definecolor{currentfill}{rgb}{1.000000,0.662745,0.054902}%
\pgfsetfillcolor{currentfill}%
\pgfsetfillopacity{0.409816}%
\pgfsetlinewidth{1.003750pt}%
\definecolor{currentstroke}{rgb}{1.000000,0.662745,0.054902}%
\pgfsetstrokecolor{currentstroke}%
\pgfsetstrokeopacity{0.409816}%
\pgfsetdash{}{0pt}%
\pgfpathmoveto{\pgfqpoint{7.712581in}{1.915555in}}%
\pgfpathcurveto{\pgfqpoint{7.721789in}{1.915555in}}{\pgfqpoint{7.730622in}{1.919213in}}{\pgfqpoint{7.737133in}{1.925725in}}%
\pgfpathcurveto{\pgfqpoint{7.743644in}{1.932236in}}{\pgfqpoint{7.747303in}{1.941068in}}{\pgfqpoint{7.747303in}{1.950277in}}%
\pgfpathcurveto{\pgfqpoint{7.747303in}{1.959485in}}{\pgfqpoint{7.743644in}{1.968318in}}{\pgfqpoint{7.737133in}{1.974829in}}%
\pgfpathcurveto{\pgfqpoint{7.730622in}{1.981341in}}{\pgfqpoint{7.721789in}{1.984999in}}{\pgfqpoint{7.712581in}{1.984999in}}%
\pgfpathcurveto{\pgfqpoint{7.703372in}{1.984999in}}{\pgfqpoint{7.694540in}{1.981341in}}{\pgfqpoint{7.688028in}{1.974829in}}%
\pgfpathcurveto{\pgfqpoint{7.681517in}{1.968318in}}{\pgfqpoint{7.677858in}{1.959485in}}{\pgfqpoint{7.677858in}{1.950277in}}%
\pgfpathcurveto{\pgfqpoint{7.677858in}{1.941068in}}{\pgfqpoint{7.681517in}{1.932236in}}{\pgfqpoint{7.688028in}{1.925725in}}%
\pgfpathcurveto{\pgfqpoint{7.694540in}{1.919213in}}{\pgfqpoint{7.703372in}{1.915555in}}{\pgfqpoint{7.712581in}{1.915555in}}%
\pgfpathlineto{\pgfqpoint{7.712581in}{1.915555in}}%
\pgfpathclose%
\pgfusepath{stroke,fill}%
\end{pgfscope}%
\begin{pgfscope}%
\pgfpathrectangle{\pgfqpoint{1.374500in}{0.082500in}}{\pgfqpoint{2.419000in}{2.419000in}}%
\pgfusepath{clip}%
\pgfsetbuttcap%
\pgfsetroundjoin%
\definecolor{currentfill}{rgb}{1.000000,0.662745,0.054902}%
\pgfsetfillcolor{currentfill}%
\pgfsetfillopacity{0.415049}%
\pgfsetlinewidth{1.003750pt}%
\definecolor{currentstroke}{rgb}{1.000000,0.662745,0.054902}%
\pgfsetstrokecolor{currentstroke}%
\pgfsetstrokeopacity{0.415049}%
\pgfsetdash{}{0pt}%
\pgfpathmoveto{\pgfqpoint{5.990630in}{1.805395in}}%
\pgfpathcurveto{\pgfqpoint{5.999839in}{1.805395in}}{\pgfqpoint{6.008671in}{1.809054in}}{\pgfqpoint{6.015183in}{1.815565in}}%
\pgfpathcurveto{\pgfqpoint{6.021694in}{1.822076in}}{\pgfqpoint{6.025353in}{1.830909in}}{\pgfqpoint{6.025353in}{1.840117in}}%
\pgfpathcurveto{\pgfqpoint{6.025353in}{1.849326in}}{\pgfqpoint{6.021694in}{1.858158in}}{\pgfqpoint{6.015183in}{1.864670in}}%
\pgfpathcurveto{\pgfqpoint{6.008671in}{1.871181in}}{\pgfqpoint{5.999839in}{1.874840in}}{\pgfqpoint{5.990630in}{1.874840in}}%
\pgfpathcurveto{\pgfqpoint{5.981422in}{1.874840in}}{\pgfqpoint{5.972589in}{1.871181in}}{\pgfqpoint{5.966078in}{1.864670in}}%
\pgfpathcurveto{\pgfqpoint{5.959567in}{1.858158in}}{\pgfqpoint{5.955908in}{1.849326in}}{\pgfqpoint{5.955908in}{1.840117in}}%
\pgfpathcurveto{\pgfqpoint{5.955908in}{1.830909in}}{\pgfqpoint{5.959567in}{1.822076in}}{\pgfqpoint{5.966078in}{1.815565in}}%
\pgfpathcurveto{\pgfqpoint{5.972589in}{1.809054in}}{\pgfqpoint{5.981422in}{1.805395in}}{\pgfqpoint{5.990630in}{1.805395in}}%
\pgfpathlineto{\pgfqpoint{5.990630in}{1.805395in}}%
\pgfpathclose%
\pgfusepath{stroke,fill}%
\end{pgfscope}%
\begin{pgfscope}%
\pgfpathrectangle{\pgfqpoint{1.374500in}{0.082500in}}{\pgfqpoint{2.419000in}{2.419000in}}%
\pgfusepath{clip}%
\pgfsetbuttcap%
\pgfsetroundjoin%
\definecolor{currentfill}{rgb}{1.000000,0.662745,0.054902}%
\pgfsetfillcolor{currentfill}%
\pgfsetfillopacity{0.415049}%
\pgfsetlinewidth{1.003750pt}%
\definecolor{currentstroke}{rgb}{1.000000,0.662745,0.054902}%
\pgfsetstrokecolor{currentstroke}%
\pgfsetstrokeopacity{0.415049}%
\pgfsetdash{}{0pt}%
\pgfpathmoveto{\pgfqpoint{10.428855in}{1.805395in}}%
\pgfpathcurveto{\pgfqpoint{10.438064in}{1.805395in}}{\pgfqpoint{10.446896in}{1.809054in}}{\pgfqpoint{10.453408in}{1.815565in}}%
\pgfpathcurveto{\pgfqpoint{10.459919in}{1.822076in}}{\pgfqpoint{10.463578in}{1.830909in}}{\pgfqpoint{10.463578in}{1.840117in}}%
\pgfpathcurveto{\pgfqpoint{10.463578in}{1.849326in}}{\pgfqpoint{10.459919in}{1.858158in}}{\pgfqpoint{10.453408in}{1.864670in}}%
\pgfpathcurveto{\pgfqpoint{10.446896in}{1.871181in}}{\pgfqpoint{10.438064in}{1.874840in}}{\pgfqpoint{10.428855in}{1.874840in}}%
\pgfpathcurveto{\pgfqpoint{10.419647in}{1.874840in}}{\pgfqpoint{10.410815in}{1.871181in}}{\pgfqpoint{10.404303in}{1.864670in}}%
\pgfpathcurveto{\pgfqpoint{10.397792in}{1.858158in}}{\pgfqpoint{10.394133in}{1.849326in}}{\pgfqpoint{10.394133in}{1.840117in}}%
\pgfpathcurveto{\pgfqpoint{10.394133in}{1.830909in}}{\pgfqpoint{10.397792in}{1.822076in}}{\pgfqpoint{10.404303in}{1.815565in}}%
\pgfpathcurveto{\pgfqpoint{10.410815in}{1.809054in}}{\pgfqpoint{10.419647in}{1.805395in}}{\pgfqpoint{10.428855in}{1.805395in}}%
\pgfpathlineto{\pgfqpoint{10.428855in}{1.805395in}}%
\pgfpathclose%
\pgfusepath{stroke,fill}%
\end{pgfscope}%
\begin{pgfscope}%
\pgfpathrectangle{\pgfqpoint{1.374500in}{0.082500in}}{\pgfqpoint{2.419000in}{2.419000in}}%
\pgfusepath{clip}%
\pgfsetbuttcap%
\pgfsetroundjoin%
\definecolor{currentfill}{rgb}{1.000000,0.662745,0.054902}%
\pgfsetfillcolor{currentfill}%
\pgfsetfillopacity{0.415049}%
\pgfsetlinewidth{1.003750pt}%
\definecolor{currentstroke}{rgb}{1.000000,0.662745,0.054902}%
\pgfsetstrokecolor{currentstroke}%
\pgfsetstrokeopacity{0.415049}%
\pgfsetdash{}{0pt}%
\pgfpathmoveto{\pgfqpoint{1.552405in}{1.805395in}}%
\pgfpathcurveto{\pgfqpoint{1.561614in}{1.805395in}}{\pgfqpoint{1.570446in}{1.809054in}}{\pgfqpoint{1.576958in}{1.815565in}}%
\pgfpathcurveto{\pgfqpoint{1.583469in}{1.822076in}}{\pgfqpoint{1.587127in}{1.830909in}}{\pgfqpoint{1.587127in}{1.840117in}}%
\pgfpathcurveto{\pgfqpoint{1.587127in}{1.849326in}}{\pgfqpoint{1.583469in}{1.858158in}}{\pgfqpoint{1.576958in}{1.864670in}}%
\pgfpathcurveto{\pgfqpoint{1.570446in}{1.871181in}}{\pgfqpoint{1.561614in}{1.874840in}}{\pgfqpoint{1.552405in}{1.874840in}}%
\pgfpathcurveto{\pgfqpoint{1.543197in}{1.874840in}}{\pgfqpoint{1.534364in}{1.871181in}}{\pgfqpoint{1.527853in}{1.864670in}}%
\pgfpathcurveto{\pgfqpoint{1.521342in}{1.858158in}}{\pgfqpoint{1.517683in}{1.849326in}}{\pgfqpoint{1.517683in}{1.840117in}}%
\pgfpathcurveto{\pgfqpoint{1.517683in}{1.830909in}}{\pgfqpoint{1.521342in}{1.822076in}}{\pgfqpoint{1.527853in}{1.815565in}}%
\pgfpathcurveto{\pgfqpoint{1.534364in}{1.809054in}}{\pgfqpoint{1.543197in}{1.805395in}}{\pgfqpoint{1.552405in}{1.805395in}}%
\pgfpathlineto{\pgfqpoint{1.552405in}{1.805395in}}%
\pgfpathclose%
\pgfusepath{stroke,fill}%
\end{pgfscope}%
\begin{pgfscope}%
\pgfpathrectangle{\pgfqpoint{1.374500in}{0.082500in}}{\pgfqpoint{2.419000in}{2.419000in}}%
\pgfusepath{clip}%
\pgfsetbuttcap%
\pgfsetroundjoin%
\definecolor{currentfill}{rgb}{1.000000,0.662745,0.054902}%
\pgfsetfillcolor{currentfill}%
\pgfsetfillopacity{0.420452}%
\pgfsetlinewidth{1.003750pt}%
\definecolor{currentstroke}{rgb}{1.000000,0.662745,0.054902}%
\pgfsetstrokecolor{currentstroke}%
\pgfsetstrokeopacity{0.420452}%
\pgfsetdash{}{0pt}%
\pgfpathmoveto{\pgfqpoint{-0.296982in}{1.691674in}}%
\pgfpathcurveto{\pgfqpoint{-0.287774in}{1.691674in}}{\pgfqpoint{-0.278941in}{1.695332in}}{\pgfqpoint{-0.272430in}{1.701843in}}%
\pgfpathcurveto{\pgfqpoint{-0.265918in}{1.708355in}}{\pgfqpoint{-0.262260in}{1.717187in}}{\pgfqpoint{-0.262260in}{1.726396in}}%
\pgfpathcurveto{\pgfqpoint{-0.262260in}{1.735604in}}{\pgfqpoint{-0.265918in}{1.744437in}}{\pgfqpoint{-0.272430in}{1.750948in}}%
\pgfpathcurveto{\pgfqpoint{-0.278941in}{1.757459in}}{\pgfqpoint{-0.287774in}{1.761118in}}{\pgfqpoint{-0.296982in}{1.761118in}}%
\pgfpathcurveto{\pgfqpoint{-0.306190in}{1.761118in}}{\pgfqpoint{-0.315023in}{1.757459in}}{\pgfqpoint{-0.321534in}{1.750948in}}%
\pgfpathcurveto{\pgfqpoint{-0.328046in}{1.744437in}}{\pgfqpoint{-0.331704in}{1.735604in}}{\pgfqpoint{-0.331704in}{1.726396in}}%
\pgfpathcurveto{\pgfqpoint{-0.331704in}{1.717187in}}{\pgfqpoint{-0.328046in}{1.708355in}}{\pgfqpoint{-0.321534in}{1.701843in}}%
\pgfpathcurveto{\pgfqpoint{-0.315023in}{1.695332in}}{\pgfqpoint{-0.306190in}{1.691674in}}{\pgfqpoint{-0.296982in}{1.691674in}}%
\pgfpathlineto{\pgfqpoint{-0.296982in}{1.691674in}}%
\pgfpathclose%
\pgfusepath{stroke,fill}%
\end{pgfscope}%
\begin{pgfscope}%
\pgfpathrectangle{\pgfqpoint{1.374500in}{0.082500in}}{\pgfqpoint{2.419000in}{2.419000in}}%
\pgfusepath{clip}%
\pgfsetbuttcap%
\pgfsetroundjoin%
\definecolor{currentfill}{rgb}{1.000000,0.662745,0.054902}%
\pgfsetfillcolor{currentfill}%
\pgfsetfillopacity{0.420452}%
\pgfsetlinewidth{1.003750pt}%
\definecolor{currentstroke}{rgb}{1.000000,0.662745,0.054902}%
\pgfsetstrokecolor{currentstroke}%
\pgfsetstrokeopacity{0.420452}%
\pgfsetdash{}{0pt}%
\pgfpathmoveto{\pgfqpoint{4.213000in}{1.691674in}}%
\pgfpathcurveto{\pgfqpoint{4.222208in}{1.691674in}}{\pgfqpoint{4.231041in}{1.695332in}}{\pgfqpoint{4.237552in}{1.701843in}}%
\pgfpathcurveto{\pgfqpoint{4.244063in}{1.708355in}}{\pgfqpoint{4.247722in}{1.717187in}}{\pgfqpoint{4.247722in}{1.726396in}}%
\pgfpathcurveto{\pgfqpoint{4.247722in}{1.735604in}}{\pgfqpoint{4.244063in}{1.744437in}}{\pgfqpoint{4.237552in}{1.750948in}}%
\pgfpathcurveto{\pgfqpoint{4.231041in}{1.757459in}}{\pgfqpoint{4.222208in}{1.761118in}}{\pgfqpoint{4.213000in}{1.761118in}}%
\pgfpathcurveto{\pgfqpoint{4.203791in}{1.761118in}}{\pgfqpoint{4.194959in}{1.757459in}}{\pgfqpoint{4.188447in}{1.750948in}}%
\pgfpathcurveto{\pgfqpoint{4.181936in}{1.744437in}}{\pgfqpoint{4.178277in}{1.735604in}}{\pgfqpoint{4.178277in}{1.726396in}}%
\pgfpathcurveto{\pgfqpoint{4.178277in}{1.717187in}}{\pgfqpoint{4.181936in}{1.708355in}}{\pgfqpoint{4.188447in}{1.701843in}}%
\pgfpathcurveto{\pgfqpoint{4.194959in}{1.695332in}}{\pgfqpoint{4.203791in}{1.691674in}}{\pgfqpoint{4.213000in}{1.691674in}}%
\pgfpathlineto{\pgfqpoint{4.213000in}{1.691674in}}%
\pgfpathclose%
\pgfusepath{stroke,fill}%
\end{pgfscope}%
\begin{pgfscope}%
\pgfpathrectangle{\pgfqpoint{1.374500in}{0.082500in}}{\pgfqpoint{2.419000in}{2.419000in}}%
\pgfusepath{clip}%
\pgfsetbuttcap%
\pgfsetroundjoin%
\definecolor{currentfill}{rgb}{1.000000,0.662745,0.054902}%
\pgfsetfillcolor{currentfill}%
\pgfsetfillopacity{0.420452}%
\pgfsetlinewidth{1.003750pt}%
\definecolor{currentstroke}{rgb}{1.000000,0.662745,0.054902}%
\pgfsetstrokecolor{currentstroke}%
\pgfsetstrokeopacity{0.420452}%
\pgfsetdash{}{0pt}%
\pgfpathmoveto{\pgfqpoint{8.722981in}{1.691674in}}%
\pgfpathcurveto{\pgfqpoint{8.732190in}{1.691674in}}{\pgfqpoint{8.741022in}{1.695332in}}{\pgfqpoint{8.747534in}{1.701843in}}%
\pgfpathcurveto{\pgfqpoint{8.754045in}{1.708355in}}{\pgfqpoint{8.757703in}{1.717187in}}{\pgfqpoint{8.757703in}{1.726396in}}%
\pgfpathcurveto{\pgfqpoint{8.757703in}{1.735604in}}{\pgfqpoint{8.754045in}{1.744437in}}{\pgfqpoint{8.747534in}{1.750948in}}%
\pgfpathcurveto{\pgfqpoint{8.741022in}{1.757459in}}{\pgfqpoint{8.732190in}{1.761118in}}{\pgfqpoint{8.722981in}{1.761118in}}%
\pgfpathcurveto{\pgfqpoint{8.713773in}{1.761118in}}{\pgfqpoint{8.704940in}{1.757459in}}{\pgfqpoint{8.698429in}{1.750948in}}%
\pgfpathcurveto{\pgfqpoint{8.691918in}{1.744437in}}{\pgfqpoint{8.688259in}{1.735604in}}{\pgfqpoint{8.688259in}{1.726396in}}%
\pgfpathcurveto{\pgfqpoint{8.688259in}{1.717187in}}{\pgfqpoint{8.691918in}{1.708355in}}{\pgfqpoint{8.698429in}{1.701843in}}%
\pgfpathcurveto{\pgfqpoint{8.704940in}{1.695332in}}{\pgfqpoint{8.713773in}{1.691674in}}{\pgfqpoint{8.722981in}{1.691674in}}%
\pgfpathlineto{\pgfqpoint{8.722981in}{1.691674in}}%
\pgfpathclose%
\pgfusepath{stroke,fill}%
\end{pgfscope}%
\begin{pgfscope}%
\pgfpathrectangle{\pgfqpoint{1.374500in}{0.082500in}}{\pgfqpoint{2.419000in}{2.419000in}}%
\pgfusepath{clip}%
\pgfsetbuttcap%
\pgfsetroundjoin%
\definecolor{currentfill}{rgb}{1.000000,0.662745,0.054902}%
\pgfsetfillcolor{currentfill}%
\pgfsetfillopacity{0.426033}%
\pgfsetlinewidth{1.003750pt}%
\definecolor{currentstroke}{rgb}{1.000000,0.662745,0.054902}%
\pgfsetstrokecolor{currentstroke}%
\pgfsetstrokeopacity{0.426033}%
\pgfsetdash{}{0pt}%
\pgfpathmoveto{\pgfqpoint{2.376943in}{1.574214in}}%
\pgfpathcurveto{\pgfqpoint{2.386152in}{1.574214in}}{\pgfqpoint{2.394984in}{1.577873in}}{\pgfqpoint{2.401496in}{1.584384in}}%
\pgfpathcurveto{\pgfqpoint{2.408007in}{1.590895in}}{\pgfqpoint{2.411666in}{1.599728in}}{\pgfqpoint{2.411666in}{1.608936in}}%
\pgfpathcurveto{\pgfqpoint{2.411666in}{1.618145in}}{\pgfqpoint{2.408007in}{1.626977in}}{\pgfqpoint{2.401496in}{1.633489in}}%
\pgfpathcurveto{\pgfqpoint{2.394984in}{1.640000in}}{\pgfqpoint{2.386152in}{1.643659in}}{\pgfqpoint{2.376943in}{1.643659in}}%
\pgfpathcurveto{\pgfqpoint{2.367735in}{1.643659in}}{\pgfqpoint{2.358902in}{1.640000in}}{\pgfqpoint{2.352391in}{1.633489in}}%
\pgfpathcurveto{\pgfqpoint{2.345880in}{1.626977in}}{\pgfqpoint{2.342221in}{1.618145in}}{\pgfqpoint{2.342221in}{1.608936in}}%
\pgfpathcurveto{\pgfqpoint{2.342221in}{1.599728in}}{\pgfqpoint{2.345880in}{1.590895in}}{\pgfqpoint{2.352391in}{1.584384in}}%
\pgfpathcurveto{\pgfqpoint{2.358902in}{1.577873in}}{\pgfqpoint{2.367735in}{1.574214in}}{\pgfqpoint{2.376943in}{1.574214in}}%
\pgfpathlineto{\pgfqpoint{2.376943in}{1.574214in}}%
\pgfpathclose%
\pgfusepath{stroke,fill}%
\end{pgfscope}%
\begin{pgfscope}%
\pgfpathrectangle{\pgfqpoint{1.374500in}{0.082500in}}{\pgfqpoint{2.419000in}{2.419000in}}%
\pgfusepath{clip}%
\pgfsetbuttcap%
\pgfsetroundjoin%
\definecolor{currentfill}{rgb}{1.000000,0.662745,0.054902}%
\pgfsetfillcolor{currentfill}%
\pgfsetfillopacity{0.426033}%
\pgfsetlinewidth{1.003750pt}%
\definecolor{currentstroke}{rgb}{1.000000,0.662745,0.054902}%
\pgfsetstrokecolor{currentstroke}%
\pgfsetstrokeopacity{0.426033}%
\pgfsetdash{}{0pt}%
\pgfpathmoveto{\pgfqpoint{11.545137in}{1.574214in}}%
\pgfpathcurveto{\pgfqpoint{11.554345in}{1.574214in}}{\pgfqpoint{11.563177in}{1.577873in}}{\pgfqpoint{11.569689in}{1.584384in}}%
\pgfpathcurveto{\pgfqpoint{11.576200in}{1.590895in}}{\pgfqpoint{11.579859in}{1.599728in}}{\pgfqpoint{11.579859in}{1.608936in}}%
\pgfpathcurveto{\pgfqpoint{11.579859in}{1.618145in}}{\pgfqpoint{11.576200in}{1.626977in}}{\pgfqpoint{11.569689in}{1.633489in}}%
\pgfpathcurveto{\pgfqpoint{11.563177in}{1.640000in}}{\pgfqpoint{11.554345in}{1.643659in}}{\pgfqpoint{11.545137in}{1.643659in}}%
\pgfpathcurveto{\pgfqpoint{11.535928in}{1.643659in}}{\pgfqpoint{11.527096in}{1.640000in}}{\pgfqpoint{11.520584in}{1.633489in}}%
\pgfpathcurveto{\pgfqpoint{11.514073in}{1.626977in}}{\pgfqpoint{11.510414in}{1.618145in}}{\pgfqpoint{11.510414in}{1.608936in}}%
\pgfpathcurveto{\pgfqpoint{11.510414in}{1.599728in}}{\pgfqpoint{11.514073in}{1.590895in}}{\pgfqpoint{11.520584in}{1.584384in}}%
\pgfpathcurveto{\pgfqpoint{11.527096in}{1.577873in}}{\pgfqpoint{11.535928in}{1.574214in}}{\pgfqpoint{11.545137in}{1.574214in}}%
\pgfpathlineto{\pgfqpoint{11.545137in}{1.574214in}}%
\pgfpathclose%
\pgfusepath{stroke,fill}%
\end{pgfscope}%
\begin{pgfscope}%
\pgfpathrectangle{\pgfqpoint{1.374500in}{0.082500in}}{\pgfqpoint{2.419000in}{2.419000in}}%
\pgfusepath{clip}%
\pgfsetbuttcap%
\pgfsetroundjoin%
\definecolor{currentfill}{rgb}{1.000000,0.662745,0.054902}%
\pgfsetfillcolor{currentfill}%
\pgfsetfillopacity{0.426033}%
\pgfsetlinewidth{1.003750pt}%
\definecolor{currentstroke}{rgb}{1.000000,0.662745,0.054902}%
\pgfsetstrokecolor{currentstroke}%
\pgfsetstrokeopacity{0.426033}%
\pgfsetdash{}{0pt}%
\pgfpathmoveto{\pgfqpoint{6.961040in}{1.574214in}}%
\pgfpathcurveto{\pgfqpoint{6.970248in}{1.574214in}}{\pgfqpoint{6.979081in}{1.577873in}}{\pgfqpoint{6.985592in}{1.584384in}}%
\pgfpathcurveto{\pgfqpoint{6.992104in}{1.590895in}}{\pgfqpoint{6.995762in}{1.599728in}}{\pgfqpoint{6.995762in}{1.608936in}}%
\pgfpathcurveto{\pgfqpoint{6.995762in}{1.618145in}}{\pgfqpoint{6.992104in}{1.626977in}}{\pgfqpoint{6.985592in}{1.633489in}}%
\pgfpathcurveto{\pgfqpoint{6.979081in}{1.640000in}}{\pgfqpoint{6.970248in}{1.643659in}}{\pgfqpoint{6.961040in}{1.643659in}}%
\pgfpathcurveto{\pgfqpoint{6.951832in}{1.643659in}}{\pgfqpoint{6.942999in}{1.640000in}}{\pgfqpoint{6.936488in}{1.633489in}}%
\pgfpathcurveto{\pgfqpoint{6.929976in}{1.626977in}}{\pgfqpoint{6.926318in}{1.618145in}}{\pgfqpoint{6.926318in}{1.608936in}}%
\pgfpathcurveto{\pgfqpoint{6.926318in}{1.599728in}}{\pgfqpoint{6.929976in}{1.590895in}}{\pgfqpoint{6.936488in}{1.584384in}}%
\pgfpathcurveto{\pgfqpoint{6.942999in}{1.577873in}}{\pgfqpoint{6.951832in}{1.574214in}}{\pgfqpoint{6.961040in}{1.574214in}}%
\pgfpathlineto{\pgfqpoint{6.961040in}{1.574214in}}%
\pgfpathclose%
\pgfusepath{stroke,fill}%
\end{pgfscope}%
\begin{pgfscope}%
\pgfpathrectangle{\pgfqpoint{1.374500in}{0.082500in}}{\pgfqpoint{2.419000in}{2.419000in}}%
\pgfusepath{clip}%
\pgfsetbuttcap%
\pgfsetroundjoin%
\definecolor{currentfill}{rgb}{1.000000,0.662745,0.054902}%
\pgfsetfillcolor{currentfill}%
\pgfsetfillopacity{0.431799}%
\pgfsetlinewidth{1.003750pt}%
\definecolor{currentstroke}{rgb}{1.000000,0.662745,0.054902}%
\pgfsetstrokecolor{currentstroke}%
\pgfsetstrokeopacity{0.431799}%
\pgfsetdash{}{0pt}%
\pgfpathmoveto{\pgfqpoint{0.479533in}{1.452830in}}%
\pgfpathcurveto{\pgfqpoint{0.488742in}{1.452830in}}{\pgfqpoint{0.497574in}{1.456488in}}{\pgfqpoint{0.504086in}{1.463000in}}%
\pgfpathcurveto{\pgfqpoint{0.510597in}{1.469511in}}{\pgfqpoint{0.514255in}{1.478344in}}{\pgfqpoint{0.514255in}{1.487552in}}%
\pgfpathcurveto{\pgfqpoint{0.514255in}{1.496761in}}{\pgfqpoint{0.510597in}{1.505593in}}{\pgfqpoint{0.504086in}{1.512104in}}%
\pgfpathcurveto{\pgfqpoint{0.497574in}{1.518616in}}{\pgfqpoint{0.488742in}{1.522274in}}{\pgfqpoint{0.479533in}{1.522274in}}%
\pgfpathcurveto{\pgfqpoint{0.470325in}{1.522274in}}{\pgfqpoint{0.461492in}{1.518616in}}{\pgfqpoint{0.454981in}{1.512104in}}%
\pgfpathcurveto{\pgfqpoint{0.448470in}{1.505593in}}{\pgfqpoint{0.444811in}{1.496761in}}{\pgfqpoint{0.444811in}{1.487552in}}%
\pgfpathcurveto{\pgfqpoint{0.444811in}{1.478344in}}{\pgfqpoint{0.448470in}{1.469511in}}{\pgfqpoint{0.454981in}{1.463000in}}%
\pgfpathcurveto{\pgfqpoint{0.461492in}{1.456488in}}{\pgfqpoint{0.470325in}{1.452830in}}{\pgfqpoint{0.479533in}{1.452830in}}%
\pgfpathlineto{\pgfqpoint{0.479533in}{1.452830in}}%
\pgfpathclose%
\pgfusepath{stroke,fill}%
\end{pgfscope}%
\begin{pgfscope}%
\pgfpathrectangle{\pgfqpoint{1.374500in}{0.082500in}}{\pgfqpoint{2.419000in}{2.419000in}}%
\pgfusepath{clip}%
\pgfsetbuttcap%
\pgfsetroundjoin%
\definecolor{currentfill}{rgb}{1.000000,0.662745,0.054902}%
\pgfsetfillcolor{currentfill}%
\pgfsetfillopacity{0.431799}%
\pgfsetlinewidth{1.003750pt}%
\definecolor{currentstroke}{rgb}{1.000000,0.662745,0.054902}%
\pgfsetstrokecolor{currentstroke}%
\pgfsetstrokeopacity{0.431799}%
\pgfsetdash{}{0pt}%
\pgfpathmoveto{\pgfqpoint{5.140221in}{1.452830in}}%
\pgfpathcurveto{\pgfqpoint{5.149430in}{1.452830in}}{\pgfqpoint{5.158262in}{1.456488in}}{\pgfqpoint{5.164774in}{1.463000in}}%
\pgfpathcurveto{\pgfqpoint{5.171285in}{1.469511in}}{\pgfqpoint{5.174944in}{1.478344in}}{\pgfqpoint{5.174944in}{1.487552in}}%
\pgfpathcurveto{\pgfqpoint{5.174944in}{1.496761in}}{\pgfqpoint{5.171285in}{1.505593in}}{\pgfqpoint{5.164774in}{1.512104in}}%
\pgfpathcurveto{\pgfqpoint{5.158262in}{1.518616in}}{\pgfqpoint{5.149430in}{1.522274in}}{\pgfqpoint{5.140221in}{1.522274in}}%
\pgfpathcurveto{\pgfqpoint{5.131013in}{1.522274in}}{\pgfqpoint{5.122180in}{1.518616in}}{\pgfqpoint{5.115669in}{1.512104in}}%
\pgfpathcurveto{\pgfqpoint{5.109158in}{1.505593in}}{\pgfqpoint{5.105499in}{1.496761in}}{\pgfqpoint{5.105499in}{1.487552in}}%
\pgfpathcurveto{\pgfqpoint{5.105499in}{1.478344in}}{\pgfqpoint{5.109158in}{1.469511in}}{\pgfqpoint{5.115669in}{1.463000in}}%
\pgfpathcurveto{\pgfqpoint{5.122180in}{1.456488in}}{\pgfqpoint{5.131013in}{1.452830in}}{\pgfqpoint{5.140221in}{1.452830in}}%
\pgfpathlineto{\pgfqpoint{5.140221in}{1.452830in}}%
\pgfpathclose%
\pgfusepath{stroke,fill}%
\end{pgfscope}%
\begin{pgfscope}%
\pgfpathrectangle{\pgfqpoint{1.374500in}{0.082500in}}{\pgfqpoint{2.419000in}{2.419000in}}%
\pgfusepath{clip}%
\pgfsetbuttcap%
\pgfsetroundjoin%
\definecolor{currentfill}{rgb}{1.000000,0.662745,0.054902}%
\pgfsetfillcolor{currentfill}%
\pgfsetfillopacity{0.431799}%
\pgfsetlinewidth{1.003750pt}%
\definecolor{currentstroke}{rgb}{1.000000,0.662745,0.054902}%
\pgfsetstrokecolor{currentstroke}%
\pgfsetstrokeopacity{0.431799}%
\pgfsetdash{}{0pt}%
\pgfpathmoveto{\pgfqpoint{9.800909in}{1.452830in}}%
\pgfpathcurveto{\pgfqpoint{9.810118in}{1.452830in}}{\pgfqpoint{9.818950in}{1.456488in}}{\pgfqpoint{9.825462in}{1.463000in}}%
\pgfpathcurveto{\pgfqpoint{9.831973in}{1.469511in}}{\pgfqpoint{9.835632in}{1.478344in}}{\pgfqpoint{9.835632in}{1.487552in}}%
\pgfpathcurveto{\pgfqpoint{9.835632in}{1.496761in}}{\pgfqpoint{9.831973in}{1.505593in}}{\pgfqpoint{9.825462in}{1.512104in}}%
\pgfpathcurveto{\pgfqpoint{9.818950in}{1.518616in}}{\pgfqpoint{9.810118in}{1.522274in}}{\pgfqpoint{9.800909in}{1.522274in}}%
\pgfpathcurveto{\pgfqpoint{9.791701in}{1.522274in}}{\pgfqpoint{9.782868in}{1.518616in}}{\pgfqpoint{9.776357in}{1.512104in}}%
\pgfpathcurveto{\pgfqpoint{9.769846in}{1.505593in}}{\pgfqpoint{9.766187in}{1.496761in}}{\pgfqpoint{9.766187in}{1.487552in}}%
\pgfpathcurveto{\pgfqpoint{9.766187in}{1.478344in}}{\pgfqpoint{9.769846in}{1.469511in}}{\pgfqpoint{9.776357in}{1.463000in}}%
\pgfpathcurveto{\pgfqpoint{9.782868in}{1.456488in}}{\pgfqpoint{9.791701in}{1.452830in}}{\pgfqpoint{9.800909in}{1.452830in}}%
\pgfpathlineto{\pgfqpoint{9.800909in}{1.452830in}}%
\pgfpathclose%
\pgfusepath{stroke,fill}%
\end{pgfscope}%
\begin{pgfscope}%
\pgfpathrectangle{\pgfqpoint{1.374500in}{0.082500in}}{\pgfqpoint{2.419000in}{2.419000in}}%
\pgfusepath{clip}%
\pgfsetbuttcap%
\pgfsetroundjoin%
\definecolor{currentfill}{rgb}{1.000000,0.662745,0.054902}%
\pgfsetfillcolor{currentfill}%
\pgfsetfillopacity{0.437762}%
\pgfsetlinewidth{1.003750pt}%
\definecolor{currentstroke}{rgb}{1.000000,0.662745,0.054902}%
\pgfsetstrokecolor{currentstroke}%
\pgfsetstrokeopacity{0.437762}%
\pgfsetdash{}{0pt}%
\pgfpathmoveto{\pgfqpoint{-1.482359in}{1.327320in}}%
\pgfpathcurveto{\pgfqpoint{-1.473150in}{1.327320in}}{\pgfqpoint{-1.464318in}{1.330979in}}{\pgfqpoint{-1.457806in}{1.337490in}}%
\pgfpathcurveto{\pgfqpoint{-1.451295in}{1.344002in}}{\pgfqpoint{-1.447636in}{1.352834in}}{\pgfqpoint{-1.447636in}{1.362043in}}%
\pgfpathcurveto{\pgfqpoint{-1.447636in}{1.371251in}}{\pgfqpoint{-1.451295in}{1.380084in}}{\pgfqpoint{-1.457806in}{1.386595in}}%
\pgfpathcurveto{\pgfqpoint{-1.464318in}{1.393106in}}{\pgfqpoint{-1.473150in}{1.396765in}}{\pgfqpoint{-1.482359in}{1.396765in}}%
\pgfpathcurveto{\pgfqpoint{-1.491567in}{1.396765in}}{\pgfqpoint{-1.500400in}{1.393106in}}{\pgfqpoint{-1.506911in}{1.386595in}}%
\pgfpathcurveto{\pgfqpoint{-1.513422in}{1.380084in}}{\pgfqpoint{-1.517081in}{1.371251in}}{\pgfqpoint{-1.517081in}{1.362043in}}%
\pgfpathcurveto{\pgfqpoint{-1.517081in}{1.352834in}}{\pgfqpoint{-1.513422in}{1.344002in}}{\pgfqpoint{-1.506911in}{1.337490in}}%
\pgfpathcurveto{\pgfqpoint{-1.500400in}{1.330979in}}{\pgfqpoint{-1.491567in}{1.327320in}}{\pgfqpoint{-1.482359in}{1.327320in}}%
\pgfpathlineto{\pgfqpoint{-1.482359in}{1.327320in}}%
\pgfpathclose%
\pgfusepath{stroke,fill}%
\end{pgfscope}%
\begin{pgfscope}%
\pgfpathrectangle{\pgfqpoint{1.374500in}{0.082500in}}{\pgfqpoint{2.419000in}{2.419000in}}%
\pgfusepath{clip}%
\pgfsetbuttcap%
\pgfsetroundjoin%
\definecolor{currentfill}{rgb}{1.000000,0.662745,0.054902}%
\pgfsetfillcolor{currentfill}%
\pgfsetfillopacity{0.437762}%
\pgfsetlinewidth{1.003750pt}%
\definecolor{currentstroke}{rgb}{1.000000,0.662745,0.054902}%
\pgfsetstrokecolor{currentstroke}%
\pgfsetstrokeopacity{0.437762}%
\pgfsetdash{}{0pt}%
\pgfpathmoveto{\pgfqpoint{3.257524in}{1.327320in}}%
\pgfpathcurveto{\pgfqpoint{3.266732in}{1.327320in}}{\pgfqpoint{3.275565in}{1.330979in}}{\pgfqpoint{3.282076in}{1.337490in}}%
\pgfpathcurveto{\pgfqpoint{3.288588in}{1.344002in}}{\pgfqpoint{3.292246in}{1.352834in}}{\pgfqpoint{3.292246in}{1.362043in}}%
\pgfpathcurveto{\pgfqpoint{3.292246in}{1.371251in}}{\pgfqpoint{3.288588in}{1.380084in}}{\pgfqpoint{3.282076in}{1.386595in}}%
\pgfpathcurveto{\pgfqpoint{3.275565in}{1.393106in}}{\pgfqpoint{3.266732in}{1.396765in}}{\pgfqpoint{3.257524in}{1.396765in}}%
\pgfpathcurveto{\pgfqpoint{3.248315in}{1.396765in}}{\pgfqpoint{3.239483in}{1.393106in}}{\pgfqpoint{3.232972in}{1.386595in}}%
\pgfpathcurveto{\pgfqpoint{3.226460in}{1.380084in}}{\pgfqpoint{3.222802in}{1.371251in}}{\pgfqpoint{3.222802in}{1.362043in}}%
\pgfpathcurveto{\pgfqpoint{3.222802in}{1.352834in}}{\pgfqpoint{3.226460in}{1.344002in}}{\pgfqpoint{3.232972in}{1.337490in}}%
\pgfpathcurveto{\pgfqpoint{3.239483in}{1.330979in}}{\pgfqpoint{3.248315in}{1.327320in}}{\pgfqpoint{3.257524in}{1.327320in}}%
\pgfpathlineto{\pgfqpoint{3.257524in}{1.327320in}}%
\pgfpathclose%
\pgfusepath{stroke,fill}%
\end{pgfscope}%
\begin{pgfscope}%
\pgfpathrectangle{\pgfqpoint{1.374500in}{0.082500in}}{\pgfqpoint{2.419000in}{2.419000in}}%
\pgfusepath{clip}%
\pgfsetbuttcap%
\pgfsetroundjoin%
\definecolor{currentfill}{rgb}{1.000000,0.662745,0.054902}%
\pgfsetfillcolor{currentfill}%
\pgfsetfillopacity{0.437762}%
\pgfsetlinewidth{1.003750pt}%
\definecolor{currentstroke}{rgb}{1.000000,0.662745,0.054902}%
\pgfsetstrokecolor{currentstroke}%
\pgfsetstrokeopacity{0.437762}%
\pgfsetdash{}{0pt}%
\pgfpathmoveto{\pgfqpoint{7.997406in}{1.327320in}}%
\pgfpathcurveto{\pgfqpoint{8.006615in}{1.327320in}}{\pgfqpoint{8.015447in}{1.330979in}}{\pgfqpoint{8.021959in}{1.337490in}}%
\pgfpathcurveto{\pgfqpoint{8.028470in}{1.344002in}}{\pgfqpoint{8.032129in}{1.352834in}}{\pgfqpoint{8.032129in}{1.362043in}}%
\pgfpathcurveto{\pgfqpoint{8.032129in}{1.371251in}}{\pgfqpoint{8.028470in}{1.380084in}}{\pgfqpoint{8.021959in}{1.386595in}}%
\pgfpathcurveto{\pgfqpoint{8.015447in}{1.393106in}}{\pgfqpoint{8.006615in}{1.396765in}}{\pgfqpoint{7.997406in}{1.396765in}}%
\pgfpathcurveto{\pgfqpoint{7.988198in}{1.396765in}}{\pgfqpoint{7.979365in}{1.393106in}}{\pgfqpoint{7.972854in}{1.386595in}}%
\pgfpathcurveto{\pgfqpoint{7.966343in}{1.380084in}}{\pgfqpoint{7.962684in}{1.371251in}}{\pgfqpoint{7.962684in}{1.362043in}}%
\pgfpathcurveto{\pgfqpoint{7.962684in}{1.352834in}}{\pgfqpoint{7.966343in}{1.344002in}}{\pgfqpoint{7.972854in}{1.337490in}}%
\pgfpathcurveto{\pgfqpoint{7.979365in}{1.330979in}}{\pgfqpoint{7.988198in}{1.327320in}}{\pgfqpoint{7.997406in}{1.327320in}}%
\pgfpathlineto{\pgfqpoint{7.997406in}{1.327320in}}%
\pgfpathclose%
\pgfusepath{stroke,fill}%
\end{pgfscope}%
\begin{pgfscope}%
\pgfpathrectangle{\pgfqpoint{1.374500in}{0.082500in}}{\pgfqpoint{2.419000in}{2.419000in}}%
\pgfusepath{clip}%
\pgfsetbuttcap%
\pgfsetroundjoin%
\definecolor{currentfill}{rgb}{1.000000,0.662745,0.054902}%
\pgfsetfillcolor{currentfill}%
\pgfsetfillopacity{0.443931}%
\pgfsetlinewidth{1.003750pt}%
\definecolor{currentstroke}{rgb}{1.000000,0.662745,0.054902}%
\pgfsetstrokecolor{currentstroke}%
\pgfsetstrokeopacity{0.443931}%
\pgfsetdash{}{0pt}%
\pgfpathmoveto{\pgfqpoint{10.953369in}{1.197472in}}%
\pgfpathcurveto{\pgfqpoint{10.962577in}{1.197472in}}{\pgfqpoint{10.971410in}{1.201130in}}{\pgfqpoint{10.977921in}{1.207642in}}%
\pgfpathcurveto{\pgfqpoint{10.984432in}{1.214153in}}{\pgfqpoint{10.988091in}{1.222986in}}{\pgfqpoint{10.988091in}{1.232194in}}%
\pgfpathcurveto{\pgfqpoint{10.988091in}{1.241403in}}{\pgfqpoint{10.984432in}{1.250235in}}{\pgfqpoint{10.977921in}{1.256746in}}%
\pgfpathcurveto{\pgfqpoint{10.971410in}{1.263258in}}{\pgfqpoint{10.962577in}{1.266916in}}{\pgfqpoint{10.953369in}{1.266916in}}%
\pgfpathcurveto{\pgfqpoint{10.944160in}{1.266916in}}{\pgfqpoint{10.935328in}{1.263258in}}{\pgfqpoint{10.928816in}{1.256746in}}%
\pgfpathcurveto{\pgfqpoint{10.922305in}{1.250235in}}{\pgfqpoint{10.918646in}{1.241403in}}{\pgfqpoint{10.918646in}{1.232194in}}%
\pgfpathcurveto{\pgfqpoint{10.918646in}{1.222986in}}{\pgfqpoint{10.922305in}{1.214153in}}{\pgfqpoint{10.928816in}{1.207642in}}%
\pgfpathcurveto{\pgfqpoint{10.935328in}{1.201130in}}{\pgfqpoint{10.944160in}{1.197472in}}{\pgfqpoint{10.953369in}{1.197472in}}%
\pgfpathlineto{\pgfqpoint{10.953369in}{1.197472in}}%
\pgfpathclose%
\pgfusepath{stroke,fill}%
\end{pgfscope}%
\begin{pgfscope}%
\pgfpathrectangle{\pgfqpoint{1.374500in}{0.082500in}}{\pgfqpoint{2.419000in}{2.419000in}}%
\pgfusepath{clip}%
\pgfsetbuttcap%
\pgfsetroundjoin%
\definecolor{currentfill}{rgb}{1.000000,0.662745,0.054902}%
\pgfsetfillcolor{currentfill}%
\pgfsetfillopacity{0.443931}%
\pgfsetlinewidth{1.003750pt}%
\definecolor{currentstroke}{rgb}{1.000000,0.662745,0.054902}%
\pgfsetstrokecolor{currentstroke}%
\pgfsetstrokeopacity{0.443931}%
\pgfsetdash{}{0pt}%
\pgfpathmoveto{\pgfqpoint{6.131554in}{1.197472in}}%
\pgfpathcurveto{\pgfqpoint{6.140762in}{1.197472in}}{\pgfqpoint{6.149595in}{1.201130in}}{\pgfqpoint{6.156106in}{1.207642in}}%
\pgfpathcurveto{\pgfqpoint{6.162617in}{1.214153in}}{\pgfqpoint{6.166276in}{1.222986in}}{\pgfqpoint{6.166276in}{1.232194in}}%
\pgfpathcurveto{\pgfqpoint{6.166276in}{1.241403in}}{\pgfqpoint{6.162617in}{1.250235in}}{\pgfqpoint{6.156106in}{1.256746in}}%
\pgfpathcurveto{\pgfqpoint{6.149595in}{1.263258in}}{\pgfqpoint{6.140762in}{1.266916in}}{\pgfqpoint{6.131554in}{1.266916in}}%
\pgfpathcurveto{\pgfqpoint{6.122345in}{1.266916in}}{\pgfqpoint{6.113513in}{1.263258in}}{\pgfqpoint{6.107001in}{1.256746in}}%
\pgfpathcurveto{\pgfqpoint{6.100490in}{1.250235in}}{\pgfqpoint{6.096832in}{1.241403in}}{\pgfqpoint{6.096832in}{1.232194in}}%
\pgfpathcurveto{\pgfqpoint{6.096832in}{1.222986in}}{\pgfqpoint{6.100490in}{1.214153in}}{\pgfqpoint{6.107001in}{1.207642in}}%
\pgfpathcurveto{\pgfqpoint{6.113513in}{1.201130in}}{\pgfqpoint{6.122345in}{1.197472in}}{\pgfqpoint{6.131554in}{1.197472in}}%
\pgfpathlineto{\pgfqpoint{6.131554in}{1.197472in}}%
\pgfpathclose%
\pgfusepath{stroke,fill}%
\end{pgfscope}%
\begin{pgfscope}%
\pgfpathrectangle{\pgfqpoint{1.374500in}{0.082500in}}{\pgfqpoint{2.419000in}{2.419000in}}%
\pgfusepath{clip}%
\pgfsetbuttcap%
\pgfsetroundjoin%
\definecolor{currentfill}{rgb}{1.000000,0.662745,0.054902}%
\pgfsetfillcolor{currentfill}%
\pgfsetfillopacity{0.443931}%
\pgfsetlinewidth{1.003750pt}%
\definecolor{currentstroke}{rgb}{1.000000,0.662745,0.054902}%
\pgfsetstrokecolor{currentstroke}%
\pgfsetstrokeopacity{0.443931}%
\pgfsetdash{}{0pt}%
\pgfpathmoveto{\pgfqpoint{1.309739in}{1.197472in}}%
\pgfpathcurveto{\pgfqpoint{1.318947in}{1.197472in}}{\pgfqpoint{1.327780in}{1.201130in}}{\pgfqpoint{1.334291in}{1.207642in}}%
\pgfpathcurveto{\pgfqpoint{1.340803in}{1.214153in}}{\pgfqpoint{1.344461in}{1.222986in}}{\pgfqpoint{1.344461in}{1.232194in}}%
\pgfpathcurveto{\pgfqpoint{1.344461in}{1.241403in}}{\pgfqpoint{1.340803in}{1.250235in}}{\pgfqpoint{1.334291in}{1.256746in}}%
\pgfpathcurveto{\pgfqpoint{1.327780in}{1.263258in}}{\pgfqpoint{1.318947in}{1.266916in}}{\pgfqpoint{1.309739in}{1.266916in}}%
\pgfpathcurveto{\pgfqpoint{1.300530in}{1.266916in}}{\pgfqpoint{1.291698in}{1.263258in}}{\pgfqpoint{1.285187in}{1.256746in}}%
\pgfpathcurveto{\pgfqpoint{1.278675in}{1.250235in}}{\pgfqpoint{1.275017in}{1.241403in}}{\pgfqpoint{1.275017in}{1.232194in}}%
\pgfpathcurveto{\pgfqpoint{1.275017in}{1.222986in}}{\pgfqpoint{1.278675in}{1.214153in}}{\pgfqpoint{1.285187in}{1.207642in}}%
\pgfpathcurveto{\pgfqpoint{1.291698in}{1.201130in}}{\pgfqpoint{1.300530in}{1.197472in}}{\pgfqpoint{1.309739in}{1.197472in}}%
\pgfpathlineto{\pgfqpoint{1.309739in}{1.197472in}}%
\pgfpathclose%
\pgfusepath{stroke,fill}%
\end{pgfscope}%
\begin{pgfscope}%
\pgfpathrectangle{\pgfqpoint{1.374500in}{0.082500in}}{\pgfqpoint{2.419000in}{2.419000in}}%
\pgfusepath{clip}%
\pgfsetbuttcap%
\pgfsetroundjoin%
\definecolor{currentfill}{rgb}{1.000000,0.662745,0.054902}%
\pgfsetfillcolor{currentfill}%
\pgfsetfillopacity{0.450317}%
\pgfsetlinewidth{1.003750pt}%
\definecolor{currentstroke}{rgb}{1.000000,0.662745,0.054902}%
\pgfsetstrokecolor{currentstroke}%
\pgfsetstrokeopacity{0.450317}%
\pgfsetdash{}{0pt}%
\pgfpathmoveto{\pgfqpoint{-0.706568in}{1.063055in}}%
\pgfpathcurveto{\pgfqpoint{-0.697360in}{1.063055in}}{\pgfqpoint{-0.688527in}{1.066714in}}{\pgfqpoint{-0.682016in}{1.073225in}}%
\pgfpathcurveto{\pgfqpoint{-0.675505in}{1.079737in}}{\pgfqpoint{-0.671846in}{1.088569in}}{\pgfqpoint{-0.671846in}{1.097778in}}%
\pgfpathcurveto{\pgfqpoint{-0.671846in}{1.106986in}}{\pgfqpoint{-0.675505in}{1.115819in}}{\pgfqpoint{-0.682016in}{1.122330in}}%
\pgfpathcurveto{\pgfqpoint{-0.688527in}{1.128841in}}{\pgfqpoint{-0.697360in}{1.132500in}}{\pgfqpoint{-0.706568in}{1.132500in}}%
\pgfpathcurveto{\pgfqpoint{-0.715777in}{1.132500in}}{\pgfqpoint{-0.724609in}{1.128841in}}{\pgfqpoint{-0.731121in}{1.122330in}}%
\pgfpathcurveto{\pgfqpoint{-0.737632in}{1.115819in}}{\pgfqpoint{-0.741291in}{1.106986in}}{\pgfqpoint{-0.741291in}{1.097778in}}%
\pgfpathcurveto{\pgfqpoint{-0.741291in}{1.088569in}}{\pgfqpoint{-0.737632in}{1.079737in}}{\pgfqpoint{-0.731121in}{1.073225in}}%
\pgfpathcurveto{\pgfqpoint{-0.724609in}{1.066714in}}{\pgfqpoint{-0.715777in}{1.063055in}}{\pgfqpoint{-0.706568in}{1.063055in}}%
\pgfpathlineto{\pgfqpoint{-0.706568in}{1.063055in}}%
\pgfpathclose%
\pgfusepath{stroke,fill}%
\end{pgfscope}%
\begin{pgfscope}%
\pgfpathrectangle{\pgfqpoint{1.374500in}{0.082500in}}{\pgfqpoint{2.419000in}{2.419000in}}%
\pgfusepath{clip}%
\pgfsetbuttcap%
\pgfsetroundjoin%
\definecolor{currentfill}{rgb}{1.000000,0.662745,0.054902}%
\pgfsetfillcolor{currentfill}%
\pgfsetfillopacity{0.450317}%
\pgfsetlinewidth{1.003750pt}%
\definecolor{currentstroke}{rgb}{1.000000,0.662745,0.054902}%
\pgfsetstrokecolor{currentstroke}%
\pgfsetstrokeopacity{0.450317}%
\pgfsetdash{}{0pt}%
\pgfpathmoveto{\pgfqpoint{4.200061in}{1.063055in}}%
\pgfpathcurveto{\pgfqpoint{4.209270in}{1.063055in}}{\pgfqpoint{4.218102in}{1.066714in}}{\pgfqpoint{4.224614in}{1.073225in}}%
\pgfpathcurveto{\pgfqpoint{4.231125in}{1.079737in}}{\pgfqpoint{4.234783in}{1.088569in}}{\pgfqpoint{4.234783in}{1.097778in}}%
\pgfpathcurveto{\pgfqpoint{4.234783in}{1.106986in}}{\pgfqpoint{4.231125in}{1.115819in}}{\pgfqpoint{4.224614in}{1.122330in}}%
\pgfpathcurveto{\pgfqpoint{4.218102in}{1.128841in}}{\pgfqpoint{4.209270in}{1.132500in}}{\pgfqpoint{4.200061in}{1.132500in}}%
\pgfpathcurveto{\pgfqpoint{4.190853in}{1.132500in}}{\pgfqpoint{4.182020in}{1.128841in}}{\pgfqpoint{4.175509in}{1.122330in}}%
\pgfpathcurveto{\pgfqpoint{4.168998in}{1.115819in}}{\pgfqpoint{4.165339in}{1.106986in}}{\pgfqpoint{4.165339in}{1.097778in}}%
\pgfpathcurveto{\pgfqpoint{4.165339in}{1.088569in}}{\pgfqpoint{4.168998in}{1.079737in}}{\pgfqpoint{4.175509in}{1.073225in}}%
\pgfpathcurveto{\pgfqpoint{4.182020in}{1.066714in}}{\pgfqpoint{4.190853in}{1.063055in}}{\pgfqpoint{4.200061in}{1.063055in}}%
\pgfpathlineto{\pgfqpoint{4.200061in}{1.063055in}}%
\pgfpathclose%
\pgfusepath{stroke,fill}%
\end{pgfscope}%
\begin{pgfscope}%
\pgfpathrectangle{\pgfqpoint{1.374500in}{0.082500in}}{\pgfqpoint{2.419000in}{2.419000in}}%
\pgfusepath{clip}%
\pgfsetbuttcap%
\pgfsetroundjoin%
\definecolor{currentfill}{rgb}{1.000000,0.662745,0.054902}%
\pgfsetfillcolor{currentfill}%
\pgfsetfillopacity{0.450317}%
\pgfsetlinewidth{1.003750pt}%
\definecolor{currentstroke}{rgb}{1.000000,0.662745,0.054902}%
\pgfsetstrokecolor{currentstroke}%
\pgfsetstrokeopacity{0.450317}%
\pgfsetdash{}{0pt}%
\pgfpathmoveto{\pgfqpoint{9.106691in}{1.063055in}}%
\pgfpathcurveto{\pgfqpoint{9.115899in}{1.063055in}}{\pgfqpoint{9.124732in}{1.066714in}}{\pgfqpoint{9.131243in}{1.073225in}}%
\pgfpathcurveto{\pgfqpoint{9.137754in}{1.079737in}}{\pgfqpoint{9.141413in}{1.088569in}}{\pgfqpoint{9.141413in}{1.097778in}}%
\pgfpathcurveto{\pgfqpoint{9.141413in}{1.106986in}}{\pgfqpoint{9.137754in}{1.115819in}}{\pgfqpoint{9.131243in}{1.122330in}}%
\pgfpathcurveto{\pgfqpoint{9.124732in}{1.128841in}}{\pgfqpoint{9.115899in}{1.132500in}}{\pgfqpoint{9.106691in}{1.132500in}}%
\pgfpathcurveto{\pgfqpoint{9.097482in}{1.132500in}}{\pgfqpoint{9.088650in}{1.128841in}}{\pgfqpoint{9.082138in}{1.122330in}}%
\pgfpathcurveto{\pgfqpoint{9.075627in}{1.115819in}}{\pgfqpoint{9.071969in}{1.106986in}}{\pgfqpoint{9.071969in}{1.097778in}}%
\pgfpathcurveto{\pgfqpoint{9.071969in}{1.088569in}}{\pgfqpoint{9.075627in}{1.079737in}}{\pgfqpoint{9.082138in}{1.073225in}}%
\pgfpathcurveto{\pgfqpoint{9.088650in}{1.066714in}}{\pgfqpoint{9.097482in}{1.063055in}}{\pgfqpoint{9.106691in}{1.063055in}}%
\pgfpathlineto{\pgfqpoint{9.106691in}{1.063055in}}%
\pgfpathclose%
\pgfusepath{stroke,fill}%
\end{pgfscope}%
\begin{pgfscope}%
\pgfpathrectangle{\pgfqpoint{1.374500in}{0.082500in}}{\pgfqpoint{2.419000in}{2.419000in}}%
\pgfusepath{clip}%
\pgfsetbuttcap%
\pgfsetroundjoin%
\definecolor{currentfill}{rgb}{1.000000,0.662745,0.054902}%
\pgfsetfillcolor{currentfill}%
\pgfsetfillopacity{0.456932}%
\pgfsetlinewidth{1.003750pt}%
\definecolor{currentstroke}{rgb}{1.000000,0.662745,0.054902}%
\pgfsetstrokecolor{currentstroke}%
\pgfsetstrokeopacity{0.456932}%
\pgfsetdash{}{0pt}%
\pgfpathmoveto{\pgfqpoint{-2.795078in}{0.923825in}}%
\pgfpathcurveto{\pgfqpoint{-2.785870in}{0.923825in}}{\pgfqpoint{-2.777037in}{0.927484in}}{\pgfqpoint{-2.770526in}{0.933995in}}%
\pgfpathcurveto{\pgfqpoint{-2.764015in}{0.940507in}}{\pgfqpoint{-2.760356in}{0.949339in}}{\pgfqpoint{-2.760356in}{0.958548in}}%
\pgfpathcurveto{\pgfqpoint{-2.760356in}{0.967756in}}{\pgfqpoint{-2.764015in}{0.976589in}}{\pgfqpoint{-2.770526in}{0.983100in}}%
\pgfpathcurveto{\pgfqpoint{-2.777037in}{0.989611in}}{\pgfqpoint{-2.785870in}{0.993270in}}{\pgfqpoint{-2.795078in}{0.993270in}}%
\pgfpathcurveto{\pgfqpoint{-2.804287in}{0.993270in}}{\pgfqpoint{-2.813119in}{0.989611in}}{\pgfqpoint{-2.819631in}{0.983100in}}%
\pgfpathcurveto{\pgfqpoint{-2.826142in}{0.976589in}}{\pgfqpoint{-2.829801in}{0.967756in}}{\pgfqpoint{-2.829801in}{0.958548in}}%
\pgfpathcurveto{\pgfqpoint{-2.829801in}{0.949339in}}{\pgfqpoint{-2.826142in}{0.940507in}}{\pgfqpoint{-2.819631in}{0.933995in}}%
\pgfpathcurveto{\pgfqpoint{-2.813119in}{0.927484in}}{\pgfqpoint{-2.804287in}{0.923825in}}{\pgfqpoint{-2.795078in}{0.923825in}}%
\pgfpathlineto{\pgfqpoint{-2.795078in}{0.923825in}}%
\pgfpathclose%
\pgfusepath{stroke,fill}%
\end{pgfscope}%
\begin{pgfscope}%
\pgfpathrectangle{\pgfqpoint{1.374500in}{0.082500in}}{\pgfqpoint{2.419000in}{2.419000in}}%
\pgfusepath{clip}%
\pgfsetbuttcap%
\pgfsetroundjoin%
\definecolor{currentfill}{rgb}{1.000000,0.662745,0.054902}%
\pgfsetfillcolor{currentfill}%
\pgfsetfillopacity{0.456932}%
\pgfsetlinewidth{1.003750pt}%
\definecolor{currentstroke}{rgb}{1.000000,0.662745,0.054902}%
\pgfsetstrokecolor{currentstroke}%
\pgfsetstrokeopacity{0.456932}%
\pgfsetdash{}{0pt}%
\pgfpathmoveto{\pgfqpoint{2.199403in}{0.923825in}}%
\pgfpathcurveto{\pgfqpoint{2.208611in}{0.923825in}}{\pgfqpoint{2.217444in}{0.927484in}}{\pgfqpoint{2.223955in}{0.933995in}}%
\pgfpathcurveto{\pgfqpoint{2.230467in}{0.940507in}}{\pgfqpoint{2.234125in}{0.949339in}}{\pgfqpoint{2.234125in}{0.958548in}}%
\pgfpathcurveto{\pgfqpoint{2.234125in}{0.967756in}}{\pgfqpoint{2.230467in}{0.976589in}}{\pgfqpoint{2.223955in}{0.983100in}}%
\pgfpathcurveto{\pgfqpoint{2.217444in}{0.989611in}}{\pgfqpoint{2.208611in}{0.993270in}}{\pgfqpoint{2.199403in}{0.993270in}}%
\pgfpathcurveto{\pgfqpoint{2.190195in}{0.993270in}}{\pgfqpoint{2.181362in}{0.989611in}}{\pgfqpoint{2.174851in}{0.983100in}}%
\pgfpathcurveto{\pgfqpoint{2.168339in}{0.976589in}}{\pgfqpoint{2.164681in}{0.967756in}}{\pgfqpoint{2.164681in}{0.958548in}}%
\pgfpathcurveto{\pgfqpoint{2.164681in}{0.949339in}}{\pgfqpoint{2.168339in}{0.940507in}}{\pgfqpoint{2.174851in}{0.933995in}}%
\pgfpathcurveto{\pgfqpoint{2.181362in}{0.927484in}}{\pgfqpoint{2.190195in}{0.923825in}}{\pgfqpoint{2.199403in}{0.923825in}}%
\pgfpathlineto{\pgfqpoint{2.199403in}{0.923825in}}%
\pgfpathclose%
\pgfusepath{stroke,fill}%
\end{pgfscope}%
\begin{pgfscope}%
\pgfpathrectangle{\pgfqpoint{1.374500in}{0.082500in}}{\pgfqpoint{2.419000in}{2.419000in}}%
\pgfusepath{clip}%
\pgfsetbuttcap%
\pgfsetroundjoin%
\definecolor{currentfill}{rgb}{1.000000,0.662745,0.054902}%
\pgfsetfillcolor{currentfill}%
\pgfsetfillopacity{0.456932}%
\pgfsetlinewidth{1.003750pt}%
\definecolor{currentstroke}{rgb}{1.000000,0.662745,0.054902}%
\pgfsetstrokecolor{currentstroke}%
\pgfsetstrokeopacity{0.456932}%
\pgfsetdash{}{0pt}%
\pgfpathmoveto{\pgfqpoint{7.193884in}{0.923825in}}%
\pgfpathcurveto{\pgfqpoint{7.203093in}{0.923825in}}{\pgfqpoint{7.211925in}{0.927484in}}{\pgfqpoint{7.218437in}{0.933995in}}%
\pgfpathcurveto{\pgfqpoint{7.224948in}{0.940507in}}{\pgfqpoint{7.228607in}{0.949339in}}{\pgfqpoint{7.228607in}{0.958548in}}%
\pgfpathcurveto{\pgfqpoint{7.228607in}{0.967756in}}{\pgfqpoint{7.224948in}{0.976589in}}{\pgfqpoint{7.218437in}{0.983100in}}%
\pgfpathcurveto{\pgfqpoint{7.211925in}{0.989611in}}{\pgfqpoint{7.203093in}{0.993270in}}{\pgfqpoint{7.193884in}{0.993270in}}%
\pgfpathcurveto{\pgfqpoint{7.184676in}{0.993270in}}{\pgfqpoint{7.175843in}{0.989611in}}{\pgfqpoint{7.169332in}{0.983100in}}%
\pgfpathcurveto{\pgfqpoint{7.162821in}{0.976589in}}{\pgfqpoint{7.159162in}{0.967756in}}{\pgfqpoint{7.159162in}{0.958548in}}%
\pgfpathcurveto{\pgfqpoint{7.159162in}{0.949339in}}{\pgfqpoint{7.162821in}{0.940507in}}{\pgfqpoint{7.169332in}{0.933995in}}%
\pgfpathcurveto{\pgfqpoint{7.175843in}{0.927484in}}{\pgfqpoint{7.184676in}{0.923825in}}{\pgfqpoint{7.193884in}{0.923825in}}%
\pgfpathlineto{\pgfqpoint{7.193884in}{0.923825in}}%
\pgfpathclose%
\pgfusepath{stroke,fill}%
\end{pgfscope}%
\begin{pgfscope}%
\pgfpathrectangle{\pgfqpoint{1.374500in}{0.082500in}}{\pgfqpoint{2.419000in}{2.419000in}}%
\pgfusepath{clip}%
\pgfsetbuttcap%
\pgfsetroundjoin%
\definecolor{currentfill}{rgb}{1.000000,0.662745,0.054902}%
\pgfsetfillcolor{currentfill}%
\pgfsetfillopacity{0.463788}%
\pgfsetlinewidth{1.003750pt}%
\definecolor{currentstroke}{rgb}{1.000000,0.662745,0.054902}%
\pgfsetstrokecolor{currentstroke}%
\pgfsetstrokeopacity{0.463788}%
\pgfsetdash{}{0pt}%
\pgfpathmoveto{\pgfqpoint{0.125796in}{0.779519in}}%
\pgfpathcurveto{\pgfqpoint{0.135005in}{0.779519in}}{\pgfqpoint{0.143837in}{0.783177in}}{\pgfqpoint{0.150349in}{0.789689in}}%
\pgfpathcurveto{\pgfqpoint{0.156860in}{0.796200in}}{\pgfqpoint{0.160518in}{0.805033in}}{\pgfqpoint{0.160518in}{0.814241in}}%
\pgfpathcurveto{\pgfqpoint{0.160518in}{0.823450in}}{\pgfqpoint{0.156860in}{0.832282in}}{\pgfqpoint{0.150349in}{0.838793in}}%
\pgfpathcurveto{\pgfqpoint{0.143837in}{0.845305in}}{\pgfqpoint{0.135005in}{0.848963in}}{\pgfqpoint{0.125796in}{0.848963in}}%
\pgfpathcurveto{\pgfqpoint{0.116588in}{0.848963in}}{\pgfqpoint{0.107755in}{0.845305in}}{\pgfqpoint{0.101244in}{0.838793in}}%
\pgfpathcurveto{\pgfqpoint{0.094733in}{0.832282in}}{\pgfqpoint{0.091074in}{0.823450in}}{\pgfqpoint{0.091074in}{0.814241in}}%
\pgfpathcurveto{\pgfqpoint{0.091074in}{0.805033in}}{\pgfqpoint{0.094733in}{0.796200in}}{\pgfqpoint{0.101244in}{0.789689in}}%
\pgfpathcurveto{\pgfqpoint{0.107755in}{0.783177in}}{\pgfqpoint{0.116588in}{0.779519in}}{\pgfqpoint{0.125796in}{0.779519in}}%
\pgfpathlineto{\pgfqpoint{0.125796in}{0.779519in}}%
\pgfpathclose%
\pgfusepath{stroke,fill}%
\end{pgfscope}%
\begin{pgfscope}%
\pgfpathrectangle{\pgfqpoint{1.374500in}{0.082500in}}{\pgfqpoint{2.419000in}{2.419000in}}%
\pgfusepath{clip}%
\pgfsetbuttcap%
\pgfsetroundjoin%
\definecolor{currentfill}{rgb}{1.000000,0.662745,0.054902}%
\pgfsetfillcolor{currentfill}%
\pgfsetfillopacity{0.463788}%
\pgfsetlinewidth{1.003750pt}%
\definecolor{currentstroke}{rgb}{1.000000,0.662745,0.054902}%
\pgfsetstrokecolor{currentstroke}%
\pgfsetstrokeopacity{0.463788}%
\pgfsetdash{}{0pt}%
\pgfpathmoveto{\pgfqpoint{10.296869in}{0.779519in}}%
\pgfpathcurveto{\pgfqpoint{10.306078in}{0.779519in}}{\pgfqpoint{10.314910in}{0.783177in}}{\pgfqpoint{10.321421in}{0.789689in}}%
\pgfpathcurveto{\pgfqpoint{10.327933in}{0.796200in}}{\pgfqpoint{10.331591in}{0.805033in}}{\pgfqpoint{10.331591in}{0.814241in}}%
\pgfpathcurveto{\pgfqpoint{10.331591in}{0.823450in}}{\pgfqpoint{10.327933in}{0.832282in}}{\pgfqpoint{10.321421in}{0.838793in}}%
\pgfpathcurveto{\pgfqpoint{10.314910in}{0.845305in}}{\pgfqpoint{10.306078in}{0.848963in}}{\pgfqpoint{10.296869in}{0.848963in}}%
\pgfpathcurveto{\pgfqpoint{10.287661in}{0.848963in}}{\pgfqpoint{10.278828in}{0.845305in}}{\pgfqpoint{10.272317in}{0.838793in}}%
\pgfpathcurveto{\pgfqpoint{10.265805in}{0.832282in}}{\pgfqpoint{10.262147in}{0.823450in}}{\pgfqpoint{10.262147in}{0.814241in}}%
\pgfpathcurveto{\pgfqpoint{10.262147in}{0.805033in}}{\pgfqpoint{10.265805in}{0.796200in}}{\pgfqpoint{10.272317in}{0.789689in}}%
\pgfpathcurveto{\pgfqpoint{10.278828in}{0.783177in}}{\pgfqpoint{10.287661in}{0.779519in}}{\pgfqpoint{10.296869in}{0.779519in}}%
\pgfpathlineto{\pgfqpoint{10.296869in}{0.779519in}}%
\pgfpathclose%
\pgfusepath{stroke,fill}%
\end{pgfscope}%
\begin{pgfscope}%
\pgfpathrectangle{\pgfqpoint{1.374500in}{0.082500in}}{\pgfqpoint{2.419000in}{2.419000in}}%
\pgfusepath{clip}%
\pgfsetbuttcap%
\pgfsetroundjoin%
\definecolor{currentfill}{rgb}{1.000000,0.662745,0.054902}%
\pgfsetfillcolor{currentfill}%
\pgfsetfillopacity{0.463788}%
\pgfsetlinewidth{1.003750pt}%
\definecolor{currentstroke}{rgb}{1.000000,0.662745,0.054902}%
\pgfsetstrokecolor{currentstroke}%
\pgfsetstrokeopacity{0.463788}%
\pgfsetdash{}{0pt}%
\pgfpathmoveto{\pgfqpoint{5.211333in}{0.779519in}}%
\pgfpathcurveto{\pgfqpoint{5.220541in}{0.779519in}}{\pgfqpoint{5.229374in}{0.783177in}}{\pgfqpoint{5.235885in}{0.789689in}}%
\pgfpathcurveto{\pgfqpoint{5.242396in}{0.796200in}}{\pgfqpoint{5.246055in}{0.805033in}}{\pgfqpoint{5.246055in}{0.814241in}}%
\pgfpathcurveto{\pgfqpoint{5.246055in}{0.823450in}}{\pgfqpoint{5.242396in}{0.832282in}}{\pgfqpoint{5.235885in}{0.838793in}}%
\pgfpathcurveto{\pgfqpoint{5.229374in}{0.845305in}}{\pgfqpoint{5.220541in}{0.848963in}}{\pgfqpoint{5.211333in}{0.848963in}}%
\pgfpathcurveto{\pgfqpoint{5.202124in}{0.848963in}}{\pgfqpoint{5.193292in}{0.845305in}}{\pgfqpoint{5.186780in}{0.838793in}}%
\pgfpathcurveto{\pgfqpoint{5.180269in}{0.832282in}}{\pgfqpoint{5.176610in}{0.823450in}}{\pgfqpoint{5.176610in}{0.814241in}}%
\pgfpathcurveto{\pgfqpoint{5.176610in}{0.805033in}}{\pgfqpoint{5.180269in}{0.796200in}}{\pgfqpoint{5.186780in}{0.789689in}}%
\pgfpathcurveto{\pgfqpoint{5.193292in}{0.783177in}}{\pgfqpoint{5.202124in}{0.779519in}}{\pgfqpoint{5.211333in}{0.779519in}}%
\pgfpathlineto{\pgfqpoint{5.211333in}{0.779519in}}%
\pgfpathclose%
\pgfusepath{stroke,fill}%
\end{pgfscope}%
\begin{pgfscope}%
\pgfpathrectangle{\pgfqpoint{1.374500in}{0.082500in}}{\pgfqpoint{2.419000in}{2.419000in}}%
\pgfusepath{clip}%
\pgfsetbuttcap%
\pgfsetroundjoin%
\definecolor{currentfill}{rgb}{1.000000,0.662745,0.054902}%
\pgfsetfillcolor{currentfill}%
\pgfsetfillopacity{0.470898}%
\pgfsetlinewidth{1.003750pt}%
\definecolor{currentstroke}{rgb}{1.000000,0.662745,0.054902}%
\pgfsetstrokecolor{currentstroke}%
\pgfsetstrokeopacity{0.470898}%
\pgfsetdash{}{0pt}%
\pgfpathmoveto{\pgfqpoint{-2.024823in}{0.629853in}}%
\pgfpathcurveto{\pgfqpoint{-2.015615in}{0.629853in}}{\pgfqpoint{-2.006782in}{0.633511in}}{\pgfqpoint{-2.000271in}{0.640023in}}%
\pgfpathcurveto{\pgfqpoint{-1.993759in}{0.646534in}}{\pgfqpoint{-1.990101in}{0.655367in}}{\pgfqpoint{-1.990101in}{0.664575in}}%
\pgfpathcurveto{\pgfqpoint{-1.990101in}{0.673784in}}{\pgfqpoint{-1.993759in}{0.682616in}}{\pgfqpoint{-2.000271in}{0.689127in}}%
\pgfpathcurveto{\pgfqpoint{-2.006782in}{0.695639in}}{\pgfqpoint{-2.015615in}{0.699297in}}{\pgfqpoint{-2.024823in}{0.699297in}}%
\pgfpathcurveto{\pgfqpoint{-2.034031in}{0.699297in}}{\pgfqpoint{-2.042864in}{0.695639in}}{\pgfqpoint{-2.049375in}{0.689127in}}%
\pgfpathcurveto{\pgfqpoint{-2.055887in}{0.682616in}}{\pgfqpoint{-2.059545in}{0.673784in}}{\pgfqpoint{-2.059545in}{0.664575in}}%
\pgfpathcurveto{\pgfqpoint{-2.059545in}{0.655367in}}{\pgfqpoint{-2.055887in}{0.646534in}}{\pgfqpoint{-2.049375in}{0.640023in}}%
\pgfpathcurveto{\pgfqpoint{-2.042864in}{0.633511in}}{\pgfqpoint{-2.034031in}{0.629853in}}{\pgfqpoint{-2.024823in}{0.629853in}}%
\pgfpathlineto{\pgfqpoint{-2.024823in}{0.629853in}}%
\pgfpathclose%
\pgfusepath{stroke,fill}%
\end{pgfscope}%
\begin{pgfscope}%
\pgfpathrectangle{\pgfqpoint{1.374500in}{0.082500in}}{\pgfqpoint{2.419000in}{2.419000in}}%
\pgfusepath{clip}%
\pgfsetbuttcap%
\pgfsetroundjoin%
\definecolor{currentfill}{rgb}{1.000000,0.662745,0.054902}%
\pgfsetfillcolor{currentfill}%
\pgfsetfillopacity{0.470898}%
\pgfsetlinewidth{1.003750pt}%
\definecolor{currentstroke}{rgb}{1.000000,0.662745,0.054902}%
\pgfsetstrokecolor{currentstroke}%
\pgfsetstrokeopacity{0.470898}%
\pgfsetdash{}{0pt}%
\pgfpathmoveto{\pgfqpoint{3.155150in}{0.629853in}}%
\pgfpathcurveto{\pgfqpoint{3.164359in}{0.629853in}}{\pgfqpoint{3.173191in}{0.633511in}}{\pgfqpoint{3.179703in}{0.640023in}}%
\pgfpathcurveto{\pgfqpoint{3.186214in}{0.646534in}}{\pgfqpoint{3.189872in}{0.655367in}}{\pgfqpoint{3.189872in}{0.664575in}}%
\pgfpathcurveto{\pgfqpoint{3.189872in}{0.673784in}}{\pgfqpoint{3.186214in}{0.682616in}}{\pgfqpoint{3.179703in}{0.689127in}}%
\pgfpathcurveto{\pgfqpoint{3.173191in}{0.695639in}}{\pgfqpoint{3.164359in}{0.699297in}}{\pgfqpoint{3.155150in}{0.699297in}}%
\pgfpathcurveto{\pgfqpoint{3.145942in}{0.699297in}}{\pgfqpoint{3.137109in}{0.695639in}}{\pgfqpoint{3.130598in}{0.689127in}}%
\pgfpathcurveto{\pgfqpoint{3.124087in}{0.682616in}}{\pgfqpoint{3.120428in}{0.673784in}}{\pgfqpoint{3.120428in}{0.664575in}}%
\pgfpathcurveto{\pgfqpoint{3.120428in}{0.655367in}}{\pgfqpoint{3.124087in}{0.646534in}}{\pgfqpoint{3.130598in}{0.640023in}}%
\pgfpathcurveto{\pgfqpoint{3.137109in}{0.633511in}}{\pgfqpoint{3.145942in}{0.629853in}}{\pgfqpoint{3.155150in}{0.629853in}}%
\pgfpathlineto{\pgfqpoint{3.155150in}{0.629853in}}%
\pgfpathclose%
\pgfusepath{stroke,fill}%
\end{pgfscope}%
\begin{pgfscope}%
\pgfpathrectangle{\pgfqpoint{1.374500in}{0.082500in}}{\pgfqpoint{2.419000in}{2.419000in}}%
\pgfusepath{clip}%
\pgfsetbuttcap%
\pgfsetroundjoin%
\definecolor{currentfill}{rgb}{1.000000,0.662745,0.054902}%
\pgfsetfillcolor{currentfill}%
\pgfsetfillopacity{0.470898}%
\pgfsetlinewidth{1.003750pt}%
\definecolor{currentstroke}{rgb}{1.000000,0.662745,0.054902}%
\pgfsetstrokecolor{currentstroke}%
\pgfsetstrokeopacity{0.470898}%
\pgfsetdash{}{0pt}%
\pgfpathmoveto{\pgfqpoint{8.335124in}{0.629853in}}%
\pgfpathcurveto{\pgfqpoint{8.344332in}{0.629853in}}{\pgfqpoint{8.353165in}{0.633511in}}{\pgfqpoint{8.359676in}{0.640023in}}%
\pgfpathcurveto{\pgfqpoint{8.366187in}{0.646534in}}{\pgfqpoint{8.369846in}{0.655367in}}{\pgfqpoint{8.369846in}{0.664575in}}%
\pgfpathcurveto{\pgfqpoint{8.369846in}{0.673784in}}{\pgfqpoint{8.366187in}{0.682616in}}{\pgfqpoint{8.359676in}{0.689127in}}%
\pgfpathcurveto{\pgfqpoint{8.353165in}{0.695639in}}{\pgfqpoint{8.344332in}{0.699297in}}{\pgfqpoint{8.335124in}{0.699297in}}%
\pgfpathcurveto{\pgfqpoint{8.325915in}{0.699297in}}{\pgfqpoint{8.317083in}{0.695639in}}{\pgfqpoint{8.310571in}{0.689127in}}%
\pgfpathcurveto{\pgfqpoint{8.304060in}{0.682616in}}{\pgfqpoint{8.300401in}{0.673784in}}{\pgfqpoint{8.300401in}{0.664575in}}%
\pgfpathcurveto{\pgfqpoint{8.300401in}{0.655367in}}{\pgfqpoint{8.304060in}{0.646534in}}{\pgfqpoint{8.310571in}{0.640023in}}%
\pgfpathcurveto{\pgfqpoint{8.317083in}{0.633511in}}{\pgfqpoint{8.325915in}{0.629853in}}{\pgfqpoint{8.335124in}{0.629853in}}%
\pgfpathlineto{\pgfqpoint{8.335124in}{0.629853in}}%
\pgfpathclose%
\pgfusepath{stroke,fill}%
\end{pgfscope}%
\begin{pgfscope}%
\pgfpathrectangle{\pgfqpoint{1.374500in}{0.082500in}}{\pgfqpoint{2.419000in}{2.419000in}}%
\pgfusepath{clip}%
\pgfsetbuttcap%
\pgfsetroundjoin%
\definecolor{currentfill}{rgb}{1.000000,0.662745,0.054902}%
\pgfsetfillcolor{currentfill}%
\pgfsetfillopacity{0.478278}%
\pgfsetlinewidth{1.003750pt}%
\definecolor{currentstroke}{rgb}{1.000000,0.662745,0.054902}%
\pgfsetstrokecolor{currentstroke}%
\pgfsetstrokeopacity{0.478278}%
\pgfsetdash{}{0pt}%
\pgfpathmoveto{\pgfqpoint{1.021158in}{0.474523in}}%
\pgfpathcurveto{\pgfqpoint{1.030366in}{0.474523in}}{\pgfqpoint{1.039199in}{0.478182in}}{\pgfqpoint{1.045710in}{0.484693in}}%
\pgfpathcurveto{\pgfqpoint{1.052221in}{0.491205in}}{\pgfqpoint{1.055880in}{0.500037in}}{\pgfqpoint{1.055880in}{0.509245in}}%
\pgfpathcurveto{\pgfqpoint{1.055880in}{0.518454in}}{\pgfqpoint{1.052221in}{0.527286in}}{\pgfqpoint{1.045710in}{0.533798in}}%
\pgfpathcurveto{\pgfqpoint{1.039199in}{0.540309in}}{\pgfqpoint{1.030366in}{0.543968in}}{\pgfqpoint{1.021158in}{0.543968in}}%
\pgfpathcurveto{\pgfqpoint{1.011949in}{0.543968in}}{\pgfqpoint{1.003117in}{0.540309in}}{\pgfqpoint{0.996605in}{0.533798in}}%
\pgfpathcurveto{\pgfqpoint{0.990094in}{0.527286in}}{\pgfqpoint{0.986435in}{0.518454in}}{\pgfqpoint{0.986435in}{0.509245in}}%
\pgfpathcurveto{\pgfqpoint{0.986435in}{0.500037in}}{\pgfqpoint{0.990094in}{0.491205in}}{\pgfqpoint{0.996605in}{0.484693in}}%
\pgfpathcurveto{\pgfqpoint{1.003117in}{0.478182in}}{\pgfqpoint{1.011949in}{0.474523in}}{\pgfqpoint{1.021158in}{0.474523in}}%
\pgfpathlineto{\pgfqpoint{1.021158in}{0.474523in}}%
\pgfpathclose%
\pgfusepath{stroke,fill}%
\end{pgfscope}%
\begin{pgfscope}%
\pgfpathrectangle{\pgfqpoint{1.374500in}{0.082500in}}{\pgfqpoint{2.419000in}{2.419000in}}%
\pgfusepath{clip}%
\pgfsetbuttcap%
\pgfsetroundjoin%
\definecolor{currentfill}{rgb}{1.000000,0.662745,0.054902}%
\pgfsetfillcolor{currentfill}%
\pgfsetfillopacity{0.478278}%
\pgfsetlinewidth{1.003750pt}%
\definecolor{currentstroke}{rgb}{1.000000,0.662745,0.054902}%
\pgfsetstrokecolor{currentstroke}%
\pgfsetstrokeopacity{0.478278}%
\pgfsetdash{}{0pt}%
\pgfpathmoveto{\pgfqpoint{6.299141in}{0.474523in}}%
\pgfpathcurveto{\pgfqpoint{6.308350in}{0.474523in}}{\pgfqpoint{6.317182in}{0.478182in}}{\pgfqpoint{6.323694in}{0.484693in}}%
\pgfpathcurveto{\pgfqpoint{6.330205in}{0.491205in}}{\pgfqpoint{6.333864in}{0.500037in}}{\pgfqpoint{6.333864in}{0.509245in}}%
\pgfpathcurveto{\pgfqpoint{6.333864in}{0.518454in}}{\pgfqpoint{6.330205in}{0.527286in}}{\pgfqpoint{6.323694in}{0.533798in}}%
\pgfpathcurveto{\pgfqpoint{6.317182in}{0.540309in}}{\pgfqpoint{6.308350in}{0.543968in}}{\pgfqpoint{6.299141in}{0.543968in}}%
\pgfpathcurveto{\pgfqpoint{6.289933in}{0.543968in}}{\pgfqpoint{6.281100in}{0.540309in}}{\pgfqpoint{6.274589in}{0.533798in}}%
\pgfpathcurveto{\pgfqpoint{6.268078in}{0.527286in}}{\pgfqpoint{6.264419in}{0.518454in}}{\pgfqpoint{6.264419in}{0.509245in}}%
\pgfpathcurveto{\pgfqpoint{6.264419in}{0.500037in}}{\pgfqpoint{6.268078in}{0.491205in}}{\pgfqpoint{6.274589in}{0.484693in}}%
\pgfpathcurveto{\pgfqpoint{6.281100in}{0.478182in}}{\pgfqpoint{6.289933in}{0.474523in}}{\pgfqpoint{6.299141in}{0.474523in}}%
\pgfpathlineto{\pgfqpoint{6.299141in}{0.474523in}}%
\pgfpathclose%
\pgfusepath{stroke,fill}%
\end{pgfscope}%
\begin{pgfscope}%
\pgfpathrectangle{\pgfqpoint{1.374500in}{0.082500in}}{\pgfqpoint{2.419000in}{2.419000in}}%
\pgfusepath{clip}%
\pgfsetbuttcap%
\pgfsetroundjoin%
\definecolor{currentfill}{rgb}{1.000000,0.662745,0.054902}%
\pgfsetfillcolor{currentfill}%
\pgfsetfillopacity{0.478278}%
\pgfsetlinewidth{1.003750pt}%
\definecolor{currentstroke}{rgb}{1.000000,0.662745,0.054902}%
\pgfsetstrokecolor{currentstroke}%
\pgfsetstrokeopacity{0.478278}%
\pgfsetdash{}{0pt}%
\pgfpathmoveto{\pgfqpoint{11.577125in}{0.474523in}}%
\pgfpathcurveto{\pgfqpoint{11.586334in}{0.474523in}}{\pgfqpoint{11.595166in}{0.478182in}}{\pgfqpoint{11.601678in}{0.484693in}}%
\pgfpathcurveto{\pgfqpoint{11.608189in}{0.491205in}}{\pgfqpoint{11.611847in}{0.500037in}}{\pgfqpoint{11.611847in}{0.509245in}}%
\pgfpathcurveto{\pgfqpoint{11.611847in}{0.518454in}}{\pgfqpoint{11.608189in}{0.527286in}}{\pgfqpoint{11.601678in}{0.533798in}}%
\pgfpathcurveto{\pgfqpoint{11.595166in}{0.540309in}}{\pgfqpoint{11.586334in}{0.543968in}}{\pgfqpoint{11.577125in}{0.543968in}}%
\pgfpathcurveto{\pgfqpoint{11.567917in}{0.543968in}}{\pgfqpoint{11.559084in}{0.540309in}}{\pgfqpoint{11.552573in}{0.533798in}}%
\pgfpathcurveto{\pgfqpoint{11.546062in}{0.527286in}}{\pgfqpoint{11.542403in}{0.518454in}}{\pgfqpoint{11.542403in}{0.509245in}}%
\pgfpathcurveto{\pgfqpoint{11.542403in}{0.500037in}}{\pgfqpoint{11.546062in}{0.491205in}}{\pgfqpoint{11.552573in}{0.484693in}}%
\pgfpathcurveto{\pgfqpoint{11.559084in}{0.478182in}}{\pgfqpoint{11.567917in}{0.474523in}}{\pgfqpoint{11.577125in}{0.474523in}}%
\pgfpathlineto{\pgfqpoint{11.577125in}{0.474523in}}%
\pgfpathclose%
\pgfusepath{stroke,fill}%
\end{pgfscope}%
\begin{pgfscope}%
\pgfpathrectangle{\pgfqpoint{1.374500in}{0.082500in}}{\pgfqpoint{2.419000in}{2.419000in}}%
\pgfusepath{clip}%
\pgfsetbuttcap%
\pgfsetroundjoin%
\definecolor{currentfill}{rgb}{1.000000,0.662745,0.054902}%
\pgfsetfillcolor{currentfill}%
\pgfsetfillopacity{0.485942}%
\pgfsetlinewidth{1.003750pt}%
\definecolor{currentstroke}{rgb}{1.000000,0.662745,0.054902}%
\pgfsetstrokecolor{currentstroke}%
\pgfsetstrokeopacity{0.485942}%
\pgfsetdash{}{0pt}%
\pgfpathmoveto{\pgfqpoint{-1.195147in}{0.313202in}}%
\pgfpathcurveto{\pgfqpoint{-1.185939in}{0.313202in}}{\pgfqpoint{-1.177106in}{0.316861in}}{\pgfqpoint{-1.170595in}{0.323372in}}%
\pgfpathcurveto{\pgfqpoint{-1.164084in}{0.329883in}}{\pgfqpoint{-1.160425in}{0.338716in}}{\pgfqpoint{-1.160425in}{0.347924in}}%
\pgfpathcurveto{\pgfqpoint{-1.160425in}{0.357133in}}{\pgfqpoint{-1.164084in}{0.365965in}}{\pgfqpoint{-1.170595in}{0.372477in}}%
\pgfpathcurveto{\pgfqpoint{-1.177106in}{0.378988in}}{\pgfqpoint{-1.185939in}{0.382647in}}{\pgfqpoint{-1.195147in}{0.382647in}}%
\pgfpathcurveto{\pgfqpoint{-1.204356in}{0.382647in}}{\pgfqpoint{-1.213188in}{0.378988in}}{\pgfqpoint{-1.219700in}{0.372477in}}%
\pgfpathcurveto{\pgfqpoint{-1.226211in}{0.365965in}}{\pgfqpoint{-1.229870in}{0.357133in}}{\pgfqpoint{-1.229870in}{0.347924in}}%
\pgfpathcurveto{\pgfqpoint{-1.229870in}{0.338716in}}{\pgfqpoint{-1.226211in}{0.329883in}}{\pgfqpoint{-1.219700in}{0.323372in}}%
\pgfpathcurveto{\pgfqpoint{-1.213188in}{0.316861in}}{\pgfqpoint{-1.204356in}{0.313202in}}{\pgfqpoint{-1.195147in}{0.313202in}}%
\pgfpathlineto{\pgfqpoint{-1.195147in}{0.313202in}}%
\pgfpathclose%
\pgfusepath{stroke,fill}%
\end{pgfscope}%
\begin{pgfscope}%
\pgfpathrectangle{\pgfqpoint{1.374500in}{0.082500in}}{\pgfqpoint{2.419000in}{2.419000in}}%
\pgfusepath{clip}%
\pgfsetbuttcap%
\pgfsetroundjoin%
\definecolor{currentfill}{rgb}{1.000000,0.662745,0.054902}%
\pgfsetfillcolor{currentfill}%
\pgfsetfillopacity{0.485942}%
\pgfsetlinewidth{1.003750pt}%
\definecolor{currentstroke}{rgb}{1.000000,0.662745,0.054902}%
\pgfsetstrokecolor{currentstroke}%
\pgfsetstrokeopacity{0.485942}%
\pgfsetdash{}{0pt}%
\pgfpathmoveto{\pgfqpoint{9.564402in}{0.313202in}}%
\pgfpathcurveto{\pgfqpoint{9.573611in}{0.313202in}}{\pgfqpoint{9.582443in}{0.316861in}}{\pgfqpoint{9.588955in}{0.323372in}}%
\pgfpathcurveto{\pgfqpoint{9.595466in}{0.329883in}}{\pgfqpoint{9.599125in}{0.338716in}}{\pgfqpoint{9.599125in}{0.347924in}}%
\pgfpathcurveto{\pgfqpoint{9.599125in}{0.357133in}}{\pgfqpoint{9.595466in}{0.365965in}}{\pgfqpoint{9.588955in}{0.372477in}}%
\pgfpathcurveto{\pgfqpoint{9.582443in}{0.378988in}}{\pgfqpoint{9.573611in}{0.382647in}}{\pgfqpoint{9.564402in}{0.382647in}}%
\pgfpathcurveto{\pgfqpoint{9.555194in}{0.382647in}}{\pgfqpoint{9.546361in}{0.378988in}}{\pgfqpoint{9.539850in}{0.372477in}}%
\pgfpathcurveto{\pgfqpoint{9.533339in}{0.365965in}}{\pgfqpoint{9.529680in}{0.357133in}}{\pgfqpoint{9.529680in}{0.347924in}}%
\pgfpathcurveto{\pgfqpoint{9.529680in}{0.338716in}}{\pgfqpoint{9.533339in}{0.329883in}}{\pgfqpoint{9.539850in}{0.323372in}}%
\pgfpathcurveto{\pgfqpoint{9.546361in}{0.316861in}}{\pgfqpoint{9.555194in}{0.313202in}}{\pgfqpoint{9.564402in}{0.313202in}}%
\pgfpathlineto{\pgfqpoint{9.564402in}{0.313202in}}%
\pgfpathclose%
\pgfusepath{stroke,fill}%
\end{pgfscope}%
\begin{pgfscope}%
\pgfpathrectangle{\pgfqpoint{1.374500in}{0.082500in}}{\pgfqpoint{2.419000in}{2.419000in}}%
\pgfusepath{clip}%
\pgfsetbuttcap%
\pgfsetroundjoin%
\definecolor{currentfill}{rgb}{1.000000,0.662745,0.054902}%
\pgfsetfillcolor{currentfill}%
\pgfsetfillopacity{0.485942}%
\pgfsetlinewidth{1.003750pt}%
\definecolor{currentstroke}{rgb}{1.000000,0.662745,0.054902}%
\pgfsetstrokecolor{currentstroke}%
\pgfsetstrokeopacity{0.485942}%
\pgfsetdash{}{0pt}%
\pgfpathmoveto{\pgfqpoint{4.184628in}{0.313202in}}%
\pgfpathcurveto{\pgfqpoint{4.193836in}{0.313202in}}{\pgfqpoint{4.202668in}{0.316861in}}{\pgfqpoint{4.209180in}{0.323372in}}%
\pgfpathcurveto{\pgfqpoint{4.215691in}{0.329883in}}{\pgfqpoint{4.219350in}{0.338716in}}{\pgfqpoint{4.219350in}{0.347924in}}%
\pgfpathcurveto{\pgfqpoint{4.219350in}{0.357133in}}{\pgfqpoint{4.215691in}{0.365965in}}{\pgfqpoint{4.209180in}{0.372477in}}%
\pgfpathcurveto{\pgfqpoint{4.202668in}{0.378988in}}{\pgfqpoint{4.193836in}{0.382647in}}{\pgfqpoint{4.184628in}{0.382647in}}%
\pgfpathcurveto{\pgfqpoint{4.175419in}{0.382647in}}{\pgfqpoint{4.166587in}{0.378988in}}{\pgfqpoint{4.160075in}{0.372477in}}%
\pgfpathcurveto{\pgfqpoint{4.153564in}{0.365965in}}{\pgfqpoint{4.149905in}{0.357133in}}{\pgfqpoint{4.149905in}{0.347924in}}%
\pgfpathcurveto{\pgfqpoint{4.149905in}{0.338716in}}{\pgfqpoint{4.153564in}{0.329883in}}{\pgfqpoint{4.160075in}{0.323372in}}%
\pgfpathcurveto{\pgfqpoint{4.166587in}{0.316861in}}{\pgfqpoint{4.175419in}{0.313202in}}{\pgfqpoint{4.184628in}{0.313202in}}%
\pgfpathlineto{\pgfqpoint{4.184628in}{0.313202in}}%
\pgfpathclose%
\pgfusepath{stroke,fill}%
\end{pgfscope}%
\begin{pgfscope}%
\pgfpathrectangle{\pgfqpoint{1.374500in}{0.082500in}}{\pgfqpoint{2.419000in}{2.419000in}}%
\pgfusepath{clip}%
\pgfsetbuttcap%
\pgfsetroundjoin%
\definecolor{currentfill}{rgb}{1.000000,0.662745,0.054902}%
\pgfsetfillcolor{currentfill}%
\pgfsetfillopacity{0.493908}%
\pgfsetlinewidth{1.003750pt}%
\definecolor{currentstroke}{rgb}{1.000000,0.662745,0.054902}%
\pgfsetstrokecolor{currentstroke}%
\pgfsetstrokeopacity{0.493908}%
\pgfsetdash{}{0pt}%
\pgfpathmoveto{\pgfqpoint{-3.498620in}{0.145536in}}%
\pgfpathcurveto{\pgfqpoint{-3.489412in}{0.145536in}}{\pgfqpoint{-3.480579in}{0.149195in}}{\pgfqpoint{-3.474068in}{0.155706in}}%
\pgfpathcurveto{\pgfqpoint{-3.467557in}{0.162218in}}{\pgfqpoint{-3.463898in}{0.171050in}}{\pgfqpoint{-3.463898in}{0.180259in}}%
\pgfpathcurveto{\pgfqpoint{-3.463898in}{0.189467in}}{\pgfqpoint{-3.467557in}{0.198300in}}{\pgfqpoint{-3.474068in}{0.204811in}}%
\pgfpathcurveto{\pgfqpoint{-3.480579in}{0.211322in}}{\pgfqpoint{-3.489412in}{0.214981in}}{\pgfqpoint{-3.498620in}{0.214981in}}%
\pgfpathcurveto{\pgfqpoint{-3.507829in}{0.214981in}}{\pgfqpoint{-3.516661in}{0.211322in}}{\pgfqpoint{-3.523173in}{0.204811in}}%
\pgfpathcurveto{\pgfqpoint{-3.529684in}{0.198300in}}{\pgfqpoint{-3.533343in}{0.189467in}}{\pgfqpoint{-3.533343in}{0.180259in}}%
\pgfpathcurveto{\pgfqpoint{-3.533343in}{0.171050in}}{\pgfqpoint{-3.529684in}{0.162218in}}{\pgfqpoint{-3.523173in}{0.155706in}}%
\pgfpathcurveto{\pgfqpoint{-3.516661in}{0.149195in}}{\pgfqpoint{-3.507829in}{0.145536in}}{\pgfqpoint{-3.498620in}{0.145536in}}%
\pgfpathlineto{\pgfqpoint{-3.498620in}{0.145536in}}%
\pgfpathclose%
\pgfusepath{stroke,fill}%
\end{pgfscope}%
\begin{pgfscope}%
\pgfpathrectangle{\pgfqpoint{1.374500in}{0.082500in}}{\pgfqpoint{2.419000in}{2.419000in}}%
\pgfusepath{clip}%
\pgfsetbuttcap%
\pgfsetroundjoin%
\definecolor{currentfill}{rgb}{1.000000,0.662745,0.054902}%
\pgfsetfillcolor{currentfill}%
\pgfsetfillopacity{0.493908}%
\pgfsetlinewidth{1.003750pt}%
\definecolor{currentstroke}{rgb}{1.000000,0.662745,0.054902}%
\pgfsetstrokecolor{currentstroke}%
\pgfsetstrokeopacity{0.493908}%
\pgfsetdash{}{0pt}%
\pgfpathmoveto{\pgfqpoint{1.986949in}{0.145536in}}%
\pgfpathcurveto{\pgfqpoint{1.996157in}{0.145536in}}{\pgfqpoint{2.004990in}{0.149195in}}{\pgfqpoint{2.011501in}{0.155706in}}%
\pgfpathcurveto{\pgfqpoint{2.018013in}{0.162218in}}{\pgfqpoint{2.021671in}{0.171050in}}{\pgfqpoint{2.021671in}{0.180259in}}%
\pgfpathcurveto{\pgfqpoint{2.021671in}{0.189467in}}{\pgfqpoint{2.018013in}{0.198300in}}{\pgfqpoint{2.011501in}{0.204811in}}%
\pgfpathcurveto{\pgfqpoint{2.004990in}{0.211322in}}{\pgfqpoint{1.996157in}{0.214981in}}{\pgfqpoint{1.986949in}{0.214981in}}%
\pgfpathcurveto{\pgfqpoint{1.977740in}{0.214981in}}{\pgfqpoint{1.968908in}{0.211322in}}{\pgfqpoint{1.962397in}{0.204811in}}%
\pgfpathcurveto{\pgfqpoint{1.955885in}{0.198300in}}{\pgfqpoint{1.952227in}{0.189467in}}{\pgfqpoint{1.952227in}{0.180259in}}%
\pgfpathcurveto{\pgfqpoint{1.952227in}{0.171050in}}{\pgfqpoint{1.955885in}{0.162218in}}{\pgfqpoint{1.962397in}{0.155706in}}%
\pgfpathcurveto{\pgfqpoint{1.968908in}{0.149195in}}{\pgfqpoint{1.977740in}{0.145536in}}{\pgfqpoint{1.986949in}{0.145536in}}%
\pgfpathlineto{\pgfqpoint{1.986949in}{0.145536in}}%
\pgfpathclose%
\pgfusepath{stroke,fill}%
\end{pgfscope}%
\begin{pgfscope}%
\pgfpathrectangle{\pgfqpoint{1.374500in}{0.082500in}}{\pgfqpoint{2.419000in}{2.419000in}}%
\pgfusepath{clip}%
\pgfsetbuttcap%
\pgfsetroundjoin%
\definecolor{currentfill}{rgb}{1.000000,0.662745,0.054902}%
\pgfsetfillcolor{currentfill}%
\pgfsetfillopacity{0.493908}%
\pgfsetlinewidth{1.003750pt}%
\definecolor{currentstroke}{rgb}{1.000000,0.662745,0.054902}%
\pgfsetstrokecolor{currentstroke}%
\pgfsetstrokeopacity{0.493908}%
\pgfsetdash{}{0pt}%
\pgfpathmoveto{\pgfqpoint{7.472518in}{0.145536in}}%
\pgfpathcurveto{\pgfqpoint{7.481727in}{0.145536in}}{\pgfqpoint{7.490559in}{0.149195in}}{\pgfqpoint{7.497070in}{0.155706in}}%
\pgfpathcurveto{\pgfqpoint{7.503582in}{0.162218in}}{\pgfqpoint{7.507240in}{0.171050in}}{\pgfqpoint{7.507240in}{0.180259in}}%
\pgfpathcurveto{\pgfqpoint{7.507240in}{0.189467in}}{\pgfqpoint{7.503582in}{0.198300in}}{\pgfqpoint{7.497070in}{0.204811in}}%
\pgfpathcurveto{\pgfqpoint{7.490559in}{0.211322in}}{\pgfqpoint{7.481727in}{0.214981in}}{\pgfqpoint{7.472518in}{0.214981in}}%
\pgfpathcurveto{\pgfqpoint{7.463310in}{0.214981in}}{\pgfqpoint{7.454477in}{0.211322in}}{\pgfqpoint{7.447966in}{0.204811in}}%
\pgfpathcurveto{\pgfqpoint{7.441454in}{0.198300in}}{\pgfqpoint{7.437796in}{0.189467in}}{\pgfqpoint{7.437796in}{0.180259in}}%
\pgfpathcurveto{\pgfqpoint{7.437796in}{0.171050in}}{\pgfqpoint{7.441454in}{0.162218in}}{\pgfqpoint{7.447966in}{0.155706in}}%
\pgfpathcurveto{\pgfqpoint{7.454477in}{0.149195in}}{\pgfqpoint{7.463310in}{0.145536in}}{\pgfqpoint{7.472518in}{0.145536in}}%
\pgfpathlineto{\pgfqpoint{7.472518in}{0.145536in}}%
\pgfpathclose%
\pgfusepath{stroke,fill}%
\end{pgfscope}%
\begin{pgfscope}%
\pgfpathrectangle{\pgfqpoint{1.374500in}{0.082500in}}{\pgfqpoint{2.419000in}{2.419000in}}%
\pgfusepath{clip}%
\pgfsetbuttcap%
\pgfsetroundjoin%
\definecolor{currentfill}{rgb}{1.000000,0.662745,0.054902}%
\pgfsetfillcolor{currentfill}%
\pgfsetfillopacity{0.502193}%
\pgfsetlinewidth{1.003750pt}%
\definecolor{currentstroke}{rgb}{1.000000,0.662745,0.054902}%
\pgfsetstrokecolor{currentstroke}%
\pgfsetstrokeopacity{0.502193}%
\pgfsetdash{}{0pt}%
\pgfpathmoveto{\pgfqpoint{-0.298899in}{-0.028856in}}%
\pgfpathcurveto{\pgfqpoint{-0.289691in}{-0.028856in}}{\pgfqpoint{-0.280858in}{-0.025198in}}{\pgfqpoint{-0.274347in}{-0.018686in}}%
\pgfpathcurveto{\pgfqpoint{-0.267836in}{-0.012175in}}{\pgfqpoint{-0.264177in}{-0.003342in}}{\pgfqpoint{-0.264177in}{0.005866in}}%
\pgfpathcurveto{\pgfqpoint{-0.264177in}{0.015074in}}{\pgfqpoint{-0.267836in}{0.023907in}}{\pgfqpoint{-0.274347in}{0.030418in}}%
\pgfpathcurveto{\pgfqpoint{-0.280858in}{0.036930in}}{\pgfqpoint{-0.289691in}{0.040588in}}{\pgfqpoint{-0.298899in}{0.040588in}}%
\pgfpathcurveto{\pgfqpoint{-0.308108in}{0.040588in}}{\pgfqpoint{-0.316940in}{0.036930in}}{\pgfqpoint{-0.323452in}{0.030418in}}%
\pgfpathcurveto{\pgfqpoint{-0.329963in}{0.023907in}}{\pgfqpoint{-0.333622in}{0.015074in}}{\pgfqpoint{-0.333622in}{0.005866in}}%
\pgfpathcurveto{\pgfqpoint{-0.333622in}{-0.003342in}}{\pgfqpoint{-0.329963in}{-0.012175in}}{\pgfqpoint{-0.323452in}{-0.018686in}}%
\pgfpathcurveto{\pgfqpoint{-0.316940in}{-0.025198in}}{\pgfqpoint{-0.308108in}{-0.028856in}}{\pgfqpoint{-0.298899in}{-0.028856in}}%
\pgfpathlineto{\pgfqpoint{-0.298899in}{-0.028856in}}%
\pgfpathclose%
\pgfusepath{stroke,fill}%
\end{pgfscope}%
\begin{pgfscope}%
\pgfpathrectangle{\pgfqpoint{1.374500in}{0.082500in}}{\pgfqpoint{2.419000in}{2.419000in}}%
\pgfusepath{clip}%
\pgfsetbuttcap%
\pgfsetroundjoin%
\definecolor{currentfill}{rgb}{1.000000,0.662745,0.054902}%
\pgfsetfillcolor{currentfill}%
\pgfsetfillopacity{0.502193}%
\pgfsetlinewidth{1.003750pt}%
\definecolor{currentstroke}{rgb}{1.000000,0.662745,0.054902}%
\pgfsetstrokecolor{currentstroke}%
\pgfsetstrokeopacity{0.502193}%
\pgfsetdash{}{0pt}%
\pgfpathmoveto{\pgfqpoint{10.892317in}{-0.028856in}}%
\pgfpathcurveto{\pgfqpoint{10.901525in}{-0.028856in}}{\pgfqpoint{10.910358in}{-0.025198in}}{\pgfqpoint{10.916869in}{-0.018686in}}%
\pgfpathcurveto{\pgfqpoint{10.923381in}{-0.012175in}}{\pgfqpoint{10.927039in}{-0.003342in}}{\pgfqpoint{10.927039in}{0.005866in}}%
\pgfpathcurveto{\pgfqpoint{10.927039in}{0.015074in}}{\pgfqpoint{10.923381in}{0.023907in}}{\pgfqpoint{10.916869in}{0.030418in}}%
\pgfpathcurveto{\pgfqpoint{10.910358in}{0.036930in}}{\pgfqpoint{10.901525in}{0.040588in}}{\pgfqpoint{10.892317in}{0.040588in}}%
\pgfpathcurveto{\pgfqpoint{10.883108in}{0.040588in}}{\pgfqpoint{10.874276in}{0.036930in}}{\pgfqpoint{10.867765in}{0.030418in}}%
\pgfpathcurveto{\pgfqpoint{10.861253in}{0.023907in}}{\pgfqpoint{10.857595in}{0.015074in}}{\pgfqpoint{10.857595in}{0.005866in}}%
\pgfpathcurveto{\pgfqpoint{10.857595in}{-0.003342in}}{\pgfqpoint{10.861253in}{-0.012175in}}{\pgfqpoint{10.867765in}{-0.018686in}}%
\pgfpathcurveto{\pgfqpoint{10.874276in}{-0.025198in}}{\pgfqpoint{10.883108in}{-0.028856in}}{\pgfqpoint{10.892317in}{-0.028856in}}%
\pgfpathlineto{\pgfqpoint{10.892317in}{-0.028856in}}%
\pgfpathclose%
\pgfusepath{stroke,fill}%
\end{pgfscope}%
\begin{pgfscope}%
\pgfpathrectangle{\pgfqpoint{1.374500in}{0.082500in}}{\pgfqpoint{2.419000in}{2.419000in}}%
\pgfusepath{clip}%
\pgfsetbuttcap%
\pgfsetroundjoin%
\definecolor{currentfill}{rgb}{1.000000,0.662745,0.054902}%
\pgfsetfillcolor{currentfill}%
\pgfsetfillopacity{0.502193}%
\pgfsetlinewidth{1.003750pt}%
\definecolor{currentstroke}{rgb}{1.000000,0.662745,0.054902}%
\pgfsetstrokecolor{currentstroke}%
\pgfsetstrokeopacity{0.502193}%
\pgfsetdash{}{0pt}%
\pgfpathmoveto{\pgfqpoint{5.296709in}{-0.028856in}}%
\pgfpathcurveto{\pgfqpoint{5.305917in}{-0.028856in}}{\pgfqpoint{5.314750in}{-0.025198in}}{\pgfqpoint{5.321261in}{-0.018686in}}%
\pgfpathcurveto{\pgfqpoint{5.327772in}{-0.012175in}}{\pgfqpoint{5.331431in}{-0.003342in}}{\pgfqpoint{5.331431in}{0.005866in}}%
\pgfpathcurveto{\pgfqpoint{5.331431in}{0.015074in}}{\pgfqpoint{5.327772in}{0.023907in}}{\pgfqpoint{5.321261in}{0.030418in}}%
\pgfpathcurveto{\pgfqpoint{5.314750in}{0.036930in}}{\pgfqpoint{5.305917in}{0.040588in}}{\pgfqpoint{5.296709in}{0.040588in}}%
\pgfpathcurveto{\pgfqpoint{5.287500in}{0.040588in}}{\pgfqpoint{5.278668in}{0.036930in}}{\pgfqpoint{5.272156in}{0.030418in}}%
\pgfpathcurveto{\pgfqpoint{5.265645in}{0.023907in}}{\pgfqpoint{5.261987in}{0.015074in}}{\pgfqpoint{5.261987in}{0.005866in}}%
\pgfpathcurveto{\pgfqpoint{5.261987in}{-0.003342in}}{\pgfqpoint{5.265645in}{-0.012175in}}{\pgfqpoint{5.272156in}{-0.018686in}}%
\pgfpathcurveto{\pgfqpoint{5.278668in}{-0.025198in}}{\pgfqpoint{5.287500in}{-0.028856in}}{\pgfqpoint{5.296709in}{-0.028856in}}%
\pgfpathlineto{\pgfqpoint{5.296709in}{-0.028856in}}%
\pgfpathclose%
\pgfusepath{stroke,fill}%
\end{pgfscope}%
\begin{pgfscope}%
\pgfpathrectangle{\pgfqpoint{1.374500in}{0.082500in}}{\pgfqpoint{2.419000in}{2.419000in}}%
\pgfusepath{clip}%
\pgfsetbuttcap%
\pgfsetroundjoin%
\definecolor{currentfill}{rgb}{1.000000,0.662745,0.054902}%
\pgfsetfillcolor{currentfill}%
\pgfsetfillopacity{0.510818}%
\pgfsetlinewidth{1.003750pt}%
\definecolor{currentstroke}{rgb}{1.000000,0.662745,0.054902}%
\pgfsetstrokecolor{currentstroke}%
\pgfsetstrokeopacity{0.510818}%
\pgfsetdash{}{0pt}%
\pgfpathmoveto{\pgfqpoint{-2.678332in}{-0.210388in}}%
\pgfpathcurveto{\pgfqpoint{-2.669123in}{-0.210388in}}{\pgfqpoint{-2.660291in}{-0.206730in}}{\pgfqpoint{-2.653779in}{-0.200219in}}%
\pgfpathcurveto{\pgfqpoint{-2.647268in}{-0.193707in}}{\pgfqpoint{-2.643609in}{-0.184875in}}{\pgfqpoint{-2.643609in}{-0.175666in}}%
\pgfpathcurveto{\pgfqpoint{-2.643609in}{-0.166458in}}{\pgfqpoint{-2.647268in}{-0.157625in}}{\pgfqpoint{-2.653779in}{-0.151114in}}%
\pgfpathcurveto{\pgfqpoint{-2.660291in}{-0.144603in}}{\pgfqpoint{-2.669123in}{-0.140944in}}{\pgfqpoint{-2.678332in}{-0.140944in}}%
\pgfpathcurveto{\pgfqpoint{-2.687540in}{-0.140944in}}{\pgfqpoint{-2.696373in}{-0.144603in}}{\pgfqpoint{-2.702884in}{-0.151114in}}%
\pgfpathcurveto{\pgfqpoint{-2.709395in}{-0.157625in}}{\pgfqpoint{-2.713054in}{-0.166458in}}{\pgfqpoint{-2.713054in}{-0.175666in}}%
\pgfpathcurveto{\pgfqpoint{-2.713054in}{-0.184875in}}{\pgfqpoint{-2.709395in}{-0.193707in}}{\pgfqpoint{-2.702884in}{-0.200219in}}%
\pgfpathcurveto{\pgfqpoint{-2.696373in}{-0.206730in}}{\pgfqpoint{-2.687540in}{-0.210388in}}{\pgfqpoint{-2.678332in}{-0.210388in}}%
\pgfpathlineto{\pgfqpoint{-2.678332in}{-0.210388in}}%
\pgfpathclose%
\pgfusepath{stroke,fill}%
\end{pgfscope}%
\begin{pgfscope}%
\pgfpathrectangle{\pgfqpoint{1.374500in}{0.082500in}}{\pgfqpoint{2.419000in}{2.419000in}}%
\pgfusepath{clip}%
\pgfsetbuttcap%
\pgfsetroundjoin%
\definecolor{currentfill}{rgb}{1.000000,0.662745,0.054902}%
\pgfsetfillcolor{currentfill}%
\pgfsetfillopacity{0.510818}%
\pgfsetlinewidth{1.003750pt}%
\definecolor{currentstroke}{rgb}{1.000000,0.662745,0.054902}%
\pgfsetstrokecolor{currentstroke}%
\pgfsetstrokeopacity{0.510818}%
\pgfsetdash{}{0pt}%
\pgfpathmoveto{\pgfqpoint{3.031820in}{-0.210388in}}%
\pgfpathcurveto{\pgfqpoint{3.041029in}{-0.210388in}}{\pgfqpoint{3.049861in}{-0.206730in}}{\pgfqpoint{3.056373in}{-0.200219in}}%
\pgfpathcurveto{\pgfqpoint{3.062884in}{-0.193707in}}{\pgfqpoint{3.066543in}{-0.184875in}}{\pgfqpoint{3.066543in}{-0.175666in}}%
\pgfpathcurveto{\pgfqpoint{3.066543in}{-0.166458in}}{\pgfqpoint{3.062884in}{-0.157625in}}{\pgfqpoint{3.056373in}{-0.151114in}}%
\pgfpathcurveto{\pgfqpoint{3.049861in}{-0.144603in}}{\pgfqpoint{3.041029in}{-0.140944in}}{\pgfqpoint{3.031820in}{-0.140944in}}%
\pgfpathcurveto{\pgfqpoint{3.022612in}{-0.140944in}}{\pgfqpoint{3.013779in}{-0.144603in}}{\pgfqpoint{3.007268in}{-0.151114in}}%
\pgfpathcurveto{\pgfqpoint{3.000757in}{-0.157625in}}{\pgfqpoint{2.997098in}{-0.166458in}}{\pgfqpoint{2.997098in}{-0.175666in}}%
\pgfpathcurveto{\pgfqpoint{2.997098in}{-0.184875in}}{\pgfqpoint{3.000757in}{-0.193707in}}{\pgfqpoint{3.007268in}{-0.200219in}}%
\pgfpathcurveto{\pgfqpoint{3.013779in}{-0.206730in}}{\pgfqpoint{3.022612in}{-0.210388in}}{\pgfqpoint{3.031820in}{-0.210388in}}%
\pgfpathlineto{\pgfqpoint{3.031820in}{-0.210388in}}%
\pgfpathclose%
\pgfusepath{stroke,fill}%
\end{pgfscope}%
\begin{pgfscope}%
\pgfpathrectangle{\pgfqpoint{1.374500in}{0.082500in}}{\pgfqpoint{2.419000in}{2.419000in}}%
\pgfusepath{clip}%
\pgfsetbuttcap%
\pgfsetroundjoin%
\definecolor{currentfill}{rgb}{1.000000,0.662745,0.054902}%
\pgfsetfillcolor{currentfill}%
\pgfsetfillopacity{0.510818}%
\pgfsetlinewidth{1.003750pt}%
\definecolor{currentstroke}{rgb}{1.000000,0.662745,0.054902}%
\pgfsetstrokecolor{currentstroke}%
\pgfsetstrokeopacity{0.510818}%
\pgfsetdash{}{0pt}%
\pgfpathmoveto{\pgfqpoint{8.741972in}{-0.210388in}}%
\pgfpathcurveto{\pgfqpoint{8.751181in}{-0.210388in}}{\pgfqpoint{8.760013in}{-0.206730in}}{\pgfqpoint{8.766525in}{-0.200219in}}%
\pgfpathcurveto{\pgfqpoint{8.773036in}{-0.193707in}}{\pgfqpoint{8.776695in}{-0.184875in}}{\pgfqpoint{8.776695in}{-0.175666in}}%
\pgfpathcurveto{\pgfqpoint{8.776695in}{-0.166458in}}{\pgfqpoint{8.773036in}{-0.157625in}}{\pgfqpoint{8.766525in}{-0.151114in}}%
\pgfpathcurveto{\pgfqpoint{8.760013in}{-0.144603in}}{\pgfqpoint{8.751181in}{-0.140944in}}{\pgfqpoint{8.741972in}{-0.140944in}}%
\pgfpathcurveto{\pgfqpoint{8.732764in}{-0.140944in}}{\pgfqpoint{8.723932in}{-0.144603in}}{\pgfqpoint{8.717420in}{-0.151114in}}%
\pgfpathcurveto{\pgfqpoint{8.710909in}{-0.157625in}}{\pgfqpoint{8.707250in}{-0.166458in}}{\pgfqpoint{8.707250in}{-0.175666in}}%
\pgfpathcurveto{\pgfqpoint{8.707250in}{-0.184875in}}{\pgfqpoint{8.710909in}{-0.193707in}}{\pgfqpoint{8.717420in}{-0.200219in}}%
\pgfpathcurveto{\pgfqpoint{8.723932in}{-0.206730in}}{\pgfqpoint{8.732764in}{-0.210388in}}{\pgfqpoint{8.741972in}{-0.210388in}}%
\pgfpathlineto{\pgfqpoint{8.741972in}{-0.210388in}}%
\pgfpathclose%
\pgfusepath{stroke,fill}%
\end{pgfscope}%
\begin{pgfscope}%
\pgfpathrectangle{\pgfqpoint{1.374500in}{0.082500in}}{\pgfqpoint{2.419000in}{2.419000in}}%
\pgfusepath{clip}%
\pgfsetbuttcap%
\pgfsetroundjoin%
\definecolor{currentfill}{rgb}{1.000000,0.662745,0.054902}%
\pgfsetfillcolor{currentfill}%
\pgfsetfillopacity{0.519802}%
\pgfsetlinewidth{1.003750pt}%
\definecolor{currentstroke}{rgb}{1.000000,0.662745,0.054902}%
\pgfsetstrokecolor{currentstroke}%
\pgfsetstrokeopacity{0.519802}%
\pgfsetdash{}{0pt}%
\pgfpathmoveto{\pgfqpoint{0.672268in}{-0.399508in}}%
\pgfpathcurveto{\pgfqpoint{0.681477in}{-0.399508in}}{\pgfqpoint{0.690309in}{-0.395850in}}{\pgfqpoint{0.696820in}{-0.389338in}}%
\pgfpathcurveto{\pgfqpoint{0.703332in}{-0.382827in}}{\pgfqpoint{0.706990in}{-0.373994in}}{\pgfqpoint{0.706990in}{-0.364786in}}%
\pgfpathcurveto{\pgfqpoint{0.706990in}{-0.355577in}}{\pgfqpoint{0.703332in}{-0.346745in}}{\pgfqpoint{0.696820in}{-0.340234in}}%
\pgfpathcurveto{\pgfqpoint{0.690309in}{-0.333722in}}{\pgfqpoint{0.681477in}{-0.330064in}}{\pgfqpoint{0.672268in}{-0.330064in}}%
\pgfpathcurveto{\pgfqpoint{0.663060in}{-0.330064in}}{\pgfqpoint{0.654227in}{-0.333722in}}{\pgfqpoint{0.647716in}{-0.340234in}}%
\pgfpathcurveto{\pgfqpoint{0.641204in}{-0.346745in}}{\pgfqpoint{0.637546in}{-0.355577in}}{\pgfqpoint{0.637546in}{-0.364786in}}%
\pgfpathcurveto{\pgfqpoint{0.637546in}{-0.373994in}}{\pgfqpoint{0.641204in}{-0.382827in}}{\pgfqpoint{0.647716in}{-0.389338in}}%
\pgfpathcurveto{\pgfqpoint{0.654227in}{-0.395850in}}{\pgfqpoint{0.663060in}{-0.399508in}}{\pgfqpoint{0.672268in}{-0.399508in}}%
\pgfpathlineto{\pgfqpoint{0.672268in}{-0.399508in}}%
\pgfpathclose%
\pgfusepath{stroke,fill}%
\end{pgfscope}%
\begin{pgfscope}%
\pgfpathrectangle{\pgfqpoint{1.374500in}{0.082500in}}{\pgfqpoint{2.419000in}{2.419000in}}%
\pgfusepath{clip}%
\pgfsetbuttcap%
\pgfsetroundjoin%
\definecolor{currentfill}{rgb}{1.000000,0.662745,0.054902}%
\pgfsetfillcolor{currentfill}%
\pgfsetfillopacity{0.519802}%
\pgfsetlinewidth{1.003750pt}%
\definecolor{currentstroke}{rgb}{1.000000,0.662745,0.054902}%
\pgfsetstrokecolor{currentstroke}%
\pgfsetstrokeopacity{0.519802}%
\pgfsetdash{}{0pt}%
\pgfpathmoveto{\pgfqpoint{-5.157215in}{-0.399508in}}%
\pgfpathcurveto{\pgfqpoint{-5.148007in}{-0.399508in}}{\pgfqpoint{-5.139174in}{-0.395850in}}{\pgfqpoint{-5.132663in}{-0.389338in}}%
\pgfpathcurveto{\pgfqpoint{-5.126152in}{-0.382827in}}{\pgfqpoint{-5.122493in}{-0.373994in}}{\pgfqpoint{-5.122493in}{-0.364786in}}%
\pgfpathcurveto{\pgfqpoint{-5.122493in}{-0.355577in}}{\pgfqpoint{-5.126152in}{-0.346745in}}{\pgfqpoint{-5.132663in}{-0.340234in}}%
\pgfpathcurveto{\pgfqpoint{-5.139174in}{-0.333722in}}{\pgfqpoint{-5.148007in}{-0.330064in}}{\pgfqpoint{-5.157215in}{-0.330064in}}%
\pgfpathcurveto{\pgfqpoint{-5.166424in}{-0.330064in}}{\pgfqpoint{-5.175256in}{-0.333722in}}{\pgfqpoint{-5.181768in}{-0.340234in}}%
\pgfpathcurveto{\pgfqpoint{-5.188279in}{-0.346745in}}{\pgfqpoint{-5.191938in}{-0.355577in}}{\pgfqpoint{-5.191938in}{-0.364786in}}%
\pgfpathcurveto{\pgfqpoint{-5.191938in}{-0.373994in}}{\pgfqpoint{-5.188279in}{-0.382827in}}{\pgfqpoint{-5.181768in}{-0.389338in}}%
\pgfpathcurveto{\pgfqpoint{-5.175256in}{-0.395850in}}{\pgfqpoint{-5.166424in}{-0.399508in}}{\pgfqpoint{-5.157215in}{-0.399508in}}%
\pgfpathlineto{\pgfqpoint{-5.157215in}{-0.399508in}}%
\pgfpathclose%
\pgfusepath{stroke,fill}%
\end{pgfscope}%
\begin{pgfscope}%
\pgfpathrectangle{\pgfqpoint{1.374500in}{0.082500in}}{\pgfqpoint{2.419000in}{2.419000in}}%
\pgfusepath{clip}%
\pgfsetbuttcap%
\pgfsetroundjoin%
\definecolor{currentfill}{rgb}{1.000000,0.662745,0.054902}%
\pgfsetfillcolor{currentfill}%
\pgfsetfillopacity{0.519802}%
\pgfsetlinewidth{1.003750pt}%
\definecolor{currentstroke}{rgb}{1.000000,0.662745,0.054902}%
\pgfsetstrokecolor{currentstroke}%
\pgfsetstrokeopacity{0.519802}%
\pgfsetdash{}{0pt}%
\pgfpathmoveto{\pgfqpoint{6.501752in}{-0.399508in}}%
\pgfpathcurveto{\pgfqpoint{6.510960in}{-0.399508in}}{\pgfqpoint{6.519793in}{-0.395850in}}{\pgfqpoint{6.526304in}{-0.389338in}}%
\pgfpathcurveto{\pgfqpoint{6.532815in}{-0.382827in}}{\pgfqpoint{6.536474in}{-0.373994in}}{\pgfqpoint{6.536474in}{-0.364786in}}%
\pgfpathcurveto{\pgfqpoint{6.536474in}{-0.355577in}}{\pgfqpoint{6.532815in}{-0.346745in}}{\pgfqpoint{6.526304in}{-0.340234in}}%
\pgfpathcurveto{\pgfqpoint{6.519793in}{-0.333722in}}{\pgfqpoint{6.510960in}{-0.330064in}}{\pgfqpoint{6.501752in}{-0.330064in}}%
\pgfpathcurveto{\pgfqpoint{6.492543in}{-0.330064in}}{\pgfqpoint{6.483711in}{-0.333722in}}{\pgfqpoint{6.477199in}{-0.340234in}}%
\pgfpathcurveto{\pgfqpoint{6.470688in}{-0.346745in}}{\pgfqpoint{6.467029in}{-0.355577in}}{\pgfqpoint{6.467029in}{-0.364786in}}%
\pgfpathcurveto{\pgfqpoint{6.467029in}{-0.373994in}}{\pgfqpoint{6.470688in}{-0.382827in}}{\pgfqpoint{6.477199in}{-0.389338in}}%
\pgfpathcurveto{\pgfqpoint{6.483711in}{-0.395850in}}{\pgfqpoint{6.492543in}{-0.399508in}}{\pgfqpoint{6.501752in}{-0.399508in}}%
\pgfpathlineto{\pgfqpoint{6.501752in}{-0.399508in}}%
\pgfpathclose%
\pgfusepath{stroke,fill}%
\end{pgfscope}%
\begin{pgfscope}%
\pgfpathrectangle{\pgfqpoint{1.374500in}{0.082500in}}{\pgfqpoint{2.419000in}{2.419000in}}%
\pgfusepath{clip}%
\pgfsetbuttcap%
\pgfsetroundjoin%
\definecolor{currentfill}{rgb}{1.000000,0.662745,0.054902}%
\pgfsetfillcolor{currentfill}%
\pgfsetfillopacity{0.529171}%
\pgfsetlinewidth{1.003750pt}%
\definecolor{currentstroke}{rgb}{1.000000,0.662745,0.054902}%
\pgfsetstrokecolor{currentstroke}%
\pgfsetstrokeopacity{0.529171}%
\pgfsetdash{}{0pt}%
\pgfpathmoveto{\pgfqpoint{-1.788009in}{-0.596701in}}%
\pgfpathcurveto{\pgfqpoint{-1.778801in}{-0.596701in}}{\pgfqpoint{-1.769968in}{-0.593042in}}{\pgfqpoint{-1.763457in}{-0.586531in}}%
\pgfpathcurveto{\pgfqpoint{-1.756946in}{-0.580020in}}{\pgfqpoint{-1.753287in}{-0.571187in}}{\pgfqpoint{-1.753287in}{-0.561979in}}%
\pgfpathcurveto{\pgfqpoint{-1.753287in}{-0.552770in}}{\pgfqpoint{-1.756946in}{-0.543938in}}{\pgfqpoint{-1.763457in}{-0.537426in}}%
\pgfpathcurveto{\pgfqpoint{-1.769968in}{-0.530915in}}{\pgfqpoint{-1.778801in}{-0.527256in}}{\pgfqpoint{-1.788009in}{-0.527256in}}%
\pgfpathcurveto{\pgfqpoint{-1.797218in}{-0.527256in}}{\pgfqpoint{-1.806050in}{-0.530915in}}{\pgfqpoint{-1.812562in}{-0.537426in}}%
\pgfpathcurveto{\pgfqpoint{-1.819073in}{-0.543938in}}{\pgfqpoint{-1.822732in}{-0.552770in}}{\pgfqpoint{-1.822732in}{-0.561979in}}%
\pgfpathcurveto{\pgfqpoint{-1.822732in}{-0.571187in}}{\pgfqpoint{-1.819073in}{-0.580020in}}{\pgfqpoint{-1.812562in}{-0.586531in}}%
\pgfpathcurveto{\pgfqpoint{-1.806050in}{-0.593042in}}{\pgfqpoint{-1.797218in}{-0.596701in}}{\pgfqpoint{-1.788009in}{-0.596701in}}%
\pgfpathlineto{\pgfqpoint{-1.788009in}{-0.596701in}}%
\pgfpathclose%
\pgfusepath{stroke,fill}%
\end{pgfscope}%
\begin{pgfscope}%
\pgfpathrectangle{\pgfqpoint{1.374500in}{0.082500in}}{\pgfqpoint{2.419000in}{2.419000in}}%
\pgfusepath{clip}%
\pgfsetbuttcap%
\pgfsetroundjoin%
\definecolor{currentfill}{rgb}{1.000000,0.662745,0.054902}%
\pgfsetfillcolor{currentfill}%
\pgfsetfillopacity{0.529171}%
\pgfsetlinewidth{1.003750pt}%
\definecolor{currentstroke}{rgb}{1.000000,0.662745,0.054902}%
\pgfsetstrokecolor{currentstroke}%
\pgfsetstrokeopacity{0.529171}%
\pgfsetdash{}{0pt}%
\pgfpathmoveto{\pgfqpoint{4.165900in}{-0.596701in}}%
\pgfpathcurveto{\pgfqpoint{4.175108in}{-0.596701in}}{\pgfqpoint{4.183941in}{-0.593042in}}{\pgfqpoint{4.190452in}{-0.586531in}}%
\pgfpathcurveto{\pgfqpoint{4.196963in}{-0.580020in}}{\pgfqpoint{4.200622in}{-0.571187in}}{\pgfqpoint{4.200622in}{-0.561979in}}%
\pgfpathcurveto{\pgfqpoint{4.200622in}{-0.552770in}}{\pgfqpoint{4.196963in}{-0.543938in}}{\pgfqpoint{4.190452in}{-0.537426in}}%
\pgfpathcurveto{\pgfqpoint{4.183941in}{-0.530915in}}{\pgfqpoint{4.175108in}{-0.527256in}}{\pgfqpoint{4.165900in}{-0.527256in}}%
\pgfpathcurveto{\pgfqpoint{4.156691in}{-0.527256in}}{\pgfqpoint{4.147859in}{-0.530915in}}{\pgfqpoint{4.141347in}{-0.537426in}}%
\pgfpathcurveto{\pgfqpoint{4.134836in}{-0.543938in}}{\pgfqpoint{4.131177in}{-0.552770in}}{\pgfqpoint{4.131177in}{-0.561979in}}%
\pgfpathcurveto{\pgfqpoint{4.131177in}{-0.571187in}}{\pgfqpoint{4.134836in}{-0.580020in}}{\pgfqpoint{4.141347in}{-0.586531in}}%
\pgfpathcurveto{\pgfqpoint{4.147859in}{-0.593042in}}{\pgfqpoint{4.156691in}{-0.596701in}}{\pgfqpoint{4.165900in}{-0.596701in}}%
\pgfpathlineto{\pgfqpoint{4.165900in}{-0.596701in}}%
\pgfpathclose%
\pgfusepath{stroke,fill}%
\end{pgfscope}%
\begin{pgfscope}%
\pgfpathrectangle{\pgfqpoint{1.374500in}{0.082500in}}{\pgfqpoint{2.419000in}{2.419000in}}%
\pgfusepath{clip}%
\pgfsetbuttcap%
\pgfsetroundjoin%
\definecolor{currentfill}{rgb}{1.000000,0.662745,0.054902}%
\pgfsetfillcolor{currentfill}%
\pgfsetfillopacity{0.529171}%
\pgfsetlinewidth{1.003750pt}%
\definecolor{currentstroke}{rgb}{1.000000,0.662745,0.054902}%
\pgfsetstrokecolor{currentstroke}%
\pgfsetstrokeopacity{0.529171}%
\pgfsetdash{}{0pt}%
\pgfpathmoveto{\pgfqpoint{10.119809in}{-0.596701in}}%
\pgfpathcurveto{\pgfqpoint{10.129017in}{-0.596701in}}{\pgfqpoint{10.137850in}{-0.593042in}}{\pgfqpoint{10.144361in}{-0.586531in}}%
\pgfpathcurveto{\pgfqpoint{10.150872in}{-0.580020in}}{\pgfqpoint{10.154531in}{-0.571187in}}{\pgfqpoint{10.154531in}{-0.561979in}}%
\pgfpathcurveto{\pgfqpoint{10.154531in}{-0.552770in}}{\pgfqpoint{10.150872in}{-0.543938in}}{\pgfqpoint{10.144361in}{-0.537426in}}%
\pgfpathcurveto{\pgfqpoint{10.137850in}{-0.530915in}}{\pgfqpoint{10.129017in}{-0.527256in}}{\pgfqpoint{10.119809in}{-0.527256in}}%
\pgfpathcurveto{\pgfqpoint{10.110600in}{-0.527256in}}{\pgfqpoint{10.101768in}{-0.530915in}}{\pgfqpoint{10.095256in}{-0.537426in}}%
\pgfpathcurveto{\pgfqpoint{10.088745in}{-0.543938in}}{\pgfqpoint{10.085086in}{-0.552770in}}{\pgfqpoint{10.085086in}{-0.561979in}}%
\pgfpathcurveto{\pgfqpoint{10.085086in}{-0.571187in}}{\pgfqpoint{10.088745in}{-0.580020in}}{\pgfqpoint{10.095256in}{-0.586531in}}%
\pgfpathcurveto{\pgfqpoint{10.101768in}{-0.593042in}}{\pgfqpoint{10.110600in}{-0.596701in}}{\pgfqpoint{10.119809in}{-0.596701in}}%
\pgfpathlineto{\pgfqpoint{10.119809in}{-0.596701in}}%
\pgfpathclose%
\pgfusepath{stroke,fill}%
\end{pgfscope}%
\begin{pgfscope}%
\pgfpathrectangle{\pgfqpoint{1.374500in}{0.082500in}}{\pgfqpoint{2.419000in}{2.419000in}}%
\pgfusepath{clip}%
\pgfsetbuttcap%
\pgfsetroundjoin%
\definecolor{currentfill}{rgb}{1.000000,0.662745,0.054902}%
\pgfsetfillcolor{currentfill}%
\pgfsetfillopacity{0.538948}%
\pgfsetlinewidth{1.003750pt}%
\definecolor{currentstroke}{rgb}{1.000000,0.662745,0.054902}%
\pgfsetstrokecolor{currentstroke}%
\pgfsetstrokeopacity{0.538948}%
\pgfsetdash{}{0pt}%
\pgfpathmoveto{\pgfqpoint{-4.355603in}{-0.802495in}}%
\pgfpathcurveto{\pgfqpoint{-4.346394in}{-0.802495in}}{\pgfqpoint{-4.337562in}{-0.798837in}}{\pgfqpoint{-4.331050in}{-0.792325in}}%
\pgfpathcurveto{\pgfqpoint{-4.324539in}{-0.785814in}}{\pgfqpoint{-4.320881in}{-0.776981in}}{\pgfqpoint{-4.320881in}{-0.767773in}}%
\pgfpathcurveto{\pgfqpoint{-4.320881in}{-0.758565in}}{\pgfqpoint{-4.324539in}{-0.749732in}}{\pgfqpoint{-4.331050in}{-0.743221in}}%
\pgfpathcurveto{\pgfqpoint{-4.337562in}{-0.736709in}}{\pgfqpoint{-4.346394in}{-0.733051in}}{\pgfqpoint{-4.355603in}{-0.733051in}}%
\pgfpathcurveto{\pgfqpoint{-4.364811in}{-0.733051in}}{\pgfqpoint{-4.373644in}{-0.736709in}}{\pgfqpoint{-4.380155in}{-0.743221in}}%
\pgfpathcurveto{\pgfqpoint{-4.386666in}{-0.749732in}}{\pgfqpoint{-4.390325in}{-0.758565in}}{\pgfqpoint{-4.390325in}{-0.767773in}}%
\pgfpathcurveto{\pgfqpoint{-4.390325in}{-0.776981in}}{\pgfqpoint{-4.386666in}{-0.785814in}}{\pgfqpoint{-4.380155in}{-0.792325in}}%
\pgfpathcurveto{\pgfqpoint{-4.373644in}{-0.798837in}}{\pgfqpoint{-4.364811in}{-0.802495in}}{\pgfqpoint{-4.355603in}{-0.802495in}}%
\pgfpathlineto{\pgfqpoint{-4.355603in}{-0.802495in}}%
\pgfpathclose%
\pgfusepath{stroke,fill}%
\end{pgfscope}%
\begin{pgfscope}%
\pgfpathrectangle{\pgfqpoint{1.374500in}{0.082500in}}{\pgfqpoint{2.419000in}{2.419000in}}%
\pgfusepath{clip}%
\pgfsetbuttcap%
\pgfsetroundjoin%
\definecolor{currentfill}{rgb}{1.000000,0.662745,0.054902}%
\pgfsetfillcolor{currentfill}%
\pgfsetfillopacity{0.538948}%
\pgfsetlinewidth{1.003750pt}%
\definecolor{currentstroke}{rgb}{1.000000,0.662745,0.054902}%
\pgfsetstrokecolor{currentstroke}%
\pgfsetstrokeopacity{0.538948}%
\pgfsetdash{}{0pt}%
\pgfpathmoveto{\pgfqpoint{1.728159in}{-0.802495in}}%
\pgfpathcurveto{\pgfqpoint{1.737368in}{-0.802495in}}{\pgfqpoint{1.746200in}{-0.798837in}}{\pgfqpoint{1.752711in}{-0.792325in}}%
\pgfpathcurveto{\pgfqpoint{1.759223in}{-0.785814in}}{\pgfqpoint{1.762881in}{-0.776981in}}{\pgfqpoint{1.762881in}{-0.767773in}}%
\pgfpathcurveto{\pgfqpoint{1.762881in}{-0.758565in}}{\pgfqpoint{1.759223in}{-0.749732in}}{\pgfqpoint{1.752711in}{-0.743221in}}%
\pgfpathcurveto{\pgfqpoint{1.746200in}{-0.736709in}}{\pgfqpoint{1.737368in}{-0.733051in}}{\pgfqpoint{1.728159in}{-0.733051in}}%
\pgfpathcurveto{\pgfqpoint{1.718951in}{-0.733051in}}{\pgfqpoint{1.710118in}{-0.736709in}}{\pgfqpoint{1.703607in}{-0.743221in}}%
\pgfpathcurveto{\pgfqpoint{1.697095in}{-0.749732in}}{\pgfqpoint{1.693437in}{-0.758565in}}{\pgfqpoint{1.693437in}{-0.767773in}}%
\pgfpathcurveto{\pgfqpoint{1.693437in}{-0.776981in}}{\pgfqpoint{1.697095in}{-0.785814in}}{\pgfqpoint{1.703607in}{-0.792325in}}%
\pgfpathcurveto{\pgfqpoint{1.710118in}{-0.798837in}}{\pgfqpoint{1.718951in}{-0.802495in}}{\pgfqpoint{1.728159in}{-0.802495in}}%
\pgfpathlineto{\pgfqpoint{1.728159in}{-0.802495in}}%
\pgfpathclose%
\pgfusepath{stroke,fill}%
\end{pgfscope}%
\begin{pgfscope}%
\pgfpathrectangle{\pgfqpoint{1.374500in}{0.082500in}}{\pgfqpoint{2.419000in}{2.419000in}}%
\pgfusepath{clip}%
\pgfsetbuttcap%
\pgfsetroundjoin%
\definecolor{currentfill}{rgb}{1.000000,0.662745,0.054902}%
\pgfsetfillcolor{currentfill}%
\pgfsetfillopacity{0.538948}%
\pgfsetlinewidth{1.003750pt}%
\definecolor{currentstroke}{rgb}{1.000000,0.662745,0.054902}%
\pgfsetstrokecolor{currentstroke}%
\pgfsetstrokeopacity{0.538948}%
\pgfsetdash{}{0pt}%
\pgfpathmoveto{\pgfqpoint{7.811921in}{-0.802495in}}%
\pgfpathcurveto{\pgfqpoint{7.821129in}{-0.802495in}}{\pgfqpoint{7.829962in}{-0.798837in}}{\pgfqpoint{7.836473in}{-0.792325in}}%
\pgfpathcurveto{\pgfqpoint{7.842985in}{-0.785814in}}{\pgfqpoint{7.846643in}{-0.776981in}}{\pgfqpoint{7.846643in}{-0.767773in}}%
\pgfpathcurveto{\pgfqpoint{7.846643in}{-0.758565in}}{\pgfqpoint{7.842985in}{-0.749732in}}{\pgfqpoint{7.836473in}{-0.743221in}}%
\pgfpathcurveto{\pgfqpoint{7.829962in}{-0.736709in}}{\pgfqpoint{7.821129in}{-0.733051in}}{\pgfqpoint{7.811921in}{-0.733051in}}%
\pgfpathcurveto{\pgfqpoint{7.802713in}{-0.733051in}}{\pgfqpoint{7.793880in}{-0.736709in}}{\pgfqpoint{7.787369in}{-0.743221in}}%
\pgfpathcurveto{\pgfqpoint{7.780857in}{-0.749732in}}{\pgfqpoint{7.777199in}{-0.758565in}}{\pgfqpoint{7.777199in}{-0.767773in}}%
\pgfpathcurveto{\pgfqpoint{7.777199in}{-0.776981in}}{\pgfqpoint{7.780857in}{-0.785814in}}{\pgfqpoint{7.787369in}{-0.792325in}}%
\pgfpathcurveto{\pgfqpoint{7.793880in}{-0.798837in}}{\pgfqpoint{7.802713in}{-0.802495in}}{\pgfqpoint{7.811921in}{-0.802495in}}%
\pgfpathlineto{\pgfqpoint{7.811921in}{-0.802495in}}%
\pgfpathclose%
\pgfusepath{stroke,fill}%
\end{pgfscope}%
\begin{pgfscope}%
\pgfpathrectangle{\pgfqpoint{1.374500in}{0.082500in}}{\pgfqpoint{2.419000in}{2.419000in}}%
\pgfusepath{clip}%
\pgfsetbuttcap%
\pgfsetroundjoin%
\definecolor{currentfill}{rgb}{1.000000,0.662745,0.054902}%
\pgfsetfillcolor{currentfill}%
\pgfsetfillopacity{0.549161}%
\pgfsetlinewidth{1.003750pt}%
\definecolor{currentstroke}{rgb}{1.000000,0.662745,0.054902}%
\pgfsetstrokecolor{currentstroke}%
\pgfsetstrokeopacity{0.549161}%
\pgfsetdash{}{0pt}%
\pgfpathmoveto{\pgfqpoint{-0.818285in}{-1.017466in}}%
\pgfpathcurveto{\pgfqpoint{-0.809076in}{-1.017466in}}{\pgfqpoint{-0.800244in}{-1.013808in}}{\pgfqpoint{-0.793733in}{-1.007296in}}%
\pgfpathcurveto{\pgfqpoint{-0.787221in}{-1.000785in}}{\pgfqpoint{-0.783563in}{-0.991952in}}{\pgfqpoint{-0.783563in}{-0.982744in}}%
\pgfpathcurveto{\pgfqpoint{-0.783563in}{-0.973536in}}{\pgfqpoint{-0.787221in}{-0.964703in}}{\pgfqpoint{-0.793733in}{-0.958192in}}%
\pgfpathcurveto{\pgfqpoint{-0.800244in}{-0.951680in}}{\pgfqpoint{-0.809076in}{-0.948022in}}{\pgfqpoint{-0.818285in}{-0.948022in}}%
\pgfpathcurveto{\pgfqpoint{-0.827493in}{-0.948022in}}{\pgfqpoint{-0.836326in}{-0.951680in}}{\pgfqpoint{-0.842837in}{-0.958192in}}%
\pgfpathcurveto{\pgfqpoint{-0.849349in}{-0.964703in}}{\pgfqpoint{-0.853007in}{-0.973536in}}{\pgfqpoint{-0.853007in}{-0.982744in}}%
\pgfpathcurveto{\pgfqpoint{-0.853007in}{-0.991952in}}{\pgfqpoint{-0.849349in}{-1.000785in}}{\pgfqpoint{-0.842837in}{-1.007296in}}%
\pgfpathcurveto{\pgfqpoint{-0.836326in}{-1.013808in}}{\pgfqpoint{-0.827493in}{-1.017466in}}{\pgfqpoint{-0.818285in}{-1.017466in}}%
\pgfpathlineto{\pgfqpoint{-0.818285in}{-1.017466in}}%
\pgfpathclose%
\pgfusepath{stroke,fill}%
\end{pgfscope}%
\begin{pgfscope}%
\pgfpathrectangle{\pgfqpoint{1.374500in}{0.082500in}}{\pgfqpoint{2.419000in}{2.419000in}}%
\pgfusepath{clip}%
\pgfsetbuttcap%
\pgfsetroundjoin%
\definecolor{currentfill}{rgb}{1.000000,0.662745,0.054902}%
\pgfsetfillcolor{currentfill}%
\pgfsetfillopacity{0.549161}%
\pgfsetlinewidth{1.003750pt}%
\definecolor{currentstroke}{rgb}{1.000000,0.662745,0.054902}%
\pgfsetstrokecolor{currentstroke}%
\pgfsetstrokeopacity{0.549161}%
\pgfsetdash{}{0pt}%
\pgfpathmoveto{\pgfqpoint{5.401120in}{-1.017466in}}%
\pgfpathcurveto{\pgfqpoint{5.410329in}{-1.017466in}}{\pgfqpoint{5.419161in}{-1.013808in}}{\pgfqpoint{5.425673in}{-1.007296in}}%
\pgfpathcurveto{\pgfqpoint{5.432184in}{-1.000785in}}{\pgfqpoint{5.435843in}{-0.991952in}}{\pgfqpoint{5.435843in}{-0.982744in}}%
\pgfpathcurveto{\pgfqpoint{5.435843in}{-0.973536in}}{\pgfqpoint{5.432184in}{-0.964703in}}{\pgfqpoint{5.425673in}{-0.958192in}}%
\pgfpathcurveto{\pgfqpoint{5.419161in}{-0.951680in}}{\pgfqpoint{5.410329in}{-0.948022in}}{\pgfqpoint{5.401120in}{-0.948022in}}%
\pgfpathcurveto{\pgfqpoint{5.391912in}{-0.948022in}}{\pgfqpoint{5.383079in}{-0.951680in}}{\pgfqpoint{5.376568in}{-0.958192in}}%
\pgfpathcurveto{\pgfqpoint{5.370057in}{-0.964703in}}{\pgfqpoint{5.366398in}{-0.973536in}}{\pgfqpoint{5.366398in}{-0.982744in}}%
\pgfpathcurveto{\pgfqpoint{5.366398in}{-0.991952in}}{\pgfqpoint{5.370057in}{-1.000785in}}{\pgfqpoint{5.376568in}{-1.007296in}}%
\pgfpathcurveto{\pgfqpoint{5.383079in}{-1.013808in}}{\pgfqpoint{5.391912in}{-1.017466in}}{\pgfqpoint{5.401120in}{-1.017466in}}%
\pgfpathlineto{\pgfqpoint{5.401120in}{-1.017466in}}%
\pgfpathclose%
\pgfusepath{stroke,fill}%
\end{pgfscope}%
\begin{pgfscope}%
\pgfpathrectangle{\pgfqpoint{1.374500in}{0.082500in}}{\pgfqpoint{2.419000in}{2.419000in}}%
\pgfusepath{clip}%
\pgfsetbuttcap%
\pgfsetroundjoin%
\definecolor{currentfill}{rgb}{1.000000,0.662745,0.054902}%
\pgfsetfillcolor{currentfill}%
\pgfsetfillopacity{0.549161}%
\pgfsetlinewidth{1.003750pt}%
\definecolor{currentstroke}{rgb}{1.000000,0.662745,0.054902}%
\pgfsetstrokecolor{currentstroke}%
\pgfsetstrokeopacity{0.549161}%
\pgfsetdash{}{0pt}%
\pgfpathmoveto{\pgfqpoint{11.620525in}{-1.017466in}}%
\pgfpathcurveto{\pgfqpoint{11.629734in}{-1.017466in}}{\pgfqpoint{11.638566in}{-1.013808in}}{\pgfqpoint{11.645078in}{-1.007296in}}%
\pgfpathcurveto{\pgfqpoint{11.651589in}{-1.000785in}}{\pgfqpoint{11.655248in}{-0.991952in}}{\pgfqpoint{11.655248in}{-0.982744in}}%
\pgfpathcurveto{\pgfqpoint{11.655248in}{-0.973536in}}{\pgfqpoint{11.651589in}{-0.964703in}}{\pgfqpoint{11.645078in}{-0.958192in}}%
\pgfpathcurveto{\pgfqpoint{11.638566in}{-0.951680in}}{\pgfqpoint{11.629734in}{-0.948022in}}{\pgfqpoint{11.620525in}{-0.948022in}}%
\pgfpathcurveto{\pgfqpoint{11.611317in}{-0.948022in}}{\pgfqpoint{11.602484in}{-0.951680in}}{\pgfqpoint{11.595973in}{-0.958192in}}%
\pgfpathcurveto{\pgfqpoint{11.589462in}{-0.964703in}}{\pgfqpoint{11.585803in}{-0.973536in}}{\pgfqpoint{11.585803in}{-0.982744in}}%
\pgfpathcurveto{\pgfqpoint{11.585803in}{-0.991952in}}{\pgfqpoint{11.589462in}{-1.000785in}}{\pgfqpoint{11.595973in}{-1.007296in}}%
\pgfpathcurveto{\pgfqpoint{11.602484in}{-1.013808in}}{\pgfqpoint{11.611317in}{-1.017466in}}{\pgfqpoint{11.620525in}{-1.017466in}}%
\pgfpathlineto{\pgfqpoint{11.620525in}{-1.017466in}}%
\pgfpathclose%
\pgfusepath{stroke,fill}%
\end{pgfscope}%
\begin{pgfscope}%
\pgfpathrectangle{\pgfqpoint{1.374500in}{0.082500in}}{\pgfqpoint{2.419000in}{2.419000in}}%
\pgfusepath{clip}%
\pgfsetbuttcap%
\pgfsetroundjoin%
\definecolor{currentfill}{rgb}{1.000000,0.662745,0.054902}%
\pgfsetfillcolor{currentfill}%
\pgfsetfillopacity{0.559840}%
\pgfsetlinewidth{1.003750pt}%
\definecolor{currentstroke}{rgb}{1.000000,0.662745,0.054902}%
\pgfsetstrokecolor{currentstroke}%
\pgfsetstrokeopacity{0.559840}%
\pgfsetdash{}{0pt}%
\pgfpathmoveto{\pgfqpoint{-3.480869in}{-1.242242in}}%
\pgfpathcurveto{\pgfqpoint{-3.471661in}{-1.242242in}}{\pgfqpoint{-3.462828in}{-1.238583in}}{\pgfqpoint{-3.456317in}{-1.232072in}}%
\pgfpathcurveto{\pgfqpoint{-3.449805in}{-1.225561in}}{\pgfqpoint{-3.446147in}{-1.216728in}}{\pgfqpoint{-3.446147in}{-1.207520in}}%
\pgfpathcurveto{\pgfqpoint{-3.446147in}{-1.198311in}}{\pgfqpoint{-3.449805in}{-1.189479in}}{\pgfqpoint{-3.456317in}{-1.182967in}}%
\pgfpathcurveto{\pgfqpoint{-3.462828in}{-1.176456in}}{\pgfqpoint{-3.471661in}{-1.172797in}}{\pgfqpoint{-3.480869in}{-1.172797in}}%
\pgfpathcurveto{\pgfqpoint{-3.490078in}{-1.172797in}}{\pgfqpoint{-3.498910in}{-1.176456in}}{\pgfqpoint{-3.505421in}{-1.182967in}}%
\pgfpathcurveto{\pgfqpoint{-3.511933in}{-1.189479in}}{\pgfqpoint{-3.515591in}{-1.198311in}}{\pgfqpoint{-3.515591in}{-1.207520in}}%
\pgfpathcurveto{\pgfqpoint{-3.515591in}{-1.216728in}}{\pgfqpoint{-3.511933in}{-1.225561in}}{\pgfqpoint{-3.505421in}{-1.232072in}}%
\pgfpathcurveto{\pgfqpoint{-3.498910in}{-1.238583in}}{\pgfqpoint{-3.490078in}{-1.242242in}}{\pgfqpoint{-3.480869in}{-1.242242in}}%
\pgfpathlineto{\pgfqpoint{-3.480869in}{-1.242242in}}%
\pgfpathclose%
\pgfusepath{stroke,fill}%
\end{pgfscope}%
\begin{pgfscope}%
\pgfpathrectangle{\pgfqpoint{1.374500in}{0.082500in}}{\pgfqpoint{2.419000in}{2.419000in}}%
\pgfusepath{clip}%
\pgfsetbuttcap%
\pgfsetroundjoin%
\definecolor{currentfill}{rgb}{1.000000,0.662745,0.054902}%
\pgfsetfillcolor{currentfill}%
\pgfsetfillopacity{0.559840}%
\pgfsetlinewidth{1.003750pt}%
\definecolor{currentstroke}{rgb}{1.000000,0.662745,0.054902}%
\pgfsetstrokecolor{currentstroke}%
\pgfsetstrokeopacity{0.559840}%
\pgfsetdash{}{0pt}%
\pgfpathmoveto{\pgfqpoint{2.880366in}{-1.242242in}}%
\pgfpathcurveto{\pgfqpoint{2.889574in}{-1.242242in}}{\pgfqpoint{2.898407in}{-1.238583in}}{\pgfqpoint{2.904918in}{-1.232072in}}%
\pgfpathcurveto{\pgfqpoint{2.911430in}{-1.225561in}}{\pgfqpoint{2.915088in}{-1.216728in}}{\pgfqpoint{2.915088in}{-1.207520in}}%
\pgfpathcurveto{\pgfqpoint{2.915088in}{-1.198311in}}{\pgfqpoint{2.911430in}{-1.189479in}}{\pgfqpoint{2.904918in}{-1.182967in}}%
\pgfpathcurveto{\pgfqpoint{2.898407in}{-1.176456in}}{\pgfqpoint{2.889574in}{-1.172797in}}{\pgfqpoint{2.880366in}{-1.172797in}}%
\pgfpathcurveto{\pgfqpoint{2.871157in}{-1.172797in}}{\pgfqpoint{2.862325in}{-1.176456in}}{\pgfqpoint{2.855814in}{-1.182967in}}%
\pgfpathcurveto{\pgfqpoint{2.849302in}{-1.189479in}}{\pgfqpoint{2.845644in}{-1.198311in}}{\pgfqpoint{2.845644in}{-1.207520in}}%
\pgfpathcurveto{\pgfqpoint{2.845644in}{-1.216728in}}{\pgfqpoint{2.849302in}{-1.225561in}}{\pgfqpoint{2.855814in}{-1.232072in}}%
\pgfpathcurveto{\pgfqpoint{2.862325in}{-1.238583in}}{\pgfqpoint{2.871157in}{-1.242242in}}{\pgfqpoint{2.880366in}{-1.242242in}}%
\pgfpathlineto{\pgfqpoint{2.880366in}{-1.242242in}}%
\pgfpathclose%
\pgfusepath{stroke,fill}%
\end{pgfscope}%
\begin{pgfscope}%
\pgfpathrectangle{\pgfqpoint{1.374500in}{0.082500in}}{\pgfqpoint{2.419000in}{2.419000in}}%
\pgfusepath{clip}%
\pgfsetbuttcap%
\pgfsetroundjoin%
\definecolor{currentfill}{rgb}{1.000000,0.662745,0.054902}%
\pgfsetfillcolor{currentfill}%
\pgfsetfillopacity{0.559840}%
\pgfsetlinewidth{1.003750pt}%
\definecolor{currentstroke}{rgb}{1.000000,0.662745,0.054902}%
\pgfsetstrokecolor{currentstroke}%
\pgfsetstrokeopacity{0.559840}%
\pgfsetdash{}{0pt}%
\pgfpathmoveto{\pgfqpoint{9.241601in}{-1.242242in}}%
\pgfpathcurveto{\pgfqpoint{9.250809in}{-1.242242in}}{\pgfqpoint{9.259642in}{-1.238583in}}{\pgfqpoint{9.266153in}{-1.232072in}}%
\pgfpathcurveto{\pgfqpoint{9.272664in}{-1.225561in}}{\pgfqpoint{9.276323in}{-1.216728in}}{\pgfqpoint{9.276323in}{-1.207520in}}%
\pgfpathcurveto{\pgfqpoint{9.276323in}{-1.198311in}}{\pgfqpoint{9.272664in}{-1.189479in}}{\pgfqpoint{9.266153in}{-1.182967in}}%
\pgfpathcurveto{\pgfqpoint{9.259642in}{-1.176456in}}{\pgfqpoint{9.250809in}{-1.172797in}}{\pgfqpoint{9.241601in}{-1.172797in}}%
\pgfpathcurveto{\pgfqpoint{9.232392in}{-1.172797in}}{\pgfqpoint{9.223560in}{-1.176456in}}{\pgfqpoint{9.217048in}{-1.182967in}}%
\pgfpathcurveto{\pgfqpoint{9.210537in}{-1.189479in}}{\pgfqpoint{9.206879in}{-1.198311in}}{\pgfqpoint{9.206879in}{-1.207520in}}%
\pgfpathcurveto{\pgfqpoint{9.206879in}{-1.216728in}}{\pgfqpoint{9.210537in}{-1.225561in}}{\pgfqpoint{9.217048in}{-1.232072in}}%
\pgfpathcurveto{\pgfqpoint{9.223560in}{-1.238583in}}{\pgfqpoint{9.232392in}{-1.242242in}}{\pgfqpoint{9.241601in}{-1.242242in}}%
\pgfpathlineto{\pgfqpoint{9.241601in}{-1.242242in}}%
\pgfpathclose%
\pgfusepath{stroke,fill}%
\end{pgfscope}%
\begin{pgfscope}%
\pgfpathrectangle{\pgfqpoint{1.374500in}{0.082500in}}{\pgfqpoint{2.419000in}{2.419000in}}%
\pgfusepath{clip}%
\pgfsetbuttcap%
\pgfsetroundjoin%
\definecolor{currentfill}{rgb}{1.000000,0.662745,0.054902}%
\pgfsetfillcolor{currentfill}%
\pgfsetfillopacity{0.571017}%
\pgfsetlinewidth{1.003750pt}%
\definecolor{currentstroke}{rgb}{1.000000,0.662745,0.054902}%
\pgfsetstrokecolor{currentstroke}%
\pgfsetstrokeopacity{0.571017}%
\pgfsetdash{}{0pt}%
\pgfpathmoveto{\pgfqpoint{0.241960in}{-1.477508in}}%
\pgfpathcurveto{\pgfqpoint{0.251168in}{-1.477508in}}{\pgfqpoint{0.260001in}{-1.473850in}}{\pgfqpoint{0.266512in}{-1.467338in}}%
\pgfpathcurveto{\pgfqpoint{0.273024in}{-1.460827in}}{\pgfqpoint{0.276682in}{-1.451995in}}{\pgfqpoint{0.276682in}{-1.442786in}}%
\pgfpathcurveto{\pgfqpoint{0.276682in}{-1.433578in}}{\pgfqpoint{0.273024in}{-1.424745in}}{\pgfqpoint{0.266512in}{-1.418234in}}%
\pgfpathcurveto{\pgfqpoint{0.260001in}{-1.411722in}}{\pgfqpoint{0.251168in}{-1.408064in}}{\pgfqpoint{0.241960in}{-1.408064in}}%
\pgfpathcurveto{\pgfqpoint{0.232751in}{-1.408064in}}{\pgfqpoint{0.223919in}{-1.411722in}}{\pgfqpoint{0.217408in}{-1.418234in}}%
\pgfpathcurveto{\pgfqpoint{0.210896in}{-1.424745in}}{\pgfqpoint{0.207238in}{-1.433578in}}{\pgfqpoint{0.207238in}{-1.442786in}}%
\pgfpathcurveto{\pgfqpoint{0.207238in}{-1.451995in}}{\pgfqpoint{0.210896in}{-1.460827in}}{\pgfqpoint{0.217408in}{-1.467338in}}%
\pgfpathcurveto{\pgfqpoint{0.223919in}{-1.473850in}}{\pgfqpoint{0.232751in}{-1.477508in}}{\pgfqpoint{0.241960in}{-1.477508in}}%
\pgfpathlineto{\pgfqpoint{0.241960in}{-1.477508in}}%
\pgfpathclose%
\pgfusepath{stroke,fill}%
\end{pgfscope}%
\begin{pgfscope}%
\pgfpathrectangle{\pgfqpoint{1.374500in}{0.082500in}}{\pgfqpoint{2.419000in}{2.419000in}}%
\pgfusepath{clip}%
\pgfsetbuttcap%
\pgfsetroundjoin%
\definecolor{currentfill}{rgb}{1.000000,0.662745,0.054902}%
\pgfsetfillcolor{currentfill}%
\pgfsetfillopacity{0.571017}%
\pgfsetlinewidth{1.003750pt}%
\definecolor{currentstroke}{rgb}{1.000000,0.662745,0.054902}%
\pgfsetstrokecolor{currentstroke}%
\pgfsetstrokeopacity{0.571017}%
\pgfsetdash{}{0pt}%
\pgfpathmoveto{\pgfqpoint{-6.267724in}{-1.477508in}}%
\pgfpathcurveto{\pgfqpoint{-6.258516in}{-1.477508in}}{\pgfqpoint{-6.249683in}{-1.473850in}}{\pgfqpoint{-6.243172in}{-1.467338in}}%
\pgfpathcurveto{\pgfqpoint{-6.236661in}{-1.460827in}}{\pgfqpoint{-6.233002in}{-1.451995in}}{\pgfqpoint{-6.233002in}{-1.442786in}}%
\pgfpathcurveto{\pgfqpoint{-6.233002in}{-1.433578in}}{\pgfqpoint{-6.236661in}{-1.424745in}}{\pgfqpoint{-6.243172in}{-1.418234in}}%
\pgfpathcurveto{\pgfqpoint{-6.249683in}{-1.411722in}}{\pgfqpoint{-6.258516in}{-1.408064in}}{\pgfqpoint{-6.267724in}{-1.408064in}}%
\pgfpathcurveto{\pgfqpoint{-6.276933in}{-1.408064in}}{\pgfqpoint{-6.285765in}{-1.411722in}}{\pgfqpoint{-6.292277in}{-1.418234in}}%
\pgfpathcurveto{\pgfqpoint{-6.298788in}{-1.424745in}}{\pgfqpoint{-6.302447in}{-1.433578in}}{\pgfqpoint{-6.302447in}{-1.442786in}}%
\pgfpathcurveto{\pgfqpoint{-6.302447in}{-1.451995in}}{\pgfqpoint{-6.298788in}{-1.460827in}}{\pgfqpoint{-6.292277in}{-1.467338in}}%
\pgfpathcurveto{\pgfqpoint{-6.285765in}{-1.473850in}}{\pgfqpoint{-6.276933in}{-1.477508in}}{\pgfqpoint{-6.267724in}{-1.477508in}}%
\pgfpathlineto{\pgfqpoint{-6.267724in}{-1.477508in}}%
\pgfpathclose%
\pgfusepath{stroke,fill}%
\end{pgfscope}%
\begin{pgfscope}%
\pgfpathrectangle{\pgfqpoint{1.374500in}{0.082500in}}{\pgfqpoint{2.419000in}{2.419000in}}%
\pgfusepath{clip}%
\pgfsetbuttcap%
\pgfsetroundjoin%
\definecolor{currentfill}{rgb}{1.000000,0.662745,0.054902}%
\pgfsetfillcolor{currentfill}%
\pgfsetfillopacity{0.571017}%
\pgfsetlinewidth{1.003750pt}%
\definecolor{currentstroke}{rgb}{1.000000,0.662745,0.054902}%
\pgfsetstrokecolor{currentstroke}%
\pgfsetstrokeopacity{0.571017}%
\pgfsetdash{}{0pt}%
\pgfpathmoveto{\pgfqpoint{6.751644in}{-1.477508in}}%
\pgfpathcurveto{\pgfqpoint{6.760853in}{-1.477508in}}{\pgfqpoint{6.769685in}{-1.473850in}}{\pgfqpoint{6.776197in}{-1.467338in}}%
\pgfpathcurveto{\pgfqpoint{6.782708in}{-1.460827in}}{\pgfqpoint{6.786366in}{-1.451995in}}{\pgfqpoint{6.786366in}{-1.442786in}}%
\pgfpathcurveto{\pgfqpoint{6.786366in}{-1.433578in}}{\pgfqpoint{6.782708in}{-1.424745in}}{\pgfqpoint{6.776197in}{-1.418234in}}%
\pgfpathcurveto{\pgfqpoint{6.769685in}{-1.411722in}}{\pgfqpoint{6.760853in}{-1.408064in}}{\pgfqpoint{6.751644in}{-1.408064in}}%
\pgfpathcurveto{\pgfqpoint{6.742436in}{-1.408064in}}{\pgfqpoint{6.733603in}{-1.411722in}}{\pgfqpoint{6.727092in}{-1.418234in}}%
\pgfpathcurveto{\pgfqpoint{6.720581in}{-1.424745in}}{\pgfqpoint{6.716922in}{-1.433578in}}{\pgfqpoint{6.716922in}{-1.442786in}}%
\pgfpathcurveto{\pgfqpoint{6.716922in}{-1.451995in}}{\pgfqpoint{6.720581in}{-1.460827in}}{\pgfqpoint{6.727092in}{-1.467338in}}%
\pgfpathcurveto{\pgfqpoint{6.733603in}{-1.473850in}}{\pgfqpoint{6.742436in}{-1.477508in}}{\pgfqpoint{6.751644in}{-1.477508in}}%
\pgfpathlineto{\pgfqpoint{6.751644in}{-1.477508in}}%
\pgfpathclose%
\pgfusepath{stroke,fill}%
\end{pgfscope}%
\begin{pgfscope}%
\pgfpathrectangle{\pgfqpoint{1.374500in}{0.082500in}}{\pgfqpoint{2.419000in}{2.419000in}}%
\pgfusepath{clip}%
\pgfsetbuttcap%
\pgfsetroundjoin%
\definecolor{currentfill}{rgb}{1.000000,0.662745,0.054902}%
\pgfsetfillcolor{currentfill}%
\pgfsetfillopacity{0.582729}%
\pgfsetlinewidth{1.003750pt}%
\definecolor{currentstroke}{rgb}{1.000000,0.662745,0.054902}%
\pgfsetstrokecolor{currentstroke}%
\pgfsetstrokeopacity{0.582729}%
\pgfsetdash{}{0pt}%
\pgfpathmoveto{\pgfqpoint{4.142697in}{-1.724018in}}%
\pgfpathcurveto{\pgfqpoint{4.151905in}{-1.724018in}}{\pgfqpoint{4.160738in}{-1.720359in}}{\pgfqpoint{4.167249in}{-1.713848in}}%
\pgfpathcurveto{\pgfqpoint{4.173760in}{-1.707337in}}{\pgfqpoint{4.177419in}{-1.698504in}}{\pgfqpoint{4.177419in}{-1.689296in}}%
\pgfpathcurveto{\pgfqpoint{4.177419in}{-1.680087in}}{\pgfqpoint{4.173760in}{-1.671255in}}{\pgfqpoint{4.167249in}{-1.664743in}}%
\pgfpathcurveto{\pgfqpoint{4.160738in}{-1.658232in}}{\pgfqpoint{4.151905in}{-1.654573in}}{\pgfqpoint{4.142697in}{-1.654573in}}%
\pgfpathcurveto{\pgfqpoint{4.133488in}{-1.654573in}}{\pgfqpoint{4.124656in}{-1.658232in}}{\pgfqpoint{4.118144in}{-1.664743in}}%
\pgfpathcurveto{\pgfqpoint{4.111633in}{-1.671255in}}{\pgfqpoint{4.107975in}{-1.680087in}}{\pgfqpoint{4.107975in}{-1.689296in}}%
\pgfpathcurveto{\pgfqpoint{4.107975in}{-1.698504in}}{\pgfqpoint{4.111633in}{-1.707337in}}{\pgfqpoint{4.118144in}{-1.713848in}}%
\pgfpathcurveto{\pgfqpoint{4.124656in}{-1.720359in}}{\pgfqpoint{4.133488in}{-1.724018in}}{\pgfqpoint{4.142697in}{-1.724018in}}%
\pgfpathlineto{\pgfqpoint{4.142697in}{-1.724018in}}%
\pgfpathclose%
\pgfusepath{stroke,fill}%
\end{pgfscope}%
\begin{pgfscope}%
\pgfpathrectangle{\pgfqpoint{1.374500in}{0.082500in}}{\pgfqpoint{2.419000in}{2.419000in}}%
\pgfusepath{clip}%
\pgfsetbuttcap%
\pgfsetroundjoin%
\definecolor{currentfill}{rgb}{1.000000,0.662745,0.054902}%
\pgfsetfillcolor{currentfill}%
\pgfsetfillopacity{0.582729}%
\pgfsetlinewidth{1.003750pt}%
\definecolor{currentstroke}{rgb}{1.000000,0.662745,0.054902}%
\pgfsetstrokecolor{currentstroke}%
\pgfsetstrokeopacity{0.582729}%
\pgfsetdash{}{0pt}%
\pgfpathmoveto{\pgfqpoint{-2.522531in}{-1.724018in}}%
\pgfpathcurveto{\pgfqpoint{-2.513323in}{-1.724018in}}{\pgfqpoint{-2.504490in}{-1.720359in}}{\pgfqpoint{-2.497979in}{-1.713848in}}%
\pgfpathcurveto{\pgfqpoint{-2.491467in}{-1.707337in}}{\pgfqpoint{-2.487809in}{-1.698504in}}{\pgfqpoint{-2.487809in}{-1.689296in}}%
\pgfpathcurveto{\pgfqpoint{-2.487809in}{-1.680087in}}{\pgfqpoint{-2.491467in}{-1.671255in}}{\pgfqpoint{-2.497979in}{-1.664743in}}%
\pgfpathcurveto{\pgfqpoint{-2.504490in}{-1.658232in}}{\pgfqpoint{-2.513323in}{-1.654573in}}{\pgfqpoint{-2.522531in}{-1.654573in}}%
\pgfpathcurveto{\pgfqpoint{-2.531740in}{-1.654573in}}{\pgfqpoint{-2.540572in}{-1.658232in}}{\pgfqpoint{-2.547083in}{-1.664743in}}%
\pgfpathcurveto{\pgfqpoint{-2.553595in}{-1.671255in}}{\pgfqpoint{-2.557253in}{-1.680087in}}{\pgfqpoint{-2.557253in}{-1.689296in}}%
\pgfpathcurveto{\pgfqpoint{-2.557253in}{-1.698504in}}{\pgfqpoint{-2.553595in}{-1.707337in}}{\pgfqpoint{-2.547083in}{-1.713848in}}%
\pgfpathcurveto{\pgfqpoint{-2.540572in}{-1.720359in}}{\pgfqpoint{-2.531740in}{-1.724018in}}{\pgfqpoint{-2.522531in}{-1.724018in}}%
\pgfpathlineto{\pgfqpoint{-2.522531in}{-1.724018in}}%
\pgfpathclose%
\pgfusepath{stroke,fill}%
\end{pgfscope}%
\begin{pgfscope}%
\pgfpathrectangle{\pgfqpoint{1.374500in}{0.082500in}}{\pgfqpoint{2.419000in}{2.419000in}}%
\pgfusepath{clip}%
\pgfsetbuttcap%
\pgfsetroundjoin%
\definecolor{currentfill}{rgb}{1.000000,0.662745,0.054902}%
\pgfsetfillcolor{currentfill}%
\pgfsetfillopacity{0.582729}%
\pgfsetlinewidth{1.003750pt}%
\definecolor{currentstroke}{rgb}{1.000000,0.662745,0.054902}%
\pgfsetstrokecolor{currentstroke}%
\pgfsetstrokeopacity{0.582729}%
\pgfsetdash{}{0pt}%
\pgfpathmoveto{\pgfqpoint{10.807925in}{-1.724018in}}%
\pgfpathcurveto{\pgfqpoint{10.817133in}{-1.724018in}}{\pgfqpoint{10.825966in}{-1.720359in}}{\pgfqpoint{10.832477in}{-1.713848in}}%
\pgfpathcurveto{\pgfqpoint{10.838988in}{-1.707337in}}{\pgfqpoint{10.842647in}{-1.698504in}}{\pgfqpoint{10.842647in}{-1.689296in}}%
\pgfpathcurveto{\pgfqpoint{10.842647in}{-1.680087in}}{\pgfqpoint{10.838988in}{-1.671255in}}{\pgfqpoint{10.832477in}{-1.664743in}}%
\pgfpathcurveto{\pgfqpoint{10.825966in}{-1.658232in}}{\pgfqpoint{10.817133in}{-1.654573in}}{\pgfqpoint{10.807925in}{-1.654573in}}%
\pgfpathcurveto{\pgfqpoint{10.798716in}{-1.654573in}}{\pgfqpoint{10.789884in}{-1.658232in}}{\pgfqpoint{10.783372in}{-1.664743in}}%
\pgfpathcurveto{\pgfqpoint{10.776861in}{-1.671255in}}{\pgfqpoint{10.773203in}{-1.680087in}}{\pgfqpoint{10.773203in}{-1.689296in}}%
\pgfpathcurveto{\pgfqpoint{10.773203in}{-1.698504in}}{\pgfqpoint{10.776861in}{-1.707337in}}{\pgfqpoint{10.783372in}{-1.713848in}}%
\pgfpathcurveto{\pgfqpoint{10.789884in}{-1.720359in}}{\pgfqpoint{10.798716in}{-1.724018in}}{\pgfqpoint{10.807925in}{-1.724018in}}%
\pgfpathlineto{\pgfqpoint{10.807925in}{-1.724018in}}%
\pgfpathclose%
\pgfusepath{stroke,fill}%
\end{pgfscope}%
\begin{pgfscope}%
\pgfpathrectangle{\pgfqpoint{1.374500in}{0.082500in}}{\pgfqpoint{2.419000in}{2.419000in}}%
\pgfusepath{clip}%
\pgfsetbuttcap%
\pgfsetroundjoin%
\definecolor{currentfill}{rgb}{1.000000,0.662745,0.054902}%
\pgfsetfillcolor{currentfill}%
\pgfsetfillopacity{0.595014}%
\pgfsetlinewidth{1.003750pt}%
\definecolor{currentstroke}{rgb}{1.000000,0.662745,0.054902}%
\pgfsetstrokecolor{currentstroke}%
\pgfsetstrokeopacity{0.595014}%
\pgfsetdash{}{0pt}%
\pgfpathmoveto{\pgfqpoint{-5.422367in}{-1.982596in}}%
\pgfpathcurveto{\pgfqpoint{-5.413158in}{-1.982596in}}{\pgfqpoint{-5.404326in}{-1.978938in}}{\pgfqpoint{-5.397814in}{-1.972426in}}%
\pgfpathcurveto{\pgfqpoint{-5.391303in}{-1.965915in}}{\pgfqpoint{-5.387644in}{-1.957082in}}{\pgfqpoint{-5.387644in}{-1.947874in}}%
\pgfpathcurveto{\pgfqpoint{-5.387644in}{-1.938666in}}{\pgfqpoint{-5.391303in}{-1.929833in}}{\pgfqpoint{-5.397814in}{-1.923322in}}%
\pgfpathcurveto{\pgfqpoint{-5.404326in}{-1.916810in}}{\pgfqpoint{-5.413158in}{-1.913152in}}{\pgfqpoint{-5.422367in}{-1.913152in}}%
\pgfpathcurveto{\pgfqpoint{-5.431575in}{-1.913152in}}{\pgfqpoint{-5.440408in}{-1.916810in}}{\pgfqpoint{-5.446919in}{-1.923322in}}%
\pgfpathcurveto{\pgfqpoint{-5.453430in}{-1.929833in}}{\pgfqpoint{-5.457089in}{-1.938666in}}{\pgfqpoint{-5.457089in}{-1.947874in}}%
\pgfpathcurveto{\pgfqpoint{-5.457089in}{-1.957082in}}{\pgfqpoint{-5.453430in}{-1.965915in}}{\pgfqpoint{-5.446919in}{-1.972426in}}%
\pgfpathcurveto{\pgfqpoint{-5.440408in}{-1.978938in}}{\pgfqpoint{-5.431575in}{-1.982596in}}{\pgfqpoint{-5.422367in}{-1.982596in}}%
\pgfpathlineto{\pgfqpoint{-5.422367in}{-1.982596in}}%
\pgfpathclose%
\pgfusepath{stroke,fill}%
\end{pgfscope}%
\begin{pgfscope}%
\pgfpathrectangle{\pgfqpoint{1.374500in}{0.082500in}}{\pgfqpoint{2.419000in}{2.419000in}}%
\pgfusepath{clip}%
\pgfsetbuttcap%
\pgfsetroundjoin%
\definecolor{currentfill}{rgb}{1.000000,0.662745,0.054902}%
\pgfsetfillcolor{currentfill}%
\pgfsetfillopacity{0.595014}%
\pgfsetlinewidth{1.003750pt}%
\definecolor{currentstroke}{rgb}{1.000000,0.662745,0.054902}%
\pgfsetstrokecolor{currentstroke}%
\pgfsetstrokeopacity{0.595014}%
\pgfsetdash{}{0pt}%
\pgfpathmoveto{\pgfqpoint{1.406020in}{-1.982596in}}%
\pgfpathcurveto{\pgfqpoint{1.415228in}{-1.982596in}}{\pgfqpoint{1.424061in}{-1.978938in}}{\pgfqpoint{1.430572in}{-1.972426in}}%
\pgfpathcurveto{\pgfqpoint{1.437084in}{-1.965915in}}{\pgfqpoint{1.440742in}{-1.957082in}}{\pgfqpoint{1.440742in}{-1.947874in}}%
\pgfpathcurveto{\pgfqpoint{1.440742in}{-1.938666in}}{\pgfqpoint{1.437084in}{-1.929833in}}{\pgfqpoint{1.430572in}{-1.923322in}}%
\pgfpathcurveto{\pgfqpoint{1.424061in}{-1.916810in}}{\pgfqpoint{1.415228in}{-1.913152in}}{\pgfqpoint{1.406020in}{-1.913152in}}%
\pgfpathcurveto{\pgfqpoint{1.396812in}{-1.913152in}}{\pgfqpoint{1.387979in}{-1.916810in}}{\pgfqpoint{1.381468in}{-1.923322in}}%
\pgfpathcurveto{\pgfqpoint{1.374956in}{-1.929833in}}{\pgfqpoint{1.371298in}{-1.938666in}}{\pgfqpoint{1.371298in}{-1.947874in}}%
\pgfpathcurveto{\pgfqpoint{1.371298in}{-1.957082in}}{\pgfqpoint{1.374956in}{-1.965915in}}{\pgfqpoint{1.381468in}{-1.972426in}}%
\pgfpathcurveto{\pgfqpoint{1.387979in}{-1.978938in}}{\pgfqpoint{1.396812in}{-1.982596in}}{\pgfqpoint{1.406020in}{-1.982596in}}%
\pgfpathlineto{\pgfqpoint{1.406020in}{-1.982596in}}%
\pgfpathclose%
\pgfusepath{stroke,fill}%
\end{pgfscope}%
\begin{pgfscope}%
\pgfpathrectangle{\pgfqpoint{1.374500in}{0.082500in}}{\pgfqpoint{2.419000in}{2.419000in}}%
\pgfusepath{clip}%
\pgfsetbuttcap%
\pgfsetroundjoin%
\definecolor{currentfill}{rgb}{1.000000,0.662745,0.054902}%
\pgfsetfillcolor{currentfill}%
\pgfsetfillopacity{0.595014}%
\pgfsetlinewidth{1.003750pt}%
\definecolor{currentstroke}{rgb}{1.000000,0.662745,0.054902}%
\pgfsetstrokecolor{currentstroke}%
\pgfsetstrokeopacity{0.595014}%
\pgfsetdash{}{0pt}%
\pgfpathmoveto{\pgfqpoint{8.234407in}{-1.982596in}}%
\pgfpathcurveto{\pgfqpoint{8.243615in}{-1.982596in}}{\pgfqpoint{8.252448in}{-1.978938in}}{\pgfqpoint{8.258959in}{-1.972426in}}%
\pgfpathcurveto{\pgfqpoint{8.265470in}{-1.965915in}}{\pgfqpoint{8.269129in}{-1.957082in}}{\pgfqpoint{8.269129in}{-1.947874in}}%
\pgfpathcurveto{\pgfqpoint{8.269129in}{-1.938666in}}{\pgfqpoint{8.265470in}{-1.929833in}}{\pgfqpoint{8.258959in}{-1.923322in}}%
\pgfpathcurveto{\pgfqpoint{8.252448in}{-1.916810in}}{\pgfqpoint{8.243615in}{-1.913152in}}{\pgfqpoint{8.234407in}{-1.913152in}}%
\pgfpathcurveto{\pgfqpoint{8.225198in}{-1.913152in}}{\pgfqpoint{8.216366in}{-1.916810in}}{\pgfqpoint{8.209854in}{-1.923322in}}%
\pgfpathcurveto{\pgfqpoint{8.203343in}{-1.929833in}}{\pgfqpoint{8.199684in}{-1.938666in}}{\pgfqpoint{8.199684in}{-1.947874in}}%
\pgfpathcurveto{\pgfqpoint{8.199684in}{-1.957082in}}{\pgfqpoint{8.203343in}{-1.965915in}}{\pgfqpoint{8.209854in}{-1.972426in}}%
\pgfpathcurveto{\pgfqpoint{8.216366in}{-1.978938in}}{\pgfqpoint{8.225198in}{-1.982596in}}{\pgfqpoint{8.234407in}{-1.982596in}}%
\pgfpathlineto{\pgfqpoint{8.234407in}{-1.982596in}}%
\pgfpathclose%
\pgfusepath{stroke,fill}%
\end{pgfscope}%
\begin{pgfscope}%
\pgfpathrectangle{\pgfqpoint{1.374500in}{0.082500in}}{\pgfqpoint{2.419000in}{2.419000in}}%
\pgfusepath{clip}%
\pgfsetbuttcap%
\pgfsetroundjoin%
\definecolor{currentfill}{rgb}{1.000000,0.662745,0.054902}%
\pgfsetfillcolor{currentfill}%
\pgfsetfillopacity{0.607915}%
\pgfsetlinewidth{1.003750pt}%
\definecolor{currentstroke}{rgb}{1.000000,0.662745,0.054902}%
\pgfsetstrokecolor{currentstroke}%
\pgfsetstrokeopacity{0.607915}%
\pgfsetdash{}{0pt}%
\pgfpathmoveto{\pgfqpoint{5.531732in}{-2.254152in}}%
\pgfpathcurveto{\pgfqpoint{5.540941in}{-2.254152in}}{\pgfqpoint{5.549773in}{-2.250493in}}{\pgfqpoint{5.556284in}{-2.243982in}}%
\pgfpathcurveto{\pgfqpoint{5.562796in}{-2.237470in}}{\pgfqpoint{5.566454in}{-2.228638in}}{\pgfqpoint{5.566454in}{-2.219429in}}%
\pgfpathcurveto{\pgfqpoint{5.566454in}{-2.210221in}}{\pgfqpoint{5.562796in}{-2.201388in}}{\pgfqpoint{5.556284in}{-2.194877in}}%
\pgfpathcurveto{\pgfqpoint{5.549773in}{-2.188366in}}{\pgfqpoint{5.540941in}{-2.184707in}}{\pgfqpoint{5.531732in}{-2.184707in}}%
\pgfpathcurveto{\pgfqpoint{5.522524in}{-2.184707in}}{\pgfqpoint{5.513691in}{-2.188366in}}{\pgfqpoint{5.507180in}{-2.194877in}}%
\pgfpathcurveto{\pgfqpoint{5.500668in}{-2.201388in}}{\pgfqpoint{5.497010in}{-2.210221in}}{\pgfqpoint{5.497010in}{-2.219429in}}%
\pgfpathcurveto{\pgfqpoint{5.497010in}{-2.228638in}}{\pgfqpoint{5.500668in}{-2.237470in}}{\pgfqpoint{5.507180in}{-2.243982in}}%
\pgfpathcurveto{\pgfqpoint{5.513691in}{-2.250493in}}{\pgfqpoint{5.522524in}{-2.254152in}}{\pgfqpoint{5.531732in}{-2.254152in}}%
\pgfpathlineto{\pgfqpoint{5.531732in}{-2.254152in}}%
\pgfpathclose%
\pgfusepath{stroke,fill}%
\end{pgfscope}%
\begin{pgfscope}%
\pgfpathrectangle{\pgfqpoint{1.374500in}{0.082500in}}{\pgfqpoint{2.419000in}{2.419000in}}%
\pgfusepath{clip}%
\pgfsetbuttcap%
\pgfsetroundjoin%
\definecolor{currentfill}{rgb}{1.000000,0.662745,0.054902}%
\pgfsetfillcolor{currentfill}%
\pgfsetfillopacity{0.607915}%
\pgfsetlinewidth{1.003750pt}%
\definecolor{currentstroke}{rgb}{1.000000,0.662745,0.054902}%
\pgfsetstrokecolor{currentstroke}%
\pgfsetstrokeopacity{0.607915}%
\pgfsetdash{}{0pt}%
\pgfpathmoveto{\pgfqpoint{-1.468002in}{-2.254152in}}%
\pgfpathcurveto{\pgfqpoint{-1.458793in}{-2.254152in}}{\pgfqpoint{-1.449961in}{-2.250493in}}{\pgfqpoint{-1.443449in}{-2.243982in}}%
\pgfpathcurveto{\pgfqpoint{-1.436938in}{-2.237470in}}{\pgfqpoint{-1.433279in}{-2.228638in}}{\pgfqpoint{-1.433279in}{-2.219429in}}%
\pgfpathcurveto{\pgfqpoint{-1.433279in}{-2.210221in}}{\pgfqpoint{-1.436938in}{-2.201388in}}{\pgfqpoint{-1.443449in}{-2.194877in}}%
\pgfpathcurveto{\pgfqpoint{-1.449961in}{-2.188366in}}{\pgfqpoint{-1.458793in}{-2.184707in}}{\pgfqpoint{-1.468002in}{-2.184707in}}%
\pgfpathcurveto{\pgfqpoint{-1.477210in}{-2.184707in}}{\pgfqpoint{-1.486043in}{-2.188366in}}{\pgfqpoint{-1.492554in}{-2.194877in}}%
\pgfpathcurveto{\pgfqpoint{-1.499065in}{-2.201388in}}{\pgfqpoint{-1.502724in}{-2.210221in}}{\pgfqpoint{-1.502724in}{-2.219429in}}%
\pgfpathcurveto{\pgfqpoint{-1.502724in}{-2.228638in}}{\pgfqpoint{-1.499065in}{-2.237470in}}{\pgfqpoint{-1.492554in}{-2.243982in}}%
\pgfpathcurveto{\pgfqpoint{-1.486043in}{-2.250493in}}{\pgfqpoint{-1.477210in}{-2.254152in}}{\pgfqpoint{-1.468002in}{-2.254152in}}%
\pgfpathlineto{\pgfqpoint{-1.468002in}{-2.254152in}}%
\pgfpathclose%
\pgfusepath{stroke,fill}%
\end{pgfscope}%
\begin{pgfscope}%
\pgfpathrectangle{\pgfqpoint{1.374500in}{0.082500in}}{\pgfqpoint{2.419000in}{2.419000in}}%
\pgfusepath{clip}%
\pgfsetbuttcap%
\pgfsetroundjoin%
\definecolor{currentfill}{rgb}{1.000000,0.662745,0.054902}%
\pgfsetfillcolor{currentfill}%
\pgfsetfillopacity{0.621480}%
\pgfsetlinewidth{1.003750pt}%
\definecolor{currentstroke}{rgb}{1.000000,0.662745,0.054902}%
\pgfsetstrokecolor{currentstroke}%
\pgfsetstrokeopacity{0.621480}%
\pgfsetdash{}{0pt}%
\pgfpathmoveto{\pgfqpoint{-4.489973in}{-2.539686in}}%
\pgfpathcurveto{\pgfqpoint{-4.480765in}{-2.539686in}}{\pgfqpoint{-4.471932in}{-2.536028in}}{\pgfqpoint{-4.465421in}{-2.529516in}}%
\pgfpathcurveto{\pgfqpoint{-4.458910in}{-2.523005in}}{\pgfqpoint{-4.455251in}{-2.514173in}}{\pgfqpoint{-4.455251in}{-2.504964in}}%
\pgfpathcurveto{\pgfqpoint{-4.455251in}{-2.495756in}}{\pgfqpoint{-4.458910in}{-2.486923in}}{\pgfqpoint{-4.465421in}{-2.480412in}}%
\pgfpathcurveto{\pgfqpoint{-4.471932in}{-2.473900in}}{\pgfqpoint{-4.480765in}{-2.470242in}}{\pgfqpoint{-4.489973in}{-2.470242in}}%
\pgfpathcurveto{\pgfqpoint{-4.499182in}{-2.470242in}}{\pgfqpoint{-4.508014in}{-2.473900in}}{\pgfqpoint{-4.514526in}{-2.480412in}}%
\pgfpathcurveto{\pgfqpoint{-4.521037in}{-2.486923in}}{\pgfqpoint{-4.524696in}{-2.495756in}}{\pgfqpoint{-4.524696in}{-2.504964in}}%
\pgfpathcurveto{\pgfqpoint{-4.524696in}{-2.514173in}}{\pgfqpoint{-4.521037in}{-2.523005in}}{\pgfqpoint{-4.514526in}{-2.529516in}}%
\pgfpathcurveto{\pgfqpoint{-4.508014in}{-2.536028in}}{\pgfqpoint{-4.499182in}{-2.539686in}}{\pgfqpoint{-4.489973in}{-2.539686in}}%
\pgfpathlineto{\pgfqpoint{-4.489973in}{-2.539686in}}%
\pgfpathclose%
\pgfusepath{stroke,fill}%
\end{pgfscope}%
\begin{pgfscope}%
\pgfpathrectangle{\pgfqpoint{1.374500in}{0.082500in}}{\pgfqpoint{2.419000in}{2.419000in}}%
\pgfusepath{clip}%
\pgfsetbuttcap%
\pgfsetroundjoin%
\definecolor{currentfill}{rgb}{1.000000,0.662745,0.054902}%
\pgfsetfillcolor{currentfill}%
\pgfsetfillopacity{0.621480}%
\pgfsetlinewidth{1.003750pt}%
\definecolor{currentstroke}{rgb}{1.000000,0.662745,0.054902}%
\pgfsetstrokecolor{currentstroke}%
\pgfsetstrokeopacity{0.621480}%
\pgfsetdash{}{0pt}%
\pgfpathmoveto{\pgfqpoint{2.689928in}{-2.539686in}}%
\pgfpathcurveto{\pgfqpoint{2.699137in}{-2.539686in}}{\pgfqpoint{2.707969in}{-2.536028in}}{\pgfqpoint{2.714480in}{-2.529516in}}%
\pgfpathcurveto{\pgfqpoint{2.720992in}{-2.523005in}}{\pgfqpoint{2.724650in}{-2.514173in}}{\pgfqpoint{2.724650in}{-2.504964in}}%
\pgfpathcurveto{\pgfqpoint{2.724650in}{-2.495756in}}{\pgfqpoint{2.720992in}{-2.486923in}}{\pgfqpoint{2.714480in}{-2.480412in}}%
\pgfpathcurveto{\pgfqpoint{2.707969in}{-2.473900in}}{\pgfqpoint{2.699137in}{-2.470242in}}{\pgfqpoint{2.689928in}{-2.470242in}}%
\pgfpathcurveto{\pgfqpoint{2.680720in}{-2.470242in}}{\pgfqpoint{2.671887in}{-2.473900in}}{\pgfqpoint{2.665376in}{-2.480412in}}%
\pgfpathcurveto{\pgfqpoint{2.658864in}{-2.486923in}}{\pgfqpoint{2.655206in}{-2.495756in}}{\pgfqpoint{2.655206in}{-2.504964in}}%
\pgfpathcurveto{\pgfqpoint{2.655206in}{-2.514173in}}{\pgfqpoint{2.658864in}{-2.523005in}}{\pgfqpoint{2.665376in}{-2.529516in}}%
\pgfpathcurveto{\pgfqpoint{2.671887in}{-2.536028in}}{\pgfqpoint{2.680720in}{-2.539686in}}{\pgfqpoint{2.689928in}{-2.539686in}}%
\pgfpathlineto{\pgfqpoint{2.689928in}{-2.539686in}}%
\pgfpathclose%
\pgfusepath{stroke,fill}%
\end{pgfscope}%
\begin{pgfscope}%
\pgfpathrectangle{\pgfqpoint{1.374500in}{0.082500in}}{\pgfqpoint{2.419000in}{2.419000in}}%
\pgfusepath{clip}%
\pgfsetbuttcap%
\pgfsetroundjoin%
\definecolor{currentfill}{rgb}{1.000000,0.662745,0.054902}%
\pgfsetfillcolor{currentfill}%
\pgfsetfillopacity{0.621480}%
\pgfsetlinewidth{1.003750pt}%
\definecolor{currentstroke}{rgb}{1.000000,0.662745,0.054902}%
\pgfsetstrokecolor{currentstroke}%
\pgfsetstrokeopacity{0.621480}%
\pgfsetdash{}{0pt}%
\pgfpathmoveto{\pgfqpoint{9.869830in}{-2.539686in}}%
\pgfpathcurveto{\pgfqpoint{9.879038in}{-2.539686in}}{\pgfqpoint{9.887871in}{-2.536028in}}{\pgfqpoint{9.894382in}{-2.529516in}}%
\pgfpathcurveto{\pgfqpoint{9.900893in}{-2.523005in}}{\pgfqpoint{9.904552in}{-2.514173in}}{\pgfqpoint{9.904552in}{-2.504964in}}%
\pgfpathcurveto{\pgfqpoint{9.904552in}{-2.495756in}}{\pgfqpoint{9.900893in}{-2.486923in}}{\pgfqpoint{9.894382in}{-2.480412in}}%
\pgfpathcurveto{\pgfqpoint{9.887871in}{-2.473900in}}{\pgfqpoint{9.879038in}{-2.470242in}}{\pgfqpoint{9.869830in}{-2.470242in}}%
\pgfpathcurveto{\pgfqpoint{9.860621in}{-2.470242in}}{\pgfqpoint{9.851789in}{-2.473900in}}{\pgfqpoint{9.845277in}{-2.480412in}}%
\pgfpathcurveto{\pgfqpoint{9.838766in}{-2.486923in}}{\pgfqpoint{9.835107in}{-2.495756in}}{\pgfqpoint{9.835107in}{-2.504964in}}%
\pgfpathcurveto{\pgfqpoint{9.835107in}{-2.514173in}}{\pgfqpoint{9.838766in}{-2.523005in}}{\pgfqpoint{9.845277in}{-2.529516in}}%
\pgfpathcurveto{\pgfqpoint{9.851789in}{-2.536028in}}{\pgfqpoint{9.860621in}{-2.539686in}}{\pgfqpoint{9.869830in}{-2.539686in}}%
\pgfpathlineto{\pgfqpoint{9.869830in}{-2.539686in}}%
\pgfpathclose%
\pgfusepath{stroke,fill}%
\end{pgfscope}%
\begin{pgfscope}%
\pgfpathrectangle{\pgfqpoint{1.374500in}{0.082500in}}{\pgfqpoint{2.419000in}{2.419000in}}%
\pgfusepath{clip}%
\pgfsetbuttcap%
\pgfsetroundjoin%
\definecolor{currentfill}{rgb}{1.000000,0.662745,0.054902}%
\pgfsetfillcolor{currentfill}%
\pgfsetfillopacity{0.635763}%
\pgfsetlinewidth{1.003750pt}%
\definecolor{currentstroke}{rgb}{1.000000,0.662745,0.054902}%
\pgfsetstrokecolor{currentstroke}%
\pgfsetstrokeopacity{0.635763}%
\pgfsetdash{}{0pt}%
\pgfpathmoveto{\pgfqpoint{7.067557in}{-2.840308in}}%
\pgfpathcurveto{\pgfqpoint{7.076765in}{-2.840308in}}{\pgfqpoint{7.085598in}{-2.836650in}}{\pgfqpoint{7.092109in}{-2.830138in}}%
\pgfpathcurveto{\pgfqpoint{7.098620in}{-2.823627in}}{\pgfqpoint{7.102279in}{-2.814794in}}{\pgfqpoint{7.102279in}{-2.805586in}}%
\pgfpathcurveto{\pgfqpoint{7.102279in}{-2.796378in}}{\pgfqpoint{7.098620in}{-2.787545in}}{\pgfqpoint{7.092109in}{-2.781034in}}%
\pgfpathcurveto{\pgfqpoint{7.085598in}{-2.774522in}}{\pgfqpoint{7.076765in}{-2.770864in}}{\pgfqpoint{7.067557in}{-2.770864in}}%
\pgfpathcurveto{\pgfqpoint{7.058348in}{-2.770864in}}{\pgfqpoint{7.049516in}{-2.774522in}}{\pgfqpoint{7.043004in}{-2.781034in}}%
\pgfpathcurveto{\pgfqpoint{7.036493in}{-2.787545in}}{\pgfqpoint{7.032834in}{-2.796378in}}{\pgfqpoint{7.032834in}{-2.805586in}}%
\pgfpathcurveto{\pgfqpoint{7.032834in}{-2.814794in}}{\pgfqpoint{7.036493in}{-2.823627in}}{\pgfqpoint{7.043004in}{-2.830138in}}%
\pgfpathcurveto{\pgfqpoint{7.049516in}{-2.836650in}}{\pgfqpoint{7.058348in}{-2.840308in}}{\pgfqpoint{7.067557in}{-2.840308in}}%
\pgfpathlineto{\pgfqpoint{7.067557in}{-2.840308in}}%
\pgfpathclose%
\pgfusepath{stroke,fill}%
\end{pgfscope}%
\begin{pgfscope}%
\pgfpathrectangle{\pgfqpoint{1.374500in}{0.082500in}}{\pgfqpoint{2.419000in}{2.419000in}}%
\pgfusepath{clip}%
\pgfsetbuttcap%
\pgfsetroundjoin%
\definecolor{currentfill}{rgb}{1.000000,0.662745,0.054902}%
\pgfsetfillcolor{currentfill}%
\pgfsetfillopacity{0.635763}%
\pgfsetlinewidth{1.003750pt}%
\definecolor{currentstroke}{rgb}{1.000000,0.662745,0.054902}%
\pgfsetstrokecolor{currentstroke}%
\pgfsetstrokeopacity{0.635763}%
\pgfsetdash{}{0pt}%
\pgfpathmoveto{\pgfqpoint{-0.302033in}{-2.840308in}}%
\pgfpathcurveto{\pgfqpoint{-0.292824in}{-2.840308in}}{\pgfqpoint{-0.283992in}{-2.836650in}}{\pgfqpoint{-0.277480in}{-2.830138in}}%
\pgfpathcurveto{\pgfqpoint{-0.270969in}{-2.823627in}}{\pgfqpoint{-0.267310in}{-2.814794in}}{\pgfqpoint{-0.267310in}{-2.805586in}}%
\pgfpathcurveto{\pgfqpoint{-0.267310in}{-2.796378in}}{\pgfqpoint{-0.270969in}{-2.787545in}}{\pgfqpoint{-0.277480in}{-2.781034in}}%
\pgfpathcurveto{\pgfqpoint{-0.283992in}{-2.774522in}}{\pgfqpoint{-0.292824in}{-2.770864in}}{\pgfqpoint{-0.302033in}{-2.770864in}}%
\pgfpathcurveto{\pgfqpoint{-0.311241in}{-2.770864in}}{\pgfqpoint{-0.320074in}{-2.774522in}}{\pgfqpoint{-0.326585in}{-2.781034in}}%
\pgfpathcurveto{\pgfqpoint{-0.333096in}{-2.787545in}}{\pgfqpoint{-0.336755in}{-2.796378in}}{\pgfqpoint{-0.336755in}{-2.805586in}}%
\pgfpathcurveto{\pgfqpoint{-0.336755in}{-2.814794in}}{\pgfqpoint{-0.333096in}{-2.823627in}}{\pgfqpoint{-0.326585in}{-2.830138in}}%
\pgfpathcurveto{\pgfqpoint{-0.320074in}{-2.836650in}}{\pgfqpoint{-0.311241in}{-2.840308in}}{\pgfqpoint{-0.302033in}{-2.840308in}}%
\pgfpathlineto{\pgfqpoint{-0.302033in}{-2.840308in}}%
\pgfpathclose%
\pgfusepath{stroke,fill}%
\end{pgfscope}%
\begin{pgfscope}%
\pgfpathrectangle{\pgfqpoint{1.374500in}{0.082500in}}{\pgfqpoint{2.419000in}{2.419000in}}%
\pgfusepath{clip}%
\pgfsetbuttcap%
\pgfsetroundjoin%
\definecolor{currentfill}{rgb}{1.000000,0.662745,0.054902}%
\pgfsetfillcolor{currentfill}%
\pgfsetfillopacity{0.650820}%
\pgfsetlinewidth{1.003750pt}%
\definecolor{currentstroke}{rgb}{1.000000,0.662745,0.054902}%
\pgfsetstrokecolor{currentstroke}%
\pgfsetstrokeopacity{0.650820}%
\pgfsetdash{}{0pt}%
\pgfpathmoveto{\pgfqpoint{4.113198in}{-3.157246in}}%
\pgfpathcurveto{\pgfqpoint{4.122406in}{-3.157246in}}{\pgfqpoint{4.131239in}{-3.153587in}}{\pgfqpoint{4.137750in}{-3.147076in}}%
\pgfpathcurveto{\pgfqpoint{4.144261in}{-3.140564in}}{\pgfqpoint{4.147920in}{-3.131732in}}{\pgfqpoint{4.147920in}{-3.122523in}}%
\pgfpathcurveto{\pgfqpoint{4.147920in}{-3.113315in}}{\pgfqpoint{4.144261in}{-3.104482in}}{\pgfqpoint{4.137750in}{-3.097971in}}%
\pgfpathcurveto{\pgfqpoint{4.131239in}{-3.091460in}}{\pgfqpoint{4.122406in}{-3.087801in}}{\pgfqpoint{4.113198in}{-3.087801in}}%
\pgfpathcurveto{\pgfqpoint{4.103989in}{-3.087801in}}{\pgfqpoint{4.095157in}{-3.091460in}}{\pgfqpoint{4.088645in}{-3.097971in}}%
\pgfpathcurveto{\pgfqpoint{4.082134in}{-3.104482in}}{\pgfqpoint{4.078475in}{-3.113315in}}{\pgfqpoint{4.078475in}{-3.122523in}}%
\pgfpathcurveto{\pgfqpoint{4.078475in}{-3.131732in}}{\pgfqpoint{4.082134in}{-3.140564in}}{\pgfqpoint{4.088645in}{-3.147076in}}%
\pgfpathcurveto{\pgfqpoint{4.095157in}{-3.153587in}}{\pgfqpoint{4.103989in}{-3.157246in}}{\pgfqpoint{4.113198in}{-3.157246in}}%
\pgfpathlineto{\pgfqpoint{4.113198in}{-3.157246in}}%
\pgfpathclose%
\pgfusepath{stroke,fill}%
\end{pgfscope}%
\begin{pgfscope}%
\pgfpathrectangle{\pgfqpoint{1.374500in}{0.082500in}}{\pgfqpoint{2.419000in}{2.419000in}}%
\pgfusepath{clip}%
\pgfsetbuttcap%
\pgfsetroundjoin%
\definecolor{currentfill}{rgb}{1.000000,0.662745,0.054902}%
\pgfsetfillcolor{currentfill}%
\pgfsetfillopacity{0.650820}%
\pgfsetlinewidth{1.003750pt}%
\definecolor{currentstroke}{rgb}{1.000000,0.662745,0.054902}%
\pgfsetstrokecolor{currentstroke}%
\pgfsetstrokeopacity{0.650820}%
\pgfsetdash{}{0pt}%
\pgfpathmoveto{\pgfqpoint{-3.456374in}{-3.157246in}}%
\pgfpathcurveto{\pgfqpoint{-3.447165in}{-3.157246in}}{\pgfqpoint{-3.438333in}{-3.153587in}}{\pgfqpoint{-3.431822in}{-3.147076in}}%
\pgfpathcurveto{\pgfqpoint{-3.425310in}{-3.140564in}}{\pgfqpoint{-3.421652in}{-3.131732in}}{\pgfqpoint{-3.421652in}{-3.122523in}}%
\pgfpathcurveto{\pgfqpoint{-3.421652in}{-3.113315in}}{\pgfqpoint{-3.425310in}{-3.104482in}}{\pgfqpoint{-3.431822in}{-3.097971in}}%
\pgfpathcurveto{\pgfqpoint{-3.438333in}{-3.091460in}}{\pgfqpoint{-3.447165in}{-3.087801in}}{\pgfqpoint{-3.456374in}{-3.087801in}}%
\pgfpathcurveto{\pgfqpoint{-3.465582in}{-3.087801in}}{\pgfqpoint{-3.474415in}{-3.091460in}}{\pgfqpoint{-3.480926in}{-3.097971in}}%
\pgfpathcurveto{\pgfqpoint{-3.487438in}{-3.104482in}}{\pgfqpoint{-3.491096in}{-3.113315in}}{\pgfqpoint{-3.491096in}{-3.122523in}}%
\pgfpathcurveto{\pgfqpoint{-3.491096in}{-3.131732in}}{\pgfqpoint{-3.487438in}{-3.140564in}}{\pgfqpoint{-3.480926in}{-3.147076in}}%
\pgfpathcurveto{\pgfqpoint{-3.474415in}{-3.153587in}}{\pgfqpoint{-3.465582in}{-3.157246in}}{\pgfqpoint{-3.456374in}{-3.157246in}}%
\pgfpathlineto{\pgfqpoint{-3.456374in}{-3.157246in}}%
\pgfpathclose%
\pgfusepath{stroke,fill}%
\end{pgfscope}%
\begin{pgfscope}%
\pgfpathrectangle{\pgfqpoint{1.374500in}{0.082500in}}{\pgfqpoint{2.419000in}{2.419000in}}%
\pgfusepath{clip}%
\pgfsetbuttcap%
\pgfsetroundjoin%
\definecolor{currentfill}{rgb}{1.000000,0.662745,0.054902}%
\pgfsetfillcolor{currentfill}%
\pgfsetfillopacity{0.650820}%
\pgfsetlinewidth{1.003750pt}%
\definecolor{currentstroke}{rgb}{1.000000,0.662745,0.054902}%
\pgfsetstrokecolor{currentstroke}%
\pgfsetstrokeopacity{0.650820}%
\pgfsetdash{}{0pt}%
\pgfpathmoveto{\pgfqpoint{11.682769in}{-3.157246in}}%
\pgfpathcurveto{\pgfqpoint{11.691978in}{-3.157246in}}{\pgfqpoint{11.700810in}{-3.153587in}}{\pgfqpoint{11.707321in}{-3.147076in}}%
\pgfpathcurveto{\pgfqpoint{11.713833in}{-3.140564in}}{\pgfqpoint{11.717491in}{-3.131732in}}{\pgfqpoint{11.717491in}{-3.122523in}}%
\pgfpathcurveto{\pgfqpoint{11.717491in}{-3.113315in}}{\pgfqpoint{11.713833in}{-3.104482in}}{\pgfqpoint{11.707321in}{-3.097971in}}%
\pgfpathcurveto{\pgfqpoint{11.700810in}{-3.091460in}}{\pgfqpoint{11.691978in}{-3.087801in}}{\pgfqpoint{11.682769in}{-3.087801in}}%
\pgfpathcurveto{\pgfqpoint{11.673561in}{-3.087801in}}{\pgfqpoint{11.664728in}{-3.091460in}}{\pgfqpoint{11.658217in}{-3.097971in}}%
\pgfpathcurveto{\pgfqpoint{11.651705in}{-3.104482in}}{\pgfqpoint{11.648047in}{-3.113315in}}{\pgfqpoint{11.648047in}{-3.122523in}}%
\pgfpathcurveto{\pgfqpoint{11.648047in}{-3.131732in}}{\pgfqpoint{11.651705in}{-3.140564in}}{\pgfqpoint{11.658217in}{-3.147076in}}%
\pgfpathcurveto{\pgfqpoint{11.664728in}{-3.153587in}}{\pgfqpoint{11.673561in}{-3.157246in}}{\pgfqpoint{11.682769in}{-3.157246in}}%
\pgfpathlineto{\pgfqpoint{11.682769in}{-3.157246in}}%
\pgfpathclose%
\pgfusepath{stroke,fill}%
\end{pgfscope}%
\begin{pgfscope}%
\pgfpathrectangle{\pgfqpoint{1.374500in}{0.082500in}}{\pgfqpoint{2.419000in}{2.419000in}}%
\pgfusepath{clip}%
\pgfsetbuttcap%
\pgfsetroundjoin%
\definecolor{currentfill}{rgb}{1.000000,0.662745,0.054902}%
\pgfsetfillcolor{currentfill}%
\pgfsetfillopacity{0.666718}%
\pgfsetlinewidth{1.003750pt}%
\definecolor{currentstroke}{rgb}{1.000000,0.662745,0.054902}%
\pgfsetstrokecolor{currentstroke}%
\pgfsetstrokeopacity{0.666718}%
\pgfsetdash{}{0pt}%
\pgfpathmoveto{\pgfqpoint{0.994026in}{-3.491864in}}%
\pgfpathcurveto{\pgfqpoint{1.003235in}{-3.491864in}}{\pgfqpoint{1.012067in}{-3.488205in}}{\pgfqpoint{1.018579in}{-3.481694in}}%
\pgfpathcurveto{\pgfqpoint{1.025090in}{-3.475182in}}{\pgfqpoint{1.028749in}{-3.466350in}}{\pgfqpoint{1.028749in}{-3.457141in}}%
\pgfpathcurveto{\pgfqpoint{1.028749in}{-3.447933in}}{\pgfqpoint{1.025090in}{-3.439100in}}{\pgfqpoint{1.018579in}{-3.432589in}}%
\pgfpathcurveto{\pgfqpoint{1.012067in}{-3.426078in}}{\pgfqpoint{1.003235in}{-3.422419in}}{\pgfqpoint{0.994026in}{-3.422419in}}%
\pgfpathcurveto{\pgfqpoint{0.984818in}{-3.422419in}}{\pgfqpoint{0.975985in}{-3.426078in}}{\pgfqpoint{0.969474in}{-3.432589in}}%
\pgfpathcurveto{\pgfqpoint{0.962963in}{-3.439100in}}{\pgfqpoint{0.959304in}{-3.447933in}}{\pgfqpoint{0.959304in}{-3.457141in}}%
\pgfpathcurveto{\pgfqpoint{0.959304in}{-3.466350in}}{\pgfqpoint{0.962963in}{-3.475182in}}{\pgfqpoint{0.969474in}{-3.481694in}}%
\pgfpathcurveto{\pgfqpoint{0.975985in}{-3.488205in}}{\pgfqpoint{0.984818in}{-3.491864in}}{\pgfqpoint{0.994026in}{-3.491864in}}%
\pgfpathlineto{\pgfqpoint{0.994026in}{-3.491864in}}%
\pgfpathclose%
\pgfusepath{stroke,fill}%
\end{pgfscope}%
\begin{pgfscope}%
\pgfpathrectangle{\pgfqpoint{1.374500in}{0.082500in}}{\pgfqpoint{2.419000in}{2.419000in}}%
\pgfusepath{clip}%
\pgfsetbuttcap%
\pgfsetroundjoin%
\definecolor{currentfill}{rgb}{1.000000,0.662745,0.054902}%
\pgfsetfillcolor{currentfill}%
\pgfsetfillopacity{0.666718}%
\pgfsetlinewidth{1.003750pt}%
\definecolor{currentstroke}{rgb}{1.000000,0.662745,0.054902}%
\pgfsetstrokecolor{currentstroke}%
\pgfsetstrokeopacity{0.666718}%
\pgfsetdash{}{0pt}%
\pgfpathmoveto{\pgfqpoint{8.774736in}{-3.491864in}}%
\pgfpathcurveto{\pgfqpoint{8.783945in}{-3.491864in}}{\pgfqpoint{8.792777in}{-3.488205in}}{\pgfqpoint{8.799289in}{-3.481694in}}%
\pgfpathcurveto{\pgfqpoint{8.805800in}{-3.475182in}}{\pgfqpoint{8.809459in}{-3.466350in}}{\pgfqpoint{8.809459in}{-3.457141in}}%
\pgfpathcurveto{\pgfqpoint{8.809459in}{-3.447933in}}{\pgfqpoint{8.805800in}{-3.439100in}}{\pgfqpoint{8.799289in}{-3.432589in}}%
\pgfpathcurveto{\pgfqpoint{8.792777in}{-3.426078in}}{\pgfqpoint{8.783945in}{-3.422419in}}{\pgfqpoint{8.774736in}{-3.422419in}}%
\pgfpathcurveto{\pgfqpoint{8.765528in}{-3.422419in}}{\pgfqpoint{8.756696in}{-3.426078in}}{\pgfqpoint{8.750184in}{-3.432589in}}%
\pgfpathcurveto{\pgfqpoint{8.743673in}{-3.439100in}}{\pgfqpoint{8.740014in}{-3.447933in}}{\pgfqpoint{8.740014in}{-3.457141in}}%
\pgfpathcurveto{\pgfqpoint{8.740014in}{-3.466350in}}{\pgfqpoint{8.743673in}{-3.475182in}}{\pgfqpoint{8.750184in}{-3.481694in}}%
\pgfpathcurveto{\pgfqpoint{8.756696in}{-3.488205in}}{\pgfqpoint{8.765528in}{-3.491864in}}{\pgfqpoint{8.774736in}{-3.491864in}}%
\pgfpathlineto{\pgfqpoint{8.774736in}{-3.491864in}}%
\pgfpathclose%
\pgfusepath{stroke,fill}%
\end{pgfscope}%
\begin{pgfscope}%
\pgfpathrectangle{\pgfqpoint{1.374500in}{0.082500in}}{\pgfqpoint{2.419000in}{2.419000in}}%
\pgfusepath{clip}%
\pgfsetbuttcap%
\pgfsetroundjoin%
\definecolor{currentfill}{rgb}{1.000000,0.662745,0.054902}%
\pgfsetfillcolor{currentfill}%
\pgfsetfillopacity{0.683527}%
\pgfsetlinewidth{1.003750pt}%
\definecolor{currentstroke}{rgb}{1.000000,0.662745,0.054902}%
\pgfsetstrokecolor{currentstroke}%
\pgfsetstrokeopacity{0.683527}%
\pgfsetdash{}{0pt}%
\pgfpathmoveto{\pgfqpoint{-2.304144in}{-3.845684in}}%
\pgfpathcurveto{\pgfqpoint{-2.294936in}{-3.845684in}}{\pgfqpoint{-2.286103in}{-3.842026in}}{\pgfqpoint{-2.279592in}{-3.835514in}}%
\pgfpathcurveto{\pgfqpoint{-2.273081in}{-3.829003in}}{\pgfqpoint{-2.269422in}{-3.820171in}}{\pgfqpoint{-2.269422in}{-3.810962in}}%
\pgfpathcurveto{\pgfqpoint{-2.269422in}{-3.801754in}}{\pgfqpoint{-2.273081in}{-3.792921in}}{\pgfqpoint{-2.279592in}{-3.786410in}}%
\pgfpathcurveto{\pgfqpoint{-2.286103in}{-3.779898in}}{\pgfqpoint{-2.294936in}{-3.776240in}}{\pgfqpoint{-2.304144in}{-3.776240in}}%
\pgfpathcurveto{\pgfqpoint{-2.313353in}{-3.776240in}}{\pgfqpoint{-2.322185in}{-3.779898in}}{\pgfqpoint{-2.328697in}{-3.786410in}}%
\pgfpathcurveto{\pgfqpoint{-2.335208in}{-3.792921in}}{\pgfqpoint{-2.338867in}{-3.801754in}}{\pgfqpoint{-2.338867in}{-3.810962in}}%
\pgfpathcurveto{\pgfqpoint{-2.338867in}{-3.820171in}}{\pgfqpoint{-2.335208in}{-3.829003in}}{\pgfqpoint{-2.328697in}{-3.835514in}}%
\pgfpathcurveto{\pgfqpoint{-2.322185in}{-3.842026in}}{\pgfqpoint{-2.313353in}{-3.845684in}}{\pgfqpoint{-2.304144in}{-3.845684in}}%
\pgfpathlineto{\pgfqpoint{-2.304144in}{-3.845684in}}%
\pgfpathclose%
\pgfusepath{stroke,fill}%
\end{pgfscope}%
\begin{pgfscope}%
\pgfpathrectangle{\pgfqpoint{1.374500in}{0.082500in}}{\pgfqpoint{2.419000in}{2.419000in}}%
\pgfusepath{clip}%
\pgfsetbuttcap%
\pgfsetroundjoin%
\definecolor{currentfill}{rgb}{1.000000,0.662745,0.054902}%
\pgfsetfillcolor{currentfill}%
\pgfsetfillopacity{0.683527}%
\pgfsetlinewidth{1.003750pt}%
\definecolor{currentstroke}{rgb}{1.000000,0.662745,0.054902}%
\pgfsetstrokecolor{currentstroke}%
\pgfsetstrokeopacity{0.683527}%
\pgfsetdash{}{0pt}%
\pgfpathmoveto{\pgfqpoint{5.699821in}{-3.845684in}}%
\pgfpathcurveto{\pgfqpoint{5.709029in}{-3.845684in}}{\pgfqpoint{5.717862in}{-3.842026in}}{\pgfqpoint{5.724373in}{-3.835514in}}%
\pgfpathcurveto{\pgfqpoint{5.730885in}{-3.829003in}}{\pgfqpoint{5.734543in}{-3.820171in}}{\pgfqpoint{5.734543in}{-3.810962in}}%
\pgfpathcurveto{\pgfqpoint{5.734543in}{-3.801754in}}{\pgfqpoint{5.730885in}{-3.792921in}}{\pgfqpoint{5.724373in}{-3.786410in}}%
\pgfpathcurveto{\pgfqpoint{5.717862in}{-3.779898in}}{\pgfqpoint{5.709029in}{-3.776240in}}{\pgfqpoint{5.699821in}{-3.776240in}}%
\pgfpathcurveto{\pgfqpoint{5.690613in}{-3.776240in}}{\pgfqpoint{5.681780in}{-3.779898in}}{\pgfqpoint{5.675269in}{-3.786410in}}%
\pgfpathcurveto{\pgfqpoint{5.668757in}{-3.792921in}}{\pgfqpoint{5.665099in}{-3.801754in}}{\pgfqpoint{5.665099in}{-3.810962in}}%
\pgfpathcurveto{\pgfqpoint{5.665099in}{-3.820171in}}{\pgfqpoint{5.668757in}{-3.829003in}}{\pgfqpoint{5.675269in}{-3.835514in}}%
\pgfpathcurveto{\pgfqpoint{5.681780in}{-3.842026in}}{\pgfqpoint{5.690613in}{-3.845684in}}{\pgfqpoint{5.699821in}{-3.845684in}}%
\pgfpathlineto{\pgfqpoint{5.699821in}{-3.845684in}}%
\pgfpathclose%
\pgfusepath{stroke,fill}%
\end{pgfscope}%
\begin{pgfscope}%
\pgfpathrectangle{\pgfqpoint{1.374500in}{0.082500in}}{\pgfqpoint{2.419000in}{2.419000in}}%
\pgfusepath{clip}%
\pgfsetbuttcap%
\pgfsetroundjoin%
\definecolor{currentfill}{rgb}{1.000000,0.662745,0.054902}%
\pgfsetfillcolor{currentfill}%
\pgfsetfillopacity{0.701330}%
\pgfsetlinewidth{1.003750pt}%
\definecolor{currentstroke}{rgb}{1.000000,0.662745,0.054902}%
\pgfsetstrokecolor{currentstroke}%
\pgfsetstrokeopacity{0.701330}%
\pgfsetdash{}{0pt}%
\pgfpathmoveto{\pgfqpoint{2.443233in}{-4.220410in}}%
\pgfpathcurveto{\pgfqpoint{2.452442in}{-4.220410in}}{\pgfqpoint{2.461274in}{-4.216751in}}{\pgfqpoint{2.467785in}{-4.210240in}}%
\pgfpathcurveto{\pgfqpoint{2.474297in}{-4.203728in}}{\pgfqpoint{2.477955in}{-4.194896in}}{\pgfqpoint{2.477955in}{-4.185687in}}%
\pgfpathcurveto{\pgfqpoint{2.477955in}{-4.176479in}}{\pgfqpoint{2.474297in}{-4.167646in}}{\pgfqpoint{2.467785in}{-4.161135in}}%
\pgfpathcurveto{\pgfqpoint{2.461274in}{-4.154624in}}{\pgfqpoint{2.452442in}{-4.150965in}}{\pgfqpoint{2.443233in}{-4.150965in}}%
\pgfpathcurveto{\pgfqpoint{2.434025in}{-4.150965in}}{\pgfqpoint{2.425192in}{-4.154624in}}{\pgfqpoint{2.418681in}{-4.161135in}}%
\pgfpathcurveto{\pgfqpoint{2.412169in}{-4.167646in}}{\pgfqpoint{2.408511in}{-4.176479in}}{\pgfqpoint{2.408511in}{-4.185687in}}%
\pgfpathcurveto{\pgfqpoint{2.408511in}{-4.194896in}}{\pgfqpoint{2.412169in}{-4.203728in}}{\pgfqpoint{2.418681in}{-4.210240in}}%
\pgfpathcurveto{\pgfqpoint{2.425192in}{-4.216751in}}{\pgfqpoint{2.434025in}{-4.220410in}}{\pgfqpoint{2.443233in}{-4.220410in}}%
\pgfpathlineto{\pgfqpoint{2.443233in}{-4.220410in}}%
\pgfpathclose%
\pgfusepath{stroke,fill}%
\end{pgfscope}%
\begin{pgfscope}%
\pgfpathrectangle{\pgfqpoint{1.374500in}{0.082500in}}{\pgfqpoint{2.419000in}{2.419000in}}%
\pgfusepath{clip}%
\pgfsetbuttcap%
\pgfsetroundjoin%
\definecolor{currentfill}{rgb}{1.000000,0.662745,0.054902}%
\pgfsetfillcolor{currentfill}%
\pgfsetfillopacity{0.701330}%
\pgfsetlinewidth{1.003750pt}%
\definecolor{currentstroke}{rgb}{1.000000,0.662745,0.054902}%
\pgfsetstrokecolor{currentstroke}%
\pgfsetstrokeopacity{0.701330}%
\pgfsetdash{}{0pt}%
\pgfpathmoveto{\pgfqpoint{10.683644in}{-4.220410in}}%
\pgfpathcurveto{\pgfqpoint{10.692852in}{-4.220410in}}{\pgfqpoint{10.701685in}{-4.216751in}}{\pgfqpoint{10.708196in}{-4.210240in}}%
\pgfpathcurveto{\pgfqpoint{10.714708in}{-4.203728in}}{\pgfqpoint{10.718366in}{-4.194896in}}{\pgfqpoint{10.718366in}{-4.185687in}}%
\pgfpathcurveto{\pgfqpoint{10.718366in}{-4.176479in}}{\pgfqpoint{10.714708in}{-4.167646in}}{\pgfqpoint{10.708196in}{-4.161135in}}%
\pgfpathcurveto{\pgfqpoint{10.701685in}{-4.154624in}}{\pgfqpoint{10.692852in}{-4.150965in}}{\pgfqpoint{10.683644in}{-4.150965in}}%
\pgfpathcurveto{\pgfqpoint{10.674436in}{-4.150965in}}{\pgfqpoint{10.665603in}{-4.154624in}}{\pgfqpoint{10.659092in}{-4.161135in}}%
\pgfpathcurveto{\pgfqpoint{10.652580in}{-4.167646in}}{\pgfqpoint{10.648922in}{-4.176479in}}{\pgfqpoint{10.648922in}{-4.185687in}}%
\pgfpathcurveto{\pgfqpoint{10.648922in}{-4.194896in}}{\pgfqpoint{10.652580in}{-4.203728in}}{\pgfqpoint{10.659092in}{-4.210240in}}%
\pgfpathcurveto{\pgfqpoint{10.665603in}{-4.216751in}}{\pgfqpoint{10.674436in}{-4.220410in}}{\pgfqpoint{10.683644in}{-4.220410in}}%
\pgfpathlineto{\pgfqpoint{10.683644in}{-4.220410in}}%
\pgfpathclose%
\pgfusepath{stroke,fill}%
\end{pgfscope}%
\begin{pgfscope}%
\pgfpathrectangle{\pgfqpoint{1.374500in}{0.082500in}}{\pgfqpoint{2.419000in}{2.419000in}}%
\pgfusepath{clip}%
\pgfsetbuttcap%
\pgfsetroundjoin%
\definecolor{currentfill}{rgb}{1.000000,0.662745,0.054902}%
\pgfsetfillcolor{currentfill}%
\pgfsetfillopacity{0.720217}%
\pgfsetlinewidth{1.003750pt}%
\definecolor{currentstroke}{rgb}{1.000000,0.662745,0.054902}%
\pgfsetstrokecolor{currentstroke}%
\pgfsetstrokeopacity{0.720217}%
\pgfsetdash{}{0pt}%
\pgfpathmoveto{\pgfqpoint{-1.011618in}{-4.617948in}}%
\pgfpathcurveto{\pgfqpoint{-1.002409in}{-4.617948in}}{\pgfqpoint{-0.993577in}{-4.614290in}}{\pgfqpoint{-0.987066in}{-4.607778in}}%
\pgfpathcurveto{\pgfqpoint{-0.980554in}{-4.601267in}}{\pgfqpoint{-0.976896in}{-4.592435in}}{\pgfqpoint{-0.976896in}{-4.583226in}}%
\pgfpathcurveto{\pgfqpoint{-0.976896in}{-4.574018in}}{\pgfqpoint{-0.980554in}{-4.565185in}}{\pgfqpoint{-0.987066in}{-4.558674in}}%
\pgfpathcurveto{\pgfqpoint{-0.993577in}{-4.552162in}}{\pgfqpoint{-1.002409in}{-4.548504in}}{\pgfqpoint{-1.011618in}{-4.548504in}}%
\pgfpathcurveto{\pgfqpoint{-1.020826in}{-4.548504in}}{\pgfqpoint{-1.029659in}{-4.552162in}}{\pgfqpoint{-1.036170in}{-4.558674in}}%
\pgfpathcurveto{\pgfqpoint{-1.042682in}{-4.565185in}}{\pgfqpoint{-1.046340in}{-4.574018in}}{\pgfqpoint{-1.046340in}{-4.583226in}}%
\pgfpathcurveto{\pgfqpoint{-1.046340in}{-4.592435in}}{\pgfqpoint{-1.042682in}{-4.601267in}}{\pgfqpoint{-1.036170in}{-4.607778in}}%
\pgfpathcurveto{\pgfqpoint{-1.029659in}{-4.614290in}}{\pgfqpoint{-1.020826in}{-4.617948in}}{\pgfqpoint{-1.011618in}{-4.617948in}}%
\pgfpathlineto{\pgfqpoint{-1.011618in}{-4.617948in}}%
\pgfpathclose%
\pgfusepath{stroke,fill}%
\end{pgfscope}%
\begin{pgfscope}%
\pgfpathrectangle{\pgfqpoint{1.374500in}{0.082500in}}{\pgfqpoint{2.419000in}{2.419000in}}%
\pgfusepath{clip}%
\pgfsetbuttcap%
\pgfsetroundjoin%
\definecolor{currentfill}{rgb}{1.000000,0.662745,0.054902}%
\pgfsetfillcolor{currentfill}%
\pgfsetfillopacity{0.720217}%
\pgfsetlinewidth{1.003750pt}%
\definecolor{currentstroke}{rgb}{1.000000,0.662745,0.054902}%
\pgfsetstrokecolor{currentstroke}%
\pgfsetstrokeopacity{0.720217}%
\pgfsetdash{}{0pt}%
\pgfpathmoveto{\pgfqpoint{7.479634in}{-4.617948in}}%
\pgfpathcurveto{\pgfqpoint{7.488842in}{-4.617948in}}{\pgfqpoint{7.497675in}{-4.614290in}}{\pgfqpoint{7.504186in}{-4.607778in}}%
\pgfpathcurveto{\pgfqpoint{7.510697in}{-4.601267in}}{\pgfqpoint{7.514356in}{-4.592435in}}{\pgfqpoint{7.514356in}{-4.583226in}}%
\pgfpathcurveto{\pgfqpoint{7.514356in}{-4.574018in}}{\pgfqpoint{7.510697in}{-4.565185in}}{\pgfqpoint{7.504186in}{-4.558674in}}%
\pgfpathcurveto{\pgfqpoint{7.497675in}{-4.552162in}}{\pgfqpoint{7.488842in}{-4.548504in}}{\pgfqpoint{7.479634in}{-4.548504in}}%
\pgfpathcurveto{\pgfqpoint{7.470425in}{-4.548504in}}{\pgfqpoint{7.461593in}{-4.552162in}}{\pgfqpoint{7.455081in}{-4.558674in}}%
\pgfpathcurveto{\pgfqpoint{7.448570in}{-4.565185in}}{\pgfqpoint{7.444911in}{-4.574018in}}{\pgfqpoint{7.444911in}{-4.583226in}}%
\pgfpathcurveto{\pgfqpoint{7.444911in}{-4.592435in}}{\pgfqpoint{7.448570in}{-4.601267in}}{\pgfqpoint{7.455081in}{-4.607778in}}%
\pgfpathcurveto{\pgfqpoint{7.461593in}{-4.614290in}}{\pgfqpoint{7.470425in}{-4.617948in}}{\pgfqpoint{7.479634in}{-4.617948in}}%
\pgfpathlineto{\pgfqpoint{7.479634in}{-4.617948in}}%
\pgfpathclose%
\pgfusepath{stroke,fill}%
\end{pgfscope}%
\begin{pgfscope}%
\pgfpathrectangle{\pgfqpoint{1.374500in}{0.082500in}}{\pgfqpoint{2.419000in}{2.419000in}}%
\pgfusepath{clip}%
\pgfsetbuttcap%
\pgfsetroundjoin%
\definecolor{currentfill}{rgb}{1.000000,0.662745,0.054902}%
\pgfsetfillcolor{currentfill}%
\pgfsetfillopacity{0.740290}%
\pgfsetlinewidth{1.003750pt}%
\definecolor{currentstroke}{rgb}{1.000000,0.662745,0.054902}%
\pgfsetstrokecolor{currentstroke}%
\pgfsetstrokeopacity{0.740290}%
\pgfsetdash{}{0pt}%
\pgfpathmoveto{\pgfqpoint{4.074437in}{-5.040449in}}%
\pgfpathcurveto{\pgfqpoint{4.083645in}{-5.040449in}}{\pgfqpoint{4.092478in}{-5.036791in}}{\pgfqpoint{4.098989in}{-5.030279in}}%
\pgfpathcurveto{\pgfqpoint{4.105501in}{-5.023768in}}{\pgfqpoint{4.109159in}{-5.014935in}}{\pgfqpoint{4.109159in}{-5.005727in}}%
\pgfpathcurveto{\pgfqpoint{4.109159in}{-4.996519in}}{\pgfqpoint{4.105501in}{-4.987686in}}{\pgfqpoint{4.098989in}{-4.981175in}}%
\pgfpathcurveto{\pgfqpoint{4.092478in}{-4.974663in}}{\pgfqpoint{4.083645in}{-4.971005in}}{\pgfqpoint{4.074437in}{-4.971005in}}%
\pgfpathcurveto{\pgfqpoint{4.065228in}{-4.971005in}}{\pgfqpoint{4.056396in}{-4.974663in}}{\pgfqpoint{4.049885in}{-4.981175in}}%
\pgfpathcurveto{\pgfqpoint{4.043373in}{-4.987686in}}{\pgfqpoint{4.039715in}{-4.996519in}}{\pgfqpoint{4.039715in}{-5.005727in}}%
\pgfpathcurveto{\pgfqpoint{4.039715in}{-5.014935in}}{\pgfqpoint{4.043373in}{-5.023768in}}{\pgfqpoint{4.049885in}{-5.030279in}}%
\pgfpathcurveto{\pgfqpoint{4.056396in}{-5.036791in}}{\pgfqpoint{4.065228in}{-5.040449in}}{\pgfqpoint{4.074437in}{-5.040449in}}%
\pgfpathlineto{\pgfqpoint{4.074437in}{-5.040449in}}%
\pgfpathclose%
\pgfusepath{stroke,fill}%
\end{pgfscope}%
\begin{pgfscope}%
\pgfpathrectangle{\pgfqpoint{1.374500in}{0.082500in}}{\pgfqpoint{2.419000in}{2.419000in}}%
\pgfusepath{clip}%
\pgfsetbuttcap%
\pgfsetroundjoin%
\definecolor{currentfill}{rgb}{1.000000,0.662745,0.054902}%
\pgfsetfillcolor{currentfill}%
\pgfsetfillopacity{0.761663}%
\pgfsetlinewidth{1.003750pt}%
\definecolor{currentstroke}{rgb}{1.000000,0.662745,0.054902}%
\pgfsetstrokecolor{currentstroke}%
\pgfsetstrokeopacity{0.761663}%
\pgfsetdash{}{0pt}%
\pgfpathmoveto{\pgfqpoint{0.448491in}{-5.490340in}}%
\pgfpathcurveto{\pgfqpoint{0.457699in}{-5.490340in}}{\pgfqpoint{0.466532in}{-5.486681in}}{\pgfqpoint{0.473043in}{-5.480170in}}%
\pgfpathcurveto{\pgfqpoint{0.479554in}{-5.473659in}}{\pgfqpoint{0.483213in}{-5.464826in}}{\pgfqpoint{0.483213in}{-5.455618in}}%
\pgfpathcurveto{\pgfqpoint{0.483213in}{-5.446409in}}{\pgfqpoint{0.479554in}{-5.437577in}}{\pgfqpoint{0.473043in}{-5.431065in}}%
\pgfpathcurveto{\pgfqpoint{0.466532in}{-5.424554in}}{\pgfqpoint{0.457699in}{-5.420895in}}{\pgfqpoint{0.448491in}{-5.420895in}}%
\pgfpathcurveto{\pgfqpoint{0.439282in}{-5.420895in}}{\pgfqpoint{0.430450in}{-5.424554in}}{\pgfqpoint{0.423938in}{-5.431065in}}%
\pgfpathcurveto{\pgfqpoint{0.417427in}{-5.437577in}}{\pgfqpoint{0.413768in}{-5.446409in}}{\pgfqpoint{0.413768in}{-5.455618in}}%
\pgfpathcurveto{\pgfqpoint{0.413768in}{-5.464826in}}{\pgfqpoint{0.417427in}{-5.473659in}}{\pgfqpoint{0.423938in}{-5.480170in}}%
\pgfpathcurveto{\pgfqpoint{0.430450in}{-5.486681in}}{\pgfqpoint{0.439282in}{-5.490340in}}{\pgfqpoint{0.448491in}{-5.490340in}}%
\pgfpathlineto{\pgfqpoint{0.448491in}{-5.490340in}}%
\pgfpathclose%
\pgfusepath{stroke,fill}%
\end{pgfscope}%
\begin{pgfscope}%
\pgfpathrectangle{\pgfqpoint{1.374500in}{0.082500in}}{\pgfqpoint{2.419000in}{2.419000in}}%
\pgfusepath{clip}%
\pgfsetbuttcap%
\pgfsetroundjoin%
\definecolor{currentfill}{rgb}{1.000000,0.662745,0.054902}%
\pgfsetfillcolor{currentfill}%
\pgfsetfillopacity{0.761663}%
\pgfsetlinewidth{1.003750pt}%
\definecolor{currentstroke}{rgb}{1.000000,0.662745,0.054902}%
\pgfsetstrokecolor{currentstroke}%
\pgfsetstrokeopacity{0.761663}%
\pgfsetdash{}{0pt}%
\pgfpathmoveto{\pgfqpoint{9.490207in}{-5.490340in}}%
\pgfpathcurveto{\pgfqpoint{9.499415in}{-5.490340in}}{\pgfqpoint{9.508248in}{-5.486681in}}{\pgfqpoint{9.514759in}{-5.480170in}}%
\pgfpathcurveto{\pgfqpoint{9.521271in}{-5.473659in}}{\pgfqpoint{9.524929in}{-5.464826in}}{\pgfqpoint{9.524929in}{-5.455618in}}%
\pgfpathcurveto{\pgfqpoint{9.524929in}{-5.446409in}}{\pgfqpoint{9.521271in}{-5.437577in}}{\pgfqpoint{9.514759in}{-5.431065in}}%
\pgfpathcurveto{\pgfqpoint{9.508248in}{-5.424554in}}{\pgfqpoint{9.499415in}{-5.420895in}}{\pgfqpoint{9.490207in}{-5.420895in}}%
\pgfpathcurveto{\pgfqpoint{9.480999in}{-5.420895in}}{\pgfqpoint{9.472166in}{-5.424554in}}{\pgfqpoint{9.465655in}{-5.431065in}}%
\pgfpathcurveto{\pgfqpoint{9.459143in}{-5.437577in}}{\pgfqpoint{9.455485in}{-5.446409in}}{\pgfqpoint{9.455485in}{-5.455618in}}%
\pgfpathcurveto{\pgfqpoint{9.455485in}{-5.464826in}}{\pgfqpoint{9.459143in}{-5.473659in}}{\pgfqpoint{9.465655in}{-5.480170in}}%
\pgfpathcurveto{\pgfqpoint{9.472166in}{-5.486681in}}{\pgfqpoint{9.480999in}{-5.490340in}}{\pgfqpoint{9.490207in}{-5.490340in}}%
\pgfpathlineto{\pgfqpoint{9.490207in}{-5.490340in}}%
\pgfpathclose%
\pgfusepath{stroke,fill}%
\end{pgfscope}%
\begin{pgfscope}%
\pgfpathrectangle{\pgfqpoint{1.374500in}{0.082500in}}{\pgfqpoint{2.419000in}{2.419000in}}%
\pgfusepath{clip}%
\pgfsetbuttcap%
\pgfsetroundjoin%
\definecolor{currentfill}{rgb}{1.000000,0.662745,0.054902}%
\pgfsetfillcolor{currentfill}%
\pgfsetfillopacity{0.784469}%
\pgfsetlinewidth{1.003750pt}%
\definecolor{currentstroke}{rgb}{1.000000,0.662745,0.054902}%
\pgfsetstrokecolor{currentstroke}%
\pgfsetstrokeopacity{0.784469}%
\pgfsetdash{}{0pt}%
\pgfpathmoveto{\pgfqpoint{5.924219in}{-5.970373in}}%
\pgfpathcurveto{\pgfqpoint{5.933427in}{-5.970373in}}{\pgfqpoint{5.942260in}{-5.966714in}}{\pgfqpoint{5.948771in}{-5.960203in}}%
\pgfpathcurveto{\pgfqpoint{5.955282in}{-5.953691in}}{\pgfqpoint{5.958941in}{-5.944859in}}{\pgfqpoint{5.958941in}{-5.935650in}}%
\pgfpathcurveto{\pgfqpoint{5.958941in}{-5.926442in}}{\pgfqpoint{5.955282in}{-5.917609in}}{\pgfqpoint{5.948771in}{-5.911098in}}%
\pgfpathcurveto{\pgfqpoint{5.942260in}{-5.904587in}}{\pgfqpoint{5.933427in}{-5.900928in}}{\pgfqpoint{5.924219in}{-5.900928in}}%
\pgfpathcurveto{\pgfqpoint{5.915010in}{-5.900928in}}{\pgfqpoint{5.906178in}{-5.904587in}}{\pgfqpoint{5.899666in}{-5.911098in}}%
\pgfpathcurveto{\pgfqpoint{5.893155in}{-5.917609in}}{\pgfqpoint{5.889497in}{-5.926442in}}{\pgfqpoint{5.889497in}{-5.935650in}}%
\pgfpathcurveto{\pgfqpoint{5.889497in}{-5.944859in}}{\pgfqpoint{5.893155in}{-5.953691in}}{\pgfqpoint{5.899666in}{-5.960203in}}%
\pgfpathcurveto{\pgfqpoint{5.906178in}{-5.966714in}}{\pgfqpoint{5.915010in}{-5.970373in}}{\pgfqpoint{5.924219in}{-5.970373in}}%
\pgfpathlineto{\pgfqpoint{5.924219in}{-5.970373in}}%
\pgfpathclose%
\pgfusepath{stroke,fill}%
\end{pgfscope}%
\begin{pgfscope}%
\pgfpathrectangle{\pgfqpoint{1.374500in}{0.082500in}}{\pgfqpoint{2.419000in}{2.419000in}}%
\pgfusepath{clip}%
\pgfsetbuttcap%
\pgfsetroundjoin%
\definecolor{currentfill}{rgb}{1.000000,0.662745,0.054902}%
\pgfsetfillcolor{currentfill}%
\pgfsetfillopacity{0.808856}%
\pgfsetlinewidth{1.003750pt}%
\definecolor{currentstroke}{rgb}{1.000000,0.662745,0.054902}%
\pgfsetstrokecolor{currentstroke}%
\pgfsetstrokeopacity{0.808856}%
\pgfsetdash{}{0pt}%
\pgfpathmoveto{\pgfqpoint{2.111032in}{-6.483682in}}%
\pgfpathcurveto{\pgfqpoint{2.120240in}{-6.483682in}}{\pgfqpoint{2.129073in}{-6.480023in}}{\pgfqpoint{2.135584in}{-6.473512in}}%
\pgfpathcurveto{\pgfqpoint{2.142096in}{-6.467001in}}{\pgfqpoint{2.145754in}{-6.458168in}}{\pgfqpoint{2.145754in}{-6.448960in}}%
\pgfpathcurveto{\pgfqpoint{2.145754in}{-6.439751in}}{\pgfqpoint{2.142096in}{-6.430919in}}{\pgfqpoint{2.135584in}{-6.424407in}}%
\pgfpathcurveto{\pgfqpoint{2.129073in}{-6.417896in}}{\pgfqpoint{2.120240in}{-6.414237in}}{\pgfqpoint{2.111032in}{-6.414237in}}%
\pgfpathcurveto{\pgfqpoint{2.101824in}{-6.414237in}}{\pgfqpoint{2.092991in}{-6.417896in}}{\pgfqpoint{2.086480in}{-6.424407in}}%
\pgfpathcurveto{\pgfqpoint{2.079968in}{-6.430919in}}{\pgfqpoint{2.076310in}{-6.439751in}}{\pgfqpoint{2.076310in}{-6.448960in}}%
\pgfpathcurveto{\pgfqpoint{2.076310in}{-6.458168in}}{\pgfqpoint{2.079968in}{-6.467001in}}{\pgfqpoint{2.086480in}{-6.473512in}}%
\pgfpathcurveto{\pgfqpoint{2.092991in}{-6.480023in}}{\pgfqpoint{2.101824in}{-6.483682in}}{\pgfqpoint{2.111032in}{-6.483682in}}%
\pgfpathlineto{\pgfqpoint{2.111032in}{-6.483682in}}%
\pgfpathclose%
\pgfusepath{stroke,fill}%
\end{pgfscope}%
\begin{pgfscope}%
\pgfpathrectangle{\pgfqpoint{1.374500in}{0.082500in}}{\pgfqpoint{2.419000in}{2.419000in}}%
\pgfusepath{clip}%
\pgfsetbuttcap%
\pgfsetroundjoin%
\definecolor{currentfill}{rgb}{1.000000,0.662745,0.054902}%
\pgfsetfillcolor{currentfill}%
\pgfsetfillopacity{0.808856}%
\pgfsetlinewidth{1.003750pt}%
\definecolor{currentstroke}{rgb}{1.000000,0.662745,0.054902}%
\pgfsetstrokecolor{currentstroke}%
\pgfsetstrokeopacity{0.808856}%
\pgfsetdash{}{0pt}%
\pgfpathmoveto{\pgfqpoint{11.779531in}{-6.483682in}}%
\pgfpathcurveto{\pgfqpoint{11.788740in}{-6.483682in}}{\pgfqpoint{11.797572in}{-6.480023in}}{\pgfqpoint{11.804084in}{-6.473512in}}%
\pgfpathcurveto{\pgfqpoint{11.810595in}{-6.467001in}}{\pgfqpoint{11.814253in}{-6.458168in}}{\pgfqpoint{11.814253in}{-6.448960in}}%
\pgfpathcurveto{\pgfqpoint{11.814253in}{-6.439751in}}{\pgfqpoint{11.810595in}{-6.430919in}}{\pgfqpoint{11.804084in}{-6.424407in}}%
\pgfpathcurveto{\pgfqpoint{11.797572in}{-6.417896in}}{\pgfqpoint{11.788740in}{-6.414237in}}{\pgfqpoint{11.779531in}{-6.414237in}}%
\pgfpathcurveto{\pgfqpoint{11.770323in}{-6.414237in}}{\pgfqpoint{11.761490in}{-6.417896in}}{\pgfqpoint{11.754979in}{-6.424407in}}%
\pgfpathcurveto{\pgfqpoint{11.748468in}{-6.430919in}}{\pgfqpoint{11.744809in}{-6.439751in}}{\pgfqpoint{11.744809in}{-6.448960in}}%
\pgfpathcurveto{\pgfqpoint{11.744809in}{-6.458168in}}{\pgfqpoint{11.748468in}{-6.467001in}}{\pgfqpoint{11.754979in}{-6.473512in}}%
\pgfpathcurveto{\pgfqpoint{11.761490in}{-6.480023in}}{\pgfqpoint{11.770323in}{-6.483682in}}{\pgfqpoint{11.779531in}{-6.483682in}}%
\pgfpathlineto{\pgfqpoint{11.779531in}{-6.483682in}}%
\pgfpathclose%
\pgfusepath{stroke,fill}%
\end{pgfscope}%
\begin{pgfscope}%
\pgfpathrectangle{\pgfqpoint{1.374500in}{0.082500in}}{\pgfqpoint{2.419000in}{2.419000in}}%
\pgfusepath{clip}%
\pgfsetbuttcap%
\pgfsetroundjoin%
\definecolor{currentfill}{rgb}{1.000000,0.662745,0.054902}%
\pgfsetfillcolor{currentfill}%
\pgfsetfillopacity{0.834994}%
\pgfsetlinewidth{1.003750pt}%
\definecolor{currentstroke}{rgb}{1.000000,0.662745,0.054902}%
\pgfsetstrokecolor{currentstroke}%
\pgfsetstrokeopacity{0.834994}%
\pgfsetdash{}{0pt}%
\pgfpathmoveto{\pgfqpoint{8.039667in}{-7.033852in}}%
\pgfpathcurveto{\pgfqpoint{8.048876in}{-7.033852in}}{\pgfqpoint{8.057708in}{-7.030193in}}{\pgfqpoint{8.064219in}{-7.023682in}}%
\pgfpathcurveto{\pgfqpoint{8.070731in}{-7.017171in}}{\pgfqpoint{8.074389in}{-7.008338in}}{\pgfqpoint{8.074389in}{-6.999130in}}%
\pgfpathcurveto{\pgfqpoint{8.074389in}{-6.989921in}}{\pgfqpoint{8.070731in}{-6.981089in}}{\pgfqpoint{8.064219in}{-6.974577in}}%
\pgfpathcurveto{\pgfqpoint{8.057708in}{-6.968066in}}{\pgfqpoint{8.048876in}{-6.964407in}}{\pgfqpoint{8.039667in}{-6.964407in}}%
\pgfpathcurveto{\pgfqpoint{8.030459in}{-6.964407in}}{\pgfqpoint{8.021626in}{-6.968066in}}{\pgfqpoint{8.015115in}{-6.974577in}}%
\pgfpathcurveto{\pgfqpoint{8.008603in}{-6.981089in}}{\pgfqpoint{8.004945in}{-6.989921in}}{\pgfqpoint{8.004945in}{-6.999130in}}%
\pgfpathcurveto{\pgfqpoint{8.004945in}{-7.008338in}}{\pgfqpoint{8.008603in}{-7.017171in}}{\pgfqpoint{8.015115in}{-7.023682in}}%
\pgfpathcurveto{\pgfqpoint{8.021626in}{-7.030193in}}{\pgfqpoint{8.030459in}{-7.033852in}}{\pgfqpoint{8.039667in}{-7.033852in}}%
\pgfpathlineto{\pgfqpoint{8.039667in}{-7.033852in}}%
\pgfpathclose%
\pgfusepath{stroke,fill}%
\end{pgfscope}%
\begin{pgfscope}%
\pgfpathrectangle{\pgfqpoint{1.374500in}{0.082500in}}{\pgfqpoint{2.419000in}{2.419000in}}%
\pgfusepath{clip}%
\pgfsetbuttcap%
\pgfsetroundjoin%
\definecolor{currentfill}{rgb}{1.000000,0.662745,0.054902}%
\pgfsetfillcolor{currentfill}%
\pgfsetfillopacity{0.863079}%
\pgfsetlinewidth{1.003750pt}%
\definecolor{currentstroke}{rgb}{1.000000,0.662745,0.054902}%
\pgfsetstrokecolor{currentstroke}%
\pgfsetstrokeopacity{0.863079}%
\pgfsetdash{}{0pt}%
\pgfpathmoveto{\pgfqpoint{4.021241in}{-7.625001in}}%
\pgfpathcurveto{\pgfqpoint{4.030449in}{-7.625001in}}{\pgfqpoint{4.039282in}{-7.621343in}}{\pgfqpoint{4.045793in}{-7.614831in}}%
\pgfpathcurveto{\pgfqpoint{4.052304in}{-7.608320in}}{\pgfqpoint{4.055963in}{-7.599487in}}{\pgfqpoint{4.055963in}{-7.590279in}}%
\pgfpathcurveto{\pgfqpoint{4.055963in}{-7.581071in}}{\pgfqpoint{4.052304in}{-7.572238in}}{\pgfqpoint{4.045793in}{-7.565727in}}%
\pgfpathcurveto{\pgfqpoint{4.039282in}{-7.559215in}}{\pgfqpoint{4.030449in}{-7.555557in}}{\pgfqpoint{4.021241in}{-7.555557in}}%
\pgfpathcurveto{\pgfqpoint{4.012032in}{-7.555557in}}{\pgfqpoint{4.003200in}{-7.559215in}}{\pgfqpoint{3.996689in}{-7.565727in}}%
\pgfpathcurveto{\pgfqpoint{3.990177in}{-7.572238in}}{\pgfqpoint{3.986519in}{-7.581071in}}{\pgfqpoint{3.986519in}{-7.590279in}}%
\pgfpathcurveto{\pgfqpoint{3.986519in}{-7.599487in}}{\pgfqpoint{3.990177in}{-7.608320in}}{\pgfqpoint{3.996689in}{-7.614831in}}%
\pgfpathcurveto{\pgfqpoint{4.003200in}{-7.621343in}}{\pgfqpoint{4.012032in}{-7.625001in}}{\pgfqpoint{4.021241in}{-7.625001in}}%
\pgfpathlineto{\pgfqpoint{4.021241in}{-7.625001in}}%
\pgfpathclose%
\pgfusepath{stroke,fill}%
\end{pgfscope}%
\begin{pgfscope}%
\pgfpathrectangle{\pgfqpoint{1.374500in}{0.082500in}}{\pgfqpoint{2.419000in}{2.419000in}}%
\pgfusepath{clip}%
\pgfsetbuttcap%
\pgfsetroundjoin%
\definecolor{currentfill}{rgb}{1.000000,0.662745,0.054902}%
\pgfsetfillcolor{currentfill}%
\pgfsetfillopacity{0.893337}%
\pgfsetlinewidth{1.003750pt}%
\definecolor{currentstroke}{rgb}{1.000000,0.662745,0.054902}%
\pgfsetstrokecolor{currentstroke}%
\pgfsetstrokeopacity{0.893337}%
\pgfsetdash{}{0pt}%
\pgfpathmoveto{\pgfqpoint{10.482442in}{-8.261885in}}%
\pgfpathcurveto{\pgfqpoint{10.491651in}{-8.261885in}}{\pgfqpoint{10.500483in}{-8.258227in}}{\pgfqpoint{10.506995in}{-8.251715in}}%
\pgfpathcurveto{\pgfqpoint{10.513506in}{-8.245204in}}{\pgfqpoint{10.517165in}{-8.236371in}}{\pgfqpoint{10.517165in}{-8.227163in}}%
\pgfpathcurveto{\pgfqpoint{10.517165in}{-8.217955in}}{\pgfqpoint{10.513506in}{-8.209122in}}{\pgfqpoint{10.506995in}{-8.202611in}}%
\pgfpathcurveto{\pgfqpoint{10.500483in}{-8.196099in}}{\pgfqpoint{10.491651in}{-8.192441in}}{\pgfqpoint{10.482442in}{-8.192441in}}%
\pgfpathcurveto{\pgfqpoint{10.473234in}{-8.192441in}}{\pgfqpoint{10.464401in}{-8.196099in}}{\pgfqpoint{10.457890in}{-8.202611in}}%
\pgfpathcurveto{\pgfqpoint{10.451379in}{-8.209122in}}{\pgfqpoint{10.447720in}{-8.217955in}}{\pgfqpoint{10.447720in}{-8.227163in}}%
\pgfpathcurveto{\pgfqpoint{10.447720in}{-8.236371in}}{\pgfqpoint{10.451379in}{-8.245204in}}{\pgfqpoint{10.457890in}{-8.251715in}}%
\pgfpathcurveto{\pgfqpoint{10.464401in}{-8.258227in}}{\pgfqpoint{10.473234in}{-8.261885in}}{\pgfqpoint{10.482442in}{-8.261885in}}%
\pgfpathlineto{\pgfqpoint{10.482442in}{-8.261885in}}%
\pgfpathclose%
\pgfusepath{stroke,fill}%
\end{pgfscope}%
\begin{pgfscope}%
\pgfpathrectangle{\pgfqpoint{1.374500in}{0.082500in}}{\pgfqpoint{2.419000in}{2.419000in}}%
\pgfusepath{clip}%
\pgfsetbuttcap%
\pgfsetroundjoin%
\definecolor{currentfill}{rgb}{1.000000,0.662745,0.054902}%
\pgfsetfillcolor{currentfill}%
\pgfsetfillopacity{0.926030}%
\pgfsetlinewidth{1.003750pt}%
\definecolor{currentstroke}{rgb}{1.000000,0.662745,0.054902}%
\pgfsetstrokecolor{currentstroke}%
\pgfsetstrokeopacity{0.926030}%
\pgfsetdash{}{0pt}%
\pgfpathmoveto{\pgfqpoint{6.238913in}{-8.950025in}}%
\pgfpathcurveto{\pgfqpoint{6.248122in}{-8.950025in}}{\pgfqpoint{6.256954in}{-8.946367in}}{\pgfqpoint{6.263466in}{-8.939855in}}%
\pgfpathcurveto{\pgfqpoint{6.269977in}{-8.933344in}}{\pgfqpoint{6.273635in}{-8.924511in}}{\pgfqpoint{6.273635in}{-8.915303in}}%
\pgfpathcurveto{\pgfqpoint{6.273635in}{-8.906094in}}{\pgfqpoint{6.269977in}{-8.897262in}}{\pgfqpoint{6.263466in}{-8.890751in}}%
\pgfpathcurveto{\pgfqpoint{6.256954in}{-8.884239in}}{\pgfqpoint{6.248122in}{-8.880581in}}{\pgfqpoint{6.238913in}{-8.880581in}}%
\pgfpathcurveto{\pgfqpoint{6.229705in}{-8.880581in}}{\pgfqpoint{6.220872in}{-8.884239in}}{\pgfqpoint{6.214361in}{-8.890751in}}%
\pgfpathcurveto{\pgfqpoint{6.207850in}{-8.897262in}}{\pgfqpoint{6.204191in}{-8.906094in}}{\pgfqpoint{6.204191in}{-8.915303in}}%
\pgfpathcurveto{\pgfqpoint{6.204191in}{-8.924511in}}{\pgfqpoint{6.207850in}{-8.933344in}}{\pgfqpoint{6.214361in}{-8.939855in}}%
\pgfpathcurveto{\pgfqpoint{6.220872in}{-8.946367in}}{\pgfqpoint{6.229705in}{-8.950025in}}{\pgfqpoint{6.238913in}{-8.950025in}}%
\pgfpathlineto{\pgfqpoint{6.238913in}{-8.950025in}}%
\pgfpathclose%
\pgfusepath{stroke,fill}%
\end{pgfscope}%
\begin{pgfscope}%
\pgfpathrectangle{\pgfqpoint{1.374500in}{0.082500in}}{\pgfqpoint{2.419000in}{2.419000in}}%
\pgfusepath{clip}%
\pgfsetbuttcap%
\pgfsetroundjoin%
\definecolor{currentfill}{rgb}{1.000000,0.662745,0.054902}%
\pgfsetfillcolor{currentfill}%
\pgfsetlinewidth{1.003750pt}%
\definecolor{currentstroke}{rgb}{1.000000,0.662745,0.054902}%
\pgfsetstrokecolor{currentstroke}%
\pgfsetdash{}{0pt}%
\pgfpathmoveto{\pgfqpoint{8.844779in}{-10.506989in}}%
\pgfpathcurveto{\pgfqpoint{8.853988in}{-10.506989in}}{\pgfqpoint{8.862820in}{-10.503330in}}{\pgfqpoint{8.869332in}{-10.496819in}}%
\pgfpathcurveto{\pgfqpoint{8.875843in}{-10.490307in}}{\pgfqpoint{8.879501in}{-10.481475in}}{\pgfqpoint{8.879501in}{-10.472266in}}%
\pgfpathcurveto{\pgfqpoint{8.879501in}{-10.463058in}}{\pgfqpoint{8.875843in}{-10.454225in}}{\pgfqpoint{8.869332in}{-10.447714in}}%
\pgfpathcurveto{\pgfqpoint{8.862820in}{-10.441203in}}{\pgfqpoint{8.853988in}{-10.437544in}}{\pgfqpoint{8.844779in}{-10.437544in}}%
\pgfpathcurveto{\pgfqpoint{8.835571in}{-10.437544in}}{\pgfqpoint{8.826738in}{-10.441203in}}{\pgfqpoint{8.820227in}{-10.447714in}}%
\pgfpathcurveto{\pgfqpoint{8.813716in}{-10.454225in}}{\pgfqpoint{8.810057in}{-10.463058in}}{\pgfqpoint{8.810057in}{-10.472266in}}%
\pgfpathcurveto{\pgfqpoint{8.810057in}{-10.481475in}}{\pgfqpoint{8.813716in}{-10.490307in}}{\pgfqpoint{8.820227in}{-10.496819in}}%
\pgfpathcurveto{\pgfqpoint{8.826738in}{-10.503330in}}{\pgfqpoint{8.835571in}{-10.506989in}}{\pgfqpoint{8.844779in}{-10.506989in}}%
\pgfpathlineto{\pgfqpoint{8.844779in}{-10.506989in}}%
\pgfpathclose%
\pgfusepath{stroke,fill}%
\end{pgfscope}%
\begin{pgfscope}%
\pgfpathrectangle{\pgfqpoint{1.374500in}{0.082500in}}{\pgfqpoint{2.419000in}{2.419000in}}%
\pgfusepath{clip}%
\pgfsetbuttcap%
\pgfsetroundjoin%
\definecolor{currentfill}{rgb}{0.741176,0.121569,0.003922}%
\pgfsetfillcolor{currentfill}%
\pgfsetfillopacity{0.300000}%
\pgfsetlinewidth{1.003750pt}%
\definecolor{currentstroke}{rgb}{0.741176,0.121569,0.003922}%
\pgfsetstrokecolor{currentstroke}%
\pgfsetstrokeopacity{0.300000}%
\pgfsetdash{}{0pt}%
\pgfpathmoveto{\pgfqpoint{2.867143in}{4.263115in}}%
\pgfpathcurveto{\pgfqpoint{2.876352in}{4.263115in}}{\pgfqpoint{2.885184in}{4.266774in}}{\pgfqpoint{2.891696in}{4.273285in}}%
\pgfpathcurveto{\pgfqpoint{2.898207in}{4.279796in}}{\pgfqpoint{2.901866in}{4.288629in}}{\pgfqpoint{2.901866in}{4.297837in}}%
\pgfpathcurveto{\pgfqpoint{2.901866in}{4.307046in}}{\pgfqpoint{2.898207in}{4.315878in}}{\pgfqpoint{2.891696in}{4.322390in}}%
\pgfpathcurveto{\pgfqpoint{2.885184in}{4.328901in}}{\pgfqpoint{2.876352in}{4.332559in}}{\pgfqpoint{2.867143in}{4.332559in}}%
\pgfpathcurveto{\pgfqpoint{2.857935in}{4.332559in}}{\pgfqpoint{2.849103in}{4.328901in}}{\pgfqpoint{2.842591in}{4.322390in}}%
\pgfpathcurveto{\pgfqpoint{2.836080in}{4.315878in}}{\pgfqpoint{2.832421in}{4.307046in}}{\pgfqpoint{2.832421in}{4.297837in}}%
\pgfpathcurveto{\pgfqpoint{2.832421in}{4.288629in}}{\pgfqpoint{2.836080in}{4.279796in}}{\pgfqpoint{2.842591in}{4.273285in}}%
\pgfpathcurveto{\pgfqpoint{2.849103in}{4.266774in}}{\pgfqpoint{2.857935in}{4.263115in}}{\pgfqpoint{2.867143in}{4.263115in}}%
\pgfpathlineto{\pgfqpoint{2.867143in}{4.263115in}}%
\pgfpathclose%
\pgfusepath{stroke,fill}%
\end{pgfscope}%
\begin{pgfscope}%
\pgfpathrectangle{\pgfqpoint{1.374500in}{0.082500in}}{\pgfqpoint{2.419000in}{2.419000in}}%
\pgfusepath{clip}%
\pgfsetbuttcap%
\pgfsetroundjoin%
\definecolor{currentfill}{rgb}{0.741176,0.121569,0.003922}%
\pgfsetfillcolor{currentfill}%
\pgfsetfillopacity{0.304985}%
\pgfsetlinewidth{1.003750pt}%
\definecolor{currentstroke}{rgb}{0.741176,0.121569,0.003922}%
\pgfsetstrokecolor{currentstroke}%
\pgfsetstrokeopacity{0.304985}%
\pgfsetdash{}{0pt}%
\pgfpathmoveto{\pgfqpoint{3.450137in}{4.162067in}}%
\pgfpathcurveto{\pgfqpoint{3.459346in}{4.162067in}}{\pgfqpoint{3.468178in}{4.165726in}}{\pgfqpoint{3.474689in}{4.172237in}}%
\pgfpathcurveto{\pgfqpoint{3.481201in}{4.178749in}}{\pgfqpoint{3.484859in}{4.187581in}}{\pgfqpoint{3.484859in}{4.196790in}}%
\pgfpathcurveto{\pgfqpoint{3.484859in}{4.205998in}}{\pgfqpoint{3.481201in}{4.214831in}}{\pgfqpoint{3.474689in}{4.221342in}}%
\pgfpathcurveto{\pgfqpoint{3.468178in}{4.227853in}}{\pgfqpoint{3.459346in}{4.231512in}}{\pgfqpoint{3.450137in}{4.231512in}}%
\pgfpathcurveto{\pgfqpoint{3.440929in}{4.231512in}}{\pgfqpoint{3.432096in}{4.227853in}}{\pgfqpoint{3.425585in}{4.221342in}}%
\pgfpathcurveto{\pgfqpoint{3.419073in}{4.214831in}}{\pgfqpoint{3.415415in}{4.205998in}}{\pgfqpoint{3.415415in}{4.196790in}}%
\pgfpathcurveto{\pgfqpoint{3.415415in}{4.187581in}}{\pgfqpoint{3.419073in}{4.178749in}}{\pgfqpoint{3.425585in}{4.172237in}}%
\pgfpathcurveto{\pgfqpoint{3.432096in}{4.165726in}}{\pgfqpoint{3.440929in}{4.162067in}}{\pgfqpoint{3.450137in}{4.162067in}}%
\pgfpathlineto{\pgfqpoint{3.450137in}{4.162067in}}%
\pgfpathclose%
\pgfusepath{stroke,fill}%
\end{pgfscope}%
\begin{pgfscope}%
\pgfpathrectangle{\pgfqpoint{1.374500in}{0.082500in}}{\pgfqpoint{2.419000in}{2.419000in}}%
\pgfusepath{clip}%
\pgfsetbuttcap%
\pgfsetroundjoin%
\definecolor{currentfill}{rgb}{0.741176,0.121569,0.003922}%
\pgfsetfillcolor{currentfill}%
\pgfsetfillopacity{0.310194}%
\pgfsetlinewidth{1.003750pt}%
\definecolor{currentstroke}{rgb}{0.741176,0.121569,0.003922}%
\pgfsetstrokecolor{currentstroke}%
\pgfsetstrokeopacity{0.310194}%
\pgfsetdash{}{0pt}%
\pgfpathmoveto{\pgfqpoint{4.059459in}{4.056457in}}%
\pgfpathcurveto{\pgfqpoint{4.068667in}{4.056457in}}{\pgfqpoint{4.077500in}{4.060115in}}{\pgfqpoint{4.084011in}{4.066627in}}%
\pgfpathcurveto{\pgfqpoint{4.090523in}{4.073138in}}{\pgfqpoint{4.094181in}{4.081970in}}{\pgfqpoint{4.094181in}{4.091179in}}%
\pgfpathcurveto{\pgfqpoint{4.094181in}{4.100387in}}{\pgfqpoint{4.090523in}{4.109220in}}{\pgfqpoint{4.084011in}{4.115731in}}%
\pgfpathcurveto{\pgfqpoint{4.077500in}{4.122243in}}{\pgfqpoint{4.068667in}{4.125901in}}{\pgfqpoint{4.059459in}{4.125901in}}%
\pgfpathcurveto{\pgfqpoint{4.050251in}{4.125901in}}{\pgfqpoint{4.041418in}{4.122243in}}{\pgfqpoint{4.034907in}{4.115731in}}%
\pgfpathcurveto{\pgfqpoint{4.028395in}{4.109220in}}{\pgfqpoint{4.024737in}{4.100387in}}{\pgfqpoint{4.024737in}{4.091179in}}%
\pgfpathcurveto{\pgfqpoint{4.024737in}{4.081970in}}{\pgfqpoint{4.028395in}{4.073138in}}{\pgfqpoint{4.034907in}{4.066627in}}%
\pgfpathcurveto{\pgfqpoint{4.041418in}{4.060115in}}{\pgfqpoint{4.050251in}{4.056457in}}{\pgfqpoint{4.059459in}{4.056457in}}%
\pgfpathlineto{\pgfqpoint{4.059459in}{4.056457in}}%
\pgfpathclose%
\pgfusepath{stroke,fill}%
\end{pgfscope}%
\begin{pgfscope}%
\pgfpathrectangle{\pgfqpoint{1.374500in}{0.082500in}}{\pgfqpoint{2.419000in}{2.419000in}}%
\pgfusepath{clip}%
\pgfsetbuttcap%
\pgfsetroundjoin%
\definecolor{currentfill}{rgb}{0.741176,0.121569,0.003922}%
\pgfsetfillcolor{currentfill}%
\pgfsetfillopacity{0.312888}%
\pgfsetlinewidth{1.003750pt}%
\definecolor{currentstroke}{rgb}{0.741176,0.121569,0.003922}%
\pgfsetstrokecolor{currentstroke}%
\pgfsetstrokeopacity{0.312888}%
\pgfsetdash{}{0pt}%
\pgfpathmoveto{\pgfqpoint{2.781902in}{4.001842in}}%
\pgfpathcurveto{\pgfqpoint{2.791111in}{4.001842in}}{\pgfqpoint{2.799943in}{4.005501in}}{\pgfqpoint{2.806455in}{4.012012in}}%
\pgfpathcurveto{\pgfqpoint{2.812966in}{4.018524in}}{\pgfqpoint{2.816625in}{4.027356in}}{\pgfqpoint{2.816625in}{4.036565in}}%
\pgfpathcurveto{\pgfqpoint{2.816625in}{4.045773in}}{\pgfqpoint{2.812966in}{4.054606in}}{\pgfqpoint{2.806455in}{4.061117in}}%
\pgfpathcurveto{\pgfqpoint{2.799943in}{4.067628in}}{\pgfqpoint{2.791111in}{4.071287in}}{\pgfqpoint{2.781902in}{4.071287in}}%
\pgfpathcurveto{\pgfqpoint{2.772694in}{4.071287in}}{\pgfqpoint{2.763861in}{4.067628in}}{\pgfqpoint{2.757350in}{4.061117in}}%
\pgfpathcurveto{\pgfqpoint{2.750839in}{4.054606in}}{\pgfqpoint{2.747180in}{4.045773in}}{\pgfqpoint{2.747180in}{4.036565in}}%
\pgfpathcurveto{\pgfqpoint{2.747180in}{4.027356in}}{\pgfqpoint{2.750839in}{4.018524in}}{\pgfqpoint{2.757350in}{4.012012in}}%
\pgfpathcurveto{\pgfqpoint{2.763861in}{4.005501in}}{\pgfqpoint{2.772694in}{4.001842in}}{\pgfqpoint{2.781902in}{4.001842in}}%
\pgfpathlineto{\pgfqpoint{2.781902in}{4.001842in}}%
\pgfpathclose%
\pgfusepath{stroke,fill}%
\end{pgfscope}%
\begin{pgfscope}%
\pgfpathrectangle{\pgfqpoint{1.374500in}{0.082500in}}{\pgfqpoint{2.419000in}{2.419000in}}%
\pgfusepath{clip}%
\pgfsetbuttcap%
\pgfsetroundjoin%
\definecolor{currentfill}{rgb}{0.741176,0.121569,0.003922}%
\pgfsetfillcolor{currentfill}%
\pgfsetfillopacity{0.315644}%
\pgfsetlinewidth{1.003750pt}%
\definecolor{currentstroke}{rgb}{0.741176,0.121569,0.003922}%
\pgfsetstrokecolor{currentstroke}%
\pgfsetstrokeopacity{0.315644}%
\pgfsetdash{}{0pt}%
\pgfpathmoveto{\pgfqpoint{4.696934in}{3.945966in}}%
\pgfpathcurveto{\pgfqpoint{4.706142in}{3.945966in}}{\pgfqpoint{4.714975in}{3.949625in}}{\pgfqpoint{4.721486in}{3.956136in}}%
\pgfpathcurveto{\pgfqpoint{4.727998in}{3.962648in}}{\pgfqpoint{4.731656in}{3.971480in}}{\pgfqpoint{4.731656in}{3.980689in}}%
\pgfpathcurveto{\pgfqpoint{4.731656in}{3.989897in}}{\pgfqpoint{4.727998in}{3.998729in}}{\pgfqpoint{4.721486in}{4.005241in}}%
\pgfpathcurveto{\pgfqpoint{4.714975in}{4.011752in}}{\pgfqpoint{4.706142in}{4.015411in}}{\pgfqpoint{4.696934in}{4.015411in}}%
\pgfpathcurveto{\pgfqpoint{4.687725in}{4.015411in}}{\pgfqpoint{4.678893in}{4.011752in}}{\pgfqpoint{4.672382in}{4.005241in}}%
\pgfpathcurveto{\pgfqpoint{4.665870in}{3.998729in}}{\pgfqpoint{4.662212in}{3.989897in}}{\pgfqpoint{4.662212in}{3.980689in}}%
\pgfpathcurveto{\pgfqpoint{4.662212in}{3.971480in}}{\pgfqpoint{4.665870in}{3.962648in}}{\pgfqpoint{4.672382in}{3.956136in}}%
\pgfpathcurveto{\pgfqpoint{4.678893in}{3.949625in}}{\pgfqpoint{4.687725in}{3.945966in}}{\pgfqpoint{4.696934in}{3.945966in}}%
\pgfpathlineto{\pgfqpoint{4.696934in}{3.945966in}}%
\pgfpathclose%
\pgfusepath{stroke,fill}%
\end{pgfscope}%
\begin{pgfscope}%
\pgfpathrectangle{\pgfqpoint{1.374500in}{0.082500in}}{\pgfqpoint{2.419000in}{2.419000in}}%
\pgfusepath{clip}%
\pgfsetbuttcap%
\pgfsetroundjoin%
\definecolor{currentfill}{rgb}{0.741176,0.121569,0.003922}%
\pgfsetfillcolor{currentfill}%
\pgfsetfillopacity{0.318465}%
\pgfsetlinewidth{1.003750pt}%
\definecolor{currentstroke}{rgb}{0.741176,0.121569,0.003922}%
\pgfsetstrokecolor{currentstroke}%
\pgfsetstrokeopacity{0.318465}%
\pgfsetdash{}{0pt}%
\pgfpathmoveto{\pgfqpoint{3.396968in}{3.888784in}}%
\pgfpathcurveto{\pgfqpoint{3.406176in}{3.888784in}}{\pgfqpoint{3.415009in}{3.892443in}}{\pgfqpoint{3.421520in}{3.898954in}}%
\pgfpathcurveto{\pgfqpoint{3.428031in}{3.905466in}}{\pgfqpoint{3.431690in}{3.914298in}}{\pgfqpoint{3.431690in}{3.923507in}}%
\pgfpathcurveto{\pgfqpoint{3.431690in}{3.932715in}}{\pgfqpoint{3.428031in}{3.941547in}}{\pgfqpoint{3.421520in}{3.948059in}}%
\pgfpathcurveto{\pgfqpoint{3.415009in}{3.954570in}}{\pgfqpoint{3.406176in}{3.958229in}}{\pgfqpoint{3.396968in}{3.958229in}}%
\pgfpathcurveto{\pgfqpoint{3.387759in}{3.958229in}}{\pgfqpoint{3.378927in}{3.954570in}}{\pgfqpoint{3.372415in}{3.948059in}}%
\pgfpathcurveto{\pgfqpoint{3.365904in}{3.941547in}}{\pgfqpoint{3.362246in}{3.932715in}}{\pgfqpoint{3.362246in}{3.923507in}}%
\pgfpathcurveto{\pgfqpoint{3.362246in}{3.914298in}}{\pgfqpoint{3.365904in}{3.905466in}}{\pgfqpoint{3.372415in}{3.898954in}}%
\pgfpathcurveto{\pgfqpoint{3.378927in}{3.892443in}}{\pgfqpoint{3.387759in}{3.888784in}}{\pgfqpoint{3.396968in}{3.888784in}}%
\pgfpathlineto{\pgfqpoint{3.396968in}{3.888784in}}%
\pgfpathclose%
\pgfusepath{stroke,fill}%
\end{pgfscope}%
\begin{pgfscope}%
\pgfpathrectangle{\pgfqpoint{1.374500in}{0.082500in}}{\pgfqpoint{2.419000in}{2.419000in}}%
\pgfusepath{clip}%
\pgfsetbuttcap%
\pgfsetroundjoin%
\definecolor{currentfill}{rgb}{0.741176,0.121569,0.003922}%
\pgfsetfillcolor{currentfill}%
\pgfsetfillopacity{0.321353}%
\pgfsetlinewidth{1.003750pt}%
\definecolor{currentstroke}{rgb}{0.741176,0.121569,0.003922}%
\pgfsetstrokecolor{currentstroke}%
\pgfsetstrokeopacity{0.321353}%
\pgfsetdash{}{0pt}%
\pgfpathmoveto{\pgfqpoint{5.364559in}{3.830250in}}%
\pgfpathcurveto{\pgfqpoint{5.373767in}{3.830250in}}{\pgfqpoint{5.382600in}{3.833909in}}{\pgfqpoint{5.389111in}{3.840420in}}%
\pgfpathcurveto{\pgfqpoint{5.395623in}{3.846931in}}{\pgfqpoint{5.399281in}{3.855764in}}{\pgfqpoint{5.399281in}{3.864972in}}%
\pgfpathcurveto{\pgfqpoint{5.399281in}{3.874181in}}{\pgfqpoint{5.395623in}{3.883013in}}{\pgfqpoint{5.389111in}{3.889525in}}%
\pgfpathcurveto{\pgfqpoint{5.382600in}{3.896036in}}{\pgfqpoint{5.373767in}{3.899695in}}{\pgfqpoint{5.364559in}{3.899695in}}%
\pgfpathcurveto{\pgfqpoint{5.355350in}{3.899695in}}{\pgfqpoint{5.346518in}{3.896036in}}{\pgfqpoint{5.340007in}{3.889525in}}%
\pgfpathcurveto{\pgfqpoint{5.333495in}{3.883013in}}{\pgfqpoint{5.329837in}{3.874181in}}{\pgfqpoint{5.329837in}{3.864972in}}%
\pgfpathcurveto{\pgfqpoint{5.329837in}{3.855764in}}{\pgfqpoint{5.333495in}{3.846931in}}{\pgfqpoint{5.340007in}{3.840420in}}%
\pgfpathcurveto{\pgfqpoint{5.346518in}{3.833909in}}{\pgfqpoint{5.355350in}{3.830250in}}{\pgfqpoint{5.364559in}{3.830250in}}%
\pgfpathlineto{\pgfqpoint{5.364559in}{3.830250in}}%
\pgfpathclose%
\pgfusepath{stroke,fill}%
\end{pgfscope}%
\begin{pgfscope}%
\pgfpathrectangle{\pgfqpoint{1.374500in}{0.082500in}}{\pgfqpoint{2.419000in}{2.419000in}}%
\pgfusepath{clip}%
\pgfsetbuttcap%
\pgfsetroundjoin%
\definecolor{currentfill}{rgb}{0.741176,0.121569,0.003922}%
\pgfsetfillcolor{currentfill}%
\pgfsetfillopacity{0.321353}%
\pgfsetlinewidth{1.003750pt}%
\definecolor{currentstroke}{rgb}{0.741176,0.121569,0.003922}%
\pgfsetstrokecolor{currentstroke}%
\pgfsetstrokeopacity{0.321353}%
\pgfsetdash{}{0pt}%
\pgfpathmoveto{\pgfqpoint{2.066260in}{3.830250in}}%
\pgfpathcurveto{\pgfqpoint{2.075468in}{3.830250in}}{\pgfqpoint{2.084301in}{3.833909in}}{\pgfqpoint{2.090812in}{3.840420in}}%
\pgfpathcurveto{\pgfqpoint{2.097324in}{3.846931in}}{\pgfqpoint{2.100982in}{3.855764in}}{\pgfqpoint{2.100982in}{3.864972in}}%
\pgfpathcurveto{\pgfqpoint{2.100982in}{3.874181in}}{\pgfqpoint{2.097324in}{3.883013in}}{\pgfqpoint{2.090812in}{3.889525in}}%
\pgfpathcurveto{\pgfqpoint{2.084301in}{3.896036in}}{\pgfqpoint{2.075468in}{3.899695in}}{\pgfqpoint{2.066260in}{3.899695in}}%
\pgfpathcurveto{\pgfqpoint{2.057051in}{3.899695in}}{\pgfqpoint{2.048219in}{3.896036in}}{\pgfqpoint{2.041708in}{3.889525in}}%
\pgfpathcurveto{\pgfqpoint{2.035196in}{3.883013in}}{\pgfqpoint{2.031538in}{3.874181in}}{\pgfqpoint{2.031538in}{3.864972in}}%
\pgfpathcurveto{\pgfqpoint{2.031538in}{3.855764in}}{\pgfqpoint{2.035196in}{3.846931in}}{\pgfqpoint{2.041708in}{3.840420in}}%
\pgfpathcurveto{\pgfqpoint{2.048219in}{3.833909in}}{\pgfqpoint{2.057051in}{3.830250in}}{\pgfqpoint{2.066260in}{3.830250in}}%
\pgfpathlineto{\pgfqpoint{2.066260in}{3.830250in}}%
\pgfpathclose%
\pgfusepath{stroke,fill}%
\end{pgfscope}%
\begin{pgfscope}%
\pgfpathrectangle{\pgfqpoint{1.374500in}{0.082500in}}{\pgfqpoint{2.419000in}{2.419000in}}%
\pgfusepath{clip}%
\pgfsetbuttcap%
\pgfsetroundjoin%
\definecolor{currentfill}{rgb}{0.741176,0.121569,0.003922}%
\pgfsetfillcolor{currentfill}%
\pgfsetfillopacity{0.324309}%
\pgfsetlinewidth{1.003750pt}%
\definecolor{currentstroke}{rgb}{0.741176,0.121569,0.003922}%
\pgfsetstrokecolor{currentstroke}%
\pgfsetstrokeopacity{0.324309}%
\pgfsetdash{}{0pt}%
\pgfpathmoveto{\pgfqpoint{4.041472in}{3.770315in}}%
\pgfpathcurveto{\pgfqpoint{4.050680in}{3.770315in}}{\pgfqpoint{4.059513in}{3.773974in}}{\pgfqpoint{4.066024in}{3.780485in}}%
\pgfpathcurveto{\pgfqpoint{4.072535in}{3.786996in}}{\pgfqpoint{4.076194in}{3.795829in}}{\pgfqpoint{4.076194in}{3.805037in}}%
\pgfpathcurveto{\pgfqpoint{4.076194in}{3.814246in}}{\pgfqpoint{4.072535in}{3.823078in}}{\pgfqpoint{4.066024in}{3.829590in}}%
\pgfpathcurveto{\pgfqpoint{4.059513in}{3.836101in}}{\pgfqpoint{4.050680in}{3.839760in}}{\pgfqpoint{4.041472in}{3.839760in}}%
\pgfpathcurveto{\pgfqpoint{4.032263in}{3.839760in}}{\pgfqpoint{4.023431in}{3.836101in}}{\pgfqpoint{4.016919in}{3.829590in}}%
\pgfpathcurveto{\pgfqpoint{4.010408in}{3.823078in}}{\pgfqpoint{4.006749in}{3.814246in}}{\pgfqpoint{4.006749in}{3.805037in}}%
\pgfpathcurveto{\pgfqpoint{4.006749in}{3.795829in}}{\pgfqpoint{4.010408in}{3.786996in}}{\pgfqpoint{4.016919in}{3.780485in}}%
\pgfpathcurveto{\pgfqpoint{4.023431in}{3.773974in}}{\pgfqpoint{4.032263in}{3.770315in}}{\pgfqpoint{4.041472in}{3.770315in}}%
\pgfpathlineto{\pgfqpoint{4.041472in}{3.770315in}}%
\pgfpathclose%
\pgfusepath{stroke,fill}%
\end{pgfscope}%
\begin{pgfscope}%
\pgfpathrectangle{\pgfqpoint{1.374500in}{0.082500in}}{\pgfqpoint{2.419000in}{2.419000in}}%
\pgfusepath{clip}%
\pgfsetbuttcap%
\pgfsetroundjoin%
\definecolor{currentfill}{rgb}{0.741176,0.121569,0.003922}%
\pgfsetfillcolor{currentfill}%
\pgfsetfillopacity{0.327337}%
\pgfsetlinewidth{1.003750pt}%
\definecolor{currentstroke}{rgb}{0.741176,0.121569,0.003922}%
\pgfsetstrokecolor{currentstroke}%
\pgfsetstrokeopacity{0.327337}%
\pgfsetdash{}{0pt}%
\pgfpathmoveto{\pgfqpoint{2.686338in}{3.708928in}}%
\pgfpathcurveto{\pgfqpoint{2.695546in}{3.708928in}}{\pgfqpoint{2.704379in}{3.712587in}}{\pgfqpoint{2.710890in}{3.719098in}}%
\pgfpathcurveto{\pgfqpoint{2.717402in}{3.725610in}}{\pgfqpoint{2.721060in}{3.734442in}}{\pgfqpoint{2.721060in}{3.743651in}}%
\pgfpathcurveto{\pgfqpoint{2.721060in}{3.752859in}}{\pgfqpoint{2.717402in}{3.761692in}}{\pgfqpoint{2.710890in}{3.768203in}}%
\pgfpathcurveto{\pgfqpoint{2.704379in}{3.774714in}}{\pgfqpoint{2.695546in}{3.778373in}}{\pgfqpoint{2.686338in}{3.778373in}}%
\pgfpathcurveto{\pgfqpoint{2.677130in}{3.778373in}}{\pgfqpoint{2.668297in}{3.774714in}}{\pgfqpoint{2.661786in}{3.768203in}}%
\pgfpathcurveto{\pgfqpoint{2.655274in}{3.761692in}}{\pgfqpoint{2.651616in}{3.752859in}}{\pgfqpoint{2.651616in}{3.743651in}}%
\pgfpathcurveto{\pgfqpoint{2.651616in}{3.734442in}}{\pgfqpoint{2.655274in}{3.725610in}}{\pgfqpoint{2.661786in}{3.719098in}}%
\pgfpathcurveto{\pgfqpoint{2.668297in}{3.712587in}}{\pgfqpoint{2.677130in}{3.708928in}}{\pgfqpoint{2.686338in}{3.708928in}}%
\pgfpathlineto{\pgfqpoint{2.686338in}{3.708928in}}%
\pgfpathclose%
\pgfusepath{stroke,fill}%
\end{pgfscope}%
\begin{pgfscope}%
\pgfpathrectangle{\pgfqpoint{1.374500in}{0.082500in}}{\pgfqpoint{2.419000in}{2.419000in}}%
\pgfusepath{clip}%
\pgfsetbuttcap%
\pgfsetroundjoin%
\definecolor{currentfill}{rgb}{0.741176,0.121569,0.003922}%
\pgfsetfillcolor{currentfill}%
\pgfsetfillopacity{0.327337}%
\pgfsetlinewidth{1.003750pt}%
\definecolor{currentstroke}{rgb}{0.741176,0.121569,0.003922}%
\pgfsetstrokecolor{currentstroke}%
\pgfsetstrokeopacity{0.327337}%
\pgfsetdash{}{0pt}%
\pgfpathmoveto{\pgfqpoint{6.064525in}{3.708928in}}%
\pgfpathcurveto{\pgfqpoint{6.073733in}{3.708928in}}{\pgfqpoint{6.082566in}{3.712587in}}{\pgfqpoint{6.089077in}{3.719098in}}%
\pgfpathcurveto{\pgfqpoint{6.095588in}{3.725610in}}{\pgfqpoint{6.099247in}{3.734442in}}{\pgfqpoint{6.099247in}{3.743651in}}%
\pgfpathcurveto{\pgfqpoint{6.099247in}{3.752859in}}{\pgfqpoint{6.095588in}{3.761692in}}{\pgfqpoint{6.089077in}{3.768203in}}%
\pgfpathcurveto{\pgfqpoint{6.082566in}{3.774714in}}{\pgfqpoint{6.073733in}{3.778373in}}{\pgfqpoint{6.064525in}{3.778373in}}%
\pgfpathcurveto{\pgfqpoint{6.055316in}{3.778373in}}{\pgfqpoint{6.046484in}{3.774714in}}{\pgfqpoint{6.039972in}{3.768203in}}%
\pgfpathcurveto{\pgfqpoint{6.033461in}{3.761692in}}{\pgfqpoint{6.029803in}{3.752859in}}{\pgfqpoint{6.029803in}{3.743651in}}%
\pgfpathcurveto{\pgfqpoint{6.029803in}{3.734442in}}{\pgfqpoint{6.033461in}{3.725610in}}{\pgfqpoint{6.039972in}{3.719098in}}%
\pgfpathcurveto{\pgfqpoint{6.046484in}{3.712587in}}{\pgfqpoint{6.055316in}{3.708928in}}{\pgfqpoint{6.064525in}{3.708928in}}%
\pgfpathlineto{\pgfqpoint{6.064525in}{3.708928in}}%
\pgfpathclose%
\pgfusepath{stroke,fill}%
\end{pgfscope}%
\begin{pgfscope}%
\pgfpathrectangle{\pgfqpoint{1.374500in}{0.082500in}}{\pgfqpoint{2.419000in}{2.419000in}}%
\pgfusepath{clip}%
\pgfsetbuttcap%
\pgfsetroundjoin%
\definecolor{currentfill}{rgb}{0.741176,0.121569,0.003922}%
\pgfsetfillcolor{currentfill}%
\pgfsetfillopacity{0.330440}%
\pgfsetlinewidth{1.003750pt}%
\definecolor{currentstroke}{rgb}{0.741176,0.121569,0.003922}%
\pgfsetstrokecolor{currentstroke}%
\pgfsetstrokeopacity{0.330440}%
\pgfsetdash{}{0pt}%
\pgfpathmoveto{\pgfqpoint{4.717579in}{3.646037in}}%
\pgfpathcurveto{\pgfqpoint{4.726788in}{3.646037in}}{\pgfqpoint{4.735620in}{3.649695in}}{\pgfqpoint{4.742131in}{3.656207in}}%
\pgfpathcurveto{\pgfqpoint{4.748643in}{3.662718in}}{\pgfqpoint{4.752301in}{3.671550in}}{\pgfqpoint{4.752301in}{3.680759in}}%
\pgfpathcurveto{\pgfqpoint{4.752301in}{3.689967in}}{\pgfqpoint{4.748643in}{3.698800in}}{\pgfqpoint{4.742131in}{3.705311in}}%
\pgfpathcurveto{\pgfqpoint{4.735620in}{3.711822in}}{\pgfqpoint{4.726788in}{3.715481in}}{\pgfqpoint{4.717579in}{3.715481in}}%
\pgfpathcurveto{\pgfqpoint{4.708371in}{3.715481in}}{\pgfqpoint{4.699538in}{3.711822in}}{\pgfqpoint{4.693027in}{3.705311in}}%
\pgfpathcurveto{\pgfqpoint{4.686516in}{3.698800in}}{\pgfqpoint{4.682857in}{3.689967in}}{\pgfqpoint{4.682857in}{3.680759in}}%
\pgfpathcurveto{\pgfqpoint{4.682857in}{3.671550in}}{\pgfqpoint{4.686516in}{3.662718in}}{\pgfqpoint{4.693027in}{3.656207in}}%
\pgfpathcurveto{\pgfqpoint{4.699538in}{3.649695in}}{\pgfqpoint{4.708371in}{3.646037in}}{\pgfqpoint{4.717579in}{3.646037in}}%
\pgfpathlineto{\pgfqpoint{4.717579in}{3.646037in}}%
\pgfpathclose%
\pgfusepath{stroke,fill}%
\end{pgfscope}%
\begin{pgfscope}%
\pgfpathrectangle{\pgfqpoint{1.374500in}{0.082500in}}{\pgfqpoint{2.419000in}{2.419000in}}%
\pgfusepath{clip}%
\pgfsetbuttcap%
\pgfsetroundjoin%
\definecolor{currentfill}{rgb}{0.741176,0.121569,0.003922}%
\pgfsetfillcolor{currentfill}%
\pgfsetfillopacity{0.333619}%
\pgfsetlinewidth{1.003750pt}%
\definecolor{currentstroke}{rgb}{0.741176,0.121569,0.003922}%
\pgfsetstrokecolor{currentstroke}%
\pgfsetstrokeopacity{0.333619}%
\pgfsetdash{}{0pt}%
\pgfpathmoveto{\pgfqpoint{3.337199in}{3.581584in}}%
\pgfpathcurveto{\pgfqpoint{3.346408in}{3.581584in}}{\pgfqpoint{3.355240in}{3.585242in}}{\pgfqpoint{3.361752in}{3.591754in}}%
\pgfpathcurveto{\pgfqpoint{3.368263in}{3.598265in}}{\pgfqpoint{3.371922in}{3.607098in}}{\pgfqpoint{3.371922in}{3.616306in}}%
\pgfpathcurveto{\pgfqpoint{3.371922in}{3.625514in}}{\pgfqpoint{3.368263in}{3.634347in}}{\pgfqpoint{3.361752in}{3.640858in}}%
\pgfpathcurveto{\pgfqpoint{3.355240in}{3.647370in}}{\pgfqpoint{3.346408in}{3.651028in}}{\pgfqpoint{3.337199in}{3.651028in}}%
\pgfpathcurveto{\pgfqpoint{3.327991in}{3.651028in}}{\pgfqpoint{3.319158in}{3.647370in}}{\pgfqpoint{3.312647in}{3.640858in}}%
\pgfpathcurveto{\pgfqpoint{3.306136in}{3.634347in}}{\pgfqpoint{3.302477in}{3.625514in}}{\pgfqpoint{3.302477in}{3.616306in}}%
\pgfpathcurveto{\pgfqpoint{3.302477in}{3.607098in}}{\pgfqpoint{3.306136in}{3.598265in}}{\pgfqpoint{3.312647in}{3.591754in}}%
\pgfpathcurveto{\pgfqpoint{3.319158in}{3.585242in}}{\pgfqpoint{3.327991in}{3.581584in}}{\pgfqpoint{3.337199in}{3.581584in}}%
\pgfpathlineto{\pgfqpoint{3.337199in}{3.581584in}}%
\pgfpathclose%
\pgfusepath{stroke,fill}%
\end{pgfscope}%
\begin{pgfscope}%
\pgfpathrectangle{\pgfqpoint{1.374500in}{0.082500in}}{\pgfqpoint{2.419000in}{2.419000in}}%
\pgfusepath{clip}%
\pgfsetbuttcap%
\pgfsetroundjoin%
\definecolor{currentfill}{rgb}{0.741176,0.121569,0.003922}%
\pgfsetfillcolor{currentfill}%
\pgfsetfillopacity{0.333619}%
\pgfsetlinewidth{1.003750pt}%
\definecolor{currentstroke}{rgb}{0.741176,0.121569,0.003922}%
\pgfsetstrokecolor{currentstroke}%
\pgfsetstrokeopacity{0.333619}%
\pgfsetdash{}{0pt}%
\pgfpathmoveto{\pgfqpoint{6.799240in}{3.581584in}}%
\pgfpathcurveto{\pgfqpoint{6.808449in}{3.581584in}}{\pgfqpoint{6.817281in}{3.585242in}}{\pgfqpoint{6.823792in}{3.591754in}}%
\pgfpathcurveto{\pgfqpoint{6.830304in}{3.598265in}}{\pgfqpoint{6.833962in}{3.607098in}}{\pgfqpoint{6.833962in}{3.616306in}}%
\pgfpathcurveto{\pgfqpoint{6.833962in}{3.625514in}}{\pgfqpoint{6.830304in}{3.634347in}}{\pgfqpoint{6.823792in}{3.640858in}}%
\pgfpathcurveto{\pgfqpoint{6.817281in}{3.647370in}}{\pgfqpoint{6.808449in}{3.651028in}}{\pgfqpoint{6.799240in}{3.651028in}}%
\pgfpathcurveto{\pgfqpoint{6.790032in}{3.651028in}}{\pgfqpoint{6.781199in}{3.647370in}}{\pgfqpoint{6.774688in}{3.640858in}}%
\pgfpathcurveto{\pgfqpoint{6.768176in}{3.634347in}}{\pgfqpoint{6.764518in}{3.625514in}}{\pgfqpoint{6.764518in}{3.616306in}}%
\pgfpathcurveto{\pgfqpoint{6.764518in}{3.607098in}}{\pgfqpoint{6.768176in}{3.598265in}}{\pgfqpoint{6.774688in}{3.591754in}}%
\pgfpathcurveto{\pgfqpoint{6.781199in}{3.585242in}}{\pgfqpoint{6.790032in}{3.581584in}}{\pgfqpoint{6.799240in}{3.581584in}}%
\pgfpathlineto{\pgfqpoint{6.799240in}{3.581584in}}%
\pgfpathclose%
\pgfusepath{stroke,fill}%
\end{pgfscope}%
\begin{pgfscope}%
\pgfpathrectangle{\pgfqpoint{1.374500in}{0.082500in}}{\pgfqpoint{2.419000in}{2.419000in}}%
\pgfusepath{clip}%
\pgfsetbuttcap%
\pgfsetroundjoin%
\definecolor{currentfill}{rgb}{0.741176,0.121569,0.003922}%
\pgfsetfillcolor{currentfill}%
\pgfsetfillopacity{0.336878}%
\pgfsetlinewidth{1.003750pt}%
\definecolor{currentstroke}{rgb}{0.741176,0.121569,0.003922}%
\pgfsetstrokecolor{currentstroke}%
\pgfsetstrokeopacity{0.336878}%
\pgfsetdash{}{0pt}%
\pgfpathmoveto{\pgfqpoint{1.922125in}{3.515511in}}%
\pgfpathcurveto{\pgfqpoint{1.931334in}{3.515511in}}{\pgfqpoint{1.940166in}{3.519169in}}{\pgfqpoint{1.946677in}{3.525681in}}%
\pgfpathcurveto{\pgfqpoint{1.953189in}{3.532192in}}{\pgfqpoint{1.956847in}{3.541025in}}{\pgfqpoint{1.956847in}{3.550233in}}%
\pgfpathcurveto{\pgfqpoint{1.956847in}{3.559442in}}{\pgfqpoint{1.953189in}{3.568274in}}{\pgfqpoint{1.946677in}{3.574785in}}%
\pgfpathcurveto{\pgfqpoint{1.940166in}{3.581297in}}{\pgfqpoint{1.931334in}{3.584955in}}{\pgfqpoint{1.922125in}{3.584955in}}%
\pgfpathcurveto{\pgfqpoint{1.912917in}{3.584955in}}{\pgfqpoint{1.904084in}{3.581297in}}{\pgfqpoint{1.897573in}{3.574785in}}%
\pgfpathcurveto{\pgfqpoint{1.891061in}{3.568274in}}{\pgfqpoint{1.887403in}{3.559442in}}{\pgfqpoint{1.887403in}{3.550233in}}%
\pgfpathcurveto{\pgfqpoint{1.887403in}{3.541025in}}{\pgfqpoint{1.891061in}{3.532192in}}{\pgfqpoint{1.897573in}{3.525681in}}%
\pgfpathcurveto{\pgfqpoint{1.904084in}{3.519169in}}{\pgfqpoint{1.912917in}{3.515511in}}{\pgfqpoint{1.922125in}{3.515511in}}%
\pgfpathlineto{\pgfqpoint{1.922125in}{3.515511in}}%
\pgfpathclose%
\pgfusepath{stroke,fill}%
\end{pgfscope}%
\begin{pgfscope}%
\pgfpathrectangle{\pgfqpoint{1.374500in}{0.082500in}}{\pgfqpoint{2.419000in}{2.419000in}}%
\pgfusepath{clip}%
\pgfsetbuttcap%
\pgfsetroundjoin%
\definecolor{currentfill}{rgb}{0.741176,0.121569,0.003922}%
\pgfsetfillcolor{currentfill}%
\pgfsetfillopacity{0.336878}%
\pgfsetlinewidth{1.003750pt}%
\definecolor{currentstroke}{rgb}{0.741176,0.121569,0.003922}%
\pgfsetstrokecolor{currentstroke}%
\pgfsetstrokeopacity{0.336878}%
\pgfsetdash{}{0pt}%
\pgfpathmoveto{\pgfqpoint{5.427673in}{3.515511in}}%
\pgfpathcurveto{\pgfqpoint{5.436882in}{3.515511in}}{\pgfqpoint{5.445714in}{3.519169in}}{\pgfqpoint{5.452226in}{3.525681in}}%
\pgfpathcurveto{\pgfqpoint{5.458737in}{3.532192in}}{\pgfqpoint{5.462396in}{3.541025in}}{\pgfqpoint{5.462396in}{3.550233in}}%
\pgfpathcurveto{\pgfqpoint{5.462396in}{3.559442in}}{\pgfqpoint{5.458737in}{3.568274in}}{\pgfqpoint{5.452226in}{3.574785in}}%
\pgfpathcurveto{\pgfqpoint{5.445714in}{3.581297in}}{\pgfqpoint{5.436882in}{3.584955in}}{\pgfqpoint{5.427673in}{3.584955in}}%
\pgfpathcurveto{\pgfqpoint{5.418465in}{3.584955in}}{\pgfqpoint{5.409632in}{3.581297in}}{\pgfqpoint{5.403121in}{3.574785in}}%
\pgfpathcurveto{\pgfqpoint{5.396610in}{3.568274in}}{\pgfqpoint{5.392951in}{3.559442in}}{\pgfqpoint{5.392951in}{3.550233in}}%
\pgfpathcurveto{\pgfqpoint{5.392951in}{3.541025in}}{\pgfqpoint{5.396610in}{3.532192in}}{\pgfqpoint{5.403121in}{3.525681in}}%
\pgfpathcurveto{\pgfqpoint{5.409632in}{3.519169in}}{\pgfqpoint{5.418465in}{3.515511in}}{\pgfqpoint{5.427673in}{3.515511in}}%
\pgfpathlineto{\pgfqpoint{5.427673in}{3.515511in}}%
\pgfpathclose%
\pgfusepath{stroke,fill}%
\end{pgfscope}%
\begin{pgfscope}%
\pgfpathrectangle{\pgfqpoint{1.374500in}{0.082500in}}{\pgfqpoint{2.419000in}{2.419000in}}%
\pgfusepath{clip}%
\pgfsetbuttcap%
\pgfsetroundjoin%
\definecolor{currentfill}{rgb}{0.741176,0.121569,0.003922}%
\pgfsetfillcolor{currentfill}%
\pgfsetfillopacity{0.340220}%
\pgfsetlinewidth{1.003750pt}%
\definecolor{currentstroke}{rgb}{0.741176,0.121569,0.003922}%
\pgfsetstrokecolor{currentstroke}%
\pgfsetstrokeopacity{0.340220}%
\pgfsetdash{}{0pt}%
\pgfpathmoveto{\pgfqpoint{4.021195in}{3.447756in}}%
\pgfpathcurveto{\pgfqpoint{4.030403in}{3.447756in}}{\pgfqpoint{4.039236in}{3.451415in}}{\pgfqpoint{4.045747in}{3.457926in}}%
\pgfpathcurveto{\pgfqpoint{4.052259in}{3.464437in}}{\pgfqpoint{4.055917in}{3.473270in}}{\pgfqpoint{4.055917in}{3.482478in}}%
\pgfpathcurveto{\pgfqpoint{4.055917in}{3.491687in}}{\pgfqpoint{4.052259in}{3.500519in}}{\pgfqpoint{4.045747in}{3.507031in}}%
\pgfpathcurveto{\pgfqpoint{4.039236in}{3.513542in}}{\pgfqpoint{4.030403in}{3.517201in}}{\pgfqpoint{4.021195in}{3.517201in}}%
\pgfpathcurveto{\pgfqpoint{4.011986in}{3.517201in}}{\pgfqpoint{4.003154in}{3.513542in}}{\pgfqpoint{3.996643in}{3.507031in}}%
\pgfpathcurveto{\pgfqpoint{3.990131in}{3.500519in}}{\pgfqpoint{3.986473in}{3.491687in}}{\pgfqpoint{3.986473in}{3.482478in}}%
\pgfpathcurveto{\pgfqpoint{3.986473in}{3.473270in}}{\pgfqpoint{3.990131in}{3.464437in}}{\pgfqpoint{3.996643in}{3.457926in}}%
\pgfpathcurveto{\pgfqpoint{4.003154in}{3.451415in}}{\pgfqpoint{4.011986in}{3.447756in}}{\pgfqpoint{4.021195in}{3.447756in}}%
\pgfpathlineto{\pgfqpoint{4.021195in}{3.447756in}}%
\pgfpathclose%
\pgfusepath{stroke,fill}%
\end{pgfscope}%
\begin{pgfscope}%
\pgfpathrectangle{\pgfqpoint{1.374500in}{0.082500in}}{\pgfqpoint{2.419000in}{2.419000in}}%
\pgfusepath{clip}%
\pgfsetbuttcap%
\pgfsetroundjoin%
\definecolor{currentfill}{rgb}{0.741176,0.121569,0.003922}%
\pgfsetfillcolor{currentfill}%
\pgfsetfillopacity{0.340220}%
\pgfsetlinewidth{1.003750pt}%
\definecolor{currentstroke}{rgb}{0.741176,0.121569,0.003922}%
\pgfsetstrokecolor{currentstroke}%
\pgfsetstrokeopacity{0.340220}%
\pgfsetdash{}{0pt}%
\pgfpathmoveto{\pgfqpoint{7.571358in}{3.447756in}}%
\pgfpathcurveto{\pgfqpoint{7.580567in}{3.447756in}}{\pgfqpoint{7.589399in}{3.451415in}}{\pgfqpoint{7.595911in}{3.457926in}}%
\pgfpathcurveto{\pgfqpoint{7.602422in}{3.464437in}}{\pgfqpoint{7.606080in}{3.473270in}}{\pgfqpoint{7.606080in}{3.482478in}}%
\pgfpathcurveto{\pgfqpoint{7.606080in}{3.491687in}}{\pgfqpoint{7.602422in}{3.500519in}}{\pgfqpoint{7.595911in}{3.507031in}}%
\pgfpathcurveto{\pgfqpoint{7.589399in}{3.513542in}}{\pgfqpoint{7.580567in}{3.517201in}}{\pgfqpoint{7.571358in}{3.517201in}}%
\pgfpathcurveto{\pgfqpoint{7.562150in}{3.517201in}}{\pgfqpoint{7.553317in}{3.513542in}}{\pgfqpoint{7.546806in}{3.507031in}}%
\pgfpathcurveto{\pgfqpoint{7.540295in}{3.500519in}}{\pgfqpoint{7.536636in}{3.491687in}}{\pgfqpoint{7.536636in}{3.482478in}}%
\pgfpathcurveto{\pgfqpoint{7.536636in}{3.473270in}}{\pgfqpoint{7.540295in}{3.464437in}}{\pgfqpoint{7.546806in}{3.457926in}}%
\pgfpathcurveto{\pgfqpoint{7.553317in}{3.451415in}}{\pgfqpoint{7.562150in}{3.447756in}}{\pgfqpoint{7.571358in}{3.447756in}}%
\pgfpathlineto{\pgfqpoint{7.571358in}{3.447756in}}%
\pgfpathclose%
\pgfusepath{stroke,fill}%
\end{pgfscope}%
\begin{pgfscope}%
\pgfpathrectangle{\pgfqpoint{1.374500in}{0.082500in}}{\pgfqpoint{2.419000in}{2.419000in}}%
\pgfusepath{clip}%
\pgfsetbuttcap%
\pgfsetroundjoin%
\definecolor{currentfill}{rgb}{0.741176,0.121569,0.003922}%
\pgfsetfillcolor{currentfill}%
\pgfsetfillopacity{0.343649}%
\pgfsetlinewidth{1.003750pt}%
\definecolor{currentstroke}{rgb}{0.741176,0.121569,0.003922}%
\pgfsetstrokecolor{currentstroke}%
\pgfsetstrokeopacity{0.343649}%
\pgfsetdash{}{0pt}%
\pgfpathmoveto{\pgfqpoint{6.174383in}{3.378255in}}%
\pgfpathcurveto{\pgfqpoint{6.183592in}{3.378255in}}{\pgfqpoint{6.192424in}{3.381913in}}{\pgfqpoint{6.198935in}{3.388425in}}%
\pgfpathcurveto{\pgfqpoint{6.205447in}{3.394936in}}{\pgfqpoint{6.209105in}{3.403769in}}{\pgfqpoint{6.209105in}{3.412977in}}%
\pgfpathcurveto{\pgfqpoint{6.209105in}{3.422185in}}{\pgfqpoint{6.205447in}{3.431018in}}{\pgfqpoint{6.198935in}{3.437529in}}%
\pgfpathcurveto{\pgfqpoint{6.192424in}{3.444041in}}{\pgfqpoint{6.183592in}{3.447699in}}{\pgfqpoint{6.174383in}{3.447699in}}%
\pgfpathcurveto{\pgfqpoint{6.165175in}{3.447699in}}{\pgfqpoint{6.156342in}{3.444041in}}{\pgfqpoint{6.149831in}{3.437529in}}%
\pgfpathcurveto{\pgfqpoint{6.143319in}{3.431018in}}{\pgfqpoint{6.139661in}{3.422185in}}{\pgfqpoint{6.139661in}{3.412977in}}%
\pgfpathcurveto{\pgfqpoint{6.139661in}{3.403769in}}{\pgfqpoint{6.143319in}{3.394936in}}{\pgfqpoint{6.149831in}{3.388425in}}%
\pgfpathcurveto{\pgfqpoint{6.156342in}{3.381913in}}{\pgfqpoint{6.165175in}{3.378255in}}{\pgfqpoint{6.174383in}{3.378255in}}%
\pgfpathlineto{\pgfqpoint{6.174383in}{3.378255in}}%
\pgfpathclose%
\pgfusepath{stroke,fill}%
\end{pgfscope}%
\begin{pgfscope}%
\pgfpathrectangle{\pgfqpoint{1.374500in}{0.082500in}}{\pgfqpoint{2.419000in}{2.419000in}}%
\pgfusepath{clip}%
\pgfsetbuttcap%
\pgfsetroundjoin%
\definecolor{currentfill}{rgb}{0.741176,0.121569,0.003922}%
\pgfsetfillcolor{currentfill}%
\pgfsetfillopacity{0.343649}%
\pgfsetlinewidth{1.003750pt}%
\definecolor{currentstroke}{rgb}{0.741176,0.121569,0.003922}%
\pgfsetstrokecolor{currentstroke}%
\pgfsetstrokeopacity{0.343649}%
\pgfsetdash{}{0pt}%
\pgfpathmoveto{\pgfqpoint{2.578454in}{3.378255in}}%
\pgfpathcurveto{\pgfqpoint{2.587663in}{3.378255in}}{\pgfqpoint{2.596495in}{3.381913in}}{\pgfqpoint{2.603007in}{3.388425in}}%
\pgfpathcurveto{\pgfqpoint{2.609518in}{3.394936in}}{\pgfqpoint{2.613177in}{3.403769in}}{\pgfqpoint{2.613177in}{3.412977in}}%
\pgfpathcurveto{\pgfqpoint{2.613177in}{3.422185in}}{\pgfqpoint{2.609518in}{3.431018in}}{\pgfqpoint{2.603007in}{3.437529in}}%
\pgfpathcurveto{\pgfqpoint{2.596495in}{3.444041in}}{\pgfqpoint{2.587663in}{3.447699in}}{\pgfqpoint{2.578454in}{3.447699in}}%
\pgfpathcurveto{\pgfqpoint{2.569246in}{3.447699in}}{\pgfqpoint{2.560413in}{3.444041in}}{\pgfqpoint{2.553902in}{3.437529in}}%
\pgfpathcurveto{\pgfqpoint{2.547391in}{3.431018in}}{\pgfqpoint{2.543732in}{3.422185in}}{\pgfqpoint{2.543732in}{3.412977in}}%
\pgfpathcurveto{\pgfqpoint{2.543732in}{3.403769in}}{\pgfqpoint{2.547391in}{3.394936in}}{\pgfqpoint{2.553902in}{3.388425in}}%
\pgfpathcurveto{\pgfqpoint{2.560413in}{3.381913in}}{\pgfqpoint{2.569246in}{3.378255in}}{\pgfqpoint{2.578454in}{3.378255in}}%
\pgfpathlineto{\pgfqpoint{2.578454in}{3.378255in}}%
\pgfpathclose%
\pgfusepath{stroke,fill}%
\end{pgfscope}%
\begin{pgfscope}%
\pgfpathrectangle{\pgfqpoint{1.374500in}{0.082500in}}{\pgfqpoint{2.419000in}{2.419000in}}%
\pgfusepath{clip}%
\pgfsetbuttcap%
\pgfsetroundjoin%
\definecolor{currentfill}{rgb}{0.741176,0.121569,0.003922}%
\pgfsetfillcolor{currentfill}%
\pgfsetfillopacity{0.347167}%
\pgfsetlinewidth{1.003750pt}%
\definecolor{currentstroke}{rgb}{0.741176,0.121569,0.003922}%
\pgfsetstrokecolor{currentstroke}%
\pgfsetstrokeopacity{0.347167}%
\pgfsetdash{}{0pt}%
\pgfpathmoveto{\pgfqpoint{1.098031in}{3.306938in}}%
\pgfpathcurveto{\pgfqpoint{1.107240in}{3.306938in}}{\pgfqpoint{1.116072in}{3.310596in}}{\pgfqpoint{1.122584in}{3.317108in}}%
\pgfpathcurveto{\pgfqpoint{1.129095in}{3.323619in}}{\pgfqpoint{1.132754in}{3.332452in}}{\pgfqpoint{1.132754in}{3.341660in}}%
\pgfpathcurveto{\pgfqpoint{1.132754in}{3.350869in}}{\pgfqpoint{1.129095in}{3.359701in}}{\pgfqpoint{1.122584in}{3.366212in}}%
\pgfpathcurveto{\pgfqpoint{1.116072in}{3.372724in}}{\pgfqpoint{1.107240in}{3.376382in}}{\pgfqpoint{1.098031in}{3.376382in}}%
\pgfpathcurveto{\pgfqpoint{1.088823in}{3.376382in}}{\pgfqpoint{1.079990in}{3.372724in}}{\pgfqpoint{1.073479in}{3.366212in}}%
\pgfpathcurveto{\pgfqpoint{1.066968in}{3.359701in}}{\pgfqpoint{1.063309in}{3.350869in}}{\pgfqpoint{1.063309in}{3.341660in}}%
\pgfpathcurveto{\pgfqpoint{1.063309in}{3.332452in}}{\pgfqpoint{1.066968in}{3.323619in}}{\pgfqpoint{1.073479in}{3.317108in}}%
\pgfpathcurveto{\pgfqpoint{1.079990in}{3.310596in}}{\pgfqpoint{1.088823in}{3.306938in}}{\pgfqpoint{1.098031in}{3.306938in}}%
\pgfpathlineto{\pgfqpoint{1.098031in}{3.306938in}}%
\pgfpathclose%
\pgfusepath{stroke,fill}%
\end{pgfscope}%
\begin{pgfscope}%
\pgfpathrectangle{\pgfqpoint{1.374500in}{0.082500in}}{\pgfqpoint{2.419000in}{2.419000in}}%
\pgfusepath{clip}%
\pgfsetbuttcap%
\pgfsetroundjoin%
\definecolor{currentfill}{rgb}{0.741176,0.121569,0.003922}%
\pgfsetfillcolor{currentfill}%
\pgfsetfillopacity{0.347167}%
\pgfsetlinewidth{1.003750pt}%
\definecolor{currentstroke}{rgb}{0.741176,0.121569,0.003922}%
\pgfsetstrokecolor{currentstroke}%
\pgfsetstrokeopacity{0.347167}%
\pgfsetdash{}{0pt}%
\pgfpathmoveto{\pgfqpoint{4.740921in}{3.306938in}}%
\pgfpathcurveto{\pgfqpoint{4.750129in}{3.306938in}}{\pgfqpoint{4.758962in}{3.310596in}}{\pgfqpoint{4.765473in}{3.317108in}}%
\pgfpathcurveto{\pgfqpoint{4.771984in}{3.323619in}}{\pgfqpoint{4.775643in}{3.332452in}}{\pgfqpoint{4.775643in}{3.341660in}}%
\pgfpathcurveto{\pgfqpoint{4.775643in}{3.350869in}}{\pgfqpoint{4.771984in}{3.359701in}}{\pgfqpoint{4.765473in}{3.366212in}}%
\pgfpathcurveto{\pgfqpoint{4.758962in}{3.372724in}}{\pgfqpoint{4.750129in}{3.376382in}}{\pgfqpoint{4.740921in}{3.376382in}}%
\pgfpathcurveto{\pgfqpoint{4.731712in}{3.376382in}}{\pgfqpoint{4.722880in}{3.372724in}}{\pgfqpoint{4.716368in}{3.366212in}}%
\pgfpathcurveto{\pgfqpoint{4.709857in}{3.359701in}}{\pgfqpoint{4.706198in}{3.350869in}}{\pgfqpoint{4.706198in}{3.341660in}}%
\pgfpathcurveto{\pgfqpoint{4.706198in}{3.332452in}}{\pgfqpoint{4.709857in}{3.323619in}}{\pgfqpoint{4.716368in}{3.317108in}}%
\pgfpathcurveto{\pgfqpoint{4.722880in}{3.310596in}}{\pgfqpoint{4.731712in}{3.306938in}}{\pgfqpoint{4.740921in}{3.306938in}}%
\pgfpathlineto{\pgfqpoint{4.740921in}{3.306938in}}%
\pgfpathclose%
\pgfusepath{stroke,fill}%
\end{pgfscope}%
\begin{pgfscope}%
\pgfpathrectangle{\pgfqpoint{1.374500in}{0.082500in}}{\pgfqpoint{2.419000in}{2.419000in}}%
\pgfusepath{clip}%
\pgfsetbuttcap%
\pgfsetroundjoin%
\definecolor{currentfill}{rgb}{0.741176,0.121569,0.003922}%
\pgfsetfillcolor{currentfill}%
\pgfsetfillopacity{0.347167}%
\pgfsetlinewidth{1.003750pt}%
\definecolor{currentstroke}{rgb}{0.741176,0.121569,0.003922}%
\pgfsetstrokecolor{currentstroke}%
\pgfsetstrokeopacity{0.347167}%
\pgfsetdash{}{0pt}%
\pgfpathmoveto{\pgfqpoint{8.383810in}{3.306938in}}%
\pgfpathcurveto{\pgfqpoint{8.393018in}{3.306938in}}{\pgfqpoint{8.401851in}{3.310596in}}{\pgfqpoint{8.408362in}{3.317108in}}%
\pgfpathcurveto{\pgfqpoint{8.414874in}{3.323619in}}{\pgfqpoint{8.418532in}{3.332452in}}{\pgfqpoint{8.418532in}{3.341660in}}%
\pgfpathcurveto{\pgfqpoint{8.418532in}{3.350869in}}{\pgfqpoint{8.414874in}{3.359701in}}{\pgfqpoint{8.408362in}{3.366212in}}%
\pgfpathcurveto{\pgfqpoint{8.401851in}{3.372724in}}{\pgfqpoint{8.393018in}{3.376382in}}{\pgfqpoint{8.383810in}{3.376382in}}%
\pgfpathcurveto{\pgfqpoint{8.374602in}{3.376382in}}{\pgfqpoint{8.365769in}{3.372724in}}{\pgfqpoint{8.359258in}{3.366212in}}%
\pgfpathcurveto{\pgfqpoint{8.352746in}{3.359701in}}{\pgfqpoint{8.349088in}{3.350869in}}{\pgfqpoint{8.349088in}{3.341660in}}%
\pgfpathcurveto{\pgfqpoint{8.349088in}{3.332452in}}{\pgfqpoint{8.352746in}{3.323619in}}{\pgfqpoint{8.359258in}{3.317108in}}%
\pgfpathcurveto{\pgfqpoint{8.365769in}{3.310596in}}{\pgfqpoint{8.374602in}{3.306938in}}{\pgfqpoint{8.383810in}{3.306938in}}%
\pgfpathlineto{\pgfqpoint{8.383810in}{3.306938in}}%
\pgfpathclose%
\pgfusepath{stroke,fill}%
\end{pgfscope}%
\begin{pgfscope}%
\pgfpathrectangle{\pgfqpoint{1.374500in}{0.082500in}}{\pgfqpoint{2.419000in}{2.419000in}}%
\pgfusepath{clip}%
\pgfsetbuttcap%
\pgfsetroundjoin%
\definecolor{currentfill}{rgb}{0.741176,0.121569,0.003922}%
\pgfsetfillcolor{currentfill}%
\pgfsetfillopacity{0.350778}%
\pgfsetlinewidth{1.003750pt}%
\definecolor{currentstroke}{rgb}{0.741176,0.121569,0.003922}%
\pgfsetstrokecolor{currentstroke}%
\pgfsetstrokeopacity{0.350778}%
\pgfsetdash{}{0pt}%
\pgfpathmoveto{\pgfqpoint{3.269522in}{3.233734in}}%
\pgfpathcurveto{\pgfqpoint{3.278731in}{3.233734in}}{\pgfqpoint{3.287563in}{3.237392in}}{\pgfqpoint{3.294075in}{3.243904in}}%
\pgfpathcurveto{\pgfqpoint{3.300586in}{3.250415in}}{\pgfqpoint{3.304245in}{3.259248in}}{\pgfqpoint{3.304245in}{3.268456in}}%
\pgfpathcurveto{\pgfqpoint{3.304245in}{3.277664in}}{\pgfqpoint{3.300586in}{3.286497in}}{\pgfqpoint{3.294075in}{3.293008in}}%
\pgfpathcurveto{\pgfqpoint{3.287563in}{3.299520in}}{\pgfqpoint{3.278731in}{3.303178in}}{\pgfqpoint{3.269522in}{3.303178in}}%
\pgfpathcurveto{\pgfqpoint{3.260314in}{3.303178in}}{\pgfqpoint{3.251482in}{3.299520in}}{\pgfqpoint{3.244970in}{3.293008in}}%
\pgfpathcurveto{\pgfqpoint{3.238459in}{3.286497in}}{\pgfqpoint{3.234800in}{3.277664in}}{\pgfqpoint{3.234800in}{3.268456in}}%
\pgfpathcurveto{\pgfqpoint{3.234800in}{3.259248in}}{\pgfqpoint{3.238459in}{3.250415in}}{\pgfqpoint{3.244970in}{3.243904in}}%
\pgfpathcurveto{\pgfqpoint{3.251482in}{3.237392in}}{\pgfqpoint{3.260314in}{3.233734in}}{\pgfqpoint{3.269522in}{3.233734in}}%
\pgfpathlineto{\pgfqpoint{3.269522in}{3.233734in}}%
\pgfpathclose%
\pgfusepath{stroke,fill}%
\end{pgfscope}%
\begin{pgfscope}%
\pgfpathrectangle{\pgfqpoint{1.374500in}{0.082500in}}{\pgfqpoint{2.419000in}{2.419000in}}%
\pgfusepath{clip}%
\pgfsetbuttcap%
\pgfsetroundjoin%
\definecolor{currentfill}{rgb}{0.741176,0.121569,0.003922}%
\pgfsetfillcolor{currentfill}%
\pgfsetfillopacity{0.350778}%
\pgfsetlinewidth{1.003750pt}%
\definecolor{currentstroke}{rgb}{0.741176,0.121569,0.003922}%
\pgfsetstrokecolor{currentstroke}%
\pgfsetstrokeopacity{0.350778}%
\pgfsetdash{}{0pt}%
\pgfpathmoveto{\pgfqpoint{6.960615in}{3.233734in}}%
\pgfpathcurveto{\pgfqpoint{6.969824in}{3.233734in}}{\pgfqpoint{6.978656in}{3.237392in}}{\pgfqpoint{6.985168in}{3.243904in}}%
\pgfpathcurveto{\pgfqpoint{6.991679in}{3.250415in}}{\pgfqpoint{6.995337in}{3.259248in}}{\pgfqpoint{6.995337in}{3.268456in}}%
\pgfpathcurveto{\pgfqpoint{6.995337in}{3.277664in}}{\pgfqpoint{6.991679in}{3.286497in}}{\pgfqpoint{6.985168in}{3.293008in}}%
\pgfpathcurveto{\pgfqpoint{6.978656in}{3.299520in}}{\pgfqpoint{6.969824in}{3.303178in}}{\pgfqpoint{6.960615in}{3.303178in}}%
\pgfpathcurveto{\pgfqpoint{6.951407in}{3.303178in}}{\pgfqpoint{6.942574in}{3.299520in}}{\pgfqpoint{6.936063in}{3.293008in}}%
\pgfpathcurveto{\pgfqpoint{6.929552in}{3.286497in}}{\pgfqpoint{6.925893in}{3.277664in}}{\pgfqpoint{6.925893in}{3.268456in}}%
\pgfpathcurveto{\pgfqpoint{6.925893in}{3.259248in}}{\pgfqpoint{6.929552in}{3.250415in}}{\pgfqpoint{6.936063in}{3.243904in}}%
\pgfpathcurveto{\pgfqpoint{6.942574in}{3.237392in}}{\pgfqpoint{6.951407in}{3.233734in}}{\pgfqpoint{6.960615in}{3.233734in}}%
\pgfpathlineto{\pgfqpoint{6.960615in}{3.233734in}}%
\pgfpathclose%
\pgfusepath{stroke,fill}%
\end{pgfscope}%
\begin{pgfscope}%
\pgfpathrectangle{\pgfqpoint{1.374500in}{0.082500in}}{\pgfqpoint{2.419000in}{2.419000in}}%
\pgfusepath{clip}%
\pgfsetbuttcap%
\pgfsetroundjoin%
\definecolor{currentfill}{rgb}{0.741176,0.121569,0.003922}%
\pgfsetfillcolor{currentfill}%
\pgfsetfillopacity{0.354486}%
\pgfsetlinewidth{1.003750pt}%
\definecolor{currentstroke}{rgb}{0.741176,0.121569,0.003922}%
\pgfsetstrokecolor{currentstroke}%
\pgfsetstrokeopacity{0.354486}%
\pgfsetdash{}{0pt}%
\pgfpathmoveto{\pgfqpoint{1.758662in}{3.158566in}}%
\pgfpathcurveto{\pgfqpoint{1.767871in}{3.158566in}}{\pgfqpoint{1.776703in}{3.162225in}}{\pgfqpoint{1.783215in}{3.168736in}}%
\pgfpathcurveto{\pgfqpoint{1.789726in}{3.175248in}}{\pgfqpoint{1.793385in}{3.184080in}}{\pgfqpoint{1.793385in}{3.193289in}}%
\pgfpathcurveto{\pgfqpoint{1.793385in}{3.202497in}}{\pgfqpoint{1.789726in}{3.211330in}}{\pgfqpoint{1.783215in}{3.217841in}}%
\pgfpathcurveto{\pgfqpoint{1.776703in}{3.224352in}}{\pgfqpoint{1.767871in}{3.228011in}}{\pgfqpoint{1.758662in}{3.228011in}}%
\pgfpathcurveto{\pgfqpoint{1.749454in}{3.228011in}}{\pgfqpoint{1.740622in}{3.224352in}}{\pgfqpoint{1.734110in}{3.217841in}}%
\pgfpathcurveto{\pgfqpoint{1.727599in}{3.211330in}}{\pgfqpoint{1.723940in}{3.202497in}}{\pgfqpoint{1.723940in}{3.193289in}}%
\pgfpathcurveto{\pgfqpoint{1.723940in}{3.184080in}}{\pgfqpoint{1.727599in}{3.175248in}}{\pgfqpoint{1.734110in}{3.168736in}}%
\pgfpathcurveto{\pgfqpoint{1.740622in}{3.162225in}}{\pgfqpoint{1.749454in}{3.158566in}}{\pgfqpoint{1.758662in}{3.158566in}}%
\pgfpathlineto{\pgfqpoint{1.758662in}{3.158566in}}%
\pgfpathclose%
\pgfusepath{stroke,fill}%
\end{pgfscope}%
\begin{pgfscope}%
\pgfpathrectangle{\pgfqpoint{1.374500in}{0.082500in}}{\pgfqpoint{2.419000in}{2.419000in}}%
\pgfusepath{clip}%
\pgfsetbuttcap%
\pgfsetroundjoin%
\definecolor{currentfill}{rgb}{0.741176,0.121569,0.003922}%
\pgfsetfillcolor{currentfill}%
\pgfsetfillopacity{0.354486}%
\pgfsetlinewidth{1.003750pt}%
\definecolor{currentstroke}{rgb}{0.741176,0.121569,0.003922}%
\pgfsetstrokecolor{currentstroke}%
\pgfsetstrokeopacity{0.354486}%
\pgfsetdash{}{0pt}%
\pgfpathmoveto{\pgfqpoint{5.499251in}{3.158566in}}%
\pgfpathcurveto{\pgfqpoint{5.508460in}{3.158566in}}{\pgfqpoint{5.517292in}{3.162225in}}{\pgfqpoint{5.523804in}{3.168736in}}%
\pgfpathcurveto{\pgfqpoint{5.530315in}{3.175248in}}{\pgfqpoint{5.533974in}{3.184080in}}{\pgfqpoint{5.533974in}{3.193289in}}%
\pgfpathcurveto{\pgfqpoint{5.533974in}{3.202497in}}{\pgfqpoint{5.530315in}{3.211330in}}{\pgfqpoint{5.523804in}{3.217841in}}%
\pgfpathcurveto{\pgfqpoint{5.517292in}{3.224352in}}{\pgfqpoint{5.508460in}{3.228011in}}{\pgfqpoint{5.499251in}{3.228011in}}%
\pgfpathcurveto{\pgfqpoint{5.490043in}{3.228011in}}{\pgfqpoint{5.481210in}{3.224352in}}{\pgfqpoint{5.474699in}{3.217841in}}%
\pgfpathcurveto{\pgfqpoint{5.468188in}{3.211330in}}{\pgfqpoint{5.464529in}{3.202497in}}{\pgfqpoint{5.464529in}{3.193289in}}%
\pgfpathcurveto{\pgfqpoint{5.464529in}{3.184080in}}{\pgfqpoint{5.468188in}{3.175248in}}{\pgfqpoint{5.474699in}{3.168736in}}%
\pgfpathcurveto{\pgfqpoint{5.481210in}{3.162225in}}{\pgfqpoint{5.490043in}{3.158566in}}{\pgfqpoint{5.499251in}{3.158566in}}%
\pgfpathlineto{\pgfqpoint{5.499251in}{3.158566in}}%
\pgfpathclose%
\pgfusepath{stroke,fill}%
\end{pgfscope}%
\begin{pgfscope}%
\pgfpathrectangle{\pgfqpoint{1.374500in}{0.082500in}}{\pgfqpoint{2.419000in}{2.419000in}}%
\pgfusepath{clip}%
\pgfsetbuttcap%
\pgfsetroundjoin%
\definecolor{currentfill}{rgb}{0.741176,0.121569,0.003922}%
\pgfsetfillcolor{currentfill}%
\pgfsetfillopacity{0.354486}%
\pgfsetlinewidth{1.003750pt}%
\definecolor{currentstroke}{rgb}{0.741176,0.121569,0.003922}%
\pgfsetstrokecolor{currentstroke}%
\pgfsetstrokeopacity{0.354486}%
\pgfsetdash{}{0pt}%
\pgfpathmoveto{\pgfqpoint{9.239840in}{3.158566in}}%
\pgfpathcurveto{\pgfqpoint{9.249049in}{3.158566in}}{\pgfqpoint{9.257881in}{3.162225in}}{\pgfqpoint{9.264393in}{3.168736in}}%
\pgfpathcurveto{\pgfqpoint{9.270904in}{3.175248in}}{\pgfqpoint{9.274563in}{3.184080in}}{\pgfqpoint{9.274563in}{3.193289in}}%
\pgfpathcurveto{\pgfqpoint{9.274563in}{3.202497in}}{\pgfqpoint{9.270904in}{3.211330in}}{\pgfqpoint{9.264393in}{3.217841in}}%
\pgfpathcurveto{\pgfqpoint{9.257881in}{3.224352in}}{\pgfqpoint{9.249049in}{3.228011in}}{\pgfqpoint{9.239840in}{3.228011in}}%
\pgfpathcurveto{\pgfqpoint{9.230632in}{3.228011in}}{\pgfqpoint{9.221799in}{3.224352in}}{\pgfqpoint{9.215288in}{3.217841in}}%
\pgfpathcurveto{\pgfqpoint{9.208777in}{3.211330in}}{\pgfqpoint{9.205118in}{3.202497in}}{\pgfqpoint{9.205118in}{3.193289in}}%
\pgfpathcurveto{\pgfqpoint{9.205118in}{3.184080in}}{\pgfqpoint{9.208777in}{3.175248in}}{\pgfqpoint{9.215288in}{3.168736in}}%
\pgfpathcurveto{\pgfqpoint{9.221799in}{3.162225in}}{\pgfqpoint{9.230632in}{3.158566in}}{\pgfqpoint{9.239840in}{3.158566in}}%
\pgfpathlineto{\pgfqpoint{9.239840in}{3.158566in}}%
\pgfpathclose%
\pgfusepath{stroke,fill}%
\end{pgfscope}%
\begin{pgfscope}%
\pgfpathrectangle{\pgfqpoint{1.374500in}{0.082500in}}{\pgfqpoint{2.419000in}{2.419000in}}%
\pgfusepath{clip}%
\pgfsetbuttcap%
\pgfsetroundjoin%
\definecolor{currentfill}{rgb}{0.741176,0.121569,0.003922}%
\pgfsetfillcolor{currentfill}%
\pgfsetfillopacity{0.358294}%
\pgfsetlinewidth{1.003750pt}%
\definecolor{currentstroke}{rgb}{0.741176,0.121569,0.003922}%
\pgfsetstrokecolor{currentstroke}%
\pgfsetstrokeopacity{0.358294}%
\pgfsetdash{}{0pt}%
\pgfpathmoveto{\pgfqpoint{3.998162in}{3.081356in}}%
\pgfpathcurveto{\pgfqpoint{4.007371in}{3.081356in}}{\pgfqpoint{4.016203in}{3.085014in}}{\pgfqpoint{4.022715in}{3.091525in}}%
\pgfpathcurveto{\pgfqpoint{4.029226in}{3.098037in}}{\pgfqpoint{4.032884in}{3.106869in}}{\pgfqpoint{4.032884in}{3.116078in}}%
\pgfpathcurveto{\pgfqpoint{4.032884in}{3.125286in}}{\pgfqpoint{4.029226in}{3.134119in}}{\pgfqpoint{4.022715in}{3.140630in}}%
\pgfpathcurveto{\pgfqpoint{4.016203in}{3.147141in}}{\pgfqpoint{4.007371in}{3.150800in}}{\pgfqpoint{3.998162in}{3.150800in}}%
\pgfpathcurveto{\pgfqpoint{3.988954in}{3.150800in}}{\pgfqpoint{3.980121in}{3.147141in}}{\pgfqpoint{3.973610in}{3.140630in}}%
\pgfpathcurveto{\pgfqpoint{3.967099in}{3.134119in}}{\pgfqpoint{3.963440in}{3.125286in}}{\pgfqpoint{3.963440in}{3.116078in}}%
\pgfpathcurveto{\pgfqpoint{3.963440in}{3.106869in}}{\pgfqpoint{3.967099in}{3.098037in}}{\pgfqpoint{3.973610in}{3.091525in}}%
\pgfpathcurveto{\pgfqpoint{3.980121in}{3.085014in}}{\pgfqpoint{3.988954in}{3.081356in}}{\pgfqpoint{3.998162in}{3.081356in}}%
\pgfpathlineto{\pgfqpoint{3.998162in}{3.081356in}}%
\pgfpathclose%
\pgfusepath{stroke,fill}%
\end{pgfscope}%
\begin{pgfscope}%
\pgfpathrectangle{\pgfqpoint{1.374500in}{0.082500in}}{\pgfqpoint{2.419000in}{2.419000in}}%
\pgfusepath{clip}%
\pgfsetbuttcap%
\pgfsetroundjoin%
\definecolor{currentfill}{rgb}{0.741176,0.121569,0.003922}%
\pgfsetfillcolor{currentfill}%
\pgfsetfillopacity{0.358294}%
\pgfsetlinewidth{1.003750pt}%
\definecolor{currentstroke}{rgb}{0.741176,0.121569,0.003922}%
\pgfsetstrokecolor{currentstroke}%
\pgfsetstrokeopacity{0.358294}%
\pgfsetdash{}{0pt}%
\pgfpathmoveto{\pgfqpoint{7.789593in}{3.081356in}}%
\pgfpathcurveto{\pgfqpoint{7.798801in}{3.081356in}}{\pgfqpoint{7.807634in}{3.085014in}}{\pgfqpoint{7.814145in}{3.091525in}}%
\pgfpathcurveto{\pgfqpoint{7.820657in}{3.098037in}}{\pgfqpoint{7.824315in}{3.106869in}}{\pgfqpoint{7.824315in}{3.116078in}}%
\pgfpathcurveto{\pgfqpoint{7.824315in}{3.125286in}}{\pgfqpoint{7.820657in}{3.134119in}}{\pgfqpoint{7.814145in}{3.140630in}}%
\pgfpathcurveto{\pgfqpoint{7.807634in}{3.147141in}}{\pgfqpoint{7.798801in}{3.150800in}}{\pgfqpoint{7.789593in}{3.150800in}}%
\pgfpathcurveto{\pgfqpoint{7.780385in}{3.150800in}}{\pgfqpoint{7.771552in}{3.147141in}}{\pgfqpoint{7.765041in}{3.140630in}}%
\pgfpathcurveto{\pgfqpoint{7.758529in}{3.134119in}}{\pgfqpoint{7.754871in}{3.125286in}}{\pgfqpoint{7.754871in}{3.116078in}}%
\pgfpathcurveto{\pgfqpoint{7.754871in}{3.106869in}}{\pgfqpoint{7.758529in}{3.098037in}}{\pgfqpoint{7.765041in}{3.091525in}}%
\pgfpathcurveto{\pgfqpoint{7.771552in}{3.085014in}}{\pgfqpoint{7.780385in}{3.081356in}}{\pgfqpoint{7.789593in}{3.081356in}}%
\pgfpathlineto{\pgfqpoint{7.789593in}{3.081356in}}%
\pgfpathclose%
\pgfusepath{stroke,fill}%
\end{pgfscope}%
\begin{pgfscope}%
\pgfpathrectangle{\pgfqpoint{1.374500in}{0.082500in}}{\pgfqpoint{2.419000in}{2.419000in}}%
\pgfusepath{clip}%
\pgfsetbuttcap%
\pgfsetroundjoin%
\definecolor{currentfill}{rgb}{0.741176,0.121569,0.003922}%
\pgfsetfillcolor{currentfill}%
\pgfsetfillopacity{0.362208}%
\pgfsetlinewidth{1.003750pt}%
\definecolor{currentstroke}{rgb}{0.741176,0.121569,0.003922}%
\pgfsetstrokecolor{currentstroke}%
\pgfsetstrokeopacity{0.362208}%
\pgfsetdash{}{0pt}%
\pgfpathmoveto{\pgfqpoint{6.299379in}{3.002017in}}%
\pgfpathcurveto{\pgfqpoint{6.308587in}{3.002017in}}{\pgfqpoint{6.317420in}{3.005676in}}{\pgfqpoint{6.323931in}{3.012187in}}%
\pgfpathcurveto{\pgfqpoint{6.330443in}{3.018698in}}{\pgfqpoint{6.334101in}{3.027531in}}{\pgfqpoint{6.334101in}{3.036739in}}%
\pgfpathcurveto{\pgfqpoint{6.334101in}{3.045948in}}{\pgfqpoint{6.330443in}{3.054780in}}{\pgfqpoint{6.323931in}{3.061291in}}%
\pgfpathcurveto{\pgfqpoint{6.317420in}{3.067803in}}{\pgfqpoint{6.308587in}{3.071461in}}{\pgfqpoint{6.299379in}{3.071461in}}%
\pgfpathcurveto{\pgfqpoint{6.290171in}{3.071461in}}{\pgfqpoint{6.281338in}{3.067803in}}{\pgfqpoint{6.274827in}{3.061291in}}%
\pgfpathcurveto{\pgfqpoint{6.268315in}{3.054780in}}{\pgfqpoint{6.264657in}{3.045948in}}{\pgfqpoint{6.264657in}{3.036739in}}%
\pgfpathcurveto{\pgfqpoint{6.264657in}{3.027531in}}{\pgfqpoint{6.268315in}{3.018698in}}{\pgfqpoint{6.274827in}{3.012187in}}%
\pgfpathcurveto{\pgfqpoint{6.281338in}{3.005676in}}{\pgfqpoint{6.290171in}{3.002017in}}{\pgfqpoint{6.299379in}{3.002017in}}%
\pgfpathlineto{\pgfqpoint{6.299379in}{3.002017in}}%
\pgfpathclose%
\pgfusepath{stroke,fill}%
\end{pgfscope}%
\begin{pgfscope}%
\pgfpathrectangle{\pgfqpoint{1.374500in}{0.082500in}}{\pgfqpoint{2.419000in}{2.419000in}}%
\pgfusepath{clip}%
\pgfsetbuttcap%
\pgfsetroundjoin%
\definecolor{currentfill}{rgb}{0.741176,0.121569,0.003922}%
\pgfsetfillcolor{currentfill}%
\pgfsetfillopacity{0.362208}%
\pgfsetlinewidth{1.003750pt}%
\definecolor{currentstroke}{rgb}{0.741176,0.121569,0.003922}%
\pgfsetstrokecolor{currentstroke}%
\pgfsetstrokeopacity{0.362208}%
\pgfsetdash{}{0pt}%
\pgfpathmoveto{\pgfqpoint{2.455705in}{3.002017in}}%
\pgfpathcurveto{\pgfqpoint{2.464914in}{3.002017in}}{\pgfqpoint{2.473746in}{3.005676in}}{\pgfqpoint{2.480258in}{3.012187in}}%
\pgfpathcurveto{\pgfqpoint{2.486769in}{3.018698in}}{\pgfqpoint{2.490428in}{3.027531in}}{\pgfqpoint{2.490428in}{3.036739in}}%
\pgfpathcurveto{\pgfqpoint{2.490428in}{3.045948in}}{\pgfqpoint{2.486769in}{3.054780in}}{\pgfqpoint{2.480258in}{3.061291in}}%
\pgfpathcurveto{\pgfqpoint{2.473746in}{3.067803in}}{\pgfqpoint{2.464914in}{3.071461in}}{\pgfqpoint{2.455705in}{3.071461in}}%
\pgfpathcurveto{\pgfqpoint{2.446497in}{3.071461in}}{\pgfqpoint{2.437664in}{3.067803in}}{\pgfqpoint{2.431153in}{3.061291in}}%
\pgfpathcurveto{\pgfqpoint{2.424642in}{3.054780in}}{\pgfqpoint{2.420983in}{3.045948in}}{\pgfqpoint{2.420983in}{3.036739in}}%
\pgfpathcurveto{\pgfqpoint{2.420983in}{3.027531in}}{\pgfqpoint{2.424642in}{3.018698in}}{\pgfqpoint{2.431153in}{3.012187in}}%
\pgfpathcurveto{\pgfqpoint{2.437664in}{3.005676in}}{\pgfqpoint{2.446497in}{3.002017in}}{\pgfqpoint{2.455705in}{3.002017in}}%
\pgfpathlineto{\pgfqpoint{2.455705in}{3.002017in}}%
\pgfpathclose%
\pgfusepath{stroke,fill}%
\end{pgfscope}%
\begin{pgfscope}%
\pgfpathrectangle{\pgfqpoint{1.374500in}{0.082500in}}{\pgfqpoint{2.419000in}{2.419000in}}%
\pgfusepath{clip}%
\pgfsetbuttcap%
\pgfsetroundjoin%
\definecolor{currentfill}{rgb}{0.741176,0.121569,0.003922}%
\pgfsetfillcolor{currentfill}%
\pgfsetfillopacity{0.362208}%
\pgfsetlinewidth{1.003750pt}%
\definecolor{currentstroke}{rgb}{0.741176,0.121569,0.003922}%
\pgfsetstrokecolor{currentstroke}%
\pgfsetstrokeopacity{0.362208}%
\pgfsetdash{}{0pt}%
\pgfpathmoveto{\pgfqpoint{10.143053in}{3.002017in}}%
\pgfpathcurveto{\pgfqpoint{10.152261in}{3.002017in}}{\pgfqpoint{10.161093in}{3.005676in}}{\pgfqpoint{10.167605in}{3.012187in}}%
\pgfpathcurveto{\pgfqpoint{10.174116in}{3.018698in}}{\pgfqpoint{10.177775in}{3.027531in}}{\pgfqpoint{10.177775in}{3.036739in}}%
\pgfpathcurveto{\pgfqpoint{10.177775in}{3.045948in}}{\pgfqpoint{10.174116in}{3.054780in}}{\pgfqpoint{10.167605in}{3.061291in}}%
\pgfpathcurveto{\pgfqpoint{10.161093in}{3.067803in}}{\pgfqpoint{10.152261in}{3.071461in}}{\pgfqpoint{10.143053in}{3.071461in}}%
\pgfpathcurveto{\pgfqpoint{10.133844in}{3.071461in}}{\pgfqpoint{10.125012in}{3.067803in}}{\pgfqpoint{10.118500in}{3.061291in}}%
\pgfpathcurveto{\pgfqpoint{10.111989in}{3.054780in}}{\pgfqpoint{10.108330in}{3.045948in}}{\pgfqpoint{10.108330in}{3.036739in}}%
\pgfpathcurveto{\pgfqpoint{10.108330in}{3.027531in}}{\pgfqpoint{10.111989in}{3.018698in}}{\pgfqpoint{10.118500in}{3.012187in}}%
\pgfpathcurveto{\pgfqpoint{10.125012in}{3.005676in}}{\pgfqpoint{10.133844in}{3.002017in}}{\pgfqpoint{10.143053in}{3.002017in}}%
\pgfpathlineto{\pgfqpoint{10.143053in}{3.002017in}}%
\pgfpathclose%
\pgfusepath{stroke,fill}%
\end{pgfscope}%
\begin{pgfscope}%
\pgfpathrectangle{\pgfqpoint{1.374500in}{0.082500in}}{\pgfqpoint{2.419000in}{2.419000in}}%
\pgfusepath{clip}%
\pgfsetbuttcap%
\pgfsetroundjoin%
\definecolor{currentfill}{rgb}{0.741176,0.121569,0.003922}%
\pgfsetfillcolor{currentfill}%
\pgfsetfillopacity{0.366231}%
\pgfsetlinewidth{1.003750pt}%
\definecolor{currentstroke}{rgb}{0.741176,0.121569,0.003922}%
\pgfsetstrokecolor{currentstroke}%
\pgfsetstrokeopacity{0.366231}%
\pgfsetdash{}{0pt}%
\pgfpathmoveto{\pgfqpoint{0.870147in}{2.920461in}}%
\pgfpathcurveto{\pgfqpoint{0.879356in}{2.920461in}}{\pgfqpoint{0.888188in}{2.924120in}}{\pgfqpoint{0.894699in}{2.930631in}}%
\pgfpathcurveto{\pgfqpoint{0.901211in}{2.937143in}}{\pgfqpoint{0.904869in}{2.945975in}}{\pgfqpoint{0.904869in}{2.955184in}}%
\pgfpathcurveto{\pgfqpoint{0.904869in}{2.964392in}}{\pgfqpoint{0.901211in}{2.973225in}}{\pgfqpoint{0.894699in}{2.979736in}}%
\pgfpathcurveto{\pgfqpoint{0.888188in}{2.986247in}}{\pgfqpoint{0.879356in}{2.989906in}}{\pgfqpoint{0.870147in}{2.989906in}}%
\pgfpathcurveto{\pgfqpoint{0.860939in}{2.989906in}}{\pgfqpoint{0.852106in}{2.986247in}}{\pgfqpoint{0.845595in}{2.979736in}}%
\pgfpathcurveto{\pgfqpoint{0.839083in}{2.973225in}}{\pgfqpoint{0.835425in}{2.964392in}}{\pgfqpoint{0.835425in}{2.955184in}}%
\pgfpathcurveto{\pgfqpoint{0.835425in}{2.945975in}}{\pgfqpoint{0.839083in}{2.937143in}}{\pgfqpoint{0.845595in}{2.930631in}}%
\pgfpathcurveto{\pgfqpoint{0.852106in}{2.924120in}}{\pgfqpoint{0.860939in}{2.920461in}}{\pgfqpoint{0.870147in}{2.920461in}}%
\pgfpathlineto{\pgfqpoint{0.870147in}{2.920461in}}%
\pgfpathclose%
\pgfusepath{stroke,fill}%
\end{pgfscope}%
\begin{pgfscope}%
\pgfpathrectangle{\pgfqpoint{1.374500in}{0.082500in}}{\pgfqpoint{2.419000in}{2.419000in}}%
\pgfusepath{clip}%
\pgfsetbuttcap%
\pgfsetroundjoin%
\definecolor{currentfill}{rgb}{0.741176,0.121569,0.003922}%
\pgfsetfillcolor{currentfill}%
\pgfsetfillopacity{0.366231}%
\pgfsetlinewidth{1.003750pt}%
\definecolor{currentstroke}{rgb}{0.741176,0.121569,0.003922}%
\pgfsetstrokecolor{currentstroke}%
\pgfsetstrokeopacity{0.366231}%
\pgfsetdash{}{0pt}%
\pgfpathmoveto{\pgfqpoint{4.767523in}{2.920461in}}%
\pgfpathcurveto{\pgfqpoint{4.776732in}{2.920461in}}{\pgfqpoint{4.785564in}{2.924120in}}{\pgfqpoint{4.792076in}{2.930631in}}%
\pgfpathcurveto{\pgfqpoint{4.798587in}{2.937143in}}{\pgfqpoint{4.802246in}{2.945975in}}{\pgfqpoint{4.802246in}{2.955184in}}%
\pgfpathcurveto{\pgfqpoint{4.802246in}{2.964392in}}{\pgfqpoint{4.798587in}{2.973225in}}{\pgfqpoint{4.792076in}{2.979736in}}%
\pgfpathcurveto{\pgfqpoint{4.785564in}{2.986247in}}{\pgfqpoint{4.776732in}{2.989906in}}{\pgfqpoint{4.767523in}{2.989906in}}%
\pgfpathcurveto{\pgfqpoint{4.758315in}{2.989906in}}{\pgfqpoint{4.749482in}{2.986247in}}{\pgfqpoint{4.742971in}{2.979736in}}%
\pgfpathcurveto{\pgfqpoint{4.736460in}{2.973225in}}{\pgfqpoint{4.732801in}{2.964392in}}{\pgfqpoint{4.732801in}{2.955184in}}%
\pgfpathcurveto{\pgfqpoint{4.732801in}{2.945975in}}{\pgfqpoint{4.736460in}{2.937143in}}{\pgfqpoint{4.742971in}{2.930631in}}%
\pgfpathcurveto{\pgfqpoint{4.749482in}{2.924120in}}{\pgfqpoint{4.758315in}{2.920461in}}{\pgfqpoint{4.767523in}{2.920461in}}%
\pgfpathlineto{\pgfqpoint{4.767523in}{2.920461in}}%
\pgfpathclose%
\pgfusepath{stroke,fill}%
\end{pgfscope}%
\begin{pgfscope}%
\pgfpathrectangle{\pgfqpoint{1.374500in}{0.082500in}}{\pgfqpoint{2.419000in}{2.419000in}}%
\pgfusepath{clip}%
\pgfsetbuttcap%
\pgfsetroundjoin%
\definecolor{currentfill}{rgb}{0.741176,0.121569,0.003922}%
\pgfsetfillcolor{currentfill}%
\pgfsetfillopacity{0.366231}%
\pgfsetlinewidth{1.003750pt}%
\definecolor{currentstroke}{rgb}{0.741176,0.121569,0.003922}%
\pgfsetstrokecolor{currentstroke}%
\pgfsetstrokeopacity{0.366231}%
\pgfsetdash{}{0pt}%
\pgfpathmoveto{\pgfqpoint{8.664900in}{2.920461in}}%
\pgfpathcurveto{\pgfqpoint{8.674108in}{2.920461in}}{\pgfqpoint{8.682941in}{2.924120in}}{\pgfqpoint{8.689452in}{2.930631in}}%
\pgfpathcurveto{\pgfqpoint{8.695963in}{2.937143in}}{\pgfqpoint{8.699622in}{2.945975in}}{\pgfqpoint{8.699622in}{2.955184in}}%
\pgfpathcurveto{\pgfqpoint{8.699622in}{2.964392in}}{\pgfqpoint{8.695963in}{2.973225in}}{\pgfqpoint{8.689452in}{2.979736in}}%
\pgfpathcurveto{\pgfqpoint{8.682941in}{2.986247in}}{\pgfqpoint{8.674108in}{2.989906in}}{\pgfqpoint{8.664900in}{2.989906in}}%
\pgfpathcurveto{\pgfqpoint{8.655691in}{2.989906in}}{\pgfqpoint{8.646859in}{2.986247in}}{\pgfqpoint{8.640347in}{2.979736in}}%
\pgfpathcurveto{\pgfqpoint{8.633836in}{2.973225in}}{\pgfqpoint{8.630177in}{2.964392in}}{\pgfqpoint{8.630177in}{2.955184in}}%
\pgfpathcurveto{\pgfqpoint{8.630177in}{2.945975in}}{\pgfqpoint{8.633836in}{2.937143in}}{\pgfqpoint{8.640347in}{2.930631in}}%
\pgfpathcurveto{\pgfqpoint{8.646859in}{2.924120in}}{\pgfqpoint{8.655691in}{2.920461in}}{\pgfqpoint{8.664900in}{2.920461in}}%
\pgfpathlineto{\pgfqpoint{8.664900in}{2.920461in}}%
\pgfpathclose%
\pgfusepath{stroke,fill}%
\end{pgfscope}%
\begin{pgfscope}%
\pgfpathrectangle{\pgfqpoint{1.374500in}{0.082500in}}{\pgfqpoint{2.419000in}{2.419000in}}%
\pgfusepath{clip}%
\pgfsetbuttcap%
\pgfsetroundjoin%
\definecolor{currentfill}{rgb}{0.741176,0.121569,0.003922}%
\pgfsetfillcolor{currentfill}%
\pgfsetfillopacity{0.370368}%
\pgfsetlinewidth{1.003750pt}%
\definecolor{currentstroke}{rgb}{0.741176,0.121569,0.003922}%
\pgfsetstrokecolor{currentstroke}%
\pgfsetstrokeopacity{0.370368}%
\pgfsetdash{}{0pt}%
\pgfpathmoveto{\pgfqpoint{3.192256in}{2.836595in}}%
\pgfpathcurveto{\pgfqpoint{3.201464in}{2.836595in}}{\pgfqpoint{3.210297in}{2.840253in}}{\pgfqpoint{3.216808in}{2.846764in}}%
\pgfpathcurveto{\pgfqpoint{3.223320in}{2.853276in}}{\pgfqpoint{3.226978in}{2.862108in}}{\pgfqpoint{3.226978in}{2.871317in}}%
\pgfpathcurveto{\pgfqpoint{3.226978in}{2.880525in}}{\pgfqpoint{3.223320in}{2.889358in}}{\pgfqpoint{3.216808in}{2.895869in}}%
\pgfpathcurveto{\pgfqpoint{3.210297in}{2.902380in}}{\pgfqpoint{3.201464in}{2.906039in}}{\pgfqpoint{3.192256in}{2.906039in}}%
\pgfpathcurveto{\pgfqpoint{3.183047in}{2.906039in}}{\pgfqpoint{3.174215in}{2.902380in}}{\pgfqpoint{3.167704in}{2.895869in}}%
\pgfpathcurveto{\pgfqpoint{3.161192in}{2.889358in}}{\pgfqpoint{3.157534in}{2.880525in}}{\pgfqpoint{3.157534in}{2.871317in}}%
\pgfpathcurveto{\pgfqpoint{3.157534in}{2.862108in}}{\pgfqpoint{3.161192in}{2.853276in}}{\pgfqpoint{3.167704in}{2.846764in}}%
\pgfpathcurveto{\pgfqpoint{3.174215in}{2.840253in}}{\pgfqpoint{3.183047in}{2.836595in}}{\pgfqpoint{3.192256in}{2.836595in}}%
\pgfpathlineto{\pgfqpoint{3.192256in}{2.836595in}}%
\pgfpathclose%
\pgfusepath{stroke,fill}%
\end{pgfscope}%
\begin{pgfscope}%
\pgfpathrectangle{\pgfqpoint{1.374500in}{0.082500in}}{\pgfqpoint{2.419000in}{2.419000in}}%
\pgfusepath{clip}%
\pgfsetbuttcap%
\pgfsetroundjoin%
\definecolor{currentfill}{rgb}{0.741176,0.121569,0.003922}%
\pgfsetfillcolor{currentfill}%
\pgfsetfillopacity{0.370368}%
\pgfsetlinewidth{1.003750pt}%
\definecolor{currentstroke}{rgb}{0.741176,0.121569,0.003922}%
\pgfsetstrokecolor{currentstroke}%
\pgfsetstrokeopacity{0.370368}%
\pgfsetdash{}{0pt}%
\pgfpathmoveto{\pgfqpoint{7.144857in}{2.836595in}}%
\pgfpathcurveto{\pgfqpoint{7.154065in}{2.836595in}}{\pgfqpoint{7.162898in}{2.840253in}}{\pgfqpoint{7.169409in}{2.846764in}}%
\pgfpathcurveto{\pgfqpoint{7.175920in}{2.853276in}}{\pgfqpoint{7.179579in}{2.862108in}}{\pgfqpoint{7.179579in}{2.871317in}}%
\pgfpathcurveto{\pgfqpoint{7.179579in}{2.880525in}}{\pgfqpoint{7.175920in}{2.889358in}}{\pgfqpoint{7.169409in}{2.895869in}}%
\pgfpathcurveto{\pgfqpoint{7.162898in}{2.902380in}}{\pgfqpoint{7.154065in}{2.906039in}}{\pgfqpoint{7.144857in}{2.906039in}}%
\pgfpathcurveto{\pgfqpoint{7.135648in}{2.906039in}}{\pgfqpoint{7.126816in}{2.902380in}}{\pgfqpoint{7.120304in}{2.895869in}}%
\pgfpathcurveto{\pgfqpoint{7.113793in}{2.889358in}}{\pgfqpoint{7.110134in}{2.880525in}}{\pgfqpoint{7.110134in}{2.871317in}}%
\pgfpathcurveto{\pgfqpoint{7.110134in}{2.862108in}}{\pgfqpoint{7.113793in}{2.853276in}}{\pgfqpoint{7.120304in}{2.846764in}}%
\pgfpathcurveto{\pgfqpoint{7.126816in}{2.840253in}}{\pgfqpoint{7.135648in}{2.836595in}}{\pgfqpoint{7.144857in}{2.836595in}}%
\pgfpathlineto{\pgfqpoint{7.144857in}{2.836595in}}%
\pgfpathclose%
\pgfusepath{stroke,fill}%
\end{pgfscope}%
\begin{pgfscope}%
\pgfpathrectangle{\pgfqpoint{1.374500in}{0.082500in}}{\pgfqpoint{2.419000in}{2.419000in}}%
\pgfusepath{clip}%
\pgfsetbuttcap%
\pgfsetroundjoin%
\definecolor{currentfill}{rgb}{0.741176,0.121569,0.003922}%
\pgfsetfillcolor{currentfill}%
\pgfsetfillopacity{0.370368}%
\pgfsetlinewidth{1.003750pt}%
\definecolor{currentstroke}{rgb}{0.741176,0.121569,0.003922}%
\pgfsetstrokecolor{currentstroke}%
\pgfsetstrokeopacity{0.370368}%
\pgfsetdash{}{0pt}%
\pgfpathmoveto{\pgfqpoint{11.097458in}{2.836595in}}%
\pgfpathcurveto{\pgfqpoint{11.106666in}{2.836595in}}{\pgfqpoint{11.115498in}{2.840253in}}{\pgfqpoint{11.122010in}{2.846764in}}%
\pgfpathcurveto{\pgfqpoint{11.128521in}{2.853276in}}{\pgfqpoint{11.132180in}{2.862108in}}{\pgfqpoint{11.132180in}{2.871317in}}%
\pgfpathcurveto{\pgfqpoint{11.132180in}{2.880525in}}{\pgfqpoint{11.128521in}{2.889358in}}{\pgfqpoint{11.122010in}{2.895869in}}%
\pgfpathcurveto{\pgfqpoint{11.115498in}{2.902380in}}{\pgfqpoint{11.106666in}{2.906039in}}{\pgfqpoint{11.097458in}{2.906039in}}%
\pgfpathcurveto{\pgfqpoint{11.088249in}{2.906039in}}{\pgfqpoint{11.079417in}{2.902380in}}{\pgfqpoint{11.072905in}{2.895869in}}%
\pgfpathcurveto{\pgfqpoint{11.066394in}{2.889358in}}{\pgfqpoint{11.062735in}{2.880525in}}{\pgfqpoint{11.062735in}{2.871317in}}%
\pgfpathcurveto{\pgfqpoint{11.062735in}{2.862108in}}{\pgfqpoint{11.066394in}{2.853276in}}{\pgfqpoint{11.072905in}{2.846764in}}%
\pgfpathcurveto{\pgfqpoint{11.079417in}{2.840253in}}{\pgfqpoint{11.088249in}{2.836595in}}{\pgfqpoint{11.097458in}{2.836595in}}%
\pgfpathlineto{\pgfqpoint{11.097458in}{2.836595in}}%
\pgfpathclose%
\pgfusepath{stroke,fill}%
\end{pgfscope}%
\begin{pgfscope}%
\pgfpathrectangle{\pgfqpoint{1.374500in}{0.082500in}}{\pgfqpoint{2.419000in}{2.419000in}}%
\pgfusepath{clip}%
\pgfsetbuttcap%
\pgfsetroundjoin%
\definecolor{currentfill}{rgb}{0.741176,0.121569,0.003922}%
\pgfsetfillcolor{currentfill}%
\pgfsetfillopacity{0.374624}%
\pgfsetlinewidth{1.003750pt}%
\definecolor{currentstroke}{rgb}{0.741176,0.121569,0.003922}%
\pgfsetstrokecolor{currentstroke}%
\pgfsetstrokeopacity{0.374624}%
\pgfsetdash{}{0pt}%
\pgfpathmoveto{\pgfqpoint{1.571705in}{2.750317in}}%
\pgfpathcurveto{\pgfqpoint{1.580913in}{2.750317in}}{\pgfqpoint{1.589746in}{2.753975in}}{\pgfqpoint{1.596257in}{2.760487in}}%
\pgfpathcurveto{\pgfqpoint{1.602768in}{2.766998in}}{\pgfqpoint{1.606427in}{2.775831in}}{\pgfqpoint{1.606427in}{2.785039in}}%
\pgfpathcurveto{\pgfqpoint{1.606427in}{2.794248in}}{\pgfqpoint{1.602768in}{2.803080in}}{\pgfqpoint{1.596257in}{2.809591in}}%
\pgfpathcurveto{\pgfqpoint{1.589746in}{2.816103in}}{\pgfqpoint{1.580913in}{2.819761in}}{\pgfqpoint{1.571705in}{2.819761in}}%
\pgfpathcurveto{\pgfqpoint{1.562496in}{2.819761in}}{\pgfqpoint{1.553664in}{2.816103in}}{\pgfqpoint{1.547152in}{2.809591in}}%
\pgfpathcurveto{\pgfqpoint{1.540641in}{2.803080in}}{\pgfqpoint{1.536983in}{2.794248in}}{\pgfqpoint{1.536983in}{2.785039in}}%
\pgfpathcurveto{\pgfqpoint{1.536983in}{2.775831in}}{\pgfqpoint{1.540641in}{2.766998in}}{\pgfqpoint{1.547152in}{2.760487in}}%
\pgfpathcurveto{\pgfqpoint{1.553664in}{2.753975in}}{\pgfqpoint{1.562496in}{2.750317in}}{\pgfqpoint{1.571705in}{2.750317in}}%
\pgfpathlineto{\pgfqpoint{1.571705in}{2.750317in}}%
\pgfpathclose%
\pgfusepath{stroke,fill}%
\end{pgfscope}%
\begin{pgfscope}%
\pgfpathrectangle{\pgfqpoint{1.374500in}{0.082500in}}{\pgfqpoint{2.419000in}{2.419000in}}%
\pgfusepath{clip}%
\pgfsetbuttcap%
\pgfsetroundjoin%
\definecolor{currentfill}{rgb}{0.741176,0.121569,0.003922}%
\pgfsetfillcolor{currentfill}%
\pgfsetfillopacity{0.374624}%
\pgfsetlinewidth{1.003750pt}%
\definecolor{currentstroke}{rgb}{0.741176,0.121569,0.003922}%
\pgfsetstrokecolor{currentstroke}%
\pgfsetstrokeopacity{0.374624}%
\pgfsetdash{}{0pt}%
\pgfpathmoveto{\pgfqpoint{5.581118in}{2.750317in}}%
\pgfpathcurveto{\pgfqpoint{5.590326in}{2.750317in}}{\pgfqpoint{5.599159in}{2.753975in}}{\pgfqpoint{5.605670in}{2.760487in}}%
\pgfpathcurveto{\pgfqpoint{5.612181in}{2.766998in}}{\pgfqpoint{5.615840in}{2.775831in}}{\pgfqpoint{5.615840in}{2.785039in}}%
\pgfpathcurveto{\pgfqpoint{5.615840in}{2.794248in}}{\pgfqpoint{5.612181in}{2.803080in}}{\pgfqpoint{5.605670in}{2.809591in}}%
\pgfpathcurveto{\pgfqpoint{5.599159in}{2.816103in}}{\pgfqpoint{5.590326in}{2.819761in}}{\pgfqpoint{5.581118in}{2.819761in}}%
\pgfpathcurveto{\pgfqpoint{5.571909in}{2.819761in}}{\pgfqpoint{5.563077in}{2.816103in}}{\pgfqpoint{5.556565in}{2.809591in}}%
\pgfpathcurveto{\pgfqpoint{5.550054in}{2.803080in}}{\pgfqpoint{5.546395in}{2.794248in}}{\pgfqpoint{5.546395in}{2.785039in}}%
\pgfpathcurveto{\pgfqpoint{5.546395in}{2.775831in}}{\pgfqpoint{5.550054in}{2.766998in}}{\pgfqpoint{5.556565in}{2.760487in}}%
\pgfpathcurveto{\pgfqpoint{5.563077in}{2.753975in}}{\pgfqpoint{5.571909in}{2.750317in}}{\pgfqpoint{5.581118in}{2.750317in}}%
\pgfpathlineto{\pgfqpoint{5.581118in}{2.750317in}}%
\pgfpathclose%
\pgfusepath{stroke,fill}%
\end{pgfscope}%
\begin{pgfscope}%
\pgfpathrectangle{\pgfqpoint{1.374500in}{0.082500in}}{\pgfqpoint{2.419000in}{2.419000in}}%
\pgfusepath{clip}%
\pgfsetbuttcap%
\pgfsetroundjoin%
\definecolor{currentfill}{rgb}{0.741176,0.121569,0.003922}%
\pgfsetfillcolor{currentfill}%
\pgfsetfillopacity{0.374624}%
\pgfsetlinewidth{1.003750pt}%
\definecolor{currentstroke}{rgb}{0.741176,0.121569,0.003922}%
\pgfsetstrokecolor{currentstroke}%
\pgfsetstrokeopacity{0.374624}%
\pgfsetdash{}{0pt}%
\pgfpathmoveto{\pgfqpoint{9.590531in}{2.750317in}}%
\pgfpathcurveto{\pgfqpoint{9.599739in}{2.750317in}}{\pgfqpoint{9.608572in}{2.753975in}}{\pgfqpoint{9.615083in}{2.760487in}}%
\pgfpathcurveto{\pgfqpoint{9.621594in}{2.766998in}}{\pgfqpoint{9.625253in}{2.775831in}}{\pgfqpoint{9.625253in}{2.785039in}}%
\pgfpathcurveto{\pgfqpoint{9.625253in}{2.794248in}}{\pgfqpoint{9.621594in}{2.803080in}}{\pgfqpoint{9.615083in}{2.809591in}}%
\pgfpathcurveto{\pgfqpoint{9.608572in}{2.816103in}}{\pgfqpoint{9.599739in}{2.819761in}}{\pgfqpoint{9.590531in}{2.819761in}}%
\pgfpathcurveto{\pgfqpoint{9.581322in}{2.819761in}}{\pgfqpoint{9.572490in}{2.816103in}}{\pgfqpoint{9.565978in}{2.809591in}}%
\pgfpathcurveto{\pgfqpoint{9.559467in}{2.803080in}}{\pgfqpoint{9.555808in}{2.794248in}}{\pgfqpoint{9.555808in}{2.785039in}}%
\pgfpathcurveto{\pgfqpoint{9.555808in}{2.775831in}}{\pgfqpoint{9.559467in}{2.766998in}}{\pgfqpoint{9.565978in}{2.760487in}}%
\pgfpathcurveto{\pgfqpoint{9.572490in}{2.753975in}}{\pgfqpoint{9.581322in}{2.750317in}}{\pgfqpoint{9.590531in}{2.750317in}}%
\pgfpathlineto{\pgfqpoint{9.590531in}{2.750317in}}%
\pgfpathclose%
\pgfusepath{stroke,fill}%
\end{pgfscope}%
\begin{pgfscope}%
\pgfpathrectangle{\pgfqpoint{1.374500in}{0.082500in}}{\pgfqpoint{2.419000in}{2.419000in}}%
\pgfusepath{clip}%
\pgfsetbuttcap%
\pgfsetroundjoin%
\definecolor{currentfill}{rgb}{0.741176,0.121569,0.003922}%
\pgfsetfillcolor{currentfill}%
\pgfsetfillopacity{0.379004}%
\pgfsetlinewidth{1.003750pt}%
\definecolor{currentstroke}{rgb}{0.741176,0.121569,0.003922}%
\pgfsetstrokecolor{currentstroke}%
\pgfsetstrokeopacity{0.379004}%
\pgfsetdash{}{0pt}%
\pgfpathmoveto{\pgfqpoint{3.971771in}{2.661523in}}%
\pgfpathcurveto{\pgfqpoint{3.980979in}{2.661523in}}{\pgfqpoint{3.989812in}{2.665181in}}{\pgfqpoint{3.996323in}{2.671693in}}%
\pgfpathcurveto{\pgfqpoint{4.002834in}{2.678204in}}{\pgfqpoint{4.006493in}{2.687037in}}{\pgfqpoint{4.006493in}{2.696245in}}%
\pgfpathcurveto{\pgfqpoint{4.006493in}{2.705453in}}{\pgfqpoint{4.002834in}{2.714286in}}{\pgfqpoint{3.996323in}{2.720797in}}%
\pgfpathcurveto{\pgfqpoint{3.989812in}{2.727309in}}{\pgfqpoint{3.980979in}{2.730967in}}{\pgfqpoint{3.971771in}{2.730967in}}%
\pgfpathcurveto{\pgfqpoint{3.962562in}{2.730967in}}{\pgfqpoint{3.953730in}{2.727309in}}{\pgfqpoint{3.947218in}{2.720797in}}%
\pgfpathcurveto{\pgfqpoint{3.940707in}{2.714286in}}{\pgfqpoint{3.937048in}{2.705453in}}{\pgfqpoint{3.937048in}{2.696245in}}%
\pgfpathcurveto{\pgfqpoint{3.937048in}{2.687037in}}{\pgfqpoint{3.940707in}{2.678204in}}{\pgfqpoint{3.947218in}{2.671693in}}%
\pgfpathcurveto{\pgfqpoint{3.953730in}{2.665181in}}{\pgfqpoint{3.962562in}{2.661523in}}{\pgfqpoint{3.971771in}{2.661523in}}%
\pgfpathlineto{\pgfqpoint{3.971771in}{2.661523in}}%
\pgfpathclose%
\pgfusepath{stroke,fill}%
\end{pgfscope}%
\begin{pgfscope}%
\pgfpathrectangle{\pgfqpoint{1.374500in}{0.082500in}}{\pgfqpoint{2.419000in}{2.419000in}}%
\pgfusepath{clip}%
\pgfsetbuttcap%
\pgfsetroundjoin%
\definecolor{currentfill}{rgb}{0.741176,0.121569,0.003922}%
\pgfsetfillcolor{currentfill}%
\pgfsetfillopacity{0.379004}%
\pgfsetlinewidth{1.003750pt}%
\definecolor{currentstroke}{rgb}{0.741176,0.121569,0.003922}%
\pgfsetstrokecolor{currentstroke}%
\pgfsetstrokeopacity{0.379004}%
\pgfsetdash{}{0pt}%
\pgfpathmoveto{\pgfqpoint{-0.096111in}{2.661523in}}%
\pgfpathcurveto{\pgfqpoint{-0.086903in}{2.661523in}}{\pgfqpoint{-0.078070in}{2.665181in}}{\pgfqpoint{-0.071559in}{2.671693in}}%
\pgfpathcurveto{\pgfqpoint{-0.065048in}{2.678204in}}{\pgfqpoint{-0.061389in}{2.687037in}}{\pgfqpoint{-0.061389in}{2.696245in}}%
\pgfpathcurveto{\pgfqpoint{-0.061389in}{2.705453in}}{\pgfqpoint{-0.065048in}{2.714286in}}{\pgfqpoint{-0.071559in}{2.720797in}}%
\pgfpathcurveto{\pgfqpoint{-0.078070in}{2.727309in}}{\pgfqpoint{-0.086903in}{2.730967in}}{\pgfqpoint{-0.096111in}{2.730967in}}%
\pgfpathcurveto{\pgfqpoint{-0.105320in}{2.730967in}}{\pgfqpoint{-0.114152in}{2.727309in}}{\pgfqpoint{-0.120664in}{2.720797in}}%
\pgfpathcurveto{\pgfqpoint{-0.127175in}{2.714286in}}{\pgfqpoint{-0.130834in}{2.705453in}}{\pgfqpoint{-0.130834in}{2.696245in}}%
\pgfpathcurveto{\pgfqpoint{-0.130834in}{2.687037in}}{\pgfqpoint{-0.127175in}{2.678204in}}{\pgfqpoint{-0.120664in}{2.671693in}}%
\pgfpathcurveto{\pgfqpoint{-0.114152in}{2.665181in}}{\pgfqpoint{-0.105320in}{2.661523in}}{\pgfqpoint{-0.096111in}{2.661523in}}%
\pgfpathlineto{\pgfqpoint{-0.096111in}{2.661523in}}%
\pgfpathclose%
\pgfusepath{stroke,fill}%
\end{pgfscope}%
\begin{pgfscope}%
\pgfpathrectangle{\pgfqpoint{1.374500in}{0.082500in}}{\pgfqpoint{2.419000in}{2.419000in}}%
\pgfusepath{clip}%
\pgfsetbuttcap%
\pgfsetroundjoin%
\definecolor{currentfill}{rgb}{0.741176,0.121569,0.003922}%
\pgfsetfillcolor{currentfill}%
\pgfsetfillopacity{0.379004}%
\pgfsetlinewidth{1.003750pt}%
\definecolor{currentstroke}{rgb}{0.741176,0.121569,0.003922}%
\pgfsetstrokecolor{currentstroke}%
\pgfsetstrokeopacity{0.379004}%
\pgfsetdash{}{0pt}%
\pgfpathmoveto{\pgfqpoint{8.039653in}{2.661523in}}%
\pgfpathcurveto{\pgfqpoint{8.048861in}{2.661523in}}{\pgfqpoint{8.057694in}{2.665181in}}{\pgfqpoint{8.064205in}{2.671693in}}%
\pgfpathcurveto{\pgfqpoint{8.070716in}{2.678204in}}{\pgfqpoint{8.074375in}{2.687037in}}{\pgfqpoint{8.074375in}{2.696245in}}%
\pgfpathcurveto{\pgfqpoint{8.074375in}{2.705453in}}{\pgfqpoint{8.070716in}{2.714286in}}{\pgfqpoint{8.064205in}{2.720797in}}%
\pgfpathcurveto{\pgfqpoint{8.057694in}{2.727309in}}{\pgfqpoint{8.048861in}{2.730967in}}{\pgfqpoint{8.039653in}{2.730967in}}%
\pgfpathcurveto{\pgfqpoint{8.030444in}{2.730967in}}{\pgfqpoint{8.021612in}{2.727309in}}{\pgfqpoint{8.015100in}{2.720797in}}%
\pgfpathcurveto{\pgfqpoint{8.008589in}{2.714286in}}{\pgfqpoint{8.004930in}{2.705453in}}{\pgfqpoint{8.004930in}{2.696245in}}%
\pgfpathcurveto{\pgfqpoint{8.004930in}{2.687037in}}{\pgfqpoint{8.008589in}{2.678204in}}{\pgfqpoint{8.015100in}{2.671693in}}%
\pgfpathcurveto{\pgfqpoint{8.021612in}{2.665181in}}{\pgfqpoint{8.030444in}{2.661523in}}{\pgfqpoint{8.039653in}{2.661523in}}%
\pgfpathlineto{\pgfqpoint{8.039653in}{2.661523in}}%
\pgfpathclose%
\pgfusepath{stroke,fill}%
\end{pgfscope}%
\begin{pgfscope}%
\pgfpathrectangle{\pgfqpoint{1.374500in}{0.082500in}}{\pgfqpoint{2.419000in}{2.419000in}}%
\pgfusepath{clip}%
\pgfsetbuttcap%
\pgfsetroundjoin%
\definecolor{currentfill}{rgb}{0.741176,0.121569,0.003922}%
\pgfsetfillcolor{currentfill}%
\pgfsetfillopacity{0.383514}%
\pgfsetlinewidth{1.003750pt}%
\definecolor{currentstroke}{rgb}{0.741176,0.121569,0.003922}%
\pgfsetstrokecolor{currentstroke}%
\pgfsetstrokeopacity{0.383514}%
\pgfsetdash{}{0pt}%
\pgfpathmoveto{\pgfqpoint{2.314791in}{2.570101in}}%
\pgfpathcurveto{\pgfqpoint{2.324000in}{2.570101in}}{\pgfqpoint{2.332832in}{2.573759in}}{\pgfqpoint{2.339343in}{2.580271in}}%
\pgfpathcurveto{\pgfqpoint{2.345855in}{2.586782in}}{\pgfqpoint{2.349513in}{2.595614in}}{\pgfqpoint{2.349513in}{2.604823in}}%
\pgfpathcurveto{\pgfqpoint{2.349513in}{2.614031in}}{\pgfqpoint{2.345855in}{2.622864in}}{\pgfqpoint{2.339343in}{2.629375in}}%
\pgfpathcurveto{\pgfqpoint{2.332832in}{2.635887in}}{\pgfqpoint{2.324000in}{2.639545in}}{\pgfqpoint{2.314791in}{2.639545in}}%
\pgfpathcurveto{\pgfqpoint{2.305583in}{2.639545in}}{\pgfqpoint{2.296750in}{2.635887in}}{\pgfqpoint{2.290239in}{2.629375in}}%
\pgfpathcurveto{\pgfqpoint{2.283727in}{2.622864in}}{\pgfqpoint{2.280069in}{2.614031in}}{\pgfqpoint{2.280069in}{2.604823in}}%
\pgfpathcurveto{\pgfqpoint{2.280069in}{2.595614in}}{\pgfqpoint{2.283727in}{2.586782in}}{\pgfqpoint{2.290239in}{2.580271in}}%
\pgfpathcurveto{\pgfqpoint{2.296750in}{2.573759in}}{\pgfqpoint{2.305583in}{2.570101in}}{\pgfqpoint{2.314791in}{2.570101in}}%
\pgfpathlineto{\pgfqpoint{2.314791in}{2.570101in}}%
\pgfpathclose%
\pgfusepath{stroke,fill}%
\end{pgfscope}%
\begin{pgfscope}%
\pgfpathrectangle{\pgfqpoint{1.374500in}{0.082500in}}{\pgfqpoint{2.419000in}{2.419000in}}%
\pgfusepath{clip}%
\pgfsetbuttcap%
\pgfsetroundjoin%
\definecolor{currentfill}{rgb}{0.741176,0.121569,0.003922}%
\pgfsetfillcolor{currentfill}%
\pgfsetfillopacity{0.383514}%
\pgfsetlinewidth{1.003750pt}%
\definecolor{currentstroke}{rgb}{0.741176,0.121569,0.003922}%
\pgfsetstrokecolor{currentstroke}%
\pgfsetstrokeopacity{0.383514}%
\pgfsetdash{}{0pt}%
\pgfpathmoveto{\pgfqpoint{10.570954in}{2.570101in}}%
\pgfpathcurveto{\pgfqpoint{10.580163in}{2.570101in}}{\pgfqpoint{10.588995in}{2.573759in}}{\pgfqpoint{10.595507in}{2.580271in}}%
\pgfpathcurveto{\pgfqpoint{10.602018in}{2.586782in}}{\pgfqpoint{10.605677in}{2.595614in}}{\pgfqpoint{10.605677in}{2.604823in}}%
\pgfpathcurveto{\pgfqpoint{10.605677in}{2.614031in}}{\pgfqpoint{10.602018in}{2.622864in}}{\pgfqpoint{10.595507in}{2.629375in}}%
\pgfpathcurveto{\pgfqpoint{10.588995in}{2.635887in}}{\pgfqpoint{10.580163in}{2.639545in}}{\pgfqpoint{10.570954in}{2.639545in}}%
\pgfpathcurveto{\pgfqpoint{10.561746in}{2.639545in}}{\pgfqpoint{10.552913in}{2.635887in}}{\pgfqpoint{10.546402in}{2.629375in}}%
\pgfpathcurveto{\pgfqpoint{10.539891in}{2.622864in}}{\pgfqpoint{10.536232in}{2.614031in}}{\pgfqpoint{10.536232in}{2.604823in}}%
\pgfpathcurveto{\pgfqpoint{10.536232in}{2.595614in}}{\pgfqpoint{10.539891in}{2.586782in}}{\pgfqpoint{10.546402in}{2.580271in}}%
\pgfpathcurveto{\pgfqpoint{10.552913in}{2.573759in}}{\pgfqpoint{10.561746in}{2.570101in}}{\pgfqpoint{10.570954in}{2.570101in}}%
\pgfpathlineto{\pgfqpoint{10.570954in}{2.570101in}}%
\pgfpathclose%
\pgfusepath{stroke,fill}%
\end{pgfscope}%
\begin{pgfscope}%
\pgfpathrectangle{\pgfqpoint{1.374500in}{0.082500in}}{\pgfqpoint{2.419000in}{2.419000in}}%
\pgfusepath{clip}%
\pgfsetbuttcap%
\pgfsetroundjoin%
\definecolor{currentfill}{rgb}{0.741176,0.121569,0.003922}%
\pgfsetfillcolor{currentfill}%
\pgfsetfillopacity{0.383514}%
\pgfsetlinewidth{1.003750pt}%
\definecolor{currentstroke}{rgb}{0.741176,0.121569,0.003922}%
\pgfsetstrokecolor{currentstroke}%
\pgfsetstrokeopacity{0.383514}%
\pgfsetdash{}{0pt}%
\pgfpathmoveto{\pgfqpoint{6.442873in}{2.570101in}}%
\pgfpathcurveto{\pgfqpoint{6.452081in}{2.570101in}}{\pgfqpoint{6.460914in}{2.573759in}}{\pgfqpoint{6.467425in}{2.580271in}}%
\pgfpathcurveto{\pgfqpoint{6.473936in}{2.586782in}}{\pgfqpoint{6.477595in}{2.595614in}}{\pgfqpoint{6.477595in}{2.604823in}}%
\pgfpathcurveto{\pgfqpoint{6.477595in}{2.614031in}}{\pgfqpoint{6.473936in}{2.622864in}}{\pgfqpoint{6.467425in}{2.629375in}}%
\pgfpathcurveto{\pgfqpoint{6.460914in}{2.635887in}}{\pgfqpoint{6.452081in}{2.639545in}}{\pgfqpoint{6.442873in}{2.639545in}}%
\pgfpathcurveto{\pgfqpoint{6.433664in}{2.639545in}}{\pgfqpoint{6.424832in}{2.635887in}}{\pgfqpoint{6.418320in}{2.629375in}}%
\pgfpathcurveto{\pgfqpoint{6.411809in}{2.622864in}}{\pgfqpoint{6.408150in}{2.614031in}}{\pgfqpoint{6.408150in}{2.604823in}}%
\pgfpathcurveto{\pgfqpoint{6.408150in}{2.595614in}}{\pgfqpoint{6.411809in}{2.586782in}}{\pgfqpoint{6.418320in}{2.580271in}}%
\pgfpathcurveto{\pgfqpoint{6.424832in}{2.573759in}}{\pgfqpoint{6.433664in}{2.570101in}}{\pgfqpoint{6.442873in}{2.570101in}}%
\pgfpathlineto{\pgfqpoint{6.442873in}{2.570101in}}%
\pgfpathclose%
\pgfusepath{stroke,fill}%
\end{pgfscope}%
\begin{pgfscope}%
\pgfpathrectangle{\pgfqpoint{1.374500in}{0.082500in}}{\pgfqpoint{2.419000in}{2.419000in}}%
\pgfusepath{clip}%
\pgfsetbuttcap%
\pgfsetroundjoin%
\definecolor{currentfill}{rgb}{0.741176,0.121569,0.003922}%
\pgfsetfillcolor{currentfill}%
\pgfsetfillopacity{0.388159}%
\pgfsetlinewidth{1.003750pt}%
\definecolor{currentstroke}{rgb}{0.741176,0.121569,0.003922}%
\pgfsetstrokecolor{currentstroke}%
\pgfsetstrokeopacity{0.388159}%
\pgfsetdash{}{0pt}%
\pgfpathmoveto{\pgfqpoint{4.798122in}{2.475932in}}%
\pgfpathcurveto{\pgfqpoint{4.807330in}{2.475932in}}{\pgfqpoint{4.816163in}{2.479591in}}{\pgfqpoint{4.822674in}{2.486102in}}%
\pgfpathcurveto{\pgfqpoint{4.829186in}{2.492613in}}{\pgfqpoint{4.832844in}{2.501446in}}{\pgfqpoint{4.832844in}{2.510654in}}%
\pgfpathcurveto{\pgfqpoint{4.832844in}{2.519863in}}{\pgfqpoint{4.829186in}{2.528695in}}{\pgfqpoint{4.822674in}{2.535207in}}%
\pgfpathcurveto{\pgfqpoint{4.816163in}{2.541718in}}{\pgfqpoint{4.807330in}{2.545376in}}{\pgfqpoint{4.798122in}{2.545376in}}%
\pgfpathcurveto{\pgfqpoint{4.788914in}{2.545376in}}{\pgfqpoint{4.780081in}{2.541718in}}{\pgfqpoint{4.773570in}{2.535207in}}%
\pgfpathcurveto{\pgfqpoint{4.767058in}{2.528695in}}{\pgfqpoint{4.763400in}{2.519863in}}{\pgfqpoint{4.763400in}{2.510654in}}%
\pgfpathcurveto{\pgfqpoint{4.763400in}{2.501446in}}{\pgfqpoint{4.767058in}{2.492613in}}{\pgfqpoint{4.773570in}{2.486102in}}%
\pgfpathcurveto{\pgfqpoint{4.780081in}{2.479591in}}{\pgfqpoint{4.788914in}{2.475932in}}{\pgfqpoint{4.798122in}{2.475932in}}%
\pgfpathlineto{\pgfqpoint{4.798122in}{2.475932in}}%
\pgfpathclose%
\pgfusepath{stroke,fill}%
\end{pgfscope}%
\begin{pgfscope}%
\pgfpathrectangle{\pgfqpoint{1.374500in}{0.082500in}}{\pgfqpoint{2.419000in}{2.419000in}}%
\pgfusepath{clip}%
\pgfsetbuttcap%
\pgfsetroundjoin%
\definecolor{currentfill}{rgb}{0.741176,0.121569,0.003922}%
\pgfsetfillcolor{currentfill}%
\pgfsetfillopacity{0.388159}%
\pgfsetlinewidth{1.003750pt}%
\definecolor{currentstroke}{rgb}{0.741176,0.121569,0.003922}%
\pgfsetstrokecolor{currentstroke}%
\pgfsetstrokeopacity{0.388159}%
\pgfsetdash{}{0pt}%
\pgfpathmoveto{\pgfqpoint{0.608032in}{2.475932in}}%
\pgfpathcurveto{\pgfqpoint{0.617241in}{2.475932in}}{\pgfqpoint{0.626073in}{2.479591in}}{\pgfqpoint{0.632585in}{2.486102in}}%
\pgfpathcurveto{\pgfqpoint{0.639096in}{2.492613in}}{\pgfqpoint{0.642755in}{2.501446in}}{\pgfqpoint{0.642755in}{2.510654in}}%
\pgfpathcurveto{\pgfqpoint{0.642755in}{2.519863in}}{\pgfqpoint{0.639096in}{2.528695in}}{\pgfqpoint{0.632585in}{2.535207in}}%
\pgfpathcurveto{\pgfqpoint{0.626073in}{2.541718in}}{\pgfqpoint{0.617241in}{2.545376in}}{\pgfqpoint{0.608032in}{2.545376in}}%
\pgfpathcurveto{\pgfqpoint{0.598824in}{2.545376in}}{\pgfqpoint{0.589991in}{2.541718in}}{\pgfqpoint{0.583480in}{2.535207in}}%
\pgfpathcurveto{\pgfqpoint{0.576969in}{2.528695in}}{\pgfqpoint{0.573310in}{2.519863in}}{\pgfqpoint{0.573310in}{2.510654in}}%
\pgfpathcurveto{\pgfqpoint{0.573310in}{2.501446in}}{\pgfqpoint{0.576969in}{2.492613in}}{\pgfqpoint{0.583480in}{2.486102in}}%
\pgfpathcurveto{\pgfqpoint{0.589991in}{2.479591in}}{\pgfqpoint{0.598824in}{2.475932in}}{\pgfqpoint{0.608032in}{2.475932in}}%
\pgfpathlineto{\pgfqpoint{0.608032in}{2.475932in}}%
\pgfpathclose%
\pgfusepath{stroke,fill}%
\end{pgfscope}%
\begin{pgfscope}%
\pgfpathrectangle{\pgfqpoint{1.374500in}{0.082500in}}{\pgfqpoint{2.419000in}{2.419000in}}%
\pgfusepath{clip}%
\pgfsetbuttcap%
\pgfsetroundjoin%
\definecolor{currentfill}{rgb}{0.741176,0.121569,0.003922}%
\pgfsetfillcolor{currentfill}%
\pgfsetfillopacity{0.388159}%
\pgfsetlinewidth{1.003750pt}%
\definecolor{currentstroke}{rgb}{0.741176,0.121569,0.003922}%
\pgfsetstrokecolor{currentstroke}%
\pgfsetstrokeopacity{0.388159}%
\pgfsetdash{}{0pt}%
\pgfpathmoveto{\pgfqpoint{8.988212in}{2.475932in}}%
\pgfpathcurveto{\pgfqpoint{8.997420in}{2.475932in}}{\pgfqpoint{9.006253in}{2.479591in}}{\pgfqpoint{9.012764in}{2.486102in}}%
\pgfpathcurveto{\pgfqpoint{9.019275in}{2.492613in}}{\pgfqpoint{9.022934in}{2.501446in}}{\pgfqpoint{9.022934in}{2.510654in}}%
\pgfpathcurveto{\pgfqpoint{9.022934in}{2.519863in}}{\pgfqpoint{9.019275in}{2.528695in}}{\pgfqpoint{9.012764in}{2.535207in}}%
\pgfpathcurveto{\pgfqpoint{9.006253in}{2.541718in}}{\pgfqpoint{8.997420in}{2.545376in}}{\pgfqpoint{8.988212in}{2.545376in}}%
\pgfpathcurveto{\pgfqpoint{8.979003in}{2.545376in}}{\pgfqpoint{8.970171in}{2.541718in}}{\pgfqpoint{8.963659in}{2.535207in}}%
\pgfpathcurveto{\pgfqpoint{8.957148in}{2.528695in}}{\pgfqpoint{8.953490in}{2.519863in}}{\pgfqpoint{8.953490in}{2.510654in}}%
\pgfpathcurveto{\pgfqpoint{8.953490in}{2.501446in}}{\pgfqpoint{8.957148in}{2.492613in}}{\pgfqpoint{8.963659in}{2.486102in}}%
\pgfpathcurveto{\pgfqpoint{8.970171in}{2.479591in}}{\pgfqpoint{8.979003in}{2.475932in}}{\pgfqpoint{8.988212in}{2.475932in}}%
\pgfpathlineto{\pgfqpoint{8.988212in}{2.475932in}}%
\pgfpathclose%
\pgfusepath{stroke,fill}%
\end{pgfscope}%
\begin{pgfscope}%
\pgfpathrectangle{\pgfqpoint{1.374500in}{0.082500in}}{\pgfqpoint{2.419000in}{2.419000in}}%
\pgfusepath{clip}%
\pgfsetbuttcap%
\pgfsetroundjoin%
\definecolor{currentfill}{rgb}{0.741176,0.121569,0.003922}%
\pgfsetfillcolor{currentfill}%
\pgfsetfillopacity{0.392946}%
\pgfsetlinewidth{1.003750pt}%
\definecolor{currentstroke}{rgb}{0.741176,0.121569,0.003922}%
\pgfsetstrokecolor{currentstroke}%
\pgfsetstrokeopacity{0.392946}%
\pgfsetdash{}{0pt}%
\pgfpathmoveto{\pgfqpoint{3.103206in}{2.378891in}}%
\pgfpathcurveto{\pgfqpoint{3.112415in}{2.378891in}}{\pgfqpoint{3.121247in}{2.382550in}}{\pgfqpoint{3.127758in}{2.389061in}}%
\pgfpathcurveto{\pgfqpoint{3.134270in}{2.395572in}}{\pgfqpoint{3.137928in}{2.404405in}}{\pgfqpoint{3.137928in}{2.413613in}}%
\pgfpathcurveto{\pgfqpoint{3.137928in}{2.422822in}}{\pgfqpoint{3.134270in}{2.431654in}}{\pgfqpoint{3.127758in}{2.438166in}}%
\pgfpathcurveto{\pgfqpoint{3.121247in}{2.444677in}}{\pgfqpoint{3.112415in}{2.448336in}}{\pgfqpoint{3.103206in}{2.448336in}}%
\pgfpathcurveto{\pgfqpoint{3.093998in}{2.448336in}}{\pgfqpoint{3.085165in}{2.444677in}}{\pgfqpoint{3.078654in}{2.438166in}}%
\pgfpathcurveto{\pgfqpoint{3.072142in}{2.431654in}}{\pgfqpoint{3.068484in}{2.422822in}}{\pgfqpoint{3.068484in}{2.413613in}}%
\pgfpathcurveto{\pgfqpoint{3.068484in}{2.404405in}}{\pgfqpoint{3.072142in}{2.395572in}}{\pgfqpoint{3.078654in}{2.389061in}}%
\pgfpathcurveto{\pgfqpoint{3.085165in}{2.382550in}}{\pgfqpoint{3.093998in}{2.378891in}}{\pgfqpoint{3.103206in}{2.378891in}}%
\pgfpathlineto{\pgfqpoint{3.103206in}{2.378891in}}%
\pgfpathclose%
\pgfusepath{stroke,fill}%
\end{pgfscope}%
\begin{pgfscope}%
\pgfpathrectangle{\pgfqpoint{1.374500in}{0.082500in}}{\pgfqpoint{2.419000in}{2.419000in}}%
\pgfusepath{clip}%
\pgfsetbuttcap%
\pgfsetroundjoin%
\definecolor{currentfill}{rgb}{0.741176,0.121569,0.003922}%
\pgfsetfillcolor{currentfill}%
\pgfsetfillopacity{0.392946}%
\pgfsetlinewidth{1.003750pt}%
\definecolor{currentstroke}{rgb}{0.741176,0.121569,0.003922}%
\pgfsetstrokecolor{currentstroke}%
\pgfsetstrokeopacity{0.392946}%
\pgfsetdash{}{0pt}%
\pgfpathmoveto{\pgfqpoint{11.611184in}{2.378891in}}%
\pgfpathcurveto{\pgfqpoint{11.620393in}{2.378891in}}{\pgfqpoint{11.629225in}{2.382550in}}{\pgfqpoint{11.635737in}{2.389061in}}%
\pgfpathcurveto{\pgfqpoint{11.642248in}{2.395572in}}{\pgfqpoint{11.645907in}{2.404405in}}{\pgfqpoint{11.645907in}{2.413613in}}%
\pgfpathcurveto{\pgfqpoint{11.645907in}{2.422822in}}{\pgfqpoint{11.642248in}{2.431654in}}{\pgfqpoint{11.635737in}{2.438166in}}%
\pgfpathcurveto{\pgfqpoint{11.629225in}{2.444677in}}{\pgfqpoint{11.620393in}{2.448336in}}{\pgfqpoint{11.611184in}{2.448336in}}%
\pgfpathcurveto{\pgfqpoint{11.601976in}{2.448336in}}{\pgfqpoint{11.593143in}{2.444677in}}{\pgfqpoint{11.586632in}{2.438166in}}%
\pgfpathcurveto{\pgfqpoint{11.580121in}{2.431654in}}{\pgfqpoint{11.576462in}{2.422822in}}{\pgfqpoint{11.576462in}{2.413613in}}%
\pgfpathcurveto{\pgfqpoint{11.576462in}{2.404405in}}{\pgfqpoint{11.580121in}{2.395572in}}{\pgfqpoint{11.586632in}{2.389061in}}%
\pgfpathcurveto{\pgfqpoint{11.593143in}{2.382550in}}{\pgfqpoint{11.601976in}{2.378891in}}{\pgfqpoint{11.611184in}{2.378891in}}%
\pgfpathlineto{\pgfqpoint{11.611184in}{2.378891in}}%
\pgfpathclose%
\pgfusepath{stroke,fill}%
\end{pgfscope}%
\begin{pgfscope}%
\pgfpathrectangle{\pgfqpoint{1.374500in}{0.082500in}}{\pgfqpoint{2.419000in}{2.419000in}}%
\pgfusepath{clip}%
\pgfsetbuttcap%
\pgfsetroundjoin%
\definecolor{currentfill}{rgb}{0.741176,0.121569,0.003922}%
\pgfsetfillcolor{currentfill}%
\pgfsetfillopacity{0.392946}%
\pgfsetlinewidth{1.003750pt}%
\definecolor{currentstroke}{rgb}{0.741176,0.121569,0.003922}%
\pgfsetstrokecolor{currentstroke}%
\pgfsetstrokeopacity{0.392946}%
\pgfsetdash{}{0pt}%
\pgfpathmoveto{\pgfqpoint{7.357195in}{2.378891in}}%
\pgfpathcurveto{\pgfqpoint{7.366404in}{2.378891in}}{\pgfqpoint{7.375236in}{2.382550in}}{\pgfqpoint{7.381748in}{2.389061in}}%
\pgfpathcurveto{\pgfqpoint{7.388259in}{2.395572in}}{\pgfqpoint{7.391917in}{2.404405in}}{\pgfqpoint{7.391917in}{2.413613in}}%
\pgfpathcurveto{\pgfqpoint{7.391917in}{2.422822in}}{\pgfqpoint{7.388259in}{2.431654in}}{\pgfqpoint{7.381748in}{2.438166in}}%
\pgfpathcurveto{\pgfqpoint{7.375236in}{2.444677in}}{\pgfqpoint{7.366404in}{2.448336in}}{\pgfqpoint{7.357195in}{2.448336in}}%
\pgfpathcurveto{\pgfqpoint{7.347987in}{2.448336in}}{\pgfqpoint{7.339154in}{2.444677in}}{\pgfqpoint{7.332643in}{2.438166in}}%
\pgfpathcurveto{\pgfqpoint{7.326132in}{2.431654in}}{\pgfqpoint{7.322473in}{2.422822in}}{\pgfqpoint{7.322473in}{2.413613in}}%
\pgfpathcurveto{\pgfqpoint{7.322473in}{2.404405in}}{\pgfqpoint{7.326132in}{2.395572in}}{\pgfqpoint{7.332643in}{2.389061in}}%
\pgfpathcurveto{\pgfqpoint{7.339154in}{2.382550in}}{\pgfqpoint{7.347987in}{2.378891in}}{\pgfqpoint{7.357195in}{2.378891in}}%
\pgfpathlineto{\pgfqpoint{7.357195in}{2.378891in}}%
\pgfpathclose%
\pgfusepath{stroke,fill}%
\end{pgfscope}%
\begin{pgfscope}%
\pgfpathrectangle{\pgfqpoint{1.374500in}{0.082500in}}{\pgfqpoint{2.419000in}{2.419000in}}%
\pgfusepath{clip}%
\pgfsetbuttcap%
\pgfsetroundjoin%
\definecolor{currentfill}{rgb}{0.741176,0.121569,0.003922}%
\pgfsetfillcolor{currentfill}%
\pgfsetfillopacity{0.397881}%
\pgfsetlinewidth{1.003750pt}%
\definecolor{currentstroke}{rgb}{0.741176,0.121569,0.003922}%
\pgfsetstrokecolor{currentstroke}%
\pgfsetstrokeopacity{0.397881}%
\pgfsetdash{}{0pt}%
\pgfpathmoveto{\pgfqpoint{5.675662in}{2.278845in}}%
\pgfpathcurveto{\pgfqpoint{5.684870in}{2.278845in}}{\pgfqpoint{5.693703in}{2.282503in}}{\pgfqpoint{5.700214in}{2.289015in}}%
\pgfpathcurveto{\pgfqpoint{5.706726in}{2.295526in}}{\pgfqpoint{5.710384in}{2.304358in}}{\pgfqpoint{5.710384in}{2.313567in}}%
\pgfpathcurveto{\pgfqpoint{5.710384in}{2.322775in}}{\pgfqpoint{5.706726in}{2.331608in}}{\pgfqpoint{5.700214in}{2.338119in}}%
\pgfpathcurveto{\pgfqpoint{5.693703in}{2.344631in}}{\pgfqpoint{5.684870in}{2.348289in}}{\pgfqpoint{5.675662in}{2.348289in}}%
\pgfpathcurveto{\pgfqpoint{5.666453in}{2.348289in}}{\pgfqpoint{5.657621in}{2.344631in}}{\pgfqpoint{5.651110in}{2.338119in}}%
\pgfpathcurveto{\pgfqpoint{5.644598in}{2.331608in}}{\pgfqpoint{5.640940in}{2.322775in}}{\pgfqpoint{5.640940in}{2.313567in}}%
\pgfpathcurveto{\pgfqpoint{5.640940in}{2.304358in}}{\pgfqpoint{5.644598in}{2.295526in}}{\pgfqpoint{5.651110in}{2.289015in}}%
\pgfpathcurveto{\pgfqpoint{5.657621in}{2.282503in}}{\pgfqpoint{5.666453in}{2.278845in}}{\pgfqpoint{5.675662in}{2.278845in}}%
\pgfpathlineto{\pgfqpoint{5.675662in}{2.278845in}}%
\pgfpathclose%
\pgfusepath{stroke,fill}%
\end{pgfscope}%
\begin{pgfscope}%
\pgfpathrectangle{\pgfqpoint{1.374500in}{0.082500in}}{\pgfqpoint{2.419000in}{2.419000in}}%
\pgfusepath{clip}%
\pgfsetbuttcap%
\pgfsetroundjoin%
\definecolor{currentfill}{rgb}{0.741176,0.121569,0.003922}%
\pgfsetfillcolor{currentfill}%
\pgfsetfillopacity{0.397881}%
\pgfsetlinewidth{1.003750pt}%
\definecolor{currentstroke}{rgb}{0.741176,0.121569,0.003922}%
\pgfsetstrokecolor{currentstroke}%
\pgfsetstrokeopacity{0.397881}%
\pgfsetdash{}{0pt}%
\pgfpathmoveto{\pgfqpoint{1.355794in}{2.278845in}}%
\pgfpathcurveto{\pgfqpoint{1.365003in}{2.278845in}}{\pgfqpoint{1.373835in}{2.282503in}}{\pgfqpoint{1.380346in}{2.289015in}}%
\pgfpathcurveto{\pgfqpoint{1.386858in}{2.295526in}}{\pgfqpoint{1.390516in}{2.304358in}}{\pgfqpoint{1.390516in}{2.313567in}}%
\pgfpathcurveto{\pgfqpoint{1.390516in}{2.322775in}}{\pgfqpoint{1.386858in}{2.331608in}}{\pgfqpoint{1.380346in}{2.338119in}}%
\pgfpathcurveto{\pgfqpoint{1.373835in}{2.344631in}}{\pgfqpoint{1.365003in}{2.348289in}}{\pgfqpoint{1.355794in}{2.348289in}}%
\pgfpathcurveto{\pgfqpoint{1.346586in}{2.348289in}}{\pgfqpoint{1.337753in}{2.344631in}}{\pgfqpoint{1.331242in}{2.338119in}}%
\pgfpathcurveto{\pgfqpoint{1.324731in}{2.331608in}}{\pgfqpoint{1.321072in}{2.322775in}}{\pgfqpoint{1.321072in}{2.313567in}}%
\pgfpathcurveto{\pgfqpoint{1.321072in}{2.304358in}}{\pgfqpoint{1.324731in}{2.295526in}}{\pgfqpoint{1.331242in}{2.289015in}}%
\pgfpathcurveto{\pgfqpoint{1.337753in}{2.282503in}}{\pgfqpoint{1.346586in}{2.278845in}}{\pgfqpoint{1.355794in}{2.278845in}}%
\pgfpathlineto{\pgfqpoint{1.355794in}{2.278845in}}%
\pgfpathclose%
\pgfusepath{stroke,fill}%
\end{pgfscope}%
\begin{pgfscope}%
\pgfpathrectangle{\pgfqpoint{1.374500in}{0.082500in}}{\pgfqpoint{2.419000in}{2.419000in}}%
\pgfusepath{clip}%
\pgfsetbuttcap%
\pgfsetroundjoin%
\definecolor{currentfill}{rgb}{0.741176,0.121569,0.003922}%
\pgfsetfillcolor{currentfill}%
\pgfsetfillopacity{0.397881}%
\pgfsetlinewidth{1.003750pt}%
\definecolor{currentstroke}{rgb}{0.741176,0.121569,0.003922}%
\pgfsetstrokecolor{currentstroke}%
\pgfsetstrokeopacity{0.397881}%
\pgfsetdash{}{0pt}%
\pgfpathmoveto{\pgfqpoint{9.995530in}{2.278845in}}%
\pgfpathcurveto{\pgfqpoint{10.004738in}{2.278845in}}{\pgfqpoint{10.013571in}{2.282503in}}{\pgfqpoint{10.020082in}{2.289015in}}%
\pgfpathcurveto{\pgfqpoint{10.026593in}{2.295526in}}{\pgfqpoint{10.030252in}{2.304358in}}{\pgfqpoint{10.030252in}{2.313567in}}%
\pgfpathcurveto{\pgfqpoint{10.030252in}{2.322775in}}{\pgfqpoint{10.026593in}{2.331608in}}{\pgfqpoint{10.020082in}{2.338119in}}%
\pgfpathcurveto{\pgfqpoint{10.013571in}{2.344631in}}{\pgfqpoint{10.004738in}{2.348289in}}{\pgfqpoint{9.995530in}{2.348289in}}%
\pgfpathcurveto{\pgfqpoint{9.986321in}{2.348289in}}{\pgfqpoint{9.977489in}{2.344631in}}{\pgfqpoint{9.970977in}{2.338119in}}%
\pgfpathcurveto{\pgfqpoint{9.964466in}{2.331608in}}{\pgfqpoint{9.960807in}{2.322775in}}{\pgfqpoint{9.960807in}{2.313567in}}%
\pgfpathcurveto{\pgfqpoint{9.960807in}{2.304358in}}{\pgfqpoint{9.964466in}{2.295526in}}{\pgfqpoint{9.970977in}{2.289015in}}%
\pgfpathcurveto{\pgfqpoint{9.977489in}{2.282503in}}{\pgfqpoint{9.986321in}{2.278845in}}{\pgfqpoint{9.995530in}{2.278845in}}%
\pgfpathlineto{\pgfqpoint{9.995530in}{2.278845in}}%
\pgfpathclose%
\pgfusepath{stroke,fill}%
\end{pgfscope}%
\begin{pgfscope}%
\pgfpathrectangle{\pgfqpoint{1.374500in}{0.082500in}}{\pgfqpoint{2.419000in}{2.419000in}}%
\pgfusepath{clip}%
\pgfsetbuttcap%
\pgfsetroundjoin%
\definecolor{currentfill}{rgb}{0.741176,0.121569,0.003922}%
\pgfsetfillcolor{currentfill}%
\pgfsetfillopacity{0.402972}%
\pgfsetlinewidth{1.003750pt}%
\definecolor{currentstroke}{rgb}{0.741176,0.121569,0.003922}%
\pgfsetstrokecolor{currentstroke}%
\pgfsetstrokeopacity{0.402972}%
\pgfsetdash{}{0pt}%
\pgfpathmoveto{\pgfqpoint{-0.446591in}{2.175651in}}%
\pgfpathcurveto{\pgfqpoint{-0.437383in}{2.175651in}}{\pgfqpoint{-0.428550in}{2.179309in}}{\pgfqpoint{-0.422039in}{2.185821in}}%
\pgfpathcurveto{\pgfqpoint{-0.415527in}{2.192332in}}{\pgfqpoint{-0.411869in}{2.201165in}}{\pgfqpoint{-0.411869in}{2.210373in}}%
\pgfpathcurveto{\pgfqpoint{-0.411869in}{2.219581in}}{\pgfqpoint{-0.415527in}{2.228414in}}{\pgfqpoint{-0.422039in}{2.234925in}}%
\pgfpathcurveto{\pgfqpoint{-0.428550in}{2.241437in}}{\pgfqpoint{-0.437383in}{2.245095in}}{\pgfqpoint{-0.446591in}{2.245095in}}%
\pgfpathcurveto{\pgfqpoint{-0.455799in}{2.245095in}}{\pgfqpoint{-0.464632in}{2.241437in}}{\pgfqpoint{-0.471143in}{2.234925in}}%
\pgfpathcurveto{\pgfqpoint{-0.477655in}{2.228414in}}{\pgfqpoint{-0.481313in}{2.219581in}}{\pgfqpoint{-0.481313in}{2.210373in}}%
\pgfpathcurveto{\pgfqpoint{-0.481313in}{2.201165in}}{\pgfqpoint{-0.477655in}{2.192332in}}{\pgfqpoint{-0.471143in}{2.185821in}}%
\pgfpathcurveto{\pgfqpoint{-0.464632in}{2.179309in}}{\pgfqpoint{-0.455799in}{2.175651in}}{\pgfqpoint{-0.446591in}{2.175651in}}%
\pgfpathlineto{\pgfqpoint{-0.446591in}{2.175651in}}%
\pgfpathclose%
\pgfusepath{stroke,fill}%
\end{pgfscope}%
\begin{pgfscope}%
\pgfpathrectangle{\pgfqpoint{1.374500in}{0.082500in}}{\pgfqpoint{2.419000in}{2.419000in}}%
\pgfusepath{clip}%
\pgfsetbuttcap%
\pgfsetroundjoin%
\definecolor{currentfill}{rgb}{0.741176,0.121569,0.003922}%
\pgfsetfillcolor{currentfill}%
\pgfsetfillopacity{0.402972}%
\pgfsetlinewidth{1.003750pt}%
\definecolor{currentstroke}{rgb}{0.741176,0.121569,0.003922}%
\pgfsetstrokecolor{currentstroke}%
\pgfsetstrokeopacity{0.402972}%
\pgfsetdash{}{0pt}%
\pgfpathmoveto{\pgfqpoint{3.941228in}{2.175651in}}%
\pgfpathcurveto{\pgfqpoint{3.950436in}{2.175651in}}{\pgfqpoint{3.959269in}{2.179309in}}{\pgfqpoint{3.965780in}{2.185821in}}%
\pgfpathcurveto{\pgfqpoint{3.972291in}{2.192332in}}{\pgfqpoint{3.975950in}{2.201165in}}{\pgfqpoint{3.975950in}{2.210373in}}%
\pgfpathcurveto{\pgfqpoint{3.975950in}{2.219581in}}{\pgfqpoint{3.972291in}{2.228414in}}{\pgfqpoint{3.965780in}{2.234925in}}%
\pgfpathcurveto{\pgfqpoint{3.959269in}{2.241437in}}{\pgfqpoint{3.950436in}{2.245095in}}{\pgfqpoint{3.941228in}{2.245095in}}%
\pgfpathcurveto{\pgfqpoint{3.932019in}{2.245095in}}{\pgfqpoint{3.923187in}{2.241437in}}{\pgfqpoint{3.916676in}{2.234925in}}%
\pgfpathcurveto{\pgfqpoint{3.910164in}{2.228414in}}{\pgfqpoint{3.906506in}{2.219581in}}{\pgfqpoint{3.906506in}{2.210373in}}%
\pgfpathcurveto{\pgfqpoint{3.906506in}{2.201165in}}{\pgfqpoint{3.910164in}{2.192332in}}{\pgfqpoint{3.916676in}{2.185821in}}%
\pgfpathcurveto{\pgfqpoint{3.923187in}{2.179309in}}{\pgfqpoint{3.932019in}{2.175651in}}{\pgfqpoint{3.941228in}{2.175651in}}%
\pgfpathlineto{\pgfqpoint{3.941228in}{2.175651in}}%
\pgfpathclose%
\pgfusepath{stroke,fill}%
\end{pgfscope}%
\begin{pgfscope}%
\pgfpathrectangle{\pgfqpoint{1.374500in}{0.082500in}}{\pgfqpoint{2.419000in}{2.419000in}}%
\pgfusepath{clip}%
\pgfsetbuttcap%
\pgfsetroundjoin%
\definecolor{currentfill}{rgb}{0.741176,0.121569,0.003922}%
\pgfsetfillcolor{currentfill}%
\pgfsetfillopacity{0.402972}%
\pgfsetlinewidth{1.003750pt}%
\definecolor{currentstroke}{rgb}{0.741176,0.121569,0.003922}%
\pgfsetstrokecolor{currentstroke}%
\pgfsetstrokeopacity{0.402972}%
\pgfsetdash{}{0pt}%
\pgfpathmoveto{\pgfqpoint{8.329047in}{2.175651in}}%
\pgfpathcurveto{\pgfqpoint{8.338255in}{2.175651in}}{\pgfqpoint{8.347088in}{2.179309in}}{\pgfqpoint{8.353599in}{2.185821in}}%
\pgfpathcurveto{\pgfqpoint{8.360110in}{2.192332in}}{\pgfqpoint{8.363769in}{2.201165in}}{\pgfqpoint{8.363769in}{2.210373in}}%
\pgfpathcurveto{\pgfqpoint{8.363769in}{2.219581in}}{\pgfqpoint{8.360110in}{2.228414in}}{\pgfqpoint{8.353599in}{2.234925in}}%
\pgfpathcurveto{\pgfqpoint{8.347088in}{2.241437in}}{\pgfqpoint{8.338255in}{2.245095in}}{\pgfqpoint{8.329047in}{2.245095in}}%
\pgfpathcurveto{\pgfqpoint{8.319838in}{2.245095in}}{\pgfqpoint{8.311006in}{2.241437in}}{\pgfqpoint{8.304494in}{2.234925in}}%
\pgfpathcurveto{\pgfqpoint{8.297983in}{2.228414in}}{\pgfqpoint{8.294324in}{2.219581in}}{\pgfqpoint{8.294324in}{2.210373in}}%
\pgfpathcurveto{\pgfqpoint{8.294324in}{2.201165in}}{\pgfqpoint{8.297983in}{2.192332in}}{\pgfqpoint{8.304494in}{2.185821in}}%
\pgfpathcurveto{\pgfqpoint{8.311006in}{2.179309in}}{\pgfqpoint{8.319838in}{2.175651in}}{\pgfqpoint{8.329047in}{2.175651in}}%
\pgfpathlineto{\pgfqpoint{8.329047in}{2.175651in}}%
\pgfpathclose%
\pgfusepath{stroke,fill}%
\end{pgfscope}%
\begin{pgfscope}%
\pgfpathrectangle{\pgfqpoint{1.374500in}{0.082500in}}{\pgfqpoint{2.419000in}{2.419000in}}%
\pgfusepath{clip}%
\pgfsetbuttcap%
\pgfsetroundjoin%
\definecolor{currentfill}{rgb}{0.741176,0.121569,0.003922}%
\pgfsetfillcolor{currentfill}%
\pgfsetfillopacity{0.408225}%
\pgfsetlinewidth{1.003750pt}%
\definecolor{currentstroke}{rgb}{0.741176,0.121569,0.003922}%
\pgfsetstrokecolor{currentstroke}%
\pgfsetstrokeopacity{0.408225}%
\pgfsetdash{}{0pt}%
\pgfpathmoveto{\pgfqpoint{2.151357in}{2.069159in}}%
\pgfpathcurveto{\pgfqpoint{2.160565in}{2.069159in}}{\pgfqpoint{2.169398in}{2.072817in}}{\pgfqpoint{2.175909in}{2.079328in}}%
\pgfpathcurveto{\pgfqpoint{2.182420in}{2.085840in}}{\pgfqpoint{2.186079in}{2.094672in}}{\pgfqpoint{2.186079in}{2.103881in}}%
\pgfpathcurveto{\pgfqpoint{2.186079in}{2.113089in}}{\pgfqpoint{2.182420in}{2.121922in}}{\pgfqpoint{2.175909in}{2.128433in}}%
\pgfpathcurveto{\pgfqpoint{2.169398in}{2.134944in}}{\pgfqpoint{2.160565in}{2.138603in}}{\pgfqpoint{2.151357in}{2.138603in}}%
\pgfpathcurveto{\pgfqpoint{2.142148in}{2.138603in}}{\pgfqpoint{2.133316in}{2.134944in}}{\pgfqpoint{2.126804in}{2.128433in}}%
\pgfpathcurveto{\pgfqpoint{2.120293in}{2.121922in}}{\pgfqpoint{2.116635in}{2.113089in}}{\pgfqpoint{2.116635in}{2.103881in}}%
\pgfpathcurveto{\pgfqpoint{2.116635in}{2.094672in}}{\pgfqpoint{2.120293in}{2.085840in}}{\pgfqpoint{2.126804in}{2.079328in}}%
\pgfpathcurveto{\pgfqpoint{2.133316in}{2.072817in}}{\pgfqpoint{2.142148in}{2.069159in}}{\pgfqpoint{2.151357in}{2.069159in}}%
\pgfpathlineto{\pgfqpoint{2.151357in}{2.069159in}}%
\pgfpathclose%
\pgfusepath{stroke,fill}%
\end{pgfscope}%
\begin{pgfscope}%
\pgfpathrectangle{\pgfqpoint{1.374500in}{0.082500in}}{\pgfqpoint{2.419000in}{2.419000in}}%
\pgfusepath{clip}%
\pgfsetbuttcap%
\pgfsetroundjoin%
\definecolor{currentfill}{rgb}{0.741176,0.121569,0.003922}%
\pgfsetfillcolor{currentfill}%
\pgfsetfillopacity{0.408225}%
\pgfsetlinewidth{1.003750pt}%
\definecolor{currentstroke}{rgb}{0.741176,0.121569,0.003922}%
\pgfsetstrokecolor{currentstroke}%
\pgfsetstrokeopacity{0.408225}%
\pgfsetdash{}{0pt}%
\pgfpathmoveto{\pgfqpoint{6.609299in}{2.069159in}}%
\pgfpathcurveto{\pgfqpoint{6.618507in}{2.069159in}}{\pgfqpoint{6.627340in}{2.072817in}}{\pgfqpoint{6.633851in}{2.079328in}}%
\pgfpathcurveto{\pgfqpoint{6.640362in}{2.085840in}}{\pgfqpoint{6.644021in}{2.094672in}}{\pgfqpoint{6.644021in}{2.103881in}}%
\pgfpathcurveto{\pgfqpoint{6.644021in}{2.113089in}}{\pgfqpoint{6.640362in}{2.121922in}}{\pgfqpoint{6.633851in}{2.128433in}}%
\pgfpathcurveto{\pgfqpoint{6.627340in}{2.134944in}}{\pgfqpoint{6.618507in}{2.138603in}}{\pgfqpoint{6.609299in}{2.138603in}}%
\pgfpathcurveto{\pgfqpoint{6.600090in}{2.138603in}}{\pgfqpoint{6.591258in}{2.134944in}}{\pgfqpoint{6.584746in}{2.128433in}}%
\pgfpathcurveto{\pgfqpoint{6.578235in}{2.121922in}}{\pgfqpoint{6.574576in}{2.113089in}}{\pgfqpoint{6.574576in}{2.103881in}}%
\pgfpathcurveto{\pgfqpoint{6.574576in}{2.094672in}}{\pgfqpoint{6.578235in}{2.085840in}}{\pgfqpoint{6.584746in}{2.079328in}}%
\pgfpathcurveto{\pgfqpoint{6.591258in}{2.072817in}}{\pgfqpoint{6.600090in}{2.069159in}}{\pgfqpoint{6.609299in}{2.069159in}}%
\pgfpathlineto{\pgfqpoint{6.609299in}{2.069159in}}%
\pgfpathclose%
\pgfusepath{stroke,fill}%
\end{pgfscope}%
\begin{pgfscope}%
\pgfpathrectangle{\pgfqpoint{1.374500in}{0.082500in}}{\pgfqpoint{2.419000in}{2.419000in}}%
\pgfusepath{clip}%
\pgfsetbuttcap%
\pgfsetroundjoin%
\definecolor{currentfill}{rgb}{0.741176,0.121569,0.003922}%
\pgfsetfillcolor{currentfill}%
\pgfsetfillopacity{0.408225}%
\pgfsetlinewidth{1.003750pt}%
\definecolor{currentstroke}{rgb}{0.741176,0.121569,0.003922}%
\pgfsetstrokecolor{currentstroke}%
\pgfsetstrokeopacity{0.408225}%
\pgfsetdash{}{0pt}%
\pgfpathmoveto{\pgfqpoint{11.067240in}{2.069159in}}%
\pgfpathcurveto{\pgfqpoint{11.076449in}{2.069159in}}{\pgfqpoint{11.085281in}{2.072817in}}{\pgfqpoint{11.091793in}{2.079328in}}%
\pgfpathcurveto{\pgfqpoint{11.098304in}{2.085840in}}{\pgfqpoint{11.101963in}{2.094672in}}{\pgfqpoint{11.101963in}{2.103881in}}%
\pgfpathcurveto{\pgfqpoint{11.101963in}{2.113089in}}{\pgfqpoint{11.098304in}{2.121922in}}{\pgfqpoint{11.091793in}{2.128433in}}%
\pgfpathcurveto{\pgfqpoint{11.085281in}{2.134944in}}{\pgfqpoint{11.076449in}{2.138603in}}{\pgfqpoint{11.067240in}{2.138603in}}%
\pgfpathcurveto{\pgfqpoint{11.058032in}{2.138603in}}{\pgfqpoint{11.049199in}{2.134944in}}{\pgfqpoint{11.042688in}{2.128433in}}%
\pgfpathcurveto{\pgfqpoint{11.036177in}{2.121922in}}{\pgfqpoint{11.032518in}{2.113089in}}{\pgfqpoint{11.032518in}{2.103881in}}%
\pgfpathcurveto{\pgfqpoint{11.032518in}{2.094672in}}{\pgfqpoint{11.036177in}{2.085840in}}{\pgfqpoint{11.042688in}{2.079328in}}%
\pgfpathcurveto{\pgfqpoint{11.049199in}{2.072817in}}{\pgfqpoint{11.058032in}{2.069159in}}{\pgfqpoint{11.067240in}{2.069159in}}%
\pgfpathlineto{\pgfqpoint{11.067240in}{2.069159in}}%
\pgfpathclose%
\pgfusepath{stroke,fill}%
\end{pgfscope}%
\begin{pgfscope}%
\pgfpathrectangle{\pgfqpoint{1.374500in}{0.082500in}}{\pgfqpoint{2.419000in}{2.419000in}}%
\pgfusepath{clip}%
\pgfsetbuttcap%
\pgfsetroundjoin%
\definecolor{currentfill}{rgb}{0.741176,0.121569,0.003922}%
\pgfsetfillcolor{currentfill}%
\pgfsetfillopacity{0.413648}%
\pgfsetlinewidth{1.003750pt}%
\definecolor{currentstroke}{rgb}{0.741176,0.121569,0.003922}%
\pgfsetstrokecolor{currentstroke}%
\pgfsetstrokeopacity{0.413648}%
\pgfsetdash{}{0pt}%
\pgfpathmoveto{\pgfqpoint{4.833690in}{1.959207in}}%
\pgfpathcurveto{\pgfqpoint{4.842899in}{1.959207in}}{\pgfqpoint{4.851731in}{1.962866in}}{\pgfqpoint{4.858243in}{1.969377in}}%
\pgfpathcurveto{\pgfqpoint{4.864754in}{1.975889in}}{\pgfqpoint{4.868412in}{1.984721in}}{\pgfqpoint{4.868412in}{1.993930in}}%
\pgfpathcurveto{\pgfqpoint{4.868412in}{2.003138in}}{\pgfqpoint{4.864754in}{2.011970in}}{\pgfqpoint{4.858243in}{2.018482in}}%
\pgfpathcurveto{\pgfqpoint{4.851731in}{2.024993in}}{\pgfqpoint{4.842899in}{2.028652in}}{\pgfqpoint{4.833690in}{2.028652in}}%
\pgfpathcurveto{\pgfqpoint{4.824482in}{2.028652in}}{\pgfqpoint{4.815649in}{2.024993in}}{\pgfqpoint{4.809138in}{2.018482in}}%
\pgfpathcurveto{\pgfqpoint{4.802627in}{2.011970in}}{\pgfqpoint{4.798968in}{2.003138in}}{\pgfqpoint{4.798968in}{1.993930in}}%
\pgfpathcurveto{\pgfqpoint{4.798968in}{1.984721in}}{\pgfqpoint{4.802627in}{1.975889in}}{\pgfqpoint{4.809138in}{1.969377in}}%
\pgfpathcurveto{\pgfqpoint{4.815649in}{1.962866in}}{\pgfqpoint{4.824482in}{1.959207in}}{\pgfqpoint{4.833690in}{1.959207in}}%
\pgfpathlineto{\pgfqpoint{4.833690in}{1.959207in}}%
\pgfpathclose%
\pgfusepath{stroke,fill}%
\end{pgfscope}%
\begin{pgfscope}%
\pgfpathrectangle{\pgfqpoint{1.374500in}{0.082500in}}{\pgfqpoint{2.419000in}{2.419000in}}%
\pgfusepath{clip}%
\pgfsetbuttcap%
\pgfsetroundjoin%
\definecolor{currentfill}{rgb}{0.741176,0.121569,0.003922}%
\pgfsetfillcolor{currentfill}%
\pgfsetfillopacity{0.413648}%
\pgfsetlinewidth{1.003750pt}%
\definecolor{currentstroke}{rgb}{0.741176,0.121569,0.003922}%
\pgfsetstrokecolor{currentstroke}%
\pgfsetstrokeopacity{0.413648}%
\pgfsetdash{}{0pt}%
\pgfpathmoveto{\pgfqpoint{0.303348in}{1.959207in}}%
\pgfpathcurveto{\pgfqpoint{0.312556in}{1.959207in}}{\pgfqpoint{0.321389in}{1.962866in}}{\pgfqpoint{0.327900in}{1.969377in}}%
\pgfpathcurveto{\pgfqpoint{0.334412in}{1.975889in}}{\pgfqpoint{0.338070in}{1.984721in}}{\pgfqpoint{0.338070in}{1.993930in}}%
\pgfpathcurveto{\pgfqpoint{0.338070in}{2.003138in}}{\pgfqpoint{0.334412in}{2.011970in}}{\pgfqpoint{0.327900in}{2.018482in}}%
\pgfpathcurveto{\pgfqpoint{0.321389in}{2.024993in}}{\pgfqpoint{0.312556in}{2.028652in}}{\pgfqpoint{0.303348in}{2.028652in}}%
\pgfpathcurveto{\pgfqpoint{0.294139in}{2.028652in}}{\pgfqpoint{0.285307in}{2.024993in}}{\pgfqpoint{0.278796in}{2.018482in}}%
\pgfpathcurveto{\pgfqpoint{0.272284in}{2.011970in}}{\pgfqpoint{0.268626in}{2.003138in}}{\pgfqpoint{0.268626in}{1.993930in}}%
\pgfpathcurveto{\pgfqpoint{0.268626in}{1.984721in}}{\pgfqpoint{0.272284in}{1.975889in}}{\pgfqpoint{0.278796in}{1.969377in}}%
\pgfpathcurveto{\pgfqpoint{0.285307in}{1.962866in}}{\pgfqpoint{0.294139in}{1.959207in}}{\pgfqpoint{0.303348in}{1.959207in}}%
\pgfpathlineto{\pgfqpoint{0.303348in}{1.959207in}}%
\pgfpathclose%
\pgfusepath{stroke,fill}%
\end{pgfscope}%
\begin{pgfscope}%
\pgfpathrectangle{\pgfqpoint{1.374500in}{0.082500in}}{\pgfqpoint{2.419000in}{2.419000in}}%
\pgfusepath{clip}%
\pgfsetbuttcap%
\pgfsetroundjoin%
\definecolor{currentfill}{rgb}{0.741176,0.121569,0.003922}%
\pgfsetfillcolor{currentfill}%
\pgfsetfillopacity{0.413648}%
\pgfsetlinewidth{1.003750pt}%
\definecolor{currentstroke}{rgb}{0.741176,0.121569,0.003922}%
\pgfsetstrokecolor{currentstroke}%
\pgfsetstrokeopacity{0.413648}%
\pgfsetdash{}{0pt}%
\pgfpathmoveto{\pgfqpoint{9.364033in}{1.959207in}}%
\pgfpathcurveto{\pgfqpoint{9.373241in}{1.959207in}}{\pgfqpoint{9.382074in}{1.962866in}}{\pgfqpoint{9.388585in}{1.969377in}}%
\pgfpathcurveto{\pgfqpoint{9.395096in}{1.975889in}}{\pgfqpoint{9.398755in}{1.984721in}}{\pgfqpoint{9.398755in}{1.993930in}}%
\pgfpathcurveto{\pgfqpoint{9.398755in}{2.003138in}}{\pgfqpoint{9.395096in}{2.011970in}}{\pgfqpoint{9.388585in}{2.018482in}}%
\pgfpathcurveto{\pgfqpoint{9.382074in}{2.024993in}}{\pgfqpoint{9.373241in}{2.028652in}}{\pgfqpoint{9.364033in}{2.028652in}}%
\pgfpathcurveto{\pgfqpoint{9.354824in}{2.028652in}}{\pgfqpoint{9.345992in}{2.024993in}}{\pgfqpoint{9.339480in}{2.018482in}}%
\pgfpathcurveto{\pgfqpoint{9.332969in}{2.011970in}}{\pgfqpoint{9.329310in}{2.003138in}}{\pgfqpoint{9.329310in}{1.993930in}}%
\pgfpathcurveto{\pgfqpoint{9.329310in}{1.984721in}}{\pgfqpoint{9.332969in}{1.975889in}}{\pgfqpoint{9.339480in}{1.969377in}}%
\pgfpathcurveto{\pgfqpoint{9.345992in}{1.962866in}}{\pgfqpoint{9.354824in}{1.959207in}}{\pgfqpoint{9.364033in}{1.959207in}}%
\pgfpathlineto{\pgfqpoint{9.364033in}{1.959207in}}%
\pgfpathclose%
\pgfusepath{stroke,fill}%
\end{pgfscope}%
\begin{pgfscope}%
\pgfpathrectangle{\pgfqpoint{1.374500in}{0.082500in}}{\pgfqpoint{2.419000in}{2.419000in}}%
\pgfusepath{clip}%
\pgfsetbuttcap%
\pgfsetroundjoin%
\definecolor{currentfill}{rgb}{0.741176,0.121569,0.003922}%
\pgfsetfillcolor{currentfill}%
\pgfsetfillopacity{0.419251}%
\pgfsetlinewidth{1.003750pt}%
\definecolor{currentstroke}{rgb}{0.741176,0.121569,0.003922}%
\pgfsetstrokecolor{currentstroke}%
\pgfsetstrokeopacity{0.419251}%
\pgfsetdash{}{0pt}%
\pgfpathmoveto{\pgfqpoint{-1.605679in}{1.845626in}}%
\pgfpathcurveto{\pgfqpoint{-1.596470in}{1.845626in}}{\pgfqpoint{-1.587638in}{1.849284in}}{\pgfqpoint{-1.581126in}{1.855796in}}%
\pgfpathcurveto{\pgfqpoint{-1.574615in}{1.862307in}}{\pgfqpoint{-1.570956in}{1.871139in}}{\pgfqpoint{-1.570956in}{1.880348in}}%
\pgfpathcurveto{\pgfqpoint{-1.570956in}{1.889556in}}{\pgfqpoint{-1.574615in}{1.898389in}}{\pgfqpoint{-1.581126in}{1.904900in}}%
\pgfpathcurveto{\pgfqpoint{-1.587638in}{1.911412in}}{\pgfqpoint{-1.596470in}{1.915070in}}{\pgfqpoint{-1.605679in}{1.915070in}}%
\pgfpathcurveto{\pgfqpoint{-1.614887in}{1.915070in}}{\pgfqpoint{-1.623719in}{1.911412in}}{\pgfqpoint{-1.630231in}{1.904900in}}%
\pgfpathcurveto{\pgfqpoint{-1.636742in}{1.898389in}}{\pgfqpoint{-1.640401in}{1.889556in}}{\pgfqpoint{-1.640401in}{1.880348in}}%
\pgfpathcurveto{\pgfqpoint{-1.640401in}{1.871139in}}{\pgfqpoint{-1.636742in}{1.862307in}}{\pgfqpoint{-1.630231in}{1.855796in}}%
\pgfpathcurveto{\pgfqpoint{-1.623719in}{1.849284in}}{\pgfqpoint{-1.614887in}{1.845626in}}{\pgfqpoint{-1.605679in}{1.845626in}}%
\pgfpathlineto{\pgfqpoint{-1.605679in}{1.845626in}}%
\pgfpathclose%
\pgfusepath{stroke,fill}%
\end{pgfscope}%
\begin{pgfscope}%
\pgfpathrectangle{\pgfqpoint{1.374500in}{0.082500in}}{\pgfqpoint{2.419000in}{2.419000in}}%
\pgfusepath{clip}%
\pgfsetbuttcap%
\pgfsetroundjoin%
\definecolor{currentfill}{rgb}{0.741176,0.121569,0.003922}%
\pgfsetfillcolor{currentfill}%
\pgfsetfillopacity{0.419251}%
\pgfsetlinewidth{1.003750pt}%
\definecolor{currentstroke}{rgb}{0.741176,0.121569,0.003922}%
\pgfsetstrokecolor{currentstroke}%
\pgfsetstrokeopacity{0.419251}%
\pgfsetdash{}{0pt}%
\pgfpathmoveto{\pgfqpoint{2.999455in}{1.845626in}}%
\pgfpathcurveto{\pgfqpoint{3.008664in}{1.845626in}}{\pgfqpoint{3.017496in}{1.849284in}}{\pgfqpoint{3.024007in}{1.855796in}}%
\pgfpathcurveto{\pgfqpoint{3.030519in}{1.862307in}}{\pgfqpoint{3.034177in}{1.871139in}}{\pgfqpoint{3.034177in}{1.880348in}}%
\pgfpathcurveto{\pgfqpoint{3.034177in}{1.889556in}}{\pgfqpoint{3.030519in}{1.898389in}}{\pgfqpoint{3.024007in}{1.904900in}}%
\pgfpathcurveto{\pgfqpoint{3.017496in}{1.911412in}}{\pgfqpoint{3.008664in}{1.915070in}}{\pgfqpoint{2.999455in}{1.915070in}}%
\pgfpathcurveto{\pgfqpoint{2.990247in}{1.915070in}}{\pgfqpoint{2.981414in}{1.911412in}}{\pgfqpoint{2.974903in}{1.904900in}}%
\pgfpathcurveto{\pgfqpoint{2.968391in}{1.898389in}}{\pgfqpoint{2.964733in}{1.889556in}}{\pgfqpoint{2.964733in}{1.880348in}}%
\pgfpathcurveto{\pgfqpoint{2.964733in}{1.871139in}}{\pgfqpoint{2.968391in}{1.862307in}}{\pgfqpoint{2.974903in}{1.855796in}}%
\pgfpathcurveto{\pgfqpoint{2.981414in}{1.849284in}}{\pgfqpoint{2.990247in}{1.845626in}}{\pgfqpoint{2.999455in}{1.845626in}}%
\pgfpathlineto{\pgfqpoint{2.999455in}{1.845626in}}%
\pgfpathclose%
\pgfusepath{stroke,fill}%
\end{pgfscope}%
\begin{pgfscope}%
\pgfpathrectangle{\pgfqpoint{1.374500in}{0.082500in}}{\pgfqpoint{2.419000in}{2.419000in}}%
\pgfusepath{clip}%
\pgfsetbuttcap%
\pgfsetroundjoin%
\definecolor{currentfill}{rgb}{0.741176,0.121569,0.003922}%
\pgfsetfillcolor{currentfill}%
\pgfsetfillopacity{0.419251}%
\pgfsetlinewidth{1.003750pt}%
\definecolor{currentstroke}{rgb}{0.741176,0.121569,0.003922}%
\pgfsetstrokecolor{currentstroke}%
\pgfsetstrokeopacity{0.419251}%
\pgfsetdash{}{0pt}%
\pgfpathmoveto{\pgfqpoint{7.604589in}{1.845626in}}%
\pgfpathcurveto{\pgfqpoint{7.613797in}{1.845626in}}{\pgfqpoint{7.622630in}{1.849284in}}{\pgfqpoint{7.629141in}{1.855796in}}%
\pgfpathcurveto{\pgfqpoint{7.635652in}{1.862307in}}{\pgfqpoint{7.639311in}{1.871139in}}{\pgfqpoint{7.639311in}{1.880348in}}%
\pgfpathcurveto{\pgfqpoint{7.639311in}{1.889556in}}{\pgfqpoint{7.635652in}{1.898389in}}{\pgfqpoint{7.629141in}{1.904900in}}%
\pgfpathcurveto{\pgfqpoint{7.622630in}{1.911412in}}{\pgfqpoint{7.613797in}{1.915070in}}{\pgfqpoint{7.604589in}{1.915070in}}%
\pgfpathcurveto{\pgfqpoint{7.595380in}{1.915070in}}{\pgfqpoint{7.586548in}{1.911412in}}{\pgfqpoint{7.580036in}{1.904900in}}%
\pgfpathcurveto{\pgfqpoint{7.573525in}{1.898389in}}{\pgfqpoint{7.569866in}{1.889556in}}{\pgfqpoint{7.569866in}{1.880348in}}%
\pgfpathcurveto{\pgfqpoint{7.569866in}{1.871139in}}{\pgfqpoint{7.573525in}{1.862307in}}{\pgfqpoint{7.580036in}{1.855796in}}%
\pgfpathcurveto{\pgfqpoint{7.586548in}{1.849284in}}{\pgfqpoint{7.595380in}{1.845626in}}{\pgfqpoint{7.604589in}{1.845626in}}%
\pgfpathlineto{\pgfqpoint{7.604589in}{1.845626in}}%
\pgfpathclose%
\pgfusepath{stroke,fill}%
\end{pgfscope}%
\begin{pgfscope}%
\pgfpathrectangle{\pgfqpoint{1.374500in}{0.082500in}}{\pgfqpoint{2.419000in}{2.419000in}}%
\pgfusepath{clip}%
\pgfsetbuttcap%
\pgfsetroundjoin%
\definecolor{currentfill}{rgb}{0.741176,0.121569,0.003922}%
\pgfsetfillcolor{currentfill}%
\pgfsetfillopacity{0.425042}%
\pgfsetlinewidth{1.003750pt}%
\definecolor{currentstroke}{rgb}{0.741176,0.121569,0.003922}%
\pgfsetstrokecolor{currentstroke}%
\pgfsetstrokeopacity{0.425042}%
\pgfsetdash{}{0pt}%
\pgfpathmoveto{\pgfqpoint{5.786076in}{1.728231in}}%
\pgfpathcurveto{\pgfqpoint{5.795285in}{1.728231in}}{\pgfqpoint{5.804117in}{1.731889in}}{\pgfqpoint{5.810629in}{1.738401in}}%
\pgfpathcurveto{\pgfqpoint{5.817140in}{1.744912in}}{\pgfqpoint{5.820799in}{1.753745in}}{\pgfqpoint{5.820799in}{1.762953in}}%
\pgfpathcurveto{\pgfqpoint{5.820799in}{1.772162in}}{\pgfqpoint{5.817140in}{1.780994in}}{\pgfqpoint{5.810629in}{1.787505in}}%
\pgfpathcurveto{\pgfqpoint{5.804117in}{1.794017in}}{\pgfqpoint{5.795285in}{1.797675in}}{\pgfqpoint{5.786076in}{1.797675in}}%
\pgfpathcurveto{\pgfqpoint{5.776868in}{1.797675in}}{\pgfqpoint{5.768035in}{1.794017in}}{\pgfqpoint{5.761524in}{1.787505in}}%
\pgfpathcurveto{\pgfqpoint{5.755013in}{1.780994in}}{\pgfqpoint{5.751354in}{1.772162in}}{\pgfqpoint{5.751354in}{1.762953in}}%
\pgfpathcurveto{\pgfqpoint{5.751354in}{1.753745in}}{\pgfqpoint{5.755013in}{1.744912in}}{\pgfqpoint{5.761524in}{1.738401in}}%
\pgfpathcurveto{\pgfqpoint{5.768035in}{1.731889in}}{\pgfqpoint{5.776868in}{1.728231in}}{\pgfqpoint{5.786076in}{1.728231in}}%
\pgfpathlineto{\pgfqpoint{5.786076in}{1.728231in}}%
\pgfpathclose%
\pgfusepath{stroke,fill}%
\end{pgfscope}%
\begin{pgfscope}%
\pgfpathrectangle{\pgfqpoint{1.374500in}{0.082500in}}{\pgfqpoint{2.419000in}{2.419000in}}%
\pgfusepath{clip}%
\pgfsetbuttcap%
\pgfsetroundjoin%
\definecolor{currentfill}{rgb}{0.741176,0.121569,0.003922}%
\pgfsetfillcolor{currentfill}%
\pgfsetfillopacity{0.425042}%
\pgfsetlinewidth{1.003750pt}%
\definecolor{currentstroke}{rgb}{0.741176,0.121569,0.003922}%
\pgfsetstrokecolor{currentstroke}%
\pgfsetstrokeopacity{0.425042}%
\pgfsetdash{}{0pt}%
\pgfpathmoveto{\pgfqpoint{10.468512in}{1.728231in}}%
\pgfpathcurveto{\pgfqpoint{10.477720in}{1.728231in}}{\pgfqpoint{10.486553in}{1.731889in}}{\pgfqpoint{10.493064in}{1.738401in}}%
\pgfpathcurveto{\pgfqpoint{10.499576in}{1.744912in}}{\pgfqpoint{10.503234in}{1.753745in}}{\pgfqpoint{10.503234in}{1.762953in}}%
\pgfpathcurveto{\pgfqpoint{10.503234in}{1.772162in}}{\pgfqpoint{10.499576in}{1.780994in}}{\pgfqpoint{10.493064in}{1.787505in}}%
\pgfpathcurveto{\pgfqpoint{10.486553in}{1.794017in}}{\pgfqpoint{10.477720in}{1.797675in}}{\pgfqpoint{10.468512in}{1.797675in}}%
\pgfpathcurveto{\pgfqpoint{10.459304in}{1.797675in}}{\pgfqpoint{10.450471in}{1.794017in}}{\pgfqpoint{10.443960in}{1.787505in}}%
\pgfpathcurveto{\pgfqpoint{10.437448in}{1.780994in}}{\pgfqpoint{10.433790in}{1.772162in}}{\pgfqpoint{10.433790in}{1.762953in}}%
\pgfpathcurveto{\pgfqpoint{10.433790in}{1.753745in}}{\pgfqpoint{10.437448in}{1.744912in}}{\pgfqpoint{10.443960in}{1.738401in}}%
\pgfpathcurveto{\pgfqpoint{10.450471in}{1.731889in}}{\pgfqpoint{10.459304in}{1.728231in}}{\pgfqpoint{10.468512in}{1.728231in}}%
\pgfpathlineto{\pgfqpoint{10.468512in}{1.728231in}}%
\pgfpathclose%
\pgfusepath{stroke,fill}%
\end{pgfscope}%
\begin{pgfscope}%
\pgfpathrectangle{\pgfqpoint{1.374500in}{0.082500in}}{\pgfqpoint{2.419000in}{2.419000in}}%
\pgfusepath{clip}%
\pgfsetbuttcap%
\pgfsetroundjoin%
\definecolor{currentfill}{rgb}{0.741176,0.121569,0.003922}%
\pgfsetfillcolor{currentfill}%
\pgfsetfillopacity{0.425042}%
\pgfsetlinewidth{1.003750pt}%
\definecolor{currentstroke}{rgb}{0.741176,0.121569,0.003922}%
\pgfsetstrokecolor{currentstroke}%
\pgfsetstrokeopacity{0.425042}%
\pgfsetdash{}{0pt}%
\pgfpathmoveto{\pgfqpoint{1.103641in}{1.728231in}}%
\pgfpathcurveto{\pgfqpoint{1.112849in}{1.728231in}}{\pgfqpoint{1.121682in}{1.731889in}}{\pgfqpoint{1.128193in}{1.738401in}}%
\pgfpathcurveto{\pgfqpoint{1.134704in}{1.744912in}}{\pgfqpoint{1.138363in}{1.753745in}}{\pgfqpoint{1.138363in}{1.762953in}}%
\pgfpathcurveto{\pgfqpoint{1.138363in}{1.772162in}}{\pgfqpoint{1.134704in}{1.780994in}}{\pgfqpoint{1.128193in}{1.787505in}}%
\pgfpathcurveto{\pgfqpoint{1.121682in}{1.794017in}}{\pgfqpoint{1.112849in}{1.797675in}}{\pgfqpoint{1.103641in}{1.797675in}}%
\pgfpathcurveto{\pgfqpoint{1.094432in}{1.797675in}}{\pgfqpoint{1.085600in}{1.794017in}}{\pgfqpoint{1.079088in}{1.787505in}}%
\pgfpathcurveto{\pgfqpoint{1.072577in}{1.780994in}}{\pgfqpoint{1.068918in}{1.772162in}}{\pgfqpoint{1.068918in}{1.762953in}}%
\pgfpathcurveto{\pgfqpoint{1.068918in}{1.753745in}}{\pgfqpoint{1.072577in}{1.744912in}}{\pgfqpoint{1.079088in}{1.738401in}}%
\pgfpathcurveto{\pgfqpoint{1.085600in}{1.731889in}}{\pgfqpoint{1.094432in}{1.728231in}}{\pgfqpoint{1.103641in}{1.728231in}}%
\pgfpathlineto{\pgfqpoint{1.103641in}{1.728231in}}%
\pgfpathclose%
\pgfusepath{stroke,fill}%
\end{pgfscope}%
\begin{pgfscope}%
\pgfpathrectangle{\pgfqpoint{1.374500in}{0.082500in}}{\pgfqpoint{2.419000in}{2.419000in}}%
\pgfusepath{clip}%
\pgfsetbuttcap%
\pgfsetroundjoin%
\definecolor{currentfill}{rgb}{0.741176,0.121569,0.003922}%
\pgfsetfillcolor{currentfill}%
\pgfsetfillopacity{0.431031}%
\pgfsetlinewidth{1.003750pt}%
\definecolor{currentstroke}{rgb}{0.741176,0.121569,0.003922}%
\pgfsetstrokecolor{currentstroke}%
\pgfsetstrokeopacity{0.431031}%
\pgfsetdash{}{0pt}%
\pgfpathmoveto{\pgfqpoint{-0.856907in}{1.606828in}}%
\pgfpathcurveto{\pgfqpoint{-0.847698in}{1.606828in}}{\pgfqpoint{-0.838866in}{1.610486in}}{\pgfqpoint{-0.832354in}{1.616997in}}%
\pgfpathcurveto{\pgfqpoint{-0.825843in}{1.623509in}}{\pgfqpoint{-0.822185in}{1.632341in}}{\pgfqpoint{-0.822185in}{1.641550in}}%
\pgfpathcurveto{\pgfqpoint{-0.822185in}{1.650758in}}{\pgfqpoint{-0.825843in}{1.659591in}}{\pgfqpoint{-0.832354in}{1.666102in}}%
\pgfpathcurveto{\pgfqpoint{-0.838866in}{1.672613in}}{\pgfqpoint{-0.847698in}{1.676272in}}{\pgfqpoint{-0.856907in}{1.676272in}}%
\pgfpathcurveto{\pgfqpoint{-0.866115in}{1.676272in}}{\pgfqpoint{-0.874948in}{1.672613in}}{\pgfqpoint{-0.881459in}{1.666102in}}%
\pgfpathcurveto{\pgfqpoint{-0.887970in}{1.659591in}}{\pgfqpoint{-0.891629in}{1.650758in}}{\pgfqpoint{-0.891629in}{1.641550in}}%
\pgfpathcurveto{\pgfqpoint{-0.891629in}{1.632341in}}{\pgfqpoint{-0.887970in}{1.623509in}}{\pgfqpoint{-0.881459in}{1.616997in}}%
\pgfpathcurveto{\pgfqpoint{-0.874948in}{1.610486in}}{\pgfqpoint{-0.866115in}{1.606828in}}{\pgfqpoint{-0.856907in}{1.606828in}}%
\pgfpathlineto{\pgfqpoint{-0.856907in}{1.606828in}}%
\pgfpathclose%
\pgfusepath{stroke,fill}%
\end{pgfscope}%
\begin{pgfscope}%
\pgfpathrectangle{\pgfqpoint{1.374500in}{0.082500in}}{\pgfqpoint{2.419000in}{2.419000in}}%
\pgfusepath{clip}%
\pgfsetbuttcap%
\pgfsetroundjoin%
\definecolor{currentfill}{rgb}{0.741176,0.121569,0.003922}%
\pgfsetfillcolor{currentfill}%
\pgfsetfillopacity{0.431031}%
\pgfsetlinewidth{1.003750pt}%
\definecolor{currentstroke}{rgb}{0.741176,0.121569,0.003922}%
\pgfsetstrokecolor{currentstroke}%
\pgfsetstrokeopacity{0.431031}%
\pgfsetdash{}{0pt}%
\pgfpathmoveto{\pgfqpoint{3.905470in}{1.606828in}}%
\pgfpathcurveto{\pgfqpoint{3.914679in}{1.606828in}}{\pgfqpoint{3.923511in}{1.610486in}}{\pgfqpoint{3.930023in}{1.616997in}}%
\pgfpathcurveto{\pgfqpoint{3.936534in}{1.623509in}}{\pgfqpoint{3.940193in}{1.632341in}}{\pgfqpoint{3.940193in}{1.641550in}}%
\pgfpathcurveto{\pgfqpoint{3.940193in}{1.650758in}}{\pgfqpoint{3.936534in}{1.659591in}}{\pgfqpoint{3.930023in}{1.666102in}}%
\pgfpathcurveto{\pgfqpoint{3.923511in}{1.672613in}}{\pgfqpoint{3.914679in}{1.676272in}}{\pgfqpoint{3.905470in}{1.676272in}}%
\pgfpathcurveto{\pgfqpoint{3.896262in}{1.676272in}}{\pgfqpoint{3.887429in}{1.672613in}}{\pgfqpoint{3.880918in}{1.666102in}}%
\pgfpathcurveto{\pgfqpoint{3.874407in}{1.659591in}}{\pgfqpoint{3.870748in}{1.650758in}}{\pgfqpoint{3.870748in}{1.641550in}}%
\pgfpathcurveto{\pgfqpoint{3.870748in}{1.632341in}}{\pgfqpoint{3.874407in}{1.623509in}}{\pgfqpoint{3.880918in}{1.616997in}}%
\pgfpathcurveto{\pgfqpoint{3.887429in}{1.610486in}}{\pgfqpoint{3.896262in}{1.606828in}}{\pgfqpoint{3.905470in}{1.606828in}}%
\pgfpathlineto{\pgfqpoint{3.905470in}{1.606828in}}%
\pgfpathclose%
\pgfusepath{stroke,fill}%
\end{pgfscope}%
\begin{pgfscope}%
\pgfpathrectangle{\pgfqpoint{1.374500in}{0.082500in}}{\pgfqpoint{2.419000in}{2.419000in}}%
\pgfusepath{clip}%
\pgfsetbuttcap%
\pgfsetroundjoin%
\definecolor{currentfill}{rgb}{0.741176,0.121569,0.003922}%
\pgfsetfillcolor{currentfill}%
\pgfsetfillopacity{0.431031}%
\pgfsetlinewidth{1.003750pt}%
\definecolor{currentstroke}{rgb}{0.741176,0.121569,0.003922}%
\pgfsetstrokecolor{currentstroke}%
\pgfsetstrokeopacity{0.431031}%
\pgfsetdash{}{0pt}%
\pgfpathmoveto{\pgfqpoint{8.667848in}{1.606828in}}%
\pgfpathcurveto{\pgfqpoint{8.677056in}{1.606828in}}{\pgfqpoint{8.685889in}{1.610486in}}{\pgfqpoint{8.692400in}{1.616997in}}%
\pgfpathcurveto{\pgfqpoint{8.698911in}{1.623509in}}{\pgfqpoint{8.702570in}{1.632341in}}{\pgfqpoint{8.702570in}{1.641550in}}%
\pgfpathcurveto{\pgfqpoint{8.702570in}{1.650758in}}{\pgfqpoint{8.698911in}{1.659591in}}{\pgfqpoint{8.692400in}{1.666102in}}%
\pgfpathcurveto{\pgfqpoint{8.685889in}{1.672613in}}{\pgfqpoint{8.677056in}{1.676272in}}{\pgfqpoint{8.667848in}{1.676272in}}%
\pgfpathcurveto{\pgfqpoint{8.658639in}{1.676272in}}{\pgfqpoint{8.649807in}{1.672613in}}{\pgfqpoint{8.643295in}{1.666102in}}%
\pgfpathcurveto{\pgfqpoint{8.636784in}{1.659591in}}{\pgfqpoint{8.633125in}{1.650758in}}{\pgfqpoint{8.633125in}{1.641550in}}%
\pgfpathcurveto{\pgfqpoint{8.633125in}{1.632341in}}{\pgfqpoint{8.636784in}{1.623509in}}{\pgfqpoint{8.643295in}{1.616997in}}%
\pgfpathcurveto{\pgfqpoint{8.649807in}{1.610486in}}{\pgfqpoint{8.658639in}{1.606828in}}{\pgfqpoint{8.667848in}{1.606828in}}%
\pgfpathlineto{\pgfqpoint{8.667848in}{1.606828in}}%
\pgfpathclose%
\pgfusepath{stroke,fill}%
\end{pgfscope}%
\begin{pgfscope}%
\pgfpathrectangle{\pgfqpoint{1.374500in}{0.082500in}}{\pgfqpoint{2.419000in}{2.419000in}}%
\pgfusepath{clip}%
\pgfsetbuttcap%
\pgfsetroundjoin%
\definecolor{currentfill}{rgb}{0.741176,0.121569,0.003922}%
\pgfsetfillcolor{currentfill}%
\pgfsetfillopacity{0.437227}%
\pgfsetlinewidth{1.003750pt}%
\definecolor{currentstroke}{rgb}{0.741176,0.121569,0.003922}%
\pgfsetstrokecolor{currentstroke}%
\pgfsetstrokeopacity{0.437227}%
\pgfsetdash{}{0pt}%
\pgfpathmoveto{\pgfqpoint{1.959535in}{1.481207in}}%
\pgfpathcurveto{\pgfqpoint{1.968744in}{1.481207in}}{\pgfqpoint{1.977576in}{1.484865in}}{\pgfqpoint{1.984088in}{1.491377in}}%
\pgfpathcurveto{\pgfqpoint{1.990599in}{1.497888in}}{\pgfqpoint{1.994258in}{1.506721in}}{\pgfqpoint{1.994258in}{1.515929in}}%
\pgfpathcurveto{\pgfqpoint{1.994258in}{1.525138in}}{\pgfqpoint{1.990599in}{1.533970in}}{\pgfqpoint{1.984088in}{1.540481in}}%
\pgfpathcurveto{\pgfqpoint{1.977576in}{1.546993in}}{\pgfqpoint{1.968744in}{1.550651in}}{\pgfqpoint{1.959535in}{1.550651in}}%
\pgfpathcurveto{\pgfqpoint{1.950327in}{1.550651in}}{\pgfqpoint{1.941494in}{1.546993in}}{\pgfqpoint{1.934983in}{1.540481in}}%
\pgfpathcurveto{\pgfqpoint{1.928472in}{1.533970in}}{\pgfqpoint{1.924813in}{1.525138in}}{\pgfqpoint{1.924813in}{1.515929in}}%
\pgfpathcurveto{\pgfqpoint{1.924813in}{1.506721in}}{\pgfqpoint{1.928472in}{1.497888in}}{\pgfqpoint{1.934983in}{1.491377in}}%
\pgfpathcurveto{\pgfqpoint{1.941494in}{1.484865in}}{\pgfqpoint{1.950327in}{1.481207in}}{\pgfqpoint{1.959535in}{1.481207in}}%
\pgfpathlineto{\pgfqpoint{1.959535in}{1.481207in}}%
\pgfpathclose%
\pgfusepath{stroke,fill}%
\end{pgfscope}%
\begin{pgfscope}%
\pgfpathrectangle{\pgfqpoint{1.374500in}{0.082500in}}{\pgfqpoint{2.419000in}{2.419000in}}%
\pgfusepath{clip}%
\pgfsetbuttcap%
\pgfsetroundjoin%
\definecolor{currentfill}{rgb}{0.741176,0.121569,0.003922}%
\pgfsetfillcolor{currentfill}%
\pgfsetfillopacity{0.437227}%
\pgfsetlinewidth{1.003750pt}%
\definecolor{currentstroke}{rgb}{0.741176,0.121569,0.003922}%
\pgfsetstrokecolor{currentstroke}%
\pgfsetstrokeopacity{0.437227}%
\pgfsetdash{}{0pt}%
\pgfpathmoveto{\pgfqpoint{11.649727in}{1.481207in}}%
\pgfpathcurveto{\pgfqpoint{11.658936in}{1.481207in}}{\pgfqpoint{11.667768in}{1.484865in}}{\pgfqpoint{11.674279in}{1.491377in}}%
\pgfpathcurveto{\pgfqpoint{11.680791in}{1.497888in}}{\pgfqpoint{11.684449in}{1.506721in}}{\pgfqpoint{11.684449in}{1.515929in}}%
\pgfpathcurveto{\pgfqpoint{11.684449in}{1.525138in}}{\pgfqpoint{11.680791in}{1.533970in}}{\pgfqpoint{11.674279in}{1.540481in}}%
\pgfpathcurveto{\pgfqpoint{11.667768in}{1.546993in}}{\pgfqpoint{11.658936in}{1.550651in}}{\pgfqpoint{11.649727in}{1.550651in}}%
\pgfpathcurveto{\pgfqpoint{11.640519in}{1.550651in}}{\pgfqpoint{11.631686in}{1.546993in}}{\pgfqpoint{11.625175in}{1.540481in}}%
\pgfpathcurveto{\pgfqpoint{11.618663in}{1.533970in}}{\pgfqpoint{11.615005in}{1.525138in}}{\pgfqpoint{11.615005in}{1.515929in}}%
\pgfpathcurveto{\pgfqpoint{11.615005in}{1.506721in}}{\pgfqpoint{11.618663in}{1.497888in}}{\pgfqpoint{11.625175in}{1.491377in}}%
\pgfpathcurveto{\pgfqpoint{11.631686in}{1.484865in}}{\pgfqpoint{11.640519in}{1.481207in}}{\pgfqpoint{11.649727in}{1.481207in}}%
\pgfpathlineto{\pgfqpoint{11.649727in}{1.481207in}}%
\pgfpathclose%
\pgfusepath{stroke,fill}%
\end{pgfscope}%
\begin{pgfscope}%
\pgfpathrectangle{\pgfqpoint{1.374500in}{0.082500in}}{\pgfqpoint{2.419000in}{2.419000in}}%
\pgfusepath{clip}%
\pgfsetbuttcap%
\pgfsetroundjoin%
\definecolor{currentfill}{rgb}{0.741176,0.121569,0.003922}%
\pgfsetfillcolor{currentfill}%
\pgfsetfillopacity{0.437227}%
\pgfsetlinewidth{1.003750pt}%
\definecolor{currentstroke}{rgb}{0.741176,0.121569,0.003922}%
\pgfsetstrokecolor{currentstroke}%
\pgfsetstrokeopacity{0.437227}%
\pgfsetdash{}{0pt}%
\pgfpathmoveto{\pgfqpoint{6.804631in}{1.481207in}}%
\pgfpathcurveto{\pgfqpoint{6.813840in}{1.481207in}}{\pgfqpoint{6.822672in}{1.484865in}}{\pgfqpoint{6.829184in}{1.491377in}}%
\pgfpathcurveto{\pgfqpoint{6.835695in}{1.497888in}}{\pgfqpoint{6.839353in}{1.506721in}}{\pgfqpoint{6.839353in}{1.515929in}}%
\pgfpathcurveto{\pgfqpoint{6.839353in}{1.525138in}}{\pgfqpoint{6.835695in}{1.533970in}}{\pgfqpoint{6.829184in}{1.540481in}}%
\pgfpathcurveto{\pgfqpoint{6.822672in}{1.546993in}}{\pgfqpoint{6.813840in}{1.550651in}}{\pgfqpoint{6.804631in}{1.550651in}}%
\pgfpathcurveto{\pgfqpoint{6.795423in}{1.550651in}}{\pgfqpoint{6.786590in}{1.546993in}}{\pgfqpoint{6.780079in}{1.540481in}}%
\pgfpathcurveto{\pgfqpoint{6.773568in}{1.533970in}}{\pgfqpoint{6.769909in}{1.525138in}}{\pgfqpoint{6.769909in}{1.515929in}}%
\pgfpathcurveto{\pgfqpoint{6.769909in}{1.506721in}}{\pgfqpoint{6.773568in}{1.497888in}}{\pgfqpoint{6.780079in}{1.491377in}}%
\pgfpathcurveto{\pgfqpoint{6.786590in}{1.484865in}}{\pgfqpoint{6.795423in}{1.481207in}}{\pgfqpoint{6.804631in}{1.481207in}}%
\pgfpathlineto{\pgfqpoint{6.804631in}{1.481207in}}%
\pgfpathclose%
\pgfusepath{stroke,fill}%
\end{pgfscope}%
\begin{pgfscope}%
\pgfpathrectangle{\pgfqpoint{1.374500in}{0.082500in}}{\pgfqpoint{2.419000in}{2.419000in}}%
\pgfusepath{clip}%
\pgfsetbuttcap%
\pgfsetroundjoin%
\definecolor{currentfill}{rgb}{0.741176,0.121569,0.003922}%
\pgfsetfillcolor{currentfill}%
\pgfsetfillopacity{0.443643}%
\pgfsetlinewidth{1.003750pt}%
\definecolor{currentstroke}{rgb}{0.741176,0.121569,0.003922}%
\pgfsetstrokecolor{currentstroke}%
\pgfsetstrokeopacity{0.443643}%
\pgfsetdash{}{0pt}%
\pgfpathmoveto{\pgfqpoint{9.806284in}{1.351145in}}%
\pgfpathcurveto{\pgfqpoint{9.815493in}{1.351145in}}{\pgfqpoint{9.824325in}{1.354804in}}{\pgfqpoint{9.830837in}{1.361315in}}%
\pgfpathcurveto{\pgfqpoint{9.837348in}{1.367827in}}{\pgfqpoint{9.841007in}{1.376659in}}{\pgfqpoint{9.841007in}{1.385868in}}%
\pgfpathcurveto{\pgfqpoint{9.841007in}{1.395076in}}{\pgfqpoint{9.837348in}{1.403908in}}{\pgfqpoint{9.830837in}{1.410420in}}%
\pgfpathcurveto{\pgfqpoint{9.824325in}{1.416931in}}{\pgfqpoint{9.815493in}{1.420590in}}{\pgfqpoint{9.806284in}{1.420590in}}%
\pgfpathcurveto{\pgfqpoint{9.797076in}{1.420590in}}{\pgfqpoint{9.788243in}{1.416931in}}{\pgfqpoint{9.781732in}{1.410420in}}%
\pgfpathcurveto{\pgfqpoint{9.775221in}{1.403908in}}{\pgfqpoint{9.771562in}{1.395076in}}{\pgfqpoint{9.771562in}{1.385868in}}%
\pgfpathcurveto{\pgfqpoint{9.771562in}{1.376659in}}{\pgfqpoint{9.775221in}{1.367827in}}{\pgfqpoint{9.781732in}{1.361315in}}%
\pgfpathcurveto{\pgfqpoint{9.788243in}{1.354804in}}{\pgfqpoint{9.797076in}{1.351145in}}{\pgfqpoint{9.806284in}{1.351145in}}%
\pgfpathlineto{\pgfqpoint{9.806284in}{1.351145in}}%
\pgfpathclose%
\pgfusepath{stroke,fill}%
\end{pgfscope}%
\begin{pgfscope}%
\pgfpathrectangle{\pgfqpoint{1.374500in}{0.082500in}}{\pgfqpoint{2.419000in}{2.419000in}}%
\pgfusepath{clip}%
\pgfsetbuttcap%
\pgfsetroundjoin%
\definecolor{currentfill}{rgb}{0.741176,0.121569,0.003922}%
\pgfsetfillcolor{currentfill}%
\pgfsetfillopacity{0.443643}%
\pgfsetlinewidth{1.003750pt}%
\definecolor{currentstroke}{rgb}{0.741176,0.121569,0.003922}%
\pgfsetstrokecolor{currentstroke}%
\pgfsetstrokeopacity{0.443643}%
\pgfsetdash{}{0pt}%
\pgfpathmoveto{\pgfqpoint{-0.055193in}{1.351145in}}%
\pgfpathcurveto{\pgfqpoint{-0.045985in}{1.351145in}}{\pgfqpoint{-0.037152in}{1.354804in}}{\pgfqpoint{-0.030641in}{1.361315in}}%
\pgfpathcurveto{\pgfqpoint{-0.024130in}{1.367827in}}{\pgfqpoint{-0.020471in}{1.376659in}}{\pgfqpoint{-0.020471in}{1.385868in}}%
\pgfpathcurveto{\pgfqpoint{-0.020471in}{1.395076in}}{\pgfqpoint{-0.024130in}{1.403908in}}{\pgfqpoint{-0.030641in}{1.410420in}}%
\pgfpathcurveto{\pgfqpoint{-0.037152in}{1.416931in}}{\pgfqpoint{-0.045985in}{1.420590in}}{\pgfqpoint{-0.055193in}{1.420590in}}%
\pgfpathcurveto{\pgfqpoint{-0.064402in}{1.420590in}}{\pgfqpoint{-0.073234in}{1.416931in}}{\pgfqpoint{-0.079746in}{1.410420in}}%
\pgfpathcurveto{\pgfqpoint{-0.086257in}{1.403908in}}{\pgfqpoint{-0.089916in}{1.395076in}}{\pgfqpoint{-0.089916in}{1.385868in}}%
\pgfpathcurveto{\pgfqpoint{-0.089916in}{1.376659in}}{\pgfqpoint{-0.086257in}{1.367827in}}{\pgfqpoint{-0.079746in}{1.361315in}}%
\pgfpathcurveto{\pgfqpoint{-0.073234in}{1.354804in}}{\pgfqpoint{-0.064402in}{1.351145in}}{\pgfqpoint{-0.055193in}{1.351145in}}%
\pgfpathlineto{\pgfqpoint{-0.055193in}{1.351145in}}%
\pgfpathclose%
\pgfusepath{stroke,fill}%
\end{pgfscope}%
\begin{pgfscope}%
\pgfpathrectangle{\pgfqpoint{1.374500in}{0.082500in}}{\pgfqpoint{2.419000in}{2.419000in}}%
\pgfusepath{clip}%
\pgfsetbuttcap%
\pgfsetroundjoin%
\definecolor{currentfill}{rgb}{0.741176,0.121569,0.003922}%
\pgfsetfillcolor{currentfill}%
\pgfsetfillopacity{0.443643}%
\pgfsetlinewidth{1.003750pt}%
\definecolor{currentstroke}{rgb}{0.741176,0.121569,0.003922}%
\pgfsetstrokecolor{currentstroke}%
\pgfsetstrokeopacity{0.443643}%
\pgfsetdash{}{0pt}%
\pgfpathmoveto{\pgfqpoint{4.875546in}{1.351145in}}%
\pgfpathcurveto{\pgfqpoint{4.884754in}{1.351145in}}{\pgfqpoint{4.893586in}{1.354804in}}{\pgfqpoint{4.900098in}{1.361315in}}%
\pgfpathcurveto{\pgfqpoint{4.906609in}{1.367827in}}{\pgfqpoint{4.910268in}{1.376659in}}{\pgfqpoint{4.910268in}{1.385868in}}%
\pgfpathcurveto{\pgfqpoint{4.910268in}{1.395076in}}{\pgfqpoint{4.906609in}{1.403908in}}{\pgfqpoint{4.900098in}{1.410420in}}%
\pgfpathcurveto{\pgfqpoint{4.893586in}{1.416931in}}{\pgfqpoint{4.884754in}{1.420590in}}{\pgfqpoint{4.875546in}{1.420590in}}%
\pgfpathcurveto{\pgfqpoint{4.866337in}{1.420590in}}{\pgfqpoint{4.857505in}{1.416931in}}{\pgfqpoint{4.850993in}{1.410420in}}%
\pgfpathcurveto{\pgfqpoint{4.844482in}{1.403908in}}{\pgfqpoint{4.840823in}{1.395076in}}{\pgfqpoint{4.840823in}{1.385868in}}%
\pgfpathcurveto{\pgfqpoint{4.840823in}{1.376659in}}{\pgfqpoint{4.844482in}{1.367827in}}{\pgfqpoint{4.850993in}{1.361315in}}%
\pgfpathcurveto{\pgfqpoint{4.857505in}{1.354804in}}{\pgfqpoint{4.866337in}{1.351145in}}{\pgfqpoint{4.875546in}{1.351145in}}%
\pgfpathlineto{\pgfqpoint{4.875546in}{1.351145in}}%
\pgfpathclose%
\pgfusepath{stroke,fill}%
\end{pgfscope}%
\begin{pgfscope}%
\pgfpathrectangle{\pgfqpoint{1.374500in}{0.082500in}}{\pgfqpoint{2.419000in}{2.419000in}}%
\pgfusepath{clip}%
\pgfsetbuttcap%
\pgfsetroundjoin%
\definecolor{currentfill}{rgb}{0.741176,0.121569,0.003922}%
\pgfsetfillcolor{currentfill}%
\pgfsetfillopacity{0.450290}%
\pgfsetlinewidth{1.003750pt}%
\definecolor{currentstroke}{rgb}{0.741176,0.121569,0.003922}%
\pgfsetstrokecolor{currentstroke}%
\pgfsetstrokeopacity{0.450290}%
\pgfsetdash{}{0pt}%
\pgfpathmoveto{\pgfqpoint{-2.142429in}{1.216403in}}%
\pgfpathcurveto{\pgfqpoint{-2.133221in}{1.216403in}}{\pgfqpoint{-2.124388in}{1.220061in}}{\pgfqpoint{-2.117877in}{1.226573in}}%
\pgfpathcurveto{\pgfqpoint{-2.111365in}{1.233084in}}{\pgfqpoint{-2.107707in}{1.241917in}}{\pgfqpoint{-2.107707in}{1.251125in}}%
\pgfpathcurveto{\pgfqpoint{-2.107707in}{1.260334in}}{\pgfqpoint{-2.111365in}{1.269166in}}{\pgfqpoint{-2.117877in}{1.275677in}}%
\pgfpathcurveto{\pgfqpoint{-2.124388in}{1.282189in}}{\pgfqpoint{-2.133221in}{1.285847in}}{\pgfqpoint{-2.142429in}{1.285847in}}%
\pgfpathcurveto{\pgfqpoint{-2.151638in}{1.285847in}}{\pgfqpoint{-2.160470in}{1.282189in}}{\pgfqpoint{-2.166981in}{1.275677in}}%
\pgfpathcurveto{\pgfqpoint{-2.173493in}{1.269166in}}{\pgfqpoint{-2.177151in}{1.260334in}}{\pgfqpoint{-2.177151in}{1.251125in}}%
\pgfpathcurveto{\pgfqpoint{-2.177151in}{1.241917in}}{\pgfqpoint{-2.173493in}{1.233084in}}{\pgfqpoint{-2.166981in}{1.226573in}}%
\pgfpathcurveto{\pgfqpoint{-2.160470in}{1.220061in}}{\pgfqpoint{-2.151638in}{1.216403in}}{\pgfqpoint{-2.142429in}{1.216403in}}%
\pgfpathlineto{\pgfqpoint{-2.142429in}{1.216403in}}%
\pgfpathclose%
\pgfusepath{stroke,fill}%
\end{pgfscope}%
\begin{pgfscope}%
\pgfpathrectangle{\pgfqpoint{1.374500in}{0.082500in}}{\pgfqpoint{2.419000in}{2.419000in}}%
\pgfusepath{clip}%
\pgfsetbuttcap%
\pgfsetroundjoin%
\definecolor{currentfill}{rgb}{0.741176,0.121569,0.003922}%
\pgfsetfillcolor{currentfill}%
\pgfsetfillopacity{0.450290}%
\pgfsetlinewidth{1.003750pt}%
\definecolor{currentstroke}{rgb}{0.741176,0.121569,0.003922}%
\pgfsetstrokecolor{currentstroke}%
\pgfsetstrokeopacity{0.450290}%
\pgfsetdash{}{0pt}%
\pgfpathmoveto{\pgfqpoint{2.877035in}{1.216403in}}%
\pgfpathcurveto{\pgfqpoint{2.886243in}{1.216403in}}{\pgfqpoint{2.895076in}{1.220061in}}{\pgfqpoint{2.901587in}{1.226573in}}%
\pgfpathcurveto{\pgfqpoint{2.908098in}{1.233084in}}{\pgfqpoint{2.911757in}{1.241917in}}{\pgfqpoint{2.911757in}{1.251125in}}%
\pgfpathcurveto{\pgfqpoint{2.911757in}{1.260334in}}{\pgfqpoint{2.908098in}{1.269166in}}{\pgfqpoint{2.901587in}{1.275677in}}%
\pgfpathcurveto{\pgfqpoint{2.895076in}{1.282189in}}{\pgfqpoint{2.886243in}{1.285847in}}{\pgfqpoint{2.877035in}{1.285847in}}%
\pgfpathcurveto{\pgfqpoint{2.867826in}{1.285847in}}{\pgfqpoint{2.858994in}{1.282189in}}{\pgfqpoint{2.852482in}{1.275677in}}%
\pgfpathcurveto{\pgfqpoint{2.845971in}{1.269166in}}{\pgfqpoint{2.842313in}{1.260334in}}{\pgfqpoint{2.842313in}{1.251125in}}%
\pgfpathcurveto{\pgfqpoint{2.842313in}{1.241917in}}{\pgfqpoint{2.845971in}{1.233084in}}{\pgfqpoint{2.852482in}{1.226573in}}%
\pgfpathcurveto{\pgfqpoint{2.858994in}{1.220061in}}{\pgfqpoint{2.867826in}{1.216403in}}{\pgfqpoint{2.877035in}{1.216403in}}%
\pgfpathlineto{\pgfqpoint{2.877035in}{1.216403in}}%
\pgfpathclose%
\pgfusepath{stroke,fill}%
\end{pgfscope}%
\begin{pgfscope}%
\pgfpathrectangle{\pgfqpoint{1.374500in}{0.082500in}}{\pgfqpoint{2.419000in}{2.419000in}}%
\pgfusepath{clip}%
\pgfsetbuttcap%
\pgfsetroundjoin%
\definecolor{currentfill}{rgb}{0.741176,0.121569,0.003922}%
\pgfsetfillcolor{currentfill}%
\pgfsetfillopacity{0.450290}%
\pgfsetlinewidth{1.003750pt}%
\definecolor{currentstroke}{rgb}{0.741176,0.121569,0.003922}%
\pgfsetstrokecolor{currentstroke}%
\pgfsetstrokeopacity{0.450290}%
\pgfsetdash{}{0pt}%
\pgfpathmoveto{\pgfqpoint{7.896499in}{1.216403in}}%
\pgfpathcurveto{\pgfqpoint{7.905707in}{1.216403in}}{\pgfqpoint{7.914540in}{1.220061in}}{\pgfqpoint{7.921051in}{1.226573in}}%
\pgfpathcurveto{\pgfqpoint{7.927562in}{1.233084in}}{\pgfqpoint{7.931221in}{1.241917in}}{\pgfqpoint{7.931221in}{1.251125in}}%
\pgfpathcurveto{\pgfqpoint{7.931221in}{1.260334in}}{\pgfqpoint{7.927562in}{1.269166in}}{\pgfqpoint{7.921051in}{1.275677in}}%
\pgfpathcurveto{\pgfqpoint{7.914540in}{1.282189in}}{\pgfqpoint{7.905707in}{1.285847in}}{\pgfqpoint{7.896499in}{1.285847in}}%
\pgfpathcurveto{\pgfqpoint{7.887290in}{1.285847in}}{\pgfqpoint{7.878458in}{1.282189in}}{\pgfqpoint{7.871946in}{1.275677in}}%
\pgfpathcurveto{\pgfqpoint{7.865435in}{1.269166in}}{\pgfqpoint{7.861776in}{1.260334in}}{\pgfqpoint{7.861776in}{1.251125in}}%
\pgfpathcurveto{\pgfqpoint{7.861776in}{1.241917in}}{\pgfqpoint{7.865435in}{1.233084in}}{\pgfqpoint{7.871946in}{1.226573in}}%
\pgfpathcurveto{\pgfqpoint{7.878458in}{1.220061in}}{\pgfqpoint{7.887290in}{1.216403in}}{\pgfqpoint{7.896499in}{1.216403in}}%
\pgfpathlineto{\pgfqpoint{7.896499in}{1.216403in}}%
\pgfpathclose%
\pgfusepath{stroke,fill}%
\end{pgfscope}%
\begin{pgfscope}%
\pgfpathrectangle{\pgfqpoint{1.374500in}{0.082500in}}{\pgfqpoint{2.419000in}{2.419000in}}%
\pgfusepath{clip}%
\pgfsetbuttcap%
\pgfsetroundjoin%
\definecolor{currentfill}{rgb}{0.741176,0.121569,0.003922}%
\pgfsetfillcolor{currentfill}%
\pgfsetfillopacity{0.457180}%
\pgfsetlinewidth{1.003750pt}%
\definecolor{currentstroke}{rgb}{0.741176,0.121569,0.003922}%
\pgfsetstrokecolor{currentstroke}%
\pgfsetstrokeopacity{0.457180}%
\pgfsetdash{}{0pt}%
\pgfpathmoveto{\pgfqpoint{0.805283in}{1.076723in}}%
\pgfpathcurveto{\pgfqpoint{0.814491in}{1.076723in}}{\pgfqpoint{0.823324in}{1.080381in}}{\pgfqpoint{0.829835in}{1.086892in}}%
\pgfpathcurveto{\pgfqpoint{0.836346in}{1.093404in}}{\pgfqpoint{0.840005in}{1.102236in}}{\pgfqpoint{0.840005in}{1.111445in}}%
\pgfpathcurveto{\pgfqpoint{0.840005in}{1.120653in}}{\pgfqpoint{0.836346in}{1.129486in}}{\pgfqpoint{0.829835in}{1.135997in}}%
\pgfpathcurveto{\pgfqpoint{0.823324in}{1.142508in}}{\pgfqpoint{0.814491in}{1.146167in}}{\pgfqpoint{0.805283in}{1.146167in}}%
\pgfpathcurveto{\pgfqpoint{0.796074in}{1.146167in}}{\pgfqpoint{0.787242in}{1.142508in}}{\pgfqpoint{0.780730in}{1.135997in}}%
\pgfpathcurveto{\pgfqpoint{0.774219in}{1.129486in}}{\pgfqpoint{0.770560in}{1.120653in}}{\pgfqpoint{0.770560in}{1.111445in}}%
\pgfpathcurveto{\pgfqpoint{0.770560in}{1.102236in}}{\pgfqpoint{0.774219in}{1.093404in}}{\pgfqpoint{0.780730in}{1.086892in}}%
\pgfpathcurveto{\pgfqpoint{0.787242in}{1.080381in}}{\pgfqpoint{0.796074in}{1.076723in}}{\pgfqpoint{0.805283in}{1.076723in}}%
\pgfpathlineto{\pgfqpoint{0.805283in}{1.076723in}}%
\pgfpathclose%
\pgfusepath{stroke,fill}%
\end{pgfscope}%
\begin{pgfscope}%
\pgfpathrectangle{\pgfqpoint{1.374500in}{0.082500in}}{\pgfqpoint{2.419000in}{2.419000in}}%
\pgfusepath{clip}%
\pgfsetbuttcap%
\pgfsetroundjoin%
\definecolor{currentfill}{rgb}{0.741176,0.121569,0.003922}%
\pgfsetfillcolor{currentfill}%
\pgfsetfillopacity{0.457180}%
\pgfsetlinewidth{1.003750pt}%
\definecolor{currentstroke}{rgb}{0.741176,0.121569,0.003922}%
\pgfsetstrokecolor{currentstroke}%
\pgfsetstrokeopacity{0.457180}%
\pgfsetdash{}{0pt}%
\pgfpathmoveto{\pgfqpoint{11.028164in}{1.076723in}}%
\pgfpathcurveto{\pgfqpoint{11.037372in}{1.076723in}}{\pgfqpoint{11.046205in}{1.080381in}}{\pgfqpoint{11.052716in}{1.086892in}}%
\pgfpathcurveto{\pgfqpoint{11.059227in}{1.093404in}}{\pgfqpoint{11.062886in}{1.102236in}}{\pgfqpoint{11.062886in}{1.111445in}}%
\pgfpathcurveto{\pgfqpoint{11.062886in}{1.120653in}}{\pgfqpoint{11.059227in}{1.129486in}}{\pgfqpoint{11.052716in}{1.135997in}}%
\pgfpathcurveto{\pgfqpoint{11.046205in}{1.142508in}}{\pgfqpoint{11.037372in}{1.146167in}}{\pgfqpoint{11.028164in}{1.146167in}}%
\pgfpathcurveto{\pgfqpoint{11.018955in}{1.146167in}}{\pgfqpoint{11.010123in}{1.142508in}}{\pgfqpoint{11.003611in}{1.135997in}}%
\pgfpathcurveto{\pgfqpoint{10.997100in}{1.129486in}}{\pgfqpoint{10.993442in}{1.120653in}}{\pgfqpoint{10.993442in}{1.111445in}}%
\pgfpathcurveto{\pgfqpoint{10.993442in}{1.102236in}}{\pgfqpoint{10.997100in}{1.093404in}}{\pgfqpoint{11.003611in}{1.086892in}}%
\pgfpathcurveto{\pgfqpoint{11.010123in}{1.080381in}}{\pgfqpoint{11.018955in}{1.076723in}}{\pgfqpoint{11.028164in}{1.076723in}}%
\pgfpathlineto{\pgfqpoint{11.028164in}{1.076723in}}%
\pgfpathclose%
\pgfusepath{stroke,fill}%
\end{pgfscope}%
\begin{pgfscope}%
\pgfpathrectangle{\pgfqpoint{1.374500in}{0.082500in}}{\pgfqpoint{2.419000in}{2.419000in}}%
\pgfusepath{clip}%
\pgfsetbuttcap%
\pgfsetroundjoin%
\definecolor{currentfill}{rgb}{0.741176,0.121569,0.003922}%
\pgfsetfillcolor{currentfill}%
\pgfsetfillopacity{0.457180}%
\pgfsetlinewidth{1.003750pt}%
\definecolor{currentstroke}{rgb}{0.741176,0.121569,0.003922}%
\pgfsetstrokecolor{currentstroke}%
\pgfsetstrokeopacity{0.457180}%
\pgfsetdash{}{0pt}%
\pgfpathmoveto{\pgfqpoint{5.916723in}{1.076723in}}%
\pgfpathcurveto{\pgfqpoint{5.925932in}{1.076723in}}{\pgfqpoint{5.934764in}{1.080381in}}{\pgfqpoint{5.941276in}{1.086892in}}%
\pgfpathcurveto{\pgfqpoint{5.947787in}{1.093404in}}{\pgfqpoint{5.951445in}{1.102236in}}{\pgfqpoint{5.951445in}{1.111445in}}%
\pgfpathcurveto{\pgfqpoint{5.951445in}{1.120653in}}{\pgfqpoint{5.947787in}{1.129486in}}{\pgfqpoint{5.941276in}{1.135997in}}%
\pgfpathcurveto{\pgfqpoint{5.934764in}{1.142508in}}{\pgfqpoint{5.925932in}{1.146167in}}{\pgfqpoint{5.916723in}{1.146167in}}%
\pgfpathcurveto{\pgfqpoint{5.907515in}{1.146167in}}{\pgfqpoint{5.898682in}{1.142508in}}{\pgfqpoint{5.892171in}{1.135997in}}%
\pgfpathcurveto{\pgfqpoint{5.885660in}{1.129486in}}{\pgfqpoint{5.882001in}{1.120653in}}{\pgfqpoint{5.882001in}{1.111445in}}%
\pgfpathcurveto{\pgfqpoint{5.882001in}{1.102236in}}{\pgfqpoint{5.885660in}{1.093404in}}{\pgfqpoint{5.892171in}{1.086892in}}%
\pgfpathcurveto{\pgfqpoint{5.898682in}{1.080381in}}{\pgfqpoint{5.907515in}{1.076723in}}{\pgfqpoint{5.916723in}{1.076723in}}%
\pgfpathlineto{\pgfqpoint{5.916723in}{1.076723in}}%
\pgfpathclose%
\pgfusepath{stroke,fill}%
\end{pgfscope}%
\begin{pgfscope}%
\pgfpathrectangle{\pgfqpoint{1.374500in}{0.082500in}}{\pgfqpoint{2.419000in}{2.419000in}}%
\pgfusepath{clip}%
\pgfsetbuttcap%
\pgfsetroundjoin%
\definecolor{currentfill}{rgb}{0.741176,0.121569,0.003922}%
\pgfsetfillcolor{currentfill}%
\pgfsetfillopacity{0.464328}%
\pgfsetlinewidth{1.003750pt}%
\definecolor{currentstroke}{rgb}{0.741176,0.121569,0.003922}%
\pgfsetstrokecolor{currentstroke}%
\pgfsetstrokeopacity{0.464328}%
\pgfsetdash{}{0pt}%
\pgfpathmoveto{\pgfqpoint{-1.343812in}{0.931828in}}%
\pgfpathcurveto{\pgfqpoint{-1.334604in}{0.931828in}}{\pgfqpoint{-1.325771in}{0.935486in}}{\pgfqpoint{-1.319260in}{0.941997in}}%
\pgfpathcurveto{\pgfqpoint{-1.312749in}{0.948509in}}{\pgfqpoint{-1.309090in}{0.957341in}}{\pgfqpoint{-1.309090in}{0.966550in}}%
\pgfpathcurveto{\pgfqpoint{-1.309090in}{0.975758in}}{\pgfqpoint{-1.312749in}{0.984591in}}{\pgfqpoint{-1.319260in}{0.991102in}}%
\pgfpathcurveto{\pgfqpoint{-1.325771in}{0.997613in}}{\pgfqpoint{-1.334604in}{1.001272in}}{\pgfqpoint{-1.343812in}{1.001272in}}%
\pgfpathcurveto{\pgfqpoint{-1.353021in}{1.001272in}}{\pgfqpoint{-1.361853in}{0.997613in}}{\pgfqpoint{-1.368365in}{0.991102in}}%
\pgfpathcurveto{\pgfqpoint{-1.374876in}{0.984591in}}{\pgfqpoint{-1.378535in}{0.975758in}}{\pgfqpoint{-1.378535in}{0.966550in}}%
\pgfpathcurveto{\pgfqpoint{-1.378535in}{0.957341in}}{\pgfqpoint{-1.374876in}{0.948509in}}{\pgfqpoint{-1.368365in}{0.941997in}}%
\pgfpathcurveto{\pgfqpoint{-1.361853in}{0.935486in}}{\pgfqpoint{-1.353021in}{0.931828in}}{\pgfqpoint{-1.343812in}{0.931828in}}%
\pgfpathlineto{\pgfqpoint{-1.343812in}{0.931828in}}%
\pgfpathclose%
\pgfusepath{stroke,fill}%
\end{pgfscope}%
\begin{pgfscope}%
\pgfpathrectangle{\pgfqpoint{1.374500in}{0.082500in}}{\pgfqpoint{2.419000in}{2.419000in}}%
\pgfusepath{clip}%
\pgfsetbuttcap%
\pgfsetroundjoin%
\definecolor{currentfill}{rgb}{0.741176,0.121569,0.003922}%
\pgfsetfillcolor{currentfill}%
\pgfsetfillopacity{0.464328}%
\pgfsetlinewidth{1.003750pt}%
\definecolor{currentstroke}{rgb}{0.741176,0.121569,0.003922}%
\pgfsetstrokecolor{currentstroke}%
\pgfsetstrokeopacity{0.464328}%
\pgfsetdash{}{0pt}%
\pgfpathmoveto{\pgfqpoint{9.069890in}{0.931828in}}%
\pgfpathcurveto{\pgfqpoint{9.079098in}{0.931828in}}{\pgfqpoint{9.087931in}{0.935486in}}{\pgfqpoint{9.094442in}{0.941997in}}%
\pgfpathcurveto{\pgfqpoint{9.100953in}{0.948509in}}{\pgfqpoint{9.104612in}{0.957341in}}{\pgfqpoint{9.104612in}{0.966550in}}%
\pgfpathcurveto{\pgfqpoint{9.104612in}{0.975758in}}{\pgfqpoint{9.100953in}{0.984591in}}{\pgfqpoint{9.094442in}{0.991102in}}%
\pgfpathcurveto{\pgfqpoint{9.087931in}{0.997613in}}{\pgfqpoint{9.079098in}{1.001272in}}{\pgfqpoint{9.069890in}{1.001272in}}%
\pgfpathcurveto{\pgfqpoint{9.060681in}{1.001272in}}{\pgfqpoint{9.051849in}{0.997613in}}{\pgfqpoint{9.045337in}{0.991102in}}%
\pgfpathcurveto{\pgfqpoint{9.038826in}{0.984591in}}{\pgfqpoint{9.035167in}{0.975758in}}{\pgfqpoint{9.035167in}{0.966550in}}%
\pgfpathcurveto{\pgfqpoint{9.035167in}{0.957341in}}{\pgfqpoint{9.038826in}{0.948509in}}{\pgfqpoint{9.045337in}{0.941997in}}%
\pgfpathcurveto{\pgfqpoint{9.051849in}{0.935486in}}{\pgfqpoint{9.060681in}{0.931828in}}{\pgfqpoint{9.069890in}{0.931828in}}%
\pgfpathlineto{\pgfqpoint{9.069890in}{0.931828in}}%
\pgfpathclose%
\pgfusepath{stroke,fill}%
\end{pgfscope}%
\begin{pgfscope}%
\pgfpathrectangle{\pgfqpoint{1.374500in}{0.082500in}}{\pgfqpoint{2.419000in}{2.419000in}}%
\pgfusepath{clip}%
\pgfsetbuttcap%
\pgfsetroundjoin%
\definecolor{currentfill}{rgb}{0.741176,0.121569,0.003922}%
\pgfsetfillcolor{currentfill}%
\pgfsetfillopacity{0.464328}%
\pgfsetlinewidth{1.003750pt}%
\definecolor{currentstroke}{rgb}{0.741176,0.121569,0.003922}%
\pgfsetstrokecolor{currentstroke}%
\pgfsetstrokeopacity{0.464328}%
\pgfsetdash{}{0pt}%
\pgfpathmoveto{\pgfqpoint{3.863039in}{0.931828in}}%
\pgfpathcurveto{\pgfqpoint{3.872247in}{0.931828in}}{\pgfqpoint{3.881080in}{0.935486in}}{\pgfqpoint{3.887591in}{0.941997in}}%
\pgfpathcurveto{\pgfqpoint{3.894102in}{0.948509in}}{\pgfqpoint{3.897761in}{0.957341in}}{\pgfqpoint{3.897761in}{0.966550in}}%
\pgfpathcurveto{\pgfqpoint{3.897761in}{0.975758in}}{\pgfqpoint{3.894102in}{0.984591in}}{\pgfqpoint{3.887591in}{0.991102in}}%
\pgfpathcurveto{\pgfqpoint{3.881080in}{0.997613in}}{\pgfqpoint{3.872247in}{1.001272in}}{\pgfqpoint{3.863039in}{1.001272in}}%
\pgfpathcurveto{\pgfqpoint{3.853830in}{1.001272in}}{\pgfqpoint{3.844998in}{0.997613in}}{\pgfqpoint{3.838486in}{0.991102in}}%
\pgfpathcurveto{\pgfqpoint{3.831975in}{0.984591in}}{\pgfqpoint{3.828316in}{0.975758in}}{\pgfqpoint{3.828316in}{0.966550in}}%
\pgfpathcurveto{\pgfqpoint{3.828316in}{0.957341in}}{\pgfqpoint{3.831975in}{0.948509in}}{\pgfqpoint{3.838486in}{0.941997in}}%
\pgfpathcurveto{\pgfqpoint{3.844998in}{0.935486in}}{\pgfqpoint{3.853830in}{0.931828in}}{\pgfqpoint{3.863039in}{0.931828in}}%
\pgfpathlineto{\pgfqpoint{3.863039in}{0.931828in}}%
\pgfpathclose%
\pgfusepath{stroke,fill}%
\end{pgfscope}%
\begin{pgfscope}%
\pgfpathrectangle{\pgfqpoint{1.374500in}{0.082500in}}{\pgfqpoint{2.419000in}{2.419000in}}%
\pgfusepath{clip}%
\pgfsetbuttcap%
\pgfsetroundjoin%
\definecolor{currentfill}{rgb}{0.741176,0.121569,0.003922}%
\pgfsetfillcolor{currentfill}%
\pgfsetfillopacity{0.471747}%
\pgfsetlinewidth{1.003750pt}%
\definecolor{currentstroke}{rgb}{0.741176,0.121569,0.003922}%
\pgfsetstrokecolor{currentstroke}%
\pgfsetstrokeopacity{0.471747}%
\pgfsetdash{}{0pt}%
\pgfpathmoveto{\pgfqpoint{-3.574664in}{0.781420in}}%
\pgfpathcurveto{\pgfqpoint{-3.565455in}{0.781420in}}{\pgfqpoint{-3.556623in}{0.785079in}}{\pgfqpoint{-3.550111in}{0.791590in}}%
\pgfpathcurveto{\pgfqpoint{-3.543600in}{0.798102in}}{\pgfqpoint{-3.539941in}{0.806934in}}{\pgfqpoint{-3.539941in}{0.816143in}}%
\pgfpathcurveto{\pgfqpoint{-3.539941in}{0.825351in}}{\pgfqpoint{-3.543600in}{0.834184in}}{\pgfqpoint{-3.550111in}{0.840695in}}%
\pgfpathcurveto{\pgfqpoint{-3.556623in}{0.847206in}}{\pgfqpoint{-3.565455in}{0.850865in}}{\pgfqpoint{-3.574664in}{0.850865in}}%
\pgfpathcurveto{\pgfqpoint{-3.583872in}{0.850865in}}{\pgfqpoint{-3.592705in}{0.847206in}}{\pgfqpoint{-3.599216in}{0.840695in}}%
\pgfpathcurveto{\pgfqpoint{-3.605727in}{0.834184in}}{\pgfqpoint{-3.609386in}{0.825351in}}{\pgfqpoint{-3.609386in}{0.816143in}}%
\pgfpathcurveto{\pgfqpoint{-3.609386in}{0.806934in}}{\pgfqpoint{-3.605727in}{0.798102in}}{\pgfqpoint{-3.599216in}{0.791590in}}%
\pgfpathcurveto{\pgfqpoint{-3.592705in}{0.785079in}}{\pgfqpoint{-3.583872in}{0.781420in}}{\pgfqpoint{-3.574664in}{0.781420in}}%
\pgfpathlineto{\pgfqpoint{-3.574664in}{0.781420in}}%
\pgfpathclose%
\pgfusepath{stroke,fill}%
\end{pgfscope}%
\begin{pgfscope}%
\pgfpathrectangle{\pgfqpoint{1.374500in}{0.082500in}}{\pgfqpoint{2.419000in}{2.419000in}}%
\pgfusepath{clip}%
\pgfsetbuttcap%
\pgfsetroundjoin%
\definecolor{currentfill}{rgb}{0.741176,0.121569,0.003922}%
\pgfsetfillcolor{currentfill}%
\pgfsetfillopacity{0.471747}%
\pgfsetlinewidth{1.003750pt}%
\definecolor{currentstroke}{rgb}{0.741176,0.121569,0.003922}%
\pgfsetstrokecolor{currentstroke}%
\pgfsetstrokeopacity{0.471747}%
\pgfsetdash{}{0pt}%
\pgfpathmoveto{\pgfqpoint{1.731227in}{0.781420in}}%
\pgfpathcurveto{\pgfqpoint{1.740436in}{0.781420in}}{\pgfqpoint{1.749268in}{0.785079in}}{\pgfqpoint{1.755780in}{0.791590in}}%
\pgfpathcurveto{\pgfqpoint{1.762291in}{0.798102in}}{\pgfqpoint{1.765950in}{0.806934in}}{\pgfqpoint{1.765950in}{0.816143in}}%
\pgfpathcurveto{\pgfqpoint{1.765950in}{0.825351in}}{\pgfqpoint{1.762291in}{0.834184in}}{\pgfqpoint{1.755780in}{0.840695in}}%
\pgfpathcurveto{\pgfqpoint{1.749268in}{0.847206in}}{\pgfqpoint{1.740436in}{0.850865in}}{\pgfqpoint{1.731227in}{0.850865in}}%
\pgfpathcurveto{\pgfqpoint{1.722019in}{0.850865in}}{\pgfqpoint{1.713186in}{0.847206in}}{\pgfqpoint{1.706675in}{0.840695in}}%
\pgfpathcurveto{\pgfqpoint{1.700164in}{0.834184in}}{\pgfqpoint{1.696505in}{0.825351in}}{\pgfqpoint{1.696505in}{0.816143in}}%
\pgfpathcurveto{\pgfqpoint{1.696505in}{0.806934in}}{\pgfqpoint{1.700164in}{0.798102in}}{\pgfqpoint{1.706675in}{0.791590in}}%
\pgfpathcurveto{\pgfqpoint{1.713186in}{0.785079in}}{\pgfqpoint{1.722019in}{0.781420in}}{\pgfqpoint{1.731227in}{0.781420in}}%
\pgfpathlineto{\pgfqpoint{1.731227in}{0.781420in}}%
\pgfpathclose%
\pgfusepath{stroke,fill}%
\end{pgfscope}%
\begin{pgfscope}%
\pgfpathrectangle{\pgfqpoint{1.374500in}{0.082500in}}{\pgfqpoint{2.419000in}{2.419000in}}%
\pgfusepath{clip}%
\pgfsetbuttcap%
\pgfsetroundjoin%
\definecolor{currentfill}{rgb}{0.741176,0.121569,0.003922}%
\pgfsetfillcolor{currentfill}%
\pgfsetfillopacity{0.471747}%
\pgfsetlinewidth{1.003750pt}%
\definecolor{currentstroke}{rgb}{0.741176,0.121569,0.003922}%
\pgfsetstrokecolor{currentstroke}%
\pgfsetstrokeopacity{0.471747}%
\pgfsetdash{}{0pt}%
\pgfpathmoveto{\pgfqpoint{7.037118in}{0.781420in}}%
\pgfpathcurveto{\pgfqpoint{7.046327in}{0.781420in}}{\pgfqpoint{7.055159in}{0.785079in}}{\pgfqpoint{7.061671in}{0.791590in}}%
\pgfpathcurveto{\pgfqpoint{7.068182in}{0.798102in}}{\pgfqpoint{7.071841in}{0.806934in}}{\pgfqpoint{7.071841in}{0.816143in}}%
\pgfpathcurveto{\pgfqpoint{7.071841in}{0.825351in}}{\pgfqpoint{7.068182in}{0.834184in}}{\pgfqpoint{7.061671in}{0.840695in}}%
\pgfpathcurveto{\pgfqpoint{7.055159in}{0.847206in}}{\pgfqpoint{7.046327in}{0.850865in}}{\pgfqpoint{7.037118in}{0.850865in}}%
\pgfpathcurveto{\pgfqpoint{7.027910in}{0.850865in}}{\pgfqpoint{7.019077in}{0.847206in}}{\pgfqpoint{7.012566in}{0.840695in}}%
\pgfpathcurveto{\pgfqpoint{7.006055in}{0.834184in}}{\pgfqpoint{7.002396in}{0.825351in}}{\pgfqpoint{7.002396in}{0.816143in}}%
\pgfpathcurveto{\pgfqpoint{7.002396in}{0.806934in}}{\pgfqpoint{7.006055in}{0.798102in}}{\pgfqpoint{7.012566in}{0.791590in}}%
\pgfpathcurveto{\pgfqpoint{7.019077in}{0.785079in}}{\pgfqpoint{7.027910in}{0.781420in}}{\pgfqpoint{7.037118in}{0.781420in}}%
\pgfpathlineto{\pgfqpoint{7.037118in}{0.781420in}}%
\pgfpathclose%
\pgfusepath{stroke,fill}%
\end{pgfscope}%
\begin{pgfscope}%
\pgfpathrectangle{\pgfqpoint{1.374500in}{0.082500in}}{\pgfqpoint{2.419000in}{2.419000in}}%
\pgfusepath{clip}%
\pgfsetbuttcap%
\pgfsetroundjoin%
\definecolor{currentfill}{rgb}{0.741176,0.121569,0.003922}%
\pgfsetfillcolor{currentfill}%
\pgfsetfillopacity{0.479454}%
\pgfsetlinewidth{1.003750pt}%
\definecolor{currentstroke}{rgb}{0.741176,0.121569,0.003922}%
\pgfsetstrokecolor{currentstroke}%
\pgfsetstrokeopacity{0.479454}%
\pgfsetdash{}{0pt}%
\pgfpathmoveto{\pgfqpoint{-0.483255in}{0.625181in}}%
\pgfpathcurveto{\pgfqpoint{-0.474047in}{0.625181in}}{\pgfqpoint{-0.465214in}{0.628839in}}{\pgfqpoint{-0.458703in}{0.635351in}}%
\pgfpathcurveto{\pgfqpoint{-0.452192in}{0.641862in}}{\pgfqpoint{-0.448533in}{0.650694in}}{\pgfqpoint{-0.448533in}{0.659903in}}%
\pgfpathcurveto{\pgfqpoint{-0.448533in}{0.669111in}}{\pgfqpoint{-0.452192in}{0.677944in}}{\pgfqpoint{-0.458703in}{0.684455in}}%
\pgfpathcurveto{\pgfqpoint{-0.465214in}{0.690966in}}{\pgfqpoint{-0.474047in}{0.694625in}}{\pgfqpoint{-0.483255in}{0.694625in}}%
\pgfpathcurveto{\pgfqpoint{-0.492464in}{0.694625in}}{\pgfqpoint{-0.501296in}{0.690966in}}{\pgfqpoint{-0.507808in}{0.684455in}}%
\pgfpathcurveto{\pgfqpoint{-0.514319in}{0.677944in}}{\pgfqpoint{-0.517977in}{0.669111in}}{\pgfqpoint{-0.517977in}{0.659903in}}%
\pgfpathcurveto{\pgfqpoint{-0.517977in}{0.650694in}}{\pgfqpoint{-0.514319in}{0.641862in}}{\pgfqpoint{-0.507808in}{0.635351in}}%
\pgfpathcurveto{\pgfqpoint{-0.501296in}{0.628839in}}{\pgfqpoint{-0.492464in}{0.625181in}}{\pgfqpoint{-0.483255in}{0.625181in}}%
\pgfpathlineto{\pgfqpoint{-0.483255in}{0.625181in}}%
\pgfpathclose%
\pgfusepath{stroke,fill}%
\end{pgfscope}%
\begin{pgfscope}%
\pgfpathrectangle{\pgfqpoint{1.374500in}{0.082500in}}{\pgfqpoint{2.419000in}{2.419000in}}%
\pgfusepath{clip}%
\pgfsetbuttcap%
\pgfsetroundjoin%
\definecolor{currentfill}{rgb}{0.741176,0.121569,0.003922}%
\pgfsetfillcolor{currentfill}%
\pgfsetfillopacity{0.479454}%
\pgfsetlinewidth{1.003750pt}%
\definecolor{currentstroke}{rgb}{0.741176,0.121569,0.003922}%
\pgfsetstrokecolor{currentstroke}%
\pgfsetstrokeopacity{0.479454}%
\pgfsetdash{}{0pt}%
\pgfpathmoveto{\pgfqpoint{10.334288in}{0.625181in}}%
\pgfpathcurveto{\pgfqpoint{10.343497in}{0.625181in}}{\pgfqpoint{10.352329in}{0.628839in}}{\pgfqpoint{10.358841in}{0.635351in}}%
\pgfpathcurveto{\pgfqpoint{10.365352in}{0.641862in}}{\pgfqpoint{10.369010in}{0.650694in}}{\pgfqpoint{10.369010in}{0.659903in}}%
\pgfpathcurveto{\pgfqpoint{10.369010in}{0.669111in}}{\pgfqpoint{10.365352in}{0.677944in}}{\pgfqpoint{10.358841in}{0.684455in}}%
\pgfpathcurveto{\pgfqpoint{10.352329in}{0.690966in}}{\pgfqpoint{10.343497in}{0.694625in}}{\pgfqpoint{10.334288in}{0.694625in}}%
\pgfpathcurveto{\pgfqpoint{10.325080in}{0.694625in}}{\pgfqpoint{10.316247in}{0.690966in}}{\pgfqpoint{10.309736in}{0.684455in}}%
\pgfpathcurveto{\pgfqpoint{10.303225in}{0.677944in}}{\pgfqpoint{10.299566in}{0.669111in}}{\pgfqpoint{10.299566in}{0.659903in}}%
\pgfpathcurveto{\pgfqpoint{10.299566in}{0.650694in}}{\pgfqpoint{10.303225in}{0.641862in}}{\pgfqpoint{10.309736in}{0.635351in}}%
\pgfpathcurveto{\pgfqpoint{10.316247in}{0.628839in}}{\pgfqpoint{10.325080in}{0.625181in}}{\pgfqpoint{10.334288in}{0.625181in}}%
\pgfpathlineto{\pgfqpoint{10.334288in}{0.625181in}}%
\pgfpathclose%
\pgfusepath{stroke,fill}%
\end{pgfscope}%
\begin{pgfscope}%
\pgfpathrectangle{\pgfqpoint{1.374500in}{0.082500in}}{\pgfqpoint{2.419000in}{2.419000in}}%
\pgfusepath{clip}%
\pgfsetbuttcap%
\pgfsetroundjoin%
\definecolor{currentfill}{rgb}{0.741176,0.121569,0.003922}%
\pgfsetfillcolor{currentfill}%
\pgfsetfillopacity{0.479454}%
\pgfsetlinewidth{1.003750pt}%
\definecolor{currentstroke}{rgb}{0.741176,0.121569,0.003922}%
\pgfsetstrokecolor{currentstroke}%
\pgfsetstrokeopacity{0.479454}%
\pgfsetdash{}{0pt}%
\pgfpathmoveto{\pgfqpoint{4.925517in}{0.625181in}}%
\pgfpathcurveto{\pgfqpoint{4.934725in}{0.625181in}}{\pgfqpoint{4.943557in}{0.628839in}}{\pgfqpoint{4.950069in}{0.635351in}}%
\pgfpathcurveto{\pgfqpoint{4.956580in}{0.641862in}}{\pgfqpoint{4.960239in}{0.650694in}}{\pgfqpoint{4.960239in}{0.659903in}}%
\pgfpathcurveto{\pgfqpoint{4.960239in}{0.669111in}}{\pgfqpoint{4.956580in}{0.677944in}}{\pgfqpoint{4.950069in}{0.684455in}}%
\pgfpathcurveto{\pgfqpoint{4.943557in}{0.690966in}}{\pgfqpoint{4.934725in}{0.694625in}}{\pgfqpoint{4.925517in}{0.694625in}}%
\pgfpathcurveto{\pgfqpoint{4.916308in}{0.694625in}}{\pgfqpoint{4.907476in}{0.690966in}}{\pgfqpoint{4.900964in}{0.684455in}}%
\pgfpathcurveto{\pgfqpoint{4.894453in}{0.677944in}}{\pgfqpoint{4.890794in}{0.669111in}}{\pgfqpoint{4.890794in}{0.659903in}}%
\pgfpathcurveto{\pgfqpoint{4.890794in}{0.650694in}}{\pgfqpoint{4.894453in}{0.641862in}}{\pgfqpoint{4.900964in}{0.635351in}}%
\pgfpathcurveto{\pgfqpoint{4.907476in}{0.628839in}}{\pgfqpoint{4.916308in}{0.625181in}}{\pgfqpoint{4.925517in}{0.625181in}}%
\pgfpathlineto{\pgfqpoint{4.925517in}{0.625181in}}%
\pgfpathclose%
\pgfusepath{stroke,fill}%
\end{pgfscope}%
\begin{pgfscope}%
\pgfpathrectangle{\pgfqpoint{1.374500in}{0.082500in}}{\pgfqpoint{2.419000in}{2.419000in}}%
\pgfusepath{clip}%
\pgfsetbuttcap%
\pgfsetroundjoin%
\definecolor{currentfill}{rgb}{0.741176,0.121569,0.003922}%
\pgfsetfillcolor{currentfill}%
\pgfsetfillopacity{0.487466}%
\pgfsetlinewidth{1.003750pt}%
\definecolor{currentstroke}{rgb}{0.741176,0.121569,0.003922}%
\pgfsetstrokecolor{currentstroke}%
\pgfsetstrokeopacity{0.487466}%
\pgfsetdash{}{0pt}%
\pgfpathmoveto{\pgfqpoint{-2.785313in}{0.462762in}}%
\pgfpathcurveto{\pgfqpoint{-2.776105in}{0.462762in}}{\pgfqpoint{-2.767272in}{0.466421in}}{\pgfqpoint{-2.760761in}{0.472932in}}%
\pgfpathcurveto{\pgfqpoint{-2.754249in}{0.479443in}}{\pgfqpoint{-2.750591in}{0.488276in}}{\pgfqpoint{-2.750591in}{0.497484in}}%
\pgfpathcurveto{\pgfqpoint{-2.750591in}{0.506693in}}{\pgfqpoint{-2.754249in}{0.515525in}}{\pgfqpoint{-2.760761in}{0.522037in}}%
\pgfpathcurveto{\pgfqpoint{-2.767272in}{0.528548in}}{\pgfqpoint{-2.776105in}{0.532206in}}{\pgfqpoint{-2.785313in}{0.532206in}}%
\pgfpathcurveto{\pgfqpoint{-2.794521in}{0.532206in}}{\pgfqpoint{-2.803354in}{0.528548in}}{\pgfqpoint{-2.809865in}{0.522037in}}%
\pgfpathcurveto{\pgfqpoint{-2.816377in}{0.515525in}}{\pgfqpoint{-2.820035in}{0.506693in}}{\pgfqpoint{-2.820035in}{0.497484in}}%
\pgfpathcurveto{\pgfqpoint{-2.820035in}{0.488276in}}{\pgfqpoint{-2.816377in}{0.479443in}}{\pgfqpoint{-2.809865in}{0.472932in}}%
\pgfpathcurveto{\pgfqpoint{-2.803354in}{0.466421in}}{\pgfqpoint{-2.794521in}{0.462762in}}{\pgfqpoint{-2.785313in}{0.462762in}}%
\pgfpathlineto{\pgfqpoint{-2.785313in}{0.462762in}}%
\pgfpathclose%
\pgfusepath{stroke,fill}%
\end{pgfscope}%
\begin{pgfscope}%
\pgfpathrectangle{\pgfqpoint{1.374500in}{0.082500in}}{\pgfqpoint{2.419000in}{2.419000in}}%
\pgfusepath{clip}%
\pgfsetbuttcap%
\pgfsetroundjoin%
\definecolor{currentfill}{rgb}{0.741176,0.121569,0.003922}%
\pgfsetfillcolor{currentfill}%
\pgfsetfillopacity{0.487466}%
\pgfsetlinewidth{1.003750pt}%
\definecolor{currentstroke}{rgb}{0.741176,0.121569,0.003922}%
\pgfsetstrokecolor{currentstroke}%
\pgfsetstrokeopacity{0.487466}%
\pgfsetdash{}{0pt}%
\pgfpathmoveto{\pgfqpoint{2.730408in}{0.462762in}}%
\pgfpathcurveto{\pgfqpoint{2.739616in}{0.462762in}}{\pgfqpoint{2.748449in}{0.466421in}}{\pgfqpoint{2.754960in}{0.472932in}}%
\pgfpathcurveto{\pgfqpoint{2.761472in}{0.479443in}}{\pgfqpoint{2.765130in}{0.488276in}}{\pgfqpoint{2.765130in}{0.497484in}}%
\pgfpathcurveto{\pgfqpoint{2.765130in}{0.506693in}}{\pgfqpoint{2.761472in}{0.515525in}}{\pgfqpoint{2.754960in}{0.522037in}}%
\pgfpathcurveto{\pgfqpoint{2.748449in}{0.528548in}}{\pgfqpoint{2.739616in}{0.532206in}}{\pgfqpoint{2.730408in}{0.532206in}}%
\pgfpathcurveto{\pgfqpoint{2.721200in}{0.532206in}}{\pgfqpoint{2.712367in}{0.528548in}}{\pgfqpoint{2.705856in}{0.522037in}}%
\pgfpathcurveto{\pgfqpoint{2.699344in}{0.515525in}}{\pgfqpoint{2.695686in}{0.506693in}}{\pgfqpoint{2.695686in}{0.497484in}}%
\pgfpathcurveto{\pgfqpoint{2.695686in}{0.488276in}}{\pgfqpoint{2.699344in}{0.479443in}}{\pgfqpoint{2.705856in}{0.472932in}}%
\pgfpathcurveto{\pgfqpoint{2.712367in}{0.466421in}}{\pgfqpoint{2.721200in}{0.462762in}}{\pgfqpoint{2.730408in}{0.462762in}}%
\pgfpathlineto{\pgfqpoint{2.730408in}{0.462762in}}%
\pgfpathclose%
\pgfusepath{stroke,fill}%
\end{pgfscope}%
\begin{pgfscope}%
\pgfpathrectangle{\pgfqpoint{1.374500in}{0.082500in}}{\pgfqpoint{2.419000in}{2.419000in}}%
\pgfusepath{clip}%
\pgfsetbuttcap%
\pgfsetroundjoin%
\definecolor{currentfill}{rgb}{0.741176,0.121569,0.003922}%
\pgfsetfillcolor{currentfill}%
\pgfsetfillopacity{0.487466}%
\pgfsetlinewidth{1.003750pt}%
\definecolor{currentstroke}{rgb}{0.741176,0.121569,0.003922}%
\pgfsetstrokecolor{currentstroke}%
\pgfsetstrokeopacity{0.487466}%
\pgfsetdash{}{0pt}%
\pgfpathmoveto{\pgfqpoint{8.246129in}{0.462762in}}%
\pgfpathcurveto{\pgfqpoint{8.255337in}{0.462762in}}{\pgfqpoint{8.264170in}{0.466421in}}{\pgfqpoint{8.270681in}{0.472932in}}%
\pgfpathcurveto{\pgfqpoint{8.277193in}{0.479443in}}{\pgfqpoint{8.280851in}{0.488276in}}{\pgfqpoint{8.280851in}{0.497484in}}%
\pgfpathcurveto{\pgfqpoint{8.280851in}{0.506693in}}{\pgfqpoint{8.277193in}{0.515525in}}{\pgfqpoint{8.270681in}{0.522037in}}%
\pgfpathcurveto{\pgfqpoint{8.264170in}{0.528548in}}{\pgfqpoint{8.255337in}{0.532206in}}{\pgfqpoint{8.246129in}{0.532206in}}%
\pgfpathcurveto{\pgfqpoint{8.236921in}{0.532206in}}{\pgfqpoint{8.228088in}{0.528548in}}{\pgfqpoint{8.221577in}{0.522037in}}%
\pgfpathcurveto{\pgfqpoint{8.215065in}{0.515525in}}{\pgfqpoint{8.211407in}{0.506693in}}{\pgfqpoint{8.211407in}{0.497484in}}%
\pgfpathcurveto{\pgfqpoint{8.211407in}{0.488276in}}{\pgfqpoint{8.215065in}{0.479443in}}{\pgfqpoint{8.221577in}{0.472932in}}%
\pgfpathcurveto{\pgfqpoint{8.228088in}{0.466421in}}{\pgfqpoint{8.236921in}{0.462762in}}{\pgfqpoint{8.246129in}{0.462762in}}%
\pgfpathlineto{\pgfqpoint{8.246129in}{0.462762in}}%
\pgfpathclose%
\pgfusepath{stroke,fill}%
\end{pgfscope}%
\begin{pgfscope}%
\pgfpathrectangle{\pgfqpoint{1.374500in}{0.082500in}}{\pgfqpoint{2.419000in}{2.419000in}}%
\pgfusepath{clip}%
\pgfsetbuttcap%
\pgfsetroundjoin%
\definecolor{currentfill}{rgb}{0.741176,0.121569,0.003922}%
\pgfsetfillcolor{currentfill}%
\pgfsetfillopacity{0.495801}%
\pgfsetlinewidth{1.003750pt}%
\definecolor{currentstroke}{rgb}{0.741176,0.121569,0.003922}%
\pgfsetstrokecolor{currentstroke}%
\pgfsetstrokeopacity{0.495801}%
\pgfsetdash{}{0pt}%
\pgfpathmoveto{\pgfqpoint{0.446739in}{0.293791in}}%
\pgfpathcurveto{\pgfqpoint{0.455948in}{0.293791in}}{\pgfqpoint{0.464780in}{0.297449in}}{\pgfqpoint{0.471292in}{0.303961in}}%
\pgfpathcurveto{\pgfqpoint{0.477803in}{0.310472in}}{\pgfqpoint{0.481461in}{0.319304in}}{\pgfqpoint{0.481461in}{0.328513in}}%
\pgfpathcurveto{\pgfqpoint{0.481461in}{0.337721in}}{\pgfqpoint{0.477803in}{0.346554in}}{\pgfqpoint{0.471292in}{0.353065in}}%
\pgfpathcurveto{\pgfqpoint{0.464780in}{0.359577in}}{\pgfqpoint{0.455948in}{0.363235in}}{\pgfqpoint{0.446739in}{0.363235in}}%
\pgfpathcurveto{\pgfqpoint{0.437531in}{0.363235in}}{\pgfqpoint{0.428698in}{0.359577in}}{\pgfqpoint{0.422187in}{0.353065in}}%
\pgfpathcurveto{\pgfqpoint{0.415676in}{0.346554in}}{\pgfqpoint{0.412017in}{0.337721in}}{\pgfqpoint{0.412017in}{0.328513in}}%
\pgfpathcurveto{\pgfqpoint{0.412017in}{0.319304in}}{\pgfqpoint{0.415676in}{0.310472in}}{\pgfqpoint{0.422187in}{0.303961in}}%
\pgfpathcurveto{\pgfqpoint{0.428698in}{0.297449in}}{\pgfqpoint{0.437531in}{0.293791in}}{\pgfqpoint{0.446739in}{0.293791in}}%
\pgfpathlineto{\pgfqpoint{0.446739in}{0.293791in}}%
\pgfpathclose%
\pgfusepath{stroke,fill}%
\end{pgfscope}%
\begin{pgfscope}%
\pgfpathrectangle{\pgfqpoint{1.374500in}{0.082500in}}{\pgfqpoint{2.419000in}{2.419000in}}%
\pgfusepath{clip}%
\pgfsetbuttcap%
\pgfsetroundjoin%
\definecolor{currentfill}{rgb}{0.741176,0.121569,0.003922}%
\pgfsetfillcolor{currentfill}%
\pgfsetfillopacity{0.495801}%
\pgfsetlinewidth{1.003750pt}%
\definecolor{currentstroke}{rgb}{0.741176,0.121569,0.003922}%
\pgfsetstrokecolor{currentstroke}%
\pgfsetstrokeopacity{0.495801}%
\pgfsetdash{}{0pt}%
\pgfpathmoveto{\pgfqpoint{6.073724in}{0.293791in}}%
\pgfpathcurveto{\pgfqpoint{6.082933in}{0.293791in}}{\pgfqpoint{6.091765in}{0.297449in}}{\pgfqpoint{6.098277in}{0.303961in}}%
\pgfpathcurveto{\pgfqpoint{6.104788in}{0.310472in}}{\pgfqpoint{6.108447in}{0.319304in}}{\pgfqpoint{6.108447in}{0.328513in}}%
\pgfpathcurveto{\pgfqpoint{6.108447in}{0.337721in}}{\pgfqpoint{6.104788in}{0.346554in}}{\pgfqpoint{6.098277in}{0.353065in}}%
\pgfpathcurveto{\pgfqpoint{6.091765in}{0.359577in}}{\pgfqpoint{6.082933in}{0.363235in}}{\pgfqpoint{6.073724in}{0.363235in}}%
\pgfpathcurveto{\pgfqpoint{6.064516in}{0.363235in}}{\pgfqpoint{6.055683in}{0.359577in}}{\pgfqpoint{6.049172in}{0.353065in}}%
\pgfpathcurveto{\pgfqpoint{6.042661in}{0.346554in}}{\pgfqpoint{6.039002in}{0.337721in}}{\pgfqpoint{6.039002in}{0.328513in}}%
\pgfpathcurveto{\pgfqpoint{6.039002in}{0.319304in}}{\pgfqpoint{6.042661in}{0.310472in}}{\pgfqpoint{6.049172in}{0.303961in}}%
\pgfpathcurveto{\pgfqpoint{6.055683in}{0.297449in}}{\pgfqpoint{6.064516in}{0.293791in}}{\pgfqpoint{6.073724in}{0.293791in}}%
\pgfpathlineto{\pgfqpoint{6.073724in}{0.293791in}}%
\pgfpathclose%
\pgfusepath{stroke,fill}%
\end{pgfscope}%
\begin{pgfscope}%
\pgfpathrectangle{\pgfqpoint{1.374500in}{0.082500in}}{\pgfqpoint{2.419000in}{2.419000in}}%
\pgfusepath{clip}%
\pgfsetbuttcap%
\pgfsetroundjoin%
\definecolor{currentfill}{rgb}{0.741176,0.121569,0.003922}%
\pgfsetfillcolor{currentfill}%
\pgfsetfillopacity{0.495801}%
\pgfsetlinewidth{1.003750pt}%
\definecolor{currentstroke}{rgb}{0.741176,0.121569,0.003922}%
\pgfsetstrokecolor{currentstroke}%
\pgfsetstrokeopacity{0.495801}%
\pgfsetdash{}{0pt}%
\pgfpathmoveto{\pgfqpoint{11.700710in}{0.293791in}}%
\pgfpathcurveto{\pgfqpoint{11.709918in}{0.293791in}}{\pgfqpoint{11.718751in}{0.297449in}}{\pgfqpoint{11.725262in}{0.303961in}}%
\pgfpathcurveto{\pgfqpoint{11.731773in}{0.310472in}}{\pgfqpoint{11.735432in}{0.319304in}}{\pgfqpoint{11.735432in}{0.328513in}}%
\pgfpathcurveto{\pgfqpoint{11.735432in}{0.337721in}}{\pgfqpoint{11.731773in}{0.346554in}}{\pgfqpoint{11.725262in}{0.353065in}}%
\pgfpathcurveto{\pgfqpoint{11.718751in}{0.359577in}}{\pgfqpoint{11.709918in}{0.363235in}}{\pgfqpoint{11.700710in}{0.363235in}}%
\pgfpathcurveto{\pgfqpoint{11.691501in}{0.363235in}}{\pgfqpoint{11.682669in}{0.359577in}}{\pgfqpoint{11.676157in}{0.353065in}}%
\pgfpathcurveto{\pgfqpoint{11.669646in}{0.346554in}}{\pgfqpoint{11.665987in}{0.337721in}}{\pgfqpoint{11.665987in}{0.328513in}}%
\pgfpathcurveto{\pgfqpoint{11.665987in}{0.319304in}}{\pgfqpoint{11.669646in}{0.310472in}}{\pgfqpoint{11.676157in}{0.303961in}}%
\pgfpathcurveto{\pgfqpoint{11.682669in}{0.297449in}}{\pgfqpoint{11.691501in}{0.293791in}}{\pgfqpoint{11.700710in}{0.293791in}}%
\pgfpathlineto{\pgfqpoint{11.700710in}{0.293791in}}%
\pgfpathclose%
\pgfusepath{stroke,fill}%
\end{pgfscope}%
\begin{pgfscope}%
\pgfpathrectangle{\pgfqpoint{1.374500in}{0.082500in}}{\pgfqpoint{2.419000in}{2.419000in}}%
\pgfusepath{clip}%
\pgfsetbuttcap%
\pgfsetroundjoin%
\definecolor{currentfill}{rgb}{0.741176,0.121569,0.003922}%
\pgfsetfillcolor{currentfill}%
\pgfsetfillopacity{0.504479}%
\pgfsetlinewidth{1.003750pt}%
\definecolor{currentstroke}{rgb}{0.741176,0.121569,0.003922}%
\pgfsetstrokecolor{currentstroke}%
\pgfsetstrokeopacity{0.504479}%
\pgfsetdash{}{0pt}%
\pgfpathmoveto{\pgfqpoint{-1.930960in}{0.117862in}}%
\pgfpathcurveto{\pgfqpoint{-1.921751in}{0.117862in}}{\pgfqpoint{-1.912919in}{0.121521in}}{\pgfqpoint{-1.906407in}{0.128032in}}%
\pgfpathcurveto{\pgfqpoint{-1.899896in}{0.134543in}}{\pgfqpoint{-1.896237in}{0.143376in}}{\pgfqpoint{-1.896237in}{0.152584in}}%
\pgfpathcurveto{\pgfqpoint{-1.896237in}{0.161793in}}{\pgfqpoint{-1.899896in}{0.170625in}}{\pgfqpoint{-1.906407in}{0.177137in}}%
\pgfpathcurveto{\pgfqpoint{-1.912919in}{0.183648in}}{\pgfqpoint{-1.921751in}{0.187306in}}{\pgfqpoint{-1.930960in}{0.187306in}}%
\pgfpathcurveto{\pgfqpoint{-1.940168in}{0.187306in}}{\pgfqpoint{-1.949000in}{0.183648in}}{\pgfqpoint{-1.955512in}{0.177137in}}%
\pgfpathcurveto{\pgfqpoint{-1.962023in}{0.170625in}}{\pgfqpoint{-1.965682in}{0.161793in}}{\pgfqpoint{-1.965682in}{0.152584in}}%
\pgfpathcurveto{\pgfqpoint{-1.965682in}{0.143376in}}{\pgfqpoint{-1.962023in}{0.134543in}}{\pgfqpoint{-1.955512in}{0.128032in}}%
\pgfpathcurveto{\pgfqpoint{-1.949000in}{0.121521in}}{\pgfqpoint{-1.940168in}{0.117862in}}{\pgfqpoint{-1.930960in}{0.117862in}}%
\pgfpathlineto{\pgfqpoint{-1.930960in}{0.117862in}}%
\pgfpathclose%
\pgfusepath{stroke,fill}%
\end{pgfscope}%
\begin{pgfscope}%
\pgfpathrectangle{\pgfqpoint{1.374500in}{0.082500in}}{\pgfqpoint{2.419000in}{2.419000in}}%
\pgfusepath{clip}%
\pgfsetbuttcap%
\pgfsetroundjoin%
\definecolor{currentfill}{rgb}{0.741176,0.121569,0.003922}%
\pgfsetfillcolor{currentfill}%
\pgfsetfillopacity{0.504479}%
\pgfsetlinewidth{1.003750pt}%
\definecolor{currentstroke}{rgb}{0.741176,0.121569,0.003922}%
\pgfsetstrokecolor{currentstroke}%
\pgfsetstrokeopacity{0.504479}%
\pgfsetdash{}{0pt}%
\pgfpathmoveto{\pgfqpoint{3.811871in}{0.117862in}}%
\pgfpathcurveto{\pgfqpoint{3.821080in}{0.117862in}}{\pgfqpoint{3.829912in}{0.121521in}}{\pgfqpoint{3.836423in}{0.128032in}}%
\pgfpathcurveto{\pgfqpoint{3.842935in}{0.134543in}}{\pgfqpoint{3.846593in}{0.143376in}}{\pgfqpoint{3.846593in}{0.152584in}}%
\pgfpathcurveto{\pgfqpoint{3.846593in}{0.161793in}}{\pgfqpoint{3.842935in}{0.170625in}}{\pgfqpoint{3.836423in}{0.177137in}}%
\pgfpathcurveto{\pgfqpoint{3.829912in}{0.183648in}}{\pgfqpoint{3.821080in}{0.187306in}}{\pgfqpoint{3.811871in}{0.187306in}}%
\pgfpathcurveto{\pgfqpoint{3.802663in}{0.187306in}}{\pgfqpoint{3.793830in}{0.183648in}}{\pgfqpoint{3.787319in}{0.177137in}}%
\pgfpathcurveto{\pgfqpoint{3.780807in}{0.170625in}}{\pgfqpoint{3.777149in}{0.161793in}}{\pgfqpoint{3.777149in}{0.152584in}}%
\pgfpathcurveto{\pgfqpoint{3.777149in}{0.143376in}}{\pgfqpoint{3.780807in}{0.134543in}}{\pgfqpoint{3.787319in}{0.128032in}}%
\pgfpathcurveto{\pgfqpoint{3.793830in}{0.121521in}}{\pgfqpoint{3.802663in}{0.117862in}}{\pgfqpoint{3.811871in}{0.117862in}}%
\pgfpathlineto{\pgfqpoint{3.811871in}{0.117862in}}%
\pgfpathclose%
\pgfusepath{stroke,fill}%
\end{pgfscope}%
\begin{pgfscope}%
\pgfpathrectangle{\pgfqpoint{1.374500in}{0.082500in}}{\pgfqpoint{2.419000in}{2.419000in}}%
\pgfusepath{clip}%
\pgfsetbuttcap%
\pgfsetroundjoin%
\definecolor{currentfill}{rgb}{0.741176,0.121569,0.003922}%
\pgfsetfillcolor{currentfill}%
\pgfsetfillopacity{0.504479}%
\pgfsetlinewidth{1.003750pt}%
\definecolor{currentstroke}{rgb}{0.741176,0.121569,0.003922}%
\pgfsetstrokecolor{currentstroke}%
\pgfsetstrokeopacity{0.504479}%
\pgfsetdash{}{0pt}%
\pgfpathmoveto{\pgfqpoint{9.554702in}{0.117862in}}%
\pgfpathcurveto{\pgfqpoint{9.563910in}{0.117862in}}{\pgfqpoint{9.572743in}{0.121521in}}{\pgfqpoint{9.579254in}{0.128032in}}%
\pgfpathcurveto{\pgfqpoint{9.585765in}{0.134543in}}{\pgfqpoint{9.589424in}{0.143376in}}{\pgfqpoint{9.589424in}{0.152584in}}%
\pgfpathcurveto{\pgfqpoint{9.589424in}{0.161793in}}{\pgfqpoint{9.585765in}{0.170625in}}{\pgfqpoint{9.579254in}{0.177137in}}%
\pgfpathcurveto{\pgfqpoint{9.572743in}{0.183648in}}{\pgfqpoint{9.563910in}{0.187306in}}{\pgfqpoint{9.554702in}{0.187306in}}%
\pgfpathcurveto{\pgfqpoint{9.545493in}{0.187306in}}{\pgfqpoint{9.536661in}{0.183648in}}{\pgfqpoint{9.530149in}{0.177137in}}%
\pgfpathcurveto{\pgfqpoint{9.523638in}{0.170625in}}{\pgfqpoint{9.519979in}{0.161793in}}{\pgfqpoint{9.519979in}{0.152584in}}%
\pgfpathcurveto{\pgfqpoint{9.519979in}{0.143376in}}{\pgfqpoint{9.523638in}{0.134543in}}{\pgfqpoint{9.530149in}{0.128032in}}%
\pgfpathcurveto{\pgfqpoint{9.536661in}{0.121521in}}{\pgfqpoint{9.545493in}{0.117862in}}{\pgfqpoint{9.554702in}{0.117862in}}%
\pgfpathlineto{\pgfqpoint{9.554702in}{0.117862in}}%
\pgfpathclose%
\pgfusepath{stroke,fill}%
\end{pgfscope}%
\begin{pgfscope}%
\pgfpathrectangle{\pgfqpoint{1.374500in}{0.082500in}}{\pgfqpoint{2.419000in}{2.419000in}}%
\pgfusepath{clip}%
\pgfsetbuttcap%
\pgfsetroundjoin%
\definecolor{currentfill}{rgb}{0.741176,0.121569,0.003922}%
\pgfsetfillcolor{currentfill}%
\pgfsetfillopacity{0.513522}%
\pgfsetlinewidth{1.003750pt}%
\definecolor{currentstroke}{rgb}{0.741176,0.121569,0.003922}%
\pgfsetstrokecolor{currentstroke}%
\pgfsetstrokeopacity{0.513522}%
\pgfsetdash{}{0pt}%
\pgfpathmoveto{\pgfqpoint{-4.408618in}{-0.065463in}}%
\pgfpathcurveto{\pgfqpoint{-4.399409in}{-0.065463in}}{\pgfqpoint{-4.390577in}{-0.061804in}}{\pgfqpoint{-4.384065in}{-0.055293in}}%
\pgfpathcurveto{\pgfqpoint{-4.377554in}{-0.048781in}}{\pgfqpoint{-4.373896in}{-0.039949in}}{\pgfqpoint{-4.373896in}{-0.030741in}}%
\pgfpathcurveto{\pgfqpoint{-4.373896in}{-0.021532in}}{\pgfqpoint{-4.377554in}{-0.012700in}}{\pgfqpoint{-4.384065in}{-0.006188in}}%
\pgfpathcurveto{\pgfqpoint{-4.390577in}{0.000323in}}{\pgfqpoint{-4.399409in}{0.003982in}}{\pgfqpoint{-4.408618in}{0.003982in}}%
\pgfpathcurveto{\pgfqpoint{-4.417826in}{0.003982in}}{\pgfqpoint{-4.426659in}{0.000323in}}{\pgfqpoint{-4.433170in}{-0.006188in}}%
\pgfpathcurveto{\pgfqpoint{-4.439681in}{-0.012700in}}{\pgfqpoint{-4.443340in}{-0.021532in}}{\pgfqpoint{-4.443340in}{-0.030741in}}%
\pgfpathcurveto{\pgfqpoint{-4.443340in}{-0.039949in}}{\pgfqpoint{-4.439681in}{-0.048781in}}{\pgfqpoint{-4.433170in}{-0.055293in}}%
\pgfpathcurveto{\pgfqpoint{-4.426659in}{-0.061804in}}{\pgfqpoint{-4.417826in}{-0.065463in}}{\pgfqpoint{-4.408618in}{-0.065463in}}%
\pgfpathlineto{\pgfqpoint{-4.408618in}{-0.065463in}}%
\pgfpathclose%
\pgfusepath{stroke,fill}%
\end{pgfscope}%
\begin{pgfscope}%
\pgfpathrectangle{\pgfqpoint{1.374500in}{0.082500in}}{\pgfqpoint{2.419000in}{2.419000in}}%
\pgfusepath{clip}%
\pgfsetbuttcap%
\pgfsetroundjoin%
\definecolor{currentfill}{rgb}{0.741176,0.121569,0.003922}%
\pgfsetfillcolor{currentfill}%
\pgfsetfillopacity{0.513522}%
\pgfsetlinewidth{1.003750pt}%
\definecolor{currentstroke}{rgb}{0.741176,0.121569,0.003922}%
\pgfsetstrokecolor{currentstroke}%
\pgfsetstrokeopacity{0.513522}%
\pgfsetdash{}{0pt}%
\pgfpathmoveto{\pgfqpoint{1.454928in}{-0.065463in}}%
\pgfpathcurveto{\pgfqpoint{1.464137in}{-0.065463in}}{\pgfqpoint{1.472969in}{-0.061804in}}{\pgfqpoint{1.479481in}{-0.055293in}}%
\pgfpathcurveto{\pgfqpoint{1.485992in}{-0.048781in}}{\pgfqpoint{1.489651in}{-0.039949in}}{\pgfqpoint{1.489651in}{-0.030741in}}%
\pgfpathcurveto{\pgfqpoint{1.489651in}{-0.021532in}}{\pgfqpoint{1.485992in}{-0.012700in}}{\pgfqpoint{1.479481in}{-0.006188in}}%
\pgfpathcurveto{\pgfqpoint{1.472969in}{0.000323in}}{\pgfqpoint{1.464137in}{0.003982in}}{\pgfqpoint{1.454928in}{0.003982in}}%
\pgfpathcurveto{\pgfqpoint{1.445720in}{0.003982in}}{\pgfqpoint{1.436887in}{0.000323in}}{\pgfqpoint{1.430376in}{-0.006188in}}%
\pgfpathcurveto{\pgfqpoint{1.423865in}{-0.012700in}}{\pgfqpoint{1.420206in}{-0.021532in}}{\pgfqpoint{1.420206in}{-0.030741in}}%
\pgfpathcurveto{\pgfqpoint{1.420206in}{-0.039949in}}{\pgfqpoint{1.423865in}{-0.048781in}}{\pgfqpoint{1.430376in}{-0.055293in}}%
\pgfpathcurveto{\pgfqpoint{1.436887in}{-0.061804in}}{\pgfqpoint{1.445720in}{-0.065463in}}{\pgfqpoint{1.454928in}{-0.065463in}}%
\pgfpathlineto{\pgfqpoint{1.454928in}{-0.065463in}}%
\pgfpathclose%
\pgfusepath{stroke,fill}%
\end{pgfscope}%
\begin{pgfscope}%
\pgfpathrectangle{\pgfqpoint{1.374500in}{0.082500in}}{\pgfqpoint{2.419000in}{2.419000in}}%
\pgfusepath{clip}%
\pgfsetbuttcap%
\pgfsetroundjoin%
\definecolor{currentfill}{rgb}{0.741176,0.121569,0.003922}%
\pgfsetfillcolor{currentfill}%
\pgfsetfillopacity{0.513522}%
\pgfsetlinewidth{1.003750pt}%
\definecolor{currentstroke}{rgb}{0.741176,0.121569,0.003922}%
\pgfsetstrokecolor{currentstroke}%
\pgfsetstrokeopacity{0.513522}%
\pgfsetdash{}{0pt}%
\pgfpathmoveto{\pgfqpoint{7.318475in}{-0.065463in}}%
\pgfpathcurveto{\pgfqpoint{7.327683in}{-0.065463in}}{\pgfqpoint{7.336516in}{-0.061804in}}{\pgfqpoint{7.343027in}{-0.055293in}}%
\pgfpathcurveto{\pgfqpoint{7.349538in}{-0.048781in}}{\pgfqpoint{7.353197in}{-0.039949in}}{\pgfqpoint{7.353197in}{-0.030741in}}%
\pgfpathcurveto{\pgfqpoint{7.353197in}{-0.021532in}}{\pgfqpoint{7.349538in}{-0.012700in}}{\pgfqpoint{7.343027in}{-0.006188in}}%
\pgfpathcurveto{\pgfqpoint{7.336516in}{0.000323in}}{\pgfqpoint{7.327683in}{0.003982in}}{\pgfqpoint{7.318475in}{0.003982in}}%
\pgfpathcurveto{\pgfqpoint{7.309266in}{0.003982in}}{\pgfqpoint{7.300434in}{0.000323in}}{\pgfqpoint{7.293922in}{-0.006188in}}%
\pgfpathcurveto{\pgfqpoint{7.287411in}{-0.012700in}}{\pgfqpoint{7.283752in}{-0.021532in}}{\pgfqpoint{7.283752in}{-0.030741in}}%
\pgfpathcurveto{\pgfqpoint{7.283752in}{-0.039949in}}{\pgfqpoint{7.287411in}{-0.048781in}}{\pgfqpoint{7.293922in}{-0.055293in}}%
\pgfpathcurveto{\pgfqpoint{7.300434in}{-0.061804in}}{\pgfqpoint{7.309266in}{-0.065463in}}{\pgfqpoint{7.318475in}{-0.065463in}}%
\pgfpathlineto{\pgfqpoint{7.318475in}{-0.065463in}}%
\pgfpathclose%
\pgfusepath{stroke,fill}%
\end{pgfscope}%
\begin{pgfscope}%
\pgfpathrectangle{\pgfqpoint{1.374500in}{0.082500in}}{\pgfqpoint{2.419000in}{2.419000in}}%
\pgfusepath{clip}%
\pgfsetbuttcap%
\pgfsetroundjoin%
\definecolor{currentfill}{rgb}{0.741176,0.121569,0.003922}%
\pgfsetfillcolor{currentfill}%
\pgfsetfillopacity{0.522954}%
\pgfsetlinewidth{1.003750pt}%
\definecolor{currentstroke}{rgb}{0.741176,0.121569,0.003922}%
\pgfsetstrokecolor{currentstroke}%
\pgfsetstrokeopacity{0.522954}%
\pgfsetdash{}{0pt}%
\pgfpathmoveto{\pgfqpoint{-1.003229in}{-0.256660in}}%
\pgfpathcurveto{\pgfqpoint{-0.994020in}{-0.256660in}}{\pgfqpoint{-0.985188in}{-0.253002in}}{\pgfqpoint{-0.978676in}{-0.246490in}}%
\pgfpathcurveto{\pgfqpoint{-0.972165in}{-0.239979in}}{\pgfqpoint{-0.968507in}{-0.231146in}}{\pgfqpoint{-0.968507in}{-0.221938in}}%
\pgfpathcurveto{\pgfqpoint{-0.968507in}{-0.212729in}}{\pgfqpoint{-0.972165in}{-0.203897in}}{\pgfqpoint{-0.978676in}{-0.197386in}}%
\pgfpathcurveto{\pgfqpoint{-0.985188in}{-0.190874in}}{\pgfqpoint{-0.994020in}{-0.187216in}}{\pgfqpoint{-1.003229in}{-0.187216in}}%
\pgfpathcurveto{\pgfqpoint{-1.012437in}{-0.187216in}}{\pgfqpoint{-1.021270in}{-0.190874in}}{\pgfqpoint{-1.027781in}{-0.197386in}}%
\pgfpathcurveto{\pgfqpoint{-1.034292in}{-0.203897in}}{\pgfqpoint{-1.037951in}{-0.212729in}}{\pgfqpoint{-1.037951in}{-0.221938in}}%
\pgfpathcurveto{\pgfqpoint{-1.037951in}{-0.231146in}}{\pgfqpoint{-1.034292in}{-0.239979in}}{\pgfqpoint{-1.027781in}{-0.246490in}}%
\pgfpathcurveto{\pgfqpoint{-1.021270in}{-0.253002in}}{\pgfqpoint{-1.012437in}{-0.256660in}}{\pgfqpoint{-1.003229in}{-0.256660in}}%
\pgfpathlineto{\pgfqpoint{-1.003229in}{-0.256660in}}%
\pgfpathclose%
\pgfusepath{stroke,fill}%
\end{pgfscope}%
\begin{pgfscope}%
\pgfpathrectangle{\pgfqpoint{1.374500in}{0.082500in}}{\pgfqpoint{2.419000in}{2.419000in}}%
\pgfusepath{clip}%
\pgfsetbuttcap%
\pgfsetroundjoin%
\definecolor{currentfill}{rgb}{0.741176,0.121569,0.003922}%
\pgfsetfillcolor{currentfill}%
\pgfsetfillopacity{0.522954}%
\pgfsetlinewidth{1.003750pt}%
\definecolor{currentstroke}{rgb}{0.741176,0.121569,0.003922}%
\pgfsetstrokecolor{currentstroke}%
\pgfsetstrokeopacity{0.522954}%
\pgfsetdash{}{0pt}%
\pgfpathmoveto{\pgfqpoint{4.986217in}{-0.256660in}}%
\pgfpathcurveto{\pgfqpoint{4.995425in}{-0.256660in}}{\pgfqpoint{5.004258in}{-0.253002in}}{\pgfqpoint{5.010769in}{-0.246490in}}%
\pgfpathcurveto{\pgfqpoint{5.017281in}{-0.239979in}}{\pgfqpoint{5.020939in}{-0.231146in}}{\pgfqpoint{5.020939in}{-0.221938in}}%
\pgfpathcurveto{\pgfqpoint{5.020939in}{-0.212729in}}{\pgfqpoint{5.017281in}{-0.203897in}}{\pgfqpoint{5.010769in}{-0.197386in}}%
\pgfpathcurveto{\pgfqpoint{5.004258in}{-0.190874in}}{\pgfqpoint{4.995425in}{-0.187216in}}{\pgfqpoint{4.986217in}{-0.187216in}}%
\pgfpathcurveto{\pgfqpoint{4.977009in}{-0.187216in}}{\pgfqpoint{4.968176in}{-0.190874in}}{\pgfqpoint{4.961665in}{-0.197386in}}%
\pgfpathcurveto{\pgfqpoint{4.955153in}{-0.203897in}}{\pgfqpoint{4.951495in}{-0.212729in}}{\pgfqpoint{4.951495in}{-0.221938in}}%
\pgfpathcurveto{\pgfqpoint{4.951495in}{-0.231146in}}{\pgfqpoint{4.955153in}{-0.239979in}}{\pgfqpoint{4.961665in}{-0.246490in}}%
\pgfpathcurveto{\pgfqpoint{4.968176in}{-0.253002in}}{\pgfqpoint{4.977009in}{-0.256660in}}{\pgfqpoint{4.986217in}{-0.256660in}}%
\pgfpathlineto{\pgfqpoint{4.986217in}{-0.256660in}}%
\pgfpathclose%
\pgfusepath{stroke,fill}%
\end{pgfscope}%
\begin{pgfscope}%
\pgfpathrectangle{\pgfqpoint{1.374500in}{0.082500in}}{\pgfqpoint{2.419000in}{2.419000in}}%
\pgfusepath{clip}%
\pgfsetbuttcap%
\pgfsetroundjoin%
\definecolor{currentfill}{rgb}{0.741176,0.121569,0.003922}%
\pgfsetfillcolor{currentfill}%
\pgfsetfillopacity{0.522954}%
\pgfsetlinewidth{1.003750pt}%
\definecolor{currentstroke}{rgb}{0.741176,0.121569,0.003922}%
\pgfsetstrokecolor{currentstroke}%
\pgfsetstrokeopacity{0.522954}%
\pgfsetdash{}{0pt}%
\pgfpathmoveto{\pgfqpoint{10.975663in}{-0.256660in}}%
\pgfpathcurveto{\pgfqpoint{10.984871in}{-0.256660in}}{\pgfqpoint{10.993704in}{-0.253002in}}{\pgfqpoint{11.000215in}{-0.246490in}}%
\pgfpathcurveto{\pgfqpoint{11.006726in}{-0.239979in}}{\pgfqpoint{11.010385in}{-0.231146in}}{\pgfqpoint{11.010385in}{-0.221938in}}%
\pgfpathcurveto{\pgfqpoint{11.010385in}{-0.212729in}}{\pgfqpoint{11.006726in}{-0.203897in}}{\pgfqpoint{11.000215in}{-0.197386in}}%
\pgfpathcurveto{\pgfqpoint{10.993704in}{-0.190874in}}{\pgfqpoint{10.984871in}{-0.187216in}}{\pgfqpoint{10.975663in}{-0.187216in}}%
\pgfpathcurveto{\pgfqpoint{10.966454in}{-0.187216in}}{\pgfqpoint{10.957622in}{-0.190874in}}{\pgfqpoint{10.951111in}{-0.197386in}}%
\pgfpathcurveto{\pgfqpoint{10.944599in}{-0.203897in}}{\pgfqpoint{10.940941in}{-0.212729in}}{\pgfqpoint{10.940941in}{-0.221938in}}%
\pgfpathcurveto{\pgfqpoint{10.940941in}{-0.231146in}}{\pgfqpoint{10.944599in}{-0.239979in}}{\pgfqpoint{10.951111in}{-0.246490in}}%
\pgfpathcurveto{\pgfqpoint{10.957622in}{-0.253002in}}{\pgfqpoint{10.966454in}{-0.256660in}}{\pgfqpoint{10.975663in}{-0.256660in}}%
\pgfpathlineto{\pgfqpoint{10.975663in}{-0.256660in}}%
\pgfpathclose%
\pgfusepath{stroke,fill}%
\end{pgfscope}%
\begin{pgfscope}%
\pgfpathrectangle{\pgfqpoint{1.374500in}{0.082500in}}{\pgfqpoint{2.419000in}{2.419000in}}%
\pgfusepath{clip}%
\pgfsetbuttcap%
\pgfsetroundjoin%
\definecolor{currentfill}{rgb}{0.741176,0.121569,0.003922}%
\pgfsetfillcolor{currentfill}%
\pgfsetfillopacity{0.532799}%
\pgfsetlinewidth{1.003750pt}%
\definecolor{currentstroke}{rgb}{0.741176,0.121569,0.003922}%
\pgfsetstrokecolor{currentstroke}%
\pgfsetstrokeopacity{0.532799}%
\pgfsetdash{}{0pt}%
\pgfpathmoveto{\pgfqpoint{2.551607in}{-0.456248in}}%
\pgfpathcurveto{\pgfqpoint{2.560816in}{-0.456248in}}{\pgfqpoint{2.569648in}{-0.452590in}}{\pgfqpoint{2.576160in}{-0.446078in}}%
\pgfpathcurveto{\pgfqpoint{2.582671in}{-0.439567in}}{\pgfqpoint{2.586330in}{-0.430734in}}{\pgfqpoint{2.586330in}{-0.421526in}}%
\pgfpathcurveto{\pgfqpoint{2.586330in}{-0.412318in}}{\pgfqpoint{2.582671in}{-0.403485in}}{\pgfqpoint{2.576160in}{-0.396974in}}%
\pgfpathcurveto{\pgfqpoint{2.569648in}{-0.390462in}}{\pgfqpoint{2.560816in}{-0.386804in}}{\pgfqpoint{2.551607in}{-0.386804in}}%
\pgfpathcurveto{\pgfqpoint{2.542399in}{-0.386804in}}{\pgfqpoint{2.533566in}{-0.390462in}}{\pgfqpoint{2.527055in}{-0.396974in}}%
\pgfpathcurveto{\pgfqpoint{2.520544in}{-0.403485in}}{\pgfqpoint{2.516885in}{-0.412318in}}{\pgfqpoint{2.516885in}{-0.421526in}}%
\pgfpathcurveto{\pgfqpoint{2.516885in}{-0.430734in}}{\pgfqpoint{2.520544in}{-0.439567in}}{\pgfqpoint{2.527055in}{-0.446078in}}%
\pgfpathcurveto{\pgfqpoint{2.533566in}{-0.452590in}}{\pgfqpoint{2.542399in}{-0.456248in}}{\pgfqpoint{2.551607in}{-0.456248in}}%
\pgfpathlineto{\pgfqpoint{2.551607in}{-0.456248in}}%
\pgfpathclose%
\pgfusepath{stroke,fill}%
\end{pgfscope}%
\begin{pgfscope}%
\pgfpathrectangle{\pgfqpoint{1.374500in}{0.082500in}}{\pgfqpoint{2.419000in}{2.419000in}}%
\pgfusepath{clip}%
\pgfsetbuttcap%
\pgfsetroundjoin%
\definecolor{currentfill}{rgb}{0.741176,0.121569,0.003922}%
\pgfsetfillcolor{currentfill}%
\pgfsetfillopacity{0.532799}%
\pgfsetlinewidth{1.003750pt}%
\definecolor{currentstroke}{rgb}{0.741176,0.121569,0.003922}%
\pgfsetstrokecolor{currentstroke}%
\pgfsetstrokeopacity{0.532799}%
\pgfsetdash{}{0pt}%
\pgfpathmoveto{\pgfqpoint{-3.569263in}{-0.456248in}}%
\pgfpathcurveto{\pgfqpoint{-3.560055in}{-0.456248in}}{\pgfqpoint{-3.551222in}{-0.452590in}}{\pgfqpoint{-3.544711in}{-0.446078in}}%
\pgfpathcurveto{\pgfqpoint{-3.538200in}{-0.439567in}}{\pgfqpoint{-3.534541in}{-0.430734in}}{\pgfqpoint{-3.534541in}{-0.421526in}}%
\pgfpathcurveto{\pgfqpoint{-3.534541in}{-0.412318in}}{\pgfqpoint{-3.538200in}{-0.403485in}}{\pgfqpoint{-3.544711in}{-0.396974in}}%
\pgfpathcurveto{\pgfqpoint{-3.551222in}{-0.390462in}}{\pgfqpoint{-3.560055in}{-0.386804in}}{\pgfqpoint{-3.569263in}{-0.386804in}}%
\pgfpathcurveto{\pgfqpoint{-3.578472in}{-0.386804in}}{\pgfqpoint{-3.587304in}{-0.390462in}}{\pgfqpoint{-3.593815in}{-0.396974in}}%
\pgfpathcurveto{\pgfqpoint{-3.600327in}{-0.403485in}}{\pgfqpoint{-3.603985in}{-0.412318in}}{\pgfqpoint{-3.603985in}{-0.421526in}}%
\pgfpathcurveto{\pgfqpoint{-3.603985in}{-0.430734in}}{\pgfqpoint{-3.600327in}{-0.439567in}}{\pgfqpoint{-3.593815in}{-0.446078in}}%
\pgfpathcurveto{\pgfqpoint{-3.587304in}{-0.452590in}}{\pgfqpoint{-3.578472in}{-0.456248in}}{\pgfqpoint{-3.569263in}{-0.456248in}}%
\pgfpathlineto{\pgfqpoint{-3.569263in}{-0.456248in}}%
\pgfpathclose%
\pgfusepath{stroke,fill}%
\end{pgfscope}%
\begin{pgfscope}%
\pgfpathrectangle{\pgfqpoint{1.374500in}{0.082500in}}{\pgfqpoint{2.419000in}{2.419000in}}%
\pgfusepath{clip}%
\pgfsetbuttcap%
\pgfsetroundjoin%
\definecolor{currentfill}{rgb}{0.741176,0.121569,0.003922}%
\pgfsetfillcolor{currentfill}%
\pgfsetfillopacity{0.532799}%
\pgfsetlinewidth{1.003750pt}%
\definecolor{currentstroke}{rgb}{0.741176,0.121569,0.003922}%
\pgfsetstrokecolor{currentstroke}%
\pgfsetstrokeopacity{0.532799}%
\pgfsetdash{}{0pt}%
\pgfpathmoveto{\pgfqpoint{8.672478in}{-0.456248in}}%
\pgfpathcurveto{\pgfqpoint{8.681686in}{-0.456248in}}{\pgfqpoint{8.690519in}{-0.452590in}}{\pgfqpoint{8.697030in}{-0.446078in}}%
\pgfpathcurveto{\pgfqpoint{8.703541in}{-0.439567in}}{\pgfqpoint{8.707200in}{-0.430734in}}{\pgfqpoint{8.707200in}{-0.421526in}}%
\pgfpathcurveto{\pgfqpoint{8.707200in}{-0.412318in}}{\pgfqpoint{8.703541in}{-0.403485in}}{\pgfqpoint{8.697030in}{-0.396974in}}%
\pgfpathcurveto{\pgfqpoint{8.690519in}{-0.390462in}}{\pgfqpoint{8.681686in}{-0.386804in}}{\pgfqpoint{8.672478in}{-0.386804in}}%
\pgfpathcurveto{\pgfqpoint{8.663269in}{-0.386804in}}{\pgfqpoint{8.654437in}{-0.390462in}}{\pgfqpoint{8.647925in}{-0.396974in}}%
\pgfpathcurveto{\pgfqpoint{8.641414in}{-0.403485in}}{\pgfqpoint{8.637756in}{-0.412318in}}{\pgfqpoint{8.637756in}{-0.421526in}}%
\pgfpathcurveto{\pgfqpoint{8.637756in}{-0.430734in}}{\pgfqpoint{8.641414in}{-0.439567in}}{\pgfqpoint{8.647925in}{-0.446078in}}%
\pgfpathcurveto{\pgfqpoint{8.654437in}{-0.452590in}}{\pgfqpoint{8.663269in}{-0.456248in}}{\pgfqpoint{8.672478in}{-0.456248in}}%
\pgfpathlineto{\pgfqpoint{8.672478in}{-0.456248in}}%
\pgfpathclose%
\pgfusepath{stroke,fill}%
\end{pgfscope}%
\begin{pgfscope}%
\pgfpathrectangle{\pgfqpoint{1.374500in}{0.082500in}}{\pgfqpoint{2.419000in}{2.419000in}}%
\pgfusepath{clip}%
\pgfsetbuttcap%
\pgfsetroundjoin%
\definecolor{currentfill}{rgb}{0.741176,0.121569,0.003922}%
\pgfsetfillcolor{currentfill}%
\pgfsetfillopacity{0.543086}%
\pgfsetlinewidth{1.003750pt}%
\definecolor{currentstroke}{rgb}{0.741176,0.121569,0.003922}%
\pgfsetstrokecolor{currentstroke}%
\pgfsetstrokeopacity{0.543086}%
\pgfsetdash{}{0pt}%
\pgfpathmoveto{\pgfqpoint{0.007757in}{-0.664792in}}%
\pgfpathcurveto{\pgfqpoint{0.016965in}{-0.664792in}}{\pgfqpoint{0.025798in}{-0.661133in}}{\pgfqpoint{0.032309in}{-0.654622in}}%
\pgfpathcurveto{\pgfqpoint{0.038820in}{-0.648111in}}{\pgfqpoint{0.042479in}{-0.639278in}}{\pgfqpoint{0.042479in}{-0.630070in}}%
\pgfpathcurveto{\pgfqpoint{0.042479in}{-0.620861in}}{\pgfqpoint{0.038820in}{-0.612029in}}{\pgfqpoint{0.032309in}{-0.605517in}}%
\pgfpathcurveto{\pgfqpoint{0.025798in}{-0.599006in}}{\pgfqpoint{0.016965in}{-0.595347in}}{\pgfqpoint{0.007757in}{-0.595347in}}%
\pgfpathcurveto{\pgfqpoint{-0.001452in}{-0.595347in}}{\pgfqpoint{-0.010284in}{-0.599006in}}{\pgfqpoint{-0.016796in}{-0.605517in}}%
\pgfpathcurveto{\pgfqpoint{-0.023307in}{-0.612029in}}{\pgfqpoint{-0.026966in}{-0.620861in}}{\pgfqpoint{-0.026966in}{-0.630070in}}%
\pgfpathcurveto{\pgfqpoint{-0.026966in}{-0.639278in}}{\pgfqpoint{-0.023307in}{-0.648111in}}{\pgfqpoint{-0.016796in}{-0.654622in}}%
\pgfpathcurveto{\pgfqpoint{-0.010284in}{-0.661133in}}{\pgfqpoint{-0.001452in}{-0.664792in}}{\pgfqpoint{0.007757in}{-0.664792in}}%
\pgfpathlineto{\pgfqpoint{0.007757in}{-0.664792in}}%
\pgfpathclose%
\pgfusepath{stroke,fill}%
\end{pgfscope}%
\begin{pgfscope}%
\pgfpathrectangle{\pgfqpoint{1.374500in}{0.082500in}}{\pgfqpoint{2.419000in}{2.419000in}}%
\pgfusepath{clip}%
\pgfsetbuttcap%
\pgfsetroundjoin%
\definecolor{currentfill}{rgb}{0.741176,0.121569,0.003922}%
\pgfsetfillcolor{currentfill}%
\pgfsetfillopacity{0.543086}%
\pgfsetlinewidth{1.003750pt}%
\definecolor{currentstroke}{rgb}{0.741176,0.121569,0.003922}%
\pgfsetstrokecolor{currentstroke}%
\pgfsetstrokeopacity{0.543086}%
\pgfsetdash{}{0pt}%
\pgfpathmoveto{\pgfqpoint{-6.250436in}{-0.664792in}}%
\pgfpathcurveto{\pgfqpoint{-6.241227in}{-0.664792in}}{\pgfqpoint{-6.232395in}{-0.661133in}}{\pgfqpoint{-6.225883in}{-0.654622in}}%
\pgfpathcurveto{\pgfqpoint{-6.219372in}{-0.648111in}}{\pgfqpoint{-6.215713in}{-0.639278in}}{\pgfqpoint{-6.215713in}{-0.630070in}}%
\pgfpathcurveto{\pgfqpoint{-6.215713in}{-0.620861in}}{\pgfqpoint{-6.219372in}{-0.612029in}}{\pgfqpoint{-6.225883in}{-0.605517in}}%
\pgfpathcurveto{\pgfqpoint{-6.232395in}{-0.599006in}}{\pgfqpoint{-6.241227in}{-0.595347in}}{\pgfqpoint{-6.250436in}{-0.595347in}}%
\pgfpathcurveto{\pgfqpoint{-6.259644in}{-0.595347in}}{\pgfqpoint{-6.268477in}{-0.599006in}}{\pgfqpoint{-6.274988in}{-0.605517in}}%
\pgfpathcurveto{\pgfqpoint{-6.281499in}{-0.612029in}}{\pgfqpoint{-6.285158in}{-0.620861in}}{\pgfqpoint{-6.285158in}{-0.630070in}}%
\pgfpathcurveto{\pgfqpoint{-6.285158in}{-0.639278in}}{\pgfqpoint{-6.281499in}{-0.648111in}}{\pgfqpoint{-6.274988in}{-0.654622in}}%
\pgfpathcurveto{\pgfqpoint{-6.268477in}{-0.661133in}}{\pgfqpoint{-6.259644in}{-0.664792in}}{\pgfqpoint{-6.250436in}{-0.664792in}}%
\pgfpathlineto{\pgfqpoint{-6.250436in}{-0.664792in}}%
\pgfpathclose%
\pgfusepath{stroke,fill}%
\end{pgfscope}%
\begin{pgfscope}%
\pgfpathrectangle{\pgfqpoint{1.374500in}{0.082500in}}{\pgfqpoint{2.419000in}{2.419000in}}%
\pgfusepath{clip}%
\pgfsetbuttcap%
\pgfsetroundjoin%
\definecolor{currentfill}{rgb}{0.741176,0.121569,0.003922}%
\pgfsetfillcolor{currentfill}%
\pgfsetfillopacity{0.543086}%
\pgfsetlinewidth{1.003750pt}%
\definecolor{currentstroke}{rgb}{0.741176,0.121569,0.003922}%
\pgfsetstrokecolor{currentstroke}%
\pgfsetstrokeopacity{0.543086}%
\pgfsetdash{}{0pt}%
\pgfpathmoveto{\pgfqpoint{6.265949in}{-0.664792in}}%
\pgfpathcurveto{\pgfqpoint{6.275157in}{-0.664792in}}{\pgfqpoint{6.283990in}{-0.661133in}}{\pgfqpoint{6.290501in}{-0.654622in}}%
\pgfpathcurveto{\pgfqpoint{6.297012in}{-0.648111in}}{\pgfqpoint{6.300671in}{-0.639278in}}{\pgfqpoint{6.300671in}{-0.630070in}}%
\pgfpathcurveto{\pgfqpoint{6.300671in}{-0.620861in}}{\pgfqpoint{6.297012in}{-0.612029in}}{\pgfqpoint{6.290501in}{-0.605517in}}%
\pgfpathcurveto{\pgfqpoint{6.283990in}{-0.599006in}}{\pgfqpoint{6.275157in}{-0.595347in}}{\pgfqpoint{6.265949in}{-0.595347in}}%
\pgfpathcurveto{\pgfqpoint{6.256740in}{-0.595347in}}{\pgfqpoint{6.247908in}{-0.599006in}}{\pgfqpoint{6.241396in}{-0.605517in}}%
\pgfpathcurveto{\pgfqpoint{6.234885in}{-0.612029in}}{\pgfqpoint{6.231227in}{-0.620861in}}{\pgfqpoint{6.231227in}{-0.630070in}}%
\pgfpathcurveto{\pgfqpoint{6.231227in}{-0.639278in}}{\pgfqpoint{6.234885in}{-0.648111in}}{\pgfqpoint{6.241396in}{-0.654622in}}%
\pgfpathcurveto{\pgfqpoint{6.247908in}{-0.661133in}}{\pgfqpoint{6.256740in}{-0.664792in}}{\pgfqpoint{6.265949in}{-0.664792in}}%
\pgfpathlineto{\pgfqpoint{6.265949in}{-0.664792in}}%
\pgfpathclose%
\pgfusepath{stroke,fill}%
\end{pgfscope}%
\begin{pgfscope}%
\pgfpathrectangle{\pgfqpoint{1.374500in}{0.082500in}}{\pgfqpoint{2.419000in}{2.419000in}}%
\pgfusepath{clip}%
\pgfsetbuttcap%
\pgfsetroundjoin%
\definecolor{currentfill}{rgb}{0.741176,0.121569,0.003922}%
\pgfsetfillcolor{currentfill}%
\pgfsetfillopacity{0.553846}%
\pgfsetlinewidth{1.003750pt}%
\definecolor{currentstroke}{rgb}{0.741176,0.121569,0.003922}%
\pgfsetstrokecolor{currentstroke}%
\pgfsetstrokeopacity{0.553846}%
\pgfsetdash{}{0pt}%
\pgfpathmoveto{\pgfqpoint{-2.652856in}{-0.882908in}}%
\pgfpathcurveto{\pgfqpoint{-2.643648in}{-0.882908in}}{\pgfqpoint{-2.634815in}{-0.879249in}}{\pgfqpoint{-2.628304in}{-0.872738in}}%
\pgfpathcurveto{\pgfqpoint{-2.621793in}{-0.866226in}}{\pgfqpoint{-2.618134in}{-0.857394in}}{\pgfqpoint{-2.618134in}{-0.848185in}}%
\pgfpathcurveto{\pgfqpoint{-2.618134in}{-0.838977in}}{\pgfqpoint{-2.621793in}{-0.830144in}}{\pgfqpoint{-2.628304in}{-0.823633in}}%
\pgfpathcurveto{\pgfqpoint{-2.634815in}{-0.817122in}}{\pgfqpoint{-2.643648in}{-0.813463in}}{\pgfqpoint{-2.652856in}{-0.813463in}}%
\pgfpathcurveto{\pgfqpoint{-2.662065in}{-0.813463in}}{\pgfqpoint{-2.670897in}{-0.817122in}}{\pgfqpoint{-2.677409in}{-0.823633in}}%
\pgfpathcurveto{\pgfqpoint{-2.683920in}{-0.830144in}}{\pgfqpoint{-2.687578in}{-0.838977in}}{\pgfqpoint{-2.687578in}{-0.848185in}}%
\pgfpathcurveto{\pgfqpoint{-2.687578in}{-0.857394in}}{\pgfqpoint{-2.683920in}{-0.866226in}}{\pgfqpoint{-2.677409in}{-0.872738in}}%
\pgfpathcurveto{\pgfqpoint{-2.670897in}{-0.879249in}}{\pgfqpoint{-2.662065in}{-0.882908in}}{\pgfqpoint{-2.652856in}{-0.882908in}}%
\pgfpathlineto{\pgfqpoint{-2.652856in}{-0.882908in}}%
\pgfpathclose%
\pgfusepath{stroke,fill}%
\end{pgfscope}%
\begin{pgfscope}%
\pgfpathrectangle{\pgfqpoint{1.374500in}{0.082500in}}{\pgfqpoint{2.419000in}{2.419000in}}%
\pgfusepath{clip}%
\pgfsetbuttcap%
\pgfsetroundjoin%
\definecolor{currentfill}{rgb}{0.741176,0.121569,0.003922}%
\pgfsetfillcolor{currentfill}%
\pgfsetfillopacity{0.553846}%
\pgfsetlinewidth{1.003750pt}%
\definecolor{currentstroke}{rgb}{0.741176,0.121569,0.003922}%
\pgfsetstrokecolor{currentstroke}%
\pgfsetstrokeopacity{0.553846}%
\pgfsetdash{}{0pt}%
\pgfpathmoveto{\pgfqpoint{3.748961in}{-0.882908in}}%
\pgfpathcurveto{\pgfqpoint{3.758169in}{-0.882908in}}{\pgfqpoint{3.767002in}{-0.879249in}}{\pgfqpoint{3.773513in}{-0.872738in}}%
\pgfpathcurveto{\pgfqpoint{3.780024in}{-0.866226in}}{\pgfqpoint{3.783683in}{-0.857394in}}{\pgfqpoint{3.783683in}{-0.848185in}}%
\pgfpathcurveto{\pgfqpoint{3.783683in}{-0.838977in}}{\pgfqpoint{3.780024in}{-0.830144in}}{\pgfqpoint{3.773513in}{-0.823633in}}%
\pgfpathcurveto{\pgfqpoint{3.767002in}{-0.817122in}}{\pgfqpoint{3.758169in}{-0.813463in}}{\pgfqpoint{3.748961in}{-0.813463in}}%
\pgfpathcurveto{\pgfqpoint{3.739752in}{-0.813463in}}{\pgfqpoint{3.730920in}{-0.817122in}}{\pgfqpoint{3.724408in}{-0.823633in}}%
\pgfpathcurveto{\pgfqpoint{3.717897in}{-0.830144in}}{\pgfqpoint{3.714239in}{-0.838977in}}{\pgfqpoint{3.714239in}{-0.848185in}}%
\pgfpathcurveto{\pgfqpoint{3.714239in}{-0.857394in}}{\pgfqpoint{3.717897in}{-0.866226in}}{\pgfqpoint{3.724408in}{-0.872738in}}%
\pgfpathcurveto{\pgfqpoint{3.730920in}{-0.879249in}}{\pgfqpoint{3.739752in}{-0.882908in}}{\pgfqpoint{3.748961in}{-0.882908in}}%
\pgfpathlineto{\pgfqpoint{3.748961in}{-0.882908in}}%
\pgfpathclose%
\pgfusepath{stroke,fill}%
\end{pgfscope}%
\begin{pgfscope}%
\pgfpathrectangle{\pgfqpoint{1.374500in}{0.082500in}}{\pgfqpoint{2.419000in}{2.419000in}}%
\pgfusepath{clip}%
\pgfsetbuttcap%
\pgfsetroundjoin%
\definecolor{currentfill}{rgb}{0.741176,0.121569,0.003922}%
\pgfsetfillcolor{currentfill}%
\pgfsetfillopacity{0.553846}%
\pgfsetlinewidth{1.003750pt}%
\definecolor{currentstroke}{rgb}{0.741176,0.121569,0.003922}%
\pgfsetstrokecolor{currentstroke}%
\pgfsetstrokeopacity{0.553846}%
\pgfsetdash{}{0pt}%
\pgfpathmoveto{\pgfqpoint{10.150778in}{-0.882908in}}%
\pgfpathcurveto{\pgfqpoint{10.159986in}{-0.882908in}}{\pgfqpoint{10.168819in}{-0.879249in}}{\pgfqpoint{10.175330in}{-0.872738in}}%
\pgfpathcurveto{\pgfqpoint{10.181841in}{-0.866226in}}{\pgfqpoint{10.185500in}{-0.857394in}}{\pgfqpoint{10.185500in}{-0.848185in}}%
\pgfpathcurveto{\pgfqpoint{10.185500in}{-0.838977in}}{\pgfqpoint{10.181841in}{-0.830144in}}{\pgfqpoint{10.175330in}{-0.823633in}}%
\pgfpathcurveto{\pgfqpoint{10.168819in}{-0.817122in}}{\pgfqpoint{10.159986in}{-0.813463in}}{\pgfqpoint{10.150778in}{-0.813463in}}%
\pgfpathcurveto{\pgfqpoint{10.141569in}{-0.813463in}}{\pgfqpoint{10.132737in}{-0.817122in}}{\pgfqpoint{10.126225in}{-0.823633in}}%
\pgfpathcurveto{\pgfqpoint{10.119714in}{-0.830144in}}{\pgfqpoint{10.116055in}{-0.838977in}}{\pgfqpoint{10.116055in}{-0.848185in}}%
\pgfpathcurveto{\pgfqpoint{10.116055in}{-0.857394in}}{\pgfqpoint{10.119714in}{-0.866226in}}{\pgfqpoint{10.126225in}{-0.872738in}}%
\pgfpathcurveto{\pgfqpoint{10.132737in}{-0.879249in}}{\pgfqpoint{10.141569in}{-0.882908in}}{\pgfqpoint{10.150778in}{-0.882908in}}%
\pgfpathlineto{\pgfqpoint{10.150778in}{-0.882908in}}%
\pgfpathclose%
\pgfusepath{stroke,fill}%
\end{pgfscope}%
\begin{pgfscope}%
\pgfpathrectangle{\pgfqpoint{1.374500in}{0.082500in}}{\pgfqpoint{2.419000in}{2.419000in}}%
\pgfusepath{clip}%
\pgfsetbuttcap%
\pgfsetroundjoin%
\definecolor{currentfill}{rgb}{0.741176,0.121569,0.003922}%
\pgfsetfillcolor{currentfill}%
\pgfsetfillopacity{0.565111}%
\pgfsetlinewidth{1.003750pt}%
\definecolor{currentstroke}{rgb}{0.741176,0.121569,0.003922}%
\pgfsetstrokecolor{currentstroke}%
\pgfsetstrokeopacity{0.565111}%
\pgfsetdash{}{0pt}%
\pgfpathmoveto{\pgfqpoint{-5.438459in}{-1.111270in}}%
\pgfpathcurveto{\pgfqpoint{-5.429251in}{-1.111270in}}{\pgfqpoint{-5.420418in}{-1.107611in}}{\pgfqpoint{-5.413907in}{-1.101100in}}%
\pgfpathcurveto{\pgfqpoint{-5.407395in}{-1.094589in}}{\pgfqpoint{-5.403737in}{-1.085756in}}{\pgfqpoint{-5.403737in}{-1.076548in}}%
\pgfpathcurveto{\pgfqpoint{-5.403737in}{-1.067339in}}{\pgfqpoint{-5.407395in}{-1.058507in}}{\pgfqpoint{-5.413907in}{-1.051995in}}%
\pgfpathcurveto{\pgfqpoint{-5.420418in}{-1.045484in}}{\pgfqpoint{-5.429251in}{-1.041825in}}{\pgfqpoint{-5.438459in}{-1.041825in}}%
\pgfpathcurveto{\pgfqpoint{-5.447667in}{-1.041825in}}{\pgfqpoint{-5.456500in}{-1.045484in}}{\pgfqpoint{-5.463011in}{-1.051995in}}%
\pgfpathcurveto{\pgfqpoint{-5.469523in}{-1.058507in}}{\pgfqpoint{-5.473181in}{-1.067339in}}{\pgfqpoint{-5.473181in}{-1.076548in}}%
\pgfpathcurveto{\pgfqpoint{-5.473181in}{-1.085756in}}{\pgfqpoint{-5.469523in}{-1.094589in}}{\pgfqpoint{-5.463011in}{-1.101100in}}%
\pgfpathcurveto{\pgfqpoint{-5.456500in}{-1.107611in}}{\pgfqpoint{-5.447667in}{-1.111270in}}{\pgfqpoint{-5.438459in}{-1.111270in}}%
\pgfpathlineto{\pgfqpoint{-5.438459in}{-1.111270in}}%
\pgfpathclose%
\pgfusepath{stroke,fill}%
\end{pgfscope}%
\begin{pgfscope}%
\pgfpathrectangle{\pgfqpoint{1.374500in}{0.082500in}}{\pgfqpoint{2.419000in}{2.419000in}}%
\pgfusepath{clip}%
\pgfsetbuttcap%
\pgfsetroundjoin%
\definecolor{currentfill}{rgb}{0.741176,0.121569,0.003922}%
\pgfsetfillcolor{currentfill}%
\pgfsetfillopacity{0.565111}%
\pgfsetlinewidth{1.003750pt}%
\definecolor{currentstroke}{rgb}{0.741176,0.121569,0.003922}%
\pgfsetstrokecolor{currentstroke}%
\pgfsetstrokeopacity{0.565111}%
\pgfsetdash{}{0pt}%
\pgfpathmoveto{\pgfqpoint{1.113730in}{-1.111270in}}%
\pgfpathcurveto{\pgfqpoint{1.122938in}{-1.111270in}}{\pgfqpoint{1.131771in}{-1.107611in}}{\pgfqpoint{1.138282in}{-1.101100in}}%
\pgfpathcurveto{\pgfqpoint{1.144794in}{-1.094589in}}{\pgfqpoint{1.148452in}{-1.085756in}}{\pgfqpoint{1.148452in}{-1.076548in}}%
\pgfpathcurveto{\pgfqpoint{1.148452in}{-1.067339in}}{\pgfqpoint{1.144794in}{-1.058507in}}{\pgfqpoint{1.138282in}{-1.051995in}}%
\pgfpathcurveto{\pgfqpoint{1.131771in}{-1.045484in}}{\pgfqpoint{1.122938in}{-1.041825in}}{\pgfqpoint{1.113730in}{-1.041825in}}%
\pgfpathcurveto{\pgfqpoint{1.104521in}{-1.041825in}}{\pgfqpoint{1.095689in}{-1.045484in}}{\pgfqpoint{1.089178in}{-1.051995in}}%
\pgfpathcurveto{\pgfqpoint{1.082666in}{-1.058507in}}{\pgfqpoint{1.079008in}{-1.067339in}}{\pgfqpoint{1.079008in}{-1.076548in}}%
\pgfpathcurveto{\pgfqpoint{1.079008in}{-1.085756in}}{\pgfqpoint{1.082666in}{-1.094589in}}{\pgfqpoint{1.089178in}{-1.101100in}}%
\pgfpathcurveto{\pgfqpoint{1.095689in}{-1.107611in}}{\pgfqpoint{1.104521in}{-1.111270in}}{\pgfqpoint{1.113730in}{-1.111270in}}%
\pgfpathlineto{\pgfqpoint{1.113730in}{-1.111270in}}%
\pgfpathclose%
\pgfusepath{stroke,fill}%
\end{pgfscope}%
\begin{pgfscope}%
\pgfpathrectangle{\pgfqpoint{1.374500in}{0.082500in}}{\pgfqpoint{2.419000in}{2.419000in}}%
\pgfusepath{clip}%
\pgfsetbuttcap%
\pgfsetroundjoin%
\definecolor{currentfill}{rgb}{0.741176,0.121569,0.003922}%
\pgfsetfillcolor{currentfill}%
\pgfsetfillopacity{0.565111}%
\pgfsetlinewidth{1.003750pt}%
\definecolor{currentstroke}{rgb}{0.741176,0.121569,0.003922}%
\pgfsetstrokecolor{currentstroke}%
\pgfsetstrokeopacity{0.565111}%
\pgfsetdash{}{0pt}%
\pgfpathmoveto{\pgfqpoint{7.665919in}{-1.111270in}}%
\pgfpathcurveto{\pgfqpoint{7.675127in}{-1.111270in}}{\pgfqpoint{7.683960in}{-1.107611in}}{\pgfqpoint{7.690471in}{-1.101100in}}%
\pgfpathcurveto{\pgfqpoint{7.696982in}{-1.094589in}}{\pgfqpoint{7.700641in}{-1.085756in}}{\pgfqpoint{7.700641in}{-1.076548in}}%
\pgfpathcurveto{\pgfqpoint{7.700641in}{-1.067339in}}{\pgfqpoint{7.696982in}{-1.058507in}}{\pgfqpoint{7.690471in}{-1.051995in}}%
\pgfpathcurveto{\pgfqpoint{7.683960in}{-1.045484in}}{\pgfqpoint{7.675127in}{-1.041825in}}{\pgfqpoint{7.665919in}{-1.041825in}}%
\pgfpathcurveto{\pgfqpoint{7.656710in}{-1.041825in}}{\pgfqpoint{7.647878in}{-1.045484in}}{\pgfqpoint{7.641366in}{-1.051995in}}%
\pgfpathcurveto{\pgfqpoint{7.634855in}{-1.058507in}}{\pgfqpoint{7.631197in}{-1.067339in}}{\pgfqpoint{7.631197in}{-1.076548in}}%
\pgfpathcurveto{\pgfqpoint{7.631197in}{-1.085756in}}{\pgfqpoint{7.634855in}{-1.094589in}}{\pgfqpoint{7.641366in}{-1.101100in}}%
\pgfpathcurveto{\pgfqpoint{7.647878in}{-1.107611in}}{\pgfqpoint{7.656710in}{-1.111270in}}{\pgfqpoint{7.665919in}{-1.111270in}}%
\pgfpathlineto{\pgfqpoint{7.665919in}{-1.111270in}}%
\pgfpathclose%
\pgfusepath{stroke,fill}%
\end{pgfscope}%
\begin{pgfscope}%
\pgfpathrectangle{\pgfqpoint{1.374500in}{0.082500in}}{\pgfqpoint{2.419000in}{2.419000in}}%
\pgfusepath{clip}%
\pgfsetbuttcap%
\pgfsetroundjoin%
\definecolor{currentfill}{rgb}{0.741176,0.121569,0.003922}%
\pgfsetfillcolor{currentfill}%
\pgfsetfillopacity{0.576917}%
\pgfsetlinewidth{1.003750pt}%
\definecolor{currentstroke}{rgb}{0.741176,0.121569,0.003922}%
\pgfsetstrokecolor{currentstroke}%
\pgfsetstrokeopacity{0.576917}%
\pgfsetdash{}{0pt}%
\pgfpathmoveto{\pgfqpoint{5.061518in}{-1.350618in}}%
\pgfpathcurveto{\pgfqpoint{5.070727in}{-1.350618in}}{\pgfqpoint{5.079559in}{-1.346960in}}{\pgfqpoint{5.086071in}{-1.340448in}}%
\pgfpathcurveto{\pgfqpoint{5.092582in}{-1.333937in}}{\pgfqpoint{5.096241in}{-1.325104in}}{\pgfqpoint{5.096241in}{-1.315896in}}%
\pgfpathcurveto{\pgfqpoint{5.096241in}{-1.306688in}}{\pgfqpoint{5.092582in}{-1.297855in}}{\pgfqpoint{5.086071in}{-1.291344in}}%
\pgfpathcurveto{\pgfqpoint{5.079559in}{-1.284832in}}{\pgfqpoint{5.070727in}{-1.281174in}}{\pgfqpoint{5.061518in}{-1.281174in}}%
\pgfpathcurveto{\pgfqpoint{5.052310in}{-1.281174in}}{\pgfqpoint{5.043478in}{-1.284832in}}{\pgfqpoint{5.036966in}{-1.291344in}}%
\pgfpathcurveto{\pgfqpoint{5.030455in}{-1.297855in}}{\pgfqpoint{5.026796in}{-1.306688in}}{\pgfqpoint{5.026796in}{-1.315896in}}%
\pgfpathcurveto{\pgfqpoint{5.026796in}{-1.325104in}}{\pgfqpoint{5.030455in}{-1.333937in}}{\pgfqpoint{5.036966in}{-1.340448in}}%
\pgfpathcurveto{\pgfqpoint{5.043478in}{-1.346960in}}{\pgfqpoint{5.052310in}{-1.350618in}}{\pgfqpoint{5.061518in}{-1.350618in}}%
\pgfpathlineto{\pgfqpoint{5.061518in}{-1.350618in}}%
\pgfpathclose%
\pgfusepath{stroke,fill}%
\end{pgfscope}%
\begin{pgfscope}%
\pgfpathrectangle{\pgfqpoint{1.374500in}{0.082500in}}{\pgfqpoint{2.419000in}{2.419000in}}%
\pgfusepath{clip}%
\pgfsetbuttcap%
\pgfsetroundjoin%
\definecolor{currentfill}{rgb}{0.741176,0.121569,0.003922}%
\pgfsetfillcolor{currentfill}%
\pgfsetfillopacity{0.576917}%
\pgfsetlinewidth{1.003750pt}%
\definecolor{currentstroke}{rgb}{0.741176,0.121569,0.003922}%
\pgfsetstrokecolor{currentstroke}%
\pgfsetstrokeopacity{0.576917}%
\pgfsetdash{}{0pt}%
\pgfpathmoveto{\pgfqpoint{-1.648276in}{-1.350618in}}%
\pgfpathcurveto{\pgfqpoint{-1.639068in}{-1.350618in}}{\pgfqpoint{-1.630235in}{-1.346960in}}{\pgfqpoint{-1.623724in}{-1.340448in}}%
\pgfpathcurveto{\pgfqpoint{-1.617213in}{-1.333937in}}{\pgfqpoint{-1.613554in}{-1.325104in}}{\pgfqpoint{-1.613554in}{-1.315896in}}%
\pgfpathcurveto{\pgfqpoint{-1.613554in}{-1.306688in}}{\pgfqpoint{-1.617213in}{-1.297855in}}{\pgfqpoint{-1.623724in}{-1.291344in}}%
\pgfpathcurveto{\pgfqpoint{-1.630235in}{-1.284832in}}{\pgfqpoint{-1.639068in}{-1.281174in}}{\pgfqpoint{-1.648276in}{-1.281174in}}%
\pgfpathcurveto{\pgfqpoint{-1.657485in}{-1.281174in}}{\pgfqpoint{-1.666317in}{-1.284832in}}{\pgfqpoint{-1.672829in}{-1.291344in}}%
\pgfpathcurveto{\pgfqpoint{-1.679340in}{-1.297855in}}{\pgfqpoint{-1.682999in}{-1.306688in}}{\pgfqpoint{-1.682999in}{-1.315896in}}%
\pgfpathcurveto{\pgfqpoint{-1.682999in}{-1.325104in}}{\pgfqpoint{-1.679340in}{-1.333937in}}{\pgfqpoint{-1.672829in}{-1.340448in}}%
\pgfpathcurveto{\pgfqpoint{-1.666317in}{-1.346960in}}{\pgfqpoint{-1.657485in}{-1.350618in}}{\pgfqpoint{-1.648276in}{-1.350618in}}%
\pgfpathlineto{\pgfqpoint{-1.648276in}{-1.350618in}}%
\pgfpathclose%
\pgfusepath{stroke,fill}%
\end{pgfscope}%
\begin{pgfscope}%
\pgfpathrectangle{\pgfqpoint{1.374500in}{0.082500in}}{\pgfqpoint{2.419000in}{2.419000in}}%
\pgfusepath{clip}%
\pgfsetbuttcap%
\pgfsetroundjoin%
\definecolor{currentfill}{rgb}{0.741176,0.121569,0.003922}%
\pgfsetfillcolor{currentfill}%
\pgfsetfillopacity{0.576917}%
\pgfsetlinewidth{1.003750pt}%
\definecolor{currentstroke}{rgb}{0.741176,0.121569,0.003922}%
\pgfsetstrokecolor{currentstroke}%
\pgfsetstrokeopacity{0.576917}%
\pgfsetdash{}{0pt}%
\pgfpathmoveto{\pgfqpoint{11.771313in}{-1.350618in}}%
\pgfpathcurveto{\pgfqpoint{11.780522in}{-1.350618in}}{\pgfqpoint{11.789354in}{-1.346960in}}{\pgfqpoint{11.795866in}{-1.340448in}}%
\pgfpathcurveto{\pgfqpoint{11.802377in}{-1.333937in}}{\pgfqpoint{11.806036in}{-1.325104in}}{\pgfqpoint{11.806036in}{-1.315896in}}%
\pgfpathcurveto{\pgfqpoint{11.806036in}{-1.306688in}}{\pgfqpoint{11.802377in}{-1.297855in}}{\pgfqpoint{11.795866in}{-1.291344in}}%
\pgfpathcurveto{\pgfqpoint{11.789354in}{-1.284832in}}{\pgfqpoint{11.780522in}{-1.281174in}}{\pgfqpoint{11.771313in}{-1.281174in}}%
\pgfpathcurveto{\pgfqpoint{11.762105in}{-1.281174in}}{\pgfqpoint{11.753272in}{-1.284832in}}{\pgfqpoint{11.746761in}{-1.291344in}}%
\pgfpathcurveto{\pgfqpoint{11.740250in}{-1.297855in}}{\pgfqpoint{11.736591in}{-1.306688in}}{\pgfqpoint{11.736591in}{-1.315896in}}%
\pgfpathcurveto{\pgfqpoint{11.736591in}{-1.325104in}}{\pgfqpoint{11.740250in}{-1.333937in}}{\pgfqpoint{11.746761in}{-1.340448in}}%
\pgfpathcurveto{\pgfqpoint{11.753272in}{-1.346960in}}{\pgfqpoint{11.762105in}{-1.350618in}}{\pgfqpoint{11.771313in}{-1.350618in}}%
\pgfpathlineto{\pgfqpoint{11.771313in}{-1.350618in}}%
\pgfpathclose%
\pgfusepath{stroke,fill}%
\end{pgfscope}%
\begin{pgfscope}%
\pgfpathrectangle{\pgfqpoint{1.374500in}{0.082500in}}{\pgfqpoint{2.419000in}{2.419000in}}%
\pgfusepath{clip}%
\pgfsetbuttcap%
\pgfsetroundjoin%
\definecolor{currentfill}{rgb}{0.741176,0.121569,0.003922}%
\pgfsetfillcolor{currentfill}%
\pgfsetfillopacity{0.589306}%
\pgfsetlinewidth{1.003750pt}%
\definecolor{currentstroke}{rgb}{0.741176,0.121569,0.003922}%
\pgfsetstrokecolor{currentstroke}%
\pgfsetstrokeopacity{0.589306}%
\pgfsetdash{}{0pt}%
\pgfpathmoveto{\pgfqpoint{-4.546432in}{-1.601765in}}%
\pgfpathcurveto{\pgfqpoint{-4.537224in}{-1.601765in}}{\pgfqpoint{-4.528391in}{-1.598106in}}{\pgfqpoint{-4.521880in}{-1.591595in}}%
\pgfpathcurveto{\pgfqpoint{-4.515368in}{-1.585084in}}{\pgfqpoint{-4.511710in}{-1.576251in}}{\pgfqpoint{-4.511710in}{-1.567043in}}%
\pgfpathcurveto{\pgfqpoint{-4.511710in}{-1.557834in}}{\pgfqpoint{-4.515368in}{-1.549002in}}{\pgfqpoint{-4.521880in}{-1.542490in}}%
\pgfpathcurveto{\pgfqpoint{-4.528391in}{-1.535979in}}{\pgfqpoint{-4.537224in}{-1.532321in}}{\pgfqpoint{-4.546432in}{-1.532321in}}%
\pgfpathcurveto{\pgfqpoint{-4.555640in}{-1.532321in}}{\pgfqpoint{-4.564473in}{-1.535979in}}{\pgfqpoint{-4.570984in}{-1.542490in}}%
\pgfpathcurveto{\pgfqpoint{-4.577496in}{-1.549002in}}{\pgfqpoint{-4.581154in}{-1.557834in}}{\pgfqpoint{-4.581154in}{-1.567043in}}%
\pgfpathcurveto{\pgfqpoint{-4.581154in}{-1.576251in}}{\pgfqpoint{-4.577496in}{-1.585084in}}{\pgfqpoint{-4.570984in}{-1.591595in}}%
\pgfpathcurveto{\pgfqpoint{-4.564473in}{-1.598106in}}{\pgfqpoint{-4.555640in}{-1.601765in}}{\pgfqpoint{-4.546432in}{-1.601765in}}%
\pgfpathlineto{\pgfqpoint{-4.546432in}{-1.601765in}}%
\pgfpathclose%
\pgfusepath{stroke,fill}%
\end{pgfscope}%
\begin{pgfscope}%
\pgfpathrectangle{\pgfqpoint{1.374500in}{0.082500in}}{\pgfqpoint{2.419000in}{2.419000in}}%
\pgfusepath{clip}%
\pgfsetbuttcap%
\pgfsetroundjoin%
\definecolor{currentfill}{rgb}{0.741176,0.121569,0.003922}%
\pgfsetfillcolor{currentfill}%
\pgfsetfillopacity{0.589306}%
\pgfsetlinewidth{1.003750pt}%
\definecolor{currentstroke}{rgb}{0.741176,0.121569,0.003922}%
\pgfsetstrokecolor{currentstroke}%
\pgfsetstrokeopacity{0.589306}%
\pgfsetdash{}{0pt}%
\pgfpathmoveto{\pgfqpoint{2.328738in}{-1.601765in}}%
\pgfpathcurveto{\pgfqpoint{2.337946in}{-1.601765in}}{\pgfqpoint{2.346779in}{-1.598106in}}{\pgfqpoint{2.353290in}{-1.591595in}}%
\pgfpathcurveto{\pgfqpoint{2.359802in}{-1.585084in}}{\pgfqpoint{2.363460in}{-1.576251in}}{\pgfqpoint{2.363460in}{-1.567043in}}%
\pgfpathcurveto{\pgfqpoint{2.363460in}{-1.557834in}}{\pgfqpoint{2.359802in}{-1.549002in}}{\pgfqpoint{2.353290in}{-1.542490in}}%
\pgfpathcurveto{\pgfqpoint{2.346779in}{-1.535979in}}{\pgfqpoint{2.337946in}{-1.532321in}}{\pgfqpoint{2.328738in}{-1.532321in}}%
\pgfpathcurveto{\pgfqpoint{2.319529in}{-1.532321in}}{\pgfqpoint{2.310697in}{-1.535979in}}{\pgfqpoint{2.304186in}{-1.542490in}}%
\pgfpathcurveto{\pgfqpoint{2.297674in}{-1.549002in}}{\pgfqpoint{2.294016in}{-1.557834in}}{\pgfqpoint{2.294016in}{-1.567043in}}%
\pgfpathcurveto{\pgfqpoint{2.294016in}{-1.576251in}}{\pgfqpoint{2.297674in}{-1.585084in}}{\pgfqpoint{2.304186in}{-1.591595in}}%
\pgfpathcurveto{\pgfqpoint{2.310697in}{-1.598106in}}{\pgfqpoint{2.319529in}{-1.601765in}}{\pgfqpoint{2.328738in}{-1.601765in}}%
\pgfpathlineto{\pgfqpoint{2.328738in}{-1.601765in}}%
\pgfpathclose%
\pgfusepath{stroke,fill}%
\end{pgfscope}%
\begin{pgfscope}%
\pgfpathrectangle{\pgfqpoint{1.374500in}{0.082500in}}{\pgfqpoint{2.419000in}{2.419000in}}%
\pgfusepath{clip}%
\pgfsetbuttcap%
\pgfsetroundjoin%
\definecolor{currentfill}{rgb}{0.741176,0.121569,0.003922}%
\pgfsetfillcolor{currentfill}%
\pgfsetfillopacity{0.589306}%
\pgfsetlinewidth{1.003750pt}%
\definecolor{currentstroke}{rgb}{0.741176,0.121569,0.003922}%
\pgfsetstrokecolor{currentstroke}%
\pgfsetstrokeopacity{0.589306}%
\pgfsetdash{}{0pt}%
\pgfpathmoveto{\pgfqpoint{9.203908in}{-1.601765in}}%
\pgfpathcurveto{\pgfqpoint{9.213116in}{-1.601765in}}{\pgfqpoint{9.221949in}{-1.598106in}}{\pgfqpoint{9.228460in}{-1.591595in}}%
\pgfpathcurveto{\pgfqpoint{9.234971in}{-1.585084in}}{\pgfqpoint{9.238630in}{-1.576251in}}{\pgfqpoint{9.238630in}{-1.567043in}}%
\pgfpathcurveto{\pgfqpoint{9.238630in}{-1.557834in}}{\pgfqpoint{9.234971in}{-1.549002in}}{\pgfqpoint{9.228460in}{-1.542490in}}%
\pgfpathcurveto{\pgfqpoint{9.221949in}{-1.535979in}}{\pgfqpoint{9.213116in}{-1.532321in}}{\pgfqpoint{9.203908in}{-1.532321in}}%
\pgfpathcurveto{\pgfqpoint{9.194699in}{-1.532321in}}{\pgfqpoint{9.185867in}{-1.535979in}}{\pgfqpoint{9.179355in}{-1.542490in}}%
\pgfpathcurveto{\pgfqpoint{9.172844in}{-1.549002in}}{\pgfqpoint{9.169186in}{-1.557834in}}{\pgfqpoint{9.169186in}{-1.567043in}}%
\pgfpathcurveto{\pgfqpoint{9.169186in}{-1.576251in}}{\pgfqpoint{9.172844in}{-1.585084in}}{\pgfqpoint{9.179355in}{-1.591595in}}%
\pgfpathcurveto{\pgfqpoint{9.185867in}{-1.598106in}}{\pgfqpoint{9.194699in}{-1.601765in}}{\pgfqpoint{9.203908in}{-1.601765in}}%
\pgfpathlineto{\pgfqpoint{9.203908in}{-1.601765in}}%
\pgfpathclose%
\pgfusepath{stroke,fill}%
\end{pgfscope}%
\begin{pgfscope}%
\pgfpathrectangle{\pgfqpoint{1.374500in}{0.082500in}}{\pgfqpoint{2.419000in}{2.419000in}}%
\pgfusepath{clip}%
\pgfsetbuttcap%
\pgfsetroundjoin%
\definecolor{currentfill}{rgb}{0.741176,0.121569,0.003922}%
\pgfsetfillcolor{currentfill}%
\pgfsetfillopacity{0.602321}%
\pgfsetlinewidth{1.003750pt}%
\definecolor{currentstroke}{rgb}{0.741176,0.121569,0.003922}%
\pgfsetstrokecolor{currentstroke}%
\pgfsetstrokeopacity{0.602321}%
\pgfsetdash{}{0pt}%
\pgfpathmoveto{\pgfqpoint{-0.542155in}{-1.865604in}}%
\pgfpathcurveto{\pgfqpoint{-0.532947in}{-1.865604in}}{\pgfqpoint{-0.524114in}{-1.861946in}}{\pgfqpoint{-0.517603in}{-1.855435in}}%
\pgfpathcurveto{\pgfqpoint{-0.511092in}{-1.848923in}}{\pgfqpoint{-0.507433in}{-1.840091in}}{\pgfqpoint{-0.507433in}{-1.830882in}}%
\pgfpathcurveto{\pgfqpoint{-0.507433in}{-1.821674in}}{\pgfqpoint{-0.511092in}{-1.812841in}}{\pgfqpoint{-0.517603in}{-1.806330in}}%
\pgfpathcurveto{\pgfqpoint{-0.524114in}{-1.799819in}}{\pgfqpoint{-0.532947in}{-1.796160in}}{\pgfqpoint{-0.542155in}{-1.796160in}}%
\pgfpathcurveto{\pgfqpoint{-0.551364in}{-1.796160in}}{\pgfqpoint{-0.560196in}{-1.799819in}}{\pgfqpoint{-0.566708in}{-1.806330in}}%
\pgfpathcurveto{\pgfqpoint{-0.573219in}{-1.812841in}}{\pgfqpoint{-0.576877in}{-1.821674in}}{\pgfqpoint{-0.576877in}{-1.830882in}}%
\pgfpathcurveto{\pgfqpoint{-0.576877in}{-1.840091in}}{\pgfqpoint{-0.573219in}{-1.848923in}}{\pgfqpoint{-0.566708in}{-1.855435in}}%
\pgfpathcurveto{\pgfqpoint{-0.560196in}{-1.861946in}}{\pgfqpoint{-0.551364in}{-1.865604in}}{\pgfqpoint{-0.542155in}{-1.865604in}}%
\pgfpathlineto{\pgfqpoint{-0.542155in}{-1.865604in}}%
\pgfpathclose%
\pgfusepath{stroke,fill}%
\end{pgfscope}%
\begin{pgfscope}%
\pgfpathrectangle{\pgfqpoint{1.374500in}{0.082500in}}{\pgfqpoint{2.419000in}{2.419000in}}%
\pgfusepath{clip}%
\pgfsetbuttcap%
\pgfsetroundjoin%
\definecolor{currentfill}{rgb}{0.741176,0.121569,0.003922}%
\pgfsetfillcolor{currentfill}%
\pgfsetfillopacity{0.602321}%
\pgfsetlinewidth{1.003750pt}%
\definecolor{currentstroke}{rgb}{0.741176,0.121569,0.003922}%
\pgfsetstrokecolor{currentstroke}%
\pgfsetstrokeopacity{0.602321}%
\pgfsetdash{}{0pt}%
\pgfpathmoveto{\pgfqpoint{6.506748in}{-1.865604in}}%
\pgfpathcurveto{\pgfqpoint{6.515956in}{-1.865604in}}{\pgfqpoint{6.524789in}{-1.861946in}}{\pgfqpoint{6.531300in}{-1.855435in}}%
\pgfpathcurveto{\pgfqpoint{6.537811in}{-1.848923in}}{\pgfqpoint{6.541470in}{-1.840091in}}{\pgfqpoint{6.541470in}{-1.830882in}}%
\pgfpathcurveto{\pgfqpoint{6.541470in}{-1.821674in}}{\pgfqpoint{6.537811in}{-1.812841in}}{\pgfqpoint{6.531300in}{-1.806330in}}%
\pgfpathcurveto{\pgfqpoint{6.524789in}{-1.799819in}}{\pgfqpoint{6.515956in}{-1.796160in}}{\pgfqpoint{6.506748in}{-1.796160in}}%
\pgfpathcurveto{\pgfqpoint{6.497539in}{-1.796160in}}{\pgfqpoint{6.488707in}{-1.799819in}}{\pgfqpoint{6.482195in}{-1.806330in}}%
\pgfpathcurveto{\pgfqpoint{6.475684in}{-1.812841in}}{\pgfqpoint{6.472025in}{-1.821674in}}{\pgfqpoint{6.472025in}{-1.830882in}}%
\pgfpathcurveto{\pgfqpoint{6.472025in}{-1.840091in}}{\pgfqpoint{6.475684in}{-1.848923in}}{\pgfqpoint{6.482195in}{-1.855435in}}%
\pgfpathcurveto{\pgfqpoint{6.488707in}{-1.861946in}}{\pgfqpoint{6.497539in}{-1.865604in}}{\pgfqpoint{6.506748in}{-1.865604in}}%
\pgfpathlineto{\pgfqpoint{6.506748in}{-1.865604in}}%
\pgfpathclose%
\pgfusepath{stroke,fill}%
\end{pgfscope}%
\begin{pgfscope}%
\pgfpathrectangle{\pgfqpoint{1.374500in}{0.082500in}}{\pgfqpoint{2.419000in}{2.419000in}}%
\pgfusepath{clip}%
\pgfsetbuttcap%
\pgfsetroundjoin%
\definecolor{currentfill}{rgb}{0.741176,0.121569,0.003922}%
\pgfsetfillcolor{currentfill}%
\pgfsetfillopacity{0.616010}%
\pgfsetlinewidth{1.003750pt}%
\definecolor{currentstroke}{rgb}{0.741176,0.121569,0.003922}%
\pgfsetstrokecolor{currentstroke}%
\pgfsetstrokeopacity{0.616010}%
\pgfsetdash{}{0pt}%
\pgfpathmoveto{\pgfqpoint{3.669741in}{-2.143124in}}%
\pgfpathcurveto{\pgfqpoint{3.678949in}{-2.143124in}}{\pgfqpoint{3.687782in}{-2.139465in}}{\pgfqpoint{3.694293in}{-2.132954in}}%
\pgfpathcurveto{\pgfqpoint{3.700805in}{-2.126443in}}{\pgfqpoint{3.704463in}{-2.117610in}}{\pgfqpoint{3.704463in}{-2.108402in}}%
\pgfpathcurveto{\pgfqpoint{3.704463in}{-2.099193in}}{\pgfqpoint{3.700805in}{-2.090361in}}{\pgfqpoint{3.694293in}{-2.083849in}}%
\pgfpathcurveto{\pgfqpoint{3.687782in}{-2.077338in}}{\pgfqpoint{3.678949in}{-2.073679in}}{\pgfqpoint{3.669741in}{-2.073679in}}%
\pgfpathcurveto{\pgfqpoint{3.660533in}{-2.073679in}}{\pgfqpoint{3.651700in}{-2.077338in}}{\pgfqpoint{3.645189in}{-2.083849in}}%
\pgfpathcurveto{\pgfqpoint{3.638677in}{-2.090361in}}{\pgfqpoint{3.635019in}{-2.099193in}}{\pgfqpoint{3.635019in}{-2.108402in}}%
\pgfpathcurveto{\pgfqpoint{3.635019in}{-2.117610in}}{\pgfqpoint{3.638677in}{-2.126443in}}{\pgfqpoint{3.645189in}{-2.132954in}}%
\pgfpathcurveto{\pgfqpoint{3.651700in}{-2.139465in}}{\pgfqpoint{3.660533in}{-2.143124in}}{\pgfqpoint{3.669741in}{-2.143124in}}%
\pgfpathlineto{\pgfqpoint{3.669741in}{-2.143124in}}%
\pgfpathclose%
\pgfusepath{stroke,fill}%
\end{pgfscope}%
\begin{pgfscope}%
\pgfpathrectangle{\pgfqpoint{1.374500in}{0.082500in}}{\pgfqpoint{2.419000in}{2.419000in}}%
\pgfusepath{clip}%
\pgfsetbuttcap%
\pgfsetroundjoin%
\definecolor{currentfill}{rgb}{0.741176,0.121569,0.003922}%
\pgfsetfillcolor{currentfill}%
\pgfsetfillopacity{0.616010}%
\pgfsetlinewidth{1.003750pt}%
\definecolor{currentstroke}{rgb}{0.741176,0.121569,0.003922}%
\pgfsetstrokecolor{currentstroke}%
\pgfsetstrokeopacity{0.616010}%
\pgfsetdash{}{0pt}%
\pgfpathmoveto{\pgfqpoint{-3.561903in}{-2.143124in}}%
\pgfpathcurveto{\pgfqpoint{-3.552694in}{-2.143124in}}{\pgfqpoint{-3.543862in}{-2.139465in}}{\pgfqpoint{-3.537350in}{-2.132954in}}%
\pgfpathcurveto{\pgfqpoint{-3.530839in}{-2.126443in}}{\pgfqpoint{-3.527180in}{-2.117610in}}{\pgfqpoint{-3.527180in}{-2.108402in}}%
\pgfpathcurveto{\pgfqpoint{-3.527180in}{-2.099193in}}{\pgfqpoint{-3.530839in}{-2.090361in}}{\pgfqpoint{-3.537350in}{-2.083849in}}%
\pgfpathcurveto{\pgfqpoint{-3.543862in}{-2.077338in}}{\pgfqpoint{-3.552694in}{-2.073679in}}{\pgfqpoint{-3.561903in}{-2.073679in}}%
\pgfpathcurveto{\pgfqpoint{-3.571111in}{-2.073679in}}{\pgfqpoint{-3.579944in}{-2.077338in}}{\pgfqpoint{-3.586455in}{-2.083849in}}%
\pgfpathcurveto{\pgfqpoint{-3.592966in}{-2.090361in}}{\pgfqpoint{-3.596625in}{-2.099193in}}{\pgfqpoint{-3.596625in}{-2.108402in}}%
\pgfpathcurveto{\pgfqpoint{-3.596625in}{-2.117610in}}{\pgfqpoint{-3.592966in}{-2.126443in}}{\pgfqpoint{-3.586455in}{-2.132954in}}%
\pgfpathcurveto{\pgfqpoint{-3.579944in}{-2.139465in}}{\pgfqpoint{-3.571111in}{-2.143124in}}{\pgfqpoint{-3.561903in}{-2.143124in}}%
\pgfpathlineto{\pgfqpoint{-3.561903in}{-2.143124in}}%
\pgfpathclose%
\pgfusepath{stroke,fill}%
\end{pgfscope}%
\begin{pgfscope}%
\pgfpathrectangle{\pgfqpoint{1.374500in}{0.082500in}}{\pgfqpoint{2.419000in}{2.419000in}}%
\pgfusepath{clip}%
\pgfsetbuttcap%
\pgfsetroundjoin%
\definecolor{currentfill}{rgb}{0.741176,0.121569,0.003922}%
\pgfsetfillcolor{currentfill}%
\pgfsetfillopacity{0.616010}%
\pgfsetlinewidth{1.003750pt}%
\definecolor{currentstroke}{rgb}{0.741176,0.121569,0.003922}%
\pgfsetstrokecolor{currentstroke}%
\pgfsetstrokeopacity{0.616010}%
\pgfsetdash{}{0pt}%
\pgfpathmoveto{\pgfqpoint{10.901385in}{-2.143124in}}%
\pgfpathcurveto{\pgfqpoint{10.910593in}{-2.143124in}}{\pgfqpoint{10.919426in}{-2.139465in}}{\pgfqpoint{10.925937in}{-2.132954in}}%
\pgfpathcurveto{\pgfqpoint{10.932448in}{-2.126443in}}{\pgfqpoint{10.936107in}{-2.117610in}}{\pgfqpoint{10.936107in}{-2.108402in}}%
\pgfpathcurveto{\pgfqpoint{10.936107in}{-2.099193in}}{\pgfqpoint{10.932448in}{-2.090361in}}{\pgfqpoint{10.925937in}{-2.083849in}}%
\pgfpathcurveto{\pgfqpoint{10.919426in}{-2.077338in}}{\pgfqpoint{10.910593in}{-2.073679in}}{\pgfqpoint{10.901385in}{-2.073679in}}%
\pgfpathcurveto{\pgfqpoint{10.892176in}{-2.073679in}}{\pgfqpoint{10.883344in}{-2.077338in}}{\pgfqpoint{10.876832in}{-2.083849in}}%
\pgfpathcurveto{\pgfqpoint{10.870321in}{-2.090361in}}{\pgfqpoint{10.866662in}{-2.099193in}}{\pgfqpoint{10.866662in}{-2.108402in}}%
\pgfpathcurveto{\pgfqpoint{10.866662in}{-2.117610in}}{\pgfqpoint{10.870321in}{-2.126443in}}{\pgfqpoint{10.876832in}{-2.132954in}}%
\pgfpathcurveto{\pgfqpoint{10.883344in}{-2.139465in}}{\pgfqpoint{10.892176in}{-2.143124in}}{\pgfqpoint{10.901385in}{-2.143124in}}%
\pgfpathlineto{\pgfqpoint{10.901385in}{-2.143124in}}%
\pgfpathclose%
\pgfusepath{stroke,fill}%
\end{pgfscope}%
\begin{pgfscope}%
\pgfpathrectangle{\pgfqpoint{1.374500in}{0.082500in}}{\pgfqpoint{2.419000in}{2.419000in}}%
\pgfusepath{clip}%
\pgfsetbuttcap%
\pgfsetroundjoin%
\definecolor{currentfill}{rgb}{0.741176,0.121569,0.003922}%
\pgfsetfillcolor{currentfill}%
\pgfsetfillopacity{0.630429}%
\pgfsetlinewidth{1.003750pt}%
\definecolor{currentstroke}{rgb}{0.741176,0.121569,0.003922}%
\pgfsetstrokecolor{currentstroke}%
\pgfsetstrokeopacity{0.630429}%
\pgfsetdash{}{0pt}%
\pgfpathmoveto{\pgfqpoint{8.105834in}{-2.435416in}}%
\pgfpathcurveto{\pgfqpoint{8.115042in}{-2.435416in}}{\pgfqpoint{8.123875in}{-2.431757in}}{\pgfqpoint{8.130386in}{-2.425246in}}%
\pgfpathcurveto{\pgfqpoint{8.136898in}{-2.418734in}}{\pgfqpoint{8.140556in}{-2.409902in}}{\pgfqpoint{8.140556in}{-2.400693in}}%
\pgfpathcurveto{\pgfqpoint{8.140556in}{-2.391485in}}{\pgfqpoint{8.136898in}{-2.382652in}}{\pgfqpoint{8.130386in}{-2.376141in}}%
\pgfpathcurveto{\pgfqpoint{8.123875in}{-2.369630in}}{\pgfqpoint{8.115042in}{-2.365971in}}{\pgfqpoint{8.105834in}{-2.365971in}}%
\pgfpathcurveto{\pgfqpoint{8.096626in}{-2.365971in}}{\pgfqpoint{8.087793in}{-2.369630in}}{\pgfqpoint{8.081282in}{-2.376141in}}%
\pgfpathcurveto{\pgfqpoint{8.074770in}{-2.382652in}}{\pgfqpoint{8.071112in}{-2.391485in}}{\pgfqpoint{8.071112in}{-2.400693in}}%
\pgfpathcurveto{\pgfqpoint{8.071112in}{-2.409902in}}{\pgfqpoint{8.074770in}{-2.418734in}}{\pgfqpoint{8.081282in}{-2.425246in}}%
\pgfpathcurveto{\pgfqpoint{8.087793in}{-2.431757in}}{\pgfqpoint{8.096626in}{-2.435416in}}{\pgfqpoint{8.105834in}{-2.435416in}}%
\pgfpathlineto{\pgfqpoint{8.105834in}{-2.435416in}}%
\pgfpathclose%
\pgfusepath{stroke,fill}%
\end{pgfscope}%
\begin{pgfscope}%
\pgfpathrectangle{\pgfqpoint{1.374500in}{0.082500in}}{\pgfqpoint{2.419000in}{2.419000in}}%
\pgfusepath{clip}%
\pgfsetbuttcap%
\pgfsetroundjoin%
\definecolor{currentfill}{rgb}{0.741176,0.121569,0.003922}%
\pgfsetfillcolor{currentfill}%
\pgfsetfillopacity{0.630429}%
\pgfsetlinewidth{1.003750pt}%
\definecolor{currentstroke}{rgb}{0.741176,0.121569,0.003922}%
\pgfsetstrokecolor{currentstroke}%
\pgfsetstrokeopacity{0.630429}%
\pgfsetdash{}{0pt}%
\pgfpathmoveto{\pgfqpoint{0.681722in}{-2.435416in}}%
\pgfpathcurveto{\pgfqpoint{0.690931in}{-2.435416in}}{\pgfqpoint{0.699763in}{-2.431757in}}{\pgfqpoint{0.706275in}{-2.425246in}}%
\pgfpathcurveto{\pgfqpoint{0.712786in}{-2.418734in}}{\pgfqpoint{0.716445in}{-2.409902in}}{\pgfqpoint{0.716445in}{-2.400693in}}%
\pgfpathcurveto{\pgfqpoint{0.716445in}{-2.391485in}}{\pgfqpoint{0.712786in}{-2.382652in}}{\pgfqpoint{0.706275in}{-2.376141in}}%
\pgfpathcurveto{\pgfqpoint{0.699763in}{-2.369630in}}{\pgfqpoint{0.690931in}{-2.365971in}}{\pgfqpoint{0.681722in}{-2.365971in}}%
\pgfpathcurveto{\pgfqpoint{0.672514in}{-2.365971in}}{\pgfqpoint{0.663681in}{-2.369630in}}{\pgfqpoint{0.657170in}{-2.376141in}}%
\pgfpathcurveto{\pgfqpoint{0.650659in}{-2.382652in}}{\pgfqpoint{0.647000in}{-2.391485in}}{\pgfqpoint{0.647000in}{-2.400693in}}%
\pgfpathcurveto{\pgfqpoint{0.647000in}{-2.409902in}}{\pgfqpoint{0.650659in}{-2.418734in}}{\pgfqpoint{0.657170in}{-2.425246in}}%
\pgfpathcurveto{\pgfqpoint{0.663681in}{-2.431757in}}{\pgfqpoint{0.672514in}{-2.435416in}}{\pgfqpoint{0.681722in}{-2.435416in}}%
\pgfpathlineto{\pgfqpoint{0.681722in}{-2.435416in}}%
\pgfpathclose%
\pgfusepath{stroke,fill}%
\end{pgfscope}%
\begin{pgfscope}%
\pgfpathrectangle{\pgfqpoint{1.374500in}{0.082500in}}{\pgfqpoint{2.419000in}{2.419000in}}%
\pgfusepath{clip}%
\pgfsetbuttcap%
\pgfsetroundjoin%
\definecolor{currentfill}{rgb}{0.741176,0.121569,0.003922}%
\pgfsetfillcolor{currentfill}%
\pgfsetfillopacity{0.645635}%
\pgfsetlinewidth{1.003750pt}%
\definecolor{currentstroke}{rgb}{0.741176,0.121569,0.003922}%
\pgfsetstrokecolor{currentstroke}%
\pgfsetstrokeopacity{0.645635}%
\pgfsetdash{}{0pt}%
\pgfpathmoveto{\pgfqpoint{-2.469696in}{-2.743691in}}%
\pgfpathcurveto{\pgfqpoint{-2.460487in}{-2.743691in}}{\pgfqpoint{-2.451655in}{-2.740033in}}{\pgfqpoint{-2.445143in}{-2.733521in}}%
\pgfpathcurveto{\pgfqpoint{-2.438632in}{-2.727010in}}{\pgfqpoint{-2.434974in}{-2.718177in}}{\pgfqpoint{-2.434974in}{-2.708969in}}%
\pgfpathcurveto{\pgfqpoint{-2.434974in}{-2.699760in}}{\pgfqpoint{-2.438632in}{-2.690928in}}{\pgfqpoint{-2.445143in}{-2.684417in}}%
\pgfpathcurveto{\pgfqpoint{-2.451655in}{-2.677905in}}{\pgfqpoint{-2.460487in}{-2.674247in}}{\pgfqpoint{-2.469696in}{-2.674247in}}%
\pgfpathcurveto{\pgfqpoint{-2.478904in}{-2.674247in}}{\pgfqpoint{-2.487737in}{-2.677905in}}{\pgfqpoint{-2.494248in}{-2.684417in}}%
\pgfpathcurveto{\pgfqpoint{-2.500759in}{-2.690928in}}{\pgfqpoint{-2.504418in}{-2.699760in}}{\pgfqpoint{-2.504418in}{-2.708969in}}%
\pgfpathcurveto{\pgfqpoint{-2.504418in}{-2.718177in}}{\pgfqpoint{-2.500759in}{-2.727010in}}{\pgfqpoint{-2.494248in}{-2.733521in}}%
\pgfpathcurveto{\pgfqpoint{-2.487737in}{-2.740033in}}{\pgfqpoint{-2.478904in}{-2.743691in}}{\pgfqpoint{-2.469696in}{-2.743691in}}%
\pgfpathlineto{\pgfqpoint{-2.469696in}{-2.743691in}}%
\pgfpathclose%
\pgfusepath{stroke,fill}%
\end{pgfscope}%
\begin{pgfscope}%
\pgfpathrectangle{\pgfqpoint{1.374500in}{0.082500in}}{\pgfqpoint{2.419000in}{2.419000in}}%
\pgfusepath{clip}%
\pgfsetbuttcap%
\pgfsetroundjoin%
\definecolor{currentfill}{rgb}{0.741176,0.121569,0.003922}%
\pgfsetfillcolor{currentfill}%
\pgfsetfillopacity{0.645635}%
\pgfsetlinewidth{1.003750pt}%
\definecolor{currentstroke}{rgb}{0.741176,0.121569,0.003922}%
\pgfsetstrokecolor{currentstroke}%
\pgfsetstrokeopacity{0.645635}%
\pgfsetdash{}{0pt}%
\pgfpathmoveto{\pgfqpoint{5.157409in}{-2.743691in}}%
\pgfpathcurveto{\pgfqpoint{5.166618in}{-2.743691in}}{\pgfqpoint{5.175450in}{-2.740033in}}{\pgfqpoint{5.181961in}{-2.733521in}}%
\pgfpathcurveto{\pgfqpoint{5.188473in}{-2.727010in}}{\pgfqpoint{5.192131in}{-2.718177in}}{\pgfqpoint{5.192131in}{-2.708969in}}%
\pgfpathcurveto{\pgfqpoint{5.192131in}{-2.699760in}}{\pgfqpoint{5.188473in}{-2.690928in}}{\pgfqpoint{5.181961in}{-2.684417in}}%
\pgfpathcurveto{\pgfqpoint{5.175450in}{-2.677905in}}{\pgfqpoint{5.166618in}{-2.674247in}}{\pgfqpoint{5.157409in}{-2.674247in}}%
\pgfpathcurveto{\pgfqpoint{5.148201in}{-2.674247in}}{\pgfqpoint{5.139368in}{-2.677905in}}{\pgfqpoint{5.132857in}{-2.684417in}}%
\pgfpathcurveto{\pgfqpoint{5.126345in}{-2.690928in}}{\pgfqpoint{5.122687in}{-2.699760in}}{\pgfqpoint{5.122687in}{-2.708969in}}%
\pgfpathcurveto{\pgfqpoint{5.122687in}{-2.718177in}}{\pgfqpoint{5.126345in}{-2.727010in}}{\pgfqpoint{5.132857in}{-2.733521in}}%
\pgfpathcurveto{\pgfqpoint{5.139368in}{-2.740033in}}{\pgfqpoint{5.148201in}{-2.743691in}}{\pgfqpoint{5.157409in}{-2.743691in}}%
\pgfpathlineto{\pgfqpoint{5.157409in}{-2.743691in}}%
\pgfpathclose%
\pgfusepath{stroke,fill}%
\end{pgfscope}%
\begin{pgfscope}%
\pgfpathrectangle{\pgfqpoint{1.374500in}{0.082500in}}{\pgfqpoint{2.419000in}{2.419000in}}%
\pgfusepath{clip}%
\pgfsetbuttcap%
\pgfsetroundjoin%
\definecolor{currentfill}{rgb}{0.741176,0.121569,0.003922}%
\pgfsetfillcolor{currentfill}%
\pgfsetfillopacity{0.661697}%
\pgfsetlinewidth{1.003750pt}%
\definecolor{currentstroke}{rgb}{0.741176,0.121569,0.003922}%
\pgfsetstrokecolor{currentstroke}%
\pgfsetstrokeopacity{0.661697}%
\pgfsetdash{}{0pt}%
\pgfpathmoveto{\pgfqpoint{2.043218in}{-3.069298in}}%
\pgfpathcurveto{\pgfqpoint{2.052426in}{-3.069298in}}{\pgfqpoint{2.061259in}{-3.065640in}}{\pgfqpoint{2.067770in}{-3.059129in}}%
\pgfpathcurveto{\pgfqpoint{2.074281in}{-3.052617in}}{\pgfqpoint{2.077940in}{-3.043785in}}{\pgfqpoint{2.077940in}{-3.034576in}}%
\pgfpathcurveto{\pgfqpoint{2.077940in}{-3.025368in}}{\pgfqpoint{2.074281in}{-3.016535in}}{\pgfqpoint{2.067770in}{-3.010024in}}%
\pgfpathcurveto{\pgfqpoint{2.061259in}{-3.003513in}}{\pgfqpoint{2.052426in}{-2.999854in}}{\pgfqpoint{2.043218in}{-2.999854in}}%
\pgfpathcurveto{\pgfqpoint{2.034009in}{-2.999854in}}{\pgfqpoint{2.025177in}{-3.003513in}}{\pgfqpoint{2.018665in}{-3.010024in}}%
\pgfpathcurveto{\pgfqpoint{2.012154in}{-3.016535in}}{\pgfqpoint{2.008495in}{-3.025368in}}{\pgfqpoint{2.008495in}{-3.034576in}}%
\pgfpathcurveto{\pgfqpoint{2.008495in}{-3.043785in}}{\pgfqpoint{2.012154in}{-3.052617in}}{\pgfqpoint{2.018665in}{-3.059129in}}%
\pgfpathcurveto{\pgfqpoint{2.025177in}{-3.065640in}}{\pgfqpoint{2.034009in}{-3.069298in}}{\pgfqpoint{2.043218in}{-3.069298in}}%
\pgfpathlineto{\pgfqpoint{2.043218in}{-3.069298in}}%
\pgfpathclose%
\pgfusepath{stroke,fill}%
\end{pgfscope}%
\begin{pgfscope}%
\pgfpathrectangle{\pgfqpoint{1.374500in}{0.082500in}}{\pgfqpoint{2.419000in}{2.419000in}}%
\pgfusepath{clip}%
\pgfsetbuttcap%
\pgfsetroundjoin%
\definecolor{currentfill}{rgb}{0.741176,0.121569,0.003922}%
\pgfsetfillcolor{currentfill}%
\pgfsetfillopacity{0.661697}%
\pgfsetlinewidth{1.003750pt}%
\definecolor{currentstroke}{rgb}{0.741176,0.121569,0.003922}%
\pgfsetstrokecolor{currentstroke}%
\pgfsetstrokeopacity{0.661697}%
\pgfsetdash{}{0pt}%
\pgfpathmoveto{\pgfqpoint{9.884728in}{-3.069298in}}%
\pgfpathcurveto{\pgfqpoint{9.893937in}{-3.069298in}}{\pgfqpoint{9.902769in}{-3.065640in}}{\pgfqpoint{9.909281in}{-3.059129in}}%
\pgfpathcurveto{\pgfqpoint{9.915792in}{-3.052617in}}{\pgfqpoint{9.919451in}{-3.043785in}}{\pgfqpoint{9.919451in}{-3.034576in}}%
\pgfpathcurveto{\pgfqpoint{9.919451in}{-3.025368in}}{\pgfqpoint{9.915792in}{-3.016535in}}{\pgfqpoint{9.909281in}{-3.010024in}}%
\pgfpathcurveto{\pgfqpoint{9.902769in}{-3.003513in}}{\pgfqpoint{9.893937in}{-2.999854in}}{\pgfqpoint{9.884728in}{-2.999854in}}%
\pgfpathcurveto{\pgfqpoint{9.875520in}{-2.999854in}}{\pgfqpoint{9.866687in}{-3.003513in}}{\pgfqpoint{9.860176in}{-3.010024in}}%
\pgfpathcurveto{\pgfqpoint{9.853665in}{-3.016535in}}{\pgfqpoint{9.850006in}{-3.025368in}}{\pgfqpoint{9.850006in}{-3.034576in}}%
\pgfpathcurveto{\pgfqpoint{9.850006in}{-3.043785in}}{\pgfqpoint{9.853665in}{-3.052617in}}{\pgfqpoint{9.860176in}{-3.059129in}}%
\pgfpathcurveto{\pgfqpoint{9.866687in}{-3.065640in}}{\pgfqpoint{9.875520in}{-3.069298in}}{\pgfqpoint{9.884728in}{-3.069298in}}%
\pgfpathlineto{\pgfqpoint{9.884728in}{-3.069298in}}%
\pgfpathclose%
\pgfusepath{stroke,fill}%
\end{pgfscope}%
\begin{pgfscope}%
\pgfpathrectangle{\pgfqpoint{1.374500in}{0.082500in}}{\pgfqpoint{2.419000in}{2.419000in}}%
\pgfusepath{clip}%
\pgfsetbuttcap%
\pgfsetroundjoin%
\definecolor{currentfill}{rgb}{0.741176,0.121569,0.003922}%
\pgfsetfillcolor{currentfill}%
\pgfsetfillopacity{0.678688}%
\pgfsetlinewidth{1.003750pt}%
\definecolor{currentstroke}{rgb}{0.741176,0.121569,0.003922}%
\pgfsetstrokecolor{currentstroke}%
\pgfsetstrokeopacity{0.678688}%
\pgfsetdash{}{0pt}%
\pgfpathmoveto{\pgfqpoint{6.817195in}{-3.413742in}}%
\pgfpathcurveto{\pgfqpoint{6.826404in}{-3.413742in}}{\pgfqpoint{6.835236in}{-3.410083in}}{\pgfqpoint{6.841748in}{-3.403572in}}%
\pgfpathcurveto{\pgfqpoint{6.848259in}{-3.397060in}}{\pgfqpoint{6.851917in}{-3.388228in}}{\pgfqpoint{6.851917in}{-3.379019in}}%
\pgfpathcurveto{\pgfqpoint{6.851917in}{-3.369811in}}{\pgfqpoint{6.848259in}{-3.360979in}}{\pgfqpoint{6.841748in}{-3.354467in}}%
\pgfpathcurveto{\pgfqpoint{6.835236in}{-3.347956in}}{\pgfqpoint{6.826404in}{-3.344297in}}{\pgfqpoint{6.817195in}{-3.344297in}}%
\pgfpathcurveto{\pgfqpoint{6.807987in}{-3.344297in}}{\pgfqpoint{6.799154in}{-3.347956in}}{\pgfqpoint{6.792643in}{-3.354467in}}%
\pgfpathcurveto{\pgfqpoint{6.786132in}{-3.360979in}}{\pgfqpoint{6.782473in}{-3.369811in}}{\pgfqpoint{6.782473in}{-3.379019in}}%
\pgfpathcurveto{\pgfqpoint{6.782473in}{-3.388228in}}{\pgfqpoint{6.786132in}{-3.397060in}}{\pgfqpoint{6.792643in}{-3.403572in}}%
\pgfpathcurveto{\pgfqpoint{6.799154in}{-3.410083in}}{\pgfqpoint{6.807987in}{-3.413742in}}{\pgfqpoint{6.817195in}{-3.413742in}}%
\pgfpathlineto{\pgfqpoint{6.817195in}{-3.413742in}}%
\pgfpathclose%
\pgfusepath{stroke,fill}%
\end{pgfscope}%
\begin{pgfscope}%
\pgfpathrectangle{\pgfqpoint{1.374500in}{0.082500in}}{\pgfqpoint{2.419000in}{2.419000in}}%
\pgfusepath{clip}%
\pgfsetbuttcap%
\pgfsetroundjoin%
\definecolor{currentfill}{rgb}{0.741176,0.121569,0.003922}%
\pgfsetfillcolor{currentfill}%
\pgfsetfillopacity{0.678688}%
\pgfsetlinewidth{1.003750pt}%
\definecolor{currentstroke}{rgb}{0.741176,0.121569,0.003922}%
\pgfsetstrokecolor{currentstroke}%
\pgfsetstrokeopacity{0.678688}%
\pgfsetdash{}{0pt}%
\pgfpathmoveto{\pgfqpoint{-1.251124in}{-3.413742in}}%
\pgfpathcurveto{\pgfqpoint{-1.241916in}{-3.413742in}}{\pgfqpoint{-1.233083in}{-3.410083in}}{\pgfqpoint{-1.226572in}{-3.403572in}}%
\pgfpathcurveto{\pgfqpoint{-1.220061in}{-3.397060in}}{\pgfqpoint{-1.216402in}{-3.388228in}}{\pgfqpoint{-1.216402in}{-3.379019in}}%
\pgfpathcurveto{\pgfqpoint{-1.216402in}{-3.369811in}}{\pgfqpoint{-1.220061in}{-3.360979in}}{\pgfqpoint{-1.226572in}{-3.354467in}}%
\pgfpathcurveto{\pgfqpoint{-1.233083in}{-3.347956in}}{\pgfqpoint{-1.241916in}{-3.344297in}}{\pgfqpoint{-1.251124in}{-3.344297in}}%
\pgfpathcurveto{\pgfqpoint{-1.260333in}{-3.344297in}}{\pgfqpoint{-1.269165in}{-3.347956in}}{\pgfqpoint{-1.275677in}{-3.354467in}}%
\pgfpathcurveto{\pgfqpoint{-1.282188in}{-3.360979in}}{\pgfqpoint{-1.285847in}{-3.369811in}}{\pgfqpoint{-1.285847in}{-3.379019in}}%
\pgfpathcurveto{\pgfqpoint{-1.285847in}{-3.388228in}}{\pgfqpoint{-1.282188in}{-3.397060in}}{\pgfqpoint{-1.275677in}{-3.403572in}}%
\pgfpathcurveto{\pgfqpoint{-1.269165in}{-3.410083in}}{\pgfqpoint{-1.260333in}{-3.413742in}}{\pgfqpoint{-1.251124in}{-3.413742in}}%
\pgfpathlineto{\pgfqpoint{-1.251124in}{-3.413742in}}%
\pgfpathclose%
\pgfusepath{stroke,fill}%
\end{pgfscope}%
\begin{pgfscope}%
\pgfpathrectangle{\pgfqpoint{1.374500in}{0.082500in}}{\pgfqpoint{2.419000in}{2.419000in}}%
\pgfusepath{clip}%
\pgfsetbuttcap%
\pgfsetroundjoin%
\definecolor{currentfill}{rgb}{0.741176,0.121569,0.003922}%
\pgfsetfillcolor{currentfill}%
\pgfsetfillopacity{0.696691}%
\pgfsetlinewidth{1.003750pt}%
\definecolor{currentstroke}{rgb}{0.741176,0.121569,0.003922}%
\pgfsetstrokecolor{currentstroke}%
\pgfsetstrokeopacity{0.696691}%
\pgfsetdash{}{0pt}%
\pgfpathmoveto{\pgfqpoint{3.566925in}{-3.778704in}}%
\pgfpathcurveto{\pgfqpoint{3.576134in}{-3.778704in}}{\pgfqpoint{3.584966in}{-3.775045in}}{\pgfqpoint{3.591477in}{-3.768534in}}%
\pgfpathcurveto{\pgfqpoint{3.597989in}{-3.762023in}}{\pgfqpoint{3.601647in}{-3.753190in}}{\pgfqpoint{3.601647in}{-3.743982in}}%
\pgfpathcurveto{\pgfqpoint{3.601647in}{-3.734773in}}{\pgfqpoint{3.597989in}{-3.725941in}}{\pgfqpoint{3.591477in}{-3.719429in}}%
\pgfpathcurveto{\pgfqpoint{3.584966in}{-3.712918in}}{\pgfqpoint{3.576134in}{-3.709259in}}{\pgfqpoint{3.566925in}{-3.709259in}}%
\pgfpathcurveto{\pgfqpoint{3.557717in}{-3.709259in}}{\pgfqpoint{3.548884in}{-3.712918in}}{\pgfqpoint{3.542373in}{-3.719429in}}%
\pgfpathcurveto{\pgfqpoint{3.535862in}{-3.725941in}}{\pgfqpoint{3.532203in}{-3.734773in}}{\pgfqpoint{3.532203in}{-3.743982in}}%
\pgfpathcurveto{\pgfqpoint{3.532203in}{-3.753190in}}{\pgfqpoint{3.535862in}{-3.762023in}}{\pgfqpoint{3.542373in}{-3.768534in}}%
\pgfpathcurveto{\pgfqpoint{3.548884in}{-3.775045in}}{\pgfqpoint{3.557717in}{-3.778704in}}{\pgfqpoint{3.566925in}{-3.778704in}}%
\pgfpathlineto{\pgfqpoint{3.566925in}{-3.778704in}}%
\pgfpathclose%
\pgfusepath{stroke,fill}%
\end{pgfscope}%
\begin{pgfscope}%
\pgfpathrectangle{\pgfqpoint{1.374500in}{0.082500in}}{\pgfqpoint{2.419000in}{2.419000in}}%
\pgfusepath{clip}%
\pgfsetbuttcap%
\pgfsetroundjoin%
\definecolor{currentfill}{rgb}{0.741176,0.121569,0.003922}%
\pgfsetfillcolor{currentfill}%
\pgfsetfillopacity{0.696691}%
\pgfsetlinewidth{1.003750pt}%
\definecolor{currentstroke}{rgb}{0.741176,0.121569,0.003922}%
\pgfsetstrokecolor{currentstroke}%
\pgfsetstrokeopacity{0.696691}%
\pgfsetdash{}{0pt}%
\pgfpathmoveto{\pgfqpoint{11.875565in}{-3.778704in}}%
\pgfpathcurveto{\pgfqpoint{11.884773in}{-3.778704in}}{\pgfqpoint{11.893606in}{-3.775045in}}{\pgfqpoint{11.900117in}{-3.768534in}}%
\pgfpathcurveto{\pgfqpoint{11.906628in}{-3.762023in}}{\pgfqpoint{11.910287in}{-3.753190in}}{\pgfqpoint{11.910287in}{-3.743982in}}%
\pgfpathcurveto{\pgfqpoint{11.910287in}{-3.734773in}}{\pgfqpoint{11.906628in}{-3.725941in}}{\pgfqpoint{11.900117in}{-3.719429in}}%
\pgfpathcurveto{\pgfqpoint{11.893606in}{-3.712918in}}{\pgfqpoint{11.884773in}{-3.709259in}}{\pgfqpoint{11.875565in}{-3.709259in}}%
\pgfpathcurveto{\pgfqpoint{11.866356in}{-3.709259in}}{\pgfqpoint{11.857524in}{-3.712918in}}{\pgfqpoint{11.851012in}{-3.719429in}}%
\pgfpathcurveto{\pgfqpoint{11.844501in}{-3.725941in}}{\pgfqpoint{11.840843in}{-3.734773in}}{\pgfqpoint{11.840843in}{-3.743982in}}%
\pgfpathcurveto{\pgfqpoint{11.840843in}{-3.753190in}}{\pgfqpoint{11.844501in}{-3.762023in}}{\pgfqpoint{11.851012in}{-3.768534in}}%
\pgfpathcurveto{\pgfqpoint{11.857524in}{-3.775045in}}{\pgfqpoint{11.866356in}{-3.778704in}}{\pgfqpoint{11.875565in}{-3.778704in}}%
\pgfpathlineto{\pgfqpoint{11.875565in}{-3.778704in}}%
\pgfpathclose%
\pgfusepath{stroke,fill}%
\end{pgfscope}%
\begin{pgfscope}%
\pgfpathrectangle{\pgfqpoint{1.374500in}{0.082500in}}{\pgfqpoint{2.419000in}{2.419000in}}%
\pgfusepath{clip}%
\pgfsetbuttcap%
\pgfsetroundjoin%
\definecolor{currentfill}{rgb}{0.741176,0.121569,0.003922}%
\pgfsetfillcolor{currentfill}%
\pgfsetfillopacity{0.715800}%
\pgfsetlinewidth{1.003750pt}%
\definecolor{currentstroke}{rgb}{0.741176,0.121569,0.003922}%
\pgfsetstrokecolor{currentstroke}%
\pgfsetstrokeopacity{0.715800}%
\pgfsetdash{}{0pt}%
\pgfpathmoveto{\pgfqpoint{0.117088in}{-4.166075in}}%
\pgfpathcurveto{\pgfqpoint{0.126297in}{-4.166075in}}{\pgfqpoint{0.135129in}{-4.162416in}}{\pgfqpoint{0.141640in}{-4.155905in}}%
\pgfpathcurveto{\pgfqpoint{0.148152in}{-4.149394in}}{\pgfqpoint{0.151810in}{-4.140561in}}{\pgfqpoint{0.151810in}{-4.131353in}}%
\pgfpathcurveto{\pgfqpoint{0.151810in}{-4.122144in}}{\pgfqpoint{0.148152in}{-4.113312in}}{\pgfqpoint{0.141640in}{-4.106800in}}%
\pgfpathcurveto{\pgfqpoint{0.135129in}{-4.100289in}}{\pgfqpoint{0.126297in}{-4.096630in}}{\pgfqpoint{0.117088in}{-4.096630in}}%
\pgfpathcurveto{\pgfqpoint{0.107880in}{-4.096630in}}{\pgfqpoint{0.099047in}{-4.100289in}}{\pgfqpoint{0.092536in}{-4.106800in}}%
\pgfpathcurveto{\pgfqpoint{0.086024in}{-4.113312in}}{\pgfqpoint{0.082366in}{-4.122144in}}{\pgfqpoint{0.082366in}{-4.131353in}}%
\pgfpathcurveto{\pgfqpoint{0.082366in}{-4.140561in}}{\pgfqpoint{0.086024in}{-4.149394in}}{\pgfqpoint{0.092536in}{-4.155905in}}%
\pgfpathcurveto{\pgfqpoint{0.099047in}{-4.162416in}}{\pgfqpoint{0.107880in}{-4.166075in}}{\pgfqpoint{0.117088in}{-4.166075in}}%
\pgfpathlineto{\pgfqpoint{0.117088in}{-4.166075in}}%
\pgfpathclose%
\pgfusepath{stroke,fill}%
\end{pgfscope}%
\begin{pgfscope}%
\pgfpathrectangle{\pgfqpoint{1.374500in}{0.082500in}}{\pgfqpoint{2.419000in}{2.419000in}}%
\pgfusepath{clip}%
\pgfsetbuttcap%
\pgfsetroundjoin%
\definecolor{currentfill}{rgb}{0.741176,0.121569,0.003922}%
\pgfsetfillcolor{currentfill}%
\pgfsetfillopacity{0.715800}%
\pgfsetlinewidth{1.003750pt}%
\definecolor{currentstroke}{rgb}{0.741176,0.121569,0.003922}%
\pgfsetstrokecolor{currentstroke}%
\pgfsetstrokeopacity{0.715800}%
\pgfsetdash{}{0pt}%
\pgfpathmoveto{\pgfqpoint{8.680804in}{-4.166075in}}%
\pgfpathcurveto{\pgfqpoint{8.690012in}{-4.166075in}}{\pgfqpoint{8.698845in}{-4.162416in}}{\pgfqpoint{8.705356in}{-4.155905in}}%
\pgfpathcurveto{\pgfqpoint{8.711867in}{-4.149394in}}{\pgfqpoint{8.715526in}{-4.140561in}}{\pgfqpoint{8.715526in}{-4.131353in}}%
\pgfpathcurveto{\pgfqpoint{8.715526in}{-4.122144in}}{\pgfqpoint{8.711867in}{-4.113312in}}{\pgfqpoint{8.705356in}{-4.106800in}}%
\pgfpathcurveto{\pgfqpoint{8.698845in}{-4.100289in}}{\pgfqpoint{8.690012in}{-4.096630in}}{\pgfqpoint{8.680804in}{-4.096630in}}%
\pgfpathcurveto{\pgfqpoint{8.671595in}{-4.096630in}}{\pgfqpoint{8.662763in}{-4.100289in}}{\pgfqpoint{8.656251in}{-4.106800in}}%
\pgfpathcurveto{\pgfqpoint{8.649740in}{-4.113312in}}{\pgfqpoint{8.646081in}{-4.122144in}}{\pgfqpoint{8.646081in}{-4.131353in}}%
\pgfpathcurveto{\pgfqpoint{8.646081in}{-4.140561in}}{\pgfqpoint{8.649740in}{-4.149394in}}{\pgfqpoint{8.656251in}{-4.155905in}}%
\pgfpathcurveto{\pgfqpoint{8.662763in}{-4.162416in}}{\pgfqpoint{8.671595in}{-4.166075in}}{\pgfqpoint{8.680804in}{-4.166075in}}%
\pgfpathlineto{\pgfqpoint{8.680804in}{-4.166075in}}%
\pgfpathclose%
\pgfusepath{stroke,fill}%
\end{pgfscope}%
\begin{pgfscope}%
\pgfpathrectangle{\pgfqpoint{1.374500in}{0.082500in}}{\pgfqpoint{2.419000in}{2.419000in}}%
\pgfusepath{clip}%
\pgfsetbuttcap%
\pgfsetroundjoin%
\definecolor{currentfill}{rgb}{0.741176,0.121569,0.003922}%
\pgfsetfillcolor{currentfill}%
\pgfsetfillopacity{0.736118}%
\pgfsetlinewidth{1.003750pt}%
\definecolor{currentstroke}{rgb}{0.741176,0.121569,0.003922}%
\pgfsetstrokecolor{currentstroke}%
\pgfsetstrokeopacity{0.736118}%
\pgfsetdash{}{0pt}%
\pgfpathmoveto{\pgfqpoint{5.283671in}{-4.577984in}}%
\pgfpathcurveto{\pgfqpoint{5.292879in}{-4.577984in}}{\pgfqpoint{5.301712in}{-4.574325in}}{\pgfqpoint{5.308223in}{-4.567814in}}%
\pgfpathcurveto{\pgfqpoint{5.314734in}{-4.561302in}}{\pgfqpoint{5.318393in}{-4.552470in}}{\pgfqpoint{5.318393in}{-4.543261in}}%
\pgfpathcurveto{\pgfqpoint{5.318393in}{-4.534053in}}{\pgfqpoint{5.314734in}{-4.525220in}}{\pgfqpoint{5.308223in}{-4.518709in}}%
\pgfpathcurveto{\pgfqpoint{5.301712in}{-4.512198in}}{\pgfqpoint{5.292879in}{-4.508539in}}{\pgfqpoint{5.283671in}{-4.508539in}}%
\pgfpathcurveto{\pgfqpoint{5.274462in}{-4.508539in}}{\pgfqpoint{5.265630in}{-4.512198in}}{\pgfqpoint{5.259118in}{-4.518709in}}%
\pgfpathcurveto{\pgfqpoint{5.252607in}{-4.525220in}}{\pgfqpoint{5.248948in}{-4.534053in}}{\pgfqpoint{5.248948in}{-4.543261in}}%
\pgfpathcurveto{\pgfqpoint{5.248948in}{-4.552470in}}{\pgfqpoint{5.252607in}{-4.561302in}}{\pgfqpoint{5.259118in}{-4.567814in}}%
\pgfpathcurveto{\pgfqpoint{5.265630in}{-4.574325in}}{\pgfqpoint{5.274462in}{-4.577984in}}{\pgfqpoint{5.283671in}{-4.577984in}}%
\pgfpathlineto{\pgfqpoint{5.283671in}{-4.577984in}}%
\pgfpathclose%
\pgfusepath{stroke,fill}%
\end{pgfscope}%
\begin{pgfscope}%
\pgfpathrectangle{\pgfqpoint{1.374500in}{0.082500in}}{\pgfqpoint{2.419000in}{2.419000in}}%
\pgfusepath{clip}%
\pgfsetbuttcap%
\pgfsetroundjoin%
\definecolor{currentfill}{rgb}{0.741176,0.121569,0.003922}%
\pgfsetfillcolor{currentfill}%
\pgfsetfillopacity{0.757766}%
\pgfsetlinewidth{1.003750pt}%
\definecolor{currentstroke}{rgb}{0.741176,0.121569,0.003922}%
\pgfsetstrokecolor{currentstroke}%
\pgfsetstrokeopacity{0.757766}%
\pgfsetdash{}{0pt}%
\pgfpathmoveto{\pgfqpoint{1.664308in}{-5.016838in}}%
\pgfpathcurveto{\pgfqpoint{1.673517in}{-5.016838in}}{\pgfqpoint{1.682349in}{-5.013180in}}{\pgfqpoint{1.688861in}{-5.006668in}}%
\pgfpathcurveto{\pgfqpoint{1.695372in}{-5.000157in}}{\pgfqpoint{1.699031in}{-4.991324in}}{\pgfqpoint{1.699031in}{-4.982116in}}%
\pgfpathcurveto{\pgfqpoint{1.699031in}{-4.972908in}}{\pgfqpoint{1.695372in}{-4.964075in}}{\pgfqpoint{1.688861in}{-4.957564in}}%
\pgfpathcurveto{\pgfqpoint{1.682349in}{-4.951052in}}{\pgfqpoint{1.673517in}{-4.947394in}}{\pgfqpoint{1.664308in}{-4.947394in}}%
\pgfpathcurveto{\pgfqpoint{1.655100in}{-4.947394in}}{\pgfqpoint{1.646267in}{-4.951052in}}{\pgfqpoint{1.639756in}{-4.957564in}}%
\pgfpathcurveto{\pgfqpoint{1.633245in}{-4.964075in}}{\pgfqpoint{1.629586in}{-4.972908in}}{\pgfqpoint{1.629586in}{-4.982116in}}%
\pgfpathcurveto{\pgfqpoint{1.629586in}{-4.991324in}}{\pgfqpoint{1.633245in}{-5.000157in}}{\pgfqpoint{1.639756in}{-5.006668in}}%
\pgfpathcurveto{\pgfqpoint{1.646267in}{-5.013180in}}{\pgfqpoint{1.655100in}{-5.016838in}}{\pgfqpoint{1.664308in}{-5.016838in}}%
\pgfpathlineto{\pgfqpoint{1.664308in}{-5.016838in}}%
\pgfpathclose%
\pgfusepath{stroke,fill}%
\end{pgfscope}%
\begin{pgfscope}%
\pgfpathrectangle{\pgfqpoint{1.374500in}{0.082500in}}{\pgfqpoint{2.419000in}{2.419000in}}%
\pgfusepath{clip}%
\pgfsetbuttcap%
\pgfsetroundjoin%
\definecolor{currentfill}{rgb}{0.741176,0.121569,0.003922}%
\pgfsetfillcolor{currentfill}%
\pgfsetfillopacity{0.757766}%
\pgfsetlinewidth{1.003750pt}%
\definecolor{currentstroke}{rgb}{0.741176,0.121569,0.003922}%
\pgfsetstrokecolor{currentstroke}%
\pgfsetstrokeopacity{0.757766}%
\pgfsetdash{}{0pt}%
\pgfpathmoveto{\pgfqpoint{10.788234in}{-5.016838in}}%
\pgfpathcurveto{\pgfqpoint{10.797443in}{-5.016838in}}{\pgfqpoint{10.806275in}{-5.013180in}}{\pgfqpoint{10.812787in}{-5.006668in}}%
\pgfpathcurveto{\pgfqpoint{10.819298in}{-5.000157in}}{\pgfqpoint{10.822956in}{-4.991324in}}{\pgfqpoint{10.822956in}{-4.982116in}}%
\pgfpathcurveto{\pgfqpoint{10.822956in}{-4.972908in}}{\pgfqpoint{10.819298in}{-4.964075in}}{\pgfqpoint{10.812787in}{-4.957564in}}%
\pgfpathcurveto{\pgfqpoint{10.806275in}{-4.951052in}}{\pgfqpoint{10.797443in}{-4.947394in}}{\pgfqpoint{10.788234in}{-4.947394in}}%
\pgfpathcurveto{\pgfqpoint{10.779026in}{-4.947394in}}{\pgfqpoint{10.770193in}{-4.951052in}}{\pgfqpoint{10.763682in}{-4.957564in}}%
\pgfpathcurveto{\pgfqpoint{10.757171in}{-4.964075in}}{\pgfqpoint{10.753512in}{-4.972908in}}{\pgfqpoint{10.753512in}{-4.982116in}}%
\pgfpathcurveto{\pgfqpoint{10.753512in}{-4.991324in}}{\pgfqpoint{10.757171in}{-5.000157in}}{\pgfqpoint{10.763682in}{-5.006668in}}%
\pgfpathcurveto{\pgfqpoint{10.770193in}{-5.013180in}}{\pgfqpoint{10.779026in}{-5.016838in}}{\pgfqpoint{10.788234in}{-5.016838in}}%
\pgfpathlineto{\pgfqpoint{10.788234in}{-5.016838in}}%
\pgfpathclose%
\pgfusepath{stroke,fill}%
\end{pgfscope}%
\begin{pgfscope}%
\pgfpathrectangle{\pgfqpoint{1.374500in}{0.082500in}}{\pgfqpoint{2.419000in}{2.419000in}}%
\pgfusepath{clip}%
\pgfsetbuttcap%
\pgfsetroundjoin%
\definecolor{currentfill}{rgb}{0.741176,0.121569,0.003922}%
\pgfsetfillcolor{currentfill}%
\pgfsetfillopacity{0.780879}%
\pgfsetlinewidth{1.003750pt}%
\definecolor{currentstroke}{rgb}{0.741176,0.121569,0.003922}%
\pgfsetstrokecolor{currentstroke}%
\pgfsetstrokeopacity{0.780879}%
\pgfsetdash{}{0pt}%
\pgfpathmoveto{\pgfqpoint{7.232619in}{-5.485372in}}%
\pgfpathcurveto{\pgfqpoint{7.241827in}{-5.485372in}}{\pgfqpoint{7.250660in}{-5.481713in}}{\pgfqpoint{7.257171in}{-5.475202in}}%
\pgfpathcurveto{\pgfqpoint{7.263683in}{-5.468691in}}{\pgfqpoint{7.267341in}{-5.459858in}}{\pgfqpoint{7.267341in}{-5.450650in}}%
\pgfpathcurveto{\pgfqpoint{7.267341in}{-5.441441in}}{\pgfqpoint{7.263683in}{-5.432609in}}{\pgfqpoint{7.257171in}{-5.426097in}}%
\pgfpathcurveto{\pgfqpoint{7.250660in}{-5.419586in}}{\pgfqpoint{7.241827in}{-5.415928in}}{\pgfqpoint{7.232619in}{-5.415928in}}%
\pgfpathcurveto{\pgfqpoint{7.223410in}{-5.415928in}}{\pgfqpoint{7.214578in}{-5.419586in}}{\pgfqpoint{7.208067in}{-5.426097in}}%
\pgfpathcurveto{\pgfqpoint{7.201555in}{-5.432609in}}{\pgfqpoint{7.197897in}{-5.441441in}}{\pgfqpoint{7.197897in}{-5.450650in}}%
\pgfpathcurveto{\pgfqpoint{7.197897in}{-5.459858in}}{\pgfqpoint{7.201555in}{-5.468691in}}{\pgfqpoint{7.208067in}{-5.475202in}}%
\pgfpathcurveto{\pgfqpoint{7.214578in}{-5.481713in}}{\pgfqpoint{7.223410in}{-5.485372in}}{\pgfqpoint{7.232619in}{-5.485372in}}%
\pgfpathlineto{\pgfqpoint{7.232619in}{-5.485372in}}%
\pgfpathclose%
\pgfusepath{stroke,fill}%
\end{pgfscope}%
\begin{pgfscope}%
\pgfpathrectangle{\pgfqpoint{1.374500in}{0.082500in}}{\pgfqpoint{2.419000in}{2.419000in}}%
\pgfusepath{clip}%
\pgfsetbuttcap%
\pgfsetroundjoin%
\definecolor{currentfill}{rgb}{0.741176,0.121569,0.003922}%
\pgfsetfillcolor{currentfill}%
\pgfsetfillopacity{0.805608}%
\pgfsetlinewidth{1.003750pt}%
\definecolor{currentstroke}{rgb}{0.741176,0.121569,0.003922}%
\pgfsetstrokecolor{currentstroke}%
\pgfsetstrokeopacity{0.805608}%
\pgfsetdash{}{0pt}%
\pgfpathmoveto{\pgfqpoint{3.428126in}{-5.986701in}}%
\pgfpathcurveto{\pgfqpoint{3.437334in}{-5.986701in}}{\pgfqpoint{3.446167in}{-5.983043in}}{\pgfqpoint{3.452678in}{-5.976531in}}%
\pgfpathcurveto{\pgfqpoint{3.459190in}{-5.970020in}}{\pgfqpoint{3.462848in}{-5.961187in}}{\pgfqpoint{3.462848in}{-5.951979in}}%
\pgfpathcurveto{\pgfqpoint{3.462848in}{-5.942770in}}{\pgfqpoint{3.459190in}{-5.933938in}}{\pgfqpoint{3.452678in}{-5.927427in}}%
\pgfpathcurveto{\pgfqpoint{3.446167in}{-5.920915in}}{\pgfqpoint{3.437334in}{-5.917257in}}{\pgfqpoint{3.428126in}{-5.917257in}}%
\pgfpathcurveto{\pgfqpoint{3.418918in}{-5.917257in}}{\pgfqpoint{3.410085in}{-5.920915in}}{\pgfqpoint{3.403574in}{-5.927427in}}%
\pgfpathcurveto{\pgfqpoint{3.397062in}{-5.933938in}}{\pgfqpoint{3.393404in}{-5.942770in}}{\pgfqpoint{3.393404in}{-5.951979in}}%
\pgfpathcurveto{\pgfqpoint{3.393404in}{-5.961187in}}{\pgfqpoint{3.397062in}{-5.970020in}}{\pgfqpoint{3.403574in}{-5.976531in}}%
\pgfpathcurveto{\pgfqpoint{3.410085in}{-5.983043in}}{\pgfqpoint{3.418918in}{-5.986701in}}{\pgfqpoint{3.428126in}{-5.986701in}}%
\pgfpathlineto{\pgfqpoint{3.428126in}{-5.986701in}}%
\pgfpathclose%
\pgfusepath{stroke,fill}%
\end{pgfscope}%
\begin{pgfscope}%
\pgfpathrectangle{\pgfqpoint{1.374500in}{0.082500in}}{\pgfqpoint{2.419000in}{2.419000in}}%
\pgfusepath{clip}%
\pgfsetbuttcap%
\pgfsetroundjoin%
\definecolor{currentfill}{rgb}{0.741176,0.121569,0.003922}%
\pgfsetfillcolor{currentfill}%
\pgfsetfillopacity{0.832132}%
\pgfsetlinewidth{1.003750pt}%
\definecolor{currentstroke}{rgb}{0.741176,0.121569,0.003922}%
\pgfsetstrokecolor{currentstroke}%
\pgfsetstrokeopacity{0.832132}%
\pgfsetdash{}{0pt}%
\pgfpathmoveto{\pgfqpoint{9.464298in}{-6.524394in}}%
\pgfpathcurveto{\pgfqpoint{9.473506in}{-6.524394in}}{\pgfqpoint{9.482339in}{-6.520735in}}{\pgfqpoint{9.488850in}{-6.514224in}}%
\pgfpathcurveto{\pgfqpoint{9.495361in}{-6.507712in}}{\pgfqpoint{9.499020in}{-6.498880in}}{\pgfqpoint{9.499020in}{-6.489671in}}%
\pgfpathcurveto{\pgfqpoint{9.499020in}{-6.480463in}}{\pgfqpoint{9.495361in}{-6.471630in}}{\pgfqpoint{9.488850in}{-6.465119in}}%
\pgfpathcurveto{\pgfqpoint{9.482339in}{-6.458608in}}{\pgfqpoint{9.473506in}{-6.454949in}}{\pgfqpoint{9.464298in}{-6.454949in}}%
\pgfpathcurveto{\pgfqpoint{9.455089in}{-6.454949in}}{\pgfqpoint{9.446257in}{-6.458608in}}{\pgfqpoint{9.439745in}{-6.465119in}}%
\pgfpathcurveto{\pgfqpoint{9.433234in}{-6.471630in}}{\pgfqpoint{9.429576in}{-6.480463in}}{\pgfqpoint{9.429576in}{-6.489671in}}%
\pgfpathcurveto{\pgfqpoint{9.429576in}{-6.498880in}}{\pgfqpoint{9.433234in}{-6.507712in}}{\pgfqpoint{9.439745in}{-6.514224in}}%
\pgfpathcurveto{\pgfqpoint{9.446257in}{-6.520735in}}{\pgfqpoint{9.455089in}{-6.524394in}}{\pgfqpoint{9.464298in}{-6.524394in}}%
\pgfpathlineto{\pgfqpoint{9.464298in}{-6.524394in}}%
\pgfpathclose%
\pgfusepath{stroke,fill}%
\end{pgfscope}%
\begin{pgfscope}%
\pgfpathrectangle{\pgfqpoint{1.374500in}{0.082500in}}{\pgfqpoint{2.419000in}{2.419000in}}%
\pgfusepath{clip}%
\pgfsetbuttcap%
\pgfsetroundjoin%
\definecolor{currentfill}{rgb}{0.741176,0.121569,0.003922}%
\pgfsetfillcolor{currentfill}%
\pgfsetfillopacity{0.860652}%
\pgfsetlinewidth{1.003750pt}%
\definecolor{currentstroke}{rgb}{0.741176,0.121569,0.003922}%
\pgfsetstrokecolor{currentstroke}%
\pgfsetstrokeopacity{0.860652}%
\pgfsetdash{}{0pt}%
\pgfpathmoveto{\pgfqpoint{5.457447in}{-7.102555in}}%
\pgfpathcurveto{\pgfqpoint{5.466655in}{-7.102555in}}{\pgfqpoint{5.475488in}{-7.098896in}}{\pgfqpoint{5.481999in}{-7.092385in}}%
\pgfpathcurveto{\pgfqpoint{5.488510in}{-7.085874in}}{\pgfqpoint{5.492169in}{-7.077041in}}{\pgfqpoint{5.492169in}{-7.067833in}}%
\pgfpathcurveto{\pgfqpoint{5.492169in}{-7.058624in}}{\pgfqpoint{5.488510in}{-7.049792in}}{\pgfqpoint{5.481999in}{-7.043280in}}%
\pgfpathcurveto{\pgfqpoint{5.475488in}{-7.036769in}}{\pgfqpoint{5.466655in}{-7.033111in}}{\pgfqpoint{5.457447in}{-7.033111in}}%
\pgfpathcurveto{\pgfqpoint{5.448238in}{-7.033111in}}{\pgfqpoint{5.439406in}{-7.036769in}}{\pgfqpoint{5.432894in}{-7.043280in}}%
\pgfpathcurveto{\pgfqpoint{5.426383in}{-7.049792in}}{\pgfqpoint{5.422725in}{-7.058624in}}{\pgfqpoint{5.422725in}{-7.067833in}}%
\pgfpathcurveto{\pgfqpoint{5.422725in}{-7.077041in}}{\pgfqpoint{5.426383in}{-7.085874in}}{\pgfqpoint{5.432894in}{-7.092385in}}%
\pgfpathcurveto{\pgfqpoint{5.439406in}{-7.098896in}}{\pgfqpoint{5.448238in}{-7.102555in}}{\pgfqpoint{5.457447in}{-7.102555in}}%
\pgfpathlineto{\pgfqpoint{5.457447in}{-7.102555in}}%
\pgfpathclose%
\pgfusepath{stroke,fill}%
\end{pgfscope}%
\begin{pgfscope}%
\pgfpathrectangle{\pgfqpoint{1.374500in}{0.082500in}}{\pgfqpoint{2.419000in}{2.419000in}}%
\pgfusepath{clip}%
\pgfsetbuttcap%
\pgfsetroundjoin%
\definecolor{currentfill}{rgb}{0.741176,0.121569,0.003922}%
\pgfsetfillcolor{currentfill}%
\pgfsetfillopacity{0.891402}%
\pgfsetlinewidth{1.003750pt}%
\definecolor{currentstroke}{rgb}{0.741176,0.121569,0.003922}%
\pgfsetstrokecolor{currentstroke}%
\pgfsetstrokeopacity{0.891402}%
\pgfsetdash{}{0pt}%
\pgfpathmoveto{\pgfqpoint{12.045042in}{-7.725932in}}%
\pgfpathcurveto{\pgfqpoint{12.054250in}{-7.725932in}}{\pgfqpoint{12.063083in}{-7.722274in}}{\pgfqpoint{12.069594in}{-7.715763in}}%
\pgfpathcurveto{\pgfqpoint{12.076105in}{-7.709251in}}{\pgfqpoint{12.079764in}{-7.700419in}}{\pgfqpoint{12.079764in}{-7.691210in}}%
\pgfpathcurveto{\pgfqpoint{12.079764in}{-7.682002in}}{\pgfqpoint{12.076105in}{-7.673169in}}{\pgfqpoint{12.069594in}{-7.666658in}}%
\pgfpathcurveto{\pgfqpoint{12.063083in}{-7.660147in}}{\pgfqpoint{12.054250in}{-7.656488in}}{\pgfqpoint{12.045042in}{-7.656488in}}%
\pgfpathcurveto{\pgfqpoint{12.035833in}{-7.656488in}}{\pgfqpoint{12.027001in}{-7.660147in}}{\pgfqpoint{12.020489in}{-7.666658in}}%
\pgfpathcurveto{\pgfqpoint{12.013978in}{-7.673169in}}{\pgfqpoint{12.010319in}{-7.682002in}}{\pgfqpoint{12.010319in}{-7.691210in}}%
\pgfpathcurveto{\pgfqpoint{12.010319in}{-7.700419in}}{\pgfqpoint{12.013978in}{-7.709251in}}{\pgfqpoint{12.020489in}{-7.715763in}}%
\pgfpathcurveto{\pgfqpoint{12.027001in}{-7.722274in}}{\pgfqpoint{12.035833in}{-7.725932in}}{\pgfqpoint{12.045042in}{-7.725932in}}%
\pgfpathlineto{\pgfqpoint{12.045042in}{-7.725932in}}%
\pgfpathclose%
\pgfusepath{stroke,fill}%
\end{pgfscope}%
\begin{pgfscope}%
\pgfpathrectangle{\pgfqpoint{1.374500in}{0.082500in}}{\pgfqpoint{2.419000in}{2.419000in}}%
\pgfusepath{clip}%
\pgfsetbuttcap%
\pgfsetroundjoin%
\definecolor{currentfill}{rgb}{0.741176,0.121569,0.003922}%
\pgfsetfillcolor{currentfill}%
\pgfsetfillopacity{0.924655}%
\pgfsetlinewidth{1.003750pt}%
\definecolor{currentstroke}{rgb}{0.741176,0.121569,0.003922}%
\pgfsetstrokecolor{currentstroke}%
\pgfsetstrokeopacity{0.924655}%
\pgfsetdash{}{0pt}%
\pgfpathmoveto{\pgfqpoint{7.817098in}{-8.400046in}}%
\pgfpathcurveto{\pgfqpoint{7.826306in}{-8.400046in}}{\pgfqpoint{7.835139in}{-8.396388in}}{\pgfqpoint{7.841650in}{-8.389876in}}%
\pgfpathcurveto{\pgfqpoint{7.848162in}{-8.383365in}}{\pgfqpoint{7.851820in}{-8.374532in}}{\pgfqpoint{7.851820in}{-8.365324in}}%
\pgfpathcurveto{\pgfqpoint{7.851820in}{-8.356116in}}{\pgfqpoint{7.848162in}{-8.347283in}}{\pgfqpoint{7.841650in}{-8.340772in}}%
\pgfpathcurveto{\pgfqpoint{7.835139in}{-8.334260in}}{\pgfqpoint{7.826306in}{-8.330602in}}{\pgfqpoint{7.817098in}{-8.330602in}}%
\pgfpathcurveto{\pgfqpoint{7.807890in}{-8.330602in}}{\pgfqpoint{7.799057in}{-8.334260in}}{\pgfqpoint{7.792546in}{-8.340772in}}%
\pgfpathcurveto{\pgfqpoint{7.786034in}{-8.347283in}}{\pgfqpoint{7.782376in}{-8.356116in}}{\pgfqpoint{7.782376in}{-8.365324in}}%
\pgfpathcurveto{\pgfqpoint{7.782376in}{-8.374532in}}{\pgfqpoint{7.786034in}{-8.383365in}}{\pgfqpoint{7.792546in}{-8.389876in}}%
\pgfpathcurveto{\pgfqpoint{7.799057in}{-8.396388in}}{\pgfqpoint{7.807890in}{-8.400046in}}{\pgfqpoint{7.817098in}{-8.400046in}}%
\pgfpathlineto{\pgfqpoint{7.817098in}{-8.400046in}}%
\pgfpathclose%
\pgfusepath{stroke,fill}%
\end{pgfscope}%
\begin{pgfscope}%
\pgfpathrectangle{\pgfqpoint{1.374500in}{0.082500in}}{\pgfqpoint{2.419000in}{2.419000in}}%
\pgfusepath{clip}%
\pgfsetbuttcap%
\pgfsetroundjoin%
\definecolor{currentfill}{rgb}{0.741176,0.121569,0.003922}%
\pgfsetfillcolor{currentfill}%
\pgfsetlinewidth{1.003750pt}%
\definecolor{currentstroke}{rgb}{0.741176,0.121569,0.003922}%
\pgfsetstrokecolor{currentstroke}%
\pgfsetdash{}{0pt}%
\pgfpathmoveto{\pgfqpoint{10.594882in}{-9.927455in}}%
\pgfpathcurveto{\pgfqpoint{10.604091in}{-9.927455in}}{\pgfqpoint{10.612923in}{-9.923796in}}{\pgfqpoint{10.619435in}{-9.917285in}}%
\pgfpathcurveto{\pgfqpoint{10.625946in}{-9.910773in}}{\pgfqpoint{10.629604in}{-9.901941in}}{\pgfqpoint{10.629604in}{-9.892732in}}%
\pgfpathcurveto{\pgfqpoint{10.629604in}{-9.883524in}}{\pgfqpoint{10.625946in}{-9.874691in}}{\pgfqpoint{10.619435in}{-9.868180in}}%
\pgfpathcurveto{\pgfqpoint{10.612923in}{-9.861669in}}{\pgfqpoint{10.604091in}{-9.858010in}}{\pgfqpoint{10.594882in}{-9.858010in}}%
\pgfpathcurveto{\pgfqpoint{10.585674in}{-9.858010in}}{\pgfqpoint{10.576841in}{-9.861669in}}{\pgfqpoint{10.570330in}{-9.868180in}}%
\pgfpathcurveto{\pgfqpoint{10.563819in}{-9.874691in}}{\pgfqpoint{10.560160in}{-9.883524in}}{\pgfqpoint{10.560160in}{-9.892732in}}%
\pgfpathcurveto{\pgfqpoint{10.560160in}{-9.901941in}}{\pgfqpoint{10.563819in}{-9.910773in}}{\pgfqpoint{10.570330in}{-9.917285in}}%
\pgfpathcurveto{\pgfqpoint{10.576841in}{-9.923796in}}{\pgfqpoint{10.585674in}{-9.927455in}}{\pgfqpoint{10.594882in}{-9.927455in}}%
\pgfpathlineto{\pgfqpoint{10.594882in}{-9.927455in}}%
\pgfpathclose%
\pgfusepath{stroke,fill}%
\end{pgfscope}%
\begin{pgfscope}%
\pgfpathrectangle{\pgfqpoint{1.374500in}{0.082500in}}{\pgfqpoint{2.419000in}{2.419000in}}%
\pgfusepath{clip}%
\pgfsetbuttcap%
\pgfsetroundjoin%
\pgfsetlinewidth{1.505625pt}%
\definecolor{currentstroke}{rgb}{0.000000,0.000000,0.000000}%
\pgfsetstrokecolor{currentstroke}%
\pgfsetdash{}{0pt}%
\pgfusepath{stroke}%
\end{pgfscope}%
\begin{pgfscope}%
\pgfpathrectangle{\pgfqpoint{1.374500in}{0.082500in}}{\pgfqpoint{2.419000in}{2.419000in}}%
\pgfusepath{clip}%
\pgfsetbuttcap%
\pgfsetroundjoin%
\pgfsetlinewidth{1.505625pt}%
\definecolor{currentstroke}{rgb}{0.000000,0.000000,0.000000}%
\pgfsetstrokecolor{currentstroke}%
\pgfsetdash{}{0pt}%
\pgfusepath{stroke}%
\end{pgfscope}%
\end{pgfpicture}%
\makeatother%
\endgroup%

	\caption{Dressed Graphene model}
	\label{fig:eg-x model}
\end{figure}
The kinetic term is
\begin{align}
	H_0 &= -t_{\mathrm{X}} \sum_{\langle ij \rangle, \sigma} d_{i, \sigma}^{\dagger} d_{j, \sigma}
	-t_{\mathrm{Gr}} \sum_{\langle ij \rangle, \sigma}
	c_{i, \sigma}^{(\mathrm{A}), \dagger} c_{j, \sigma}^{(\mathrm{B})}
	+ V \sum_{i, \sigma \sigma^{\prime}}
	d_{i, \sigma}^{\dagger} c_{i, \sigma^{\prime}}^{(\mathrm{A})} + \mathrm{h.c.}
	\label{eq:EG-X model Hamiltonian non-interacting}
\end{align}
with
\begin{itemize}
	\item \(d\) - operators on the X atom
	\item \(c^{(\epsilon)}\) - operators on the graphene sites (\(\epsilon = \mathrm{A}, \mathrm{B}\))
	\item \(t_{\mathrm{X}}\) - nearest neighbour hopping for X
	\item \(t_{\mathrm{Gr}}\) - nearest neighbour hopping between Graphene sites
	\item \(V\) - hopping between \(\mathrm{X}\) and Graphene \(\mathrm{A}\) sites.
\end{itemize}
The (attractive) Hubbard interaction has the following form:
\begin{equation}
	H_{\mathrm{int}} = -U_{\mathrm{X}} \sum_{i} d_{i, \uparrow}^{\dagger} d_{i, \downarrow}^{\dagger} d_{i, \downarrow} d_{i, \uparrow}
	- U_{\mathrm{Gr}} \sum_{i, \epsilon=A, B} c_{i, \uparrow}^{(\epsilon) \dagger} c_{i, \downarrow}^{(\epsilon) \dagger} c_{i, \downarrow}^{\epsilon} c_{i, \uparrow}^{\epsilon}
\end{equation}
The notation using different letters for the sites connects intuitively to the physical picture, but it is more economical and in line with the notation for mean field-theory established in \cref{sec:bcs-theory} to write the Hamiltonian using a sublattice index
\begin{equation}
	\alpha = 1, 2, 3
\end{equation}
with \(1 \hateq \mathrm{Gr}_{\mathrm{A}}\), \(2 \hateq \mathrm{Gr}_{\mathrm{B}}\), \(3 \hateq \mathrm{X}\).
Then we can write the non-interacting term as
\begin{equation}
	H_0 = \sum_{\langle i, j \rangle, \alpha, \beta, \sigma} \left[\mat{t} \right]_{i\alpha,j\beta} c_{i\alpha}^{\dagger} c_{j\beta}
\end{equation}
with the matrix in the sublattice indices
\begin{equation}
	\mat{t} = \begin{pmatrix}
		0 & -t_{\mathrm{Gr}} & V \delta_{ij} \\
		-t_{\mathrm{Gr}} & 0 & 0 \\
		V \delta_{ij} & 0 & -t_{\mathrm{X}} \\
	\end{pmatrix}\;
\end{equation}
Also write the interaction part as
\begin{equation}
	H_{\mathrm{int}} = - \sum_{i \alpha} U_{\alpha} c_{i\alpha \uparrow}^{\dagger} c_{i\alpha \downarrow}^{\dagger} c_{i\alpha \downarrow} c_{i\alpha \uparrow}\;.
\end{equation}
\todo{Clean up the section from here}
Using the Fourier transformation \cref{ch:dressed graphene reciprocal space}
\begin{align}
	H_0 &= \sum_{\vb{k}, \sigma, \sigma^{\prime}} \begin{pmatrix} c_{k, \sigma}^{A, \dagger} & c_{k, \sigma}^{B, \dagger} & d_{k, \sigma}^{\dagger} \end{pmatrix}
	\begin{pmatrix}
		0 & f_{Gr} & V \\
		f_{Gr}^* & 0 & 0 \\
		V & 0 & f_{X}
	\end{pmatrix} \begin{pmatrix} c_{k, \sigma}^{A} \\ c_{k, \sigma}^{B} \\ d_{k, \sigma} \end{pmatrix}
	\label{eq:dressed graphene Hamiltonian non-interacting matrix}
\end{align}
The band structure for the non-interacting dressed graphene model is easily obtained by diagonalising the matrix in \cref{eq:dressed graphene Hamiltonian non-interacting matrix}.
This was done in \cref{fig:dressed graphene model non-interacting bands}.
\begin{figure}[t]
	\centering
	 \import{images}{dressed graphene bands.pgf}
	\caption{Bands of the non-interacting dressed Graphene model, with parameters \(t_{\mathrm{X}} = 0 \cdot t_{\mathrm{Gr}}\)}
	\label{fig:dressed graphene model non-interacting bands}
\end{figure}

\end{document}
