\documentclass[../main.tex]{subfiles}
\graphicspath{{\subfix{../images/}}, {\subfix{../}}}

\begin{document}
\chapter{Quantum Metric}

First formulated in \cite{provostRiemannianStructureManifolds1980}
	
Following Cheng - a pedagogical Introduction \todo{See what is specific to this paper, see that I can derive that myself}

Parameter dependent Hamiltonian \(\{H(\lambda)\}\), smooth dependence on parameter \(\lambda = (\lambda_1, \lambda_2, \ldots) \in \mathcal{M}\) (base manifold)

Hamiltonian acts on parametrized Hilbert space \(\mathcal{H} (\lambda)\)

Eigenenergies \(E_n (\lambda)\), eigenstates \(\ket{\phi_n (\lambda)}\)

System state \(\ket{\psi (\lambda)}\) is linear combination of \(\ket{\psi_n (\lambda)}\) at every point in \(\mathcal{M}\)

Infinitesimal variation of the parameter \(\odif{\lambda}\) \todo{Dont get it here}:
\begin{equation}
	\odif{s^2} = \vert\vert \psi (\lambda + \odif{\lambda}) - \si (\lambda) \vert \vert^2 = \braket{\fdif{\psi} | \fdif{\psi}} = \braket{\pdif{\mu} \psi | \pdif{\nu} \psi} \odif{\lambda^{\mu}} \odif{\lambda^{\nu}} = (\gamma_{\mu \nu} + \iu \sigma_{\mu \nu}) \odif{\lambda^{\mu}} \odif{\lambda^{\nu}}
\end{equation}
Last part is splitting up into real and imaginary part

\section{Quantum Metric and superfluid weight}

\todo{Write up notes about quantum metric and superfluid weight}

\end{document}
