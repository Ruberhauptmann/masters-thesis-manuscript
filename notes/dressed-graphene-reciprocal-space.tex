\documentclass[../main.tex]{subfiles}
\graphicspath{{\subfix{../images/}}, {\subfix{../}}}

\begin{document}
\raggedbottom

\chapter{Dressed Graphene Hamiltonian in Reciprocal Space}\label{ch:dressed graphene reciprocal space}

In this chapter, the Hamiltonian
\begin{align}
	H_0 &= -t_{\mathrm{X}} \sum_{\langle ij \rangle, \sigma} d_{i, \sigma}^{\dagger} d_{j, \sigma}
	-t_{\mathrm{Gr}} \sum_{\langle ij \rangle, \sigma}
	c_{i, \sigma}^{(\mathrm{A}), \dagger} c_{j, \sigma}^{(\mathrm{B})}
	+ V \sum_{i, \sigma \sigma^{\prime}}
	d_{i, \sigma}^{\dagger} c_{i, \sigma^{\prime}}^{(\mathrm{A})} + \mathrm{h.c.}
	\label{eq:EG-X model Hamiltonian non-interacting appendix}
\end{align}
from \cref{sec:dressed graphene model} will be treated to obtain the electronic band structure shown in the chapter.
The first step is to write out the sums over nearest neighbors \(\langle i, j \rangle\) explicitly, writing \(\delta_{\mathrm{X}}, \delta_{\epsilon}\) (\(\epsilon = A, B\)) for the vectors to the nearest neighbors of the \(\mathrm{X}\) atoms and Graphene \(A, B\) sites.
For example, for the \(\mathrm{X}\) atoms this gives:
\begin{align}
	-t_{\mathrm{X}} \sum_{\langle ij \rangle, \sigma} (d_{i, \sigma}^{\dagger} d_{j, \sigma} + d_{j, \sigma}^{\dagger} d_{i, \sigma})
	&= -\frac{t_X}{2} \sum_{i,\delta_{\mathrm{X}},\sigma} d_{i, \sigma}^{\dagger} d_{i + \delta_{\mathrm{X}}, \sigma}
	-\frac{t_X}{2} \sum_{j,\delta_{\mathrm{X}},\sigma} d_{j, \sigma}^{\dagger} d_{j + \delta_{\mathrm{X}}, \sigma} \label{eq:EG-X model X atoms nearest neighbor sum double counting} \\
	&= -t_X \sum_{i,\sigma} \sum_{\delta_{\mathrm{X}}} d_{i, \sigma}^{\dagger} d_{i + \delta_{\mathrm{X}}, \sigma} \;. \label{eq:EG-X model X atoms nearest neighbours written out}
\end{align}
The factor \(\nicefrac{1}{2}\) in \cref{eq:EG-X model X atoms nearest neighbor sum double counting} is to account for double counting when going to the sum over all lattice sites \(i\).
By relabeling \(j \to i\) in the second sum, the two sum are the same and \cref{eq:EG-X model X atoms nearest neighbours written out} is obtained.
Now, using the discrete Fourier transform
\begin{align}
	c_{i} &= \frac{1}{\sqrt{N}} \sum_{\vb{k}} e^{\iu \vb{k} \vb{r}_{i}} c_{\vb{k}} ,\;
	c_{i}^{\dagger} = \frac{1}{\sqrt{N}} \sum_{\vb{k}} e^{-\iu \vb{k} \vb{r}_{i}} c_{\vb{k}}^{\dagger}
\end{align}
with the completeness relation
\begin{equation}
	\sum_{i} e^{\iu \vb{k} \vb{r}_{i}} e^{-\iu \vb{k}^{\prime} \vb{r}_{i}} = N \delta_{\vb{k}, \vb{k}^{\prime}}\;,
\end{equation}
\cref{eq:EG-X model X atoms nearest neighbours written out} reads:
\begin{align}
	-t_X \frac{1}{N} \sum_{i, \sigma} \sum_{\mathrm{X}} d_{i, \sigma}^{\dagger} d_{i + \delta_{\mathrm{X}}, \sigma}
	&= -t_X \frac{1}{N} \sum_{i,\sigma} \sum_{\vb{k}, \vb{k}^{\prime}, \delta_{\mathrm{X}}} \left(e^{-\iu \vb{k} \vb{r}_i} d_{\vb{k}, \sigma}^{\dagger} \right) \left(e^{\iu \vb{k}^{\prime} \vb{r}_i} e^{\iu \vb{k}^{\prime} \delta_{\mathrm{X}}} d_{\vb{k}^{\prime}, \sigma} \right) \\
	&= -t_X \frac{1}{N} \sum_{\vb{k}, \vb{k^{\prime}}, \delta_{\mathrm{X}}, \sigma} d_{\vb{k}, \sigma}^{\dagger}   d_{\vb{k}^{\prime}, \sigma} e^{\iu \vb{k}^{\prime} \delta_{\mathrm{X}}} \sum_{i} e^{-\iu \vb{k} \vb{r}_i} e^{\iu \vb{k}^{\prime} \vb{r}_i} \\
	&= -t_X \frac{1}{N} \sum_{\vb{k}, \vb{k^{\prime}}, \sigma}  d_{\vb{k}, \sigma}^{\dagger}  d_{\vb{k}^{\prime}, \sigma} \sum_{\delta_{\mathrm{X}}} e^{\iu \vb{k}^{\prime} \delta_{\mathrm{X}}} \left(N \delta_{\vb{k}, \vb{k}^{\prime}} \right)\\
	&= -t_X \sum_{\vb{k}, \sigma}  d_{\vb{k}, \sigma}^{\dagger}d_{\vb{k}, \sigma} \sum_{\delta_{\mathrm{X}}} e^{\iu \vb{k} \delta_{\mathrm{X}}}
\end{align}
This is now diagonal in \(\vb{k}\) space.
The sum over \(\delta_{\mathrm{X}}\) can be explicitly calculated using the fact that the nearest neighbours vectors \(\delta_{\mathrm{X}}\) for the \(\mathrm{X}\) atoms are the vectors \(\delta_{AA, i}\) from \cref{sec:lattice-structure-of-graphene}, for example
\todo{Clear up definition NN vectors and results}
\begin{align}
	\vb{k} \cdot \vb{\delta_{AA, 1}} = \begin{pmatrix} k_x \\ k_y \end{pmatrix} \cdot \begin{pmatrix} 1 \\ \sqrt{3} \end{pmatrix} = k_x + \sqrt{3} k_y
\end{align}
\begin{align}
	f_{\mathrm{X}} (\vb{k}) &= -t_X \sum_{\delta_{\mathrm{X}}} e^{\iu \vb{k} \delta_{\mathrm{X}}} \\
	&= -t_X \left[ e^{\iu a (\frac{k_x}{2} + \frac{\sqrt{3} k_y}{2})}
	+ e^{\iu a k_x}
	+ e^{\iu a (\frac{k_x}{2} - \frac{\sqrt{3} k_y}{2})}
	\right. \\
	&+ \left. e^{\iu a (-\frac{k_x}{2} - \frac{\sqrt{3} k_y}{2})}
	+ e^{-\iu a k_x}
	+ e^{\iu a (-\frac{k_x}{2} + \frac{\sqrt{3} k_y}{2})} \right] \\
	&= -t_X \left( 2 \cos{(a k_x)} + 2 e^{\iu a \frac{\sqrt{3} k_y}{2}} \cos{(\frac{a}{2} k_x)} + 2 e^{-\iu a \frac{\sqrt{3} k_y}{2}} \cos{(\frac{a}{2} k_x)} \right) \\
	&= -2t_X \left( \cos{(a k_x)} + 2 \cos{(\frac{a}{2} k_x)} \cos{(\sqrt{3} \frac{ a}{2} k_y)} \right) \;.
\end{align}
The same can be done for the hopping between Graphene sites, for example :
\begin{align}
	-t_{\mathrm{Gr}} \sum_{\langle ij \rangle, \sigma \sigma^{\prime}} c_{i, \sigma}^{(A), \dagger} c_{j, \sigma^{\prime}}^{(B)}
	&= -t_{\mathrm{Gr}} \sum_{i, \sigma \sigma^{\prime}} \sum_{\delta_{AB}} c_{i, \sigma}^{(A), \dagger} c_{i + \delta_{AB} , \sigma^{\prime}}^{(B)} \\
	&= -t_{\mathrm{Gr}} \sum_{\vb{k}, \sigma, \sigma^{\prime}}  c_{\vb{k}, \sigma}^{(A) \dagger} c_{\vb{k}, \sigma^{\prime}}^{(B)} \sum_{\delta_{AB}} e^{\iu \vb{k} \delta_{AB}}
\end{align}
with again the sum over \(\delta_{AB}\)
\begin{align}
	f_{\mathrm{Gr}} (\vb{k}) &= -t_{\mathrm{Gr}} \sum_{\delta_{AB}} e^{\iu \vb{k} \delta_{AB}} \\
	&= -t_{\mathrm{Gr}} \left(
	e^{\iu \frac{a}{\sqrt{3}} k_y} +
	e^{\iu \frac{a}{2\sqrt{3}} (\sqrt{3} k_x - k_y)} +
	e^{\iu \frac{a}{2\sqrt{3}} (-\sqrt{3} k_x - k_y)} \right) \\
	&= -t_{\mathrm{Gr}} \left(
	e^{\iu \frac{a}{\sqrt{3}} k_y} +
	e^{-\iu \frac{a}{2\sqrt{3}} k_y} \left(
	e^{\iu \frac{a}{2} k_x} + e^{-\iu \frac{a}{2} k_x}
	\right) \right) \\
	&= -t_{\mathrm{Gr}} \left(
	e^{\iu \frac{a}{\sqrt{3}} k_y} +
	2 e^{-\iu \frac{a}{2\sqrt{3}} k_y}
	\cos{(\frac{a}{2} k_x)} \right)
\end{align}
We note \todo{Show that!}
\begin{align}
	\sum_{\delta_{AB}} e^{\iu \vb{k} \delta_{AB}} = \left( \sum_{\delta_{BA}} e^{\iu \vb{k} \delta_{BA}} \right)^* = \sum_{\delta_{BA}} e^{-\iu \vb{k} \delta_{BA}}
\end{align}
All in all:
\begin{align}
	H_0 &= \sum_{\vb{k}, \sigma, \sigma^{\prime}} \begin{pmatrix} c_{\vb{k}, \sigma}^{A, \dagger} & c_{\vb{k}, \sigma}^{B, \dagger} & d_{\vb{k}, \sigma}^{\dagger} \end{pmatrix}
	\begin{pmatrix}
		0 & f_{Gr} & V \\
		f_{Gr}^* & 0 & 0 \\
		V & 0 & f_{X}
	\end{pmatrix} \begin{pmatrix} c_{\vb{k}, \sigma}^{A} \\ c_{\vb{k}, \sigma}^{B} \\ d_{\vb{k}, \sigma} \end{pmatrix}
	\label{eq:EG-X Hamiltonian non-interacting matrix appendix}
\end{align}	

\end{document}