\documentclass[../notes.tex]{subfiles}
\graphicspath{{\subfix{../images/}}, {\subfix{../}}}

\begin{document}
\raggedbottom
	
\chapter{Conclusion}\label{ch:conclusion}

\subsection*{Summary}

The goal of this thesis was to explore the phenomenon of superconductivity via the calculation of the coherence length and the London penetration depth, which connect to the critical temperature and the superfluid phase stiffness respectively.
This was done by placing a \gls{fmp} constraint on the order parameter and analyzing the suppression of the order parameter and the superconducting current.

The \gls{fmp} method is based on the phenomenological Ginzburg-Landau theory, which is revised in the theoretical foundations.
It is then explained how introducing a finite momentum to the order parameter gives access to the superconducting length scales.
To calculate the length scales in microscopic theories, a way to introduce the finite momentum into these theories is needed.
In the thesis, this is done for two theories: \gls{bcs} theory and \gls{dmft}.
%\Acrshort{bcs} theory is the static mean-field description of superconductivity and es
%On the other hand, \gls{dmft} includes the full local fluctuations, so that 
This section also introduces the geometry of the state (characterized by the quantum metric) and how it connects to the superfluid weight, which is especially important in the context of superconductors with flat electronic bands.

One topic in superconductivity that has garnered recent interest are graphene based system that host flat bands due to their structural configuration, i.e. twisted Bilayer graphene or Bernal trilayer graphene.
The next section introduces a decorated graphene model which is an approximate model for a system of group-IV intercalants between a graphene sheet and the semiconducting \ce{SiC}(0001) substrate.
This system also hosts flat bands and inherits the quantum geometry from the underlying graphene, which is captured by the decorated graphene model.

In the last chapter, the \gls{fmp} method is used to explore the superconducting length scales.
The \gls{bcs} formulation is applied to the decorated graphene model, exploring how the superconductivity depends on the hybridization and how the quantum geometry of the model influence superconductivity.
\Acrshort{dmft} calculations are numerically significantly more complex and implementing it for the decorated graphene is beyond the scope of this thesis, so it is applied to a simpler model in the one band Hubbard model on a square lattice, exploring the phenomenon of the BCS-BEC crossover and seeing how the \gls{bcs} and \gls{dmft} method differ.


\subsection*{Outlook}

Starting from the mean-field results for the decorated graphene model discussed in \cref{sec:results decorated graphene model}, further investigation into the interesting superconducting behavior with the hybridization \(V\) and why the superfluid weight does not follow the quantum metric.
To this end, it is also interesting to extend mean-field treatment to the toy model with lower impurity density, see \cref{fig:decorated graphene extended unit cell}.
This model has a different \(V \to \infty\) limit, retaining a finite quantum metric in this regime, which will in turn influence superconductivity.

DMFT and BCS: see whether it is a limitation of the BCS or the method in this model. \todo{Split up figure}

\begin{figure}[t]
	\centering
	\includegraphics[width=0.9\textwidth]{images/extended decorated graphene}
	\label{fig:decorated graphene extended unit cell}
	\caption[Decorated graphene with smaller impurity density.]{\textbf{Decorated graphene with smaller impurity density.} (\textbf{a}) Taken from \cite{wittQuantumGeometryLocal2025}.}
\end{figure}

As seen in \cref{sec:results OBH}, including the full local fluctuations with \gls{dmft} reveals physics beyond the mean-field level, so investigating the decorated graphene model in both the structure treated in this thesis and with the smaller impurity density is an interesting next step.

In twisted Bilayer graphene, the fact that quantum geometry is important for superconductivity in the material is well established \cite{tanakaSuperfluidStiffnessMagicangle2025}.
The pairing mechanism in twisted bilayer graphene is still an open question so using the \gls{fmp} method to calculate \(\xi_0\) and \(\lambda_{\mathrm{L}, 0}\) to enable a more rigorous comparison to experiment can guide in this exploration, and using the \gls{fmp} method on the mean-field level as developed in this thesis can be a tool for exploration with lower computational cost.
	
\end{document}