\documentclass[../notes.tex]{subfiles}
\graphicspath{{\subfix{../images/}}, {\subfix{../}}}

\begin{document}
\raggedbottom
	
\chapter{Introduction}

The United Nations declared 2025 the `International Year of Quantum Science and Technology' \cite{unitednationsInternationalYearQuantum2024}.
This is an effort is to raise awareness of the importance of quantum science and its applications, which focuses in 3 key areas: quantum computing, quantum communications and quantum sensors.

One effect underlying many of these applications is the phenomenon of superconductivity.

It was discovered in 1911, when Heike Onnes measured that the electrical resistance of Mercury suddenly vanished completely when cooling it below around \qty{4}{\kelvin} \cite{onnesFurtherExperimentsLiquid1991}.

Natural units: \(\kB = 1, \hbar = 1, \mu_0 = 1\)

\todo{Section about SC length scales}

\subsection*{Quantum Materials}



While the mechanisms of superconductivity are not fully known in all cases, it is 

As such, superconductivity is one of the important examples of quantum mechanical effects (i.e. the pairing of electrons) manifesting on a macroscopical scale.
This makes it 

\begin{SCfigure}[50][t]
	\centering
	\import{images/}{Critical Surface.pdf_tex}
	\caption{\textbf{Critical surface of a superconductor.} For practical applications, this surface is desired to be as large as possible, making it possible to carry high currents and generate strong magnetic fields while not needing to cool the superconductor to very low temperatures. This generally is the case for high-temperature superconductors in comparison to low-temperature superconductors. }
	\label{fig:Critical Surface of a SC}
\end{SCfigure}

%Discovery of Meissner effect, perfect expulsion of external magnetic fields in 1933 \cite{meissnerNeuerEffektBei1933}.
%This started almost half a century of intensive theoretical research, which culminated in John Bardeen, Leon Cooper and J. Robert Schrieffer developing the microscopic theory now know as BCS theory \cite{bardeenTheorySuperconductivity1957}.

\subsection*{Unconventional Superconductivity}

\todo{Section about unconv. SCs and length scales (type I, II SCs, pseudogaps)}

\subsection*{BCS Superconductivity}

\subsection*{High-Temperature Superconductivity}

1986 and 1987: discovery of superconductivity with very high \(T_C\) found in cuprates \cite{bednorzPossibleHighTc1986,uchidaHighTcSuperconductivity1987}.
Cuprate superconductors are made up of layers of cooper oxide and charge reservoirs in between.
The specific charge reservoir layers determine the properties of the SC and varying them lead to a rich zoo of materials with high \(T_C\)  \cite{rybickiPerspectivePhaseDiagram2016}.

Largest commercial application to date is in magnetic resonance imaging, a medical technique using strong magnetic fields and field gradients \cite{rinckMagneticResonanceMedicine}.
Enabled due to the fact, that SCs can carry much stronger currents and thus generate much higher magnetic field strength.
Technical applications in research are much wider, ranging from strong superconducting magnets in the LHC \cite{tollestrupDevelopmentSuperconductingMagnets2008, rossiParticleAcceleratorsCuprate2023} and other particle accelerators over detectors of single photons in astrophysics \cite{irwinTransitionEdgeSensors2005} to extremely sensitive measurement devices for magnetic fields \cite{faleyHighTcSQUIDBiomagnetometers2017} and voltages \cite{klushinPresentFutureHightemperature2020} based on the Josesphon effect \cite{josephsonPossibleNewEffects1962}.


Since the first discovery of SC in cuprates, there has been a lot of work to develop superconductors with higher transition temperatures.

\subsection*{Flat Bands: Pairing and Supercurrent}

A 

\subsection*{Graphene Structures as a Platform for Correlated Physics}

One interesting development in is in twisted multilayer systems, first realized as twisted bilayer Graphene \cite{caoUnconventionalSuperconductivityMagicangle2018}.
In comparison to the complex crystal structure of e.g. the Cuprates, twisted multilayer systems have a very simple structure and can be tuned very easily: the angle of twist between the layers can be easily accessed experimentally.
The defining feature of these systems are flat electronic bands due to folding of the Brilluoin zone.
Superconductivity in these systems is enhanced due to the fact that in the flat bands, interactions between the electrons are very strongly enhanced.
Thus these systems are a very interesting playground to study strongly correlation effects in general and superconductivity in particular.

\subsection*{Organization of this thesis}

\end{document}