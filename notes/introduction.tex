\documentclass[../notes.tex]{subfiles}
\graphicspath{{\subfix{../images/}}, {\subfix{../}}}

\begin{document}
\raggedbottom
	
\chapter{Introduction}

In 1894, Albert Michelson remarked that \enquote{it seems probable that most of the grand underlying principles have been firmly established}  \cite[p. 159]{chicagoAnnualRegister1896}.
The 20th century then fundamentally changed our view of the world in the small and cold with quantum mechanics and the large and hot with general relativity.
Among the events fuelling this revolution was the 1911 discovery of the phenomenon of superconductivity in Mercury by Heike Onnes \cite{onnesFurtherExperimentsLiquid1991}.
Superconductivity is the phenomenon of the electrical resistance of a material suddenly dropping to zero below a critical temperature \(T_C\).

Discovery of Meissner effect, perfect expulsion of external magnetic fields in 1933 \cite{meissnerNeuerEffektBei1933}.
This started almost half a century of intensive theoretical research, which culminated in John Bardeen, Leon Cooper and J. Robert Schrieffer developing the microscopic theory now know as BCS theory \cite{bardeenTheorySuperconductivity1957}.

\subsection*{High-Temperature Superconductivity}

1986 and 1987: discovery of superconductivity with very high \(T_C\) found in cuprates \cite{bednorzPossibleHighTc1986,uchidaHighTcSuperconductivity1987}.
Cuprate superconductors are made up of layers of cooper oxide and charge reservoirs in between.
The specific charge reservoir layers determine the properties of the SC and varying them lead to a rich zoo of materials with high \(T_C\)  \cite{rybickiPerspectivePhaseDiagram2016}.

Largest commercial application to date is in magnetic resonance imaging, a medical technique using strong magnetic fields and field gradients \cite{rinckMagneticResonanceMedicine}.
Enabled due to the fact, that SCs can carry much stronger currents and thus generate much higher magnetic field strength.
Technical applications in research are much wider, ranging from strong superconducting magnets in the LHC \cite{tollestrupDevelopmentSuperconductingMagnets2008, rossiParticleAcceleratorsCuprate2023} and other particle accelerators over detectors of single photons in astrophysics \cite{irwinTransitionEdgeSensors2005} to extremely sensitive measurement devices for magnetic fields \cite{faleyHighTcSQUIDBiomagnetometers2017} and voltages \cite{klushinPresentFutureHightemperature2020} based on the Josesphon effect \cite{josephsonPossibleNewEffects1962}.


Since the first discovery of SC in cuprates, there has been a lot of work to develop superconductors with higher transition temperatures.

\subsection*{Flat Bands: Pairing and Supercurrent}

\subsection*{Twisted Bilayer Graphene}

One interesting development in is in twisted multilayer systems, first realized as twisted bilayer Graphene \cite{caoUnconventionalSuperconductivityMagicangle2018}.
In comparison to the complex crystal structure of e.g. the Cuprates, twisted multilayer systems have a very simple structure and can be tuned very easily: the angle of twist between the layers can be easily accessed experimentally.
The defining feature of these systems are flat electronic bands due to folding of the Brilluoin zone.
Superconductivity in these systems is enhanced due to the fact that in the flat bands, interactions between the electrons are very strongly enhanced.
Thus these systems are a very interesting playground to study strongly correlation effects in general and superconductivity in particular.

\todo{I have system that is similar}

\subsection*{Organization of this thesis}

\end{document}