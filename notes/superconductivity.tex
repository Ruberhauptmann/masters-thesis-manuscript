\documentclass[../notes.tex]{subfiles}
\graphicspath{{\subfix{../images/}}, {\subfix{../}}}

\begin{document}
\raggedbottom

\chapter{Superconductivity}\label{ch:superconductivity}

Superconductivity is an example of an emergent phenomenon: the Schrödinger equation describing all interactions between electrons gives no indication that there exists parameters for which the electrons condense into phase coherent pairs.
In this chapter I review theoretical concepts needed for understanding superconductivity and introduce the tools used to study superconductivity in the later chapters.
There are many textbooks covering these topics which can be referenced for a more detailed treatment, such as refs. \cite{colemanIntroductionManyBodyPhysics2015, tinkhamIntroductionSuperconductivity1996, bruusManyBodyQuantumTheory2004, larkinTheoryFluctuationsSuperconductors2005, bennemannSuperconductivity2008}.

Macroscopically, the superconducting state can be described by a spontaneous breaking of a \(U(1)\) phase rotation symmetry that is associated with an order parameter.
The theory of spontaneous symmetry breaking and associated phase transitions is Ginzburg-Landau theory discussed in \cref{sec:Ginzburg-Landau theory of superconductivity}, following refs. \cite{beekmanIntroductionSpontaneousSymmetry2019, colemanIntroductionManyBodyPhysics2015}.
Ginzburg-Landau theory introduces two length scales: the coherence length \(\xi_0\) describing the length scale of amplitude variations of the order parameter and the London penetration depth \(\lambda_L\), which is connected to energy cost of phase variations of the order parameter.
They also connect to the energy gap \(\Delta\) and the condensate stiffness \(D_{\mathrm{S}}\), which are often competing energy scales in superconductors.
The interplay of these length (energy) scales determine the macroscopic properties of a superconductors, so there is a great interest in accessing them in computational ways.
To this end, \cref{sec:Ginzburg-Landau theory of superconductivity} also introduces a theoretical framework based on Cooper pairs with finite momentum \cite{wittBypassingLatticeBCS2024} that will be used in later chapters to calculate these length scales from microscopic theories.

Ginzburg Landau theory is a macroscopic theory, but it can be connected to microscopic theories: if a theory finds an expression for the order parameter describing the breakdown of symmetry, it can be connected to quantities expressed by Ginzburg-Landau theory.
One such theory to describe superconductivity from a microscopic perspective is \glsxtrfull{bcs} theory in \cref{sec:bcs-theory}, which is 
A method to treat local interactions non-perturbatively is \glsxtrfull{dmft}. \Cref{sec:Dynamical Mean-Field Theory} briefly introduces the Greens function method to treat many-body problems and outlines the \glsxtrshort{dmft} self-consistency cycle.

Furthermore, \cref{sec:quantum-metric} introduces an emerging perspective in the study of novel superconductors: it turns out that the superfluid weight is connected to a quantity of the electronic band structure called the quantum metric \cite{peottaSuperfluidityTopologicallyNontrivial2015, yuQuantumGeometryQuantum2024}, which is connected to 

\section{Ginzburg-Landau Theory of Superconductivity}\label{sec:Ginzburg-Landau theory of superconductivity}

\subsection*{Spontaneous Symmetry Breaking and Order Parameter}

Symmetries are a powerful concept in physics.
Noethers theorem \cite{noetherInvarianteVariationsprobleme1918} connects the symmetries of physical theories to associated conservation laws.
An interesting facet of symmetries in physical theories is the fact, that a ground state of a system must not necessarily obey the same symmetries of its Hamiltonian, i.e. for a symmetry operation that is described by a unitary operator \(U\), the Hamiltonian commutes with \(U\) (which results in expectation values of the Hamiltonian being invariant under the symmetry operation) but the states \(\ket{\phi}\) and \(U \ket{\phi}\) are different.
This phenomenon is called spontaneous symmetry breaking and the state \(\ket{\phi}\) is said to be symmetry-broken.

One consequence of this fact is that for a given symmetry-broken state \(\ket{\phi}\), there exists multiple states that can be reached by repeatedly applying \(U\) to \(\ket{\phi}\) and all have the same energy.
To differentiate the symmetry-broken states an operator can be defined that has all these equivalent states as eigenvectors with different eigenvalues and zero expectation value for symmetric states.
This is the microscopic notion of an order parameter.

The original notion of an order parameter was motivated from macroscopic observables that can then be related to the microscopic order parameter operator introduced above.
Macroscopically I characterize the symmetry breaking by an order parameter \(\Psi\) which generally can be a complex-valued vector that becomes non-zero below the transition temperature \(T_C\)
\begin{equation}
	\vert \Psi \vert =
	\begin{cases}
		0 & T > T_C \\
		\vert \Psi_0 \vert > 0 & T < T_C
	\end{cases} \;.
\end{equation}

In the example of a ferromagnet, a finite magnetization of a material is associated with a finite expectation value for the \(z\)-component of the spin operator, \(m_z = \braket{\ope{S_z}}\).
The order parameter describes the `degree of order` \cite{landauTheoryPhaseTransitions1937}.
Similarly to a magnetically ordered state, the SC state is characterized by an order parameter.
The theory of phase transitions in superconductors was developed by Ginzburg and Landau \cite{ginzburgTheorySuperconductivity1950}.
Landau theory and conversely Ginzburg-Landau theory is not concerned with the the microscopic properties of the order parameter, but describes the changes in thermodynamic properties of matter with the development of an order parameter.
In superconductivity, the order parameter is described by coherent pairs of electrons with opposite momentum and spin.
This will be explained in more detail later, but might be helpful to think of Cooper pairs already when discussing Ginzburg-Landau theory.  

\subsection*{Landau and Ginzburg-Landau Theory}\label{sub:Landau and Ginzburg-Landau Theory}

The free energy is a thermodynamic quantity:
\begin{equation}
	F = E - T S
\end{equation}
with the energy of the system \(E\), temperature \(T\) and entropy \(S\).
A system in thermodynamic equilibrium has minimal free energy.
The fundamental idea underlying Landau theory is to write the free energy \(F[\Psi]\) as function of the order parameter \(\Psi\) and expand it as a polynomial:
\begin{equation}
	F_L[\Psi] = \int \odif[order={d}]{x} f_L [\Psi] \;,
\end{equation}
where
\begin{equation}
	f_L [\Psi] = \frac{r}{2} \Psi^2 + \frac{u}{4} \Psi^4
\end{equation}
is called the free energy density.
\todo{Do minima first, then redefine r}
Provided the parameters \(r\) and \(u\) are greater than \(0\), there is a minimum of \(f_L [\Psi]\) that lies at \(\Psi = 0\).
Landau theory assumes that at the phase transition temperature \(T_C\) the parameter \(r\) changes sign, so it can be written in first order as
\begin{equation}
	r = a(T - T_C) \;.
\end{equation}

\begin{figure}[t]
	\centering
	\begin{subfigure}[b]{0.5\textwidth}
		\centering
		\caption{\hfill\null}\label{sfig:Landau free energy}
		%% Creator: Matplotlib, PGF backend
%%
%% To include the figure in your LaTeX document, write
%%   \input{<filename>.pgf}
%%
%% Make sure the required packages are loaded in your preamble
%%   \usepackage{pgf}
%%
%% Also ensure that all the required font packages are loaded; for instance,
%% the lmodern package is sometimes necessary when using math font.
%%   \usepackage{lmodern}
%%
%% Figures using additional raster images can only be included by \input if
%% they are in the same directory as the main LaTeX file. For loading figures
%% from other directories you can use the `import` package
%%   \usepackage{import}
%%
%% and then include the figures with
%%   \import{<path to file>}{<filename>.pgf}
%%
%% Matplotlib used the following preamble
%%   \def\mathdefault#1{#1}
%%   \everymath=\expandafter{\the\everymath\displaystyle}
%%   \IfFileExists{scrextend.sty}{
%%     \usepackage[fontsize=10.000000pt]{scrextend}
%%   }{
%%     \renewcommand{\normalsize}{\fontsize{10.000000}{12.000000}\selectfont}
%%     \normalsize
%%   }
%%   \usepackage{fontspec}\usepackage{unicode-math}\setmathfont{texgyrepagella-math.otf}\setmainfont{texgyrepagella-math}
%%   \makeatletter\@ifpackageloaded{underscore}{}{\usepackage[strings]{underscore}}\makeatother
%%
\begingroup%
\makeatletter%
\begin{pgfpicture}%
\pgfpathrectangle{\pgfpointorigin}{\pgfqpoint{2.252757in}{2.135967in}}%
\pgfusepath{use as bounding box, clip}%
\begin{pgfscope}%
\pgfsetbuttcap%
\pgfsetmiterjoin%
\definecolor{currentfill}{rgb}{1.000000,1.000000,1.000000}%
\pgfsetfillcolor{currentfill}%
\pgfsetlinewidth{0.000000pt}%
\definecolor{currentstroke}{rgb}{1.000000,1.000000,1.000000}%
\pgfsetstrokecolor{currentstroke}%
\pgfsetdash{}{0pt}%
\pgfpathmoveto{\pgfqpoint{0.000000in}{0.000000in}}%
\pgfpathlineto{\pgfqpoint{2.252757in}{0.000000in}}%
\pgfpathlineto{\pgfqpoint{2.252757in}{2.135967in}}%
\pgfpathlineto{\pgfqpoint{0.000000in}{2.135967in}}%
\pgfpathlineto{\pgfqpoint{0.000000in}{0.000000in}}%
\pgfpathclose%
\pgfusepath{fill}%
\end{pgfscope}%
\begin{pgfscope}%
\pgfsetbuttcap%
\pgfsetmiterjoin%
\definecolor{currentfill}{rgb}{1.000000,1.000000,1.000000}%
\pgfsetfillcolor{currentfill}%
\pgfsetlinewidth{0.000000pt}%
\definecolor{currentstroke}{rgb}{0.000000,0.000000,0.000000}%
\pgfsetstrokecolor{currentstroke}%
\pgfsetstrokeopacity{0.000000}%
\pgfsetdash{}{0pt}%
\pgfpathmoveto{\pgfqpoint{0.050000in}{0.050000in}}%
\pgfpathlineto{\pgfqpoint{2.057250in}{0.050000in}}%
\pgfpathlineto{\pgfqpoint{2.057250in}{2.044300in}}%
\pgfpathlineto{\pgfqpoint{0.050000in}{2.044300in}}%
\pgfpathlineto{\pgfqpoint{0.050000in}{0.050000in}}%
\pgfpathclose%
\pgfusepath{fill}%
\end{pgfscope}%
\begin{pgfscope}%
\definecolor{textcolor}{rgb}{0.000000,0.000000,0.000000}%
\pgfsetstrokecolor{textcolor}%
\pgfsetfillcolor{textcolor}%
\pgftext[x=2.197758in,y=0.887606in,,top]{\color{textcolor}{\rmfamily\fontsize{10.000000}{12.000000}\selectfont\catcode`\^=\active\def^{\ifmmode\sp\else\^{}\fi}\catcode`\%=\active\def%{\%}$\Psi$}}%
\end{pgfscope}%
\begin{pgfscope}%
\pgfpathrectangle{\pgfqpoint{0.050000in}{0.050000in}}{\pgfqpoint{2.007250in}{1.994300in}}%
\pgfusepath{clip}%
\pgfsetrectcap%
\pgfsetroundjoin%
\pgfsetlinewidth{1.003750pt}%
\definecolor{currentstroke}{rgb}{0.800000,0.400000,0.466667}%
\pgfsetstrokecolor{currentstroke}%
\pgfsetdash{}{0pt}%
\pgfpathmoveto{\pgfqpoint{0.163523in}{2.054300in}}%
\pgfpathlineto{\pgfqpoint{0.185121in}{1.863607in}}%
\pgfpathlineto{\pgfqpoint{0.207062in}{1.687653in}}%
\pgfpathlineto{\pgfqpoint{0.229003in}{1.528604in}}%
\pgfpathlineto{\pgfqpoint{0.250944in}{1.385497in}}%
\pgfpathlineto{\pgfqpoint{0.272886in}{1.257393in}}%
\pgfpathlineto{\pgfqpoint{0.294827in}{1.143376in}}%
\pgfpathlineto{\pgfqpoint{0.316768in}{1.042560in}}%
\pgfpathlineto{\pgfqpoint{0.335052in}{0.968006in}}%
\pgfpathlineto{\pgfqpoint{0.353336in}{0.901533in}}%
\pgfpathlineto{\pgfqpoint{0.371621in}{0.842667in}}%
\pgfpathlineto{\pgfqpoint{0.389905in}{0.790946in}}%
\pgfpathlineto{\pgfqpoint{0.408189in}{0.745921in}}%
\pgfpathlineto{\pgfqpoint{0.426474in}{0.707154in}}%
\pgfpathlineto{\pgfqpoint{0.444758in}{0.674221in}}%
\pgfpathlineto{\pgfqpoint{0.463042in}{0.646709in}}%
\pgfpathlineto{\pgfqpoint{0.477670in}{0.628332in}}%
\pgfpathlineto{\pgfqpoint{0.492297in}{0.612969in}}%
\pgfpathlineto{\pgfqpoint{0.506925in}{0.600424in}}%
\pgfpathlineto{\pgfqpoint{0.521552in}{0.590509in}}%
\pgfpathlineto{\pgfqpoint{0.536179in}{0.583039in}}%
\pgfpathlineto{\pgfqpoint{0.550807in}{0.577834in}}%
\pgfpathlineto{\pgfqpoint{0.565434in}{0.574720in}}%
\pgfpathlineto{\pgfqpoint{0.580062in}{0.573527in}}%
\pgfpathlineto{\pgfqpoint{0.598346in}{0.574489in}}%
\pgfpathlineto{\pgfqpoint{0.616630in}{0.577885in}}%
\pgfpathlineto{\pgfqpoint{0.634915in}{0.583420in}}%
\pgfpathlineto{\pgfqpoint{0.656856in}{0.592481in}}%
\pgfpathlineto{\pgfqpoint{0.682454in}{0.605796in}}%
\pgfpathlineto{\pgfqpoint{0.711709in}{0.623776in}}%
\pgfpathlineto{\pgfqpoint{0.748277in}{0.648975in}}%
\pgfpathlineto{\pgfqpoint{0.865297in}{0.731897in}}%
\pgfpathlineto{\pgfqpoint{0.898208in}{0.751694in}}%
\pgfpathlineto{\pgfqpoint{0.927463in}{0.766842in}}%
\pgfpathlineto{\pgfqpoint{0.953061in}{0.777886in}}%
\pgfpathlineto{\pgfqpoint{0.978659in}{0.786646in}}%
\pgfpathlineto{\pgfqpoint{1.004257in}{0.792963in}}%
\pgfpathlineto{\pgfqpoint{1.029855in}{0.796722in}}%
\pgfpathlineto{\pgfqpoint{1.051797in}{0.797856in}}%
\pgfpathlineto{\pgfqpoint{1.073738in}{0.797046in}}%
\pgfpathlineto{\pgfqpoint{1.095679in}{0.794302in}}%
\pgfpathlineto{\pgfqpoint{1.121277in}{0.788706in}}%
\pgfpathlineto{\pgfqpoint{1.146875in}{0.780630in}}%
\pgfpathlineto{\pgfqpoint{1.172473in}{0.770219in}}%
\pgfpathlineto{\pgfqpoint{1.201728in}{0.755715in}}%
\pgfpathlineto{\pgfqpoint{1.230983in}{0.738784in}}%
\pgfpathlineto{\pgfqpoint{1.267551in}{0.714891in}}%
\pgfpathlineto{\pgfqpoint{1.318747in}{0.678275in}}%
\pgfpathlineto{\pgfqpoint{1.388228in}{0.628622in}}%
\pgfpathlineto{\pgfqpoint{1.421139in}{0.607901in}}%
\pgfpathlineto{\pgfqpoint{1.446737in}{0.594217in}}%
\pgfpathlineto{\pgfqpoint{1.468679in}{0.584758in}}%
\pgfpathlineto{\pgfqpoint{1.490620in}{0.577885in}}%
\pgfpathlineto{\pgfqpoint{1.508904in}{0.574489in}}%
\pgfpathlineto{\pgfqpoint{1.527188in}{0.573527in}}%
\pgfpathlineto{\pgfqpoint{1.541816in}{0.574720in}}%
\pgfpathlineto{\pgfqpoint{1.556443in}{0.577834in}}%
\pgfpathlineto{\pgfqpoint{1.571071in}{0.583039in}}%
\pgfpathlineto{\pgfqpoint{1.585698in}{0.590509in}}%
\pgfpathlineto{\pgfqpoint{1.600325in}{0.600424in}}%
\pgfpathlineto{\pgfqpoint{1.614953in}{0.612969in}}%
\pgfpathlineto{\pgfqpoint{1.629580in}{0.628332in}}%
\pgfpathlineto{\pgfqpoint{1.644208in}{0.646709in}}%
\pgfpathlineto{\pgfqpoint{1.658835in}{0.668298in}}%
\pgfpathlineto{\pgfqpoint{1.673463in}{0.693304in}}%
\pgfpathlineto{\pgfqpoint{1.691747in}{0.729687in}}%
\pgfpathlineto{\pgfqpoint{1.710031in}{0.772157in}}%
\pgfpathlineto{\pgfqpoint{1.728316in}{0.821146in}}%
\pgfpathlineto{\pgfqpoint{1.746600in}{0.877100in}}%
\pgfpathlineto{\pgfqpoint{1.764884in}{0.940474in}}%
\pgfpathlineto{\pgfqpoint{1.783168in}{1.011738in}}%
\pgfpathlineto{\pgfqpoint{1.801453in}{1.091373in}}%
\pgfpathlineto{\pgfqpoint{1.819737in}{1.179872in}}%
\pgfpathlineto{\pgfqpoint{1.841678in}{1.298484in}}%
\pgfpathlineto{\pgfqpoint{1.863619in}{1.431487in}}%
\pgfpathlineto{\pgfqpoint{1.885560in}{1.579802in}}%
\pgfpathlineto{\pgfqpoint{1.907502in}{1.744377in}}%
\pgfpathlineto{\pgfqpoint{1.929443in}{1.926185in}}%
\pgfpathlineto{\pgfqpoint{1.943727in}{2.054300in}}%
\pgfpathlineto{\pgfqpoint{1.943727in}{2.054300in}}%
\pgfusepath{stroke}%
\end{pgfscope}%
\begin{pgfscope}%
\pgfpathrectangle{\pgfqpoint{0.050000in}{0.050000in}}{\pgfqpoint{2.007250in}{1.994300in}}%
\pgfusepath{clip}%
\pgfsetrectcap%
\pgfsetroundjoin%
\pgfsetlinewidth{1.003750pt}%
\definecolor{currentstroke}{rgb}{0.200000,0.133333,0.533333}%
\pgfsetstrokecolor{currentstroke}%
\pgfsetdash{}{0pt}%
\pgfpathmoveto{\pgfqpoint{0.141239in}{1.148426in}}%
\pgfpathlineto{\pgfqpoint{0.163180in}{0.988971in}}%
\pgfpathlineto{\pgfqpoint{0.185121in}{0.847129in}}%
\pgfpathlineto{\pgfqpoint{0.207062in}{0.721885in}}%
\pgfpathlineto{\pgfqpoint{0.225346in}{0.629479in}}%
\pgfpathlineto{\pgfqpoint{0.243631in}{0.547353in}}%
\pgfpathlineto{\pgfqpoint{0.261915in}{0.474958in}}%
\pgfpathlineto{\pgfqpoint{0.280199in}{0.411761in}}%
\pgfpathlineto{\pgfqpoint{0.298484in}{0.357239in}}%
\pgfpathlineto{\pgfqpoint{0.316768in}{0.310880in}}%
\pgfpathlineto{\pgfqpoint{0.331395in}{0.279336in}}%
\pgfpathlineto{\pgfqpoint{0.346023in}{0.252449in}}%
\pgfpathlineto{\pgfqpoint{0.360650in}{0.229973in}}%
\pgfpathlineto{\pgfqpoint{0.375278in}{0.211670in}}%
\pgfpathlineto{\pgfqpoint{0.389905in}{0.197305in}}%
\pgfpathlineto{\pgfqpoint{0.404532in}{0.186648in}}%
\pgfpathlineto{\pgfqpoint{0.419160in}{0.179474in}}%
\pgfpathlineto{\pgfqpoint{0.430131in}{0.176249in}}%
\pgfpathlineto{\pgfqpoint{0.441101in}{0.174769in}}%
\pgfpathlineto{\pgfqpoint{0.455729in}{0.175356in}}%
\pgfpathlineto{\pgfqpoint{0.470356in}{0.178681in}}%
\pgfpathlineto{\pgfqpoint{0.484983in}{0.184542in}}%
\pgfpathlineto{\pgfqpoint{0.499611in}{0.192741in}}%
\pgfpathlineto{\pgfqpoint{0.517895in}{0.205988in}}%
\pgfpathlineto{\pgfqpoint{0.536179in}{0.222225in}}%
\pgfpathlineto{\pgfqpoint{0.558121in}{0.245162in}}%
\pgfpathlineto{\pgfqpoint{0.580062in}{0.271316in}}%
\pgfpathlineto{\pgfqpoint{0.605660in}{0.305149in}}%
\pgfpathlineto{\pgfqpoint{0.638571in}{0.352611in}}%
\pgfpathlineto{\pgfqpoint{0.682454in}{0.420143in}}%
\pgfpathlineto{\pgfqpoint{0.777532in}{0.567512in}}%
\pgfpathlineto{\pgfqpoint{0.814101in}{0.619565in}}%
\pgfpathlineto{\pgfqpoint{0.843356in}{0.657810in}}%
\pgfpathlineto{\pgfqpoint{0.872610in}{0.692363in}}%
\pgfpathlineto{\pgfqpoint{0.898208in}{0.719144in}}%
\pgfpathlineto{\pgfqpoint{0.920150in}{0.739287in}}%
\pgfpathlineto{\pgfqpoint{0.942091in}{0.756658in}}%
\pgfpathlineto{\pgfqpoint{0.964032in}{0.771114in}}%
\pgfpathlineto{\pgfqpoint{0.985973in}{0.782539in}}%
\pgfpathlineto{\pgfqpoint{1.004257in}{0.789679in}}%
\pgfpathlineto{\pgfqpoint{1.022542in}{0.794612in}}%
\pgfpathlineto{\pgfqpoint{1.040826in}{0.797311in}}%
\pgfpathlineto{\pgfqpoint{1.059110in}{0.797761in}}%
\pgfpathlineto{\pgfqpoint{1.077395in}{0.795961in}}%
\pgfpathlineto{\pgfqpoint{1.095679in}{0.791919in}}%
\pgfpathlineto{\pgfqpoint{1.113963in}{0.785658in}}%
\pgfpathlineto{\pgfqpoint{1.132247in}{0.777211in}}%
\pgfpathlineto{\pgfqpoint{1.150532in}{0.766626in}}%
\pgfpathlineto{\pgfqpoint{1.172473in}{0.751185in}}%
\pgfpathlineto{\pgfqpoint{1.194414in}{0.732873in}}%
\pgfpathlineto{\pgfqpoint{1.216355in}{0.711842in}}%
\pgfpathlineto{\pgfqpoint{1.241953in}{0.684101in}}%
\pgfpathlineto{\pgfqpoint{1.267551in}{0.653220in}}%
\pgfpathlineto{\pgfqpoint{1.296806in}{0.614555in}}%
\pgfpathlineto{\pgfqpoint{1.333375in}{0.562093in}}%
\pgfpathlineto{\pgfqpoint{1.377257in}{0.494952in}}%
\pgfpathlineto{\pgfqpoint{1.479649in}{0.336387in}}%
\pgfpathlineto{\pgfqpoint{1.512561in}{0.290262in}}%
\pgfpathlineto{\pgfqpoint{1.538159in}{0.257873in}}%
\pgfpathlineto{\pgfqpoint{1.560100in}{0.233256in}}%
\pgfpathlineto{\pgfqpoint{1.582041in}{0.212144in}}%
\pgfpathlineto{\pgfqpoint{1.600325in}{0.197658in}}%
\pgfpathlineto{\pgfqpoint{1.618610in}{0.186380in}}%
\pgfpathlineto{\pgfqpoint{1.633237in}{0.179916in}}%
\pgfpathlineto{\pgfqpoint{1.647865in}{0.175939in}}%
\pgfpathlineto{\pgfqpoint{1.662492in}{0.174648in}}%
\pgfpathlineto{\pgfqpoint{1.673463in}{0.175566in}}%
\pgfpathlineto{\pgfqpoint{1.684433in}{0.178200in}}%
\pgfpathlineto{\pgfqpoint{1.695404in}{0.182639in}}%
\pgfpathlineto{\pgfqpoint{1.706374in}{0.188976in}}%
\pgfpathlineto{\pgfqpoint{1.717345in}{0.197305in}}%
\pgfpathlineto{\pgfqpoint{1.731972in}{0.211670in}}%
\pgfpathlineto{\pgfqpoint{1.746600in}{0.229973in}}%
\pgfpathlineto{\pgfqpoint{1.761227in}{0.252449in}}%
\pgfpathlineto{\pgfqpoint{1.775855in}{0.279336in}}%
\pgfpathlineto{\pgfqpoint{1.790482in}{0.310880in}}%
\pgfpathlineto{\pgfqpoint{1.805110in}{0.347330in}}%
\pgfpathlineto{\pgfqpoint{1.819737in}{0.388940in}}%
\pgfpathlineto{\pgfqpoint{1.838021in}{0.448605in}}%
\pgfpathlineto{\pgfqpoint{1.856306in}{0.517257in}}%
\pgfpathlineto{\pgfqpoint{1.874590in}{0.595426in}}%
\pgfpathlineto{\pgfqpoint{1.892874in}{0.683653in}}%
\pgfpathlineto{\pgfqpoint{1.911158in}{0.782494in}}%
\pgfpathlineto{\pgfqpoint{1.929443in}{0.892516in}}%
\pgfpathlineto{\pgfqpoint{1.947727in}{1.014296in}}%
\pgfpathlineto{\pgfqpoint{1.966011in}{1.148426in}}%
\pgfpathlineto{\pgfqpoint{1.966011in}{1.148426in}}%
\pgfusepath{stroke}%
\end{pgfscope}%
\begin{pgfscope}%
\pgfpathrectangle{\pgfqpoint{0.050000in}{0.050000in}}{\pgfqpoint{2.007250in}{1.994300in}}%
\pgfusepath{clip}%
\pgfsetrectcap%
\pgfsetroundjoin%
\pgfsetlinewidth{1.003750pt}%
\definecolor{currentstroke}{rgb}{0.866667,0.800000,0.466667}%
\pgfsetstrokecolor{currentstroke}%
\pgfsetdash{}{0pt}%
\pgfpathmoveto{\pgfqpoint{0.710063in}{2.054300in}}%
\pgfpathlineto{\pgfqpoint{0.733650in}{1.880372in}}%
\pgfpathlineto{\pgfqpoint{0.759248in}{1.707893in}}%
\pgfpathlineto{\pgfqpoint{0.784846in}{1.551767in}}%
\pgfpathlineto{\pgfqpoint{0.806787in}{1.430566in}}%
\pgfpathlineto{\pgfqpoint{0.828728in}{1.320701in}}%
\pgfpathlineto{\pgfqpoint{0.850669in}{1.221900in}}%
\pgfpathlineto{\pgfqpoint{0.872610in}{1.133915in}}%
\pgfpathlineto{\pgfqpoint{0.890895in}{1.068697in}}%
\pgfpathlineto{\pgfqpoint{0.909179in}{1.010721in}}%
\pgfpathlineto{\pgfqpoint{0.927463in}{0.959885in}}%
\pgfpathlineto{\pgfqpoint{0.945748in}{0.916098in}}%
\pgfpathlineto{\pgfqpoint{0.960375in}{0.886092in}}%
\pgfpathlineto{\pgfqpoint{0.975003in}{0.860512in}}%
\pgfpathlineto{\pgfqpoint{0.989630in}{0.839330in}}%
\pgfpathlineto{\pgfqpoint{1.004257in}{0.822522in}}%
\pgfpathlineto{\pgfqpoint{1.015228in}{0.812773in}}%
\pgfpathlineto{\pgfqpoint{1.026199in}{0.805468in}}%
\pgfpathlineto{\pgfqpoint{1.037169in}{0.800600in}}%
\pgfpathlineto{\pgfqpoint{1.048140in}{0.798167in}}%
\pgfpathlineto{\pgfqpoint{1.059110in}{0.798167in}}%
\pgfpathlineto{\pgfqpoint{1.070081in}{0.800600in}}%
\pgfpathlineto{\pgfqpoint{1.081051in}{0.805468in}}%
\pgfpathlineto{\pgfqpoint{1.092022in}{0.812773in}}%
\pgfpathlineto{\pgfqpoint{1.102993in}{0.822522in}}%
\pgfpathlineto{\pgfqpoint{1.113963in}{0.834719in}}%
\pgfpathlineto{\pgfqpoint{1.128591in}{0.854805in}}%
\pgfpathlineto{\pgfqpoint{1.143218in}{0.879283in}}%
\pgfpathlineto{\pgfqpoint{1.157845in}{0.908180in}}%
\pgfpathlineto{\pgfqpoint{1.172473in}{0.941529in}}%
\pgfpathlineto{\pgfqpoint{1.190757in}{0.989535in}}%
\pgfpathlineto{\pgfqpoint{1.209042in}{1.044643in}}%
\pgfpathlineto{\pgfqpoint{1.227326in}{1.106951in}}%
\pgfpathlineto{\pgfqpoint{1.245610in}{1.176570in}}%
\pgfpathlineto{\pgfqpoint{1.267551in}{1.269933in}}%
\pgfpathlineto{\pgfqpoint{1.289492in}{1.374234in}}%
\pgfpathlineto{\pgfqpoint{1.311434in}{1.489731in}}%
\pgfpathlineto{\pgfqpoint{1.333375in}{1.616710in}}%
\pgfpathlineto{\pgfqpoint{1.355316in}{1.755481in}}%
\pgfpathlineto{\pgfqpoint{1.380914in}{1.932732in}}%
\pgfpathlineto{\pgfqpoint{1.397187in}{2.054300in}}%
\pgfpathlineto{\pgfqpoint{1.397187in}{2.054300in}}%
\pgfusepath{stroke}%
\end{pgfscope}%
\begin{pgfscope}%
\pgfpathrectangle{\pgfqpoint{0.050000in}{0.050000in}}{\pgfqpoint{2.007250in}{1.994300in}}%
\pgfusepath{clip}%
\pgfsetrectcap%
\pgfsetroundjoin%
\pgfsetlinewidth{1.003750pt}%
\definecolor{currentstroke}{rgb}{0.066667,0.466667,0.200000}%
\pgfsetstrokecolor{currentstroke}%
\pgfsetdash{}{0pt}%
\pgfpathmoveto{\pgfqpoint{0.328840in}{2.054300in}}%
\pgfpathlineto{\pgfqpoint{0.349680in}{1.915874in}}%
\pgfpathlineto{\pgfqpoint{0.371621in}{1.782868in}}%
\pgfpathlineto{\pgfqpoint{0.393562in}{1.662098in}}%
\pgfpathlineto{\pgfqpoint{0.415503in}{1.552790in}}%
\pgfpathlineto{\pgfqpoint{0.437444in}{1.454194in}}%
\pgfpathlineto{\pgfqpoint{0.459385in}{1.365586in}}%
\pgfpathlineto{\pgfqpoint{0.481327in}{1.286268in}}%
\pgfpathlineto{\pgfqpoint{0.503268in}{1.215567in}}%
\pgfpathlineto{\pgfqpoint{0.525209in}{1.152835in}}%
\pgfpathlineto{\pgfqpoint{0.547150in}{1.097449in}}%
\pgfpathlineto{\pgfqpoint{0.569091in}{1.048812in}}%
\pgfpathlineto{\pgfqpoint{0.591032in}{1.006353in}}%
\pgfpathlineto{\pgfqpoint{0.612973in}{0.969523in}}%
\pgfpathlineto{\pgfqpoint{0.631258in}{0.942757in}}%
\pgfpathlineto{\pgfqpoint{0.649542in}{0.919250in}}%
\pgfpathlineto{\pgfqpoint{0.671483in}{0.894956in}}%
\pgfpathlineto{\pgfqpoint{0.693424in}{0.874505in}}%
\pgfpathlineto{\pgfqpoint{0.715366in}{0.857469in}}%
\pgfpathlineto{\pgfqpoint{0.737307in}{0.843444in}}%
\pgfpathlineto{\pgfqpoint{0.759248in}{0.832053in}}%
\pgfpathlineto{\pgfqpoint{0.781189in}{0.822944in}}%
\pgfpathlineto{\pgfqpoint{0.806787in}{0.814765in}}%
\pgfpathlineto{\pgfqpoint{0.836042in}{0.808067in}}%
\pgfpathlineto{\pgfqpoint{0.868954in}{0.803158in}}%
\pgfpathlineto{\pgfqpoint{0.905522in}{0.800053in}}%
\pgfpathlineto{\pgfqpoint{0.953061in}{0.798328in}}%
\pgfpathlineto{\pgfqpoint{1.040826in}{0.797863in}}%
\pgfpathlineto{\pgfqpoint{1.176130in}{0.798888in}}%
\pgfpathlineto{\pgfqpoint{1.220012in}{0.801352in}}%
\pgfpathlineto{\pgfqpoint{1.256581in}{0.805587in}}%
\pgfpathlineto{\pgfqpoint{1.285836in}{0.811100in}}%
\pgfpathlineto{\pgfqpoint{1.311434in}{0.817976in}}%
\pgfpathlineto{\pgfqpoint{1.337032in}{0.827234in}}%
\pgfpathlineto{\pgfqpoint{1.358973in}{0.837442in}}%
\pgfpathlineto{\pgfqpoint{1.380914in}{0.850104in}}%
\pgfpathlineto{\pgfqpoint{1.402855in}{0.865586in}}%
\pgfpathlineto{\pgfqpoint{1.424796in}{0.884277in}}%
\pgfpathlineto{\pgfqpoint{1.443081in}{0.902605in}}%
\pgfpathlineto{\pgfqpoint{1.461365in}{0.923704in}}%
\pgfpathlineto{\pgfqpoint{1.479649in}{0.947840in}}%
\pgfpathlineto{\pgfqpoint{1.497933in}{0.975293in}}%
\pgfpathlineto{\pgfqpoint{1.516218in}{1.006353in}}%
\pgfpathlineto{\pgfqpoint{1.534502in}{1.041322in}}%
\pgfpathlineto{\pgfqpoint{1.552786in}{1.080516in}}%
\pgfpathlineto{\pgfqpoint{1.571071in}{1.124262in}}%
\pgfpathlineto{\pgfqpoint{1.593012in}{1.183245in}}%
\pgfpathlineto{\pgfqpoint{1.614953in}{1.249882in}}%
\pgfpathlineto{\pgfqpoint{1.636894in}{1.324808in}}%
\pgfpathlineto{\pgfqpoint{1.658835in}{1.408685in}}%
\pgfpathlineto{\pgfqpoint{1.680776in}{1.502198in}}%
\pgfpathlineto{\pgfqpoint{1.702718in}{1.606059in}}%
\pgfpathlineto{\pgfqpoint{1.724659in}{1.721003in}}%
\pgfpathlineto{\pgfqpoint{1.746600in}{1.847792in}}%
\pgfpathlineto{\pgfqpoint{1.768541in}{1.987214in}}%
\pgfpathlineto{\pgfqpoint{1.778410in}{2.054300in}}%
\pgfpathlineto{\pgfqpoint{1.778410in}{2.054300in}}%
\pgfusepath{stroke}%
\end{pgfscope}%
\begin{pgfscope}%
\pgfsetbuttcap%
\pgfsetmiterjoin%
\definecolor{currentfill}{rgb}{0.000000,0.000000,0.000000}%
\pgfsetfillcolor{currentfill}%
\pgfsetlinewidth{1.003750pt}%
\definecolor{currentstroke}{rgb}{0.000000,0.000000,0.000000}%
\pgfsetstrokecolor{currentstroke}%
\pgfsetdash{}{0pt}%
\pgfsys@defobject{currentmarker}{\pgfqpoint{-0.041667in}{-0.041667in}}{\pgfqpoint{0.041667in}{0.041667in}}{%
\pgfpathmoveto{\pgfqpoint{0.041667in}{-0.000000in}}%
\pgfpathlineto{\pgfqpoint{-0.041667in}{0.041667in}}%
\pgfpathlineto{\pgfqpoint{-0.041667in}{-0.041667in}}%
\pgfpathlineto{\pgfqpoint{0.041667in}{-0.000000in}}%
\pgfpathclose%
\pgfusepath{stroke,fill}%
}%
\begin{pgfscope}%
\pgfsys@transformshift{2.057250in}{0.797862in}%
\pgfsys@useobject{currentmarker}{}%
\end{pgfscope}%
\end{pgfscope}%
\begin{pgfscope}%
\pgfsetbuttcap%
\pgfsetmiterjoin%
\definecolor{currentfill}{rgb}{0.000000,0.000000,0.000000}%
\pgfsetfillcolor{currentfill}%
\pgfsetlinewidth{1.003750pt}%
\definecolor{currentstroke}{rgb}{0.000000,0.000000,0.000000}%
\pgfsetstrokecolor{currentstroke}%
\pgfsetdash{}{0pt}%
\pgfsys@defobject{currentmarker}{\pgfqpoint{-0.041667in}{-0.041667in}}{\pgfqpoint{0.041667in}{0.041667in}}{%
\pgfpathmoveto{\pgfqpoint{0.000000in}{0.041667in}}%
\pgfpathlineto{\pgfqpoint{-0.041667in}{-0.041667in}}%
\pgfpathlineto{\pgfqpoint{0.041667in}{-0.041667in}}%
\pgfpathlineto{\pgfqpoint{0.000000in}{0.041667in}}%
\pgfpathclose%
\pgfusepath{stroke,fill}%
}%
\begin{pgfscope}%
\pgfsys@transformshift{1.053625in}{2.044300in}%
\pgfsys@useobject{currentmarker}{}%
\end{pgfscope}%
\end{pgfscope}%
\begin{pgfscope}%
\pgfsetrectcap%
\pgfsetmiterjoin%
\pgfsetlinewidth{0.501875pt}%
\definecolor{currentstroke}{rgb}{0.000000,0.000000,0.000000}%
\pgfsetstrokecolor{currentstroke}%
\pgfsetdash{}{0pt}%
\pgfpathmoveto{\pgfqpoint{1.053625in}{0.050000in}}%
\pgfpathlineto{\pgfqpoint{1.053625in}{2.044300in}}%
\pgfusepath{stroke}%
\end{pgfscope}%
\begin{pgfscope}%
\pgfsetrectcap%
\pgfsetmiterjoin%
\pgfsetlinewidth{0.501875pt}%
\definecolor{currentstroke}{rgb}{0.000000,0.000000,0.000000}%
\pgfsetstrokecolor{currentstroke}%
\pgfsetdash{}{0pt}%
\pgfpathmoveto{\pgfqpoint{0.050000in}{0.797862in}}%
\pgfpathlineto{\pgfqpoint{2.057250in}{0.797862in}}%
\pgfusepath{stroke}%
\end{pgfscope}%
\begin{pgfscope}%
\definecolor{textcolor}{rgb}{0.800000,0.400000,0.466667}%
\pgfsetstrokecolor{textcolor}%
\pgfsetfillcolor{textcolor}%
\pgftext[x=0.203668in, y=2.025482in, left, base,rotate=280.000000]{\color{textcolor}{\rmfamily\fontsize{8.330000}{9.996000}\selectfont\catcode`\^=\active\def^{\ifmmode\sp\else\^{}\fi}\catcode`\%=\active\def%{\%}$T < T_C$}}%
\end{pgfscope}%
\begin{pgfscope}%
\definecolor{textcolor}{rgb}{0.866667,0.800000,0.466667}%
\pgfsetstrokecolor{textcolor}%
\pgfsetfillcolor{textcolor}%
\pgftext[x=0.751100in, y=2.025482in, left, base,rotate=280.000000]{\color{textcolor}{\rmfamily\fontsize{8.330000}{9.996000}\selectfont\catcode`\^=\active\def^{\ifmmode\sp\else\^{}\fi}\catcode`\%=\active\def%{\%}$T > T_C$}}%
\end{pgfscope}%
\begin{pgfscope}%
\definecolor{textcolor}{rgb}{0.066667,0.466667,0.200000}%
\pgfsetstrokecolor{textcolor}%
\pgfsetfillcolor{textcolor}%
\pgftext[x=0.386145in, y=2.024684in, left, base,rotate=280.000000]{\color{textcolor}{\rmfamily\fontsize{8.330000}{9.996000}\selectfont\catcode`\^=\active\def^{\ifmmode\sp\else\^{}\fi}\catcode`\%=\active\def%{\%}$T = T_C$}}%
\end{pgfscope}%
\end{pgfpicture}%
\makeatother%
\endgroup%

	\end{subfigure}%
	\begin{subfigure}[b]{0.5\textwidth}
		\centering
		\caption{\hfill\null}\label{sfig:Ginzburg Landau free energy}
		%% Creator: Matplotlib, PGF backend
%%
%% To include the figure in your LaTeX document, write
%%   \input{<filename>.pgf}
%%
%% Make sure the required packages are loaded in your preamble
%%   \usepackage{pgf}
%%
%% Also ensure that all the required font packages are loaded; for instance,
%% the lmodern package is sometimes necessary when using math font.
%%   \usepackage{lmodern}
%%
%% Figures using additional raster images can only be included by \input if
%% they are in the same directory as the main LaTeX file. For loading figures
%% from other directories you can use the `import` package
%%   \usepackage{import}
%%
%% and then include the figures with
%%   \import{<path to file>}{<filename>.pgf}
%%
%% Matplotlib used the following preamble
%%   \def\mathdefault#1{#1}
%%   \everymath=\expandafter{\the\everymath\displaystyle}
%%   \IfFileExists{scrextend.sty}{
%%     \usepackage[fontsize=11.000000pt]{scrextend}
%%   }{
%%     \renewcommand{\normalsize}{\fontsize{11.000000}{13.200000}\selectfont}
%%     \normalsize
%%   }
%%   \usepackage{fontspec}\usepackage{unicode-math}\setmathfont{texgyrepagella-math.otf}\setmainfont{texgyrepagella-math}
%%   \makeatletter\@ifpackageloaded{underscore}{}{\usepackage[strings]{underscore}}\makeatother
%%
\begingroup%
\makeatletter%
\begin{pgfpicture}%
\pgfpathrectangle{\pgfpointorigin}{\pgfqpoint{2.584000in}{2.584000in}}%
\pgfusepath{use as bounding box, clip}%
\begin{pgfscope}%
\pgfsetbuttcap%
\pgfsetmiterjoin%
\definecolor{currentfill}{rgb}{1.000000,1.000000,1.000000}%
\pgfsetfillcolor{currentfill}%
\pgfsetlinewidth{0.000000pt}%
\definecolor{currentstroke}{rgb}{1.000000,1.000000,1.000000}%
\pgfsetstrokecolor{currentstroke}%
\pgfsetdash{}{0pt}%
\pgfpathmoveto{\pgfqpoint{0.000000in}{0.000000in}}%
\pgfpathlineto{\pgfqpoint{2.584000in}{0.000000in}}%
\pgfpathlineto{\pgfqpoint{2.584000in}{2.584000in}}%
\pgfpathlineto{\pgfqpoint{0.000000in}{2.584000in}}%
\pgfpathlineto{\pgfqpoint{0.000000in}{0.000000in}}%
\pgfpathclose%
\pgfusepath{fill}%
\end{pgfscope}%
\begin{pgfscope}%
\pgfsetbuttcap%
\pgfsetmiterjoin%
\definecolor{currentfill}{rgb}{1.000000,1.000000,1.000000}%
\pgfsetfillcolor{currentfill}%
\pgfsetlinewidth{0.000000pt}%
\definecolor{currentstroke}{rgb}{0.000000,0.000000,0.000000}%
\pgfsetstrokecolor{currentstroke}%
\pgfsetstrokeopacity{0.000000}%
\pgfsetdash{}{0pt}%
\pgfpathmoveto{\pgfqpoint{0.329460in}{0.284240in}}%
\pgfpathlineto{\pgfqpoint{2.319140in}{0.284240in}}%
\pgfpathlineto{\pgfqpoint{2.319140in}{2.273920in}}%
\pgfpathlineto{\pgfqpoint{0.329460in}{2.273920in}}%
\pgfpathlineto{\pgfqpoint{0.329460in}{0.284240in}}%
\pgfpathclose%
\pgfusepath{fill}%
\end{pgfscope}%
\begin{pgfscope}%
\pgfpathrectangle{\pgfqpoint{0.329460in}{0.284240in}}{\pgfqpoint{1.989680in}{1.989680in}}%
\pgfusepath{clip}%
\pgfsetbuttcap%
\pgfsetroundjoin%
\pgfsetlinewidth{1.505625pt}%
\definecolor{currentstroke}{rgb}{0.000000,0.000000,0.000000}%
\pgfsetstrokecolor{currentstroke}%
\pgfsetdash{}{0pt}%
\pgfpathmoveto{\pgfqpoint{0.360703in}{1.352635in}}%
\pgfpathlineto{\pgfqpoint{1.351188in}{1.168388in}}%
\pgfusepath{stroke}%
\end{pgfscope}%
\begin{pgfscope}%
\pgfpathrectangle{\pgfqpoint{0.329460in}{0.284240in}}{\pgfqpoint{1.989680in}{1.989680in}}%
\pgfusepath{clip}%
\pgfsetbuttcap%
\pgfsetroundjoin%
\pgfsetlinewidth{1.505625pt}%
\definecolor{currentstroke}{rgb}{0.000000,0.000000,0.000000}%
\pgfsetstrokecolor{currentstroke}%
\pgfsetdash{}{0pt}%
\pgfpathmoveto{\pgfqpoint{0.360703in}{1.352635in}}%
\pgfpathlineto{\pgfqpoint{0.410800in}{1.273131in}}%
\pgfusepath{stroke}%
\end{pgfscope}%
\begin{pgfscope}%
\pgfpathrectangle{\pgfqpoint{0.329460in}{0.284240in}}{\pgfqpoint{1.989680in}{1.989680in}}%
\pgfusepath{clip}%
\pgfsetbuttcap%
\pgfsetroundjoin%
\pgfsetlinewidth{1.505625pt}%
\definecolor{currentstroke}{rgb}{0.000000,0.000000,0.000000}%
\pgfsetstrokecolor{currentstroke}%
\pgfsetdash{}{0pt}%
\pgfpathmoveto{\pgfqpoint{0.360703in}{1.352635in}}%
\pgfpathlineto{\pgfqpoint{0.403606in}{1.415377in}}%
\pgfusepath{stroke}%
\end{pgfscope}%
\begin{pgfscope}%
\definecolor{textcolor}{rgb}{0.000000,0.000000,0.000000}%
\pgfsetstrokecolor{textcolor}%
\pgfsetfillcolor{textcolor}%
\pgftext[x=0.406327in,y=1.370744in,left,base]{\color{textcolor}{\rmfamily\fontsize{11.000000}{13.200000}\selectfont\catcode`\^=\active\def^{\ifmmode\sp\else\^{}\fi}\catcode`\%=\active\def%{\%}$\Re \Psi$}}%
\end{pgfscope}%
\begin{pgfscope}%
\definecolor{textcolor}{rgb}{0.000000,0.000000,0.000000}%
\pgfsetstrokecolor{textcolor}%
\pgfsetfillcolor{textcolor}%
\pgftext[x=1.012489in,y=0.576048in,left,base]{\color{textcolor}{\rmfamily\fontsize{11.000000}{13.200000}\selectfont\catcode`\^=\active\def^{\ifmmode\sp\else\^{}\fi}\catcode`\%=\active\def%{\%}$\Im \Psi$}}%
\end{pgfscope}%
\begin{pgfscope}%
\definecolor{textcolor}{rgb}{0.000000,0.000000,0.000000}%
\pgfsetstrokecolor{textcolor}%
\pgfsetfillcolor{textcolor}%
\pgftext[x=1.407079in,y=2.005447in,left,base]{\color{textcolor}{\rmfamily\fontsize{11.000000}{13.200000}\selectfont\catcode`\^=\active\def^{\ifmmode\sp\else\^{}\fi}\catcode`\%=\active\def%{\%}$f_{\mathrm{L}}$}}%
\end{pgfscope}%
\begin{pgfscope}%
\pgfpathrectangle{\pgfqpoint{0.329460in}{0.284240in}}{\pgfqpoint{1.989680in}{1.989680in}}%
\pgfusepath{clip}%
\pgfsetbuttcap%
\pgfsetroundjoin%
\pgfsetlinewidth{1.505625pt}%
\definecolor{currentstroke}{rgb}{0.000000,0.000000,0.000000}%
\pgfsetstrokecolor{currentstroke}%
\pgfsetdash{}{0pt}%
\pgfpathmoveto{\pgfqpoint{1.351188in}{2.025761in}}%
\pgfpathlineto{\pgfqpoint{1.351188in}{1.168388in}}%
\pgfusepath{stroke}%
\end{pgfscope}%
\begin{pgfscope}%
\pgfpathrectangle{\pgfqpoint{0.329460in}{0.284240in}}{\pgfqpoint{1.989680in}{1.989680in}}%
\pgfusepath{clip}%
\pgfsetbuttcap%
\pgfsetroundjoin%
\pgfsetlinewidth{1.505625pt}%
\definecolor{currentstroke}{rgb}{0.000000,0.000000,0.000000}%
\pgfsetstrokecolor{currentstroke}%
\pgfsetdash{}{0pt}%
\pgfpathmoveto{\pgfqpoint{1.351188in}{2.025761in}}%
\pgfpathlineto{\pgfqpoint{1.345165in}{1.897636in}}%
\pgfusepath{stroke}%
\end{pgfscope}%
\begin{pgfscope}%
\pgfpathrectangle{\pgfqpoint{0.329460in}{0.284240in}}{\pgfqpoint{1.989680in}{1.989680in}}%
\pgfusepath{clip}%
\pgfsetbuttcap%
\pgfsetroundjoin%
\pgfsetlinewidth{1.505625pt}%
\definecolor{currentstroke}{rgb}{0.000000,0.000000,0.000000}%
\pgfsetstrokecolor{currentstroke}%
\pgfsetdash{}{0pt}%
\pgfpathmoveto{\pgfqpoint{1.351188in}{2.025761in}}%
\pgfpathlineto{\pgfqpoint{1.357212in}{1.895652in}}%
\pgfusepath{stroke}%
\end{pgfscope}%
\begin{pgfscope}%
\pgfpathrectangle{\pgfqpoint{0.329460in}{0.284240in}}{\pgfqpoint{1.989680in}{1.989680in}}%
\pgfusepath{clip}%
\pgfsetbuttcap%
\pgfsetroundjoin%
\definecolor{currentfill}{rgb}{0.282327,0.094955,0.417331}%
\pgfsetfillcolor{currentfill}%
\pgfsetlinewidth{0.000000pt}%
\definecolor{currentstroke}{rgb}{0.000000,0.000000,0.000000}%
\pgfsetstrokecolor{currentstroke}%
\pgfsetdash{}{0pt}%
\pgfpathmoveto{\pgfqpoint{1.282033in}{1.353486in}}%
\pgfpathlineto{\pgfqpoint{1.281633in}{1.359403in}}%
\pgfpathlineto{\pgfqpoint{1.281232in}{1.365615in}}%
\pgfpathlineto{\pgfqpoint{1.280829in}{1.372125in}}%
\pgfpathlineto{\pgfqpoint{1.280426in}{1.378940in}}%
\pgfpathlineto{\pgfqpoint{1.298313in}{1.379904in}}%
\pgfpathlineto{\pgfqpoint{1.316246in}{1.380589in}}%
\pgfpathlineto{\pgfqpoint{1.334209in}{1.380994in}}%
\pgfpathlineto{\pgfqpoint{1.352186in}{1.381119in}}%
\pgfpathlineto{\pgfqpoint{1.352181in}{1.374294in}}%
\pgfpathlineto{\pgfqpoint{1.352175in}{1.367773in}}%
\pgfpathlineto{\pgfqpoint{1.352169in}{1.361552in}}%
\pgfpathlineto{\pgfqpoint{1.352164in}{1.355625in}}%
\pgfpathlineto{\pgfqpoint{1.334594in}{1.355502in}}%
\pgfpathlineto{\pgfqpoint{1.317040in}{1.355105in}}%
\pgfpathlineto{\pgfqpoint{1.299514in}{1.354433in}}%
\pgfpathlineto{\pgfqpoint{1.282033in}{1.353486in}}%
\pgfpathclose%
\pgfusepath{fill}%
\end{pgfscope}%
\begin{pgfscope}%
\pgfpathrectangle{\pgfqpoint{0.329460in}{0.284240in}}{\pgfqpoint{1.989680in}{1.989680in}}%
\pgfusepath{clip}%
\pgfsetbuttcap%
\pgfsetroundjoin%
\definecolor{currentfill}{rgb}{0.282327,0.094955,0.417331}%
\pgfsetfillcolor{currentfill}%
\pgfsetlinewidth{0.000000pt}%
\definecolor{currentstroke}{rgb}{0.000000,0.000000,0.000000}%
\pgfsetstrokecolor{currentstroke}%
\pgfsetdash{}{0pt}%
\pgfpathmoveto{\pgfqpoint{1.352164in}{1.355625in}}%
\pgfpathlineto{\pgfqpoint{1.352169in}{1.361552in}}%
\pgfpathlineto{\pgfqpoint{1.352175in}{1.367773in}}%
\pgfpathlineto{\pgfqpoint{1.352181in}{1.374294in}}%
\pgfpathlineto{\pgfqpoint{1.352186in}{1.381119in}}%
\pgfpathlineto{\pgfqpoint{1.370163in}{1.380963in}}%
\pgfpathlineto{\pgfqpoint{1.388123in}{1.380527in}}%
\pgfpathlineto{\pgfqpoint{1.406052in}{1.379811in}}%
\pgfpathlineto{\pgfqpoint{1.423933in}{1.378816in}}%
\pgfpathlineto{\pgfqpoint{1.423518in}{1.372001in}}%
\pgfpathlineto{\pgfqpoint{1.423105in}{1.365491in}}%
\pgfpathlineto{\pgfqpoint{1.422692in}{1.359281in}}%
\pgfpathlineto{\pgfqpoint{1.422281in}{1.353364in}}%
\pgfpathlineto{\pgfqpoint{1.404806in}{1.354341in}}%
\pgfpathlineto{\pgfqpoint{1.387285in}{1.355044in}}%
\pgfpathlineto{\pgfqpoint{1.369732in}{1.355472in}}%
\pgfpathlineto{\pgfqpoint{1.352164in}{1.355625in}}%
\pgfpathclose%
\pgfusepath{fill}%
\end{pgfscope}%
\begin{pgfscope}%
\pgfpathrectangle{\pgfqpoint{0.329460in}{0.284240in}}{\pgfqpoint{1.989680in}{1.989680in}}%
\pgfusepath{clip}%
\pgfsetbuttcap%
\pgfsetroundjoin%
\definecolor{currentfill}{rgb}{0.282884,0.135920,0.453427}%
\pgfsetfillcolor{currentfill}%
\pgfsetlinewidth{0.000000pt}%
\definecolor{currentstroke}{rgb}{0.000000,0.000000,0.000000}%
\pgfsetstrokecolor{currentstroke}%
\pgfsetdash{}{0pt}%
\pgfpathmoveto{\pgfqpoint{1.280426in}{1.378940in}}%
\pgfpathlineto{\pgfqpoint{1.280021in}{1.386064in}}%
\pgfpathlineto{\pgfqpoint{1.279615in}{1.393503in}}%
\pgfpathlineto{\pgfqpoint{1.279208in}{1.401261in}}%
\pgfpathlineto{\pgfqpoint{1.278799in}{1.409344in}}%
\pgfpathlineto{\pgfqpoint{1.297098in}{1.410326in}}%
\pgfpathlineto{\pgfqpoint{1.315443in}{1.411023in}}%
\pgfpathlineto{\pgfqpoint{1.333819in}{1.411435in}}%
\pgfpathlineto{\pgfqpoint{1.352209in}{1.411562in}}%
\pgfpathlineto{\pgfqpoint{1.352204in}{1.403469in}}%
\pgfpathlineto{\pgfqpoint{1.352198in}{1.395701in}}%
\pgfpathlineto{\pgfqpoint{1.352192in}{1.388253in}}%
\pgfpathlineto{\pgfqpoint{1.352186in}{1.381119in}}%
\pgfpathlineto{\pgfqpoint{1.334209in}{1.380994in}}%
\pgfpathlineto{\pgfqpoint{1.316246in}{1.380589in}}%
\pgfpathlineto{\pgfqpoint{1.298313in}{1.379904in}}%
\pgfpathlineto{\pgfqpoint{1.280426in}{1.378940in}}%
\pgfpathclose%
\pgfusepath{fill}%
\end{pgfscope}%
\begin{pgfscope}%
\pgfpathrectangle{\pgfqpoint{0.329460in}{0.284240in}}{\pgfqpoint{1.989680in}{1.989680in}}%
\pgfusepath{clip}%
\pgfsetbuttcap%
\pgfsetroundjoin%
\definecolor{currentfill}{rgb}{0.282884,0.135920,0.453427}%
\pgfsetfillcolor{currentfill}%
\pgfsetlinewidth{0.000000pt}%
\definecolor{currentstroke}{rgb}{0.000000,0.000000,0.000000}%
\pgfsetstrokecolor{currentstroke}%
\pgfsetdash{}{0pt}%
\pgfpathmoveto{\pgfqpoint{1.352186in}{1.381119in}}%
\pgfpathlineto{\pgfqpoint{1.352192in}{1.388253in}}%
\pgfpathlineto{\pgfqpoint{1.352198in}{1.395701in}}%
\pgfpathlineto{\pgfqpoint{1.352204in}{1.403469in}}%
\pgfpathlineto{\pgfqpoint{1.352209in}{1.411562in}}%
\pgfpathlineto{\pgfqpoint{1.370599in}{1.411404in}}%
\pgfpathlineto{\pgfqpoint{1.388972in}{1.410960in}}%
\pgfpathlineto{\pgfqpoint{1.407313in}{1.410231in}}%
\pgfpathlineto{\pgfqpoint{1.425605in}{1.409218in}}%
\pgfpathlineto{\pgfqpoint{1.425185in}{1.401135in}}%
\pgfpathlineto{\pgfqpoint{1.424767in}{1.393377in}}%
\pgfpathlineto{\pgfqpoint{1.424349in}{1.385939in}}%
\pgfpathlineto{\pgfqpoint{1.423933in}{1.378816in}}%
\pgfpathlineto{\pgfqpoint{1.406052in}{1.379811in}}%
\pgfpathlineto{\pgfqpoint{1.388123in}{1.380527in}}%
\pgfpathlineto{\pgfqpoint{1.370163in}{1.380963in}}%
\pgfpathlineto{\pgfqpoint{1.352186in}{1.381119in}}%
\pgfpathclose%
\pgfusepath{fill}%
\end{pgfscope}%
\begin{pgfscope}%
\pgfpathrectangle{\pgfqpoint{0.329460in}{0.284240in}}{\pgfqpoint{1.989680in}{1.989680in}}%
\pgfusepath{clip}%
\pgfsetbuttcap%
\pgfsetroundjoin%
\definecolor{currentfill}{rgb}{0.277941,0.056324,0.381191}%
\pgfsetfillcolor{currentfill}%
\pgfsetlinewidth{0.000000pt}%
\definecolor{currentstroke}{rgb}{0.000000,0.000000,0.000000}%
\pgfsetstrokecolor{currentstroke}%
\pgfsetdash{}{0pt}%
\pgfpathmoveto{\pgfqpoint{1.283624in}{1.332664in}}%
\pgfpathlineto{\pgfqpoint{1.283228in}{1.337452in}}%
\pgfpathlineto{\pgfqpoint{1.282830in}{1.342516in}}%
\pgfpathlineto{\pgfqpoint{1.282432in}{1.347859in}}%
\pgfpathlineto{\pgfqpoint{1.282033in}{1.353486in}}%
\pgfpathlineto{\pgfqpoint{1.299514in}{1.354433in}}%
\pgfpathlineto{\pgfqpoint{1.317040in}{1.355105in}}%
\pgfpathlineto{\pgfqpoint{1.334594in}{1.355502in}}%
\pgfpathlineto{\pgfqpoint{1.352164in}{1.355625in}}%
\pgfpathlineto{\pgfqpoint{1.352158in}{1.349987in}}%
\pgfpathlineto{\pgfqpoint{1.352153in}{1.344633in}}%
\pgfpathlineto{\pgfqpoint{1.352147in}{1.339559in}}%
\pgfpathlineto{\pgfqpoint{1.352141in}{1.334760in}}%
\pgfpathlineto{\pgfqpoint{1.334976in}{1.334640in}}%
\pgfpathlineto{\pgfqpoint{1.317825in}{1.334251in}}%
\pgfpathlineto{\pgfqpoint{1.300702in}{1.333592in}}%
\pgfpathlineto{\pgfqpoint{1.283624in}{1.332664in}}%
\pgfpathclose%
\pgfusepath{fill}%
\end{pgfscope}%
\begin{pgfscope}%
\pgfpathrectangle{\pgfqpoint{0.329460in}{0.284240in}}{\pgfqpoint{1.989680in}{1.989680in}}%
\pgfusepath{clip}%
\pgfsetbuttcap%
\pgfsetroundjoin%
\definecolor{currentfill}{rgb}{0.277941,0.056324,0.381191}%
\pgfsetfillcolor{currentfill}%
\pgfsetlinewidth{0.000000pt}%
\definecolor{currentstroke}{rgb}{0.000000,0.000000,0.000000}%
\pgfsetstrokecolor{currentstroke}%
\pgfsetdash{}{0pt}%
\pgfpathmoveto{\pgfqpoint{1.352141in}{1.334760in}}%
\pgfpathlineto{\pgfqpoint{1.352147in}{1.339559in}}%
\pgfpathlineto{\pgfqpoint{1.352153in}{1.344633in}}%
\pgfpathlineto{\pgfqpoint{1.352158in}{1.349987in}}%
\pgfpathlineto{\pgfqpoint{1.352164in}{1.355625in}}%
\pgfpathlineto{\pgfqpoint{1.369732in}{1.355472in}}%
\pgfpathlineto{\pgfqpoint{1.387285in}{1.355044in}}%
\pgfpathlineto{\pgfqpoint{1.404806in}{1.354341in}}%
\pgfpathlineto{\pgfqpoint{1.422281in}{1.353364in}}%
\pgfpathlineto{\pgfqpoint{1.421870in}{1.347737in}}%
\pgfpathlineto{\pgfqpoint{1.421461in}{1.342395in}}%
\pgfpathlineto{\pgfqpoint{1.421053in}{1.337332in}}%
\pgfpathlineto{\pgfqpoint{1.420646in}{1.332544in}}%
\pgfpathlineto{\pgfqpoint{1.403573in}{1.333502in}}%
\pgfpathlineto{\pgfqpoint{1.386455in}{1.334191in}}%
\pgfpathlineto{\pgfqpoint{1.369306in}{1.334610in}}%
\pgfpathlineto{\pgfqpoint{1.352141in}{1.334760in}}%
\pgfpathclose%
\pgfusepath{fill}%
\end{pgfscope}%
\begin{pgfscope}%
\pgfpathrectangle{\pgfqpoint{0.329460in}{0.284240in}}{\pgfqpoint{1.989680in}{1.989680in}}%
\pgfusepath{clip}%
\pgfsetbuttcap%
\pgfsetroundjoin%
\definecolor{currentfill}{rgb}{0.276194,0.190074,0.493001}%
\pgfsetfillcolor{currentfill}%
\pgfsetlinewidth{0.000000pt}%
\definecolor{currentstroke}{rgb}{0.000000,0.000000,0.000000}%
\pgfsetstrokecolor{currentstroke}%
\pgfsetdash{}{0pt}%
\pgfpathmoveto{\pgfqpoint{1.278799in}{1.409344in}}%
\pgfpathlineto{\pgfqpoint{1.278390in}{1.417758in}}%
\pgfpathlineto{\pgfqpoint{1.277978in}{1.426508in}}%
\pgfpathlineto{\pgfqpoint{1.277566in}{1.435598in}}%
\pgfpathlineto{\pgfqpoint{1.277151in}{1.445036in}}%
\pgfpathlineto{\pgfqpoint{1.295867in}{1.446034in}}%
\pgfpathlineto{\pgfqpoint{1.314630in}{1.446743in}}%
\pgfpathlineto{\pgfqpoint{1.333424in}{1.447162in}}%
\pgfpathlineto{\pgfqpoint{1.352233in}{1.447291in}}%
\pgfpathlineto{\pgfqpoint{1.352227in}{1.437844in}}%
\pgfpathlineto{\pgfqpoint{1.352221in}{1.428744in}}%
\pgfpathlineto{\pgfqpoint{1.352215in}{1.419986in}}%
\pgfpathlineto{\pgfqpoint{1.352209in}{1.411562in}}%
\pgfpathlineto{\pgfqpoint{1.333819in}{1.411435in}}%
\pgfpathlineto{\pgfqpoint{1.315443in}{1.411023in}}%
\pgfpathlineto{\pgfqpoint{1.297098in}{1.410326in}}%
\pgfpathlineto{\pgfqpoint{1.278799in}{1.409344in}}%
\pgfpathclose%
\pgfusepath{fill}%
\end{pgfscope}%
\begin{pgfscope}%
\pgfpathrectangle{\pgfqpoint{0.329460in}{0.284240in}}{\pgfqpoint{1.989680in}{1.989680in}}%
\pgfusepath{clip}%
\pgfsetbuttcap%
\pgfsetroundjoin%
\definecolor{currentfill}{rgb}{0.276194,0.190074,0.493001}%
\pgfsetfillcolor{currentfill}%
\pgfsetlinewidth{0.000000pt}%
\definecolor{currentstroke}{rgb}{0.000000,0.000000,0.000000}%
\pgfsetstrokecolor{currentstroke}%
\pgfsetdash{}{0pt}%
\pgfpathmoveto{\pgfqpoint{1.352209in}{1.411562in}}%
\pgfpathlineto{\pgfqpoint{1.352215in}{1.419986in}}%
\pgfpathlineto{\pgfqpoint{1.352221in}{1.428744in}}%
\pgfpathlineto{\pgfqpoint{1.352227in}{1.437844in}}%
\pgfpathlineto{\pgfqpoint{1.352233in}{1.447291in}}%
\pgfpathlineto{\pgfqpoint{1.371041in}{1.447130in}}%
\pgfpathlineto{\pgfqpoint{1.389832in}{1.446679in}}%
\pgfpathlineto{\pgfqpoint{1.408590in}{1.445938in}}%
\pgfpathlineto{\pgfqpoint{1.427300in}{1.444907in}}%
\pgfpathlineto{\pgfqpoint{1.426874in}{1.435470in}}%
\pgfpathlineto{\pgfqpoint{1.426449in}{1.426380in}}%
\pgfpathlineto{\pgfqpoint{1.426027in}{1.417631in}}%
\pgfpathlineto{\pgfqpoint{1.425605in}{1.409218in}}%
\pgfpathlineto{\pgfqpoint{1.407313in}{1.410231in}}%
\pgfpathlineto{\pgfqpoint{1.388972in}{1.410960in}}%
\pgfpathlineto{\pgfqpoint{1.370599in}{1.411404in}}%
\pgfpathlineto{\pgfqpoint{1.352209in}{1.411562in}}%
\pgfpathclose%
\pgfusepath{fill}%
\end{pgfscope}%
\begin{pgfscope}%
\pgfpathrectangle{\pgfqpoint{0.329460in}{0.284240in}}{\pgfqpoint{1.989680in}{1.989680in}}%
\pgfusepath{clip}%
\pgfsetbuttcap%
\pgfsetroundjoin%
\definecolor{currentfill}{rgb}{0.272594,0.025563,0.353093}%
\pgfsetfillcolor{currentfill}%
\pgfsetlinewidth{0.000000pt}%
\definecolor{currentstroke}{rgb}{0.000000,0.000000,0.000000}%
\pgfsetstrokecolor{currentstroke}%
\pgfsetdash{}{0pt}%
\pgfpathmoveto{\pgfqpoint{1.285200in}{1.316169in}}%
\pgfpathlineto{\pgfqpoint{1.284807in}{1.319903in}}%
\pgfpathlineto{\pgfqpoint{1.284414in}{1.323894in}}%
\pgfpathlineto{\pgfqpoint{1.284019in}{1.328146in}}%
\pgfpathlineto{\pgfqpoint{1.283624in}{1.332664in}}%
\pgfpathlineto{\pgfqpoint{1.300702in}{1.333592in}}%
\pgfpathlineto{\pgfqpoint{1.317825in}{1.334251in}}%
\pgfpathlineto{\pgfqpoint{1.334976in}{1.334640in}}%
\pgfpathlineto{\pgfqpoint{1.352141in}{1.334760in}}%
\pgfpathlineto{\pgfqpoint{1.352136in}{1.330232in}}%
\pgfpathlineto{\pgfqpoint{1.352130in}{1.325969in}}%
\pgfpathlineto{\pgfqpoint{1.352125in}{1.321967in}}%
\pgfpathlineto{\pgfqpoint{1.352119in}{1.318221in}}%
\pgfpathlineto{\pgfqpoint{1.335354in}{1.318104in}}%
\pgfpathlineto{\pgfqpoint{1.318603in}{1.317722in}}%
\pgfpathlineto{\pgfqpoint{1.301880in}{1.317077in}}%
\pgfpathlineto{\pgfqpoint{1.285200in}{1.316169in}}%
\pgfpathclose%
\pgfusepath{fill}%
\end{pgfscope}%
\begin{pgfscope}%
\pgfpathrectangle{\pgfqpoint{0.329460in}{0.284240in}}{\pgfqpoint{1.989680in}{1.989680in}}%
\pgfusepath{clip}%
\pgfsetbuttcap%
\pgfsetroundjoin%
\definecolor{currentfill}{rgb}{0.272594,0.025563,0.353093}%
\pgfsetfillcolor{currentfill}%
\pgfsetlinewidth{0.000000pt}%
\definecolor{currentstroke}{rgb}{0.000000,0.000000,0.000000}%
\pgfsetstrokecolor{currentstroke}%
\pgfsetdash{}{0pt}%
\pgfpathmoveto{\pgfqpoint{1.352119in}{1.318221in}}%
\pgfpathlineto{\pgfqpoint{1.352125in}{1.321967in}}%
\pgfpathlineto{\pgfqpoint{1.352130in}{1.325969in}}%
\pgfpathlineto{\pgfqpoint{1.352136in}{1.330232in}}%
\pgfpathlineto{\pgfqpoint{1.352141in}{1.334760in}}%
\pgfpathlineto{\pgfqpoint{1.369306in}{1.334610in}}%
\pgfpathlineto{\pgfqpoint{1.386455in}{1.334191in}}%
\pgfpathlineto{\pgfqpoint{1.403573in}{1.333502in}}%
\pgfpathlineto{\pgfqpoint{1.420646in}{1.332544in}}%
\pgfpathlineto{\pgfqpoint{1.420239in}{1.328027in}}%
\pgfpathlineto{\pgfqpoint{1.419834in}{1.323775in}}%
\pgfpathlineto{\pgfqpoint{1.419429in}{1.319785in}}%
\pgfpathlineto{\pgfqpoint{1.419025in}{1.316051in}}%
\pgfpathlineto{\pgfqpoint{1.402351in}{1.316989in}}%
\pgfpathlineto{\pgfqpoint{1.385632in}{1.317664in}}%
\pgfpathlineto{\pgfqpoint{1.368883in}{1.318075in}}%
\pgfpathlineto{\pgfqpoint{1.352119in}{1.318221in}}%
\pgfpathclose%
\pgfusepath{fill}%
\end{pgfscope}%
\begin{pgfscope}%
\pgfpathrectangle{\pgfqpoint{0.329460in}{0.284240in}}{\pgfqpoint{1.989680in}{1.989680in}}%
\pgfusepath{clip}%
\pgfsetbuttcap%
\pgfsetroundjoin%
\definecolor{currentfill}{rgb}{0.282327,0.094955,0.417331}%
\pgfsetfillcolor{currentfill}%
\pgfsetlinewidth{0.000000pt}%
\definecolor{currentstroke}{rgb}{0.000000,0.000000,0.000000}%
\pgfsetstrokecolor{currentstroke}%
\pgfsetdash{}{0pt}%
\pgfpathmoveto{\pgfqpoint{1.212858in}{1.346974in}}%
\pgfpathlineto{\pgfqpoint{1.212057in}{1.352860in}}%
\pgfpathlineto{\pgfqpoint{1.211253in}{1.359040in}}%
\pgfpathlineto{\pgfqpoint{1.210447in}{1.365520in}}%
\pgfpathlineto{\pgfqpoint{1.209639in}{1.372304in}}%
\pgfpathlineto{\pgfqpoint{1.227191in}{1.374378in}}%
\pgfpathlineto{\pgfqpoint{1.244849in}{1.376176in}}%
\pgfpathlineto{\pgfqpoint{1.262599in}{1.377697in}}%
\pgfpathlineto{\pgfqpoint{1.280426in}{1.378940in}}%
\pgfpathlineto{\pgfqpoint{1.280829in}{1.372125in}}%
\pgfpathlineto{\pgfqpoint{1.281232in}{1.365615in}}%
\pgfpathlineto{\pgfqpoint{1.281633in}{1.359403in}}%
\pgfpathlineto{\pgfqpoint{1.282033in}{1.353486in}}%
\pgfpathlineto{\pgfqpoint{1.264612in}{1.352266in}}%
\pgfpathlineto{\pgfqpoint{1.247266in}{1.350774in}}%
\pgfpathlineto{\pgfqpoint{1.230009in}{1.349009in}}%
\pgfpathlineto{\pgfqpoint{1.212858in}{1.346974in}}%
\pgfpathclose%
\pgfusepath{fill}%
\end{pgfscope}%
\begin{pgfscope}%
\pgfpathrectangle{\pgfqpoint{0.329460in}{0.284240in}}{\pgfqpoint{1.989680in}{1.989680in}}%
\pgfusepath{clip}%
\pgfsetbuttcap%
\pgfsetroundjoin%
\definecolor{currentfill}{rgb}{0.282327,0.094955,0.417331}%
\pgfsetfillcolor{currentfill}%
\pgfsetlinewidth{0.000000pt}%
\definecolor{currentstroke}{rgb}{0.000000,0.000000,0.000000}%
\pgfsetstrokecolor{currentstroke}%
\pgfsetdash{}{0pt}%
\pgfpathmoveto{\pgfqpoint{1.422281in}{1.353364in}}%
\pgfpathlineto{\pgfqpoint{1.422692in}{1.359281in}}%
\pgfpathlineto{\pgfqpoint{1.423105in}{1.365491in}}%
\pgfpathlineto{\pgfqpoint{1.423518in}{1.372001in}}%
\pgfpathlineto{\pgfqpoint{1.423933in}{1.378816in}}%
\pgfpathlineto{\pgfqpoint{1.441752in}{1.377541in}}%
\pgfpathlineto{\pgfqpoint{1.459493in}{1.375989in}}%
\pgfpathlineto{\pgfqpoint{1.477140in}{1.374161in}}%
\pgfpathlineto{\pgfqpoint{1.494679in}{1.372057in}}%
\pgfpathlineto{\pgfqpoint{1.493860in}{1.365274in}}%
\pgfpathlineto{\pgfqpoint{1.493043in}{1.358795in}}%
\pgfpathlineto{\pgfqpoint{1.492228in}{1.352616in}}%
\pgfpathlineto{\pgfqpoint{1.491416in}{1.346732in}}%
\pgfpathlineto{\pgfqpoint{1.474277in}{1.348796in}}%
\pgfpathlineto{\pgfqpoint{1.457031in}{1.350591in}}%
\pgfpathlineto{\pgfqpoint{1.439694in}{1.352114in}}%
\pgfpathlineto{\pgfqpoint{1.422281in}{1.353364in}}%
\pgfpathclose%
\pgfusepath{fill}%
\end{pgfscope}%
\begin{pgfscope}%
\pgfpathrectangle{\pgfqpoint{0.329460in}{0.284240in}}{\pgfqpoint{1.989680in}{1.989680in}}%
\pgfusepath{clip}%
\pgfsetbuttcap%
\pgfsetroundjoin%
\definecolor{currentfill}{rgb}{0.277941,0.056324,0.381191}%
\pgfsetfillcolor{currentfill}%
\pgfsetlinewidth{0.000000pt}%
\definecolor{currentstroke}{rgb}{0.000000,0.000000,0.000000}%
\pgfsetstrokecolor{currentstroke}%
\pgfsetdash{}{0pt}%
\pgfpathmoveto{\pgfqpoint{1.216043in}{1.326281in}}%
\pgfpathlineto{\pgfqpoint{1.215250in}{1.331036in}}%
\pgfpathlineto{\pgfqpoint{1.214454in}{1.336067in}}%
\pgfpathlineto{\pgfqpoint{1.213657in}{1.341379in}}%
\pgfpathlineto{\pgfqpoint{1.212858in}{1.346974in}}%
\pgfpathlineto{\pgfqpoint{1.230009in}{1.349009in}}%
\pgfpathlineto{\pgfqpoint{1.247266in}{1.350774in}}%
\pgfpathlineto{\pgfqpoint{1.264612in}{1.352266in}}%
\pgfpathlineto{\pgfqpoint{1.282033in}{1.353486in}}%
\pgfpathlineto{\pgfqpoint{1.282432in}{1.347859in}}%
\pgfpathlineto{\pgfqpoint{1.282830in}{1.342516in}}%
\pgfpathlineto{\pgfqpoint{1.283228in}{1.337452in}}%
\pgfpathlineto{\pgfqpoint{1.283624in}{1.332664in}}%
\pgfpathlineto{\pgfqpoint{1.266604in}{1.331468in}}%
\pgfpathlineto{\pgfqpoint{1.249657in}{1.330005in}}%
\pgfpathlineto{\pgfqpoint{1.232799in}{1.328275in}}%
\pgfpathlineto{\pgfqpoint{1.216043in}{1.326281in}}%
\pgfpathclose%
\pgfusepath{fill}%
\end{pgfscope}%
\begin{pgfscope}%
\pgfpathrectangle{\pgfqpoint{0.329460in}{0.284240in}}{\pgfqpoint{1.989680in}{1.989680in}}%
\pgfusepath{clip}%
\pgfsetbuttcap%
\pgfsetroundjoin%
\definecolor{currentfill}{rgb}{0.282884,0.135920,0.453427}%
\pgfsetfillcolor{currentfill}%
\pgfsetlinewidth{0.000000pt}%
\definecolor{currentstroke}{rgb}{0.000000,0.000000,0.000000}%
\pgfsetstrokecolor{currentstroke}%
\pgfsetdash{}{0pt}%
\pgfpathmoveto{\pgfqpoint{1.209639in}{1.372304in}}%
\pgfpathlineto{\pgfqpoint{1.208829in}{1.379398in}}%
\pgfpathlineto{\pgfqpoint{1.208016in}{1.386807in}}%
\pgfpathlineto{\pgfqpoint{1.207200in}{1.394535in}}%
\pgfpathlineto{\pgfqpoint{1.206382in}{1.402590in}}%
\pgfpathlineto{\pgfqpoint{1.224339in}{1.404700in}}%
\pgfpathlineto{\pgfqpoint{1.242404in}{1.406530in}}%
\pgfpathlineto{\pgfqpoint{1.260563in}{1.408079in}}%
\pgfpathlineto{\pgfqpoint{1.278799in}{1.409344in}}%
\pgfpathlineto{\pgfqpoint{1.279208in}{1.401261in}}%
\pgfpathlineto{\pgfqpoint{1.279615in}{1.393503in}}%
\pgfpathlineto{\pgfqpoint{1.280021in}{1.386064in}}%
\pgfpathlineto{\pgfqpoint{1.280426in}{1.378940in}}%
\pgfpathlineto{\pgfqpoint{1.262599in}{1.377697in}}%
\pgfpathlineto{\pgfqpoint{1.244849in}{1.376176in}}%
\pgfpathlineto{\pgfqpoint{1.227191in}{1.374378in}}%
\pgfpathlineto{\pgfqpoint{1.209639in}{1.372304in}}%
\pgfpathclose%
\pgfusepath{fill}%
\end{pgfscope}%
\begin{pgfscope}%
\pgfpathrectangle{\pgfqpoint{0.329460in}{0.284240in}}{\pgfqpoint{1.989680in}{1.989680in}}%
\pgfusepath{clip}%
\pgfsetbuttcap%
\pgfsetroundjoin%
\definecolor{currentfill}{rgb}{0.277941,0.056324,0.381191}%
\pgfsetfillcolor{currentfill}%
\pgfsetlinewidth{0.000000pt}%
\definecolor{currentstroke}{rgb}{0.000000,0.000000,0.000000}%
\pgfsetstrokecolor{currentstroke}%
\pgfsetdash{}{0pt}%
\pgfpathmoveto{\pgfqpoint{1.420646in}{1.332544in}}%
\pgfpathlineto{\pgfqpoint{1.421053in}{1.337332in}}%
\pgfpathlineto{\pgfqpoint{1.421461in}{1.342395in}}%
\pgfpathlineto{\pgfqpoint{1.421870in}{1.347737in}}%
\pgfpathlineto{\pgfqpoint{1.422281in}{1.353364in}}%
\pgfpathlineto{\pgfqpoint{1.439694in}{1.352114in}}%
\pgfpathlineto{\pgfqpoint{1.457031in}{1.350591in}}%
\pgfpathlineto{\pgfqpoint{1.474277in}{1.348796in}}%
\pgfpathlineto{\pgfqpoint{1.491416in}{1.346732in}}%
\pgfpathlineto{\pgfqpoint{1.490606in}{1.341137in}}%
\pgfpathlineto{\pgfqpoint{1.489797in}{1.335827in}}%
\pgfpathlineto{\pgfqpoint{1.488991in}{1.330797in}}%
\pgfpathlineto{\pgfqpoint{1.488187in}{1.326043in}}%
\pgfpathlineto{\pgfqpoint{1.471444in}{1.328067in}}%
\pgfpathlineto{\pgfqpoint{1.454596in}{1.329826in}}%
\pgfpathlineto{\pgfqpoint{1.437658in}{1.331319in}}%
\pgfpathlineto{\pgfqpoint{1.420646in}{1.332544in}}%
\pgfpathclose%
\pgfusepath{fill}%
\end{pgfscope}%
\begin{pgfscope}%
\pgfpathrectangle{\pgfqpoint{0.329460in}{0.284240in}}{\pgfqpoint{1.989680in}{1.989680in}}%
\pgfusepath{clip}%
\pgfsetbuttcap%
\pgfsetroundjoin%
\definecolor{currentfill}{rgb}{0.282884,0.135920,0.453427}%
\pgfsetfillcolor{currentfill}%
\pgfsetlinewidth{0.000000pt}%
\definecolor{currentstroke}{rgb}{0.000000,0.000000,0.000000}%
\pgfsetstrokecolor{currentstroke}%
\pgfsetdash{}{0pt}%
\pgfpathmoveto{\pgfqpoint{1.423933in}{1.378816in}}%
\pgfpathlineto{\pgfqpoint{1.424349in}{1.385939in}}%
\pgfpathlineto{\pgfqpoint{1.424767in}{1.393377in}}%
\pgfpathlineto{\pgfqpoint{1.425185in}{1.401135in}}%
\pgfpathlineto{\pgfqpoint{1.425605in}{1.409218in}}%
\pgfpathlineto{\pgfqpoint{1.443834in}{1.407921in}}%
\pgfpathlineto{\pgfqpoint{1.461983in}{1.406341in}}%
\pgfpathlineto{\pgfqpoint{1.480037in}{1.404480in}}%
\pgfpathlineto{\pgfqpoint{1.497981in}{1.402338in}}%
\pgfpathlineto{\pgfqpoint{1.497151in}{1.394285in}}%
\pgfpathlineto{\pgfqpoint{1.496324in}{1.386557in}}%
\pgfpathlineto{\pgfqpoint{1.495500in}{1.379150in}}%
\pgfpathlineto{\pgfqpoint{1.494679in}{1.372057in}}%
\pgfpathlineto{\pgfqpoint{1.477140in}{1.374161in}}%
\pgfpathlineto{\pgfqpoint{1.459493in}{1.375989in}}%
\pgfpathlineto{\pgfqpoint{1.441752in}{1.377541in}}%
\pgfpathlineto{\pgfqpoint{1.423933in}{1.378816in}}%
\pgfpathclose%
\pgfusepath{fill}%
\end{pgfscope}%
\begin{pgfscope}%
\pgfpathrectangle{\pgfqpoint{0.329460in}{0.284240in}}{\pgfqpoint{1.989680in}{1.989680in}}%
\pgfusepath{clip}%
\pgfsetbuttcap%
\pgfsetroundjoin%
\definecolor{currentfill}{rgb}{0.268510,0.009605,0.335427}%
\pgfsetfillcolor{currentfill}%
\pgfsetlinewidth{0.000000pt}%
\definecolor{currentstroke}{rgb}{0.000000,0.000000,0.000000}%
\pgfsetstrokecolor{currentstroke}%
\pgfsetdash{}{0pt}%
\pgfpathmoveto{\pgfqpoint{1.286764in}{1.303709in}}%
\pgfpathlineto{\pgfqpoint{1.286374in}{1.306461in}}%
\pgfpathlineto{\pgfqpoint{1.285983in}{1.309452in}}%
\pgfpathlineto{\pgfqpoint{1.285592in}{1.312686in}}%
\pgfpathlineto{\pgfqpoint{1.285200in}{1.316169in}}%
\pgfpathlineto{\pgfqpoint{1.301880in}{1.317077in}}%
\pgfpathlineto{\pgfqpoint{1.318603in}{1.317722in}}%
\pgfpathlineto{\pgfqpoint{1.335354in}{1.318104in}}%
\pgfpathlineto{\pgfqpoint{1.352119in}{1.318221in}}%
\pgfpathlineto{\pgfqpoint{1.352114in}{1.314728in}}%
\pgfpathlineto{\pgfqpoint{1.352108in}{1.311483in}}%
\pgfpathlineto{\pgfqpoint{1.352103in}{1.308480in}}%
\pgfpathlineto{\pgfqpoint{1.352097in}{1.305717in}}%
\pgfpathlineto{\pgfqpoint{1.335729in}{1.305602in}}%
\pgfpathlineto{\pgfqpoint{1.319375in}{1.305229in}}%
\pgfpathlineto{\pgfqpoint{1.303048in}{1.304598in}}%
\pgfpathlineto{\pgfqpoint{1.286764in}{1.303709in}}%
\pgfpathclose%
\pgfusepath{fill}%
\end{pgfscope}%
\begin{pgfscope}%
\pgfpathrectangle{\pgfqpoint{0.329460in}{0.284240in}}{\pgfqpoint{1.989680in}{1.989680in}}%
\pgfusepath{clip}%
\pgfsetbuttcap%
\pgfsetroundjoin%
\definecolor{currentfill}{rgb}{0.268510,0.009605,0.335427}%
\pgfsetfillcolor{currentfill}%
\pgfsetlinewidth{0.000000pt}%
\definecolor{currentstroke}{rgb}{0.000000,0.000000,0.000000}%
\pgfsetstrokecolor{currentstroke}%
\pgfsetdash{}{0pt}%
\pgfpathmoveto{\pgfqpoint{1.352097in}{1.305717in}}%
\pgfpathlineto{\pgfqpoint{1.352103in}{1.308480in}}%
\pgfpathlineto{\pgfqpoint{1.352108in}{1.311483in}}%
\pgfpathlineto{\pgfqpoint{1.352114in}{1.314728in}}%
\pgfpathlineto{\pgfqpoint{1.352119in}{1.318221in}}%
\pgfpathlineto{\pgfqpoint{1.368883in}{1.318075in}}%
\pgfpathlineto{\pgfqpoint{1.385632in}{1.317664in}}%
\pgfpathlineto{\pgfqpoint{1.402351in}{1.316989in}}%
\pgfpathlineto{\pgfqpoint{1.419025in}{1.316051in}}%
\pgfpathlineto{\pgfqpoint{1.418622in}{1.312570in}}%
\pgfpathlineto{\pgfqpoint{1.418220in}{1.309336in}}%
\pgfpathlineto{\pgfqpoint{1.417819in}{1.306346in}}%
\pgfpathlineto{\pgfqpoint{1.417418in}{1.303594in}}%
\pgfpathlineto{\pgfqpoint{1.401139in}{1.304512in}}%
\pgfpathlineto{\pgfqpoint{1.384816in}{1.305172in}}%
\pgfpathlineto{\pgfqpoint{1.368464in}{1.305574in}}%
\pgfpathlineto{\pgfqpoint{1.352097in}{1.305717in}}%
\pgfpathclose%
\pgfusepath{fill}%
\end{pgfscope}%
\begin{pgfscope}%
\pgfpathrectangle{\pgfqpoint{0.329460in}{0.284240in}}{\pgfqpoint{1.989680in}{1.989680in}}%
\pgfusepath{clip}%
\pgfsetbuttcap%
\pgfsetroundjoin%
\definecolor{currentfill}{rgb}{0.260571,0.246922,0.522828}%
\pgfsetfillcolor{currentfill}%
\pgfsetlinewidth{0.000000pt}%
\definecolor{currentstroke}{rgb}{0.000000,0.000000,0.000000}%
\pgfsetstrokecolor{currentstroke}%
\pgfsetdash{}{0pt}%
\pgfpathmoveto{\pgfqpoint{1.277151in}{1.445036in}}%
\pgfpathlineto{\pgfqpoint{1.276736in}{1.454826in}}%
\pgfpathlineto{\pgfqpoint{1.276318in}{1.464975in}}%
\pgfpathlineto{\pgfqpoint{1.275900in}{1.475487in}}%
\pgfpathlineto{\pgfqpoint{1.275479in}{1.486370in}}%
\pgfpathlineto{\pgfqpoint{1.294618in}{1.487383in}}%
\pgfpathlineto{\pgfqpoint{1.313804in}{1.488103in}}%
\pgfpathlineto{\pgfqpoint{1.333022in}{1.488529in}}%
\pgfpathlineto{\pgfqpoint{1.352256in}{1.488660in}}%
\pgfpathlineto{\pgfqpoint{1.352250in}{1.477769in}}%
\pgfpathlineto{\pgfqpoint{1.352244in}{1.467248in}}%
\pgfpathlineto{\pgfqpoint{1.352239in}{1.457090in}}%
\pgfpathlineto{\pgfqpoint{1.352233in}{1.447291in}}%
\pgfpathlineto{\pgfqpoint{1.333424in}{1.447162in}}%
\pgfpathlineto{\pgfqpoint{1.314630in}{1.446743in}}%
\pgfpathlineto{\pgfqpoint{1.295867in}{1.446034in}}%
\pgfpathlineto{\pgfqpoint{1.277151in}{1.445036in}}%
\pgfpathclose%
\pgfusepath{fill}%
\end{pgfscope}%
\begin{pgfscope}%
\pgfpathrectangle{\pgfqpoint{0.329460in}{0.284240in}}{\pgfqpoint{1.989680in}{1.989680in}}%
\pgfusepath{clip}%
\pgfsetbuttcap%
\pgfsetroundjoin%
\definecolor{currentfill}{rgb}{0.260571,0.246922,0.522828}%
\pgfsetfillcolor{currentfill}%
\pgfsetlinewidth{0.000000pt}%
\definecolor{currentstroke}{rgb}{0.000000,0.000000,0.000000}%
\pgfsetstrokecolor{currentstroke}%
\pgfsetdash{}{0pt}%
\pgfpathmoveto{\pgfqpoint{1.352233in}{1.447291in}}%
\pgfpathlineto{\pgfqpoint{1.352239in}{1.457090in}}%
\pgfpathlineto{\pgfqpoint{1.352244in}{1.467248in}}%
\pgfpathlineto{\pgfqpoint{1.352250in}{1.477769in}}%
\pgfpathlineto{\pgfqpoint{1.352256in}{1.488660in}}%
\pgfpathlineto{\pgfqpoint{1.371489in}{1.488496in}}%
\pgfpathlineto{\pgfqpoint{1.390705in}{1.488038in}}%
\pgfpathlineto{\pgfqpoint{1.409887in}{1.487285in}}%
\pgfpathlineto{\pgfqpoint{1.429019in}{1.486239in}}%
\pgfpathlineto{\pgfqpoint{1.428587in}{1.475357in}}%
\pgfpathlineto{\pgfqpoint{1.428156in}{1.464845in}}%
\pgfpathlineto{\pgfqpoint{1.427727in}{1.454697in}}%
\pgfpathlineto{\pgfqpoint{1.427300in}{1.444907in}}%
\pgfpathlineto{\pgfqpoint{1.408590in}{1.445938in}}%
\pgfpathlineto{\pgfqpoint{1.389832in}{1.446679in}}%
\pgfpathlineto{\pgfqpoint{1.371041in}{1.447130in}}%
\pgfpathlineto{\pgfqpoint{1.352233in}{1.447291in}}%
\pgfpathclose%
\pgfusepath{fill}%
\end{pgfscope}%
\begin{pgfscope}%
\pgfpathrectangle{\pgfqpoint{0.329460in}{0.284240in}}{\pgfqpoint{1.989680in}{1.989680in}}%
\pgfusepath{clip}%
\pgfsetbuttcap%
\pgfsetroundjoin%
\definecolor{currentfill}{rgb}{0.272594,0.025563,0.353093}%
\pgfsetfillcolor{currentfill}%
\pgfsetlinewidth{0.000000pt}%
\definecolor{currentstroke}{rgb}{0.000000,0.000000,0.000000}%
\pgfsetstrokecolor{currentstroke}%
\pgfsetdash{}{0pt}%
\pgfpathmoveto{\pgfqpoint{1.219199in}{1.309918in}}%
\pgfpathlineto{\pgfqpoint{1.218412in}{1.313618in}}%
\pgfpathlineto{\pgfqpoint{1.217624in}{1.317576in}}%
\pgfpathlineto{\pgfqpoint{1.216834in}{1.321795in}}%
\pgfpathlineto{\pgfqpoint{1.216043in}{1.326281in}}%
\pgfpathlineto{\pgfqpoint{1.232799in}{1.328275in}}%
\pgfpathlineto{\pgfqpoint{1.249657in}{1.330005in}}%
\pgfpathlineto{\pgfqpoint{1.266604in}{1.331468in}}%
\pgfpathlineto{\pgfqpoint{1.283624in}{1.332664in}}%
\pgfpathlineto{\pgfqpoint{1.284019in}{1.328146in}}%
\pgfpathlineto{\pgfqpoint{1.284414in}{1.323894in}}%
\pgfpathlineto{\pgfqpoint{1.284807in}{1.319903in}}%
\pgfpathlineto{\pgfqpoint{1.285200in}{1.316169in}}%
\pgfpathlineto{\pgfqpoint{1.268577in}{1.314997in}}%
\pgfpathlineto{\pgfqpoint{1.252027in}{1.313564in}}%
\pgfpathlineto{\pgfqpoint{1.235562in}{1.311871in}}%
\pgfpathlineto{\pgfqpoint{1.219199in}{1.309918in}}%
\pgfpathclose%
\pgfusepath{fill}%
\end{pgfscope}%
\begin{pgfscope}%
\pgfpathrectangle{\pgfqpoint{0.329460in}{0.284240in}}{\pgfqpoint{1.989680in}{1.989680in}}%
\pgfusepath{clip}%
\pgfsetbuttcap%
\pgfsetroundjoin%
\definecolor{currentfill}{rgb}{0.272594,0.025563,0.353093}%
\pgfsetfillcolor{currentfill}%
\pgfsetlinewidth{0.000000pt}%
\definecolor{currentstroke}{rgb}{0.000000,0.000000,0.000000}%
\pgfsetstrokecolor{currentstroke}%
\pgfsetdash{}{0pt}%
\pgfpathmoveto{\pgfqpoint{1.419025in}{1.316051in}}%
\pgfpathlineto{\pgfqpoint{1.419429in}{1.319785in}}%
\pgfpathlineto{\pgfqpoint{1.419834in}{1.323775in}}%
\pgfpathlineto{\pgfqpoint{1.420239in}{1.328027in}}%
\pgfpathlineto{\pgfqpoint{1.420646in}{1.332544in}}%
\pgfpathlineto{\pgfqpoint{1.437658in}{1.331319in}}%
\pgfpathlineto{\pgfqpoint{1.454596in}{1.329826in}}%
\pgfpathlineto{\pgfqpoint{1.471444in}{1.328067in}}%
\pgfpathlineto{\pgfqpoint{1.488187in}{1.326043in}}%
\pgfpathlineto{\pgfqpoint{1.487385in}{1.321559in}}%
\pgfpathlineto{\pgfqpoint{1.486584in}{1.317341in}}%
\pgfpathlineto{\pgfqpoint{1.485785in}{1.313384in}}%
\pgfpathlineto{\pgfqpoint{1.484988in}{1.309685in}}%
\pgfpathlineto{\pgfqpoint{1.468636in}{1.311666in}}%
\pgfpathlineto{\pgfqpoint{1.452182in}{1.313389in}}%
\pgfpathlineto{\pgfqpoint{1.435641in}{1.314851in}}%
\pgfpathlineto{\pgfqpoint{1.419025in}{1.316051in}}%
\pgfpathclose%
\pgfusepath{fill}%
\end{pgfscope}%
\begin{pgfscope}%
\pgfpathrectangle{\pgfqpoint{0.329460in}{0.284240in}}{\pgfqpoint{1.989680in}{1.989680in}}%
\pgfusepath{clip}%
\pgfsetbuttcap%
\pgfsetroundjoin%
\definecolor{currentfill}{rgb}{0.276194,0.190074,0.493001}%
\pgfsetfillcolor{currentfill}%
\pgfsetlinewidth{0.000000pt}%
\definecolor{currentstroke}{rgb}{0.000000,0.000000,0.000000}%
\pgfsetstrokecolor{currentstroke}%
\pgfsetdash{}{0pt}%
\pgfpathmoveto{\pgfqpoint{1.206382in}{1.402590in}}%
\pgfpathlineto{\pgfqpoint{1.205561in}{1.410974in}}%
\pgfpathlineto{\pgfqpoint{1.204738in}{1.419695in}}%
\pgfpathlineto{\pgfqpoint{1.203911in}{1.428758in}}%
\pgfpathlineto{\pgfqpoint{1.203082in}{1.438168in}}%
\pgfpathlineto{\pgfqpoint{1.221448in}{1.440314in}}%
\pgfpathlineto{\pgfqpoint{1.239926in}{1.442175in}}%
\pgfpathlineto{\pgfqpoint{1.258499in}{1.443749in}}%
\pgfpathlineto{\pgfqpoint{1.277151in}{1.445036in}}%
\pgfpathlineto{\pgfqpoint{1.277566in}{1.435598in}}%
\pgfpathlineto{\pgfqpoint{1.277978in}{1.426508in}}%
\pgfpathlineto{\pgfqpoint{1.278390in}{1.417758in}}%
\pgfpathlineto{\pgfqpoint{1.278799in}{1.409344in}}%
\pgfpathlineto{\pgfqpoint{1.260563in}{1.408079in}}%
\pgfpathlineto{\pgfqpoint{1.242404in}{1.406530in}}%
\pgfpathlineto{\pgfqpoint{1.224339in}{1.404700in}}%
\pgfpathlineto{\pgfqpoint{1.206382in}{1.402590in}}%
\pgfpathclose%
\pgfusepath{fill}%
\end{pgfscope}%
\begin{pgfscope}%
\pgfpathrectangle{\pgfqpoint{0.329460in}{0.284240in}}{\pgfqpoint{1.989680in}{1.989680in}}%
\pgfusepath{clip}%
\pgfsetbuttcap%
\pgfsetroundjoin%
\definecolor{currentfill}{rgb}{0.276194,0.190074,0.493001}%
\pgfsetfillcolor{currentfill}%
\pgfsetlinewidth{0.000000pt}%
\definecolor{currentstroke}{rgb}{0.000000,0.000000,0.000000}%
\pgfsetstrokecolor{currentstroke}%
\pgfsetdash{}{0pt}%
\pgfpathmoveto{\pgfqpoint{1.425605in}{1.409218in}}%
\pgfpathlineto{\pgfqpoint{1.426027in}{1.417631in}}%
\pgfpathlineto{\pgfqpoint{1.426449in}{1.426380in}}%
\pgfpathlineto{\pgfqpoint{1.426874in}{1.435470in}}%
\pgfpathlineto{\pgfqpoint{1.427300in}{1.444907in}}%
\pgfpathlineto{\pgfqpoint{1.445944in}{1.443589in}}%
\pgfpathlineto{\pgfqpoint{1.464507in}{1.441982in}}%
\pgfpathlineto{\pgfqpoint{1.482973in}{1.440089in}}%
\pgfpathlineto{\pgfqpoint{1.501327in}{1.437912in}}%
\pgfpathlineto{\pgfqpoint{1.500486in}{1.428503in}}%
\pgfpathlineto{\pgfqpoint{1.499648in}{1.419441in}}%
\pgfpathlineto{\pgfqpoint{1.498813in}{1.410721in}}%
\pgfpathlineto{\pgfqpoint{1.497981in}{1.402338in}}%
\pgfpathlineto{\pgfqpoint{1.480037in}{1.404480in}}%
\pgfpathlineto{\pgfqpoint{1.461983in}{1.406341in}}%
\pgfpathlineto{\pgfqpoint{1.443834in}{1.407921in}}%
\pgfpathlineto{\pgfqpoint{1.425605in}{1.409218in}}%
\pgfpathclose%
\pgfusepath{fill}%
\end{pgfscope}%
\begin{pgfscope}%
\pgfpathrectangle{\pgfqpoint{0.329460in}{0.284240in}}{\pgfqpoint{1.989680in}{1.989680in}}%
\pgfusepath{clip}%
\pgfsetbuttcap%
\pgfsetroundjoin%
\definecolor{currentfill}{rgb}{0.268510,0.009605,0.335427}%
\pgfsetfillcolor{currentfill}%
\pgfsetlinewidth{0.000000pt}%
\definecolor{currentstroke}{rgb}{0.000000,0.000000,0.000000}%
\pgfsetstrokecolor{currentstroke}%
\pgfsetdash{}{0pt}%
\pgfpathmoveto{\pgfqpoint{1.222329in}{1.297594in}}%
\pgfpathlineto{\pgfqpoint{1.221549in}{1.300312in}}%
\pgfpathlineto{\pgfqpoint{1.220767in}{1.303269in}}%
\pgfpathlineto{\pgfqpoint{1.219983in}{1.306469in}}%
\pgfpathlineto{\pgfqpoint{1.219199in}{1.309918in}}%
\pgfpathlineto{\pgfqpoint{1.235562in}{1.311871in}}%
\pgfpathlineto{\pgfqpoint{1.252027in}{1.313564in}}%
\pgfpathlineto{\pgfqpoint{1.268577in}{1.314997in}}%
\pgfpathlineto{\pgfqpoint{1.285200in}{1.316169in}}%
\pgfpathlineto{\pgfqpoint{1.285592in}{1.312686in}}%
\pgfpathlineto{\pgfqpoint{1.285983in}{1.309452in}}%
\pgfpathlineto{\pgfqpoint{1.286374in}{1.306461in}}%
\pgfpathlineto{\pgfqpoint{1.286764in}{1.303709in}}%
\pgfpathlineto{\pgfqpoint{1.270535in}{1.302563in}}%
\pgfpathlineto{\pgfqpoint{1.254377in}{1.301162in}}%
\pgfpathlineto{\pgfqpoint{1.238304in}{1.299505in}}%
\pgfpathlineto{\pgfqpoint{1.222329in}{1.297594in}}%
\pgfpathclose%
\pgfusepath{fill}%
\end{pgfscope}%
\begin{pgfscope}%
\pgfpathrectangle{\pgfqpoint{0.329460in}{0.284240in}}{\pgfqpoint{1.989680in}{1.989680in}}%
\pgfusepath{clip}%
\pgfsetbuttcap%
\pgfsetroundjoin%
\definecolor{currentfill}{rgb}{0.268510,0.009605,0.335427}%
\pgfsetfillcolor{currentfill}%
\pgfsetlinewidth{0.000000pt}%
\definecolor{currentstroke}{rgb}{0.000000,0.000000,0.000000}%
\pgfsetstrokecolor{currentstroke}%
\pgfsetdash{}{0pt}%
\pgfpathmoveto{\pgfqpoint{1.417418in}{1.303594in}}%
\pgfpathlineto{\pgfqpoint{1.417819in}{1.306346in}}%
\pgfpathlineto{\pgfqpoint{1.418220in}{1.309336in}}%
\pgfpathlineto{\pgfqpoint{1.418622in}{1.312570in}}%
\pgfpathlineto{\pgfqpoint{1.419025in}{1.316051in}}%
\pgfpathlineto{\pgfqpoint{1.435641in}{1.314851in}}%
\pgfpathlineto{\pgfqpoint{1.452182in}{1.313389in}}%
\pgfpathlineto{\pgfqpoint{1.468636in}{1.311666in}}%
\pgfpathlineto{\pgfqpoint{1.484988in}{1.309685in}}%
\pgfpathlineto{\pgfqpoint{1.484192in}{1.306237in}}%
\pgfpathlineto{\pgfqpoint{1.483398in}{1.303038in}}%
\pgfpathlineto{\pgfqpoint{1.482605in}{1.300083in}}%
\pgfpathlineto{\pgfqpoint{1.481814in}{1.297366in}}%
\pgfpathlineto{\pgfqpoint{1.465851in}{1.299305in}}%
\pgfpathlineto{\pgfqpoint{1.449788in}{1.300990in}}%
\pgfpathlineto{\pgfqpoint{1.433639in}{1.302420in}}%
\pgfpathlineto{\pgfqpoint{1.417418in}{1.303594in}}%
\pgfpathclose%
\pgfusepath{fill}%
\end{pgfscope}%
\begin{pgfscope}%
\pgfpathrectangle{\pgfqpoint{0.329460in}{0.284240in}}{\pgfqpoint{1.989680in}{1.989680in}}%
\pgfusepath{clip}%
\pgfsetbuttcap%
\pgfsetroundjoin%
\definecolor{currentfill}{rgb}{0.260571,0.246922,0.522828}%
\pgfsetfillcolor{currentfill}%
\pgfsetlinewidth{0.000000pt}%
\definecolor{currentstroke}{rgb}{0.000000,0.000000,0.000000}%
\pgfsetstrokecolor{currentstroke}%
\pgfsetdash{}{0pt}%
\pgfpathmoveto{\pgfqpoint{1.203082in}{1.438168in}}%
\pgfpathlineto{\pgfqpoint{1.202249in}{1.447930in}}%
\pgfpathlineto{\pgfqpoint{1.201414in}{1.458052in}}%
\pgfpathlineto{\pgfqpoint{1.200575in}{1.468537in}}%
\pgfpathlineto{\pgfqpoint{1.199732in}{1.479394in}}%
\pgfpathlineto{\pgfqpoint{1.218516in}{1.481574in}}%
\pgfpathlineto{\pgfqpoint{1.237412in}{1.483464in}}%
\pgfpathlineto{\pgfqpoint{1.256405in}{1.485063in}}%
\pgfpathlineto{\pgfqpoint{1.275479in}{1.486370in}}%
\pgfpathlineto{\pgfqpoint{1.275900in}{1.475487in}}%
\pgfpathlineto{\pgfqpoint{1.276318in}{1.464975in}}%
\pgfpathlineto{\pgfqpoint{1.276736in}{1.454826in}}%
\pgfpathlineto{\pgfqpoint{1.277151in}{1.445036in}}%
\pgfpathlineto{\pgfqpoint{1.258499in}{1.443749in}}%
\pgfpathlineto{\pgfqpoint{1.239926in}{1.442175in}}%
\pgfpathlineto{\pgfqpoint{1.221448in}{1.440314in}}%
\pgfpathlineto{\pgfqpoint{1.203082in}{1.438168in}}%
\pgfpathclose%
\pgfusepath{fill}%
\end{pgfscope}%
\begin{pgfscope}%
\pgfpathrectangle{\pgfqpoint{0.329460in}{0.284240in}}{\pgfqpoint{1.989680in}{1.989680in}}%
\pgfusepath{clip}%
\pgfsetbuttcap%
\pgfsetroundjoin%
\definecolor{currentfill}{rgb}{0.282327,0.094955,0.417331}%
\pgfsetfillcolor{currentfill}%
\pgfsetlinewidth{0.000000pt}%
\definecolor{currentstroke}{rgb}{0.000000,0.000000,0.000000}%
\pgfsetstrokecolor{currentstroke}%
\pgfsetdash{}{0pt}%
\pgfpathmoveto{\pgfqpoint{1.491416in}{1.346732in}}%
\pgfpathlineto{\pgfqpoint{1.492228in}{1.352616in}}%
\pgfpathlineto{\pgfqpoint{1.493043in}{1.358795in}}%
\pgfpathlineto{\pgfqpoint{1.493860in}{1.365274in}}%
\pgfpathlineto{\pgfqpoint{1.494679in}{1.372057in}}%
\pgfpathlineto{\pgfqpoint{1.512094in}{1.369679in}}%
\pgfpathlineto{\pgfqpoint{1.529369in}{1.367029in}}%
\pgfpathlineto{\pgfqpoint{1.546490in}{1.364109in}}%
\pgfpathlineto{\pgfqpoint{1.545374in}{1.357363in}}%
\pgfpathlineto{\pgfqpoint{1.544261in}{1.350921in}}%
\pgfpathlineto{\pgfqpoint{1.543151in}{1.344780in}}%
\pgfpathlineto{\pgfqpoint{1.542044in}{1.338933in}}%
\pgfpathlineto{\pgfqpoint{1.525314in}{1.341798in}}%
\pgfpathlineto{\pgfqpoint{1.508433in}{1.344398in}}%
\pgfpathlineto{\pgfqpoint{1.491416in}{1.346732in}}%
\pgfpathclose%
\pgfusepath{fill}%
\end{pgfscope}%
\begin{pgfscope}%
\pgfpathrectangle{\pgfqpoint{0.329460in}{0.284240in}}{\pgfqpoint{1.989680in}{1.989680in}}%
\pgfusepath{clip}%
\pgfsetbuttcap%
\pgfsetroundjoin%
\definecolor{currentfill}{rgb}{0.267004,0.004874,0.329415}%
\pgfsetfillcolor{currentfill}%
\pgfsetlinewidth{0.000000pt}%
\definecolor{currentstroke}{rgb}{0.000000,0.000000,0.000000}%
\pgfsetstrokecolor{currentstroke}%
\pgfsetdash{}{0pt}%
\pgfpathmoveto{\pgfqpoint{1.288317in}{1.295007in}}%
\pgfpathlineto{\pgfqpoint{1.287929in}{1.296845in}}%
\pgfpathlineto{\pgfqpoint{1.287541in}{1.298905in}}%
\pgfpathlineto{\pgfqpoint{1.287153in}{1.301192in}}%
\pgfpathlineto{\pgfqpoint{1.286764in}{1.303709in}}%
\pgfpathlineto{\pgfqpoint{1.303048in}{1.304598in}}%
\pgfpathlineto{\pgfqpoint{1.319375in}{1.305229in}}%
\pgfpathlineto{\pgfqpoint{1.335729in}{1.305602in}}%
\pgfpathlineto{\pgfqpoint{1.352097in}{1.305717in}}%
\pgfpathlineto{\pgfqpoint{1.352092in}{1.303189in}}%
\pgfpathlineto{\pgfqpoint{1.352086in}{1.300891in}}%
\pgfpathlineto{\pgfqpoint{1.352081in}{1.298819in}}%
\pgfpathlineto{\pgfqpoint{1.352075in}{1.296970in}}%
\pgfpathlineto{\pgfqpoint{1.336102in}{1.296858in}}%
\pgfpathlineto{\pgfqpoint{1.320142in}{1.296493in}}%
\pgfpathlineto{\pgfqpoint{1.304209in}{1.295876in}}%
\pgfpathlineto{\pgfqpoint{1.288317in}{1.295007in}}%
\pgfpathclose%
\pgfusepath{fill}%
\end{pgfscope}%
\begin{pgfscope}%
\pgfpathrectangle{\pgfqpoint{0.329460in}{0.284240in}}{\pgfqpoint{1.989680in}{1.989680in}}%
\pgfusepath{clip}%
\pgfsetbuttcap%
\pgfsetroundjoin%
\definecolor{currentfill}{rgb}{0.267004,0.004874,0.329415}%
\pgfsetfillcolor{currentfill}%
\pgfsetlinewidth{0.000000pt}%
\definecolor{currentstroke}{rgb}{0.000000,0.000000,0.000000}%
\pgfsetstrokecolor{currentstroke}%
\pgfsetdash{}{0pt}%
\pgfpathmoveto{\pgfqpoint{1.352075in}{1.296970in}}%
\pgfpathlineto{\pgfqpoint{1.352081in}{1.298819in}}%
\pgfpathlineto{\pgfqpoint{1.352086in}{1.300891in}}%
\pgfpathlineto{\pgfqpoint{1.352092in}{1.303189in}}%
\pgfpathlineto{\pgfqpoint{1.352097in}{1.305717in}}%
\pgfpathlineto{\pgfqpoint{1.368464in}{1.305574in}}%
\pgfpathlineto{\pgfqpoint{1.384816in}{1.305172in}}%
\pgfpathlineto{\pgfqpoint{1.401139in}{1.304512in}}%
\pgfpathlineto{\pgfqpoint{1.417418in}{1.303594in}}%
\pgfpathlineto{\pgfqpoint{1.417018in}{1.301078in}}%
\pgfpathlineto{\pgfqpoint{1.416618in}{1.298792in}}%
\pgfpathlineto{\pgfqpoint{1.416219in}{1.296733in}}%
\pgfpathlineto{\pgfqpoint{1.415821in}{1.294895in}}%
\pgfpathlineto{\pgfqpoint{1.399935in}{1.295792in}}%
\pgfpathlineto{\pgfqpoint{1.384006in}{1.296437in}}%
\pgfpathlineto{\pgfqpoint{1.368048in}{1.296830in}}%
\pgfpathlineto{\pgfqpoint{1.352075in}{1.296970in}}%
\pgfpathclose%
\pgfusepath{fill}%
\end{pgfscope}%
\begin{pgfscope}%
\pgfpathrectangle{\pgfqpoint{0.329460in}{0.284240in}}{\pgfqpoint{1.989680in}{1.989680in}}%
\pgfusepath{clip}%
\pgfsetbuttcap%
\pgfsetroundjoin%
\definecolor{currentfill}{rgb}{0.260571,0.246922,0.522828}%
\pgfsetfillcolor{currentfill}%
\pgfsetlinewidth{0.000000pt}%
\definecolor{currentstroke}{rgb}{0.000000,0.000000,0.000000}%
\pgfsetstrokecolor{currentstroke}%
\pgfsetdash{}{0pt}%
\pgfpathmoveto{\pgfqpoint{1.427300in}{1.444907in}}%
\pgfpathlineto{\pgfqpoint{1.427727in}{1.454697in}}%
\pgfpathlineto{\pgfqpoint{1.428156in}{1.464845in}}%
\pgfpathlineto{\pgfqpoint{1.428587in}{1.475357in}}%
\pgfpathlineto{\pgfqpoint{1.429019in}{1.486239in}}%
\pgfpathlineto{\pgfqpoint{1.448085in}{1.484900in}}%
\pgfpathlineto{\pgfqpoint{1.467068in}{1.483268in}}%
\pgfpathlineto{\pgfqpoint{1.485953in}{1.481346in}}%
\pgfpathlineto{\pgfqpoint{1.504722in}{1.479134in}}%
\pgfpathlineto{\pgfqpoint{1.503868in}{1.468278in}}%
\pgfpathlineto{\pgfqpoint{1.503018in}{1.457794in}}%
\pgfpathlineto{\pgfqpoint{1.502171in}{1.447673in}}%
\pgfpathlineto{\pgfqpoint{1.501327in}{1.437912in}}%
\pgfpathlineto{\pgfqpoint{1.482973in}{1.440089in}}%
\pgfpathlineto{\pgfqpoint{1.464507in}{1.441982in}}%
\pgfpathlineto{\pgfqpoint{1.445944in}{1.443589in}}%
\pgfpathlineto{\pgfqpoint{1.427300in}{1.444907in}}%
\pgfpathclose%
\pgfusepath{fill}%
\end{pgfscope}%
\begin{pgfscope}%
\pgfpathrectangle{\pgfqpoint{0.329460in}{0.284240in}}{\pgfqpoint{1.989680in}{1.989680in}}%
\pgfusepath{clip}%
\pgfsetbuttcap%
\pgfsetroundjoin%
\definecolor{currentfill}{rgb}{0.277941,0.056324,0.381191}%
\pgfsetfillcolor{currentfill}%
\pgfsetlinewidth{0.000000pt}%
\definecolor{currentstroke}{rgb}{0.000000,0.000000,0.000000}%
\pgfsetstrokecolor{currentstroke}%
\pgfsetdash{}{0pt}%
\pgfpathmoveto{\pgfqpoint{1.488187in}{1.326043in}}%
\pgfpathlineto{\pgfqpoint{1.488991in}{1.330797in}}%
\pgfpathlineto{\pgfqpoint{1.489797in}{1.335827in}}%
\pgfpathlineto{\pgfqpoint{1.490606in}{1.341137in}}%
\pgfpathlineto{\pgfqpoint{1.491416in}{1.346732in}}%
\pgfpathlineto{\pgfqpoint{1.508433in}{1.344398in}}%
\pgfpathlineto{\pgfqpoint{1.525314in}{1.341798in}}%
\pgfpathlineto{\pgfqpoint{1.542044in}{1.338933in}}%
\pgfpathlineto{\pgfqpoint{1.540940in}{1.333376in}}%
\pgfpathlineto{\pgfqpoint{1.539838in}{1.328105in}}%
\pgfpathlineto{\pgfqpoint{1.538740in}{1.323114in}}%
\pgfpathlineto{\pgfqpoint{1.537644in}{1.318398in}}%
\pgfpathlineto{\pgfqpoint{1.521302in}{1.321207in}}%
\pgfpathlineto{\pgfqpoint{1.504811in}{1.323755in}}%
\pgfpathlineto{\pgfqpoint{1.488187in}{1.326043in}}%
\pgfpathclose%
\pgfusepath{fill}%
\end{pgfscope}%
\begin{pgfscope}%
\pgfpathrectangle{\pgfqpoint{0.329460in}{0.284240in}}{\pgfqpoint{1.989680in}{1.989680in}}%
\pgfusepath{clip}%
\pgfsetbuttcap%
\pgfsetroundjoin%
\definecolor{currentfill}{rgb}{0.282884,0.135920,0.453427}%
\pgfsetfillcolor{currentfill}%
\pgfsetlinewidth{0.000000pt}%
\definecolor{currentstroke}{rgb}{0.000000,0.000000,0.000000}%
\pgfsetstrokecolor{currentstroke}%
\pgfsetdash{}{0pt}%
\pgfpathmoveto{\pgfqpoint{1.494679in}{1.372057in}}%
\pgfpathlineto{\pgfqpoint{1.495500in}{1.379150in}}%
\pgfpathlineto{\pgfqpoint{1.496324in}{1.386557in}}%
\pgfpathlineto{\pgfqpoint{1.497151in}{1.394285in}}%
\pgfpathlineto{\pgfqpoint{1.497981in}{1.402338in}}%
\pgfpathlineto{\pgfqpoint{1.515798in}{1.399917in}}%
\pgfpathlineto{\pgfqpoint{1.533473in}{1.397220in}}%
\pgfpathlineto{\pgfqpoint{1.550991in}{1.394247in}}%
\pgfpathlineto{\pgfqpoint{1.549860in}{1.386229in}}%
\pgfpathlineto{\pgfqpoint{1.548733in}{1.378537in}}%
\pgfpathlineto{\pgfqpoint{1.547610in}{1.371166in}}%
\pgfpathlineto{\pgfqpoint{1.546490in}{1.364109in}}%
\pgfpathlineto{\pgfqpoint{1.529369in}{1.367029in}}%
\pgfpathlineto{\pgfqpoint{1.512094in}{1.369679in}}%
\pgfpathlineto{\pgfqpoint{1.494679in}{1.372057in}}%
\pgfpathclose%
\pgfusepath{fill}%
\end{pgfscope}%
\begin{pgfscope}%
\pgfpathrectangle{\pgfqpoint{0.329460in}{0.284240in}}{\pgfqpoint{1.989680in}{1.989680in}}%
\pgfusepath{clip}%
\pgfsetbuttcap%
\pgfsetroundjoin%
\definecolor{currentfill}{rgb}{0.282327,0.094955,0.417331}%
\pgfsetfillcolor{currentfill}%
\pgfsetlinewidth{0.000000pt}%
\definecolor{currentstroke}{rgb}{0.000000,0.000000,0.000000}%
\pgfsetstrokecolor{currentstroke}%
\pgfsetdash{}{0pt}%
\pgfpathmoveto{\pgfqpoint{1.145600in}{1.336165in}}%
\pgfpathlineto{\pgfqpoint{1.144407in}{1.341999in}}%
\pgfpathlineto{\pgfqpoint{1.143210in}{1.348127in}}%
\pgfpathlineto{\pgfqpoint{1.142011in}{1.354555in}}%
\pgfpathlineto{\pgfqpoint{1.140807in}{1.361289in}}%
\pgfpathlineto{\pgfqpoint{1.157779in}{1.364447in}}%
\pgfpathlineto{\pgfqpoint{1.174918in}{1.367337in}}%
\pgfpathlineto{\pgfqpoint{1.192210in}{1.369957in}}%
\pgfpathlineto{\pgfqpoint{1.209639in}{1.372304in}}%
\pgfpathlineto{\pgfqpoint{1.210447in}{1.365520in}}%
\pgfpathlineto{\pgfqpoint{1.211253in}{1.359040in}}%
\pgfpathlineto{\pgfqpoint{1.212057in}{1.352860in}}%
\pgfpathlineto{\pgfqpoint{1.212858in}{1.346974in}}%
\pgfpathlineto{\pgfqpoint{1.195826in}{1.344671in}}%
\pgfpathlineto{\pgfqpoint{1.178929in}{1.342100in}}%
\pgfpathlineto{\pgfqpoint{1.162182in}{1.339264in}}%
\pgfpathlineto{\pgfqpoint{1.145600in}{1.336165in}}%
\pgfpathclose%
\pgfusepath{fill}%
\end{pgfscope}%
\begin{pgfscope}%
\pgfpathrectangle{\pgfqpoint{0.329460in}{0.284240in}}{\pgfqpoint{1.989680in}{1.989680in}}%
\pgfusepath{clip}%
\pgfsetbuttcap%
\pgfsetroundjoin%
\definecolor{currentfill}{rgb}{0.233603,0.313828,0.543914}%
\pgfsetfillcolor{currentfill}%
\pgfsetlinewidth{0.000000pt}%
\definecolor{currentstroke}{rgb}{0.000000,0.000000,0.000000}%
\pgfsetstrokecolor{currentstroke}%
\pgfsetdash{}{0pt}%
\pgfpathmoveto{\pgfqpoint{1.275479in}{1.486370in}}%
\pgfpathlineto{\pgfqpoint{1.275057in}{1.497628in}}%
\pgfpathlineto{\pgfqpoint{1.274633in}{1.509269in}}%
\pgfpathlineto{\pgfqpoint{1.274207in}{1.521298in}}%
\pgfpathlineto{\pgfqpoint{1.273780in}{1.533721in}}%
\pgfpathlineto{\pgfqpoint{1.293348in}{1.534749in}}%
\pgfpathlineto{\pgfqpoint{1.312965in}{1.535480in}}%
\pgfpathlineto{\pgfqpoint{1.332615in}{1.535911in}}%
\pgfpathlineto{\pgfqpoint{1.352280in}{1.536044in}}%
\pgfpathlineto{\pgfqpoint{1.352274in}{1.523613in}}%
\pgfpathlineto{\pgfqpoint{1.352268in}{1.511576in}}%
\pgfpathlineto{\pgfqpoint{1.352262in}{1.499927in}}%
\pgfpathlineto{\pgfqpoint{1.352256in}{1.488660in}}%
\pgfpathlineto{\pgfqpoint{1.333022in}{1.488529in}}%
\pgfpathlineto{\pgfqpoint{1.313804in}{1.488103in}}%
\pgfpathlineto{\pgfqpoint{1.294618in}{1.487383in}}%
\pgfpathlineto{\pgfqpoint{1.275479in}{1.486370in}}%
\pgfpathclose%
\pgfusepath{fill}%
\end{pgfscope}%
\begin{pgfscope}%
\pgfpathrectangle{\pgfqpoint{0.329460in}{0.284240in}}{\pgfqpoint{1.989680in}{1.989680in}}%
\pgfusepath{clip}%
\pgfsetbuttcap%
\pgfsetroundjoin%
\definecolor{currentfill}{rgb}{0.233603,0.313828,0.543914}%
\pgfsetfillcolor{currentfill}%
\pgfsetlinewidth{0.000000pt}%
\definecolor{currentstroke}{rgb}{0.000000,0.000000,0.000000}%
\pgfsetstrokecolor{currentstroke}%
\pgfsetdash{}{0pt}%
\pgfpathmoveto{\pgfqpoint{1.352256in}{1.488660in}}%
\pgfpathlineto{\pgfqpoint{1.352262in}{1.499927in}}%
\pgfpathlineto{\pgfqpoint{1.352268in}{1.511576in}}%
\pgfpathlineto{\pgfqpoint{1.352274in}{1.523613in}}%
\pgfpathlineto{\pgfqpoint{1.352280in}{1.536044in}}%
\pgfpathlineto{\pgfqpoint{1.371945in}{1.535878in}}%
\pgfpathlineto{\pgfqpoint{1.391592in}{1.535413in}}%
\pgfpathlineto{\pgfqpoint{1.411204in}{1.534650in}}%
\pgfpathlineto{\pgfqpoint{1.430766in}{1.533588in}}%
\pgfpathlineto{\pgfqpoint{1.430327in}{1.521165in}}%
\pgfpathlineto{\pgfqpoint{1.429889in}{1.509137in}}%
\pgfpathlineto{\pgfqpoint{1.429453in}{1.497497in}}%
\pgfpathlineto{\pgfqpoint{1.429019in}{1.486239in}}%
\pgfpathlineto{\pgfqpoint{1.409887in}{1.487285in}}%
\pgfpathlineto{\pgfqpoint{1.390705in}{1.488038in}}%
\pgfpathlineto{\pgfqpoint{1.371489in}{1.488496in}}%
\pgfpathlineto{\pgfqpoint{1.352256in}{1.488660in}}%
\pgfpathclose%
\pgfusepath{fill}%
\end{pgfscope}%
\begin{pgfscope}%
\pgfpathrectangle{\pgfqpoint{0.329460in}{0.284240in}}{\pgfqpoint{1.989680in}{1.989680in}}%
\pgfusepath{clip}%
\pgfsetbuttcap%
\pgfsetroundjoin%
\definecolor{currentfill}{rgb}{0.277941,0.056324,0.381191}%
\pgfsetfillcolor{currentfill}%
\pgfsetlinewidth{0.000000pt}%
\definecolor{currentstroke}{rgb}{0.000000,0.000000,0.000000}%
\pgfsetstrokecolor{currentstroke}%
\pgfsetdash{}{0pt}%
\pgfpathmoveto{\pgfqpoint{1.150341in}{1.315685in}}%
\pgfpathlineto{\pgfqpoint{1.149160in}{1.320387in}}%
\pgfpathlineto{\pgfqpoint{1.147976in}{1.325364in}}%
\pgfpathlineto{\pgfqpoint{1.146789in}{1.330622in}}%
\pgfpathlineto{\pgfqpoint{1.145600in}{1.336165in}}%
\pgfpathlineto{\pgfqpoint{1.162182in}{1.339264in}}%
\pgfpathlineto{\pgfqpoint{1.178929in}{1.342100in}}%
\pgfpathlineto{\pgfqpoint{1.195826in}{1.344671in}}%
\pgfpathlineto{\pgfqpoint{1.212858in}{1.346974in}}%
\pgfpathlineto{\pgfqpoint{1.213657in}{1.341379in}}%
\pgfpathlineto{\pgfqpoint{1.214454in}{1.336067in}}%
\pgfpathlineto{\pgfqpoint{1.215250in}{1.331036in}}%
\pgfpathlineto{\pgfqpoint{1.216043in}{1.326281in}}%
\pgfpathlineto{\pgfqpoint{1.199405in}{1.324023in}}%
\pgfpathlineto{\pgfqpoint{1.182899in}{1.321503in}}%
\pgfpathlineto{\pgfqpoint{1.166539in}{1.318723in}}%
\pgfpathlineto{\pgfqpoint{1.150341in}{1.315685in}}%
\pgfpathclose%
\pgfusepath{fill}%
\end{pgfscope}%
\begin{pgfscope}%
\pgfpathrectangle{\pgfqpoint{0.329460in}{0.284240in}}{\pgfqpoint{1.989680in}{1.989680in}}%
\pgfusepath{clip}%
\pgfsetbuttcap%
\pgfsetroundjoin%
\definecolor{currentfill}{rgb}{0.282884,0.135920,0.453427}%
\pgfsetfillcolor{currentfill}%
\pgfsetlinewidth{0.000000pt}%
\definecolor{currentstroke}{rgb}{0.000000,0.000000,0.000000}%
\pgfsetstrokecolor{currentstroke}%
\pgfsetdash{}{0pt}%
\pgfpathmoveto{\pgfqpoint{1.140807in}{1.361289in}}%
\pgfpathlineto{\pgfqpoint{1.139601in}{1.368332in}}%
\pgfpathlineto{\pgfqpoint{1.138390in}{1.375691in}}%
\pgfpathlineto{\pgfqpoint{1.137176in}{1.383370in}}%
\pgfpathlineto{\pgfqpoint{1.135958in}{1.391376in}}%
\pgfpathlineto{\pgfqpoint{1.153323in}{1.394591in}}%
\pgfpathlineto{\pgfqpoint{1.170859in}{1.397533in}}%
\pgfpathlineto{\pgfqpoint{1.188551in}{1.400200in}}%
\pgfpathlineto{\pgfqpoint{1.206382in}{1.402590in}}%
\pgfpathlineto{\pgfqpoint{1.207200in}{1.394535in}}%
\pgfpathlineto{\pgfqpoint{1.208016in}{1.386807in}}%
\pgfpathlineto{\pgfqpoint{1.208829in}{1.379398in}}%
\pgfpathlineto{\pgfqpoint{1.209639in}{1.372304in}}%
\pgfpathlineto{\pgfqpoint{1.192210in}{1.369957in}}%
\pgfpathlineto{\pgfqpoint{1.174918in}{1.367337in}}%
\pgfpathlineto{\pgfqpoint{1.157779in}{1.364447in}}%
\pgfpathlineto{\pgfqpoint{1.140807in}{1.361289in}}%
\pgfpathclose%
\pgfusepath{fill}%
\end{pgfscope}%
\begin{pgfscope}%
\pgfpathrectangle{\pgfqpoint{0.329460in}{0.284240in}}{\pgfqpoint{1.989680in}{1.989680in}}%
\pgfusepath{clip}%
\pgfsetbuttcap%
\pgfsetroundjoin%
\definecolor{currentfill}{rgb}{0.272594,0.025563,0.353093}%
\pgfsetfillcolor{currentfill}%
\pgfsetlinewidth{0.000000pt}%
\definecolor{currentstroke}{rgb}{0.000000,0.000000,0.000000}%
\pgfsetstrokecolor{currentstroke}%
\pgfsetdash{}{0pt}%
\pgfpathmoveto{\pgfqpoint{1.484988in}{1.309685in}}%
\pgfpathlineto{\pgfqpoint{1.485785in}{1.313384in}}%
\pgfpathlineto{\pgfqpoint{1.486584in}{1.317341in}}%
\pgfpathlineto{\pgfqpoint{1.487385in}{1.321559in}}%
\pgfpathlineto{\pgfqpoint{1.488187in}{1.326043in}}%
\pgfpathlineto{\pgfqpoint{1.504811in}{1.323755in}}%
\pgfpathlineto{\pgfqpoint{1.521302in}{1.321207in}}%
\pgfpathlineto{\pgfqpoint{1.537644in}{1.318398in}}%
\pgfpathlineto{\pgfqpoint{1.536550in}{1.313954in}}%
\pgfpathlineto{\pgfqpoint{1.535459in}{1.309775in}}%
\pgfpathlineto{\pgfqpoint{1.534371in}{1.305859in}}%
\pgfpathlineto{\pgfqpoint{1.533284in}{1.302199in}}%
\pgfpathlineto{\pgfqpoint{1.517326in}{1.304949in}}%
\pgfpathlineto{\pgfqpoint{1.501223in}{1.307445in}}%
\pgfpathlineto{\pgfqpoint{1.484988in}{1.309685in}}%
\pgfpathclose%
\pgfusepath{fill}%
\end{pgfscope}%
\begin{pgfscope}%
\pgfpathrectangle{\pgfqpoint{0.329460in}{0.284240in}}{\pgfqpoint{1.989680in}{1.989680in}}%
\pgfusepath{clip}%
\pgfsetbuttcap%
\pgfsetroundjoin%
\definecolor{currentfill}{rgb}{0.276194,0.190074,0.493001}%
\pgfsetfillcolor{currentfill}%
\pgfsetlinewidth{0.000000pt}%
\definecolor{currentstroke}{rgb}{0.000000,0.000000,0.000000}%
\pgfsetstrokecolor{currentstroke}%
\pgfsetdash{}{0pt}%
\pgfpathmoveto{\pgfqpoint{1.497981in}{1.402338in}}%
\pgfpathlineto{\pgfqpoint{1.498813in}{1.410721in}}%
\pgfpathlineto{\pgfqpoint{1.499648in}{1.419441in}}%
\pgfpathlineto{\pgfqpoint{1.500486in}{1.428503in}}%
\pgfpathlineto{\pgfqpoint{1.501327in}{1.437912in}}%
\pgfpathlineto{\pgfqpoint{1.519551in}{1.435450in}}%
\pgfpathlineto{\pgfqpoint{1.537631in}{1.432707in}}%
\pgfpathlineto{\pgfqpoint{1.555551in}{1.429684in}}%
\pgfpathlineto{\pgfqpoint{1.554405in}{1.420309in}}%
\pgfpathlineto{\pgfqpoint{1.553263in}{1.411281in}}%
\pgfpathlineto{\pgfqpoint{1.552125in}{1.402596in}}%
\pgfpathlineto{\pgfqpoint{1.550991in}{1.394247in}}%
\pgfpathlineto{\pgfqpoint{1.533473in}{1.397220in}}%
\pgfpathlineto{\pgfqpoint{1.515798in}{1.399917in}}%
\pgfpathlineto{\pgfqpoint{1.497981in}{1.402338in}}%
\pgfpathclose%
\pgfusepath{fill}%
\end{pgfscope}%
\begin{pgfscope}%
\pgfpathrectangle{\pgfqpoint{0.329460in}{0.284240in}}{\pgfqpoint{1.989680in}{1.989680in}}%
\pgfusepath{clip}%
\pgfsetbuttcap%
\pgfsetroundjoin%
\definecolor{currentfill}{rgb}{0.272594,0.025563,0.353093}%
\pgfsetfillcolor{currentfill}%
\pgfsetlinewidth{0.000000pt}%
\definecolor{currentstroke}{rgb}{0.000000,0.000000,0.000000}%
\pgfsetstrokecolor{currentstroke}%
\pgfsetdash{}{0pt}%
\pgfpathmoveto{\pgfqpoint{1.155039in}{1.299543in}}%
\pgfpathlineto{\pgfqpoint{1.153868in}{1.303188in}}%
\pgfpathlineto{\pgfqpoint{1.152695in}{1.307090in}}%
\pgfpathlineto{\pgfqpoint{1.151520in}{1.311255in}}%
\pgfpathlineto{\pgfqpoint{1.150341in}{1.315685in}}%
\pgfpathlineto{\pgfqpoint{1.166539in}{1.318723in}}%
\pgfpathlineto{\pgfqpoint{1.182899in}{1.321503in}}%
\pgfpathlineto{\pgfqpoint{1.199405in}{1.324023in}}%
\pgfpathlineto{\pgfqpoint{1.216043in}{1.326281in}}%
\pgfpathlineto{\pgfqpoint{1.216834in}{1.321795in}}%
\pgfpathlineto{\pgfqpoint{1.217624in}{1.317576in}}%
\pgfpathlineto{\pgfqpoint{1.218412in}{1.313618in}}%
\pgfpathlineto{\pgfqpoint{1.219199in}{1.309918in}}%
\pgfpathlineto{\pgfqpoint{1.202950in}{1.307706in}}%
\pgfpathlineto{\pgfqpoint{1.186831in}{1.305239in}}%
\pgfpathlineto{\pgfqpoint{1.170856in}{1.302517in}}%
\pgfpathlineto{\pgfqpoint{1.155039in}{1.299543in}}%
\pgfpathclose%
\pgfusepath{fill}%
\end{pgfscope}%
\begin{pgfscope}%
\pgfpathrectangle{\pgfqpoint{0.329460in}{0.284240in}}{\pgfqpoint{1.989680in}{1.989680in}}%
\pgfusepath{clip}%
\pgfsetbuttcap%
\pgfsetroundjoin%
\definecolor{currentfill}{rgb}{0.267004,0.004874,0.329415}%
\pgfsetfillcolor{currentfill}%
\pgfsetlinewidth{0.000000pt}%
\definecolor{currentstroke}{rgb}{0.000000,0.000000,0.000000}%
\pgfsetstrokecolor{currentstroke}%
\pgfsetdash{}{0pt}%
\pgfpathmoveto{\pgfqpoint{1.225439in}{1.289032in}}%
\pgfpathlineto{\pgfqpoint{1.224663in}{1.290834in}}%
\pgfpathlineto{\pgfqpoint{1.223886in}{1.292860in}}%
\pgfpathlineto{\pgfqpoint{1.223108in}{1.295112in}}%
\pgfpathlineto{\pgfqpoint{1.222329in}{1.297594in}}%
\pgfpathlineto{\pgfqpoint{1.238304in}{1.299505in}}%
\pgfpathlineto{\pgfqpoint{1.254377in}{1.301162in}}%
\pgfpathlineto{\pgfqpoint{1.270535in}{1.302563in}}%
\pgfpathlineto{\pgfqpoint{1.286764in}{1.303709in}}%
\pgfpathlineto{\pgfqpoint{1.287153in}{1.301192in}}%
\pgfpathlineto{\pgfqpoint{1.287541in}{1.298905in}}%
\pgfpathlineto{\pgfqpoint{1.287929in}{1.296845in}}%
\pgfpathlineto{\pgfqpoint{1.288317in}{1.295007in}}%
\pgfpathlineto{\pgfqpoint{1.272480in}{1.293888in}}%
\pgfpathlineto{\pgfqpoint{1.256712in}{1.292518in}}%
\pgfpathlineto{\pgfqpoint{1.241027in}{1.290899in}}%
\pgfpathlineto{\pgfqpoint{1.225439in}{1.289032in}}%
\pgfpathclose%
\pgfusepath{fill}%
\end{pgfscope}%
\begin{pgfscope}%
\pgfpathrectangle{\pgfqpoint{0.329460in}{0.284240in}}{\pgfqpoint{1.989680in}{1.989680in}}%
\pgfusepath{clip}%
\pgfsetbuttcap%
\pgfsetroundjoin%
\definecolor{currentfill}{rgb}{0.267004,0.004874,0.329415}%
\pgfsetfillcolor{currentfill}%
\pgfsetlinewidth{0.000000pt}%
\definecolor{currentstroke}{rgb}{0.000000,0.000000,0.000000}%
\pgfsetstrokecolor{currentstroke}%
\pgfsetdash{}{0pt}%
\pgfpathmoveto{\pgfqpoint{1.415821in}{1.294895in}}%
\pgfpathlineto{\pgfqpoint{1.416219in}{1.296733in}}%
\pgfpathlineto{\pgfqpoint{1.416618in}{1.298792in}}%
\pgfpathlineto{\pgfqpoint{1.417018in}{1.301078in}}%
\pgfpathlineto{\pgfqpoint{1.417418in}{1.303594in}}%
\pgfpathlineto{\pgfqpoint{1.433639in}{1.302420in}}%
\pgfpathlineto{\pgfqpoint{1.449788in}{1.300990in}}%
\pgfpathlineto{\pgfqpoint{1.465851in}{1.299305in}}%
\pgfpathlineto{\pgfqpoint{1.481814in}{1.297366in}}%
\pgfpathlineto{\pgfqpoint{1.481024in}{1.294885in}}%
\pgfpathlineto{\pgfqpoint{1.480236in}{1.292634in}}%
\pgfpathlineto{\pgfqpoint{1.479448in}{1.290610in}}%
\pgfpathlineto{\pgfqpoint{1.478662in}{1.288809in}}%
\pgfpathlineto{\pgfqpoint{1.463085in}{1.290703in}}%
\pgfpathlineto{\pgfqpoint{1.447410in}{1.292350in}}%
\pgfpathlineto{\pgfqpoint{1.431651in}{1.293748in}}%
\pgfpathlineto{\pgfqpoint{1.415821in}{1.294895in}}%
\pgfpathclose%
\pgfusepath{fill}%
\end{pgfscope}%
\begin{pgfscope}%
\pgfpathrectangle{\pgfqpoint{0.329460in}{0.284240in}}{\pgfqpoint{1.989680in}{1.989680in}}%
\pgfusepath{clip}%
\pgfsetbuttcap%
\pgfsetroundjoin%
\definecolor{currentfill}{rgb}{0.276194,0.190074,0.493001}%
\pgfsetfillcolor{currentfill}%
\pgfsetlinewidth{0.000000pt}%
\definecolor{currentstroke}{rgb}{0.000000,0.000000,0.000000}%
\pgfsetstrokecolor{currentstroke}%
\pgfsetdash{}{0pt}%
\pgfpathmoveto{\pgfqpoint{1.135958in}{1.391376in}}%
\pgfpathlineto{\pgfqpoint{1.134735in}{1.399712in}}%
\pgfpathlineto{\pgfqpoint{1.133509in}{1.408385in}}%
\pgfpathlineto{\pgfqpoint{1.132278in}{1.417401in}}%
\pgfpathlineto{\pgfqpoint{1.131043in}{1.426764in}}%
\pgfpathlineto{\pgfqpoint{1.148807in}{1.430034in}}%
\pgfpathlineto{\pgfqpoint{1.166745in}{1.433026in}}%
\pgfpathlineto{\pgfqpoint{1.184842in}{1.435738in}}%
\pgfpathlineto{\pgfqpoint{1.203082in}{1.438168in}}%
\pgfpathlineto{\pgfqpoint{1.203911in}{1.428758in}}%
\pgfpathlineto{\pgfqpoint{1.204738in}{1.419695in}}%
\pgfpathlineto{\pgfqpoint{1.205561in}{1.410974in}}%
\pgfpathlineto{\pgfqpoint{1.206382in}{1.402590in}}%
\pgfpathlineto{\pgfqpoint{1.188551in}{1.400200in}}%
\pgfpathlineto{\pgfqpoint{1.170859in}{1.397533in}}%
\pgfpathlineto{\pgfqpoint{1.153323in}{1.394591in}}%
\pgfpathlineto{\pgfqpoint{1.135958in}{1.391376in}}%
\pgfpathclose%
\pgfusepath{fill}%
\end{pgfscope}%
\begin{pgfscope}%
\pgfpathrectangle{\pgfqpoint{0.329460in}{0.284240in}}{\pgfqpoint{1.989680in}{1.989680in}}%
\pgfusepath{clip}%
\pgfsetbuttcap%
\pgfsetroundjoin%
\definecolor{currentfill}{rgb}{0.233603,0.313828,0.543914}%
\pgfsetfillcolor{currentfill}%
\pgfsetlinewidth{0.000000pt}%
\definecolor{currentstroke}{rgb}{0.000000,0.000000,0.000000}%
\pgfsetstrokecolor{currentstroke}%
\pgfsetdash{}{0pt}%
\pgfpathmoveto{\pgfqpoint{1.199732in}{1.479394in}}%
\pgfpathlineto{\pgfqpoint{1.198887in}{1.490626in}}%
\pgfpathlineto{\pgfqpoint{1.198038in}{1.502242in}}%
\pgfpathlineto{\pgfqpoint{1.197185in}{1.514245in}}%
\pgfpathlineto{\pgfqpoint{1.196328in}{1.526644in}}%
\pgfpathlineto{\pgfqpoint{1.215535in}{1.528856in}}%
\pgfpathlineto{\pgfqpoint{1.234856in}{1.530773in}}%
\pgfpathlineto{\pgfqpoint{1.254277in}{1.532395in}}%
\pgfpathlineto{\pgfqpoint{1.273780in}{1.533721in}}%
\pgfpathlineto{\pgfqpoint{1.274207in}{1.521298in}}%
\pgfpathlineto{\pgfqpoint{1.274633in}{1.509269in}}%
\pgfpathlineto{\pgfqpoint{1.275057in}{1.497628in}}%
\pgfpathlineto{\pgfqpoint{1.275479in}{1.486370in}}%
\pgfpathlineto{\pgfqpoint{1.256405in}{1.485063in}}%
\pgfpathlineto{\pgfqpoint{1.237412in}{1.483464in}}%
\pgfpathlineto{\pgfqpoint{1.218516in}{1.481574in}}%
\pgfpathlineto{\pgfqpoint{1.199732in}{1.479394in}}%
\pgfpathclose%
\pgfusepath{fill}%
\end{pgfscope}%
\begin{pgfscope}%
\pgfpathrectangle{\pgfqpoint{0.329460in}{0.284240in}}{\pgfqpoint{1.989680in}{1.989680in}}%
\pgfusepath{clip}%
\pgfsetbuttcap%
\pgfsetroundjoin%
\definecolor{currentfill}{rgb}{0.268510,0.009605,0.335427}%
\pgfsetfillcolor{currentfill}%
\pgfsetlinewidth{0.000000pt}%
\definecolor{currentstroke}{rgb}{0.000000,0.000000,0.000000}%
\pgfsetstrokecolor{currentstroke}%
\pgfsetdash{}{0pt}%
\pgfpathmoveto{\pgfqpoint{1.481814in}{1.297366in}}%
\pgfpathlineto{\pgfqpoint{1.482605in}{1.300083in}}%
\pgfpathlineto{\pgfqpoint{1.483398in}{1.303038in}}%
\pgfpathlineto{\pgfqpoint{1.484192in}{1.306237in}}%
\pgfpathlineto{\pgfqpoint{1.484988in}{1.309685in}}%
\pgfpathlineto{\pgfqpoint{1.501223in}{1.307445in}}%
\pgfpathlineto{\pgfqpoint{1.517326in}{1.304949in}}%
\pgfpathlineto{\pgfqpoint{1.533284in}{1.302199in}}%
\pgfpathlineto{\pgfqpoint{1.532200in}{1.298792in}}%
\pgfpathlineto{\pgfqpoint{1.531118in}{1.295634in}}%
\pgfpathlineto{\pgfqpoint{1.530038in}{1.292719in}}%
\pgfpathlineto{\pgfqpoint{1.528960in}{1.290044in}}%
\pgfpathlineto{\pgfqpoint{1.513382in}{1.292734in}}%
\pgfpathlineto{\pgfqpoint{1.497662in}{1.295175in}}%
\pgfpathlineto{\pgfqpoint{1.481814in}{1.297366in}}%
\pgfpathclose%
\pgfusepath{fill}%
\end{pgfscope}%
\begin{pgfscope}%
\pgfpathrectangle{\pgfqpoint{0.329460in}{0.284240in}}{\pgfqpoint{1.989680in}{1.989680in}}%
\pgfusepath{clip}%
\pgfsetbuttcap%
\pgfsetroundjoin%
\definecolor{currentfill}{rgb}{0.267004,0.004874,0.329415}%
\pgfsetfillcolor{currentfill}%
\pgfsetlinewidth{0.000000pt}%
\definecolor{currentstroke}{rgb}{0.000000,0.000000,0.000000}%
\pgfsetstrokecolor{currentstroke}%
\pgfsetdash{}{0pt}%
\pgfpathmoveto{\pgfqpoint{1.289861in}{1.289796in}}%
\pgfpathlineto{\pgfqpoint{1.289476in}{1.290786in}}%
\pgfpathlineto{\pgfqpoint{1.289090in}{1.291982in}}%
\pgfpathlineto{\pgfqpoint{1.288703in}{1.293387in}}%
\pgfpathlineto{\pgfqpoint{1.288317in}{1.295007in}}%
\pgfpathlineto{\pgfqpoint{1.304209in}{1.295876in}}%
\pgfpathlineto{\pgfqpoint{1.320142in}{1.296493in}}%
\pgfpathlineto{\pgfqpoint{1.336102in}{1.296858in}}%
\pgfpathlineto{\pgfqpoint{1.352075in}{1.296970in}}%
\pgfpathlineto{\pgfqpoint{1.352070in}{1.295338in}}%
\pgfpathlineto{\pgfqpoint{1.352064in}{1.293921in}}%
\pgfpathlineto{\pgfqpoint{1.352059in}{1.292713in}}%
\pgfpathlineto{\pgfqpoint{1.352053in}{1.291712in}}%
\pgfpathlineto{\pgfqpoint{1.336472in}{1.291602in}}%
\pgfpathlineto{\pgfqpoint{1.320904in}{1.291246in}}%
\pgfpathlineto{\pgfqpoint{1.305362in}{1.290644in}}%
\pgfpathlineto{\pgfqpoint{1.289861in}{1.289796in}}%
\pgfpathclose%
\pgfusepath{fill}%
\end{pgfscope}%
\begin{pgfscope}%
\pgfpathrectangle{\pgfqpoint{0.329460in}{0.284240in}}{\pgfqpoint{1.989680in}{1.989680in}}%
\pgfusepath{clip}%
\pgfsetbuttcap%
\pgfsetroundjoin%
\definecolor{currentfill}{rgb}{0.233603,0.313828,0.543914}%
\pgfsetfillcolor{currentfill}%
\pgfsetlinewidth{0.000000pt}%
\definecolor{currentstroke}{rgb}{0.000000,0.000000,0.000000}%
\pgfsetstrokecolor{currentstroke}%
\pgfsetdash{}{0pt}%
\pgfpathmoveto{\pgfqpoint{1.429019in}{1.486239in}}%
\pgfpathlineto{\pgfqpoint{1.429453in}{1.497497in}}%
\pgfpathlineto{\pgfqpoint{1.429889in}{1.509137in}}%
\pgfpathlineto{\pgfqpoint{1.430327in}{1.521165in}}%
\pgfpathlineto{\pgfqpoint{1.430766in}{1.533588in}}%
\pgfpathlineto{\pgfqpoint{1.450260in}{1.532230in}}%
\pgfpathlineto{\pgfqpoint{1.469671in}{1.530575in}}%
\pgfpathlineto{\pgfqpoint{1.488980in}{1.528624in}}%
\pgfpathlineto{\pgfqpoint{1.508173in}{1.526380in}}%
\pgfpathlineto{\pgfqpoint{1.507305in}{1.513983in}}%
\pgfpathlineto{\pgfqpoint{1.506440in}{1.501980in}}%
\pgfpathlineto{\pgfqpoint{1.505579in}{1.490365in}}%
\pgfpathlineto{\pgfqpoint{1.504722in}{1.479134in}}%
\pgfpathlineto{\pgfqpoint{1.485953in}{1.481346in}}%
\pgfpathlineto{\pgfqpoint{1.467068in}{1.483268in}}%
\pgfpathlineto{\pgfqpoint{1.448085in}{1.484900in}}%
\pgfpathlineto{\pgfqpoint{1.429019in}{1.486239in}}%
\pgfpathclose%
\pgfusepath{fill}%
\end{pgfscope}%
\begin{pgfscope}%
\pgfpathrectangle{\pgfqpoint{0.329460in}{0.284240in}}{\pgfqpoint{1.989680in}{1.989680in}}%
\pgfusepath{clip}%
\pgfsetbuttcap%
\pgfsetroundjoin%
\definecolor{currentfill}{rgb}{0.267004,0.004874,0.329415}%
\pgfsetfillcolor{currentfill}%
\pgfsetlinewidth{0.000000pt}%
\definecolor{currentstroke}{rgb}{0.000000,0.000000,0.000000}%
\pgfsetstrokecolor{currentstroke}%
\pgfsetdash{}{0pt}%
\pgfpathmoveto{\pgfqpoint{1.352053in}{1.291712in}}%
\pgfpathlineto{\pgfqpoint{1.352059in}{1.292713in}}%
\pgfpathlineto{\pgfqpoint{1.352064in}{1.293921in}}%
\pgfpathlineto{\pgfqpoint{1.352070in}{1.295338in}}%
\pgfpathlineto{\pgfqpoint{1.352075in}{1.296970in}}%
\pgfpathlineto{\pgfqpoint{1.368048in}{1.296830in}}%
\pgfpathlineto{\pgfqpoint{1.384006in}{1.296437in}}%
\pgfpathlineto{\pgfqpoint{1.399935in}{1.295792in}}%
\pgfpathlineto{\pgfqpoint{1.415821in}{1.294895in}}%
\pgfpathlineto{\pgfqpoint{1.415423in}{1.293276in}}%
\pgfpathlineto{\pgfqpoint{1.415026in}{1.291871in}}%
\pgfpathlineto{\pgfqpoint{1.414630in}{1.290676in}}%
\pgfpathlineto{\pgfqpoint{1.414233in}{1.289687in}}%
\pgfpathlineto{\pgfqpoint{1.398737in}{1.290562in}}%
\pgfpathlineto{\pgfqpoint{1.383200in}{1.291191in}}%
\pgfpathlineto{\pgfqpoint{1.367634in}{1.291575in}}%
\pgfpathlineto{\pgfqpoint{1.352053in}{1.291712in}}%
\pgfpathclose%
\pgfusepath{fill}%
\end{pgfscope}%
\begin{pgfscope}%
\pgfpathrectangle{\pgfqpoint{0.329460in}{0.284240in}}{\pgfqpoint{1.989680in}{1.989680in}}%
\pgfusepath{clip}%
\pgfsetbuttcap%
\pgfsetroundjoin%
\definecolor{currentfill}{rgb}{0.268510,0.009605,0.335427}%
\pgfsetfillcolor{currentfill}%
\pgfsetlinewidth{0.000000pt}%
\definecolor{currentstroke}{rgb}{0.000000,0.000000,0.000000}%
\pgfsetstrokecolor{currentstroke}%
\pgfsetdash{}{0pt}%
\pgfpathmoveto{\pgfqpoint{1.159699in}{1.287446in}}%
\pgfpathlineto{\pgfqpoint{1.158537in}{1.290106in}}%
\pgfpathlineto{\pgfqpoint{1.157373in}{1.293006in}}%
\pgfpathlineto{\pgfqpoint{1.156207in}{1.296150in}}%
\pgfpathlineto{\pgfqpoint{1.155039in}{1.299543in}}%
\pgfpathlineto{\pgfqpoint{1.170856in}{1.302517in}}%
\pgfpathlineto{\pgfqpoint{1.186831in}{1.305239in}}%
\pgfpathlineto{\pgfqpoint{1.202950in}{1.307706in}}%
\pgfpathlineto{\pgfqpoint{1.219199in}{1.309918in}}%
\pgfpathlineto{\pgfqpoint{1.219983in}{1.306469in}}%
\pgfpathlineto{\pgfqpoint{1.220767in}{1.303269in}}%
\pgfpathlineto{\pgfqpoint{1.221549in}{1.300312in}}%
\pgfpathlineto{\pgfqpoint{1.222329in}{1.297594in}}%
\pgfpathlineto{\pgfqpoint{1.206467in}{1.295431in}}%
\pgfpathlineto{\pgfqpoint{1.190733in}{1.293018in}}%
\pgfpathlineto{\pgfqpoint{1.175138in}{1.290355in}}%
\pgfpathlineto{\pgfqpoint{1.159699in}{1.287446in}}%
\pgfpathclose%
\pgfusepath{fill}%
\end{pgfscope}%
\begin{pgfscope}%
\pgfpathrectangle{\pgfqpoint{0.329460in}{0.284240in}}{\pgfqpoint{1.989680in}{1.989680in}}%
\pgfusepath{clip}%
\pgfsetbuttcap%
\pgfsetroundjoin%
\definecolor{currentfill}{rgb}{0.260571,0.246922,0.522828}%
\pgfsetfillcolor{currentfill}%
\pgfsetlinewidth{0.000000pt}%
\definecolor{currentstroke}{rgb}{0.000000,0.000000,0.000000}%
\pgfsetstrokecolor{currentstroke}%
\pgfsetdash{}{0pt}%
\pgfpathmoveto{\pgfqpoint{1.501327in}{1.437912in}}%
\pgfpathlineto{\pgfqpoint{1.502171in}{1.447673in}}%
\pgfpathlineto{\pgfqpoint{1.503018in}{1.457794in}}%
\pgfpathlineto{\pgfqpoint{1.503868in}{1.468278in}}%
\pgfpathlineto{\pgfqpoint{1.504722in}{1.479134in}}%
\pgfpathlineto{\pgfqpoint{1.523360in}{1.476634in}}%
\pgfpathlineto{\pgfqpoint{1.541851in}{1.473848in}}%
\pgfpathlineto{\pgfqpoint{1.560179in}{1.470777in}}%
\pgfpathlineto{\pgfqpoint{1.559015in}{1.459954in}}%
\pgfpathlineto{\pgfqpoint{1.557856in}{1.449501in}}%
\pgfpathlineto{\pgfqpoint{1.556701in}{1.439413in}}%
\pgfpathlineto{\pgfqpoint{1.555551in}{1.429684in}}%
\pgfpathlineto{\pgfqpoint{1.537631in}{1.432707in}}%
\pgfpathlineto{\pgfqpoint{1.519551in}{1.435450in}}%
\pgfpathlineto{\pgfqpoint{1.501327in}{1.437912in}}%
\pgfpathclose%
\pgfusepath{fill}%
\end{pgfscope}%
\begin{pgfscope}%
\pgfpathrectangle{\pgfqpoint{0.329460in}{0.284240in}}{\pgfqpoint{1.989680in}{1.989680in}}%
\pgfusepath{clip}%
\pgfsetbuttcap%
\pgfsetroundjoin%
\definecolor{currentfill}{rgb}{0.260571,0.246922,0.522828}%
\pgfsetfillcolor{currentfill}%
\pgfsetlinewidth{0.000000pt}%
\definecolor{currentstroke}{rgb}{0.000000,0.000000,0.000000}%
\pgfsetstrokecolor{currentstroke}%
\pgfsetdash{}{0pt}%
\pgfpathmoveto{\pgfqpoint{1.131043in}{1.426764in}}%
\pgfpathlineto{\pgfqpoint{1.129803in}{1.436481in}}%
\pgfpathlineto{\pgfqpoint{1.128558in}{1.446557in}}%
\pgfpathlineto{\pgfqpoint{1.127309in}{1.456999in}}%
\pgfpathlineto{\pgfqpoint{1.126055in}{1.467811in}}%
\pgfpathlineto{\pgfqpoint{1.144224in}{1.471132in}}%
\pgfpathlineto{\pgfqpoint{1.162571in}{1.474171in}}%
\pgfpathlineto{\pgfqpoint{1.181079in}{1.476926in}}%
\pgfpathlineto{\pgfqpoint{1.199732in}{1.479394in}}%
\pgfpathlineto{\pgfqpoint{1.200575in}{1.468537in}}%
\pgfpathlineto{\pgfqpoint{1.201414in}{1.458052in}}%
\pgfpathlineto{\pgfqpoint{1.202249in}{1.447930in}}%
\pgfpathlineto{\pgfqpoint{1.203082in}{1.438168in}}%
\pgfpathlineto{\pgfqpoint{1.184842in}{1.435738in}}%
\pgfpathlineto{\pgfqpoint{1.166745in}{1.433026in}}%
\pgfpathlineto{\pgfqpoint{1.148807in}{1.430034in}}%
\pgfpathlineto{\pgfqpoint{1.131043in}{1.426764in}}%
\pgfpathclose%
\pgfusepath{fill}%
\end{pgfscope}%
\begin{pgfscope}%
\pgfpathrectangle{\pgfqpoint{0.329460in}{0.284240in}}{\pgfqpoint{1.989680in}{1.989680in}}%
\pgfusepath{clip}%
\pgfsetbuttcap%
\pgfsetroundjoin%
\definecolor{currentfill}{rgb}{0.201239,0.383670,0.554294}%
\pgfsetfillcolor{currentfill}%
\pgfsetlinewidth{0.000000pt}%
\definecolor{currentstroke}{rgb}{0.000000,0.000000,0.000000}%
\pgfsetstrokecolor{currentstroke}%
\pgfsetdash{}{0pt}%
\pgfpathmoveto{\pgfqpoint{1.273780in}{1.533721in}}%
\pgfpathlineto{\pgfqpoint{1.273350in}{1.546545in}}%
\pgfpathlineto{\pgfqpoint{1.272919in}{1.559777in}}%
\pgfpathlineto{\pgfqpoint{1.272485in}{1.573422in}}%
\pgfpathlineto{\pgfqpoint{1.272050in}{1.587488in}}%
\pgfpathlineto{\pgfqpoint{1.292055in}{1.588530in}}%
\pgfpathlineto{\pgfqpoint{1.312111in}{1.589270in}}%
\pgfpathlineto{\pgfqpoint{1.332200in}{1.589708in}}%
\pgfpathlineto{\pgfqpoint{1.352305in}{1.589842in}}%
\pgfpathlineto{\pgfqpoint{1.352298in}{1.575769in}}%
\pgfpathlineto{\pgfqpoint{1.352292in}{1.562116in}}%
\pgfpathlineto{\pgfqpoint{1.352286in}{1.548876in}}%
\pgfpathlineto{\pgfqpoint{1.352280in}{1.536044in}}%
\pgfpathlineto{\pgfqpoint{1.332615in}{1.535911in}}%
\pgfpathlineto{\pgfqpoint{1.312965in}{1.535480in}}%
\pgfpathlineto{\pgfqpoint{1.293348in}{1.534749in}}%
\pgfpathlineto{\pgfqpoint{1.273780in}{1.533721in}}%
\pgfpathclose%
\pgfusepath{fill}%
\end{pgfscope}%
\begin{pgfscope}%
\pgfpathrectangle{\pgfqpoint{0.329460in}{0.284240in}}{\pgfqpoint{1.989680in}{1.989680in}}%
\pgfusepath{clip}%
\pgfsetbuttcap%
\pgfsetroundjoin%
\definecolor{currentfill}{rgb}{0.201239,0.383670,0.554294}%
\pgfsetfillcolor{currentfill}%
\pgfsetlinewidth{0.000000pt}%
\definecolor{currentstroke}{rgb}{0.000000,0.000000,0.000000}%
\pgfsetstrokecolor{currentstroke}%
\pgfsetdash{}{0pt}%
\pgfpathmoveto{\pgfqpoint{1.352280in}{1.536044in}}%
\pgfpathlineto{\pgfqpoint{1.352286in}{1.548876in}}%
\pgfpathlineto{\pgfqpoint{1.352292in}{1.562116in}}%
\pgfpathlineto{\pgfqpoint{1.352298in}{1.575769in}}%
\pgfpathlineto{\pgfqpoint{1.352305in}{1.589842in}}%
\pgfpathlineto{\pgfqpoint{1.372408in}{1.589674in}}%
\pgfpathlineto{\pgfqpoint{1.392494in}{1.589203in}}%
\pgfpathlineto{\pgfqpoint{1.412545in}{1.588429in}}%
\pgfpathlineto{\pgfqpoint{1.432545in}{1.587354in}}%
\pgfpathlineto{\pgfqpoint{1.432097in}{1.573288in}}%
\pgfpathlineto{\pgfqpoint{1.431651in}{1.559643in}}%
\pgfpathlineto{\pgfqpoint{1.431208in}{1.546412in}}%
\pgfpathlineto{\pgfqpoint{1.430766in}{1.533588in}}%
\pgfpathlineto{\pgfqpoint{1.411204in}{1.534650in}}%
\pgfpathlineto{\pgfqpoint{1.391592in}{1.535413in}}%
\pgfpathlineto{\pgfqpoint{1.371945in}{1.535878in}}%
\pgfpathlineto{\pgfqpoint{1.352280in}{1.536044in}}%
\pgfpathclose%
\pgfusepath{fill}%
\end{pgfscope}%
\begin{pgfscope}%
\pgfpathrectangle{\pgfqpoint{0.329460in}{0.284240in}}{\pgfqpoint{1.989680in}{1.989680in}}%
\pgfusepath{clip}%
\pgfsetbuttcap%
\pgfsetroundjoin%
\definecolor{currentfill}{rgb}{0.267004,0.004874,0.329415}%
\pgfsetfillcolor{currentfill}%
\pgfsetlinewidth{0.000000pt}%
\definecolor{currentstroke}{rgb}{0.000000,0.000000,0.000000}%
\pgfsetstrokecolor{currentstroke}%
\pgfsetdash{}{0pt}%
\pgfpathmoveto{\pgfqpoint{1.228531in}{1.283962in}}%
\pgfpathlineto{\pgfqpoint{1.227759in}{1.284916in}}%
\pgfpathlineto{\pgfqpoint{1.226987in}{1.286076in}}%
\pgfpathlineto{\pgfqpoint{1.226213in}{1.287447in}}%
\pgfpathlineto{\pgfqpoint{1.225439in}{1.289032in}}%
\pgfpathlineto{\pgfqpoint{1.241027in}{1.290899in}}%
\pgfpathlineto{\pgfqpoint{1.256712in}{1.292518in}}%
\pgfpathlineto{\pgfqpoint{1.272480in}{1.293888in}}%
\pgfpathlineto{\pgfqpoint{1.288317in}{1.295007in}}%
\pgfpathlineto{\pgfqpoint{1.288703in}{1.293387in}}%
\pgfpathlineto{\pgfqpoint{1.289090in}{1.291982in}}%
\pgfpathlineto{\pgfqpoint{1.289476in}{1.290786in}}%
\pgfpathlineto{\pgfqpoint{1.289861in}{1.289796in}}%
\pgfpathlineto{\pgfqpoint{1.274414in}{1.288703in}}%
\pgfpathlineto{\pgfqpoint{1.259034in}{1.287365in}}%
\pgfpathlineto{\pgfqpoint{1.243735in}{1.285785in}}%
\pgfpathlineto{\pgfqpoint{1.228531in}{1.283962in}}%
\pgfpathclose%
\pgfusepath{fill}%
\end{pgfscope}%
\begin{pgfscope}%
\pgfpathrectangle{\pgfqpoint{0.329460in}{0.284240in}}{\pgfqpoint{1.989680in}{1.989680in}}%
\pgfusepath{clip}%
\pgfsetbuttcap%
\pgfsetroundjoin%
\definecolor{currentfill}{rgb}{0.267004,0.004874,0.329415}%
\pgfsetfillcolor{currentfill}%
\pgfsetlinewidth{0.000000pt}%
\definecolor{currentstroke}{rgb}{0.000000,0.000000,0.000000}%
\pgfsetstrokecolor{currentstroke}%
\pgfsetdash{}{0pt}%
\pgfpathmoveto{\pgfqpoint{1.414233in}{1.289687in}}%
\pgfpathlineto{\pgfqpoint{1.414630in}{1.290676in}}%
\pgfpathlineto{\pgfqpoint{1.415026in}{1.291871in}}%
\pgfpathlineto{\pgfqpoint{1.415423in}{1.293276in}}%
\pgfpathlineto{\pgfqpoint{1.415821in}{1.294895in}}%
\pgfpathlineto{\pgfqpoint{1.431651in}{1.293748in}}%
\pgfpathlineto{\pgfqpoint{1.447410in}{1.292350in}}%
\pgfpathlineto{\pgfqpoint{1.463085in}{1.290703in}}%
\pgfpathlineto{\pgfqpoint{1.478662in}{1.288809in}}%
\pgfpathlineto{\pgfqpoint{1.477877in}{1.287226in}}%
\pgfpathlineto{\pgfqpoint{1.477093in}{1.285856in}}%
\pgfpathlineto{\pgfqpoint{1.476310in}{1.284698in}}%
\pgfpathlineto{\pgfqpoint{1.475528in}{1.283745in}}%
\pgfpathlineto{\pgfqpoint{1.460335in}{1.285594in}}%
\pgfpathlineto{\pgfqpoint{1.445046in}{1.287202in}}%
\pgfpathlineto{\pgfqpoint{1.429674in}{1.288566in}}%
\pgfpathlineto{\pgfqpoint{1.414233in}{1.289687in}}%
\pgfpathclose%
\pgfusepath{fill}%
\end{pgfscope}%
\begin{pgfscope}%
\pgfpathrectangle{\pgfqpoint{0.329460in}{0.284240in}}{\pgfqpoint{1.989680in}{1.989680in}}%
\pgfusepath{clip}%
\pgfsetbuttcap%
\pgfsetroundjoin%
\definecolor{currentfill}{rgb}{0.267004,0.004874,0.329415}%
\pgfsetfillcolor{currentfill}%
\pgfsetlinewidth{0.000000pt}%
\definecolor{currentstroke}{rgb}{0.000000,0.000000,0.000000}%
\pgfsetstrokecolor{currentstroke}%
\pgfsetdash{}{0pt}%
\pgfpathmoveto{\pgfqpoint{1.478662in}{1.288809in}}%
\pgfpathlineto{\pgfqpoint{1.479448in}{1.290610in}}%
\pgfpathlineto{\pgfqpoint{1.480236in}{1.292634in}}%
\pgfpathlineto{\pgfqpoint{1.481024in}{1.294885in}}%
\pgfpathlineto{\pgfqpoint{1.481814in}{1.297366in}}%
\pgfpathlineto{\pgfqpoint{1.497662in}{1.295175in}}%
\pgfpathlineto{\pgfqpoint{1.513382in}{1.292734in}}%
\pgfpathlineto{\pgfqpoint{1.528960in}{1.290044in}}%
\pgfpathlineto{\pgfqpoint{1.527884in}{1.287604in}}%
\pgfpathlineto{\pgfqpoint{1.526809in}{1.285395in}}%
\pgfpathlineto{\pgfqpoint{1.525736in}{1.283413in}}%
\pgfpathlineto{\pgfqpoint{1.524665in}{1.281654in}}%
\pgfpathlineto{\pgfqpoint{1.509466in}{1.284282in}}%
\pgfpathlineto{\pgfqpoint{1.494127in}{1.286668in}}%
\pgfpathlineto{\pgfqpoint{1.478662in}{1.288809in}}%
\pgfpathclose%
\pgfusepath{fill}%
\end{pgfscope}%
\begin{pgfscope}%
\pgfpathrectangle{\pgfqpoint{0.329460in}{0.284240in}}{\pgfqpoint{1.989680in}{1.989680in}}%
\pgfusepath{clip}%
\pgfsetbuttcap%
\pgfsetroundjoin%
\definecolor{currentfill}{rgb}{0.282327,0.094955,0.417331}%
\pgfsetfillcolor{currentfill}%
\pgfsetlinewidth{0.000000pt}%
\definecolor{currentstroke}{rgb}{0.000000,0.000000,0.000000}%
\pgfsetstrokecolor{currentstroke}%
\pgfsetdash{}{0pt}%
\pgfpathmoveto{\pgfqpoint{1.542044in}{1.338933in}}%
\pgfpathlineto{\pgfqpoint{1.543151in}{1.344780in}}%
\pgfpathlineto{\pgfqpoint{1.544261in}{1.350921in}}%
\pgfpathlineto{\pgfqpoint{1.545374in}{1.357363in}}%
\pgfpathlineto{\pgfqpoint{1.546490in}{1.364109in}}%
\pgfpathlineto{\pgfqpoint{1.563442in}{1.360921in}}%
\pgfpathlineto{\pgfqpoint{1.580210in}{1.357468in}}%
\pgfpathlineto{\pgfqpoint{1.596778in}{1.353751in}}%
\pgfpathlineto{\pgfqpoint{1.613133in}{1.349774in}}%
\pgfpathlineto{\pgfqpoint{1.611632in}{1.343094in}}%
\pgfpathlineto{\pgfqpoint{1.610135in}{1.336720in}}%
\pgfpathlineto{\pgfqpoint{1.608643in}{1.330646in}}%
\pgfpathlineto{\pgfqpoint{1.607155in}{1.324868in}}%
\pgfpathlineto{\pgfqpoint{1.591178in}{1.328770in}}%
\pgfpathlineto{\pgfqpoint{1.574990in}{1.332416in}}%
\pgfpathlineto{\pgfqpoint{1.558607in}{1.335805in}}%
\pgfpathlineto{\pgfqpoint{1.542044in}{1.338933in}}%
\pgfpathclose%
\pgfusepath{fill}%
\end{pgfscope}%
\begin{pgfscope}%
\pgfpathrectangle{\pgfqpoint{0.329460in}{0.284240in}}{\pgfqpoint{1.989680in}{1.989680in}}%
\pgfusepath{clip}%
\pgfsetbuttcap%
\pgfsetroundjoin%
\definecolor{currentfill}{rgb}{0.277941,0.056324,0.381191}%
\pgfsetfillcolor{currentfill}%
\pgfsetlinewidth{0.000000pt}%
\definecolor{currentstroke}{rgb}{0.000000,0.000000,0.000000}%
\pgfsetstrokecolor{currentstroke}%
\pgfsetdash{}{0pt}%
\pgfpathmoveto{\pgfqpoint{1.537644in}{1.318398in}}%
\pgfpathlineto{\pgfqpoint{1.538740in}{1.323114in}}%
\pgfpathlineto{\pgfqpoint{1.539838in}{1.328105in}}%
\pgfpathlineto{\pgfqpoint{1.540940in}{1.333376in}}%
\pgfpathlineto{\pgfqpoint{1.542044in}{1.338933in}}%
\pgfpathlineto{\pgfqpoint{1.558607in}{1.335805in}}%
\pgfpathlineto{\pgfqpoint{1.574990in}{1.332416in}}%
\pgfpathlineto{\pgfqpoint{1.591178in}{1.328770in}}%
\pgfpathlineto{\pgfqpoint{1.607155in}{1.324868in}}%
\pgfpathlineto{\pgfqpoint{1.605671in}{1.319380in}}%
\pgfpathlineto{\pgfqpoint{1.604191in}{1.314178in}}%
\pgfpathlineto{\pgfqpoint{1.602714in}{1.309257in}}%
\pgfpathlineto{\pgfqpoint{1.601241in}{1.304612in}}%
\pgfpathlineto{\pgfqpoint{1.585636in}{1.308436in}}%
\pgfpathlineto{\pgfqpoint{1.569825in}{1.312011in}}%
\pgfpathlineto{\pgfqpoint{1.553823in}{1.315332in}}%
\pgfpathlineto{\pgfqpoint{1.537644in}{1.318398in}}%
\pgfpathclose%
\pgfusepath{fill}%
\end{pgfscope}%
\begin{pgfscope}%
\pgfpathrectangle{\pgfqpoint{0.329460in}{0.284240in}}{\pgfqpoint{1.989680in}{1.989680in}}%
\pgfusepath{clip}%
\pgfsetbuttcap%
\pgfsetroundjoin%
\definecolor{currentfill}{rgb}{0.267004,0.004874,0.329415}%
\pgfsetfillcolor{currentfill}%
\pgfsetlinewidth{0.000000pt}%
\definecolor{currentstroke}{rgb}{0.000000,0.000000,0.000000}%
\pgfsetstrokecolor{currentstroke}%
\pgfsetdash{}{0pt}%
\pgfpathmoveto{\pgfqpoint{1.164327in}{1.279115in}}%
\pgfpathlineto{\pgfqpoint{1.163173in}{1.280859in}}%
\pgfpathlineto{\pgfqpoint{1.162017in}{1.282827in}}%
\pgfpathlineto{\pgfqpoint{1.160859in}{1.285021in}}%
\pgfpathlineto{\pgfqpoint{1.159699in}{1.287446in}}%
\pgfpathlineto{\pgfqpoint{1.175138in}{1.290355in}}%
\pgfpathlineto{\pgfqpoint{1.190733in}{1.293018in}}%
\pgfpathlineto{\pgfqpoint{1.206467in}{1.295431in}}%
\pgfpathlineto{\pgfqpoint{1.222329in}{1.297594in}}%
\pgfpathlineto{\pgfqpoint{1.223108in}{1.295112in}}%
\pgfpathlineto{\pgfqpoint{1.223886in}{1.292860in}}%
\pgfpathlineto{\pgfqpoint{1.224663in}{1.290834in}}%
\pgfpathlineto{\pgfqpoint{1.225439in}{1.289032in}}%
\pgfpathlineto{\pgfqpoint{1.209961in}{1.286918in}}%
\pgfpathlineto{\pgfqpoint{1.194607in}{1.284559in}}%
\pgfpathlineto{\pgfqpoint{1.179391in}{1.281958in}}%
\pgfpathlineto{\pgfqpoint{1.164327in}{1.279115in}}%
\pgfpathclose%
\pgfusepath{fill}%
\end{pgfscope}%
\begin{pgfscope}%
\pgfpathrectangle{\pgfqpoint{0.329460in}{0.284240in}}{\pgfqpoint{1.989680in}{1.989680in}}%
\pgfusepath{clip}%
\pgfsetbuttcap%
\pgfsetroundjoin%
\definecolor{currentfill}{rgb}{0.282884,0.135920,0.453427}%
\pgfsetfillcolor{currentfill}%
\pgfsetlinewidth{0.000000pt}%
\definecolor{currentstroke}{rgb}{0.000000,0.000000,0.000000}%
\pgfsetstrokecolor{currentstroke}%
\pgfsetdash{}{0pt}%
\pgfpathmoveto{\pgfqpoint{1.546490in}{1.364109in}}%
\pgfpathlineto{\pgfqpoint{1.547610in}{1.371166in}}%
\pgfpathlineto{\pgfqpoint{1.548733in}{1.378537in}}%
\pgfpathlineto{\pgfqpoint{1.549860in}{1.386229in}}%
\pgfpathlineto{\pgfqpoint{1.550991in}{1.394247in}}%
\pgfpathlineto{\pgfqpoint{1.568336in}{1.391002in}}%
\pgfpathlineto{\pgfqpoint{1.585493in}{1.387486in}}%
\pgfpathlineto{\pgfqpoint{1.602447in}{1.383702in}}%
\pgfpathlineto{\pgfqpoint{1.619183in}{1.379653in}}%
\pgfpathlineto{\pgfqpoint{1.617663in}{1.371699in}}%
\pgfpathlineto{\pgfqpoint{1.616148in}{1.364071in}}%
\pgfpathlineto{\pgfqpoint{1.614638in}{1.356765in}}%
\pgfpathlineto{\pgfqpoint{1.613133in}{1.349774in}}%
\pgfpathlineto{\pgfqpoint{1.596778in}{1.353751in}}%
\pgfpathlineto{\pgfqpoint{1.580210in}{1.357468in}}%
\pgfpathlineto{\pgfqpoint{1.563442in}{1.360921in}}%
\pgfpathlineto{\pgfqpoint{1.546490in}{1.364109in}}%
\pgfpathclose%
\pgfusepath{fill}%
\end{pgfscope}%
\begin{pgfscope}%
\pgfpathrectangle{\pgfqpoint{0.329460in}{0.284240in}}{\pgfqpoint{1.989680in}{1.989680in}}%
\pgfusepath{clip}%
\pgfsetbuttcap%
\pgfsetroundjoin%
\definecolor{currentfill}{rgb}{0.272594,0.025563,0.353093}%
\pgfsetfillcolor{currentfill}%
\pgfsetlinewidth{0.000000pt}%
\definecolor{currentstroke}{rgb}{0.000000,0.000000,0.000000}%
\pgfsetstrokecolor{currentstroke}%
\pgfsetdash{}{0pt}%
\pgfpathmoveto{\pgfqpoint{1.533284in}{1.302199in}}%
\pgfpathlineto{\pgfqpoint{1.534371in}{1.305859in}}%
\pgfpathlineto{\pgfqpoint{1.535459in}{1.309775in}}%
\pgfpathlineto{\pgfqpoint{1.536550in}{1.313954in}}%
\pgfpathlineto{\pgfqpoint{1.537644in}{1.318398in}}%
\pgfpathlineto{\pgfqpoint{1.553823in}{1.315332in}}%
\pgfpathlineto{\pgfqpoint{1.569825in}{1.312011in}}%
\pgfpathlineto{\pgfqpoint{1.585636in}{1.308436in}}%
\pgfpathlineto{\pgfqpoint{1.601241in}{1.304612in}}%
\pgfpathlineto{\pgfqpoint{1.599771in}{1.300239in}}%
\pgfpathlineto{\pgfqpoint{1.598305in}{1.296132in}}%
\pgfpathlineto{\pgfqpoint{1.596842in}{1.292288in}}%
\pgfpathlineto{\pgfqpoint{1.595382in}{1.288701in}}%
\pgfpathlineto{\pgfqpoint{1.580146in}{1.292445in}}%
\pgfpathlineto{\pgfqpoint{1.564708in}{1.295945in}}%
\pgfpathlineto{\pgfqpoint{1.549083in}{1.299197in}}%
\pgfpathlineto{\pgfqpoint{1.533284in}{1.302199in}}%
\pgfpathclose%
\pgfusepath{fill}%
\end{pgfscope}%
\begin{pgfscope}%
\pgfpathrectangle{\pgfqpoint{0.329460in}{0.284240in}}{\pgfqpoint{1.989680in}{1.989680in}}%
\pgfusepath{clip}%
\pgfsetbuttcap%
\pgfsetroundjoin%
\definecolor{currentfill}{rgb}{0.268510,0.009605,0.335427}%
\pgfsetfillcolor{currentfill}%
\pgfsetlinewidth{0.000000pt}%
\definecolor{currentstroke}{rgb}{0.000000,0.000000,0.000000}%
\pgfsetstrokecolor{currentstroke}%
\pgfsetdash{}{0pt}%
\pgfpathmoveto{\pgfqpoint{1.291398in}{1.287818in}}%
\pgfpathlineto{\pgfqpoint{1.291015in}{1.288023in}}%
\pgfpathlineto{\pgfqpoint{1.290630in}{1.288418in}}%
\pgfpathlineto{\pgfqpoint{1.290246in}{1.289008in}}%
\pgfpathlineto{\pgfqpoint{1.289861in}{1.289796in}}%
\pgfpathlineto{\pgfqpoint{1.305362in}{1.290644in}}%
\pgfpathlineto{\pgfqpoint{1.320904in}{1.291246in}}%
\pgfpathlineto{\pgfqpoint{1.336472in}{1.291602in}}%
\pgfpathlineto{\pgfqpoint{1.352053in}{1.291712in}}%
\pgfpathlineto{\pgfqpoint{1.352048in}{1.290912in}}%
\pgfpathlineto{\pgfqpoint{1.352042in}{1.290311in}}%
\pgfpathlineto{\pgfqpoint{1.352037in}{1.289903in}}%
\pgfpathlineto{\pgfqpoint{1.352032in}{1.289686in}}%
\pgfpathlineto{\pgfqpoint{1.336841in}{1.289579in}}%
\pgfpathlineto{\pgfqpoint{1.321663in}{1.289232in}}%
\pgfpathlineto{\pgfqpoint{1.306511in}{1.288645in}}%
\pgfpathlineto{\pgfqpoint{1.291398in}{1.287818in}}%
\pgfpathclose%
\pgfusepath{fill}%
\end{pgfscope}%
\begin{pgfscope}%
\pgfpathrectangle{\pgfqpoint{0.329460in}{0.284240in}}{\pgfqpoint{1.989680in}{1.989680in}}%
\pgfusepath{clip}%
\pgfsetbuttcap%
\pgfsetroundjoin%
\definecolor{currentfill}{rgb}{0.268510,0.009605,0.335427}%
\pgfsetfillcolor{currentfill}%
\pgfsetlinewidth{0.000000pt}%
\definecolor{currentstroke}{rgb}{0.000000,0.000000,0.000000}%
\pgfsetstrokecolor{currentstroke}%
\pgfsetdash{}{0pt}%
\pgfpathmoveto{\pgfqpoint{1.352032in}{1.289686in}}%
\pgfpathlineto{\pgfqpoint{1.352037in}{1.289903in}}%
\pgfpathlineto{\pgfqpoint{1.352042in}{1.290311in}}%
\pgfpathlineto{\pgfqpoint{1.352048in}{1.290912in}}%
\pgfpathlineto{\pgfqpoint{1.352053in}{1.291712in}}%
\pgfpathlineto{\pgfqpoint{1.367634in}{1.291575in}}%
\pgfpathlineto{\pgfqpoint{1.383200in}{1.291191in}}%
\pgfpathlineto{\pgfqpoint{1.398737in}{1.290562in}}%
\pgfpathlineto{\pgfqpoint{1.414233in}{1.289687in}}%
\pgfpathlineto{\pgfqpoint{1.413838in}{1.288899in}}%
\pgfpathlineto{\pgfqpoint{1.413442in}{1.288310in}}%
\pgfpathlineto{\pgfqpoint{1.413047in}{1.287916in}}%
\pgfpathlineto{\pgfqpoint{1.412653in}{1.287711in}}%
\pgfpathlineto{\pgfqpoint{1.397546in}{1.288565in}}%
\pgfpathlineto{\pgfqpoint{1.382397in}{1.289179in}}%
\pgfpathlineto{\pgfqpoint{1.367221in}{1.289553in}}%
\pgfpathlineto{\pgfqpoint{1.352032in}{1.289686in}}%
\pgfpathclose%
\pgfusepath{fill}%
\end{pgfscope}%
\begin{pgfscope}%
\pgfpathrectangle{\pgfqpoint{0.329460in}{0.284240in}}{\pgfqpoint{1.989680in}{1.989680in}}%
\pgfusepath{clip}%
\pgfsetbuttcap%
\pgfsetroundjoin%
\definecolor{currentfill}{rgb}{0.282327,0.094955,0.417331}%
\pgfsetfillcolor{currentfill}%
\pgfsetlinewidth{0.000000pt}%
\definecolor{currentstroke}{rgb}{0.000000,0.000000,0.000000}%
\pgfsetstrokecolor{currentstroke}%
\pgfsetdash{}{0pt}%
\pgfpathmoveto{\pgfqpoint{1.081206in}{1.321187in}}%
\pgfpathlineto{\pgfqpoint{1.079635in}{1.326948in}}%
\pgfpathlineto{\pgfqpoint{1.078060in}{1.333004in}}%
\pgfpathlineto{\pgfqpoint{1.076481in}{1.339360in}}%
\pgfpathlineto{\pgfqpoint{1.074897in}{1.346023in}}%
\pgfpathlineto{\pgfqpoint{1.091049in}{1.350229in}}%
\pgfpathlineto{\pgfqpoint{1.107427in}{1.354177in}}%
\pgfpathlineto{\pgfqpoint{1.124019in}{1.357864in}}%
\pgfpathlineto{\pgfqpoint{1.140807in}{1.361289in}}%
\pgfpathlineto{\pgfqpoint{1.142011in}{1.354555in}}%
\pgfpathlineto{\pgfqpoint{1.143210in}{1.348127in}}%
\pgfpathlineto{\pgfqpoint{1.144407in}{1.341999in}}%
\pgfpathlineto{\pgfqpoint{1.145600in}{1.336165in}}%
\pgfpathlineto{\pgfqpoint{1.129196in}{1.332806in}}%
\pgfpathlineto{\pgfqpoint{1.112986in}{1.329187in}}%
\pgfpathlineto{\pgfqpoint{1.096985in}{1.325314in}}%
\pgfpathlineto{\pgfqpoint{1.081206in}{1.321187in}}%
\pgfpathclose%
\pgfusepath{fill}%
\end{pgfscope}%
\begin{pgfscope}%
\pgfpathrectangle{\pgfqpoint{0.329460in}{0.284240in}}{\pgfqpoint{1.989680in}{1.989680in}}%
\pgfusepath{clip}%
\pgfsetbuttcap%
\pgfsetroundjoin%
\definecolor{currentfill}{rgb}{0.201239,0.383670,0.554294}%
\pgfsetfillcolor{currentfill}%
\pgfsetlinewidth{0.000000pt}%
\definecolor{currentstroke}{rgb}{0.000000,0.000000,0.000000}%
\pgfsetstrokecolor{currentstroke}%
\pgfsetdash{}{0pt}%
\pgfpathmoveto{\pgfqpoint{1.196328in}{1.526644in}}%
\pgfpathlineto{\pgfqpoint{1.195468in}{1.539444in}}%
\pgfpathlineto{\pgfqpoint{1.194604in}{1.552652in}}%
\pgfpathlineto{\pgfqpoint{1.193736in}{1.566275in}}%
\pgfpathlineto{\pgfqpoint{1.192864in}{1.580318in}}%
\pgfpathlineto{\pgfqpoint{1.212501in}{1.582559in}}%
\pgfpathlineto{\pgfqpoint{1.232255in}{1.584502in}}%
\pgfpathlineto{\pgfqpoint{1.252111in}{1.586145in}}%
\pgfpathlineto{\pgfqpoint{1.272050in}{1.587488in}}%
\pgfpathlineto{\pgfqpoint{1.272485in}{1.573422in}}%
\pgfpathlineto{\pgfqpoint{1.272919in}{1.559777in}}%
\pgfpathlineto{\pgfqpoint{1.273350in}{1.546545in}}%
\pgfpathlineto{\pgfqpoint{1.273780in}{1.533721in}}%
\pgfpathlineto{\pgfqpoint{1.254277in}{1.532395in}}%
\pgfpathlineto{\pgfqpoint{1.234856in}{1.530773in}}%
\pgfpathlineto{\pgfqpoint{1.215535in}{1.528856in}}%
\pgfpathlineto{\pgfqpoint{1.196328in}{1.526644in}}%
\pgfpathclose%
\pgfusepath{fill}%
\end{pgfscope}%
\begin{pgfscope}%
\pgfpathrectangle{\pgfqpoint{0.329460in}{0.284240in}}{\pgfqpoint{1.989680in}{1.989680in}}%
\pgfusepath{clip}%
\pgfsetbuttcap%
\pgfsetroundjoin%
\definecolor{currentfill}{rgb}{0.233603,0.313828,0.543914}%
\pgfsetfillcolor{currentfill}%
\pgfsetlinewidth{0.000000pt}%
\definecolor{currentstroke}{rgb}{0.000000,0.000000,0.000000}%
\pgfsetstrokecolor{currentstroke}%
\pgfsetdash{}{0pt}%
\pgfpathmoveto{\pgfqpoint{1.504722in}{1.479134in}}%
\pgfpathlineto{\pgfqpoint{1.505579in}{1.490365in}}%
\pgfpathlineto{\pgfqpoint{1.506440in}{1.501980in}}%
\pgfpathlineto{\pgfqpoint{1.507305in}{1.513983in}}%
\pgfpathlineto{\pgfqpoint{1.508173in}{1.526380in}}%
\pgfpathlineto{\pgfqpoint{1.527232in}{1.523844in}}%
\pgfpathlineto{\pgfqpoint{1.546141in}{1.521018in}}%
\pgfpathlineto{\pgfqpoint{1.564883in}{1.517903in}}%
\pgfpathlineto{\pgfqpoint{1.563700in}{1.505534in}}%
\pgfpathlineto{\pgfqpoint{1.562521in}{1.493561in}}%
\pgfpathlineto{\pgfqpoint{1.561348in}{1.481978in}}%
\pgfpathlineto{\pgfqpoint{1.560179in}{1.470777in}}%
\pgfpathlineto{\pgfqpoint{1.541851in}{1.473848in}}%
\pgfpathlineto{\pgfqpoint{1.523360in}{1.476634in}}%
\pgfpathlineto{\pgfqpoint{1.504722in}{1.479134in}}%
\pgfpathclose%
\pgfusepath{fill}%
\end{pgfscope}%
\begin{pgfscope}%
\pgfpathrectangle{\pgfqpoint{0.329460in}{0.284240in}}{\pgfqpoint{1.989680in}{1.989680in}}%
\pgfusepath{clip}%
\pgfsetbuttcap%
\pgfsetroundjoin%
\definecolor{currentfill}{rgb}{0.277941,0.056324,0.381191}%
\pgfsetfillcolor{currentfill}%
\pgfsetlinewidth{0.000000pt}%
\definecolor{currentstroke}{rgb}{0.000000,0.000000,0.000000}%
\pgfsetstrokecolor{currentstroke}%
\pgfsetdash{}{0pt}%
\pgfpathmoveto{\pgfqpoint{1.087447in}{1.301005in}}%
\pgfpathlineto{\pgfqpoint{1.085893in}{1.305631in}}%
\pgfpathlineto{\pgfqpoint{1.084334in}{1.310534in}}%
\pgfpathlineto{\pgfqpoint{1.082772in}{1.315718in}}%
\pgfpathlineto{\pgfqpoint{1.081206in}{1.321187in}}%
\pgfpathlineto{\pgfqpoint{1.096985in}{1.325314in}}%
\pgfpathlineto{\pgfqpoint{1.112986in}{1.329187in}}%
\pgfpathlineto{\pgfqpoint{1.129196in}{1.332806in}}%
\pgfpathlineto{\pgfqpoint{1.145600in}{1.336165in}}%
\pgfpathlineto{\pgfqpoint{1.146789in}{1.330622in}}%
\pgfpathlineto{\pgfqpoint{1.147976in}{1.325364in}}%
\pgfpathlineto{\pgfqpoint{1.149160in}{1.320387in}}%
\pgfpathlineto{\pgfqpoint{1.150341in}{1.315685in}}%
\pgfpathlineto{\pgfqpoint{1.134319in}{1.312392in}}%
\pgfpathlineto{\pgfqpoint{1.118486in}{1.308846in}}%
\pgfpathlineto{\pgfqpoint{1.102857in}{1.305049in}}%
\pgfpathlineto{\pgfqpoint{1.087447in}{1.301005in}}%
\pgfpathclose%
\pgfusepath{fill}%
\end{pgfscope}%
\begin{pgfscope}%
\pgfpathrectangle{\pgfqpoint{0.329460in}{0.284240in}}{\pgfqpoint{1.989680in}{1.989680in}}%
\pgfusepath{clip}%
\pgfsetbuttcap%
\pgfsetroundjoin%
\definecolor{currentfill}{rgb}{0.201239,0.383670,0.554294}%
\pgfsetfillcolor{currentfill}%
\pgfsetlinewidth{0.000000pt}%
\definecolor{currentstroke}{rgb}{0.000000,0.000000,0.000000}%
\pgfsetstrokecolor{currentstroke}%
\pgfsetdash{}{0pt}%
\pgfpathmoveto{\pgfqpoint{1.430766in}{1.533588in}}%
\pgfpathlineto{\pgfqpoint{1.431208in}{1.546412in}}%
\pgfpathlineto{\pgfqpoint{1.431651in}{1.559643in}}%
\pgfpathlineto{\pgfqpoint{1.432097in}{1.573288in}}%
\pgfpathlineto{\pgfqpoint{1.432545in}{1.587354in}}%
\pgfpathlineto{\pgfqpoint{1.452475in}{1.585977in}}%
\pgfpathlineto{\pgfqpoint{1.472320in}{1.584300in}}%
\pgfpathlineto{\pgfqpoint{1.492062in}{1.582325in}}%
\pgfpathlineto{\pgfqpoint{1.511686in}{1.580051in}}%
\pgfpathlineto{\pgfqpoint{1.510801in}{1.566008in}}%
\pgfpathlineto{\pgfqpoint{1.509921in}{1.552387in}}%
\pgfpathlineto{\pgfqpoint{1.509045in}{1.539180in}}%
\pgfpathlineto{\pgfqpoint{1.508173in}{1.526380in}}%
\pgfpathlineto{\pgfqpoint{1.488980in}{1.528624in}}%
\pgfpathlineto{\pgfqpoint{1.469671in}{1.530575in}}%
\pgfpathlineto{\pgfqpoint{1.450260in}{1.532230in}}%
\pgfpathlineto{\pgfqpoint{1.430766in}{1.533588in}}%
\pgfpathclose%
\pgfusepath{fill}%
\end{pgfscope}%
\begin{pgfscope}%
\pgfpathrectangle{\pgfqpoint{0.329460in}{0.284240in}}{\pgfqpoint{1.989680in}{1.989680in}}%
\pgfusepath{clip}%
\pgfsetbuttcap%
\pgfsetroundjoin%
\definecolor{currentfill}{rgb}{0.276194,0.190074,0.493001}%
\pgfsetfillcolor{currentfill}%
\pgfsetlinewidth{0.000000pt}%
\definecolor{currentstroke}{rgb}{0.000000,0.000000,0.000000}%
\pgfsetstrokecolor{currentstroke}%
\pgfsetdash{}{0pt}%
\pgfpathmoveto{\pgfqpoint{1.550991in}{1.394247in}}%
\pgfpathlineto{\pgfqpoint{1.552125in}{1.402596in}}%
\pgfpathlineto{\pgfqpoint{1.553263in}{1.411281in}}%
\pgfpathlineto{\pgfqpoint{1.554405in}{1.420309in}}%
\pgfpathlineto{\pgfqpoint{1.555551in}{1.429684in}}%
\pgfpathlineto{\pgfqpoint{1.573295in}{1.426384in}}%
\pgfpathlineto{\pgfqpoint{1.590847in}{1.422808in}}%
\pgfpathlineto{\pgfqpoint{1.608192in}{1.418960in}}%
\pgfpathlineto{\pgfqpoint{1.625314in}{1.414842in}}%
\pgfpathlineto{\pgfqpoint{1.623773in}{1.405528in}}%
\pgfpathlineto{\pgfqpoint{1.622238in}{1.396562in}}%
\pgfpathlineto{\pgfqpoint{1.620708in}{1.387939in}}%
\pgfpathlineto{\pgfqpoint{1.619183in}{1.379653in}}%
\pgfpathlineto{\pgfqpoint{1.602447in}{1.383702in}}%
\pgfpathlineto{\pgfqpoint{1.585493in}{1.387486in}}%
\pgfpathlineto{\pgfqpoint{1.568336in}{1.391002in}}%
\pgfpathlineto{\pgfqpoint{1.550991in}{1.394247in}}%
\pgfpathclose%
\pgfusepath{fill}%
\end{pgfscope}%
\begin{pgfscope}%
\pgfpathrectangle{\pgfqpoint{0.329460in}{0.284240in}}{\pgfqpoint{1.989680in}{1.989680in}}%
\pgfusepath{clip}%
\pgfsetbuttcap%
\pgfsetroundjoin%
\definecolor{currentfill}{rgb}{0.282884,0.135920,0.453427}%
\pgfsetfillcolor{currentfill}%
\pgfsetlinewidth{0.000000pt}%
\definecolor{currentstroke}{rgb}{0.000000,0.000000,0.000000}%
\pgfsetstrokecolor{currentstroke}%
\pgfsetdash{}{0pt}%
\pgfpathmoveto{\pgfqpoint{1.074897in}{1.346023in}}%
\pgfpathlineto{\pgfqpoint{1.073308in}{1.352996in}}%
\pgfpathlineto{\pgfqpoint{1.071714in}{1.360285in}}%
\pgfpathlineto{\pgfqpoint{1.070116in}{1.367896in}}%
\pgfpathlineto{\pgfqpoint{1.068512in}{1.375833in}}%
\pgfpathlineto{\pgfqpoint{1.085041in}{1.380116in}}%
\pgfpathlineto{\pgfqpoint{1.101801in}{1.384135in}}%
\pgfpathlineto{\pgfqpoint{1.118779in}{1.387889in}}%
\pgfpathlineto{\pgfqpoint{1.135958in}{1.391376in}}%
\pgfpathlineto{\pgfqpoint{1.137176in}{1.383370in}}%
\pgfpathlineto{\pgfqpoint{1.138390in}{1.375691in}}%
\pgfpathlineto{\pgfqpoint{1.139601in}{1.368332in}}%
\pgfpathlineto{\pgfqpoint{1.140807in}{1.361289in}}%
\pgfpathlineto{\pgfqpoint{1.124019in}{1.357864in}}%
\pgfpathlineto{\pgfqpoint{1.107427in}{1.354177in}}%
\pgfpathlineto{\pgfqpoint{1.091049in}{1.350229in}}%
\pgfpathlineto{\pgfqpoint{1.074897in}{1.346023in}}%
\pgfpathclose%
\pgfusepath{fill}%
\end{pgfscope}%
\begin{pgfscope}%
\pgfpathrectangle{\pgfqpoint{0.329460in}{0.284240in}}{\pgfqpoint{1.989680in}{1.989680in}}%
\pgfusepath{clip}%
\pgfsetbuttcap%
\pgfsetroundjoin%
\definecolor{currentfill}{rgb}{0.233603,0.313828,0.543914}%
\pgfsetfillcolor{currentfill}%
\pgfsetlinewidth{0.000000pt}%
\definecolor{currentstroke}{rgb}{0.000000,0.000000,0.000000}%
\pgfsetstrokecolor{currentstroke}%
\pgfsetdash{}{0pt}%
\pgfpathmoveto{\pgfqpoint{1.126055in}{1.467811in}}%
\pgfpathlineto{\pgfqpoint{1.124795in}{1.479001in}}%
\pgfpathlineto{\pgfqpoint{1.123531in}{1.490573in}}%
\pgfpathlineto{\pgfqpoint{1.122261in}{1.502536in}}%
\pgfpathlineto{\pgfqpoint{1.120985in}{1.514894in}}%
\pgfpathlineto{\pgfqpoint{1.139566in}{1.518263in}}%
\pgfpathlineto{\pgfqpoint{1.158328in}{1.521346in}}%
\pgfpathlineto{\pgfqpoint{1.177254in}{1.524140in}}%
\pgfpathlineto{\pgfqpoint{1.196328in}{1.526644in}}%
\pgfpathlineto{\pgfqpoint{1.197185in}{1.514245in}}%
\pgfpathlineto{\pgfqpoint{1.198038in}{1.502242in}}%
\pgfpathlineto{\pgfqpoint{1.198887in}{1.490626in}}%
\pgfpathlineto{\pgfqpoint{1.199732in}{1.479394in}}%
\pgfpathlineto{\pgfqpoint{1.181079in}{1.476926in}}%
\pgfpathlineto{\pgfqpoint{1.162571in}{1.474171in}}%
\pgfpathlineto{\pgfqpoint{1.144224in}{1.471132in}}%
\pgfpathlineto{\pgfqpoint{1.126055in}{1.467811in}}%
\pgfpathclose%
\pgfusepath{fill}%
\end{pgfscope}%
\begin{pgfscope}%
\pgfpathrectangle{\pgfqpoint{0.329460in}{0.284240in}}{\pgfqpoint{1.989680in}{1.989680in}}%
\pgfusepath{clip}%
\pgfsetbuttcap%
\pgfsetroundjoin%
\definecolor{currentfill}{rgb}{0.272594,0.025563,0.353093}%
\pgfsetfillcolor{currentfill}%
\pgfsetlinewidth{0.000000pt}%
\definecolor{currentstroke}{rgb}{0.000000,0.000000,0.000000}%
\pgfsetstrokecolor{currentstroke}%
\pgfsetdash{}{0pt}%
\pgfpathmoveto{\pgfqpoint{1.093630in}{1.285169in}}%
\pgfpathlineto{\pgfqpoint{1.092090in}{1.288737in}}%
\pgfpathlineto{\pgfqpoint{1.090546in}{1.292563in}}%
\pgfpathlineto{\pgfqpoint{1.088998in}{1.296650in}}%
\pgfpathlineto{\pgfqpoint{1.087447in}{1.301005in}}%
\pgfpathlineto{\pgfqpoint{1.102857in}{1.305049in}}%
\pgfpathlineto{\pgfqpoint{1.118486in}{1.308846in}}%
\pgfpathlineto{\pgfqpoint{1.134319in}{1.312392in}}%
\pgfpathlineto{\pgfqpoint{1.150341in}{1.315685in}}%
\pgfpathlineto{\pgfqpoint{1.151520in}{1.311255in}}%
\pgfpathlineto{\pgfqpoint{1.152695in}{1.307090in}}%
\pgfpathlineto{\pgfqpoint{1.153868in}{1.303188in}}%
\pgfpathlineto{\pgfqpoint{1.155039in}{1.299543in}}%
\pgfpathlineto{\pgfqpoint{1.139394in}{1.296318in}}%
\pgfpathlineto{\pgfqpoint{1.123935in}{1.292846in}}%
\pgfpathlineto{\pgfqpoint{1.108676in}{1.289129in}}%
\pgfpathlineto{\pgfqpoint{1.093630in}{1.285169in}}%
\pgfpathclose%
\pgfusepath{fill}%
\end{pgfscope}%
\begin{pgfscope}%
\pgfpathrectangle{\pgfqpoint{0.329460in}{0.284240in}}{\pgfqpoint{1.989680in}{1.989680in}}%
\pgfusepath{clip}%
\pgfsetbuttcap%
\pgfsetroundjoin%
\definecolor{currentfill}{rgb}{0.268510,0.009605,0.335427}%
\pgfsetfillcolor{currentfill}%
\pgfsetlinewidth{0.000000pt}%
\definecolor{currentstroke}{rgb}{0.000000,0.000000,0.000000}%
\pgfsetstrokecolor{currentstroke}%
\pgfsetdash{}{0pt}%
\pgfpathmoveto{\pgfqpoint{1.528960in}{1.290044in}}%
\pgfpathlineto{\pgfqpoint{1.530038in}{1.292719in}}%
\pgfpathlineto{\pgfqpoint{1.531118in}{1.295634in}}%
\pgfpathlineto{\pgfqpoint{1.532200in}{1.298792in}}%
\pgfpathlineto{\pgfqpoint{1.533284in}{1.302199in}}%
\pgfpathlineto{\pgfqpoint{1.549083in}{1.299197in}}%
\pgfpathlineto{\pgfqpoint{1.564708in}{1.295945in}}%
\pgfpathlineto{\pgfqpoint{1.580146in}{1.292445in}}%
\pgfpathlineto{\pgfqpoint{1.595382in}{1.288701in}}%
\pgfpathlineto{\pgfqpoint{1.593925in}{1.285367in}}%
\pgfpathlineto{\pgfqpoint{1.592471in}{1.282282in}}%
\pgfpathlineto{\pgfqpoint{1.591019in}{1.279442in}}%
\pgfpathlineto{\pgfqpoint{1.589571in}{1.276841in}}%
\pgfpathlineto{\pgfqpoint{1.574700in}{1.280503in}}%
\pgfpathlineto{\pgfqpoint{1.559633in}{1.283926in}}%
\pgfpathlineto{\pgfqpoint{1.544381in}{1.287107in}}%
\pgfpathlineto{\pgfqpoint{1.528960in}{1.290044in}}%
\pgfpathclose%
\pgfusepath{fill}%
\end{pgfscope}%
\begin{pgfscope}%
\pgfpathrectangle{\pgfqpoint{0.329460in}{0.284240in}}{\pgfqpoint{1.989680in}{1.989680in}}%
\pgfusepath{clip}%
\pgfsetbuttcap%
\pgfsetroundjoin%
\definecolor{currentfill}{rgb}{0.267004,0.004874,0.329415}%
\pgfsetfillcolor{currentfill}%
\pgfsetlinewidth{0.000000pt}%
\definecolor{currentstroke}{rgb}{0.000000,0.000000,0.000000}%
\pgfsetstrokecolor{currentstroke}%
\pgfsetdash{}{0pt}%
\pgfpathmoveto{\pgfqpoint{1.475528in}{1.283745in}}%
\pgfpathlineto{\pgfqpoint{1.476310in}{1.284698in}}%
\pgfpathlineto{\pgfqpoint{1.477093in}{1.285856in}}%
\pgfpathlineto{\pgfqpoint{1.477877in}{1.287226in}}%
\pgfpathlineto{\pgfqpoint{1.478662in}{1.288809in}}%
\pgfpathlineto{\pgfqpoint{1.494127in}{1.286668in}}%
\pgfpathlineto{\pgfqpoint{1.509466in}{1.284282in}}%
\pgfpathlineto{\pgfqpoint{1.524665in}{1.281654in}}%
\pgfpathlineto{\pgfqpoint{1.523596in}{1.280113in}}%
\pgfpathlineto{\pgfqpoint{1.522527in}{1.278786in}}%
\pgfpathlineto{\pgfqpoint{1.521461in}{1.277670in}}%
\pgfpathlineto{\pgfqpoint{1.520395in}{1.276760in}}%
\pgfpathlineto{\pgfqpoint{1.505571in}{1.279326in}}%
\pgfpathlineto{\pgfqpoint{1.490611in}{1.281655in}}%
\pgfpathlineto{\pgfqpoint{1.475528in}{1.283745in}}%
\pgfpathclose%
\pgfusepath{fill}%
\end{pgfscope}%
\begin{pgfscope}%
\pgfpathrectangle{\pgfqpoint{0.329460in}{0.284240in}}{\pgfqpoint{1.989680in}{1.989680in}}%
\pgfusepath{clip}%
\pgfsetbuttcap%
\pgfsetroundjoin%
\definecolor{currentfill}{rgb}{0.268510,0.009605,0.335427}%
\pgfsetfillcolor{currentfill}%
\pgfsetlinewidth{0.000000pt}%
\definecolor{currentstroke}{rgb}{0.000000,0.000000,0.000000}%
\pgfsetstrokecolor{currentstroke}%
\pgfsetdash{}{0pt}%
\pgfpathmoveto{\pgfqpoint{1.231608in}{1.282129in}}%
\pgfpathlineto{\pgfqpoint{1.230840in}{1.282297in}}%
\pgfpathlineto{\pgfqpoint{1.230071in}{1.282657in}}%
\pgfpathlineto{\pgfqpoint{1.229301in}{1.283210in}}%
\pgfpathlineto{\pgfqpoint{1.228531in}{1.283962in}}%
\pgfpathlineto{\pgfqpoint{1.243735in}{1.285785in}}%
\pgfpathlineto{\pgfqpoint{1.259034in}{1.287365in}}%
\pgfpathlineto{\pgfqpoint{1.274414in}{1.288703in}}%
\pgfpathlineto{\pgfqpoint{1.289861in}{1.289796in}}%
\pgfpathlineto{\pgfqpoint{1.290246in}{1.289008in}}%
\pgfpathlineto{\pgfqpoint{1.290630in}{1.288418in}}%
\pgfpathlineto{\pgfqpoint{1.291015in}{1.288023in}}%
\pgfpathlineto{\pgfqpoint{1.291398in}{1.287818in}}%
\pgfpathlineto{\pgfqpoint{1.276339in}{1.286752in}}%
\pgfpathlineto{\pgfqpoint{1.261345in}{1.285448in}}%
\pgfpathlineto{\pgfqpoint{1.246430in}{1.283906in}}%
\pgfpathlineto{\pgfqpoint{1.231608in}{1.282129in}}%
\pgfpathclose%
\pgfusepath{fill}%
\end{pgfscope}%
\begin{pgfscope}%
\pgfpathrectangle{\pgfqpoint{0.329460in}{0.284240in}}{\pgfqpoint{1.989680in}{1.989680in}}%
\pgfusepath{clip}%
\pgfsetbuttcap%
\pgfsetroundjoin%
\definecolor{currentfill}{rgb}{0.268510,0.009605,0.335427}%
\pgfsetfillcolor{currentfill}%
\pgfsetlinewidth{0.000000pt}%
\definecolor{currentstroke}{rgb}{0.000000,0.000000,0.000000}%
\pgfsetstrokecolor{currentstroke}%
\pgfsetdash{}{0pt}%
\pgfpathmoveto{\pgfqpoint{1.412653in}{1.287711in}}%
\pgfpathlineto{\pgfqpoint{1.413047in}{1.287916in}}%
\pgfpathlineto{\pgfqpoint{1.413442in}{1.288310in}}%
\pgfpathlineto{\pgfqpoint{1.413838in}{1.288899in}}%
\pgfpathlineto{\pgfqpoint{1.414233in}{1.289687in}}%
\pgfpathlineto{\pgfqpoint{1.429674in}{1.288566in}}%
\pgfpathlineto{\pgfqpoint{1.445046in}{1.287202in}}%
\pgfpathlineto{\pgfqpoint{1.460335in}{1.285594in}}%
\pgfpathlineto{\pgfqpoint{1.475528in}{1.283745in}}%
\pgfpathlineto{\pgfqpoint{1.474746in}{1.282994in}}%
\pgfpathlineto{\pgfqpoint{1.473966in}{1.282442in}}%
\pgfpathlineto{\pgfqpoint{1.473186in}{1.282084in}}%
\pgfpathlineto{\pgfqpoint{1.472408in}{1.281917in}}%
\pgfpathlineto{\pgfqpoint{1.457597in}{1.283720in}}%
\pgfpathlineto{\pgfqpoint{1.442692in}{1.285288in}}%
\pgfpathlineto{\pgfqpoint{1.427706in}{1.286619in}}%
\pgfpathlineto{\pgfqpoint{1.412653in}{1.287711in}}%
\pgfpathclose%
\pgfusepath{fill}%
\end{pgfscope}%
\begin{pgfscope}%
\pgfpathrectangle{\pgfqpoint{0.329460in}{0.284240in}}{\pgfqpoint{1.989680in}{1.989680in}}%
\pgfusepath{clip}%
\pgfsetbuttcap%
\pgfsetroundjoin%
\definecolor{currentfill}{rgb}{0.267004,0.004874,0.329415}%
\pgfsetfillcolor{currentfill}%
\pgfsetlinewidth{0.000000pt}%
\definecolor{currentstroke}{rgb}{0.000000,0.000000,0.000000}%
\pgfsetstrokecolor{currentstroke}%
\pgfsetdash{}{0pt}%
\pgfpathmoveto{\pgfqpoint{1.168929in}{1.274282in}}%
\pgfpathlineto{\pgfqpoint{1.167780in}{1.275176in}}%
\pgfpathlineto{\pgfqpoint{1.166631in}{1.276277in}}%
\pgfpathlineto{\pgfqpoint{1.165480in}{1.277589in}}%
\pgfpathlineto{\pgfqpoint{1.164327in}{1.279115in}}%
\pgfpathlineto{\pgfqpoint{1.179391in}{1.281958in}}%
\pgfpathlineto{\pgfqpoint{1.194607in}{1.284559in}}%
\pgfpathlineto{\pgfqpoint{1.209961in}{1.286918in}}%
\pgfpathlineto{\pgfqpoint{1.225439in}{1.289032in}}%
\pgfpathlineto{\pgfqpoint{1.226213in}{1.287447in}}%
\pgfpathlineto{\pgfqpoint{1.226987in}{1.286076in}}%
\pgfpathlineto{\pgfqpoint{1.227759in}{1.284916in}}%
\pgfpathlineto{\pgfqpoint{1.228531in}{1.283962in}}%
\pgfpathlineto{\pgfqpoint{1.213434in}{1.281899in}}%
\pgfpathlineto{\pgfqpoint{1.198460in}{1.279596in}}%
\pgfpathlineto{\pgfqpoint{1.183620in}{1.277057in}}%
\pgfpathlineto{\pgfqpoint{1.168929in}{1.274282in}}%
\pgfpathclose%
\pgfusepath{fill}%
\end{pgfscope}%
\begin{pgfscope}%
\pgfpathrectangle{\pgfqpoint{0.329460in}{0.284240in}}{\pgfqpoint{1.989680in}{1.989680in}}%
\pgfusepath{clip}%
\pgfsetbuttcap%
\pgfsetroundjoin%
\definecolor{currentfill}{rgb}{0.172719,0.448791,0.557885}%
\pgfsetfillcolor{currentfill}%
\pgfsetlinewidth{0.000000pt}%
\definecolor{currentstroke}{rgb}{0.000000,0.000000,0.000000}%
\pgfsetstrokecolor{currentstroke}%
\pgfsetdash{}{0pt}%
\pgfpathmoveto{\pgfqpoint{1.272050in}{1.587488in}}%
\pgfpathlineto{\pgfqpoint{1.271612in}{1.601982in}}%
\pgfpathlineto{\pgfqpoint{1.271172in}{1.616910in}}%
\pgfpathlineto{\pgfqpoint{1.270730in}{1.632279in}}%
\pgfpathlineto{\pgfqpoint{1.291070in}{1.633330in}}%
\pgfpathlineto{\pgfqpoint{1.311460in}{1.634076in}}%
\pgfpathlineto{\pgfqpoint{1.331883in}{1.634518in}}%
\pgfpathlineto{\pgfqpoint{1.352323in}{1.634654in}}%
\pgfpathlineto{\pgfqpoint{1.352317in}{1.619278in}}%
\pgfpathlineto{\pgfqpoint{1.352311in}{1.604343in}}%
\pgfpathlineto{\pgfqpoint{1.352305in}{1.589842in}}%
\pgfpathlineto{\pgfqpoint{1.332200in}{1.589708in}}%
\pgfpathlineto{\pgfqpoint{1.312111in}{1.589270in}}%
\pgfpathlineto{\pgfqpoint{1.292055in}{1.588530in}}%
\pgfpathlineto{\pgfqpoint{1.272050in}{1.587488in}}%
\pgfpathclose%
\pgfusepath{fill}%
\end{pgfscope}%
\begin{pgfscope}%
\pgfpathrectangle{\pgfqpoint{0.329460in}{0.284240in}}{\pgfqpoint{1.989680in}{1.989680in}}%
\pgfusepath{clip}%
\pgfsetbuttcap%
\pgfsetroundjoin%
\definecolor{currentfill}{rgb}{0.172719,0.448791,0.557885}%
\pgfsetfillcolor{currentfill}%
\pgfsetlinewidth{0.000000pt}%
\definecolor{currentstroke}{rgb}{0.000000,0.000000,0.000000}%
\pgfsetstrokecolor{currentstroke}%
\pgfsetdash{}{0pt}%
\pgfpathmoveto{\pgfqpoint{1.352305in}{1.589842in}}%
\pgfpathlineto{\pgfqpoint{1.352311in}{1.604343in}}%
\pgfpathlineto{\pgfqpoint{1.352317in}{1.619278in}}%
\pgfpathlineto{\pgfqpoint{1.352323in}{1.634654in}}%
\pgfpathlineto{\pgfqpoint{1.372762in}{1.634484in}}%
\pgfpathlineto{\pgfqpoint{1.393183in}{1.634008in}}%
\pgfpathlineto{\pgfqpoint{1.413568in}{1.633228in}}%
\pgfpathlineto{\pgfqpoint{1.433901in}{1.632143in}}%
\pgfpathlineto{\pgfqpoint{1.433446in}{1.616774in}}%
\pgfpathlineto{\pgfqpoint{1.432994in}{1.601847in}}%
\pgfpathlineto{\pgfqpoint{1.432545in}{1.587354in}}%
\pgfpathlineto{\pgfqpoint{1.412545in}{1.588429in}}%
\pgfpathlineto{\pgfqpoint{1.392494in}{1.589203in}}%
\pgfpathlineto{\pgfqpoint{1.372408in}{1.589674in}}%
\pgfpathlineto{\pgfqpoint{1.352305in}{1.589842in}}%
\pgfpathclose%
\pgfusepath{fill}%
\end{pgfscope}%
\begin{pgfscope}%
\pgfpathrectangle{\pgfqpoint{0.329460in}{0.284240in}}{\pgfqpoint{1.989680in}{1.989680in}}%
\pgfusepath{clip}%
\pgfsetbuttcap%
\pgfsetroundjoin%
\definecolor{currentfill}{rgb}{0.276194,0.190074,0.493001}%
\pgfsetfillcolor{currentfill}%
\pgfsetlinewidth{0.000000pt}%
\definecolor{currentstroke}{rgb}{0.000000,0.000000,0.000000}%
\pgfsetstrokecolor{currentstroke}%
\pgfsetdash{}{0pt}%
\pgfpathmoveto{\pgfqpoint{1.068512in}{1.375833in}}%
\pgfpathlineto{\pgfqpoint{1.066902in}{1.384103in}}%
\pgfpathlineto{\pgfqpoint{1.065287in}{1.392710in}}%
\pgfpathlineto{\pgfqpoint{1.063666in}{1.401659in}}%
\pgfpathlineto{\pgfqpoint{1.062040in}{1.410958in}}%
\pgfpathlineto{\pgfqpoint{1.078952in}{1.415313in}}%
\pgfpathlineto{\pgfqpoint{1.096100in}{1.419401in}}%
\pgfpathlineto{\pgfqpoint{1.113469in}{1.423219in}}%
\pgfpathlineto{\pgfqpoint{1.131043in}{1.426764in}}%
\pgfpathlineto{\pgfqpoint{1.132278in}{1.417401in}}%
\pgfpathlineto{\pgfqpoint{1.133509in}{1.408385in}}%
\pgfpathlineto{\pgfqpoint{1.134735in}{1.399712in}}%
\pgfpathlineto{\pgfqpoint{1.135958in}{1.391376in}}%
\pgfpathlineto{\pgfqpoint{1.118779in}{1.387889in}}%
\pgfpathlineto{\pgfqpoint{1.101801in}{1.384135in}}%
\pgfpathlineto{\pgfqpoint{1.085041in}{1.380116in}}%
\pgfpathlineto{\pgfqpoint{1.068512in}{1.375833in}}%
\pgfpathclose%
\pgfusepath{fill}%
\end{pgfscope}%
\begin{pgfscope}%
\pgfpathrectangle{\pgfqpoint{0.329460in}{0.284240in}}{\pgfqpoint{1.989680in}{1.989680in}}%
\pgfusepath{clip}%
\pgfsetbuttcap%
\pgfsetroundjoin%
\definecolor{currentfill}{rgb}{0.268510,0.009605,0.335427}%
\pgfsetfillcolor{currentfill}%
\pgfsetlinewidth{0.000000pt}%
\definecolor{currentstroke}{rgb}{0.000000,0.000000,0.000000}%
\pgfsetstrokecolor{currentstroke}%
\pgfsetdash{}{0pt}%
\pgfpathmoveto{\pgfqpoint{1.099763in}{1.273387in}}%
\pgfpathlineto{\pgfqpoint{1.098234in}{1.275968in}}%
\pgfpathlineto{\pgfqpoint{1.096702in}{1.278789in}}%
\pgfpathlineto{\pgfqpoint{1.095168in}{1.281855in}}%
\pgfpathlineto{\pgfqpoint{1.093630in}{1.285169in}}%
\pgfpathlineto{\pgfqpoint{1.108676in}{1.289129in}}%
\pgfpathlineto{\pgfqpoint{1.123935in}{1.292846in}}%
\pgfpathlineto{\pgfqpoint{1.139394in}{1.296318in}}%
\pgfpathlineto{\pgfqpoint{1.155039in}{1.299543in}}%
\pgfpathlineto{\pgfqpoint{1.156207in}{1.296150in}}%
\pgfpathlineto{\pgfqpoint{1.157373in}{1.293006in}}%
\pgfpathlineto{\pgfqpoint{1.158537in}{1.290106in}}%
\pgfpathlineto{\pgfqpoint{1.159699in}{1.287446in}}%
\pgfpathlineto{\pgfqpoint{1.144428in}{1.284292in}}%
\pgfpathlineto{\pgfqpoint{1.129339in}{1.280896in}}%
\pgfpathlineto{\pgfqpoint{1.114447in}{1.277260in}}%
\pgfpathlineto{\pgfqpoint{1.099763in}{1.273387in}}%
\pgfpathclose%
\pgfusepath{fill}%
\end{pgfscope}%
\begin{pgfscope}%
\pgfpathrectangle{\pgfqpoint{0.329460in}{0.284240in}}{\pgfqpoint{1.989680in}{1.989680in}}%
\pgfusepath{clip}%
\pgfsetbuttcap%
\pgfsetroundjoin%
\definecolor{currentfill}{rgb}{0.260571,0.246922,0.522828}%
\pgfsetfillcolor{currentfill}%
\pgfsetlinewidth{0.000000pt}%
\definecolor{currentstroke}{rgb}{0.000000,0.000000,0.000000}%
\pgfsetstrokecolor{currentstroke}%
\pgfsetdash{}{0pt}%
\pgfpathmoveto{\pgfqpoint{1.555551in}{1.429684in}}%
\pgfpathlineto{\pgfqpoint{1.556701in}{1.439413in}}%
\pgfpathlineto{\pgfqpoint{1.557856in}{1.449501in}}%
\pgfpathlineto{\pgfqpoint{1.559015in}{1.459954in}}%
\pgfpathlineto{\pgfqpoint{1.560179in}{1.470777in}}%
\pgfpathlineto{\pgfqpoint{1.578327in}{1.467425in}}%
\pgfpathlineto{\pgfqpoint{1.596281in}{1.463793in}}%
\pgfpathlineto{\pgfqpoint{1.614023in}{1.459884in}}%
\pgfpathlineto{\pgfqpoint{1.631538in}{1.455701in}}%
\pgfpathlineto{\pgfqpoint{1.629973in}{1.444934in}}%
\pgfpathlineto{\pgfqpoint{1.628414in}{1.434540in}}%
\pgfpathlineto{\pgfqpoint{1.626861in}{1.424511in}}%
\pgfpathlineto{\pgfqpoint{1.625314in}{1.414842in}}%
\pgfpathlineto{\pgfqpoint{1.608192in}{1.418960in}}%
\pgfpathlineto{\pgfqpoint{1.590847in}{1.422808in}}%
\pgfpathlineto{\pgfqpoint{1.573295in}{1.426384in}}%
\pgfpathlineto{\pgfqpoint{1.555551in}{1.429684in}}%
\pgfpathclose%
\pgfusepath{fill}%
\end{pgfscope}%
\begin{pgfscope}%
\pgfpathrectangle{\pgfqpoint{0.329460in}{0.284240in}}{\pgfqpoint{1.989680in}{1.989680in}}%
\pgfusepath{clip}%
\pgfsetbuttcap%
\pgfsetroundjoin%
\definecolor{currentfill}{rgb}{0.267004,0.004874,0.329415}%
\pgfsetfillcolor{currentfill}%
\pgfsetlinewidth{0.000000pt}%
\definecolor{currentstroke}{rgb}{0.000000,0.000000,0.000000}%
\pgfsetstrokecolor{currentstroke}%
\pgfsetdash{}{0pt}%
\pgfpathmoveto{\pgfqpoint{1.524665in}{1.281654in}}%
\pgfpathlineto{\pgfqpoint{1.525736in}{1.283413in}}%
\pgfpathlineto{\pgfqpoint{1.526809in}{1.285395in}}%
\pgfpathlineto{\pgfqpoint{1.527884in}{1.287604in}}%
\pgfpathlineto{\pgfqpoint{1.528960in}{1.290044in}}%
\pgfpathlineto{\pgfqpoint{1.544381in}{1.287107in}}%
\pgfpathlineto{\pgfqpoint{1.559633in}{1.283926in}}%
\pgfpathlineto{\pgfqpoint{1.574700in}{1.280503in}}%
\pgfpathlineto{\pgfqpoint{1.589571in}{1.276841in}}%
\pgfpathlineto{\pgfqpoint{1.588124in}{1.274476in}}%
\pgfpathlineto{\pgfqpoint{1.586681in}{1.272343in}}%
\pgfpathlineto{\pgfqpoint{1.585239in}{1.270436in}}%
\pgfpathlineto{\pgfqpoint{1.583800in}{1.268753in}}%
\pgfpathlineto{\pgfqpoint{1.569292in}{1.272331in}}%
\pgfpathlineto{\pgfqpoint{1.554592in}{1.275676in}}%
\pgfpathlineto{\pgfqpoint{1.539712in}{1.278784in}}%
\pgfpathlineto{\pgfqpoint{1.524665in}{1.281654in}}%
\pgfpathclose%
\pgfusepath{fill}%
\end{pgfscope}%
\begin{pgfscope}%
\pgfpathrectangle{\pgfqpoint{0.329460in}{0.284240in}}{\pgfqpoint{1.989680in}{1.989680in}}%
\pgfusepath{clip}%
\pgfsetbuttcap%
\pgfsetroundjoin%
\definecolor{currentfill}{rgb}{0.271305,0.019942,0.347269}%
\pgfsetfillcolor{currentfill}%
\pgfsetlinewidth{0.000000pt}%
\definecolor{currentstroke}{rgb}{0.000000,0.000000,0.000000}%
\pgfsetstrokecolor{currentstroke}%
\pgfsetdash{}{0pt}%
\pgfpathmoveto{\pgfqpoint{1.292930in}{1.288825in}}%
\pgfpathlineto{\pgfqpoint{1.292548in}{1.288307in}}%
\pgfpathlineto{\pgfqpoint{1.292165in}{1.287964in}}%
\pgfpathlineto{\pgfqpoint{1.291782in}{1.287799in}}%
\pgfpathlineto{\pgfqpoint{1.291398in}{1.287818in}}%
\pgfpathlineto{\pgfqpoint{1.306511in}{1.288645in}}%
\pgfpathlineto{\pgfqpoint{1.321663in}{1.289232in}}%
\pgfpathlineto{\pgfqpoint{1.336841in}{1.289579in}}%
\pgfpathlineto{\pgfqpoint{1.352032in}{1.289686in}}%
\pgfpathlineto{\pgfqpoint{1.352026in}{1.289656in}}%
\pgfpathlineto{\pgfqpoint{1.352021in}{1.289808in}}%
\pgfpathlineto{\pgfqpoint{1.352015in}{1.290139in}}%
\pgfpathlineto{\pgfqpoint{1.352010in}{1.290646in}}%
\pgfpathlineto{\pgfqpoint{1.337208in}{1.290542in}}%
\pgfpathlineto{\pgfqpoint{1.322419in}{1.290203in}}%
\pgfpathlineto{\pgfqpoint{1.307656in}{1.289631in}}%
\pgfpathlineto{\pgfqpoint{1.292930in}{1.288825in}}%
\pgfpathclose%
\pgfusepath{fill}%
\end{pgfscope}%
\begin{pgfscope}%
\pgfpathrectangle{\pgfqpoint{0.329460in}{0.284240in}}{\pgfqpoint{1.989680in}{1.989680in}}%
\pgfusepath{clip}%
\pgfsetbuttcap%
\pgfsetroundjoin%
\definecolor{currentfill}{rgb}{0.271305,0.019942,0.347269}%
\pgfsetfillcolor{currentfill}%
\pgfsetlinewidth{0.000000pt}%
\definecolor{currentstroke}{rgb}{0.000000,0.000000,0.000000}%
\pgfsetstrokecolor{currentstroke}%
\pgfsetdash{}{0pt}%
\pgfpathmoveto{\pgfqpoint{1.352010in}{1.290646in}}%
\pgfpathlineto{\pgfqpoint{1.352015in}{1.290139in}}%
\pgfpathlineto{\pgfqpoint{1.352021in}{1.289808in}}%
\pgfpathlineto{\pgfqpoint{1.352026in}{1.289656in}}%
\pgfpathlineto{\pgfqpoint{1.352032in}{1.289686in}}%
\pgfpathlineto{\pgfqpoint{1.367221in}{1.289553in}}%
\pgfpathlineto{\pgfqpoint{1.382397in}{1.289179in}}%
\pgfpathlineto{\pgfqpoint{1.397546in}{1.288565in}}%
\pgfpathlineto{\pgfqpoint{1.412653in}{1.287711in}}%
\pgfpathlineto{\pgfqpoint{1.412259in}{1.287693in}}%
\pgfpathlineto{\pgfqpoint{1.411865in}{1.287858in}}%
\pgfpathlineto{\pgfqpoint{1.411471in}{1.288202in}}%
\pgfpathlineto{\pgfqpoint{1.411078in}{1.288721in}}%
\pgfpathlineto{\pgfqpoint{1.396358in}{1.289553in}}%
\pgfpathlineto{\pgfqpoint{1.381598in}{1.290151in}}%
\pgfpathlineto{\pgfqpoint{1.366811in}{1.290516in}}%
\pgfpathlineto{\pgfqpoint{1.352010in}{1.290646in}}%
\pgfpathclose%
\pgfusepath{fill}%
\end{pgfscope}%
\begin{pgfscope}%
\pgfpathrectangle{\pgfqpoint{0.329460in}{0.284240in}}{\pgfqpoint{1.989680in}{1.989680in}}%
\pgfusepath{clip}%
\pgfsetbuttcap%
\pgfsetroundjoin%
\definecolor{currentfill}{rgb}{0.201239,0.383670,0.554294}%
\pgfsetfillcolor{currentfill}%
\pgfsetlinewidth{0.000000pt}%
\definecolor{currentstroke}{rgb}{0.000000,0.000000,0.000000}%
\pgfsetstrokecolor{currentstroke}%
\pgfsetdash{}{0pt}%
\pgfpathmoveto{\pgfqpoint{1.508173in}{1.526380in}}%
\pgfpathlineto{\pgfqpoint{1.509045in}{1.539180in}}%
\pgfpathlineto{\pgfqpoint{1.509921in}{1.552387in}}%
\pgfpathlineto{\pgfqpoint{1.510801in}{1.566008in}}%
\pgfpathlineto{\pgfqpoint{1.511686in}{1.580051in}}%
\pgfpathlineto{\pgfqpoint{1.531173in}{1.577481in}}%
\pgfpathlineto{\pgfqpoint{1.550507in}{1.574617in}}%
\pgfpathlineto{\pgfqpoint{1.569672in}{1.571461in}}%
\pgfpathlineto{\pgfqpoint{1.568466in}{1.557446in}}%
\pgfpathlineto{\pgfqpoint{1.567266in}{1.543852in}}%
\pgfpathlineto{\pgfqpoint{1.566072in}{1.530673in}}%
\pgfpathlineto{\pgfqpoint{1.564883in}{1.517903in}}%
\pgfpathlineto{\pgfqpoint{1.546141in}{1.521018in}}%
\pgfpathlineto{\pgfqpoint{1.527232in}{1.523844in}}%
\pgfpathlineto{\pgfqpoint{1.508173in}{1.526380in}}%
\pgfpathclose%
\pgfusepath{fill}%
\end{pgfscope}%
\begin{pgfscope}%
\pgfpathrectangle{\pgfqpoint{0.329460in}{0.284240in}}{\pgfqpoint{1.989680in}{1.989680in}}%
\pgfusepath{clip}%
\pgfsetbuttcap%
\pgfsetroundjoin%
\definecolor{currentfill}{rgb}{0.172719,0.448791,0.557885}%
\pgfsetfillcolor{currentfill}%
\pgfsetlinewidth{0.000000pt}%
\definecolor{currentstroke}{rgb}{0.000000,0.000000,0.000000}%
\pgfsetstrokecolor{currentstroke}%
\pgfsetdash{}{0pt}%
\pgfpathmoveto{\pgfqpoint{1.192864in}{1.580318in}}%
\pgfpathlineto{\pgfqpoint{1.191987in}{1.594790in}}%
\pgfpathlineto{\pgfqpoint{1.191106in}{1.609696in}}%
\pgfpathlineto{\pgfqpoint{1.190221in}{1.625044in}}%
\pgfpathlineto{\pgfqpoint{1.210187in}{1.627305in}}%
\pgfpathlineto{\pgfqpoint{1.230272in}{1.629265in}}%
\pgfpathlineto{\pgfqpoint{1.250459in}{1.630923in}}%
\pgfpathlineto{\pgfqpoint{1.270730in}{1.632279in}}%
\pgfpathlineto{\pgfqpoint{1.271172in}{1.616910in}}%
\pgfpathlineto{\pgfqpoint{1.271612in}{1.601982in}}%
\pgfpathlineto{\pgfqpoint{1.272050in}{1.587488in}}%
\pgfpathlineto{\pgfqpoint{1.252111in}{1.586145in}}%
\pgfpathlineto{\pgfqpoint{1.232255in}{1.584502in}}%
\pgfpathlineto{\pgfqpoint{1.212501in}{1.582559in}}%
\pgfpathlineto{\pgfqpoint{1.192864in}{1.580318in}}%
\pgfpathclose%
\pgfusepath{fill}%
\end{pgfscope}%
\begin{pgfscope}%
\pgfpathrectangle{\pgfqpoint{0.329460in}{0.284240in}}{\pgfqpoint{1.989680in}{1.989680in}}%
\pgfusepath{clip}%
\pgfsetbuttcap%
\pgfsetroundjoin%
\definecolor{currentfill}{rgb}{0.260571,0.246922,0.522828}%
\pgfsetfillcolor{currentfill}%
\pgfsetlinewidth{0.000000pt}%
\definecolor{currentstroke}{rgb}{0.000000,0.000000,0.000000}%
\pgfsetstrokecolor{currentstroke}%
\pgfsetdash{}{0pt}%
\pgfpathmoveto{\pgfqpoint{1.062040in}{1.410958in}}%
\pgfpathlineto{\pgfqpoint{1.060407in}{1.420611in}}%
\pgfpathlineto{\pgfqpoint{1.058768in}{1.430624in}}%
\pgfpathlineto{\pgfqpoint{1.057123in}{1.441004in}}%
\pgfpathlineto{\pgfqpoint{1.055471in}{1.451755in}}%
\pgfpathlineto{\pgfqpoint{1.072772in}{1.456179in}}%
\pgfpathlineto{\pgfqpoint{1.090313in}{1.460332in}}%
\pgfpathlineto{\pgfqpoint{1.108079in}{1.464210in}}%
\pgfpathlineto{\pgfqpoint{1.126055in}{1.467811in}}%
\pgfpathlineto{\pgfqpoint{1.127309in}{1.456999in}}%
\pgfpathlineto{\pgfqpoint{1.128558in}{1.446557in}}%
\pgfpathlineto{\pgfqpoint{1.129803in}{1.436481in}}%
\pgfpathlineto{\pgfqpoint{1.131043in}{1.426764in}}%
\pgfpathlineto{\pgfqpoint{1.113469in}{1.423219in}}%
\pgfpathlineto{\pgfqpoint{1.096100in}{1.419401in}}%
\pgfpathlineto{\pgfqpoint{1.078952in}{1.415313in}}%
\pgfpathlineto{\pgfqpoint{1.062040in}{1.410958in}}%
\pgfpathclose%
\pgfusepath{fill}%
\end{pgfscope}%
\begin{pgfscope}%
\pgfpathrectangle{\pgfqpoint{0.329460in}{0.284240in}}{\pgfqpoint{1.989680in}{1.989680in}}%
\pgfusepath{clip}%
\pgfsetbuttcap%
\pgfsetroundjoin%
\definecolor{currentfill}{rgb}{0.172719,0.448791,0.557885}%
\pgfsetfillcolor{currentfill}%
\pgfsetlinewidth{0.000000pt}%
\definecolor{currentstroke}{rgb}{0.000000,0.000000,0.000000}%
\pgfsetstrokecolor{currentstroke}%
\pgfsetdash{}{0pt}%
\pgfpathmoveto{\pgfqpoint{1.432545in}{1.587354in}}%
\pgfpathlineto{\pgfqpoint{1.432994in}{1.601847in}}%
\pgfpathlineto{\pgfqpoint{1.433446in}{1.616774in}}%
\pgfpathlineto{\pgfqpoint{1.433901in}{1.632143in}}%
\pgfpathlineto{\pgfqpoint{1.454164in}{1.630754in}}%
\pgfpathlineto{\pgfqpoint{1.474340in}{1.629062in}}%
\pgfpathlineto{\pgfqpoint{1.494413in}{1.627068in}}%
\pgfpathlineto{\pgfqpoint{1.514365in}{1.624774in}}%
\pgfpathlineto{\pgfqpoint{1.513467in}{1.609427in}}%
\pgfpathlineto{\pgfqpoint{1.512574in}{1.594522in}}%
\pgfpathlineto{\pgfqpoint{1.511686in}{1.580051in}}%
\pgfpathlineto{\pgfqpoint{1.492062in}{1.582325in}}%
\pgfpathlineto{\pgfqpoint{1.472320in}{1.584300in}}%
\pgfpathlineto{\pgfqpoint{1.452475in}{1.585977in}}%
\pgfpathlineto{\pgfqpoint{1.432545in}{1.587354in}}%
\pgfpathclose%
\pgfusepath{fill}%
\end{pgfscope}%
\begin{pgfscope}%
\pgfpathrectangle{\pgfqpoint{0.329460in}{0.284240in}}{\pgfqpoint{1.989680in}{1.989680in}}%
\pgfusepath{clip}%
\pgfsetbuttcap%
\pgfsetroundjoin%
\definecolor{currentfill}{rgb}{0.267004,0.004874,0.329415}%
\pgfsetfillcolor{currentfill}%
\pgfsetlinewidth{0.000000pt}%
\definecolor{currentstroke}{rgb}{0.000000,0.000000,0.000000}%
\pgfsetstrokecolor{currentstroke}%
\pgfsetdash{}{0pt}%
\pgfpathmoveto{\pgfqpoint{1.105853in}{1.265378in}}%
\pgfpathlineto{\pgfqpoint{1.104334in}{1.267042in}}%
\pgfpathlineto{\pgfqpoint{1.102813in}{1.268928in}}%
\pgfpathlineto{\pgfqpoint{1.101289in}{1.271042in}}%
\pgfpathlineto{\pgfqpoint{1.099763in}{1.273387in}}%
\pgfpathlineto{\pgfqpoint{1.114447in}{1.277260in}}%
\pgfpathlineto{\pgfqpoint{1.129339in}{1.280896in}}%
\pgfpathlineto{\pgfqpoint{1.144428in}{1.284292in}}%
\pgfpathlineto{\pgfqpoint{1.159699in}{1.287446in}}%
\pgfpathlineto{\pgfqpoint{1.160859in}{1.285021in}}%
\pgfpathlineto{\pgfqpoint{1.162017in}{1.282827in}}%
\pgfpathlineto{\pgfqpoint{1.163173in}{1.280859in}}%
\pgfpathlineto{\pgfqpoint{1.164327in}{1.279115in}}%
\pgfpathlineto{\pgfqpoint{1.149428in}{1.276033in}}%
\pgfpathlineto{\pgfqpoint{1.134707in}{1.272715in}}%
\pgfpathlineto{\pgfqpoint{1.120177in}{1.269162in}}%
\pgfpathlineto{\pgfqpoint{1.105853in}{1.265378in}}%
\pgfpathclose%
\pgfusepath{fill}%
\end{pgfscope}%
\begin{pgfscope}%
\pgfpathrectangle{\pgfqpoint{0.329460in}{0.284240in}}{\pgfqpoint{1.989680in}{1.989680in}}%
\pgfusepath{clip}%
\pgfsetbuttcap%
\pgfsetroundjoin%
\definecolor{currentfill}{rgb}{0.268510,0.009605,0.335427}%
\pgfsetfillcolor{currentfill}%
\pgfsetlinewidth{0.000000pt}%
\definecolor{currentstroke}{rgb}{0.000000,0.000000,0.000000}%
\pgfsetstrokecolor{currentstroke}%
\pgfsetdash{}{0pt}%
\pgfpathmoveto{\pgfqpoint{1.472408in}{1.281917in}}%
\pgfpathlineto{\pgfqpoint{1.473186in}{1.282084in}}%
\pgfpathlineto{\pgfqpoint{1.473966in}{1.282442in}}%
\pgfpathlineto{\pgfqpoint{1.474746in}{1.282994in}}%
\pgfpathlineto{\pgfqpoint{1.475528in}{1.283745in}}%
\pgfpathlineto{\pgfqpoint{1.490611in}{1.281655in}}%
\pgfpathlineto{\pgfqpoint{1.505571in}{1.279326in}}%
\pgfpathlineto{\pgfqpoint{1.520395in}{1.276760in}}%
\pgfpathlineto{\pgfqpoint{1.519331in}{1.276053in}}%
\pgfpathlineto{\pgfqpoint{1.518268in}{1.275544in}}%
\pgfpathlineto{\pgfqpoint{1.517205in}{1.275229in}}%
\pgfpathlineto{\pgfqpoint{1.516144in}{1.275105in}}%
\pgfpathlineto{\pgfqpoint{1.501695in}{1.277608in}}%
\pgfpathlineto{\pgfqpoint{1.487111in}{1.279879in}}%
\pgfpathlineto{\pgfqpoint{1.472408in}{1.281917in}}%
\pgfpathclose%
\pgfusepath{fill}%
\end{pgfscope}%
\begin{pgfscope}%
\pgfpathrectangle{\pgfqpoint{0.329460in}{0.284240in}}{\pgfqpoint{1.989680in}{1.989680in}}%
\pgfusepath{clip}%
\pgfsetbuttcap%
\pgfsetroundjoin%
\definecolor{currentfill}{rgb}{0.201239,0.383670,0.554294}%
\pgfsetfillcolor{currentfill}%
\pgfsetlinewidth{0.000000pt}%
\definecolor{currentstroke}{rgb}{0.000000,0.000000,0.000000}%
\pgfsetstrokecolor{currentstroke}%
\pgfsetdash{}{0pt}%
\pgfpathmoveto{\pgfqpoint{1.120985in}{1.514894in}}%
\pgfpathlineto{\pgfqpoint{1.119704in}{1.527654in}}%
\pgfpathlineto{\pgfqpoint{1.118416in}{1.540822in}}%
\pgfpathlineto{\pgfqpoint{1.117123in}{1.554406in}}%
\pgfpathlineto{\pgfqpoint{1.115824in}{1.568412in}}%
\pgfpathlineto{\pgfqpoint{1.134824in}{1.571826in}}%
\pgfpathlineto{\pgfqpoint{1.154009in}{1.574950in}}%
\pgfpathlineto{\pgfqpoint{1.173361in}{1.577781in}}%
\pgfpathlineto{\pgfqpoint{1.192864in}{1.580318in}}%
\pgfpathlineto{\pgfqpoint{1.193736in}{1.566275in}}%
\pgfpathlineto{\pgfqpoint{1.194604in}{1.552652in}}%
\pgfpathlineto{\pgfqpoint{1.195468in}{1.539444in}}%
\pgfpathlineto{\pgfqpoint{1.196328in}{1.526644in}}%
\pgfpathlineto{\pgfqpoint{1.177254in}{1.524140in}}%
\pgfpathlineto{\pgfqpoint{1.158328in}{1.521346in}}%
\pgfpathlineto{\pgfqpoint{1.139566in}{1.518263in}}%
\pgfpathlineto{\pgfqpoint{1.120985in}{1.514894in}}%
\pgfpathclose%
\pgfusepath{fill}%
\end{pgfscope}%
\begin{pgfscope}%
\pgfpathrectangle{\pgfqpoint{0.329460in}{0.284240in}}{\pgfqpoint{1.989680in}{1.989680in}}%
\pgfusepath{clip}%
\pgfsetbuttcap%
\pgfsetroundjoin%
\definecolor{currentfill}{rgb}{0.268510,0.009605,0.335427}%
\pgfsetfillcolor{currentfill}%
\pgfsetlinewidth{0.000000pt}%
\definecolor{currentstroke}{rgb}{0.000000,0.000000,0.000000}%
\pgfsetstrokecolor{currentstroke}%
\pgfsetdash{}{0pt}%
\pgfpathmoveto{\pgfqpoint{1.173509in}{1.272689in}}%
\pgfpathlineto{\pgfqpoint{1.172365in}{1.272797in}}%
\pgfpathlineto{\pgfqpoint{1.171221in}{1.273096in}}%
\pgfpathlineto{\pgfqpoint{1.170075in}{1.273590in}}%
\pgfpathlineto{\pgfqpoint{1.168929in}{1.274282in}}%
\pgfpathlineto{\pgfqpoint{1.183620in}{1.277057in}}%
\pgfpathlineto{\pgfqpoint{1.198460in}{1.279596in}}%
\pgfpathlineto{\pgfqpoint{1.213434in}{1.281899in}}%
\pgfpathlineto{\pgfqpoint{1.228531in}{1.283962in}}%
\pgfpathlineto{\pgfqpoint{1.229301in}{1.283210in}}%
\pgfpathlineto{\pgfqpoint{1.230071in}{1.282657in}}%
\pgfpathlineto{\pgfqpoint{1.230840in}{1.282297in}}%
\pgfpathlineto{\pgfqpoint{1.231608in}{1.282129in}}%
\pgfpathlineto{\pgfqpoint{1.216892in}{1.280117in}}%
\pgfpathlineto{\pgfqpoint{1.202295in}{1.277871in}}%
\pgfpathlineto{\pgfqpoint{1.187829in}{1.275395in}}%
\pgfpathlineto{\pgfqpoint{1.173509in}{1.272689in}}%
\pgfpathclose%
\pgfusepath{fill}%
\end{pgfscope}%
\begin{pgfscope}%
\pgfpathrectangle{\pgfqpoint{0.329460in}{0.284240in}}{\pgfqpoint{1.989680in}{1.989680in}}%
\pgfusepath{clip}%
\pgfsetbuttcap%
\pgfsetroundjoin%
\definecolor{currentfill}{rgb}{0.271305,0.019942,0.347269}%
\pgfsetfillcolor{currentfill}%
\pgfsetlinewidth{0.000000pt}%
\definecolor{currentstroke}{rgb}{0.000000,0.000000,0.000000}%
\pgfsetstrokecolor{currentstroke}%
\pgfsetdash{}{0pt}%
\pgfpathmoveto{\pgfqpoint{1.234675in}{1.283283in}}%
\pgfpathlineto{\pgfqpoint{1.233909in}{1.282728in}}%
\pgfpathlineto{\pgfqpoint{1.233143in}{1.282348in}}%
\pgfpathlineto{\pgfqpoint{1.232376in}{1.282147in}}%
\pgfpathlineto{\pgfqpoint{1.231608in}{1.282129in}}%
\pgfpathlineto{\pgfqpoint{1.246430in}{1.283906in}}%
\pgfpathlineto{\pgfqpoint{1.261345in}{1.285448in}}%
\pgfpathlineto{\pgfqpoint{1.276339in}{1.286752in}}%
\pgfpathlineto{\pgfqpoint{1.291398in}{1.287818in}}%
\pgfpathlineto{\pgfqpoint{1.291782in}{1.287799in}}%
\pgfpathlineto{\pgfqpoint{1.292165in}{1.287964in}}%
\pgfpathlineto{\pgfqpoint{1.292548in}{1.288307in}}%
\pgfpathlineto{\pgfqpoint{1.292930in}{1.288825in}}%
\pgfpathlineto{\pgfqpoint{1.278257in}{1.287787in}}%
\pgfpathlineto{\pgfqpoint{1.263648in}{1.286516in}}%
\pgfpathlineto{\pgfqpoint{1.249116in}{1.285015in}}%
\pgfpathlineto{\pgfqpoint{1.234675in}{1.283283in}}%
\pgfpathclose%
\pgfusepath{fill}%
\end{pgfscope}%
\begin{pgfscope}%
\pgfpathrectangle{\pgfqpoint{0.329460in}{0.284240in}}{\pgfqpoint{1.989680in}{1.989680in}}%
\pgfusepath{clip}%
\pgfsetbuttcap%
\pgfsetroundjoin%
\definecolor{currentfill}{rgb}{0.271305,0.019942,0.347269}%
\pgfsetfillcolor{currentfill}%
\pgfsetlinewidth{0.000000pt}%
\definecolor{currentstroke}{rgb}{0.000000,0.000000,0.000000}%
\pgfsetstrokecolor{currentstroke}%
\pgfsetdash{}{0pt}%
\pgfpathmoveto{\pgfqpoint{1.411078in}{1.288721in}}%
\pgfpathlineto{\pgfqpoint{1.411471in}{1.288202in}}%
\pgfpathlineto{\pgfqpoint{1.411865in}{1.287858in}}%
\pgfpathlineto{\pgfqpoint{1.412259in}{1.287693in}}%
\pgfpathlineto{\pgfqpoint{1.412653in}{1.287711in}}%
\pgfpathlineto{\pgfqpoint{1.427706in}{1.286619in}}%
\pgfpathlineto{\pgfqpoint{1.442692in}{1.285288in}}%
\pgfpathlineto{\pgfqpoint{1.457597in}{1.283720in}}%
\pgfpathlineto{\pgfqpoint{1.472408in}{1.281917in}}%
\pgfpathlineto{\pgfqpoint{1.471629in}{1.281936in}}%
\pgfpathlineto{\pgfqpoint{1.470852in}{1.282138in}}%
\pgfpathlineto{\pgfqpoint{1.470075in}{1.282520in}}%
\pgfpathlineto{\pgfqpoint{1.469299in}{1.283076in}}%
\pgfpathlineto{\pgfqpoint{1.454868in}{1.284833in}}%
\pgfpathlineto{\pgfqpoint{1.440346in}{1.286361in}}%
\pgfpathlineto{\pgfqpoint{1.425745in}{1.287657in}}%
\pgfpathlineto{\pgfqpoint{1.411078in}{1.288721in}}%
\pgfpathclose%
\pgfusepath{fill}%
\end{pgfscope}%
\begin{pgfscope}%
\pgfpathrectangle{\pgfqpoint{0.329460in}{0.284240in}}{\pgfqpoint{1.989680in}{1.989680in}}%
\pgfusepath{clip}%
\pgfsetbuttcap%
\pgfsetroundjoin%
\definecolor{currentfill}{rgb}{0.233603,0.313828,0.543914}%
\pgfsetfillcolor{currentfill}%
\pgfsetlinewidth{0.000000pt}%
\definecolor{currentstroke}{rgb}{0.000000,0.000000,0.000000}%
\pgfsetstrokecolor{currentstroke}%
\pgfsetdash{}{0pt}%
\pgfpathmoveto{\pgfqpoint{1.560179in}{1.470777in}}%
\pgfpathlineto{\pgfqpoint{1.561348in}{1.481978in}}%
\pgfpathlineto{\pgfqpoint{1.562521in}{1.493561in}}%
\pgfpathlineto{\pgfqpoint{1.563700in}{1.505534in}}%
\pgfpathlineto{\pgfqpoint{1.564883in}{1.517903in}}%
\pgfpathlineto{\pgfqpoint{1.583443in}{1.514502in}}%
\pgfpathlineto{\pgfqpoint{1.601804in}{1.510817in}}%
\pgfpathlineto{\pgfqpoint{1.619950in}{1.506851in}}%
\pgfpathlineto{\pgfqpoint{1.637864in}{1.502606in}}%
\pgfpathlineto{\pgfqpoint{1.636272in}{1.490292in}}%
\pgfpathlineto{\pgfqpoint{1.634688in}{1.478373in}}%
\pgfpathlineto{\pgfqpoint{1.633110in}{1.466845in}}%
\pgfpathlineto{\pgfqpoint{1.631538in}{1.455701in}}%
\pgfpathlineto{\pgfqpoint{1.614023in}{1.459884in}}%
\pgfpathlineto{\pgfqpoint{1.596281in}{1.463793in}}%
\pgfpathlineto{\pgfqpoint{1.578327in}{1.467425in}}%
\pgfpathlineto{\pgfqpoint{1.560179in}{1.470777in}}%
\pgfpathclose%
\pgfusepath{fill}%
\end{pgfscope}%
\begin{pgfscope}%
\pgfpathrectangle{\pgfqpoint{0.329460in}{0.284240in}}{\pgfqpoint{1.989680in}{1.989680in}}%
\pgfusepath{clip}%
\pgfsetbuttcap%
\pgfsetroundjoin%
\definecolor{currentfill}{rgb}{0.267004,0.004874,0.329415}%
\pgfsetfillcolor{currentfill}%
\pgfsetlinewidth{0.000000pt}%
\definecolor{currentstroke}{rgb}{0.000000,0.000000,0.000000}%
\pgfsetstrokecolor{currentstroke}%
\pgfsetdash{}{0pt}%
\pgfpathmoveto{\pgfqpoint{1.520395in}{1.276760in}}%
\pgfpathlineto{\pgfqpoint{1.521461in}{1.277670in}}%
\pgfpathlineto{\pgfqpoint{1.522527in}{1.278786in}}%
\pgfpathlineto{\pgfqpoint{1.523596in}{1.280113in}}%
\pgfpathlineto{\pgfqpoint{1.524665in}{1.281654in}}%
\pgfpathlineto{\pgfqpoint{1.539712in}{1.278784in}}%
\pgfpathlineto{\pgfqpoint{1.554592in}{1.275676in}}%
\pgfpathlineto{\pgfqpoint{1.569292in}{1.272331in}}%
\pgfpathlineto{\pgfqpoint{1.583800in}{1.268753in}}%
\pgfpathlineto{\pgfqpoint{1.582362in}{1.267289in}}%
\pgfpathlineto{\pgfqpoint{1.580927in}{1.266039in}}%
\pgfpathlineto{\pgfqpoint{1.579494in}{1.265000in}}%
\pgfpathlineto{\pgfqpoint{1.578062in}{1.264168in}}%
\pgfpathlineto{\pgfqpoint{1.563916in}{1.267660in}}%
\pgfpathlineto{\pgfqpoint{1.549580in}{1.270925in}}%
\pgfpathlineto{\pgfqpoint{1.535069in}{1.273959in}}%
\pgfpathlineto{\pgfqpoint{1.520395in}{1.276760in}}%
\pgfpathclose%
\pgfusepath{fill}%
\end{pgfscope}%
\begin{pgfscope}%
\pgfpathrectangle{\pgfqpoint{0.329460in}{0.284240in}}{\pgfqpoint{1.989680in}{1.989680in}}%
\pgfusepath{clip}%
\pgfsetbuttcap%
\pgfsetroundjoin%
\definecolor{currentfill}{rgb}{0.277941,0.056324,0.381191}%
\pgfsetfillcolor{currentfill}%
\pgfsetlinewidth{0.000000pt}%
\definecolor{currentstroke}{rgb}{0.000000,0.000000,0.000000}%
\pgfsetstrokecolor{currentstroke}%
\pgfsetdash{}{0pt}%
\pgfpathmoveto{\pgfqpoint{1.601241in}{1.304612in}}%
\pgfpathlineto{\pgfqpoint{1.602714in}{1.309257in}}%
\pgfpathlineto{\pgfqpoint{1.604191in}{1.314178in}}%
\pgfpathlineto{\pgfqpoint{1.605671in}{1.319380in}}%
\pgfpathlineto{\pgfqpoint{1.607155in}{1.324868in}}%
\pgfpathlineto{\pgfqpoint{1.622908in}{1.320713in}}%
\pgfpathlineto{\pgfqpoint{1.638423in}{1.316310in}}%
\pgfpathlineto{\pgfqpoint{1.653685in}{1.311661in}}%
\pgfpathlineto{\pgfqpoint{1.668680in}{1.306769in}}%
\pgfpathlineto{\pgfqpoint{1.666834in}{1.301370in}}%
\pgfpathlineto{\pgfqpoint{1.664992in}{1.296259in}}%
\pgfpathlineto{\pgfqpoint{1.663156in}{1.291428in}}%
\pgfpathlineto{\pgfqpoint{1.661323in}{1.286875in}}%
\pgfpathlineto{\pgfqpoint{1.646681in}{1.291669in}}%
\pgfpathlineto{\pgfqpoint{1.631777in}{1.296225in}}%
\pgfpathlineto{\pgfqpoint{1.616626in}{1.300541in}}%
\pgfpathlineto{\pgfqpoint{1.601241in}{1.304612in}}%
\pgfpathclose%
\pgfusepath{fill}%
\end{pgfscope}%
\begin{pgfscope}%
\pgfpathrectangle{\pgfqpoint{0.329460in}{0.284240in}}{\pgfqpoint{1.989680in}{1.989680in}}%
\pgfusepath{clip}%
\pgfsetbuttcap%
\pgfsetroundjoin%
\definecolor{currentfill}{rgb}{0.282327,0.094955,0.417331}%
\pgfsetfillcolor{currentfill}%
\pgfsetlinewidth{0.000000pt}%
\definecolor{currentstroke}{rgb}{0.000000,0.000000,0.000000}%
\pgfsetstrokecolor{currentstroke}%
\pgfsetdash{}{0pt}%
\pgfpathmoveto{\pgfqpoint{1.607155in}{1.324868in}}%
\pgfpathlineto{\pgfqpoint{1.608643in}{1.330646in}}%
\pgfpathlineto{\pgfqpoint{1.610135in}{1.336720in}}%
\pgfpathlineto{\pgfqpoint{1.611632in}{1.343094in}}%
\pgfpathlineto{\pgfqpoint{1.613133in}{1.349774in}}%
\pgfpathlineto{\pgfqpoint{1.629258in}{1.345540in}}%
\pgfpathlineto{\pgfqpoint{1.645140in}{1.341051in}}%
\pgfpathlineto{\pgfqpoint{1.660765in}{1.336312in}}%
\pgfpathlineto{\pgfqpoint{1.676117in}{1.331326in}}%
\pgfpathlineto{\pgfqpoint{1.674249in}{1.324732in}}%
\pgfpathlineto{\pgfqpoint{1.672388in}{1.318445in}}%
\pgfpathlineto{\pgfqpoint{1.670531in}{1.312459in}}%
\pgfpathlineto{\pgfqpoint{1.668680in}{1.306769in}}%
\pgfpathlineto{\pgfqpoint{1.653685in}{1.311661in}}%
\pgfpathlineto{\pgfqpoint{1.638423in}{1.316310in}}%
\pgfpathlineto{\pgfqpoint{1.622908in}{1.320713in}}%
\pgfpathlineto{\pgfqpoint{1.607155in}{1.324868in}}%
\pgfpathclose%
\pgfusepath{fill}%
\end{pgfscope}%
\begin{pgfscope}%
\pgfpathrectangle{\pgfqpoint{0.329460in}{0.284240in}}{\pgfqpoint{1.989680in}{1.989680in}}%
\pgfusepath{clip}%
\pgfsetbuttcap%
\pgfsetroundjoin%
\definecolor{currentfill}{rgb}{0.272594,0.025563,0.353093}%
\pgfsetfillcolor{currentfill}%
\pgfsetlinewidth{0.000000pt}%
\definecolor{currentstroke}{rgb}{0.000000,0.000000,0.000000}%
\pgfsetstrokecolor{currentstroke}%
\pgfsetdash{}{0pt}%
\pgfpathmoveto{\pgfqpoint{1.595382in}{1.288701in}}%
\pgfpathlineto{\pgfqpoint{1.596842in}{1.292288in}}%
\pgfpathlineto{\pgfqpoint{1.598305in}{1.296132in}}%
\pgfpathlineto{\pgfqpoint{1.599771in}{1.300239in}}%
\pgfpathlineto{\pgfqpoint{1.601241in}{1.304612in}}%
\pgfpathlineto{\pgfqpoint{1.616626in}{1.300541in}}%
\pgfpathlineto{\pgfqpoint{1.631777in}{1.296225in}}%
\pgfpathlineto{\pgfqpoint{1.646681in}{1.291669in}}%
\pgfpathlineto{\pgfqpoint{1.661323in}{1.286875in}}%
\pgfpathlineto{\pgfqpoint{1.659495in}{1.282594in}}%
\pgfpathlineto{\pgfqpoint{1.657672in}{1.278580in}}%
\pgfpathlineto{\pgfqpoint{1.655852in}{1.274829in}}%
\pgfpathlineto{\pgfqpoint{1.654036in}{1.271336in}}%
\pgfpathlineto{\pgfqpoint{1.639743in}{1.276029in}}%
\pgfpathlineto{\pgfqpoint{1.625194in}{1.280489in}}%
\pgfpathlineto{\pgfqpoint{1.610403in}{1.284715in}}%
\pgfpathlineto{\pgfqpoint{1.595382in}{1.288701in}}%
\pgfpathclose%
\pgfusepath{fill}%
\end{pgfscope}%
\begin{pgfscope}%
\pgfpathrectangle{\pgfqpoint{0.329460in}{0.284240in}}{\pgfqpoint{1.989680in}{1.989680in}}%
\pgfusepath{clip}%
\pgfsetbuttcap%
\pgfsetroundjoin%
\definecolor{currentfill}{rgb}{0.282884,0.135920,0.453427}%
\pgfsetfillcolor{currentfill}%
\pgfsetlinewidth{0.000000pt}%
\definecolor{currentstroke}{rgb}{0.000000,0.000000,0.000000}%
\pgfsetstrokecolor{currentstroke}%
\pgfsetdash{}{0pt}%
\pgfpathmoveto{\pgfqpoint{1.613133in}{1.349774in}}%
\pgfpathlineto{\pgfqpoint{1.614638in}{1.356765in}}%
\pgfpathlineto{\pgfqpoint{1.616148in}{1.364071in}}%
\pgfpathlineto{\pgfqpoint{1.617663in}{1.371699in}}%
\pgfpathlineto{\pgfqpoint{1.619183in}{1.379653in}}%
\pgfpathlineto{\pgfqpoint{1.635685in}{1.375341in}}%
\pgfpathlineto{\pgfqpoint{1.651939in}{1.370771in}}%
\pgfpathlineto{\pgfqpoint{1.667931in}{1.365946in}}%
\pgfpathlineto{\pgfqpoint{1.683644in}{1.360868in}}%
\pgfpathlineto{\pgfqpoint{1.681753in}{1.352997in}}%
\pgfpathlineto{\pgfqpoint{1.679868in}{1.345453in}}%
\pgfpathlineto{\pgfqpoint{1.677990in}{1.338231in}}%
\pgfpathlineto{\pgfqpoint{1.676117in}{1.331326in}}%
\pgfpathlineto{\pgfqpoint{1.660765in}{1.336312in}}%
\pgfpathlineto{\pgfqpoint{1.645140in}{1.341051in}}%
\pgfpathlineto{\pgfqpoint{1.629258in}{1.345540in}}%
\pgfpathlineto{\pgfqpoint{1.613133in}{1.349774in}}%
\pgfpathclose%
\pgfusepath{fill}%
\end{pgfscope}%
\begin{pgfscope}%
\pgfpathrectangle{\pgfqpoint{0.329460in}{0.284240in}}{\pgfqpoint{1.989680in}{1.989680in}}%
\pgfusepath{clip}%
\pgfsetbuttcap%
\pgfsetroundjoin%
\definecolor{currentfill}{rgb}{0.267004,0.004874,0.329415}%
\pgfsetfillcolor{currentfill}%
\pgfsetlinewidth{0.000000pt}%
\definecolor{currentstroke}{rgb}{0.000000,0.000000,0.000000}%
\pgfsetstrokecolor{currentstroke}%
\pgfsetdash{}{0pt}%
\pgfpathmoveto{\pgfqpoint{1.111908in}{1.260874in}}%
\pgfpathlineto{\pgfqpoint{1.110397in}{1.261686in}}%
\pgfpathlineto{\pgfqpoint{1.108884in}{1.262705in}}%
\pgfpathlineto{\pgfqpoint{1.107370in}{1.263934in}}%
\pgfpathlineto{\pgfqpoint{1.105853in}{1.265378in}}%
\pgfpathlineto{\pgfqpoint{1.120177in}{1.269162in}}%
\pgfpathlineto{\pgfqpoint{1.134707in}{1.272715in}}%
\pgfpathlineto{\pgfqpoint{1.149428in}{1.276033in}}%
\pgfpathlineto{\pgfqpoint{1.164327in}{1.279115in}}%
\pgfpathlineto{\pgfqpoint{1.165480in}{1.277589in}}%
\pgfpathlineto{\pgfqpoint{1.166631in}{1.276277in}}%
\pgfpathlineto{\pgfqpoint{1.167780in}{1.275176in}}%
\pgfpathlineto{\pgfqpoint{1.168929in}{1.274282in}}%
\pgfpathlineto{\pgfqpoint{1.154399in}{1.271274in}}%
\pgfpathlineto{\pgfqpoint{1.140043in}{1.268034in}}%
\pgfpathlineto{\pgfqpoint{1.125875in}{1.264567in}}%
\pgfpathlineto{\pgfqpoint{1.111908in}{1.260874in}}%
\pgfpathclose%
\pgfusepath{fill}%
\end{pgfscope}%
\begin{pgfscope}%
\pgfpathrectangle{\pgfqpoint{0.329460in}{0.284240in}}{\pgfqpoint{1.989680in}{1.989680in}}%
\pgfusepath{clip}%
\pgfsetbuttcap%
\pgfsetroundjoin%
\definecolor{currentfill}{rgb}{0.233603,0.313828,0.543914}%
\pgfsetfillcolor{currentfill}%
\pgfsetlinewidth{0.000000pt}%
\definecolor{currentstroke}{rgb}{0.000000,0.000000,0.000000}%
\pgfsetstrokecolor{currentstroke}%
\pgfsetdash{}{0pt}%
\pgfpathmoveto{\pgfqpoint{1.055471in}{1.451755in}}%
\pgfpathlineto{\pgfqpoint{1.053812in}{1.462884in}}%
\pgfpathlineto{\pgfqpoint{1.052147in}{1.474398in}}%
\pgfpathlineto{\pgfqpoint{1.050474in}{1.486302in}}%
\pgfpathlineto{\pgfqpoint{1.048793in}{1.498603in}}%
\pgfpathlineto{\pgfqpoint{1.066489in}{1.503092in}}%
\pgfpathlineto{\pgfqpoint{1.084431in}{1.507305in}}%
\pgfpathlineto{\pgfqpoint{1.102601in}{1.511240in}}%
\pgfpathlineto{\pgfqpoint{1.120985in}{1.514894in}}%
\pgfpathlineto{\pgfqpoint{1.122261in}{1.502536in}}%
\pgfpathlineto{\pgfqpoint{1.123531in}{1.490573in}}%
\pgfpathlineto{\pgfqpoint{1.124795in}{1.479001in}}%
\pgfpathlineto{\pgfqpoint{1.126055in}{1.467811in}}%
\pgfpathlineto{\pgfqpoint{1.108079in}{1.464210in}}%
\pgfpathlineto{\pgfqpoint{1.090313in}{1.460332in}}%
\pgfpathlineto{\pgfqpoint{1.072772in}{1.456179in}}%
\pgfpathlineto{\pgfqpoint{1.055471in}{1.451755in}}%
\pgfpathclose%
\pgfusepath{fill}%
\end{pgfscope}%
\begin{pgfscope}%
\pgfpathrectangle{\pgfqpoint{0.329460in}{0.284240in}}{\pgfqpoint{1.989680in}{1.989680in}}%
\pgfusepath{clip}%
\pgfsetbuttcap%
\pgfsetroundjoin%
\definecolor{currentfill}{rgb}{0.277941,0.056324,0.381191}%
\pgfsetfillcolor{currentfill}%
\pgfsetlinewidth{0.000000pt}%
\definecolor{currentstroke}{rgb}{0.000000,0.000000,0.000000}%
\pgfsetstrokecolor{currentstroke}%
\pgfsetdash{}{0pt}%
\pgfpathmoveto{\pgfqpoint{1.028267in}{1.282418in}}%
\pgfpathlineto{\pgfqpoint{1.026358in}{1.286948in}}%
\pgfpathlineto{\pgfqpoint{1.024444in}{1.291755in}}%
\pgfpathlineto{\pgfqpoint{1.022525in}{1.296844in}}%
\pgfpathlineto{\pgfqpoint{1.020601in}{1.302220in}}%
\pgfpathlineto{\pgfqpoint{1.035348in}{1.307324in}}%
\pgfpathlineto{\pgfqpoint{1.050373in}{1.312189in}}%
\pgfpathlineto{\pgfqpoint{1.065664in}{1.316811in}}%
\pgfpathlineto{\pgfqpoint{1.081206in}{1.321187in}}%
\pgfpathlineto{\pgfqpoint{1.082772in}{1.315718in}}%
\pgfpathlineto{\pgfqpoint{1.084334in}{1.310534in}}%
\pgfpathlineto{\pgfqpoint{1.085893in}{1.305631in}}%
\pgfpathlineto{\pgfqpoint{1.087447in}{1.301005in}}%
\pgfpathlineto{\pgfqpoint{1.072269in}{1.296716in}}%
\pgfpathlineto{\pgfqpoint{1.057337in}{1.292187in}}%
\pgfpathlineto{\pgfqpoint{1.042665in}{1.287419in}}%
\pgfpathlineto{\pgfqpoint{1.028267in}{1.282418in}}%
\pgfpathclose%
\pgfusepath{fill}%
\end{pgfscope}%
\begin{pgfscope}%
\pgfpathrectangle{\pgfqpoint{0.329460in}{0.284240in}}{\pgfqpoint{1.989680in}{1.989680in}}%
\pgfusepath{clip}%
\pgfsetbuttcap%
\pgfsetroundjoin%
\definecolor{currentfill}{rgb}{0.282327,0.094955,0.417331}%
\pgfsetfillcolor{currentfill}%
\pgfsetlinewidth{0.000000pt}%
\definecolor{currentstroke}{rgb}{0.000000,0.000000,0.000000}%
\pgfsetstrokecolor{currentstroke}%
\pgfsetdash{}{0pt}%
\pgfpathmoveto{\pgfqpoint{1.020601in}{1.302220in}}%
\pgfpathlineto{\pgfqpoint{1.018672in}{1.307888in}}%
\pgfpathlineto{\pgfqpoint{1.016738in}{1.313852in}}%
\pgfpathlineto{\pgfqpoint{1.014798in}{1.320117in}}%
\pgfpathlineto{\pgfqpoint{1.012852in}{1.326689in}}%
\pgfpathlineto{\pgfqpoint{1.027950in}{1.331892in}}%
\pgfpathlineto{\pgfqpoint{1.043333in}{1.336851in}}%
\pgfpathlineto{\pgfqpoint{1.058987in}{1.341562in}}%
\pgfpathlineto{\pgfqpoint{1.074897in}{1.346023in}}%
\pgfpathlineto{\pgfqpoint{1.076481in}{1.339360in}}%
\pgfpathlineto{\pgfqpoint{1.078060in}{1.333004in}}%
\pgfpathlineto{\pgfqpoint{1.079635in}{1.326948in}}%
\pgfpathlineto{\pgfqpoint{1.081206in}{1.321187in}}%
\pgfpathlineto{\pgfqpoint{1.065664in}{1.316811in}}%
\pgfpathlineto{\pgfqpoint{1.050373in}{1.312189in}}%
\pgfpathlineto{\pgfqpoint{1.035348in}{1.307324in}}%
\pgfpathlineto{\pgfqpoint{1.020601in}{1.302220in}}%
\pgfpathclose%
\pgfusepath{fill}%
\end{pgfscope}%
\begin{pgfscope}%
\pgfpathrectangle{\pgfqpoint{0.329460in}{0.284240in}}{\pgfqpoint{1.989680in}{1.989680in}}%
\pgfusepath{clip}%
\pgfsetbuttcap%
\pgfsetroundjoin%
\definecolor{currentfill}{rgb}{0.268510,0.009605,0.335427}%
\pgfsetfillcolor{currentfill}%
\pgfsetlinewidth{0.000000pt}%
\definecolor{currentstroke}{rgb}{0.000000,0.000000,0.000000}%
\pgfsetstrokecolor{currentstroke}%
\pgfsetdash{}{0pt}%
\pgfpathmoveto{\pgfqpoint{1.589571in}{1.276841in}}%
\pgfpathlineto{\pgfqpoint{1.591019in}{1.279442in}}%
\pgfpathlineto{\pgfqpoint{1.592471in}{1.282282in}}%
\pgfpathlineto{\pgfqpoint{1.593925in}{1.285367in}}%
\pgfpathlineto{\pgfqpoint{1.595382in}{1.288701in}}%
\pgfpathlineto{\pgfqpoint{1.610403in}{1.284715in}}%
\pgfpathlineto{\pgfqpoint{1.625194in}{1.280489in}}%
\pgfpathlineto{\pgfqpoint{1.639743in}{1.276029in}}%
\pgfpathlineto{\pgfqpoint{1.654036in}{1.271336in}}%
\pgfpathlineto{\pgfqpoint{1.652224in}{1.268097in}}%
\pgfpathlineto{\pgfqpoint{1.650416in}{1.265107in}}%
\pgfpathlineto{\pgfqpoint{1.648611in}{1.262363in}}%
\pgfpathlineto{\pgfqpoint{1.646810in}{1.259858in}}%
\pgfpathlineto{\pgfqpoint{1.632863in}{1.264448in}}%
\pgfpathlineto{\pgfqpoint{1.618665in}{1.268810in}}%
\pgfpathlineto{\pgfqpoint{1.604230in}{1.272942in}}%
\pgfpathlineto{\pgfqpoint{1.589571in}{1.276841in}}%
\pgfpathclose%
\pgfusepath{fill}%
\end{pgfscope}%
\begin{pgfscope}%
\pgfpathrectangle{\pgfqpoint{0.329460in}{0.284240in}}{\pgfqpoint{1.989680in}{1.989680in}}%
\pgfusepath{clip}%
\pgfsetbuttcap%
\pgfsetroundjoin%
\definecolor{currentfill}{rgb}{0.274952,0.037752,0.364543}%
\pgfsetfillcolor{currentfill}%
\pgfsetlinewidth{0.000000pt}%
\definecolor{currentstroke}{rgb}{0.000000,0.000000,0.000000}%
\pgfsetstrokecolor{currentstroke}%
\pgfsetdash{}{0pt}%
\pgfpathmoveto{\pgfqpoint{1.294458in}{1.292579in}}%
\pgfpathlineto{\pgfqpoint{1.294077in}{1.291396in}}%
\pgfpathlineto{\pgfqpoint{1.293695in}{1.290374in}}%
\pgfpathlineto{\pgfqpoint{1.293313in}{1.289516in}}%
\pgfpathlineto{\pgfqpoint{1.292930in}{1.288825in}}%
\pgfpathlineto{\pgfqpoint{1.307656in}{1.289631in}}%
\pgfpathlineto{\pgfqpoint{1.322419in}{1.290203in}}%
\pgfpathlineto{\pgfqpoint{1.337208in}{1.290542in}}%
\pgfpathlineto{\pgfqpoint{1.352010in}{1.290646in}}%
\pgfpathlineto{\pgfqpoint{1.352005in}{1.291324in}}%
\pgfpathlineto{\pgfqpoint{1.351999in}{1.292170in}}%
\pgfpathlineto{\pgfqpoint{1.351994in}{1.293180in}}%
\pgfpathlineto{\pgfqpoint{1.351988in}{1.294351in}}%
\pgfpathlineto{\pgfqpoint{1.337575in}{1.294249in}}%
\pgfpathlineto{\pgfqpoint{1.323174in}{1.293920in}}%
\pgfpathlineto{\pgfqpoint{1.308797in}{1.293363in}}%
\pgfpathlineto{\pgfqpoint{1.294458in}{1.292579in}}%
\pgfpathclose%
\pgfusepath{fill}%
\end{pgfscope}%
\begin{pgfscope}%
\pgfpathrectangle{\pgfqpoint{0.329460in}{0.284240in}}{\pgfqpoint{1.989680in}{1.989680in}}%
\pgfusepath{clip}%
\pgfsetbuttcap%
\pgfsetroundjoin%
\definecolor{currentfill}{rgb}{0.274952,0.037752,0.364543}%
\pgfsetfillcolor{currentfill}%
\pgfsetlinewidth{0.000000pt}%
\definecolor{currentstroke}{rgb}{0.000000,0.000000,0.000000}%
\pgfsetstrokecolor{currentstroke}%
\pgfsetdash{}{0pt}%
\pgfpathmoveto{\pgfqpoint{1.351988in}{1.294351in}}%
\pgfpathlineto{\pgfqpoint{1.351994in}{1.293180in}}%
\pgfpathlineto{\pgfqpoint{1.351999in}{1.292170in}}%
\pgfpathlineto{\pgfqpoint{1.352005in}{1.291324in}}%
\pgfpathlineto{\pgfqpoint{1.352010in}{1.290646in}}%
\pgfpathlineto{\pgfqpoint{1.366811in}{1.290516in}}%
\pgfpathlineto{\pgfqpoint{1.381598in}{1.290151in}}%
\pgfpathlineto{\pgfqpoint{1.396358in}{1.289553in}}%
\pgfpathlineto{\pgfqpoint{1.411078in}{1.288721in}}%
\pgfpathlineto{\pgfqpoint{1.410685in}{1.289412in}}%
\pgfpathlineto{\pgfqpoint{1.410292in}{1.290271in}}%
\pgfpathlineto{\pgfqpoint{1.409900in}{1.291294in}}%
\pgfpathlineto{\pgfqpoint{1.409507in}{1.292478in}}%
\pgfpathlineto{\pgfqpoint{1.395173in}{1.293287in}}%
\pgfpathlineto{\pgfqpoint{1.380800in}{1.293870in}}%
\pgfpathlineto{\pgfqpoint{1.366401in}{1.294224in}}%
\pgfpathlineto{\pgfqpoint{1.351988in}{1.294351in}}%
\pgfpathclose%
\pgfusepath{fill}%
\end{pgfscope}%
\begin{pgfscope}%
\pgfpathrectangle{\pgfqpoint{0.329460in}{0.284240in}}{\pgfqpoint{1.989680in}{1.989680in}}%
\pgfusepath{clip}%
\pgfsetbuttcap%
\pgfsetroundjoin%
\definecolor{currentfill}{rgb}{0.172719,0.448791,0.557885}%
\pgfsetfillcolor{currentfill}%
\pgfsetlinewidth{0.000000pt}%
\definecolor{currentstroke}{rgb}{0.000000,0.000000,0.000000}%
\pgfsetstrokecolor{currentstroke}%
\pgfsetdash{}{0pt}%
\pgfpathmoveto{\pgfqpoint{1.511686in}{1.580051in}}%
\pgfpathlineto{\pgfqpoint{1.512574in}{1.594522in}}%
\pgfpathlineto{\pgfqpoint{1.513467in}{1.609427in}}%
\pgfpathlineto{\pgfqpoint{1.514365in}{1.624774in}}%
\pgfpathlineto{\pgfqpoint{1.534178in}{1.622181in}}%
\pgfpathlineto{\pgfqpoint{1.553837in}{1.619291in}}%
\pgfpathlineto{\pgfqpoint{1.573324in}{1.616106in}}%
\pgfpathlineto{\pgfqpoint{1.572100in}{1.600784in}}%
\pgfpathlineto{\pgfqpoint{1.570883in}{1.585905in}}%
\pgfpathlineto{\pgfqpoint{1.569672in}{1.571461in}}%
\pgfpathlineto{\pgfqpoint{1.550507in}{1.574617in}}%
\pgfpathlineto{\pgfqpoint{1.531173in}{1.577481in}}%
\pgfpathlineto{\pgfqpoint{1.511686in}{1.580051in}}%
\pgfpathclose%
\pgfusepath{fill}%
\end{pgfscope}%
\begin{pgfscope}%
\pgfpathrectangle{\pgfqpoint{0.329460in}{0.284240in}}{\pgfqpoint{1.989680in}{1.989680in}}%
\pgfusepath{clip}%
\pgfsetbuttcap%
\pgfsetroundjoin%
\definecolor{currentfill}{rgb}{0.276194,0.190074,0.493001}%
\pgfsetfillcolor{currentfill}%
\pgfsetlinewidth{0.000000pt}%
\definecolor{currentstroke}{rgb}{0.000000,0.000000,0.000000}%
\pgfsetstrokecolor{currentstroke}%
\pgfsetdash{}{0pt}%
\pgfpathmoveto{\pgfqpoint{1.619183in}{1.379653in}}%
\pgfpathlineto{\pgfqpoint{1.620708in}{1.387939in}}%
\pgfpathlineto{\pgfqpoint{1.622238in}{1.396562in}}%
\pgfpathlineto{\pgfqpoint{1.623773in}{1.405528in}}%
\pgfpathlineto{\pgfqpoint{1.625314in}{1.414842in}}%
\pgfpathlineto{\pgfqpoint{1.642199in}{1.410458in}}%
\pgfpathlineto{\pgfqpoint{1.658831in}{1.405810in}}%
\pgfpathlineto{\pgfqpoint{1.675194in}{1.400902in}}%
\pgfpathlineto{\pgfqpoint{1.691275in}{1.395737in}}%
\pgfpathlineto{\pgfqpoint{1.689357in}{1.386502in}}%
\pgfpathlineto{\pgfqpoint{1.687446in}{1.377615in}}%
\pgfpathlineto{\pgfqpoint{1.685542in}{1.369073in}}%
\pgfpathlineto{\pgfqpoint{1.683644in}{1.360868in}}%
\pgfpathlineto{\pgfqpoint{1.667931in}{1.365946in}}%
\pgfpathlineto{\pgfqpoint{1.651939in}{1.370771in}}%
\pgfpathlineto{\pgfqpoint{1.635685in}{1.375341in}}%
\pgfpathlineto{\pgfqpoint{1.619183in}{1.379653in}}%
\pgfpathclose%
\pgfusepath{fill}%
\end{pgfscope}%
\begin{pgfscope}%
\pgfpathrectangle{\pgfqpoint{0.329460in}{0.284240in}}{\pgfqpoint{1.989680in}{1.989680in}}%
\pgfusepath{clip}%
\pgfsetbuttcap%
\pgfsetroundjoin%
\definecolor{currentfill}{rgb}{0.271305,0.019942,0.347269}%
\pgfsetfillcolor{currentfill}%
\pgfsetlinewidth{0.000000pt}%
\definecolor{currentstroke}{rgb}{0.000000,0.000000,0.000000}%
\pgfsetstrokecolor{currentstroke}%
\pgfsetdash{}{0pt}%
\pgfpathmoveto{\pgfqpoint{1.469299in}{1.283076in}}%
\pgfpathlineto{\pgfqpoint{1.470075in}{1.282520in}}%
\pgfpathlineto{\pgfqpoint{1.470852in}{1.282138in}}%
\pgfpathlineto{\pgfqpoint{1.471629in}{1.281936in}}%
\pgfpathlineto{\pgfqpoint{1.472408in}{1.281917in}}%
\pgfpathlineto{\pgfqpoint{1.487111in}{1.279879in}}%
\pgfpathlineto{\pgfqpoint{1.501695in}{1.277608in}}%
\pgfpathlineto{\pgfqpoint{1.516144in}{1.275105in}}%
\pgfpathlineto{\pgfqpoint{1.515084in}{1.275169in}}%
\pgfpathlineto{\pgfqpoint{1.514025in}{1.275415in}}%
\pgfpathlineto{\pgfqpoint{1.512967in}{1.275840in}}%
\pgfpathlineto{\pgfqpoint{1.511909in}{1.276441in}}%
\pgfpathlineto{\pgfqpoint{1.497832in}{1.278879in}}%
\pgfpathlineto{\pgfqpoint{1.483624in}{1.281091in}}%
\pgfpathlineto{\pgfqpoint{1.469299in}{1.283076in}}%
\pgfpathclose%
\pgfusepath{fill}%
\end{pgfscope}%
\begin{pgfscope}%
\pgfpathrectangle{\pgfqpoint{0.329460in}{0.284240in}}{\pgfqpoint{1.989680in}{1.989680in}}%
\pgfusepath{clip}%
\pgfsetbuttcap%
\pgfsetroundjoin%
\definecolor{currentfill}{rgb}{0.272594,0.025563,0.353093}%
\pgfsetfillcolor{currentfill}%
\pgfsetlinewidth{0.000000pt}%
\definecolor{currentstroke}{rgb}{0.000000,0.000000,0.000000}%
\pgfsetstrokecolor{currentstroke}%
\pgfsetdash{}{0pt}%
\pgfpathmoveto{\pgfqpoint{1.035860in}{1.266973in}}%
\pgfpathlineto{\pgfqpoint{1.033968in}{1.270442in}}%
\pgfpathlineto{\pgfqpoint{1.032072in}{1.274169in}}%
\pgfpathlineto{\pgfqpoint{1.030172in}{1.278159in}}%
\pgfpathlineto{\pgfqpoint{1.028267in}{1.282418in}}%
\pgfpathlineto{\pgfqpoint{1.042665in}{1.287419in}}%
\pgfpathlineto{\pgfqpoint{1.057337in}{1.292187in}}%
\pgfpathlineto{\pgfqpoint{1.072269in}{1.296716in}}%
\pgfpathlineto{\pgfqpoint{1.087447in}{1.301005in}}%
\pgfpathlineto{\pgfqpoint{1.088998in}{1.296650in}}%
\pgfpathlineto{\pgfqpoint{1.090546in}{1.292563in}}%
\pgfpathlineto{\pgfqpoint{1.092090in}{1.288737in}}%
\pgfpathlineto{\pgfqpoint{1.093630in}{1.285169in}}%
\pgfpathlineto{\pgfqpoint{1.078813in}{1.280971in}}%
\pgfpathlineto{\pgfqpoint{1.064236in}{1.276536in}}%
\pgfpathlineto{\pgfqpoint{1.049914in}{1.271869in}}%
\pgfpathlineto{\pgfqpoint{1.035860in}{1.266973in}}%
\pgfpathclose%
\pgfusepath{fill}%
\end{pgfscope}%
\begin{pgfscope}%
\pgfpathrectangle{\pgfqpoint{0.329460in}{0.284240in}}{\pgfqpoint{1.989680in}{1.989680in}}%
\pgfusepath{clip}%
\pgfsetbuttcap%
\pgfsetroundjoin%
\definecolor{currentfill}{rgb}{0.282884,0.135920,0.453427}%
\pgfsetfillcolor{currentfill}%
\pgfsetlinewidth{0.000000pt}%
\definecolor{currentstroke}{rgb}{0.000000,0.000000,0.000000}%
\pgfsetstrokecolor{currentstroke}%
\pgfsetdash{}{0pt}%
\pgfpathmoveto{\pgfqpoint{1.012852in}{1.326689in}}%
\pgfpathlineto{\pgfqpoint{1.010901in}{1.333573in}}%
\pgfpathlineto{\pgfqpoint{1.008943in}{1.340773in}}%
\pgfpathlineto{\pgfqpoint{1.006979in}{1.348296in}}%
\pgfpathlineto{\pgfqpoint{1.005008in}{1.356147in}}%
\pgfpathlineto{\pgfqpoint{1.020463in}{1.361445in}}%
\pgfpathlineto{\pgfqpoint{1.036208in}{1.366495in}}%
\pgfpathlineto{\pgfqpoint{1.052229in}{1.371292in}}%
\pgfpathlineto{\pgfqpoint{1.068512in}{1.375833in}}%
\pgfpathlineto{\pgfqpoint{1.070116in}{1.367896in}}%
\pgfpathlineto{\pgfqpoint{1.071714in}{1.360285in}}%
\pgfpathlineto{\pgfqpoint{1.073308in}{1.352996in}}%
\pgfpathlineto{\pgfqpoint{1.074897in}{1.346023in}}%
\pgfpathlineto{\pgfqpoint{1.058987in}{1.341562in}}%
\pgfpathlineto{\pgfqpoint{1.043333in}{1.336851in}}%
\pgfpathlineto{\pgfqpoint{1.027950in}{1.331892in}}%
\pgfpathlineto{\pgfqpoint{1.012852in}{1.326689in}}%
\pgfpathclose%
\pgfusepath{fill}%
\end{pgfscope}%
\begin{pgfscope}%
\pgfpathrectangle{\pgfqpoint{0.329460in}{0.284240in}}{\pgfqpoint{1.989680in}{1.989680in}}%
\pgfusepath{clip}%
\pgfsetbuttcap%
\pgfsetroundjoin%
\definecolor{currentfill}{rgb}{0.172719,0.448791,0.557885}%
\pgfsetfillcolor{currentfill}%
\pgfsetlinewidth{0.000000pt}%
\definecolor{currentstroke}{rgb}{0.000000,0.000000,0.000000}%
\pgfsetstrokecolor{currentstroke}%
\pgfsetdash{}{0pt}%
\pgfpathmoveto{\pgfqpoint{1.115824in}{1.568412in}}%
\pgfpathlineto{\pgfqpoint{1.114518in}{1.582846in}}%
\pgfpathlineto{\pgfqpoint{1.113206in}{1.597717in}}%
\pgfpathlineto{\pgfqpoint{1.111887in}{1.613030in}}%
\pgfpathlineto{\pgfqpoint{1.131208in}{1.616475in}}%
\pgfpathlineto{\pgfqpoint{1.150714in}{1.619627in}}%
\pgfpathlineto{\pgfqpoint{1.170391in}{1.622484in}}%
\pgfpathlineto{\pgfqpoint{1.190221in}{1.625044in}}%
\pgfpathlineto{\pgfqpoint{1.191106in}{1.609696in}}%
\pgfpathlineto{\pgfqpoint{1.191987in}{1.594790in}}%
\pgfpathlineto{\pgfqpoint{1.192864in}{1.580318in}}%
\pgfpathlineto{\pgfqpoint{1.173361in}{1.577781in}}%
\pgfpathlineto{\pgfqpoint{1.154009in}{1.574950in}}%
\pgfpathlineto{\pgfqpoint{1.134824in}{1.571826in}}%
\pgfpathlineto{\pgfqpoint{1.115824in}{1.568412in}}%
\pgfpathclose%
\pgfusepath{fill}%
\end{pgfscope}%
\begin{pgfscope}%
\pgfpathrectangle{\pgfqpoint{0.329460in}{0.284240in}}{\pgfqpoint{1.989680in}{1.989680in}}%
\pgfusepath{clip}%
\pgfsetbuttcap%
\pgfsetroundjoin%
\definecolor{currentfill}{rgb}{0.271305,0.019942,0.347269}%
\pgfsetfillcolor{currentfill}%
\pgfsetlinewidth{0.000000pt}%
\definecolor{currentstroke}{rgb}{0.000000,0.000000,0.000000}%
\pgfsetstrokecolor{currentstroke}%
\pgfsetdash{}{0pt}%
\pgfpathmoveto{\pgfqpoint{1.178072in}{1.274087in}}%
\pgfpathlineto{\pgfqpoint{1.176933in}{1.273470in}}%
\pgfpathlineto{\pgfqpoint{1.175792in}{1.273029in}}%
\pgfpathlineto{\pgfqpoint{1.174651in}{1.272767in}}%
\pgfpathlineto{\pgfqpoint{1.173509in}{1.272689in}}%
\pgfpathlineto{\pgfqpoint{1.187829in}{1.275395in}}%
\pgfpathlineto{\pgfqpoint{1.202295in}{1.277871in}}%
\pgfpathlineto{\pgfqpoint{1.216892in}{1.280117in}}%
\pgfpathlineto{\pgfqpoint{1.231608in}{1.282129in}}%
\pgfpathlineto{\pgfqpoint{1.232376in}{1.282147in}}%
\pgfpathlineto{\pgfqpoint{1.233143in}{1.282348in}}%
\pgfpathlineto{\pgfqpoint{1.233909in}{1.282728in}}%
\pgfpathlineto{\pgfqpoint{1.234675in}{1.283283in}}%
\pgfpathlineto{\pgfqpoint{1.220337in}{1.281323in}}%
\pgfpathlineto{\pgfqpoint{1.206116in}{1.279136in}}%
\pgfpathlineto{\pgfqpoint{1.192023in}{1.276723in}}%
\pgfpathlineto{\pgfqpoint{1.178072in}{1.274087in}}%
\pgfpathclose%
\pgfusepath{fill}%
\end{pgfscope}%
\begin{pgfscope}%
\pgfpathrectangle{\pgfqpoint{0.329460in}{0.284240in}}{\pgfqpoint{1.989680in}{1.989680in}}%
\pgfusepath{clip}%
\pgfsetbuttcap%
\pgfsetroundjoin%
\definecolor{currentfill}{rgb}{0.268510,0.009605,0.335427}%
\pgfsetfillcolor{currentfill}%
\pgfsetlinewidth{0.000000pt}%
\definecolor{currentstroke}{rgb}{0.000000,0.000000,0.000000}%
\pgfsetstrokecolor{currentstroke}%
\pgfsetdash{}{0pt}%
\pgfpathmoveto{\pgfqpoint{1.516144in}{1.275105in}}%
\pgfpathlineto{\pgfqpoint{1.517205in}{1.275229in}}%
\pgfpathlineto{\pgfqpoint{1.518268in}{1.275544in}}%
\pgfpathlineto{\pgfqpoint{1.519331in}{1.276053in}}%
\pgfpathlineto{\pgfqpoint{1.520395in}{1.276760in}}%
\pgfpathlineto{\pgfqpoint{1.535069in}{1.273959in}}%
\pgfpathlineto{\pgfqpoint{1.549580in}{1.270925in}}%
\pgfpathlineto{\pgfqpoint{1.563916in}{1.267660in}}%
\pgfpathlineto{\pgfqpoint{1.578062in}{1.264168in}}%
\pgfpathlineto{\pgfqpoint{1.576632in}{1.263538in}}%
\pgfpathlineto{\pgfqpoint{1.575204in}{1.263108in}}%
\pgfpathlineto{\pgfqpoint{1.573777in}{1.262872in}}%
\pgfpathlineto{\pgfqpoint{1.572352in}{1.262827in}}%
\pgfpathlineto{\pgfqpoint{1.558565in}{1.266232in}}%
\pgfpathlineto{\pgfqpoint{1.544592in}{1.269416in}}%
\pgfpathlineto{\pgfqpoint{1.530448in}{1.272374in}}%
\pgfpathlineto{\pgfqpoint{1.516144in}{1.275105in}}%
\pgfpathclose%
\pgfusepath{fill}%
\end{pgfscope}%
\begin{pgfscope}%
\pgfpathrectangle{\pgfqpoint{0.329460in}{0.284240in}}{\pgfqpoint{1.989680in}{1.989680in}}%
\pgfusepath{clip}%
\pgfsetbuttcap%
\pgfsetroundjoin%
\definecolor{currentfill}{rgb}{0.201239,0.383670,0.554294}%
\pgfsetfillcolor{currentfill}%
\pgfsetlinewidth{0.000000pt}%
\definecolor{currentstroke}{rgb}{0.000000,0.000000,0.000000}%
\pgfsetstrokecolor{currentstroke}%
\pgfsetdash{}{0pt}%
\pgfpathmoveto{\pgfqpoint{1.564883in}{1.517903in}}%
\pgfpathlineto{\pgfqpoint{1.566072in}{1.530673in}}%
\pgfpathlineto{\pgfqpoint{1.567266in}{1.543852in}}%
\pgfpathlineto{\pgfqpoint{1.568466in}{1.557446in}}%
\pgfpathlineto{\pgfqpoint{1.569672in}{1.571461in}}%
\pgfpathlineto{\pgfqpoint{1.588650in}{1.568015in}}%
\pgfpathlineto{\pgfqpoint{1.607426in}{1.564281in}}%
\pgfpathlineto{\pgfqpoint{1.625983in}{1.560262in}}%
\pgfpathlineto{\pgfqpoint{1.644305in}{1.555961in}}%
\pgfpathlineto{\pgfqpoint{1.642683in}{1.541995in}}%
\pgfpathlineto{\pgfqpoint{1.641070in}{1.528451in}}%
\pgfpathlineto{\pgfqpoint{1.639463in}{1.515324in}}%
\pgfpathlineto{\pgfqpoint{1.637864in}{1.502606in}}%
\pgfpathlineto{\pgfqpoint{1.619950in}{1.506851in}}%
\pgfpathlineto{\pgfqpoint{1.601804in}{1.510817in}}%
\pgfpathlineto{\pgfqpoint{1.583443in}{1.514502in}}%
\pgfpathlineto{\pgfqpoint{1.564883in}{1.517903in}}%
\pgfpathclose%
\pgfusepath{fill}%
\end{pgfscope}%
\begin{pgfscope}%
\pgfpathrectangle{\pgfqpoint{0.329460in}{0.284240in}}{\pgfqpoint{1.989680in}{1.989680in}}%
\pgfusepath{clip}%
\pgfsetbuttcap%
\pgfsetroundjoin%
\definecolor{currentfill}{rgb}{0.274952,0.037752,0.364543}%
\pgfsetfillcolor{currentfill}%
\pgfsetlinewidth{0.000000pt}%
\definecolor{currentstroke}{rgb}{0.000000,0.000000,0.000000}%
\pgfsetstrokecolor{currentstroke}%
\pgfsetdash{}{0pt}%
\pgfpathmoveto{\pgfqpoint{1.237734in}{1.287186in}}%
\pgfpathlineto{\pgfqpoint{1.236969in}{1.285965in}}%
\pgfpathlineto{\pgfqpoint{1.236205in}{1.284905in}}%
\pgfpathlineto{\pgfqpoint{1.235440in}{1.284010in}}%
\pgfpathlineto{\pgfqpoint{1.234675in}{1.283283in}}%
\pgfpathlineto{\pgfqpoint{1.249116in}{1.285015in}}%
\pgfpathlineto{\pgfqpoint{1.263648in}{1.286516in}}%
\pgfpathlineto{\pgfqpoint{1.278257in}{1.287787in}}%
\pgfpathlineto{\pgfqpoint{1.292930in}{1.288825in}}%
\pgfpathlineto{\pgfqpoint{1.293313in}{1.289516in}}%
\pgfpathlineto{\pgfqpoint{1.293695in}{1.290374in}}%
\pgfpathlineto{\pgfqpoint{1.294077in}{1.291396in}}%
\pgfpathlineto{\pgfqpoint{1.294458in}{1.292579in}}%
\pgfpathlineto{\pgfqpoint{1.280170in}{1.291569in}}%
\pgfpathlineto{\pgfqpoint{1.265944in}{1.290332in}}%
\pgfpathlineto{\pgfqpoint{1.251795in}{1.288871in}}%
\pgfpathlineto{\pgfqpoint{1.237734in}{1.287186in}}%
\pgfpathclose%
\pgfusepath{fill}%
\end{pgfscope}%
\begin{pgfscope}%
\pgfpathrectangle{\pgfqpoint{0.329460in}{0.284240in}}{\pgfqpoint{1.989680in}{1.989680in}}%
\pgfusepath{clip}%
\pgfsetbuttcap%
\pgfsetroundjoin%
\definecolor{currentfill}{rgb}{0.274952,0.037752,0.364543}%
\pgfsetfillcolor{currentfill}%
\pgfsetlinewidth{0.000000pt}%
\definecolor{currentstroke}{rgb}{0.000000,0.000000,0.000000}%
\pgfsetstrokecolor{currentstroke}%
\pgfsetdash{}{0pt}%
\pgfpathmoveto{\pgfqpoint{1.409507in}{1.292478in}}%
\pgfpathlineto{\pgfqpoint{1.409900in}{1.291294in}}%
\pgfpathlineto{\pgfqpoint{1.410292in}{1.290271in}}%
\pgfpathlineto{\pgfqpoint{1.410685in}{1.289412in}}%
\pgfpathlineto{\pgfqpoint{1.411078in}{1.288721in}}%
\pgfpathlineto{\pgfqpoint{1.425745in}{1.287657in}}%
\pgfpathlineto{\pgfqpoint{1.440346in}{1.286361in}}%
\pgfpathlineto{\pgfqpoint{1.454868in}{1.284833in}}%
\pgfpathlineto{\pgfqpoint{1.469299in}{1.283076in}}%
\pgfpathlineto{\pgfqpoint{1.468523in}{1.283805in}}%
\pgfpathlineto{\pgfqpoint{1.467748in}{1.284702in}}%
\pgfpathlineto{\pgfqpoint{1.466973in}{1.285763in}}%
\pgfpathlineto{\pgfqpoint{1.466198in}{1.286985in}}%
\pgfpathlineto{\pgfqpoint{1.452147in}{1.288694in}}%
\pgfpathlineto{\pgfqpoint{1.438007in}{1.290181in}}%
\pgfpathlineto{\pgfqpoint{1.423789in}{1.291442in}}%
\pgfpathlineto{\pgfqpoint{1.409507in}{1.292478in}}%
\pgfpathclose%
\pgfusepath{fill}%
\end{pgfscope}%
\begin{pgfscope}%
\pgfpathrectangle{\pgfqpoint{0.329460in}{0.284240in}}{\pgfqpoint{1.989680in}{1.989680in}}%
\pgfusepath{clip}%
\pgfsetbuttcap%
\pgfsetroundjoin%
\definecolor{currentfill}{rgb}{0.268510,0.009605,0.335427}%
\pgfsetfillcolor{currentfill}%
\pgfsetlinewidth{0.000000pt}%
\definecolor{currentstroke}{rgb}{0.000000,0.000000,0.000000}%
\pgfsetstrokecolor{currentstroke}%
\pgfsetdash{}{0pt}%
\pgfpathmoveto{\pgfqpoint{1.043389in}{1.255591in}}%
\pgfpathlineto{\pgfqpoint{1.041512in}{1.258071in}}%
\pgfpathlineto{\pgfqpoint{1.039632in}{1.260792in}}%
\pgfpathlineto{\pgfqpoint{1.037748in}{1.263757in}}%
\pgfpathlineto{\pgfqpoint{1.035860in}{1.266973in}}%
\pgfpathlineto{\pgfqpoint{1.049914in}{1.271869in}}%
\pgfpathlineto{\pgfqpoint{1.064236in}{1.276536in}}%
\pgfpathlineto{\pgfqpoint{1.078813in}{1.280971in}}%
\pgfpathlineto{\pgfqpoint{1.093630in}{1.285169in}}%
\pgfpathlineto{\pgfqpoint{1.095168in}{1.281855in}}%
\pgfpathlineto{\pgfqpoint{1.096702in}{1.278789in}}%
\pgfpathlineto{\pgfqpoint{1.098234in}{1.275968in}}%
\pgfpathlineto{\pgfqpoint{1.099763in}{1.273387in}}%
\pgfpathlineto{\pgfqpoint{1.085302in}{1.269281in}}%
\pgfpathlineto{\pgfqpoint{1.071078in}{1.264944in}}%
\pgfpathlineto{\pgfqpoint{1.057102in}{1.260379in}}%
\pgfpathlineto{\pgfqpoint{1.043389in}{1.255591in}}%
\pgfpathclose%
\pgfusepath{fill}%
\end{pgfscope}%
\begin{pgfscope}%
\pgfpathrectangle{\pgfqpoint{0.329460in}{0.284240in}}{\pgfqpoint{1.989680in}{1.989680in}}%
\pgfusepath{clip}%
\pgfsetbuttcap%
\pgfsetroundjoin%
\definecolor{currentfill}{rgb}{0.267004,0.004874,0.329415}%
\pgfsetfillcolor{currentfill}%
\pgfsetlinewidth{0.000000pt}%
\definecolor{currentstroke}{rgb}{0.000000,0.000000,0.000000}%
\pgfsetstrokecolor{currentstroke}%
\pgfsetdash{}{0pt}%
\pgfpathmoveto{\pgfqpoint{1.583800in}{1.268753in}}%
\pgfpathlineto{\pgfqpoint{1.585239in}{1.270436in}}%
\pgfpathlineto{\pgfqpoint{1.586681in}{1.272343in}}%
\pgfpathlineto{\pgfqpoint{1.588124in}{1.274476in}}%
\pgfpathlineto{\pgfqpoint{1.589571in}{1.276841in}}%
\pgfpathlineto{\pgfqpoint{1.604230in}{1.272942in}}%
\pgfpathlineto{\pgfqpoint{1.618665in}{1.268810in}}%
\pgfpathlineto{\pgfqpoint{1.632863in}{1.264448in}}%
\pgfpathlineto{\pgfqpoint{1.646810in}{1.259858in}}%
\pgfpathlineto{\pgfqpoint{1.645011in}{1.257590in}}%
\pgfpathlineto{\pgfqpoint{1.643216in}{1.255554in}}%
\pgfpathlineto{\pgfqpoint{1.641423in}{1.253746in}}%
\pgfpathlineto{\pgfqpoint{1.639634in}{1.252161in}}%
\pgfpathlineto{\pgfqpoint{1.626030in}{1.256645in}}%
\pgfpathlineto{\pgfqpoint{1.612182in}{1.260907in}}%
\pgfpathlineto{\pgfqpoint{1.598100in}{1.264944in}}%
\pgfpathlineto{\pgfqpoint{1.583800in}{1.268753in}}%
\pgfpathclose%
\pgfusepath{fill}%
\end{pgfscope}%
\begin{pgfscope}%
\pgfpathrectangle{\pgfqpoint{0.329460in}{0.284240in}}{\pgfqpoint{1.989680in}{1.989680in}}%
\pgfusepath{clip}%
\pgfsetbuttcap%
\pgfsetroundjoin%
\definecolor{currentfill}{rgb}{0.268510,0.009605,0.335427}%
\pgfsetfillcolor{currentfill}%
\pgfsetlinewidth{0.000000pt}%
\definecolor{currentstroke}{rgb}{0.000000,0.000000,0.000000}%
\pgfsetstrokecolor{currentstroke}%
\pgfsetdash{}{0pt}%
\pgfpathmoveto{\pgfqpoint{1.117933in}{1.259615in}}%
\pgfpathlineto{\pgfqpoint{1.116429in}{1.259639in}}%
\pgfpathlineto{\pgfqpoint{1.114924in}{1.259855in}}%
\pgfpathlineto{\pgfqpoint{1.113416in}{1.260265in}}%
\pgfpathlineto{\pgfqpoint{1.111908in}{1.260874in}}%
\pgfpathlineto{\pgfqpoint{1.125875in}{1.264567in}}%
\pgfpathlineto{\pgfqpoint{1.140043in}{1.268034in}}%
\pgfpathlineto{\pgfqpoint{1.154399in}{1.271274in}}%
\pgfpathlineto{\pgfqpoint{1.168929in}{1.274282in}}%
\pgfpathlineto{\pgfqpoint{1.170075in}{1.273590in}}%
\pgfpathlineto{\pgfqpoint{1.171221in}{1.273096in}}%
\pgfpathlineto{\pgfqpoint{1.172365in}{1.272797in}}%
\pgfpathlineto{\pgfqpoint{1.173509in}{1.272689in}}%
\pgfpathlineto{\pgfqpoint{1.159346in}{1.269755in}}%
\pgfpathlineto{\pgfqpoint{1.145354in}{1.266597in}}%
\pgfpathlineto{\pgfqpoint{1.131546in}{1.263216in}}%
\pgfpathlineto{\pgfqpoint{1.117933in}{1.259615in}}%
\pgfpathclose%
\pgfusepath{fill}%
\end{pgfscope}%
\begin{pgfscope}%
\pgfpathrectangle{\pgfqpoint{0.329460in}{0.284240in}}{\pgfqpoint{1.989680in}{1.989680in}}%
\pgfusepath{clip}%
\pgfsetbuttcap%
\pgfsetroundjoin%
\definecolor{currentfill}{rgb}{0.276194,0.190074,0.493001}%
\pgfsetfillcolor{currentfill}%
\pgfsetlinewidth{0.000000pt}%
\definecolor{currentstroke}{rgb}{0.000000,0.000000,0.000000}%
\pgfsetstrokecolor{currentstroke}%
\pgfsetdash{}{0pt}%
\pgfpathmoveto{\pgfqpoint{1.005008in}{1.356147in}}%
\pgfpathlineto{\pgfqpoint{1.003031in}{1.364330in}}%
\pgfpathlineto{\pgfqpoint{1.001046in}{1.372853in}}%
\pgfpathlineto{\pgfqpoint{0.999055in}{1.381719in}}%
\pgfpathlineto{\pgfqpoint{0.997056in}{1.390935in}}%
\pgfpathlineto{\pgfqpoint{1.012873in}{1.396324in}}%
\pgfpathlineto{\pgfqpoint{1.028985in}{1.401460in}}%
\pgfpathlineto{\pgfqpoint{1.045379in}{1.406339in}}%
\pgfpathlineto{\pgfqpoint{1.062040in}{1.410958in}}%
\pgfpathlineto{\pgfqpoint{1.063666in}{1.401659in}}%
\pgfpathlineto{\pgfqpoint{1.065287in}{1.392710in}}%
\pgfpathlineto{\pgfqpoint{1.066902in}{1.384103in}}%
\pgfpathlineto{\pgfqpoint{1.068512in}{1.375833in}}%
\pgfpathlineto{\pgfqpoint{1.052229in}{1.371292in}}%
\pgfpathlineto{\pgfqpoint{1.036208in}{1.366495in}}%
\pgfpathlineto{\pgfqpoint{1.020463in}{1.361445in}}%
\pgfpathlineto{\pgfqpoint{1.005008in}{1.356147in}}%
\pgfpathclose%
\pgfusepath{fill}%
\end{pgfscope}%
\begin{pgfscope}%
\pgfpathrectangle{\pgfqpoint{0.329460in}{0.284240in}}{\pgfqpoint{1.989680in}{1.989680in}}%
\pgfusepath{clip}%
\pgfsetbuttcap%
\pgfsetroundjoin%
\definecolor{currentfill}{rgb}{0.260571,0.246922,0.522828}%
\pgfsetfillcolor{currentfill}%
\pgfsetlinewidth{0.000000pt}%
\definecolor{currentstroke}{rgb}{0.000000,0.000000,0.000000}%
\pgfsetstrokecolor{currentstroke}%
\pgfsetdash{}{0pt}%
\pgfpathmoveto{\pgfqpoint{1.625314in}{1.414842in}}%
\pgfpathlineto{\pgfqpoint{1.626861in}{1.424511in}}%
\pgfpathlineto{\pgfqpoint{1.628414in}{1.434540in}}%
\pgfpathlineto{\pgfqpoint{1.629973in}{1.444934in}}%
\pgfpathlineto{\pgfqpoint{1.631538in}{1.455701in}}%
\pgfpathlineto{\pgfqpoint{1.648811in}{1.451247in}}%
\pgfpathlineto{\pgfqpoint{1.665826in}{1.446525in}}%
\pgfpathlineto{\pgfqpoint{1.682568in}{1.441538in}}%
\pgfpathlineto{\pgfqpoint{1.699021in}{1.436292in}}%
\pgfpathlineto{\pgfqpoint{1.697073in}{1.425600in}}%
\pgfpathlineto{\pgfqpoint{1.695132in}{1.415281in}}%
\pgfpathlineto{\pgfqpoint{1.693200in}{1.405328in}}%
\pgfpathlineto{\pgfqpoint{1.691275in}{1.395737in}}%
\pgfpathlineto{\pgfqpoint{1.675194in}{1.400902in}}%
\pgfpathlineto{\pgfqpoint{1.658831in}{1.405810in}}%
\pgfpathlineto{\pgfqpoint{1.642199in}{1.410458in}}%
\pgfpathlineto{\pgfqpoint{1.625314in}{1.414842in}}%
\pgfpathclose%
\pgfusepath{fill}%
\end{pgfscope}%
\begin{pgfscope}%
\pgfpathrectangle{\pgfqpoint{0.329460in}{0.284240in}}{\pgfqpoint{1.989680in}{1.989680in}}%
\pgfusepath{clip}%
\pgfsetbuttcap%
\pgfsetroundjoin%
\definecolor{currentfill}{rgb}{0.201239,0.383670,0.554294}%
\pgfsetfillcolor{currentfill}%
\pgfsetlinewidth{0.000000pt}%
\definecolor{currentstroke}{rgb}{0.000000,0.000000,0.000000}%
\pgfsetstrokecolor{currentstroke}%
\pgfsetdash{}{0pt}%
\pgfpathmoveto{\pgfqpoint{1.048793in}{1.498603in}}%
\pgfpathlineto{\pgfqpoint{1.047106in}{1.511306in}}%
\pgfpathlineto{\pgfqpoint{1.045410in}{1.524420in}}%
\pgfpathlineto{\pgfqpoint{1.043707in}{1.537950in}}%
\pgfpathlineto{\pgfqpoint{1.041995in}{1.551903in}}%
\pgfpathlineto{\pgfqpoint{1.060094in}{1.556452in}}%
\pgfpathlineto{\pgfqpoint{1.078442in}{1.560722in}}%
\pgfpathlineto{\pgfqpoint{1.097025in}{1.564710in}}%
\pgfpathlineto{\pgfqpoint{1.115824in}{1.568412in}}%
\pgfpathlineto{\pgfqpoint{1.117123in}{1.554406in}}%
\pgfpathlineto{\pgfqpoint{1.118416in}{1.540822in}}%
\pgfpathlineto{\pgfqpoint{1.119704in}{1.527654in}}%
\pgfpathlineto{\pgfqpoint{1.120985in}{1.514894in}}%
\pgfpathlineto{\pgfqpoint{1.102601in}{1.511240in}}%
\pgfpathlineto{\pgfqpoint{1.084431in}{1.507305in}}%
\pgfpathlineto{\pgfqpoint{1.066489in}{1.503092in}}%
\pgfpathlineto{\pgfqpoint{1.048793in}{1.498603in}}%
\pgfpathclose%
\pgfusepath{fill}%
\end{pgfscope}%
\begin{pgfscope}%
\pgfpathrectangle{\pgfqpoint{0.329460in}{0.284240in}}{\pgfqpoint{1.989680in}{1.989680in}}%
\pgfusepath{clip}%
\pgfsetbuttcap%
\pgfsetroundjoin%
\definecolor{currentfill}{rgb}{0.267004,0.004874,0.329415}%
\pgfsetfillcolor{currentfill}%
\pgfsetlinewidth{0.000000pt}%
\definecolor{currentstroke}{rgb}{0.000000,0.000000,0.000000}%
\pgfsetstrokecolor{currentstroke}%
\pgfsetdash{}{0pt}%
\pgfpathmoveto{\pgfqpoint{1.050866in}{1.247993in}}%
\pgfpathlineto{\pgfqpoint{1.049001in}{1.249553in}}%
\pgfpathlineto{\pgfqpoint{1.047134in}{1.251336in}}%
\pgfpathlineto{\pgfqpoint{1.045263in}{1.253348in}}%
\pgfpathlineto{\pgfqpoint{1.043389in}{1.255591in}}%
\pgfpathlineto{\pgfqpoint{1.057102in}{1.260379in}}%
\pgfpathlineto{\pgfqpoint{1.071078in}{1.264944in}}%
\pgfpathlineto{\pgfqpoint{1.085302in}{1.269281in}}%
\pgfpathlineto{\pgfqpoint{1.099763in}{1.273387in}}%
\pgfpathlineto{\pgfqpoint{1.101289in}{1.271042in}}%
\pgfpathlineto{\pgfqpoint{1.102813in}{1.268928in}}%
\pgfpathlineto{\pgfqpoint{1.104334in}{1.267042in}}%
\pgfpathlineto{\pgfqpoint{1.105853in}{1.265378in}}%
\pgfpathlineto{\pgfqpoint{1.091747in}{1.261366in}}%
\pgfpathlineto{\pgfqpoint{1.077872in}{1.257129in}}%
\pgfpathlineto{\pgfqpoint{1.064240in}{1.252670in}}%
\pgfpathlineto{\pgfqpoint{1.050866in}{1.247993in}}%
\pgfpathclose%
\pgfusepath{fill}%
\end{pgfscope}%
\begin{pgfscope}%
\pgfpathrectangle{\pgfqpoint{0.329460in}{0.284240in}}{\pgfqpoint{1.989680in}{1.989680in}}%
\pgfusepath{clip}%
\pgfsetbuttcap%
\pgfsetroundjoin%
\definecolor{currentfill}{rgb}{0.267004,0.004874,0.329415}%
\pgfsetfillcolor{currentfill}%
\pgfsetlinewidth{0.000000pt}%
\definecolor{currentstroke}{rgb}{0.000000,0.000000,0.000000}%
\pgfsetstrokecolor{currentstroke}%
\pgfsetdash{}{0pt}%
\pgfpathmoveto{\pgfqpoint{1.578062in}{1.264168in}}%
\pgfpathlineto{\pgfqpoint{1.579494in}{1.265000in}}%
\pgfpathlineto{\pgfqpoint{1.580927in}{1.266039in}}%
\pgfpathlineto{\pgfqpoint{1.582362in}{1.267289in}}%
\pgfpathlineto{\pgfqpoint{1.583800in}{1.268753in}}%
\pgfpathlineto{\pgfqpoint{1.598100in}{1.264944in}}%
\pgfpathlineto{\pgfqpoint{1.612182in}{1.260907in}}%
\pgfpathlineto{\pgfqpoint{1.626030in}{1.256645in}}%
\pgfpathlineto{\pgfqpoint{1.639634in}{1.252161in}}%
\pgfpathlineto{\pgfqpoint{1.637847in}{1.250796in}}%
\pgfpathlineto{\pgfqpoint{1.636062in}{1.249645in}}%
\pgfpathlineto{\pgfqpoint{1.634280in}{1.248706in}}%
\pgfpathlineto{\pgfqpoint{1.632501in}{1.247974in}}%
\pgfpathlineto{\pgfqpoint{1.619238in}{1.252350in}}%
\pgfpathlineto{\pgfqpoint{1.605736in}{1.256510in}}%
\pgfpathlineto{\pgfqpoint{1.592007in}{1.260450in}}%
\pgfpathlineto{\pgfqpoint{1.578062in}{1.264168in}}%
\pgfpathclose%
\pgfusepath{fill}%
\end{pgfscope}%
\begin{pgfscope}%
\pgfpathrectangle{\pgfqpoint{0.329460in}{0.284240in}}{\pgfqpoint{1.989680in}{1.989680in}}%
\pgfusepath{clip}%
\pgfsetbuttcap%
\pgfsetroundjoin%
\definecolor{currentfill}{rgb}{0.274952,0.037752,0.364543}%
\pgfsetfillcolor{currentfill}%
\pgfsetlinewidth{0.000000pt}%
\definecolor{currentstroke}{rgb}{0.000000,0.000000,0.000000}%
\pgfsetstrokecolor{currentstroke}%
\pgfsetdash{}{0pt}%
\pgfpathmoveto{\pgfqpoint{1.466198in}{1.286985in}}%
\pgfpathlineto{\pgfqpoint{1.466973in}{1.285763in}}%
\pgfpathlineto{\pgfqpoint{1.467748in}{1.284702in}}%
\pgfpathlineto{\pgfqpoint{1.468523in}{1.283805in}}%
\pgfpathlineto{\pgfqpoint{1.469299in}{1.283076in}}%
\pgfpathlineto{\pgfqpoint{1.483624in}{1.281091in}}%
\pgfpathlineto{\pgfqpoint{1.497832in}{1.278879in}}%
\pgfpathlineto{\pgfqpoint{1.511909in}{1.276441in}}%
\pgfpathlineto{\pgfqpoint{1.510853in}{1.277214in}}%
\pgfpathlineto{\pgfqpoint{1.509796in}{1.278155in}}%
\pgfpathlineto{\pgfqpoint{1.508741in}{1.279261in}}%
\pgfpathlineto{\pgfqpoint{1.507686in}{1.280528in}}%
\pgfpathlineto{\pgfqpoint{1.493980in}{1.282900in}}%
\pgfpathlineto{\pgfqpoint{1.480146in}{1.285052in}}%
\pgfpathlineto{\pgfqpoint{1.466198in}{1.286985in}}%
\pgfpathclose%
\pgfusepath{fill}%
\end{pgfscope}%
\begin{pgfscope}%
\pgfpathrectangle{\pgfqpoint{0.329460in}{0.284240in}}{\pgfqpoint{1.989680in}{1.989680in}}%
\pgfusepath{clip}%
\pgfsetbuttcap%
\pgfsetroundjoin%
\definecolor{currentfill}{rgb}{0.271305,0.019942,0.347269}%
\pgfsetfillcolor{currentfill}%
\pgfsetlinewidth{0.000000pt}%
\definecolor{currentstroke}{rgb}{0.000000,0.000000,0.000000}%
\pgfsetstrokecolor{currentstroke}%
\pgfsetdash{}{0pt}%
\pgfpathmoveto{\pgfqpoint{1.511909in}{1.276441in}}%
\pgfpathlineto{\pgfqpoint{1.512967in}{1.275840in}}%
\pgfpathlineto{\pgfqpoint{1.514025in}{1.275415in}}%
\pgfpathlineto{\pgfqpoint{1.515084in}{1.275169in}}%
\pgfpathlineto{\pgfqpoint{1.516144in}{1.275105in}}%
\pgfpathlineto{\pgfqpoint{1.530448in}{1.272374in}}%
\pgfpathlineto{\pgfqpoint{1.544592in}{1.269416in}}%
\pgfpathlineto{\pgfqpoint{1.558565in}{1.266232in}}%
\pgfpathlineto{\pgfqpoint{1.572352in}{1.262827in}}%
\pgfpathlineto{\pgfqpoint{1.570928in}{1.262969in}}%
\pgfpathlineto{\pgfqpoint{1.569505in}{1.263295in}}%
\pgfpathlineto{\pgfqpoint{1.568083in}{1.263800in}}%
\pgfpathlineto{\pgfqpoint{1.566663in}{1.264481in}}%
\pgfpathlineto{\pgfqpoint{1.553233in}{1.267798in}}%
\pgfpathlineto{\pgfqpoint{1.539623in}{1.270899in}}%
\pgfpathlineto{\pgfqpoint{1.525844in}{1.273781in}}%
\pgfpathlineto{\pgfqpoint{1.511909in}{1.276441in}}%
\pgfpathclose%
\pgfusepath{fill}%
\end{pgfscope}%
\begin{pgfscope}%
\pgfpathrectangle{\pgfqpoint{0.329460in}{0.284240in}}{\pgfqpoint{1.989680in}{1.989680in}}%
\pgfusepath{clip}%
\pgfsetbuttcap%
\pgfsetroundjoin%
\definecolor{currentfill}{rgb}{0.260571,0.246922,0.522828}%
\pgfsetfillcolor{currentfill}%
\pgfsetlinewidth{0.000000pt}%
\definecolor{currentstroke}{rgb}{0.000000,0.000000,0.000000}%
\pgfsetstrokecolor{currentstroke}%
\pgfsetdash{}{0pt}%
\pgfpathmoveto{\pgfqpoint{0.997056in}{1.390935in}}%
\pgfpathlineto{\pgfqpoint{0.995050in}{1.400506in}}%
\pgfpathlineto{\pgfqpoint{0.993036in}{1.410439in}}%
\pgfpathlineto{\pgfqpoint{0.991014in}{1.420739in}}%
\pgfpathlineto{\pgfqpoint{0.988984in}{1.431412in}}%
\pgfpathlineto{\pgfqpoint{1.005168in}{1.436887in}}%
\pgfpathlineto{\pgfqpoint{1.021654in}{1.442105in}}%
\pgfpathlineto{\pgfqpoint{1.038427in}{1.447063in}}%
\pgfpathlineto{\pgfqpoint{1.055471in}{1.451755in}}%
\pgfpathlineto{\pgfqpoint{1.057123in}{1.441004in}}%
\pgfpathlineto{\pgfqpoint{1.058768in}{1.430624in}}%
\pgfpathlineto{\pgfqpoint{1.060407in}{1.420611in}}%
\pgfpathlineto{\pgfqpoint{1.062040in}{1.410958in}}%
\pgfpathlineto{\pgfqpoint{1.045379in}{1.406339in}}%
\pgfpathlineto{\pgfqpoint{1.028985in}{1.401460in}}%
\pgfpathlineto{\pgfqpoint{1.012873in}{1.396324in}}%
\pgfpathlineto{\pgfqpoint{0.997056in}{1.390935in}}%
\pgfpathclose%
\pgfusepath{fill}%
\end{pgfscope}%
\begin{pgfscope}%
\pgfpathrectangle{\pgfqpoint{0.329460in}{0.284240in}}{\pgfqpoint{1.989680in}{1.989680in}}%
\pgfusepath{clip}%
\pgfsetbuttcap%
\pgfsetroundjoin%
\definecolor{currentfill}{rgb}{0.279566,0.067836,0.391917}%
\pgfsetfillcolor{currentfill}%
\pgfsetlinewidth{0.000000pt}%
\definecolor{currentstroke}{rgb}{0.000000,0.000000,0.000000}%
\pgfsetstrokecolor{currentstroke}%
\pgfsetdash{}{0pt}%
\pgfpathmoveto{\pgfqpoint{1.295984in}{1.298848in}}%
\pgfpathlineto{\pgfqpoint{1.295603in}{1.297057in}}%
\pgfpathlineto{\pgfqpoint{1.295221in}{1.295413in}}%
\pgfpathlineto{\pgfqpoint{1.294840in}{1.293919in}}%
\pgfpathlineto{\pgfqpoint{1.294458in}{1.292579in}}%
\pgfpathlineto{\pgfqpoint{1.308797in}{1.293363in}}%
\pgfpathlineto{\pgfqpoint{1.323174in}{1.293920in}}%
\pgfpathlineto{\pgfqpoint{1.337575in}{1.294249in}}%
\pgfpathlineto{\pgfqpoint{1.351988in}{1.294351in}}%
\pgfpathlineto{\pgfqpoint{1.351983in}{1.295679in}}%
\pgfpathlineto{\pgfqpoint{1.351978in}{1.297160in}}%
\pgfpathlineto{\pgfqpoint{1.351972in}{1.298792in}}%
\pgfpathlineto{\pgfqpoint{1.351967in}{1.300570in}}%
\pgfpathlineto{\pgfqpoint{1.337941in}{1.300472in}}%
\pgfpathlineto{\pgfqpoint{1.323927in}{1.300151in}}%
\pgfpathlineto{\pgfqpoint{1.309937in}{1.299610in}}%
\pgfpathlineto{\pgfqpoint{1.295984in}{1.298848in}}%
\pgfpathclose%
\pgfusepath{fill}%
\end{pgfscope}%
\begin{pgfscope}%
\pgfpathrectangle{\pgfqpoint{0.329460in}{0.284240in}}{\pgfqpoint{1.989680in}{1.989680in}}%
\pgfusepath{clip}%
\pgfsetbuttcap%
\pgfsetroundjoin%
\definecolor{currentfill}{rgb}{0.279566,0.067836,0.391917}%
\pgfsetfillcolor{currentfill}%
\pgfsetlinewidth{0.000000pt}%
\definecolor{currentstroke}{rgb}{0.000000,0.000000,0.000000}%
\pgfsetstrokecolor{currentstroke}%
\pgfsetdash{}{0pt}%
\pgfpathmoveto{\pgfqpoint{1.351967in}{1.300570in}}%
\pgfpathlineto{\pgfqpoint{1.351972in}{1.298792in}}%
\pgfpathlineto{\pgfqpoint{1.351978in}{1.297160in}}%
\pgfpathlineto{\pgfqpoint{1.351983in}{1.295679in}}%
\pgfpathlineto{\pgfqpoint{1.351988in}{1.294351in}}%
\pgfpathlineto{\pgfqpoint{1.366401in}{1.294224in}}%
\pgfpathlineto{\pgfqpoint{1.380800in}{1.293870in}}%
\pgfpathlineto{\pgfqpoint{1.395173in}{1.293287in}}%
\pgfpathlineto{\pgfqpoint{1.409507in}{1.292478in}}%
\pgfpathlineto{\pgfqpoint{1.409115in}{1.293819in}}%
\pgfpathlineto{\pgfqpoint{1.408723in}{1.295314in}}%
\pgfpathlineto{\pgfqpoint{1.408331in}{1.296958in}}%
\pgfpathlineto{\pgfqpoint{1.407939in}{1.298750in}}%
\pgfpathlineto{\pgfqpoint{1.393991in}{1.299536in}}%
\pgfpathlineto{\pgfqpoint{1.380004in}{1.300102in}}%
\pgfpathlineto{\pgfqpoint{1.365992in}{1.300447in}}%
\pgfpathlineto{\pgfqpoint{1.351967in}{1.300570in}}%
\pgfpathclose%
\pgfusepath{fill}%
\end{pgfscope}%
\begin{pgfscope}%
\pgfpathrectangle{\pgfqpoint{0.329460in}{0.284240in}}{\pgfqpoint{1.989680in}{1.989680in}}%
\pgfusepath{clip}%
\pgfsetbuttcap%
\pgfsetroundjoin%
\definecolor{currentfill}{rgb}{0.274952,0.037752,0.364543}%
\pgfsetfillcolor{currentfill}%
\pgfsetlinewidth{0.000000pt}%
\definecolor{currentstroke}{rgb}{0.000000,0.000000,0.000000}%
\pgfsetstrokecolor{currentstroke}%
\pgfsetdash{}{0pt}%
\pgfpathmoveto{\pgfqpoint{1.182623in}{1.278237in}}%
\pgfpathlineto{\pgfqpoint{1.181486in}{1.276955in}}%
\pgfpathlineto{\pgfqpoint{1.180349in}{1.275833in}}%
\pgfpathlineto{\pgfqpoint{1.179211in}{1.274876in}}%
\pgfpathlineto{\pgfqpoint{1.178072in}{1.274087in}}%
\pgfpathlineto{\pgfqpoint{1.192023in}{1.276723in}}%
\pgfpathlineto{\pgfqpoint{1.206116in}{1.279136in}}%
\pgfpathlineto{\pgfqpoint{1.220337in}{1.281323in}}%
\pgfpathlineto{\pgfqpoint{1.234675in}{1.283283in}}%
\pgfpathlineto{\pgfqpoint{1.235440in}{1.284010in}}%
\pgfpathlineto{\pgfqpoint{1.236205in}{1.284905in}}%
\pgfpathlineto{\pgfqpoint{1.236969in}{1.285965in}}%
\pgfpathlineto{\pgfqpoint{1.237734in}{1.287186in}}%
\pgfpathlineto{\pgfqpoint{1.223773in}{1.285278in}}%
\pgfpathlineto{\pgfqpoint{1.209926in}{1.283150in}}%
\pgfpathlineto{\pgfqpoint{1.196206in}{1.280802in}}%
\pgfpathlineto{\pgfqpoint{1.182623in}{1.278237in}}%
\pgfpathclose%
\pgfusepath{fill}%
\end{pgfscope}%
\begin{pgfscope}%
\pgfpathrectangle{\pgfqpoint{0.329460in}{0.284240in}}{\pgfqpoint{1.989680in}{1.989680in}}%
\pgfusepath{clip}%
\pgfsetbuttcap%
\pgfsetroundjoin%
\definecolor{currentfill}{rgb}{0.233603,0.313828,0.543914}%
\pgfsetfillcolor{currentfill}%
\pgfsetlinewidth{0.000000pt}%
\definecolor{currentstroke}{rgb}{0.000000,0.000000,0.000000}%
\pgfsetstrokecolor{currentstroke}%
\pgfsetdash{}{0pt}%
\pgfpathmoveto{\pgfqpoint{1.631538in}{1.455701in}}%
\pgfpathlineto{\pgfqpoint{1.633110in}{1.466845in}}%
\pgfpathlineto{\pgfqpoint{1.634688in}{1.478373in}}%
\pgfpathlineto{\pgfqpoint{1.636272in}{1.490292in}}%
\pgfpathlineto{\pgfqpoint{1.637864in}{1.502606in}}%
\pgfpathlineto{\pgfqpoint{1.655532in}{1.498087in}}%
\pgfpathlineto{\pgfqpoint{1.672937in}{1.493295in}}%
\pgfpathlineto{\pgfqpoint{1.690063in}{1.488236in}}%
\pgfpathlineto{\pgfqpoint{1.706895in}{1.482911in}}%
\pgfpathlineto{\pgfqpoint{1.704914in}{1.470666in}}%
\pgfpathlineto{\pgfqpoint{1.702941in}{1.458819in}}%
\pgfpathlineto{\pgfqpoint{1.700977in}{1.447362in}}%
\pgfpathlineto{\pgfqpoint{1.699021in}{1.436292in}}%
\pgfpathlineto{\pgfqpoint{1.682568in}{1.441538in}}%
\pgfpathlineto{\pgfqpoint{1.665826in}{1.446525in}}%
\pgfpathlineto{\pgfqpoint{1.648811in}{1.451247in}}%
\pgfpathlineto{\pgfqpoint{1.631538in}{1.455701in}}%
\pgfpathclose%
\pgfusepath{fill}%
\end{pgfscope}%
\begin{pgfscope}%
\pgfpathrectangle{\pgfqpoint{0.329460in}{0.284240in}}{\pgfqpoint{1.989680in}{1.989680in}}%
\pgfusepath{clip}%
\pgfsetbuttcap%
\pgfsetroundjoin%
\definecolor{currentfill}{rgb}{0.271305,0.019942,0.347269}%
\pgfsetfillcolor{currentfill}%
\pgfsetlinewidth{0.000000pt}%
\definecolor{currentstroke}{rgb}{0.000000,0.000000,0.000000}%
\pgfsetstrokecolor{currentstroke}%
\pgfsetdash{}{0pt}%
\pgfpathmoveto{\pgfqpoint{1.123936in}{1.261353in}}%
\pgfpathlineto{\pgfqpoint{1.122437in}{1.260651in}}%
\pgfpathlineto{\pgfqpoint{1.120937in}{1.260125in}}%
\pgfpathlineto{\pgfqpoint{1.119436in}{1.259778in}}%
\pgfpathlineto{\pgfqpoint{1.117933in}{1.259615in}}%
\pgfpathlineto{\pgfqpoint{1.131546in}{1.263216in}}%
\pgfpathlineto{\pgfqpoint{1.145354in}{1.266597in}}%
\pgfpathlineto{\pgfqpoint{1.159346in}{1.269755in}}%
\pgfpathlineto{\pgfqpoint{1.173509in}{1.272689in}}%
\pgfpathlineto{\pgfqpoint{1.174651in}{1.272767in}}%
\pgfpathlineto{\pgfqpoint{1.175792in}{1.273029in}}%
\pgfpathlineto{\pgfqpoint{1.176933in}{1.273470in}}%
\pgfpathlineto{\pgfqpoint{1.178072in}{1.274087in}}%
\pgfpathlineto{\pgfqpoint{1.164276in}{1.271230in}}%
\pgfpathlineto{\pgfqpoint{1.150646in}{1.268153in}}%
\pgfpathlineto{\pgfqpoint{1.137195in}{1.264860in}}%
\pgfpathlineto{\pgfqpoint{1.123936in}{1.261353in}}%
\pgfpathclose%
\pgfusepath{fill}%
\end{pgfscope}%
\begin{pgfscope}%
\pgfpathrectangle{\pgfqpoint{0.329460in}{0.284240in}}{\pgfqpoint{1.989680in}{1.989680in}}%
\pgfusepath{clip}%
\pgfsetbuttcap%
\pgfsetroundjoin%
\definecolor{currentfill}{rgb}{0.172719,0.448791,0.557885}%
\pgfsetfillcolor{currentfill}%
\pgfsetlinewidth{0.000000pt}%
\definecolor{currentstroke}{rgb}{0.000000,0.000000,0.000000}%
\pgfsetstrokecolor{currentstroke}%
\pgfsetdash{}{0pt}%
\pgfpathmoveto{\pgfqpoint{1.569672in}{1.571461in}}%
\pgfpathlineto{\pgfqpoint{1.570883in}{1.585905in}}%
\pgfpathlineto{\pgfqpoint{1.572100in}{1.600784in}}%
\pgfpathlineto{\pgfqpoint{1.573324in}{1.616106in}}%
\pgfpathlineto{\pgfqpoint{1.592622in}{1.612629in}}%
\pgfpathlineto{\pgfqpoint{1.611715in}{1.608861in}}%
\pgfpathlineto{\pgfqpoint{1.630586in}{1.604805in}}%
\pgfpathlineto{\pgfqpoint{1.649218in}{1.600465in}}%
\pgfpathlineto{\pgfqpoint{1.647572in}{1.585189in}}%
\pgfpathlineto{\pgfqpoint{1.645934in}{1.570356in}}%
\pgfpathlineto{\pgfqpoint{1.644305in}{1.555961in}}%
\pgfpathlineto{\pgfqpoint{1.625983in}{1.560262in}}%
\pgfpathlineto{\pgfqpoint{1.607426in}{1.564281in}}%
\pgfpathlineto{\pgfqpoint{1.588650in}{1.568015in}}%
\pgfpathlineto{\pgfqpoint{1.569672in}{1.571461in}}%
\pgfpathclose%
\pgfusepath{fill}%
\end{pgfscope}%
\begin{pgfscope}%
\pgfpathrectangle{\pgfqpoint{0.329460in}{0.284240in}}{\pgfqpoint{1.989680in}{1.989680in}}%
\pgfusepath{clip}%
\pgfsetbuttcap%
\pgfsetroundjoin%
\definecolor{currentfill}{rgb}{0.267004,0.004874,0.329415}%
\pgfsetfillcolor{currentfill}%
\pgfsetlinewidth{0.000000pt}%
\definecolor{currentstroke}{rgb}{0.000000,0.000000,0.000000}%
\pgfsetstrokecolor{currentstroke}%
\pgfsetdash{}{0pt}%
\pgfpathmoveto{\pgfqpoint{1.058297in}{1.243906in}}%
\pgfpathlineto{\pgfqpoint{1.056443in}{1.244613in}}%
\pgfpathlineto{\pgfqpoint{1.054587in}{1.245527in}}%
\pgfpathlineto{\pgfqpoint{1.052727in}{1.246652in}}%
\pgfpathlineto{\pgfqpoint{1.050866in}{1.247993in}}%
\pgfpathlineto{\pgfqpoint{1.064240in}{1.252670in}}%
\pgfpathlineto{\pgfqpoint{1.077872in}{1.257129in}}%
\pgfpathlineto{\pgfqpoint{1.091747in}{1.261366in}}%
\pgfpathlineto{\pgfqpoint{1.105853in}{1.265378in}}%
\pgfpathlineto{\pgfqpoint{1.107370in}{1.263934in}}%
\pgfpathlineto{\pgfqpoint{1.108884in}{1.262705in}}%
\pgfpathlineto{\pgfqpoint{1.110397in}{1.261686in}}%
\pgfpathlineto{\pgfqpoint{1.111908in}{1.260874in}}%
\pgfpathlineto{\pgfqpoint{1.098153in}{1.256958in}}%
\pgfpathlineto{\pgfqpoint{1.084625in}{1.252823in}}%
\pgfpathlineto{\pgfqpoint{1.071336in}{1.248471in}}%
\pgfpathlineto{\pgfqpoint{1.058297in}{1.243906in}}%
\pgfpathclose%
\pgfusepath{fill}%
\end{pgfscope}%
\begin{pgfscope}%
\pgfpathrectangle{\pgfqpoint{0.329460in}{0.284240in}}{\pgfqpoint{1.989680in}{1.989680in}}%
\pgfusepath{clip}%
\pgfsetbuttcap%
\pgfsetroundjoin%
\definecolor{currentfill}{rgb}{0.279566,0.067836,0.391917}%
\pgfsetfillcolor{currentfill}%
\pgfsetlinewidth{0.000000pt}%
\definecolor{currentstroke}{rgb}{0.000000,0.000000,0.000000}%
\pgfsetstrokecolor{currentstroke}%
\pgfsetdash{}{0pt}%
\pgfpathmoveto{\pgfqpoint{1.240787in}{1.293605in}}%
\pgfpathlineto{\pgfqpoint{1.240024in}{1.291776in}}%
\pgfpathlineto{\pgfqpoint{1.239261in}{1.290095in}}%
\pgfpathlineto{\pgfqpoint{1.238497in}{1.288563in}}%
\pgfpathlineto{\pgfqpoint{1.237734in}{1.287186in}}%
\pgfpathlineto{\pgfqpoint{1.251795in}{1.288871in}}%
\pgfpathlineto{\pgfqpoint{1.265944in}{1.290332in}}%
\pgfpathlineto{\pgfqpoint{1.280170in}{1.291569in}}%
\pgfpathlineto{\pgfqpoint{1.294458in}{1.292579in}}%
\pgfpathlineto{\pgfqpoint{1.294840in}{1.293919in}}%
\pgfpathlineto{\pgfqpoint{1.295221in}{1.295413in}}%
\pgfpathlineto{\pgfqpoint{1.295603in}{1.297057in}}%
\pgfpathlineto{\pgfqpoint{1.295984in}{1.298848in}}%
\pgfpathlineto{\pgfqpoint{1.282080in}{1.297865in}}%
\pgfpathlineto{\pgfqpoint{1.268238in}{1.296663in}}%
\pgfpathlineto{\pgfqpoint{1.254469in}{1.295243in}}%
\pgfpathlineto{\pgfqpoint{1.240787in}{1.293605in}}%
\pgfpathclose%
\pgfusepath{fill}%
\end{pgfscope}%
\begin{pgfscope}%
\pgfpathrectangle{\pgfqpoint{0.329460in}{0.284240in}}{\pgfqpoint{1.989680in}{1.989680in}}%
\pgfusepath{clip}%
\pgfsetbuttcap%
\pgfsetroundjoin%
\definecolor{currentfill}{rgb}{0.279566,0.067836,0.391917}%
\pgfsetfillcolor{currentfill}%
\pgfsetlinewidth{0.000000pt}%
\definecolor{currentstroke}{rgb}{0.000000,0.000000,0.000000}%
\pgfsetstrokecolor{currentstroke}%
\pgfsetdash{}{0pt}%
\pgfpathmoveto{\pgfqpoint{1.407939in}{1.298750in}}%
\pgfpathlineto{\pgfqpoint{1.408331in}{1.296958in}}%
\pgfpathlineto{\pgfqpoint{1.408723in}{1.295314in}}%
\pgfpathlineto{\pgfqpoint{1.409115in}{1.293819in}}%
\pgfpathlineto{\pgfqpoint{1.409507in}{1.292478in}}%
\pgfpathlineto{\pgfqpoint{1.423789in}{1.291442in}}%
\pgfpathlineto{\pgfqpoint{1.438007in}{1.290181in}}%
\pgfpathlineto{\pgfqpoint{1.452147in}{1.288694in}}%
\pgfpathlineto{\pgfqpoint{1.466198in}{1.286985in}}%
\pgfpathlineto{\pgfqpoint{1.465424in}{1.288364in}}%
\pgfpathlineto{\pgfqpoint{1.464650in}{1.289897in}}%
\pgfpathlineto{\pgfqpoint{1.463876in}{1.291580in}}%
\pgfpathlineto{\pgfqpoint{1.463103in}{1.293409in}}%
\pgfpathlineto{\pgfqpoint{1.449431in}{1.295071in}}%
\pgfpathlineto{\pgfqpoint{1.435671in}{1.296516in}}%
\pgfpathlineto{\pgfqpoint{1.421837in}{1.297743in}}%
\pgfpathlineto{\pgfqpoint{1.407939in}{1.298750in}}%
\pgfpathclose%
\pgfusepath{fill}%
\end{pgfscope}%
\begin{pgfscope}%
\pgfpathrectangle{\pgfqpoint{0.329460in}{0.284240in}}{\pgfqpoint{1.989680in}{1.989680in}}%
\pgfusepath{clip}%
\pgfsetbuttcap%
\pgfsetroundjoin%
\definecolor{currentfill}{rgb}{0.277941,0.056324,0.381191}%
\pgfsetfillcolor{currentfill}%
\pgfsetlinewidth{0.000000pt}%
\definecolor{currentstroke}{rgb}{0.000000,0.000000,0.000000}%
\pgfsetstrokecolor{currentstroke}%
\pgfsetdash{}{0pt}%
\pgfpathmoveto{\pgfqpoint{1.661323in}{1.286875in}}%
\pgfpathlineto{\pgfqpoint{1.663156in}{1.291428in}}%
\pgfpathlineto{\pgfqpoint{1.664992in}{1.296259in}}%
\pgfpathlineto{\pgfqpoint{1.666834in}{1.301370in}}%
\pgfpathlineto{\pgfqpoint{1.668680in}{1.306769in}}%
\pgfpathlineto{\pgfqpoint{1.683394in}{1.301639in}}%
\pgfpathlineto{\pgfqpoint{1.697814in}{1.296274in}}%
\pgfpathlineto{\pgfqpoint{1.711926in}{1.290679in}}%
\pgfpathlineto{\pgfqpoint{1.725716in}{1.284857in}}%
\pgfpathlineto{\pgfqpoint{1.723531in}{1.279568in}}%
\pgfpathlineto{\pgfqpoint{1.721351in}{1.274566in}}%
\pgfpathlineto{\pgfqpoint{1.719177in}{1.269846in}}%
\pgfpathlineto{\pgfqpoint{1.717008in}{1.265404in}}%
\pgfpathlineto{\pgfqpoint{1.703546in}{1.271108in}}%
\pgfpathlineto{\pgfqpoint{1.689769in}{1.276591in}}%
\pgfpathlineto{\pgfqpoint{1.675690in}{1.281848in}}%
\pgfpathlineto{\pgfqpoint{1.661323in}{1.286875in}}%
\pgfpathclose%
\pgfusepath{fill}%
\end{pgfscope}%
\begin{pgfscope}%
\pgfpathrectangle{\pgfqpoint{0.329460in}{0.284240in}}{\pgfqpoint{1.989680in}{1.989680in}}%
\pgfusepath{clip}%
\pgfsetbuttcap%
\pgfsetroundjoin%
\definecolor{currentfill}{rgb}{0.282327,0.094955,0.417331}%
\pgfsetfillcolor{currentfill}%
\pgfsetlinewidth{0.000000pt}%
\definecolor{currentstroke}{rgb}{0.000000,0.000000,0.000000}%
\pgfsetstrokecolor{currentstroke}%
\pgfsetdash{}{0pt}%
\pgfpathmoveto{\pgfqpoint{1.668680in}{1.306769in}}%
\pgfpathlineto{\pgfqpoint{1.670531in}{1.312459in}}%
\pgfpathlineto{\pgfqpoint{1.672388in}{1.318445in}}%
\pgfpathlineto{\pgfqpoint{1.674249in}{1.324732in}}%
\pgfpathlineto{\pgfqpoint{1.676117in}{1.331326in}}%
\pgfpathlineto{\pgfqpoint{1.691182in}{1.326096in}}%
\pgfpathlineto{\pgfqpoint{1.705947in}{1.320627in}}%
\pgfpathlineto{\pgfqpoint{1.720398in}{1.314923in}}%
\pgfpathlineto{\pgfqpoint{1.734520in}{1.308988in}}%
\pgfpathlineto{\pgfqpoint{1.732310in}{1.302499in}}%
\pgfpathlineto{\pgfqpoint{1.730106in}{1.296318in}}%
\pgfpathlineto{\pgfqpoint{1.727908in}{1.290439in}}%
\pgfpathlineto{\pgfqpoint{1.725716in}{1.284857in}}%
\pgfpathlineto{\pgfqpoint{1.711926in}{1.290679in}}%
\pgfpathlineto{\pgfqpoint{1.697814in}{1.296274in}}%
\pgfpathlineto{\pgfqpoint{1.683394in}{1.301639in}}%
\pgfpathlineto{\pgfqpoint{1.668680in}{1.306769in}}%
\pgfpathclose%
\pgfusepath{fill}%
\end{pgfscope}%
\begin{pgfscope}%
\pgfpathrectangle{\pgfqpoint{0.329460in}{0.284240in}}{\pgfqpoint{1.989680in}{1.989680in}}%
\pgfusepath{clip}%
\pgfsetbuttcap%
\pgfsetroundjoin%
\definecolor{currentfill}{rgb}{0.268510,0.009605,0.335427}%
\pgfsetfillcolor{currentfill}%
\pgfsetlinewidth{0.000000pt}%
\definecolor{currentstroke}{rgb}{0.000000,0.000000,0.000000}%
\pgfsetstrokecolor{currentstroke}%
\pgfsetdash{}{0pt}%
\pgfpathmoveto{\pgfqpoint{1.572352in}{1.262827in}}%
\pgfpathlineto{\pgfqpoint{1.573777in}{1.262872in}}%
\pgfpathlineto{\pgfqpoint{1.575204in}{1.263108in}}%
\pgfpathlineto{\pgfqpoint{1.576632in}{1.263538in}}%
\pgfpathlineto{\pgfqpoint{1.578062in}{1.264168in}}%
\pgfpathlineto{\pgfqpoint{1.592007in}{1.260450in}}%
\pgfpathlineto{\pgfqpoint{1.605736in}{1.256510in}}%
\pgfpathlineto{\pgfqpoint{1.619238in}{1.252350in}}%
\pgfpathlineto{\pgfqpoint{1.632501in}{1.247974in}}%
\pgfpathlineto{\pgfqpoint{1.630723in}{1.247445in}}%
\pgfpathlineto{\pgfqpoint{1.628948in}{1.247116in}}%
\pgfpathlineto{\pgfqpoint{1.627174in}{1.246981in}}%
\pgfpathlineto{\pgfqpoint{1.625402in}{1.247038in}}%
\pgfpathlineto{\pgfqpoint{1.612479in}{1.251304in}}%
\pgfpathlineto{\pgfqpoint{1.599322in}{1.255360in}}%
\pgfpathlineto{\pgfqpoint{1.585942in}{1.259202in}}%
\pgfpathlineto{\pgfqpoint{1.572352in}{1.262827in}}%
\pgfpathclose%
\pgfusepath{fill}%
\end{pgfscope}%
\begin{pgfscope}%
\pgfpathrectangle{\pgfqpoint{0.329460in}{0.284240in}}{\pgfqpoint{1.989680in}{1.989680in}}%
\pgfusepath{clip}%
\pgfsetbuttcap%
\pgfsetroundjoin%
\definecolor{currentfill}{rgb}{0.272594,0.025563,0.353093}%
\pgfsetfillcolor{currentfill}%
\pgfsetlinewidth{0.000000pt}%
\definecolor{currentstroke}{rgb}{0.000000,0.000000,0.000000}%
\pgfsetstrokecolor{currentstroke}%
\pgfsetdash{}{0pt}%
\pgfpathmoveto{\pgfqpoint{1.654036in}{1.271336in}}%
\pgfpathlineto{\pgfqpoint{1.655852in}{1.274829in}}%
\pgfpathlineto{\pgfqpoint{1.657672in}{1.278580in}}%
\pgfpathlineto{\pgfqpoint{1.659495in}{1.282594in}}%
\pgfpathlineto{\pgfqpoint{1.661323in}{1.286875in}}%
\pgfpathlineto{\pgfqpoint{1.675690in}{1.281848in}}%
\pgfpathlineto{\pgfqpoint{1.689769in}{1.276591in}}%
\pgfpathlineto{\pgfqpoint{1.703546in}{1.271108in}}%
\pgfpathlineto{\pgfqpoint{1.717008in}{1.265404in}}%
\pgfpathlineto{\pgfqpoint{1.714845in}{1.261235in}}%
\pgfpathlineto{\pgfqpoint{1.712686in}{1.257335in}}%
\pgfpathlineto{\pgfqpoint{1.710533in}{1.253697in}}%
\pgfpathlineto{\pgfqpoint{1.708384in}{1.250319in}}%
\pgfpathlineto{\pgfqpoint{1.695247in}{1.255902in}}%
\pgfpathlineto{\pgfqpoint{1.681801in}{1.261269in}}%
\pgfpathlineto{\pgfqpoint{1.668060in}{1.266415in}}%
\pgfpathlineto{\pgfqpoint{1.654036in}{1.271336in}}%
\pgfpathclose%
\pgfusepath{fill}%
\end{pgfscope}%
\begin{pgfscope}%
\pgfpathrectangle{\pgfqpoint{0.329460in}{0.284240in}}{\pgfqpoint{1.989680in}{1.989680in}}%
\pgfusepath{clip}%
\pgfsetbuttcap%
\pgfsetroundjoin%
\definecolor{currentfill}{rgb}{0.172719,0.448791,0.557885}%
\pgfsetfillcolor{currentfill}%
\pgfsetlinewidth{0.000000pt}%
\definecolor{currentstroke}{rgb}{0.000000,0.000000,0.000000}%
\pgfsetstrokecolor{currentstroke}%
\pgfsetdash{}{0pt}%
\pgfpathmoveto{\pgfqpoint{1.041995in}{1.551903in}}%
\pgfpathlineto{\pgfqpoint{1.040275in}{1.566286in}}%
\pgfpathlineto{\pgfqpoint{1.038547in}{1.581106in}}%
\pgfpathlineto{\pgfqpoint{1.036809in}{1.596370in}}%
\pgfpathlineto{\pgfqpoint{1.055215in}{1.600961in}}%
\pgfpathlineto{\pgfqpoint{1.073875in}{1.605270in}}%
\pgfpathlineto{\pgfqpoint{1.092771in}{1.609294in}}%
\pgfpathlineto{\pgfqpoint{1.111887in}{1.613030in}}%
\pgfpathlineto{\pgfqpoint{1.113206in}{1.597717in}}%
\pgfpathlineto{\pgfqpoint{1.114518in}{1.582846in}}%
\pgfpathlineto{\pgfqpoint{1.115824in}{1.568412in}}%
\pgfpathlineto{\pgfqpoint{1.097025in}{1.564710in}}%
\pgfpathlineto{\pgfqpoint{1.078442in}{1.560722in}}%
\pgfpathlineto{\pgfqpoint{1.060094in}{1.556452in}}%
\pgfpathlineto{\pgfqpoint{1.041995in}{1.551903in}}%
\pgfpathclose%
\pgfusepath{fill}%
\end{pgfscope}%
\begin{pgfscope}%
\pgfpathrectangle{\pgfqpoint{0.329460in}{0.284240in}}{\pgfqpoint{1.989680in}{1.989680in}}%
\pgfusepath{clip}%
\pgfsetbuttcap%
\pgfsetroundjoin%
\definecolor{currentfill}{rgb}{0.233603,0.313828,0.543914}%
\pgfsetfillcolor{currentfill}%
\pgfsetlinewidth{0.000000pt}%
\definecolor{currentstroke}{rgb}{0.000000,0.000000,0.000000}%
\pgfsetstrokecolor{currentstroke}%
\pgfsetdash{}{0pt}%
\pgfpathmoveto{\pgfqpoint{0.988984in}{1.431412in}}%
\pgfpathlineto{\pgfqpoint{0.986946in}{1.442464in}}%
\pgfpathlineto{\pgfqpoint{0.984899in}{1.453902in}}%
\pgfpathlineto{\pgfqpoint{0.982843in}{1.465732in}}%
\pgfpathlineto{\pgfqpoint{0.980777in}{1.477959in}}%
\pgfpathlineto{\pgfqpoint{0.997335in}{1.483516in}}%
\pgfpathlineto{\pgfqpoint{1.014201in}{1.488811in}}%
\pgfpathlineto{\pgfqpoint{1.031359in}{1.493841in}}%
\pgfpathlineto{\pgfqpoint{1.048793in}{1.498603in}}%
\pgfpathlineto{\pgfqpoint{1.050474in}{1.486302in}}%
\pgfpathlineto{\pgfqpoint{1.052147in}{1.474398in}}%
\pgfpathlineto{\pgfqpoint{1.053812in}{1.462884in}}%
\pgfpathlineto{\pgfqpoint{1.055471in}{1.451755in}}%
\pgfpathlineto{\pgfqpoint{1.038427in}{1.447063in}}%
\pgfpathlineto{\pgfqpoint{1.021654in}{1.442105in}}%
\pgfpathlineto{\pgfqpoint{1.005168in}{1.436887in}}%
\pgfpathlineto{\pgfqpoint{0.988984in}{1.431412in}}%
\pgfpathclose%
\pgfusepath{fill}%
\end{pgfscope}%
\begin{pgfscope}%
\pgfpathrectangle{\pgfqpoint{0.329460in}{0.284240in}}{\pgfqpoint{1.989680in}{1.989680in}}%
\pgfusepath{clip}%
\pgfsetbuttcap%
\pgfsetroundjoin%
\definecolor{currentfill}{rgb}{0.282884,0.135920,0.453427}%
\pgfsetfillcolor{currentfill}%
\pgfsetlinewidth{0.000000pt}%
\definecolor{currentstroke}{rgb}{0.000000,0.000000,0.000000}%
\pgfsetstrokecolor{currentstroke}%
\pgfsetdash{}{0pt}%
\pgfpathmoveto{\pgfqpoint{1.676117in}{1.331326in}}%
\pgfpathlineto{\pgfqpoint{1.677990in}{1.338231in}}%
\pgfpathlineto{\pgfqpoint{1.679868in}{1.345453in}}%
\pgfpathlineto{\pgfqpoint{1.681753in}{1.352997in}}%
\pgfpathlineto{\pgfqpoint{1.683644in}{1.360868in}}%
\pgfpathlineto{\pgfqpoint{1.699066in}{1.355543in}}%
\pgfpathlineto{\pgfqpoint{1.714181in}{1.349974in}}%
\pgfpathlineto{\pgfqpoint{1.728975in}{1.344165in}}%
\pgfpathlineto{\pgfqpoint{1.743434in}{1.338120in}}%
\pgfpathlineto{\pgfqpoint{1.741195in}{1.330350in}}%
\pgfpathlineto{\pgfqpoint{1.738963in}{1.322908in}}%
\pgfpathlineto{\pgfqpoint{1.736738in}{1.315789in}}%
\pgfpathlineto{\pgfqpoint{1.734520in}{1.308988in}}%
\pgfpathlineto{\pgfqpoint{1.720398in}{1.314923in}}%
\pgfpathlineto{\pgfqpoint{1.705947in}{1.320627in}}%
\pgfpathlineto{\pgfqpoint{1.691182in}{1.326096in}}%
\pgfpathlineto{\pgfqpoint{1.676117in}{1.331326in}}%
\pgfpathclose%
\pgfusepath{fill}%
\end{pgfscope}%
\begin{pgfscope}%
\pgfpathrectangle{\pgfqpoint{0.329460in}{0.284240in}}{\pgfqpoint{1.989680in}{1.989680in}}%
\pgfusepath{clip}%
\pgfsetbuttcap%
\pgfsetroundjoin%
\definecolor{currentfill}{rgb}{0.268510,0.009605,0.335427}%
\pgfsetfillcolor{currentfill}%
\pgfsetlinewidth{0.000000pt}%
\definecolor{currentstroke}{rgb}{0.000000,0.000000,0.000000}%
\pgfsetstrokecolor{currentstroke}%
\pgfsetdash{}{0pt}%
\pgfpathmoveto{\pgfqpoint{1.646810in}{1.259858in}}%
\pgfpathlineto{\pgfqpoint{1.648611in}{1.262363in}}%
\pgfpathlineto{\pgfqpoint{1.650416in}{1.265107in}}%
\pgfpathlineto{\pgfqpoint{1.652224in}{1.268097in}}%
\pgfpathlineto{\pgfqpoint{1.654036in}{1.271336in}}%
\pgfpathlineto{\pgfqpoint{1.668060in}{1.266415in}}%
\pgfpathlineto{\pgfqpoint{1.681801in}{1.261269in}}%
\pgfpathlineto{\pgfqpoint{1.695247in}{1.255902in}}%
\pgfpathlineto{\pgfqpoint{1.708384in}{1.250319in}}%
\pgfpathlineto{\pgfqpoint{1.706240in}{1.247196in}}%
\pgfpathlineto{\pgfqpoint{1.704100in}{1.244322in}}%
\pgfpathlineto{\pgfqpoint{1.701964in}{1.241694in}}%
\pgfpathlineto{\pgfqpoint{1.699832in}{1.239307in}}%
\pgfpathlineto{\pgfqpoint{1.687017in}{1.244766in}}%
\pgfpathlineto{\pgfqpoint{1.673899in}{1.250014in}}%
\pgfpathlineto{\pgfqpoint{1.660493in}{1.255046in}}%
\pgfpathlineto{\pgfqpoint{1.646810in}{1.259858in}}%
\pgfpathclose%
\pgfusepath{fill}%
\end{pgfscope}%
\begin{pgfscope}%
\pgfpathrectangle{\pgfqpoint{0.329460in}{0.284240in}}{\pgfqpoint{1.989680in}{1.989680in}}%
\pgfusepath{clip}%
\pgfsetbuttcap%
\pgfsetroundjoin%
\definecolor{currentfill}{rgb}{0.274952,0.037752,0.364543}%
\pgfsetfillcolor{currentfill}%
\pgfsetlinewidth{0.000000pt}%
\definecolor{currentstroke}{rgb}{0.000000,0.000000,0.000000}%
\pgfsetstrokecolor{currentstroke}%
\pgfsetdash{}{0pt}%
\pgfpathmoveto{\pgfqpoint{1.507686in}{1.280528in}}%
\pgfpathlineto{\pgfqpoint{1.508741in}{1.279261in}}%
\pgfpathlineto{\pgfqpoint{1.509796in}{1.278155in}}%
\pgfpathlineto{\pgfqpoint{1.510853in}{1.277214in}}%
\pgfpathlineto{\pgfqpoint{1.511909in}{1.276441in}}%
\pgfpathlineto{\pgfqpoint{1.525844in}{1.273781in}}%
\pgfpathlineto{\pgfqpoint{1.539623in}{1.270899in}}%
\pgfpathlineto{\pgfqpoint{1.553233in}{1.267798in}}%
\pgfpathlineto{\pgfqpoint{1.566663in}{1.264481in}}%
\pgfpathlineto{\pgfqpoint{1.565243in}{1.265334in}}%
\pgfpathlineto{\pgfqpoint{1.563825in}{1.266356in}}%
\pgfpathlineto{\pgfqpoint{1.562407in}{1.267543in}}%
\pgfpathlineto{\pgfqpoint{1.560990in}{1.268890in}}%
\pgfpathlineto{\pgfqpoint{1.547916in}{1.272118in}}%
\pgfpathlineto{\pgfqpoint{1.534666in}{1.275135in}}%
\pgfpathlineto{\pgfqpoint{1.521252in}{1.277939in}}%
\pgfpathlineto{\pgfqpoint{1.507686in}{1.280528in}}%
\pgfpathclose%
\pgfusepath{fill}%
\end{pgfscope}%
\begin{pgfscope}%
\pgfpathrectangle{\pgfqpoint{0.329460in}{0.284240in}}{\pgfqpoint{1.989680in}{1.989680in}}%
\pgfusepath{clip}%
\pgfsetbuttcap%
\pgfsetroundjoin%
\definecolor{currentfill}{rgb}{0.268510,0.009605,0.335427}%
\pgfsetfillcolor{currentfill}%
\pgfsetlinewidth{0.000000pt}%
\definecolor{currentstroke}{rgb}{0.000000,0.000000,0.000000}%
\pgfsetstrokecolor{currentstroke}%
\pgfsetdash{}{0pt}%
\pgfpathmoveto{\pgfqpoint{1.065692in}{1.243073in}}%
\pgfpathlineto{\pgfqpoint{1.063847in}{1.242990in}}%
\pgfpathlineto{\pgfqpoint{1.061999in}{1.243099in}}%
\pgfpathlineto{\pgfqpoint{1.060149in}{1.243403in}}%
\pgfpathlineto{\pgfqpoint{1.058297in}{1.243906in}}%
\pgfpathlineto{\pgfqpoint{1.071336in}{1.248471in}}%
\pgfpathlineto{\pgfqpoint{1.084625in}{1.252823in}}%
\pgfpathlineto{\pgfqpoint{1.098153in}{1.256958in}}%
\pgfpathlineto{\pgfqpoint{1.111908in}{1.260874in}}%
\pgfpathlineto{\pgfqpoint{1.113416in}{1.260265in}}%
\pgfpathlineto{\pgfqpoint{1.114924in}{1.259855in}}%
\pgfpathlineto{\pgfqpoint{1.116429in}{1.259639in}}%
\pgfpathlineto{\pgfqpoint{1.117933in}{1.259615in}}%
\pgfpathlineto{\pgfqpoint{1.104529in}{1.255797in}}%
\pgfpathlineto{\pgfqpoint{1.091346in}{1.251766in}}%
\pgfpathlineto{\pgfqpoint{1.078397in}{1.247523in}}%
\pgfpathlineto{\pgfqpoint{1.065692in}{1.243073in}}%
\pgfpathclose%
\pgfusepath{fill}%
\end{pgfscope}%
\begin{pgfscope}%
\pgfpathrectangle{\pgfqpoint{0.329460in}{0.284240in}}{\pgfqpoint{1.989680in}{1.989680in}}%
\pgfusepath{clip}%
\pgfsetbuttcap%
\pgfsetroundjoin%
\definecolor{currentfill}{rgb}{0.279566,0.067836,0.391917}%
\pgfsetfillcolor{currentfill}%
\pgfsetlinewidth{0.000000pt}%
\definecolor{currentstroke}{rgb}{0.000000,0.000000,0.000000}%
\pgfsetstrokecolor{currentstroke}%
\pgfsetdash{}{0pt}%
\pgfpathmoveto{\pgfqpoint{1.463103in}{1.293409in}}%
\pgfpathlineto{\pgfqpoint{1.463876in}{1.291580in}}%
\pgfpathlineto{\pgfqpoint{1.464650in}{1.289897in}}%
\pgfpathlineto{\pgfqpoint{1.465424in}{1.288364in}}%
\pgfpathlineto{\pgfqpoint{1.466198in}{1.286985in}}%
\pgfpathlineto{\pgfqpoint{1.480146in}{1.285052in}}%
\pgfpathlineto{\pgfqpoint{1.493980in}{1.282900in}}%
\pgfpathlineto{\pgfqpoint{1.507686in}{1.280528in}}%
\pgfpathlineto{\pgfqpoint{1.506631in}{1.281952in}}%
\pgfpathlineto{\pgfqpoint{1.505577in}{1.283530in}}%
\pgfpathlineto{\pgfqpoint{1.504523in}{1.285258in}}%
\pgfpathlineto{\pgfqpoint{1.503469in}{1.287133in}}%
\pgfpathlineto{\pgfqpoint{1.490134in}{1.289439in}}%
\pgfpathlineto{\pgfqpoint{1.476674in}{1.291531in}}%
\pgfpathlineto{\pgfqpoint{1.463103in}{1.293409in}}%
\pgfpathclose%
\pgfusepath{fill}%
\end{pgfscope}%
\begin{pgfscope}%
\pgfpathrectangle{\pgfqpoint{0.329460in}{0.284240in}}{\pgfqpoint{1.989680in}{1.989680in}}%
\pgfusepath{clip}%
\pgfsetbuttcap%
\pgfsetroundjoin%
\definecolor{currentfill}{rgb}{0.277941,0.056324,0.381191}%
\pgfsetfillcolor{currentfill}%
\pgfsetlinewidth{0.000000pt}%
\definecolor{currentstroke}{rgb}{0.000000,0.000000,0.000000}%
\pgfsetstrokecolor{currentstroke}%
\pgfsetdash{}{0pt}%
\pgfpathmoveto{\pgfqpoint{0.973676in}{1.260152in}}%
\pgfpathlineto{\pgfqpoint{0.971436in}{1.264567in}}%
\pgfpathlineto{\pgfqpoint{0.969190in}{1.269259in}}%
\pgfpathlineto{\pgfqpoint{0.966939in}{1.274234in}}%
\pgfpathlineto{\pgfqpoint{0.964682in}{1.279497in}}%
\pgfpathlineto{\pgfqpoint{0.978175in}{1.285515in}}%
\pgfpathlineto{\pgfqpoint{0.992002in}{1.291312in}}%
\pgfpathlineto{\pgfqpoint{1.006148in}{1.296881in}}%
\pgfpathlineto{\pgfqpoint{1.020601in}{1.302220in}}%
\pgfpathlineto{\pgfqpoint{1.022525in}{1.296844in}}%
\pgfpathlineto{\pgfqpoint{1.024444in}{1.291755in}}%
\pgfpathlineto{\pgfqpoint{1.026358in}{1.286948in}}%
\pgfpathlineto{\pgfqpoint{1.028267in}{1.282418in}}%
\pgfpathlineto{\pgfqpoint{1.014156in}{1.277186in}}%
\pgfpathlineto{\pgfqpoint{1.000345in}{1.271728in}}%
\pgfpathlineto{\pgfqpoint{0.986847in}{1.266049in}}%
\pgfpathlineto{\pgfqpoint{0.973676in}{1.260152in}}%
\pgfpathclose%
\pgfusepath{fill}%
\end{pgfscope}%
\begin{pgfscope}%
\pgfpathrectangle{\pgfqpoint{0.329460in}{0.284240in}}{\pgfqpoint{1.989680in}{1.989680in}}%
\pgfusepath{clip}%
\pgfsetbuttcap%
\pgfsetroundjoin%
\definecolor{currentfill}{rgb}{0.282327,0.094955,0.417331}%
\pgfsetfillcolor{currentfill}%
\pgfsetlinewidth{0.000000pt}%
\definecolor{currentstroke}{rgb}{0.000000,0.000000,0.000000}%
\pgfsetstrokecolor{currentstroke}%
\pgfsetdash{}{0pt}%
\pgfpathmoveto{\pgfqpoint{0.964682in}{1.279497in}}%
\pgfpathlineto{\pgfqpoint{0.962418in}{1.285052in}}%
\pgfpathlineto{\pgfqpoint{0.960148in}{1.290905in}}%
\pgfpathlineto{\pgfqpoint{0.957871in}{1.297060in}}%
\pgfpathlineto{\pgfqpoint{0.955588in}{1.303523in}}%
\pgfpathlineto{\pgfqpoint{0.969407in}{1.309659in}}%
\pgfpathlineto{\pgfqpoint{0.983567in}{1.315568in}}%
\pgfpathlineto{\pgfqpoint{0.998053in}{1.321246in}}%
\pgfpathlineto{\pgfqpoint{1.012852in}{1.326689in}}%
\pgfpathlineto{\pgfqpoint{1.014798in}{1.320117in}}%
\pgfpathlineto{\pgfqpoint{1.016738in}{1.313852in}}%
\pgfpathlineto{\pgfqpoint{1.018672in}{1.307888in}}%
\pgfpathlineto{\pgfqpoint{1.020601in}{1.302220in}}%
\pgfpathlineto{\pgfqpoint{1.006148in}{1.296881in}}%
\pgfpathlineto{\pgfqpoint{0.992002in}{1.291312in}}%
\pgfpathlineto{\pgfqpoint{0.978175in}{1.285515in}}%
\pgfpathlineto{\pgfqpoint{0.964682in}{1.279497in}}%
\pgfpathclose%
\pgfusepath{fill}%
\end{pgfscope}%
\begin{pgfscope}%
\pgfpathrectangle{\pgfqpoint{0.329460in}{0.284240in}}{\pgfqpoint{1.989680in}{1.989680in}}%
\pgfusepath{clip}%
\pgfsetbuttcap%
\pgfsetroundjoin%
\definecolor{currentfill}{rgb}{0.279566,0.067836,0.391917}%
\pgfsetfillcolor{currentfill}%
\pgfsetlinewidth{0.000000pt}%
\definecolor{currentstroke}{rgb}{0.000000,0.000000,0.000000}%
\pgfsetstrokecolor{currentstroke}%
\pgfsetdash{}{0pt}%
\pgfpathmoveto{\pgfqpoint{1.187166in}{1.284906in}}%
\pgfpathlineto{\pgfqpoint{1.186031in}{1.283015in}}%
\pgfpathlineto{\pgfqpoint{1.184895in}{1.281271in}}%
\pgfpathlineto{\pgfqpoint{1.183760in}{1.279677in}}%
\pgfpathlineto{\pgfqpoint{1.182623in}{1.278237in}}%
\pgfpathlineto{\pgfqpoint{1.196206in}{1.280802in}}%
\pgfpathlineto{\pgfqpoint{1.209926in}{1.283150in}}%
\pgfpathlineto{\pgfqpoint{1.223773in}{1.285278in}}%
\pgfpathlineto{\pgfqpoint{1.237734in}{1.287186in}}%
\pgfpathlineto{\pgfqpoint{1.238497in}{1.288563in}}%
\pgfpathlineto{\pgfqpoint{1.239261in}{1.290095in}}%
\pgfpathlineto{\pgfqpoint{1.240024in}{1.291776in}}%
\pgfpathlineto{\pgfqpoint{1.240787in}{1.293605in}}%
\pgfpathlineto{\pgfqpoint{1.227203in}{1.291750in}}%
\pgfpathlineto{\pgfqpoint{1.213731in}{1.289682in}}%
\pgfpathlineto{\pgfqpoint{1.200381in}{1.287400in}}%
\pgfpathlineto{\pgfqpoint{1.187166in}{1.284906in}}%
\pgfpathclose%
\pgfusepath{fill}%
\end{pgfscope}%
\begin{pgfscope}%
\pgfpathrectangle{\pgfqpoint{0.329460in}{0.284240in}}{\pgfqpoint{1.989680in}{1.989680in}}%
\pgfusepath{clip}%
\pgfsetbuttcap%
\pgfsetroundjoin%
\definecolor{currentfill}{rgb}{0.272594,0.025563,0.353093}%
\pgfsetfillcolor{currentfill}%
\pgfsetlinewidth{0.000000pt}%
\definecolor{currentstroke}{rgb}{0.000000,0.000000,0.000000}%
\pgfsetstrokecolor{currentstroke}%
\pgfsetdash{}{0pt}%
\pgfpathmoveto{\pgfqpoint{0.982583in}{1.245179in}}%
\pgfpathlineto{\pgfqpoint{0.980364in}{1.248529in}}%
\pgfpathlineto{\pgfqpoint{0.978139in}{1.252138in}}%
\pgfpathlineto{\pgfqpoint{0.975910in}{1.256011in}}%
\pgfpathlineto{\pgfqpoint{0.973676in}{1.260152in}}%
\pgfpathlineto{\pgfqpoint{0.986847in}{1.266049in}}%
\pgfpathlineto{\pgfqpoint{1.000345in}{1.271728in}}%
\pgfpathlineto{\pgfqpoint{1.014156in}{1.277186in}}%
\pgfpathlineto{\pgfqpoint{1.028267in}{1.282418in}}%
\pgfpathlineto{\pgfqpoint{1.030172in}{1.278159in}}%
\pgfpathlineto{\pgfqpoint{1.032072in}{1.274169in}}%
\pgfpathlineto{\pgfqpoint{1.033968in}{1.270442in}}%
\pgfpathlineto{\pgfqpoint{1.035860in}{1.266973in}}%
\pgfpathlineto{\pgfqpoint{1.022086in}{1.261851in}}%
\pgfpathlineto{\pgfqpoint{1.008607in}{1.256509in}}%
\pgfpathlineto{\pgfqpoint{0.995435in}{1.250950in}}%
\pgfpathlineto{\pgfqpoint{0.982583in}{1.245179in}}%
\pgfpathclose%
\pgfusepath{fill}%
\end{pgfscope}%
\begin{pgfscope}%
\pgfpathrectangle{\pgfqpoint{0.329460in}{0.284240in}}{\pgfqpoint{1.989680in}{1.989680in}}%
\pgfusepath{clip}%
\pgfsetbuttcap%
\pgfsetroundjoin%
\definecolor{currentfill}{rgb}{0.282327,0.094955,0.417331}%
\pgfsetfillcolor{currentfill}%
\pgfsetlinewidth{0.000000pt}%
\definecolor{currentstroke}{rgb}{0.000000,0.000000,0.000000}%
\pgfsetstrokecolor{currentstroke}%
\pgfsetdash{}{0pt}%
\pgfpathmoveto{\pgfqpoint{1.297508in}{1.307407in}}%
\pgfpathlineto{\pgfqpoint{1.297127in}{1.305065in}}%
\pgfpathlineto{\pgfqpoint{1.296746in}{1.302855in}}%
\pgfpathlineto{\pgfqpoint{1.296365in}{1.300782in}}%
\pgfpathlineto{\pgfqpoint{1.295984in}{1.298848in}}%
\pgfpathlineto{\pgfqpoint{1.309937in}{1.299610in}}%
\pgfpathlineto{\pgfqpoint{1.323927in}{1.300151in}}%
\pgfpathlineto{\pgfqpoint{1.337941in}{1.300472in}}%
\pgfpathlineto{\pgfqpoint{1.351967in}{1.300570in}}%
\pgfpathlineto{\pgfqpoint{1.351962in}{1.302491in}}%
\pgfpathlineto{\pgfqpoint{1.351956in}{1.304552in}}%
\pgfpathlineto{\pgfqpoint{1.351951in}{1.306749in}}%
\pgfpathlineto{\pgfqpoint{1.351945in}{1.309079in}}%
\pgfpathlineto{\pgfqpoint{1.338307in}{1.308984in}}%
\pgfpathlineto{\pgfqpoint{1.324679in}{1.308673in}}%
\pgfpathlineto{\pgfqpoint{1.311076in}{1.308147in}}%
\pgfpathlineto{\pgfqpoint{1.297508in}{1.307407in}}%
\pgfpathclose%
\pgfusepath{fill}%
\end{pgfscope}%
\begin{pgfscope}%
\pgfpathrectangle{\pgfqpoint{0.329460in}{0.284240in}}{\pgfqpoint{1.989680in}{1.989680in}}%
\pgfusepath{clip}%
\pgfsetbuttcap%
\pgfsetroundjoin%
\definecolor{currentfill}{rgb}{0.282327,0.094955,0.417331}%
\pgfsetfillcolor{currentfill}%
\pgfsetlinewidth{0.000000pt}%
\definecolor{currentstroke}{rgb}{0.000000,0.000000,0.000000}%
\pgfsetstrokecolor{currentstroke}%
\pgfsetdash{}{0pt}%
\pgfpathmoveto{\pgfqpoint{1.351945in}{1.309079in}}%
\pgfpathlineto{\pgfqpoint{1.351951in}{1.306749in}}%
\pgfpathlineto{\pgfqpoint{1.351956in}{1.304552in}}%
\pgfpathlineto{\pgfqpoint{1.351962in}{1.302491in}}%
\pgfpathlineto{\pgfqpoint{1.351967in}{1.300570in}}%
\pgfpathlineto{\pgfqpoint{1.365992in}{1.300447in}}%
\pgfpathlineto{\pgfqpoint{1.380004in}{1.300102in}}%
\pgfpathlineto{\pgfqpoint{1.393991in}{1.299536in}}%
\pgfpathlineto{\pgfqpoint{1.407939in}{1.298750in}}%
\pgfpathlineto{\pgfqpoint{1.407547in}{1.300684in}}%
\pgfpathlineto{\pgfqpoint{1.407155in}{1.302758in}}%
\pgfpathlineto{\pgfqpoint{1.406764in}{1.304968in}}%
\pgfpathlineto{\pgfqpoint{1.406372in}{1.307312in}}%
\pgfpathlineto{\pgfqpoint{1.392809in}{1.308076in}}%
\pgfpathlineto{\pgfqpoint{1.379209in}{1.308625in}}%
\pgfpathlineto{\pgfqpoint{1.365583in}{1.308960in}}%
\pgfpathlineto{\pgfqpoint{1.351945in}{1.309079in}}%
\pgfpathclose%
\pgfusepath{fill}%
\end{pgfscope}%
\begin{pgfscope}%
\pgfpathrectangle{\pgfqpoint{0.329460in}{0.284240in}}{\pgfqpoint{1.989680in}{1.989680in}}%
\pgfusepath{clip}%
\pgfsetbuttcap%
\pgfsetroundjoin%
\definecolor{currentfill}{rgb}{0.274952,0.037752,0.364543}%
\pgfsetfillcolor{currentfill}%
\pgfsetlinewidth{0.000000pt}%
\definecolor{currentstroke}{rgb}{0.000000,0.000000,0.000000}%
\pgfsetstrokecolor{currentstroke}%
\pgfsetdash{}{0pt}%
\pgfpathmoveto{\pgfqpoint{1.129922in}{1.265847in}}%
\pgfpathlineto{\pgfqpoint{1.128427in}{1.264478in}}%
\pgfpathlineto{\pgfqpoint{1.126931in}{1.263270in}}%
\pgfpathlineto{\pgfqpoint{1.125434in}{1.262227in}}%
\pgfpathlineto{\pgfqpoint{1.123936in}{1.261353in}}%
\pgfpathlineto{\pgfqpoint{1.137195in}{1.264860in}}%
\pgfpathlineto{\pgfqpoint{1.150646in}{1.268153in}}%
\pgfpathlineto{\pgfqpoint{1.164276in}{1.271230in}}%
\pgfpathlineto{\pgfqpoint{1.178072in}{1.274087in}}%
\pgfpathlineto{\pgfqpoint{1.179211in}{1.274876in}}%
\pgfpathlineto{\pgfqpoint{1.180349in}{1.275833in}}%
\pgfpathlineto{\pgfqpoint{1.181486in}{1.276955in}}%
\pgfpathlineto{\pgfqpoint{1.182623in}{1.278237in}}%
\pgfpathlineto{\pgfqpoint{1.169192in}{1.275457in}}%
\pgfpathlineto{\pgfqpoint{1.155923in}{1.272463in}}%
\pgfpathlineto{\pgfqpoint{1.142829in}{1.269259in}}%
\pgfpathlineto{\pgfqpoint{1.129922in}{1.265847in}}%
\pgfpathclose%
\pgfusepath{fill}%
\end{pgfscope}%
\begin{pgfscope}%
\pgfpathrectangle{\pgfqpoint{0.329460in}{0.284240in}}{\pgfqpoint{1.989680in}{1.989680in}}%
\pgfusepath{clip}%
\pgfsetbuttcap%
\pgfsetroundjoin%
\definecolor{currentfill}{rgb}{0.276194,0.190074,0.493001}%
\pgfsetfillcolor{currentfill}%
\pgfsetlinewidth{0.000000pt}%
\definecolor{currentstroke}{rgb}{0.000000,0.000000,0.000000}%
\pgfsetstrokecolor{currentstroke}%
\pgfsetdash{}{0pt}%
\pgfpathmoveto{\pgfqpoint{1.683644in}{1.360868in}}%
\pgfpathlineto{\pgfqpoint{1.685542in}{1.369073in}}%
\pgfpathlineto{\pgfqpoint{1.687446in}{1.377615in}}%
\pgfpathlineto{\pgfqpoint{1.689357in}{1.386502in}}%
\pgfpathlineto{\pgfqpoint{1.691275in}{1.395737in}}%
\pgfpathlineto{\pgfqpoint{1.707057in}{1.390320in}}%
\pgfpathlineto{\pgfqpoint{1.722527in}{1.384655in}}%
\pgfpathlineto{\pgfqpoint{1.737670in}{1.378746in}}%
\pgfpathlineto{\pgfqpoint{1.752471in}{1.372597in}}%
\pgfpathlineto{\pgfqpoint{1.750199in}{1.363458in}}%
\pgfpathlineto{\pgfqpoint{1.747936in}{1.354669in}}%
\pgfpathlineto{\pgfqpoint{1.745681in}{1.346225in}}%
\pgfpathlineto{\pgfqpoint{1.743434in}{1.338120in}}%
\pgfpathlineto{\pgfqpoint{1.728975in}{1.344165in}}%
\pgfpathlineto{\pgfqpoint{1.714181in}{1.349974in}}%
\pgfpathlineto{\pgfqpoint{1.699066in}{1.355543in}}%
\pgfpathlineto{\pgfqpoint{1.683644in}{1.360868in}}%
\pgfpathclose%
\pgfusepath{fill}%
\end{pgfscope}%
\begin{pgfscope}%
\pgfpathrectangle{\pgfqpoint{0.329460in}{0.284240in}}{\pgfqpoint{1.989680in}{1.989680in}}%
\pgfusepath{clip}%
\pgfsetbuttcap%
\pgfsetroundjoin%
\definecolor{currentfill}{rgb}{0.201239,0.383670,0.554294}%
\pgfsetfillcolor{currentfill}%
\pgfsetlinewidth{0.000000pt}%
\definecolor{currentstroke}{rgb}{0.000000,0.000000,0.000000}%
\pgfsetstrokecolor{currentstroke}%
\pgfsetdash{}{0pt}%
\pgfpathmoveto{\pgfqpoint{1.637864in}{1.502606in}}%
\pgfpathlineto{\pgfqpoint{1.639463in}{1.515324in}}%
\pgfpathlineto{\pgfqpoint{1.641070in}{1.528451in}}%
\pgfpathlineto{\pgfqpoint{1.642683in}{1.541995in}}%
\pgfpathlineto{\pgfqpoint{1.644305in}{1.555961in}}%
\pgfpathlineto{\pgfqpoint{1.662375in}{1.551381in}}%
\pgfpathlineto{\pgfqpoint{1.680177in}{1.546525in}}%
\pgfpathlineto{\pgfqpoint{1.697695in}{1.541397in}}%
\pgfpathlineto{\pgfqpoint{1.714913in}{1.536000in}}%
\pgfpathlineto{\pgfqpoint{1.712895in}{1.522098in}}%
\pgfpathlineto{\pgfqpoint{1.710885in}{1.508621in}}%
\pgfpathlineto{\pgfqpoint{1.708886in}{1.495561in}}%
\pgfpathlineto{\pgfqpoint{1.706895in}{1.482911in}}%
\pgfpathlineto{\pgfqpoint{1.690063in}{1.488236in}}%
\pgfpathlineto{\pgfqpoint{1.672937in}{1.493295in}}%
\pgfpathlineto{\pgfqpoint{1.655532in}{1.498087in}}%
\pgfpathlineto{\pgfqpoint{1.637864in}{1.502606in}}%
\pgfpathclose%
\pgfusepath{fill}%
\end{pgfscope}%
\begin{pgfscope}%
\pgfpathrectangle{\pgfqpoint{0.329460in}{0.284240in}}{\pgfqpoint{1.989680in}{1.989680in}}%
\pgfusepath{clip}%
\pgfsetbuttcap%
\pgfsetroundjoin%
\definecolor{currentfill}{rgb}{0.267004,0.004874,0.329415}%
\pgfsetfillcolor{currentfill}%
\pgfsetlinewidth{0.000000pt}%
\definecolor{currentstroke}{rgb}{0.000000,0.000000,0.000000}%
\pgfsetstrokecolor{currentstroke}%
\pgfsetdash{}{0pt}%
\pgfpathmoveto{\pgfqpoint{1.639634in}{1.252161in}}%
\pgfpathlineto{\pgfqpoint{1.641423in}{1.253746in}}%
\pgfpathlineto{\pgfqpoint{1.643216in}{1.255554in}}%
\pgfpathlineto{\pgfqpoint{1.645011in}{1.257590in}}%
\pgfpathlineto{\pgfqpoint{1.646810in}{1.259858in}}%
\pgfpathlineto{\pgfqpoint{1.660493in}{1.255046in}}%
\pgfpathlineto{\pgfqpoint{1.673899in}{1.250014in}}%
\pgfpathlineto{\pgfqpoint{1.687017in}{1.244766in}}%
\pgfpathlineto{\pgfqpoint{1.699832in}{1.239307in}}%
\pgfpathlineto{\pgfqpoint{1.697704in}{1.237157in}}%
\pgfpathlineto{\pgfqpoint{1.695580in}{1.235240in}}%
\pgfpathlineto{\pgfqpoint{1.693459in}{1.233551in}}%
\pgfpathlineto{\pgfqpoint{1.691342in}{1.232086in}}%
\pgfpathlineto{\pgfqpoint{1.678846in}{1.237418in}}%
\pgfpathlineto{\pgfqpoint{1.666054in}{1.242544in}}%
\pgfpathlineto{\pgfqpoint{1.652979in}{1.247460in}}%
\pgfpathlineto{\pgfqpoint{1.639634in}{1.252161in}}%
\pgfpathclose%
\pgfusepath{fill}%
\end{pgfscope}%
\begin{pgfscope}%
\pgfpathrectangle{\pgfqpoint{0.329460in}{0.284240in}}{\pgfqpoint{1.989680in}{1.989680in}}%
\pgfusepath{clip}%
\pgfsetbuttcap%
\pgfsetroundjoin%
\definecolor{currentfill}{rgb}{0.282884,0.135920,0.453427}%
\pgfsetfillcolor{currentfill}%
\pgfsetlinewidth{0.000000pt}%
\definecolor{currentstroke}{rgb}{0.000000,0.000000,0.000000}%
\pgfsetstrokecolor{currentstroke}%
\pgfsetdash{}{0pt}%
\pgfpathmoveto{\pgfqpoint{0.955588in}{1.303523in}}%
\pgfpathlineto{\pgfqpoint{0.953297in}{1.310298in}}%
\pgfpathlineto{\pgfqpoint{0.950999in}{1.317392in}}%
\pgfpathlineto{\pgfqpoint{0.948694in}{1.324809in}}%
\pgfpathlineto{\pgfqpoint{0.946381in}{1.332554in}}%
\pgfpathlineto{\pgfqpoint{0.960531in}{1.338804in}}%
\pgfpathlineto{\pgfqpoint{0.975028in}{1.344822in}}%
\pgfpathlineto{\pgfqpoint{0.989858in}{1.350604in}}%
\pgfpathlineto{\pgfqpoint{1.005008in}{1.356147in}}%
\pgfpathlineto{\pgfqpoint{1.006979in}{1.348296in}}%
\pgfpathlineto{\pgfqpoint{1.008943in}{1.340773in}}%
\pgfpathlineto{\pgfqpoint{1.010901in}{1.333573in}}%
\pgfpathlineto{\pgfqpoint{1.012852in}{1.326689in}}%
\pgfpathlineto{\pgfqpoint{0.998053in}{1.321246in}}%
\pgfpathlineto{\pgfqpoint{0.983567in}{1.315568in}}%
\pgfpathlineto{\pgfqpoint{0.969407in}{1.309659in}}%
\pgfpathlineto{\pgfqpoint{0.955588in}{1.303523in}}%
\pgfpathclose%
\pgfusepath{fill}%
\end{pgfscope}%
\begin{pgfscope}%
\pgfpathrectangle{\pgfqpoint{0.329460in}{0.284240in}}{\pgfqpoint{1.989680in}{1.989680in}}%
\pgfusepath{clip}%
\pgfsetbuttcap%
\pgfsetroundjoin%
\definecolor{currentfill}{rgb}{0.268510,0.009605,0.335427}%
\pgfsetfillcolor{currentfill}%
\pgfsetlinewidth{0.000000pt}%
\definecolor{currentstroke}{rgb}{0.000000,0.000000,0.000000}%
\pgfsetstrokecolor{currentstroke}%
\pgfsetdash{}{0pt}%
\pgfpathmoveto{\pgfqpoint{0.991415in}{1.234281in}}%
\pgfpathlineto{\pgfqpoint{0.989213in}{1.236639in}}%
\pgfpathlineto{\pgfqpoint{0.987008in}{1.239238in}}%
\pgfpathlineto{\pgfqpoint{0.984798in}{1.242083in}}%
\pgfpathlineto{\pgfqpoint{0.982583in}{1.245179in}}%
\pgfpathlineto{\pgfqpoint{0.995435in}{1.250950in}}%
\pgfpathlineto{\pgfqpoint{1.008607in}{1.256509in}}%
\pgfpathlineto{\pgfqpoint{1.022086in}{1.261851in}}%
\pgfpathlineto{\pgfqpoint{1.035860in}{1.266973in}}%
\pgfpathlineto{\pgfqpoint{1.037748in}{1.263757in}}%
\pgfpathlineto{\pgfqpoint{1.039632in}{1.260792in}}%
\pgfpathlineto{\pgfqpoint{1.041512in}{1.258071in}}%
\pgfpathlineto{\pgfqpoint{1.043389in}{1.255591in}}%
\pgfpathlineto{\pgfqpoint{1.029951in}{1.250583in}}%
\pgfpathlineto{\pgfqpoint{1.016801in}{1.245360in}}%
\pgfpathlineto{\pgfqpoint{1.003952in}{1.239924in}}%
\pgfpathlineto{\pgfqpoint{0.991415in}{1.234281in}}%
\pgfpathclose%
\pgfusepath{fill}%
\end{pgfscope}%
\begin{pgfscope}%
\pgfpathrectangle{\pgfqpoint{0.329460in}{0.284240in}}{\pgfqpoint{1.989680in}{1.989680in}}%
\pgfusepath{clip}%
\pgfsetbuttcap%
\pgfsetroundjoin%
\definecolor{currentfill}{rgb}{0.271305,0.019942,0.347269}%
\pgfsetfillcolor{currentfill}%
\pgfsetlinewidth{0.000000pt}%
\definecolor{currentstroke}{rgb}{0.000000,0.000000,0.000000}%
\pgfsetstrokecolor{currentstroke}%
\pgfsetdash{}{0pt}%
\pgfpathmoveto{\pgfqpoint{1.566663in}{1.264481in}}%
\pgfpathlineto{\pgfqpoint{1.568083in}{1.263800in}}%
\pgfpathlineto{\pgfqpoint{1.569505in}{1.263295in}}%
\pgfpathlineto{\pgfqpoint{1.570928in}{1.262969in}}%
\pgfpathlineto{\pgfqpoint{1.572352in}{1.262827in}}%
\pgfpathlineto{\pgfqpoint{1.585942in}{1.259202in}}%
\pgfpathlineto{\pgfqpoint{1.599322in}{1.255360in}}%
\pgfpathlineto{\pgfqpoint{1.612479in}{1.251304in}}%
\pgfpathlineto{\pgfqpoint{1.625402in}{1.247038in}}%
\pgfpathlineto{\pgfqpoint{1.623632in}{1.247283in}}%
\pgfpathlineto{\pgfqpoint{1.621864in}{1.247711in}}%
\pgfpathlineto{\pgfqpoint{1.620097in}{1.248320in}}%
\pgfpathlineto{\pgfqpoint{1.618331in}{1.249104in}}%
\pgfpathlineto{\pgfqpoint{1.605746in}{1.253259in}}%
\pgfpathlineto{\pgfqpoint{1.592932in}{1.257208in}}%
\pgfpathlineto{\pgfqpoint{1.579900in}{1.260950in}}%
\pgfpathlineto{\pgfqpoint{1.566663in}{1.264481in}}%
\pgfpathclose%
\pgfusepath{fill}%
\end{pgfscope}%
\begin{pgfscope}%
\pgfpathrectangle{\pgfqpoint{0.329460in}{0.284240in}}{\pgfqpoint{1.989680in}{1.989680in}}%
\pgfusepath{clip}%
\pgfsetbuttcap%
\pgfsetroundjoin%
\definecolor{currentfill}{rgb}{0.282327,0.094955,0.417331}%
\pgfsetfillcolor{currentfill}%
\pgfsetlinewidth{0.000000pt}%
\definecolor{currentstroke}{rgb}{0.000000,0.000000,0.000000}%
\pgfsetstrokecolor{currentstroke}%
\pgfsetdash{}{0pt}%
\pgfpathmoveto{\pgfqpoint{1.243838in}{1.302316in}}%
\pgfpathlineto{\pgfqpoint{1.243075in}{1.299935in}}%
\pgfpathlineto{\pgfqpoint{1.242313in}{1.297687in}}%
\pgfpathlineto{\pgfqpoint{1.241550in}{1.295576in}}%
\pgfpathlineto{\pgfqpoint{1.240787in}{1.293605in}}%
\pgfpathlineto{\pgfqpoint{1.254469in}{1.295243in}}%
\pgfpathlineto{\pgfqpoint{1.268238in}{1.296663in}}%
\pgfpathlineto{\pgfqpoint{1.282080in}{1.297865in}}%
\pgfpathlineto{\pgfqpoint{1.295984in}{1.298848in}}%
\pgfpathlineto{\pgfqpoint{1.296365in}{1.300782in}}%
\pgfpathlineto{\pgfqpoint{1.296746in}{1.302855in}}%
\pgfpathlineto{\pgfqpoint{1.297127in}{1.305065in}}%
\pgfpathlineto{\pgfqpoint{1.297508in}{1.307407in}}%
\pgfpathlineto{\pgfqpoint{1.283988in}{1.306453in}}%
\pgfpathlineto{\pgfqpoint{1.270529in}{1.305286in}}%
\pgfpathlineto{\pgfqpoint{1.257141in}{1.303906in}}%
\pgfpathlineto{\pgfqpoint{1.243838in}{1.302316in}}%
\pgfpathclose%
\pgfusepath{fill}%
\end{pgfscope}%
\begin{pgfscope}%
\pgfpathrectangle{\pgfqpoint{0.329460in}{0.284240in}}{\pgfqpoint{1.989680in}{1.989680in}}%
\pgfusepath{clip}%
\pgfsetbuttcap%
\pgfsetroundjoin%
\definecolor{currentfill}{rgb}{0.282327,0.094955,0.417331}%
\pgfsetfillcolor{currentfill}%
\pgfsetlinewidth{0.000000pt}%
\definecolor{currentstroke}{rgb}{0.000000,0.000000,0.000000}%
\pgfsetstrokecolor{currentstroke}%
\pgfsetdash{}{0pt}%
\pgfpathmoveto{\pgfqpoint{1.406372in}{1.307312in}}%
\pgfpathlineto{\pgfqpoint{1.406764in}{1.304968in}}%
\pgfpathlineto{\pgfqpoint{1.407155in}{1.302758in}}%
\pgfpathlineto{\pgfqpoint{1.407547in}{1.300684in}}%
\pgfpathlineto{\pgfqpoint{1.407939in}{1.298750in}}%
\pgfpathlineto{\pgfqpoint{1.421837in}{1.297743in}}%
\pgfpathlineto{\pgfqpoint{1.435671in}{1.296516in}}%
\pgfpathlineto{\pgfqpoint{1.449431in}{1.295071in}}%
\pgfpathlineto{\pgfqpoint{1.463103in}{1.293409in}}%
\pgfpathlineto{\pgfqpoint{1.462329in}{1.295382in}}%
\pgfpathlineto{\pgfqpoint{1.461556in}{1.297495in}}%
\pgfpathlineto{\pgfqpoint{1.460783in}{1.299744in}}%
\pgfpathlineto{\pgfqpoint{1.460010in}{1.302126in}}%
\pgfpathlineto{\pgfqpoint{1.446717in}{1.303740in}}%
\pgfpathlineto{\pgfqpoint{1.433338in}{1.305143in}}%
\pgfpathlineto{\pgfqpoint{1.419886in}{1.306334in}}%
\pgfpathlineto{\pgfqpoint{1.406372in}{1.307312in}}%
\pgfpathclose%
\pgfusepath{fill}%
\end{pgfscope}%
\begin{pgfscope}%
\pgfpathrectangle{\pgfqpoint{0.329460in}{0.284240in}}{\pgfqpoint{1.989680in}{1.989680in}}%
\pgfusepath{clip}%
\pgfsetbuttcap%
\pgfsetroundjoin%
\definecolor{currentfill}{rgb}{0.201239,0.383670,0.554294}%
\pgfsetfillcolor{currentfill}%
\pgfsetlinewidth{0.000000pt}%
\definecolor{currentstroke}{rgb}{0.000000,0.000000,0.000000}%
\pgfsetstrokecolor{currentstroke}%
\pgfsetdash{}{0pt}%
\pgfpathmoveto{\pgfqpoint{0.980777in}{1.477959in}}%
\pgfpathlineto{\pgfqpoint{0.978703in}{1.490592in}}%
\pgfpathlineto{\pgfqpoint{0.976619in}{1.503635in}}%
\pgfpathlineto{\pgfqpoint{0.974525in}{1.517096in}}%
\pgfpathlineto{\pgfqpoint{0.972421in}{1.530981in}}%
\pgfpathlineto{\pgfqpoint{0.989360in}{1.536613in}}%
\pgfpathlineto{\pgfqpoint{1.006612in}{1.541980in}}%
\pgfpathlineto{\pgfqpoint{1.024163in}{1.547078in}}%
\pgfpathlineto{\pgfqpoint{1.041995in}{1.551903in}}%
\pgfpathlineto{\pgfqpoint{1.043707in}{1.537950in}}%
\pgfpathlineto{\pgfqpoint{1.045410in}{1.524420in}}%
\pgfpathlineto{\pgfqpoint{1.047106in}{1.511306in}}%
\pgfpathlineto{\pgfqpoint{1.048793in}{1.498603in}}%
\pgfpathlineto{\pgfqpoint{1.031359in}{1.493841in}}%
\pgfpathlineto{\pgfqpoint{1.014201in}{1.488811in}}%
\pgfpathlineto{\pgfqpoint{0.997335in}{1.483516in}}%
\pgfpathlineto{\pgfqpoint{0.980777in}{1.477959in}}%
\pgfpathclose%
\pgfusepath{fill}%
\end{pgfscope}%
\begin{pgfscope}%
\pgfpathrectangle{\pgfqpoint{0.329460in}{0.284240in}}{\pgfqpoint{1.989680in}{1.989680in}}%
\pgfusepath{clip}%
\pgfsetbuttcap%
\pgfsetroundjoin%
\definecolor{currentfill}{rgb}{0.276194,0.190074,0.493001}%
\pgfsetfillcolor{currentfill}%
\pgfsetlinewidth{0.000000pt}%
\definecolor{currentstroke}{rgb}{0.000000,0.000000,0.000000}%
\pgfsetstrokecolor{currentstroke}%
\pgfsetdash{}{0pt}%
\pgfpathmoveto{\pgfqpoint{0.946381in}{1.332554in}}%
\pgfpathlineto{\pgfqpoint{0.944060in}{1.340634in}}%
\pgfpathlineto{\pgfqpoint{0.941730in}{1.349053in}}%
\pgfpathlineto{\pgfqpoint{0.939392in}{1.357818in}}%
\pgfpathlineto{\pgfqpoint{0.937046in}{1.366934in}}%
\pgfpathlineto{\pgfqpoint{0.951532in}{1.373292in}}%
\pgfpathlineto{\pgfqpoint{0.966371in}{1.379414in}}%
\pgfpathlineto{\pgfqpoint{0.981551in}{1.385297in}}%
\pgfpathlineto{\pgfqpoint{0.997056in}{1.390935in}}%
\pgfpathlineto{\pgfqpoint{0.999055in}{1.381719in}}%
\pgfpathlineto{\pgfqpoint{1.001046in}{1.372853in}}%
\pgfpathlineto{\pgfqpoint{1.003031in}{1.364330in}}%
\pgfpathlineto{\pgfqpoint{1.005008in}{1.356147in}}%
\pgfpathlineto{\pgfqpoint{0.989858in}{1.350604in}}%
\pgfpathlineto{\pgfqpoint{0.975028in}{1.344822in}}%
\pgfpathlineto{\pgfqpoint{0.960531in}{1.338804in}}%
\pgfpathlineto{\pgfqpoint{0.946381in}{1.332554in}}%
\pgfpathclose%
\pgfusepath{fill}%
\end{pgfscope}%
\begin{pgfscope}%
\pgfpathrectangle{\pgfqpoint{0.329460in}{0.284240in}}{\pgfqpoint{1.989680in}{1.989680in}}%
\pgfusepath{clip}%
\pgfsetbuttcap%
\pgfsetroundjoin%
\definecolor{currentfill}{rgb}{0.267004,0.004874,0.329415}%
\pgfsetfillcolor{currentfill}%
\pgfsetlinewidth{0.000000pt}%
\definecolor{currentstroke}{rgb}{0.000000,0.000000,0.000000}%
\pgfsetstrokecolor{currentstroke}%
\pgfsetdash{}{0pt}%
\pgfpathmoveto{\pgfqpoint{1.000183in}{1.227176in}}%
\pgfpathlineto{\pgfqpoint{0.997996in}{1.228612in}}%
\pgfpathlineto{\pgfqpoint{0.995806in}{1.230272in}}%
\pgfpathlineto{\pgfqpoint{0.993612in}{1.232160in}}%
\pgfpathlineto{\pgfqpoint{0.991415in}{1.234281in}}%
\pgfpathlineto{\pgfqpoint{1.003952in}{1.239924in}}%
\pgfpathlineto{\pgfqpoint{1.016801in}{1.245360in}}%
\pgfpathlineto{\pgfqpoint{1.029951in}{1.250583in}}%
\pgfpathlineto{\pgfqpoint{1.043389in}{1.255591in}}%
\pgfpathlineto{\pgfqpoint{1.045263in}{1.253348in}}%
\pgfpathlineto{\pgfqpoint{1.047134in}{1.251336in}}%
\pgfpathlineto{\pgfqpoint{1.049001in}{1.249553in}}%
\pgfpathlineto{\pgfqpoint{1.050866in}{1.247993in}}%
\pgfpathlineto{\pgfqpoint{1.037760in}{1.243100in}}%
\pgfpathlineto{\pgfqpoint{1.024936in}{1.237998in}}%
\pgfpathlineto{\pgfqpoint{1.012407in}{1.232688in}}%
\pgfpathlineto{\pgfqpoint{1.000183in}{1.227176in}}%
\pgfpathclose%
\pgfusepath{fill}%
\end{pgfscope}%
\begin{pgfscope}%
\pgfpathrectangle{\pgfqpoint{0.329460in}{0.284240in}}{\pgfqpoint{1.989680in}{1.989680in}}%
\pgfusepath{clip}%
\pgfsetbuttcap%
\pgfsetroundjoin%
\definecolor{currentfill}{rgb}{0.271305,0.019942,0.347269}%
\pgfsetfillcolor{currentfill}%
\pgfsetlinewidth{0.000000pt}%
\definecolor{currentstroke}{rgb}{0.000000,0.000000,0.000000}%
\pgfsetstrokecolor{currentstroke}%
\pgfsetdash{}{0pt}%
\pgfpathmoveto{\pgfqpoint{1.073059in}{1.245242in}}%
\pgfpathlineto{\pgfqpoint{1.071219in}{1.244431in}}%
\pgfpathlineto{\pgfqpoint{1.069379in}{1.243797in}}%
\pgfpathlineto{\pgfqpoint{1.067536in}{1.243343in}}%
\pgfpathlineto{\pgfqpoint{1.065692in}{1.243073in}}%
\pgfpathlineto{\pgfqpoint{1.078397in}{1.247523in}}%
\pgfpathlineto{\pgfqpoint{1.091346in}{1.251766in}}%
\pgfpathlineto{\pgfqpoint{1.104529in}{1.255797in}}%
\pgfpathlineto{\pgfqpoint{1.117933in}{1.259615in}}%
\pgfpathlineto{\pgfqpoint{1.119436in}{1.259778in}}%
\pgfpathlineto{\pgfqpoint{1.120937in}{1.260125in}}%
\pgfpathlineto{\pgfqpoint{1.122437in}{1.260651in}}%
\pgfpathlineto{\pgfqpoint{1.123936in}{1.261353in}}%
\pgfpathlineto{\pgfqpoint{1.110881in}{1.257635in}}%
\pgfpathlineto{\pgfqpoint{1.098042in}{1.253708in}}%
\pgfpathlineto{\pgfqpoint{1.085430in}{1.249576in}}%
\pgfpathlineto{\pgfqpoint{1.073059in}{1.245242in}}%
\pgfpathclose%
\pgfusepath{fill}%
\end{pgfscope}%
\begin{pgfscope}%
\pgfpathrectangle{\pgfqpoint{0.329460in}{0.284240in}}{\pgfqpoint{1.989680in}{1.989680in}}%
\pgfusepath{clip}%
\pgfsetbuttcap%
\pgfsetroundjoin%
\definecolor{currentfill}{rgb}{0.260571,0.246922,0.522828}%
\pgfsetfillcolor{currentfill}%
\pgfsetlinewidth{0.000000pt}%
\definecolor{currentstroke}{rgb}{0.000000,0.000000,0.000000}%
\pgfsetstrokecolor{currentstroke}%
\pgfsetdash{}{0pt}%
\pgfpathmoveto{\pgfqpoint{1.691275in}{1.395737in}}%
\pgfpathlineto{\pgfqpoint{1.693200in}{1.405328in}}%
\pgfpathlineto{\pgfqpoint{1.695132in}{1.415281in}}%
\pgfpathlineto{\pgfqpoint{1.697073in}{1.425600in}}%
\pgfpathlineto{\pgfqpoint{1.699021in}{1.436292in}}%
\pgfpathlineto{\pgfqpoint{1.715170in}{1.430788in}}%
\pgfpathlineto{\pgfqpoint{1.731000in}{1.425032in}}%
\pgfpathlineto{\pgfqpoint{1.746497in}{1.419027in}}%
\pgfpathlineto{\pgfqpoint{1.761646in}{1.412779in}}%
\pgfpathlineto{\pgfqpoint{1.759338in}{1.402178in}}%
\pgfpathlineto{\pgfqpoint{1.757040in}{1.391952in}}%
\pgfpathlineto{\pgfqpoint{1.754751in}{1.382093in}}%
\pgfpathlineto{\pgfqpoint{1.752471in}{1.372597in}}%
\pgfpathlineto{\pgfqpoint{1.737670in}{1.378746in}}%
\pgfpathlineto{\pgfqpoint{1.722527in}{1.384655in}}%
\pgfpathlineto{\pgfqpoint{1.707057in}{1.390320in}}%
\pgfpathlineto{\pgfqpoint{1.691275in}{1.395737in}}%
\pgfpathclose%
\pgfusepath{fill}%
\end{pgfscope}%
\begin{pgfscope}%
\pgfpathrectangle{\pgfqpoint{0.329460in}{0.284240in}}{\pgfqpoint{1.989680in}{1.989680in}}%
\pgfusepath{clip}%
\pgfsetbuttcap%
\pgfsetroundjoin%
\definecolor{currentfill}{rgb}{0.267004,0.004874,0.329415}%
\pgfsetfillcolor{currentfill}%
\pgfsetlinewidth{0.000000pt}%
\definecolor{currentstroke}{rgb}{0.000000,0.000000,0.000000}%
\pgfsetstrokecolor{currentstroke}%
\pgfsetdash{}{0pt}%
\pgfpathmoveto{\pgfqpoint{1.632501in}{1.247974in}}%
\pgfpathlineto{\pgfqpoint{1.634280in}{1.248706in}}%
\pgfpathlineto{\pgfqpoint{1.636062in}{1.249645in}}%
\pgfpathlineto{\pgfqpoint{1.637847in}{1.250796in}}%
\pgfpathlineto{\pgfqpoint{1.639634in}{1.252161in}}%
\pgfpathlineto{\pgfqpoint{1.652979in}{1.247460in}}%
\pgfpathlineto{\pgfqpoint{1.666054in}{1.242544in}}%
\pgfpathlineto{\pgfqpoint{1.678846in}{1.237418in}}%
\pgfpathlineto{\pgfqpoint{1.691342in}{1.232086in}}%
\pgfpathlineto{\pgfqpoint{1.689228in}{1.230841in}}%
\pgfpathlineto{\pgfqpoint{1.687117in}{1.229811in}}%
\pgfpathlineto{\pgfqpoint{1.685009in}{1.228994in}}%
\pgfpathlineto{\pgfqpoint{1.682904in}{1.228384in}}%
\pgfpathlineto{\pgfqpoint{1.670724in}{1.233587in}}%
\pgfpathlineto{\pgfqpoint{1.658256in}{1.238589in}}%
\pgfpathlineto{\pgfqpoint{1.645511in}{1.243386in}}%
\pgfpathlineto{\pgfqpoint{1.632501in}{1.247974in}}%
\pgfpathclose%
\pgfusepath{fill}%
\end{pgfscope}%
\begin{pgfscope}%
\pgfpathrectangle{\pgfqpoint{0.329460in}{0.284240in}}{\pgfqpoint{1.989680in}{1.989680in}}%
\pgfusepath{clip}%
\pgfsetbuttcap%
\pgfsetroundjoin%
\definecolor{currentfill}{rgb}{0.279566,0.067836,0.391917}%
\pgfsetfillcolor{currentfill}%
\pgfsetlinewidth{0.000000pt}%
\definecolor{currentstroke}{rgb}{0.000000,0.000000,0.000000}%
\pgfsetstrokecolor{currentstroke}%
\pgfsetdash{}{0pt}%
\pgfpathmoveto{\pgfqpoint{1.503469in}{1.287133in}}%
\pgfpathlineto{\pgfqpoint{1.504523in}{1.285258in}}%
\pgfpathlineto{\pgfqpoint{1.505577in}{1.283530in}}%
\pgfpathlineto{\pgfqpoint{1.506631in}{1.281952in}}%
\pgfpathlineto{\pgfqpoint{1.507686in}{1.280528in}}%
\pgfpathlineto{\pgfqpoint{1.521252in}{1.277939in}}%
\pgfpathlineto{\pgfqpoint{1.534666in}{1.275135in}}%
\pgfpathlineto{\pgfqpoint{1.547916in}{1.272118in}}%
\pgfpathlineto{\pgfqpoint{1.560990in}{1.268890in}}%
\pgfpathlineto{\pgfqpoint{1.559573in}{1.270396in}}%
\pgfpathlineto{\pgfqpoint{1.558157in}{1.272055in}}%
\pgfpathlineto{\pgfqpoint{1.556742in}{1.273865in}}%
\pgfpathlineto{\pgfqpoint{1.555327in}{1.275821in}}%
\pgfpathlineto{\pgfqpoint{1.542609in}{1.278958in}}%
\pgfpathlineto{\pgfqpoint{1.529719in}{1.281891in}}%
\pgfpathlineto{\pgfqpoint{1.516668in}{1.284616in}}%
\pgfpathlineto{\pgfqpoint{1.503469in}{1.287133in}}%
\pgfpathclose%
\pgfusepath{fill}%
\end{pgfscope}%
\begin{pgfscope}%
\pgfpathrectangle{\pgfqpoint{0.329460in}{0.284240in}}{\pgfqpoint{1.989680in}{1.989680in}}%
\pgfusepath{clip}%
\pgfsetbuttcap%
\pgfsetroundjoin%
\definecolor{currentfill}{rgb}{0.282327,0.094955,0.417331}%
\pgfsetfillcolor{currentfill}%
\pgfsetlinewidth{0.000000pt}%
\definecolor{currentstroke}{rgb}{0.000000,0.000000,0.000000}%
\pgfsetstrokecolor{currentstroke}%
\pgfsetdash{}{0pt}%
\pgfpathmoveto{\pgfqpoint{1.460010in}{1.302126in}}%
\pgfpathlineto{\pgfqpoint{1.460783in}{1.299744in}}%
\pgfpathlineto{\pgfqpoint{1.461556in}{1.297495in}}%
\pgfpathlineto{\pgfqpoint{1.462329in}{1.295382in}}%
\pgfpathlineto{\pgfqpoint{1.463103in}{1.293409in}}%
\pgfpathlineto{\pgfqpoint{1.476674in}{1.291531in}}%
\pgfpathlineto{\pgfqpoint{1.490134in}{1.289439in}}%
\pgfpathlineto{\pgfqpoint{1.503469in}{1.287133in}}%
\pgfpathlineto{\pgfqpoint{1.502416in}{1.289151in}}%
\pgfpathlineto{\pgfqpoint{1.501363in}{1.291310in}}%
\pgfpathlineto{\pgfqpoint{1.500310in}{1.293604in}}%
\pgfpathlineto{\pgfqpoint{1.499257in}{1.296032in}}%
\pgfpathlineto{\pgfqpoint{1.486292in}{1.298270in}}%
\pgfpathlineto{\pgfqpoint{1.473206in}{1.300302in}}%
\pgfpathlineto{\pgfqpoint{1.460010in}{1.302126in}}%
\pgfpathclose%
\pgfusepath{fill}%
\end{pgfscope}%
\begin{pgfscope}%
\pgfpathrectangle{\pgfqpoint{0.329460in}{0.284240in}}{\pgfqpoint{1.989680in}{1.989680in}}%
\pgfusepath{clip}%
\pgfsetbuttcap%
\pgfsetroundjoin%
\definecolor{currentfill}{rgb}{0.279566,0.067836,0.391917}%
\pgfsetfillcolor{currentfill}%
\pgfsetlinewidth{0.000000pt}%
\definecolor{currentstroke}{rgb}{0.000000,0.000000,0.000000}%
\pgfsetstrokecolor{currentstroke}%
\pgfsetdash{}{0pt}%
\pgfpathmoveto{\pgfqpoint{1.135897in}{1.272863in}}%
\pgfpathlineto{\pgfqpoint{1.134404in}{1.270885in}}%
\pgfpathlineto{\pgfqpoint{1.132911in}{1.269054in}}%
\pgfpathlineto{\pgfqpoint{1.131417in}{1.267373in}}%
\pgfpathlineto{\pgfqpoint{1.129922in}{1.265847in}}%
\pgfpathlineto{\pgfqpoint{1.142829in}{1.269259in}}%
\pgfpathlineto{\pgfqpoint{1.155923in}{1.272463in}}%
\pgfpathlineto{\pgfqpoint{1.169192in}{1.275457in}}%
\pgfpathlineto{\pgfqpoint{1.182623in}{1.278237in}}%
\pgfpathlineto{\pgfqpoint{1.183760in}{1.279677in}}%
\pgfpathlineto{\pgfqpoint{1.184895in}{1.281271in}}%
\pgfpathlineto{\pgfqpoint{1.186031in}{1.283015in}}%
\pgfpathlineto{\pgfqpoint{1.187166in}{1.284906in}}%
\pgfpathlineto{\pgfqpoint{1.174099in}{1.282204in}}%
\pgfpathlineto{\pgfqpoint{1.161190in}{1.279294in}}%
\pgfpathlineto{\pgfqpoint{1.148452in}{1.276180in}}%
\pgfpathlineto{\pgfqpoint{1.135897in}{1.272863in}}%
\pgfpathclose%
\pgfusepath{fill}%
\end{pgfscope}%
\begin{pgfscope}%
\pgfpathrectangle{\pgfqpoint{0.329460in}{0.284240in}}{\pgfqpoint{1.989680in}{1.989680in}}%
\pgfusepath{clip}%
\pgfsetbuttcap%
\pgfsetroundjoin%
\definecolor{currentfill}{rgb}{0.172719,0.448791,0.557885}%
\pgfsetfillcolor{currentfill}%
\pgfsetlinewidth{0.000000pt}%
\definecolor{currentstroke}{rgb}{0.000000,0.000000,0.000000}%
\pgfsetstrokecolor{currentstroke}%
\pgfsetdash{}{0pt}%
\pgfpathmoveto{\pgfqpoint{1.644305in}{1.555961in}}%
\pgfpathlineto{\pgfqpoint{1.645934in}{1.570356in}}%
\pgfpathlineto{\pgfqpoint{1.647572in}{1.585189in}}%
\pgfpathlineto{\pgfqpoint{1.649218in}{1.600465in}}%
\pgfpathlineto{\pgfqpoint{1.667595in}{1.595842in}}%
\pgfpathlineto{\pgfqpoint{1.685700in}{1.590942in}}%
\pgfpathlineto{\pgfqpoint{1.703517in}{1.585767in}}%
\pgfpathlineto{\pgfqpoint{1.721031in}{1.580320in}}%
\pgfpathlineto{\pgfqpoint{1.718981in}{1.565104in}}%
\pgfpathlineto{\pgfqpoint{1.716942in}{1.550333in}}%
\pgfpathlineto{\pgfqpoint{1.714913in}{1.536000in}}%
\pgfpathlineto{\pgfqpoint{1.697695in}{1.541397in}}%
\pgfpathlineto{\pgfqpoint{1.680177in}{1.546525in}}%
\pgfpathlineto{\pgfqpoint{1.662375in}{1.551381in}}%
\pgfpathlineto{\pgfqpoint{1.644305in}{1.555961in}}%
\pgfpathclose%
\pgfusepath{fill}%
\end{pgfscope}%
\begin{pgfscope}%
\pgfpathrectangle{\pgfqpoint{0.329460in}{0.284240in}}{\pgfqpoint{1.989680in}{1.989680in}}%
\pgfusepath{clip}%
\pgfsetbuttcap%
\pgfsetroundjoin%
\definecolor{currentfill}{rgb}{0.282327,0.094955,0.417331}%
\pgfsetfillcolor{currentfill}%
\pgfsetlinewidth{0.000000pt}%
\definecolor{currentstroke}{rgb}{0.000000,0.000000,0.000000}%
\pgfsetstrokecolor{currentstroke}%
\pgfsetdash{}{0pt}%
\pgfpathmoveto{\pgfqpoint{1.191705in}{1.293870in}}%
\pgfpathlineto{\pgfqpoint{1.190570in}{1.291426in}}%
\pgfpathlineto{\pgfqpoint{1.189436in}{1.289115in}}%
\pgfpathlineto{\pgfqpoint{1.188301in}{1.286941in}}%
\pgfpathlineto{\pgfqpoint{1.187166in}{1.284906in}}%
\pgfpathlineto{\pgfqpoint{1.200381in}{1.287400in}}%
\pgfpathlineto{\pgfqpoint{1.213731in}{1.289682in}}%
\pgfpathlineto{\pgfqpoint{1.227203in}{1.291750in}}%
\pgfpathlineto{\pgfqpoint{1.240787in}{1.293605in}}%
\pgfpathlineto{\pgfqpoint{1.241550in}{1.295576in}}%
\pgfpathlineto{\pgfqpoint{1.242313in}{1.297687in}}%
\pgfpathlineto{\pgfqpoint{1.243075in}{1.299935in}}%
\pgfpathlineto{\pgfqpoint{1.243838in}{1.302316in}}%
\pgfpathlineto{\pgfqpoint{1.230631in}{1.300515in}}%
\pgfpathlineto{\pgfqpoint{1.217531in}{1.298506in}}%
\pgfpathlineto{\pgfqpoint{1.204552in}{1.296291in}}%
\pgfpathlineto{\pgfqpoint{1.191705in}{1.293870in}}%
\pgfpathclose%
\pgfusepath{fill}%
\end{pgfscope}%
\begin{pgfscope}%
\pgfpathrectangle{\pgfqpoint{0.329460in}{0.284240in}}{\pgfqpoint{1.989680in}{1.989680in}}%
\pgfusepath{clip}%
\pgfsetbuttcap%
\pgfsetroundjoin%
\definecolor{currentfill}{rgb}{0.267004,0.004874,0.329415}%
\pgfsetfillcolor{currentfill}%
\pgfsetlinewidth{0.000000pt}%
\definecolor{currentstroke}{rgb}{0.000000,0.000000,0.000000}%
\pgfsetstrokecolor{currentstroke}%
\pgfsetdash{}{0pt}%
\pgfpathmoveto{\pgfqpoint{1.008897in}{1.223593in}}%
\pgfpathlineto{\pgfqpoint{1.006723in}{1.224173in}}%
\pgfpathlineto{\pgfqpoint{1.004546in}{1.224961in}}%
\pgfpathlineto{\pgfqpoint{1.002366in}{1.225961in}}%
\pgfpathlineto{\pgfqpoint{1.000183in}{1.227176in}}%
\pgfpathlineto{\pgfqpoint{1.012407in}{1.232688in}}%
\pgfpathlineto{\pgfqpoint{1.024936in}{1.237998in}}%
\pgfpathlineto{\pgfqpoint{1.037760in}{1.243100in}}%
\pgfpathlineto{\pgfqpoint{1.050866in}{1.247993in}}%
\pgfpathlineto{\pgfqpoint{1.052727in}{1.246652in}}%
\pgfpathlineto{\pgfqpoint{1.054587in}{1.245527in}}%
\pgfpathlineto{\pgfqpoint{1.056443in}{1.244613in}}%
\pgfpathlineto{\pgfqpoint{1.058297in}{1.243906in}}%
\pgfpathlineto{\pgfqpoint{1.045522in}{1.239132in}}%
\pgfpathlineto{\pgfqpoint{1.033022in}{1.234153in}}%
\pgfpathlineto{\pgfqpoint{1.020810in}{1.228972in}}%
\pgfpathlineto{\pgfqpoint{1.008897in}{1.223593in}}%
\pgfpathclose%
\pgfusepath{fill}%
\end{pgfscope}%
\begin{pgfscope}%
\pgfpathrectangle{\pgfqpoint{0.329460in}{0.284240in}}{\pgfqpoint{1.989680in}{1.989680in}}%
\pgfusepath{clip}%
\pgfsetbuttcap%
\pgfsetroundjoin%
\definecolor{currentfill}{rgb}{0.274952,0.037752,0.364543}%
\pgfsetfillcolor{currentfill}%
\pgfsetlinewidth{0.000000pt}%
\definecolor{currentstroke}{rgb}{0.000000,0.000000,0.000000}%
\pgfsetstrokecolor{currentstroke}%
\pgfsetdash{}{0pt}%
\pgfpathmoveto{\pgfqpoint{1.560990in}{1.268890in}}%
\pgfpathlineto{\pgfqpoint{1.562407in}{1.267543in}}%
\pgfpathlineto{\pgfqpoint{1.563825in}{1.266356in}}%
\pgfpathlineto{\pgfqpoint{1.565243in}{1.265334in}}%
\pgfpathlineto{\pgfqpoint{1.566663in}{1.264481in}}%
\pgfpathlineto{\pgfqpoint{1.579900in}{1.260950in}}%
\pgfpathlineto{\pgfqpoint{1.592932in}{1.257208in}}%
\pgfpathlineto{\pgfqpoint{1.605746in}{1.253259in}}%
\pgfpathlineto{\pgfqpoint{1.618331in}{1.249104in}}%
\pgfpathlineto{\pgfqpoint{1.616567in}{1.250061in}}%
\pgfpathlineto{\pgfqpoint{1.614804in}{1.251186in}}%
\pgfpathlineto{\pgfqpoint{1.613042in}{1.252477in}}%
\pgfpathlineto{\pgfqpoint{1.611281in}{1.253929in}}%
\pgfpathlineto{\pgfqpoint{1.599032in}{1.257971in}}%
\pgfpathlineto{\pgfqpoint{1.586560in}{1.261814in}}%
\pgfpathlineto{\pgfqpoint{1.573875in}{1.265455in}}%
\pgfpathlineto{\pgfqpoint{1.560990in}{1.268890in}}%
\pgfpathclose%
\pgfusepath{fill}%
\end{pgfscope}%
\begin{pgfscope}%
\pgfpathrectangle{\pgfqpoint{0.329460in}{0.284240in}}{\pgfqpoint{1.989680in}{1.989680in}}%
\pgfusepath{clip}%
\pgfsetbuttcap%
\pgfsetroundjoin%
\definecolor{currentfill}{rgb}{0.260571,0.246922,0.522828}%
\pgfsetfillcolor{currentfill}%
\pgfsetlinewidth{0.000000pt}%
\definecolor{currentstroke}{rgb}{0.000000,0.000000,0.000000}%
\pgfsetstrokecolor{currentstroke}%
\pgfsetdash{}{0pt}%
\pgfpathmoveto{\pgfqpoint{0.937046in}{1.366934in}}%
\pgfpathlineto{\pgfqpoint{0.934691in}{1.376407in}}%
\pgfpathlineto{\pgfqpoint{0.932326in}{1.386242in}}%
\pgfpathlineto{\pgfqpoint{0.929952in}{1.396446in}}%
\pgfpathlineto{\pgfqpoint{0.927568in}{1.407024in}}%
\pgfpathlineto{\pgfqpoint{0.942395in}{1.413485in}}%
\pgfpathlineto{\pgfqpoint{0.957583in}{1.419706in}}%
\pgfpathlineto{\pgfqpoint{0.973118in}{1.425684in}}%
\pgfpathlineto{\pgfqpoint{0.988984in}{1.431412in}}%
\pgfpathlineto{\pgfqpoint{0.991014in}{1.420739in}}%
\pgfpathlineto{\pgfqpoint{0.993036in}{1.410439in}}%
\pgfpathlineto{\pgfqpoint{0.995050in}{1.400506in}}%
\pgfpathlineto{\pgfqpoint{0.997056in}{1.390935in}}%
\pgfpathlineto{\pgfqpoint{0.981551in}{1.385297in}}%
\pgfpathlineto{\pgfqpoint{0.966371in}{1.379414in}}%
\pgfpathlineto{\pgfqpoint{0.951532in}{1.373292in}}%
\pgfpathlineto{\pgfqpoint{0.937046in}{1.366934in}}%
\pgfpathclose%
\pgfusepath{fill}%
\end{pgfscope}%
\begin{pgfscope}%
\pgfpathrectangle{\pgfqpoint{0.329460in}{0.284240in}}{\pgfqpoint{1.989680in}{1.989680in}}%
\pgfusepath{clip}%
\pgfsetbuttcap%
\pgfsetroundjoin%
\definecolor{currentfill}{rgb}{0.283072,0.130895,0.449241}%
\pgfsetfillcolor{currentfill}%
\pgfsetlinewidth{0.000000pt}%
\definecolor{currentstroke}{rgb}{0.000000,0.000000,0.000000}%
\pgfsetstrokecolor{currentstroke}%
\pgfsetdash{}{0pt}%
\pgfpathmoveto{\pgfqpoint{1.299032in}{1.318039in}}%
\pgfpathlineto{\pgfqpoint{1.298651in}{1.315198in}}%
\pgfpathlineto{\pgfqpoint{1.298270in}{1.312477in}}%
\pgfpathlineto{\pgfqpoint{1.297889in}{1.309879in}}%
\pgfpathlineto{\pgfqpoint{1.297508in}{1.307407in}}%
\pgfpathlineto{\pgfqpoint{1.311076in}{1.308147in}}%
\pgfpathlineto{\pgfqpoint{1.324679in}{1.308673in}}%
\pgfpathlineto{\pgfqpoint{1.338307in}{1.308984in}}%
\pgfpathlineto{\pgfqpoint{1.351945in}{1.309079in}}%
\pgfpathlineto{\pgfqpoint{1.351940in}{1.311539in}}%
\pgfpathlineto{\pgfqpoint{1.351935in}{1.314125in}}%
\pgfpathlineto{\pgfqpoint{1.351929in}{1.316833in}}%
\pgfpathlineto{\pgfqpoint{1.351924in}{1.319661in}}%
\pgfpathlineto{\pgfqpoint{1.338672in}{1.319568in}}%
\pgfpathlineto{\pgfqpoint{1.325432in}{1.319267in}}%
\pgfpathlineto{\pgfqpoint{1.312215in}{1.318757in}}%
\pgfpathlineto{\pgfqpoint{1.299032in}{1.318039in}}%
\pgfpathclose%
\pgfusepath{fill}%
\end{pgfscope}%
\begin{pgfscope}%
\pgfpathrectangle{\pgfqpoint{0.329460in}{0.284240in}}{\pgfqpoint{1.989680in}{1.989680in}}%
\pgfusepath{clip}%
\pgfsetbuttcap%
\pgfsetroundjoin%
\definecolor{currentfill}{rgb}{0.283072,0.130895,0.449241}%
\pgfsetfillcolor{currentfill}%
\pgfsetlinewidth{0.000000pt}%
\definecolor{currentstroke}{rgb}{0.000000,0.000000,0.000000}%
\pgfsetstrokecolor{currentstroke}%
\pgfsetdash{}{0pt}%
\pgfpathmoveto{\pgfqpoint{1.351924in}{1.319661in}}%
\pgfpathlineto{\pgfqpoint{1.351929in}{1.316833in}}%
\pgfpathlineto{\pgfqpoint{1.351935in}{1.314125in}}%
\pgfpathlineto{\pgfqpoint{1.351940in}{1.311539in}}%
\pgfpathlineto{\pgfqpoint{1.351945in}{1.309079in}}%
\pgfpathlineto{\pgfqpoint{1.365583in}{1.308960in}}%
\pgfpathlineto{\pgfqpoint{1.379209in}{1.308625in}}%
\pgfpathlineto{\pgfqpoint{1.392809in}{1.308076in}}%
\pgfpathlineto{\pgfqpoint{1.406372in}{1.307312in}}%
\pgfpathlineto{\pgfqpoint{1.405980in}{1.309784in}}%
\pgfpathlineto{\pgfqpoint{1.405588in}{1.312383in}}%
\pgfpathlineto{\pgfqpoint{1.405197in}{1.315105in}}%
\pgfpathlineto{\pgfqpoint{1.404805in}{1.317947in}}%
\pgfpathlineto{\pgfqpoint{1.391627in}{1.318688in}}%
\pgfpathlineto{\pgfqpoint{1.378413in}{1.319221in}}%
\pgfpathlineto{\pgfqpoint{1.365175in}{1.319545in}}%
\pgfpathlineto{\pgfqpoint{1.351924in}{1.319661in}}%
\pgfpathclose%
\pgfusepath{fill}%
\end{pgfscope}%
\begin{pgfscope}%
\pgfpathrectangle{\pgfqpoint{0.329460in}{0.284240in}}{\pgfqpoint{1.989680in}{1.989680in}}%
\pgfusepath{clip}%
\pgfsetbuttcap%
\pgfsetroundjoin%
\definecolor{currentfill}{rgb}{0.268510,0.009605,0.335427}%
\pgfsetfillcolor{currentfill}%
\pgfsetlinewidth{0.000000pt}%
\definecolor{currentstroke}{rgb}{0.000000,0.000000,0.000000}%
\pgfsetstrokecolor{currentstroke}%
\pgfsetdash{}{0pt}%
\pgfpathmoveto{\pgfqpoint{1.625402in}{1.247038in}}%
\pgfpathlineto{\pgfqpoint{1.627174in}{1.246981in}}%
\pgfpathlineto{\pgfqpoint{1.628948in}{1.247116in}}%
\pgfpathlineto{\pgfqpoint{1.630723in}{1.247445in}}%
\pgfpathlineto{\pgfqpoint{1.632501in}{1.247974in}}%
\pgfpathlineto{\pgfqpoint{1.645511in}{1.243386in}}%
\pgfpathlineto{\pgfqpoint{1.658256in}{1.238589in}}%
\pgfpathlineto{\pgfqpoint{1.670724in}{1.233587in}}%
\pgfpathlineto{\pgfqpoint{1.682904in}{1.228384in}}%
\pgfpathlineto{\pgfqpoint{1.680802in}{1.227977in}}%
\pgfpathlineto{\pgfqpoint{1.678701in}{1.227771in}}%
\pgfpathlineto{\pgfqpoint{1.676604in}{1.227760in}}%
\pgfpathlineto{\pgfqpoint{1.674508in}{1.227941in}}%
\pgfpathlineto{\pgfqpoint{1.662643in}{1.233013in}}%
\pgfpathlineto{\pgfqpoint{1.650496in}{1.237889in}}%
\pgfpathlineto{\pgfqpoint{1.638079in}{1.242565in}}%
\pgfpathlineto{\pgfqpoint{1.625402in}{1.247038in}}%
\pgfpathclose%
\pgfusepath{fill}%
\end{pgfscope}%
\begin{pgfscope}%
\pgfpathrectangle{\pgfqpoint{0.329460in}{0.284240in}}{\pgfqpoint{1.989680in}{1.989680in}}%
\pgfusepath{clip}%
\pgfsetbuttcap%
\pgfsetroundjoin%
\definecolor{currentfill}{rgb}{0.233603,0.313828,0.543914}%
\pgfsetfillcolor{currentfill}%
\pgfsetlinewidth{0.000000pt}%
\definecolor{currentstroke}{rgb}{0.000000,0.000000,0.000000}%
\pgfsetstrokecolor{currentstroke}%
\pgfsetdash{}{0pt}%
\pgfpathmoveto{\pgfqpoint{1.699021in}{1.436292in}}%
\pgfpathlineto{\pgfqpoint{1.700977in}{1.447362in}}%
\pgfpathlineto{\pgfqpoint{1.702941in}{1.458819in}}%
\pgfpathlineto{\pgfqpoint{1.704914in}{1.470666in}}%
\pgfpathlineto{\pgfqpoint{1.706895in}{1.482911in}}%
\pgfpathlineto{\pgfqpoint{1.723418in}{1.477326in}}%
\pgfpathlineto{\pgfqpoint{1.739615in}{1.471484in}}%
\pgfpathlineto{\pgfqpoint{1.755473in}{1.465390in}}%
\pgfpathlineto{\pgfqpoint{1.770975in}{1.459049in}}%
\pgfpathlineto{\pgfqpoint{1.768627in}{1.446889in}}%
\pgfpathlineto{\pgfqpoint{1.766290in}{1.435128in}}%
\pgfpathlineto{\pgfqpoint{1.763963in}{1.423760in}}%
\pgfpathlineto{\pgfqpoint{1.761646in}{1.412779in}}%
\pgfpathlineto{\pgfqpoint{1.746497in}{1.419027in}}%
\pgfpathlineto{\pgfqpoint{1.731000in}{1.425032in}}%
\pgfpathlineto{\pgfqpoint{1.715170in}{1.430788in}}%
\pgfpathlineto{\pgfqpoint{1.699021in}{1.436292in}}%
\pgfpathclose%
\pgfusepath{fill}%
\end{pgfscope}%
\begin{pgfscope}%
\pgfpathrectangle{\pgfqpoint{0.329460in}{0.284240in}}{\pgfqpoint{1.989680in}{1.989680in}}%
\pgfusepath{clip}%
\pgfsetbuttcap%
\pgfsetroundjoin%
\definecolor{currentfill}{rgb}{0.274952,0.037752,0.364543}%
\pgfsetfillcolor{currentfill}%
\pgfsetlinewidth{0.000000pt}%
\definecolor{currentstroke}{rgb}{0.000000,0.000000,0.000000}%
\pgfsetstrokecolor{currentstroke}%
\pgfsetdash{}{0pt}%
\pgfpathmoveto{\pgfqpoint{1.080404in}{1.250172in}}%
\pgfpathlineto{\pgfqpoint{1.078569in}{1.248693in}}%
\pgfpathlineto{\pgfqpoint{1.076733in}{1.247376in}}%
\pgfpathlineto{\pgfqpoint{1.074897in}{1.246225in}}%
\pgfpathlineto{\pgfqpoint{1.073059in}{1.245242in}}%
\pgfpathlineto{\pgfqpoint{1.085430in}{1.249576in}}%
\pgfpathlineto{\pgfqpoint{1.098042in}{1.253708in}}%
\pgfpathlineto{\pgfqpoint{1.110881in}{1.257635in}}%
\pgfpathlineto{\pgfqpoint{1.123936in}{1.261353in}}%
\pgfpathlineto{\pgfqpoint{1.125434in}{1.262227in}}%
\pgfpathlineto{\pgfqpoint{1.126931in}{1.263270in}}%
\pgfpathlineto{\pgfqpoint{1.128427in}{1.264478in}}%
\pgfpathlineto{\pgfqpoint{1.129922in}{1.265847in}}%
\pgfpathlineto{\pgfqpoint{1.117215in}{1.262229in}}%
\pgfpathlineto{\pgfqpoint{1.104718in}{1.258408in}}%
\pgfpathlineto{\pgfqpoint{1.092444in}{1.254388in}}%
\pgfpathlineto{\pgfqpoint{1.080404in}{1.250172in}}%
\pgfpathclose%
\pgfusepath{fill}%
\end{pgfscope}%
\begin{pgfscope}%
\pgfpathrectangle{\pgfqpoint{0.329460in}{0.284240in}}{\pgfqpoint{1.989680in}{1.989680in}}%
\pgfusepath{clip}%
\pgfsetbuttcap%
\pgfsetroundjoin%
\definecolor{currentfill}{rgb}{0.172719,0.448791,0.557885}%
\pgfsetfillcolor{currentfill}%
\pgfsetlinewidth{0.000000pt}%
\definecolor{currentstroke}{rgb}{0.000000,0.000000,0.000000}%
\pgfsetstrokecolor{currentstroke}%
\pgfsetdash{}{0pt}%
\pgfpathmoveto{\pgfqpoint{0.972421in}{1.530981in}}%
\pgfpathlineto{\pgfqpoint{0.970306in}{1.545298in}}%
\pgfpathlineto{\pgfqpoint{0.968181in}{1.560054in}}%
\pgfpathlineto{\pgfqpoint{0.966045in}{1.575254in}}%
\pgfpathlineto{\pgfqpoint{0.983275in}{1.580939in}}%
\pgfpathlineto{\pgfqpoint{1.000823in}{1.586355in}}%
\pgfpathlineto{\pgfqpoint{1.018673in}{1.591500in}}%
\pgfpathlineto{\pgfqpoint{1.036809in}{1.596370in}}%
\pgfpathlineto{\pgfqpoint{1.038547in}{1.581106in}}%
\pgfpathlineto{\pgfqpoint{1.040275in}{1.566286in}}%
\pgfpathlineto{\pgfqpoint{1.041995in}{1.551903in}}%
\pgfpathlineto{\pgfqpoint{1.024163in}{1.547078in}}%
\pgfpathlineto{\pgfqpoint{1.006612in}{1.541980in}}%
\pgfpathlineto{\pgfqpoint{0.989360in}{1.536613in}}%
\pgfpathlineto{\pgfqpoint{0.972421in}{1.530981in}}%
\pgfpathclose%
\pgfusepath{fill}%
\end{pgfscope}%
\begin{pgfscope}%
\pgfpathrectangle{\pgfqpoint{0.329460in}{0.284240in}}{\pgfqpoint{1.989680in}{1.989680in}}%
\pgfusepath{clip}%
\pgfsetbuttcap%
\pgfsetroundjoin%
\definecolor{currentfill}{rgb}{0.283072,0.130895,0.449241}%
\pgfsetfillcolor{currentfill}%
\pgfsetlinewidth{0.000000pt}%
\definecolor{currentstroke}{rgb}{0.000000,0.000000,0.000000}%
\pgfsetstrokecolor{currentstroke}%
\pgfsetdash{}{0pt}%
\pgfpathmoveto{\pgfqpoint{1.246888in}{1.313101in}}%
\pgfpathlineto{\pgfqpoint{1.246126in}{1.310222in}}%
\pgfpathlineto{\pgfqpoint{1.245363in}{1.307462in}}%
\pgfpathlineto{\pgfqpoint{1.244600in}{1.304826in}}%
\pgfpathlineto{\pgfqpoint{1.243838in}{1.302316in}}%
\pgfpathlineto{\pgfqpoint{1.257141in}{1.303906in}}%
\pgfpathlineto{\pgfqpoint{1.270529in}{1.305286in}}%
\pgfpathlineto{\pgfqpoint{1.283988in}{1.306453in}}%
\pgfpathlineto{\pgfqpoint{1.297508in}{1.307407in}}%
\pgfpathlineto{\pgfqpoint{1.297889in}{1.309879in}}%
\pgfpathlineto{\pgfqpoint{1.298270in}{1.312477in}}%
\pgfpathlineto{\pgfqpoint{1.298651in}{1.315198in}}%
\pgfpathlineto{\pgfqpoint{1.299032in}{1.318039in}}%
\pgfpathlineto{\pgfqpoint{1.285897in}{1.317114in}}%
\pgfpathlineto{\pgfqpoint{1.272820in}{1.315982in}}%
\pgfpathlineto{\pgfqpoint{1.259813in}{1.314644in}}%
\pgfpathlineto{\pgfqpoint{1.246888in}{1.313101in}}%
\pgfpathclose%
\pgfusepath{fill}%
\end{pgfscope}%
\begin{pgfscope}%
\pgfpathrectangle{\pgfqpoint{0.329460in}{0.284240in}}{\pgfqpoint{1.989680in}{1.989680in}}%
\pgfusepath{clip}%
\pgfsetbuttcap%
\pgfsetroundjoin%
\definecolor{currentfill}{rgb}{0.283072,0.130895,0.449241}%
\pgfsetfillcolor{currentfill}%
\pgfsetlinewidth{0.000000pt}%
\definecolor{currentstroke}{rgb}{0.000000,0.000000,0.000000}%
\pgfsetstrokecolor{currentstroke}%
\pgfsetdash{}{0pt}%
\pgfpathmoveto{\pgfqpoint{1.404805in}{1.317947in}}%
\pgfpathlineto{\pgfqpoint{1.405197in}{1.315105in}}%
\pgfpathlineto{\pgfqpoint{1.405588in}{1.312383in}}%
\pgfpathlineto{\pgfqpoint{1.405980in}{1.309784in}}%
\pgfpathlineto{\pgfqpoint{1.406372in}{1.307312in}}%
\pgfpathlineto{\pgfqpoint{1.419886in}{1.306334in}}%
\pgfpathlineto{\pgfqpoint{1.433338in}{1.305143in}}%
\pgfpathlineto{\pgfqpoint{1.446717in}{1.303740in}}%
\pgfpathlineto{\pgfqpoint{1.460010in}{1.302126in}}%
\pgfpathlineto{\pgfqpoint{1.459237in}{1.304638in}}%
\pgfpathlineto{\pgfqpoint{1.458464in}{1.307276in}}%
\pgfpathlineto{\pgfqpoint{1.457691in}{1.310036in}}%
\pgfpathlineto{\pgfqpoint{1.456917in}{1.312917in}}%
\pgfpathlineto{\pgfqpoint{1.444003in}{1.314482in}}%
\pgfpathlineto{\pgfqpoint{1.431004in}{1.315843in}}%
\pgfpathlineto{\pgfqpoint{1.417935in}{1.316998in}}%
\pgfpathlineto{\pgfqpoint{1.404805in}{1.317947in}}%
\pgfpathclose%
\pgfusepath{fill}%
\end{pgfscope}%
\begin{pgfscope}%
\pgfpathrectangle{\pgfqpoint{0.329460in}{0.284240in}}{\pgfqpoint{1.989680in}{1.989680in}}%
\pgfusepath{clip}%
\pgfsetbuttcap%
\pgfsetroundjoin%
\definecolor{currentfill}{rgb}{0.268510,0.009605,0.335427}%
\pgfsetfillcolor{currentfill}%
\pgfsetlinewidth{0.000000pt}%
\definecolor{currentstroke}{rgb}{0.000000,0.000000,0.000000}%
\pgfsetstrokecolor{currentstroke}%
\pgfsetdash{}{0pt}%
\pgfpathmoveto{\pgfqpoint{1.017566in}{1.223272in}}%
\pgfpathlineto{\pgfqpoint{1.015402in}{1.223060in}}%
\pgfpathlineto{\pgfqpoint{1.013236in}{1.223041in}}%
\pgfpathlineto{\pgfqpoint{1.011068in}{1.223217in}}%
\pgfpathlineto{\pgfqpoint{1.008897in}{1.223593in}}%
\pgfpathlineto{\pgfqpoint{1.020810in}{1.228972in}}%
\pgfpathlineto{\pgfqpoint{1.033022in}{1.234153in}}%
\pgfpathlineto{\pgfqpoint{1.045522in}{1.239132in}}%
\pgfpathlineto{\pgfqpoint{1.058297in}{1.243906in}}%
\pgfpathlineto{\pgfqpoint{1.060149in}{1.243403in}}%
\pgfpathlineto{\pgfqpoint{1.061999in}{1.243099in}}%
\pgfpathlineto{\pgfqpoint{1.063847in}{1.242990in}}%
\pgfpathlineto{\pgfqpoint{1.065692in}{1.243073in}}%
\pgfpathlineto{\pgfqpoint{1.053245in}{1.238418in}}%
\pgfpathlineto{\pgfqpoint{1.041068in}{1.233564in}}%
\pgfpathlineto{\pgfqpoint{1.029171in}{1.228514in}}%
\pgfpathlineto{\pgfqpoint{1.017566in}{1.223272in}}%
\pgfpathclose%
\pgfusepath{fill}%
\end{pgfscope}%
\begin{pgfscope}%
\pgfpathrectangle{\pgfqpoint{0.329460in}{0.284240in}}{\pgfqpoint{1.989680in}{1.989680in}}%
\pgfusepath{clip}%
\pgfsetbuttcap%
\pgfsetroundjoin%
\definecolor{currentfill}{rgb}{0.277941,0.056324,0.381191}%
\pgfsetfillcolor{currentfill}%
\pgfsetlinewidth{0.000000pt}%
\definecolor{currentstroke}{rgb}{0.000000,0.000000,0.000000}%
\pgfsetstrokecolor{currentstroke}%
\pgfsetdash{}{0pt}%
\pgfpathmoveto{\pgfqpoint{1.717008in}{1.265404in}}%
\pgfpathlineto{\pgfqpoint{1.719177in}{1.269846in}}%
\pgfpathlineto{\pgfqpoint{1.721351in}{1.274566in}}%
\pgfpathlineto{\pgfqpoint{1.723531in}{1.279568in}}%
\pgfpathlineto{\pgfqpoint{1.725716in}{1.284857in}}%
\pgfpathlineto{\pgfqpoint{1.739172in}{1.278815in}}%
\pgfpathlineto{\pgfqpoint{1.752279in}{1.272555in}}%
\pgfpathlineto{\pgfqpoint{1.765026in}{1.266084in}}%
\pgfpathlineto{\pgfqpoint{1.777400in}{1.259407in}}%
\pgfpathlineto{\pgfqpoint{1.774903in}{1.254245in}}%
\pgfpathlineto{\pgfqpoint{1.772412in}{1.249371in}}%
\pgfpathlineto{\pgfqpoint{1.769928in}{1.244781in}}%
\pgfpathlineto{\pgfqpoint{1.767450in}{1.240470in}}%
\pgfpathlineto{\pgfqpoint{1.755376in}{1.247012in}}%
\pgfpathlineto{\pgfqpoint{1.742936in}{1.253351in}}%
\pgfpathlineto{\pgfqpoint{1.730142in}{1.259484in}}%
\pgfpathlineto{\pgfqpoint{1.717008in}{1.265404in}}%
\pgfpathclose%
\pgfusepath{fill}%
\end{pgfscope}%
\begin{pgfscope}%
\pgfpathrectangle{\pgfqpoint{0.329460in}{0.284240in}}{\pgfqpoint{1.989680in}{1.989680in}}%
\pgfusepath{clip}%
\pgfsetbuttcap%
\pgfsetroundjoin%
\definecolor{currentfill}{rgb}{0.272594,0.025563,0.353093}%
\pgfsetfillcolor{currentfill}%
\pgfsetlinewidth{0.000000pt}%
\definecolor{currentstroke}{rgb}{0.000000,0.000000,0.000000}%
\pgfsetstrokecolor{currentstroke}%
\pgfsetdash{}{0pt}%
\pgfpathmoveto{\pgfqpoint{1.708384in}{1.250319in}}%
\pgfpathlineto{\pgfqpoint{1.710533in}{1.253697in}}%
\pgfpathlineto{\pgfqpoint{1.712686in}{1.257335in}}%
\pgfpathlineto{\pgfqpoint{1.714845in}{1.261235in}}%
\pgfpathlineto{\pgfqpoint{1.717008in}{1.265404in}}%
\pgfpathlineto{\pgfqpoint{1.730142in}{1.259484in}}%
\pgfpathlineto{\pgfqpoint{1.742936in}{1.253351in}}%
\pgfpathlineto{\pgfqpoint{1.755376in}{1.247012in}}%
\pgfpathlineto{\pgfqpoint{1.767450in}{1.240470in}}%
\pgfpathlineto{\pgfqpoint{1.764979in}{1.236433in}}%
\pgfpathlineto{\pgfqpoint{1.762513in}{1.232664in}}%
\pgfpathlineto{\pgfqpoint{1.760053in}{1.229160in}}%
\pgfpathlineto{\pgfqpoint{1.757599in}{1.225916in}}%
\pgfpathlineto{\pgfqpoint{1.745820in}{1.232318in}}%
\pgfpathlineto{\pgfqpoint{1.733683in}{1.238523in}}%
\pgfpathlineto{\pgfqpoint{1.721200in}{1.244525in}}%
\pgfpathlineto{\pgfqpoint{1.708384in}{1.250319in}}%
\pgfpathclose%
\pgfusepath{fill}%
\end{pgfscope}%
\begin{pgfscope}%
\pgfpathrectangle{\pgfqpoint{0.329460in}{0.284240in}}{\pgfqpoint{1.989680in}{1.989680in}}%
\pgfusepath{clip}%
\pgfsetbuttcap%
\pgfsetroundjoin%
\definecolor{currentfill}{rgb}{0.282327,0.094955,0.417331}%
\pgfsetfillcolor{currentfill}%
\pgfsetlinewidth{0.000000pt}%
\definecolor{currentstroke}{rgb}{0.000000,0.000000,0.000000}%
\pgfsetstrokecolor{currentstroke}%
\pgfsetdash{}{0pt}%
\pgfpathmoveto{\pgfqpoint{1.499257in}{1.296032in}}%
\pgfpathlineto{\pgfqpoint{1.500310in}{1.293604in}}%
\pgfpathlineto{\pgfqpoint{1.501363in}{1.291310in}}%
\pgfpathlineto{\pgfqpoint{1.502416in}{1.289151in}}%
\pgfpathlineto{\pgfqpoint{1.503469in}{1.287133in}}%
\pgfpathlineto{\pgfqpoint{1.516668in}{1.284616in}}%
\pgfpathlineto{\pgfqpoint{1.529719in}{1.281891in}}%
\pgfpathlineto{\pgfqpoint{1.542609in}{1.278958in}}%
\pgfpathlineto{\pgfqpoint{1.555327in}{1.275821in}}%
\pgfpathlineto{\pgfqpoint{1.553912in}{1.277922in}}%
\pgfpathlineto{\pgfqpoint{1.552498in}{1.280162in}}%
\pgfpathlineto{\pgfqpoint{1.551084in}{1.282539in}}%
\pgfpathlineto{\pgfqpoint{1.549670in}{1.285050in}}%
\pgfpathlineto{\pgfqpoint{1.537307in}{1.288095in}}%
\pgfpathlineto{\pgfqpoint{1.524776in}{1.290942in}}%
\pgfpathlineto{\pgfqpoint{1.512089in}{1.293589in}}%
\pgfpathlineto{\pgfqpoint{1.499257in}{1.296032in}}%
\pgfpathclose%
\pgfusepath{fill}%
\end{pgfscope}%
\begin{pgfscope}%
\pgfpathrectangle{\pgfqpoint{0.329460in}{0.284240in}}{\pgfqpoint{1.989680in}{1.989680in}}%
\pgfusepath{clip}%
\pgfsetbuttcap%
\pgfsetroundjoin%
\definecolor{currentfill}{rgb}{0.282327,0.094955,0.417331}%
\pgfsetfillcolor{currentfill}%
\pgfsetlinewidth{0.000000pt}%
\definecolor{currentstroke}{rgb}{0.000000,0.000000,0.000000}%
\pgfsetstrokecolor{currentstroke}%
\pgfsetdash{}{0pt}%
\pgfpathmoveto{\pgfqpoint{1.725716in}{1.284857in}}%
\pgfpathlineto{\pgfqpoint{1.727908in}{1.290439in}}%
\pgfpathlineto{\pgfqpoint{1.730106in}{1.296318in}}%
\pgfpathlineto{\pgfqpoint{1.732310in}{1.302499in}}%
\pgfpathlineto{\pgfqpoint{1.734520in}{1.308988in}}%
\pgfpathlineto{\pgfqpoint{1.748301in}{1.302827in}}%
\pgfpathlineto{\pgfqpoint{1.761727in}{1.296445in}}%
\pgfpathlineto{\pgfqpoint{1.774785in}{1.289847in}}%
\pgfpathlineto{\pgfqpoint{1.787461in}{1.283037in}}%
\pgfpathlineto{\pgfqpoint{1.784934in}{1.276672in}}%
\pgfpathlineto{\pgfqpoint{1.782416in}{1.270615in}}%
\pgfpathlineto{\pgfqpoint{1.779904in}{1.264862in}}%
\pgfpathlineto{\pgfqpoint{1.777400in}{1.259407in}}%
\pgfpathlineto{\pgfqpoint{1.765026in}{1.266084in}}%
\pgfpathlineto{\pgfqpoint{1.752279in}{1.272555in}}%
\pgfpathlineto{\pgfqpoint{1.739172in}{1.278815in}}%
\pgfpathlineto{\pgfqpoint{1.725716in}{1.284857in}}%
\pgfpathclose%
\pgfusepath{fill}%
\end{pgfscope}%
\begin{pgfscope}%
\pgfpathrectangle{\pgfqpoint{0.329460in}{0.284240in}}{\pgfqpoint{1.989680in}{1.989680in}}%
\pgfusepath{clip}%
\pgfsetbuttcap%
\pgfsetroundjoin%
\definecolor{currentfill}{rgb}{0.233603,0.313828,0.543914}%
\pgfsetfillcolor{currentfill}%
\pgfsetlinewidth{0.000000pt}%
\definecolor{currentstroke}{rgb}{0.000000,0.000000,0.000000}%
\pgfsetstrokecolor{currentstroke}%
\pgfsetdash{}{0pt}%
\pgfpathmoveto{\pgfqpoint{0.927568in}{1.407024in}}%
\pgfpathlineto{\pgfqpoint{0.925174in}{1.417984in}}%
\pgfpathlineto{\pgfqpoint{0.922770in}{1.429330in}}%
\pgfpathlineto{\pgfqpoint{0.920356in}{1.441069in}}%
\pgfpathlineto{\pgfqpoint{0.917930in}{1.453208in}}%
\pgfpathlineto{\pgfqpoint{0.933105in}{1.459765in}}%
\pgfpathlineto{\pgfqpoint{0.948647in}{1.466080in}}%
\pgfpathlineto{\pgfqpoint{0.964543in}{1.472146in}}%
\pgfpathlineto{\pgfqpoint{0.980777in}{1.477959in}}%
\pgfpathlineto{\pgfqpoint{0.982843in}{1.465732in}}%
\pgfpathlineto{\pgfqpoint{0.984899in}{1.453902in}}%
\pgfpathlineto{\pgfqpoint{0.986946in}{1.442464in}}%
\pgfpathlineto{\pgfqpoint{0.988984in}{1.431412in}}%
\pgfpathlineto{\pgfqpoint{0.973118in}{1.425684in}}%
\pgfpathlineto{\pgfqpoint{0.957583in}{1.419706in}}%
\pgfpathlineto{\pgfqpoint{0.942395in}{1.413485in}}%
\pgfpathlineto{\pgfqpoint{0.927568in}{1.407024in}}%
\pgfpathclose%
\pgfusepath{fill}%
\end{pgfscope}%
\begin{pgfscope}%
\pgfpathrectangle{\pgfqpoint{0.329460in}{0.284240in}}{\pgfqpoint{1.989680in}{1.989680in}}%
\pgfusepath{clip}%
\pgfsetbuttcap%
\pgfsetroundjoin%
\definecolor{currentfill}{rgb}{0.271305,0.019942,0.347269}%
\pgfsetfillcolor{currentfill}%
\pgfsetlinewidth{0.000000pt}%
\definecolor{currentstroke}{rgb}{0.000000,0.000000,0.000000}%
\pgfsetstrokecolor{currentstroke}%
\pgfsetdash{}{0pt}%
\pgfpathmoveto{\pgfqpoint{1.618331in}{1.249104in}}%
\pgfpathlineto{\pgfqpoint{1.620097in}{1.248320in}}%
\pgfpathlineto{\pgfqpoint{1.621864in}{1.247711in}}%
\pgfpathlineto{\pgfqpoint{1.623632in}{1.247283in}}%
\pgfpathlineto{\pgfqpoint{1.625402in}{1.247038in}}%
\pgfpathlineto{\pgfqpoint{1.638079in}{1.242565in}}%
\pgfpathlineto{\pgfqpoint{1.650496in}{1.237889in}}%
\pgfpathlineto{\pgfqpoint{1.662643in}{1.233013in}}%
\pgfpathlineto{\pgfqpoint{1.674508in}{1.227941in}}%
\pgfpathlineto{\pgfqpoint{1.672415in}{1.228310in}}%
\pgfpathlineto{\pgfqpoint{1.670324in}{1.228863in}}%
\pgfpathlineto{\pgfqpoint{1.668234in}{1.229596in}}%
\pgfpathlineto{\pgfqpoint{1.666146in}{1.230506in}}%
\pgfpathlineto{\pgfqpoint{1.654595in}{1.235445in}}%
\pgfpathlineto{\pgfqpoint{1.642767in}{1.240194in}}%
\pgfpathlineto{\pgfqpoint{1.630676in}{1.244748in}}%
\pgfpathlineto{\pgfqpoint{1.618331in}{1.249104in}}%
\pgfpathclose%
\pgfusepath{fill}%
\end{pgfscope}%
\begin{pgfscope}%
\pgfpathrectangle{\pgfqpoint{0.329460in}{0.284240in}}{\pgfqpoint{1.989680in}{1.989680in}}%
\pgfusepath{clip}%
\pgfsetbuttcap%
\pgfsetroundjoin%
\definecolor{currentfill}{rgb}{0.268510,0.009605,0.335427}%
\pgfsetfillcolor{currentfill}%
\pgfsetlinewidth{0.000000pt}%
\definecolor{currentstroke}{rgb}{0.000000,0.000000,0.000000}%
\pgfsetstrokecolor{currentstroke}%
\pgfsetdash{}{0pt}%
\pgfpathmoveto{\pgfqpoint{1.699832in}{1.239307in}}%
\pgfpathlineto{\pgfqpoint{1.701964in}{1.241694in}}%
\pgfpathlineto{\pgfqpoint{1.704100in}{1.244322in}}%
\pgfpathlineto{\pgfqpoint{1.706240in}{1.247196in}}%
\pgfpathlineto{\pgfqpoint{1.708384in}{1.250319in}}%
\pgfpathlineto{\pgfqpoint{1.721200in}{1.244525in}}%
\pgfpathlineto{\pgfqpoint{1.733683in}{1.238523in}}%
\pgfpathlineto{\pgfqpoint{1.745820in}{1.232318in}}%
\pgfpathlineto{\pgfqpoint{1.757599in}{1.225916in}}%
\pgfpathlineto{\pgfqpoint{1.755150in}{1.222928in}}%
\pgfpathlineto{\pgfqpoint{1.752706in}{1.220190in}}%
\pgfpathlineto{\pgfqpoint{1.750266in}{1.217698in}}%
\pgfpathlineto{\pgfqpoint{1.747832in}{1.215449in}}%
\pgfpathlineto{\pgfqpoint{1.736345in}{1.221707in}}%
\pgfpathlineto{\pgfqpoint{1.724508in}{1.227773in}}%
\pgfpathlineto{\pgfqpoint{1.712333in}{1.233641in}}%
\pgfpathlineto{\pgfqpoint{1.699832in}{1.239307in}}%
\pgfpathclose%
\pgfusepath{fill}%
\end{pgfscope}%
\begin{pgfscope}%
\pgfpathrectangle{\pgfqpoint{0.329460in}{0.284240in}}{\pgfqpoint{1.989680in}{1.989680in}}%
\pgfusepath{clip}%
\pgfsetbuttcap%
\pgfsetroundjoin%
\definecolor{currentfill}{rgb}{0.279566,0.067836,0.391917}%
\pgfsetfillcolor{currentfill}%
\pgfsetlinewidth{0.000000pt}%
\definecolor{currentstroke}{rgb}{0.000000,0.000000,0.000000}%
\pgfsetstrokecolor{currentstroke}%
\pgfsetdash{}{0pt}%
\pgfpathmoveto{\pgfqpoint{1.555327in}{1.275821in}}%
\pgfpathlineto{\pgfqpoint{1.556742in}{1.273865in}}%
\pgfpathlineto{\pgfqpoint{1.558157in}{1.272055in}}%
\pgfpathlineto{\pgfqpoint{1.559573in}{1.270396in}}%
\pgfpathlineto{\pgfqpoint{1.560990in}{1.268890in}}%
\pgfpathlineto{\pgfqpoint{1.573875in}{1.265455in}}%
\pgfpathlineto{\pgfqpoint{1.586560in}{1.261814in}}%
\pgfpathlineto{\pgfqpoint{1.599032in}{1.257971in}}%
\pgfpathlineto{\pgfqpoint{1.611281in}{1.253929in}}%
\pgfpathlineto{\pgfqpoint{1.609521in}{1.255540in}}%
\pgfpathlineto{\pgfqpoint{1.607761in}{1.257304in}}%
\pgfpathlineto{\pgfqpoint{1.606003in}{1.259219in}}%
\pgfpathlineto{\pgfqpoint{1.604245in}{1.261282in}}%
\pgfpathlineto{\pgfqpoint{1.592331in}{1.265210in}}%
\pgfpathlineto{\pgfqpoint{1.580200in}{1.268944in}}%
\pgfpathlineto{\pgfqpoint{1.567861in}{1.272483in}}%
\pgfpathlineto{\pgfqpoint{1.555327in}{1.275821in}}%
\pgfpathclose%
\pgfusepath{fill}%
\end{pgfscope}%
\begin{pgfscope}%
\pgfpathrectangle{\pgfqpoint{0.329460in}{0.284240in}}{\pgfqpoint{1.989680in}{1.989680in}}%
\pgfusepath{clip}%
\pgfsetbuttcap%
\pgfsetroundjoin%
\definecolor{currentfill}{rgb}{0.282327,0.094955,0.417331}%
\pgfsetfillcolor{currentfill}%
\pgfsetlinewidth{0.000000pt}%
\definecolor{currentstroke}{rgb}{0.000000,0.000000,0.000000}%
\pgfsetstrokecolor{currentstroke}%
\pgfsetdash{}{0pt}%
\pgfpathmoveto{\pgfqpoint{1.141866in}{1.282178in}}%
\pgfpathlineto{\pgfqpoint{1.140374in}{1.279646in}}%
\pgfpathlineto{\pgfqpoint{1.138882in}{1.277247in}}%
\pgfpathlineto{\pgfqpoint{1.137390in}{1.274985in}}%
\pgfpathlineto{\pgfqpoint{1.135897in}{1.272863in}}%
\pgfpathlineto{\pgfqpoint{1.148452in}{1.276180in}}%
\pgfpathlineto{\pgfqpoint{1.161190in}{1.279294in}}%
\pgfpathlineto{\pgfqpoint{1.174099in}{1.282204in}}%
\pgfpathlineto{\pgfqpoint{1.187166in}{1.284906in}}%
\pgfpathlineto{\pgfqpoint{1.188301in}{1.286941in}}%
\pgfpathlineto{\pgfqpoint{1.189436in}{1.289115in}}%
\pgfpathlineto{\pgfqpoint{1.190570in}{1.291426in}}%
\pgfpathlineto{\pgfqpoint{1.191705in}{1.293870in}}%
\pgfpathlineto{\pgfqpoint{1.179001in}{1.291246in}}%
\pgfpathlineto{\pgfqpoint{1.166452in}{1.288421in}}%
\pgfpathlineto{\pgfqpoint{1.154070in}{1.285398in}}%
\pgfpathlineto{\pgfqpoint{1.141866in}{1.282178in}}%
\pgfpathclose%
\pgfusepath{fill}%
\end{pgfscope}%
\begin{pgfscope}%
\pgfpathrectangle{\pgfqpoint{0.329460in}{0.284240in}}{\pgfqpoint{1.989680in}{1.989680in}}%
\pgfusepath{clip}%
\pgfsetbuttcap%
\pgfsetroundjoin%
\definecolor{currentfill}{rgb}{0.282884,0.135920,0.453427}%
\pgfsetfillcolor{currentfill}%
\pgfsetlinewidth{0.000000pt}%
\definecolor{currentstroke}{rgb}{0.000000,0.000000,0.000000}%
\pgfsetstrokecolor{currentstroke}%
\pgfsetdash{}{0pt}%
\pgfpathmoveto{\pgfqpoint{1.734520in}{1.308988in}}%
\pgfpathlineto{\pgfqpoint{1.736738in}{1.315789in}}%
\pgfpathlineto{\pgfqpoint{1.738963in}{1.322908in}}%
\pgfpathlineto{\pgfqpoint{1.741195in}{1.330350in}}%
\pgfpathlineto{\pgfqpoint{1.743434in}{1.338120in}}%
\pgfpathlineto{\pgfqpoint{1.757545in}{1.331846in}}%
\pgfpathlineto{\pgfqpoint{1.771293in}{1.325345in}}%
\pgfpathlineto{\pgfqpoint{1.784666in}{1.318625in}}%
\pgfpathlineto{\pgfqpoint{1.797649in}{1.311689in}}%
\pgfpathlineto{\pgfqpoint{1.795089in}{1.304037in}}%
\pgfpathlineto{\pgfqpoint{1.792538in}{1.296714in}}%
\pgfpathlineto{\pgfqpoint{1.789995in}{1.289716in}}%
\pgfpathlineto{\pgfqpoint{1.787461in}{1.283037in}}%
\pgfpathlineto{\pgfqpoint{1.774785in}{1.289847in}}%
\pgfpathlineto{\pgfqpoint{1.761727in}{1.296445in}}%
\pgfpathlineto{\pgfqpoint{1.748301in}{1.302827in}}%
\pgfpathlineto{\pgfqpoint{1.734520in}{1.308988in}}%
\pgfpathclose%
\pgfusepath{fill}%
\end{pgfscope}%
\begin{pgfscope}%
\pgfpathrectangle{\pgfqpoint{0.329460in}{0.284240in}}{\pgfqpoint{1.989680in}{1.989680in}}%
\pgfusepath{clip}%
\pgfsetbuttcap%
\pgfsetroundjoin%
\definecolor{currentfill}{rgb}{0.283072,0.130895,0.449241}%
\pgfsetfillcolor{currentfill}%
\pgfsetlinewidth{0.000000pt}%
\definecolor{currentstroke}{rgb}{0.000000,0.000000,0.000000}%
\pgfsetstrokecolor{currentstroke}%
\pgfsetdash{}{0pt}%
\pgfpathmoveto{\pgfqpoint{1.456917in}{1.312917in}}%
\pgfpathlineto{\pgfqpoint{1.457691in}{1.310036in}}%
\pgfpathlineto{\pgfqpoint{1.458464in}{1.307276in}}%
\pgfpathlineto{\pgfqpoint{1.459237in}{1.304638in}}%
\pgfpathlineto{\pgfqpoint{1.460010in}{1.302126in}}%
\pgfpathlineto{\pgfqpoint{1.473206in}{1.300302in}}%
\pgfpathlineto{\pgfqpoint{1.486292in}{1.298270in}}%
\pgfpathlineto{\pgfqpoint{1.499257in}{1.296032in}}%
\pgfpathlineto{\pgfqpoint{1.498204in}{1.298589in}}%
\pgfpathlineto{\pgfqpoint{1.497151in}{1.301273in}}%
\pgfpathlineto{\pgfqpoint{1.496098in}{1.304080in}}%
\pgfpathlineto{\pgfqpoint{1.495045in}{1.307006in}}%
\pgfpathlineto{\pgfqpoint{1.482450in}{1.309178in}}%
\pgfpathlineto{\pgfqpoint{1.469737in}{1.311148in}}%
\pgfpathlineto{\pgfqpoint{1.456917in}{1.312917in}}%
\pgfpathclose%
\pgfusepath{fill}%
\end{pgfscope}%
\begin{pgfscope}%
\pgfpathrectangle{\pgfqpoint{0.329460in}{0.284240in}}{\pgfqpoint{1.989680in}{1.989680in}}%
\pgfusepath{clip}%
\pgfsetbuttcap%
\pgfsetroundjoin%
\definecolor{currentfill}{rgb}{0.283072,0.130895,0.449241}%
\pgfsetfillcolor{currentfill}%
\pgfsetlinewidth{0.000000pt}%
\definecolor{currentstroke}{rgb}{0.000000,0.000000,0.000000}%
\pgfsetstrokecolor{currentstroke}%
\pgfsetdash{}{0pt}%
\pgfpathmoveto{\pgfqpoint{1.196243in}{1.304910in}}%
\pgfpathlineto{\pgfqpoint{1.195108in}{1.301967in}}%
\pgfpathlineto{\pgfqpoint{1.193974in}{1.299144in}}%
\pgfpathlineto{\pgfqpoint{1.192839in}{1.296444in}}%
\pgfpathlineto{\pgfqpoint{1.191705in}{1.293870in}}%
\pgfpathlineto{\pgfqpoint{1.204552in}{1.296291in}}%
\pgfpathlineto{\pgfqpoint{1.217531in}{1.298506in}}%
\pgfpathlineto{\pgfqpoint{1.230631in}{1.300515in}}%
\pgfpathlineto{\pgfqpoint{1.243838in}{1.302316in}}%
\pgfpathlineto{\pgfqpoint{1.244600in}{1.304826in}}%
\pgfpathlineto{\pgfqpoint{1.245363in}{1.307462in}}%
\pgfpathlineto{\pgfqpoint{1.246126in}{1.310222in}}%
\pgfpathlineto{\pgfqpoint{1.246888in}{1.313101in}}%
\pgfpathlineto{\pgfqpoint{1.234057in}{1.311355in}}%
\pgfpathlineto{\pgfqpoint{1.221332in}{1.309406in}}%
\pgfpathlineto{\pgfqpoint{1.208723in}{1.307258in}}%
\pgfpathlineto{\pgfqpoint{1.196243in}{1.304910in}}%
\pgfpathclose%
\pgfusepath{fill}%
\end{pgfscope}%
\begin{pgfscope}%
\pgfpathrectangle{\pgfqpoint{0.329460in}{0.284240in}}{\pgfqpoint{1.989680in}{1.989680in}}%
\pgfusepath{clip}%
\pgfsetbuttcap%
\pgfsetroundjoin%
\definecolor{currentfill}{rgb}{0.267004,0.004874,0.329415}%
\pgfsetfillcolor{currentfill}%
\pgfsetlinewidth{0.000000pt}%
\definecolor{currentstroke}{rgb}{0.000000,0.000000,0.000000}%
\pgfsetstrokecolor{currentstroke}%
\pgfsetdash{}{0pt}%
\pgfpathmoveto{\pgfqpoint{1.691342in}{1.232086in}}%
\pgfpathlineto{\pgfqpoint{1.693459in}{1.233551in}}%
\pgfpathlineto{\pgfqpoint{1.695580in}{1.235240in}}%
\pgfpathlineto{\pgfqpoint{1.697704in}{1.237157in}}%
\pgfpathlineto{\pgfqpoint{1.699832in}{1.239307in}}%
\pgfpathlineto{\pgfqpoint{1.712333in}{1.233641in}}%
\pgfpathlineto{\pgfqpoint{1.724508in}{1.227773in}}%
\pgfpathlineto{\pgfqpoint{1.736345in}{1.221707in}}%
\pgfpathlineto{\pgfqpoint{1.747832in}{1.215449in}}%
\pgfpathlineto{\pgfqpoint{1.745402in}{1.213437in}}%
\pgfpathlineto{\pgfqpoint{1.742976in}{1.211658in}}%
\pgfpathlineto{\pgfqpoint{1.740555in}{1.210109in}}%
\pgfpathlineto{\pgfqpoint{1.738137in}{1.208784in}}%
\pgfpathlineto{\pgfqpoint{1.726940in}{1.214896in}}%
\pgfpathlineto{\pgfqpoint{1.715401in}{1.220820in}}%
\pgfpathlineto{\pgfqpoint{1.703531in}{1.226552in}}%
\pgfpathlineto{\pgfqpoint{1.691342in}{1.232086in}}%
\pgfpathclose%
\pgfusepath{fill}%
\end{pgfscope}%
\begin{pgfscope}%
\pgfpathrectangle{\pgfqpoint{0.329460in}{0.284240in}}{\pgfqpoint{1.989680in}{1.989680in}}%
\pgfusepath{clip}%
\pgfsetbuttcap%
\pgfsetroundjoin%
\definecolor{currentfill}{rgb}{0.277941,0.056324,0.381191}%
\pgfsetfillcolor{currentfill}%
\pgfsetlinewidth{0.000000pt}%
\definecolor{currentstroke}{rgb}{0.000000,0.000000,0.000000}%
\pgfsetstrokecolor{currentstroke}%
\pgfsetdash{}{0pt}%
\pgfpathmoveto{\pgfqpoint{0.924508in}{1.234490in}}%
\pgfpathlineto{\pgfqpoint{0.921966in}{1.238770in}}%
\pgfpathlineto{\pgfqpoint{0.919418in}{1.243329in}}%
\pgfpathlineto{\pgfqpoint{0.916862in}{1.248171in}}%
\pgfpathlineto{\pgfqpoint{0.914300in}{1.253302in}}%
\pgfpathlineto{\pgfqpoint{0.926331in}{1.260159in}}%
\pgfpathlineto{\pgfqpoint{0.938747in}{1.266814in}}%
\pgfpathlineto{\pgfqpoint{0.951535in}{1.273261in}}%
\pgfpathlineto{\pgfqpoint{0.964682in}{1.279497in}}%
\pgfpathlineto{\pgfqpoint{0.966939in}{1.274234in}}%
\pgfpathlineto{\pgfqpoint{0.969190in}{1.269259in}}%
\pgfpathlineto{\pgfqpoint{0.971436in}{1.264567in}}%
\pgfpathlineto{\pgfqpoint{0.973676in}{1.260152in}}%
\pgfpathlineto{\pgfqpoint{0.960844in}{1.254043in}}%
\pgfpathlineto{\pgfqpoint{0.948364in}{1.247726in}}%
\pgfpathlineto{\pgfqpoint{0.936248in}{1.241207in}}%
\pgfpathlineto{\pgfqpoint{0.924508in}{1.234490in}}%
\pgfpathclose%
\pgfusepath{fill}%
\end{pgfscope}%
\begin{pgfscope}%
\pgfpathrectangle{\pgfqpoint{0.329460in}{0.284240in}}{\pgfqpoint{1.989680in}{1.989680in}}%
\pgfusepath{clip}%
\pgfsetbuttcap%
\pgfsetroundjoin%
\definecolor{currentfill}{rgb}{0.201239,0.383670,0.554294}%
\pgfsetfillcolor{currentfill}%
\pgfsetlinewidth{0.000000pt}%
\definecolor{currentstroke}{rgb}{0.000000,0.000000,0.000000}%
\pgfsetstrokecolor{currentstroke}%
\pgfsetdash{}{0pt}%
\pgfpathmoveto{\pgfqpoint{1.706895in}{1.482911in}}%
\pgfpathlineto{\pgfqpoint{1.708886in}{1.495561in}}%
\pgfpathlineto{\pgfqpoint{1.710885in}{1.508621in}}%
\pgfpathlineto{\pgfqpoint{1.712895in}{1.522098in}}%
\pgfpathlineto{\pgfqpoint{1.714913in}{1.536000in}}%
\pgfpathlineto{\pgfqpoint{1.731816in}{1.530339in}}%
\pgfpathlineto{\pgfqpoint{1.748388in}{1.524418in}}%
\pgfpathlineto{\pgfqpoint{1.764613in}{1.518241in}}%
\pgfpathlineto{\pgfqpoint{1.780476in}{1.511812in}}%
\pgfpathlineto{\pgfqpoint{1.778084in}{1.497990in}}%
\pgfpathlineto{\pgfqpoint{1.775703in}{1.484592in}}%
\pgfpathlineto{\pgfqpoint{1.773333in}{1.471614in}}%
\pgfpathlineto{\pgfqpoint{1.770975in}{1.459049in}}%
\pgfpathlineto{\pgfqpoint{1.755473in}{1.465390in}}%
\pgfpathlineto{\pgfqpoint{1.739615in}{1.471484in}}%
\pgfpathlineto{\pgfqpoint{1.723418in}{1.477326in}}%
\pgfpathlineto{\pgfqpoint{1.706895in}{1.482911in}}%
\pgfpathclose%
\pgfusepath{fill}%
\end{pgfscope}%
\begin{pgfscope}%
\pgfpathrectangle{\pgfqpoint{0.329460in}{0.284240in}}{\pgfqpoint{1.989680in}{1.989680in}}%
\pgfusepath{clip}%
\pgfsetbuttcap%
\pgfsetroundjoin%
\definecolor{currentfill}{rgb}{0.272594,0.025563,0.353093}%
\pgfsetfillcolor{currentfill}%
\pgfsetlinewidth{0.000000pt}%
\definecolor{currentstroke}{rgb}{0.000000,0.000000,0.000000}%
\pgfsetstrokecolor{currentstroke}%
\pgfsetdash{}{0pt}%
\pgfpathmoveto{\pgfqpoint{0.934616in}{1.220065in}}%
\pgfpathlineto{\pgfqpoint{0.932098in}{1.223276in}}%
\pgfpathlineto{\pgfqpoint{0.929574in}{1.226748in}}%
\pgfpathlineto{\pgfqpoint{0.927044in}{1.230484in}}%
\pgfpathlineto{\pgfqpoint{0.924508in}{1.234490in}}%
\pgfpathlineto{\pgfqpoint{0.936248in}{1.241207in}}%
\pgfpathlineto{\pgfqpoint{0.948364in}{1.247726in}}%
\pgfpathlineto{\pgfqpoint{0.960844in}{1.254043in}}%
\pgfpathlineto{\pgfqpoint{0.973676in}{1.260152in}}%
\pgfpathlineto{\pgfqpoint{0.975910in}{1.256011in}}%
\pgfpathlineto{\pgfqpoint{0.978139in}{1.252138in}}%
\pgfpathlineto{\pgfqpoint{0.980364in}{1.248529in}}%
\pgfpathlineto{\pgfqpoint{0.982583in}{1.245179in}}%
\pgfpathlineto{\pgfqpoint{0.970063in}{1.239200in}}%
\pgfpathlineto{\pgfqpoint{0.957887in}{1.233017in}}%
\pgfpathlineto{\pgfqpoint{0.946067in}{1.226637in}}%
\pgfpathlineto{\pgfqpoint{0.934616in}{1.220065in}}%
\pgfpathclose%
\pgfusepath{fill}%
\end{pgfscope}%
\begin{pgfscope}%
\pgfpathrectangle{\pgfqpoint{0.329460in}{0.284240in}}{\pgfqpoint{1.989680in}{1.989680in}}%
\pgfusepath{clip}%
\pgfsetbuttcap%
\pgfsetroundjoin%
\definecolor{currentfill}{rgb}{0.271305,0.019942,0.347269}%
\pgfsetfillcolor{currentfill}%
\pgfsetlinewidth{0.000000pt}%
\definecolor{currentstroke}{rgb}{0.000000,0.000000,0.000000}%
\pgfsetstrokecolor{currentstroke}%
\pgfsetdash{}{0pt}%
\pgfpathmoveto{\pgfqpoint{1.026201in}{1.225960in}}%
\pgfpathlineto{\pgfqpoint{1.024045in}{1.225019in}}%
\pgfpathlineto{\pgfqpoint{1.021887in}{1.224255in}}%
\pgfpathlineto{\pgfqpoint{1.019728in}{1.223671in}}%
\pgfpathlineto{\pgfqpoint{1.017566in}{1.223272in}}%
\pgfpathlineto{\pgfqpoint{1.029171in}{1.228514in}}%
\pgfpathlineto{\pgfqpoint{1.041068in}{1.233564in}}%
\pgfpathlineto{\pgfqpoint{1.053245in}{1.238418in}}%
\pgfpathlineto{\pgfqpoint{1.065692in}{1.243073in}}%
\pgfpathlineto{\pgfqpoint{1.067536in}{1.243343in}}%
\pgfpathlineto{\pgfqpoint{1.069379in}{1.243797in}}%
\pgfpathlineto{\pgfqpoint{1.071219in}{1.244431in}}%
\pgfpathlineto{\pgfqpoint{1.073059in}{1.245242in}}%
\pgfpathlineto{\pgfqpoint{1.060939in}{1.240709in}}%
\pgfpathlineto{\pgfqpoint{1.049081in}{1.235982in}}%
\pgfpathlineto{\pgfqpoint{1.037498in}{1.231064in}}%
\pgfpathlineto{\pgfqpoint{1.026201in}{1.225960in}}%
\pgfpathclose%
\pgfusepath{fill}%
\end{pgfscope}%
\begin{pgfscope}%
\pgfpathrectangle{\pgfqpoint{0.329460in}{0.284240in}}{\pgfqpoint{1.989680in}{1.989680in}}%
\pgfusepath{clip}%
\pgfsetbuttcap%
\pgfsetroundjoin%
\definecolor{currentfill}{rgb}{0.279566,0.067836,0.391917}%
\pgfsetfillcolor{currentfill}%
\pgfsetlinewidth{0.000000pt}%
\definecolor{currentstroke}{rgb}{0.000000,0.000000,0.000000}%
\pgfsetstrokecolor{currentstroke}%
\pgfsetdash{}{0pt}%
\pgfpathmoveto{\pgfqpoint{1.087733in}{1.257630in}}%
\pgfpathlineto{\pgfqpoint{1.085902in}{1.255541in}}%
\pgfpathlineto{\pgfqpoint{1.084070in}{1.253599in}}%
\pgfpathlineto{\pgfqpoint{1.082237in}{1.251809in}}%
\pgfpathlineto{\pgfqpoint{1.080404in}{1.250172in}}%
\pgfpathlineto{\pgfqpoint{1.092444in}{1.254388in}}%
\pgfpathlineto{\pgfqpoint{1.104718in}{1.258408in}}%
\pgfpathlineto{\pgfqpoint{1.117215in}{1.262229in}}%
\pgfpathlineto{\pgfqpoint{1.129922in}{1.265847in}}%
\pgfpathlineto{\pgfqpoint{1.131417in}{1.267373in}}%
\pgfpathlineto{\pgfqpoint{1.132911in}{1.269054in}}%
\pgfpathlineto{\pgfqpoint{1.134404in}{1.270885in}}%
\pgfpathlineto{\pgfqpoint{1.135897in}{1.272863in}}%
\pgfpathlineto{\pgfqpoint{1.123536in}{1.269347in}}%
\pgfpathlineto{\pgfqpoint{1.111381in}{1.265634in}}%
\pgfpathlineto{\pgfqpoint{1.099443in}{1.261727in}}%
\pgfpathlineto{\pgfqpoint{1.087733in}{1.257630in}}%
\pgfpathclose%
\pgfusepath{fill}%
\end{pgfscope}%
\begin{pgfscope}%
\pgfpathrectangle{\pgfqpoint{0.329460in}{0.284240in}}{\pgfqpoint{1.989680in}{1.989680in}}%
\pgfusepath{clip}%
\pgfsetbuttcap%
\pgfsetroundjoin%
\definecolor{currentfill}{rgb}{0.282327,0.094955,0.417331}%
\pgfsetfillcolor{currentfill}%
\pgfsetlinewidth{0.000000pt}%
\definecolor{currentstroke}{rgb}{0.000000,0.000000,0.000000}%
\pgfsetstrokecolor{currentstroke}%
\pgfsetdash{}{0pt}%
\pgfpathmoveto{\pgfqpoint{0.914300in}{1.253302in}}%
\pgfpathlineto{\pgfqpoint{0.911731in}{1.258727in}}%
\pgfpathlineto{\pgfqpoint{0.909154in}{1.264450in}}%
\pgfpathlineto{\pgfqpoint{0.906569in}{1.270477in}}%
\pgfpathlineto{\pgfqpoint{0.903977in}{1.276812in}}%
\pgfpathlineto{\pgfqpoint{0.916303in}{1.283804in}}%
\pgfpathlineto{\pgfqpoint{0.929023in}{1.290590in}}%
\pgfpathlineto{\pgfqpoint{0.942122in}{1.297165in}}%
\pgfpathlineto{\pgfqpoint{0.955588in}{1.303523in}}%
\pgfpathlineto{\pgfqpoint{0.957871in}{1.297060in}}%
\pgfpathlineto{\pgfqpoint{0.960148in}{1.290905in}}%
\pgfpathlineto{\pgfqpoint{0.962418in}{1.285052in}}%
\pgfpathlineto{\pgfqpoint{0.964682in}{1.279497in}}%
\pgfpathlineto{\pgfqpoint{0.951535in}{1.273261in}}%
\pgfpathlineto{\pgfqpoint{0.938747in}{1.266814in}}%
\pgfpathlineto{\pgfqpoint{0.926331in}{1.260159in}}%
\pgfpathlineto{\pgfqpoint{0.914300in}{1.253302in}}%
\pgfpathclose%
\pgfusepath{fill}%
\end{pgfscope}%
\begin{pgfscope}%
\pgfpathrectangle{\pgfqpoint{0.329460in}{0.284240in}}{\pgfqpoint{1.989680in}{1.989680in}}%
\pgfusepath{clip}%
\pgfsetbuttcap%
\pgfsetroundjoin%
\definecolor{currentfill}{rgb}{0.280255,0.165693,0.476498}%
\pgfsetfillcolor{currentfill}%
\pgfsetlinewidth{0.000000pt}%
\definecolor{currentstroke}{rgb}{0.000000,0.000000,0.000000}%
\pgfsetstrokecolor{currentstroke}%
\pgfsetdash{}{0pt}%
\pgfpathmoveto{\pgfqpoint{1.300558in}{1.330533in}}%
\pgfpathlineto{\pgfqpoint{1.300176in}{1.327246in}}%
\pgfpathlineto{\pgfqpoint{1.299795in}{1.324066in}}%
\pgfpathlineto{\pgfqpoint{1.299413in}{1.320996in}}%
\pgfpathlineto{\pgfqpoint{1.299032in}{1.318039in}}%
\pgfpathlineto{\pgfqpoint{1.312215in}{1.318757in}}%
\pgfpathlineto{\pgfqpoint{1.325432in}{1.319267in}}%
\pgfpathlineto{\pgfqpoint{1.338672in}{1.319568in}}%
\pgfpathlineto{\pgfqpoint{1.351924in}{1.319661in}}%
\pgfpathlineto{\pgfqpoint{1.351918in}{1.322606in}}%
\pgfpathlineto{\pgfqpoint{1.351913in}{1.325663in}}%
\pgfpathlineto{\pgfqpoint{1.351908in}{1.328830in}}%
\pgfpathlineto{\pgfqpoint{1.351902in}{1.332104in}}%
\pgfpathlineto{\pgfqpoint{1.339038in}{1.332014in}}%
\pgfpathlineto{\pgfqpoint{1.326185in}{1.331722in}}%
\pgfpathlineto{\pgfqpoint{1.313354in}{1.331228in}}%
\pgfpathlineto{\pgfqpoint{1.300558in}{1.330533in}}%
\pgfpathclose%
\pgfusepath{fill}%
\end{pgfscope}%
\begin{pgfscope}%
\pgfpathrectangle{\pgfqpoint{0.329460in}{0.284240in}}{\pgfqpoint{1.989680in}{1.989680in}}%
\pgfusepath{clip}%
\pgfsetbuttcap%
\pgfsetroundjoin%
\definecolor{currentfill}{rgb}{0.280255,0.165693,0.476498}%
\pgfsetfillcolor{currentfill}%
\pgfsetlinewidth{0.000000pt}%
\definecolor{currentstroke}{rgb}{0.000000,0.000000,0.000000}%
\pgfsetstrokecolor{currentstroke}%
\pgfsetdash{}{0pt}%
\pgfpathmoveto{\pgfqpoint{1.351902in}{1.332104in}}%
\pgfpathlineto{\pgfqpoint{1.351908in}{1.328830in}}%
\pgfpathlineto{\pgfqpoint{1.351913in}{1.325663in}}%
\pgfpathlineto{\pgfqpoint{1.351918in}{1.322606in}}%
\pgfpathlineto{\pgfqpoint{1.351924in}{1.319661in}}%
\pgfpathlineto{\pgfqpoint{1.365175in}{1.319545in}}%
\pgfpathlineto{\pgfqpoint{1.378413in}{1.319221in}}%
\pgfpathlineto{\pgfqpoint{1.391627in}{1.318688in}}%
\pgfpathlineto{\pgfqpoint{1.404805in}{1.317947in}}%
\pgfpathlineto{\pgfqpoint{1.404413in}{1.320904in}}%
\pgfpathlineto{\pgfqpoint{1.404021in}{1.323975in}}%
\pgfpathlineto{\pgfqpoint{1.403629in}{1.327156in}}%
\pgfpathlineto{\pgfqpoint{1.403237in}{1.330443in}}%
\pgfpathlineto{\pgfqpoint{1.390445in}{1.331161in}}%
\pgfpathlineto{\pgfqpoint{1.377617in}{1.331677in}}%
\pgfpathlineto{\pgfqpoint{1.364766in}{1.331992in}}%
\pgfpathlineto{\pgfqpoint{1.351902in}{1.332104in}}%
\pgfpathclose%
\pgfusepath{fill}%
\end{pgfscope}%
\begin{pgfscope}%
\pgfpathrectangle{\pgfqpoint{0.329460in}{0.284240in}}{\pgfqpoint{1.989680in}{1.989680in}}%
\pgfusepath{clip}%
\pgfsetbuttcap%
\pgfsetroundjoin%
\definecolor{currentfill}{rgb}{0.268510,0.009605,0.335427}%
\pgfsetfillcolor{currentfill}%
\pgfsetlinewidth{0.000000pt}%
\definecolor{currentstroke}{rgb}{0.000000,0.000000,0.000000}%
\pgfsetstrokecolor{currentstroke}%
\pgfsetdash{}{0pt}%
\pgfpathmoveto{\pgfqpoint{0.944636in}{1.209728in}}%
\pgfpathlineto{\pgfqpoint{0.942138in}{1.211945in}}%
\pgfpathlineto{\pgfqpoint{0.939636in}{1.214403in}}%
\pgfpathlineto{\pgfqpoint{0.937129in}{1.217108in}}%
\pgfpathlineto{\pgfqpoint{0.934616in}{1.220065in}}%
\pgfpathlineto{\pgfqpoint{0.946067in}{1.226637in}}%
\pgfpathlineto{\pgfqpoint{0.957887in}{1.233017in}}%
\pgfpathlineto{\pgfqpoint{0.970063in}{1.239200in}}%
\pgfpathlineto{\pgfqpoint{0.982583in}{1.245179in}}%
\pgfpathlineto{\pgfqpoint{0.984798in}{1.242083in}}%
\pgfpathlineto{\pgfqpoint{0.987008in}{1.239238in}}%
\pgfpathlineto{\pgfqpoint{0.989213in}{1.236639in}}%
\pgfpathlineto{\pgfqpoint{0.991415in}{1.234281in}}%
\pgfpathlineto{\pgfqpoint{0.979203in}{1.228435in}}%
\pgfpathlineto{\pgfqpoint{0.967328in}{1.222391in}}%
\pgfpathlineto{\pgfqpoint{0.955802in}{1.216154in}}%
\pgfpathlineto{\pgfqpoint{0.944636in}{1.209728in}}%
\pgfpathclose%
\pgfusepath{fill}%
\end{pgfscope}%
\begin{pgfscope}%
\pgfpathrectangle{\pgfqpoint{0.329460in}{0.284240in}}{\pgfqpoint{1.989680in}{1.989680in}}%
\pgfusepath{clip}%
\pgfsetbuttcap%
\pgfsetroundjoin%
\definecolor{currentfill}{rgb}{0.276194,0.190074,0.493001}%
\pgfsetfillcolor{currentfill}%
\pgfsetlinewidth{0.000000pt}%
\definecolor{currentstroke}{rgb}{0.000000,0.000000,0.000000}%
\pgfsetstrokecolor{currentstroke}%
\pgfsetdash{}{0pt}%
\pgfpathmoveto{\pgfqpoint{1.743434in}{1.338120in}}%
\pgfpathlineto{\pgfqpoint{1.745681in}{1.346225in}}%
\pgfpathlineto{\pgfqpoint{1.747936in}{1.354669in}}%
\pgfpathlineto{\pgfqpoint{1.750199in}{1.363458in}}%
\pgfpathlineto{\pgfqpoint{1.752471in}{1.372597in}}%
\pgfpathlineto{\pgfqpoint{1.766916in}{1.366213in}}%
\pgfpathlineto{\pgfqpoint{1.780992in}{1.359600in}}%
\pgfpathlineto{\pgfqpoint{1.794684in}{1.352762in}}%
\pgfpathlineto{\pgfqpoint{1.807980in}{1.345704in}}%
\pgfpathlineto{\pgfqpoint{1.805383in}{1.336678in}}%
\pgfpathlineto{\pgfqpoint{1.802795in}{1.328004in}}%
\pgfpathlineto{\pgfqpoint{1.800218in}{1.319676in}}%
\pgfpathlineto{\pgfqpoint{1.797649in}{1.311689in}}%
\pgfpathlineto{\pgfqpoint{1.784666in}{1.318625in}}%
\pgfpathlineto{\pgfqpoint{1.771293in}{1.325345in}}%
\pgfpathlineto{\pgfqpoint{1.757545in}{1.331846in}}%
\pgfpathlineto{\pgfqpoint{1.743434in}{1.338120in}}%
\pgfpathclose%
\pgfusepath{fill}%
\end{pgfscope}%
\begin{pgfscope}%
\pgfpathrectangle{\pgfqpoint{0.329460in}{0.284240in}}{\pgfqpoint{1.989680in}{1.989680in}}%
\pgfusepath{clip}%
\pgfsetbuttcap%
\pgfsetroundjoin%
\definecolor{currentfill}{rgb}{0.282884,0.135920,0.453427}%
\pgfsetfillcolor{currentfill}%
\pgfsetlinewidth{0.000000pt}%
\definecolor{currentstroke}{rgb}{0.000000,0.000000,0.000000}%
\pgfsetstrokecolor{currentstroke}%
\pgfsetdash{}{0pt}%
\pgfpathmoveto{\pgfqpoint{0.903977in}{1.276812in}}%
\pgfpathlineto{\pgfqpoint{0.901376in}{1.283462in}}%
\pgfpathlineto{\pgfqpoint{0.898767in}{1.290430in}}%
\pgfpathlineto{\pgfqpoint{0.896149in}{1.297724in}}%
\pgfpathlineto{\pgfqpoint{0.893523in}{1.305347in}}%
\pgfpathlineto{\pgfqpoint{0.906149in}{1.312470in}}%
\pgfpathlineto{\pgfqpoint{0.919176in}{1.319382in}}%
\pgfpathlineto{\pgfqpoint{0.932591in}{1.326079in}}%
\pgfpathlineto{\pgfqpoint{0.946381in}{1.332554in}}%
\pgfpathlineto{\pgfqpoint{0.948694in}{1.324809in}}%
\pgfpathlineto{\pgfqpoint{0.950999in}{1.317392in}}%
\pgfpathlineto{\pgfqpoint{0.953297in}{1.310298in}}%
\pgfpathlineto{\pgfqpoint{0.955588in}{1.303523in}}%
\pgfpathlineto{\pgfqpoint{0.942122in}{1.297165in}}%
\pgfpathlineto{\pgfqpoint{0.929023in}{1.290590in}}%
\pgfpathlineto{\pgfqpoint{0.916303in}{1.283804in}}%
\pgfpathlineto{\pgfqpoint{0.903977in}{1.276812in}}%
\pgfpathclose%
\pgfusepath{fill}%
\end{pgfscope}%
\begin{pgfscope}%
\pgfpathrectangle{\pgfqpoint{0.329460in}{0.284240in}}{\pgfqpoint{1.989680in}{1.989680in}}%
\pgfusepath{clip}%
\pgfsetbuttcap%
\pgfsetroundjoin%
\definecolor{currentfill}{rgb}{0.267004,0.004874,0.329415}%
\pgfsetfillcolor{currentfill}%
\pgfsetlinewidth{0.000000pt}%
\definecolor{currentstroke}{rgb}{0.000000,0.000000,0.000000}%
\pgfsetstrokecolor{currentstroke}%
\pgfsetdash{}{0pt}%
\pgfpathmoveto{\pgfqpoint{1.682904in}{1.228384in}}%
\pgfpathlineto{\pgfqpoint{1.685009in}{1.228994in}}%
\pgfpathlineto{\pgfqpoint{1.687117in}{1.229811in}}%
\pgfpathlineto{\pgfqpoint{1.689228in}{1.230841in}}%
\pgfpathlineto{\pgfqpoint{1.691342in}{1.232086in}}%
\pgfpathlineto{\pgfqpoint{1.703531in}{1.226552in}}%
\pgfpathlineto{\pgfqpoint{1.715401in}{1.220820in}}%
\pgfpathlineto{\pgfqpoint{1.726940in}{1.214896in}}%
\pgfpathlineto{\pgfqpoint{1.738137in}{1.208784in}}%
\pgfpathlineto{\pgfqpoint{1.735724in}{1.207679in}}%
\pgfpathlineto{\pgfqpoint{1.733314in}{1.206791in}}%
\pgfpathlineto{\pgfqpoint{1.730907in}{1.206116in}}%
\pgfpathlineto{\pgfqpoint{1.728504in}{1.205649in}}%
\pgfpathlineto{\pgfqpoint{1.717594in}{1.211612in}}%
\pgfpathlineto{\pgfqpoint{1.706351in}{1.217392in}}%
\pgfpathlineto{\pgfqpoint{1.694783in}{1.222984in}}%
\pgfpathlineto{\pgfqpoint{1.682904in}{1.228384in}}%
\pgfpathclose%
\pgfusepath{fill}%
\end{pgfscope}%
\begin{pgfscope}%
\pgfpathrectangle{\pgfqpoint{0.329460in}{0.284240in}}{\pgfqpoint{1.989680in}{1.989680in}}%
\pgfusepath{clip}%
\pgfsetbuttcap%
\pgfsetroundjoin%
\definecolor{currentfill}{rgb}{0.274952,0.037752,0.364543}%
\pgfsetfillcolor{currentfill}%
\pgfsetlinewidth{0.000000pt}%
\definecolor{currentstroke}{rgb}{0.000000,0.000000,0.000000}%
\pgfsetstrokecolor{currentstroke}%
\pgfsetdash{}{0pt}%
\pgfpathmoveto{\pgfqpoint{1.611281in}{1.253929in}}%
\pgfpathlineto{\pgfqpoint{1.613042in}{1.252477in}}%
\pgfpathlineto{\pgfqpoint{1.614804in}{1.251186in}}%
\pgfpathlineto{\pgfqpoint{1.616567in}{1.250061in}}%
\pgfpathlineto{\pgfqpoint{1.618331in}{1.249104in}}%
\pgfpathlineto{\pgfqpoint{1.630676in}{1.244748in}}%
\pgfpathlineto{\pgfqpoint{1.642767in}{1.240194in}}%
\pgfpathlineto{\pgfqpoint{1.654595in}{1.235445in}}%
\pgfpathlineto{\pgfqpoint{1.666146in}{1.230506in}}%
\pgfpathlineto{\pgfqpoint{1.664060in}{1.231589in}}%
\pgfpathlineto{\pgfqpoint{1.661976in}{1.232841in}}%
\pgfpathlineto{\pgfqpoint{1.659892in}{1.234259in}}%
\pgfpathlineto{\pgfqpoint{1.657810in}{1.235838in}}%
\pgfpathlineto{\pgfqpoint{1.646570in}{1.240642in}}%
\pgfpathlineto{\pgfqpoint{1.635061in}{1.245261in}}%
\pgfpathlineto{\pgfqpoint{1.623294in}{1.249692in}}%
\pgfpathlineto{\pgfqpoint{1.611281in}{1.253929in}}%
\pgfpathclose%
\pgfusepath{fill}%
\end{pgfscope}%
\begin{pgfscope}%
\pgfpathrectangle{\pgfqpoint{0.329460in}{0.284240in}}{\pgfqpoint{1.989680in}{1.989680in}}%
\pgfusepath{clip}%
\pgfsetbuttcap%
\pgfsetroundjoin%
\definecolor{currentfill}{rgb}{0.280255,0.165693,0.476498}%
\pgfsetfillcolor{currentfill}%
\pgfsetlinewidth{0.000000pt}%
\definecolor{currentstroke}{rgb}{0.000000,0.000000,0.000000}%
\pgfsetstrokecolor{currentstroke}%
\pgfsetdash{}{0pt}%
\pgfpathmoveto{\pgfqpoint{1.249941in}{1.325749in}}%
\pgfpathlineto{\pgfqpoint{1.249177in}{1.322424in}}%
\pgfpathlineto{\pgfqpoint{1.248414in}{1.319205in}}%
\pgfpathlineto{\pgfqpoint{1.247651in}{1.316096in}}%
\pgfpathlineto{\pgfqpoint{1.246888in}{1.313101in}}%
\pgfpathlineto{\pgfqpoint{1.259813in}{1.314644in}}%
\pgfpathlineto{\pgfqpoint{1.272820in}{1.315982in}}%
\pgfpathlineto{\pgfqpoint{1.285897in}{1.317114in}}%
\pgfpathlineto{\pgfqpoint{1.299032in}{1.318039in}}%
\pgfpathlineto{\pgfqpoint{1.299413in}{1.320996in}}%
\pgfpathlineto{\pgfqpoint{1.299795in}{1.324066in}}%
\pgfpathlineto{\pgfqpoint{1.300176in}{1.327246in}}%
\pgfpathlineto{\pgfqpoint{1.300558in}{1.330533in}}%
\pgfpathlineto{\pgfqpoint{1.287806in}{1.329636in}}%
\pgfpathlineto{\pgfqpoint{1.275112in}{1.328539in}}%
\pgfpathlineto{\pgfqpoint{1.262487in}{1.327243in}}%
\pgfpathlineto{\pgfqpoint{1.249941in}{1.325749in}}%
\pgfpathclose%
\pgfusepath{fill}%
\end{pgfscope}%
\begin{pgfscope}%
\pgfpathrectangle{\pgfqpoint{0.329460in}{0.284240in}}{\pgfqpoint{1.989680in}{1.989680in}}%
\pgfusepath{clip}%
\pgfsetbuttcap%
\pgfsetroundjoin%
\definecolor{currentfill}{rgb}{0.280255,0.165693,0.476498}%
\pgfsetfillcolor{currentfill}%
\pgfsetlinewidth{0.000000pt}%
\definecolor{currentstroke}{rgb}{0.000000,0.000000,0.000000}%
\pgfsetstrokecolor{currentstroke}%
\pgfsetdash{}{0pt}%
\pgfpathmoveto{\pgfqpoint{1.403237in}{1.330443in}}%
\pgfpathlineto{\pgfqpoint{1.403629in}{1.327156in}}%
\pgfpathlineto{\pgfqpoint{1.404021in}{1.323975in}}%
\pgfpathlineto{\pgfqpoint{1.404413in}{1.320904in}}%
\pgfpathlineto{\pgfqpoint{1.404805in}{1.317947in}}%
\pgfpathlineto{\pgfqpoint{1.417935in}{1.316998in}}%
\pgfpathlineto{\pgfqpoint{1.431004in}{1.315843in}}%
\pgfpathlineto{\pgfqpoint{1.444003in}{1.314482in}}%
\pgfpathlineto{\pgfqpoint{1.456917in}{1.312917in}}%
\pgfpathlineto{\pgfqpoint{1.456144in}{1.315914in}}%
\pgfpathlineto{\pgfqpoint{1.455370in}{1.319024in}}%
\pgfpathlineto{\pgfqpoint{1.454597in}{1.322244in}}%
\pgfpathlineto{\pgfqpoint{1.453823in}{1.325570in}}%
\pgfpathlineto{\pgfqpoint{1.441287in}{1.327087in}}%
\pgfpathlineto{\pgfqpoint{1.428669in}{1.328405in}}%
\pgfpathlineto{\pgfqpoint{1.415982in}{1.329524in}}%
\pgfpathlineto{\pgfqpoint{1.403237in}{1.330443in}}%
\pgfpathclose%
\pgfusepath{fill}%
\end{pgfscope}%
\begin{pgfscope}%
\pgfpathrectangle{\pgfqpoint{0.329460in}{0.284240in}}{\pgfqpoint{1.989680in}{1.989680in}}%
\pgfusepath{clip}%
\pgfsetbuttcap%
\pgfsetroundjoin%
\definecolor{currentfill}{rgb}{0.267004,0.004874,0.329415}%
\pgfsetfillcolor{currentfill}%
\pgfsetlinewidth{0.000000pt}%
\definecolor{currentstroke}{rgb}{0.000000,0.000000,0.000000}%
\pgfsetstrokecolor{currentstroke}%
\pgfsetdash{}{0pt}%
\pgfpathmoveto{\pgfqpoint{0.954582in}{1.203197in}}%
\pgfpathlineto{\pgfqpoint{0.952102in}{1.204488in}}%
\pgfpathlineto{\pgfqpoint{0.949617in}{1.206004in}}%
\pgfpathlineto{\pgfqpoint{0.947129in}{1.207750in}}%
\pgfpathlineto{\pgfqpoint{0.944636in}{1.209728in}}%
\pgfpathlineto{\pgfqpoint{0.955802in}{1.216154in}}%
\pgfpathlineto{\pgfqpoint{0.967328in}{1.222391in}}%
\pgfpathlineto{\pgfqpoint{0.979203in}{1.228435in}}%
\pgfpathlineto{\pgfqpoint{0.991415in}{1.234281in}}%
\pgfpathlineto{\pgfqpoint{0.993612in}{1.232160in}}%
\pgfpathlineto{\pgfqpoint{0.995806in}{1.230272in}}%
\pgfpathlineto{\pgfqpoint{0.997996in}{1.228612in}}%
\pgfpathlineto{\pgfqpoint{1.000183in}{1.227176in}}%
\pgfpathlineto{\pgfqpoint{0.988277in}{1.221466in}}%
\pgfpathlineto{\pgfqpoint{0.976700in}{1.215563in}}%
\pgfpathlineto{\pgfqpoint{0.965465in}{1.209472in}}%
\pgfpathlineto{\pgfqpoint{0.954582in}{1.203197in}}%
\pgfpathclose%
\pgfusepath{fill}%
\end{pgfscope}%
\begin{pgfscope}%
\pgfpathrectangle{\pgfqpoint{0.329460in}{0.284240in}}{\pgfqpoint{1.989680in}{1.989680in}}%
\pgfusepath{clip}%
\pgfsetbuttcap%
\pgfsetroundjoin%
\definecolor{currentfill}{rgb}{0.201239,0.383670,0.554294}%
\pgfsetfillcolor{currentfill}%
\pgfsetlinewidth{0.000000pt}%
\definecolor{currentstroke}{rgb}{0.000000,0.000000,0.000000}%
\pgfsetstrokecolor{currentstroke}%
\pgfsetdash{}{0pt}%
\pgfpathmoveto{\pgfqpoint{0.917930in}{1.453208in}}%
\pgfpathlineto{\pgfqpoint{0.915494in}{1.465753in}}%
\pgfpathlineto{\pgfqpoint{0.913046in}{1.478711in}}%
\pgfpathlineto{\pgfqpoint{0.910586in}{1.492088in}}%
\pgfpathlineto{\pgfqpoint{0.908115in}{1.505891in}}%
\pgfpathlineto{\pgfqpoint{0.923643in}{1.512539in}}%
\pgfpathlineto{\pgfqpoint{0.939547in}{1.518940in}}%
\pgfpathlineto{\pgfqpoint{0.955812in}{1.525089in}}%
\pgfpathlineto{\pgfqpoint{0.972421in}{1.530981in}}%
\pgfpathlineto{\pgfqpoint{0.974525in}{1.517096in}}%
\pgfpathlineto{\pgfqpoint{0.976619in}{1.503635in}}%
\pgfpathlineto{\pgfqpoint{0.978703in}{1.490592in}}%
\pgfpathlineto{\pgfqpoint{0.980777in}{1.477959in}}%
\pgfpathlineto{\pgfqpoint{0.964543in}{1.472146in}}%
\pgfpathlineto{\pgfqpoint{0.948647in}{1.466080in}}%
\pgfpathlineto{\pgfqpoint{0.933105in}{1.459765in}}%
\pgfpathlineto{\pgfqpoint{0.917930in}{1.453208in}}%
\pgfpathclose%
\pgfusepath{fill}%
\end{pgfscope}%
\begin{pgfscope}%
\pgfpathrectangle{\pgfqpoint{0.329460in}{0.284240in}}{\pgfqpoint{1.989680in}{1.989680in}}%
\pgfusepath{clip}%
\pgfsetbuttcap%
\pgfsetroundjoin%
\definecolor{currentfill}{rgb}{0.283072,0.130895,0.449241}%
\pgfsetfillcolor{currentfill}%
\pgfsetlinewidth{0.000000pt}%
\definecolor{currentstroke}{rgb}{0.000000,0.000000,0.000000}%
\pgfsetstrokecolor{currentstroke}%
\pgfsetdash{}{0pt}%
\pgfpathmoveto{\pgfqpoint{1.495045in}{1.307006in}}%
\pgfpathlineto{\pgfqpoint{1.496098in}{1.304080in}}%
\pgfpathlineto{\pgfqpoint{1.497151in}{1.301273in}}%
\pgfpathlineto{\pgfqpoint{1.498204in}{1.298589in}}%
\pgfpathlineto{\pgfqpoint{1.499257in}{1.296032in}}%
\pgfpathlineto{\pgfqpoint{1.512089in}{1.293589in}}%
\pgfpathlineto{\pgfqpoint{1.524776in}{1.290942in}}%
\pgfpathlineto{\pgfqpoint{1.537307in}{1.288095in}}%
\pgfpathlineto{\pgfqpoint{1.549670in}{1.285050in}}%
\pgfpathlineto{\pgfqpoint{1.548256in}{1.287690in}}%
\pgfpathlineto{\pgfqpoint{1.546843in}{1.290456in}}%
\pgfpathlineto{\pgfqpoint{1.545429in}{1.293346in}}%
\pgfpathlineto{\pgfqpoint{1.544015in}{1.296356in}}%
\pgfpathlineto{\pgfqpoint{1.532007in}{1.299309in}}%
\pgfpathlineto{\pgfqpoint{1.519835in}{1.302070in}}%
\pgfpathlineto{\pgfqpoint{1.507511in}{1.304637in}}%
\pgfpathlineto{\pgfqpoint{1.495045in}{1.307006in}}%
\pgfpathclose%
\pgfusepath{fill}%
\end{pgfscope}%
\begin{pgfscope}%
\pgfpathrectangle{\pgfqpoint{0.329460in}{0.284240in}}{\pgfqpoint{1.989680in}{1.989680in}}%
\pgfusepath{clip}%
\pgfsetbuttcap%
\pgfsetroundjoin%
\definecolor{currentfill}{rgb}{0.282327,0.094955,0.417331}%
\pgfsetfillcolor{currentfill}%
\pgfsetlinewidth{0.000000pt}%
\definecolor{currentstroke}{rgb}{0.000000,0.000000,0.000000}%
\pgfsetstrokecolor{currentstroke}%
\pgfsetdash{}{0pt}%
\pgfpathmoveto{\pgfqpoint{1.549670in}{1.285050in}}%
\pgfpathlineto{\pgfqpoint{1.551084in}{1.282539in}}%
\pgfpathlineto{\pgfqpoint{1.552498in}{1.280162in}}%
\pgfpathlineto{\pgfqpoint{1.553912in}{1.277922in}}%
\pgfpathlineto{\pgfqpoint{1.555327in}{1.275821in}}%
\pgfpathlineto{\pgfqpoint{1.567861in}{1.272483in}}%
\pgfpathlineto{\pgfqpoint{1.580200in}{1.268944in}}%
\pgfpathlineto{\pgfqpoint{1.592331in}{1.265210in}}%
\pgfpathlineto{\pgfqpoint{1.604245in}{1.261282in}}%
\pgfpathlineto{\pgfqpoint{1.602487in}{1.263488in}}%
\pgfpathlineto{\pgfqpoint{1.600730in}{1.265834in}}%
\pgfpathlineto{\pgfqpoint{1.598973in}{1.268318in}}%
\pgfpathlineto{\pgfqpoint{1.597216in}{1.270934in}}%
\pgfpathlineto{\pgfqpoint{1.585638in}{1.274748in}}%
\pgfpathlineto{\pgfqpoint{1.573847in}{1.278373in}}%
\pgfpathlineto{\pgfqpoint{1.561854in}{1.281808in}}%
\pgfpathlineto{\pgfqpoint{1.549670in}{1.285050in}}%
\pgfpathclose%
\pgfusepath{fill}%
\end{pgfscope}%
\begin{pgfscope}%
\pgfpathrectangle{\pgfqpoint{0.329460in}{0.284240in}}{\pgfqpoint{1.989680in}{1.989680in}}%
\pgfusepath{clip}%
\pgfsetbuttcap%
\pgfsetroundjoin%
\definecolor{currentfill}{rgb}{0.276194,0.190074,0.493001}%
\pgfsetfillcolor{currentfill}%
\pgfsetlinewidth{0.000000pt}%
\definecolor{currentstroke}{rgb}{0.000000,0.000000,0.000000}%
\pgfsetstrokecolor{currentstroke}%
\pgfsetdash{}{0pt}%
\pgfpathmoveto{\pgfqpoint{0.893523in}{1.305347in}}%
\pgfpathlineto{\pgfqpoint{0.890887in}{1.313306in}}%
\pgfpathlineto{\pgfqpoint{0.888241in}{1.321606in}}%
\pgfpathlineto{\pgfqpoint{0.885586in}{1.330253in}}%
\pgfpathlineto{\pgfqpoint{0.882921in}{1.339252in}}%
\pgfpathlineto{\pgfqpoint{0.895853in}{1.346499in}}%
\pgfpathlineto{\pgfqpoint{0.909193in}{1.353533in}}%
\pgfpathlineto{\pgfqpoint{0.922928in}{1.360346in}}%
\pgfpathlineto{\pgfqpoint{0.937046in}{1.366934in}}%
\pgfpathlineto{\pgfqpoint{0.939392in}{1.357818in}}%
\pgfpathlineto{\pgfqpoint{0.941730in}{1.349053in}}%
\pgfpathlineto{\pgfqpoint{0.944060in}{1.340634in}}%
\pgfpathlineto{\pgfqpoint{0.946381in}{1.332554in}}%
\pgfpathlineto{\pgfqpoint{0.932591in}{1.326079in}}%
\pgfpathlineto{\pgfqpoint{0.919176in}{1.319382in}}%
\pgfpathlineto{\pgfqpoint{0.906149in}{1.312470in}}%
\pgfpathlineto{\pgfqpoint{0.893523in}{1.305347in}}%
\pgfpathclose%
\pgfusepath{fill}%
\end{pgfscope}%
\begin{pgfscope}%
\pgfpathrectangle{\pgfqpoint{0.329460in}{0.284240in}}{\pgfqpoint{1.989680in}{1.989680in}}%
\pgfusepath{clip}%
\pgfsetbuttcap%
\pgfsetroundjoin%
\definecolor{currentfill}{rgb}{0.260571,0.246922,0.522828}%
\pgfsetfillcolor{currentfill}%
\pgfsetlinewidth{0.000000pt}%
\definecolor{currentstroke}{rgb}{0.000000,0.000000,0.000000}%
\pgfsetstrokecolor{currentstroke}%
\pgfsetdash{}{0pt}%
\pgfpathmoveto{\pgfqpoint{1.752471in}{1.372597in}}%
\pgfpathlineto{\pgfqpoint{1.754751in}{1.382093in}}%
\pgfpathlineto{\pgfqpoint{1.757040in}{1.391952in}}%
\pgfpathlineto{\pgfqpoint{1.759338in}{1.402178in}}%
\pgfpathlineto{\pgfqpoint{1.761646in}{1.412779in}}%
\pgfpathlineto{\pgfqpoint{1.776432in}{1.406292in}}%
\pgfpathlineto{\pgfqpoint{1.790840in}{1.399571in}}%
\pgfpathlineto{\pgfqpoint{1.804858in}{1.392622in}}%
\pgfpathlineto{\pgfqpoint{1.818471in}{1.385449in}}%
\pgfpathlineto{\pgfqpoint{1.815832in}{1.374955in}}%
\pgfpathlineto{\pgfqpoint{1.813204in}{1.364837in}}%
\pgfpathlineto{\pgfqpoint{1.810587in}{1.355089in}}%
\pgfpathlineto{\pgfqpoint{1.807980in}{1.345704in}}%
\pgfpathlineto{\pgfqpoint{1.794684in}{1.352762in}}%
\pgfpathlineto{\pgfqpoint{1.780992in}{1.359600in}}%
\pgfpathlineto{\pgfqpoint{1.766916in}{1.366213in}}%
\pgfpathlineto{\pgfqpoint{1.752471in}{1.372597in}}%
\pgfpathclose%
\pgfusepath{fill}%
\end{pgfscope}%
\begin{pgfscope}%
\pgfpathrectangle{\pgfqpoint{0.329460in}{0.284240in}}{\pgfqpoint{1.989680in}{1.989680in}}%
\pgfusepath{clip}%
\pgfsetbuttcap%
\pgfsetroundjoin%
\definecolor{currentfill}{rgb}{0.274952,0.037752,0.364543}%
\pgfsetfillcolor{currentfill}%
\pgfsetlinewidth{0.000000pt}%
\definecolor{currentstroke}{rgb}{0.000000,0.000000,0.000000}%
\pgfsetstrokecolor{currentstroke}%
\pgfsetdash{}{0pt}%
\pgfpathmoveto{\pgfqpoint{1.034809in}{1.231416in}}%
\pgfpathlineto{\pgfqpoint{1.032659in}{1.229805in}}%
\pgfpathlineto{\pgfqpoint{1.030508in}{1.228357in}}%
\pgfpathlineto{\pgfqpoint{1.028355in}{1.227074in}}%
\pgfpathlineto{\pgfqpoint{1.026201in}{1.225960in}}%
\pgfpathlineto{\pgfqpoint{1.037498in}{1.231064in}}%
\pgfpathlineto{\pgfqpoint{1.049081in}{1.235982in}}%
\pgfpathlineto{\pgfqpoint{1.060939in}{1.240709in}}%
\pgfpathlineto{\pgfqpoint{1.073059in}{1.245242in}}%
\pgfpathlineto{\pgfqpoint{1.074897in}{1.246225in}}%
\pgfpathlineto{\pgfqpoint{1.076733in}{1.247376in}}%
\pgfpathlineto{\pgfqpoint{1.078569in}{1.248693in}}%
\pgfpathlineto{\pgfqpoint{1.080404in}{1.250172in}}%
\pgfpathlineto{\pgfqpoint{1.068609in}{1.245763in}}%
\pgfpathlineto{\pgfqpoint{1.057071in}{1.241165in}}%
\pgfpathlineto{\pgfqpoint{1.045800in}{1.236381in}}%
\pgfpathlineto{\pgfqpoint{1.034809in}{1.231416in}}%
\pgfpathclose%
\pgfusepath{fill}%
\end{pgfscope}%
\begin{pgfscope}%
\pgfpathrectangle{\pgfqpoint{0.329460in}{0.284240in}}{\pgfqpoint{1.989680in}{1.989680in}}%
\pgfusepath{clip}%
\pgfsetbuttcap%
\pgfsetroundjoin%
\definecolor{currentfill}{rgb}{0.283072,0.130895,0.449241}%
\pgfsetfillcolor{currentfill}%
\pgfsetlinewidth{0.000000pt}%
\definecolor{currentstroke}{rgb}{0.000000,0.000000,0.000000}%
\pgfsetstrokecolor{currentstroke}%
\pgfsetdash{}{0pt}%
\pgfpathmoveto{\pgfqpoint{1.147833in}{1.293571in}}%
\pgfpathlineto{\pgfqpoint{1.146341in}{1.290540in}}%
\pgfpathlineto{\pgfqpoint{1.144849in}{1.287628in}}%
\pgfpathlineto{\pgfqpoint{1.143357in}{1.284840in}}%
\pgfpathlineto{\pgfqpoint{1.141866in}{1.282178in}}%
\pgfpathlineto{\pgfqpoint{1.154070in}{1.285398in}}%
\pgfpathlineto{\pgfqpoint{1.166452in}{1.288421in}}%
\pgfpathlineto{\pgfqpoint{1.179001in}{1.291246in}}%
\pgfpathlineto{\pgfqpoint{1.191705in}{1.293870in}}%
\pgfpathlineto{\pgfqpoint{1.192839in}{1.296444in}}%
\pgfpathlineto{\pgfqpoint{1.193974in}{1.299144in}}%
\pgfpathlineto{\pgfqpoint{1.195108in}{1.301967in}}%
\pgfpathlineto{\pgfqpoint{1.196243in}{1.304910in}}%
\pgfpathlineto{\pgfqpoint{1.183902in}{1.302365in}}%
\pgfpathlineto{\pgfqpoint{1.171713in}{1.299626in}}%
\pgfpathlineto{\pgfqpoint{1.159686in}{1.296693in}}%
\pgfpathlineto{\pgfqpoint{1.147833in}{1.293571in}}%
\pgfpathclose%
\pgfusepath{fill}%
\end{pgfscope}%
\begin{pgfscope}%
\pgfpathrectangle{\pgfqpoint{0.329460in}{0.284240in}}{\pgfqpoint{1.989680in}{1.989680in}}%
\pgfusepath{clip}%
\pgfsetbuttcap%
\pgfsetroundjoin%
\definecolor{currentfill}{rgb}{0.267004,0.004874,0.329415}%
\pgfsetfillcolor{currentfill}%
\pgfsetlinewidth{0.000000pt}%
\definecolor{currentstroke}{rgb}{0.000000,0.000000,0.000000}%
\pgfsetstrokecolor{currentstroke}%
\pgfsetdash{}{0pt}%
\pgfpathmoveto{\pgfqpoint{0.964464in}{1.200198in}}%
\pgfpathlineto{\pgfqpoint{0.961999in}{1.200631in}}%
\pgfpathlineto{\pgfqpoint{0.959530in}{1.201273in}}%
\pgfpathlineto{\pgfqpoint{0.957058in}{1.202127in}}%
\pgfpathlineto{\pgfqpoint{0.954582in}{1.203197in}}%
\pgfpathlineto{\pgfqpoint{0.965465in}{1.209472in}}%
\pgfpathlineto{\pgfqpoint{0.976700in}{1.215563in}}%
\pgfpathlineto{\pgfqpoint{0.988277in}{1.221466in}}%
\pgfpathlineto{\pgfqpoint{1.000183in}{1.227176in}}%
\pgfpathlineto{\pgfqpoint{1.002366in}{1.225961in}}%
\pgfpathlineto{\pgfqpoint{1.004546in}{1.224961in}}%
\pgfpathlineto{\pgfqpoint{1.006723in}{1.224173in}}%
\pgfpathlineto{\pgfqpoint{1.008897in}{1.223593in}}%
\pgfpathlineto{\pgfqpoint{0.997294in}{1.218022in}}%
\pgfpathlineto{\pgfqpoint{0.986014in}{1.212263in}}%
\pgfpathlineto{\pgfqpoint{0.975067in}{1.206320in}}%
\pgfpathlineto{\pgfqpoint{0.964464in}{1.200198in}}%
\pgfpathclose%
\pgfusepath{fill}%
\end{pgfscope}%
\begin{pgfscope}%
\pgfpathrectangle{\pgfqpoint{0.329460in}{0.284240in}}{\pgfqpoint{1.989680in}{1.989680in}}%
\pgfusepath{clip}%
\pgfsetbuttcap%
\pgfsetroundjoin%
\definecolor{currentfill}{rgb}{0.268510,0.009605,0.335427}%
\pgfsetfillcolor{currentfill}%
\pgfsetlinewidth{0.000000pt}%
\definecolor{currentstroke}{rgb}{0.000000,0.000000,0.000000}%
\pgfsetstrokecolor{currentstroke}%
\pgfsetdash{}{0pt}%
\pgfpathmoveto{\pgfqpoint{1.674508in}{1.227941in}}%
\pgfpathlineto{\pgfqpoint{1.676604in}{1.227760in}}%
\pgfpathlineto{\pgfqpoint{1.678701in}{1.227771in}}%
\pgfpathlineto{\pgfqpoint{1.680802in}{1.227977in}}%
\pgfpathlineto{\pgfqpoint{1.682904in}{1.228384in}}%
\pgfpathlineto{\pgfqpoint{1.694783in}{1.222984in}}%
\pgfpathlineto{\pgfqpoint{1.706351in}{1.217392in}}%
\pgfpathlineto{\pgfqpoint{1.717594in}{1.211612in}}%
\pgfpathlineto{\pgfqpoint{1.728504in}{1.205649in}}%
\pgfpathlineto{\pgfqpoint{1.726103in}{1.205385in}}%
\pgfpathlineto{\pgfqpoint{1.723706in}{1.205323in}}%
\pgfpathlineto{\pgfqpoint{1.721312in}{1.205456in}}%
\pgfpathlineto{\pgfqpoint{1.718920in}{1.205782in}}%
\pgfpathlineto{\pgfqpoint{1.708297in}{1.211593in}}%
\pgfpathlineto{\pgfqpoint{1.697346in}{1.217227in}}%
\pgfpathlineto{\pgfqpoint{1.686080in}{1.222677in}}%
\pgfpathlineto{\pgfqpoint{1.674508in}{1.227941in}}%
\pgfpathclose%
\pgfusepath{fill}%
\end{pgfscope}%
\begin{pgfscope}%
\pgfpathrectangle{\pgfqpoint{0.329460in}{0.284240in}}{\pgfqpoint{1.989680in}{1.989680in}}%
\pgfusepath{clip}%
\pgfsetbuttcap%
\pgfsetroundjoin%
\definecolor{currentfill}{rgb}{0.172719,0.448791,0.557885}%
\pgfsetfillcolor{currentfill}%
\pgfsetlinewidth{0.000000pt}%
\definecolor{currentstroke}{rgb}{0.000000,0.000000,0.000000}%
\pgfsetstrokecolor{currentstroke}%
\pgfsetdash{}{0pt}%
\pgfpathmoveto{\pgfqpoint{1.714913in}{1.536000in}}%
\pgfpathlineto{\pgfqpoint{1.716942in}{1.550333in}}%
\pgfpathlineto{\pgfqpoint{1.718981in}{1.565104in}}%
\pgfpathlineto{\pgfqpoint{1.721031in}{1.580320in}}%
\pgfpathlineto{\pgfqpoint{1.738224in}{1.574607in}}%
\pgfpathlineto{\pgfqpoint{1.755081in}{1.568630in}}%
\pgfpathlineto{\pgfqpoint{1.771587in}{1.562395in}}%
\pgfpathlineto{\pgfqpoint{1.787725in}{1.555906in}}%
\pgfpathlineto{\pgfqpoint{1.785296in}{1.540764in}}%
\pgfpathlineto{\pgfqpoint{1.782880in}{1.526068in}}%
\pgfpathlineto{\pgfqpoint{1.780476in}{1.511812in}}%
\pgfpathlineto{\pgfqpoint{1.764613in}{1.518241in}}%
\pgfpathlineto{\pgfqpoint{1.748388in}{1.524418in}}%
\pgfpathlineto{\pgfqpoint{1.731816in}{1.530339in}}%
\pgfpathlineto{\pgfqpoint{1.714913in}{1.536000in}}%
\pgfpathclose%
\pgfusepath{fill}%
\end{pgfscope}%
\begin{pgfscope}%
\pgfpathrectangle{\pgfqpoint{0.329460in}{0.284240in}}{\pgfqpoint{1.989680in}{1.989680in}}%
\pgfusepath{clip}%
\pgfsetbuttcap%
\pgfsetroundjoin%
\definecolor{currentfill}{rgb}{0.280255,0.165693,0.476498}%
\pgfsetfillcolor{currentfill}%
\pgfsetlinewidth{0.000000pt}%
\definecolor{currentstroke}{rgb}{0.000000,0.000000,0.000000}%
\pgfsetstrokecolor{currentstroke}%
\pgfsetdash{}{0pt}%
\pgfpathmoveto{\pgfqpoint{1.453823in}{1.325570in}}%
\pgfpathlineto{\pgfqpoint{1.454597in}{1.322244in}}%
\pgfpathlineto{\pgfqpoint{1.455370in}{1.319024in}}%
\pgfpathlineto{\pgfqpoint{1.456144in}{1.315914in}}%
\pgfpathlineto{\pgfqpoint{1.456917in}{1.312917in}}%
\pgfpathlineto{\pgfqpoint{1.469737in}{1.311148in}}%
\pgfpathlineto{\pgfqpoint{1.482450in}{1.309178in}}%
\pgfpathlineto{\pgfqpoint{1.495045in}{1.307006in}}%
\pgfpathlineto{\pgfqpoint{1.493992in}{1.310049in}}%
\pgfpathlineto{\pgfqpoint{1.492939in}{1.313206in}}%
\pgfpathlineto{\pgfqpoint{1.491885in}{1.316472in}}%
\pgfpathlineto{\pgfqpoint{1.490831in}{1.319845in}}%
\pgfpathlineto{\pgfqpoint{1.478606in}{1.321948in}}%
\pgfpathlineto{\pgfqpoint{1.466266in}{1.323857in}}%
\pgfpathlineto{\pgfqpoint{1.453823in}{1.325570in}}%
\pgfpathclose%
\pgfusepath{fill}%
\end{pgfscope}%
\begin{pgfscope}%
\pgfpathrectangle{\pgfqpoint{0.329460in}{0.284240in}}{\pgfqpoint{1.989680in}{1.989680in}}%
\pgfusepath{clip}%
\pgfsetbuttcap%
\pgfsetroundjoin%
\definecolor{currentfill}{rgb}{0.282327,0.094955,0.417331}%
\pgfsetfillcolor{currentfill}%
\pgfsetlinewidth{0.000000pt}%
\definecolor{currentstroke}{rgb}{0.000000,0.000000,0.000000}%
\pgfsetstrokecolor{currentstroke}%
\pgfsetdash{}{0pt}%
\pgfpathmoveto{\pgfqpoint{1.095055in}{1.267390in}}%
\pgfpathlineto{\pgfqpoint{1.093225in}{1.264746in}}%
\pgfpathlineto{\pgfqpoint{1.091395in}{1.262236in}}%
\pgfpathlineto{\pgfqpoint{1.089564in}{1.259863in}}%
\pgfpathlineto{\pgfqpoint{1.087733in}{1.257630in}}%
\pgfpathlineto{\pgfqpoint{1.099443in}{1.261727in}}%
\pgfpathlineto{\pgfqpoint{1.111381in}{1.265634in}}%
\pgfpathlineto{\pgfqpoint{1.123536in}{1.269347in}}%
\pgfpathlineto{\pgfqpoint{1.135897in}{1.272863in}}%
\pgfpathlineto{\pgfqpoint{1.137390in}{1.274985in}}%
\pgfpathlineto{\pgfqpoint{1.138882in}{1.277247in}}%
\pgfpathlineto{\pgfqpoint{1.140374in}{1.279646in}}%
\pgfpathlineto{\pgfqpoint{1.141866in}{1.282178in}}%
\pgfpathlineto{\pgfqpoint{1.129851in}{1.278764in}}%
\pgfpathlineto{\pgfqpoint{1.118037in}{1.275160in}}%
\pgfpathlineto{\pgfqpoint{1.106434in}{1.271367in}}%
\pgfpathlineto{\pgfqpoint{1.095055in}{1.267390in}}%
\pgfpathclose%
\pgfusepath{fill}%
\end{pgfscope}%
\begin{pgfscope}%
\pgfpathrectangle{\pgfqpoint{0.329460in}{0.284240in}}{\pgfqpoint{1.989680in}{1.989680in}}%
\pgfusepath{clip}%
\pgfsetbuttcap%
\pgfsetroundjoin%
\definecolor{currentfill}{rgb}{0.280255,0.165693,0.476498}%
\pgfsetfillcolor{currentfill}%
\pgfsetlinewidth{0.000000pt}%
\definecolor{currentstroke}{rgb}{0.000000,0.000000,0.000000}%
\pgfsetstrokecolor{currentstroke}%
\pgfsetdash{}{0pt}%
\pgfpathmoveto{\pgfqpoint{1.200784in}{1.317814in}}%
\pgfpathlineto{\pgfqpoint{1.199648in}{1.314424in}}%
\pgfpathlineto{\pgfqpoint{1.198513in}{1.311142in}}%
\pgfpathlineto{\pgfqpoint{1.197378in}{1.307969in}}%
\pgfpathlineto{\pgfqpoint{1.196243in}{1.304910in}}%
\pgfpathlineto{\pgfqpoint{1.208723in}{1.307258in}}%
\pgfpathlineto{\pgfqpoint{1.221332in}{1.309406in}}%
\pgfpathlineto{\pgfqpoint{1.234057in}{1.311355in}}%
\pgfpathlineto{\pgfqpoint{1.246888in}{1.313101in}}%
\pgfpathlineto{\pgfqpoint{1.247651in}{1.316096in}}%
\pgfpathlineto{\pgfqpoint{1.248414in}{1.319205in}}%
\pgfpathlineto{\pgfqpoint{1.249177in}{1.322424in}}%
\pgfpathlineto{\pgfqpoint{1.249941in}{1.325749in}}%
\pgfpathlineto{\pgfqpoint{1.237486in}{1.324057in}}%
\pgfpathlineto{\pgfqpoint{1.225135in}{1.322170in}}%
\pgfpathlineto{\pgfqpoint{1.212896in}{1.320088in}}%
\pgfpathlineto{\pgfqpoint{1.200784in}{1.317814in}}%
\pgfpathclose%
\pgfusepath{fill}%
\end{pgfscope}%
\begin{pgfscope}%
\pgfpathrectangle{\pgfqpoint{0.329460in}{0.284240in}}{\pgfqpoint{1.989680in}{1.989680in}}%
\pgfusepath{clip}%
\pgfsetbuttcap%
\pgfsetroundjoin%
\definecolor{currentfill}{rgb}{0.279566,0.067836,0.391917}%
\pgfsetfillcolor{currentfill}%
\pgfsetlinewidth{0.000000pt}%
\definecolor{currentstroke}{rgb}{0.000000,0.000000,0.000000}%
\pgfsetstrokecolor{currentstroke}%
\pgfsetdash{}{0pt}%
\pgfpathmoveto{\pgfqpoint{1.604245in}{1.261282in}}%
\pgfpathlineto{\pgfqpoint{1.606003in}{1.259219in}}%
\pgfpathlineto{\pgfqpoint{1.607761in}{1.257304in}}%
\pgfpathlineto{\pgfqpoint{1.609521in}{1.255540in}}%
\pgfpathlineto{\pgfqpoint{1.611281in}{1.253929in}}%
\pgfpathlineto{\pgfqpoint{1.623294in}{1.249692in}}%
\pgfpathlineto{\pgfqpoint{1.635061in}{1.245261in}}%
\pgfpathlineto{\pgfqpoint{1.646570in}{1.240642in}}%
\pgfpathlineto{\pgfqpoint{1.657810in}{1.235838in}}%
\pgfpathlineto{\pgfqpoint{1.655729in}{1.237576in}}%
\pgfpathlineto{\pgfqpoint{1.653649in}{1.239468in}}%
\pgfpathlineto{\pgfqpoint{1.651570in}{1.241511in}}%
\pgfpathlineto{\pgfqpoint{1.649492in}{1.243702in}}%
\pgfpathlineto{\pgfqpoint{1.638563in}{1.248370in}}%
\pgfpathlineto{\pgfqpoint{1.627371in}{1.252858in}}%
\pgfpathlineto{\pgfqpoint{1.615928in}{1.257163in}}%
\pgfpathlineto{\pgfqpoint{1.604245in}{1.261282in}}%
\pgfpathclose%
\pgfusepath{fill}%
\end{pgfscope}%
\begin{pgfscope}%
\pgfpathrectangle{\pgfqpoint{0.329460in}{0.284240in}}{\pgfqpoint{1.989680in}{1.989680in}}%
\pgfusepath{clip}%
\pgfsetbuttcap%
\pgfsetroundjoin%
\definecolor{currentfill}{rgb}{0.274128,0.199721,0.498911}%
\pgfsetfillcolor{currentfill}%
\pgfsetlinewidth{0.000000pt}%
\definecolor{currentstroke}{rgb}{0.000000,0.000000,0.000000}%
\pgfsetstrokecolor{currentstroke}%
\pgfsetdash{}{0pt}%
\pgfpathmoveto{\pgfqpoint{1.302085in}{1.344682in}}%
\pgfpathlineto{\pgfqpoint{1.301703in}{1.341000in}}%
\pgfpathlineto{\pgfqpoint{1.301321in}{1.337413in}}%
\pgfpathlineto{\pgfqpoint{1.300939in}{1.333923in}}%
\pgfpathlineto{\pgfqpoint{1.300558in}{1.330533in}}%
\pgfpathlineto{\pgfqpoint{1.313354in}{1.331228in}}%
\pgfpathlineto{\pgfqpoint{1.326185in}{1.331722in}}%
\pgfpathlineto{\pgfqpoint{1.339038in}{1.332014in}}%
\pgfpathlineto{\pgfqpoint{1.351902in}{1.332104in}}%
\pgfpathlineto{\pgfqpoint{1.351897in}{1.335481in}}%
\pgfpathlineto{\pgfqpoint{1.351892in}{1.338959in}}%
\pgfpathlineto{\pgfqpoint{1.351886in}{1.342534in}}%
\pgfpathlineto{\pgfqpoint{1.351881in}{1.346203in}}%
\pgfpathlineto{\pgfqpoint{1.339405in}{1.346116in}}%
\pgfpathlineto{\pgfqpoint{1.326939in}{1.345833in}}%
\pgfpathlineto{\pgfqpoint{1.314495in}{1.345355in}}%
\pgfpathlineto{\pgfqpoint{1.302085in}{1.344682in}}%
\pgfpathclose%
\pgfusepath{fill}%
\end{pgfscope}%
\begin{pgfscope}%
\pgfpathrectangle{\pgfqpoint{0.329460in}{0.284240in}}{\pgfqpoint{1.989680in}{1.989680in}}%
\pgfusepath{clip}%
\pgfsetbuttcap%
\pgfsetroundjoin%
\definecolor{currentfill}{rgb}{0.274128,0.199721,0.498911}%
\pgfsetfillcolor{currentfill}%
\pgfsetlinewidth{0.000000pt}%
\definecolor{currentstroke}{rgb}{0.000000,0.000000,0.000000}%
\pgfsetstrokecolor{currentstroke}%
\pgfsetdash{}{0pt}%
\pgfpathmoveto{\pgfqpoint{1.351881in}{1.346203in}}%
\pgfpathlineto{\pgfqpoint{1.351886in}{1.342534in}}%
\pgfpathlineto{\pgfqpoint{1.351892in}{1.338959in}}%
\pgfpathlineto{\pgfqpoint{1.351897in}{1.335481in}}%
\pgfpathlineto{\pgfqpoint{1.351902in}{1.332104in}}%
\pgfpathlineto{\pgfqpoint{1.364766in}{1.331992in}}%
\pgfpathlineto{\pgfqpoint{1.377617in}{1.331677in}}%
\pgfpathlineto{\pgfqpoint{1.390445in}{1.331161in}}%
\pgfpathlineto{\pgfqpoint{1.403237in}{1.330443in}}%
\pgfpathlineto{\pgfqpoint{1.402845in}{1.333834in}}%
\pgfpathlineto{\pgfqpoint{1.402452in}{1.337325in}}%
\pgfpathlineto{\pgfqpoint{1.402059in}{1.340913in}}%
\pgfpathlineto{\pgfqpoint{1.401667in}{1.344595in}}%
\pgfpathlineto{\pgfqpoint{1.389260in}{1.345290in}}%
\pgfpathlineto{\pgfqpoint{1.376820in}{1.345790in}}%
\pgfpathlineto{\pgfqpoint{1.364356in}{1.346094in}}%
\pgfpathlineto{\pgfqpoint{1.351881in}{1.346203in}}%
\pgfpathclose%
\pgfusepath{fill}%
\end{pgfscope}%
\begin{pgfscope}%
\pgfpathrectangle{\pgfqpoint{0.329460in}{0.284240in}}{\pgfqpoint{1.989680in}{1.989680in}}%
\pgfusepath{clip}%
\pgfsetbuttcap%
\pgfsetroundjoin%
\definecolor{currentfill}{rgb}{0.260571,0.246922,0.522828}%
\pgfsetfillcolor{currentfill}%
\pgfsetlinewidth{0.000000pt}%
\definecolor{currentstroke}{rgb}{0.000000,0.000000,0.000000}%
\pgfsetstrokecolor{currentstroke}%
\pgfsetdash{}{0pt}%
\pgfpathmoveto{\pgfqpoint{0.882921in}{1.339252in}}%
\pgfpathlineto{\pgfqpoint{0.880246in}{1.348609in}}%
\pgfpathlineto{\pgfqpoint{0.877560in}{1.358331in}}%
\pgfpathlineto{\pgfqpoint{0.874863in}{1.368422in}}%
\pgfpathlineto{\pgfqpoint{0.872155in}{1.378890in}}%
\pgfpathlineto{\pgfqpoint{0.885397in}{1.386257in}}%
\pgfpathlineto{\pgfqpoint{0.899055in}{1.393405in}}%
\pgfpathlineto{\pgfqpoint{0.913117in}{1.400329in}}%
\pgfpathlineto{\pgfqpoint{0.927568in}{1.407024in}}%
\pgfpathlineto{\pgfqpoint{0.929952in}{1.396446in}}%
\pgfpathlineto{\pgfqpoint{0.932326in}{1.386242in}}%
\pgfpathlineto{\pgfqpoint{0.934691in}{1.376407in}}%
\pgfpathlineto{\pgfqpoint{0.937046in}{1.366934in}}%
\pgfpathlineto{\pgfqpoint{0.922928in}{1.360346in}}%
\pgfpathlineto{\pgfqpoint{0.909193in}{1.353533in}}%
\pgfpathlineto{\pgfqpoint{0.895853in}{1.346499in}}%
\pgfpathlineto{\pgfqpoint{0.882921in}{1.339252in}}%
\pgfpathclose%
\pgfusepath{fill}%
\end{pgfscope}%
\begin{pgfscope}%
\pgfpathrectangle{\pgfqpoint{0.329460in}{0.284240in}}{\pgfqpoint{1.989680in}{1.989680in}}%
\pgfusepath{clip}%
\pgfsetbuttcap%
\pgfsetroundjoin%
\definecolor{currentfill}{rgb}{0.268510,0.009605,0.335427}%
\pgfsetfillcolor{currentfill}%
\pgfsetlinewidth{0.000000pt}%
\definecolor{currentstroke}{rgb}{0.000000,0.000000,0.000000}%
\pgfsetstrokecolor{currentstroke}%
\pgfsetdash{}{0pt}%
\pgfpathmoveto{\pgfqpoint{0.974295in}{1.200470in}}%
\pgfpathlineto{\pgfqpoint{0.971841in}{1.200110in}}%
\pgfpathlineto{\pgfqpoint{0.969385in}{1.199942in}}%
\pgfpathlineto{\pgfqpoint{0.966926in}{1.199970in}}%
\pgfpathlineto{\pgfqpoint{0.964464in}{1.200198in}}%
\pgfpathlineto{\pgfqpoint{0.975067in}{1.206320in}}%
\pgfpathlineto{\pgfqpoint{0.986014in}{1.212263in}}%
\pgfpathlineto{\pgfqpoint{0.997294in}{1.218022in}}%
\pgfpathlineto{\pgfqpoint{1.008897in}{1.223593in}}%
\pgfpathlineto{\pgfqpoint{1.011068in}{1.223217in}}%
\pgfpathlineto{\pgfqpoint{1.013236in}{1.223041in}}%
\pgfpathlineto{\pgfqpoint{1.015402in}{1.223060in}}%
\pgfpathlineto{\pgfqpoint{1.017566in}{1.223272in}}%
\pgfpathlineto{\pgfqpoint{1.006265in}{1.217841in}}%
\pgfpathlineto{\pgfqpoint{0.995279in}{1.212228in}}%
\pgfpathlineto{\pgfqpoint{0.984619in}{1.206436in}}%
\pgfpathlineto{\pgfqpoint{0.974295in}{1.200470in}}%
\pgfpathclose%
\pgfusepath{fill}%
\end{pgfscope}%
\begin{pgfscope}%
\pgfpathrectangle{\pgfqpoint{0.329460in}{0.284240in}}{\pgfqpoint{1.989680in}{1.989680in}}%
\pgfusepath{clip}%
\pgfsetbuttcap%
\pgfsetroundjoin%
\definecolor{currentfill}{rgb}{0.172719,0.448791,0.557885}%
\pgfsetfillcolor{currentfill}%
\pgfsetlinewidth{0.000000pt}%
\definecolor{currentstroke}{rgb}{0.000000,0.000000,0.000000}%
\pgfsetstrokecolor{currentstroke}%
\pgfsetdash{}{0pt}%
\pgfpathmoveto{\pgfqpoint{0.908115in}{1.505891in}}%
\pgfpathlineto{\pgfqpoint{0.905631in}{1.520128in}}%
\pgfpathlineto{\pgfqpoint{0.903135in}{1.534805in}}%
\pgfpathlineto{\pgfqpoint{0.900625in}{1.549929in}}%
\pgfpathlineto{\pgfqpoint{0.916424in}{1.556640in}}%
\pgfpathlineto{\pgfqpoint{0.932604in}{1.563100in}}%
\pgfpathlineto{\pgfqpoint{0.949150in}{1.569307in}}%
\pgfpathlineto{\pgfqpoint{0.966045in}{1.575254in}}%
\pgfpathlineto{\pgfqpoint{0.968181in}{1.560054in}}%
\pgfpathlineto{\pgfqpoint{0.970306in}{1.545298in}}%
\pgfpathlineto{\pgfqpoint{0.972421in}{1.530981in}}%
\pgfpathlineto{\pgfqpoint{0.955812in}{1.525089in}}%
\pgfpathlineto{\pgfqpoint{0.939547in}{1.518940in}}%
\pgfpathlineto{\pgfqpoint{0.923643in}{1.512539in}}%
\pgfpathlineto{\pgfqpoint{0.908115in}{1.505891in}}%
\pgfpathclose%
\pgfusepath{fill}%
\end{pgfscope}%
\begin{pgfscope}%
\pgfpathrectangle{\pgfqpoint{0.329460in}{0.284240in}}{\pgfqpoint{1.989680in}{1.989680in}}%
\pgfusepath{clip}%
\pgfsetbuttcap%
\pgfsetroundjoin%
\definecolor{currentfill}{rgb}{0.271305,0.019942,0.347269}%
\pgfsetfillcolor{currentfill}%
\pgfsetlinewidth{0.000000pt}%
\definecolor{currentstroke}{rgb}{0.000000,0.000000,0.000000}%
\pgfsetstrokecolor{currentstroke}%
\pgfsetdash{}{0pt}%
\pgfpathmoveto{\pgfqpoint{1.666146in}{1.230506in}}%
\pgfpathlineto{\pgfqpoint{1.668234in}{1.229596in}}%
\pgfpathlineto{\pgfqpoint{1.670324in}{1.228863in}}%
\pgfpathlineto{\pgfqpoint{1.672415in}{1.228310in}}%
\pgfpathlineto{\pgfqpoint{1.674508in}{1.227941in}}%
\pgfpathlineto{\pgfqpoint{1.686080in}{1.222677in}}%
\pgfpathlineto{\pgfqpoint{1.697346in}{1.217227in}}%
\pgfpathlineto{\pgfqpoint{1.708297in}{1.211593in}}%
\pgfpathlineto{\pgfqpoint{1.718920in}{1.205782in}}%
\pgfpathlineto{\pgfqpoint{1.716531in}{1.206296in}}%
\pgfpathlineto{\pgfqpoint{1.714144in}{1.206995in}}%
\pgfpathlineto{\pgfqpoint{1.711760in}{1.207874in}}%
\pgfpathlineto{\pgfqpoint{1.709377in}{1.208931in}}%
\pgfpathlineto{\pgfqpoint{1.699037in}{1.214589in}}%
\pgfpathlineto{\pgfqpoint{1.688379in}{1.220074in}}%
\pgfpathlineto{\pgfqpoint{1.677411in}{1.225381in}}%
\pgfpathlineto{\pgfqpoint{1.666146in}{1.230506in}}%
\pgfpathclose%
\pgfusepath{fill}%
\end{pgfscope}%
\begin{pgfscope}%
\pgfpathrectangle{\pgfqpoint{0.329460in}{0.284240in}}{\pgfqpoint{1.989680in}{1.989680in}}%
\pgfusepath{clip}%
\pgfsetbuttcap%
\pgfsetroundjoin%
\definecolor{currentfill}{rgb}{0.233603,0.313828,0.543914}%
\pgfsetfillcolor{currentfill}%
\pgfsetlinewidth{0.000000pt}%
\definecolor{currentstroke}{rgb}{0.000000,0.000000,0.000000}%
\pgfsetstrokecolor{currentstroke}%
\pgfsetdash{}{0pt}%
\pgfpathmoveto{\pgfqpoint{1.761646in}{1.412779in}}%
\pgfpathlineto{\pgfqpoint{1.763963in}{1.423760in}}%
\pgfpathlineto{\pgfqpoint{1.766290in}{1.435128in}}%
\pgfpathlineto{\pgfqpoint{1.768627in}{1.446889in}}%
\pgfpathlineto{\pgfqpoint{1.770975in}{1.459049in}}%
\pgfpathlineto{\pgfqpoint{1.786107in}{1.452464in}}%
\pgfpathlineto{\pgfqpoint{1.800855in}{1.445642in}}%
\pgfpathlineto{\pgfqpoint{1.815204in}{1.438588in}}%
\pgfpathlineto{\pgfqpoint{1.829140in}{1.431306in}}%
\pgfpathlineto{\pgfqpoint{1.826455in}{1.419247in}}%
\pgfpathlineto{\pgfqpoint{1.823782in}{1.407588in}}%
\pgfpathlineto{\pgfqpoint{1.821121in}{1.396324in}}%
\pgfpathlineto{\pgfqpoint{1.818471in}{1.385449in}}%
\pgfpathlineto{\pgfqpoint{1.804858in}{1.392622in}}%
\pgfpathlineto{\pgfqpoint{1.790840in}{1.399571in}}%
\pgfpathlineto{\pgfqpoint{1.776432in}{1.406292in}}%
\pgfpathlineto{\pgfqpoint{1.761646in}{1.412779in}}%
\pgfpathclose%
\pgfusepath{fill}%
\end{pgfscope}%
\begin{pgfscope}%
\pgfpathrectangle{\pgfqpoint{0.329460in}{0.284240in}}{\pgfqpoint{1.989680in}{1.989680in}}%
\pgfusepath{clip}%
\pgfsetbuttcap%
\pgfsetroundjoin%
\definecolor{currentfill}{rgb}{0.279566,0.067836,0.391917}%
\pgfsetfillcolor{currentfill}%
\pgfsetlinewidth{0.000000pt}%
\definecolor{currentstroke}{rgb}{0.000000,0.000000,0.000000}%
\pgfsetstrokecolor{currentstroke}%
\pgfsetdash{}{0pt}%
\pgfpathmoveto{\pgfqpoint{1.043398in}{1.239405in}}%
\pgfpathlineto{\pgfqpoint{1.041252in}{1.237183in}}%
\pgfpathlineto{\pgfqpoint{1.039105in}{1.235108in}}%
\pgfpathlineto{\pgfqpoint{1.036957in}{1.233184in}}%
\pgfpathlineto{\pgfqpoint{1.034809in}{1.231416in}}%
\pgfpathlineto{\pgfqpoint{1.045800in}{1.236381in}}%
\pgfpathlineto{\pgfqpoint{1.057071in}{1.241165in}}%
\pgfpathlineto{\pgfqpoint{1.068609in}{1.245763in}}%
\pgfpathlineto{\pgfqpoint{1.080404in}{1.250172in}}%
\pgfpathlineto{\pgfqpoint{1.082237in}{1.251809in}}%
\pgfpathlineto{\pgfqpoint{1.084070in}{1.253599in}}%
\pgfpathlineto{\pgfqpoint{1.085902in}{1.255541in}}%
\pgfpathlineto{\pgfqpoint{1.087733in}{1.257630in}}%
\pgfpathlineto{\pgfqpoint{1.076263in}{1.253346in}}%
\pgfpathlineto{\pgfqpoint{1.065043in}{1.248877in}}%
\pgfpathlineto{\pgfqpoint{1.054085in}{1.244229in}}%
\pgfpathlineto{\pgfqpoint{1.043398in}{1.239405in}}%
\pgfpathclose%
\pgfusepath{fill}%
\end{pgfscope}%
\begin{pgfscope}%
\pgfpathrectangle{\pgfqpoint{0.329460in}{0.284240in}}{\pgfqpoint{1.989680in}{1.989680in}}%
\pgfusepath{clip}%
\pgfsetbuttcap%
\pgfsetroundjoin%
\definecolor{currentfill}{rgb}{0.274128,0.199721,0.498911}%
\pgfsetfillcolor{currentfill}%
\pgfsetlinewidth{0.000000pt}%
\definecolor{currentstroke}{rgb}{0.000000,0.000000,0.000000}%
\pgfsetstrokecolor{currentstroke}%
\pgfsetdash{}{0pt}%
\pgfpathmoveto{\pgfqpoint{1.252997in}{1.340053in}}%
\pgfpathlineto{\pgfqpoint{1.252233in}{1.336333in}}%
\pgfpathlineto{\pgfqpoint{1.251469in}{1.332706in}}%
\pgfpathlineto{\pgfqpoint{1.250705in}{1.329177in}}%
\pgfpathlineto{\pgfqpoint{1.249941in}{1.325749in}}%
\pgfpathlineto{\pgfqpoint{1.262487in}{1.327243in}}%
\pgfpathlineto{\pgfqpoint{1.275112in}{1.328539in}}%
\pgfpathlineto{\pgfqpoint{1.287806in}{1.329636in}}%
\pgfpathlineto{\pgfqpoint{1.300558in}{1.330533in}}%
\pgfpathlineto{\pgfqpoint{1.300939in}{1.333923in}}%
\pgfpathlineto{\pgfqpoint{1.301321in}{1.337413in}}%
\pgfpathlineto{\pgfqpoint{1.301703in}{1.341000in}}%
\pgfpathlineto{\pgfqpoint{1.302085in}{1.344682in}}%
\pgfpathlineto{\pgfqpoint{1.289719in}{1.343815in}}%
\pgfpathlineto{\pgfqpoint{1.277408in}{1.342753in}}%
\pgfpathlineto{\pgfqpoint{1.265164in}{1.341499in}}%
\pgfpathlineto{\pgfqpoint{1.252997in}{1.340053in}}%
\pgfpathclose%
\pgfusepath{fill}%
\end{pgfscope}%
\begin{pgfscope}%
\pgfpathrectangle{\pgfqpoint{0.329460in}{0.284240in}}{\pgfqpoint{1.989680in}{1.989680in}}%
\pgfusepath{clip}%
\pgfsetbuttcap%
\pgfsetroundjoin%
\definecolor{currentfill}{rgb}{0.274128,0.199721,0.498911}%
\pgfsetfillcolor{currentfill}%
\pgfsetlinewidth{0.000000pt}%
\definecolor{currentstroke}{rgb}{0.000000,0.000000,0.000000}%
\pgfsetstrokecolor{currentstroke}%
\pgfsetdash{}{0pt}%
\pgfpathmoveto{\pgfqpoint{1.401667in}{1.344595in}}%
\pgfpathlineto{\pgfqpoint{1.402059in}{1.340913in}}%
\pgfpathlineto{\pgfqpoint{1.402452in}{1.337325in}}%
\pgfpathlineto{\pgfqpoint{1.402845in}{1.333834in}}%
\pgfpathlineto{\pgfqpoint{1.403237in}{1.330443in}}%
\pgfpathlineto{\pgfqpoint{1.415982in}{1.329524in}}%
\pgfpathlineto{\pgfqpoint{1.428669in}{1.328405in}}%
\pgfpathlineto{\pgfqpoint{1.441287in}{1.327087in}}%
\pgfpathlineto{\pgfqpoint{1.453823in}{1.325570in}}%
\pgfpathlineto{\pgfqpoint{1.453049in}{1.329001in}}%
\pgfpathlineto{\pgfqpoint{1.452274in}{1.332531in}}%
\pgfpathlineto{\pgfqpoint{1.451499in}{1.336159in}}%
\pgfpathlineto{\pgfqpoint{1.450724in}{1.339881in}}%
\pgfpathlineto{\pgfqpoint{1.438567in}{1.341348in}}%
\pgfpathlineto{\pgfqpoint{1.426331in}{1.342623in}}%
\pgfpathlineto{\pgfqpoint{1.414027in}{1.343706in}}%
\pgfpathlineto{\pgfqpoint{1.401667in}{1.344595in}}%
\pgfpathclose%
\pgfusepath{fill}%
\end{pgfscope}%
\begin{pgfscope}%
\pgfpathrectangle{\pgfqpoint{0.329460in}{0.284240in}}{\pgfqpoint{1.989680in}{1.989680in}}%
\pgfusepath{clip}%
\pgfsetbuttcap%
\pgfsetroundjoin%
\definecolor{currentfill}{rgb}{0.280255,0.165693,0.476498}%
\pgfsetfillcolor{currentfill}%
\pgfsetlinewidth{0.000000pt}%
\definecolor{currentstroke}{rgb}{0.000000,0.000000,0.000000}%
\pgfsetstrokecolor{currentstroke}%
\pgfsetdash{}{0pt}%
\pgfpathmoveto{\pgfqpoint{1.490831in}{1.319845in}}%
\pgfpathlineto{\pgfqpoint{1.491885in}{1.316472in}}%
\pgfpathlineto{\pgfqpoint{1.492939in}{1.313206in}}%
\pgfpathlineto{\pgfqpoint{1.493992in}{1.310049in}}%
\pgfpathlineto{\pgfqpoint{1.495045in}{1.307006in}}%
\pgfpathlineto{\pgfqpoint{1.507511in}{1.304637in}}%
\pgfpathlineto{\pgfqpoint{1.519835in}{1.302070in}}%
\pgfpathlineto{\pgfqpoint{1.532007in}{1.299309in}}%
\pgfpathlineto{\pgfqpoint{1.544015in}{1.296356in}}%
\pgfpathlineto{\pgfqpoint{1.542601in}{1.299482in}}%
\pgfpathlineto{\pgfqpoint{1.541186in}{1.302722in}}%
\pgfpathlineto{\pgfqpoint{1.539771in}{1.306072in}}%
\pgfpathlineto{\pgfqpoint{1.538356in}{1.309528in}}%
\pgfpathlineto{\pgfqpoint{1.526703in}{1.312389in}}%
\pgfpathlineto{\pgfqpoint{1.514890in}{1.315063in}}%
\pgfpathlineto{\pgfqpoint{1.502929in}{1.317549in}}%
\pgfpathlineto{\pgfqpoint{1.490831in}{1.319845in}}%
\pgfpathclose%
\pgfusepath{fill}%
\end{pgfscope}%
\begin{pgfscope}%
\pgfpathrectangle{\pgfqpoint{0.329460in}{0.284240in}}{\pgfqpoint{1.989680in}{1.989680in}}%
\pgfusepath{clip}%
\pgfsetbuttcap%
\pgfsetroundjoin%
\definecolor{currentfill}{rgb}{0.283072,0.130895,0.449241}%
\pgfsetfillcolor{currentfill}%
\pgfsetlinewidth{0.000000pt}%
\definecolor{currentstroke}{rgb}{0.000000,0.000000,0.000000}%
\pgfsetstrokecolor{currentstroke}%
\pgfsetdash{}{0pt}%
\pgfpathmoveto{\pgfqpoint{1.544015in}{1.296356in}}%
\pgfpathlineto{\pgfqpoint{1.545429in}{1.293346in}}%
\pgfpathlineto{\pgfqpoint{1.546843in}{1.290456in}}%
\pgfpathlineto{\pgfqpoint{1.548256in}{1.287690in}}%
\pgfpathlineto{\pgfqpoint{1.549670in}{1.285050in}}%
\pgfpathlineto{\pgfqpoint{1.561854in}{1.281808in}}%
\pgfpathlineto{\pgfqpoint{1.573847in}{1.278373in}}%
\pgfpathlineto{\pgfqpoint{1.585638in}{1.274748in}}%
\pgfpathlineto{\pgfqpoint{1.597216in}{1.270934in}}%
\pgfpathlineto{\pgfqpoint{1.595460in}{1.273681in}}%
\pgfpathlineto{\pgfqpoint{1.593704in}{1.276555in}}%
\pgfpathlineto{\pgfqpoint{1.591947in}{1.279552in}}%
\pgfpathlineto{\pgfqpoint{1.590191in}{1.282669in}}%
\pgfpathlineto{\pgfqpoint{1.578947in}{1.286366in}}%
\pgfpathlineto{\pgfqpoint{1.567496in}{1.289882in}}%
\pgfpathlineto{\pgfqpoint{1.555848in}{1.293212in}}%
\pgfpathlineto{\pgfqpoint{1.544015in}{1.296356in}}%
\pgfpathclose%
\pgfusepath{fill}%
\end{pgfscope}%
\begin{pgfscope}%
\pgfpathrectangle{\pgfqpoint{0.329460in}{0.284240in}}{\pgfqpoint{1.989680in}{1.989680in}}%
\pgfusepath{clip}%
\pgfsetbuttcap%
\pgfsetroundjoin%
\definecolor{currentfill}{rgb}{0.272594,0.025563,0.353093}%
\pgfsetfillcolor{currentfill}%
\pgfsetlinewidth{0.000000pt}%
\definecolor{currentstroke}{rgb}{0.000000,0.000000,0.000000}%
\pgfsetstrokecolor{currentstroke}%
\pgfsetdash{}{0pt}%
\pgfpathmoveto{\pgfqpoint{1.757599in}{1.225916in}}%
\pgfpathlineto{\pgfqpoint{1.760053in}{1.229160in}}%
\pgfpathlineto{\pgfqpoint{1.762513in}{1.232664in}}%
\pgfpathlineto{\pgfqpoint{1.764979in}{1.236433in}}%
\pgfpathlineto{\pgfqpoint{1.767450in}{1.240470in}}%
\pgfpathlineto{\pgfqpoint{1.779147in}{1.233732in}}%
\pgfpathlineto{\pgfqpoint{1.790455in}{1.226802in}}%
\pgfpathlineto{\pgfqpoint{1.801361in}{1.219688in}}%
\pgfpathlineto{\pgfqpoint{1.811855in}{1.212393in}}%
\pgfpathlineto{\pgfqpoint{1.809107in}{1.208505in}}%
\pgfpathlineto{\pgfqpoint{1.806367in}{1.204887in}}%
\pgfpathlineto{\pgfqpoint{1.803633in}{1.201535in}}%
\pgfpathlineto{\pgfqpoint{1.800905in}{1.198443in}}%
\pgfpathlineto{\pgfqpoint{1.790673in}{1.205580in}}%
\pgfpathlineto{\pgfqpoint{1.780037in}{1.212542in}}%
\pgfpathlineto{\pgfqpoint{1.769008in}{1.219323in}}%
\pgfpathlineto{\pgfqpoint{1.757599in}{1.225916in}}%
\pgfpathclose%
\pgfusepath{fill}%
\end{pgfscope}%
\begin{pgfscope}%
\pgfpathrectangle{\pgfqpoint{0.329460in}{0.284240in}}{\pgfqpoint{1.989680in}{1.989680in}}%
\pgfusepath{clip}%
\pgfsetbuttcap%
\pgfsetroundjoin%
\definecolor{currentfill}{rgb}{0.277941,0.056324,0.381191}%
\pgfsetfillcolor{currentfill}%
\pgfsetlinewidth{0.000000pt}%
\definecolor{currentstroke}{rgb}{0.000000,0.000000,0.000000}%
\pgfsetstrokecolor{currentstroke}%
\pgfsetdash{}{0pt}%
\pgfpathmoveto{\pgfqpoint{1.767450in}{1.240470in}}%
\pgfpathlineto{\pgfqpoint{1.769928in}{1.244781in}}%
\pgfpathlineto{\pgfqpoint{1.772412in}{1.249371in}}%
\pgfpathlineto{\pgfqpoint{1.774903in}{1.254245in}}%
\pgfpathlineto{\pgfqpoint{1.777400in}{1.259407in}}%
\pgfpathlineto{\pgfqpoint{1.789388in}{1.252528in}}%
\pgfpathlineto{\pgfqpoint{1.800978in}{1.245454in}}%
\pgfpathlineto{\pgfqpoint{1.812158in}{1.238190in}}%
\pgfpathlineto{\pgfqpoint{1.822916in}{1.230743in}}%
\pgfpathlineto{\pgfqpoint{1.820139in}{1.225726in}}%
\pgfpathlineto{\pgfqpoint{1.817371in}{1.220999in}}%
\pgfpathlineto{\pgfqpoint{1.814609in}{1.216556in}}%
\pgfpathlineto{\pgfqpoint{1.811855in}{1.212393in}}%
\pgfpathlineto{\pgfqpoint{1.801361in}{1.219688in}}%
\pgfpathlineto{\pgfqpoint{1.790455in}{1.226802in}}%
\pgfpathlineto{\pgfqpoint{1.779147in}{1.233732in}}%
\pgfpathlineto{\pgfqpoint{1.767450in}{1.240470in}}%
\pgfpathclose%
\pgfusepath{fill}%
\end{pgfscope}%
\begin{pgfscope}%
\pgfpathrectangle{\pgfqpoint{0.329460in}{0.284240in}}{\pgfqpoint{1.989680in}{1.989680in}}%
\pgfusepath{clip}%
\pgfsetbuttcap%
\pgfsetroundjoin%
\definecolor{currentfill}{rgb}{0.280255,0.165693,0.476498}%
\pgfsetfillcolor{currentfill}%
\pgfsetlinewidth{0.000000pt}%
\definecolor{currentstroke}{rgb}{0.000000,0.000000,0.000000}%
\pgfsetstrokecolor{currentstroke}%
\pgfsetdash{}{0pt}%
\pgfpathmoveto{\pgfqpoint{1.153803in}{1.306831in}}%
\pgfpathlineto{\pgfqpoint{1.152310in}{1.303352in}}%
\pgfpathlineto{\pgfqpoint{1.150817in}{1.299981in}}%
\pgfpathlineto{\pgfqpoint{1.149325in}{1.296719in}}%
\pgfpathlineto{\pgfqpoint{1.147833in}{1.293571in}}%
\pgfpathlineto{\pgfqpoint{1.159686in}{1.296693in}}%
\pgfpathlineto{\pgfqpoint{1.171713in}{1.299626in}}%
\pgfpathlineto{\pgfqpoint{1.183902in}{1.302365in}}%
\pgfpathlineto{\pgfqpoint{1.196243in}{1.304910in}}%
\pgfpathlineto{\pgfqpoint{1.197378in}{1.307969in}}%
\pgfpathlineto{\pgfqpoint{1.198513in}{1.311142in}}%
\pgfpathlineto{\pgfqpoint{1.199648in}{1.314424in}}%
\pgfpathlineto{\pgfqpoint{1.200784in}{1.317814in}}%
\pgfpathlineto{\pgfqpoint{1.188807in}{1.315349in}}%
\pgfpathlineto{\pgfqpoint{1.176977in}{1.312695in}}%
\pgfpathlineto{\pgfqpoint{1.165305in}{1.309855in}}%
\pgfpathlineto{\pgfqpoint{1.153803in}{1.306831in}}%
\pgfpathclose%
\pgfusepath{fill}%
\end{pgfscope}%
\begin{pgfscope}%
\pgfpathrectangle{\pgfqpoint{0.329460in}{0.284240in}}{\pgfqpoint{1.989680in}{1.989680in}}%
\pgfusepath{clip}%
\pgfsetbuttcap%
\pgfsetroundjoin%
\definecolor{currentfill}{rgb}{0.271305,0.019942,0.347269}%
\pgfsetfillcolor{currentfill}%
\pgfsetlinewidth{0.000000pt}%
\definecolor{currentstroke}{rgb}{0.000000,0.000000,0.000000}%
\pgfsetstrokecolor{currentstroke}%
\pgfsetdash{}{0pt}%
\pgfpathmoveto{\pgfqpoint{0.984084in}{1.203760in}}%
\pgfpathlineto{\pgfqpoint{0.981640in}{1.202668in}}%
\pgfpathlineto{\pgfqpoint{0.979194in}{1.201753in}}%
\pgfpathlineto{\pgfqpoint{0.976746in}{1.201019in}}%
\pgfpathlineto{\pgfqpoint{0.974295in}{1.200470in}}%
\pgfpathlineto{\pgfqpoint{0.984619in}{1.206436in}}%
\pgfpathlineto{\pgfqpoint{0.995279in}{1.212228in}}%
\pgfpathlineto{\pgfqpoint{1.006265in}{1.217841in}}%
\pgfpathlineto{\pgfqpoint{1.017566in}{1.223272in}}%
\pgfpathlineto{\pgfqpoint{1.019728in}{1.223671in}}%
\pgfpathlineto{\pgfqpoint{1.021887in}{1.224255in}}%
\pgfpathlineto{\pgfqpoint{1.024045in}{1.225019in}}%
\pgfpathlineto{\pgfqpoint{1.026201in}{1.225960in}}%
\pgfpathlineto{\pgfqpoint{1.015200in}{1.220673in}}%
\pgfpathlineto{\pgfqpoint{1.004506in}{1.215207in}}%
\pgfpathlineto{\pgfqpoint{0.994131in}{1.209568in}}%
\pgfpathlineto{\pgfqpoint{0.984084in}{1.203760in}}%
\pgfpathclose%
\pgfusepath{fill}%
\end{pgfscope}%
\begin{pgfscope}%
\pgfpathrectangle{\pgfqpoint{0.329460in}{0.284240in}}{\pgfqpoint{1.989680in}{1.989680in}}%
\pgfusepath{clip}%
\pgfsetbuttcap%
\pgfsetroundjoin%
\definecolor{currentfill}{rgb}{0.283072,0.130895,0.449241}%
\pgfsetfillcolor{currentfill}%
\pgfsetlinewidth{0.000000pt}%
\definecolor{currentstroke}{rgb}{0.000000,0.000000,0.000000}%
\pgfsetstrokecolor{currentstroke}%
\pgfsetdash{}{0pt}%
\pgfpathmoveto{\pgfqpoint{1.102373in}{1.279232in}}%
\pgfpathlineto{\pgfqpoint{1.100543in}{1.276088in}}%
\pgfpathlineto{\pgfqpoint{1.098714in}{1.273064in}}%
\pgfpathlineto{\pgfqpoint{1.096884in}{1.270164in}}%
\pgfpathlineto{\pgfqpoint{1.095055in}{1.267390in}}%
\pgfpathlineto{\pgfqpoint{1.106434in}{1.271367in}}%
\pgfpathlineto{\pgfqpoint{1.118037in}{1.275160in}}%
\pgfpathlineto{\pgfqpoint{1.129851in}{1.278764in}}%
\pgfpathlineto{\pgfqpoint{1.141866in}{1.282178in}}%
\pgfpathlineto{\pgfqpoint{1.143357in}{1.284840in}}%
\pgfpathlineto{\pgfqpoint{1.144849in}{1.287628in}}%
\pgfpathlineto{\pgfqpoint{1.146341in}{1.290540in}}%
\pgfpathlineto{\pgfqpoint{1.147833in}{1.293571in}}%
\pgfpathlineto{\pgfqpoint{1.136164in}{1.290261in}}%
\pgfpathlineto{\pgfqpoint{1.124690in}{1.286766in}}%
\pgfpathlineto{\pgfqpoint{1.113423in}{1.283088in}}%
\pgfpathlineto{\pgfqpoint{1.102373in}{1.279232in}}%
\pgfpathclose%
\pgfusepath{fill}%
\end{pgfscope}%
\begin{pgfscope}%
\pgfpathrectangle{\pgfqpoint{0.329460in}{0.284240in}}{\pgfqpoint{1.989680in}{1.989680in}}%
\pgfusepath{clip}%
\pgfsetbuttcap%
\pgfsetroundjoin%
\definecolor{currentfill}{rgb}{0.268510,0.009605,0.335427}%
\pgfsetfillcolor{currentfill}%
\pgfsetlinewidth{0.000000pt}%
\definecolor{currentstroke}{rgb}{0.000000,0.000000,0.000000}%
\pgfsetstrokecolor{currentstroke}%
\pgfsetdash{}{0pt}%
\pgfpathmoveto{\pgfqpoint{1.747832in}{1.215449in}}%
\pgfpathlineto{\pgfqpoint{1.750266in}{1.217698in}}%
\pgfpathlineto{\pgfqpoint{1.752706in}{1.220190in}}%
\pgfpathlineto{\pgfqpoint{1.755150in}{1.222928in}}%
\pgfpathlineto{\pgfqpoint{1.757599in}{1.225916in}}%
\pgfpathlineto{\pgfqpoint{1.769008in}{1.219323in}}%
\pgfpathlineto{\pgfqpoint{1.780037in}{1.212542in}}%
\pgfpathlineto{\pgfqpoint{1.790673in}{1.205580in}}%
\pgfpathlineto{\pgfqpoint{1.800905in}{1.198443in}}%
\pgfpathlineto{\pgfqpoint{1.798183in}{1.195608in}}%
\pgfpathlineto{\pgfqpoint{1.795467in}{1.193024in}}%
\pgfpathlineto{\pgfqpoint{1.792756in}{1.190688in}}%
\pgfpathlineto{\pgfqpoint{1.790051in}{1.188594in}}%
\pgfpathlineto{\pgfqpoint{1.780077in}{1.195570in}}%
\pgfpathlineto{\pgfqpoint{1.769709in}{1.202375in}}%
\pgfpathlineto{\pgfqpoint{1.758957in}{1.209003in}}%
\pgfpathlineto{\pgfqpoint{1.747832in}{1.215449in}}%
\pgfpathclose%
\pgfusepath{fill}%
\end{pgfscope}%
\begin{pgfscope}%
\pgfpathrectangle{\pgfqpoint{0.329460in}{0.284240in}}{\pgfqpoint{1.989680in}{1.989680in}}%
\pgfusepath{clip}%
\pgfsetbuttcap%
\pgfsetroundjoin%
\definecolor{currentfill}{rgb}{0.282327,0.094955,0.417331}%
\pgfsetfillcolor{currentfill}%
\pgfsetlinewidth{0.000000pt}%
\definecolor{currentstroke}{rgb}{0.000000,0.000000,0.000000}%
\pgfsetstrokecolor{currentstroke}%
\pgfsetdash{}{0pt}%
\pgfpathmoveto{\pgfqpoint{1.777400in}{1.259407in}}%
\pgfpathlineto{\pgfqpoint{1.779904in}{1.264862in}}%
\pgfpathlineto{\pgfqpoint{1.782416in}{1.270615in}}%
\pgfpathlineto{\pgfqpoint{1.784934in}{1.276672in}}%
\pgfpathlineto{\pgfqpoint{1.787461in}{1.283037in}}%
\pgfpathlineto{\pgfqpoint{1.799743in}{1.276023in}}%
\pgfpathlineto{\pgfqpoint{1.811620in}{1.268808in}}%
\pgfpathlineto{\pgfqpoint{1.823077in}{1.261400in}}%
\pgfpathlineto{\pgfqpoint{1.834104in}{1.253804in}}%
\pgfpathlineto{\pgfqpoint{1.831294in}{1.247579in}}%
\pgfpathlineto{\pgfqpoint{1.828493in}{1.241664in}}%
\pgfpathlineto{\pgfqpoint{1.825700in}{1.236054in}}%
\pgfpathlineto{\pgfqpoint{1.822916in}{1.230743in}}%
\pgfpathlineto{\pgfqpoint{1.812158in}{1.238190in}}%
\pgfpathlineto{\pgfqpoint{1.800978in}{1.245454in}}%
\pgfpathlineto{\pgfqpoint{1.789388in}{1.252528in}}%
\pgfpathlineto{\pgfqpoint{1.777400in}{1.259407in}}%
\pgfpathclose%
\pgfusepath{fill}%
\end{pgfscope}%
\begin{pgfscope}%
\pgfpathrectangle{\pgfqpoint{0.329460in}{0.284240in}}{\pgfqpoint{1.989680in}{1.989680in}}%
\pgfusepath{clip}%
\pgfsetbuttcap%
\pgfsetroundjoin%
\definecolor{currentfill}{rgb}{0.233603,0.313828,0.543914}%
\pgfsetfillcolor{currentfill}%
\pgfsetlinewidth{0.000000pt}%
\definecolor{currentstroke}{rgb}{0.000000,0.000000,0.000000}%
\pgfsetstrokecolor{currentstroke}%
\pgfsetdash{}{0pt}%
\pgfpathmoveto{\pgfqpoint{0.872155in}{1.378890in}}%
\pgfpathlineto{\pgfqpoint{0.869436in}{1.389740in}}%
\pgfpathlineto{\pgfqpoint{0.866705in}{1.400979in}}%
\pgfpathlineto{\pgfqpoint{0.863961in}{1.412613in}}%
\pgfpathlineto{\pgfqpoint{0.861206in}{1.424648in}}%
\pgfpathlineto{\pgfqpoint{0.874763in}{1.432126in}}%
\pgfpathlineto{\pgfqpoint{0.888745in}{1.439383in}}%
\pgfpathlineto{\pgfqpoint{0.903139in}{1.446412in}}%
\pgfpathlineto{\pgfqpoint{0.917930in}{1.453208in}}%
\pgfpathlineto{\pgfqpoint{0.920356in}{1.441069in}}%
\pgfpathlineto{\pgfqpoint{0.922770in}{1.429330in}}%
\pgfpathlineto{\pgfqpoint{0.925174in}{1.417984in}}%
\pgfpathlineto{\pgfqpoint{0.927568in}{1.407024in}}%
\pgfpathlineto{\pgfqpoint{0.913117in}{1.400329in}}%
\pgfpathlineto{\pgfqpoint{0.899055in}{1.393405in}}%
\pgfpathlineto{\pgfqpoint{0.885397in}{1.386257in}}%
\pgfpathlineto{\pgfqpoint{0.872155in}{1.378890in}}%
\pgfpathclose%
\pgfusepath{fill}%
\end{pgfscope}%
\begin{pgfscope}%
\pgfpathrectangle{\pgfqpoint{0.329460in}{0.284240in}}{\pgfqpoint{1.989680in}{1.989680in}}%
\pgfusepath{clip}%
\pgfsetbuttcap%
\pgfsetroundjoin%
\definecolor{currentfill}{rgb}{0.274128,0.199721,0.498911}%
\pgfsetfillcolor{currentfill}%
\pgfsetlinewidth{0.000000pt}%
\definecolor{currentstroke}{rgb}{0.000000,0.000000,0.000000}%
\pgfsetstrokecolor{currentstroke}%
\pgfsetdash{}{0pt}%
\pgfpathmoveto{\pgfqpoint{1.450724in}{1.339881in}}%
\pgfpathlineto{\pgfqpoint{1.451499in}{1.336159in}}%
\pgfpathlineto{\pgfqpoint{1.452274in}{1.332531in}}%
\pgfpathlineto{\pgfqpoint{1.453049in}{1.329001in}}%
\pgfpathlineto{\pgfqpoint{1.453823in}{1.325570in}}%
\pgfpathlineto{\pgfqpoint{1.466266in}{1.323857in}}%
\pgfpathlineto{\pgfqpoint{1.478606in}{1.321948in}}%
\pgfpathlineto{\pgfqpoint{1.490831in}{1.319845in}}%
\pgfpathlineto{\pgfqpoint{1.489777in}{1.323321in}}%
\pgfpathlineto{\pgfqpoint{1.488722in}{1.326898in}}%
\pgfpathlineto{\pgfqpoint{1.487667in}{1.330573in}}%
\pgfpathlineto{\pgfqpoint{1.486611in}{1.334341in}}%
\pgfpathlineto{\pgfqpoint{1.474757in}{1.336376in}}%
\pgfpathlineto{\pgfqpoint{1.462791in}{1.338223in}}%
\pgfpathlineto{\pgfqpoint{1.450724in}{1.339881in}}%
\pgfpathclose%
\pgfusepath{fill}%
\end{pgfscope}%
\begin{pgfscope}%
\pgfpathrectangle{\pgfqpoint{0.329460in}{0.284240in}}{\pgfqpoint{1.989680in}{1.989680in}}%
\pgfusepath{clip}%
\pgfsetbuttcap%
\pgfsetroundjoin%
\definecolor{currentfill}{rgb}{0.282327,0.094955,0.417331}%
\pgfsetfillcolor{currentfill}%
\pgfsetlinewidth{0.000000pt}%
\definecolor{currentstroke}{rgb}{0.000000,0.000000,0.000000}%
\pgfsetstrokecolor{currentstroke}%
\pgfsetdash{}{0pt}%
\pgfpathmoveto{\pgfqpoint{1.597216in}{1.270934in}}%
\pgfpathlineto{\pgfqpoint{1.598973in}{1.268318in}}%
\pgfpathlineto{\pgfqpoint{1.600730in}{1.265834in}}%
\pgfpathlineto{\pgfqpoint{1.602487in}{1.263488in}}%
\pgfpathlineto{\pgfqpoint{1.604245in}{1.261282in}}%
\pgfpathlineto{\pgfqpoint{1.615928in}{1.257163in}}%
\pgfpathlineto{\pgfqpoint{1.627371in}{1.252858in}}%
\pgfpathlineto{\pgfqpoint{1.638563in}{1.248370in}}%
\pgfpathlineto{\pgfqpoint{1.649492in}{1.243702in}}%
\pgfpathlineto{\pgfqpoint{1.647414in}{1.246037in}}%
\pgfpathlineto{\pgfqpoint{1.645337in}{1.248512in}}%
\pgfpathlineto{\pgfqpoint{1.643260in}{1.251124in}}%
\pgfpathlineto{\pgfqpoint{1.641184in}{1.253870in}}%
\pgfpathlineto{\pgfqpoint{1.630565in}{1.258401in}}%
\pgfpathlineto{\pgfqpoint{1.619691in}{1.262758in}}%
\pgfpathlineto{\pgfqpoint{1.608571in}{1.266937in}}%
\pgfpathlineto{\pgfqpoint{1.597216in}{1.270934in}}%
\pgfpathclose%
\pgfusepath{fill}%
\end{pgfscope}%
\begin{pgfscope}%
\pgfpathrectangle{\pgfqpoint{0.329460in}{0.284240in}}{\pgfqpoint{1.989680in}{1.989680in}}%
\pgfusepath{clip}%
\pgfsetbuttcap%
\pgfsetroundjoin%
\definecolor{currentfill}{rgb}{0.274128,0.199721,0.498911}%
\pgfsetfillcolor{currentfill}%
\pgfsetlinewidth{0.000000pt}%
\definecolor{currentstroke}{rgb}{0.000000,0.000000,0.000000}%
\pgfsetstrokecolor{currentstroke}%
\pgfsetdash{}{0pt}%
\pgfpathmoveto{\pgfqpoint{1.205330in}{1.332376in}}%
\pgfpathlineto{\pgfqpoint{1.204193in}{1.328591in}}%
\pgfpathlineto{\pgfqpoint{1.203056in}{1.324900in}}%
\pgfpathlineto{\pgfqpoint{1.201919in}{1.321307in}}%
\pgfpathlineto{\pgfqpoint{1.200784in}{1.317814in}}%
\pgfpathlineto{\pgfqpoint{1.212896in}{1.320088in}}%
\pgfpathlineto{\pgfqpoint{1.225135in}{1.322170in}}%
\pgfpathlineto{\pgfqpoint{1.237486in}{1.324057in}}%
\pgfpathlineto{\pgfqpoint{1.249941in}{1.325749in}}%
\pgfpathlineto{\pgfqpoint{1.250705in}{1.329177in}}%
\pgfpathlineto{\pgfqpoint{1.251469in}{1.332706in}}%
\pgfpathlineto{\pgfqpoint{1.252233in}{1.336333in}}%
\pgfpathlineto{\pgfqpoint{1.252997in}{1.340053in}}%
\pgfpathlineto{\pgfqpoint{1.240920in}{1.338416in}}%
\pgfpathlineto{\pgfqpoint{1.228942in}{1.336590in}}%
\pgfpathlineto{\pgfqpoint{1.217075in}{1.334576in}}%
\pgfpathlineto{\pgfqpoint{1.205330in}{1.332376in}}%
\pgfpathclose%
\pgfusepath{fill}%
\end{pgfscope}%
\begin{pgfscope}%
\pgfpathrectangle{\pgfqpoint{0.329460in}{0.284240in}}{\pgfqpoint{1.989680in}{1.989680in}}%
\pgfusepath{clip}%
\pgfsetbuttcap%
\pgfsetroundjoin%
\definecolor{currentfill}{rgb}{0.274952,0.037752,0.364543}%
\pgfsetfillcolor{currentfill}%
\pgfsetlinewidth{0.000000pt}%
\definecolor{currentstroke}{rgb}{0.000000,0.000000,0.000000}%
\pgfsetstrokecolor{currentstroke}%
\pgfsetdash{}{0pt}%
\pgfpathmoveto{\pgfqpoint{1.657810in}{1.235838in}}%
\pgfpathlineto{\pgfqpoint{1.659892in}{1.234259in}}%
\pgfpathlineto{\pgfqpoint{1.661976in}{1.232841in}}%
\pgfpathlineto{\pgfqpoint{1.664060in}{1.231589in}}%
\pgfpathlineto{\pgfqpoint{1.666146in}{1.230506in}}%
\pgfpathlineto{\pgfqpoint{1.677411in}{1.225381in}}%
\pgfpathlineto{\pgfqpoint{1.688379in}{1.220074in}}%
\pgfpathlineto{\pgfqpoint{1.699037in}{1.214589in}}%
\pgfpathlineto{\pgfqpoint{1.709377in}{1.208931in}}%
\pgfpathlineto{\pgfqpoint{1.706996in}{1.210161in}}%
\pgfpathlineto{\pgfqpoint{1.704618in}{1.211561in}}%
\pgfpathlineto{\pgfqpoint{1.702240in}{1.213126in}}%
\pgfpathlineto{\pgfqpoint{1.699865in}{1.214854in}}%
\pgfpathlineto{\pgfqpoint{1.689808in}{1.220356in}}%
\pgfpathlineto{\pgfqpoint{1.679439in}{1.225691in}}%
\pgfpathlineto{\pgfqpoint{1.668770in}{1.230853in}}%
\pgfpathlineto{\pgfqpoint{1.657810in}{1.235838in}}%
\pgfpathclose%
\pgfusepath{fill}%
\end{pgfscope}%
\begin{pgfscope}%
\pgfpathrectangle{\pgfqpoint{0.329460in}{0.284240in}}{\pgfqpoint{1.989680in}{1.989680in}}%
\pgfusepath{clip}%
\pgfsetbuttcap%
\pgfsetroundjoin%
\definecolor{currentfill}{rgb}{0.267004,0.004874,0.329415}%
\pgfsetfillcolor{currentfill}%
\pgfsetlinewidth{0.000000pt}%
\definecolor{currentstroke}{rgb}{0.000000,0.000000,0.000000}%
\pgfsetstrokecolor{currentstroke}%
\pgfsetdash{}{0pt}%
\pgfpathmoveto{\pgfqpoint{1.738137in}{1.208784in}}%
\pgfpathlineto{\pgfqpoint{1.740555in}{1.210109in}}%
\pgfpathlineto{\pgfqpoint{1.742976in}{1.211658in}}%
\pgfpathlineto{\pgfqpoint{1.745402in}{1.213437in}}%
\pgfpathlineto{\pgfqpoint{1.747832in}{1.215449in}}%
\pgfpathlineto{\pgfqpoint{1.758957in}{1.209003in}}%
\pgfpathlineto{\pgfqpoint{1.769709in}{1.202375in}}%
\pgfpathlineto{\pgfqpoint{1.780077in}{1.195570in}}%
\pgfpathlineto{\pgfqpoint{1.790051in}{1.188594in}}%
\pgfpathlineto{\pgfqpoint{1.787351in}{1.186739in}}%
\pgfpathlineto{\pgfqpoint{1.784656in}{1.185118in}}%
\pgfpathlineto{\pgfqpoint{1.781965in}{1.183726in}}%
\pgfpathlineto{\pgfqpoint{1.779280in}{1.182561in}}%
\pgfpathlineto{\pgfqpoint{1.769562in}{1.189372in}}%
\pgfpathlineto{\pgfqpoint{1.759459in}{1.196016in}}%
\pgfpathlineto{\pgfqpoint{1.748981in}{1.202489in}}%
\pgfpathlineto{\pgfqpoint{1.738137in}{1.208784in}}%
\pgfpathclose%
\pgfusepath{fill}%
\end{pgfscope}%
\begin{pgfscope}%
\pgfpathrectangle{\pgfqpoint{0.329460in}{0.284240in}}{\pgfqpoint{1.989680in}{1.989680in}}%
\pgfusepath{clip}%
\pgfsetbuttcap%
\pgfsetroundjoin%
\definecolor{currentfill}{rgb}{0.282884,0.135920,0.453427}%
\pgfsetfillcolor{currentfill}%
\pgfsetlinewidth{0.000000pt}%
\definecolor{currentstroke}{rgb}{0.000000,0.000000,0.000000}%
\pgfsetstrokecolor{currentstroke}%
\pgfsetdash{}{0pt}%
\pgfpathmoveto{\pgfqpoint{1.787461in}{1.283037in}}%
\pgfpathlineto{\pgfqpoint{1.789995in}{1.289716in}}%
\pgfpathlineto{\pgfqpoint{1.792538in}{1.296714in}}%
\pgfpathlineto{\pgfqpoint{1.795089in}{1.304037in}}%
\pgfpathlineto{\pgfqpoint{1.797649in}{1.311689in}}%
\pgfpathlineto{\pgfqpoint{1.810230in}{1.304543in}}%
\pgfpathlineto{\pgfqpoint{1.822397in}{1.297193in}}%
\pgfpathlineto{\pgfqpoint{1.834136in}{1.289646in}}%
\pgfpathlineto{\pgfqpoint{1.845435in}{1.281907in}}%
\pgfpathlineto{\pgfqpoint{1.842588in}{1.274390in}}%
\pgfpathlineto{\pgfqpoint{1.839750in}{1.267204in}}%
\pgfpathlineto{\pgfqpoint{1.836922in}{1.260344in}}%
\pgfpathlineto{\pgfqpoint{1.834104in}{1.253804in}}%
\pgfpathlineto{\pgfqpoint{1.823077in}{1.261400in}}%
\pgfpathlineto{\pgfqpoint{1.811620in}{1.268808in}}%
\pgfpathlineto{\pgfqpoint{1.799743in}{1.276023in}}%
\pgfpathlineto{\pgfqpoint{1.787461in}{1.283037in}}%
\pgfpathclose%
\pgfusepath{fill}%
\end{pgfscope}%
\begin{pgfscope}%
\pgfpathrectangle{\pgfqpoint{0.329460in}{0.284240in}}{\pgfqpoint{1.989680in}{1.989680in}}%
\pgfusepath{clip}%
\pgfsetbuttcap%
\pgfsetroundjoin%
\definecolor{currentfill}{rgb}{0.272594,0.025563,0.353093}%
\pgfsetfillcolor{currentfill}%
\pgfsetlinewidth{0.000000pt}%
\definecolor{currentstroke}{rgb}{0.000000,0.000000,0.000000}%
\pgfsetstrokecolor{currentstroke}%
\pgfsetdash{}{0pt}%
\pgfpathmoveto{\pgfqpoint{0.892723in}{1.191957in}}%
\pgfpathlineto{\pgfqpoint{0.889939in}{1.195013in}}%
\pgfpathlineto{\pgfqpoint{0.887149in}{1.198329in}}%
\pgfpathlineto{\pgfqpoint{0.884352in}{1.201911in}}%
\pgfpathlineto{\pgfqpoint{0.881548in}{1.205764in}}%
\pgfpathlineto{\pgfqpoint{0.891666in}{1.213212in}}%
\pgfpathlineto{\pgfqpoint{0.902206in}{1.220487in}}%
\pgfpathlineto{\pgfqpoint{0.913157in}{1.227582in}}%
\pgfpathlineto{\pgfqpoint{0.924508in}{1.234490in}}%
\pgfpathlineto{\pgfqpoint{0.927044in}{1.230484in}}%
\pgfpathlineto{\pgfqpoint{0.929574in}{1.226748in}}%
\pgfpathlineto{\pgfqpoint{0.932098in}{1.223276in}}%
\pgfpathlineto{\pgfqpoint{0.934616in}{1.220065in}}%
\pgfpathlineto{\pgfqpoint{0.923545in}{1.213304in}}%
\pgfpathlineto{\pgfqpoint{0.912865in}{1.206363in}}%
\pgfpathlineto{\pgfqpoint{0.902587in}{1.199245in}}%
\pgfpathlineto{\pgfqpoint{0.892723in}{1.191957in}}%
\pgfpathclose%
\pgfusepath{fill}%
\end{pgfscope}%
\begin{pgfscope}%
\pgfpathrectangle{\pgfqpoint{0.329460in}{0.284240in}}{\pgfqpoint{1.989680in}{1.989680in}}%
\pgfusepath{clip}%
\pgfsetbuttcap%
\pgfsetroundjoin%
\definecolor{currentfill}{rgb}{0.201239,0.383670,0.554294}%
\pgfsetfillcolor{currentfill}%
\pgfsetlinewidth{0.000000pt}%
\definecolor{currentstroke}{rgb}{0.000000,0.000000,0.000000}%
\pgfsetstrokecolor{currentstroke}%
\pgfsetdash{}{0pt}%
\pgfpathmoveto{\pgfqpoint{1.770975in}{1.459049in}}%
\pgfpathlineto{\pgfqpoint{1.773333in}{1.471614in}}%
\pgfpathlineto{\pgfqpoint{1.775703in}{1.484592in}}%
\pgfpathlineto{\pgfqpoint{1.778084in}{1.497990in}}%
\pgfpathlineto{\pgfqpoint{1.780476in}{1.511812in}}%
\pgfpathlineto{\pgfqpoint{1.795962in}{1.505138in}}%
\pgfpathlineto{\pgfqpoint{1.811056in}{1.498221in}}%
\pgfpathlineto{\pgfqpoint{1.825743in}{1.491069in}}%
\pgfpathlineto{\pgfqpoint{1.840008in}{1.483686in}}%
\pgfpathlineto{\pgfqpoint{1.837271in}{1.469957in}}%
\pgfpathlineto{\pgfqpoint{1.834548in}{1.456655in}}%
\pgfpathlineto{\pgfqpoint{1.831838in}{1.443773in}}%
\pgfpathlineto{\pgfqpoint{1.829140in}{1.431306in}}%
\pgfpathlineto{\pgfqpoint{1.815204in}{1.438588in}}%
\pgfpathlineto{\pgfqpoint{1.800855in}{1.445642in}}%
\pgfpathlineto{\pgfqpoint{1.786107in}{1.452464in}}%
\pgfpathlineto{\pgfqpoint{1.770975in}{1.459049in}}%
\pgfpathclose%
\pgfusepath{fill}%
\end{pgfscope}%
\begin{pgfscope}%
\pgfpathrectangle{\pgfqpoint{0.329460in}{0.284240in}}{\pgfqpoint{1.989680in}{1.989680in}}%
\pgfusepath{clip}%
\pgfsetbuttcap%
\pgfsetroundjoin%
\definecolor{currentfill}{rgb}{0.277941,0.056324,0.381191}%
\pgfsetfillcolor{currentfill}%
\pgfsetlinewidth{0.000000pt}%
\definecolor{currentstroke}{rgb}{0.000000,0.000000,0.000000}%
\pgfsetstrokecolor{currentstroke}%
\pgfsetdash{}{0pt}%
\pgfpathmoveto{\pgfqpoint{0.881548in}{1.205764in}}%
\pgfpathlineto{\pgfqpoint{0.878737in}{1.209891in}}%
\pgfpathlineto{\pgfqpoint{0.875919in}{1.214299in}}%
\pgfpathlineto{\pgfqpoint{0.873093in}{1.218991in}}%
\pgfpathlineto{\pgfqpoint{0.870259in}{1.223973in}}%
\pgfpathlineto{\pgfqpoint{0.880633in}{1.231579in}}%
\pgfpathlineto{\pgfqpoint{0.891439in}{1.239006in}}%
\pgfpathlineto{\pgfqpoint{0.902665in}{1.246249in}}%
\pgfpathlineto{\pgfqpoint{0.914300in}{1.253302in}}%
\pgfpathlineto{\pgfqpoint{0.916862in}{1.248171in}}%
\pgfpathlineto{\pgfqpoint{0.919418in}{1.243329in}}%
\pgfpathlineto{\pgfqpoint{0.921966in}{1.238770in}}%
\pgfpathlineto{\pgfqpoint{0.924508in}{1.234490in}}%
\pgfpathlineto{\pgfqpoint{0.913157in}{1.227582in}}%
\pgfpathlineto{\pgfqpoint{0.902206in}{1.220487in}}%
\pgfpathlineto{\pgfqpoint{0.891666in}{1.213212in}}%
\pgfpathlineto{\pgfqpoint{0.881548in}{1.205764in}}%
\pgfpathclose%
\pgfusepath{fill}%
\end{pgfscope}%
\begin{pgfscope}%
\pgfpathrectangle{\pgfqpoint{0.329460in}{0.284240in}}{\pgfqpoint{1.989680in}{1.989680in}}%
\pgfusepath{clip}%
\pgfsetbuttcap%
\pgfsetroundjoin%
\definecolor{currentfill}{rgb}{0.282327,0.094955,0.417331}%
\pgfsetfillcolor{currentfill}%
\pgfsetlinewidth{0.000000pt}%
\definecolor{currentstroke}{rgb}{0.000000,0.000000,0.000000}%
\pgfsetstrokecolor{currentstroke}%
\pgfsetdash{}{0pt}%
\pgfpathmoveto{\pgfqpoint{1.051975in}{1.249700in}}%
\pgfpathlineto{\pgfqpoint{1.049831in}{1.246922in}}%
\pgfpathlineto{\pgfqpoint{1.047687in}{1.244278in}}%
\pgfpathlineto{\pgfqpoint{1.045543in}{1.241771in}}%
\pgfpathlineto{\pgfqpoint{1.043398in}{1.239405in}}%
\pgfpathlineto{\pgfqpoint{1.054085in}{1.244229in}}%
\pgfpathlineto{\pgfqpoint{1.065043in}{1.248877in}}%
\pgfpathlineto{\pgfqpoint{1.076263in}{1.253346in}}%
\pgfpathlineto{\pgfqpoint{1.087733in}{1.257630in}}%
\pgfpathlineto{\pgfqpoint{1.089564in}{1.259863in}}%
\pgfpathlineto{\pgfqpoint{1.091395in}{1.262236in}}%
\pgfpathlineto{\pgfqpoint{1.093225in}{1.264746in}}%
\pgfpathlineto{\pgfqpoint{1.095055in}{1.267390in}}%
\pgfpathlineto{\pgfqpoint{1.083908in}{1.263231in}}%
\pgfpathlineto{\pgfqpoint{1.073006in}{1.258894in}}%
\pgfpathlineto{\pgfqpoint{1.062358in}{1.254382in}}%
\pgfpathlineto{\pgfqpoint{1.051975in}{1.249700in}}%
\pgfpathclose%
\pgfusepath{fill}%
\end{pgfscope}%
\begin{pgfscope}%
\pgfpathrectangle{\pgfqpoint{0.329460in}{0.284240in}}{\pgfqpoint{1.989680in}{1.989680in}}%
\pgfusepath{clip}%
\pgfsetbuttcap%
\pgfsetroundjoin%
\definecolor{currentfill}{rgb}{0.268510,0.009605,0.335427}%
\pgfsetfillcolor{currentfill}%
\pgfsetlinewidth{0.000000pt}%
\definecolor{currentstroke}{rgb}{0.000000,0.000000,0.000000}%
\pgfsetstrokecolor{currentstroke}%
\pgfsetdash{}{0pt}%
\pgfpathmoveto{\pgfqpoint{0.903799in}{1.182255in}}%
\pgfpathlineto{\pgfqpoint{0.901038in}{1.184312in}}%
\pgfpathlineto{\pgfqpoint{0.898272in}{1.186611in}}%
\pgfpathlineto{\pgfqpoint{0.895501in}{1.189158in}}%
\pgfpathlineto{\pgfqpoint{0.892723in}{1.191957in}}%
\pgfpathlineto{\pgfqpoint{0.902587in}{1.199245in}}%
\pgfpathlineto{\pgfqpoint{0.912865in}{1.206363in}}%
\pgfpathlineto{\pgfqpoint{0.923545in}{1.213304in}}%
\pgfpathlineto{\pgfqpoint{0.934616in}{1.220065in}}%
\pgfpathlineto{\pgfqpoint{0.937129in}{1.217108in}}%
\pgfpathlineto{\pgfqpoint{0.939636in}{1.214403in}}%
\pgfpathlineto{\pgfqpoint{0.942138in}{1.211945in}}%
\pgfpathlineto{\pgfqpoint{0.944636in}{1.209728in}}%
\pgfpathlineto{\pgfqpoint{0.933842in}{1.203120in}}%
\pgfpathlineto{\pgfqpoint{0.923430in}{1.196335in}}%
\pgfpathlineto{\pgfqpoint{0.913413in}{1.189378in}}%
\pgfpathlineto{\pgfqpoint{0.903799in}{1.182255in}}%
\pgfpathclose%
\pgfusepath{fill}%
\end{pgfscope}%
\begin{pgfscope}%
\pgfpathrectangle{\pgfqpoint{0.329460in}{0.284240in}}{\pgfqpoint{1.989680in}{1.989680in}}%
\pgfusepath{clip}%
\pgfsetbuttcap%
\pgfsetroundjoin%
\definecolor{currentfill}{rgb}{0.263663,0.237631,0.518762}%
\pgfsetfillcolor{currentfill}%
\pgfsetlinewidth{0.000000pt}%
\definecolor{currentstroke}{rgb}{0.000000,0.000000,0.000000}%
\pgfsetstrokecolor{currentstroke}%
\pgfsetdash{}{0pt}%
\pgfpathmoveto{\pgfqpoint{1.303615in}{1.360287in}}%
\pgfpathlineto{\pgfqpoint{1.303232in}{1.356260in}}%
\pgfpathlineto{\pgfqpoint{1.302850in}{1.352315in}}%
\pgfpathlineto{\pgfqpoint{1.302467in}{1.348455in}}%
\pgfpathlineto{\pgfqpoint{1.302085in}{1.344682in}}%
\pgfpathlineto{\pgfqpoint{1.314495in}{1.345355in}}%
\pgfpathlineto{\pgfqpoint{1.326939in}{1.345833in}}%
\pgfpathlineto{\pgfqpoint{1.339405in}{1.346116in}}%
\pgfpathlineto{\pgfqpoint{1.351881in}{1.346203in}}%
\pgfpathlineto{\pgfqpoint{1.351875in}{1.349962in}}%
\pgfpathlineto{\pgfqpoint{1.351870in}{1.353810in}}%
\pgfpathlineto{\pgfqpoint{1.351865in}{1.357742in}}%
\pgfpathlineto{\pgfqpoint{1.351859in}{1.361757in}}%
\pgfpathlineto{\pgfqpoint{1.339772in}{1.361673in}}%
\pgfpathlineto{\pgfqpoint{1.327695in}{1.361400in}}%
\pgfpathlineto{\pgfqpoint{1.315639in}{1.360938in}}%
\pgfpathlineto{\pgfqpoint{1.303615in}{1.360287in}}%
\pgfpathclose%
\pgfusepath{fill}%
\end{pgfscope}%
\begin{pgfscope}%
\pgfpathrectangle{\pgfqpoint{0.329460in}{0.284240in}}{\pgfqpoint{1.989680in}{1.989680in}}%
\pgfusepath{clip}%
\pgfsetbuttcap%
\pgfsetroundjoin%
\definecolor{currentfill}{rgb}{0.267004,0.004874,0.329415}%
\pgfsetfillcolor{currentfill}%
\pgfsetlinewidth{0.000000pt}%
\definecolor{currentstroke}{rgb}{0.000000,0.000000,0.000000}%
\pgfsetstrokecolor{currentstroke}%
\pgfsetdash{}{0pt}%
\pgfpathmoveto{\pgfqpoint{1.728504in}{1.205649in}}%
\pgfpathlineto{\pgfqpoint{1.730907in}{1.206116in}}%
\pgfpathlineto{\pgfqpoint{1.733314in}{1.206791in}}%
\pgfpathlineto{\pgfqpoint{1.735724in}{1.207679in}}%
\pgfpathlineto{\pgfqpoint{1.738137in}{1.208784in}}%
\pgfpathlineto{\pgfqpoint{1.748981in}{1.202489in}}%
\pgfpathlineto{\pgfqpoint{1.759459in}{1.196016in}}%
\pgfpathlineto{\pgfqpoint{1.769562in}{1.189372in}}%
\pgfpathlineto{\pgfqpoint{1.779280in}{1.182561in}}%
\pgfpathlineto{\pgfqpoint{1.776598in}{1.181616in}}%
\pgfpathlineto{\pgfqpoint{1.773921in}{1.180888in}}%
\pgfpathlineto{\pgfqpoint{1.771248in}{1.180374in}}%
\pgfpathlineto{\pgfqpoint{1.768579in}{1.180068in}}%
\pgfpathlineto{\pgfqpoint{1.759115in}{1.186712in}}%
\pgfpathlineto{\pgfqpoint{1.749275in}{1.193193in}}%
\pgfpathlineto{\pgfqpoint{1.739067in}{1.199507in}}%
\pgfpathlineto{\pgfqpoint{1.728504in}{1.205649in}}%
\pgfpathclose%
\pgfusepath{fill}%
\end{pgfscope}%
\begin{pgfscope}%
\pgfpathrectangle{\pgfqpoint{0.329460in}{0.284240in}}{\pgfqpoint{1.989680in}{1.989680in}}%
\pgfusepath{clip}%
\pgfsetbuttcap%
\pgfsetroundjoin%
\definecolor{currentfill}{rgb}{0.263663,0.237631,0.518762}%
\pgfsetfillcolor{currentfill}%
\pgfsetlinewidth{0.000000pt}%
\definecolor{currentstroke}{rgb}{0.000000,0.000000,0.000000}%
\pgfsetstrokecolor{currentstroke}%
\pgfsetdash{}{0pt}%
\pgfpathmoveto{\pgfqpoint{1.351859in}{1.361757in}}%
\pgfpathlineto{\pgfqpoint{1.351865in}{1.357742in}}%
\pgfpathlineto{\pgfqpoint{1.351870in}{1.353810in}}%
\pgfpathlineto{\pgfqpoint{1.351875in}{1.349962in}}%
\pgfpathlineto{\pgfqpoint{1.351881in}{1.346203in}}%
\pgfpathlineto{\pgfqpoint{1.364356in}{1.346094in}}%
\pgfpathlineto{\pgfqpoint{1.376820in}{1.345790in}}%
\pgfpathlineto{\pgfqpoint{1.389260in}{1.345290in}}%
\pgfpathlineto{\pgfqpoint{1.401667in}{1.344595in}}%
\pgfpathlineto{\pgfqpoint{1.401274in}{1.348368in}}%
\pgfpathlineto{\pgfqpoint{1.400880in}{1.352230in}}%
\pgfpathlineto{\pgfqpoint{1.400487in}{1.356176in}}%
\pgfpathlineto{\pgfqpoint{1.400093in}{1.360203in}}%
\pgfpathlineto{\pgfqpoint{1.388074in}{1.360875in}}%
\pgfpathlineto{\pgfqpoint{1.376021in}{1.361358in}}%
\pgfpathlineto{\pgfqpoint{1.363946in}{1.361652in}}%
\pgfpathlineto{\pgfqpoint{1.351859in}{1.361757in}}%
\pgfpathclose%
\pgfusepath{fill}%
\end{pgfscope}%
\begin{pgfscope}%
\pgfpathrectangle{\pgfqpoint{0.329460in}{0.284240in}}{\pgfqpoint{1.989680in}{1.989680in}}%
\pgfusepath{clip}%
\pgfsetbuttcap%
\pgfsetroundjoin%
\definecolor{currentfill}{rgb}{0.274952,0.037752,0.364543}%
\pgfsetfillcolor{currentfill}%
\pgfsetlinewidth{0.000000pt}%
\definecolor{currentstroke}{rgb}{0.000000,0.000000,0.000000}%
\pgfsetstrokecolor{currentstroke}%
\pgfsetdash{}{0pt}%
\pgfpathmoveto{\pgfqpoint{0.993841in}{1.209825in}}%
\pgfpathlineto{\pgfqpoint{0.991404in}{1.208062in}}%
\pgfpathlineto{\pgfqpoint{0.988966in}{1.206461in}}%
\pgfpathlineto{\pgfqpoint{0.986526in}{1.205026in}}%
\pgfpathlineto{\pgfqpoint{0.984084in}{1.203760in}}%
\pgfpathlineto{\pgfqpoint{0.994131in}{1.209568in}}%
\pgfpathlineto{\pgfqpoint{1.004506in}{1.215207in}}%
\pgfpathlineto{\pgfqpoint{1.015200in}{1.220673in}}%
\pgfpathlineto{\pgfqpoint{1.026201in}{1.225960in}}%
\pgfpathlineto{\pgfqpoint{1.028355in}{1.227074in}}%
\pgfpathlineto{\pgfqpoint{1.030508in}{1.228357in}}%
\pgfpathlineto{\pgfqpoint{1.032659in}{1.229805in}}%
\pgfpathlineto{\pgfqpoint{1.034809in}{1.231416in}}%
\pgfpathlineto{\pgfqpoint{1.024106in}{1.226273in}}%
\pgfpathlineto{\pgfqpoint{1.013704in}{1.220957in}}%
\pgfpathlineto{\pgfqpoint{1.003612in}{1.215473in}}%
\pgfpathlineto{\pgfqpoint{0.993841in}{1.209825in}}%
\pgfpathclose%
\pgfusepath{fill}%
\end{pgfscope}%
\begin{pgfscope}%
\pgfpathrectangle{\pgfqpoint{0.329460in}{0.284240in}}{\pgfqpoint{1.989680in}{1.989680in}}%
\pgfusepath{clip}%
\pgfsetbuttcap%
\pgfsetroundjoin%
\definecolor{currentfill}{rgb}{0.282327,0.094955,0.417331}%
\pgfsetfillcolor{currentfill}%
\pgfsetlinewidth{0.000000pt}%
\definecolor{currentstroke}{rgb}{0.000000,0.000000,0.000000}%
\pgfsetstrokecolor{currentstroke}%
\pgfsetdash{}{0pt}%
\pgfpathmoveto{\pgfqpoint{0.870259in}{1.223973in}}%
\pgfpathlineto{\pgfqpoint{0.867417in}{1.229250in}}%
\pgfpathlineto{\pgfqpoint{0.864567in}{1.234827in}}%
\pgfpathlineto{\pgfqpoint{0.861708in}{1.240708in}}%
\pgfpathlineto{\pgfqpoint{0.858841in}{1.246900in}}%
\pgfpathlineto{\pgfqpoint{0.869475in}{1.254657in}}%
\pgfpathlineto{\pgfqpoint{0.880550in}{1.262233in}}%
\pgfpathlineto{\pgfqpoint{0.892055in}{1.269620in}}%
\pgfpathlineto{\pgfqpoint{0.903977in}{1.276812in}}%
\pgfpathlineto{\pgfqpoint{0.906569in}{1.270477in}}%
\pgfpathlineto{\pgfqpoint{0.909154in}{1.264450in}}%
\pgfpathlineto{\pgfqpoint{0.911731in}{1.258727in}}%
\pgfpathlineto{\pgfqpoint{0.914300in}{1.253302in}}%
\pgfpathlineto{\pgfqpoint{0.902665in}{1.246249in}}%
\pgfpathlineto{\pgfqpoint{0.891439in}{1.239006in}}%
\pgfpathlineto{\pgfqpoint{0.880633in}{1.231579in}}%
\pgfpathlineto{\pgfqpoint{0.870259in}{1.223973in}}%
\pgfpathclose%
\pgfusepath{fill}%
\end{pgfscope}%
\begin{pgfscope}%
\pgfpathrectangle{\pgfqpoint{0.329460in}{0.284240in}}{\pgfqpoint{1.989680in}{1.989680in}}%
\pgfusepath{clip}%
\pgfsetbuttcap%
\pgfsetroundjoin%
\definecolor{currentfill}{rgb}{0.267004,0.004874,0.329415}%
\pgfsetfillcolor{currentfill}%
\pgfsetlinewidth{0.000000pt}%
\definecolor{currentstroke}{rgb}{0.000000,0.000000,0.000000}%
\pgfsetstrokecolor{currentstroke}%
\pgfsetdash{}{0pt}%
\pgfpathmoveto{\pgfqpoint{0.914790in}{1.176371in}}%
\pgfpathlineto{\pgfqpoint{0.912049in}{1.177499in}}%
\pgfpathlineto{\pgfqpoint{0.909304in}{1.178853in}}%
\pgfpathlineto{\pgfqpoint{0.906554in}{1.180437in}}%
\pgfpathlineto{\pgfqpoint{0.903799in}{1.182255in}}%
\pgfpathlineto{\pgfqpoint{0.913413in}{1.189378in}}%
\pgfpathlineto{\pgfqpoint{0.923430in}{1.196335in}}%
\pgfpathlineto{\pgfqpoint{0.933842in}{1.203120in}}%
\pgfpathlineto{\pgfqpoint{0.944636in}{1.209728in}}%
\pgfpathlineto{\pgfqpoint{0.947129in}{1.207750in}}%
\pgfpathlineto{\pgfqpoint{0.949617in}{1.206004in}}%
\pgfpathlineto{\pgfqpoint{0.952102in}{1.204488in}}%
\pgfpathlineto{\pgfqpoint{0.954582in}{1.203197in}}%
\pgfpathlineto{\pgfqpoint{0.944062in}{1.196744in}}%
\pgfpathlineto{\pgfqpoint{0.933916in}{1.190118in}}%
\pgfpathlineto{\pgfqpoint{0.924156in}{1.183325in}}%
\pgfpathlineto{\pgfqpoint{0.914790in}{1.176371in}}%
\pgfpathclose%
\pgfusepath{fill}%
\end{pgfscope}%
\begin{pgfscope}%
\pgfpathrectangle{\pgfqpoint{0.329460in}{0.284240in}}{\pgfqpoint{1.989680in}{1.989680in}}%
\pgfusepath{clip}%
\pgfsetbuttcap%
\pgfsetroundjoin%
\definecolor{currentfill}{rgb}{0.280255,0.165693,0.476498}%
\pgfsetfillcolor{currentfill}%
\pgfsetlinewidth{0.000000pt}%
\definecolor{currentstroke}{rgb}{0.000000,0.000000,0.000000}%
\pgfsetstrokecolor{currentstroke}%
\pgfsetdash{}{0pt}%
\pgfpathmoveto{\pgfqpoint{1.538356in}{1.309528in}}%
\pgfpathlineto{\pgfqpoint{1.539771in}{1.306072in}}%
\pgfpathlineto{\pgfqpoint{1.541186in}{1.302722in}}%
\pgfpathlineto{\pgfqpoint{1.542601in}{1.299482in}}%
\pgfpathlineto{\pgfqpoint{1.544015in}{1.296356in}}%
\pgfpathlineto{\pgfqpoint{1.555848in}{1.293212in}}%
\pgfpathlineto{\pgfqpoint{1.567496in}{1.289882in}}%
\pgfpathlineto{\pgfqpoint{1.578947in}{1.286366in}}%
\pgfpathlineto{\pgfqpoint{1.590191in}{1.282669in}}%
\pgfpathlineto{\pgfqpoint{1.588434in}{1.285902in}}%
\pgfpathlineto{\pgfqpoint{1.586677in}{1.289250in}}%
\pgfpathlineto{\pgfqpoint{1.584920in}{1.292707in}}%
\pgfpathlineto{\pgfqpoint{1.583162in}{1.296272in}}%
\pgfpathlineto{\pgfqpoint{1.572253in}{1.299852in}}%
\pgfpathlineto{\pgfqpoint{1.561142in}{1.303257in}}%
\pgfpathlineto{\pgfqpoint{1.549840in}{1.306483in}}%
\pgfpathlineto{\pgfqpoint{1.538356in}{1.309528in}}%
\pgfpathclose%
\pgfusepath{fill}%
\end{pgfscope}%
\begin{pgfscope}%
\pgfpathrectangle{\pgfqpoint{0.329460in}{0.284240in}}{\pgfqpoint{1.989680in}{1.989680in}}%
\pgfusepath{clip}%
\pgfsetbuttcap%
\pgfsetroundjoin%
\definecolor{currentfill}{rgb}{0.276194,0.190074,0.493001}%
\pgfsetfillcolor{currentfill}%
\pgfsetlinewidth{0.000000pt}%
\definecolor{currentstroke}{rgb}{0.000000,0.000000,0.000000}%
\pgfsetstrokecolor{currentstroke}%
\pgfsetdash{}{0pt}%
\pgfpathmoveto{\pgfqpoint{1.797649in}{1.311689in}}%
\pgfpathlineto{\pgfqpoint{1.800218in}{1.319676in}}%
\pgfpathlineto{\pgfqpoint{1.802795in}{1.328004in}}%
\pgfpathlineto{\pgfqpoint{1.805383in}{1.336678in}}%
\pgfpathlineto{\pgfqpoint{1.807980in}{1.345704in}}%
\pgfpathlineto{\pgfqpoint{1.820865in}{1.338433in}}%
\pgfpathlineto{\pgfqpoint{1.833326in}{1.330955in}}%
\pgfpathlineto{\pgfqpoint{1.845351in}{1.323274in}}%
\pgfpathlineto{\pgfqpoint{1.856928in}{1.315397in}}%
\pgfpathlineto{\pgfqpoint{1.854038in}{1.306500in}}%
\pgfpathlineto{\pgfqpoint{1.851160in}{1.297956in}}%
\pgfpathlineto{\pgfqpoint{1.848292in}{1.289761in}}%
\pgfpathlineto{\pgfqpoint{1.845435in}{1.281907in}}%
\pgfpathlineto{\pgfqpoint{1.834136in}{1.289646in}}%
\pgfpathlineto{\pgfqpoint{1.822397in}{1.297193in}}%
\pgfpathlineto{\pgfqpoint{1.810230in}{1.304543in}}%
\pgfpathlineto{\pgfqpoint{1.797649in}{1.311689in}}%
\pgfpathclose%
\pgfusepath{fill}%
\end{pgfscope}%
\begin{pgfscope}%
\pgfpathrectangle{\pgfqpoint{0.329460in}{0.284240in}}{\pgfqpoint{1.989680in}{1.989680in}}%
\pgfusepath{clip}%
\pgfsetbuttcap%
\pgfsetroundjoin%
\definecolor{currentfill}{rgb}{0.274128,0.199721,0.498911}%
\pgfsetfillcolor{currentfill}%
\pgfsetlinewidth{0.000000pt}%
\definecolor{currentstroke}{rgb}{0.000000,0.000000,0.000000}%
\pgfsetstrokecolor{currentstroke}%
\pgfsetdash{}{0pt}%
\pgfpathmoveto{\pgfqpoint{1.486611in}{1.334341in}}%
\pgfpathlineto{\pgfqpoint{1.487667in}{1.330573in}}%
\pgfpathlineto{\pgfqpoint{1.488722in}{1.326898in}}%
\pgfpathlineto{\pgfqpoint{1.489777in}{1.323321in}}%
\pgfpathlineto{\pgfqpoint{1.490831in}{1.319845in}}%
\pgfpathlineto{\pgfqpoint{1.502929in}{1.317549in}}%
\pgfpathlineto{\pgfqpoint{1.514890in}{1.315063in}}%
\pgfpathlineto{\pgfqpoint{1.526703in}{1.312389in}}%
\pgfpathlineto{\pgfqpoint{1.538356in}{1.309528in}}%
\pgfpathlineto{\pgfqpoint{1.536941in}{1.313088in}}%
\pgfpathlineto{\pgfqpoint{1.535525in}{1.316749in}}%
\pgfpathlineto{\pgfqpoint{1.534109in}{1.320507in}}%
\pgfpathlineto{\pgfqpoint{1.532691in}{1.324360in}}%
\pgfpathlineto{\pgfqpoint{1.521393in}{1.327127in}}%
\pgfpathlineto{\pgfqpoint{1.509940in}{1.329715in}}%
\pgfpathlineto{\pgfqpoint{1.498342in}{1.332120in}}%
\pgfpathlineto{\pgfqpoint{1.486611in}{1.334341in}}%
\pgfpathclose%
\pgfusepath{fill}%
\end{pgfscope}%
\begin{pgfscope}%
\pgfpathrectangle{\pgfqpoint{0.329460in}{0.284240in}}{\pgfqpoint{1.989680in}{1.989680in}}%
\pgfusepath{clip}%
\pgfsetbuttcap%
\pgfsetroundjoin%
\definecolor{currentfill}{rgb}{0.263663,0.237631,0.518762}%
\pgfsetfillcolor{currentfill}%
\pgfsetlinewidth{0.000000pt}%
\definecolor{currentstroke}{rgb}{0.000000,0.000000,0.000000}%
\pgfsetstrokecolor{currentstroke}%
\pgfsetdash{}{0pt}%
\pgfpathmoveto{\pgfqpoint{1.256060in}{1.355814in}}%
\pgfpathlineto{\pgfqpoint{1.255294in}{1.351748in}}%
\pgfpathlineto{\pgfqpoint{1.254528in}{1.347764in}}%
\pgfpathlineto{\pgfqpoint{1.253762in}{1.343865in}}%
\pgfpathlineto{\pgfqpoint{1.252997in}{1.340053in}}%
\pgfpathlineto{\pgfqpoint{1.265164in}{1.341499in}}%
\pgfpathlineto{\pgfqpoint{1.277408in}{1.342753in}}%
\pgfpathlineto{\pgfqpoint{1.289719in}{1.343815in}}%
\pgfpathlineto{\pgfqpoint{1.302085in}{1.344682in}}%
\pgfpathlineto{\pgfqpoint{1.302467in}{1.348455in}}%
\pgfpathlineto{\pgfqpoint{1.302850in}{1.352315in}}%
\pgfpathlineto{\pgfqpoint{1.303232in}{1.356260in}}%
\pgfpathlineto{\pgfqpoint{1.303615in}{1.360287in}}%
\pgfpathlineto{\pgfqpoint{1.291635in}{1.359449in}}%
\pgfpathlineto{\pgfqpoint{1.279708in}{1.358423in}}%
\pgfpathlineto{\pgfqpoint{1.267846in}{1.357211in}}%
\pgfpathlineto{\pgfqpoint{1.256060in}{1.355814in}}%
\pgfpathclose%
\pgfusepath{fill}%
\end{pgfscope}%
\begin{pgfscope}%
\pgfpathrectangle{\pgfqpoint{0.329460in}{0.284240in}}{\pgfqpoint{1.989680in}{1.989680in}}%
\pgfusepath{clip}%
\pgfsetbuttcap%
\pgfsetroundjoin%
\definecolor{currentfill}{rgb}{0.263663,0.237631,0.518762}%
\pgfsetfillcolor{currentfill}%
\pgfsetlinewidth{0.000000pt}%
\definecolor{currentstroke}{rgb}{0.000000,0.000000,0.000000}%
\pgfsetstrokecolor{currentstroke}%
\pgfsetdash{}{0pt}%
\pgfpathmoveto{\pgfqpoint{1.400093in}{1.360203in}}%
\pgfpathlineto{\pgfqpoint{1.400487in}{1.356176in}}%
\pgfpathlineto{\pgfqpoint{1.400880in}{1.352230in}}%
\pgfpathlineto{\pgfqpoint{1.401274in}{1.348368in}}%
\pgfpathlineto{\pgfqpoint{1.401667in}{1.344595in}}%
\pgfpathlineto{\pgfqpoint{1.414027in}{1.343706in}}%
\pgfpathlineto{\pgfqpoint{1.426331in}{1.342623in}}%
\pgfpathlineto{\pgfqpoint{1.438567in}{1.341348in}}%
\pgfpathlineto{\pgfqpoint{1.450724in}{1.339881in}}%
\pgfpathlineto{\pgfqpoint{1.449949in}{1.343694in}}%
\pgfpathlineto{\pgfqpoint{1.449173in}{1.347594in}}%
\pgfpathlineto{\pgfqpoint{1.448396in}{1.351580in}}%
\pgfpathlineto{\pgfqpoint{1.447619in}{1.355647in}}%
\pgfpathlineto{\pgfqpoint{1.435842in}{1.357065in}}%
\pgfpathlineto{\pgfqpoint{1.423988in}{1.358298in}}%
\pgfpathlineto{\pgfqpoint{1.412068in}{1.359344in}}%
\pgfpathlineto{\pgfqpoint{1.400093in}{1.360203in}}%
\pgfpathclose%
\pgfusepath{fill}%
\end{pgfscope}%
\begin{pgfscope}%
\pgfpathrectangle{\pgfqpoint{0.329460in}{0.284240in}}{\pgfqpoint{1.989680in}{1.989680in}}%
\pgfusepath{clip}%
\pgfsetbuttcap%
\pgfsetroundjoin%
\definecolor{currentfill}{rgb}{0.282884,0.135920,0.453427}%
\pgfsetfillcolor{currentfill}%
\pgfsetlinewidth{0.000000pt}%
\definecolor{currentstroke}{rgb}{0.000000,0.000000,0.000000}%
\pgfsetstrokecolor{currentstroke}%
\pgfsetdash{}{0pt}%
\pgfpathmoveto{\pgfqpoint{0.858841in}{1.246900in}}%
\pgfpathlineto{\pgfqpoint{0.855964in}{1.253406in}}%
\pgfpathlineto{\pgfqpoint{0.853077in}{1.260234in}}%
\pgfpathlineto{\pgfqpoint{0.850181in}{1.267387in}}%
\pgfpathlineto{\pgfqpoint{0.847275in}{1.274872in}}%
\pgfpathlineto{\pgfqpoint{0.858173in}{1.282776in}}%
\pgfpathlineto{\pgfqpoint{0.869522in}{1.290494in}}%
\pgfpathlineto{\pgfqpoint{0.881309in}{1.298020in}}%
\pgfpathlineto{\pgfqpoint{0.893523in}{1.305347in}}%
\pgfpathlineto{\pgfqpoint{0.896149in}{1.297724in}}%
\pgfpathlineto{\pgfqpoint{0.898767in}{1.290430in}}%
\pgfpathlineto{\pgfqpoint{0.901376in}{1.283462in}}%
\pgfpathlineto{\pgfqpoint{0.903977in}{1.276812in}}%
\pgfpathlineto{\pgfqpoint{0.892055in}{1.269620in}}%
\pgfpathlineto{\pgfqpoint{0.880550in}{1.262233in}}%
\pgfpathlineto{\pgfqpoint{0.869475in}{1.254657in}}%
\pgfpathlineto{\pgfqpoint{0.858841in}{1.246900in}}%
\pgfpathclose%
\pgfusepath{fill}%
\end{pgfscope}%
\begin{pgfscope}%
\pgfpathrectangle{\pgfqpoint{0.329460in}{0.284240in}}{\pgfqpoint{1.989680in}{1.989680in}}%
\pgfusepath{clip}%
\pgfsetbuttcap%
\pgfsetroundjoin%
\definecolor{currentfill}{rgb}{0.279566,0.067836,0.391917}%
\pgfsetfillcolor{currentfill}%
\pgfsetlinewidth{0.000000pt}%
\definecolor{currentstroke}{rgb}{0.000000,0.000000,0.000000}%
\pgfsetstrokecolor{currentstroke}%
\pgfsetdash{}{0pt}%
\pgfpathmoveto{\pgfqpoint{1.649492in}{1.243702in}}%
\pgfpathlineto{\pgfqpoint{1.651570in}{1.241511in}}%
\pgfpathlineto{\pgfqpoint{1.653649in}{1.239468in}}%
\pgfpathlineto{\pgfqpoint{1.655729in}{1.237576in}}%
\pgfpathlineto{\pgfqpoint{1.657810in}{1.235838in}}%
\pgfpathlineto{\pgfqpoint{1.668770in}{1.230853in}}%
\pgfpathlineto{\pgfqpoint{1.679439in}{1.225691in}}%
\pgfpathlineto{\pgfqpoint{1.689808in}{1.220356in}}%
\pgfpathlineto{\pgfqpoint{1.699865in}{1.214854in}}%
\pgfpathlineto{\pgfqpoint{1.697490in}{1.216740in}}%
\pgfpathlineto{\pgfqpoint{1.695117in}{1.218781in}}%
\pgfpathlineto{\pgfqpoint{1.692745in}{1.220973in}}%
\pgfpathlineto{\pgfqpoint{1.690375in}{1.223314in}}%
\pgfpathlineto{\pgfqpoint{1.680599in}{1.228660in}}%
\pgfpathlineto{\pgfqpoint{1.670520in}{1.233843in}}%
\pgfpathlineto{\pgfqpoint{1.660148in}{1.238858in}}%
\pgfpathlineto{\pgfqpoint{1.649492in}{1.243702in}}%
\pgfpathclose%
\pgfusepath{fill}%
\end{pgfscope}%
\begin{pgfscope}%
\pgfpathrectangle{\pgfqpoint{0.329460in}{0.284240in}}{\pgfqpoint{1.989680in}{1.989680in}}%
\pgfusepath{clip}%
\pgfsetbuttcap%
\pgfsetroundjoin%
\definecolor{currentfill}{rgb}{0.201239,0.383670,0.554294}%
\pgfsetfillcolor{currentfill}%
\pgfsetlinewidth{0.000000pt}%
\definecolor{currentstroke}{rgb}{0.000000,0.000000,0.000000}%
\pgfsetstrokecolor{currentstroke}%
\pgfsetdash{}{0pt}%
\pgfpathmoveto{\pgfqpoint{0.861206in}{1.424648in}}%
\pgfpathlineto{\pgfqpoint{0.858437in}{1.437091in}}%
\pgfpathlineto{\pgfqpoint{0.855656in}{1.449949in}}%
\pgfpathlineto{\pgfqpoint{0.852860in}{1.463228in}}%
\pgfpathlineto{\pgfqpoint{0.850052in}{1.476935in}}%
\pgfpathlineto{\pgfqpoint{0.863931in}{1.484518in}}%
\pgfpathlineto{\pgfqpoint{0.878244in}{1.491875in}}%
\pgfpathlineto{\pgfqpoint{0.892977in}{1.499002in}}%
\pgfpathlineto{\pgfqpoint{0.908115in}{1.505891in}}%
\pgfpathlineto{\pgfqpoint{0.910586in}{1.492088in}}%
\pgfpathlineto{\pgfqpoint{0.913046in}{1.478711in}}%
\pgfpathlineto{\pgfqpoint{0.915494in}{1.465753in}}%
\pgfpathlineto{\pgfqpoint{0.917930in}{1.453208in}}%
\pgfpathlineto{\pgfqpoint{0.903139in}{1.446412in}}%
\pgfpathlineto{\pgfqpoint{0.888745in}{1.439383in}}%
\pgfpathlineto{\pgfqpoint{0.874763in}{1.432126in}}%
\pgfpathlineto{\pgfqpoint{0.861206in}{1.424648in}}%
\pgfpathclose%
\pgfusepath{fill}%
\end{pgfscope}%
\begin{pgfscope}%
\pgfpathrectangle{\pgfqpoint{0.329460in}{0.284240in}}{\pgfqpoint{1.989680in}{1.989680in}}%
\pgfusepath{clip}%
\pgfsetbuttcap%
\pgfsetroundjoin%
\definecolor{currentfill}{rgb}{0.268510,0.009605,0.335427}%
\pgfsetfillcolor{currentfill}%
\pgfsetlinewidth{0.000000pt}%
\definecolor{currentstroke}{rgb}{0.000000,0.000000,0.000000}%
\pgfsetstrokecolor{currentstroke}%
\pgfsetdash{}{0pt}%
\pgfpathmoveto{\pgfqpoint{1.718920in}{1.205782in}}%
\pgfpathlineto{\pgfqpoint{1.721312in}{1.205456in}}%
\pgfpathlineto{\pgfqpoint{1.723706in}{1.205323in}}%
\pgfpathlineto{\pgfqpoint{1.726103in}{1.205385in}}%
\pgfpathlineto{\pgfqpoint{1.728504in}{1.205649in}}%
\pgfpathlineto{\pgfqpoint{1.739067in}{1.199507in}}%
\pgfpathlineto{\pgfqpoint{1.749275in}{1.193193in}}%
\pgfpathlineto{\pgfqpoint{1.759115in}{1.186712in}}%
\pgfpathlineto{\pgfqpoint{1.768579in}{1.180068in}}%
\pgfpathlineto{\pgfqpoint{1.765913in}{1.179967in}}%
\pgfpathlineto{\pgfqpoint{1.763250in}{1.180067in}}%
\pgfpathlineto{\pgfqpoint{1.760591in}{1.180364in}}%
\pgfpathlineto{\pgfqpoint{1.757935in}{1.180854in}}%
\pgfpathlineto{\pgfqpoint{1.748724in}{1.187328in}}%
\pgfpathlineto{\pgfqpoint{1.739144in}{1.193644in}}%
\pgfpathlineto{\pgfqpoint{1.729206in}{1.199797in}}%
\pgfpathlineto{\pgfqpoint{1.718920in}{1.205782in}}%
\pgfpathclose%
\pgfusepath{fill}%
\end{pgfscope}%
\begin{pgfscope}%
\pgfpathrectangle{\pgfqpoint{0.329460in}{0.284240in}}{\pgfqpoint{1.989680in}{1.989680in}}%
\pgfusepath{clip}%
\pgfsetbuttcap%
\pgfsetroundjoin%
\definecolor{currentfill}{rgb}{0.274128,0.199721,0.498911}%
\pgfsetfillcolor{currentfill}%
\pgfsetlinewidth{0.000000pt}%
\definecolor{currentstroke}{rgb}{0.000000,0.000000,0.000000}%
\pgfsetstrokecolor{currentstroke}%
\pgfsetdash{}{0pt}%
\pgfpathmoveto{\pgfqpoint{1.159779in}{1.321750in}}%
\pgfpathlineto{\pgfqpoint{1.158284in}{1.317876in}}%
\pgfpathlineto{\pgfqpoint{1.156790in}{1.314096in}}%
\pgfpathlineto{\pgfqpoint{1.155296in}{1.310413in}}%
\pgfpathlineto{\pgfqpoint{1.153803in}{1.306831in}}%
\pgfpathlineto{\pgfqpoint{1.165305in}{1.309855in}}%
\pgfpathlineto{\pgfqpoint{1.176977in}{1.312695in}}%
\pgfpathlineto{\pgfqpoint{1.188807in}{1.315349in}}%
\pgfpathlineto{\pgfqpoint{1.200784in}{1.317814in}}%
\pgfpathlineto{\pgfqpoint{1.201919in}{1.321307in}}%
\pgfpathlineto{\pgfqpoint{1.203056in}{1.324900in}}%
\pgfpathlineto{\pgfqpoint{1.204193in}{1.328591in}}%
\pgfpathlineto{\pgfqpoint{1.205330in}{1.332376in}}%
\pgfpathlineto{\pgfqpoint{1.193717in}{1.329991in}}%
\pgfpathlineto{\pgfqpoint{1.182247in}{1.327424in}}%
\pgfpathlineto{\pgfqpoint{1.170931in}{1.324676in}}%
\pgfpathlineto{\pgfqpoint{1.159779in}{1.321750in}}%
\pgfpathclose%
\pgfusepath{fill}%
\end{pgfscope}%
\begin{pgfscope}%
\pgfpathrectangle{\pgfqpoint{0.329460in}{0.284240in}}{\pgfqpoint{1.989680in}{1.989680in}}%
\pgfusepath{clip}%
\pgfsetbuttcap%
\pgfsetroundjoin%
\definecolor{currentfill}{rgb}{0.280255,0.165693,0.476498}%
\pgfsetfillcolor{currentfill}%
\pgfsetlinewidth{0.000000pt}%
\definecolor{currentstroke}{rgb}{0.000000,0.000000,0.000000}%
\pgfsetstrokecolor{currentstroke}%
\pgfsetdash{}{0pt}%
\pgfpathmoveto{\pgfqpoint{1.109694in}{1.292943in}}%
\pgfpathlineto{\pgfqpoint{1.107863in}{1.289352in}}%
\pgfpathlineto{\pgfqpoint{1.106033in}{1.285867in}}%
\pgfpathlineto{\pgfqpoint{1.104203in}{1.282493in}}%
\pgfpathlineto{\pgfqpoint{1.102373in}{1.279232in}}%
\pgfpathlineto{\pgfqpoint{1.113423in}{1.283088in}}%
\pgfpathlineto{\pgfqpoint{1.124690in}{1.286766in}}%
\pgfpathlineto{\pgfqpoint{1.136164in}{1.290261in}}%
\pgfpathlineto{\pgfqpoint{1.147833in}{1.293571in}}%
\pgfpathlineto{\pgfqpoint{1.149325in}{1.296719in}}%
\pgfpathlineto{\pgfqpoint{1.150817in}{1.299981in}}%
\pgfpathlineto{\pgfqpoint{1.152310in}{1.303352in}}%
\pgfpathlineto{\pgfqpoint{1.153803in}{1.306831in}}%
\pgfpathlineto{\pgfqpoint{1.142480in}{1.303625in}}%
\pgfpathlineto{\pgfqpoint{1.131347in}{1.300239in}}%
\pgfpathlineto{\pgfqpoint{1.120415in}{1.296678in}}%
\pgfpathlineto{\pgfqpoint{1.109694in}{1.292943in}}%
\pgfpathclose%
\pgfusepath{fill}%
\end{pgfscope}%
\begin{pgfscope}%
\pgfpathrectangle{\pgfqpoint{0.329460in}{0.284240in}}{\pgfqpoint{1.989680in}{1.989680in}}%
\pgfusepath{clip}%
\pgfsetbuttcap%
\pgfsetroundjoin%
\definecolor{currentfill}{rgb}{0.283072,0.130895,0.449241}%
\pgfsetfillcolor{currentfill}%
\pgfsetlinewidth{0.000000pt}%
\definecolor{currentstroke}{rgb}{0.000000,0.000000,0.000000}%
\pgfsetstrokecolor{currentstroke}%
\pgfsetdash{}{0pt}%
\pgfpathmoveto{\pgfqpoint{1.590191in}{1.282669in}}%
\pgfpathlineto{\pgfqpoint{1.591947in}{1.279552in}}%
\pgfpathlineto{\pgfqpoint{1.593704in}{1.276555in}}%
\pgfpathlineto{\pgfqpoint{1.595460in}{1.273681in}}%
\pgfpathlineto{\pgfqpoint{1.597216in}{1.270934in}}%
\pgfpathlineto{\pgfqpoint{1.608571in}{1.266937in}}%
\pgfpathlineto{\pgfqpoint{1.619691in}{1.262758in}}%
\pgfpathlineto{\pgfqpoint{1.630565in}{1.258401in}}%
\pgfpathlineto{\pgfqpoint{1.641184in}{1.253870in}}%
\pgfpathlineto{\pgfqpoint{1.639108in}{1.256747in}}%
\pgfpathlineto{\pgfqpoint{1.637032in}{1.259750in}}%
\pgfpathlineto{\pgfqpoint{1.634957in}{1.262877in}}%
\pgfpathlineto{\pgfqpoint{1.632881in}{1.266124in}}%
\pgfpathlineto{\pgfqpoint{1.622572in}{1.270517in}}%
\pgfpathlineto{\pgfqpoint{1.612013in}{1.274741in}}%
\pgfpathlineto{\pgfqpoint{1.601216in}{1.278793in}}%
\pgfpathlineto{\pgfqpoint{1.590191in}{1.282669in}}%
\pgfpathclose%
\pgfusepath{fill}%
\end{pgfscope}%
\begin{pgfscope}%
\pgfpathrectangle{\pgfqpoint{0.329460in}{0.284240in}}{\pgfqpoint{1.989680in}{1.989680in}}%
\pgfusepath{clip}%
\pgfsetbuttcap%
\pgfsetroundjoin%
\definecolor{currentfill}{rgb}{0.267004,0.004874,0.329415}%
\pgfsetfillcolor{currentfill}%
\pgfsetlinewidth{0.000000pt}%
\definecolor{currentstroke}{rgb}{0.000000,0.000000,0.000000}%
\pgfsetstrokecolor{currentstroke}%
\pgfsetdash{}{0pt}%
\pgfpathmoveto{\pgfqpoint{0.925709in}{1.174031in}}%
\pgfpathlineto{\pgfqpoint{0.922985in}{1.174298in}}%
\pgfpathlineto{\pgfqpoint{0.920258in}{1.174775in}}%
\pgfpathlineto{\pgfqpoint{0.917526in}{1.175464in}}%
\pgfpathlineto{\pgfqpoint{0.914790in}{1.176371in}}%
\pgfpathlineto{\pgfqpoint{0.924156in}{1.183325in}}%
\pgfpathlineto{\pgfqpoint{0.933916in}{1.190118in}}%
\pgfpathlineto{\pgfqpoint{0.944062in}{1.196744in}}%
\pgfpathlineto{\pgfqpoint{0.954582in}{1.203197in}}%
\pgfpathlineto{\pgfqpoint{0.957058in}{1.202127in}}%
\pgfpathlineto{\pgfqpoint{0.959530in}{1.201273in}}%
\pgfpathlineto{\pgfqpoint{0.961999in}{1.200631in}}%
\pgfpathlineto{\pgfqpoint{0.964464in}{1.200198in}}%
\pgfpathlineto{\pgfqpoint{0.954217in}{1.193903in}}%
\pgfpathlineto{\pgfqpoint{0.944335in}{1.187440in}}%
\pgfpathlineto{\pgfqpoint{0.934829in}{1.180814in}}%
\pgfpathlineto{\pgfqpoint{0.925709in}{1.174031in}}%
\pgfpathclose%
\pgfusepath{fill}%
\end{pgfscope}%
\begin{pgfscope}%
\pgfpathrectangle{\pgfqpoint{0.329460in}{0.284240in}}{\pgfqpoint{1.989680in}{1.989680in}}%
\pgfusepath{clip}%
\pgfsetbuttcap%
\pgfsetroundjoin%
\definecolor{currentfill}{rgb}{0.263663,0.237631,0.518762}%
\pgfsetfillcolor{currentfill}%
\pgfsetlinewidth{0.000000pt}%
\definecolor{currentstroke}{rgb}{0.000000,0.000000,0.000000}%
\pgfsetstrokecolor{currentstroke}%
\pgfsetdash{}{0pt}%
\pgfpathmoveto{\pgfqpoint{1.447619in}{1.355647in}}%
\pgfpathlineto{\pgfqpoint{1.448396in}{1.351580in}}%
\pgfpathlineto{\pgfqpoint{1.449173in}{1.347594in}}%
\pgfpathlineto{\pgfqpoint{1.449949in}{1.343694in}}%
\pgfpathlineto{\pgfqpoint{1.450724in}{1.339881in}}%
\pgfpathlineto{\pgfqpoint{1.462791in}{1.338223in}}%
\pgfpathlineto{\pgfqpoint{1.474757in}{1.336376in}}%
\pgfpathlineto{\pgfqpoint{1.486611in}{1.334341in}}%
\pgfpathlineto{\pgfqpoint{1.485555in}{1.338200in}}%
\pgfpathlineto{\pgfqpoint{1.484498in}{1.342148in}}%
\pgfpathlineto{\pgfqpoint{1.483441in}{1.346180in}}%
\pgfpathlineto{\pgfqpoint{1.482383in}{1.350294in}}%
\pgfpathlineto{\pgfqpoint{1.470901in}{1.352260in}}%
\pgfpathlineto{\pgfqpoint{1.459309in}{1.354045in}}%
\pgfpathlineto{\pgfqpoint{1.447619in}{1.355647in}}%
\pgfpathclose%
\pgfusepath{fill}%
\end{pgfscope}%
\begin{pgfscope}%
\pgfpathrectangle{\pgfqpoint{0.329460in}{0.284240in}}{\pgfqpoint{1.989680in}{1.989680in}}%
\pgfusepath{clip}%
\pgfsetbuttcap%
\pgfsetroundjoin%
\definecolor{currentfill}{rgb}{0.279566,0.067836,0.391917}%
\pgfsetfillcolor{currentfill}%
\pgfsetlinewidth{0.000000pt}%
\definecolor{currentstroke}{rgb}{0.000000,0.000000,0.000000}%
\pgfsetstrokecolor{currentstroke}%
\pgfsetdash{}{0pt}%
\pgfpathmoveto{\pgfqpoint{1.003574in}{1.218429in}}%
\pgfpathlineto{\pgfqpoint{1.001143in}{1.216052in}}%
\pgfpathlineto{\pgfqpoint{0.998710in}{1.213824in}}%
\pgfpathlineto{\pgfqpoint{0.996276in}{1.211747in}}%
\pgfpathlineto{\pgfqpoint{0.993841in}{1.209825in}}%
\pgfpathlineto{\pgfqpoint{1.003612in}{1.215473in}}%
\pgfpathlineto{\pgfqpoint{1.013704in}{1.220957in}}%
\pgfpathlineto{\pgfqpoint{1.024106in}{1.226273in}}%
\pgfpathlineto{\pgfqpoint{1.034809in}{1.231416in}}%
\pgfpathlineto{\pgfqpoint{1.036957in}{1.233184in}}%
\pgfpathlineto{\pgfqpoint{1.039105in}{1.235108in}}%
\pgfpathlineto{\pgfqpoint{1.041252in}{1.237183in}}%
\pgfpathlineto{\pgfqpoint{1.043398in}{1.239405in}}%
\pgfpathlineto{\pgfqpoint{1.032993in}{1.234408in}}%
\pgfpathlineto{\pgfqpoint{1.022881in}{1.229244in}}%
\pgfpathlineto{\pgfqpoint{1.013071in}{1.223916in}}%
\pgfpathlineto{\pgfqpoint{1.003574in}{1.218429in}}%
\pgfpathclose%
\pgfusepath{fill}%
\end{pgfscope}%
\begin{pgfscope}%
\pgfpathrectangle{\pgfqpoint{0.329460in}{0.284240in}}{\pgfqpoint{1.989680in}{1.989680in}}%
\pgfusepath{clip}%
\pgfsetbuttcap%
\pgfsetroundjoin%
\definecolor{currentfill}{rgb}{0.172719,0.448791,0.557885}%
\pgfsetfillcolor{currentfill}%
\pgfsetlinewidth{0.000000pt}%
\definecolor{currentstroke}{rgb}{0.000000,0.000000,0.000000}%
\pgfsetstrokecolor{currentstroke}%
\pgfsetdash{}{0pt}%
\pgfpathmoveto{\pgfqpoint{1.780476in}{1.511812in}}%
\pgfpathlineto{\pgfqpoint{1.782880in}{1.526068in}}%
\pgfpathlineto{\pgfqpoint{1.785296in}{1.540764in}}%
\pgfpathlineto{\pgfqpoint{1.787725in}{1.555906in}}%
\pgfpathlineto{\pgfqpoint{1.803481in}{1.549168in}}%
\pgfpathlineto{\pgfqpoint{1.818839in}{1.542187in}}%
\pgfpathlineto{\pgfqpoint{1.833785in}{1.534966in}}%
\pgfpathlineto{\pgfqpoint{1.848302in}{1.527513in}}%
\pgfpathlineto{\pgfqpoint{1.845523in}{1.512458in}}%
\pgfpathlineto{\pgfqpoint{1.842759in}{1.497851in}}%
\pgfpathlineto{\pgfqpoint{1.840008in}{1.483686in}}%
\pgfpathlineto{\pgfqpoint{1.825743in}{1.491069in}}%
\pgfpathlineto{\pgfqpoint{1.811056in}{1.498221in}}%
\pgfpathlineto{\pgfqpoint{1.795962in}{1.505138in}}%
\pgfpathlineto{\pgfqpoint{1.780476in}{1.511812in}}%
\pgfpathclose%
\pgfusepath{fill}%
\end{pgfscope}%
\begin{pgfscope}%
\pgfpathrectangle{\pgfqpoint{0.329460in}{0.284240in}}{\pgfqpoint{1.989680in}{1.989680in}}%
\pgfusepath{clip}%
\pgfsetbuttcap%
\pgfsetroundjoin%
\definecolor{currentfill}{rgb}{0.263663,0.237631,0.518762}%
\pgfsetfillcolor{currentfill}%
\pgfsetlinewidth{0.000000pt}%
\definecolor{currentstroke}{rgb}{0.000000,0.000000,0.000000}%
\pgfsetstrokecolor{currentstroke}%
\pgfsetdash{}{0pt}%
\pgfpathmoveto{\pgfqpoint{1.209885in}{1.348395in}}%
\pgfpathlineto{\pgfqpoint{1.208745in}{1.344265in}}%
\pgfpathlineto{\pgfqpoint{1.207606in}{1.340216in}}%
\pgfpathlineto{\pgfqpoint{1.206468in}{1.336252in}}%
\pgfpathlineto{\pgfqpoint{1.205330in}{1.332376in}}%
\pgfpathlineto{\pgfqpoint{1.217075in}{1.334576in}}%
\pgfpathlineto{\pgfqpoint{1.228942in}{1.336590in}}%
\pgfpathlineto{\pgfqpoint{1.240920in}{1.338416in}}%
\pgfpathlineto{\pgfqpoint{1.252997in}{1.340053in}}%
\pgfpathlineto{\pgfqpoint{1.253762in}{1.343865in}}%
\pgfpathlineto{\pgfqpoint{1.254528in}{1.347764in}}%
\pgfpathlineto{\pgfqpoint{1.255294in}{1.351748in}}%
\pgfpathlineto{\pgfqpoint{1.256060in}{1.355814in}}%
\pgfpathlineto{\pgfqpoint{1.244360in}{1.354232in}}%
\pgfpathlineto{\pgfqpoint{1.232757in}{1.352468in}}%
\pgfpathlineto{\pgfqpoint{1.221262in}{1.350522in}}%
\pgfpathlineto{\pgfqpoint{1.209885in}{1.348395in}}%
\pgfpathclose%
\pgfusepath{fill}%
\end{pgfscope}%
\begin{pgfscope}%
\pgfpathrectangle{\pgfqpoint{0.329460in}{0.284240in}}{\pgfqpoint{1.989680in}{1.989680in}}%
\pgfusepath{clip}%
\pgfsetbuttcap%
\pgfsetroundjoin%
\definecolor{currentfill}{rgb}{0.276194,0.190074,0.493001}%
\pgfsetfillcolor{currentfill}%
\pgfsetlinewidth{0.000000pt}%
\definecolor{currentstroke}{rgb}{0.000000,0.000000,0.000000}%
\pgfsetstrokecolor{currentstroke}%
\pgfsetdash{}{0pt}%
\pgfpathmoveto{\pgfqpoint{0.847275in}{1.274872in}}%
\pgfpathlineto{\pgfqpoint{0.844359in}{1.282694in}}%
\pgfpathlineto{\pgfqpoint{0.841431in}{1.290858in}}%
\pgfpathlineto{\pgfqpoint{0.838493in}{1.299371in}}%
\pgfpathlineto{\pgfqpoint{0.835544in}{1.308237in}}%
\pgfpathlineto{\pgfqpoint{0.846711in}{1.316282in}}%
\pgfpathlineto{\pgfqpoint{0.858338in}{1.324137in}}%
\pgfpathlineto{\pgfqpoint{0.870412in}{1.331796in}}%
\pgfpathlineto{\pgfqpoint{0.882921in}{1.339252in}}%
\pgfpathlineto{\pgfqpoint{0.885586in}{1.330253in}}%
\pgfpathlineto{\pgfqpoint{0.888241in}{1.321606in}}%
\pgfpathlineto{\pgfqpoint{0.890887in}{1.313306in}}%
\pgfpathlineto{\pgfqpoint{0.893523in}{1.305347in}}%
\pgfpathlineto{\pgfqpoint{0.881309in}{1.298020in}}%
\pgfpathlineto{\pgfqpoint{0.869522in}{1.290494in}}%
\pgfpathlineto{\pgfqpoint{0.858173in}{1.282776in}}%
\pgfpathlineto{\pgfqpoint{0.847275in}{1.274872in}}%
\pgfpathclose%
\pgfusepath{fill}%
\end{pgfscope}%
\begin{pgfscope}%
\pgfpathrectangle{\pgfqpoint{0.329460in}{0.284240in}}{\pgfqpoint{1.989680in}{1.989680in}}%
\pgfusepath{clip}%
\pgfsetbuttcap%
\pgfsetroundjoin%
\definecolor{currentfill}{rgb}{0.260571,0.246922,0.522828}%
\pgfsetfillcolor{currentfill}%
\pgfsetlinewidth{0.000000pt}%
\definecolor{currentstroke}{rgb}{0.000000,0.000000,0.000000}%
\pgfsetstrokecolor{currentstroke}%
\pgfsetdash{}{0pt}%
\pgfpathmoveto{\pgfqpoint{1.807980in}{1.345704in}}%
\pgfpathlineto{\pgfqpoint{1.810587in}{1.355089in}}%
\pgfpathlineto{\pgfqpoint{1.813204in}{1.364837in}}%
\pgfpathlineto{\pgfqpoint{1.815832in}{1.374955in}}%
\pgfpathlineto{\pgfqpoint{1.818471in}{1.385449in}}%
\pgfpathlineto{\pgfqpoint{1.831665in}{1.378058in}}%
\pgfpathlineto{\pgfqpoint{1.844426in}{1.370456in}}%
\pgfpathlineto{\pgfqpoint{1.856743in}{1.362649in}}%
\pgfpathlineto{\pgfqpoint{1.868601in}{1.354642in}}%
\pgfpathlineto{\pgfqpoint{1.865665in}{1.344271in}}%
\pgfpathlineto{\pgfqpoint{1.862741in}{1.334277in}}%
\pgfpathlineto{\pgfqpoint{1.859828in}{1.324654in}}%
\pgfpathlineto{\pgfqpoint{1.856928in}{1.315397in}}%
\pgfpathlineto{\pgfqpoint{1.845351in}{1.323274in}}%
\pgfpathlineto{\pgfqpoint{1.833326in}{1.330955in}}%
\pgfpathlineto{\pgfqpoint{1.820865in}{1.338433in}}%
\pgfpathlineto{\pgfqpoint{1.807980in}{1.345704in}}%
\pgfpathclose%
\pgfusepath{fill}%
\end{pgfscope}%
\begin{pgfscope}%
\pgfpathrectangle{\pgfqpoint{0.329460in}{0.284240in}}{\pgfqpoint{1.989680in}{1.989680in}}%
\pgfusepath{clip}%
\pgfsetbuttcap%
\pgfsetroundjoin%
\definecolor{currentfill}{rgb}{0.283072,0.130895,0.449241}%
\pgfsetfillcolor{currentfill}%
\pgfsetlinewidth{0.000000pt}%
\definecolor{currentstroke}{rgb}{0.000000,0.000000,0.000000}%
\pgfsetstrokecolor{currentstroke}%
\pgfsetdash{}{0pt}%
\pgfpathmoveto{\pgfqpoint{1.060548in}{1.262081in}}%
\pgfpathlineto{\pgfqpoint{1.058405in}{1.258802in}}%
\pgfpathlineto{\pgfqpoint{1.056262in}{1.255643in}}%
\pgfpathlineto{\pgfqpoint{1.054118in}{1.252608in}}%
\pgfpathlineto{\pgfqpoint{1.051975in}{1.249700in}}%
\pgfpathlineto{\pgfqpoint{1.062358in}{1.254382in}}%
\pgfpathlineto{\pgfqpoint{1.073006in}{1.258894in}}%
\pgfpathlineto{\pgfqpoint{1.083908in}{1.263231in}}%
\pgfpathlineto{\pgfqpoint{1.095055in}{1.267390in}}%
\pgfpathlineto{\pgfqpoint{1.096884in}{1.270164in}}%
\pgfpathlineto{\pgfqpoint{1.098714in}{1.273064in}}%
\pgfpathlineto{\pgfqpoint{1.100543in}{1.276088in}}%
\pgfpathlineto{\pgfqpoint{1.102373in}{1.279232in}}%
\pgfpathlineto{\pgfqpoint{1.091550in}{1.275200in}}%
\pgfpathlineto{\pgfqpoint{1.080964in}{1.270995in}}%
\pgfpathlineto{\pgfqpoint{1.070627in}{1.266621in}}%
\pgfpathlineto{\pgfqpoint{1.060548in}{1.262081in}}%
\pgfpathclose%
\pgfusepath{fill}%
\end{pgfscope}%
\begin{pgfscope}%
\pgfpathrectangle{\pgfqpoint{0.329460in}{0.284240in}}{\pgfqpoint{1.989680in}{1.989680in}}%
\pgfusepath{clip}%
\pgfsetbuttcap%
\pgfsetroundjoin%
\definecolor{currentfill}{rgb}{0.268510,0.009605,0.335427}%
\pgfsetfillcolor{currentfill}%
\pgfsetlinewidth{0.000000pt}%
\definecolor{currentstroke}{rgb}{0.000000,0.000000,0.000000}%
\pgfsetstrokecolor{currentstroke}%
\pgfsetdash{}{0pt}%
\pgfpathmoveto{\pgfqpoint{0.936569in}{1.174972in}}%
\pgfpathlineto{\pgfqpoint{0.933859in}{1.174443in}}%
\pgfpathlineto{\pgfqpoint{0.931146in}{1.174107in}}%
\pgfpathlineto{\pgfqpoint{0.928429in}{1.173969in}}%
\pgfpathlineto{\pgfqpoint{0.925709in}{1.174031in}}%
\pgfpathlineto{\pgfqpoint{0.934829in}{1.180814in}}%
\pgfpathlineto{\pgfqpoint{0.944335in}{1.187440in}}%
\pgfpathlineto{\pgfqpoint{0.954217in}{1.193903in}}%
\pgfpathlineto{\pgfqpoint{0.964464in}{1.200198in}}%
\pgfpathlineto{\pgfqpoint{0.966926in}{1.199970in}}%
\pgfpathlineto{\pgfqpoint{0.969385in}{1.199942in}}%
\pgfpathlineto{\pgfqpoint{0.971841in}{1.200110in}}%
\pgfpathlineto{\pgfqpoint{0.974295in}{1.200470in}}%
\pgfpathlineto{\pgfqpoint{0.964318in}{1.194336in}}%
\pgfpathlineto{\pgfqpoint{0.954698in}{1.188037in}}%
\pgfpathlineto{\pgfqpoint{0.945445in}{1.181581in}}%
\pgfpathlineto{\pgfqpoint{0.936569in}{1.174972in}}%
\pgfpathclose%
\pgfusepath{fill}%
\end{pgfscope}%
\begin{pgfscope}%
\pgfpathrectangle{\pgfqpoint{0.329460in}{0.284240in}}{\pgfqpoint{1.989680in}{1.989680in}}%
\pgfusepath{clip}%
\pgfsetbuttcap%
\pgfsetroundjoin%
\definecolor{currentfill}{rgb}{0.271305,0.019942,0.347269}%
\pgfsetfillcolor{currentfill}%
\pgfsetlinewidth{0.000000pt}%
\definecolor{currentstroke}{rgb}{0.000000,0.000000,0.000000}%
\pgfsetstrokecolor{currentstroke}%
\pgfsetdash{}{0pt}%
\pgfpathmoveto{\pgfqpoint{1.709377in}{1.208931in}}%
\pgfpathlineto{\pgfqpoint{1.711760in}{1.207874in}}%
\pgfpathlineto{\pgfqpoint{1.714144in}{1.206995in}}%
\pgfpathlineto{\pgfqpoint{1.716531in}{1.206296in}}%
\pgfpathlineto{\pgfqpoint{1.718920in}{1.205782in}}%
\pgfpathlineto{\pgfqpoint{1.729206in}{1.199797in}}%
\pgfpathlineto{\pgfqpoint{1.739144in}{1.193644in}}%
\pgfpathlineto{\pgfqpoint{1.748724in}{1.187328in}}%
\pgfpathlineto{\pgfqpoint{1.757935in}{1.180854in}}%
\pgfpathlineto{\pgfqpoint{1.755282in}{1.181533in}}%
\pgfpathlineto{\pgfqpoint{1.752632in}{1.182397in}}%
\pgfpathlineto{\pgfqpoint{1.749984in}{1.183442in}}%
\pgfpathlineto{\pgfqpoint{1.747339in}{1.184665in}}%
\pgfpathlineto{\pgfqpoint{1.738378in}{1.190966in}}%
\pgfpathlineto{\pgfqpoint{1.729057in}{1.197115in}}%
\pgfpathlineto{\pgfqpoint{1.719387in}{1.203105in}}%
\pgfpathlineto{\pgfqpoint{1.709377in}{1.208931in}}%
\pgfpathclose%
\pgfusepath{fill}%
\end{pgfscope}%
\begin{pgfscope}%
\pgfpathrectangle{\pgfqpoint{0.329460in}{0.284240in}}{\pgfqpoint{1.989680in}{1.989680in}}%
\pgfusepath{clip}%
\pgfsetbuttcap%
\pgfsetroundjoin%
\definecolor{currentfill}{rgb}{0.282327,0.094955,0.417331}%
\pgfsetfillcolor{currentfill}%
\pgfsetlinewidth{0.000000pt}%
\definecolor{currentstroke}{rgb}{0.000000,0.000000,0.000000}%
\pgfsetstrokecolor{currentstroke}%
\pgfsetdash{}{0pt}%
\pgfpathmoveto{\pgfqpoint{1.641184in}{1.253870in}}%
\pgfpathlineto{\pgfqpoint{1.643260in}{1.251124in}}%
\pgfpathlineto{\pgfqpoint{1.645337in}{1.248512in}}%
\pgfpathlineto{\pgfqpoint{1.647414in}{1.246037in}}%
\pgfpathlineto{\pgfqpoint{1.649492in}{1.243702in}}%
\pgfpathlineto{\pgfqpoint{1.660148in}{1.238858in}}%
\pgfpathlineto{\pgfqpoint{1.670520in}{1.233843in}}%
\pgfpathlineto{\pgfqpoint{1.680599in}{1.228660in}}%
\pgfpathlineto{\pgfqpoint{1.690375in}{1.223314in}}%
\pgfpathlineto{\pgfqpoint{1.688005in}{1.225799in}}%
\pgfpathlineto{\pgfqpoint{1.685635in}{1.228424in}}%
\pgfpathlineto{\pgfqpoint{1.683267in}{1.231187in}}%
\pgfpathlineto{\pgfqpoint{1.680899in}{1.234084in}}%
\pgfpathlineto{\pgfqpoint{1.671404in}{1.239272in}}%
\pgfpathlineto{\pgfqpoint{1.661614in}{1.244302in}}%
\pgfpathlineto{\pgfqpoint{1.651537in}{1.249169in}}%
\pgfpathlineto{\pgfqpoint{1.641184in}{1.253870in}}%
\pgfpathclose%
\pgfusepath{fill}%
\end{pgfscope}%
\begin{pgfscope}%
\pgfpathrectangle{\pgfqpoint{0.329460in}{0.284240in}}{\pgfqpoint{1.989680in}{1.989680in}}%
\pgfusepath{clip}%
\pgfsetbuttcap%
\pgfsetroundjoin%
\definecolor{currentfill}{rgb}{0.248629,0.278775,0.534556}%
\pgfsetfillcolor{currentfill}%
\pgfsetlinewidth{0.000000pt}%
\definecolor{currentstroke}{rgb}{0.000000,0.000000,0.000000}%
\pgfsetstrokecolor{currentstroke}%
\pgfsetdash{}{0pt}%
\pgfpathmoveto{\pgfqpoint{1.305150in}{1.377153in}}%
\pgfpathlineto{\pgfqpoint{1.304766in}{1.372829in}}%
\pgfpathlineto{\pgfqpoint{1.304382in}{1.368575in}}%
\pgfpathlineto{\pgfqpoint{1.303999in}{1.364393in}}%
\pgfpathlineto{\pgfqpoint{1.303615in}{1.360287in}}%
\pgfpathlineto{\pgfqpoint{1.315639in}{1.360938in}}%
\pgfpathlineto{\pgfqpoint{1.327695in}{1.361400in}}%
\pgfpathlineto{\pgfqpoint{1.339772in}{1.361673in}}%
\pgfpathlineto{\pgfqpoint{1.351859in}{1.361757in}}%
\pgfpathlineto{\pgfqpoint{1.351854in}{1.365850in}}%
\pgfpathlineto{\pgfqpoint{1.351848in}{1.370018in}}%
\pgfpathlineto{\pgfqpoint{1.351843in}{1.374260in}}%
\pgfpathlineto{\pgfqpoint{1.351838in}{1.378572in}}%
\pgfpathlineto{\pgfqpoint{1.340140in}{1.378490in}}%
\pgfpathlineto{\pgfqpoint{1.328452in}{1.378227in}}%
\pgfpathlineto{\pgfqpoint{1.316785in}{1.377781in}}%
\pgfpathlineto{\pgfqpoint{1.305150in}{1.377153in}}%
\pgfpathclose%
\pgfusepath{fill}%
\end{pgfscope}%
\begin{pgfscope}%
\pgfpathrectangle{\pgfqpoint{0.329460in}{0.284240in}}{\pgfqpoint{1.989680in}{1.989680in}}%
\pgfusepath{clip}%
\pgfsetbuttcap%
\pgfsetroundjoin%
\definecolor{currentfill}{rgb}{0.248629,0.278775,0.534556}%
\pgfsetfillcolor{currentfill}%
\pgfsetlinewidth{0.000000pt}%
\definecolor{currentstroke}{rgb}{0.000000,0.000000,0.000000}%
\pgfsetstrokecolor{currentstroke}%
\pgfsetdash{}{0pt}%
\pgfpathmoveto{\pgfqpoint{1.351838in}{1.378572in}}%
\pgfpathlineto{\pgfqpoint{1.351843in}{1.374260in}}%
\pgfpathlineto{\pgfqpoint{1.351848in}{1.370018in}}%
\pgfpathlineto{\pgfqpoint{1.351854in}{1.365850in}}%
\pgfpathlineto{\pgfqpoint{1.351859in}{1.361757in}}%
\pgfpathlineto{\pgfqpoint{1.363946in}{1.361652in}}%
\pgfpathlineto{\pgfqpoint{1.376021in}{1.361358in}}%
\pgfpathlineto{\pgfqpoint{1.388074in}{1.360875in}}%
\pgfpathlineto{\pgfqpoint{1.400093in}{1.360203in}}%
\pgfpathlineto{\pgfqpoint{1.399699in}{1.364310in}}%
\pgfpathlineto{\pgfqpoint{1.399305in}{1.368492in}}%
\pgfpathlineto{\pgfqpoint{1.398911in}{1.372747in}}%
\pgfpathlineto{\pgfqpoint{1.398516in}{1.377072in}}%
\pgfpathlineto{\pgfqpoint{1.386884in}{1.377720in}}%
\pgfpathlineto{\pgfqpoint{1.375220in}{1.378186in}}%
\pgfpathlineto{\pgfqpoint{1.363535in}{1.378470in}}%
\pgfpathlineto{\pgfqpoint{1.351838in}{1.378572in}}%
\pgfpathclose%
\pgfusepath{fill}%
\end{pgfscope}%
\begin{pgfscope}%
\pgfpathrectangle{\pgfqpoint{0.329460in}{0.284240in}}{\pgfqpoint{1.989680in}{1.989680in}}%
\pgfusepath{clip}%
\pgfsetbuttcap%
\pgfsetroundjoin%
\definecolor{currentfill}{rgb}{0.274128,0.199721,0.498911}%
\pgfsetfillcolor{currentfill}%
\pgfsetlinewidth{0.000000pt}%
\definecolor{currentstroke}{rgb}{0.000000,0.000000,0.000000}%
\pgfsetstrokecolor{currentstroke}%
\pgfsetdash{}{0pt}%
\pgfpathmoveto{\pgfqpoint{1.532691in}{1.324360in}}%
\pgfpathlineto{\pgfqpoint{1.534109in}{1.320507in}}%
\pgfpathlineto{\pgfqpoint{1.535525in}{1.316749in}}%
\pgfpathlineto{\pgfqpoint{1.536941in}{1.313088in}}%
\pgfpathlineto{\pgfqpoint{1.538356in}{1.309528in}}%
\pgfpathlineto{\pgfqpoint{1.549840in}{1.306483in}}%
\pgfpathlineto{\pgfqpoint{1.561142in}{1.303257in}}%
\pgfpathlineto{\pgfqpoint{1.572253in}{1.299852in}}%
\pgfpathlineto{\pgfqpoint{1.583162in}{1.296272in}}%
\pgfpathlineto{\pgfqpoint{1.581404in}{1.299940in}}%
\pgfpathlineto{\pgfqpoint{1.579646in}{1.303709in}}%
\pgfpathlineto{\pgfqpoint{1.577887in}{1.307575in}}%
\pgfpathlineto{\pgfqpoint{1.576127in}{1.311536in}}%
\pgfpathlineto{\pgfqpoint{1.565552in}{1.315000in}}%
\pgfpathlineto{\pgfqpoint{1.554781in}{1.318294in}}%
\pgfpathlineto{\pgfqpoint{1.543824in}{1.321414in}}%
\pgfpathlineto{\pgfqpoint{1.532691in}{1.324360in}}%
\pgfpathclose%
\pgfusepath{fill}%
\end{pgfscope}%
\begin{pgfscope}%
\pgfpathrectangle{\pgfqpoint{0.329460in}{0.284240in}}{\pgfqpoint{1.989680in}{1.989680in}}%
\pgfusepath{clip}%
\pgfsetbuttcap%
\pgfsetroundjoin%
\definecolor{currentfill}{rgb}{0.172719,0.448791,0.557885}%
\pgfsetfillcolor{currentfill}%
\pgfsetlinewidth{0.000000pt}%
\definecolor{currentstroke}{rgb}{0.000000,0.000000,0.000000}%
\pgfsetstrokecolor{currentstroke}%
\pgfsetdash{}{0pt}%
\pgfpathmoveto{\pgfqpoint{0.850052in}{1.476935in}}%
\pgfpathlineto{\pgfqpoint{0.847229in}{1.491078in}}%
\pgfpathlineto{\pgfqpoint{0.844391in}{1.505663in}}%
\pgfpathlineto{\pgfqpoint{0.841539in}{1.520697in}}%
\pgfpathlineto{\pgfqpoint{0.855664in}{1.528353in}}%
\pgfpathlineto{\pgfqpoint{0.870230in}{1.535780in}}%
\pgfpathlineto{\pgfqpoint{0.885222in}{1.542974in}}%
\pgfpathlineto{\pgfqpoint{0.900625in}{1.549929in}}%
\pgfpathlineto{\pgfqpoint{0.903135in}{1.534805in}}%
\pgfpathlineto{\pgfqpoint{0.905631in}{1.520128in}}%
\pgfpathlineto{\pgfqpoint{0.908115in}{1.505891in}}%
\pgfpathlineto{\pgfqpoint{0.892977in}{1.499002in}}%
\pgfpathlineto{\pgfqpoint{0.878244in}{1.491875in}}%
\pgfpathlineto{\pgfqpoint{0.863931in}{1.484518in}}%
\pgfpathlineto{\pgfqpoint{0.850052in}{1.476935in}}%
\pgfpathclose%
\pgfusepath{fill}%
\end{pgfscope}%
\begin{pgfscope}%
\pgfpathrectangle{\pgfqpoint{0.329460in}{0.284240in}}{\pgfqpoint{1.989680in}{1.989680in}}%
\pgfusepath{clip}%
\pgfsetbuttcap%
\pgfsetroundjoin%
\definecolor{currentfill}{rgb}{0.263663,0.237631,0.518762}%
\pgfsetfillcolor{currentfill}%
\pgfsetlinewidth{0.000000pt}%
\definecolor{currentstroke}{rgb}{0.000000,0.000000,0.000000}%
\pgfsetstrokecolor{currentstroke}%
\pgfsetdash{}{0pt}%
\pgfpathmoveto{\pgfqpoint{1.482383in}{1.350294in}}%
\pgfpathlineto{\pgfqpoint{1.483441in}{1.346180in}}%
\pgfpathlineto{\pgfqpoint{1.484498in}{1.342148in}}%
\pgfpathlineto{\pgfqpoint{1.485555in}{1.338200in}}%
\pgfpathlineto{\pgfqpoint{1.486611in}{1.334341in}}%
\pgfpathlineto{\pgfqpoint{1.498342in}{1.332120in}}%
\pgfpathlineto{\pgfqpoint{1.509940in}{1.329715in}}%
\pgfpathlineto{\pgfqpoint{1.521393in}{1.327127in}}%
\pgfpathlineto{\pgfqpoint{1.532691in}{1.324360in}}%
\pgfpathlineto{\pgfqpoint{1.531274in}{1.328303in}}%
\pgfpathlineto{\pgfqpoint{1.529855in}{1.332335in}}%
\pgfpathlineto{\pgfqpoint{1.528436in}{1.336452in}}%
\pgfpathlineto{\pgfqpoint{1.527016in}{1.340650in}}%
\pgfpathlineto{\pgfqpoint{1.516074in}{1.343324in}}%
\pgfpathlineto{\pgfqpoint{1.504980in}{1.345824in}}%
\pgfpathlineto{\pgfqpoint{1.493747in}{1.348148in}}%
\pgfpathlineto{\pgfqpoint{1.482383in}{1.350294in}}%
\pgfpathclose%
\pgfusepath{fill}%
\end{pgfscope}%
\begin{pgfscope}%
\pgfpathrectangle{\pgfqpoint{0.329460in}{0.284240in}}{\pgfqpoint{1.989680in}{1.989680in}}%
\pgfusepath{clip}%
\pgfsetbuttcap%
\pgfsetroundjoin%
\definecolor{currentfill}{rgb}{0.260571,0.246922,0.522828}%
\pgfsetfillcolor{currentfill}%
\pgfsetlinewidth{0.000000pt}%
\definecolor{currentstroke}{rgb}{0.000000,0.000000,0.000000}%
\pgfsetstrokecolor{currentstroke}%
\pgfsetdash{}{0pt}%
\pgfpathmoveto{\pgfqpoint{0.835544in}{1.308237in}}%
\pgfpathlineto{\pgfqpoint{0.832583in}{1.317464in}}%
\pgfpathlineto{\pgfqpoint{0.829610in}{1.327056in}}%
\pgfpathlineto{\pgfqpoint{0.826625in}{1.337021in}}%
\pgfpathlineto{\pgfqpoint{0.823628in}{1.347363in}}%
\pgfpathlineto{\pgfqpoint{0.835068in}{1.355541in}}%
\pgfpathlineto{\pgfqpoint{0.846978in}{1.363526in}}%
\pgfpathlineto{\pgfqpoint{0.859345in}{1.371311in}}%
\pgfpathlineto{\pgfqpoint{0.872155in}{1.378890in}}%
\pgfpathlineto{\pgfqpoint{0.874863in}{1.368422in}}%
\pgfpathlineto{\pgfqpoint{0.877560in}{1.358331in}}%
\pgfpathlineto{\pgfqpoint{0.880246in}{1.348609in}}%
\pgfpathlineto{\pgfqpoint{0.882921in}{1.339252in}}%
\pgfpathlineto{\pgfqpoint{0.870412in}{1.331796in}}%
\pgfpathlineto{\pgfqpoint{0.858338in}{1.324137in}}%
\pgfpathlineto{\pgfqpoint{0.846711in}{1.316282in}}%
\pgfpathlineto{\pgfqpoint{0.835544in}{1.308237in}}%
\pgfpathclose%
\pgfusepath{fill}%
\end{pgfscope}%
\begin{pgfscope}%
\pgfpathrectangle{\pgfqpoint{0.329460in}{0.284240in}}{\pgfqpoint{1.989680in}{1.989680in}}%
\pgfusepath{clip}%
\pgfsetbuttcap%
\pgfsetroundjoin%
\definecolor{currentfill}{rgb}{0.271305,0.019942,0.347269}%
\pgfsetfillcolor{currentfill}%
\pgfsetlinewidth{0.000000pt}%
\definecolor{currentstroke}{rgb}{0.000000,0.000000,0.000000}%
\pgfsetstrokecolor{currentstroke}%
\pgfsetdash{}{0pt}%
\pgfpathmoveto{\pgfqpoint{0.947380in}{1.178940in}}%
\pgfpathlineto{\pgfqpoint{0.944681in}{1.177677in}}%
\pgfpathlineto{\pgfqpoint{0.941980in}{1.176593in}}%
\pgfpathlineto{\pgfqpoint{0.939276in}{1.175690in}}%
\pgfpathlineto{\pgfqpoint{0.936569in}{1.174972in}}%
\pgfpathlineto{\pgfqpoint{0.945445in}{1.181581in}}%
\pgfpathlineto{\pgfqpoint{0.954698in}{1.188037in}}%
\pgfpathlineto{\pgfqpoint{0.964318in}{1.194336in}}%
\pgfpathlineto{\pgfqpoint{0.974295in}{1.200470in}}%
\pgfpathlineto{\pgfqpoint{0.976746in}{1.201019in}}%
\pgfpathlineto{\pgfqpoint{0.979194in}{1.201753in}}%
\pgfpathlineto{\pgfqpoint{0.981640in}{1.202668in}}%
\pgfpathlineto{\pgfqpoint{0.984084in}{1.203760in}}%
\pgfpathlineto{\pgfqpoint{0.974375in}{1.197788in}}%
\pgfpathlineto{\pgfqpoint{0.965015in}{1.191657in}}%
\pgfpathlineto{\pgfqpoint{0.956014in}{1.185373in}}%
\pgfpathlineto{\pgfqpoint{0.947380in}{1.178940in}}%
\pgfpathclose%
\pgfusepath{fill}%
\end{pgfscope}%
\begin{pgfscope}%
\pgfpathrectangle{\pgfqpoint{0.329460in}{0.284240in}}{\pgfqpoint{1.989680in}{1.989680in}}%
\pgfusepath{clip}%
\pgfsetbuttcap%
\pgfsetroundjoin%
\definecolor{currentfill}{rgb}{0.280255,0.165693,0.476498}%
\pgfsetfillcolor{currentfill}%
\pgfsetlinewidth{0.000000pt}%
\definecolor{currentstroke}{rgb}{0.000000,0.000000,0.000000}%
\pgfsetstrokecolor{currentstroke}%
\pgfsetdash{}{0pt}%
\pgfpathmoveto{\pgfqpoint{1.583162in}{1.296272in}}%
\pgfpathlineto{\pgfqpoint{1.584920in}{1.292707in}}%
\pgfpathlineto{\pgfqpoint{1.586677in}{1.289250in}}%
\pgfpathlineto{\pgfqpoint{1.588434in}{1.285902in}}%
\pgfpathlineto{\pgfqpoint{1.590191in}{1.282669in}}%
\pgfpathlineto{\pgfqpoint{1.601216in}{1.278793in}}%
\pgfpathlineto{\pgfqpoint{1.612013in}{1.274741in}}%
\pgfpathlineto{\pgfqpoint{1.622572in}{1.270517in}}%
\pgfpathlineto{\pgfqpoint{1.632881in}{1.266124in}}%
\pgfpathlineto{\pgfqpoint{1.630805in}{1.269489in}}%
\pgfpathlineto{\pgfqpoint{1.628729in}{1.272966in}}%
\pgfpathlineto{\pgfqpoint{1.626653in}{1.276555in}}%
\pgfpathlineto{\pgfqpoint{1.624576in}{1.280250in}}%
\pgfpathlineto{\pgfqpoint{1.614576in}{1.284504in}}%
\pgfpathlineto{\pgfqpoint{1.604334in}{1.288594in}}%
\pgfpathlineto{\pgfqpoint{1.593859in}{1.292518in}}%
\pgfpathlineto{\pgfqpoint{1.583162in}{1.296272in}}%
\pgfpathclose%
\pgfusepath{fill}%
\end{pgfscope}%
\begin{pgfscope}%
\pgfpathrectangle{\pgfqpoint{0.329460in}{0.284240in}}{\pgfqpoint{1.989680in}{1.989680in}}%
\pgfusepath{clip}%
\pgfsetbuttcap%
\pgfsetroundjoin%
\definecolor{currentfill}{rgb}{0.248629,0.278775,0.534556}%
\pgfsetfillcolor{currentfill}%
\pgfsetlinewidth{0.000000pt}%
\definecolor{currentstroke}{rgb}{0.000000,0.000000,0.000000}%
\pgfsetstrokecolor{currentstroke}%
\pgfsetdash{}{0pt}%
\pgfpathmoveto{\pgfqpoint{1.259130in}{1.372836in}}%
\pgfpathlineto{\pgfqpoint{1.258362in}{1.368473in}}%
\pgfpathlineto{\pgfqpoint{1.257594in}{1.364180in}}%
\pgfpathlineto{\pgfqpoint{1.256827in}{1.359959in}}%
\pgfpathlineto{\pgfqpoint{1.256060in}{1.355814in}}%
\pgfpathlineto{\pgfqpoint{1.267846in}{1.357211in}}%
\pgfpathlineto{\pgfqpoint{1.279708in}{1.358423in}}%
\pgfpathlineto{\pgfqpoint{1.291635in}{1.359449in}}%
\pgfpathlineto{\pgfqpoint{1.303615in}{1.360287in}}%
\pgfpathlineto{\pgfqpoint{1.303999in}{1.364393in}}%
\pgfpathlineto{\pgfqpoint{1.304382in}{1.368575in}}%
\pgfpathlineto{\pgfqpoint{1.304766in}{1.372829in}}%
\pgfpathlineto{\pgfqpoint{1.305150in}{1.377153in}}%
\pgfpathlineto{\pgfqpoint{1.293556in}{1.376344in}}%
\pgfpathlineto{\pgfqpoint{1.282014in}{1.375355in}}%
\pgfpathlineto{\pgfqpoint{1.270536in}{1.374185in}}%
\pgfpathlineto{\pgfqpoint{1.259130in}{1.372836in}}%
\pgfpathclose%
\pgfusepath{fill}%
\end{pgfscope}%
\begin{pgfscope}%
\pgfpathrectangle{\pgfqpoint{0.329460in}{0.284240in}}{\pgfqpoint{1.989680in}{1.989680in}}%
\pgfusepath{clip}%
\pgfsetbuttcap%
\pgfsetroundjoin%
\definecolor{currentfill}{rgb}{0.248629,0.278775,0.534556}%
\pgfsetfillcolor{currentfill}%
\pgfsetlinewidth{0.000000pt}%
\definecolor{currentstroke}{rgb}{0.000000,0.000000,0.000000}%
\pgfsetstrokecolor{currentstroke}%
\pgfsetdash{}{0pt}%
\pgfpathmoveto{\pgfqpoint{1.398516in}{1.377072in}}%
\pgfpathlineto{\pgfqpoint{1.398911in}{1.372747in}}%
\pgfpathlineto{\pgfqpoint{1.399305in}{1.368492in}}%
\pgfpathlineto{\pgfqpoint{1.399699in}{1.364310in}}%
\pgfpathlineto{\pgfqpoint{1.400093in}{1.360203in}}%
\pgfpathlineto{\pgfqpoint{1.412068in}{1.359344in}}%
\pgfpathlineto{\pgfqpoint{1.423988in}{1.358298in}}%
\pgfpathlineto{\pgfqpoint{1.435842in}{1.357065in}}%
\pgfpathlineto{\pgfqpoint{1.447619in}{1.355647in}}%
\pgfpathlineto{\pgfqpoint{1.446842in}{1.359794in}}%
\pgfpathlineto{\pgfqpoint{1.446064in}{1.364016in}}%
\pgfpathlineto{\pgfqpoint{1.445286in}{1.368311in}}%
\pgfpathlineto{\pgfqpoint{1.444507in}{1.372676in}}%
\pgfpathlineto{\pgfqpoint{1.433110in}{1.374044in}}%
\pgfpathlineto{\pgfqpoint{1.421639in}{1.375233in}}%
\pgfpathlineto{\pgfqpoint{1.410104in}{1.376243in}}%
\pgfpathlineto{\pgfqpoint{1.398516in}{1.377072in}}%
\pgfpathclose%
\pgfusepath{fill}%
\end{pgfscope}%
\begin{pgfscope}%
\pgfpathrectangle{\pgfqpoint{0.329460in}{0.284240in}}{\pgfqpoint{1.989680in}{1.989680in}}%
\pgfusepath{clip}%
\pgfsetbuttcap%
\pgfsetroundjoin%
\definecolor{currentfill}{rgb}{0.274128,0.199721,0.498911}%
\pgfsetfillcolor{currentfill}%
\pgfsetlinewidth{0.000000pt}%
\definecolor{currentstroke}{rgb}{0.000000,0.000000,0.000000}%
\pgfsetstrokecolor{currentstroke}%
\pgfsetdash{}{0pt}%
\pgfpathmoveto{\pgfqpoint{1.117022in}{1.308317in}}%
\pgfpathlineto{\pgfqpoint{1.115189in}{1.304329in}}%
\pgfpathlineto{\pgfqpoint{1.113356in}{1.300435in}}%
\pgfpathlineto{\pgfqpoint{1.111525in}{1.296639in}}%
\pgfpathlineto{\pgfqpoint{1.109694in}{1.292943in}}%
\pgfpathlineto{\pgfqpoint{1.120415in}{1.296678in}}%
\pgfpathlineto{\pgfqpoint{1.131347in}{1.300239in}}%
\pgfpathlineto{\pgfqpoint{1.142480in}{1.303625in}}%
\pgfpathlineto{\pgfqpoint{1.153803in}{1.306831in}}%
\pgfpathlineto{\pgfqpoint{1.155296in}{1.310413in}}%
\pgfpathlineto{\pgfqpoint{1.156790in}{1.314096in}}%
\pgfpathlineto{\pgfqpoint{1.158284in}{1.317876in}}%
\pgfpathlineto{\pgfqpoint{1.159779in}{1.321750in}}%
\pgfpathlineto{\pgfqpoint{1.148802in}{1.318649in}}%
\pgfpathlineto{\pgfqpoint{1.138010in}{1.315374in}}%
\pgfpathlineto{\pgfqpoint{1.127413in}{1.311929in}}%
\pgfpathlineto{\pgfqpoint{1.117022in}{1.308317in}}%
\pgfpathclose%
\pgfusepath{fill}%
\end{pgfscope}%
\begin{pgfscope}%
\pgfpathrectangle{\pgfqpoint{0.329460in}{0.284240in}}{\pgfqpoint{1.989680in}{1.989680in}}%
\pgfusepath{clip}%
\pgfsetbuttcap%
\pgfsetroundjoin%
\definecolor{currentfill}{rgb}{0.274952,0.037752,0.364543}%
\pgfsetfillcolor{currentfill}%
\pgfsetlinewidth{0.000000pt}%
\definecolor{currentstroke}{rgb}{0.000000,0.000000,0.000000}%
\pgfsetstrokecolor{currentstroke}%
\pgfsetdash{}{0pt}%
\pgfpathmoveto{\pgfqpoint{1.699865in}{1.214854in}}%
\pgfpathlineto{\pgfqpoint{1.702240in}{1.213126in}}%
\pgfpathlineto{\pgfqpoint{1.704618in}{1.211561in}}%
\pgfpathlineto{\pgfqpoint{1.706996in}{1.210161in}}%
\pgfpathlineto{\pgfqpoint{1.709377in}{1.208931in}}%
\pgfpathlineto{\pgfqpoint{1.719387in}{1.203105in}}%
\pgfpathlineto{\pgfqpoint{1.729057in}{1.197115in}}%
\pgfpathlineto{\pgfqpoint{1.738378in}{1.190966in}}%
\pgfpathlineto{\pgfqpoint{1.747339in}{1.184665in}}%
\pgfpathlineto{\pgfqpoint{1.744696in}{1.186062in}}%
\pgfpathlineto{\pgfqpoint{1.742055in}{1.187628in}}%
\pgfpathlineto{\pgfqpoint{1.739416in}{1.189361in}}%
\pgfpathlineto{\pgfqpoint{1.736779in}{1.191256in}}%
\pgfpathlineto{\pgfqpoint{1.728067in}{1.197384in}}%
\pgfpathlineto{\pgfqpoint{1.719004in}{1.203362in}}%
\pgfpathlineto{\pgfqpoint{1.709600in}{1.209187in}}%
\pgfpathlineto{\pgfqpoint{1.699865in}{1.214854in}}%
\pgfpathclose%
\pgfusepath{fill}%
\end{pgfscope}%
\begin{pgfscope}%
\pgfpathrectangle{\pgfqpoint{0.329460in}{0.284240in}}{\pgfqpoint{1.989680in}{1.989680in}}%
\pgfusepath{clip}%
\pgfsetbuttcap%
\pgfsetroundjoin%
\definecolor{currentfill}{rgb}{0.282327,0.094955,0.417331}%
\pgfsetfillcolor{currentfill}%
\pgfsetlinewidth{0.000000pt}%
\definecolor{currentstroke}{rgb}{0.000000,0.000000,0.000000}%
\pgfsetstrokecolor{currentstroke}%
\pgfsetdash{}{0pt}%
\pgfpathmoveto{\pgfqpoint{1.013293in}{1.229344in}}%
\pgfpathlineto{\pgfqpoint{1.010864in}{1.226411in}}%
\pgfpathlineto{\pgfqpoint{1.008435in}{1.223611in}}%
\pgfpathlineto{\pgfqpoint{1.006005in}{1.220949in}}%
\pgfpathlineto{\pgfqpoint{1.003574in}{1.218429in}}%
\pgfpathlineto{\pgfqpoint{1.013071in}{1.223916in}}%
\pgfpathlineto{\pgfqpoint{1.022881in}{1.229244in}}%
\pgfpathlineto{\pgfqpoint{1.032993in}{1.234408in}}%
\pgfpathlineto{\pgfqpoint{1.043398in}{1.239405in}}%
\pgfpathlineto{\pgfqpoint{1.045543in}{1.241771in}}%
\pgfpathlineto{\pgfqpoint{1.047687in}{1.244278in}}%
\pgfpathlineto{\pgfqpoint{1.049831in}{1.246922in}}%
\pgfpathlineto{\pgfqpoint{1.051975in}{1.249700in}}%
\pgfpathlineto{\pgfqpoint{1.041867in}{1.244851in}}%
\pgfpathlineto{\pgfqpoint{1.032045in}{1.239839in}}%
\pgfpathlineto{\pgfqpoint{1.022517in}{1.234668in}}%
\pgfpathlineto{\pgfqpoint{1.013293in}{1.229344in}}%
\pgfpathclose%
\pgfusepath{fill}%
\end{pgfscope}%
\begin{pgfscope}%
\pgfpathrectangle{\pgfqpoint{0.329460in}{0.284240in}}{\pgfqpoint{1.989680in}{1.989680in}}%
\pgfusepath{clip}%
\pgfsetbuttcap%
\pgfsetroundjoin%
\definecolor{currentfill}{rgb}{0.233603,0.313828,0.543914}%
\pgfsetfillcolor{currentfill}%
\pgfsetlinewidth{0.000000pt}%
\definecolor{currentstroke}{rgb}{0.000000,0.000000,0.000000}%
\pgfsetstrokecolor{currentstroke}%
\pgfsetdash{}{0pt}%
\pgfpathmoveto{\pgfqpoint{1.818471in}{1.385449in}}%
\pgfpathlineto{\pgfqpoint{1.821121in}{1.396324in}}%
\pgfpathlineto{\pgfqpoint{1.823782in}{1.407588in}}%
\pgfpathlineto{\pgfqpoint{1.826455in}{1.419247in}}%
\pgfpathlineto{\pgfqpoint{1.829140in}{1.431306in}}%
\pgfpathlineto{\pgfqpoint{1.842649in}{1.423804in}}%
\pgfpathlineto{\pgfqpoint{1.855717in}{1.416086in}}%
\pgfpathlineto{\pgfqpoint{1.868330in}{1.408158in}}%
\pgfpathlineto{\pgfqpoint{1.880476in}{1.400028in}}%
\pgfpathlineto{\pgfqpoint{1.877487in}{1.388084in}}%
\pgfpathlineto{\pgfqpoint{1.874512in}{1.376542in}}%
\pgfpathlineto{\pgfqpoint{1.871550in}{1.365397in}}%
\pgfpathlineto{\pgfqpoint{1.868601in}{1.354642in}}%
\pgfpathlineto{\pgfqpoint{1.856743in}{1.362649in}}%
\pgfpathlineto{\pgfqpoint{1.844426in}{1.370456in}}%
\pgfpathlineto{\pgfqpoint{1.831665in}{1.378058in}}%
\pgfpathlineto{\pgfqpoint{1.818471in}{1.385449in}}%
\pgfpathclose%
\pgfusepath{fill}%
\end{pgfscope}%
\begin{pgfscope}%
\pgfpathrectangle{\pgfqpoint{0.329460in}{0.284240in}}{\pgfqpoint{1.989680in}{1.989680in}}%
\pgfusepath{clip}%
\pgfsetbuttcap%
\pgfsetroundjoin%
\definecolor{currentfill}{rgb}{0.263663,0.237631,0.518762}%
\pgfsetfillcolor{currentfill}%
\pgfsetlinewidth{0.000000pt}%
\definecolor{currentstroke}{rgb}{0.000000,0.000000,0.000000}%
\pgfsetstrokecolor{currentstroke}%
\pgfsetdash{}{0pt}%
\pgfpathmoveto{\pgfqpoint{1.165767in}{1.338129in}}%
\pgfpathlineto{\pgfqpoint{1.164269in}{1.333908in}}%
\pgfpathlineto{\pgfqpoint{1.162771in}{1.329770in}}%
\pgfpathlineto{\pgfqpoint{1.161275in}{1.325716in}}%
\pgfpathlineto{\pgfqpoint{1.159779in}{1.321750in}}%
\pgfpathlineto{\pgfqpoint{1.170931in}{1.324676in}}%
\pgfpathlineto{\pgfqpoint{1.182247in}{1.327424in}}%
\pgfpathlineto{\pgfqpoint{1.193717in}{1.329991in}}%
\pgfpathlineto{\pgfqpoint{1.205330in}{1.332376in}}%
\pgfpathlineto{\pgfqpoint{1.206468in}{1.336252in}}%
\pgfpathlineto{\pgfqpoint{1.207606in}{1.340216in}}%
\pgfpathlineto{\pgfqpoint{1.208745in}{1.344265in}}%
\pgfpathlineto{\pgfqpoint{1.209885in}{1.348395in}}%
\pgfpathlineto{\pgfqpoint{1.198636in}{1.346091in}}%
\pgfpathlineto{\pgfqpoint{1.187527in}{1.343611in}}%
\pgfpathlineto{\pgfqpoint{1.176567in}{1.340956in}}%
\pgfpathlineto{\pgfqpoint{1.165767in}{1.338129in}}%
\pgfpathclose%
\pgfusepath{fill}%
\end{pgfscope}%
\begin{pgfscope}%
\pgfpathrectangle{\pgfqpoint{0.329460in}{0.284240in}}{\pgfqpoint{1.989680in}{1.989680in}}%
\pgfusepath{clip}%
\pgfsetbuttcap%
\pgfsetroundjoin%
\definecolor{currentfill}{rgb}{0.280255,0.165693,0.476498}%
\pgfsetfillcolor{currentfill}%
\pgfsetlinewidth{0.000000pt}%
\definecolor{currentstroke}{rgb}{0.000000,0.000000,0.000000}%
\pgfsetstrokecolor{currentstroke}%
\pgfsetdash{}{0pt}%
\pgfpathmoveto{\pgfqpoint{1.069122in}{1.276335in}}%
\pgfpathlineto{\pgfqpoint{1.066978in}{1.272608in}}%
\pgfpathlineto{\pgfqpoint{1.064834in}{1.268987in}}%
\pgfpathlineto{\pgfqpoint{1.062691in}{1.265478in}}%
\pgfpathlineto{\pgfqpoint{1.060548in}{1.262081in}}%
\pgfpathlineto{\pgfqpoint{1.070627in}{1.266621in}}%
\pgfpathlineto{\pgfqpoint{1.080964in}{1.270995in}}%
\pgfpathlineto{\pgfqpoint{1.091550in}{1.275200in}}%
\pgfpathlineto{\pgfqpoint{1.102373in}{1.279232in}}%
\pgfpathlineto{\pgfqpoint{1.104203in}{1.282493in}}%
\pgfpathlineto{\pgfqpoint{1.106033in}{1.285867in}}%
\pgfpathlineto{\pgfqpoint{1.107863in}{1.289352in}}%
\pgfpathlineto{\pgfqpoint{1.109694in}{1.292943in}}%
\pgfpathlineto{\pgfqpoint{1.099194in}{1.289038in}}%
\pgfpathlineto{\pgfqpoint{1.088925in}{1.284966in}}%
\pgfpathlineto{\pgfqpoint{1.078898in}{1.280731in}}%
\pgfpathlineto{\pgfqpoint{1.069122in}{1.276335in}}%
\pgfpathclose%
\pgfusepath{fill}%
\end{pgfscope}%
\begin{pgfscope}%
\pgfpathrectangle{\pgfqpoint{0.329460in}{0.284240in}}{\pgfqpoint{1.989680in}{1.989680in}}%
\pgfusepath{clip}%
\pgfsetbuttcap%
\pgfsetroundjoin%
\definecolor{currentfill}{rgb}{0.272594,0.025563,0.353093}%
\pgfsetfillcolor{currentfill}%
\pgfsetlinewidth{0.000000pt}%
\definecolor{currentstroke}{rgb}{0.000000,0.000000,0.000000}%
\pgfsetstrokecolor{currentstroke}%
\pgfsetdash{}{0pt}%
\pgfpathmoveto{\pgfqpoint{1.800905in}{1.198443in}}%
\pgfpathlineto{\pgfqpoint{1.803633in}{1.201535in}}%
\pgfpathlineto{\pgfqpoint{1.806367in}{1.204887in}}%
\pgfpathlineto{\pgfqpoint{1.809107in}{1.208505in}}%
\pgfpathlineto{\pgfqpoint{1.811855in}{1.212393in}}%
\pgfpathlineto{\pgfqpoint{1.821925in}{1.204926in}}%
\pgfpathlineto{\pgfqpoint{1.831560in}{1.197291in}}%
\pgfpathlineto{\pgfqpoint{1.840750in}{1.189495in}}%
\pgfpathlineto{\pgfqpoint{1.849485in}{1.181546in}}%
\pgfpathlineto{\pgfqpoint{1.846499in}{1.177824in}}%
\pgfpathlineto{\pgfqpoint{1.843520in}{1.174373in}}%
\pgfpathlineto{\pgfqpoint{1.840549in}{1.171188in}}%
\pgfpathlineto{\pgfqpoint{1.837584in}{1.168266in}}%
\pgfpathlineto{\pgfqpoint{1.829072in}{1.176042in}}%
\pgfpathlineto{\pgfqpoint{1.820115in}{1.183668in}}%
\pgfpathlineto{\pgfqpoint{1.810722in}{1.191137in}}%
\pgfpathlineto{\pgfqpoint{1.800905in}{1.198443in}}%
\pgfpathclose%
\pgfusepath{fill}%
\end{pgfscope}%
\begin{pgfscope}%
\pgfpathrectangle{\pgfqpoint{0.329460in}{0.284240in}}{\pgfqpoint{1.989680in}{1.989680in}}%
\pgfusepath{clip}%
\pgfsetbuttcap%
\pgfsetroundjoin%
\definecolor{currentfill}{rgb}{0.268510,0.009605,0.335427}%
\pgfsetfillcolor{currentfill}%
\pgfsetlinewidth{0.000000pt}%
\definecolor{currentstroke}{rgb}{0.000000,0.000000,0.000000}%
\pgfsetstrokecolor{currentstroke}%
\pgfsetdash{}{0pt}%
\pgfpathmoveto{\pgfqpoint{1.790051in}{1.188594in}}%
\pgfpathlineto{\pgfqpoint{1.792756in}{1.190688in}}%
\pgfpathlineto{\pgfqpoint{1.795467in}{1.193024in}}%
\pgfpathlineto{\pgfqpoint{1.798183in}{1.195608in}}%
\pgfpathlineto{\pgfqpoint{1.800905in}{1.198443in}}%
\pgfpathlineto{\pgfqpoint{1.810722in}{1.191137in}}%
\pgfpathlineto{\pgfqpoint{1.820115in}{1.183668in}}%
\pgfpathlineto{\pgfqpoint{1.829072in}{1.176042in}}%
\pgfpathlineto{\pgfqpoint{1.837584in}{1.168266in}}%
\pgfpathlineto{\pgfqpoint{1.834626in}{1.165601in}}%
\pgfpathlineto{\pgfqpoint{1.831675in}{1.163188in}}%
\pgfpathlineto{\pgfqpoint{1.828730in}{1.161023in}}%
\pgfpathlineto{\pgfqpoint{1.825790in}{1.159103in}}%
\pgfpathlineto{\pgfqpoint{1.817499in}{1.166701in}}%
\pgfpathlineto{\pgfqpoint{1.808772in}{1.174154in}}%
\pgfpathlineto{\pgfqpoint{1.799619in}{1.181454in}}%
\pgfpathlineto{\pgfqpoint{1.790051in}{1.188594in}}%
\pgfpathclose%
\pgfusepath{fill}%
\end{pgfscope}%
\begin{pgfscope}%
\pgfpathrectangle{\pgfqpoint{0.329460in}{0.284240in}}{\pgfqpoint{1.989680in}{1.989680in}}%
\pgfusepath{clip}%
\pgfsetbuttcap%
\pgfsetroundjoin%
\definecolor{currentfill}{rgb}{0.277941,0.056324,0.381191}%
\pgfsetfillcolor{currentfill}%
\pgfsetlinewidth{0.000000pt}%
\definecolor{currentstroke}{rgb}{0.000000,0.000000,0.000000}%
\pgfsetstrokecolor{currentstroke}%
\pgfsetdash{}{0pt}%
\pgfpathmoveto{\pgfqpoint{1.811855in}{1.212393in}}%
\pgfpathlineto{\pgfqpoint{1.814609in}{1.216556in}}%
\pgfpathlineto{\pgfqpoint{1.817371in}{1.220999in}}%
\pgfpathlineto{\pgfqpoint{1.820139in}{1.225726in}}%
\pgfpathlineto{\pgfqpoint{1.822916in}{1.230743in}}%
\pgfpathlineto{\pgfqpoint{1.833241in}{1.223117in}}%
\pgfpathlineto{\pgfqpoint{1.843123in}{1.215321in}}%
\pgfpathlineto{\pgfqpoint{1.852549in}{1.207361in}}%
\pgfpathlineto{\pgfqpoint{1.861510in}{1.199242in}}%
\pgfpathlineto{\pgfqpoint{1.858491in}{1.194387in}}%
\pgfpathlineto{\pgfqpoint{1.855481in}{1.189823in}}%
\pgfpathlineto{\pgfqpoint{1.852479in}{1.185544in}}%
\pgfpathlineto{\pgfqpoint{1.849485in}{1.181546in}}%
\pgfpathlineto{\pgfqpoint{1.840750in}{1.189495in}}%
\pgfpathlineto{\pgfqpoint{1.831560in}{1.197291in}}%
\pgfpathlineto{\pgfqpoint{1.821925in}{1.204926in}}%
\pgfpathlineto{\pgfqpoint{1.811855in}{1.212393in}}%
\pgfpathclose%
\pgfusepath{fill}%
\end{pgfscope}%
\begin{pgfscope}%
\pgfpathrectangle{\pgfqpoint{0.329460in}{0.284240in}}{\pgfqpoint{1.989680in}{1.989680in}}%
\pgfusepath{clip}%
\pgfsetbuttcap%
\pgfsetroundjoin%
\definecolor{currentfill}{rgb}{0.248629,0.278775,0.534556}%
\pgfsetfillcolor{currentfill}%
\pgfsetlinewidth{0.000000pt}%
\definecolor{currentstroke}{rgb}{0.000000,0.000000,0.000000}%
\pgfsetstrokecolor{currentstroke}%
\pgfsetdash{}{0pt}%
\pgfpathmoveto{\pgfqpoint{1.444507in}{1.372676in}}%
\pgfpathlineto{\pgfqpoint{1.445286in}{1.368311in}}%
\pgfpathlineto{\pgfqpoint{1.446064in}{1.364016in}}%
\pgfpathlineto{\pgfqpoint{1.446842in}{1.359794in}}%
\pgfpathlineto{\pgfqpoint{1.447619in}{1.355647in}}%
\pgfpathlineto{\pgfqpoint{1.459309in}{1.354045in}}%
\pgfpathlineto{\pgfqpoint{1.470901in}{1.352260in}}%
\pgfpathlineto{\pgfqpoint{1.482383in}{1.350294in}}%
\pgfpathlineto{\pgfqpoint{1.481325in}{1.354487in}}%
\pgfpathlineto{\pgfqpoint{1.480266in}{1.358756in}}%
\pgfpathlineto{\pgfqpoint{1.479206in}{1.363098in}}%
\pgfpathlineto{\pgfqpoint{1.478145in}{1.367510in}}%
\pgfpathlineto{\pgfqpoint{1.467035in}{1.369407in}}%
\pgfpathlineto{\pgfqpoint{1.455818in}{1.371130in}}%
\pgfpathlineto{\pgfqpoint{1.444507in}{1.372676in}}%
\pgfpathclose%
\pgfusepath{fill}%
\end{pgfscope}%
\begin{pgfscope}%
\pgfpathrectangle{\pgfqpoint{0.329460in}{0.284240in}}{\pgfqpoint{1.989680in}{1.989680in}}%
\pgfusepath{clip}%
\pgfsetbuttcap%
\pgfsetroundjoin%
\definecolor{currentfill}{rgb}{0.248629,0.278775,0.534556}%
\pgfsetfillcolor{currentfill}%
\pgfsetlinewidth{0.000000pt}%
\definecolor{currentstroke}{rgb}{0.000000,0.000000,0.000000}%
\pgfsetstrokecolor{currentstroke}%
\pgfsetdash{}{0pt}%
\pgfpathmoveto{\pgfqpoint{1.214451in}{1.365677in}}%
\pgfpathlineto{\pgfqpoint{1.213308in}{1.361249in}}%
\pgfpathlineto{\pgfqpoint{1.212166in}{1.356891in}}%
\pgfpathlineto{\pgfqpoint{1.211025in}{1.352605in}}%
\pgfpathlineto{\pgfqpoint{1.209885in}{1.348395in}}%
\pgfpathlineto{\pgfqpoint{1.221262in}{1.350522in}}%
\pgfpathlineto{\pgfqpoint{1.232757in}{1.352468in}}%
\pgfpathlineto{\pgfqpoint{1.244360in}{1.354232in}}%
\pgfpathlineto{\pgfqpoint{1.256060in}{1.355814in}}%
\pgfpathlineto{\pgfqpoint{1.256827in}{1.359959in}}%
\pgfpathlineto{\pgfqpoint{1.257594in}{1.364180in}}%
\pgfpathlineto{\pgfqpoint{1.258362in}{1.368473in}}%
\pgfpathlineto{\pgfqpoint{1.259130in}{1.372836in}}%
\pgfpathlineto{\pgfqpoint{1.247809in}{1.371310in}}%
\pgfpathlineto{\pgfqpoint{1.236582in}{1.369607in}}%
\pgfpathlineto{\pgfqpoint{1.225459in}{1.367729in}}%
\pgfpathlineto{\pgfqpoint{1.214451in}{1.365677in}}%
\pgfpathclose%
\pgfusepath{fill}%
\end{pgfscope}%
\begin{pgfscope}%
\pgfpathrectangle{\pgfqpoint{0.329460in}{0.284240in}}{\pgfqpoint{1.989680in}{1.989680in}}%
\pgfusepath{clip}%
\pgfsetbuttcap%
\pgfsetroundjoin%
\definecolor{currentfill}{rgb}{0.274952,0.037752,0.364543}%
\pgfsetfillcolor{currentfill}%
\pgfsetlinewidth{0.000000pt}%
\definecolor{currentstroke}{rgb}{0.000000,0.000000,0.000000}%
\pgfsetstrokecolor{currentstroke}%
\pgfsetdash{}{0pt}%
\pgfpathmoveto{\pgfqpoint{0.958153in}{1.185690in}}%
\pgfpathlineto{\pgfqpoint{0.955463in}{1.183754in}}%
\pgfpathlineto{\pgfqpoint{0.952771in}{1.181982in}}%
\pgfpathlineto{\pgfqpoint{0.950076in}{1.180376in}}%
\pgfpathlineto{\pgfqpoint{0.947380in}{1.178940in}}%
\pgfpathlineto{\pgfqpoint{0.956014in}{1.185373in}}%
\pgfpathlineto{\pgfqpoint{0.965015in}{1.191657in}}%
\pgfpathlineto{\pgfqpoint{0.974375in}{1.197788in}}%
\pgfpathlineto{\pgfqpoint{0.984084in}{1.203760in}}%
\pgfpathlineto{\pgfqpoint{0.986526in}{1.205026in}}%
\pgfpathlineto{\pgfqpoint{0.988966in}{1.206461in}}%
\pgfpathlineto{\pgfqpoint{0.991404in}{1.208062in}}%
\pgfpathlineto{\pgfqpoint{0.993841in}{1.209825in}}%
\pgfpathlineto{\pgfqpoint{0.984399in}{1.204017in}}%
\pgfpathlineto{\pgfqpoint{0.975298in}{1.198055in}}%
\pgfpathlineto{\pgfqpoint{0.966546in}{1.191944in}}%
\pgfpathlineto{\pgfqpoint{0.958153in}{1.185690in}}%
\pgfpathclose%
\pgfusepath{fill}%
\end{pgfscope}%
\begin{pgfscope}%
\pgfpathrectangle{\pgfqpoint{0.329460in}{0.284240in}}{\pgfqpoint{1.989680in}{1.989680in}}%
\pgfusepath{clip}%
\pgfsetbuttcap%
\pgfsetroundjoin%
\definecolor{currentfill}{rgb}{0.267004,0.004874,0.329415}%
\pgfsetfillcolor{currentfill}%
\pgfsetlinewidth{0.000000pt}%
\definecolor{currentstroke}{rgb}{0.000000,0.000000,0.000000}%
\pgfsetstrokecolor{currentstroke}%
\pgfsetdash{}{0pt}%
\pgfpathmoveto{\pgfqpoint{1.779280in}{1.182561in}}%
\pgfpathlineto{\pgfqpoint{1.781965in}{1.183726in}}%
\pgfpathlineto{\pgfqpoint{1.784656in}{1.185118in}}%
\pgfpathlineto{\pgfqpoint{1.787351in}{1.186739in}}%
\pgfpathlineto{\pgfqpoint{1.790051in}{1.188594in}}%
\pgfpathlineto{\pgfqpoint{1.799619in}{1.181454in}}%
\pgfpathlineto{\pgfqpoint{1.808772in}{1.174154in}}%
\pgfpathlineto{\pgfqpoint{1.817499in}{1.166701in}}%
\pgfpathlineto{\pgfqpoint{1.825790in}{1.159103in}}%
\pgfpathlineto{\pgfqpoint{1.822857in}{1.157421in}}%
\pgfpathlineto{\pgfqpoint{1.819929in}{1.155974in}}%
\pgfpathlineto{\pgfqpoint{1.817007in}{1.154758in}}%
\pgfpathlineto{\pgfqpoint{1.814089in}{1.153768in}}%
\pgfpathlineto{\pgfqpoint{1.806015in}{1.161186in}}%
\pgfpathlineto{\pgfqpoint{1.797516in}{1.168462in}}%
\pgfpathlineto{\pgfqpoint{1.788601in}{1.175589in}}%
\pgfpathlineto{\pgfqpoint{1.779280in}{1.182561in}}%
\pgfpathclose%
\pgfusepath{fill}%
\end{pgfscope}%
\begin{pgfscope}%
\pgfpathrectangle{\pgfqpoint{0.329460in}{0.284240in}}{\pgfqpoint{1.989680in}{1.989680in}}%
\pgfusepath{clip}%
\pgfsetbuttcap%
\pgfsetroundjoin%
\definecolor{currentfill}{rgb}{0.283072,0.130895,0.449241}%
\pgfsetfillcolor{currentfill}%
\pgfsetlinewidth{0.000000pt}%
\definecolor{currentstroke}{rgb}{0.000000,0.000000,0.000000}%
\pgfsetstrokecolor{currentstroke}%
\pgfsetdash{}{0pt}%
\pgfpathmoveto{\pgfqpoint{1.632881in}{1.266124in}}%
\pgfpathlineto{\pgfqpoint{1.634957in}{1.262877in}}%
\pgfpathlineto{\pgfqpoint{1.637032in}{1.259750in}}%
\pgfpathlineto{\pgfqpoint{1.639108in}{1.256747in}}%
\pgfpathlineto{\pgfqpoint{1.641184in}{1.253870in}}%
\pgfpathlineto{\pgfqpoint{1.651537in}{1.249169in}}%
\pgfpathlineto{\pgfqpoint{1.661614in}{1.244302in}}%
\pgfpathlineto{\pgfqpoint{1.671404in}{1.239272in}}%
\pgfpathlineto{\pgfqpoint{1.680899in}{1.234084in}}%
\pgfpathlineto{\pgfqpoint{1.678531in}{1.237112in}}%
\pgfpathlineto{\pgfqpoint{1.676164in}{1.240267in}}%
\pgfpathlineto{\pgfqpoint{1.673796in}{1.243545in}}%
\pgfpathlineto{\pgfqpoint{1.671429in}{1.246944in}}%
\pgfpathlineto{\pgfqpoint{1.662215in}{1.251973in}}%
\pgfpathlineto{\pgfqpoint{1.652712in}{1.256848in}}%
\pgfpathlineto{\pgfqpoint{1.642931in}{1.261567in}}%
\pgfpathlineto{\pgfqpoint{1.632881in}{1.266124in}}%
\pgfpathclose%
\pgfusepath{fill}%
\end{pgfscope}%
\begin{pgfscope}%
\pgfpathrectangle{\pgfqpoint{0.329460in}{0.284240in}}{\pgfqpoint{1.989680in}{1.989680in}}%
\pgfusepath{clip}%
\pgfsetbuttcap%
\pgfsetroundjoin%
\definecolor{currentfill}{rgb}{0.282327,0.094955,0.417331}%
\pgfsetfillcolor{currentfill}%
\pgfsetlinewidth{0.000000pt}%
\definecolor{currentstroke}{rgb}{0.000000,0.000000,0.000000}%
\pgfsetstrokecolor{currentstroke}%
\pgfsetdash{}{0pt}%
\pgfpathmoveto{\pgfqpoint{1.822916in}{1.230743in}}%
\pgfpathlineto{\pgfqpoint{1.825700in}{1.236054in}}%
\pgfpathlineto{\pgfqpoint{1.828493in}{1.241664in}}%
\pgfpathlineto{\pgfqpoint{1.831294in}{1.247579in}}%
\pgfpathlineto{\pgfqpoint{1.834104in}{1.253804in}}%
\pgfpathlineto{\pgfqpoint{1.844688in}{1.246027in}}%
\pgfpathlineto{\pgfqpoint{1.854819in}{1.238074in}}%
\pgfpathlineto{\pgfqpoint{1.864485in}{1.229954in}}%
\pgfpathlineto{\pgfqpoint{1.873675in}{1.221671in}}%
\pgfpathlineto{\pgfqpoint{1.870620in}{1.215603in}}%
\pgfpathlineto{\pgfqpoint{1.867574in}{1.209845in}}%
\pgfpathlineto{\pgfqpoint{1.864537in}{1.204393in}}%
\pgfpathlineto{\pgfqpoint{1.861510in}{1.199242in}}%
\pgfpathlineto{\pgfqpoint{1.852549in}{1.207361in}}%
\pgfpathlineto{\pgfqpoint{1.843123in}{1.215321in}}%
\pgfpathlineto{\pgfqpoint{1.833241in}{1.223117in}}%
\pgfpathlineto{\pgfqpoint{1.822916in}{1.230743in}}%
\pgfpathclose%
\pgfusepath{fill}%
\end{pgfscope}%
\begin{pgfscope}%
\pgfpathrectangle{\pgfqpoint{0.329460in}{0.284240in}}{\pgfqpoint{1.989680in}{1.989680in}}%
\pgfusepath{clip}%
\pgfsetbuttcap%
\pgfsetroundjoin%
\definecolor{currentfill}{rgb}{0.233603,0.313828,0.543914}%
\pgfsetfillcolor{currentfill}%
\pgfsetlinewidth{0.000000pt}%
\definecolor{currentstroke}{rgb}{0.000000,0.000000,0.000000}%
\pgfsetstrokecolor{currentstroke}%
\pgfsetdash{}{0pt}%
\pgfpathmoveto{\pgfqpoint{0.823628in}{1.347363in}}%
\pgfpathlineto{\pgfqpoint{0.820618in}{1.358089in}}%
\pgfpathlineto{\pgfqpoint{0.817594in}{1.369206in}}%
\pgfpathlineto{\pgfqpoint{0.814557in}{1.380720in}}%
\pgfpathlineto{\pgfqpoint{0.811506in}{1.392637in}}%
\pgfpathlineto{\pgfqpoint{0.823225in}{1.400941in}}%
\pgfpathlineto{\pgfqpoint{0.835424in}{1.409049in}}%
\pgfpathlineto{\pgfqpoint{0.848088in}{1.416954in}}%
\pgfpathlineto{\pgfqpoint{0.861206in}{1.424648in}}%
\pgfpathlineto{\pgfqpoint{0.863961in}{1.412613in}}%
\pgfpathlineto{\pgfqpoint{0.866705in}{1.400979in}}%
\pgfpathlineto{\pgfqpoint{0.869436in}{1.389740in}}%
\pgfpathlineto{\pgfqpoint{0.872155in}{1.378890in}}%
\pgfpathlineto{\pgfqpoint{0.859345in}{1.371311in}}%
\pgfpathlineto{\pgfqpoint{0.846978in}{1.363526in}}%
\pgfpathlineto{\pgfqpoint{0.835068in}{1.355541in}}%
\pgfpathlineto{\pgfqpoint{0.823628in}{1.347363in}}%
\pgfpathclose%
\pgfusepath{fill}%
\end{pgfscope}%
\begin{pgfscope}%
\pgfpathrectangle{\pgfqpoint{0.329460in}{0.284240in}}{\pgfqpoint{1.989680in}{1.989680in}}%
\pgfusepath{clip}%
\pgfsetbuttcap%
\pgfsetroundjoin%
\definecolor{currentfill}{rgb}{0.279566,0.067836,0.391917}%
\pgfsetfillcolor{currentfill}%
\pgfsetlinewidth{0.000000pt}%
\definecolor{currentstroke}{rgb}{0.000000,0.000000,0.000000}%
\pgfsetstrokecolor{currentstroke}%
\pgfsetdash{}{0pt}%
\pgfpathmoveto{\pgfqpoint{1.690375in}{1.223314in}}%
\pgfpathlineto{\pgfqpoint{1.692745in}{1.220973in}}%
\pgfpathlineto{\pgfqpoint{1.695117in}{1.218781in}}%
\pgfpathlineto{\pgfqpoint{1.697490in}{1.216740in}}%
\pgfpathlineto{\pgfqpoint{1.699865in}{1.214854in}}%
\pgfpathlineto{\pgfqpoint{1.709600in}{1.209187in}}%
\pgfpathlineto{\pgfqpoint{1.719004in}{1.203362in}}%
\pgfpathlineto{\pgfqpoint{1.728067in}{1.197384in}}%
\pgfpathlineto{\pgfqpoint{1.736779in}{1.191256in}}%
\pgfpathlineto{\pgfqpoint{1.734144in}{1.193311in}}%
\pgfpathlineto{\pgfqpoint{1.731510in}{1.195521in}}%
\pgfpathlineto{\pgfqpoint{1.728877in}{1.197882in}}%
\pgfpathlineto{\pgfqpoint{1.726246in}{1.200392in}}%
\pgfpathlineto{\pgfqpoint{1.717782in}{1.206344in}}%
\pgfpathlineto{\pgfqpoint{1.708975in}{1.212151in}}%
\pgfpathlineto{\pgfqpoint{1.699836in}{1.217809in}}%
\pgfpathlineto{\pgfqpoint{1.690375in}{1.223314in}}%
\pgfpathclose%
\pgfusepath{fill}%
\end{pgfscope}%
\begin{pgfscope}%
\pgfpathrectangle{\pgfqpoint{0.329460in}{0.284240in}}{\pgfqpoint{1.989680in}{1.989680in}}%
\pgfusepath{clip}%
\pgfsetbuttcap%
\pgfsetroundjoin%
\definecolor{currentfill}{rgb}{0.267004,0.004874,0.329415}%
\pgfsetfillcolor{currentfill}%
\pgfsetlinewidth{0.000000pt}%
\definecolor{currentstroke}{rgb}{0.000000,0.000000,0.000000}%
\pgfsetstrokecolor{currentstroke}%
\pgfsetdash{}{0pt}%
\pgfpathmoveto{\pgfqpoint{1.768579in}{1.180068in}}%
\pgfpathlineto{\pgfqpoint{1.771248in}{1.180374in}}%
\pgfpathlineto{\pgfqpoint{1.773921in}{1.180888in}}%
\pgfpathlineto{\pgfqpoint{1.776598in}{1.181616in}}%
\pgfpathlineto{\pgfqpoint{1.779280in}{1.182561in}}%
\pgfpathlineto{\pgfqpoint{1.788601in}{1.175589in}}%
\pgfpathlineto{\pgfqpoint{1.797516in}{1.168462in}}%
\pgfpathlineto{\pgfqpoint{1.806015in}{1.161186in}}%
\pgfpathlineto{\pgfqpoint{1.814089in}{1.153768in}}%
\pgfpathlineto{\pgfqpoint{1.811177in}{1.153001in}}%
\pgfpathlineto{\pgfqpoint{1.808269in}{1.152451in}}%
\pgfpathlineto{\pgfqpoint{1.805366in}{1.152115in}}%
\pgfpathlineto{\pgfqpoint{1.802467in}{1.151988in}}%
\pgfpathlineto{\pgfqpoint{1.794608in}{1.159222in}}%
\pgfpathlineto{\pgfqpoint{1.786335in}{1.166317in}}%
\pgfpathlineto{\pgfqpoint{1.777655in}{1.173268in}}%
\pgfpathlineto{\pgfqpoint{1.768579in}{1.180068in}}%
\pgfpathclose%
\pgfusepath{fill}%
\end{pgfscope}%
\begin{pgfscope}%
\pgfpathrectangle{\pgfqpoint{0.329460in}{0.284240in}}{\pgfqpoint{1.989680in}{1.989680in}}%
\pgfusepath{clip}%
\pgfsetbuttcap%
\pgfsetroundjoin%
\definecolor{currentfill}{rgb}{0.263663,0.237631,0.518762}%
\pgfsetfillcolor{currentfill}%
\pgfsetlinewidth{0.000000pt}%
\definecolor{currentstroke}{rgb}{0.000000,0.000000,0.000000}%
\pgfsetstrokecolor{currentstroke}%
\pgfsetdash{}{0pt}%
\pgfpathmoveto{\pgfqpoint{1.527016in}{1.340650in}}%
\pgfpathlineto{\pgfqpoint{1.528436in}{1.336452in}}%
\pgfpathlineto{\pgfqpoint{1.529855in}{1.332335in}}%
\pgfpathlineto{\pgfqpoint{1.531274in}{1.328303in}}%
\pgfpathlineto{\pgfqpoint{1.532691in}{1.324360in}}%
\pgfpathlineto{\pgfqpoint{1.543824in}{1.321414in}}%
\pgfpathlineto{\pgfqpoint{1.554781in}{1.318294in}}%
\pgfpathlineto{\pgfqpoint{1.565552in}{1.315000in}}%
\pgfpathlineto{\pgfqpoint{1.576127in}{1.311536in}}%
\pgfpathlineto{\pgfqpoint{1.574366in}{1.315588in}}%
\pgfpathlineto{\pgfqpoint{1.572605in}{1.319728in}}%
\pgfpathlineto{\pgfqpoint{1.570843in}{1.323954in}}%
\pgfpathlineto{\pgfqpoint{1.569079in}{1.328261in}}%
\pgfpathlineto{\pgfqpoint{1.558840in}{1.331607in}}%
\pgfpathlineto{\pgfqpoint{1.548409in}{1.334789in}}%
\pgfpathlineto{\pgfqpoint{1.537798in}{1.337804in}}%
\pgfpathlineto{\pgfqpoint{1.527016in}{1.340650in}}%
\pgfpathclose%
\pgfusepath{fill}%
\end{pgfscope}%
\begin{pgfscope}%
\pgfpathrectangle{\pgfqpoint{0.329460in}{0.284240in}}{\pgfqpoint{1.989680in}{1.989680in}}%
\pgfusepath{clip}%
\pgfsetbuttcap%
\pgfsetroundjoin%
\definecolor{currentfill}{rgb}{0.272594,0.025563,0.353093}%
\pgfsetfillcolor{currentfill}%
\pgfsetlinewidth{0.000000pt}%
\definecolor{currentstroke}{rgb}{0.000000,0.000000,0.000000}%
\pgfsetstrokecolor{currentstroke}%
\pgfsetdash{}{0pt}%
\pgfpathmoveto{\pgfqpoint{0.857606in}{1.161233in}}%
\pgfpathlineto{\pgfqpoint{0.854595in}{1.164116in}}%
\pgfpathlineto{\pgfqpoint{0.851576in}{1.167261in}}%
\pgfpathlineto{\pgfqpoint{0.848550in}{1.170673in}}%
\pgfpathlineto{\pgfqpoint{0.845516in}{1.174356in}}%
\pgfpathlineto{\pgfqpoint{0.853838in}{1.182436in}}%
\pgfpathlineto{\pgfqpoint{0.862624in}{1.190369in}}%
\pgfpathlineto{\pgfqpoint{0.871864in}{1.198147in}}%
\pgfpathlineto{\pgfqpoint{0.881548in}{1.205764in}}%
\pgfpathlineto{\pgfqpoint{0.884352in}{1.201911in}}%
\pgfpathlineto{\pgfqpoint{0.887149in}{1.198329in}}%
\pgfpathlineto{\pgfqpoint{0.889939in}{1.195013in}}%
\pgfpathlineto{\pgfqpoint{0.892723in}{1.191957in}}%
\pgfpathlineto{\pgfqpoint{0.883283in}{1.184506in}}%
\pgfpathlineto{\pgfqpoint{0.874277in}{1.176897in}}%
\pgfpathlineto{\pgfqpoint{0.865715in}{1.169137in}}%
\pgfpathlineto{\pgfqpoint{0.857606in}{1.161233in}}%
\pgfpathclose%
\pgfusepath{fill}%
\end{pgfscope}%
\begin{pgfscope}%
\pgfpathrectangle{\pgfqpoint{0.329460in}{0.284240in}}{\pgfqpoint{1.989680in}{1.989680in}}%
\pgfusepath{clip}%
\pgfsetbuttcap%
\pgfsetroundjoin%
\definecolor{currentfill}{rgb}{0.283072,0.130895,0.449241}%
\pgfsetfillcolor{currentfill}%
\pgfsetlinewidth{0.000000pt}%
\definecolor{currentstroke}{rgb}{0.000000,0.000000,0.000000}%
\pgfsetstrokecolor{currentstroke}%
\pgfsetdash{}{0pt}%
\pgfpathmoveto{\pgfqpoint{1.023004in}{1.242349in}}%
\pgfpathlineto{\pgfqpoint{1.020577in}{1.238914in}}%
\pgfpathlineto{\pgfqpoint{1.018149in}{1.235599in}}%
\pgfpathlineto{\pgfqpoint{1.015721in}{1.232408in}}%
\pgfpathlineto{\pgfqpoint{1.013293in}{1.229344in}}%
\pgfpathlineto{\pgfqpoint{1.022517in}{1.234668in}}%
\pgfpathlineto{\pgfqpoint{1.032045in}{1.239839in}}%
\pgfpathlineto{\pgfqpoint{1.041867in}{1.244851in}}%
\pgfpathlineto{\pgfqpoint{1.051975in}{1.249700in}}%
\pgfpathlineto{\pgfqpoint{1.054118in}{1.252608in}}%
\pgfpathlineto{\pgfqpoint{1.056262in}{1.255643in}}%
\pgfpathlineto{\pgfqpoint{1.058405in}{1.258802in}}%
\pgfpathlineto{\pgfqpoint{1.060548in}{1.262081in}}%
\pgfpathlineto{\pgfqpoint{1.050736in}{1.257381in}}%
\pgfpathlineto{\pgfqpoint{1.041202in}{1.252522in}}%
\pgfpathlineto{\pgfqpoint{1.031955in}{1.247510in}}%
\pgfpathlineto{\pgfqpoint{1.023004in}{1.242349in}}%
\pgfpathclose%
\pgfusepath{fill}%
\end{pgfscope}%
\begin{pgfscope}%
\pgfpathrectangle{\pgfqpoint{0.329460in}{0.284240in}}{\pgfqpoint{1.989680in}{1.989680in}}%
\pgfusepath{clip}%
\pgfsetbuttcap%
\pgfsetroundjoin%
\definecolor{currentfill}{rgb}{0.268510,0.009605,0.335427}%
\pgfsetfillcolor{currentfill}%
\pgfsetlinewidth{0.000000pt}%
\definecolor{currentstroke}{rgb}{0.000000,0.000000,0.000000}%
\pgfsetstrokecolor{currentstroke}%
\pgfsetdash{}{0pt}%
\pgfpathmoveto{\pgfqpoint{0.869587in}{1.152231in}}%
\pgfpathlineto{\pgfqpoint{0.866602in}{1.154111in}}%
\pgfpathlineto{\pgfqpoint{0.863610in}{1.156235in}}%
\pgfpathlineto{\pgfqpoint{0.860611in}{1.158608in}}%
\pgfpathlineto{\pgfqpoint{0.857606in}{1.161233in}}%
\pgfpathlineto{\pgfqpoint{0.865715in}{1.169137in}}%
\pgfpathlineto{\pgfqpoint{0.874277in}{1.176897in}}%
\pgfpathlineto{\pgfqpoint{0.883283in}{1.184506in}}%
\pgfpathlineto{\pgfqpoint{0.892723in}{1.191957in}}%
\pgfpathlineto{\pgfqpoint{0.895501in}{1.189158in}}%
\pgfpathlineto{\pgfqpoint{0.898272in}{1.186611in}}%
\pgfpathlineto{\pgfqpoint{0.901038in}{1.184312in}}%
\pgfpathlineto{\pgfqpoint{0.903799in}{1.182255in}}%
\pgfpathlineto{\pgfqpoint{0.894600in}{1.174973in}}%
\pgfpathlineto{\pgfqpoint{0.885825in}{1.167537in}}%
\pgfpathlineto{\pgfqpoint{0.877484in}{1.159954in}}%
\pgfpathlineto{\pgfqpoint{0.869587in}{1.152231in}}%
\pgfpathclose%
\pgfusepath{fill}%
\end{pgfscope}%
\begin{pgfscope}%
\pgfpathrectangle{\pgfqpoint{0.329460in}{0.284240in}}{\pgfqpoint{1.989680in}{1.989680in}}%
\pgfusepath{clip}%
\pgfsetbuttcap%
\pgfsetroundjoin%
\definecolor{currentfill}{rgb}{0.282884,0.135920,0.453427}%
\pgfsetfillcolor{currentfill}%
\pgfsetlinewidth{0.000000pt}%
\definecolor{currentstroke}{rgb}{0.000000,0.000000,0.000000}%
\pgfsetstrokecolor{currentstroke}%
\pgfsetdash{}{0pt}%
\pgfpathmoveto{\pgfqpoint{1.834104in}{1.253804in}}%
\pgfpathlineto{\pgfqpoint{1.836922in}{1.260344in}}%
\pgfpathlineto{\pgfqpoint{1.839750in}{1.267204in}}%
\pgfpathlineto{\pgfqpoint{1.842588in}{1.274390in}}%
\pgfpathlineto{\pgfqpoint{1.845435in}{1.281907in}}%
\pgfpathlineto{\pgfqpoint{1.856283in}{1.273982in}}%
\pgfpathlineto{\pgfqpoint{1.866667in}{1.265879in}}%
\pgfpathlineto{\pgfqpoint{1.876576in}{1.257604in}}%
\pgfpathlineto{\pgfqpoint{1.885999in}{1.249164in}}%
\pgfpathlineto{\pgfqpoint{1.882902in}{1.241797in}}%
\pgfpathlineto{\pgfqpoint{1.879816in}{1.234763in}}%
\pgfpathlineto{\pgfqpoint{1.876741in}{1.228057in}}%
\pgfpathlineto{\pgfqpoint{1.873675in}{1.221671in}}%
\pgfpathlineto{\pgfqpoint{1.864485in}{1.229954in}}%
\pgfpathlineto{\pgfqpoint{1.854819in}{1.238074in}}%
\pgfpathlineto{\pgfqpoint{1.844688in}{1.246027in}}%
\pgfpathlineto{\pgfqpoint{1.834104in}{1.253804in}}%
\pgfpathclose%
\pgfusepath{fill}%
\end{pgfscope}%
\begin{pgfscope}%
\pgfpathrectangle{\pgfqpoint{0.329460in}{0.284240in}}{\pgfqpoint{1.989680in}{1.989680in}}%
\pgfusepath{clip}%
\pgfsetbuttcap%
\pgfsetroundjoin%
\definecolor{currentfill}{rgb}{0.231674,0.318106,0.544834}%
\pgfsetfillcolor{currentfill}%
\pgfsetlinewidth{0.000000pt}%
\definecolor{currentstroke}{rgb}{0.000000,0.000000,0.000000}%
\pgfsetstrokecolor{currentstroke}%
\pgfsetdash{}{0pt}%
\pgfpathmoveto{\pgfqpoint{1.306689in}{1.395091in}}%
\pgfpathlineto{\pgfqpoint{1.306304in}{1.390516in}}%
\pgfpathlineto{\pgfqpoint{1.305919in}{1.386000in}}%
\pgfpathlineto{\pgfqpoint{1.305534in}{1.381545in}}%
\pgfpathlineto{\pgfqpoint{1.305150in}{1.377153in}}%
\pgfpathlineto{\pgfqpoint{1.316785in}{1.377781in}}%
\pgfpathlineto{\pgfqpoint{1.328452in}{1.378227in}}%
\pgfpathlineto{\pgfqpoint{1.340140in}{1.378490in}}%
\pgfpathlineto{\pgfqpoint{1.351838in}{1.378572in}}%
\pgfpathlineto{\pgfqpoint{1.351832in}{1.382950in}}%
\pgfpathlineto{\pgfqpoint{1.351827in}{1.387392in}}%
\pgfpathlineto{\pgfqpoint{1.351821in}{1.391896in}}%
\pgfpathlineto{\pgfqpoint{1.351816in}{1.396457in}}%
\pgfpathlineto{\pgfqpoint{1.340509in}{1.396379in}}%
\pgfpathlineto{\pgfqpoint{1.329212in}{1.396125in}}%
\pgfpathlineto{\pgfqpoint{1.317935in}{1.395696in}}%
\pgfpathlineto{\pgfqpoint{1.306689in}{1.395091in}}%
\pgfpathclose%
\pgfusepath{fill}%
\end{pgfscope}%
\begin{pgfscope}%
\pgfpathrectangle{\pgfqpoint{0.329460in}{0.284240in}}{\pgfqpoint{1.989680in}{1.989680in}}%
\pgfusepath{clip}%
\pgfsetbuttcap%
\pgfsetroundjoin%
\definecolor{currentfill}{rgb}{0.231674,0.318106,0.544834}%
\pgfsetfillcolor{currentfill}%
\pgfsetlinewidth{0.000000pt}%
\definecolor{currentstroke}{rgb}{0.000000,0.000000,0.000000}%
\pgfsetstrokecolor{currentstroke}%
\pgfsetdash{}{0pt}%
\pgfpathmoveto{\pgfqpoint{1.351816in}{1.396457in}}%
\pgfpathlineto{\pgfqpoint{1.351821in}{1.391896in}}%
\pgfpathlineto{\pgfqpoint{1.351827in}{1.387392in}}%
\pgfpathlineto{\pgfqpoint{1.351832in}{1.382950in}}%
\pgfpathlineto{\pgfqpoint{1.351838in}{1.378572in}}%
\pgfpathlineto{\pgfqpoint{1.363535in}{1.378470in}}%
\pgfpathlineto{\pgfqpoint{1.375220in}{1.378186in}}%
\pgfpathlineto{\pgfqpoint{1.386884in}{1.377720in}}%
\pgfpathlineto{\pgfqpoint{1.398516in}{1.377072in}}%
\pgfpathlineto{\pgfqpoint{1.398121in}{1.381464in}}%
\pgfpathlineto{\pgfqpoint{1.397725in}{1.385920in}}%
\pgfpathlineto{\pgfqpoint{1.397330in}{1.390437in}}%
\pgfpathlineto{\pgfqpoint{1.396934in}{1.395013in}}%
\pgfpathlineto{\pgfqpoint{1.385691in}{1.395637in}}%
\pgfpathlineto{\pgfqpoint{1.374417in}{1.396086in}}%
\pgfpathlineto{\pgfqpoint{1.363122in}{1.396360in}}%
\pgfpathlineto{\pgfqpoint{1.351816in}{1.396457in}}%
\pgfpathclose%
\pgfusepath{fill}%
\end{pgfscope}%
\begin{pgfscope}%
\pgfpathrectangle{\pgfqpoint{0.329460in}{0.284240in}}{\pgfqpoint{1.989680in}{1.989680in}}%
\pgfusepath{clip}%
\pgfsetbuttcap%
\pgfsetroundjoin%
\definecolor{currentfill}{rgb}{0.277941,0.056324,0.381191}%
\pgfsetfillcolor{currentfill}%
\pgfsetlinewidth{0.000000pt}%
\definecolor{currentstroke}{rgb}{0.000000,0.000000,0.000000}%
\pgfsetstrokecolor{currentstroke}%
\pgfsetdash{}{0pt}%
\pgfpathmoveto{\pgfqpoint{0.845516in}{1.174356in}}%
\pgfpathlineto{\pgfqpoint{0.842474in}{1.178315in}}%
\pgfpathlineto{\pgfqpoint{0.839424in}{1.182555in}}%
\pgfpathlineto{\pgfqpoint{0.836366in}{1.187082in}}%
\pgfpathlineto{\pgfqpoint{0.833299in}{1.191899in}}%
\pgfpathlineto{\pgfqpoint{0.841838in}{1.200152in}}%
\pgfpathlineto{\pgfqpoint{0.850851in}{1.208253in}}%
\pgfpathlineto{\pgfqpoint{0.860328in}{1.216196in}}%
\pgfpathlineto{\pgfqpoint{0.870259in}{1.223973in}}%
\pgfpathlineto{\pgfqpoint{0.873093in}{1.218991in}}%
\pgfpathlineto{\pgfqpoint{0.875919in}{1.214299in}}%
\pgfpathlineto{\pgfqpoint{0.878737in}{1.209891in}}%
\pgfpathlineto{\pgfqpoint{0.881548in}{1.205764in}}%
\pgfpathlineto{\pgfqpoint{0.871864in}{1.198147in}}%
\pgfpathlineto{\pgfqpoint{0.862624in}{1.190369in}}%
\pgfpathlineto{\pgfqpoint{0.853838in}{1.182436in}}%
\pgfpathlineto{\pgfqpoint{0.845516in}{1.174356in}}%
\pgfpathclose%
\pgfusepath{fill}%
\end{pgfscope}%
\begin{pgfscope}%
\pgfpathrectangle{\pgfqpoint{0.329460in}{0.284240in}}{\pgfqpoint{1.989680in}{1.989680in}}%
\pgfusepath{clip}%
\pgfsetbuttcap%
\pgfsetroundjoin%
\definecolor{currentfill}{rgb}{0.274128,0.199721,0.498911}%
\pgfsetfillcolor{currentfill}%
\pgfsetlinewidth{0.000000pt}%
\definecolor{currentstroke}{rgb}{0.000000,0.000000,0.000000}%
\pgfsetstrokecolor{currentstroke}%
\pgfsetdash{}{0pt}%
\pgfpathmoveto{\pgfqpoint{1.576127in}{1.311536in}}%
\pgfpathlineto{\pgfqpoint{1.577887in}{1.307575in}}%
\pgfpathlineto{\pgfqpoint{1.579646in}{1.303709in}}%
\pgfpathlineto{\pgfqpoint{1.581404in}{1.299940in}}%
\pgfpathlineto{\pgfqpoint{1.583162in}{1.296272in}}%
\pgfpathlineto{\pgfqpoint{1.593859in}{1.292518in}}%
\pgfpathlineto{\pgfqpoint{1.604334in}{1.288594in}}%
\pgfpathlineto{\pgfqpoint{1.614576in}{1.284504in}}%
\pgfpathlineto{\pgfqpoint{1.624576in}{1.280250in}}%
\pgfpathlineto{\pgfqpoint{1.622499in}{1.284050in}}%
\pgfpathlineto{\pgfqpoint{1.620421in}{1.287950in}}%
\pgfpathlineto{\pgfqpoint{1.618343in}{1.291948in}}%
\pgfpathlineto{\pgfqpoint{1.616263in}{1.296040in}}%
\pgfpathlineto{\pgfqpoint{1.606573in}{1.300154in}}%
\pgfpathlineto{\pgfqpoint{1.596647in}{1.304110in}}%
\pgfpathlineto{\pgfqpoint{1.586495in}{1.307905in}}%
\pgfpathlineto{\pgfqpoint{1.576127in}{1.311536in}}%
\pgfpathclose%
\pgfusepath{fill}%
\end{pgfscope}%
\begin{pgfscope}%
\pgfpathrectangle{\pgfqpoint{0.329460in}{0.284240in}}{\pgfqpoint{1.989680in}{1.989680in}}%
\pgfusepath{clip}%
\pgfsetbuttcap%
\pgfsetroundjoin%
\definecolor{currentfill}{rgb}{0.201239,0.383670,0.554294}%
\pgfsetfillcolor{currentfill}%
\pgfsetlinewidth{0.000000pt}%
\definecolor{currentstroke}{rgb}{0.000000,0.000000,0.000000}%
\pgfsetstrokecolor{currentstroke}%
\pgfsetdash{}{0pt}%
\pgfpathmoveto{\pgfqpoint{1.829140in}{1.431306in}}%
\pgfpathlineto{\pgfqpoint{1.831838in}{1.443773in}}%
\pgfpathlineto{\pgfqpoint{1.834548in}{1.456655in}}%
\pgfpathlineto{\pgfqpoint{1.837271in}{1.469957in}}%
\pgfpathlineto{\pgfqpoint{1.840008in}{1.483686in}}%
\pgfpathlineto{\pgfqpoint{1.853838in}{1.476079in}}%
\pgfpathlineto{\pgfqpoint{1.867218in}{1.468253in}}%
\pgfpathlineto{\pgfqpoint{1.880135in}{1.460214in}}%
\pgfpathlineto{\pgfqpoint{1.892575in}{1.451969in}}%
\pgfpathlineto{\pgfqpoint{1.889528in}{1.438346in}}%
\pgfpathlineto{\pgfqpoint{1.886496in}{1.425153in}}%
\pgfpathlineto{\pgfqpoint{1.883479in}{1.412382in}}%
\pgfpathlineto{\pgfqpoint{1.880476in}{1.400028in}}%
\pgfpathlineto{\pgfqpoint{1.868330in}{1.408158in}}%
\pgfpathlineto{\pgfqpoint{1.855717in}{1.416086in}}%
\pgfpathlineto{\pgfqpoint{1.842649in}{1.423804in}}%
\pgfpathlineto{\pgfqpoint{1.829140in}{1.431306in}}%
\pgfpathclose%
\pgfusepath{fill}%
\end{pgfscope}%
\begin{pgfscope}%
\pgfpathrectangle{\pgfqpoint{0.329460in}{0.284240in}}{\pgfqpoint{1.989680in}{1.989680in}}%
\pgfusepath{clip}%
\pgfsetbuttcap%
\pgfsetroundjoin%
\definecolor{currentfill}{rgb}{0.248629,0.278775,0.534556}%
\pgfsetfillcolor{currentfill}%
\pgfsetlinewidth{0.000000pt}%
\definecolor{currentstroke}{rgb}{0.000000,0.000000,0.000000}%
\pgfsetstrokecolor{currentstroke}%
\pgfsetdash{}{0pt}%
\pgfpathmoveto{\pgfqpoint{1.478145in}{1.367510in}}%
\pgfpathlineto{\pgfqpoint{1.479206in}{1.363098in}}%
\pgfpathlineto{\pgfqpoint{1.480266in}{1.358756in}}%
\pgfpathlineto{\pgfqpoint{1.481325in}{1.354487in}}%
\pgfpathlineto{\pgfqpoint{1.482383in}{1.350294in}}%
\pgfpathlineto{\pgfqpoint{1.493747in}{1.348148in}}%
\pgfpathlineto{\pgfqpoint{1.504980in}{1.345824in}}%
\pgfpathlineto{\pgfqpoint{1.516074in}{1.343324in}}%
\pgfpathlineto{\pgfqpoint{1.527016in}{1.340650in}}%
\pgfpathlineto{\pgfqpoint{1.525595in}{1.344928in}}%
\pgfpathlineto{\pgfqpoint{1.524174in}{1.349281in}}%
\pgfpathlineto{\pgfqpoint{1.522751in}{1.353707in}}%
\pgfpathlineto{\pgfqpoint{1.521328in}{1.358204in}}%
\pgfpathlineto{\pgfqpoint{1.510741in}{1.360784in}}%
\pgfpathlineto{\pgfqpoint{1.500009in}{1.363196in}}%
\pgfpathlineto{\pgfqpoint{1.489140in}{1.365439in}}%
\pgfpathlineto{\pgfqpoint{1.478145in}{1.367510in}}%
\pgfpathclose%
\pgfusepath{fill}%
\end{pgfscope}%
\begin{pgfscope}%
\pgfpathrectangle{\pgfqpoint{0.329460in}{0.284240in}}{\pgfqpoint{1.989680in}{1.989680in}}%
\pgfusepath{clip}%
\pgfsetbuttcap%
\pgfsetroundjoin%
\definecolor{currentfill}{rgb}{0.267004,0.004874,0.329415}%
\pgfsetfillcolor{currentfill}%
\pgfsetlinewidth{0.000000pt}%
\definecolor{currentstroke}{rgb}{0.000000,0.000000,0.000000}%
\pgfsetstrokecolor{currentstroke}%
\pgfsetdash{}{0pt}%
\pgfpathmoveto{\pgfqpoint{0.881474in}{1.147060in}}%
\pgfpathlineto{\pgfqpoint{0.878510in}{1.148009in}}%
\pgfpathlineto{\pgfqpoint{0.875541in}{1.149184in}}%
\pgfpathlineto{\pgfqpoint{0.872567in}{1.150590in}}%
\pgfpathlineto{\pgfqpoint{0.869587in}{1.152231in}}%
\pgfpathlineto{\pgfqpoint{0.877484in}{1.159954in}}%
\pgfpathlineto{\pgfqpoint{0.885825in}{1.167537in}}%
\pgfpathlineto{\pgfqpoint{0.894600in}{1.174973in}}%
\pgfpathlineto{\pgfqpoint{0.903799in}{1.182255in}}%
\pgfpathlineto{\pgfqpoint{0.906554in}{1.180437in}}%
\pgfpathlineto{\pgfqpoint{0.909304in}{1.178853in}}%
\pgfpathlineto{\pgfqpoint{0.912049in}{1.177499in}}%
\pgfpathlineto{\pgfqpoint{0.914790in}{1.176371in}}%
\pgfpathlineto{\pgfqpoint{0.905829in}{1.169261in}}%
\pgfpathlineto{\pgfqpoint{0.897283in}{1.162002in}}%
\pgfpathlineto{\pgfqpoint{0.889162in}{1.154599in}}%
\pgfpathlineto{\pgfqpoint{0.881474in}{1.147060in}}%
\pgfpathclose%
\pgfusepath{fill}%
\end{pgfscope}%
\begin{pgfscope}%
\pgfpathrectangle{\pgfqpoint{0.329460in}{0.284240in}}{\pgfqpoint{1.989680in}{1.989680in}}%
\pgfusepath{clip}%
\pgfsetbuttcap%
\pgfsetroundjoin%
\definecolor{currentfill}{rgb}{0.263663,0.237631,0.518762}%
\pgfsetfillcolor{currentfill}%
\pgfsetlinewidth{0.000000pt}%
\definecolor{currentstroke}{rgb}{0.000000,0.000000,0.000000}%
\pgfsetstrokecolor{currentstroke}%
\pgfsetdash{}{0pt}%
\pgfpathmoveto{\pgfqpoint{1.124362in}{1.325151in}}%
\pgfpathlineto{\pgfqpoint{1.122525in}{1.320816in}}%
\pgfpathlineto{\pgfqpoint{1.120690in}{1.316563in}}%
\pgfpathlineto{\pgfqpoint{1.118855in}{1.312396in}}%
\pgfpathlineto{\pgfqpoint{1.117022in}{1.308317in}}%
\pgfpathlineto{\pgfqpoint{1.127413in}{1.311929in}}%
\pgfpathlineto{\pgfqpoint{1.138010in}{1.315374in}}%
\pgfpathlineto{\pgfqpoint{1.148802in}{1.318649in}}%
\pgfpathlineto{\pgfqpoint{1.159779in}{1.321750in}}%
\pgfpathlineto{\pgfqpoint{1.161275in}{1.325716in}}%
\pgfpathlineto{\pgfqpoint{1.162771in}{1.329770in}}%
\pgfpathlineto{\pgfqpoint{1.164269in}{1.333908in}}%
\pgfpathlineto{\pgfqpoint{1.165767in}{1.338129in}}%
\pgfpathlineto{\pgfqpoint{1.155136in}{1.335133in}}%
\pgfpathlineto{\pgfqpoint{1.144685in}{1.331969in}}%
\pgfpathlineto{\pgfqpoint{1.134424in}{1.328641in}}%
\pgfpathlineto{\pgfqpoint{1.124362in}{1.325151in}}%
\pgfpathclose%
\pgfusepath{fill}%
\end{pgfscope}%
\begin{pgfscope}%
\pgfpathrectangle{\pgfqpoint{0.329460in}{0.284240in}}{\pgfqpoint{1.989680in}{1.989680in}}%
\pgfusepath{clip}%
\pgfsetbuttcap%
\pgfsetroundjoin%
\definecolor{currentfill}{rgb}{0.268510,0.009605,0.335427}%
\pgfsetfillcolor{currentfill}%
\pgfsetlinewidth{0.000000pt}%
\definecolor{currentstroke}{rgb}{0.000000,0.000000,0.000000}%
\pgfsetstrokecolor{currentstroke}%
\pgfsetdash{}{0pt}%
\pgfpathmoveto{\pgfqpoint{1.757935in}{1.180854in}}%
\pgfpathlineto{\pgfqpoint{1.760591in}{1.180364in}}%
\pgfpathlineto{\pgfqpoint{1.763250in}{1.180067in}}%
\pgfpathlineto{\pgfqpoint{1.765913in}{1.179967in}}%
\pgfpathlineto{\pgfqpoint{1.768579in}{1.180068in}}%
\pgfpathlineto{\pgfqpoint{1.777655in}{1.173268in}}%
\pgfpathlineto{\pgfqpoint{1.786335in}{1.166317in}}%
\pgfpathlineto{\pgfqpoint{1.794608in}{1.159222in}}%
\pgfpathlineto{\pgfqpoint{1.802467in}{1.151988in}}%
\pgfpathlineto{\pgfqpoint{1.799572in}{1.152067in}}%
\pgfpathlineto{\pgfqpoint{1.796681in}{1.152347in}}%
\pgfpathlineto{\pgfqpoint{1.793793in}{1.152825in}}%
\pgfpathlineto{\pgfqpoint{1.790910in}{1.153496in}}%
\pgfpathlineto{\pgfqpoint{1.783265in}{1.160543in}}%
\pgfpathlineto{\pgfqpoint{1.775215in}{1.167456in}}%
\pgfpathlineto{\pgfqpoint{1.766769in}{1.174228in}}%
\pgfpathlineto{\pgfqpoint{1.757935in}{1.180854in}}%
\pgfpathclose%
\pgfusepath{fill}%
\end{pgfscope}%
\begin{pgfscope}%
\pgfpathrectangle{\pgfqpoint{0.329460in}{0.284240in}}{\pgfqpoint{1.989680in}{1.989680in}}%
\pgfusepath{clip}%
\pgfsetbuttcap%
\pgfsetroundjoin%
\definecolor{currentfill}{rgb}{0.279566,0.067836,0.391917}%
\pgfsetfillcolor{currentfill}%
\pgfsetlinewidth{0.000000pt}%
\definecolor{currentstroke}{rgb}{0.000000,0.000000,0.000000}%
\pgfsetstrokecolor{currentstroke}%
\pgfsetdash{}{0pt}%
\pgfpathmoveto{\pgfqpoint{0.968899in}{1.194985in}}%
\pgfpathlineto{\pgfqpoint{0.966214in}{1.192435in}}%
\pgfpathlineto{\pgfqpoint{0.963529in}{1.190034in}}%
\pgfpathlineto{\pgfqpoint{0.960842in}{1.187784in}}%
\pgfpathlineto{\pgfqpoint{0.958153in}{1.185690in}}%
\pgfpathlineto{\pgfqpoint{0.966546in}{1.191944in}}%
\pgfpathlineto{\pgfqpoint{0.975298in}{1.198055in}}%
\pgfpathlineto{\pgfqpoint{0.984399in}{1.204017in}}%
\pgfpathlineto{\pgfqpoint{0.993841in}{1.209825in}}%
\pgfpathlineto{\pgfqpoint{0.996276in}{1.211747in}}%
\pgfpathlineto{\pgfqpoint{0.998710in}{1.213824in}}%
\pgfpathlineto{\pgfqpoint{1.001143in}{1.216052in}}%
\pgfpathlineto{\pgfqpoint{1.003574in}{1.218429in}}%
\pgfpathlineto{\pgfqpoint{0.994399in}{1.212787in}}%
\pgfpathlineto{\pgfqpoint{0.985555in}{1.206996in}}%
\pgfpathlineto{\pgfqpoint{0.977052in}{1.201060in}}%
\pgfpathlineto{\pgfqpoint{0.968899in}{1.194985in}}%
\pgfpathclose%
\pgfusepath{fill}%
\end{pgfscope}%
\begin{pgfscope}%
\pgfpathrectangle{\pgfqpoint{0.329460in}{0.284240in}}{\pgfqpoint{1.989680in}{1.989680in}}%
\pgfusepath{clip}%
\pgfsetbuttcap%
\pgfsetroundjoin%
\definecolor{currentfill}{rgb}{0.282327,0.094955,0.417331}%
\pgfsetfillcolor{currentfill}%
\pgfsetlinewidth{0.000000pt}%
\definecolor{currentstroke}{rgb}{0.000000,0.000000,0.000000}%
\pgfsetstrokecolor{currentstroke}%
\pgfsetdash{}{0pt}%
\pgfpathmoveto{\pgfqpoint{0.833299in}{1.191899in}}%
\pgfpathlineto{\pgfqpoint{0.830223in}{1.197013in}}%
\pgfpathlineto{\pgfqpoint{0.827138in}{1.202427in}}%
\pgfpathlineto{\pgfqpoint{0.824043in}{1.208148in}}%
\pgfpathlineto{\pgfqpoint{0.820939in}{1.214180in}}%
\pgfpathlineto{\pgfqpoint{0.829697in}{1.222599in}}%
\pgfpathlineto{\pgfqpoint{0.838941in}{1.230864in}}%
\pgfpathlineto{\pgfqpoint{0.848659in}{1.238966in}}%
\pgfpathlineto{\pgfqpoint{0.858841in}{1.246900in}}%
\pgfpathlineto{\pgfqpoint{0.861708in}{1.240708in}}%
\pgfpathlineto{\pgfqpoint{0.864567in}{1.234827in}}%
\pgfpathlineto{\pgfqpoint{0.867417in}{1.229250in}}%
\pgfpathlineto{\pgfqpoint{0.870259in}{1.223973in}}%
\pgfpathlineto{\pgfqpoint{0.860328in}{1.216196in}}%
\pgfpathlineto{\pgfqpoint{0.850851in}{1.208253in}}%
\pgfpathlineto{\pgfqpoint{0.841838in}{1.200152in}}%
\pgfpathlineto{\pgfqpoint{0.833299in}{1.191899in}}%
\pgfpathclose%
\pgfusepath{fill}%
\end{pgfscope}%
\begin{pgfscope}%
\pgfpathrectangle{\pgfqpoint{0.329460in}{0.284240in}}{\pgfqpoint{1.989680in}{1.989680in}}%
\pgfusepath{clip}%
\pgfsetbuttcap%
\pgfsetroundjoin%
\definecolor{currentfill}{rgb}{0.231674,0.318106,0.544834}%
\pgfsetfillcolor{currentfill}%
\pgfsetlinewidth{0.000000pt}%
\definecolor{currentstroke}{rgb}{0.000000,0.000000,0.000000}%
\pgfsetstrokecolor{currentstroke}%
\pgfsetdash{}{0pt}%
\pgfpathmoveto{\pgfqpoint{1.262210in}{1.390930in}}%
\pgfpathlineto{\pgfqpoint{1.261439in}{1.386317in}}%
\pgfpathlineto{\pgfqpoint{1.260669in}{1.381761in}}%
\pgfpathlineto{\pgfqpoint{1.259899in}{1.377267in}}%
\pgfpathlineto{\pgfqpoint{1.259130in}{1.372836in}}%
\pgfpathlineto{\pgfqpoint{1.270536in}{1.374185in}}%
\pgfpathlineto{\pgfqpoint{1.282014in}{1.375355in}}%
\pgfpathlineto{\pgfqpoint{1.293556in}{1.376344in}}%
\pgfpathlineto{\pgfqpoint{1.305150in}{1.377153in}}%
\pgfpathlineto{\pgfqpoint{1.305534in}{1.381545in}}%
\pgfpathlineto{\pgfqpoint{1.305919in}{1.386000in}}%
\pgfpathlineto{\pgfqpoint{1.306304in}{1.390516in}}%
\pgfpathlineto{\pgfqpoint{1.306689in}{1.395091in}}%
\pgfpathlineto{\pgfqpoint{1.295483in}{1.394311in}}%
\pgfpathlineto{\pgfqpoint{1.284328in}{1.393357in}}%
\pgfpathlineto{\pgfqpoint{1.273233in}{1.392230in}}%
\pgfpathlineto{\pgfqpoint{1.262210in}{1.390930in}}%
\pgfpathclose%
\pgfusepath{fill}%
\end{pgfscope}%
\begin{pgfscope}%
\pgfpathrectangle{\pgfqpoint{0.329460in}{0.284240in}}{\pgfqpoint{1.989680in}{1.989680in}}%
\pgfusepath{clip}%
\pgfsetbuttcap%
\pgfsetroundjoin%
\definecolor{currentfill}{rgb}{0.231674,0.318106,0.544834}%
\pgfsetfillcolor{currentfill}%
\pgfsetlinewidth{0.000000pt}%
\definecolor{currentstroke}{rgb}{0.000000,0.000000,0.000000}%
\pgfsetstrokecolor{currentstroke}%
\pgfsetdash{}{0pt}%
\pgfpathmoveto{\pgfqpoint{1.396934in}{1.395013in}}%
\pgfpathlineto{\pgfqpoint{1.397330in}{1.390437in}}%
\pgfpathlineto{\pgfqpoint{1.397725in}{1.385920in}}%
\pgfpathlineto{\pgfqpoint{1.398121in}{1.381464in}}%
\pgfpathlineto{\pgfqpoint{1.398516in}{1.377072in}}%
\pgfpathlineto{\pgfqpoint{1.410104in}{1.376243in}}%
\pgfpathlineto{\pgfqpoint{1.421639in}{1.375233in}}%
\pgfpathlineto{\pgfqpoint{1.433110in}{1.374044in}}%
\pgfpathlineto{\pgfqpoint{1.444507in}{1.372676in}}%
\pgfpathlineto{\pgfqpoint{1.443727in}{1.377107in}}%
\pgfpathlineto{\pgfqpoint{1.442947in}{1.381603in}}%
\pgfpathlineto{\pgfqpoint{1.442166in}{1.386160in}}%
\pgfpathlineto{\pgfqpoint{1.441385in}{1.390775in}}%
\pgfpathlineto{\pgfqpoint{1.430370in}{1.392094in}}%
\pgfpathlineto{\pgfqpoint{1.419283in}{1.393240in}}%
\pgfpathlineto{\pgfqpoint{1.408134in}{1.394214in}}%
\pgfpathlineto{\pgfqpoint{1.396934in}{1.395013in}}%
\pgfpathclose%
\pgfusepath{fill}%
\end{pgfscope}%
\begin{pgfscope}%
\pgfpathrectangle{\pgfqpoint{0.329460in}{0.284240in}}{\pgfqpoint{1.989680in}{1.989680in}}%
\pgfusepath{clip}%
\pgfsetbuttcap%
\pgfsetroundjoin%
\definecolor{currentfill}{rgb}{0.248629,0.278775,0.534556}%
\pgfsetfillcolor{currentfill}%
\pgfsetlinewidth{0.000000pt}%
\definecolor{currentstroke}{rgb}{0.000000,0.000000,0.000000}%
\pgfsetstrokecolor{currentstroke}%
\pgfsetdash{}{0pt}%
\pgfpathmoveto{\pgfqpoint{1.171768in}{1.355771in}}%
\pgfpathlineto{\pgfqpoint{1.170266in}{1.351253in}}%
\pgfpathlineto{\pgfqpoint{1.168765in}{1.346804in}}%
\pgfpathlineto{\pgfqpoint{1.167266in}{1.342429in}}%
\pgfpathlineto{\pgfqpoint{1.165767in}{1.338129in}}%
\pgfpathlineto{\pgfqpoint{1.176567in}{1.340956in}}%
\pgfpathlineto{\pgfqpoint{1.187527in}{1.343611in}}%
\pgfpathlineto{\pgfqpoint{1.198636in}{1.346091in}}%
\pgfpathlineto{\pgfqpoint{1.209885in}{1.348395in}}%
\pgfpathlineto{\pgfqpoint{1.211025in}{1.352605in}}%
\pgfpathlineto{\pgfqpoint{1.212166in}{1.356891in}}%
\pgfpathlineto{\pgfqpoint{1.213308in}{1.361249in}}%
\pgfpathlineto{\pgfqpoint{1.214451in}{1.365677in}}%
\pgfpathlineto{\pgfqpoint{1.203568in}{1.363454in}}%
\pgfpathlineto{\pgfqpoint{1.192819in}{1.361060in}}%
\pgfpathlineto{\pgfqpoint{1.182216in}{1.358499in}}%
\pgfpathlineto{\pgfqpoint{1.171768in}{1.355771in}}%
\pgfpathclose%
\pgfusepath{fill}%
\end{pgfscope}%
\begin{pgfscope}%
\pgfpathrectangle{\pgfqpoint{0.329460in}{0.284240in}}{\pgfqpoint{1.989680in}{1.989680in}}%
\pgfusepath{clip}%
\pgfsetbuttcap%
\pgfsetroundjoin%
\definecolor{currentfill}{rgb}{0.274128,0.199721,0.498911}%
\pgfsetfillcolor{currentfill}%
\pgfsetlinewidth{0.000000pt}%
\definecolor{currentstroke}{rgb}{0.000000,0.000000,0.000000}%
\pgfsetstrokecolor{currentstroke}%
\pgfsetdash{}{0pt}%
\pgfpathmoveto{\pgfqpoint{1.077703in}{1.292254in}}%
\pgfpathlineto{\pgfqpoint{1.075557in}{1.288129in}}%
\pgfpathlineto{\pgfqpoint{1.073411in}{1.284099in}}%
\pgfpathlineto{\pgfqpoint{1.071266in}{1.280167in}}%
\pgfpathlineto{\pgfqpoint{1.069122in}{1.276335in}}%
\pgfpathlineto{\pgfqpoint{1.078898in}{1.280731in}}%
\pgfpathlineto{\pgfqpoint{1.088925in}{1.284966in}}%
\pgfpathlineto{\pgfqpoint{1.099194in}{1.289038in}}%
\pgfpathlineto{\pgfqpoint{1.109694in}{1.292943in}}%
\pgfpathlineto{\pgfqpoint{1.111525in}{1.296639in}}%
\pgfpathlineto{\pgfqpoint{1.113356in}{1.300435in}}%
\pgfpathlineto{\pgfqpoint{1.115189in}{1.304329in}}%
\pgfpathlineto{\pgfqpoint{1.117022in}{1.308317in}}%
\pgfpathlineto{\pgfqpoint{1.106845in}{1.304540in}}%
\pgfpathlineto{\pgfqpoint{1.096894in}{1.300601in}}%
\pgfpathlineto{\pgfqpoint{1.087176in}{1.296505in}}%
\pgfpathlineto{\pgfqpoint{1.077703in}{1.292254in}}%
\pgfpathclose%
\pgfusepath{fill}%
\end{pgfscope}%
\begin{pgfscope}%
\pgfpathrectangle{\pgfqpoint{0.329460in}{0.284240in}}{\pgfqpoint{1.989680in}{1.989680in}}%
\pgfusepath{clip}%
\pgfsetbuttcap%
\pgfsetroundjoin%
\definecolor{currentfill}{rgb}{0.267004,0.004874,0.329415}%
\pgfsetfillcolor{currentfill}%
\pgfsetlinewidth{0.000000pt}%
\definecolor{currentstroke}{rgb}{0.000000,0.000000,0.000000}%
\pgfsetstrokecolor{currentstroke}%
\pgfsetdash{}{0pt}%
\pgfpathmoveto{\pgfqpoint{0.893279in}{1.145447in}}%
\pgfpathlineto{\pgfqpoint{0.890335in}{1.145531in}}%
\pgfpathlineto{\pgfqpoint{0.887386in}{1.145826in}}%
\pgfpathlineto{\pgfqpoint{0.884432in}{1.146334in}}%
\pgfpathlineto{\pgfqpoint{0.881474in}{1.147060in}}%
\pgfpathlineto{\pgfqpoint{0.889162in}{1.154599in}}%
\pgfpathlineto{\pgfqpoint{0.897283in}{1.162002in}}%
\pgfpathlineto{\pgfqpoint{0.905829in}{1.169261in}}%
\pgfpathlineto{\pgfqpoint{0.914790in}{1.176371in}}%
\pgfpathlineto{\pgfqpoint{0.917526in}{1.175464in}}%
\pgfpathlineto{\pgfqpoint{0.920258in}{1.174775in}}%
\pgfpathlineto{\pgfqpoint{0.922985in}{1.174298in}}%
\pgfpathlineto{\pgfqpoint{0.925709in}{1.174031in}}%
\pgfpathlineto{\pgfqpoint{0.916985in}{1.167097in}}%
\pgfpathlineto{\pgfqpoint{0.908666in}{1.160017in}}%
\pgfpathlineto{\pgfqpoint{0.900761in}{1.152798in}}%
\pgfpathlineto{\pgfqpoint{0.893279in}{1.145447in}}%
\pgfpathclose%
\pgfusepath{fill}%
\end{pgfscope}%
\begin{pgfscope}%
\pgfpathrectangle{\pgfqpoint{0.329460in}{0.284240in}}{\pgfqpoint{1.989680in}{1.989680in}}%
\pgfusepath{clip}%
\pgfsetbuttcap%
\pgfsetroundjoin%
\definecolor{currentfill}{rgb}{0.276194,0.190074,0.493001}%
\pgfsetfillcolor{currentfill}%
\pgfsetlinewidth{0.000000pt}%
\definecolor{currentstroke}{rgb}{0.000000,0.000000,0.000000}%
\pgfsetstrokecolor{currentstroke}%
\pgfsetdash{}{0pt}%
\pgfpathmoveto{\pgfqpoint{1.845435in}{1.281907in}}%
\pgfpathlineto{\pgfqpoint{1.848292in}{1.289761in}}%
\pgfpathlineto{\pgfqpoint{1.851160in}{1.297956in}}%
\pgfpathlineto{\pgfqpoint{1.854038in}{1.306500in}}%
\pgfpathlineto{\pgfqpoint{1.856928in}{1.315397in}}%
\pgfpathlineto{\pgfqpoint{1.868043in}{1.307332in}}%
\pgfpathlineto{\pgfqpoint{1.878685in}{1.299084in}}%
\pgfpathlineto{\pgfqpoint{1.888842in}{1.290661in}}%
\pgfpathlineto{\pgfqpoint{1.898502in}{1.282069in}}%
\pgfpathlineto{\pgfqpoint{1.895359in}{1.273315in}}%
\pgfpathlineto{\pgfqpoint{1.892227in}{1.264917in}}%
\pgfpathlineto{\pgfqpoint{1.889108in}{1.256869in}}%
\pgfpathlineto{\pgfqpoint{1.885999in}{1.249164in}}%
\pgfpathlineto{\pgfqpoint{1.876576in}{1.257604in}}%
\pgfpathlineto{\pgfqpoint{1.866667in}{1.265879in}}%
\pgfpathlineto{\pgfqpoint{1.856283in}{1.273982in}}%
\pgfpathlineto{\pgfqpoint{1.845435in}{1.281907in}}%
\pgfpathclose%
\pgfusepath{fill}%
\end{pgfscope}%
\begin{pgfscope}%
\pgfpathrectangle{\pgfqpoint{0.329460in}{0.284240in}}{\pgfqpoint{1.989680in}{1.989680in}}%
\pgfusepath{clip}%
\pgfsetbuttcap%
\pgfsetroundjoin%
\definecolor{currentfill}{rgb}{0.280255,0.165693,0.476498}%
\pgfsetfillcolor{currentfill}%
\pgfsetlinewidth{0.000000pt}%
\definecolor{currentstroke}{rgb}{0.000000,0.000000,0.000000}%
\pgfsetstrokecolor{currentstroke}%
\pgfsetdash{}{0pt}%
\pgfpathmoveto{\pgfqpoint{1.624576in}{1.280250in}}%
\pgfpathlineto{\pgfqpoint{1.626653in}{1.276555in}}%
\pgfpathlineto{\pgfqpoint{1.628729in}{1.272966in}}%
\pgfpathlineto{\pgfqpoint{1.630805in}{1.269489in}}%
\pgfpathlineto{\pgfqpoint{1.632881in}{1.266124in}}%
\pgfpathlineto{\pgfqpoint{1.642931in}{1.261567in}}%
\pgfpathlineto{\pgfqpoint{1.652712in}{1.256848in}}%
\pgfpathlineto{\pgfqpoint{1.662215in}{1.251973in}}%
\pgfpathlineto{\pgfqpoint{1.671429in}{1.246944in}}%
\pgfpathlineto{\pgfqpoint{1.669062in}{1.250460in}}%
\pgfpathlineto{\pgfqpoint{1.666695in}{1.254090in}}%
\pgfpathlineto{\pgfqpoint{1.664328in}{1.257831in}}%
\pgfpathlineto{\pgfqpoint{1.661960in}{1.261679in}}%
\pgfpathlineto{\pgfqpoint{1.653025in}{1.266548in}}%
\pgfpathlineto{\pgfqpoint{1.643810in}{1.271268in}}%
\pgfpathlineto{\pgfqpoint{1.634324in}{1.275837in}}%
\pgfpathlineto{\pgfqpoint{1.624576in}{1.280250in}}%
\pgfpathclose%
\pgfusepath{fill}%
\end{pgfscope}%
\begin{pgfscope}%
\pgfpathrectangle{\pgfqpoint{0.329460in}{0.284240in}}{\pgfqpoint{1.989680in}{1.989680in}}%
\pgfusepath{clip}%
\pgfsetbuttcap%
\pgfsetroundjoin%
\definecolor{currentfill}{rgb}{0.282327,0.094955,0.417331}%
\pgfsetfillcolor{currentfill}%
\pgfsetlinewidth{0.000000pt}%
\definecolor{currentstroke}{rgb}{0.000000,0.000000,0.000000}%
\pgfsetstrokecolor{currentstroke}%
\pgfsetdash{}{0pt}%
\pgfpathmoveto{\pgfqpoint{1.680899in}{1.234084in}}%
\pgfpathlineto{\pgfqpoint{1.683267in}{1.231187in}}%
\pgfpathlineto{\pgfqpoint{1.685635in}{1.228424in}}%
\pgfpathlineto{\pgfqpoint{1.688005in}{1.225799in}}%
\pgfpathlineto{\pgfqpoint{1.690375in}{1.223314in}}%
\pgfpathlineto{\pgfqpoint{1.699836in}{1.217809in}}%
\pgfpathlineto{\pgfqpoint{1.708975in}{1.212151in}}%
\pgfpathlineto{\pgfqpoint{1.717782in}{1.206344in}}%
\pgfpathlineto{\pgfqpoint{1.726246in}{1.200392in}}%
\pgfpathlineto{\pgfqpoint{1.723616in}{1.203047in}}%
\pgfpathlineto{\pgfqpoint{1.720987in}{1.205842in}}%
\pgfpathlineto{\pgfqpoint{1.718359in}{1.208776in}}%
\pgfpathlineto{\pgfqpoint{1.715731in}{1.211843in}}%
\pgfpathlineto{\pgfqpoint{1.707513in}{1.217618in}}%
\pgfpathlineto{\pgfqpoint{1.698962in}{1.223252in}}%
\pgfpathlineto{\pgfqpoint{1.690088in}{1.228743in}}%
\pgfpathlineto{\pgfqpoint{1.680899in}{1.234084in}}%
\pgfpathclose%
\pgfusepath{fill}%
\end{pgfscope}%
\begin{pgfscope}%
\pgfpathrectangle{\pgfqpoint{0.329460in}{0.284240in}}{\pgfqpoint{1.989680in}{1.989680in}}%
\pgfusepath{clip}%
\pgfsetbuttcap%
\pgfsetroundjoin%
\definecolor{currentfill}{rgb}{0.282884,0.135920,0.453427}%
\pgfsetfillcolor{currentfill}%
\pgfsetlinewidth{0.000000pt}%
\definecolor{currentstroke}{rgb}{0.000000,0.000000,0.000000}%
\pgfsetstrokecolor{currentstroke}%
\pgfsetdash{}{0pt}%
\pgfpathmoveto{\pgfqpoint{0.820939in}{1.214180in}}%
\pgfpathlineto{\pgfqpoint{0.817824in}{1.220529in}}%
\pgfpathlineto{\pgfqpoint{0.814699in}{1.227199in}}%
\pgfpathlineto{\pgfqpoint{0.811563in}{1.234198in}}%
\pgfpathlineto{\pgfqpoint{0.808416in}{1.241529in}}%
\pgfpathlineto{\pgfqpoint{0.817398in}{1.250110in}}%
\pgfpathlineto{\pgfqpoint{0.826876in}{1.258532in}}%
\pgfpathlineto{\pgfqpoint{0.836839in}{1.266788in}}%
\pgfpathlineto{\pgfqpoint{0.847275in}{1.274872in}}%
\pgfpathlineto{\pgfqpoint{0.850181in}{1.267387in}}%
\pgfpathlineto{\pgfqpoint{0.853077in}{1.260234in}}%
\pgfpathlineto{\pgfqpoint{0.855964in}{1.253406in}}%
\pgfpathlineto{\pgfqpoint{0.858841in}{1.246900in}}%
\pgfpathlineto{\pgfqpoint{0.848659in}{1.238966in}}%
\pgfpathlineto{\pgfqpoint{0.838941in}{1.230864in}}%
\pgfpathlineto{\pgfqpoint{0.829697in}{1.222599in}}%
\pgfpathlineto{\pgfqpoint{0.820939in}{1.214180in}}%
\pgfpathclose%
\pgfusepath{fill}%
\end{pgfscope}%
\begin{pgfscope}%
\pgfpathrectangle{\pgfqpoint{0.329460in}{0.284240in}}{\pgfqpoint{1.989680in}{1.989680in}}%
\pgfusepath{clip}%
\pgfsetbuttcap%
\pgfsetroundjoin%
\definecolor{currentfill}{rgb}{0.201239,0.383670,0.554294}%
\pgfsetfillcolor{currentfill}%
\pgfsetlinewidth{0.000000pt}%
\definecolor{currentstroke}{rgb}{0.000000,0.000000,0.000000}%
\pgfsetstrokecolor{currentstroke}%
\pgfsetdash{}{0pt}%
\pgfpathmoveto{\pgfqpoint{0.811506in}{1.392637in}}%
\pgfpathlineto{\pgfqpoint{0.808440in}{1.404964in}}%
\pgfpathlineto{\pgfqpoint{0.805360in}{1.417707in}}%
\pgfpathlineto{\pgfqpoint{0.802265in}{1.430875in}}%
\pgfpathlineto{\pgfqpoint{0.799154in}{1.444472in}}%
\pgfpathlineto{\pgfqpoint{0.811159in}{1.452895in}}%
\pgfpathlineto{\pgfqpoint{0.823652in}{1.461117in}}%
\pgfpathlineto{\pgfqpoint{0.836621in}{1.469133in}}%
\pgfpathlineto{\pgfqpoint{0.850052in}{1.476935in}}%
\pgfpathlineto{\pgfqpoint{0.852860in}{1.463228in}}%
\pgfpathlineto{\pgfqpoint{0.855656in}{1.449949in}}%
\pgfpathlineto{\pgfqpoint{0.858437in}{1.437091in}}%
\pgfpathlineto{\pgfqpoint{0.861206in}{1.424648in}}%
\pgfpathlineto{\pgfqpoint{0.848088in}{1.416954in}}%
\pgfpathlineto{\pgfqpoint{0.835424in}{1.409049in}}%
\pgfpathlineto{\pgfqpoint{0.823225in}{1.400941in}}%
\pgfpathlineto{\pgfqpoint{0.811506in}{1.392637in}}%
\pgfpathclose%
\pgfusepath{fill}%
\end{pgfscope}%
\begin{pgfscope}%
\pgfpathrectangle{\pgfqpoint{0.329460in}{0.284240in}}{\pgfqpoint{1.989680in}{1.989680in}}%
\pgfusepath{clip}%
\pgfsetbuttcap%
\pgfsetroundjoin%
\definecolor{currentfill}{rgb}{0.271305,0.019942,0.347269}%
\pgfsetfillcolor{currentfill}%
\pgfsetlinewidth{0.000000pt}%
\definecolor{currentstroke}{rgb}{0.000000,0.000000,0.000000}%
\pgfsetstrokecolor{currentstroke}%
\pgfsetdash{}{0pt}%
\pgfpathmoveto{\pgfqpoint{1.747339in}{1.184665in}}%
\pgfpathlineto{\pgfqpoint{1.749984in}{1.183442in}}%
\pgfpathlineto{\pgfqpoint{1.752632in}{1.182397in}}%
\pgfpathlineto{\pgfqpoint{1.755282in}{1.181533in}}%
\pgfpathlineto{\pgfqpoint{1.757935in}{1.180854in}}%
\pgfpathlineto{\pgfqpoint{1.766769in}{1.174228in}}%
\pgfpathlineto{\pgfqpoint{1.775215in}{1.167456in}}%
\pgfpathlineto{\pgfqpoint{1.783265in}{1.160543in}}%
\pgfpathlineto{\pgfqpoint{1.790910in}{1.153496in}}%
\pgfpathlineto{\pgfqpoint{1.788029in}{1.154357in}}%
\pgfpathlineto{\pgfqpoint{1.785152in}{1.155404in}}%
\pgfpathlineto{\pgfqpoint{1.782277in}{1.156632in}}%
\pgfpathlineto{\pgfqpoint{1.779406in}{1.158039in}}%
\pgfpathlineto{\pgfqpoint{1.771974in}{1.164897in}}%
\pgfpathlineto{\pgfqpoint{1.764146in}{1.171625in}}%
\pgfpathlineto{\pgfqpoint{1.755932in}{1.178216in}}%
\pgfpathlineto{\pgfqpoint{1.747339in}{1.184665in}}%
\pgfpathclose%
\pgfusepath{fill}%
\end{pgfscope}%
\begin{pgfscope}%
\pgfpathrectangle{\pgfqpoint{0.329460in}{0.284240in}}{\pgfqpoint{1.989680in}{1.989680in}}%
\pgfusepath{clip}%
\pgfsetbuttcap%
\pgfsetroundjoin%
\definecolor{currentfill}{rgb}{0.231674,0.318106,0.544834}%
\pgfsetfillcolor{currentfill}%
\pgfsetlinewidth{0.000000pt}%
\definecolor{currentstroke}{rgb}{0.000000,0.000000,0.000000}%
\pgfsetstrokecolor{currentstroke}%
\pgfsetdash{}{0pt}%
\pgfpathmoveto{\pgfqpoint{1.441385in}{1.390775in}}%
\pgfpathlineto{\pgfqpoint{1.442166in}{1.386160in}}%
\pgfpathlineto{\pgfqpoint{1.442947in}{1.381603in}}%
\pgfpathlineto{\pgfqpoint{1.443727in}{1.377107in}}%
\pgfpathlineto{\pgfqpoint{1.444507in}{1.372676in}}%
\pgfpathlineto{\pgfqpoint{1.455818in}{1.371130in}}%
\pgfpathlineto{\pgfqpoint{1.467035in}{1.369407in}}%
\pgfpathlineto{\pgfqpoint{1.478145in}{1.367510in}}%
\pgfpathlineto{\pgfqpoint{1.477084in}{1.371988in}}%
\pgfpathlineto{\pgfqpoint{1.476022in}{1.376531in}}%
\pgfpathlineto{\pgfqpoint{1.474958in}{1.381135in}}%
\pgfpathlineto{\pgfqpoint{1.473894in}{1.385797in}}%
\pgfpathlineto{\pgfqpoint{1.463157in}{1.387626in}}%
\pgfpathlineto{\pgfqpoint{1.452317in}{1.389285in}}%
\pgfpathlineto{\pgfqpoint{1.441385in}{1.390775in}}%
\pgfpathclose%
\pgfusepath{fill}%
\end{pgfscope}%
\begin{pgfscope}%
\pgfpathrectangle{\pgfqpoint{0.329460in}{0.284240in}}{\pgfqpoint{1.989680in}{1.989680in}}%
\pgfusepath{clip}%
\pgfsetbuttcap%
\pgfsetroundjoin%
\definecolor{currentfill}{rgb}{0.268510,0.009605,0.335427}%
\pgfsetfillcolor{currentfill}%
\pgfsetlinewidth{0.000000pt}%
\definecolor{currentstroke}{rgb}{0.000000,0.000000,0.000000}%
\pgfsetstrokecolor{currentstroke}%
\pgfsetdash{}{0pt}%
\pgfpathmoveto{\pgfqpoint{0.905018in}{1.147124in}}%
\pgfpathlineto{\pgfqpoint{0.902089in}{1.146411in}}%
\pgfpathlineto{\pgfqpoint{0.899156in}{1.145890in}}%
\pgfpathlineto{\pgfqpoint{0.896220in}{1.145568in}}%
\pgfpathlineto{\pgfqpoint{0.893279in}{1.145447in}}%
\pgfpathlineto{\pgfqpoint{0.900761in}{1.152798in}}%
\pgfpathlineto{\pgfqpoint{0.908666in}{1.160017in}}%
\pgfpathlineto{\pgfqpoint{0.916985in}{1.167097in}}%
\pgfpathlineto{\pgfqpoint{0.925709in}{1.174031in}}%
\pgfpathlineto{\pgfqpoint{0.928429in}{1.173969in}}%
\pgfpathlineto{\pgfqpoint{0.931146in}{1.174107in}}%
\pgfpathlineto{\pgfqpoint{0.933859in}{1.174443in}}%
\pgfpathlineto{\pgfqpoint{0.936569in}{1.174972in}}%
\pgfpathlineto{\pgfqpoint{0.928079in}{1.168216in}}%
\pgfpathlineto{\pgfqpoint{0.919985in}{1.161318in}}%
\pgfpathlineto{\pgfqpoint{0.912295in}{1.154286in}}%
\pgfpathlineto{\pgfqpoint{0.905018in}{1.147124in}}%
\pgfpathclose%
\pgfusepath{fill}%
\end{pgfscope}%
\begin{pgfscope}%
\pgfpathrectangle{\pgfqpoint{0.329460in}{0.284240in}}{\pgfqpoint{1.989680in}{1.989680in}}%
\pgfusepath{clip}%
\pgfsetbuttcap%
\pgfsetroundjoin%
\definecolor{currentfill}{rgb}{0.231674,0.318106,0.544834}%
\pgfsetfillcolor{currentfill}%
\pgfsetlinewidth{0.000000pt}%
\definecolor{currentstroke}{rgb}{0.000000,0.000000,0.000000}%
\pgfsetstrokecolor{currentstroke}%
\pgfsetdash{}{0pt}%
\pgfpathmoveto{\pgfqpoint{1.219030in}{1.384031in}}%
\pgfpathlineto{\pgfqpoint{1.217884in}{1.379353in}}%
\pgfpathlineto{\pgfqpoint{1.216739in}{1.374732in}}%
\pgfpathlineto{\pgfqpoint{1.215594in}{1.370173in}}%
\pgfpathlineto{\pgfqpoint{1.214451in}{1.365677in}}%
\pgfpathlineto{\pgfqpoint{1.225459in}{1.367729in}}%
\pgfpathlineto{\pgfqpoint{1.236582in}{1.369607in}}%
\pgfpathlineto{\pgfqpoint{1.247809in}{1.371310in}}%
\pgfpathlineto{\pgfqpoint{1.259130in}{1.372836in}}%
\pgfpathlineto{\pgfqpoint{1.259899in}{1.377267in}}%
\pgfpathlineto{\pgfqpoint{1.260669in}{1.381761in}}%
\pgfpathlineto{\pgfqpoint{1.261439in}{1.386317in}}%
\pgfpathlineto{\pgfqpoint{1.262210in}{1.390930in}}%
\pgfpathlineto{\pgfqpoint{1.251268in}{1.389459in}}%
\pgfpathlineto{\pgfqpoint{1.240418in}{1.387818in}}%
\pgfpathlineto{\pgfqpoint{1.229668in}{1.386009in}}%
\pgfpathlineto{\pgfqpoint{1.219030in}{1.384031in}}%
\pgfpathclose%
\pgfusepath{fill}%
\end{pgfscope}%
\begin{pgfscope}%
\pgfpathrectangle{\pgfqpoint{0.329460in}{0.284240in}}{\pgfqpoint{1.989680in}{1.989680in}}%
\pgfusepath{clip}%
\pgfsetbuttcap%
\pgfsetroundjoin%
\definecolor{currentfill}{rgb}{0.280255,0.165693,0.476498}%
\pgfsetfillcolor{currentfill}%
\pgfsetlinewidth{0.000000pt}%
\definecolor{currentstroke}{rgb}{0.000000,0.000000,0.000000}%
\pgfsetstrokecolor{currentstroke}%
\pgfsetdash{}{0pt}%
\pgfpathmoveto{\pgfqpoint{1.032716in}{1.257231in}}%
\pgfpathlineto{\pgfqpoint{1.030288in}{1.253346in}}%
\pgfpathlineto{\pgfqpoint{1.027860in}{1.249569in}}%
\pgfpathlineto{\pgfqpoint{1.025432in}{1.245902in}}%
\pgfpathlineto{\pgfqpoint{1.023004in}{1.242349in}}%
\pgfpathlineto{\pgfqpoint{1.031955in}{1.247510in}}%
\pgfpathlineto{\pgfqpoint{1.041202in}{1.252522in}}%
\pgfpathlineto{\pgfqpoint{1.050736in}{1.257381in}}%
\pgfpathlineto{\pgfqpoint{1.060548in}{1.262081in}}%
\pgfpathlineto{\pgfqpoint{1.062691in}{1.265478in}}%
\pgfpathlineto{\pgfqpoint{1.064834in}{1.268987in}}%
\pgfpathlineto{\pgfqpoint{1.066978in}{1.272608in}}%
\pgfpathlineto{\pgfqpoint{1.069122in}{1.276335in}}%
\pgfpathlineto{\pgfqpoint{1.059606in}{1.271784in}}%
\pgfpathlineto{\pgfqpoint{1.050361in}{1.267080in}}%
\pgfpathlineto{\pgfqpoint{1.041394in}{1.262227in}}%
\pgfpathlineto{\pgfqpoint{1.032716in}{1.257231in}}%
\pgfpathclose%
\pgfusepath{fill}%
\end{pgfscope}%
\begin{pgfscope}%
\pgfpathrectangle{\pgfqpoint{0.329460in}{0.284240in}}{\pgfqpoint{1.989680in}{1.989680in}}%
\pgfusepath{clip}%
\pgfsetbuttcap%
\pgfsetroundjoin%
\definecolor{currentfill}{rgb}{0.282327,0.094955,0.417331}%
\pgfsetfillcolor{currentfill}%
\pgfsetlinewidth{0.000000pt}%
\definecolor{currentstroke}{rgb}{0.000000,0.000000,0.000000}%
\pgfsetstrokecolor{currentstroke}%
\pgfsetdash{}{0pt}%
\pgfpathmoveto{\pgfqpoint{0.979626in}{1.206598in}}%
\pgfpathlineto{\pgfqpoint{0.976945in}{1.203490in}}%
\pgfpathlineto{\pgfqpoint{0.974264in}{1.200516in}}%
\pgfpathlineto{\pgfqpoint{0.971582in}{1.197680in}}%
\pgfpathlineto{\pgfqpoint{0.968899in}{1.194985in}}%
\pgfpathlineto{\pgfqpoint{0.977052in}{1.201060in}}%
\pgfpathlineto{\pgfqpoint{0.985555in}{1.206996in}}%
\pgfpathlineto{\pgfqpoint{0.994399in}{1.212787in}}%
\pgfpathlineto{\pgfqpoint{1.003574in}{1.218429in}}%
\pgfpathlineto{\pgfqpoint{1.006005in}{1.220949in}}%
\pgfpathlineto{\pgfqpoint{1.008435in}{1.223611in}}%
\pgfpathlineto{\pgfqpoint{1.010864in}{1.226411in}}%
\pgfpathlineto{\pgfqpoint{1.013293in}{1.229344in}}%
\pgfpathlineto{\pgfqpoint{1.004383in}{1.223870in}}%
\pgfpathlineto{\pgfqpoint{0.995796in}{1.218251in}}%
\pgfpathlineto{\pgfqpoint{0.987541in}{1.212492in}}%
\pgfpathlineto{\pgfqpoint{0.979626in}{1.206598in}}%
\pgfpathclose%
\pgfusepath{fill}%
\end{pgfscope}%
\begin{pgfscope}%
\pgfpathrectangle{\pgfqpoint{0.329460in}{0.284240in}}{\pgfqpoint{1.989680in}{1.989680in}}%
\pgfusepath{clip}%
\pgfsetbuttcap%
\pgfsetroundjoin%
\definecolor{currentfill}{rgb}{0.172719,0.448791,0.557885}%
\pgfsetfillcolor{currentfill}%
\pgfsetlinewidth{0.000000pt}%
\definecolor{currentstroke}{rgb}{0.000000,0.000000,0.000000}%
\pgfsetstrokecolor{currentstroke}%
\pgfsetdash{}{0pt}%
\pgfpathmoveto{\pgfqpoint{1.840008in}{1.483686in}}%
\pgfpathlineto{\pgfqpoint{1.842759in}{1.497851in}}%
\pgfpathlineto{\pgfqpoint{1.845523in}{1.512458in}}%
\pgfpathlineto{\pgfqpoint{1.848302in}{1.527513in}}%
\pgfpathlineto{\pgfqpoint{1.862378in}{1.519833in}}%
\pgfpathlineto{\pgfqpoint{1.875997in}{1.511931in}}%
\pgfpathlineto{\pgfqpoint{1.889146in}{1.503814in}}%
\pgfpathlineto{\pgfqpoint{1.901810in}{1.495489in}}%
\pgfpathlineto{\pgfqpoint{1.898715in}{1.480533in}}%
\pgfpathlineto{\pgfqpoint{1.895637in}{1.466029in}}%
\pgfpathlineto{\pgfqpoint{1.892575in}{1.451969in}}%
\pgfpathlineto{\pgfqpoint{1.880135in}{1.460214in}}%
\pgfpathlineto{\pgfqpoint{1.867218in}{1.468253in}}%
\pgfpathlineto{\pgfqpoint{1.853838in}{1.476079in}}%
\pgfpathlineto{\pgfqpoint{1.840008in}{1.483686in}}%
\pgfpathclose%
\pgfusepath{fill}%
\end{pgfscope}%
\begin{pgfscope}%
\pgfpathrectangle{\pgfqpoint{0.329460in}{0.284240in}}{\pgfqpoint{1.989680in}{1.989680in}}%
\pgfusepath{clip}%
\pgfsetbuttcap%
\pgfsetroundjoin%
\definecolor{currentfill}{rgb}{0.248629,0.278775,0.534556}%
\pgfsetfillcolor{currentfill}%
\pgfsetlinewidth{0.000000pt}%
\definecolor{currentstroke}{rgb}{0.000000,0.000000,0.000000}%
\pgfsetstrokecolor{currentstroke}%
\pgfsetdash{}{0pt}%
\pgfpathmoveto{\pgfqpoint{1.521328in}{1.358204in}}%
\pgfpathlineto{\pgfqpoint{1.522751in}{1.353707in}}%
\pgfpathlineto{\pgfqpoint{1.524174in}{1.349281in}}%
\pgfpathlineto{\pgfqpoint{1.525595in}{1.344928in}}%
\pgfpathlineto{\pgfqpoint{1.527016in}{1.340650in}}%
\pgfpathlineto{\pgfqpoint{1.537798in}{1.337804in}}%
\pgfpathlineto{\pgfqpoint{1.548409in}{1.334789in}}%
\pgfpathlineto{\pgfqpoint{1.558840in}{1.331607in}}%
\pgfpathlineto{\pgfqpoint{1.569079in}{1.328261in}}%
\pgfpathlineto{\pgfqpoint{1.567315in}{1.332647in}}%
\pgfpathlineto{\pgfqpoint{1.565550in}{1.337109in}}%
\pgfpathlineto{\pgfqpoint{1.563784in}{1.341645in}}%
\pgfpathlineto{\pgfqpoint{1.562016in}{1.346250in}}%
\pgfpathlineto{\pgfqpoint{1.552112in}{1.349479in}}%
\pgfpathlineto{\pgfqpoint{1.542023in}{1.352549in}}%
\pgfpathlineto{\pgfqpoint{1.531758in}{1.355458in}}%
\pgfpathlineto{\pgfqpoint{1.521328in}{1.358204in}}%
\pgfpathclose%
\pgfusepath{fill}%
\end{pgfscope}%
\begin{pgfscope}%
\pgfpathrectangle{\pgfqpoint{0.329460in}{0.284240in}}{\pgfqpoint{1.989680in}{1.989680in}}%
\pgfusepath{clip}%
\pgfsetbuttcap%
\pgfsetroundjoin%
\definecolor{currentfill}{rgb}{0.263663,0.237631,0.518762}%
\pgfsetfillcolor{currentfill}%
\pgfsetlinewidth{0.000000pt}%
\definecolor{currentstroke}{rgb}{0.000000,0.000000,0.000000}%
\pgfsetstrokecolor{currentstroke}%
\pgfsetdash{}{0pt}%
\pgfpathmoveto{\pgfqpoint{1.569079in}{1.328261in}}%
\pgfpathlineto{\pgfqpoint{1.570843in}{1.323954in}}%
\pgfpathlineto{\pgfqpoint{1.572605in}{1.319728in}}%
\pgfpathlineto{\pgfqpoint{1.574366in}{1.315588in}}%
\pgfpathlineto{\pgfqpoint{1.576127in}{1.311536in}}%
\pgfpathlineto{\pgfqpoint{1.586495in}{1.307905in}}%
\pgfpathlineto{\pgfqpoint{1.596647in}{1.304110in}}%
\pgfpathlineto{\pgfqpoint{1.606573in}{1.300154in}}%
\pgfpathlineto{\pgfqpoint{1.616263in}{1.296040in}}%
\pgfpathlineto{\pgfqpoint{1.614184in}{1.300224in}}%
\pgfpathlineto{\pgfqpoint{1.612103in}{1.304496in}}%
\pgfpathlineto{\pgfqpoint{1.610021in}{1.308853in}}%
\pgfpathlineto{\pgfqpoint{1.607938in}{1.313292in}}%
\pgfpathlineto{\pgfqpoint{1.598558in}{1.317266in}}%
\pgfpathlineto{\pgfqpoint{1.588948in}{1.321087in}}%
\pgfpathlineto{\pgfqpoint{1.579119in}{1.324753in}}%
\pgfpathlineto{\pgfqpoint{1.569079in}{1.328261in}}%
\pgfpathclose%
\pgfusepath{fill}%
\end{pgfscope}%
\begin{pgfscope}%
\pgfpathrectangle{\pgfqpoint{0.329460in}{0.284240in}}{\pgfqpoint{1.989680in}{1.989680in}}%
\pgfusepath{clip}%
\pgfsetbuttcap%
\pgfsetroundjoin%
\definecolor{currentfill}{rgb}{0.276194,0.190074,0.493001}%
\pgfsetfillcolor{currentfill}%
\pgfsetlinewidth{0.000000pt}%
\definecolor{currentstroke}{rgb}{0.000000,0.000000,0.000000}%
\pgfsetstrokecolor{currentstroke}%
\pgfsetdash{}{0pt}%
\pgfpathmoveto{\pgfqpoint{0.808416in}{1.241529in}}%
\pgfpathlineto{\pgfqpoint{0.805258in}{1.249199in}}%
\pgfpathlineto{\pgfqpoint{0.802088in}{1.257213in}}%
\pgfpathlineto{\pgfqpoint{0.798906in}{1.265577in}}%
\pgfpathlineto{\pgfqpoint{0.795711in}{1.274296in}}%
\pgfpathlineto{\pgfqpoint{0.804921in}{1.283032in}}%
\pgfpathlineto{\pgfqpoint{0.814638in}{1.291605in}}%
\pgfpathlineto{\pgfqpoint{0.824849in}{1.300009in}}%
\pgfpathlineto{\pgfqpoint{0.835544in}{1.308237in}}%
\pgfpathlineto{\pgfqpoint{0.838493in}{1.299371in}}%
\pgfpathlineto{\pgfqpoint{0.841431in}{1.290858in}}%
\pgfpathlineto{\pgfqpoint{0.844359in}{1.282694in}}%
\pgfpathlineto{\pgfqpoint{0.847275in}{1.274872in}}%
\pgfpathlineto{\pgfqpoint{0.836839in}{1.266788in}}%
\pgfpathlineto{\pgfqpoint{0.826876in}{1.258532in}}%
\pgfpathlineto{\pgfqpoint{0.817398in}{1.250110in}}%
\pgfpathlineto{\pgfqpoint{0.808416in}{1.241529in}}%
\pgfpathclose%
\pgfusepath{fill}%
\end{pgfscope}%
\begin{pgfscope}%
\pgfpathrectangle{\pgfqpoint{0.329460in}{0.284240in}}{\pgfqpoint{1.989680in}{1.989680in}}%
\pgfusepath{clip}%
\pgfsetbuttcap%
\pgfsetroundjoin%
\definecolor{currentfill}{rgb}{0.260571,0.246922,0.522828}%
\pgfsetfillcolor{currentfill}%
\pgfsetlinewidth{0.000000pt}%
\definecolor{currentstroke}{rgb}{0.000000,0.000000,0.000000}%
\pgfsetstrokecolor{currentstroke}%
\pgfsetdash{}{0pt}%
\pgfpathmoveto{\pgfqpoint{1.856928in}{1.315397in}}%
\pgfpathlineto{\pgfqpoint{1.859828in}{1.324654in}}%
\pgfpathlineto{\pgfqpoint{1.862741in}{1.334277in}}%
\pgfpathlineto{\pgfqpoint{1.865665in}{1.344271in}}%
\pgfpathlineto{\pgfqpoint{1.868601in}{1.354642in}}%
\pgfpathlineto{\pgfqpoint{1.879989in}{1.346442in}}%
\pgfpathlineto{\pgfqpoint{1.890893in}{1.338057in}}%
\pgfpathlineto{\pgfqpoint{1.901302in}{1.329492in}}%
\pgfpathlineto{\pgfqpoint{1.911205in}{1.320756in}}%
\pgfpathlineto{\pgfqpoint{1.908009in}{1.310521in}}%
\pgfpathlineto{\pgfqpoint{1.904827in}{1.300666in}}%
\pgfpathlineto{\pgfqpoint{1.901658in}{1.291184in}}%
\pgfpathlineto{\pgfqpoint{1.898502in}{1.282069in}}%
\pgfpathlineto{\pgfqpoint{1.888842in}{1.290661in}}%
\pgfpathlineto{\pgfqpoint{1.878685in}{1.299084in}}%
\pgfpathlineto{\pgfqpoint{1.868043in}{1.307332in}}%
\pgfpathlineto{\pgfqpoint{1.856928in}{1.315397in}}%
\pgfpathclose%
\pgfusepath{fill}%
\end{pgfscope}%
\begin{pgfscope}%
\pgfpathrectangle{\pgfqpoint{0.329460in}{0.284240in}}{\pgfqpoint{1.989680in}{1.989680in}}%
\pgfusepath{clip}%
\pgfsetbuttcap%
\pgfsetroundjoin%
\definecolor{currentfill}{rgb}{0.271305,0.019942,0.347269}%
\pgfsetfillcolor{currentfill}%
\pgfsetlinewidth{0.000000pt}%
\definecolor{currentstroke}{rgb}{0.000000,0.000000,0.000000}%
\pgfsetstrokecolor{currentstroke}%
\pgfsetdash{}{0pt}%
\pgfpathmoveto{\pgfqpoint{0.916701in}{1.151838in}}%
\pgfpathlineto{\pgfqpoint{0.913785in}{1.150389in}}%
\pgfpathlineto{\pgfqpoint{0.910866in}{1.149117in}}%
\pgfpathlineto{\pgfqpoint{0.907943in}{1.148028in}}%
\pgfpathlineto{\pgfqpoint{0.905018in}{1.147124in}}%
\pgfpathlineto{\pgfqpoint{0.912295in}{1.154286in}}%
\pgfpathlineto{\pgfqpoint{0.919985in}{1.161318in}}%
\pgfpathlineto{\pgfqpoint{0.928079in}{1.168216in}}%
\pgfpathlineto{\pgfqpoint{0.936569in}{1.174972in}}%
\pgfpathlineto{\pgfqpoint{0.939276in}{1.175690in}}%
\pgfpathlineto{\pgfqpoint{0.941980in}{1.176593in}}%
\pgfpathlineto{\pgfqpoint{0.944681in}{1.177677in}}%
\pgfpathlineto{\pgfqpoint{0.947380in}{1.178940in}}%
\pgfpathlineto{\pgfqpoint{0.939123in}{1.172364in}}%
\pgfpathlineto{\pgfqpoint{0.931252in}{1.165651in}}%
\pgfpathlineto{\pgfqpoint{0.923775in}{1.158807in}}%
\pgfpathlineto{\pgfqpoint{0.916701in}{1.151838in}}%
\pgfpathclose%
\pgfusepath{fill}%
\end{pgfscope}%
\begin{pgfscope}%
\pgfpathrectangle{\pgfqpoint{0.329460in}{0.284240in}}{\pgfqpoint{1.989680in}{1.989680in}}%
\pgfusepath{clip}%
\pgfsetbuttcap%
\pgfsetroundjoin%
\definecolor{currentfill}{rgb}{0.274952,0.037752,0.364543}%
\pgfsetfillcolor{currentfill}%
\pgfsetlinewidth{0.000000pt}%
\definecolor{currentstroke}{rgb}{0.000000,0.000000,0.000000}%
\pgfsetstrokecolor{currentstroke}%
\pgfsetdash{}{0pt}%
\pgfpathmoveto{\pgfqpoint{1.736779in}{1.191256in}}%
\pgfpathlineto{\pgfqpoint{1.739416in}{1.189361in}}%
\pgfpathlineto{\pgfqpoint{1.742055in}{1.187628in}}%
\pgfpathlineto{\pgfqpoint{1.744696in}{1.186062in}}%
\pgfpathlineto{\pgfqpoint{1.747339in}{1.184665in}}%
\pgfpathlineto{\pgfqpoint{1.755932in}{1.178216in}}%
\pgfpathlineto{\pgfqpoint{1.764146in}{1.171625in}}%
\pgfpathlineto{\pgfqpoint{1.771974in}{1.164897in}}%
\pgfpathlineto{\pgfqpoint{1.779406in}{1.158039in}}%
\pgfpathlineto{\pgfqpoint{1.776537in}{1.159620in}}%
\pgfpathlineto{\pgfqpoint{1.773671in}{1.161371in}}%
\pgfpathlineto{\pgfqpoint{1.770807in}{1.163289in}}%
\pgfpathlineto{\pgfqpoint{1.767945in}{1.165370in}}%
\pgfpathlineto{\pgfqpoint{1.760723in}{1.172037in}}%
\pgfpathlineto{\pgfqpoint{1.753116in}{1.178578in}}%
\pgfpathlineto{\pgfqpoint{1.745132in}{1.184986in}}%
\pgfpathlineto{\pgfqpoint{1.736779in}{1.191256in}}%
\pgfpathclose%
\pgfusepath{fill}%
\end{pgfscope}%
\begin{pgfscope}%
\pgfpathrectangle{\pgfqpoint{0.329460in}{0.284240in}}{\pgfqpoint{1.989680in}{1.989680in}}%
\pgfusepath{clip}%
\pgfsetbuttcap%
\pgfsetroundjoin%
\definecolor{currentfill}{rgb}{0.248629,0.278775,0.534556}%
\pgfsetfillcolor{currentfill}%
\pgfsetlinewidth{0.000000pt}%
\definecolor{currentstroke}{rgb}{0.000000,0.000000,0.000000}%
\pgfsetstrokecolor{currentstroke}%
\pgfsetdash{}{0pt}%
\pgfpathmoveto{\pgfqpoint{1.131718in}{1.343250in}}%
\pgfpathlineto{\pgfqpoint{1.129877in}{1.338617in}}%
\pgfpathlineto{\pgfqpoint{1.128038in}{1.334054in}}%
\pgfpathlineto{\pgfqpoint{1.126199in}{1.329564in}}%
\pgfpathlineto{\pgfqpoint{1.124362in}{1.325151in}}%
\pgfpathlineto{\pgfqpoint{1.134424in}{1.328641in}}%
\pgfpathlineto{\pgfqpoint{1.144685in}{1.331969in}}%
\pgfpathlineto{\pgfqpoint{1.155136in}{1.335133in}}%
\pgfpathlineto{\pgfqpoint{1.165767in}{1.338129in}}%
\pgfpathlineto{\pgfqpoint{1.167266in}{1.342429in}}%
\pgfpathlineto{\pgfqpoint{1.168765in}{1.346804in}}%
\pgfpathlineto{\pgfqpoint{1.170266in}{1.351253in}}%
\pgfpathlineto{\pgfqpoint{1.171768in}{1.355771in}}%
\pgfpathlineto{\pgfqpoint{1.161484in}{1.352880in}}%
\pgfpathlineto{\pgfqpoint{1.151375in}{1.349828in}}%
\pgfpathlineto{\pgfqpoint{1.141450in}{1.346617in}}%
\pgfpathlineto{\pgfqpoint{1.131718in}{1.343250in}}%
\pgfpathclose%
\pgfusepath{fill}%
\end{pgfscope}%
\begin{pgfscope}%
\pgfpathrectangle{\pgfqpoint{0.329460in}{0.284240in}}{\pgfqpoint{1.989680in}{1.989680in}}%
\pgfusepath{clip}%
\pgfsetbuttcap%
\pgfsetroundjoin%
\definecolor{currentfill}{rgb}{0.212395,0.359683,0.551710}%
\pgfsetfillcolor{currentfill}%
\pgfsetlinewidth{0.000000pt}%
\definecolor{currentstroke}{rgb}{0.000000,0.000000,0.000000}%
\pgfsetstrokecolor{currentstroke}%
\pgfsetdash{}{0pt}%
\pgfpathmoveto{\pgfqpoint{1.308233in}{1.413914in}}%
\pgfpathlineto{\pgfqpoint{1.307847in}{1.409135in}}%
\pgfpathlineto{\pgfqpoint{1.307460in}{1.404403in}}%
\pgfpathlineto{\pgfqpoint{1.307074in}{1.399721in}}%
\pgfpathlineto{\pgfqpoint{1.306689in}{1.395091in}}%
\pgfpathlineto{\pgfqpoint{1.317935in}{1.395696in}}%
\pgfpathlineto{\pgfqpoint{1.329212in}{1.396125in}}%
\pgfpathlineto{\pgfqpoint{1.340509in}{1.396379in}}%
\pgfpathlineto{\pgfqpoint{1.351816in}{1.396457in}}%
\pgfpathlineto{\pgfqpoint{1.351810in}{1.401075in}}%
\pgfpathlineto{\pgfqpoint{1.351805in}{1.405744in}}%
\pgfpathlineto{\pgfqpoint{1.351799in}{1.410463in}}%
\pgfpathlineto{\pgfqpoint{1.351794in}{1.415230in}}%
\pgfpathlineto{\pgfqpoint{1.340880in}{1.415154in}}%
\pgfpathlineto{\pgfqpoint{1.329975in}{1.414910in}}%
\pgfpathlineto{\pgfqpoint{1.319090in}{1.414496in}}%
\pgfpathlineto{\pgfqpoint{1.308233in}{1.413914in}}%
\pgfpathclose%
\pgfusepath{fill}%
\end{pgfscope}%
\begin{pgfscope}%
\pgfpathrectangle{\pgfqpoint{0.329460in}{0.284240in}}{\pgfqpoint{1.989680in}{1.989680in}}%
\pgfusepath{clip}%
\pgfsetbuttcap%
\pgfsetroundjoin%
\definecolor{currentfill}{rgb}{0.212395,0.359683,0.551710}%
\pgfsetfillcolor{currentfill}%
\pgfsetlinewidth{0.000000pt}%
\definecolor{currentstroke}{rgb}{0.000000,0.000000,0.000000}%
\pgfsetstrokecolor{currentstroke}%
\pgfsetdash{}{0pt}%
\pgfpathmoveto{\pgfqpoint{1.351794in}{1.415230in}}%
\pgfpathlineto{\pgfqpoint{1.351799in}{1.410463in}}%
\pgfpathlineto{\pgfqpoint{1.351805in}{1.405744in}}%
\pgfpathlineto{\pgfqpoint{1.351810in}{1.401075in}}%
\pgfpathlineto{\pgfqpoint{1.351816in}{1.396457in}}%
\pgfpathlineto{\pgfqpoint{1.363122in}{1.396360in}}%
\pgfpathlineto{\pgfqpoint{1.374417in}{1.396086in}}%
\pgfpathlineto{\pgfqpoint{1.385691in}{1.395637in}}%
\pgfpathlineto{\pgfqpoint{1.396934in}{1.395013in}}%
\pgfpathlineto{\pgfqpoint{1.396537in}{1.399643in}}%
\pgfpathlineto{\pgfqpoint{1.396140in}{1.404327in}}%
\pgfpathlineto{\pgfqpoint{1.395743in}{1.409059in}}%
\pgfpathlineto{\pgfqpoint{1.395346in}{1.413839in}}%
\pgfpathlineto{\pgfqpoint{1.384493in}{1.414440in}}%
\pgfpathlineto{\pgfqpoint{1.373611in}{1.414872in}}%
\pgfpathlineto{\pgfqpoint{1.362708in}{1.415136in}}%
\pgfpathlineto{\pgfqpoint{1.351794in}{1.415230in}}%
\pgfpathclose%
\pgfusepath{fill}%
\end{pgfscope}%
\begin{pgfscope}%
\pgfpathrectangle{\pgfqpoint{0.329460in}{0.284240in}}{\pgfqpoint{1.989680in}{1.989680in}}%
\pgfusepath{clip}%
\pgfsetbuttcap%
\pgfsetroundjoin%
\definecolor{currentfill}{rgb}{0.231674,0.318106,0.544834}%
\pgfsetfillcolor{currentfill}%
\pgfsetlinewidth{0.000000pt}%
\definecolor{currentstroke}{rgb}{0.000000,0.000000,0.000000}%
\pgfsetstrokecolor{currentstroke}%
\pgfsetdash{}{0pt}%
\pgfpathmoveto{\pgfqpoint{1.473894in}{1.385797in}}%
\pgfpathlineto{\pgfqpoint{1.474958in}{1.381135in}}%
\pgfpathlineto{\pgfqpoint{1.476022in}{1.376531in}}%
\pgfpathlineto{\pgfqpoint{1.477084in}{1.371988in}}%
\pgfpathlineto{\pgfqpoint{1.478145in}{1.367510in}}%
\pgfpathlineto{\pgfqpoint{1.489140in}{1.365439in}}%
\pgfpathlineto{\pgfqpoint{1.500009in}{1.363196in}}%
\pgfpathlineto{\pgfqpoint{1.510741in}{1.360784in}}%
\pgfpathlineto{\pgfqpoint{1.521328in}{1.358204in}}%
\pgfpathlineto{\pgfqpoint{1.519903in}{1.362767in}}%
\pgfpathlineto{\pgfqpoint{1.518477in}{1.367394in}}%
\pgfpathlineto{\pgfqpoint{1.517051in}{1.372083in}}%
\pgfpathlineto{\pgfqpoint{1.515623in}{1.376830in}}%
\pgfpathlineto{\pgfqpoint{1.505393in}{1.379316in}}%
\pgfpathlineto{\pgfqpoint{1.495022in}{1.381641in}}%
\pgfpathlineto{\pgfqpoint{1.484519in}{1.383802in}}%
\pgfpathlineto{\pgfqpoint{1.473894in}{1.385797in}}%
\pgfpathclose%
\pgfusepath{fill}%
\end{pgfscope}%
\begin{pgfscope}%
\pgfpathrectangle{\pgfqpoint{0.329460in}{0.284240in}}{\pgfqpoint{1.989680in}{1.989680in}}%
\pgfusepath{clip}%
\pgfsetbuttcap%
\pgfsetroundjoin%
\definecolor{currentfill}{rgb}{0.283072,0.130895,0.449241}%
\pgfsetfillcolor{currentfill}%
\pgfsetlinewidth{0.000000pt}%
\definecolor{currentstroke}{rgb}{0.000000,0.000000,0.000000}%
\pgfsetstrokecolor{currentstroke}%
\pgfsetdash{}{0pt}%
\pgfpathmoveto{\pgfqpoint{1.671429in}{1.246944in}}%
\pgfpathlineto{\pgfqpoint{1.673796in}{1.243545in}}%
\pgfpathlineto{\pgfqpoint{1.676164in}{1.240267in}}%
\pgfpathlineto{\pgfqpoint{1.678531in}{1.237112in}}%
\pgfpathlineto{\pgfqpoint{1.680899in}{1.234084in}}%
\pgfpathlineto{\pgfqpoint{1.690088in}{1.228743in}}%
\pgfpathlineto{\pgfqpoint{1.698962in}{1.223252in}}%
\pgfpathlineto{\pgfqpoint{1.707513in}{1.217618in}}%
\pgfpathlineto{\pgfqpoint{1.715731in}{1.211843in}}%
\pgfpathlineto{\pgfqpoint{1.713104in}{1.215042in}}%
\pgfpathlineto{\pgfqpoint{1.710477in}{1.218368in}}%
\pgfpathlineto{\pgfqpoint{1.707851in}{1.221818in}}%
\pgfpathlineto{\pgfqpoint{1.705225in}{1.225389in}}%
\pgfpathlineto{\pgfqpoint{1.697253in}{1.230984in}}%
\pgfpathlineto{\pgfqpoint{1.688958in}{1.236445in}}%
\pgfpathlineto{\pgfqpoint{1.680347in}{1.241767in}}%
\pgfpathlineto{\pgfqpoint{1.671429in}{1.246944in}}%
\pgfpathclose%
\pgfusepath{fill}%
\end{pgfscope}%
\begin{pgfscope}%
\pgfpathrectangle{\pgfqpoint{0.329460in}{0.284240in}}{\pgfqpoint{1.989680in}{1.989680in}}%
\pgfusepath{clip}%
\pgfsetbuttcap%
\pgfsetroundjoin%
\definecolor{currentfill}{rgb}{0.263663,0.237631,0.518762}%
\pgfsetfillcolor{currentfill}%
\pgfsetlinewidth{0.000000pt}%
\definecolor{currentstroke}{rgb}{0.000000,0.000000,0.000000}%
\pgfsetstrokecolor{currentstroke}%
\pgfsetdash{}{0pt}%
\pgfpathmoveto{\pgfqpoint{1.086297in}{1.309635in}}%
\pgfpathlineto{\pgfqpoint{1.084147in}{1.305163in}}%
\pgfpathlineto{\pgfqpoint{1.081999in}{1.300774in}}%
\pgfpathlineto{\pgfqpoint{1.079850in}{1.296470in}}%
\pgfpathlineto{\pgfqpoint{1.077703in}{1.292254in}}%
\pgfpathlineto{\pgfqpoint{1.087176in}{1.296505in}}%
\pgfpathlineto{\pgfqpoint{1.096894in}{1.300601in}}%
\pgfpathlineto{\pgfqpoint{1.106845in}{1.304540in}}%
\pgfpathlineto{\pgfqpoint{1.117022in}{1.308317in}}%
\pgfpathlineto{\pgfqpoint{1.118855in}{1.312396in}}%
\pgfpathlineto{\pgfqpoint{1.120690in}{1.316563in}}%
\pgfpathlineto{\pgfqpoint{1.122525in}{1.320816in}}%
\pgfpathlineto{\pgfqpoint{1.124362in}{1.325151in}}%
\pgfpathlineto{\pgfqpoint{1.114509in}{1.321502in}}%
\pgfpathlineto{\pgfqpoint{1.104874in}{1.317698in}}%
\pgfpathlineto{\pgfqpoint{1.095467in}{1.313741in}}%
\pgfpathlineto{\pgfqpoint{1.086297in}{1.309635in}}%
\pgfpathclose%
\pgfusepath{fill}%
\end{pgfscope}%
\begin{pgfscope}%
\pgfpathrectangle{\pgfqpoint{0.329460in}{0.284240in}}{\pgfqpoint{1.989680in}{1.989680in}}%
\pgfusepath{clip}%
\pgfsetbuttcap%
\pgfsetroundjoin%
\definecolor{currentfill}{rgb}{0.231674,0.318106,0.544834}%
\pgfsetfillcolor{currentfill}%
\pgfsetlinewidth{0.000000pt}%
\definecolor{currentstroke}{rgb}{0.000000,0.000000,0.000000}%
\pgfsetstrokecolor{currentstroke}%
\pgfsetdash{}{0pt}%
\pgfpathmoveto{\pgfqpoint{1.177786in}{1.374486in}}%
\pgfpathlineto{\pgfqpoint{1.176280in}{1.369717in}}%
\pgfpathlineto{\pgfqpoint{1.174775in}{1.365006in}}%
\pgfpathlineto{\pgfqpoint{1.173271in}{1.360357in}}%
\pgfpathlineto{\pgfqpoint{1.171768in}{1.355771in}}%
\pgfpathlineto{\pgfqpoint{1.182216in}{1.358499in}}%
\pgfpathlineto{\pgfqpoint{1.192819in}{1.361060in}}%
\pgfpathlineto{\pgfqpoint{1.203568in}{1.363454in}}%
\pgfpathlineto{\pgfqpoint{1.214451in}{1.365677in}}%
\pgfpathlineto{\pgfqpoint{1.215594in}{1.370173in}}%
\pgfpathlineto{\pgfqpoint{1.216739in}{1.374732in}}%
\pgfpathlineto{\pgfqpoint{1.217884in}{1.379353in}}%
\pgfpathlineto{\pgfqpoint{1.219030in}{1.384031in}}%
\pgfpathlineto{\pgfqpoint{1.208513in}{1.381889in}}%
\pgfpathlineto{\pgfqpoint{1.198127in}{1.379582in}}%
\pgfpathlineto{\pgfqpoint{1.187882in}{1.377114in}}%
\pgfpathlineto{\pgfqpoint{1.177786in}{1.374486in}}%
\pgfpathclose%
\pgfusepath{fill}%
\end{pgfscope}%
\begin{pgfscope}%
\pgfpathrectangle{\pgfqpoint{0.329460in}{0.284240in}}{\pgfqpoint{1.989680in}{1.989680in}}%
\pgfusepath{clip}%
\pgfsetbuttcap%
\pgfsetroundjoin%
\definecolor{currentfill}{rgb}{0.274128,0.199721,0.498911}%
\pgfsetfillcolor{currentfill}%
\pgfsetlinewidth{0.000000pt}%
\definecolor{currentstroke}{rgb}{0.000000,0.000000,0.000000}%
\pgfsetstrokecolor{currentstroke}%
\pgfsetdash{}{0pt}%
\pgfpathmoveto{\pgfqpoint{1.616263in}{1.296040in}}%
\pgfpathlineto{\pgfqpoint{1.618343in}{1.291948in}}%
\pgfpathlineto{\pgfqpoint{1.620421in}{1.287950in}}%
\pgfpathlineto{\pgfqpoint{1.622499in}{1.284050in}}%
\pgfpathlineto{\pgfqpoint{1.624576in}{1.280250in}}%
\pgfpathlineto{\pgfqpoint{1.634324in}{1.275837in}}%
\pgfpathlineto{\pgfqpoint{1.643810in}{1.271268in}}%
\pgfpathlineto{\pgfqpoint{1.653025in}{1.266548in}}%
\pgfpathlineto{\pgfqpoint{1.661960in}{1.261679in}}%
\pgfpathlineto{\pgfqpoint{1.659592in}{1.265631in}}%
\pgfpathlineto{\pgfqpoint{1.657223in}{1.269684in}}%
\pgfpathlineto{\pgfqpoint{1.654853in}{1.273835in}}%
\pgfpathlineto{\pgfqpoint{1.652483in}{1.278081in}}%
\pgfpathlineto{\pgfqpoint{1.643828in}{1.282789in}}%
\pgfpathlineto{\pgfqpoint{1.634900in}{1.287354in}}%
\pgfpathlineto{\pgfqpoint{1.625709in}{1.291772in}}%
\pgfpathlineto{\pgfqpoint{1.616263in}{1.296040in}}%
\pgfpathclose%
\pgfusepath{fill}%
\end{pgfscope}%
\begin{pgfscope}%
\pgfpathrectangle{\pgfqpoint{0.329460in}{0.284240in}}{\pgfqpoint{1.989680in}{1.989680in}}%
\pgfusepath{clip}%
\pgfsetbuttcap%
\pgfsetroundjoin%
\definecolor{currentfill}{rgb}{0.212395,0.359683,0.551710}%
\pgfsetfillcolor{currentfill}%
\pgfsetlinewidth{0.000000pt}%
\definecolor{currentstroke}{rgb}{0.000000,0.000000,0.000000}%
\pgfsetstrokecolor{currentstroke}%
\pgfsetdash{}{0pt}%
\pgfpathmoveto{\pgfqpoint{1.265301in}{1.409911in}}%
\pgfpathlineto{\pgfqpoint{1.264527in}{1.405093in}}%
\pgfpathlineto{\pgfqpoint{1.263754in}{1.400321in}}%
\pgfpathlineto{\pgfqpoint{1.262982in}{1.395599in}}%
\pgfpathlineto{\pgfqpoint{1.262210in}{1.390930in}}%
\pgfpathlineto{\pgfqpoint{1.273233in}{1.392230in}}%
\pgfpathlineto{\pgfqpoint{1.284328in}{1.393357in}}%
\pgfpathlineto{\pgfqpoint{1.295483in}{1.394311in}}%
\pgfpathlineto{\pgfqpoint{1.306689in}{1.395091in}}%
\pgfpathlineto{\pgfqpoint{1.307074in}{1.399721in}}%
\pgfpathlineto{\pgfqpoint{1.307460in}{1.404403in}}%
\pgfpathlineto{\pgfqpoint{1.307847in}{1.409135in}}%
\pgfpathlineto{\pgfqpoint{1.308233in}{1.413914in}}%
\pgfpathlineto{\pgfqpoint{1.297417in}{1.413164in}}%
\pgfpathlineto{\pgfqpoint{1.286649in}{1.412246in}}%
\pgfpathlineto{\pgfqpoint{1.275940in}{1.411161in}}%
\pgfpathlineto{\pgfqpoint{1.265301in}{1.409911in}}%
\pgfpathclose%
\pgfusepath{fill}%
\end{pgfscope}%
\begin{pgfscope}%
\pgfpathrectangle{\pgfqpoint{0.329460in}{0.284240in}}{\pgfqpoint{1.989680in}{1.989680in}}%
\pgfusepath{clip}%
\pgfsetbuttcap%
\pgfsetroundjoin%
\definecolor{currentfill}{rgb}{0.212395,0.359683,0.551710}%
\pgfsetfillcolor{currentfill}%
\pgfsetlinewidth{0.000000pt}%
\definecolor{currentstroke}{rgb}{0.000000,0.000000,0.000000}%
\pgfsetstrokecolor{currentstroke}%
\pgfsetdash{}{0pt}%
\pgfpathmoveto{\pgfqpoint{1.395346in}{1.413839in}}%
\pgfpathlineto{\pgfqpoint{1.395743in}{1.409059in}}%
\pgfpathlineto{\pgfqpoint{1.396140in}{1.404327in}}%
\pgfpathlineto{\pgfqpoint{1.396537in}{1.399643in}}%
\pgfpathlineto{\pgfqpoint{1.396934in}{1.395013in}}%
\pgfpathlineto{\pgfqpoint{1.408134in}{1.394214in}}%
\pgfpathlineto{\pgfqpoint{1.419283in}{1.393240in}}%
\pgfpathlineto{\pgfqpoint{1.430370in}{1.392094in}}%
\pgfpathlineto{\pgfqpoint{1.441385in}{1.390775in}}%
\pgfpathlineto{\pgfqpoint{1.440603in}{1.395446in}}%
\pgfpathlineto{\pgfqpoint{1.439820in}{1.400169in}}%
\pgfpathlineto{\pgfqpoint{1.439036in}{1.404942in}}%
\pgfpathlineto{\pgfqpoint{1.438252in}{1.409762in}}%
\pgfpathlineto{\pgfqpoint{1.427621in}{1.411031in}}%
\pgfpathlineto{\pgfqpoint{1.416919in}{1.412134in}}%
\pgfpathlineto{\pgfqpoint{1.406158in}{1.413070in}}%
\pgfpathlineto{\pgfqpoint{1.395346in}{1.413839in}}%
\pgfpathclose%
\pgfusepath{fill}%
\end{pgfscope}%
\begin{pgfscope}%
\pgfpathrectangle{\pgfqpoint{0.329460in}{0.284240in}}{\pgfqpoint{1.989680in}{1.989680in}}%
\pgfusepath{clip}%
\pgfsetbuttcap%
\pgfsetroundjoin%
\definecolor{currentfill}{rgb}{0.172719,0.448791,0.557885}%
\pgfsetfillcolor{currentfill}%
\pgfsetlinewidth{0.000000pt}%
\definecolor{currentstroke}{rgb}{0.000000,0.000000,0.000000}%
\pgfsetstrokecolor{currentstroke}%
\pgfsetdash{}{0pt}%
\pgfpathmoveto{\pgfqpoint{0.799154in}{1.444472in}}%
\pgfpathlineto{\pgfqpoint{0.796028in}{1.458507in}}%
\pgfpathlineto{\pgfqpoint{0.792885in}{1.472987in}}%
\pgfpathlineto{\pgfqpoint{0.789726in}{1.487919in}}%
\pgfpathlineto{\pgfqpoint{0.801948in}{1.496424in}}%
\pgfpathlineto{\pgfqpoint{0.814667in}{1.504726in}}%
\pgfpathlineto{\pgfqpoint{0.827868in}{1.512819in}}%
\pgfpathlineto{\pgfqpoint{0.841539in}{1.520697in}}%
\pgfpathlineto{\pgfqpoint{0.844391in}{1.505663in}}%
\pgfpathlineto{\pgfqpoint{0.847229in}{1.491078in}}%
\pgfpathlineto{\pgfqpoint{0.850052in}{1.476935in}}%
\pgfpathlineto{\pgfqpoint{0.836621in}{1.469133in}}%
\pgfpathlineto{\pgfqpoint{0.823652in}{1.461117in}}%
\pgfpathlineto{\pgfqpoint{0.811159in}{1.452895in}}%
\pgfpathlineto{\pgfqpoint{0.799154in}{1.444472in}}%
\pgfpathclose%
\pgfusepath{fill}%
\end{pgfscope}%
\begin{pgfscope}%
\pgfpathrectangle{\pgfqpoint{0.329460in}{0.284240in}}{\pgfqpoint{1.989680in}{1.989680in}}%
\pgfusepath{clip}%
\pgfsetbuttcap%
\pgfsetroundjoin%
\definecolor{currentfill}{rgb}{0.274952,0.037752,0.364543}%
\pgfsetfillcolor{currentfill}%
\pgfsetlinewidth{0.000000pt}%
\definecolor{currentstroke}{rgb}{0.000000,0.000000,0.000000}%
\pgfsetstrokecolor{currentstroke}%
\pgfsetdash{}{0pt}%
\pgfpathmoveto{\pgfqpoint{0.928342in}{1.159342in}}%
\pgfpathlineto{\pgfqpoint{0.925435in}{1.157218in}}%
\pgfpathlineto{\pgfqpoint{0.922526in}{1.155257in}}%
\pgfpathlineto{\pgfqpoint{0.919615in}{1.153462in}}%
\pgfpathlineto{\pgfqpoint{0.916701in}{1.151838in}}%
\pgfpathlineto{\pgfqpoint{0.923775in}{1.158807in}}%
\pgfpathlineto{\pgfqpoint{0.931252in}{1.165651in}}%
\pgfpathlineto{\pgfqpoint{0.939123in}{1.172364in}}%
\pgfpathlineto{\pgfqpoint{0.947380in}{1.178940in}}%
\pgfpathlineto{\pgfqpoint{0.950076in}{1.180376in}}%
\pgfpathlineto{\pgfqpoint{0.952771in}{1.181982in}}%
\pgfpathlineto{\pgfqpoint{0.955463in}{1.183754in}}%
\pgfpathlineto{\pgfqpoint{0.958153in}{1.185690in}}%
\pgfpathlineto{\pgfqpoint{0.950128in}{1.179296in}}%
\pgfpathlineto{\pgfqpoint{0.942478in}{1.172770in}}%
\pgfpathlineto{\pgfqpoint{0.935214in}{1.166117in}}%
\pgfpathlineto{\pgfqpoint{0.928342in}{1.159342in}}%
\pgfpathclose%
\pgfusepath{fill}%
\end{pgfscope}%
\begin{pgfscope}%
\pgfpathrectangle{\pgfqpoint{0.329460in}{0.284240in}}{\pgfqpoint{1.989680in}{1.989680in}}%
\pgfusepath{clip}%
\pgfsetbuttcap%
\pgfsetroundjoin%
\definecolor{currentfill}{rgb}{0.283072,0.130895,0.449241}%
\pgfsetfillcolor{currentfill}%
\pgfsetlinewidth{0.000000pt}%
\definecolor{currentstroke}{rgb}{0.000000,0.000000,0.000000}%
\pgfsetstrokecolor{currentstroke}%
\pgfsetdash{}{0pt}%
\pgfpathmoveto{\pgfqpoint{0.990342in}{1.220305in}}%
\pgfpathlineto{\pgfqpoint{0.987664in}{1.216694in}}%
\pgfpathlineto{\pgfqpoint{0.984985in}{1.213203in}}%
\pgfpathlineto{\pgfqpoint{0.982305in}{1.209837in}}%
\pgfpathlineto{\pgfqpoint{0.979626in}{1.206598in}}%
\pgfpathlineto{\pgfqpoint{0.987541in}{1.212492in}}%
\pgfpathlineto{\pgfqpoint{0.995796in}{1.218251in}}%
\pgfpathlineto{\pgfqpoint{1.004383in}{1.223870in}}%
\pgfpathlineto{\pgfqpoint{1.013293in}{1.229344in}}%
\pgfpathlineto{\pgfqpoint{1.015721in}{1.232408in}}%
\pgfpathlineto{\pgfqpoint{1.018149in}{1.235599in}}%
\pgfpathlineto{\pgfqpoint{1.020577in}{1.238914in}}%
\pgfpathlineto{\pgfqpoint{1.023004in}{1.242349in}}%
\pgfpathlineto{\pgfqpoint{1.014359in}{1.237044in}}%
\pgfpathlineto{\pgfqpoint{1.006028in}{1.231598in}}%
\pgfpathlineto{\pgfqpoint{0.998019in}{1.226017in}}%
\pgfpathlineto{\pgfqpoint{0.990342in}{1.220305in}}%
\pgfpathclose%
\pgfusepath{fill}%
\end{pgfscope}%
\begin{pgfscope}%
\pgfpathrectangle{\pgfqpoint{0.329460in}{0.284240in}}{\pgfqpoint{1.989680in}{1.989680in}}%
\pgfusepath{clip}%
\pgfsetbuttcap%
\pgfsetroundjoin%
\definecolor{currentfill}{rgb}{0.260571,0.246922,0.522828}%
\pgfsetfillcolor{currentfill}%
\pgfsetlinewidth{0.000000pt}%
\definecolor{currentstroke}{rgb}{0.000000,0.000000,0.000000}%
\pgfsetstrokecolor{currentstroke}%
\pgfsetdash{}{0pt}%
\pgfpathmoveto{\pgfqpoint{0.795711in}{1.274296in}}%
\pgfpathlineto{\pgfqpoint{0.792504in}{1.283378in}}%
\pgfpathlineto{\pgfqpoint{0.789284in}{1.292827in}}%
\pgfpathlineto{\pgfqpoint{0.786050in}{1.302649in}}%
\pgfpathlineto{\pgfqpoint{0.782803in}{1.312852in}}%
\pgfpathlineto{\pgfqpoint{0.792245in}{1.321735in}}%
\pgfpathlineto{\pgfqpoint{0.802204in}{1.330452in}}%
\pgfpathlineto{\pgfqpoint{0.812669in}{1.338997in}}%
\pgfpathlineto{\pgfqpoint{0.823628in}{1.347363in}}%
\pgfpathlineto{\pgfqpoint{0.826625in}{1.337021in}}%
\pgfpathlineto{\pgfqpoint{0.829610in}{1.327056in}}%
\pgfpathlineto{\pgfqpoint{0.832583in}{1.317464in}}%
\pgfpathlineto{\pgfqpoint{0.835544in}{1.308237in}}%
\pgfpathlineto{\pgfqpoint{0.824849in}{1.300009in}}%
\pgfpathlineto{\pgfqpoint{0.814638in}{1.291605in}}%
\pgfpathlineto{\pgfqpoint{0.804921in}{1.283032in}}%
\pgfpathlineto{\pgfqpoint{0.795711in}{1.274296in}}%
\pgfpathclose%
\pgfusepath{fill}%
\end{pgfscope}%
\begin{pgfscope}%
\pgfpathrectangle{\pgfqpoint{0.329460in}{0.284240in}}{\pgfqpoint{1.989680in}{1.989680in}}%
\pgfusepath{clip}%
\pgfsetbuttcap%
\pgfsetroundjoin%
\definecolor{currentfill}{rgb}{0.274128,0.199721,0.498911}%
\pgfsetfillcolor{currentfill}%
\pgfsetlinewidth{0.000000pt}%
\definecolor{currentstroke}{rgb}{0.000000,0.000000,0.000000}%
\pgfsetstrokecolor{currentstroke}%
\pgfsetdash{}{0pt}%
\pgfpathmoveto{\pgfqpoint{1.042434in}{1.273779in}}%
\pgfpathlineto{\pgfqpoint{1.040003in}{1.269497in}}%
\pgfpathlineto{\pgfqpoint{1.037573in}{1.265309in}}%
\pgfpathlineto{\pgfqpoint{1.035144in}{1.261220in}}%
\pgfpathlineto{\pgfqpoint{1.032716in}{1.257231in}}%
\pgfpathlineto{\pgfqpoint{1.041394in}{1.262227in}}%
\pgfpathlineto{\pgfqpoint{1.050361in}{1.267080in}}%
\pgfpathlineto{\pgfqpoint{1.059606in}{1.271784in}}%
\pgfpathlineto{\pgfqpoint{1.069122in}{1.276335in}}%
\pgfpathlineto{\pgfqpoint{1.071266in}{1.280167in}}%
\pgfpathlineto{\pgfqpoint{1.073411in}{1.284099in}}%
\pgfpathlineto{\pgfqpoint{1.075557in}{1.288129in}}%
\pgfpathlineto{\pgfqpoint{1.077703in}{1.292254in}}%
\pgfpathlineto{\pgfqpoint{1.068483in}{1.287852in}}%
\pgfpathlineto{\pgfqpoint{1.059526in}{1.283303in}}%
\pgfpathlineto{\pgfqpoint{1.050840in}{1.278611in}}%
\pgfpathlineto{\pgfqpoint{1.042434in}{1.273779in}}%
\pgfpathclose%
\pgfusepath{fill}%
\end{pgfscope}%
\begin{pgfscope}%
\pgfpathrectangle{\pgfqpoint{0.329460in}{0.284240in}}{\pgfqpoint{1.989680in}{1.989680in}}%
\pgfusepath{clip}%
\pgfsetbuttcap%
\pgfsetroundjoin%
\definecolor{currentfill}{rgb}{0.279566,0.067836,0.391917}%
\pgfsetfillcolor{currentfill}%
\pgfsetlinewidth{0.000000pt}%
\definecolor{currentstroke}{rgb}{0.000000,0.000000,0.000000}%
\pgfsetstrokecolor{currentstroke}%
\pgfsetdash{}{0pt}%
\pgfpathmoveto{\pgfqpoint{1.726246in}{1.200392in}}%
\pgfpathlineto{\pgfqpoint{1.728877in}{1.197882in}}%
\pgfpathlineto{\pgfqpoint{1.731510in}{1.195521in}}%
\pgfpathlineto{\pgfqpoint{1.734144in}{1.193311in}}%
\pgfpathlineto{\pgfqpoint{1.736779in}{1.191256in}}%
\pgfpathlineto{\pgfqpoint{1.745132in}{1.184986in}}%
\pgfpathlineto{\pgfqpoint{1.753116in}{1.178578in}}%
\pgfpathlineto{\pgfqpoint{1.760723in}{1.172037in}}%
\pgfpathlineto{\pgfqpoint{1.767945in}{1.165370in}}%
\pgfpathlineto{\pgfqpoint{1.765085in}{1.167611in}}%
\pgfpathlineto{\pgfqpoint{1.762226in}{1.170007in}}%
\pgfpathlineto{\pgfqpoint{1.759370in}{1.172555in}}%
\pgfpathlineto{\pgfqpoint{1.756515in}{1.175252in}}%
\pgfpathlineto{\pgfqpoint{1.749503in}{1.181726in}}%
\pgfpathlineto{\pgfqpoint{1.742115in}{1.188078in}}%
\pgfpathlineto{\pgfqpoint{1.734360in}{1.194302in}}%
\pgfpathlineto{\pgfqpoint{1.726246in}{1.200392in}}%
\pgfpathclose%
\pgfusepath{fill}%
\end{pgfscope}%
\begin{pgfscope}%
\pgfpathrectangle{\pgfqpoint{0.329460in}{0.284240in}}{\pgfqpoint{1.989680in}{1.989680in}}%
\pgfusepath{clip}%
\pgfsetbuttcap%
\pgfsetroundjoin%
\definecolor{currentfill}{rgb}{0.212395,0.359683,0.551710}%
\pgfsetfillcolor{currentfill}%
\pgfsetlinewidth{0.000000pt}%
\definecolor{currentstroke}{rgb}{0.000000,0.000000,0.000000}%
\pgfsetstrokecolor{currentstroke}%
\pgfsetdash{}{0pt}%
\pgfpathmoveto{\pgfqpoint{1.438252in}{1.409762in}}%
\pgfpathlineto{\pgfqpoint{1.439036in}{1.404942in}}%
\pgfpathlineto{\pgfqpoint{1.439820in}{1.400169in}}%
\pgfpathlineto{\pgfqpoint{1.440603in}{1.395446in}}%
\pgfpathlineto{\pgfqpoint{1.441385in}{1.390775in}}%
\pgfpathlineto{\pgfqpoint{1.452317in}{1.389285in}}%
\pgfpathlineto{\pgfqpoint{1.463157in}{1.387626in}}%
\pgfpathlineto{\pgfqpoint{1.473894in}{1.385797in}}%
\pgfpathlineto{\pgfqpoint{1.472829in}{1.390515in}}%
\pgfpathlineto{\pgfqpoint{1.471764in}{1.395285in}}%
\pgfpathlineto{\pgfqpoint{1.470697in}{1.400105in}}%
\pgfpathlineto{\pgfqpoint{1.469629in}{1.404971in}}%
\pgfpathlineto{\pgfqpoint{1.459266in}{1.406731in}}%
\pgfpathlineto{\pgfqpoint{1.448804in}{1.408328in}}%
\pgfpathlineto{\pgfqpoint{1.438252in}{1.409762in}}%
\pgfpathclose%
\pgfusepath{fill}%
\end{pgfscope}%
\begin{pgfscope}%
\pgfpathrectangle{\pgfqpoint{0.329460in}{0.284240in}}{\pgfqpoint{1.989680in}{1.989680in}}%
\pgfusepath{clip}%
\pgfsetbuttcap%
\pgfsetroundjoin%
\definecolor{currentfill}{rgb}{0.233603,0.313828,0.543914}%
\pgfsetfillcolor{currentfill}%
\pgfsetlinewidth{0.000000pt}%
\definecolor{currentstroke}{rgb}{0.000000,0.000000,0.000000}%
\pgfsetstrokecolor{currentstroke}%
\pgfsetdash{}{0pt}%
\pgfpathmoveto{\pgfqpoint{1.868601in}{1.354642in}}%
\pgfpathlineto{\pgfqpoint{1.871550in}{1.365397in}}%
\pgfpathlineto{\pgfqpoint{1.874512in}{1.376542in}}%
\pgfpathlineto{\pgfqpoint{1.877487in}{1.388084in}}%
\pgfpathlineto{\pgfqpoint{1.880476in}{1.400028in}}%
\pgfpathlineto{\pgfqpoint{1.892142in}{1.391702in}}%
\pgfpathlineto{\pgfqpoint{1.903314in}{1.383186in}}%
\pgfpathlineto{\pgfqpoint{1.913981in}{1.374489in}}%
\pgfpathlineto{\pgfqpoint{1.924130in}{1.365615in}}%
\pgfpathlineto{\pgfqpoint{1.920876in}{1.353800in}}%
\pgfpathlineto{\pgfqpoint{1.917638in}{1.342389in}}%
\pgfpathlineto{\pgfqpoint{1.914414in}{1.331376in}}%
\pgfpathlineto{\pgfqpoint{1.911205in}{1.320756in}}%
\pgfpathlineto{\pgfqpoint{1.901302in}{1.329492in}}%
\pgfpathlineto{\pgfqpoint{1.890893in}{1.338057in}}%
\pgfpathlineto{\pgfqpoint{1.879989in}{1.346442in}}%
\pgfpathlineto{\pgfqpoint{1.868601in}{1.354642in}}%
\pgfpathclose%
\pgfusepath{fill}%
\end{pgfscope}%
\begin{pgfscope}%
\pgfpathrectangle{\pgfqpoint{0.329460in}{0.284240in}}{\pgfqpoint{1.989680in}{1.989680in}}%
\pgfusepath{clip}%
\pgfsetbuttcap%
\pgfsetroundjoin%
\definecolor{currentfill}{rgb}{0.212395,0.359683,0.551710}%
\pgfsetfillcolor{currentfill}%
\pgfsetlinewidth{0.000000pt}%
\definecolor{currentstroke}{rgb}{0.000000,0.000000,0.000000}%
\pgfsetstrokecolor{currentstroke}%
\pgfsetdash{}{0pt}%
\pgfpathmoveto{\pgfqpoint{1.223626in}{1.403273in}}%
\pgfpathlineto{\pgfqpoint{1.222475in}{1.398389in}}%
\pgfpathlineto{\pgfqpoint{1.221326in}{1.393553in}}%
\pgfpathlineto{\pgfqpoint{1.220178in}{1.388766in}}%
\pgfpathlineto{\pgfqpoint{1.219030in}{1.384031in}}%
\pgfpathlineto{\pgfqpoint{1.229668in}{1.386009in}}%
\pgfpathlineto{\pgfqpoint{1.240418in}{1.387818in}}%
\pgfpathlineto{\pgfqpoint{1.251268in}{1.389459in}}%
\pgfpathlineto{\pgfqpoint{1.262210in}{1.390930in}}%
\pgfpathlineto{\pgfqpoint{1.262982in}{1.395599in}}%
\pgfpathlineto{\pgfqpoint{1.263754in}{1.400321in}}%
\pgfpathlineto{\pgfqpoint{1.264527in}{1.405093in}}%
\pgfpathlineto{\pgfqpoint{1.265301in}{1.409911in}}%
\pgfpathlineto{\pgfqpoint{1.254740in}{1.408495in}}%
\pgfpathlineto{\pgfqpoint{1.244267in}{1.406916in}}%
\pgfpathlineto{\pgfqpoint{1.233892in}{1.405175in}}%
\pgfpathlineto{\pgfqpoint{1.223626in}{1.403273in}}%
\pgfpathclose%
\pgfusepath{fill}%
\end{pgfscope}%
\begin{pgfscope}%
\pgfpathrectangle{\pgfqpoint{0.329460in}{0.284240in}}{\pgfqpoint{1.989680in}{1.989680in}}%
\pgfusepath{clip}%
\pgfsetbuttcap%
\pgfsetroundjoin%
\definecolor{currentfill}{rgb}{0.268510,0.009605,0.335427}%
\pgfsetfillcolor{currentfill}%
\pgfsetlinewidth{0.000000pt}%
\definecolor{currentstroke}{rgb}{0.000000,0.000000,0.000000}%
\pgfsetstrokecolor{currentstroke}%
\pgfsetdash{}{0pt}%
\pgfpathmoveto{\pgfqpoint{1.825790in}{1.159103in}}%
\pgfpathlineto{\pgfqpoint{1.828730in}{1.161023in}}%
\pgfpathlineto{\pgfqpoint{1.831675in}{1.163188in}}%
\pgfpathlineto{\pgfqpoint{1.834626in}{1.165601in}}%
\pgfpathlineto{\pgfqpoint{1.837584in}{1.168266in}}%
\pgfpathlineto{\pgfqpoint{1.845641in}{1.160346in}}%
\pgfpathlineto{\pgfqpoint{1.853234in}{1.152290in}}%
\pgfpathlineto{\pgfqpoint{1.860354in}{1.144105in}}%
\pgfpathlineto{\pgfqpoint{1.866992in}{1.135797in}}%
\pgfpathlineto{\pgfqpoint{1.863840in}{1.133317in}}%
\pgfpathlineto{\pgfqpoint{1.860694in}{1.131090in}}%
\pgfpathlineto{\pgfqpoint{1.857555in}{1.129113in}}%
\pgfpathlineto{\pgfqpoint{1.854423in}{1.127380in}}%
\pgfpathlineto{\pgfqpoint{1.847962in}{1.135496in}}%
\pgfpathlineto{\pgfqpoint{1.841031in}{1.143493in}}%
\pgfpathlineto{\pgfqpoint{1.833638in}{1.151364in}}%
\pgfpathlineto{\pgfqpoint{1.825790in}{1.159103in}}%
\pgfpathclose%
\pgfusepath{fill}%
\end{pgfscope}%
\begin{pgfscope}%
\pgfpathrectangle{\pgfqpoint{0.329460in}{0.284240in}}{\pgfqpoint{1.989680in}{1.989680in}}%
\pgfusepath{clip}%
\pgfsetbuttcap%
\pgfsetroundjoin%
\definecolor{currentfill}{rgb}{0.272594,0.025563,0.353093}%
\pgfsetfillcolor{currentfill}%
\pgfsetlinewidth{0.000000pt}%
\definecolor{currentstroke}{rgb}{0.000000,0.000000,0.000000}%
\pgfsetstrokecolor{currentstroke}%
\pgfsetdash{}{0pt}%
\pgfpathmoveto{\pgfqpoint{1.837584in}{1.168266in}}%
\pgfpathlineto{\pgfqpoint{1.840549in}{1.171188in}}%
\pgfpathlineto{\pgfqpoint{1.843520in}{1.174373in}}%
\pgfpathlineto{\pgfqpoint{1.846499in}{1.177824in}}%
\pgfpathlineto{\pgfqpoint{1.849485in}{1.181546in}}%
\pgfpathlineto{\pgfqpoint{1.857755in}{1.173449in}}%
\pgfpathlineto{\pgfqpoint{1.865550in}{1.165212in}}%
\pgfpathlineto{\pgfqpoint{1.872861in}{1.156843in}}%
\pgfpathlineto{\pgfqpoint{1.879679in}{1.148348in}}%
\pgfpathlineto{\pgfqpoint{1.876496in}{1.144807in}}%
\pgfpathlineto{\pgfqpoint{1.873320in}{1.141537in}}%
\pgfpathlineto{\pgfqpoint{1.870152in}{1.138536in}}%
\pgfpathlineto{\pgfqpoint{1.866992in}{1.135797in}}%
\pgfpathlineto{\pgfqpoint{1.860354in}{1.144105in}}%
\pgfpathlineto{\pgfqpoint{1.853234in}{1.152290in}}%
\pgfpathlineto{\pgfqpoint{1.845641in}{1.160346in}}%
\pgfpathlineto{\pgfqpoint{1.837584in}{1.168266in}}%
\pgfpathclose%
\pgfusepath{fill}%
\end{pgfscope}%
\begin{pgfscope}%
\pgfpathrectangle{\pgfqpoint{0.329460in}{0.284240in}}{\pgfqpoint{1.989680in}{1.989680in}}%
\pgfusepath{clip}%
\pgfsetbuttcap%
\pgfsetroundjoin%
\definecolor{currentfill}{rgb}{0.267004,0.004874,0.329415}%
\pgfsetfillcolor{currentfill}%
\pgfsetlinewidth{0.000000pt}%
\definecolor{currentstroke}{rgb}{0.000000,0.000000,0.000000}%
\pgfsetstrokecolor{currentstroke}%
\pgfsetdash{}{0pt}%
\pgfpathmoveto{\pgfqpoint{1.814089in}{1.153768in}}%
\pgfpathlineto{\pgfqpoint{1.817007in}{1.154758in}}%
\pgfpathlineto{\pgfqpoint{1.819929in}{1.155974in}}%
\pgfpathlineto{\pgfqpoint{1.822857in}{1.157421in}}%
\pgfpathlineto{\pgfqpoint{1.825790in}{1.159103in}}%
\pgfpathlineto{\pgfqpoint{1.833638in}{1.151364in}}%
\pgfpathlineto{\pgfqpoint{1.841031in}{1.143493in}}%
\pgfpathlineto{\pgfqpoint{1.847962in}{1.135496in}}%
\pgfpathlineto{\pgfqpoint{1.854423in}{1.127380in}}%
\pgfpathlineto{\pgfqpoint{1.851297in}{1.125887in}}%
\pgfpathlineto{\pgfqpoint{1.848178in}{1.124630in}}%
\pgfpathlineto{\pgfqpoint{1.845064in}{1.123604in}}%
\pgfpathlineto{\pgfqpoint{1.841956in}{1.122805in}}%
\pgfpathlineto{\pgfqpoint{1.835670in}{1.130726in}}%
\pgfpathlineto{\pgfqpoint{1.828925in}{1.138532in}}%
\pgfpathlineto{\pgfqpoint{1.821729in}{1.146215in}}%
\pgfpathlineto{\pgfqpoint{1.814089in}{1.153768in}}%
\pgfpathclose%
\pgfusepath{fill}%
\end{pgfscope}%
\begin{pgfscope}%
\pgfpathrectangle{\pgfqpoint{0.329460in}{0.284240in}}{\pgfqpoint{1.989680in}{1.989680in}}%
\pgfusepath{clip}%
\pgfsetbuttcap%
\pgfsetroundjoin%
\definecolor{currentfill}{rgb}{0.277941,0.056324,0.381191}%
\pgfsetfillcolor{currentfill}%
\pgfsetlinewidth{0.000000pt}%
\definecolor{currentstroke}{rgb}{0.000000,0.000000,0.000000}%
\pgfsetstrokecolor{currentstroke}%
\pgfsetdash{}{0pt}%
\pgfpathmoveto{\pgfqpoint{1.849485in}{1.181546in}}%
\pgfpathlineto{\pgfqpoint{1.852479in}{1.185544in}}%
\pgfpathlineto{\pgfqpoint{1.855481in}{1.189823in}}%
\pgfpathlineto{\pgfqpoint{1.858491in}{1.194387in}}%
\pgfpathlineto{\pgfqpoint{1.861510in}{1.199242in}}%
\pgfpathlineto{\pgfqpoint{1.869995in}{1.190973in}}%
\pgfpathlineto{\pgfqpoint{1.877995in}{1.182561in}}%
\pgfpathlineto{\pgfqpoint{1.885500in}{1.174012in}}%
\pgfpathlineto{\pgfqpoint{1.892501in}{1.165334in}}%
\pgfpathlineto{\pgfqpoint{1.889282in}{1.160655in}}%
\pgfpathlineto{\pgfqpoint{1.886072in}{1.156267in}}%
\pgfpathlineto{\pgfqpoint{1.882871in}{1.152167in}}%
\pgfpathlineto{\pgfqpoint{1.879679in}{1.148348in}}%
\pgfpathlineto{\pgfqpoint{1.872861in}{1.156843in}}%
\pgfpathlineto{\pgfqpoint{1.865550in}{1.165212in}}%
\pgfpathlineto{\pgfqpoint{1.857755in}{1.173449in}}%
\pgfpathlineto{\pgfqpoint{1.849485in}{1.181546in}}%
\pgfpathclose%
\pgfusepath{fill}%
\end{pgfscope}%
\begin{pgfscope}%
\pgfpathrectangle{\pgfqpoint{0.329460in}{0.284240in}}{\pgfqpoint{1.989680in}{1.989680in}}%
\pgfusepath{clip}%
\pgfsetbuttcap%
\pgfsetroundjoin%
\definecolor{currentfill}{rgb}{0.248629,0.278775,0.534556}%
\pgfsetfillcolor{currentfill}%
\pgfsetlinewidth{0.000000pt}%
\definecolor{currentstroke}{rgb}{0.000000,0.000000,0.000000}%
\pgfsetstrokecolor{currentstroke}%
\pgfsetdash{}{0pt}%
\pgfpathmoveto{\pgfqpoint{1.562016in}{1.346250in}}%
\pgfpathlineto{\pgfqpoint{1.563784in}{1.341645in}}%
\pgfpathlineto{\pgfqpoint{1.565550in}{1.337109in}}%
\pgfpathlineto{\pgfqpoint{1.567315in}{1.332647in}}%
\pgfpathlineto{\pgfqpoint{1.569079in}{1.328261in}}%
\pgfpathlineto{\pgfqpoint{1.579119in}{1.324753in}}%
\pgfpathlineto{\pgfqpoint{1.588948in}{1.321087in}}%
\pgfpathlineto{\pgfqpoint{1.598558in}{1.317266in}}%
\pgfpathlineto{\pgfqpoint{1.607938in}{1.313292in}}%
\pgfpathlineto{\pgfqpoint{1.605855in}{1.317810in}}%
\pgfpathlineto{\pgfqpoint{1.603770in}{1.322405in}}%
\pgfpathlineto{\pgfqpoint{1.601684in}{1.327072in}}%
\pgfpathlineto{\pgfqpoint{1.599596in}{1.331810in}}%
\pgfpathlineto{\pgfqpoint{1.590525in}{1.335643in}}%
\pgfpathlineto{\pgfqpoint{1.581232in}{1.339329in}}%
\pgfpathlineto{\pgfqpoint{1.571726in}{1.342866in}}%
\pgfpathlineto{\pgfqpoint{1.562016in}{1.346250in}}%
\pgfpathclose%
\pgfusepath{fill}%
\end{pgfscope}%
\begin{pgfscope}%
\pgfpathrectangle{\pgfqpoint{0.329460in}{0.284240in}}{\pgfqpoint{1.989680in}{1.989680in}}%
\pgfusepath{clip}%
\pgfsetbuttcap%
\pgfsetroundjoin%
\definecolor{currentfill}{rgb}{0.231674,0.318106,0.544834}%
\pgfsetfillcolor{currentfill}%
\pgfsetlinewidth{0.000000pt}%
\definecolor{currentstroke}{rgb}{0.000000,0.000000,0.000000}%
\pgfsetstrokecolor{currentstroke}%
\pgfsetdash{}{0pt}%
\pgfpathmoveto{\pgfqpoint{1.515623in}{1.376830in}}%
\pgfpathlineto{\pgfqpoint{1.517051in}{1.372083in}}%
\pgfpathlineto{\pgfqpoint{1.518477in}{1.367394in}}%
\pgfpathlineto{\pgfqpoint{1.519903in}{1.362767in}}%
\pgfpathlineto{\pgfqpoint{1.521328in}{1.358204in}}%
\pgfpathlineto{\pgfqpoint{1.531758in}{1.355458in}}%
\pgfpathlineto{\pgfqpoint{1.542023in}{1.352549in}}%
\pgfpathlineto{\pgfqpoint{1.552112in}{1.349479in}}%
\pgfpathlineto{\pgfqpoint{1.562016in}{1.346250in}}%
\pgfpathlineto{\pgfqpoint{1.560247in}{1.350922in}}%
\pgfpathlineto{\pgfqpoint{1.558478in}{1.355659in}}%
\pgfpathlineto{\pgfqpoint{1.556706in}{1.360457in}}%
\pgfpathlineto{\pgfqpoint{1.554934in}{1.365313in}}%
\pgfpathlineto{\pgfqpoint{1.545366in}{1.368423in}}%
\pgfpathlineto{\pgfqpoint{1.535618in}{1.371381in}}%
\pgfpathlineto{\pgfqpoint{1.525701in}{1.374184in}}%
\pgfpathlineto{\pgfqpoint{1.515623in}{1.376830in}}%
\pgfpathclose%
\pgfusepath{fill}%
\end{pgfscope}%
\begin{pgfscope}%
\pgfpathrectangle{\pgfqpoint{0.329460in}{0.284240in}}{\pgfqpoint{1.989680in}{1.989680in}}%
\pgfusepath{clip}%
\pgfsetbuttcap%
\pgfsetroundjoin%
\definecolor{currentfill}{rgb}{0.280255,0.165693,0.476498}%
\pgfsetfillcolor{currentfill}%
\pgfsetlinewidth{0.000000pt}%
\definecolor{currentstroke}{rgb}{0.000000,0.000000,0.000000}%
\pgfsetstrokecolor{currentstroke}%
\pgfsetdash{}{0pt}%
\pgfpathmoveto{\pgfqpoint{1.661960in}{1.261679in}}%
\pgfpathlineto{\pgfqpoint{1.664328in}{1.257831in}}%
\pgfpathlineto{\pgfqpoint{1.666695in}{1.254090in}}%
\pgfpathlineto{\pgfqpoint{1.669062in}{1.250460in}}%
\pgfpathlineto{\pgfqpoint{1.671429in}{1.246944in}}%
\pgfpathlineto{\pgfqpoint{1.680347in}{1.241767in}}%
\pgfpathlineto{\pgfqpoint{1.688958in}{1.236445in}}%
\pgfpathlineto{\pgfqpoint{1.697253in}{1.230984in}}%
\pgfpathlineto{\pgfqpoint{1.705225in}{1.225389in}}%
\pgfpathlineto{\pgfqpoint{1.702599in}{1.229077in}}%
\pgfpathlineto{\pgfqpoint{1.699974in}{1.232879in}}%
\pgfpathlineto{\pgfqpoint{1.697348in}{1.236792in}}%
\pgfpathlineto{\pgfqpoint{1.694721in}{1.240812in}}%
\pgfpathlineto{\pgfqpoint{1.686995in}{1.246229in}}%
\pgfpathlineto{\pgfqpoint{1.678953in}{1.251515in}}%
\pgfpathlineto{\pgfqpoint{1.670606in}{1.256667in}}%
\pgfpathlineto{\pgfqpoint{1.661960in}{1.261679in}}%
\pgfpathclose%
\pgfusepath{fill}%
\end{pgfscope}%
\begin{pgfscope}%
\pgfpathrectangle{\pgfqpoint{0.329460in}{0.284240in}}{\pgfqpoint{1.989680in}{1.989680in}}%
\pgfusepath{clip}%
\pgfsetbuttcap%
\pgfsetroundjoin%
\definecolor{currentfill}{rgb}{0.267004,0.004874,0.329415}%
\pgfsetfillcolor{currentfill}%
\pgfsetlinewidth{0.000000pt}%
\definecolor{currentstroke}{rgb}{0.000000,0.000000,0.000000}%
\pgfsetstrokecolor{currentstroke}%
\pgfsetdash{}{0pt}%
\pgfpathmoveto{\pgfqpoint{1.802467in}{1.151988in}}%
\pgfpathlineto{\pgfqpoint{1.805366in}{1.152115in}}%
\pgfpathlineto{\pgfqpoint{1.808269in}{1.152451in}}%
\pgfpathlineto{\pgfqpoint{1.811177in}{1.153001in}}%
\pgfpathlineto{\pgfqpoint{1.814089in}{1.153768in}}%
\pgfpathlineto{\pgfqpoint{1.821729in}{1.146215in}}%
\pgfpathlineto{\pgfqpoint{1.828925in}{1.138532in}}%
\pgfpathlineto{\pgfqpoint{1.835670in}{1.130726in}}%
\pgfpathlineto{\pgfqpoint{1.841956in}{1.122805in}}%
\pgfpathlineto{\pgfqpoint{1.838853in}{1.122230in}}%
\pgfpathlineto{\pgfqpoint{1.835755in}{1.121873in}}%
\pgfpathlineto{\pgfqpoint{1.832663in}{1.121730in}}%
\pgfpathlineto{\pgfqpoint{1.829575in}{1.121798in}}%
\pgfpathlineto{\pgfqpoint{1.823463in}{1.129520in}}%
\pgfpathlineto{\pgfqpoint{1.816902in}{1.137130in}}%
\pgfpathlineto{\pgfqpoint{1.809901in}{1.144622in}}%
\pgfpathlineto{\pgfqpoint{1.802467in}{1.151988in}}%
\pgfpathclose%
\pgfusepath{fill}%
\end{pgfscope}%
\begin{pgfscope}%
\pgfpathrectangle{\pgfqpoint{0.329460in}{0.284240in}}{\pgfqpoint{1.989680in}{1.989680in}}%
\pgfusepath{clip}%
\pgfsetbuttcap%
\pgfsetroundjoin%
\definecolor{currentfill}{rgb}{0.279566,0.067836,0.391917}%
\pgfsetfillcolor{currentfill}%
\pgfsetlinewidth{0.000000pt}%
\definecolor{currentstroke}{rgb}{0.000000,0.000000,0.000000}%
\pgfsetstrokecolor{currentstroke}%
\pgfsetdash{}{0pt}%
\pgfpathmoveto{\pgfqpoint{0.939949in}{1.169399in}}%
\pgfpathlineto{\pgfqpoint{0.937050in}{1.166658in}}%
\pgfpathlineto{\pgfqpoint{0.934149in}{1.164066in}}%
\pgfpathlineto{\pgfqpoint{0.931246in}{1.161626in}}%
\pgfpathlineto{\pgfqpoint{0.928342in}{1.159342in}}%
\pgfpathlineto{\pgfqpoint{0.935214in}{1.166117in}}%
\pgfpathlineto{\pgfqpoint{0.942478in}{1.172770in}}%
\pgfpathlineto{\pgfqpoint{0.950128in}{1.179296in}}%
\pgfpathlineto{\pgfqpoint{0.958153in}{1.185690in}}%
\pgfpathlineto{\pgfqpoint{0.960842in}{1.187784in}}%
\pgfpathlineto{\pgfqpoint{0.963529in}{1.190034in}}%
\pgfpathlineto{\pgfqpoint{0.966214in}{1.192435in}}%
\pgfpathlineto{\pgfqpoint{0.968899in}{1.194985in}}%
\pgfpathlineto{\pgfqpoint{0.961103in}{1.188776in}}%
\pgfpathlineto{\pgfqpoint{0.953675in}{1.182438in}}%
\pgfpathlineto{\pgfqpoint{0.946620in}{1.175977in}}%
\pgfpathlineto{\pgfqpoint{0.939949in}{1.169399in}}%
\pgfpathclose%
\pgfusepath{fill}%
\end{pgfscope}%
\begin{pgfscope}%
\pgfpathrectangle{\pgfqpoint{0.329460in}{0.284240in}}{\pgfqpoint{1.989680in}{1.989680in}}%
\pgfusepath{clip}%
\pgfsetbuttcap%
\pgfsetroundjoin%
\definecolor{currentfill}{rgb}{0.282327,0.094955,0.417331}%
\pgfsetfillcolor{currentfill}%
\pgfsetlinewidth{0.000000pt}%
\definecolor{currentstroke}{rgb}{0.000000,0.000000,0.000000}%
\pgfsetstrokecolor{currentstroke}%
\pgfsetdash{}{0pt}%
\pgfpathmoveto{\pgfqpoint{1.861510in}{1.199242in}}%
\pgfpathlineto{\pgfqpoint{1.864537in}{1.204393in}}%
\pgfpathlineto{\pgfqpoint{1.867574in}{1.209845in}}%
\pgfpathlineto{\pgfqpoint{1.870620in}{1.215603in}}%
\pgfpathlineto{\pgfqpoint{1.873675in}{1.221671in}}%
\pgfpathlineto{\pgfqpoint{1.882379in}{1.213235in}}%
\pgfpathlineto{\pgfqpoint{1.890587in}{1.204652in}}%
\pgfpathlineto{\pgfqpoint{1.898289in}{1.195929in}}%
\pgfpathlineto{\pgfqpoint{1.905476in}{1.187074in}}%
\pgfpathlineto{\pgfqpoint{1.902217in}{1.181176in}}%
\pgfpathlineto{\pgfqpoint{1.898968in}{1.175590in}}%
\pgfpathlineto{\pgfqpoint{1.895730in}{1.170311in}}%
\pgfpathlineto{\pgfqpoint{1.892501in}{1.165334in}}%
\pgfpathlineto{\pgfqpoint{1.885500in}{1.174012in}}%
\pgfpathlineto{\pgfqpoint{1.877995in}{1.182561in}}%
\pgfpathlineto{\pgfqpoint{1.869995in}{1.190973in}}%
\pgfpathlineto{\pgfqpoint{1.861510in}{1.199242in}}%
\pgfpathclose%
\pgfusepath{fill}%
\end{pgfscope}%
\begin{pgfscope}%
\pgfpathrectangle{\pgfqpoint{0.329460in}{0.284240in}}{\pgfqpoint{1.989680in}{1.989680in}}%
\pgfusepath{clip}%
\pgfsetbuttcap%
\pgfsetroundjoin%
\definecolor{currentfill}{rgb}{0.231674,0.318106,0.544834}%
\pgfsetfillcolor{currentfill}%
\pgfsetlinewidth{0.000000pt}%
\definecolor{currentstroke}{rgb}{0.000000,0.000000,0.000000}%
\pgfsetstrokecolor{currentstroke}%
\pgfsetdash{}{0pt}%
\pgfpathmoveto{\pgfqpoint{1.139095in}{1.362422in}}%
\pgfpathlineto{\pgfqpoint{1.137248in}{1.357539in}}%
\pgfpathlineto{\pgfqpoint{1.135404in}{1.352713in}}%
\pgfpathlineto{\pgfqpoint{1.133560in}{1.347949in}}%
\pgfpathlineto{\pgfqpoint{1.131718in}{1.343250in}}%
\pgfpathlineto{\pgfqpoint{1.141450in}{1.346617in}}%
\pgfpathlineto{\pgfqpoint{1.151375in}{1.349828in}}%
\pgfpathlineto{\pgfqpoint{1.161484in}{1.352880in}}%
\pgfpathlineto{\pgfqpoint{1.171768in}{1.355771in}}%
\pgfpathlineto{\pgfqpoint{1.173271in}{1.360357in}}%
\pgfpathlineto{\pgfqpoint{1.174775in}{1.365006in}}%
\pgfpathlineto{\pgfqpoint{1.176280in}{1.369717in}}%
\pgfpathlineto{\pgfqpoint{1.177786in}{1.374486in}}%
\pgfpathlineto{\pgfqpoint{1.167851in}{1.371700in}}%
\pgfpathlineto{\pgfqpoint{1.158084in}{1.368759in}}%
\pgfpathlineto{\pgfqpoint{1.148496in}{1.365666in}}%
\pgfpathlineto{\pgfqpoint{1.139095in}{1.362422in}}%
\pgfpathclose%
\pgfusepath{fill}%
\end{pgfscope}%
\begin{pgfscope}%
\pgfpathrectangle{\pgfqpoint{0.329460in}{0.284240in}}{\pgfqpoint{1.989680in}{1.989680in}}%
\pgfusepath{clip}%
\pgfsetbuttcap%
\pgfsetroundjoin%
\definecolor{currentfill}{rgb}{0.248629,0.278775,0.534556}%
\pgfsetfillcolor{currentfill}%
\pgfsetlinewidth{0.000000pt}%
\definecolor{currentstroke}{rgb}{0.000000,0.000000,0.000000}%
\pgfsetstrokecolor{currentstroke}%
\pgfsetdash{}{0pt}%
\pgfpathmoveto{\pgfqpoint{1.094909in}{1.328282in}}%
\pgfpathlineto{\pgfqpoint{1.092754in}{1.323512in}}%
\pgfpathlineto{\pgfqpoint{1.090601in}{1.318812in}}%
\pgfpathlineto{\pgfqpoint{1.088449in}{1.314186in}}%
\pgfpathlineto{\pgfqpoint{1.086297in}{1.309635in}}%
\pgfpathlineto{\pgfqpoint{1.095467in}{1.313741in}}%
\pgfpathlineto{\pgfqpoint{1.104874in}{1.317698in}}%
\pgfpathlineto{\pgfqpoint{1.114509in}{1.321502in}}%
\pgfpathlineto{\pgfqpoint{1.124362in}{1.325151in}}%
\pgfpathlineto{\pgfqpoint{1.126199in}{1.329564in}}%
\pgfpathlineto{\pgfqpoint{1.128038in}{1.334054in}}%
\pgfpathlineto{\pgfqpoint{1.129877in}{1.338617in}}%
\pgfpathlineto{\pgfqpoint{1.131718in}{1.343250in}}%
\pgfpathlineto{\pgfqpoint{1.122189in}{1.339730in}}%
\pgfpathlineto{\pgfqpoint{1.112872in}{1.336060in}}%
\pgfpathlineto{\pgfqpoint{1.103776in}{1.332243in}}%
\pgfpathlineto{\pgfqpoint{1.094909in}{1.328282in}}%
\pgfpathclose%
\pgfusepath{fill}%
\end{pgfscope}%
\begin{pgfscope}%
\pgfpathrectangle{\pgfqpoint{0.329460in}{0.284240in}}{\pgfqpoint{1.989680in}{1.989680in}}%
\pgfusepath{clip}%
\pgfsetbuttcap%
\pgfsetroundjoin%
\definecolor{currentfill}{rgb}{0.263663,0.237631,0.518762}%
\pgfsetfillcolor{currentfill}%
\pgfsetlinewidth{0.000000pt}%
\definecolor{currentstroke}{rgb}{0.000000,0.000000,0.000000}%
\pgfsetstrokecolor{currentstroke}%
\pgfsetdash{}{0pt}%
\pgfpathmoveto{\pgfqpoint{1.607938in}{1.313292in}}%
\pgfpathlineto{\pgfqpoint{1.610021in}{1.308853in}}%
\pgfpathlineto{\pgfqpoint{1.612103in}{1.304496in}}%
\pgfpathlineto{\pgfqpoint{1.614184in}{1.300224in}}%
\pgfpathlineto{\pgfqpoint{1.616263in}{1.296040in}}%
\pgfpathlineto{\pgfqpoint{1.625709in}{1.291772in}}%
\pgfpathlineto{\pgfqpoint{1.634900in}{1.287354in}}%
\pgfpathlineto{\pgfqpoint{1.643828in}{1.282789in}}%
\pgfpathlineto{\pgfqpoint{1.652483in}{1.278081in}}%
\pgfpathlineto{\pgfqpoint{1.650113in}{1.282418in}}%
\pgfpathlineto{\pgfqpoint{1.647741in}{1.286843in}}%
\pgfpathlineto{\pgfqpoint{1.645368in}{1.291354in}}%
\pgfpathlineto{\pgfqpoint{1.642994in}{1.295947in}}%
\pgfpathlineto{\pgfqpoint{1.634618in}{1.300493in}}%
\pgfpathlineto{\pgfqpoint{1.625978in}{1.304902in}}%
\pgfpathlineto{\pgfqpoint{1.617082in}{1.309170in}}%
\pgfpathlineto{\pgfqpoint{1.607938in}{1.313292in}}%
\pgfpathclose%
\pgfusepath{fill}%
\end{pgfscope}%
\begin{pgfscope}%
\pgfpathrectangle{\pgfqpoint{0.329460in}{0.284240in}}{\pgfqpoint{1.989680in}{1.989680in}}%
\pgfusepath{clip}%
\pgfsetbuttcap%
\pgfsetroundjoin%
\definecolor{currentfill}{rgb}{0.212395,0.359683,0.551710}%
\pgfsetfillcolor{currentfill}%
\pgfsetlinewidth{0.000000pt}%
\definecolor{currentstroke}{rgb}{0.000000,0.000000,0.000000}%
\pgfsetstrokecolor{currentstroke}%
\pgfsetdash{}{0pt}%
\pgfpathmoveto{\pgfqpoint{1.469629in}{1.404971in}}%
\pgfpathlineto{\pgfqpoint{1.470697in}{1.400105in}}%
\pgfpathlineto{\pgfqpoint{1.471764in}{1.395285in}}%
\pgfpathlineto{\pgfqpoint{1.472829in}{1.390515in}}%
\pgfpathlineto{\pgfqpoint{1.473894in}{1.385797in}}%
\pgfpathlineto{\pgfqpoint{1.484519in}{1.383802in}}%
\pgfpathlineto{\pgfqpoint{1.495022in}{1.381641in}}%
\pgfpathlineto{\pgfqpoint{1.505393in}{1.379316in}}%
\pgfpathlineto{\pgfqpoint{1.515623in}{1.376830in}}%
\pgfpathlineto{\pgfqpoint{1.514193in}{1.381632in}}%
\pgfpathlineto{\pgfqpoint{1.512763in}{1.386487in}}%
\pgfpathlineto{\pgfqpoint{1.511331in}{1.391392in}}%
\pgfpathlineto{\pgfqpoint{1.509899in}{1.396344in}}%
\pgfpathlineto{\pgfqpoint{1.500027in}{1.398736in}}%
\pgfpathlineto{\pgfqpoint{1.490019in}{1.400972in}}%
\pgfpathlineto{\pgfqpoint{1.479883in}{1.403051in}}%
\pgfpathlineto{\pgfqpoint{1.469629in}{1.404971in}}%
\pgfpathclose%
\pgfusepath{fill}%
\end{pgfscope}%
\begin{pgfscope}%
\pgfpathrectangle{\pgfqpoint{0.329460in}{0.284240in}}{\pgfqpoint{1.989680in}{1.989680in}}%
\pgfusepath{clip}%
\pgfsetbuttcap%
\pgfsetroundjoin%
\definecolor{currentfill}{rgb}{0.282327,0.094955,0.417331}%
\pgfsetfillcolor{currentfill}%
\pgfsetlinewidth{0.000000pt}%
\definecolor{currentstroke}{rgb}{0.000000,0.000000,0.000000}%
\pgfsetstrokecolor{currentstroke}%
\pgfsetdash{}{0pt}%
\pgfpathmoveto{\pgfqpoint{1.715731in}{1.211843in}}%
\pgfpathlineto{\pgfqpoint{1.718359in}{1.208776in}}%
\pgfpathlineto{\pgfqpoint{1.720987in}{1.205842in}}%
\pgfpathlineto{\pgfqpoint{1.723616in}{1.203047in}}%
\pgfpathlineto{\pgfqpoint{1.726246in}{1.200392in}}%
\pgfpathlineto{\pgfqpoint{1.734360in}{1.194302in}}%
\pgfpathlineto{\pgfqpoint{1.742115in}{1.188078in}}%
\pgfpathlineto{\pgfqpoint{1.749503in}{1.181726in}}%
\pgfpathlineto{\pgfqpoint{1.756515in}{1.175252in}}%
\pgfpathlineto{\pgfqpoint{1.753661in}{1.178094in}}%
\pgfpathlineto{\pgfqpoint{1.750809in}{1.181078in}}%
\pgfpathlineto{\pgfqpoint{1.747958in}{1.184199in}}%
\pgfpathlineto{\pgfqpoint{1.745107in}{1.187455in}}%
\pgfpathlineto{\pgfqpoint{1.738304in}{1.193735in}}%
\pgfpathlineto{\pgfqpoint{1.731135in}{1.199897in}}%
\pgfpathlineto{\pgfqpoint{1.723608in}{1.205935in}}%
\pgfpathlineto{\pgfqpoint{1.715731in}{1.211843in}}%
\pgfpathclose%
\pgfusepath{fill}%
\end{pgfscope}%
\begin{pgfscope}%
\pgfpathrectangle{\pgfqpoint{0.329460in}{0.284240in}}{\pgfqpoint{1.989680in}{1.989680in}}%
\pgfusepath{clip}%
\pgfsetbuttcap%
\pgfsetroundjoin%
\definecolor{currentfill}{rgb}{0.195860,0.395433,0.555276}%
\pgfsetfillcolor{currentfill}%
\pgfsetlinewidth{0.000000pt}%
\definecolor{currentstroke}{rgb}{0.000000,0.000000,0.000000}%
\pgfsetstrokecolor{currentstroke}%
\pgfsetdash{}{0pt}%
\pgfpathmoveto{\pgfqpoint{1.309784in}{1.433444in}}%
\pgfpathlineto{\pgfqpoint{1.309396in}{1.428505in}}%
\pgfpathlineto{\pgfqpoint{1.309008in}{1.423602in}}%
\pgfpathlineto{\pgfqpoint{1.308621in}{1.418738in}}%
\pgfpathlineto{\pgfqpoint{1.308233in}{1.413914in}}%
\pgfpathlineto{\pgfqpoint{1.319090in}{1.414496in}}%
\pgfpathlineto{\pgfqpoint{1.329975in}{1.414910in}}%
\pgfpathlineto{\pgfqpoint{1.340880in}{1.415154in}}%
\pgfpathlineto{\pgfqpoint{1.351794in}{1.415230in}}%
\pgfpathlineto{\pgfqpoint{1.351789in}{1.420040in}}%
\pgfpathlineto{\pgfqpoint{1.351783in}{1.424892in}}%
\pgfpathlineto{\pgfqpoint{1.351778in}{1.429782in}}%
\pgfpathlineto{\pgfqpoint{1.351772in}{1.434708in}}%
\pgfpathlineto{\pgfqpoint{1.341252in}{1.434636in}}%
\pgfpathlineto{\pgfqpoint{1.330741in}{1.434401in}}%
\pgfpathlineto{\pgfqpoint{1.320248in}{1.434004in}}%
\pgfpathlineto{\pgfqpoint{1.309784in}{1.433444in}}%
\pgfpathclose%
\pgfusepath{fill}%
\end{pgfscope}%
\begin{pgfscope}%
\pgfpathrectangle{\pgfqpoint{0.329460in}{0.284240in}}{\pgfqpoint{1.989680in}{1.989680in}}%
\pgfusepath{clip}%
\pgfsetbuttcap%
\pgfsetroundjoin%
\definecolor{currentfill}{rgb}{0.195860,0.395433,0.555276}%
\pgfsetfillcolor{currentfill}%
\pgfsetlinewidth{0.000000pt}%
\definecolor{currentstroke}{rgb}{0.000000,0.000000,0.000000}%
\pgfsetstrokecolor{currentstroke}%
\pgfsetdash{}{0pt}%
\pgfpathmoveto{\pgfqpoint{1.351772in}{1.434708in}}%
\pgfpathlineto{\pgfqpoint{1.351778in}{1.429782in}}%
\pgfpathlineto{\pgfqpoint{1.351783in}{1.424892in}}%
\pgfpathlineto{\pgfqpoint{1.351789in}{1.420040in}}%
\pgfpathlineto{\pgfqpoint{1.351794in}{1.415230in}}%
\pgfpathlineto{\pgfqpoint{1.362708in}{1.415136in}}%
\pgfpathlineto{\pgfqpoint{1.373611in}{1.414872in}}%
\pgfpathlineto{\pgfqpoint{1.384493in}{1.414440in}}%
\pgfpathlineto{\pgfqpoint{1.395346in}{1.413839in}}%
\pgfpathlineto{\pgfqpoint{1.394948in}{1.418663in}}%
\pgfpathlineto{\pgfqpoint{1.394549in}{1.423528in}}%
\pgfpathlineto{\pgfqpoint{1.394151in}{1.428432in}}%
\pgfpathlineto{\pgfqpoint{1.393752in}{1.433372in}}%
\pgfpathlineto{\pgfqpoint{1.383291in}{1.433949in}}%
\pgfpathlineto{\pgfqpoint{1.372801in}{1.434365in}}%
\pgfpathlineto{\pgfqpoint{1.362292in}{1.434618in}}%
\pgfpathlineto{\pgfqpoint{1.351772in}{1.434708in}}%
\pgfpathclose%
\pgfusepath{fill}%
\end{pgfscope}%
\begin{pgfscope}%
\pgfpathrectangle{\pgfqpoint{0.329460in}{0.284240in}}{\pgfqpoint{1.989680in}{1.989680in}}%
\pgfusepath{clip}%
\pgfsetbuttcap%
\pgfsetroundjoin%
\definecolor{currentfill}{rgb}{0.233603,0.313828,0.543914}%
\pgfsetfillcolor{currentfill}%
\pgfsetlinewidth{0.000000pt}%
\definecolor{currentstroke}{rgb}{0.000000,0.000000,0.000000}%
\pgfsetstrokecolor{currentstroke}%
\pgfsetdash{}{0pt}%
\pgfpathmoveto{\pgfqpoint{0.782803in}{1.312852in}}%
\pgfpathlineto{\pgfqpoint{0.779541in}{1.323441in}}%
\pgfpathlineto{\pgfqpoint{0.776265in}{1.334422in}}%
\pgfpathlineto{\pgfqpoint{0.772974in}{1.345802in}}%
\pgfpathlineto{\pgfqpoint{0.769668in}{1.357587in}}%
\pgfpathlineto{\pgfqpoint{0.779347in}{1.366610in}}%
\pgfpathlineto{\pgfqpoint{0.789554in}{1.375464in}}%
\pgfpathlineto{\pgfqpoint{0.800278in}{1.384142in}}%
\pgfpathlineto{\pgfqpoint{0.811506in}{1.392637in}}%
\pgfpathlineto{\pgfqpoint{0.814557in}{1.380720in}}%
\pgfpathlineto{\pgfqpoint{0.817594in}{1.369206in}}%
\pgfpathlineto{\pgfqpoint{0.820618in}{1.358089in}}%
\pgfpathlineto{\pgfqpoint{0.823628in}{1.347363in}}%
\pgfpathlineto{\pgfqpoint{0.812669in}{1.338997in}}%
\pgfpathlineto{\pgfqpoint{0.802204in}{1.330452in}}%
\pgfpathlineto{\pgfqpoint{0.792245in}{1.321735in}}%
\pgfpathlineto{\pgfqpoint{0.782803in}{1.312852in}}%
\pgfpathclose%
\pgfusepath{fill}%
\end{pgfscope}%
\begin{pgfscope}%
\pgfpathrectangle{\pgfqpoint{0.329460in}{0.284240in}}{\pgfqpoint{1.989680in}{1.989680in}}%
\pgfusepath{clip}%
\pgfsetbuttcap%
\pgfsetroundjoin%
\definecolor{currentfill}{rgb}{0.268510,0.009605,0.335427}%
\pgfsetfillcolor{currentfill}%
\pgfsetlinewidth{0.000000pt}%
\definecolor{currentstroke}{rgb}{0.000000,0.000000,0.000000}%
\pgfsetstrokecolor{currentstroke}%
\pgfsetdash{}{0pt}%
\pgfpathmoveto{\pgfqpoint{0.842611in}{1.120071in}}%
\pgfpathlineto{\pgfqpoint{0.839442in}{1.121761in}}%
\pgfpathlineto{\pgfqpoint{0.836266in}{1.123695in}}%
\pgfpathlineto{\pgfqpoint{0.833083in}{1.125879in}}%
\pgfpathlineto{\pgfqpoint{0.829893in}{1.128316in}}%
\pgfpathlineto{\pgfqpoint{0.836096in}{1.136726in}}%
\pgfpathlineto{\pgfqpoint{0.842788in}{1.145020in}}%
\pgfpathlineto{\pgfqpoint{0.849961in}{1.153192in}}%
\pgfpathlineto{\pgfqpoint{0.857606in}{1.161233in}}%
\pgfpathlineto{\pgfqpoint{0.860611in}{1.158608in}}%
\pgfpathlineto{\pgfqpoint{0.863610in}{1.156235in}}%
\pgfpathlineto{\pgfqpoint{0.866602in}{1.154111in}}%
\pgfpathlineto{\pgfqpoint{0.869587in}{1.152231in}}%
\pgfpathlineto{\pgfqpoint{0.862143in}{1.144374in}}%
\pgfpathlineto{\pgfqpoint{0.855160in}{1.136390in}}%
\pgfpathlineto{\pgfqpoint{0.848646in}{1.128287in}}%
\pgfpathlineto{\pgfqpoint{0.842611in}{1.120071in}}%
\pgfpathclose%
\pgfusepath{fill}%
\end{pgfscope}%
\begin{pgfscope}%
\pgfpathrectangle{\pgfqpoint{0.329460in}{0.284240in}}{\pgfqpoint{1.989680in}{1.989680in}}%
\pgfusepath{clip}%
\pgfsetbuttcap%
\pgfsetroundjoin%
\definecolor{currentfill}{rgb}{0.280255,0.165693,0.476498}%
\pgfsetfillcolor{currentfill}%
\pgfsetlinewidth{0.000000pt}%
\definecolor{currentstroke}{rgb}{0.000000,0.000000,0.000000}%
\pgfsetstrokecolor{currentstroke}%
\pgfsetdash{}{0pt}%
\pgfpathmoveto{\pgfqpoint{1.001056in}{1.235892in}}%
\pgfpathlineto{\pgfqpoint{0.998378in}{1.231830in}}%
\pgfpathlineto{\pgfqpoint{0.995699in}{1.227877in}}%
\pgfpathlineto{\pgfqpoint{0.993021in}{1.224034in}}%
\pgfpathlineto{\pgfqpoint{0.990342in}{1.220305in}}%
\pgfpathlineto{\pgfqpoint{0.998019in}{1.226017in}}%
\pgfpathlineto{\pgfqpoint{1.006028in}{1.231598in}}%
\pgfpathlineto{\pgfqpoint{1.014359in}{1.237044in}}%
\pgfpathlineto{\pgfqpoint{1.023004in}{1.242349in}}%
\pgfpathlineto{\pgfqpoint{1.025432in}{1.245902in}}%
\pgfpathlineto{\pgfqpoint{1.027860in}{1.249569in}}%
\pgfpathlineto{\pgfqpoint{1.030288in}{1.253346in}}%
\pgfpathlineto{\pgfqpoint{1.032716in}{1.257231in}}%
\pgfpathlineto{\pgfqpoint{1.024334in}{1.252094in}}%
\pgfpathlineto{\pgfqpoint{1.016258in}{1.246823in}}%
\pgfpathlineto{\pgfqpoint{1.008496in}{1.241420in}}%
\pgfpathlineto{\pgfqpoint{1.001056in}{1.235892in}}%
\pgfpathclose%
\pgfusepath{fill}%
\end{pgfscope}%
\begin{pgfscope}%
\pgfpathrectangle{\pgfqpoint{0.329460in}{0.284240in}}{\pgfqpoint{1.989680in}{1.989680in}}%
\pgfusepath{clip}%
\pgfsetbuttcap%
\pgfsetroundjoin%
\definecolor{currentfill}{rgb}{0.268510,0.009605,0.335427}%
\pgfsetfillcolor{currentfill}%
\pgfsetlinewidth{0.000000pt}%
\definecolor{currentstroke}{rgb}{0.000000,0.000000,0.000000}%
\pgfsetstrokecolor{currentstroke}%
\pgfsetdash{}{0pt}%
\pgfpathmoveto{\pgfqpoint{1.790910in}{1.153496in}}%
\pgfpathlineto{\pgfqpoint{1.793793in}{1.152825in}}%
\pgfpathlineto{\pgfqpoint{1.796681in}{1.152347in}}%
\pgfpathlineto{\pgfqpoint{1.799572in}{1.152067in}}%
\pgfpathlineto{\pgfqpoint{1.802467in}{1.151988in}}%
\pgfpathlineto{\pgfqpoint{1.809901in}{1.144622in}}%
\pgfpathlineto{\pgfqpoint{1.816902in}{1.137130in}}%
\pgfpathlineto{\pgfqpoint{1.823463in}{1.129520in}}%
\pgfpathlineto{\pgfqpoint{1.829575in}{1.121798in}}%
\pgfpathlineto{\pgfqpoint{1.826492in}{1.122071in}}%
\pgfpathlineto{\pgfqpoint{1.823413in}{1.122547in}}%
\pgfpathlineto{\pgfqpoint{1.820338in}{1.123221in}}%
\pgfpathlineto{\pgfqpoint{1.817267in}{1.124090in}}%
\pgfpathlineto{\pgfqpoint{1.811326in}{1.131611in}}%
\pgfpathlineto{\pgfqpoint{1.804948in}{1.139024in}}%
\pgfpathlineto{\pgfqpoint{1.798140in}{1.146321in}}%
\pgfpathlineto{\pgfqpoint{1.790910in}{1.153496in}}%
\pgfpathclose%
\pgfusepath{fill}%
\end{pgfscope}%
\begin{pgfscope}%
\pgfpathrectangle{\pgfqpoint{0.329460in}{0.284240in}}{\pgfqpoint{1.989680in}{1.989680in}}%
\pgfusepath{clip}%
\pgfsetbuttcap%
\pgfsetroundjoin%
\definecolor{currentfill}{rgb}{0.272594,0.025563,0.353093}%
\pgfsetfillcolor{currentfill}%
\pgfsetlinewidth{0.000000pt}%
\definecolor{currentstroke}{rgb}{0.000000,0.000000,0.000000}%
\pgfsetstrokecolor{currentstroke}%
\pgfsetdash{}{0pt}%
\pgfpathmoveto{\pgfqpoint{0.829893in}{1.128316in}}%
\pgfpathlineto{\pgfqpoint{0.826696in}{1.131012in}}%
\pgfpathlineto{\pgfqpoint{0.823491in}{1.133971in}}%
\pgfpathlineto{\pgfqpoint{0.820277in}{1.137198in}}%
\pgfpathlineto{\pgfqpoint{0.817056in}{1.140698in}}%
\pgfpathlineto{\pgfqpoint{0.823429in}{1.149298in}}%
\pgfpathlineto{\pgfqpoint{0.830302in}{1.157779in}}%
\pgfpathlineto{\pgfqpoint{0.837667in}{1.166134in}}%
\pgfpathlineto{\pgfqpoint{0.845516in}{1.174356in}}%
\pgfpathlineto{\pgfqpoint{0.848550in}{1.170673in}}%
\pgfpathlineto{\pgfqpoint{0.851576in}{1.167261in}}%
\pgfpathlineto{\pgfqpoint{0.854595in}{1.164116in}}%
\pgfpathlineto{\pgfqpoint{0.857606in}{1.161233in}}%
\pgfpathlineto{\pgfqpoint{0.849961in}{1.153192in}}%
\pgfpathlineto{\pgfqpoint{0.842788in}{1.145020in}}%
\pgfpathlineto{\pgfqpoint{0.836096in}{1.136726in}}%
\pgfpathlineto{\pgfqpoint{0.829893in}{1.128316in}}%
\pgfpathclose%
\pgfusepath{fill}%
\end{pgfscope}%
\begin{pgfscope}%
\pgfpathrectangle{\pgfqpoint{0.329460in}{0.284240in}}{\pgfqpoint{1.989680in}{1.989680in}}%
\pgfusepath{clip}%
\pgfsetbuttcap%
\pgfsetroundjoin%
\definecolor{currentfill}{rgb}{0.267004,0.004874,0.329415}%
\pgfsetfillcolor{currentfill}%
\pgfsetlinewidth{0.000000pt}%
\definecolor{currentstroke}{rgb}{0.000000,0.000000,0.000000}%
\pgfsetstrokecolor{currentstroke}%
\pgfsetdash{}{0pt}%
\pgfpathmoveto{\pgfqpoint{0.855225in}{1.115673in}}%
\pgfpathlineto{\pgfqpoint{0.852080in}{1.116428in}}%
\pgfpathlineto{\pgfqpoint{0.848930in}{1.117409in}}%
\pgfpathlineto{\pgfqpoint{0.845773in}{1.118622in}}%
\pgfpathlineto{\pgfqpoint{0.842611in}{1.120071in}}%
\pgfpathlineto{\pgfqpoint{0.848646in}{1.128287in}}%
\pgfpathlineto{\pgfqpoint{0.855160in}{1.136390in}}%
\pgfpathlineto{\pgfqpoint{0.862143in}{1.144374in}}%
\pgfpathlineto{\pgfqpoint{0.869587in}{1.152231in}}%
\pgfpathlineto{\pgfqpoint{0.872567in}{1.150590in}}%
\pgfpathlineto{\pgfqpoint{0.875541in}{1.149184in}}%
\pgfpathlineto{\pgfqpoint{0.878510in}{1.148009in}}%
\pgfpathlineto{\pgfqpoint{0.881474in}{1.147060in}}%
\pgfpathlineto{\pgfqpoint{0.874227in}{1.139392in}}%
\pgfpathlineto{\pgfqpoint{0.867432in}{1.131599in}}%
\pgfpathlineto{\pgfqpoint{0.861095in}{1.123691in}}%
\pgfpathlineto{\pgfqpoint{0.855225in}{1.115673in}}%
\pgfpathclose%
\pgfusepath{fill}%
\end{pgfscope}%
\begin{pgfscope}%
\pgfpathrectangle{\pgfqpoint{0.329460in}{0.284240in}}{\pgfqpoint{1.989680in}{1.989680in}}%
\pgfusepath{clip}%
\pgfsetbuttcap%
\pgfsetroundjoin%
\definecolor{currentfill}{rgb}{0.212395,0.359683,0.551710}%
\pgfsetfillcolor{currentfill}%
\pgfsetlinewidth{0.000000pt}%
\definecolor{currentstroke}{rgb}{0.000000,0.000000,0.000000}%
\pgfsetstrokecolor{currentstroke}%
\pgfsetdash{}{0pt}%
\pgfpathmoveto{\pgfqpoint{1.183825in}{1.394088in}}%
\pgfpathlineto{\pgfqpoint{1.182313in}{1.389115in}}%
\pgfpathlineto{\pgfqpoint{1.180803in}{1.384188in}}%
\pgfpathlineto{\pgfqpoint{1.179294in}{1.379311in}}%
\pgfpathlineto{\pgfqpoint{1.177786in}{1.374486in}}%
\pgfpathlineto{\pgfqpoint{1.187882in}{1.377114in}}%
\pgfpathlineto{\pgfqpoint{1.198127in}{1.379582in}}%
\pgfpathlineto{\pgfqpoint{1.208513in}{1.381889in}}%
\pgfpathlineto{\pgfqpoint{1.219030in}{1.384031in}}%
\pgfpathlineto{\pgfqpoint{1.220178in}{1.388766in}}%
\pgfpathlineto{\pgfqpoint{1.221326in}{1.393553in}}%
\pgfpathlineto{\pgfqpoint{1.222475in}{1.398389in}}%
\pgfpathlineto{\pgfqpoint{1.223626in}{1.403273in}}%
\pgfpathlineto{\pgfqpoint{1.213476in}{1.401211in}}%
\pgfpathlineto{\pgfqpoint{1.203453in}{1.398992in}}%
\pgfpathlineto{\pgfqpoint{1.193566in}{1.396617in}}%
\pgfpathlineto{\pgfqpoint{1.183825in}{1.394088in}}%
\pgfpathclose%
\pgfusepath{fill}%
\end{pgfscope}%
\begin{pgfscope}%
\pgfpathrectangle{\pgfqpoint{0.329460in}{0.284240in}}{\pgfqpoint{1.989680in}{1.989680in}}%
\pgfusepath{clip}%
\pgfsetbuttcap%
\pgfsetroundjoin%
\definecolor{currentfill}{rgb}{0.277941,0.056324,0.381191}%
\pgfsetfillcolor{currentfill}%
\pgfsetlinewidth{0.000000pt}%
\definecolor{currentstroke}{rgb}{0.000000,0.000000,0.000000}%
\pgfsetstrokecolor{currentstroke}%
\pgfsetdash{}{0pt}%
\pgfpathmoveto{\pgfqpoint{0.817056in}{1.140698in}}%
\pgfpathlineto{\pgfqpoint{0.813826in}{1.144475in}}%
\pgfpathlineto{\pgfqpoint{0.810587in}{1.148534in}}%
\pgfpathlineto{\pgfqpoint{0.807339in}{1.152881in}}%
\pgfpathlineto{\pgfqpoint{0.804081in}{1.157519in}}%
\pgfpathlineto{\pgfqpoint{0.810627in}{1.166305in}}%
\pgfpathlineto{\pgfqpoint{0.817684in}{1.174968in}}%
\pgfpathlineto{\pgfqpoint{0.825245in}{1.183502in}}%
\pgfpathlineto{\pgfqpoint{0.833299in}{1.191899in}}%
\pgfpathlineto{\pgfqpoint{0.836366in}{1.187082in}}%
\pgfpathlineto{\pgfqpoint{0.839424in}{1.182555in}}%
\pgfpathlineto{\pgfqpoint{0.842474in}{1.178315in}}%
\pgfpathlineto{\pgfqpoint{0.845516in}{1.174356in}}%
\pgfpathlineto{\pgfqpoint{0.837667in}{1.166134in}}%
\pgfpathlineto{\pgfqpoint{0.830302in}{1.157779in}}%
\pgfpathlineto{\pgfqpoint{0.823429in}{1.149298in}}%
\pgfpathlineto{\pgfqpoint{0.817056in}{1.140698in}}%
\pgfpathclose%
\pgfusepath{fill}%
\end{pgfscope}%
\begin{pgfscope}%
\pgfpathrectangle{\pgfqpoint{0.329460in}{0.284240in}}{\pgfqpoint{1.989680in}{1.989680in}}%
\pgfusepath{clip}%
\pgfsetbuttcap%
\pgfsetroundjoin%
\definecolor{currentfill}{rgb}{0.195860,0.395433,0.555276}%
\pgfsetfillcolor{currentfill}%
\pgfsetlinewidth{0.000000pt}%
\definecolor{currentstroke}{rgb}{0.000000,0.000000,0.000000}%
\pgfsetstrokecolor{currentstroke}%
\pgfsetdash{}{0pt}%
\pgfpathmoveto{\pgfqpoint{1.268403in}{1.429598in}}%
\pgfpathlineto{\pgfqpoint{1.267626in}{1.424620in}}%
\pgfpathlineto{\pgfqpoint{1.266850in}{1.419677in}}%
\pgfpathlineto{\pgfqpoint{1.266075in}{1.414773in}}%
\pgfpathlineto{\pgfqpoint{1.265301in}{1.409911in}}%
\pgfpathlineto{\pgfqpoint{1.275940in}{1.411161in}}%
\pgfpathlineto{\pgfqpoint{1.286649in}{1.412246in}}%
\pgfpathlineto{\pgfqpoint{1.297417in}{1.413164in}}%
\pgfpathlineto{\pgfqpoint{1.308233in}{1.413914in}}%
\pgfpathlineto{\pgfqpoint{1.308621in}{1.418738in}}%
\pgfpathlineto{\pgfqpoint{1.309008in}{1.423602in}}%
\pgfpathlineto{\pgfqpoint{1.309396in}{1.428505in}}%
\pgfpathlineto{\pgfqpoint{1.309784in}{1.433444in}}%
\pgfpathlineto{\pgfqpoint{1.299358in}{1.432723in}}%
\pgfpathlineto{\pgfqpoint{1.288980in}{1.431841in}}%
\pgfpathlineto{\pgfqpoint{1.278658in}{1.430799in}}%
\pgfpathlineto{\pgfqpoint{1.268403in}{1.429598in}}%
\pgfpathclose%
\pgfusepath{fill}%
\end{pgfscope}%
\begin{pgfscope}%
\pgfpathrectangle{\pgfqpoint{0.329460in}{0.284240in}}{\pgfqpoint{1.989680in}{1.989680in}}%
\pgfusepath{clip}%
\pgfsetbuttcap%
\pgfsetroundjoin%
\definecolor{currentfill}{rgb}{0.195860,0.395433,0.555276}%
\pgfsetfillcolor{currentfill}%
\pgfsetlinewidth{0.000000pt}%
\definecolor{currentstroke}{rgb}{0.000000,0.000000,0.000000}%
\pgfsetstrokecolor{currentstroke}%
\pgfsetdash{}{0pt}%
\pgfpathmoveto{\pgfqpoint{1.393752in}{1.433372in}}%
\pgfpathlineto{\pgfqpoint{1.394151in}{1.428432in}}%
\pgfpathlineto{\pgfqpoint{1.394549in}{1.423528in}}%
\pgfpathlineto{\pgfqpoint{1.394948in}{1.418663in}}%
\pgfpathlineto{\pgfqpoint{1.395346in}{1.413839in}}%
\pgfpathlineto{\pgfqpoint{1.406158in}{1.413070in}}%
\pgfpathlineto{\pgfqpoint{1.416919in}{1.412134in}}%
\pgfpathlineto{\pgfqpoint{1.427621in}{1.411031in}}%
\pgfpathlineto{\pgfqpoint{1.438252in}{1.409762in}}%
\pgfpathlineto{\pgfqpoint{1.437467in}{1.414626in}}%
\pgfpathlineto{\pgfqpoint{1.436681in}{1.419531in}}%
\pgfpathlineto{\pgfqpoint{1.435894in}{1.424475in}}%
\pgfpathlineto{\pgfqpoint{1.435107in}{1.429455in}}%
\pgfpathlineto{\pgfqpoint{1.424860in}{1.430674in}}%
\pgfpathlineto{\pgfqpoint{1.414545in}{1.431734in}}%
\pgfpathlineto{\pgfqpoint{1.404173in}{1.432633in}}%
\pgfpathlineto{\pgfqpoint{1.393752in}{1.433372in}}%
\pgfpathclose%
\pgfusepath{fill}%
\end{pgfscope}%
\begin{pgfscope}%
\pgfpathrectangle{\pgfqpoint{0.329460in}{0.284240in}}{\pgfqpoint{1.989680in}{1.989680in}}%
\pgfusepath{clip}%
\pgfsetbuttcap%
\pgfsetroundjoin%
\definecolor{currentfill}{rgb}{0.263663,0.237631,0.518762}%
\pgfsetfillcolor{currentfill}%
\pgfsetlinewidth{0.000000pt}%
\definecolor{currentstroke}{rgb}{0.000000,0.000000,0.000000}%
\pgfsetstrokecolor{currentstroke}%
\pgfsetdash{}{0pt}%
\pgfpathmoveto{\pgfqpoint{1.052164in}{1.291793in}}%
\pgfpathlineto{\pgfqpoint{1.049730in}{1.287163in}}%
\pgfpathlineto{\pgfqpoint{1.047297in}{1.282615in}}%
\pgfpathlineto{\pgfqpoint{1.044865in}{1.278153in}}%
\pgfpathlineto{\pgfqpoint{1.042434in}{1.273779in}}%
\pgfpathlineto{\pgfqpoint{1.050840in}{1.278611in}}%
\pgfpathlineto{\pgfqpoint{1.059526in}{1.283303in}}%
\pgfpathlineto{\pgfqpoint{1.068483in}{1.287852in}}%
\pgfpathlineto{\pgfqpoint{1.077703in}{1.292254in}}%
\pgfpathlineto{\pgfqpoint{1.079850in}{1.296470in}}%
\pgfpathlineto{\pgfqpoint{1.081999in}{1.300774in}}%
\pgfpathlineto{\pgfqpoint{1.084147in}{1.305163in}}%
\pgfpathlineto{\pgfqpoint{1.086297in}{1.309635in}}%
\pgfpathlineto{\pgfqpoint{1.077373in}{1.305383in}}%
\pgfpathlineto{\pgfqpoint{1.068704in}{1.300990in}}%
\pgfpathlineto{\pgfqpoint{1.060298in}{1.296458in}}%
\pgfpathlineto{\pgfqpoint{1.052164in}{1.291793in}}%
\pgfpathclose%
\pgfusepath{fill}%
\end{pgfscope}%
\begin{pgfscope}%
\pgfpathrectangle{\pgfqpoint{0.329460in}{0.284240in}}{\pgfqpoint{1.989680in}{1.989680in}}%
\pgfusepath{clip}%
\pgfsetbuttcap%
\pgfsetroundjoin%
\definecolor{currentfill}{rgb}{0.282884,0.135920,0.453427}%
\pgfsetfillcolor{currentfill}%
\pgfsetlinewidth{0.000000pt}%
\definecolor{currentstroke}{rgb}{0.000000,0.000000,0.000000}%
\pgfsetstrokecolor{currentstroke}%
\pgfsetdash{}{0pt}%
\pgfpathmoveto{\pgfqpoint{1.873675in}{1.221671in}}%
\pgfpathlineto{\pgfqpoint{1.876741in}{1.228057in}}%
\pgfpathlineto{\pgfqpoint{1.879816in}{1.234763in}}%
\pgfpathlineto{\pgfqpoint{1.882902in}{1.241797in}}%
\pgfpathlineto{\pgfqpoint{1.885999in}{1.249164in}}%
\pgfpathlineto{\pgfqpoint{1.894926in}{1.240566in}}%
\pgfpathlineto{\pgfqpoint{1.903346in}{1.231818in}}%
\pgfpathlineto{\pgfqpoint{1.911248in}{1.222927in}}%
\pgfpathlineto{\pgfqpoint{1.918624in}{1.213901in}}%
\pgfpathlineto{\pgfqpoint{1.915320in}{1.206698in}}%
\pgfpathlineto{\pgfqpoint{1.912027in}{1.199830in}}%
\pgfpathlineto{\pgfqpoint{1.908746in}{1.193290in}}%
\pgfpathlineto{\pgfqpoint{1.905476in}{1.187074in}}%
\pgfpathlineto{\pgfqpoint{1.898289in}{1.195929in}}%
\pgfpathlineto{\pgfqpoint{1.890587in}{1.204652in}}%
\pgfpathlineto{\pgfqpoint{1.882379in}{1.213235in}}%
\pgfpathlineto{\pgfqpoint{1.873675in}{1.221671in}}%
\pgfpathclose%
\pgfusepath{fill}%
\end{pgfscope}%
\begin{pgfscope}%
\pgfpathrectangle{\pgfqpoint{0.329460in}{0.284240in}}{\pgfqpoint{1.989680in}{1.989680in}}%
\pgfusepath{clip}%
\pgfsetbuttcap%
\pgfsetroundjoin%
\definecolor{currentfill}{rgb}{0.267004,0.004874,0.329415}%
\pgfsetfillcolor{currentfill}%
\pgfsetlinewidth{0.000000pt}%
\definecolor{currentstroke}{rgb}{0.000000,0.000000,0.000000}%
\pgfsetstrokecolor{currentstroke}%
\pgfsetdash{}{0pt}%
\pgfpathmoveto{\pgfqpoint{0.867750in}{1.114845in}}%
\pgfpathlineto{\pgfqpoint{0.864626in}{1.114732in}}%
\pgfpathlineto{\pgfqpoint{0.861498in}{1.114830in}}%
\pgfpathlineto{\pgfqpoint{0.858364in}{1.115142in}}%
\pgfpathlineto{\pgfqpoint{0.855225in}{1.115673in}}%
\pgfpathlineto{\pgfqpoint{0.861095in}{1.123691in}}%
\pgfpathlineto{\pgfqpoint{0.867432in}{1.131599in}}%
\pgfpathlineto{\pgfqpoint{0.874227in}{1.139392in}}%
\pgfpathlineto{\pgfqpoint{0.881474in}{1.147060in}}%
\pgfpathlineto{\pgfqpoint{0.884432in}{1.146334in}}%
\pgfpathlineto{\pgfqpoint{0.887386in}{1.145826in}}%
\pgfpathlineto{\pgfqpoint{0.890335in}{1.145531in}}%
\pgfpathlineto{\pgfqpoint{0.893279in}{1.145447in}}%
\pgfpathlineto{\pgfqpoint{0.886229in}{1.137969in}}%
\pgfpathlineto{\pgfqpoint{0.879619in}{1.130371in}}%
\pgfpathlineto{\pgfqpoint{0.873457in}{1.122661in}}%
\pgfpathlineto{\pgfqpoint{0.867750in}{1.114845in}}%
\pgfpathclose%
\pgfusepath{fill}%
\end{pgfscope}%
\begin{pgfscope}%
\pgfpathrectangle{\pgfqpoint{0.329460in}{0.284240in}}{\pgfqpoint{1.989680in}{1.989680in}}%
\pgfusepath{clip}%
\pgfsetbuttcap%
\pgfsetroundjoin%
\definecolor{currentfill}{rgb}{0.201239,0.383670,0.554294}%
\pgfsetfillcolor{currentfill}%
\pgfsetlinewidth{0.000000pt}%
\definecolor{currentstroke}{rgb}{0.000000,0.000000,0.000000}%
\pgfsetstrokecolor{currentstroke}%
\pgfsetdash{}{0pt}%
\pgfpathmoveto{\pgfqpoint{1.880476in}{1.400028in}}%
\pgfpathlineto{\pgfqpoint{1.883479in}{1.412382in}}%
\pgfpathlineto{\pgfqpoint{1.886496in}{1.425153in}}%
\pgfpathlineto{\pgfqpoint{1.889528in}{1.438346in}}%
\pgfpathlineto{\pgfqpoint{1.892575in}{1.451969in}}%
\pgfpathlineto{\pgfqpoint{1.904524in}{1.443524in}}%
\pgfpathlineto{\pgfqpoint{1.915970in}{1.434887in}}%
\pgfpathlineto{\pgfqpoint{1.926900in}{1.426065in}}%
\pgfpathlineto{\pgfqpoint{1.937302in}{1.417064in}}%
\pgfpathlineto{\pgfqpoint{1.933984in}{1.403560in}}%
\pgfpathlineto{\pgfqpoint{1.930683in}{1.390489in}}%
\pgfpathlineto{\pgfqpoint{1.927399in}{1.377843in}}%
\pgfpathlineto{\pgfqpoint{1.924130in}{1.365615in}}%
\pgfpathlineto{\pgfqpoint{1.913981in}{1.374489in}}%
\pgfpathlineto{\pgfqpoint{1.903314in}{1.383186in}}%
\pgfpathlineto{\pgfqpoint{1.892142in}{1.391702in}}%
\pgfpathlineto{\pgfqpoint{1.880476in}{1.400028in}}%
\pgfpathclose%
\pgfusepath{fill}%
\end{pgfscope}%
\begin{pgfscope}%
\pgfpathrectangle{\pgfqpoint{0.329460in}{0.284240in}}{\pgfqpoint{1.989680in}{1.989680in}}%
\pgfusepath{clip}%
\pgfsetbuttcap%
\pgfsetroundjoin%
\definecolor{currentfill}{rgb}{0.282327,0.094955,0.417331}%
\pgfsetfillcolor{currentfill}%
\pgfsetlinewidth{0.000000pt}%
\definecolor{currentstroke}{rgb}{0.000000,0.000000,0.000000}%
\pgfsetstrokecolor{currentstroke}%
\pgfsetdash{}{0pt}%
\pgfpathmoveto{\pgfqpoint{0.951533in}{1.181778in}}%
\pgfpathlineto{\pgfqpoint{0.948639in}{1.178478in}}%
\pgfpathlineto{\pgfqpoint{0.945743in}{1.175312in}}%
\pgfpathlineto{\pgfqpoint{0.942847in}{1.172285in}}%
\pgfpathlineto{\pgfqpoint{0.939949in}{1.169399in}}%
\pgfpathlineto{\pgfqpoint{0.946620in}{1.175977in}}%
\pgfpathlineto{\pgfqpoint{0.953675in}{1.182438in}}%
\pgfpathlineto{\pgfqpoint{0.961103in}{1.188776in}}%
\pgfpathlineto{\pgfqpoint{0.968899in}{1.194985in}}%
\pgfpathlineto{\pgfqpoint{0.971582in}{1.197680in}}%
\pgfpathlineto{\pgfqpoint{0.974264in}{1.200516in}}%
\pgfpathlineto{\pgfqpoint{0.976945in}{1.203490in}}%
\pgfpathlineto{\pgfqpoint{0.979626in}{1.206598in}}%
\pgfpathlineto{\pgfqpoint{0.972059in}{1.200574in}}%
\pgfpathlineto{\pgfqpoint{0.964850in}{1.194426in}}%
\pgfpathlineto{\pgfqpoint{0.958006in}{1.188159in}}%
\pgfpathlineto{\pgfqpoint{0.951533in}{1.181778in}}%
\pgfpathclose%
\pgfusepath{fill}%
\end{pgfscope}%
\begin{pgfscope}%
\pgfpathrectangle{\pgfqpoint{0.329460in}{0.284240in}}{\pgfqpoint{1.989680in}{1.989680in}}%
\pgfusepath{clip}%
\pgfsetbuttcap%
\pgfsetroundjoin%
\definecolor{currentfill}{rgb}{0.271305,0.019942,0.347269}%
\pgfsetfillcolor{currentfill}%
\pgfsetlinewidth{0.000000pt}%
\definecolor{currentstroke}{rgb}{0.000000,0.000000,0.000000}%
\pgfsetstrokecolor{currentstroke}%
\pgfsetdash{}{0pt}%
\pgfpathmoveto{\pgfqpoint{1.779406in}{1.158039in}}%
\pgfpathlineto{\pgfqpoint{1.782277in}{1.156632in}}%
\pgfpathlineto{\pgfqpoint{1.785152in}{1.155404in}}%
\pgfpathlineto{\pgfqpoint{1.788029in}{1.154357in}}%
\pgfpathlineto{\pgfqpoint{1.790910in}{1.153496in}}%
\pgfpathlineto{\pgfqpoint{1.798140in}{1.146321in}}%
\pgfpathlineto{\pgfqpoint{1.804948in}{1.139024in}}%
\pgfpathlineto{\pgfqpoint{1.811326in}{1.131611in}}%
\pgfpathlineto{\pgfqpoint{1.817267in}{1.124090in}}%
\pgfpathlineto{\pgfqpoint{1.814200in}{1.125148in}}%
\pgfpathlineto{\pgfqpoint{1.811136in}{1.126393in}}%
\pgfpathlineto{\pgfqpoint{1.808076in}{1.127820in}}%
\pgfpathlineto{\pgfqpoint{1.805019in}{1.129426in}}%
\pgfpathlineto{\pgfqpoint{1.799248in}{1.136743in}}%
\pgfpathlineto{\pgfqpoint{1.793051in}{1.143956in}}%
\pgfpathlineto{\pgfqpoint{1.786434in}{1.151057in}}%
\pgfpathlineto{\pgfqpoint{1.779406in}{1.158039in}}%
\pgfpathclose%
\pgfusepath{fill}%
\end{pgfscope}%
\begin{pgfscope}%
\pgfpathrectangle{\pgfqpoint{0.329460in}{0.284240in}}{\pgfqpoint{1.989680in}{1.989680in}}%
\pgfusepath{clip}%
\pgfsetbuttcap%
\pgfsetroundjoin%
\definecolor{currentfill}{rgb}{0.282327,0.094955,0.417331}%
\pgfsetfillcolor{currentfill}%
\pgfsetlinewidth{0.000000pt}%
\definecolor{currentstroke}{rgb}{0.000000,0.000000,0.000000}%
\pgfsetstrokecolor{currentstroke}%
\pgfsetdash{}{0pt}%
\pgfpathmoveto{\pgfqpoint{0.804081in}{1.157519in}}%
\pgfpathlineto{\pgfqpoint{0.800814in}{1.162455in}}%
\pgfpathlineto{\pgfqpoint{0.797537in}{1.167694in}}%
\pgfpathlineto{\pgfqpoint{0.794249in}{1.173240in}}%
\pgfpathlineto{\pgfqpoint{0.790951in}{1.179099in}}%
\pgfpathlineto{\pgfqpoint{0.797672in}{1.188064in}}%
\pgfpathlineto{\pgfqpoint{0.804916in}{1.196905in}}%
\pgfpathlineto{\pgfqpoint{0.812675in}{1.205612in}}%
\pgfpathlineto{\pgfqpoint{0.820939in}{1.214180in}}%
\pgfpathlineto{\pgfqpoint{0.824043in}{1.208148in}}%
\pgfpathlineto{\pgfqpoint{0.827138in}{1.202427in}}%
\pgfpathlineto{\pgfqpoint{0.830223in}{1.197013in}}%
\pgfpathlineto{\pgfqpoint{0.833299in}{1.191899in}}%
\pgfpathlineto{\pgfqpoint{0.825245in}{1.183502in}}%
\pgfpathlineto{\pgfqpoint{0.817684in}{1.174968in}}%
\pgfpathlineto{\pgfqpoint{0.810627in}{1.166305in}}%
\pgfpathlineto{\pgfqpoint{0.804081in}{1.157519in}}%
\pgfpathclose%
\pgfusepath{fill}%
\end{pgfscope}%
\begin{pgfscope}%
\pgfpathrectangle{\pgfqpoint{0.329460in}{0.284240in}}{\pgfqpoint{1.989680in}{1.989680in}}%
\pgfusepath{clip}%
\pgfsetbuttcap%
\pgfsetroundjoin%
\definecolor{currentfill}{rgb}{0.274128,0.199721,0.498911}%
\pgfsetfillcolor{currentfill}%
\pgfsetlinewidth{0.000000pt}%
\definecolor{currentstroke}{rgb}{0.000000,0.000000,0.000000}%
\pgfsetstrokecolor{currentstroke}%
\pgfsetdash{}{0pt}%
\pgfpathmoveto{\pgfqpoint{1.652483in}{1.278081in}}%
\pgfpathlineto{\pgfqpoint{1.654853in}{1.273835in}}%
\pgfpathlineto{\pgfqpoint{1.657223in}{1.269684in}}%
\pgfpathlineto{\pgfqpoint{1.659592in}{1.265631in}}%
\pgfpathlineto{\pgfqpoint{1.661960in}{1.261679in}}%
\pgfpathlineto{\pgfqpoint{1.670606in}{1.256667in}}%
\pgfpathlineto{\pgfqpoint{1.678953in}{1.251515in}}%
\pgfpathlineto{\pgfqpoint{1.686995in}{1.246229in}}%
\pgfpathlineto{\pgfqpoint{1.694721in}{1.240812in}}%
\pgfpathlineto{\pgfqpoint{1.692095in}{1.244937in}}%
\pgfpathlineto{\pgfqpoint{1.689468in}{1.249163in}}%
\pgfpathlineto{\pgfqpoint{1.686840in}{1.253487in}}%
\pgfpathlineto{\pgfqpoint{1.684212in}{1.257905in}}%
\pgfpathlineto{\pgfqpoint{1.676731in}{1.263142in}}%
\pgfpathlineto{\pgfqpoint{1.668943in}{1.268253in}}%
\pgfpathlineto{\pgfqpoint{1.660858in}{1.273234in}}%
\pgfpathlineto{\pgfqpoint{1.652483in}{1.278081in}}%
\pgfpathclose%
\pgfusepath{fill}%
\end{pgfscope}%
\begin{pgfscope}%
\pgfpathrectangle{\pgfqpoint{0.329460in}{0.284240in}}{\pgfqpoint{1.989680in}{1.989680in}}%
\pgfusepath{clip}%
\pgfsetbuttcap%
\pgfsetroundjoin%
\definecolor{currentfill}{rgb}{0.268510,0.009605,0.335427}%
\pgfsetfillcolor{currentfill}%
\pgfsetlinewidth{0.000000pt}%
\definecolor{currentstroke}{rgb}{0.000000,0.000000,0.000000}%
\pgfsetstrokecolor{currentstroke}%
\pgfsetdash{}{0pt}%
\pgfpathmoveto{\pgfqpoint{0.880201in}{1.117318in}}%
\pgfpathlineto{\pgfqpoint{0.877095in}{1.116404in}}%
\pgfpathlineto{\pgfqpoint{0.873984in}{1.115685in}}%
\pgfpathlineto{\pgfqpoint{0.870869in}{1.115164in}}%
\pgfpathlineto{\pgfqpoint{0.867750in}{1.114845in}}%
\pgfpathlineto{\pgfqpoint{0.873457in}{1.122661in}}%
\pgfpathlineto{\pgfqpoint{0.879619in}{1.130371in}}%
\pgfpathlineto{\pgfqpoint{0.886229in}{1.137969in}}%
\pgfpathlineto{\pgfqpoint{0.893279in}{1.145447in}}%
\pgfpathlineto{\pgfqpoint{0.896220in}{1.145568in}}%
\pgfpathlineto{\pgfqpoint{0.899156in}{1.145890in}}%
\pgfpathlineto{\pgfqpoint{0.902089in}{1.146411in}}%
\pgfpathlineto{\pgfqpoint{0.905018in}{1.147124in}}%
\pgfpathlineto{\pgfqpoint{0.898162in}{1.139840in}}%
\pgfpathlineto{\pgfqpoint{0.891736in}{1.132440in}}%
\pgfpathlineto{\pgfqpoint{0.885746in}{1.124931in}}%
\pgfpathlineto{\pgfqpoint{0.880201in}{1.117318in}}%
\pgfpathclose%
\pgfusepath{fill}%
\end{pgfscope}%
\begin{pgfscope}%
\pgfpathrectangle{\pgfqpoint{0.329460in}{0.284240in}}{\pgfqpoint{1.989680in}{1.989680in}}%
\pgfusepath{clip}%
\pgfsetbuttcap%
\pgfsetroundjoin%
\definecolor{currentfill}{rgb}{0.195860,0.395433,0.555276}%
\pgfsetfillcolor{currentfill}%
\pgfsetlinewidth{0.000000pt}%
\definecolor{currentstroke}{rgb}{0.000000,0.000000,0.000000}%
\pgfsetstrokecolor{currentstroke}%
\pgfsetdash{}{0pt}%
\pgfpathmoveto{\pgfqpoint{1.435107in}{1.429455in}}%
\pgfpathlineto{\pgfqpoint{1.435894in}{1.424475in}}%
\pgfpathlineto{\pgfqpoint{1.436681in}{1.419531in}}%
\pgfpathlineto{\pgfqpoint{1.437467in}{1.414626in}}%
\pgfpathlineto{\pgfqpoint{1.438252in}{1.409762in}}%
\pgfpathlineto{\pgfqpoint{1.448804in}{1.408328in}}%
\pgfpathlineto{\pgfqpoint{1.459266in}{1.406731in}}%
\pgfpathlineto{\pgfqpoint{1.469629in}{1.404971in}}%
\pgfpathlineto{\pgfqpoint{1.468560in}{1.409882in}}%
\pgfpathlineto{\pgfqpoint{1.467490in}{1.414835in}}%
\pgfpathlineto{\pgfqpoint{1.466419in}{1.419826in}}%
\pgfpathlineto{\pgfqpoint{1.465347in}{1.424852in}}%
\pgfpathlineto{\pgfqpoint{1.455360in}{1.426543in}}%
\pgfpathlineto{\pgfqpoint{1.445276in}{1.428077in}}%
\pgfpathlineto{\pgfqpoint{1.435107in}{1.429455in}}%
\pgfpathclose%
\pgfusepath{fill}%
\end{pgfscope}%
\begin{pgfscope}%
\pgfpathrectangle{\pgfqpoint{0.329460in}{0.284240in}}{\pgfqpoint{1.989680in}{1.989680in}}%
\pgfusepath{clip}%
\pgfsetbuttcap%
\pgfsetroundjoin%
\definecolor{currentfill}{rgb}{0.195860,0.395433,0.555276}%
\pgfsetfillcolor{currentfill}%
\pgfsetlinewidth{0.000000pt}%
\definecolor{currentstroke}{rgb}{0.000000,0.000000,0.000000}%
\pgfsetstrokecolor{currentstroke}%
\pgfsetdash{}{0pt}%
\pgfpathmoveto{\pgfqpoint{1.228239in}{1.423220in}}%
\pgfpathlineto{\pgfqpoint{1.227084in}{1.418177in}}%
\pgfpathlineto{\pgfqpoint{1.225930in}{1.413169in}}%
\pgfpathlineto{\pgfqpoint{1.224777in}{1.408200in}}%
\pgfpathlineto{\pgfqpoint{1.223626in}{1.403273in}}%
\pgfpathlineto{\pgfqpoint{1.233892in}{1.405175in}}%
\pgfpathlineto{\pgfqpoint{1.244267in}{1.406916in}}%
\pgfpathlineto{\pgfqpoint{1.254740in}{1.408495in}}%
\pgfpathlineto{\pgfqpoint{1.265301in}{1.409911in}}%
\pgfpathlineto{\pgfqpoint{1.266075in}{1.414773in}}%
\pgfpathlineto{\pgfqpoint{1.266850in}{1.419677in}}%
\pgfpathlineto{\pgfqpoint{1.267626in}{1.424620in}}%
\pgfpathlineto{\pgfqpoint{1.268403in}{1.429598in}}%
\pgfpathlineto{\pgfqpoint{1.258225in}{1.428238in}}%
\pgfpathlineto{\pgfqpoint{1.248131in}{1.426721in}}%
\pgfpathlineto{\pgfqpoint{1.238133in}{1.425048in}}%
\pgfpathlineto{\pgfqpoint{1.228239in}{1.423220in}}%
\pgfpathclose%
\pgfusepath{fill}%
\end{pgfscope}%
\begin{pgfscope}%
\pgfpathrectangle{\pgfqpoint{0.329460in}{0.284240in}}{\pgfqpoint{1.989680in}{1.989680in}}%
\pgfusepath{clip}%
\pgfsetbuttcap%
\pgfsetroundjoin%
\definecolor{currentfill}{rgb}{0.283072,0.130895,0.449241}%
\pgfsetfillcolor{currentfill}%
\pgfsetlinewidth{0.000000pt}%
\definecolor{currentstroke}{rgb}{0.000000,0.000000,0.000000}%
\pgfsetstrokecolor{currentstroke}%
\pgfsetdash{}{0pt}%
\pgfpathmoveto{\pgfqpoint{1.705225in}{1.225389in}}%
\pgfpathlineto{\pgfqpoint{1.707851in}{1.221818in}}%
\pgfpathlineto{\pgfqpoint{1.710477in}{1.218368in}}%
\pgfpathlineto{\pgfqpoint{1.713104in}{1.215042in}}%
\pgfpathlineto{\pgfqpoint{1.715731in}{1.211843in}}%
\pgfpathlineto{\pgfqpoint{1.723608in}{1.205935in}}%
\pgfpathlineto{\pgfqpoint{1.731135in}{1.199897in}}%
\pgfpathlineto{\pgfqpoint{1.738304in}{1.193735in}}%
\pgfpathlineto{\pgfqpoint{1.745107in}{1.187455in}}%
\pgfpathlineto{\pgfqpoint{1.742258in}{1.190842in}}%
\pgfpathlineto{\pgfqpoint{1.739409in}{1.194357in}}%
\pgfpathlineto{\pgfqpoint{1.736560in}{1.197997in}}%
\pgfpathlineto{\pgfqpoint{1.733712in}{1.201757in}}%
\pgfpathlineto{\pgfqpoint{1.727116in}{1.207842in}}%
\pgfpathlineto{\pgfqpoint{1.720165in}{1.213812in}}%
\pgfpathlineto{\pgfqpoint{1.712865in}{1.219663in}}%
\pgfpathlineto{\pgfqpoint{1.705225in}{1.225389in}}%
\pgfpathclose%
\pgfusepath{fill}%
\end{pgfscope}%
\begin{pgfscope}%
\pgfpathrectangle{\pgfqpoint{0.329460in}{0.284240in}}{\pgfqpoint{1.989680in}{1.989680in}}%
\pgfusepath{clip}%
\pgfsetbuttcap%
\pgfsetroundjoin%
\definecolor{currentfill}{rgb}{0.231674,0.318106,0.544834}%
\pgfsetfillcolor{currentfill}%
\pgfsetlinewidth{0.000000pt}%
\definecolor{currentstroke}{rgb}{0.000000,0.000000,0.000000}%
\pgfsetstrokecolor{currentstroke}%
\pgfsetdash{}{0pt}%
\pgfpathmoveto{\pgfqpoint{1.554934in}{1.365313in}}%
\pgfpathlineto{\pgfqpoint{1.556706in}{1.360457in}}%
\pgfpathlineto{\pgfqpoint{1.558478in}{1.355659in}}%
\pgfpathlineto{\pgfqpoint{1.560247in}{1.350922in}}%
\pgfpathlineto{\pgfqpoint{1.562016in}{1.346250in}}%
\pgfpathlineto{\pgfqpoint{1.571726in}{1.342866in}}%
\pgfpathlineto{\pgfqpoint{1.581232in}{1.339329in}}%
\pgfpathlineto{\pgfqpoint{1.590525in}{1.335643in}}%
\pgfpathlineto{\pgfqpoint{1.599596in}{1.331810in}}%
\pgfpathlineto{\pgfqpoint{1.597507in}{1.336614in}}%
\pgfpathlineto{\pgfqpoint{1.595417in}{1.341483in}}%
\pgfpathlineto{\pgfqpoint{1.593325in}{1.346413in}}%
\pgfpathlineto{\pgfqpoint{1.591232in}{1.351402in}}%
\pgfpathlineto{\pgfqpoint{1.582471in}{1.355094in}}%
\pgfpathlineto{\pgfqpoint{1.573495in}{1.358645in}}%
\pgfpathlineto{\pgfqpoint{1.564313in}{1.362053in}}%
\pgfpathlineto{\pgfqpoint{1.554934in}{1.365313in}}%
\pgfpathclose%
\pgfusepath{fill}%
\end{pgfscope}%
\begin{pgfscope}%
\pgfpathrectangle{\pgfqpoint{0.329460in}{0.284240in}}{\pgfqpoint{1.989680in}{1.989680in}}%
\pgfusepath{clip}%
\pgfsetbuttcap%
\pgfsetroundjoin%
\definecolor{currentfill}{rgb}{0.276194,0.190074,0.493001}%
\pgfsetfillcolor{currentfill}%
\pgfsetlinewidth{0.000000pt}%
\definecolor{currentstroke}{rgb}{0.000000,0.000000,0.000000}%
\pgfsetstrokecolor{currentstroke}%
\pgfsetdash{}{0pt}%
\pgfpathmoveto{\pgfqpoint{1.885999in}{1.249164in}}%
\pgfpathlineto{\pgfqpoint{1.889108in}{1.256869in}}%
\pgfpathlineto{\pgfqpoint{1.892227in}{1.264917in}}%
\pgfpathlineto{\pgfqpoint{1.895359in}{1.273315in}}%
\pgfpathlineto{\pgfqpoint{1.898502in}{1.282069in}}%
\pgfpathlineto{\pgfqpoint{1.907656in}{1.273316in}}%
\pgfpathlineto{\pgfqpoint{1.916291in}{1.264409in}}%
\pgfpathlineto{\pgfqpoint{1.924397in}{1.255357in}}%
\pgfpathlineto{\pgfqpoint{1.931966in}{1.246166in}}%
\pgfpathlineto{\pgfqpoint{1.928611in}{1.237570in}}%
\pgfpathlineto{\pgfqpoint{1.925269in}{1.229331in}}%
\pgfpathlineto{\pgfqpoint{1.921940in}{1.221443in}}%
\pgfpathlineto{\pgfqpoint{1.918624in}{1.213901in}}%
\pgfpathlineto{\pgfqpoint{1.911248in}{1.222927in}}%
\pgfpathlineto{\pgfqpoint{1.903346in}{1.231818in}}%
\pgfpathlineto{\pgfqpoint{1.894926in}{1.240566in}}%
\pgfpathlineto{\pgfqpoint{1.885999in}{1.249164in}}%
\pgfpathclose%
\pgfusepath{fill}%
\end{pgfscope}%
\begin{pgfscope}%
\pgfpathrectangle{\pgfqpoint{0.329460in}{0.284240in}}{\pgfqpoint{1.989680in}{1.989680in}}%
\pgfusepath{clip}%
\pgfsetbuttcap%
\pgfsetroundjoin%
\definecolor{currentfill}{rgb}{0.212395,0.359683,0.551710}%
\pgfsetfillcolor{currentfill}%
\pgfsetlinewidth{0.000000pt}%
\definecolor{currentstroke}{rgb}{0.000000,0.000000,0.000000}%
\pgfsetstrokecolor{currentstroke}%
\pgfsetdash{}{0pt}%
\pgfpathmoveto{\pgfqpoint{1.509899in}{1.396344in}}%
\pgfpathlineto{\pgfqpoint{1.511331in}{1.391392in}}%
\pgfpathlineto{\pgfqpoint{1.512763in}{1.386487in}}%
\pgfpathlineto{\pgfqpoint{1.514193in}{1.381632in}}%
\pgfpathlineto{\pgfqpoint{1.515623in}{1.376830in}}%
\pgfpathlineto{\pgfqpoint{1.525701in}{1.374184in}}%
\pgfpathlineto{\pgfqpoint{1.535618in}{1.371381in}}%
\pgfpathlineto{\pgfqpoint{1.545366in}{1.368423in}}%
\pgfpathlineto{\pgfqpoint{1.554934in}{1.365313in}}%
\pgfpathlineto{\pgfqpoint{1.553159in}{1.370224in}}%
\pgfpathlineto{\pgfqpoint{1.551384in}{1.375189in}}%
\pgfpathlineto{\pgfqpoint{1.549607in}{1.380202in}}%
\pgfpathlineto{\pgfqpoint{1.547828in}{1.385263in}}%
\pgfpathlineto{\pgfqpoint{1.538597in}{1.388256in}}%
\pgfpathlineto{\pgfqpoint{1.529192in}{1.391101in}}%
\pgfpathlineto{\pgfqpoint{1.519623in}{1.393798in}}%
\pgfpathlineto{\pgfqpoint{1.509899in}{1.396344in}}%
\pgfpathclose%
\pgfusepath{fill}%
\end{pgfscope}%
\begin{pgfscope}%
\pgfpathrectangle{\pgfqpoint{0.329460in}{0.284240in}}{\pgfqpoint{1.989680in}{1.989680in}}%
\pgfusepath{clip}%
\pgfsetbuttcap%
\pgfsetroundjoin%
\definecolor{currentfill}{rgb}{0.282884,0.135920,0.453427}%
\pgfsetfillcolor{currentfill}%
\pgfsetlinewidth{0.000000pt}%
\definecolor{currentstroke}{rgb}{0.000000,0.000000,0.000000}%
\pgfsetstrokecolor{currentstroke}%
\pgfsetdash{}{0pt}%
\pgfpathmoveto{\pgfqpoint{0.790951in}{1.179099in}}%
\pgfpathlineto{\pgfqpoint{0.787642in}{1.185276in}}%
\pgfpathlineto{\pgfqpoint{0.784321in}{1.191777in}}%
\pgfpathlineto{\pgfqpoint{0.780989in}{1.198606in}}%
\pgfpathlineto{\pgfqpoint{0.777645in}{1.205771in}}%
\pgfpathlineto{\pgfqpoint{0.784544in}{1.214910in}}%
\pgfpathlineto{\pgfqpoint{0.791979in}{1.223921in}}%
\pgfpathlineto{\pgfqpoint{0.799940in}{1.232797in}}%
\pgfpathlineto{\pgfqpoint{0.808416in}{1.241529in}}%
\pgfpathlineto{\pgfqpoint{0.811563in}{1.234198in}}%
\pgfpathlineto{\pgfqpoint{0.814699in}{1.227199in}}%
\pgfpathlineto{\pgfqpoint{0.817824in}{1.220529in}}%
\pgfpathlineto{\pgfqpoint{0.820939in}{1.214180in}}%
\pgfpathlineto{\pgfqpoint{0.812675in}{1.205612in}}%
\pgfpathlineto{\pgfqpoint{0.804916in}{1.196905in}}%
\pgfpathlineto{\pgfqpoint{0.797672in}{1.188064in}}%
\pgfpathlineto{\pgfqpoint{0.790951in}{1.179099in}}%
\pgfpathclose%
\pgfusepath{fill}%
\end{pgfscope}%
\begin{pgfscope}%
\pgfpathrectangle{\pgfqpoint{0.329460in}{0.284240in}}{\pgfqpoint{1.989680in}{1.989680in}}%
\pgfusepath{clip}%
\pgfsetbuttcap%
\pgfsetroundjoin%
\definecolor{currentfill}{rgb}{0.274952,0.037752,0.364543}%
\pgfsetfillcolor{currentfill}%
\pgfsetlinewidth{0.000000pt}%
\definecolor{currentstroke}{rgb}{0.000000,0.000000,0.000000}%
\pgfsetstrokecolor{currentstroke}%
\pgfsetdash{}{0pt}%
\pgfpathmoveto{\pgfqpoint{1.767945in}{1.165370in}}%
\pgfpathlineto{\pgfqpoint{1.770807in}{1.163289in}}%
\pgfpathlineto{\pgfqpoint{1.773671in}{1.161371in}}%
\pgfpathlineto{\pgfqpoint{1.776537in}{1.159620in}}%
\pgfpathlineto{\pgfqpoint{1.779406in}{1.158039in}}%
\pgfpathlineto{\pgfqpoint{1.786434in}{1.151057in}}%
\pgfpathlineto{\pgfqpoint{1.793051in}{1.143956in}}%
\pgfpathlineto{\pgfqpoint{1.799248in}{1.136743in}}%
\pgfpathlineto{\pgfqpoint{1.805019in}{1.129426in}}%
\pgfpathlineto{\pgfqpoint{1.801965in}{1.131206in}}%
\pgfpathlineto{\pgfqpoint{1.798914in}{1.133157in}}%
\pgfpathlineto{\pgfqpoint{1.795865in}{1.135276in}}%
\pgfpathlineto{\pgfqpoint{1.792819in}{1.137558in}}%
\pgfpathlineto{\pgfqpoint{1.787217in}{1.144670in}}%
\pgfpathlineto{\pgfqpoint{1.781199in}{1.151681in}}%
\pgfpathlineto{\pgfqpoint{1.774773in}{1.158583in}}%
\pgfpathlineto{\pgfqpoint{1.767945in}{1.165370in}}%
\pgfpathclose%
\pgfusepath{fill}%
\end{pgfscope}%
\begin{pgfscope}%
\pgfpathrectangle{\pgfqpoint{0.329460in}{0.284240in}}{\pgfqpoint{1.989680in}{1.989680in}}%
\pgfusepath{clip}%
\pgfsetbuttcap%
\pgfsetroundjoin%
\definecolor{currentfill}{rgb}{0.274128,0.199721,0.498911}%
\pgfsetfillcolor{currentfill}%
\pgfsetlinewidth{0.000000pt}%
\definecolor{currentstroke}{rgb}{0.000000,0.000000,0.000000}%
\pgfsetstrokecolor{currentstroke}%
\pgfsetdash{}{0pt}%
\pgfpathmoveto{\pgfqpoint{1.011776in}{1.253148in}}%
\pgfpathlineto{\pgfqpoint{1.009095in}{1.248689in}}%
\pgfpathlineto{\pgfqpoint{1.006415in}{1.244324in}}%
\pgfpathlineto{\pgfqpoint{1.003736in}{1.240057in}}%
\pgfpathlineto{\pgfqpoint{1.001056in}{1.235892in}}%
\pgfpathlineto{\pgfqpoint{1.008496in}{1.241420in}}%
\pgfpathlineto{\pgfqpoint{1.016258in}{1.246823in}}%
\pgfpathlineto{\pgfqpoint{1.024334in}{1.252094in}}%
\pgfpathlineto{\pgfqpoint{1.032716in}{1.257231in}}%
\pgfpathlineto{\pgfqpoint{1.035144in}{1.261220in}}%
\pgfpathlineto{\pgfqpoint{1.037573in}{1.265309in}}%
\pgfpathlineto{\pgfqpoint{1.040003in}{1.269497in}}%
\pgfpathlineto{\pgfqpoint{1.042434in}{1.273779in}}%
\pgfpathlineto{\pgfqpoint{1.034316in}{1.268813in}}%
\pgfpathlineto{\pgfqpoint{1.026495in}{1.263716in}}%
\pgfpathlineto{\pgfqpoint{1.018979in}{1.258493in}}%
\pgfpathlineto{\pgfqpoint{1.011776in}{1.253148in}}%
\pgfpathclose%
\pgfusepath{fill}%
\end{pgfscope}%
\begin{pgfscope}%
\pgfpathrectangle{\pgfqpoint{0.329460in}{0.284240in}}{\pgfqpoint{1.989680in}{1.989680in}}%
\pgfusepath{clip}%
\pgfsetbuttcap%
\pgfsetroundjoin%
\definecolor{currentfill}{rgb}{0.248629,0.278775,0.534556}%
\pgfsetfillcolor{currentfill}%
\pgfsetlinewidth{0.000000pt}%
\definecolor{currentstroke}{rgb}{0.000000,0.000000,0.000000}%
\pgfsetstrokecolor{currentstroke}%
\pgfsetdash{}{0pt}%
\pgfpathmoveto{\pgfqpoint{1.599596in}{1.331810in}}%
\pgfpathlineto{\pgfqpoint{1.601684in}{1.327072in}}%
\pgfpathlineto{\pgfqpoint{1.603770in}{1.322405in}}%
\pgfpathlineto{\pgfqpoint{1.605855in}{1.317810in}}%
\pgfpathlineto{\pgfqpoint{1.607938in}{1.313292in}}%
\pgfpathlineto{\pgfqpoint{1.617082in}{1.309170in}}%
\pgfpathlineto{\pgfqpoint{1.625978in}{1.304902in}}%
\pgfpathlineto{\pgfqpoint{1.634618in}{1.300493in}}%
\pgfpathlineto{\pgfqpoint{1.642994in}{1.295947in}}%
\pgfpathlineto{\pgfqpoint{1.640620in}{1.300619in}}%
\pgfpathlineto{\pgfqpoint{1.638243in}{1.305367in}}%
\pgfpathlineto{\pgfqpoint{1.635866in}{1.310188in}}%
\pgfpathlineto{\pgfqpoint{1.633487in}{1.315079in}}%
\pgfpathlineto{\pgfqpoint{1.625391in}{1.319465in}}%
\pgfpathlineto{\pgfqpoint{1.617038in}{1.323717in}}%
\pgfpathlineto{\pgfqpoint{1.608437in}{1.327833in}}%
\pgfpathlineto{\pgfqpoint{1.599596in}{1.331810in}}%
\pgfpathclose%
\pgfusepath{fill}%
\end{pgfscope}%
\begin{pgfscope}%
\pgfpathrectangle{\pgfqpoint{0.329460in}{0.284240in}}{\pgfqpoint{1.989680in}{1.989680in}}%
\pgfusepath{clip}%
\pgfsetbuttcap%
\pgfsetroundjoin%
\definecolor{currentfill}{rgb}{0.271305,0.019942,0.347269}%
\pgfsetfillcolor{currentfill}%
\pgfsetlinewidth{0.000000pt}%
\definecolor{currentstroke}{rgb}{0.000000,0.000000,0.000000}%
\pgfsetstrokecolor{currentstroke}%
\pgfsetdash{}{0pt}%
\pgfpathmoveto{\pgfqpoint{0.892591in}{1.122838in}}%
\pgfpathlineto{\pgfqpoint{0.889498in}{1.121186in}}%
\pgfpathlineto{\pgfqpoint{0.886403in}{1.119713in}}%
\pgfpathlineto{\pgfqpoint{0.883304in}{1.118422in}}%
\pgfpathlineto{\pgfqpoint{0.880201in}{1.117318in}}%
\pgfpathlineto{\pgfqpoint{0.885746in}{1.124931in}}%
\pgfpathlineto{\pgfqpoint{0.891736in}{1.132440in}}%
\pgfpathlineto{\pgfqpoint{0.898162in}{1.139840in}}%
\pgfpathlineto{\pgfqpoint{0.905018in}{1.147124in}}%
\pgfpathlineto{\pgfqpoint{0.907943in}{1.148028in}}%
\pgfpathlineto{\pgfqpoint{0.910866in}{1.149117in}}%
\pgfpathlineto{\pgfqpoint{0.913785in}{1.150389in}}%
\pgfpathlineto{\pgfqpoint{0.916701in}{1.151838in}}%
\pgfpathlineto{\pgfqpoint{0.910039in}{1.144750in}}%
\pgfpathlineto{\pgfqpoint{0.903794in}{1.137550in}}%
\pgfpathlineto{\pgfqpoint{0.897976in}{1.130244in}}%
\pgfpathlineto{\pgfqpoint{0.892591in}{1.122838in}}%
\pgfpathclose%
\pgfusepath{fill}%
\end{pgfscope}%
\begin{pgfscope}%
\pgfpathrectangle{\pgfqpoint{0.329460in}{0.284240in}}{\pgfqpoint{1.989680in}{1.989680in}}%
\pgfusepath{clip}%
\pgfsetbuttcap%
\pgfsetroundjoin%
\definecolor{currentfill}{rgb}{0.231674,0.318106,0.544834}%
\pgfsetfillcolor{currentfill}%
\pgfsetlinewidth{0.000000pt}%
\definecolor{currentstroke}{rgb}{0.000000,0.000000,0.000000}%
\pgfsetstrokecolor{currentstroke}%
\pgfsetdash{}{0pt}%
\pgfpathmoveto{\pgfqpoint{1.103543in}{1.348004in}}%
\pgfpathlineto{\pgfqpoint{1.101382in}{1.342983in}}%
\pgfpathlineto{\pgfqpoint{1.099223in}{1.338020in}}%
\pgfpathlineto{\pgfqpoint{1.097065in}{1.333119in}}%
\pgfpathlineto{\pgfqpoint{1.094909in}{1.328282in}}%
\pgfpathlineto{\pgfqpoint{1.103776in}{1.332243in}}%
\pgfpathlineto{\pgfqpoint{1.112872in}{1.336060in}}%
\pgfpathlineto{\pgfqpoint{1.122189in}{1.339730in}}%
\pgfpathlineto{\pgfqpoint{1.131718in}{1.343250in}}%
\pgfpathlineto{\pgfqpoint{1.133560in}{1.347949in}}%
\pgfpathlineto{\pgfqpoint{1.135404in}{1.352713in}}%
\pgfpathlineto{\pgfqpoint{1.137248in}{1.357539in}}%
\pgfpathlineto{\pgfqpoint{1.139095in}{1.362422in}}%
\pgfpathlineto{\pgfqpoint{1.129890in}{1.359031in}}%
\pgfpathlineto{\pgfqpoint{1.120891in}{1.355496in}}%
\pgfpathlineto{\pgfqpoint{1.112105in}{1.351819in}}%
\pgfpathlineto{\pgfqpoint{1.103543in}{1.348004in}}%
\pgfpathclose%
\pgfusepath{fill}%
\end{pgfscope}%
\begin{pgfscope}%
\pgfpathrectangle{\pgfqpoint{0.329460in}{0.284240in}}{\pgfqpoint{1.989680in}{1.989680in}}%
\pgfusepath{clip}%
\pgfsetbuttcap%
\pgfsetroundjoin%
\definecolor{currentfill}{rgb}{0.201239,0.383670,0.554294}%
\pgfsetfillcolor{currentfill}%
\pgfsetlinewidth{0.000000pt}%
\definecolor{currentstroke}{rgb}{0.000000,0.000000,0.000000}%
\pgfsetstrokecolor{currentstroke}%
\pgfsetdash{}{0pt}%
\pgfpathmoveto{\pgfqpoint{0.769668in}{1.357587in}}%
\pgfpathlineto{\pgfqpoint{0.766345in}{1.369785in}}%
\pgfpathlineto{\pgfqpoint{0.763007in}{1.382402in}}%
\pgfpathlineto{\pgfqpoint{0.759652in}{1.395444in}}%
\pgfpathlineto{\pgfqpoint{0.756281in}{1.408920in}}%
\pgfpathlineto{\pgfqpoint{0.766203in}{1.418073in}}%
\pgfpathlineto{\pgfqpoint{0.776664in}{1.427054in}}%
\pgfpathlineto{\pgfqpoint{0.787652in}{1.435856in}}%
\pgfpathlineto{\pgfqpoint{0.799154in}{1.444472in}}%
\pgfpathlineto{\pgfqpoint{0.802265in}{1.430875in}}%
\pgfpathlineto{\pgfqpoint{0.805360in}{1.417707in}}%
\pgfpathlineto{\pgfqpoint{0.808440in}{1.404964in}}%
\pgfpathlineto{\pgfqpoint{0.811506in}{1.392637in}}%
\pgfpathlineto{\pgfqpoint{0.800278in}{1.384142in}}%
\pgfpathlineto{\pgfqpoint{0.789554in}{1.375464in}}%
\pgfpathlineto{\pgfqpoint{0.779347in}{1.366610in}}%
\pgfpathlineto{\pgfqpoint{0.769668in}{1.357587in}}%
\pgfpathclose%
\pgfusepath{fill}%
\end{pgfscope}%
\begin{pgfscope}%
\pgfpathrectangle{\pgfqpoint{0.329460in}{0.284240in}}{\pgfqpoint{1.989680in}{1.989680in}}%
\pgfusepath{clip}%
\pgfsetbuttcap%
\pgfsetroundjoin%
\definecolor{currentfill}{rgb}{0.212395,0.359683,0.551710}%
\pgfsetfillcolor{currentfill}%
\pgfsetlinewidth{0.000000pt}%
\definecolor{currentstroke}{rgb}{0.000000,0.000000,0.000000}%
\pgfsetstrokecolor{currentstroke}%
\pgfsetdash{}{0pt}%
\pgfpathmoveto{\pgfqpoint{1.146494in}{1.382483in}}%
\pgfpathlineto{\pgfqpoint{1.144642in}{1.377394in}}%
\pgfpathlineto{\pgfqpoint{1.142791in}{1.372353in}}%
\pgfpathlineto{\pgfqpoint{1.140942in}{1.367361in}}%
\pgfpathlineto{\pgfqpoint{1.139095in}{1.362422in}}%
\pgfpathlineto{\pgfqpoint{1.148496in}{1.365666in}}%
\pgfpathlineto{\pgfqpoint{1.158084in}{1.368759in}}%
\pgfpathlineto{\pgfqpoint{1.167851in}{1.371700in}}%
\pgfpathlineto{\pgfqpoint{1.177786in}{1.374486in}}%
\pgfpathlineto{\pgfqpoint{1.179294in}{1.379311in}}%
\pgfpathlineto{\pgfqpoint{1.180803in}{1.384188in}}%
\pgfpathlineto{\pgfqpoint{1.182313in}{1.389115in}}%
\pgfpathlineto{\pgfqpoint{1.183825in}{1.394088in}}%
\pgfpathlineto{\pgfqpoint{1.174238in}{1.391408in}}%
\pgfpathlineto{\pgfqpoint{1.164815in}{1.388579in}}%
\pgfpathlineto{\pgfqpoint{1.155564in}{1.385603in}}%
\pgfpathlineto{\pgfqpoint{1.146494in}{1.382483in}}%
\pgfpathclose%
\pgfusepath{fill}%
\end{pgfscope}%
\begin{pgfscope}%
\pgfpathrectangle{\pgfqpoint{0.329460in}{0.284240in}}{\pgfqpoint{1.989680in}{1.989680in}}%
\pgfusepath{clip}%
\pgfsetbuttcap%
\pgfsetroundjoin%
\definecolor{currentfill}{rgb}{0.283072,0.130895,0.449241}%
\pgfsetfillcolor{currentfill}%
\pgfsetlinewidth{0.000000pt}%
\definecolor{currentstroke}{rgb}{0.000000,0.000000,0.000000}%
\pgfsetstrokecolor{currentstroke}%
\pgfsetdash{}{0pt}%
\pgfpathmoveto{\pgfqpoint{0.963104in}{1.196256in}}%
\pgfpathlineto{\pgfqpoint{0.960212in}{1.192452in}}%
\pgfpathlineto{\pgfqpoint{0.957320in}{1.188768in}}%
\pgfpathlineto{\pgfqpoint{0.954427in}{1.185209in}}%
\pgfpathlineto{\pgfqpoint{0.951533in}{1.181778in}}%
\pgfpathlineto{\pgfqpoint{0.958006in}{1.188159in}}%
\pgfpathlineto{\pgfqpoint{0.964850in}{1.194426in}}%
\pgfpathlineto{\pgfqpoint{0.972059in}{1.200574in}}%
\pgfpathlineto{\pgfqpoint{0.979626in}{1.206598in}}%
\pgfpathlineto{\pgfqpoint{0.982305in}{1.209837in}}%
\pgfpathlineto{\pgfqpoint{0.984985in}{1.213203in}}%
\pgfpathlineto{\pgfqpoint{0.987664in}{1.216694in}}%
\pgfpathlineto{\pgfqpoint{0.990342in}{1.220305in}}%
\pgfpathlineto{\pgfqpoint{0.983004in}{1.214468in}}%
\pgfpathlineto{\pgfqpoint{0.976014in}{1.208511in}}%
\pgfpathlineto{\pgfqpoint{0.969378in}{1.202438in}}%
\pgfpathlineto{\pgfqpoint{0.963104in}{1.196256in}}%
\pgfpathclose%
\pgfusepath{fill}%
\end{pgfscope}%
\begin{pgfscope}%
\pgfpathrectangle{\pgfqpoint{0.329460in}{0.284240in}}{\pgfqpoint{1.989680in}{1.989680in}}%
\pgfusepath{clip}%
\pgfsetbuttcap%
\pgfsetroundjoin%
\definecolor{currentfill}{rgb}{0.195860,0.395433,0.555276}%
\pgfsetfillcolor{currentfill}%
\pgfsetlinewidth{0.000000pt}%
\definecolor{currentstroke}{rgb}{0.000000,0.000000,0.000000}%
\pgfsetstrokecolor{currentstroke}%
\pgfsetdash{}{0pt}%
\pgfpathmoveto{\pgfqpoint{1.465347in}{1.424852in}}%
\pgfpathlineto{\pgfqpoint{1.466419in}{1.419826in}}%
\pgfpathlineto{\pgfqpoint{1.467490in}{1.414835in}}%
\pgfpathlineto{\pgfqpoint{1.468560in}{1.409882in}}%
\pgfpathlineto{\pgfqpoint{1.469629in}{1.404971in}}%
\pgfpathlineto{\pgfqpoint{1.479883in}{1.403051in}}%
\pgfpathlineto{\pgfqpoint{1.490019in}{1.400972in}}%
\pgfpathlineto{\pgfqpoint{1.500027in}{1.398736in}}%
\pgfpathlineto{\pgfqpoint{1.509899in}{1.396344in}}%
\pgfpathlineto{\pgfqpoint{1.508464in}{1.401339in}}%
\pgfpathlineto{\pgfqpoint{1.507029in}{1.406376in}}%
\pgfpathlineto{\pgfqpoint{1.505591in}{1.411452in}}%
\pgfpathlineto{\pgfqpoint{1.504153in}{1.416564in}}%
\pgfpathlineto{\pgfqpoint{1.494641in}{1.418862in}}%
\pgfpathlineto{\pgfqpoint{1.484997in}{1.421010in}}%
\pgfpathlineto{\pgfqpoint{1.475229in}{1.423008in}}%
\pgfpathlineto{\pgfqpoint{1.465347in}{1.424852in}}%
\pgfpathclose%
\pgfusepath{fill}%
\end{pgfscope}%
\begin{pgfscope}%
\pgfpathrectangle{\pgfqpoint{0.329460in}{0.284240in}}{\pgfqpoint{1.989680in}{1.989680in}}%
\pgfusepath{clip}%
\pgfsetbuttcap%
\pgfsetroundjoin%
\definecolor{currentfill}{rgb}{0.248629,0.278775,0.534556}%
\pgfsetfillcolor{currentfill}%
\pgfsetlinewidth{0.000000pt}%
\definecolor{currentstroke}{rgb}{0.000000,0.000000,0.000000}%
\pgfsetstrokecolor{currentstroke}%
\pgfsetdash{}{0pt}%
\pgfpathmoveto{\pgfqpoint{1.061912in}{1.311073in}}%
\pgfpathlineto{\pgfqpoint{1.059473in}{1.306145in}}%
\pgfpathlineto{\pgfqpoint{1.057036in}{1.301287in}}%
\pgfpathlineto{\pgfqpoint{1.054599in}{1.296502in}}%
\pgfpathlineto{\pgfqpoint{1.052164in}{1.291793in}}%
\pgfpathlineto{\pgfqpoint{1.060298in}{1.296458in}}%
\pgfpathlineto{\pgfqpoint{1.068704in}{1.300990in}}%
\pgfpathlineto{\pgfqpoint{1.077373in}{1.305383in}}%
\pgfpathlineto{\pgfqpoint{1.086297in}{1.309635in}}%
\pgfpathlineto{\pgfqpoint{1.088449in}{1.314186in}}%
\pgfpathlineto{\pgfqpoint{1.090601in}{1.318812in}}%
\pgfpathlineto{\pgfqpoint{1.092754in}{1.323512in}}%
\pgfpathlineto{\pgfqpoint{1.094909in}{1.328282in}}%
\pgfpathlineto{\pgfqpoint{1.086281in}{1.324181in}}%
\pgfpathlineto{\pgfqpoint{1.077900in}{1.319944in}}%
\pgfpathlineto{\pgfqpoint{1.069774in}{1.315573in}}%
\pgfpathlineto{\pgfqpoint{1.061912in}{1.311073in}}%
\pgfpathclose%
\pgfusepath{fill}%
\end{pgfscope}%
\begin{pgfscope}%
\pgfpathrectangle{\pgfqpoint{0.329460in}{0.284240in}}{\pgfqpoint{1.989680in}{1.989680in}}%
\pgfusepath{clip}%
\pgfsetbuttcap%
\pgfsetroundjoin%
\definecolor{currentfill}{rgb}{0.179019,0.433756,0.557430}%
\pgfsetfillcolor{currentfill}%
\pgfsetlinewidth{0.000000pt}%
\definecolor{currentstroke}{rgb}{0.000000,0.000000,0.000000}%
\pgfsetstrokecolor{currentstroke}%
\pgfsetdash{}{0pt}%
\pgfpathmoveto{\pgfqpoint{1.311342in}{1.453505in}}%
\pgfpathlineto{\pgfqpoint{1.310952in}{1.448450in}}%
\pgfpathlineto{\pgfqpoint{1.310562in}{1.443419in}}%
\pgfpathlineto{\pgfqpoint{1.310173in}{1.438417in}}%
\pgfpathlineto{\pgfqpoint{1.309784in}{1.433444in}}%
\pgfpathlineto{\pgfqpoint{1.320248in}{1.434004in}}%
\pgfpathlineto{\pgfqpoint{1.330741in}{1.434401in}}%
\pgfpathlineto{\pgfqpoint{1.341252in}{1.434636in}}%
\pgfpathlineto{\pgfqpoint{1.351772in}{1.434708in}}%
\pgfpathlineto{\pgfqpoint{1.351767in}{1.439667in}}%
\pgfpathlineto{\pgfqpoint{1.351761in}{1.444657in}}%
\pgfpathlineto{\pgfqpoint{1.351756in}{1.449675in}}%
\pgfpathlineto{\pgfqpoint{1.351750in}{1.454718in}}%
\pgfpathlineto{\pgfqpoint{1.341625in}{1.454648in}}%
\pgfpathlineto{\pgfqpoint{1.331510in}{1.454423in}}%
\pgfpathlineto{\pgfqpoint{1.321412in}{1.454042in}}%
\pgfpathlineto{\pgfqpoint{1.311342in}{1.453505in}}%
\pgfpathclose%
\pgfusepath{fill}%
\end{pgfscope}%
\begin{pgfscope}%
\pgfpathrectangle{\pgfqpoint{0.329460in}{0.284240in}}{\pgfqpoint{1.989680in}{1.989680in}}%
\pgfusepath{clip}%
\pgfsetbuttcap%
\pgfsetroundjoin%
\definecolor{currentfill}{rgb}{0.179019,0.433756,0.557430}%
\pgfsetfillcolor{currentfill}%
\pgfsetlinewidth{0.000000pt}%
\definecolor{currentstroke}{rgb}{0.000000,0.000000,0.000000}%
\pgfsetstrokecolor{currentstroke}%
\pgfsetdash{}{0pt}%
\pgfpathmoveto{\pgfqpoint{1.351750in}{1.454718in}}%
\pgfpathlineto{\pgfqpoint{1.351756in}{1.449675in}}%
\pgfpathlineto{\pgfqpoint{1.351761in}{1.444657in}}%
\pgfpathlineto{\pgfqpoint{1.351767in}{1.439667in}}%
\pgfpathlineto{\pgfqpoint{1.351772in}{1.434708in}}%
\pgfpathlineto{\pgfqpoint{1.362292in}{1.434618in}}%
\pgfpathlineto{\pgfqpoint{1.372801in}{1.434365in}}%
\pgfpathlineto{\pgfqpoint{1.383291in}{1.433949in}}%
\pgfpathlineto{\pgfqpoint{1.393752in}{1.433372in}}%
\pgfpathlineto{\pgfqpoint{1.393352in}{1.438345in}}%
\pgfpathlineto{\pgfqpoint{1.392952in}{1.443349in}}%
\pgfpathlineto{\pgfqpoint{1.392551in}{1.448380in}}%
\pgfpathlineto{\pgfqpoint{1.392150in}{1.453436in}}%
\pgfpathlineto{\pgfqpoint{1.382084in}{1.453990in}}%
\pgfpathlineto{\pgfqpoint{1.371989in}{1.454388in}}%
\pgfpathlineto{\pgfqpoint{1.361874in}{1.454631in}}%
\pgfpathlineto{\pgfqpoint{1.351750in}{1.454718in}}%
\pgfpathclose%
\pgfusepath{fill}%
\end{pgfscope}%
\begin{pgfscope}%
\pgfpathrectangle{\pgfqpoint{0.329460in}{0.284240in}}{\pgfqpoint{1.989680in}{1.989680in}}%
\pgfusepath{clip}%
\pgfsetbuttcap%
\pgfsetroundjoin%
\definecolor{currentfill}{rgb}{0.195860,0.395433,0.555276}%
\pgfsetfillcolor{currentfill}%
\pgfsetlinewidth{0.000000pt}%
\definecolor{currentstroke}{rgb}{0.000000,0.000000,0.000000}%
\pgfsetstrokecolor{currentstroke}%
\pgfsetdash{}{0pt}%
\pgfpathmoveto{\pgfqpoint{1.189886in}{1.414398in}}%
\pgfpathlineto{\pgfqpoint{1.188368in}{1.409264in}}%
\pgfpathlineto{\pgfqpoint{1.186852in}{1.404166in}}%
\pgfpathlineto{\pgfqpoint{1.185338in}{1.399106in}}%
\pgfpathlineto{\pgfqpoint{1.183825in}{1.394088in}}%
\pgfpathlineto{\pgfqpoint{1.193566in}{1.396617in}}%
\pgfpathlineto{\pgfqpoint{1.203453in}{1.398992in}}%
\pgfpathlineto{\pgfqpoint{1.213476in}{1.401211in}}%
\pgfpathlineto{\pgfqpoint{1.223626in}{1.403273in}}%
\pgfpathlineto{\pgfqpoint{1.224777in}{1.408200in}}%
\pgfpathlineto{\pgfqpoint{1.225930in}{1.413169in}}%
\pgfpathlineto{\pgfqpoint{1.227084in}{1.418177in}}%
\pgfpathlineto{\pgfqpoint{1.228239in}{1.423220in}}%
\pgfpathlineto{\pgfqpoint{1.218458in}{1.421240in}}%
\pgfpathlineto{\pgfqpoint{1.208799in}{1.419108in}}%
\pgfpathlineto{\pgfqpoint{1.199272in}{1.416826in}}%
\pgfpathlineto{\pgfqpoint{1.189886in}{1.414398in}}%
\pgfpathclose%
\pgfusepath{fill}%
\end{pgfscope}%
\begin{pgfscope}%
\pgfpathrectangle{\pgfqpoint{0.329460in}{0.284240in}}{\pgfqpoint{1.989680in}{1.989680in}}%
\pgfusepath{clip}%
\pgfsetbuttcap%
\pgfsetroundjoin%
\definecolor{currentfill}{rgb}{0.172719,0.448791,0.557885}%
\pgfsetfillcolor{currentfill}%
\pgfsetlinewidth{0.000000pt}%
\definecolor{currentstroke}{rgb}{0.000000,0.000000,0.000000}%
\pgfsetstrokecolor{currentstroke}%
\pgfsetdash{}{0pt}%
\pgfpathmoveto{\pgfqpoint{1.892575in}{1.451969in}}%
\pgfpathlineto{\pgfqpoint{1.895637in}{1.466029in}}%
\pgfpathlineto{\pgfqpoint{1.898715in}{1.480533in}}%
\pgfpathlineto{\pgfqpoint{1.901810in}{1.495489in}}%
\pgfpathlineto{\pgfqpoint{1.913976in}{1.486962in}}%
\pgfpathlineto{\pgfqpoint{1.925632in}{1.478240in}}%
\pgfpathlineto{\pgfqpoint{1.936763in}{1.469330in}}%
\pgfpathlineto{\pgfqpoint{1.947358in}{1.460240in}}%
\pgfpathlineto{\pgfqpoint{1.943988in}{1.445396in}}%
\pgfpathlineto{\pgfqpoint{1.940636in}{1.431007in}}%
\pgfpathlineto{\pgfqpoint{1.937302in}{1.417064in}}%
\pgfpathlineto{\pgfqpoint{1.926900in}{1.426065in}}%
\pgfpathlineto{\pgfqpoint{1.915970in}{1.434887in}}%
\pgfpathlineto{\pgfqpoint{1.904524in}{1.443524in}}%
\pgfpathlineto{\pgfqpoint{1.892575in}{1.451969in}}%
\pgfpathclose%
\pgfusepath{fill}%
\end{pgfscope}%
\begin{pgfscope}%
\pgfpathrectangle{\pgfqpoint{0.329460in}{0.284240in}}{\pgfqpoint{1.989680in}{1.989680in}}%
\pgfusepath{clip}%
\pgfsetbuttcap%
\pgfsetroundjoin%
\definecolor{currentfill}{rgb}{0.274952,0.037752,0.364543}%
\pgfsetfillcolor{currentfill}%
\pgfsetlinewidth{0.000000pt}%
\definecolor{currentstroke}{rgb}{0.000000,0.000000,0.000000}%
\pgfsetstrokecolor{currentstroke}%
\pgfsetdash{}{0pt}%
\pgfpathmoveto{\pgfqpoint{0.904931in}{1.131156in}}%
\pgfpathlineto{\pgfqpoint{0.901850in}{1.128827in}}%
\pgfpathlineto{\pgfqpoint{0.898766in}{1.126662in}}%
\pgfpathlineto{\pgfqpoint{0.895680in}{1.124664in}}%
\pgfpathlineto{\pgfqpoint{0.892591in}{1.122838in}}%
\pgfpathlineto{\pgfqpoint{0.897976in}{1.130244in}}%
\pgfpathlineto{\pgfqpoint{0.903794in}{1.137550in}}%
\pgfpathlineto{\pgfqpoint{0.910039in}{1.144750in}}%
\pgfpathlineto{\pgfqpoint{0.916701in}{1.151838in}}%
\pgfpathlineto{\pgfqpoint{0.919615in}{1.153462in}}%
\pgfpathlineto{\pgfqpoint{0.922526in}{1.155257in}}%
\pgfpathlineto{\pgfqpoint{0.925435in}{1.157218in}}%
\pgfpathlineto{\pgfqpoint{0.928342in}{1.159342in}}%
\pgfpathlineto{\pgfqpoint{0.921870in}{1.152453in}}%
\pgfpathlineto{\pgfqpoint{0.915806in}{1.145454in}}%
\pgfpathlineto{\pgfqpoint{0.910158in}{1.138353in}}%
\pgfpathlineto{\pgfqpoint{0.904931in}{1.131156in}}%
\pgfpathclose%
\pgfusepath{fill}%
\end{pgfscope}%
\begin{pgfscope}%
\pgfpathrectangle{\pgfqpoint{0.329460in}{0.284240in}}{\pgfqpoint{1.989680in}{1.989680in}}%
\pgfusepath{clip}%
\pgfsetbuttcap%
\pgfsetroundjoin%
\definecolor{currentfill}{rgb}{0.279566,0.067836,0.391917}%
\pgfsetfillcolor{currentfill}%
\pgfsetlinewidth{0.000000pt}%
\definecolor{currentstroke}{rgb}{0.000000,0.000000,0.000000}%
\pgfsetstrokecolor{currentstroke}%
\pgfsetdash{}{0pt}%
\pgfpathmoveto{\pgfqpoint{1.756515in}{1.175252in}}%
\pgfpathlineto{\pgfqpoint{1.759370in}{1.172555in}}%
\pgfpathlineto{\pgfqpoint{1.762226in}{1.170007in}}%
\pgfpathlineto{\pgfqpoint{1.765085in}{1.167611in}}%
\pgfpathlineto{\pgfqpoint{1.767945in}{1.165370in}}%
\pgfpathlineto{\pgfqpoint{1.774773in}{1.158583in}}%
\pgfpathlineto{\pgfqpoint{1.781199in}{1.151681in}}%
\pgfpathlineto{\pgfqpoint{1.787217in}{1.144670in}}%
\pgfpathlineto{\pgfqpoint{1.792819in}{1.137558in}}%
\pgfpathlineto{\pgfqpoint{1.789775in}{1.140000in}}%
\pgfpathlineto{\pgfqpoint{1.786733in}{1.142598in}}%
\pgfpathlineto{\pgfqpoint{1.783693in}{1.145349in}}%
\pgfpathlineto{\pgfqpoint{1.780655in}{1.148248in}}%
\pgfpathlineto{\pgfqpoint{1.775221in}{1.155153in}}%
\pgfpathlineto{\pgfqpoint{1.769381in}{1.161959in}}%
\pgfpathlineto{\pgfqpoint{1.763144in}{1.168661in}}%
\pgfpathlineto{\pgfqpoint{1.756515in}{1.175252in}}%
\pgfpathclose%
\pgfusepath{fill}%
\end{pgfscope}%
\begin{pgfscope}%
\pgfpathrectangle{\pgfqpoint{0.329460in}{0.284240in}}{\pgfqpoint{1.989680in}{1.989680in}}%
\pgfusepath{clip}%
\pgfsetbuttcap%
\pgfsetroundjoin%
\definecolor{currentfill}{rgb}{0.276194,0.190074,0.493001}%
\pgfsetfillcolor{currentfill}%
\pgfsetlinewidth{0.000000pt}%
\definecolor{currentstroke}{rgb}{0.000000,0.000000,0.000000}%
\pgfsetstrokecolor{currentstroke}%
\pgfsetdash{}{0pt}%
\pgfpathmoveto{\pgfqpoint{0.777645in}{1.205771in}}%
\pgfpathlineto{\pgfqpoint{0.774288in}{1.213275in}}%
\pgfpathlineto{\pgfqpoint{0.770919in}{1.221126in}}%
\pgfpathlineto{\pgfqpoint{0.767537in}{1.229328in}}%
\pgfpathlineto{\pgfqpoint{0.764142in}{1.237887in}}%
\pgfpathlineto{\pgfqpoint{0.771223in}{1.247194in}}%
\pgfpathlineto{\pgfqpoint{0.778852in}{1.256370in}}%
\pgfpathlineto{\pgfqpoint{0.787018in}{1.265406in}}%
\pgfpathlineto{\pgfqpoint{0.795711in}{1.274296in}}%
\pgfpathlineto{\pgfqpoint{0.798906in}{1.265577in}}%
\pgfpathlineto{\pgfqpoint{0.802088in}{1.257213in}}%
\pgfpathlineto{\pgfqpoint{0.805258in}{1.249199in}}%
\pgfpathlineto{\pgfqpoint{0.808416in}{1.241529in}}%
\pgfpathlineto{\pgfqpoint{0.799940in}{1.232797in}}%
\pgfpathlineto{\pgfqpoint{0.791979in}{1.223921in}}%
\pgfpathlineto{\pgfqpoint{0.784544in}{1.214910in}}%
\pgfpathlineto{\pgfqpoint{0.777645in}{1.205771in}}%
\pgfpathclose%
\pgfusepath{fill}%
\end{pgfscope}%
\begin{pgfscope}%
\pgfpathrectangle{\pgfqpoint{0.329460in}{0.284240in}}{\pgfqpoint{1.989680in}{1.989680in}}%
\pgfusepath{clip}%
\pgfsetbuttcap%
\pgfsetroundjoin%
\definecolor{currentfill}{rgb}{0.263663,0.237631,0.518762}%
\pgfsetfillcolor{currentfill}%
\pgfsetlinewidth{0.000000pt}%
\definecolor{currentstroke}{rgb}{0.000000,0.000000,0.000000}%
\pgfsetstrokecolor{currentstroke}%
\pgfsetdash{}{0pt}%
\pgfpathmoveto{\pgfqpoint{1.642994in}{1.295947in}}%
\pgfpathlineto{\pgfqpoint{1.645368in}{1.291354in}}%
\pgfpathlineto{\pgfqpoint{1.647741in}{1.286843in}}%
\pgfpathlineto{\pgfqpoint{1.650113in}{1.282418in}}%
\pgfpathlineto{\pgfqpoint{1.652483in}{1.278081in}}%
\pgfpathlineto{\pgfqpoint{1.660858in}{1.273234in}}%
\pgfpathlineto{\pgfqpoint{1.668943in}{1.268253in}}%
\pgfpathlineto{\pgfqpoint{1.676731in}{1.263142in}}%
\pgfpathlineto{\pgfqpoint{1.684212in}{1.257905in}}%
\pgfpathlineto{\pgfqpoint{1.681583in}{1.262415in}}%
\pgfpathlineto{\pgfqpoint{1.678953in}{1.267014in}}%
\pgfpathlineto{\pgfqpoint{1.676323in}{1.271698in}}%
\pgfpathlineto{\pgfqpoint{1.673691in}{1.276464in}}%
\pgfpathlineto{\pgfqpoint{1.666454in}{1.281521in}}%
\pgfpathlineto{\pgfqpoint{1.658920in}{1.286456in}}%
\pgfpathlineto{\pgfqpoint{1.651098in}{1.291266in}}%
\pgfpathlineto{\pgfqpoint{1.642994in}{1.295947in}}%
\pgfpathclose%
\pgfusepath{fill}%
\end{pgfscope}%
\begin{pgfscope}%
\pgfpathrectangle{\pgfqpoint{0.329460in}{0.284240in}}{\pgfqpoint{1.989680in}{1.989680in}}%
\pgfusepath{clip}%
\pgfsetbuttcap%
\pgfsetroundjoin%
\definecolor{currentfill}{rgb}{0.280255,0.165693,0.476498}%
\pgfsetfillcolor{currentfill}%
\pgfsetlinewidth{0.000000pt}%
\definecolor{currentstroke}{rgb}{0.000000,0.000000,0.000000}%
\pgfsetstrokecolor{currentstroke}%
\pgfsetdash{}{0pt}%
\pgfpathmoveto{\pgfqpoint{1.694721in}{1.240812in}}%
\pgfpathlineto{\pgfqpoint{1.697348in}{1.236792in}}%
\pgfpathlineto{\pgfqpoint{1.699974in}{1.232879in}}%
\pgfpathlineto{\pgfqpoint{1.702599in}{1.229077in}}%
\pgfpathlineto{\pgfqpoint{1.705225in}{1.225389in}}%
\pgfpathlineto{\pgfqpoint{1.712865in}{1.219663in}}%
\pgfpathlineto{\pgfqpoint{1.720165in}{1.213812in}}%
\pgfpathlineto{\pgfqpoint{1.727116in}{1.207842in}}%
\pgfpathlineto{\pgfqpoint{1.733712in}{1.201757in}}%
\pgfpathlineto{\pgfqpoint{1.730865in}{1.205634in}}%
\pgfpathlineto{\pgfqpoint{1.728017in}{1.209626in}}%
\pgfpathlineto{\pgfqpoint{1.725169in}{1.213729in}}%
\pgfpathlineto{\pgfqpoint{1.722322in}{1.217940in}}%
\pgfpathlineto{\pgfqpoint{1.715933in}{1.223829in}}%
\pgfpathlineto{\pgfqpoint{1.709198in}{1.229607in}}%
\pgfpathlineto{\pgfqpoint{1.702125in}{1.235270in}}%
\pgfpathlineto{\pgfqpoint{1.694721in}{1.240812in}}%
\pgfpathclose%
\pgfusepath{fill}%
\end{pgfscope}%
\begin{pgfscope}%
\pgfpathrectangle{\pgfqpoint{0.329460in}{0.284240in}}{\pgfqpoint{1.989680in}{1.989680in}}%
\pgfusepath{clip}%
\pgfsetbuttcap%
\pgfsetroundjoin%
\definecolor{currentfill}{rgb}{0.260571,0.246922,0.522828}%
\pgfsetfillcolor{currentfill}%
\pgfsetlinewidth{0.000000pt}%
\definecolor{currentstroke}{rgb}{0.000000,0.000000,0.000000}%
\pgfsetstrokecolor{currentstroke}%
\pgfsetdash{}{0pt}%
\pgfpathmoveto{\pgfqpoint{1.898502in}{1.282069in}}%
\pgfpathlineto{\pgfqpoint{1.901658in}{1.291184in}}%
\pgfpathlineto{\pgfqpoint{1.904827in}{1.300666in}}%
\pgfpathlineto{\pgfqpoint{1.908009in}{1.310521in}}%
\pgfpathlineto{\pgfqpoint{1.911205in}{1.320756in}}%
\pgfpathlineto{\pgfqpoint{1.920589in}{1.311855in}}%
\pgfpathlineto{\pgfqpoint{1.929444in}{1.302797in}}%
\pgfpathlineto{\pgfqpoint{1.937759in}{1.293591in}}%
\pgfpathlineto{\pgfqpoint{1.945524in}{1.284243in}}%
\pgfpathlineto{\pgfqpoint{1.942113in}{1.274158in}}%
\pgfpathlineto{\pgfqpoint{1.938717in}{1.264454in}}%
\pgfpathlineto{\pgfqpoint{1.935334in}{1.255125in}}%
\pgfpathlineto{\pgfqpoint{1.931966in}{1.246166in}}%
\pgfpathlineto{\pgfqpoint{1.924397in}{1.255357in}}%
\pgfpathlineto{\pgfqpoint{1.916291in}{1.264409in}}%
\pgfpathlineto{\pgfqpoint{1.907656in}{1.273316in}}%
\pgfpathlineto{\pgfqpoint{1.898502in}{1.282069in}}%
\pgfpathclose%
\pgfusepath{fill}%
\end{pgfscope}%
\begin{pgfscope}%
\pgfpathrectangle{\pgfqpoint{0.329460in}{0.284240in}}{\pgfqpoint{1.989680in}{1.989680in}}%
\pgfusepath{clip}%
\pgfsetbuttcap%
\pgfsetroundjoin%
\definecolor{currentfill}{rgb}{0.179019,0.433756,0.557430}%
\pgfsetfillcolor{currentfill}%
\pgfsetlinewidth{0.000000pt}%
\definecolor{currentstroke}{rgb}{0.000000,0.000000,0.000000}%
\pgfsetstrokecolor{currentstroke}%
\pgfsetdash{}{0pt}%
\pgfpathmoveto{\pgfqpoint{1.271519in}{1.449816in}}%
\pgfpathlineto{\pgfqpoint{1.270739in}{1.444721in}}%
\pgfpathlineto{\pgfqpoint{1.269959in}{1.439651in}}%
\pgfpathlineto{\pgfqpoint{1.269181in}{1.434609in}}%
\pgfpathlineto{\pgfqpoint{1.268403in}{1.429598in}}%
\pgfpathlineto{\pgfqpoint{1.278658in}{1.430799in}}%
\pgfpathlineto{\pgfqpoint{1.288980in}{1.431841in}}%
\pgfpathlineto{\pgfqpoint{1.299358in}{1.432723in}}%
\pgfpathlineto{\pgfqpoint{1.309784in}{1.433444in}}%
\pgfpathlineto{\pgfqpoint{1.310173in}{1.438417in}}%
\pgfpathlineto{\pgfqpoint{1.310562in}{1.443419in}}%
\pgfpathlineto{\pgfqpoint{1.310952in}{1.448450in}}%
\pgfpathlineto{\pgfqpoint{1.311342in}{1.453505in}}%
\pgfpathlineto{\pgfqpoint{1.301308in}{1.452814in}}%
\pgfpathlineto{\pgfqpoint{1.291320in}{1.451968in}}%
\pgfpathlineto{\pgfqpoint{1.281387in}{1.450968in}}%
\pgfpathlineto{\pgfqpoint{1.271519in}{1.449816in}}%
\pgfpathclose%
\pgfusepath{fill}%
\end{pgfscope}%
\begin{pgfscope}%
\pgfpathrectangle{\pgfqpoint{0.329460in}{0.284240in}}{\pgfqpoint{1.989680in}{1.989680in}}%
\pgfusepath{clip}%
\pgfsetbuttcap%
\pgfsetroundjoin%
\definecolor{currentfill}{rgb}{0.179019,0.433756,0.557430}%
\pgfsetfillcolor{currentfill}%
\pgfsetlinewidth{0.000000pt}%
\definecolor{currentstroke}{rgb}{0.000000,0.000000,0.000000}%
\pgfsetstrokecolor{currentstroke}%
\pgfsetdash{}{0pt}%
\pgfpathmoveto{\pgfqpoint{1.392150in}{1.453436in}}%
\pgfpathlineto{\pgfqpoint{1.392551in}{1.448380in}}%
\pgfpathlineto{\pgfqpoint{1.392952in}{1.443349in}}%
\pgfpathlineto{\pgfqpoint{1.393352in}{1.438345in}}%
\pgfpathlineto{\pgfqpoint{1.393752in}{1.433372in}}%
\pgfpathlineto{\pgfqpoint{1.404173in}{1.432633in}}%
\pgfpathlineto{\pgfqpoint{1.414545in}{1.431734in}}%
\pgfpathlineto{\pgfqpoint{1.424860in}{1.430674in}}%
\pgfpathlineto{\pgfqpoint{1.435107in}{1.429455in}}%
\pgfpathlineto{\pgfqpoint{1.434318in}{1.434467in}}%
\pgfpathlineto{\pgfqpoint{1.433529in}{1.439511in}}%
\pgfpathlineto{\pgfqpoint{1.432739in}{1.444582in}}%
\pgfpathlineto{\pgfqpoint{1.431948in}{1.449678in}}%
\pgfpathlineto{\pgfqpoint{1.422087in}{1.450848in}}%
\pgfpathlineto{\pgfqpoint{1.412162in}{1.451864in}}%
\pgfpathlineto{\pgfqpoint{1.402179in}{1.452727in}}%
\pgfpathlineto{\pgfqpoint{1.392150in}{1.453436in}}%
\pgfpathclose%
\pgfusepath{fill}%
\end{pgfscope}%
\begin{pgfscope}%
\pgfpathrectangle{\pgfqpoint{0.329460in}{0.284240in}}{\pgfqpoint{1.989680in}{1.989680in}}%
\pgfusepath{clip}%
\pgfsetbuttcap%
\pgfsetroundjoin%
\definecolor{currentfill}{rgb}{0.263663,0.237631,0.518762}%
\pgfsetfillcolor{currentfill}%
\pgfsetlinewidth{0.000000pt}%
\definecolor{currentstroke}{rgb}{0.000000,0.000000,0.000000}%
\pgfsetstrokecolor{currentstroke}%
\pgfsetdash{}{0pt}%
\pgfpathmoveto{\pgfqpoint{1.022506in}{1.271872in}}%
\pgfpathlineto{\pgfqpoint{1.019822in}{1.267064in}}%
\pgfpathlineto{\pgfqpoint{1.017139in}{1.262339in}}%
\pgfpathlineto{\pgfqpoint{1.014457in}{1.257700in}}%
\pgfpathlineto{\pgfqpoint{1.011776in}{1.253148in}}%
\pgfpathlineto{\pgfqpoint{1.018979in}{1.258493in}}%
\pgfpathlineto{\pgfqpoint{1.026495in}{1.263716in}}%
\pgfpathlineto{\pgfqpoint{1.034316in}{1.268813in}}%
\pgfpathlineto{\pgfqpoint{1.042434in}{1.273779in}}%
\pgfpathlineto{\pgfqpoint{1.044865in}{1.278153in}}%
\pgfpathlineto{\pgfqpoint{1.047297in}{1.282615in}}%
\pgfpathlineto{\pgfqpoint{1.049730in}{1.287163in}}%
\pgfpathlineto{\pgfqpoint{1.052164in}{1.291793in}}%
\pgfpathlineto{\pgfqpoint{1.044310in}{1.286997in}}%
\pgfpathlineto{\pgfqpoint{1.036744in}{1.282075in}}%
\pgfpathlineto{\pgfqpoint{1.029473in}{1.277032in}}%
\pgfpathlineto{\pgfqpoint{1.022506in}{1.271872in}}%
\pgfpathclose%
\pgfusepath{fill}%
\end{pgfscope}%
\begin{pgfscope}%
\pgfpathrectangle{\pgfqpoint{0.329460in}{0.284240in}}{\pgfqpoint{1.989680in}{1.989680in}}%
\pgfusepath{clip}%
\pgfsetbuttcap%
\pgfsetroundjoin%
\definecolor{currentfill}{rgb}{0.212395,0.359683,0.551710}%
\pgfsetfillcolor{currentfill}%
\pgfsetlinewidth{0.000000pt}%
\definecolor{currentstroke}{rgb}{0.000000,0.000000,0.000000}%
\pgfsetstrokecolor{currentstroke}%
\pgfsetdash{}{0pt}%
\pgfpathmoveto{\pgfqpoint{1.547828in}{1.385263in}}%
\pgfpathlineto{\pgfqpoint{1.549607in}{1.380202in}}%
\pgfpathlineto{\pgfqpoint{1.551384in}{1.375189in}}%
\pgfpathlineto{\pgfqpoint{1.553159in}{1.370224in}}%
\pgfpathlineto{\pgfqpoint{1.554934in}{1.365313in}}%
\pgfpathlineto{\pgfqpoint{1.564313in}{1.362053in}}%
\pgfpathlineto{\pgfqpoint{1.573495in}{1.358645in}}%
\pgfpathlineto{\pgfqpoint{1.582471in}{1.355094in}}%
\pgfpathlineto{\pgfqpoint{1.591232in}{1.351402in}}%
\pgfpathlineto{\pgfqpoint{1.589137in}{1.356446in}}%
\pgfpathlineto{\pgfqpoint{1.587041in}{1.361542in}}%
\pgfpathlineto{\pgfqpoint{1.584943in}{1.366689in}}%
\pgfpathlineto{\pgfqpoint{1.582843in}{1.371882in}}%
\pgfpathlineto{\pgfqpoint{1.574393in}{1.375433in}}%
\pgfpathlineto{\pgfqpoint{1.565735in}{1.378850in}}%
\pgfpathlineto{\pgfqpoint{1.556877in}{1.382127in}}%
\pgfpathlineto{\pgfqpoint{1.547828in}{1.385263in}}%
\pgfpathclose%
\pgfusepath{fill}%
\end{pgfscope}%
\begin{pgfscope}%
\pgfpathrectangle{\pgfqpoint{0.329460in}{0.284240in}}{\pgfqpoint{1.989680in}{1.989680in}}%
\pgfusepath{clip}%
\pgfsetbuttcap%
\pgfsetroundjoin%
\definecolor{currentfill}{rgb}{0.179019,0.433756,0.557430}%
\pgfsetfillcolor{currentfill}%
\pgfsetlinewidth{0.000000pt}%
\definecolor{currentstroke}{rgb}{0.000000,0.000000,0.000000}%
\pgfsetstrokecolor{currentstroke}%
\pgfsetdash{}{0pt}%
\pgfpathmoveto{\pgfqpoint{1.431948in}{1.449678in}}%
\pgfpathlineto{\pgfqpoint{1.432739in}{1.444582in}}%
\pgfpathlineto{\pgfqpoint{1.433529in}{1.439511in}}%
\pgfpathlineto{\pgfqpoint{1.434318in}{1.434467in}}%
\pgfpathlineto{\pgfqpoint{1.435107in}{1.429455in}}%
\pgfpathlineto{\pgfqpoint{1.445276in}{1.428077in}}%
\pgfpathlineto{\pgfqpoint{1.455360in}{1.426543in}}%
\pgfpathlineto{\pgfqpoint{1.465347in}{1.424852in}}%
\pgfpathlineto{\pgfqpoint{1.464274in}{1.429912in}}%
\pgfpathlineto{\pgfqpoint{1.463199in}{1.435003in}}%
\pgfpathlineto{\pgfqpoint{1.462124in}{1.440121in}}%
\pgfpathlineto{\pgfqpoint{1.461047in}{1.445264in}}%
\pgfpathlineto{\pgfqpoint{1.451437in}{1.446885in}}%
\pgfpathlineto{\pgfqpoint{1.441734in}{1.448357in}}%
\pgfpathlineto{\pgfqpoint{1.431948in}{1.449678in}}%
\pgfpathclose%
\pgfusepath{fill}%
\end{pgfscope}%
\begin{pgfscope}%
\pgfpathrectangle{\pgfqpoint{0.329460in}{0.284240in}}{\pgfqpoint{1.989680in}{1.989680in}}%
\pgfusepath{clip}%
\pgfsetbuttcap%
\pgfsetroundjoin%
\definecolor{currentfill}{rgb}{0.280255,0.165693,0.476498}%
\pgfsetfillcolor{currentfill}%
\pgfsetlinewidth{0.000000pt}%
\definecolor{currentstroke}{rgb}{0.000000,0.000000,0.000000}%
\pgfsetstrokecolor{currentstroke}%
\pgfsetdash{}{0pt}%
\pgfpathmoveto{\pgfqpoint{0.974670in}{1.212617in}}%
\pgfpathlineto{\pgfqpoint{0.971779in}{1.208362in}}%
\pgfpathlineto{\pgfqpoint{0.968887in}{1.204215in}}%
\pgfpathlineto{\pgfqpoint{0.965996in}{1.200178in}}%
\pgfpathlineto{\pgfqpoint{0.963104in}{1.196256in}}%
\pgfpathlineto{\pgfqpoint{0.969378in}{1.202438in}}%
\pgfpathlineto{\pgfqpoint{0.976014in}{1.208511in}}%
\pgfpathlineto{\pgfqpoint{0.983004in}{1.214468in}}%
\pgfpathlineto{\pgfqpoint{0.990342in}{1.220305in}}%
\pgfpathlineto{\pgfqpoint{0.993021in}{1.224034in}}%
\pgfpathlineto{\pgfqpoint{0.995699in}{1.227877in}}%
\pgfpathlineto{\pgfqpoint{0.998378in}{1.231830in}}%
\pgfpathlineto{\pgfqpoint{1.001056in}{1.235892in}}%
\pgfpathlineto{\pgfqpoint{0.993946in}{1.230242in}}%
\pgfpathlineto{\pgfqpoint{0.987174in}{1.224476in}}%
\pgfpathlineto{\pgfqpoint{0.980746in}{1.218600in}}%
\pgfpathlineto{\pgfqpoint{0.974670in}{1.212617in}}%
\pgfpathclose%
\pgfusepath{fill}%
\end{pgfscope}%
\begin{pgfscope}%
\pgfpathrectangle{\pgfqpoint{0.329460in}{0.284240in}}{\pgfqpoint{1.989680in}{1.989680in}}%
\pgfusepath{clip}%
\pgfsetbuttcap%
\pgfsetroundjoin%
\definecolor{currentfill}{rgb}{0.179019,0.433756,0.557430}%
\pgfsetfillcolor{currentfill}%
\pgfsetlinewidth{0.000000pt}%
\definecolor{currentstroke}{rgb}{0.000000,0.000000,0.000000}%
\pgfsetstrokecolor{currentstroke}%
\pgfsetdash{}{0pt}%
\pgfpathmoveto{\pgfqpoint{1.232871in}{1.443699in}}%
\pgfpathlineto{\pgfqpoint{1.231711in}{1.438539in}}%
\pgfpathlineto{\pgfqpoint{1.230552in}{1.433404in}}%
\pgfpathlineto{\pgfqpoint{1.229395in}{1.428297in}}%
\pgfpathlineto{\pgfqpoint{1.228239in}{1.423220in}}%
\pgfpathlineto{\pgfqpoint{1.238133in}{1.425048in}}%
\pgfpathlineto{\pgfqpoint{1.248131in}{1.426721in}}%
\pgfpathlineto{\pgfqpoint{1.258225in}{1.428238in}}%
\pgfpathlineto{\pgfqpoint{1.268403in}{1.429598in}}%
\pgfpathlineto{\pgfqpoint{1.269181in}{1.434609in}}%
\pgfpathlineto{\pgfqpoint{1.269959in}{1.439651in}}%
\pgfpathlineto{\pgfqpoint{1.270739in}{1.444721in}}%
\pgfpathlineto{\pgfqpoint{1.271519in}{1.449816in}}%
\pgfpathlineto{\pgfqpoint{1.261724in}{1.448511in}}%
\pgfpathlineto{\pgfqpoint{1.252012in}{1.447056in}}%
\pgfpathlineto{\pgfqpoint{1.242391in}{1.445452in}}%
\pgfpathlineto{\pgfqpoint{1.232871in}{1.443699in}}%
\pgfpathclose%
\pgfusepath{fill}%
\end{pgfscope}%
\begin{pgfscope}%
\pgfpathrectangle{\pgfqpoint{0.329460in}{0.284240in}}{\pgfqpoint{1.989680in}{1.989680in}}%
\pgfusepath{clip}%
\pgfsetbuttcap%
\pgfsetroundjoin%
\definecolor{currentfill}{rgb}{0.279566,0.067836,0.391917}%
\pgfsetfillcolor{currentfill}%
\pgfsetlinewidth{0.000000pt}%
\definecolor{currentstroke}{rgb}{0.000000,0.000000,0.000000}%
\pgfsetstrokecolor{currentstroke}%
\pgfsetdash{}{0pt}%
\pgfpathmoveto{\pgfqpoint{0.917234in}{1.142033in}}%
\pgfpathlineto{\pgfqpoint{0.914161in}{1.139086in}}%
\pgfpathlineto{\pgfqpoint{0.911087in}{1.136289in}}%
\pgfpathlineto{\pgfqpoint{0.908010in}{1.133644in}}%
\pgfpathlineto{\pgfqpoint{0.904931in}{1.131156in}}%
\pgfpathlineto{\pgfqpoint{0.910158in}{1.138353in}}%
\pgfpathlineto{\pgfqpoint{0.915806in}{1.145454in}}%
\pgfpathlineto{\pgfqpoint{0.921870in}{1.152453in}}%
\pgfpathlineto{\pgfqpoint{0.928342in}{1.159342in}}%
\pgfpathlineto{\pgfqpoint{0.931246in}{1.161626in}}%
\pgfpathlineto{\pgfqpoint{0.934149in}{1.164066in}}%
\pgfpathlineto{\pgfqpoint{0.937050in}{1.166658in}}%
\pgfpathlineto{\pgfqpoint{0.939949in}{1.169399in}}%
\pgfpathlineto{\pgfqpoint{0.933667in}{1.162709in}}%
\pgfpathlineto{\pgfqpoint{0.927783in}{1.155914in}}%
\pgfpathlineto{\pgfqpoint{0.922303in}{1.149020in}}%
\pgfpathlineto{\pgfqpoint{0.917234in}{1.142033in}}%
\pgfpathclose%
\pgfusepath{fill}%
\end{pgfscope}%
\begin{pgfscope}%
\pgfpathrectangle{\pgfqpoint{0.329460in}{0.284240in}}{\pgfqpoint{1.989680in}{1.989680in}}%
\pgfusepath{clip}%
\pgfsetbuttcap%
\pgfsetroundjoin%
\definecolor{currentfill}{rgb}{0.195860,0.395433,0.555276}%
\pgfsetfillcolor{currentfill}%
\pgfsetlinewidth{0.000000pt}%
\definecolor{currentstroke}{rgb}{0.000000,0.000000,0.000000}%
\pgfsetstrokecolor{currentstroke}%
\pgfsetdash{}{0pt}%
\pgfpathmoveto{\pgfqpoint{1.504153in}{1.416564in}}%
\pgfpathlineto{\pgfqpoint{1.505591in}{1.411452in}}%
\pgfpathlineto{\pgfqpoint{1.507029in}{1.406376in}}%
\pgfpathlineto{\pgfqpoint{1.508464in}{1.401339in}}%
\pgfpathlineto{\pgfqpoint{1.509899in}{1.396344in}}%
\pgfpathlineto{\pgfqpoint{1.519623in}{1.393798in}}%
\pgfpathlineto{\pgfqpoint{1.529192in}{1.391101in}}%
\pgfpathlineto{\pgfqpoint{1.538597in}{1.388256in}}%
\pgfpathlineto{\pgfqpoint{1.547828in}{1.385263in}}%
\pgfpathlineto{\pgfqpoint{1.546048in}{1.390368in}}%
\pgfpathlineto{\pgfqpoint{1.544266in}{1.395515in}}%
\pgfpathlineto{\pgfqpoint{1.542482in}{1.400700in}}%
\pgfpathlineto{\pgfqpoint{1.540697in}{1.405921in}}%
\pgfpathlineto{\pgfqpoint{1.531804in}{1.408795in}}%
\pgfpathlineto{\pgfqpoint{1.522743in}{1.411528in}}%
\pgfpathlineto{\pgfqpoint{1.513523in}{1.414119in}}%
\pgfpathlineto{\pgfqpoint{1.504153in}{1.416564in}}%
\pgfpathclose%
\pgfusepath{fill}%
\end{pgfscope}%
\begin{pgfscope}%
\pgfpathrectangle{\pgfqpoint{0.329460in}{0.284240in}}{\pgfqpoint{1.989680in}{1.989680in}}%
\pgfusepath{clip}%
\pgfsetbuttcap%
\pgfsetroundjoin%
\definecolor{currentfill}{rgb}{0.231674,0.318106,0.544834}%
\pgfsetfillcolor{currentfill}%
\pgfsetlinewidth{0.000000pt}%
\definecolor{currentstroke}{rgb}{0.000000,0.000000,0.000000}%
\pgfsetstrokecolor{currentstroke}%
\pgfsetdash{}{0pt}%
\pgfpathmoveto{\pgfqpoint{1.591232in}{1.351402in}}%
\pgfpathlineto{\pgfqpoint{1.593325in}{1.346413in}}%
\pgfpathlineto{\pgfqpoint{1.595417in}{1.341483in}}%
\pgfpathlineto{\pgfqpoint{1.597507in}{1.336614in}}%
\pgfpathlineto{\pgfqpoint{1.599596in}{1.331810in}}%
\pgfpathlineto{\pgfqpoint{1.608437in}{1.327833in}}%
\pgfpathlineto{\pgfqpoint{1.617038in}{1.323717in}}%
\pgfpathlineto{\pgfqpoint{1.625391in}{1.319465in}}%
\pgfpathlineto{\pgfqpoint{1.633487in}{1.315079in}}%
\pgfpathlineto{\pgfqpoint{1.631107in}{1.320038in}}%
\pgfpathlineto{\pgfqpoint{1.628726in}{1.325061in}}%
\pgfpathlineto{\pgfqpoint{1.626342in}{1.330145in}}%
\pgfpathlineto{\pgfqpoint{1.623957in}{1.335288in}}%
\pgfpathlineto{\pgfqpoint{1.616140in}{1.339511in}}%
\pgfpathlineto{\pgfqpoint{1.608075in}{1.343607in}}%
\pgfpathlineto{\pgfqpoint{1.599770in}{1.347571in}}%
\pgfpathlineto{\pgfqpoint{1.591232in}{1.351402in}}%
\pgfpathclose%
\pgfusepath{fill}%
\end{pgfscope}%
\begin{pgfscope}%
\pgfpathrectangle{\pgfqpoint{0.329460in}{0.284240in}}{\pgfqpoint{1.989680in}{1.989680in}}%
\pgfusepath{clip}%
\pgfsetbuttcap%
\pgfsetroundjoin%
\definecolor{currentfill}{rgb}{0.172719,0.448791,0.557885}%
\pgfsetfillcolor{currentfill}%
\pgfsetlinewidth{0.000000pt}%
\definecolor{currentstroke}{rgb}{0.000000,0.000000,0.000000}%
\pgfsetstrokecolor{currentstroke}%
\pgfsetdash{}{0pt}%
\pgfpathmoveto{\pgfqpoint{0.756281in}{1.408920in}}%
\pgfpathlineto{\pgfqpoint{0.752892in}{1.422835in}}%
\pgfpathlineto{\pgfqpoint{0.749485in}{1.437197in}}%
\pgfpathlineto{\pgfqpoint{0.746059in}{1.452014in}}%
\pgfpathlineto{\pgfqpoint{0.756167in}{1.461258in}}%
\pgfpathlineto{\pgfqpoint{0.766822in}{1.470329in}}%
\pgfpathlineto{\pgfqpoint{0.778013in}{1.479218in}}%
\pgfpathlineto{\pgfqpoint{0.789726in}{1.487919in}}%
\pgfpathlineto{\pgfqpoint{0.792885in}{1.472987in}}%
\pgfpathlineto{\pgfqpoint{0.796028in}{1.458507in}}%
\pgfpathlineto{\pgfqpoint{0.799154in}{1.444472in}}%
\pgfpathlineto{\pgfqpoint{0.787652in}{1.435856in}}%
\pgfpathlineto{\pgfqpoint{0.776664in}{1.427054in}}%
\pgfpathlineto{\pgfqpoint{0.766203in}{1.418073in}}%
\pgfpathlineto{\pgfqpoint{0.756281in}{1.408920in}}%
\pgfpathclose%
\pgfusepath{fill}%
\end{pgfscope}%
\begin{pgfscope}%
\pgfpathrectangle{\pgfqpoint{0.329460in}{0.284240in}}{\pgfqpoint{1.989680in}{1.989680in}}%
\pgfusepath{clip}%
\pgfsetbuttcap%
\pgfsetroundjoin%
\definecolor{currentfill}{rgb}{0.282327,0.094955,0.417331}%
\pgfsetfillcolor{currentfill}%
\pgfsetlinewidth{0.000000pt}%
\definecolor{currentstroke}{rgb}{0.000000,0.000000,0.000000}%
\pgfsetstrokecolor{currentstroke}%
\pgfsetdash{}{0pt}%
\pgfpathmoveto{\pgfqpoint{1.745107in}{1.187455in}}%
\pgfpathlineto{\pgfqpoint{1.747958in}{1.184199in}}%
\pgfpathlineto{\pgfqpoint{1.750809in}{1.181078in}}%
\pgfpathlineto{\pgfqpoint{1.753661in}{1.178094in}}%
\pgfpathlineto{\pgfqpoint{1.756515in}{1.175252in}}%
\pgfpathlineto{\pgfqpoint{1.763144in}{1.168661in}}%
\pgfpathlineto{\pgfqpoint{1.769381in}{1.161959in}}%
\pgfpathlineto{\pgfqpoint{1.775221in}{1.155153in}}%
\pgfpathlineto{\pgfqpoint{1.780655in}{1.148248in}}%
\pgfpathlineto{\pgfqpoint{1.777619in}{1.151293in}}%
\pgfpathlineto{\pgfqpoint{1.774584in}{1.154480in}}%
\pgfpathlineto{\pgfqpoint{1.771551in}{1.157805in}}%
\pgfpathlineto{\pgfqpoint{1.768518in}{1.161265in}}%
\pgfpathlineto{\pgfqpoint{1.763250in}{1.167961in}}%
\pgfpathlineto{\pgfqpoint{1.757587in}{1.174562in}}%
\pgfpathlineto{\pgfqpoint{1.751537in}{1.181062in}}%
\pgfpathlineto{\pgfqpoint{1.745107in}{1.187455in}}%
\pgfpathclose%
\pgfusepath{fill}%
\end{pgfscope}%
\begin{pgfscope}%
\pgfpathrectangle{\pgfqpoint{0.329460in}{0.284240in}}{\pgfqpoint{1.989680in}{1.989680in}}%
\pgfusepath{clip}%
\pgfsetbuttcap%
\pgfsetroundjoin%
\definecolor{currentfill}{rgb}{0.212395,0.359683,0.551710}%
\pgfsetfillcolor{currentfill}%
\pgfsetlinewidth{0.000000pt}%
\definecolor{currentstroke}{rgb}{0.000000,0.000000,0.000000}%
\pgfsetstrokecolor{currentstroke}%
\pgfsetdash{}{0pt}%
\pgfpathmoveto{\pgfqpoint{1.112203in}{1.368614in}}%
\pgfpathlineto{\pgfqpoint{1.110035in}{1.363388in}}%
\pgfpathlineto{\pgfqpoint{1.107869in}{1.358209in}}%
\pgfpathlineto{\pgfqpoint{1.105705in}{1.353080in}}%
\pgfpathlineto{\pgfqpoint{1.103543in}{1.348004in}}%
\pgfpathlineto{\pgfqpoint{1.112105in}{1.351819in}}%
\pgfpathlineto{\pgfqpoint{1.120891in}{1.355496in}}%
\pgfpathlineto{\pgfqpoint{1.129890in}{1.359031in}}%
\pgfpathlineto{\pgfqpoint{1.139095in}{1.362422in}}%
\pgfpathlineto{\pgfqpoint{1.140942in}{1.367361in}}%
\pgfpathlineto{\pgfqpoint{1.142791in}{1.372353in}}%
\pgfpathlineto{\pgfqpoint{1.144642in}{1.377394in}}%
\pgfpathlineto{\pgfqpoint{1.146494in}{1.382483in}}%
\pgfpathlineto{\pgfqpoint{1.137615in}{1.379221in}}%
\pgfpathlineto{\pgfqpoint{1.128934in}{1.375820in}}%
\pgfpathlineto{\pgfqpoint{1.120461in}{1.372283in}}%
\pgfpathlineto{\pgfqpoint{1.112203in}{1.368614in}}%
\pgfpathclose%
\pgfusepath{fill}%
\end{pgfscope}%
\begin{pgfscope}%
\pgfpathrectangle{\pgfqpoint{0.329460in}{0.284240in}}{\pgfqpoint{1.989680in}{1.989680in}}%
\pgfusepath{clip}%
\pgfsetbuttcap%
\pgfsetroundjoin%
\definecolor{currentfill}{rgb}{0.268510,0.009605,0.335427}%
\pgfsetfillcolor{currentfill}%
\pgfsetlinewidth{0.000000pt}%
\definecolor{currentstroke}{rgb}{0.000000,0.000000,0.000000}%
\pgfsetstrokecolor{currentstroke}%
\pgfsetdash{}{0pt}%
\pgfpathmoveto{\pgfqpoint{1.854423in}{1.127380in}}%
\pgfpathlineto{\pgfqpoint{1.857555in}{1.129113in}}%
\pgfpathlineto{\pgfqpoint{1.860694in}{1.131090in}}%
\pgfpathlineto{\pgfqpoint{1.863840in}{1.133317in}}%
\pgfpathlineto{\pgfqpoint{1.866992in}{1.135797in}}%
\pgfpathlineto{\pgfqpoint{1.873141in}{1.127375in}}%
\pgfpathlineto{\pgfqpoint{1.878791in}{1.118846in}}%
\pgfpathlineto{\pgfqpoint{1.883937in}{1.110217in}}%
\pgfpathlineto{\pgfqpoint{1.888570in}{1.101497in}}%
\pgfpathlineto{\pgfqpoint{1.885268in}{1.099214in}}%
\pgfpathlineto{\pgfqpoint{1.881973in}{1.097186in}}%
\pgfpathlineto{\pgfqpoint{1.878686in}{1.095408in}}%
\pgfpathlineto{\pgfqpoint{1.875406in}{1.093876in}}%
\pgfpathlineto{\pgfqpoint{1.870904in}{1.102392in}}%
\pgfpathlineto{\pgfqpoint{1.865901in}{1.110820in}}%
\pgfpathlineto{\pgfqpoint{1.860405in}{1.119152in}}%
\pgfpathlineto{\pgfqpoint{1.854423in}{1.127380in}}%
\pgfpathclose%
\pgfusepath{fill}%
\end{pgfscope}%
\begin{pgfscope}%
\pgfpathrectangle{\pgfqpoint{0.329460in}{0.284240in}}{\pgfqpoint{1.989680in}{1.989680in}}%
\pgfusepath{clip}%
\pgfsetbuttcap%
\pgfsetroundjoin%
\definecolor{currentfill}{rgb}{0.267004,0.004874,0.329415}%
\pgfsetfillcolor{currentfill}%
\pgfsetlinewidth{0.000000pt}%
\definecolor{currentstroke}{rgb}{0.000000,0.000000,0.000000}%
\pgfsetstrokecolor{currentstroke}%
\pgfsetdash{}{0pt}%
\pgfpathmoveto{\pgfqpoint{1.841956in}{1.122805in}}%
\pgfpathlineto{\pgfqpoint{1.845064in}{1.123604in}}%
\pgfpathlineto{\pgfqpoint{1.848178in}{1.124630in}}%
\pgfpathlineto{\pgfqpoint{1.851297in}{1.125887in}}%
\pgfpathlineto{\pgfqpoint{1.854423in}{1.127380in}}%
\pgfpathlineto{\pgfqpoint{1.860405in}{1.119152in}}%
\pgfpathlineto{\pgfqpoint{1.865901in}{1.110820in}}%
\pgfpathlineto{\pgfqpoint{1.870904in}{1.102392in}}%
\pgfpathlineto{\pgfqpoint{1.875406in}{1.093876in}}%
\pgfpathlineto{\pgfqpoint{1.872133in}{1.092584in}}%
\pgfpathlineto{\pgfqpoint{1.868866in}{1.091529in}}%
\pgfpathlineto{\pgfqpoint{1.865606in}{1.090707in}}%
\pgfpathlineto{\pgfqpoint{1.862352in}{1.090112in}}%
\pgfpathlineto{\pgfqpoint{1.857979in}{1.098422in}}%
\pgfpathlineto{\pgfqpoint{1.853117in}{1.106646in}}%
\pgfpathlineto{\pgfqpoint{1.847774in}{1.114776in}}%
\pgfpathlineto{\pgfqpoint{1.841956in}{1.122805in}}%
\pgfpathclose%
\pgfusepath{fill}%
\end{pgfscope}%
\begin{pgfscope}%
\pgfpathrectangle{\pgfqpoint{0.329460in}{0.284240in}}{\pgfqpoint{1.989680in}{1.989680in}}%
\pgfusepath{clip}%
\pgfsetbuttcap%
\pgfsetroundjoin%
\definecolor{currentfill}{rgb}{0.260571,0.246922,0.522828}%
\pgfsetfillcolor{currentfill}%
\pgfsetlinewidth{0.000000pt}%
\definecolor{currentstroke}{rgb}{0.000000,0.000000,0.000000}%
\pgfsetstrokecolor{currentstroke}%
\pgfsetdash{}{0pt}%
\pgfpathmoveto{\pgfqpoint{0.764142in}{1.237887in}}%
\pgfpathlineto{\pgfqpoint{0.760732in}{1.246810in}}%
\pgfpathlineto{\pgfqpoint{0.757309in}{1.256103in}}%
\pgfpathlineto{\pgfqpoint{0.753871in}{1.265772in}}%
\pgfpathlineto{\pgfqpoint{0.750418in}{1.275822in}}%
\pgfpathlineto{\pgfqpoint{0.757686in}{1.285288in}}%
\pgfpathlineto{\pgfqpoint{0.765513in}{1.294621in}}%
\pgfpathlineto{\pgfqpoint{0.773888in}{1.303811in}}%
\pgfpathlineto{\pgfqpoint{0.782803in}{1.312852in}}%
\pgfpathlineto{\pgfqpoint{0.786050in}{1.302649in}}%
\pgfpathlineto{\pgfqpoint{0.789284in}{1.292827in}}%
\pgfpathlineto{\pgfqpoint{0.792504in}{1.283378in}}%
\pgfpathlineto{\pgfqpoint{0.795711in}{1.274296in}}%
\pgfpathlineto{\pgfqpoint{0.787018in}{1.265406in}}%
\pgfpathlineto{\pgfqpoint{0.778852in}{1.256370in}}%
\pgfpathlineto{\pgfqpoint{0.771223in}{1.247194in}}%
\pgfpathlineto{\pgfqpoint{0.764142in}{1.237887in}}%
\pgfpathclose%
\pgfusepath{fill}%
\end{pgfscope}%
\begin{pgfscope}%
\pgfpathrectangle{\pgfqpoint{0.329460in}{0.284240in}}{\pgfqpoint{1.989680in}{1.989680in}}%
\pgfusepath{clip}%
\pgfsetbuttcap%
\pgfsetroundjoin%
\definecolor{currentfill}{rgb}{0.195860,0.395433,0.555276}%
\pgfsetfillcolor{currentfill}%
\pgfsetlinewidth{0.000000pt}%
\definecolor{currentstroke}{rgb}{0.000000,0.000000,0.000000}%
\pgfsetstrokecolor{currentstroke}%
\pgfsetdash{}{0pt}%
\pgfpathmoveto{\pgfqpoint{1.153921in}{1.403250in}}%
\pgfpathlineto{\pgfqpoint{1.152062in}{1.398002in}}%
\pgfpathlineto{\pgfqpoint{1.150204in}{1.392789in}}%
\pgfpathlineto{\pgfqpoint{1.148348in}{1.387615in}}%
\pgfpathlineto{\pgfqpoint{1.146494in}{1.382483in}}%
\pgfpathlineto{\pgfqpoint{1.155564in}{1.385603in}}%
\pgfpathlineto{\pgfqpoint{1.164815in}{1.388579in}}%
\pgfpathlineto{\pgfqpoint{1.174238in}{1.391408in}}%
\pgfpathlineto{\pgfqpoint{1.183825in}{1.394088in}}%
\pgfpathlineto{\pgfqpoint{1.185338in}{1.399106in}}%
\pgfpathlineto{\pgfqpoint{1.186852in}{1.404166in}}%
\pgfpathlineto{\pgfqpoint{1.188368in}{1.409264in}}%
\pgfpathlineto{\pgfqpoint{1.189886in}{1.414398in}}%
\pgfpathlineto{\pgfqpoint{1.180649in}{1.411823in}}%
\pgfpathlineto{\pgfqpoint{1.171570in}{1.409106in}}%
\pgfpathlineto{\pgfqpoint{1.162658in}{1.406247in}}%
\pgfpathlineto{\pgfqpoint{1.153921in}{1.403250in}}%
\pgfpathclose%
\pgfusepath{fill}%
\end{pgfscope}%
\begin{pgfscope}%
\pgfpathrectangle{\pgfqpoint{0.329460in}{0.284240in}}{\pgfqpoint{1.989680in}{1.989680in}}%
\pgfusepath{clip}%
\pgfsetbuttcap%
\pgfsetroundjoin%
\definecolor{currentfill}{rgb}{0.272594,0.025563,0.353093}%
\pgfsetfillcolor{currentfill}%
\pgfsetlinewidth{0.000000pt}%
\definecolor{currentstroke}{rgb}{0.000000,0.000000,0.000000}%
\pgfsetstrokecolor{currentstroke}%
\pgfsetdash{}{0pt}%
\pgfpathmoveto{\pgfqpoint{1.866992in}{1.135797in}}%
\pgfpathlineto{\pgfqpoint{1.870152in}{1.138536in}}%
\pgfpathlineto{\pgfqpoint{1.873320in}{1.141537in}}%
\pgfpathlineto{\pgfqpoint{1.876496in}{1.144807in}}%
\pgfpathlineto{\pgfqpoint{1.879679in}{1.148348in}}%
\pgfpathlineto{\pgfqpoint{1.885996in}{1.139735in}}%
\pgfpathlineto{\pgfqpoint{1.891804in}{1.131013in}}%
\pgfpathlineto{\pgfqpoint{1.897094in}{1.122188in}}%
\pgfpathlineto{\pgfqpoint{1.901860in}{1.113269in}}%
\pgfpathlineto{\pgfqpoint{1.898525in}{1.109921in}}%
\pgfpathlineto{\pgfqpoint{1.895198in}{1.106846in}}%
\pgfpathlineto{\pgfqpoint{1.891880in}{1.104040in}}%
\pgfpathlineto{\pgfqpoint{1.888570in}{1.101497in}}%
\pgfpathlineto{\pgfqpoint{1.883937in}{1.110217in}}%
\pgfpathlineto{\pgfqpoint{1.878791in}{1.118846in}}%
\pgfpathlineto{\pgfqpoint{1.873141in}{1.127375in}}%
\pgfpathlineto{\pgfqpoint{1.866992in}{1.135797in}}%
\pgfpathclose%
\pgfusepath{fill}%
\end{pgfscope}%
\begin{pgfscope}%
\pgfpathrectangle{\pgfqpoint{0.329460in}{0.284240in}}{\pgfqpoint{1.989680in}{1.989680in}}%
\pgfusepath{clip}%
\pgfsetbuttcap%
\pgfsetroundjoin%
\definecolor{currentfill}{rgb}{0.267004,0.004874,0.329415}%
\pgfsetfillcolor{currentfill}%
\pgfsetlinewidth{0.000000pt}%
\definecolor{currentstroke}{rgb}{0.000000,0.000000,0.000000}%
\pgfsetstrokecolor{currentstroke}%
\pgfsetdash{}{0pt}%
\pgfpathmoveto{\pgfqpoint{1.829575in}{1.121798in}}%
\pgfpathlineto{\pgfqpoint{1.832663in}{1.121730in}}%
\pgfpathlineto{\pgfqpoint{1.835755in}{1.121873in}}%
\pgfpathlineto{\pgfqpoint{1.838853in}{1.122230in}}%
\pgfpathlineto{\pgfqpoint{1.841956in}{1.122805in}}%
\pgfpathlineto{\pgfqpoint{1.847774in}{1.114776in}}%
\pgfpathlineto{\pgfqpoint{1.853117in}{1.106646in}}%
\pgfpathlineto{\pgfqpoint{1.857979in}{1.098422in}}%
\pgfpathlineto{\pgfqpoint{1.862352in}{1.090112in}}%
\pgfpathlineto{\pgfqpoint{1.859104in}{1.089741in}}%
\pgfpathlineto{\pgfqpoint{1.855862in}{1.089590in}}%
\pgfpathlineto{\pgfqpoint{1.852625in}{1.089654in}}%
\pgfpathlineto{\pgfqpoint{1.849393in}{1.089928in}}%
\pgfpathlineto{\pgfqpoint{1.845146in}{1.098028in}}%
\pgfpathlineto{\pgfqpoint{1.840423in}{1.106044in}}%
\pgfpathlineto{\pgfqpoint{1.835231in}{1.113970in}}%
\pgfpathlineto{\pgfqpoint{1.829575in}{1.121798in}}%
\pgfpathclose%
\pgfusepath{fill}%
\end{pgfscope}%
\begin{pgfscope}%
\pgfpathrectangle{\pgfqpoint{0.329460in}{0.284240in}}{\pgfqpoint{1.989680in}{1.989680in}}%
\pgfusepath{clip}%
\pgfsetbuttcap%
\pgfsetroundjoin%
\definecolor{currentfill}{rgb}{0.231674,0.318106,0.544834}%
\pgfsetfillcolor{currentfill}%
\pgfsetlinewidth{0.000000pt}%
\definecolor{currentstroke}{rgb}{0.000000,0.000000,0.000000}%
\pgfsetstrokecolor{currentstroke}%
\pgfsetdash{}{0pt}%
\pgfpathmoveto{\pgfqpoint{1.071684in}{1.331429in}}%
\pgfpathlineto{\pgfqpoint{1.069239in}{1.326250in}}%
\pgfpathlineto{\pgfqpoint{1.066795in}{1.321129in}}%
\pgfpathlineto{\pgfqpoint{1.064353in}{1.316069in}}%
\pgfpathlineto{\pgfqpoint{1.061912in}{1.311073in}}%
\pgfpathlineto{\pgfqpoint{1.069774in}{1.315573in}}%
\pgfpathlineto{\pgfqpoint{1.077900in}{1.319944in}}%
\pgfpathlineto{\pgfqpoint{1.086281in}{1.324181in}}%
\pgfpathlineto{\pgfqpoint{1.094909in}{1.328282in}}%
\pgfpathlineto{\pgfqpoint{1.097065in}{1.333119in}}%
\pgfpathlineto{\pgfqpoint{1.099223in}{1.338020in}}%
\pgfpathlineto{\pgfqpoint{1.101382in}{1.342983in}}%
\pgfpathlineto{\pgfqpoint{1.103543in}{1.348004in}}%
\pgfpathlineto{\pgfqpoint{1.095211in}{1.344054in}}%
\pgfpathlineto{\pgfqpoint{1.087119in}{1.339972in}}%
\pgfpathlineto{\pgfqpoint{1.079274in}{1.335763in}}%
\pgfpathlineto{\pgfqpoint{1.071684in}{1.331429in}}%
\pgfpathclose%
\pgfusepath{fill}%
\end{pgfscope}%
\begin{pgfscope}%
\pgfpathrectangle{\pgfqpoint{0.329460in}{0.284240in}}{\pgfqpoint{1.989680in}{1.989680in}}%
\pgfusepath{clip}%
\pgfsetbuttcap%
\pgfsetroundjoin%
\definecolor{currentfill}{rgb}{0.274128,0.199721,0.498911}%
\pgfsetfillcolor{currentfill}%
\pgfsetlinewidth{0.000000pt}%
\definecolor{currentstroke}{rgb}{0.000000,0.000000,0.000000}%
\pgfsetstrokecolor{currentstroke}%
\pgfsetdash{}{0pt}%
\pgfpathmoveto{\pgfqpoint{1.684212in}{1.257905in}}%
\pgfpathlineto{\pgfqpoint{1.686840in}{1.253487in}}%
\pgfpathlineto{\pgfqpoint{1.689468in}{1.249163in}}%
\pgfpathlineto{\pgfqpoint{1.692095in}{1.244937in}}%
\pgfpathlineto{\pgfqpoint{1.694721in}{1.240812in}}%
\pgfpathlineto{\pgfqpoint{1.702125in}{1.235270in}}%
\pgfpathlineto{\pgfqpoint{1.709198in}{1.229607in}}%
\pgfpathlineto{\pgfqpoint{1.715933in}{1.223829in}}%
\pgfpathlineto{\pgfqpoint{1.722322in}{1.217940in}}%
\pgfpathlineto{\pgfqpoint{1.719474in}{1.222255in}}%
\pgfpathlineto{\pgfqpoint{1.716626in}{1.226672in}}%
\pgfpathlineto{\pgfqpoint{1.713777in}{1.231186in}}%
\pgfpathlineto{\pgfqpoint{1.710928in}{1.235796in}}%
\pgfpathlineto{\pgfqpoint{1.704745in}{1.241488in}}%
\pgfpathlineto{\pgfqpoint{1.698227in}{1.247073in}}%
\pgfpathlineto{\pgfqpoint{1.691380in}{1.252547in}}%
\pgfpathlineto{\pgfqpoint{1.684212in}{1.257905in}}%
\pgfpathclose%
\pgfusepath{fill}%
\end{pgfscope}%
\begin{pgfscope}%
\pgfpathrectangle{\pgfqpoint{0.329460in}{0.284240in}}{\pgfqpoint{1.989680in}{1.989680in}}%
\pgfusepath{clip}%
\pgfsetbuttcap%
\pgfsetroundjoin%
\definecolor{currentfill}{rgb}{0.233603,0.313828,0.543914}%
\pgfsetfillcolor{currentfill}%
\pgfsetlinewidth{0.000000pt}%
\definecolor{currentstroke}{rgb}{0.000000,0.000000,0.000000}%
\pgfsetstrokecolor{currentstroke}%
\pgfsetdash{}{0pt}%
\pgfpathmoveto{\pgfqpoint{1.911205in}{1.320756in}}%
\pgfpathlineto{\pgfqpoint{1.914414in}{1.331376in}}%
\pgfpathlineto{\pgfqpoint{1.917638in}{1.342389in}}%
\pgfpathlineto{\pgfqpoint{1.920876in}{1.353800in}}%
\pgfpathlineto{\pgfqpoint{1.924130in}{1.365615in}}%
\pgfpathlineto{\pgfqpoint{1.933750in}{1.356575in}}%
\pgfpathlineto{\pgfqpoint{1.942830in}{1.347374in}}%
\pgfpathlineto{\pgfqpoint{1.951358in}{1.338022in}}%
\pgfpathlineto{\pgfqpoint{1.959324in}{1.328525in}}%
\pgfpathlineto{\pgfqpoint{1.955850in}{1.316850in}}%
\pgfpathlineto{\pgfqpoint{1.952393in}{1.305583in}}%
\pgfpathlineto{\pgfqpoint{1.948951in}{1.294716in}}%
\pgfpathlineto{\pgfqpoint{1.945524in}{1.284243in}}%
\pgfpathlineto{\pgfqpoint{1.937759in}{1.293591in}}%
\pgfpathlineto{\pgfqpoint{1.929444in}{1.302797in}}%
\pgfpathlineto{\pgfqpoint{1.920589in}{1.311855in}}%
\pgfpathlineto{\pgfqpoint{1.911205in}{1.320756in}}%
\pgfpathclose%
\pgfusepath{fill}%
\end{pgfscope}%
\begin{pgfscope}%
\pgfpathrectangle{\pgfqpoint{0.329460in}{0.284240in}}{\pgfqpoint{1.989680in}{1.989680in}}%
\pgfusepath{clip}%
\pgfsetbuttcap%
\pgfsetroundjoin%
\definecolor{currentfill}{rgb}{0.277941,0.056324,0.381191}%
\pgfsetfillcolor{currentfill}%
\pgfsetlinewidth{0.000000pt}%
\definecolor{currentstroke}{rgb}{0.000000,0.000000,0.000000}%
\pgfsetstrokecolor{currentstroke}%
\pgfsetdash{}{0pt}%
\pgfpathmoveto{\pgfqpoint{1.879679in}{1.148348in}}%
\pgfpathlineto{\pgfqpoint{1.882871in}{1.152167in}}%
\pgfpathlineto{\pgfqpoint{1.886072in}{1.156267in}}%
\pgfpathlineto{\pgfqpoint{1.889282in}{1.160655in}}%
\pgfpathlineto{\pgfqpoint{1.892501in}{1.165334in}}%
\pgfpathlineto{\pgfqpoint{1.898989in}{1.156536in}}%
\pgfpathlineto{\pgfqpoint{1.904956in}{1.147625in}}%
\pgfpathlineto{\pgfqpoint{1.910394in}{1.138608in}}%
\pgfpathlineto{\pgfqpoint{1.915296in}{1.129495in}}%
\pgfpathlineto{\pgfqpoint{1.911922in}{1.125004in}}%
\pgfpathlineto{\pgfqpoint{1.908559in}{1.120806in}}%
\pgfpathlineto{\pgfqpoint{1.905205in}{1.116896in}}%
\pgfpathlineto{\pgfqpoint{1.901860in}{1.113269in}}%
\pgfpathlineto{\pgfqpoint{1.897094in}{1.122188in}}%
\pgfpathlineto{\pgfqpoint{1.891804in}{1.131013in}}%
\pgfpathlineto{\pgfqpoint{1.885996in}{1.139735in}}%
\pgfpathlineto{\pgfqpoint{1.879679in}{1.148348in}}%
\pgfpathclose%
\pgfusepath{fill}%
\end{pgfscope}%
\begin{pgfscope}%
\pgfpathrectangle{\pgfqpoint{0.329460in}{0.284240in}}{\pgfqpoint{1.989680in}{1.989680in}}%
\pgfusepath{clip}%
\pgfsetbuttcap%
\pgfsetroundjoin%
\definecolor{currentfill}{rgb}{0.179019,0.433756,0.557430}%
\pgfsetfillcolor{currentfill}%
\pgfsetlinewidth{0.000000pt}%
\definecolor{currentstroke}{rgb}{0.000000,0.000000,0.000000}%
\pgfsetstrokecolor{currentstroke}%
\pgfsetdash{}{0pt}%
\pgfpathmoveto{\pgfqpoint{1.461047in}{1.445264in}}%
\pgfpathlineto{\pgfqpoint{1.462124in}{1.440121in}}%
\pgfpathlineto{\pgfqpoint{1.463199in}{1.435003in}}%
\pgfpathlineto{\pgfqpoint{1.464274in}{1.429912in}}%
\pgfpathlineto{\pgfqpoint{1.465347in}{1.424852in}}%
\pgfpathlineto{\pgfqpoint{1.475229in}{1.423008in}}%
\pgfpathlineto{\pgfqpoint{1.484997in}{1.421010in}}%
\pgfpathlineto{\pgfqpoint{1.494641in}{1.418862in}}%
\pgfpathlineto{\pgfqpoint{1.504153in}{1.416564in}}%
\pgfpathlineto{\pgfqpoint{1.502713in}{1.421709in}}%
\pgfpathlineto{\pgfqpoint{1.501271in}{1.426884in}}%
\pgfpathlineto{\pgfqpoint{1.499828in}{1.432087in}}%
\pgfpathlineto{\pgfqpoint{1.498384in}{1.437315in}}%
\pgfpathlineto{\pgfqpoint{1.489233in}{1.439519in}}%
\pgfpathlineto{\pgfqpoint{1.479954in}{1.441579in}}%
\pgfpathlineto{\pgfqpoint{1.470556in}{1.443495in}}%
\pgfpathlineto{\pgfqpoint{1.461047in}{1.445264in}}%
\pgfpathclose%
\pgfusepath{fill}%
\end{pgfscope}%
\begin{pgfscope}%
\pgfpathrectangle{\pgfqpoint{0.329460in}{0.284240in}}{\pgfqpoint{1.989680in}{1.989680in}}%
\pgfusepath{clip}%
\pgfsetbuttcap%
\pgfsetroundjoin%
\definecolor{currentfill}{rgb}{0.248629,0.278775,0.534556}%
\pgfsetfillcolor{currentfill}%
\pgfsetlinewidth{0.000000pt}%
\definecolor{currentstroke}{rgb}{0.000000,0.000000,0.000000}%
\pgfsetstrokecolor{currentstroke}%
\pgfsetdash{}{0pt}%
\pgfpathmoveto{\pgfqpoint{1.633487in}{1.315079in}}%
\pgfpathlineto{\pgfqpoint{1.635866in}{1.310188in}}%
\pgfpathlineto{\pgfqpoint{1.638243in}{1.305367in}}%
\pgfpathlineto{\pgfqpoint{1.640620in}{1.300619in}}%
\pgfpathlineto{\pgfqpoint{1.642994in}{1.295947in}}%
\pgfpathlineto{\pgfqpoint{1.651098in}{1.291266in}}%
\pgfpathlineto{\pgfqpoint{1.658920in}{1.286456in}}%
\pgfpathlineto{\pgfqpoint{1.666454in}{1.281521in}}%
\pgfpathlineto{\pgfqpoint{1.673691in}{1.276464in}}%
\pgfpathlineto{\pgfqpoint{1.671058in}{1.281310in}}%
\pgfpathlineto{\pgfqpoint{1.668424in}{1.286232in}}%
\pgfpathlineto{\pgfqpoint{1.665789in}{1.291227in}}%
\pgfpathlineto{\pgfqpoint{1.663152in}{1.296292in}}%
\pgfpathlineto{\pgfqpoint{1.656160in}{1.301168in}}%
\pgfpathlineto{\pgfqpoint{1.648880in}{1.305927in}}%
\pgfpathlineto{\pgfqpoint{1.641320in}{1.310566in}}%
\pgfpathlineto{\pgfqpoint{1.633487in}{1.315079in}}%
\pgfpathclose%
\pgfusepath{fill}%
\end{pgfscope}%
\begin{pgfscope}%
\pgfpathrectangle{\pgfqpoint{0.329460in}{0.284240in}}{\pgfqpoint{1.989680in}{1.989680in}}%
\pgfusepath{clip}%
\pgfsetbuttcap%
\pgfsetroundjoin%
\definecolor{currentfill}{rgb}{0.268510,0.009605,0.335427}%
\pgfsetfillcolor{currentfill}%
\pgfsetlinewidth{0.000000pt}%
\definecolor{currentstroke}{rgb}{0.000000,0.000000,0.000000}%
\pgfsetstrokecolor{currentstroke}%
\pgfsetdash{}{0pt}%
\pgfpathmoveto{\pgfqpoint{1.817267in}{1.124090in}}%
\pgfpathlineto{\pgfqpoint{1.820338in}{1.123221in}}%
\pgfpathlineto{\pgfqpoint{1.823413in}{1.122547in}}%
\pgfpathlineto{\pgfqpoint{1.826492in}{1.122071in}}%
\pgfpathlineto{\pgfqpoint{1.829575in}{1.121798in}}%
\pgfpathlineto{\pgfqpoint{1.835231in}{1.113970in}}%
\pgfpathlineto{\pgfqpoint{1.840423in}{1.106044in}}%
\pgfpathlineto{\pgfqpoint{1.845146in}{1.098028in}}%
\pgfpathlineto{\pgfqpoint{1.849393in}{1.089928in}}%
\pgfpathlineto{\pgfqpoint{1.846166in}{1.090410in}}%
\pgfpathlineto{\pgfqpoint{1.842944in}{1.091094in}}%
\pgfpathlineto{\pgfqpoint{1.839726in}{1.091978in}}%
\pgfpathlineto{\pgfqpoint{1.836513in}{1.093056in}}%
\pgfpathlineto{\pgfqpoint{1.832392in}{1.100942in}}%
\pgfpathlineto{\pgfqpoint{1.827806in}{1.108748in}}%
\pgfpathlineto{\pgfqpoint{1.822763in}{1.116466in}}%
\pgfpathlineto{\pgfqpoint{1.817267in}{1.124090in}}%
\pgfpathclose%
\pgfusepath{fill}%
\end{pgfscope}%
\begin{pgfscope}%
\pgfpathrectangle{\pgfqpoint{0.329460in}{0.284240in}}{\pgfqpoint{1.989680in}{1.989680in}}%
\pgfusepath{clip}%
\pgfsetbuttcap%
\pgfsetroundjoin%
\definecolor{currentfill}{rgb}{0.282327,0.094955,0.417331}%
\pgfsetfillcolor{currentfill}%
\pgfsetlinewidth{0.000000pt}%
\definecolor{currentstroke}{rgb}{0.000000,0.000000,0.000000}%
\pgfsetstrokecolor{currentstroke}%
\pgfsetdash{}{0pt}%
\pgfpathmoveto{\pgfqpoint{0.929510in}{1.155238in}}%
\pgfpathlineto{\pgfqpoint{0.926443in}{1.151731in}}%
\pgfpathlineto{\pgfqpoint{0.923375in}{1.148358in}}%
\pgfpathlineto{\pgfqpoint{0.920305in}{1.145125in}}%
\pgfpathlineto{\pgfqpoint{0.917234in}{1.142033in}}%
\pgfpathlineto{\pgfqpoint{0.922303in}{1.149020in}}%
\pgfpathlineto{\pgfqpoint{0.927783in}{1.155914in}}%
\pgfpathlineto{\pgfqpoint{0.933667in}{1.162709in}}%
\pgfpathlineto{\pgfqpoint{0.939949in}{1.169399in}}%
\pgfpathlineto{\pgfqpoint{0.942847in}{1.172285in}}%
\pgfpathlineto{\pgfqpoint{0.945743in}{1.175312in}}%
\pgfpathlineto{\pgfqpoint{0.948639in}{1.178478in}}%
\pgfpathlineto{\pgfqpoint{0.951533in}{1.181778in}}%
\pgfpathlineto{\pgfqpoint{0.945441in}{1.175290in}}%
\pgfpathlineto{\pgfqpoint{0.939735in}{1.168699in}}%
\pgfpathlineto{\pgfqpoint{0.934423in}{1.162013in}}%
\pgfpathlineto{\pgfqpoint{0.929510in}{1.155238in}}%
\pgfpathclose%
\pgfusepath{fill}%
\end{pgfscope}%
\begin{pgfscope}%
\pgfpathrectangle{\pgfqpoint{0.329460in}{0.284240in}}{\pgfqpoint{1.989680in}{1.989680in}}%
\pgfusepath{clip}%
\pgfsetbuttcap%
\pgfsetroundjoin%
\definecolor{currentfill}{rgb}{0.179019,0.433756,0.557430}%
\pgfsetfillcolor{currentfill}%
\pgfsetlinewidth{0.000000pt}%
\definecolor{currentstroke}{rgb}{0.000000,0.000000,0.000000}%
\pgfsetstrokecolor{currentstroke}%
\pgfsetdash{}{0pt}%
\pgfpathmoveto{\pgfqpoint{1.195972in}{1.435238in}}%
\pgfpathlineto{\pgfqpoint{1.194448in}{1.429987in}}%
\pgfpathlineto{\pgfqpoint{1.192925in}{1.424762in}}%
\pgfpathlineto{\pgfqpoint{1.191405in}{1.419565in}}%
\pgfpathlineto{\pgfqpoint{1.189886in}{1.414398in}}%
\pgfpathlineto{\pgfqpoint{1.199272in}{1.416826in}}%
\pgfpathlineto{\pgfqpoint{1.208799in}{1.419108in}}%
\pgfpathlineto{\pgfqpoint{1.218458in}{1.421240in}}%
\pgfpathlineto{\pgfqpoint{1.228239in}{1.423220in}}%
\pgfpathlineto{\pgfqpoint{1.229395in}{1.428297in}}%
\pgfpathlineto{\pgfqpoint{1.230552in}{1.433404in}}%
\pgfpathlineto{\pgfqpoint{1.231711in}{1.438539in}}%
\pgfpathlineto{\pgfqpoint{1.232871in}{1.443699in}}%
\pgfpathlineto{\pgfqpoint{1.223460in}{1.441799in}}%
\pgfpathlineto{\pgfqpoint{1.214167in}{1.439755in}}%
\pgfpathlineto{\pgfqpoint{1.205002in}{1.437567in}}%
\pgfpathlineto{\pgfqpoint{1.195972in}{1.435238in}}%
\pgfpathclose%
\pgfusepath{fill}%
\end{pgfscope}%
\begin{pgfscope}%
\pgfpathrectangle{\pgfqpoint{0.329460in}{0.284240in}}{\pgfqpoint{1.989680in}{1.989680in}}%
\pgfusepath{clip}%
\pgfsetbuttcap%
\pgfsetroundjoin%
\definecolor{currentfill}{rgb}{0.163625,0.471133,0.558148}%
\pgfsetfillcolor{currentfill}%
\pgfsetlinewidth{0.000000pt}%
\definecolor{currentstroke}{rgb}{0.000000,0.000000,0.000000}%
\pgfsetstrokecolor{currentstroke}%
\pgfsetdash{}{0pt}%
\pgfpathmoveto{\pgfqpoint{1.312906in}{1.473927in}}%
\pgfpathlineto{\pgfqpoint{1.312514in}{1.468797in}}%
\pgfpathlineto{\pgfqpoint{1.312123in}{1.463682in}}%
\pgfpathlineto{\pgfqpoint{1.311732in}{1.458584in}}%
\pgfpathlineto{\pgfqpoint{1.311342in}{1.453505in}}%
\pgfpathlineto{\pgfqpoint{1.321412in}{1.454042in}}%
\pgfpathlineto{\pgfqpoint{1.331510in}{1.454423in}}%
\pgfpathlineto{\pgfqpoint{1.341625in}{1.454648in}}%
\pgfpathlineto{\pgfqpoint{1.351750in}{1.454718in}}%
\pgfpathlineto{\pgfqpoint{1.351745in}{1.459783in}}%
\pgfpathlineto{\pgfqpoint{1.351739in}{1.464868in}}%
\pgfpathlineto{\pgfqpoint{1.351734in}{1.469970in}}%
\pgfpathlineto{\pgfqpoint{1.351728in}{1.475088in}}%
\pgfpathlineto{\pgfqpoint{1.342001in}{1.475021in}}%
\pgfpathlineto{\pgfqpoint{1.332282in}{1.474806in}}%
\pgfpathlineto{\pgfqpoint{1.322581in}{1.474441in}}%
\pgfpathlineto{\pgfqpoint{1.312906in}{1.473927in}}%
\pgfpathclose%
\pgfusepath{fill}%
\end{pgfscope}%
\begin{pgfscope}%
\pgfpathrectangle{\pgfqpoint{0.329460in}{0.284240in}}{\pgfqpoint{1.989680in}{1.989680in}}%
\pgfusepath{clip}%
\pgfsetbuttcap%
\pgfsetroundjoin%
\definecolor{currentfill}{rgb}{0.163625,0.471133,0.558148}%
\pgfsetfillcolor{currentfill}%
\pgfsetlinewidth{0.000000pt}%
\definecolor{currentstroke}{rgb}{0.000000,0.000000,0.000000}%
\pgfsetstrokecolor{currentstroke}%
\pgfsetdash{}{0pt}%
\pgfpathmoveto{\pgfqpoint{1.351728in}{1.475088in}}%
\pgfpathlineto{\pgfqpoint{1.351734in}{1.469970in}}%
\pgfpathlineto{\pgfqpoint{1.351739in}{1.464868in}}%
\pgfpathlineto{\pgfqpoint{1.351745in}{1.459783in}}%
\pgfpathlineto{\pgfqpoint{1.351750in}{1.454718in}}%
\pgfpathlineto{\pgfqpoint{1.361874in}{1.454631in}}%
\pgfpathlineto{\pgfqpoint{1.371989in}{1.454388in}}%
\pgfpathlineto{\pgfqpoint{1.382084in}{1.453990in}}%
\pgfpathlineto{\pgfqpoint{1.392150in}{1.453436in}}%
\pgfpathlineto{\pgfqpoint{1.391749in}{1.458515in}}%
\pgfpathlineto{\pgfqpoint{1.391347in}{1.463614in}}%
\pgfpathlineto{\pgfqpoint{1.390945in}{1.468730in}}%
\pgfpathlineto{\pgfqpoint{1.390542in}{1.473861in}}%
\pgfpathlineto{\pgfqpoint{1.380871in}{1.474391in}}%
\pgfpathlineto{\pgfqpoint{1.371172in}{1.474772in}}%
\pgfpathlineto{\pgfqpoint{1.361455in}{1.475005in}}%
\pgfpathlineto{\pgfqpoint{1.351728in}{1.475088in}}%
\pgfpathclose%
\pgfusepath{fill}%
\end{pgfscope}%
\begin{pgfscope}%
\pgfpathrectangle{\pgfqpoint{0.329460in}{0.284240in}}{\pgfqpoint{1.989680in}{1.989680in}}%
\pgfusepath{clip}%
\pgfsetbuttcap%
\pgfsetroundjoin%
\definecolor{currentfill}{rgb}{0.283072,0.130895,0.449241}%
\pgfsetfillcolor{currentfill}%
\pgfsetlinewidth{0.000000pt}%
\definecolor{currentstroke}{rgb}{0.000000,0.000000,0.000000}%
\pgfsetstrokecolor{currentstroke}%
\pgfsetdash{}{0pt}%
\pgfpathmoveto{\pgfqpoint{1.733712in}{1.201757in}}%
\pgfpathlineto{\pgfqpoint{1.736560in}{1.197997in}}%
\pgfpathlineto{\pgfqpoint{1.739409in}{1.194357in}}%
\pgfpathlineto{\pgfqpoint{1.742258in}{1.190842in}}%
\pgfpathlineto{\pgfqpoint{1.745107in}{1.187455in}}%
\pgfpathlineto{\pgfqpoint{1.751537in}{1.181062in}}%
\pgfpathlineto{\pgfqpoint{1.757587in}{1.174562in}}%
\pgfpathlineto{\pgfqpoint{1.763250in}{1.167961in}}%
\pgfpathlineto{\pgfqpoint{1.768518in}{1.161265in}}%
\pgfpathlineto{\pgfqpoint{1.765487in}{1.164856in}}%
\pgfpathlineto{\pgfqpoint{1.762456in}{1.168576in}}%
\pgfpathlineto{\pgfqpoint{1.759427in}{1.172419in}}%
\pgfpathlineto{\pgfqpoint{1.756398in}{1.176384in}}%
\pgfpathlineto{\pgfqpoint{1.751294in}{1.182871in}}%
\pgfpathlineto{\pgfqpoint{1.745808in}{1.189266in}}%
\pgfpathlineto{\pgfqpoint{1.739945in}{1.195563in}}%
\pgfpathlineto{\pgfqpoint{1.733712in}{1.201757in}}%
\pgfpathclose%
\pgfusepath{fill}%
\end{pgfscope}%
\begin{pgfscope}%
\pgfpathrectangle{\pgfqpoint{0.329460in}{0.284240in}}{\pgfqpoint{1.989680in}{1.989680in}}%
\pgfusepath{clip}%
\pgfsetbuttcap%
\pgfsetroundjoin%
\definecolor{currentfill}{rgb}{0.282327,0.094955,0.417331}%
\pgfsetfillcolor{currentfill}%
\pgfsetlinewidth{0.000000pt}%
\definecolor{currentstroke}{rgb}{0.000000,0.000000,0.000000}%
\pgfsetstrokecolor{currentstroke}%
\pgfsetdash{}{0pt}%
\pgfpathmoveto{\pgfqpoint{1.892501in}{1.165334in}}%
\pgfpathlineto{\pgfqpoint{1.895730in}{1.170311in}}%
\pgfpathlineto{\pgfqpoint{1.898968in}{1.175590in}}%
\pgfpathlineto{\pgfqpoint{1.902217in}{1.181176in}}%
\pgfpathlineto{\pgfqpoint{1.905476in}{1.187074in}}%
\pgfpathlineto{\pgfqpoint{1.912138in}{1.178095in}}%
\pgfpathlineto{\pgfqpoint{1.918268in}{1.169001in}}%
\pgfpathlineto{\pgfqpoint{1.923856in}{1.159798in}}%
\pgfpathlineto{\pgfqpoint{1.928895in}{1.150497in}}%
\pgfpathlineto{\pgfqpoint{1.925479in}{1.144781in}}%
\pgfpathlineto{\pgfqpoint{1.922074in}{1.139378in}}%
\pgfpathlineto{\pgfqpoint{1.918679in}{1.134285in}}%
\pgfpathlineto{\pgfqpoint{1.915296in}{1.129495in}}%
\pgfpathlineto{\pgfqpoint{1.910394in}{1.138608in}}%
\pgfpathlineto{\pgfqpoint{1.904956in}{1.147625in}}%
\pgfpathlineto{\pgfqpoint{1.898989in}{1.156536in}}%
\pgfpathlineto{\pgfqpoint{1.892501in}{1.165334in}}%
\pgfpathclose%
\pgfusepath{fill}%
\end{pgfscope}%
\begin{pgfscope}%
\pgfpathrectangle{\pgfqpoint{0.329460in}{0.284240in}}{\pgfqpoint{1.989680in}{1.989680in}}%
\pgfusepath{clip}%
\pgfsetbuttcap%
\pgfsetroundjoin%
\definecolor{currentfill}{rgb}{0.274128,0.199721,0.498911}%
\pgfsetfillcolor{currentfill}%
\pgfsetlinewidth{0.000000pt}%
\definecolor{currentstroke}{rgb}{0.000000,0.000000,0.000000}%
\pgfsetstrokecolor{currentstroke}%
\pgfsetdash{}{0pt}%
\pgfpathmoveto{\pgfqpoint{0.986239in}{1.230651in}}%
\pgfpathlineto{\pgfqpoint{0.983346in}{1.225997in}}%
\pgfpathlineto{\pgfqpoint{0.980454in}{1.221438in}}%
\pgfpathlineto{\pgfqpoint{0.977562in}{1.216977in}}%
\pgfpathlineto{\pgfqpoint{0.974670in}{1.212617in}}%
\pgfpathlineto{\pgfqpoint{0.980746in}{1.218600in}}%
\pgfpathlineto{\pgfqpoint{0.987174in}{1.224476in}}%
\pgfpathlineto{\pgfqpoint{0.993946in}{1.230242in}}%
\pgfpathlineto{\pgfqpoint{1.001056in}{1.235892in}}%
\pgfpathlineto{\pgfqpoint{1.003736in}{1.240057in}}%
\pgfpathlineto{\pgfqpoint{1.006415in}{1.244324in}}%
\pgfpathlineto{\pgfqpoint{1.009095in}{1.248689in}}%
\pgfpathlineto{\pgfqpoint{1.011776in}{1.253148in}}%
\pgfpathlineto{\pgfqpoint{1.004893in}{1.247687in}}%
\pgfpathlineto{\pgfqpoint{0.998338in}{1.242114in}}%
\pgfpathlineto{\pgfqpoint{0.992117in}{1.236433in}}%
\pgfpathlineto{\pgfqpoint{0.986239in}{1.230651in}}%
\pgfpathclose%
\pgfusepath{fill}%
\end{pgfscope}%
\begin{pgfscope}%
\pgfpathrectangle{\pgfqpoint{0.329460in}{0.284240in}}{\pgfqpoint{1.989680in}{1.989680in}}%
\pgfusepath{clip}%
\pgfsetbuttcap%
\pgfsetroundjoin%
\definecolor{currentfill}{rgb}{0.268510,0.009605,0.335427}%
\pgfsetfillcolor{currentfill}%
\pgfsetlinewidth{0.000000pt}%
\definecolor{currentstroke}{rgb}{0.000000,0.000000,0.000000}%
\pgfsetstrokecolor{currentstroke}%
\pgfsetdash{}{0pt}%
\pgfpathmoveto{\pgfqpoint{0.823392in}{1.086237in}}%
\pgfpathlineto{\pgfqpoint{0.820086in}{1.087724in}}%
\pgfpathlineto{\pgfqpoint{0.816772in}{1.089456in}}%
\pgfpathlineto{\pgfqpoint{0.813451in}{1.091439in}}%
\pgfpathlineto{\pgfqpoint{0.810122in}{1.093676in}}%
\pgfpathlineto{\pgfqpoint{0.814295in}{1.102470in}}%
\pgfpathlineto{\pgfqpoint{0.818985in}{1.111181in}}%
\pgfpathlineto{\pgfqpoint{0.824187in}{1.119799in}}%
\pgfpathlineto{\pgfqpoint{0.829893in}{1.128316in}}%
\pgfpathlineto{\pgfqpoint{0.833083in}{1.125879in}}%
\pgfpathlineto{\pgfqpoint{0.836266in}{1.123695in}}%
\pgfpathlineto{\pgfqpoint{0.839442in}{1.121761in}}%
\pgfpathlineto{\pgfqpoint{0.842611in}{1.120071in}}%
\pgfpathlineto{\pgfqpoint{0.837060in}{1.111751in}}%
\pgfpathlineto{\pgfqpoint{0.832003in}{1.103333in}}%
\pgfpathlineto{\pgfqpoint{0.827444in}{1.094826in}}%
\pgfpathlineto{\pgfqpoint{0.823392in}{1.086237in}}%
\pgfpathclose%
\pgfusepath{fill}%
\end{pgfscope}%
\begin{pgfscope}%
\pgfpathrectangle{\pgfqpoint{0.329460in}{0.284240in}}{\pgfqpoint{1.989680in}{1.989680in}}%
\pgfusepath{clip}%
\pgfsetbuttcap%
\pgfsetroundjoin%
\definecolor{currentfill}{rgb}{0.267004,0.004874,0.329415}%
\pgfsetfillcolor{currentfill}%
\pgfsetlinewidth{0.000000pt}%
\definecolor{currentstroke}{rgb}{0.000000,0.000000,0.000000}%
\pgfsetstrokecolor{currentstroke}%
\pgfsetdash{}{0pt}%
\pgfpathmoveto{\pgfqpoint{0.836550in}{1.082660in}}%
\pgfpathlineto{\pgfqpoint{0.833270in}{1.083208in}}%
\pgfpathlineto{\pgfqpoint{0.829984in}{1.083984in}}%
\pgfpathlineto{\pgfqpoint{0.826691in}{1.084992in}}%
\pgfpathlineto{\pgfqpoint{0.823392in}{1.086237in}}%
\pgfpathlineto{\pgfqpoint{0.827444in}{1.094826in}}%
\pgfpathlineto{\pgfqpoint{0.832003in}{1.103333in}}%
\pgfpathlineto{\pgfqpoint{0.837060in}{1.111751in}}%
\pgfpathlineto{\pgfqpoint{0.842611in}{1.120071in}}%
\pgfpathlineto{\pgfqpoint{0.845773in}{1.118622in}}%
\pgfpathlineto{\pgfqpoint{0.848930in}{1.117409in}}%
\pgfpathlineto{\pgfqpoint{0.852080in}{1.116428in}}%
\pgfpathlineto{\pgfqpoint{0.855225in}{1.115673in}}%
\pgfpathlineto{\pgfqpoint{0.849828in}{1.107554in}}%
\pgfpathlineto{\pgfqpoint{0.844913in}{1.099340in}}%
\pgfpathlineto{\pgfqpoint{0.840485in}{1.091040in}}%
\pgfpathlineto{\pgfqpoint{0.836550in}{1.082660in}}%
\pgfpathclose%
\pgfusepath{fill}%
\end{pgfscope}%
\begin{pgfscope}%
\pgfpathrectangle{\pgfqpoint{0.329460in}{0.284240in}}{\pgfqpoint{1.989680in}{1.989680in}}%
\pgfusepath{clip}%
\pgfsetbuttcap%
\pgfsetroundjoin%
\definecolor{currentfill}{rgb}{0.163625,0.471133,0.558148}%
\pgfsetfillcolor{currentfill}%
\pgfsetlinewidth{0.000000pt}%
\definecolor{currentstroke}{rgb}{0.000000,0.000000,0.000000}%
\pgfsetstrokecolor{currentstroke}%
\pgfsetdash{}{0pt}%
\pgfpathmoveto{\pgfqpoint{1.274649in}{1.470394in}}%
\pgfpathlineto{\pgfqpoint{1.273865in}{1.465225in}}%
\pgfpathlineto{\pgfqpoint{1.273082in}{1.460071in}}%
\pgfpathlineto{\pgfqpoint{1.272300in}{1.454933in}}%
\pgfpathlineto{\pgfqpoint{1.271519in}{1.449816in}}%
\pgfpathlineto{\pgfqpoint{1.281387in}{1.450968in}}%
\pgfpathlineto{\pgfqpoint{1.291320in}{1.451968in}}%
\pgfpathlineto{\pgfqpoint{1.301308in}{1.452814in}}%
\pgfpathlineto{\pgfqpoint{1.311342in}{1.453505in}}%
\pgfpathlineto{\pgfqpoint{1.311732in}{1.458584in}}%
\pgfpathlineto{\pgfqpoint{1.312123in}{1.463682in}}%
\pgfpathlineto{\pgfqpoint{1.312514in}{1.468797in}}%
\pgfpathlineto{\pgfqpoint{1.312906in}{1.473927in}}%
\pgfpathlineto{\pgfqpoint{1.303267in}{1.473265in}}%
\pgfpathlineto{\pgfqpoint{1.293671in}{1.472455in}}%
\pgfpathlineto{\pgfqpoint{1.284129in}{1.471498in}}%
\pgfpathlineto{\pgfqpoint{1.274649in}{1.470394in}}%
\pgfpathclose%
\pgfusepath{fill}%
\end{pgfscope}%
\begin{pgfscope}%
\pgfpathrectangle{\pgfqpoint{0.329460in}{0.284240in}}{\pgfqpoint{1.989680in}{1.989680in}}%
\pgfusepath{clip}%
\pgfsetbuttcap%
\pgfsetroundjoin%
\definecolor{currentfill}{rgb}{0.163625,0.471133,0.558148}%
\pgfsetfillcolor{currentfill}%
\pgfsetlinewidth{0.000000pt}%
\definecolor{currentstroke}{rgb}{0.000000,0.000000,0.000000}%
\pgfsetstrokecolor{currentstroke}%
\pgfsetdash{}{0pt}%
\pgfpathmoveto{\pgfqpoint{1.390542in}{1.473861in}}%
\pgfpathlineto{\pgfqpoint{1.390945in}{1.468730in}}%
\pgfpathlineto{\pgfqpoint{1.391347in}{1.463614in}}%
\pgfpathlineto{\pgfqpoint{1.391749in}{1.458515in}}%
\pgfpathlineto{\pgfqpoint{1.392150in}{1.453436in}}%
\pgfpathlineto{\pgfqpoint{1.402179in}{1.452727in}}%
\pgfpathlineto{\pgfqpoint{1.412162in}{1.451864in}}%
\pgfpathlineto{\pgfqpoint{1.422087in}{1.450848in}}%
\pgfpathlineto{\pgfqpoint{1.431948in}{1.449678in}}%
\pgfpathlineto{\pgfqpoint{1.431156in}{1.454797in}}%
\pgfpathlineto{\pgfqpoint{1.430363in}{1.459936in}}%
\pgfpathlineto{\pgfqpoint{1.429570in}{1.465092in}}%
\pgfpathlineto{\pgfqpoint{1.428775in}{1.470263in}}%
\pgfpathlineto{\pgfqpoint{1.419302in}{1.471382in}}%
\pgfpathlineto{\pgfqpoint{1.409767in}{1.472356in}}%
\pgfpathlineto{\pgfqpoint{1.400177in}{1.473182in}}%
\pgfpathlineto{\pgfqpoint{1.390542in}{1.473861in}}%
\pgfpathclose%
\pgfusepath{fill}%
\end{pgfscope}%
\begin{pgfscope}%
\pgfpathrectangle{\pgfqpoint{0.329460in}{0.284240in}}{\pgfqpoint{1.989680in}{1.989680in}}%
\pgfusepath{clip}%
\pgfsetbuttcap%
\pgfsetroundjoin%
\definecolor{currentfill}{rgb}{0.248629,0.278775,0.534556}%
\pgfsetfillcolor{currentfill}%
\pgfsetlinewidth{0.000000pt}%
\definecolor{currentstroke}{rgb}{0.000000,0.000000,0.000000}%
\pgfsetstrokecolor{currentstroke}%
\pgfsetdash{}{0pt}%
\pgfpathmoveto{\pgfqpoint{1.033255in}{1.291864in}}%
\pgfpathlineto{\pgfqpoint{1.030566in}{1.286757in}}%
\pgfpathlineto{\pgfqpoint{1.027878in}{1.281721in}}%
\pgfpathlineto{\pgfqpoint{1.025192in}{1.276758in}}%
\pgfpathlineto{\pgfqpoint{1.022506in}{1.271872in}}%
\pgfpathlineto{\pgfqpoint{1.029473in}{1.277032in}}%
\pgfpathlineto{\pgfqpoint{1.036744in}{1.282075in}}%
\pgfpathlineto{\pgfqpoint{1.044310in}{1.286997in}}%
\pgfpathlineto{\pgfqpoint{1.052164in}{1.291793in}}%
\pgfpathlineto{\pgfqpoint{1.054599in}{1.296502in}}%
\pgfpathlineto{\pgfqpoint{1.057036in}{1.301287in}}%
\pgfpathlineto{\pgfqpoint{1.059473in}{1.306145in}}%
\pgfpathlineto{\pgfqpoint{1.061912in}{1.311073in}}%
\pgfpathlineto{\pgfqpoint{1.054322in}{1.306448in}}%
\pgfpathlineto{\pgfqpoint{1.047010in}{1.301702in}}%
\pgfpathlineto{\pgfqpoint{1.039986in}{1.296839in}}%
\pgfpathlineto{\pgfqpoint{1.033255in}{1.291864in}}%
\pgfpathclose%
\pgfusepath{fill}%
\end{pgfscope}%
\begin{pgfscope}%
\pgfpathrectangle{\pgfqpoint{0.329460in}{0.284240in}}{\pgfqpoint{1.989680in}{1.989680in}}%
\pgfusepath{clip}%
\pgfsetbuttcap%
\pgfsetroundjoin%
\definecolor{currentfill}{rgb}{0.272594,0.025563,0.353093}%
\pgfsetfillcolor{currentfill}%
\pgfsetlinewidth{0.000000pt}%
\definecolor{currentstroke}{rgb}{0.000000,0.000000,0.000000}%
\pgfsetstrokecolor{currentstroke}%
\pgfsetdash{}{0pt}%
\pgfpathmoveto{\pgfqpoint{0.810122in}{1.093676in}}%
\pgfpathlineto{\pgfqpoint{0.806785in}{1.096173in}}%
\pgfpathlineto{\pgfqpoint{0.803440in}{1.098934in}}%
\pgfpathlineto{\pgfqpoint{0.800087in}{1.101965in}}%
\pgfpathlineto{\pgfqpoint{0.796724in}{1.105269in}}%
\pgfpathlineto{\pgfqpoint{0.801018in}{1.114264in}}%
\pgfpathlineto{\pgfqpoint{0.805843in}{1.123173in}}%
\pgfpathlineto{\pgfqpoint{0.811191in}{1.131987in}}%
\pgfpathlineto{\pgfqpoint{0.817056in}{1.140698in}}%
\pgfpathlineto{\pgfqpoint{0.820277in}{1.137198in}}%
\pgfpathlineto{\pgfqpoint{0.823491in}{1.133971in}}%
\pgfpathlineto{\pgfqpoint{0.826696in}{1.131012in}}%
\pgfpathlineto{\pgfqpoint{0.829893in}{1.128316in}}%
\pgfpathlineto{\pgfqpoint{0.824187in}{1.119799in}}%
\pgfpathlineto{\pgfqpoint{0.818985in}{1.111181in}}%
\pgfpathlineto{\pgfqpoint{0.814295in}{1.102470in}}%
\pgfpathlineto{\pgfqpoint{0.810122in}{1.093676in}}%
\pgfpathclose%
\pgfusepath{fill}%
\end{pgfscope}%
\begin{pgfscope}%
\pgfpathrectangle{\pgfqpoint{0.329460in}{0.284240in}}{\pgfqpoint{1.989680in}{1.989680in}}%
\pgfusepath{clip}%
\pgfsetbuttcap%
\pgfsetroundjoin%
\definecolor{currentfill}{rgb}{0.271305,0.019942,0.347269}%
\pgfsetfillcolor{currentfill}%
\pgfsetlinewidth{0.000000pt}%
\definecolor{currentstroke}{rgb}{0.000000,0.000000,0.000000}%
\pgfsetstrokecolor{currentstroke}%
\pgfsetdash{}{0pt}%
\pgfpathmoveto{\pgfqpoint{1.805019in}{1.129426in}}%
\pgfpathlineto{\pgfqpoint{1.808076in}{1.127820in}}%
\pgfpathlineto{\pgfqpoint{1.811136in}{1.126393in}}%
\pgfpathlineto{\pgfqpoint{1.814200in}{1.125148in}}%
\pgfpathlineto{\pgfqpoint{1.817267in}{1.124090in}}%
\pgfpathlineto{\pgfqpoint{1.822763in}{1.116466in}}%
\pgfpathlineto{\pgfqpoint{1.827806in}{1.108748in}}%
\pgfpathlineto{\pgfqpoint{1.832392in}{1.100942in}}%
\pgfpathlineto{\pgfqpoint{1.836513in}{1.093056in}}%
\pgfpathlineto{\pgfqpoint{1.833304in}{1.094324in}}%
\pgfpathlineto{\pgfqpoint{1.830098in}{1.095780in}}%
\pgfpathlineto{\pgfqpoint{1.826897in}{1.097418in}}%
\pgfpathlineto{\pgfqpoint{1.823699in}{1.099236in}}%
\pgfpathlineto{\pgfqpoint{1.819702in}{1.106907in}}%
\pgfpathlineto{\pgfqpoint{1.815252in}{1.114501in}}%
\pgfpathlineto{\pgfqpoint{1.810356in}{1.122009in}}%
\pgfpathlineto{\pgfqpoint{1.805019in}{1.129426in}}%
\pgfpathclose%
\pgfusepath{fill}%
\end{pgfscope}%
\begin{pgfscope}%
\pgfpathrectangle{\pgfqpoint{0.329460in}{0.284240in}}{\pgfqpoint{1.989680in}{1.989680in}}%
\pgfusepath{clip}%
\pgfsetbuttcap%
\pgfsetroundjoin%
\definecolor{currentfill}{rgb}{0.267004,0.004874,0.329415}%
\pgfsetfillcolor{currentfill}%
\pgfsetlinewidth{0.000000pt}%
\definecolor{currentstroke}{rgb}{0.000000,0.000000,0.000000}%
\pgfsetstrokecolor{currentstroke}%
\pgfsetdash{}{0pt}%
\pgfpathmoveto{\pgfqpoint{0.849612in}{1.082665in}}%
\pgfpathlineto{\pgfqpoint{0.846355in}{1.082343in}}%
\pgfpathlineto{\pgfqpoint{0.843092in}{1.082232in}}%
\pgfpathlineto{\pgfqpoint{0.839824in}{1.082336in}}%
\pgfpathlineto{\pgfqpoint{0.836550in}{1.082660in}}%
\pgfpathlineto{\pgfqpoint{0.840485in}{1.091040in}}%
\pgfpathlineto{\pgfqpoint{0.844913in}{1.099340in}}%
\pgfpathlineto{\pgfqpoint{0.849828in}{1.107554in}}%
\pgfpathlineto{\pgfqpoint{0.855225in}{1.115673in}}%
\pgfpathlineto{\pgfqpoint{0.858364in}{1.115142in}}%
\pgfpathlineto{\pgfqpoint{0.861498in}{1.114830in}}%
\pgfpathlineto{\pgfqpoint{0.864626in}{1.114732in}}%
\pgfpathlineto{\pgfqpoint{0.867750in}{1.114845in}}%
\pgfpathlineto{\pgfqpoint{0.862506in}{1.106930in}}%
\pgfpathlineto{\pgfqpoint{0.857730in}{1.098923in}}%
\pgfpathlineto{\pgfqpoint{0.853431in}{1.090832in}}%
\pgfpathlineto{\pgfqpoint{0.849612in}{1.082665in}}%
\pgfpathclose%
\pgfusepath{fill}%
\end{pgfscope}%
\begin{pgfscope}%
\pgfpathrectangle{\pgfqpoint{0.329460in}{0.284240in}}{\pgfqpoint{1.989680in}{1.989680in}}%
\pgfusepath{clip}%
\pgfsetbuttcap%
\pgfsetroundjoin%
\definecolor{currentfill}{rgb}{0.195860,0.395433,0.555276}%
\pgfsetfillcolor{currentfill}%
\pgfsetlinewidth{0.000000pt}%
\definecolor{currentstroke}{rgb}{0.000000,0.000000,0.000000}%
\pgfsetstrokecolor{currentstroke}%
\pgfsetdash{}{0pt}%
\pgfpathmoveto{\pgfqpoint{1.540697in}{1.405921in}}%
\pgfpathlineto{\pgfqpoint{1.542482in}{1.400700in}}%
\pgfpathlineto{\pgfqpoint{1.544266in}{1.395515in}}%
\pgfpathlineto{\pgfqpoint{1.546048in}{1.390368in}}%
\pgfpathlineto{\pgfqpoint{1.547828in}{1.385263in}}%
\pgfpathlineto{\pgfqpoint{1.556877in}{1.382127in}}%
\pgfpathlineto{\pgfqpoint{1.565735in}{1.378850in}}%
\pgfpathlineto{\pgfqpoint{1.574393in}{1.375433in}}%
\pgfpathlineto{\pgfqpoint{1.582843in}{1.371882in}}%
\pgfpathlineto{\pgfqpoint{1.580741in}{1.377119in}}%
\pgfpathlineto{\pgfqpoint{1.578637in}{1.382398in}}%
\pgfpathlineto{\pgfqpoint{1.576532in}{1.387716in}}%
\pgfpathlineto{\pgfqpoint{1.574424in}{1.393069in}}%
\pgfpathlineto{\pgfqpoint{1.566286in}{1.396480in}}%
\pgfpathlineto{\pgfqpoint{1.557946in}{1.399761in}}%
\pgfpathlineto{\pgfqpoint{1.549414in}{1.402908in}}%
\pgfpathlineto{\pgfqpoint{1.540697in}{1.405921in}}%
\pgfpathclose%
\pgfusepath{fill}%
\end{pgfscope}%
\begin{pgfscope}%
\pgfpathrectangle{\pgfqpoint{0.329460in}{0.284240in}}{\pgfqpoint{1.989680in}{1.989680in}}%
\pgfusepath{clip}%
\pgfsetbuttcap%
\pgfsetroundjoin%
\definecolor{currentfill}{rgb}{0.233603,0.313828,0.543914}%
\pgfsetfillcolor{currentfill}%
\pgfsetlinewidth{0.000000pt}%
\definecolor{currentstroke}{rgb}{0.000000,0.000000,0.000000}%
\pgfsetstrokecolor{currentstroke}%
\pgfsetdash{}{0pt}%
\pgfpathmoveto{\pgfqpoint{0.750418in}{1.275822in}}%
\pgfpathlineto{\pgfqpoint{0.746950in}{1.286260in}}%
\pgfpathlineto{\pgfqpoint{0.743466in}{1.297093in}}%
\pgfpathlineto{\pgfqpoint{0.739966in}{1.308328in}}%
\pgfpathlineto{\pgfqpoint{0.736450in}{1.319970in}}%
\pgfpathlineto{\pgfqpoint{0.743908in}{1.329587in}}%
\pgfpathlineto{\pgfqpoint{0.751937in}{1.339068in}}%
\pgfpathlineto{\pgfqpoint{0.760527in}{1.348404in}}%
\pgfpathlineto{\pgfqpoint{0.769668in}{1.357587in}}%
\pgfpathlineto{\pgfqpoint{0.772974in}{1.345802in}}%
\pgfpathlineto{\pgfqpoint{0.776265in}{1.334422in}}%
\pgfpathlineto{\pgfqpoint{0.779541in}{1.323441in}}%
\pgfpathlineto{\pgfqpoint{0.782803in}{1.312852in}}%
\pgfpathlineto{\pgfqpoint{0.773888in}{1.303811in}}%
\pgfpathlineto{\pgfqpoint{0.765513in}{1.294621in}}%
\pgfpathlineto{\pgfqpoint{0.757686in}{1.285288in}}%
\pgfpathlineto{\pgfqpoint{0.750418in}{1.275822in}}%
\pgfpathclose%
\pgfusepath{fill}%
\end{pgfscope}%
\begin{pgfscope}%
\pgfpathrectangle{\pgfqpoint{0.329460in}{0.284240in}}{\pgfqpoint{1.989680in}{1.989680in}}%
\pgfusepath{clip}%
\pgfsetbuttcap%
\pgfsetroundjoin%
\definecolor{currentfill}{rgb}{0.277941,0.056324,0.381191}%
\pgfsetfillcolor{currentfill}%
\pgfsetlinewidth{0.000000pt}%
\definecolor{currentstroke}{rgb}{0.000000,0.000000,0.000000}%
\pgfsetstrokecolor{currentstroke}%
\pgfsetdash{}{0pt}%
\pgfpathmoveto{\pgfqpoint{0.796724in}{1.105269in}}%
\pgfpathlineto{\pgfqpoint{0.793352in}{1.108851in}}%
\pgfpathlineto{\pgfqpoint{0.789971in}{1.112718in}}%
\pgfpathlineto{\pgfqpoint{0.786580in}{1.116872in}}%
\pgfpathlineto{\pgfqpoint{0.783179in}{1.121320in}}%
\pgfpathlineto{\pgfqpoint{0.787597in}{1.130512in}}%
\pgfpathlineto{\pgfqpoint{0.792559in}{1.139615in}}%
\pgfpathlineto{\pgfqpoint{0.798056in}{1.148620in}}%
\pgfpathlineto{\pgfqpoint{0.804081in}{1.157519in}}%
\pgfpathlineto{\pgfqpoint{0.807339in}{1.152881in}}%
\pgfpathlineto{\pgfqpoint{0.810587in}{1.148534in}}%
\pgfpathlineto{\pgfqpoint{0.813826in}{1.144475in}}%
\pgfpathlineto{\pgfqpoint{0.817056in}{1.140698in}}%
\pgfpathlineto{\pgfqpoint{0.811191in}{1.131987in}}%
\pgfpathlineto{\pgfqpoint{0.805843in}{1.123173in}}%
\pgfpathlineto{\pgfqpoint{0.801018in}{1.114264in}}%
\pgfpathlineto{\pgfqpoint{0.796724in}{1.105269in}}%
\pgfpathclose%
\pgfusepath{fill}%
\end{pgfscope}%
\begin{pgfscope}%
\pgfpathrectangle{\pgfqpoint{0.329460in}{0.284240in}}{\pgfqpoint{1.989680in}{1.989680in}}%
\pgfusepath{clip}%
\pgfsetbuttcap%
\pgfsetroundjoin%
\definecolor{currentfill}{rgb}{0.212395,0.359683,0.551710}%
\pgfsetfillcolor{currentfill}%
\pgfsetlinewidth{0.000000pt}%
\definecolor{currentstroke}{rgb}{0.000000,0.000000,0.000000}%
\pgfsetstrokecolor{currentstroke}%
\pgfsetdash{}{0pt}%
\pgfpathmoveto{\pgfqpoint{1.582843in}{1.371882in}}%
\pgfpathlineto{\pgfqpoint{1.584943in}{1.366689in}}%
\pgfpathlineto{\pgfqpoint{1.587041in}{1.361542in}}%
\pgfpathlineto{\pgfqpoint{1.589137in}{1.356446in}}%
\pgfpathlineto{\pgfqpoint{1.591232in}{1.351402in}}%
\pgfpathlineto{\pgfqpoint{1.599770in}{1.347571in}}%
\pgfpathlineto{\pgfqpoint{1.608075in}{1.343607in}}%
\pgfpathlineto{\pgfqpoint{1.616140in}{1.339511in}}%
\pgfpathlineto{\pgfqpoint{1.623957in}{1.335288in}}%
\pgfpathlineto{\pgfqpoint{1.621571in}{1.340486in}}%
\pgfpathlineto{\pgfqpoint{1.619182in}{1.345736in}}%
\pgfpathlineto{\pgfqpoint{1.616792in}{1.351037in}}%
\pgfpathlineto{\pgfqpoint{1.614400in}{1.356384in}}%
\pgfpathlineto{\pgfqpoint{1.606863in}{1.360446in}}%
\pgfpathlineto{\pgfqpoint{1.599086in}{1.364385in}}%
\pgfpathlineto{\pgfqpoint{1.591076in}{1.368198in}}%
\pgfpathlineto{\pgfqpoint{1.582843in}{1.371882in}}%
\pgfpathclose%
\pgfusepath{fill}%
\end{pgfscope}%
\begin{pgfscope}%
\pgfpathrectangle{\pgfqpoint{0.329460in}{0.284240in}}{\pgfqpoint{1.989680in}{1.989680in}}%
\pgfusepath{clip}%
\pgfsetbuttcap%
\pgfsetroundjoin%
\definecolor{currentfill}{rgb}{0.268510,0.009605,0.335427}%
\pgfsetfillcolor{currentfill}%
\pgfsetlinewidth{0.000000pt}%
\definecolor{currentstroke}{rgb}{0.000000,0.000000,0.000000}%
\pgfsetstrokecolor{currentstroke}%
\pgfsetdash{}{0pt}%
\pgfpathmoveto{\pgfqpoint{0.862594in}{1.085983in}}%
\pgfpathlineto{\pgfqpoint{0.859355in}{1.084857in}}%
\pgfpathlineto{\pgfqpoint{0.856112in}{1.083926in}}%
\pgfpathlineto{\pgfqpoint{0.852865in}{1.083194in}}%
\pgfpathlineto{\pgfqpoint{0.849612in}{1.082665in}}%
\pgfpathlineto{\pgfqpoint{0.853431in}{1.090832in}}%
\pgfpathlineto{\pgfqpoint{0.857730in}{1.098923in}}%
\pgfpathlineto{\pgfqpoint{0.862506in}{1.106930in}}%
\pgfpathlineto{\pgfqpoint{0.867750in}{1.114845in}}%
\pgfpathlineto{\pgfqpoint{0.870869in}{1.115164in}}%
\pgfpathlineto{\pgfqpoint{0.873984in}{1.115685in}}%
\pgfpathlineto{\pgfqpoint{0.877095in}{1.116404in}}%
\pgfpathlineto{\pgfqpoint{0.880201in}{1.117318in}}%
\pgfpathlineto{\pgfqpoint{0.875107in}{1.109610in}}%
\pgfpathlineto{\pgfqpoint{0.870470in}{1.101814in}}%
\pgfpathlineto{\pgfqpoint{0.866297in}{1.093936in}}%
\pgfpathlineto{\pgfqpoint{0.862594in}{1.085983in}}%
\pgfpathclose%
\pgfusepath{fill}%
\end{pgfscope}%
\begin{pgfscope}%
\pgfpathrectangle{\pgfqpoint{0.329460in}{0.284240in}}{\pgfqpoint{1.989680in}{1.989680in}}%
\pgfusepath{clip}%
\pgfsetbuttcap%
\pgfsetroundjoin%
\definecolor{currentfill}{rgb}{0.163625,0.471133,0.558148}%
\pgfsetfillcolor{currentfill}%
\pgfsetlinewidth{0.000000pt}%
\definecolor{currentstroke}{rgb}{0.000000,0.000000,0.000000}%
\pgfsetstrokecolor{currentstroke}%
\pgfsetdash{}{0pt}%
\pgfpathmoveto{\pgfqpoint{1.428775in}{1.470263in}}%
\pgfpathlineto{\pgfqpoint{1.429570in}{1.465092in}}%
\pgfpathlineto{\pgfqpoint{1.430363in}{1.459936in}}%
\pgfpathlineto{\pgfqpoint{1.431156in}{1.454797in}}%
\pgfpathlineto{\pgfqpoint{1.431948in}{1.449678in}}%
\pgfpathlineto{\pgfqpoint{1.441734in}{1.448357in}}%
\pgfpathlineto{\pgfqpoint{1.451437in}{1.446885in}}%
\pgfpathlineto{\pgfqpoint{1.461047in}{1.445264in}}%
\pgfpathlineto{\pgfqpoint{1.459969in}{1.450430in}}%
\pgfpathlineto{\pgfqpoint{1.458890in}{1.455616in}}%
\pgfpathlineto{\pgfqpoint{1.457809in}{1.460819in}}%
\pgfpathlineto{\pgfqpoint{1.456727in}{1.466037in}}%
\pgfpathlineto{\pgfqpoint{1.447496in}{1.467589in}}%
\pgfpathlineto{\pgfqpoint{1.438176in}{1.468998in}}%
\pgfpathlineto{\pgfqpoint{1.428775in}{1.470263in}}%
\pgfpathclose%
\pgfusepath{fill}%
\end{pgfscope}%
\begin{pgfscope}%
\pgfpathrectangle{\pgfqpoint{0.329460in}{0.284240in}}{\pgfqpoint{1.989680in}{1.989680in}}%
\pgfusepath{clip}%
\pgfsetbuttcap%
\pgfsetroundjoin%
\definecolor{currentfill}{rgb}{0.179019,0.433756,0.557430}%
\pgfsetfillcolor{currentfill}%
\pgfsetlinewidth{0.000000pt}%
\definecolor{currentstroke}{rgb}{0.000000,0.000000,0.000000}%
\pgfsetstrokecolor{currentstroke}%
\pgfsetdash{}{0pt}%
\pgfpathmoveto{\pgfqpoint{1.498384in}{1.437315in}}%
\pgfpathlineto{\pgfqpoint{1.499828in}{1.432087in}}%
\pgfpathlineto{\pgfqpoint{1.501271in}{1.426884in}}%
\pgfpathlineto{\pgfqpoint{1.502713in}{1.421709in}}%
\pgfpathlineto{\pgfqpoint{1.504153in}{1.416564in}}%
\pgfpathlineto{\pgfqpoint{1.513523in}{1.414119in}}%
\pgfpathlineto{\pgfqpoint{1.522743in}{1.411528in}}%
\pgfpathlineto{\pgfqpoint{1.531804in}{1.408795in}}%
\pgfpathlineto{\pgfqpoint{1.540697in}{1.405921in}}%
\pgfpathlineto{\pgfqpoint{1.538910in}{1.411175in}}%
\pgfpathlineto{\pgfqpoint{1.537121in}{1.416459in}}%
\pgfpathlineto{\pgfqpoint{1.535330in}{1.421772in}}%
\pgfpathlineto{\pgfqpoint{1.533537in}{1.427109in}}%
\pgfpathlineto{\pgfqpoint{1.524983in}{1.429865in}}%
\pgfpathlineto{\pgfqpoint{1.516268in}{1.432486in}}%
\pgfpathlineto{\pgfqpoint{1.507398in}{1.434970in}}%
\pgfpathlineto{\pgfqpoint{1.498384in}{1.437315in}}%
\pgfpathclose%
\pgfusepath{fill}%
\end{pgfscope}%
\begin{pgfscope}%
\pgfpathrectangle{\pgfqpoint{0.329460in}{0.284240in}}{\pgfqpoint{1.989680in}{1.989680in}}%
\pgfusepath{clip}%
\pgfsetbuttcap%
\pgfsetroundjoin%
\definecolor{currentfill}{rgb}{0.163625,0.471133,0.558148}%
\pgfsetfillcolor{currentfill}%
\pgfsetlinewidth{0.000000pt}%
\definecolor{currentstroke}{rgb}{0.000000,0.000000,0.000000}%
\pgfsetstrokecolor{currentstroke}%
\pgfsetdash{}{0pt}%
\pgfpathmoveto{\pgfqpoint{1.237524in}{1.464538in}}%
\pgfpathlineto{\pgfqpoint{1.236359in}{1.459304in}}%
\pgfpathlineto{\pgfqpoint{1.235195in}{1.454084in}}%
\pgfpathlineto{\pgfqpoint{1.234032in}{1.448881in}}%
\pgfpathlineto{\pgfqpoint{1.232871in}{1.443699in}}%
\pgfpathlineto{\pgfqpoint{1.242391in}{1.445452in}}%
\pgfpathlineto{\pgfqpoint{1.252012in}{1.447056in}}%
\pgfpathlineto{\pgfqpoint{1.261724in}{1.448511in}}%
\pgfpathlineto{\pgfqpoint{1.271519in}{1.449816in}}%
\pgfpathlineto{\pgfqpoint{1.272300in}{1.454933in}}%
\pgfpathlineto{\pgfqpoint{1.273082in}{1.460071in}}%
\pgfpathlineto{\pgfqpoint{1.273865in}{1.465225in}}%
\pgfpathlineto{\pgfqpoint{1.274649in}{1.470394in}}%
\pgfpathlineto{\pgfqpoint{1.265240in}{1.469145in}}%
\pgfpathlineto{\pgfqpoint{1.255910in}{1.467752in}}%
\pgfpathlineto{\pgfqpoint{1.246669in}{1.466216in}}%
\pgfpathlineto{\pgfqpoint{1.237524in}{1.464538in}}%
\pgfpathclose%
\pgfusepath{fill}%
\end{pgfscope}%
\begin{pgfscope}%
\pgfpathrectangle{\pgfqpoint{0.329460in}{0.284240in}}{\pgfqpoint{1.989680in}{1.989680in}}%
\pgfusepath{clip}%
\pgfsetbuttcap%
\pgfsetroundjoin%
\definecolor{currentfill}{rgb}{0.282884,0.135920,0.453427}%
\pgfsetfillcolor{currentfill}%
\pgfsetlinewidth{0.000000pt}%
\definecolor{currentstroke}{rgb}{0.000000,0.000000,0.000000}%
\pgfsetstrokecolor{currentstroke}%
\pgfsetdash{}{0pt}%
\pgfpathmoveto{\pgfqpoint{1.905476in}{1.187074in}}%
\pgfpathlineto{\pgfqpoint{1.908746in}{1.193290in}}%
\pgfpathlineto{\pgfqpoint{1.912027in}{1.199830in}}%
\pgfpathlineto{\pgfqpoint{1.915320in}{1.206698in}}%
\pgfpathlineto{\pgfqpoint{1.918624in}{1.213901in}}%
\pgfpathlineto{\pgfqpoint{1.925464in}{1.204748in}}%
\pgfpathlineto{\pgfqpoint{1.931758in}{1.195476in}}%
\pgfpathlineto{\pgfqpoint{1.937500in}{1.186093in}}%
\pgfpathlineto{\pgfqpoint{1.942680in}{1.176609in}}%
\pgfpathlineto{\pgfqpoint{1.939215in}{1.169583in}}%
\pgfpathlineto{\pgfqpoint{1.935763in}{1.162892in}}%
\pgfpathlineto{\pgfqpoint{1.932323in}{1.156532in}}%
\pgfpathlineto{\pgfqpoint{1.928895in}{1.150497in}}%
\pgfpathlineto{\pgfqpoint{1.923856in}{1.159798in}}%
\pgfpathlineto{\pgfqpoint{1.918268in}{1.169001in}}%
\pgfpathlineto{\pgfqpoint{1.912138in}{1.178095in}}%
\pgfpathlineto{\pgfqpoint{1.905476in}{1.187074in}}%
\pgfpathclose%
\pgfusepath{fill}%
\end{pgfscope}%
\begin{pgfscope}%
\pgfpathrectangle{\pgfqpoint{0.329460in}{0.284240in}}{\pgfqpoint{1.989680in}{1.989680in}}%
\pgfusepath{clip}%
\pgfsetbuttcap%
\pgfsetroundjoin%
\definecolor{currentfill}{rgb}{0.283072,0.130895,0.449241}%
\pgfsetfillcolor{currentfill}%
\pgfsetlinewidth{0.000000pt}%
\definecolor{currentstroke}{rgb}{0.000000,0.000000,0.000000}%
\pgfsetstrokecolor{currentstroke}%
\pgfsetdash{}{0pt}%
\pgfpathmoveto{\pgfqpoint{0.941768in}{1.170546in}}%
\pgfpathlineto{\pgfqpoint{0.938705in}{1.166534in}}%
\pgfpathlineto{\pgfqpoint{0.935640in}{1.162643in}}%
\pgfpathlineto{\pgfqpoint{0.932576in}{1.158876in}}%
\pgfpathlineto{\pgfqpoint{0.929510in}{1.155238in}}%
\pgfpathlineto{\pgfqpoint{0.934423in}{1.162013in}}%
\pgfpathlineto{\pgfqpoint{0.939735in}{1.168699in}}%
\pgfpathlineto{\pgfqpoint{0.945441in}{1.175290in}}%
\pgfpathlineto{\pgfqpoint{0.951533in}{1.181778in}}%
\pgfpathlineto{\pgfqpoint{0.954427in}{1.185209in}}%
\pgfpathlineto{\pgfqpoint{0.957320in}{1.188768in}}%
\pgfpathlineto{\pgfqpoint{0.960212in}{1.192452in}}%
\pgfpathlineto{\pgfqpoint{0.963104in}{1.196256in}}%
\pgfpathlineto{\pgfqpoint{0.957200in}{1.189970in}}%
\pgfpathlineto{\pgfqpoint{0.951672in}{1.183586in}}%
\pgfpathlineto{\pgfqpoint{0.946525in}{1.177109in}}%
\pgfpathlineto{\pgfqpoint{0.941768in}{1.170546in}}%
\pgfpathclose%
\pgfusepath{fill}%
\end{pgfscope}%
\begin{pgfscope}%
\pgfpathrectangle{\pgfqpoint{0.329460in}{0.284240in}}{\pgfqpoint{1.989680in}{1.989680in}}%
\pgfusepath{clip}%
\pgfsetbuttcap%
\pgfsetroundjoin%
\definecolor{currentfill}{rgb}{0.263663,0.237631,0.518762}%
\pgfsetfillcolor{currentfill}%
\pgfsetlinewidth{0.000000pt}%
\definecolor{currentstroke}{rgb}{0.000000,0.000000,0.000000}%
\pgfsetstrokecolor{currentstroke}%
\pgfsetdash{}{0pt}%
\pgfpathmoveto{\pgfqpoint{1.673691in}{1.276464in}}%
\pgfpathlineto{\pgfqpoint{1.676323in}{1.271698in}}%
\pgfpathlineto{\pgfqpoint{1.678953in}{1.267014in}}%
\pgfpathlineto{\pgfqpoint{1.681583in}{1.262415in}}%
\pgfpathlineto{\pgfqpoint{1.684212in}{1.257905in}}%
\pgfpathlineto{\pgfqpoint{1.691380in}{1.252547in}}%
\pgfpathlineto{\pgfqpoint{1.698227in}{1.247073in}}%
\pgfpathlineto{\pgfqpoint{1.704745in}{1.241488in}}%
\pgfpathlineto{\pgfqpoint{1.710928in}{1.235796in}}%
\pgfpathlineto{\pgfqpoint{1.708078in}{1.240497in}}%
\pgfpathlineto{\pgfqpoint{1.705228in}{1.245287in}}%
\pgfpathlineto{\pgfqpoint{1.702376in}{1.250162in}}%
\pgfpathlineto{\pgfqpoint{1.699524in}{1.255119in}}%
\pgfpathlineto{\pgfqpoint{1.693547in}{1.260614in}}%
\pgfpathlineto{\pgfqpoint{1.687245in}{1.266006in}}%
\pgfpathlineto{\pgfqpoint{1.680624in}{1.271291in}}%
\pgfpathlineto{\pgfqpoint{1.673691in}{1.276464in}}%
\pgfpathclose%
\pgfusepath{fill}%
\end{pgfscope}%
\begin{pgfscope}%
\pgfpathrectangle{\pgfqpoint{0.329460in}{0.284240in}}{\pgfqpoint{1.989680in}{1.989680in}}%
\pgfusepath{clip}%
\pgfsetbuttcap%
\pgfsetroundjoin%
\definecolor{currentfill}{rgb}{0.274952,0.037752,0.364543}%
\pgfsetfillcolor{currentfill}%
\pgfsetlinewidth{0.000000pt}%
\definecolor{currentstroke}{rgb}{0.000000,0.000000,0.000000}%
\pgfsetstrokecolor{currentstroke}%
\pgfsetdash{}{0pt}%
\pgfpathmoveto{\pgfqpoint{1.792819in}{1.137558in}}%
\pgfpathlineto{\pgfqpoint{1.795865in}{1.135276in}}%
\pgfpathlineto{\pgfqpoint{1.798914in}{1.133157in}}%
\pgfpathlineto{\pgfqpoint{1.801965in}{1.131206in}}%
\pgfpathlineto{\pgfqpoint{1.805019in}{1.129426in}}%
\pgfpathlineto{\pgfqpoint{1.810356in}{1.122009in}}%
\pgfpathlineto{\pgfqpoint{1.815252in}{1.114501in}}%
\pgfpathlineto{\pgfqpoint{1.819702in}{1.106907in}}%
\pgfpathlineto{\pgfqpoint{1.823699in}{1.099236in}}%
\pgfpathlineto{\pgfqpoint{1.820504in}{1.101229in}}%
\pgfpathlineto{\pgfqpoint{1.817313in}{1.103393in}}%
\pgfpathlineto{\pgfqpoint{1.814124in}{1.105725in}}%
\pgfpathlineto{\pgfqpoint{1.810938in}{1.108221in}}%
\pgfpathlineto{\pgfqpoint{1.807064in}{1.115675in}}%
\pgfpathlineto{\pgfqpoint{1.802749in}{1.123054in}}%
\pgfpathlineto{\pgfqpoint{1.797998in}{1.130350in}}%
\pgfpathlineto{\pgfqpoint{1.792819in}{1.137558in}}%
\pgfpathclose%
\pgfusepath{fill}%
\end{pgfscope}%
\begin{pgfscope}%
\pgfpathrectangle{\pgfqpoint{0.329460in}{0.284240in}}{\pgfqpoint{1.989680in}{1.989680in}}%
\pgfusepath{clip}%
\pgfsetbuttcap%
\pgfsetroundjoin%
\definecolor{currentfill}{rgb}{0.195860,0.395433,0.555276}%
\pgfsetfillcolor{currentfill}%
\pgfsetlinewidth{0.000000pt}%
\definecolor{currentstroke}{rgb}{0.000000,0.000000,0.000000}%
\pgfsetstrokecolor{currentstroke}%
\pgfsetdash{}{0pt}%
\pgfpathmoveto{\pgfqpoint{1.120892in}{1.389931in}}%
\pgfpathlineto{\pgfqpoint{1.118717in}{1.384545in}}%
\pgfpathlineto{\pgfqpoint{1.116543in}{1.379195in}}%
\pgfpathlineto{\pgfqpoint{1.114372in}{1.373884in}}%
\pgfpathlineto{\pgfqpoint{1.112203in}{1.368614in}}%
\pgfpathlineto{\pgfqpoint{1.120461in}{1.372283in}}%
\pgfpathlineto{\pgfqpoint{1.128934in}{1.375820in}}%
\pgfpathlineto{\pgfqpoint{1.137615in}{1.379221in}}%
\pgfpathlineto{\pgfqpoint{1.146494in}{1.382483in}}%
\pgfpathlineto{\pgfqpoint{1.148348in}{1.387615in}}%
\pgfpathlineto{\pgfqpoint{1.150204in}{1.392789in}}%
\pgfpathlineto{\pgfqpoint{1.152062in}{1.398002in}}%
\pgfpathlineto{\pgfqpoint{1.153921in}{1.403250in}}%
\pgfpathlineto{\pgfqpoint{1.145368in}{1.400117in}}%
\pgfpathlineto{\pgfqpoint{1.137006in}{1.396851in}}%
\pgfpathlineto{\pgfqpoint{1.128845in}{1.393454in}}%
\pgfpathlineto{\pgfqpoint{1.120892in}{1.389931in}}%
\pgfpathclose%
\pgfusepath{fill}%
\end{pgfscope}%
\begin{pgfscope}%
\pgfpathrectangle{\pgfqpoint{0.329460in}{0.284240in}}{\pgfqpoint{1.989680in}{1.989680in}}%
\pgfusepath{clip}%
\pgfsetbuttcap%
\pgfsetroundjoin%
\definecolor{currentfill}{rgb}{0.212395,0.359683,0.551710}%
\pgfsetfillcolor{currentfill}%
\pgfsetlinewidth{0.000000pt}%
\definecolor{currentstroke}{rgb}{0.000000,0.000000,0.000000}%
\pgfsetstrokecolor{currentstroke}%
\pgfsetdash{}{0pt}%
\pgfpathmoveto{\pgfqpoint{1.081483in}{1.352674in}}%
\pgfpathlineto{\pgfqpoint{1.079030in}{1.347290in}}%
\pgfpathlineto{\pgfqpoint{1.076580in}{1.341952in}}%
\pgfpathlineto{\pgfqpoint{1.074131in}{1.336664in}}%
\pgfpathlineto{\pgfqpoint{1.071684in}{1.331429in}}%
\pgfpathlineto{\pgfqpoint{1.079274in}{1.335763in}}%
\pgfpathlineto{\pgfqpoint{1.087119in}{1.339972in}}%
\pgfpathlineto{\pgfqpoint{1.095211in}{1.344054in}}%
\pgfpathlineto{\pgfqpoint{1.103543in}{1.348004in}}%
\pgfpathlineto{\pgfqpoint{1.105705in}{1.353080in}}%
\pgfpathlineto{\pgfqpoint{1.107869in}{1.358209in}}%
\pgfpathlineto{\pgfqpoint{1.110035in}{1.363388in}}%
\pgfpathlineto{\pgfqpoint{1.112203in}{1.368614in}}%
\pgfpathlineto{\pgfqpoint{1.104168in}{1.364815in}}%
\pgfpathlineto{\pgfqpoint{1.096365in}{1.360889in}}%
\pgfpathlineto{\pgfqpoint{1.088800in}{1.356841in}}%
\pgfpathlineto{\pgfqpoint{1.081483in}{1.352674in}}%
\pgfpathclose%
\pgfusepath{fill}%
\end{pgfscope}%
\begin{pgfscope}%
\pgfpathrectangle{\pgfqpoint{0.329460in}{0.284240in}}{\pgfqpoint{1.989680in}{1.989680in}}%
\pgfusepath{clip}%
\pgfsetbuttcap%
\pgfsetroundjoin%
\definecolor{currentfill}{rgb}{0.282327,0.094955,0.417331}%
\pgfsetfillcolor{currentfill}%
\pgfsetlinewidth{0.000000pt}%
\definecolor{currentstroke}{rgb}{0.000000,0.000000,0.000000}%
\pgfsetstrokecolor{currentstroke}%
\pgfsetdash{}{0pt}%
\pgfpathmoveto{\pgfqpoint{0.783179in}{1.121320in}}%
\pgfpathlineto{\pgfqpoint{0.779767in}{1.126067in}}%
\pgfpathlineto{\pgfqpoint{0.776345in}{1.131118in}}%
\pgfpathlineto{\pgfqpoint{0.772912in}{1.136478in}}%
\pgfpathlineto{\pgfqpoint{0.769467in}{1.142152in}}%
\pgfpathlineto{\pgfqpoint{0.774013in}{1.151535in}}%
\pgfpathlineto{\pgfqpoint{0.779113in}{1.160826in}}%
\pgfpathlineto{\pgfqpoint{0.784762in}{1.170017in}}%
\pgfpathlineto{\pgfqpoint{0.790951in}{1.179099in}}%
\pgfpathlineto{\pgfqpoint{0.794249in}{1.173240in}}%
\pgfpathlineto{\pgfqpoint{0.797537in}{1.167694in}}%
\pgfpathlineto{\pgfqpoint{0.800814in}{1.162455in}}%
\pgfpathlineto{\pgfqpoint{0.804081in}{1.157519in}}%
\pgfpathlineto{\pgfqpoint{0.798056in}{1.148620in}}%
\pgfpathlineto{\pgfqpoint{0.792559in}{1.139615in}}%
\pgfpathlineto{\pgfqpoint{0.787597in}{1.130512in}}%
\pgfpathlineto{\pgfqpoint{0.783179in}{1.121320in}}%
\pgfpathclose%
\pgfusepath{fill}%
\end{pgfscope}%
\begin{pgfscope}%
\pgfpathrectangle{\pgfqpoint{0.329460in}{0.284240in}}{\pgfqpoint{1.989680in}{1.989680in}}%
\pgfusepath{clip}%
\pgfsetbuttcap%
\pgfsetroundjoin%
\definecolor{currentfill}{rgb}{0.201239,0.383670,0.554294}%
\pgfsetfillcolor{currentfill}%
\pgfsetlinewidth{0.000000pt}%
\definecolor{currentstroke}{rgb}{0.000000,0.000000,0.000000}%
\pgfsetstrokecolor{currentstroke}%
\pgfsetdash{}{0pt}%
\pgfpathmoveto{\pgfqpoint{1.924130in}{1.365615in}}%
\pgfpathlineto{\pgfqpoint{1.927399in}{1.377843in}}%
\pgfpathlineto{\pgfqpoint{1.930683in}{1.390489in}}%
\pgfpathlineto{\pgfqpoint{1.933984in}{1.403560in}}%
\pgfpathlineto{\pgfqpoint{1.937302in}{1.417064in}}%
\pgfpathlineto{\pgfqpoint{1.947163in}{1.407893in}}%
\pgfpathlineto{\pgfqpoint{1.956473in}{1.398558in}}%
\pgfpathlineto{\pgfqpoint{1.965219in}{1.389069in}}%
\pgfpathlineto{\pgfqpoint{1.973392in}{1.379433in}}%
\pgfpathlineto{\pgfqpoint{1.969848in}{1.366061in}}%
\pgfpathlineto{\pgfqpoint{1.966323in}{1.353124in}}%
\pgfpathlineto{\pgfqpoint{1.962815in}{1.340614in}}%
\pgfpathlineto{\pgfqpoint{1.959324in}{1.328525in}}%
\pgfpathlineto{\pgfqpoint{1.951358in}{1.338022in}}%
\pgfpathlineto{\pgfqpoint{1.942830in}{1.347374in}}%
\pgfpathlineto{\pgfqpoint{1.933750in}{1.356575in}}%
\pgfpathlineto{\pgfqpoint{1.924130in}{1.365615in}}%
\pgfpathclose%
\pgfusepath{fill}%
\end{pgfscope}%
\begin{pgfscope}%
\pgfpathrectangle{\pgfqpoint{0.329460in}{0.284240in}}{\pgfqpoint{1.989680in}{1.989680in}}%
\pgfusepath{clip}%
\pgfsetbuttcap%
\pgfsetroundjoin%
\definecolor{currentfill}{rgb}{0.179019,0.433756,0.557430}%
\pgfsetfillcolor{currentfill}%
\pgfsetlinewidth{0.000000pt}%
\definecolor{currentstroke}{rgb}{0.000000,0.000000,0.000000}%
\pgfsetstrokecolor{currentstroke}%
\pgfsetdash{}{0pt}%
\pgfpathmoveto{\pgfqpoint{1.161377in}{1.424548in}}%
\pgfpathlineto{\pgfqpoint{1.159510in}{1.419183in}}%
\pgfpathlineto{\pgfqpoint{1.157645in}{1.413843in}}%
\pgfpathlineto{\pgfqpoint{1.155782in}{1.408531in}}%
\pgfpathlineto{\pgfqpoint{1.153921in}{1.403250in}}%
\pgfpathlineto{\pgfqpoint{1.162658in}{1.406247in}}%
\pgfpathlineto{\pgfqpoint{1.171570in}{1.409106in}}%
\pgfpathlineto{\pgfqpoint{1.180649in}{1.411823in}}%
\pgfpathlineto{\pgfqpoint{1.189886in}{1.414398in}}%
\pgfpathlineto{\pgfqpoint{1.191405in}{1.419565in}}%
\pgfpathlineto{\pgfqpoint{1.192925in}{1.424762in}}%
\pgfpathlineto{\pgfqpoint{1.194448in}{1.429987in}}%
\pgfpathlineto{\pgfqpoint{1.195972in}{1.435238in}}%
\pgfpathlineto{\pgfqpoint{1.187086in}{1.432769in}}%
\pgfpathlineto{\pgfqpoint{1.178352in}{1.430163in}}%
\pgfpathlineto{\pgfqpoint{1.169780in}{1.427422in}}%
\pgfpathlineto{\pgfqpoint{1.161377in}{1.424548in}}%
\pgfpathclose%
\pgfusepath{fill}%
\end{pgfscope}%
\begin{pgfscope}%
\pgfpathrectangle{\pgfqpoint{0.329460in}{0.284240in}}{\pgfqpoint{1.989680in}{1.989680in}}%
\pgfusepath{clip}%
\pgfsetbuttcap%
\pgfsetroundjoin%
\definecolor{currentfill}{rgb}{0.271305,0.019942,0.347269}%
\pgfsetfillcolor{currentfill}%
\pgfsetlinewidth{0.000000pt}%
\definecolor{currentstroke}{rgb}{0.000000,0.000000,0.000000}%
\pgfsetstrokecolor{currentstroke}%
\pgfsetdash{}{0pt}%
\pgfpathmoveto{\pgfqpoint{0.875508in}{1.092358in}}%
\pgfpathlineto{\pgfqpoint{0.872285in}{1.090491in}}%
\pgfpathlineto{\pgfqpoint{0.869058in}{1.088804in}}%
\pgfpathlineto{\pgfqpoint{0.865828in}{1.087300in}}%
\pgfpathlineto{\pgfqpoint{0.862594in}{1.085983in}}%
\pgfpathlineto{\pgfqpoint{0.866297in}{1.093936in}}%
\pgfpathlineto{\pgfqpoint{0.870470in}{1.101814in}}%
\pgfpathlineto{\pgfqpoint{0.875107in}{1.109610in}}%
\pgfpathlineto{\pgfqpoint{0.880201in}{1.117318in}}%
\pgfpathlineto{\pgfqpoint{0.883304in}{1.118422in}}%
\pgfpathlineto{\pgfqpoint{0.886403in}{1.119713in}}%
\pgfpathlineto{\pgfqpoint{0.889498in}{1.121186in}}%
\pgfpathlineto{\pgfqpoint{0.892591in}{1.122838in}}%
\pgfpathlineto{\pgfqpoint{0.887645in}{1.115339in}}%
\pgfpathlineto{\pgfqpoint{0.883146in}{1.107755in}}%
\pgfpathlineto{\pgfqpoint{0.879098in}{1.100092in}}%
\pgfpathlineto{\pgfqpoint{0.875508in}{1.092358in}}%
\pgfpathclose%
\pgfusepath{fill}%
\end{pgfscope}%
\begin{pgfscope}%
\pgfpathrectangle{\pgfqpoint{0.329460in}{0.284240in}}{\pgfqpoint{1.989680in}{1.989680in}}%
\pgfusepath{clip}%
\pgfsetbuttcap%
\pgfsetroundjoin%
\definecolor{currentfill}{rgb}{0.280255,0.165693,0.476498}%
\pgfsetfillcolor{currentfill}%
\pgfsetlinewidth{0.000000pt}%
\definecolor{currentstroke}{rgb}{0.000000,0.000000,0.000000}%
\pgfsetstrokecolor{currentstroke}%
\pgfsetdash{}{0pt}%
\pgfpathmoveto{\pgfqpoint{1.722322in}{1.217940in}}%
\pgfpathlineto{\pgfqpoint{1.725169in}{1.213729in}}%
\pgfpathlineto{\pgfqpoint{1.728017in}{1.209626in}}%
\pgfpathlineto{\pgfqpoint{1.730865in}{1.205634in}}%
\pgfpathlineto{\pgfqpoint{1.733712in}{1.201757in}}%
\pgfpathlineto{\pgfqpoint{1.739945in}{1.195563in}}%
\pgfpathlineto{\pgfqpoint{1.745808in}{1.189266in}}%
\pgfpathlineto{\pgfqpoint{1.751294in}{1.182871in}}%
\pgfpathlineto{\pgfqpoint{1.756398in}{1.176384in}}%
\pgfpathlineto{\pgfqpoint{1.753369in}{1.180467in}}%
\pgfpathlineto{\pgfqpoint{1.750341in}{1.184665in}}%
\pgfpathlineto{\pgfqpoint{1.747313in}{1.188973in}}%
\pgfpathlineto{\pgfqpoint{1.744285in}{1.193389in}}%
\pgfpathlineto{\pgfqpoint{1.739346in}{1.199665in}}%
\pgfpathlineto{\pgfqpoint{1.734035in}{1.205853in}}%
\pgfpathlineto{\pgfqpoint{1.728358in}{1.211946in}}%
\pgfpathlineto{\pgfqpoint{1.722322in}{1.217940in}}%
\pgfpathclose%
\pgfusepath{fill}%
\end{pgfscope}%
\begin{pgfscope}%
\pgfpathrectangle{\pgfqpoint{0.329460in}{0.284240in}}{\pgfqpoint{1.989680in}{1.989680in}}%
\pgfusepath{clip}%
\pgfsetbuttcap%
\pgfsetroundjoin%
\definecolor{currentfill}{rgb}{0.231674,0.318106,0.544834}%
\pgfsetfillcolor{currentfill}%
\pgfsetlinewidth{0.000000pt}%
\definecolor{currentstroke}{rgb}{0.000000,0.000000,0.000000}%
\pgfsetstrokecolor{currentstroke}%
\pgfsetdash{}{0pt}%
\pgfpathmoveto{\pgfqpoint{1.623957in}{1.335288in}}%
\pgfpathlineto{\pgfqpoint{1.626342in}{1.330145in}}%
\pgfpathlineto{\pgfqpoint{1.628726in}{1.325061in}}%
\pgfpathlineto{\pgfqpoint{1.631107in}{1.320038in}}%
\pgfpathlineto{\pgfqpoint{1.633487in}{1.315079in}}%
\pgfpathlineto{\pgfqpoint{1.641320in}{1.310566in}}%
\pgfpathlineto{\pgfqpoint{1.648880in}{1.305927in}}%
\pgfpathlineto{\pgfqpoint{1.656160in}{1.301168in}}%
\pgfpathlineto{\pgfqpoint{1.663152in}{1.296292in}}%
\pgfpathlineto{\pgfqpoint{1.660514in}{1.301424in}}%
\pgfpathlineto{\pgfqpoint{1.657874in}{1.306621in}}%
\pgfpathlineto{\pgfqpoint{1.655233in}{1.311879in}}%
\pgfpathlineto{\pgfqpoint{1.652590in}{1.317195in}}%
\pgfpathlineto{\pgfqpoint{1.645842in}{1.321890in}}%
\pgfpathlineto{\pgfqpoint{1.638816in}{1.326473in}}%
\pgfpathlineto{\pgfqpoint{1.631519in}{1.330940in}}%
\pgfpathlineto{\pgfqpoint{1.623957in}{1.335288in}}%
\pgfpathclose%
\pgfusepath{fill}%
\end{pgfscope}%
\begin{pgfscope}%
\pgfpathrectangle{\pgfqpoint{0.329460in}{0.284240in}}{\pgfqpoint{1.989680in}{1.989680in}}%
\pgfusepath{clip}%
\pgfsetbuttcap%
\pgfsetroundjoin%
\definecolor{currentfill}{rgb}{0.263663,0.237631,0.518762}%
\pgfsetfillcolor{currentfill}%
\pgfsetlinewidth{0.000000pt}%
\definecolor{currentstroke}{rgb}{0.000000,0.000000,0.000000}%
\pgfsetstrokecolor{currentstroke}%
\pgfsetdash{}{0pt}%
\pgfpathmoveto{\pgfqpoint{0.997817in}{1.250153in}}%
\pgfpathlineto{\pgfqpoint{0.994921in}{1.245151in}}%
\pgfpathlineto{\pgfqpoint{0.992026in}{1.240231in}}%
\pgfpathlineto{\pgfqpoint{0.989132in}{1.235397in}}%
\pgfpathlineto{\pgfqpoint{0.986239in}{1.230651in}}%
\pgfpathlineto{\pgfqpoint{0.992117in}{1.236433in}}%
\pgfpathlineto{\pgfqpoint{0.998338in}{1.242114in}}%
\pgfpathlineto{\pgfqpoint{1.004893in}{1.247687in}}%
\pgfpathlineto{\pgfqpoint{1.011776in}{1.253148in}}%
\pgfpathlineto{\pgfqpoint{1.014457in}{1.257700in}}%
\pgfpathlineto{\pgfqpoint{1.017139in}{1.262339in}}%
\pgfpathlineto{\pgfqpoint{1.019822in}{1.267064in}}%
\pgfpathlineto{\pgfqpoint{1.022506in}{1.271872in}}%
\pgfpathlineto{\pgfqpoint{1.015850in}{1.266599in}}%
\pgfpathlineto{\pgfqpoint{1.009512in}{1.261218in}}%
\pgfpathlineto{\pgfqpoint{1.003499in}{1.255735in}}%
\pgfpathlineto{\pgfqpoint{0.997817in}{1.250153in}}%
\pgfpathclose%
\pgfusepath{fill}%
\end{pgfscope}%
\begin{pgfscope}%
\pgfpathrectangle{\pgfqpoint{0.329460in}{0.284240in}}{\pgfqpoint{1.989680in}{1.989680in}}%
\pgfusepath{clip}%
\pgfsetbuttcap%
\pgfsetroundjoin%
\definecolor{currentfill}{rgb}{0.163625,0.471133,0.558148}%
\pgfsetfillcolor{currentfill}%
\pgfsetlinewidth{0.000000pt}%
\definecolor{currentstroke}{rgb}{0.000000,0.000000,0.000000}%
\pgfsetstrokecolor{currentstroke}%
\pgfsetdash{}{0pt}%
\pgfpathmoveto{\pgfqpoint{1.456727in}{1.466037in}}%
\pgfpathlineto{\pgfqpoint{1.457809in}{1.460819in}}%
\pgfpathlineto{\pgfqpoint{1.458890in}{1.455616in}}%
\pgfpathlineto{\pgfqpoint{1.459969in}{1.450430in}}%
\pgfpathlineto{\pgfqpoint{1.461047in}{1.445264in}}%
\pgfpathlineto{\pgfqpoint{1.470556in}{1.443495in}}%
\pgfpathlineto{\pgfqpoint{1.479954in}{1.441579in}}%
\pgfpathlineto{\pgfqpoint{1.489233in}{1.439519in}}%
\pgfpathlineto{\pgfqpoint{1.498384in}{1.437315in}}%
\pgfpathlineto{\pgfqpoint{1.496937in}{1.442566in}}%
\pgfpathlineto{\pgfqpoint{1.495490in}{1.447836in}}%
\pgfpathlineto{\pgfqpoint{1.494040in}{1.453124in}}%
\pgfpathlineto{\pgfqpoint{1.492589in}{1.458426in}}%
\pgfpathlineto{\pgfqpoint{1.483800in}{1.460536in}}%
\pgfpathlineto{\pgfqpoint{1.474888in}{1.462509in}}%
\pgfpathlineto{\pgfqpoint{1.465861in}{1.464343in}}%
\pgfpathlineto{\pgfqpoint{1.456727in}{1.466037in}}%
\pgfpathclose%
\pgfusepath{fill}%
\end{pgfscope}%
\begin{pgfscope}%
\pgfpathrectangle{\pgfqpoint{0.329460in}{0.284240in}}{\pgfqpoint{1.989680in}{1.989680in}}%
\pgfusepath{clip}%
\pgfsetbuttcap%
\pgfsetroundjoin%
\definecolor{currentfill}{rgb}{0.279566,0.067836,0.391917}%
\pgfsetfillcolor{currentfill}%
\pgfsetlinewidth{0.000000pt}%
\definecolor{currentstroke}{rgb}{0.000000,0.000000,0.000000}%
\pgfsetstrokecolor{currentstroke}%
\pgfsetdash{}{0pt}%
\pgfpathmoveto{\pgfqpoint{1.780655in}{1.148248in}}%
\pgfpathlineto{\pgfqpoint{1.783693in}{1.145349in}}%
\pgfpathlineto{\pgfqpoint{1.786733in}{1.142598in}}%
\pgfpathlineto{\pgfqpoint{1.789775in}{1.140000in}}%
\pgfpathlineto{\pgfqpoint{1.792819in}{1.137558in}}%
\pgfpathlineto{\pgfqpoint{1.797998in}{1.130350in}}%
\pgfpathlineto{\pgfqpoint{1.802749in}{1.123054in}}%
\pgfpathlineto{\pgfqpoint{1.807064in}{1.115675in}}%
\pgfpathlineto{\pgfqpoint{1.810938in}{1.108221in}}%
\pgfpathlineto{\pgfqpoint{1.807755in}{1.110877in}}%
\pgfpathlineto{\pgfqpoint{1.804574in}{1.113690in}}%
\pgfpathlineto{\pgfqpoint{1.801395in}{1.116656in}}%
\pgfpathlineto{\pgfqpoint{1.798219in}{1.119771in}}%
\pgfpathlineto{\pgfqpoint{1.794466in}{1.127006in}}%
\pgfpathlineto{\pgfqpoint{1.790284in}{1.134168in}}%
\pgfpathlineto{\pgfqpoint{1.785679in}{1.141251in}}%
\pgfpathlineto{\pgfqpoint{1.780655in}{1.148248in}}%
\pgfpathclose%
\pgfusepath{fill}%
\end{pgfscope}%
\begin{pgfscope}%
\pgfpathrectangle{\pgfqpoint{0.329460in}{0.284240in}}{\pgfqpoint{1.989680in}{1.989680in}}%
\pgfusepath{clip}%
\pgfsetbuttcap%
\pgfsetroundjoin%
\definecolor{currentfill}{rgb}{0.274952,0.037752,0.364543}%
\pgfsetfillcolor{currentfill}%
\pgfsetlinewidth{0.000000pt}%
\definecolor{currentstroke}{rgb}{0.000000,0.000000,0.000000}%
\pgfsetstrokecolor{currentstroke}%
\pgfsetdash{}{0pt}%
\pgfpathmoveto{\pgfqpoint{0.888368in}{1.101538in}}%
\pgfpathlineto{\pgfqpoint{0.885157in}{1.098993in}}%
\pgfpathlineto{\pgfqpoint{0.881944in}{1.096612in}}%
\pgfpathlineto{\pgfqpoint{0.878727in}{1.094399in}}%
\pgfpathlineto{\pgfqpoint{0.875508in}{1.092358in}}%
\pgfpathlineto{\pgfqpoint{0.879098in}{1.100092in}}%
\pgfpathlineto{\pgfqpoint{0.883146in}{1.107755in}}%
\pgfpathlineto{\pgfqpoint{0.887645in}{1.115339in}}%
\pgfpathlineto{\pgfqpoint{0.892591in}{1.122838in}}%
\pgfpathlineto{\pgfqpoint{0.895680in}{1.124664in}}%
\pgfpathlineto{\pgfqpoint{0.898766in}{1.126662in}}%
\pgfpathlineto{\pgfqpoint{0.901850in}{1.128827in}}%
\pgfpathlineto{\pgfqpoint{0.904931in}{1.131156in}}%
\pgfpathlineto{\pgfqpoint{0.900133in}{1.123869in}}%
\pgfpathlineto{\pgfqpoint{0.895769in}{1.116499in}}%
\pgfpathlineto{\pgfqpoint{0.891846in}{1.109053in}}%
\pgfpathlineto{\pgfqpoint{0.888368in}{1.101538in}}%
\pgfpathclose%
\pgfusepath{fill}%
\end{pgfscope}%
\begin{pgfscope}%
\pgfpathrectangle{\pgfqpoint{0.329460in}{0.284240in}}{\pgfqpoint{1.989680in}{1.989680in}}%
\pgfusepath{clip}%
\pgfsetbuttcap%
\pgfsetroundjoin%
\definecolor{currentfill}{rgb}{0.163625,0.471133,0.558148}%
\pgfsetfillcolor{currentfill}%
\pgfsetlinewidth{0.000000pt}%
\definecolor{currentstroke}{rgb}{0.000000,0.000000,0.000000}%
\pgfsetstrokecolor{currentstroke}%
\pgfsetdash{}{0pt}%
\pgfpathmoveto{\pgfqpoint{1.202084in}{1.456438in}}%
\pgfpathlineto{\pgfqpoint{1.200553in}{1.451113in}}%
\pgfpathlineto{\pgfqpoint{1.199024in}{1.445803in}}%
\pgfpathlineto{\pgfqpoint{1.197497in}{1.440511in}}%
\pgfpathlineto{\pgfqpoint{1.195972in}{1.435238in}}%
\pgfpathlineto{\pgfqpoint{1.205002in}{1.437567in}}%
\pgfpathlineto{\pgfqpoint{1.214167in}{1.439755in}}%
\pgfpathlineto{\pgfqpoint{1.223460in}{1.441799in}}%
\pgfpathlineto{\pgfqpoint{1.232871in}{1.443699in}}%
\pgfpathlineto{\pgfqpoint{1.234032in}{1.448881in}}%
\pgfpathlineto{\pgfqpoint{1.235195in}{1.454084in}}%
\pgfpathlineto{\pgfqpoint{1.236359in}{1.459304in}}%
\pgfpathlineto{\pgfqpoint{1.237524in}{1.464538in}}%
\pgfpathlineto{\pgfqpoint{1.228485in}{1.462719in}}%
\pgfpathlineto{\pgfqpoint{1.219559in}{1.460762in}}%
\pgfpathlineto{\pgfqpoint{1.210756in}{1.458668in}}%
\pgfpathlineto{\pgfqpoint{1.202084in}{1.456438in}}%
\pgfpathclose%
\pgfusepath{fill}%
\end{pgfscope}%
\begin{pgfscope}%
\pgfpathrectangle{\pgfqpoint{0.329460in}{0.284240in}}{\pgfqpoint{1.989680in}{1.989680in}}%
\pgfusepath{clip}%
\pgfsetbuttcap%
\pgfsetroundjoin%
\definecolor{currentfill}{rgb}{0.231674,0.318106,0.544834}%
\pgfsetfillcolor{currentfill}%
\pgfsetlinewidth{0.000000pt}%
\definecolor{currentstroke}{rgb}{0.000000,0.000000,0.000000}%
\pgfsetstrokecolor{currentstroke}%
\pgfsetdash{}{0pt}%
\pgfpathmoveto{\pgfqpoint{1.044027in}{1.312931in}}%
\pgfpathlineto{\pgfqpoint{1.041331in}{1.307574in}}%
\pgfpathlineto{\pgfqpoint{1.038638in}{1.302275in}}%
\pgfpathlineto{\pgfqpoint{1.035945in}{1.297037in}}%
\pgfpathlineto{\pgfqpoint{1.033255in}{1.291864in}}%
\pgfpathlineto{\pgfqpoint{1.039986in}{1.296839in}}%
\pgfpathlineto{\pgfqpoint{1.047010in}{1.301702in}}%
\pgfpathlineto{\pgfqpoint{1.054322in}{1.306448in}}%
\pgfpathlineto{\pgfqpoint{1.061912in}{1.311073in}}%
\pgfpathlineto{\pgfqpoint{1.064353in}{1.316069in}}%
\pgfpathlineto{\pgfqpoint{1.066795in}{1.321129in}}%
\pgfpathlineto{\pgfqpoint{1.069239in}{1.326250in}}%
\pgfpathlineto{\pgfqpoint{1.071684in}{1.331429in}}%
\pgfpathlineto{\pgfqpoint{1.064357in}{1.326975in}}%
\pgfpathlineto{\pgfqpoint{1.057300in}{1.322405in}}%
\pgfpathlineto{\pgfqpoint{1.050521in}{1.317722in}}%
\pgfpathlineto{\pgfqpoint{1.044027in}{1.312931in}}%
\pgfpathclose%
\pgfusepath{fill}%
\end{pgfscope}%
\begin{pgfscope}%
\pgfpathrectangle{\pgfqpoint{0.329460in}{0.284240in}}{\pgfqpoint{1.989680in}{1.989680in}}%
\pgfusepath{clip}%
\pgfsetbuttcap%
\pgfsetroundjoin%
\definecolor{currentfill}{rgb}{0.276194,0.190074,0.493001}%
\pgfsetfillcolor{currentfill}%
\pgfsetlinewidth{0.000000pt}%
\definecolor{currentstroke}{rgb}{0.000000,0.000000,0.000000}%
\pgfsetstrokecolor{currentstroke}%
\pgfsetdash{}{0pt}%
\pgfpathmoveto{\pgfqpoint{1.918624in}{1.213901in}}%
\pgfpathlineto{\pgfqpoint{1.921940in}{1.221443in}}%
\pgfpathlineto{\pgfqpoint{1.925269in}{1.229331in}}%
\pgfpathlineto{\pgfqpoint{1.928611in}{1.237570in}}%
\pgfpathlineto{\pgfqpoint{1.931966in}{1.246166in}}%
\pgfpathlineto{\pgfqpoint{1.938986in}{1.236846in}}%
\pgfpathlineto{\pgfqpoint{1.945450in}{1.227403in}}%
\pgfpathlineto{\pgfqpoint{1.951348in}{1.217848in}}%
\pgfpathlineto{\pgfqpoint{1.956671in}{1.208188in}}%
\pgfpathlineto{\pgfqpoint{1.953153in}{1.199761in}}%
\pgfpathlineto{\pgfqpoint{1.949648in}{1.191692in}}%
\pgfpathlineto{\pgfqpoint{1.946157in}{1.183977in}}%
\pgfpathlineto{\pgfqpoint{1.942680in}{1.176609in}}%
\pgfpathlineto{\pgfqpoint{1.937500in}{1.186093in}}%
\pgfpathlineto{\pgfqpoint{1.931758in}{1.195476in}}%
\pgfpathlineto{\pgfqpoint{1.925464in}{1.204748in}}%
\pgfpathlineto{\pgfqpoint{1.918624in}{1.213901in}}%
\pgfpathclose%
\pgfusepath{fill}%
\end{pgfscope}%
\begin{pgfscope}%
\pgfpathrectangle{\pgfqpoint{0.329460in}{0.284240in}}{\pgfqpoint{1.989680in}{1.989680in}}%
\pgfusepath{clip}%
\pgfsetbuttcap%
\pgfsetroundjoin%
\definecolor{currentfill}{rgb}{0.282884,0.135920,0.453427}%
\pgfsetfillcolor{currentfill}%
\pgfsetlinewidth{0.000000pt}%
\definecolor{currentstroke}{rgb}{0.000000,0.000000,0.000000}%
\pgfsetstrokecolor{currentstroke}%
\pgfsetdash{}{0pt}%
\pgfpathmoveto{\pgfqpoint{0.769467in}{1.142152in}}%
\pgfpathlineto{\pgfqpoint{0.766011in}{1.148146in}}%
\pgfpathlineto{\pgfqpoint{0.762543in}{1.154465in}}%
\pgfpathlineto{\pgfqpoint{0.759062in}{1.161114in}}%
\pgfpathlineto{\pgfqpoint{0.755568in}{1.168100in}}%
\pgfpathlineto{\pgfqpoint{0.760243in}{1.177668in}}%
\pgfpathlineto{\pgfqpoint{0.765486in}{1.187141in}}%
\pgfpathlineto{\pgfqpoint{0.771289in}{1.196512in}}%
\pgfpathlineto{\pgfqpoint{0.777645in}{1.205771in}}%
\pgfpathlineto{\pgfqpoint{0.780989in}{1.198606in}}%
\pgfpathlineto{\pgfqpoint{0.784321in}{1.191777in}}%
\pgfpathlineto{\pgfqpoint{0.787642in}{1.185276in}}%
\pgfpathlineto{\pgfqpoint{0.790951in}{1.179099in}}%
\pgfpathlineto{\pgfqpoint{0.784762in}{1.170017in}}%
\pgfpathlineto{\pgfqpoint{0.779113in}{1.160826in}}%
\pgfpathlineto{\pgfqpoint{0.774013in}{1.151535in}}%
\pgfpathlineto{\pgfqpoint{0.769467in}{1.142152in}}%
\pgfpathclose%
\pgfusepath{fill}%
\end{pgfscope}%
\begin{pgfscope}%
\pgfpathrectangle{\pgfqpoint{0.329460in}{0.284240in}}{\pgfqpoint{1.989680in}{1.989680in}}%
\pgfusepath{clip}%
\pgfsetbuttcap%
\pgfsetroundjoin%
\definecolor{currentfill}{rgb}{0.147607,0.511733,0.557049}%
\pgfsetfillcolor{currentfill}%
\pgfsetlinewidth{0.000000pt}%
\definecolor{currentstroke}{rgb}{0.000000,0.000000,0.000000}%
\pgfsetstrokecolor{currentstroke}%
\pgfsetdash{}{0pt}%
\pgfpathmoveto{\pgfqpoint{1.314479in}{1.494543in}}%
\pgfpathlineto{\pgfqpoint{1.314085in}{1.489380in}}%
\pgfpathlineto{\pgfqpoint{1.313691in}{1.484221in}}%
\pgfpathlineto{\pgfqpoint{1.313299in}{1.479069in}}%
\pgfpathlineto{\pgfqpoint{1.312906in}{1.473927in}}%
\pgfpathlineto{\pgfqpoint{1.322581in}{1.474441in}}%
\pgfpathlineto{\pgfqpoint{1.332282in}{1.474806in}}%
\pgfpathlineto{\pgfqpoint{1.342001in}{1.475021in}}%
\pgfpathlineto{\pgfqpoint{1.351728in}{1.475088in}}%
\pgfpathlineto{\pgfqpoint{1.351722in}{1.480217in}}%
\pgfpathlineto{\pgfqpoint{1.351717in}{1.485356in}}%
\pgfpathlineto{\pgfqpoint{1.351711in}{1.490502in}}%
\pgfpathlineto{\pgfqpoint{1.351706in}{1.495653in}}%
\pgfpathlineto{\pgfqpoint{1.342378in}{1.495589in}}%
\pgfpathlineto{\pgfqpoint{1.333059in}{1.495383in}}%
\pgfpathlineto{\pgfqpoint{1.323756in}{1.495034in}}%
\pgfpathlineto{\pgfqpoint{1.314479in}{1.494543in}}%
\pgfpathclose%
\pgfusepath{fill}%
\end{pgfscope}%
\begin{pgfscope}%
\pgfpathrectangle{\pgfqpoint{0.329460in}{0.284240in}}{\pgfqpoint{1.989680in}{1.989680in}}%
\pgfusepath{clip}%
\pgfsetbuttcap%
\pgfsetroundjoin%
\definecolor{currentfill}{rgb}{0.147607,0.511733,0.557049}%
\pgfsetfillcolor{currentfill}%
\pgfsetlinewidth{0.000000pt}%
\definecolor{currentstroke}{rgb}{0.000000,0.000000,0.000000}%
\pgfsetstrokecolor{currentstroke}%
\pgfsetdash{}{0pt}%
\pgfpathmoveto{\pgfqpoint{1.351706in}{1.495653in}}%
\pgfpathlineto{\pgfqpoint{1.351711in}{1.490502in}}%
\pgfpathlineto{\pgfqpoint{1.351717in}{1.485356in}}%
\pgfpathlineto{\pgfqpoint{1.351722in}{1.480217in}}%
\pgfpathlineto{\pgfqpoint{1.351728in}{1.475088in}}%
\pgfpathlineto{\pgfqpoint{1.361455in}{1.475005in}}%
\pgfpathlineto{\pgfqpoint{1.371172in}{1.474772in}}%
\pgfpathlineto{\pgfqpoint{1.380871in}{1.474391in}}%
\pgfpathlineto{\pgfqpoint{1.390542in}{1.473861in}}%
\pgfpathlineto{\pgfqpoint{1.390139in}{1.479004in}}%
\pgfpathlineto{\pgfqpoint{1.389735in}{1.484156in}}%
\pgfpathlineto{\pgfqpoint{1.389330in}{1.489316in}}%
\pgfpathlineto{\pgfqpoint{1.388926in}{1.494480in}}%
\pgfpathlineto{\pgfqpoint{1.379652in}{1.494987in}}%
\pgfpathlineto{\pgfqpoint{1.370351in}{1.495351in}}%
\pgfpathlineto{\pgfqpoint{1.361033in}{1.495573in}}%
\pgfpathlineto{\pgfqpoint{1.351706in}{1.495653in}}%
\pgfpathclose%
\pgfusepath{fill}%
\end{pgfscope}%
\begin{pgfscope}%
\pgfpathrectangle{\pgfqpoint{0.329460in}{0.284240in}}{\pgfqpoint{1.989680in}{1.989680in}}%
\pgfusepath{clip}%
\pgfsetbuttcap%
\pgfsetroundjoin%
\definecolor{currentfill}{rgb}{0.280255,0.165693,0.476498}%
\pgfsetfillcolor{currentfill}%
\pgfsetlinewidth{0.000000pt}%
\definecolor{currentstroke}{rgb}{0.000000,0.000000,0.000000}%
\pgfsetstrokecolor{currentstroke}%
\pgfsetdash{}{0pt}%
\pgfpathmoveto{\pgfqpoint{0.954018in}{1.187741in}}%
\pgfpathlineto{\pgfqpoint{0.950956in}{1.183277in}}%
\pgfpathlineto{\pgfqpoint{0.947893in}{1.178921in}}%
\pgfpathlineto{\pgfqpoint{0.944831in}{1.174677in}}%
\pgfpathlineto{\pgfqpoint{0.941768in}{1.170546in}}%
\pgfpathlineto{\pgfqpoint{0.946525in}{1.177109in}}%
\pgfpathlineto{\pgfqpoint{0.951672in}{1.183586in}}%
\pgfpathlineto{\pgfqpoint{0.957200in}{1.189970in}}%
\pgfpathlineto{\pgfqpoint{0.963104in}{1.196256in}}%
\pgfpathlineto{\pgfqpoint{0.965996in}{1.200178in}}%
\pgfpathlineto{\pgfqpoint{0.968887in}{1.204215in}}%
\pgfpathlineto{\pgfqpoint{0.971779in}{1.208362in}}%
\pgfpathlineto{\pgfqpoint{0.974670in}{1.212617in}}%
\pgfpathlineto{\pgfqpoint{0.968953in}{1.206535in}}%
\pgfpathlineto{\pgfqpoint{0.963601in}{1.200357in}}%
\pgfpathlineto{\pgfqpoint{0.958621in}{1.194091in}}%
\pgfpathlineto{\pgfqpoint{0.954018in}{1.187741in}}%
\pgfpathclose%
\pgfusepath{fill}%
\end{pgfscope}%
\begin{pgfscope}%
\pgfpathrectangle{\pgfqpoint{0.329460in}{0.284240in}}{\pgfqpoint{1.989680in}{1.989680in}}%
\pgfusepath{clip}%
\pgfsetbuttcap%
\pgfsetroundjoin%
\definecolor{currentfill}{rgb}{0.147607,0.511733,0.557049}%
\pgfsetfillcolor{currentfill}%
\pgfsetlinewidth{0.000000pt}%
\definecolor{currentstroke}{rgb}{0.000000,0.000000,0.000000}%
\pgfsetstrokecolor{currentstroke}%
\pgfsetdash{}{0pt}%
\pgfpathmoveto{\pgfqpoint{1.277795in}{1.491167in}}%
\pgfpathlineto{\pgfqpoint{1.277007in}{1.485965in}}%
\pgfpathlineto{\pgfqpoint{1.276220in}{1.480767in}}%
\pgfpathlineto{\pgfqpoint{1.275434in}{1.475576in}}%
\pgfpathlineto{\pgfqpoint{1.274649in}{1.470394in}}%
\pgfpathlineto{\pgfqpoint{1.284129in}{1.471498in}}%
\pgfpathlineto{\pgfqpoint{1.293671in}{1.472455in}}%
\pgfpathlineto{\pgfqpoint{1.303267in}{1.473265in}}%
\pgfpathlineto{\pgfqpoint{1.312906in}{1.473927in}}%
\pgfpathlineto{\pgfqpoint{1.313299in}{1.479069in}}%
\pgfpathlineto{\pgfqpoint{1.313691in}{1.484221in}}%
\pgfpathlineto{\pgfqpoint{1.314085in}{1.489380in}}%
\pgfpathlineto{\pgfqpoint{1.314479in}{1.494543in}}%
\pgfpathlineto{\pgfqpoint{1.305235in}{1.493911in}}%
\pgfpathlineto{\pgfqpoint{1.296034in}{1.493137in}}%
\pgfpathlineto{\pgfqpoint{1.286885in}{1.492222in}}%
\pgfpathlineto{\pgfqpoint{1.277795in}{1.491167in}}%
\pgfpathclose%
\pgfusepath{fill}%
\end{pgfscope}%
\begin{pgfscope}%
\pgfpathrectangle{\pgfqpoint{0.329460in}{0.284240in}}{\pgfqpoint{1.989680in}{1.989680in}}%
\pgfusepath{clip}%
\pgfsetbuttcap%
\pgfsetroundjoin%
\definecolor{currentfill}{rgb}{0.147607,0.511733,0.557049}%
\pgfsetfillcolor{currentfill}%
\pgfsetlinewidth{0.000000pt}%
\definecolor{currentstroke}{rgb}{0.000000,0.000000,0.000000}%
\pgfsetstrokecolor{currentstroke}%
\pgfsetdash{}{0pt}%
\pgfpathmoveto{\pgfqpoint{1.388926in}{1.494480in}}%
\pgfpathlineto{\pgfqpoint{1.389330in}{1.489316in}}%
\pgfpathlineto{\pgfqpoint{1.389735in}{1.484156in}}%
\pgfpathlineto{\pgfqpoint{1.390139in}{1.479004in}}%
\pgfpathlineto{\pgfqpoint{1.390542in}{1.473861in}}%
\pgfpathlineto{\pgfqpoint{1.400177in}{1.473182in}}%
\pgfpathlineto{\pgfqpoint{1.409767in}{1.472356in}}%
\pgfpathlineto{\pgfqpoint{1.419302in}{1.471382in}}%
\pgfpathlineto{\pgfqpoint{1.428775in}{1.470263in}}%
\pgfpathlineto{\pgfqpoint{1.427979in}{1.475446in}}%
\pgfpathlineto{\pgfqpoint{1.427183in}{1.480638in}}%
\pgfpathlineto{\pgfqpoint{1.426385in}{1.485838in}}%
\pgfpathlineto{\pgfqpoint{1.425586in}{1.491042in}}%
\pgfpathlineto{\pgfqpoint{1.416504in}{1.492112in}}%
\pgfpathlineto{\pgfqpoint{1.407360in}{1.493042in}}%
\pgfpathlineto{\pgfqpoint{1.398165in}{1.493832in}}%
\pgfpathlineto{\pgfqpoint{1.388926in}{1.494480in}}%
\pgfpathclose%
\pgfusepath{fill}%
\end{pgfscope}%
\begin{pgfscope}%
\pgfpathrectangle{\pgfqpoint{0.329460in}{0.284240in}}{\pgfqpoint{1.989680in}{1.989680in}}%
\pgfusepath{clip}%
\pgfsetbuttcap%
\pgfsetroundjoin%
\definecolor{currentfill}{rgb}{0.201239,0.383670,0.554294}%
\pgfsetfillcolor{currentfill}%
\pgfsetlinewidth{0.000000pt}%
\definecolor{currentstroke}{rgb}{0.000000,0.000000,0.000000}%
\pgfsetstrokecolor{currentstroke}%
\pgfsetdash{}{0pt}%
\pgfpathmoveto{\pgfqpoint{0.736450in}{1.319970in}}%
\pgfpathlineto{\pgfqpoint{0.732916in}{1.332026in}}%
\pgfpathlineto{\pgfqpoint{0.729365in}{1.344504in}}%
\pgfpathlineto{\pgfqpoint{0.725796in}{1.357410in}}%
\pgfpathlineto{\pgfqpoint{0.722209in}{1.370752in}}%
\pgfpathlineto{\pgfqpoint{0.729863in}{1.380511in}}%
\pgfpathlineto{\pgfqpoint{0.738100in}{1.390131in}}%
\pgfpathlineto{\pgfqpoint{0.746909in}{1.399603in}}%
\pgfpathlineto{\pgfqpoint{0.756281in}{1.408920in}}%
\pgfpathlineto{\pgfqpoint{0.759652in}{1.395444in}}%
\pgfpathlineto{\pgfqpoint{0.763007in}{1.382402in}}%
\pgfpathlineto{\pgfqpoint{0.766345in}{1.369785in}}%
\pgfpathlineto{\pgfqpoint{0.769668in}{1.357587in}}%
\pgfpathlineto{\pgfqpoint{0.760527in}{1.348404in}}%
\pgfpathlineto{\pgfqpoint{0.751937in}{1.339068in}}%
\pgfpathlineto{\pgfqpoint{0.743908in}{1.329587in}}%
\pgfpathlineto{\pgfqpoint{0.736450in}{1.319970in}}%
\pgfpathclose%
\pgfusepath{fill}%
\end{pgfscope}%
\begin{pgfscope}%
\pgfpathrectangle{\pgfqpoint{0.329460in}{0.284240in}}{\pgfqpoint{1.989680in}{1.989680in}}%
\pgfusepath{clip}%
\pgfsetbuttcap%
\pgfsetroundjoin%
\definecolor{currentfill}{rgb}{0.248629,0.278775,0.534556}%
\pgfsetfillcolor{currentfill}%
\pgfsetlinewidth{0.000000pt}%
\definecolor{currentstroke}{rgb}{0.000000,0.000000,0.000000}%
\pgfsetstrokecolor{currentstroke}%
\pgfsetdash{}{0pt}%
\pgfpathmoveto{\pgfqpoint{1.663152in}{1.296292in}}%
\pgfpathlineto{\pgfqpoint{1.665789in}{1.291227in}}%
\pgfpathlineto{\pgfqpoint{1.668424in}{1.286232in}}%
\pgfpathlineto{\pgfqpoint{1.671058in}{1.281310in}}%
\pgfpathlineto{\pgfqpoint{1.673691in}{1.276464in}}%
\pgfpathlineto{\pgfqpoint{1.680624in}{1.271291in}}%
\pgfpathlineto{\pgfqpoint{1.687245in}{1.266006in}}%
\pgfpathlineto{\pgfqpoint{1.693547in}{1.260614in}}%
\pgfpathlineto{\pgfqpoint{1.699524in}{1.255119in}}%
\pgfpathlineto{\pgfqpoint{1.696671in}{1.260156in}}%
\pgfpathlineto{\pgfqpoint{1.693816in}{1.265269in}}%
\pgfpathlineto{\pgfqpoint{1.690960in}{1.270456in}}%
\pgfpathlineto{\pgfqpoint{1.688103in}{1.275712in}}%
\pgfpathlineto{\pgfqpoint{1.682332in}{1.281009in}}%
\pgfpathlineto{\pgfqpoint{1.676245in}{1.286208in}}%
\pgfpathlineto{\pgfqpoint{1.669850in}{1.291304in}}%
\pgfpathlineto{\pgfqpoint{1.663152in}{1.296292in}}%
\pgfpathclose%
\pgfusepath{fill}%
\end{pgfscope}%
\begin{pgfscope}%
\pgfpathrectangle{\pgfqpoint{0.329460in}{0.284240in}}{\pgfqpoint{1.989680in}{1.989680in}}%
\pgfusepath{clip}%
\pgfsetbuttcap%
\pgfsetroundjoin%
\definecolor{currentfill}{rgb}{0.179019,0.433756,0.557430}%
\pgfsetfillcolor{currentfill}%
\pgfsetlinewidth{0.000000pt}%
\definecolor{currentstroke}{rgb}{0.000000,0.000000,0.000000}%
\pgfsetstrokecolor{currentstroke}%
\pgfsetdash{}{0pt}%
\pgfpathmoveto{\pgfqpoint{1.533537in}{1.427109in}}%
\pgfpathlineto{\pgfqpoint{1.535330in}{1.421772in}}%
\pgfpathlineto{\pgfqpoint{1.537121in}{1.416459in}}%
\pgfpathlineto{\pgfqpoint{1.538910in}{1.411175in}}%
\pgfpathlineto{\pgfqpoint{1.540697in}{1.405921in}}%
\pgfpathlineto{\pgfqpoint{1.549414in}{1.402908in}}%
\pgfpathlineto{\pgfqpoint{1.557946in}{1.399761in}}%
\pgfpathlineto{\pgfqpoint{1.566286in}{1.396480in}}%
\pgfpathlineto{\pgfqpoint{1.574424in}{1.393069in}}%
\pgfpathlineto{\pgfqpoint{1.572315in}{1.398456in}}%
\pgfpathlineto{\pgfqpoint{1.570203in}{1.403872in}}%
\pgfpathlineto{\pgfqpoint{1.568089in}{1.409317in}}%
\pgfpathlineto{\pgfqpoint{1.565973in}{1.414787in}}%
\pgfpathlineto{\pgfqpoint{1.558147in}{1.418057in}}%
\pgfpathlineto{\pgfqpoint{1.550127in}{1.421202in}}%
\pgfpathlineto{\pgfqpoint{1.541922in}{1.424221in}}%
\pgfpathlineto{\pgfqpoint{1.533537in}{1.427109in}}%
\pgfpathclose%
\pgfusepath{fill}%
\end{pgfscope}%
\begin{pgfscope}%
\pgfpathrectangle{\pgfqpoint{0.329460in}{0.284240in}}{\pgfqpoint{1.989680in}{1.989680in}}%
\pgfusepath{clip}%
\pgfsetbuttcap%
\pgfsetroundjoin%
\definecolor{currentfill}{rgb}{0.195860,0.395433,0.555276}%
\pgfsetfillcolor{currentfill}%
\pgfsetlinewidth{0.000000pt}%
\definecolor{currentstroke}{rgb}{0.000000,0.000000,0.000000}%
\pgfsetstrokecolor{currentstroke}%
\pgfsetdash{}{0pt}%
\pgfpathmoveto{\pgfqpoint{1.574424in}{1.393069in}}%
\pgfpathlineto{\pgfqpoint{1.576532in}{1.387716in}}%
\pgfpathlineto{\pgfqpoint{1.578637in}{1.382398in}}%
\pgfpathlineto{\pgfqpoint{1.580741in}{1.377119in}}%
\pgfpathlineto{\pgfqpoint{1.582843in}{1.371882in}}%
\pgfpathlineto{\pgfqpoint{1.591076in}{1.368198in}}%
\pgfpathlineto{\pgfqpoint{1.599086in}{1.364385in}}%
\pgfpathlineto{\pgfqpoint{1.606863in}{1.360446in}}%
\pgfpathlineto{\pgfqpoint{1.614400in}{1.356384in}}%
\pgfpathlineto{\pgfqpoint{1.612006in}{1.361776in}}%
\pgfpathlineto{\pgfqpoint{1.609610in}{1.367209in}}%
\pgfpathlineto{\pgfqpoint{1.607212in}{1.372680in}}%
\pgfpathlineto{\pgfqpoint{1.604811in}{1.378187in}}%
\pgfpathlineto{\pgfqpoint{1.597555in}{1.382087in}}%
\pgfpathlineto{\pgfqpoint{1.590066in}{1.385870in}}%
\pgfpathlineto{\pgfqpoint{1.582354in}{1.389531in}}%
\pgfpathlineto{\pgfqpoint{1.574424in}{1.393069in}}%
\pgfpathclose%
\pgfusepath{fill}%
\end{pgfscope}%
\begin{pgfscope}%
\pgfpathrectangle{\pgfqpoint{0.329460in}{0.284240in}}{\pgfqpoint{1.989680in}{1.989680in}}%
\pgfusepath{clip}%
\pgfsetbuttcap%
\pgfsetroundjoin%
\definecolor{currentfill}{rgb}{0.274128,0.199721,0.498911}%
\pgfsetfillcolor{currentfill}%
\pgfsetlinewidth{0.000000pt}%
\definecolor{currentstroke}{rgb}{0.000000,0.000000,0.000000}%
\pgfsetstrokecolor{currentstroke}%
\pgfsetdash{}{0pt}%
\pgfpathmoveto{\pgfqpoint{1.710928in}{1.235796in}}%
\pgfpathlineto{\pgfqpoint{1.713777in}{1.231186in}}%
\pgfpathlineto{\pgfqpoint{1.716626in}{1.226672in}}%
\pgfpathlineto{\pgfqpoint{1.719474in}{1.222255in}}%
\pgfpathlineto{\pgfqpoint{1.722322in}{1.217940in}}%
\pgfpathlineto{\pgfqpoint{1.728358in}{1.211946in}}%
\pgfpathlineto{\pgfqpoint{1.734035in}{1.205853in}}%
\pgfpathlineto{\pgfqpoint{1.739346in}{1.199665in}}%
\pgfpathlineto{\pgfqpoint{1.744285in}{1.193389in}}%
\pgfpathlineto{\pgfqpoint{1.741257in}{1.197910in}}%
\pgfpathlineto{\pgfqpoint{1.738228in}{1.202533in}}%
\pgfpathlineto{\pgfqpoint{1.735200in}{1.207254in}}%
\pgfpathlineto{\pgfqpoint{1.732171in}{1.212069in}}%
\pgfpathlineto{\pgfqpoint{1.727396in}{1.218134in}}%
\pgfpathlineto{\pgfqpoint{1.722260in}{1.224114in}}%
\pgfpathlineto{\pgfqpoint{1.716768in}{1.230003in}}%
\pgfpathlineto{\pgfqpoint{1.710928in}{1.235796in}}%
\pgfpathclose%
\pgfusepath{fill}%
\end{pgfscope}%
\begin{pgfscope}%
\pgfpathrectangle{\pgfqpoint{0.329460in}{0.284240in}}{\pgfqpoint{1.989680in}{1.989680in}}%
\pgfusepath{clip}%
\pgfsetbuttcap%
\pgfsetroundjoin%
\definecolor{currentfill}{rgb}{0.279566,0.067836,0.391917}%
\pgfsetfillcolor{currentfill}%
\pgfsetlinewidth{0.000000pt}%
\definecolor{currentstroke}{rgb}{0.000000,0.000000,0.000000}%
\pgfsetstrokecolor{currentstroke}%
\pgfsetdash{}{0pt}%
\pgfpathmoveto{\pgfqpoint{0.901185in}{1.113284in}}%
\pgfpathlineto{\pgfqpoint{0.897984in}{1.110120in}}%
\pgfpathlineto{\pgfqpoint{0.894781in}{1.107105in}}%
\pgfpathlineto{\pgfqpoint{0.891575in}{1.104243in}}%
\pgfpathlineto{\pgfqpoint{0.888368in}{1.101538in}}%
\pgfpathlineto{\pgfqpoint{0.891846in}{1.109053in}}%
\pgfpathlineto{\pgfqpoint{0.895769in}{1.116499in}}%
\pgfpathlineto{\pgfqpoint{0.900133in}{1.123869in}}%
\pgfpathlineto{\pgfqpoint{0.904931in}{1.131156in}}%
\pgfpathlineto{\pgfqpoint{0.908010in}{1.133644in}}%
\pgfpathlineto{\pgfqpoint{0.911087in}{1.136289in}}%
\pgfpathlineto{\pgfqpoint{0.914161in}{1.139086in}}%
\pgfpathlineto{\pgfqpoint{0.917234in}{1.142033in}}%
\pgfpathlineto{\pgfqpoint{0.912582in}{1.134959in}}%
\pgfpathlineto{\pgfqpoint{0.908353in}{1.127805in}}%
\pgfpathlineto{\pgfqpoint{0.904552in}{1.120578in}}%
\pgfpathlineto{\pgfqpoint{0.901185in}{1.113284in}}%
\pgfpathclose%
\pgfusepath{fill}%
\end{pgfscope}%
\begin{pgfscope}%
\pgfpathrectangle{\pgfqpoint{0.329460in}{0.284240in}}{\pgfqpoint{1.989680in}{1.989680in}}%
\pgfusepath{clip}%
\pgfsetbuttcap%
\pgfsetroundjoin%
\definecolor{currentfill}{rgb}{0.282327,0.094955,0.417331}%
\pgfsetfillcolor{currentfill}%
\pgfsetlinewidth{0.000000pt}%
\definecolor{currentstroke}{rgb}{0.000000,0.000000,0.000000}%
\pgfsetstrokecolor{currentstroke}%
\pgfsetdash{}{0pt}%
\pgfpathmoveto{\pgfqpoint{1.768518in}{1.161265in}}%
\pgfpathlineto{\pgfqpoint{1.771551in}{1.157805in}}%
\pgfpathlineto{\pgfqpoint{1.774584in}{1.154480in}}%
\pgfpathlineto{\pgfqpoint{1.777619in}{1.151293in}}%
\pgfpathlineto{\pgfqpoint{1.780655in}{1.148248in}}%
\pgfpathlineto{\pgfqpoint{1.785679in}{1.141251in}}%
\pgfpathlineto{\pgfqpoint{1.790284in}{1.134168in}}%
\pgfpathlineto{\pgfqpoint{1.794466in}{1.127006in}}%
\pgfpathlineto{\pgfqpoint{1.798219in}{1.119771in}}%
\pgfpathlineto{\pgfqpoint{1.795044in}{1.123031in}}%
\pgfpathlineto{\pgfqpoint{1.791872in}{1.126434in}}%
\pgfpathlineto{\pgfqpoint{1.788700in}{1.129976in}}%
\pgfpathlineto{\pgfqpoint{1.785530in}{1.133652in}}%
\pgfpathlineto{\pgfqpoint{1.781898in}{1.140667in}}%
\pgfpathlineto{\pgfqpoint{1.777848in}{1.147611in}}%
\pgfpathlineto{\pgfqpoint{1.773386in}{1.154479in}}%
\pgfpathlineto{\pgfqpoint{1.768518in}{1.161265in}}%
\pgfpathclose%
\pgfusepath{fill}%
\end{pgfscope}%
\begin{pgfscope}%
\pgfpathrectangle{\pgfqpoint{0.329460in}{0.284240in}}{\pgfqpoint{1.989680in}{1.989680in}}%
\pgfusepath{clip}%
\pgfsetbuttcap%
\pgfsetroundjoin%
\definecolor{currentfill}{rgb}{0.179019,0.433756,0.557430}%
\pgfsetfillcolor{currentfill}%
\pgfsetlinewidth{0.000000pt}%
\definecolor{currentstroke}{rgb}{0.000000,0.000000,0.000000}%
\pgfsetstrokecolor{currentstroke}%
\pgfsetdash{}{0pt}%
\pgfpathmoveto{\pgfqpoint{1.129614in}{1.411778in}}%
\pgfpathlineto{\pgfqpoint{1.127431in}{1.406276in}}%
\pgfpathlineto{\pgfqpoint{1.125249in}{1.400799in}}%
\pgfpathlineto{\pgfqpoint{1.123069in}{1.395349in}}%
\pgfpathlineto{\pgfqpoint{1.120892in}{1.389931in}}%
\pgfpathlineto{\pgfqpoint{1.128845in}{1.393454in}}%
\pgfpathlineto{\pgfqpoint{1.137006in}{1.396851in}}%
\pgfpathlineto{\pgfqpoint{1.145368in}{1.400117in}}%
\pgfpathlineto{\pgfqpoint{1.153921in}{1.403250in}}%
\pgfpathlineto{\pgfqpoint{1.155782in}{1.408531in}}%
\pgfpathlineto{\pgfqpoint{1.157645in}{1.413843in}}%
\pgfpathlineto{\pgfqpoint{1.159510in}{1.419183in}}%
\pgfpathlineto{\pgfqpoint{1.161377in}{1.424548in}}%
\pgfpathlineto{\pgfqpoint{1.153150in}{1.421544in}}%
\pgfpathlineto{\pgfqpoint{1.145110in}{1.418412in}}%
\pgfpathlineto{\pgfqpoint{1.137262in}{1.415156in}}%
\pgfpathlineto{\pgfqpoint{1.129614in}{1.411778in}}%
\pgfpathclose%
\pgfusepath{fill}%
\end{pgfscope}%
\begin{pgfscope}%
\pgfpathrectangle{\pgfqpoint{0.329460in}{0.284240in}}{\pgfqpoint{1.989680in}{1.989680in}}%
\pgfusepath{clip}%
\pgfsetbuttcap%
\pgfsetroundjoin%
\definecolor{currentfill}{rgb}{0.163625,0.471133,0.558148}%
\pgfsetfillcolor{currentfill}%
\pgfsetlinewidth{0.000000pt}%
\definecolor{currentstroke}{rgb}{0.000000,0.000000,0.000000}%
\pgfsetstrokecolor{currentstroke}%
\pgfsetdash{}{0pt}%
\pgfpathmoveto{\pgfqpoint{1.492589in}{1.458426in}}%
\pgfpathlineto{\pgfqpoint{1.494040in}{1.453124in}}%
\pgfpathlineto{\pgfqpoint{1.495490in}{1.447836in}}%
\pgfpathlineto{\pgfqpoint{1.496937in}{1.442566in}}%
\pgfpathlineto{\pgfqpoint{1.498384in}{1.437315in}}%
\pgfpathlineto{\pgfqpoint{1.507398in}{1.434970in}}%
\pgfpathlineto{\pgfqpoint{1.516268in}{1.432486in}}%
\pgfpathlineto{\pgfqpoint{1.524983in}{1.429865in}}%
\pgfpathlineto{\pgfqpoint{1.533537in}{1.427109in}}%
\pgfpathlineto{\pgfqpoint{1.531743in}{1.432469in}}%
\pgfpathlineto{\pgfqpoint{1.529946in}{1.437848in}}%
\pgfpathlineto{\pgfqpoint{1.528148in}{1.443245in}}%
\pgfpathlineto{\pgfqpoint{1.526347in}{1.448657in}}%
\pgfpathlineto{\pgfqpoint{1.518133in}{1.451295in}}%
\pgfpathlineto{\pgfqpoint{1.509764in}{1.453804in}}%
\pgfpathlineto{\pgfqpoint{1.501246in}{1.456182in}}%
\pgfpathlineto{\pgfqpoint{1.492589in}{1.458426in}}%
\pgfpathclose%
\pgfusepath{fill}%
\end{pgfscope}%
\begin{pgfscope}%
\pgfpathrectangle{\pgfqpoint{0.329460in}{0.284240in}}{\pgfqpoint{1.989680in}{1.989680in}}%
\pgfusepath{clip}%
\pgfsetbuttcap%
\pgfsetroundjoin%
\definecolor{currentfill}{rgb}{0.147607,0.511733,0.557049}%
\pgfsetfillcolor{currentfill}%
\pgfsetlinewidth{0.000000pt}%
\definecolor{currentstroke}{rgb}{0.000000,0.000000,0.000000}%
\pgfsetstrokecolor{currentstroke}%
\pgfsetdash{}{0pt}%
\pgfpathmoveto{\pgfqpoint{1.425586in}{1.491042in}}%
\pgfpathlineto{\pgfqpoint{1.426385in}{1.485838in}}%
\pgfpathlineto{\pgfqpoint{1.427183in}{1.480638in}}%
\pgfpathlineto{\pgfqpoint{1.427979in}{1.475446in}}%
\pgfpathlineto{\pgfqpoint{1.428775in}{1.470263in}}%
\pgfpathlineto{\pgfqpoint{1.438176in}{1.468998in}}%
\pgfpathlineto{\pgfqpoint{1.447496in}{1.467589in}}%
\pgfpathlineto{\pgfqpoint{1.456727in}{1.466037in}}%
\pgfpathlineto{\pgfqpoint{1.455644in}{1.471266in}}%
\pgfpathlineto{\pgfqpoint{1.454560in}{1.476506in}}%
\pgfpathlineto{\pgfqpoint{1.453474in}{1.481752in}}%
\pgfpathlineto{\pgfqpoint{1.452387in}{1.487003in}}%
\pgfpathlineto{\pgfqpoint{1.443537in}{1.488486in}}%
\pgfpathlineto{\pgfqpoint{1.434600in}{1.489833in}}%
\pgfpathlineto{\pgfqpoint{1.425586in}{1.491042in}}%
\pgfpathclose%
\pgfusepath{fill}%
\end{pgfscope}%
\begin{pgfscope}%
\pgfpathrectangle{\pgfqpoint{0.329460in}{0.284240in}}{\pgfqpoint{1.989680in}{1.989680in}}%
\pgfusepath{clip}%
\pgfsetbuttcap%
\pgfsetroundjoin%
\definecolor{currentfill}{rgb}{0.172719,0.448791,0.557885}%
\pgfsetfillcolor{currentfill}%
\pgfsetlinewidth{0.000000pt}%
\definecolor{currentstroke}{rgb}{0.000000,0.000000,0.000000}%
\pgfsetstrokecolor{currentstroke}%
\pgfsetdash{}{0pt}%
\pgfpathmoveto{\pgfqpoint{1.937302in}{1.417064in}}%
\pgfpathlineto{\pgfqpoint{1.940636in}{1.431007in}}%
\pgfpathlineto{\pgfqpoint{1.943988in}{1.445396in}}%
\pgfpathlineto{\pgfqpoint{1.947358in}{1.460240in}}%
\pgfpathlineto{\pgfqpoint{1.957405in}{1.450977in}}%
\pgfpathlineto{\pgfqpoint{1.966890in}{1.441549in}}%
\pgfpathlineto{\pgfqpoint{1.975804in}{1.431964in}}%
\pgfpathlineto{\pgfqpoint{1.984134in}{1.422230in}}%
\pgfpathlineto{\pgfqpoint{1.980534in}{1.407510in}}%
\pgfpathlineto{\pgfqpoint{1.976953in}{1.393247in}}%
\pgfpathlineto{\pgfqpoint{1.973392in}{1.379433in}}%
\pgfpathlineto{\pgfqpoint{1.965219in}{1.389069in}}%
\pgfpathlineto{\pgfqpoint{1.956473in}{1.398558in}}%
\pgfpathlineto{\pgfqpoint{1.947163in}{1.407893in}}%
\pgfpathlineto{\pgfqpoint{1.937302in}{1.417064in}}%
\pgfpathclose%
\pgfusepath{fill}%
\end{pgfscope}%
\begin{pgfscope}%
\pgfpathrectangle{\pgfqpoint{0.329460in}{0.284240in}}{\pgfqpoint{1.989680in}{1.989680in}}%
\pgfusepath{clip}%
\pgfsetbuttcap%
\pgfsetroundjoin%
\definecolor{currentfill}{rgb}{0.147607,0.511733,0.557049}%
\pgfsetfillcolor{currentfill}%
\pgfsetlinewidth{0.000000pt}%
\definecolor{currentstroke}{rgb}{0.000000,0.000000,0.000000}%
\pgfsetstrokecolor{currentstroke}%
\pgfsetdash{}{0pt}%
\pgfpathmoveto{\pgfqpoint{1.242199in}{1.485571in}}%
\pgfpathlineto{\pgfqpoint{1.241028in}{1.480304in}}%
\pgfpathlineto{\pgfqpoint{1.239859in}{1.475040in}}%
\pgfpathlineto{\pgfqpoint{1.238691in}{1.469784in}}%
\pgfpathlineto{\pgfqpoint{1.237524in}{1.464538in}}%
\pgfpathlineto{\pgfqpoint{1.246669in}{1.466216in}}%
\pgfpathlineto{\pgfqpoint{1.255910in}{1.467752in}}%
\pgfpathlineto{\pgfqpoint{1.265240in}{1.469145in}}%
\pgfpathlineto{\pgfqpoint{1.274649in}{1.470394in}}%
\pgfpathlineto{\pgfqpoint{1.275434in}{1.475576in}}%
\pgfpathlineto{\pgfqpoint{1.276220in}{1.480767in}}%
\pgfpathlineto{\pgfqpoint{1.277007in}{1.485965in}}%
\pgfpathlineto{\pgfqpoint{1.277795in}{1.491167in}}%
\pgfpathlineto{\pgfqpoint{1.268773in}{1.489974in}}%
\pgfpathlineto{\pgfqpoint{1.259827in}{1.488643in}}%
\pgfpathlineto{\pgfqpoint{1.250967in}{1.487175in}}%
\pgfpathlineto{\pgfqpoint{1.242199in}{1.485571in}}%
\pgfpathclose%
\pgfusepath{fill}%
\end{pgfscope}%
\begin{pgfscope}%
\pgfpathrectangle{\pgfqpoint{0.329460in}{0.284240in}}{\pgfqpoint{1.989680in}{1.989680in}}%
\pgfusepath{clip}%
\pgfsetbuttcap%
\pgfsetroundjoin%
\definecolor{currentfill}{rgb}{0.195860,0.395433,0.555276}%
\pgfsetfillcolor{currentfill}%
\pgfsetlinewidth{0.000000pt}%
\definecolor{currentstroke}{rgb}{0.000000,0.000000,0.000000}%
\pgfsetstrokecolor{currentstroke}%
\pgfsetdash{}{0pt}%
\pgfpathmoveto{\pgfqpoint{1.091314in}{1.374625in}}%
\pgfpathlineto{\pgfqpoint{1.088853in}{1.369081in}}%
\pgfpathlineto{\pgfqpoint{1.086394in}{1.363572in}}%
\pgfpathlineto{\pgfqpoint{1.083938in}{1.358102in}}%
\pgfpathlineto{\pgfqpoint{1.081483in}{1.352674in}}%
\pgfpathlineto{\pgfqpoint{1.088800in}{1.356841in}}%
\pgfpathlineto{\pgfqpoint{1.096365in}{1.360889in}}%
\pgfpathlineto{\pgfqpoint{1.104168in}{1.364815in}}%
\pgfpathlineto{\pgfqpoint{1.112203in}{1.368614in}}%
\pgfpathlineto{\pgfqpoint{1.114372in}{1.373884in}}%
\pgfpathlineto{\pgfqpoint{1.116543in}{1.379195in}}%
\pgfpathlineto{\pgfqpoint{1.118717in}{1.384545in}}%
\pgfpathlineto{\pgfqpoint{1.120892in}{1.389931in}}%
\pgfpathlineto{\pgfqpoint{1.113155in}{1.386282in}}%
\pgfpathlineto{\pgfqpoint{1.105641in}{1.382513in}}%
\pgfpathlineto{\pgfqpoint{1.098358in}{1.378626in}}%
\pgfpathlineto{\pgfqpoint{1.091314in}{1.374625in}}%
\pgfpathclose%
\pgfusepath{fill}%
\end{pgfscope}%
\begin{pgfscope}%
\pgfpathrectangle{\pgfqpoint{0.329460in}{0.284240in}}{\pgfqpoint{1.989680in}{1.989680in}}%
\pgfusepath{clip}%
\pgfsetbuttcap%
\pgfsetroundjoin%
\definecolor{currentfill}{rgb}{0.212395,0.359683,0.551710}%
\pgfsetfillcolor{currentfill}%
\pgfsetlinewidth{0.000000pt}%
\definecolor{currentstroke}{rgb}{0.000000,0.000000,0.000000}%
\pgfsetstrokecolor{currentstroke}%
\pgfsetdash{}{0pt}%
\pgfpathmoveto{\pgfqpoint{1.614400in}{1.356384in}}%
\pgfpathlineto{\pgfqpoint{1.616792in}{1.351037in}}%
\pgfpathlineto{\pgfqpoint{1.619182in}{1.345736in}}%
\pgfpathlineto{\pgfqpoint{1.621571in}{1.340486in}}%
\pgfpathlineto{\pgfqpoint{1.623957in}{1.335288in}}%
\pgfpathlineto{\pgfqpoint{1.631519in}{1.330940in}}%
\pgfpathlineto{\pgfqpoint{1.638816in}{1.326473in}}%
\pgfpathlineto{\pgfqpoint{1.645842in}{1.321890in}}%
\pgfpathlineto{\pgfqpoint{1.652590in}{1.317195in}}%
\pgfpathlineto{\pgfqpoint{1.649945in}{1.322567in}}%
\pgfpathlineto{\pgfqpoint{1.647299in}{1.327992in}}%
\pgfpathlineto{\pgfqpoint{1.644650in}{1.333466in}}%
\pgfpathlineto{\pgfqpoint{1.642000in}{1.338987in}}%
\pgfpathlineto{\pgfqpoint{1.635496in}{1.343501in}}%
\pgfpathlineto{\pgfqpoint{1.628724in}{1.347908in}}%
\pgfpathlineto{\pgfqpoint{1.621690in}{1.352204in}}%
\pgfpathlineto{\pgfqpoint{1.614400in}{1.356384in}}%
\pgfpathclose%
\pgfusepath{fill}%
\end{pgfscope}%
\begin{pgfscope}%
\pgfpathrectangle{\pgfqpoint{0.329460in}{0.284240in}}{\pgfqpoint{1.989680in}{1.989680in}}%
\pgfusepath{clip}%
\pgfsetbuttcap%
\pgfsetroundjoin%
\definecolor{currentfill}{rgb}{0.276194,0.190074,0.493001}%
\pgfsetfillcolor{currentfill}%
\pgfsetlinewidth{0.000000pt}%
\definecolor{currentstroke}{rgb}{0.000000,0.000000,0.000000}%
\pgfsetstrokecolor{currentstroke}%
\pgfsetdash{}{0pt}%
\pgfpathmoveto{\pgfqpoint{0.755568in}{1.168100in}}%
\pgfpathlineto{\pgfqpoint{0.752062in}{1.175428in}}%
\pgfpathlineto{\pgfqpoint{0.748542in}{1.183103in}}%
\pgfpathlineto{\pgfqpoint{0.745008in}{1.191132in}}%
\pgfpathlineto{\pgfqpoint{0.741460in}{1.199521in}}%
\pgfpathlineto{\pgfqpoint{0.746267in}{1.209266in}}%
\pgfpathlineto{\pgfqpoint{0.751655in}{1.218915in}}%
\pgfpathlineto{\pgfqpoint{0.757616in}{1.228458in}}%
\pgfpathlineto{\pgfqpoint{0.764142in}{1.237887in}}%
\pgfpathlineto{\pgfqpoint{0.767537in}{1.229328in}}%
\pgfpathlineto{\pgfqpoint{0.770919in}{1.221126in}}%
\pgfpathlineto{\pgfqpoint{0.774288in}{1.213275in}}%
\pgfpathlineto{\pgfqpoint{0.777645in}{1.205771in}}%
\pgfpathlineto{\pgfqpoint{0.771289in}{1.196512in}}%
\pgfpathlineto{\pgfqpoint{0.765486in}{1.187141in}}%
\pgfpathlineto{\pgfqpoint{0.760243in}{1.177668in}}%
\pgfpathlineto{\pgfqpoint{0.755568in}{1.168100in}}%
\pgfpathclose%
\pgfusepath{fill}%
\end{pgfscope}%
\begin{pgfscope}%
\pgfpathrectangle{\pgfqpoint{0.329460in}{0.284240in}}{\pgfqpoint{1.989680in}{1.989680in}}%
\pgfusepath{clip}%
\pgfsetbuttcap%
\pgfsetroundjoin%
\definecolor{currentfill}{rgb}{0.248629,0.278775,0.534556}%
\pgfsetfillcolor{currentfill}%
\pgfsetlinewidth{0.000000pt}%
\definecolor{currentstroke}{rgb}{0.000000,0.000000,0.000000}%
\pgfsetstrokecolor{currentstroke}%
\pgfsetdash{}{0pt}%
\pgfpathmoveto{\pgfqpoint{1.009412in}{1.270925in}}%
\pgfpathlineto{\pgfqpoint{1.006511in}{1.265624in}}%
\pgfpathlineto{\pgfqpoint{1.003612in}{1.260393in}}%
\pgfpathlineto{\pgfqpoint{1.000714in}{1.255235in}}%
\pgfpathlineto{\pgfqpoint{0.997817in}{1.250153in}}%
\pgfpathlineto{\pgfqpoint{1.003499in}{1.255735in}}%
\pgfpathlineto{\pgfqpoint{1.009512in}{1.261218in}}%
\pgfpathlineto{\pgfqpoint{1.015850in}{1.266599in}}%
\pgfpathlineto{\pgfqpoint{1.022506in}{1.271872in}}%
\pgfpathlineto{\pgfqpoint{1.025192in}{1.276758in}}%
\pgfpathlineto{\pgfqpoint{1.027878in}{1.281721in}}%
\pgfpathlineto{\pgfqpoint{1.030566in}{1.286757in}}%
\pgfpathlineto{\pgfqpoint{1.033255in}{1.291864in}}%
\pgfpathlineto{\pgfqpoint{1.026825in}{1.286780in}}%
\pgfpathlineto{\pgfqpoint{1.020704in}{1.281592in}}%
\pgfpathlineto{\pgfqpoint{1.014897in}{1.276306in}}%
\pgfpathlineto{\pgfqpoint{1.009412in}{1.270925in}}%
\pgfpathclose%
\pgfusepath{fill}%
\end{pgfscope}%
\begin{pgfscope}%
\pgfpathrectangle{\pgfqpoint{0.329460in}{0.284240in}}{\pgfqpoint{1.989680in}{1.989680in}}%
\pgfusepath{clip}%
\pgfsetbuttcap%
\pgfsetroundjoin%
\definecolor{currentfill}{rgb}{0.260571,0.246922,0.522828}%
\pgfsetfillcolor{currentfill}%
\pgfsetlinewidth{0.000000pt}%
\definecolor{currentstroke}{rgb}{0.000000,0.000000,0.000000}%
\pgfsetstrokecolor{currentstroke}%
\pgfsetdash{}{0pt}%
\pgfpathmoveto{\pgfqpoint{1.931966in}{1.246166in}}%
\pgfpathlineto{\pgfqpoint{1.935334in}{1.255125in}}%
\pgfpathlineto{\pgfqpoint{1.938717in}{1.264454in}}%
\pgfpathlineto{\pgfqpoint{1.942113in}{1.274158in}}%
\pgfpathlineto{\pgfqpoint{1.945524in}{1.284243in}}%
\pgfpathlineto{\pgfqpoint{1.952730in}{1.274762in}}%
\pgfpathlineto{\pgfqpoint{1.959366in}{1.265157in}}%
\pgfpathlineto{\pgfqpoint{1.965424in}{1.255436in}}%
\pgfpathlineto{\pgfqpoint{1.970894in}{1.245609in}}%
\pgfpathlineto{\pgfqpoint{1.967316in}{1.235684in}}%
\pgfpathlineto{\pgfqpoint{1.963752in}{1.226144in}}%
\pgfpathlineto{\pgfqpoint{1.960205in}{1.216980in}}%
\pgfpathlineto{\pgfqpoint{1.956671in}{1.208188in}}%
\pgfpathlineto{\pgfqpoint{1.951348in}{1.217848in}}%
\pgfpathlineto{\pgfqpoint{1.945450in}{1.227403in}}%
\pgfpathlineto{\pgfqpoint{1.938986in}{1.236846in}}%
\pgfpathlineto{\pgfqpoint{1.931966in}{1.246166in}}%
\pgfpathclose%
\pgfusepath{fill}%
\end{pgfscope}%
\begin{pgfscope}%
\pgfpathrectangle{\pgfqpoint{0.329460in}{0.284240in}}{\pgfqpoint{1.989680in}{1.989680in}}%
\pgfusepath{clip}%
\pgfsetbuttcap%
\pgfsetroundjoin%
\definecolor{currentfill}{rgb}{0.163625,0.471133,0.558148}%
\pgfsetfillcolor{currentfill}%
\pgfsetlinewidth{0.000000pt}%
\definecolor{currentstroke}{rgb}{0.000000,0.000000,0.000000}%
\pgfsetstrokecolor{currentstroke}%
\pgfsetdash{}{0pt}%
\pgfpathmoveto{\pgfqpoint{1.168864in}{1.446206in}}%
\pgfpathlineto{\pgfqpoint{1.166989in}{1.440767in}}%
\pgfpathlineto{\pgfqpoint{1.165116in}{1.435342in}}%
\pgfpathlineto{\pgfqpoint{1.163245in}{1.429935in}}%
\pgfpathlineto{\pgfqpoint{1.161377in}{1.424548in}}%
\pgfpathlineto{\pgfqpoint{1.169780in}{1.427422in}}%
\pgfpathlineto{\pgfqpoint{1.178352in}{1.430163in}}%
\pgfpathlineto{\pgfqpoint{1.187086in}{1.432769in}}%
\pgfpathlineto{\pgfqpoint{1.195972in}{1.435238in}}%
\pgfpathlineto{\pgfqpoint{1.197497in}{1.440511in}}%
\pgfpathlineto{\pgfqpoint{1.199024in}{1.445803in}}%
\pgfpathlineto{\pgfqpoint{1.200553in}{1.451113in}}%
\pgfpathlineto{\pgfqpoint{1.202084in}{1.456438in}}%
\pgfpathlineto{\pgfqpoint{1.193551in}{1.454075in}}%
\pgfpathlineto{\pgfqpoint{1.185164in}{1.451580in}}%
\pgfpathlineto{\pgfqpoint{1.176933in}{1.448956in}}%
\pgfpathlineto{\pgfqpoint{1.168864in}{1.446206in}}%
\pgfpathclose%
\pgfusepath{fill}%
\end{pgfscope}%
\begin{pgfscope}%
\pgfpathrectangle{\pgfqpoint{0.329460in}{0.284240in}}{\pgfqpoint{1.989680in}{1.989680in}}%
\pgfusepath{clip}%
\pgfsetbuttcap%
\pgfsetroundjoin%
\definecolor{currentfill}{rgb}{0.274128,0.199721,0.498911}%
\pgfsetfillcolor{currentfill}%
\pgfsetlinewidth{0.000000pt}%
\definecolor{currentstroke}{rgb}{0.000000,0.000000,0.000000}%
\pgfsetstrokecolor{currentstroke}%
\pgfsetdash{}{0pt}%
\pgfpathmoveto{\pgfqpoint{0.966267in}{1.206611in}}%
\pgfpathlineto{\pgfqpoint{0.963205in}{1.201748in}}%
\pgfpathlineto{\pgfqpoint{0.960142in}{1.196980in}}%
\pgfpathlineto{\pgfqpoint{0.957080in}{1.192310in}}%
\pgfpathlineto{\pgfqpoint{0.954018in}{1.187741in}}%
\pgfpathlineto{\pgfqpoint{0.958621in}{1.194091in}}%
\pgfpathlineto{\pgfqpoint{0.963601in}{1.200357in}}%
\pgfpathlineto{\pgfqpoint{0.968953in}{1.206535in}}%
\pgfpathlineto{\pgfqpoint{0.974670in}{1.212617in}}%
\pgfpathlineto{\pgfqpoint{0.977562in}{1.216977in}}%
\pgfpathlineto{\pgfqpoint{0.980454in}{1.221438in}}%
\pgfpathlineto{\pgfqpoint{0.983346in}{1.225997in}}%
\pgfpathlineto{\pgfqpoint{0.986239in}{1.230651in}}%
\pgfpathlineto{\pgfqpoint{0.980708in}{1.224773in}}%
\pgfpathlineto{\pgfqpoint{0.975532in}{1.218803in}}%
\pgfpathlineto{\pgfqpoint{0.970717in}{1.212747in}}%
\pgfpathlineto{\pgfqpoint{0.966267in}{1.206611in}}%
\pgfpathclose%
\pgfusepath{fill}%
\end{pgfscope}%
\begin{pgfscope}%
\pgfpathrectangle{\pgfqpoint{0.329460in}{0.284240in}}{\pgfqpoint{1.989680in}{1.989680in}}%
\pgfusepath{clip}%
\pgfsetbuttcap%
\pgfsetroundjoin%
\definecolor{currentfill}{rgb}{0.282327,0.094955,0.417331}%
\pgfsetfillcolor{currentfill}%
\pgfsetlinewidth{0.000000pt}%
\definecolor{currentstroke}{rgb}{0.000000,0.000000,0.000000}%
\pgfsetstrokecolor{currentstroke}%
\pgfsetdash{}{0pt}%
\pgfpathmoveto{\pgfqpoint{0.913970in}{1.127364in}}%
\pgfpathlineto{\pgfqpoint{0.910776in}{1.123638in}}%
\pgfpathlineto{\pgfqpoint{0.907581in}{1.120047in}}%
\pgfpathlineto{\pgfqpoint{0.904383in}{1.116594in}}%
\pgfpathlineto{\pgfqpoint{0.901185in}{1.113284in}}%
\pgfpathlineto{\pgfqpoint{0.904552in}{1.120578in}}%
\pgfpathlineto{\pgfqpoint{0.908353in}{1.127805in}}%
\pgfpathlineto{\pgfqpoint{0.912582in}{1.134959in}}%
\pgfpathlineto{\pgfqpoint{0.917234in}{1.142033in}}%
\pgfpathlineto{\pgfqpoint{0.920305in}{1.145125in}}%
\pgfpathlineto{\pgfqpoint{0.923375in}{1.148358in}}%
\pgfpathlineto{\pgfqpoint{0.926443in}{1.151731in}}%
\pgfpathlineto{\pgfqpoint{0.929510in}{1.155238in}}%
\pgfpathlineto{\pgfqpoint{0.925003in}{1.148378in}}%
\pgfpathlineto{\pgfqpoint{0.920907in}{1.141442in}}%
\pgfpathlineto{\pgfqpoint{0.917228in}{1.134435in}}%
\pgfpathlineto{\pgfqpoint{0.913970in}{1.127364in}}%
\pgfpathclose%
\pgfusepath{fill}%
\end{pgfscope}%
\begin{pgfscope}%
\pgfpathrectangle{\pgfqpoint{0.329460in}{0.284240in}}{\pgfqpoint{1.989680in}{1.989680in}}%
\pgfusepath{clip}%
\pgfsetbuttcap%
\pgfsetroundjoin%
\definecolor{currentfill}{rgb}{0.212395,0.359683,0.551710}%
\pgfsetfillcolor{currentfill}%
\pgfsetlinewidth{0.000000pt}%
\definecolor{currentstroke}{rgb}{0.000000,0.000000,0.000000}%
\pgfsetstrokecolor{currentstroke}%
\pgfsetdash{}{0pt}%
\pgfpathmoveto{\pgfqpoint{1.054826in}{1.334888in}}%
\pgfpathlineto{\pgfqpoint{1.052124in}{1.329325in}}%
\pgfpathlineto{\pgfqpoint{1.049423in}{1.323810in}}%
\pgfpathlineto{\pgfqpoint{1.046724in}{1.318344in}}%
\pgfpathlineto{\pgfqpoint{1.044027in}{1.312931in}}%
\pgfpathlineto{\pgfqpoint{1.050521in}{1.317722in}}%
\pgfpathlineto{\pgfqpoint{1.057300in}{1.322405in}}%
\pgfpathlineto{\pgfqpoint{1.064357in}{1.326975in}}%
\pgfpathlineto{\pgfqpoint{1.071684in}{1.331429in}}%
\pgfpathlineto{\pgfqpoint{1.074131in}{1.336664in}}%
\pgfpathlineto{\pgfqpoint{1.076580in}{1.341952in}}%
\pgfpathlineto{\pgfqpoint{1.079030in}{1.347290in}}%
\pgfpathlineto{\pgfqpoint{1.081483in}{1.352674in}}%
\pgfpathlineto{\pgfqpoint{1.074420in}{1.348391in}}%
\pgfpathlineto{\pgfqpoint{1.067618in}{1.343996in}}%
\pgfpathlineto{\pgfqpoint{1.061085in}{1.339494in}}%
\pgfpathlineto{\pgfqpoint{1.054826in}{1.334888in}}%
\pgfpathclose%
\pgfusepath{fill}%
\end{pgfscope}%
\begin{pgfscope}%
\pgfpathrectangle{\pgfqpoint{0.329460in}{0.284240in}}{\pgfqpoint{1.989680in}{1.989680in}}%
\pgfusepath{clip}%
\pgfsetbuttcap%
\pgfsetroundjoin%
\definecolor{currentfill}{rgb}{0.267004,0.004874,0.329415}%
\pgfsetfillcolor{currentfill}%
\pgfsetlinewidth{0.000000pt}%
\definecolor{currentstroke}{rgb}{0.000000,0.000000,0.000000}%
\pgfsetstrokecolor{currentstroke}%
\pgfsetdash{}{0pt}%
\pgfpathmoveto{\pgfqpoint{1.862352in}{1.090112in}}%
\pgfpathlineto{\pgfqpoint{1.865606in}{1.090707in}}%
\pgfpathlineto{\pgfqpoint{1.868866in}{1.091529in}}%
\pgfpathlineto{\pgfqpoint{1.872133in}{1.092584in}}%
\pgfpathlineto{\pgfqpoint{1.875406in}{1.093876in}}%
\pgfpathlineto{\pgfqpoint{1.879402in}{1.085278in}}%
\pgfpathlineto{\pgfqpoint{1.882886in}{1.076608in}}%
\pgfpathlineto{\pgfqpoint{1.885851in}{1.067873in}}%
\pgfpathlineto{\pgfqpoint{1.888294in}{1.059082in}}%
\pgfpathlineto{\pgfqpoint{1.884922in}{1.058002in}}%
\pgfpathlineto{\pgfqpoint{1.881558in}{1.057160in}}%
\pgfpathlineto{\pgfqpoint{1.878200in}{1.056550in}}%
\pgfpathlineto{\pgfqpoint{1.874848in}{1.056170in}}%
\pgfpathlineto{\pgfqpoint{1.872485in}{1.064745in}}%
\pgfpathlineto{\pgfqpoint{1.869611in}{1.073266in}}%
\pgfpathlineto{\pgfqpoint{1.866232in}{1.081724in}}%
\pgfpathlineto{\pgfqpoint{1.862352in}{1.090112in}}%
\pgfpathclose%
\pgfusepath{fill}%
\end{pgfscope}%
\begin{pgfscope}%
\pgfpathrectangle{\pgfqpoint{0.329460in}{0.284240in}}{\pgfqpoint{1.989680in}{1.989680in}}%
\pgfusepath{clip}%
\pgfsetbuttcap%
\pgfsetroundjoin%
\definecolor{currentfill}{rgb}{0.283072,0.130895,0.449241}%
\pgfsetfillcolor{currentfill}%
\pgfsetlinewidth{0.000000pt}%
\definecolor{currentstroke}{rgb}{0.000000,0.000000,0.000000}%
\pgfsetstrokecolor{currentstroke}%
\pgfsetdash{}{0pt}%
\pgfpathmoveto{\pgfqpoint{1.756398in}{1.176384in}}%
\pgfpathlineto{\pgfqpoint{1.759427in}{1.172419in}}%
\pgfpathlineto{\pgfqpoint{1.762456in}{1.168576in}}%
\pgfpathlineto{\pgfqpoint{1.765487in}{1.164856in}}%
\pgfpathlineto{\pgfqpoint{1.768518in}{1.161265in}}%
\pgfpathlineto{\pgfqpoint{1.773386in}{1.154479in}}%
\pgfpathlineto{\pgfqpoint{1.777848in}{1.147611in}}%
\pgfpathlineto{\pgfqpoint{1.781898in}{1.140667in}}%
\pgfpathlineto{\pgfqpoint{1.785530in}{1.133652in}}%
\pgfpathlineto{\pgfqpoint{1.782362in}{1.137461in}}%
\pgfpathlineto{\pgfqpoint{1.779195in}{1.141397in}}%
\pgfpathlineto{\pgfqpoint{1.776028in}{1.145458in}}%
\pgfpathlineto{\pgfqpoint{1.772862in}{1.149641in}}%
\pgfpathlineto{\pgfqpoint{1.769349in}{1.156434in}}%
\pgfpathlineto{\pgfqpoint{1.765431in}{1.163160in}}%
\pgfpathlineto{\pgfqpoint{1.761112in}{1.169812in}}%
\pgfpathlineto{\pgfqpoint{1.756398in}{1.176384in}}%
\pgfpathclose%
\pgfusepath{fill}%
\end{pgfscope}%
\begin{pgfscope}%
\pgfpathrectangle{\pgfqpoint{0.329460in}{0.284240in}}{\pgfqpoint{1.989680in}{1.989680in}}%
\pgfusepath{clip}%
\pgfsetbuttcap%
\pgfsetroundjoin%
\definecolor{currentfill}{rgb}{0.147607,0.511733,0.557049}%
\pgfsetfillcolor{currentfill}%
\pgfsetlinewidth{0.000000pt}%
\definecolor{currentstroke}{rgb}{0.000000,0.000000,0.000000}%
\pgfsetstrokecolor{currentstroke}%
\pgfsetdash{}{0pt}%
\pgfpathmoveto{\pgfqpoint{1.452387in}{1.487003in}}%
\pgfpathlineto{\pgfqpoint{1.453474in}{1.481752in}}%
\pgfpathlineto{\pgfqpoint{1.454560in}{1.476506in}}%
\pgfpathlineto{\pgfqpoint{1.455644in}{1.471266in}}%
\pgfpathlineto{\pgfqpoint{1.456727in}{1.466037in}}%
\pgfpathlineto{\pgfqpoint{1.465861in}{1.464343in}}%
\pgfpathlineto{\pgfqpoint{1.474888in}{1.462509in}}%
\pgfpathlineto{\pgfqpoint{1.483800in}{1.460536in}}%
\pgfpathlineto{\pgfqpoint{1.492589in}{1.458426in}}%
\pgfpathlineto{\pgfqpoint{1.491136in}{1.463741in}}%
\pgfpathlineto{\pgfqpoint{1.489681in}{1.469065in}}%
\pgfpathlineto{\pgfqpoint{1.488225in}{1.474396in}}%
\pgfpathlineto{\pgfqpoint{1.486767in}{1.479732in}}%
\pgfpathlineto{\pgfqpoint{1.478342in}{1.481747in}}%
\pgfpathlineto{\pgfqpoint{1.469798in}{1.483632in}}%
\pgfpathlineto{\pgfqpoint{1.461144in}{1.485385in}}%
\pgfpathlineto{\pgfqpoint{1.452387in}{1.487003in}}%
\pgfpathclose%
\pgfusepath{fill}%
\end{pgfscope}%
\begin{pgfscope}%
\pgfpathrectangle{\pgfqpoint{0.329460in}{0.284240in}}{\pgfqpoint{1.989680in}{1.989680in}}%
\pgfusepath{clip}%
\pgfsetbuttcap%
\pgfsetroundjoin%
\definecolor{currentfill}{rgb}{0.268510,0.009605,0.335427}%
\pgfsetfillcolor{currentfill}%
\pgfsetlinewidth{0.000000pt}%
\definecolor{currentstroke}{rgb}{0.000000,0.000000,0.000000}%
\pgfsetstrokecolor{currentstroke}%
\pgfsetdash{}{0pt}%
\pgfpathmoveto{\pgfqpoint{1.875406in}{1.093876in}}%
\pgfpathlineto{\pgfqpoint{1.878686in}{1.095408in}}%
\pgfpathlineto{\pgfqpoint{1.881973in}{1.097186in}}%
\pgfpathlineto{\pgfqpoint{1.885268in}{1.099214in}}%
\pgfpathlineto{\pgfqpoint{1.888570in}{1.101497in}}%
\pgfpathlineto{\pgfqpoint{1.892684in}{1.092694in}}%
\pgfpathlineto{\pgfqpoint{1.896274in}{1.083816in}}%
\pgfpathlineto{\pgfqpoint{1.899333in}{1.074871in}}%
\pgfpathlineto{\pgfqpoint{1.901856in}{1.065867in}}%
\pgfpathlineto{\pgfqpoint{1.898454in}{1.063792in}}%
\pgfpathlineto{\pgfqpoint{1.895059in}{1.061973in}}%
\pgfpathlineto{\pgfqpoint{1.891673in}{1.060404in}}%
\pgfpathlineto{\pgfqpoint{1.888294in}{1.059082in}}%
\pgfpathlineto{\pgfqpoint{1.885851in}{1.067873in}}%
\pgfpathlineto{\pgfqpoint{1.882886in}{1.076608in}}%
\pgfpathlineto{\pgfqpoint{1.879402in}{1.085278in}}%
\pgfpathlineto{\pgfqpoint{1.875406in}{1.093876in}}%
\pgfpathclose%
\pgfusepath{fill}%
\end{pgfscope}%
\begin{pgfscope}%
\pgfpathrectangle{\pgfqpoint{0.329460in}{0.284240in}}{\pgfqpoint{1.989680in}{1.989680in}}%
\pgfusepath{clip}%
\pgfsetbuttcap%
\pgfsetroundjoin%
\definecolor{currentfill}{rgb}{0.267004,0.004874,0.329415}%
\pgfsetfillcolor{currentfill}%
\pgfsetlinewidth{0.000000pt}%
\definecolor{currentstroke}{rgb}{0.000000,0.000000,0.000000}%
\pgfsetstrokecolor{currentstroke}%
\pgfsetdash{}{0pt}%
\pgfpathmoveto{\pgfqpoint{1.849393in}{1.089928in}}%
\pgfpathlineto{\pgfqpoint{1.852625in}{1.089654in}}%
\pgfpathlineto{\pgfqpoint{1.855862in}{1.089590in}}%
\pgfpathlineto{\pgfqpoint{1.859104in}{1.089741in}}%
\pgfpathlineto{\pgfqpoint{1.862352in}{1.090112in}}%
\pgfpathlineto{\pgfqpoint{1.866232in}{1.081724in}}%
\pgfpathlineto{\pgfqpoint{1.869611in}{1.073266in}}%
\pgfpathlineto{\pgfqpoint{1.872485in}{1.064745in}}%
\pgfpathlineto{\pgfqpoint{1.874848in}{1.056170in}}%
\pgfpathlineto{\pgfqpoint{1.871503in}{1.056014in}}%
\pgfpathlineto{\pgfqpoint{1.868164in}{1.056078in}}%
\pgfpathlineto{\pgfqpoint{1.864831in}{1.056358in}}%
\pgfpathlineto{\pgfqpoint{1.861504in}{1.056850in}}%
\pgfpathlineto{\pgfqpoint{1.859217in}{1.065206in}}%
\pgfpathlineto{\pgfqpoint{1.856434in}{1.073510in}}%
\pgfpathlineto{\pgfqpoint{1.853157in}{1.081753in}}%
\pgfpathlineto{\pgfqpoint{1.849393in}{1.089928in}}%
\pgfpathclose%
\pgfusepath{fill}%
\end{pgfscope}%
\begin{pgfscope}%
\pgfpathrectangle{\pgfqpoint{0.329460in}{0.284240in}}{\pgfqpoint{1.989680in}{1.989680in}}%
\pgfusepath{clip}%
\pgfsetbuttcap%
\pgfsetroundjoin%
\definecolor{currentfill}{rgb}{0.147607,0.511733,0.557049}%
\pgfsetfillcolor{currentfill}%
\pgfsetlinewidth{0.000000pt}%
\definecolor{currentstroke}{rgb}{0.000000,0.000000,0.000000}%
\pgfsetstrokecolor{currentstroke}%
\pgfsetdash{}{0pt}%
\pgfpathmoveto{\pgfqpoint{1.208225in}{1.477832in}}%
\pgfpathlineto{\pgfqpoint{1.206687in}{1.472474in}}%
\pgfpathlineto{\pgfqpoint{1.205151in}{1.467121in}}%
\pgfpathlineto{\pgfqpoint{1.203617in}{1.461774in}}%
\pgfpathlineto{\pgfqpoint{1.202084in}{1.456438in}}%
\pgfpathlineto{\pgfqpoint{1.210756in}{1.458668in}}%
\pgfpathlineto{\pgfqpoint{1.219559in}{1.460762in}}%
\pgfpathlineto{\pgfqpoint{1.228485in}{1.462719in}}%
\pgfpathlineto{\pgfqpoint{1.237524in}{1.464538in}}%
\pgfpathlineto{\pgfqpoint{1.238691in}{1.469784in}}%
\pgfpathlineto{\pgfqpoint{1.239859in}{1.475040in}}%
\pgfpathlineto{\pgfqpoint{1.241028in}{1.480304in}}%
\pgfpathlineto{\pgfqpoint{1.242199in}{1.485571in}}%
\pgfpathlineto{\pgfqpoint{1.233533in}{1.483834in}}%
\pgfpathlineto{\pgfqpoint{1.224977in}{1.481963in}}%
\pgfpathlineto{\pgfqpoint{1.216538in}{1.479962in}}%
\pgfpathlineto{\pgfqpoint{1.208225in}{1.477832in}}%
\pgfpathclose%
\pgfusepath{fill}%
\end{pgfscope}%
\begin{pgfscope}%
\pgfpathrectangle{\pgfqpoint{0.329460in}{0.284240in}}{\pgfqpoint{1.989680in}{1.989680in}}%
\pgfusepath{clip}%
\pgfsetbuttcap%
\pgfsetroundjoin%
\definecolor{currentfill}{rgb}{0.263663,0.237631,0.518762}%
\pgfsetfillcolor{currentfill}%
\pgfsetlinewidth{0.000000pt}%
\definecolor{currentstroke}{rgb}{0.000000,0.000000,0.000000}%
\pgfsetstrokecolor{currentstroke}%
\pgfsetdash{}{0pt}%
\pgfpathmoveto{\pgfqpoint{1.699524in}{1.255119in}}%
\pgfpathlineto{\pgfqpoint{1.702376in}{1.250162in}}%
\pgfpathlineto{\pgfqpoint{1.705228in}{1.245287in}}%
\pgfpathlineto{\pgfqpoint{1.708078in}{1.240497in}}%
\pgfpathlineto{\pgfqpoint{1.710928in}{1.235796in}}%
\pgfpathlineto{\pgfqpoint{1.716768in}{1.230003in}}%
\pgfpathlineto{\pgfqpoint{1.722260in}{1.224114in}}%
\pgfpathlineto{\pgfqpoint{1.727396in}{1.218134in}}%
\pgfpathlineto{\pgfqpoint{1.732171in}{1.212069in}}%
\pgfpathlineto{\pgfqpoint{1.729142in}{1.216977in}}%
\pgfpathlineto{\pgfqpoint{1.726112in}{1.221973in}}%
\pgfpathlineto{\pgfqpoint{1.723081in}{1.227054in}}%
\pgfpathlineto{\pgfqpoint{1.720049in}{1.232218in}}%
\pgfpathlineto{\pgfqpoint{1.715437in}{1.238072in}}%
\pgfpathlineto{\pgfqpoint{1.710475in}{1.243843in}}%
\pgfpathlineto{\pgfqpoint{1.705169in}{1.249527in}}%
\pgfpathlineto{\pgfqpoint{1.699524in}{1.255119in}}%
\pgfpathclose%
\pgfusepath{fill}%
\end{pgfscope}%
\begin{pgfscope}%
\pgfpathrectangle{\pgfqpoint{0.329460in}{0.284240in}}{\pgfqpoint{1.989680in}{1.989680in}}%
\pgfusepath{clip}%
\pgfsetbuttcap%
\pgfsetroundjoin%
\definecolor{currentfill}{rgb}{0.272594,0.025563,0.353093}%
\pgfsetfillcolor{currentfill}%
\pgfsetlinewidth{0.000000pt}%
\definecolor{currentstroke}{rgb}{0.000000,0.000000,0.000000}%
\pgfsetstrokecolor{currentstroke}%
\pgfsetdash{}{0pt}%
\pgfpathmoveto{\pgfqpoint{1.888570in}{1.101497in}}%
\pgfpathlineto{\pgfqpoint{1.891880in}{1.104040in}}%
\pgfpathlineto{\pgfqpoint{1.895198in}{1.106846in}}%
\pgfpathlineto{\pgfqpoint{1.898525in}{1.109921in}}%
\pgfpathlineto{\pgfqpoint{1.901860in}{1.113269in}}%
\pgfpathlineto{\pgfqpoint{1.906095in}{1.104264in}}%
\pgfpathlineto{\pgfqpoint{1.909793in}{1.095182in}}%
\pgfpathlineto{\pgfqpoint{1.912947in}{1.086031in}}%
\pgfpathlineto{\pgfqpoint{1.915553in}{1.076820in}}%
\pgfpathlineto{\pgfqpoint{1.912115in}{1.073675in}}%
\pgfpathlineto{\pgfqpoint{1.908686in}{1.070804in}}%
\pgfpathlineto{\pgfqpoint{1.905267in}{1.068203in}}%
\pgfpathlineto{\pgfqpoint{1.901856in}{1.065867in}}%
\pgfpathlineto{\pgfqpoint{1.899333in}{1.074871in}}%
\pgfpathlineto{\pgfqpoint{1.896274in}{1.083816in}}%
\pgfpathlineto{\pgfqpoint{1.892684in}{1.092694in}}%
\pgfpathlineto{\pgfqpoint{1.888570in}{1.101497in}}%
\pgfpathclose%
\pgfusepath{fill}%
\end{pgfscope}%
\begin{pgfscope}%
\pgfpathrectangle{\pgfqpoint{0.329460in}{0.284240in}}{\pgfqpoint{1.989680in}{1.989680in}}%
\pgfusepath{clip}%
\pgfsetbuttcap%
\pgfsetroundjoin%
\definecolor{currentfill}{rgb}{0.231674,0.318106,0.544834}%
\pgfsetfillcolor{currentfill}%
\pgfsetlinewidth{0.000000pt}%
\definecolor{currentstroke}{rgb}{0.000000,0.000000,0.000000}%
\pgfsetstrokecolor{currentstroke}%
\pgfsetdash{}{0pt}%
\pgfpathmoveto{\pgfqpoint{1.652590in}{1.317195in}}%
\pgfpathlineto{\pgfqpoint{1.655233in}{1.311879in}}%
\pgfpathlineto{\pgfqpoint{1.657874in}{1.306621in}}%
\pgfpathlineto{\pgfqpoint{1.660514in}{1.301424in}}%
\pgfpathlineto{\pgfqpoint{1.663152in}{1.296292in}}%
\pgfpathlineto{\pgfqpoint{1.669850in}{1.291304in}}%
\pgfpathlineto{\pgfqpoint{1.676245in}{1.286208in}}%
\pgfpathlineto{\pgfqpoint{1.682332in}{1.281009in}}%
\pgfpathlineto{\pgfqpoint{1.688103in}{1.275712in}}%
\pgfpathlineto{\pgfqpoint{1.685245in}{1.281036in}}%
\pgfpathlineto{\pgfqpoint{1.682385in}{1.286424in}}%
\pgfpathlineto{\pgfqpoint{1.679523in}{1.291874in}}%
\pgfpathlineto{\pgfqpoint{1.676660in}{1.297381in}}%
\pgfpathlineto{\pgfqpoint{1.671094in}{1.302481in}}%
\pgfpathlineto{\pgfqpoint{1.665222in}{1.307486in}}%
\pgfpathlineto{\pgfqpoint{1.659052in}{1.312392in}}%
\pgfpathlineto{\pgfqpoint{1.652590in}{1.317195in}}%
\pgfpathclose%
\pgfusepath{fill}%
\end{pgfscope}%
\begin{pgfscope}%
\pgfpathrectangle{\pgfqpoint{0.329460in}{0.284240in}}{\pgfqpoint{1.989680in}{1.989680in}}%
\pgfusepath{clip}%
\pgfsetbuttcap%
\pgfsetroundjoin%
\definecolor{currentfill}{rgb}{0.268510,0.009605,0.335427}%
\pgfsetfillcolor{currentfill}%
\pgfsetlinewidth{0.000000pt}%
\definecolor{currentstroke}{rgb}{0.000000,0.000000,0.000000}%
\pgfsetstrokecolor{currentstroke}%
\pgfsetdash{}{0pt}%
\pgfpathmoveto{\pgfqpoint{1.836513in}{1.093056in}}%
\pgfpathlineto{\pgfqpoint{1.839726in}{1.091978in}}%
\pgfpathlineto{\pgfqpoint{1.842944in}{1.091094in}}%
\pgfpathlineto{\pgfqpoint{1.846166in}{1.090410in}}%
\pgfpathlineto{\pgfqpoint{1.849393in}{1.089928in}}%
\pgfpathlineto{\pgfqpoint{1.853157in}{1.081753in}}%
\pgfpathlineto{\pgfqpoint{1.856434in}{1.073510in}}%
\pgfpathlineto{\pgfqpoint{1.859217in}{1.065206in}}%
\pgfpathlineto{\pgfqpoint{1.861504in}{1.056850in}}%
\pgfpathlineto{\pgfqpoint{1.858181in}{1.057549in}}%
\pgfpathlineto{\pgfqpoint{1.854864in}{1.058452in}}%
\pgfpathlineto{\pgfqpoint{1.851552in}{1.059554in}}%
\pgfpathlineto{\pgfqpoint{1.848244in}{1.060852in}}%
\pgfpathlineto{\pgfqpoint{1.846034in}{1.068986in}}%
\pgfpathlineto{\pgfqpoint{1.843339in}{1.077070in}}%
\pgfpathlineto{\pgfqpoint{1.840164in}{1.085096in}}%
\pgfpathlineto{\pgfqpoint{1.836513in}{1.093056in}}%
\pgfpathclose%
\pgfusepath{fill}%
\end{pgfscope}%
\begin{pgfscope}%
\pgfpathrectangle{\pgfqpoint{0.329460in}{0.284240in}}{\pgfqpoint{1.989680in}{1.989680in}}%
\pgfusepath{clip}%
\pgfsetbuttcap%
\pgfsetroundjoin%
\definecolor{currentfill}{rgb}{0.172719,0.448791,0.557885}%
\pgfsetfillcolor{currentfill}%
\pgfsetlinewidth{0.000000pt}%
\definecolor{currentstroke}{rgb}{0.000000,0.000000,0.000000}%
\pgfsetstrokecolor{currentstroke}%
\pgfsetdash{}{0pt}%
\pgfpathmoveto{\pgfqpoint{0.722209in}{1.370752in}}%
\pgfpathlineto{\pgfqpoint{0.718604in}{1.384535in}}%
\pgfpathlineto{\pgfqpoint{0.714979in}{1.398769in}}%
\pgfpathlineto{\pgfqpoint{0.711334in}{1.413460in}}%
\pgfpathlineto{\pgfqpoint{0.719137in}{1.423319in}}%
\pgfpathlineto{\pgfqpoint{0.727533in}{1.433036in}}%
\pgfpathlineto{\pgfqpoint{0.736511in}{1.442604in}}%
\pgfpathlineto{\pgfqpoint{0.746059in}{1.452014in}}%
\pgfpathlineto{\pgfqpoint{0.749485in}{1.437197in}}%
\pgfpathlineto{\pgfqpoint{0.752892in}{1.422835in}}%
\pgfpathlineto{\pgfqpoint{0.756281in}{1.408920in}}%
\pgfpathlineto{\pgfqpoint{0.746909in}{1.399603in}}%
\pgfpathlineto{\pgfqpoint{0.738100in}{1.390131in}}%
\pgfpathlineto{\pgfqpoint{0.729863in}{1.380511in}}%
\pgfpathlineto{\pgfqpoint{0.722209in}{1.370752in}}%
\pgfpathclose%
\pgfusepath{fill}%
\end{pgfscope}%
\begin{pgfscope}%
\pgfpathrectangle{\pgfqpoint{0.329460in}{0.284240in}}{\pgfqpoint{1.989680in}{1.989680in}}%
\pgfusepath{clip}%
\pgfsetbuttcap%
\pgfsetroundjoin%
\definecolor{currentfill}{rgb}{0.133743,0.548535,0.553541}%
\pgfsetfillcolor{currentfill}%
\pgfsetlinewidth{0.000000pt}%
\definecolor{currentstroke}{rgb}{0.000000,0.000000,0.000000}%
\pgfsetstrokecolor{currentstroke}%
\pgfsetdash{}{0pt}%
\pgfpathmoveto{\pgfqpoint{1.316059in}{1.515194in}}%
\pgfpathlineto{\pgfqpoint{1.315663in}{1.510036in}}%
\pgfpathlineto{\pgfqpoint{1.315268in}{1.504874in}}%
\pgfpathlineto{\pgfqpoint{1.314873in}{1.499709in}}%
\pgfpathlineto{\pgfqpoint{1.314479in}{1.494543in}}%
\pgfpathlineto{\pgfqpoint{1.323756in}{1.495034in}}%
\pgfpathlineto{\pgfqpoint{1.333059in}{1.495383in}}%
\pgfpathlineto{\pgfqpoint{1.342378in}{1.495589in}}%
\pgfpathlineto{\pgfqpoint{1.351706in}{1.495653in}}%
\pgfpathlineto{\pgfqpoint{1.351700in}{1.500805in}}%
\pgfpathlineto{\pgfqpoint{1.351695in}{1.505958in}}%
\pgfpathlineto{\pgfqpoint{1.351689in}{1.511107in}}%
\pgfpathlineto{\pgfqpoint{1.351684in}{1.516251in}}%
\pgfpathlineto{\pgfqpoint{1.342757in}{1.516191in}}%
\pgfpathlineto{\pgfqpoint{1.333839in}{1.515994in}}%
\pgfpathlineto{\pgfqpoint{1.324937in}{1.515662in}}%
\pgfpathlineto{\pgfqpoint{1.316059in}{1.515194in}}%
\pgfpathclose%
\pgfusepath{fill}%
\end{pgfscope}%
\begin{pgfscope}%
\pgfpathrectangle{\pgfqpoint{0.329460in}{0.284240in}}{\pgfqpoint{1.989680in}{1.989680in}}%
\pgfusepath{clip}%
\pgfsetbuttcap%
\pgfsetroundjoin%
\definecolor{currentfill}{rgb}{0.133743,0.548535,0.553541}%
\pgfsetfillcolor{currentfill}%
\pgfsetlinewidth{0.000000pt}%
\definecolor{currentstroke}{rgb}{0.000000,0.000000,0.000000}%
\pgfsetstrokecolor{currentstroke}%
\pgfsetdash{}{0pt}%
\pgfpathmoveto{\pgfqpoint{1.351684in}{1.516251in}}%
\pgfpathlineto{\pgfqpoint{1.351689in}{1.511107in}}%
\pgfpathlineto{\pgfqpoint{1.351695in}{1.505958in}}%
\pgfpathlineto{\pgfqpoint{1.351700in}{1.500805in}}%
\pgfpathlineto{\pgfqpoint{1.351706in}{1.495653in}}%
\pgfpathlineto{\pgfqpoint{1.361033in}{1.495573in}}%
\pgfpathlineto{\pgfqpoint{1.370351in}{1.495351in}}%
\pgfpathlineto{\pgfqpoint{1.379652in}{1.494987in}}%
\pgfpathlineto{\pgfqpoint{1.388926in}{1.494480in}}%
\pgfpathlineto{\pgfqpoint{1.388520in}{1.499646in}}%
\pgfpathlineto{\pgfqpoint{1.388114in}{1.504812in}}%
\pgfpathlineto{\pgfqpoint{1.387708in}{1.509975in}}%
\pgfpathlineto{\pgfqpoint{1.387301in}{1.515133in}}%
\pgfpathlineto{\pgfqpoint{1.378426in}{1.515616in}}%
\pgfpathlineto{\pgfqpoint{1.369526in}{1.515964in}}%
\pgfpathlineto{\pgfqpoint{1.360609in}{1.516176in}}%
\pgfpathlineto{\pgfqpoint{1.351684in}{1.516251in}}%
\pgfpathclose%
\pgfusepath{fill}%
\end{pgfscope}%
\begin{pgfscope}%
\pgfpathrectangle{\pgfqpoint{0.329460in}{0.284240in}}{\pgfqpoint{1.989680in}{1.989680in}}%
\pgfusepath{clip}%
\pgfsetbuttcap%
\pgfsetroundjoin%
\definecolor{currentfill}{rgb}{0.260571,0.246922,0.522828}%
\pgfsetfillcolor{currentfill}%
\pgfsetlinewidth{0.000000pt}%
\definecolor{currentstroke}{rgb}{0.000000,0.000000,0.000000}%
\pgfsetstrokecolor{currentstroke}%
\pgfsetdash{}{0pt}%
\pgfpathmoveto{\pgfqpoint{0.741460in}{1.199521in}}%
\pgfpathlineto{\pgfqpoint{0.737897in}{1.208275in}}%
\pgfpathlineto{\pgfqpoint{0.734319in}{1.217400in}}%
\pgfpathlineto{\pgfqpoint{0.730726in}{1.226903in}}%
\pgfpathlineto{\pgfqpoint{0.727117in}{1.236790in}}%
\pgfpathlineto{\pgfqpoint{0.732059in}{1.246706in}}%
\pgfpathlineto{\pgfqpoint{0.737596in}{1.256522in}}%
\pgfpathlineto{\pgfqpoint{0.743718in}{1.266230in}}%
\pgfpathlineto{\pgfqpoint{0.750418in}{1.275822in}}%
\pgfpathlineto{\pgfqpoint{0.753871in}{1.265772in}}%
\pgfpathlineto{\pgfqpoint{0.757309in}{1.256103in}}%
\pgfpathlineto{\pgfqpoint{0.760732in}{1.246810in}}%
\pgfpathlineto{\pgfqpoint{0.764142in}{1.237887in}}%
\pgfpathlineto{\pgfqpoint{0.757616in}{1.228458in}}%
\pgfpathlineto{\pgfqpoint{0.751655in}{1.218915in}}%
\pgfpathlineto{\pgfqpoint{0.746267in}{1.209266in}}%
\pgfpathlineto{\pgfqpoint{0.741460in}{1.199521in}}%
\pgfpathclose%
\pgfusepath{fill}%
\end{pgfscope}%
\begin{pgfscope}%
\pgfpathrectangle{\pgfqpoint{0.329460in}{0.284240in}}{\pgfqpoint{1.989680in}{1.989680in}}%
\pgfusepath{clip}%
\pgfsetbuttcap%
\pgfsetroundjoin%
\definecolor{currentfill}{rgb}{0.179019,0.433756,0.557430}%
\pgfsetfillcolor{currentfill}%
\pgfsetlinewidth{0.000000pt}%
\definecolor{currentstroke}{rgb}{0.000000,0.000000,0.000000}%
\pgfsetstrokecolor{currentstroke}%
\pgfsetdash{}{0pt}%
\pgfpathmoveto{\pgfqpoint{1.565973in}{1.414787in}}%
\pgfpathlineto{\pgfqpoint{1.568089in}{1.409317in}}%
\pgfpathlineto{\pgfqpoint{1.570203in}{1.403872in}}%
\pgfpathlineto{\pgfqpoint{1.572315in}{1.398456in}}%
\pgfpathlineto{\pgfqpoint{1.574424in}{1.393069in}}%
\pgfpathlineto{\pgfqpoint{1.582354in}{1.389531in}}%
\pgfpathlineto{\pgfqpoint{1.590066in}{1.385870in}}%
\pgfpathlineto{\pgfqpoint{1.597555in}{1.382087in}}%
\pgfpathlineto{\pgfqpoint{1.604811in}{1.378187in}}%
\pgfpathlineto{\pgfqpoint{1.602409in}{1.383728in}}%
\pgfpathlineto{\pgfqpoint{1.600004in}{1.389298in}}%
\pgfpathlineto{\pgfqpoint{1.597597in}{1.394897in}}%
\pgfpathlineto{\pgfqpoint{1.595188in}{1.400520in}}%
\pgfpathlineto{\pgfqpoint{1.588212in}{1.404259in}}%
\pgfpathlineto{\pgfqpoint{1.581013in}{1.407885in}}%
\pgfpathlineto{\pgfqpoint{1.573598in}{1.411395in}}%
\pgfpathlineto{\pgfqpoint{1.565973in}{1.414787in}}%
\pgfpathclose%
\pgfusepath{fill}%
\end{pgfscope}%
\begin{pgfscope}%
\pgfpathrectangle{\pgfqpoint{0.329460in}{0.284240in}}{\pgfqpoint{1.989680in}{1.989680in}}%
\pgfusepath{clip}%
\pgfsetbuttcap%
\pgfsetroundjoin%
\definecolor{currentfill}{rgb}{0.277941,0.056324,0.381191}%
\pgfsetfillcolor{currentfill}%
\pgfsetlinewidth{0.000000pt}%
\definecolor{currentstroke}{rgb}{0.000000,0.000000,0.000000}%
\pgfsetstrokecolor{currentstroke}%
\pgfsetdash{}{0pt}%
\pgfpathmoveto{\pgfqpoint{1.901860in}{1.113269in}}%
\pgfpathlineto{\pgfqpoint{1.905205in}{1.116896in}}%
\pgfpathlineto{\pgfqpoint{1.908559in}{1.120806in}}%
\pgfpathlineto{\pgfqpoint{1.911922in}{1.125004in}}%
\pgfpathlineto{\pgfqpoint{1.915296in}{1.129495in}}%
\pgfpathlineto{\pgfqpoint{1.919654in}{1.120294in}}%
\pgfpathlineto{\pgfqpoint{1.923461in}{1.111013in}}%
\pgfpathlineto{\pgfqpoint{1.926713in}{1.101661in}}%
\pgfpathlineto{\pgfqpoint{1.929402in}{1.092246in}}%
\pgfpathlineto{\pgfqpoint{1.925925in}{1.087953in}}%
\pgfpathlineto{\pgfqpoint{1.922457in}{1.083954in}}%
\pgfpathlineto{\pgfqpoint{1.919000in}{1.080245in}}%
\pgfpathlineto{\pgfqpoint{1.915553in}{1.076820in}}%
\pgfpathlineto{\pgfqpoint{1.912947in}{1.086031in}}%
\pgfpathlineto{\pgfqpoint{1.909793in}{1.095182in}}%
\pgfpathlineto{\pgfqpoint{1.906095in}{1.104264in}}%
\pgfpathlineto{\pgfqpoint{1.901860in}{1.113269in}}%
\pgfpathclose%
\pgfusepath{fill}%
\end{pgfscope}%
\begin{pgfscope}%
\pgfpathrectangle{\pgfqpoint{0.329460in}{0.284240in}}{\pgfqpoint{1.989680in}{1.989680in}}%
\pgfusepath{clip}%
\pgfsetbuttcap%
\pgfsetroundjoin%
\definecolor{currentfill}{rgb}{0.163625,0.471133,0.558148}%
\pgfsetfillcolor{currentfill}%
\pgfsetlinewidth{0.000000pt}%
\definecolor{currentstroke}{rgb}{0.000000,0.000000,0.000000}%
\pgfsetstrokecolor{currentstroke}%
\pgfsetdash{}{0pt}%
\pgfpathmoveto{\pgfqpoint{1.526347in}{1.448657in}}%
\pgfpathlineto{\pgfqpoint{1.528148in}{1.443245in}}%
\pgfpathlineto{\pgfqpoint{1.529946in}{1.437848in}}%
\pgfpathlineto{\pgfqpoint{1.531743in}{1.432469in}}%
\pgfpathlineto{\pgfqpoint{1.533537in}{1.427109in}}%
\pgfpathlineto{\pgfqpoint{1.541922in}{1.424221in}}%
\pgfpathlineto{\pgfqpoint{1.550127in}{1.421202in}}%
\pgfpathlineto{\pgfqpoint{1.558147in}{1.418057in}}%
\pgfpathlineto{\pgfqpoint{1.565973in}{1.414787in}}%
\pgfpathlineto{\pgfqpoint{1.563855in}{1.420279in}}%
\pgfpathlineto{\pgfqpoint{1.561735in}{1.425791in}}%
\pgfpathlineto{\pgfqpoint{1.559612in}{1.431320in}}%
\pgfpathlineto{\pgfqpoint{1.557488in}{1.436863in}}%
\pgfpathlineto{\pgfqpoint{1.549975in}{1.439993in}}%
\pgfpathlineto{\pgfqpoint{1.542276in}{1.443003in}}%
\pgfpathlineto{\pgfqpoint{1.534397in}{1.445892in}}%
\pgfpathlineto{\pgfqpoint{1.526347in}{1.448657in}}%
\pgfpathclose%
\pgfusepath{fill}%
\end{pgfscope}%
\begin{pgfscope}%
\pgfpathrectangle{\pgfqpoint{0.329460in}{0.284240in}}{\pgfqpoint{1.989680in}{1.989680in}}%
\pgfusepath{clip}%
\pgfsetbuttcap%
\pgfsetroundjoin%
\definecolor{currentfill}{rgb}{0.271305,0.019942,0.347269}%
\pgfsetfillcolor{currentfill}%
\pgfsetlinewidth{0.000000pt}%
\definecolor{currentstroke}{rgb}{0.000000,0.000000,0.000000}%
\pgfsetstrokecolor{currentstroke}%
\pgfsetdash{}{0pt}%
\pgfpathmoveto{\pgfqpoint{1.823699in}{1.099236in}}%
\pgfpathlineto{\pgfqpoint{1.826897in}{1.097418in}}%
\pgfpathlineto{\pgfqpoint{1.830098in}{1.095780in}}%
\pgfpathlineto{\pgfqpoint{1.833304in}{1.094324in}}%
\pgfpathlineto{\pgfqpoint{1.836513in}{1.093056in}}%
\pgfpathlineto{\pgfqpoint{1.840164in}{1.085096in}}%
\pgfpathlineto{\pgfqpoint{1.843339in}{1.077070in}}%
\pgfpathlineto{\pgfqpoint{1.846034in}{1.068986in}}%
\pgfpathlineto{\pgfqpoint{1.848244in}{1.060852in}}%
\pgfpathlineto{\pgfqpoint{1.844941in}{1.062341in}}%
\pgfpathlineto{\pgfqpoint{1.841642in}{1.064018in}}%
\pgfpathlineto{\pgfqpoint{1.838348in}{1.065878in}}%
\pgfpathlineto{\pgfqpoint{1.835057in}{1.067917in}}%
\pgfpathlineto{\pgfqpoint{1.832921in}{1.075827in}}%
\pgfpathlineto{\pgfqpoint{1.830313in}{1.083689in}}%
\pgfpathlineto{\pgfqpoint{1.827238in}{1.091494in}}%
\pgfpathlineto{\pgfqpoint{1.823699in}{1.099236in}}%
\pgfpathclose%
\pgfusepath{fill}%
\end{pgfscope}%
\begin{pgfscope}%
\pgfpathrectangle{\pgfqpoint{0.329460in}{0.284240in}}{\pgfqpoint{1.989680in}{1.989680in}}%
\pgfusepath{clip}%
\pgfsetbuttcap%
\pgfsetroundjoin%
\definecolor{currentfill}{rgb}{0.133743,0.548535,0.553541}%
\pgfsetfillcolor{currentfill}%
\pgfsetlinewidth{0.000000pt}%
\definecolor{currentstroke}{rgb}{0.000000,0.000000,0.000000}%
\pgfsetstrokecolor{currentstroke}%
\pgfsetdash{}{0pt}%
\pgfpathmoveto{\pgfqpoint{1.280956in}{1.511974in}}%
\pgfpathlineto{\pgfqpoint{1.280164in}{1.506778in}}%
\pgfpathlineto{\pgfqpoint{1.279373in}{1.501576in}}%
\pgfpathlineto{\pgfqpoint{1.278584in}{1.496372in}}%
\pgfpathlineto{\pgfqpoint{1.277795in}{1.491167in}}%
\pgfpathlineto{\pgfqpoint{1.286885in}{1.492222in}}%
\pgfpathlineto{\pgfqpoint{1.296034in}{1.493137in}}%
\pgfpathlineto{\pgfqpoint{1.305235in}{1.493911in}}%
\pgfpathlineto{\pgfqpoint{1.314479in}{1.494543in}}%
\pgfpathlineto{\pgfqpoint{1.314873in}{1.499709in}}%
\pgfpathlineto{\pgfqpoint{1.315268in}{1.504874in}}%
\pgfpathlineto{\pgfqpoint{1.315663in}{1.510036in}}%
\pgfpathlineto{\pgfqpoint{1.316059in}{1.515194in}}%
\pgfpathlineto{\pgfqpoint{1.307214in}{1.514590in}}%
\pgfpathlineto{\pgfqpoint{1.298409in}{1.513852in}}%
\pgfpathlineto{\pgfqpoint{1.289654in}{1.512980in}}%
\pgfpathlineto{\pgfqpoint{1.280956in}{1.511974in}}%
\pgfpathclose%
\pgfusepath{fill}%
\end{pgfscope}%
\begin{pgfscope}%
\pgfpathrectangle{\pgfqpoint{0.329460in}{0.284240in}}{\pgfqpoint{1.989680in}{1.989680in}}%
\pgfusepath{clip}%
\pgfsetbuttcap%
\pgfsetroundjoin%
\definecolor{currentfill}{rgb}{0.133743,0.548535,0.553541}%
\pgfsetfillcolor{currentfill}%
\pgfsetlinewidth{0.000000pt}%
\definecolor{currentstroke}{rgb}{0.000000,0.000000,0.000000}%
\pgfsetstrokecolor{currentstroke}%
\pgfsetdash{}{0pt}%
\pgfpathmoveto{\pgfqpoint{1.387301in}{1.515133in}}%
\pgfpathlineto{\pgfqpoint{1.387708in}{1.509975in}}%
\pgfpathlineto{\pgfqpoint{1.388114in}{1.504812in}}%
\pgfpathlineto{\pgfqpoint{1.388520in}{1.499646in}}%
\pgfpathlineto{\pgfqpoint{1.388926in}{1.494480in}}%
\pgfpathlineto{\pgfqpoint{1.398165in}{1.493832in}}%
\pgfpathlineto{\pgfqpoint{1.407360in}{1.493042in}}%
\pgfpathlineto{\pgfqpoint{1.416504in}{1.492112in}}%
\pgfpathlineto{\pgfqpoint{1.425586in}{1.491042in}}%
\pgfpathlineto{\pgfqpoint{1.424787in}{1.496248in}}%
\pgfpathlineto{\pgfqpoint{1.423986in}{1.501454in}}%
\pgfpathlineto{\pgfqpoint{1.423184in}{1.506657in}}%
\pgfpathlineto{\pgfqpoint{1.422382in}{1.511854in}}%
\pgfpathlineto{\pgfqpoint{1.413690in}{1.512875in}}%
\pgfpathlineto{\pgfqpoint{1.404941in}{1.513762in}}%
\pgfpathlineto{\pgfqpoint{1.396142in}{1.514515in}}%
\pgfpathlineto{\pgfqpoint{1.387301in}{1.515133in}}%
\pgfpathclose%
\pgfusepath{fill}%
\end{pgfscope}%
\begin{pgfscope}%
\pgfpathrectangle{\pgfqpoint{0.329460in}{0.284240in}}{\pgfqpoint{1.989680in}{1.989680in}}%
\pgfusepath{clip}%
\pgfsetbuttcap%
\pgfsetroundjoin%
\definecolor{currentfill}{rgb}{0.283072,0.130895,0.449241}%
\pgfsetfillcolor{currentfill}%
\pgfsetlinewidth{0.000000pt}%
\definecolor{currentstroke}{rgb}{0.000000,0.000000,0.000000}%
\pgfsetstrokecolor{currentstroke}%
\pgfsetdash{}{0pt}%
\pgfpathmoveto{\pgfqpoint{0.926734in}{1.143551in}}%
\pgfpathlineto{\pgfqpoint{0.923544in}{1.139319in}}%
\pgfpathlineto{\pgfqpoint{0.920354in}{1.135208in}}%
\pgfpathlineto{\pgfqpoint{0.917163in}{1.131222in}}%
\pgfpathlineto{\pgfqpoint{0.913970in}{1.127364in}}%
\pgfpathlineto{\pgfqpoint{0.917228in}{1.134435in}}%
\pgfpathlineto{\pgfqpoint{0.920907in}{1.141442in}}%
\pgfpathlineto{\pgfqpoint{0.925003in}{1.148378in}}%
\pgfpathlineto{\pgfqpoint{0.929510in}{1.155238in}}%
\pgfpathlineto{\pgfqpoint{0.932576in}{1.158876in}}%
\pgfpathlineto{\pgfqpoint{0.935640in}{1.162643in}}%
\pgfpathlineto{\pgfqpoint{0.938705in}{1.166534in}}%
\pgfpathlineto{\pgfqpoint{0.941768in}{1.170546in}}%
\pgfpathlineto{\pgfqpoint{0.937405in}{1.163903in}}%
\pgfpathlineto{\pgfqpoint{0.933441in}{1.157185in}}%
\pgfpathlineto{\pgfqpoint{0.929883in}{1.150399in}}%
\pgfpathlineto{\pgfqpoint{0.926734in}{1.143551in}}%
\pgfpathclose%
\pgfusepath{fill}%
\end{pgfscope}%
\begin{pgfscope}%
\pgfpathrectangle{\pgfqpoint{0.329460in}{0.284240in}}{\pgfqpoint{1.989680in}{1.989680in}}%
\pgfusepath{clip}%
\pgfsetbuttcap%
\pgfsetroundjoin%
\definecolor{currentfill}{rgb}{0.231674,0.318106,0.544834}%
\pgfsetfillcolor{currentfill}%
\pgfsetlinewidth{0.000000pt}%
\definecolor{currentstroke}{rgb}{0.000000,0.000000,0.000000}%
\pgfsetstrokecolor{currentstroke}%
\pgfsetdash{}{0pt}%
\pgfpathmoveto{\pgfqpoint{1.021029in}{1.292773in}}%
\pgfpathlineto{\pgfqpoint{1.018122in}{1.287220in}}%
\pgfpathlineto{\pgfqpoint{1.015217in}{1.281726in}}%
\pgfpathlineto{\pgfqpoint{1.012314in}{1.276294in}}%
\pgfpathlineto{\pgfqpoint{1.009412in}{1.270925in}}%
\pgfpathlineto{\pgfqpoint{1.014897in}{1.276306in}}%
\pgfpathlineto{\pgfqpoint{1.020704in}{1.281592in}}%
\pgfpathlineto{\pgfqpoint{1.026825in}{1.286780in}}%
\pgfpathlineto{\pgfqpoint{1.033255in}{1.291864in}}%
\pgfpathlineto{\pgfqpoint{1.035945in}{1.297037in}}%
\pgfpathlineto{\pgfqpoint{1.038638in}{1.302275in}}%
\pgfpathlineto{\pgfqpoint{1.041331in}{1.307574in}}%
\pgfpathlineto{\pgfqpoint{1.044027in}{1.312931in}}%
\pgfpathlineto{\pgfqpoint{1.037824in}{1.308036in}}%
\pgfpathlineto{\pgfqpoint{1.031919in}{1.303042in}}%
\pgfpathlineto{\pgfqpoint{1.026318in}{1.297953in}}%
\pgfpathlineto{\pgfqpoint{1.021029in}{1.292773in}}%
\pgfpathclose%
\pgfusepath{fill}%
\end{pgfscope}%
\begin{pgfscope}%
\pgfpathrectangle{\pgfqpoint{0.329460in}{0.284240in}}{\pgfqpoint{1.989680in}{1.989680in}}%
\pgfusepath{clip}%
\pgfsetbuttcap%
\pgfsetroundjoin%
\definecolor{currentfill}{rgb}{0.267004,0.004874,0.329415}%
\pgfsetfillcolor{currentfill}%
\pgfsetlinewidth{0.000000pt}%
\definecolor{currentstroke}{rgb}{0.000000,0.000000,0.000000}%
\pgfsetstrokecolor{currentstroke}%
\pgfsetdash{}{0pt}%
\pgfpathmoveto{\pgfqpoint{0.825857in}{1.048509in}}%
\pgfpathlineto{\pgfqpoint{0.822491in}{1.048840in}}%
\pgfpathlineto{\pgfqpoint{0.819119in}{1.049401in}}%
\pgfpathlineto{\pgfqpoint{0.815739in}{1.050195in}}%
\pgfpathlineto{\pgfqpoint{0.812353in}{1.051227in}}%
\pgfpathlineto{\pgfqpoint{0.814327in}{1.060061in}}%
\pgfpathlineto{\pgfqpoint{0.816828in}{1.068846in}}%
\pgfpathlineto{\pgfqpoint{0.819851in}{1.077575in}}%
\pgfpathlineto{\pgfqpoint{0.823392in}{1.086237in}}%
\pgfpathlineto{\pgfqpoint{0.826691in}{1.084992in}}%
\pgfpathlineto{\pgfqpoint{0.829984in}{1.083984in}}%
\pgfpathlineto{\pgfqpoint{0.833270in}{1.083208in}}%
\pgfpathlineto{\pgfqpoint{0.836550in}{1.082660in}}%
\pgfpathlineto{\pgfqpoint{0.833115in}{1.074209in}}%
\pgfpathlineto{\pgfqpoint{0.830185in}{1.065695in}}%
\pgfpathlineto{\pgfqpoint{0.827764in}{1.057125in}}%
\pgfpathlineto{\pgfqpoint{0.825857in}{1.048509in}}%
\pgfpathclose%
\pgfusepath{fill}%
\end{pgfscope}%
\begin{pgfscope}%
\pgfpathrectangle{\pgfqpoint{0.329460in}{0.284240in}}{\pgfqpoint{1.989680in}{1.989680in}}%
\pgfusepath{clip}%
\pgfsetbuttcap%
\pgfsetroundjoin%
\definecolor{currentfill}{rgb}{0.263663,0.237631,0.518762}%
\pgfsetfillcolor{currentfill}%
\pgfsetlinewidth{0.000000pt}%
\definecolor{currentstroke}{rgb}{0.000000,0.000000,0.000000}%
\pgfsetstrokecolor{currentstroke}%
\pgfsetdash{}{0pt}%
\pgfpathmoveto{\pgfqpoint{0.978524in}{1.226951in}}%
\pgfpathlineto{\pgfqpoint{0.975459in}{1.221739in}}%
\pgfpathlineto{\pgfqpoint{0.972395in}{1.216610in}}%
\pgfpathlineto{\pgfqpoint{0.969331in}{1.211566in}}%
\pgfpathlineto{\pgfqpoint{0.966267in}{1.206611in}}%
\pgfpathlineto{\pgfqpoint{0.970717in}{1.212747in}}%
\pgfpathlineto{\pgfqpoint{0.975532in}{1.218803in}}%
\pgfpathlineto{\pgfqpoint{0.980708in}{1.224773in}}%
\pgfpathlineto{\pgfqpoint{0.986239in}{1.230651in}}%
\pgfpathlineto{\pgfqpoint{0.989132in}{1.235397in}}%
\pgfpathlineto{\pgfqpoint{0.992026in}{1.240231in}}%
\pgfpathlineto{\pgfqpoint{0.994921in}{1.245151in}}%
\pgfpathlineto{\pgfqpoint{0.997817in}{1.250153in}}%
\pgfpathlineto{\pgfqpoint{0.992472in}{1.244479in}}%
\pgfpathlineto{\pgfqpoint{0.987472in}{1.238717in}}%
\pgfpathlineto{\pgfqpoint{0.982821in}{1.232872in}}%
\pgfpathlineto{\pgfqpoint{0.978524in}{1.226951in}}%
\pgfpathclose%
\pgfusepath{fill}%
\end{pgfscope}%
\begin{pgfscope}%
\pgfpathrectangle{\pgfqpoint{0.329460in}{0.284240in}}{\pgfqpoint{1.989680in}{1.989680in}}%
\pgfusepath{clip}%
\pgfsetbuttcap%
\pgfsetroundjoin%
\definecolor{currentfill}{rgb}{0.233603,0.313828,0.543914}%
\pgfsetfillcolor{currentfill}%
\pgfsetlinewidth{0.000000pt}%
\definecolor{currentstroke}{rgb}{0.000000,0.000000,0.000000}%
\pgfsetstrokecolor{currentstroke}%
\pgfsetdash{}{0pt}%
\pgfpathmoveto{\pgfqpoint{1.945524in}{1.284243in}}%
\pgfpathlineto{\pgfqpoint{1.948951in}{1.294716in}}%
\pgfpathlineto{\pgfqpoint{1.952393in}{1.305583in}}%
\pgfpathlineto{\pgfqpoint{1.955850in}{1.316850in}}%
\pgfpathlineto{\pgfqpoint{1.959324in}{1.328525in}}%
\pgfpathlineto{\pgfqpoint{1.966719in}{1.318893in}}%
\pgfpathlineto{\pgfqpoint{1.973531in}{1.309134in}}%
\pgfpathlineto{\pgfqpoint{1.979753in}{1.299256in}}%
\pgfpathlineto{\pgfqpoint{1.985375in}{1.289269in}}%
\pgfpathlineto{\pgfqpoint{1.981729in}{1.277746in}}%
\pgfpathlineto{\pgfqpoint{1.978101in}{1.266633in}}%
\pgfpathlineto{\pgfqpoint{1.974489in}{1.255923in}}%
\pgfpathlineto{\pgfqpoint{1.970894in}{1.245609in}}%
\pgfpathlineto{\pgfqpoint{1.965424in}{1.255436in}}%
\pgfpathlineto{\pgfqpoint{1.959366in}{1.265157in}}%
\pgfpathlineto{\pgfqpoint{1.952730in}{1.274762in}}%
\pgfpathlineto{\pgfqpoint{1.945524in}{1.284243in}}%
\pgfpathclose%
\pgfusepath{fill}%
\end{pgfscope}%
\begin{pgfscope}%
\pgfpathrectangle{\pgfqpoint{0.329460in}{0.284240in}}{\pgfqpoint{1.989680in}{1.989680in}}%
\pgfusepath{clip}%
\pgfsetbuttcap%
\pgfsetroundjoin%
\definecolor{currentfill}{rgb}{0.268510,0.009605,0.335427}%
\pgfsetfillcolor{currentfill}%
\pgfsetlinewidth{0.000000pt}%
\definecolor{currentstroke}{rgb}{0.000000,0.000000,0.000000}%
\pgfsetstrokecolor{currentstroke}%
\pgfsetdash{}{0pt}%
\pgfpathmoveto{\pgfqpoint{0.812353in}{1.051227in}}%
\pgfpathlineto{\pgfqpoint{0.808959in}{1.052501in}}%
\pgfpathlineto{\pgfqpoint{0.805557in}{1.054022in}}%
\pgfpathlineto{\pgfqpoint{0.802148in}{1.055795in}}%
\pgfpathlineto{\pgfqpoint{0.798730in}{1.057823in}}%
\pgfpathlineto{\pgfqpoint{0.800773in}{1.066870in}}%
\pgfpathlineto{\pgfqpoint{0.803356in}{1.075868in}}%
\pgfpathlineto{\pgfqpoint{0.806474in}{1.084806in}}%
\pgfpathlineto{\pgfqpoint{0.810122in}{1.093676in}}%
\pgfpathlineto{\pgfqpoint{0.813451in}{1.091439in}}%
\pgfpathlineto{\pgfqpoint{0.816772in}{1.089456in}}%
\pgfpathlineto{\pgfqpoint{0.820086in}{1.087724in}}%
\pgfpathlineto{\pgfqpoint{0.823392in}{1.086237in}}%
\pgfpathlineto{\pgfqpoint{0.819851in}{1.077575in}}%
\pgfpathlineto{\pgfqpoint{0.816828in}{1.068846in}}%
\pgfpathlineto{\pgfqpoint{0.814327in}{1.060061in}}%
\pgfpathlineto{\pgfqpoint{0.812353in}{1.051227in}}%
\pgfpathclose%
\pgfusepath{fill}%
\end{pgfscope}%
\begin{pgfscope}%
\pgfpathrectangle{\pgfqpoint{0.329460in}{0.284240in}}{\pgfqpoint{1.989680in}{1.989680in}}%
\pgfusepath{clip}%
\pgfsetbuttcap%
\pgfsetroundjoin%
\definecolor{currentfill}{rgb}{0.267004,0.004874,0.329415}%
\pgfsetfillcolor{currentfill}%
\pgfsetlinewidth{0.000000pt}%
\definecolor{currentstroke}{rgb}{0.000000,0.000000,0.000000}%
\pgfsetstrokecolor{currentstroke}%
\pgfsetdash{}{0pt}%
\pgfpathmoveto{\pgfqpoint{0.839260in}{1.049385in}}%
\pgfpathlineto{\pgfqpoint{0.835918in}{1.048844in}}%
\pgfpathlineto{\pgfqpoint{0.832571in}{1.048514in}}%
\pgfpathlineto{\pgfqpoint{0.829217in}{1.048401in}}%
\pgfpathlineto{\pgfqpoint{0.825857in}{1.048509in}}%
\pgfpathlineto{\pgfqpoint{0.827764in}{1.057125in}}%
\pgfpathlineto{\pgfqpoint{0.830185in}{1.065695in}}%
\pgfpathlineto{\pgfqpoint{0.833115in}{1.074209in}}%
\pgfpathlineto{\pgfqpoint{0.836550in}{1.082660in}}%
\pgfpathlineto{\pgfqpoint{0.839824in}{1.082336in}}%
\pgfpathlineto{\pgfqpoint{0.843092in}{1.082232in}}%
\pgfpathlineto{\pgfqpoint{0.846355in}{1.082343in}}%
\pgfpathlineto{\pgfqpoint{0.849612in}{1.082665in}}%
\pgfpathlineto{\pgfqpoint{0.846281in}{1.074429in}}%
\pgfpathlineto{\pgfqpoint{0.843443in}{1.066131in}}%
\pgfpathlineto{\pgfqpoint{0.841101in}{1.057781in}}%
\pgfpathlineto{\pgfqpoint{0.839260in}{1.049385in}}%
\pgfpathclose%
\pgfusepath{fill}%
\end{pgfscope}%
\begin{pgfscope}%
\pgfpathrectangle{\pgfqpoint{0.329460in}{0.284240in}}{\pgfqpoint{1.989680in}{1.989680in}}%
\pgfusepath{clip}%
\pgfsetbuttcap%
\pgfsetroundjoin%
\definecolor{currentfill}{rgb}{0.195860,0.395433,0.555276}%
\pgfsetfillcolor{currentfill}%
\pgfsetlinewidth{0.000000pt}%
\definecolor{currentstroke}{rgb}{0.000000,0.000000,0.000000}%
\pgfsetstrokecolor{currentstroke}%
\pgfsetdash{}{0pt}%
\pgfpathmoveto{\pgfqpoint{1.604811in}{1.378187in}}%
\pgfpathlineto{\pgfqpoint{1.607212in}{1.372680in}}%
\pgfpathlineto{\pgfqpoint{1.609610in}{1.367209in}}%
\pgfpathlineto{\pgfqpoint{1.612006in}{1.361776in}}%
\pgfpathlineto{\pgfqpoint{1.614400in}{1.356384in}}%
\pgfpathlineto{\pgfqpoint{1.621690in}{1.352204in}}%
\pgfpathlineto{\pgfqpoint{1.628724in}{1.347908in}}%
\pgfpathlineto{\pgfqpoint{1.635496in}{1.343501in}}%
\pgfpathlineto{\pgfqpoint{1.642000in}{1.338987in}}%
\pgfpathlineto{\pgfqpoint{1.639347in}{1.344552in}}%
\pgfpathlineto{\pgfqpoint{1.636692in}{1.350159in}}%
\pgfpathlineto{\pgfqpoint{1.634036in}{1.355804in}}%
\pgfpathlineto{\pgfqpoint{1.631377in}{1.361485in}}%
\pgfpathlineto{\pgfqpoint{1.625118in}{1.365819in}}%
\pgfpathlineto{\pgfqpoint{1.618600in}{1.370049in}}%
\pgfpathlineto{\pgfqpoint{1.611829in}{1.374174in}}%
\pgfpathlineto{\pgfqpoint{1.604811in}{1.378187in}}%
\pgfpathclose%
\pgfusepath{fill}%
\end{pgfscope}%
\begin{pgfscope}%
\pgfpathrectangle{\pgfqpoint{0.329460in}{0.284240in}}{\pgfqpoint{1.989680in}{1.989680in}}%
\pgfusepath{clip}%
\pgfsetbuttcap%
\pgfsetroundjoin%
\definecolor{currentfill}{rgb}{0.280255,0.165693,0.476498}%
\pgfsetfillcolor{currentfill}%
\pgfsetlinewidth{0.000000pt}%
\definecolor{currentstroke}{rgb}{0.000000,0.000000,0.000000}%
\pgfsetstrokecolor{currentstroke}%
\pgfsetdash{}{0pt}%
\pgfpathmoveto{\pgfqpoint{1.744285in}{1.193389in}}%
\pgfpathlineto{\pgfqpoint{1.747313in}{1.188973in}}%
\pgfpathlineto{\pgfqpoint{1.750341in}{1.184665in}}%
\pgfpathlineto{\pgfqpoint{1.753369in}{1.180467in}}%
\pgfpathlineto{\pgfqpoint{1.756398in}{1.176384in}}%
\pgfpathlineto{\pgfqpoint{1.761112in}{1.169812in}}%
\pgfpathlineto{\pgfqpoint{1.765431in}{1.163160in}}%
\pgfpathlineto{\pgfqpoint{1.769349in}{1.156434in}}%
\pgfpathlineto{\pgfqpoint{1.772862in}{1.149641in}}%
\pgfpathlineto{\pgfqpoint{1.769698in}{1.153942in}}%
\pgfpathlineto{\pgfqpoint{1.766533in}{1.158357in}}%
\pgfpathlineto{\pgfqpoint{1.763369in}{1.162884in}}%
\pgfpathlineto{\pgfqpoint{1.760206in}{1.167518in}}%
\pgfpathlineto{\pgfqpoint{1.756811in}{1.174089in}}%
\pgfpathlineto{\pgfqpoint{1.753023in}{1.180595in}}%
\pgfpathlineto{\pgfqpoint{1.748845in}{1.187031in}}%
\pgfpathlineto{\pgfqpoint{1.744285in}{1.193389in}}%
\pgfpathclose%
\pgfusepath{fill}%
\end{pgfscope}%
\begin{pgfscope}%
\pgfpathrectangle{\pgfqpoint{0.329460in}{0.284240in}}{\pgfqpoint{1.989680in}{1.989680in}}%
\pgfusepath{clip}%
\pgfsetbuttcap%
\pgfsetroundjoin%
\definecolor{currentfill}{rgb}{0.163625,0.471133,0.558148}%
\pgfsetfillcolor{currentfill}%
\pgfsetlinewidth{0.000000pt}%
\definecolor{currentstroke}{rgb}{0.000000,0.000000,0.000000}%
\pgfsetstrokecolor{currentstroke}%
\pgfsetdash{}{0pt}%
\pgfpathmoveto{\pgfqpoint{1.138372in}{1.433984in}}%
\pgfpathlineto{\pgfqpoint{1.136179in}{1.428408in}}%
\pgfpathlineto{\pgfqpoint{1.133989in}{1.422846in}}%
\pgfpathlineto{\pgfqpoint{1.131800in}{1.417302in}}%
\pgfpathlineto{\pgfqpoint{1.129614in}{1.411778in}}%
\pgfpathlineto{\pgfqpoint{1.137262in}{1.415156in}}%
\pgfpathlineto{\pgfqpoint{1.145110in}{1.418412in}}%
\pgfpathlineto{\pgfqpoint{1.153150in}{1.421544in}}%
\pgfpathlineto{\pgfqpoint{1.161377in}{1.424548in}}%
\pgfpathlineto{\pgfqpoint{1.163245in}{1.429935in}}%
\pgfpathlineto{\pgfqpoint{1.165116in}{1.435342in}}%
\pgfpathlineto{\pgfqpoint{1.166989in}{1.440767in}}%
\pgfpathlineto{\pgfqpoint{1.168864in}{1.446206in}}%
\pgfpathlineto{\pgfqpoint{1.160966in}{1.443330in}}%
\pgfpathlineto{\pgfqpoint{1.153247in}{1.440333in}}%
\pgfpathlineto{\pgfqpoint{1.145713in}{1.437217in}}%
\pgfpathlineto{\pgfqpoint{1.138372in}{1.433984in}}%
\pgfpathclose%
\pgfusepath{fill}%
\end{pgfscope}%
\begin{pgfscope}%
\pgfpathrectangle{\pgfqpoint{0.329460in}{0.284240in}}{\pgfqpoint{1.989680in}{1.989680in}}%
\pgfusepath{clip}%
\pgfsetbuttcap%
\pgfsetroundjoin%
\definecolor{currentfill}{rgb}{0.179019,0.433756,0.557430}%
\pgfsetfillcolor{currentfill}%
\pgfsetlinewidth{0.000000pt}%
\definecolor{currentstroke}{rgb}{0.000000,0.000000,0.000000}%
\pgfsetstrokecolor{currentstroke}%
\pgfsetdash{}{0pt}%
\pgfpathmoveto{\pgfqpoint{1.101180in}{1.397106in}}%
\pgfpathlineto{\pgfqpoint{1.098710in}{1.391445in}}%
\pgfpathlineto{\pgfqpoint{1.096242in}{1.385810in}}%
\pgfpathlineto{\pgfqpoint{1.093777in}{1.380202in}}%
\pgfpathlineto{\pgfqpoint{1.091314in}{1.374625in}}%
\pgfpathlineto{\pgfqpoint{1.098358in}{1.378626in}}%
\pgfpathlineto{\pgfqpoint{1.105641in}{1.382513in}}%
\pgfpathlineto{\pgfqpoint{1.113155in}{1.386282in}}%
\pgfpathlineto{\pgfqpoint{1.120892in}{1.389931in}}%
\pgfpathlineto{\pgfqpoint{1.123069in}{1.395349in}}%
\pgfpathlineto{\pgfqpoint{1.125249in}{1.400799in}}%
\pgfpathlineto{\pgfqpoint{1.127431in}{1.406276in}}%
\pgfpathlineto{\pgfqpoint{1.129614in}{1.411778in}}%
\pgfpathlineto{\pgfqpoint{1.122175in}{1.408280in}}%
\pgfpathlineto{\pgfqpoint{1.114952in}{1.404667in}}%
\pgfpathlineto{\pgfqpoint{1.107951in}{1.400941in}}%
\pgfpathlineto{\pgfqpoint{1.101180in}{1.397106in}}%
\pgfpathclose%
\pgfusepath{fill}%
\end{pgfscope}%
\begin{pgfscope}%
\pgfpathrectangle{\pgfqpoint{0.329460in}{0.284240in}}{\pgfqpoint{1.989680in}{1.989680in}}%
\pgfusepath{clip}%
\pgfsetbuttcap%
\pgfsetroundjoin%
\definecolor{currentfill}{rgb}{0.147607,0.511733,0.557049}%
\pgfsetfillcolor{currentfill}%
\pgfsetlinewidth{0.000000pt}%
\definecolor{currentstroke}{rgb}{0.000000,0.000000,0.000000}%
\pgfsetstrokecolor{currentstroke}%
\pgfsetdash{}{0pt}%
\pgfpathmoveto{\pgfqpoint{1.486767in}{1.479732in}}%
\pgfpathlineto{\pgfqpoint{1.488225in}{1.474396in}}%
\pgfpathlineto{\pgfqpoint{1.489681in}{1.469065in}}%
\pgfpathlineto{\pgfqpoint{1.491136in}{1.463741in}}%
\pgfpathlineto{\pgfqpoint{1.492589in}{1.458426in}}%
\pgfpathlineto{\pgfqpoint{1.501246in}{1.456182in}}%
\pgfpathlineto{\pgfqpoint{1.509764in}{1.453804in}}%
\pgfpathlineto{\pgfqpoint{1.518133in}{1.451295in}}%
\pgfpathlineto{\pgfqpoint{1.526347in}{1.448657in}}%
\pgfpathlineto{\pgfqpoint{1.524545in}{1.454080in}}%
\pgfpathlineto{\pgfqpoint{1.522740in}{1.459513in}}%
\pgfpathlineto{\pgfqpoint{1.520933in}{1.464953in}}%
\pgfpathlineto{\pgfqpoint{1.519124in}{1.470398in}}%
\pgfpathlineto{\pgfqpoint{1.511252in}{1.472918in}}%
\pgfpathlineto{\pgfqpoint{1.503230in}{1.475315in}}%
\pgfpathlineto{\pgfqpoint{1.495066in}{1.477587in}}%
\pgfpathlineto{\pgfqpoint{1.486767in}{1.479732in}}%
\pgfpathclose%
\pgfusepath{fill}%
\end{pgfscope}%
\begin{pgfscope}%
\pgfpathrectangle{\pgfqpoint{0.329460in}{0.284240in}}{\pgfqpoint{1.989680in}{1.989680in}}%
\pgfusepath{clip}%
\pgfsetbuttcap%
\pgfsetroundjoin%
\definecolor{currentfill}{rgb}{0.268510,0.009605,0.335427}%
\pgfsetfillcolor{currentfill}%
\pgfsetlinewidth{0.000000pt}%
\definecolor{currentstroke}{rgb}{0.000000,0.000000,0.000000}%
\pgfsetstrokecolor{currentstroke}%
\pgfsetdash{}{0pt}%
\pgfpathmoveto{\pgfqpoint{0.852576in}{1.053586in}}%
\pgfpathlineto{\pgfqpoint{0.849255in}{1.052238in}}%
\pgfpathlineto{\pgfqpoint{0.845928in}{1.051086in}}%
\pgfpathlineto{\pgfqpoint{0.842597in}{1.050134in}}%
\pgfpathlineto{\pgfqpoint{0.839260in}{1.049385in}}%
\pgfpathlineto{\pgfqpoint{0.841101in}{1.057781in}}%
\pgfpathlineto{\pgfqpoint{0.843443in}{1.066131in}}%
\pgfpathlineto{\pgfqpoint{0.846281in}{1.074429in}}%
\pgfpathlineto{\pgfqpoint{0.849612in}{1.082665in}}%
\pgfpathlineto{\pgfqpoint{0.852865in}{1.083194in}}%
\pgfpathlineto{\pgfqpoint{0.856112in}{1.083926in}}%
\pgfpathlineto{\pgfqpoint{0.859355in}{1.084857in}}%
\pgfpathlineto{\pgfqpoint{0.862594in}{1.085983in}}%
\pgfpathlineto{\pgfqpoint{0.859365in}{1.077965in}}%
\pgfpathlineto{\pgfqpoint{0.856617in}{1.069887in}}%
\pgfpathlineto{\pgfqpoint{0.854352in}{1.061758in}}%
\pgfpathlineto{\pgfqpoint{0.852576in}{1.053586in}}%
\pgfpathclose%
\pgfusepath{fill}%
\end{pgfscope}%
\begin{pgfscope}%
\pgfpathrectangle{\pgfqpoint{0.329460in}{0.284240in}}{\pgfqpoint{1.989680in}{1.989680in}}%
\pgfusepath{clip}%
\pgfsetbuttcap%
\pgfsetroundjoin%
\definecolor{currentfill}{rgb}{0.133743,0.548535,0.553541}%
\pgfsetfillcolor{currentfill}%
\pgfsetlinewidth{0.000000pt}%
\definecolor{currentstroke}{rgb}{0.000000,0.000000,0.000000}%
\pgfsetstrokecolor{currentstroke}%
\pgfsetdash{}{0pt}%
\pgfpathmoveto{\pgfqpoint{1.422382in}{1.511854in}}%
\pgfpathlineto{\pgfqpoint{1.423184in}{1.506657in}}%
\pgfpathlineto{\pgfqpoint{1.423986in}{1.501454in}}%
\pgfpathlineto{\pgfqpoint{1.424787in}{1.496248in}}%
\pgfpathlineto{\pgfqpoint{1.425586in}{1.491042in}}%
\pgfpathlineto{\pgfqpoint{1.434600in}{1.489833in}}%
\pgfpathlineto{\pgfqpoint{1.443537in}{1.488486in}}%
\pgfpathlineto{\pgfqpoint{1.452387in}{1.487003in}}%
\pgfpathlineto{\pgfqpoint{1.451299in}{1.492256in}}%
\pgfpathlineto{\pgfqpoint{1.450209in}{1.497509in}}%
\pgfpathlineto{\pgfqpoint{1.449118in}{1.502759in}}%
\pgfpathlineto{\pgfqpoint{1.448025in}{1.508003in}}%
\pgfpathlineto{\pgfqpoint{1.439557in}{1.509418in}}%
\pgfpathlineto{\pgfqpoint{1.431007in}{1.510702in}}%
\pgfpathlineto{\pgfqpoint{1.422382in}{1.511854in}}%
\pgfpathclose%
\pgfusepath{fill}%
\end{pgfscope}%
\begin{pgfscope}%
\pgfpathrectangle{\pgfqpoint{0.329460in}{0.284240in}}{\pgfqpoint{1.989680in}{1.989680in}}%
\pgfusepath{clip}%
\pgfsetbuttcap%
\pgfsetroundjoin%
\definecolor{currentfill}{rgb}{0.274952,0.037752,0.364543}%
\pgfsetfillcolor{currentfill}%
\pgfsetlinewidth{0.000000pt}%
\definecolor{currentstroke}{rgb}{0.000000,0.000000,0.000000}%
\pgfsetstrokecolor{currentstroke}%
\pgfsetdash{}{0pt}%
\pgfpathmoveto{\pgfqpoint{1.810938in}{1.108221in}}%
\pgfpathlineto{\pgfqpoint{1.814124in}{1.105725in}}%
\pgfpathlineto{\pgfqpoint{1.817313in}{1.103393in}}%
\pgfpathlineto{\pgfqpoint{1.820504in}{1.101229in}}%
\pgfpathlineto{\pgfqpoint{1.823699in}{1.099236in}}%
\pgfpathlineto{\pgfqpoint{1.827238in}{1.091494in}}%
\pgfpathlineto{\pgfqpoint{1.830313in}{1.083689in}}%
\pgfpathlineto{\pgfqpoint{1.832921in}{1.075827in}}%
\pgfpathlineto{\pgfqpoint{1.835057in}{1.067917in}}%
\pgfpathlineto{\pgfqpoint{1.831769in}{1.070133in}}%
\pgfpathlineto{\pgfqpoint{1.828485in}{1.072520in}}%
\pgfpathlineto{\pgfqpoint{1.825205in}{1.075075in}}%
\pgfpathlineto{\pgfqpoint{1.821927in}{1.077795in}}%
\pgfpathlineto{\pgfqpoint{1.819865in}{1.085479in}}%
\pgfpathlineto{\pgfqpoint{1.817344in}{1.093116in}}%
\pgfpathlineto{\pgfqpoint{1.814366in}{1.100699in}}%
\pgfpathlineto{\pgfqpoint{1.810938in}{1.108221in}}%
\pgfpathclose%
\pgfusepath{fill}%
\end{pgfscope}%
\begin{pgfscope}%
\pgfpathrectangle{\pgfqpoint{0.329460in}{0.284240in}}{\pgfqpoint{1.989680in}{1.989680in}}%
\pgfusepath{clip}%
\pgfsetbuttcap%
\pgfsetroundjoin%
\definecolor{currentfill}{rgb}{0.272594,0.025563,0.353093}%
\pgfsetfillcolor{currentfill}%
\pgfsetlinewidth{0.000000pt}%
\definecolor{currentstroke}{rgb}{0.000000,0.000000,0.000000}%
\pgfsetstrokecolor{currentstroke}%
\pgfsetdash{}{0pt}%
\pgfpathmoveto{\pgfqpoint{0.798730in}{1.057823in}}%
\pgfpathlineto{\pgfqpoint{0.795304in}{1.060111in}}%
\pgfpathlineto{\pgfqpoint{0.791869in}{1.062665in}}%
\pgfpathlineto{\pgfqpoint{0.788425in}{1.065490in}}%
\pgfpathlineto{\pgfqpoint{0.784971in}{1.068589in}}%
\pgfpathlineto{\pgfqpoint{0.787085in}{1.077846in}}%
\pgfpathlineto{\pgfqpoint{0.789751in}{1.087051in}}%
\pgfpathlineto{\pgfqpoint{0.792966in}{1.096195in}}%
\pgfpathlineto{\pgfqpoint{0.796724in}{1.105269in}}%
\pgfpathlineto{\pgfqpoint{0.800087in}{1.101965in}}%
\pgfpathlineto{\pgfqpoint{0.803440in}{1.098934in}}%
\pgfpathlineto{\pgfqpoint{0.806785in}{1.096173in}}%
\pgfpathlineto{\pgfqpoint{0.810122in}{1.093676in}}%
\pgfpathlineto{\pgfqpoint{0.806474in}{1.084806in}}%
\pgfpathlineto{\pgfqpoint{0.803356in}{1.075868in}}%
\pgfpathlineto{\pgfqpoint{0.800773in}{1.066870in}}%
\pgfpathlineto{\pgfqpoint{0.798730in}{1.057823in}}%
\pgfpathclose%
\pgfusepath{fill}%
\end{pgfscope}%
\begin{pgfscope}%
\pgfpathrectangle{\pgfqpoint{0.329460in}{0.284240in}}{\pgfqpoint{1.989680in}{1.989680in}}%
\pgfusepath{clip}%
\pgfsetbuttcap%
\pgfsetroundjoin%
\definecolor{currentfill}{rgb}{0.282327,0.094955,0.417331}%
\pgfsetfillcolor{currentfill}%
\pgfsetlinewidth{0.000000pt}%
\definecolor{currentstroke}{rgb}{0.000000,0.000000,0.000000}%
\pgfsetstrokecolor{currentstroke}%
\pgfsetdash{}{0pt}%
\pgfpathmoveto{\pgfqpoint{1.915296in}{1.129495in}}%
\pgfpathlineto{\pgfqpoint{1.918679in}{1.134285in}}%
\pgfpathlineto{\pgfqpoint{1.922074in}{1.139378in}}%
\pgfpathlineto{\pgfqpoint{1.925479in}{1.144781in}}%
\pgfpathlineto{\pgfqpoint{1.928895in}{1.150497in}}%
\pgfpathlineto{\pgfqpoint{1.933378in}{1.141104in}}%
\pgfpathlineto{\pgfqpoint{1.937299in}{1.131629in}}%
\pgfpathlineto{\pgfqpoint{1.940649in}{1.122082in}}%
\pgfpathlineto{\pgfqpoint{1.943425in}{1.112470in}}%
\pgfpathlineto{\pgfqpoint{1.939902in}{1.106946in}}%
\pgfpathlineto{\pgfqpoint{1.936391in}{1.101738in}}%
\pgfpathlineto{\pgfqpoint{1.932891in}{1.096840in}}%
\pgfpathlineto{\pgfqpoint{1.929402in}{1.092246in}}%
\pgfpathlineto{\pgfqpoint{1.926713in}{1.101661in}}%
\pgfpathlineto{\pgfqpoint{1.923461in}{1.111013in}}%
\pgfpathlineto{\pgfqpoint{1.919654in}{1.120294in}}%
\pgfpathlineto{\pgfqpoint{1.915296in}{1.129495in}}%
\pgfpathclose%
\pgfusepath{fill}%
\end{pgfscope}%
\begin{pgfscope}%
\pgfpathrectangle{\pgfqpoint{0.329460in}{0.284240in}}{\pgfqpoint{1.989680in}{1.989680in}}%
\pgfusepath{clip}%
\pgfsetbuttcap%
\pgfsetroundjoin%
\definecolor{currentfill}{rgb}{0.133743,0.548535,0.553541}%
\pgfsetfillcolor{currentfill}%
\pgfsetlinewidth{0.000000pt}%
\definecolor{currentstroke}{rgb}{0.000000,0.000000,0.000000}%
\pgfsetstrokecolor{currentstroke}%
\pgfsetdash{}{0pt}%
\pgfpathmoveto{\pgfqpoint{1.246898in}{1.506638in}}%
\pgfpathlineto{\pgfqpoint{1.245721in}{1.501377in}}%
\pgfpathlineto{\pgfqpoint{1.244546in}{1.496110in}}%
\pgfpathlineto{\pgfqpoint{1.243372in}{1.490841in}}%
\pgfpathlineto{\pgfqpoint{1.242199in}{1.485571in}}%
\pgfpathlineto{\pgfqpoint{1.250967in}{1.487175in}}%
\pgfpathlineto{\pgfqpoint{1.259827in}{1.488643in}}%
\pgfpathlineto{\pgfqpoint{1.268773in}{1.489974in}}%
\pgfpathlineto{\pgfqpoint{1.277795in}{1.491167in}}%
\pgfpathlineto{\pgfqpoint{1.278584in}{1.496372in}}%
\pgfpathlineto{\pgfqpoint{1.279373in}{1.501576in}}%
\pgfpathlineto{\pgfqpoint{1.280164in}{1.506778in}}%
\pgfpathlineto{\pgfqpoint{1.280956in}{1.511974in}}%
\pgfpathlineto{\pgfqpoint{1.272323in}{1.510836in}}%
\pgfpathlineto{\pgfqpoint{1.263764in}{1.509567in}}%
\pgfpathlineto{\pgfqpoint{1.255286in}{1.508167in}}%
\pgfpathlineto{\pgfqpoint{1.246898in}{1.506638in}}%
\pgfpathclose%
\pgfusepath{fill}%
\end{pgfscope}%
\begin{pgfscope}%
\pgfpathrectangle{\pgfqpoint{0.329460in}{0.284240in}}{\pgfqpoint{1.989680in}{1.989680in}}%
\pgfusepath{clip}%
\pgfsetbuttcap%
\pgfsetroundjoin%
\definecolor{currentfill}{rgb}{0.147607,0.511733,0.557049}%
\pgfsetfillcolor{currentfill}%
\pgfsetlinewidth{0.000000pt}%
\definecolor{currentstroke}{rgb}{0.000000,0.000000,0.000000}%
\pgfsetstrokecolor{currentstroke}%
\pgfsetdash{}{0pt}%
\pgfpathmoveto{\pgfqpoint{1.176386in}{1.468056in}}%
\pgfpathlineto{\pgfqpoint{1.174502in}{1.462584in}}%
\pgfpathlineto{\pgfqpoint{1.172620in}{1.457117in}}%
\pgfpathlineto{\pgfqpoint{1.170741in}{1.451657in}}%
\pgfpathlineto{\pgfqpoint{1.168864in}{1.446206in}}%
\pgfpathlineto{\pgfqpoint{1.176933in}{1.448956in}}%
\pgfpathlineto{\pgfqpoint{1.185164in}{1.451580in}}%
\pgfpathlineto{\pgfqpoint{1.193551in}{1.454075in}}%
\pgfpathlineto{\pgfqpoint{1.202084in}{1.456438in}}%
\pgfpathlineto{\pgfqpoint{1.203617in}{1.461774in}}%
\pgfpathlineto{\pgfqpoint{1.205151in}{1.467121in}}%
\pgfpathlineto{\pgfqpoint{1.206687in}{1.472474in}}%
\pgfpathlineto{\pgfqpoint{1.208225in}{1.477832in}}%
\pgfpathlineto{\pgfqpoint{1.200046in}{1.475574in}}%
\pgfpathlineto{\pgfqpoint{1.192007in}{1.473191in}}%
\pgfpathlineto{\pgfqpoint{1.184118in}{1.470684in}}%
\pgfpathlineto{\pgfqpoint{1.176386in}{1.468056in}}%
\pgfpathclose%
\pgfusepath{fill}%
\end{pgfscope}%
\begin{pgfscope}%
\pgfpathrectangle{\pgfqpoint{0.329460in}{0.284240in}}{\pgfqpoint{1.989680in}{1.989680in}}%
\pgfusepath{clip}%
\pgfsetbuttcap%
\pgfsetroundjoin%
\definecolor{currentfill}{rgb}{0.195860,0.395433,0.555276}%
\pgfsetfillcolor{currentfill}%
\pgfsetlinewidth{0.000000pt}%
\definecolor{currentstroke}{rgb}{0.000000,0.000000,0.000000}%
\pgfsetstrokecolor{currentstroke}%
\pgfsetdash{}{0pt}%
\pgfpathmoveto{\pgfqpoint{1.065659in}{1.357550in}}%
\pgfpathlineto{\pgfqpoint{1.062948in}{1.351828in}}%
\pgfpathlineto{\pgfqpoint{1.060238in}{1.346142in}}%
\pgfpathlineto{\pgfqpoint{1.057531in}{1.340494in}}%
\pgfpathlineto{\pgfqpoint{1.054826in}{1.334888in}}%
\pgfpathlineto{\pgfqpoint{1.061085in}{1.339494in}}%
\pgfpathlineto{\pgfqpoint{1.067618in}{1.343996in}}%
\pgfpathlineto{\pgfqpoint{1.074420in}{1.348391in}}%
\pgfpathlineto{\pgfqpoint{1.081483in}{1.352674in}}%
\pgfpathlineto{\pgfqpoint{1.083938in}{1.358102in}}%
\pgfpathlineto{\pgfqpoint{1.086394in}{1.363572in}}%
\pgfpathlineto{\pgfqpoint{1.088853in}{1.369081in}}%
\pgfpathlineto{\pgfqpoint{1.091314in}{1.374625in}}%
\pgfpathlineto{\pgfqpoint{1.084515in}{1.370513in}}%
\pgfpathlineto{\pgfqpoint{1.077968in}{1.366294in}}%
\pgfpathlineto{\pgfqpoint{1.071681in}{1.361972in}}%
\pgfpathlineto{\pgfqpoint{1.065659in}{1.357550in}}%
\pgfpathclose%
\pgfusepath{fill}%
\end{pgfscope}%
\begin{pgfscope}%
\pgfpathrectangle{\pgfqpoint{0.329460in}{0.284240in}}{\pgfqpoint{1.989680in}{1.989680in}}%
\pgfusepath{clip}%
\pgfsetbuttcap%
\pgfsetroundjoin%
\definecolor{currentfill}{rgb}{0.248629,0.278775,0.534556}%
\pgfsetfillcolor{currentfill}%
\pgfsetlinewidth{0.000000pt}%
\definecolor{currentstroke}{rgb}{0.000000,0.000000,0.000000}%
\pgfsetstrokecolor{currentstroke}%
\pgfsetdash{}{0pt}%
\pgfpathmoveto{\pgfqpoint{1.688103in}{1.275712in}}%
\pgfpathlineto{\pgfqpoint{1.690960in}{1.270456in}}%
\pgfpathlineto{\pgfqpoint{1.693816in}{1.265269in}}%
\pgfpathlineto{\pgfqpoint{1.696671in}{1.260156in}}%
\pgfpathlineto{\pgfqpoint{1.699524in}{1.255119in}}%
\pgfpathlineto{\pgfqpoint{1.705169in}{1.249527in}}%
\pgfpathlineto{\pgfqpoint{1.710475in}{1.243843in}}%
\pgfpathlineto{\pgfqpoint{1.715437in}{1.238072in}}%
\pgfpathlineto{\pgfqpoint{1.720049in}{1.232218in}}%
\pgfpathlineto{\pgfqpoint{1.717017in}{1.237462in}}%
\pgfpathlineto{\pgfqpoint{1.713983in}{1.242781in}}%
\pgfpathlineto{\pgfqpoint{1.710949in}{1.248174in}}%
\pgfpathlineto{\pgfqpoint{1.707913in}{1.253638in}}%
\pgfpathlineto{\pgfqpoint{1.703463in}{1.259279in}}%
\pgfpathlineto{\pgfqpoint{1.698675in}{1.264842in}}%
\pgfpathlineto{\pgfqpoint{1.693553in}{1.270322in}}%
\pgfpathlineto{\pgfqpoint{1.688103in}{1.275712in}}%
\pgfpathclose%
\pgfusepath{fill}%
\end{pgfscope}%
\begin{pgfscope}%
\pgfpathrectangle{\pgfqpoint{0.329460in}{0.284240in}}{\pgfqpoint{1.989680in}{1.989680in}}%
\pgfusepath{clip}%
\pgfsetbuttcap%
\pgfsetroundjoin%
\definecolor{currentfill}{rgb}{0.271305,0.019942,0.347269}%
\pgfsetfillcolor{currentfill}%
\pgfsetlinewidth{0.000000pt}%
\definecolor{currentstroke}{rgb}{0.000000,0.000000,0.000000}%
\pgfsetstrokecolor{currentstroke}%
\pgfsetdash{}{0pt}%
\pgfpathmoveto{\pgfqpoint{0.865820in}{1.060852in}}%
\pgfpathlineto{\pgfqpoint{0.862515in}{1.058762in}}%
\pgfpathlineto{\pgfqpoint{0.859206in}{1.056852in}}%
\pgfpathlineto{\pgfqpoint{0.855893in}{1.055125in}}%
\pgfpathlineto{\pgfqpoint{0.852576in}{1.053586in}}%
\pgfpathlineto{\pgfqpoint{0.854352in}{1.061758in}}%
\pgfpathlineto{\pgfqpoint{0.856617in}{1.069887in}}%
\pgfpathlineto{\pgfqpoint{0.859365in}{1.077965in}}%
\pgfpathlineto{\pgfqpoint{0.862594in}{1.085983in}}%
\pgfpathlineto{\pgfqpoint{0.865828in}{1.087300in}}%
\pgfpathlineto{\pgfqpoint{0.869058in}{1.088804in}}%
\pgfpathlineto{\pgfqpoint{0.872285in}{1.090491in}}%
\pgfpathlineto{\pgfqpoint{0.875508in}{1.092358in}}%
\pgfpathlineto{\pgfqpoint{0.872380in}{1.084559in}}%
\pgfpathlineto{\pgfqpoint{0.869721in}{1.076703in}}%
\pgfpathlineto{\pgfqpoint{0.867532in}{1.068798in}}%
\pgfpathlineto{\pgfqpoint{0.865820in}{1.060852in}}%
\pgfpathclose%
\pgfusepath{fill}%
\end{pgfscope}%
\begin{pgfscope}%
\pgfpathrectangle{\pgfqpoint{0.329460in}{0.284240in}}{\pgfqpoint{1.989680in}{1.989680in}}%
\pgfusepath{clip}%
\pgfsetbuttcap%
\pgfsetroundjoin%
\definecolor{currentfill}{rgb}{0.277941,0.056324,0.381191}%
\pgfsetfillcolor{currentfill}%
\pgfsetlinewidth{0.000000pt}%
\definecolor{currentstroke}{rgb}{0.000000,0.000000,0.000000}%
\pgfsetstrokecolor{currentstroke}%
\pgfsetdash{}{0pt}%
\pgfpathmoveto{\pgfqpoint{0.784971in}{1.068589in}}%
\pgfpathlineto{\pgfqpoint{0.781508in}{1.071968in}}%
\pgfpathlineto{\pgfqpoint{0.778035in}{1.075631in}}%
\pgfpathlineto{\pgfqpoint{0.774552in}{1.079585in}}%
\pgfpathlineto{\pgfqpoint{0.771058in}{1.083833in}}%
\pgfpathlineto{\pgfqpoint{0.773244in}{1.093295in}}%
\pgfpathlineto{\pgfqpoint{0.775996in}{1.102703in}}%
\pgfpathlineto{\pgfqpoint{0.779310in}{1.112048in}}%
\pgfpathlineto{\pgfqpoint{0.783179in}{1.121320in}}%
\pgfpathlineto{\pgfqpoint{0.786580in}{1.116872in}}%
\pgfpathlineto{\pgfqpoint{0.789971in}{1.112718in}}%
\pgfpathlineto{\pgfqpoint{0.793352in}{1.108851in}}%
\pgfpathlineto{\pgfqpoint{0.796724in}{1.105269in}}%
\pgfpathlineto{\pgfqpoint{0.792966in}{1.096195in}}%
\pgfpathlineto{\pgfqpoint{0.789751in}{1.087051in}}%
\pgfpathlineto{\pgfqpoint{0.787085in}{1.077846in}}%
\pgfpathlineto{\pgfqpoint{0.784971in}{1.068589in}}%
\pgfpathclose%
\pgfusepath{fill}%
\end{pgfscope}%
\begin{pgfscope}%
\pgfpathrectangle{\pgfqpoint{0.329460in}{0.284240in}}{\pgfqpoint{1.989680in}{1.989680in}}%
\pgfusepath{clip}%
\pgfsetbuttcap%
\pgfsetroundjoin%
\definecolor{currentfill}{rgb}{0.280255,0.165693,0.476498}%
\pgfsetfillcolor{currentfill}%
\pgfsetlinewidth{0.000000pt}%
\definecolor{currentstroke}{rgb}{0.000000,0.000000,0.000000}%
\pgfsetstrokecolor{currentstroke}%
\pgfsetdash{}{0pt}%
\pgfpathmoveto{\pgfqpoint{0.939486in}{1.161628in}}%
\pgfpathlineto{\pgfqpoint{0.936299in}{1.156944in}}%
\pgfpathlineto{\pgfqpoint{0.933111in}{1.152367in}}%
\pgfpathlineto{\pgfqpoint{0.929923in}{1.147902in}}%
\pgfpathlineto{\pgfqpoint{0.926734in}{1.143551in}}%
\pgfpathlineto{\pgfqpoint{0.929883in}{1.150399in}}%
\pgfpathlineto{\pgfqpoint{0.933441in}{1.157185in}}%
\pgfpathlineto{\pgfqpoint{0.937405in}{1.163903in}}%
\pgfpathlineto{\pgfqpoint{0.941768in}{1.170546in}}%
\pgfpathlineto{\pgfqpoint{0.944831in}{1.174677in}}%
\pgfpathlineto{\pgfqpoint{0.947893in}{1.178921in}}%
\pgfpathlineto{\pgfqpoint{0.950956in}{1.183277in}}%
\pgfpathlineto{\pgfqpoint{0.954018in}{1.187741in}}%
\pgfpathlineto{\pgfqpoint{0.949798in}{1.181314in}}%
\pgfpathlineto{\pgfqpoint{0.945966in}{1.174815in}}%
\pgfpathlineto{\pgfqpoint{0.942527in}{1.168251in}}%
\pgfpathlineto{\pgfqpoint{0.939486in}{1.161628in}}%
\pgfpathclose%
\pgfusepath{fill}%
\end{pgfscope}%
\begin{pgfscope}%
\pgfpathrectangle{\pgfqpoint{0.329460in}{0.284240in}}{\pgfqpoint{1.989680in}{1.989680in}}%
\pgfusepath{clip}%
\pgfsetbuttcap%
\pgfsetroundjoin%
\definecolor{currentfill}{rgb}{0.212395,0.359683,0.551710}%
\pgfsetfillcolor{currentfill}%
\pgfsetlinewidth{0.000000pt}%
\definecolor{currentstroke}{rgb}{0.000000,0.000000,0.000000}%
\pgfsetstrokecolor{currentstroke}%
\pgfsetdash{}{0pt}%
\pgfpathmoveto{\pgfqpoint{1.642000in}{1.338987in}}%
\pgfpathlineto{\pgfqpoint{1.644650in}{1.333466in}}%
\pgfpathlineto{\pgfqpoint{1.647299in}{1.327992in}}%
\pgfpathlineto{\pgfqpoint{1.649945in}{1.322567in}}%
\pgfpathlineto{\pgfqpoint{1.652590in}{1.317195in}}%
\pgfpathlineto{\pgfqpoint{1.659052in}{1.312392in}}%
\pgfpathlineto{\pgfqpoint{1.665222in}{1.307486in}}%
\pgfpathlineto{\pgfqpoint{1.671094in}{1.302481in}}%
\pgfpathlineto{\pgfqpoint{1.676660in}{1.297381in}}%
\pgfpathlineto{\pgfqpoint{1.673795in}{1.302945in}}%
\pgfpathlineto{\pgfqpoint{1.670928in}{1.308561in}}%
\pgfpathlineto{\pgfqpoint{1.668059in}{1.314226in}}%
\pgfpathlineto{\pgfqpoint{1.665188in}{1.319939in}}%
\pgfpathlineto{\pgfqpoint{1.659827in}{1.324841in}}%
\pgfpathlineto{\pgfqpoint{1.654172in}{1.329653in}}%
\pgfpathlineto{\pgfqpoint{1.648227in}{1.334369in}}%
\pgfpathlineto{\pgfqpoint{1.642000in}{1.338987in}}%
\pgfpathclose%
\pgfusepath{fill}%
\end{pgfscope}%
\begin{pgfscope}%
\pgfpathrectangle{\pgfqpoint{0.329460in}{0.284240in}}{\pgfqpoint{1.989680in}{1.989680in}}%
\pgfusepath{clip}%
\pgfsetbuttcap%
\pgfsetroundjoin%
\definecolor{currentfill}{rgb}{0.279566,0.067836,0.391917}%
\pgfsetfillcolor{currentfill}%
\pgfsetlinewidth{0.000000pt}%
\definecolor{currentstroke}{rgb}{0.000000,0.000000,0.000000}%
\pgfsetstrokecolor{currentstroke}%
\pgfsetdash{}{0pt}%
\pgfpathmoveto{\pgfqpoint{1.798219in}{1.119771in}}%
\pgfpathlineto{\pgfqpoint{1.801395in}{1.116656in}}%
\pgfpathlineto{\pgfqpoint{1.804574in}{1.113690in}}%
\pgfpathlineto{\pgfqpoint{1.807755in}{1.110877in}}%
\pgfpathlineto{\pgfqpoint{1.810938in}{1.108221in}}%
\pgfpathlineto{\pgfqpoint{1.814366in}{1.100699in}}%
\pgfpathlineto{\pgfqpoint{1.817344in}{1.093116in}}%
\pgfpathlineto{\pgfqpoint{1.819865in}{1.085479in}}%
\pgfpathlineto{\pgfqpoint{1.821927in}{1.077795in}}%
\pgfpathlineto{\pgfqpoint{1.818653in}{1.080675in}}%
\pgfpathlineto{\pgfqpoint{1.815381in}{1.083712in}}%
\pgfpathlineto{\pgfqpoint{1.812111in}{1.086903in}}%
\pgfpathlineto{\pgfqpoint{1.808844in}{1.090243in}}%
\pgfpathlineto{\pgfqpoint{1.806854in}{1.097699in}}%
\pgfpathlineto{\pgfqpoint{1.804418in}{1.105111in}}%
\pgfpathlineto{\pgfqpoint{1.801538in}{1.112470in}}%
\pgfpathlineto{\pgfqpoint{1.798219in}{1.119771in}}%
\pgfpathclose%
\pgfusepath{fill}%
\end{pgfscope}%
\begin{pgfscope}%
\pgfpathrectangle{\pgfqpoint{0.329460in}{0.284240in}}{\pgfqpoint{1.989680in}{1.989680in}}%
\pgfusepath{clip}%
\pgfsetbuttcap%
\pgfsetroundjoin%
\definecolor{currentfill}{rgb}{0.133743,0.548535,0.553541}%
\pgfsetfillcolor{currentfill}%
\pgfsetlinewidth{0.000000pt}%
\definecolor{currentstroke}{rgb}{0.000000,0.000000,0.000000}%
\pgfsetstrokecolor{currentstroke}%
\pgfsetdash{}{0pt}%
\pgfpathmoveto{\pgfqpoint{1.448025in}{1.508003in}}%
\pgfpathlineto{\pgfqpoint{1.449118in}{1.502759in}}%
\pgfpathlineto{\pgfqpoint{1.450209in}{1.497509in}}%
\pgfpathlineto{\pgfqpoint{1.451299in}{1.492256in}}%
\pgfpathlineto{\pgfqpoint{1.452387in}{1.487003in}}%
\pgfpathlineto{\pgfqpoint{1.461144in}{1.485385in}}%
\pgfpathlineto{\pgfqpoint{1.469798in}{1.483632in}}%
\pgfpathlineto{\pgfqpoint{1.478342in}{1.481747in}}%
\pgfpathlineto{\pgfqpoint{1.486767in}{1.479732in}}%
\pgfpathlineto{\pgfqpoint{1.485307in}{1.485069in}}%
\pgfpathlineto{\pgfqpoint{1.483845in}{1.490407in}}%
\pgfpathlineto{\pgfqpoint{1.482382in}{1.495741in}}%
\pgfpathlineto{\pgfqpoint{1.480916in}{1.501070in}}%
\pgfpathlineto{\pgfqpoint{1.472856in}{1.502992in}}%
\pgfpathlineto{\pgfqpoint{1.464683in}{1.504789in}}%
\pgfpathlineto{\pgfqpoint{1.456403in}{1.506460in}}%
\pgfpathlineto{\pgfqpoint{1.448025in}{1.508003in}}%
\pgfpathclose%
\pgfusepath{fill}%
\end{pgfscope}%
\begin{pgfscope}%
\pgfpathrectangle{\pgfqpoint{0.329460in}{0.284240in}}{\pgfqpoint{1.989680in}{1.989680in}}%
\pgfusepath{clip}%
\pgfsetbuttcap%
\pgfsetroundjoin%
\definecolor{currentfill}{rgb}{0.233603,0.313828,0.543914}%
\pgfsetfillcolor{currentfill}%
\pgfsetlinewidth{0.000000pt}%
\definecolor{currentstroke}{rgb}{0.000000,0.000000,0.000000}%
\pgfsetstrokecolor{currentstroke}%
\pgfsetdash{}{0pt}%
\pgfpathmoveto{\pgfqpoint{0.727117in}{1.236790in}}%
\pgfpathlineto{\pgfqpoint{0.723491in}{1.247068in}}%
\pgfpathlineto{\pgfqpoint{0.719849in}{1.257742in}}%
\pgfpathlineto{\pgfqpoint{0.716190in}{1.268820in}}%
\pgfpathlineto{\pgfqpoint{0.712514in}{1.280307in}}%
\pgfpathlineto{\pgfqpoint{0.717595in}{1.290384in}}%
\pgfpathlineto{\pgfqpoint{0.723284in}{1.300359in}}%
\pgfpathlineto{\pgfqpoint{0.729572in}{1.310224in}}%
\pgfpathlineto{\pgfqpoint{0.736450in}{1.319970in}}%
\pgfpathlineto{\pgfqpoint{0.739966in}{1.308328in}}%
\pgfpathlineto{\pgfqpoint{0.743466in}{1.297093in}}%
\pgfpathlineto{\pgfqpoint{0.746950in}{1.286260in}}%
\pgfpathlineto{\pgfqpoint{0.750418in}{1.275822in}}%
\pgfpathlineto{\pgfqpoint{0.743718in}{1.266230in}}%
\pgfpathlineto{\pgfqpoint{0.737596in}{1.256522in}}%
\pgfpathlineto{\pgfqpoint{0.732059in}{1.246706in}}%
\pgfpathlineto{\pgfqpoint{0.727117in}{1.236790in}}%
\pgfpathclose%
\pgfusepath{fill}%
\end{pgfscope}%
\begin{pgfscope}%
\pgfpathrectangle{\pgfqpoint{0.329460in}{0.284240in}}{\pgfqpoint{1.989680in}{1.989680in}}%
\pgfusepath{clip}%
\pgfsetbuttcap%
\pgfsetroundjoin%
\definecolor{currentfill}{rgb}{0.274952,0.037752,0.364543}%
\pgfsetfillcolor{currentfill}%
\pgfsetlinewidth{0.000000pt}%
\definecolor{currentstroke}{rgb}{0.000000,0.000000,0.000000}%
\pgfsetstrokecolor{currentstroke}%
\pgfsetdash{}{0pt}%
\pgfpathmoveto{\pgfqpoint{0.879004in}{1.070932in}}%
\pgfpathlineto{\pgfqpoint{0.875713in}{1.068161in}}%
\pgfpathlineto{\pgfqpoint{0.872418in}{1.065555in}}%
\pgfpathlineto{\pgfqpoint{0.869121in}{1.063117in}}%
\pgfpathlineto{\pgfqpoint{0.865820in}{1.060852in}}%
\pgfpathlineto{\pgfqpoint{0.867532in}{1.068798in}}%
\pgfpathlineto{\pgfqpoint{0.869721in}{1.076703in}}%
\pgfpathlineto{\pgfqpoint{0.872380in}{1.084559in}}%
\pgfpathlineto{\pgfqpoint{0.875508in}{1.092358in}}%
\pgfpathlineto{\pgfqpoint{0.878727in}{1.094399in}}%
\pgfpathlineto{\pgfqpoint{0.881944in}{1.096612in}}%
\pgfpathlineto{\pgfqpoint{0.885157in}{1.098993in}}%
\pgfpathlineto{\pgfqpoint{0.888368in}{1.101538in}}%
\pgfpathlineto{\pgfqpoint{0.885340in}{1.093961in}}%
\pgfpathlineto{\pgfqpoint{0.882767in}{1.086330in}}%
\pgfpathlineto{\pgfqpoint{0.880654in}{1.078651in}}%
\pgfpathlineto{\pgfqpoint{0.879004in}{1.070932in}}%
\pgfpathclose%
\pgfusepath{fill}%
\end{pgfscope}%
\begin{pgfscope}%
\pgfpathrectangle{\pgfqpoint{0.329460in}{0.284240in}}{\pgfqpoint{1.989680in}{1.989680in}}%
\pgfusepath{clip}%
\pgfsetbuttcap%
\pgfsetroundjoin%
\definecolor{currentfill}{rgb}{0.282884,0.135920,0.453427}%
\pgfsetfillcolor{currentfill}%
\pgfsetlinewidth{0.000000pt}%
\definecolor{currentstroke}{rgb}{0.000000,0.000000,0.000000}%
\pgfsetstrokecolor{currentstroke}%
\pgfsetdash{}{0pt}%
\pgfpathmoveto{\pgfqpoint{1.928895in}{1.150497in}}%
\pgfpathlineto{\pgfqpoint{1.932323in}{1.156532in}}%
\pgfpathlineto{\pgfqpoint{1.935763in}{1.162892in}}%
\pgfpathlineto{\pgfqpoint{1.939215in}{1.169583in}}%
\pgfpathlineto{\pgfqpoint{1.942680in}{1.176609in}}%
\pgfpathlineto{\pgfqpoint{1.947291in}{1.167032in}}%
\pgfpathlineto{\pgfqpoint{1.951326in}{1.157370in}}%
\pgfpathlineto{\pgfqpoint{1.954779in}{1.147633in}}%
\pgfpathlineto{\pgfqpoint{1.957643in}{1.137829in}}%
\pgfpathlineto{\pgfqpoint{1.954069in}{1.130989in}}%
\pgfpathlineto{\pgfqpoint{1.950508in}{1.124486in}}%
\pgfpathlineto{\pgfqpoint{1.946961in}{1.118314in}}%
\pgfpathlineto{\pgfqpoint{1.943425in}{1.112470in}}%
\pgfpathlineto{\pgfqpoint{1.940649in}{1.122082in}}%
\pgfpathlineto{\pgfqpoint{1.937299in}{1.131629in}}%
\pgfpathlineto{\pgfqpoint{1.933378in}{1.141104in}}%
\pgfpathlineto{\pgfqpoint{1.928895in}{1.150497in}}%
\pgfpathclose%
\pgfusepath{fill}%
\end{pgfscope}%
\begin{pgfscope}%
\pgfpathrectangle{\pgfqpoint{0.329460in}{0.284240in}}{\pgfqpoint{1.989680in}{1.989680in}}%
\pgfusepath{clip}%
\pgfsetbuttcap%
\pgfsetroundjoin%
\definecolor{currentfill}{rgb}{0.274128,0.199721,0.498911}%
\pgfsetfillcolor{currentfill}%
\pgfsetlinewidth{0.000000pt}%
\definecolor{currentstroke}{rgb}{0.000000,0.000000,0.000000}%
\pgfsetstrokecolor{currentstroke}%
\pgfsetdash{}{0pt}%
\pgfpathmoveto{\pgfqpoint{1.732171in}{1.212069in}}%
\pgfpathlineto{\pgfqpoint{1.735200in}{1.207254in}}%
\pgfpathlineto{\pgfqpoint{1.738228in}{1.202533in}}%
\pgfpathlineto{\pgfqpoint{1.741257in}{1.197910in}}%
\pgfpathlineto{\pgfqpoint{1.744285in}{1.193389in}}%
\pgfpathlineto{\pgfqpoint{1.748845in}{1.187031in}}%
\pgfpathlineto{\pgfqpoint{1.753023in}{1.180595in}}%
\pgfpathlineto{\pgfqpoint{1.756811in}{1.174089in}}%
\pgfpathlineto{\pgfqpoint{1.760206in}{1.167518in}}%
\pgfpathlineto{\pgfqpoint{1.757042in}{1.172258in}}%
\pgfpathlineto{\pgfqpoint{1.753879in}{1.177099in}}%
\pgfpathlineto{\pgfqpoint{1.750715in}{1.182038in}}%
\pgfpathlineto{\pgfqpoint{1.747552in}{1.187072in}}%
\pgfpathlineto{\pgfqpoint{1.744274in}{1.193421in}}%
\pgfpathlineto{\pgfqpoint{1.740615in}{1.199707in}}%
\pgfpathlineto{\pgfqpoint{1.736579in}{1.205925in}}%
\pgfpathlineto{\pgfqpoint{1.732171in}{1.212069in}}%
\pgfpathclose%
\pgfusepath{fill}%
\end{pgfscope}%
\begin{pgfscope}%
\pgfpathrectangle{\pgfqpoint{0.329460in}{0.284240in}}{\pgfqpoint{1.989680in}{1.989680in}}%
\pgfusepath{clip}%
\pgfsetbuttcap%
\pgfsetroundjoin%
\definecolor{currentfill}{rgb}{0.133743,0.548535,0.553541}%
\pgfsetfillcolor{currentfill}%
\pgfsetlinewidth{0.000000pt}%
\definecolor{currentstroke}{rgb}{0.000000,0.000000,0.000000}%
\pgfsetstrokecolor{currentstroke}%
\pgfsetdash{}{0pt}%
\pgfpathmoveto{\pgfqpoint{1.214396in}{1.499258in}}%
\pgfpathlineto{\pgfqpoint{1.212851in}{1.493907in}}%
\pgfpathlineto{\pgfqpoint{1.211307in}{1.488551in}}%
\pgfpathlineto{\pgfqpoint{1.209765in}{1.483191in}}%
\pgfpathlineto{\pgfqpoint{1.208225in}{1.477832in}}%
\pgfpathlineto{\pgfqpoint{1.216538in}{1.479962in}}%
\pgfpathlineto{\pgfqpoint{1.224977in}{1.481963in}}%
\pgfpathlineto{\pgfqpoint{1.233533in}{1.483834in}}%
\pgfpathlineto{\pgfqpoint{1.242199in}{1.485571in}}%
\pgfpathlineto{\pgfqpoint{1.243372in}{1.490841in}}%
\pgfpathlineto{\pgfqpoint{1.244546in}{1.496110in}}%
\pgfpathlineto{\pgfqpoint{1.245721in}{1.501377in}}%
\pgfpathlineto{\pgfqpoint{1.246898in}{1.506638in}}%
\pgfpathlineto{\pgfqpoint{1.238607in}{1.504981in}}%
\pgfpathlineto{\pgfqpoint{1.230421in}{1.503198in}}%
\pgfpathlineto{\pgfqpoint{1.222349in}{1.501289in}}%
\pgfpathlineto{\pgfqpoint{1.214396in}{1.499258in}}%
\pgfpathclose%
\pgfusepath{fill}%
\end{pgfscope}%
\begin{pgfscope}%
\pgfpathrectangle{\pgfqpoint{0.329460in}{0.284240in}}{\pgfqpoint{1.989680in}{1.989680in}}%
\pgfusepath{clip}%
\pgfsetbuttcap%
\pgfsetroundjoin%
\definecolor{currentfill}{rgb}{0.248629,0.278775,0.534556}%
\pgfsetfillcolor{currentfill}%
\pgfsetlinewidth{0.000000pt}%
\definecolor{currentstroke}{rgb}{0.000000,0.000000,0.000000}%
\pgfsetstrokecolor{currentstroke}%
\pgfsetdash{}{0pt}%
\pgfpathmoveto{\pgfqpoint{0.990796in}{1.248561in}}%
\pgfpathlineto{\pgfqpoint{0.987726in}{1.243050in}}%
\pgfpathlineto{\pgfqpoint{0.984658in}{1.237610in}}%
\pgfpathlineto{\pgfqpoint{0.981591in}{1.232242in}}%
\pgfpathlineto{\pgfqpoint{0.978524in}{1.226951in}}%
\pgfpathlineto{\pgfqpoint{0.982821in}{1.232872in}}%
\pgfpathlineto{\pgfqpoint{0.987472in}{1.238717in}}%
\pgfpathlineto{\pgfqpoint{0.992472in}{1.244479in}}%
\pgfpathlineto{\pgfqpoint{0.997817in}{1.250153in}}%
\pgfpathlineto{\pgfqpoint{1.000714in}{1.255235in}}%
\pgfpathlineto{\pgfqpoint{1.003612in}{1.260393in}}%
\pgfpathlineto{\pgfqpoint{1.006511in}{1.265624in}}%
\pgfpathlineto{\pgfqpoint{1.009412in}{1.270925in}}%
\pgfpathlineto{\pgfqpoint{1.004253in}{1.265455in}}%
\pgfpathlineto{\pgfqpoint{0.999427in}{1.259901in}}%
\pgfpathlineto{\pgfqpoint{0.994940in}{1.254268in}}%
\pgfpathlineto{\pgfqpoint{0.990796in}{1.248561in}}%
\pgfpathclose%
\pgfusepath{fill}%
\end{pgfscope}%
\begin{pgfscope}%
\pgfpathrectangle{\pgfqpoint{0.329460in}{0.284240in}}{\pgfqpoint{1.989680in}{1.989680in}}%
\pgfusepath{clip}%
\pgfsetbuttcap%
\pgfsetroundjoin%
\definecolor{currentfill}{rgb}{0.282327,0.094955,0.417331}%
\pgfsetfillcolor{currentfill}%
\pgfsetlinewidth{0.000000pt}%
\definecolor{currentstroke}{rgb}{0.000000,0.000000,0.000000}%
\pgfsetstrokecolor{currentstroke}%
\pgfsetdash{}{0pt}%
\pgfpathmoveto{\pgfqpoint{0.771058in}{1.083833in}}%
\pgfpathlineto{\pgfqpoint{0.767553in}{1.088382in}}%
\pgfpathlineto{\pgfqpoint{0.764037in}{1.093235in}}%
\pgfpathlineto{\pgfqpoint{0.760509in}{1.098399in}}%
\pgfpathlineto{\pgfqpoint{0.756970in}{1.103879in}}%
\pgfpathlineto{\pgfqpoint{0.759230in}{1.113541in}}%
\pgfpathlineto{\pgfqpoint{0.762070in}{1.123146in}}%
\pgfpathlineto{\pgfqpoint{0.765484in}{1.132686in}}%
\pgfpathlineto{\pgfqpoint{0.769467in}{1.142152in}}%
\pgfpathlineto{\pgfqpoint{0.772912in}{1.136478in}}%
\pgfpathlineto{\pgfqpoint{0.776345in}{1.131118in}}%
\pgfpathlineto{\pgfqpoint{0.779767in}{1.126067in}}%
\pgfpathlineto{\pgfqpoint{0.783179in}{1.121320in}}%
\pgfpathlineto{\pgfqpoint{0.779310in}{1.112048in}}%
\pgfpathlineto{\pgfqpoint{0.775996in}{1.102703in}}%
\pgfpathlineto{\pgfqpoint{0.773244in}{1.093295in}}%
\pgfpathlineto{\pgfqpoint{0.771058in}{1.083833in}}%
\pgfpathclose%
\pgfusepath{fill}%
\end{pgfscope}%
\begin{pgfscope}%
\pgfpathrectangle{\pgfqpoint{0.329460in}{0.284240in}}{\pgfqpoint{1.989680in}{1.989680in}}%
\pgfusepath{clip}%
\pgfsetbuttcap%
\pgfsetroundjoin%
\definecolor{currentfill}{rgb}{0.212395,0.359683,0.551710}%
\pgfsetfillcolor{currentfill}%
\pgfsetlinewidth{0.000000pt}%
\definecolor{currentstroke}{rgb}{0.000000,0.000000,0.000000}%
\pgfsetstrokecolor{currentstroke}%
\pgfsetdash{}{0pt}%
\pgfpathmoveto{\pgfqpoint{1.032674in}{1.315509in}}%
\pgfpathlineto{\pgfqpoint{1.029759in}{1.309752in}}%
\pgfpathlineto{\pgfqpoint{1.026847in}{1.304042in}}%
\pgfpathlineto{\pgfqpoint{1.023937in}{1.298381in}}%
\pgfpathlineto{\pgfqpoint{1.021029in}{1.292773in}}%
\pgfpathlineto{\pgfqpoint{1.026318in}{1.297953in}}%
\pgfpathlineto{\pgfqpoint{1.031919in}{1.303042in}}%
\pgfpathlineto{\pgfqpoint{1.037824in}{1.308036in}}%
\pgfpathlineto{\pgfqpoint{1.044027in}{1.312931in}}%
\pgfpathlineto{\pgfqpoint{1.046724in}{1.318344in}}%
\pgfpathlineto{\pgfqpoint{1.049423in}{1.323810in}}%
\pgfpathlineto{\pgfqpoint{1.052124in}{1.329325in}}%
\pgfpathlineto{\pgfqpoint{1.054826in}{1.334888in}}%
\pgfpathlineto{\pgfqpoint{1.048850in}{1.330182in}}%
\pgfpathlineto{\pgfqpoint{1.043162in}{1.325380in}}%
\pgfpathlineto{\pgfqpoint{1.037768in}{1.320488in}}%
\pgfpathlineto{\pgfqpoint{1.032674in}{1.315509in}}%
\pgfpathclose%
\pgfusepath{fill}%
\end{pgfscope}%
\begin{pgfscope}%
\pgfpathrectangle{\pgfqpoint{0.329460in}{0.284240in}}{\pgfqpoint{1.989680in}{1.989680in}}%
\pgfusepath{clip}%
\pgfsetbuttcap%
\pgfsetroundjoin%
\definecolor{currentfill}{rgb}{0.163625,0.471133,0.558148}%
\pgfsetfillcolor{currentfill}%
\pgfsetlinewidth{0.000000pt}%
\definecolor{currentstroke}{rgb}{0.000000,0.000000,0.000000}%
\pgfsetstrokecolor{currentstroke}%
\pgfsetdash{}{0pt}%
\pgfpathmoveto{\pgfqpoint{1.557488in}{1.436863in}}%
\pgfpathlineto{\pgfqpoint{1.559612in}{1.431320in}}%
\pgfpathlineto{\pgfqpoint{1.561735in}{1.425791in}}%
\pgfpathlineto{\pgfqpoint{1.563855in}{1.420279in}}%
\pgfpathlineto{\pgfqpoint{1.565973in}{1.414787in}}%
\pgfpathlineto{\pgfqpoint{1.573598in}{1.411395in}}%
\pgfpathlineto{\pgfqpoint{1.581013in}{1.407885in}}%
\pgfpathlineto{\pgfqpoint{1.588212in}{1.404259in}}%
\pgfpathlineto{\pgfqpoint{1.595188in}{1.400520in}}%
\pgfpathlineto{\pgfqpoint{1.592776in}{1.406166in}}%
\pgfpathlineto{\pgfqpoint{1.590362in}{1.411832in}}%
\pgfpathlineto{\pgfqpoint{1.587945in}{1.417514in}}%
\pgfpathlineto{\pgfqpoint{1.585526in}{1.423211in}}%
\pgfpathlineto{\pgfqpoint{1.578832in}{1.426788in}}%
\pgfpathlineto{\pgfqpoint{1.571923in}{1.430258in}}%
\pgfpathlineto{\pgfqpoint{1.564806in}{1.433617in}}%
\pgfpathlineto{\pgfqpoint{1.557488in}{1.436863in}}%
\pgfpathclose%
\pgfusepath{fill}%
\end{pgfscope}%
\begin{pgfscope}%
\pgfpathrectangle{\pgfqpoint{0.329460in}{0.284240in}}{\pgfqpoint{1.989680in}{1.989680in}}%
\pgfusepath{clip}%
\pgfsetbuttcap%
\pgfsetroundjoin%
\definecolor{currentfill}{rgb}{0.122606,0.585371,0.546557}%
\pgfsetfillcolor{currentfill}%
\pgfsetlinewidth{0.000000pt}%
\definecolor{currentstroke}{rgb}{0.000000,0.000000,0.000000}%
\pgfsetstrokecolor{currentstroke}%
\pgfsetdash{}{0pt}%
\pgfpathmoveto{\pgfqpoint{1.317648in}{1.535722in}}%
\pgfpathlineto{\pgfqpoint{1.317250in}{1.530609in}}%
\pgfpathlineto{\pgfqpoint{1.316852in}{1.525482in}}%
\pgfpathlineto{\pgfqpoint{1.316455in}{1.520343in}}%
\pgfpathlineto{\pgfqpoint{1.316059in}{1.515194in}}%
\pgfpathlineto{\pgfqpoint{1.324937in}{1.515662in}}%
\pgfpathlineto{\pgfqpoint{1.333839in}{1.515994in}}%
\pgfpathlineto{\pgfqpoint{1.342757in}{1.516191in}}%
\pgfpathlineto{\pgfqpoint{1.351684in}{1.516251in}}%
\pgfpathlineto{\pgfqpoint{1.351678in}{1.521388in}}%
\pgfpathlineto{\pgfqpoint{1.351672in}{1.526515in}}%
\pgfpathlineto{\pgfqpoint{1.351667in}{1.531629in}}%
\pgfpathlineto{\pgfqpoint{1.351661in}{1.536728in}}%
\pgfpathlineto{\pgfqpoint{1.343139in}{1.536671in}}%
\pgfpathlineto{\pgfqpoint{1.334624in}{1.536484in}}%
\pgfpathlineto{\pgfqpoint{1.326124in}{1.536167in}}%
\pgfpathlineto{\pgfqpoint{1.317648in}{1.535722in}}%
\pgfpathclose%
\pgfusepath{fill}%
\end{pgfscope}%
\begin{pgfscope}%
\pgfpathrectangle{\pgfqpoint{0.329460in}{0.284240in}}{\pgfqpoint{1.989680in}{1.989680in}}%
\pgfusepath{clip}%
\pgfsetbuttcap%
\pgfsetroundjoin%
\definecolor{currentfill}{rgb}{0.122606,0.585371,0.546557}%
\pgfsetfillcolor{currentfill}%
\pgfsetlinewidth{0.000000pt}%
\definecolor{currentstroke}{rgb}{0.000000,0.000000,0.000000}%
\pgfsetstrokecolor{currentstroke}%
\pgfsetdash{}{0pt}%
\pgfpathmoveto{\pgfqpoint{1.351661in}{1.536728in}}%
\pgfpathlineto{\pgfqpoint{1.351667in}{1.531629in}}%
\pgfpathlineto{\pgfqpoint{1.351672in}{1.526515in}}%
\pgfpathlineto{\pgfqpoint{1.351678in}{1.521388in}}%
\pgfpathlineto{\pgfqpoint{1.351684in}{1.516251in}}%
\pgfpathlineto{\pgfqpoint{1.360609in}{1.516176in}}%
\pgfpathlineto{\pgfqpoint{1.369526in}{1.515964in}}%
\pgfpathlineto{\pgfqpoint{1.378426in}{1.515616in}}%
\pgfpathlineto{\pgfqpoint{1.387301in}{1.515133in}}%
\pgfpathlineto{\pgfqpoint{1.386893in}{1.520283in}}%
\pgfpathlineto{\pgfqpoint{1.386485in}{1.525424in}}%
\pgfpathlineto{\pgfqpoint{1.386077in}{1.530551in}}%
\pgfpathlineto{\pgfqpoint{1.385668in}{1.535664in}}%
\pgfpathlineto{\pgfqpoint{1.377194in}{1.536124in}}%
\pgfpathlineto{\pgfqpoint{1.368697in}{1.536455in}}%
\pgfpathlineto{\pgfqpoint{1.360183in}{1.536656in}}%
\pgfpathlineto{\pgfqpoint{1.351661in}{1.536728in}}%
\pgfpathclose%
\pgfusepath{fill}%
\end{pgfscope}%
\begin{pgfscope}%
\pgfpathrectangle{\pgfqpoint{0.329460in}{0.284240in}}{\pgfqpoint{1.989680in}{1.989680in}}%
\pgfusepath{clip}%
\pgfsetbuttcap%
\pgfsetroundjoin%
\definecolor{currentfill}{rgb}{0.147607,0.511733,0.557049}%
\pgfsetfillcolor{currentfill}%
\pgfsetlinewidth{0.000000pt}%
\definecolor{currentstroke}{rgb}{0.000000,0.000000,0.000000}%
\pgfsetstrokecolor{currentstroke}%
\pgfsetdash{}{0pt}%
\pgfpathmoveto{\pgfqpoint{1.519124in}{1.470398in}}%
\pgfpathlineto{\pgfqpoint{1.520933in}{1.464953in}}%
\pgfpathlineto{\pgfqpoint{1.522740in}{1.459513in}}%
\pgfpathlineto{\pgfqpoint{1.524545in}{1.454080in}}%
\pgfpathlineto{\pgfqpoint{1.526347in}{1.448657in}}%
\pgfpathlineto{\pgfqpoint{1.534397in}{1.445892in}}%
\pgfpathlineto{\pgfqpoint{1.542276in}{1.443003in}}%
\pgfpathlineto{\pgfqpoint{1.549975in}{1.439993in}}%
\pgfpathlineto{\pgfqpoint{1.557488in}{1.436863in}}%
\pgfpathlineto{\pgfqpoint{1.555360in}{1.442419in}}%
\pgfpathlineto{\pgfqpoint{1.553231in}{1.447984in}}%
\pgfpathlineto{\pgfqpoint{1.551099in}{1.453556in}}%
\pgfpathlineto{\pgfqpoint{1.548964in}{1.459132in}}%
\pgfpathlineto{\pgfqpoint{1.541766in}{1.462122in}}%
\pgfpathlineto{\pgfqpoint{1.534389in}{1.464997in}}%
\pgfpathlineto{\pgfqpoint{1.526839in}{1.467757in}}%
\pgfpathlineto{\pgfqpoint{1.519124in}{1.470398in}}%
\pgfpathclose%
\pgfusepath{fill}%
\end{pgfscope}%
\begin{pgfscope}%
\pgfpathrectangle{\pgfqpoint{0.329460in}{0.284240in}}{\pgfqpoint{1.989680in}{1.989680in}}%
\pgfusepath{clip}%
\pgfsetbuttcap%
\pgfsetroundjoin%
\definecolor{currentfill}{rgb}{0.282327,0.094955,0.417331}%
\pgfsetfillcolor{currentfill}%
\pgfsetlinewidth{0.000000pt}%
\definecolor{currentstroke}{rgb}{0.000000,0.000000,0.000000}%
\pgfsetstrokecolor{currentstroke}%
\pgfsetdash{}{0pt}%
\pgfpathmoveto{\pgfqpoint{1.785530in}{1.133652in}}%
\pgfpathlineto{\pgfqpoint{1.788700in}{1.129976in}}%
\pgfpathlineto{\pgfqpoint{1.791872in}{1.126434in}}%
\pgfpathlineto{\pgfqpoint{1.795044in}{1.123031in}}%
\pgfpathlineto{\pgfqpoint{1.798219in}{1.119771in}}%
\pgfpathlineto{\pgfqpoint{1.801538in}{1.112470in}}%
\pgfpathlineto{\pgfqpoint{1.804418in}{1.105111in}}%
\pgfpathlineto{\pgfqpoint{1.806854in}{1.097699in}}%
\pgfpathlineto{\pgfqpoint{1.808844in}{1.090243in}}%
\pgfpathlineto{\pgfqpoint{1.805579in}{1.093730in}}%
\pgfpathlineto{\pgfqpoint{1.802316in}{1.097358in}}%
\pgfpathlineto{\pgfqpoint{1.799055in}{1.101126in}}%
\pgfpathlineto{\pgfqpoint{1.795796in}{1.105029in}}%
\pgfpathlineto{\pgfqpoint{1.793877in}{1.112256in}}%
\pgfpathlineto{\pgfqpoint{1.791525in}{1.119440in}}%
\pgfpathlineto{\pgfqpoint{1.788741in}{1.126575in}}%
\pgfpathlineto{\pgfqpoint{1.785530in}{1.133652in}}%
\pgfpathclose%
\pgfusepath{fill}%
\end{pgfscope}%
\begin{pgfscope}%
\pgfpathrectangle{\pgfqpoint{0.329460in}{0.284240in}}{\pgfqpoint{1.989680in}{1.989680in}}%
\pgfusepath{clip}%
\pgfsetbuttcap%
\pgfsetroundjoin%
\definecolor{currentfill}{rgb}{0.201239,0.383670,0.554294}%
\pgfsetfillcolor{currentfill}%
\pgfsetlinewidth{0.000000pt}%
\definecolor{currentstroke}{rgb}{0.000000,0.000000,0.000000}%
\pgfsetstrokecolor{currentstroke}%
\pgfsetdash{}{0pt}%
\pgfpathmoveto{\pgfqpoint{1.959324in}{1.328525in}}%
\pgfpathlineto{\pgfqpoint{1.962815in}{1.340614in}}%
\pgfpathlineto{\pgfqpoint{1.966323in}{1.353124in}}%
\pgfpathlineto{\pgfqpoint{1.969848in}{1.366061in}}%
\pgfpathlineto{\pgfqpoint{1.973392in}{1.379433in}}%
\pgfpathlineto{\pgfqpoint{1.980980in}{1.369659in}}%
\pgfpathlineto{\pgfqpoint{1.987973in}{1.359755in}}%
\pgfpathlineto{\pgfqpoint{1.994363in}{1.349730in}}%
\pgfpathlineto{\pgfqpoint{2.000140in}{1.339594in}}%
\pgfpathlineto{\pgfqpoint{1.996420in}{1.326364in}}%
\pgfpathlineto{\pgfqpoint{1.992720in}{1.313571in}}%
\pgfpathlineto{\pgfqpoint{1.989038in}{1.301208in}}%
\pgfpathlineto{\pgfqpoint{1.985375in}{1.289269in}}%
\pgfpathlineto{\pgfqpoint{1.979753in}{1.299256in}}%
\pgfpathlineto{\pgfqpoint{1.973531in}{1.309134in}}%
\pgfpathlineto{\pgfqpoint{1.966719in}{1.318893in}}%
\pgfpathlineto{\pgfqpoint{1.959324in}{1.328525in}}%
\pgfpathclose%
\pgfusepath{fill}%
\end{pgfscope}%
\begin{pgfscope}%
\pgfpathrectangle{\pgfqpoint{0.329460in}{0.284240in}}{\pgfqpoint{1.989680in}{1.989680in}}%
\pgfusepath{clip}%
\pgfsetbuttcap%
\pgfsetroundjoin%
\definecolor{currentfill}{rgb}{0.179019,0.433756,0.557430}%
\pgfsetfillcolor{currentfill}%
\pgfsetlinewidth{0.000000pt}%
\definecolor{currentstroke}{rgb}{0.000000,0.000000,0.000000}%
\pgfsetstrokecolor{currentstroke}%
\pgfsetdash{}{0pt}%
\pgfpathmoveto{\pgfqpoint{1.595188in}{1.400520in}}%
\pgfpathlineto{\pgfqpoint{1.597597in}{1.394897in}}%
\pgfpathlineto{\pgfqpoint{1.600004in}{1.389298in}}%
\pgfpathlineto{\pgfqpoint{1.602409in}{1.383728in}}%
\pgfpathlineto{\pgfqpoint{1.604811in}{1.378187in}}%
\pgfpathlineto{\pgfqpoint{1.611829in}{1.374174in}}%
\pgfpathlineto{\pgfqpoint{1.618600in}{1.370049in}}%
\pgfpathlineto{\pgfqpoint{1.625118in}{1.365819in}}%
\pgfpathlineto{\pgfqpoint{1.631377in}{1.361485in}}%
\pgfpathlineto{\pgfqpoint{1.628715in}{1.367199in}}%
\pgfpathlineto{\pgfqpoint{1.626052in}{1.372943in}}%
\pgfpathlineto{\pgfqpoint{1.623386in}{1.378715in}}%
\pgfpathlineto{\pgfqpoint{1.620717in}{1.384512in}}%
\pgfpathlineto{\pgfqpoint{1.614704in}{1.388665in}}%
\pgfpathlineto{\pgfqpoint{1.608440in}{1.392720in}}%
\pgfpathlineto{\pgfqpoint{1.601933in}{1.396673in}}%
\pgfpathlineto{\pgfqpoint{1.595188in}{1.400520in}}%
\pgfpathclose%
\pgfusepath{fill}%
\end{pgfscope}%
\begin{pgfscope}%
\pgfpathrectangle{\pgfqpoint{0.329460in}{0.284240in}}{\pgfqpoint{1.989680in}{1.989680in}}%
\pgfusepath{clip}%
\pgfsetbuttcap%
\pgfsetroundjoin%
\definecolor{currentfill}{rgb}{0.279566,0.067836,0.391917}%
\pgfsetfillcolor{currentfill}%
\pgfsetlinewidth{0.000000pt}%
\definecolor{currentstroke}{rgb}{0.000000,0.000000,0.000000}%
\pgfsetstrokecolor{currentstroke}%
\pgfsetdash{}{0pt}%
\pgfpathmoveto{\pgfqpoint{0.892141in}{1.083584in}}%
\pgfpathlineto{\pgfqpoint{0.888860in}{1.080193in}}%
\pgfpathlineto{\pgfqpoint{0.885578in}{1.076951in}}%
\pgfpathlineto{\pgfqpoint{0.882292in}{1.073863in}}%
\pgfpathlineto{\pgfqpoint{0.879004in}{1.070932in}}%
\pgfpathlineto{\pgfqpoint{0.880654in}{1.078651in}}%
\pgfpathlineto{\pgfqpoint{0.882767in}{1.086330in}}%
\pgfpathlineto{\pgfqpoint{0.885340in}{1.093961in}}%
\pgfpathlineto{\pgfqpoint{0.888368in}{1.101538in}}%
\pgfpathlineto{\pgfqpoint{0.891575in}{1.104243in}}%
\pgfpathlineto{\pgfqpoint{0.894781in}{1.107105in}}%
\pgfpathlineto{\pgfqpoint{0.897984in}{1.110120in}}%
\pgfpathlineto{\pgfqpoint{0.901185in}{1.113284in}}%
\pgfpathlineto{\pgfqpoint{0.898256in}{1.105931in}}%
\pgfpathlineto{\pgfqpoint{0.895769in}{1.098525in}}%
\pgfpathlineto{\pgfqpoint{0.893730in}{1.091074in}}%
\pgfpathlineto{\pgfqpoint{0.892141in}{1.083584in}}%
\pgfpathclose%
\pgfusepath{fill}%
\end{pgfscope}%
\begin{pgfscope}%
\pgfpathrectangle{\pgfqpoint{0.329460in}{0.284240in}}{\pgfqpoint{1.989680in}{1.989680in}}%
\pgfusepath{clip}%
\pgfsetbuttcap%
\pgfsetroundjoin%
\definecolor{currentfill}{rgb}{0.122606,0.585371,0.546557}%
\pgfsetfillcolor{currentfill}%
\pgfsetlinewidth{0.000000pt}%
\definecolor{currentstroke}{rgb}{0.000000,0.000000,0.000000}%
\pgfsetstrokecolor{currentstroke}%
\pgfsetdash{}{0pt}%
\pgfpathmoveto{\pgfqpoint{1.284134in}{1.532659in}}%
\pgfpathlineto{\pgfqpoint{1.283338in}{1.527507in}}%
\pgfpathlineto{\pgfqpoint{1.282543in}{1.522341in}}%
\pgfpathlineto{\pgfqpoint{1.281749in}{1.517163in}}%
\pgfpathlineto{\pgfqpoint{1.280956in}{1.511974in}}%
\pgfpathlineto{\pgfqpoint{1.289654in}{1.512980in}}%
\pgfpathlineto{\pgfqpoint{1.298409in}{1.513852in}}%
\pgfpathlineto{\pgfqpoint{1.307214in}{1.514590in}}%
\pgfpathlineto{\pgfqpoint{1.316059in}{1.515194in}}%
\pgfpathlineto{\pgfqpoint{1.316455in}{1.520343in}}%
\pgfpathlineto{\pgfqpoint{1.316852in}{1.525482in}}%
\pgfpathlineto{\pgfqpoint{1.317250in}{1.530609in}}%
\pgfpathlineto{\pgfqpoint{1.317648in}{1.535722in}}%
\pgfpathlineto{\pgfqpoint{1.309203in}{1.535148in}}%
\pgfpathlineto{\pgfqpoint{1.300797in}{1.534445in}}%
\pgfpathlineto{\pgfqpoint{1.292438in}{1.533615in}}%
\pgfpathlineto{\pgfqpoint{1.284134in}{1.532659in}}%
\pgfpathclose%
\pgfusepath{fill}%
\end{pgfscope}%
\begin{pgfscope}%
\pgfpathrectangle{\pgfqpoint{0.329460in}{0.284240in}}{\pgfqpoint{1.989680in}{1.989680in}}%
\pgfusepath{clip}%
\pgfsetbuttcap%
\pgfsetroundjoin%
\definecolor{currentfill}{rgb}{0.122606,0.585371,0.546557}%
\pgfsetfillcolor{currentfill}%
\pgfsetlinewidth{0.000000pt}%
\definecolor{currentstroke}{rgb}{0.000000,0.000000,0.000000}%
\pgfsetstrokecolor{currentstroke}%
\pgfsetdash{}{0pt}%
\pgfpathmoveto{\pgfqpoint{1.385668in}{1.535664in}}%
\pgfpathlineto{\pgfqpoint{1.386077in}{1.530551in}}%
\pgfpathlineto{\pgfqpoint{1.386485in}{1.525424in}}%
\pgfpathlineto{\pgfqpoint{1.386893in}{1.520283in}}%
\pgfpathlineto{\pgfqpoint{1.387301in}{1.515133in}}%
\pgfpathlineto{\pgfqpoint{1.396142in}{1.514515in}}%
\pgfpathlineto{\pgfqpoint{1.404941in}{1.513762in}}%
\pgfpathlineto{\pgfqpoint{1.413690in}{1.512875in}}%
\pgfpathlineto{\pgfqpoint{1.422382in}{1.511854in}}%
\pgfpathlineto{\pgfqpoint{1.421578in}{1.517044in}}%
\pgfpathlineto{\pgfqpoint{1.420773in}{1.522224in}}%
\pgfpathlineto{\pgfqpoint{1.419967in}{1.527392in}}%
\pgfpathlineto{\pgfqpoint{1.419160in}{1.532544in}}%
\pgfpathlineto{\pgfqpoint{1.410863in}{1.533515in}}%
\pgfpathlineto{\pgfqpoint{1.402510in}{1.534359in}}%
\pgfpathlineto{\pgfqpoint{1.394109in}{1.535076in}}%
\pgfpathlineto{\pgfqpoint{1.385668in}{1.535664in}}%
\pgfpathclose%
\pgfusepath{fill}%
\end{pgfscope}%
\begin{pgfscope}%
\pgfpathrectangle{\pgfqpoint{0.329460in}{0.284240in}}{\pgfqpoint{1.989680in}{1.989680in}}%
\pgfusepath{clip}%
\pgfsetbuttcap%
\pgfsetroundjoin%
\definecolor{currentfill}{rgb}{0.163625,0.471133,0.558148}%
\pgfsetfillcolor{currentfill}%
\pgfsetlinewidth{0.000000pt}%
\definecolor{currentstroke}{rgb}{0.000000,0.000000,0.000000}%
\pgfsetstrokecolor{currentstroke}%
\pgfsetdash{}{0pt}%
\pgfpathmoveto{\pgfqpoint{1.111085in}{1.419944in}}%
\pgfpathlineto{\pgfqpoint{1.108605in}{1.414210in}}%
\pgfpathlineto{\pgfqpoint{1.106128in}{1.408491in}}%
\pgfpathlineto{\pgfqpoint{1.103653in}{1.402789in}}%
\pgfpathlineto{\pgfqpoint{1.101180in}{1.397106in}}%
\pgfpathlineto{\pgfqpoint{1.107951in}{1.400941in}}%
\pgfpathlineto{\pgfqpoint{1.114952in}{1.404667in}}%
\pgfpathlineto{\pgfqpoint{1.122175in}{1.408280in}}%
\pgfpathlineto{\pgfqpoint{1.129614in}{1.411778in}}%
\pgfpathlineto{\pgfqpoint{1.131800in}{1.417302in}}%
\pgfpathlineto{\pgfqpoint{1.133989in}{1.422846in}}%
\pgfpathlineto{\pgfqpoint{1.136179in}{1.428408in}}%
\pgfpathlineto{\pgfqpoint{1.138372in}{1.433984in}}%
\pgfpathlineto{\pgfqpoint{1.131232in}{1.430637in}}%
\pgfpathlineto{\pgfqpoint{1.124300in}{1.427179in}}%
\pgfpathlineto{\pgfqpoint{1.117582in}{1.423614in}}%
\pgfpathlineto{\pgfqpoint{1.111085in}{1.419944in}}%
\pgfpathclose%
\pgfusepath{fill}%
\end{pgfscope}%
\begin{pgfscope}%
\pgfpathrectangle{\pgfqpoint{0.329460in}{0.284240in}}{\pgfqpoint{1.989680in}{1.989680in}}%
\pgfusepath{clip}%
\pgfsetbuttcap%
\pgfsetroundjoin%
\definecolor{currentfill}{rgb}{0.274128,0.199721,0.498911}%
\pgfsetfillcolor{currentfill}%
\pgfsetlinewidth{0.000000pt}%
\definecolor{currentstroke}{rgb}{0.000000,0.000000,0.000000}%
\pgfsetstrokecolor{currentstroke}%
\pgfsetdash{}{0pt}%
\pgfpathmoveto{\pgfqpoint{0.952235in}{1.181382in}}%
\pgfpathlineto{\pgfqpoint{0.949047in}{1.176298in}}%
\pgfpathlineto{\pgfqpoint{0.945860in}{1.171309in}}%
\pgfpathlineto{\pgfqpoint{0.942673in}{1.166418in}}%
\pgfpathlineto{\pgfqpoint{0.939486in}{1.161628in}}%
\pgfpathlineto{\pgfqpoint{0.942527in}{1.168251in}}%
\pgfpathlineto{\pgfqpoint{0.945966in}{1.174815in}}%
\pgfpathlineto{\pgfqpoint{0.949798in}{1.181314in}}%
\pgfpathlineto{\pgfqpoint{0.954018in}{1.187741in}}%
\pgfpathlineto{\pgfqpoint{0.957080in}{1.192310in}}%
\pgfpathlineto{\pgfqpoint{0.960142in}{1.196980in}}%
\pgfpathlineto{\pgfqpoint{0.963205in}{1.201748in}}%
\pgfpathlineto{\pgfqpoint{0.966267in}{1.206611in}}%
\pgfpathlineto{\pgfqpoint{0.962190in}{1.200401in}}%
\pgfpathlineto{\pgfqpoint{0.958488in}{1.194122in}}%
\pgfpathlineto{\pgfqpoint{0.955169in}{1.187781in}}%
\pgfpathlineto{\pgfqpoint{0.952235in}{1.181382in}}%
\pgfpathclose%
\pgfusepath{fill}%
\end{pgfscope}%
\begin{pgfscope}%
\pgfpathrectangle{\pgfqpoint{0.329460in}{0.284240in}}{\pgfqpoint{1.989680in}{1.989680in}}%
\pgfusepath{clip}%
\pgfsetbuttcap%
\pgfsetroundjoin%
\definecolor{currentfill}{rgb}{0.147607,0.511733,0.557049}%
\pgfsetfillcolor{currentfill}%
\pgfsetlinewidth{0.000000pt}%
\definecolor{currentstroke}{rgb}{0.000000,0.000000,0.000000}%
\pgfsetstrokecolor{currentstroke}%
\pgfsetdash{}{0pt}%
\pgfpathmoveto{\pgfqpoint{1.147169in}{1.456382in}}%
\pgfpathlineto{\pgfqpoint{1.144966in}{1.450773in}}%
\pgfpathlineto{\pgfqpoint{1.142766in}{1.445169in}}%
\pgfpathlineto{\pgfqpoint{1.140568in}{1.439572in}}%
\pgfpathlineto{\pgfqpoint{1.138372in}{1.433984in}}%
\pgfpathlineto{\pgfqpoint{1.145713in}{1.437217in}}%
\pgfpathlineto{\pgfqpoint{1.153247in}{1.440333in}}%
\pgfpathlineto{\pgfqpoint{1.160966in}{1.443330in}}%
\pgfpathlineto{\pgfqpoint{1.168864in}{1.446206in}}%
\pgfpathlineto{\pgfqpoint{1.170741in}{1.451657in}}%
\pgfpathlineto{\pgfqpoint{1.172620in}{1.457117in}}%
\pgfpathlineto{\pgfqpoint{1.174502in}{1.462584in}}%
\pgfpathlineto{\pgfqpoint{1.176386in}{1.468056in}}%
\pgfpathlineto{\pgfqpoint{1.168817in}{1.465310in}}%
\pgfpathlineto{\pgfqpoint{1.161420in}{1.462447in}}%
\pgfpathlineto{\pgfqpoint{1.154201in}{1.459470in}}%
\pgfpathlineto{\pgfqpoint{1.147169in}{1.456382in}}%
\pgfpathclose%
\pgfusepath{fill}%
\end{pgfscope}%
\begin{pgfscope}%
\pgfpathrectangle{\pgfqpoint{0.329460in}{0.284240in}}{\pgfqpoint{1.989680in}{1.989680in}}%
\pgfusepath{clip}%
\pgfsetbuttcap%
\pgfsetroundjoin%
\definecolor{currentfill}{rgb}{0.231674,0.318106,0.544834}%
\pgfsetfillcolor{currentfill}%
\pgfsetlinewidth{0.000000pt}%
\definecolor{currentstroke}{rgb}{0.000000,0.000000,0.000000}%
\pgfsetstrokecolor{currentstroke}%
\pgfsetdash{}{0pt}%
\pgfpathmoveto{\pgfqpoint{1.676660in}{1.297381in}}%
\pgfpathlineto{\pgfqpoint{1.679523in}{1.291874in}}%
\pgfpathlineto{\pgfqpoint{1.682385in}{1.286424in}}%
\pgfpathlineto{\pgfqpoint{1.685245in}{1.281036in}}%
\pgfpathlineto{\pgfqpoint{1.688103in}{1.275712in}}%
\pgfpathlineto{\pgfqpoint{1.693553in}{1.270322in}}%
\pgfpathlineto{\pgfqpoint{1.698675in}{1.264842in}}%
\pgfpathlineto{\pgfqpoint{1.703463in}{1.259279in}}%
\pgfpathlineto{\pgfqpoint{1.707913in}{1.253638in}}%
\pgfpathlineto{\pgfqpoint{1.704876in}{1.259168in}}%
\pgfpathlineto{\pgfqpoint{1.701837in}{1.264763in}}%
\pgfpathlineto{\pgfqpoint{1.698797in}{1.270419in}}%
\pgfpathlineto{\pgfqpoint{1.695755in}{1.276133in}}%
\pgfpathlineto{\pgfqpoint{1.691468in}{1.281563in}}%
\pgfpathlineto{\pgfqpoint{1.686853in}{1.286918in}}%
\pgfpathlineto{\pgfqpoint{1.681915in}{1.292192in}}%
\pgfpathlineto{\pgfqpoint{1.676660in}{1.297381in}}%
\pgfpathclose%
\pgfusepath{fill}%
\end{pgfscope}%
\begin{pgfscope}%
\pgfpathrectangle{\pgfqpoint{0.329460in}{0.284240in}}{\pgfqpoint{1.989680in}{1.989680in}}%
\pgfusepath{clip}%
\pgfsetbuttcap%
\pgfsetroundjoin%
\definecolor{currentfill}{rgb}{0.133743,0.548535,0.553541}%
\pgfsetfillcolor{currentfill}%
\pgfsetlinewidth{0.000000pt}%
\definecolor{currentstroke}{rgb}{0.000000,0.000000,0.000000}%
\pgfsetstrokecolor{currentstroke}%
\pgfsetdash{}{0pt}%
\pgfpathmoveto{\pgfqpoint{1.480916in}{1.501070in}}%
\pgfpathlineto{\pgfqpoint{1.482382in}{1.495741in}}%
\pgfpathlineto{\pgfqpoint{1.483845in}{1.490407in}}%
\pgfpathlineto{\pgfqpoint{1.485307in}{1.485069in}}%
\pgfpathlineto{\pgfqpoint{1.486767in}{1.479732in}}%
\pgfpathlineto{\pgfqpoint{1.495066in}{1.477587in}}%
\pgfpathlineto{\pgfqpoint{1.503230in}{1.475315in}}%
\pgfpathlineto{\pgfqpoint{1.511252in}{1.472918in}}%
\pgfpathlineto{\pgfqpoint{1.519124in}{1.470398in}}%
\pgfpathlineto{\pgfqpoint{1.517313in}{1.475844in}}%
\pgfpathlineto{\pgfqpoint{1.515500in}{1.481290in}}%
\pgfpathlineto{\pgfqpoint{1.513684in}{1.486733in}}%
\pgfpathlineto{\pgfqpoint{1.511866in}{1.492171in}}%
\pgfpathlineto{\pgfqpoint{1.504337in}{1.494574in}}%
\pgfpathlineto{\pgfqpoint{1.496664in}{1.496859in}}%
\pgfpathlineto{\pgfqpoint{1.488855in}{1.499025in}}%
\pgfpathlineto{\pgfqpoint{1.480916in}{1.501070in}}%
\pgfpathclose%
\pgfusepath{fill}%
\end{pgfscope}%
\begin{pgfscope}%
\pgfpathrectangle{\pgfqpoint{0.329460in}{0.284240in}}{\pgfqpoint{1.989680in}{1.989680in}}%
\pgfusepath{clip}%
\pgfsetbuttcap%
\pgfsetroundjoin%
\definecolor{currentfill}{rgb}{0.179019,0.433756,0.557430}%
\pgfsetfillcolor{currentfill}%
\pgfsetlinewidth{0.000000pt}%
\definecolor{currentstroke}{rgb}{0.000000,0.000000,0.000000}%
\pgfsetstrokecolor{currentstroke}%
\pgfsetdash{}{0pt}%
\pgfpathmoveto{\pgfqpoint{1.076528in}{1.380741in}}%
\pgfpathlineto{\pgfqpoint{1.073807in}{1.374903in}}%
\pgfpathlineto{\pgfqpoint{1.071089in}{1.369090in}}%
\pgfpathlineto{\pgfqpoint{1.068373in}{1.363305in}}%
\pgfpathlineto{\pgfqpoint{1.065659in}{1.357550in}}%
\pgfpathlineto{\pgfqpoint{1.071681in}{1.361972in}}%
\pgfpathlineto{\pgfqpoint{1.077968in}{1.366294in}}%
\pgfpathlineto{\pgfqpoint{1.084515in}{1.370513in}}%
\pgfpathlineto{\pgfqpoint{1.091314in}{1.374625in}}%
\pgfpathlineto{\pgfqpoint{1.093777in}{1.380202in}}%
\pgfpathlineto{\pgfqpoint{1.096242in}{1.385810in}}%
\pgfpathlineto{\pgfqpoint{1.098710in}{1.391445in}}%
\pgfpathlineto{\pgfqpoint{1.101180in}{1.397106in}}%
\pgfpathlineto{\pgfqpoint{1.094646in}{1.393165in}}%
\pgfpathlineto{\pgfqpoint{1.088355in}{1.389121in}}%
\pgfpathlineto{\pgfqpoint{1.082314in}{1.384978in}}%
\pgfpathlineto{\pgfqpoint{1.076528in}{1.380741in}}%
\pgfpathclose%
\pgfusepath{fill}%
\end{pgfscope}%
\begin{pgfscope}%
\pgfpathrectangle{\pgfqpoint{0.329460in}{0.284240in}}{\pgfqpoint{1.989680in}{1.989680in}}%
\pgfusepath{clip}%
\pgfsetbuttcap%
\pgfsetroundjoin%
\definecolor{currentfill}{rgb}{0.276194,0.190074,0.493001}%
\pgfsetfillcolor{currentfill}%
\pgfsetlinewidth{0.000000pt}%
\definecolor{currentstroke}{rgb}{0.000000,0.000000,0.000000}%
\pgfsetstrokecolor{currentstroke}%
\pgfsetdash{}{0pt}%
\pgfpathmoveto{\pgfqpoint{1.942680in}{1.176609in}}%
\pgfpathlineto{\pgfqpoint{1.946157in}{1.183977in}}%
\pgfpathlineto{\pgfqpoint{1.949648in}{1.191692in}}%
\pgfpathlineto{\pgfqpoint{1.953153in}{1.199761in}}%
\pgfpathlineto{\pgfqpoint{1.956671in}{1.208188in}}%
\pgfpathlineto{\pgfqpoint{1.961413in}{1.198432in}}%
\pgfpathlineto{\pgfqpoint{1.965566in}{1.188590in}}%
\pgfpathlineto{\pgfqpoint{1.969124in}{1.178670in}}%
\pgfpathlineto{\pgfqpoint{1.972079in}{1.168683in}}%
\pgfpathlineto{\pgfqpoint{1.968448in}{1.160434in}}%
\pgfpathlineto{\pgfqpoint{1.964832in}{1.152546in}}%
\pgfpathlineto{\pgfqpoint{1.961230in}{1.145013in}}%
\pgfpathlineto{\pgfqpoint{1.957643in}{1.137829in}}%
\pgfpathlineto{\pgfqpoint{1.954779in}{1.147633in}}%
\pgfpathlineto{\pgfqpoint{1.951326in}{1.157370in}}%
\pgfpathlineto{\pgfqpoint{1.947291in}{1.167032in}}%
\pgfpathlineto{\pgfqpoint{1.942680in}{1.176609in}}%
\pgfpathclose%
\pgfusepath{fill}%
\end{pgfscope}%
\begin{pgfscope}%
\pgfpathrectangle{\pgfqpoint{0.329460in}{0.284240in}}{\pgfqpoint{1.989680in}{1.989680in}}%
\pgfusepath{clip}%
\pgfsetbuttcap%
\pgfsetroundjoin%
\definecolor{currentfill}{rgb}{0.282884,0.135920,0.453427}%
\pgfsetfillcolor{currentfill}%
\pgfsetlinewidth{0.000000pt}%
\definecolor{currentstroke}{rgb}{0.000000,0.000000,0.000000}%
\pgfsetstrokecolor{currentstroke}%
\pgfsetdash{}{0pt}%
\pgfpathmoveto{\pgfqpoint{0.756970in}{1.103879in}}%
\pgfpathlineto{\pgfqpoint{0.753418in}{1.109681in}}%
\pgfpathlineto{\pgfqpoint{0.749853in}{1.115809in}}%
\pgfpathlineto{\pgfqpoint{0.746276in}{1.122269in}}%
\pgfpathlineto{\pgfqpoint{0.742685in}{1.129067in}}%
\pgfpathlineto{\pgfqpoint{0.745021in}{1.138921in}}%
\pgfpathlineto{\pgfqpoint{0.747951in}{1.148718in}}%
\pgfpathlineto{\pgfqpoint{0.751469in}{1.158447in}}%
\pgfpathlineto{\pgfqpoint{0.755568in}{1.168100in}}%
\pgfpathlineto{\pgfqpoint{0.759062in}{1.161114in}}%
\pgfpathlineto{\pgfqpoint{0.762543in}{1.154465in}}%
\pgfpathlineto{\pgfqpoint{0.766011in}{1.148146in}}%
\pgfpathlineto{\pgfqpoint{0.769467in}{1.142152in}}%
\pgfpathlineto{\pgfqpoint{0.765484in}{1.132686in}}%
\pgfpathlineto{\pgfqpoint{0.762070in}{1.123146in}}%
\pgfpathlineto{\pgfqpoint{0.759230in}{1.113541in}}%
\pgfpathlineto{\pgfqpoint{0.756970in}{1.103879in}}%
\pgfpathclose%
\pgfusepath{fill}%
\end{pgfscope}%
\begin{pgfscope}%
\pgfpathrectangle{\pgfqpoint{0.329460in}{0.284240in}}{\pgfqpoint{1.989680in}{1.989680in}}%
\pgfusepath{clip}%
\pgfsetbuttcap%
\pgfsetroundjoin%
\definecolor{currentfill}{rgb}{0.122606,0.585371,0.546557}%
\pgfsetfillcolor{currentfill}%
\pgfsetlinewidth{0.000000pt}%
\definecolor{currentstroke}{rgb}{0.000000,0.000000,0.000000}%
\pgfsetstrokecolor{currentstroke}%
\pgfsetdash{}{0pt}%
\pgfpathmoveto{\pgfqpoint{1.419160in}{1.532544in}}%
\pgfpathlineto{\pgfqpoint{1.419967in}{1.527392in}}%
\pgfpathlineto{\pgfqpoint{1.420773in}{1.522224in}}%
\pgfpathlineto{\pgfqpoint{1.421578in}{1.517044in}}%
\pgfpathlineto{\pgfqpoint{1.422382in}{1.511854in}}%
\pgfpathlineto{\pgfqpoint{1.431007in}{1.510702in}}%
\pgfpathlineto{\pgfqpoint{1.439557in}{1.509418in}}%
\pgfpathlineto{\pgfqpoint{1.448025in}{1.508003in}}%
\pgfpathlineto{\pgfqpoint{1.446931in}{1.513240in}}%
\pgfpathlineto{\pgfqpoint{1.445836in}{1.518467in}}%
\pgfpathlineto{\pgfqpoint{1.444739in}{1.523681in}}%
\pgfpathlineto{\pgfqpoint{1.443641in}{1.528880in}}%
\pgfpathlineto{\pgfqpoint{1.435557in}{1.530226in}}%
\pgfpathlineto{\pgfqpoint{1.427394in}{1.531448in}}%
\pgfpathlineto{\pgfqpoint{1.419160in}{1.532544in}}%
\pgfpathclose%
\pgfusepath{fill}%
\end{pgfscope}%
\begin{pgfscope}%
\pgfpathrectangle{\pgfqpoint{0.329460in}{0.284240in}}{\pgfqpoint{1.989680in}{1.989680in}}%
\pgfusepath{clip}%
\pgfsetbuttcap%
\pgfsetroundjoin%
\definecolor{currentfill}{rgb}{0.122606,0.585371,0.546557}%
\pgfsetfillcolor{currentfill}%
\pgfsetlinewidth{0.000000pt}%
\definecolor{currentstroke}{rgb}{0.000000,0.000000,0.000000}%
\pgfsetstrokecolor{currentstroke}%
\pgfsetdash{}{0pt}%
\pgfpathmoveto{\pgfqpoint{1.251621in}{1.527581in}}%
\pgfpathlineto{\pgfqpoint{1.250438in}{1.522365in}}%
\pgfpathlineto{\pgfqpoint{1.249257in}{1.517134in}}%
\pgfpathlineto{\pgfqpoint{1.248077in}{1.511891in}}%
\pgfpathlineto{\pgfqpoint{1.246898in}{1.506638in}}%
\pgfpathlineto{\pgfqpoint{1.255286in}{1.508167in}}%
\pgfpathlineto{\pgfqpoint{1.263764in}{1.509567in}}%
\pgfpathlineto{\pgfqpoint{1.272323in}{1.510836in}}%
\pgfpathlineto{\pgfqpoint{1.280956in}{1.511974in}}%
\pgfpathlineto{\pgfqpoint{1.281749in}{1.517163in}}%
\pgfpathlineto{\pgfqpoint{1.282543in}{1.522341in}}%
\pgfpathlineto{\pgfqpoint{1.283338in}{1.527507in}}%
\pgfpathlineto{\pgfqpoint{1.284134in}{1.532659in}}%
\pgfpathlineto{\pgfqpoint{1.275893in}{1.531576in}}%
\pgfpathlineto{\pgfqpoint{1.267722in}{1.530368in}}%
\pgfpathlineto{\pgfqpoint{1.259629in}{1.529036in}}%
\pgfpathlineto{\pgfqpoint{1.251621in}{1.527581in}}%
\pgfpathclose%
\pgfusepath{fill}%
\end{pgfscope}%
\begin{pgfscope}%
\pgfpathrectangle{\pgfqpoint{0.329460in}{0.284240in}}{\pgfqpoint{1.989680in}{1.989680in}}%
\pgfusepath{clip}%
\pgfsetbuttcap%
\pgfsetroundjoin%
\definecolor{currentfill}{rgb}{0.263663,0.237631,0.518762}%
\pgfsetfillcolor{currentfill}%
\pgfsetlinewidth{0.000000pt}%
\definecolor{currentstroke}{rgb}{0.000000,0.000000,0.000000}%
\pgfsetstrokecolor{currentstroke}%
\pgfsetdash{}{0pt}%
\pgfpathmoveto{\pgfqpoint{1.720049in}{1.232218in}}%
\pgfpathlineto{\pgfqpoint{1.723081in}{1.227054in}}%
\pgfpathlineto{\pgfqpoint{1.726112in}{1.221973in}}%
\pgfpathlineto{\pgfqpoint{1.729142in}{1.216977in}}%
\pgfpathlineto{\pgfqpoint{1.732171in}{1.212069in}}%
\pgfpathlineto{\pgfqpoint{1.736579in}{1.205925in}}%
\pgfpathlineto{\pgfqpoint{1.740615in}{1.199707in}}%
\pgfpathlineto{\pgfqpoint{1.744274in}{1.193421in}}%
\pgfpathlineto{\pgfqpoint{1.747552in}{1.187072in}}%
\pgfpathlineto{\pgfqpoint{1.744388in}{1.192199in}}%
\pgfpathlineto{\pgfqpoint{1.741223in}{1.197414in}}%
\pgfpathlineto{\pgfqpoint{1.738058in}{1.202714in}}%
\pgfpathlineto{\pgfqpoint{1.734892in}{1.208097in}}%
\pgfpathlineto{\pgfqpoint{1.731732in}{1.214223in}}%
\pgfpathlineto{\pgfqpoint{1.728202in}{1.220288in}}%
\pgfpathlineto{\pgfqpoint{1.724306in}{1.226289in}}%
\pgfpathlineto{\pgfqpoint{1.720049in}{1.232218in}}%
\pgfpathclose%
\pgfusepath{fill}%
\end{pgfscope}%
\begin{pgfscope}%
\pgfpathrectangle{\pgfqpoint{0.329460in}{0.284240in}}{\pgfqpoint{1.989680in}{1.989680in}}%
\pgfusepath{clip}%
\pgfsetbuttcap%
\pgfsetroundjoin%
\definecolor{currentfill}{rgb}{0.133743,0.548535,0.553541}%
\pgfsetfillcolor{currentfill}%
\pgfsetlinewidth{0.000000pt}%
\definecolor{currentstroke}{rgb}{0.000000,0.000000,0.000000}%
\pgfsetstrokecolor{currentstroke}%
\pgfsetdash{}{0pt}%
\pgfpathmoveto{\pgfqpoint{1.183943in}{1.489939in}}%
\pgfpathlineto{\pgfqpoint{1.182050in}{1.484474in}}%
\pgfpathlineto{\pgfqpoint{1.180160in}{1.479003in}}%
\pgfpathlineto{\pgfqpoint{1.178271in}{1.473530in}}%
\pgfpathlineto{\pgfqpoint{1.176386in}{1.468056in}}%
\pgfpathlineto{\pgfqpoint{1.184118in}{1.470684in}}%
\pgfpathlineto{\pgfqpoint{1.192007in}{1.473191in}}%
\pgfpathlineto{\pgfqpoint{1.200046in}{1.475574in}}%
\pgfpathlineto{\pgfqpoint{1.208225in}{1.477832in}}%
\pgfpathlineto{\pgfqpoint{1.209765in}{1.483191in}}%
\pgfpathlineto{\pgfqpoint{1.211307in}{1.488551in}}%
\pgfpathlineto{\pgfqpoint{1.212851in}{1.493907in}}%
\pgfpathlineto{\pgfqpoint{1.214396in}{1.499258in}}%
\pgfpathlineto{\pgfqpoint{1.206572in}{1.497105in}}%
\pgfpathlineto{\pgfqpoint{1.198883in}{1.494833in}}%
\pgfpathlineto{\pgfqpoint{1.191338in}{1.492444in}}%
\pgfpathlineto{\pgfqpoint{1.183943in}{1.489939in}}%
\pgfpathclose%
\pgfusepath{fill}%
\end{pgfscope}%
\begin{pgfscope}%
\pgfpathrectangle{\pgfqpoint{0.329460in}{0.284240in}}{\pgfqpoint{1.989680in}{1.989680in}}%
\pgfusepath{clip}%
\pgfsetbuttcap%
\pgfsetroundjoin%
\definecolor{currentfill}{rgb}{0.195860,0.395433,0.555276}%
\pgfsetfillcolor{currentfill}%
\pgfsetlinewidth{0.000000pt}%
\definecolor{currentstroke}{rgb}{0.000000,0.000000,0.000000}%
\pgfsetstrokecolor{currentstroke}%
\pgfsetdash{}{0pt}%
\pgfpathmoveto{\pgfqpoint{1.631377in}{1.361485in}}%
\pgfpathlineto{\pgfqpoint{1.634036in}{1.355804in}}%
\pgfpathlineto{\pgfqpoint{1.636692in}{1.350159in}}%
\pgfpathlineto{\pgfqpoint{1.639347in}{1.344552in}}%
\pgfpathlineto{\pgfqpoint{1.642000in}{1.338987in}}%
\pgfpathlineto{\pgfqpoint{1.648227in}{1.334369in}}%
\pgfpathlineto{\pgfqpoint{1.654172in}{1.329653in}}%
\pgfpathlineto{\pgfqpoint{1.659827in}{1.324841in}}%
\pgfpathlineto{\pgfqpoint{1.665188in}{1.319939in}}%
\pgfpathlineto{\pgfqpoint{1.662316in}{1.325695in}}%
\pgfpathlineto{\pgfqpoint{1.659441in}{1.331493in}}%
\pgfpathlineto{\pgfqpoint{1.656564in}{1.337330in}}%
\pgfpathlineto{\pgfqpoint{1.653684in}{1.343202in}}%
\pgfpathlineto{\pgfqpoint{1.648528in}{1.347907in}}%
\pgfpathlineto{\pgfqpoint{1.643088in}{1.352525in}}%
\pgfpathlineto{\pgfqpoint{1.637369in}{1.357053in}}%
\pgfpathlineto{\pgfqpoint{1.631377in}{1.361485in}}%
\pgfpathclose%
\pgfusepath{fill}%
\end{pgfscope}%
\begin{pgfscope}%
\pgfpathrectangle{\pgfqpoint{0.329460in}{0.284240in}}{\pgfqpoint{1.989680in}{1.989680in}}%
\pgfusepath{clip}%
\pgfsetbuttcap%
\pgfsetroundjoin%
\definecolor{currentfill}{rgb}{0.282327,0.094955,0.417331}%
\pgfsetfillcolor{currentfill}%
\pgfsetlinewidth{0.000000pt}%
\definecolor{currentstroke}{rgb}{0.000000,0.000000,0.000000}%
\pgfsetstrokecolor{currentstroke}%
\pgfsetdash{}{0pt}%
\pgfpathmoveto{\pgfqpoint{0.905242in}{1.098575in}}%
\pgfpathlineto{\pgfqpoint{0.901970in}{1.094620in}}%
\pgfpathlineto{\pgfqpoint{0.898695in}{1.090801in}}%
\pgfpathlineto{\pgfqpoint{0.895419in}{1.087121in}}%
\pgfpathlineto{\pgfqpoint{0.892141in}{1.083584in}}%
\pgfpathlineto{\pgfqpoint{0.893730in}{1.091074in}}%
\pgfpathlineto{\pgfqpoint{0.895769in}{1.098525in}}%
\pgfpathlineto{\pgfqpoint{0.898256in}{1.105931in}}%
\pgfpathlineto{\pgfqpoint{0.901185in}{1.113284in}}%
\pgfpathlineto{\pgfqpoint{0.904383in}{1.116594in}}%
\pgfpathlineto{\pgfqpoint{0.907581in}{1.120047in}}%
\pgfpathlineto{\pgfqpoint{0.910776in}{1.123638in}}%
\pgfpathlineto{\pgfqpoint{0.913970in}{1.127364in}}%
\pgfpathlineto{\pgfqpoint{0.911139in}{1.120236in}}%
\pgfpathlineto{\pgfqpoint{0.908738in}{1.113057in}}%
\pgfpathlineto{\pgfqpoint{0.906771in}{1.105834in}}%
\pgfpathlineto{\pgfqpoint{0.905242in}{1.098575in}}%
\pgfpathclose%
\pgfusepath{fill}%
\end{pgfscope}%
\begin{pgfscope}%
\pgfpathrectangle{\pgfqpoint{0.329460in}{0.284240in}}{\pgfqpoint{1.989680in}{1.989680in}}%
\pgfusepath{clip}%
\pgfsetbuttcap%
\pgfsetroundjoin%
\definecolor{currentfill}{rgb}{0.283072,0.130895,0.449241}%
\pgfsetfillcolor{currentfill}%
\pgfsetlinewidth{0.000000pt}%
\definecolor{currentstroke}{rgb}{0.000000,0.000000,0.000000}%
\pgfsetstrokecolor{currentstroke}%
\pgfsetdash{}{0pt}%
\pgfpathmoveto{\pgfqpoint{1.772862in}{1.149641in}}%
\pgfpathlineto{\pgfqpoint{1.776028in}{1.145458in}}%
\pgfpathlineto{\pgfqpoint{1.779195in}{1.141397in}}%
\pgfpathlineto{\pgfqpoint{1.782362in}{1.137461in}}%
\pgfpathlineto{\pgfqpoint{1.785530in}{1.133652in}}%
\pgfpathlineto{\pgfqpoint{1.788741in}{1.126575in}}%
\pgfpathlineto{\pgfqpoint{1.791525in}{1.119440in}}%
\pgfpathlineto{\pgfqpoint{1.793877in}{1.112256in}}%
\pgfpathlineto{\pgfqpoint{1.795796in}{1.105029in}}%
\pgfpathlineto{\pgfqpoint{1.792538in}{1.109064in}}%
\pgfpathlineto{\pgfqpoint{1.789281in}{1.113227in}}%
\pgfpathlineto{\pgfqpoint{1.786026in}{1.117516in}}%
\pgfpathlineto{\pgfqpoint{1.782772in}{1.121926in}}%
\pgfpathlineto{\pgfqpoint{1.780924in}{1.128923in}}%
\pgfpathlineto{\pgfqpoint{1.778654in}{1.135879in}}%
\pgfpathlineto{\pgfqpoint{1.775966in}{1.142787in}}%
\pgfpathlineto{\pgfqpoint{1.772862in}{1.149641in}}%
\pgfpathclose%
\pgfusepath{fill}%
\end{pgfscope}%
\begin{pgfscope}%
\pgfpathrectangle{\pgfqpoint{0.329460in}{0.284240in}}{\pgfqpoint{1.989680in}{1.989680in}}%
\pgfusepath{clip}%
\pgfsetbuttcap%
\pgfsetroundjoin%
\definecolor{currentfill}{rgb}{0.231674,0.318106,0.544834}%
\pgfsetfillcolor{currentfill}%
\pgfsetlinewidth{0.000000pt}%
\definecolor{currentstroke}{rgb}{0.000000,0.000000,0.000000}%
\pgfsetstrokecolor{currentstroke}%
\pgfsetdash{}{0pt}%
\pgfpathmoveto{\pgfqpoint{1.003088in}{1.271248in}}%
\pgfpathlineto{\pgfqpoint{1.000013in}{1.265485in}}%
\pgfpathlineto{\pgfqpoint{0.996939in}{1.259782in}}%
\pgfpathlineto{\pgfqpoint{0.993867in}{1.254139in}}%
\pgfpathlineto{\pgfqpoint{0.990796in}{1.248561in}}%
\pgfpathlineto{\pgfqpoint{0.994940in}{1.254268in}}%
\pgfpathlineto{\pgfqpoint{0.999427in}{1.259901in}}%
\pgfpathlineto{\pgfqpoint{1.004253in}{1.265455in}}%
\pgfpathlineto{\pgfqpoint{1.009412in}{1.270925in}}%
\pgfpathlineto{\pgfqpoint{1.012314in}{1.276294in}}%
\pgfpathlineto{\pgfqpoint{1.015217in}{1.281726in}}%
\pgfpathlineto{\pgfqpoint{1.018122in}{1.287220in}}%
\pgfpathlineto{\pgfqpoint{1.021029in}{1.292773in}}%
\pgfpathlineto{\pgfqpoint{1.016055in}{1.287508in}}%
\pgfpathlineto{\pgfqpoint{1.011404in}{1.282162in}}%
\pgfpathlineto{\pgfqpoint{1.007080in}{1.276740in}}%
\pgfpathlineto{\pgfqpoint{1.003088in}{1.271248in}}%
\pgfpathclose%
\pgfusepath{fill}%
\end{pgfscope}%
\begin{pgfscope}%
\pgfpathrectangle{\pgfqpoint{0.329460in}{0.284240in}}{\pgfqpoint{1.989680in}{1.989680in}}%
\pgfusepath{clip}%
\pgfsetbuttcap%
\pgfsetroundjoin%
\definecolor{currentfill}{rgb}{0.201239,0.383670,0.554294}%
\pgfsetfillcolor{currentfill}%
\pgfsetlinewidth{0.000000pt}%
\definecolor{currentstroke}{rgb}{0.000000,0.000000,0.000000}%
\pgfsetstrokecolor{currentstroke}%
\pgfsetdash{}{0pt}%
\pgfpathmoveto{\pgfqpoint{0.712514in}{1.280307in}}%
\pgfpathlineto{\pgfqpoint{0.708819in}{1.292212in}}%
\pgfpathlineto{\pgfqpoint{0.705106in}{1.304540in}}%
\pgfpathlineto{\pgfqpoint{0.701374in}{1.317300in}}%
\pgfpathlineto{\pgfqpoint{0.697622in}{1.330497in}}%
\pgfpathlineto{\pgfqpoint{0.702847in}{1.340725in}}%
\pgfpathlineto{\pgfqpoint{0.708692in}{1.350850in}}%
\pgfpathlineto{\pgfqpoint{0.715149in}{1.360862in}}%
\pgfpathlineto{\pgfqpoint{0.722209in}{1.370752in}}%
\pgfpathlineto{\pgfqpoint{0.725796in}{1.357410in}}%
\pgfpathlineto{\pgfqpoint{0.729365in}{1.344504in}}%
\pgfpathlineto{\pgfqpoint{0.732916in}{1.332026in}}%
\pgfpathlineto{\pgfqpoint{0.736450in}{1.319970in}}%
\pgfpathlineto{\pgfqpoint{0.729572in}{1.310224in}}%
\pgfpathlineto{\pgfqpoint{0.723284in}{1.300359in}}%
\pgfpathlineto{\pgfqpoint{0.717595in}{1.290384in}}%
\pgfpathlineto{\pgfqpoint{0.712514in}{1.280307in}}%
\pgfpathclose%
\pgfusepath{fill}%
\end{pgfscope}%
\begin{pgfscope}%
\pgfpathrectangle{\pgfqpoint{0.329460in}{0.284240in}}{\pgfqpoint{1.989680in}{1.989680in}}%
\pgfusepath{clip}%
\pgfsetbuttcap%
\pgfsetroundjoin%
\definecolor{currentfill}{rgb}{0.195860,0.395433,0.555276}%
\pgfsetfillcolor{currentfill}%
\pgfsetlinewidth{0.000000pt}%
\definecolor{currentstroke}{rgb}{0.000000,0.000000,0.000000}%
\pgfsetstrokecolor{currentstroke}%
\pgfsetdash{}{0pt}%
\pgfpathmoveto{\pgfqpoint{1.044351in}{1.338950in}}%
\pgfpathlineto{\pgfqpoint{1.041428in}{1.333034in}}%
\pgfpathlineto{\pgfqpoint{1.038508in}{1.327153in}}%
\pgfpathlineto{\pgfqpoint{1.035590in}{1.321310in}}%
\pgfpathlineto{\pgfqpoint{1.032674in}{1.315509in}}%
\pgfpathlineto{\pgfqpoint{1.037768in}{1.320488in}}%
\pgfpathlineto{\pgfqpoint{1.043162in}{1.325380in}}%
\pgfpathlineto{\pgfqpoint{1.048850in}{1.330182in}}%
\pgfpathlineto{\pgfqpoint{1.054826in}{1.334888in}}%
\pgfpathlineto{\pgfqpoint{1.057531in}{1.340494in}}%
\pgfpathlineto{\pgfqpoint{1.060238in}{1.346142in}}%
\pgfpathlineto{\pgfqpoint{1.062948in}{1.351828in}}%
\pgfpathlineto{\pgfqpoint{1.065659in}{1.357550in}}%
\pgfpathlineto{\pgfqpoint{1.059909in}{1.353033in}}%
\pgfpathlineto{\pgfqpoint{1.054437in}{1.348424in}}%
\pgfpathlineto{\pgfqpoint{1.049249in}{1.343729in}}%
\pgfpathlineto{\pgfqpoint{1.044351in}{1.338950in}}%
\pgfpathclose%
\pgfusepath{fill}%
\end{pgfscope}%
\begin{pgfscope}%
\pgfpathrectangle{\pgfqpoint{0.329460in}{0.284240in}}{\pgfqpoint{1.989680in}{1.989680in}}%
\pgfusepath{clip}%
\pgfsetbuttcap%
\pgfsetroundjoin%
\definecolor{currentfill}{rgb}{0.122606,0.585371,0.546557}%
\pgfsetfillcolor{currentfill}%
\pgfsetlinewidth{0.000000pt}%
\definecolor{currentstroke}{rgb}{0.000000,0.000000,0.000000}%
\pgfsetstrokecolor{currentstroke}%
\pgfsetdash{}{0pt}%
\pgfpathmoveto{\pgfqpoint{1.443641in}{1.528880in}}%
\pgfpathlineto{\pgfqpoint{1.444739in}{1.523681in}}%
\pgfpathlineto{\pgfqpoint{1.445836in}{1.518467in}}%
\pgfpathlineto{\pgfqpoint{1.446931in}{1.513240in}}%
\pgfpathlineto{\pgfqpoint{1.448025in}{1.508003in}}%
\pgfpathlineto{\pgfqpoint{1.456403in}{1.506460in}}%
\pgfpathlineto{\pgfqpoint{1.464683in}{1.504789in}}%
\pgfpathlineto{\pgfqpoint{1.472856in}{1.502992in}}%
\pgfpathlineto{\pgfqpoint{1.480916in}{1.501070in}}%
\pgfpathlineto{\pgfqpoint{1.479449in}{1.506391in}}%
\pgfpathlineto{\pgfqpoint{1.477980in}{1.511702in}}%
\pgfpathlineto{\pgfqpoint{1.476509in}{1.517001in}}%
\pgfpathlineto{\pgfqpoint{1.475036in}{1.522284in}}%
\pgfpathlineto{\pgfqpoint{1.467343in}{1.524113in}}%
\pgfpathlineto{\pgfqpoint{1.459541in}{1.525822in}}%
\pgfpathlineto{\pgfqpoint{1.451638in}{1.527412in}}%
\pgfpathlineto{\pgfqpoint{1.443641in}{1.528880in}}%
\pgfpathclose%
\pgfusepath{fill}%
\end{pgfscope}%
\begin{pgfscope}%
\pgfpathrectangle{\pgfqpoint{0.329460in}{0.284240in}}{\pgfqpoint{1.989680in}{1.989680in}}%
\pgfusepath{clip}%
\pgfsetbuttcap%
\pgfsetroundjoin%
\definecolor{currentfill}{rgb}{0.263663,0.237631,0.518762}%
\pgfsetfillcolor{currentfill}%
\pgfsetlinewidth{0.000000pt}%
\definecolor{currentstroke}{rgb}{0.000000,0.000000,0.000000}%
\pgfsetstrokecolor{currentstroke}%
\pgfsetdash{}{0pt}%
\pgfpathmoveto{\pgfqpoint{0.964988in}{1.202608in}}%
\pgfpathlineto{\pgfqpoint{0.961799in}{1.197174in}}%
\pgfpathlineto{\pgfqpoint{0.958610in}{1.191824in}}%
\pgfpathlineto{\pgfqpoint{0.955422in}{1.186559in}}%
\pgfpathlineto{\pgfqpoint{0.952235in}{1.181382in}}%
\pgfpathlineto{\pgfqpoint{0.955169in}{1.187781in}}%
\pgfpathlineto{\pgfqpoint{0.958488in}{1.194122in}}%
\pgfpathlineto{\pgfqpoint{0.962190in}{1.200401in}}%
\pgfpathlineto{\pgfqpoint{0.966267in}{1.206611in}}%
\pgfpathlineto{\pgfqpoint{0.969331in}{1.211566in}}%
\pgfpathlineto{\pgfqpoint{0.972395in}{1.216610in}}%
\pgfpathlineto{\pgfqpoint{0.975459in}{1.221739in}}%
\pgfpathlineto{\pgfqpoint{0.978524in}{1.226951in}}%
\pgfpathlineto{\pgfqpoint{0.974588in}{1.220958in}}%
\pgfpathlineto{\pgfqpoint{0.971017in}{1.214900in}}%
\pgfpathlineto{\pgfqpoint{0.967816in}{1.208781in}}%
\pgfpathlineto{\pgfqpoint{0.964988in}{1.202608in}}%
\pgfpathclose%
\pgfusepath{fill}%
\end{pgfscope}%
\begin{pgfscope}%
\pgfpathrectangle{\pgfqpoint{0.329460in}{0.284240in}}{\pgfqpoint{1.989680in}{1.989680in}}%
\pgfusepath{clip}%
\pgfsetbuttcap%
\pgfsetroundjoin%
\definecolor{currentfill}{rgb}{0.122606,0.585371,0.546557}%
\pgfsetfillcolor{currentfill}%
\pgfsetlinewidth{0.000000pt}%
\definecolor{currentstroke}{rgb}{0.000000,0.000000,0.000000}%
\pgfsetstrokecolor{currentstroke}%
\pgfsetdash{}{0pt}%
\pgfpathmoveto{\pgfqpoint{1.220599in}{1.520561in}}%
\pgfpathlineto{\pgfqpoint{1.219045in}{1.515255in}}%
\pgfpathlineto{\pgfqpoint{1.217493in}{1.509935in}}%
\pgfpathlineto{\pgfqpoint{1.215944in}{1.504601in}}%
\pgfpathlineto{\pgfqpoint{1.214396in}{1.499258in}}%
\pgfpathlineto{\pgfqpoint{1.222349in}{1.501289in}}%
\pgfpathlineto{\pgfqpoint{1.230421in}{1.503198in}}%
\pgfpathlineto{\pgfqpoint{1.238607in}{1.504981in}}%
\pgfpathlineto{\pgfqpoint{1.246898in}{1.506638in}}%
\pgfpathlineto{\pgfqpoint{1.248077in}{1.511891in}}%
\pgfpathlineto{\pgfqpoint{1.249257in}{1.517134in}}%
\pgfpathlineto{\pgfqpoint{1.250438in}{1.522365in}}%
\pgfpathlineto{\pgfqpoint{1.251621in}{1.527581in}}%
\pgfpathlineto{\pgfqpoint{1.243707in}{1.526005in}}%
\pgfpathlineto{\pgfqpoint{1.235894in}{1.524308in}}%
\pgfpathlineto{\pgfqpoint{1.228188in}{1.522493in}}%
\pgfpathlineto{\pgfqpoint{1.220599in}{1.520561in}}%
\pgfpathclose%
\pgfusepath{fill}%
\end{pgfscope}%
\begin{pgfscope}%
\pgfpathrectangle{\pgfqpoint{0.329460in}{0.284240in}}{\pgfqpoint{1.989680in}{1.989680in}}%
\pgfusepath{clip}%
\pgfsetbuttcap%
\pgfsetroundjoin%
\definecolor{currentfill}{rgb}{0.147607,0.511733,0.557049}%
\pgfsetfillcolor{currentfill}%
\pgfsetlinewidth{0.000000pt}%
\definecolor{currentstroke}{rgb}{0.000000,0.000000,0.000000}%
\pgfsetstrokecolor{currentstroke}%
\pgfsetdash{}{0pt}%
\pgfpathmoveto{\pgfqpoint{1.548964in}{1.459132in}}%
\pgfpathlineto{\pgfqpoint{1.551099in}{1.453556in}}%
\pgfpathlineto{\pgfqpoint{1.553231in}{1.447984in}}%
\pgfpathlineto{\pgfqpoint{1.555360in}{1.442419in}}%
\pgfpathlineto{\pgfqpoint{1.557488in}{1.436863in}}%
\pgfpathlineto{\pgfqpoint{1.564806in}{1.433617in}}%
\pgfpathlineto{\pgfqpoint{1.571923in}{1.430258in}}%
\pgfpathlineto{\pgfqpoint{1.578832in}{1.426788in}}%
\pgfpathlineto{\pgfqpoint{1.585526in}{1.423211in}}%
\pgfpathlineto{\pgfqpoint{1.583105in}{1.428920in}}%
\pgfpathlineto{\pgfqpoint{1.580680in}{1.434639in}}%
\pgfpathlineto{\pgfqpoint{1.578253in}{1.440364in}}%
\pgfpathlineto{\pgfqpoint{1.575824in}{1.446094in}}%
\pgfpathlineto{\pgfqpoint{1.569413in}{1.449510in}}%
\pgfpathlineto{\pgfqpoint{1.562794in}{1.452823in}}%
\pgfpathlineto{\pgfqpoint{1.555976in}{1.456032in}}%
\pgfpathlineto{\pgfqpoint{1.548964in}{1.459132in}}%
\pgfpathclose%
\pgfusepath{fill}%
\end{pgfscope}%
\begin{pgfscope}%
\pgfpathrectangle{\pgfqpoint{0.329460in}{0.284240in}}{\pgfqpoint{1.989680in}{1.989680in}}%
\pgfusepath{clip}%
\pgfsetbuttcap%
\pgfsetroundjoin%
\definecolor{currentfill}{rgb}{0.283072,0.130895,0.449241}%
\pgfsetfillcolor{currentfill}%
\pgfsetlinewidth{0.000000pt}%
\definecolor{currentstroke}{rgb}{0.000000,0.000000,0.000000}%
\pgfsetstrokecolor{currentstroke}%
\pgfsetdash{}{0pt}%
\pgfpathmoveto{\pgfqpoint{0.918318in}{1.115677in}}%
\pgfpathlineto{\pgfqpoint{0.915051in}{1.111216in}}%
\pgfpathlineto{\pgfqpoint{0.911783in}{1.106876in}}%
\pgfpathlineto{\pgfqpoint{0.908513in}{1.102661in}}%
\pgfpathlineto{\pgfqpoint{0.905242in}{1.098575in}}%
\pgfpathlineto{\pgfqpoint{0.906771in}{1.105834in}}%
\pgfpathlineto{\pgfqpoint{0.908738in}{1.113057in}}%
\pgfpathlineto{\pgfqpoint{0.911139in}{1.120236in}}%
\pgfpathlineto{\pgfqpoint{0.913970in}{1.127364in}}%
\pgfpathlineto{\pgfqpoint{0.917163in}{1.131222in}}%
\pgfpathlineto{\pgfqpoint{0.920354in}{1.135208in}}%
\pgfpathlineto{\pgfqpoint{0.923544in}{1.139319in}}%
\pgfpathlineto{\pgfqpoint{0.926734in}{1.143551in}}%
\pgfpathlineto{\pgfqpoint{0.923999in}{1.136649in}}%
\pgfpathlineto{\pgfqpoint{0.921683in}{1.129698in}}%
\pgfpathlineto{\pgfqpoint{0.919788in}{1.122705in}}%
\pgfpathlineto{\pgfqpoint{0.918318in}{1.115677in}}%
\pgfpathclose%
\pgfusepath{fill}%
\end{pgfscope}%
\begin{pgfscope}%
\pgfpathrectangle{\pgfqpoint{0.329460in}{0.284240in}}{\pgfqpoint{1.989680in}{1.989680in}}%
\pgfusepath{clip}%
\pgfsetbuttcap%
\pgfsetroundjoin%
\definecolor{currentfill}{rgb}{0.267004,0.004874,0.329415}%
\pgfsetfillcolor{currentfill}%
\pgfsetlinewidth{0.000000pt}%
\definecolor{currentstroke}{rgb}{0.000000,0.000000,0.000000}%
\pgfsetstrokecolor{currentstroke}%
\pgfsetdash{}{0pt}%
\pgfpathmoveto{\pgfqpoint{1.861504in}{1.056850in}}%
\pgfpathlineto{\pgfqpoint{1.864831in}{1.056358in}}%
\pgfpathlineto{\pgfqpoint{1.868164in}{1.056078in}}%
\pgfpathlineto{\pgfqpoint{1.871503in}{1.056014in}}%
\pgfpathlineto{\pgfqpoint{1.874848in}{1.056170in}}%
\pgfpathlineto{\pgfqpoint{1.876698in}{1.047549in}}%
\pgfpathlineto{\pgfqpoint{1.878029in}{1.038890in}}%
\pgfpathlineto{\pgfqpoint{1.878838in}{1.030202in}}%
\pgfpathlineto{\pgfqpoint{1.879122in}{1.021493in}}%
\pgfpathlineto{\pgfqpoint{1.875732in}{1.021559in}}%
\pgfpathlineto{\pgfqpoint{1.872348in}{1.021845in}}%
\pgfpathlineto{\pgfqpoint{1.868970in}{1.022349in}}%
\pgfpathlineto{\pgfqpoint{1.865598in}{1.023065in}}%
\pgfpathlineto{\pgfqpoint{1.865338in}{1.031549in}}%
\pgfpathlineto{\pgfqpoint{1.864568in}{1.040013in}}%
\pgfpathlineto{\pgfqpoint{1.863288in}{1.048450in}}%
\pgfpathlineto{\pgfqpoint{1.861504in}{1.056850in}}%
\pgfpathclose%
\pgfusepath{fill}%
\end{pgfscope}%
\begin{pgfscope}%
\pgfpathrectangle{\pgfqpoint{0.329460in}{0.284240in}}{\pgfqpoint{1.989680in}{1.989680in}}%
\pgfusepath{clip}%
\pgfsetbuttcap%
\pgfsetroundjoin%
\definecolor{currentfill}{rgb}{0.267004,0.004874,0.329415}%
\pgfsetfillcolor{currentfill}%
\pgfsetlinewidth{0.000000pt}%
\definecolor{currentstroke}{rgb}{0.000000,0.000000,0.000000}%
\pgfsetstrokecolor{currentstroke}%
\pgfsetdash{}{0pt}%
\pgfpathmoveto{\pgfqpoint{1.874848in}{1.056170in}}%
\pgfpathlineto{\pgfqpoint{1.878200in}{1.056550in}}%
\pgfpathlineto{\pgfqpoint{1.881558in}{1.057160in}}%
\pgfpathlineto{\pgfqpoint{1.884922in}{1.058002in}}%
\pgfpathlineto{\pgfqpoint{1.888294in}{1.059082in}}%
\pgfpathlineto{\pgfqpoint{1.890209in}{1.050243in}}%
\pgfpathlineto{\pgfqpoint{1.891593in}{1.041365in}}%
\pgfpathlineto{\pgfqpoint{1.892441in}{1.032456in}}%
\pgfpathlineto{\pgfqpoint{1.892752in}{1.023526in}}%
\pgfpathlineto{\pgfqpoint{1.889333in}{1.022664in}}%
\pgfpathlineto{\pgfqpoint{1.885923in}{1.022041in}}%
\pgfpathlineto{\pgfqpoint{1.882519in}{1.021652in}}%
\pgfpathlineto{\pgfqpoint{1.879122in}{1.021493in}}%
\pgfpathlineto{\pgfqpoint{1.878838in}{1.030202in}}%
\pgfpathlineto{\pgfqpoint{1.878029in}{1.038890in}}%
\pgfpathlineto{\pgfqpoint{1.876698in}{1.047549in}}%
\pgfpathlineto{\pgfqpoint{1.874848in}{1.056170in}}%
\pgfpathclose%
\pgfusepath{fill}%
\end{pgfscope}%
\begin{pgfscope}%
\pgfpathrectangle{\pgfqpoint{0.329460in}{0.284240in}}{\pgfqpoint{1.989680in}{1.989680in}}%
\pgfusepath{clip}%
\pgfsetbuttcap%
\pgfsetroundjoin%
\definecolor{currentfill}{rgb}{0.276194,0.190074,0.493001}%
\pgfsetfillcolor{currentfill}%
\pgfsetlinewidth{0.000000pt}%
\definecolor{currentstroke}{rgb}{0.000000,0.000000,0.000000}%
\pgfsetstrokecolor{currentstroke}%
\pgfsetdash{}{0pt}%
\pgfpathmoveto{\pgfqpoint{0.742685in}{1.129067in}}%
\pgfpathlineto{\pgfqpoint{0.739080in}{1.136209in}}%
\pgfpathlineto{\pgfqpoint{0.735462in}{1.143700in}}%
\pgfpathlineto{\pgfqpoint{0.731828in}{1.151547in}}%
\pgfpathlineto{\pgfqpoint{0.728180in}{1.159755in}}%
\pgfpathlineto{\pgfqpoint{0.730595in}{1.169796in}}%
\pgfpathlineto{\pgfqpoint{0.733617in}{1.179776in}}%
\pgfpathlineto{\pgfqpoint{0.737241in}{1.189688in}}%
\pgfpathlineto{\pgfqpoint{0.741460in}{1.199521in}}%
\pgfpathlineto{\pgfqpoint{0.745008in}{1.191132in}}%
\pgfpathlineto{\pgfqpoint{0.748542in}{1.183103in}}%
\pgfpathlineto{\pgfqpoint{0.752062in}{1.175428in}}%
\pgfpathlineto{\pgfqpoint{0.755568in}{1.168100in}}%
\pgfpathlineto{\pgfqpoint{0.751469in}{1.158447in}}%
\pgfpathlineto{\pgfqpoint{0.747951in}{1.148718in}}%
\pgfpathlineto{\pgfqpoint{0.745021in}{1.138921in}}%
\pgfpathlineto{\pgfqpoint{0.742685in}{1.129067in}}%
\pgfpathclose%
\pgfusepath{fill}%
\end{pgfscope}%
\begin{pgfscope}%
\pgfpathrectangle{\pgfqpoint{0.329460in}{0.284240in}}{\pgfqpoint{1.989680in}{1.989680in}}%
\pgfusepath{clip}%
\pgfsetbuttcap%
\pgfsetroundjoin%
\definecolor{currentfill}{rgb}{0.163625,0.471133,0.558148}%
\pgfsetfillcolor{currentfill}%
\pgfsetlinewidth{0.000000pt}%
\definecolor{currentstroke}{rgb}{0.000000,0.000000,0.000000}%
\pgfsetstrokecolor{currentstroke}%
\pgfsetdash{}{0pt}%
\pgfpathmoveto{\pgfqpoint{1.585526in}{1.423211in}}%
\pgfpathlineto{\pgfqpoint{1.587945in}{1.417514in}}%
\pgfpathlineto{\pgfqpoint{1.590362in}{1.411832in}}%
\pgfpathlineto{\pgfqpoint{1.592776in}{1.406166in}}%
\pgfpathlineto{\pgfqpoint{1.595188in}{1.400520in}}%
\pgfpathlineto{\pgfqpoint{1.601933in}{1.396673in}}%
\pgfpathlineto{\pgfqpoint{1.608440in}{1.392720in}}%
\pgfpathlineto{\pgfqpoint{1.614704in}{1.388665in}}%
\pgfpathlineto{\pgfqpoint{1.620717in}{1.384512in}}%
\pgfpathlineto{\pgfqpoint{1.618046in}{1.390331in}}%
\pgfpathlineto{\pgfqpoint{1.615373in}{1.396170in}}%
\pgfpathlineto{\pgfqpoint{1.612697in}{1.402026in}}%
\pgfpathlineto{\pgfqpoint{1.610018in}{1.407896in}}%
\pgfpathlineto{\pgfqpoint{1.604250in}{1.411869in}}%
\pgfpathlineto{\pgfqpoint{1.598242in}{1.415748in}}%
\pgfpathlineto{\pgfqpoint{1.591998in}{1.419530in}}%
\pgfpathlineto{\pgfqpoint{1.585526in}{1.423211in}}%
\pgfpathclose%
\pgfusepath{fill}%
\end{pgfscope}%
\begin{pgfscope}%
\pgfpathrectangle{\pgfqpoint{0.329460in}{0.284240in}}{\pgfqpoint{1.989680in}{1.989680in}}%
\pgfusepath{clip}%
\pgfsetbuttcap%
\pgfsetroundjoin%
\definecolor{currentfill}{rgb}{0.133743,0.548535,0.553541}%
\pgfsetfillcolor{currentfill}%
\pgfsetlinewidth{0.000000pt}%
\definecolor{currentstroke}{rgb}{0.000000,0.000000,0.000000}%
\pgfsetstrokecolor{currentstroke}%
\pgfsetdash{}{0pt}%
\pgfpathmoveto{\pgfqpoint{1.511866in}{1.492171in}}%
\pgfpathlineto{\pgfqpoint{1.513684in}{1.486733in}}%
\pgfpathlineto{\pgfqpoint{1.515500in}{1.481290in}}%
\pgfpathlineto{\pgfqpoint{1.517313in}{1.475844in}}%
\pgfpathlineto{\pgfqpoint{1.519124in}{1.470398in}}%
\pgfpathlineto{\pgfqpoint{1.526839in}{1.467757in}}%
\pgfpathlineto{\pgfqpoint{1.534389in}{1.464997in}}%
\pgfpathlineto{\pgfqpoint{1.541766in}{1.462122in}}%
\pgfpathlineto{\pgfqpoint{1.548964in}{1.459132in}}%
\pgfpathlineto{\pgfqpoint{1.546828in}{1.464710in}}%
\pgfpathlineto{\pgfqpoint{1.544688in}{1.470288in}}%
\pgfpathlineto{\pgfqpoint{1.542546in}{1.475863in}}%
\pgfpathlineto{\pgfqpoint{1.540402in}{1.481432in}}%
\pgfpathlineto{\pgfqpoint{1.533519in}{1.484281in}}%
\pgfpathlineto{\pgfqpoint{1.526465in}{1.487022in}}%
\pgfpathlineto{\pgfqpoint{1.519245in}{1.489653in}}%
\pgfpathlineto{\pgfqpoint{1.511866in}{1.492171in}}%
\pgfpathclose%
\pgfusepath{fill}%
\end{pgfscope}%
\begin{pgfscope}%
\pgfpathrectangle{\pgfqpoint{0.329460in}{0.284240in}}{\pgfqpoint{1.989680in}{1.989680in}}%
\pgfusepath{clip}%
\pgfsetbuttcap%
\pgfsetroundjoin%
\definecolor{currentfill}{rgb}{0.212395,0.359683,0.551710}%
\pgfsetfillcolor{currentfill}%
\pgfsetlinewidth{0.000000pt}%
\definecolor{currentstroke}{rgb}{0.000000,0.000000,0.000000}%
\pgfsetstrokecolor{currentstroke}%
\pgfsetdash{}{0pt}%
\pgfpathmoveto{\pgfqpoint{1.665188in}{1.319939in}}%
\pgfpathlineto{\pgfqpoint{1.668059in}{1.314226in}}%
\pgfpathlineto{\pgfqpoint{1.670928in}{1.308561in}}%
\pgfpathlineto{\pgfqpoint{1.673795in}{1.302945in}}%
\pgfpathlineto{\pgfqpoint{1.676660in}{1.297381in}}%
\pgfpathlineto{\pgfqpoint{1.681915in}{1.292192in}}%
\pgfpathlineto{\pgfqpoint{1.686853in}{1.286918in}}%
\pgfpathlineto{\pgfqpoint{1.691468in}{1.281563in}}%
\pgfpathlineto{\pgfqpoint{1.695755in}{1.276133in}}%
\pgfpathlineto{\pgfqpoint{1.692712in}{1.281903in}}%
\pgfpathlineto{\pgfqpoint{1.689667in}{1.287725in}}%
\pgfpathlineto{\pgfqpoint{1.686620in}{1.293597in}}%
\pgfpathlineto{\pgfqpoint{1.683571in}{1.299516in}}%
\pgfpathlineto{\pgfqpoint{1.679445in}{1.304735in}}%
\pgfpathlineto{\pgfqpoint{1.675003in}{1.309881in}}%
\pgfpathlineto{\pgfqpoint{1.670249in}{1.314951in}}%
\pgfpathlineto{\pgfqpoint{1.665188in}{1.319939in}}%
\pgfpathclose%
\pgfusepath{fill}%
\end{pgfscope}%
\begin{pgfscope}%
\pgfpathrectangle{\pgfqpoint{0.329460in}{0.284240in}}{\pgfqpoint{1.989680in}{1.989680in}}%
\pgfusepath{clip}%
\pgfsetbuttcap%
\pgfsetroundjoin%
\definecolor{currentfill}{rgb}{0.280255,0.165693,0.476498}%
\pgfsetfillcolor{currentfill}%
\pgfsetlinewidth{0.000000pt}%
\definecolor{currentstroke}{rgb}{0.000000,0.000000,0.000000}%
\pgfsetstrokecolor{currentstroke}%
\pgfsetdash{}{0pt}%
\pgfpathmoveto{\pgfqpoint{1.760206in}{1.167518in}}%
\pgfpathlineto{\pgfqpoint{1.763369in}{1.162884in}}%
\pgfpathlineto{\pgfqpoint{1.766533in}{1.158357in}}%
\pgfpathlineto{\pgfqpoint{1.769698in}{1.153942in}}%
\pgfpathlineto{\pgfqpoint{1.772862in}{1.149641in}}%
\pgfpathlineto{\pgfqpoint{1.775966in}{1.142787in}}%
\pgfpathlineto{\pgfqpoint{1.778654in}{1.135879in}}%
\pgfpathlineto{\pgfqpoint{1.780924in}{1.128923in}}%
\pgfpathlineto{\pgfqpoint{1.782772in}{1.121926in}}%
\pgfpathlineto{\pgfqpoint{1.779519in}{1.126454in}}%
\pgfpathlineto{\pgfqpoint{1.776266in}{1.131097in}}%
\pgfpathlineto{\pgfqpoint{1.773014in}{1.135851in}}%
\pgfpathlineto{\pgfqpoint{1.769763in}{1.140714in}}%
\pgfpathlineto{\pgfqpoint{1.767985in}{1.147481in}}%
\pgfpathlineto{\pgfqpoint{1.765796in}{1.154208in}}%
\pgfpathlineto{\pgfqpoint{1.763202in}{1.160889in}}%
\pgfpathlineto{\pgfqpoint{1.760206in}{1.167518in}}%
\pgfpathclose%
\pgfusepath{fill}%
\end{pgfscope}%
\begin{pgfscope}%
\pgfpathrectangle{\pgfqpoint{0.329460in}{0.284240in}}{\pgfqpoint{1.989680in}{1.989680in}}%
\pgfusepath{clip}%
\pgfsetbuttcap%
\pgfsetroundjoin%
\definecolor{currentfill}{rgb}{0.172719,0.448791,0.557885}%
\pgfsetfillcolor{currentfill}%
\pgfsetlinewidth{0.000000pt}%
\definecolor{currentstroke}{rgb}{0.000000,0.000000,0.000000}%
\pgfsetstrokecolor{currentstroke}%
\pgfsetdash{}{0pt}%
\pgfpathmoveto{\pgfqpoint{1.973392in}{1.379433in}}%
\pgfpathlineto{\pgfqpoint{1.976953in}{1.393247in}}%
\pgfpathlineto{\pgfqpoint{1.980534in}{1.407510in}}%
\pgfpathlineto{\pgfqpoint{1.984134in}{1.422230in}}%
\pgfpathlineto{\pgfqpoint{1.991871in}{1.412356in}}%
\pgfpathlineto{\pgfqpoint{1.999004in}{1.402351in}}%
\pgfpathlineto{\pgfqpoint{2.005523in}{1.392223in}}%
\pgfpathlineto{\pgfqpoint{2.011418in}{1.381981in}}%
\pgfpathlineto{\pgfqpoint{2.007638in}{1.367395in}}%
\pgfpathlineto{\pgfqpoint{2.003879in}{1.353268in}}%
\pgfpathlineto{\pgfqpoint{2.000140in}{1.339594in}}%
\pgfpathlineto{\pgfqpoint{1.994363in}{1.349730in}}%
\pgfpathlineto{\pgfqpoint{1.987973in}{1.359755in}}%
\pgfpathlineto{\pgfqpoint{1.980980in}{1.369659in}}%
\pgfpathlineto{\pgfqpoint{1.973392in}{1.379433in}}%
\pgfpathclose%
\pgfusepath{fill}%
\end{pgfscope}%
\begin{pgfscope}%
\pgfpathrectangle{\pgfqpoint{0.329460in}{0.284240in}}{\pgfqpoint{1.989680in}{1.989680in}}%
\pgfusepath{clip}%
\pgfsetbuttcap%
\pgfsetroundjoin%
\definecolor{currentfill}{rgb}{0.268510,0.009605,0.335427}%
\pgfsetfillcolor{currentfill}%
\pgfsetlinewidth{0.000000pt}%
\definecolor{currentstroke}{rgb}{0.000000,0.000000,0.000000}%
\pgfsetstrokecolor{currentstroke}%
\pgfsetdash{}{0pt}%
\pgfpathmoveto{\pgfqpoint{1.848244in}{1.060852in}}%
\pgfpathlineto{\pgfqpoint{1.851552in}{1.059554in}}%
\pgfpathlineto{\pgfqpoint{1.854864in}{1.058452in}}%
\pgfpathlineto{\pgfqpoint{1.858181in}{1.057549in}}%
\pgfpathlineto{\pgfqpoint{1.861504in}{1.056850in}}%
\pgfpathlineto{\pgfqpoint{1.863288in}{1.048450in}}%
\pgfpathlineto{\pgfqpoint{1.864568in}{1.040013in}}%
\pgfpathlineto{\pgfqpoint{1.865338in}{1.031549in}}%
\pgfpathlineto{\pgfqpoint{1.865598in}{1.023065in}}%
\pgfpathlineto{\pgfqpoint{1.862232in}{1.023989in}}%
\pgfpathlineto{\pgfqpoint{1.858871in}{1.025117in}}%
\pgfpathlineto{\pgfqpoint{1.855515in}{1.026445in}}%
\pgfpathlineto{\pgfqpoint{1.852164in}{1.027970in}}%
\pgfpathlineto{\pgfqpoint{1.851929in}{1.036226in}}%
\pgfpathlineto{\pgfqpoint{1.851195in}{1.044464in}}%
\pgfpathlineto{\pgfqpoint{1.849966in}{1.052675in}}%
\pgfpathlineto{\pgfqpoint{1.848244in}{1.060852in}}%
\pgfpathclose%
\pgfusepath{fill}%
\end{pgfscope}%
\begin{pgfscope}%
\pgfpathrectangle{\pgfqpoint{0.329460in}{0.284240in}}{\pgfqpoint{1.989680in}{1.989680in}}%
\pgfusepath{clip}%
\pgfsetbuttcap%
\pgfsetroundjoin%
\definecolor{currentfill}{rgb}{0.268510,0.009605,0.335427}%
\pgfsetfillcolor{currentfill}%
\pgfsetlinewidth{0.000000pt}%
\definecolor{currentstroke}{rgb}{0.000000,0.000000,0.000000}%
\pgfsetstrokecolor{currentstroke}%
\pgfsetdash{}{0pt}%
\pgfpathmoveto{\pgfqpoint{1.888294in}{1.059082in}}%
\pgfpathlineto{\pgfqpoint{1.891673in}{1.060404in}}%
\pgfpathlineto{\pgfqpoint{1.895059in}{1.061973in}}%
\pgfpathlineto{\pgfqpoint{1.898454in}{1.063792in}}%
\pgfpathlineto{\pgfqpoint{1.901856in}{1.065867in}}%
\pgfpathlineto{\pgfqpoint{1.903839in}{1.056815in}}%
\pgfpathlineto{\pgfqpoint{1.905277in}{1.047721in}}%
\pgfpathlineto{\pgfqpoint{1.906166in}{1.038595in}}%
\pgfpathlineto{\pgfqpoint{1.906504in}{1.029447in}}%
\pgfpathlineto{\pgfqpoint{1.903053in}{1.027586in}}%
\pgfpathlineto{\pgfqpoint{1.899611in}{1.025982in}}%
\pgfpathlineto{\pgfqpoint{1.896177in}{1.024630in}}%
\pgfpathlineto{\pgfqpoint{1.892752in}{1.023526in}}%
\pgfpathlineto{\pgfqpoint{1.892441in}{1.032456in}}%
\pgfpathlineto{\pgfqpoint{1.891593in}{1.041365in}}%
\pgfpathlineto{\pgfqpoint{1.890209in}{1.050243in}}%
\pgfpathlineto{\pgfqpoint{1.888294in}{1.059082in}}%
\pgfpathclose%
\pgfusepath{fill}%
\end{pgfscope}%
\begin{pgfscope}%
\pgfpathrectangle{\pgfqpoint{0.329460in}{0.284240in}}{\pgfqpoint{1.989680in}{1.989680in}}%
\pgfusepath{clip}%
\pgfsetbuttcap%
\pgfsetroundjoin%
\definecolor{currentfill}{rgb}{0.120081,0.622161,0.534946}%
\pgfsetfillcolor{currentfill}%
\pgfsetlinewidth{0.000000pt}%
\definecolor{currentstroke}{rgb}{0.000000,0.000000,0.000000}%
\pgfsetstrokecolor{currentstroke}%
\pgfsetdash{}{0pt}%
\pgfpathmoveto{\pgfqpoint{1.319245in}{1.555977in}}%
\pgfpathlineto{\pgfqpoint{1.318845in}{1.550947in}}%
\pgfpathlineto{\pgfqpoint{1.318445in}{1.545893in}}%
\pgfpathlineto{\pgfqpoint{1.318046in}{1.540817in}}%
\pgfpathlineto{\pgfqpoint{1.317648in}{1.535722in}}%
\pgfpathlineto{\pgfqpoint{1.326124in}{1.536167in}}%
\pgfpathlineto{\pgfqpoint{1.334624in}{1.536484in}}%
\pgfpathlineto{\pgfqpoint{1.343139in}{1.536671in}}%
\pgfpathlineto{\pgfqpoint{1.351661in}{1.536728in}}%
\pgfpathlineto{\pgfqpoint{1.351655in}{1.541811in}}%
\pgfpathlineto{\pgfqpoint{1.351650in}{1.546874in}}%
\pgfpathlineto{\pgfqpoint{1.351644in}{1.551915in}}%
\pgfpathlineto{\pgfqpoint{1.351639in}{1.556932in}}%
\pgfpathlineto{\pgfqpoint{1.343522in}{1.556878in}}%
\pgfpathlineto{\pgfqpoint{1.335412in}{1.556700in}}%
\pgfpathlineto{\pgfqpoint{1.327317in}{1.556400in}}%
\pgfpathlineto{\pgfqpoint{1.319245in}{1.555977in}}%
\pgfpathclose%
\pgfusepath{fill}%
\end{pgfscope}%
\begin{pgfscope}%
\pgfpathrectangle{\pgfqpoint{0.329460in}{0.284240in}}{\pgfqpoint{1.989680in}{1.989680in}}%
\pgfusepath{clip}%
\pgfsetbuttcap%
\pgfsetroundjoin%
\definecolor{currentfill}{rgb}{0.120081,0.622161,0.534946}%
\pgfsetfillcolor{currentfill}%
\pgfsetlinewidth{0.000000pt}%
\definecolor{currentstroke}{rgb}{0.000000,0.000000,0.000000}%
\pgfsetstrokecolor{currentstroke}%
\pgfsetdash{}{0pt}%
\pgfpathmoveto{\pgfqpoint{1.351639in}{1.556932in}}%
\pgfpathlineto{\pgfqpoint{1.351644in}{1.551915in}}%
\pgfpathlineto{\pgfqpoint{1.351650in}{1.546874in}}%
\pgfpathlineto{\pgfqpoint{1.351655in}{1.541811in}}%
\pgfpathlineto{\pgfqpoint{1.351661in}{1.536728in}}%
\pgfpathlineto{\pgfqpoint{1.360183in}{1.536656in}}%
\pgfpathlineto{\pgfqpoint{1.368697in}{1.536455in}}%
\pgfpathlineto{\pgfqpoint{1.377194in}{1.536124in}}%
\pgfpathlineto{\pgfqpoint{1.385668in}{1.535664in}}%
\pgfpathlineto{\pgfqpoint{1.385258in}{1.540760in}}%
\pgfpathlineto{\pgfqpoint{1.384848in}{1.545837in}}%
\pgfpathlineto{\pgfqpoint{1.384437in}{1.550891in}}%
\pgfpathlineto{\pgfqpoint{1.384025in}{1.555922in}}%
\pgfpathlineto{\pgfqpoint{1.375956in}{1.556359in}}%
\pgfpathlineto{\pgfqpoint{1.367863in}{1.556673in}}%
\pgfpathlineto{\pgfqpoint{1.359755in}{1.556864in}}%
\pgfpathlineto{\pgfqpoint{1.351639in}{1.556932in}}%
\pgfpathclose%
\pgfusepath{fill}%
\end{pgfscope}%
\begin{pgfscope}%
\pgfpathrectangle{\pgfqpoint{0.329460in}{0.284240in}}{\pgfqpoint{1.989680in}{1.989680in}}%
\pgfusepath{clip}%
\pgfsetbuttcap%
\pgfsetroundjoin%
\definecolor{currentfill}{rgb}{0.260571,0.246922,0.522828}%
\pgfsetfillcolor{currentfill}%
\pgfsetlinewidth{0.000000pt}%
\definecolor{currentstroke}{rgb}{0.000000,0.000000,0.000000}%
\pgfsetstrokecolor{currentstroke}%
\pgfsetdash{}{0pt}%
\pgfpathmoveto{\pgfqpoint{1.956671in}{1.208188in}}%
\pgfpathlineto{\pgfqpoint{1.960205in}{1.216980in}}%
\pgfpathlineto{\pgfqpoint{1.963752in}{1.226144in}}%
\pgfpathlineto{\pgfqpoint{1.967316in}{1.235684in}}%
\pgfpathlineto{\pgfqpoint{1.970894in}{1.245609in}}%
\pgfpathlineto{\pgfqpoint{1.975771in}{1.235683in}}%
\pgfpathlineto{\pgfqpoint{1.980044in}{1.225668in}}%
\pgfpathlineto{\pgfqpoint{1.983709in}{1.215574in}}%
\pgfpathlineto{\pgfqpoint{1.986757in}{1.205409in}}%
\pgfpathlineto{\pgfqpoint{1.983063in}{1.195656in}}%
\pgfpathlineto{\pgfqpoint{1.979386in}{1.186287in}}%
\pgfpathlineto{\pgfqpoint{1.975724in}{1.177298in}}%
\pgfpathlineto{\pgfqpoint{1.972079in}{1.168683in}}%
\pgfpathlineto{\pgfqpoint{1.969124in}{1.178670in}}%
\pgfpathlineto{\pgfqpoint{1.965566in}{1.188590in}}%
\pgfpathlineto{\pgfqpoint{1.961413in}{1.198432in}}%
\pgfpathlineto{\pgfqpoint{1.956671in}{1.208188in}}%
\pgfpathclose%
\pgfusepath{fill}%
\end{pgfscope}%
\begin{pgfscope}%
\pgfpathrectangle{\pgfqpoint{0.329460in}{0.284240in}}{\pgfqpoint{1.989680in}{1.989680in}}%
\pgfusepath{clip}%
\pgfsetbuttcap%
\pgfsetroundjoin%
\definecolor{currentfill}{rgb}{0.248629,0.278775,0.534556}%
\pgfsetfillcolor{currentfill}%
\pgfsetlinewidth{0.000000pt}%
\definecolor{currentstroke}{rgb}{0.000000,0.000000,0.000000}%
\pgfsetstrokecolor{currentstroke}%
\pgfsetdash{}{0pt}%
\pgfpathmoveto{\pgfqpoint{1.707913in}{1.253638in}}%
\pgfpathlineto{\pgfqpoint{1.710949in}{1.248174in}}%
\pgfpathlineto{\pgfqpoint{1.713983in}{1.242781in}}%
\pgfpathlineto{\pgfqpoint{1.717017in}{1.237462in}}%
\pgfpathlineto{\pgfqpoint{1.720049in}{1.232218in}}%
\pgfpathlineto{\pgfqpoint{1.724306in}{1.226289in}}%
\pgfpathlineto{\pgfqpoint{1.728202in}{1.220288in}}%
\pgfpathlineto{\pgfqpoint{1.731732in}{1.214223in}}%
\pgfpathlineto{\pgfqpoint{1.734892in}{1.208097in}}%
\pgfpathlineto{\pgfqpoint{1.731726in}{1.213560in}}%
\pgfpathlineto{\pgfqpoint{1.728558in}{1.219099in}}%
\pgfpathlineto{\pgfqpoint{1.725390in}{1.224711in}}%
\pgfpathlineto{\pgfqpoint{1.722220in}{1.230393in}}%
\pgfpathlineto{\pgfqpoint{1.719176in}{1.236295in}}%
\pgfpathlineto{\pgfqpoint{1.715774in}{1.242140in}}%
\pgfpathlineto{\pgfqpoint{1.712018in}{1.247923in}}%
\pgfpathlineto{\pgfqpoint{1.707913in}{1.253638in}}%
\pgfpathclose%
\pgfusepath{fill}%
\end{pgfscope}%
\begin{pgfscope}%
\pgfpathrectangle{\pgfqpoint{0.329460in}{0.284240in}}{\pgfqpoint{1.989680in}{1.989680in}}%
\pgfusepath{clip}%
\pgfsetbuttcap%
\pgfsetroundjoin%
\definecolor{currentfill}{rgb}{0.147607,0.511733,0.557049}%
\pgfsetfillcolor{currentfill}%
\pgfsetlinewidth{0.000000pt}%
\definecolor{currentstroke}{rgb}{0.000000,0.000000,0.000000}%
\pgfsetstrokecolor{currentstroke}%
\pgfsetdash{}{0pt}%
\pgfpathmoveto{\pgfqpoint{1.121031in}{1.442974in}}%
\pgfpathlineto{\pgfqpoint{1.118540in}{1.437207in}}%
\pgfpathlineto{\pgfqpoint{1.116053in}{1.431445in}}%
\pgfpathlineto{\pgfqpoint{1.113567in}{1.425690in}}%
\pgfpathlineto{\pgfqpoint{1.111085in}{1.419944in}}%
\pgfpathlineto{\pgfqpoint{1.117582in}{1.423614in}}%
\pgfpathlineto{\pgfqpoint{1.124300in}{1.427179in}}%
\pgfpathlineto{\pgfqpoint{1.131232in}{1.430637in}}%
\pgfpathlineto{\pgfqpoint{1.138372in}{1.433984in}}%
\pgfpathlineto{\pgfqpoint{1.140568in}{1.439572in}}%
\pgfpathlineto{\pgfqpoint{1.142766in}{1.445169in}}%
\pgfpathlineto{\pgfqpoint{1.144966in}{1.450773in}}%
\pgfpathlineto{\pgfqpoint{1.147169in}{1.456382in}}%
\pgfpathlineto{\pgfqpoint{1.140329in}{1.453185in}}%
\pgfpathlineto{\pgfqpoint{1.133688in}{1.449883in}}%
\pgfpathlineto{\pgfqpoint{1.127253in}{1.446478in}}%
\pgfpathlineto{\pgfqpoint{1.121031in}{1.442974in}}%
\pgfpathclose%
\pgfusepath{fill}%
\end{pgfscope}%
\begin{pgfscope}%
\pgfpathrectangle{\pgfqpoint{0.329460in}{0.284240in}}{\pgfqpoint{1.989680in}{1.989680in}}%
\pgfusepath{clip}%
\pgfsetbuttcap%
\pgfsetroundjoin%
\definecolor{currentfill}{rgb}{0.271305,0.019942,0.347269}%
\pgfsetfillcolor{currentfill}%
\pgfsetlinewidth{0.000000pt}%
\definecolor{currentstroke}{rgb}{0.000000,0.000000,0.000000}%
\pgfsetstrokecolor{currentstroke}%
\pgfsetdash{}{0pt}%
\pgfpathmoveto{\pgfqpoint{1.835057in}{1.067917in}}%
\pgfpathlineto{\pgfqpoint{1.838348in}{1.065878in}}%
\pgfpathlineto{\pgfqpoint{1.841642in}{1.064018in}}%
\pgfpathlineto{\pgfqpoint{1.844941in}{1.062341in}}%
\pgfpathlineto{\pgfqpoint{1.848244in}{1.060852in}}%
\pgfpathlineto{\pgfqpoint{1.849966in}{1.052675in}}%
\pgfpathlineto{\pgfqpoint{1.851195in}{1.044464in}}%
\pgfpathlineto{\pgfqpoint{1.851929in}{1.036226in}}%
\pgfpathlineto{\pgfqpoint{1.852164in}{1.027970in}}%
\pgfpathlineto{\pgfqpoint{1.848818in}{1.029686in}}%
\pgfpathlineto{\pgfqpoint{1.845477in}{1.031590in}}%
\pgfpathlineto{\pgfqpoint{1.842140in}{1.033678in}}%
\pgfpathlineto{\pgfqpoint{1.838806in}{1.035947in}}%
\pgfpathlineto{\pgfqpoint{1.838594in}{1.043973in}}%
\pgfpathlineto{\pgfqpoint{1.837896in}{1.051983in}}%
\pgfpathlineto{\pgfqpoint{1.836716in}{1.059966in}}%
\pgfpathlineto{\pgfqpoint{1.835057in}{1.067917in}}%
\pgfpathclose%
\pgfusepath{fill}%
\end{pgfscope}%
\begin{pgfscope}%
\pgfpathrectangle{\pgfqpoint{0.329460in}{0.284240in}}{\pgfqpoint{1.989680in}{1.989680in}}%
\pgfusepath{clip}%
\pgfsetbuttcap%
\pgfsetroundjoin%
\definecolor{currentfill}{rgb}{0.133743,0.548535,0.553541}%
\pgfsetfillcolor{currentfill}%
\pgfsetlinewidth{0.000000pt}%
\definecolor{currentstroke}{rgb}{0.000000,0.000000,0.000000}%
\pgfsetstrokecolor{currentstroke}%
\pgfsetdash{}{0pt}%
\pgfpathmoveto{\pgfqpoint{1.156005in}{1.478810in}}%
\pgfpathlineto{\pgfqpoint{1.153792in}{1.473209in}}%
\pgfpathlineto{\pgfqpoint{1.151582in}{1.467602in}}%
\pgfpathlineto{\pgfqpoint{1.149374in}{1.461992in}}%
\pgfpathlineto{\pgfqpoint{1.147169in}{1.456382in}}%
\pgfpathlineto{\pgfqpoint{1.154201in}{1.459470in}}%
\pgfpathlineto{\pgfqpoint{1.161420in}{1.462447in}}%
\pgfpathlineto{\pgfqpoint{1.168817in}{1.465310in}}%
\pgfpathlineto{\pgfqpoint{1.176386in}{1.468056in}}%
\pgfpathlineto{\pgfqpoint{1.178271in}{1.473530in}}%
\pgfpathlineto{\pgfqpoint{1.180160in}{1.479003in}}%
\pgfpathlineto{\pgfqpoint{1.182050in}{1.484474in}}%
\pgfpathlineto{\pgfqpoint{1.183943in}{1.489939in}}%
\pgfpathlineto{\pgfqpoint{1.176705in}{1.487320in}}%
\pgfpathlineto{\pgfqpoint{1.169631in}{1.484591in}}%
\pgfpathlineto{\pgfqpoint{1.162729in}{1.481754in}}%
\pgfpathlineto{\pgfqpoint{1.156005in}{1.478810in}}%
\pgfpathclose%
\pgfusepath{fill}%
\end{pgfscope}%
\begin{pgfscope}%
\pgfpathrectangle{\pgfqpoint{0.329460in}{0.284240in}}{\pgfqpoint{1.989680in}{1.989680in}}%
\pgfusepath{clip}%
\pgfsetbuttcap%
\pgfsetroundjoin%
\definecolor{currentfill}{rgb}{0.120081,0.622161,0.534946}%
\pgfsetfillcolor{currentfill}%
\pgfsetlinewidth{0.000000pt}%
\definecolor{currentstroke}{rgb}{0.000000,0.000000,0.000000}%
\pgfsetstrokecolor{currentstroke}%
\pgfsetdash{}{0pt}%
\pgfpathmoveto{\pgfqpoint{1.287329in}{1.553069in}}%
\pgfpathlineto{\pgfqpoint{1.286529in}{1.548000in}}%
\pgfpathlineto{\pgfqpoint{1.285729in}{1.542907in}}%
\pgfpathlineto{\pgfqpoint{1.284931in}{1.537793in}}%
\pgfpathlineto{\pgfqpoint{1.284134in}{1.532659in}}%
\pgfpathlineto{\pgfqpoint{1.292438in}{1.533615in}}%
\pgfpathlineto{\pgfqpoint{1.300797in}{1.534445in}}%
\pgfpathlineto{\pgfqpoint{1.309203in}{1.535148in}}%
\pgfpathlineto{\pgfqpoint{1.317648in}{1.535722in}}%
\pgfpathlineto{\pgfqpoint{1.318046in}{1.540817in}}%
\pgfpathlineto{\pgfqpoint{1.318445in}{1.545893in}}%
\pgfpathlineto{\pgfqpoint{1.318845in}{1.550947in}}%
\pgfpathlineto{\pgfqpoint{1.319245in}{1.555977in}}%
\pgfpathlineto{\pgfqpoint{1.311202in}{1.555432in}}%
\pgfpathlineto{\pgfqpoint{1.303197in}{1.554765in}}%
\pgfpathlineto{\pgfqpoint{1.295237in}{1.553978in}}%
\pgfpathlineto{\pgfqpoint{1.287329in}{1.553069in}}%
\pgfpathclose%
\pgfusepath{fill}%
\end{pgfscope}%
\begin{pgfscope}%
\pgfpathrectangle{\pgfqpoint{0.329460in}{0.284240in}}{\pgfqpoint{1.989680in}{1.989680in}}%
\pgfusepath{clip}%
\pgfsetbuttcap%
\pgfsetroundjoin%
\definecolor{currentfill}{rgb}{0.120081,0.622161,0.534946}%
\pgfsetfillcolor{currentfill}%
\pgfsetlinewidth{0.000000pt}%
\definecolor{currentstroke}{rgb}{0.000000,0.000000,0.000000}%
\pgfsetstrokecolor{currentstroke}%
\pgfsetdash{}{0pt}%
\pgfpathmoveto{\pgfqpoint{1.384025in}{1.555922in}}%
\pgfpathlineto{\pgfqpoint{1.384437in}{1.550891in}}%
\pgfpathlineto{\pgfqpoint{1.384848in}{1.545837in}}%
\pgfpathlineto{\pgfqpoint{1.385258in}{1.540760in}}%
\pgfpathlineto{\pgfqpoint{1.385668in}{1.535664in}}%
\pgfpathlineto{\pgfqpoint{1.394109in}{1.535076in}}%
\pgfpathlineto{\pgfqpoint{1.402510in}{1.534359in}}%
\pgfpathlineto{\pgfqpoint{1.410863in}{1.533515in}}%
\pgfpathlineto{\pgfqpoint{1.419160in}{1.532544in}}%
\pgfpathlineto{\pgfqpoint{1.418352in}{1.537680in}}%
\pgfpathlineto{\pgfqpoint{1.417543in}{1.542796in}}%
\pgfpathlineto{\pgfqpoint{1.416733in}{1.547891in}}%
\pgfpathlineto{\pgfqpoint{1.415921in}{1.552961in}}%
\pgfpathlineto{\pgfqpoint{1.408020in}{1.553883in}}%
\pgfpathlineto{\pgfqpoint{1.400065in}{1.554684in}}%
\pgfpathlineto{\pgfqpoint{1.392064in}{1.555364in}}%
\pgfpathlineto{\pgfqpoint{1.384025in}{1.555922in}}%
\pgfpathclose%
\pgfusepath{fill}%
\end{pgfscope}%
\begin{pgfscope}%
\pgfpathrectangle{\pgfqpoint{0.329460in}{0.284240in}}{\pgfqpoint{1.989680in}{1.989680in}}%
\pgfusepath{clip}%
\pgfsetbuttcap%
\pgfsetroundjoin%
\definecolor{currentfill}{rgb}{0.272594,0.025563,0.353093}%
\pgfsetfillcolor{currentfill}%
\pgfsetlinewidth{0.000000pt}%
\definecolor{currentstroke}{rgb}{0.000000,0.000000,0.000000}%
\pgfsetstrokecolor{currentstroke}%
\pgfsetdash{}{0pt}%
\pgfpathmoveto{\pgfqpoint{1.901856in}{1.065867in}}%
\pgfpathlineto{\pgfqpoint{1.905267in}{1.068203in}}%
\pgfpathlineto{\pgfqpoint{1.908686in}{1.070804in}}%
\pgfpathlineto{\pgfqpoint{1.912115in}{1.073675in}}%
\pgfpathlineto{\pgfqpoint{1.915553in}{1.076820in}}%
\pgfpathlineto{\pgfqpoint{1.917604in}{1.067557in}}%
\pgfpathlineto{\pgfqpoint{1.919098in}{1.058252in}}%
\pgfpathlineto{\pgfqpoint{1.920030in}{1.048914in}}%
\pgfpathlineto{\pgfqpoint{1.920396in}{1.039551in}}%
\pgfpathlineto{\pgfqpoint{1.916909in}{1.036616in}}%
\pgfpathlineto{\pgfqpoint{1.913431in}{1.033957in}}%
\pgfpathlineto{\pgfqpoint{1.909963in}{1.031569in}}%
\pgfpathlineto{\pgfqpoint{1.906504in}{1.029447in}}%
\pgfpathlineto{\pgfqpoint{1.906166in}{1.038595in}}%
\pgfpathlineto{\pgfqpoint{1.905277in}{1.047721in}}%
\pgfpathlineto{\pgfqpoint{1.903839in}{1.056815in}}%
\pgfpathlineto{\pgfqpoint{1.901856in}{1.065867in}}%
\pgfpathclose%
\pgfusepath{fill}%
\end{pgfscope}%
\begin{pgfscope}%
\pgfpathrectangle{\pgfqpoint{0.329460in}{0.284240in}}{\pgfqpoint{1.989680in}{1.989680in}}%
\pgfusepath{clip}%
\pgfsetbuttcap%
\pgfsetroundjoin%
\definecolor{currentfill}{rgb}{0.163625,0.471133,0.558148}%
\pgfsetfillcolor{currentfill}%
\pgfsetlinewidth{0.000000pt}%
\definecolor{currentstroke}{rgb}{0.000000,0.000000,0.000000}%
\pgfsetstrokecolor{currentstroke}%
\pgfsetdash{}{0pt}%
\pgfpathmoveto{\pgfqpoint{1.087438in}{1.404288in}}%
\pgfpathlineto{\pgfqpoint{1.084706in}{1.398377in}}%
\pgfpathlineto{\pgfqpoint{1.081978in}{1.392480in}}%
\pgfpathlineto{\pgfqpoint{1.079252in}{1.386601in}}%
\pgfpathlineto{\pgfqpoint{1.076528in}{1.380741in}}%
\pgfpathlineto{\pgfqpoint{1.082314in}{1.384978in}}%
\pgfpathlineto{\pgfqpoint{1.088355in}{1.389121in}}%
\pgfpathlineto{\pgfqpoint{1.094646in}{1.393165in}}%
\pgfpathlineto{\pgfqpoint{1.101180in}{1.397106in}}%
\pgfpathlineto{\pgfqpoint{1.103653in}{1.402789in}}%
\pgfpathlineto{\pgfqpoint{1.106128in}{1.408491in}}%
\pgfpathlineto{\pgfqpoint{1.108605in}{1.414210in}}%
\pgfpathlineto{\pgfqpoint{1.111085in}{1.419944in}}%
\pgfpathlineto{\pgfqpoint{1.104816in}{1.416173in}}%
\pgfpathlineto{\pgfqpoint{1.098781in}{1.412305in}}%
\pgfpathlineto{\pgfqpoint{1.092986in}{1.408342in}}%
\pgfpathlineto{\pgfqpoint{1.087438in}{1.404288in}}%
\pgfpathclose%
\pgfusepath{fill}%
\end{pgfscope}%
\begin{pgfscope}%
\pgfpathrectangle{\pgfqpoint{0.329460in}{0.284240in}}{\pgfqpoint{1.989680in}{1.989680in}}%
\pgfusepath{clip}%
\pgfsetbuttcap%
\pgfsetroundjoin%
\definecolor{currentfill}{rgb}{0.179019,0.433756,0.557430}%
\pgfsetfillcolor{currentfill}%
\pgfsetlinewidth{0.000000pt}%
\definecolor{currentstroke}{rgb}{0.000000,0.000000,0.000000}%
\pgfsetstrokecolor{currentstroke}%
\pgfsetdash{}{0pt}%
\pgfpathmoveto{\pgfqpoint{1.620717in}{1.384512in}}%
\pgfpathlineto{\pgfqpoint{1.623386in}{1.378715in}}%
\pgfpathlineto{\pgfqpoint{1.626052in}{1.372943in}}%
\pgfpathlineto{\pgfqpoint{1.628715in}{1.367199in}}%
\pgfpathlineto{\pgfqpoint{1.631377in}{1.361485in}}%
\pgfpathlineto{\pgfqpoint{1.637369in}{1.357053in}}%
\pgfpathlineto{\pgfqpoint{1.643088in}{1.352525in}}%
\pgfpathlineto{\pgfqpoint{1.648528in}{1.347907in}}%
\pgfpathlineto{\pgfqpoint{1.653684in}{1.343202in}}%
\pgfpathlineto{\pgfqpoint{1.650803in}{1.349107in}}%
\pgfpathlineto{\pgfqpoint{1.647919in}{1.355042in}}%
\pgfpathlineto{\pgfqpoint{1.645032in}{1.361004in}}%
\pgfpathlineto{\pgfqpoint{1.642143in}{1.366992in}}%
\pgfpathlineto{\pgfqpoint{1.637192in}{1.371500in}}%
\pgfpathlineto{\pgfqpoint{1.631967in}{1.375926in}}%
\pgfpathlineto{\pgfqpoint{1.626474in}{1.380264in}}%
\pgfpathlineto{\pgfqpoint{1.620717in}{1.384512in}}%
\pgfpathclose%
\pgfusepath{fill}%
\end{pgfscope}%
\begin{pgfscope}%
\pgfpathrectangle{\pgfqpoint{0.329460in}{0.284240in}}{\pgfqpoint{1.989680in}{1.989680in}}%
\pgfusepath{clip}%
\pgfsetbuttcap%
\pgfsetroundjoin%
\definecolor{currentfill}{rgb}{0.122606,0.585371,0.546557}%
\pgfsetfillcolor{currentfill}%
\pgfsetlinewidth{0.000000pt}%
\definecolor{currentstroke}{rgb}{0.000000,0.000000,0.000000}%
\pgfsetstrokecolor{currentstroke}%
\pgfsetdash{}{0pt}%
\pgfpathmoveto{\pgfqpoint{1.475036in}{1.522284in}}%
\pgfpathlineto{\pgfqpoint{1.476509in}{1.517001in}}%
\pgfpathlineto{\pgfqpoint{1.477980in}{1.511702in}}%
\pgfpathlineto{\pgfqpoint{1.479449in}{1.506391in}}%
\pgfpathlineto{\pgfqpoint{1.480916in}{1.501070in}}%
\pgfpathlineto{\pgfqpoint{1.488855in}{1.499025in}}%
\pgfpathlineto{\pgfqpoint{1.496664in}{1.496859in}}%
\pgfpathlineto{\pgfqpoint{1.504337in}{1.494574in}}%
\pgfpathlineto{\pgfqpoint{1.511866in}{1.492171in}}%
\pgfpathlineto{\pgfqpoint{1.510046in}{1.497601in}}%
\pgfpathlineto{\pgfqpoint{1.508224in}{1.503020in}}%
\pgfpathlineto{\pgfqpoint{1.506400in}{1.508427in}}%
\pgfpathlineto{\pgfqpoint{1.504573in}{1.513819in}}%
\pgfpathlineto{\pgfqpoint{1.497388in}{1.516105in}}%
\pgfpathlineto{\pgfqpoint{1.490066in}{1.518279in}}%
\pgfpathlineto{\pgfqpoint{1.482612in}{1.520339in}}%
\pgfpathlineto{\pgfqpoint{1.475036in}{1.522284in}}%
\pgfpathclose%
\pgfusepath{fill}%
\end{pgfscope}%
\begin{pgfscope}%
\pgfpathrectangle{\pgfqpoint{0.329460in}{0.284240in}}{\pgfqpoint{1.989680in}{1.989680in}}%
\pgfusepath{clip}%
\pgfsetbuttcap%
\pgfsetroundjoin%
\definecolor{currentfill}{rgb}{0.212395,0.359683,0.551710}%
\pgfsetfillcolor{currentfill}%
\pgfsetlinewidth{0.000000pt}%
\definecolor{currentstroke}{rgb}{0.000000,0.000000,0.000000}%
\pgfsetstrokecolor{currentstroke}%
\pgfsetdash{}{0pt}%
\pgfpathmoveto{\pgfqpoint{1.015407in}{1.294821in}}%
\pgfpathlineto{\pgfqpoint{1.012324in}{1.288855in}}%
\pgfpathlineto{\pgfqpoint{1.009244in}{1.282935in}}%
\pgfpathlineto{\pgfqpoint{1.006165in}{1.277065in}}%
\pgfpathlineto{\pgfqpoint{1.003088in}{1.271248in}}%
\pgfpathlineto{\pgfqpoint{1.007080in}{1.276740in}}%
\pgfpathlineto{\pgfqpoint{1.011404in}{1.282162in}}%
\pgfpathlineto{\pgfqpoint{1.016055in}{1.287508in}}%
\pgfpathlineto{\pgfqpoint{1.021029in}{1.292773in}}%
\pgfpathlineto{\pgfqpoint{1.023937in}{1.298381in}}%
\pgfpathlineto{\pgfqpoint{1.026847in}{1.304042in}}%
\pgfpathlineto{\pgfqpoint{1.029759in}{1.309752in}}%
\pgfpathlineto{\pgfqpoint{1.032674in}{1.315509in}}%
\pgfpathlineto{\pgfqpoint{1.027885in}{1.310448in}}%
\pgfpathlineto{\pgfqpoint{1.023408in}{1.305310in}}%
\pgfpathlineto{\pgfqpoint{1.019247in}{1.300099in}}%
\pgfpathlineto{\pgfqpoint{1.015407in}{1.294821in}}%
\pgfpathclose%
\pgfusepath{fill}%
\end{pgfscope}%
\begin{pgfscope}%
\pgfpathrectangle{\pgfqpoint{0.329460in}{0.284240in}}{\pgfqpoint{1.989680in}{1.989680in}}%
\pgfusepath{clip}%
\pgfsetbuttcap%
\pgfsetroundjoin%
\definecolor{currentfill}{rgb}{0.274952,0.037752,0.364543}%
\pgfsetfillcolor{currentfill}%
\pgfsetlinewidth{0.000000pt}%
\definecolor{currentstroke}{rgb}{0.000000,0.000000,0.000000}%
\pgfsetstrokecolor{currentstroke}%
\pgfsetdash{}{0pt}%
\pgfpathmoveto{\pgfqpoint{1.821927in}{1.077795in}}%
\pgfpathlineto{\pgfqpoint{1.825205in}{1.075075in}}%
\pgfpathlineto{\pgfqpoint{1.828485in}{1.072520in}}%
\pgfpathlineto{\pgfqpoint{1.831769in}{1.070133in}}%
\pgfpathlineto{\pgfqpoint{1.835057in}{1.067917in}}%
\pgfpathlineto{\pgfqpoint{1.836716in}{1.059966in}}%
\pgfpathlineto{\pgfqpoint{1.837896in}{1.051983in}}%
\pgfpathlineto{\pgfqpoint{1.838594in}{1.043973in}}%
\pgfpathlineto{\pgfqpoint{1.838806in}{1.035947in}}%
\pgfpathlineto{\pgfqpoint{1.835477in}{1.038391in}}%
\pgfpathlineto{\pgfqpoint{1.832152in}{1.041008in}}%
\pgfpathlineto{\pgfqpoint{1.828830in}{1.043793in}}%
\pgfpathlineto{\pgfqpoint{1.825511in}{1.046743in}}%
\pgfpathlineto{\pgfqpoint{1.825321in}{1.054539in}}%
\pgfpathlineto{\pgfqpoint{1.824658in}{1.062317in}}%
\pgfpathlineto{\pgfqpoint{1.823526in}{1.070072in}}%
\pgfpathlineto{\pgfqpoint{1.821927in}{1.077795in}}%
\pgfpathclose%
\pgfusepath{fill}%
\end{pgfscope}%
\begin{pgfscope}%
\pgfpathrectangle{\pgfqpoint{0.329460in}{0.284240in}}{\pgfqpoint{1.989680in}{1.989680in}}%
\pgfusepath{clip}%
\pgfsetbuttcap%
\pgfsetroundjoin%
\definecolor{currentfill}{rgb}{0.280255,0.165693,0.476498}%
\pgfsetfillcolor{currentfill}%
\pgfsetlinewidth{0.000000pt}%
\definecolor{currentstroke}{rgb}{0.000000,0.000000,0.000000}%
\pgfsetstrokecolor{currentstroke}%
\pgfsetdash{}{0pt}%
\pgfpathmoveto{\pgfqpoint{0.931378in}{1.134672in}}%
\pgfpathlineto{\pgfqpoint{0.928114in}{1.129758in}}%
\pgfpathlineto{\pgfqpoint{0.924849in}{1.124952in}}%
\pgfpathlineto{\pgfqpoint{0.921584in}{1.120257in}}%
\pgfpathlineto{\pgfqpoint{0.918318in}{1.115677in}}%
\pgfpathlineto{\pgfqpoint{0.919788in}{1.122705in}}%
\pgfpathlineto{\pgfqpoint{0.921683in}{1.129698in}}%
\pgfpathlineto{\pgfqpoint{0.923999in}{1.136649in}}%
\pgfpathlineto{\pgfqpoint{0.926734in}{1.143551in}}%
\pgfpathlineto{\pgfqpoint{0.929923in}{1.147902in}}%
\pgfpathlineto{\pgfqpoint{0.933111in}{1.152367in}}%
\pgfpathlineto{\pgfqpoint{0.936299in}{1.156944in}}%
\pgfpathlineto{\pgfqpoint{0.939486in}{1.161628in}}%
\pgfpathlineto{\pgfqpoint{0.936847in}{1.154952in}}%
\pgfpathlineto{\pgfqpoint{0.934614in}{1.148230in}}%
\pgfpathlineto{\pgfqpoint{0.932790in}{1.141468in}}%
\pgfpathlineto{\pgfqpoint{0.931378in}{1.134672in}}%
\pgfpathclose%
\pgfusepath{fill}%
\end{pgfscope}%
\begin{pgfscope}%
\pgfpathrectangle{\pgfqpoint{0.329460in}{0.284240in}}{\pgfqpoint{1.989680in}{1.989680in}}%
\pgfusepath{clip}%
\pgfsetbuttcap%
\pgfsetroundjoin%
\definecolor{currentfill}{rgb}{0.120081,0.622161,0.534946}%
\pgfsetfillcolor{currentfill}%
\pgfsetlinewidth{0.000000pt}%
\definecolor{currentstroke}{rgb}{0.000000,0.000000,0.000000}%
\pgfsetstrokecolor{currentstroke}%
\pgfsetdash{}{0pt}%
\pgfpathmoveto{\pgfqpoint{1.415921in}{1.552961in}}%
\pgfpathlineto{\pgfqpoint{1.416733in}{1.547891in}}%
\pgfpathlineto{\pgfqpoint{1.417543in}{1.542796in}}%
\pgfpathlineto{\pgfqpoint{1.418352in}{1.537680in}}%
\pgfpathlineto{\pgfqpoint{1.419160in}{1.532544in}}%
\pgfpathlineto{\pgfqpoint{1.427394in}{1.531448in}}%
\pgfpathlineto{\pgfqpoint{1.435557in}{1.530226in}}%
\pgfpathlineto{\pgfqpoint{1.443641in}{1.528880in}}%
\pgfpathlineto{\pgfqpoint{1.442541in}{1.534063in}}%
\pgfpathlineto{\pgfqpoint{1.441440in}{1.539226in}}%
\pgfpathlineto{\pgfqpoint{1.440337in}{1.544367in}}%
\pgfpathlineto{\pgfqpoint{1.439233in}{1.549484in}}%
\pgfpathlineto{\pgfqpoint{1.431535in}{1.550761in}}%
\pgfpathlineto{\pgfqpoint{1.423762in}{1.551920in}}%
\pgfpathlineto{\pgfqpoint{1.415921in}{1.552961in}}%
\pgfpathclose%
\pgfusepath{fill}%
\end{pgfscope}%
\begin{pgfscope}%
\pgfpathrectangle{\pgfqpoint{0.329460in}{0.284240in}}{\pgfqpoint{1.989680in}{1.989680in}}%
\pgfusepath{clip}%
\pgfsetbuttcap%
\pgfsetroundjoin%
\definecolor{currentfill}{rgb}{0.248629,0.278775,0.534556}%
\pgfsetfillcolor{currentfill}%
\pgfsetlinewidth{0.000000pt}%
\definecolor{currentstroke}{rgb}{0.000000,0.000000,0.000000}%
\pgfsetstrokecolor{currentstroke}%
\pgfsetdash{}{0pt}%
\pgfpathmoveto{\pgfqpoint{0.977753in}{1.225103in}}%
\pgfpathlineto{\pgfqpoint{0.974560in}{1.219371in}}%
\pgfpathlineto{\pgfqpoint{0.971368in}{1.213709in}}%
\pgfpathlineto{\pgfqpoint{0.968178in}{1.208120in}}%
\pgfpathlineto{\pgfqpoint{0.964988in}{1.202608in}}%
\pgfpathlineto{\pgfqpoint{0.967816in}{1.208781in}}%
\pgfpathlineto{\pgfqpoint{0.971017in}{1.214900in}}%
\pgfpathlineto{\pgfqpoint{0.974588in}{1.220958in}}%
\pgfpathlineto{\pgfqpoint{0.978524in}{1.226951in}}%
\pgfpathlineto{\pgfqpoint{0.981591in}{1.232242in}}%
\pgfpathlineto{\pgfqpoint{0.984658in}{1.237610in}}%
\pgfpathlineto{\pgfqpoint{0.987726in}{1.243050in}}%
\pgfpathlineto{\pgfqpoint{0.990796in}{1.248561in}}%
\pgfpathlineto{\pgfqpoint{0.987001in}{1.242786in}}%
\pgfpathlineto{\pgfqpoint{0.983559in}{1.236947in}}%
\pgfpathlineto{\pgfqpoint{0.980475in}{1.231051in}}%
\pgfpathlineto{\pgfqpoint{0.977753in}{1.225103in}}%
\pgfpathclose%
\pgfusepath{fill}%
\end{pgfscope}%
\begin{pgfscope}%
\pgfpathrectangle{\pgfqpoint{0.329460in}{0.284240in}}{\pgfqpoint{1.989680in}{1.989680in}}%
\pgfusepath{clip}%
\pgfsetbuttcap%
\pgfsetroundjoin%
\definecolor{currentfill}{rgb}{0.122606,0.585371,0.546557}%
\pgfsetfillcolor{currentfill}%
\pgfsetlinewidth{0.000000pt}%
\definecolor{currentstroke}{rgb}{0.000000,0.000000,0.000000}%
\pgfsetstrokecolor{currentstroke}%
\pgfsetdash{}{0pt}%
\pgfpathmoveto{\pgfqpoint{1.191537in}{1.511696in}}%
\pgfpathlineto{\pgfqpoint{1.189635in}{1.506277in}}%
\pgfpathlineto{\pgfqpoint{1.187735in}{1.500842in}}%
\pgfpathlineto{\pgfqpoint{1.185838in}{1.495396in}}%
\pgfpathlineto{\pgfqpoint{1.183943in}{1.489939in}}%
\pgfpathlineto{\pgfqpoint{1.191338in}{1.492444in}}%
\pgfpathlineto{\pgfqpoint{1.198883in}{1.494833in}}%
\pgfpathlineto{\pgfqpoint{1.206572in}{1.497105in}}%
\pgfpathlineto{\pgfqpoint{1.214396in}{1.499258in}}%
\pgfpathlineto{\pgfqpoint{1.215944in}{1.504601in}}%
\pgfpathlineto{\pgfqpoint{1.217493in}{1.509935in}}%
\pgfpathlineto{\pgfqpoint{1.219045in}{1.515255in}}%
\pgfpathlineto{\pgfqpoint{1.220599in}{1.520561in}}%
\pgfpathlineto{\pgfqpoint{1.213131in}{1.518513in}}%
\pgfpathlineto{\pgfqpoint{1.205794in}{1.516352in}}%
\pgfpathlineto{\pgfqpoint{1.198594in}{1.514079in}}%
\pgfpathlineto{\pgfqpoint{1.191537in}{1.511696in}}%
\pgfpathclose%
\pgfusepath{fill}%
\end{pgfscope}%
\begin{pgfscope}%
\pgfpathrectangle{\pgfqpoint{0.329460in}{0.284240in}}{\pgfqpoint{1.989680in}{1.989680in}}%
\pgfusepath{clip}%
\pgfsetbuttcap%
\pgfsetroundjoin%
\definecolor{currentfill}{rgb}{0.267004,0.004874,0.329415}%
\pgfsetfillcolor{currentfill}%
\pgfsetlinewidth{0.000000pt}%
\definecolor{currentstroke}{rgb}{0.000000,0.000000,0.000000}%
\pgfsetstrokecolor{currentstroke}%
\pgfsetdash{}{0pt}%
\pgfpathmoveto{\pgfqpoint{0.836978in}{1.015514in}}%
\pgfpathlineto{\pgfqpoint{0.833603in}{1.014748in}}%
\pgfpathlineto{\pgfqpoint{0.830223in}{1.014194in}}%
\pgfpathlineto{\pgfqpoint{0.826836in}{1.013857in}}%
\pgfpathlineto{\pgfqpoint{0.823443in}{1.013741in}}%
\pgfpathlineto{\pgfqpoint{0.823259in}{1.022461in}}%
\pgfpathlineto{\pgfqpoint{0.823602in}{1.031168in}}%
\pgfpathlineto{\pgfqpoint{0.824469in}{1.039853in}}%
\pgfpathlineto{\pgfqpoint{0.825857in}{1.048509in}}%
\pgfpathlineto{\pgfqpoint{0.829217in}{1.048401in}}%
\pgfpathlineto{\pgfqpoint{0.832571in}{1.048514in}}%
\pgfpathlineto{\pgfqpoint{0.835918in}{1.048844in}}%
\pgfpathlineto{\pgfqpoint{0.839260in}{1.049385in}}%
\pgfpathlineto{\pgfqpoint{0.837924in}{1.040952in}}%
\pgfpathlineto{\pgfqpoint{0.837097in}{1.032490in}}%
\pgfpathlineto{\pgfqpoint{0.836781in}{1.024008in}}%
\pgfpathlineto{\pgfqpoint{0.836978in}{1.015514in}}%
\pgfpathclose%
\pgfusepath{fill}%
\end{pgfscope}%
\begin{pgfscope}%
\pgfpathrectangle{\pgfqpoint{0.329460in}{0.284240in}}{\pgfqpoint{1.989680in}{1.989680in}}%
\pgfusepath{clip}%
\pgfsetbuttcap%
\pgfsetroundjoin%
\definecolor{currentfill}{rgb}{0.267004,0.004874,0.329415}%
\pgfsetfillcolor{currentfill}%
\pgfsetlinewidth{0.000000pt}%
\definecolor{currentstroke}{rgb}{0.000000,0.000000,0.000000}%
\pgfsetstrokecolor{currentstroke}%
\pgfsetdash{}{0pt}%
\pgfpathmoveto{\pgfqpoint{0.823443in}{1.013741in}}%
\pgfpathlineto{\pgfqpoint{0.820043in}{1.013851in}}%
\pgfpathlineto{\pgfqpoint{0.816637in}{1.014190in}}%
\pgfpathlineto{\pgfqpoint{0.813223in}{1.014764in}}%
\pgfpathlineto{\pgfqpoint{0.809801in}{1.015576in}}%
\pgfpathlineto{\pgfqpoint{0.809631in}{1.024519in}}%
\pgfpathlineto{\pgfqpoint{0.810002in}{1.033447in}}%
\pgfpathlineto{\pgfqpoint{0.810910in}{1.042353in}}%
\pgfpathlineto{\pgfqpoint{0.812353in}{1.051227in}}%
\pgfpathlineto{\pgfqpoint{0.815739in}{1.050195in}}%
\pgfpathlineto{\pgfqpoint{0.819119in}{1.049401in}}%
\pgfpathlineto{\pgfqpoint{0.822491in}{1.048840in}}%
\pgfpathlineto{\pgfqpoint{0.825857in}{1.048509in}}%
\pgfpathlineto{\pgfqpoint{0.824469in}{1.039853in}}%
\pgfpathlineto{\pgfqpoint{0.823602in}{1.031168in}}%
\pgfpathlineto{\pgfqpoint{0.823259in}{1.022461in}}%
\pgfpathlineto{\pgfqpoint{0.823443in}{1.013741in}}%
\pgfpathclose%
\pgfusepath{fill}%
\end{pgfscope}%
\begin{pgfscope}%
\pgfpathrectangle{\pgfqpoint{0.329460in}{0.284240in}}{\pgfqpoint{1.989680in}{1.989680in}}%
\pgfusepath{clip}%
\pgfsetbuttcap%
\pgfsetroundjoin%
\definecolor{currentfill}{rgb}{0.120081,0.622161,0.534946}%
\pgfsetfillcolor{currentfill}%
\pgfsetlinewidth{0.000000pt}%
\definecolor{currentstroke}{rgb}{0.000000,0.000000,0.000000}%
\pgfsetstrokecolor{currentstroke}%
\pgfsetdash{}{0pt}%
\pgfpathmoveto{\pgfqpoint{1.256369in}{1.548251in}}%
\pgfpathlineto{\pgfqpoint{1.255180in}{1.543117in}}%
\pgfpathlineto{\pgfqpoint{1.253992in}{1.537960in}}%
\pgfpathlineto{\pgfqpoint{1.252806in}{1.532780in}}%
\pgfpathlineto{\pgfqpoint{1.251621in}{1.527581in}}%
\pgfpathlineto{\pgfqpoint{1.259629in}{1.529036in}}%
\pgfpathlineto{\pgfqpoint{1.267722in}{1.530368in}}%
\pgfpathlineto{\pgfqpoint{1.275893in}{1.531576in}}%
\pgfpathlineto{\pgfqpoint{1.284134in}{1.532659in}}%
\pgfpathlineto{\pgfqpoint{1.284931in}{1.537793in}}%
\pgfpathlineto{\pgfqpoint{1.285729in}{1.542907in}}%
\pgfpathlineto{\pgfqpoint{1.286529in}{1.548000in}}%
\pgfpathlineto{\pgfqpoint{1.287329in}{1.553069in}}%
\pgfpathlineto{\pgfqpoint{1.279481in}{1.552042in}}%
\pgfpathlineto{\pgfqpoint{1.271700in}{1.550895in}}%
\pgfpathlineto{\pgfqpoint{1.263994in}{1.549631in}}%
\pgfpathlineto{\pgfqpoint{1.256369in}{1.548251in}}%
\pgfpathclose%
\pgfusepath{fill}%
\end{pgfscope}%
\begin{pgfscope}%
\pgfpathrectangle{\pgfqpoint{0.329460in}{0.284240in}}{\pgfqpoint{1.989680in}{1.989680in}}%
\pgfusepath{clip}%
\pgfsetbuttcap%
\pgfsetroundjoin%
\definecolor{currentfill}{rgb}{0.277941,0.056324,0.381191}%
\pgfsetfillcolor{currentfill}%
\pgfsetlinewidth{0.000000pt}%
\definecolor{currentstroke}{rgb}{0.000000,0.000000,0.000000}%
\pgfsetstrokecolor{currentstroke}%
\pgfsetdash{}{0pt}%
\pgfpathmoveto{\pgfqpoint{1.915553in}{1.076820in}}%
\pgfpathlineto{\pgfqpoint{1.919000in}{1.080245in}}%
\pgfpathlineto{\pgfqpoint{1.922457in}{1.083954in}}%
\pgfpathlineto{\pgfqpoint{1.925925in}{1.087953in}}%
\pgfpathlineto{\pgfqpoint{1.929402in}{1.092246in}}%
\pgfpathlineto{\pgfqpoint{1.931525in}{1.082779in}}%
\pgfpathlineto{\pgfqpoint{1.933076in}{1.073267in}}%
\pgfpathlineto{\pgfqpoint{1.934051in}{1.063721in}}%
\pgfpathlineto{\pgfqpoint{1.934447in}{1.054149in}}%
\pgfpathlineto{\pgfqpoint{1.930918in}{1.050061in}}%
\pgfpathlineto{\pgfqpoint{1.927400in}{1.046269in}}%
\pgfpathlineto{\pgfqpoint{1.923893in}{1.042767in}}%
\pgfpathlineto{\pgfqpoint{1.920396in}{1.039551in}}%
\pgfpathlineto{\pgfqpoint{1.920030in}{1.048914in}}%
\pgfpathlineto{\pgfqpoint{1.919098in}{1.058252in}}%
\pgfpathlineto{\pgfqpoint{1.917604in}{1.067557in}}%
\pgfpathlineto{\pgfqpoint{1.915553in}{1.076820in}}%
\pgfpathclose%
\pgfusepath{fill}%
\end{pgfscope}%
\begin{pgfscope}%
\pgfpathrectangle{\pgfqpoint{0.329460in}{0.284240in}}{\pgfqpoint{1.989680in}{1.989680in}}%
\pgfusepath{clip}%
\pgfsetbuttcap%
\pgfsetroundjoin%
\definecolor{currentfill}{rgb}{0.268510,0.009605,0.335427}%
\pgfsetfillcolor{currentfill}%
\pgfsetlinewidth{0.000000pt}%
\definecolor{currentstroke}{rgb}{0.000000,0.000000,0.000000}%
\pgfsetstrokecolor{currentstroke}%
\pgfsetdash{}{0pt}%
\pgfpathmoveto{\pgfqpoint{0.850422in}{1.020622in}}%
\pgfpathlineto{\pgfqpoint{0.847068in}{1.019047in}}%
\pgfpathlineto{\pgfqpoint{0.843710in}{1.017667in}}%
\pgfpathlineto{\pgfqpoint{0.840347in}{1.016488in}}%
\pgfpathlineto{\pgfqpoint{0.836978in}{1.015514in}}%
\pgfpathlineto{\pgfqpoint{0.836781in}{1.024008in}}%
\pgfpathlineto{\pgfqpoint{0.837097in}{1.032490in}}%
\pgfpathlineto{\pgfqpoint{0.837924in}{1.040952in}}%
\pgfpathlineto{\pgfqpoint{0.839260in}{1.049385in}}%
\pgfpathlineto{\pgfqpoint{0.842597in}{1.050134in}}%
\pgfpathlineto{\pgfqpoint{0.845928in}{1.051086in}}%
\pgfpathlineto{\pgfqpoint{0.849255in}{1.052238in}}%
\pgfpathlineto{\pgfqpoint{0.852576in}{1.053586in}}%
\pgfpathlineto{\pgfqpoint{0.851292in}{1.045378in}}%
\pgfpathlineto{\pgfqpoint{0.850503in}{1.037143in}}%
\pgfpathlineto{\pgfqpoint{0.850212in}{1.028888in}}%
\pgfpathlineto{\pgfqpoint{0.850422in}{1.020622in}}%
\pgfpathclose%
\pgfusepath{fill}%
\end{pgfscope}%
\begin{pgfscope}%
\pgfpathrectangle{\pgfqpoint{0.329460in}{0.284240in}}{\pgfqpoint{1.989680in}{1.989680in}}%
\pgfusepath{clip}%
\pgfsetbuttcap%
\pgfsetroundjoin%
\definecolor{currentfill}{rgb}{0.274128,0.199721,0.498911}%
\pgfsetfillcolor{currentfill}%
\pgfsetlinewidth{0.000000pt}%
\definecolor{currentstroke}{rgb}{0.000000,0.000000,0.000000}%
\pgfsetstrokecolor{currentstroke}%
\pgfsetdash{}{0pt}%
\pgfpathmoveto{\pgfqpoint{1.747552in}{1.187072in}}%
\pgfpathlineto{\pgfqpoint{1.750715in}{1.182038in}}%
\pgfpathlineto{\pgfqpoint{1.753879in}{1.177099in}}%
\pgfpathlineto{\pgfqpoint{1.757042in}{1.172258in}}%
\pgfpathlineto{\pgfqpoint{1.760206in}{1.167518in}}%
\pgfpathlineto{\pgfqpoint{1.763202in}{1.160889in}}%
\pgfpathlineto{\pgfqpoint{1.765796in}{1.154208in}}%
\pgfpathlineto{\pgfqpoint{1.767985in}{1.147481in}}%
\pgfpathlineto{\pgfqpoint{1.769763in}{1.140714in}}%
\pgfpathlineto{\pgfqpoint{1.766512in}{1.145682in}}%
\pgfpathlineto{\pgfqpoint{1.763261in}{1.150751in}}%
\pgfpathlineto{\pgfqpoint{1.760011in}{1.155918in}}%
\pgfpathlineto{\pgfqpoint{1.756760in}{1.161181in}}%
\pgfpathlineto{\pgfqpoint{1.755050in}{1.167716in}}%
\pgfpathlineto{\pgfqpoint{1.752943in}{1.174214in}}%
\pgfpathlineto{\pgfqpoint{1.750443in}{1.180668in}}%
\pgfpathlineto{\pgfqpoint{1.747552in}{1.187072in}}%
\pgfpathclose%
\pgfusepath{fill}%
\end{pgfscope}%
\begin{pgfscope}%
\pgfpathrectangle{\pgfqpoint{0.329460in}{0.284240in}}{\pgfqpoint{1.989680in}{1.989680in}}%
\pgfusepath{clip}%
\pgfsetbuttcap%
\pgfsetroundjoin%
\definecolor{currentfill}{rgb}{0.268510,0.009605,0.335427}%
\pgfsetfillcolor{currentfill}%
\pgfsetlinewidth{0.000000pt}%
\definecolor{currentstroke}{rgb}{0.000000,0.000000,0.000000}%
\pgfsetstrokecolor{currentstroke}%
\pgfsetdash{}{0pt}%
\pgfpathmoveto{\pgfqpoint{0.809801in}{1.015576in}}%
\pgfpathlineto{\pgfqpoint{0.806373in}{1.016632in}}%
\pgfpathlineto{\pgfqpoint{0.802936in}{1.017935in}}%
\pgfpathlineto{\pgfqpoint{0.799490in}{1.019490in}}%
\pgfpathlineto{\pgfqpoint{0.796037in}{1.021303in}}%
\pgfpathlineto{\pgfqpoint{0.795882in}{1.030464in}}%
\pgfpathlineto{\pgfqpoint{0.796281in}{1.039611in}}%
\pgfpathlineto{\pgfqpoint{0.797231in}{1.048733in}}%
\pgfpathlineto{\pgfqpoint{0.798730in}{1.057823in}}%
\pgfpathlineto{\pgfqpoint{0.802148in}{1.055795in}}%
\pgfpathlineto{\pgfqpoint{0.805557in}{1.054022in}}%
\pgfpathlineto{\pgfqpoint{0.808959in}{1.052501in}}%
\pgfpathlineto{\pgfqpoint{0.812353in}{1.051227in}}%
\pgfpathlineto{\pgfqpoint{0.810910in}{1.042353in}}%
\pgfpathlineto{\pgfqpoint{0.810002in}{1.033447in}}%
\pgfpathlineto{\pgfqpoint{0.809631in}{1.024519in}}%
\pgfpathlineto{\pgfqpoint{0.809801in}{1.015576in}}%
\pgfpathclose%
\pgfusepath{fill}%
\end{pgfscope}%
\begin{pgfscope}%
\pgfpathrectangle{\pgfqpoint{0.329460in}{0.284240in}}{\pgfqpoint{1.989680in}{1.989680in}}%
\pgfusepath{clip}%
\pgfsetbuttcap%
\pgfsetroundjoin%
\definecolor{currentfill}{rgb}{0.179019,0.433756,0.557430}%
\pgfsetfillcolor{currentfill}%
\pgfsetlinewidth{0.000000pt}%
\definecolor{currentstroke}{rgb}{0.000000,0.000000,0.000000}%
\pgfsetstrokecolor{currentstroke}%
\pgfsetdash{}{0pt}%
\pgfpathmoveto{\pgfqpoint{1.056065in}{1.362919in}}%
\pgfpathlineto{\pgfqpoint{1.053133in}{1.356887in}}%
\pgfpathlineto{\pgfqpoint{1.050203in}{1.350879in}}%
\pgfpathlineto{\pgfqpoint{1.047276in}{1.344900in}}%
\pgfpathlineto{\pgfqpoint{1.044351in}{1.338950in}}%
\pgfpathlineto{\pgfqpoint{1.049249in}{1.343729in}}%
\pgfpathlineto{\pgfqpoint{1.054437in}{1.348424in}}%
\pgfpathlineto{\pgfqpoint{1.059909in}{1.353033in}}%
\pgfpathlineto{\pgfqpoint{1.065659in}{1.357550in}}%
\pgfpathlineto{\pgfqpoint{1.068373in}{1.363305in}}%
\pgfpathlineto{\pgfqpoint{1.071089in}{1.369090in}}%
\pgfpathlineto{\pgfqpoint{1.073807in}{1.374903in}}%
\pgfpathlineto{\pgfqpoint{1.076528in}{1.380741in}}%
\pgfpathlineto{\pgfqpoint{1.071005in}{1.376412in}}%
\pgfpathlineto{\pgfqpoint{1.065750in}{1.371996in}}%
\pgfpathlineto{\pgfqpoint{1.060768in}{1.367497in}}%
\pgfpathlineto{\pgfqpoint{1.056065in}{1.362919in}}%
\pgfpathclose%
\pgfusepath{fill}%
\end{pgfscope}%
\begin{pgfscope}%
\pgfpathrectangle{\pgfqpoint{0.329460in}{0.284240in}}{\pgfqpoint{1.989680in}{1.989680in}}%
\pgfusepath{clip}%
\pgfsetbuttcap%
\pgfsetroundjoin%
\definecolor{currentfill}{rgb}{0.172719,0.448791,0.557885}%
\pgfsetfillcolor{currentfill}%
\pgfsetlinewidth{0.000000pt}%
\definecolor{currentstroke}{rgb}{0.000000,0.000000,0.000000}%
\pgfsetstrokecolor{currentstroke}%
\pgfsetdash{}{0pt}%
\pgfpathmoveto{\pgfqpoint{0.697622in}{1.330497in}}%
\pgfpathlineto{\pgfqpoint{0.693851in}{1.344139in}}%
\pgfpathlineto{\pgfqpoint{0.690059in}{1.358234in}}%
\pgfpathlineto{\pgfqpoint{0.686246in}{1.372790in}}%
\pgfpathlineto{\pgfqpoint{0.691581in}{1.383124in}}%
\pgfpathlineto{\pgfqpoint{0.697546in}{1.393354in}}%
\pgfpathlineto{\pgfqpoint{0.704134in}{1.403469in}}%
\pgfpathlineto{\pgfqpoint{0.711334in}{1.413460in}}%
\pgfpathlineto{\pgfqpoint{0.714979in}{1.398769in}}%
\pgfpathlineto{\pgfqpoint{0.718604in}{1.384535in}}%
\pgfpathlineto{\pgfqpoint{0.722209in}{1.370752in}}%
\pgfpathlineto{\pgfqpoint{0.715149in}{1.360862in}}%
\pgfpathlineto{\pgfqpoint{0.708692in}{1.350850in}}%
\pgfpathlineto{\pgfqpoint{0.702847in}{1.340725in}}%
\pgfpathlineto{\pgfqpoint{0.697622in}{1.330497in}}%
\pgfpathclose%
\pgfusepath{fill}%
\end{pgfscope}%
\begin{pgfscope}%
\pgfpathrectangle{\pgfqpoint{0.329460in}{0.284240in}}{\pgfqpoint{1.989680in}{1.989680in}}%
\pgfusepath{clip}%
\pgfsetbuttcap%
\pgfsetroundjoin%
\definecolor{currentfill}{rgb}{0.271305,0.019942,0.347269}%
\pgfsetfillcolor{currentfill}%
\pgfsetlinewidth{0.000000pt}%
\definecolor{currentstroke}{rgb}{0.000000,0.000000,0.000000}%
\pgfsetstrokecolor{currentstroke}%
\pgfsetdash{}{0pt}%
\pgfpathmoveto{\pgfqpoint{0.863788in}{1.028804in}}%
\pgfpathlineto{\pgfqpoint{0.860453in}{1.026484in}}%
\pgfpathlineto{\pgfqpoint{0.857114in}{1.024344in}}%
\pgfpathlineto{\pgfqpoint{0.853770in}{1.022389in}}%
\pgfpathlineto{\pgfqpoint{0.850422in}{1.020622in}}%
\pgfpathlineto{\pgfqpoint{0.850212in}{1.028888in}}%
\pgfpathlineto{\pgfqpoint{0.850503in}{1.037143in}}%
\pgfpathlineto{\pgfqpoint{0.851292in}{1.045378in}}%
\pgfpathlineto{\pgfqpoint{0.852576in}{1.053586in}}%
\pgfpathlineto{\pgfqpoint{0.855893in}{1.055125in}}%
\pgfpathlineto{\pgfqpoint{0.859206in}{1.056852in}}%
\pgfpathlineto{\pgfqpoint{0.862515in}{1.058762in}}%
\pgfpathlineto{\pgfqpoint{0.865820in}{1.060852in}}%
\pgfpathlineto{\pgfqpoint{0.864586in}{1.052871in}}%
\pgfpathlineto{\pgfqpoint{0.863835in}{1.044864in}}%
\pgfpathlineto{\pgfqpoint{0.863568in}{1.036839in}}%
\pgfpathlineto{\pgfqpoint{0.863788in}{1.028804in}}%
\pgfpathclose%
\pgfusepath{fill}%
\end{pgfscope}%
\begin{pgfscope}%
\pgfpathrectangle{\pgfqpoint{0.329460in}{0.284240in}}{\pgfqpoint{1.989680in}{1.989680in}}%
\pgfusepath{clip}%
\pgfsetbuttcap%
\pgfsetroundjoin%
\definecolor{currentfill}{rgb}{0.260571,0.246922,0.522828}%
\pgfsetfillcolor{currentfill}%
\pgfsetlinewidth{0.000000pt}%
\definecolor{currentstroke}{rgb}{0.000000,0.000000,0.000000}%
\pgfsetstrokecolor{currentstroke}%
\pgfsetdash{}{0pt}%
\pgfpathmoveto{\pgfqpoint{0.728180in}{1.159755in}}%
\pgfpathlineto{\pgfqpoint{0.724517in}{1.168331in}}%
\pgfpathlineto{\pgfqpoint{0.720838in}{1.177280in}}%
\pgfpathlineto{\pgfqpoint{0.717142in}{1.186608in}}%
\pgfpathlineto{\pgfqpoint{0.713430in}{1.196323in}}%
\pgfpathlineto{\pgfqpoint{0.715926in}{1.206542in}}%
\pgfpathlineto{\pgfqpoint{0.719043in}{1.216699in}}%
\pgfpathlineto{\pgfqpoint{0.722776in}{1.226785in}}%
\pgfpathlineto{\pgfqpoint{0.727117in}{1.236790in}}%
\pgfpathlineto{\pgfqpoint{0.730726in}{1.226903in}}%
\pgfpathlineto{\pgfqpoint{0.734319in}{1.217400in}}%
\pgfpathlineto{\pgfqpoint{0.737897in}{1.208275in}}%
\pgfpathlineto{\pgfqpoint{0.741460in}{1.199521in}}%
\pgfpathlineto{\pgfqpoint{0.737241in}{1.189688in}}%
\pgfpathlineto{\pgfqpoint{0.733617in}{1.179776in}}%
\pgfpathlineto{\pgfqpoint{0.730595in}{1.169796in}}%
\pgfpathlineto{\pgfqpoint{0.728180in}{1.159755in}}%
\pgfpathclose%
\pgfusepath{fill}%
\end{pgfscope}%
\begin{pgfscope}%
\pgfpathrectangle{\pgfqpoint{0.329460in}{0.284240in}}{\pgfqpoint{1.989680in}{1.989680in}}%
\pgfusepath{clip}%
\pgfsetbuttcap%
\pgfsetroundjoin%
\definecolor{currentfill}{rgb}{0.279566,0.067836,0.391917}%
\pgfsetfillcolor{currentfill}%
\pgfsetlinewidth{0.000000pt}%
\definecolor{currentstroke}{rgb}{0.000000,0.000000,0.000000}%
\pgfsetstrokecolor{currentstroke}%
\pgfsetdash{}{0pt}%
\pgfpathmoveto{\pgfqpoint{1.808844in}{1.090243in}}%
\pgfpathlineto{\pgfqpoint{1.812111in}{1.086903in}}%
\pgfpathlineto{\pgfqpoint{1.815381in}{1.083712in}}%
\pgfpathlineto{\pgfqpoint{1.818653in}{1.080675in}}%
\pgfpathlineto{\pgfqpoint{1.821927in}{1.077795in}}%
\pgfpathlineto{\pgfqpoint{1.823526in}{1.070072in}}%
\pgfpathlineto{\pgfqpoint{1.824658in}{1.062317in}}%
\pgfpathlineto{\pgfqpoint{1.825321in}{1.054539in}}%
\pgfpathlineto{\pgfqpoint{1.825511in}{1.046743in}}%
\pgfpathlineto{\pgfqpoint{1.822196in}{1.049855in}}%
\pgfpathlineto{\pgfqpoint{1.818884in}{1.053123in}}%
\pgfpathlineto{\pgfqpoint{1.815574in}{1.056545in}}%
\pgfpathlineto{\pgfqpoint{1.812267in}{1.060117in}}%
\pgfpathlineto{\pgfqpoint{1.812097in}{1.067679in}}%
\pgfpathlineto{\pgfqpoint{1.811468in}{1.075226in}}%
\pgfpathlineto{\pgfqpoint{1.810383in}{1.082750in}}%
\pgfpathlineto{\pgfqpoint{1.808844in}{1.090243in}}%
\pgfpathclose%
\pgfusepath{fill}%
\end{pgfscope}%
\begin{pgfscope}%
\pgfpathrectangle{\pgfqpoint{0.329460in}{0.284240in}}{\pgfqpoint{1.989680in}{1.989680in}}%
\pgfusepath{clip}%
\pgfsetbuttcap%
\pgfsetroundjoin%
\definecolor{currentfill}{rgb}{0.272594,0.025563,0.353093}%
\pgfsetfillcolor{currentfill}%
\pgfsetlinewidth{0.000000pt}%
\definecolor{currentstroke}{rgb}{0.000000,0.000000,0.000000}%
\pgfsetstrokecolor{currentstroke}%
\pgfsetdash{}{0pt}%
\pgfpathmoveto{\pgfqpoint{0.796037in}{1.021303in}}%
\pgfpathlineto{\pgfqpoint{0.792574in}{1.023377in}}%
\pgfpathlineto{\pgfqpoint{0.789103in}{1.025717in}}%
\pgfpathlineto{\pgfqpoint{0.785622in}{1.028329in}}%
\pgfpathlineto{\pgfqpoint{0.782131in}{1.031216in}}%
\pgfpathlineto{\pgfqpoint{0.781992in}{1.040593in}}%
\pgfpathlineto{\pgfqpoint{0.782421in}{1.049953in}}%
\pgfpathlineto{\pgfqpoint{0.783415in}{1.059288in}}%
\pgfpathlineto{\pgfqpoint{0.784971in}{1.068589in}}%
\pgfpathlineto{\pgfqpoint{0.788425in}{1.065490in}}%
\pgfpathlineto{\pgfqpoint{0.791869in}{1.062665in}}%
\pgfpathlineto{\pgfqpoint{0.795304in}{1.060111in}}%
\pgfpathlineto{\pgfqpoint{0.798730in}{1.057823in}}%
\pgfpathlineto{\pgfqpoint{0.797231in}{1.048733in}}%
\pgfpathlineto{\pgfqpoint{0.796281in}{1.039611in}}%
\pgfpathlineto{\pgfqpoint{0.795882in}{1.030464in}}%
\pgfpathlineto{\pgfqpoint{0.796037in}{1.021303in}}%
\pgfpathclose%
\pgfusepath{fill}%
\end{pgfscope}%
\begin{pgfscope}%
\pgfpathrectangle{\pgfqpoint{0.329460in}{0.284240in}}{\pgfqpoint{1.989680in}{1.989680in}}%
\pgfusepath{clip}%
\pgfsetbuttcap%
\pgfsetroundjoin%
\definecolor{currentfill}{rgb}{0.231674,0.318106,0.544834}%
\pgfsetfillcolor{currentfill}%
\pgfsetlinewidth{0.000000pt}%
\definecolor{currentstroke}{rgb}{0.000000,0.000000,0.000000}%
\pgfsetstrokecolor{currentstroke}%
\pgfsetdash{}{0pt}%
\pgfpathmoveto{\pgfqpoint{1.695755in}{1.276133in}}%
\pgfpathlineto{\pgfqpoint{1.698797in}{1.270419in}}%
\pgfpathlineto{\pgfqpoint{1.701837in}{1.264763in}}%
\pgfpathlineto{\pgfqpoint{1.704876in}{1.259168in}}%
\pgfpathlineto{\pgfqpoint{1.707913in}{1.253638in}}%
\pgfpathlineto{\pgfqpoint{1.712018in}{1.247923in}}%
\pgfpathlineto{\pgfqpoint{1.715774in}{1.242140in}}%
\pgfpathlineto{\pgfqpoint{1.719176in}{1.236295in}}%
\pgfpathlineto{\pgfqpoint{1.722220in}{1.230393in}}%
\pgfpathlineto{\pgfqpoint{1.719050in}{1.236142in}}%
\pgfpathlineto{\pgfqpoint{1.715878in}{1.241956in}}%
\pgfpathlineto{\pgfqpoint{1.712705in}{1.247831in}}%
\pgfpathlineto{\pgfqpoint{1.709530in}{1.253764in}}%
\pgfpathlineto{\pgfqpoint{1.706601in}{1.259443in}}%
\pgfpathlineto{\pgfqpoint{1.703326in}{1.265068in}}%
\pgfpathlineto{\pgfqpoint{1.699710in}{1.270633in}}%
\pgfpathlineto{\pgfqpoint{1.695755in}{1.276133in}}%
\pgfpathclose%
\pgfusepath{fill}%
\end{pgfscope}%
\begin{pgfscope}%
\pgfpathrectangle{\pgfqpoint{0.329460in}{0.284240in}}{\pgfqpoint{1.989680in}{1.989680in}}%
\pgfusepath{clip}%
\pgfsetbuttcap%
\pgfsetroundjoin%
\definecolor{currentfill}{rgb}{0.195860,0.395433,0.555276}%
\pgfsetfillcolor{currentfill}%
\pgfsetlinewidth{0.000000pt}%
\definecolor{currentstroke}{rgb}{0.000000,0.000000,0.000000}%
\pgfsetstrokecolor{currentstroke}%
\pgfsetdash{}{0pt}%
\pgfpathmoveto{\pgfqpoint{1.653684in}{1.343202in}}%
\pgfpathlineto{\pgfqpoint{1.656564in}{1.337330in}}%
\pgfpathlineto{\pgfqpoint{1.659441in}{1.331493in}}%
\pgfpathlineto{\pgfqpoint{1.662316in}{1.325695in}}%
\pgfpathlineto{\pgfqpoint{1.665188in}{1.319939in}}%
\pgfpathlineto{\pgfqpoint{1.670249in}{1.314951in}}%
\pgfpathlineto{\pgfqpoint{1.675003in}{1.309881in}}%
\pgfpathlineto{\pgfqpoint{1.679445in}{1.304735in}}%
\pgfpathlineto{\pgfqpoint{1.683571in}{1.299516in}}%
\pgfpathlineto{\pgfqpoint{1.680520in}{1.305479in}}%
\pgfpathlineto{\pgfqpoint{1.677467in}{1.311483in}}%
\pgfpathlineto{\pgfqpoint{1.674412in}{1.317525in}}%
\pgfpathlineto{\pgfqpoint{1.671355in}{1.323603in}}%
\pgfpathlineto{\pgfqpoint{1.667390in}{1.328611in}}%
\pgfpathlineto{\pgfqpoint{1.663121in}{1.333549in}}%
\pgfpathlineto{\pgfqpoint{1.658550in}{1.338414in}}%
\pgfpathlineto{\pgfqpoint{1.653684in}{1.343202in}}%
\pgfpathclose%
\pgfusepath{fill}%
\end{pgfscope}%
\begin{pgfscope}%
\pgfpathrectangle{\pgfqpoint{0.329460in}{0.284240in}}{\pgfqpoint{1.989680in}{1.989680in}}%
\pgfusepath{clip}%
\pgfsetbuttcap%
\pgfsetroundjoin%
\definecolor{currentfill}{rgb}{0.120081,0.622161,0.534946}%
\pgfsetfillcolor{currentfill}%
\pgfsetlinewidth{0.000000pt}%
\definecolor{currentstroke}{rgb}{0.000000,0.000000,0.000000}%
\pgfsetstrokecolor{currentstroke}%
\pgfsetdash{}{0pt}%
\pgfpathmoveto{\pgfqpoint{1.439233in}{1.549484in}}%
\pgfpathlineto{\pgfqpoint{1.440337in}{1.544367in}}%
\pgfpathlineto{\pgfqpoint{1.441440in}{1.539226in}}%
\pgfpathlineto{\pgfqpoint{1.442541in}{1.534063in}}%
\pgfpathlineto{\pgfqpoint{1.443641in}{1.528880in}}%
\pgfpathlineto{\pgfqpoint{1.451638in}{1.527412in}}%
\pgfpathlineto{\pgfqpoint{1.459541in}{1.525822in}}%
\pgfpathlineto{\pgfqpoint{1.467343in}{1.524113in}}%
\pgfpathlineto{\pgfqpoint{1.475036in}{1.522284in}}%
\pgfpathlineto{\pgfqpoint{1.473561in}{1.527551in}}%
\pgfpathlineto{\pgfqpoint{1.472084in}{1.532798in}}%
\pgfpathlineto{\pgfqpoint{1.470605in}{1.538023in}}%
\pgfpathlineto{\pgfqpoint{1.469125in}{1.543224in}}%
\pgfpathlineto{\pgfqpoint{1.461801in}{1.544959in}}%
\pgfpathlineto{\pgfqpoint{1.454373in}{1.546582in}}%
\pgfpathlineto{\pgfqpoint{1.446848in}{1.548090in}}%
\pgfpathlineto{\pgfqpoint{1.439233in}{1.549484in}}%
\pgfpathclose%
\pgfusepath{fill}%
\end{pgfscope}%
\begin{pgfscope}%
\pgfpathrectangle{\pgfqpoint{0.329460in}{0.284240in}}{\pgfqpoint{1.989680in}{1.989680in}}%
\pgfusepath{clip}%
\pgfsetbuttcap%
\pgfsetroundjoin%
\definecolor{currentfill}{rgb}{0.133743,0.548535,0.553541}%
\pgfsetfillcolor{currentfill}%
\pgfsetlinewidth{0.000000pt}%
\definecolor{currentstroke}{rgb}{0.000000,0.000000,0.000000}%
\pgfsetstrokecolor{currentstroke}%
\pgfsetdash{}{0pt}%
\pgfpathmoveto{\pgfqpoint{1.540402in}{1.481432in}}%
\pgfpathlineto{\pgfqpoint{1.542546in}{1.475863in}}%
\pgfpathlineto{\pgfqpoint{1.544688in}{1.470288in}}%
\pgfpathlineto{\pgfqpoint{1.546828in}{1.464710in}}%
\pgfpathlineto{\pgfqpoint{1.548964in}{1.459132in}}%
\pgfpathlineto{\pgfqpoint{1.555976in}{1.456032in}}%
\pgfpathlineto{\pgfqpoint{1.562794in}{1.452823in}}%
\pgfpathlineto{\pgfqpoint{1.569413in}{1.449510in}}%
\pgfpathlineto{\pgfqpoint{1.575824in}{1.446094in}}%
\pgfpathlineto{\pgfqpoint{1.573392in}{1.451825in}}%
\pgfpathlineto{\pgfqpoint{1.570957in}{1.457555in}}%
\pgfpathlineto{\pgfqpoint{1.568519in}{1.463283in}}%
\pgfpathlineto{\pgfqpoint{1.566079in}{1.469005in}}%
\pgfpathlineto{\pgfqpoint{1.559950in}{1.472260in}}%
\pgfpathlineto{\pgfqpoint{1.553624in}{1.475419in}}%
\pgfpathlineto{\pgfqpoint{1.547106in}{1.478477in}}%
\pgfpathlineto{\pgfqpoint{1.540402in}{1.481432in}}%
\pgfpathclose%
\pgfusepath{fill}%
\end{pgfscope}%
\begin{pgfscope}%
\pgfpathrectangle{\pgfqpoint{0.329460in}{0.284240in}}{\pgfqpoint{1.989680in}{1.989680in}}%
\pgfusepath{clip}%
\pgfsetbuttcap%
\pgfsetroundjoin%
\definecolor{currentfill}{rgb}{0.274952,0.037752,0.364543}%
\pgfsetfillcolor{currentfill}%
\pgfsetlinewidth{0.000000pt}%
\definecolor{currentstroke}{rgb}{0.000000,0.000000,0.000000}%
\pgfsetstrokecolor{currentstroke}%
\pgfsetdash{}{0pt}%
\pgfpathmoveto{\pgfqpoint{0.877092in}{1.039807in}}%
\pgfpathlineto{\pgfqpoint{0.873771in}{1.036805in}}%
\pgfpathlineto{\pgfqpoint{0.870447in}{1.033968in}}%
\pgfpathlineto{\pgfqpoint{0.867120in}{1.031300in}}%
\pgfpathlineto{\pgfqpoint{0.863788in}{1.028804in}}%
\pgfpathlineto{\pgfqpoint{0.863568in}{1.036839in}}%
\pgfpathlineto{\pgfqpoint{0.863835in}{1.044864in}}%
\pgfpathlineto{\pgfqpoint{0.864586in}{1.052871in}}%
\pgfpathlineto{\pgfqpoint{0.865820in}{1.060852in}}%
\pgfpathlineto{\pgfqpoint{0.869121in}{1.063117in}}%
\pgfpathlineto{\pgfqpoint{0.872418in}{1.065555in}}%
\pgfpathlineto{\pgfqpoint{0.875713in}{1.068161in}}%
\pgfpathlineto{\pgfqpoint{0.879004in}{1.070932in}}%
\pgfpathlineto{\pgfqpoint{0.877820in}{1.063180in}}%
\pgfpathlineto{\pgfqpoint{0.877105in}{1.055404in}}%
\pgfpathlineto{\pgfqpoint{0.876862in}{1.047610in}}%
\pgfpathlineto{\pgfqpoint{0.877092in}{1.039807in}}%
\pgfpathclose%
\pgfusepath{fill}%
\end{pgfscope}%
\begin{pgfscope}%
\pgfpathrectangle{\pgfqpoint{0.329460in}{0.284240in}}{\pgfqpoint{1.989680in}{1.989680in}}%
\pgfusepath{clip}%
\pgfsetbuttcap%
\pgfsetroundjoin%
\definecolor{currentfill}{rgb}{0.282327,0.094955,0.417331}%
\pgfsetfillcolor{currentfill}%
\pgfsetlinewidth{0.000000pt}%
\definecolor{currentstroke}{rgb}{0.000000,0.000000,0.000000}%
\pgfsetstrokecolor{currentstroke}%
\pgfsetdash{}{0pt}%
\pgfpathmoveto{\pgfqpoint{1.929402in}{1.092246in}}%
\pgfpathlineto{\pgfqpoint{1.932891in}{1.096840in}}%
\pgfpathlineto{\pgfqpoint{1.936391in}{1.101738in}}%
\pgfpathlineto{\pgfqpoint{1.939902in}{1.106946in}}%
\pgfpathlineto{\pgfqpoint{1.943425in}{1.112470in}}%
\pgfpathlineto{\pgfqpoint{1.945621in}{1.102803in}}%
\pgfpathlineto{\pgfqpoint{1.947231in}{1.093090in}}%
\pgfpathlineto{\pgfqpoint{1.948251in}{1.083342in}}%
\pgfpathlineto{\pgfqpoint{1.948678in}{1.073566in}}%
\pgfpathlineto{\pgfqpoint{1.945103in}{1.068242in}}%
\pgfpathlineto{\pgfqpoint{1.941539in}{1.063234in}}%
\pgfpathlineto{\pgfqpoint{1.937987in}{1.058539in}}%
\pgfpathlineto{\pgfqpoint{1.934447in}{1.054149in}}%
\pgfpathlineto{\pgfqpoint{1.934051in}{1.063721in}}%
\pgfpathlineto{\pgfqpoint{1.933076in}{1.073267in}}%
\pgfpathlineto{\pgfqpoint{1.931525in}{1.082779in}}%
\pgfpathlineto{\pgfqpoint{1.929402in}{1.092246in}}%
\pgfpathclose%
\pgfusepath{fill}%
\end{pgfscope}%
\begin{pgfscope}%
\pgfpathrectangle{\pgfqpoint{0.329460in}{0.284240in}}{\pgfqpoint{1.989680in}{1.989680in}}%
\pgfusepath{clip}%
\pgfsetbuttcap%
\pgfsetroundjoin%
\definecolor{currentfill}{rgb}{0.147607,0.511733,0.557049}%
\pgfsetfillcolor{currentfill}%
\pgfsetlinewidth{0.000000pt}%
\definecolor{currentstroke}{rgb}{0.000000,0.000000,0.000000}%
\pgfsetstrokecolor{currentstroke}%
\pgfsetdash{}{0pt}%
\pgfpathmoveto{\pgfqpoint{1.575824in}{1.446094in}}%
\pgfpathlineto{\pgfqpoint{1.578253in}{1.440364in}}%
\pgfpathlineto{\pgfqpoint{1.580680in}{1.434639in}}%
\pgfpathlineto{\pgfqpoint{1.583105in}{1.428920in}}%
\pgfpathlineto{\pgfqpoint{1.585526in}{1.423211in}}%
\pgfpathlineto{\pgfqpoint{1.591998in}{1.419530in}}%
\pgfpathlineto{\pgfqpoint{1.598242in}{1.415748in}}%
\pgfpathlineto{\pgfqpoint{1.604250in}{1.411869in}}%
\pgfpathlineto{\pgfqpoint{1.610018in}{1.407896in}}%
\pgfpathlineto{\pgfqpoint{1.607336in}{1.413778in}}%
\pgfpathlineto{\pgfqpoint{1.604652in}{1.419669in}}%
\pgfpathlineto{\pgfqpoint{1.601965in}{1.425566in}}%
\pgfpathlineto{\pgfqpoint{1.599276in}{1.431468in}}%
\pgfpathlineto{\pgfqpoint{1.593754in}{1.435262in}}%
\pgfpathlineto{\pgfqpoint{1.588001in}{1.438967in}}%
\pgfpathlineto{\pgfqpoint{1.582022in}{1.442578in}}%
\pgfpathlineto{\pgfqpoint{1.575824in}{1.446094in}}%
\pgfpathclose%
\pgfusepath{fill}%
\end{pgfscope}%
\begin{pgfscope}%
\pgfpathrectangle{\pgfqpoint{0.329460in}{0.284240in}}{\pgfqpoint{1.989680in}{1.989680in}}%
\pgfusepath{clip}%
\pgfsetbuttcap%
\pgfsetroundjoin%
\definecolor{currentfill}{rgb}{0.233603,0.313828,0.543914}%
\pgfsetfillcolor{currentfill}%
\pgfsetlinewidth{0.000000pt}%
\definecolor{currentstroke}{rgb}{0.000000,0.000000,0.000000}%
\pgfsetstrokecolor{currentstroke}%
\pgfsetdash{}{0pt}%
\pgfpathmoveto{\pgfqpoint{1.970894in}{1.245609in}}%
\pgfpathlineto{\pgfqpoint{1.974489in}{1.255923in}}%
\pgfpathlineto{\pgfqpoint{1.978101in}{1.266633in}}%
\pgfpathlineto{\pgfqpoint{1.981729in}{1.277746in}}%
\pgfpathlineto{\pgfqpoint{1.985375in}{1.289269in}}%
\pgfpathlineto{\pgfqpoint{1.990388in}{1.279182in}}%
\pgfpathlineto{\pgfqpoint{1.994786in}{1.269003in}}%
\pgfpathlineto{\pgfqpoint{1.998561in}{1.258743in}}%
\pgfpathlineto{\pgfqpoint{2.001705in}{1.248411in}}%
\pgfpathlineto{\pgfqpoint{1.997941in}{1.237049in}}%
\pgfpathlineto{\pgfqpoint{1.994196in}{1.226100in}}%
\pgfpathlineto{\pgfqpoint{1.990468in}{1.215555in}}%
\pgfpathlineto{\pgfqpoint{1.986757in}{1.205409in}}%
\pgfpathlineto{\pgfqpoint{1.983709in}{1.215574in}}%
\pgfpathlineto{\pgfqpoint{1.980044in}{1.225668in}}%
\pgfpathlineto{\pgfqpoint{1.975771in}{1.235683in}}%
\pgfpathlineto{\pgfqpoint{1.970894in}{1.245609in}}%
\pgfpathclose%
\pgfusepath{fill}%
\end{pgfscope}%
\begin{pgfscope}%
\pgfpathrectangle{\pgfqpoint{0.329460in}{0.284240in}}{\pgfqpoint{1.989680in}{1.989680in}}%
\pgfusepath{clip}%
\pgfsetbuttcap%
\pgfsetroundjoin%
\definecolor{currentfill}{rgb}{0.120081,0.622161,0.534946}%
\pgfsetfillcolor{currentfill}%
\pgfsetlinewidth{0.000000pt}%
\definecolor{currentstroke}{rgb}{0.000000,0.000000,0.000000}%
\pgfsetstrokecolor{currentstroke}%
\pgfsetdash{}{0pt}%
\pgfpathmoveto{\pgfqpoint{1.226833in}{1.541589in}}%
\pgfpathlineto{\pgfqpoint{1.225271in}{1.536366in}}%
\pgfpathlineto{\pgfqpoint{1.223712in}{1.531118in}}%
\pgfpathlineto{\pgfqpoint{1.222154in}{1.525849in}}%
\pgfpathlineto{\pgfqpoint{1.220599in}{1.520561in}}%
\pgfpathlineto{\pgfqpoint{1.228188in}{1.522493in}}%
\pgfpathlineto{\pgfqpoint{1.235894in}{1.524308in}}%
\pgfpathlineto{\pgfqpoint{1.243707in}{1.526005in}}%
\pgfpathlineto{\pgfqpoint{1.251621in}{1.527581in}}%
\pgfpathlineto{\pgfqpoint{1.252806in}{1.532780in}}%
\pgfpathlineto{\pgfqpoint{1.253992in}{1.537960in}}%
\pgfpathlineto{\pgfqpoint{1.255180in}{1.543117in}}%
\pgfpathlineto{\pgfqpoint{1.256369in}{1.548251in}}%
\pgfpathlineto{\pgfqpoint{1.248834in}{1.546755in}}%
\pgfpathlineto{\pgfqpoint{1.241395in}{1.545145in}}%
\pgfpathlineto{\pgfqpoint{1.234059in}{1.543422in}}%
\pgfpathlineto{\pgfqpoint{1.226833in}{1.541589in}}%
\pgfpathclose%
\pgfusepath{fill}%
\end{pgfscope}%
\begin{pgfscope}%
\pgfpathrectangle{\pgfqpoint{0.329460in}{0.284240in}}{\pgfqpoint{1.989680in}{1.989680in}}%
\pgfusepath{clip}%
\pgfsetbuttcap%
\pgfsetroundjoin%
\definecolor{currentfill}{rgb}{0.274128,0.199721,0.498911}%
\pgfsetfillcolor{currentfill}%
\pgfsetlinewidth{0.000000pt}%
\definecolor{currentstroke}{rgb}{0.000000,0.000000,0.000000}%
\pgfsetstrokecolor{currentstroke}%
\pgfsetdash{}{0pt}%
\pgfpathmoveto{\pgfqpoint{0.944431in}{1.155345in}}%
\pgfpathlineto{\pgfqpoint{0.941168in}{1.150031in}}%
\pgfpathlineto{\pgfqpoint{0.937905in}{1.144812in}}%
\pgfpathlineto{\pgfqpoint{0.934641in}{1.139691in}}%
\pgfpathlineto{\pgfqpoint{0.931378in}{1.134672in}}%
\pgfpathlineto{\pgfqpoint{0.932790in}{1.141468in}}%
\pgfpathlineto{\pgfqpoint{0.934614in}{1.148230in}}%
\pgfpathlineto{\pgfqpoint{0.936847in}{1.154952in}}%
\pgfpathlineto{\pgfqpoint{0.939486in}{1.161628in}}%
\pgfpathlineto{\pgfqpoint{0.942673in}{1.166418in}}%
\pgfpathlineto{\pgfqpoint{0.945860in}{1.171309in}}%
\pgfpathlineto{\pgfqpoint{0.949047in}{1.176298in}}%
\pgfpathlineto{\pgfqpoint{0.952235in}{1.181382in}}%
\pgfpathlineto{\pgfqpoint{0.949690in}{1.174934in}}%
\pgfpathlineto{\pgfqpoint{0.947539in}{1.168440in}}%
\pgfpathlineto{\pgfqpoint{0.945785in}{1.161909in}}%
\pgfpathlineto{\pgfqpoint{0.944431in}{1.155345in}}%
\pgfpathclose%
\pgfusepath{fill}%
\end{pgfscope}%
\begin{pgfscope}%
\pgfpathrectangle{\pgfqpoint{0.329460in}{0.284240in}}{\pgfqpoint{1.989680in}{1.989680in}}%
\pgfusepath{clip}%
\pgfsetbuttcap%
\pgfsetroundjoin%
\definecolor{currentfill}{rgb}{0.122606,0.585371,0.546557}%
\pgfsetfillcolor{currentfill}%
\pgfsetlinewidth{0.000000pt}%
\definecolor{currentstroke}{rgb}{0.000000,0.000000,0.000000}%
\pgfsetstrokecolor{currentstroke}%
\pgfsetdash{}{0pt}%
\pgfpathmoveto{\pgfqpoint{1.504573in}{1.513819in}}%
\pgfpathlineto{\pgfqpoint{1.506400in}{1.508427in}}%
\pgfpathlineto{\pgfqpoint{1.508224in}{1.503020in}}%
\pgfpathlineto{\pgfqpoint{1.510046in}{1.497601in}}%
\pgfpathlineto{\pgfqpoint{1.511866in}{1.492171in}}%
\pgfpathlineto{\pgfqpoint{1.519245in}{1.489653in}}%
\pgfpathlineto{\pgfqpoint{1.526465in}{1.487022in}}%
\pgfpathlineto{\pgfqpoint{1.533519in}{1.484281in}}%
\pgfpathlineto{\pgfqpoint{1.540402in}{1.481432in}}%
\pgfpathlineto{\pgfqpoint{1.538255in}{1.486993in}}%
\pgfpathlineto{\pgfqpoint{1.536105in}{1.492544in}}%
\pgfpathlineto{\pgfqpoint{1.533953in}{1.498082in}}%
\pgfpathlineto{\pgfqpoint{1.531798in}{1.503605in}}%
\pgfpathlineto{\pgfqpoint{1.525232in}{1.506315in}}%
\pgfpathlineto{\pgfqpoint{1.518502in}{1.508922in}}%
\pgfpathlineto{\pgfqpoint{1.511613in}{1.511425in}}%
\pgfpathlineto{\pgfqpoint{1.504573in}{1.513819in}}%
\pgfpathclose%
\pgfusepath{fill}%
\end{pgfscope}%
\begin{pgfscope}%
\pgfpathrectangle{\pgfqpoint{0.329460in}{0.284240in}}{\pgfqpoint{1.989680in}{1.989680in}}%
\pgfusepath{clip}%
\pgfsetbuttcap%
\pgfsetroundjoin%
\definecolor{currentfill}{rgb}{0.282327,0.094955,0.417331}%
\pgfsetfillcolor{currentfill}%
\pgfsetlinewidth{0.000000pt}%
\definecolor{currentstroke}{rgb}{0.000000,0.000000,0.000000}%
\pgfsetstrokecolor{currentstroke}%
\pgfsetdash{}{0pt}%
\pgfpathmoveto{\pgfqpoint{1.795796in}{1.105029in}}%
\pgfpathlineto{\pgfqpoint{1.799055in}{1.101126in}}%
\pgfpathlineto{\pgfqpoint{1.802316in}{1.097358in}}%
\pgfpathlineto{\pgfqpoint{1.805579in}{1.093730in}}%
\pgfpathlineto{\pgfqpoint{1.808844in}{1.090243in}}%
\pgfpathlineto{\pgfqpoint{1.810383in}{1.082750in}}%
\pgfpathlineto{\pgfqpoint{1.811468in}{1.075226in}}%
\pgfpathlineto{\pgfqpoint{1.812097in}{1.067679in}}%
\pgfpathlineto{\pgfqpoint{1.812267in}{1.060117in}}%
\pgfpathlineto{\pgfqpoint{1.808962in}{1.063836in}}%
\pgfpathlineto{\pgfqpoint{1.805660in}{1.067697in}}%
\pgfpathlineto{\pgfqpoint{1.802359in}{1.071697in}}%
\pgfpathlineto{\pgfqpoint{1.799061in}{1.075833in}}%
\pgfpathlineto{\pgfqpoint{1.798911in}{1.083161in}}%
\pgfpathlineto{\pgfqpoint{1.798315in}{1.090475in}}%
\pgfpathlineto{\pgfqpoint{1.797276in}{1.097766in}}%
\pgfpathlineto{\pgfqpoint{1.795796in}{1.105029in}}%
\pgfpathclose%
\pgfusepath{fill}%
\end{pgfscope}%
\begin{pgfscope}%
\pgfpathrectangle{\pgfqpoint{0.329460in}{0.284240in}}{\pgfqpoint{1.989680in}{1.989680in}}%
\pgfusepath{clip}%
\pgfsetbuttcap%
\pgfsetroundjoin%
\definecolor{currentfill}{rgb}{0.133743,0.548535,0.553541}%
\pgfsetfillcolor{currentfill}%
\pgfsetlinewidth{0.000000pt}%
\definecolor{currentstroke}{rgb}{0.000000,0.000000,0.000000}%
\pgfsetstrokecolor{currentstroke}%
\pgfsetdash{}{0pt}%
\pgfpathmoveto{\pgfqpoint{1.131021in}{1.466031in}}%
\pgfpathlineto{\pgfqpoint{1.128519in}{1.460273in}}%
\pgfpathlineto{\pgfqpoint{1.126020in}{1.454509in}}%
\pgfpathlineto{\pgfqpoint{1.123524in}{1.448742in}}%
\pgfpathlineto{\pgfqpoint{1.121031in}{1.442974in}}%
\pgfpathlineto{\pgfqpoint{1.127253in}{1.446478in}}%
\pgfpathlineto{\pgfqpoint{1.133688in}{1.449883in}}%
\pgfpathlineto{\pgfqpoint{1.140329in}{1.453185in}}%
\pgfpathlineto{\pgfqpoint{1.147169in}{1.456382in}}%
\pgfpathlineto{\pgfqpoint{1.149374in}{1.461992in}}%
\pgfpathlineto{\pgfqpoint{1.151582in}{1.467602in}}%
\pgfpathlineto{\pgfqpoint{1.153792in}{1.473209in}}%
\pgfpathlineto{\pgfqpoint{1.156005in}{1.478810in}}%
\pgfpathlineto{\pgfqpoint{1.149466in}{1.475763in}}%
\pgfpathlineto{\pgfqpoint{1.143118in}{1.472616in}}%
\pgfpathlineto{\pgfqpoint{1.136967in}{1.469371in}}%
\pgfpathlineto{\pgfqpoint{1.131021in}{1.466031in}}%
\pgfpathclose%
\pgfusepath{fill}%
\end{pgfscope}%
\begin{pgfscope}%
\pgfpathrectangle{\pgfqpoint{0.329460in}{0.284240in}}{\pgfqpoint{1.989680in}{1.989680in}}%
\pgfusepath{clip}%
\pgfsetbuttcap%
\pgfsetroundjoin%
\definecolor{currentfill}{rgb}{0.277941,0.056324,0.381191}%
\pgfsetfillcolor{currentfill}%
\pgfsetlinewidth{0.000000pt}%
\definecolor{currentstroke}{rgb}{0.000000,0.000000,0.000000}%
\pgfsetstrokecolor{currentstroke}%
\pgfsetdash{}{0pt}%
\pgfpathmoveto{\pgfqpoint{0.782131in}{1.031216in}}%
\pgfpathlineto{\pgfqpoint{0.778630in}{1.034385in}}%
\pgfpathlineto{\pgfqpoint{0.775119in}{1.037840in}}%
\pgfpathlineto{\pgfqpoint{0.771597in}{1.041586in}}%
\pgfpathlineto{\pgfqpoint{0.768065in}{1.045628in}}%
\pgfpathlineto{\pgfqpoint{0.767943in}{1.055214in}}%
\pgfpathlineto{\pgfqpoint{0.768404in}{1.064783in}}%
\pgfpathlineto{\pgfqpoint{0.769443in}{1.074326in}}%
\pgfpathlineto{\pgfqpoint{0.771058in}{1.083833in}}%
\pgfpathlineto{\pgfqpoint{0.774552in}{1.079585in}}%
\pgfpathlineto{\pgfqpoint{0.778035in}{1.075631in}}%
\pgfpathlineto{\pgfqpoint{0.781508in}{1.071968in}}%
\pgfpathlineto{\pgfqpoint{0.784971in}{1.068589in}}%
\pgfpathlineto{\pgfqpoint{0.783415in}{1.059288in}}%
\pgfpathlineto{\pgfqpoint{0.782421in}{1.049953in}}%
\pgfpathlineto{\pgfqpoint{0.781992in}{1.040593in}}%
\pgfpathlineto{\pgfqpoint{0.782131in}{1.031216in}}%
\pgfpathclose%
\pgfusepath{fill}%
\end{pgfscope}%
\begin{pgfscope}%
\pgfpathrectangle{\pgfqpoint{0.329460in}{0.284240in}}{\pgfqpoint{1.989680in}{1.989680in}}%
\pgfusepath{clip}%
\pgfsetbuttcap%
\pgfsetroundjoin%
\definecolor{currentfill}{rgb}{0.195860,0.395433,0.555276}%
\pgfsetfillcolor{currentfill}%
\pgfsetlinewidth{0.000000pt}%
\definecolor{currentstroke}{rgb}{0.000000,0.000000,0.000000}%
\pgfsetstrokecolor{currentstroke}%
\pgfsetdash{}{0pt}%
\pgfpathmoveto{\pgfqpoint{1.027757in}{1.319099in}}%
\pgfpathlineto{\pgfqpoint{1.024666in}{1.312973in}}%
\pgfpathlineto{\pgfqpoint{1.021578in}{1.306883in}}%
\pgfpathlineto{\pgfqpoint{1.018491in}{1.300832in}}%
\pgfpathlineto{\pgfqpoint{1.015407in}{1.294821in}}%
\pgfpathlineto{\pgfqpoint{1.019247in}{1.300099in}}%
\pgfpathlineto{\pgfqpoint{1.023408in}{1.305310in}}%
\pgfpathlineto{\pgfqpoint{1.027885in}{1.310448in}}%
\pgfpathlineto{\pgfqpoint{1.032674in}{1.315509in}}%
\pgfpathlineto{\pgfqpoint{1.035590in}{1.321310in}}%
\pgfpathlineto{\pgfqpoint{1.038508in}{1.327153in}}%
\pgfpathlineto{\pgfqpoint{1.041428in}{1.333034in}}%
\pgfpathlineto{\pgfqpoint{1.044351in}{1.338950in}}%
\pgfpathlineto{\pgfqpoint{1.039748in}{1.334093in}}%
\pgfpathlineto{\pgfqpoint{1.035444in}{1.329163in}}%
\pgfpathlineto{\pgfqpoint{1.031446in}{1.324163in}}%
\pgfpathlineto{\pgfqpoint{1.027757in}{1.319099in}}%
\pgfpathclose%
\pgfusepath{fill}%
\end{pgfscope}%
\begin{pgfscope}%
\pgfpathrectangle{\pgfqpoint{0.329460in}{0.284240in}}{\pgfqpoint{1.989680in}{1.989680in}}%
\pgfusepath{clip}%
\pgfsetbuttcap%
\pgfsetroundjoin%
\definecolor{currentfill}{rgb}{0.134692,0.658636,0.517649}%
\pgfsetfillcolor{currentfill}%
\pgfsetlinewidth{0.000000pt}%
\definecolor{currentstroke}{rgb}{0.000000,0.000000,0.000000}%
\pgfsetstrokecolor{currentstroke}%
\pgfsetdash{}{0pt}%
\pgfpathmoveto{\pgfqpoint{1.320851in}{1.575813in}}%
\pgfpathlineto{\pgfqpoint{1.320449in}{1.570901in}}%
\pgfpathlineto{\pgfqpoint{1.320047in}{1.565956in}}%
\pgfpathlineto{\pgfqpoint{1.319646in}{1.560981in}}%
\pgfpathlineto{\pgfqpoint{1.319245in}{1.555977in}}%
\pgfpathlineto{\pgfqpoint{1.327317in}{1.556400in}}%
\pgfpathlineto{\pgfqpoint{1.335412in}{1.556700in}}%
\pgfpathlineto{\pgfqpoint{1.343522in}{1.556878in}}%
\pgfpathlineto{\pgfqpoint{1.351639in}{1.556932in}}%
\pgfpathlineto{\pgfqpoint{1.351633in}{1.561923in}}%
\pgfpathlineto{\pgfqpoint{1.351627in}{1.566886in}}%
\pgfpathlineto{\pgfqpoint{1.351622in}{1.571818in}}%
\pgfpathlineto{\pgfqpoint{1.351616in}{1.576717in}}%
\pgfpathlineto{\pgfqpoint{1.343907in}{1.576666in}}%
\pgfpathlineto{\pgfqpoint{1.336205in}{1.576497in}}%
\pgfpathlineto{\pgfqpoint{1.328518in}{1.576213in}}%
\pgfpathlineto{\pgfqpoint{1.320851in}{1.575813in}}%
\pgfpathclose%
\pgfusepath{fill}%
\end{pgfscope}%
\begin{pgfscope}%
\pgfpathrectangle{\pgfqpoint{0.329460in}{0.284240in}}{\pgfqpoint{1.989680in}{1.989680in}}%
\pgfusepath{clip}%
\pgfsetbuttcap%
\pgfsetroundjoin%
\definecolor{currentfill}{rgb}{0.134692,0.658636,0.517649}%
\pgfsetfillcolor{currentfill}%
\pgfsetlinewidth{0.000000pt}%
\definecolor{currentstroke}{rgb}{0.000000,0.000000,0.000000}%
\pgfsetstrokecolor{currentstroke}%
\pgfsetdash{}{0pt}%
\pgfpathmoveto{\pgfqpoint{1.351616in}{1.576717in}}%
\pgfpathlineto{\pgfqpoint{1.351622in}{1.571818in}}%
\pgfpathlineto{\pgfqpoint{1.351627in}{1.566886in}}%
\pgfpathlineto{\pgfqpoint{1.351633in}{1.561923in}}%
\pgfpathlineto{\pgfqpoint{1.351639in}{1.556932in}}%
\pgfpathlineto{\pgfqpoint{1.359755in}{1.556864in}}%
\pgfpathlineto{\pgfqpoint{1.367863in}{1.556673in}}%
\pgfpathlineto{\pgfqpoint{1.375956in}{1.556359in}}%
\pgfpathlineto{\pgfqpoint{1.384025in}{1.555922in}}%
\pgfpathlineto{\pgfqpoint{1.383613in}{1.560927in}}%
\pgfpathlineto{\pgfqpoint{1.383201in}{1.565903in}}%
\pgfpathlineto{\pgfqpoint{1.382788in}{1.570849in}}%
\pgfpathlineto{\pgfqpoint{1.382374in}{1.575761in}}%
\pgfpathlineto{\pgfqpoint{1.374710in}{1.576174in}}%
\pgfpathlineto{\pgfqpoint{1.367025in}{1.576472in}}%
\pgfpathlineto{\pgfqpoint{1.359324in}{1.576653in}}%
\pgfpathlineto{\pgfqpoint{1.351616in}{1.576717in}}%
\pgfpathclose%
\pgfusepath{fill}%
\end{pgfscope}%
\begin{pgfscope}%
\pgfpathrectangle{\pgfqpoint{0.329460in}{0.284240in}}{\pgfqpoint{1.989680in}{1.989680in}}%
\pgfusepath{clip}%
\pgfsetbuttcap%
\pgfsetroundjoin%
\definecolor{currentfill}{rgb}{0.231674,0.318106,0.544834}%
\pgfsetfillcolor{currentfill}%
\pgfsetlinewidth{0.000000pt}%
\definecolor{currentstroke}{rgb}{0.000000,0.000000,0.000000}%
\pgfsetstrokecolor{currentstroke}%
\pgfsetdash{}{0pt}%
\pgfpathmoveto{\pgfqpoint{0.990536in}{1.248674in}}%
\pgfpathlineto{\pgfqpoint{0.987338in}{1.242691in}}%
\pgfpathlineto{\pgfqpoint{0.984142in}{1.236766in}}%
\pgfpathlineto{\pgfqpoint{0.980946in}{1.230903in}}%
\pgfpathlineto{\pgfqpoint{0.977753in}{1.225103in}}%
\pgfpathlineto{\pgfqpoint{0.980475in}{1.231051in}}%
\pgfpathlineto{\pgfqpoint{0.983559in}{1.236947in}}%
\pgfpathlineto{\pgfqpoint{0.987001in}{1.242786in}}%
\pgfpathlineto{\pgfqpoint{0.990796in}{1.248561in}}%
\pgfpathlineto{\pgfqpoint{0.993867in}{1.254139in}}%
\pgfpathlineto{\pgfqpoint{0.996939in}{1.259782in}}%
\pgfpathlineto{\pgfqpoint{1.000013in}{1.265485in}}%
\pgfpathlineto{\pgfqpoint{1.003088in}{1.271248in}}%
\pgfpathlineto{\pgfqpoint{0.999434in}{1.265689in}}%
\pgfpathlineto{\pgfqpoint{0.996121in}{1.260071in}}%
\pgfpathlineto{\pgfqpoint{0.993153in}{1.254398in}}%
\pgfpathlineto{\pgfqpoint{0.990536in}{1.248674in}}%
\pgfpathclose%
\pgfusepath{fill}%
\end{pgfscope}%
\begin{pgfscope}%
\pgfpathrectangle{\pgfqpoint{0.329460in}{0.284240in}}{\pgfqpoint{1.989680in}{1.989680in}}%
\pgfusepath{clip}%
\pgfsetbuttcap%
\pgfsetroundjoin%
\definecolor{currentfill}{rgb}{0.263663,0.237631,0.518762}%
\pgfsetfillcolor{currentfill}%
\pgfsetlinewidth{0.000000pt}%
\definecolor{currentstroke}{rgb}{0.000000,0.000000,0.000000}%
\pgfsetstrokecolor{currentstroke}%
\pgfsetdash{}{0pt}%
\pgfpathmoveto{\pgfqpoint{1.734892in}{1.208097in}}%
\pgfpathlineto{\pgfqpoint{1.738058in}{1.202714in}}%
\pgfpathlineto{\pgfqpoint{1.741223in}{1.197414in}}%
\pgfpathlineto{\pgfqpoint{1.744388in}{1.192199in}}%
\pgfpathlineto{\pgfqpoint{1.747552in}{1.187072in}}%
\pgfpathlineto{\pgfqpoint{1.750443in}{1.180668in}}%
\pgfpathlineto{\pgfqpoint{1.752943in}{1.174214in}}%
\pgfpathlineto{\pgfqpoint{1.755050in}{1.167716in}}%
\pgfpathlineto{\pgfqpoint{1.756760in}{1.161181in}}%
\pgfpathlineto{\pgfqpoint{1.753509in}{1.166535in}}%
\pgfpathlineto{\pgfqpoint{1.750258in}{1.171979in}}%
\pgfpathlineto{\pgfqpoint{1.747007in}{1.177508in}}%
\pgfpathlineto{\pgfqpoint{1.743755in}{1.183119in}}%
\pgfpathlineto{\pgfqpoint{1.742113in}{1.189423in}}%
\pgfpathlineto{\pgfqpoint{1.740087in}{1.195692in}}%
\pgfpathlineto{\pgfqpoint{1.737678in}{1.201919in}}%
\pgfpathlineto{\pgfqpoint{1.734892in}{1.208097in}}%
\pgfpathclose%
\pgfusepath{fill}%
\end{pgfscope}%
\begin{pgfscope}%
\pgfpathrectangle{\pgfqpoint{0.329460in}{0.284240in}}{\pgfqpoint{1.989680in}{1.989680in}}%
\pgfusepath{clip}%
\pgfsetbuttcap%
\pgfsetroundjoin%
\definecolor{currentfill}{rgb}{0.147607,0.511733,0.557049}%
\pgfsetfillcolor{currentfill}%
\pgfsetlinewidth{0.000000pt}%
\definecolor{currentstroke}{rgb}{0.000000,0.000000,0.000000}%
\pgfsetstrokecolor{currentstroke}%
\pgfsetdash{}{0pt}%
\pgfpathmoveto{\pgfqpoint{1.098390in}{1.428024in}}%
\pgfpathlineto{\pgfqpoint{1.095648in}{1.422081in}}%
\pgfpathlineto{\pgfqpoint{1.092908in}{1.416143in}}%
\pgfpathlineto{\pgfqpoint{1.090172in}{1.410211in}}%
\pgfpathlineto{\pgfqpoint{1.087438in}{1.404288in}}%
\pgfpathlineto{\pgfqpoint{1.092986in}{1.408342in}}%
\pgfpathlineto{\pgfqpoint{1.098781in}{1.412305in}}%
\pgfpathlineto{\pgfqpoint{1.104816in}{1.416173in}}%
\pgfpathlineto{\pgfqpoint{1.111085in}{1.419944in}}%
\pgfpathlineto{\pgfqpoint{1.113567in}{1.425690in}}%
\pgfpathlineto{\pgfqpoint{1.116053in}{1.431445in}}%
\pgfpathlineto{\pgfqpoint{1.118540in}{1.437207in}}%
\pgfpathlineto{\pgfqpoint{1.121031in}{1.442974in}}%
\pgfpathlineto{\pgfqpoint{1.115028in}{1.439373in}}%
\pgfpathlineto{\pgfqpoint{1.109249in}{1.435678in}}%
\pgfpathlineto{\pgfqpoint{1.103701in}{1.431894in}}%
\pgfpathlineto{\pgfqpoint{1.098390in}{1.428024in}}%
\pgfpathclose%
\pgfusepath{fill}%
\end{pgfscope}%
\begin{pgfscope}%
\pgfpathrectangle{\pgfqpoint{0.329460in}{0.284240in}}{\pgfqpoint{1.989680in}{1.989680in}}%
\pgfusepath{clip}%
\pgfsetbuttcap%
\pgfsetroundjoin%
\definecolor{currentfill}{rgb}{0.279566,0.067836,0.391917}%
\pgfsetfillcolor{currentfill}%
\pgfsetlinewidth{0.000000pt}%
\definecolor{currentstroke}{rgb}{0.000000,0.000000,0.000000}%
\pgfsetstrokecolor{currentstroke}%
\pgfsetdash{}{0pt}%
\pgfpathmoveto{\pgfqpoint{0.890344in}{1.053389in}}%
\pgfpathlineto{\pgfqpoint{0.887035in}{1.049765in}}%
\pgfpathlineto{\pgfqpoint{0.883723in}{1.046290in}}%
\pgfpathlineto{\pgfqpoint{0.880409in}{1.042970in}}%
\pgfpathlineto{\pgfqpoint{0.877092in}{1.039807in}}%
\pgfpathlineto{\pgfqpoint{0.876862in}{1.047610in}}%
\pgfpathlineto{\pgfqpoint{0.877105in}{1.055404in}}%
\pgfpathlineto{\pgfqpoint{0.877820in}{1.063180in}}%
\pgfpathlineto{\pgfqpoint{0.879004in}{1.070932in}}%
\pgfpathlineto{\pgfqpoint{0.882292in}{1.073863in}}%
\pgfpathlineto{\pgfqpoint{0.885578in}{1.076951in}}%
\pgfpathlineto{\pgfqpoint{0.888860in}{1.080193in}}%
\pgfpathlineto{\pgfqpoint{0.892141in}{1.083584in}}%
\pgfpathlineto{\pgfqpoint{0.891005in}{1.076063in}}%
\pgfpathlineto{\pgfqpoint{0.890326in}{1.068519in}}%
\pgfpathlineto{\pgfqpoint{0.890105in}{1.060958in}}%
\pgfpathlineto{\pgfqpoint{0.890344in}{1.053389in}}%
\pgfpathclose%
\pgfusepath{fill}%
\end{pgfscope}%
\begin{pgfscope}%
\pgfpathrectangle{\pgfqpoint{0.329460in}{0.284240in}}{\pgfqpoint{1.989680in}{1.989680in}}%
\pgfusepath{clip}%
\pgfsetbuttcap%
\pgfsetroundjoin%
\definecolor{currentfill}{rgb}{0.163625,0.471133,0.558148}%
\pgfsetfillcolor{currentfill}%
\pgfsetlinewidth{0.000000pt}%
\definecolor{currentstroke}{rgb}{0.000000,0.000000,0.000000}%
\pgfsetstrokecolor{currentstroke}%
\pgfsetdash{}{0pt}%
\pgfpathmoveto{\pgfqpoint{1.610018in}{1.407896in}}%
\pgfpathlineto{\pgfqpoint{1.612697in}{1.402026in}}%
\pgfpathlineto{\pgfqpoint{1.615373in}{1.396170in}}%
\pgfpathlineto{\pgfqpoint{1.618046in}{1.390331in}}%
\pgfpathlineto{\pgfqpoint{1.620717in}{1.384512in}}%
\pgfpathlineto{\pgfqpoint{1.626474in}{1.380264in}}%
\pgfpathlineto{\pgfqpoint{1.631967in}{1.375926in}}%
\pgfpathlineto{\pgfqpoint{1.637192in}{1.371500in}}%
\pgfpathlineto{\pgfqpoint{1.642143in}{1.366992in}}%
\pgfpathlineto{\pgfqpoint{1.639252in}{1.373002in}}%
\pgfpathlineto{\pgfqpoint{1.636358in}{1.379031in}}%
\pgfpathlineto{\pgfqpoint{1.633461in}{1.385077in}}%
\pgfpathlineto{\pgfqpoint{1.630562in}{1.391137in}}%
\pgfpathlineto{\pgfqpoint{1.625816in}{1.395449in}}%
\pgfpathlineto{\pgfqpoint{1.620806in}{1.399682in}}%
\pgfpathlineto{\pgfqpoint{1.615539in}{1.403832in}}%
\pgfpathlineto{\pgfqpoint{1.610018in}{1.407896in}}%
\pgfpathclose%
\pgfusepath{fill}%
\end{pgfscope}%
\begin{pgfscope}%
\pgfpathrectangle{\pgfqpoint{0.329460in}{0.284240in}}{\pgfqpoint{1.989680in}{1.989680in}}%
\pgfusepath{clip}%
\pgfsetbuttcap%
\pgfsetroundjoin%
\definecolor{currentfill}{rgb}{0.122606,0.585371,0.546557}%
\pgfsetfillcolor{currentfill}%
\pgfsetlinewidth{0.000000pt}%
\definecolor{currentstroke}{rgb}{0.000000,0.000000,0.000000}%
\pgfsetstrokecolor{currentstroke}%
\pgfsetdash{}{0pt}%
\pgfpathmoveto{\pgfqpoint{1.164884in}{1.501112in}}%
\pgfpathlineto{\pgfqpoint{1.162660in}{1.495557in}}%
\pgfpathlineto{\pgfqpoint{1.160439in}{1.489986in}}%
\pgfpathlineto{\pgfqpoint{1.158221in}{1.484403in}}%
\pgfpathlineto{\pgfqpoint{1.156005in}{1.478810in}}%
\pgfpathlineto{\pgfqpoint{1.162729in}{1.481754in}}%
\pgfpathlineto{\pgfqpoint{1.169631in}{1.484591in}}%
\pgfpathlineto{\pgfqpoint{1.176705in}{1.487320in}}%
\pgfpathlineto{\pgfqpoint{1.183943in}{1.489939in}}%
\pgfpathlineto{\pgfqpoint{1.185838in}{1.495396in}}%
\pgfpathlineto{\pgfqpoint{1.187735in}{1.500842in}}%
\pgfpathlineto{\pgfqpoint{1.189635in}{1.506277in}}%
\pgfpathlineto{\pgfqpoint{1.191537in}{1.511696in}}%
\pgfpathlineto{\pgfqpoint{1.184631in}{1.509206in}}%
\pgfpathlineto{\pgfqpoint{1.177882in}{1.506610in}}%
\pgfpathlineto{\pgfqpoint{1.171298in}{1.503911in}}%
\pgfpathlineto{\pgfqpoint{1.164884in}{1.501112in}}%
\pgfpathclose%
\pgfusepath{fill}%
\end{pgfscope}%
\begin{pgfscope}%
\pgfpathrectangle{\pgfqpoint{0.329460in}{0.284240in}}{\pgfqpoint{1.989680in}{1.989680in}}%
\pgfusepath{clip}%
\pgfsetbuttcap%
\pgfsetroundjoin%
\definecolor{currentfill}{rgb}{0.134692,0.658636,0.517649}%
\pgfsetfillcolor{currentfill}%
\pgfsetlinewidth{0.000000pt}%
\definecolor{currentstroke}{rgb}{0.000000,0.000000,0.000000}%
\pgfsetstrokecolor{currentstroke}%
\pgfsetdash{}{0pt}%
\pgfpathmoveto{\pgfqpoint{1.290542in}{1.573061in}}%
\pgfpathlineto{\pgfqpoint{1.289737in}{1.568110in}}%
\pgfpathlineto{\pgfqpoint{1.288933in}{1.563127in}}%
\pgfpathlineto{\pgfqpoint{1.288131in}{1.558112in}}%
\pgfpathlineto{\pgfqpoint{1.287329in}{1.553069in}}%
\pgfpathlineto{\pgfqpoint{1.295237in}{1.553978in}}%
\pgfpathlineto{\pgfqpoint{1.303197in}{1.554765in}}%
\pgfpathlineto{\pgfqpoint{1.311202in}{1.555432in}}%
\pgfpathlineto{\pgfqpoint{1.319245in}{1.555977in}}%
\pgfpathlineto{\pgfqpoint{1.319646in}{1.560981in}}%
\pgfpathlineto{\pgfqpoint{1.320047in}{1.565956in}}%
\pgfpathlineto{\pgfqpoint{1.320449in}{1.570901in}}%
\pgfpathlineto{\pgfqpoint{1.320851in}{1.575813in}}%
\pgfpathlineto{\pgfqpoint{1.313213in}{1.575297in}}%
\pgfpathlineto{\pgfqpoint{1.305611in}{1.574666in}}%
\pgfpathlineto{\pgfqpoint{1.298051in}{1.573921in}}%
\pgfpathlineto{\pgfqpoint{1.290542in}{1.573061in}}%
\pgfpathclose%
\pgfusepath{fill}%
\end{pgfscope}%
\begin{pgfscope}%
\pgfpathrectangle{\pgfqpoint{0.329460in}{0.284240in}}{\pgfqpoint{1.989680in}{1.989680in}}%
\pgfusepath{clip}%
\pgfsetbuttcap%
\pgfsetroundjoin%
\definecolor{currentfill}{rgb}{0.134692,0.658636,0.517649}%
\pgfsetfillcolor{currentfill}%
\pgfsetlinewidth{0.000000pt}%
\definecolor{currentstroke}{rgb}{0.000000,0.000000,0.000000}%
\pgfsetstrokecolor{currentstroke}%
\pgfsetdash{}{0pt}%
\pgfpathmoveto{\pgfqpoint{1.382374in}{1.575761in}}%
\pgfpathlineto{\pgfqpoint{1.382788in}{1.570849in}}%
\pgfpathlineto{\pgfqpoint{1.383201in}{1.565903in}}%
\pgfpathlineto{\pgfqpoint{1.383613in}{1.560927in}}%
\pgfpathlineto{\pgfqpoint{1.384025in}{1.555922in}}%
\pgfpathlineto{\pgfqpoint{1.392064in}{1.555364in}}%
\pgfpathlineto{\pgfqpoint{1.400065in}{1.554684in}}%
\pgfpathlineto{\pgfqpoint{1.408020in}{1.553883in}}%
\pgfpathlineto{\pgfqpoint{1.415921in}{1.552961in}}%
\pgfpathlineto{\pgfqpoint{1.415109in}{1.558005in}}%
\pgfpathlineto{\pgfqpoint{1.414295in}{1.563021in}}%
\pgfpathlineto{\pgfqpoint{1.413480in}{1.568006in}}%
\pgfpathlineto{\pgfqpoint{1.412664in}{1.572959in}}%
\pgfpathlineto{\pgfqpoint{1.405161in}{1.573831in}}%
\pgfpathlineto{\pgfqpoint{1.397606in}{1.574589in}}%
\pgfpathlineto{\pgfqpoint{1.390008in}{1.575233in}}%
\pgfpathlineto{\pgfqpoint{1.382374in}{1.575761in}}%
\pgfpathclose%
\pgfusepath{fill}%
\end{pgfscope}%
\begin{pgfscope}%
\pgfpathrectangle{\pgfqpoint{0.329460in}{0.284240in}}{\pgfqpoint{1.989680in}{1.989680in}}%
\pgfusepath{clip}%
\pgfsetbuttcap%
\pgfsetroundjoin%
\definecolor{currentfill}{rgb}{0.120081,0.622161,0.534946}%
\pgfsetfillcolor{currentfill}%
\pgfsetlinewidth{0.000000pt}%
\definecolor{currentstroke}{rgb}{0.000000,0.000000,0.000000}%
\pgfsetstrokecolor{currentstroke}%
\pgfsetdash{}{0pt}%
\pgfpathmoveto{\pgfqpoint{1.469125in}{1.543224in}}%
\pgfpathlineto{\pgfqpoint{1.470605in}{1.538023in}}%
\pgfpathlineto{\pgfqpoint{1.472084in}{1.532798in}}%
\pgfpathlineto{\pgfqpoint{1.473561in}{1.527551in}}%
\pgfpathlineto{\pgfqpoint{1.475036in}{1.522284in}}%
\pgfpathlineto{\pgfqpoint{1.482612in}{1.520339in}}%
\pgfpathlineto{\pgfqpoint{1.490066in}{1.518279in}}%
\pgfpathlineto{\pgfqpoint{1.497388in}{1.516105in}}%
\pgfpathlineto{\pgfqpoint{1.504573in}{1.513819in}}%
\pgfpathlineto{\pgfqpoint{1.502744in}{1.519194in}}%
\pgfpathlineto{\pgfqpoint{1.500912in}{1.524549in}}%
\pgfpathlineto{\pgfqpoint{1.499078in}{1.529883in}}%
\pgfpathlineto{\pgfqpoint{1.497242in}{1.535192in}}%
\pgfpathlineto{\pgfqpoint{1.490403in}{1.537360in}}%
\pgfpathlineto{\pgfqpoint{1.483433in}{1.539423in}}%
\pgfpathlineto{\pgfqpoint{1.476338in}{1.541378in}}%
\pgfpathlineto{\pgfqpoint{1.469125in}{1.543224in}}%
\pgfpathclose%
\pgfusepath{fill}%
\end{pgfscope}%
\begin{pgfscope}%
\pgfpathrectangle{\pgfqpoint{0.329460in}{0.284240in}}{\pgfqpoint{1.989680in}{1.989680in}}%
\pgfusepath{clip}%
\pgfsetbuttcap%
\pgfsetroundjoin%
\definecolor{currentfill}{rgb}{0.282884,0.135920,0.453427}%
\pgfsetfillcolor{currentfill}%
\pgfsetlinewidth{0.000000pt}%
\definecolor{currentstroke}{rgb}{0.000000,0.000000,0.000000}%
\pgfsetstrokecolor{currentstroke}%
\pgfsetdash{}{0pt}%
\pgfpathmoveto{\pgfqpoint{1.943425in}{1.112470in}}%
\pgfpathlineto{\pgfqpoint{1.946961in}{1.118314in}}%
\pgfpathlineto{\pgfqpoint{1.950508in}{1.124486in}}%
\pgfpathlineto{\pgfqpoint{1.954069in}{1.130989in}}%
\pgfpathlineto{\pgfqpoint{1.957643in}{1.137829in}}%
\pgfpathlineto{\pgfqpoint{1.959913in}{1.127969in}}%
\pgfpathlineto{\pgfqpoint{1.961584in}{1.118062in}}%
\pgfpathlineto{\pgfqpoint{1.962651in}{1.108117in}}%
\pgfpathlineto{\pgfqpoint{1.963111in}{1.098144in}}%
\pgfpathlineto{\pgfqpoint{1.959483in}{1.091496in}}%
\pgfpathlineto{\pgfqpoint{1.955868in}{1.085188in}}%
\pgfpathlineto{\pgfqpoint{1.952267in}{1.079213in}}%
\pgfpathlineto{\pgfqpoint{1.948678in}{1.073566in}}%
\pgfpathlineto{\pgfqpoint{1.948251in}{1.083342in}}%
\pgfpathlineto{\pgfqpoint{1.947231in}{1.093090in}}%
\pgfpathlineto{\pgfqpoint{1.945621in}{1.102803in}}%
\pgfpathlineto{\pgfqpoint{1.943425in}{1.112470in}}%
\pgfpathclose%
\pgfusepath{fill}%
\end{pgfscope}%
\begin{pgfscope}%
\pgfpathrectangle{\pgfqpoint{0.329460in}{0.284240in}}{\pgfqpoint{1.989680in}{1.989680in}}%
\pgfusepath{clip}%
\pgfsetbuttcap%
\pgfsetroundjoin%
\definecolor{currentfill}{rgb}{0.163625,0.471133,0.558148}%
\pgfsetfillcolor{currentfill}%
\pgfsetlinewidth{0.000000pt}%
\definecolor{currentstroke}{rgb}{0.000000,0.000000,0.000000}%
\pgfsetstrokecolor{currentstroke}%
\pgfsetdash{}{0pt}%
\pgfpathmoveto{\pgfqpoint{1.067820in}{1.387241in}}%
\pgfpathlineto{\pgfqpoint{1.064878in}{1.381137in}}%
\pgfpathlineto{\pgfqpoint{1.061937in}{1.375046in}}%
\pgfpathlineto{\pgfqpoint{1.059000in}{1.368973in}}%
\pgfpathlineto{\pgfqpoint{1.056065in}{1.362919in}}%
\pgfpathlineto{\pgfqpoint{1.060768in}{1.367497in}}%
\pgfpathlineto{\pgfqpoint{1.065750in}{1.371996in}}%
\pgfpathlineto{\pgfqpoint{1.071005in}{1.376412in}}%
\pgfpathlineto{\pgfqpoint{1.076528in}{1.380741in}}%
\pgfpathlineto{\pgfqpoint{1.079252in}{1.386601in}}%
\pgfpathlineto{\pgfqpoint{1.081978in}{1.392480in}}%
\pgfpathlineto{\pgfqpoint{1.084706in}{1.398377in}}%
\pgfpathlineto{\pgfqpoint{1.087438in}{1.404288in}}%
\pgfpathlineto{\pgfqpoint{1.082141in}{1.400147in}}%
\pgfpathlineto{\pgfqpoint{1.077103in}{1.395923in}}%
\pgfpathlineto{\pgfqpoint{1.072327in}{1.391620in}}%
\pgfpathlineto{\pgfqpoint{1.067820in}{1.387241in}}%
\pgfpathclose%
\pgfusepath{fill}%
\end{pgfscope}%
\begin{pgfscope}%
\pgfpathrectangle{\pgfqpoint{0.329460in}{0.284240in}}{\pgfqpoint{1.989680in}{1.989680in}}%
\pgfusepath{clip}%
\pgfsetbuttcap%
\pgfsetroundjoin%
\definecolor{currentfill}{rgb}{0.283072,0.130895,0.449241}%
\pgfsetfillcolor{currentfill}%
\pgfsetlinewidth{0.000000pt}%
\definecolor{currentstroke}{rgb}{0.000000,0.000000,0.000000}%
\pgfsetstrokecolor{currentstroke}%
\pgfsetdash{}{0pt}%
\pgfpathmoveto{\pgfqpoint{1.782772in}{1.121926in}}%
\pgfpathlineto{\pgfqpoint{1.786026in}{1.117516in}}%
\pgfpathlineto{\pgfqpoint{1.789281in}{1.113227in}}%
\pgfpathlineto{\pgfqpoint{1.792538in}{1.109064in}}%
\pgfpathlineto{\pgfqpoint{1.795796in}{1.105029in}}%
\pgfpathlineto{\pgfqpoint{1.797276in}{1.097766in}}%
\pgfpathlineto{\pgfqpoint{1.798315in}{1.090475in}}%
\pgfpathlineto{\pgfqpoint{1.798911in}{1.083161in}}%
\pgfpathlineto{\pgfqpoint{1.799061in}{1.075833in}}%
\pgfpathlineto{\pgfqpoint{1.795764in}{1.080101in}}%
\pgfpathlineto{\pgfqpoint{1.792469in}{1.084498in}}%
\pgfpathlineto{\pgfqpoint{1.789176in}{1.089020in}}%
\pgfpathlineto{\pgfqpoint{1.785883in}{1.093664in}}%
\pgfpathlineto{\pgfqpoint{1.785752in}{1.100757in}}%
\pgfpathlineto{\pgfqpoint{1.785189in}{1.107836in}}%
\pgfpathlineto{\pgfqpoint{1.784194in}{1.114895in}}%
\pgfpathlineto{\pgfqpoint{1.782772in}{1.121926in}}%
\pgfpathclose%
\pgfusepath{fill}%
\end{pgfscope}%
\begin{pgfscope}%
\pgfpathrectangle{\pgfqpoint{0.329460in}{0.284240in}}{\pgfqpoint{1.989680in}{1.989680in}}%
\pgfusepath{clip}%
\pgfsetbuttcap%
\pgfsetroundjoin%
\definecolor{currentfill}{rgb}{0.120081,0.622161,0.534946}%
\pgfsetfillcolor{currentfill}%
\pgfsetlinewidth{0.000000pt}%
\definecolor{currentstroke}{rgb}{0.000000,0.000000,0.000000}%
\pgfsetstrokecolor{currentstroke}%
\pgfsetdash{}{0pt}%
\pgfpathmoveto{\pgfqpoint{1.199170in}{1.533177in}}%
\pgfpathlineto{\pgfqpoint{1.197258in}{1.527841in}}%
\pgfpathlineto{\pgfqpoint{1.195348in}{1.522480in}}%
\pgfpathlineto{\pgfqpoint{1.193442in}{1.517098in}}%
\pgfpathlineto{\pgfqpoint{1.191537in}{1.511696in}}%
\pgfpathlineto{\pgfqpoint{1.198594in}{1.514079in}}%
\pgfpathlineto{\pgfqpoint{1.205794in}{1.516352in}}%
\pgfpathlineto{\pgfqpoint{1.213131in}{1.518513in}}%
\pgfpathlineto{\pgfqpoint{1.220599in}{1.520561in}}%
\pgfpathlineto{\pgfqpoint{1.222154in}{1.525849in}}%
\pgfpathlineto{\pgfqpoint{1.223712in}{1.531118in}}%
\pgfpathlineto{\pgfqpoint{1.225271in}{1.536366in}}%
\pgfpathlineto{\pgfqpoint{1.226833in}{1.541589in}}%
\pgfpathlineto{\pgfqpoint{1.219725in}{1.539646in}}%
\pgfpathlineto{\pgfqpoint{1.212740in}{1.537595in}}%
\pgfpathlineto{\pgfqpoint{1.205886in}{1.535438in}}%
\pgfpathlineto{\pgfqpoint{1.199170in}{1.533177in}}%
\pgfpathclose%
\pgfusepath{fill}%
\end{pgfscope}%
\begin{pgfscope}%
\pgfpathrectangle{\pgfqpoint{0.329460in}{0.284240in}}{\pgfqpoint{1.989680in}{1.989680in}}%
\pgfusepath{clip}%
\pgfsetbuttcap%
\pgfsetroundjoin%
\definecolor{currentfill}{rgb}{0.282327,0.094955,0.417331}%
\pgfsetfillcolor{currentfill}%
\pgfsetlinewidth{0.000000pt}%
\definecolor{currentstroke}{rgb}{0.000000,0.000000,0.000000}%
\pgfsetstrokecolor{currentstroke}%
\pgfsetdash{}{0pt}%
\pgfpathmoveto{\pgfqpoint{0.768065in}{1.045628in}}%
\pgfpathlineto{\pgfqpoint{0.764521in}{1.049971in}}%
\pgfpathlineto{\pgfqpoint{0.760965in}{1.054621in}}%
\pgfpathlineto{\pgfqpoint{0.757397in}{1.059583in}}%
\pgfpathlineto{\pgfqpoint{0.753818in}{1.064862in}}%
\pgfpathlineto{\pgfqpoint{0.753715in}{1.074653in}}%
\pgfpathlineto{\pgfqpoint{0.754208in}{1.084426in}}%
\pgfpathlineto{\pgfqpoint{0.755294in}{1.094171in}}%
\pgfpathlineto{\pgfqpoint{0.756970in}{1.103879in}}%
\pgfpathlineto{\pgfqpoint{0.760509in}{1.098399in}}%
\pgfpathlineto{\pgfqpoint{0.764037in}{1.093235in}}%
\pgfpathlineto{\pgfqpoint{0.767553in}{1.088382in}}%
\pgfpathlineto{\pgfqpoint{0.771058in}{1.083833in}}%
\pgfpathlineto{\pgfqpoint{0.769443in}{1.074326in}}%
\pgfpathlineto{\pgfqpoint{0.768404in}{1.064783in}}%
\pgfpathlineto{\pgfqpoint{0.767943in}{1.055214in}}%
\pgfpathlineto{\pgfqpoint{0.768065in}{1.045628in}}%
\pgfpathclose%
\pgfusepath{fill}%
\end{pgfscope}%
\begin{pgfscope}%
\pgfpathrectangle{\pgfqpoint{0.329460in}{0.284240in}}{\pgfqpoint{1.989680in}{1.989680in}}%
\pgfusepath{clip}%
\pgfsetbuttcap%
\pgfsetroundjoin%
\definecolor{currentfill}{rgb}{0.282327,0.094955,0.417331}%
\pgfsetfillcolor{currentfill}%
\pgfsetlinewidth{0.000000pt}%
\definecolor{currentstroke}{rgb}{0.000000,0.000000,0.000000}%
\pgfsetstrokecolor{currentstroke}%
\pgfsetdash{}{0pt}%
\pgfpathmoveto{\pgfqpoint{0.903556in}{1.069314in}}%
\pgfpathlineto{\pgfqpoint{0.900256in}{1.065125in}}%
\pgfpathlineto{\pgfqpoint{0.896954in}{1.061073in}}%
\pgfpathlineto{\pgfqpoint{0.893650in}{1.057159in}}%
\pgfpathlineto{\pgfqpoint{0.890344in}{1.053389in}}%
\pgfpathlineto{\pgfqpoint{0.890105in}{1.060958in}}%
\pgfpathlineto{\pgfqpoint{0.890326in}{1.068519in}}%
\pgfpathlineto{\pgfqpoint{0.891005in}{1.076063in}}%
\pgfpathlineto{\pgfqpoint{0.892141in}{1.083584in}}%
\pgfpathlineto{\pgfqpoint{0.895419in}{1.087121in}}%
\pgfpathlineto{\pgfqpoint{0.898695in}{1.090801in}}%
\pgfpathlineto{\pgfqpoint{0.901970in}{1.094620in}}%
\pgfpathlineto{\pgfqpoint{0.905242in}{1.098575in}}%
\pgfpathlineto{\pgfqpoint{0.904154in}{1.091286in}}%
\pgfpathlineto{\pgfqpoint{0.903509in}{1.083975in}}%
\pgfpathlineto{\pgfqpoint{0.903309in}{1.076648in}}%
\pgfpathlineto{\pgfqpoint{0.903556in}{1.069314in}}%
\pgfpathclose%
\pgfusepath{fill}%
\end{pgfscope}%
\begin{pgfscope}%
\pgfpathrectangle{\pgfqpoint{0.329460in}{0.284240in}}{\pgfqpoint{1.989680in}{1.989680in}}%
\pgfusepath{clip}%
\pgfsetbuttcap%
\pgfsetroundjoin%
\definecolor{currentfill}{rgb}{0.212395,0.359683,0.551710}%
\pgfsetfillcolor{currentfill}%
\pgfsetlinewidth{0.000000pt}%
\definecolor{currentstroke}{rgb}{0.000000,0.000000,0.000000}%
\pgfsetstrokecolor{currentstroke}%
\pgfsetdash{}{0pt}%
\pgfpathmoveto{\pgfqpoint{1.683571in}{1.299516in}}%
\pgfpathlineto{\pgfqpoint{1.686620in}{1.293597in}}%
\pgfpathlineto{\pgfqpoint{1.689667in}{1.287725in}}%
\pgfpathlineto{\pgfqpoint{1.692712in}{1.281903in}}%
\pgfpathlineto{\pgfqpoint{1.695755in}{1.276133in}}%
\pgfpathlineto{\pgfqpoint{1.699710in}{1.270633in}}%
\pgfpathlineto{\pgfqpoint{1.703326in}{1.265068in}}%
\pgfpathlineto{\pgfqpoint{1.706601in}{1.259443in}}%
\pgfpathlineto{\pgfqpoint{1.709530in}{1.253764in}}%
\pgfpathlineto{\pgfqpoint{1.706354in}{1.259753in}}%
\pgfpathlineto{\pgfqpoint{1.703176in}{1.265794in}}%
\pgfpathlineto{\pgfqpoint{1.699996in}{1.271884in}}%
\pgfpathlineto{\pgfqpoint{1.696814in}{1.278022in}}%
\pgfpathlineto{\pgfqpoint{1.694001in}{1.283478in}}%
\pgfpathlineto{\pgfqpoint{1.690853in}{1.288883in}}%
\pgfpathlineto{\pgfqpoint{1.687375in}{1.294231in}}%
\pgfpathlineto{\pgfqpoint{1.683571in}{1.299516in}}%
\pgfpathclose%
\pgfusepath{fill}%
\end{pgfscope}%
\begin{pgfscope}%
\pgfpathrectangle{\pgfqpoint{0.329460in}{0.284240in}}{\pgfqpoint{1.989680in}{1.989680in}}%
\pgfusepath{clip}%
\pgfsetbuttcap%
\pgfsetroundjoin%
\definecolor{currentfill}{rgb}{0.134692,0.658636,0.517649}%
\pgfsetfillcolor{currentfill}%
\pgfsetlinewidth{0.000000pt}%
\definecolor{currentstroke}{rgb}{0.000000,0.000000,0.000000}%
\pgfsetstrokecolor{currentstroke}%
\pgfsetdash{}{0pt}%
\pgfpathmoveto{\pgfqpoint{1.412664in}{1.572959in}}%
\pgfpathlineto{\pgfqpoint{1.413480in}{1.568006in}}%
\pgfpathlineto{\pgfqpoint{1.414295in}{1.563021in}}%
\pgfpathlineto{\pgfqpoint{1.415109in}{1.558005in}}%
\pgfpathlineto{\pgfqpoint{1.415921in}{1.552961in}}%
\pgfpathlineto{\pgfqpoint{1.423762in}{1.551920in}}%
\pgfpathlineto{\pgfqpoint{1.431535in}{1.550761in}}%
\pgfpathlineto{\pgfqpoint{1.439233in}{1.549484in}}%
\pgfpathlineto{\pgfqpoint{1.438127in}{1.554575in}}%
\pgfpathlineto{\pgfqpoint{1.437020in}{1.559637in}}%
\pgfpathlineto{\pgfqpoint{1.435911in}{1.564669in}}%
\pgfpathlineto{\pgfqpoint{1.434801in}{1.569667in}}%
\pgfpathlineto{\pgfqpoint{1.427491in}{1.570876in}}%
\pgfpathlineto{\pgfqpoint{1.420110in}{1.571973in}}%
\pgfpathlineto{\pgfqpoint{1.412664in}{1.572959in}}%
\pgfpathclose%
\pgfusepath{fill}%
\end{pgfscope}%
\begin{pgfscope}%
\pgfpathrectangle{\pgfqpoint{0.329460in}{0.284240in}}{\pgfqpoint{1.989680in}{1.989680in}}%
\pgfusepath{clip}%
\pgfsetbuttcap%
\pgfsetroundjoin%
\definecolor{currentfill}{rgb}{0.263663,0.237631,0.518762}%
\pgfsetfillcolor{currentfill}%
\pgfsetlinewidth{0.000000pt}%
\definecolor{currentstroke}{rgb}{0.000000,0.000000,0.000000}%
\pgfsetstrokecolor{currentstroke}%
\pgfsetdash{}{0pt}%
\pgfpathmoveto{\pgfqpoint{0.957485in}{1.177490in}}%
\pgfpathlineto{\pgfqpoint{0.954221in}{1.171827in}}%
\pgfpathlineto{\pgfqpoint{0.950958in}{1.166247in}}%
\pgfpathlineto{\pgfqpoint{0.947694in}{1.160752in}}%
\pgfpathlineto{\pgfqpoint{0.944431in}{1.155345in}}%
\pgfpathlineto{\pgfqpoint{0.945785in}{1.161909in}}%
\pgfpathlineto{\pgfqpoint{0.947539in}{1.168440in}}%
\pgfpathlineto{\pgfqpoint{0.949690in}{1.174934in}}%
\pgfpathlineto{\pgfqpoint{0.952235in}{1.181382in}}%
\pgfpathlineto{\pgfqpoint{0.955422in}{1.186559in}}%
\pgfpathlineto{\pgfqpoint{0.958610in}{1.191824in}}%
\pgfpathlineto{\pgfqpoint{0.961799in}{1.197174in}}%
\pgfpathlineto{\pgfqpoint{0.964988in}{1.202608in}}%
\pgfpathlineto{\pgfqpoint{0.962537in}{1.196386in}}%
\pgfpathlineto{\pgfqpoint{0.960468in}{1.190122in}}%
\pgfpathlineto{\pgfqpoint{0.958783in}{1.183821in}}%
\pgfpathlineto{\pgfqpoint{0.957485in}{1.177490in}}%
\pgfpathclose%
\pgfusepath{fill}%
\end{pgfscope}%
\begin{pgfscope}%
\pgfpathrectangle{\pgfqpoint{0.329460in}{0.284240in}}{\pgfqpoint{1.989680in}{1.989680in}}%
\pgfusepath{clip}%
\pgfsetbuttcap%
\pgfsetroundjoin%
\definecolor{currentfill}{rgb}{0.233603,0.313828,0.543914}%
\pgfsetfillcolor{currentfill}%
\pgfsetlinewidth{0.000000pt}%
\definecolor{currentstroke}{rgb}{0.000000,0.000000,0.000000}%
\pgfsetstrokecolor{currentstroke}%
\pgfsetdash{}{0pt}%
\pgfpathmoveto{\pgfqpoint{0.713430in}{1.196323in}}%
\pgfpathlineto{\pgfqpoint{0.709702in}{1.206431in}}%
\pgfpathlineto{\pgfqpoint{0.705955in}{1.216937in}}%
\pgfpathlineto{\pgfqpoint{0.702191in}{1.227850in}}%
\pgfpathlineto{\pgfqpoint{0.698409in}{1.239174in}}%
\pgfpathlineto{\pgfqpoint{0.700988in}{1.249562in}}%
\pgfpathlineto{\pgfqpoint{0.704203in}{1.259887in}}%
\pgfpathlineto{\pgfqpoint{0.708047in}{1.270138in}}%
\pgfpathlineto{\pgfqpoint{0.712514in}{1.280307in}}%
\pgfpathlineto{\pgfqpoint{0.716190in}{1.268820in}}%
\pgfpathlineto{\pgfqpoint{0.719849in}{1.257742in}}%
\pgfpathlineto{\pgfqpoint{0.723491in}{1.247068in}}%
\pgfpathlineto{\pgfqpoint{0.727117in}{1.236790in}}%
\pgfpathlineto{\pgfqpoint{0.722776in}{1.226785in}}%
\pgfpathlineto{\pgfqpoint{0.719043in}{1.216699in}}%
\pgfpathlineto{\pgfqpoint{0.715926in}{1.206542in}}%
\pgfpathlineto{\pgfqpoint{0.713430in}{1.196323in}}%
\pgfpathclose%
\pgfusepath{fill}%
\end{pgfscope}%
\begin{pgfscope}%
\pgfpathrectangle{\pgfqpoint{0.329460in}{0.284240in}}{\pgfqpoint{1.989680in}{1.989680in}}%
\pgfusepath{clip}%
\pgfsetbuttcap%
\pgfsetroundjoin%
\definecolor{currentfill}{rgb}{0.134692,0.658636,0.517649}%
\pgfsetfillcolor{currentfill}%
\pgfsetlinewidth{0.000000pt}%
\definecolor{currentstroke}{rgb}{0.000000,0.000000,0.000000}%
\pgfsetstrokecolor{currentstroke}%
\pgfsetdash{}{0pt}%
\pgfpathmoveto{\pgfqpoint{1.261143in}{1.568500in}}%
\pgfpathlineto{\pgfqpoint{1.259947in}{1.563485in}}%
\pgfpathlineto{\pgfqpoint{1.258753in}{1.558437in}}%
\pgfpathlineto{\pgfqpoint{1.257560in}{1.553358in}}%
\pgfpathlineto{\pgfqpoint{1.256369in}{1.548251in}}%
\pgfpathlineto{\pgfqpoint{1.263994in}{1.549631in}}%
\pgfpathlineto{\pgfqpoint{1.271700in}{1.550895in}}%
\pgfpathlineto{\pgfqpoint{1.279481in}{1.552042in}}%
\pgfpathlineto{\pgfqpoint{1.287329in}{1.553069in}}%
\pgfpathlineto{\pgfqpoint{1.288131in}{1.558112in}}%
\pgfpathlineto{\pgfqpoint{1.288933in}{1.563127in}}%
\pgfpathlineto{\pgfqpoint{1.289737in}{1.568110in}}%
\pgfpathlineto{\pgfqpoint{1.290542in}{1.573061in}}%
\pgfpathlineto{\pgfqpoint{1.283089in}{1.572088in}}%
\pgfpathlineto{\pgfqpoint{1.275700in}{1.571003in}}%
\pgfpathlineto{\pgfqpoint{1.268383in}{1.569807in}}%
\pgfpathlineto{\pgfqpoint{1.261143in}{1.568500in}}%
\pgfpathclose%
\pgfusepath{fill}%
\end{pgfscope}%
\begin{pgfscope}%
\pgfpathrectangle{\pgfqpoint{0.329460in}{0.284240in}}{\pgfqpoint{1.989680in}{1.989680in}}%
\pgfusepath{clip}%
\pgfsetbuttcap%
\pgfsetroundjoin%
\definecolor{currentfill}{rgb}{0.179019,0.433756,0.557430}%
\pgfsetfillcolor{currentfill}%
\pgfsetlinewidth{0.000000pt}%
\definecolor{currentstroke}{rgb}{0.000000,0.000000,0.000000}%
\pgfsetstrokecolor{currentstroke}%
\pgfsetdash{}{0pt}%
\pgfpathmoveto{\pgfqpoint{1.642143in}{1.366992in}}%
\pgfpathlineto{\pgfqpoint{1.645032in}{1.361004in}}%
\pgfpathlineto{\pgfqpoint{1.647919in}{1.355042in}}%
\pgfpathlineto{\pgfqpoint{1.650803in}{1.349107in}}%
\pgfpathlineto{\pgfqpoint{1.653684in}{1.343202in}}%
\pgfpathlineto{\pgfqpoint{1.658550in}{1.338414in}}%
\pgfpathlineto{\pgfqpoint{1.663121in}{1.333549in}}%
\pgfpathlineto{\pgfqpoint{1.667390in}{1.328611in}}%
\pgfpathlineto{\pgfqpoint{1.671355in}{1.323603in}}%
\pgfpathlineto{\pgfqpoint{1.668295in}{1.329714in}}%
\pgfpathlineto{\pgfqpoint{1.665233in}{1.335855in}}%
\pgfpathlineto{\pgfqpoint{1.662169in}{1.342023in}}%
\pgfpathlineto{\pgfqpoint{1.659102in}{1.348216in}}%
\pgfpathlineto{\pgfqpoint{1.655299in}{1.353013in}}%
\pgfpathlineto{\pgfqpoint{1.651202in}{1.357744in}}%
\pgfpathlineto{\pgfqpoint{1.646815in}{1.362405in}}%
\pgfpathlineto{\pgfqpoint{1.642143in}{1.366992in}}%
\pgfpathclose%
\pgfusepath{fill}%
\end{pgfscope}%
\begin{pgfscope}%
\pgfpathrectangle{\pgfqpoint{0.329460in}{0.284240in}}{\pgfqpoint{1.989680in}{1.989680in}}%
\pgfusepath{clip}%
\pgfsetbuttcap%
\pgfsetroundjoin%
\definecolor{currentfill}{rgb}{0.248629,0.278775,0.534556}%
\pgfsetfillcolor{currentfill}%
\pgfsetlinewidth{0.000000pt}%
\definecolor{currentstroke}{rgb}{0.000000,0.000000,0.000000}%
\pgfsetstrokecolor{currentstroke}%
\pgfsetdash{}{0pt}%
\pgfpathmoveto{\pgfqpoint{1.722220in}{1.230393in}}%
\pgfpathlineto{\pgfqpoint{1.725390in}{1.224711in}}%
\pgfpathlineto{\pgfqpoint{1.728558in}{1.219099in}}%
\pgfpathlineto{\pgfqpoint{1.731726in}{1.213560in}}%
\pgfpathlineto{\pgfqpoint{1.734892in}{1.208097in}}%
\pgfpathlineto{\pgfqpoint{1.737678in}{1.201919in}}%
\pgfpathlineto{\pgfqpoint{1.740087in}{1.195692in}}%
\pgfpathlineto{\pgfqpoint{1.742113in}{1.189423in}}%
\pgfpathlineto{\pgfqpoint{1.743755in}{1.183119in}}%
\pgfpathlineto{\pgfqpoint{1.740503in}{1.188810in}}%
\pgfpathlineto{\pgfqpoint{1.737250in}{1.194577in}}%
\pgfpathlineto{\pgfqpoint{1.733996in}{1.200418in}}%
\pgfpathlineto{\pgfqpoint{1.730741in}{1.206328in}}%
\pgfpathlineto{\pgfqpoint{1.729166in}{1.212401in}}%
\pgfpathlineto{\pgfqpoint{1.727219in}{1.218440in}}%
\pgfpathlineto{\pgfqpoint{1.724902in}{1.224439in}}%
\pgfpathlineto{\pgfqpoint{1.722220in}{1.230393in}}%
\pgfpathclose%
\pgfusepath{fill}%
\end{pgfscope}%
\begin{pgfscope}%
\pgfpathrectangle{\pgfqpoint{0.329460in}{0.284240in}}{\pgfqpoint{1.989680in}{1.989680in}}%
\pgfusepath{clip}%
\pgfsetbuttcap%
\pgfsetroundjoin%
\definecolor{currentfill}{rgb}{0.122606,0.585371,0.546557}%
\pgfsetfillcolor{currentfill}%
\pgfsetlinewidth{0.000000pt}%
\definecolor{currentstroke}{rgb}{0.000000,0.000000,0.000000}%
\pgfsetstrokecolor{currentstroke}%
\pgfsetdash{}{0pt}%
\pgfpathmoveto{\pgfqpoint{1.531798in}{1.503605in}}%
\pgfpathlineto{\pgfqpoint{1.533953in}{1.498082in}}%
\pgfpathlineto{\pgfqpoint{1.536105in}{1.492544in}}%
\pgfpathlineto{\pgfqpoint{1.538255in}{1.486993in}}%
\pgfpathlineto{\pgfqpoint{1.540402in}{1.481432in}}%
\pgfpathlineto{\pgfqpoint{1.547106in}{1.478477in}}%
\pgfpathlineto{\pgfqpoint{1.553624in}{1.475419in}}%
\pgfpathlineto{\pgfqpoint{1.559950in}{1.472260in}}%
\pgfpathlineto{\pgfqpoint{1.566079in}{1.469005in}}%
\pgfpathlineto{\pgfqpoint{1.563635in}{1.474718in}}%
\pgfpathlineto{\pgfqpoint{1.561189in}{1.480422in}}%
\pgfpathlineto{\pgfqpoint{1.558740in}{1.486112in}}%
\pgfpathlineto{\pgfqpoint{1.556288in}{1.491788in}}%
\pgfpathlineto{\pgfqpoint{1.550444in}{1.494884in}}%
\pgfpathlineto{\pgfqpoint{1.544411in}{1.497887in}}%
\pgfpathlineto{\pgfqpoint{1.538193in}{1.500795in}}%
\pgfpathlineto{\pgfqpoint{1.531798in}{1.503605in}}%
\pgfpathclose%
\pgfusepath{fill}%
\end{pgfscope}%
\begin{pgfscope}%
\pgfpathrectangle{\pgfqpoint{0.329460in}{0.284240in}}{\pgfqpoint{1.989680in}{1.989680in}}%
\pgfusepath{clip}%
\pgfsetbuttcap%
\pgfsetroundjoin%
\definecolor{currentfill}{rgb}{0.133743,0.548535,0.553541}%
\pgfsetfillcolor{currentfill}%
\pgfsetlinewidth{0.000000pt}%
\definecolor{currentstroke}{rgb}{0.000000,0.000000,0.000000}%
\pgfsetstrokecolor{currentstroke}%
\pgfsetdash{}{0pt}%
\pgfpathmoveto{\pgfqpoint{1.566079in}{1.469005in}}%
\pgfpathlineto{\pgfqpoint{1.568519in}{1.463283in}}%
\pgfpathlineto{\pgfqpoint{1.570957in}{1.457555in}}%
\pgfpathlineto{\pgfqpoint{1.573392in}{1.451825in}}%
\pgfpathlineto{\pgfqpoint{1.575824in}{1.446094in}}%
\pgfpathlineto{\pgfqpoint{1.582022in}{1.442578in}}%
\pgfpathlineto{\pgfqpoint{1.588001in}{1.438967in}}%
\pgfpathlineto{\pgfqpoint{1.593754in}{1.435262in}}%
\pgfpathlineto{\pgfqpoint{1.599276in}{1.431468in}}%
\pgfpathlineto{\pgfqpoint{1.596583in}{1.437372in}}%
\pgfpathlineto{\pgfqpoint{1.593888in}{1.443275in}}%
\pgfpathlineto{\pgfqpoint{1.591189in}{1.449175in}}%
\pgfpathlineto{\pgfqpoint{1.588488in}{1.455068in}}%
\pgfpathlineto{\pgfqpoint{1.583213in}{1.458683in}}%
\pgfpathlineto{\pgfqpoint{1.577716in}{1.462213in}}%
\pgfpathlineto{\pgfqpoint{1.572002in}{1.465655in}}%
\pgfpathlineto{\pgfqpoint{1.566079in}{1.469005in}}%
\pgfpathclose%
\pgfusepath{fill}%
\end{pgfscope}%
\begin{pgfscope}%
\pgfpathrectangle{\pgfqpoint{0.329460in}{0.284240in}}{\pgfqpoint{1.989680in}{1.989680in}}%
\pgfusepath{clip}%
\pgfsetbuttcap%
\pgfsetroundjoin%
\definecolor{currentfill}{rgb}{0.212395,0.359683,0.551710}%
\pgfsetfillcolor{currentfill}%
\pgfsetlinewidth{0.000000pt}%
\definecolor{currentstroke}{rgb}{0.000000,0.000000,0.000000}%
\pgfsetstrokecolor{currentstroke}%
\pgfsetdash{}{0pt}%
\pgfpathmoveto{\pgfqpoint{1.003343in}{1.273132in}}%
\pgfpathlineto{\pgfqpoint{1.000139in}{1.266945in}}%
\pgfpathlineto{\pgfqpoint{0.996936in}{1.260804in}}%
\pgfpathlineto{\pgfqpoint{0.993735in}{1.254713in}}%
\pgfpathlineto{\pgfqpoint{0.990536in}{1.248674in}}%
\pgfpathlineto{\pgfqpoint{0.993153in}{1.254398in}}%
\pgfpathlineto{\pgfqpoint{0.996121in}{1.260071in}}%
\pgfpathlineto{\pgfqpoint{0.999434in}{1.265689in}}%
\pgfpathlineto{\pgfqpoint{1.003088in}{1.271248in}}%
\pgfpathlineto{\pgfqpoint{1.006165in}{1.277065in}}%
\pgfpathlineto{\pgfqpoint{1.009244in}{1.282935in}}%
\pgfpathlineto{\pgfqpoint{1.012324in}{1.288855in}}%
\pgfpathlineto{\pgfqpoint{1.015407in}{1.294821in}}%
\pgfpathlineto{\pgfqpoint{1.011892in}{1.289480in}}%
\pgfpathlineto{\pgfqpoint{1.008708in}{1.284082in}}%
\pgfpathlineto{\pgfqpoint{1.005857in}{1.278630in}}%
\pgfpathlineto{\pgfqpoint{1.003343in}{1.273132in}}%
\pgfpathclose%
\pgfusepath{fill}%
\end{pgfscope}%
\begin{pgfscope}%
\pgfpathrectangle{\pgfqpoint{0.329460in}{0.284240in}}{\pgfqpoint{1.989680in}{1.989680in}}%
\pgfusepath{clip}%
\pgfsetbuttcap%
\pgfsetroundjoin%
\definecolor{currentfill}{rgb}{0.134692,0.658636,0.517649}%
\pgfsetfillcolor{currentfill}%
\pgfsetlinewidth{0.000000pt}%
\definecolor{currentstroke}{rgb}{0.000000,0.000000,0.000000}%
\pgfsetstrokecolor{currentstroke}%
\pgfsetdash{}{0pt}%
\pgfpathmoveto{\pgfqpoint{1.434801in}{1.569667in}}%
\pgfpathlineto{\pgfqpoint{1.435911in}{1.564669in}}%
\pgfpathlineto{\pgfqpoint{1.437020in}{1.559637in}}%
\pgfpathlineto{\pgfqpoint{1.438127in}{1.554575in}}%
\pgfpathlineto{\pgfqpoint{1.439233in}{1.549484in}}%
\pgfpathlineto{\pgfqpoint{1.446848in}{1.548090in}}%
\pgfpathlineto{\pgfqpoint{1.454373in}{1.546582in}}%
\pgfpathlineto{\pgfqpoint{1.461801in}{1.544959in}}%
\pgfpathlineto{\pgfqpoint{1.469125in}{1.543224in}}%
\pgfpathlineto{\pgfqpoint{1.467642in}{1.548399in}}%
\pgfpathlineto{\pgfqpoint{1.466158in}{1.553545in}}%
\pgfpathlineto{\pgfqpoint{1.464671in}{1.558661in}}%
\pgfpathlineto{\pgfqpoint{1.463182in}{1.563743in}}%
\pgfpathlineto{\pgfqpoint{1.456229in}{1.565385in}}%
\pgfpathlineto{\pgfqpoint{1.449176in}{1.566921in}}%
\pgfpathlineto{\pgfqpoint{1.442031in}{1.568348in}}%
\pgfpathlineto{\pgfqpoint{1.434801in}{1.569667in}}%
\pgfpathclose%
\pgfusepath{fill}%
\end{pgfscope}%
\begin{pgfscope}%
\pgfpathrectangle{\pgfqpoint{0.329460in}{0.284240in}}{\pgfqpoint{1.989680in}{1.989680in}}%
\pgfusepath{clip}%
\pgfsetbuttcap%
\pgfsetroundjoin%
\definecolor{currentfill}{rgb}{0.283072,0.130895,0.449241}%
\pgfsetfillcolor{currentfill}%
\pgfsetlinewidth{0.000000pt}%
\definecolor{currentstroke}{rgb}{0.000000,0.000000,0.000000}%
\pgfsetstrokecolor{currentstroke}%
\pgfsetdash{}{0pt}%
\pgfpathmoveto{\pgfqpoint{0.916739in}{1.087354in}}%
\pgfpathlineto{\pgfqpoint{0.913446in}{1.082657in}}%
\pgfpathlineto{\pgfqpoint{0.910151in}{1.078083in}}%
\pgfpathlineto{\pgfqpoint{0.906854in}{1.073634in}}%
\pgfpathlineto{\pgfqpoint{0.903556in}{1.069314in}}%
\pgfpathlineto{\pgfqpoint{0.903309in}{1.076648in}}%
\pgfpathlineto{\pgfqpoint{0.903509in}{1.083975in}}%
\pgfpathlineto{\pgfqpoint{0.904154in}{1.091286in}}%
\pgfpathlineto{\pgfqpoint{0.905242in}{1.098575in}}%
\pgfpathlineto{\pgfqpoint{0.908513in}{1.102661in}}%
\pgfpathlineto{\pgfqpoint{0.911783in}{1.106876in}}%
\pgfpathlineto{\pgfqpoint{0.915051in}{1.111216in}}%
\pgfpathlineto{\pgfqpoint{0.918318in}{1.115677in}}%
\pgfpathlineto{\pgfqpoint{0.917276in}{1.108621in}}%
\pgfpathlineto{\pgfqpoint{0.916664in}{1.101544in}}%
\pgfpathlineto{\pgfqpoint{0.916485in}{1.094452in}}%
\pgfpathlineto{\pgfqpoint{0.916739in}{1.087354in}}%
\pgfpathclose%
\pgfusepath{fill}%
\end{pgfscope}%
\begin{pgfscope}%
\pgfpathrectangle{\pgfqpoint{0.329460in}{0.284240in}}{\pgfqpoint{1.989680in}{1.989680in}}%
\pgfusepath{clip}%
\pgfsetbuttcap%
\pgfsetroundjoin%
\definecolor{currentfill}{rgb}{0.280255,0.165693,0.476498}%
\pgfsetfillcolor{currentfill}%
\pgfsetlinewidth{0.000000pt}%
\definecolor{currentstroke}{rgb}{0.000000,0.000000,0.000000}%
\pgfsetstrokecolor{currentstroke}%
\pgfsetdash{}{0pt}%
\pgfpathmoveto{\pgfqpoint{1.769763in}{1.140714in}}%
\pgfpathlineto{\pgfqpoint{1.773014in}{1.135851in}}%
\pgfpathlineto{\pgfqpoint{1.776266in}{1.131097in}}%
\pgfpathlineto{\pgfqpoint{1.779519in}{1.126454in}}%
\pgfpathlineto{\pgfqpoint{1.782772in}{1.121926in}}%
\pgfpathlineto{\pgfqpoint{1.784194in}{1.114895in}}%
\pgfpathlineto{\pgfqpoint{1.785189in}{1.107836in}}%
\pgfpathlineto{\pgfqpoint{1.785752in}{1.100757in}}%
\pgfpathlineto{\pgfqpoint{1.785883in}{1.093664in}}%
\pgfpathlineto{\pgfqpoint{1.782592in}{1.098426in}}%
\pgfpathlineto{\pgfqpoint{1.779302in}{1.103303in}}%
\pgfpathlineto{\pgfqpoint{1.776013in}{1.108291in}}%
\pgfpathlineto{\pgfqpoint{1.772725in}{1.113388in}}%
\pgfpathlineto{\pgfqpoint{1.772611in}{1.120245in}}%
\pgfpathlineto{\pgfqpoint{1.772079in}{1.127090in}}%
\pgfpathlineto{\pgfqpoint{1.771129in}{1.133915in}}%
\pgfpathlineto{\pgfqpoint{1.769763in}{1.140714in}}%
\pgfpathclose%
\pgfusepath{fill}%
\end{pgfscope}%
\begin{pgfscope}%
\pgfpathrectangle{\pgfqpoint{0.329460in}{0.284240in}}{\pgfqpoint{1.989680in}{1.989680in}}%
\pgfusepath{clip}%
\pgfsetbuttcap%
\pgfsetroundjoin%
\definecolor{currentfill}{rgb}{0.179019,0.433756,0.557430}%
\pgfsetfillcolor{currentfill}%
\pgfsetlinewidth{0.000000pt}%
\definecolor{currentstroke}{rgb}{0.000000,0.000000,0.000000}%
\pgfsetstrokecolor{currentstroke}%
\pgfsetdash{}{0pt}%
\pgfpathmoveto{\pgfqpoint{1.040143in}{1.343901in}}%
\pgfpathlineto{\pgfqpoint{1.037043in}{1.337661in}}%
\pgfpathlineto{\pgfqpoint{1.033945in}{1.331445in}}%
\pgfpathlineto{\pgfqpoint{1.030850in}{1.325257in}}%
\pgfpathlineto{\pgfqpoint{1.027757in}{1.319099in}}%
\pgfpathlineto{\pgfqpoint{1.031446in}{1.324163in}}%
\pgfpathlineto{\pgfqpoint{1.035444in}{1.329163in}}%
\pgfpathlineto{\pgfqpoint{1.039748in}{1.334093in}}%
\pgfpathlineto{\pgfqpoint{1.044351in}{1.338950in}}%
\pgfpathlineto{\pgfqpoint{1.047276in}{1.344900in}}%
\pgfpathlineto{\pgfqpoint{1.050203in}{1.350879in}}%
\pgfpathlineto{\pgfqpoint{1.053133in}{1.356887in}}%
\pgfpathlineto{\pgfqpoint{1.056065in}{1.362919in}}%
\pgfpathlineto{\pgfqpoint{1.051647in}{1.358266in}}%
\pgfpathlineto{\pgfqpoint{1.047517in}{1.353542in}}%
\pgfpathlineto{\pgfqpoint{1.043681in}{1.348752in}}%
\pgfpathlineto{\pgfqpoint{1.040143in}{1.343901in}}%
\pgfpathclose%
\pgfusepath{fill}%
\end{pgfscope}%
\begin{pgfscope}%
\pgfpathrectangle{\pgfqpoint{0.329460in}{0.284240in}}{\pgfqpoint{1.989680in}{1.989680in}}%
\pgfusepath{clip}%
\pgfsetbuttcap%
\pgfsetroundjoin%
\definecolor{currentfill}{rgb}{0.201239,0.383670,0.554294}%
\pgfsetfillcolor{currentfill}%
\pgfsetlinewidth{0.000000pt}%
\definecolor{currentstroke}{rgb}{0.000000,0.000000,0.000000}%
\pgfsetstrokecolor{currentstroke}%
\pgfsetdash{}{0pt}%
\pgfpathmoveto{\pgfqpoint{1.985375in}{1.289269in}}%
\pgfpathlineto{\pgfqpoint{1.989038in}{1.301208in}}%
\pgfpathlineto{\pgfqpoint{1.992720in}{1.313571in}}%
\pgfpathlineto{\pgfqpoint{1.996420in}{1.326364in}}%
\pgfpathlineto{\pgfqpoint{2.000140in}{1.339594in}}%
\pgfpathlineto{\pgfqpoint{2.005295in}{1.329355in}}%
\pgfpathlineto{\pgfqpoint{2.009820in}{1.319022in}}%
\pgfpathlineto{\pgfqpoint{2.013708in}{1.308606in}}%
\pgfpathlineto{\pgfqpoint{2.016952in}{1.298116in}}%
\pgfpathlineto{\pgfqpoint{2.013111in}{1.285037in}}%
\pgfpathlineto{\pgfqpoint{2.009290in}{1.272398in}}%
\pgfpathlineto{\pgfqpoint{2.005488in}{1.260192in}}%
\pgfpathlineto{\pgfqpoint{2.001705in}{1.248411in}}%
\pgfpathlineto{\pgfqpoint{1.998561in}{1.258743in}}%
\pgfpathlineto{\pgfqpoint{1.994786in}{1.269003in}}%
\pgfpathlineto{\pgfqpoint{1.990388in}{1.279182in}}%
\pgfpathlineto{\pgfqpoint{1.985375in}{1.289269in}}%
\pgfpathclose%
\pgfusepath{fill}%
\end{pgfscope}%
\begin{pgfscope}%
\pgfpathrectangle{\pgfqpoint{0.329460in}{0.284240in}}{\pgfqpoint{1.989680in}{1.989680in}}%
\pgfusepath{clip}%
\pgfsetbuttcap%
\pgfsetroundjoin%
\definecolor{currentfill}{rgb}{0.134692,0.658636,0.517649}%
\pgfsetfillcolor{currentfill}%
\pgfsetlinewidth{0.000000pt}%
\definecolor{currentstroke}{rgb}{0.000000,0.000000,0.000000}%
\pgfsetstrokecolor{currentstroke}%
\pgfsetdash{}{0pt}%
\pgfpathmoveto{\pgfqpoint{1.233101in}{1.562195in}}%
\pgfpathlineto{\pgfqpoint{1.231531in}{1.557091in}}%
\pgfpathlineto{\pgfqpoint{1.229963in}{1.551954in}}%
\pgfpathlineto{\pgfqpoint{1.228397in}{1.546785in}}%
\pgfpathlineto{\pgfqpoint{1.226833in}{1.541589in}}%
\pgfpathlineto{\pgfqpoint{1.234059in}{1.543422in}}%
\pgfpathlineto{\pgfqpoint{1.241395in}{1.545145in}}%
\pgfpathlineto{\pgfqpoint{1.248834in}{1.546755in}}%
\pgfpathlineto{\pgfqpoint{1.256369in}{1.548251in}}%
\pgfpathlineto{\pgfqpoint{1.257560in}{1.553358in}}%
\pgfpathlineto{\pgfqpoint{1.258753in}{1.558437in}}%
\pgfpathlineto{\pgfqpoint{1.259947in}{1.563485in}}%
\pgfpathlineto{\pgfqpoint{1.261143in}{1.568500in}}%
\pgfpathlineto{\pgfqpoint{1.253988in}{1.567085in}}%
\pgfpathlineto{\pgfqpoint{1.246925in}{1.565561in}}%
\pgfpathlineto{\pgfqpoint{1.239960in}{1.563931in}}%
\pgfpathlineto{\pgfqpoint{1.233101in}{1.562195in}}%
\pgfpathclose%
\pgfusepath{fill}%
\end{pgfscope}%
\begin{pgfscope}%
\pgfpathrectangle{\pgfqpoint{0.329460in}{0.284240in}}{\pgfqpoint{1.989680in}{1.989680in}}%
\pgfusepath{clip}%
\pgfsetbuttcap%
\pgfsetroundjoin%
\definecolor{currentfill}{rgb}{0.120081,0.622161,0.534946}%
\pgfsetfillcolor{currentfill}%
\pgfsetlinewidth{0.000000pt}%
\definecolor{currentstroke}{rgb}{0.000000,0.000000,0.000000}%
\pgfsetstrokecolor{currentstroke}%
\pgfsetdash{}{0pt}%
\pgfpathmoveto{\pgfqpoint{1.497242in}{1.535192in}}%
\pgfpathlineto{\pgfqpoint{1.499078in}{1.529883in}}%
\pgfpathlineto{\pgfqpoint{1.500912in}{1.524549in}}%
\pgfpathlineto{\pgfqpoint{1.502744in}{1.519194in}}%
\pgfpathlineto{\pgfqpoint{1.504573in}{1.513819in}}%
\pgfpathlineto{\pgfqpoint{1.511613in}{1.511425in}}%
\pgfpathlineto{\pgfqpoint{1.518502in}{1.508922in}}%
\pgfpathlineto{\pgfqpoint{1.525232in}{1.506315in}}%
\pgfpathlineto{\pgfqpoint{1.531798in}{1.503605in}}%
\pgfpathlineto{\pgfqpoint{1.529641in}{1.509111in}}%
\pgfpathlineto{\pgfqpoint{1.527481in}{1.514597in}}%
\pgfpathlineto{\pgfqpoint{1.525318in}{1.520061in}}%
\pgfpathlineto{\pgfqpoint{1.523153in}{1.525501in}}%
\pgfpathlineto{\pgfqpoint{1.516905in}{1.528072in}}%
\pgfpathlineto{\pgfqpoint{1.510499in}{1.530546in}}%
\pgfpathlineto{\pgfqpoint{1.503943in}{1.532920in}}%
\pgfpathlineto{\pgfqpoint{1.497242in}{1.535192in}}%
\pgfpathclose%
\pgfusepath{fill}%
\end{pgfscope}%
\begin{pgfscope}%
\pgfpathrectangle{\pgfqpoint{0.329460in}{0.284240in}}{\pgfqpoint{1.989680in}{1.989680in}}%
\pgfusepath{clip}%
\pgfsetbuttcap%
\pgfsetroundjoin%
\definecolor{currentfill}{rgb}{0.282884,0.135920,0.453427}%
\pgfsetfillcolor{currentfill}%
\pgfsetlinewidth{0.000000pt}%
\definecolor{currentstroke}{rgb}{0.000000,0.000000,0.000000}%
\pgfsetstrokecolor{currentstroke}%
\pgfsetdash{}{0pt}%
\pgfpathmoveto{\pgfqpoint{0.753818in}{1.064862in}}%
\pgfpathlineto{\pgfqpoint{0.750225in}{1.070464in}}%
\pgfpathlineto{\pgfqpoint{0.746619in}{1.076395in}}%
\pgfpathlineto{\pgfqpoint{0.743000in}{1.082659in}}%
\pgfpathlineto{\pgfqpoint{0.739367in}{1.089263in}}%
\pgfpathlineto{\pgfqpoint{0.739285in}{1.099253in}}%
\pgfpathlineto{\pgfqpoint{0.739812in}{1.109223in}}%
\pgfpathlineto{\pgfqpoint{0.740947in}{1.119165in}}%
\pgfpathlineto{\pgfqpoint{0.742685in}{1.129067in}}%
\pgfpathlineto{\pgfqpoint{0.746276in}{1.122269in}}%
\pgfpathlineto{\pgfqpoint{0.749853in}{1.115809in}}%
\pgfpathlineto{\pgfqpoint{0.753418in}{1.109681in}}%
\pgfpathlineto{\pgfqpoint{0.756970in}{1.103879in}}%
\pgfpathlineto{\pgfqpoint{0.755294in}{1.094171in}}%
\pgfpathlineto{\pgfqpoint{0.754208in}{1.084426in}}%
\pgfpathlineto{\pgfqpoint{0.753715in}{1.074653in}}%
\pgfpathlineto{\pgfqpoint{0.753818in}{1.064862in}}%
\pgfpathclose%
\pgfusepath{fill}%
\end{pgfscope}%
\begin{pgfscope}%
\pgfpathrectangle{\pgfqpoint{0.329460in}{0.284240in}}{\pgfqpoint{1.989680in}{1.989680in}}%
\pgfusepath{clip}%
\pgfsetbuttcap%
\pgfsetroundjoin%
\definecolor{currentfill}{rgb}{0.122606,0.585371,0.546557}%
\pgfsetfillcolor{currentfill}%
\pgfsetlinewidth{0.000000pt}%
\definecolor{currentstroke}{rgb}{0.000000,0.000000,0.000000}%
\pgfsetstrokecolor{currentstroke}%
\pgfsetdash{}{0pt}%
\pgfpathmoveto{\pgfqpoint{1.141056in}{1.488961in}}%
\pgfpathlineto{\pgfqpoint{1.138543in}{1.483249in}}%
\pgfpathlineto{\pgfqpoint{1.136033in}{1.477522in}}%
\pgfpathlineto{\pgfqpoint{1.133525in}{1.471782in}}%
\pgfpathlineto{\pgfqpoint{1.131021in}{1.466031in}}%
\pgfpathlineto{\pgfqpoint{1.136967in}{1.469371in}}%
\pgfpathlineto{\pgfqpoint{1.143118in}{1.472616in}}%
\pgfpathlineto{\pgfqpoint{1.149466in}{1.475763in}}%
\pgfpathlineto{\pgfqpoint{1.156005in}{1.478810in}}%
\pgfpathlineto{\pgfqpoint{1.158221in}{1.484403in}}%
\pgfpathlineto{\pgfqpoint{1.160439in}{1.489986in}}%
\pgfpathlineto{\pgfqpoint{1.162660in}{1.495557in}}%
\pgfpathlineto{\pgfqpoint{1.164884in}{1.501112in}}%
\pgfpathlineto{\pgfqpoint{1.158646in}{1.498215in}}%
\pgfpathlineto{\pgfqpoint{1.152592in}{1.495222in}}%
\pgfpathlineto{\pgfqpoint{1.146727in}{1.492136in}}%
\pgfpathlineto{\pgfqpoint{1.141056in}{1.488961in}}%
\pgfpathclose%
\pgfusepath{fill}%
\end{pgfscope}%
\begin{pgfscope}%
\pgfpathrectangle{\pgfqpoint{0.329460in}{0.284240in}}{\pgfqpoint{1.989680in}{1.989680in}}%
\pgfusepath{clip}%
\pgfsetbuttcap%
\pgfsetroundjoin%
\definecolor{currentfill}{rgb}{0.147607,0.511733,0.557049}%
\pgfsetfillcolor{currentfill}%
\pgfsetlinewidth{0.000000pt}%
\definecolor{currentstroke}{rgb}{0.000000,0.000000,0.000000}%
\pgfsetstrokecolor{currentstroke}%
\pgfsetdash{}{0pt}%
\pgfpathmoveto{\pgfqpoint{1.599276in}{1.431468in}}%
\pgfpathlineto{\pgfqpoint{1.601965in}{1.425566in}}%
\pgfpathlineto{\pgfqpoint{1.604652in}{1.419669in}}%
\pgfpathlineto{\pgfqpoint{1.607336in}{1.413778in}}%
\pgfpathlineto{\pgfqpoint{1.610018in}{1.407896in}}%
\pgfpathlineto{\pgfqpoint{1.615539in}{1.403832in}}%
\pgfpathlineto{\pgfqpoint{1.620806in}{1.399682in}}%
\pgfpathlineto{\pgfqpoint{1.625816in}{1.395449in}}%
\pgfpathlineto{\pgfqpoint{1.630562in}{1.391137in}}%
\pgfpathlineto{\pgfqpoint{1.627660in}{1.397209in}}%
\pgfpathlineto{\pgfqpoint{1.624755in}{1.403290in}}%
\pgfpathlineto{\pgfqpoint{1.621847in}{1.409378in}}%
\pgfpathlineto{\pgfqpoint{1.618937in}{1.415469in}}%
\pgfpathlineto{\pgfqpoint{1.614396in}{1.419585in}}%
\pgfpathlineto{\pgfqpoint{1.609602in}{1.423626in}}%
\pgfpathlineto{\pgfqpoint{1.604560in}{1.427589in}}%
\pgfpathlineto{\pgfqpoint{1.599276in}{1.431468in}}%
\pgfpathclose%
\pgfusepath{fill}%
\end{pgfscope}%
\begin{pgfscope}%
\pgfpathrectangle{\pgfqpoint{0.329460in}{0.284240in}}{\pgfqpoint{1.989680in}{1.989680in}}%
\pgfusepath{clip}%
\pgfsetbuttcap%
\pgfsetroundjoin%
\definecolor{currentfill}{rgb}{0.133743,0.548535,0.553541}%
\pgfsetfillcolor{currentfill}%
\pgfsetlinewidth{0.000000pt}%
\definecolor{currentstroke}{rgb}{0.000000,0.000000,0.000000}%
\pgfsetstrokecolor{currentstroke}%
\pgfsetdash{}{0pt}%
\pgfpathmoveto{\pgfqpoint{1.109388in}{1.451786in}}%
\pgfpathlineto{\pgfqpoint{1.106635in}{1.445852in}}%
\pgfpathlineto{\pgfqpoint{1.103884in}{1.439912in}}%
\pgfpathlineto{\pgfqpoint{1.101135in}{1.433968in}}%
\pgfpathlineto{\pgfqpoint{1.098390in}{1.428024in}}%
\pgfpathlineto{\pgfqpoint{1.103701in}{1.431894in}}%
\pgfpathlineto{\pgfqpoint{1.109249in}{1.435678in}}%
\pgfpathlineto{\pgfqpoint{1.115028in}{1.439373in}}%
\pgfpathlineto{\pgfqpoint{1.121031in}{1.442974in}}%
\pgfpathlineto{\pgfqpoint{1.123524in}{1.448742in}}%
\pgfpathlineto{\pgfqpoint{1.126020in}{1.454509in}}%
\pgfpathlineto{\pgfqpoint{1.128519in}{1.460273in}}%
\pgfpathlineto{\pgfqpoint{1.131021in}{1.466031in}}%
\pgfpathlineto{\pgfqpoint{1.125284in}{1.462600in}}%
\pgfpathlineto{\pgfqpoint{1.119762in}{1.459080in}}%
\pgfpathlineto{\pgfqpoint{1.114462in}{1.455474in}}%
\pgfpathlineto{\pgfqpoint{1.109388in}{1.451786in}}%
\pgfpathclose%
\pgfusepath{fill}%
\end{pgfscope}%
\begin{pgfscope}%
\pgfpathrectangle{\pgfqpoint{0.329460in}{0.284240in}}{\pgfqpoint{1.989680in}{1.989680in}}%
\pgfusepath{clip}%
\pgfsetbuttcap%
\pgfsetroundjoin%
\definecolor{currentfill}{rgb}{0.276194,0.190074,0.493001}%
\pgfsetfillcolor{currentfill}%
\pgfsetlinewidth{0.000000pt}%
\definecolor{currentstroke}{rgb}{0.000000,0.000000,0.000000}%
\pgfsetstrokecolor{currentstroke}%
\pgfsetdash{}{0pt}%
\pgfpathmoveto{\pgfqpoint{1.957643in}{1.137829in}}%
\pgfpathlineto{\pgfqpoint{1.961230in}{1.145013in}}%
\pgfpathlineto{\pgfqpoint{1.964832in}{1.152546in}}%
\pgfpathlineto{\pgfqpoint{1.968448in}{1.160434in}}%
\pgfpathlineto{\pgfqpoint{1.972079in}{1.168683in}}%
\pgfpathlineto{\pgfqpoint{1.974426in}{1.158636in}}%
\pgfpathlineto{\pgfqpoint{1.976160in}{1.148541in}}%
\pgfpathlineto{\pgfqpoint{1.977276in}{1.138407in}}%
\pgfpathlineto{\pgfqpoint{1.977770in}{1.128243in}}%
\pgfpathlineto{\pgfqpoint{1.974083in}{1.120180in}}%
\pgfpathlineto{\pgfqpoint{1.970411in}{1.112480in}}%
\pgfpathlineto{\pgfqpoint{1.966754in}{1.105136in}}%
\pgfpathlineto{\pgfqpoint{1.963111in}{1.098144in}}%
\pgfpathlineto{\pgfqpoint{1.962651in}{1.108117in}}%
\pgfpathlineto{\pgfqpoint{1.961584in}{1.118062in}}%
\pgfpathlineto{\pgfqpoint{1.959913in}{1.127969in}}%
\pgfpathlineto{\pgfqpoint{1.957643in}{1.137829in}}%
\pgfpathclose%
\pgfusepath{fill}%
\end{pgfscope}%
\begin{pgfscope}%
\pgfpathrectangle{\pgfqpoint{0.329460in}{0.284240in}}{\pgfqpoint{1.989680in}{1.989680in}}%
\pgfusepath{clip}%
\pgfsetbuttcap%
\pgfsetroundjoin%
\definecolor{currentfill}{rgb}{0.248629,0.278775,0.534556}%
\pgfsetfillcolor{currentfill}%
\pgfsetlinewidth{0.000000pt}%
\definecolor{currentstroke}{rgb}{0.000000,0.000000,0.000000}%
\pgfsetstrokecolor{currentstroke}%
\pgfsetdash{}{0pt}%
\pgfpathmoveto{\pgfqpoint{0.970548in}{1.200906in}}%
\pgfpathlineto{\pgfqpoint{0.967281in}{1.194944in}}%
\pgfpathlineto{\pgfqpoint{0.964015in}{1.189052in}}%
\pgfpathlineto{\pgfqpoint{0.960750in}{1.183233in}}%
\pgfpathlineto{\pgfqpoint{0.957485in}{1.177490in}}%
\pgfpathlineto{\pgfqpoint{0.958783in}{1.183821in}}%
\pgfpathlineto{\pgfqpoint{0.960468in}{1.190122in}}%
\pgfpathlineto{\pgfqpoint{0.962537in}{1.196386in}}%
\pgfpathlineto{\pgfqpoint{0.964988in}{1.202608in}}%
\pgfpathlineto{\pgfqpoint{0.968178in}{1.208120in}}%
\pgfpathlineto{\pgfqpoint{0.971368in}{1.213709in}}%
\pgfpathlineto{\pgfqpoint{0.974560in}{1.219371in}}%
\pgfpathlineto{\pgfqpoint{0.977753in}{1.225103in}}%
\pgfpathlineto{\pgfqpoint{0.975396in}{1.219109in}}%
\pgfpathlineto{\pgfqpoint{0.973407in}{1.213074in}}%
\pgfpathlineto{\pgfqpoint{0.971790in}{1.207004in}}%
\pgfpathlineto{\pgfqpoint{0.970548in}{1.200906in}}%
\pgfpathclose%
\pgfusepath{fill}%
\end{pgfscope}%
\begin{pgfscope}%
\pgfpathrectangle{\pgfqpoint{0.329460in}{0.284240in}}{\pgfqpoint{1.989680in}{1.989680in}}%
\pgfusepath{clip}%
\pgfsetbuttcap%
\pgfsetroundjoin%
\definecolor{currentfill}{rgb}{0.120081,0.622161,0.534946}%
\pgfsetfillcolor{currentfill}%
\pgfsetlinewidth{0.000000pt}%
\definecolor{currentstroke}{rgb}{0.000000,0.000000,0.000000}%
\pgfsetstrokecolor{currentstroke}%
\pgfsetdash{}{0pt}%
\pgfpathmoveto{\pgfqpoint{1.173806in}{1.523136in}}%
\pgfpathlineto{\pgfqpoint{1.171571in}{1.517664in}}%
\pgfpathlineto{\pgfqpoint{1.169339in}{1.512168in}}%
\pgfpathlineto{\pgfqpoint{1.167110in}{1.506650in}}%
\pgfpathlineto{\pgfqpoint{1.164884in}{1.501112in}}%
\pgfpathlineto{\pgfqpoint{1.171298in}{1.503911in}}%
\pgfpathlineto{\pgfqpoint{1.177882in}{1.506610in}}%
\pgfpathlineto{\pgfqpoint{1.184631in}{1.509206in}}%
\pgfpathlineto{\pgfqpoint{1.191537in}{1.511696in}}%
\pgfpathlineto{\pgfqpoint{1.193442in}{1.517098in}}%
\pgfpathlineto{\pgfqpoint{1.195348in}{1.522480in}}%
\pgfpathlineto{\pgfqpoint{1.197258in}{1.527841in}}%
\pgfpathlineto{\pgfqpoint{1.199170in}{1.533177in}}%
\pgfpathlineto{\pgfqpoint{1.192597in}{1.530814in}}%
\pgfpathlineto{\pgfqpoint{1.186175in}{1.528352in}}%
\pgfpathlineto{\pgfqpoint{1.179909in}{1.525792in}}%
\pgfpathlineto{\pgfqpoint{1.173806in}{1.523136in}}%
\pgfpathclose%
\pgfusepath{fill}%
\end{pgfscope}%
\begin{pgfscope}%
\pgfpathrectangle{\pgfqpoint{0.329460in}{0.284240in}}{\pgfqpoint{1.989680in}{1.989680in}}%
\pgfusepath{clip}%
\pgfsetbuttcap%
\pgfsetroundjoin%
\definecolor{currentfill}{rgb}{0.166383,0.690856,0.496502}%
\pgfsetfillcolor{currentfill}%
\pgfsetlinewidth{0.000000pt}%
\definecolor{currentstroke}{rgb}{0.000000,0.000000,0.000000}%
\pgfsetstrokecolor{currentstroke}%
\pgfsetdash{}{0pt}%
\pgfpathmoveto{\pgfqpoint{1.322466in}{1.595090in}}%
\pgfpathlineto{\pgfqpoint{1.322062in}{1.590331in}}%
\pgfpathlineto{\pgfqpoint{1.321658in}{1.585530in}}%
\pgfpathlineto{\pgfqpoint{1.321254in}{1.580690in}}%
\pgfpathlineto{\pgfqpoint{1.320851in}{1.575813in}}%
\pgfpathlineto{\pgfqpoint{1.328518in}{1.576213in}}%
\pgfpathlineto{\pgfqpoint{1.336205in}{1.576497in}}%
\pgfpathlineto{\pgfqpoint{1.343907in}{1.576666in}}%
\pgfpathlineto{\pgfqpoint{1.351616in}{1.576717in}}%
\pgfpathlineto{\pgfqpoint{1.351610in}{1.581582in}}%
\pgfpathlineto{\pgfqpoint{1.351604in}{1.586409in}}%
\pgfpathlineto{\pgfqpoint{1.351599in}{1.591197in}}%
\pgfpathlineto{\pgfqpoint{1.351593in}{1.595943in}}%
\pgfpathlineto{\pgfqpoint{1.344295in}{1.595894in}}%
\pgfpathlineto{\pgfqpoint{1.337003in}{1.595736in}}%
\pgfpathlineto{\pgfqpoint{1.329725in}{1.595468in}}%
\pgfpathlineto{\pgfqpoint{1.322466in}{1.595090in}}%
\pgfpathclose%
\pgfusepath{fill}%
\end{pgfscope}%
\begin{pgfscope}%
\pgfpathrectangle{\pgfqpoint{0.329460in}{0.284240in}}{\pgfqpoint{1.989680in}{1.989680in}}%
\pgfusepath{clip}%
\pgfsetbuttcap%
\pgfsetroundjoin%
\definecolor{currentfill}{rgb}{0.166383,0.690856,0.496502}%
\pgfsetfillcolor{currentfill}%
\pgfsetlinewidth{0.000000pt}%
\definecolor{currentstroke}{rgb}{0.000000,0.000000,0.000000}%
\pgfsetstrokecolor{currentstroke}%
\pgfsetdash{}{0pt}%
\pgfpathmoveto{\pgfqpoint{1.351593in}{1.595943in}}%
\pgfpathlineto{\pgfqpoint{1.351599in}{1.591197in}}%
\pgfpathlineto{\pgfqpoint{1.351604in}{1.586409in}}%
\pgfpathlineto{\pgfqpoint{1.351610in}{1.581582in}}%
\pgfpathlineto{\pgfqpoint{1.351616in}{1.576717in}}%
\pgfpathlineto{\pgfqpoint{1.359324in}{1.576653in}}%
\pgfpathlineto{\pgfqpoint{1.367025in}{1.576472in}}%
\pgfpathlineto{\pgfqpoint{1.374710in}{1.576174in}}%
\pgfpathlineto{\pgfqpoint{1.382374in}{1.575761in}}%
\pgfpathlineto{\pgfqpoint{1.381960in}{1.580639in}}%
\pgfpathlineto{\pgfqpoint{1.381545in}{1.585480in}}%
\pgfpathlineto{\pgfqpoint{1.381130in}{1.590281in}}%
\pgfpathlineto{\pgfqpoint{1.380714in}{1.595041in}}%
\pgfpathlineto{\pgfqpoint{1.373458in}{1.595431in}}%
\pgfpathlineto{\pgfqpoint{1.366182in}{1.595711in}}%
\pgfpathlineto{\pgfqpoint{1.358891in}{1.595882in}}%
\pgfpathlineto{\pgfqpoint{1.351593in}{1.595943in}}%
\pgfpathclose%
\pgfusepath{fill}%
\end{pgfscope}%
\begin{pgfscope}%
\pgfpathrectangle{\pgfqpoint{0.329460in}{0.284240in}}{\pgfqpoint{1.989680in}{1.989680in}}%
\pgfusepath{clip}%
\pgfsetbuttcap%
\pgfsetroundjoin%
\definecolor{currentfill}{rgb}{0.195860,0.395433,0.555276}%
\pgfsetfillcolor{currentfill}%
\pgfsetlinewidth{0.000000pt}%
\definecolor{currentstroke}{rgb}{0.000000,0.000000,0.000000}%
\pgfsetstrokecolor{currentstroke}%
\pgfsetdash{}{0pt}%
\pgfpathmoveto{\pgfqpoint{1.671355in}{1.323603in}}%
\pgfpathlineto{\pgfqpoint{1.674412in}{1.317525in}}%
\pgfpathlineto{\pgfqpoint{1.677467in}{1.311483in}}%
\pgfpathlineto{\pgfqpoint{1.680520in}{1.305479in}}%
\pgfpathlineto{\pgfqpoint{1.683571in}{1.299516in}}%
\pgfpathlineto{\pgfqpoint{1.687375in}{1.294231in}}%
\pgfpathlineto{\pgfqpoint{1.690853in}{1.288883in}}%
\pgfpathlineto{\pgfqpoint{1.694001in}{1.283478in}}%
\pgfpathlineto{\pgfqpoint{1.696814in}{1.278022in}}%
\pgfpathlineto{\pgfqpoint{1.693631in}{1.284203in}}%
\pgfpathlineto{\pgfqpoint{1.690446in}{1.290425in}}%
\pgfpathlineto{\pgfqpoint{1.687259in}{1.296686in}}%
\pgfpathlineto{\pgfqpoint{1.684069in}{1.302982in}}%
\pgfpathlineto{\pgfqpoint{1.681370in}{1.308216in}}%
\pgfpathlineto{\pgfqpoint{1.678349in}{1.313401in}}%
\pgfpathlineto{\pgfqpoint{1.675009in}{1.318532in}}%
\pgfpathlineto{\pgfqpoint{1.671355in}{1.323603in}}%
\pgfpathclose%
\pgfusepath{fill}%
\end{pgfscope}%
\begin{pgfscope}%
\pgfpathrectangle{\pgfqpoint{0.329460in}{0.284240in}}{\pgfqpoint{1.989680in}{1.989680in}}%
\pgfusepath{clip}%
\pgfsetbuttcap%
\pgfsetroundjoin%
\definecolor{currentfill}{rgb}{0.267004,0.004874,0.329415}%
\pgfsetfillcolor{currentfill}%
\pgfsetlinewidth{0.000000pt}%
\definecolor{currentstroke}{rgb}{0.000000,0.000000,0.000000}%
\pgfsetstrokecolor{currentstroke}%
\pgfsetdash{}{0pt}%
\pgfpathmoveto{\pgfqpoint{1.865598in}{1.023065in}}%
\pgfpathlineto{\pgfqpoint{1.868970in}{1.022349in}}%
\pgfpathlineto{\pgfqpoint{1.872348in}{1.021845in}}%
\pgfpathlineto{\pgfqpoint{1.875732in}{1.021559in}}%
\pgfpathlineto{\pgfqpoint{1.879122in}{1.021493in}}%
\pgfpathlineto{\pgfqpoint{1.878879in}{1.012772in}}%
\pgfpathlineto{\pgfqpoint{1.878107in}{1.004048in}}%
\pgfpathlineto{\pgfqpoint{1.876805in}{0.995329in}}%
\pgfpathlineto{\pgfqpoint{1.874972in}{0.986625in}}%
\pgfpathlineto{\pgfqpoint{1.871590in}{0.986916in}}%
\pgfpathlineto{\pgfqpoint{1.868215in}{0.987429in}}%
\pgfpathlineto{\pgfqpoint{1.864846in}{0.988159in}}%
\pgfpathlineto{\pgfqpoint{1.861484in}{0.989103in}}%
\pgfpathlineto{\pgfqpoint{1.863288in}{0.997580in}}%
\pgfpathlineto{\pgfqpoint{1.864574in}{1.006072in}}%
\pgfpathlineto{\pgfqpoint{1.865344in}{1.014570in}}%
\pgfpathlineto{\pgfqpoint{1.865598in}{1.023065in}}%
\pgfpathclose%
\pgfusepath{fill}%
\end{pgfscope}%
\begin{pgfscope}%
\pgfpathrectangle{\pgfqpoint{0.329460in}{0.284240in}}{\pgfqpoint{1.989680in}{1.989680in}}%
\pgfusepath{clip}%
\pgfsetbuttcap%
\pgfsetroundjoin%
\definecolor{currentfill}{rgb}{0.268510,0.009605,0.335427}%
\pgfsetfillcolor{currentfill}%
\pgfsetlinewidth{0.000000pt}%
\definecolor{currentstroke}{rgb}{0.000000,0.000000,0.000000}%
\pgfsetstrokecolor{currentstroke}%
\pgfsetdash{}{0pt}%
\pgfpathmoveto{\pgfqpoint{1.852164in}{1.027970in}}%
\pgfpathlineto{\pgfqpoint{1.855515in}{1.026445in}}%
\pgfpathlineto{\pgfqpoint{1.858871in}{1.025117in}}%
\pgfpathlineto{\pgfqpoint{1.862232in}{1.023989in}}%
\pgfpathlineto{\pgfqpoint{1.865598in}{1.023065in}}%
\pgfpathlineto{\pgfqpoint{1.865344in}{1.014570in}}%
\pgfpathlineto{\pgfqpoint{1.864574in}{1.006072in}}%
\pgfpathlineto{\pgfqpoint{1.863288in}{0.997580in}}%
\pgfpathlineto{\pgfqpoint{1.861484in}{0.989103in}}%
\pgfpathlineto{\pgfqpoint{1.858127in}{0.990255in}}%
\pgfpathlineto{\pgfqpoint{1.854776in}{0.991612in}}%
\pgfpathlineto{\pgfqpoint{1.851430in}{0.993170in}}%
\pgfpathlineto{\pgfqpoint{1.848090in}{0.994924in}}%
\pgfpathlineto{\pgfqpoint{1.849863in}{1.003172in}}%
\pgfpathlineto{\pgfqpoint{1.851133in}{1.011434in}}%
\pgfpathlineto{\pgfqpoint{1.851899in}{1.019703in}}%
\pgfpathlineto{\pgfqpoint{1.852164in}{1.027970in}}%
\pgfpathclose%
\pgfusepath{fill}%
\end{pgfscope}%
\begin{pgfscope}%
\pgfpathrectangle{\pgfqpoint{0.329460in}{0.284240in}}{\pgfqpoint{1.989680in}{1.989680in}}%
\pgfusepath{clip}%
\pgfsetbuttcap%
\pgfsetroundjoin%
\definecolor{currentfill}{rgb}{0.147607,0.511733,0.557049}%
\pgfsetfillcolor{currentfill}%
\pgfsetlinewidth{0.000000pt}%
\definecolor{currentstroke}{rgb}{0.000000,0.000000,0.000000}%
\pgfsetstrokecolor{currentstroke}%
\pgfsetdash{}{0pt}%
\pgfpathmoveto{\pgfqpoint{1.079619in}{1.411751in}}%
\pgfpathlineto{\pgfqpoint{1.076665in}{1.405615in}}%
\pgfpathlineto{\pgfqpoint{1.073714in}{1.399483in}}%
\pgfpathlineto{\pgfqpoint{1.070766in}{1.393358in}}%
\pgfpathlineto{\pgfqpoint{1.067820in}{1.387241in}}%
\pgfpathlineto{\pgfqpoint{1.072327in}{1.391620in}}%
\pgfpathlineto{\pgfqpoint{1.077103in}{1.395923in}}%
\pgfpathlineto{\pgfqpoint{1.082141in}{1.400147in}}%
\pgfpathlineto{\pgfqpoint{1.087438in}{1.404288in}}%
\pgfpathlineto{\pgfqpoint{1.090172in}{1.410211in}}%
\pgfpathlineto{\pgfqpoint{1.092908in}{1.416143in}}%
\pgfpathlineto{\pgfqpoint{1.095648in}{1.422081in}}%
\pgfpathlineto{\pgfqpoint{1.098390in}{1.428024in}}%
\pgfpathlineto{\pgfqpoint{1.093321in}{1.424071in}}%
\pgfpathlineto{\pgfqpoint{1.088500in}{1.420038in}}%
\pgfpathlineto{\pgfqpoint{1.083930in}{1.415930in}}%
\pgfpathlineto{\pgfqpoint{1.079619in}{1.411751in}}%
\pgfpathclose%
\pgfusepath{fill}%
\end{pgfscope}%
\begin{pgfscope}%
\pgfpathrectangle{\pgfqpoint{0.329460in}{0.284240in}}{\pgfqpoint{1.989680in}{1.989680in}}%
\pgfusepath{clip}%
\pgfsetbuttcap%
\pgfsetroundjoin%
\definecolor{currentfill}{rgb}{0.267004,0.004874,0.329415}%
\pgfsetfillcolor{currentfill}%
\pgfsetlinewidth{0.000000pt}%
\definecolor{currentstroke}{rgb}{0.000000,0.000000,0.000000}%
\pgfsetstrokecolor{currentstroke}%
\pgfsetdash{}{0pt}%
\pgfpathmoveto{\pgfqpoint{1.879122in}{1.021493in}}%
\pgfpathlineto{\pgfqpoint{1.882519in}{1.021652in}}%
\pgfpathlineto{\pgfqpoint{1.885923in}{1.022041in}}%
\pgfpathlineto{\pgfqpoint{1.889333in}{1.022664in}}%
\pgfpathlineto{\pgfqpoint{1.892752in}{1.023526in}}%
\pgfpathlineto{\pgfqpoint{1.892521in}{1.014582in}}%
\pgfpathlineto{\pgfqpoint{1.891748in}{1.005635in}}%
\pgfpathlineto{\pgfqpoint{1.890431in}{0.996692in}}%
\pgfpathlineto{\pgfqpoint{1.888569in}{0.987764in}}%
\pgfpathlineto{\pgfqpoint{1.885158in}{0.987125in}}%
\pgfpathlineto{\pgfqpoint{1.881756in}{0.986725in}}%
\pgfpathlineto{\pgfqpoint{1.878360in}{0.986560in}}%
\pgfpathlineto{\pgfqpoint{1.874972in}{0.986625in}}%
\pgfpathlineto{\pgfqpoint{1.876805in}{0.995329in}}%
\pgfpathlineto{\pgfqpoint{1.878107in}{1.004048in}}%
\pgfpathlineto{\pgfqpoint{1.878879in}{1.012772in}}%
\pgfpathlineto{\pgfqpoint{1.879122in}{1.021493in}}%
\pgfpathclose%
\pgfusepath{fill}%
\end{pgfscope}%
\begin{pgfscope}%
\pgfpathrectangle{\pgfqpoint{0.329460in}{0.284240in}}{\pgfqpoint{1.989680in}{1.989680in}}%
\pgfusepath{clip}%
\pgfsetbuttcap%
\pgfsetroundjoin%
\definecolor{currentfill}{rgb}{0.280255,0.165693,0.476498}%
\pgfsetfillcolor{currentfill}%
\pgfsetlinewidth{0.000000pt}%
\definecolor{currentstroke}{rgb}{0.000000,0.000000,0.000000}%
\pgfsetstrokecolor{currentstroke}%
\pgfsetdash{}{0pt}%
\pgfpathmoveto{\pgfqpoint{0.929903in}{1.107288in}}%
\pgfpathlineto{\pgfqpoint{0.926614in}{1.102139in}}%
\pgfpathlineto{\pgfqpoint{0.923323in}{1.097097in}}%
\pgfpathlineto{\pgfqpoint{0.920032in}{1.092168in}}%
\pgfpathlineto{\pgfqpoint{0.916739in}{1.087354in}}%
\pgfpathlineto{\pgfqpoint{0.916485in}{1.094452in}}%
\pgfpathlineto{\pgfqpoint{0.916664in}{1.101544in}}%
\pgfpathlineto{\pgfqpoint{0.917276in}{1.108621in}}%
\pgfpathlineto{\pgfqpoint{0.918318in}{1.115677in}}%
\pgfpathlineto{\pgfqpoint{0.921584in}{1.120257in}}%
\pgfpathlineto{\pgfqpoint{0.924849in}{1.124952in}}%
\pgfpathlineto{\pgfqpoint{0.928114in}{1.129758in}}%
\pgfpathlineto{\pgfqpoint{0.931378in}{1.134672in}}%
\pgfpathlineto{\pgfqpoint{0.930381in}{1.127849in}}%
\pgfpathlineto{\pgfqpoint{0.929802in}{1.121007in}}%
\pgfpathlineto{\pgfqpoint{0.929642in}{1.114150in}}%
\pgfpathlineto{\pgfqpoint{0.929903in}{1.107288in}}%
\pgfpathclose%
\pgfusepath{fill}%
\end{pgfscope}%
\begin{pgfscope}%
\pgfpathrectangle{\pgfqpoint{0.329460in}{0.284240in}}{\pgfqpoint{1.989680in}{1.989680in}}%
\pgfusepath{clip}%
\pgfsetbuttcap%
\pgfsetroundjoin%
\definecolor{currentfill}{rgb}{0.166383,0.690856,0.496502}%
\pgfsetfillcolor{currentfill}%
\pgfsetlinewidth{0.000000pt}%
\definecolor{currentstroke}{rgb}{0.000000,0.000000,0.000000}%
\pgfsetstrokecolor{currentstroke}%
\pgfsetdash{}{0pt}%
\pgfpathmoveto{\pgfqpoint{1.293772in}{1.592493in}}%
\pgfpathlineto{\pgfqpoint{1.292963in}{1.587695in}}%
\pgfpathlineto{\pgfqpoint{1.292155in}{1.582856in}}%
\pgfpathlineto{\pgfqpoint{1.291348in}{1.577977in}}%
\pgfpathlineto{\pgfqpoint{1.290542in}{1.573061in}}%
\pgfpathlineto{\pgfqpoint{1.298051in}{1.573921in}}%
\pgfpathlineto{\pgfqpoint{1.305611in}{1.574666in}}%
\pgfpathlineto{\pgfqpoint{1.313213in}{1.575297in}}%
\pgfpathlineto{\pgfqpoint{1.320851in}{1.575813in}}%
\pgfpathlineto{\pgfqpoint{1.321254in}{1.580690in}}%
\pgfpathlineto{\pgfqpoint{1.321658in}{1.585530in}}%
\pgfpathlineto{\pgfqpoint{1.322062in}{1.590331in}}%
\pgfpathlineto{\pgfqpoint{1.322466in}{1.595090in}}%
\pgfpathlineto{\pgfqpoint{1.315235in}{1.594603in}}%
\pgfpathlineto{\pgfqpoint{1.308038in}{1.594008in}}%
\pgfpathlineto{\pgfqpoint{1.300881in}{1.593304in}}%
\pgfpathlineto{\pgfqpoint{1.293772in}{1.592493in}}%
\pgfpathclose%
\pgfusepath{fill}%
\end{pgfscope}%
\begin{pgfscope}%
\pgfpathrectangle{\pgfqpoint{0.329460in}{0.284240in}}{\pgfqpoint{1.989680in}{1.989680in}}%
\pgfusepath{clip}%
\pgfsetbuttcap%
\pgfsetroundjoin%
\definecolor{currentfill}{rgb}{0.166383,0.690856,0.496502}%
\pgfsetfillcolor{currentfill}%
\pgfsetlinewidth{0.000000pt}%
\definecolor{currentstroke}{rgb}{0.000000,0.000000,0.000000}%
\pgfsetstrokecolor{currentstroke}%
\pgfsetdash{}{0pt}%
\pgfpathmoveto{\pgfqpoint{1.380714in}{1.595041in}}%
\pgfpathlineto{\pgfqpoint{1.381130in}{1.590281in}}%
\pgfpathlineto{\pgfqpoint{1.381545in}{1.585480in}}%
\pgfpathlineto{\pgfqpoint{1.381960in}{1.580639in}}%
\pgfpathlineto{\pgfqpoint{1.382374in}{1.575761in}}%
\pgfpathlineto{\pgfqpoint{1.390008in}{1.575233in}}%
\pgfpathlineto{\pgfqpoint{1.397606in}{1.574589in}}%
\pgfpathlineto{\pgfqpoint{1.405161in}{1.573831in}}%
\pgfpathlineto{\pgfqpoint{1.412664in}{1.572959in}}%
\pgfpathlineto{\pgfqpoint{1.411847in}{1.577876in}}%
\pgfpathlineto{\pgfqpoint{1.411029in}{1.582756in}}%
\pgfpathlineto{\pgfqpoint{1.410210in}{1.587597in}}%
\pgfpathlineto{\pgfqpoint{1.409390in}{1.592396in}}%
\pgfpathlineto{\pgfqpoint{1.402286in}{1.593219in}}%
\pgfpathlineto{\pgfqpoint{1.395134in}{1.593935in}}%
\pgfpathlineto{\pgfqpoint{1.387941in}{1.594542in}}%
\pgfpathlineto{\pgfqpoint{1.380714in}{1.595041in}}%
\pgfpathclose%
\pgfusepath{fill}%
\end{pgfscope}%
\begin{pgfscope}%
\pgfpathrectangle{\pgfqpoint{0.329460in}{0.284240in}}{\pgfqpoint{1.989680in}{1.989680in}}%
\pgfusepath{clip}%
\pgfsetbuttcap%
\pgfsetroundjoin%
\definecolor{currentfill}{rgb}{0.134692,0.658636,0.517649}%
\pgfsetfillcolor{currentfill}%
\pgfsetlinewidth{0.000000pt}%
\definecolor{currentstroke}{rgb}{0.000000,0.000000,0.000000}%
\pgfsetstrokecolor{currentstroke}%
\pgfsetdash{}{0pt}%
\pgfpathmoveto{\pgfqpoint{1.463182in}{1.563743in}}%
\pgfpathlineto{\pgfqpoint{1.464671in}{1.558661in}}%
\pgfpathlineto{\pgfqpoint{1.466158in}{1.553545in}}%
\pgfpathlineto{\pgfqpoint{1.467642in}{1.548399in}}%
\pgfpathlineto{\pgfqpoint{1.469125in}{1.543224in}}%
\pgfpathlineto{\pgfqpoint{1.476338in}{1.541378in}}%
\pgfpathlineto{\pgfqpoint{1.483433in}{1.539423in}}%
\pgfpathlineto{\pgfqpoint{1.490403in}{1.537360in}}%
\pgfpathlineto{\pgfqpoint{1.497242in}{1.535192in}}%
\pgfpathlineto{\pgfqpoint{1.495404in}{1.540475in}}%
\pgfpathlineto{\pgfqpoint{1.493563in}{1.545729in}}%
\pgfpathlineto{\pgfqpoint{1.491719in}{1.550952in}}%
\pgfpathlineto{\pgfqpoint{1.489873in}{1.556142in}}%
\pgfpathlineto{\pgfqpoint{1.483382in}{1.558194in}}%
\pgfpathlineto{\pgfqpoint{1.476765in}{1.560146in}}%
\pgfpathlineto{\pgfqpoint{1.470030in}{1.561996in}}%
\pgfpathlineto{\pgfqpoint{1.463182in}{1.563743in}}%
\pgfpathclose%
\pgfusepath{fill}%
\end{pgfscope}%
\begin{pgfscope}%
\pgfpathrectangle{\pgfqpoint{0.329460in}{0.284240in}}{\pgfqpoint{1.989680in}{1.989680in}}%
\pgfusepath{clip}%
\pgfsetbuttcap%
\pgfsetroundjoin%
\definecolor{currentfill}{rgb}{0.271305,0.019942,0.347269}%
\pgfsetfillcolor{currentfill}%
\pgfsetlinewidth{0.000000pt}%
\definecolor{currentstroke}{rgb}{0.000000,0.000000,0.000000}%
\pgfsetstrokecolor{currentstroke}%
\pgfsetdash{}{0pt}%
\pgfpathmoveto{\pgfqpoint{1.838806in}{1.035947in}}%
\pgfpathlineto{\pgfqpoint{1.842140in}{1.033678in}}%
\pgfpathlineto{\pgfqpoint{1.845477in}{1.031590in}}%
\pgfpathlineto{\pgfqpoint{1.848818in}{1.029686in}}%
\pgfpathlineto{\pgfqpoint{1.852164in}{1.027970in}}%
\pgfpathlineto{\pgfqpoint{1.851899in}{1.019703in}}%
\pgfpathlineto{\pgfqpoint{1.851133in}{1.011434in}}%
\pgfpathlineto{\pgfqpoint{1.849863in}{1.003172in}}%
\pgfpathlineto{\pgfqpoint{1.848090in}{0.994924in}}%
\pgfpathlineto{\pgfqpoint{1.844754in}{0.996871in}}%
\pgfpathlineto{\pgfqpoint{1.841423in}{0.999006in}}%
\pgfpathlineto{\pgfqpoint{1.838097in}{1.001326in}}%
\pgfpathlineto{\pgfqpoint{1.834775in}{1.003826in}}%
\pgfpathlineto{\pgfqpoint{1.836517in}{1.011842in}}%
\pgfpathlineto{\pgfqpoint{1.837769in}{1.019873in}}%
\pgfpathlineto{\pgfqpoint{1.838532in}{1.027910in}}%
\pgfpathlineto{\pgfqpoint{1.838806in}{1.035947in}}%
\pgfpathclose%
\pgfusepath{fill}%
\end{pgfscope}%
\begin{pgfscope}%
\pgfpathrectangle{\pgfqpoint{0.329460in}{0.284240in}}{\pgfqpoint{1.989680in}{1.989680in}}%
\pgfusepath{clip}%
\pgfsetbuttcap%
\pgfsetroundjoin%
\definecolor{currentfill}{rgb}{0.274128,0.199721,0.498911}%
\pgfsetfillcolor{currentfill}%
\pgfsetlinewidth{0.000000pt}%
\definecolor{currentstroke}{rgb}{0.000000,0.000000,0.000000}%
\pgfsetstrokecolor{currentstroke}%
\pgfsetdash{}{0pt}%
\pgfpathmoveto{\pgfqpoint{1.756760in}{1.161181in}}%
\pgfpathlineto{\pgfqpoint{1.760011in}{1.155918in}}%
\pgfpathlineto{\pgfqpoint{1.763261in}{1.150751in}}%
\pgfpathlineto{\pgfqpoint{1.766512in}{1.145682in}}%
\pgfpathlineto{\pgfqpoint{1.769763in}{1.140714in}}%
\pgfpathlineto{\pgfqpoint{1.771129in}{1.133915in}}%
\pgfpathlineto{\pgfqpoint{1.772079in}{1.127090in}}%
\pgfpathlineto{\pgfqpoint{1.772611in}{1.120245in}}%
\pgfpathlineto{\pgfqpoint{1.772725in}{1.113388in}}%
\pgfpathlineto{\pgfqpoint{1.769437in}{1.118590in}}%
\pgfpathlineto{\pgfqpoint{1.766149in}{1.123893in}}%
\pgfpathlineto{\pgfqpoint{1.762862in}{1.129295in}}%
\pgfpathlineto{\pgfqpoint{1.759575in}{1.134792in}}%
\pgfpathlineto{\pgfqpoint{1.759479in}{1.141414in}}%
\pgfpathlineto{\pgfqpoint{1.758977in}{1.148023in}}%
\pgfpathlineto{\pgfqpoint{1.758070in}{1.154614in}}%
\pgfpathlineto{\pgfqpoint{1.756760in}{1.161181in}}%
\pgfpathclose%
\pgfusepath{fill}%
\end{pgfscope}%
\begin{pgfscope}%
\pgfpathrectangle{\pgfqpoint{0.329460in}{0.284240in}}{\pgfqpoint{1.989680in}{1.989680in}}%
\pgfusepath{clip}%
\pgfsetbuttcap%
\pgfsetroundjoin%
\definecolor{currentfill}{rgb}{0.231674,0.318106,0.544834}%
\pgfsetfillcolor{currentfill}%
\pgfsetlinewidth{0.000000pt}%
\definecolor{currentstroke}{rgb}{0.000000,0.000000,0.000000}%
\pgfsetstrokecolor{currentstroke}%
\pgfsetdash{}{0pt}%
\pgfpathmoveto{\pgfqpoint{1.709530in}{1.253764in}}%
\pgfpathlineto{\pgfqpoint{1.712705in}{1.247831in}}%
\pgfpathlineto{\pgfqpoint{1.715878in}{1.241956in}}%
\pgfpathlineto{\pgfqpoint{1.719050in}{1.236142in}}%
\pgfpathlineto{\pgfqpoint{1.722220in}{1.230393in}}%
\pgfpathlineto{\pgfqpoint{1.724902in}{1.224439in}}%
\pgfpathlineto{\pgfqpoint{1.727219in}{1.218440in}}%
\pgfpathlineto{\pgfqpoint{1.729166in}{1.212401in}}%
\pgfpathlineto{\pgfqpoint{1.730741in}{1.206328in}}%
\pgfpathlineto{\pgfqpoint{1.727485in}{1.212306in}}%
\pgfpathlineto{\pgfqpoint{1.724229in}{1.218348in}}%
\pgfpathlineto{\pgfqpoint{1.720970in}{1.224451in}}%
\pgfpathlineto{\pgfqpoint{1.717711in}{1.230612in}}%
\pgfpathlineto{\pgfqpoint{1.716202in}{1.236454in}}%
\pgfpathlineto{\pgfqpoint{1.714334in}{1.242264in}}%
\pgfpathlineto{\pgfqpoint{1.712108in}{1.248036in}}%
\pgfpathlineto{\pgfqpoint{1.709530in}{1.253764in}}%
\pgfpathclose%
\pgfusepath{fill}%
\end{pgfscope}%
\begin{pgfscope}%
\pgfpathrectangle{\pgfqpoint{0.329460in}{0.284240in}}{\pgfqpoint{1.989680in}{1.989680in}}%
\pgfusepath{clip}%
\pgfsetbuttcap%
\pgfsetroundjoin%
\definecolor{currentfill}{rgb}{0.163625,0.471133,0.558148}%
\pgfsetfillcolor{currentfill}%
\pgfsetlinewidth{0.000000pt}%
\definecolor{currentstroke}{rgb}{0.000000,0.000000,0.000000}%
\pgfsetstrokecolor{currentstroke}%
\pgfsetdash{}{0pt}%
\pgfpathmoveto{\pgfqpoint{1.630562in}{1.391137in}}%
\pgfpathlineto{\pgfqpoint{1.633461in}{1.385077in}}%
\pgfpathlineto{\pgfqpoint{1.636358in}{1.379031in}}%
\pgfpathlineto{\pgfqpoint{1.639252in}{1.373002in}}%
\pgfpathlineto{\pgfqpoint{1.642143in}{1.366992in}}%
\pgfpathlineto{\pgfqpoint{1.646815in}{1.362405in}}%
\pgfpathlineto{\pgfqpoint{1.651202in}{1.357744in}}%
\pgfpathlineto{\pgfqpoint{1.655299in}{1.353013in}}%
\pgfpathlineto{\pgfqpoint{1.659102in}{1.348216in}}%
\pgfpathlineto{\pgfqpoint{1.656033in}{1.354431in}}%
\pgfpathlineto{\pgfqpoint{1.652961in}{1.360666in}}%
\pgfpathlineto{\pgfqpoint{1.649887in}{1.366917in}}%
\pgfpathlineto{\pgfqpoint{1.646810in}{1.373182in}}%
\pgfpathlineto{\pgfqpoint{1.643167in}{1.377769in}}%
\pgfpathlineto{\pgfqpoint{1.639242in}{1.382293in}}%
\pgfpathlineto{\pgfqpoint{1.635039in}{1.386750in}}%
\pgfpathlineto{\pgfqpoint{1.630562in}{1.391137in}}%
\pgfpathclose%
\pgfusepath{fill}%
\end{pgfscope}%
\begin{pgfscope}%
\pgfpathrectangle{\pgfqpoint{0.329460in}{0.284240in}}{\pgfqpoint{1.989680in}{1.989680in}}%
\pgfusepath{clip}%
\pgfsetbuttcap%
\pgfsetroundjoin%
\definecolor{currentfill}{rgb}{0.268510,0.009605,0.335427}%
\pgfsetfillcolor{currentfill}%
\pgfsetlinewidth{0.000000pt}%
\definecolor{currentstroke}{rgb}{0.000000,0.000000,0.000000}%
\pgfsetstrokecolor{currentstroke}%
\pgfsetdash{}{0pt}%
\pgfpathmoveto{\pgfqpoint{1.892752in}{1.023526in}}%
\pgfpathlineto{\pgfqpoint{1.896177in}{1.024630in}}%
\pgfpathlineto{\pgfqpoint{1.899611in}{1.025982in}}%
\pgfpathlineto{\pgfqpoint{1.903053in}{1.027586in}}%
\pgfpathlineto{\pgfqpoint{1.906504in}{1.029447in}}%
\pgfpathlineto{\pgfqpoint{1.906287in}{1.020284in}}%
\pgfpathlineto{\pgfqpoint{1.905514in}{1.011117in}}%
\pgfpathlineto{\pgfqpoint{1.904182in}{1.001954in}}%
\pgfpathlineto{\pgfqpoint{1.902292in}{0.992805in}}%
\pgfpathlineto{\pgfqpoint{1.898848in}{0.991163in}}%
\pgfpathlineto{\pgfqpoint{1.895413in}{0.989778in}}%
\pgfpathlineto{\pgfqpoint{1.891987in}{0.988647in}}%
\pgfpathlineto{\pgfqpoint{1.888569in}{0.987764in}}%
\pgfpathlineto{\pgfqpoint{1.890431in}{0.996692in}}%
\pgfpathlineto{\pgfqpoint{1.891748in}{1.005635in}}%
\pgfpathlineto{\pgfqpoint{1.892521in}{1.014582in}}%
\pgfpathlineto{\pgfqpoint{1.892752in}{1.023526in}}%
\pgfpathclose%
\pgfusepath{fill}%
\end{pgfscope}%
\begin{pgfscope}%
\pgfpathrectangle{\pgfqpoint{0.329460in}{0.284240in}}{\pgfqpoint{1.989680in}{1.989680in}}%
\pgfusepath{clip}%
\pgfsetbuttcap%
\pgfsetroundjoin%
\definecolor{currentfill}{rgb}{0.134692,0.658636,0.517649}%
\pgfsetfillcolor{currentfill}%
\pgfsetlinewidth{0.000000pt}%
\definecolor{currentstroke}{rgb}{0.000000,0.000000,0.000000}%
\pgfsetstrokecolor{currentstroke}%
\pgfsetdash{}{0pt}%
\pgfpathmoveto{\pgfqpoint{1.206842in}{1.554236in}}%
\pgfpathlineto{\pgfqpoint{1.204920in}{1.549019in}}%
\pgfpathlineto{\pgfqpoint{1.203001in}{1.543768in}}%
\pgfpathlineto{\pgfqpoint{1.201084in}{1.538487in}}%
\pgfpathlineto{\pgfqpoint{1.199170in}{1.533177in}}%
\pgfpathlineto{\pgfqpoint{1.205886in}{1.535438in}}%
\pgfpathlineto{\pgfqpoint{1.212740in}{1.537595in}}%
\pgfpathlineto{\pgfqpoint{1.219725in}{1.539646in}}%
\pgfpathlineto{\pgfqpoint{1.226833in}{1.541589in}}%
\pgfpathlineto{\pgfqpoint{1.228397in}{1.546785in}}%
\pgfpathlineto{\pgfqpoint{1.229963in}{1.551954in}}%
\pgfpathlineto{\pgfqpoint{1.231531in}{1.557091in}}%
\pgfpathlineto{\pgfqpoint{1.233101in}{1.562195in}}%
\pgfpathlineto{\pgfqpoint{1.226352in}{1.560357in}}%
\pgfpathlineto{\pgfqpoint{1.219722in}{1.558416in}}%
\pgfpathlineto{\pgfqpoint{1.213217in}{1.556375in}}%
\pgfpathlineto{\pgfqpoint{1.206842in}{1.554236in}}%
\pgfpathclose%
\pgfusepath{fill}%
\end{pgfscope}%
\begin{pgfscope}%
\pgfpathrectangle{\pgfqpoint{0.329460in}{0.284240in}}{\pgfqpoint{1.989680in}{1.989680in}}%
\pgfusepath{clip}%
\pgfsetbuttcap%
\pgfsetroundjoin%
\definecolor{currentfill}{rgb}{0.274952,0.037752,0.364543}%
\pgfsetfillcolor{currentfill}%
\pgfsetlinewidth{0.000000pt}%
\definecolor{currentstroke}{rgb}{0.000000,0.000000,0.000000}%
\pgfsetstrokecolor{currentstroke}%
\pgfsetdash{}{0pt}%
\pgfpathmoveto{\pgfqpoint{1.825511in}{1.046743in}}%
\pgfpathlineto{\pgfqpoint{1.828830in}{1.043793in}}%
\pgfpathlineto{\pgfqpoint{1.832152in}{1.041008in}}%
\pgfpathlineto{\pgfqpoint{1.835477in}{1.038391in}}%
\pgfpathlineto{\pgfqpoint{1.838806in}{1.035947in}}%
\pgfpathlineto{\pgfqpoint{1.838532in}{1.027910in}}%
\pgfpathlineto{\pgfqpoint{1.837769in}{1.019873in}}%
\pgfpathlineto{\pgfqpoint{1.836517in}{1.011842in}}%
\pgfpathlineto{\pgfqpoint{1.834775in}{1.003826in}}%
\pgfpathlineto{\pgfqpoint{1.831457in}{1.006503in}}%
\pgfpathlineto{\pgfqpoint{1.828143in}{1.009353in}}%
\pgfpathlineto{\pgfqpoint{1.824833in}{1.012371in}}%
\pgfpathlineto{\pgfqpoint{1.821526in}{1.015555in}}%
\pgfpathlineto{\pgfqpoint{1.823237in}{1.023338in}}%
\pgfpathlineto{\pgfqpoint{1.824471in}{1.031135in}}%
\pgfpathlineto{\pgfqpoint{1.825229in}{1.038940in}}%
\pgfpathlineto{\pgfqpoint{1.825511in}{1.046743in}}%
\pgfpathclose%
\pgfusepath{fill}%
\end{pgfscope}%
\begin{pgfscope}%
\pgfpathrectangle{\pgfqpoint{0.329460in}{0.284240in}}{\pgfqpoint{1.989680in}{1.989680in}}%
\pgfusepath{clip}%
\pgfsetbuttcap%
\pgfsetroundjoin%
\definecolor{currentfill}{rgb}{0.201239,0.383670,0.554294}%
\pgfsetfillcolor{currentfill}%
\pgfsetlinewidth{0.000000pt}%
\definecolor{currentstroke}{rgb}{0.000000,0.000000,0.000000}%
\pgfsetstrokecolor{currentstroke}%
\pgfsetdash{}{0pt}%
\pgfpathmoveto{\pgfqpoint{0.698409in}{1.239174in}}%
\pgfpathlineto{\pgfqpoint{0.694607in}{1.250919in}}%
\pgfpathlineto{\pgfqpoint{0.690787in}{1.263089in}}%
\pgfpathlineto{\pgfqpoint{0.686946in}{1.275693in}}%
\pgfpathlineto{\pgfqpoint{0.683085in}{1.288737in}}%
\pgfpathlineto{\pgfqpoint{0.685751in}{1.299285in}}%
\pgfpathlineto{\pgfqpoint{0.689067in}{1.309767in}}%
\pgfpathlineto{\pgfqpoint{0.693026in}{1.320174in}}%
\pgfpathlineto{\pgfqpoint{0.697622in}{1.330497in}}%
\pgfpathlineto{\pgfqpoint{0.701374in}{1.317300in}}%
\pgfpathlineto{\pgfqpoint{0.705106in}{1.304540in}}%
\pgfpathlineto{\pgfqpoint{0.708819in}{1.292212in}}%
\pgfpathlineto{\pgfqpoint{0.712514in}{1.280307in}}%
\pgfpathlineto{\pgfqpoint{0.708047in}{1.270138in}}%
\pgfpathlineto{\pgfqpoint{0.704203in}{1.259887in}}%
\pgfpathlineto{\pgfqpoint{0.700988in}{1.249562in}}%
\pgfpathlineto{\pgfqpoint{0.698409in}{1.239174in}}%
\pgfpathclose%
\pgfusepath{fill}%
\end{pgfscope}%
\begin{pgfscope}%
\pgfpathrectangle{\pgfqpoint{0.329460in}{0.284240in}}{\pgfqpoint{1.989680in}{1.989680in}}%
\pgfusepath{clip}%
\pgfsetbuttcap%
\pgfsetroundjoin%
\definecolor{currentfill}{rgb}{0.166383,0.690856,0.496502}%
\pgfsetfillcolor{currentfill}%
\pgfsetlinewidth{0.000000pt}%
\definecolor{currentstroke}{rgb}{0.000000,0.000000,0.000000}%
\pgfsetstrokecolor{currentstroke}%
\pgfsetdash{}{0pt}%
\pgfpathmoveto{\pgfqpoint{1.409390in}{1.592396in}}%
\pgfpathlineto{\pgfqpoint{1.410210in}{1.587597in}}%
\pgfpathlineto{\pgfqpoint{1.411029in}{1.582756in}}%
\pgfpathlineto{\pgfqpoint{1.411847in}{1.577876in}}%
\pgfpathlineto{\pgfqpoint{1.412664in}{1.572959in}}%
\pgfpathlineto{\pgfqpoint{1.420110in}{1.571973in}}%
\pgfpathlineto{\pgfqpoint{1.427491in}{1.570876in}}%
\pgfpathlineto{\pgfqpoint{1.434801in}{1.569667in}}%
\pgfpathlineto{\pgfqpoint{1.433689in}{1.574631in}}%
\pgfpathlineto{\pgfqpoint{1.432576in}{1.579558in}}%
\pgfpathlineto{\pgfqpoint{1.431461in}{1.584445in}}%
\pgfpathlineto{\pgfqpoint{1.430344in}{1.589291in}}%
\pgfpathlineto{\pgfqpoint{1.423426in}{1.590431in}}%
\pgfpathlineto{\pgfqpoint{1.416438in}{1.591467in}}%
\pgfpathlineto{\pgfqpoint{1.409390in}{1.592396in}}%
\pgfpathclose%
\pgfusepath{fill}%
\end{pgfscope}%
\begin{pgfscope}%
\pgfpathrectangle{\pgfqpoint{0.329460in}{0.284240in}}{\pgfqpoint{1.989680in}{1.989680in}}%
\pgfusepath{clip}%
\pgfsetbuttcap%
\pgfsetroundjoin%
\definecolor{currentfill}{rgb}{0.195860,0.395433,0.555276}%
\pgfsetfillcolor{currentfill}%
\pgfsetlinewidth{0.000000pt}%
\definecolor{currentstroke}{rgb}{0.000000,0.000000,0.000000}%
\pgfsetstrokecolor{currentstroke}%
\pgfsetdash{}{0pt}%
\pgfpathmoveto{\pgfqpoint{1.016180in}{1.298291in}}%
\pgfpathlineto{\pgfqpoint{1.012968in}{1.291946in}}%
\pgfpathlineto{\pgfqpoint{1.009758in}{1.285635in}}%
\pgfpathlineto{\pgfqpoint{1.006549in}{1.279363in}}%
\pgfpathlineto{\pgfqpoint{1.003343in}{1.273132in}}%
\pgfpathlineto{\pgfqpoint{1.005857in}{1.278630in}}%
\pgfpathlineto{\pgfqpoint{1.008708in}{1.284082in}}%
\pgfpathlineto{\pgfqpoint{1.011892in}{1.289480in}}%
\pgfpathlineto{\pgfqpoint{1.015407in}{1.294821in}}%
\pgfpathlineto{\pgfqpoint{1.018491in}{1.300832in}}%
\pgfpathlineto{\pgfqpoint{1.021578in}{1.306883in}}%
\pgfpathlineto{\pgfqpoint{1.024666in}{1.312973in}}%
\pgfpathlineto{\pgfqpoint{1.027757in}{1.319099in}}%
\pgfpathlineto{\pgfqpoint{1.024382in}{1.313974in}}%
\pgfpathlineto{\pgfqpoint{1.021325in}{1.308795in}}%
\pgfpathlineto{\pgfqpoint{1.018590in}{1.303566in}}%
\pgfpathlineto{\pgfqpoint{1.016180in}{1.298291in}}%
\pgfpathclose%
\pgfusepath{fill}%
\end{pgfscope}%
\begin{pgfscope}%
\pgfpathrectangle{\pgfqpoint{0.329460in}{0.284240in}}{\pgfqpoint{1.989680in}{1.989680in}}%
\pgfusepath{clip}%
\pgfsetbuttcap%
\pgfsetroundjoin%
\definecolor{currentfill}{rgb}{0.166383,0.690856,0.496502}%
\pgfsetfillcolor{currentfill}%
\pgfsetlinewidth{0.000000pt}%
\definecolor{currentstroke}{rgb}{0.000000,0.000000,0.000000}%
\pgfsetstrokecolor{currentstroke}%
\pgfsetdash{}{0pt}%
\pgfpathmoveto{\pgfqpoint{1.265943in}{1.588190in}}%
\pgfpathlineto{\pgfqpoint{1.264741in}{1.583327in}}%
\pgfpathlineto{\pgfqpoint{1.263540in}{1.578424in}}%
\pgfpathlineto{\pgfqpoint{1.262341in}{1.573481in}}%
\pgfpathlineto{\pgfqpoint{1.261143in}{1.568500in}}%
\pgfpathlineto{\pgfqpoint{1.268383in}{1.569807in}}%
\pgfpathlineto{\pgfqpoint{1.275700in}{1.571003in}}%
\pgfpathlineto{\pgfqpoint{1.283089in}{1.572088in}}%
\pgfpathlineto{\pgfqpoint{1.290542in}{1.573061in}}%
\pgfpathlineto{\pgfqpoint{1.291348in}{1.577977in}}%
\pgfpathlineto{\pgfqpoint{1.292155in}{1.582856in}}%
\pgfpathlineto{\pgfqpoint{1.292963in}{1.587695in}}%
\pgfpathlineto{\pgfqpoint{1.293772in}{1.592493in}}%
\pgfpathlineto{\pgfqpoint{1.286717in}{1.591575in}}%
\pgfpathlineto{\pgfqpoint{1.279723in}{1.590551in}}%
\pgfpathlineto{\pgfqpoint{1.272796in}{1.589422in}}%
\pgfpathlineto{\pgfqpoint{1.265943in}{1.588190in}}%
\pgfpathclose%
\pgfusepath{fill}%
\end{pgfscope}%
\begin{pgfscope}%
\pgfpathrectangle{\pgfqpoint{0.329460in}{0.284240in}}{\pgfqpoint{1.989680in}{1.989680in}}%
\pgfusepath{clip}%
\pgfsetbuttcap%
\pgfsetroundjoin%
\definecolor{currentfill}{rgb}{0.276194,0.190074,0.493001}%
\pgfsetfillcolor{currentfill}%
\pgfsetlinewidth{0.000000pt}%
\definecolor{currentstroke}{rgb}{0.000000,0.000000,0.000000}%
\pgfsetstrokecolor{currentstroke}%
\pgfsetdash{}{0pt}%
\pgfpathmoveto{\pgfqpoint{0.739367in}{1.089263in}}%
\pgfpathlineto{\pgfqpoint{0.735720in}{1.096213in}}%
\pgfpathlineto{\pgfqpoint{0.732059in}{1.103513in}}%
\pgfpathlineto{\pgfqpoint{0.728382in}{1.111171in}}%
\pgfpathlineto{\pgfqpoint{0.724691in}{1.119192in}}%
\pgfpathlineto{\pgfqpoint{0.724629in}{1.129373in}}%
\pgfpathlineto{\pgfqpoint{0.725193in}{1.139534in}}%
\pgfpathlineto{\pgfqpoint{0.726378in}{1.149665in}}%
\pgfpathlineto{\pgfqpoint{0.728180in}{1.159755in}}%
\pgfpathlineto{\pgfqpoint{0.731828in}{1.151547in}}%
\pgfpathlineto{\pgfqpoint{0.735462in}{1.143700in}}%
\pgfpathlineto{\pgfqpoint{0.739080in}{1.136209in}}%
\pgfpathlineto{\pgfqpoint{0.742685in}{1.129067in}}%
\pgfpathlineto{\pgfqpoint{0.740947in}{1.119165in}}%
\pgfpathlineto{\pgfqpoint{0.739812in}{1.109223in}}%
\pgfpathlineto{\pgfqpoint{0.739285in}{1.099253in}}%
\pgfpathlineto{\pgfqpoint{0.739367in}{1.089263in}}%
\pgfpathclose%
\pgfusepath{fill}%
\end{pgfscope}%
\begin{pgfscope}%
\pgfpathrectangle{\pgfqpoint{0.329460in}{0.284240in}}{\pgfqpoint{1.989680in}{1.989680in}}%
\pgfusepath{clip}%
\pgfsetbuttcap%
\pgfsetroundjoin%
\definecolor{currentfill}{rgb}{0.272594,0.025563,0.353093}%
\pgfsetfillcolor{currentfill}%
\pgfsetlinewidth{0.000000pt}%
\definecolor{currentstroke}{rgb}{0.000000,0.000000,0.000000}%
\pgfsetstrokecolor{currentstroke}%
\pgfsetdash{}{0pt}%
\pgfpathmoveto{\pgfqpoint{1.906504in}{1.029447in}}%
\pgfpathlineto{\pgfqpoint{1.909963in}{1.031569in}}%
\pgfpathlineto{\pgfqpoint{1.913431in}{1.033957in}}%
\pgfpathlineto{\pgfqpoint{1.916909in}{1.036616in}}%
\pgfpathlineto{\pgfqpoint{1.920396in}{1.039551in}}%
\pgfpathlineto{\pgfqpoint{1.920194in}{1.030174in}}%
\pgfpathlineto{\pgfqpoint{1.919421in}{1.020791in}}%
\pgfpathlineto{\pgfqpoint{1.918076in}{1.011412in}}%
\pgfpathlineto{\pgfqpoint{1.916158in}{1.002046in}}%
\pgfpathlineto{\pgfqpoint{1.912677in}{0.999325in}}%
\pgfpathlineto{\pgfqpoint{1.909206in}{0.996881in}}%
\pgfpathlineto{\pgfqpoint{1.905744in}{0.994710in}}%
\pgfpathlineto{\pgfqpoint{1.902292in}{0.992805in}}%
\pgfpathlineto{\pgfqpoint{1.904182in}{1.001954in}}%
\pgfpathlineto{\pgfqpoint{1.905514in}{1.011117in}}%
\pgfpathlineto{\pgfqpoint{1.906287in}{1.020284in}}%
\pgfpathlineto{\pgfqpoint{1.906504in}{1.029447in}}%
\pgfpathclose%
\pgfusepath{fill}%
\end{pgfscope}%
\begin{pgfscope}%
\pgfpathrectangle{\pgfqpoint{0.329460in}{0.284240in}}{\pgfqpoint{1.989680in}{1.989680in}}%
\pgfusepath{clip}%
\pgfsetbuttcap%
\pgfsetroundjoin%
\definecolor{currentfill}{rgb}{0.163625,0.471133,0.558148}%
\pgfsetfillcolor{currentfill}%
\pgfsetlinewidth{0.000000pt}%
\definecolor{currentstroke}{rgb}{0.000000,0.000000,0.000000}%
\pgfsetstrokecolor{currentstroke}%
\pgfsetdash{}{0pt}%
\pgfpathmoveto{\pgfqpoint{1.052569in}{1.369056in}}%
\pgfpathlineto{\pgfqpoint{1.049458in}{1.362744in}}%
\pgfpathlineto{\pgfqpoint{1.046351in}{1.356445in}}%
\pgfpathlineto{\pgfqpoint{1.043246in}{1.350164in}}%
\pgfpathlineto{\pgfqpoint{1.040143in}{1.343901in}}%
\pgfpathlineto{\pgfqpoint{1.043681in}{1.348752in}}%
\pgfpathlineto{\pgfqpoint{1.047517in}{1.353542in}}%
\pgfpathlineto{\pgfqpoint{1.051647in}{1.358266in}}%
\pgfpathlineto{\pgfqpoint{1.056065in}{1.362919in}}%
\pgfpathlineto{\pgfqpoint{1.059000in}{1.368973in}}%
\pgfpathlineto{\pgfqpoint{1.061937in}{1.375046in}}%
\pgfpathlineto{\pgfqpoint{1.064878in}{1.381137in}}%
\pgfpathlineto{\pgfqpoint{1.067820in}{1.387241in}}%
\pgfpathlineto{\pgfqpoint{1.063586in}{1.382791in}}%
\pgfpathlineto{\pgfqpoint{1.059630in}{1.378274in}}%
\pgfpathlineto{\pgfqpoint{1.055956in}{1.373694in}}%
\pgfpathlineto{\pgfqpoint{1.052569in}{1.369056in}}%
\pgfpathclose%
\pgfusepath{fill}%
\end{pgfscope}%
\begin{pgfscope}%
\pgfpathrectangle{\pgfqpoint{0.329460in}{0.284240in}}{\pgfqpoint{1.989680in}{1.989680in}}%
\pgfusepath{clip}%
\pgfsetbuttcap%
\pgfsetroundjoin%
\definecolor{currentfill}{rgb}{0.122606,0.585371,0.546557}%
\pgfsetfillcolor{currentfill}%
\pgfsetlinewidth{0.000000pt}%
\definecolor{currentstroke}{rgb}{0.000000,0.000000,0.000000}%
\pgfsetstrokecolor{currentstroke}%
\pgfsetdash{}{0pt}%
\pgfpathmoveto{\pgfqpoint{1.556288in}{1.491788in}}%
\pgfpathlineto{\pgfqpoint{1.558740in}{1.486112in}}%
\pgfpathlineto{\pgfqpoint{1.561189in}{1.480422in}}%
\pgfpathlineto{\pgfqpoint{1.563635in}{1.474718in}}%
\pgfpathlineto{\pgfqpoint{1.566079in}{1.469005in}}%
\pgfpathlineto{\pgfqpoint{1.572002in}{1.465655in}}%
\pgfpathlineto{\pgfqpoint{1.577716in}{1.462213in}}%
\pgfpathlineto{\pgfqpoint{1.583213in}{1.458683in}}%
\pgfpathlineto{\pgfqpoint{1.588488in}{1.455068in}}%
\pgfpathlineto{\pgfqpoint{1.585784in}{1.460954in}}%
\pgfpathlineto{\pgfqpoint{1.583077in}{1.466829in}}%
\pgfpathlineto{\pgfqpoint{1.580366in}{1.472691in}}%
\pgfpathlineto{\pgfqpoint{1.577653in}{1.478538in}}%
\pgfpathlineto{\pgfqpoint{1.572625in}{1.481975in}}%
\pgfpathlineto{\pgfqpoint{1.567384in}{1.485330in}}%
\pgfpathlineto{\pgfqpoint{1.561937in}{1.488603in}}%
\pgfpathlineto{\pgfqpoint{1.556288in}{1.491788in}}%
\pgfpathclose%
\pgfusepath{fill}%
\end{pgfscope}%
\begin{pgfscope}%
\pgfpathrectangle{\pgfqpoint{0.329460in}{0.284240in}}{\pgfqpoint{1.989680in}{1.989680in}}%
\pgfusepath{clip}%
\pgfsetbuttcap%
\pgfsetroundjoin%
\definecolor{currentfill}{rgb}{0.267004,0.004874,0.329415}%
\pgfsetfillcolor{currentfill}%
\pgfsetlinewidth{0.000000pt}%
\definecolor{currentstroke}{rgb}{0.000000,0.000000,0.000000}%
\pgfsetstrokecolor{currentstroke}%
\pgfsetdash{}{0pt}%
\pgfpathmoveto{\pgfqpoint{0.842929in}{0.981587in}}%
\pgfpathlineto{\pgfqpoint{0.839576in}{0.980593in}}%
\pgfpathlineto{\pgfqpoint{0.836217in}{0.979812in}}%
\pgfpathlineto{\pgfqpoint{0.832851in}{0.979248in}}%
\pgfpathlineto{\pgfqpoint{0.829479in}{0.978907in}}%
\pgfpathlineto{\pgfqpoint{0.827173in}{0.987591in}}%
\pgfpathlineto{\pgfqpoint{0.825399in}{0.996297in}}%
\pgfpathlineto{\pgfqpoint{0.824156in}{1.005017in}}%
\pgfpathlineto{\pgfqpoint{0.823443in}{1.013741in}}%
\pgfpathlineto{\pgfqpoint{0.826836in}{1.013857in}}%
\pgfpathlineto{\pgfqpoint{0.830223in}{1.014194in}}%
\pgfpathlineto{\pgfqpoint{0.833603in}{1.014748in}}%
\pgfpathlineto{\pgfqpoint{0.836978in}{1.015514in}}%
\pgfpathlineto{\pgfqpoint{0.837690in}{1.007016in}}%
\pgfpathlineto{\pgfqpoint{0.838919in}{0.998523in}}%
\pgfpathlineto{\pgfqpoint{0.840665in}{0.990044in}}%
\pgfpathlineto{\pgfqpoint{0.842929in}{0.981587in}}%
\pgfpathclose%
\pgfusepath{fill}%
\end{pgfscope}%
\begin{pgfscope}%
\pgfpathrectangle{\pgfqpoint{0.329460in}{0.284240in}}{\pgfqpoint{1.989680in}{1.989680in}}%
\pgfusepath{clip}%
\pgfsetbuttcap%
\pgfsetroundjoin%
\definecolor{currentfill}{rgb}{0.231674,0.318106,0.544834}%
\pgfsetfillcolor{currentfill}%
\pgfsetlinewidth{0.000000pt}%
\definecolor{currentstroke}{rgb}{0.000000,0.000000,0.000000}%
\pgfsetstrokecolor{currentstroke}%
\pgfsetdash{}{0pt}%
\pgfpathmoveto{\pgfqpoint{0.983626in}{1.225396in}}%
\pgfpathlineto{\pgfqpoint{0.980355in}{1.219183in}}%
\pgfpathlineto{\pgfqpoint{0.977085in}{1.213029in}}%
\pgfpathlineto{\pgfqpoint{0.973816in}{1.206935in}}%
\pgfpathlineto{\pgfqpoint{0.970548in}{1.200906in}}%
\pgfpathlineto{\pgfqpoint{0.971790in}{1.207004in}}%
\pgfpathlineto{\pgfqpoint{0.973407in}{1.213074in}}%
\pgfpathlineto{\pgfqpoint{0.975396in}{1.219109in}}%
\pgfpathlineto{\pgfqpoint{0.977753in}{1.225103in}}%
\pgfpathlineto{\pgfqpoint{0.980946in}{1.230903in}}%
\pgfpathlineto{\pgfqpoint{0.984142in}{1.236766in}}%
\pgfpathlineto{\pgfqpoint{0.987338in}{1.242691in}}%
\pgfpathlineto{\pgfqpoint{0.990536in}{1.248674in}}%
\pgfpathlineto{\pgfqpoint{0.988271in}{1.242907in}}%
\pgfpathlineto{\pgfqpoint{0.986363in}{1.237101in}}%
\pgfpathlineto{\pgfqpoint{0.984814in}{1.231262in}}%
\pgfpathlineto{\pgfqpoint{0.983626in}{1.225396in}}%
\pgfpathclose%
\pgfusepath{fill}%
\end{pgfscope}%
\begin{pgfscope}%
\pgfpathrectangle{\pgfqpoint{0.329460in}{0.284240in}}{\pgfqpoint{1.989680in}{1.989680in}}%
\pgfusepath{clip}%
\pgfsetbuttcap%
\pgfsetroundjoin%
\definecolor{currentfill}{rgb}{0.279566,0.067836,0.391917}%
\pgfsetfillcolor{currentfill}%
\pgfsetlinewidth{0.000000pt}%
\definecolor{currentstroke}{rgb}{0.000000,0.000000,0.000000}%
\pgfsetstrokecolor{currentstroke}%
\pgfsetdash{}{0pt}%
\pgfpathmoveto{\pgfqpoint{1.812267in}{1.060117in}}%
\pgfpathlineto{\pgfqpoint{1.815574in}{1.056545in}}%
\pgfpathlineto{\pgfqpoint{1.818884in}{1.053123in}}%
\pgfpathlineto{\pgfqpoint{1.822196in}{1.049855in}}%
\pgfpathlineto{\pgfqpoint{1.825511in}{1.046743in}}%
\pgfpathlineto{\pgfqpoint{1.825229in}{1.038940in}}%
\pgfpathlineto{\pgfqpoint{1.824471in}{1.031135in}}%
\pgfpathlineto{\pgfqpoint{1.823237in}{1.023338in}}%
\pgfpathlineto{\pgfqpoint{1.821526in}{1.015555in}}%
\pgfpathlineto{\pgfqpoint{1.818223in}{1.018900in}}%
\pgfpathlineto{\pgfqpoint{1.814923in}{1.022403in}}%
\pgfpathlineto{\pgfqpoint{1.811626in}{1.026060in}}%
\pgfpathlineto{\pgfqpoint{1.808331in}{1.029867in}}%
\pgfpathlineto{\pgfqpoint{1.810009in}{1.037414in}}%
\pgfpathlineto{\pgfqpoint{1.811224in}{1.044977in}}%
\pgfpathlineto{\pgfqpoint{1.811976in}{1.052548in}}%
\pgfpathlineto{\pgfqpoint{1.812267in}{1.060117in}}%
\pgfpathclose%
\pgfusepath{fill}%
\end{pgfscope}%
\begin{pgfscope}%
\pgfpathrectangle{\pgfqpoint{0.329460in}{0.284240in}}{\pgfqpoint{1.989680in}{1.989680in}}%
\pgfusepath{clip}%
\pgfsetbuttcap%
\pgfsetroundjoin%
\definecolor{currentfill}{rgb}{0.268510,0.009605,0.335427}%
\pgfsetfillcolor{currentfill}%
\pgfsetlinewidth{0.000000pt}%
\definecolor{currentstroke}{rgb}{0.000000,0.000000,0.000000}%
\pgfsetstrokecolor{currentstroke}%
\pgfsetdash{}{0pt}%
\pgfpathmoveto{\pgfqpoint{0.856285in}{0.987612in}}%
\pgfpathlineto{\pgfqpoint{0.852954in}{0.985807in}}%
\pgfpathlineto{\pgfqpoint{0.849618in}{0.984198in}}%
\pgfpathlineto{\pgfqpoint{0.846276in}{0.982790in}}%
\pgfpathlineto{\pgfqpoint{0.842929in}{0.981587in}}%
\pgfpathlineto{\pgfqpoint{0.840665in}{0.990044in}}%
\pgfpathlineto{\pgfqpoint{0.838919in}{0.998523in}}%
\pgfpathlineto{\pgfqpoint{0.837690in}{1.007016in}}%
\pgfpathlineto{\pgfqpoint{0.836978in}{1.015514in}}%
\pgfpathlineto{\pgfqpoint{0.840347in}{1.016488in}}%
\pgfpathlineto{\pgfqpoint{0.843710in}{1.017667in}}%
\pgfpathlineto{\pgfqpoint{0.847068in}{1.019047in}}%
\pgfpathlineto{\pgfqpoint{0.850422in}{1.020622in}}%
\pgfpathlineto{\pgfqpoint{0.851132in}{1.012353in}}%
\pgfpathlineto{\pgfqpoint{0.852346in}{1.004089in}}%
\pgfpathlineto{\pgfqpoint{0.854064in}{0.995840in}}%
\pgfpathlineto{\pgfqpoint{0.856285in}{0.987612in}}%
\pgfpathclose%
\pgfusepath{fill}%
\end{pgfscope}%
\begin{pgfscope}%
\pgfpathrectangle{\pgfqpoint{0.329460in}{0.284240in}}{\pgfqpoint{1.989680in}{1.989680in}}%
\pgfusepath{clip}%
\pgfsetbuttcap%
\pgfsetroundjoin%
\definecolor{currentfill}{rgb}{0.120081,0.622161,0.534946}%
\pgfsetfillcolor{currentfill}%
\pgfsetlinewidth{0.000000pt}%
\definecolor{currentstroke}{rgb}{0.000000,0.000000,0.000000}%
\pgfsetstrokecolor{currentstroke}%
\pgfsetdash{}{0pt}%
\pgfpathmoveto{\pgfqpoint{1.523153in}{1.525501in}}%
\pgfpathlineto{\pgfqpoint{1.525318in}{1.520061in}}%
\pgfpathlineto{\pgfqpoint{1.527481in}{1.514597in}}%
\pgfpathlineto{\pgfqpoint{1.529641in}{1.509111in}}%
\pgfpathlineto{\pgfqpoint{1.531798in}{1.503605in}}%
\pgfpathlineto{\pgfqpoint{1.538193in}{1.500795in}}%
\pgfpathlineto{\pgfqpoint{1.544411in}{1.497887in}}%
\pgfpathlineto{\pgfqpoint{1.550444in}{1.494884in}}%
\pgfpathlineto{\pgfqpoint{1.556288in}{1.491788in}}%
\pgfpathlineto{\pgfqpoint{1.553834in}{1.497446in}}%
\pgfpathlineto{\pgfqpoint{1.551376in}{1.503084in}}%
\pgfpathlineto{\pgfqpoint{1.548915in}{1.508700in}}%
\pgfpathlineto{\pgfqpoint{1.546452in}{1.514292in}}%
\pgfpathlineto{\pgfqpoint{1.540893in}{1.517228in}}%
\pgfpathlineto{\pgfqpoint{1.535153in}{1.520077in}}%
\pgfpathlineto{\pgfqpoint{1.529237in}{1.522835in}}%
\pgfpathlineto{\pgfqpoint{1.523153in}{1.525501in}}%
\pgfpathclose%
\pgfusepath{fill}%
\end{pgfscope}%
\begin{pgfscope}%
\pgfpathrectangle{\pgfqpoint{0.329460in}{0.284240in}}{\pgfqpoint{1.989680in}{1.989680in}}%
\pgfusepath{clip}%
\pgfsetbuttcap%
\pgfsetroundjoin%
\definecolor{currentfill}{rgb}{0.274128,0.199721,0.498911}%
\pgfsetfillcolor{currentfill}%
\pgfsetlinewidth{0.000000pt}%
\definecolor{currentstroke}{rgb}{0.000000,0.000000,0.000000}%
\pgfsetstrokecolor{currentstroke}%
\pgfsetdash{}{0pt}%
\pgfpathmoveto{\pgfqpoint{0.943057in}{1.128902in}}%
\pgfpathlineto{\pgfqpoint{0.939769in}{1.123353in}}%
\pgfpathlineto{\pgfqpoint{0.936481in}{1.117898in}}%
\pgfpathlineto{\pgfqpoint{0.933192in}{1.112542in}}%
\pgfpathlineto{\pgfqpoint{0.929903in}{1.107288in}}%
\pgfpathlineto{\pgfqpoint{0.929642in}{1.114150in}}%
\pgfpathlineto{\pgfqpoint{0.929802in}{1.121007in}}%
\pgfpathlineto{\pgfqpoint{0.930381in}{1.127849in}}%
\pgfpathlineto{\pgfqpoint{0.931378in}{1.134672in}}%
\pgfpathlineto{\pgfqpoint{0.934641in}{1.139691in}}%
\pgfpathlineto{\pgfqpoint{0.937905in}{1.144812in}}%
\pgfpathlineto{\pgfqpoint{0.941168in}{1.150031in}}%
\pgfpathlineto{\pgfqpoint{0.944431in}{1.155345in}}%
\pgfpathlineto{\pgfqpoint{0.943479in}{1.148756in}}%
\pgfpathlineto{\pgfqpoint{0.942931in}{1.142149in}}%
\pgfpathlineto{\pgfqpoint{0.942790in}{1.135528in}}%
\pgfpathlineto{\pgfqpoint{0.943057in}{1.128902in}}%
\pgfpathclose%
\pgfusepath{fill}%
\end{pgfscope}%
\begin{pgfscope}%
\pgfpathrectangle{\pgfqpoint{0.329460in}{0.284240in}}{\pgfqpoint{1.989680in}{1.989680in}}%
\pgfusepath{clip}%
\pgfsetbuttcap%
\pgfsetroundjoin%
\definecolor{currentfill}{rgb}{0.267004,0.004874,0.329415}%
\pgfsetfillcolor{currentfill}%
\pgfsetlinewidth{0.000000pt}%
\definecolor{currentstroke}{rgb}{0.000000,0.000000,0.000000}%
\pgfsetstrokecolor{currentstroke}%
\pgfsetdash{}{0pt}%
\pgfpathmoveto{\pgfqpoint{0.829479in}{0.978907in}}%
\pgfpathlineto{\pgfqpoint{0.826100in}{0.978792in}}%
\pgfpathlineto{\pgfqpoint{0.822714in}{0.978907in}}%
\pgfpathlineto{\pgfqpoint{0.819321in}{0.979257in}}%
\pgfpathlineto{\pgfqpoint{0.815920in}{0.979847in}}%
\pgfpathlineto{\pgfqpoint{0.813572in}{0.988755in}}%
\pgfpathlineto{\pgfqpoint{0.811771in}{0.997685in}}%
\pgfpathlineto{\pgfqpoint{0.810514in}{1.006629in}}%
\pgfpathlineto{\pgfqpoint{0.809801in}{1.015576in}}%
\pgfpathlineto{\pgfqpoint{0.813223in}{1.014764in}}%
\pgfpathlineto{\pgfqpoint{0.816637in}{1.014190in}}%
\pgfpathlineto{\pgfqpoint{0.820043in}{1.013851in}}%
\pgfpathlineto{\pgfqpoint{0.823443in}{1.013741in}}%
\pgfpathlineto{\pgfqpoint{0.824156in}{1.005017in}}%
\pgfpathlineto{\pgfqpoint{0.825399in}{0.996297in}}%
\pgfpathlineto{\pgfqpoint{0.827173in}{0.987591in}}%
\pgfpathlineto{\pgfqpoint{0.829479in}{0.978907in}}%
\pgfpathclose%
\pgfusepath{fill}%
\end{pgfscope}%
\begin{pgfscope}%
\pgfpathrectangle{\pgfqpoint{0.329460in}{0.284240in}}{\pgfqpoint{1.989680in}{1.989680in}}%
\pgfusepath{clip}%
\pgfsetbuttcap%
\pgfsetroundjoin%
\definecolor{currentfill}{rgb}{0.260571,0.246922,0.522828}%
\pgfsetfillcolor{currentfill}%
\pgfsetlinewidth{0.000000pt}%
\definecolor{currentstroke}{rgb}{0.000000,0.000000,0.000000}%
\pgfsetstrokecolor{currentstroke}%
\pgfsetdash{}{0pt}%
\pgfpathmoveto{\pgfqpoint{1.972079in}{1.168683in}}%
\pgfpathlineto{\pgfqpoint{1.975724in}{1.177298in}}%
\pgfpathlineto{\pgfqpoint{1.979386in}{1.186287in}}%
\pgfpathlineto{\pgfqpoint{1.983063in}{1.195656in}}%
\pgfpathlineto{\pgfqpoint{1.986757in}{1.205409in}}%
\pgfpathlineto{\pgfqpoint{1.989183in}{1.195185in}}%
\pgfpathlineto{\pgfqpoint{1.990982in}{1.184909in}}%
\pgfpathlineto{\pgfqpoint{1.992149in}{1.174593in}}%
\pgfpathlineto{\pgfqpoint{1.992680in}{1.164247in}}%
\pgfpathlineto{\pgfqpoint{1.988927in}{1.154671in}}%
\pgfpathlineto{\pgfqpoint{1.985192in}{1.145482in}}%
\pgfpathlineto{\pgfqpoint{1.981473in}{1.136675in}}%
\pgfpathlineto{\pgfqpoint{1.977770in}{1.128243in}}%
\pgfpathlineto{\pgfqpoint{1.977276in}{1.138407in}}%
\pgfpathlineto{\pgfqpoint{1.976160in}{1.148541in}}%
\pgfpathlineto{\pgfqpoint{1.974426in}{1.158636in}}%
\pgfpathlineto{\pgfqpoint{1.972079in}{1.168683in}}%
\pgfpathclose%
\pgfusepath{fill}%
\end{pgfscope}%
\begin{pgfscope}%
\pgfpathrectangle{\pgfqpoint{0.329460in}{0.284240in}}{\pgfqpoint{1.989680in}{1.989680in}}%
\pgfusepath{clip}%
\pgfsetbuttcap%
\pgfsetroundjoin%
\definecolor{currentfill}{rgb}{0.271305,0.019942,0.347269}%
\pgfsetfillcolor{currentfill}%
\pgfsetlinewidth{0.000000pt}%
\definecolor{currentstroke}{rgb}{0.000000,0.000000,0.000000}%
\pgfsetstrokecolor{currentstroke}%
\pgfsetdash{}{0pt}%
\pgfpathmoveto{\pgfqpoint{0.869560in}{0.996720in}}%
\pgfpathlineto{\pgfqpoint{0.866248in}{0.994168in}}%
\pgfpathlineto{\pgfqpoint{0.862932in}{0.991797in}}%
\pgfpathlineto{\pgfqpoint{0.859611in}{0.989610in}}%
\pgfpathlineto{\pgfqpoint{0.856285in}{0.987612in}}%
\pgfpathlineto{\pgfqpoint{0.854064in}{0.995840in}}%
\pgfpathlineto{\pgfqpoint{0.852346in}{1.004089in}}%
\pgfpathlineto{\pgfqpoint{0.851132in}{1.012353in}}%
\pgfpathlineto{\pgfqpoint{0.850422in}{1.020622in}}%
\pgfpathlineto{\pgfqpoint{0.853770in}{1.022389in}}%
\pgfpathlineto{\pgfqpoint{0.857114in}{1.024344in}}%
\pgfpathlineto{\pgfqpoint{0.860453in}{1.026484in}}%
\pgfpathlineto{\pgfqpoint{0.863788in}{1.028804in}}%
\pgfpathlineto{\pgfqpoint{0.864497in}{1.020766in}}%
\pgfpathlineto{\pgfqpoint{0.865695in}{1.012734in}}%
\pgfpathlineto{\pgfqpoint{0.867382in}{1.004716in}}%
\pgfpathlineto{\pgfqpoint{0.869560in}{0.996720in}}%
\pgfpathclose%
\pgfusepath{fill}%
\end{pgfscope}%
\begin{pgfscope}%
\pgfpathrectangle{\pgfqpoint{0.329460in}{0.284240in}}{\pgfqpoint{1.989680in}{1.989680in}}%
\pgfusepath{clip}%
\pgfsetbuttcap%
\pgfsetroundjoin%
\definecolor{currentfill}{rgb}{0.166383,0.690856,0.496502}%
\pgfsetfillcolor{currentfill}%
\pgfsetlinewidth{0.000000pt}%
\definecolor{currentstroke}{rgb}{0.000000,0.000000,0.000000}%
\pgfsetstrokecolor{currentstroke}%
\pgfsetdash{}{0pt}%
\pgfpathmoveto{\pgfqpoint{1.430344in}{1.589291in}}%
\pgfpathlineto{\pgfqpoint{1.431461in}{1.584445in}}%
\pgfpathlineto{\pgfqpoint{1.432576in}{1.579558in}}%
\pgfpathlineto{\pgfqpoint{1.433689in}{1.574631in}}%
\pgfpathlineto{\pgfqpoint{1.434801in}{1.569667in}}%
\pgfpathlineto{\pgfqpoint{1.442031in}{1.568348in}}%
\pgfpathlineto{\pgfqpoint{1.449176in}{1.566921in}}%
\pgfpathlineto{\pgfqpoint{1.456229in}{1.565385in}}%
\pgfpathlineto{\pgfqpoint{1.463182in}{1.563743in}}%
\pgfpathlineto{\pgfqpoint{1.461692in}{1.568791in}}%
\pgfpathlineto{\pgfqpoint{1.460199in}{1.573801in}}%
\pgfpathlineto{\pgfqpoint{1.458704in}{1.578772in}}%
\pgfpathlineto{\pgfqpoint{1.457208in}{1.583701in}}%
\pgfpathlineto{\pgfqpoint{1.450627in}{1.585250in}}%
\pgfpathlineto{\pgfqpoint{1.443952in}{1.586699in}}%
\pgfpathlineto{\pgfqpoint{1.437189in}{1.588046in}}%
\pgfpathlineto{\pgfqpoint{1.430344in}{1.589291in}}%
\pgfpathclose%
\pgfusepath{fill}%
\end{pgfscope}%
\begin{pgfscope}%
\pgfpathrectangle{\pgfqpoint{0.329460in}{0.284240in}}{\pgfqpoint{1.989680in}{1.989680in}}%
\pgfusepath{clip}%
\pgfsetbuttcap%
\pgfsetroundjoin%
\definecolor{currentfill}{rgb}{0.263663,0.237631,0.518762}%
\pgfsetfillcolor{currentfill}%
\pgfsetlinewidth{0.000000pt}%
\definecolor{currentstroke}{rgb}{0.000000,0.000000,0.000000}%
\pgfsetstrokecolor{currentstroke}%
\pgfsetdash{}{0pt}%
\pgfpathmoveto{\pgfqpoint{1.743755in}{1.183119in}}%
\pgfpathlineto{\pgfqpoint{1.747007in}{1.177508in}}%
\pgfpathlineto{\pgfqpoint{1.750258in}{1.171979in}}%
\pgfpathlineto{\pgfqpoint{1.753509in}{1.166535in}}%
\pgfpathlineto{\pgfqpoint{1.756760in}{1.161181in}}%
\pgfpathlineto{\pgfqpoint{1.758070in}{1.154614in}}%
\pgfpathlineto{\pgfqpoint{1.758977in}{1.148023in}}%
\pgfpathlineto{\pgfqpoint{1.759479in}{1.141414in}}%
\pgfpathlineto{\pgfqpoint{1.759575in}{1.134792in}}%
\pgfpathlineto{\pgfqpoint{1.756289in}{1.140381in}}%
\pgfpathlineto{\pgfqpoint{1.753002in}{1.146059in}}%
\pgfpathlineto{\pgfqpoint{1.749715in}{1.151823in}}%
\pgfpathlineto{\pgfqpoint{1.746428in}{1.157668in}}%
\pgfpathlineto{\pgfqpoint{1.746348in}{1.164054in}}%
\pgfpathlineto{\pgfqpoint{1.745875in}{1.170428in}}%
\pgfpathlineto{\pgfqpoint{1.745010in}{1.176785in}}%
\pgfpathlineto{\pgfqpoint{1.743755in}{1.183119in}}%
\pgfpathclose%
\pgfusepath{fill}%
\end{pgfscope}%
\begin{pgfscope}%
\pgfpathrectangle{\pgfqpoint{0.329460in}{0.284240in}}{\pgfqpoint{1.989680in}{1.989680in}}%
\pgfusepath{clip}%
\pgfsetbuttcap%
\pgfsetroundjoin%
\definecolor{currentfill}{rgb}{0.133743,0.548535,0.553541}%
\pgfsetfillcolor{currentfill}%
\pgfsetlinewidth{0.000000pt}%
\definecolor{currentstroke}{rgb}{0.000000,0.000000,0.000000}%
\pgfsetstrokecolor{currentstroke}%
\pgfsetdash{}{0pt}%
\pgfpathmoveto{\pgfqpoint{1.588488in}{1.455068in}}%
\pgfpathlineto{\pgfqpoint{1.591189in}{1.449175in}}%
\pgfpathlineto{\pgfqpoint{1.593888in}{1.443275in}}%
\pgfpathlineto{\pgfqpoint{1.596583in}{1.437372in}}%
\pgfpathlineto{\pgfqpoint{1.599276in}{1.431468in}}%
\pgfpathlineto{\pgfqpoint{1.604560in}{1.427589in}}%
\pgfpathlineto{\pgfqpoint{1.609602in}{1.423626in}}%
\pgfpathlineto{\pgfqpoint{1.614396in}{1.419585in}}%
\pgfpathlineto{\pgfqpoint{1.618937in}{1.415469in}}%
\pgfpathlineto{\pgfqpoint{1.616023in}{1.421562in}}%
\pgfpathlineto{\pgfqpoint{1.613107in}{1.427655in}}%
\pgfpathlineto{\pgfqpoint{1.610187in}{1.433743in}}%
\pgfpathlineto{\pgfqpoint{1.607265in}{1.439826in}}%
\pgfpathlineto{\pgfqpoint{1.602929in}{1.443747in}}%
\pgfpathlineto{\pgfqpoint{1.598352in}{1.447597in}}%
\pgfpathlineto{\pgfqpoint{1.593536in}{1.451372in}}%
\pgfpathlineto{\pgfqpoint{1.588488in}{1.455068in}}%
\pgfpathclose%
\pgfusepath{fill}%
\end{pgfscope}%
\begin{pgfscope}%
\pgfpathrectangle{\pgfqpoint{0.329460in}{0.284240in}}{\pgfqpoint{1.989680in}{1.989680in}}%
\pgfusepath{clip}%
\pgfsetbuttcap%
\pgfsetroundjoin%
\definecolor{currentfill}{rgb}{0.172719,0.448791,0.557885}%
\pgfsetfillcolor{currentfill}%
\pgfsetlinewidth{0.000000pt}%
\definecolor{currentstroke}{rgb}{0.000000,0.000000,0.000000}%
\pgfsetstrokecolor{currentstroke}%
\pgfsetdash{}{0pt}%
\pgfpathmoveto{\pgfqpoint{2.000140in}{1.339594in}}%
\pgfpathlineto{\pgfqpoint{2.003879in}{1.353268in}}%
\pgfpathlineto{\pgfqpoint{2.007638in}{1.367395in}}%
\pgfpathlineto{\pgfqpoint{2.011418in}{1.381981in}}%
\pgfpathlineto{\pgfqpoint{2.016682in}{1.371635in}}%
\pgfpathlineto{\pgfqpoint{2.021306in}{1.361194in}}%
\pgfpathlineto{\pgfqpoint{2.025282in}{1.350668in}}%
\pgfpathlineto{\pgfqpoint{2.028602in}{1.340067in}}%
\pgfpathlineto{\pgfqpoint{2.024697in}{1.325624in}}%
\pgfpathlineto{\pgfqpoint{2.020814in}{1.311642in}}%
\pgfpathlineto{\pgfqpoint{2.016952in}{1.298116in}}%
\pgfpathlineto{\pgfqpoint{2.013708in}{1.308606in}}%
\pgfpathlineto{\pgfqpoint{2.009820in}{1.319022in}}%
\pgfpathlineto{\pgfqpoint{2.005295in}{1.329355in}}%
\pgfpathlineto{\pgfqpoint{2.000140in}{1.339594in}}%
\pgfpathclose%
\pgfusepath{fill}%
\end{pgfscope}%
\begin{pgfscope}%
\pgfpathrectangle{\pgfqpoint{0.329460in}{0.284240in}}{\pgfqpoint{1.989680in}{1.989680in}}%
\pgfusepath{clip}%
\pgfsetbuttcap%
\pgfsetroundjoin%
\definecolor{currentfill}{rgb}{0.120081,0.622161,0.534946}%
\pgfsetfillcolor{currentfill}%
\pgfsetlinewidth{0.000000pt}%
\definecolor{currentstroke}{rgb}{0.000000,0.000000,0.000000}%
\pgfsetstrokecolor{currentstroke}%
\pgfsetdash{}{0pt}%
\pgfpathmoveto{\pgfqpoint{1.151138in}{1.511610in}}%
\pgfpathlineto{\pgfqpoint{1.148613in}{1.505982in}}%
\pgfpathlineto{\pgfqpoint{1.146091in}{1.500330in}}%
\pgfpathlineto{\pgfqpoint{1.143572in}{1.494655in}}%
\pgfpathlineto{\pgfqpoint{1.141056in}{1.488961in}}%
\pgfpathlineto{\pgfqpoint{1.146727in}{1.492136in}}%
\pgfpathlineto{\pgfqpoint{1.152592in}{1.495222in}}%
\pgfpathlineto{\pgfqpoint{1.158646in}{1.498215in}}%
\pgfpathlineto{\pgfqpoint{1.164884in}{1.501112in}}%
\pgfpathlineto{\pgfqpoint{1.167110in}{1.506650in}}%
\pgfpathlineto{\pgfqpoint{1.169339in}{1.512168in}}%
\pgfpathlineto{\pgfqpoint{1.171571in}{1.517664in}}%
\pgfpathlineto{\pgfqpoint{1.173806in}{1.523136in}}%
\pgfpathlineto{\pgfqpoint{1.167871in}{1.520388in}}%
\pgfpathlineto{\pgfqpoint{1.162111in}{1.517549in}}%
\pgfpathlineto{\pgfqpoint{1.156532in}{1.514622in}}%
\pgfpathlineto{\pgfqpoint{1.151138in}{1.511610in}}%
\pgfpathclose%
\pgfusepath{fill}%
\end{pgfscope}%
\begin{pgfscope}%
\pgfpathrectangle{\pgfqpoint{0.329460in}{0.284240in}}{\pgfqpoint{1.989680in}{1.989680in}}%
\pgfusepath{clip}%
\pgfsetbuttcap%
\pgfsetroundjoin%
\definecolor{currentfill}{rgb}{0.122606,0.585371,0.546557}%
\pgfsetfillcolor{currentfill}%
\pgfsetlinewidth{0.000000pt}%
\definecolor{currentstroke}{rgb}{0.000000,0.000000,0.000000}%
\pgfsetstrokecolor{currentstroke}%
\pgfsetdash{}{0pt}%
\pgfpathmoveto{\pgfqpoint{1.120435in}{1.475418in}}%
\pgfpathlineto{\pgfqpoint{1.117668in}{1.469531in}}%
\pgfpathlineto{\pgfqpoint{1.114905in}{1.463628in}}%
\pgfpathlineto{\pgfqpoint{1.112145in}{1.457713in}}%
\pgfpathlineto{\pgfqpoint{1.109388in}{1.451786in}}%
\pgfpathlineto{\pgfqpoint{1.114462in}{1.455474in}}%
\pgfpathlineto{\pgfqpoint{1.119762in}{1.459080in}}%
\pgfpathlineto{\pgfqpoint{1.125284in}{1.462600in}}%
\pgfpathlineto{\pgfqpoint{1.131021in}{1.466031in}}%
\pgfpathlineto{\pgfqpoint{1.133525in}{1.471782in}}%
\pgfpathlineto{\pgfqpoint{1.136033in}{1.477522in}}%
\pgfpathlineto{\pgfqpoint{1.138543in}{1.483249in}}%
\pgfpathlineto{\pgfqpoint{1.141056in}{1.488961in}}%
\pgfpathlineto{\pgfqpoint{1.135586in}{1.485698in}}%
\pgfpathlineto{\pgfqpoint{1.130322in}{1.482352in}}%
\pgfpathlineto{\pgfqpoint{1.125270in}{1.478924in}}%
\pgfpathlineto{\pgfqpoint{1.120435in}{1.475418in}}%
\pgfpathclose%
\pgfusepath{fill}%
\end{pgfscope}%
\begin{pgfscope}%
\pgfpathrectangle{\pgfqpoint{0.329460in}{0.284240in}}{\pgfqpoint{1.989680in}{1.989680in}}%
\pgfusepath{clip}%
\pgfsetbuttcap%
\pgfsetroundjoin%
\definecolor{currentfill}{rgb}{0.166383,0.690856,0.496502}%
\pgfsetfillcolor{currentfill}%
\pgfsetlinewidth{0.000000pt}%
\definecolor{currentstroke}{rgb}{0.000000,0.000000,0.000000}%
\pgfsetstrokecolor{currentstroke}%
\pgfsetdash{}{0pt}%
\pgfpathmoveto{\pgfqpoint{1.239402in}{1.582241in}}%
\pgfpathlineto{\pgfqpoint{1.237823in}{1.577290in}}%
\pgfpathlineto{\pgfqpoint{1.236247in}{1.572297in}}%
\pgfpathlineto{\pgfqpoint{1.234673in}{1.567265in}}%
\pgfpathlineto{\pgfqpoint{1.233101in}{1.562195in}}%
\pgfpathlineto{\pgfqpoint{1.239960in}{1.563931in}}%
\pgfpathlineto{\pgfqpoint{1.246925in}{1.565561in}}%
\pgfpathlineto{\pgfqpoint{1.253988in}{1.567085in}}%
\pgfpathlineto{\pgfqpoint{1.261143in}{1.568500in}}%
\pgfpathlineto{\pgfqpoint{1.262341in}{1.573481in}}%
\pgfpathlineto{\pgfqpoint{1.263540in}{1.578424in}}%
\pgfpathlineto{\pgfqpoint{1.264741in}{1.583327in}}%
\pgfpathlineto{\pgfqpoint{1.265943in}{1.588190in}}%
\pgfpathlineto{\pgfqpoint{1.259171in}{1.586854in}}%
\pgfpathlineto{\pgfqpoint{1.252486in}{1.585416in}}%
\pgfpathlineto{\pgfqpoint{1.245894in}{1.583878in}}%
\pgfpathlineto{\pgfqpoint{1.239402in}{1.582241in}}%
\pgfpathclose%
\pgfusepath{fill}%
\end{pgfscope}%
\begin{pgfscope}%
\pgfpathrectangle{\pgfqpoint{0.329460in}{0.284240in}}{\pgfqpoint{1.989680in}{1.989680in}}%
\pgfusepath{clip}%
\pgfsetbuttcap%
\pgfsetroundjoin%
\definecolor{currentfill}{rgb}{0.134692,0.658636,0.517649}%
\pgfsetfillcolor{currentfill}%
\pgfsetlinewidth{0.000000pt}%
\definecolor{currentstroke}{rgb}{0.000000,0.000000,0.000000}%
\pgfsetstrokecolor{currentstroke}%
\pgfsetdash{}{0pt}%
\pgfpathmoveto{\pgfqpoint{1.489873in}{1.556142in}}%
\pgfpathlineto{\pgfqpoint{1.491719in}{1.550952in}}%
\pgfpathlineto{\pgfqpoint{1.493563in}{1.545729in}}%
\pgfpathlineto{\pgfqpoint{1.495404in}{1.540475in}}%
\pgfpathlineto{\pgfqpoint{1.497242in}{1.535192in}}%
\pgfpathlineto{\pgfqpoint{1.503943in}{1.532920in}}%
\pgfpathlineto{\pgfqpoint{1.510499in}{1.530546in}}%
\pgfpathlineto{\pgfqpoint{1.516905in}{1.528072in}}%
\pgfpathlineto{\pgfqpoint{1.523153in}{1.525501in}}%
\pgfpathlineto{\pgfqpoint{1.520984in}{1.530915in}}%
\pgfpathlineto{\pgfqpoint{1.518813in}{1.536299in}}%
\pgfpathlineto{\pgfqpoint{1.516640in}{1.541653in}}%
\pgfpathlineto{\pgfqpoint{1.514463in}{1.546974in}}%
\pgfpathlineto{\pgfqpoint{1.508534in}{1.549406in}}%
\pgfpathlineto{\pgfqpoint{1.502456in}{1.551746in}}%
\pgfpathlineto{\pgfqpoint{1.496233in}{1.553992in}}%
\pgfpathlineto{\pgfqpoint{1.489873in}{1.556142in}}%
\pgfpathclose%
\pgfusepath{fill}%
\end{pgfscope}%
\begin{pgfscope}%
\pgfpathrectangle{\pgfqpoint{0.329460in}{0.284240in}}{\pgfqpoint{1.989680in}{1.989680in}}%
\pgfusepath{clip}%
\pgfsetbuttcap%
\pgfsetroundjoin%
\definecolor{currentfill}{rgb}{0.179019,0.433756,0.557430}%
\pgfsetfillcolor{currentfill}%
\pgfsetlinewidth{0.000000pt}%
\definecolor{currentstroke}{rgb}{0.000000,0.000000,0.000000}%
\pgfsetstrokecolor{currentstroke}%
\pgfsetdash{}{0pt}%
\pgfpathmoveto{\pgfqpoint{1.659102in}{1.348216in}}%
\pgfpathlineto{\pgfqpoint{1.662169in}{1.342023in}}%
\pgfpathlineto{\pgfqpoint{1.665233in}{1.335855in}}%
\pgfpathlineto{\pgfqpoint{1.668295in}{1.329714in}}%
\pgfpathlineto{\pgfqpoint{1.671355in}{1.323603in}}%
\pgfpathlineto{\pgfqpoint{1.675009in}{1.318532in}}%
\pgfpathlineto{\pgfqpoint{1.678349in}{1.313401in}}%
\pgfpathlineto{\pgfqpoint{1.681370in}{1.308216in}}%
\pgfpathlineto{\pgfqpoint{1.684069in}{1.302982in}}%
\pgfpathlineto{\pgfqpoint{1.680878in}{1.309310in}}%
\pgfpathlineto{\pgfqpoint{1.677684in}{1.315669in}}%
\pgfpathlineto{\pgfqpoint{1.674488in}{1.322055in}}%
\pgfpathlineto{\pgfqpoint{1.671289in}{1.328465in}}%
\pgfpathlineto{\pgfqpoint{1.668704in}{1.333478in}}%
\pgfpathlineto{\pgfqpoint{1.665809in}{1.338444in}}%
\pgfpathlineto{\pgfqpoint{1.662607in}{1.343359in}}%
\pgfpathlineto{\pgfqpoint{1.659102in}{1.348216in}}%
\pgfpathclose%
\pgfusepath{fill}%
\end{pgfscope}%
\begin{pgfscope}%
\pgfpathrectangle{\pgfqpoint{0.329460in}{0.284240in}}{\pgfqpoint{1.989680in}{1.989680in}}%
\pgfusepath{clip}%
\pgfsetbuttcap%
\pgfsetroundjoin%
\definecolor{currentfill}{rgb}{0.268510,0.009605,0.335427}%
\pgfsetfillcolor{currentfill}%
\pgfsetlinewidth{0.000000pt}%
\definecolor{currentstroke}{rgb}{0.000000,0.000000,0.000000}%
\pgfsetstrokecolor{currentstroke}%
\pgfsetdash{}{0pt}%
\pgfpathmoveto{\pgfqpoint{0.815920in}{0.979847in}}%
\pgfpathlineto{\pgfqpoint{0.812511in}{0.980680in}}%
\pgfpathlineto{\pgfqpoint{0.809093in}{0.981763in}}%
\pgfpathlineto{\pgfqpoint{0.805668in}{0.983098in}}%
\pgfpathlineto{\pgfqpoint{0.802234in}{0.984692in}}%
\pgfpathlineto{\pgfqpoint{0.799846in}{0.993820in}}%
\pgfpathlineto{\pgfqpoint{0.798017in}{1.002972in}}%
\pgfpathlineto{\pgfqpoint{0.796748in}{1.012135in}}%
\pgfpathlineto{\pgfqpoint{0.796037in}{1.021303in}}%
\pgfpathlineto{\pgfqpoint{0.799490in}{1.019490in}}%
\pgfpathlineto{\pgfqpoint{0.802936in}{1.017935in}}%
\pgfpathlineto{\pgfqpoint{0.806373in}{1.016632in}}%
\pgfpathlineto{\pgfqpoint{0.809801in}{1.015576in}}%
\pgfpathlineto{\pgfqpoint{0.810514in}{1.006629in}}%
\pgfpathlineto{\pgfqpoint{0.811771in}{0.997685in}}%
\pgfpathlineto{\pgfqpoint{0.813572in}{0.988755in}}%
\pgfpathlineto{\pgfqpoint{0.815920in}{0.979847in}}%
\pgfpathclose%
\pgfusepath{fill}%
\end{pgfscope}%
\begin{pgfscope}%
\pgfpathrectangle{\pgfqpoint{0.329460in}{0.284240in}}{\pgfqpoint{1.989680in}{1.989680in}}%
\pgfusepath{clip}%
\pgfsetbuttcap%
\pgfsetroundjoin%
\definecolor{currentfill}{rgb}{0.274952,0.037752,0.364543}%
\pgfsetfillcolor{currentfill}%
\pgfsetlinewidth{0.000000pt}%
\definecolor{currentstroke}{rgb}{0.000000,0.000000,0.000000}%
\pgfsetstrokecolor{currentstroke}%
\pgfsetdash{}{0pt}%
\pgfpathmoveto{\pgfqpoint{0.882769in}{1.008656in}}%
\pgfpathlineto{\pgfqpoint{0.879473in}{1.005420in}}%
\pgfpathlineto{\pgfqpoint{0.876172in}{1.002350in}}%
\pgfpathlineto{\pgfqpoint{0.872868in}{0.999448in}}%
\pgfpathlineto{\pgfqpoint{0.869560in}{0.996720in}}%
\pgfpathlineto{\pgfqpoint{0.867382in}{1.004716in}}%
\pgfpathlineto{\pgfqpoint{0.865695in}{1.012734in}}%
\pgfpathlineto{\pgfqpoint{0.864497in}{1.020766in}}%
\pgfpathlineto{\pgfqpoint{0.863788in}{1.028804in}}%
\pgfpathlineto{\pgfqpoint{0.867120in}{1.031300in}}%
\pgfpathlineto{\pgfqpoint{0.870447in}{1.033968in}}%
\pgfpathlineto{\pgfqpoint{0.873771in}{1.036805in}}%
\pgfpathlineto{\pgfqpoint{0.877092in}{1.039807in}}%
\pgfpathlineto{\pgfqpoint{0.877797in}{1.032002in}}%
\pgfpathlineto{\pgfqpoint{0.878978in}{1.024203in}}%
\pgfpathlineto{\pgfqpoint{0.880635in}{1.016419in}}%
\pgfpathlineto{\pgfqpoint{0.882769in}{1.008656in}}%
\pgfpathclose%
\pgfusepath{fill}%
\end{pgfscope}%
\begin{pgfscope}%
\pgfpathrectangle{\pgfqpoint{0.329460in}{0.284240in}}{\pgfqpoint{1.989680in}{1.989680in}}%
\pgfusepath{clip}%
\pgfsetbuttcap%
\pgfsetroundjoin%
\definecolor{currentfill}{rgb}{0.282327,0.094955,0.417331}%
\pgfsetfillcolor{currentfill}%
\pgfsetlinewidth{0.000000pt}%
\definecolor{currentstroke}{rgb}{0.000000,0.000000,0.000000}%
\pgfsetstrokecolor{currentstroke}%
\pgfsetdash{}{0pt}%
\pgfpathmoveto{\pgfqpoint{1.799061in}{1.075833in}}%
\pgfpathlineto{\pgfqpoint{1.802359in}{1.071697in}}%
\pgfpathlineto{\pgfqpoint{1.805660in}{1.067697in}}%
\pgfpathlineto{\pgfqpoint{1.808962in}{1.063836in}}%
\pgfpathlineto{\pgfqpoint{1.812267in}{1.060117in}}%
\pgfpathlineto{\pgfqpoint{1.811976in}{1.052548in}}%
\pgfpathlineto{\pgfqpoint{1.811224in}{1.044977in}}%
\pgfpathlineto{\pgfqpoint{1.810009in}{1.037414in}}%
\pgfpathlineto{\pgfqpoint{1.808331in}{1.029867in}}%
\pgfpathlineto{\pgfqpoint{1.805040in}{1.033820in}}%
\pgfpathlineto{\pgfqpoint{1.801750in}{1.037917in}}%
\pgfpathlineto{\pgfqpoint{1.798463in}{1.042153in}}%
\pgfpathlineto{\pgfqpoint{1.795179in}{1.046524in}}%
\pgfpathlineto{\pgfqpoint{1.796823in}{1.053837in}}%
\pgfpathlineto{\pgfqpoint{1.798018in}{1.061164in}}%
\pgfpathlineto{\pgfqpoint{1.798764in}{1.068498in}}%
\pgfpathlineto{\pgfqpoint{1.799061in}{1.075833in}}%
\pgfpathclose%
\pgfusepath{fill}%
\end{pgfscope}%
\begin{pgfscope}%
\pgfpathrectangle{\pgfqpoint{0.329460in}{0.284240in}}{\pgfqpoint{1.989680in}{1.989680in}}%
\pgfusepath{clip}%
\pgfsetbuttcap%
\pgfsetroundjoin%
\definecolor{currentfill}{rgb}{0.277941,0.056324,0.381191}%
\pgfsetfillcolor{currentfill}%
\pgfsetlinewidth{0.000000pt}%
\definecolor{currentstroke}{rgb}{0.000000,0.000000,0.000000}%
\pgfsetstrokecolor{currentstroke}%
\pgfsetdash{}{0pt}%
\pgfpathmoveto{\pgfqpoint{1.920396in}{1.039551in}}%
\pgfpathlineto{\pgfqpoint{1.923893in}{1.042767in}}%
\pgfpathlineto{\pgfqpoint{1.927400in}{1.046269in}}%
\pgfpathlineto{\pgfqpoint{1.930918in}{1.050061in}}%
\pgfpathlineto{\pgfqpoint{1.934447in}{1.054149in}}%
\pgfpathlineto{\pgfqpoint{1.934261in}{1.044562in}}%
\pgfpathlineto{\pgfqpoint{1.933490in}{1.034968in}}%
\pgfpathlineto{\pgfqpoint{1.932133in}{1.025377in}}%
\pgfpathlineto{\pgfqpoint{1.930188in}{1.015800in}}%
\pgfpathlineto{\pgfqpoint{1.926664in}{1.011921in}}%
\pgfpathlineto{\pgfqpoint{1.923152in}{1.008339in}}%
\pgfpathlineto{\pgfqpoint{1.919650in}{1.005049in}}%
\pgfpathlineto{\pgfqpoint{1.916158in}{1.002046in}}%
\pgfpathlineto{\pgfqpoint{1.918076in}{1.011412in}}%
\pgfpathlineto{\pgfqpoint{1.919421in}{1.020791in}}%
\pgfpathlineto{\pgfqpoint{1.920194in}{1.030174in}}%
\pgfpathlineto{\pgfqpoint{1.920396in}{1.039551in}}%
\pgfpathclose%
\pgfusepath{fill}%
\end{pgfscope}%
\begin{pgfscope}%
\pgfpathrectangle{\pgfqpoint{0.329460in}{0.284240in}}{\pgfqpoint{1.989680in}{1.989680in}}%
\pgfusepath{clip}%
\pgfsetbuttcap%
\pgfsetroundjoin%
\definecolor{currentfill}{rgb}{0.212395,0.359683,0.551710}%
\pgfsetfillcolor{currentfill}%
\pgfsetlinewidth{0.000000pt}%
\definecolor{currentstroke}{rgb}{0.000000,0.000000,0.000000}%
\pgfsetstrokecolor{currentstroke}%
\pgfsetdash{}{0pt}%
\pgfpathmoveto{\pgfqpoint{1.696814in}{1.278022in}}%
\pgfpathlineto{\pgfqpoint{1.699996in}{1.271884in}}%
\pgfpathlineto{\pgfqpoint{1.703176in}{1.265794in}}%
\pgfpathlineto{\pgfqpoint{1.706354in}{1.259753in}}%
\pgfpathlineto{\pgfqpoint{1.709530in}{1.253764in}}%
\pgfpathlineto{\pgfqpoint{1.712108in}{1.248036in}}%
\pgfpathlineto{\pgfqpoint{1.714334in}{1.242264in}}%
\pgfpathlineto{\pgfqpoint{1.716202in}{1.236454in}}%
\pgfpathlineto{\pgfqpoint{1.717711in}{1.230612in}}%
\pgfpathlineto{\pgfqpoint{1.714450in}{1.236828in}}%
\pgfpathlineto{\pgfqpoint{1.711188in}{1.243097in}}%
\pgfpathlineto{\pgfqpoint{1.707924in}{1.249416in}}%
\pgfpathlineto{\pgfqpoint{1.704659in}{1.255781in}}%
\pgfpathlineto{\pgfqpoint{1.703215in}{1.261392in}}%
\pgfpathlineto{\pgfqpoint{1.701425in}{1.266973in}}%
\pgfpathlineto{\pgfqpoint{1.699290in}{1.272518in}}%
\pgfpathlineto{\pgfqpoint{1.696814in}{1.278022in}}%
\pgfpathclose%
\pgfusepath{fill}%
\end{pgfscope}%
\begin{pgfscope}%
\pgfpathrectangle{\pgfqpoint{0.329460in}{0.284240in}}{\pgfqpoint{1.989680in}{1.989680in}}%
\pgfusepath{clip}%
\pgfsetbuttcap%
\pgfsetroundjoin%
\definecolor{currentfill}{rgb}{0.134692,0.658636,0.517649}%
\pgfsetfillcolor{currentfill}%
\pgfsetlinewidth{0.000000pt}%
\definecolor{currentstroke}{rgb}{0.000000,0.000000,0.000000}%
\pgfsetstrokecolor{currentstroke}%
\pgfsetdash{}{0pt}%
\pgfpathmoveto{\pgfqpoint{1.182772in}{1.544736in}}%
\pgfpathlineto{\pgfqpoint{1.180526in}{1.539384in}}%
\pgfpathlineto{\pgfqpoint{1.178283in}{1.533998in}}%
\pgfpathlineto{\pgfqpoint{1.176043in}{1.528582in}}%
\pgfpathlineto{\pgfqpoint{1.173806in}{1.523136in}}%
\pgfpathlineto{\pgfqpoint{1.179909in}{1.525792in}}%
\pgfpathlineto{\pgfqpoint{1.186175in}{1.528352in}}%
\pgfpathlineto{\pgfqpoint{1.192597in}{1.530814in}}%
\pgfpathlineto{\pgfqpoint{1.199170in}{1.533177in}}%
\pgfpathlineto{\pgfqpoint{1.201084in}{1.538487in}}%
\pgfpathlineto{\pgfqpoint{1.203001in}{1.543768in}}%
\pgfpathlineto{\pgfqpoint{1.204920in}{1.549019in}}%
\pgfpathlineto{\pgfqpoint{1.206842in}{1.554236in}}%
\pgfpathlineto{\pgfqpoint{1.200604in}{1.552000in}}%
\pgfpathlineto{\pgfqpoint{1.194509in}{1.549671in}}%
\pgfpathlineto{\pgfqpoint{1.188563in}{1.547248in}}%
\pgfpathlineto{\pgfqpoint{1.182772in}{1.544736in}}%
\pgfpathclose%
\pgfusepath{fill}%
\end{pgfscope}%
\begin{pgfscope}%
\pgfpathrectangle{\pgfqpoint{0.329460in}{0.284240in}}{\pgfqpoint{1.989680in}{1.989680in}}%
\pgfusepath{clip}%
\pgfsetbuttcap%
\pgfsetroundjoin%
\definecolor{currentfill}{rgb}{0.133743,0.548535,0.553541}%
\pgfsetfillcolor{currentfill}%
\pgfsetlinewidth{0.000000pt}%
\definecolor{currentstroke}{rgb}{0.000000,0.000000,0.000000}%
\pgfsetstrokecolor{currentstroke}%
\pgfsetdash{}{0pt}%
\pgfpathmoveto{\pgfqpoint{1.091464in}{1.436284in}}%
\pgfpathlineto{\pgfqpoint{1.088498in}{1.430157in}}%
\pgfpathlineto{\pgfqpoint{1.085535in}{1.424024in}}%
\pgfpathlineto{\pgfqpoint{1.082576in}{1.417888in}}%
\pgfpathlineto{\pgfqpoint{1.079619in}{1.411751in}}%
\pgfpathlineto{\pgfqpoint{1.083930in}{1.415930in}}%
\pgfpathlineto{\pgfqpoint{1.088500in}{1.420038in}}%
\pgfpathlineto{\pgfqpoint{1.093321in}{1.424071in}}%
\pgfpathlineto{\pgfqpoint{1.098390in}{1.428024in}}%
\pgfpathlineto{\pgfqpoint{1.101135in}{1.433968in}}%
\pgfpathlineto{\pgfqpoint{1.103884in}{1.439912in}}%
\pgfpathlineto{\pgfqpoint{1.106635in}{1.445852in}}%
\pgfpathlineto{\pgfqpoint{1.109388in}{1.451786in}}%
\pgfpathlineto{\pgfqpoint{1.104547in}{1.448020in}}%
\pgfpathlineto{\pgfqpoint{1.099942in}{1.444178in}}%
\pgfpathlineto{\pgfqpoint{1.095580in}{1.440265in}}%
\pgfpathlineto{\pgfqpoint{1.091464in}{1.436284in}}%
\pgfpathclose%
\pgfusepath{fill}%
\end{pgfscope}%
\begin{pgfscope}%
\pgfpathrectangle{\pgfqpoint{0.329460in}{0.284240in}}{\pgfqpoint{1.989680in}{1.989680in}}%
\pgfusepath{clip}%
\pgfsetbuttcap%
\pgfsetroundjoin%
\definecolor{currentfill}{rgb}{0.279566,0.067836,0.391917}%
\pgfsetfillcolor{currentfill}%
\pgfsetlinewidth{0.000000pt}%
\definecolor{currentstroke}{rgb}{0.000000,0.000000,0.000000}%
\pgfsetstrokecolor{currentstroke}%
\pgfsetdash{}{0pt}%
\pgfpathmoveto{\pgfqpoint{0.895924in}{1.023176in}}%
\pgfpathlineto{\pgfqpoint{0.892640in}{1.019317in}}%
\pgfpathlineto{\pgfqpoint{0.889353in}{1.015608in}}%
\pgfpathlineto{\pgfqpoint{0.886063in}{1.012053in}}%
\pgfpathlineto{\pgfqpoint{0.882769in}{1.008656in}}%
\pgfpathlineto{\pgfqpoint{0.880635in}{1.016419in}}%
\pgfpathlineto{\pgfqpoint{0.878978in}{1.024203in}}%
\pgfpathlineto{\pgfqpoint{0.877797in}{1.032002in}}%
\pgfpathlineto{\pgfqpoint{0.877092in}{1.039807in}}%
\pgfpathlineto{\pgfqpoint{0.880409in}{1.042970in}}%
\pgfpathlineto{\pgfqpoint{0.883723in}{1.046290in}}%
\pgfpathlineto{\pgfqpoint{0.887035in}{1.049765in}}%
\pgfpathlineto{\pgfqpoint{0.890344in}{1.053389in}}%
\pgfpathlineto{\pgfqpoint{0.891045in}{1.045818in}}%
\pgfpathlineto{\pgfqpoint{0.892208in}{1.038254in}}%
\pgfpathlineto{\pgfqpoint{0.893835in}{1.030704in}}%
\pgfpathlineto{\pgfqpoint{0.895924in}{1.023176in}}%
\pgfpathclose%
\pgfusepath{fill}%
\end{pgfscope}%
\begin{pgfscope}%
\pgfpathrectangle{\pgfqpoint{0.329460in}{0.284240in}}{\pgfqpoint{1.989680in}{1.989680in}}%
\pgfusepath{clip}%
\pgfsetbuttcap%
\pgfsetroundjoin%
\definecolor{currentfill}{rgb}{0.147607,0.511733,0.557049}%
\pgfsetfillcolor{currentfill}%
\pgfsetlinewidth{0.000000pt}%
\definecolor{currentstroke}{rgb}{0.000000,0.000000,0.000000}%
\pgfsetstrokecolor{currentstroke}%
\pgfsetdash{}{0pt}%
\pgfpathmoveto{\pgfqpoint{1.618937in}{1.415469in}}%
\pgfpathlineto{\pgfqpoint{1.621847in}{1.409378in}}%
\pgfpathlineto{\pgfqpoint{1.624755in}{1.403290in}}%
\pgfpathlineto{\pgfqpoint{1.627660in}{1.397209in}}%
\pgfpathlineto{\pgfqpoint{1.630562in}{1.391137in}}%
\pgfpathlineto{\pgfqpoint{1.635039in}{1.386750in}}%
\pgfpathlineto{\pgfqpoint{1.639242in}{1.382293in}}%
\pgfpathlineto{\pgfqpoint{1.643167in}{1.377769in}}%
\pgfpathlineto{\pgfqpoint{1.646810in}{1.373182in}}%
\pgfpathlineto{\pgfqpoint{1.643730in}{1.379459in}}%
\pgfpathlineto{\pgfqpoint{1.640647in}{1.385744in}}%
\pgfpathlineto{\pgfqpoint{1.637562in}{1.392036in}}%
\pgfpathlineto{\pgfqpoint{1.634474in}{1.398332in}}%
\pgfpathlineto{\pgfqpoint{1.630992in}{1.402709in}}%
\pgfpathlineto{\pgfqpoint{1.627239in}{1.407027in}}%
\pgfpathlineto{\pgfqpoint{1.623219in}{1.411282in}}%
\pgfpathlineto{\pgfqpoint{1.618937in}{1.415469in}}%
\pgfpathclose%
\pgfusepath{fill}%
\end{pgfscope}%
\begin{pgfscope}%
\pgfpathrectangle{\pgfqpoint{0.329460in}{0.284240in}}{\pgfqpoint{1.989680in}{1.989680in}}%
\pgfusepath{clip}%
\pgfsetbuttcap%
\pgfsetroundjoin%
\definecolor{currentfill}{rgb}{0.272594,0.025563,0.353093}%
\pgfsetfillcolor{currentfill}%
\pgfsetlinewidth{0.000000pt}%
\definecolor{currentstroke}{rgb}{0.000000,0.000000,0.000000}%
\pgfsetstrokecolor{currentstroke}%
\pgfsetdash{}{0pt}%
\pgfpathmoveto{\pgfqpoint{0.802234in}{0.984692in}}%
\pgfpathlineto{\pgfqpoint{0.798790in}{0.986548in}}%
\pgfpathlineto{\pgfqpoint{0.795338in}{0.988671in}}%
\pgfpathlineto{\pgfqpoint{0.791876in}{0.991067in}}%
\pgfpathlineto{\pgfqpoint{0.788403in}{0.993740in}}%
\pgfpathlineto{\pgfqpoint{0.785975in}{1.003086in}}%
\pgfpathlineto{\pgfqpoint{0.784121in}{1.012453in}}%
\pgfpathlineto{\pgfqpoint{0.782840in}{1.021833in}}%
\pgfpathlineto{\pgfqpoint{0.782131in}{1.031216in}}%
\pgfpathlineto{\pgfqpoint{0.785622in}{1.028329in}}%
\pgfpathlineto{\pgfqpoint{0.789103in}{1.025717in}}%
\pgfpathlineto{\pgfqpoint{0.792574in}{1.023377in}}%
\pgfpathlineto{\pgfqpoint{0.796037in}{1.021303in}}%
\pgfpathlineto{\pgfqpoint{0.796748in}{1.012135in}}%
\pgfpathlineto{\pgfqpoint{0.798017in}{1.002972in}}%
\pgfpathlineto{\pgfqpoint{0.799846in}{0.993820in}}%
\pgfpathlineto{\pgfqpoint{0.802234in}{0.984692in}}%
\pgfpathclose%
\pgfusepath{fill}%
\end{pgfscope}%
\begin{pgfscope}%
\pgfpathrectangle{\pgfqpoint{0.329460in}{0.284240in}}{\pgfqpoint{1.989680in}{1.989680in}}%
\pgfusepath{clip}%
\pgfsetbuttcap%
\pgfsetroundjoin%
\definecolor{currentfill}{rgb}{0.220124,0.725509,0.466226}%
\pgfsetfillcolor{currentfill}%
\pgfsetlinewidth{0.000000pt}%
\definecolor{currentstroke}{rgb}{0.000000,0.000000,0.000000}%
\pgfsetstrokecolor{currentstroke}%
\pgfsetdash{}{0pt}%
\pgfpathmoveto{\pgfqpoint{1.324091in}{1.613673in}}%
\pgfpathlineto{\pgfqpoint{1.323684in}{1.609099in}}%
\pgfpathlineto{\pgfqpoint{1.323277in}{1.604476in}}%
\pgfpathlineto{\pgfqpoint{1.322872in}{1.599806in}}%
\pgfpathlineto{\pgfqpoint{1.322466in}{1.595090in}}%
\pgfpathlineto{\pgfqpoint{1.329725in}{1.595468in}}%
\pgfpathlineto{\pgfqpoint{1.337003in}{1.595736in}}%
\pgfpathlineto{\pgfqpoint{1.344295in}{1.595894in}}%
\pgfpathlineto{\pgfqpoint{1.351593in}{1.595943in}}%
\pgfpathlineto{\pgfqpoint{1.351587in}{1.600646in}}%
\pgfpathlineto{\pgfqpoint{1.351582in}{1.605304in}}%
\pgfpathlineto{\pgfqpoint{1.351576in}{1.609914in}}%
\pgfpathlineto{\pgfqpoint{1.351570in}{1.614475in}}%
\pgfpathlineto{\pgfqpoint{1.344685in}{1.614429in}}%
\pgfpathlineto{\pgfqpoint{1.337805in}{1.614280in}}%
\pgfpathlineto{\pgfqpoint{1.330938in}{1.614028in}}%
\pgfpathlineto{\pgfqpoint{1.324091in}{1.613673in}}%
\pgfpathclose%
\pgfusepath{fill}%
\end{pgfscope}%
\begin{pgfscope}%
\pgfpathrectangle{\pgfqpoint{0.329460in}{0.284240in}}{\pgfqpoint{1.989680in}{1.989680in}}%
\pgfusepath{clip}%
\pgfsetbuttcap%
\pgfsetroundjoin%
\definecolor{currentfill}{rgb}{0.220124,0.725509,0.466226}%
\pgfsetfillcolor{currentfill}%
\pgfsetlinewidth{0.000000pt}%
\definecolor{currentstroke}{rgb}{0.000000,0.000000,0.000000}%
\pgfsetstrokecolor{currentstroke}%
\pgfsetdash{}{0pt}%
\pgfpathmoveto{\pgfqpoint{1.351570in}{1.614475in}}%
\pgfpathlineto{\pgfqpoint{1.351576in}{1.609914in}}%
\pgfpathlineto{\pgfqpoint{1.351582in}{1.605304in}}%
\pgfpathlineto{\pgfqpoint{1.351587in}{1.600646in}}%
\pgfpathlineto{\pgfqpoint{1.351593in}{1.595943in}}%
\pgfpathlineto{\pgfqpoint{1.358891in}{1.595882in}}%
\pgfpathlineto{\pgfqpoint{1.366182in}{1.595711in}}%
\pgfpathlineto{\pgfqpoint{1.373458in}{1.595431in}}%
\pgfpathlineto{\pgfqpoint{1.380714in}{1.595041in}}%
\pgfpathlineto{\pgfqpoint{1.380297in}{1.599758in}}%
\pgfpathlineto{\pgfqpoint{1.379880in}{1.604429in}}%
\pgfpathlineto{\pgfqpoint{1.379462in}{1.609053in}}%
\pgfpathlineto{\pgfqpoint{1.379044in}{1.613627in}}%
\pgfpathlineto{\pgfqpoint{1.372199in}{1.613993in}}%
\pgfpathlineto{\pgfqpoint{1.365334in}{1.614257in}}%
\pgfpathlineto{\pgfqpoint{1.358455in}{1.614418in}}%
\pgfpathlineto{\pgfqpoint{1.351570in}{1.614475in}}%
\pgfpathclose%
\pgfusepath{fill}%
\end{pgfscope}%
\begin{pgfscope}%
\pgfpathrectangle{\pgfqpoint{0.329460in}{0.284240in}}{\pgfqpoint{1.989680in}{1.989680in}}%
\pgfusepath{clip}%
\pgfsetbuttcap%
\pgfsetroundjoin%
\definecolor{currentfill}{rgb}{0.263663,0.237631,0.518762}%
\pgfsetfillcolor{currentfill}%
\pgfsetlinewidth{0.000000pt}%
\definecolor{currentstroke}{rgb}{0.000000,0.000000,0.000000}%
\pgfsetstrokecolor{currentstroke}%
\pgfsetdash{}{0pt}%
\pgfpathmoveto{\pgfqpoint{0.956208in}{1.151989in}}%
\pgfpathlineto{\pgfqpoint{0.952920in}{1.146090in}}%
\pgfpathlineto{\pgfqpoint{0.949632in}{1.140274in}}%
\pgfpathlineto{\pgfqpoint{0.946345in}{1.134544in}}%
\pgfpathlineto{\pgfqpoint{0.943057in}{1.128902in}}%
\pgfpathlineto{\pgfqpoint{0.942790in}{1.135528in}}%
\pgfpathlineto{\pgfqpoint{0.942931in}{1.142149in}}%
\pgfpathlineto{\pgfqpoint{0.943479in}{1.148756in}}%
\pgfpathlineto{\pgfqpoint{0.944431in}{1.155345in}}%
\pgfpathlineto{\pgfqpoint{0.947694in}{1.160752in}}%
\pgfpathlineto{\pgfqpoint{0.950958in}{1.166247in}}%
\pgfpathlineto{\pgfqpoint{0.954221in}{1.171827in}}%
\pgfpathlineto{\pgfqpoint{0.957485in}{1.177490in}}%
\pgfpathlineto{\pgfqpoint{0.956577in}{1.171135in}}%
\pgfpathlineto{\pgfqpoint{0.956060in}{1.164763in}}%
\pgfpathlineto{\pgfqpoint{0.955937in}{1.158378in}}%
\pgfpathlineto{\pgfqpoint{0.956208in}{1.151989in}}%
\pgfpathclose%
\pgfusepath{fill}%
\end{pgfscope}%
\begin{pgfscope}%
\pgfpathrectangle{\pgfqpoint{0.329460in}{0.284240in}}{\pgfqpoint{1.989680in}{1.989680in}}%
\pgfusepath{clip}%
\pgfsetbuttcap%
\pgfsetroundjoin%
\definecolor{currentfill}{rgb}{0.179019,0.433756,0.557430}%
\pgfsetfillcolor{currentfill}%
\pgfsetlinewidth{0.000000pt}%
\definecolor{currentstroke}{rgb}{0.000000,0.000000,0.000000}%
\pgfsetstrokecolor{currentstroke}%
\pgfsetdash{}{0pt}%
\pgfpathmoveto{\pgfqpoint{1.029051in}{1.323974in}}%
\pgfpathlineto{\pgfqpoint{1.025830in}{1.317514in}}%
\pgfpathlineto{\pgfqpoint{1.022611in}{1.311078in}}%
\pgfpathlineto{\pgfqpoint{1.019394in}{1.304670in}}%
\pgfpathlineto{\pgfqpoint{1.016180in}{1.298291in}}%
\pgfpathlineto{\pgfqpoint{1.018590in}{1.303566in}}%
\pgfpathlineto{\pgfqpoint{1.021325in}{1.308795in}}%
\pgfpathlineto{\pgfqpoint{1.024382in}{1.313974in}}%
\pgfpathlineto{\pgfqpoint{1.027757in}{1.319099in}}%
\pgfpathlineto{\pgfqpoint{1.030850in}{1.325257in}}%
\pgfpathlineto{\pgfqpoint{1.033945in}{1.331445in}}%
\pgfpathlineto{\pgfqpoint{1.037043in}{1.337661in}}%
\pgfpathlineto{\pgfqpoint{1.040143in}{1.343901in}}%
\pgfpathlineto{\pgfqpoint{1.036907in}{1.338993in}}%
\pgfpathlineto{\pgfqpoint{1.033978in}{1.334033in}}%
\pgfpathlineto{\pgfqpoint{1.031358in}{1.329024in}}%
\pgfpathlineto{\pgfqpoint{1.029051in}{1.323974in}}%
\pgfpathclose%
\pgfusepath{fill}%
\end{pgfscope}%
\begin{pgfscope}%
\pgfpathrectangle{\pgfqpoint{0.329460in}{0.284240in}}{\pgfqpoint{1.989680in}{1.989680in}}%
\pgfusepath{clip}%
\pgfsetbuttcap%
\pgfsetroundjoin%
\definecolor{currentfill}{rgb}{0.166383,0.690856,0.496502}%
\pgfsetfillcolor{currentfill}%
\pgfsetlinewidth{0.000000pt}%
\definecolor{currentstroke}{rgb}{0.000000,0.000000,0.000000}%
\pgfsetstrokecolor{currentstroke}%
\pgfsetdash{}{0pt}%
\pgfpathmoveto{\pgfqpoint{1.457208in}{1.583701in}}%
\pgfpathlineto{\pgfqpoint{1.458704in}{1.578772in}}%
\pgfpathlineto{\pgfqpoint{1.460199in}{1.573801in}}%
\pgfpathlineto{\pgfqpoint{1.461692in}{1.568791in}}%
\pgfpathlineto{\pgfqpoint{1.463182in}{1.563743in}}%
\pgfpathlineto{\pgfqpoint{1.470030in}{1.561996in}}%
\pgfpathlineto{\pgfqpoint{1.476765in}{1.560146in}}%
\pgfpathlineto{\pgfqpoint{1.483382in}{1.558194in}}%
\pgfpathlineto{\pgfqpoint{1.489873in}{1.556142in}}%
\pgfpathlineto{\pgfqpoint{1.488025in}{1.561297in}}%
\pgfpathlineto{\pgfqpoint{1.486175in}{1.566415in}}%
\pgfpathlineto{\pgfqpoint{1.484322in}{1.571493in}}%
\pgfpathlineto{\pgfqpoint{1.482466in}{1.576530in}}%
\pgfpathlineto{\pgfqpoint{1.476324in}{1.578466in}}%
\pgfpathlineto{\pgfqpoint{1.470063in}{1.580307in}}%
\pgfpathlineto{\pgfqpoint{1.463689in}{1.582053in}}%
\pgfpathlineto{\pgfqpoint{1.457208in}{1.583701in}}%
\pgfpathclose%
\pgfusepath{fill}%
\end{pgfscope}%
\begin{pgfscope}%
\pgfpathrectangle{\pgfqpoint{0.329460in}{0.284240in}}{\pgfqpoint{1.989680in}{1.989680in}}%
\pgfusepath{clip}%
\pgfsetbuttcap%
\pgfsetroundjoin%
\definecolor{currentfill}{rgb}{0.220124,0.725509,0.466226}%
\pgfsetfillcolor{currentfill}%
\pgfsetlinewidth{0.000000pt}%
\definecolor{currentstroke}{rgb}{0.000000,0.000000,0.000000}%
\pgfsetstrokecolor{currentstroke}%
\pgfsetdash{}{0pt}%
\pgfpathmoveto{\pgfqpoint{1.297020in}{1.611230in}}%
\pgfpathlineto{\pgfqpoint{1.296207in}{1.606618in}}%
\pgfpathlineto{\pgfqpoint{1.295394in}{1.601957in}}%
\pgfpathlineto{\pgfqpoint{1.294583in}{1.597248in}}%
\pgfpathlineto{\pgfqpoint{1.293772in}{1.592493in}}%
\pgfpathlineto{\pgfqpoint{1.300881in}{1.593304in}}%
\pgfpathlineto{\pgfqpoint{1.308038in}{1.594008in}}%
\pgfpathlineto{\pgfqpoint{1.315235in}{1.594603in}}%
\pgfpathlineto{\pgfqpoint{1.322466in}{1.595090in}}%
\pgfpathlineto{\pgfqpoint{1.322872in}{1.599806in}}%
\pgfpathlineto{\pgfqpoint{1.323277in}{1.604476in}}%
\pgfpathlineto{\pgfqpoint{1.323684in}{1.609099in}}%
\pgfpathlineto{\pgfqpoint{1.324091in}{1.613673in}}%
\pgfpathlineto{\pgfqpoint{1.317269in}{1.613215in}}%
\pgfpathlineto{\pgfqpoint{1.310479in}{1.612655in}}%
\pgfpathlineto{\pgfqpoint{1.303727in}{1.611993in}}%
\pgfpathlineto{\pgfqpoint{1.297020in}{1.611230in}}%
\pgfpathclose%
\pgfusepath{fill}%
\end{pgfscope}%
\begin{pgfscope}%
\pgfpathrectangle{\pgfqpoint{0.329460in}{0.284240in}}{\pgfqpoint{1.989680in}{1.989680in}}%
\pgfusepath{clip}%
\pgfsetbuttcap%
\pgfsetroundjoin%
\definecolor{currentfill}{rgb}{0.283072,0.130895,0.449241}%
\pgfsetfillcolor{currentfill}%
\pgfsetlinewidth{0.000000pt}%
\definecolor{currentstroke}{rgb}{0.000000,0.000000,0.000000}%
\pgfsetstrokecolor{currentstroke}%
\pgfsetdash{}{0pt}%
\pgfpathmoveto{\pgfqpoint{1.785883in}{1.093664in}}%
\pgfpathlineto{\pgfqpoint{1.789176in}{1.089020in}}%
\pgfpathlineto{\pgfqpoint{1.792469in}{1.084498in}}%
\pgfpathlineto{\pgfqpoint{1.795764in}{1.080101in}}%
\pgfpathlineto{\pgfqpoint{1.799061in}{1.075833in}}%
\pgfpathlineto{\pgfqpoint{1.798764in}{1.068498in}}%
\pgfpathlineto{\pgfqpoint{1.798018in}{1.061164in}}%
\pgfpathlineto{\pgfqpoint{1.796823in}{1.053837in}}%
\pgfpathlineto{\pgfqpoint{1.795179in}{1.046524in}}%
\pgfpathlineto{\pgfqpoint{1.791896in}{1.051029in}}%
\pgfpathlineto{\pgfqpoint{1.788615in}{1.055661in}}%
\pgfpathlineto{\pgfqpoint{1.785335in}{1.060420in}}%
\pgfpathlineto{\pgfqpoint{1.782058in}{1.065300in}}%
\pgfpathlineto{\pgfqpoint{1.783668in}{1.072375in}}%
\pgfpathlineto{\pgfqpoint{1.784842in}{1.079466in}}%
\pgfpathlineto{\pgfqpoint{1.785581in}{1.086565in}}%
\pgfpathlineto{\pgfqpoint{1.785883in}{1.093664in}}%
\pgfpathclose%
\pgfusepath{fill}%
\end{pgfscope}%
\begin{pgfscope}%
\pgfpathrectangle{\pgfqpoint{0.329460in}{0.284240in}}{\pgfqpoint{1.989680in}{1.989680in}}%
\pgfusepath{clip}%
\pgfsetbuttcap%
\pgfsetroundjoin%
\definecolor{currentfill}{rgb}{0.220124,0.725509,0.466226}%
\pgfsetfillcolor{currentfill}%
\pgfsetlinewidth{0.000000pt}%
\definecolor{currentstroke}{rgb}{0.000000,0.000000,0.000000}%
\pgfsetstrokecolor{currentstroke}%
\pgfsetdash{}{0pt}%
\pgfpathmoveto{\pgfqpoint{1.379044in}{1.613627in}}%
\pgfpathlineto{\pgfqpoint{1.379462in}{1.609053in}}%
\pgfpathlineto{\pgfqpoint{1.379880in}{1.604429in}}%
\pgfpathlineto{\pgfqpoint{1.380297in}{1.599758in}}%
\pgfpathlineto{\pgfqpoint{1.380714in}{1.595041in}}%
\pgfpathlineto{\pgfqpoint{1.387941in}{1.594542in}}%
\pgfpathlineto{\pgfqpoint{1.395134in}{1.593935in}}%
\pgfpathlineto{\pgfqpoint{1.402286in}{1.593219in}}%
\pgfpathlineto{\pgfqpoint{1.409390in}{1.592396in}}%
\pgfpathlineto{\pgfqpoint{1.408568in}{1.597152in}}%
\pgfpathlineto{\pgfqpoint{1.407746in}{1.601863in}}%
\pgfpathlineto{\pgfqpoint{1.406922in}{1.606526in}}%
\pgfpathlineto{\pgfqpoint{1.406097in}{1.611139in}}%
\pgfpathlineto{\pgfqpoint{1.399396in}{1.611913in}}%
\pgfpathlineto{\pgfqpoint{1.392649in}{1.612586in}}%
\pgfpathlineto{\pgfqpoint{1.385863in}{1.613158in}}%
\pgfpathlineto{\pgfqpoint{1.379044in}{1.613627in}}%
\pgfpathclose%
\pgfusepath{fill}%
\end{pgfscope}%
\begin{pgfscope}%
\pgfpathrectangle{\pgfqpoint{0.329460in}{0.284240in}}{\pgfqpoint{1.989680in}{1.989680in}}%
\pgfusepath{clip}%
\pgfsetbuttcap%
\pgfsetroundjoin%
\definecolor{currentfill}{rgb}{0.212395,0.359683,0.551710}%
\pgfsetfillcolor{currentfill}%
\pgfsetlinewidth{0.000000pt}%
\definecolor{currentstroke}{rgb}{0.000000,0.000000,0.000000}%
\pgfsetstrokecolor{currentstroke}%
\pgfsetdash{}{0pt}%
\pgfpathmoveto{\pgfqpoint{0.996726in}{1.250771in}}%
\pgfpathlineto{\pgfqpoint{0.993449in}{1.244355in}}%
\pgfpathlineto{\pgfqpoint{0.990173in}{1.237985in}}%
\pgfpathlineto{\pgfqpoint{0.986899in}{1.231664in}}%
\pgfpathlineto{\pgfqpoint{0.983626in}{1.225396in}}%
\pgfpathlineto{\pgfqpoint{0.984814in}{1.231262in}}%
\pgfpathlineto{\pgfqpoint{0.986363in}{1.237101in}}%
\pgfpathlineto{\pgfqpoint{0.988271in}{1.242907in}}%
\pgfpathlineto{\pgfqpoint{0.990536in}{1.248674in}}%
\pgfpathlineto{\pgfqpoint{0.993735in}{1.254713in}}%
\pgfpathlineto{\pgfqpoint{0.996936in}{1.260804in}}%
\pgfpathlineto{\pgfqpoint{1.000139in}{1.266945in}}%
\pgfpathlineto{\pgfqpoint{1.003343in}{1.273132in}}%
\pgfpathlineto{\pgfqpoint{1.001170in}{1.267591in}}%
\pgfpathlineto{\pgfqpoint{0.999342in}{1.262014in}}%
\pgfpathlineto{\pgfqpoint{0.997859in}{1.256405in}}%
\pgfpathlineto{\pgfqpoint{0.996726in}{1.250771in}}%
\pgfpathclose%
\pgfusepath{fill}%
\end{pgfscope}%
\begin{pgfscope}%
\pgfpathrectangle{\pgfqpoint{0.329460in}{0.284240in}}{\pgfqpoint{1.989680in}{1.989680in}}%
\pgfusepath{clip}%
\pgfsetbuttcap%
\pgfsetroundjoin%
\definecolor{currentfill}{rgb}{0.248629,0.278775,0.534556}%
\pgfsetfillcolor{currentfill}%
\pgfsetlinewidth{0.000000pt}%
\definecolor{currentstroke}{rgb}{0.000000,0.000000,0.000000}%
\pgfsetstrokecolor{currentstroke}%
\pgfsetdash{}{0pt}%
\pgfpathmoveto{\pgfqpoint{1.730741in}{1.206328in}}%
\pgfpathlineto{\pgfqpoint{1.733996in}{1.200418in}}%
\pgfpathlineto{\pgfqpoint{1.737250in}{1.194577in}}%
\pgfpathlineto{\pgfqpoint{1.740503in}{1.188810in}}%
\pgfpathlineto{\pgfqpoint{1.743755in}{1.183119in}}%
\pgfpathlineto{\pgfqpoint{1.745010in}{1.176785in}}%
\pgfpathlineto{\pgfqpoint{1.745875in}{1.170428in}}%
\pgfpathlineto{\pgfqpoint{1.746348in}{1.164054in}}%
\pgfpathlineto{\pgfqpoint{1.746428in}{1.157668in}}%
\pgfpathlineto{\pgfqpoint{1.743140in}{1.163594in}}%
\pgfpathlineto{\pgfqpoint{1.739852in}{1.169595in}}%
\pgfpathlineto{\pgfqpoint{1.736563in}{1.175670in}}%
\pgfpathlineto{\pgfqpoint{1.733274in}{1.181814in}}%
\pgfpathlineto{\pgfqpoint{1.733210in}{1.187964in}}%
\pgfpathlineto{\pgfqpoint{1.732765in}{1.194103in}}%
\pgfpathlineto{\pgfqpoint{1.731942in}{1.200227in}}%
\pgfpathlineto{\pgfqpoint{1.730741in}{1.206328in}}%
\pgfpathclose%
\pgfusepath{fill}%
\end{pgfscope}%
\begin{pgfscope}%
\pgfpathrectangle{\pgfqpoint{0.329460in}{0.284240in}}{\pgfqpoint{1.989680in}{1.989680in}}%
\pgfusepath{clip}%
\pgfsetbuttcap%
\pgfsetroundjoin%
\definecolor{currentfill}{rgb}{0.260571,0.246922,0.522828}%
\pgfsetfillcolor{currentfill}%
\pgfsetlinewidth{0.000000pt}%
\definecolor{currentstroke}{rgb}{0.000000,0.000000,0.000000}%
\pgfsetstrokecolor{currentstroke}%
\pgfsetdash{}{0pt}%
\pgfpathmoveto{\pgfqpoint{0.724691in}{1.119192in}}%
\pgfpathlineto{\pgfqpoint{0.720983in}{1.127583in}}%
\pgfpathlineto{\pgfqpoint{0.717259in}{1.136349in}}%
\pgfpathlineto{\pgfqpoint{0.713519in}{1.145497in}}%
\pgfpathlineto{\pgfqpoint{0.709761in}{1.155033in}}%
\pgfpathlineto{\pgfqpoint{0.709723in}{1.165398in}}%
\pgfpathlineto{\pgfqpoint{0.710324in}{1.175741in}}%
\pgfpathlineto{\pgfqpoint{0.711561in}{1.186053in}}%
\pgfpathlineto{\pgfqpoint{0.713430in}{1.196323in}}%
\pgfpathlineto{\pgfqpoint{0.717142in}{1.186608in}}%
\pgfpathlineto{\pgfqpoint{0.720838in}{1.177280in}}%
\pgfpathlineto{\pgfqpoint{0.724517in}{1.168331in}}%
\pgfpathlineto{\pgfqpoint{0.728180in}{1.159755in}}%
\pgfpathlineto{\pgfqpoint{0.726378in}{1.149665in}}%
\pgfpathlineto{\pgfqpoint{0.725193in}{1.139534in}}%
\pgfpathlineto{\pgfqpoint{0.724629in}{1.129373in}}%
\pgfpathlineto{\pgfqpoint{0.724691in}{1.119192in}}%
\pgfpathclose%
\pgfusepath{fill}%
\end{pgfscope}%
\begin{pgfscope}%
\pgfpathrectangle{\pgfqpoint{0.329460in}{0.284240in}}{\pgfqpoint{1.989680in}{1.989680in}}%
\pgfusepath{clip}%
\pgfsetbuttcap%
\pgfsetroundjoin%
\definecolor{currentfill}{rgb}{0.166383,0.690856,0.496502}%
\pgfsetfillcolor{currentfill}%
\pgfsetlinewidth{0.000000pt}%
\definecolor{currentstroke}{rgb}{0.000000,0.000000,0.000000}%
\pgfsetstrokecolor{currentstroke}%
\pgfsetdash{}{0pt}%
\pgfpathmoveto{\pgfqpoint{1.214554in}{1.574732in}}%
\pgfpathlineto{\pgfqpoint{1.212622in}{1.569668in}}%
\pgfpathlineto{\pgfqpoint{1.210693in}{1.564563in}}%
\pgfpathlineto{\pgfqpoint{1.208766in}{1.559418in}}%
\pgfpathlineto{\pgfqpoint{1.206842in}{1.554236in}}%
\pgfpathlineto{\pgfqpoint{1.213217in}{1.556375in}}%
\pgfpathlineto{\pgfqpoint{1.219722in}{1.558416in}}%
\pgfpathlineto{\pgfqpoint{1.226352in}{1.560357in}}%
\pgfpathlineto{\pgfqpoint{1.233101in}{1.562195in}}%
\pgfpathlineto{\pgfqpoint{1.234673in}{1.567265in}}%
\pgfpathlineto{\pgfqpoint{1.236247in}{1.572297in}}%
\pgfpathlineto{\pgfqpoint{1.237823in}{1.577290in}}%
\pgfpathlineto{\pgfqpoint{1.239402in}{1.582241in}}%
\pgfpathlineto{\pgfqpoint{1.233015in}{1.580506in}}%
\pgfpathlineto{\pgfqpoint{1.226741in}{1.578675in}}%
\pgfpathlineto{\pgfqpoint{1.220585in}{1.576750in}}%
\pgfpathlineto{\pgfqpoint{1.214554in}{1.574732in}}%
\pgfpathclose%
\pgfusepath{fill}%
\end{pgfscope}%
\begin{pgfscope}%
\pgfpathrectangle{\pgfqpoint{0.329460in}{0.284240in}}{\pgfqpoint{1.989680in}{1.989680in}}%
\pgfusepath{clip}%
\pgfsetbuttcap%
\pgfsetroundjoin%
\definecolor{currentfill}{rgb}{0.282327,0.094955,0.417331}%
\pgfsetfillcolor{currentfill}%
\pgfsetlinewidth{0.000000pt}%
\definecolor{currentstroke}{rgb}{0.000000,0.000000,0.000000}%
\pgfsetstrokecolor{currentstroke}%
\pgfsetdash{}{0pt}%
\pgfpathmoveto{\pgfqpoint{0.909036in}{1.040043in}}%
\pgfpathlineto{\pgfqpoint{0.905761in}{1.035619in}}%
\pgfpathlineto{\pgfqpoint{0.902485in}{1.031331in}}%
\pgfpathlineto{\pgfqpoint{0.899206in}{1.027182in}}%
\pgfpathlineto{\pgfqpoint{0.895924in}{1.023176in}}%
\pgfpathlineto{\pgfqpoint{0.893835in}{1.030704in}}%
\pgfpathlineto{\pgfqpoint{0.892208in}{1.038254in}}%
\pgfpathlineto{\pgfqpoint{0.891045in}{1.045818in}}%
\pgfpathlineto{\pgfqpoint{0.890344in}{1.053389in}}%
\pgfpathlineto{\pgfqpoint{0.893650in}{1.057159in}}%
\pgfpathlineto{\pgfqpoint{0.896954in}{1.061073in}}%
\pgfpathlineto{\pgfqpoint{0.900256in}{1.065125in}}%
\pgfpathlineto{\pgfqpoint{0.903556in}{1.069314in}}%
\pgfpathlineto{\pgfqpoint{0.904252in}{1.061978in}}%
\pgfpathlineto{\pgfqpoint{0.905397in}{1.054650in}}%
\pgfpathlineto{\pgfqpoint{0.906992in}{1.047336in}}%
\pgfpathlineto{\pgfqpoint{0.909036in}{1.040043in}}%
\pgfpathclose%
\pgfusepath{fill}%
\end{pgfscope}%
\begin{pgfscope}%
\pgfpathrectangle{\pgfqpoint{0.329460in}{0.284240in}}{\pgfqpoint{1.989680in}{1.989680in}}%
\pgfusepath{clip}%
\pgfsetbuttcap%
\pgfsetroundjoin%
\definecolor{currentfill}{rgb}{0.147607,0.511733,0.557049}%
\pgfsetfillcolor{currentfill}%
\pgfsetlinewidth{0.000000pt}%
\definecolor{currentstroke}{rgb}{0.000000,0.000000,0.000000}%
\pgfsetstrokecolor{currentstroke}%
\pgfsetdash{}{0pt}%
\pgfpathmoveto{\pgfqpoint{1.065038in}{1.394394in}}%
\pgfpathlineto{\pgfqpoint{1.061916in}{1.388051in}}%
\pgfpathlineto{\pgfqpoint{1.058798in}{1.381712in}}%
\pgfpathlineto{\pgfqpoint{1.055682in}{1.375380in}}%
\pgfpathlineto{\pgfqpoint{1.052569in}{1.369056in}}%
\pgfpathlineto{\pgfqpoint{1.055956in}{1.373694in}}%
\pgfpathlineto{\pgfqpoint{1.059630in}{1.378274in}}%
\pgfpathlineto{\pgfqpoint{1.063586in}{1.382791in}}%
\pgfpathlineto{\pgfqpoint{1.067820in}{1.387241in}}%
\pgfpathlineto{\pgfqpoint{1.070766in}{1.393358in}}%
\pgfpathlineto{\pgfqpoint{1.073714in}{1.399483in}}%
\pgfpathlineto{\pgfqpoint{1.076665in}{1.405615in}}%
\pgfpathlineto{\pgfqpoint{1.079619in}{1.411751in}}%
\pgfpathlineto{\pgfqpoint{1.075569in}{1.407503in}}%
\pgfpathlineto{\pgfqpoint{1.071787in}{1.403192in}}%
\pgfpathlineto{\pgfqpoint{1.068275in}{1.398821in}}%
\pgfpathlineto{\pgfqpoint{1.065038in}{1.394394in}}%
\pgfpathclose%
\pgfusepath{fill}%
\end{pgfscope}%
\begin{pgfscope}%
\pgfpathrectangle{\pgfqpoint{0.329460in}{0.284240in}}{\pgfqpoint{1.989680in}{1.989680in}}%
\pgfusepath{clip}%
\pgfsetbuttcap%
\pgfsetroundjoin%
\definecolor{currentfill}{rgb}{0.282327,0.094955,0.417331}%
\pgfsetfillcolor{currentfill}%
\pgfsetlinewidth{0.000000pt}%
\definecolor{currentstroke}{rgb}{0.000000,0.000000,0.000000}%
\pgfsetstrokecolor{currentstroke}%
\pgfsetdash{}{0pt}%
\pgfpathmoveto{\pgfqpoint{1.934447in}{1.054149in}}%
\pgfpathlineto{\pgfqpoint{1.937987in}{1.058539in}}%
\pgfpathlineto{\pgfqpoint{1.941539in}{1.063234in}}%
\pgfpathlineto{\pgfqpoint{1.945103in}{1.068242in}}%
\pgfpathlineto{\pgfqpoint{1.948678in}{1.073566in}}%
\pgfpathlineto{\pgfqpoint{1.948509in}{1.063774in}}%
\pgfpathlineto{\pgfqpoint{1.947741in}{1.053974in}}%
\pgfpathlineto{\pgfqpoint{1.946372in}{1.044177in}}%
\pgfpathlineto{\pgfqpoint{1.944401in}{1.034393in}}%
\pgfpathlineto{\pgfqpoint{1.940829in}{1.029272in}}%
\pgfpathlineto{\pgfqpoint{1.937270in}{1.024470in}}%
\pgfpathlineto{\pgfqpoint{1.933723in}{1.019981in}}%
\pgfpathlineto{\pgfqpoint{1.930188in}{1.015800in}}%
\pgfpathlineto{\pgfqpoint{1.932133in}{1.025377in}}%
\pgfpathlineto{\pgfqpoint{1.933490in}{1.034968in}}%
\pgfpathlineto{\pgfqpoint{1.934261in}{1.044562in}}%
\pgfpathlineto{\pgfqpoint{1.934447in}{1.054149in}}%
\pgfpathclose%
\pgfusepath{fill}%
\end{pgfscope}%
\begin{pgfscope}%
\pgfpathrectangle{\pgfqpoint{0.329460in}{0.284240in}}{\pgfqpoint{1.989680in}{1.989680in}}%
\pgfusepath{clip}%
\pgfsetbuttcap%
\pgfsetroundjoin%
\definecolor{currentfill}{rgb}{0.172719,0.448791,0.557885}%
\pgfsetfillcolor{currentfill}%
\pgfsetlinewidth{0.000000pt}%
\definecolor{currentstroke}{rgb}{0.000000,0.000000,0.000000}%
\pgfsetstrokecolor{currentstroke}%
\pgfsetdash{}{0pt}%
\pgfpathmoveto{\pgfqpoint{0.683085in}{1.288737in}}%
\pgfpathlineto{\pgfqpoint{0.679204in}{1.302230in}}%
\pgfpathlineto{\pgfqpoint{0.675301in}{1.316178in}}%
\pgfpathlineto{\pgfqpoint{0.671377in}{1.330589in}}%
\pgfpathlineto{\pgfqpoint{0.674109in}{1.341249in}}%
\pgfpathlineto{\pgfqpoint{0.677503in}{1.351842in}}%
\pgfpathlineto{\pgfqpoint{0.681551in}{1.362359in}}%
\pgfpathlineto{\pgfqpoint{0.686246in}{1.372790in}}%
\pgfpathlineto{\pgfqpoint{0.690059in}{1.358234in}}%
\pgfpathlineto{\pgfqpoint{0.693851in}{1.344139in}}%
\pgfpathlineto{\pgfqpoint{0.697622in}{1.330497in}}%
\pgfpathlineto{\pgfqpoint{0.693026in}{1.320174in}}%
\pgfpathlineto{\pgfqpoint{0.689067in}{1.309767in}}%
\pgfpathlineto{\pgfqpoint{0.685751in}{1.299285in}}%
\pgfpathlineto{\pgfqpoint{0.683085in}{1.288737in}}%
\pgfpathclose%
\pgfusepath{fill}%
\end{pgfscope}%
\begin{pgfscope}%
\pgfpathrectangle{\pgfqpoint{0.329460in}{0.284240in}}{\pgfqpoint{1.989680in}{1.989680in}}%
\pgfusepath{clip}%
\pgfsetbuttcap%
\pgfsetroundjoin%
\definecolor{currentfill}{rgb}{0.220124,0.725509,0.466226}%
\pgfsetfillcolor{currentfill}%
\pgfsetlinewidth{0.000000pt}%
\definecolor{currentstroke}{rgb}{0.000000,0.000000,0.000000}%
\pgfsetstrokecolor{currentstroke}%
\pgfsetdash{}{0pt}%
\pgfpathmoveto{\pgfqpoint{1.406097in}{1.611139in}}%
\pgfpathlineto{\pgfqpoint{1.406922in}{1.606526in}}%
\pgfpathlineto{\pgfqpoint{1.407746in}{1.601863in}}%
\pgfpathlineto{\pgfqpoint{1.408568in}{1.597152in}}%
\pgfpathlineto{\pgfqpoint{1.409390in}{1.592396in}}%
\pgfpathlineto{\pgfqpoint{1.416438in}{1.591467in}}%
\pgfpathlineto{\pgfqpoint{1.423426in}{1.590431in}}%
\pgfpathlineto{\pgfqpoint{1.430344in}{1.589291in}}%
\pgfpathlineto{\pgfqpoint{1.429227in}{1.594093in}}%
\pgfpathlineto{\pgfqpoint{1.428107in}{1.598850in}}%
\pgfpathlineto{\pgfqpoint{1.426986in}{1.603559in}}%
\pgfpathlineto{\pgfqpoint{1.425864in}{1.608219in}}%
\pgfpathlineto{\pgfqpoint{1.419337in}{1.609291in}}%
\pgfpathlineto{\pgfqpoint{1.412746in}{1.610265in}}%
\pgfpathlineto{\pgfqpoint{1.406097in}{1.611139in}}%
\pgfpathclose%
\pgfusepath{fill}%
\end{pgfscope}%
\begin{pgfscope}%
\pgfpathrectangle{\pgfqpoint{0.329460in}{0.284240in}}{\pgfqpoint{1.989680in}{1.989680in}}%
\pgfusepath{clip}%
\pgfsetbuttcap%
\pgfsetroundjoin%
\definecolor{currentfill}{rgb}{0.277941,0.056324,0.381191}%
\pgfsetfillcolor{currentfill}%
\pgfsetlinewidth{0.000000pt}%
\definecolor{currentstroke}{rgb}{0.000000,0.000000,0.000000}%
\pgfsetstrokecolor{currentstroke}%
\pgfsetdash{}{0pt}%
\pgfpathmoveto{\pgfqpoint{0.788403in}{0.993740in}}%
\pgfpathlineto{\pgfqpoint{0.784921in}{0.996695in}}%
\pgfpathlineto{\pgfqpoint{0.781428in}{0.999938in}}%
\pgfpathlineto{\pgfqpoint{0.777925in}{1.003473in}}%
\pgfpathlineto{\pgfqpoint{0.774410in}{1.007305in}}%
\pgfpathlineto{\pgfqpoint{0.771942in}{1.016863in}}%
\pgfpathlineto{\pgfqpoint{0.770062in}{1.026442in}}%
\pgfpathlineto{\pgfqpoint{0.768770in}{1.036034in}}%
\pgfpathlineto{\pgfqpoint{0.768065in}{1.045628in}}%
\pgfpathlineto{\pgfqpoint{0.771597in}{1.041586in}}%
\pgfpathlineto{\pgfqpoint{0.775119in}{1.037840in}}%
\pgfpathlineto{\pgfqpoint{0.778630in}{1.034385in}}%
\pgfpathlineto{\pgfqpoint{0.782131in}{1.031216in}}%
\pgfpathlineto{\pgfqpoint{0.782840in}{1.021833in}}%
\pgfpathlineto{\pgfqpoint{0.784121in}{1.012453in}}%
\pgfpathlineto{\pgfqpoint{0.785975in}{1.003086in}}%
\pgfpathlineto{\pgfqpoint{0.788403in}{0.993740in}}%
\pgfpathclose%
\pgfusepath{fill}%
\end{pgfscope}%
\begin{pgfscope}%
\pgfpathrectangle{\pgfqpoint{0.329460in}{0.284240in}}{\pgfqpoint{1.989680in}{1.989680in}}%
\pgfusepath{clip}%
\pgfsetbuttcap%
\pgfsetroundjoin%
\definecolor{currentfill}{rgb}{0.220124,0.725509,0.466226}%
\pgfsetfillcolor{currentfill}%
\pgfsetlinewidth{0.000000pt}%
\definecolor{currentstroke}{rgb}{0.000000,0.000000,0.000000}%
\pgfsetstrokecolor{currentstroke}%
\pgfsetdash{}{0pt}%
\pgfpathmoveto{\pgfqpoint{1.270769in}{1.607183in}}%
\pgfpathlineto{\pgfqpoint{1.269560in}{1.602507in}}%
\pgfpathlineto{\pgfqpoint{1.268353in}{1.597782in}}%
\pgfpathlineto{\pgfqpoint{1.267147in}{1.593008in}}%
\pgfpathlineto{\pgfqpoint{1.265943in}{1.588190in}}%
\pgfpathlineto{\pgfqpoint{1.272796in}{1.589422in}}%
\pgfpathlineto{\pgfqpoint{1.279723in}{1.590551in}}%
\pgfpathlineto{\pgfqpoint{1.286717in}{1.591575in}}%
\pgfpathlineto{\pgfqpoint{1.293772in}{1.592493in}}%
\pgfpathlineto{\pgfqpoint{1.294583in}{1.597248in}}%
\pgfpathlineto{\pgfqpoint{1.295394in}{1.601957in}}%
\pgfpathlineto{\pgfqpoint{1.296207in}{1.606618in}}%
\pgfpathlineto{\pgfqpoint{1.297020in}{1.611230in}}%
\pgfpathlineto{\pgfqpoint{1.290365in}{1.610367in}}%
\pgfpathlineto{\pgfqpoint{1.283767in}{1.609404in}}%
\pgfpathlineto{\pgfqpoint{1.277233in}{1.608343in}}%
\pgfpathlineto{\pgfqpoint{1.270769in}{1.607183in}}%
\pgfpathclose%
\pgfusepath{fill}%
\end{pgfscope}%
\begin{pgfscope}%
\pgfpathrectangle{\pgfqpoint{0.329460in}{0.284240in}}{\pgfqpoint{1.989680in}{1.989680in}}%
\pgfusepath{clip}%
\pgfsetbuttcap%
\pgfsetroundjoin%
\definecolor{currentfill}{rgb}{0.120081,0.622161,0.534946}%
\pgfsetfillcolor{currentfill}%
\pgfsetlinewidth{0.000000pt}%
\definecolor{currentstroke}{rgb}{0.000000,0.000000,0.000000}%
\pgfsetstrokecolor{currentstroke}%
\pgfsetdash{}{0pt}%
\pgfpathmoveto{\pgfqpoint{1.546452in}{1.514292in}}%
\pgfpathlineto{\pgfqpoint{1.548915in}{1.508700in}}%
\pgfpathlineto{\pgfqpoint{1.551376in}{1.503084in}}%
\pgfpathlineto{\pgfqpoint{1.553834in}{1.497446in}}%
\pgfpathlineto{\pgfqpoint{1.556288in}{1.491788in}}%
\pgfpathlineto{\pgfqpoint{1.561937in}{1.488603in}}%
\pgfpathlineto{\pgfqpoint{1.567384in}{1.485330in}}%
\pgfpathlineto{\pgfqpoint{1.572625in}{1.481975in}}%
\pgfpathlineto{\pgfqpoint{1.577653in}{1.478538in}}%
\pgfpathlineto{\pgfqpoint{1.574937in}{1.484367in}}%
\pgfpathlineto{\pgfqpoint{1.572217in}{1.490176in}}%
\pgfpathlineto{\pgfqpoint{1.569495in}{1.495963in}}%
\pgfpathlineto{\pgfqpoint{1.566769in}{1.501726in}}%
\pgfpathlineto{\pgfqpoint{1.561988in}{1.504985in}}%
\pgfpathlineto{\pgfqpoint{1.557005in}{1.508167in}}%
\pgfpathlineto{\pgfqpoint{1.551824in}{1.511271in}}%
\pgfpathlineto{\pgfqpoint{1.546452in}{1.514292in}}%
\pgfpathclose%
\pgfusepath{fill}%
\end{pgfscope}%
\begin{pgfscope}%
\pgfpathrectangle{\pgfqpoint{0.329460in}{0.284240in}}{\pgfqpoint{1.989680in}{1.989680in}}%
\pgfusepath{clip}%
\pgfsetbuttcap%
\pgfsetroundjoin%
\definecolor{currentfill}{rgb}{0.134692,0.658636,0.517649}%
\pgfsetfillcolor{currentfill}%
\pgfsetlinewidth{0.000000pt}%
\definecolor{currentstroke}{rgb}{0.000000,0.000000,0.000000}%
\pgfsetstrokecolor{currentstroke}%
\pgfsetdash{}{0pt}%
\pgfpathmoveto{\pgfqpoint{1.514463in}{1.546974in}}%
\pgfpathlineto{\pgfqpoint{1.516640in}{1.541653in}}%
\pgfpathlineto{\pgfqpoint{1.518813in}{1.536299in}}%
\pgfpathlineto{\pgfqpoint{1.520984in}{1.530915in}}%
\pgfpathlineto{\pgfqpoint{1.523153in}{1.525501in}}%
\pgfpathlineto{\pgfqpoint{1.529237in}{1.522835in}}%
\pgfpathlineto{\pgfqpoint{1.535153in}{1.520077in}}%
\pgfpathlineto{\pgfqpoint{1.540893in}{1.517228in}}%
\pgfpathlineto{\pgfqpoint{1.546452in}{1.514292in}}%
\pgfpathlineto{\pgfqpoint{1.543985in}{1.519857in}}%
\pgfpathlineto{\pgfqpoint{1.541516in}{1.525393in}}%
\pgfpathlineto{\pgfqpoint{1.539043in}{1.530898in}}%
\pgfpathlineto{\pgfqpoint{1.536568in}{1.536370in}}%
\pgfpathlineto{\pgfqpoint{1.531294in}{1.539147in}}%
\pgfpathlineto{\pgfqpoint{1.525849in}{1.541842in}}%
\pgfpathlineto{\pgfqpoint{1.520237in}{1.544452in}}%
\pgfpathlineto{\pgfqpoint{1.514463in}{1.546974in}}%
\pgfpathclose%
\pgfusepath{fill}%
\end{pgfscope}%
\begin{pgfscope}%
\pgfpathrectangle{\pgfqpoint{0.329460in}{0.284240in}}{\pgfqpoint{1.989680in}{1.989680in}}%
\pgfusepath{clip}%
\pgfsetbuttcap%
\pgfsetroundjoin%
\definecolor{currentfill}{rgb}{0.195860,0.395433,0.555276}%
\pgfsetfillcolor{currentfill}%
\pgfsetlinewidth{0.000000pt}%
\definecolor{currentstroke}{rgb}{0.000000,0.000000,0.000000}%
\pgfsetstrokecolor{currentstroke}%
\pgfsetdash{}{0pt}%
\pgfpathmoveto{\pgfqpoint{1.684069in}{1.302982in}}%
\pgfpathlineto{\pgfqpoint{1.687259in}{1.296686in}}%
\pgfpathlineto{\pgfqpoint{1.690446in}{1.290425in}}%
\pgfpathlineto{\pgfqpoint{1.693631in}{1.284203in}}%
\pgfpathlineto{\pgfqpoint{1.696814in}{1.278022in}}%
\pgfpathlineto{\pgfqpoint{1.699290in}{1.272518in}}%
\pgfpathlineto{\pgfqpoint{1.701425in}{1.266973in}}%
\pgfpathlineto{\pgfqpoint{1.703215in}{1.261392in}}%
\pgfpathlineto{\pgfqpoint{1.704659in}{1.255781in}}%
\pgfpathlineto{\pgfqpoint{1.701392in}{1.262189in}}%
\pgfpathlineto{\pgfqpoint{1.698123in}{1.268639in}}%
\pgfpathlineto{\pgfqpoint{1.694852in}{1.275126in}}%
\pgfpathlineto{\pgfqpoint{1.691580in}{1.281649in}}%
\pgfpathlineto{\pgfqpoint{1.690201in}{1.287031in}}%
\pgfpathlineto{\pgfqpoint{1.688488in}{1.292384in}}%
\pgfpathlineto{\pgfqpoint{1.686443in}{1.297703in}}%
\pgfpathlineto{\pgfqpoint{1.684069in}{1.302982in}}%
\pgfpathclose%
\pgfusepath{fill}%
\end{pgfscope}%
\begin{pgfscope}%
\pgfpathrectangle{\pgfqpoint{0.329460in}{0.284240in}}{\pgfqpoint{1.989680in}{1.989680in}}%
\pgfusepath{clip}%
\pgfsetbuttcap%
\pgfsetroundjoin%
\definecolor{currentfill}{rgb}{0.163625,0.471133,0.558148}%
\pgfsetfillcolor{currentfill}%
\pgfsetlinewidth{0.000000pt}%
\definecolor{currentstroke}{rgb}{0.000000,0.000000,0.000000}%
\pgfsetstrokecolor{currentstroke}%
\pgfsetdash{}{0pt}%
\pgfpathmoveto{\pgfqpoint{1.646810in}{1.373182in}}%
\pgfpathlineto{\pgfqpoint{1.649887in}{1.366917in}}%
\pgfpathlineto{\pgfqpoint{1.652961in}{1.360666in}}%
\pgfpathlineto{\pgfqpoint{1.656033in}{1.354431in}}%
\pgfpathlineto{\pgfqpoint{1.659102in}{1.348216in}}%
\pgfpathlineto{\pgfqpoint{1.662607in}{1.343359in}}%
\pgfpathlineto{\pgfqpoint{1.665809in}{1.338444in}}%
\pgfpathlineto{\pgfqpoint{1.668704in}{1.333478in}}%
\pgfpathlineto{\pgfqpoint{1.671289in}{1.328465in}}%
\pgfpathlineto{\pgfqpoint{1.668089in}{1.334897in}}%
\pgfpathlineto{\pgfqpoint{1.664885in}{1.341349in}}%
\pgfpathlineto{\pgfqpoint{1.661679in}{1.347817in}}%
\pgfpathlineto{\pgfqpoint{1.658471in}{1.354298in}}%
\pgfpathlineto{\pgfqpoint{1.655999in}{1.359091in}}%
\pgfpathlineto{\pgfqpoint{1.653229in}{1.363839in}}%
\pgfpathlineto{\pgfqpoint{1.650165in}{1.368537in}}%
\pgfpathlineto{\pgfqpoint{1.646810in}{1.373182in}}%
\pgfpathclose%
\pgfusepath{fill}%
\end{pgfscope}%
\begin{pgfscope}%
\pgfpathrectangle{\pgfqpoint{0.329460in}{0.284240in}}{\pgfqpoint{1.989680in}{1.989680in}}%
\pgfusepath{clip}%
\pgfsetbuttcap%
\pgfsetroundjoin%
\definecolor{currentfill}{rgb}{0.233603,0.313828,0.543914}%
\pgfsetfillcolor{currentfill}%
\pgfsetlinewidth{0.000000pt}%
\definecolor{currentstroke}{rgb}{0.000000,0.000000,0.000000}%
\pgfsetstrokecolor{currentstroke}%
\pgfsetdash{}{0pt}%
\pgfpathmoveto{\pgfqpoint{1.986757in}{1.205409in}}%
\pgfpathlineto{\pgfqpoint{1.990468in}{1.215555in}}%
\pgfpathlineto{\pgfqpoint{1.994196in}{1.226100in}}%
\pgfpathlineto{\pgfqpoint{1.997941in}{1.237049in}}%
\pgfpathlineto{\pgfqpoint{2.001705in}{1.248411in}}%
\pgfpathlineto{\pgfqpoint{2.004214in}{1.238017in}}%
\pgfpathlineto{\pgfqpoint{2.006080in}{1.227570in}}%
\pgfpathlineto{\pgfqpoint{2.007300in}{1.217082in}}%
\pgfpathlineto{\pgfqpoint{2.007868in}{1.206561in}}%
\pgfpathlineto{\pgfqpoint{2.004043in}{1.195368in}}%
\pgfpathlineto{\pgfqpoint{2.000237in}{1.184589in}}%
\pgfpathlineto{\pgfqpoint{1.996450in}{1.174217in}}%
\pgfpathlineto{\pgfqpoint{1.992680in}{1.164247in}}%
\pgfpathlineto{\pgfqpoint{1.992149in}{1.174593in}}%
\pgfpathlineto{\pgfqpoint{1.990982in}{1.184909in}}%
\pgfpathlineto{\pgfqpoint{1.989183in}{1.195185in}}%
\pgfpathlineto{\pgfqpoint{1.986757in}{1.205409in}}%
\pgfpathclose%
\pgfusepath{fill}%
\end{pgfscope}%
\begin{pgfscope}%
\pgfpathrectangle{\pgfqpoint{0.329460in}{0.284240in}}{\pgfqpoint{1.989680in}{1.989680in}}%
\pgfusepath{clip}%
\pgfsetbuttcap%
\pgfsetroundjoin%
\definecolor{currentfill}{rgb}{0.122606,0.585371,0.546557}%
\pgfsetfillcolor{currentfill}%
\pgfsetlinewidth{0.000000pt}%
\definecolor{currentstroke}{rgb}{0.000000,0.000000,0.000000}%
\pgfsetstrokecolor{currentstroke}%
\pgfsetdash{}{0pt}%
\pgfpathmoveto{\pgfqpoint{1.577653in}{1.478538in}}%
\pgfpathlineto{\pgfqpoint{1.580366in}{1.472691in}}%
\pgfpathlineto{\pgfqpoint{1.583077in}{1.466829in}}%
\pgfpathlineto{\pgfqpoint{1.585784in}{1.460954in}}%
\pgfpathlineto{\pgfqpoint{1.588488in}{1.455068in}}%
\pgfpathlineto{\pgfqpoint{1.593536in}{1.451372in}}%
\pgfpathlineto{\pgfqpoint{1.598352in}{1.447597in}}%
\pgfpathlineto{\pgfqpoint{1.602929in}{1.443747in}}%
\pgfpathlineto{\pgfqpoint{1.607265in}{1.439826in}}%
\pgfpathlineto{\pgfqpoint{1.604339in}{1.445900in}}%
\pgfpathlineto{\pgfqpoint{1.601411in}{1.451964in}}%
\pgfpathlineto{\pgfqpoint{1.598479in}{1.458015in}}%
\pgfpathlineto{\pgfqpoint{1.595544in}{1.464050in}}%
\pgfpathlineto{\pgfqpoint{1.591415in}{1.467776in}}%
\pgfpathlineto{\pgfqpoint{1.587053in}{1.471436in}}%
\pgfpathlineto{\pgfqpoint{1.582464in}{1.475024in}}%
\pgfpathlineto{\pgfqpoint{1.577653in}{1.478538in}}%
\pgfpathclose%
\pgfusepath{fill}%
\end{pgfscope}%
\begin{pgfscope}%
\pgfpathrectangle{\pgfqpoint{0.329460in}{0.284240in}}{\pgfqpoint{1.989680in}{1.989680in}}%
\pgfusepath{clip}%
\pgfsetbuttcap%
\pgfsetroundjoin%
\definecolor{currentfill}{rgb}{0.280255,0.165693,0.476498}%
\pgfsetfillcolor{currentfill}%
\pgfsetlinewidth{0.000000pt}%
\definecolor{currentstroke}{rgb}{0.000000,0.000000,0.000000}%
\pgfsetstrokecolor{currentstroke}%
\pgfsetdash{}{0pt}%
\pgfpathmoveto{\pgfqpoint{1.772725in}{1.113388in}}%
\pgfpathlineto{\pgfqpoint{1.776013in}{1.108291in}}%
\pgfpathlineto{\pgfqpoint{1.779302in}{1.103303in}}%
\pgfpathlineto{\pgfqpoint{1.782592in}{1.098426in}}%
\pgfpathlineto{\pgfqpoint{1.785883in}{1.093664in}}%
\pgfpathlineto{\pgfqpoint{1.785581in}{1.086565in}}%
\pgfpathlineto{\pgfqpoint{1.784842in}{1.079466in}}%
\pgfpathlineto{\pgfqpoint{1.783668in}{1.072375in}}%
\pgfpathlineto{\pgfqpoint{1.782058in}{1.065300in}}%
\pgfpathlineto{\pgfqpoint{1.778781in}{1.070298in}}%
\pgfpathlineto{\pgfqpoint{1.775506in}{1.075412in}}%
\pgfpathlineto{\pgfqpoint{1.772232in}{1.080637in}}%
\pgfpathlineto{\pgfqpoint{1.768959in}{1.085971in}}%
\pgfpathlineto{\pgfqpoint{1.770534in}{1.092810in}}%
\pgfpathlineto{\pgfqpoint{1.771687in}{1.099663in}}%
\pgfpathlineto{\pgfqpoint{1.772417in}{1.106525in}}%
\pgfpathlineto{\pgfqpoint{1.772725in}{1.113388in}}%
\pgfpathclose%
\pgfusepath{fill}%
\end{pgfscope}%
\begin{pgfscope}%
\pgfpathrectangle{\pgfqpoint{0.329460in}{0.284240in}}{\pgfqpoint{1.989680in}{1.989680in}}%
\pgfusepath{clip}%
\pgfsetbuttcap%
\pgfsetroundjoin%
\definecolor{currentfill}{rgb}{0.283072,0.130895,0.449241}%
\pgfsetfillcolor{currentfill}%
\pgfsetlinewidth{0.000000pt}%
\definecolor{currentstroke}{rgb}{0.000000,0.000000,0.000000}%
\pgfsetstrokecolor{currentstroke}%
\pgfsetdash{}{0pt}%
\pgfpathmoveto{\pgfqpoint{0.922115in}{1.059029in}}%
\pgfpathlineto{\pgfqpoint{0.918848in}{1.054096in}}%
\pgfpathlineto{\pgfqpoint{0.915579in}{1.049285in}}%
\pgfpathlineto{\pgfqpoint{0.912308in}{1.044600in}}%
\pgfpathlineto{\pgfqpoint{0.909036in}{1.040043in}}%
\pgfpathlineto{\pgfqpoint{0.906992in}{1.047336in}}%
\pgfpathlineto{\pgfqpoint{0.905397in}{1.054650in}}%
\pgfpathlineto{\pgfqpoint{0.904252in}{1.061978in}}%
\pgfpathlineto{\pgfqpoint{0.903556in}{1.069314in}}%
\pgfpathlineto{\pgfqpoint{0.906854in}{1.073634in}}%
\pgfpathlineto{\pgfqpoint{0.910151in}{1.078083in}}%
\pgfpathlineto{\pgfqpoint{0.913446in}{1.082657in}}%
\pgfpathlineto{\pgfqpoint{0.916739in}{1.087354in}}%
\pgfpathlineto{\pgfqpoint{0.917429in}{1.080255in}}%
\pgfpathlineto{\pgfqpoint{0.918555in}{1.073163in}}%
\pgfpathlineto{\pgfqpoint{0.920117in}{1.066085in}}%
\pgfpathlineto{\pgfqpoint{0.922115in}{1.059029in}}%
\pgfpathclose%
\pgfusepath{fill}%
\end{pgfscope}%
\begin{pgfscope}%
\pgfpathrectangle{\pgfqpoint{0.329460in}{0.284240in}}{\pgfqpoint{1.989680in}{1.989680in}}%
\pgfusepath{clip}%
\pgfsetbuttcap%
\pgfsetroundjoin%
\definecolor{currentfill}{rgb}{0.248629,0.278775,0.534556}%
\pgfsetfillcolor{currentfill}%
\pgfsetlinewidth{0.000000pt}%
\definecolor{currentstroke}{rgb}{0.000000,0.000000,0.000000}%
\pgfsetstrokecolor{currentstroke}%
\pgfsetdash{}{0pt}%
\pgfpathmoveto{\pgfqpoint{0.969364in}{1.176345in}}%
\pgfpathlineto{\pgfqpoint{0.966074in}{1.170147in}}%
\pgfpathlineto{\pgfqpoint{0.962785in}{1.164020in}}%
\pgfpathlineto{\pgfqpoint{0.959496in}{1.157966in}}%
\pgfpathlineto{\pgfqpoint{0.956208in}{1.151989in}}%
\pgfpathlineto{\pgfqpoint{0.955937in}{1.158378in}}%
\pgfpathlineto{\pgfqpoint{0.956060in}{1.164763in}}%
\pgfpathlineto{\pgfqpoint{0.956577in}{1.171135in}}%
\pgfpathlineto{\pgfqpoint{0.957485in}{1.177490in}}%
\pgfpathlineto{\pgfqpoint{0.960750in}{1.183233in}}%
\pgfpathlineto{\pgfqpoint{0.964015in}{1.189052in}}%
\pgfpathlineto{\pgfqpoint{0.967281in}{1.194944in}}%
\pgfpathlineto{\pgfqpoint{0.970548in}{1.200906in}}%
\pgfpathlineto{\pgfqpoint{0.969683in}{1.194785in}}%
\pgfpathlineto{\pgfqpoint{0.969196in}{1.188647in}}%
\pgfpathlineto{\pgfqpoint{0.969089in}{1.182498in}}%
\pgfpathlineto{\pgfqpoint{0.969364in}{1.176345in}}%
\pgfpathclose%
\pgfusepath{fill}%
\end{pgfscope}%
\begin{pgfscope}%
\pgfpathrectangle{\pgfqpoint{0.329460in}{0.284240in}}{\pgfqpoint{1.989680in}{1.989680in}}%
\pgfusepath{clip}%
\pgfsetbuttcap%
\pgfsetroundjoin%
\definecolor{currentfill}{rgb}{0.120081,0.622161,0.534946}%
\pgfsetfillcolor{currentfill}%
\pgfsetlinewidth{0.000000pt}%
\definecolor{currentstroke}{rgb}{0.000000,0.000000,0.000000}%
\pgfsetstrokecolor{currentstroke}%
\pgfsetdash{}{0pt}%
\pgfpathmoveto{\pgfqpoint{1.131530in}{1.498767in}}%
\pgfpathlineto{\pgfqpoint{1.128751in}{1.492965in}}%
\pgfpathlineto{\pgfqpoint{1.125976in}{1.487137in}}%
\pgfpathlineto{\pgfqpoint{1.123204in}{1.481288in}}%
\pgfpathlineto{\pgfqpoint{1.120435in}{1.475418in}}%
\pgfpathlineto{\pgfqpoint{1.125270in}{1.478924in}}%
\pgfpathlineto{\pgfqpoint{1.130322in}{1.482352in}}%
\pgfpathlineto{\pgfqpoint{1.135586in}{1.485698in}}%
\pgfpathlineto{\pgfqpoint{1.141056in}{1.488961in}}%
\pgfpathlineto{\pgfqpoint{1.143572in}{1.494655in}}%
\pgfpathlineto{\pgfqpoint{1.146091in}{1.500330in}}%
\pgfpathlineto{\pgfqpoint{1.148613in}{1.505982in}}%
\pgfpathlineto{\pgfqpoint{1.151138in}{1.511610in}}%
\pgfpathlineto{\pgfqpoint{1.145936in}{1.508516in}}%
\pgfpathlineto{\pgfqpoint{1.140931in}{1.505342in}}%
\pgfpathlineto{\pgfqpoint{1.136127in}{1.502092in}}%
\pgfpathlineto{\pgfqpoint{1.131530in}{1.498767in}}%
\pgfpathclose%
\pgfusepath{fill}%
\end{pgfscope}%
\begin{pgfscope}%
\pgfpathrectangle{\pgfqpoint{0.329460in}{0.284240in}}{\pgfqpoint{1.989680in}{1.989680in}}%
\pgfusepath{clip}%
\pgfsetbuttcap%
\pgfsetroundjoin%
\definecolor{currentfill}{rgb}{0.134692,0.658636,0.517649}%
\pgfsetfillcolor{currentfill}%
\pgfsetlinewidth{0.000000pt}%
\definecolor{currentstroke}{rgb}{0.000000,0.000000,0.000000}%
\pgfsetstrokecolor{currentstroke}%
\pgfsetdash{}{0pt}%
\pgfpathmoveto{\pgfqpoint{1.161269in}{1.533833in}}%
\pgfpathlineto{\pgfqpoint{1.158732in}{1.528325in}}%
\pgfpathlineto{\pgfqpoint{1.156198in}{1.522784in}}%
\pgfpathlineto{\pgfqpoint{1.153666in}{1.517212in}}%
\pgfpathlineto{\pgfqpoint{1.151138in}{1.511610in}}%
\pgfpathlineto{\pgfqpoint{1.156532in}{1.514622in}}%
\pgfpathlineto{\pgfqpoint{1.162111in}{1.517549in}}%
\pgfpathlineto{\pgfqpoint{1.167871in}{1.520388in}}%
\pgfpathlineto{\pgfqpoint{1.173806in}{1.523136in}}%
\pgfpathlineto{\pgfqpoint{1.176043in}{1.528582in}}%
\pgfpathlineto{\pgfqpoint{1.178283in}{1.533998in}}%
\pgfpathlineto{\pgfqpoint{1.180526in}{1.539384in}}%
\pgfpathlineto{\pgfqpoint{1.182772in}{1.544736in}}%
\pgfpathlineto{\pgfqpoint{1.177142in}{1.542136in}}%
\pgfpathlineto{\pgfqpoint{1.171677in}{1.539451in}}%
\pgfpathlineto{\pgfqpoint{1.166385in}{1.536682in}}%
\pgfpathlineto{\pgfqpoint{1.161269in}{1.533833in}}%
\pgfpathclose%
\pgfusepath{fill}%
\end{pgfscope}%
\begin{pgfscope}%
\pgfpathrectangle{\pgfqpoint{0.329460in}{0.284240in}}{\pgfqpoint{1.989680in}{1.989680in}}%
\pgfusepath{clip}%
\pgfsetbuttcap%
\pgfsetroundjoin%
\definecolor{currentfill}{rgb}{0.220124,0.725509,0.466226}%
\pgfsetfillcolor{currentfill}%
\pgfsetlinewidth{0.000000pt}%
\definecolor{currentstroke}{rgb}{0.000000,0.000000,0.000000}%
\pgfsetstrokecolor{currentstroke}%
\pgfsetdash{}{0pt}%
\pgfpathmoveto{\pgfqpoint{1.425864in}{1.608219in}}%
\pgfpathlineto{\pgfqpoint{1.426986in}{1.603559in}}%
\pgfpathlineto{\pgfqpoint{1.428107in}{1.598850in}}%
\pgfpathlineto{\pgfqpoint{1.429227in}{1.594093in}}%
\pgfpathlineto{\pgfqpoint{1.430344in}{1.589291in}}%
\pgfpathlineto{\pgfqpoint{1.437189in}{1.588046in}}%
\pgfpathlineto{\pgfqpoint{1.443952in}{1.586699in}}%
\pgfpathlineto{\pgfqpoint{1.450627in}{1.585250in}}%
\pgfpathlineto{\pgfqpoint{1.457208in}{1.583701in}}%
\pgfpathlineto{\pgfqpoint{1.455709in}{1.588587in}}%
\pgfpathlineto{\pgfqpoint{1.454209in}{1.593427in}}%
\pgfpathlineto{\pgfqpoint{1.452706in}{1.598220in}}%
\pgfpathlineto{\pgfqpoint{1.451201in}{1.602963in}}%
\pgfpathlineto{\pgfqpoint{1.444994in}{1.604419in}}%
\pgfpathlineto{\pgfqpoint{1.438698in}{1.605782in}}%
\pgfpathlineto{\pgfqpoint{1.432320in}{1.607048in}}%
\pgfpathlineto{\pgfqpoint{1.425864in}{1.608219in}}%
\pgfpathclose%
\pgfusepath{fill}%
\end{pgfscope}%
\begin{pgfscope}%
\pgfpathrectangle{\pgfqpoint{0.329460in}{0.284240in}}{\pgfqpoint{1.989680in}{1.989680in}}%
\pgfusepath{clip}%
\pgfsetbuttcap%
\pgfsetroundjoin%
\definecolor{currentfill}{rgb}{0.166383,0.690856,0.496502}%
\pgfsetfillcolor{currentfill}%
\pgfsetlinewidth{0.000000pt}%
\definecolor{currentstroke}{rgb}{0.000000,0.000000,0.000000}%
\pgfsetstrokecolor{currentstroke}%
\pgfsetdash{}{0pt}%
\pgfpathmoveto{\pgfqpoint{1.482466in}{1.576530in}}%
\pgfpathlineto{\pgfqpoint{1.484322in}{1.571493in}}%
\pgfpathlineto{\pgfqpoint{1.486175in}{1.566415in}}%
\pgfpathlineto{\pgfqpoint{1.488025in}{1.561297in}}%
\pgfpathlineto{\pgfqpoint{1.489873in}{1.556142in}}%
\pgfpathlineto{\pgfqpoint{1.496233in}{1.553992in}}%
\pgfpathlineto{\pgfqpoint{1.502456in}{1.551746in}}%
\pgfpathlineto{\pgfqpoint{1.508534in}{1.549406in}}%
\pgfpathlineto{\pgfqpoint{1.514463in}{1.546974in}}%
\pgfpathlineto{\pgfqpoint{1.512284in}{1.552259in}}%
\pgfpathlineto{\pgfqpoint{1.510102in}{1.557507in}}%
\pgfpathlineto{\pgfqpoint{1.507918in}{1.562715in}}%
\pgfpathlineto{\pgfqpoint{1.505730in}{1.567882in}}%
\pgfpathlineto{\pgfqpoint{1.500121in}{1.570176in}}%
\pgfpathlineto{\pgfqpoint{1.494371in}{1.572383in}}%
\pgfpathlineto{\pgfqpoint{1.488484in}{1.574502in}}%
\pgfpathlineto{\pgfqpoint{1.482466in}{1.576530in}}%
\pgfpathclose%
\pgfusepath{fill}%
\end{pgfscope}%
\begin{pgfscope}%
\pgfpathrectangle{\pgfqpoint{0.329460in}{0.284240in}}{\pgfqpoint{1.989680in}{1.989680in}}%
\pgfusepath{clip}%
\pgfsetbuttcap%
\pgfsetroundjoin%
\definecolor{currentfill}{rgb}{0.220124,0.725509,0.466226}%
\pgfsetfillcolor{currentfill}%
\pgfsetlinewidth{0.000000pt}%
\definecolor{currentstroke}{rgb}{0.000000,0.000000,0.000000}%
\pgfsetstrokecolor{currentstroke}%
\pgfsetdash{}{0pt}%
\pgfpathmoveto{\pgfqpoint{1.245736in}{1.601590in}}%
\pgfpathlineto{\pgfqpoint{1.244149in}{1.596825in}}%
\pgfpathlineto{\pgfqpoint{1.242565in}{1.592010in}}%
\pgfpathlineto{\pgfqpoint{1.240982in}{1.587148in}}%
\pgfpathlineto{\pgfqpoint{1.239402in}{1.582241in}}%
\pgfpathlineto{\pgfqpoint{1.245894in}{1.583878in}}%
\pgfpathlineto{\pgfqpoint{1.252486in}{1.585416in}}%
\pgfpathlineto{\pgfqpoint{1.259171in}{1.586854in}}%
\pgfpathlineto{\pgfqpoint{1.265943in}{1.588190in}}%
\pgfpathlineto{\pgfqpoint{1.267147in}{1.593008in}}%
\pgfpathlineto{\pgfqpoint{1.268353in}{1.597782in}}%
\pgfpathlineto{\pgfqpoint{1.269560in}{1.602507in}}%
\pgfpathlineto{\pgfqpoint{1.270769in}{1.607183in}}%
\pgfpathlineto{\pgfqpoint{1.264381in}{1.605927in}}%
\pgfpathlineto{\pgfqpoint{1.258076in}{1.604575in}}%
\pgfpathlineto{\pgfqpoint{1.251859in}{1.603129in}}%
\pgfpathlineto{\pgfqpoint{1.245736in}{1.601590in}}%
\pgfpathclose%
\pgfusepath{fill}%
\end{pgfscope}%
\begin{pgfscope}%
\pgfpathrectangle{\pgfqpoint{0.329460in}{0.284240in}}{\pgfqpoint{1.989680in}{1.989680in}}%
\pgfusepath{clip}%
\pgfsetbuttcap%
\pgfsetroundjoin%
\definecolor{currentfill}{rgb}{0.122606,0.585371,0.546557}%
\pgfsetfillcolor{currentfill}%
\pgfsetlinewidth{0.000000pt}%
\definecolor{currentstroke}{rgb}{0.000000,0.000000,0.000000}%
\pgfsetstrokecolor{currentstroke}%
\pgfsetdash{}{0pt}%
\pgfpathmoveto{\pgfqpoint{1.103358in}{1.460683in}}%
\pgfpathlineto{\pgfqpoint{1.100380in}{1.454604in}}%
\pgfpathlineto{\pgfqpoint{1.097405in}{1.448510in}}%
\pgfpathlineto{\pgfqpoint{1.094433in}{1.442402in}}%
\pgfpathlineto{\pgfqpoint{1.091464in}{1.436284in}}%
\pgfpathlineto{\pgfqpoint{1.095580in}{1.440265in}}%
\pgfpathlineto{\pgfqpoint{1.099942in}{1.444178in}}%
\pgfpathlineto{\pgfqpoint{1.104547in}{1.448020in}}%
\pgfpathlineto{\pgfqpoint{1.109388in}{1.451786in}}%
\pgfpathlineto{\pgfqpoint{1.112145in}{1.457713in}}%
\pgfpathlineto{\pgfqpoint{1.114905in}{1.463628in}}%
\pgfpathlineto{\pgfqpoint{1.117668in}{1.469531in}}%
\pgfpathlineto{\pgfqpoint{1.120435in}{1.475418in}}%
\pgfpathlineto{\pgfqpoint{1.115821in}{1.471838in}}%
\pgfpathlineto{\pgfqpoint{1.111434in}{1.468186in}}%
\pgfpathlineto{\pgfqpoint{1.107278in}{1.464467in}}%
\pgfpathlineto{\pgfqpoint{1.103358in}{1.460683in}}%
\pgfpathclose%
\pgfusepath{fill}%
\end{pgfscope}%
\begin{pgfscope}%
\pgfpathrectangle{\pgfqpoint{0.329460in}{0.284240in}}{\pgfqpoint{1.989680in}{1.989680in}}%
\pgfusepath{clip}%
\pgfsetbuttcap%
\pgfsetroundjoin%
\definecolor{currentfill}{rgb}{0.231674,0.318106,0.544834}%
\pgfsetfillcolor{currentfill}%
\pgfsetlinewidth{0.000000pt}%
\definecolor{currentstroke}{rgb}{0.000000,0.000000,0.000000}%
\pgfsetstrokecolor{currentstroke}%
\pgfsetdash{}{0pt}%
\pgfpathmoveto{\pgfqpoint{1.717711in}{1.230612in}}%
\pgfpathlineto{\pgfqpoint{1.720970in}{1.224451in}}%
\pgfpathlineto{\pgfqpoint{1.724229in}{1.218348in}}%
\pgfpathlineto{\pgfqpoint{1.727485in}{1.212306in}}%
\pgfpathlineto{\pgfqpoint{1.730741in}{1.206328in}}%
\pgfpathlineto{\pgfqpoint{1.731942in}{1.200227in}}%
\pgfpathlineto{\pgfqpoint{1.732765in}{1.194103in}}%
\pgfpathlineto{\pgfqpoint{1.733210in}{1.187964in}}%
\pgfpathlineto{\pgfqpoint{1.733274in}{1.181814in}}%
\pgfpathlineto{\pgfqpoint{1.729984in}{1.188026in}}%
\pgfpathlineto{\pgfqpoint{1.726693in}{1.194302in}}%
\pgfpathlineto{\pgfqpoint{1.723401in}{1.200639in}}%
\pgfpathlineto{\pgfqpoint{1.720108in}{1.207034in}}%
\pgfpathlineto{\pgfqpoint{1.720058in}{1.212948in}}%
\pgfpathlineto{\pgfqpoint{1.719641in}{1.218853in}}%
\pgfpathlineto{\pgfqpoint{1.718858in}{1.224743in}}%
\pgfpathlineto{\pgfqpoint{1.717711in}{1.230612in}}%
\pgfpathclose%
\pgfusepath{fill}%
\end{pgfscope}%
\begin{pgfscope}%
\pgfpathrectangle{\pgfqpoint{0.329460in}{0.284240in}}{\pgfqpoint{1.989680in}{1.989680in}}%
\pgfusepath{clip}%
\pgfsetbuttcap%
\pgfsetroundjoin%
\definecolor{currentfill}{rgb}{0.163625,0.471133,0.558148}%
\pgfsetfillcolor{currentfill}%
\pgfsetlinewidth{0.000000pt}%
\definecolor{currentstroke}{rgb}{0.000000,0.000000,0.000000}%
\pgfsetstrokecolor{currentstroke}%
\pgfsetdash{}{0pt}%
\pgfpathmoveto{\pgfqpoint{1.041960in}{1.350005in}}%
\pgfpathlineto{\pgfqpoint{1.038729in}{1.343473in}}%
\pgfpathlineto{\pgfqpoint{1.035500in}{1.336956in}}%
\pgfpathlineto{\pgfqpoint{1.032274in}{1.330455in}}%
\pgfpathlineto{\pgfqpoint{1.029051in}{1.323974in}}%
\pgfpathlineto{\pgfqpoint{1.031358in}{1.329024in}}%
\pgfpathlineto{\pgfqpoint{1.033978in}{1.334033in}}%
\pgfpathlineto{\pgfqpoint{1.036907in}{1.338993in}}%
\pgfpathlineto{\pgfqpoint{1.040143in}{1.343901in}}%
\pgfpathlineto{\pgfqpoint{1.043246in}{1.350164in}}%
\pgfpathlineto{\pgfqpoint{1.046351in}{1.356445in}}%
\pgfpathlineto{\pgfqpoint{1.049458in}{1.362744in}}%
\pgfpathlineto{\pgfqpoint{1.052569in}{1.369056in}}%
\pgfpathlineto{\pgfqpoint{1.049472in}{1.364363in}}%
\pgfpathlineto{\pgfqpoint{1.046669in}{1.359621in}}%
\pgfpathlineto{\pgfqpoint{1.044164in}{1.354833in}}%
\pgfpathlineto{\pgfqpoint{1.041960in}{1.350005in}}%
\pgfpathclose%
\pgfusepath{fill}%
\end{pgfscope}%
\begin{pgfscope}%
\pgfpathrectangle{\pgfqpoint{0.329460in}{0.284240in}}{\pgfqpoint{1.989680in}{1.989680in}}%
\pgfusepath{clip}%
\pgfsetbuttcap%
\pgfsetroundjoin%
\definecolor{currentfill}{rgb}{0.133743,0.548535,0.553541}%
\pgfsetfillcolor{currentfill}%
\pgfsetlinewidth{0.000000pt}%
\definecolor{currentstroke}{rgb}{0.000000,0.000000,0.000000}%
\pgfsetstrokecolor{currentstroke}%
\pgfsetdash{}{0pt}%
\pgfpathmoveto{\pgfqpoint{1.607265in}{1.439826in}}%
\pgfpathlineto{\pgfqpoint{1.610187in}{1.433743in}}%
\pgfpathlineto{\pgfqpoint{1.613107in}{1.427655in}}%
\pgfpathlineto{\pgfqpoint{1.616023in}{1.421562in}}%
\pgfpathlineto{\pgfqpoint{1.618937in}{1.415469in}}%
\pgfpathlineto{\pgfqpoint{1.623219in}{1.411282in}}%
\pgfpathlineto{\pgfqpoint{1.627239in}{1.407027in}}%
\pgfpathlineto{\pgfqpoint{1.630992in}{1.402709in}}%
\pgfpathlineto{\pgfqpoint{1.634474in}{1.398332in}}%
\pgfpathlineto{\pgfqpoint{1.631382in}{1.404629in}}%
\pgfpathlineto{\pgfqpoint{1.628288in}{1.410925in}}%
\pgfpathlineto{\pgfqpoint{1.625191in}{1.417217in}}%
\pgfpathlineto{\pgfqpoint{1.622091in}{1.423503in}}%
\pgfpathlineto{\pgfqpoint{1.618770in}{1.427672in}}%
\pgfpathlineto{\pgfqpoint{1.615190in}{1.431785in}}%
\pgfpathlineto{\pgfqpoint{1.611353in}{1.435837in}}%
\pgfpathlineto{\pgfqpoint{1.607265in}{1.439826in}}%
\pgfpathclose%
\pgfusepath{fill}%
\end{pgfscope}%
\begin{pgfscope}%
\pgfpathrectangle{\pgfqpoint{0.329460in}{0.284240in}}{\pgfqpoint{1.989680in}{1.989680in}}%
\pgfusepath{clip}%
\pgfsetbuttcap%
\pgfsetroundjoin%
\definecolor{currentfill}{rgb}{0.195860,0.395433,0.555276}%
\pgfsetfillcolor{currentfill}%
\pgfsetlinewidth{0.000000pt}%
\definecolor{currentstroke}{rgb}{0.000000,0.000000,0.000000}%
\pgfsetstrokecolor{currentstroke}%
\pgfsetdash{}{0pt}%
\pgfpathmoveto{\pgfqpoint{1.009852in}{1.276845in}}%
\pgfpathlineto{\pgfqpoint{1.006568in}{1.270271in}}%
\pgfpathlineto{\pgfqpoint{1.003285in}{1.263732in}}%
\pgfpathlineto{\pgfqpoint{1.000005in}{1.257231in}}%
\pgfpathlineto{\pgfqpoint{0.996726in}{1.250771in}}%
\pgfpathlineto{\pgfqpoint{0.997859in}{1.256405in}}%
\pgfpathlineto{\pgfqpoint{0.999342in}{1.262014in}}%
\pgfpathlineto{\pgfqpoint{1.001170in}{1.267591in}}%
\pgfpathlineto{\pgfqpoint{1.003343in}{1.273132in}}%
\pgfpathlineto{\pgfqpoint{1.006549in}{1.279363in}}%
\pgfpathlineto{\pgfqpoint{1.009758in}{1.285635in}}%
\pgfpathlineto{\pgfqpoint{1.012968in}{1.291946in}}%
\pgfpathlineto{\pgfqpoint{1.016180in}{1.298291in}}%
\pgfpathlineto{\pgfqpoint{1.014098in}{1.292977in}}%
\pgfpathlineto{\pgfqpoint{1.012348in}{1.287628in}}%
\pgfpathlineto{\pgfqpoint{1.010932in}{1.282249in}}%
\pgfpathlineto{\pgfqpoint{1.009852in}{1.276845in}}%
\pgfpathclose%
\pgfusepath{fill}%
\end{pgfscope}%
\begin{pgfscope}%
\pgfpathrectangle{\pgfqpoint{0.329460in}{0.284240in}}{\pgfqpoint{1.989680in}{1.989680in}}%
\pgfusepath{clip}%
\pgfsetbuttcap%
\pgfsetroundjoin%
\definecolor{currentfill}{rgb}{0.282884,0.135920,0.453427}%
\pgfsetfillcolor{currentfill}%
\pgfsetlinewidth{0.000000pt}%
\definecolor{currentstroke}{rgb}{0.000000,0.000000,0.000000}%
\pgfsetstrokecolor{currentstroke}%
\pgfsetdash{}{0pt}%
\pgfpathmoveto{\pgfqpoint{1.948678in}{1.073566in}}%
\pgfpathlineto{\pgfqpoint{1.952267in}{1.079213in}}%
\pgfpathlineto{\pgfqpoint{1.955868in}{1.085188in}}%
\pgfpathlineto{\pgfqpoint{1.959483in}{1.091496in}}%
\pgfpathlineto{\pgfqpoint{1.963111in}{1.098144in}}%
\pgfpathlineto{\pgfqpoint{1.962961in}{1.088152in}}%
\pgfpathlineto{\pgfqpoint{1.962196in}{1.078153in}}%
\pgfpathlineto{\pgfqpoint{1.960817in}{1.068156in}}%
\pgfpathlineto{\pgfqpoint{1.958820in}{1.058171in}}%
\pgfpathlineto{\pgfqpoint{1.955195in}{1.051721in}}%
\pgfpathlineto{\pgfqpoint{1.951583in}{1.045612in}}%
\pgfpathlineto{\pgfqpoint{1.947986in}{1.039837in}}%
\pgfpathlineto{\pgfqpoint{1.944401in}{1.034393in}}%
\pgfpathlineto{\pgfqpoint{1.946372in}{1.044177in}}%
\pgfpathlineto{\pgfqpoint{1.947741in}{1.053974in}}%
\pgfpathlineto{\pgfqpoint{1.948509in}{1.063774in}}%
\pgfpathlineto{\pgfqpoint{1.948678in}{1.073566in}}%
\pgfpathclose%
\pgfusepath{fill}%
\end{pgfscope}%
\begin{pgfscope}%
\pgfpathrectangle{\pgfqpoint{0.329460in}{0.284240in}}{\pgfqpoint{1.989680in}{1.989680in}}%
\pgfusepath{clip}%
\pgfsetbuttcap%
\pgfsetroundjoin%
\definecolor{currentfill}{rgb}{0.282327,0.094955,0.417331}%
\pgfsetfillcolor{currentfill}%
\pgfsetlinewidth{0.000000pt}%
\definecolor{currentstroke}{rgb}{0.000000,0.000000,0.000000}%
\pgfsetstrokecolor{currentstroke}%
\pgfsetdash{}{0pt}%
\pgfpathmoveto{\pgfqpoint{0.774410in}{1.007305in}}%
\pgfpathlineto{\pgfqpoint{0.770884in}{1.011440in}}%
\pgfpathlineto{\pgfqpoint{0.767345in}{1.015883in}}%
\pgfpathlineto{\pgfqpoint{0.763795in}{1.020639in}}%
\pgfpathlineto{\pgfqpoint{0.760232in}{1.025714in}}%
\pgfpathlineto{\pgfqpoint{0.757725in}{1.035479in}}%
\pgfpathlineto{\pgfqpoint{0.755821in}{1.045265in}}%
\pgfpathlineto{\pgfqpoint{0.754519in}{1.055063in}}%
\pgfpathlineto{\pgfqpoint{0.753818in}{1.064862in}}%
\pgfpathlineto{\pgfqpoint{0.757397in}{1.059583in}}%
\pgfpathlineto{\pgfqpoint{0.760965in}{1.054621in}}%
\pgfpathlineto{\pgfqpoint{0.764521in}{1.049971in}}%
\pgfpathlineto{\pgfqpoint{0.768065in}{1.045628in}}%
\pgfpathlineto{\pgfqpoint{0.768770in}{1.036034in}}%
\pgfpathlineto{\pgfqpoint{0.770062in}{1.026442in}}%
\pgfpathlineto{\pgfqpoint{0.771942in}{1.016863in}}%
\pgfpathlineto{\pgfqpoint{0.774410in}{1.007305in}}%
\pgfpathclose%
\pgfusepath{fill}%
\end{pgfscope}%
\begin{pgfscope}%
\pgfpathrectangle{\pgfqpoint{0.329460in}{0.284240in}}{\pgfqpoint{1.989680in}{1.989680in}}%
\pgfusepath{clip}%
\pgfsetbuttcap%
\pgfsetroundjoin%
\definecolor{currentfill}{rgb}{0.166383,0.690856,0.496502}%
\pgfsetfillcolor{currentfill}%
\pgfsetlinewidth{0.000000pt}%
\definecolor{currentstroke}{rgb}{0.000000,0.000000,0.000000}%
\pgfsetstrokecolor{currentstroke}%
\pgfsetdash{}{0pt}%
\pgfpathmoveto{\pgfqpoint{1.191783in}{1.565771in}}%
\pgfpathlineto{\pgfqpoint{1.189526in}{1.560573in}}%
\pgfpathlineto{\pgfqpoint{1.187272in}{1.555333in}}%
\pgfpathlineto{\pgfqpoint{1.185021in}{1.550053in}}%
\pgfpathlineto{\pgfqpoint{1.182772in}{1.544736in}}%
\pgfpathlineto{\pgfqpoint{1.188563in}{1.547248in}}%
\pgfpathlineto{\pgfqpoint{1.194509in}{1.549671in}}%
\pgfpathlineto{\pgfqpoint{1.200604in}{1.552000in}}%
\pgfpathlineto{\pgfqpoint{1.206842in}{1.554236in}}%
\pgfpathlineto{\pgfqpoint{1.208766in}{1.559418in}}%
\pgfpathlineto{\pgfqpoint{1.210693in}{1.564563in}}%
\pgfpathlineto{\pgfqpoint{1.212622in}{1.569668in}}%
\pgfpathlineto{\pgfqpoint{1.214554in}{1.574732in}}%
\pgfpathlineto{\pgfqpoint{1.208652in}{1.572623in}}%
\pgfpathlineto{\pgfqpoint{1.202886in}{1.570425in}}%
\pgfpathlineto{\pgfqpoint{1.197261in}{1.568141in}}%
\pgfpathlineto{\pgfqpoint{1.191783in}{1.565771in}}%
\pgfpathclose%
\pgfusepath{fill}%
\end{pgfscope}%
\begin{pgfscope}%
\pgfpathrectangle{\pgfqpoint{0.329460in}{0.284240in}}{\pgfqpoint{1.989680in}{1.989680in}}%
\pgfusepath{clip}%
\pgfsetbuttcap%
\pgfsetroundjoin%
\definecolor{currentfill}{rgb}{0.280255,0.165693,0.476498}%
\pgfsetfillcolor{currentfill}%
\pgfsetlinewidth{0.000000pt}%
\definecolor{currentstroke}{rgb}{0.000000,0.000000,0.000000}%
\pgfsetstrokecolor{currentstroke}%
\pgfsetdash{}{0pt}%
\pgfpathmoveto{\pgfqpoint{0.935172in}{1.079910in}}%
\pgfpathlineto{\pgfqpoint{0.931909in}{1.074524in}}%
\pgfpathlineto{\pgfqpoint{0.928646in}{1.069246in}}%
\pgfpathlineto{\pgfqpoint{0.925381in}{1.064080in}}%
\pgfpathlineto{\pgfqpoint{0.922115in}{1.059029in}}%
\pgfpathlineto{\pgfqpoint{0.920117in}{1.066085in}}%
\pgfpathlineto{\pgfqpoint{0.918555in}{1.073163in}}%
\pgfpathlineto{\pgfqpoint{0.917429in}{1.080255in}}%
\pgfpathlineto{\pgfqpoint{0.916739in}{1.087354in}}%
\pgfpathlineto{\pgfqpoint{0.920032in}{1.092168in}}%
\pgfpathlineto{\pgfqpoint{0.923323in}{1.097097in}}%
\pgfpathlineto{\pgfqpoint{0.926614in}{1.102139in}}%
\pgfpathlineto{\pgfqpoint{0.929903in}{1.107288in}}%
\pgfpathlineto{\pgfqpoint{0.930586in}{1.100426in}}%
\pgfpathlineto{\pgfqpoint{0.931692in}{1.093571in}}%
\pgfpathlineto{\pgfqpoint{0.933220in}{1.086730in}}%
\pgfpathlineto{\pgfqpoint{0.935172in}{1.079910in}}%
\pgfpathclose%
\pgfusepath{fill}%
\end{pgfscope}%
\begin{pgfscope}%
\pgfpathrectangle{\pgfqpoint{0.329460in}{0.284240in}}{\pgfqpoint{1.989680in}{1.989680in}}%
\pgfusepath{clip}%
\pgfsetbuttcap%
\pgfsetroundjoin%
\definecolor{currentfill}{rgb}{0.274128,0.199721,0.498911}%
\pgfsetfillcolor{currentfill}%
\pgfsetlinewidth{0.000000pt}%
\definecolor{currentstroke}{rgb}{0.000000,0.000000,0.000000}%
\pgfsetstrokecolor{currentstroke}%
\pgfsetdash{}{0pt}%
\pgfpathmoveto{\pgfqpoint{1.759575in}{1.134792in}}%
\pgfpathlineto{\pgfqpoint{1.762862in}{1.129295in}}%
\pgfpathlineto{\pgfqpoint{1.766149in}{1.123893in}}%
\pgfpathlineto{\pgfqpoint{1.769437in}{1.118590in}}%
\pgfpathlineto{\pgfqpoint{1.772725in}{1.113388in}}%
\pgfpathlineto{\pgfqpoint{1.772417in}{1.106525in}}%
\pgfpathlineto{\pgfqpoint{1.771687in}{1.099663in}}%
\pgfpathlineto{\pgfqpoint{1.770534in}{1.092810in}}%
\pgfpathlineto{\pgfqpoint{1.768959in}{1.085971in}}%
\pgfpathlineto{\pgfqpoint{1.765686in}{1.091410in}}%
\pgfpathlineto{\pgfqpoint{1.762415in}{1.096950in}}%
\pgfpathlineto{\pgfqpoint{1.759144in}{1.102589in}}%
\pgfpathlineto{\pgfqpoint{1.755873in}{1.108323in}}%
\pgfpathlineto{\pgfqpoint{1.757413in}{1.114924in}}%
\pgfpathlineto{\pgfqpoint{1.758543in}{1.121541in}}%
\pgfpathlineto{\pgfqpoint{1.759263in}{1.128166in}}%
\pgfpathlineto{\pgfqpoint{1.759575in}{1.134792in}}%
\pgfpathclose%
\pgfusepath{fill}%
\end{pgfscope}%
\begin{pgfscope}%
\pgfpathrectangle{\pgfqpoint{0.329460in}{0.284240in}}{\pgfqpoint{1.989680in}{1.989680in}}%
\pgfusepath{clip}%
\pgfsetbuttcap%
\pgfsetroundjoin%
\definecolor{currentfill}{rgb}{0.281477,0.755203,0.432552}%
\pgfsetfillcolor{currentfill}%
\pgfsetlinewidth{0.000000pt}%
\definecolor{currentstroke}{rgb}{0.000000,0.000000,0.000000}%
\pgfsetstrokecolor{currentstroke}%
\pgfsetdash{}{0pt}%
\pgfpathmoveto{\pgfqpoint{1.325724in}{1.631432in}}%
\pgfpathlineto{\pgfqpoint{1.325315in}{1.627076in}}%
\pgfpathlineto{\pgfqpoint{1.324906in}{1.622663in}}%
\pgfpathlineto{\pgfqpoint{1.324498in}{1.618194in}}%
\pgfpathlineto{\pgfqpoint{1.324091in}{1.613673in}}%
\pgfpathlineto{\pgfqpoint{1.330938in}{1.614028in}}%
\pgfpathlineto{\pgfqpoint{1.337805in}{1.614280in}}%
\pgfpathlineto{\pgfqpoint{1.344685in}{1.614429in}}%
\pgfpathlineto{\pgfqpoint{1.351570in}{1.614475in}}%
\pgfpathlineto{\pgfqpoint{1.351564in}{1.618984in}}%
\pgfpathlineto{\pgfqpoint{1.351559in}{1.623440in}}%
\pgfpathlineto{\pgfqpoint{1.351553in}{1.627841in}}%
\pgfpathlineto{\pgfqpoint{1.351547in}{1.632184in}}%
\pgfpathlineto{\pgfqpoint{1.345077in}{1.632141in}}%
\pgfpathlineto{\pgfqpoint{1.338612in}{1.632001in}}%
\pgfpathlineto{\pgfqpoint{1.332159in}{1.631765in}}%
\pgfpathlineto{\pgfqpoint{1.325724in}{1.631432in}}%
\pgfpathclose%
\pgfusepath{fill}%
\end{pgfscope}%
\begin{pgfscope}%
\pgfpathrectangle{\pgfqpoint{0.329460in}{0.284240in}}{\pgfqpoint{1.989680in}{1.989680in}}%
\pgfusepath{clip}%
\pgfsetbuttcap%
\pgfsetroundjoin%
\definecolor{currentfill}{rgb}{0.281477,0.755203,0.432552}%
\pgfsetfillcolor{currentfill}%
\pgfsetlinewidth{0.000000pt}%
\definecolor{currentstroke}{rgb}{0.000000,0.000000,0.000000}%
\pgfsetstrokecolor{currentstroke}%
\pgfsetdash{}{0pt}%
\pgfpathmoveto{\pgfqpoint{1.351547in}{1.632184in}}%
\pgfpathlineto{\pgfqpoint{1.351553in}{1.627841in}}%
\pgfpathlineto{\pgfqpoint{1.351559in}{1.623440in}}%
\pgfpathlineto{\pgfqpoint{1.351564in}{1.618984in}}%
\pgfpathlineto{\pgfqpoint{1.351570in}{1.614475in}}%
\pgfpathlineto{\pgfqpoint{1.358455in}{1.614418in}}%
\pgfpathlineto{\pgfqpoint{1.365334in}{1.614257in}}%
\pgfpathlineto{\pgfqpoint{1.372199in}{1.613993in}}%
\pgfpathlineto{\pgfqpoint{1.379044in}{1.613627in}}%
\pgfpathlineto{\pgfqpoint{1.378625in}{1.618149in}}%
\pgfpathlineto{\pgfqpoint{1.378205in}{1.622619in}}%
\pgfpathlineto{\pgfqpoint{1.377785in}{1.627032in}}%
\pgfpathlineto{\pgfqpoint{1.377365in}{1.631389in}}%
\pgfpathlineto{\pgfqpoint{1.370932in}{1.631732in}}%
\pgfpathlineto{\pgfqpoint{1.364481in}{1.631979in}}%
\pgfpathlineto{\pgfqpoint{1.358017in}{1.632130in}}%
\pgfpathlineto{\pgfqpoint{1.351547in}{1.632184in}}%
\pgfpathclose%
\pgfusepath{fill}%
\end{pgfscope}%
\begin{pgfscope}%
\pgfpathrectangle{\pgfqpoint{0.329460in}{0.284240in}}{\pgfqpoint{1.989680in}{1.989680in}}%
\pgfusepath{clip}%
\pgfsetbuttcap%
\pgfsetroundjoin%
\definecolor{currentfill}{rgb}{0.133743,0.548535,0.553541}%
\pgfsetfillcolor{currentfill}%
\pgfsetlinewidth{0.000000pt}%
\definecolor{currentstroke}{rgb}{0.000000,0.000000,0.000000}%
\pgfsetstrokecolor{currentstroke}%
\pgfsetdash{}{0pt}%
\pgfpathmoveto{\pgfqpoint{1.077553in}{1.419753in}}%
\pgfpathlineto{\pgfqpoint{1.074420in}{1.413420in}}%
\pgfpathlineto{\pgfqpoint{1.071289in}{1.407081in}}%
\pgfpathlineto{\pgfqpoint{1.068162in}{1.400738in}}%
\pgfpathlineto{\pgfqpoint{1.065038in}{1.394394in}}%
\pgfpathlineto{\pgfqpoint{1.068275in}{1.398821in}}%
\pgfpathlineto{\pgfqpoint{1.071787in}{1.403192in}}%
\pgfpathlineto{\pgfqpoint{1.075569in}{1.407503in}}%
\pgfpathlineto{\pgfqpoint{1.079619in}{1.411751in}}%
\pgfpathlineto{\pgfqpoint{1.082576in}{1.417888in}}%
\pgfpathlineto{\pgfqpoint{1.085535in}{1.424024in}}%
\pgfpathlineto{\pgfqpoint{1.088498in}{1.430157in}}%
\pgfpathlineto{\pgfqpoint{1.091464in}{1.436284in}}%
\pgfpathlineto{\pgfqpoint{1.087599in}{1.432238in}}%
\pgfpathlineto{\pgfqpoint{1.083990in}{1.428132in}}%
\pgfpathlineto{\pgfqpoint{1.080640in}{1.423969in}}%
\pgfpathlineto{\pgfqpoint{1.077553in}{1.419753in}}%
\pgfpathclose%
\pgfusepath{fill}%
\end{pgfscope}%
\begin{pgfscope}%
\pgfpathrectangle{\pgfqpoint{0.329460in}{0.284240in}}{\pgfqpoint{1.989680in}{1.989680in}}%
\pgfusepath{clip}%
\pgfsetbuttcap%
\pgfsetroundjoin%
\definecolor{currentfill}{rgb}{0.220124,0.725509,0.466226}%
\pgfsetfillcolor{currentfill}%
\pgfsetlinewidth{0.000000pt}%
\definecolor{currentstroke}{rgb}{0.000000,0.000000,0.000000}%
\pgfsetstrokecolor{currentstroke}%
\pgfsetdash{}{0pt}%
\pgfpathmoveto{\pgfqpoint{1.451201in}{1.602963in}}%
\pgfpathlineto{\pgfqpoint{1.452706in}{1.598220in}}%
\pgfpathlineto{\pgfqpoint{1.454209in}{1.593427in}}%
\pgfpathlineto{\pgfqpoint{1.455709in}{1.588587in}}%
\pgfpathlineto{\pgfqpoint{1.457208in}{1.583701in}}%
\pgfpathlineto{\pgfqpoint{1.463689in}{1.582053in}}%
\pgfpathlineto{\pgfqpoint{1.470063in}{1.580307in}}%
\pgfpathlineto{\pgfqpoint{1.476324in}{1.578466in}}%
\pgfpathlineto{\pgfqpoint{1.482466in}{1.576530in}}%
\pgfpathlineto{\pgfqpoint{1.480608in}{1.581523in}}%
\pgfpathlineto{\pgfqpoint{1.478748in}{1.586471in}}%
\pgfpathlineto{\pgfqpoint{1.476885in}{1.591371in}}%
\pgfpathlineto{\pgfqpoint{1.475020in}{1.596221in}}%
\pgfpathlineto{\pgfqpoint{1.469228in}{1.598040in}}%
\pgfpathlineto{\pgfqpoint{1.463324in}{1.599772in}}%
\pgfpathlineto{\pgfqpoint{1.457313in}{1.601413in}}%
\pgfpathlineto{\pgfqpoint{1.451201in}{1.602963in}}%
\pgfpathclose%
\pgfusepath{fill}%
\end{pgfscope}%
\begin{pgfscope}%
\pgfpathrectangle{\pgfqpoint{0.329460in}{0.284240in}}{\pgfqpoint{1.989680in}{1.989680in}}%
\pgfusepath{clip}%
\pgfsetbuttcap%
\pgfsetroundjoin%
\definecolor{currentfill}{rgb}{0.233603,0.313828,0.543914}%
\pgfsetfillcolor{currentfill}%
\pgfsetlinewidth{0.000000pt}%
\definecolor{currentstroke}{rgb}{0.000000,0.000000,0.000000}%
\pgfsetstrokecolor{currentstroke}%
\pgfsetdash{}{0pt}%
\pgfpathmoveto{\pgfqpoint{0.709761in}{1.155033in}}%
\pgfpathlineto{\pgfqpoint{0.705986in}{1.164964in}}%
\pgfpathlineto{\pgfqpoint{0.702193in}{1.175296in}}%
\pgfpathlineto{\pgfqpoint{0.698382in}{1.186036in}}%
\pgfpathlineto{\pgfqpoint{0.694552in}{1.197191in}}%
\pgfpathlineto{\pgfqpoint{0.694538in}{1.207731in}}%
\pgfpathlineto{\pgfqpoint{0.695179in}{1.218249in}}%
\pgfpathlineto{\pgfqpoint{0.696470in}{1.228733in}}%
\pgfpathlineto{\pgfqpoint{0.698409in}{1.239174in}}%
\pgfpathlineto{\pgfqpoint{0.702191in}{1.227850in}}%
\pgfpathlineto{\pgfqpoint{0.705955in}{1.216937in}}%
\pgfpathlineto{\pgfqpoint{0.709702in}{1.206431in}}%
\pgfpathlineto{\pgfqpoint{0.713430in}{1.196323in}}%
\pgfpathlineto{\pgfqpoint{0.711561in}{1.186053in}}%
\pgfpathlineto{\pgfqpoint{0.710324in}{1.175741in}}%
\pgfpathlineto{\pgfqpoint{0.709723in}{1.165398in}}%
\pgfpathlineto{\pgfqpoint{0.709761in}{1.155033in}}%
\pgfpathclose%
\pgfusepath{fill}%
\end{pgfscope}%
\begin{pgfscope}%
\pgfpathrectangle{\pgfqpoint{0.329460in}{0.284240in}}{\pgfqpoint{1.989680in}{1.989680in}}%
\pgfusepath{clip}%
\pgfsetbuttcap%
\pgfsetroundjoin%
\definecolor{currentfill}{rgb}{0.231674,0.318106,0.544834}%
\pgfsetfillcolor{currentfill}%
\pgfsetlinewidth{0.000000pt}%
\definecolor{currentstroke}{rgb}{0.000000,0.000000,0.000000}%
\pgfsetstrokecolor{currentstroke}%
\pgfsetdash{}{0pt}%
\pgfpathmoveto{\pgfqpoint{0.982532in}{1.201774in}}%
\pgfpathlineto{\pgfqpoint{0.979239in}{1.195327in}}%
\pgfpathlineto{\pgfqpoint{0.975946in}{1.188937in}}%
\pgfpathlineto{\pgfqpoint{0.972655in}{1.182609in}}%
\pgfpathlineto{\pgfqpoint{0.969364in}{1.176345in}}%
\pgfpathlineto{\pgfqpoint{0.969089in}{1.182498in}}%
\pgfpathlineto{\pgfqpoint{0.969196in}{1.188647in}}%
\pgfpathlineto{\pgfqpoint{0.969683in}{1.194785in}}%
\pgfpathlineto{\pgfqpoint{0.970548in}{1.200906in}}%
\pgfpathlineto{\pgfqpoint{0.973816in}{1.206935in}}%
\pgfpathlineto{\pgfqpoint{0.977085in}{1.213029in}}%
\pgfpathlineto{\pgfqpoint{0.980355in}{1.219183in}}%
\pgfpathlineto{\pgfqpoint{0.983626in}{1.225396in}}%
\pgfpathlineto{\pgfqpoint{0.982803in}{1.219508in}}%
\pgfpathlineto{\pgfqpoint{0.982345in}{1.213605in}}%
\pgfpathlineto{\pgfqpoint{0.982255in}{1.207691in}}%
\pgfpathlineto{\pgfqpoint{0.982532in}{1.201774in}}%
\pgfpathclose%
\pgfusepath{fill}%
\end{pgfscope}%
\begin{pgfscope}%
\pgfpathrectangle{\pgfqpoint{0.329460in}{0.284240in}}{\pgfqpoint{1.989680in}{1.989680in}}%
\pgfusepath{clip}%
\pgfsetbuttcap%
\pgfsetroundjoin%
\definecolor{currentfill}{rgb}{0.281477,0.755203,0.432552}%
\pgfsetfillcolor{currentfill}%
\pgfsetlinewidth{0.000000pt}%
\definecolor{currentstroke}{rgb}{0.000000,0.000000,0.000000}%
\pgfsetstrokecolor{currentstroke}%
\pgfsetdash{}{0pt}%
\pgfpathmoveto{\pgfqpoint{1.300287in}{1.629144in}}%
\pgfpathlineto{\pgfqpoint{1.299469in}{1.624749in}}%
\pgfpathlineto{\pgfqpoint{1.298651in}{1.620298in}}%
\pgfpathlineto{\pgfqpoint{1.297835in}{1.615791in}}%
\pgfpathlineto{\pgfqpoint{1.297020in}{1.611230in}}%
\pgfpathlineto{\pgfqpoint{1.303727in}{1.611993in}}%
\pgfpathlineto{\pgfqpoint{1.310479in}{1.612655in}}%
\pgfpathlineto{\pgfqpoint{1.317269in}{1.613215in}}%
\pgfpathlineto{\pgfqpoint{1.324091in}{1.613673in}}%
\pgfpathlineto{\pgfqpoint{1.324498in}{1.618194in}}%
\pgfpathlineto{\pgfqpoint{1.324906in}{1.622663in}}%
\pgfpathlineto{\pgfqpoint{1.325315in}{1.627076in}}%
\pgfpathlineto{\pgfqpoint{1.325724in}{1.631432in}}%
\pgfpathlineto{\pgfqpoint{1.319313in}{1.631003in}}%
\pgfpathlineto{\pgfqpoint{1.312933in}{1.630478in}}%
\pgfpathlineto{\pgfqpoint{1.306589in}{1.629858in}}%
\pgfpathlineto{\pgfqpoint{1.300287in}{1.629144in}}%
\pgfpathclose%
\pgfusepath{fill}%
\end{pgfscope}%
\begin{pgfscope}%
\pgfpathrectangle{\pgfqpoint{0.329460in}{0.284240in}}{\pgfqpoint{1.989680in}{1.989680in}}%
\pgfusepath{clip}%
\pgfsetbuttcap%
\pgfsetroundjoin%
\definecolor{currentfill}{rgb}{0.281477,0.755203,0.432552}%
\pgfsetfillcolor{currentfill}%
\pgfsetlinewidth{0.000000pt}%
\definecolor{currentstroke}{rgb}{0.000000,0.000000,0.000000}%
\pgfsetstrokecolor{currentstroke}%
\pgfsetdash{}{0pt}%
\pgfpathmoveto{\pgfqpoint{1.377365in}{1.631389in}}%
\pgfpathlineto{\pgfqpoint{1.377785in}{1.627032in}}%
\pgfpathlineto{\pgfqpoint{1.378205in}{1.622619in}}%
\pgfpathlineto{\pgfqpoint{1.378625in}{1.618149in}}%
\pgfpathlineto{\pgfqpoint{1.379044in}{1.613627in}}%
\pgfpathlineto{\pgfqpoint{1.385863in}{1.613158in}}%
\pgfpathlineto{\pgfqpoint{1.392649in}{1.612586in}}%
\pgfpathlineto{\pgfqpoint{1.399396in}{1.611913in}}%
\pgfpathlineto{\pgfqpoint{1.406097in}{1.611139in}}%
\pgfpathlineto{\pgfqpoint{1.405271in}{1.615701in}}%
\pgfpathlineto{\pgfqpoint{1.404444in}{1.620210in}}%
\pgfpathlineto{\pgfqpoint{1.403615in}{1.624663in}}%
\pgfpathlineto{\pgfqpoint{1.402786in}{1.629058in}}%
\pgfpathlineto{\pgfqpoint{1.396489in}{1.629784in}}%
\pgfpathlineto{\pgfqpoint{1.390149in}{1.630414in}}%
\pgfpathlineto{\pgfqpoint{1.383772in}{1.630949in}}%
\pgfpathlineto{\pgfqpoint{1.377365in}{1.631389in}}%
\pgfpathclose%
\pgfusepath{fill}%
\end{pgfscope}%
\begin{pgfscope}%
\pgfpathrectangle{\pgfqpoint{0.329460in}{0.284240in}}{\pgfqpoint{1.989680in}{1.989680in}}%
\pgfusepath{clip}%
\pgfsetbuttcap%
\pgfsetroundjoin%
\definecolor{currentfill}{rgb}{0.179019,0.433756,0.557430}%
\pgfsetfillcolor{currentfill}%
\pgfsetlinewidth{0.000000pt}%
\definecolor{currentstroke}{rgb}{0.000000,0.000000,0.000000}%
\pgfsetstrokecolor{currentstroke}%
\pgfsetdash{}{0pt}%
\pgfpathmoveto{\pgfqpoint{1.671289in}{1.328465in}}%
\pgfpathlineto{\pgfqpoint{1.674488in}{1.322055in}}%
\pgfpathlineto{\pgfqpoint{1.677684in}{1.315669in}}%
\pgfpathlineto{\pgfqpoint{1.680878in}{1.309310in}}%
\pgfpathlineto{\pgfqpoint{1.684069in}{1.302982in}}%
\pgfpathlineto{\pgfqpoint{1.686443in}{1.297703in}}%
\pgfpathlineto{\pgfqpoint{1.688488in}{1.292384in}}%
\pgfpathlineto{\pgfqpoint{1.690201in}{1.287031in}}%
\pgfpathlineto{\pgfqpoint{1.691580in}{1.281649in}}%
\pgfpathlineto{\pgfqpoint{1.688305in}{1.288205in}}%
\pgfpathlineto{\pgfqpoint{1.685028in}{1.294790in}}%
\pgfpathlineto{\pgfqpoint{1.681750in}{1.301402in}}%
\pgfpathlineto{\pgfqpoint{1.678468in}{1.308039in}}%
\pgfpathlineto{\pgfqpoint{1.677153in}{1.313192in}}%
\pgfpathlineto{\pgfqpoint{1.675517in}{1.318317in}}%
\pgfpathlineto{\pgfqpoint{1.673561in}{1.323410in}}%
\pgfpathlineto{\pgfqpoint{1.671289in}{1.328465in}}%
\pgfpathclose%
\pgfusepath{fill}%
\end{pgfscope}%
\begin{pgfscope}%
\pgfpathrectangle{\pgfqpoint{0.329460in}{0.284240in}}{\pgfqpoint{1.989680in}{1.989680in}}%
\pgfusepath{clip}%
\pgfsetbuttcap%
\pgfsetroundjoin%
\definecolor{currentfill}{rgb}{0.220124,0.725509,0.466226}%
\pgfsetfillcolor{currentfill}%
\pgfsetlinewidth{0.000000pt}%
\definecolor{currentstroke}{rgb}{0.000000,0.000000,0.000000}%
\pgfsetstrokecolor{currentstroke}%
\pgfsetdash{}{0pt}%
\pgfpathmoveto{\pgfqpoint{1.222306in}{1.594530in}}%
\pgfpathlineto{\pgfqpoint{1.220364in}{1.589653in}}%
\pgfpathlineto{\pgfqpoint{1.218425in}{1.584726in}}%
\pgfpathlineto{\pgfqpoint{1.216488in}{1.579752in}}%
\pgfpathlineto{\pgfqpoint{1.214554in}{1.574732in}}%
\pgfpathlineto{\pgfqpoint{1.220585in}{1.576750in}}%
\pgfpathlineto{\pgfqpoint{1.226741in}{1.578675in}}%
\pgfpathlineto{\pgfqpoint{1.233015in}{1.580506in}}%
\pgfpathlineto{\pgfqpoint{1.239402in}{1.582241in}}%
\pgfpathlineto{\pgfqpoint{1.240982in}{1.587148in}}%
\pgfpathlineto{\pgfqpoint{1.242565in}{1.592010in}}%
\pgfpathlineto{\pgfqpoint{1.244149in}{1.596825in}}%
\pgfpathlineto{\pgfqpoint{1.245736in}{1.601590in}}%
\pgfpathlineto{\pgfqpoint{1.239714in}{1.599959in}}%
\pgfpathlineto{\pgfqpoint{1.233797in}{1.598237in}}%
\pgfpathlineto{\pgfqpoint{1.227993in}{1.596427in}}%
\pgfpathlineto{\pgfqpoint{1.222306in}{1.594530in}}%
\pgfpathclose%
\pgfusepath{fill}%
\end{pgfscope}%
\begin{pgfscope}%
\pgfpathrectangle{\pgfqpoint{0.329460in}{0.284240in}}{\pgfqpoint{1.989680in}{1.989680in}}%
\pgfusepath{clip}%
\pgfsetbuttcap%
\pgfsetroundjoin%
\definecolor{currentfill}{rgb}{0.268510,0.009605,0.335427}%
\pgfsetfillcolor{currentfill}%
\pgfsetlinewidth{0.000000pt}%
\definecolor{currentstroke}{rgb}{0.000000,0.000000,0.000000}%
\pgfsetstrokecolor{currentstroke}%
\pgfsetdash{}{0pt}%
\pgfpathmoveto{\pgfqpoint{1.848090in}{0.994924in}}%
\pgfpathlineto{\pgfqpoint{1.851430in}{0.993170in}}%
\pgfpathlineto{\pgfqpoint{1.854776in}{0.991612in}}%
\pgfpathlineto{\pgfqpoint{1.858127in}{0.990255in}}%
\pgfpathlineto{\pgfqpoint{1.861484in}{0.989103in}}%
\pgfpathlineto{\pgfqpoint{1.859163in}{0.980649in}}%
\pgfpathlineto{\pgfqpoint{1.856324in}{0.972227in}}%
\pgfpathlineto{\pgfqpoint{1.852969in}{0.963845in}}%
\pgfpathlineto{\pgfqpoint{1.849100in}{0.955514in}}%
\pgfpathlineto{\pgfqpoint{1.845808in}{0.956894in}}%
\pgfpathlineto{\pgfqpoint{1.842521in}{0.958480in}}%
\pgfpathlineto{\pgfqpoint{1.839241in}{0.960266in}}%
\pgfpathlineto{\pgfqpoint{1.835965in}{0.962250in}}%
\pgfpathlineto{\pgfqpoint{1.839749in}{0.970354in}}%
\pgfpathlineto{\pgfqpoint{1.843032in}{0.978507in}}%
\pgfpathlineto{\pgfqpoint{1.845813in}{0.986700in}}%
\pgfpathlineto{\pgfqpoint{1.848090in}{0.994924in}}%
\pgfpathclose%
\pgfusepath{fill}%
\end{pgfscope}%
\begin{pgfscope}%
\pgfpathrectangle{\pgfqpoint{0.329460in}{0.284240in}}{\pgfqpoint{1.989680in}{1.989680in}}%
\pgfusepath{clip}%
\pgfsetbuttcap%
\pgfsetroundjoin%
\definecolor{currentfill}{rgb}{0.147607,0.511733,0.557049}%
\pgfsetfillcolor{currentfill}%
\pgfsetlinewidth{0.000000pt}%
\definecolor{currentstroke}{rgb}{0.000000,0.000000,0.000000}%
\pgfsetstrokecolor{currentstroke}%
\pgfsetdash{}{0pt}%
\pgfpathmoveto{\pgfqpoint{1.634474in}{1.398332in}}%
\pgfpathlineto{\pgfqpoint{1.637562in}{1.392036in}}%
\pgfpathlineto{\pgfqpoint{1.640647in}{1.385744in}}%
\pgfpathlineto{\pgfqpoint{1.643730in}{1.379459in}}%
\pgfpathlineto{\pgfqpoint{1.646810in}{1.373182in}}%
\pgfpathlineto{\pgfqpoint{1.650165in}{1.368537in}}%
\pgfpathlineto{\pgfqpoint{1.653229in}{1.363839in}}%
\pgfpathlineto{\pgfqpoint{1.655999in}{1.359091in}}%
\pgfpathlineto{\pgfqpoint{1.658471in}{1.354298in}}%
\pgfpathlineto{\pgfqpoint{1.655260in}{1.360792in}}%
\pgfpathlineto{\pgfqpoint{1.652046in}{1.367293in}}%
\pgfpathlineto{\pgfqpoint{1.648830in}{1.373801in}}%
\pgfpathlineto{\pgfqpoint{1.645611in}{1.380313in}}%
\pgfpathlineto{\pgfqpoint{1.643252in}{1.384885in}}%
\pgfpathlineto{\pgfqpoint{1.640607in}{1.389416in}}%
\pgfpathlineto{\pgfqpoint{1.637680in}{1.393899in}}%
\pgfpathlineto{\pgfqpoint{1.634474in}{1.398332in}}%
\pgfpathclose%
\pgfusepath{fill}%
\end{pgfscope}%
\begin{pgfscope}%
\pgfpathrectangle{\pgfqpoint{0.329460in}{0.284240in}}{\pgfqpoint{1.989680in}{1.989680in}}%
\pgfusepath{clip}%
\pgfsetbuttcap%
\pgfsetroundjoin%
\definecolor{currentfill}{rgb}{0.271305,0.019942,0.347269}%
\pgfsetfillcolor{currentfill}%
\pgfsetlinewidth{0.000000pt}%
\definecolor{currentstroke}{rgb}{0.000000,0.000000,0.000000}%
\pgfsetstrokecolor{currentstroke}%
\pgfsetdash{}{0pt}%
\pgfpathmoveto{\pgfqpoint{1.834775in}{1.003826in}}%
\pgfpathlineto{\pgfqpoint{1.838097in}{1.001326in}}%
\pgfpathlineto{\pgfqpoint{1.841423in}{0.999006in}}%
\pgfpathlineto{\pgfqpoint{1.844754in}{0.996871in}}%
\pgfpathlineto{\pgfqpoint{1.848090in}{0.994924in}}%
\pgfpathlineto{\pgfqpoint{1.845813in}{0.986700in}}%
\pgfpathlineto{\pgfqpoint{1.843032in}{0.978507in}}%
\pgfpathlineto{\pgfqpoint{1.839749in}{0.970354in}}%
\pgfpathlineto{\pgfqpoint{1.835965in}{0.962250in}}%
\pgfpathlineto{\pgfqpoint{1.832695in}{0.964427in}}%
\pgfpathlineto{\pgfqpoint{1.829429in}{0.966793in}}%
\pgfpathlineto{\pgfqpoint{1.826168in}{0.969344in}}%
\pgfpathlineto{\pgfqpoint{1.822912in}{0.972075in}}%
\pgfpathlineto{\pgfqpoint{1.826610in}{0.979949in}}%
\pgfpathlineto{\pgfqpoint{1.829820in}{0.987872in}}%
\pgfpathlineto{\pgfqpoint{1.832543in}{0.995833in}}%
\pgfpathlineto{\pgfqpoint{1.834775in}{1.003826in}}%
\pgfpathclose%
\pgfusepath{fill}%
\end{pgfscope}%
\begin{pgfscope}%
\pgfpathrectangle{\pgfqpoint{0.329460in}{0.284240in}}{\pgfqpoint{1.989680in}{1.989680in}}%
\pgfusepath{clip}%
\pgfsetbuttcap%
\pgfsetroundjoin%
\definecolor{currentfill}{rgb}{0.267004,0.004874,0.329415}%
\pgfsetfillcolor{currentfill}%
\pgfsetlinewidth{0.000000pt}%
\definecolor{currentstroke}{rgb}{0.000000,0.000000,0.000000}%
\pgfsetstrokecolor{currentstroke}%
\pgfsetdash{}{0pt}%
\pgfpathmoveto{\pgfqpoint{1.861484in}{0.989103in}}%
\pgfpathlineto{\pgfqpoint{1.864846in}{0.988159in}}%
\pgfpathlineto{\pgfqpoint{1.868215in}{0.987429in}}%
\pgfpathlineto{\pgfqpoint{1.871590in}{0.986916in}}%
\pgfpathlineto{\pgfqpoint{1.874972in}{0.986625in}}%
\pgfpathlineto{\pgfqpoint{1.872607in}{0.977944in}}%
\pgfpathlineto{\pgfqpoint{1.869711in}{0.969295in}}%
\pgfpathlineto{\pgfqpoint{1.866285in}{0.960687in}}%
\pgfpathlineto{\pgfqpoint{1.862330in}{0.952130in}}%
\pgfpathlineto{\pgfqpoint{1.859013in}{0.952647in}}%
\pgfpathlineto{\pgfqpoint{1.855702in}{0.953386in}}%
\pgfpathlineto{\pgfqpoint{1.852398in}{0.954343in}}%
\pgfpathlineto{\pgfqpoint{1.849100in}{0.955514in}}%
\pgfpathlineto{\pgfqpoint{1.852969in}{0.963845in}}%
\pgfpathlineto{\pgfqpoint{1.856324in}{0.972227in}}%
\pgfpathlineto{\pgfqpoint{1.859163in}{0.980649in}}%
\pgfpathlineto{\pgfqpoint{1.861484in}{0.989103in}}%
\pgfpathclose%
\pgfusepath{fill}%
\end{pgfscope}%
\begin{pgfscope}%
\pgfpathrectangle{\pgfqpoint{0.329460in}{0.284240in}}{\pgfqpoint{1.989680in}{1.989680in}}%
\pgfusepath{clip}%
\pgfsetbuttcap%
\pgfsetroundjoin%
\definecolor{currentfill}{rgb}{0.134692,0.658636,0.517649}%
\pgfsetfillcolor{currentfill}%
\pgfsetlinewidth{0.000000pt}%
\definecolor{currentstroke}{rgb}{0.000000,0.000000,0.000000}%
\pgfsetstrokecolor{currentstroke}%
\pgfsetdash{}{0pt}%
\pgfpathmoveto{\pgfqpoint{1.536568in}{1.536370in}}%
\pgfpathlineto{\pgfqpoint{1.539043in}{1.530898in}}%
\pgfpathlineto{\pgfqpoint{1.541516in}{1.525393in}}%
\pgfpathlineto{\pgfqpoint{1.543985in}{1.519857in}}%
\pgfpathlineto{\pgfqpoint{1.546452in}{1.514292in}}%
\pgfpathlineto{\pgfqpoint{1.551824in}{1.511271in}}%
\pgfpathlineto{\pgfqpoint{1.557005in}{1.508167in}}%
\pgfpathlineto{\pgfqpoint{1.561988in}{1.504985in}}%
\pgfpathlineto{\pgfqpoint{1.566769in}{1.501726in}}%
\pgfpathlineto{\pgfqpoint{1.564040in}{1.507461in}}%
\pgfpathlineto{\pgfqpoint{1.561308in}{1.513168in}}%
\pgfpathlineto{\pgfqpoint{1.558573in}{1.518843in}}%
\pgfpathlineto{\pgfqpoint{1.555835in}{1.524485in}}%
\pgfpathlineto{\pgfqpoint{1.551302in}{1.527567in}}%
\pgfpathlineto{\pgfqpoint{1.546577in}{1.530577in}}%
\pgfpathlineto{\pgfqpoint{1.541663in}{1.533512in}}%
\pgfpathlineto{\pgfqpoint{1.536568in}{1.536370in}}%
\pgfpathclose%
\pgfusepath{fill}%
\end{pgfscope}%
\begin{pgfscope}%
\pgfpathrectangle{\pgfqpoint{0.329460in}{0.284240in}}{\pgfqpoint{1.989680in}{1.989680in}}%
\pgfusepath{clip}%
\pgfsetbuttcap%
\pgfsetroundjoin%
\definecolor{currentfill}{rgb}{0.274128,0.199721,0.498911}%
\pgfsetfillcolor{currentfill}%
\pgfsetlinewidth{0.000000pt}%
\definecolor{currentstroke}{rgb}{0.000000,0.000000,0.000000}%
\pgfsetstrokecolor{currentstroke}%
\pgfsetdash{}{0pt}%
\pgfpathmoveto{\pgfqpoint{0.948214in}{1.102472in}}%
\pgfpathlineto{\pgfqpoint{0.944954in}{1.096686in}}%
\pgfpathlineto{\pgfqpoint{0.941694in}{1.090994in}}%
\pgfpathlineto{\pgfqpoint{0.938433in}{1.085401in}}%
\pgfpathlineto{\pgfqpoint{0.935172in}{1.079910in}}%
\pgfpathlineto{\pgfqpoint{0.933220in}{1.086730in}}%
\pgfpathlineto{\pgfqpoint{0.931692in}{1.093571in}}%
\pgfpathlineto{\pgfqpoint{0.930586in}{1.100426in}}%
\pgfpathlineto{\pgfqpoint{0.929903in}{1.107288in}}%
\pgfpathlineto{\pgfqpoint{0.933192in}{1.112542in}}%
\pgfpathlineto{\pgfqpoint{0.936481in}{1.117898in}}%
\pgfpathlineto{\pgfqpoint{0.939769in}{1.123353in}}%
\pgfpathlineto{\pgfqpoint{0.943057in}{1.128902in}}%
\pgfpathlineto{\pgfqpoint{0.943732in}{1.122277in}}%
\pgfpathlineto{\pgfqpoint{0.944817in}{1.115659in}}%
\pgfpathlineto{\pgfqpoint{0.946311in}{1.109055in}}%
\pgfpathlineto{\pgfqpoint{0.948214in}{1.102472in}}%
\pgfpathclose%
\pgfusepath{fill}%
\end{pgfscope}%
\begin{pgfscope}%
\pgfpathrectangle{\pgfqpoint{0.329460in}{0.284240in}}{\pgfqpoint{1.989680in}{1.989680in}}%
\pgfusepath{clip}%
\pgfsetbuttcap%
\pgfsetroundjoin%
\definecolor{currentfill}{rgb}{0.281477,0.755203,0.432552}%
\pgfsetfillcolor{currentfill}%
\pgfsetlinewidth{0.000000pt}%
\definecolor{currentstroke}{rgb}{0.000000,0.000000,0.000000}%
\pgfsetstrokecolor{currentstroke}%
\pgfsetdash{}{0pt}%
\pgfpathmoveto{\pgfqpoint{1.402786in}{1.629058in}}%
\pgfpathlineto{\pgfqpoint{1.403615in}{1.624663in}}%
\pgfpathlineto{\pgfqpoint{1.404444in}{1.620210in}}%
\pgfpathlineto{\pgfqpoint{1.405271in}{1.615701in}}%
\pgfpathlineto{\pgfqpoint{1.406097in}{1.611139in}}%
\pgfpathlineto{\pgfqpoint{1.412746in}{1.610265in}}%
\pgfpathlineto{\pgfqpoint{1.419337in}{1.609291in}}%
\pgfpathlineto{\pgfqpoint{1.425864in}{1.608219in}}%
\pgfpathlineto{\pgfqpoint{1.424740in}{1.612827in}}%
\pgfpathlineto{\pgfqpoint{1.423614in}{1.617381in}}%
\pgfpathlineto{\pgfqpoint{1.422487in}{1.621881in}}%
\pgfpathlineto{\pgfqpoint{1.421359in}{1.626322in}}%
\pgfpathlineto{\pgfqpoint{1.415226in}{1.627327in}}%
\pgfpathlineto{\pgfqpoint{1.409033in}{1.628239in}}%
\pgfpathlineto{\pgfqpoint{1.402786in}{1.629058in}}%
\pgfpathclose%
\pgfusepath{fill}%
\end{pgfscope}%
\begin{pgfscope}%
\pgfpathrectangle{\pgfqpoint{0.329460in}{0.284240in}}{\pgfqpoint{1.989680in}{1.989680in}}%
\pgfusepath{clip}%
\pgfsetbuttcap%
\pgfsetroundjoin%
\definecolor{currentfill}{rgb}{0.274952,0.037752,0.364543}%
\pgfsetfillcolor{currentfill}%
\pgfsetlinewidth{0.000000pt}%
\definecolor{currentstroke}{rgb}{0.000000,0.000000,0.000000}%
\pgfsetstrokecolor{currentstroke}%
\pgfsetdash{}{0pt}%
\pgfpathmoveto{\pgfqpoint{1.821526in}{1.015555in}}%
\pgfpathlineto{\pgfqpoint{1.824833in}{1.012371in}}%
\pgfpathlineto{\pgfqpoint{1.828143in}{1.009353in}}%
\pgfpathlineto{\pgfqpoint{1.831457in}{1.006503in}}%
\pgfpathlineto{\pgfqpoint{1.834775in}{1.003826in}}%
\pgfpathlineto{\pgfqpoint{1.832543in}{0.995833in}}%
\pgfpathlineto{\pgfqpoint{1.829820in}{0.987872in}}%
\pgfpathlineto{\pgfqpoint{1.826610in}{0.979949in}}%
\pgfpathlineto{\pgfqpoint{1.822912in}{0.972075in}}%
\pgfpathlineto{\pgfqpoint{1.819660in}{0.974984in}}%
\pgfpathlineto{\pgfqpoint{1.816412in}{0.978066in}}%
\pgfpathlineto{\pgfqpoint{1.813168in}{0.981317in}}%
\pgfpathlineto{\pgfqpoint{1.809927in}{0.984734in}}%
\pgfpathlineto{\pgfqpoint{1.813538in}{0.992377in}}%
\pgfpathlineto{\pgfqpoint{1.816676in}{1.000067in}}%
\pgfpathlineto{\pgfqpoint{1.819339in}{1.007795in}}%
\pgfpathlineto{\pgfqpoint{1.821526in}{1.015555in}}%
\pgfpathclose%
\pgfusepath{fill}%
\end{pgfscope}%
\begin{pgfscope}%
\pgfpathrectangle{\pgfqpoint{0.329460in}{0.284240in}}{\pgfqpoint{1.989680in}{1.989680in}}%
\pgfusepath{clip}%
\pgfsetbuttcap%
\pgfsetroundjoin%
\definecolor{currentfill}{rgb}{0.120081,0.622161,0.534946}%
\pgfsetfillcolor{currentfill}%
\pgfsetlinewidth{0.000000pt}%
\definecolor{currentstroke}{rgb}{0.000000,0.000000,0.000000}%
\pgfsetstrokecolor{currentstroke}%
\pgfsetdash{}{0pt}%
\pgfpathmoveto{\pgfqpoint{1.566769in}{1.501726in}}%
\pgfpathlineto{\pgfqpoint{1.569495in}{1.495963in}}%
\pgfpathlineto{\pgfqpoint{1.572217in}{1.490176in}}%
\pgfpathlineto{\pgfqpoint{1.574937in}{1.484367in}}%
\pgfpathlineto{\pgfqpoint{1.577653in}{1.478538in}}%
\pgfpathlineto{\pgfqpoint{1.582464in}{1.475024in}}%
\pgfpathlineto{\pgfqpoint{1.587053in}{1.471436in}}%
\pgfpathlineto{\pgfqpoint{1.591415in}{1.467776in}}%
\pgfpathlineto{\pgfqpoint{1.595544in}{1.464050in}}%
\pgfpathlineto{\pgfqpoint{1.592607in}{1.470067in}}%
\pgfpathlineto{\pgfqpoint{1.589666in}{1.476064in}}%
\pgfpathlineto{\pgfqpoint{1.586721in}{1.482038in}}%
\pgfpathlineto{\pgfqpoint{1.583774in}{1.487988in}}%
\pgfpathlineto{\pgfqpoint{1.579850in}{1.491522in}}%
\pgfpathlineto{\pgfqpoint{1.575705in}{1.494991in}}%
\pgfpathlineto{\pgfqpoint{1.571343in}{1.498394in}}%
\pgfpathlineto{\pgfqpoint{1.566769in}{1.501726in}}%
\pgfpathclose%
\pgfusepath{fill}%
\end{pgfscope}%
\begin{pgfscope}%
\pgfpathrectangle{\pgfqpoint{0.329460in}{0.284240in}}{\pgfqpoint{1.989680in}{1.989680in}}%
\pgfusepath{clip}%
\pgfsetbuttcap%
\pgfsetroundjoin%
\definecolor{currentfill}{rgb}{0.212395,0.359683,0.551710}%
\pgfsetfillcolor{currentfill}%
\pgfsetlinewidth{0.000000pt}%
\definecolor{currentstroke}{rgb}{0.000000,0.000000,0.000000}%
\pgfsetstrokecolor{currentstroke}%
\pgfsetdash{}{0pt}%
\pgfpathmoveto{\pgfqpoint{1.704659in}{1.255781in}}%
\pgfpathlineto{\pgfqpoint{1.707924in}{1.249416in}}%
\pgfpathlineto{\pgfqpoint{1.711188in}{1.243097in}}%
\pgfpathlineto{\pgfqpoint{1.714450in}{1.236828in}}%
\pgfpathlineto{\pgfqpoint{1.717711in}{1.230612in}}%
\pgfpathlineto{\pgfqpoint{1.718858in}{1.224743in}}%
\pgfpathlineto{\pgfqpoint{1.719641in}{1.218853in}}%
\pgfpathlineto{\pgfqpoint{1.720058in}{1.212948in}}%
\pgfpathlineto{\pgfqpoint{1.720108in}{1.207034in}}%
\pgfpathlineto{\pgfqpoint{1.716814in}{1.213484in}}%
\pgfpathlineto{\pgfqpoint{1.713518in}{1.219987in}}%
\pgfpathlineto{\pgfqpoint{1.710221in}{1.226538in}}%
\pgfpathlineto{\pgfqpoint{1.706923in}{1.233136in}}%
\pgfpathlineto{\pgfqpoint{1.706887in}{1.238816in}}%
\pgfpathlineto{\pgfqpoint{1.706497in}{1.244487in}}%
\pgfpathlineto{\pgfqpoint{1.705754in}{1.250144in}}%
\pgfpathlineto{\pgfqpoint{1.704659in}{1.255781in}}%
\pgfpathclose%
\pgfusepath{fill}%
\end{pgfscope}%
\begin{pgfscope}%
\pgfpathrectangle{\pgfqpoint{0.329460in}{0.284240in}}{\pgfqpoint{1.989680in}{1.989680in}}%
\pgfusepath{clip}%
\pgfsetbuttcap%
\pgfsetroundjoin%
\definecolor{currentfill}{rgb}{0.267004,0.004874,0.329415}%
\pgfsetfillcolor{currentfill}%
\pgfsetlinewidth{0.000000pt}%
\definecolor{currentstroke}{rgb}{0.000000,0.000000,0.000000}%
\pgfsetstrokecolor{currentstroke}%
\pgfsetdash{}{0pt}%
\pgfpathmoveto{\pgfqpoint{1.874972in}{0.986625in}}%
\pgfpathlineto{\pgfqpoint{1.878360in}{0.986560in}}%
\pgfpathlineto{\pgfqpoint{1.881756in}{0.986725in}}%
\pgfpathlineto{\pgfqpoint{1.885158in}{0.987125in}}%
\pgfpathlineto{\pgfqpoint{1.888569in}{0.987764in}}%
\pgfpathlineto{\pgfqpoint{1.886161in}{0.978859in}}%
\pgfpathlineto{\pgfqpoint{1.883208in}{0.969986in}}%
\pgfpathlineto{\pgfqpoint{1.879711in}{0.961155in}}%
\pgfpathlineto{\pgfqpoint{1.875671in}{0.952375in}}%
\pgfpathlineto{\pgfqpoint{1.872324in}{0.951958in}}%
\pgfpathlineto{\pgfqpoint{1.868985in}{0.951781in}}%
\pgfpathlineto{\pgfqpoint{1.865654in}{0.951840in}}%
\pgfpathlineto{\pgfqpoint{1.862330in}{0.952130in}}%
\pgfpathlineto{\pgfqpoint{1.866285in}{0.960687in}}%
\pgfpathlineto{\pgfqpoint{1.869711in}{0.969295in}}%
\pgfpathlineto{\pgfqpoint{1.872607in}{0.977944in}}%
\pgfpathlineto{\pgfqpoint{1.874972in}{0.986625in}}%
\pgfpathclose%
\pgfusepath{fill}%
\end{pgfscope}%
\begin{pgfscope}%
\pgfpathrectangle{\pgfqpoint{0.329460in}{0.284240in}}{\pgfqpoint{1.989680in}{1.989680in}}%
\pgfusepath{clip}%
\pgfsetbuttcap%
\pgfsetroundjoin%
\definecolor{currentfill}{rgb}{0.281477,0.755203,0.432552}%
\pgfsetfillcolor{currentfill}%
\pgfsetlinewidth{0.000000pt}%
\definecolor{currentstroke}{rgb}{0.000000,0.000000,0.000000}%
\pgfsetstrokecolor{currentstroke}%
\pgfsetdash{}{0pt}%
\pgfpathmoveto{\pgfqpoint{1.275622in}{1.625352in}}%
\pgfpathlineto{\pgfqpoint{1.274406in}{1.620894in}}%
\pgfpathlineto{\pgfqpoint{1.273192in}{1.616379in}}%
\pgfpathlineto{\pgfqpoint{1.271980in}{1.611808in}}%
\pgfpathlineto{\pgfqpoint{1.270769in}{1.607183in}}%
\pgfpathlineto{\pgfqpoint{1.277233in}{1.608343in}}%
\pgfpathlineto{\pgfqpoint{1.283767in}{1.609404in}}%
\pgfpathlineto{\pgfqpoint{1.290365in}{1.610367in}}%
\pgfpathlineto{\pgfqpoint{1.297020in}{1.611230in}}%
\pgfpathlineto{\pgfqpoint{1.297835in}{1.615791in}}%
\pgfpathlineto{\pgfqpoint{1.298651in}{1.620298in}}%
\pgfpathlineto{\pgfqpoint{1.299469in}{1.624749in}}%
\pgfpathlineto{\pgfqpoint{1.300287in}{1.629144in}}%
\pgfpathlineto{\pgfqpoint{1.294033in}{1.628335in}}%
\pgfpathlineto{\pgfqpoint{1.287834in}{1.627433in}}%
\pgfpathlineto{\pgfqpoint{1.281695in}{1.626439in}}%
\pgfpathlineto{\pgfqpoint{1.275622in}{1.625352in}}%
\pgfpathclose%
\pgfusepath{fill}%
\end{pgfscope}%
\begin{pgfscope}%
\pgfpathrectangle{\pgfqpoint{0.329460in}{0.284240in}}{\pgfqpoint{1.989680in}{1.989680in}}%
\pgfusepath{clip}%
\pgfsetbuttcap%
\pgfsetroundjoin%
\definecolor{currentfill}{rgb}{0.166383,0.690856,0.496502}%
\pgfsetfillcolor{currentfill}%
\pgfsetlinewidth{0.000000pt}%
\definecolor{currentstroke}{rgb}{0.000000,0.000000,0.000000}%
\pgfsetstrokecolor{currentstroke}%
\pgfsetdash{}{0pt}%
\pgfpathmoveto{\pgfqpoint{1.505730in}{1.567882in}}%
\pgfpathlineto{\pgfqpoint{1.507918in}{1.562715in}}%
\pgfpathlineto{\pgfqpoint{1.510102in}{1.557507in}}%
\pgfpathlineto{\pgfqpoint{1.512284in}{1.552259in}}%
\pgfpathlineto{\pgfqpoint{1.514463in}{1.546974in}}%
\pgfpathlineto{\pgfqpoint{1.520237in}{1.544452in}}%
\pgfpathlineto{\pgfqpoint{1.525849in}{1.541842in}}%
\pgfpathlineto{\pgfqpoint{1.531294in}{1.539147in}}%
\pgfpathlineto{\pgfqpoint{1.536568in}{1.536370in}}%
\pgfpathlineto{\pgfqpoint{1.534089in}{1.541806in}}%
\pgfpathlineto{\pgfqpoint{1.531608in}{1.547205in}}%
\pgfpathlineto{\pgfqpoint{1.529123in}{1.552564in}}%
\pgfpathlineto{\pgfqpoint{1.526635in}{1.557881in}}%
\pgfpathlineto{\pgfqpoint{1.521649in}{1.560500in}}%
\pgfpathlineto{\pgfqpoint{1.516499in}{1.563042in}}%
\pgfpathlineto{\pgfqpoint{1.511191in}{1.565503in}}%
\pgfpathlineto{\pgfqpoint{1.505730in}{1.567882in}}%
\pgfpathclose%
\pgfusepath{fill}%
\end{pgfscope}%
\begin{pgfscope}%
\pgfpathrectangle{\pgfqpoint{0.329460in}{0.284240in}}{\pgfqpoint{1.989680in}{1.989680in}}%
\pgfusepath{clip}%
\pgfsetbuttcap%
\pgfsetroundjoin%
\definecolor{currentfill}{rgb}{0.263663,0.237631,0.518762}%
\pgfsetfillcolor{currentfill}%
\pgfsetlinewidth{0.000000pt}%
\definecolor{currentstroke}{rgb}{0.000000,0.000000,0.000000}%
\pgfsetstrokecolor{currentstroke}%
\pgfsetdash{}{0pt}%
\pgfpathmoveto{\pgfqpoint{1.746428in}{1.157668in}}%
\pgfpathlineto{\pgfqpoint{1.749715in}{1.151823in}}%
\pgfpathlineto{\pgfqpoint{1.753002in}{1.146059in}}%
\pgfpathlineto{\pgfqpoint{1.756289in}{1.140381in}}%
\pgfpathlineto{\pgfqpoint{1.759575in}{1.134792in}}%
\pgfpathlineto{\pgfqpoint{1.759263in}{1.128166in}}%
\pgfpathlineto{\pgfqpoint{1.758543in}{1.121541in}}%
\pgfpathlineto{\pgfqpoint{1.757413in}{1.114924in}}%
\pgfpathlineto{\pgfqpoint{1.755873in}{1.108323in}}%
\pgfpathlineto{\pgfqpoint{1.752603in}{1.114149in}}%
\pgfpathlineto{\pgfqpoint{1.749332in}{1.120063in}}%
\pgfpathlineto{\pgfqpoint{1.746062in}{1.126063in}}%
\pgfpathlineto{\pgfqpoint{1.742792in}{1.132146in}}%
\pgfpathlineto{\pgfqpoint{1.744295in}{1.138511in}}%
\pgfpathlineto{\pgfqpoint{1.745402in}{1.144891in}}%
\pgfpathlineto{\pgfqpoint{1.746113in}{1.151279in}}%
\pgfpathlineto{\pgfqpoint{1.746428in}{1.157668in}}%
\pgfpathclose%
\pgfusepath{fill}%
\end{pgfscope}%
\begin{pgfscope}%
\pgfpathrectangle{\pgfqpoint{0.329460in}{0.284240in}}{\pgfqpoint{1.989680in}{1.989680in}}%
\pgfusepath{clip}%
\pgfsetbuttcap%
\pgfsetroundjoin%
\definecolor{currentfill}{rgb}{0.282884,0.135920,0.453427}%
\pgfsetfillcolor{currentfill}%
\pgfsetlinewidth{0.000000pt}%
\definecolor{currentstroke}{rgb}{0.000000,0.000000,0.000000}%
\pgfsetstrokecolor{currentstroke}%
\pgfsetdash{}{0pt}%
\pgfpathmoveto{\pgfqpoint{0.760232in}{1.025714in}}%
\pgfpathlineto{\pgfqpoint{0.756657in}{1.031114in}}%
\pgfpathlineto{\pgfqpoint{0.753068in}{1.036843in}}%
\pgfpathlineto{\pgfqpoint{0.749465in}{1.042908in}}%
\pgfpathlineto{\pgfqpoint{0.745849in}{1.049314in}}%
\pgfpathlineto{\pgfqpoint{0.743303in}{1.059280in}}%
\pgfpathlineto{\pgfqpoint{0.741375in}{1.069267in}}%
\pgfpathlineto{\pgfqpoint{0.740063in}{1.079264in}}%
\pgfpathlineto{\pgfqpoint{0.739367in}{1.089263in}}%
\pgfpathlineto{\pgfqpoint{0.743000in}{1.082659in}}%
\pgfpathlineto{\pgfqpoint{0.746619in}{1.076395in}}%
\pgfpathlineto{\pgfqpoint{0.750225in}{1.070464in}}%
\pgfpathlineto{\pgfqpoint{0.753818in}{1.064862in}}%
\pgfpathlineto{\pgfqpoint{0.754519in}{1.055063in}}%
\pgfpathlineto{\pgfqpoint{0.755821in}{1.045265in}}%
\pgfpathlineto{\pgfqpoint{0.757725in}{1.035479in}}%
\pgfpathlineto{\pgfqpoint{0.760232in}{1.025714in}}%
\pgfpathclose%
\pgfusepath{fill}%
\end{pgfscope}%
\begin{pgfscope}%
\pgfpathrectangle{\pgfqpoint{0.329460in}{0.284240in}}{\pgfqpoint{1.989680in}{1.989680in}}%
\pgfusepath{clip}%
\pgfsetbuttcap%
\pgfsetroundjoin%
\definecolor{currentfill}{rgb}{0.179019,0.433756,0.557430}%
\pgfsetfillcolor{currentfill}%
\pgfsetlinewidth{0.000000pt}%
\definecolor{currentstroke}{rgb}{0.000000,0.000000,0.000000}%
\pgfsetstrokecolor{currentstroke}%
\pgfsetdash{}{0pt}%
\pgfpathmoveto{\pgfqpoint{1.023009in}{1.303439in}}%
\pgfpathlineto{\pgfqpoint{1.019717in}{1.296752in}}%
\pgfpathlineto{\pgfqpoint{1.016426in}{1.290088in}}%
\pgfpathlineto{\pgfqpoint{1.013138in}{1.283452in}}%
\pgfpathlineto{\pgfqpoint{1.009852in}{1.276845in}}%
\pgfpathlineto{\pgfqpoint{1.010932in}{1.282249in}}%
\pgfpathlineto{\pgfqpoint{1.012348in}{1.287628in}}%
\pgfpathlineto{\pgfqpoint{1.014098in}{1.292977in}}%
\pgfpathlineto{\pgfqpoint{1.016180in}{1.298291in}}%
\pgfpathlineto{\pgfqpoint{1.019394in}{1.304670in}}%
\pgfpathlineto{\pgfqpoint{1.022611in}{1.311078in}}%
\pgfpathlineto{\pgfqpoint{1.025830in}{1.317514in}}%
\pgfpathlineto{\pgfqpoint{1.029051in}{1.323974in}}%
\pgfpathlineto{\pgfqpoint{1.027060in}{1.318885in}}%
\pgfpathlineto{\pgfqpoint{1.025388in}{1.313763in}}%
\pgfpathlineto{\pgfqpoint{1.024037in}{1.308612in}}%
\pgfpathlineto{\pgfqpoint{1.023009in}{1.303439in}}%
\pgfpathclose%
\pgfusepath{fill}%
\end{pgfscope}%
\begin{pgfscope}%
\pgfpathrectangle{\pgfqpoint{0.329460in}{0.284240in}}{\pgfqpoint{1.989680in}{1.989680in}}%
\pgfusepath{clip}%
\pgfsetbuttcap%
\pgfsetroundjoin%
\definecolor{currentfill}{rgb}{0.134692,0.658636,0.517649}%
\pgfsetfillcolor{currentfill}%
\pgfsetlinewidth{0.000000pt}%
\definecolor{currentstroke}{rgb}{0.000000,0.000000,0.000000}%
\pgfsetstrokecolor{currentstroke}%
\pgfsetdash{}{0pt}%
\pgfpathmoveto{\pgfqpoint{1.142676in}{1.521687in}}%
\pgfpathlineto{\pgfqpoint{1.139885in}{1.516005in}}%
\pgfpathlineto{\pgfqpoint{1.137097in}{1.510290in}}%
\pgfpathlineto{\pgfqpoint{1.134312in}{1.504543in}}%
\pgfpathlineto{\pgfqpoint{1.131530in}{1.498767in}}%
\pgfpathlineto{\pgfqpoint{1.136127in}{1.502092in}}%
\pgfpathlineto{\pgfqpoint{1.140931in}{1.505342in}}%
\pgfpathlineto{\pgfqpoint{1.145936in}{1.508516in}}%
\pgfpathlineto{\pgfqpoint{1.151138in}{1.511610in}}%
\pgfpathlineto{\pgfqpoint{1.153666in}{1.517212in}}%
\pgfpathlineto{\pgfqpoint{1.156198in}{1.522784in}}%
\pgfpathlineto{\pgfqpoint{1.158732in}{1.528325in}}%
\pgfpathlineto{\pgfqpoint{1.161269in}{1.533833in}}%
\pgfpathlineto{\pgfqpoint{1.156335in}{1.530907in}}%
\pgfpathlineto{\pgfqpoint{1.151589in}{1.527905in}}%
\pgfpathlineto{\pgfqpoint{1.147034in}{1.524831in}}%
\pgfpathlineto{\pgfqpoint{1.142676in}{1.521687in}}%
\pgfpathclose%
\pgfusepath{fill}%
\end{pgfscope}%
\begin{pgfscope}%
\pgfpathrectangle{\pgfqpoint{0.329460in}{0.284240in}}{\pgfqpoint{1.989680in}{1.989680in}}%
\pgfusepath{clip}%
\pgfsetbuttcap%
\pgfsetroundjoin%
\definecolor{currentfill}{rgb}{0.276194,0.190074,0.493001}%
\pgfsetfillcolor{currentfill}%
\pgfsetlinewidth{0.000000pt}%
\definecolor{currentstroke}{rgb}{0.000000,0.000000,0.000000}%
\pgfsetstrokecolor{currentstroke}%
\pgfsetdash{}{0pt}%
\pgfpathmoveto{\pgfqpoint{1.963111in}{1.098144in}}%
\pgfpathlineto{\pgfqpoint{1.966754in}{1.105136in}}%
\pgfpathlineto{\pgfqpoint{1.970411in}{1.112480in}}%
\pgfpathlineto{\pgfqpoint{1.974083in}{1.120180in}}%
\pgfpathlineto{\pgfqpoint{1.977770in}{1.128243in}}%
\pgfpathlineto{\pgfqpoint{1.977639in}{1.118060in}}%
\pgfpathlineto{\pgfqpoint{1.976880in}{1.107868in}}%
\pgfpathlineto{\pgfqpoint{1.975490in}{1.097678in}}%
\pgfpathlineto{\pgfqpoint{1.973469in}{1.087499in}}%
\pgfpathlineto{\pgfqpoint{1.969784in}{1.079626in}}%
\pgfpathlineto{\pgfqpoint{1.966114in}{1.072118in}}%
\pgfpathlineto{\pgfqpoint{1.962460in}{1.064968in}}%
\pgfpathlineto{\pgfqpoint{1.958820in}{1.058171in}}%
\pgfpathlineto{\pgfqpoint{1.960817in}{1.068156in}}%
\pgfpathlineto{\pgfqpoint{1.962196in}{1.078153in}}%
\pgfpathlineto{\pgfqpoint{1.962961in}{1.088152in}}%
\pgfpathlineto{\pgfqpoint{1.963111in}{1.098144in}}%
\pgfpathclose%
\pgfusepath{fill}%
\end{pgfscope}%
\begin{pgfscope}%
\pgfpathrectangle{\pgfqpoint{0.329460in}{0.284240in}}{\pgfqpoint{1.989680in}{1.989680in}}%
\pgfusepath{clip}%
\pgfsetbuttcap%
\pgfsetroundjoin%
\definecolor{currentfill}{rgb}{0.147607,0.511733,0.557049}%
\pgfsetfillcolor{currentfill}%
\pgfsetlinewidth{0.000000pt}%
\definecolor{currentstroke}{rgb}{0.000000,0.000000,0.000000}%
\pgfsetstrokecolor{currentstroke}%
\pgfsetdash{}{0pt}%
\pgfpathmoveto{\pgfqpoint{1.054910in}{1.376216in}}%
\pgfpathlineto{\pgfqpoint{1.051668in}{1.369655in}}%
\pgfpathlineto{\pgfqpoint{1.048429in}{1.363098in}}%
\pgfpathlineto{\pgfqpoint{1.045193in}{1.356547in}}%
\pgfpathlineto{\pgfqpoint{1.041960in}{1.350005in}}%
\pgfpathlineto{\pgfqpoint{1.044164in}{1.354833in}}%
\pgfpathlineto{\pgfqpoint{1.046669in}{1.359621in}}%
\pgfpathlineto{\pgfqpoint{1.049472in}{1.364363in}}%
\pgfpathlineto{\pgfqpoint{1.052569in}{1.369056in}}%
\pgfpathlineto{\pgfqpoint{1.055682in}{1.375380in}}%
\pgfpathlineto{\pgfqpoint{1.058798in}{1.381712in}}%
\pgfpathlineto{\pgfqpoint{1.061916in}{1.388051in}}%
\pgfpathlineto{\pgfqpoint{1.065038in}{1.394394in}}%
\pgfpathlineto{\pgfqpoint{1.062079in}{1.389916in}}%
\pgfpathlineto{\pgfqpoint{1.059403in}{1.385391in}}%
\pgfpathlineto{\pgfqpoint{1.057012in}{1.380823in}}%
\pgfpathlineto{\pgfqpoint{1.054910in}{1.376216in}}%
\pgfpathclose%
\pgfusepath{fill}%
\end{pgfscope}%
\begin{pgfscope}%
\pgfpathrectangle{\pgfqpoint{0.329460in}{0.284240in}}{\pgfqpoint{1.989680in}{1.989680in}}%
\pgfusepath{clip}%
\pgfsetbuttcap%
\pgfsetroundjoin%
\definecolor{currentfill}{rgb}{0.279566,0.067836,0.391917}%
\pgfsetfillcolor{currentfill}%
\pgfsetlinewidth{0.000000pt}%
\definecolor{currentstroke}{rgb}{0.000000,0.000000,0.000000}%
\pgfsetstrokecolor{currentstroke}%
\pgfsetdash{}{0pt}%
\pgfpathmoveto{\pgfqpoint{1.808331in}{1.029867in}}%
\pgfpathlineto{\pgfqpoint{1.811626in}{1.026060in}}%
\pgfpathlineto{\pgfqpoint{1.814923in}{1.022403in}}%
\pgfpathlineto{\pgfqpoint{1.818223in}{1.018900in}}%
\pgfpathlineto{\pgfqpoint{1.821526in}{1.015555in}}%
\pgfpathlineto{\pgfqpoint{1.819339in}{1.007795in}}%
\pgfpathlineto{\pgfqpoint{1.816676in}{1.000067in}}%
\pgfpathlineto{\pgfqpoint{1.813538in}{0.992377in}}%
\pgfpathlineto{\pgfqpoint{1.809927in}{0.984734in}}%
\pgfpathlineto{\pgfqpoint{1.806690in}{0.988312in}}%
\pgfpathlineto{\pgfqpoint{1.803456in}{0.992049in}}%
\pgfpathlineto{\pgfqpoint{1.800226in}{0.995939in}}%
\pgfpathlineto{\pgfqpoint{1.796998in}{0.999980in}}%
\pgfpathlineto{\pgfqpoint{1.800523in}{1.007390in}}%
\pgfpathlineto{\pgfqpoint{1.803587in}{1.014847in}}%
\pgfpathlineto{\pgfqpoint{1.806190in}{1.022341in}}%
\pgfpathlineto{\pgfqpoint{1.808331in}{1.029867in}}%
\pgfpathclose%
\pgfusepath{fill}%
\end{pgfscope}%
\begin{pgfscope}%
\pgfpathrectangle{\pgfqpoint{0.329460in}{0.284240in}}{\pgfqpoint{1.989680in}{1.989680in}}%
\pgfusepath{clip}%
\pgfsetbuttcap%
\pgfsetroundjoin%
\definecolor{currentfill}{rgb}{0.201239,0.383670,0.554294}%
\pgfsetfillcolor{currentfill}%
\pgfsetlinewidth{0.000000pt}%
\definecolor{currentstroke}{rgb}{0.000000,0.000000,0.000000}%
\pgfsetstrokecolor{currentstroke}%
\pgfsetdash{}{0pt}%
\pgfpathmoveto{\pgfqpoint{2.001705in}{1.248411in}}%
\pgfpathlineto{\pgfqpoint{2.005488in}{1.260192in}}%
\pgfpathlineto{\pgfqpoint{2.009290in}{1.272398in}}%
\pgfpathlineto{\pgfqpoint{2.013111in}{1.285037in}}%
\pgfpathlineto{\pgfqpoint{2.016952in}{1.298116in}}%
\pgfpathlineto{\pgfqpoint{2.019545in}{1.287562in}}%
\pgfpathlineto{\pgfqpoint{2.021482in}{1.276954in}}%
\pgfpathlineto{\pgfqpoint{2.022757in}{1.266303in}}%
\pgfpathlineto{\pgfqpoint{2.023365in}{1.255618in}}%
\pgfpathlineto{\pgfqpoint{2.019460in}{1.242697in}}%
\pgfpathlineto{\pgfqpoint{2.015576in}{1.230219in}}%
\pgfpathlineto{\pgfqpoint{2.011712in}{1.218176in}}%
\pgfpathlineto{\pgfqpoint{2.007868in}{1.206561in}}%
\pgfpathlineto{\pgfqpoint{2.007300in}{1.217082in}}%
\pgfpathlineto{\pgfqpoint{2.006080in}{1.227570in}}%
\pgfpathlineto{\pgfqpoint{2.004214in}{1.238017in}}%
\pgfpathlineto{\pgfqpoint{2.001705in}{1.248411in}}%
\pgfpathclose%
\pgfusepath{fill}%
\end{pgfscope}%
\begin{pgfscope}%
\pgfpathrectangle{\pgfqpoint{0.329460in}{0.284240in}}{\pgfqpoint{1.989680in}{1.989680in}}%
\pgfusepath{clip}%
\pgfsetbuttcap%
\pgfsetroundjoin%
\definecolor{currentfill}{rgb}{0.166383,0.690856,0.496502}%
\pgfsetfillcolor{currentfill}%
\pgfsetlinewidth{0.000000pt}%
\definecolor{currentstroke}{rgb}{0.000000,0.000000,0.000000}%
\pgfsetstrokecolor{currentstroke}%
\pgfsetdash{}{0pt}%
\pgfpathmoveto{\pgfqpoint{1.171448in}{1.555489in}}%
\pgfpathlineto{\pgfqpoint{1.168899in}{1.550136in}}%
\pgfpathlineto{\pgfqpoint{1.166353in}{1.544741in}}%
\pgfpathlineto{\pgfqpoint{1.163809in}{1.539306in}}%
\pgfpathlineto{\pgfqpoint{1.161269in}{1.533833in}}%
\pgfpathlineto{\pgfqpoint{1.166385in}{1.536682in}}%
\pgfpathlineto{\pgfqpoint{1.171677in}{1.539451in}}%
\pgfpathlineto{\pgfqpoint{1.177142in}{1.542136in}}%
\pgfpathlineto{\pgfqpoint{1.182772in}{1.544736in}}%
\pgfpathlineto{\pgfqpoint{1.185021in}{1.550053in}}%
\pgfpathlineto{\pgfqpoint{1.187272in}{1.555333in}}%
\pgfpathlineto{\pgfqpoint{1.189526in}{1.560573in}}%
\pgfpathlineto{\pgfqpoint{1.191783in}{1.565771in}}%
\pgfpathlineto{\pgfqpoint{1.186458in}{1.563319in}}%
\pgfpathlineto{\pgfqpoint{1.181290in}{1.560787in}}%
\pgfpathlineto{\pgfqpoint{1.176286in}{1.558176in}}%
\pgfpathlineto{\pgfqpoint{1.171448in}{1.555489in}}%
\pgfpathclose%
\pgfusepath{fill}%
\end{pgfscope}%
\begin{pgfscope}%
\pgfpathrectangle{\pgfqpoint{0.329460in}{0.284240in}}{\pgfqpoint{1.989680in}{1.989680in}}%
\pgfusepath{clip}%
\pgfsetbuttcap%
\pgfsetroundjoin%
\definecolor{currentfill}{rgb}{0.268510,0.009605,0.335427}%
\pgfsetfillcolor{currentfill}%
\pgfsetlinewidth{0.000000pt}%
\definecolor{currentstroke}{rgb}{0.000000,0.000000,0.000000}%
\pgfsetstrokecolor{currentstroke}%
\pgfsetdash{}{0pt}%
\pgfpathmoveto{\pgfqpoint{1.888569in}{0.987764in}}%
\pgfpathlineto{\pgfqpoint{1.891987in}{0.988647in}}%
\pgfpathlineto{\pgfqpoint{1.895413in}{0.989778in}}%
\pgfpathlineto{\pgfqpoint{1.898848in}{0.991163in}}%
\pgfpathlineto{\pgfqpoint{1.902292in}{0.992805in}}%
\pgfpathlineto{\pgfqpoint{1.899842in}{0.983679in}}%
\pgfpathlineto{\pgfqpoint{1.896832in}{0.974586in}}%
\pgfpathlineto{\pgfqpoint{1.893264in}{0.965535in}}%
\pgfpathlineto{\pgfqpoint{1.889139in}{0.956535in}}%
\pgfpathlineto{\pgfqpoint{1.885759in}{0.955112in}}%
\pgfpathlineto{\pgfqpoint{1.882388in}{0.953947in}}%
\pgfpathlineto{\pgfqpoint{1.879025in}{0.953036in}}%
\pgfpathlineto{\pgfqpoint{1.875671in}{0.952375in}}%
\pgfpathlineto{\pgfqpoint{1.879711in}{0.961155in}}%
\pgfpathlineto{\pgfqpoint{1.883208in}{0.969986in}}%
\pgfpathlineto{\pgfqpoint{1.886161in}{0.978859in}}%
\pgfpathlineto{\pgfqpoint{1.888569in}{0.987764in}}%
\pgfpathclose%
\pgfusepath{fill}%
\end{pgfscope}%
\begin{pgfscope}%
\pgfpathrectangle{\pgfqpoint{0.329460in}{0.284240in}}{\pgfqpoint{1.989680in}{1.989680in}}%
\pgfusepath{clip}%
\pgfsetbuttcap%
\pgfsetroundjoin%
\definecolor{currentfill}{rgb}{0.120081,0.622161,0.534946}%
\pgfsetfillcolor{currentfill}%
\pgfsetlinewidth{0.000000pt}%
\definecolor{currentstroke}{rgb}{0.000000,0.000000,0.000000}%
\pgfsetstrokecolor{currentstroke}%
\pgfsetdash{}{0pt}%
\pgfpathmoveto{\pgfqpoint{1.115302in}{1.484797in}}%
\pgfpathlineto{\pgfqpoint{1.112311in}{1.478803in}}%
\pgfpathlineto{\pgfqpoint{1.109324in}{1.472785in}}%
\pgfpathlineto{\pgfqpoint{1.106339in}{1.466744in}}%
\pgfpathlineto{\pgfqpoint{1.103358in}{1.460683in}}%
\pgfpathlineto{\pgfqpoint{1.107278in}{1.464467in}}%
\pgfpathlineto{\pgfqpoint{1.111434in}{1.468186in}}%
\pgfpathlineto{\pgfqpoint{1.115821in}{1.471838in}}%
\pgfpathlineto{\pgfqpoint{1.120435in}{1.475418in}}%
\pgfpathlineto{\pgfqpoint{1.123204in}{1.481288in}}%
\pgfpathlineto{\pgfqpoint{1.125976in}{1.487137in}}%
\pgfpathlineto{\pgfqpoint{1.128751in}{1.492965in}}%
\pgfpathlineto{\pgfqpoint{1.131530in}{1.498767in}}%
\pgfpathlineto{\pgfqpoint{1.127145in}{1.495372in}}%
\pgfpathlineto{\pgfqpoint{1.122975in}{1.491910in}}%
\pgfpathlineto{\pgfqpoint{1.119026in}{1.488384in}}%
\pgfpathlineto{\pgfqpoint{1.115302in}{1.484797in}}%
\pgfpathclose%
\pgfusepath{fill}%
\end{pgfscope}%
\begin{pgfscope}%
\pgfpathrectangle{\pgfqpoint{0.329460in}{0.284240in}}{\pgfqpoint{1.989680in}{1.989680in}}%
\pgfusepath{clip}%
\pgfsetbuttcap%
\pgfsetroundjoin%
\definecolor{currentfill}{rgb}{0.122606,0.585371,0.546557}%
\pgfsetfillcolor{currentfill}%
\pgfsetlinewidth{0.000000pt}%
\definecolor{currentstroke}{rgb}{0.000000,0.000000,0.000000}%
\pgfsetstrokecolor{currentstroke}%
\pgfsetdash{}{0pt}%
\pgfpathmoveto{\pgfqpoint{1.595544in}{1.464050in}}%
\pgfpathlineto{\pgfqpoint{1.598479in}{1.458015in}}%
\pgfpathlineto{\pgfqpoint{1.601411in}{1.451964in}}%
\pgfpathlineto{\pgfqpoint{1.604339in}{1.445900in}}%
\pgfpathlineto{\pgfqpoint{1.607265in}{1.439826in}}%
\pgfpathlineto{\pgfqpoint{1.611353in}{1.435837in}}%
\pgfpathlineto{\pgfqpoint{1.615190in}{1.431785in}}%
\pgfpathlineto{\pgfqpoint{1.618770in}{1.427672in}}%
\pgfpathlineto{\pgfqpoint{1.622091in}{1.423503in}}%
\pgfpathlineto{\pgfqpoint{1.618988in}{1.429780in}}%
\pgfpathlineto{\pgfqpoint{1.615882in}{1.436047in}}%
\pgfpathlineto{\pgfqpoint{1.612773in}{1.442300in}}%
\pgfpathlineto{\pgfqpoint{1.609661in}{1.448538in}}%
\pgfpathlineto{\pgfqpoint{1.606500in}{1.452499in}}%
\pgfpathlineto{\pgfqpoint{1.603091in}{1.456407in}}%
\pgfpathlineto{\pgfqpoint{1.599438in}{1.460259in}}%
\pgfpathlineto{\pgfqpoint{1.595544in}{1.464050in}}%
\pgfpathclose%
\pgfusepath{fill}%
\end{pgfscope}%
\begin{pgfscope}%
\pgfpathrectangle{\pgfqpoint{0.329460in}{0.284240in}}{\pgfqpoint{1.989680in}{1.989680in}}%
\pgfusepath{clip}%
\pgfsetbuttcap%
\pgfsetroundjoin%
\definecolor{currentfill}{rgb}{0.281477,0.755203,0.432552}%
\pgfsetfillcolor{currentfill}%
\pgfsetlinewidth{0.000000pt}%
\definecolor{currentstroke}{rgb}{0.000000,0.000000,0.000000}%
\pgfsetstrokecolor{currentstroke}%
\pgfsetdash{}{0pt}%
\pgfpathmoveto{\pgfqpoint{1.421359in}{1.626322in}}%
\pgfpathlineto{\pgfqpoint{1.422487in}{1.621881in}}%
\pgfpathlineto{\pgfqpoint{1.423614in}{1.617381in}}%
\pgfpathlineto{\pgfqpoint{1.424740in}{1.612827in}}%
\pgfpathlineto{\pgfqpoint{1.425864in}{1.608219in}}%
\pgfpathlineto{\pgfqpoint{1.432320in}{1.607048in}}%
\pgfpathlineto{\pgfqpoint{1.438698in}{1.605782in}}%
\pgfpathlineto{\pgfqpoint{1.444994in}{1.604419in}}%
\pgfpathlineto{\pgfqpoint{1.451201in}{1.602963in}}%
\pgfpathlineto{\pgfqpoint{1.449695in}{1.607654in}}%
\pgfpathlineto{\pgfqpoint{1.448186in}{1.612292in}}%
\pgfpathlineto{\pgfqpoint{1.446675in}{1.616874in}}%
\pgfpathlineto{\pgfqpoint{1.445162in}{1.621399in}}%
\pgfpathlineto{\pgfqpoint{1.439332in}{1.622763in}}%
\pgfpathlineto{\pgfqpoint{1.433417in}{1.624039in}}%
\pgfpathlineto{\pgfqpoint{1.427424in}{1.625226in}}%
\pgfpathlineto{\pgfqpoint{1.421359in}{1.626322in}}%
\pgfpathclose%
\pgfusepath{fill}%
\end{pgfscope}%
\begin{pgfscope}%
\pgfpathrectangle{\pgfqpoint{0.329460in}{0.284240in}}{\pgfqpoint{1.989680in}{1.989680in}}%
\pgfusepath{clip}%
\pgfsetbuttcap%
\pgfsetroundjoin%
\definecolor{currentfill}{rgb}{0.220124,0.725509,0.466226}%
\pgfsetfillcolor{currentfill}%
\pgfsetlinewidth{0.000000pt}%
\definecolor{currentstroke}{rgb}{0.000000,0.000000,0.000000}%
\pgfsetstrokecolor{currentstroke}%
\pgfsetdash{}{0pt}%
\pgfpathmoveto{\pgfqpoint{1.475020in}{1.596221in}}%
\pgfpathlineto{\pgfqpoint{1.476885in}{1.591371in}}%
\pgfpathlineto{\pgfqpoint{1.478748in}{1.586471in}}%
\pgfpathlineto{\pgfqpoint{1.480608in}{1.581523in}}%
\pgfpathlineto{\pgfqpoint{1.482466in}{1.576530in}}%
\pgfpathlineto{\pgfqpoint{1.488484in}{1.574502in}}%
\pgfpathlineto{\pgfqpoint{1.494371in}{1.572383in}}%
\pgfpathlineto{\pgfqpoint{1.500121in}{1.570176in}}%
\pgfpathlineto{\pgfqpoint{1.505730in}{1.567882in}}%
\pgfpathlineto{\pgfqpoint{1.503540in}{1.573005in}}%
\pgfpathlineto{\pgfqpoint{1.501347in}{1.578082in}}%
\pgfpathlineto{\pgfqpoint{1.499151in}{1.583111in}}%
\pgfpathlineto{\pgfqpoint{1.496952in}{1.588090in}}%
\pgfpathlineto{\pgfqpoint{1.491665in}{1.590247in}}%
\pgfpathlineto{\pgfqpoint{1.486244in}{1.592322in}}%
\pgfpathlineto{\pgfqpoint{1.480694in}{1.594314in}}%
\pgfpathlineto{\pgfqpoint{1.475020in}{1.596221in}}%
\pgfpathclose%
\pgfusepath{fill}%
\end{pgfscope}%
\begin{pgfscope}%
\pgfpathrectangle{\pgfqpoint{0.329460in}{0.284240in}}{\pgfqpoint{1.989680in}{1.989680in}}%
\pgfusepath{clip}%
\pgfsetbuttcap%
\pgfsetroundjoin%
\definecolor{currentfill}{rgb}{0.268510,0.009605,0.335427}%
\pgfsetfillcolor{currentfill}%
\pgfsetlinewidth{0.000000pt}%
\definecolor{currentstroke}{rgb}{0.000000,0.000000,0.000000}%
\pgfsetstrokecolor{currentstroke}%
\pgfsetdash{}{0pt}%
\pgfpathmoveto{\pgfqpoint{0.870192in}{0.955095in}}%
\pgfpathlineto{\pgfqpoint{0.866939in}{0.953060in}}%
\pgfpathlineto{\pgfqpoint{0.863680in}{0.951223in}}%
\pgfpathlineto{\pgfqpoint{0.860416in}{0.949587in}}%
\pgfpathlineto{\pgfqpoint{0.857146in}{0.948156in}}%
\pgfpathlineto{\pgfqpoint{0.852820in}{0.956437in}}%
\pgfpathlineto{\pgfqpoint{0.849008in}{0.964774in}}%
\pgfpathlineto{\pgfqpoint{0.845710in}{0.973161in}}%
\pgfpathlineto{\pgfqpoint{0.842929in}{0.981587in}}%
\pgfpathlineto{\pgfqpoint{0.846276in}{0.982790in}}%
\pgfpathlineto{\pgfqpoint{0.849618in}{0.984198in}}%
\pgfpathlineto{\pgfqpoint{0.852954in}{0.985807in}}%
\pgfpathlineto{\pgfqpoint{0.856285in}{0.987612in}}%
\pgfpathlineto{\pgfqpoint{0.859009in}{0.979415in}}%
\pgfpathlineto{\pgfqpoint{0.862237in}{0.971258in}}%
\pgfpathlineto{\pgfqpoint{0.865965in}{0.963148in}}%
\pgfpathlineto{\pgfqpoint{0.870192in}{0.955095in}}%
\pgfpathclose%
\pgfusepath{fill}%
\end{pgfscope}%
\begin{pgfscope}%
\pgfpathrectangle{\pgfqpoint{0.329460in}{0.284240in}}{\pgfqpoint{1.989680in}{1.989680in}}%
\pgfusepath{clip}%
\pgfsetbuttcap%
\pgfsetroundjoin%
\definecolor{currentfill}{rgb}{0.281477,0.755203,0.432552}%
\pgfsetfillcolor{currentfill}%
\pgfsetlinewidth{0.000000pt}%
\definecolor{currentstroke}{rgb}{0.000000,0.000000,0.000000}%
\pgfsetstrokecolor{currentstroke}%
\pgfsetdash{}{0pt}%
\pgfpathmoveto{\pgfqpoint{1.252105in}{1.620113in}}%
\pgfpathlineto{\pgfqpoint{1.250510in}{1.615566in}}%
\pgfpathlineto{\pgfqpoint{1.248916in}{1.610962in}}%
\pgfpathlineto{\pgfqpoint{1.247325in}{1.606303in}}%
\pgfpathlineto{\pgfqpoint{1.245736in}{1.601590in}}%
\pgfpathlineto{\pgfqpoint{1.251859in}{1.603129in}}%
\pgfpathlineto{\pgfqpoint{1.258076in}{1.604575in}}%
\pgfpathlineto{\pgfqpoint{1.264381in}{1.605927in}}%
\pgfpathlineto{\pgfqpoint{1.270769in}{1.607183in}}%
\pgfpathlineto{\pgfqpoint{1.271980in}{1.611808in}}%
\pgfpathlineto{\pgfqpoint{1.273192in}{1.616379in}}%
\pgfpathlineto{\pgfqpoint{1.274406in}{1.620894in}}%
\pgfpathlineto{\pgfqpoint{1.275622in}{1.625352in}}%
\pgfpathlineto{\pgfqpoint{1.269620in}{1.624176in}}%
\pgfpathlineto{\pgfqpoint{1.263697in}{1.622909in}}%
\pgfpathlineto{\pgfqpoint{1.257856in}{1.621555in}}%
\pgfpathlineto{\pgfqpoint{1.252105in}{1.620113in}}%
\pgfpathclose%
\pgfusepath{fill}%
\end{pgfscope}%
\begin{pgfscope}%
\pgfpathrectangle{\pgfqpoint{0.329460in}{0.284240in}}{\pgfqpoint{1.989680in}{1.989680in}}%
\pgfusepath{clip}%
\pgfsetbuttcap%
\pgfsetroundjoin%
\definecolor{currentfill}{rgb}{0.212395,0.359683,0.551710}%
\pgfsetfillcolor{currentfill}%
\pgfsetlinewidth{0.000000pt}%
\definecolor{currentstroke}{rgb}{0.000000,0.000000,0.000000}%
\pgfsetstrokecolor{currentstroke}%
\pgfsetdash{}{0pt}%
\pgfpathmoveto{\pgfqpoint{0.995718in}{1.228085in}}%
\pgfpathlineto{\pgfqpoint{0.992420in}{1.221435in}}%
\pgfpathlineto{\pgfqpoint{0.989123in}{1.214831in}}%
\pgfpathlineto{\pgfqpoint{0.985827in}{1.208276in}}%
\pgfpathlineto{\pgfqpoint{0.982532in}{1.201774in}}%
\pgfpathlineto{\pgfqpoint{0.982255in}{1.207691in}}%
\pgfpathlineto{\pgfqpoint{0.982345in}{1.213605in}}%
\pgfpathlineto{\pgfqpoint{0.982803in}{1.219508in}}%
\pgfpathlineto{\pgfqpoint{0.983626in}{1.225396in}}%
\pgfpathlineto{\pgfqpoint{0.986899in}{1.231664in}}%
\pgfpathlineto{\pgfqpoint{0.990173in}{1.237985in}}%
\pgfpathlineto{\pgfqpoint{0.993449in}{1.244355in}}%
\pgfpathlineto{\pgfqpoint{0.996726in}{1.250771in}}%
\pgfpathlineto{\pgfqpoint{0.995944in}{1.245116in}}%
\pgfpathlineto{\pgfqpoint{0.995514in}{1.239446in}}%
\pgfpathlineto{\pgfqpoint{0.995439in}{1.233768in}}%
\pgfpathlineto{\pgfqpoint{0.995718in}{1.228085in}}%
\pgfpathclose%
\pgfusepath{fill}%
\end{pgfscope}%
\begin{pgfscope}%
\pgfpathrectangle{\pgfqpoint{0.329460in}{0.284240in}}{\pgfqpoint{1.989680in}{1.989680in}}%
\pgfusepath{clip}%
\pgfsetbuttcap%
\pgfsetroundjoin%
\definecolor{currentfill}{rgb}{0.271305,0.019942,0.347269}%
\pgfsetfillcolor{currentfill}%
\pgfsetlinewidth{0.000000pt}%
\definecolor{currentstroke}{rgb}{0.000000,0.000000,0.000000}%
\pgfsetstrokecolor{currentstroke}%
\pgfsetdash{}{0pt}%
\pgfpathmoveto{\pgfqpoint{0.883157in}{0.965123in}}%
\pgfpathlineto{\pgfqpoint{0.879923in}{0.962340in}}%
\pgfpathlineto{\pgfqpoint{0.876684in}{0.959739in}}%
\pgfpathlineto{\pgfqpoint{0.873441in}{0.957322in}}%
\pgfpathlineto{\pgfqpoint{0.870192in}{0.955095in}}%
\pgfpathlineto{\pgfqpoint{0.865965in}{0.963148in}}%
\pgfpathlineto{\pgfqpoint{0.862237in}{0.971258in}}%
\pgfpathlineto{\pgfqpoint{0.859009in}{0.979415in}}%
\pgfpathlineto{\pgfqpoint{0.856285in}{0.987612in}}%
\pgfpathlineto{\pgfqpoint{0.859611in}{0.989610in}}%
\pgfpathlineto{\pgfqpoint{0.862932in}{0.991797in}}%
\pgfpathlineto{\pgfqpoint{0.866248in}{0.994168in}}%
\pgfpathlineto{\pgfqpoint{0.869560in}{0.996720in}}%
\pgfpathlineto{\pgfqpoint{0.872228in}{0.988754in}}%
\pgfpathlineto{\pgfqpoint{0.875385in}{0.980828in}}%
\pgfpathlineto{\pgfqpoint{0.879028in}{0.972948in}}%
\pgfpathlineto{\pgfqpoint{0.883157in}{0.965123in}}%
\pgfpathclose%
\pgfusepath{fill}%
\end{pgfscope}%
\begin{pgfscope}%
\pgfpathrectangle{\pgfqpoint{0.329460in}{0.284240in}}{\pgfqpoint{1.989680in}{1.989680in}}%
\pgfusepath{clip}%
\pgfsetbuttcap%
\pgfsetroundjoin%
\definecolor{currentfill}{rgb}{0.267004,0.004874,0.329415}%
\pgfsetfillcolor{currentfill}%
\pgfsetlinewidth{0.000000pt}%
\definecolor{currentstroke}{rgb}{0.000000,0.000000,0.000000}%
\pgfsetstrokecolor{currentstroke}%
\pgfsetdash{}{0pt}%
\pgfpathmoveto{\pgfqpoint{0.857146in}{0.948156in}}%
\pgfpathlineto{\pgfqpoint{0.853870in}{0.946935in}}%
\pgfpathlineto{\pgfqpoint{0.850587in}{0.945928in}}%
\pgfpathlineto{\pgfqpoint{0.847299in}{0.945139in}}%
\pgfpathlineto{\pgfqpoint{0.844003in}{0.944573in}}%
\pgfpathlineto{\pgfqpoint{0.839580in}{0.953078in}}%
\pgfpathlineto{\pgfqpoint{0.835683in}{0.961641in}}%
\pgfpathlineto{\pgfqpoint{0.832316in}{0.970254in}}%
\pgfpathlineto{\pgfqpoint{0.829479in}{0.978907in}}%
\pgfpathlineto{\pgfqpoint{0.832851in}{0.979248in}}%
\pgfpathlineto{\pgfqpoint{0.836217in}{0.979812in}}%
\pgfpathlineto{\pgfqpoint{0.839576in}{0.980593in}}%
\pgfpathlineto{\pgfqpoint{0.842929in}{0.981587in}}%
\pgfpathlineto{\pgfqpoint{0.845710in}{0.973161in}}%
\pgfpathlineto{\pgfqpoint{0.849008in}{0.964774in}}%
\pgfpathlineto{\pgfqpoint{0.852820in}{0.956437in}}%
\pgfpathlineto{\pgfqpoint{0.857146in}{0.948156in}}%
\pgfpathclose%
\pgfusepath{fill}%
\end{pgfscope}%
\begin{pgfscope}%
\pgfpathrectangle{\pgfqpoint{0.329460in}{0.284240in}}{\pgfqpoint{1.989680in}{1.989680in}}%
\pgfusepath{clip}%
\pgfsetbuttcap%
\pgfsetroundjoin%
\definecolor{currentfill}{rgb}{0.282327,0.094955,0.417331}%
\pgfsetfillcolor{currentfill}%
\pgfsetlinewidth{0.000000pt}%
\definecolor{currentstroke}{rgb}{0.000000,0.000000,0.000000}%
\pgfsetstrokecolor{currentstroke}%
\pgfsetdash{}{0pt}%
\pgfpathmoveto{\pgfqpoint{1.795179in}{1.046524in}}%
\pgfpathlineto{\pgfqpoint{1.798463in}{1.042153in}}%
\pgfpathlineto{\pgfqpoint{1.801750in}{1.037917in}}%
\pgfpathlineto{\pgfqpoint{1.805040in}{1.033820in}}%
\pgfpathlineto{\pgfqpoint{1.808331in}{1.029867in}}%
\pgfpathlineto{\pgfqpoint{1.806190in}{1.022341in}}%
\pgfpathlineto{\pgfqpoint{1.803587in}{1.014847in}}%
\pgfpathlineto{\pgfqpoint{1.800523in}{1.007390in}}%
\pgfpathlineto{\pgfqpoint{1.796998in}{0.999980in}}%
\pgfpathlineto{\pgfqpoint{1.793773in}{1.004168in}}%
\pgfpathlineto{\pgfqpoint{1.790551in}{1.008499in}}%
\pgfpathlineto{\pgfqpoint{1.787332in}{1.012970in}}%
\pgfpathlineto{\pgfqpoint{1.784114in}{1.017576in}}%
\pgfpathlineto{\pgfqpoint{1.787551in}{1.024753in}}%
\pgfpathlineto{\pgfqpoint{1.790542in}{1.031975in}}%
\pgfpathlineto{\pgfqpoint{1.793085in}{1.039235in}}%
\pgfpathlineto{\pgfqpoint{1.795179in}{1.046524in}}%
\pgfpathclose%
\pgfusepath{fill}%
\end{pgfscope}%
\begin{pgfscope}%
\pgfpathrectangle{\pgfqpoint{0.329460in}{0.284240in}}{\pgfqpoint{1.989680in}{1.989680in}}%
\pgfusepath{clip}%
\pgfsetbuttcap%
\pgfsetroundjoin%
\definecolor{currentfill}{rgb}{0.220124,0.725509,0.466226}%
\pgfsetfillcolor{currentfill}%
\pgfsetlinewidth{0.000000pt}%
\definecolor{currentstroke}{rgb}{0.000000,0.000000,0.000000}%
\pgfsetstrokecolor{currentstroke}%
\pgfsetdash{}{0pt}%
\pgfpathmoveto{\pgfqpoint{1.200840in}{1.586107in}}%
\pgfpathlineto{\pgfqpoint{1.198572in}{1.581096in}}%
\pgfpathlineto{\pgfqpoint{1.196306in}{1.576035in}}%
\pgfpathlineto{\pgfqpoint{1.194043in}{1.570926in}}%
\pgfpathlineto{\pgfqpoint{1.191783in}{1.565771in}}%
\pgfpathlineto{\pgfqpoint{1.197261in}{1.568141in}}%
\pgfpathlineto{\pgfqpoint{1.202886in}{1.570425in}}%
\pgfpathlineto{\pgfqpoint{1.208652in}{1.572623in}}%
\pgfpathlineto{\pgfqpoint{1.214554in}{1.574732in}}%
\pgfpathlineto{\pgfqpoint{1.216488in}{1.579752in}}%
\pgfpathlineto{\pgfqpoint{1.218425in}{1.584726in}}%
\pgfpathlineto{\pgfqpoint{1.220364in}{1.589653in}}%
\pgfpathlineto{\pgfqpoint{1.222306in}{1.594530in}}%
\pgfpathlineto{\pgfqpoint{1.216741in}{1.592547in}}%
\pgfpathlineto{\pgfqpoint{1.211306in}{1.590481in}}%
\pgfpathlineto{\pgfqpoint{1.206003in}{1.588334in}}%
\pgfpathlineto{\pgfqpoint{1.200840in}{1.586107in}}%
\pgfpathclose%
\pgfusepath{fill}%
\end{pgfscope}%
\begin{pgfscope}%
\pgfpathrectangle{\pgfqpoint{0.329460in}{0.284240in}}{\pgfqpoint{1.989680in}{1.989680in}}%
\pgfusepath{clip}%
\pgfsetbuttcap%
\pgfsetroundjoin%
\definecolor{currentfill}{rgb}{0.263663,0.237631,0.518762}%
\pgfsetfillcolor{currentfill}%
\pgfsetlinewidth{0.000000pt}%
\definecolor{currentstroke}{rgb}{0.000000,0.000000,0.000000}%
\pgfsetstrokecolor{currentstroke}%
\pgfsetdash{}{0pt}%
\pgfpathmoveto{\pgfqpoint{0.961251in}{1.126506in}}%
\pgfpathlineto{\pgfqpoint{0.957992in}{1.120371in}}%
\pgfpathlineto{\pgfqpoint{0.954733in}{1.114318in}}%
\pgfpathlineto{\pgfqpoint{0.951474in}{1.108351in}}%
\pgfpathlineto{\pgfqpoint{0.948214in}{1.102472in}}%
\pgfpathlineto{\pgfqpoint{0.946311in}{1.109055in}}%
\pgfpathlineto{\pgfqpoint{0.944817in}{1.115659in}}%
\pgfpathlineto{\pgfqpoint{0.943732in}{1.122277in}}%
\pgfpathlineto{\pgfqpoint{0.943057in}{1.128902in}}%
\pgfpathlineto{\pgfqpoint{0.946345in}{1.134544in}}%
\pgfpathlineto{\pgfqpoint{0.949632in}{1.140274in}}%
\pgfpathlineto{\pgfqpoint{0.952920in}{1.146090in}}%
\pgfpathlineto{\pgfqpoint{0.956208in}{1.151989in}}%
\pgfpathlineto{\pgfqpoint{0.956875in}{1.145600in}}%
\pgfpathlineto{\pgfqpoint{0.957937in}{1.139219in}}%
\pgfpathlineto{\pgfqpoint{0.959396in}{1.132852in}}%
\pgfpathlineto{\pgfqpoint{0.961251in}{1.126506in}}%
\pgfpathclose%
\pgfusepath{fill}%
\end{pgfscope}%
\begin{pgfscope}%
\pgfpathrectangle{\pgfqpoint{0.329460in}{0.284240in}}{\pgfqpoint{1.989680in}{1.989680in}}%
\pgfusepath{clip}%
\pgfsetbuttcap%
\pgfsetroundjoin%
\definecolor{currentfill}{rgb}{0.274952,0.037752,0.364543}%
\pgfsetfillcolor{currentfill}%
\pgfsetlinewidth{0.000000pt}%
\definecolor{currentstroke}{rgb}{0.000000,0.000000,0.000000}%
\pgfsetstrokecolor{currentstroke}%
\pgfsetdash{}{0pt}%
\pgfpathmoveto{\pgfqpoint{0.896054in}{0.977986in}}%
\pgfpathlineto{\pgfqpoint{0.892836in}{0.974518in}}%
\pgfpathlineto{\pgfqpoint{0.889614in}{0.971216in}}%
\pgfpathlineto{\pgfqpoint{0.886388in}{0.968083in}}%
\pgfpathlineto{\pgfqpoint{0.883157in}{0.965123in}}%
\pgfpathlineto{\pgfqpoint{0.879028in}{0.972948in}}%
\pgfpathlineto{\pgfqpoint{0.875385in}{0.980828in}}%
\pgfpathlineto{\pgfqpoint{0.872228in}{0.988754in}}%
\pgfpathlineto{\pgfqpoint{0.869560in}{0.996720in}}%
\pgfpathlineto{\pgfqpoint{0.872868in}{0.999448in}}%
\pgfpathlineto{\pgfqpoint{0.876172in}{1.002350in}}%
\pgfpathlineto{\pgfqpoint{0.879473in}{1.005420in}}%
\pgfpathlineto{\pgfqpoint{0.882769in}{1.008656in}}%
\pgfpathlineto{\pgfqpoint{0.885379in}{1.000924in}}%
\pgfpathlineto{\pgfqpoint{0.888465in}{0.993229in}}%
\pgfpathlineto{\pgfqpoint{0.892024in}{0.985581in}}%
\pgfpathlineto{\pgfqpoint{0.896054in}{0.977986in}}%
\pgfpathclose%
\pgfusepath{fill}%
\end{pgfscope}%
\begin{pgfscope}%
\pgfpathrectangle{\pgfqpoint{0.329460in}{0.284240in}}{\pgfqpoint{1.989680in}{1.989680in}}%
\pgfusepath{clip}%
\pgfsetbuttcap%
\pgfsetroundjoin%
\definecolor{currentfill}{rgb}{0.122606,0.585371,0.546557}%
\pgfsetfillcolor{currentfill}%
\pgfsetlinewidth{0.000000pt}%
\definecolor{currentstroke}{rgb}{0.000000,0.000000,0.000000}%
\pgfsetstrokecolor{currentstroke}%
\pgfsetdash{}{0pt}%
\pgfpathmoveto{\pgfqpoint{1.090116in}{1.444975in}}%
\pgfpathlineto{\pgfqpoint{1.086971in}{1.438691in}}%
\pgfpathlineto{\pgfqpoint{1.083828in}{1.432391in}}%
\pgfpathlineto{\pgfqpoint{1.080689in}{1.426077in}}%
\pgfpathlineto{\pgfqpoint{1.077553in}{1.419753in}}%
\pgfpathlineto{\pgfqpoint{1.080640in}{1.423969in}}%
\pgfpathlineto{\pgfqpoint{1.083990in}{1.428132in}}%
\pgfpathlineto{\pgfqpoint{1.087599in}{1.432238in}}%
\pgfpathlineto{\pgfqpoint{1.091464in}{1.436284in}}%
\pgfpathlineto{\pgfqpoint{1.094433in}{1.442402in}}%
\pgfpathlineto{\pgfqpoint{1.097405in}{1.448510in}}%
\pgfpathlineto{\pgfqpoint{1.100380in}{1.454604in}}%
\pgfpathlineto{\pgfqpoint{1.103358in}{1.460683in}}%
\pgfpathlineto{\pgfqpoint{1.099678in}{1.456838in}}%
\pgfpathlineto{\pgfqpoint{1.096242in}{1.452936in}}%
\pgfpathlineto{\pgfqpoint{1.093053in}{1.448980in}}%
\pgfpathlineto{\pgfqpoint{1.090116in}{1.444975in}}%
\pgfpathclose%
\pgfusepath{fill}%
\end{pgfscope}%
\begin{pgfscope}%
\pgfpathrectangle{\pgfqpoint{0.329460in}{0.284240in}}{\pgfqpoint{1.989680in}{1.989680in}}%
\pgfusepath{clip}%
\pgfsetbuttcap%
\pgfsetroundjoin%
\definecolor{currentfill}{rgb}{0.163625,0.471133,0.558148}%
\pgfsetfillcolor{currentfill}%
\pgfsetlinewidth{0.000000pt}%
\definecolor{currentstroke}{rgb}{0.000000,0.000000,0.000000}%
\pgfsetstrokecolor{currentstroke}%
\pgfsetdash{}{0pt}%
\pgfpathmoveto{\pgfqpoint{1.658471in}{1.354298in}}%
\pgfpathlineto{\pgfqpoint{1.661679in}{1.347817in}}%
\pgfpathlineto{\pgfqpoint{1.664885in}{1.341349in}}%
\pgfpathlineto{\pgfqpoint{1.668089in}{1.334897in}}%
\pgfpathlineto{\pgfqpoint{1.671289in}{1.328465in}}%
\pgfpathlineto{\pgfqpoint{1.673561in}{1.323410in}}%
\pgfpathlineto{\pgfqpoint{1.675517in}{1.318317in}}%
\pgfpathlineto{\pgfqpoint{1.677153in}{1.313192in}}%
\pgfpathlineto{\pgfqpoint{1.678468in}{1.308039in}}%
\pgfpathlineto{\pgfqpoint{1.675185in}{1.314697in}}%
\pgfpathlineto{\pgfqpoint{1.671900in}{1.321374in}}%
\pgfpathlineto{\pgfqpoint{1.668612in}{1.328067in}}%
\pgfpathlineto{\pgfqpoint{1.665321in}{1.334774in}}%
\pgfpathlineto{\pgfqpoint{1.664069in}{1.339699in}}%
\pgfpathlineto{\pgfqpoint{1.662509in}{1.344598in}}%
\pgfpathlineto{\pgfqpoint{1.660642in}{1.349466in}}%
\pgfpathlineto{\pgfqpoint{1.658471in}{1.354298in}}%
\pgfpathclose%
\pgfusepath{fill}%
\end{pgfscope}%
\begin{pgfscope}%
\pgfpathrectangle{\pgfqpoint{0.329460in}{0.284240in}}{\pgfqpoint{1.989680in}{1.989680in}}%
\pgfusepath{clip}%
\pgfsetbuttcap%
\pgfsetroundjoin%
\definecolor{currentfill}{rgb}{0.267004,0.004874,0.329415}%
\pgfsetfillcolor{currentfill}%
\pgfsetlinewidth{0.000000pt}%
\definecolor{currentstroke}{rgb}{0.000000,0.000000,0.000000}%
\pgfsetstrokecolor{currentstroke}%
\pgfsetdash{}{0pt}%
\pgfpathmoveto{\pgfqpoint{0.844003in}{0.944573in}}%
\pgfpathlineto{\pgfqpoint{0.840701in}{0.944233in}}%
\pgfpathlineto{\pgfqpoint{0.837391in}{0.944125in}}%
\pgfpathlineto{\pgfqpoint{0.834075in}{0.944253in}}%
\pgfpathlineto{\pgfqpoint{0.830750in}{0.944621in}}%
\pgfpathlineto{\pgfqpoint{0.826229in}{0.953348in}}%
\pgfpathlineto{\pgfqpoint{0.822249in}{0.962134in}}%
\pgfpathlineto{\pgfqpoint{0.818812in}{0.970970in}}%
\pgfpathlineto{\pgfqpoint{0.815920in}{0.979847in}}%
\pgfpathlineto{\pgfqpoint{0.819321in}{0.979257in}}%
\pgfpathlineto{\pgfqpoint{0.822714in}{0.978907in}}%
\pgfpathlineto{\pgfqpoint{0.826100in}{0.978792in}}%
\pgfpathlineto{\pgfqpoint{0.829479in}{0.978907in}}%
\pgfpathlineto{\pgfqpoint{0.832316in}{0.970254in}}%
\pgfpathlineto{\pgfqpoint{0.835683in}{0.961641in}}%
\pgfpathlineto{\pgfqpoint{0.839580in}{0.953078in}}%
\pgfpathlineto{\pgfqpoint{0.844003in}{0.944573in}}%
\pgfpathclose%
\pgfusepath{fill}%
\end{pgfscope}%
\begin{pgfscope}%
\pgfpathrectangle{\pgfqpoint{0.329460in}{0.284240in}}{\pgfqpoint{1.989680in}{1.989680in}}%
\pgfusepath{clip}%
\pgfsetbuttcap%
\pgfsetroundjoin%
\definecolor{currentfill}{rgb}{0.272594,0.025563,0.353093}%
\pgfsetfillcolor{currentfill}%
\pgfsetlinewidth{0.000000pt}%
\definecolor{currentstroke}{rgb}{0.000000,0.000000,0.000000}%
\pgfsetstrokecolor{currentstroke}%
\pgfsetdash{}{0pt}%
\pgfpathmoveto{\pgfqpoint{1.902292in}{0.992805in}}%
\pgfpathlineto{\pgfqpoint{1.905744in}{0.994710in}}%
\pgfpathlineto{\pgfqpoint{1.909206in}{0.996881in}}%
\pgfpathlineto{\pgfqpoint{1.912677in}{0.999325in}}%
\pgfpathlineto{\pgfqpoint{1.916158in}{1.002046in}}%
\pgfpathlineto{\pgfqpoint{1.913666in}{0.992704in}}%
\pgfpathlineto{\pgfqpoint{1.910601in}{0.983394in}}%
\pgfpathlineto{\pgfqpoint{1.906962in}{0.974126in}}%
\pgfpathlineto{\pgfqpoint{1.902752in}{0.964911in}}%
\pgfpathlineto{\pgfqpoint{1.899334in}{0.962405in}}%
\pgfpathlineto{\pgfqpoint{1.895926in}{0.960177in}}%
\pgfpathlineto{\pgfqpoint{1.892528in}{0.958222in}}%
\pgfpathlineto{\pgfqpoint{1.889139in}{0.956535in}}%
\pgfpathlineto{\pgfqpoint{1.893264in}{0.965535in}}%
\pgfpathlineto{\pgfqpoint{1.896832in}{0.974586in}}%
\pgfpathlineto{\pgfqpoint{1.899842in}{0.983679in}}%
\pgfpathlineto{\pgfqpoint{1.902292in}{0.992805in}}%
\pgfpathclose%
\pgfusepath{fill}%
\end{pgfscope}%
\begin{pgfscope}%
\pgfpathrectangle{\pgfqpoint{0.329460in}{0.284240in}}{\pgfqpoint{1.989680in}{1.989680in}}%
\pgfusepath{clip}%
\pgfsetbuttcap%
\pgfsetroundjoin%
\definecolor{currentfill}{rgb}{0.248629,0.278775,0.534556}%
\pgfsetfillcolor{currentfill}%
\pgfsetlinewidth{0.000000pt}%
\definecolor{currentstroke}{rgb}{0.000000,0.000000,0.000000}%
\pgfsetstrokecolor{currentstroke}%
\pgfsetdash{}{0pt}%
\pgfpathmoveto{\pgfqpoint{1.733274in}{1.181814in}}%
\pgfpathlineto{\pgfqpoint{1.736563in}{1.175670in}}%
\pgfpathlineto{\pgfqpoint{1.739852in}{1.169595in}}%
\pgfpathlineto{\pgfqpoint{1.743140in}{1.163594in}}%
\pgfpathlineto{\pgfqpoint{1.746428in}{1.157668in}}%
\pgfpathlineto{\pgfqpoint{1.746113in}{1.151279in}}%
\pgfpathlineto{\pgfqpoint{1.745402in}{1.144891in}}%
\pgfpathlineto{\pgfqpoint{1.744295in}{1.138511in}}%
\pgfpathlineto{\pgfqpoint{1.742792in}{1.132146in}}%
\pgfpathlineto{\pgfqpoint{1.739522in}{1.138308in}}%
\pgfpathlineto{\pgfqpoint{1.736251in}{1.144546in}}%
\pgfpathlineto{\pgfqpoint{1.732980in}{1.150857in}}%
\pgfpathlineto{\pgfqpoint{1.729709in}{1.157239in}}%
\pgfpathlineto{\pgfqpoint{1.731174in}{1.163367in}}%
\pgfpathlineto{\pgfqpoint{1.732257in}{1.169510in}}%
\pgfpathlineto{\pgfqpoint{1.732957in}{1.175661in}}%
\pgfpathlineto{\pgfqpoint{1.733274in}{1.181814in}}%
\pgfpathclose%
\pgfusepath{fill}%
\end{pgfscope}%
\begin{pgfscope}%
\pgfpathrectangle{\pgfqpoint{0.329460in}{0.284240in}}{\pgfqpoint{1.989680in}{1.989680in}}%
\pgfusepath{clip}%
\pgfsetbuttcap%
\pgfsetroundjoin%
\definecolor{currentfill}{rgb}{0.279566,0.067836,0.391917}%
\pgfsetfillcolor{currentfill}%
\pgfsetlinewidth{0.000000pt}%
\definecolor{currentstroke}{rgb}{0.000000,0.000000,0.000000}%
\pgfsetstrokecolor{currentstroke}%
\pgfsetdash{}{0pt}%
\pgfpathmoveto{\pgfqpoint{0.908894in}{0.993438in}}%
\pgfpathlineto{\pgfqpoint{0.905689in}{0.989345in}}%
\pgfpathlineto{\pgfqpoint{0.902480in}{0.985404in}}%
\pgfpathlineto{\pgfqpoint{0.899269in}{0.981616in}}%
\pgfpathlineto{\pgfqpoint{0.896054in}{0.977986in}}%
\pgfpathlineto{\pgfqpoint{0.892024in}{0.985581in}}%
\pgfpathlineto{\pgfqpoint{0.888465in}{0.993229in}}%
\pgfpathlineto{\pgfqpoint{0.885379in}{1.000924in}}%
\pgfpathlineto{\pgfqpoint{0.882769in}{1.008656in}}%
\pgfpathlineto{\pgfqpoint{0.886063in}{1.012053in}}%
\pgfpathlineto{\pgfqpoint{0.889353in}{1.015608in}}%
\pgfpathlineto{\pgfqpoint{0.892640in}{1.019317in}}%
\pgfpathlineto{\pgfqpoint{0.895924in}{1.023176in}}%
\pgfpathlineto{\pgfqpoint{0.898476in}{1.015678in}}%
\pgfpathlineto{\pgfqpoint{0.901489in}{1.008217in}}%
\pgfpathlineto{\pgfqpoint{0.904963in}{1.000801in}}%
\pgfpathlineto{\pgfqpoint{0.908894in}{0.993438in}}%
\pgfpathclose%
\pgfusepath{fill}%
\end{pgfscope}%
\begin{pgfscope}%
\pgfpathrectangle{\pgfqpoint{0.329460in}{0.284240in}}{\pgfqpoint{1.989680in}{1.989680in}}%
\pgfusepath{clip}%
\pgfsetbuttcap%
\pgfsetroundjoin%
\definecolor{currentfill}{rgb}{0.195860,0.395433,0.555276}%
\pgfsetfillcolor{currentfill}%
\pgfsetlinewidth{0.000000pt}%
\definecolor{currentstroke}{rgb}{0.000000,0.000000,0.000000}%
\pgfsetstrokecolor{currentstroke}%
\pgfsetdash{}{0pt}%
\pgfpathmoveto{\pgfqpoint{1.691580in}{1.281649in}}%
\pgfpathlineto{\pgfqpoint{1.694852in}{1.275126in}}%
\pgfpathlineto{\pgfqpoint{1.698123in}{1.268639in}}%
\pgfpathlineto{\pgfqpoint{1.701392in}{1.262189in}}%
\pgfpathlineto{\pgfqpoint{1.704659in}{1.255781in}}%
\pgfpathlineto{\pgfqpoint{1.705754in}{1.250144in}}%
\pgfpathlineto{\pgfqpoint{1.706497in}{1.244487in}}%
\pgfpathlineto{\pgfqpoint{1.706887in}{1.238816in}}%
\pgfpathlineto{\pgfqpoint{1.706923in}{1.233136in}}%
\pgfpathlineto{\pgfqpoint{1.703623in}{1.239778in}}%
\pgfpathlineto{\pgfqpoint{1.700322in}{1.246460in}}%
\pgfpathlineto{\pgfqpoint{1.697019in}{1.253181in}}%
\pgfpathlineto{\pgfqpoint{1.693714in}{1.259936in}}%
\pgfpathlineto{\pgfqpoint{1.693691in}{1.265381in}}%
\pgfpathlineto{\pgfqpoint{1.693327in}{1.270819in}}%
\pgfpathlineto{\pgfqpoint{1.692622in}{1.276243in}}%
\pgfpathlineto{\pgfqpoint{1.691580in}{1.281649in}}%
\pgfpathclose%
\pgfusepath{fill}%
\end{pgfscope}%
\begin{pgfscope}%
\pgfpathrectangle{\pgfqpoint{0.329460in}{0.284240in}}{\pgfqpoint{1.989680in}{1.989680in}}%
\pgfusepath{clip}%
\pgfsetbuttcap%
\pgfsetroundjoin%
\definecolor{currentfill}{rgb}{0.133743,0.548535,0.553541}%
\pgfsetfillcolor{currentfill}%
\pgfsetlinewidth{0.000000pt}%
\definecolor{currentstroke}{rgb}{0.000000,0.000000,0.000000}%
\pgfsetstrokecolor{currentstroke}%
\pgfsetdash{}{0pt}%
\pgfpathmoveto{\pgfqpoint{1.622091in}{1.423503in}}%
\pgfpathlineto{\pgfqpoint{1.625191in}{1.417217in}}%
\pgfpathlineto{\pgfqpoint{1.628288in}{1.410925in}}%
\pgfpathlineto{\pgfqpoint{1.631382in}{1.404629in}}%
\pgfpathlineto{\pgfqpoint{1.634474in}{1.398332in}}%
\pgfpathlineto{\pgfqpoint{1.637680in}{1.393899in}}%
\pgfpathlineto{\pgfqpoint{1.640607in}{1.389416in}}%
\pgfpathlineto{\pgfqpoint{1.643252in}{1.384885in}}%
\pgfpathlineto{\pgfqpoint{1.645611in}{1.380313in}}%
\pgfpathlineto{\pgfqpoint{1.642389in}{1.386825in}}%
\pgfpathlineto{\pgfqpoint{1.639164in}{1.393336in}}%
\pgfpathlineto{\pgfqpoint{1.635936in}{1.399844in}}%
\pgfpathlineto{\pgfqpoint{1.632705in}{1.406345in}}%
\pgfpathlineto{\pgfqpoint{1.630459in}{1.410698in}}%
\pgfpathlineto{\pgfqpoint{1.627939in}{1.415012in}}%
\pgfpathlineto{\pgfqpoint{1.625149in}{1.419282in}}%
\pgfpathlineto{\pgfqpoint{1.622091in}{1.423503in}}%
\pgfpathclose%
\pgfusepath{fill}%
\end{pgfscope}%
\begin{pgfscope}%
\pgfpathrectangle{\pgfqpoint{0.329460in}{0.284240in}}{\pgfqpoint{1.989680in}{1.989680in}}%
\pgfusepath{clip}%
\pgfsetbuttcap%
\pgfsetroundjoin%
\definecolor{currentfill}{rgb}{0.283072,0.130895,0.449241}%
\pgfsetfillcolor{currentfill}%
\pgfsetlinewidth{0.000000pt}%
\definecolor{currentstroke}{rgb}{0.000000,0.000000,0.000000}%
\pgfsetstrokecolor{currentstroke}%
\pgfsetdash{}{0pt}%
\pgfpathmoveto{\pgfqpoint{1.782058in}{1.065300in}}%
\pgfpathlineto{\pgfqpoint{1.785335in}{1.060420in}}%
\pgfpathlineto{\pgfqpoint{1.788615in}{1.055661in}}%
\pgfpathlineto{\pgfqpoint{1.791896in}{1.051029in}}%
\pgfpathlineto{\pgfqpoint{1.795179in}{1.046524in}}%
\pgfpathlineto{\pgfqpoint{1.793085in}{1.039235in}}%
\pgfpathlineto{\pgfqpoint{1.790542in}{1.031975in}}%
\pgfpathlineto{\pgfqpoint{1.787551in}{1.024753in}}%
\pgfpathlineto{\pgfqpoint{1.784114in}{1.017576in}}%
\pgfpathlineto{\pgfqpoint{1.780899in}{1.022315in}}%
\pgfpathlineto{\pgfqpoint{1.777686in}{1.027183in}}%
\pgfpathlineto{\pgfqpoint{1.774474in}{1.032177in}}%
\pgfpathlineto{\pgfqpoint{1.771265in}{1.037292in}}%
\pgfpathlineto{\pgfqpoint{1.774613in}{1.044235in}}%
\pgfpathlineto{\pgfqpoint{1.777529in}{1.051222in}}%
\pgfpathlineto{\pgfqpoint{1.780011in}{1.058246in}}%
\pgfpathlineto{\pgfqpoint{1.782058in}{1.065300in}}%
\pgfpathclose%
\pgfusepath{fill}%
\end{pgfscope}%
\begin{pgfscope}%
\pgfpathrectangle{\pgfqpoint{0.329460in}{0.284240in}}{\pgfqpoint{1.989680in}{1.989680in}}%
\pgfusepath{clip}%
\pgfsetbuttcap%
\pgfsetroundjoin%
\definecolor{currentfill}{rgb}{0.344074,0.780029,0.397381}%
\pgfsetfillcolor{currentfill}%
\pgfsetlinewidth{0.000000pt}%
\definecolor{currentstroke}{rgb}{0.000000,0.000000,0.000000}%
\pgfsetstrokecolor{currentstroke}%
\pgfsetdash{}{0pt}%
\pgfpathmoveto{\pgfqpoint{1.327366in}{1.648244in}}%
\pgfpathlineto{\pgfqpoint{1.326955in}{1.644136in}}%
\pgfpathlineto{\pgfqpoint{1.326544in}{1.639964in}}%
\pgfpathlineto{\pgfqpoint{1.326134in}{1.635728in}}%
\pgfpathlineto{\pgfqpoint{1.325724in}{1.631432in}}%
\pgfpathlineto{\pgfqpoint{1.332159in}{1.631765in}}%
\pgfpathlineto{\pgfqpoint{1.338612in}{1.632001in}}%
\pgfpathlineto{\pgfqpoint{1.345077in}{1.632141in}}%
\pgfpathlineto{\pgfqpoint{1.351547in}{1.632184in}}%
\pgfpathlineto{\pgfqpoint{1.351541in}{1.636468in}}%
\pgfpathlineto{\pgfqpoint{1.351536in}{1.640691in}}%
\pgfpathlineto{\pgfqpoint{1.351530in}{1.644851in}}%
\pgfpathlineto{\pgfqpoint{1.351524in}{1.648946in}}%
\pgfpathlineto{\pgfqpoint{1.345471in}{1.648906in}}%
\pgfpathlineto{\pgfqpoint{1.339423in}{1.648775in}}%
\pgfpathlineto{\pgfqpoint{1.333386in}{1.648555in}}%
\pgfpathlineto{\pgfqpoint{1.327366in}{1.648244in}}%
\pgfpathclose%
\pgfusepath{fill}%
\end{pgfscope}%
\begin{pgfscope}%
\pgfpathrectangle{\pgfqpoint{0.329460in}{0.284240in}}{\pgfqpoint{1.989680in}{1.989680in}}%
\pgfusepath{clip}%
\pgfsetbuttcap%
\pgfsetroundjoin%
\definecolor{currentfill}{rgb}{0.344074,0.780029,0.397381}%
\pgfsetfillcolor{currentfill}%
\pgfsetlinewidth{0.000000pt}%
\definecolor{currentstroke}{rgb}{0.000000,0.000000,0.000000}%
\pgfsetstrokecolor{currentstroke}%
\pgfsetdash{}{0pt}%
\pgfpathmoveto{\pgfqpoint{1.351524in}{1.648946in}}%
\pgfpathlineto{\pgfqpoint{1.351530in}{1.644851in}}%
\pgfpathlineto{\pgfqpoint{1.351536in}{1.640691in}}%
\pgfpathlineto{\pgfqpoint{1.351541in}{1.636468in}}%
\pgfpathlineto{\pgfqpoint{1.351547in}{1.632184in}}%
\pgfpathlineto{\pgfqpoint{1.358017in}{1.632130in}}%
\pgfpathlineto{\pgfqpoint{1.364481in}{1.631979in}}%
\pgfpathlineto{\pgfqpoint{1.370932in}{1.631732in}}%
\pgfpathlineto{\pgfqpoint{1.377365in}{1.631389in}}%
\pgfpathlineto{\pgfqpoint{1.376943in}{1.635686in}}%
\pgfpathlineto{\pgfqpoint{1.376522in}{1.639922in}}%
\pgfpathlineto{\pgfqpoint{1.376099in}{1.644096in}}%
\pgfpathlineto{\pgfqpoint{1.375676in}{1.648204in}}%
\pgfpathlineto{\pgfqpoint{1.369659in}{1.648525in}}%
\pgfpathlineto{\pgfqpoint{1.363624in}{1.648755in}}%
\pgfpathlineto{\pgfqpoint{1.357577in}{1.648896in}}%
\pgfpathlineto{\pgfqpoint{1.351524in}{1.648946in}}%
\pgfpathclose%
\pgfusepath{fill}%
\end{pgfscope}%
\begin{pgfscope}%
\pgfpathrectangle{\pgfqpoint{0.329460in}{0.284240in}}{\pgfqpoint{1.989680in}{1.989680in}}%
\pgfusepath{clip}%
\pgfsetbuttcap%
\pgfsetroundjoin%
\definecolor{currentfill}{rgb}{0.281477,0.755203,0.432552}%
\pgfsetfillcolor{currentfill}%
\pgfsetlinewidth{0.000000pt}%
\definecolor{currentstroke}{rgb}{0.000000,0.000000,0.000000}%
\pgfsetstrokecolor{currentstroke}%
\pgfsetdash{}{0pt}%
\pgfpathmoveto{\pgfqpoint{1.445162in}{1.621399in}}%
\pgfpathlineto{\pgfqpoint{1.446675in}{1.616874in}}%
\pgfpathlineto{\pgfqpoint{1.448186in}{1.612292in}}%
\pgfpathlineto{\pgfqpoint{1.449695in}{1.607654in}}%
\pgfpathlineto{\pgfqpoint{1.451201in}{1.602963in}}%
\pgfpathlineto{\pgfqpoint{1.457313in}{1.601413in}}%
\pgfpathlineto{\pgfqpoint{1.463324in}{1.599772in}}%
\pgfpathlineto{\pgfqpoint{1.469228in}{1.598040in}}%
\pgfpathlineto{\pgfqpoint{1.475020in}{1.596221in}}%
\pgfpathlineto{\pgfqpoint{1.473152in}{1.601019in}}%
\pgfpathlineto{\pgfqpoint{1.471282in}{1.605763in}}%
\pgfpathlineto{\pgfqpoint{1.469410in}{1.610453in}}%
\pgfpathlineto{\pgfqpoint{1.467535in}{1.615084in}}%
\pgfpathlineto{\pgfqpoint{1.462095in}{1.616788in}}%
\pgfpathlineto{\pgfqpoint{1.456550in}{1.618410in}}%
\pgfpathlineto{\pgfqpoint{1.450904in}{1.619947in}}%
\pgfpathlineto{\pgfqpoint{1.445162in}{1.621399in}}%
\pgfpathclose%
\pgfusepath{fill}%
\end{pgfscope}%
\begin{pgfscope}%
\pgfpathrectangle{\pgfqpoint{0.329460in}{0.284240in}}{\pgfqpoint{1.989680in}{1.989680in}}%
\pgfusepath{clip}%
\pgfsetbuttcap%
\pgfsetroundjoin%
\definecolor{currentfill}{rgb}{0.268510,0.009605,0.335427}%
\pgfsetfillcolor{currentfill}%
\pgfsetlinewidth{0.000000pt}%
\definecolor{currentstroke}{rgb}{0.000000,0.000000,0.000000}%
\pgfsetstrokecolor{currentstroke}%
\pgfsetdash{}{0pt}%
\pgfpathmoveto{\pgfqpoint{0.830750in}{0.944621in}}%
\pgfpathlineto{\pgfqpoint{0.827418in}{0.945233in}}%
\pgfpathlineto{\pgfqpoint{0.824077in}{0.946095in}}%
\pgfpathlineto{\pgfqpoint{0.820728in}{0.947212in}}%
\pgfpathlineto{\pgfqpoint{0.817370in}{0.948587in}}%
\pgfpathlineto{\pgfqpoint{0.812751in}{0.957532in}}%
\pgfpathlineto{\pgfqpoint{0.808687in}{0.966538in}}%
\pgfpathlineto{\pgfqpoint{0.805181in}{0.975594in}}%
\pgfpathlineto{\pgfqpoint{0.802234in}{0.984692in}}%
\pgfpathlineto{\pgfqpoint{0.805668in}{0.983098in}}%
\pgfpathlineto{\pgfqpoint{0.809093in}{0.981763in}}%
\pgfpathlineto{\pgfqpoint{0.812511in}{0.980680in}}%
\pgfpathlineto{\pgfqpoint{0.815920in}{0.979847in}}%
\pgfpathlineto{\pgfqpoint{0.818812in}{0.970970in}}%
\pgfpathlineto{\pgfqpoint{0.822249in}{0.962134in}}%
\pgfpathlineto{\pgfqpoint{0.826229in}{0.953348in}}%
\pgfpathlineto{\pgfqpoint{0.830750in}{0.944621in}}%
\pgfpathclose%
\pgfusepath{fill}%
\end{pgfscope}%
\begin{pgfscope}%
\pgfpathrectangle{\pgfqpoint{0.329460in}{0.284240in}}{\pgfqpoint{1.989680in}{1.989680in}}%
\pgfusepath{clip}%
\pgfsetbuttcap%
\pgfsetroundjoin%
\definecolor{currentfill}{rgb}{0.276194,0.190074,0.493001}%
\pgfsetfillcolor{currentfill}%
\pgfsetlinewidth{0.000000pt}%
\definecolor{currentstroke}{rgb}{0.000000,0.000000,0.000000}%
\pgfsetstrokecolor{currentstroke}%
\pgfsetdash{}{0pt}%
\pgfpathmoveto{\pgfqpoint{0.745849in}{1.049314in}}%
\pgfpathlineto{\pgfqpoint{0.742218in}{1.056067in}}%
\pgfpathlineto{\pgfqpoint{0.738572in}{1.063174in}}%
\pgfpathlineto{\pgfqpoint{0.734911in}{1.070639in}}%
\pgfpathlineto{\pgfqpoint{0.731235in}{1.078469in}}%
\pgfpathlineto{\pgfqpoint{0.728651in}{1.088629in}}%
\pgfpathlineto{\pgfqpoint{0.726699in}{1.098810in}}%
\pgfpathlineto{\pgfqpoint{0.725380in}{1.109001in}}%
\pgfpathlineto{\pgfqpoint{0.724691in}{1.119192in}}%
\pgfpathlineto{\pgfqpoint{0.728382in}{1.111171in}}%
\pgfpathlineto{\pgfqpoint{0.732059in}{1.103513in}}%
\pgfpathlineto{\pgfqpoint{0.735720in}{1.096213in}}%
\pgfpathlineto{\pgfqpoint{0.739367in}{1.089263in}}%
\pgfpathlineto{\pgfqpoint{0.740063in}{1.079264in}}%
\pgfpathlineto{\pgfqpoint{0.741375in}{1.069267in}}%
\pgfpathlineto{\pgfqpoint{0.743303in}{1.059280in}}%
\pgfpathlineto{\pgfqpoint{0.745849in}{1.049314in}}%
\pgfpathclose%
\pgfusepath{fill}%
\end{pgfscope}%
\begin{pgfscope}%
\pgfpathrectangle{\pgfqpoint{0.329460in}{0.284240in}}{\pgfqpoint{1.989680in}{1.989680in}}%
\pgfusepath{clip}%
\pgfsetbuttcap%
\pgfsetroundjoin%
\definecolor{currentfill}{rgb}{0.344074,0.780029,0.397381}%
\pgfsetfillcolor{currentfill}%
\pgfsetlinewidth{0.000000pt}%
\definecolor{currentstroke}{rgb}{0.000000,0.000000,0.000000}%
\pgfsetstrokecolor{currentstroke}%
\pgfsetdash{}{0pt}%
\pgfpathmoveto{\pgfqpoint{1.303571in}{1.646110in}}%
\pgfpathlineto{\pgfqpoint{1.302748in}{1.641964in}}%
\pgfpathlineto{\pgfqpoint{1.301927in}{1.637753in}}%
\pgfpathlineto{\pgfqpoint{1.301106in}{1.633479in}}%
\pgfpathlineto{\pgfqpoint{1.300287in}{1.629144in}}%
\pgfpathlineto{\pgfqpoint{1.306589in}{1.629858in}}%
\pgfpathlineto{\pgfqpoint{1.312933in}{1.630478in}}%
\pgfpathlineto{\pgfqpoint{1.319313in}{1.631003in}}%
\pgfpathlineto{\pgfqpoint{1.325724in}{1.631432in}}%
\pgfpathlineto{\pgfqpoint{1.326134in}{1.635728in}}%
\pgfpathlineto{\pgfqpoint{1.326544in}{1.639964in}}%
\pgfpathlineto{\pgfqpoint{1.326955in}{1.644136in}}%
\pgfpathlineto{\pgfqpoint{1.327366in}{1.648244in}}%
\pgfpathlineto{\pgfqpoint{1.321369in}{1.647844in}}%
\pgfpathlineto{\pgfqpoint{1.315401in}{1.647355in}}%
\pgfpathlineto{\pgfqpoint{1.309466in}{1.646777in}}%
\pgfpathlineto{\pgfqpoint{1.303571in}{1.646110in}}%
\pgfpathclose%
\pgfusepath{fill}%
\end{pgfscope}%
\begin{pgfscope}%
\pgfpathrectangle{\pgfqpoint{0.329460in}{0.284240in}}{\pgfqpoint{1.989680in}{1.989680in}}%
\pgfusepath{clip}%
\pgfsetbuttcap%
\pgfsetroundjoin%
\definecolor{currentfill}{rgb}{0.344074,0.780029,0.397381}%
\pgfsetfillcolor{currentfill}%
\pgfsetlinewidth{0.000000pt}%
\definecolor{currentstroke}{rgb}{0.000000,0.000000,0.000000}%
\pgfsetstrokecolor{currentstroke}%
\pgfsetdash{}{0pt}%
\pgfpathmoveto{\pgfqpoint{1.375676in}{1.648204in}}%
\pgfpathlineto{\pgfqpoint{1.376099in}{1.644096in}}%
\pgfpathlineto{\pgfqpoint{1.376522in}{1.639922in}}%
\pgfpathlineto{\pgfqpoint{1.376943in}{1.635686in}}%
\pgfpathlineto{\pgfqpoint{1.377365in}{1.631389in}}%
\pgfpathlineto{\pgfqpoint{1.383772in}{1.630949in}}%
\pgfpathlineto{\pgfqpoint{1.390149in}{1.630414in}}%
\pgfpathlineto{\pgfqpoint{1.396489in}{1.629784in}}%
\pgfpathlineto{\pgfqpoint{1.402786in}{1.629058in}}%
\pgfpathlineto{\pgfqpoint{1.401955in}{1.633395in}}%
\pgfpathlineto{\pgfqpoint{1.401123in}{1.637670in}}%
\pgfpathlineto{\pgfqpoint{1.400290in}{1.641883in}}%
\pgfpathlineto{\pgfqpoint{1.399456in}{1.646031in}}%
\pgfpathlineto{\pgfqpoint{1.393566in}{1.646707in}}%
\pgfpathlineto{\pgfqpoint{1.387636in}{1.647295in}}%
\pgfpathlineto{\pgfqpoint{1.381670in}{1.647794in}}%
\pgfpathlineto{\pgfqpoint{1.375676in}{1.648204in}}%
\pgfpathclose%
\pgfusepath{fill}%
\end{pgfscope}%
\begin{pgfscope}%
\pgfpathrectangle{\pgfqpoint{0.329460in}{0.284240in}}{\pgfqpoint{1.989680in}{1.989680in}}%
\pgfusepath{clip}%
\pgfsetbuttcap%
\pgfsetroundjoin%
\definecolor{currentfill}{rgb}{0.163625,0.471133,0.558148}%
\pgfsetfillcolor{currentfill}%
\pgfsetlinewidth{0.000000pt}%
\definecolor{currentstroke}{rgb}{0.000000,0.000000,0.000000}%
\pgfsetstrokecolor{currentstroke}%
\pgfsetdash{}{0pt}%
\pgfpathmoveto{\pgfqpoint{1.036202in}{1.330379in}}%
\pgfpathlineto{\pgfqpoint{1.032900in}{1.323621in}}%
\pgfpathlineto{\pgfqpoint{1.029601in}{1.316876in}}%
\pgfpathlineto{\pgfqpoint{1.026304in}{1.310148in}}%
\pgfpathlineto{\pgfqpoint{1.023009in}{1.303439in}}%
\pgfpathlineto{\pgfqpoint{1.024037in}{1.308612in}}%
\pgfpathlineto{\pgfqpoint{1.025388in}{1.313763in}}%
\pgfpathlineto{\pgfqpoint{1.027060in}{1.318885in}}%
\pgfpathlineto{\pgfqpoint{1.029051in}{1.323974in}}%
\pgfpathlineto{\pgfqpoint{1.032274in}{1.330455in}}%
\pgfpathlineto{\pgfqpoint{1.035500in}{1.336956in}}%
\pgfpathlineto{\pgfqpoint{1.038729in}{1.343473in}}%
\pgfpathlineto{\pgfqpoint{1.041960in}{1.350005in}}%
\pgfpathlineto{\pgfqpoint{1.040059in}{1.345140in}}%
\pgfpathlineto{\pgfqpoint{1.038464in}{1.340245in}}%
\pgfpathlineto{\pgfqpoint{1.037178in}{1.335323in}}%
\pgfpathlineto{\pgfqpoint{1.036202in}{1.330379in}}%
\pgfpathclose%
\pgfusepath{fill}%
\end{pgfscope}%
\begin{pgfscope}%
\pgfpathrectangle{\pgfqpoint{0.329460in}{0.284240in}}{\pgfqpoint{1.989680in}{1.989680in}}%
\pgfusepath{clip}%
\pgfsetbuttcap%
\pgfsetroundjoin%
\definecolor{currentfill}{rgb}{0.281477,0.755203,0.432552}%
\pgfsetfillcolor{currentfill}%
\pgfsetlinewidth{0.000000pt}%
\definecolor{currentstroke}{rgb}{0.000000,0.000000,0.000000}%
\pgfsetstrokecolor{currentstroke}%
\pgfsetdash{}{0pt}%
\pgfpathmoveto{\pgfqpoint{1.230098in}{1.613501in}}%
\pgfpathlineto{\pgfqpoint{1.228146in}{1.608842in}}%
\pgfpathlineto{\pgfqpoint{1.226197in}{1.604127in}}%
\pgfpathlineto{\pgfqpoint{1.224250in}{1.599355in}}%
\pgfpathlineto{\pgfqpoint{1.222306in}{1.594530in}}%
\pgfpathlineto{\pgfqpoint{1.227993in}{1.596427in}}%
\pgfpathlineto{\pgfqpoint{1.233797in}{1.598237in}}%
\pgfpathlineto{\pgfqpoint{1.239714in}{1.599959in}}%
\pgfpathlineto{\pgfqpoint{1.245736in}{1.601590in}}%
\pgfpathlineto{\pgfqpoint{1.247325in}{1.606303in}}%
\pgfpathlineto{\pgfqpoint{1.248916in}{1.610962in}}%
\pgfpathlineto{\pgfqpoint{1.250510in}{1.615566in}}%
\pgfpathlineto{\pgfqpoint{1.252105in}{1.620113in}}%
\pgfpathlineto{\pgfqpoint{1.246448in}{1.618585in}}%
\pgfpathlineto{\pgfqpoint{1.240891in}{1.616973in}}%
\pgfpathlineto{\pgfqpoint{1.235439in}{1.615277in}}%
\pgfpathlineto{\pgfqpoint{1.230098in}{1.613501in}}%
\pgfpathclose%
\pgfusepath{fill}%
\end{pgfscope}%
\begin{pgfscope}%
\pgfpathrectangle{\pgfqpoint{0.329460in}{0.284240in}}{\pgfqpoint{1.989680in}{1.989680in}}%
\pgfusepath{clip}%
\pgfsetbuttcap%
\pgfsetroundjoin%
\definecolor{currentfill}{rgb}{0.166383,0.690856,0.496502}%
\pgfsetfillcolor{currentfill}%
\pgfsetlinewidth{0.000000pt}%
\definecolor{currentstroke}{rgb}{0.000000,0.000000,0.000000}%
\pgfsetstrokecolor{currentstroke}%
\pgfsetdash{}{0pt}%
\pgfpathmoveto{\pgfqpoint{1.526635in}{1.557881in}}%
\pgfpathlineto{\pgfqpoint{1.529123in}{1.552564in}}%
\pgfpathlineto{\pgfqpoint{1.531608in}{1.547205in}}%
\pgfpathlineto{\pgfqpoint{1.534089in}{1.541806in}}%
\pgfpathlineto{\pgfqpoint{1.536568in}{1.536370in}}%
\pgfpathlineto{\pgfqpoint{1.541663in}{1.533512in}}%
\pgfpathlineto{\pgfqpoint{1.546577in}{1.530577in}}%
\pgfpathlineto{\pgfqpoint{1.551302in}{1.527567in}}%
\pgfpathlineto{\pgfqpoint{1.555835in}{1.524485in}}%
\pgfpathlineto{\pgfqpoint{1.553094in}{1.530091in}}%
\pgfpathlineto{\pgfqpoint{1.550349in}{1.535660in}}%
\pgfpathlineto{\pgfqpoint{1.547601in}{1.541188in}}%
\pgfpathlineto{\pgfqpoint{1.544850in}{1.546675in}}%
\pgfpathlineto{\pgfqpoint{1.540565in}{1.549581in}}%
\pgfpathlineto{\pgfqpoint{1.536098in}{1.552419in}}%
\pgfpathlineto{\pgfqpoint{1.531454in}{1.555186in}}%
\pgfpathlineto{\pgfqpoint{1.526635in}{1.557881in}}%
\pgfpathclose%
\pgfusepath{fill}%
\end{pgfscope}%
\begin{pgfscope}%
\pgfpathrectangle{\pgfqpoint{0.329460in}{0.284240in}}{\pgfqpoint{1.989680in}{1.989680in}}%
\pgfusepath{clip}%
\pgfsetbuttcap%
\pgfsetroundjoin%
\definecolor{currentfill}{rgb}{0.201239,0.383670,0.554294}%
\pgfsetfillcolor{currentfill}%
\pgfsetlinewidth{0.000000pt}%
\definecolor{currentstroke}{rgb}{0.000000,0.000000,0.000000}%
\pgfsetstrokecolor{currentstroke}%
\pgfsetdash{}{0pt}%
\pgfpathmoveto{\pgfqpoint{0.694552in}{1.197191in}}%
\pgfpathlineto{\pgfqpoint{0.690702in}{1.208768in}}%
\pgfpathlineto{\pgfqpoint{0.686833in}{1.220774in}}%
\pgfpathlineto{\pgfqpoint{0.682943in}{1.233216in}}%
\pgfpathlineto{\pgfqpoint{0.679033in}{1.246102in}}%
\pgfpathlineto{\pgfqpoint{0.679045in}{1.256807in}}%
\pgfpathlineto{\pgfqpoint{0.679727in}{1.267488in}}%
\pgfpathlineto{\pgfqpoint{0.681076in}{1.278135in}}%
\pgfpathlineto{\pgfqpoint{0.683085in}{1.288737in}}%
\pgfpathlineto{\pgfqpoint{0.686946in}{1.275693in}}%
\pgfpathlineto{\pgfqpoint{0.690787in}{1.263089in}}%
\pgfpathlineto{\pgfqpoint{0.694607in}{1.250919in}}%
\pgfpathlineto{\pgfqpoint{0.698409in}{1.239174in}}%
\pgfpathlineto{\pgfqpoint{0.696470in}{1.228733in}}%
\pgfpathlineto{\pgfqpoint{0.695179in}{1.218249in}}%
\pgfpathlineto{\pgfqpoint{0.694538in}{1.207731in}}%
\pgfpathlineto{\pgfqpoint{0.694552in}{1.197191in}}%
\pgfpathclose%
\pgfusepath{fill}%
\end{pgfscope}%
\begin{pgfscope}%
\pgfpathrectangle{\pgfqpoint{0.329460in}{0.284240in}}{\pgfqpoint{1.989680in}{1.989680in}}%
\pgfusepath{clip}%
\pgfsetbuttcap%
\pgfsetroundjoin%
\definecolor{currentfill}{rgb}{0.282327,0.094955,0.417331}%
\pgfsetfillcolor{currentfill}%
\pgfsetlinewidth{0.000000pt}%
\definecolor{currentstroke}{rgb}{0.000000,0.000000,0.000000}%
\pgfsetstrokecolor{currentstroke}%
\pgfsetdash{}{0pt}%
\pgfpathmoveto{\pgfqpoint{0.921689in}{1.011240in}}%
\pgfpathlineto{\pgfqpoint{0.918494in}{1.006582in}}%
\pgfpathlineto{\pgfqpoint{0.915297in}{1.002060in}}%
\pgfpathlineto{\pgfqpoint{0.912097in}{0.997677in}}%
\pgfpathlineto{\pgfqpoint{0.908894in}{0.993438in}}%
\pgfpathlineto{\pgfqpoint{0.904963in}{1.000801in}}%
\pgfpathlineto{\pgfqpoint{0.901489in}{1.008217in}}%
\pgfpathlineto{\pgfqpoint{0.898476in}{1.015678in}}%
\pgfpathlineto{\pgfqpoint{0.895924in}{1.023176in}}%
\pgfpathlineto{\pgfqpoint{0.899206in}{1.027182in}}%
\pgfpathlineto{\pgfqpoint{0.902485in}{1.031331in}}%
\pgfpathlineto{\pgfqpoint{0.905761in}{1.035619in}}%
\pgfpathlineto{\pgfqpoint{0.909036in}{1.040043in}}%
\pgfpathlineto{\pgfqpoint{0.911529in}{1.032780in}}%
\pgfpathlineto{\pgfqpoint{0.914470in}{1.025554in}}%
\pgfpathlineto{\pgfqpoint{0.917857in}{1.018371in}}%
\pgfpathlineto{\pgfqpoint{0.921689in}{1.011240in}}%
\pgfpathclose%
\pgfusepath{fill}%
\end{pgfscope}%
\begin{pgfscope}%
\pgfpathrectangle{\pgfqpoint{0.329460in}{0.284240in}}{\pgfqpoint{1.989680in}{1.989680in}}%
\pgfusepath{clip}%
\pgfsetbuttcap%
\pgfsetroundjoin%
\definecolor{currentfill}{rgb}{0.134692,0.658636,0.517649}%
\pgfsetfillcolor{currentfill}%
\pgfsetlinewidth{0.000000pt}%
\definecolor{currentstroke}{rgb}{0.000000,0.000000,0.000000}%
\pgfsetstrokecolor{currentstroke}%
\pgfsetdash{}{0pt}%
\pgfpathmoveto{\pgfqpoint{1.555835in}{1.524485in}}%
\pgfpathlineto{\pgfqpoint{1.558573in}{1.518843in}}%
\pgfpathlineto{\pgfqpoint{1.561308in}{1.513168in}}%
\pgfpathlineto{\pgfqpoint{1.564040in}{1.507461in}}%
\pgfpathlineto{\pgfqpoint{1.566769in}{1.501726in}}%
\pgfpathlineto{\pgfqpoint{1.571343in}{1.498394in}}%
\pgfpathlineto{\pgfqpoint{1.575705in}{1.494991in}}%
\pgfpathlineto{\pgfqpoint{1.579850in}{1.491522in}}%
\pgfpathlineto{\pgfqpoint{1.583774in}{1.487988in}}%
\pgfpathlineto{\pgfqpoint{1.580823in}{1.493911in}}%
\pgfpathlineto{\pgfqpoint{1.577869in}{1.499805in}}%
\pgfpathlineto{\pgfqpoint{1.574912in}{1.505667in}}%
\pgfpathlineto{\pgfqpoint{1.571952in}{1.511495in}}%
\pgfpathlineto{\pgfqpoint{1.568233in}{1.514836in}}%
\pgfpathlineto{\pgfqpoint{1.564305in}{1.518116in}}%
\pgfpathlineto{\pgfqpoint{1.560171in}{1.521334in}}%
\pgfpathlineto{\pgfqpoint{1.555835in}{1.524485in}}%
\pgfpathclose%
\pgfusepath{fill}%
\end{pgfscope}%
\begin{pgfscope}%
\pgfpathrectangle{\pgfqpoint{0.329460in}{0.284240in}}{\pgfqpoint{1.989680in}{1.989680in}}%
\pgfusepath{clip}%
\pgfsetbuttcap%
\pgfsetroundjoin%
\definecolor{currentfill}{rgb}{0.248629,0.278775,0.534556}%
\pgfsetfillcolor{currentfill}%
\pgfsetlinewidth{0.000000pt}%
\definecolor{currentstroke}{rgb}{0.000000,0.000000,0.000000}%
\pgfsetstrokecolor{currentstroke}%
\pgfsetdash{}{0pt}%
\pgfpathmoveto{\pgfqpoint{0.974290in}{1.151809in}}%
\pgfpathlineto{\pgfqpoint{0.971030in}{1.145375in}}%
\pgfpathlineto{\pgfqpoint{0.967770in}{1.139011in}}%
\pgfpathlineto{\pgfqpoint{0.964510in}{1.132721in}}%
\pgfpathlineto{\pgfqpoint{0.961251in}{1.126506in}}%
\pgfpathlineto{\pgfqpoint{0.959396in}{1.132852in}}%
\pgfpathlineto{\pgfqpoint{0.957937in}{1.139219in}}%
\pgfpathlineto{\pgfqpoint{0.956875in}{1.145600in}}%
\pgfpathlineto{\pgfqpoint{0.956208in}{1.151989in}}%
\pgfpathlineto{\pgfqpoint{0.959496in}{1.157966in}}%
\pgfpathlineto{\pgfqpoint{0.962785in}{1.164020in}}%
\pgfpathlineto{\pgfqpoint{0.966074in}{1.170147in}}%
\pgfpathlineto{\pgfqpoint{0.969364in}{1.176345in}}%
\pgfpathlineto{\pgfqpoint{0.970021in}{1.170193in}}%
\pgfpathlineto{\pgfqpoint{0.971062in}{1.164049in}}%
\pgfpathlineto{\pgfqpoint{0.972484in}{1.157919in}}%
\pgfpathlineto{\pgfqpoint{0.974290in}{1.151809in}}%
\pgfpathclose%
\pgfusepath{fill}%
\end{pgfscope}%
\begin{pgfscope}%
\pgfpathrectangle{\pgfqpoint{0.329460in}{0.284240in}}{\pgfqpoint{1.989680in}{1.989680in}}%
\pgfusepath{clip}%
\pgfsetbuttcap%
\pgfsetroundjoin%
\definecolor{currentfill}{rgb}{0.133743,0.548535,0.553541}%
\pgfsetfillcolor{currentfill}%
\pgfsetlinewidth{0.000000pt}%
\definecolor{currentstroke}{rgb}{0.000000,0.000000,0.000000}%
\pgfsetstrokecolor{currentstroke}%
\pgfsetdash{}{0pt}%
\pgfpathmoveto{\pgfqpoint{1.067905in}{1.402444in}}%
\pgfpathlineto{\pgfqpoint{1.064652in}{1.395894in}}%
\pgfpathlineto{\pgfqpoint{1.061402in}{1.389338in}}%
\pgfpathlineto{\pgfqpoint{1.058154in}{1.382778in}}%
\pgfpathlineto{\pgfqpoint{1.054910in}{1.376216in}}%
\pgfpathlineto{\pgfqpoint{1.057012in}{1.380823in}}%
\pgfpathlineto{\pgfqpoint{1.059403in}{1.385391in}}%
\pgfpathlineto{\pgfqpoint{1.062079in}{1.389916in}}%
\pgfpathlineto{\pgfqpoint{1.065038in}{1.394394in}}%
\pgfpathlineto{\pgfqpoint{1.068162in}{1.400738in}}%
\pgfpathlineto{\pgfqpoint{1.071289in}{1.407081in}}%
\pgfpathlineto{\pgfqpoint{1.074420in}{1.413420in}}%
\pgfpathlineto{\pgfqpoint{1.077553in}{1.419753in}}%
\pgfpathlineto{\pgfqpoint{1.074733in}{1.415489in}}%
\pgfpathlineto{\pgfqpoint{1.072183in}{1.411180in}}%
\pgfpathlineto{\pgfqpoint{1.069906in}{1.406830in}}%
\pgfpathlineto{\pgfqpoint{1.067905in}{1.402444in}}%
\pgfpathclose%
\pgfusepath{fill}%
\end{pgfscope}%
\begin{pgfscope}%
\pgfpathrectangle{\pgfqpoint{0.329460in}{0.284240in}}{\pgfqpoint{1.989680in}{1.989680in}}%
\pgfusepath{clip}%
\pgfsetbuttcap%
\pgfsetroundjoin%
\definecolor{currentfill}{rgb}{0.277941,0.056324,0.381191}%
\pgfsetfillcolor{currentfill}%
\pgfsetlinewidth{0.000000pt}%
\definecolor{currentstroke}{rgb}{0.000000,0.000000,0.000000}%
\pgfsetstrokecolor{currentstroke}%
\pgfsetdash{}{0pt}%
\pgfpathmoveto{\pgfqpoint{1.916158in}{1.002046in}}%
\pgfpathlineto{\pgfqpoint{1.919650in}{1.005049in}}%
\pgfpathlineto{\pgfqpoint{1.923152in}{1.008339in}}%
\pgfpathlineto{\pgfqpoint{1.926664in}{1.011921in}}%
\pgfpathlineto{\pgfqpoint{1.930188in}{1.015800in}}%
\pgfpathlineto{\pgfqpoint{1.927655in}{1.006245in}}%
\pgfpathlineto{\pgfqpoint{1.924533in}{0.996723in}}%
\pgfpathlineto{\pgfqpoint{1.920824in}{0.987244in}}%
\pgfpathlineto{\pgfqpoint{1.916528in}{0.977818in}}%
\pgfpathlineto{\pgfqpoint{1.913068in}{0.974149in}}%
\pgfpathlineto{\pgfqpoint{1.909618in}{0.970778in}}%
\pgfpathlineto{\pgfqpoint{1.906180in}{0.967701in}}%
\pgfpathlineto{\pgfqpoint{1.902752in}{0.964911in}}%
\pgfpathlineto{\pgfqpoint{1.906962in}{0.974126in}}%
\pgfpathlineto{\pgfqpoint{1.910601in}{0.983394in}}%
\pgfpathlineto{\pgfqpoint{1.913666in}{0.992704in}}%
\pgfpathlineto{\pgfqpoint{1.916158in}{1.002046in}}%
\pgfpathclose%
\pgfusepath{fill}%
\end{pgfscope}%
\begin{pgfscope}%
\pgfpathrectangle{\pgfqpoint{0.329460in}{0.284240in}}{\pgfqpoint{1.989680in}{1.989680in}}%
\pgfusepath{clip}%
\pgfsetbuttcap%
\pgfsetroundjoin%
\definecolor{currentfill}{rgb}{0.220124,0.725509,0.466226}%
\pgfsetfillcolor{currentfill}%
\pgfsetlinewidth{0.000000pt}%
\definecolor{currentstroke}{rgb}{0.000000,0.000000,0.000000}%
\pgfsetstrokecolor{currentstroke}%
\pgfsetdash{}{0pt}%
\pgfpathmoveto{\pgfqpoint{1.496952in}{1.588090in}}%
\pgfpathlineto{\pgfqpoint{1.499151in}{1.583111in}}%
\pgfpathlineto{\pgfqpoint{1.501347in}{1.578082in}}%
\pgfpathlineto{\pgfqpoint{1.503540in}{1.573005in}}%
\pgfpathlineto{\pgfqpoint{1.505730in}{1.567882in}}%
\pgfpathlineto{\pgfqpoint{1.511191in}{1.565503in}}%
\pgfpathlineto{\pgfqpoint{1.516499in}{1.563042in}}%
\pgfpathlineto{\pgfqpoint{1.521649in}{1.560500in}}%
\pgfpathlineto{\pgfqpoint{1.526635in}{1.557881in}}%
\pgfpathlineto{\pgfqpoint{1.524145in}{1.563154in}}%
\pgfpathlineto{\pgfqpoint{1.521651in}{1.568382in}}%
\pgfpathlineto{\pgfqpoint{1.519154in}{1.573561in}}%
\pgfpathlineto{\pgfqpoint{1.516655in}{1.578691in}}%
\pgfpathlineto{\pgfqpoint{1.511956in}{1.581152in}}%
\pgfpathlineto{\pgfqpoint{1.507103in}{1.583541in}}%
\pgfpathlineto{\pgfqpoint{1.502100in}{1.585854in}}%
\pgfpathlineto{\pgfqpoint{1.496952in}{1.588090in}}%
\pgfpathclose%
\pgfusepath{fill}%
\end{pgfscope}%
\begin{pgfscope}%
\pgfpathrectangle{\pgfqpoint{0.329460in}{0.284240in}}{\pgfqpoint{1.989680in}{1.989680in}}%
\pgfusepath{clip}%
\pgfsetbuttcap%
\pgfsetroundjoin%
\definecolor{currentfill}{rgb}{0.195860,0.395433,0.555276}%
\pgfsetfillcolor{currentfill}%
\pgfsetlinewidth{0.000000pt}%
\definecolor{currentstroke}{rgb}{0.000000,0.000000,0.000000}%
\pgfsetstrokecolor{currentstroke}%
\pgfsetdash{}{0pt}%
\pgfpathmoveto{\pgfqpoint{1.008928in}{1.255094in}}%
\pgfpathlineto{\pgfqpoint{1.005623in}{1.248286in}}%
\pgfpathlineto{\pgfqpoint{1.002320in}{1.241514in}}%
\pgfpathlineto{\pgfqpoint{0.999018in}{1.234779in}}%
\pgfpathlineto{\pgfqpoint{0.995718in}{1.228085in}}%
\pgfpathlineto{\pgfqpoint{0.995439in}{1.233768in}}%
\pgfpathlineto{\pgfqpoint{0.995514in}{1.239446in}}%
\pgfpathlineto{\pgfqpoint{0.995944in}{1.245116in}}%
\pgfpathlineto{\pgfqpoint{0.996726in}{1.250771in}}%
\pgfpathlineto{\pgfqpoint{1.000005in}{1.257231in}}%
\pgfpathlineto{\pgfqpoint{1.003285in}{1.263732in}}%
\pgfpathlineto{\pgfqpoint{1.006568in}{1.270271in}}%
\pgfpathlineto{\pgfqpoint{1.009852in}{1.276845in}}%
\pgfpathlineto{\pgfqpoint{1.009110in}{1.271423in}}%
\pgfpathlineto{\pgfqpoint{1.008708in}{1.265986in}}%
\pgfpathlineto{\pgfqpoint{1.008647in}{1.260541in}}%
\pgfpathlineto{\pgfqpoint{1.008928in}{1.255094in}}%
\pgfpathclose%
\pgfusepath{fill}%
\end{pgfscope}%
\begin{pgfscope}%
\pgfpathrectangle{\pgfqpoint{0.329460in}{0.284240in}}{\pgfqpoint{1.989680in}{1.989680in}}%
\pgfusepath{clip}%
\pgfsetbuttcap%
\pgfsetroundjoin%
\definecolor{currentfill}{rgb}{0.344074,0.780029,0.397381}%
\pgfsetfillcolor{currentfill}%
\pgfsetlinewidth{0.000000pt}%
\definecolor{currentstroke}{rgb}{0.000000,0.000000,0.000000}%
\pgfsetstrokecolor{currentstroke}%
\pgfsetdash{}{0pt}%
\pgfpathmoveto{\pgfqpoint{1.399456in}{1.646031in}}%
\pgfpathlineto{\pgfqpoint{1.400290in}{1.641883in}}%
\pgfpathlineto{\pgfqpoint{1.401123in}{1.637670in}}%
\pgfpathlineto{\pgfqpoint{1.401955in}{1.633395in}}%
\pgfpathlineto{\pgfqpoint{1.402786in}{1.629058in}}%
\pgfpathlineto{\pgfqpoint{1.409033in}{1.628239in}}%
\pgfpathlineto{\pgfqpoint{1.415226in}{1.627327in}}%
\pgfpathlineto{\pgfqpoint{1.421359in}{1.626322in}}%
\pgfpathlineto{\pgfqpoint{1.420228in}{1.630705in}}%
\pgfpathlineto{\pgfqpoint{1.419097in}{1.635026in}}%
\pgfpathlineto{\pgfqpoint{1.417964in}{1.639285in}}%
\pgfpathlineto{\pgfqpoint{1.416829in}{1.643478in}}%
\pgfpathlineto{\pgfqpoint{1.411093in}{1.644416in}}%
\pgfpathlineto{\pgfqpoint{1.405301in}{1.645267in}}%
\pgfpathlineto{\pgfqpoint{1.399456in}{1.646031in}}%
\pgfpathclose%
\pgfusepath{fill}%
\end{pgfscope}%
\begin{pgfscope}%
\pgfpathrectangle{\pgfqpoint{0.329460in}{0.284240in}}{\pgfqpoint{1.989680in}{1.989680in}}%
\pgfusepath{clip}%
\pgfsetbuttcap%
\pgfsetroundjoin%
\definecolor{currentfill}{rgb}{0.260571,0.246922,0.522828}%
\pgfsetfillcolor{currentfill}%
\pgfsetlinewidth{0.000000pt}%
\definecolor{currentstroke}{rgb}{0.000000,0.000000,0.000000}%
\pgfsetstrokecolor{currentstroke}%
\pgfsetdash{}{0pt}%
\pgfpathmoveto{\pgfqpoint{1.977770in}{1.128243in}}%
\pgfpathlineto{\pgfqpoint{1.981473in}{1.136675in}}%
\pgfpathlineto{\pgfqpoint{1.985192in}{1.145482in}}%
\pgfpathlineto{\pgfqpoint{1.988927in}{1.154671in}}%
\pgfpathlineto{\pgfqpoint{1.992680in}{1.164247in}}%
\pgfpathlineto{\pgfqpoint{1.992570in}{1.153880in}}%
\pgfpathlineto{\pgfqpoint{1.991817in}{1.143503in}}%
\pgfpathlineto{\pgfqpoint{1.990419in}{1.133127in}}%
\pgfpathlineto{\pgfqpoint{1.988373in}{1.122762in}}%
\pgfpathlineto{\pgfqpoint{1.984621in}{1.113368in}}%
\pgfpathlineto{\pgfqpoint{1.980887in}{1.104364in}}%
\pgfpathlineto{\pgfqpoint{1.977170in}{1.095743in}}%
\pgfpathlineto{\pgfqpoint{1.973469in}{1.087499in}}%
\pgfpathlineto{\pgfqpoint{1.975490in}{1.097678in}}%
\pgfpathlineto{\pgfqpoint{1.976880in}{1.107868in}}%
\pgfpathlineto{\pgfqpoint{1.977639in}{1.118060in}}%
\pgfpathlineto{\pgfqpoint{1.977770in}{1.128243in}}%
\pgfpathclose%
\pgfusepath{fill}%
\end{pgfscope}%
\begin{pgfscope}%
\pgfpathrectangle{\pgfqpoint{0.329460in}{0.284240in}}{\pgfqpoint{1.989680in}{1.989680in}}%
\pgfusepath{clip}%
\pgfsetbuttcap%
\pgfsetroundjoin%
\definecolor{currentfill}{rgb}{0.166383,0.690856,0.496502}%
\pgfsetfillcolor{currentfill}%
\pgfsetlinewidth{0.000000pt}%
\definecolor{currentstroke}{rgb}{0.000000,0.000000,0.000000}%
\pgfsetstrokecolor{currentstroke}%
\pgfsetdash{}{0pt}%
\pgfpathmoveto{\pgfqpoint{1.153874in}{1.544037in}}%
\pgfpathlineto{\pgfqpoint{1.151069in}{1.538511in}}%
\pgfpathlineto{\pgfqpoint{1.148268in}{1.532942in}}%
\pgfpathlineto{\pgfqpoint{1.145471in}{1.527334in}}%
\pgfpathlineto{\pgfqpoint{1.142676in}{1.521687in}}%
\pgfpathlineto{\pgfqpoint{1.147034in}{1.524831in}}%
\pgfpathlineto{\pgfqpoint{1.151589in}{1.527905in}}%
\pgfpathlineto{\pgfqpoint{1.156335in}{1.530907in}}%
\pgfpathlineto{\pgfqpoint{1.161269in}{1.533833in}}%
\pgfpathlineto{\pgfqpoint{1.163809in}{1.539306in}}%
\pgfpathlineto{\pgfqpoint{1.166353in}{1.544741in}}%
\pgfpathlineto{\pgfqpoint{1.168899in}{1.550136in}}%
\pgfpathlineto{\pgfqpoint{1.171448in}{1.555489in}}%
\pgfpathlineto{\pgfqpoint{1.166784in}{1.552730in}}%
\pgfpathlineto{\pgfqpoint{1.162297in}{1.549899in}}%
\pgfpathlineto{\pgfqpoint{1.157992in}{1.547001in}}%
\pgfpathlineto{\pgfqpoint{1.153874in}{1.544037in}}%
\pgfpathclose%
\pgfusepath{fill}%
\end{pgfscope}%
\begin{pgfscope}%
\pgfpathrectangle{\pgfqpoint{0.329460in}{0.284240in}}{\pgfqpoint{1.989680in}{1.989680in}}%
\pgfusepath{clip}%
\pgfsetbuttcap%
\pgfsetroundjoin%
\definecolor{currentfill}{rgb}{0.280255,0.165693,0.476498}%
\pgfsetfillcolor{currentfill}%
\pgfsetlinewidth{0.000000pt}%
\definecolor{currentstroke}{rgb}{0.000000,0.000000,0.000000}%
\pgfsetstrokecolor{currentstroke}%
\pgfsetdash{}{0pt}%
\pgfpathmoveto{\pgfqpoint{1.768959in}{1.085971in}}%
\pgfpathlineto{\pgfqpoint{1.772232in}{1.080637in}}%
\pgfpathlineto{\pgfqpoint{1.775506in}{1.075412in}}%
\pgfpathlineto{\pgfqpoint{1.778781in}{1.070298in}}%
\pgfpathlineto{\pgfqpoint{1.782058in}{1.065300in}}%
\pgfpathlineto{\pgfqpoint{1.780011in}{1.058246in}}%
\pgfpathlineto{\pgfqpoint{1.777529in}{1.051222in}}%
\pgfpathlineto{\pgfqpoint{1.774613in}{1.044235in}}%
\pgfpathlineto{\pgfqpoint{1.771265in}{1.037292in}}%
\pgfpathlineto{\pgfqpoint{1.768056in}{1.042526in}}%
\pgfpathlineto{\pgfqpoint{1.764850in}{1.047875in}}%
\pgfpathlineto{\pgfqpoint{1.761644in}{1.053336in}}%
\pgfpathlineto{\pgfqpoint{1.758440in}{1.058905in}}%
\pgfpathlineto{\pgfqpoint{1.761700in}{1.065614in}}%
\pgfpathlineto{\pgfqpoint{1.764541in}{1.072366in}}%
\pgfpathlineto{\pgfqpoint{1.766961in}{1.079154in}}%
\pgfpathlineto{\pgfqpoint{1.768959in}{1.085971in}}%
\pgfpathclose%
\pgfusepath{fill}%
\end{pgfscope}%
\begin{pgfscope}%
\pgfpathrectangle{\pgfqpoint{0.329460in}{0.284240in}}{\pgfqpoint{1.989680in}{1.989680in}}%
\pgfusepath{clip}%
\pgfsetbuttcap%
\pgfsetroundjoin%
\definecolor{currentfill}{rgb}{0.344074,0.780029,0.397381}%
\pgfsetfillcolor{currentfill}%
\pgfsetlinewidth{0.000000pt}%
\definecolor{currentstroke}{rgb}{0.000000,0.000000,0.000000}%
\pgfsetstrokecolor{currentstroke}%
\pgfsetdash{}{0pt}%
\pgfpathmoveto{\pgfqpoint{1.280500in}{1.642574in}}%
\pgfpathlineto{\pgfqpoint{1.279278in}{1.638364in}}%
\pgfpathlineto{\pgfqpoint{1.278058in}{1.634089in}}%
\pgfpathlineto{\pgfqpoint{1.276839in}{1.629751in}}%
\pgfpathlineto{\pgfqpoint{1.275622in}{1.625352in}}%
\pgfpathlineto{\pgfqpoint{1.281695in}{1.626439in}}%
\pgfpathlineto{\pgfqpoint{1.287834in}{1.627433in}}%
\pgfpathlineto{\pgfqpoint{1.294033in}{1.628335in}}%
\pgfpathlineto{\pgfqpoint{1.300287in}{1.629144in}}%
\pgfpathlineto{\pgfqpoint{1.301106in}{1.633479in}}%
\pgfpathlineto{\pgfqpoint{1.301927in}{1.637753in}}%
\pgfpathlineto{\pgfqpoint{1.302748in}{1.641964in}}%
\pgfpathlineto{\pgfqpoint{1.303571in}{1.646110in}}%
\pgfpathlineto{\pgfqpoint{1.297722in}{1.645356in}}%
\pgfpathlineto{\pgfqpoint{1.291923in}{1.644514in}}%
\pgfpathlineto{\pgfqpoint{1.286181in}{1.643587in}}%
\pgfpathlineto{\pgfqpoint{1.280500in}{1.642574in}}%
\pgfpathclose%
\pgfusepath{fill}%
\end{pgfscope}%
\begin{pgfscope}%
\pgfpathrectangle{\pgfqpoint{0.329460in}{0.284240in}}{\pgfqpoint{1.989680in}{1.989680in}}%
\pgfusepath{clip}%
\pgfsetbuttcap%
\pgfsetroundjoin%
\definecolor{currentfill}{rgb}{0.120081,0.622161,0.534946}%
\pgfsetfillcolor{currentfill}%
\pgfsetlinewidth{0.000000pt}%
\definecolor{currentstroke}{rgb}{0.000000,0.000000,0.000000}%
\pgfsetstrokecolor{currentstroke}%
\pgfsetdash{}{0pt}%
\pgfpathmoveto{\pgfqpoint{1.583774in}{1.487988in}}%
\pgfpathlineto{\pgfqpoint{1.586721in}{1.482038in}}%
\pgfpathlineto{\pgfqpoint{1.589666in}{1.476064in}}%
\pgfpathlineto{\pgfqpoint{1.592607in}{1.470067in}}%
\pgfpathlineto{\pgfqpoint{1.595544in}{1.464050in}}%
\pgfpathlineto{\pgfqpoint{1.599438in}{1.460259in}}%
\pgfpathlineto{\pgfqpoint{1.603091in}{1.456407in}}%
\pgfpathlineto{\pgfqpoint{1.606500in}{1.452499in}}%
\pgfpathlineto{\pgfqpoint{1.609661in}{1.448538in}}%
\pgfpathlineto{\pgfqpoint{1.606545in}{1.454757in}}%
\pgfpathlineto{\pgfqpoint{1.603427in}{1.460956in}}%
\pgfpathlineto{\pgfqpoint{1.600305in}{1.467132in}}%
\pgfpathlineto{\pgfqpoint{1.597180in}{1.473284in}}%
\pgfpathlineto{\pgfqpoint{1.594179in}{1.477039in}}%
\pgfpathlineto{\pgfqpoint{1.590943in}{1.480743in}}%
\pgfpathlineto{\pgfqpoint{1.587473in}{1.484394in}}%
\pgfpathlineto{\pgfqpoint{1.583774in}{1.487988in}}%
\pgfpathclose%
\pgfusepath{fill}%
\end{pgfscope}%
\begin{pgfscope}%
\pgfpathrectangle{\pgfqpoint{0.329460in}{0.284240in}}{\pgfqpoint{1.989680in}{1.989680in}}%
\pgfusepath{clip}%
\pgfsetbuttcap%
\pgfsetroundjoin%
\definecolor{currentfill}{rgb}{0.231674,0.318106,0.544834}%
\pgfsetfillcolor{currentfill}%
\pgfsetlinewidth{0.000000pt}%
\definecolor{currentstroke}{rgb}{0.000000,0.000000,0.000000}%
\pgfsetstrokecolor{currentstroke}%
\pgfsetdash{}{0pt}%
\pgfpathmoveto{\pgfqpoint{1.720108in}{1.207034in}}%
\pgfpathlineto{\pgfqpoint{1.723401in}{1.200639in}}%
\pgfpathlineto{\pgfqpoint{1.726693in}{1.194302in}}%
\pgfpathlineto{\pgfqpoint{1.729984in}{1.188026in}}%
\pgfpathlineto{\pgfqpoint{1.733274in}{1.181814in}}%
\pgfpathlineto{\pgfqpoint{1.732957in}{1.175661in}}%
\pgfpathlineto{\pgfqpoint{1.732257in}{1.169510in}}%
\pgfpathlineto{\pgfqpoint{1.731174in}{1.163367in}}%
\pgfpathlineto{\pgfqpoint{1.729709in}{1.157239in}}%
\pgfpathlineto{\pgfqpoint{1.726437in}{1.163687in}}%
\pgfpathlineto{\pgfqpoint{1.723164in}{1.170199in}}%
\pgfpathlineto{\pgfqpoint{1.719891in}{1.176772in}}%
\pgfpathlineto{\pgfqpoint{1.716617in}{1.183403in}}%
\pgfpathlineto{\pgfqpoint{1.718044in}{1.189295in}}%
\pgfpathlineto{\pgfqpoint{1.719101in}{1.195201in}}%
\pgfpathlineto{\pgfqpoint{1.719789in}{1.201116in}}%
\pgfpathlineto{\pgfqpoint{1.720108in}{1.207034in}}%
\pgfpathclose%
\pgfusepath{fill}%
\end{pgfscope}%
\begin{pgfscope}%
\pgfpathrectangle{\pgfqpoint{0.329460in}{0.284240in}}{\pgfqpoint{1.989680in}{1.989680in}}%
\pgfusepath{clip}%
\pgfsetbuttcap%
\pgfsetroundjoin%
\definecolor{currentfill}{rgb}{0.272594,0.025563,0.353093}%
\pgfsetfillcolor{currentfill}%
\pgfsetlinewidth{0.000000pt}%
\definecolor{currentstroke}{rgb}{0.000000,0.000000,0.000000}%
\pgfsetstrokecolor{currentstroke}%
\pgfsetdash{}{0pt}%
\pgfpathmoveto{\pgfqpoint{0.817370in}{0.948587in}}%
\pgfpathlineto{\pgfqpoint{0.814003in}{0.950225in}}%
\pgfpathlineto{\pgfqpoint{0.810626in}{0.952133in}}%
\pgfpathlineto{\pgfqpoint{0.807240in}{0.954313in}}%
\pgfpathlineto{\pgfqpoint{0.803845in}{0.956772in}}%
\pgfpathlineto{\pgfqpoint{0.799128in}{0.965932in}}%
\pgfpathlineto{\pgfqpoint{0.794981in}{0.975154in}}%
\pgfpathlineto{\pgfqpoint{0.791405in}{0.984426in}}%
\pgfpathlineto{\pgfqpoint{0.788403in}{0.993740in}}%
\pgfpathlineto{\pgfqpoint{0.791876in}{0.991067in}}%
\pgfpathlineto{\pgfqpoint{0.795338in}{0.988671in}}%
\pgfpathlineto{\pgfqpoint{0.798790in}{0.986548in}}%
\pgfpathlineto{\pgfqpoint{0.802234in}{0.984692in}}%
\pgfpathlineto{\pgfqpoint{0.805181in}{0.975594in}}%
\pgfpathlineto{\pgfqpoint{0.808687in}{0.966538in}}%
\pgfpathlineto{\pgfqpoint{0.812751in}{0.957532in}}%
\pgfpathlineto{\pgfqpoint{0.817370in}{0.948587in}}%
\pgfpathclose%
\pgfusepath{fill}%
\end{pgfscope}%
\begin{pgfscope}%
\pgfpathrectangle{\pgfqpoint{0.329460in}{0.284240in}}{\pgfqpoint{1.989680in}{1.989680in}}%
\pgfusepath{clip}%
\pgfsetbuttcap%
\pgfsetroundjoin%
\definecolor{currentfill}{rgb}{0.134692,0.658636,0.517649}%
\pgfsetfillcolor{currentfill}%
\pgfsetlinewidth{0.000000pt}%
\definecolor{currentstroke}{rgb}{0.000000,0.000000,0.000000}%
\pgfsetstrokecolor{currentstroke}%
\pgfsetdash{}{0pt}%
\pgfpathmoveto{\pgfqpoint{1.127298in}{1.508477in}}%
\pgfpathlineto{\pgfqpoint{1.124294in}{1.502606in}}%
\pgfpathlineto{\pgfqpoint{1.121294in}{1.496700in}}%
\pgfpathlineto{\pgfqpoint{1.118296in}{1.490763in}}%
\pgfpathlineto{\pgfqpoint{1.115302in}{1.484797in}}%
\pgfpathlineto{\pgfqpoint{1.119026in}{1.488384in}}%
\pgfpathlineto{\pgfqpoint{1.122975in}{1.491910in}}%
\pgfpathlineto{\pgfqpoint{1.127145in}{1.495372in}}%
\pgfpathlineto{\pgfqpoint{1.131530in}{1.498767in}}%
\pgfpathlineto{\pgfqpoint{1.134312in}{1.504543in}}%
\pgfpathlineto{\pgfqpoint{1.137097in}{1.510290in}}%
\pgfpathlineto{\pgfqpoint{1.139885in}{1.516005in}}%
\pgfpathlineto{\pgfqpoint{1.142676in}{1.521687in}}%
\pgfpathlineto{\pgfqpoint{1.138519in}{1.518477in}}%
\pgfpathlineto{\pgfqpoint{1.134568in}{1.515203in}}%
\pgfpathlineto{\pgfqpoint{1.130826in}{1.511869in}}%
\pgfpathlineto{\pgfqpoint{1.127298in}{1.508477in}}%
\pgfpathclose%
\pgfusepath{fill}%
\end{pgfscope}%
\begin{pgfscope}%
\pgfpathrectangle{\pgfqpoint{0.329460in}{0.284240in}}{\pgfqpoint{1.989680in}{1.989680in}}%
\pgfusepath{clip}%
\pgfsetbuttcap%
\pgfsetroundjoin%
\definecolor{currentfill}{rgb}{0.283072,0.130895,0.449241}%
\pgfsetfillcolor{currentfill}%
\pgfsetlinewidth{0.000000pt}%
\definecolor{currentstroke}{rgb}{0.000000,0.000000,0.000000}%
\pgfsetstrokecolor{currentstroke}%
\pgfsetdash{}{0pt}%
\pgfpathmoveto{\pgfqpoint{0.934449in}{1.031163in}}%
\pgfpathlineto{\pgfqpoint{0.931262in}{1.025996in}}%
\pgfpathlineto{\pgfqpoint{0.928073in}{1.020951in}}%
\pgfpathlineto{\pgfqpoint{0.924882in}{1.016031in}}%
\pgfpathlineto{\pgfqpoint{0.921689in}{1.011240in}}%
\pgfpathlineto{\pgfqpoint{0.917857in}{1.018371in}}%
\pgfpathlineto{\pgfqpoint{0.914470in}{1.025554in}}%
\pgfpathlineto{\pgfqpoint{0.911529in}{1.032780in}}%
\pgfpathlineto{\pgfqpoint{0.909036in}{1.040043in}}%
\pgfpathlineto{\pgfqpoint{0.912308in}{1.044600in}}%
\pgfpathlineto{\pgfqpoint{0.915579in}{1.049285in}}%
\pgfpathlineto{\pgfqpoint{0.918848in}{1.054096in}}%
\pgfpathlineto{\pgfqpoint{0.922115in}{1.059029in}}%
\pgfpathlineto{\pgfqpoint{0.924548in}{1.052001in}}%
\pgfpathlineto{\pgfqpoint{0.927416in}{1.045010in}}%
\pgfpathlineto{\pgfqpoint{0.930717in}{1.038061in}}%
\pgfpathlineto{\pgfqpoint{0.934449in}{1.031163in}}%
\pgfpathclose%
\pgfusepath{fill}%
\end{pgfscope}%
\begin{pgfscope}%
\pgfpathrectangle{\pgfqpoint{0.329460in}{0.284240in}}{\pgfqpoint{1.989680in}{1.989680in}}%
\pgfusepath{clip}%
\pgfsetbuttcap%
\pgfsetroundjoin%
\definecolor{currentfill}{rgb}{0.220124,0.725509,0.466226}%
\pgfsetfillcolor{currentfill}%
\pgfsetlinewidth{0.000000pt}%
\definecolor{currentstroke}{rgb}{0.000000,0.000000,0.000000}%
\pgfsetstrokecolor{currentstroke}%
\pgfsetdash{}{0pt}%
\pgfpathmoveto{\pgfqpoint{1.181677in}{1.576443in}}%
\pgfpathlineto{\pgfqpoint{1.179116in}{1.571277in}}%
\pgfpathlineto{\pgfqpoint{1.176557in}{1.566062in}}%
\pgfpathlineto{\pgfqpoint{1.174001in}{1.560799in}}%
\pgfpathlineto{\pgfqpoint{1.171448in}{1.555489in}}%
\pgfpathlineto{\pgfqpoint{1.176286in}{1.558176in}}%
\pgfpathlineto{\pgfqpoint{1.181290in}{1.560787in}}%
\pgfpathlineto{\pgfqpoint{1.186458in}{1.563319in}}%
\pgfpathlineto{\pgfqpoint{1.191783in}{1.565771in}}%
\pgfpathlineto{\pgfqpoint{1.194043in}{1.570926in}}%
\pgfpathlineto{\pgfqpoint{1.196306in}{1.576035in}}%
\pgfpathlineto{\pgfqpoint{1.198572in}{1.581096in}}%
\pgfpathlineto{\pgfqpoint{1.200840in}{1.586107in}}%
\pgfpathlineto{\pgfqpoint{1.195821in}{1.583802in}}%
\pgfpathlineto{\pgfqpoint{1.190951in}{1.581421in}}%
\pgfpathlineto{\pgfqpoint{1.186235in}{1.578968in}}%
\pgfpathlineto{\pgfqpoint{1.181677in}{1.576443in}}%
\pgfpathclose%
\pgfusepath{fill}%
\end{pgfscope}%
\begin{pgfscope}%
\pgfpathrectangle{\pgfqpoint{0.329460in}{0.284240in}}{\pgfqpoint{1.989680in}{1.989680in}}%
\pgfusepath{clip}%
\pgfsetbuttcap%
\pgfsetroundjoin%
\definecolor{currentfill}{rgb}{0.147607,0.511733,0.557049}%
\pgfsetfillcolor{currentfill}%
\pgfsetlinewidth{0.000000pt}%
\definecolor{currentstroke}{rgb}{0.000000,0.000000,0.000000}%
\pgfsetstrokecolor{currentstroke}%
\pgfsetdash{}{0pt}%
\pgfpathmoveto{\pgfqpoint{1.645611in}{1.380313in}}%
\pgfpathlineto{\pgfqpoint{1.648830in}{1.373801in}}%
\pgfpathlineto{\pgfqpoint{1.652046in}{1.367293in}}%
\pgfpathlineto{\pgfqpoint{1.655260in}{1.360792in}}%
\pgfpathlineto{\pgfqpoint{1.658471in}{1.354298in}}%
\pgfpathlineto{\pgfqpoint{1.660642in}{1.349466in}}%
\pgfpathlineto{\pgfqpoint{1.662509in}{1.344598in}}%
\pgfpathlineto{\pgfqpoint{1.664069in}{1.339699in}}%
\pgfpathlineto{\pgfqpoint{1.665321in}{1.334774in}}%
\pgfpathlineto{\pgfqpoint{1.662028in}{1.341492in}}%
\pgfpathlineto{\pgfqpoint{1.658733in}{1.348219in}}%
\pgfpathlineto{\pgfqpoint{1.655435in}{1.354951in}}%
\pgfpathlineto{\pgfqpoint{1.652134in}{1.361687in}}%
\pgfpathlineto{\pgfqpoint{1.650945in}{1.366385in}}%
\pgfpathlineto{\pgfqpoint{1.649460in}{1.371058in}}%
\pgfpathlineto{\pgfqpoint{1.647681in}{1.375702in}}%
\pgfpathlineto{\pgfqpoint{1.645611in}{1.380313in}}%
\pgfpathclose%
\pgfusepath{fill}%
\end{pgfscope}%
\begin{pgfscope}%
\pgfpathrectangle{\pgfqpoint{0.329460in}{0.284240in}}{\pgfqpoint{1.989680in}{1.989680in}}%
\pgfusepath{clip}%
\pgfsetbuttcap%
\pgfsetroundjoin%
\definecolor{currentfill}{rgb}{0.281477,0.755203,0.432552}%
\pgfsetfillcolor{currentfill}%
\pgfsetlinewidth{0.000000pt}%
\definecolor{currentstroke}{rgb}{0.000000,0.000000,0.000000}%
\pgfsetstrokecolor{currentstroke}%
\pgfsetdash{}{0pt}%
\pgfpathmoveto{\pgfqpoint{1.467535in}{1.615084in}}%
\pgfpathlineto{\pgfqpoint{1.469410in}{1.610453in}}%
\pgfpathlineto{\pgfqpoint{1.471282in}{1.605763in}}%
\pgfpathlineto{\pgfqpoint{1.473152in}{1.601019in}}%
\pgfpathlineto{\pgfqpoint{1.475020in}{1.596221in}}%
\pgfpathlineto{\pgfqpoint{1.480694in}{1.594314in}}%
\pgfpathlineto{\pgfqpoint{1.486244in}{1.592322in}}%
\pgfpathlineto{\pgfqpoint{1.491665in}{1.590247in}}%
\pgfpathlineto{\pgfqpoint{1.496952in}{1.588090in}}%
\pgfpathlineto{\pgfqpoint{1.494751in}{1.593018in}}%
\pgfpathlineto{\pgfqpoint{1.492547in}{1.597892in}}%
\pgfpathlineto{\pgfqpoint{1.490340in}{1.602710in}}%
\pgfpathlineto{\pgfqpoint{1.488130in}{1.607470in}}%
\pgfpathlineto{\pgfqpoint{1.483166in}{1.609490in}}%
\pgfpathlineto{\pgfqpoint{1.478076in}{1.611433in}}%
\pgfpathlineto{\pgfqpoint{1.472863in}{1.613298in}}%
\pgfpathlineto{\pgfqpoint{1.467535in}{1.615084in}}%
\pgfpathclose%
\pgfusepath{fill}%
\end{pgfscope}%
\begin{pgfscope}%
\pgfpathrectangle{\pgfqpoint{0.329460in}{0.284240in}}{\pgfqpoint{1.989680in}{1.989680in}}%
\pgfusepath{clip}%
\pgfsetbuttcap%
\pgfsetroundjoin%
\definecolor{currentfill}{rgb}{0.172719,0.448791,0.557885}%
\pgfsetfillcolor{currentfill}%
\pgfsetlinewidth{0.000000pt}%
\definecolor{currentstroke}{rgb}{0.000000,0.000000,0.000000}%
\pgfsetstrokecolor{currentstroke}%
\pgfsetdash{}{0pt}%
\pgfpathmoveto{\pgfqpoint{2.016952in}{1.298116in}}%
\pgfpathlineto{\pgfqpoint{2.020814in}{1.311642in}}%
\pgfpathlineto{\pgfqpoint{2.024697in}{1.325624in}}%
\pgfpathlineto{\pgfqpoint{2.028602in}{1.340067in}}%
\pgfpathlineto{\pgfqpoint{2.031261in}{1.329401in}}%
\pgfpathlineto{\pgfqpoint{2.033252in}{1.318679in}}%
\pgfpathlineto{\pgfqpoint{2.034569in}{1.307913in}}%
\pgfpathlineto{\pgfqpoint{2.035209in}{1.297113in}}%
\pgfpathlineto{\pgfqpoint{2.031238in}{1.282818in}}%
\pgfpathlineto{\pgfqpoint{2.027291in}{1.268989in}}%
\pgfpathlineto{\pgfqpoint{2.023365in}{1.255618in}}%
\pgfpathlineto{\pgfqpoint{2.022757in}{1.266303in}}%
\pgfpathlineto{\pgfqpoint{2.021482in}{1.276954in}}%
\pgfpathlineto{\pgfqpoint{2.019545in}{1.287562in}}%
\pgfpathlineto{\pgfqpoint{2.016952in}{1.298116in}}%
\pgfpathclose%
\pgfusepath{fill}%
\end{pgfscope}%
\begin{pgfscope}%
\pgfpathrectangle{\pgfqpoint{0.329460in}{0.284240in}}{\pgfqpoint{1.989680in}{1.989680in}}%
\pgfusepath{clip}%
\pgfsetbuttcap%
\pgfsetroundjoin%
\definecolor{currentfill}{rgb}{0.344074,0.780029,0.397381}%
\pgfsetfillcolor{currentfill}%
\pgfsetlinewidth{0.000000pt}%
\definecolor{currentstroke}{rgb}{0.000000,0.000000,0.000000}%
\pgfsetstrokecolor{currentstroke}%
\pgfsetdash{}{0pt}%
\pgfpathmoveto{\pgfqpoint{1.416829in}{1.643478in}}%
\pgfpathlineto{\pgfqpoint{1.417964in}{1.639285in}}%
\pgfpathlineto{\pgfqpoint{1.419097in}{1.635026in}}%
\pgfpathlineto{\pgfqpoint{1.420228in}{1.630705in}}%
\pgfpathlineto{\pgfqpoint{1.421359in}{1.626322in}}%
\pgfpathlineto{\pgfqpoint{1.427424in}{1.625226in}}%
\pgfpathlineto{\pgfqpoint{1.433417in}{1.624039in}}%
\pgfpathlineto{\pgfqpoint{1.439332in}{1.622763in}}%
\pgfpathlineto{\pgfqpoint{1.445162in}{1.621399in}}%
\pgfpathlineto{\pgfqpoint{1.443648in}{1.625864in}}%
\pgfpathlineto{\pgfqpoint{1.442131in}{1.630269in}}%
\pgfpathlineto{\pgfqpoint{1.440612in}{1.634610in}}%
\pgfpathlineto{\pgfqpoint{1.439091in}{1.638886in}}%
\pgfpathlineto{\pgfqpoint{1.433638in}{1.640159in}}%
\pgfpathlineto{\pgfqpoint{1.428107in}{1.641349in}}%
\pgfpathlineto{\pgfqpoint{1.422502in}{1.642456in}}%
\pgfpathlineto{\pgfqpoint{1.416829in}{1.643478in}}%
\pgfpathclose%
\pgfusepath{fill}%
\end{pgfscope}%
\begin{pgfscope}%
\pgfpathrectangle{\pgfqpoint{0.329460in}{0.284240in}}{\pgfqpoint{1.989680in}{1.989680in}}%
\pgfusepath{clip}%
\pgfsetbuttcap%
\pgfsetroundjoin%
\definecolor{currentfill}{rgb}{0.179019,0.433756,0.557430}%
\pgfsetfillcolor{currentfill}%
\pgfsetlinewidth{0.000000pt}%
\definecolor{currentstroke}{rgb}{0.000000,0.000000,0.000000}%
\pgfsetstrokecolor{currentstroke}%
\pgfsetdash{}{0pt}%
\pgfpathmoveto{\pgfqpoint{1.678468in}{1.308039in}}%
\pgfpathlineto{\pgfqpoint{1.681750in}{1.301402in}}%
\pgfpathlineto{\pgfqpoint{1.685028in}{1.294790in}}%
\pgfpathlineto{\pgfqpoint{1.688305in}{1.288205in}}%
\pgfpathlineto{\pgfqpoint{1.691580in}{1.281649in}}%
\pgfpathlineto{\pgfqpoint{1.692622in}{1.276243in}}%
\pgfpathlineto{\pgfqpoint{1.693327in}{1.270819in}}%
\pgfpathlineto{\pgfqpoint{1.693691in}{1.265381in}}%
\pgfpathlineto{\pgfqpoint{1.693714in}{1.259936in}}%
\pgfpathlineto{\pgfqpoint{1.690407in}{1.266724in}}%
\pgfpathlineto{\pgfqpoint{1.687099in}{1.273541in}}%
\pgfpathlineto{\pgfqpoint{1.683789in}{1.280385in}}%
\pgfpathlineto{\pgfqpoint{1.680476in}{1.287253in}}%
\pgfpathlineto{\pgfqpoint{1.680465in}{1.292465in}}%
\pgfpathlineto{\pgfqpoint{1.680126in}{1.297670in}}%
\pgfpathlineto{\pgfqpoint{1.679460in}{1.302863in}}%
\pgfpathlineto{\pgfqpoint{1.678468in}{1.308039in}}%
\pgfpathclose%
\pgfusepath{fill}%
\end{pgfscope}%
\begin{pgfscope}%
\pgfpathrectangle{\pgfqpoint{0.329460in}{0.284240in}}{\pgfqpoint{1.989680in}{1.989680in}}%
\pgfusepath{clip}%
\pgfsetbuttcap%
\pgfsetroundjoin%
\definecolor{currentfill}{rgb}{0.120081,0.622161,0.534946}%
\pgfsetfillcolor{currentfill}%
\pgfsetlinewidth{0.000000pt}%
\definecolor{currentstroke}{rgb}{0.000000,0.000000,0.000000}%
\pgfsetstrokecolor{currentstroke}%
\pgfsetdash{}{0pt}%
\pgfpathmoveto{\pgfqpoint{1.102730in}{1.469907in}}%
\pgfpathlineto{\pgfqpoint{1.099572in}{1.463709in}}%
\pgfpathlineto{\pgfqpoint{1.096417in}{1.457486in}}%
\pgfpathlineto{\pgfqpoint{1.093265in}{1.451241in}}%
\pgfpathlineto{\pgfqpoint{1.090116in}{1.444975in}}%
\pgfpathlineto{\pgfqpoint{1.093053in}{1.448980in}}%
\pgfpathlineto{\pgfqpoint{1.096242in}{1.452936in}}%
\pgfpathlineto{\pgfqpoint{1.099678in}{1.456838in}}%
\pgfpathlineto{\pgfqpoint{1.103358in}{1.460683in}}%
\pgfpathlineto{\pgfqpoint{1.106339in}{1.466744in}}%
\pgfpathlineto{\pgfqpoint{1.109324in}{1.472785in}}%
\pgfpathlineto{\pgfqpoint{1.112311in}{1.478803in}}%
\pgfpathlineto{\pgfqpoint{1.115302in}{1.484797in}}%
\pgfpathlineto{\pgfqpoint{1.111807in}{1.481152in}}%
\pgfpathlineto{\pgfqpoint{1.108544in}{1.477453in}}%
\pgfpathlineto{\pgfqpoint{1.105517in}{1.473703in}}%
\pgfpathlineto{\pgfqpoint{1.102730in}{1.469907in}}%
\pgfpathclose%
\pgfusepath{fill}%
\end{pgfscope}%
\begin{pgfscope}%
\pgfpathrectangle{\pgfqpoint{0.329460in}{0.284240in}}{\pgfqpoint{1.989680in}{1.989680in}}%
\pgfusepath{clip}%
\pgfsetbuttcap%
\pgfsetroundjoin%
\definecolor{currentfill}{rgb}{0.344074,0.780029,0.397381}%
\pgfsetfillcolor{currentfill}%
\pgfsetlinewidth{0.000000pt}%
\definecolor{currentstroke}{rgb}{0.000000,0.000000,0.000000}%
\pgfsetstrokecolor{currentstroke}%
\pgfsetdash{}{0pt}%
\pgfpathmoveto{\pgfqpoint{1.258507in}{1.637687in}}%
\pgfpathlineto{\pgfqpoint{1.256904in}{1.633389in}}%
\pgfpathlineto{\pgfqpoint{1.255302in}{1.629026in}}%
\pgfpathlineto{\pgfqpoint{1.253702in}{1.624600in}}%
\pgfpathlineto{\pgfqpoint{1.252105in}{1.620113in}}%
\pgfpathlineto{\pgfqpoint{1.257856in}{1.621555in}}%
\pgfpathlineto{\pgfqpoint{1.263697in}{1.622909in}}%
\pgfpathlineto{\pgfqpoint{1.269620in}{1.624176in}}%
\pgfpathlineto{\pgfqpoint{1.275622in}{1.625352in}}%
\pgfpathlineto{\pgfqpoint{1.276839in}{1.629751in}}%
\pgfpathlineto{\pgfqpoint{1.278058in}{1.634089in}}%
\pgfpathlineto{\pgfqpoint{1.279278in}{1.638364in}}%
\pgfpathlineto{\pgfqpoint{1.280500in}{1.642574in}}%
\pgfpathlineto{\pgfqpoint{1.274888in}{1.641476in}}%
\pgfpathlineto{\pgfqpoint{1.269348in}{1.640295in}}%
\pgfpathlineto{\pgfqpoint{1.263886in}{1.639032in}}%
\pgfpathlineto{\pgfqpoint{1.258507in}{1.637687in}}%
\pgfpathclose%
\pgfusepath{fill}%
\end{pgfscope}%
\begin{pgfscope}%
\pgfpathrectangle{\pgfqpoint{0.329460in}{0.284240in}}{\pgfqpoint{1.989680in}{1.989680in}}%
\pgfusepath{clip}%
\pgfsetbuttcap%
\pgfsetroundjoin%
\definecolor{currentfill}{rgb}{0.281477,0.755203,0.432552}%
\pgfsetfillcolor{currentfill}%
\pgfsetlinewidth{0.000000pt}%
\definecolor{currentstroke}{rgb}{0.000000,0.000000,0.000000}%
\pgfsetstrokecolor{currentstroke}%
\pgfsetdash{}{0pt}%
\pgfpathmoveto{\pgfqpoint{1.209943in}{1.605613in}}%
\pgfpathlineto{\pgfqpoint{1.207663in}{1.600821in}}%
\pgfpathlineto{\pgfqpoint{1.205386in}{1.595971in}}%
\pgfpathlineto{\pgfqpoint{1.203112in}{1.591066in}}%
\pgfpathlineto{\pgfqpoint{1.200840in}{1.586107in}}%
\pgfpathlineto{\pgfqpoint{1.206003in}{1.588334in}}%
\pgfpathlineto{\pgfqpoint{1.211306in}{1.590481in}}%
\pgfpathlineto{\pgfqpoint{1.216741in}{1.592547in}}%
\pgfpathlineto{\pgfqpoint{1.222306in}{1.594530in}}%
\pgfpathlineto{\pgfqpoint{1.224250in}{1.599355in}}%
\pgfpathlineto{\pgfqpoint{1.226197in}{1.604127in}}%
\pgfpathlineto{\pgfqpoint{1.228146in}{1.608842in}}%
\pgfpathlineto{\pgfqpoint{1.230098in}{1.613501in}}%
\pgfpathlineto{\pgfqpoint{1.224873in}{1.611644in}}%
\pgfpathlineto{\pgfqpoint{1.219769in}{1.609709in}}%
\pgfpathlineto{\pgfqpoint{1.214790in}{1.607698in}}%
\pgfpathlineto{\pgfqpoint{1.209943in}{1.605613in}}%
\pgfpathclose%
\pgfusepath{fill}%
\end{pgfscope}%
\begin{pgfscope}%
\pgfpathrectangle{\pgfqpoint{0.329460in}{0.284240in}}{\pgfqpoint{1.989680in}{1.989680in}}%
\pgfusepath{clip}%
\pgfsetbuttcap%
\pgfsetroundjoin%
\definecolor{currentfill}{rgb}{0.274128,0.199721,0.498911}%
\pgfsetfillcolor{currentfill}%
\pgfsetlinewidth{0.000000pt}%
\definecolor{currentstroke}{rgb}{0.000000,0.000000,0.000000}%
\pgfsetstrokecolor{currentstroke}%
\pgfsetdash{}{0pt}%
\pgfpathmoveto{\pgfqpoint{1.755873in}{1.108323in}}%
\pgfpathlineto{\pgfqpoint{1.759144in}{1.102589in}}%
\pgfpathlineto{\pgfqpoint{1.762415in}{1.096950in}}%
\pgfpathlineto{\pgfqpoint{1.765686in}{1.091410in}}%
\pgfpathlineto{\pgfqpoint{1.768959in}{1.085971in}}%
\pgfpathlineto{\pgfqpoint{1.766961in}{1.079154in}}%
\pgfpathlineto{\pgfqpoint{1.764541in}{1.072366in}}%
\pgfpathlineto{\pgfqpoint{1.761700in}{1.065614in}}%
\pgfpathlineto{\pgfqpoint{1.758440in}{1.058905in}}%
\pgfpathlineto{\pgfqpoint{1.755237in}{1.064580in}}%
\pgfpathlineto{\pgfqpoint{1.752034in}{1.070356in}}%
\pgfpathlineto{\pgfqpoint{1.748833in}{1.076230in}}%
\pgfpathlineto{\pgfqpoint{1.745632in}{1.082199in}}%
\pgfpathlineto{\pgfqpoint{1.748802in}{1.088674in}}%
\pgfpathlineto{\pgfqpoint{1.751567in}{1.095191in}}%
\pgfpathlineto{\pgfqpoint{1.753924in}{1.101743in}}%
\pgfpathlineto{\pgfqpoint{1.755873in}{1.108323in}}%
\pgfpathclose%
\pgfusepath{fill}%
\end{pgfscope}%
\begin{pgfscope}%
\pgfpathrectangle{\pgfqpoint{0.329460in}{0.284240in}}{\pgfqpoint{1.989680in}{1.989680in}}%
\pgfusepath{clip}%
\pgfsetbuttcap%
\pgfsetroundjoin%
\definecolor{currentfill}{rgb}{0.122606,0.585371,0.546557}%
\pgfsetfillcolor{currentfill}%
\pgfsetlinewidth{0.000000pt}%
\definecolor{currentstroke}{rgb}{0.000000,0.000000,0.000000}%
\pgfsetstrokecolor{currentstroke}%
\pgfsetdash{}{0pt}%
\pgfpathmoveto{\pgfqpoint{1.609661in}{1.448538in}}%
\pgfpathlineto{\pgfqpoint{1.612773in}{1.442300in}}%
\pgfpathlineto{\pgfqpoint{1.615882in}{1.436047in}}%
\pgfpathlineto{\pgfqpoint{1.618988in}{1.429780in}}%
\pgfpathlineto{\pgfqpoint{1.622091in}{1.423503in}}%
\pgfpathlineto{\pgfqpoint{1.625149in}{1.419282in}}%
\pgfpathlineto{\pgfqpoint{1.627939in}{1.415012in}}%
\pgfpathlineto{\pgfqpoint{1.630459in}{1.410698in}}%
\pgfpathlineto{\pgfqpoint{1.632705in}{1.406345in}}%
\pgfpathlineto{\pgfqpoint{1.629472in}{1.412837in}}%
\pgfpathlineto{\pgfqpoint{1.626235in}{1.419317in}}%
\pgfpathlineto{\pgfqpoint{1.622996in}{1.425785in}}%
\pgfpathlineto{\pgfqpoint{1.619753in}{1.432236in}}%
\pgfpathlineto{\pgfqpoint{1.617619in}{1.436372in}}%
\pgfpathlineto{\pgfqpoint{1.615223in}{1.440470in}}%
\pgfpathlineto{\pgfqpoint{1.612569in}{1.444527in}}%
\pgfpathlineto{\pgfqpoint{1.609661in}{1.448538in}}%
\pgfpathclose%
\pgfusepath{fill}%
\end{pgfscope}%
\begin{pgfscope}%
\pgfpathrectangle{\pgfqpoint{0.329460in}{0.284240in}}{\pgfqpoint{1.989680in}{1.989680in}}%
\pgfusepath{clip}%
\pgfsetbuttcap%
\pgfsetroundjoin%
\definecolor{currentfill}{rgb}{0.280255,0.165693,0.476498}%
\pgfsetfillcolor{currentfill}%
\pgfsetlinewidth{0.000000pt}%
\definecolor{currentstroke}{rgb}{0.000000,0.000000,0.000000}%
\pgfsetstrokecolor{currentstroke}%
\pgfsetdash{}{0pt}%
\pgfpathmoveto{\pgfqpoint{0.947183in}{1.052984in}}%
\pgfpathlineto{\pgfqpoint{0.944001in}{1.047363in}}%
\pgfpathlineto{\pgfqpoint{0.940819in}{1.041850in}}%
\pgfpathlineto{\pgfqpoint{0.937634in}{1.036449in}}%
\pgfpathlineto{\pgfqpoint{0.934449in}{1.031163in}}%
\pgfpathlineto{\pgfqpoint{0.930717in}{1.038061in}}%
\pgfpathlineto{\pgfqpoint{0.927416in}{1.045010in}}%
\pgfpathlineto{\pgfqpoint{0.924548in}{1.052001in}}%
\pgfpathlineto{\pgfqpoint{0.922115in}{1.059029in}}%
\pgfpathlineto{\pgfqpoint{0.925381in}{1.064080in}}%
\pgfpathlineto{\pgfqpoint{0.928646in}{1.069246in}}%
\pgfpathlineto{\pgfqpoint{0.931909in}{1.074524in}}%
\pgfpathlineto{\pgfqpoint{0.935172in}{1.079910in}}%
\pgfpathlineto{\pgfqpoint{0.937545in}{1.073119in}}%
\pgfpathlineto{\pgfqpoint{0.940339in}{1.066363in}}%
\pgfpathlineto{\pgfqpoint{0.943552in}{1.059648in}}%
\pgfpathlineto{\pgfqpoint{0.947183in}{1.052984in}}%
\pgfpathclose%
\pgfusepath{fill}%
\end{pgfscope}%
\begin{pgfscope}%
\pgfpathrectangle{\pgfqpoint{0.329460in}{0.284240in}}{\pgfqpoint{1.989680in}{1.989680in}}%
\pgfusepath{clip}%
\pgfsetbuttcap%
\pgfsetroundjoin%
\definecolor{currentfill}{rgb}{0.231674,0.318106,0.544834}%
\pgfsetfillcolor{currentfill}%
\pgfsetlinewidth{0.000000pt}%
\definecolor{currentstroke}{rgb}{0.000000,0.000000,0.000000}%
\pgfsetstrokecolor{currentstroke}%
\pgfsetdash{}{0pt}%
\pgfpathmoveto{\pgfqpoint{0.987337in}{1.178182in}}%
\pgfpathlineto{\pgfqpoint{0.984074in}{1.171499in}}%
\pgfpathlineto{\pgfqpoint{0.980812in}{1.164874in}}%
\pgfpathlineto{\pgfqpoint{0.977550in}{1.158309in}}%
\pgfpathlineto{\pgfqpoint{0.974290in}{1.151809in}}%
\pgfpathlineto{\pgfqpoint{0.972484in}{1.157919in}}%
\pgfpathlineto{\pgfqpoint{0.971062in}{1.164049in}}%
\pgfpathlineto{\pgfqpoint{0.970021in}{1.170193in}}%
\pgfpathlineto{\pgfqpoint{0.969364in}{1.176345in}}%
\pgfpathlineto{\pgfqpoint{0.972655in}{1.182609in}}%
\pgfpathlineto{\pgfqpoint{0.975946in}{1.188937in}}%
\pgfpathlineto{\pgfqpoint{0.979239in}{1.195327in}}%
\pgfpathlineto{\pgfqpoint{0.982532in}{1.201774in}}%
\pgfpathlineto{\pgfqpoint{0.983179in}{1.195858in}}%
\pgfpathlineto{\pgfqpoint{0.984196in}{1.189950in}}%
\pgfpathlineto{\pgfqpoint{0.985582in}{1.184056in}}%
\pgfpathlineto{\pgfqpoint{0.987337in}{1.178182in}}%
\pgfpathclose%
\pgfusepath{fill}%
\end{pgfscope}%
\begin{pgfscope}%
\pgfpathrectangle{\pgfqpoint{0.329460in}{0.284240in}}{\pgfqpoint{1.989680in}{1.989680in}}%
\pgfusepath{clip}%
\pgfsetbuttcap%
\pgfsetroundjoin%
\definecolor{currentfill}{rgb}{0.282327,0.094955,0.417331}%
\pgfsetfillcolor{currentfill}%
\pgfsetlinewidth{0.000000pt}%
\definecolor{currentstroke}{rgb}{0.000000,0.000000,0.000000}%
\pgfsetstrokecolor{currentstroke}%
\pgfsetdash{}{0pt}%
\pgfpathmoveto{\pgfqpoint{1.930188in}{1.015800in}}%
\pgfpathlineto{\pgfqpoint{1.933723in}{1.019981in}}%
\pgfpathlineto{\pgfqpoint{1.937270in}{1.024470in}}%
\pgfpathlineto{\pgfqpoint{1.940829in}{1.029272in}}%
\pgfpathlineto{\pgfqpoint{1.944401in}{1.034393in}}%
\pgfpathlineto{\pgfqpoint{1.941827in}{1.024631in}}%
\pgfpathlineto{\pgfqpoint{1.938650in}{1.014903in}}%
\pgfpathlineto{\pgfqpoint{1.934870in}{1.005217in}}%
\pgfpathlineto{\pgfqpoint{1.930489in}{0.995584in}}%
\pgfpathlineto{\pgfqpoint{1.926981in}{0.990668in}}%
\pgfpathlineto{\pgfqpoint{1.923484in}{0.986072in}}%
\pgfpathlineto{\pgfqpoint{1.920000in}{0.981790in}}%
\pgfpathlineto{\pgfqpoint{1.916528in}{0.977818in}}%
\pgfpathlineto{\pgfqpoint{1.920824in}{0.987244in}}%
\pgfpathlineto{\pgfqpoint{1.924533in}{0.996723in}}%
\pgfpathlineto{\pgfqpoint{1.927655in}{1.006245in}}%
\pgfpathlineto{\pgfqpoint{1.930188in}{1.015800in}}%
\pgfpathclose%
\pgfusepath{fill}%
\end{pgfscope}%
\begin{pgfscope}%
\pgfpathrectangle{\pgfqpoint{0.329460in}{0.284240in}}{\pgfqpoint{1.989680in}{1.989680in}}%
\pgfusepath{clip}%
\pgfsetbuttcap%
\pgfsetroundjoin%
\definecolor{currentfill}{rgb}{0.277941,0.056324,0.381191}%
\pgfsetfillcolor{currentfill}%
\pgfsetlinewidth{0.000000pt}%
\definecolor{currentstroke}{rgb}{0.000000,0.000000,0.000000}%
\pgfsetstrokecolor{currentstroke}%
\pgfsetdash{}{0pt}%
\pgfpathmoveto{\pgfqpoint{0.803845in}{0.956772in}}%
\pgfpathlineto{\pgfqpoint{0.800439in}{0.959514in}}%
\pgfpathlineto{\pgfqpoint{0.797022in}{0.962544in}}%
\pgfpathlineto{\pgfqpoint{0.793595in}{0.965868in}}%
\pgfpathlineto{\pgfqpoint{0.790156in}{0.969491in}}%
\pgfpathlineto{\pgfqpoint{0.785341in}{0.978862in}}%
\pgfpathlineto{\pgfqpoint{0.781110in}{0.988295in}}%
\pgfpathlineto{\pgfqpoint{0.777466in}{0.997779in}}%
\pgfpathlineto{\pgfqpoint{0.774410in}{1.007305in}}%
\pgfpathlineto{\pgfqpoint{0.777925in}{1.003473in}}%
\pgfpathlineto{\pgfqpoint{0.781428in}{0.999938in}}%
\pgfpathlineto{\pgfqpoint{0.784921in}{0.996695in}}%
\pgfpathlineto{\pgfqpoint{0.788403in}{0.993740in}}%
\pgfpathlineto{\pgfqpoint{0.791405in}{0.984426in}}%
\pgfpathlineto{\pgfqpoint{0.794981in}{0.975154in}}%
\pgfpathlineto{\pgfqpoint{0.799128in}{0.965932in}}%
\pgfpathlineto{\pgfqpoint{0.803845in}{0.956772in}}%
\pgfpathclose%
\pgfusepath{fill}%
\end{pgfscope}%
\begin{pgfscope}%
\pgfpathrectangle{\pgfqpoint{0.329460in}{0.284240in}}{\pgfqpoint{1.989680in}{1.989680in}}%
\pgfusepath{clip}%
\pgfsetbuttcap%
\pgfsetroundjoin%
\definecolor{currentfill}{rgb}{0.147607,0.511733,0.557049}%
\pgfsetfillcolor{currentfill}%
\pgfsetlinewidth{0.000000pt}%
\definecolor{currentstroke}{rgb}{0.000000,0.000000,0.000000}%
\pgfsetstrokecolor{currentstroke}%
\pgfsetdash{}{0pt}%
\pgfpathmoveto{\pgfqpoint{1.049434in}{1.357494in}}%
\pgfpathlineto{\pgfqpoint{1.046122in}{1.350708in}}%
\pgfpathlineto{\pgfqpoint{1.042813in}{1.343925in}}%
\pgfpathlineto{\pgfqpoint{1.039506in}{1.337148in}}%
\pgfpathlineto{\pgfqpoint{1.036202in}{1.330379in}}%
\pgfpathlineto{\pgfqpoint{1.037178in}{1.335323in}}%
\pgfpathlineto{\pgfqpoint{1.038464in}{1.340245in}}%
\pgfpathlineto{\pgfqpoint{1.040059in}{1.345140in}}%
\pgfpathlineto{\pgfqpoint{1.041960in}{1.350005in}}%
\pgfpathlineto{\pgfqpoint{1.045193in}{1.356547in}}%
\pgfpathlineto{\pgfqpoint{1.048429in}{1.363098in}}%
\pgfpathlineto{\pgfqpoint{1.051668in}{1.369655in}}%
\pgfpathlineto{\pgfqpoint{1.054910in}{1.376216in}}%
\pgfpathlineto{\pgfqpoint{1.053099in}{1.371575in}}%
\pgfpathlineto{\pgfqpoint{1.051581in}{1.366905in}}%
\pgfpathlineto{\pgfqpoint{1.050358in}{1.362210in}}%
\pgfpathlineto{\pgfqpoint{1.049434in}{1.357494in}}%
\pgfpathclose%
\pgfusepath{fill}%
\end{pgfscope}%
\begin{pgfscope}%
\pgfpathrectangle{\pgfqpoint{0.329460in}{0.284240in}}{\pgfqpoint{1.989680in}{1.989680in}}%
\pgfusepath{clip}%
\pgfsetbuttcap%
\pgfsetroundjoin%
\definecolor{currentfill}{rgb}{0.179019,0.433756,0.557430}%
\pgfsetfillcolor{currentfill}%
\pgfsetlinewidth{0.000000pt}%
\definecolor{currentstroke}{rgb}{0.000000,0.000000,0.000000}%
\pgfsetstrokecolor{currentstroke}%
\pgfsetdash{}{0pt}%
\pgfpathmoveto{\pgfqpoint{1.022165in}{1.282618in}}%
\pgfpathlineto{\pgfqpoint{1.018853in}{1.275698in}}%
\pgfpathlineto{\pgfqpoint{1.015543in}{1.268802in}}%
\pgfpathlineto{\pgfqpoint{1.012234in}{1.261933in}}%
\pgfpathlineto{\pgfqpoint{1.008928in}{1.255094in}}%
\pgfpathlineto{\pgfqpoint{1.008647in}{1.260541in}}%
\pgfpathlineto{\pgfqpoint{1.008708in}{1.265986in}}%
\pgfpathlineto{\pgfqpoint{1.009110in}{1.271423in}}%
\pgfpathlineto{\pgfqpoint{1.009852in}{1.276845in}}%
\pgfpathlineto{\pgfqpoint{1.013138in}{1.283452in}}%
\pgfpathlineto{\pgfqpoint{1.016426in}{1.290088in}}%
\pgfpathlineto{\pgfqpoint{1.019717in}{1.296752in}}%
\pgfpathlineto{\pgfqpoint{1.023009in}{1.303439in}}%
\pgfpathlineto{\pgfqpoint{1.022307in}{1.298248in}}%
\pgfpathlineto{\pgfqpoint{1.021931in}{1.293044in}}%
\pgfpathlineto{\pgfqpoint{1.021884in}{1.287832in}}%
\pgfpathlineto{\pgfqpoint{1.022165in}{1.282618in}}%
\pgfpathclose%
\pgfusepath{fill}%
\end{pgfscope}%
\begin{pgfscope}%
\pgfpathrectangle{\pgfqpoint{0.329460in}{0.284240in}}{\pgfqpoint{1.989680in}{1.989680in}}%
\pgfusepath{clip}%
\pgfsetbuttcap%
\pgfsetroundjoin%
\definecolor{currentfill}{rgb}{0.212395,0.359683,0.551710}%
\pgfsetfillcolor{currentfill}%
\pgfsetlinewidth{0.000000pt}%
\definecolor{currentstroke}{rgb}{0.000000,0.000000,0.000000}%
\pgfsetstrokecolor{currentstroke}%
\pgfsetdash{}{0pt}%
\pgfpathmoveto{\pgfqpoint{1.706923in}{1.233136in}}%
\pgfpathlineto{\pgfqpoint{1.710221in}{1.226538in}}%
\pgfpathlineto{\pgfqpoint{1.713518in}{1.219987in}}%
\pgfpathlineto{\pgfqpoint{1.716814in}{1.213484in}}%
\pgfpathlineto{\pgfqpoint{1.720108in}{1.207034in}}%
\pgfpathlineto{\pgfqpoint{1.719789in}{1.201116in}}%
\pgfpathlineto{\pgfqpoint{1.719101in}{1.195201in}}%
\pgfpathlineto{\pgfqpoint{1.718044in}{1.189295in}}%
\pgfpathlineto{\pgfqpoint{1.716617in}{1.183403in}}%
\pgfpathlineto{\pgfqpoint{1.713342in}{1.190089in}}%
\pgfpathlineto{\pgfqpoint{1.710065in}{1.196827in}}%
\pgfpathlineto{\pgfqpoint{1.706788in}{1.203614in}}%
\pgfpathlineto{\pgfqpoint{1.703509in}{1.210447in}}%
\pgfpathlineto{\pgfqpoint{1.704897in}{1.216104in}}%
\pgfpathlineto{\pgfqpoint{1.705928in}{1.221775in}}%
\pgfpathlineto{\pgfqpoint{1.706604in}{1.227454in}}%
\pgfpathlineto{\pgfqpoint{1.706923in}{1.233136in}}%
\pgfpathclose%
\pgfusepath{fill}%
\end{pgfscope}%
\begin{pgfscope}%
\pgfpathrectangle{\pgfqpoint{0.329460in}{0.284240in}}{\pgfqpoint{1.989680in}{1.989680in}}%
\pgfusepath{clip}%
\pgfsetbuttcap%
\pgfsetroundjoin%
\definecolor{currentfill}{rgb}{0.260571,0.246922,0.522828}%
\pgfsetfillcolor{currentfill}%
\pgfsetlinewidth{0.000000pt}%
\definecolor{currentstroke}{rgb}{0.000000,0.000000,0.000000}%
\pgfsetstrokecolor{currentstroke}%
\pgfsetdash{}{0pt}%
\pgfpathmoveto{\pgfqpoint{0.731235in}{1.078469in}}%
\pgfpathlineto{\pgfqpoint{0.727542in}{1.086671in}}%
\pgfpathlineto{\pgfqpoint{0.723834in}{1.095250in}}%
\pgfpathlineto{\pgfqpoint{0.720108in}{1.104213in}}%
\pgfpathlineto{\pgfqpoint{0.716365in}{1.113567in}}%
\pgfpathlineto{\pgfqpoint{0.713743in}{1.123913in}}%
\pgfpathlineto{\pgfqpoint{0.711769in}{1.134280in}}%
\pgfpathlineto{\pgfqpoint{0.710443in}{1.144657in}}%
\pgfpathlineto{\pgfqpoint{0.709761in}{1.155033in}}%
\pgfpathlineto{\pgfqpoint{0.713519in}{1.145497in}}%
\pgfpathlineto{\pgfqpoint{0.717259in}{1.136349in}}%
\pgfpathlineto{\pgfqpoint{0.720983in}{1.127583in}}%
\pgfpathlineto{\pgfqpoint{0.724691in}{1.119192in}}%
\pgfpathlineto{\pgfqpoint{0.725380in}{1.109001in}}%
\pgfpathlineto{\pgfqpoint{0.726699in}{1.098810in}}%
\pgfpathlineto{\pgfqpoint{0.728651in}{1.088629in}}%
\pgfpathlineto{\pgfqpoint{0.731235in}{1.078469in}}%
\pgfpathclose%
\pgfusepath{fill}%
\end{pgfscope}%
\begin{pgfscope}%
\pgfpathrectangle{\pgfqpoint{0.329460in}{0.284240in}}{\pgfqpoint{1.989680in}{1.989680in}}%
\pgfusepath{clip}%
\pgfsetbuttcap%
\pgfsetroundjoin%
\definecolor{currentfill}{rgb}{0.344074,0.780029,0.397381}%
\pgfsetfillcolor{currentfill}%
\pgfsetlinewidth{0.000000pt}%
\definecolor{currentstroke}{rgb}{0.000000,0.000000,0.000000}%
\pgfsetstrokecolor{currentstroke}%
\pgfsetdash{}{0pt}%
\pgfpathmoveto{\pgfqpoint{1.439091in}{1.638886in}}%
\pgfpathlineto{\pgfqpoint{1.440612in}{1.634610in}}%
\pgfpathlineto{\pgfqpoint{1.442131in}{1.630269in}}%
\pgfpathlineto{\pgfqpoint{1.443648in}{1.625864in}}%
\pgfpathlineto{\pgfqpoint{1.445162in}{1.621399in}}%
\pgfpathlineto{\pgfqpoint{1.450904in}{1.619947in}}%
\pgfpathlineto{\pgfqpoint{1.456550in}{1.618410in}}%
\pgfpathlineto{\pgfqpoint{1.462095in}{1.616788in}}%
\pgfpathlineto{\pgfqpoint{1.467535in}{1.615084in}}%
\pgfpathlineto{\pgfqpoint{1.465658in}{1.619656in}}%
\pgfpathlineto{\pgfqpoint{1.463778in}{1.624167in}}%
\pgfpathlineto{\pgfqpoint{1.461896in}{1.628615in}}%
\pgfpathlineto{\pgfqpoint{1.460011in}{1.632997in}}%
\pgfpathlineto{\pgfqpoint{1.454925in}{1.634587in}}%
\pgfpathlineto{\pgfqpoint{1.449740in}{1.636099in}}%
\pgfpathlineto{\pgfqpoint{1.444460in}{1.637533in}}%
\pgfpathlineto{\pgfqpoint{1.439091in}{1.638886in}}%
\pgfpathclose%
\pgfusepath{fill}%
\end{pgfscope}%
\begin{pgfscope}%
\pgfpathrectangle{\pgfqpoint{0.329460in}{0.284240in}}{\pgfqpoint{1.989680in}{1.989680in}}%
\pgfusepath{clip}%
\pgfsetbuttcap%
\pgfsetroundjoin%
\definecolor{currentfill}{rgb}{0.122606,0.585371,0.546557}%
\pgfsetfillcolor{currentfill}%
\pgfsetlinewidth{0.000000pt}%
\definecolor{currentstroke}{rgb}{0.000000,0.000000,0.000000}%
\pgfsetstrokecolor{currentstroke}%
\pgfsetdash{}{0pt}%
\pgfpathmoveto{\pgfqpoint{1.080946in}{1.428531in}}%
\pgfpathlineto{\pgfqpoint{1.077681in}{1.422031in}}%
\pgfpathlineto{\pgfqpoint{1.074420in}{1.415515in}}%
\pgfpathlineto{\pgfqpoint{1.071161in}{1.408985in}}%
\pgfpathlineto{\pgfqpoint{1.067905in}{1.402444in}}%
\pgfpathlineto{\pgfqpoint{1.069906in}{1.406830in}}%
\pgfpathlineto{\pgfqpoint{1.072183in}{1.411180in}}%
\pgfpathlineto{\pgfqpoint{1.074733in}{1.415489in}}%
\pgfpathlineto{\pgfqpoint{1.077553in}{1.419753in}}%
\pgfpathlineto{\pgfqpoint{1.080689in}{1.426077in}}%
\pgfpathlineto{\pgfqpoint{1.083828in}{1.432391in}}%
\pgfpathlineto{\pgfqpoint{1.086971in}{1.438691in}}%
\pgfpathlineto{\pgfqpoint{1.090116in}{1.444975in}}%
\pgfpathlineto{\pgfqpoint{1.087434in}{1.440923in}}%
\pgfpathlineto{\pgfqpoint{1.085010in}{1.436829in}}%
\pgfpathlineto{\pgfqpoint{1.082846in}{1.432697in}}%
\pgfpathlineto{\pgfqpoint{1.080946in}{1.428531in}}%
\pgfpathclose%
\pgfusepath{fill}%
\end{pgfscope}%
\begin{pgfscope}%
\pgfpathrectangle{\pgfqpoint{0.329460in}{0.284240in}}{\pgfqpoint{1.989680in}{1.989680in}}%
\pgfusepath{clip}%
\pgfsetbuttcap%
\pgfsetroundjoin%
\definecolor{currentfill}{rgb}{0.412913,0.803041,0.357269}%
\pgfsetfillcolor{currentfill}%
\pgfsetlinewidth{0.000000pt}%
\definecolor{currentstroke}{rgb}{0.000000,0.000000,0.000000}%
\pgfsetstrokecolor{currentstroke}%
\pgfsetdash{}{0pt}%
\pgfpathmoveto{\pgfqpoint{1.329018in}{1.663993in}}%
\pgfpathlineto{\pgfqpoint{1.328604in}{1.660162in}}%
\pgfpathlineto{\pgfqpoint{1.328191in}{1.656259in}}%
\pgfpathlineto{\pgfqpoint{1.327778in}{1.652286in}}%
\pgfpathlineto{\pgfqpoint{1.327366in}{1.648244in}}%
\pgfpathlineto{\pgfqpoint{1.333386in}{1.648555in}}%
\pgfpathlineto{\pgfqpoint{1.339423in}{1.648775in}}%
\pgfpathlineto{\pgfqpoint{1.345471in}{1.648906in}}%
\pgfpathlineto{\pgfqpoint{1.351524in}{1.648946in}}%
\pgfpathlineto{\pgfqpoint{1.351518in}{1.652975in}}%
\pgfpathlineto{\pgfqpoint{1.351512in}{1.656935in}}%
\pgfpathlineto{\pgfqpoint{1.351506in}{1.660826in}}%
\pgfpathlineto{\pgfqpoint{1.351501in}{1.664644in}}%
\pgfpathlineto{\pgfqpoint{1.345867in}{1.664607in}}%
\pgfpathlineto{\pgfqpoint{1.340238in}{1.664486in}}%
\pgfpathlineto{\pgfqpoint{1.334620in}{1.664281in}}%
\pgfpathlineto{\pgfqpoint{1.329018in}{1.663993in}}%
\pgfpathclose%
\pgfusepath{fill}%
\end{pgfscope}%
\begin{pgfscope}%
\pgfpathrectangle{\pgfqpoint{0.329460in}{0.284240in}}{\pgfqpoint{1.989680in}{1.989680in}}%
\pgfusepath{clip}%
\pgfsetbuttcap%
\pgfsetroundjoin%
\definecolor{currentfill}{rgb}{0.412913,0.803041,0.357269}%
\pgfsetfillcolor{currentfill}%
\pgfsetlinewidth{0.000000pt}%
\definecolor{currentstroke}{rgb}{0.000000,0.000000,0.000000}%
\pgfsetstrokecolor{currentstroke}%
\pgfsetdash{}{0pt}%
\pgfpathmoveto{\pgfqpoint{1.351501in}{1.664644in}}%
\pgfpathlineto{\pgfqpoint{1.351506in}{1.660826in}}%
\pgfpathlineto{\pgfqpoint{1.351512in}{1.656935in}}%
\pgfpathlineto{\pgfqpoint{1.351518in}{1.652975in}}%
\pgfpathlineto{\pgfqpoint{1.351524in}{1.648946in}}%
\pgfpathlineto{\pgfqpoint{1.357577in}{1.648896in}}%
\pgfpathlineto{\pgfqpoint{1.363624in}{1.648755in}}%
\pgfpathlineto{\pgfqpoint{1.369659in}{1.648525in}}%
\pgfpathlineto{\pgfqpoint{1.375676in}{1.648204in}}%
\pgfpathlineto{\pgfqpoint{1.375253in}{1.652246in}}%
\pgfpathlineto{\pgfqpoint{1.374829in}{1.656220in}}%
\pgfpathlineto{\pgfqpoint{1.374404in}{1.660124in}}%
\pgfpathlineto{\pgfqpoint{1.373979in}{1.663956in}}%
\pgfpathlineto{\pgfqpoint{1.368378in}{1.664253in}}%
\pgfpathlineto{\pgfqpoint{1.362762in}{1.664467in}}%
\pgfpathlineto{\pgfqpoint{1.357134in}{1.664598in}}%
\pgfpathlineto{\pgfqpoint{1.351501in}{1.664644in}}%
\pgfpathclose%
\pgfusepath{fill}%
\end{pgfscope}%
\begin{pgfscope}%
\pgfpathrectangle{\pgfqpoint{0.329460in}{0.284240in}}{\pgfqpoint{1.989680in}{1.989680in}}%
\pgfusepath{clip}%
\pgfsetbuttcap%
\pgfsetroundjoin%
\definecolor{currentfill}{rgb}{0.220124,0.725509,0.466226}%
\pgfsetfillcolor{currentfill}%
\pgfsetlinewidth{0.000000pt}%
\definecolor{currentstroke}{rgb}{0.000000,0.000000,0.000000}%
\pgfsetstrokecolor{currentstroke}%
\pgfsetdash{}{0pt}%
\pgfpathmoveto{\pgfqpoint{1.516655in}{1.578691in}}%
\pgfpathlineto{\pgfqpoint{1.519154in}{1.573561in}}%
\pgfpathlineto{\pgfqpoint{1.521651in}{1.568382in}}%
\pgfpathlineto{\pgfqpoint{1.524145in}{1.563154in}}%
\pgfpathlineto{\pgfqpoint{1.526635in}{1.557881in}}%
\pgfpathlineto{\pgfqpoint{1.531454in}{1.555186in}}%
\pgfpathlineto{\pgfqpoint{1.536098in}{1.552419in}}%
\pgfpathlineto{\pgfqpoint{1.540565in}{1.549581in}}%
\pgfpathlineto{\pgfqpoint{1.544850in}{1.546675in}}%
\pgfpathlineto{\pgfqpoint{1.542095in}{1.552117in}}%
\pgfpathlineto{\pgfqpoint{1.539338in}{1.557514in}}%
\pgfpathlineto{\pgfqpoint{1.536577in}{1.562862in}}%
\pgfpathlineto{\pgfqpoint{1.533813in}{1.568160in}}%
\pgfpathlineto{\pgfqpoint{1.529778in}{1.570890in}}%
\pgfpathlineto{\pgfqpoint{1.525570in}{1.573557in}}%
\pgfpathlineto{\pgfqpoint{1.521194in}{1.576158in}}%
\pgfpathlineto{\pgfqpoint{1.516655in}{1.578691in}}%
\pgfpathclose%
\pgfusepath{fill}%
\end{pgfscope}%
\begin{pgfscope}%
\pgfpathrectangle{\pgfqpoint{0.329460in}{0.284240in}}{\pgfqpoint{1.989680in}{1.989680in}}%
\pgfusepath{clip}%
\pgfsetbuttcap%
\pgfsetroundjoin%
\definecolor{currentfill}{rgb}{0.166383,0.690856,0.496502}%
\pgfsetfillcolor{currentfill}%
\pgfsetlinewidth{0.000000pt}%
\definecolor{currentstroke}{rgb}{0.000000,0.000000,0.000000}%
\pgfsetstrokecolor{currentstroke}%
\pgfsetdash{}{0pt}%
\pgfpathmoveto{\pgfqpoint{1.544850in}{1.546675in}}%
\pgfpathlineto{\pgfqpoint{1.547601in}{1.541188in}}%
\pgfpathlineto{\pgfqpoint{1.550349in}{1.535660in}}%
\pgfpathlineto{\pgfqpoint{1.553094in}{1.530091in}}%
\pgfpathlineto{\pgfqpoint{1.555835in}{1.524485in}}%
\pgfpathlineto{\pgfqpoint{1.560171in}{1.521334in}}%
\pgfpathlineto{\pgfqpoint{1.564305in}{1.518116in}}%
\pgfpathlineto{\pgfqpoint{1.568233in}{1.514836in}}%
\pgfpathlineto{\pgfqpoint{1.571952in}{1.511495in}}%
\pgfpathlineto{\pgfqpoint{1.568988in}{1.517288in}}%
\pgfpathlineto{\pgfqpoint{1.566021in}{1.523042in}}%
\pgfpathlineto{\pgfqpoint{1.563051in}{1.528757in}}%
\pgfpathlineto{\pgfqpoint{1.560077in}{1.534429in}}%
\pgfpathlineto{\pgfqpoint{1.556565in}{1.537578in}}%
\pgfpathlineto{\pgfqpoint{1.552854in}{1.540671in}}%
\pgfpathlineto{\pgfqpoint{1.548947in}{1.543704in}}%
\pgfpathlineto{\pgfqpoint{1.544850in}{1.546675in}}%
\pgfpathclose%
\pgfusepath{fill}%
\end{pgfscope}%
\begin{pgfscope}%
\pgfpathrectangle{\pgfqpoint{0.329460in}{0.284240in}}{\pgfqpoint{1.989680in}{1.989680in}}%
\pgfusepath{clip}%
\pgfsetbuttcap%
\pgfsetroundjoin%
\definecolor{currentfill}{rgb}{0.344074,0.780029,0.397381}%
\pgfsetfillcolor{currentfill}%
\pgfsetlinewidth{0.000000pt}%
\definecolor{currentstroke}{rgb}{0.000000,0.000000,0.000000}%
\pgfsetstrokecolor{currentstroke}%
\pgfsetdash{}{0pt}%
\pgfpathmoveto{\pgfqpoint{1.237931in}{1.631521in}}%
\pgfpathlineto{\pgfqpoint{1.235969in}{1.627112in}}%
\pgfpathlineto{\pgfqpoint{1.234009in}{1.622637in}}%
\pgfpathlineto{\pgfqpoint{1.232052in}{1.618100in}}%
\pgfpathlineto{\pgfqpoint{1.230098in}{1.613501in}}%
\pgfpathlineto{\pgfqpoint{1.235439in}{1.615277in}}%
\pgfpathlineto{\pgfqpoint{1.240891in}{1.616973in}}%
\pgfpathlineto{\pgfqpoint{1.246448in}{1.618585in}}%
\pgfpathlineto{\pgfqpoint{1.252105in}{1.620113in}}%
\pgfpathlineto{\pgfqpoint{1.253702in}{1.624600in}}%
\pgfpathlineto{\pgfqpoint{1.255302in}{1.629026in}}%
\pgfpathlineto{\pgfqpoint{1.256904in}{1.633389in}}%
\pgfpathlineto{\pgfqpoint{1.258507in}{1.637687in}}%
\pgfpathlineto{\pgfqpoint{1.253217in}{1.636262in}}%
\pgfpathlineto{\pgfqpoint{1.248021in}{1.634759in}}%
\pgfpathlineto{\pgfqpoint{1.242924in}{1.633178in}}%
\pgfpathlineto{\pgfqpoint{1.237931in}{1.631521in}}%
\pgfpathclose%
\pgfusepath{fill}%
\end{pgfscope}%
\begin{pgfscope}%
\pgfpathrectangle{\pgfqpoint{0.329460in}{0.284240in}}{\pgfqpoint{1.989680in}{1.989680in}}%
\pgfusepath{clip}%
\pgfsetbuttcap%
\pgfsetroundjoin%
\definecolor{currentfill}{rgb}{0.274128,0.199721,0.498911}%
\pgfsetfillcolor{currentfill}%
\pgfsetlinewidth{0.000000pt}%
\definecolor{currentstroke}{rgb}{0.000000,0.000000,0.000000}%
\pgfsetstrokecolor{currentstroke}%
\pgfsetdash{}{0pt}%
\pgfpathmoveto{\pgfqpoint{0.959901in}{1.076485in}}%
\pgfpathlineto{\pgfqpoint{0.956722in}{1.070464in}}%
\pgfpathlineto{\pgfqpoint{0.953543in}{1.064537in}}%
\pgfpathlineto{\pgfqpoint{0.950364in}{1.058710in}}%
\pgfpathlineto{\pgfqpoint{0.947183in}{1.052984in}}%
\pgfpathlineto{\pgfqpoint{0.943552in}{1.059648in}}%
\pgfpathlineto{\pgfqpoint{0.940339in}{1.066363in}}%
\pgfpathlineto{\pgfqpoint{0.937545in}{1.073119in}}%
\pgfpathlineto{\pgfqpoint{0.935172in}{1.079910in}}%
\pgfpathlineto{\pgfqpoint{0.938433in}{1.085401in}}%
\pgfpathlineto{\pgfqpoint{0.941694in}{1.090994in}}%
\pgfpathlineto{\pgfqpoint{0.944954in}{1.096686in}}%
\pgfpathlineto{\pgfqpoint{0.948214in}{1.102472in}}%
\pgfpathlineto{\pgfqpoint{0.950526in}{1.095917in}}%
\pgfpathlineto{\pgfqpoint{0.953246in}{1.089396in}}%
\pgfpathlineto{\pgfqpoint{0.956371in}{1.082917in}}%
\pgfpathlineto{\pgfqpoint{0.959901in}{1.076485in}}%
\pgfpathclose%
\pgfusepath{fill}%
\end{pgfscope}%
\begin{pgfscope}%
\pgfpathrectangle{\pgfqpoint{0.329460in}{0.284240in}}{\pgfqpoint{1.989680in}{1.989680in}}%
\pgfusepath{clip}%
\pgfsetbuttcap%
\pgfsetroundjoin%
\definecolor{currentfill}{rgb}{0.412913,0.803041,0.357269}%
\pgfsetfillcolor{currentfill}%
\pgfsetlinewidth{0.000000pt}%
\definecolor{currentstroke}{rgb}{0.000000,0.000000,0.000000}%
\pgfsetstrokecolor{currentstroke}%
\pgfsetdash{}{0pt}%
\pgfpathmoveto{\pgfqpoint{1.306873in}{1.662012in}}%
\pgfpathlineto{\pgfqpoint{1.306046in}{1.658143in}}%
\pgfpathlineto{\pgfqpoint{1.305220in}{1.654201in}}%
\pgfpathlineto{\pgfqpoint{1.304395in}{1.650190in}}%
\pgfpathlineto{\pgfqpoint{1.303571in}{1.646110in}}%
\pgfpathlineto{\pgfqpoint{1.309466in}{1.646777in}}%
\pgfpathlineto{\pgfqpoint{1.315401in}{1.647355in}}%
\pgfpathlineto{\pgfqpoint{1.321369in}{1.647844in}}%
\pgfpathlineto{\pgfqpoint{1.327366in}{1.648244in}}%
\pgfpathlineto{\pgfqpoint{1.327778in}{1.652286in}}%
\pgfpathlineto{\pgfqpoint{1.328191in}{1.656259in}}%
\pgfpathlineto{\pgfqpoint{1.328604in}{1.660162in}}%
\pgfpathlineto{\pgfqpoint{1.329018in}{1.663993in}}%
\pgfpathlineto{\pgfqpoint{1.323437in}{1.663622in}}%
\pgfpathlineto{\pgfqpoint{1.317882in}{1.663168in}}%
\pgfpathlineto{\pgfqpoint{1.312359in}{1.662631in}}%
\pgfpathlineto{\pgfqpoint{1.306873in}{1.662012in}}%
\pgfpathclose%
\pgfusepath{fill}%
\end{pgfscope}%
\begin{pgfscope}%
\pgfpathrectangle{\pgfqpoint{0.329460in}{0.284240in}}{\pgfqpoint{1.989680in}{1.989680in}}%
\pgfusepath{clip}%
\pgfsetbuttcap%
\pgfsetroundjoin%
\definecolor{currentfill}{rgb}{0.412913,0.803041,0.357269}%
\pgfsetfillcolor{currentfill}%
\pgfsetlinewidth{0.000000pt}%
\definecolor{currentstroke}{rgb}{0.000000,0.000000,0.000000}%
\pgfsetstrokecolor{currentstroke}%
\pgfsetdash{}{0pt}%
\pgfpathmoveto{\pgfqpoint{1.373979in}{1.663956in}}%
\pgfpathlineto{\pgfqpoint{1.374404in}{1.660124in}}%
\pgfpathlineto{\pgfqpoint{1.374829in}{1.656220in}}%
\pgfpathlineto{\pgfqpoint{1.375253in}{1.652246in}}%
\pgfpathlineto{\pgfqpoint{1.375676in}{1.648204in}}%
\pgfpathlineto{\pgfqpoint{1.381670in}{1.647794in}}%
\pgfpathlineto{\pgfqpoint{1.387636in}{1.647295in}}%
\pgfpathlineto{\pgfqpoint{1.393566in}{1.646707in}}%
\pgfpathlineto{\pgfqpoint{1.399456in}{1.646031in}}%
\pgfpathlineto{\pgfqpoint{1.398621in}{1.650112in}}%
\pgfpathlineto{\pgfqpoint{1.397785in}{1.654125in}}%
\pgfpathlineto{\pgfqpoint{1.396947in}{1.658067in}}%
\pgfpathlineto{\pgfqpoint{1.396109in}{1.661938in}}%
\pgfpathlineto{\pgfqpoint{1.390628in}{1.662566in}}%
\pgfpathlineto{\pgfqpoint{1.385109in}{1.663112in}}%
\pgfpathlineto{\pgfqpoint{1.379557in}{1.663575in}}%
\pgfpathlineto{\pgfqpoint{1.373979in}{1.663956in}}%
\pgfpathclose%
\pgfusepath{fill}%
\end{pgfscope}%
\begin{pgfscope}%
\pgfpathrectangle{\pgfqpoint{0.329460in}{0.284240in}}{\pgfqpoint{1.989680in}{1.989680in}}%
\pgfusepath{clip}%
\pgfsetbuttcap%
\pgfsetroundjoin%
\definecolor{currentfill}{rgb}{0.263663,0.237631,0.518762}%
\pgfsetfillcolor{currentfill}%
\pgfsetlinewidth{0.000000pt}%
\definecolor{currentstroke}{rgb}{0.000000,0.000000,0.000000}%
\pgfsetstrokecolor{currentstroke}%
\pgfsetdash{}{0pt}%
\pgfpathmoveto{\pgfqpoint{1.742792in}{1.132146in}}%
\pgfpathlineto{\pgfqpoint{1.746062in}{1.126063in}}%
\pgfpathlineto{\pgfqpoint{1.749332in}{1.120063in}}%
\pgfpathlineto{\pgfqpoint{1.752603in}{1.114149in}}%
\pgfpathlineto{\pgfqpoint{1.755873in}{1.108323in}}%
\pgfpathlineto{\pgfqpoint{1.753924in}{1.101743in}}%
\pgfpathlineto{\pgfqpoint{1.751567in}{1.095191in}}%
\pgfpathlineto{\pgfqpoint{1.748802in}{1.088674in}}%
\pgfpathlineto{\pgfqpoint{1.745632in}{1.082199in}}%
\pgfpathlineto{\pgfqpoint{1.742431in}{1.088261in}}%
\pgfpathlineto{\pgfqpoint{1.739231in}{1.094411in}}%
\pgfpathlineto{\pgfqpoint{1.736031in}{1.100647in}}%
\pgfpathlineto{\pgfqpoint{1.732832in}{1.106965in}}%
\pgfpathlineto{\pgfqpoint{1.735912in}{1.113205in}}%
\pgfpathlineto{\pgfqpoint{1.738600in}{1.119487in}}%
\pgfpathlineto{\pgfqpoint{1.740893in}{1.125803in}}%
\pgfpathlineto{\pgfqpoint{1.742792in}{1.132146in}}%
\pgfpathclose%
\pgfusepath{fill}%
\end{pgfscope}%
\begin{pgfscope}%
\pgfpathrectangle{\pgfqpoint{0.329460in}{0.284240in}}{\pgfqpoint{1.989680in}{1.989680in}}%
\pgfusepath{clip}%
\pgfsetbuttcap%
\pgfsetroundjoin%
\definecolor{currentfill}{rgb}{0.281477,0.755203,0.432552}%
\pgfsetfillcolor{currentfill}%
\pgfsetlinewidth{0.000000pt}%
\definecolor{currentstroke}{rgb}{0.000000,0.000000,0.000000}%
\pgfsetstrokecolor{currentstroke}%
\pgfsetdash{}{0pt}%
\pgfpathmoveto{\pgfqpoint{1.488130in}{1.607470in}}%
\pgfpathlineto{\pgfqpoint{1.490340in}{1.602710in}}%
\pgfpathlineto{\pgfqpoint{1.492547in}{1.597892in}}%
\pgfpathlineto{\pgfqpoint{1.494751in}{1.593018in}}%
\pgfpathlineto{\pgfqpoint{1.496952in}{1.588090in}}%
\pgfpathlineto{\pgfqpoint{1.502100in}{1.585854in}}%
\pgfpathlineto{\pgfqpoint{1.507103in}{1.583541in}}%
\pgfpathlineto{\pgfqpoint{1.511956in}{1.581152in}}%
\pgfpathlineto{\pgfqpoint{1.516655in}{1.578691in}}%
\pgfpathlineto{\pgfqpoint{1.514152in}{1.583768in}}%
\pgfpathlineto{\pgfqpoint{1.511646in}{1.588792in}}%
\pgfpathlineto{\pgfqpoint{1.509137in}{1.593759in}}%
\pgfpathlineto{\pgfqpoint{1.506625in}{1.598669in}}%
\pgfpathlineto{\pgfqpoint{1.502215in}{1.600974in}}%
\pgfpathlineto{\pgfqpoint{1.497659in}{1.603210in}}%
\pgfpathlineto{\pgfqpoint{1.492963in}{1.605376in}}%
\pgfpathlineto{\pgfqpoint{1.488130in}{1.607470in}}%
\pgfpathclose%
\pgfusepath{fill}%
\end{pgfscope}%
\begin{pgfscope}%
\pgfpathrectangle{\pgfqpoint{0.329460in}{0.284240in}}{\pgfqpoint{1.989680in}{1.989680in}}%
\pgfusepath{clip}%
\pgfsetbuttcap%
\pgfsetroundjoin%
\definecolor{currentfill}{rgb}{0.134692,0.658636,0.517649}%
\pgfsetfillcolor{currentfill}%
\pgfsetlinewidth{0.000000pt}%
\definecolor{currentstroke}{rgb}{0.000000,0.000000,0.000000}%
\pgfsetstrokecolor{currentstroke}%
\pgfsetdash{}{0pt}%
\pgfpathmoveto{\pgfqpoint{1.571952in}{1.511495in}}%
\pgfpathlineto{\pgfqpoint{1.574912in}{1.505667in}}%
\pgfpathlineto{\pgfqpoint{1.577869in}{1.499805in}}%
\pgfpathlineto{\pgfqpoint{1.580823in}{1.493911in}}%
\pgfpathlineto{\pgfqpoint{1.583774in}{1.487988in}}%
\pgfpathlineto{\pgfqpoint{1.587473in}{1.484394in}}%
\pgfpathlineto{\pgfqpoint{1.590943in}{1.480743in}}%
\pgfpathlineto{\pgfqpoint{1.594179in}{1.477039in}}%
\pgfpathlineto{\pgfqpoint{1.597180in}{1.473284in}}%
\pgfpathlineto{\pgfqpoint{1.594051in}{1.479408in}}%
\pgfpathlineto{\pgfqpoint{1.590920in}{1.485502in}}%
\pgfpathlineto{\pgfqpoint{1.587785in}{1.491565in}}%
\pgfpathlineto{\pgfqpoint{1.584647in}{1.497594in}}%
\pgfpathlineto{\pgfqpoint{1.581807in}{1.501143in}}%
\pgfpathlineto{\pgfqpoint{1.578742in}{1.504646in}}%
\pgfpathlineto{\pgfqpoint{1.575456in}{1.508097in}}%
\pgfpathlineto{\pgfqpoint{1.571952in}{1.511495in}}%
\pgfpathclose%
\pgfusepath{fill}%
\end{pgfscope}%
\begin{pgfscope}%
\pgfpathrectangle{\pgfqpoint{0.329460in}{0.284240in}}{\pgfqpoint{1.989680in}{1.989680in}}%
\pgfusepath{clip}%
\pgfsetbuttcap%
\pgfsetroundjoin%
\definecolor{currentfill}{rgb}{0.220124,0.725509,0.466226}%
\pgfsetfillcolor{currentfill}%
\pgfsetlinewidth{0.000000pt}%
\definecolor{currentstroke}{rgb}{0.000000,0.000000,0.000000}%
\pgfsetstrokecolor{currentstroke}%
\pgfsetdash{}{0pt}%
\pgfpathmoveto{\pgfqpoint{1.165123in}{1.565681in}}%
\pgfpathlineto{\pgfqpoint{1.162306in}{1.560344in}}%
\pgfpathlineto{\pgfqpoint{1.159492in}{1.554956in}}%
\pgfpathlineto{\pgfqpoint{1.156681in}{1.549519in}}%
\pgfpathlineto{\pgfqpoint{1.153874in}{1.544037in}}%
\pgfpathlineto{\pgfqpoint{1.157992in}{1.547001in}}%
\pgfpathlineto{\pgfqpoint{1.162297in}{1.549899in}}%
\pgfpathlineto{\pgfqpoint{1.166784in}{1.552730in}}%
\pgfpathlineto{\pgfqpoint{1.171448in}{1.555489in}}%
\pgfpathlineto{\pgfqpoint{1.174001in}{1.560799in}}%
\pgfpathlineto{\pgfqpoint{1.176557in}{1.566062in}}%
\pgfpathlineto{\pgfqpoint{1.179116in}{1.571277in}}%
\pgfpathlineto{\pgfqpoint{1.181677in}{1.576443in}}%
\pgfpathlineto{\pgfqpoint{1.177283in}{1.573849in}}%
\pgfpathlineto{\pgfqpoint{1.173056in}{1.571190in}}%
\pgfpathlineto{\pgfqpoint{1.169002in}{1.568466in}}%
\pgfpathlineto{\pgfqpoint{1.165123in}{1.565681in}}%
\pgfpathclose%
\pgfusepath{fill}%
\end{pgfscope}%
\begin{pgfscope}%
\pgfpathrectangle{\pgfqpoint{0.329460in}{0.284240in}}{\pgfqpoint{1.989680in}{1.989680in}}%
\pgfusepath{clip}%
\pgfsetbuttcap%
\pgfsetroundjoin%
\definecolor{currentfill}{rgb}{0.172719,0.448791,0.557885}%
\pgfsetfillcolor{currentfill}%
\pgfsetlinewidth{0.000000pt}%
\definecolor{currentstroke}{rgb}{0.000000,0.000000,0.000000}%
\pgfsetstrokecolor{currentstroke}%
\pgfsetdash{}{0pt}%
\pgfpathmoveto{\pgfqpoint{0.679033in}{1.246102in}}%
\pgfpathlineto{\pgfqpoint{0.675101in}{1.259438in}}%
\pgfpathlineto{\pgfqpoint{0.671147in}{1.273232in}}%
\pgfpathlineto{\pgfqpoint{0.667171in}{1.287492in}}%
\pgfpathlineto{\pgfqpoint{0.667204in}{1.298314in}}%
\pgfpathlineto{\pgfqpoint{0.667919in}{1.309111in}}%
\pgfpathlineto{\pgfqpoint{0.669311in}{1.319873in}}%
\pgfpathlineto{\pgfqpoint{0.671377in}{1.330589in}}%
\pgfpathlineto{\pgfqpoint{0.675301in}{1.316178in}}%
\pgfpathlineto{\pgfqpoint{0.679204in}{1.302230in}}%
\pgfpathlineto{\pgfqpoint{0.683085in}{1.288737in}}%
\pgfpathlineto{\pgfqpoint{0.681076in}{1.278135in}}%
\pgfpathlineto{\pgfqpoint{0.679727in}{1.267488in}}%
\pgfpathlineto{\pgfqpoint{0.679045in}{1.256807in}}%
\pgfpathlineto{\pgfqpoint{0.679033in}{1.246102in}}%
\pgfpathclose%
\pgfusepath{fill}%
\end{pgfscope}%
\begin{pgfscope}%
\pgfpathrectangle{\pgfqpoint{0.329460in}{0.284240in}}{\pgfqpoint{1.989680in}{1.989680in}}%
\pgfusepath{clip}%
\pgfsetbuttcap%
\pgfsetroundjoin%
\definecolor{currentfill}{rgb}{0.166383,0.690856,0.496502}%
\pgfsetfillcolor{currentfill}%
\pgfsetlinewidth{0.000000pt}%
\definecolor{currentstroke}{rgb}{0.000000,0.000000,0.000000}%
\pgfsetstrokecolor{currentstroke}%
\pgfsetdash{}{0pt}%
\pgfpathmoveto{\pgfqpoint{1.139347in}{1.531585in}}%
\pgfpathlineto{\pgfqpoint{1.136330in}{1.525869in}}%
\pgfpathlineto{\pgfqpoint{1.133316in}{1.520111in}}%
\pgfpathlineto{\pgfqpoint{1.130305in}{1.514313in}}%
\pgfpathlineto{\pgfqpoint{1.127298in}{1.508477in}}%
\pgfpathlineto{\pgfqpoint{1.130826in}{1.511869in}}%
\pgfpathlineto{\pgfqpoint{1.134568in}{1.515203in}}%
\pgfpathlineto{\pgfqpoint{1.138519in}{1.518477in}}%
\pgfpathlineto{\pgfqpoint{1.142676in}{1.521687in}}%
\pgfpathlineto{\pgfqpoint{1.145471in}{1.527334in}}%
\pgfpathlineto{\pgfqpoint{1.148268in}{1.532942in}}%
\pgfpathlineto{\pgfqpoint{1.151069in}{1.538511in}}%
\pgfpathlineto{\pgfqpoint{1.153874in}{1.544037in}}%
\pgfpathlineto{\pgfqpoint{1.149946in}{1.541011in}}%
\pgfpathlineto{\pgfqpoint{1.146213in}{1.537925in}}%
\pgfpathlineto{\pgfqpoint{1.142678in}{1.534782in}}%
\pgfpathlineto{\pgfqpoint{1.139347in}{1.531585in}}%
\pgfpathclose%
\pgfusepath{fill}%
\end{pgfscope}%
\begin{pgfscope}%
\pgfpathrectangle{\pgfqpoint{0.329460in}{0.284240in}}{\pgfqpoint{1.989680in}{1.989680in}}%
\pgfusepath{clip}%
\pgfsetbuttcap%
\pgfsetroundjoin%
\definecolor{currentfill}{rgb}{0.163625,0.471133,0.558148}%
\pgfsetfillcolor{currentfill}%
\pgfsetlinewidth{0.000000pt}%
\definecolor{currentstroke}{rgb}{0.000000,0.000000,0.000000}%
\pgfsetstrokecolor{currentstroke}%
\pgfsetdash{}{0pt}%
\pgfpathmoveto{\pgfqpoint{1.665321in}{1.334774in}}%
\pgfpathlineto{\pgfqpoint{1.668612in}{1.328067in}}%
\pgfpathlineto{\pgfqpoint{1.671900in}{1.321374in}}%
\pgfpathlineto{\pgfqpoint{1.675185in}{1.314697in}}%
\pgfpathlineto{\pgfqpoint{1.678468in}{1.308039in}}%
\pgfpathlineto{\pgfqpoint{1.679460in}{1.302863in}}%
\pgfpathlineto{\pgfqpoint{1.680126in}{1.297670in}}%
\pgfpathlineto{\pgfqpoint{1.680465in}{1.292465in}}%
\pgfpathlineto{\pgfqpoint{1.680476in}{1.287253in}}%
\pgfpathlineto{\pgfqpoint{1.677162in}{1.294142in}}%
\pgfpathlineto{\pgfqpoint{1.673845in}{1.301050in}}%
\pgfpathlineto{\pgfqpoint{1.670527in}{1.307975in}}%
\pgfpathlineto{\pgfqpoint{1.667206in}{1.314912in}}%
\pgfpathlineto{\pgfqpoint{1.667206in}{1.319892in}}%
\pgfpathlineto{\pgfqpoint{1.666891in}{1.324866in}}%
\pgfpathlineto{\pgfqpoint{1.666262in}{1.329828in}}%
\pgfpathlineto{\pgfqpoint{1.665321in}{1.334774in}}%
\pgfpathclose%
\pgfusepath{fill}%
\end{pgfscope}%
\begin{pgfscope}%
\pgfpathrectangle{\pgfqpoint{0.329460in}{0.284240in}}{\pgfqpoint{1.989680in}{1.989680in}}%
\pgfusepath{clip}%
\pgfsetbuttcap%
\pgfsetroundjoin%
\definecolor{currentfill}{rgb}{0.133743,0.548535,0.553541}%
\pgfsetfillcolor{currentfill}%
\pgfsetlinewidth{0.000000pt}%
\definecolor{currentstroke}{rgb}{0.000000,0.000000,0.000000}%
\pgfsetstrokecolor{currentstroke}%
\pgfsetdash{}{0pt}%
\pgfpathmoveto{\pgfqpoint{1.632705in}{1.406345in}}%
\pgfpathlineto{\pgfqpoint{1.635936in}{1.399844in}}%
\pgfpathlineto{\pgfqpoint{1.639164in}{1.393336in}}%
\pgfpathlineto{\pgfqpoint{1.642389in}{1.386825in}}%
\pgfpathlineto{\pgfqpoint{1.645611in}{1.380313in}}%
\pgfpathlineto{\pgfqpoint{1.647681in}{1.375702in}}%
\pgfpathlineto{\pgfqpoint{1.649460in}{1.371058in}}%
\pgfpathlineto{\pgfqpoint{1.650945in}{1.366385in}}%
\pgfpathlineto{\pgfqpoint{1.652134in}{1.361687in}}%
\pgfpathlineto{\pgfqpoint{1.648831in}{1.368424in}}%
\pgfpathlineto{\pgfqpoint{1.645525in}{1.375158in}}%
\pgfpathlineto{\pgfqpoint{1.642216in}{1.381889in}}%
\pgfpathlineto{\pgfqpoint{1.638905in}{1.388613in}}%
\pgfpathlineto{\pgfqpoint{1.637777in}{1.393085in}}%
\pgfpathlineto{\pgfqpoint{1.636367in}{1.397534in}}%
\pgfpathlineto{\pgfqpoint{1.634676in}{1.401955in}}%
\pgfpathlineto{\pgfqpoint{1.632705in}{1.406345in}}%
\pgfpathclose%
\pgfusepath{fill}%
\end{pgfscope}%
\begin{pgfscope}%
\pgfpathrectangle{\pgfqpoint{0.329460in}{0.284240in}}{\pgfqpoint{1.989680in}{1.989680in}}%
\pgfusepath{clip}%
\pgfsetbuttcap%
\pgfsetroundjoin%
\definecolor{currentfill}{rgb}{0.212395,0.359683,0.551710}%
\pgfsetfillcolor{currentfill}%
\pgfsetlinewidth{0.000000pt}%
\definecolor{currentstroke}{rgb}{0.000000,0.000000,0.000000}%
\pgfsetstrokecolor{currentstroke}%
\pgfsetdash{}{0pt}%
\pgfpathmoveto{\pgfqpoint{1.000398in}{1.205435in}}%
\pgfpathlineto{\pgfqpoint{0.997131in}{1.198550in}}%
\pgfpathlineto{\pgfqpoint{0.993865in}{1.191711in}}%
\pgfpathlineto{\pgfqpoint{0.990600in}{1.184920in}}%
\pgfpathlineto{\pgfqpoint{0.987337in}{1.178182in}}%
\pgfpathlineto{\pgfqpoint{0.985582in}{1.184056in}}%
\pgfpathlineto{\pgfqpoint{0.984196in}{1.189950in}}%
\pgfpathlineto{\pgfqpoint{0.983179in}{1.195858in}}%
\pgfpathlineto{\pgfqpoint{0.982532in}{1.201774in}}%
\pgfpathlineto{\pgfqpoint{0.985827in}{1.208276in}}%
\pgfpathlineto{\pgfqpoint{0.989123in}{1.214831in}}%
\pgfpathlineto{\pgfqpoint{0.992420in}{1.221435in}}%
\pgfpathlineto{\pgfqpoint{0.995718in}{1.228085in}}%
\pgfpathlineto{\pgfqpoint{0.996354in}{1.222405in}}%
\pgfpathlineto{\pgfqpoint{0.997346in}{1.216733in}}%
\pgfpathlineto{\pgfqpoint{0.998694in}{1.211075in}}%
\pgfpathlineto{\pgfqpoint{1.000398in}{1.205435in}}%
\pgfpathclose%
\pgfusepath{fill}%
\end{pgfscope}%
\begin{pgfscope}%
\pgfpathrectangle{\pgfqpoint{0.329460in}{0.284240in}}{\pgfqpoint{1.989680in}{1.989680in}}%
\pgfusepath{clip}%
\pgfsetbuttcap%
\pgfsetroundjoin%
\definecolor{currentfill}{rgb}{0.412913,0.803041,0.357269}%
\pgfsetfillcolor{currentfill}%
\pgfsetlinewidth{0.000000pt}%
\definecolor{currentstroke}{rgb}{0.000000,0.000000,0.000000}%
\pgfsetstrokecolor{currentstroke}%
\pgfsetdash{}{0pt}%
\pgfpathmoveto{\pgfqpoint{1.396109in}{1.661938in}}%
\pgfpathlineto{\pgfqpoint{1.396947in}{1.658067in}}%
\pgfpathlineto{\pgfqpoint{1.397785in}{1.654125in}}%
\pgfpathlineto{\pgfqpoint{1.398621in}{1.650112in}}%
\pgfpathlineto{\pgfqpoint{1.399456in}{1.646031in}}%
\pgfpathlineto{\pgfqpoint{1.405301in}{1.645267in}}%
\pgfpathlineto{\pgfqpoint{1.411093in}{1.644416in}}%
\pgfpathlineto{\pgfqpoint{1.416829in}{1.643478in}}%
\pgfpathlineto{\pgfqpoint{1.415693in}{1.647606in}}%
\pgfpathlineto{\pgfqpoint{1.414555in}{1.651664in}}%
\pgfpathlineto{\pgfqpoint{1.413416in}{1.655653in}}%
\pgfpathlineto{\pgfqpoint{1.412275in}{1.659570in}}%
\pgfpathlineto{\pgfqpoint{1.406938in}{1.660440in}}%
\pgfpathlineto{\pgfqpoint{1.401547in}{1.661229in}}%
\pgfpathlineto{\pgfqpoint{1.396109in}{1.661938in}}%
\pgfpathclose%
\pgfusepath{fill}%
\end{pgfscope}%
\begin{pgfscope}%
\pgfpathrectangle{\pgfqpoint{0.329460in}{0.284240in}}{\pgfqpoint{1.989680in}{1.989680in}}%
\pgfusepath{clip}%
\pgfsetbuttcap%
\pgfsetroundjoin%
\definecolor{currentfill}{rgb}{0.233603,0.313828,0.543914}%
\pgfsetfillcolor{currentfill}%
\pgfsetlinewidth{0.000000pt}%
\definecolor{currentstroke}{rgb}{0.000000,0.000000,0.000000}%
\pgfsetstrokecolor{currentstroke}%
\pgfsetdash{}{0pt}%
\pgfpathmoveto{\pgfqpoint{1.992680in}{1.164247in}}%
\pgfpathlineto{\pgfqpoint{1.996450in}{1.174217in}}%
\pgfpathlineto{\pgfqpoint{2.000237in}{1.184589in}}%
\pgfpathlineto{\pgfqpoint{2.004043in}{1.195368in}}%
\pgfpathlineto{\pgfqpoint{2.007868in}{1.206561in}}%
\pgfpathlineto{\pgfqpoint{2.007781in}{1.196019in}}%
\pgfpathlineto{\pgfqpoint{2.007036in}{1.185467in}}%
\pgfpathlineto{\pgfqpoint{2.005629in}{1.174913in}}%
\pgfpathlineto{\pgfqpoint{2.003560in}{1.164371in}}%
\pgfpathlineto{\pgfqpoint{1.999735in}{1.153350in}}%
\pgfpathlineto{\pgfqpoint{1.995929in}{1.142747in}}%
\pgfpathlineto{\pgfqpoint{1.992142in}{1.132553in}}%
\pgfpathlineto{\pgfqpoint{1.988373in}{1.122762in}}%
\pgfpathlineto{\pgfqpoint{1.990419in}{1.133127in}}%
\pgfpathlineto{\pgfqpoint{1.991817in}{1.143503in}}%
\pgfpathlineto{\pgfqpoint{1.992570in}{1.153880in}}%
\pgfpathlineto{\pgfqpoint{1.992680in}{1.164247in}}%
\pgfpathclose%
\pgfusepath{fill}%
\end{pgfscope}%
\begin{pgfscope}%
\pgfpathrectangle{\pgfqpoint{0.329460in}{0.284240in}}{\pgfqpoint{1.989680in}{1.989680in}}%
\pgfusepath{clip}%
\pgfsetbuttcap%
\pgfsetroundjoin%
\definecolor{currentfill}{rgb}{0.412913,0.803041,0.357269}%
\pgfsetfillcolor{currentfill}%
\pgfsetlinewidth{0.000000pt}%
\definecolor{currentstroke}{rgb}{0.000000,0.000000,0.000000}%
\pgfsetstrokecolor{currentstroke}%
\pgfsetdash{}{0pt}%
\pgfpathmoveto{\pgfqpoint{1.285405in}{1.658730in}}%
\pgfpathlineto{\pgfqpoint{1.284177in}{1.654797in}}%
\pgfpathlineto{\pgfqpoint{1.282950in}{1.650792in}}%
\pgfpathlineto{\pgfqpoint{1.281724in}{1.646717in}}%
\pgfpathlineto{\pgfqpoint{1.280500in}{1.642574in}}%
\pgfpathlineto{\pgfqpoint{1.286181in}{1.643587in}}%
\pgfpathlineto{\pgfqpoint{1.291923in}{1.644514in}}%
\pgfpathlineto{\pgfqpoint{1.297722in}{1.645356in}}%
\pgfpathlineto{\pgfqpoint{1.303571in}{1.646110in}}%
\pgfpathlineto{\pgfqpoint{1.304395in}{1.650190in}}%
\pgfpathlineto{\pgfqpoint{1.305220in}{1.654201in}}%
\pgfpathlineto{\pgfqpoint{1.306046in}{1.658143in}}%
\pgfpathlineto{\pgfqpoint{1.306873in}{1.662012in}}%
\pgfpathlineto{\pgfqpoint{1.301430in}{1.661312in}}%
\pgfpathlineto{\pgfqpoint{1.296034in}{1.660531in}}%
\pgfpathlineto{\pgfqpoint{1.290691in}{1.659670in}}%
\pgfpathlineto{\pgfqpoint{1.285405in}{1.658730in}}%
\pgfpathclose%
\pgfusepath{fill}%
\end{pgfscope}%
\begin{pgfscope}%
\pgfpathrectangle{\pgfqpoint{0.329460in}{0.284240in}}{\pgfqpoint{1.989680in}{1.989680in}}%
\pgfusepath{clip}%
\pgfsetbuttcap%
\pgfsetroundjoin%
\definecolor{currentfill}{rgb}{0.271305,0.019942,0.347269}%
\pgfsetfillcolor{currentfill}%
\pgfsetlinewidth{0.000000pt}%
\definecolor{currentstroke}{rgb}{0.000000,0.000000,0.000000}%
\pgfsetstrokecolor{currentstroke}%
\pgfsetdash{}{0pt}%
\pgfpathmoveto{\pgfqpoint{1.822912in}{0.972075in}}%
\pgfpathlineto{\pgfqpoint{1.826168in}{0.969344in}}%
\pgfpathlineto{\pgfqpoint{1.829429in}{0.966793in}}%
\pgfpathlineto{\pgfqpoint{1.832695in}{0.964427in}}%
\pgfpathlineto{\pgfqpoint{1.835965in}{0.962250in}}%
\pgfpathlineto{\pgfqpoint{1.831682in}{0.954204in}}%
\pgfpathlineto{\pgfqpoint{1.826903in}{0.946223in}}%
\pgfpathlineto{\pgfqpoint{1.821631in}{0.938316in}}%
\pgfpathlineto{\pgfqpoint{1.815870in}{0.930492in}}%
\pgfpathlineto{\pgfqpoint{1.812720in}{0.932895in}}%
\pgfpathlineto{\pgfqpoint{1.809575in}{0.935487in}}%
\pgfpathlineto{\pgfqpoint{1.806434in}{0.938264in}}%
\pgfpathlineto{\pgfqpoint{1.803298in}{0.941223in}}%
\pgfpathlineto{\pgfqpoint{1.808919in}{0.948823in}}%
\pgfpathlineto{\pgfqpoint{1.814064in}{0.956504in}}%
\pgfpathlineto{\pgfqpoint{1.818729in}{0.964257in}}%
\pgfpathlineto{\pgfqpoint{1.822912in}{0.972075in}}%
\pgfpathclose%
\pgfusepath{fill}%
\end{pgfscope}%
\begin{pgfscope}%
\pgfpathrectangle{\pgfqpoint{0.329460in}{0.284240in}}{\pgfqpoint{1.989680in}{1.989680in}}%
\pgfusepath{clip}%
\pgfsetbuttcap%
\pgfsetroundjoin%
\definecolor{currentfill}{rgb}{0.281477,0.755203,0.432552}%
\pgfsetfillcolor{currentfill}%
\pgfsetlinewidth{0.000000pt}%
\definecolor{currentstroke}{rgb}{0.000000,0.000000,0.000000}%
\pgfsetstrokecolor{currentstroke}%
\pgfsetdash{}{0pt}%
\pgfpathmoveto{\pgfqpoint{1.191956in}{1.596565in}}%
\pgfpathlineto{\pgfqpoint{1.189381in}{1.591619in}}%
\pgfpathlineto{\pgfqpoint{1.186810in}{1.586616in}}%
\pgfpathlineto{\pgfqpoint{1.184242in}{1.581556in}}%
\pgfpathlineto{\pgfqpoint{1.181677in}{1.576443in}}%
\pgfpathlineto{\pgfqpoint{1.186235in}{1.578968in}}%
\pgfpathlineto{\pgfqpoint{1.190951in}{1.581421in}}%
\pgfpathlineto{\pgfqpoint{1.195821in}{1.583802in}}%
\pgfpathlineto{\pgfqpoint{1.200840in}{1.586107in}}%
\pgfpathlineto{\pgfqpoint{1.203112in}{1.591066in}}%
\pgfpathlineto{\pgfqpoint{1.205386in}{1.595971in}}%
\pgfpathlineto{\pgfqpoint{1.207663in}{1.600821in}}%
\pgfpathlineto{\pgfqpoint{1.209943in}{1.605613in}}%
\pgfpathlineto{\pgfqpoint{1.205231in}{1.603454in}}%
\pgfpathlineto{\pgfqpoint{1.200659in}{1.601226in}}%
\pgfpathlineto{\pgfqpoint{1.196233in}{1.598928in}}%
\pgfpathlineto{\pgfqpoint{1.191956in}{1.596565in}}%
\pgfpathclose%
\pgfusepath{fill}%
\end{pgfscope}%
\begin{pgfscope}%
\pgfpathrectangle{\pgfqpoint{0.329460in}{0.284240in}}{\pgfqpoint{1.989680in}{1.989680in}}%
\pgfusepath{clip}%
\pgfsetbuttcap%
\pgfsetroundjoin%
\definecolor{currentfill}{rgb}{0.268510,0.009605,0.335427}%
\pgfsetfillcolor{currentfill}%
\pgfsetlinewidth{0.000000pt}%
\definecolor{currentstroke}{rgb}{0.000000,0.000000,0.000000}%
\pgfsetstrokecolor{currentstroke}%
\pgfsetdash{}{0pt}%
\pgfpathmoveto{\pgfqpoint{1.835965in}{0.962250in}}%
\pgfpathlineto{\pgfqpoint{1.839241in}{0.960266in}}%
\pgfpathlineto{\pgfqpoint{1.842521in}{0.958480in}}%
\pgfpathlineto{\pgfqpoint{1.845808in}{0.956894in}}%
\pgfpathlineto{\pgfqpoint{1.849100in}{0.955514in}}%
\pgfpathlineto{\pgfqpoint{1.844717in}{0.947240in}}%
\pgfpathlineto{\pgfqpoint{1.839825in}{0.939034in}}%
\pgfpathlineto{\pgfqpoint{1.834425in}{0.930903in}}%
\pgfpathlineto{\pgfqpoint{1.828522in}{0.922857in}}%
\pgfpathlineto{\pgfqpoint{1.825351in}{0.924461in}}%
\pgfpathlineto{\pgfqpoint{1.822185in}{0.926271in}}%
\pgfpathlineto{\pgfqpoint{1.819025in}{0.928283in}}%
\pgfpathlineto{\pgfqpoint{1.815870in}{0.930492in}}%
\pgfpathlineto{\pgfqpoint{1.821631in}{0.938316in}}%
\pgfpathlineto{\pgfqpoint{1.826903in}{0.946223in}}%
\pgfpathlineto{\pgfqpoint{1.831682in}{0.954204in}}%
\pgfpathlineto{\pgfqpoint{1.835965in}{0.962250in}}%
\pgfpathclose%
\pgfusepath{fill}%
\end{pgfscope}%
\begin{pgfscope}%
\pgfpathrectangle{\pgfqpoint{0.329460in}{0.284240in}}{\pgfqpoint{1.989680in}{1.989680in}}%
\pgfusepath{clip}%
\pgfsetbuttcap%
\pgfsetroundjoin%
\definecolor{currentfill}{rgb}{0.282327,0.094955,0.417331}%
\pgfsetfillcolor{currentfill}%
\pgfsetlinewidth{0.000000pt}%
\definecolor{currentstroke}{rgb}{0.000000,0.000000,0.000000}%
\pgfsetstrokecolor{currentstroke}%
\pgfsetdash{}{0pt}%
\pgfpathmoveto{\pgfqpoint{0.790156in}{0.969491in}}%
\pgfpathlineto{\pgfqpoint{0.786706in}{0.973418in}}%
\pgfpathlineto{\pgfqpoint{0.783244in}{0.977654in}}%
\pgfpathlineto{\pgfqpoint{0.779770in}{0.982204in}}%
\pgfpathlineto{\pgfqpoint{0.776283in}{0.987075in}}%
\pgfpathlineto{\pgfqpoint{0.771370in}{0.996652in}}%
\pgfpathlineto{\pgfqpoint{0.767055in}{1.006291in}}%
\pgfpathlineto{\pgfqpoint{0.763343in}{1.015982in}}%
\pgfpathlineto{\pgfqpoint{0.760232in}{1.025714in}}%
\pgfpathlineto{\pgfqpoint{0.763795in}{1.020639in}}%
\pgfpathlineto{\pgfqpoint{0.767345in}{1.015883in}}%
\pgfpathlineto{\pgfqpoint{0.770884in}{1.011440in}}%
\pgfpathlineto{\pgfqpoint{0.774410in}{1.007305in}}%
\pgfpathlineto{\pgfqpoint{0.777466in}{0.997779in}}%
\pgfpathlineto{\pgfqpoint{0.781110in}{0.988295in}}%
\pgfpathlineto{\pgfqpoint{0.785341in}{0.978862in}}%
\pgfpathlineto{\pgfqpoint{0.790156in}{0.969491in}}%
\pgfpathclose%
\pgfusepath{fill}%
\end{pgfscope}%
\begin{pgfscope}%
\pgfpathrectangle{\pgfqpoint{0.329460in}{0.284240in}}{\pgfqpoint{1.989680in}{1.989680in}}%
\pgfusepath{clip}%
\pgfsetbuttcap%
\pgfsetroundjoin%
\definecolor{currentfill}{rgb}{0.274952,0.037752,0.364543}%
\pgfsetfillcolor{currentfill}%
\pgfsetlinewidth{0.000000pt}%
\definecolor{currentstroke}{rgb}{0.000000,0.000000,0.000000}%
\pgfsetstrokecolor{currentstroke}%
\pgfsetdash{}{0pt}%
\pgfpathmoveto{\pgfqpoint{1.809927in}{0.984734in}}%
\pgfpathlineto{\pgfqpoint{1.813168in}{0.981317in}}%
\pgfpathlineto{\pgfqpoint{1.816412in}{0.978066in}}%
\pgfpathlineto{\pgfqpoint{1.819660in}{0.974984in}}%
\pgfpathlineto{\pgfqpoint{1.822912in}{0.972075in}}%
\pgfpathlineto{\pgfqpoint{1.818729in}{0.964257in}}%
\pgfpathlineto{\pgfqpoint{1.814064in}{0.956504in}}%
\pgfpathlineto{\pgfqpoint{1.808919in}{0.948823in}}%
\pgfpathlineto{\pgfqpoint{1.803298in}{0.941223in}}%
\pgfpathlineto{\pgfqpoint{1.800167in}{0.944359in}}%
\pgfpathlineto{\pgfqpoint{1.797040in}{0.947668in}}%
\pgfpathlineto{\pgfqpoint{1.793916in}{0.951148in}}%
\pgfpathlineto{\pgfqpoint{1.790796in}{0.954793in}}%
\pgfpathlineto{\pgfqpoint{1.796276in}{0.962167in}}%
\pgfpathlineto{\pgfqpoint{1.801293in}{0.969621in}}%
\pgfpathlineto{\pgfqpoint{1.805844in}{0.977146in}}%
\pgfpathlineto{\pgfqpoint{1.809927in}{0.984734in}}%
\pgfpathclose%
\pgfusepath{fill}%
\end{pgfscope}%
\begin{pgfscope}%
\pgfpathrectangle{\pgfqpoint{0.329460in}{0.284240in}}{\pgfqpoint{1.989680in}{1.989680in}}%
\pgfusepath{clip}%
\pgfsetbuttcap%
\pgfsetroundjoin%
\definecolor{currentfill}{rgb}{0.134692,0.658636,0.517649}%
\pgfsetfillcolor{currentfill}%
\pgfsetlinewidth{0.000000pt}%
\definecolor{currentstroke}{rgb}{0.000000,0.000000,0.000000}%
\pgfsetstrokecolor{currentstroke}%
\pgfsetdash{}{0pt}%
\pgfpathmoveto{\pgfqpoint{1.115395in}{1.494402in}}%
\pgfpathlineto{\pgfqpoint{1.112224in}{1.488327in}}%
\pgfpathlineto{\pgfqpoint{1.109056in}{1.482218in}}%
\pgfpathlineto{\pgfqpoint{1.105891in}{1.476077in}}%
\pgfpathlineto{\pgfqpoint{1.102730in}{1.469907in}}%
\pgfpathlineto{\pgfqpoint{1.105517in}{1.473703in}}%
\pgfpathlineto{\pgfqpoint{1.108544in}{1.477453in}}%
\pgfpathlineto{\pgfqpoint{1.111807in}{1.481152in}}%
\pgfpathlineto{\pgfqpoint{1.115302in}{1.484797in}}%
\pgfpathlineto{\pgfqpoint{1.118296in}{1.490763in}}%
\pgfpathlineto{\pgfqpoint{1.121294in}{1.496700in}}%
\pgfpathlineto{\pgfqpoint{1.124294in}{1.502606in}}%
\pgfpathlineto{\pgfqpoint{1.127298in}{1.508477in}}%
\pgfpathlineto{\pgfqpoint{1.123987in}{1.505032in}}%
\pgfpathlineto{\pgfqpoint{1.120898in}{1.501535in}}%
\pgfpathlineto{\pgfqpoint{1.118032in}{1.497991in}}%
\pgfpathlineto{\pgfqpoint{1.115395in}{1.494402in}}%
\pgfpathclose%
\pgfusepath{fill}%
\end{pgfscope}%
\begin{pgfscope}%
\pgfpathrectangle{\pgfqpoint{0.329460in}{0.284240in}}{\pgfqpoint{1.989680in}{1.989680in}}%
\pgfusepath{clip}%
\pgfsetbuttcap%
\pgfsetroundjoin%
\definecolor{currentfill}{rgb}{0.282884,0.135920,0.453427}%
\pgfsetfillcolor{currentfill}%
\pgfsetlinewidth{0.000000pt}%
\definecolor{currentstroke}{rgb}{0.000000,0.000000,0.000000}%
\pgfsetstrokecolor{currentstroke}%
\pgfsetdash{}{0pt}%
\pgfpathmoveto{\pgfqpoint{1.944401in}{1.034393in}}%
\pgfpathlineto{\pgfqpoint{1.947986in}{1.039837in}}%
\pgfpathlineto{\pgfqpoint{1.951583in}{1.045612in}}%
\pgfpathlineto{\pgfqpoint{1.955195in}{1.051721in}}%
\pgfpathlineto{\pgfqpoint{1.958820in}{1.058171in}}%
\pgfpathlineto{\pgfqpoint{1.956205in}{1.048209in}}%
\pgfpathlineto{\pgfqpoint{1.952973in}{1.038279in}}%
\pgfpathlineto{\pgfqpoint{1.949123in}{1.028392in}}%
\pgfpathlineto{\pgfqpoint{1.944656in}{1.018559in}}%
\pgfpathlineto{\pgfqpoint{1.941094in}{1.012308in}}%
\pgfpathlineto{\pgfqpoint{1.937546in}{1.006398in}}%
\pgfpathlineto{\pgfqpoint{1.934011in}{1.000826in}}%
\pgfpathlineto{\pgfqpoint{1.930489in}{0.995584in}}%
\pgfpathlineto{\pgfqpoint{1.934870in}{1.005217in}}%
\pgfpathlineto{\pgfqpoint{1.938650in}{1.014903in}}%
\pgfpathlineto{\pgfqpoint{1.941827in}{1.024631in}}%
\pgfpathlineto{\pgfqpoint{1.944401in}{1.034393in}}%
\pgfpathclose%
\pgfusepath{fill}%
\end{pgfscope}%
\begin{pgfscope}%
\pgfpathrectangle{\pgfqpoint{0.329460in}{0.284240in}}{\pgfqpoint{1.989680in}{1.989680in}}%
\pgfusepath{clip}%
\pgfsetbuttcap%
\pgfsetroundjoin%
\definecolor{currentfill}{rgb}{0.267004,0.004874,0.329415}%
\pgfsetfillcolor{currentfill}%
\pgfsetlinewidth{0.000000pt}%
\definecolor{currentstroke}{rgb}{0.000000,0.000000,0.000000}%
\pgfsetstrokecolor{currentstroke}%
\pgfsetdash{}{0pt}%
\pgfpathmoveto{\pgfqpoint{1.849100in}{0.955514in}}%
\pgfpathlineto{\pgfqpoint{1.852398in}{0.954343in}}%
\pgfpathlineto{\pgfqpoint{1.855702in}{0.953386in}}%
\pgfpathlineto{\pgfqpoint{1.859013in}{0.952647in}}%
\pgfpathlineto{\pgfqpoint{1.862330in}{0.952130in}}%
\pgfpathlineto{\pgfqpoint{1.857848in}{0.943632in}}%
\pgfpathlineto{\pgfqpoint{1.852842in}{0.935202in}}%
\pgfpathlineto{\pgfqpoint{1.847315in}{0.926850in}}%
\pgfpathlineto{\pgfqpoint{1.841271in}{0.918584in}}%
\pgfpathlineto{\pgfqpoint{1.838074in}{0.919322in}}%
\pgfpathlineto{\pgfqpoint{1.834884in}{0.920284in}}%
\pgfpathlineto{\pgfqpoint{1.831700in}{0.921463in}}%
\pgfpathlineto{\pgfqpoint{1.828522in}{0.922857in}}%
\pgfpathlineto{\pgfqpoint{1.834425in}{0.930903in}}%
\pgfpathlineto{\pgfqpoint{1.839825in}{0.939034in}}%
\pgfpathlineto{\pgfqpoint{1.844717in}{0.947240in}}%
\pgfpathlineto{\pgfqpoint{1.849100in}{0.955514in}}%
\pgfpathclose%
\pgfusepath{fill}%
\end{pgfscope}%
\begin{pgfscope}%
\pgfpathrectangle{\pgfqpoint{0.329460in}{0.284240in}}{\pgfqpoint{1.989680in}{1.989680in}}%
\pgfusepath{clip}%
\pgfsetbuttcap%
\pgfsetroundjoin%
\definecolor{currentfill}{rgb}{0.344074,0.780029,0.397381}%
\pgfsetfillcolor{currentfill}%
\pgfsetlinewidth{0.000000pt}%
\definecolor{currentstroke}{rgb}{0.000000,0.000000,0.000000}%
\pgfsetstrokecolor{currentstroke}%
\pgfsetdash{}{0pt}%
\pgfpathmoveto{\pgfqpoint{1.460011in}{1.632997in}}%
\pgfpathlineto{\pgfqpoint{1.461896in}{1.628615in}}%
\pgfpathlineto{\pgfqpoint{1.463778in}{1.624167in}}%
\pgfpathlineto{\pgfqpoint{1.465658in}{1.619656in}}%
\pgfpathlineto{\pgfqpoint{1.467535in}{1.615084in}}%
\pgfpathlineto{\pgfqpoint{1.472863in}{1.613298in}}%
\pgfpathlineto{\pgfqpoint{1.478076in}{1.611433in}}%
\pgfpathlineto{\pgfqpoint{1.483166in}{1.609490in}}%
\pgfpathlineto{\pgfqpoint{1.488130in}{1.607470in}}%
\pgfpathlineto{\pgfqpoint{1.485918in}{1.612171in}}%
\pgfpathlineto{\pgfqpoint{1.483702in}{1.616811in}}%
\pgfpathlineto{\pgfqpoint{1.481484in}{1.621387in}}%
\pgfpathlineto{\pgfqpoint{1.479264in}{1.625898in}}%
\pgfpathlineto{\pgfqpoint{1.474624in}{1.627781in}}%
\pgfpathlineto{\pgfqpoint{1.469865in}{1.629593in}}%
\pgfpathlineto{\pgfqpoint{1.464993in}{1.631332in}}%
\pgfpathlineto{\pgfqpoint{1.460011in}{1.632997in}}%
\pgfpathclose%
\pgfusepath{fill}%
\end{pgfscope}%
\begin{pgfscope}%
\pgfpathrectangle{\pgfqpoint{0.329460in}{0.284240in}}{\pgfqpoint{1.989680in}{1.989680in}}%
\pgfusepath{clip}%
\pgfsetbuttcap%
\pgfsetroundjoin%
\definecolor{currentfill}{rgb}{0.279566,0.067836,0.391917}%
\pgfsetfillcolor{currentfill}%
\pgfsetlinewidth{0.000000pt}%
\definecolor{currentstroke}{rgb}{0.000000,0.000000,0.000000}%
\pgfsetstrokecolor{currentstroke}%
\pgfsetdash{}{0pt}%
\pgfpathmoveto{\pgfqpoint{1.796998in}{0.999980in}}%
\pgfpathlineto{\pgfqpoint{1.800226in}{0.995939in}}%
\pgfpathlineto{\pgfqpoint{1.803456in}{0.992049in}}%
\pgfpathlineto{\pgfqpoint{1.806690in}{0.988312in}}%
\pgfpathlineto{\pgfqpoint{1.809927in}{0.984734in}}%
\pgfpathlineto{\pgfqpoint{1.805844in}{0.977146in}}%
\pgfpathlineto{\pgfqpoint{1.801293in}{0.969621in}}%
\pgfpathlineto{\pgfqpoint{1.796276in}{0.962167in}}%
\pgfpathlineto{\pgfqpoint{1.790796in}{0.954793in}}%
\pgfpathlineto{\pgfqpoint{1.787680in}{0.958599in}}%
\pgfpathlineto{\pgfqpoint{1.784567in}{0.962565in}}%
\pgfpathlineto{\pgfqpoint{1.781458in}{0.966684in}}%
\pgfpathlineto{\pgfqpoint{1.778351in}{0.970954in}}%
\pgfpathlineto{\pgfqpoint{1.783689in}{0.978103in}}%
\pgfpathlineto{\pgfqpoint{1.788579in}{0.985328in}}%
\pgfpathlineto{\pgfqpoint{1.793016in}{0.992623in}}%
\pgfpathlineto{\pgfqpoint{1.796998in}{0.999980in}}%
\pgfpathclose%
\pgfusepath{fill}%
\end{pgfscope}%
\begin{pgfscope}%
\pgfpathrectangle{\pgfqpoint{0.329460in}{0.284240in}}{\pgfqpoint{1.989680in}{1.989680in}}%
\pgfusepath{clip}%
\pgfsetbuttcap%
\pgfsetroundjoin%
\definecolor{currentfill}{rgb}{0.120081,0.622161,0.534946}%
\pgfsetfillcolor{currentfill}%
\pgfsetlinewidth{0.000000pt}%
\definecolor{currentstroke}{rgb}{0.000000,0.000000,0.000000}%
\pgfsetstrokecolor{currentstroke}%
\pgfsetdash{}{0pt}%
\pgfpathmoveto{\pgfqpoint{1.597180in}{1.473284in}}%
\pgfpathlineto{\pgfqpoint{1.600305in}{1.467132in}}%
\pgfpathlineto{\pgfqpoint{1.603427in}{1.460956in}}%
\pgfpathlineto{\pgfqpoint{1.606545in}{1.454757in}}%
\pgfpathlineto{\pgfqpoint{1.609661in}{1.448538in}}%
\pgfpathlineto{\pgfqpoint{1.612569in}{1.444527in}}%
\pgfpathlineto{\pgfqpoint{1.615223in}{1.440470in}}%
\pgfpathlineto{\pgfqpoint{1.617619in}{1.436372in}}%
\pgfpathlineto{\pgfqpoint{1.619753in}{1.432236in}}%
\pgfpathlineto{\pgfqpoint{1.616507in}{1.438669in}}%
\pgfpathlineto{\pgfqpoint{1.613258in}{1.445081in}}%
\pgfpathlineto{\pgfqpoint{1.610006in}{1.451470in}}%
\pgfpathlineto{\pgfqpoint{1.606751in}{1.457834in}}%
\pgfpathlineto{\pgfqpoint{1.604729in}{1.461754in}}%
\pgfpathlineto{\pgfqpoint{1.602457in}{1.465638in}}%
\pgfpathlineto{\pgfqpoint{1.599940in}{1.469482in}}%
\pgfpathlineto{\pgfqpoint{1.597180in}{1.473284in}}%
\pgfpathclose%
\pgfusepath{fill}%
\end{pgfscope}%
\begin{pgfscope}%
\pgfpathrectangle{\pgfqpoint{0.329460in}{0.284240in}}{\pgfqpoint{1.989680in}{1.989680in}}%
\pgfusepath{clip}%
\pgfsetbuttcap%
\pgfsetroundjoin%
\definecolor{currentfill}{rgb}{0.195860,0.395433,0.555276}%
\pgfsetfillcolor{currentfill}%
\pgfsetlinewidth{0.000000pt}%
\definecolor{currentstroke}{rgb}{0.000000,0.000000,0.000000}%
\pgfsetstrokecolor{currentstroke}%
\pgfsetdash{}{0pt}%
\pgfpathmoveto{\pgfqpoint{1.693714in}{1.259936in}}%
\pgfpathlineto{\pgfqpoint{1.697019in}{1.253181in}}%
\pgfpathlineto{\pgfqpoint{1.700322in}{1.246460in}}%
\pgfpathlineto{\pgfqpoint{1.703623in}{1.239778in}}%
\pgfpathlineto{\pgfqpoint{1.706923in}{1.233136in}}%
\pgfpathlineto{\pgfqpoint{1.706604in}{1.227454in}}%
\pgfpathlineto{\pgfqpoint{1.705928in}{1.221775in}}%
\pgfpathlineto{\pgfqpoint{1.704897in}{1.216104in}}%
\pgfpathlineto{\pgfqpoint{1.703509in}{1.210447in}}%
\pgfpathlineto{\pgfqpoint{1.700230in}{1.217324in}}%
\pgfpathlineto{\pgfqpoint{1.696948in}{1.224241in}}%
\pgfpathlineto{\pgfqpoint{1.693666in}{1.231196in}}%
\pgfpathlineto{\pgfqpoint{1.690382in}{1.238185in}}%
\pgfpathlineto{\pgfqpoint{1.691729in}{1.243607in}}%
\pgfpathlineto{\pgfqpoint{1.692734in}{1.249044in}}%
\pgfpathlineto{\pgfqpoint{1.693395in}{1.254488in}}%
\pgfpathlineto{\pgfqpoint{1.693714in}{1.259936in}}%
\pgfpathclose%
\pgfusepath{fill}%
\end{pgfscope}%
\begin{pgfscope}%
\pgfpathrectangle{\pgfqpoint{0.329460in}{0.284240in}}{\pgfqpoint{1.989680in}{1.989680in}}%
\pgfusepath{clip}%
\pgfsetbuttcap%
\pgfsetroundjoin%
\definecolor{currentfill}{rgb}{0.263663,0.237631,0.518762}%
\pgfsetfillcolor{currentfill}%
\pgfsetlinewidth{0.000000pt}%
\definecolor{currentstroke}{rgb}{0.000000,0.000000,0.000000}%
\pgfsetstrokecolor{currentstroke}%
\pgfsetdash{}{0pt}%
\pgfpathmoveto{\pgfqpoint{0.972609in}{1.101457in}}%
\pgfpathlineto{\pgfqpoint{0.969432in}{1.095087in}}%
\pgfpathlineto{\pgfqpoint{0.966255in}{1.088800in}}%
\pgfpathlineto{\pgfqpoint{0.963078in}{1.082598in}}%
\pgfpathlineto{\pgfqpoint{0.959901in}{1.076485in}}%
\pgfpathlineto{\pgfqpoint{0.956371in}{1.082917in}}%
\pgfpathlineto{\pgfqpoint{0.953246in}{1.089396in}}%
\pgfpathlineto{\pgfqpoint{0.950526in}{1.095917in}}%
\pgfpathlineto{\pgfqpoint{0.948214in}{1.102472in}}%
\pgfpathlineto{\pgfqpoint{0.951474in}{1.108351in}}%
\pgfpathlineto{\pgfqpoint{0.954733in}{1.114318in}}%
\pgfpathlineto{\pgfqpoint{0.957992in}{1.120371in}}%
\pgfpathlineto{\pgfqpoint{0.961251in}{1.126506in}}%
\pgfpathlineto{\pgfqpoint{0.963501in}{1.120187in}}%
\pgfpathlineto{\pgfqpoint{0.966145in}{1.113901in}}%
\pgfpathlineto{\pgfqpoint{0.969182in}{1.107656in}}%
\pgfpathlineto{\pgfqpoint{0.972609in}{1.101457in}}%
\pgfpathclose%
\pgfusepath{fill}%
\end{pgfscope}%
\begin{pgfscope}%
\pgfpathrectangle{\pgfqpoint{0.329460in}{0.284240in}}{\pgfqpoint{1.989680in}{1.989680in}}%
\pgfusepath{clip}%
\pgfsetbuttcap%
\pgfsetroundjoin%
\definecolor{currentfill}{rgb}{0.133743,0.548535,0.553541}%
\pgfsetfillcolor{currentfill}%
\pgfsetlinewidth{0.000000pt}%
\definecolor{currentstroke}{rgb}{0.000000,0.000000,0.000000}%
\pgfsetstrokecolor{currentstroke}%
\pgfsetdash{}{0pt}%
\pgfpathmoveto{\pgfqpoint{1.062707in}{1.384623in}}%
\pgfpathlineto{\pgfqpoint{1.059385in}{1.377848in}}%
\pgfpathlineto{\pgfqpoint{1.056065in}{1.371067in}}%
\pgfpathlineto{\pgfqpoint{1.052748in}{1.364282in}}%
\pgfpathlineto{\pgfqpoint{1.049434in}{1.357494in}}%
\pgfpathlineto{\pgfqpoint{1.050358in}{1.362210in}}%
\pgfpathlineto{\pgfqpoint{1.051581in}{1.366905in}}%
\pgfpathlineto{\pgfqpoint{1.053099in}{1.371575in}}%
\pgfpathlineto{\pgfqpoint{1.054910in}{1.376216in}}%
\pgfpathlineto{\pgfqpoint{1.058154in}{1.382778in}}%
\pgfpathlineto{\pgfqpoint{1.061402in}{1.389338in}}%
\pgfpathlineto{\pgfqpoint{1.064652in}{1.395894in}}%
\pgfpathlineto{\pgfqpoint{1.067905in}{1.402444in}}%
\pgfpathlineto{\pgfqpoint{1.066182in}{1.398026in}}%
\pgfpathlineto{\pgfqpoint{1.064741in}{1.393580in}}%
\pgfpathlineto{\pgfqpoint{1.063582in}{1.389111in}}%
\pgfpathlineto{\pgfqpoint{1.062707in}{1.384623in}}%
\pgfpathclose%
\pgfusepath{fill}%
\end{pgfscope}%
\begin{pgfscope}%
\pgfpathrectangle{\pgfqpoint{0.329460in}{0.284240in}}{\pgfqpoint{1.989680in}{1.989680in}}%
\pgfusepath{clip}%
\pgfsetbuttcap%
\pgfsetroundjoin%
\definecolor{currentfill}{rgb}{0.412913,0.803041,0.357269}%
\pgfsetfillcolor{currentfill}%
\pgfsetlinewidth{0.000000pt}%
\definecolor{currentstroke}{rgb}{0.000000,0.000000,0.000000}%
\pgfsetstrokecolor{currentstroke}%
\pgfsetdash{}{0pt}%
\pgfpathmoveto{\pgfqpoint{1.412275in}{1.659570in}}%
\pgfpathlineto{\pgfqpoint{1.413416in}{1.655653in}}%
\pgfpathlineto{\pgfqpoint{1.414555in}{1.651664in}}%
\pgfpathlineto{\pgfqpoint{1.415693in}{1.647606in}}%
\pgfpathlineto{\pgfqpoint{1.416829in}{1.643478in}}%
\pgfpathlineto{\pgfqpoint{1.422502in}{1.642456in}}%
\pgfpathlineto{\pgfqpoint{1.428107in}{1.641349in}}%
\pgfpathlineto{\pgfqpoint{1.433638in}{1.640159in}}%
\pgfpathlineto{\pgfqpoint{1.439091in}{1.638886in}}%
\pgfpathlineto{\pgfqpoint{1.437568in}{1.643096in}}%
\pgfpathlineto{\pgfqpoint{1.436044in}{1.647238in}}%
\pgfpathlineto{\pgfqpoint{1.434517in}{1.651309in}}%
\pgfpathlineto{\pgfqpoint{1.432988in}{1.655308in}}%
\pgfpathlineto{\pgfqpoint{1.427915in}{1.656489in}}%
\pgfpathlineto{\pgfqpoint{1.422769in}{1.657594in}}%
\pgfpathlineto{\pgfqpoint{1.417553in}{1.658621in}}%
\pgfpathlineto{\pgfqpoint{1.412275in}{1.659570in}}%
\pgfpathclose%
\pgfusepath{fill}%
\end{pgfscope}%
\begin{pgfscope}%
\pgfpathrectangle{\pgfqpoint{0.329460in}{0.284240in}}{\pgfqpoint{1.989680in}{1.989680in}}%
\pgfusepath{clip}%
\pgfsetbuttcap%
\pgfsetroundjoin%
\definecolor{currentfill}{rgb}{0.163625,0.471133,0.558148}%
\pgfsetfillcolor{currentfill}%
\pgfsetlinewidth{0.000000pt}%
\definecolor{currentstroke}{rgb}{0.000000,0.000000,0.000000}%
\pgfsetstrokecolor{currentstroke}%
\pgfsetdash{}{0pt}%
\pgfpathmoveto{\pgfqpoint{1.035434in}{1.310484in}}%
\pgfpathlineto{\pgfqpoint{1.032114in}{1.303495in}}%
\pgfpathlineto{\pgfqpoint{1.028795in}{1.296519in}}%
\pgfpathlineto{\pgfqpoint{1.025479in}{1.289559in}}%
\pgfpathlineto{\pgfqpoint{1.022165in}{1.282618in}}%
\pgfpathlineto{\pgfqpoint{1.021884in}{1.287832in}}%
\pgfpathlineto{\pgfqpoint{1.021931in}{1.293044in}}%
\pgfpathlineto{\pgfqpoint{1.022307in}{1.298248in}}%
\pgfpathlineto{\pgfqpoint{1.023009in}{1.303439in}}%
\pgfpathlineto{\pgfqpoint{1.026304in}{1.310148in}}%
\pgfpathlineto{\pgfqpoint{1.029601in}{1.316876in}}%
\pgfpathlineto{\pgfqpoint{1.032900in}{1.323621in}}%
\pgfpathlineto{\pgfqpoint{1.036202in}{1.330379in}}%
\pgfpathlineto{\pgfqpoint{1.035539in}{1.325418in}}%
\pgfpathlineto{\pgfqpoint{1.035189in}{1.320445in}}%
\pgfpathlineto{\pgfqpoint{1.035154in}{1.315466in}}%
\pgfpathlineto{\pgfqpoint{1.035434in}{1.310484in}}%
\pgfpathclose%
\pgfusepath{fill}%
\end{pgfscope}%
\begin{pgfscope}%
\pgfpathrectangle{\pgfqpoint{0.329460in}{0.284240in}}{\pgfqpoint{1.989680in}{1.989680in}}%
\pgfusepath{clip}%
\pgfsetbuttcap%
\pgfsetroundjoin%
\definecolor{currentfill}{rgb}{0.248629,0.278775,0.534556}%
\pgfsetfillcolor{currentfill}%
\pgfsetlinewidth{0.000000pt}%
\definecolor{currentstroke}{rgb}{0.000000,0.000000,0.000000}%
\pgfsetstrokecolor{currentstroke}%
\pgfsetdash{}{0pt}%
\pgfpathmoveto{\pgfqpoint{1.729709in}{1.157239in}}%
\pgfpathlineto{\pgfqpoint{1.732980in}{1.150857in}}%
\pgfpathlineto{\pgfqpoint{1.736251in}{1.144546in}}%
\pgfpathlineto{\pgfqpoint{1.739522in}{1.138308in}}%
\pgfpathlineto{\pgfqpoint{1.742792in}{1.132146in}}%
\pgfpathlineto{\pgfqpoint{1.740893in}{1.125803in}}%
\pgfpathlineto{\pgfqpoint{1.738600in}{1.119487in}}%
\pgfpathlineto{\pgfqpoint{1.735912in}{1.113205in}}%
\pgfpathlineto{\pgfqpoint{1.732832in}{1.106965in}}%
\pgfpathlineto{\pgfqpoint{1.729632in}{1.113362in}}%
\pgfpathlineto{\pgfqpoint{1.726432in}{1.119835in}}%
\pgfpathlineto{\pgfqpoint{1.723233in}{1.126381in}}%
\pgfpathlineto{\pgfqpoint{1.720033in}{1.132997in}}%
\pgfpathlineto{\pgfqpoint{1.723022in}{1.139004in}}%
\pgfpathlineto{\pgfqpoint{1.725632in}{1.145051in}}%
\pgfpathlineto{\pgfqpoint{1.727861in}{1.151131in}}%
\pgfpathlineto{\pgfqpoint{1.729709in}{1.157239in}}%
\pgfpathclose%
\pgfusepath{fill}%
\end{pgfscope}%
\begin{pgfscope}%
\pgfpathrectangle{\pgfqpoint{0.329460in}{0.284240in}}{\pgfqpoint{1.989680in}{1.989680in}}%
\pgfusepath{clip}%
\pgfsetbuttcap%
\pgfsetroundjoin%
\definecolor{currentfill}{rgb}{0.344074,0.780029,0.397381}%
\pgfsetfillcolor{currentfill}%
\pgfsetlinewidth{0.000000pt}%
\definecolor{currentstroke}{rgb}{0.000000,0.000000,0.000000}%
\pgfsetstrokecolor{currentstroke}%
\pgfsetdash{}{0pt}%
\pgfpathmoveto{\pgfqpoint{1.219090in}{1.624166in}}%
\pgfpathlineto{\pgfqpoint{1.216799in}{1.619624in}}%
\pgfpathlineto{\pgfqpoint{1.214511in}{1.615016in}}%
\pgfpathlineto{\pgfqpoint{1.212225in}{1.610345in}}%
\pgfpathlineto{\pgfqpoint{1.209943in}{1.605613in}}%
\pgfpathlineto{\pgfqpoint{1.214790in}{1.607698in}}%
\pgfpathlineto{\pgfqpoint{1.219769in}{1.609709in}}%
\pgfpathlineto{\pgfqpoint{1.224873in}{1.611644in}}%
\pgfpathlineto{\pgfqpoint{1.230098in}{1.613501in}}%
\pgfpathlineto{\pgfqpoint{1.232052in}{1.618100in}}%
\pgfpathlineto{\pgfqpoint{1.234009in}{1.622637in}}%
\pgfpathlineto{\pgfqpoint{1.235969in}{1.627112in}}%
\pgfpathlineto{\pgfqpoint{1.237931in}{1.631521in}}%
\pgfpathlineto{\pgfqpoint{1.233046in}{1.629790in}}%
\pgfpathlineto{\pgfqpoint{1.228274in}{1.627986in}}%
\pgfpathlineto{\pgfqpoint{1.223621in}{1.626111in}}%
\pgfpathlineto{\pgfqpoint{1.219090in}{1.624166in}}%
\pgfpathclose%
\pgfusepath{fill}%
\end{pgfscope}%
\begin{pgfscope}%
\pgfpathrectangle{\pgfqpoint{0.329460in}{0.284240in}}{\pgfqpoint{1.989680in}{1.989680in}}%
\pgfusepath{clip}%
\pgfsetbuttcap%
\pgfsetroundjoin%
\definecolor{currentfill}{rgb}{0.412913,0.803041,0.357269}%
\pgfsetfillcolor{currentfill}%
\pgfsetlinewidth{0.000000pt}%
\definecolor{currentstroke}{rgb}{0.000000,0.000000,0.000000}%
\pgfsetstrokecolor{currentstroke}%
\pgfsetdash{}{0pt}%
\pgfpathmoveto{\pgfqpoint{1.264943in}{1.654195in}}%
\pgfpathlineto{\pgfqpoint{1.263331in}{1.650174in}}%
\pgfpathlineto{\pgfqpoint{1.261721in}{1.646081in}}%
\pgfpathlineto{\pgfqpoint{1.260113in}{1.641918in}}%
\pgfpathlineto{\pgfqpoint{1.258507in}{1.637687in}}%
\pgfpathlineto{\pgfqpoint{1.263886in}{1.639032in}}%
\pgfpathlineto{\pgfqpoint{1.269348in}{1.640295in}}%
\pgfpathlineto{\pgfqpoint{1.274888in}{1.641476in}}%
\pgfpathlineto{\pgfqpoint{1.280500in}{1.642574in}}%
\pgfpathlineto{\pgfqpoint{1.281724in}{1.646717in}}%
\pgfpathlineto{\pgfqpoint{1.282950in}{1.650792in}}%
\pgfpathlineto{\pgfqpoint{1.284177in}{1.654797in}}%
\pgfpathlineto{\pgfqpoint{1.285405in}{1.658730in}}%
\pgfpathlineto{\pgfqpoint{1.280183in}{1.657712in}}%
\pgfpathlineto{\pgfqpoint{1.275028in}{1.656616in}}%
\pgfpathlineto{\pgfqpoint{1.269947in}{1.655443in}}%
\pgfpathlineto{\pgfqpoint{1.264943in}{1.654195in}}%
\pgfpathclose%
\pgfusepath{fill}%
\end{pgfscope}%
\begin{pgfscope}%
\pgfpathrectangle{\pgfqpoint{0.329460in}{0.284240in}}{\pgfqpoint{1.989680in}{1.989680in}}%
\pgfusepath{clip}%
\pgfsetbuttcap%
\pgfsetroundjoin%
\definecolor{currentfill}{rgb}{0.267004,0.004874,0.329415}%
\pgfsetfillcolor{currentfill}%
\pgfsetlinewidth{0.000000pt}%
\definecolor{currentstroke}{rgb}{0.000000,0.000000,0.000000}%
\pgfsetstrokecolor{currentstroke}%
\pgfsetdash{}{0pt}%
\pgfpathmoveto{\pgfqpoint{1.862330in}{0.952130in}}%
\pgfpathlineto{\pgfqpoint{1.865654in}{0.951840in}}%
\pgfpathlineto{\pgfqpoint{1.868985in}{0.951781in}}%
\pgfpathlineto{\pgfqpoint{1.872324in}{0.951958in}}%
\pgfpathlineto{\pgfqpoint{1.875671in}{0.952375in}}%
\pgfpathlineto{\pgfqpoint{1.871089in}{0.943655in}}%
\pgfpathlineto{\pgfqpoint{1.865970in}{0.935005in}}%
\pgfpathlineto{\pgfqpoint{1.860315in}{0.926434in}}%
\pgfpathlineto{\pgfqpoint{1.854129in}{0.917950in}}%
\pgfpathlineto{\pgfqpoint{1.850903in}{0.917752in}}%
\pgfpathlineto{\pgfqpoint{1.847685in}{0.917795in}}%
\pgfpathlineto{\pgfqpoint{1.844474in}{0.918074in}}%
\pgfpathlineto{\pgfqpoint{1.841271in}{0.918584in}}%
\pgfpathlineto{\pgfqpoint{1.847315in}{0.926850in}}%
\pgfpathlineto{\pgfqpoint{1.852842in}{0.935202in}}%
\pgfpathlineto{\pgfqpoint{1.857848in}{0.943632in}}%
\pgfpathlineto{\pgfqpoint{1.862330in}{0.952130in}}%
\pgfpathclose%
\pgfusepath{fill}%
\end{pgfscope}%
\begin{pgfscope}%
\pgfpathrectangle{\pgfqpoint{0.329460in}{0.284240in}}{\pgfqpoint{1.989680in}{1.989680in}}%
\pgfusepath{clip}%
\pgfsetbuttcap%
\pgfsetroundjoin%
\definecolor{currentfill}{rgb}{0.282327,0.094955,0.417331}%
\pgfsetfillcolor{currentfill}%
\pgfsetlinewidth{0.000000pt}%
\definecolor{currentstroke}{rgb}{0.000000,0.000000,0.000000}%
\pgfsetstrokecolor{currentstroke}%
\pgfsetdash{}{0pt}%
\pgfpathmoveto{\pgfqpoint{1.784114in}{1.017576in}}%
\pgfpathlineto{\pgfqpoint{1.787332in}{1.012970in}}%
\pgfpathlineto{\pgfqpoint{1.790551in}{1.008499in}}%
\pgfpathlineto{\pgfqpoint{1.793773in}{1.004168in}}%
\pgfpathlineto{\pgfqpoint{1.796998in}{0.999980in}}%
\pgfpathlineto{\pgfqpoint{1.793016in}{0.992623in}}%
\pgfpathlineto{\pgfqpoint{1.788579in}{0.985328in}}%
\pgfpathlineto{\pgfqpoint{1.783689in}{0.978103in}}%
\pgfpathlineto{\pgfqpoint{1.778351in}{0.970954in}}%
\pgfpathlineto{\pgfqpoint{1.775248in}{0.975372in}}%
\pgfpathlineto{\pgfqpoint{1.772147in}{0.979932in}}%
\pgfpathlineto{\pgfqpoint{1.769049in}{0.984632in}}%
\pgfpathlineto{\pgfqpoint{1.765953in}{0.989469in}}%
\pgfpathlineto{\pgfqpoint{1.771149in}{0.996391in}}%
\pgfpathlineto{\pgfqpoint{1.775910in}{1.003387in}}%
\pgfpathlineto{\pgfqpoint{1.780233in}{1.010452in}}%
\pgfpathlineto{\pgfqpoint{1.784114in}{1.017576in}}%
\pgfpathclose%
\pgfusepath{fill}%
\end{pgfscope}%
\begin{pgfscope}%
\pgfpathrectangle{\pgfqpoint{0.329460in}{0.284240in}}{\pgfqpoint{1.989680in}{1.989680in}}%
\pgfusepath{clip}%
\pgfsetbuttcap%
\pgfsetroundjoin%
\definecolor{currentfill}{rgb}{0.120081,0.622161,0.534946}%
\pgfsetfillcolor{currentfill}%
\pgfsetlinewidth{0.000000pt}%
\definecolor{currentstroke}{rgb}{0.000000,0.000000,0.000000}%
\pgfsetstrokecolor{currentstroke}%
\pgfsetdash{}{0pt}%
\pgfpathmoveto{\pgfqpoint{1.094037in}{1.454324in}}%
\pgfpathlineto{\pgfqpoint{1.090759in}{1.447911in}}%
\pgfpathlineto{\pgfqpoint{1.087485in}{1.441473in}}%
\pgfpathlineto{\pgfqpoint{1.084214in}{1.435013in}}%
\pgfpathlineto{\pgfqpoint{1.080946in}{1.428531in}}%
\pgfpathlineto{\pgfqpoint{1.082846in}{1.432697in}}%
\pgfpathlineto{\pgfqpoint{1.085010in}{1.436829in}}%
\pgfpathlineto{\pgfqpoint{1.087434in}{1.440923in}}%
\pgfpathlineto{\pgfqpoint{1.090116in}{1.444975in}}%
\pgfpathlineto{\pgfqpoint{1.093265in}{1.451241in}}%
\pgfpathlineto{\pgfqpoint{1.096417in}{1.457486in}}%
\pgfpathlineto{\pgfqpoint{1.099572in}{1.463709in}}%
\pgfpathlineto{\pgfqpoint{1.102730in}{1.469907in}}%
\pgfpathlineto{\pgfqpoint{1.100185in}{1.466067in}}%
\pgfpathlineto{\pgfqpoint{1.097887in}{1.462187in}}%
\pgfpathlineto{\pgfqpoint{1.095836in}{1.458272in}}%
\pgfpathlineto{\pgfqpoint{1.094037in}{1.454324in}}%
\pgfpathclose%
\pgfusepath{fill}%
\end{pgfscope}%
\begin{pgfscope}%
\pgfpathrectangle{\pgfqpoint{0.329460in}{0.284240in}}{\pgfqpoint{1.989680in}{1.989680in}}%
\pgfusepath{clip}%
\pgfsetbuttcap%
\pgfsetroundjoin%
\definecolor{currentfill}{rgb}{0.271305,0.019942,0.347269}%
\pgfsetfillcolor{currentfill}%
\pgfsetlinewidth{0.000000pt}%
\definecolor{currentstroke}{rgb}{0.000000,0.000000,0.000000}%
\pgfsetstrokecolor{currentstroke}%
\pgfsetdash{}{0pt}%
\pgfpathmoveto{\pgfqpoint{0.904469in}{0.934542in}}%
\pgfpathlineto{\pgfqpoint{0.901367in}{0.931534in}}%
\pgfpathlineto{\pgfqpoint{0.898261in}{0.928707in}}%
\pgfpathlineto{\pgfqpoint{0.895150in}{0.926066in}}%
\pgfpathlineto{\pgfqpoint{0.892034in}{0.923614in}}%
\pgfpathlineto{\pgfqpoint{0.885841in}{0.931357in}}%
\pgfpathlineto{\pgfqpoint{0.880134in}{0.939191in}}%
\pgfpathlineto{\pgfqpoint{0.874916in}{0.947106in}}%
\pgfpathlineto{\pgfqpoint{0.870192in}{0.955095in}}%
\pgfpathlineto{\pgfqpoint{0.873441in}{0.957322in}}%
\pgfpathlineto{\pgfqpoint{0.876684in}{0.959739in}}%
\pgfpathlineto{\pgfqpoint{0.879923in}{0.962340in}}%
\pgfpathlineto{\pgfqpoint{0.883157in}{0.965123in}}%
\pgfpathlineto{\pgfqpoint{0.887769in}{0.957362in}}%
\pgfpathlineto{\pgfqpoint{0.892861in}{0.949672in}}%
\pgfpathlineto{\pgfqpoint{0.898429in}{0.942063in}}%
\pgfpathlineto{\pgfqpoint{0.904469in}{0.934542in}}%
\pgfpathclose%
\pgfusepath{fill}%
\end{pgfscope}%
\begin{pgfscope}%
\pgfpathrectangle{\pgfqpoint{0.329460in}{0.284240in}}{\pgfqpoint{1.989680in}{1.989680in}}%
\pgfusepath{clip}%
\pgfsetbuttcap%
\pgfsetroundjoin%
\definecolor{currentfill}{rgb}{0.268510,0.009605,0.335427}%
\pgfsetfillcolor{currentfill}%
\pgfsetlinewidth{0.000000pt}%
\definecolor{currentstroke}{rgb}{0.000000,0.000000,0.000000}%
\pgfsetstrokecolor{currentstroke}%
\pgfsetdash{}{0pt}%
\pgfpathmoveto{\pgfqpoint{0.892034in}{0.923614in}}%
\pgfpathlineto{\pgfqpoint{0.888914in}{0.921356in}}%
\pgfpathlineto{\pgfqpoint{0.885788in}{0.919295in}}%
\pgfpathlineto{\pgfqpoint{0.882656in}{0.917437in}}%
\pgfpathlineto{\pgfqpoint{0.879519in}{0.915784in}}%
\pgfpathlineto{\pgfqpoint{0.873172in}{0.923747in}}%
\pgfpathlineto{\pgfqpoint{0.867325in}{0.931803in}}%
\pgfpathlineto{\pgfqpoint{0.861982in}{0.939942in}}%
\pgfpathlineto{\pgfqpoint{0.857146in}{0.948156in}}%
\pgfpathlineto{\pgfqpoint{0.860416in}{0.949587in}}%
\pgfpathlineto{\pgfqpoint{0.863680in}{0.951223in}}%
\pgfpathlineto{\pgfqpoint{0.866939in}{0.953060in}}%
\pgfpathlineto{\pgfqpoint{0.870192in}{0.955095in}}%
\pgfpathlineto{\pgfqpoint{0.874916in}{0.947106in}}%
\pgfpathlineto{\pgfqpoint{0.880134in}{0.939191in}}%
\pgfpathlineto{\pgfqpoint{0.885841in}{0.931357in}}%
\pgfpathlineto{\pgfqpoint{0.892034in}{0.923614in}}%
\pgfpathclose%
\pgfusepath{fill}%
\end{pgfscope}%
\begin{pgfscope}%
\pgfpathrectangle{\pgfqpoint{0.329460in}{0.284240in}}{\pgfqpoint{1.989680in}{1.989680in}}%
\pgfusepath{clip}%
\pgfsetbuttcap%
\pgfsetroundjoin%
\definecolor{currentfill}{rgb}{0.274952,0.037752,0.364543}%
\pgfsetfillcolor{currentfill}%
\pgfsetlinewidth{0.000000pt}%
\definecolor{currentstroke}{rgb}{0.000000,0.000000,0.000000}%
\pgfsetstrokecolor{currentstroke}%
\pgfsetdash{}{0pt}%
\pgfpathmoveto{\pgfqpoint{0.916834in}{0.948310in}}%
\pgfpathlineto{\pgfqpoint{0.913748in}{0.944615in}}%
\pgfpathlineto{\pgfqpoint{0.910659in}{0.941087in}}%
\pgfpathlineto{\pgfqpoint{0.907566in}{0.937728in}}%
\pgfpathlineto{\pgfqpoint{0.904469in}{0.934542in}}%
\pgfpathlineto{\pgfqpoint{0.898429in}{0.942063in}}%
\pgfpathlineto{\pgfqpoint{0.892861in}{0.949672in}}%
\pgfpathlineto{\pgfqpoint{0.887769in}{0.957362in}}%
\pgfpathlineto{\pgfqpoint{0.883157in}{0.965123in}}%
\pgfpathlineto{\pgfqpoint{0.886388in}{0.968083in}}%
\pgfpathlineto{\pgfqpoint{0.889614in}{0.971216in}}%
\pgfpathlineto{\pgfqpoint{0.892836in}{0.974518in}}%
\pgfpathlineto{\pgfqpoint{0.896054in}{0.977986in}}%
\pgfpathlineto{\pgfqpoint{0.900554in}{0.970454in}}%
\pgfpathlineto{\pgfqpoint{0.905519in}{0.962992in}}%
\pgfpathlineto{\pgfqpoint{0.910947in}{0.955608in}}%
\pgfpathlineto{\pgfqpoint{0.916834in}{0.948310in}}%
\pgfpathclose%
\pgfusepath{fill}%
\end{pgfscope}%
\begin{pgfscope}%
\pgfpathrectangle{\pgfqpoint{0.329460in}{0.284240in}}{\pgfqpoint{1.989680in}{1.989680in}}%
\pgfusepath{clip}%
\pgfsetbuttcap%
\pgfsetroundjoin%
\definecolor{currentfill}{rgb}{0.195860,0.395433,0.555276}%
\pgfsetfillcolor{currentfill}%
\pgfsetlinewidth{0.000000pt}%
\definecolor{currentstroke}{rgb}{0.000000,0.000000,0.000000}%
\pgfsetstrokecolor{currentstroke}%
\pgfsetdash{}{0pt}%
\pgfpathmoveto{\pgfqpoint{1.013478in}{1.233382in}}%
\pgfpathlineto{\pgfqpoint{1.010206in}{1.226340in}}%
\pgfpathlineto{\pgfqpoint{1.006935in}{1.219333in}}%
\pgfpathlineto{\pgfqpoint{1.003666in}{1.212364in}}%
\pgfpathlineto{\pgfqpoint{1.000398in}{1.205435in}}%
\pgfpathlineto{\pgfqpoint{0.998694in}{1.211075in}}%
\pgfpathlineto{\pgfqpoint{0.997346in}{1.216733in}}%
\pgfpathlineto{\pgfqpoint{0.996354in}{1.222405in}}%
\pgfpathlineto{\pgfqpoint{0.995718in}{1.228085in}}%
\pgfpathlineto{\pgfqpoint{0.999018in}{1.234779in}}%
\pgfpathlineto{\pgfqpoint{1.002320in}{1.241514in}}%
\pgfpathlineto{\pgfqpoint{1.005623in}{1.248286in}}%
\pgfpathlineto{\pgfqpoint{1.008928in}{1.255094in}}%
\pgfpathlineto{\pgfqpoint{1.009551in}{1.249648in}}%
\pgfpathlineto{\pgfqpoint{1.010518in}{1.244211in}}%
\pgfpathlineto{\pgfqpoint{1.011827in}{1.238787in}}%
\pgfpathlineto{\pgfqpoint{1.013478in}{1.233382in}}%
\pgfpathclose%
\pgfusepath{fill}%
\end{pgfscope}%
\begin{pgfscope}%
\pgfpathrectangle{\pgfqpoint{0.329460in}{0.284240in}}{\pgfqpoint{1.989680in}{1.989680in}}%
\pgfusepath{clip}%
\pgfsetbuttcap%
\pgfsetroundjoin%
\definecolor{currentfill}{rgb}{0.220124,0.725509,0.466226}%
\pgfsetfillcolor{currentfill}%
\pgfsetlinewidth{0.000000pt}%
\definecolor{currentstroke}{rgb}{0.000000,0.000000,0.000000}%
\pgfsetstrokecolor{currentstroke}%
\pgfsetdash{}{0pt}%
\pgfpathmoveto{\pgfqpoint{1.533813in}{1.568160in}}%
\pgfpathlineto{\pgfqpoint{1.536577in}{1.562862in}}%
\pgfpathlineto{\pgfqpoint{1.539338in}{1.557514in}}%
\pgfpathlineto{\pgfqpoint{1.542095in}{1.552117in}}%
\pgfpathlineto{\pgfqpoint{1.544850in}{1.546675in}}%
\pgfpathlineto{\pgfqpoint{1.548947in}{1.543704in}}%
\pgfpathlineto{\pgfqpoint{1.552854in}{1.540671in}}%
\pgfpathlineto{\pgfqpoint{1.556565in}{1.537578in}}%
\pgfpathlineto{\pgfqpoint{1.560077in}{1.534429in}}%
\pgfpathlineto{\pgfqpoint{1.557100in}{1.540057in}}%
\pgfpathlineto{\pgfqpoint{1.554120in}{1.545639in}}%
\pgfpathlineto{\pgfqpoint{1.551136in}{1.551172in}}%
\pgfpathlineto{\pgfqpoint{1.548149in}{1.556655in}}%
\pgfpathlineto{\pgfqpoint{1.544844in}{1.559613in}}%
\pgfpathlineto{\pgfqpoint{1.541350in}{1.562519in}}%
\pgfpathlineto{\pgfqpoint{1.537672in}{1.565368in}}%
\pgfpathlineto{\pgfqpoint{1.533813in}{1.568160in}}%
\pgfpathclose%
\pgfusepath{fill}%
\end{pgfscope}%
\begin{pgfscope}%
\pgfpathrectangle{\pgfqpoint{0.329460in}{0.284240in}}{\pgfqpoint{1.989680in}{1.989680in}}%
\pgfusepath{clip}%
\pgfsetbuttcap%
\pgfsetroundjoin%
\definecolor{currentfill}{rgb}{0.147607,0.511733,0.557049}%
\pgfsetfillcolor{currentfill}%
\pgfsetlinewidth{0.000000pt}%
\definecolor{currentstroke}{rgb}{0.000000,0.000000,0.000000}%
\pgfsetstrokecolor{currentstroke}%
\pgfsetdash{}{0pt}%
\pgfpathmoveto{\pgfqpoint{1.652134in}{1.361687in}}%
\pgfpathlineto{\pgfqpoint{1.655435in}{1.354951in}}%
\pgfpathlineto{\pgfqpoint{1.658733in}{1.348219in}}%
\pgfpathlineto{\pgfqpoint{1.662028in}{1.341492in}}%
\pgfpathlineto{\pgfqpoint{1.665321in}{1.334774in}}%
\pgfpathlineto{\pgfqpoint{1.666262in}{1.329828in}}%
\pgfpathlineto{\pgfqpoint{1.666891in}{1.324866in}}%
\pgfpathlineto{\pgfqpoint{1.667206in}{1.319892in}}%
\pgfpathlineto{\pgfqpoint{1.667206in}{1.314912in}}%
\pgfpathlineto{\pgfqpoint{1.663883in}{1.321860in}}%
\pgfpathlineto{\pgfqpoint{1.660557in}{1.328817in}}%
\pgfpathlineto{\pgfqpoint{1.657229in}{1.335779in}}%
\pgfpathlineto{\pgfqpoint{1.653899in}{1.342744in}}%
\pgfpathlineto{\pgfqpoint{1.653909in}{1.347493in}}%
\pgfpathlineto{\pgfqpoint{1.653618in}{1.352237in}}%
\pgfpathlineto{\pgfqpoint{1.653026in}{1.356969in}}%
\pgfpathlineto{\pgfqpoint{1.652134in}{1.361687in}}%
\pgfpathclose%
\pgfusepath{fill}%
\end{pgfscope}%
\begin{pgfscope}%
\pgfpathrectangle{\pgfqpoint{0.329460in}{0.284240in}}{\pgfqpoint{1.989680in}{1.989680in}}%
\pgfusepath{clip}%
\pgfsetbuttcap%
\pgfsetroundjoin%
\definecolor{currentfill}{rgb}{0.281477,0.755203,0.432552}%
\pgfsetfillcolor{currentfill}%
\pgfsetlinewidth{0.000000pt}%
\definecolor{currentstroke}{rgb}{0.000000,0.000000,0.000000}%
\pgfsetstrokecolor{currentstroke}%
\pgfsetdash{}{0pt}%
\pgfpathmoveto{\pgfqpoint{1.506625in}{1.598669in}}%
\pgfpathlineto{\pgfqpoint{1.509137in}{1.593759in}}%
\pgfpathlineto{\pgfqpoint{1.511646in}{1.588792in}}%
\pgfpathlineto{\pgfqpoint{1.514152in}{1.583768in}}%
\pgfpathlineto{\pgfqpoint{1.516655in}{1.578691in}}%
\pgfpathlineto{\pgfqpoint{1.521194in}{1.576158in}}%
\pgfpathlineto{\pgfqpoint{1.525570in}{1.573557in}}%
\pgfpathlineto{\pgfqpoint{1.529778in}{1.570890in}}%
\pgfpathlineto{\pgfqpoint{1.533813in}{1.568160in}}%
\pgfpathlineto{\pgfqpoint{1.531046in}{1.573406in}}%
\pgfpathlineto{\pgfqpoint{1.528276in}{1.578597in}}%
\pgfpathlineto{\pgfqpoint{1.525502in}{1.583733in}}%
\pgfpathlineto{\pgfqpoint{1.522725in}{1.588811in}}%
\pgfpathlineto{\pgfqpoint{1.518939in}{1.591367in}}%
\pgfpathlineto{\pgfqpoint{1.514992in}{1.593863in}}%
\pgfpathlineto{\pgfqpoint{1.510885in}{1.596298in}}%
\pgfpathlineto{\pgfqpoint{1.506625in}{1.598669in}}%
\pgfpathclose%
\pgfusepath{fill}%
\end{pgfscope}%
\begin{pgfscope}%
\pgfpathrectangle{\pgfqpoint{0.329460in}{0.284240in}}{\pgfqpoint{1.989680in}{1.989680in}}%
\pgfusepath{clip}%
\pgfsetbuttcap%
\pgfsetroundjoin%
\definecolor{currentfill}{rgb}{0.283072,0.130895,0.449241}%
\pgfsetfillcolor{currentfill}%
\pgfsetlinewidth{0.000000pt}%
\definecolor{currentstroke}{rgb}{0.000000,0.000000,0.000000}%
\pgfsetstrokecolor{currentstroke}%
\pgfsetdash{}{0pt}%
\pgfpathmoveto{\pgfqpoint{1.771265in}{1.037292in}}%
\pgfpathlineto{\pgfqpoint{1.774474in}{1.032177in}}%
\pgfpathlineto{\pgfqpoint{1.777686in}{1.027183in}}%
\pgfpathlineto{\pgfqpoint{1.780899in}{1.022315in}}%
\pgfpathlineto{\pgfqpoint{1.784114in}{1.017576in}}%
\pgfpathlineto{\pgfqpoint{1.780233in}{1.010452in}}%
\pgfpathlineto{\pgfqpoint{1.775910in}{1.003387in}}%
\pgfpathlineto{\pgfqpoint{1.771149in}{0.996391in}}%
\pgfpathlineto{\pgfqpoint{1.765953in}{0.989469in}}%
\pgfpathlineto{\pgfqpoint{1.762859in}{0.994438in}}%
\pgfpathlineto{\pgfqpoint{1.759768in}{0.999536in}}%
\pgfpathlineto{\pgfqpoint{1.756678in}{1.004760in}}%
\pgfpathlineto{\pgfqpoint{1.753591in}{1.010106in}}%
\pgfpathlineto{\pgfqpoint{1.758645in}{1.016800in}}%
\pgfpathlineto{\pgfqpoint{1.763278in}{1.023567in}}%
\pgfpathlineto{\pgfqpoint{1.767485in}{1.030400in}}%
\pgfpathlineto{\pgfqpoint{1.771265in}{1.037292in}}%
\pgfpathclose%
\pgfusepath{fill}%
\end{pgfscope}%
\begin{pgfscope}%
\pgfpathrectangle{\pgfqpoint{0.329460in}{0.284240in}}{\pgfqpoint{1.989680in}{1.989680in}}%
\pgfusepath{clip}%
\pgfsetbuttcap%
\pgfsetroundjoin%
\definecolor{currentfill}{rgb}{0.268510,0.009605,0.335427}%
\pgfsetfillcolor{currentfill}%
\pgfsetlinewidth{0.000000pt}%
\definecolor{currentstroke}{rgb}{0.000000,0.000000,0.000000}%
\pgfsetstrokecolor{currentstroke}%
\pgfsetdash{}{0pt}%
\pgfpathmoveto{\pgfqpoint{1.875671in}{0.952375in}}%
\pgfpathlineto{\pgfqpoint{1.879025in}{0.953036in}}%
\pgfpathlineto{\pgfqpoint{1.882388in}{0.953947in}}%
\pgfpathlineto{\pgfqpoint{1.885759in}{0.955112in}}%
\pgfpathlineto{\pgfqpoint{1.889139in}{0.956535in}}%
\pgfpathlineto{\pgfqpoint{1.884458in}{0.947597in}}%
\pgfpathlineto{\pgfqpoint{1.879225in}{0.938729in}}%
\pgfpathlineto{\pgfqpoint{1.873442in}{0.929942in}}%
\pgfpathlineto{\pgfqpoint{1.867113in}{0.921244in}}%
\pgfpathlineto{\pgfqpoint{1.863854in}{0.920036in}}%
\pgfpathlineto{\pgfqpoint{1.860604in}{0.919088in}}%
\pgfpathlineto{\pgfqpoint{1.857362in}{0.918394in}}%
\pgfpathlineto{\pgfqpoint{1.854129in}{0.917950in}}%
\pgfpathlineto{\pgfqpoint{1.860315in}{0.926434in}}%
\pgfpathlineto{\pgfqpoint{1.865970in}{0.935005in}}%
\pgfpathlineto{\pgfqpoint{1.871089in}{0.943655in}}%
\pgfpathlineto{\pgfqpoint{1.875671in}{0.952375in}}%
\pgfpathclose%
\pgfusepath{fill}%
\end{pgfscope}%
\begin{pgfscope}%
\pgfpathrectangle{\pgfqpoint{0.329460in}{0.284240in}}{\pgfqpoint{1.989680in}{1.989680in}}%
\pgfusepath{clip}%
\pgfsetbuttcap%
\pgfsetroundjoin%
\definecolor{currentfill}{rgb}{0.412913,0.803041,0.357269}%
\pgfsetfillcolor{currentfill}%
\pgfsetlinewidth{0.000000pt}%
\definecolor{currentstroke}{rgb}{0.000000,0.000000,0.000000}%
\pgfsetstrokecolor{currentstroke}%
\pgfsetdash{}{0pt}%
\pgfpathmoveto{\pgfqpoint{1.432988in}{1.655308in}}%
\pgfpathlineto{\pgfqpoint{1.434517in}{1.651309in}}%
\pgfpathlineto{\pgfqpoint{1.436044in}{1.647238in}}%
\pgfpathlineto{\pgfqpoint{1.437568in}{1.643096in}}%
\pgfpathlineto{\pgfqpoint{1.439091in}{1.638886in}}%
\pgfpathlineto{\pgfqpoint{1.444460in}{1.637533in}}%
\pgfpathlineto{\pgfqpoint{1.449740in}{1.636099in}}%
\pgfpathlineto{\pgfqpoint{1.454925in}{1.634587in}}%
\pgfpathlineto{\pgfqpoint{1.460011in}{1.632997in}}%
\pgfpathlineto{\pgfqpoint{1.458124in}{1.637314in}}%
\pgfpathlineto{\pgfqpoint{1.456235in}{1.641561in}}%
\pgfpathlineto{\pgfqpoint{1.454343in}{1.645738in}}%
\pgfpathlineto{\pgfqpoint{1.452449in}{1.649844in}}%
\pgfpathlineto{\pgfqpoint{1.447718in}{1.651319in}}%
\pgfpathlineto{\pgfqpoint{1.442895in}{1.652722in}}%
\pgfpathlineto{\pgfqpoint{1.437983in}{1.654052in}}%
\pgfpathlineto{\pgfqpoint{1.432988in}{1.655308in}}%
\pgfpathclose%
\pgfusepath{fill}%
\end{pgfscope}%
\begin{pgfscope}%
\pgfpathrectangle{\pgfqpoint{0.329460in}{0.284240in}}{\pgfqpoint{1.989680in}{1.989680in}}%
\pgfusepath{clip}%
\pgfsetbuttcap%
\pgfsetroundjoin%
\definecolor{currentfill}{rgb}{0.279566,0.067836,0.391917}%
\pgfsetfillcolor{currentfill}%
\pgfsetlinewidth{0.000000pt}%
\definecolor{currentstroke}{rgb}{0.000000,0.000000,0.000000}%
\pgfsetstrokecolor{currentstroke}%
\pgfsetdash{}{0pt}%
\pgfpathmoveto{\pgfqpoint{0.929142in}{0.964671in}}%
\pgfpathlineto{\pgfqpoint{0.926069in}{0.960351in}}%
\pgfpathlineto{\pgfqpoint{0.922994in}{0.956182in}}%
\pgfpathlineto{\pgfqpoint{0.919916in}{0.952167in}}%
\pgfpathlineto{\pgfqpoint{0.916834in}{0.948310in}}%
\pgfpathlineto{\pgfqpoint{0.910947in}{0.955608in}}%
\pgfpathlineto{\pgfqpoint{0.905519in}{0.962992in}}%
\pgfpathlineto{\pgfqpoint{0.900554in}{0.970454in}}%
\pgfpathlineto{\pgfqpoint{0.896054in}{0.977986in}}%
\pgfpathlineto{\pgfqpoint{0.899269in}{0.981616in}}%
\pgfpathlineto{\pgfqpoint{0.902480in}{0.985404in}}%
\pgfpathlineto{\pgfqpoint{0.905689in}{0.989345in}}%
\pgfpathlineto{\pgfqpoint{0.908894in}{0.993438in}}%
\pgfpathlineto{\pgfqpoint{0.913281in}{0.986136in}}%
\pgfpathlineto{\pgfqpoint{0.918120in}{0.978902in}}%
\pgfpathlineto{\pgfqpoint{0.923409in}{0.971745in}}%
\pgfpathlineto{\pgfqpoint{0.929142in}{0.964671in}}%
\pgfpathclose%
\pgfusepath{fill}%
\end{pgfscope}%
\begin{pgfscope}%
\pgfpathrectangle{\pgfqpoint{0.329460in}{0.284240in}}{\pgfqpoint{1.989680in}{1.989680in}}%
\pgfusepath{clip}%
\pgfsetbuttcap%
\pgfsetroundjoin%
\definecolor{currentfill}{rgb}{0.267004,0.004874,0.329415}%
\pgfsetfillcolor{currentfill}%
\pgfsetlinewidth{0.000000pt}%
\definecolor{currentstroke}{rgb}{0.000000,0.000000,0.000000}%
\pgfsetstrokecolor{currentstroke}%
\pgfsetdash{}{0pt}%
\pgfpathmoveto{\pgfqpoint{0.879519in}{0.915784in}}%
\pgfpathlineto{\pgfqpoint{0.876376in}{0.914341in}}%
\pgfpathlineto{\pgfqpoint{0.873226in}{0.913113in}}%
\pgfpathlineto{\pgfqpoint{0.870070in}{0.912103in}}%
\pgfpathlineto{\pgfqpoint{0.866908in}{0.911317in}}%
\pgfpathlineto{\pgfqpoint{0.860408in}{0.919498in}}%
\pgfpathlineto{\pgfqpoint{0.854421in}{0.927774in}}%
\pgfpathlineto{\pgfqpoint{0.848951in}{0.936135in}}%
\pgfpathlineto{\pgfqpoint{0.844003in}{0.944573in}}%
\pgfpathlineto{\pgfqpoint{0.847299in}{0.945139in}}%
\pgfpathlineto{\pgfqpoint{0.850587in}{0.945928in}}%
\pgfpathlineto{\pgfqpoint{0.853870in}{0.946935in}}%
\pgfpathlineto{\pgfqpoint{0.857146in}{0.948156in}}%
\pgfpathlineto{\pgfqpoint{0.861982in}{0.939942in}}%
\pgfpathlineto{\pgfqpoint{0.867325in}{0.931803in}}%
\pgfpathlineto{\pgfqpoint{0.873172in}{0.923747in}}%
\pgfpathlineto{\pgfqpoint{0.879519in}{0.915784in}}%
\pgfpathclose%
\pgfusepath{fill}%
\end{pgfscope}%
\begin{pgfscope}%
\pgfpathrectangle{\pgfqpoint{0.329460in}{0.284240in}}{\pgfqpoint{1.989680in}{1.989680in}}%
\pgfusepath{clip}%
\pgfsetbuttcap%
\pgfsetroundjoin%
\definecolor{currentfill}{rgb}{0.248629,0.278775,0.534556}%
\pgfsetfillcolor{currentfill}%
\pgfsetlinewidth{0.000000pt}%
\definecolor{currentstroke}{rgb}{0.000000,0.000000,0.000000}%
\pgfsetstrokecolor{currentstroke}%
\pgfsetdash{}{0pt}%
\pgfpathmoveto{\pgfqpoint{0.985316in}{1.127696in}}%
\pgfpathlineto{\pgfqpoint{0.982139in}{1.121028in}}%
\pgfpathlineto{\pgfqpoint{0.978962in}{1.114431in}}%
\pgfpathlineto{\pgfqpoint{0.975786in}{1.107906in}}%
\pgfpathlineto{\pgfqpoint{0.972609in}{1.101457in}}%
\pgfpathlineto{\pgfqpoint{0.969182in}{1.107656in}}%
\pgfpathlineto{\pgfqpoint{0.966145in}{1.113901in}}%
\pgfpathlineto{\pgfqpoint{0.963501in}{1.120187in}}%
\pgfpathlineto{\pgfqpoint{0.961251in}{1.126506in}}%
\pgfpathlineto{\pgfqpoint{0.964510in}{1.132721in}}%
\pgfpathlineto{\pgfqpoint{0.967770in}{1.139011in}}%
\pgfpathlineto{\pgfqpoint{0.971030in}{1.145375in}}%
\pgfpathlineto{\pgfqpoint{0.974290in}{1.151809in}}%
\pgfpathlineto{\pgfqpoint{0.976477in}{1.145725in}}%
\pgfpathlineto{\pgfqpoint{0.979045in}{1.139674in}}%
\pgfpathlineto{\pgfqpoint{0.981992in}{1.133663in}}%
\pgfpathlineto{\pgfqpoint{0.985316in}{1.127696in}}%
\pgfpathclose%
\pgfusepath{fill}%
\end{pgfscope}%
\begin{pgfscope}%
\pgfpathrectangle{\pgfqpoint{0.329460in}{0.284240in}}{\pgfqpoint{1.989680in}{1.989680in}}%
\pgfusepath{clip}%
\pgfsetbuttcap%
\pgfsetroundjoin%
\definecolor{currentfill}{rgb}{0.233603,0.313828,0.543914}%
\pgfsetfillcolor{currentfill}%
\pgfsetlinewidth{0.000000pt}%
\definecolor{currentstroke}{rgb}{0.000000,0.000000,0.000000}%
\pgfsetstrokecolor{currentstroke}%
\pgfsetdash{}{0pt}%
\pgfpathmoveto{\pgfqpoint{0.716365in}{1.113567in}}%
\pgfpathlineto{\pgfqpoint{0.712605in}{1.123317in}}%
\pgfpathlineto{\pgfqpoint{0.708826in}{1.133471in}}%
\pgfpathlineto{\pgfqpoint{0.705028in}{1.144036in}}%
\pgfpathlineto{\pgfqpoint{0.701212in}{1.155017in}}%
\pgfpathlineto{\pgfqpoint{0.698552in}{1.165541in}}%
\pgfpathlineto{\pgfqpoint{0.696557in}{1.176086in}}%
\pgfpathlineto{\pgfqpoint{0.695224in}{1.186639in}}%
\pgfpathlineto{\pgfqpoint{0.694552in}{1.197191in}}%
\pgfpathlineto{\pgfqpoint{0.698382in}{1.186036in}}%
\pgfpathlineto{\pgfqpoint{0.702193in}{1.175296in}}%
\pgfpathlineto{\pgfqpoint{0.705986in}{1.164964in}}%
\pgfpathlineto{\pgfqpoint{0.709761in}{1.155033in}}%
\pgfpathlineto{\pgfqpoint{0.710443in}{1.144657in}}%
\pgfpathlineto{\pgfqpoint{0.711769in}{1.134280in}}%
\pgfpathlineto{\pgfqpoint{0.713743in}{1.123913in}}%
\pgfpathlineto{\pgfqpoint{0.716365in}{1.113567in}}%
\pgfpathclose%
\pgfusepath{fill}%
\end{pgfscope}%
\begin{pgfscope}%
\pgfpathrectangle{\pgfqpoint{0.329460in}{0.284240in}}{\pgfqpoint{1.989680in}{1.989680in}}%
\pgfusepath{clip}%
\pgfsetbuttcap%
\pgfsetroundjoin%
\definecolor{currentfill}{rgb}{0.122606,0.585371,0.546557}%
\pgfsetfillcolor{currentfill}%
\pgfsetlinewidth{0.000000pt}%
\definecolor{currentstroke}{rgb}{0.000000,0.000000,0.000000}%
\pgfsetstrokecolor{currentstroke}%
\pgfsetdash{}{0pt}%
\pgfpathmoveto{\pgfqpoint{1.619753in}{1.432236in}}%
\pgfpathlineto{\pgfqpoint{1.622996in}{1.425785in}}%
\pgfpathlineto{\pgfqpoint{1.626235in}{1.419317in}}%
\pgfpathlineto{\pgfqpoint{1.629472in}{1.412837in}}%
\pgfpathlineto{\pgfqpoint{1.632705in}{1.406345in}}%
\pgfpathlineto{\pgfqpoint{1.634676in}{1.401955in}}%
\pgfpathlineto{\pgfqpoint{1.636367in}{1.397534in}}%
\pgfpathlineto{\pgfqpoint{1.637777in}{1.393085in}}%
\pgfpathlineto{\pgfqpoint{1.638905in}{1.388613in}}%
\pgfpathlineto{\pgfqpoint{1.635590in}{1.395328in}}%
\pgfpathlineto{\pgfqpoint{1.632273in}{1.402032in}}%
\pgfpathlineto{\pgfqpoint{1.628953in}{1.408721in}}%
\pgfpathlineto{\pgfqpoint{1.625630in}{1.415394in}}%
\pgfpathlineto{\pgfqpoint{1.624564in}{1.419641in}}%
\pgfpathlineto{\pgfqpoint{1.623228in}{1.423867in}}%
\pgfpathlineto{\pgfqpoint{1.621624in}{1.428066in}}%
\pgfpathlineto{\pgfqpoint{1.619753in}{1.432236in}}%
\pgfpathclose%
\pgfusepath{fill}%
\end{pgfscope}%
\begin{pgfscope}%
\pgfpathrectangle{\pgfqpoint{0.329460in}{0.284240in}}{\pgfqpoint{1.989680in}{1.989680in}}%
\pgfusepath{clip}%
\pgfsetbuttcap%
\pgfsetroundjoin%
\definecolor{currentfill}{rgb}{0.166383,0.690856,0.496502}%
\pgfsetfillcolor{currentfill}%
\pgfsetlinewidth{0.000000pt}%
\definecolor{currentstroke}{rgb}{0.000000,0.000000,0.000000}%
\pgfsetstrokecolor{currentstroke}%
\pgfsetdash{}{0pt}%
\pgfpathmoveto{\pgfqpoint{1.560077in}{1.534429in}}%
\pgfpathlineto{\pgfqpoint{1.563051in}{1.528757in}}%
\pgfpathlineto{\pgfqpoint{1.566021in}{1.523042in}}%
\pgfpathlineto{\pgfqpoint{1.568988in}{1.517288in}}%
\pgfpathlineto{\pgfqpoint{1.571952in}{1.511495in}}%
\pgfpathlineto{\pgfqpoint{1.575456in}{1.508097in}}%
\pgfpathlineto{\pgfqpoint{1.578742in}{1.504646in}}%
\pgfpathlineto{\pgfqpoint{1.581807in}{1.501143in}}%
\pgfpathlineto{\pgfqpoint{1.584647in}{1.497594in}}%
\pgfpathlineto{\pgfqpoint{1.581506in}{1.503587in}}%
\pgfpathlineto{\pgfqpoint{1.578361in}{1.509542in}}%
\pgfpathlineto{\pgfqpoint{1.575213in}{1.515456in}}%
\pgfpathlineto{\pgfqpoint{1.572062in}{1.521328in}}%
\pgfpathlineto{\pgfqpoint{1.569382in}{1.524673in}}%
\pgfpathlineto{\pgfqpoint{1.566489in}{1.527973in}}%
\pgfpathlineto{\pgfqpoint{1.563386in}{1.531226in}}%
\pgfpathlineto{\pgfqpoint{1.560077in}{1.534429in}}%
\pgfpathclose%
\pgfusepath{fill}%
\end{pgfscope}%
\begin{pgfscope}%
\pgfpathrectangle{\pgfqpoint{0.329460in}{0.284240in}}{\pgfqpoint{1.989680in}{1.989680in}}%
\pgfusepath{clip}%
\pgfsetbuttcap%
\pgfsetroundjoin%
\definecolor{currentfill}{rgb}{0.487026,0.823929,0.312321}%
\pgfsetfillcolor{currentfill}%
\pgfsetlinewidth{0.000000pt}%
\definecolor{currentstroke}{rgb}{0.000000,0.000000,0.000000}%
\pgfsetstrokecolor{currentstroke}%
\pgfsetdash{}{0pt}%
\pgfpathmoveto{\pgfqpoint{1.330678in}{1.678568in}}%
\pgfpathlineto{\pgfqpoint{1.330262in}{1.675040in}}%
\pgfpathlineto{\pgfqpoint{1.329847in}{1.671434in}}%
\pgfpathlineto{\pgfqpoint{1.329432in}{1.667751in}}%
\pgfpathlineto{\pgfqpoint{1.329018in}{1.663993in}}%
\pgfpathlineto{\pgfqpoint{1.334620in}{1.664281in}}%
\pgfpathlineto{\pgfqpoint{1.340238in}{1.664486in}}%
\pgfpathlineto{\pgfqpoint{1.345867in}{1.664607in}}%
\pgfpathlineto{\pgfqpoint{1.351501in}{1.664644in}}%
\pgfpathlineto{\pgfqpoint{1.351495in}{1.668390in}}%
\pgfpathlineto{\pgfqpoint{1.351489in}{1.672060in}}%
\pgfpathlineto{\pgfqpoint{1.351483in}{1.675653in}}%
\pgfpathlineto{\pgfqpoint{1.351477in}{1.679168in}}%
\pgfpathlineto{\pgfqpoint{1.346265in}{1.679134in}}%
\pgfpathlineto{\pgfqpoint{1.341058in}{1.679022in}}%
\pgfpathlineto{\pgfqpoint{1.335861in}{1.678833in}}%
\pgfpathlineto{\pgfqpoint{1.330678in}{1.678568in}}%
\pgfpathclose%
\pgfusepath{fill}%
\end{pgfscope}%
\begin{pgfscope}%
\pgfpathrectangle{\pgfqpoint{0.329460in}{0.284240in}}{\pgfqpoint{1.989680in}{1.989680in}}%
\pgfusepath{clip}%
\pgfsetbuttcap%
\pgfsetroundjoin%
\definecolor{currentfill}{rgb}{0.487026,0.823929,0.312321}%
\pgfsetfillcolor{currentfill}%
\pgfsetlinewidth{0.000000pt}%
\definecolor{currentstroke}{rgb}{0.000000,0.000000,0.000000}%
\pgfsetstrokecolor{currentstroke}%
\pgfsetdash{}{0pt}%
\pgfpathmoveto{\pgfqpoint{1.351477in}{1.679168in}}%
\pgfpathlineto{\pgfqpoint{1.351483in}{1.675653in}}%
\pgfpathlineto{\pgfqpoint{1.351489in}{1.672060in}}%
\pgfpathlineto{\pgfqpoint{1.351495in}{1.668390in}}%
\pgfpathlineto{\pgfqpoint{1.351501in}{1.664644in}}%
\pgfpathlineto{\pgfqpoint{1.357134in}{1.664598in}}%
\pgfpathlineto{\pgfqpoint{1.362762in}{1.664467in}}%
\pgfpathlineto{\pgfqpoint{1.368378in}{1.664253in}}%
\pgfpathlineto{\pgfqpoint{1.373979in}{1.663956in}}%
\pgfpathlineto{\pgfqpoint{1.373553in}{1.667715in}}%
\pgfpathlineto{\pgfqpoint{1.373126in}{1.671398in}}%
\pgfpathlineto{\pgfqpoint{1.372699in}{1.675005in}}%
\pgfpathlineto{\pgfqpoint{1.372272in}{1.678533in}}%
\pgfpathlineto{\pgfqpoint{1.367091in}{1.678808in}}%
\pgfpathlineto{\pgfqpoint{1.361895in}{1.679005in}}%
\pgfpathlineto{\pgfqpoint{1.356689in}{1.679125in}}%
\pgfpathlineto{\pgfqpoint{1.351477in}{1.679168in}}%
\pgfpathclose%
\pgfusepath{fill}%
\end{pgfscope}%
\begin{pgfscope}%
\pgfpathrectangle{\pgfqpoint{0.329460in}{0.284240in}}{\pgfqpoint{1.989680in}{1.989680in}}%
\pgfusepath{clip}%
\pgfsetbuttcap%
\pgfsetroundjoin%
\definecolor{currentfill}{rgb}{0.412913,0.803041,0.357269}%
\pgfsetfillcolor{currentfill}%
\pgfsetlinewidth{0.000000pt}%
\definecolor{currentstroke}{rgb}{0.000000,0.000000,0.000000}%
\pgfsetstrokecolor{currentstroke}%
\pgfsetdash{}{0pt}%
\pgfpathmoveto{\pgfqpoint{1.245803in}{1.648474in}}%
\pgfpathlineto{\pgfqpoint{1.243831in}{1.644342in}}%
\pgfpathlineto{\pgfqpoint{1.241862in}{1.640138in}}%
\pgfpathlineto{\pgfqpoint{1.239895in}{1.635864in}}%
\pgfpathlineto{\pgfqpoint{1.237931in}{1.631521in}}%
\pgfpathlineto{\pgfqpoint{1.242924in}{1.633178in}}%
\pgfpathlineto{\pgfqpoint{1.248021in}{1.634759in}}%
\pgfpathlineto{\pgfqpoint{1.253217in}{1.636262in}}%
\pgfpathlineto{\pgfqpoint{1.258507in}{1.637687in}}%
\pgfpathlineto{\pgfqpoint{1.260113in}{1.641918in}}%
\pgfpathlineto{\pgfqpoint{1.261721in}{1.646081in}}%
\pgfpathlineto{\pgfqpoint{1.263331in}{1.650174in}}%
\pgfpathlineto{\pgfqpoint{1.264943in}{1.654195in}}%
\pgfpathlineto{\pgfqpoint{1.260022in}{1.652873in}}%
\pgfpathlineto{\pgfqpoint{1.255189in}{1.651478in}}%
\pgfpathlineto{\pgfqpoint{1.250447in}{1.650011in}}%
\pgfpathlineto{\pgfqpoint{1.245803in}{1.648474in}}%
\pgfpathclose%
\pgfusepath{fill}%
\end{pgfscope}%
\begin{pgfscope}%
\pgfpathrectangle{\pgfqpoint{0.329460in}{0.284240in}}{\pgfqpoint{1.989680in}{1.989680in}}%
\pgfusepath{clip}%
\pgfsetbuttcap%
\pgfsetroundjoin%
\definecolor{currentfill}{rgb}{0.282884,0.135920,0.453427}%
\pgfsetfillcolor{currentfill}%
\pgfsetlinewidth{0.000000pt}%
\definecolor{currentstroke}{rgb}{0.000000,0.000000,0.000000}%
\pgfsetstrokecolor{currentstroke}%
\pgfsetdash{}{0pt}%
\pgfpathmoveto{\pgfqpoint{0.776283in}{0.987075in}}%
\pgfpathlineto{\pgfqpoint{0.772784in}{0.992272in}}%
\pgfpathlineto{\pgfqpoint{0.769271in}{0.997800in}}%
\pgfpathlineto{\pgfqpoint{0.765745in}{1.003665in}}%
\pgfpathlineto{\pgfqpoint{0.762205in}{1.009872in}}%
\pgfpathlineto{\pgfqpoint{0.757192in}{1.019649in}}%
\pgfpathlineto{\pgfqpoint{0.752794in}{1.029488in}}%
\pgfpathlineto{\pgfqpoint{0.749013in}{1.039380in}}%
\pgfpathlineto{\pgfqpoint{0.745849in}{1.049314in}}%
\pgfpathlineto{\pgfqpoint{0.749465in}{1.042908in}}%
\pgfpathlineto{\pgfqpoint{0.753068in}{1.036843in}}%
\pgfpathlineto{\pgfqpoint{0.756657in}{1.031114in}}%
\pgfpathlineto{\pgfqpoint{0.760232in}{1.025714in}}%
\pgfpathlineto{\pgfqpoint{0.763343in}{1.015982in}}%
\pgfpathlineto{\pgfqpoint{0.767055in}{1.006291in}}%
\pgfpathlineto{\pgfqpoint{0.771370in}{0.996652in}}%
\pgfpathlineto{\pgfqpoint{0.776283in}{0.987075in}}%
\pgfpathclose%
\pgfusepath{fill}%
\end{pgfscope}%
\begin{pgfscope}%
\pgfpathrectangle{\pgfqpoint{0.329460in}{0.284240in}}{\pgfqpoint{1.989680in}{1.989680in}}%
\pgfusepath{clip}%
\pgfsetbuttcap%
\pgfsetroundjoin%
\definecolor{currentfill}{rgb}{0.231674,0.318106,0.544834}%
\pgfsetfillcolor{currentfill}%
\pgfsetlinewidth{0.000000pt}%
\definecolor{currentstroke}{rgb}{0.000000,0.000000,0.000000}%
\pgfsetstrokecolor{currentstroke}%
\pgfsetdash{}{0pt}%
\pgfpathmoveto{\pgfqpoint{1.716617in}{1.183403in}}%
\pgfpathlineto{\pgfqpoint{1.719891in}{1.176772in}}%
\pgfpathlineto{\pgfqpoint{1.723164in}{1.170199in}}%
\pgfpathlineto{\pgfqpoint{1.726437in}{1.163687in}}%
\pgfpathlineto{\pgfqpoint{1.729709in}{1.157239in}}%
\pgfpathlineto{\pgfqpoint{1.727861in}{1.151131in}}%
\pgfpathlineto{\pgfqpoint{1.725632in}{1.145051in}}%
\pgfpathlineto{\pgfqpoint{1.723022in}{1.139004in}}%
\pgfpathlineto{\pgfqpoint{1.720033in}{1.132997in}}%
\pgfpathlineto{\pgfqpoint{1.716832in}{1.139680in}}%
\pgfpathlineto{\pgfqpoint{1.713631in}{1.146427in}}%
\pgfpathlineto{\pgfqpoint{1.710430in}{1.153234in}}%
\pgfpathlineto{\pgfqpoint{1.707228in}{1.160099in}}%
\pgfpathlineto{\pgfqpoint{1.710125in}{1.165873in}}%
\pgfpathlineto{\pgfqpoint{1.712656in}{1.171686in}}%
\pgfpathlineto{\pgfqpoint{1.714821in}{1.177531in}}%
\pgfpathlineto{\pgfqpoint{1.716617in}{1.183403in}}%
\pgfpathclose%
\pgfusepath{fill}%
\end{pgfscope}%
\begin{pgfscope}%
\pgfpathrectangle{\pgfqpoint{0.329460in}{0.284240in}}{\pgfqpoint{1.989680in}{1.989680in}}%
\pgfusepath{clip}%
\pgfsetbuttcap%
\pgfsetroundjoin%
\definecolor{currentfill}{rgb}{0.220124,0.725509,0.466226}%
\pgfsetfillcolor{currentfill}%
\pgfsetlinewidth{0.000000pt}%
\definecolor{currentstroke}{rgb}{0.000000,0.000000,0.000000}%
\pgfsetstrokecolor{currentstroke}%
\pgfsetdash{}{0pt}%
\pgfpathmoveto{\pgfqpoint{1.151448in}{1.553983in}}%
\pgfpathlineto{\pgfqpoint{1.148418in}{1.548457in}}%
\pgfpathlineto{\pgfqpoint{1.145391in}{1.542881in}}%
\pgfpathlineto{\pgfqpoint{1.142367in}{1.537256in}}%
\pgfpathlineto{\pgfqpoint{1.139347in}{1.531585in}}%
\pgfpathlineto{\pgfqpoint{1.142678in}{1.534782in}}%
\pgfpathlineto{\pgfqpoint{1.146213in}{1.537925in}}%
\pgfpathlineto{\pgfqpoint{1.149946in}{1.541011in}}%
\pgfpathlineto{\pgfqpoint{1.153874in}{1.544037in}}%
\pgfpathlineto{\pgfqpoint{1.156681in}{1.549519in}}%
\pgfpathlineto{\pgfqpoint{1.159492in}{1.554956in}}%
\pgfpathlineto{\pgfqpoint{1.162306in}{1.560344in}}%
\pgfpathlineto{\pgfqpoint{1.165123in}{1.565681in}}%
\pgfpathlineto{\pgfqpoint{1.161425in}{1.562838in}}%
\pgfpathlineto{\pgfqpoint{1.157910in}{1.559939in}}%
\pgfpathlineto{\pgfqpoint{1.154584in}{1.556986in}}%
\pgfpathlineto{\pgfqpoint{1.151448in}{1.553983in}}%
\pgfpathclose%
\pgfusepath{fill}%
\end{pgfscope}%
\begin{pgfscope}%
\pgfpathrectangle{\pgfqpoint{0.329460in}{0.284240in}}{\pgfqpoint{1.989680in}{1.989680in}}%
\pgfusepath{clip}%
\pgfsetbuttcap%
\pgfsetroundjoin%
\definecolor{currentfill}{rgb}{0.281477,0.755203,0.432552}%
\pgfsetfillcolor{currentfill}%
\pgfsetlinewidth{0.000000pt}%
\definecolor{currentstroke}{rgb}{0.000000,0.000000,0.000000}%
\pgfsetstrokecolor{currentstroke}%
\pgfsetdash{}{0pt}%
\pgfpathmoveto{\pgfqpoint{1.176424in}{1.586491in}}%
\pgfpathlineto{\pgfqpoint{1.173594in}{1.581374in}}%
\pgfpathlineto{\pgfqpoint{1.170767in}{1.576199in}}%
\pgfpathlineto{\pgfqpoint{1.167944in}{1.570967in}}%
\pgfpathlineto{\pgfqpoint{1.165123in}{1.565681in}}%
\pgfpathlineto{\pgfqpoint{1.169002in}{1.568466in}}%
\pgfpathlineto{\pgfqpoint{1.173056in}{1.571190in}}%
\pgfpathlineto{\pgfqpoint{1.177283in}{1.573849in}}%
\pgfpathlineto{\pgfqpoint{1.181677in}{1.576443in}}%
\pgfpathlineto{\pgfqpoint{1.184242in}{1.581556in}}%
\pgfpathlineto{\pgfqpoint{1.186810in}{1.586616in}}%
\pgfpathlineto{\pgfqpoint{1.189381in}{1.591619in}}%
\pgfpathlineto{\pgfqpoint{1.191956in}{1.596565in}}%
\pgfpathlineto{\pgfqpoint{1.187832in}{1.594137in}}%
\pgfpathlineto{\pgfqpoint{1.183866in}{1.591647in}}%
\pgfpathlineto{\pgfqpoint{1.180063in}{1.589098in}}%
\pgfpathlineto{\pgfqpoint{1.176424in}{1.586491in}}%
\pgfpathclose%
\pgfusepath{fill}%
\end{pgfscope}%
\begin{pgfscope}%
\pgfpathrectangle{\pgfqpoint{0.329460in}{0.284240in}}{\pgfqpoint{1.989680in}{1.989680in}}%
\pgfusepath{clip}%
\pgfsetbuttcap%
\pgfsetroundjoin%
\definecolor{currentfill}{rgb}{0.344074,0.780029,0.397381}%
\pgfsetfillcolor{currentfill}%
\pgfsetlinewidth{0.000000pt}%
\definecolor{currentstroke}{rgb}{0.000000,0.000000,0.000000}%
\pgfsetstrokecolor{currentstroke}%
\pgfsetdash{}{0pt}%
\pgfpathmoveto{\pgfqpoint{1.479264in}{1.625898in}}%
\pgfpathlineto{\pgfqpoint{1.481484in}{1.621387in}}%
\pgfpathlineto{\pgfqpoint{1.483702in}{1.616811in}}%
\pgfpathlineto{\pgfqpoint{1.485918in}{1.612171in}}%
\pgfpathlineto{\pgfqpoint{1.488130in}{1.607470in}}%
\pgfpathlineto{\pgfqpoint{1.492963in}{1.605376in}}%
\pgfpathlineto{\pgfqpoint{1.497659in}{1.603210in}}%
\pgfpathlineto{\pgfqpoint{1.502215in}{1.600974in}}%
\pgfpathlineto{\pgfqpoint{1.506625in}{1.598669in}}%
\pgfpathlineto{\pgfqpoint{1.504110in}{1.603519in}}%
\pgfpathlineto{\pgfqpoint{1.501592in}{1.608308in}}%
\pgfpathlineto{\pgfqpoint{1.499071in}{1.613033in}}%
\pgfpathlineto{\pgfqpoint{1.496547in}{1.617693in}}%
\pgfpathlineto{\pgfqpoint{1.492426in}{1.619842in}}%
\pgfpathlineto{\pgfqpoint{1.488169in}{1.621927in}}%
\pgfpathlineto{\pgfqpoint{1.483780in}{1.623946in}}%
\pgfpathlineto{\pgfqpoint{1.479264in}{1.625898in}}%
\pgfpathclose%
\pgfusepath{fill}%
\end{pgfscope}%
\begin{pgfscope}%
\pgfpathrectangle{\pgfqpoint{0.329460in}{0.284240in}}{\pgfqpoint{1.989680in}{1.989680in}}%
\pgfusepath{clip}%
\pgfsetbuttcap%
\pgfsetroundjoin%
\definecolor{currentfill}{rgb}{0.179019,0.433756,0.557430}%
\pgfsetfillcolor{currentfill}%
\pgfsetlinewidth{0.000000pt}%
\definecolor{currentstroke}{rgb}{0.000000,0.000000,0.000000}%
\pgfsetstrokecolor{currentstroke}%
\pgfsetdash{}{0pt}%
\pgfpathmoveto{\pgfqpoint{1.680476in}{1.287253in}}%
\pgfpathlineto{\pgfqpoint{1.683789in}{1.280385in}}%
\pgfpathlineto{\pgfqpoint{1.687099in}{1.273541in}}%
\pgfpathlineto{\pgfqpoint{1.690407in}{1.266724in}}%
\pgfpathlineto{\pgfqpoint{1.693714in}{1.259936in}}%
\pgfpathlineto{\pgfqpoint{1.693395in}{1.254488in}}%
\pgfpathlineto{\pgfqpoint{1.692734in}{1.249044in}}%
\pgfpathlineto{\pgfqpoint{1.691729in}{1.243607in}}%
\pgfpathlineto{\pgfqpoint{1.690382in}{1.238185in}}%
\pgfpathlineto{\pgfqpoint{1.687096in}{1.245207in}}%
\pgfpathlineto{\pgfqpoint{1.683809in}{1.252258in}}%
\pgfpathlineto{\pgfqpoint{1.680520in}{1.259335in}}%
\pgfpathlineto{\pgfqpoint{1.677229in}{1.266437in}}%
\pgfpathlineto{\pgfqpoint{1.678535in}{1.271625in}}%
\pgfpathlineto{\pgfqpoint{1.679512in}{1.276828in}}%
\pgfpathlineto{\pgfqpoint{1.680159in}{1.282039in}}%
\pgfpathlineto{\pgfqpoint{1.680476in}{1.287253in}}%
\pgfpathclose%
\pgfusepath{fill}%
\end{pgfscope}%
\begin{pgfscope}%
\pgfpathrectangle{\pgfqpoint{0.329460in}{0.284240in}}{\pgfqpoint{1.989680in}{1.989680in}}%
\pgfusepath{clip}%
\pgfsetbuttcap%
\pgfsetroundjoin%
\definecolor{currentfill}{rgb}{0.487026,0.823929,0.312321}%
\pgfsetfillcolor{currentfill}%
\pgfsetlinewidth{0.000000pt}%
\definecolor{currentstroke}{rgb}{0.000000,0.000000,0.000000}%
\pgfsetstrokecolor{currentstroke}%
\pgfsetdash{}{0pt}%
\pgfpathmoveto{\pgfqpoint{1.310193in}{1.676740in}}%
\pgfpathlineto{\pgfqpoint{1.309361in}{1.673174in}}%
\pgfpathlineto{\pgfqpoint{1.308531in}{1.669529in}}%
\pgfpathlineto{\pgfqpoint{1.307701in}{1.665808in}}%
\pgfpathlineto{\pgfqpoint{1.306873in}{1.662012in}}%
\pgfpathlineto{\pgfqpoint{1.312359in}{1.662631in}}%
\pgfpathlineto{\pgfqpoint{1.317882in}{1.663168in}}%
\pgfpathlineto{\pgfqpoint{1.323437in}{1.663622in}}%
\pgfpathlineto{\pgfqpoint{1.329018in}{1.663993in}}%
\pgfpathlineto{\pgfqpoint{1.329432in}{1.667751in}}%
\pgfpathlineto{\pgfqpoint{1.329847in}{1.671434in}}%
\pgfpathlineto{\pgfqpoint{1.330262in}{1.675040in}}%
\pgfpathlineto{\pgfqpoint{1.330678in}{1.678568in}}%
\pgfpathlineto{\pgfqpoint{1.325515in}{1.678225in}}%
\pgfpathlineto{\pgfqpoint{1.320376in}{1.677806in}}%
\pgfpathlineto{\pgfqpoint{1.315267in}{1.677310in}}%
\pgfpathlineto{\pgfqpoint{1.310193in}{1.676740in}}%
\pgfpathclose%
\pgfusepath{fill}%
\end{pgfscope}%
\begin{pgfscope}%
\pgfpathrectangle{\pgfqpoint{0.329460in}{0.284240in}}{\pgfqpoint{1.989680in}{1.989680in}}%
\pgfusepath{clip}%
\pgfsetbuttcap%
\pgfsetroundjoin%
\definecolor{currentfill}{rgb}{0.487026,0.823929,0.312321}%
\pgfsetfillcolor{currentfill}%
\pgfsetlinewidth{0.000000pt}%
\definecolor{currentstroke}{rgb}{0.000000,0.000000,0.000000}%
\pgfsetstrokecolor{currentstroke}%
\pgfsetdash{}{0pt}%
\pgfpathmoveto{\pgfqpoint{1.372272in}{1.678533in}}%
\pgfpathlineto{\pgfqpoint{1.372699in}{1.675005in}}%
\pgfpathlineto{\pgfqpoint{1.373126in}{1.671398in}}%
\pgfpathlineto{\pgfqpoint{1.373553in}{1.667715in}}%
\pgfpathlineto{\pgfqpoint{1.373979in}{1.663956in}}%
\pgfpathlineto{\pgfqpoint{1.379557in}{1.663575in}}%
\pgfpathlineto{\pgfqpoint{1.385109in}{1.663112in}}%
\pgfpathlineto{\pgfqpoint{1.390628in}{1.662566in}}%
\pgfpathlineto{\pgfqpoint{1.396109in}{1.661938in}}%
\pgfpathlineto{\pgfqpoint{1.395269in}{1.665736in}}%
\pgfpathlineto{\pgfqpoint{1.394429in}{1.669458in}}%
\pgfpathlineto{\pgfqpoint{1.393587in}{1.673104in}}%
\pgfpathlineto{\pgfqpoint{1.392744in}{1.676671in}}%
\pgfpathlineto{\pgfqpoint{1.387673in}{1.677251in}}%
\pgfpathlineto{\pgfqpoint{1.382568in}{1.677754in}}%
\pgfpathlineto{\pgfqpoint{1.377432in}{1.678182in}}%
\pgfpathlineto{\pgfqpoint{1.372272in}{1.678533in}}%
\pgfpathclose%
\pgfusepath{fill}%
\end{pgfscope}%
\begin{pgfscope}%
\pgfpathrectangle{\pgfqpoint{0.329460in}{0.284240in}}{\pgfqpoint{1.989680in}{1.989680in}}%
\pgfusepath{clip}%
\pgfsetbuttcap%
\pgfsetroundjoin%
\definecolor{currentfill}{rgb}{0.282327,0.094955,0.417331}%
\pgfsetfillcolor{currentfill}%
\pgfsetlinewidth{0.000000pt}%
\definecolor{currentstroke}{rgb}{0.000000,0.000000,0.000000}%
\pgfsetstrokecolor{currentstroke}%
\pgfsetdash{}{0pt}%
\pgfpathmoveto{\pgfqpoint{0.941403in}{0.983385in}}%
\pgfpathlineto{\pgfqpoint{0.938342in}{0.978499in}}%
\pgfpathlineto{\pgfqpoint{0.935278in}{0.973749in}}%
\pgfpathlineto{\pgfqpoint{0.932211in}{0.969138in}}%
\pgfpathlineto{\pgfqpoint{0.929142in}{0.964671in}}%
\pgfpathlineto{\pgfqpoint{0.923409in}{0.971745in}}%
\pgfpathlineto{\pgfqpoint{0.918120in}{0.978902in}}%
\pgfpathlineto{\pgfqpoint{0.913281in}{0.986136in}}%
\pgfpathlineto{\pgfqpoint{0.908894in}{0.993438in}}%
\pgfpathlineto{\pgfqpoint{0.912097in}{0.997677in}}%
\pgfpathlineto{\pgfqpoint{0.915297in}{1.002060in}}%
\pgfpathlineto{\pgfqpoint{0.918494in}{1.006582in}}%
\pgfpathlineto{\pgfqpoint{0.921689in}{1.011240in}}%
\pgfpathlineto{\pgfqpoint{0.925963in}{1.004169in}}%
\pgfpathlineto{\pgfqpoint{0.930675in}{0.997164in}}%
\pgfpathlineto{\pgfqpoint{0.935823in}{0.990234in}}%
\pgfpathlineto{\pgfqpoint{0.941403in}{0.983385in}}%
\pgfpathclose%
\pgfusepath{fill}%
\end{pgfscope}%
\begin{pgfscope}%
\pgfpathrectangle{\pgfqpoint{0.329460in}{0.284240in}}{\pgfqpoint{1.989680in}{1.989680in}}%
\pgfusepath{clip}%
\pgfsetbuttcap%
\pgfsetroundjoin%
\definecolor{currentfill}{rgb}{0.267004,0.004874,0.329415}%
\pgfsetfillcolor{currentfill}%
\pgfsetlinewidth{0.000000pt}%
\definecolor{currentstroke}{rgb}{0.000000,0.000000,0.000000}%
\pgfsetstrokecolor{currentstroke}%
\pgfsetdash{}{0pt}%
\pgfpathmoveto{\pgfqpoint{0.866908in}{0.911317in}}%
\pgfpathlineto{\pgfqpoint{0.863739in}{0.910758in}}%
\pgfpathlineto{\pgfqpoint{0.860562in}{0.910431in}}%
\pgfpathlineto{\pgfqpoint{0.857379in}{0.910341in}}%
\pgfpathlineto{\pgfqpoint{0.854188in}{0.910491in}}%
\pgfpathlineto{\pgfqpoint{0.847533in}{0.918888in}}%
\pgfpathlineto{\pgfqpoint{0.841406in}{0.927382in}}%
\pgfpathlineto{\pgfqpoint{0.835810in}{0.935962in}}%
\pgfpathlineto{\pgfqpoint{0.830750in}{0.944621in}}%
\pgfpathlineto{\pgfqpoint{0.834075in}{0.944253in}}%
\pgfpathlineto{\pgfqpoint{0.837391in}{0.944125in}}%
\pgfpathlineto{\pgfqpoint{0.840701in}{0.944233in}}%
\pgfpathlineto{\pgfqpoint{0.844003in}{0.944573in}}%
\pgfpathlineto{\pgfqpoint{0.848951in}{0.936135in}}%
\pgfpathlineto{\pgfqpoint{0.854421in}{0.927774in}}%
\pgfpathlineto{\pgfqpoint{0.860408in}{0.919498in}}%
\pgfpathlineto{\pgfqpoint{0.866908in}{0.911317in}}%
\pgfpathclose%
\pgfusepath{fill}%
\end{pgfscope}%
\begin{pgfscope}%
\pgfpathrectangle{\pgfqpoint{0.329460in}{0.284240in}}{\pgfqpoint{1.989680in}{1.989680in}}%
\pgfusepath{clip}%
\pgfsetbuttcap%
\pgfsetroundjoin%
\definecolor{currentfill}{rgb}{0.280255,0.165693,0.476498}%
\pgfsetfillcolor{currentfill}%
\pgfsetlinewidth{0.000000pt}%
\definecolor{currentstroke}{rgb}{0.000000,0.000000,0.000000}%
\pgfsetstrokecolor{currentstroke}%
\pgfsetdash{}{0pt}%
\pgfpathmoveto{\pgfqpoint{1.758440in}{1.058905in}}%
\pgfpathlineto{\pgfqpoint{1.761644in}{1.053336in}}%
\pgfpathlineto{\pgfqpoint{1.764850in}{1.047875in}}%
\pgfpathlineto{\pgfqpoint{1.768056in}{1.042526in}}%
\pgfpathlineto{\pgfqpoint{1.771265in}{1.037292in}}%
\pgfpathlineto{\pgfqpoint{1.767485in}{1.030400in}}%
\pgfpathlineto{\pgfqpoint{1.763278in}{1.023567in}}%
\pgfpathlineto{\pgfqpoint{1.758645in}{1.016800in}}%
\pgfpathlineto{\pgfqpoint{1.753591in}{1.010106in}}%
\pgfpathlineto{\pgfqpoint{1.750505in}{1.015570in}}%
\pgfpathlineto{\pgfqpoint{1.747420in}{1.021149in}}%
\pgfpathlineto{\pgfqpoint{1.744338in}{1.026840in}}%
\pgfpathlineto{\pgfqpoint{1.741256in}{1.032640in}}%
\pgfpathlineto{\pgfqpoint{1.746168in}{1.039107in}}%
\pgfpathlineto{\pgfqpoint{1.750671in}{1.045645in}}%
\pgfpathlineto{\pgfqpoint{1.754763in}{1.052247in}}%
\pgfpathlineto{\pgfqpoint{1.758440in}{1.058905in}}%
\pgfpathclose%
\pgfusepath{fill}%
\end{pgfscope}%
\begin{pgfscope}%
\pgfpathrectangle{\pgfqpoint{0.329460in}{0.284240in}}{\pgfqpoint{1.989680in}{1.989680in}}%
\pgfusepath{clip}%
\pgfsetbuttcap%
\pgfsetroundjoin%
\definecolor{currentfill}{rgb}{0.147607,0.511733,0.557049}%
\pgfsetfillcolor{currentfill}%
\pgfsetlinewidth{0.000000pt}%
\definecolor{currentstroke}{rgb}{0.000000,0.000000,0.000000}%
\pgfsetstrokecolor{currentstroke}%
\pgfsetdash{}{0pt}%
\pgfpathmoveto{\pgfqpoint{1.048739in}{1.338521in}}%
\pgfpathlineto{\pgfqpoint{1.045409in}{1.331505in}}%
\pgfpathlineto{\pgfqpoint{1.042082in}{1.324492in}}%
\pgfpathlineto{\pgfqpoint{1.038757in}{1.317484in}}%
\pgfpathlineto{\pgfqpoint{1.035434in}{1.310484in}}%
\pgfpathlineto{\pgfqpoint{1.035154in}{1.315466in}}%
\pgfpathlineto{\pgfqpoint{1.035189in}{1.320445in}}%
\pgfpathlineto{\pgfqpoint{1.035539in}{1.325418in}}%
\pgfpathlineto{\pgfqpoint{1.036202in}{1.330379in}}%
\pgfpathlineto{\pgfqpoint{1.039506in}{1.337148in}}%
\pgfpathlineto{\pgfqpoint{1.042813in}{1.343925in}}%
\pgfpathlineto{\pgfqpoint{1.046122in}{1.350708in}}%
\pgfpathlineto{\pgfqpoint{1.049434in}{1.357494in}}%
\pgfpathlineto{\pgfqpoint{1.048808in}{1.352763in}}%
\pgfpathlineto{\pgfqpoint{1.048484in}{1.348021in}}%
\pgfpathlineto{\pgfqpoint{1.048460in}{1.343272in}}%
\pgfpathlineto{\pgfqpoint{1.048739in}{1.338521in}}%
\pgfpathclose%
\pgfusepath{fill}%
\end{pgfscope}%
\begin{pgfscope}%
\pgfpathrectangle{\pgfqpoint{0.329460in}{0.284240in}}{\pgfqpoint{1.989680in}{1.989680in}}%
\pgfusepath{clip}%
\pgfsetbuttcap%
\pgfsetroundjoin%
\definecolor{currentfill}{rgb}{0.166383,0.690856,0.496502}%
\pgfsetfillcolor{currentfill}%
\pgfsetlinewidth{0.000000pt}%
\definecolor{currentstroke}{rgb}{0.000000,0.000000,0.000000}%
\pgfsetstrokecolor{currentstroke}%
\pgfsetdash{}{0pt}%
\pgfpathmoveto{\pgfqpoint{1.128112in}{1.518320in}}%
\pgfpathlineto{\pgfqpoint{1.124928in}{1.512402in}}%
\pgfpathlineto{\pgfqpoint{1.121747in}{1.506442in}}%
\pgfpathlineto{\pgfqpoint{1.118569in}{1.500441in}}%
\pgfpathlineto{\pgfqpoint{1.115395in}{1.494402in}}%
\pgfpathlineto{\pgfqpoint{1.118032in}{1.497991in}}%
\pgfpathlineto{\pgfqpoint{1.120898in}{1.501535in}}%
\pgfpathlineto{\pgfqpoint{1.123987in}{1.505032in}}%
\pgfpathlineto{\pgfqpoint{1.127298in}{1.508477in}}%
\pgfpathlineto{\pgfqpoint{1.130305in}{1.514313in}}%
\pgfpathlineto{\pgfqpoint{1.133316in}{1.520111in}}%
\pgfpathlineto{\pgfqpoint{1.136330in}{1.525869in}}%
\pgfpathlineto{\pgfqpoint{1.139347in}{1.531585in}}%
\pgfpathlineto{\pgfqpoint{1.136221in}{1.528337in}}%
\pgfpathlineto{\pgfqpoint{1.133304in}{1.525042in}}%
\pgfpathlineto{\pgfqpoint{1.130600in}{1.521701in}}%
\pgfpathlineto{\pgfqpoint{1.128112in}{1.518320in}}%
\pgfpathclose%
\pgfusepath{fill}%
\end{pgfscope}%
\begin{pgfscope}%
\pgfpathrectangle{\pgfqpoint{0.329460in}{0.284240in}}{\pgfqpoint{1.989680in}{1.989680in}}%
\pgfusepath{clip}%
\pgfsetbuttcap%
\pgfsetroundjoin%
\definecolor{currentfill}{rgb}{0.276194,0.190074,0.493001}%
\pgfsetfillcolor{currentfill}%
\pgfsetlinewidth{0.000000pt}%
\definecolor{currentstroke}{rgb}{0.000000,0.000000,0.000000}%
\pgfsetstrokecolor{currentstroke}%
\pgfsetdash{}{0pt}%
\pgfpathmoveto{\pgfqpoint{1.958820in}{1.058171in}}%
\pgfpathlineto{\pgfqpoint{1.962460in}{1.064968in}}%
\pgfpathlineto{\pgfqpoint{1.966114in}{1.072118in}}%
\pgfpathlineto{\pgfqpoint{1.969784in}{1.079626in}}%
\pgfpathlineto{\pgfqpoint{1.973469in}{1.087499in}}%
\pgfpathlineto{\pgfqpoint{1.970814in}{1.077342in}}%
\pgfpathlineto{\pgfqpoint{1.967526in}{1.067218in}}%
\pgfpathlineto{\pgfqpoint{1.963606in}{1.057137in}}%
\pgfpathlineto{\pgfqpoint{1.959053in}{1.047111in}}%
\pgfpathlineto{\pgfqpoint{1.955431in}{1.039429in}}%
\pgfpathlineto{\pgfqpoint{1.951825in}{1.032114in}}%
\pgfpathlineto{\pgfqpoint{1.948233in}{1.025160in}}%
\pgfpathlineto{\pgfqpoint{1.944656in}{1.018559in}}%
\pgfpathlineto{\pgfqpoint{1.949123in}{1.028392in}}%
\pgfpathlineto{\pgfqpoint{1.952973in}{1.038279in}}%
\pgfpathlineto{\pgfqpoint{1.956205in}{1.048209in}}%
\pgfpathlineto{\pgfqpoint{1.958820in}{1.058171in}}%
\pgfpathclose%
\pgfusepath{fill}%
\end{pgfscope}%
\begin{pgfscope}%
\pgfpathrectangle{\pgfqpoint{0.329460in}{0.284240in}}{\pgfqpoint{1.989680in}{1.989680in}}%
\pgfusepath{clip}%
\pgfsetbuttcap%
\pgfsetroundjoin%
\definecolor{currentfill}{rgb}{0.344074,0.780029,0.397381}%
\pgfsetfillcolor{currentfill}%
\pgfsetlinewidth{0.000000pt}%
\definecolor{currentstroke}{rgb}{0.000000,0.000000,0.000000}%
\pgfsetstrokecolor{currentstroke}%
\pgfsetdash{}{0pt}%
\pgfpathmoveto{\pgfqpoint{1.202283in}{1.615732in}}%
\pgfpathlineto{\pgfqpoint{1.199697in}{1.611036in}}%
\pgfpathlineto{\pgfqpoint{1.197113in}{1.606275in}}%
\pgfpathlineto{\pgfqpoint{1.194533in}{1.601451in}}%
\pgfpathlineto{\pgfqpoint{1.191956in}{1.596565in}}%
\pgfpathlineto{\pgfqpoint{1.196233in}{1.598928in}}%
\pgfpathlineto{\pgfqpoint{1.200659in}{1.601226in}}%
\pgfpathlineto{\pgfqpoint{1.205231in}{1.603454in}}%
\pgfpathlineto{\pgfqpoint{1.209943in}{1.605613in}}%
\pgfpathlineto{\pgfqpoint{1.212225in}{1.610345in}}%
\pgfpathlineto{\pgfqpoint{1.214511in}{1.615016in}}%
\pgfpathlineto{\pgfqpoint{1.216799in}{1.619624in}}%
\pgfpathlineto{\pgfqpoint{1.219090in}{1.624166in}}%
\pgfpathlineto{\pgfqpoint{1.214687in}{1.622154in}}%
\pgfpathlineto{\pgfqpoint{1.210415in}{1.620077in}}%
\pgfpathlineto{\pgfqpoint{1.206279in}{1.617935in}}%
\pgfpathlineto{\pgfqpoint{1.202283in}{1.615732in}}%
\pgfpathclose%
\pgfusepath{fill}%
\end{pgfscope}%
\begin{pgfscope}%
\pgfpathrectangle{\pgfqpoint{0.329460in}{0.284240in}}{\pgfqpoint{1.989680in}{1.989680in}}%
\pgfusepath{clip}%
\pgfsetbuttcap%
\pgfsetroundjoin%
\definecolor{currentfill}{rgb}{0.122606,0.585371,0.546557}%
\pgfsetfillcolor{currentfill}%
\pgfsetlinewidth{0.000000pt}%
\definecolor{currentstroke}{rgb}{0.000000,0.000000,0.000000}%
\pgfsetstrokecolor{currentstroke}%
\pgfsetdash{}{0pt}%
\pgfpathmoveto{\pgfqpoint{1.076025in}{1.411605in}}%
\pgfpathlineto{\pgfqpoint{1.072692in}{1.404881in}}%
\pgfpathlineto{\pgfqpoint{1.069361in}{1.398142in}}%
\pgfpathlineto{\pgfqpoint{1.066033in}{1.391388in}}%
\pgfpathlineto{\pgfqpoint{1.062707in}{1.384623in}}%
\pgfpathlineto{\pgfqpoint{1.063582in}{1.389111in}}%
\pgfpathlineto{\pgfqpoint{1.064741in}{1.393580in}}%
\pgfpathlineto{\pgfqpoint{1.066182in}{1.398026in}}%
\pgfpathlineto{\pgfqpoint{1.067905in}{1.402444in}}%
\pgfpathlineto{\pgfqpoint{1.071161in}{1.408985in}}%
\pgfpathlineto{\pgfqpoint{1.074420in}{1.415515in}}%
\pgfpathlineto{\pgfqpoint{1.077681in}{1.422031in}}%
\pgfpathlineto{\pgfqpoint{1.080946in}{1.428531in}}%
\pgfpathlineto{\pgfqpoint{1.079312in}{1.424335in}}%
\pgfpathlineto{\pgfqpoint{1.077946in}{1.420112in}}%
\pgfpathlineto{\pgfqpoint{1.076850in}{1.415867in}}%
\pgfpathlineto{\pgfqpoint{1.076025in}{1.411605in}}%
\pgfpathclose%
\pgfusepath{fill}%
\end{pgfscope}%
\begin{pgfscope}%
\pgfpathrectangle{\pgfqpoint{0.329460in}{0.284240in}}{\pgfqpoint{1.989680in}{1.989680in}}%
\pgfusepath{clip}%
\pgfsetbuttcap%
\pgfsetroundjoin%
\definecolor{currentfill}{rgb}{0.134692,0.658636,0.517649}%
\pgfsetfillcolor{currentfill}%
\pgfsetlinewidth{0.000000pt}%
\definecolor{currentstroke}{rgb}{0.000000,0.000000,0.000000}%
\pgfsetstrokecolor{currentstroke}%
\pgfsetdash{}{0pt}%
\pgfpathmoveto{\pgfqpoint{1.584647in}{1.497594in}}%
\pgfpathlineto{\pgfqpoint{1.587785in}{1.491565in}}%
\pgfpathlineto{\pgfqpoint{1.590920in}{1.485502in}}%
\pgfpathlineto{\pgfqpoint{1.594051in}{1.479408in}}%
\pgfpathlineto{\pgfqpoint{1.597180in}{1.473284in}}%
\pgfpathlineto{\pgfqpoint{1.599940in}{1.469482in}}%
\pgfpathlineto{\pgfqpoint{1.602457in}{1.465638in}}%
\pgfpathlineto{\pgfqpoint{1.604729in}{1.461754in}}%
\pgfpathlineto{\pgfqpoint{1.606751in}{1.457834in}}%
\pgfpathlineto{\pgfqpoint{1.603493in}{1.464171in}}%
\pgfpathlineto{\pgfqpoint{1.600232in}{1.470478in}}%
\pgfpathlineto{\pgfqpoint{1.596967in}{1.476752in}}%
\pgfpathlineto{\pgfqpoint{1.593700in}{1.482993in}}%
\pgfpathlineto{\pgfqpoint{1.591788in}{1.486697in}}%
\pgfpathlineto{\pgfqpoint{1.589641in}{1.490367in}}%
\pgfpathlineto{\pgfqpoint{1.587259in}{1.494001in}}%
\pgfpathlineto{\pgfqpoint{1.584647in}{1.497594in}}%
\pgfpathclose%
\pgfusepath{fill}%
\end{pgfscope}%
\begin{pgfscope}%
\pgfpathrectangle{\pgfqpoint{0.329460in}{0.284240in}}{\pgfqpoint{1.989680in}{1.989680in}}%
\pgfusepath{clip}%
\pgfsetbuttcap%
\pgfsetroundjoin%
\definecolor{currentfill}{rgb}{0.487026,0.823929,0.312321}%
\pgfsetfillcolor{currentfill}%
\pgfsetlinewidth{0.000000pt}%
\definecolor{currentstroke}{rgb}{0.000000,0.000000,0.000000}%
\pgfsetstrokecolor{currentstroke}%
\pgfsetdash{}{0pt}%
\pgfpathmoveto{\pgfqpoint{1.392744in}{1.676671in}}%
\pgfpathlineto{\pgfqpoint{1.393587in}{1.673104in}}%
\pgfpathlineto{\pgfqpoint{1.394429in}{1.669458in}}%
\pgfpathlineto{\pgfqpoint{1.395269in}{1.665736in}}%
\pgfpathlineto{\pgfqpoint{1.396109in}{1.661938in}}%
\pgfpathlineto{\pgfqpoint{1.401547in}{1.661229in}}%
\pgfpathlineto{\pgfqpoint{1.406938in}{1.660440in}}%
\pgfpathlineto{\pgfqpoint{1.412275in}{1.659570in}}%
\pgfpathlineto{\pgfqpoint{1.411133in}{1.663413in}}%
\pgfpathlineto{\pgfqpoint{1.409989in}{1.667181in}}%
\pgfpathlineto{\pgfqpoint{1.408844in}{1.670873in}}%
\pgfpathlineto{\pgfqpoint{1.407697in}{1.674486in}}%
\pgfpathlineto{\pgfqpoint{1.402760in}{1.675288in}}%
\pgfpathlineto{\pgfqpoint{1.397774in}{1.676017in}}%
\pgfpathlineto{\pgfqpoint{1.392744in}{1.676671in}}%
\pgfpathclose%
\pgfusepath{fill}%
\end{pgfscope}%
\begin{pgfscope}%
\pgfpathrectangle{\pgfqpoint{0.329460in}{0.284240in}}{\pgfqpoint{1.989680in}{1.989680in}}%
\pgfusepath{clip}%
\pgfsetbuttcap%
\pgfsetroundjoin%
\definecolor{currentfill}{rgb}{0.272594,0.025563,0.353093}%
\pgfsetfillcolor{currentfill}%
\pgfsetlinewidth{0.000000pt}%
\definecolor{currentstroke}{rgb}{0.000000,0.000000,0.000000}%
\pgfsetstrokecolor{currentstroke}%
\pgfsetdash{}{0pt}%
\pgfpathmoveto{\pgfqpoint{1.889139in}{0.956535in}}%
\pgfpathlineto{\pgfqpoint{1.892528in}{0.958222in}}%
\pgfpathlineto{\pgfqpoint{1.895926in}{0.960177in}}%
\pgfpathlineto{\pgfqpoint{1.899334in}{0.962405in}}%
\pgfpathlineto{\pgfqpoint{1.902752in}{0.964911in}}%
\pgfpathlineto{\pgfqpoint{1.897971in}{0.955758in}}%
\pgfpathlineto{\pgfqpoint{1.892624in}{0.946677in}}%
\pgfpathlineto{\pgfqpoint{1.886712in}{0.937677in}}%
\pgfpathlineto{\pgfqpoint{1.880241in}{0.928769in}}%
\pgfpathlineto{\pgfqpoint{1.876944in}{0.926474in}}%
\pgfpathlineto{\pgfqpoint{1.873658in}{0.924458in}}%
\pgfpathlineto{\pgfqpoint{1.870381in}{0.922717in}}%
\pgfpathlineto{\pgfqpoint{1.867113in}{0.921244in}}%
\pgfpathlineto{\pgfqpoint{1.873442in}{0.929942in}}%
\pgfpathlineto{\pgfqpoint{1.879225in}{0.938729in}}%
\pgfpathlineto{\pgfqpoint{1.884458in}{0.947597in}}%
\pgfpathlineto{\pgfqpoint{1.889139in}{0.956535in}}%
\pgfpathclose%
\pgfusepath{fill}%
\end{pgfscope}%
\begin{pgfscope}%
\pgfpathrectangle{\pgfqpoint{0.329460in}{0.284240in}}{\pgfqpoint{1.989680in}{1.989680in}}%
\pgfusepath{clip}%
\pgfsetbuttcap%
\pgfsetroundjoin%
\definecolor{currentfill}{rgb}{0.487026,0.823929,0.312321}%
\pgfsetfillcolor{currentfill}%
\pgfsetlinewidth{0.000000pt}%
\definecolor{currentstroke}{rgb}{0.000000,0.000000,0.000000}%
\pgfsetstrokecolor{currentstroke}%
\pgfsetdash{}{0pt}%
\pgfpathmoveto{\pgfqpoint{1.290335in}{1.673711in}}%
\pgfpathlineto{\pgfqpoint{1.289101in}{1.670082in}}%
\pgfpathlineto{\pgfqpoint{1.287867in}{1.666374in}}%
\pgfpathlineto{\pgfqpoint{1.286635in}{1.662590in}}%
\pgfpathlineto{\pgfqpoint{1.285405in}{1.658730in}}%
\pgfpathlineto{\pgfqpoint{1.290691in}{1.659670in}}%
\pgfpathlineto{\pgfqpoint{1.296034in}{1.660531in}}%
\pgfpathlineto{\pgfqpoint{1.301430in}{1.661312in}}%
\pgfpathlineto{\pgfqpoint{1.306873in}{1.662012in}}%
\pgfpathlineto{\pgfqpoint{1.307701in}{1.665808in}}%
\pgfpathlineto{\pgfqpoint{1.308531in}{1.669529in}}%
\pgfpathlineto{\pgfqpoint{1.309361in}{1.673174in}}%
\pgfpathlineto{\pgfqpoint{1.310193in}{1.676740in}}%
\pgfpathlineto{\pgfqpoint{1.305158in}{1.676094in}}%
\pgfpathlineto{\pgfqpoint{1.300166in}{1.675373in}}%
\pgfpathlineto{\pgfqpoint{1.295224in}{1.674579in}}%
\pgfpathlineto{\pgfqpoint{1.290335in}{1.673711in}}%
\pgfpathclose%
\pgfusepath{fill}%
\end{pgfscope}%
\begin{pgfscope}%
\pgfpathrectangle{\pgfqpoint{0.329460in}{0.284240in}}{\pgfqpoint{1.989680in}{1.989680in}}%
\pgfusepath{clip}%
\pgfsetbuttcap%
\pgfsetroundjoin%
\definecolor{currentfill}{rgb}{0.283072,0.130895,0.449241}%
\pgfsetfillcolor{currentfill}%
\pgfsetlinewidth{0.000000pt}%
\definecolor{currentstroke}{rgb}{0.000000,0.000000,0.000000}%
\pgfsetstrokecolor{currentstroke}%
\pgfsetdash{}{0pt}%
\pgfpathmoveto{\pgfqpoint{0.953628in}{1.004222in}}%
\pgfpathlineto{\pgfqpoint{0.950575in}{0.998826in}}%
\pgfpathlineto{\pgfqpoint{0.947519in}{0.993553in}}%
\pgfpathlineto{\pgfqpoint{0.944462in}{0.988405in}}%
\pgfpathlineto{\pgfqpoint{0.941403in}{0.983385in}}%
\pgfpathlineto{\pgfqpoint{0.935823in}{0.990234in}}%
\pgfpathlineto{\pgfqpoint{0.930675in}{0.997164in}}%
\pgfpathlineto{\pgfqpoint{0.925963in}{1.004169in}}%
\pgfpathlineto{\pgfqpoint{0.921689in}{1.011240in}}%
\pgfpathlineto{\pgfqpoint{0.924882in}{1.016031in}}%
\pgfpathlineto{\pgfqpoint{0.928073in}{1.020951in}}%
\pgfpathlineto{\pgfqpoint{0.931262in}{1.025996in}}%
\pgfpathlineto{\pgfqpoint{0.934449in}{1.031163in}}%
\pgfpathlineto{\pgfqpoint{0.938609in}{1.024323in}}%
\pgfpathlineto{\pgfqpoint{0.943194in}{1.017548in}}%
\pgfpathlineto{\pgfqpoint{0.948202in}{1.010846in}}%
\pgfpathlineto{\pgfqpoint{0.953628in}{1.004222in}}%
\pgfpathclose%
\pgfusepath{fill}%
\end{pgfscope}%
\begin{pgfscope}%
\pgfpathrectangle{\pgfqpoint{0.329460in}{0.284240in}}{\pgfqpoint{1.989680in}{1.989680in}}%
\pgfusepath{clip}%
\pgfsetbuttcap%
\pgfsetroundjoin%
\definecolor{currentfill}{rgb}{0.412913,0.803041,0.357269}%
\pgfsetfillcolor{currentfill}%
\pgfsetlinewidth{0.000000pt}%
\definecolor{currentstroke}{rgb}{0.000000,0.000000,0.000000}%
\pgfsetstrokecolor{currentstroke}%
\pgfsetdash{}{0pt}%
\pgfpathmoveto{\pgfqpoint{1.452449in}{1.649844in}}%
\pgfpathlineto{\pgfqpoint{1.454343in}{1.645738in}}%
\pgfpathlineto{\pgfqpoint{1.456235in}{1.641561in}}%
\pgfpathlineto{\pgfqpoint{1.458124in}{1.637314in}}%
\pgfpathlineto{\pgfqpoint{1.460011in}{1.632997in}}%
\pgfpathlineto{\pgfqpoint{1.464993in}{1.631332in}}%
\pgfpathlineto{\pgfqpoint{1.469865in}{1.629593in}}%
\pgfpathlineto{\pgfqpoint{1.474624in}{1.627781in}}%
\pgfpathlineto{\pgfqpoint{1.479264in}{1.625898in}}%
\pgfpathlineto{\pgfqpoint{1.477040in}{1.630343in}}%
\pgfpathlineto{\pgfqpoint{1.474814in}{1.634718in}}%
\pgfpathlineto{\pgfqpoint{1.472585in}{1.639024in}}%
\pgfpathlineto{\pgfqpoint{1.470353in}{1.643257in}}%
\pgfpathlineto{\pgfqpoint{1.466039in}{1.645004in}}%
\pgfpathlineto{\pgfqpoint{1.461614in}{1.646685in}}%
\pgfpathlineto{\pgfqpoint{1.457082in}{1.648299in}}%
\pgfpathlineto{\pgfqpoint{1.452449in}{1.649844in}}%
\pgfpathclose%
\pgfusepath{fill}%
\end{pgfscope}%
\begin{pgfscope}%
\pgfpathrectangle{\pgfqpoint{0.329460in}{0.284240in}}{\pgfqpoint{1.989680in}{1.989680in}}%
\pgfusepath{clip}%
\pgfsetbuttcap%
\pgfsetroundjoin%
\definecolor{currentfill}{rgb}{0.231674,0.318106,0.544834}%
\pgfsetfillcolor{currentfill}%
\pgfsetlinewidth{0.000000pt}%
\definecolor{currentstroke}{rgb}{0.000000,0.000000,0.000000}%
\pgfsetstrokecolor{currentstroke}%
\pgfsetdash{}{0pt}%
\pgfpathmoveto{\pgfqpoint{0.998027in}{1.155004in}}%
\pgfpathlineto{\pgfqpoint{0.994849in}{1.148087in}}%
\pgfpathlineto{\pgfqpoint{0.991671in}{1.141229in}}%
\pgfpathlineto{\pgfqpoint{0.988493in}{1.134430in}}%
\pgfpathlineto{\pgfqpoint{0.985316in}{1.127696in}}%
\pgfpathlineto{\pgfqpoint{0.981992in}{1.133663in}}%
\pgfpathlineto{\pgfqpoint{0.979045in}{1.139674in}}%
\pgfpathlineto{\pgfqpoint{0.976477in}{1.145725in}}%
\pgfpathlineto{\pgfqpoint{0.974290in}{1.151809in}}%
\pgfpathlineto{\pgfqpoint{0.977550in}{1.158309in}}%
\pgfpathlineto{\pgfqpoint{0.980812in}{1.164874in}}%
\pgfpathlineto{\pgfqpoint{0.984074in}{1.171499in}}%
\pgfpathlineto{\pgfqpoint{0.987337in}{1.178182in}}%
\pgfpathlineto{\pgfqpoint{0.989460in}{1.172334in}}%
\pgfpathlineto{\pgfqpoint{0.991951in}{1.166517in}}%
\pgfpathlineto{\pgfqpoint{0.994807in}{1.160739in}}%
\pgfpathlineto{\pgfqpoint{0.998027in}{1.155004in}}%
\pgfpathclose%
\pgfusepath{fill}%
\end{pgfscope}%
\begin{pgfscope}%
\pgfpathrectangle{\pgfqpoint{0.329460in}{0.284240in}}{\pgfqpoint{1.989680in}{1.989680in}}%
\pgfusepath{clip}%
\pgfsetbuttcap%
\pgfsetroundjoin%
\definecolor{currentfill}{rgb}{0.179019,0.433756,0.557430}%
\pgfsetfillcolor{currentfill}%
\pgfsetlinewidth{0.000000pt}%
\definecolor{currentstroke}{rgb}{0.000000,0.000000,0.000000}%
\pgfsetstrokecolor{currentstroke}%
\pgfsetdash{}{0pt}%
\pgfpathmoveto{\pgfqpoint{1.026583in}{1.261841in}}%
\pgfpathlineto{\pgfqpoint{1.023304in}{1.254687in}}%
\pgfpathlineto{\pgfqpoint{1.020027in}{1.247558in}}%
\pgfpathlineto{\pgfqpoint{1.016752in}{1.240455in}}%
\pgfpathlineto{\pgfqpoint{1.013478in}{1.233382in}}%
\pgfpathlineto{\pgfqpoint{1.011827in}{1.238787in}}%
\pgfpathlineto{\pgfqpoint{1.010518in}{1.244211in}}%
\pgfpathlineto{\pgfqpoint{1.009551in}{1.249648in}}%
\pgfpathlineto{\pgfqpoint{1.008928in}{1.255094in}}%
\pgfpathlineto{\pgfqpoint{1.012234in}{1.261933in}}%
\pgfpathlineto{\pgfqpoint{1.015543in}{1.268802in}}%
\pgfpathlineto{\pgfqpoint{1.018853in}{1.275698in}}%
\pgfpathlineto{\pgfqpoint{1.022165in}{1.282618in}}%
\pgfpathlineto{\pgfqpoint{1.022775in}{1.277406in}}%
\pgfpathlineto{\pgfqpoint{1.023715in}{1.272203in}}%
\pgfpathlineto{\pgfqpoint{1.024985in}{1.267012in}}%
\pgfpathlineto{\pgfqpoint{1.026583in}{1.261841in}}%
\pgfpathclose%
\pgfusepath{fill}%
\end{pgfscope}%
\begin{pgfscope}%
\pgfpathrectangle{\pgfqpoint{0.329460in}{0.284240in}}{\pgfqpoint{1.989680in}{1.989680in}}%
\pgfusepath{clip}%
\pgfsetbuttcap%
\pgfsetroundjoin%
\definecolor{currentfill}{rgb}{0.268510,0.009605,0.335427}%
\pgfsetfillcolor{currentfill}%
\pgfsetlinewidth{0.000000pt}%
\definecolor{currentstroke}{rgb}{0.000000,0.000000,0.000000}%
\pgfsetstrokecolor{currentstroke}%
\pgfsetdash{}{0pt}%
\pgfpathmoveto{\pgfqpoint{0.854188in}{0.910491in}}%
\pgfpathlineto{\pgfqpoint{0.850989in}{0.910887in}}%
\pgfpathlineto{\pgfqpoint{0.847781in}{0.911534in}}%
\pgfpathlineto{\pgfqpoint{0.844566in}{0.912435in}}%
\pgfpathlineto{\pgfqpoint{0.841342in}{0.913596in}}%
\pgfpathlineto{\pgfqpoint{0.834532in}{0.922206in}}%
\pgfpathlineto{\pgfqpoint{0.828264in}{0.930914in}}%
\pgfpathlineto{\pgfqpoint{0.822542in}{0.939711in}}%
\pgfpathlineto{\pgfqpoint{0.817370in}{0.948587in}}%
\pgfpathlineto{\pgfqpoint{0.820728in}{0.947212in}}%
\pgfpathlineto{\pgfqpoint{0.824077in}{0.946095in}}%
\pgfpathlineto{\pgfqpoint{0.827418in}{0.945233in}}%
\pgfpathlineto{\pgfqpoint{0.830750in}{0.944621in}}%
\pgfpathlineto{\pgfqpoint{0.835810in}{0.935962in}}%
\pgfpathlineto{\pgfqpoint{0.841406in}{0.927382in}}%
\pgfpathlineto{\pgfqpoint{0.847533in}{0.918888in}}%
\pgfpathlineto{\pgfqpoint{0.854188in}{0.910491in}}%
\pgfpathclose%
\pgfusepath{fill}%
\end{pgfscope}%
\begin{pgfscope}%
\pgfpathrectangle{\pgfqpoint{0.329460in}{0.284240in}}{\pgfqpoint{1.989680in}{1.989680in}}%
\pgfusepath{clip}%
\pgfsetbuttcap%
\pgfsetroundjoin%
\definecolor{currentfill}{rgb}{0.201239,0.383670,0.554294}%
\pgfsetfillcolor{currentfill}%
\pgfsetlinewidth{0.000000pt}%
\definecolor{currentstroke}{rgb}{0.000000,0.000000,0.000000}%
\pgfsetstrokecolor{currentstroke}%
\pgfsetdash{}{0pt}%
\pgfpathmoveto{\pgfqpoint{2.007868in}{1.206561in}}%
\pgfpathlineto{\pgfqpoint{2.011712in}{1.218176in}}%
\pgfpathlineto{\pgfqpoint{2.015576in}{1.230219in}}%
\pgfpathlineto{\pgfqpoint{2.019460in}{1.242697in}}%
\pgfpathlineto{\pgfqpoint{2.023365in}{1.255618in}}%
\pgfpathlineto{\pgfqpoint{2.023302in}{1.244911in}}%
\pgfpathlineto{\pgfqpoint{2.022566in}{1.234192in}}%
\pgfpathlineto{\pgfqpoint{2.021152in}{1.223472in}}%
\pgfpathlineto{\pgfqpoint{2.019060in}{1.212761in}}%
\pgfpathlineto{\pgfqpoint{2.015154in}{1.200003in}}%
\pgfpathlineto{\pgfqpoint{2.011269in}{1.187690in}}%
\pgfpathlineto{\pgfqpoint{2.007405in}{1.175815in}}%
\pgfpathlineto{\pgfqpoint{2.003560in}{1.164371in}}%
\pgfpathlineto{\pgfqpoint{2.005629in}{1.174913in}}%
\pgfpathlineto{\pgfqpoint{2.007036in}{1.185467in}}%
\pgfpathlineto{\pgfqpoint{2.007781in}{1.196019in}}%
\pgfpathlineto{\pgfqpoint{2.007868in}{1.206561in}}%
\pgfpathclose%
\pgfusepath{fill}%
\end{pgfscope}%
\begin{pgfscope}%
\pgfpathrectangle{\pgfqpoint{0.329460in}{0.284240in}}{\pgfqpoint{1.989680in}{1.989680in}}%
\pgfusepath{clip}%
\pgfsetbuttcap%
\pgfsetroundjoin%
\definecolor{currentfill}{rgb}{0.487026,0.823929,0.312321}%
\pgfsetfillcolor{currentfill}%
\pgfsetlinewidth{0.000000pt}%
\definecolor{currentstroke}{rgb}{0.000000,0.000000,0.000000}%
\pgfsetstrokecolor{currentstroke}%
\pgfsetdash{}{0pt}%
\pgfpathmoveto{\pgfqpoint{1.407697in}{1.674486in}}%
\pgfpathlineto{\pgfqpoint{1.408844in}{1.670873in}}%
\pgfpathlineto{\pgfqpoint{1.409989in}{1.667181in}}%
\pgfpathlineto{\pgfqpoint{1.411133in}{1.663413in}}%
\pgfpathlineto{\pgfqpoint{1.412275in}{1.659570in}}%
\pgfpathlineto{\pgfqpoint{1.417553in}{1.658621in}}%
\pgfpathlineto{\pgfqpoint{1.422769in}{1.657594in}}%
\pgfpathlineto{\pgfqpoint{1.427915in}{1.656489in}}%
\pgfpathlineto{\pgfqpoint{1.432988in}{1.655308in}}%
\pgfpathlineto{\pgfqpoint{1.431458in}{1.659234in}}%
\pgfpathlineto{\pgfqpoint{1.429925in}{1.663085in}}%
\pgfpathlineto{\pgfqpoint{1.428391in}{1.666859in}}%
\pgfpathlineto{\pgfqpoint{1.426854in}{1.670554in}}%
\pgfpathlineto{\pgfqpoint{1.422163in}{1.671643in}}%
\pgfpathlineto{\pgfqpoint{1.417403in}{1.672663in}}%
\pgfpathlineto{\pgfqpoint{1.412579in}{1.673610in}}%
\pgfpathlineto{\pgfqpoint{1.407697in}{1.674486in}}%
\pgfpathclose%
\pgfusepath{fill}%
\end{pgfscope}%
\begin{pgfscope}%
\pgfpathrectangle{\pgfqpoint{0.329460in}{0.284240in}}{\pgfqpoint{1.989680in}{1.989680in}}%
\pgfusepath{clip}%
\pgfsetbuttcap%
\pgfsetroundjoin%
\definecolor{currentfill}{rgb}{0.274128,0.199721,0.498911}%
\pgfsetfillcolor{currentfill}%
\pgfsetlinewidth{0.000000pt}%
\definecolor{currentstroke}{rgb}{0.000000,0.000000,0.000000}%
\pgfsetstrokecolor{currentstroke}%
\pgfsetdash{}{0pt}%
\pgfpathmoveto{\pgfqpoint{1.745632in}{1.082199in}}%
\pgfpathlineto{\pgfqpoint{1.748833in}{1.076230in}}%
\pgfpathlineto{\pgfqpoint{1.752034in}{1.070356in}}%
\pgfpathlineto{\pgfqpoint{1.755237in}{1.064580in}}%
\pgfpathlineto{\pgfqpoint{1.758440in}{1.058905in}}%
\pgfpathlineto{\pgfqpoint{1.754763in}{1.052247in}}%
\pgfpathlineto{\pgfqpoint{1.750671in}{1.045645in}}%
\pgfpathlineto{\pgfqpoint{1.746168in}{1.039107in}}%
\pgfpathlineto{\pgfqpoint{1.741256in}{1.032640in}}%
\pgfpathlineto{\pgfqpoint{1.738175in}{1.038545in}}%
\pgfpathlineto{\pgfqpoint{1.735096in}{1.044551in}}%
\pgfpathlineto{\pgfqpoint{1.732017in}{1.050655in}}%
\pgfpathlineto{\pgfqpoint{1.728940in}{1.056855in}}%
\pgfpathlineto{\pgfqpoint{1.733708in}{1.063095in}}%
\pgfpathlineto{\pgfqpoint{1.738082in}{1.069403in}}%
\pgfpathlineto{\pgfqpoint{1.742058in}{1.075774in}}%
\pgfpathlineto{\pgfqpoint{1.745632in}{1.082199in}}%
\pgfpathclose%
\pgfusepath{fill}%
\end{pgfscope}%
\begin{pgfscope}%
\pgfpathrectangle{\pgfqpoint{0.329460in}{0.284240in}}{\pgfqpoint{1.989680in}{1.989680in}}%
\pgfusepath{clip}%
\pgfsetbuttcap%
\pgfsetroundjoin%
\definecolor{currentfill}{rgb}{0.134692,0.658636,0.517649}%
\pgfsetfillcolor{currentfill}%
\pgfsetlinewidth{0.000000pt}%
\definecolor{currentstroke}{rgb}{0.000000,0.000000,0.000000}%
\pgfsetstrokecolor{currentstroke}%
\pgfsetdash{}{0pt}%
\pgfpathmoveto{\pgfqpoint{1.107177in}{1.479675in}}%
\pgfpathlineto{\pgfqpoint{1.103887in}{1.473387in}}%
\pgfpathlineto{\pgfqpoint{1.100600in}{1.467064in}}%
\pgfpathlineto{\pgfqpoint{1.097317in}{1.460709in}}%
\pgfpathlineto{\pgfqpoint{1.094037in}{1.454324in}}%
\pgfpathlineto{\pgfqpoint{1.095836in}{1.458272in}}%
\pgfpathlineto{\pgfqpoint{1.097887in}{1.462187in}}%
\pgfpathlineto{\pgfqpoint{1.100185in}{1.466067in}}%
\pgfpathlineto{\pgfqpoint{1.102730in}{1.469907in}}%
\pgfpathlineto{\pgfqpoint{1.105891in}{1.476077in}}%
\pgfpathlineto{\pgfqpoint{1.109056in}{1.482218in}}%
\pgfpathlineto{\pgfqpoint{1.112224in}{1.488327in}}%
\pgfpathlineto{\pgfqpoint{1.115395in}{1.494402in}}%
\pgfpathlineto{\pgfqpoint{1.112988in}{1.490773in}}%
\pgfpathlineto{\pgfqpoint{1.110814in}{1.487106in}}%
\pgfpathlineto{\pgfqpoint{1.108876in}{1.483406in}}%
\pgfpathlineto{\pgfqpoint{1.107177in}{1.479675in}}%
\pgfpathclose%
\pgfusepath{fill}%
\end{pgfscope}%
\begin{pgfscope}%
\pgfpathrectangle{\pgfqpoint{0.329460in}{0.284240in}}{\pgfqpoint{1.989680in}{1.989680in}}%
\pgfusepath{clip}%
\pgfsetbuttcap%
\pgfsetroundjoin%
\definecolor{currentfill}{rgb}{0.412913,0.803041,0.357269}%
\pgfsetfillcolor{currentfill}%
\pgfsetlinewidth{0.000000pt}%
\definecolor{currentstroke}{rgb}{0.000000,0.000000,0.000000}%
\pgfsetstrokecolor{currentstroke}%
\pgfsetdash{}{0pt}%
\pgfpathmoveto{\pgfqpoint{1.228283in}{1.641651in}}%
\pgfpathlineto{\pgfqpoint{1.225981in}{1.637386in}}%
\pgfpathlineto{\pgfqpoint{1.223681in}{1.633049in}}%
\pgfpathlineto{\pgfqpoint{1.221384in}{1.628642in}}%
\pgfpathlineto{\pgfqpoint{1.219090in}{1.624166in}}%
\pgfpathlineto{\pgfqpoint{1.223621in}{1.626111in}}%
\pgfpathlineto{\pgfqpoint{1.228274in}{1.627986in}}%
\pgfpathlineto{\pgfqpoint{1.233046in}{1.629790in}}%
\pgfpathlineto{\pgfqpoint{1.237931in}{1.631521in}}%
\pgfpathlineto{\pgfqpoint{1.239895in}{1.635864in}}%
\pgfpathlineto{\pgfqpoint{1.241862in}{1.640138in}}%
\pgfpathlineto{\pgfqpoint{1.243831in}{1.644342in}}%
\pgfpathlineto{\pgfqpoint{1.245803in}{1.648474in}}%
\pgfpathlineto{\pgfqpoint{1.241260in}{1.646868in}}%
\pgfpathlineto{\pgfqpoint{1.236823in}{1.645194in}}%
\pgfpathlineto{\pgfqpoint{1.232495in}{1.643455in}}%
\pgfpathlineto{\pgfqpoint{1.228283in}{1.641651in}}%
\pgfpathclose%
\pgfusepath{fill}%
\end{pgfscope}%
\begin{pgfscope}%
\pgfpathrectangle{\pgfqpoint{0.329460in}{0.284240in}}{\pgfqpoint{1.989680in}{1.989680in}}%
\pgfusepath{clip}%
\pgfsetbuttcap%
\pgfsetroundjoin%
\definecolor{currentfill}{rgb}{0.212395,0.359683,0.551710}%
\pgfsetfillcolor{currentfill}%
\pgfsetlinewidth{0.000000pt}%
\definecolor{currentstroke}{rgb}{0.000000,0.000000,0.000000}%
\pgfsetstrokecolor{currentstroke}%
\pgfsetdash{}{0pt}%
\pgfpathmoveto{\pgfqpoint{1.703509in}{1.210447in}}%
\pgfpathlineto{\pgfqpoint{1.706788in}{1.203614in}}%
\pgfpathlineto{\pgfqpoint{1.710065in}{1.196827in}}%
\pgfpathlineto{\pgfqpoint{1.713342in}{1.190089in}}%
\pgfpathlineto{\pgfqpoint{1.716617in}{1.183403in}}%
\pgfpathlineto{\pgfqpoint{1.714821in}{1.177531in}}%
\pgfpathlineto{\pgfqpoint{1.712656in}{1.171686in}}%
\pgfpathlineto{\pgfqpoint{1.710125in}{1.165873in}}%
\pgfpathlineto{\pgfqpoint{1.707228in}{1.160099in}}%
\pgfpathlineto{\pgfqpoint{1.704025in}{1.167019in}}%
\pgfpathlineto{\pgfqpoint{1.700822in}{1.173991in}}%
\pgfpathlineto{\pgfqpoint{1.697618in}{1.181012in}}%
\pgfpathlineto{\pgfqpoint{1.694412in}{1.188078in}}%
\pgfpathlineto{\pgfqpoint{1.697216in}{1.193620in}}%
\pgfpathlineto{\pgfqpoint{1.699668in}{1.199199in}}%
\pgfpathlineto{\pgfqpoint{1.701766in}{1.204810in}}%
\pgfpathlineto{\pgfqpoint{1.703509in}{1.210447in}}%
\pgfpathclose%
\pgfusepath{fill}%
\end{pgfscope}%
\begin{pgfscope}%
\pgfpathrectangle{\pgfqpoint{0.329460in}{0.284240in}}{\pgfqpoint{1.989680in}{1.989680in}}%
\pgfusepath{clip}%
\pgfsetbuttcap%
\pgfsetroundjoin%
\definecolor{currentfill}{rgb}{0.133743,0.548535,0.553541}%
\pgfsetfillcolor{currentfill}%
\pgfsetlinewidth{0.000000pt}%
\definecolor{currentstroke}{rgb}{0.000000,0.000000,0.000000}%
\pgfsetstrokecolor{currentstroke}%
\pgfsetdash{}{0pt}%
\pgfpathmoveto{\pgfqpoint{1.638905in}{1.388613in}}%
\pgfpathlineto{\pgfqpoint{1.642216in}{1.381889in}}%
\pgfpathlineto{\pgfqpoint{1.645525in}{1.375158in}}%
\pgfpathlineto{\pgfqpoint{1.648831in}{1.368424in}}%
\pgfpathlineto{\pgfqpoint{1.652134in}{1.361687in}}%
\pgfpathlineto{\pgfqpoint{1.653026in}{1.356969in}}%
\pgfpathlineto{\pgfqpoint{1.653618in}{1.352237in}}%
\pgfpathlineto{\pgfqpoint{1.653909in}{1.347493in}}%
\pgfpathlineto{\pgfqpoint{1.653899in}{1.342744in}}%
\pgfpathlineto{\pgfqpoint{1.650566in}{1.349710in}}%
\pgfpathlineto{\pgfqpoint{1.647231in}{1.356673in}}%
\pgfpathlineto{\pgfqpoint{1.643893in}{1.363633in}}%
\pgfpathlineto{\pgfqpoint{1.640553in}{1.370585in}}%
\pgfpathlineto{\pgfqpoint{1.640572in}{1.375104in}}%
\pgfpathlineto{\pgfqpoint{1.640303in}{1.379618in}}%
\pgfpathlineto{\pgfqpoint{1.639747in}{1.384123in}}%
\pgfpathlineto{\pgfqpoint{1.638905in}{1.388613in}}%
\pgfpathclose%
\pgfusepath{fill}%
\end{pgfscope}%
\begin{pgfscope}%
\pgfpathrectangle{\pgfqpoint{0.329460in}{0.284240in}}{\pgfqpoint{1.989680in}{1.989680in}}%
\pgfusepath{clip}%
\pgfsetbuttcap%
\pgfsetroundjoin%
\definecolor{currentfill}{rgb}{0.487026,0.823929,0.312321}%
\pgfsetfillcolor{currentfill}%
\pgfsetlinewidth{0.000000pt}%
\definecolor{currentstroke}{rgb}{0.000000,0.000000,0.000000}%
\pgfsetstrokecolor{currentstroke}%
\pgfsetdash{}{0pt}%
\pgfpathmoveto{\pgfqpoint{1.271412in}{1.669527in}}%
\pgfpathlineto{\pgfqpoint{1.269791in}{1.665810in}}%
\pgfpathlineto{\pgfqpoint{1.268173in}{1.662015in}}%
\pgfpathlineto{\pgfqpoint{1.266557in}{1.658143in}}%
\pgfpathlineto{\pgfqpoint{1.264943in}{1.654195in}}%
\pgfpathlineto{\pgfqpoint{1.269947in}{1.655443in}}%
\pgfpathlineto{\pgfqpoint{1.275028in}{1.656616in}}%
\pgfpathlineto{\pgfqpoint{1.280183in}{1.657712in}}%
\pgfpathlineto{\pgfqpoint{1.285405in}{1.658730in}}%
\pgfpathlineto{\pgfqpoint{1.286635in}{1.662590in}}%
\pgfpathlineto{\pgfqpoint{1.287867in}{1.666374in}}%
\pgfpathlineto{\pgfqpoint{1.289101in}{1.670082in}}%
\pgfpathlineto{\pgfqpoint{1.290335in}{1.673711in}}%
\pgfpathlineto{\pgfqpoint{1.285505in}{1.672771in}}%
\pgfpathlineto{\pgfqpoint{1.280738in}{1.671760in}}%
\pgfpathlineto{\pgfqpoint{1.276039in}{1.670678in}}%
\pgfpathlineto{\pgfqpoint{1.271412in}{1.669527in}}%
\pgfpathclose%
\pgfusepath{fill}%
\end{pgfscope}%
\begin{pgfscope}%
\pgfpathrectangle{\pgfqpoint{0.329460in}{0.284240in}}{\pgfqpoint{1.989680in}{1.989680in}}%
\pgfusepath{clip}%
\pgfsetbuttcap%
\pgfsetroundjoin%
\definecolor{currentfill}{rgb}{0.280255,0.165693,0.476498}%
\pgfsetfillcolor{currentfill}%
\pgfsetlinewidth{0.000000pt}%
\definecolor{currentstroke}{rgb}{0.000000,0.000000,0.000000}%
\pgfsetstrokecolor{currentstroke}%
\pgfsetdash{}{0pt}%
\pgfpathmoveto{\pgfqpoint{0.965825in}{1.026957in}}%
\pgfpathlineto{\pgfqpoint{0.962778in}{1.021107in}}%
\pgfpathlineto{\pgfqpoint{0.959729in}{1.015366in}}%
\pgfpathlineto{\pgfqpoint{0.956679in}{1.009737in}}%
\pgfpathlineto{\pgfqpoint{0.953628in}{1.004222in}}%
\pgfpathlineto{\pgfqpoint{0.948202in}{1.010846in}}%
\pgfpathlineto{\pgfqpoint{0.943194in}{1.017548in}}%
\pgfpathlineto{\pgfqpoint{0.938609in}{1.024323in}}%
\pgfpathlineto{\pgfqpoint{0.934449in}{1.031163in}}%
\pgfpathlineto{\pgfqpoint{0.937634in}{1.036449in}}%
\pgfpathlineto{\pgfqpoint{0.940819in}{1.041850in}}%
\pgfpathlineto{\pgfqpoint{0.944001in}{1.047363in}}%
\pgfpathlineto{\pgfqpoint{0.947183in}{1.052984in}}%
\pgfpathlineto{\pgfqpoint{0.951229in}{1.046375in}}%
\pgfpathlineto{\pgfqpoint{0.955687in}{1.039830in}}%
\pgfpathlineto{\pgfqpoint{0.960553in}{1.033355in}}%
\pgfpathlineto{\pgfqpoint{0.965825in}{1.026957in}}%
\pgfpathclose%
\pgfusepath{fill}%
\end{pgfscope}%
\begin{pgfscope}%
\pgfpathrectangle{\pgfqpoint{0.329460in}{0.284240in}}{\pgfqpoint{1.989680in}{1.989680in}}%
\pgfusepath{clip}%
\pgfsetbuttcap%
\pgfsetroundjoin%
\definecolor{currentfill}{rgb}{0.120081,0.622161,0.534946}%
\pgfsetfillcolor{currentfill}%
\pgfsetlinewidth{0.000000pt}%
\definecolor{currentstroke}{rgb}{0.000000,0.000000,0.000000}%
\pgfsetstrokecolor{currentstroke}%
\pgfsetdash{}{0pt}%
\pgfpathmoveto{\pgfqpoint{1.606751in}{1.457834in}}%
\pgfpathlineto{\pgfqpoint{1.610006in}{1.451470in}}%
\pgfpathlineto{\pgfqpoint{1.613258in}{1.445081in}}%
\pgfpathlineto{\pgfqpoint{1.616507in}{1.438669in}}%
\pgfpathlineto{\pgfqpoint{1.619753in}{1.432236in}}%
\pgfpathlineto{\pgfqpoint{1.621624in}{1.428066in}}%
\pgfpathlineto{\pgfqpoint{1.623228in}{1.423867in}}%
\pgfpathlineto{\pgfqpoint{1.624564in}{1.419641in}}%
\pgfpathlineto{\pgfqpoint{1.625630in}{1.415394in}}%
\pgfpathlineto{\pgfqpoint{1.622304in}{1.422049in}}%
\pgfpathlineto{\pgfqpoint{1.618976in}{1.428683in}}%
\pgfpathlineto{\pgfqpoint{1.615644in}{1.435293in}}%
\pgfpathlineto{\pgfqpoint{1.612309in}{1.441878in}}%
\pgfpathlineto{\pgfqpoint{1.611304in}{1.445901in}}%
\pgfpathlineto{\pgfqpoint{1.610041in}{1.449904in}}%
\pgfpathlineto{\pgfqpoint{1.608523in}{1.453883in}}%
\pgfpathlineto{\pgfqpoint{1.606751in}{1.457834in}}%
\pgfpathclose%
\pgfusepath{fill}%
\end{pgfscope}%
\begin{pgfscope}%
\pgfpathrectangle{\pgfqpoint{0.329460in}{0.284240in}}{\pgfqpoint{1.989680in}{1.989680in}}%
\pgfusepath{clip}%
\pgfsetbuttcap%
\pgfsetroundjoin%
\definecolor{currentfill}{rgb}{0.163625,0.471133,0.558148}%
\pgfsetfillcolor{currentfill}%
\pgfsetlinewidth{0.000000pt}%
\definecolor{currentstroke}{rgb}{0.000000,0.000000,0.000000}%
\pgfsetstrokecolor{currentstroke}%
\pgfsetdash{}{0pt}%
\pgfpathmoveto{\pgfqpoint{1.667206in}{1.314912in}}%
\pgfpathlineto{\pgfqpoint{1.670527in}{1.307975in}}%
\pgfpathlineto{\pgfqpoint{1.673845in}{1.301050in}}%
\pgfpathlineto{\pgfqpoint{1.677162in}{1.294142in}}%
\pgfpathlineto{\pgfqpoint{1.680476in}{1.287253in}}%
\pgfpathlineto{\pgfqpoint{1.680159in}{1.282039in}}%
\pgfpathlineto{\pgfqpoint{1.679512in}{1.276828in}}%
\pgfpathlineto{\pgfqpoint{1.678535in}{1.271625in}}%
\pgfpathlineto{\pgfqpoint{1.677229in}{1.266437in}}%
\pgfpathlineto{\pgfqpoint{1.673936in}{1.273559in}}%
\pgfpathlineto{\pgfqpoint{1.670642in}{1.280700in}}%
\pgfpathlineto{\pgfqpoint{1.667346in}{1.287857in}}%
\pgfpathlineto{\pgfqpoint{1.664047in}{1.295026in}}%
\pgfpathlineto{\pgfqpoint{1.665311in}{1.299982in}}%
\pgfpathlineto{\pgfqpoint{1.666259in}{1.304952in}}%
\pgfpathlineto{\pgfqpoint{1.666890in}{1.309930in}}%
\pgfpathlineto{\pgfqpoint{1.667206in}{1.314912in}}%
\pgfpathclose%
\pgfusepath{fill}%
\end{pgfscope}%
\begin{pgfscope}%
\pgfpathrectangle{\pgfqpoint{0.329460in}{0.284240in}}{\pgfqpoint{1.989680in}{1.989680in}}%
\pgfusepath{clip}%
\pgfsetbuttcap%
\pgfsetroundjoin%
\definecolor{currentfill}{rgb}{0.281477,0.755203,0.432552}%
\pgfsetfillcolor{currentfill}%
\pgfsetlinewidth{0.000000pt}%
\definecolor{currentstroke}{rgb}{0.000000,0.000000,0.000000}%
\pgfsetstrokecolor{currentstroke}%
\pgfsetdash{}{0pt}%
\pgfpathmoveto{\pgfqpoint{1.522725in}{1.588811in}}%
\pgfpathlineto{\pgfqpoint{1.525502in}{1.583733in}}%
\pgfpathlineto{\pgfqpoint{1.528276in}{1.578597in}}%
\pgfpathlineto{\pgfqpoint{1.531046in}{1.573406in}}%
\pgfpathlineto{\pgfqpoint{1.533813in}{1.568160in}}%
\pgfpathlineto{\pgfqpoint{1.537672in}{1.565368in}}%
\pgfpathlineto{\pgfqpoint{1.541350in}{1.562519in}}%
\pgfpathlineto{\pgfqpoint{1.544844in}{1.559613in}}%
\pgfpathlineto{\pgfqpoint{1.548149in}{1.556655in}}%
\pgfpathlineto{\pgfqpoint{1.545159in}{1.562085in}}%
\pgfpathlineto{\pgfqpoint{1.542166in}{1.567462in}}%
\pgfpathlineto{\pgfqpoint{1.539169in}{1.572782in}}%
\pgfpathlineto{\pgfqpoint{1.536169in}{1.578043in}}%
\pgfpathlineto{\pgfqpoint{1.533070in}{1.580812in}}%
\pgfpathlineto{\pgfqpoint{1.529794in}{1.583531in}}%
\pgfpathlineto{\pgfqpoint{1.526345in}{1.586198in}}%
\pgfpathlineto{\pgfqpoint{1.522725in}{1.588811in}}%
\pgfpathclose%
\pgfusepath{fill}%
\end{pgfscope}%
\begin{pgfscope}%
\pgfpathrectangle{\pgfqpoint{0.329460in}{0.284240in}}{\pgfqpoint{1.989680in}{1.989680in}}%
\pgfusepath{clip}%
\pgfsetbuttcap%
\pgfsetroundjoin%
\definecolor{currentfill}{rgb}{0.344074,0.780029,0.397381}%
\pgfsetfillcolor{currentfill}%
\pgfsetlinewidth{0.000000pt}%
\definecolor{currentstroke}{rgb}{0.000000,0.000000,0.000000}%
\pgfsetstrokecolor{currentstroke}%
\pgfsetdash{}{0pt}%
\pgfpathmoveto{\pgfqpoint{1.496547in}{1.617693in}}%
\pgfpathlineto{\pgfqpoint{1.499071in}{1.613033in}}%
\pgfpathlineto{\pgfqpoint{1.501592in}{1.608308in}}%
\pgfpathlineto{\pgfqpoint{1.504110in}{1.603519in}}%
\pgfpathlineto{\pgfqpoint{1.506625in}{1.598669in}}%
\pgfpathlineto{\pgfqpoint{1.510885in}{1.596298in}}%
\pgfpathlineto{\pgfqpoint{1.514992in}{1.593863in}}%
\pgfpathlineto{\pgfqpoint{1.518939in}{1.591367in}}%
\pgfpathlineto{\pgfqpoint{1.522725in}{1.588811in}}%
\pgfpathlineto{\pgfqpoint{1.519945in}{1.593829in}}%
\pgfpathlineto{\pgfqpoint{1.517162in}{1.598785in}}%
\pgfpathlineto{\pgfqpoint{1.514376in}{1.603678in}}%
\pgfpathlineto{\pgfqpoint{1.511586in}{1.608505in}}%
\pgfpathlineto{\pgfqpoint{1.508051in}{1.610887in}}%
\pgfpathlineto{\pgfqpoint{1.504363in}{1.613214in}}%
\pgfpathlineto{\pgfqpoint{1.500527in}{1.615483in}}%
\pgfpathlineto{\pgfqpoint{1.496547in}{1.617693in}}%
\pgfpathclose%
\pgfusepath{fill}%
\end{pgfscope}%
\begin{pgfscope}%
\pgfpathrectangle{\pgfqpoint{0.329460in}{0.284240in}}{\pgfqpoint{1.989680in}{1.989680in}}%
\pgfusepath{clip}%
\pgfsetbuttcap%
\pgfsetroundjoin%
\definecolor{currentfill}{rgb}{0.277941,0.056324,0.381191}%
\pgfsetfillcolor{currentfill}%
\pgfsetlinewidth{0.000000pt}%
\definecolor{currentstroke}{rgb}{0.000000,0.000000,0.000000}%
\pgfsetstrokecolor{currentstroke}%
\pgfsetdash{}{0pt}%
\pgfpathmoveto{\pgfqpoint{1.902752in}{0.964911in}}%
\pgfpathlineto{\pgfqpoint{1.906180in}{0.967701in}}%
\pgfpathlineto{\pgfqpoint{1.909618in}{0.970778in}}%
\pgfpathlineto{\pgfqpoint{1.913068in}{0.974149in}}%
\pgfpathlineto{\pgfqpoint{1.916528in}{0.977818in}}%
\pgfpathlineto{\pgfqpoint{1.911648in}{0.968454in}}%
\pgfpathlineto{\pgfqpoint{1.906186in}{0.959163in}}%
\pgfpathlineto{\pgfqpoint{1.900145in}{0.949955in}}%
\pgfpathlineto{\pgfqpoint{1.893530in}{0.940840in}}%
\pgfpathlineto{\pgfqpoint{1.890191in}{0.937378in}}%
\pgfpathlineto{\pgfqpoint{1.886864in}{0.934216in}}%
\pgfpathlineto{\pgfqpoint{1.883547in}{0.931348in}}%
\pgfpathlineto{\pgfqpoint{1.880241in}{0.928769in}}%
\pgfpathlineto{\pgfqpoint{1.886712in}{0.937677in}}%
\pgfpathlineto{\pgfqpoint{1.892624in}{0.946677in}}%
\pgfpathlineto{\pgfqpoint{1.897971in}{0.955758in}}%
\pgfpathlineto{\pgfqpoint{1.902752in}{0.964911in}}%
\pgfpathclose%
\pgfusepath{fill}%
\end{pgfscope}%
\begin{pgfscope}%
\pgfpathrectangle{\pgfqpoint{0.329460in}{0.284240in}}{\pgfqpoint{1.989680in}{1.989680in}}%
\pgfusepath{clip}%
\pgfsetbuttcap%
\pgfsetroundjoin%
\definecolor{currentfill}{rgb}{0.220124,0.725509,0.466226}%
\pgfsetfillcolor{currentfill}%
\pgfsetlinewidth{0.000000pt}%
\definecolor{currentstroke}{rgb}{0.000000,0.000000,0.000000}%
\pgfsetstrokecolor{currentstroke}%
\pgfsetdash{}{0pt}%
\pgfpathmoveto{\pgfqpoint{1.548149in}{1.556655in}}%
\pgfpathlineto{\pgfqpoint{1.551136in}{1.551172in}}%
\pgfpathlineto{\pgfqpoint{1.554120in}{1.545639in}}%
\pgfpathlineto{\pgfqpoint{1.557100in}{1.540057in}}%
\pgfpathlineto{\pgfqpoint{1.560077in}{1.534429in}}%
\pgfpathlineto{\pgfqpoint{1.563386in}{1.531226in}}%
\pgfpathlineto{\pgfqpoint{1.566489in}{1.527973in}}%
\pgfpathlineto{\pgfqpoint{1.569382in}{1.524673in}}%
\pgfpathlineto{\pgfqpoint{1.572062in}{1.521328in}}%
\pgfpathlineto{\pgfqpoint{1.568908in}{1.527155in}}%
\pgfpathlineto{\pgfqpoint{1.565750in}{1.532936in}}%
\pgfpathlineto{\pgfqpoint{1.562589in}{1.538668in}}%
\pgfpathlineto{\pgfqpoint{1.559424in}{1.544349in}}%
\pgfpathlineto{\pgfqpoint{1.556904in}{1.547491in}}%
\pgfpathlineto{\pgfqpoint{1.554183in}{1.550591in}}%
\pgfpathlineto{\pgfqpoint{1.551264in}{1.553646in}}%
\pgfpathlineto{\pgfqpoint{1.548149in}{1.556655in}}%
\pgfpathclose%
\pgfusepath{fill}%
\end{pgfscope}%
\begin{pgfscope}%
\pgfpathrectangle{\pgfqpoint{0.329460in}{0.284240in}}{\pgfqpoint{1.989680in}{1.989680in}}%
\pgfusepath{clip}%
\pgfsetbuttcap%
\pgfsetroundjoin%
\definecolor{currentfill}{rgb}{0.272594,0.025563,0.353093}%
\pgfsetfillcolor{currentfill}%
\pgfsetlinewidth{0.000000pt}%
\definecolor{currentstroke}{rgb}{0.000000,0.000000,0.000000}%
\pgfsetstrokecolor{currentstroke}%
\pgfsetdash{}{0pt}%
\pgfpathmoveto{\pgfqpoint{0.841342in}{0.913596in}}%
\pgfpathlineto{\pgfqpoint{0.838109in}{0.915022in}}%
\pgfpathlineto{\pgfqpoint{0.834867in}{0.916717in}}%
\pgfpathlineto{\pgfqpoint{0.831615in}{0.918686in}}%
\pgfpathlineto{\pgfqpoint{0.828353in}{0.920935in}}%
\pgfpathlineto{\pgfqpoint{0.821388in}{0.929754in}}%
\pgfpathlineto{\pgfqpoint{0.814978in}{0.938673in}}%
\pgfpathlineto{\pgfqpoint{0.809129in}{0.947682in}}%
\pgfpathlineto{\pgfqpoint{0.803845in}{0.956772in}}%
\pgfpathlineto{\pgfqpoint{0.807240in}{0.954313in}}%
\pgfpathlineto{\pgfqpoint{0.810626in}{0.952133in}}%
\pgfpathlineto{\pgfqpoint{0.814003in}{0.950225in}}%
\pgfpathlineto{\pgfqpoint{0.817370in}{0.948587in}}%
\pgfpathlineto{\pgfqpoint{0.822542in}{0.939711in}}%
\pgfpathlineto{\pgfqpoint{0.828264in}{0.930914in}}%
\pgfpathlineto{\pgfqpoint{0.834532in}{0.922206in}}%
\pgfpathlineto{\pgfqpoint{0.841342in}{0.913596in}}%
\pgfpathclose%
\pgfusepath{fill}%
\end{pgfscope}%
\begin{pgfscope}%
\pgfpathrectangle{\pgfqpoint{0.329460in}{0.284240in}}{\pgfqpoint{1.989680in}{1.989680in}}%
\pgfusepath{clip}%
\pgfsetbuttcap%
\pgfsetroundjoin%
\definecolor{currentfill}{rgb}{0.276194,0.190074,0.493001}%
\pgfsetfillcolor{currentfill}%
\pgfsetlinewidth{0.000000pt}%
\definecolor{currentstroke}{rgb}{0.000000,0.000000,0.000000}%
\pgfsetstrokecolor{currentstroke}%
\pgfsetdash{}{0pt}%
\pgfpathmoveto{\pgfqpoint{0.762205in}{1.009872in}}%
\pgfpathlineto{\pgfqpoint{0.758650in}{1.016429in}}%
\pgfpathlineto{\pgfqpoint{0.755081in}{1.023341in}}%
\pgfpathlineto{\pgfqpoint{0.751497in}{1.030613in}}%
\pgfpathlineto{\pgfqpoint{0.747897in}{1.038252in}}%
\pgfpathlineto{\pgfqpoint{0.742785in}{1.048222in}}%
\pgfpathlineto{\pgfqpoint{0.738303in}{1.058255in}}%
\pgfpathlineto{\pgfqpoint{0.734452in}{1.068341in}}%
\pgfpathlineto{\pgfqpoint{0.731235in}{1.078469in}}%
\pgfpathlineto{\pgfqpoint{0.734911in}{1.070639in}}%
\pgfpathlineto{\pgfqpoint{0.738572in}{1.063174in}}%
\pgfpathlineto{\pgfqpoint{0.742218in}{1.056067in}}%
\pgfpathlineto{\pgfqpoint{0.745849in}{1.049314in}}%
\pgfpathlineto{\pgfqpoint{0.749013in}{1.039380in}}%
\pgfpathlineto{\pgfqpoint{0.752794in}{1.029488in}}%
\pgfpathlineto{\pgfqpoint{0.757192in}{1.019649in}}%
\pgfpathlineto{\pgfqpoint{0.762205in}{1.009872in}}%
\pgfpathclose%
\pgfusepath{fill}%
\end{pgfscope}%
\begin{pgfscope}%
\pgfpathrectangle{\pgfqpoint{0.329460in}{0.284240in}}{\pgfqpoint{1.989680in}{1.989680in}}%
\pgfusepath{clip}%
\pgfsetbuttcap%
\pgfsetroundjoin%
\definecolor{currentfill}{rgb}{0.487026,0.823929,0.312321}%
\pgfsetfillcolor{currentfill}%
\pgfsetlinewidth{0.000000pt}%
\definecolor{currentstroke}{rgb}{0.000000,0.000000,0.000000}%
\pgfsetstrokecolor{currentstroke}%
\pgfsetdash{}{0pt}%
\pgfpathmoveto{\pgfqpoint{1.426854in}{1.670554in}}%
\pgfpathlineto{\pgfqpoint{1.428391in}{1.666859in}}%
\pgfpathlineto{\pgfqpoint{1.429925in}{1.663085in}}%
\pgfpathlineto{\pgfqpoint{1.431458in}{1.659234in}}%
\pgfpathlineto{\pgfqpoint{1.432988in}{1.655308in}}%
\pgfpathlineto{\pgfqpoint{1.437983in}{1.654052in}}%
\pgfpathlineto{\pgfqpoint{1.442895in}{1.652722in}}%
\pgfpathlineto{\pgfqpoint{1.447718in}{1.651319in}}%
\pgfpathlineto{\pgfqpoint{1.452449in}{1.649844in}}%
\pgfpathlineto{\pgfqpoint{1.450552in}{1.653876in}}%
\pgfpathlineto{\pgfqpoint{1.448653in}{1.657832in}}%
\pgfpathlineto{\pgfqpoint{1.446752in}{1.661712in}}%
\pgfpathlineto{\pgfqpoint{1.444849in}{1.665513in}}%
\pgfpathlineto{\pgfqpoint{1.440475in}{1.666873in}}%
\pgfpathlineto{\pgfqpoint{1.436015in}{1.668168in}}%
\pgfpathlineto{\pgfqpoint{1.431473in}{1.669395in}}%
\pgfpathlineto{\pgfqpoint{1.426854in}{1.670554in}}%
\pgfpathclose%
\pgfusepath{fill}%
\end{pgfscope}%
\begin{pgfscope}%
\pgfpathrectangle{\pgfqpoint{0.329460in}{0.284240in}}{\pgfqpoint{1.989680in}{1.989680in}}%
\pgfusepath{clip}%
\pgfsetbuttcap%
\pgfsetroundjoin%
\definecolor{currentfill}{rgb}{0.133743,0.548535,0.553541}%
\pgfsetfillcolor{currentfill}%
\pgfsetlinewidth{0.000000pt}%
\definecolor{currentstroke}{rgb}{0.000000,0.000000,0.000000}%
\pgfsetstrokecolor{currentstroke}%
\pgfsetdash{}{0pt}%
\pgfpathmoveto{\pgfqpoint{1.062082in}{1.366567in}}%
\pgfpathlineto{\pgfqpoint{1.058743in}{1.359563in}}%
\pgfpathlineto{\pgfqpoint{1.055406in}{1.352553in}}%
\pgfpathlineto{\pgfqpoint{1.052071in}{1.345538in}}%
\pgfpathlineto{\pgfqpoint{1.048739in}{1.338521in}}%
\pgfpathlineto{\pgfqpoint{1.048460in}{1.343272in}}%
\pgfpathlineto{\pgfqpoint{1.048484in}{1.348021in}}%
\pgfpathlineto{\pgfqpoint{1.048808in}{1.352763in}}%
\pgfpathlineto{\pgfqpoint{1.049434in}{1.357494in}}%
\pgfpathlineto{\pgfqpoint{1.052748in}{1.364282in}}%
\pgfpathlineto{\pgfqpoint{1.056065in}{1.371067in}}%
\pgfpathlineto{\pgfqpoint{1.059385in}{1.377848in}}%
\pgfpathlineto{\pgfqpoint{1.062707in}{1.384623in}}%
\pgfpathlineto{\pgfqpoint{1.062119in}{1.380119in}}%
\pgfpathlineto{\pgfqpoint{1.061819in}{1.375606in}}%
\pgfpathlineto{\pgfqpoint{1.061806in}{1.371087in}}%
\pgfpathlineto{\pgfqpoint{1.062082in}{1.366567in}}%
\pgfpathclose%
\pgfusepath{fill}%
\end{pgfscope}%
\begin{pgfscope}%
\pgfpathrectangle{\pgfqpoint{0.329460in}{0.284240in}}{\pgfqpoint{1.989680in}{1.989680in}}%
\pgfusepath{clip}%
\pgfsetbuttcap%
\pgfsetroundjoin%
\definecolor{currentfill}{rgb}{0.263663,0.237631,0.518762}%
\pgfsetfillcolor{currentfill}%
\pgfsetlinewidth{0.000000pt}%
\definecolor{currentstroke}{rgb}{0.000000,0.000000,0.000000}%
\pgfsetstrokecolor{currentstroke}%
\pgfsetdash{}{0pt}%
\pgfpathmoveto{\pgfqpoint{1.732832in}{1.106965in}}%
\pgfpathlineto{\pgfqpoint{1.736031in}{1.100647in}}%
\pgfpathlineto{\pgfqpoint{1.739231in}{1.094411in}}%
\pgfpathlineto{\pgfqpoint{1.742431in}{1.088261in}}%
\pgfpathlineto{\pgfqpoint{1.745632in}{1.082199in}}%
\pgfpathlineto{\pgfqpoint{1.742058in}{1.075774in}}%
\pgfpathlineto{\pgfqpoint{1.738082in}{1.069403in}}%
\pgfpathlineto{\pgfqpoint{1.733708in}{1.063095in}}%
\pgfpathlineto{\pgfqpoint{1.728940in}{1.056855in}}%
\pgfpathlineto{\pgfqpoint{1.725863in}{1.063147in}}%
\pgfpathlineto{\pgfqpoint{1.722786in}{1.069527in}}%
\pgfpathlineto{\pgfqpoint{1.719710in}{1.075993in}}%
\pgfpathlineto{\pgfqpoint{1.716635in}{1.082540in}}%
\pgfpathlineto{\pgfqpoint{1.721260in}{1.088553in}}%
\pgfpathlineto{\pgfqpoint{1.725503in}{1.094632in}}%
\pgfpathlineto{\pgfqpoint{1.729361in}{1.100771in}}%
\pgfpathlineto{\pgfqpoint{1.732832in}{1.106965in}}%
\pgfpathclose%
\pgfusepath{fill}%
\end{pgfscope}%
\begin{pgfscope}%
\pgfpathrectangle{\pgfqpoint{0.329460in}{0.284240in}}{\pgfqpoint{1.989680in}{1.989680in}}%
\pgfusepath{clip}%
\pgfsetbuttcap%
\pgfsetroundjoin%
\definecolor{currentfill}{rgb}{0.281477,0.755203,0.432552}%
\pgfsetfillcolor{currentfill}%
\pgfsetlinewidth{0.000000pt}%
\definecolor{currentstroke}{rgb}{0.000000,0.000000,0.000000}%
\pgfsetstrokecolor{currentstroke}%
\pgfsetdash{}{0pt}%
\pgfpathmoveto{\pgfqpoint{1.163602in}{1.575543in}}%
\pgfpathlineto{\pgfqpoint{1.160559in}{1.570239in}}%
\pgfpathlineto{\pgfqpoint{1.157519in}{1.564876in}}%
\pgfpathlineto{\pgfqpoint{1.154482in}{1.559457in}}%
\pgfpathlineto{\pgfqpoint{1.151448in}{1.553983in}}%
\pgfpathlineto{\pgfqpoint{1.154584in}{1.556986in}}%
\pgfpathlineto{\pgfqpoint{1.157910in}{1.559939in}}%
\pgfpathlineto{\pgfqpoint{1.161425in}{1.562838in}}%
\pgfpathlineto{\pgfqpoint{1.165123in}{1.565681in}}%
\pgfpathlineto{\pgfqpoint{1.167944in}{1.570967in}}%
\pgfpathlineto{\pgfqpoint{1.170767in}{1.576199in}}%
\pgfpathlineto{\pgfqpoint{1.173594in}{1.581374in}}%
\pgfpathlineto{\pgfqpoint{1.176424in}{1.586491in}}%
\pgfpathlineto{\pgfqpoint{1.172956in}{1.583830in}}%
\pgfpathlineto{\pgfqpoint{1.169660in}{1.581116in}}%
\pgfpathlineto{\pgfqpoint{1.166541in}{1.578353in}}%
\pgfpathlineto{\pgfqpoint{1.163602in}{1.575543in}}%
\pgfpathclose%
\pgfusepath{fill}%
\end{pgfscope}%
\begin{pgfscope}%
\pgfpathrectangle{\pgfqpoint{0.329460in}{0.284240in}}{\pgfqpoint{1.989680in}{1.989680in}}%
\pgfusepath{clip}%
\pgfsetbuttcap%
\pgfsetroundjoin%
\definecolor{currentfill}{rgb}{0.212395,0.359683,0.551710}%
\pgfsetfillcolor{currentfill}%
\pgfsetlinewidth{0.000000pt}%
\definecolor{currentstroke}{rgb}{0.000000,0.000000,0.000000}%
\pgfsetstrokecolor{currentstroke}%
\pgfsetdash{}{0pt}%
\pgfpathmoveto{\pgfqpoint{1.010749in}{1.183188in}}%
\pgfpathlineto{\pgfqpoint{1.007568in}{1.176070in}}%
\pgfpathlineto{\pgfqpoint{1.004387in}{1.168998in}}%
\pgfpathlineto{\pgfqpoint{1.001207in}{1.161975in}}%
\pgfpathlineto{\pgfqpoint{0.998027in}{1.155004in}}%
\pgfpathlineto{\pgfqpoint{0.994807in}{1.160739in}}%
\pgfpathlineto{\pgfqpoint{0.991951in}{1.166517in}}%
\pgfpathlineto{\pgfqpoint{0.989460in}{1.172334in}}%
\pgfpathlineto{\pgfqpoint{0.987337in}{1.178182in}}%
\pgfpathlineto{\pgfqpoint{0.990600in}{1.184920in}}%
\pgfpathlineto{\pgfqpoint{0.993865in}{1.191711in}}%
\pgfpathlineto{\pgfqpoint{0.997131in}{1.198550in}}%
\pgfpathlineto{\pgfqpoint{1.000398in}{1.205435in}}%
\pgfpathlineto{\pgfqpoint{1.002456in}{1.199821in}}%
\pgfpathlineto{\pgfqpoint{1.004869in}{1.194238in}}%
\pgfpathlineto{\pgfqpoint{1.007634in}{1.188692in}}%
\pgfpathlineto{\pgfqpoint{1.010749in}{1.183188in}}%
\pgfpathclose%
\pgfusepath{fill}%
\end{pgfscope}%
\begin{pgfscope}%
\pgfpathrectangle{\pgfqpoint{0.329460in}{0.284240in}}{\pgfqpoint{1.989680in}{1.989680in}}%
\pgfusepath{clip}%
\pgfsetbuttcap%
\pgfsetroundjoin%
\definecolor{currentfill}{rgb}{0.274128,0.199721,0.498911}%
\pgfsetfillcolor{currentfill}%
\pgfsetlinewidth{0.000000pt}%
\definecolor{currentstroke}{rgb}{0.000000,0.000000,0.000000}%
\pgfsetstrokecolor{currentstroke}%
\pgfsetdash{}{0pt}%
\pgfpathmoveto{\pgfqpoint{0.978003in}{1.051372in}}%
\pgfpathlineto{\pgfqpoint{0.974960in}{1.045122in}}%
\pgfpathlineto{\pgfqpoint{0.971916in}{1.038968in}}%
\pgfpathlineto{\pgfqpoint{0.968871in}{1.032912in}}%
\pgfpathlineto{\pgfqpoint{0.965825in}{1.026957in}}%
\pgfpathlineto{\pgfqpoint{0.960553in}{1.033355in}}%
\pgfpathlineto{\pgfqpoint{0.955687in}{1.039830in}}%
\pgfpathlineto{\pgfqpoint{0.951229in}{1.046375in}}%
\pgfpathlineto{\pgfqpoint{0.947183in}{1.052984in}}%
\pgfpathlineto{\pgfqpoint{0.950364in}{1.058710in}}%
\pgfpathlineto{\pgfqpoint{0.953543in}{1.064537in}}%
\pgfpathlineto{\pgfqpoint{0.956722in}{1.070464in}}%
\pgfpathlineto{\pgfqpoint{0.959901in}{1.076485in}}%
\pgfpathlineto{\pgfqpoint{0.963832in}{1.070108in}}%
\pgfpathlineto{\pgfqpoint{0.968161in}{1.063792in}}%
\pgfpathlineto{\pgfqpoint{0.972886in}{1.057545in}}%
\pgfpathlineto{\pgfqpoint{0.978003in}{1.051372in}}%
\pgfpathclose%
\pgfusepath{fill}%
\end{pgfscope}%
\begin{pgfscope}%
\pgfpathrectangle{\pgfqpoint{0.329460in}{0.284240in}}{\pgfqpoint{1.989680in}{1.989680in}}%
\pgfusepath{clip}%
\pgfsetbuttcap%
\pgfsetroundjoin%
\definecolor{currentfill}{rgb}{0.344074,0.780029,0.397381}%
\pgfsetfillcolor{currentfill}%
\pgfsetlinewidth{0.000000pt}%
\definecolor{currentstroke}{rgb}{0.000000,0.000000,0.000000}%
\pgfsetstrokecolor{currentstroke}%
\pgfsetdash{}{0pt}%
\pgfpathmoveto{\pgfqpoint{1.187777in}{1.606343in}}%
\pgfpathlineto{\pgfqpoint{1.184934in}{1.601477in}}%
\pgfpathlineto{\pgfqpoint{1.182094in}{1.596544in}}%
\pgfpathlineto{\pgfqpoint{1.179258in}{1.591549in}}%
\pgfpathlineto{\pgfqpoint{1.176424in}{1.586491in}}%
\pgfpathlineto{\pgfqpoint{1.180063in}{1.589098in}}%
\pgfpathlineto{\pgfqpoint{1.183866in}{1.591647in}}%
\pgfpathlineto{\pgfqpoint{1.187832in}{1.594137in}}%
\pgfpathlineto{\pgfqpoint{1.191956in}{1.596565in}}%
\pgfpathlineto{\pgfqpoint{1.194533in}{1.601451in}}%
\pgfpathlineto{\pgfqpoint{1.197113in}{1.606275in}}%
\pgfpathlineto{\pgfqpoint{1.199697in}{1.611036in}}%
\pgfpathlineto{\pgfqpoint{1.202283in}{1.615732in}}%
\pgfpathlineto{\pgfqpoint{1.198431in}{1.613469in}}%
\pgfpathlineto{\pgfqpoint{1.194727in}{1.611148in}}%
\pgfpathlineto{\pgfqpoint{1.191174in}{1.608772in}}%
\pgfpathlineto{\pgfqpoint{1.187777in}{1.606343in}}%
\pgfpathclose%
\pgfusepath{fill}%
\end{pgfscope}%
\begin{pgfscope}%
\pgfpathrectangle{\pgfqpoint{0.329460in}{0.284240in}}{\pgfqpoint{1.989680in}{1.989680in}}%
\pgfusepath{clip}%
\pgfsetbuttcap%
\pgfsetroundjoin%
\definecolor{currentfill}{rgb}{0.120081,0.622161,0.534946}%
\pgfsetfillcolor{currentfill}%
\pgfsetlinewidth{0.000000pt}%
\definecolor{currentstroke}{rgb}{0.000000,0.000000,0.000000}%
\pgfsetstrokecolor{currentstroke}%
\pgfsetdash{}{0pt}%
\pgfpathmoveto{\pgfqpoint{1.089389in}{1.438288in}}%
\pgfpathlineto{\pgfqpoint{1.086044in}{1.431653in}}%
\pgfpathlineto{\pgfqpoint{1.082701in}{1.424993in}}%
\pgfpathlineto{\pgfqpoint{1.079362in}{1.418309in}}%
\pgfpathlineto{\pgfqpoint{1.076025in}{1.411605in}}%
\pgfpathlineto{\pgfqpoint{1.076850in}{1.415867in}}%
\pgfpathlineto{\pgfqpoint{1.077946in}{1.420112in}}%
\pgfpathlineto{\pgfqpoint{1.079312in}{1.424335in}}%
\pgfpathlineto{\pgfqpoint{1.080946in}{1.428531in}}%
\pgfpathlineto{\pgfqpoint{1.084214in}{1.435013in}}%
\pgfpathlineto{\pgfqpoint{1.087485in}{1.441473in}}%
\pgfpathlineto{\pgfqpoint{1.090759in}{1.447911in}}%
\pgfpathlineto{\pgfqpoint{1.094037in}{1.454324in}}%
\pgfpathlineto{\pgfqpoint{1.092490in}{1.450348in}}%
\pgfpathlineto{\pgfqpoint{1.091199in}{1.446347in}}%
\pgfpathlineto{\pgfqpoint{1.090165in}{1.442326in}}%
\pgfpathlineto{\pgfqpoint{1.089389in}{1.438288in}}%
\pgfpathclose%
\pgfusepath{fill}%
\end{pgfscope}%
\begin{pgfscope}%
\pgfpathrectangle{\pgfqpoint{0.329460in}{0.284240in}}{\pgfqpoint{1.989680in}{1.989680in}}%
\pgfusepath{clip}%
\pgfsetbuttcap%
\pgfsetroundjoin%
\definecolor{currentfill}{rgb}{0.487026,0.823929,0.312321}%
\pgfsetfillcolor{currentfill}%
\pgfsetlinewidth{0.000000pt}%
\definecolor{currentstroke}{rgb}{0.000000,0.000000,0.000000}%
\pgfsetstrokecolor{currentstroke}%
\pgfsetdash{}{0pt}%
\pgfpathmoveto{\pgfqpoint{1.253714in}{1.664249in}}%
\pgfpathlineto{\pgfqpoint{1.251733in}{1.660422in}}%
\pgfpathlineto{\pgfqpoint{1.249754in}{1.656516in}}%
\pgfpathlineto{\pgfqpoint{1.247777in}{1.652532in}}%
\pgfpathlineto{\pgfqpoint{1.245803in}{1.648474in}}%
\pgfpathlineto{\pgfqpoint{1.250447in}{1.650011in}}%
\pgfpathlineto{\pgfqpoint{1.255189in}{1.651478in}}%
\pgfpathlineto{\pgfqpoint{1.260022in}{1.652873in}}%
\pgfpathlineto{\pgfqpoint{1.264943in}{1.654195in}}%
\pgfpathlineto{\pgfqpoint{1.266557in}{1.658143in}}%
\pgfpathlineto{\pgfqpoint{1.268173in}{1.662015in}}%
\pgfpathlineto{\pgfqpoint{1.269791in}{1.665810in}}%
\pgfpathlineto{\pgfqpoint{1.271412in}{1.669527in}}%
\pgfpathlineto{\pgfqpoint{1.266861in}{1.668307in}}%
\pgfpathlineto{\pgfqpoint{1.262392in}{1.667020in}}%
\pgfpathlineto{\pgfqpoint{1.258008in}{1.665667in}}%
\pgfpathlineto{\pgfqpoint{1.253714in}{1.664249in}}%
\pgfpathclose%
\pgfusepath{fill}%
\end{pgfscope}%
\begin{pgfscope}%
\pgfpathrectangle{\pgfqpoint{0.329460in}{0.284240in}}{\pgfqpoint{1.989680in}{1.989680in}}%
\pgfusepath{clip}%
\pgfsetbuttcap%
\pgfsetroundjoin%
\definecolor{currentfill}{rgb}{0.412913,0.803041,0.357269}%
\pgfsetfillcolor{currentfill}%
\pgfsetlinewidth{0.000000pt}%
\definecolor{currentstroke}{rgb}{0.000000,0.000000,0.000000}%
\pgfsetstrokecolor{currentstroke}%
\pgfsetdash{}{0pt}%
\pgfpathmoveto{\pgfqpoint{1.470353in}{1.643257in}}%
\pgfpathlineto{\pgfqpoint{1.472585in}{1.639024in}}%
\pgfpathlineto{\pgfqpoint{1.474814in}{1.634718in}}%
\pgfpathlineto{\pgfqpoint{1.477040in}{1.630343in}}%
\pgfpathlineto{\pgfqpoint{1.479264in}{1.625898in}}%
\pgfpathlineto{\pgfqpoint{1.483780in}{1.623946in}}%
\pgfpathlineto{\pgfqpoint{1.488169in}{1.621927in}}%
\pgfpathlineto{\pgfqpoint{1.492426in}{1.619842in}}%
\pgfpathlineto{\pgfqpoint{1.496547in}{1.617693in}}%
\pgfpathlineto{\pgfqpoint{1.494020in}{1.622287in}}%
\pgfpathlineto{\pgfqpoint{1.491490in}{1.626811in}}%
\pgfpathlineto{\pgfqpoint{1.488957in}{1.631265in}}%
\pgfpathlineto{\pgfqpoint{1.486421in}{1.635647in}}%
\pgfpathlineto{\pgfqpoint{1.482591in}{1.637639in}}%
\pgfpathlineto{\pgfqpoint{1.478634in}{1.639573in}}%
\pgfpathlineto{\pgfqpoint{1.474553in}{1.641447in}}%
\pgfpathlineto{\pgfqpoint{1.470353in}{1.643257in}}%
\pgfpathclose%
\pgfusepath{fill}%
\end{pgfscope}%
\begin{pgfscope}%
\pgfpathrectangle{\pgfqpoint{0.329460in}{0.284240in}}{\pgfqpoint{1.989680in}{1.989680in}}%
\pgfusepath{clip}%
\pgfsetbuttcap%
\pgfsetroundjoin%
\definecolor{currentfill}{rgb}{0.565498,0.842430,0.262877}%
\pgfsetfillcolor{currentfill}%
\pgfsetlinewidth{0.000000pt}%
\definecolor{currentstroke}{rgb}{0.000000,0.000000,0.000000}%
\pgfsetstrokecolor{currentstroke}%
\pgfsetdash{}{0pt}%
\pgfpathmoveto{\pgfqpoint{1.332347in}{1.691863in}}%
\pgfpathlineto{\pgfqpoint{1.331929in}{1.688665in}}%
\pgfpathlineto{\pgfqpoint{1.331511in}{1.685382in}}%
\pgfpathlineto{\pgfqpoint{1.331094in}{1.682015in}}%
\pgfpathlineto{\pgfqpoint{1.330678in}{1.678568in}}%
\pgfpathlineto{\pgfqpoint{1.335861in}{1.678833in}}%
\pgfpathlineto{\pgfqpoint{1.341058in}{1.679022in}}%
\pgfpathlineto{\pgfqpoint{1.346265in}{1.679134in}}%
\pgfpathlineto{\pgfqpoint{1.351477in}{1.679168in}}%
\pgfpathlineto{\pgfqpoint{1.351471in}{1.682604in}}%
\pgfpathlineto{\pgfqpoint{1.351465in}{1.685957in}}%
\pgfpathlineto{\pgfqpoint{1.351459in}{1.689228in}}%
\pgfpathlineto{\pgfqpoint{1.351454in}{1.692414in}}%
\pgfpathlineto{\pgfqpoint{1.346666in}{1.692383in}}%
\pgfpathlineto{\pgfqpoint{1.341882in}{1.692280in}}%
\pgfpathlineto{\pgfqpoint{1.337108in}{1.692107in}}%
\pgfpathlineto{\pgfqpoint{1.332347in}{1.691863in}}%
\pgfpathclose%
\pgfusepath{fill}%
\end{pgfscope}%
\begin{pgfscope}%
\pgfpathrectangle{\pgfqpoint{0.329460in}{0.284240in}}{\pgfqpoint{1.989680in}{1.989680in}}%
\pgfusepath{clip}%
\pgfsetbuttcap%
\pgfsetroundjoin%
\definecolor{currentfill}{rgb}{0.565498,0.842430,0.262877}%
\pgfsetfillcolor{currentfill}%
\pgfsetlinewidth{0.000000pt}%
\definecolor{currentstroke}{rgb}{0.000000,0.000000,0.000000}%
\pgfsetstrokecolor{currentstroke}%
\pgfsetdash{}{0pt}%
\pgfpathmoveto{\pgfqpoint{1.351454in}{1.692414in}}%
\pgfpathlineto{\pgfqpoint{1.351459in}{1.689228in}}%
\pgfpathlineto{\pgfqpoint{1.351465in}{1.685957in}}%
\pgfpathlineto{\pgfqpoint{1.351471in}{1.682604in}}%
\pgfpathlineto{\pgfqpoint{1.351477in}{1.679168in}}%
\pgfpathlineto{\pgfqpoint{1.356689in}{1.679125in}}%
\pgfpathlineto{\pgfqpoint{1.361895in}{1.679005in}}%
\pgfpathlineto{\pgfqpoint{1.367091in}{1.678808in}}%
\pgfpathlineto{\pgfqpoint{1.372272in}{1.678533in}}%
\pgfpathlineto{\pgfqpoint{1.371844in}{1.681982in}}%
\pgfpathlineto{\pgfqpoint{1.371415in}{1.685349in}}%
\pgfpathlineto{\pgfqpoint{1.370986in}{1.688633in}}%
\pgfpathlineto{\pgfqpoint{1.370556in}{1.691832in}}%
\pgfpathlineto{\pgfqpoint{1.365797in}{1.692084in}}%
\pgfpathlineto{\pgfqpoint{1.361024in}{1.692265in}}%
\pgfpathlineto{\pgfqpoint{1.356241in}{1.692375in}}%
\pgfpathlineto{\pgfqpoint{1.351454in}{1.692414in}}%
\pgfpathclose%
\pgfusepath{fill}%
\end{pgfscope}%
\begin{pgfscope}%
\pgfpathrectangle{\pgfqpoint{0.329460in}{0.284240in}}{\pgfqpoint{1.989680in}{1.989680in}}%
\pgfusepath{clip}%
\pgfsetbuttcap%
\pgfsetroundjoin%
\definecolor{currentfill}{rgb}{0.220124,0.725509,0.466226}%
\pgfsetfillcolor{currentfill}%
\pgfsetlinewidth{0.000000pt}%
\definecolor{currentstroke}{rgb}{0.000000,0.000000,0.000000}%
\pgfsetstrokecolor{currentstroke}%
\pgfsetdash{}{0pt}%
\pgfpathmoveto{\pgfqpoint{1.140882in}{1.541524in}}%
\pgfpathlineto{\pgfqpoint{1.137684in}{1.535797in}}%
\pgfpathlineto{\pgfqpoint{1.134490in}{1.530019in}}%
\pgfpathlineto{\pgfqpoint{1.131299in}{1.524193in}}%
\pgfpathlineto{\pgfqpoint{1.128112in}{1.518320in}}%
\pgfpathlineto{\pgfqpoint{1.130600in}{1.521701in}}%
\pgfpathlineto{\pgfqpoint{1.133304in}{1.525042in}}%
\pgfpathlineto{\pgfqpoint{1.136221in}{1.528337in}}%
\pgfpathlineto{\pgfqpoint{1.139347in}{1.531585in}}%
\pgfpathlineto{\pgfqpoint{1.142367in}{1.537256in}}%
\pgfpathlineto{\pgfqpoint{1.145391in}{1.542881in}}%
\pgfpathlineto{\pgfqpoint{1.148418in}{1.548457in}}%
\pgfpathlineto{\pgfqpoint{1.151448in}{1.553983in}}%
\pgfpathlineto{\pgfqpoint{1.148507in}{1.550932in}}%
\pgfpathlineto{\pgfqpoint{1.145764in}{1.547837in}}%
\pgfpathlineto{\pgfqpoint{1.143221in}{1.544700in}}%
\pgfpathlineto{\pgfqpoint{1.140882in}{1.541524in}}%
\pgfpathclose%
\pgfusepath{fill}%
\end{pgfscope}%
\begin{pgfscope}%
\pgfpathrectangle{\pgfqpoint{0.329460in}{0.284240in}}{\pgfqpoint{1.989680in}{1.989680in}}%
\pgfusepath{clip}%
\pgfsetbuttcap%
\pgfsetroundjoin%
\definecolor{currentfill}{rgb}{0.163625,0.471133,0.558148}%
\pgfsetfillcolor{currentfill}%
\pgfsetlinewidth{0.000000pt}%
\definecolor{currentstroke}{rgb}{0.000000,0.000000,0.000000}%
\pgfsetstrokecolor{currentstroke}%
\pgfsetdash{}{0pt}%
\pgfpathmoveto{\pgfqpoint{1.039716in}{1.290636in}}%
\pgfpathlineto{\pgfqpoint{1.036430in}{1.283415in}}%
\pgfpathlineto{\pgfqpoint{1.033146in}{1.276207in}}%
\pgfpathlineto{\pgfqpoint{1.029864in}{1.269015in}}%
\pgfpathlineto{\pgfqpoint{1.026583in}{1.261841in}}%
\pgfpathlineto{\pgfqpoint{1.024985in}{1.267012in}}%
\pgfpathlineto{\pgfqpoint{1.023715in}{1.272203in}}%
\pgfpathlineto{\pgfqpoint{1.022775in}{1.277406in}}%
\pgfpathlineto{\pgfqpoint{1.022165in}{1.282618in}}%
\pgfpathlineto{\pgfqpoint{1.025479in}{1.289559in}}%
\pgfpathlineto{\pgfqpoint{1.028795in}{1.296519in}}%
\pgfpathlineto{\pgfqpoint{1.032114in}{1.303495in}}%
\pgfpathlineto{\pgfqpoint{1.035434in}{1.310484in}}%
\pgfpathlineto{\pgfqpoint{1.036031in}{1.305505in}}%
\pgfpathlineto{\pgfqpoint{1.036943in}{1.300534in}}%
\pgfpathlineto{\pgfqpoint{1.038172in}{1.295576in}}%
\pgfpathlineto{\pgfqpoint{1.039716in}{1.290636in}}%
\pgfpathclose%
\pgfusepath{fill}%
\end{pgfscope}%
\begin{pgfscope}%
\pgfpathrectangle{\pgfqpoint{0.329460in}{0.284240in}}{\pgfqpoint{1.989680in}{1.989680in}}%
\pgfusepath{clip}%
\pgfsetbuttcap%
\pgfsetroundjoin%
\definecolor{currentfill}{rgb}{0.166383,0.690856,0.496502}%
\pgfsetfillcolor{currentfill}%
\pgfsetlinewidth{0.000000pt}%
\definecolor{currentstroke}{rgb}{0.000000,0.000000,0.000000}%
\pgfsetstrokecolor{currentstroke}%
\pgfsetdash{}{0pt}%
\pgfpathmoveto{\pgfqpoint{1.572062in}{1.521328in}}%
\pgfpathlineto{\pgfqpoint{1.575213in}{1.515456in}}%
\pgfpathlineto{\pgfqpoint{1.578361in}{1.509542in}}%
\pgfpathlineto{\pgfqpoint{1.581506in}{1.503587in}}%
\pgfpathlineto{\pgfqpoint{1.584647in}{1.497594in}}%
\pgfpathlineto{\pgfqpoint{1.587259in}{1.494001in}}%
\pgfpathlineto{\pgfqpoint{1.589641in}{1.490367in}}%
\pgfpathlineto{\pgfqpoint{1.591788in}{1.486697in}}%
\pgfpathlineto{\pgfqpoint{1.593700in}{1.482993in}}%
\pgfpathlineto{\pgfqpoint{1.590429in}{1.489197in}}%
\pgfpathlineto{\pgfqpoint{1.587154in}{1.495363in}}%
\pgfpathlineto{\pgfqpoint{1.583877in}{1.501488in}}%
\pgfpathlineto{\pgfqpoint{1.580597in}{1.507570in}}%
\pgfpathlineto{\pgfqpoint{1.578796in}{1.511059in}}%
\pgfpathlineto{\pgfqpoint{1.576772in}{1.514518in}}%
\pgfpathlineto{\pgfqpoint{1.574526in}{1.517942in}}%
\pgfpathlineto{\pgfqpoint{1.572062in}{1.521328in}}%
\pgfpathclose%
\pgfusepath{fill}%
\end{pgfscope}%
\begin{pgfscope}%
\pgfpathrectangle{\pgfqpoint{0.329460in}{0.284240in}}{\pgfqpoint{1.989680in}{1.989680in}}%
\pgfusepath{clip}%
\pgfsetbuttcap%
\pgfsetroundjoin%
\definecolor{currentfill}{rgb}{0.565498,0.842430,0.262877}%
\pgfsetfillcolor{currentfill}%
\pgfsetlinewidth{0.000000pt}%
\definecolor{currentstroke}{rgb}{0.000000,0.000000,0.000000}%
\pgfsetstrokecolor{currentstroke}%
\pgfsetdash{}{0pt}%
\pgfpathmoveto{\pgfqpoint{1.313530in}{1.690188in}}%
\pgfpathlineto{\pgfqpoint{1.312694in}{1.686951in}}%
\pgfpathlineto{\pgfqpoint{1.311859in}{1.683630in}}%
\pgfpathlineto{\pgfqpoint{1.311025in}{1.680225in}}%
\pgfpathlineto{\pgfqpoint{1.310193in}{1.676740in}}%
\pgfpathlineto{\pgfqpoint{1.315267in}{1.677310in}}%
\pgfpathlineto{\pgfqpoint{1.320376in}{1.677806in}}%
\pgfpathlineto{\pgfqpoint{1.325515in}{1.678225in}}%
\pgfpathlineto{\pgfqpoint{1.330678in}{1.678568in}}%
\pgfpathlineto{\pgfqpoint{1.331094in}{1.682015in}}%
\pgfpathlineto{\pgfqpoint{1.331511in}{1.685382in}}%
\pgfpathlineto{\pgfqpoint{1.331929in}{1.688665in}}%
\pgfpathlineto{\pgfqpoint{1.332347in}{1.691863in}}%
\pgfpathlineto{\pgfqpoint{1.327604in}{1.691549in}}%
\pgfpathlineto{\pgfqpoint{1.322884in}{1.691165in}}%
\pgfpathlineto{\pgfqpoint{1.318191in}{1.690711in}}%
\pgfpathlineto{\pgfqpoint{1.313530in}{1.690188in}}%
\pgfpathclose%
\pgfusepath{fill}%
\end{pgfscope}%
\begin{pgfscope}%
\pgfpathrectangle{\pgfqpoint{0.329460in}{0.284240in}}{\pgfqpoint{1.989680in}{1.989680in}}%
\pgfusepath{clip}%
\pgfsetbuttcap%
\pgfsetroundjoin%
\definecolor{currentfill}{rgb}{0.565498,0.842430,0.262877}%
\pgfsetfillcolor{currentfill}%
\pgfsetlinewidth{0.000000pt}%
\definecolor{currentstroke}{rgb}{0.000000,0.000000,0.000000}%
\pgfsetstrokecolor{currentstroke}%
\pgfsetdash{}{0pt}%
\pgfpathmoveto{\pgfqpoint{1.370556in}{1.691832in}}%
\pgfpathlineto{\pgfqpoint{1.370986in}{1.688633in}}%
\pgfpathlineto{\pgfqpoint{1.371415in}{1.685349in}}%
\pgfpathlineto{\pgfqpoint{1.371844in}{1.681982in}}%
\pgfpathlineto{\pgfqpoint{1.372272in}{1.678533in}}%
\pgfpathlineto{\pgfqpoint{1.377432in}{1.678182in}}%
\pgfpathlineto{\pgfqpoint{1.382568in}{1.677754in}}%
\pgfpathlineto{\pgfqpoint{1.387673in}{1.677251in}}%
\pgfpathlineto{\pgfqpoint{1.392744in}{1.676671in}}%
\pgfpathlineto{\pgfqpoint{1.391900in}{1.680159in}}%
\pgfpathlineto{\pgfqpoint{1.391055in}{1.683565in}}%
\pgfpathlineto{\pgfqpoint{1.390209in}{1.686887in}}%
\pgfpathlineto{\pgfqpoint{1.389361in}{1.690126in}}%
\pgfpathlineto{\pgfqpoint{1.384704in}{1.690657in}}%
\pgfpathlineto{\pgfqpoint{1.380014in}{1.691118in}}%
\pgfpathlineto{\pgfqpoint{1.375297in}{1.691510in}}%
\pgfpathlineto{\pgfqpoint{1.370556in}{1.691832in}}%
\pgfpathclose%
\pgfusepath{fill}%
\end{pgfscope}%
\begin{pgfscope}%
\pgfpathrectangle{\pgfqpoint{0.329460in}{0.284240in}}{\pgfqpoint{1.989680in}{1.989680in}}%
\pgfusepath{clip}%
\pgfsetbuttcap%
\pgfsetroundjoin%
\definecolor{currentfill}{rgb}{0.195860,0.395433,0.555276}%
\pgfsetfillcolor{currentfill}%
\pgfsetlinewidth{0.000000pt}%
\definecolor{currentstroke}{rgb}{0.000000,0.000000,0.000000}%
\pgfsetstrokecolor{currentstroke}%
\pgfsetdash{}{0pt}%
\pgfpathmoveto{\pgfqpoint{1.690382in}{1.238185in}}%
\pgfpathlineto{\pgfqpoint{1.693666in}{1.231196in}}%
\pgfpathlineto{\pgfqpoint{1.696948in}{1.224241in}}%
\pgfpathlineto{\pgfqpoint{1.700230in}{1.217324in}}%
\pgfpathlineto{\pgfqpoint{1.703509in}{1.210447in}}%
\pgfpathlineto{\pgfqpoint{1.701766in}{1.204810in}}%
\pgfpathlineto{\pgfqpoint{1.699668in}{1.199199in}}%
\pgfpathlineto{\pgfqpoint{1.697216in}{1.193620in}}%
\pgfpathlineto{\pgfqpoint{1.694412in}{1.188078in}}%
\pgfpathlineto{\pgfqpoint{1.691206in}{1.195188in}}%
\pgfpathlineto{\pgfqpoint{1.687998in}{1.202338in}}%
\pgfpathlineto{\pgfqpoint{1.684790in}{1.209526in}}%
\pgfpathlineto{\pgfqpoint{1.681580in}{1.216748in}}%
\pgfpathlineto{\pgfqpoint{1.684290in}{1.222058in}}%
\pgfpathlineto{\pgfqpoint{1.686661in}{1.227405in}}%
\pgfpathlineto{\pgfqpoint{1.688692in}{1.232783in}}%
\pgfpathlineto{\pgfqpoint{1.690382in}{1.238185in}}%
\pgfpathclose%
\pgfusepath{fill}%
\end{pgfscope}%
\begin{pgfscope}%
\pgfpathrectangle{\pgfqpoint{0.329460in}{0.284240in}}{\pgfqpoint{1.989680in}{1.989680in}}%
\pgfusepath{clip}%
\pgfsetbuttcap%
\pgfsetroundjoin%
\definecolor{currentfill}{rgb}{0.412913,0.803041,0.357269}%
\pgfsetfillcolor{currentfill}%
\pgfsetlinewidth{0.000000pt}%
\definecolor{currentstroke}{rgb}{0.000000,0.000000,0.000000}%
\pgfsetstrokecolor{currentstroke}%
\pgfsetdash{}{0pt}%
\pgfpathmoveto{\pgfqpoint{1.212659in}{1.633827in}}%
\pgfpathlineto{\pgfqpoint{1.210060in}{1.629410in}}%
\pgfpathlineto{\pgfqpoint{1.207465in}{1.624921in}}%
\pgfpathlineto{\pgfqpoint{1.204873in}{1.620361in}}%
\pgfpathlineto{\pgfqpoint{1.202283in}{1.615732in}}%
\pgfpathlineto{\pgfqpoint{1.206279in}{1.617935in}}%
\pgfpathlineto{\pgfqpoint{1.210415in}{1.620077in}}%
\pgfpathlineto{\pgfqpoint{1.214687in}{1.622154in}}%
\pgfpathlineto{\pgfqpoint{1.219090in}{1.624166in}}%
\pgfpathlineto{\pgfqpoint{1.221384in}{1.628642in}}%
\pgfpathlineto{\pgfqpoint{1.223681in}{1.633049in}}%
\pgfpathlineto{\pgfqpoint{1.225981in}{1.637386in}}%
\pgfpathlineto{\pgfqpoint{1.228283in}{1.641651in}}%
\pgfpathlineto{\pgfqpoint{1.224189in}{1.639785in}}%
\pgfpathlineto{\pgfqpoint{1.220218in}{1.637857in}}%
\pgfpathlineto{\pgfqpoint{1.216373in}{1.635871in}}%
\pgfpathlineto{\pgfqpoint{1.212659in}{1.633827in}}%
\pgfpathclose%
\pgfusepath{fill}%
\end{pgfscope}%
\begin{pgfscope}%
\pgfpathrectangle{\pgfqpoint{0.329460in}{0.284240in}}{\pgfqpoint{1.989680in}{1.989680in}}%
\pgfusepath{clip}%
\pgfsetbuttcap%
\pgfsetroundjoin%
\definecolor{currentfill}{rgb}{0.201239,0.383670,0.554294}%
\pgfsetfillcolor{currentfill}%
\pgfsetlinewidth{0.000000pt}%
\definecolor{currentstroke}{rgb}{0.000000,0.000000,0.000000}%
\pgfsetstrokecolor{currentstroke}%
\pgfsetdash{}{0pt}%
\pgfpathmoveto{\pgfqpoint{0.701212in}{1.155017in}}%
\pgfpathlineto{\pgfqpoint{0.697376in}{1.166423in}}%
\pgfpathlineto{\pgfqpoint{0.693520in}{1.178260in}}%
\pgfpathlineto{\pgfqpoint{0.689643in}{1.190536in}}%
\pgfpathlineto{\pgfqpoint{0.685745in}{1.203258in}}%
\pgfpathlineto{\pgfqpoint{0.683049in}{1.213950in}}%
\pgfpathlineto{\pgfqpoint{0.681032in}{1.224663in}}%
\pgfpathlineto{\pgfqpoint{0.679694in}{1.235383in}}%
\pgfpathlineto{\pgfqpoint{0.679033in}{1.246102in}}%
\pgfpathlineto{\pgfqpoint{0.682943in}{1.233216in}}%
\pgfpathlineto{\pgfqpoint{0.686833in}{1.220774in}}%
\pgfpathlineto{\pgfqpoint{0.690702in}{1.208768in}}%
\pgfpathlineto{\pgfqpoint{0.694552in}{1.197191in}}%
\pgfpathlineto{\pgfqpoint{0.695224in}{1.186639in}}%
\pgfpathlineto{\pgfqpoint{0.696557in}{1.176086in}}%
\pgfpathlineto{\pgfqpoint{0.698552in}{1.165541in}}%
\pgfpathlineto{\pgfqpoint{0.701212in}{1.155017in}}%
\pgfpathclose%
\pgfusepath{fill}%
\end{pgfscope}%
\begin{pgfscope}%
\pgfpathrectangle{\pgfqpoint{0.329460in}{0.284240in}}{\pgfqpoint{1.989680in}{1.989680in}}%
\pgfusepath{clip}%
\pgfsetbuttcap%
\pgfsetroundjoin%
\definecolor{currentfill}{rgb}{0.260571,0.246922,0.522828}%
\pgfsetfillcolor{currentfill}%
\pgfsetlinewidth{0.000000pt}%
\definecolor{currentstroke}{rgb}{0.000000,0.000000,0.000000}%
\pgfsetstrokecolor{currentstroke}%
\pgfsetdash{}{0pt}%
\pgfpathmoveto{\pgfqpoint{1.973469in}{1.087499in}}%
\pgfpathlineto{\pgfqpoint{1.977170in}{1.095743in}}%
\pgfpathlineto{\pgfqpoint{1.980887in}{1.104364in}}%
\pgfpathlineto{\pgfqpoint{1.984621in}{1.113368in}}%
\pgfpathlineto{\pgfqpoint{1.988373in}{1.122762in}}%
\pgfpathlineto{\pgfqpoint{1.985679in}{1.112419in}}%
\pgfpathlineto{\pgfqpoint{1.982336in}{1.102108in}}%
\pgfpathlineto{\pgfqpoint{1.978345in}{1.091841in}}%
\pgfpathlineto{\pgfqpoint{1.973706in}{1.081627in}}%
\pgfpathlineto{\pgfqpoint{1.970017in}{1.072417in}}%
\pgfpathlineto{\pgfqpoint{1.966346in}{1.063598in}}%
\pgfpathlineto{\pgfqpoint{1.962692in}{1.055165in}}%
\pgfpathlineto{\pgfqpoint{1.959053in}{1.047111in}}%
\pgfpathlineto{\pgfqpoint{1.963606in}{1.057137in}}%
\pgfpathlineto{\pgfqpoint{1.967526in}{1.067218in}}%
\pgfpathlineto{\pgfqpoint{1.970814in}{1.077342in}}%
\pgfpathlineto{\pgfqpoint{1.973469in}{1.087499in}}%
\pgfpathclose%
\pgfusepath{fill}%
\end{pgfscope}%
\begin{pgfscope}%
\pgfpathrectangle{\pgfqpoint{0.329460in}{0.284240in}}{\pgfqpoint{1.989680in}{1.989680in}}%
\pgfusepath{clip}%
\pgfsetbuttcap%
\pgfsetroundjoin%
\definecolor{currentfill}{rgb}{0.565498,0.842430,0.262877}%
\pgfsetfillcolor{currentfill}%
\pgfsetlinewidth{0.000000pt}%
\definecolor{currentstroke}{rgb}{0.000000,0.000000,0.000000}%
\pgfsetstrokecolor{currentstroke}%
\pgfsetdash{}{0pt}%
\pgfpathmoveto{\pgfqpoint{1.389361in}{1.690126in}}%
\pgfpathlineto{\pgfqpoint{1.390209in}{1.686887in}}%
\pgfpathlineto{\pgfqpoint{1.391055in}{1.683565in}}%
\pgfpathlineto{\pgfqpoint{1.391900in}{1.680159in}}%
\pgfpathlineto{\pgfqpoint{1.392744in}{1.676671in}}%
\pgfpathlineto{\pgfqpoint{1.397774in}{1.676017in}}%
\pgfpathlineto{\pgfqpoint{1.402760in}{1.675288in}}%
\pgfpathlineto{\pgfqpoint{1.407697in}{1.674486in}}%
\pgfpathlineto{\pgfqpoint{1.406549in}{1.678019in}}%
\pgfpathlineto{\pgfqpoint{1.405399in}{1.681470in}}%
\pgfpathlineto{\pgfqpoint{1.404248in}{1.684839in}}%
\pgfpathlineto{\pgfqpoint{1.403096in}{1.688123in}}%
\pgfpathlineto{\pgfqpoint{1.398562in}{1.688858in}}%
\pgfpathlineto{\pgfqpoint{1.393982in}{1.689526in}}%
\pgfpathlineto{\pgfqpoint{1.389361in}{1.690126in}}%
\pgfpathclose%
\pgfusepath{fill}%
\end{pgfscope}%
\begin{pgfscope}%
\pgfpathrectangle{\pgfqpoint{0.329460in}{0.284240in}}{\pgfqpoint{1.989680in}{1.989680in}}%
\pgfusepath{clip}%
\pgfsetbuttcap%
\pgfsetroundjoin%
\definecolor{currentfill}{rgb}{0.122606,0.585371,0.546557}%
\pgfsetfillcolor{currentfill}%
\pgfsetlinewidth{0.000000pt}%
\definecolor{currentstroke}{rgb}{0.000000,0.000000,0.000000}%
\pgfsetstrokecolor{currentstroke}%
\pgfsetdash{}{0pt}%
\pgfpathmoveto{\pgfqpoint{1.625630in}{1.415394in}}%
\pgfpathlineto{\pgfqpoint{1.628953in}{1.408721in}}%
\pgfpathlineto{\pgfqpoint{1.632273in}{1.402032in}}%
\pgfpathlineto{\pgfqpoint{1.635590in}{1.395328in}}%
\pgfpathlineto{\pgfqpoint{1.638905in}{1.388613in}}%
\pgfpathlineto{\pgfqpoint{1.639747in}{1.384123in}}%
\pgfpathlineto{\pgfqpoint{1.640303in}{1.379618in}}%
\pgfpathlineto{\pgfqpoint{1.640572in}{1.375104in}}%
\pgfpathlineto{\pgfqpoint{1.640553in}{1.370585in}}%
\pgfpathlineto{\pgfqpoint{1.637210in}{1.377527in}}%
\pgfpathlineto{\pgfqpoint{1.633864in}{1.384458in}}%
\pgfpathlineto{\pgfqpoint{1.630516in}{1.391375in}}%
\pgfpathlineto{\pgfqpoint{1.627165in}{1.398275in}}%
\pgfpathlineto{\pgfqpoint{1.627193in}{1.402566in}}%
\pgfpathlineto{\pgfqpoint{1.626946in}{1.406853in}}%
\pgfpathlineto{\pgfqpoint{1.626425in}{1.411130in}}%
\pgfpathlineto{\pgfqpoint{1.625630in}{1.415394in}}%
\pgfpathclose%
\pgfusepath{fill}%
\end{pgfscope}%
\begin{pgfscope}%
\pgfpathrectangle{\pgfqpoint{0.329460in}{0.284240in}}{\pgfqpoint{1.989680in}{1.989680in}}%
\pgfusepath{clip}%
\pgfsetbuttcap%
\pgfsetroundjoin%
\definecolor{currentfill}{rgb}{0.565498,0.842430,0.262877}%
\pgfsetfillcolor{currentfill}%
\pgfsetlinewidth{0.000000pt}%
\definecolor{currentstroke}{rgb}{0.000000,0.000000,0.000000}%
\pgfsetstrokecolor{currentstroke}%
\pgfsetdash{}{0pt}%
\pgfpathmoveto{\pgfqpoint{1.295291in}{1.687413in}}%
\pgfpathlineto{\pgfqpoint{1.294050in}{1.684113in}}%
\pgfpathlineto{\pgfqpoint{1.292810in}{1.680728in}}%
\pgfpathlineto{\pgfqpoint{1.291572in}{1.677260in}}%
\pgfpathlineto{\pgfqpoint{1.290335in}{1.673711in}}%
\pgfpathlineto{\pgfqpoint{1.295224in}{1.674579in}}%
\pgfpathlineto{\pgfqpoint{1.300166in}{1.675373in}}%
\pgfpathlineto{\pgfqpoint{1.305158in}{1.676094in}}%
\pgfpathlineto{\pgfqpoint{1.310193in}{1.676740in}}%
\pgfpathlineto{\pgfqpoint{1.311025in}{1.680225in}}%
\pgfpathlineto{\pgfqpoint{1.311859in}{1.683630in}}%
\pgfpathlineto{\pgfqpoint{1.312694in}{1.686951in}}%
\pgfpathlineto{\pgfqpoint{1.313530in}{1.690188in}}%
\pgfpathlineto{\pgfqpoint{1.308905in}{1.689596in}}%
\pgfpathlineto{\pgfqpoint{1.304320in}{1.688936in}}%
\pgfpathlineto{\pgfqpoint{1.299781in}{1.688208in}}%
\pgfpathlineto{\pgfqpoint{1.295291in}{1.687413in}}%
\pgfpathclose%
\pgfusepath{fill}%
\end{pgfscope}%
\begin{pgfscope}%
\pgfpathrectangle{\pgfqpoint{0.329460in}{0.284240in}}{\pgfqpoint{1.989680in}{1.989680in}}%
\pgfusepath{clip}%
\pgfsetbuttcap%
\pgfsetroundjoin%
\definecolor{currentfill}{rgb}{0.263663,0.237631,0.518762}%
\pgfsetfillcolor{currentfill}%
\pgfsetlinewidth{0.000000pt}%
\definecolor{currentstroke}{rgb}{0.000000,0.000000,0.000000}%
\pgfsetstrokecolor{currentstroke}%
\pgfsetdash{}{0pt}%
\pgfpathmoveto{\pgfqpoint{0.990169in}{1.077257in}}%
\pgfpathlineto{\pgfqpoint{0.987128in}{1.070659in}}%
\pgfpathlineto{\pgfqpoint{0.984087in}{1.064144in}}%
\pgfpathlineto{\pgfqpoint{0.981045in}{1.057714in}}%
\pgfpathlineto{\pgfqpoint{0.978003in}{1.051372in}}%
\pgfpathlineto{\pgfqpoint{0.972886in}{1.057545in}}%
\pgfpathlineto{\pgfqpoint{0.968161in}{1.063792in}}%
\pgfpathlineto{\pgfqpoint{0.963832in}{1.070108in}}%
\pgfpathlineto{\pgfqpoint{0.959901in}{1.076485in}}%
\pgfpathlineto{\pgfqpoint{0.963078in}{1.082598in}}%
\pgfpathlineto{\pgfqpoint{0.966255in}{1.088800in}}%
\pgfpathlineto{\pgfqpoint{0.969432in}{1.095087in}}%
\pgfpathlineto{\pgfqpoint{0.972609in}{1.101457in}}%
\pgfpathlineto{\pgfqpoint{0.976424in}{1.095311in}}%
\pgfpathlineto{\pgfqpoint{0.980625in}{1.089225in}}%
\pgfpathlineto{\pgfqpoint{0.985208in}{1.083205in}}%
\pgfpathlineto{\pgfqpoint{0.990169in}{1.077257in}}%
\pgfpathclose%
\pgfusepath{fill}%
\end{pgfscope}%
\begin{pgfscope}%
\pgfpathrectangle{\pgfqpoint{0.329460in}{0.284240in}}{\pgfqpoint{1.989680in}{1.989680in}}%
\pgfusepath{clip}%
\pgfsetbuttcap%
\pgfsetroundjoin%
\definecolor{currentfill}{rgb}{0.166383,0.690856,0.496502}%
\pgfsetfillcolor{currentfill}%
\pgfsetlinewidth{0.000000pt}%
\definecolor{currentstroke}{rgb}{0.000000,0.000000,0.000000}%
\pgfsetstrokecolor{currentstroke}%
\pgfsetdash{}{0pt}%
\pgfpathmoveto{\pgfqpoint{1.120368in}{1.504444in}}%
\pgfpathlineto{\pgfqpoint{1.117065in}{1.498314in}}%
\pgfpathlineto{\pgfqpoint{1.113766in}{1.492141in}}%
\pgfpathlineto{\pgfqpoint{1.110470in}{1.485928in}}%
\pgfpathlineto{\pgfqpoint{1.107177in}{1.479675in}}%
\pgfpathlineto{\pgfqpoint{1.108876in}{1.483406in}}%
\pgfpathlineto{\pgfqpoint{1.110814in}{1.487106in}}%
\pgfpathlineto{\pgfqpoint{1.112988in}{1.490773in}}%
\pgfpathlineto{\pgfqpoint{1.115395in}{1.494402in}}%
\pgfpathlineto{\pgfqpoint{1.118569in}{1.500441in}}%
\pgfpathlineto{\pgfqpoint{1.121747in}{1.506442in}}%
\pgfpathlineto{\pgfqpoint{1.124928in}{1.512402in}}%
\pgfpathlineto{\pgfqpoint{1.128112in}{1.518320in}}%
\pgfpathlineto{\pgfqpoint{1.125842in}{1.514900in}}%
\pgfpathlineto{\pgfqpoint{1.123793in}{1.511445in}}%
\pgfpathlineto{\pgfqpoint{1.121968in}{1.507959in}}%
\pgfpathlineto{\pgfqpoint{1.120368in}{1.504444in}}%
\pgfpathclose%
\pgfusepath{fill}%
\end{pgfscope}%
\begin{pgfscope}%
\pgfpathrectangle{\pgfqpoint{0.329460in}{0.284240in}}{\pgfqpoint{1.989680in}{1.989680in}}%
\pgfusepath{clip}%
\pgfsetbuttcap%
\pgfsetroundjoin%
\definecolor{currentfill}{rgb}{0.248629,0.278775,0.534556}%
\pgfsetfillcolor{currentfill}%
\pgfsetlinewidth{0.000000pt}%
\definecolor{currentstroke}{rgb}{0.000000,0.000000,0.000000}%
\pgfsetstrokecolor{currentstroke}%
\pgfsetdash{}{0pt}%
\pgfpathmoveto{\pgfqpoint{1.720033in}{1.132997in}}%
\pgfpathlineto{\pgfqpoint{1.723233in}{1.126381in}}%
\pgfpathlineto{\pgfqpoint{1.726432in}{1.119835in}}%
\pgfpathlineto{\pgfqpoint{1.729632in}{1.113362in}}%
\pgfpathlineto{\pgfqpoint{1.732832in}{1.106965in}}%
\pgfpathlineto{\pgfqpoint{1.729361in}{1.100771in}}%
\pgfpathlineto{\pgfqpoint{1.725503in}{1.094632in}}%
\pgfpathlineto{\pgfqpoint{1.721260in}{1.088553in}}%
\pgfpathlineto{\pgfqpoint{1.716635in}{1.082540in}}%
\pgfpathlineto{\pgfqpoint{1.713559in}{1.089167in}}%
\pgfpathlineto{\pgfqpoint{1.710484in}{1.095870in}}%
\pgfpathlineto{\pgfqpoint{1.707409in}{1.102646in}}%
\pgfpathlineto{\pgfqpoint{1.704334in}{1.109491in}}%
\pgfpathlineto{\pgfqpoint{1.708814in}{1.115277in}}%
\pgfpathlineto{\pgfqpoint{1.712926in}{1.121127in}}%
\pgfpathlineto{\pgfqpoint{1.716667in}{1.127036in}}%
\pgfpathlineto{\pgfqpoint{1.720033in}{1.132997in}}%
\pgfpathclose%
\pgfusepath{fill}%
\end{pgfscope}%
\begin{pgfscope}%
\pgfpathrectangle{\pgfqpoint{0.329460in}{0.284240in}}{\pgfqpoint{1.989680in}{1.989680in}}%
\pgfusepath{clip}%
\pgfsetbuttcap%
\pgfsetroundjoin%
\definecolor{currentfill}{rgb}{0.147607,0.511733,0.557049}%
\pgfsetfillcolor{currentfill}%
\pgfsetlinewidth{0.000000pt}%
\definecolor{currentstroke}{rgb}{0.000000,0.000000,0.000000}%
\pgfsetstrokecolor{currentstroke}%
\pgfsetdash{}{0pt}%
\pgfpathmoveto{\pgfqpoint{1.653899in}{1.342744in}}%
\pgfpathlineto{\pgfqpoint{1.657229in}{1.335779in}}%
\pgfpathlineto{\pgfqpoint{1.660557in}{1.328817in}}%
\pgfpathlineto{\pgfqpoint{1.663883in}{1.321860in}}%
\pgfpathlineto{\pgfqpoint{1.667206in}{1.314912in}}%
\pgfpathlineto{\pgfqpoint{1.666890in}{1.309930in}}%
\pgfpathlineto{\pgfqpoint{1.666259in}{1.304952in}}%
\pgfpathlineto{\pgfqpoint{1.665311in}{1.299982in}}%
\pgfpathlineto{\pgfqpoint{1.664047in}{1.295026in}}%
\pgfpathlineto{\pgfqpoint{1.660747in}{1.302206in}}%
\pgfpathlineto{\pgfqpoint{1.657444in}{1.309394in}}%
\pgfpathlineto{\pgfqpoint{1.654140in}{1.316588in}}%
\pgfpathlineto{\pgfqpoint{1.650833in}{1.323784in}}%
\pgfpathlineto{\pgfqpoint{1.652053in}{1.328509in}}%
\pgfpathlineto{\pgfqpoint{1.652971in}{1.333247in}}%
\pgfpathlineto{\pgfqpoint{1.653586in}{1.337994in}}%
\pgfpathlineto{\pgfqpoint{1.653899in}{1.342744in}}%
\pgfpathclose%
\pgfusepath{fill}%
\end{pgfscope}%
\begin{pgfscope}%
\pgfpathrectangle{\pgfqpoint{0.329460in}{0.284240in}}{\pgfqpoint{1.989680in}{1.989680in}}%
\pgfusepath{clip}%
\pgfsetbuttcap%
\pgfsetroundjoin%
\definecolor{currentfill}{rgb}{0.487026,0.823929,0.312321}%
\pgfsetfillcolor{currentfill}%
\pgfsetlinewidth{0.000000pt}%
\definecolor{currentstroke}{rgb}{0.000000,0.000000,0.000000}%
\pgfsetstrokecolor{currentstroke}%
\pgfsetdash{}{0pt}%
\pgfpathmoveto{\pgfqpoint{1.444849in}{1.665513in}}%
\pgfpathlineto{\pgfqpoint{1.446752in}{1.661712in}}%
\pgfpathlineto{\pgfqpoint{1.448653in}{1.657832in}}%
\pgfpathlineto{\pgfqpoint{1.450552in}{1.653876in}}%
\pgfpathlineto{\pgfqpoint{1.452449in}{1.649844in}}%
\pgfpathlineto{\pgfqpoint{1.457082in}{1.648299in}}%
\pgfpathlineto{\pgfqpoint{1.461614in}{1.646685in}}%
\pgfpathlineto{\pgfqpoint{1.466039in}{1.645004in}}%
\pgfpathlineto{\pgfqpoint{1.470353in}{1.643257in}}%
\pgfpathlineto{\pgfqpoint{1.468119in}{1.647417in}}%
\pgfpathlineto{\pgfqpoint{1.465882in}{1.651502in}}%
\pgfpathlineto{\pgfqpoint{1.463642in}{1.655509in}}%
\pgfpathlineto{\pgfqpoint{1.461400in}{1.659438in}}%
\pgfpathlineto{\pgfqpoint{1.457412in}{1.661049in}}%
\pgfpathlineto{\pgfqpoint{1.453322in}{1.662599in}}%
\pgfpathlineto{\pgfqpoint{1.449132in}{1.664088in}}%
\pgfpathlineto{\pgfqpoint{1.444849in}{1.665513in}}%
\pgfpathclose%
\pgfusepath{fill}%
\end{pgfscope}%
\begin{pgfscope}%
\pgfpathrectangle{\pgfqpoint{0.329460in}{0.284240in}}{\pgfqpoint{1.989680in}{1.989680in}}%
\pgfusepath{clip}%
\pgfsetbuttcap%
\pgfsetroundjoin%
\definecolor{currentfill}{rgb}{0.277941,0.056324,0.381191}%
\pgfsetfillcolor{currentfill}%
\pgfsetlinewidth{0.000000pt}%
\definecolor{currentstroke}{rgb}{0.000000,0.000000,0.000000}%
\pgfsetstrokecolor{currentstroke}%
\pgfsetdash{}{0pt}%
\pgfpathmoveto{\pgfqpoint{0.828353in}{0.920935in}}%
\pgfpathlineto{\pgfqpoint{0.825082in}{0.923468in}}%
\pgfpathlineto{\pgfqpoint{0.821800in}{0.926290in}}%
\pgfpathlineto{\pgfqpoint{0.818508in}{0.929407in}}%
\pgfpathlineto{\pgfqpoint{0.815205in}{0.932824in}}%
\pgfpathlineto{\pgfqpoint{0.808082in}{0.941848in}}%
\pgfpathlineto{\pgfqpoint{0.801530in}{0.950974in}}%
\pgfpathlineto{\pgfqpoint{0.795554in}{0.960192in}}%
\pgfpathlineto{\pgfqpoint{0.790156in}{0.969491in}}%
\pgfpathlineto{\pgfqpoint{0.793595in}{0.965868in}}%
\pgfpathlineto{\pgfqpoint{0.797022in}{0.962544in}}%
\pgfpathlineto{\pgfqpoint{0.800439in}{0.959514in}}%
\pgfpathlineto{\pgfqpoint{0.803845in}{0.956772in}}%
\pgfpathlineto{\pgfqpoint{0.809129in}{0.947682in}}%
\pgfpathlineto{\pgfqpoint{0.814978in}{0.938673in}}%
\pgfpathlineto{\pgfqpoint{0.821388in}{0.929754in}}%
\pgfpathlineto{\pgfqpoint{0.828353in}{0.920935in}}%
\pgfpathclose%
\pgfusepath{fill}%
\end{pgfscope}%
\begin{pgfscope}%
\pgfpathrectangle{\pgfqpoint{0.329460in}{0.284240in}}{\pgfqpoint{1.989680in}{1.989680in}}%
\pgfusepath{clip}%
\pgfsetbuttcap%
\pgfsetroundjoin%
\definecolor{currentfill}{rgb}{0.282327,0.094955,0.417331}%
\pgfsetfillcolor{currentfill}%
\pgfsetlinewidth{0.000000pt}%
\definecolor{currentstroke}{rgb}{0.000000,0.000000,0.000000}%
\pgfsetstrokecolor{currentstroke}%
\pgfsetdash{}{0pt}%
\pgfpathmoveto{\pgfqpoint{1.916528in}{0.977818in}}%
\pgfpathlineto{\pgfqpoint{1.920000in}{0.981790in}}%
\pgfpathlineto{\pgfqpoint{1.923484in}{0.986072in}}%
\pgfpathlineto{\pgfqpoint{1.926981in}{0.990668in}}%
\pgfpathlineto{\pgfqpoint{1.930489in}{0.995584in}}%
\pgfpathlineto{\pgfqpoint{1.925509in}{0.986015in}}%
\pgfpathlineto{\pgfqpoint{1.919932in}{0.976520in}}%
\pgfpathlineto{\pgfqpoint{1.913761in}{0.967108in}}%
\pgfpathlineto{\pgfqpoint{1.907001in}{0.957791in}}%
\pgfpathlineto{\pgfqpoint{1.903615in}{0.953077in}}%
\pgfpathlineto{\pgfqpoint{1.900241in}{0.948684in}}%
\pgfpathlineto{\pgfqpoint{1.896880in}{0.944607in}}%
\pgfpathlineto{\pgfqpoint{1.893530in}{0.940840in}}%
\pgfpathlineto{\pgfqpoint{1.900145in}{0.949955in}}%
\pgfpathlineto{\pgfqpoint{1.906186in}{0.959163in}}%
\pgfpathlineto{\pgfqpoint{1.911648in}{0.968454in}}%
\pgfpathlineto{\pgfqpoint{1.916528in}{0.977818in}}%
\pgfpathclose%
\pgfusepath{fill}%
\end{pgfscope}%
\begin{pgfscope}%
\pgfpathrectangle{\pgfqpoint{0.329460in}{0.284240in}}{\pgfqpoint{1.989680in}{1.989680in}}%
\pgfusepath{clip}%
\pgfsetbuttcap%
\pgfsetroundjoin%
\definecolor{currentfill}{rgb}{0.134692,0.658636,0.517649}%
\pgfsetfillcolor{currentfill}%
\pgfsetlinewidth{0.000000pt}%
\definecolor{currentstroke}{rgb}{0.000000,0.000000,0.000000}%
\pgfsetstrokecolor{currentstroke}%
\pgfsetdash{}{0pt}%
\pgfpathmoveto{\pgfqpoint{1.593700in}{1.482993in}}%
\pgfpathlineto{\pgfqpoint{1.596967in}{1.476752in}}%
\pgfpathlineto{\pgfqpoint{1.600232in}{1.470478in}}%
\pgfpathlineto{\pgfqpoint{1.603493in}{1.464171in}}%
\pgfpathlineto{\pgfqpoint{1.606751in}{1.457834in}}%
\pgfpathlineto{\pgfqpoint{1.608523in}{1.453883in}}%
\pgfpathlineto{\pgfqpoint{1.610041in}{1.449904in}}%
\pgfpathlineto{\pgfqpoint{1.611304in}{1.445901in}}%
\pgfpathlineto{\pgfqpoint{1.612309in}{1.441878in}}%
\pgfpathlineto{\pgfqpoint{1.608972in}{1.448435in}}%
\pgfpathlineto{\pgfqpoint{1.605631in}{1.454962in}}%
\pgfpathlineto{\pgfqpoint{1.602287in}{1.461456in}}%
\pgfpathlineto{\pgfqpoint{1.598940in}{1.467916in}}%
\pgfpathlineto{\pgfqpoint{1.597995in}{1.471717in}}%
\pgfpathlineto{\pgfqpoint{1.596805in}{1.475499in}}%
\pgfpathlineto{\pgfqpoint{1.595372in}{1.479259in}}%
\pgfpathlineto{\pgfqpoint{1.593700in}{1.482993in}}%
\pgfpathclose%
\pgfusepath{fill}%
\end{pgfscope}%
\begin{pgfscope}%
\pgfpathrectangle{\pgfqpoint{0.329460in}{0.284240in}}{\pgfqpoint{1.989680in}{1.989680in}}%
\pgfusepath{clip}%
\pgfsetbuttcap%
\pgfsetroundjoin%
\definecolor{currentfill}{rgb}{0.274952,0.037752,0.364543}%
\pgfsetfillcolor{currentfill}%
\pgfsetlinewidth{0.000000pt}%
\definecolor{currentstroke}{rgb}{0.000000,0.000000,0.000000}%
\pgfsetstrokecolor{currentstroke}%
\pgfsetdash{}{0pt}%
\pgfpathmoveto{\pgfqpoint{1.790796in}{0.954793in}}%
\pgfpathlineto{\pgfqpoint{1.793916in}{0.951148in}}%
\pgfpathlineto{\pgfqpoint{1.797040in}{0.947668in}}%
\pgfpathlineto{\pgfqpoint{1.800167in}{0.944359in}}%
\pgfpathlineto{\pgfqpoint{1.803298in}{0.941223in}}%
\pgfpathlineto{\pgfqpoint{1.797207in}{0.933712in}}%
\pgfpathlineto{\pgfqpoint{1.790648in}{0.926299in}}%
\pgfpathlineto{\pgfqpoint{1.783628in}{0.918991in}}%
\pgfpathlineto{\pgfqpoint{1.776153in}{0.911797in}}%
\pgfpathlineto{\pgfqpoint{1.773196in}{0.915151in}}%
\pgfpathlineto{\pgfqpoint{1.770244in}{0.918680in}}%
\pgfpathlineto{\pgfqpoint{1.767295in}{0.922378in}}%
\pgfpathlineto{\pgfqpoint{1.764350in}{0.926243in}}%
\pgfpathlineto{\pgfqpoint{1.771631in}{0.933222in}}%
\pgfpathlineto{\pgfqpoint{1.778469in}{0.940312in}}%
\pgfpathlineto{\pgfqpoint{1.784859in}{0.947505in}}%
\pgfpathlineto{\pgfqpoint{1.790796in}{0.954793in}}%
\pgfpathclose%
\pgfusepath{fill}%
\end{pgfscope}%
\begin{pgfscope}%
\pgfpathrectangle{\pgfqpoint{0.329460in}{0.284240in}}{\pgfqpoint{1.989680in}{1.989680in}}%
\pgfusepath{clip}%
\pgfsetbuttcap%
\pgfsetroundjoin%
\definecolor{currentfill}{rgb}{0.271305,0.019942,0.347269}%
\pgfsetfillcolor{currentfill}%
\pgfsetlinewidth{0.000000pt}%
\definecolor{currentstroke}{rgb}{0.000000,0.000000,0.000000}%
\pgfsetstrokecolor{currentstroke}%
\pgfsetdash{}{0pt}%
\pgfpathmoveto{\pgfqpoint{1.803298in}{0.941223in}}%
\pgfpathlineto{\pgfqpoint{1.806434in}{0.938264in}}%
\pgfpathlineto{\pgfqpoint{1.809575in}{0.935487in}}%
\pgfpathlineto{\pgfqpoint{1.812720in}{0.932895in}}%
\pgfpathlineto{\pgfqpoint{1.815870in}{0.930492in}}%
\pgfpathlineto{\pgfqpoint{1.809623in}{0.922760in}}%
\pgfpathlineto{\pgfqpoint{1.802896in}{0.915127in}}%
\pgfpathlineto{\pgfqpoint{1.795694in}{0.907603in}}%
\pgfpathlineto{\pgfqpoint{1.788023in}{0.900195in}}%
\pgfpathlineto{\pgfqpoint{1.785049in}{0.902815in}}%
\pgfpathlineto{\pgfqpoint{1.782079in}{0.905624in}}%
\pgfpathlineto{\pgfqpoint{1.779113in}{0.908620in}}%
\pgfpathlineto{\pgfqpoint{1.776153in}{0.911797in}}%
\pgfpathlineto{\pgfqpoint{1.783628in}{0.918991in}}%
\pgfpathlineto{\pgfqpoint{1.790648in}{0.926299in}}%
\pgfpathlineto{\pgfqpoint{1.797207in}{0.933712in}}%
\pgfpathlineto{\pgfqpoint{1.803298in}{0.941223in}}%
\pgfpathclose%
\pgfusepath{fill}%
\end{pgfscope}%
\begin{pgfscope}%
\pgfpathrectangle{\pgfqpoint{0.329460in}{0.284240in}}{\pgfqpoint{1.989680in}{1.989680in}}%
\pgfusepath{clip}%
\pgfsetbuttcap%
\pgfsetroundjoin%
\definecolor{currentfill}{rgb}{0.195860,0.395433,0.555276}%
\pgfsetfillcolor{currentfill}%
\pgfsetlinewidth{0.000000pt}%
\definecolor{currentstroke}{rgb}{0.000000,0.000000,0.000000}%
\pgfsetstrokecolor{currentstroke}%
\pgfsetdash{}{0pt}%
\pgfpathmoveto{\pgfqpoint{1.023487in}{1.212062in}}%
\pgfpathlineto{\pgfqpoint{1.020301in}{1.204789in}}%
\pgfpathlineto{\pgfqpoint{1.017116in}{1.197550in}}%
\pgfpathlineto{\pgfqpoint{1.013932in}{1.190349in}}%
\pgfpathlineto{\pgfqpoint{1.010749in}{1.183188in}}%
\pgfpathlineto{\pgfqpoint{1.007634in}{1.188692in}}%
\pgfpathlineto{\pgfqpoint{1.004869in}{1.194238in}}%
\pgfpathlineto{\pgfqpoint{1.002456in}{1.199821in}}%
\pgfpathlineto{\pgfqpoint{1.000398in}{1.205435in}}%
\pgfpathlineto{\pgfqpoint{1.003666in}{1.212364in}}%
\pgfpathlineto{\pgfqpoint{1.006935in}{1.219333in}}%
\pgfpathlineto{\pgfqpoint{1.010206in}{1.226340in}}%
\pgfpathlineto{\pgfqpoint{1.013478in}{1.233382in}}%
\pgfpathlineto{\pgfqpoint{1.015471in}{1.228001in}}%
\pgfpathlineto{\pgfqpoint{1.017805in}{1.222651in}}%
\pgfpathlineto{\pgfqpoint{1.020477in}{1.217336in}}%
\pgfpathlineto{\pgfqpoint{1.023487in}{1.212062in}}%
\pgfpathclose%
\pgfusepath{fill}%
\end{pgfscope}%
\begin{pgfscope}%
\pgfpathrectangle{\pgfqpoint{0.329460in}{0.284240in}}{\pgfqpoint{1.989680in}{1.989680in}}%
\pgfusepath{clip}%
\pgfsetbuttcap%
\pgfsetroundjoin%
\definecolor{currentfill}{rgb}{0.487026,0.823929,0.312321}%
\pgfsetfillcolor{currentfill}%
\pgfsetlinewidth{0.000000pt}%
\definecolor{currentstroke}{rgb}{0.000000,0.000000,0.000000}%
\pgfsetstrokecolor{currentstroke}%
\pgfsetdash{}{0pt}%
\pgfpathmoveto{\pgfqpoint{1.237519in}{1.657956in}}%
\pgfpathlineto{\pgfqpoint{1.235206in}{1.653996in}}%
\pgfpathlineto{\pgfqpoint{1.232896in}{1.649957in}}%
\pgfpathlineto{\pgfqpoint{1.230588in}{1.645842in}}%
\pgfpathlineto{\pgfqpoint{1.228283in}{1.641651in}}%
\pgfpathlineto{\pgfqpoint{1.232495in}{1.643455in}}%
\pgfpathlineto{\pgfqpoint{1.236823in}{1.645194in}}%
\pgfpathlineto{\pgfqpoint{1.241260in}{1.646868in}}%
\pgfpathlineto{\pgfqpoint{1.245803in}{1.648474in}}%
\pgfpathlineto{\pgfqpoint{1.247777in}{1.652532in}}%
\pgfpathlineto{\pgfqpoint{1.249754in}{1.656516in}}%
\pgfpathlineto{\pgfqpoint{1.251733in}{1.660422in}}%
\pgfpathlineto{\pgfqpoint{1.253714in}{1.664249in}}%
\pgfpathlineto{\pgfqpoint{1.249514in}{1.662768in}}%
\pgfpathlineto{\pgfqpoint{1.245412in}{1.661224in}}%
\pgfpathlineto{\pgfqpoint{1.241413in}{1.659620in}}%
\pgfpathlineto{\pgfqpoint{1.237519in}{1.657956in}}%
\pgfpathclose%
\pgfusepath{fill}%
\end{pgfscope}%
\begin{pgfscope}%
\pgfpathrectangle{\pgfqpoint{0.329460in}{0.284240in}}{\pgfqpoint{1.989680in}{1.989680in}}%
\pgfusepath{clip}%
\pgfsetbuttcap%
\pgfsetroundjoin%
\definecolor{currentfill}{rgb}{0.565498,0.842430,0.262877}%
\pgfsetfillcolor{currentfill}%
\pgfsetlinewidth{0.000000pt}%
\definecolor{currentstroke}{rgb}{0.000000,0.000000,0.000000}%
\pgfsetstrokecolor{currentstroke}%
\pgfsetdash{}{0pt}%
\pgfpathmoveto{\pgfqpoint{1.403096in}{1.688123in}}%
\pgfpathlineto{\pgfqpoint{1.404248in}{1.684839in}}%
\pgfpathlineto{\pgfqpoint{1.405399in}{1.681470in}}%
\pgfpathlineto{\pgfqpoint{1.406549in}{1.678019in}}%
\pgfpathlineto{\pgfqpoint{1.407697in}{1.674486in}}%
\pgfpathlineto{\pgfqpoint{1.412579in}{1.673610in}}%
\pgfpathlineto{\pgfqpoint{1.417403in}{1.672663in}}%
\pgfpathlineto{\pgfqpoint{1.422163in}{1.671643in}}%
\pgfpathlineto{\pgfqpoint{1.426854in}{1.670554in}}%
\pgfpathlineto{\pgfqpoint{1.425316in}{1.674169in}}%
\pgfpathlineto{\pgfqpoint{1.423775in}{1.677703in}}%
\pgfpathlineto{\pgfqpoint{1.422233in}{1.681153in}}%
\pgfpathlineto{\pgfqpoint{1.420689in}{1.684519in}}%
\pgfpathlineto{\pgfqpoint{1.416381in}{1.685518in}}%
\pgfpathlineto{\pgfqpoint{1.412010in}{1.686452in}}%
\pgfpathlineto{\pgfqpoint{1.407580in}{1.687320in}}%
\pgfpathlineto{\pgfqpoint{1.403096in}{1.688123in}}%
\pgfpathclose%
\pgfusepath{fill}%
\end{pgfscope}%
\begin{pgfscope}%
\pgfpathrectangle{\pgfqpoint{0.329460in}{0.284240in}}{\pgfqpoint{1.989680in}{1.989680in}}%
\pgfusepath{clip}%
\pgfsetbuttcap%
\pgfsetroundjoin%
\definecolor{currentfill}{rgb}{0.279566,0.067836,0.391917}%
\pgfsetfillcolor{currentfill}%
\pgfsetlinewidth{0.000000pt}%
\definecolor{currentstroke}{rgb}{0.000000,0.000000,0.000000}%
\pgfsetstrokecolor{currentstroke}%
\pgfsetdash{}{0pt}%
\pgfpathmoveto{\pgfqpoint{1.778351in}{0.970954in}}%
\pgfpathlineto{\pgfqpoint{1.781458in}{0.966684in}}%
\pgfpathlineto{\pgfqpoint{1.784567in}{0.962565in}}%
\pgfpathlineto{\pgfqpoint{1.787680in}{0.958599in}}%
\pgfpathlineto{\pgfqpoint{1.790796in}{0.954793in}}%
\pgfpathlineto{\pgfqpoint{1.784859in}{0.947505in}}%
\pgfpathlineto{\pgfqpoint{1.778469in}{0.940312in}}%
\pgfpathlineto{\pgfqpoint{1.771631in}{0.933222in}}%
\pgfpathlineto{\pgfqpoint{1.764350in}{0.926243in}}%
\pgfpathlineto{\pgfqpoint{1.761409in}{0.930269in}}%
\pgfpathlineto{\pgfqpoint{1.758471in}{0.934454in}}%
\pgfpathlineto{\pgfqpoint{1.755536in}{0.938794in}}%
\pgfpathlineto{\pgfqpoint{1.752604in}{0.943284in}}%
\pgfpathlineto{\pgfqpoint{1.759690in}{0.950048in}}%
\pgfpathlineto{\pgfqpoint{1.766347in}{0.956919in}}%
\pgfpathlineto{\pgfqpoint{1.772569in}{0.963891in}}%
\pgfpathlineto{\pgfqpoint{1.778351in}{0.970954in}}%
\pgfpathclose%
\pgfusepath{fill}%
\end{pgfscope}%
\begin{pgfscope}%
\pgfpathrectangle{\pgfqpoint{0.329460in}{0.284240in}}{\pgfqpoint{1.989680in}{1.989680in}}%
\pgfusepath{clip}%
\pgfsetbuttcap%
\pgfsetroundjoin%
\definecolor{currentfill}{rgb}{0.268510,0.009605,0.335427}%
\pgfsetfillcolor{currentfill}%
\pgfsetlinewidth{0.000000pt}%
\definecolor{currentstroke}{rgb}{0.000000,0.000000,0.000000}%
\pgfsetstrokecolor{currentstroke}%
\pgfsetdash{}{0pt}%
\pgfpathmoveto{\pgfqpoint{1.815870in}{0.930492in}}%
\pgfpathlineto{\pgfqpoint{1.819025in}{0.928283in}}%
\pgfpathlineto{\pgfqpoint{1.822185in}{0.926271in}}%
\pgfpathlineto{\pgfqpoint{1.825351in}{0.924461in}}%
\pgfpathlineto{\pgfqpoint{1.828522in}{0.922857in}}%
\pgfpathlineto{\pgfqpoint{1.822121in}{0.914905in}}%
\pgfpathlineto{\pgfqpoint{1.815225in}{0.907055in}}%
\pgfpathlineto{\pgfqpoint{1.807841in}{0.899315in}}%
\pgfpathlineto{\pgfqpoint{1.799974in}{0.891695in}}%
\pgfpathlineto{\pgfqpoint{1.796978in}{0.893515in}}%
\pgfpathlineto{\pgfqpoint{1.793988in}{0.895541in}}%
\pgfpathlineto{\pgfqpoint{1.791003in}{0.897769in}}%
\pgfpathlineto{\pgfqpoint{1.788023in}{0.900195in}}%
\pgfpathlineto{\pgfqpoint{1.795694in}{0.907603in}}%
\pgfpathlineto{\pgfqpoint{1.802896in}{0.915127in}}%
\pgfpathlineto{\pgfqpoint{1.809623in}{0.922760in}}%
\pgfpathlineto{\pgfqpoint{1.815870in}{0.930492in}}%
\pgfpathclose%
\pgfusepath{fill}%
\end{pgfscope}%
\begin{pgfscope}%
\pgfpathrectangle{\pgfqpoint{0.329460in}{0.284240in}}{\pgfqpoint{1.989680in}{1.989680in}}%
\pgfusepath{clip}%
\pgfsetbuttcap%
\pgfsetroundjoin%
\definecolor{currentfill}{rgb}{0.565498,0.842430,0.262877}%
\pgfsetfillcolor{currentfill}%
\pgfsetlinewidth{0.000000pt}%
\definecolor{currentstroke}{rgb}{0.000000,0.000000,0.000000}%
\pgfsetstrokecolor{currentstroke}%
\pgfsetdash{}{0pt}%
\pgfpathmoveto{\pgfqpoint{1.277912in}{1.683579in}}%
\pgfpathlineto{\pgfqpoint{1.276284in}{1.680191in}}%
\pgfpathlineto{\pgfqpoint{1.274658in}{1.676719in}}%
\pgfpathlineto{\pgfqpoint{1.273034in}{1.673164in}}%
\pgfpathlineto{\pgfqpoint{1.271412in}{1.669527in}}%
\pgfpathlineto{\pgfqpoint{1.276039in}{1.670678in}}%
\pgfpathlineto{\pgfqpoint{1.280738in}{1.671760in}}%
\pgfpathlineto{\pgfqpoint{1.285505in}{1.672771in}}%
\pgfpathlineto{\pgfqpoint{1.290335in}{1.673711in}}%
\pgfpathlineto{\pgfqpoint{1.291572in}{1.677260in}}%
\pgfpathlineto{\pgfqpoint{1.292810in}{1.680728in}}%
\pgfpathlineto{\pgfqpoint{1.294050in}{1.684113in}}%
\pgfpathlineto{\pgfqpoint{1.295291in}{1.687413in}}%
\pgfpathlineto{\pgfqpoint{1.290855in}{1.686551in}}%
\pgfpathlineto{\pgfqpoint{1.286477in}{1.685625in}}%
\pgfpathlineto{\pgfqpoint{1.282161in}{1.684634in}}%
\pgfpathlineto{\pgfqpoint{1.277912in}{1.683579in}}%
\pgfpathclose%
\pgfusepath{fill}%
\end{pgfscope}%
\begin{pgfscope}%
\pgfpathrectangle{\pgfqpoint{0.329460in}{0.284240in}}{\pgfqpoint{1.989680in}{1.989680in}}%
\pgfusepath{clip}%
\pgfsetbuttcap%
\pgfsetroundjoin%
\definecolor{currentfill}{rgb}{0.344074,0.780029,0.397381}%
\pgfsetfillcolor{currentfill}%
\pgfsetlinewidth{0.000000pt}%
\definecolor{currentstroke}{rgb}{0.000000,0.000000,0.000000}%
\pgfsetstrokecolor{currentstroke}%
\pgfsetdash{}{0pt}%
\pgfpathmoveto{\pgfqpoint{1.511586in}{1.608505in}}%
\pgfpathlineto{\pgfqpoint{1.514376in}{1.603678in}}%
\pgfpathlineto{\pgfqpoint{1.517162in}{1.598785in}}%
\pgfpathlineto{\pgfqpoint{1.519945in}{1.593829in}}%
\pgfpathlineto{\pgfqpoint{1.522725in}{1.588811in}}%
\pgfpathlineto{\pgfqpoint{1.526345in}{1.586198in}}%
\pgfpathlineto{\pgfqpoint{1.529794in}{1.583531in}}%
\pgfpathlineto{\pgfqpoint{1.533070in}{1.580812in}}%
\pgfpathlineto{\pgfqpoint{1.536169in}{1.578043in}}%
\pgfpathlineto{\pgfqpoint{1.533166in}{1.583245in}}%
\pgfpathlineto{\pgfqpoint{1.530159in}{1.588385in}}%
\pgfpathlineto{\pgfqpoint{1.527150in}{1.593461in}}%
\pgfpathlineto{\pgfqpoint{1.524137in}{1.598471in}}%
\pgfpathlineto{\pgfqpoint{1.521244in}{1.601051in}}%
\pgfpathlineto{\pgfqpoint{1.518186in}{1.603585in}}%
\pgfpathlineto{\pgfqpoint{1.514966in}{1.606070in}}%
\pgfpathlineto{\pgfqpoint{1.511586in}{1.608505in}}%
\pgfpathclose%
\pgfusepath{fill}%
\end{pgfscope}%
\begin{pgfscope}%
\pgfpathrectangle{\pgfqpoint{0.329460in}{0.284240in}}{\pgfqpoint{1.989680in}{1.989680in}}%
\pgfusepath{clip}%
\pgfsetbuttcap%
\pgfsetroundjoin%
\definecolor{currentfill}{rgb}{0.122606,0.585371,0.546557}%
\pgfsetfillcolor{currentfill}%
\pgfsetlinewidth{0.000000pt}%
\definecolor{currentstroke}{rgb}{0.000000,0.000000,0.000000}%
\pgfsetstrokecolor{currentstroke}%
\pgfsetdash{}{0pt}%
\pgfpathmoveto{\pgfqpoint{1.075467in}{1.394460in}}%
\pgfpathlineto{\pgfqpoint{1.072117in}{1.387510in}}%
\pgfpathlineto{\pgfqpoint{1.068769in}{1.380542in}}%
\pgfpathlineto{\pgfqpoint{1.065425in}{1.373560in}}%
\pgfpathlineto{\pgfqpoint{1.062082in}{1.366567in}}%
\pgfpathlineto{\pgfqpoint{1.061806in}{1.371087in}}%
\pgfpathlineto{\pgfqpoint{1.061819in}{1.375606in}}%
\pgfpathlineto{\pgfqpoint{1.062119in}{1.380119in}}%
\pgfpathlineto{\pgfqpoint{1.062707in}{1.384623in}}%
\pgfpathlineto{\pgfqpoint{1.066033in}{1.391388in}}%
\pgfpathlineto{\pgfqpoint{1.069361in}{1.398142in}}%
\pgfpathlineto{\pgfqpoint{1.072692in}{1.404881in}}%
\pgfpathlineto{\pgfqpoint{1.076025in}{1.411605in}}%
\pgfpathlineto{\pgfqpoint{1.075474in}{1.407329in}}%
\pgfpathlineto{\pgfqpoint{1.075196in}{1.403043in}}%
\pgfpathlineto{\pgfqpoint{1.075194in}{1.398752in}}%
\pgfpathlineto{\pgfqpoint{1.075467in}{1.394460in}}%
\pgfpathclose%
\pgfusepath{fill}%
\end{pgfscope}%
\begin{pgfscope}%
\pgfpathrectangle{\pgfqpoint{0.329460in}{0.284240in}}{\pgfqpoint{1.989680in}{1.989680in}}%
\pgfusepath{clip}%
\pgfsetbuttcap%
\pgfsetroundjoin%
\definecolor{currentfill}{rgb}{0.281477,0.755203,0.432552}%
\pgfsetfillcolor{currentfill}%
\pgfsetlinewidth{0.000000pt}%
\definecolor{currentstroke}{rgb}{0.000000,0.000000,0.000000}%
\pgfsetstrokecolor{currentstroke}%
\pgfsetdash{}{0pt}%
\pgfpathmoveto{\pgfqpoint{1.536169in}{1.578043in}}%
\pgfpathlineto{\pgfqpoint{1.539169in}{1.572782in}}%
\pgfpathlineto{\pgfqpoint{1.542166in}{1.567462in}}%
\pgfpathlineto{\pgfqpoint{1.545159in}{1.562085in}}%
\pgfpathlineto{\pgfqpoint{1.548149in}{1.556655in}}%
\pgfpathlineto{\pgfqpoint{1.551264in}{1.553646in}}%
\pgfpathlineto{\pgfqpoint{1.554183in}{1.550591in}}%
\pgfpathlineto{\pgfqpoint{1.556904in}{1.547491in}}%
\pgfpathlineto{\pgfqpoint{1.559424in}{1.544349in}}%
\pgfpathlineto{\pgfqpoint{1.556257in}{1.549978in}}%
\pgfpathlineto{\pgfqpoint{1.553086in}{1.555552in}}%
\pgfpathlineto{\pgfqpoint{1.549911in}{1.561070in}}%
\pgfpathlineto{\pgfqpoint{1.546734in}{1.566529in}}%
\pgfpathlineto{\pgfqpoint{1.544373in}{1.569468in}}%
\pgfpathlineto{\pgfqpoint{1.541824in}{1.572369in}}%
\pgfpathlineto{\pgfqpoint{1.539088in}{1.575228in}}%
\pgfpathlineto{\pgfqpoint{1.536169in}{1.578043in}}%
\pgfpathclose%
\pgfusepath{fill}%
\end{pgfscope}%
\begin{pgfscope}%
\pgfpathrectangle{\pgfqpoint{0.329460in}{0.284240in}}{\pgfqpoint{1.989680in}{1.989680in}}%
\pgfusepath{clip}%
\pgfsetbuttcap%
\pgfsetroundjoin%
\definecolor{currentfill}{rgb}{0.282327,0.094955,0.417331}%
\pgfsetfillcolor{currentfill}%
\pgfsetlinewidth{0.000000pt}%
\definecolor{currentstroke}{rgb}{0.000000,0.000000,0.000000}%
\pgfsetstrokecolor{currentstroke}%
\pgfsetdash{}{0pt}%
\pgfpathmoveto{\pgfqpoint{1.765953in}{0.989469in}}%
\pgfpathlineto{\pgfqpoint{1.769049in}{0.984632in}}%
\pgfpathlineto{\pgfqpoint{1.772147in}{0.979932in}}%
\pgfpathlineto{\pgfqpoint{1.775248in}{0.975372in}}%
\pgfpathlineto{\pgfqpoint{1.778351in}{0.970954in}}%
\pgfpathlineto{\pgfqpoint{1.772569in}{0.963891in}}%
\pgfpathlineto{\pgfqpoint{1.766347in}{0.956919in}}%
\pgfpathlineto{\pgfqpoint{1.759690in}{0.950048in}}%
\pgfpathlineto{\pgfqpoint{1.752604in}{0.943284in}}%
\pgfpathlineto{\pgfqpoint{1.749676in}{0.947922in}}%
\pgfpathlineto{\pgfqpoint{1.746750in}{0.952703in}}%
\pgfpathlineto{\pgfqpoint{1.743826in}{0.957624in}}%
\pgfpathlineto{\pgfqpoint{1.740905in}{0.962682in}}%
\pgfpathlineto{\pgfqpoint{1.747797in}{0.969229in}}%
\pgfpathlineto{\pgfqpoint{1.754272in}{0.975881in}}%
\pgfpathlineto{\pgfqpoint{1.760326in}{0.982630in}}%
\pgfpathlineto{\pgfqpoint{1.765953in}{0.989469in}}%
\pgfpathclose%
\pgfusepath{fill}%
\end{pgfscope}%
\begin{pgfscope}%
\pgfpathrectangle{\pgfqpoint{0.329460in}{0.284240in}}{\pgfqpoint{1.989680in}{1.989680in}}%
\pgfusepath{clip}%
\pgfsetbuttcap%
\pgfsetroundjoin%
\definecolor{currentfill}{rgb}{0.179019,0.433756,0.557430}%
\pgfsetfillcolor{currentfill}%
\pgfsetlinewidth{0.000000pt}%
\definecolor{currentstroke}{rgb}{0.000000,0.000000,0.000000}%
\pgfsetstrokecolor{currentstroke}%
\pgfsetdash{}{0pt}%
\pgfpathmoveto{\pgfqpoint{1.677229in}{1.266437in}}%
\pgfpathlineto{\pgfqpoint{1.680520in}{1.259335in}}%
\pgfpathlineto{\pgfqpoint{1.683809in}{1.252258in}}%
\pgfpathlineto{\pgfqpoint{1.687096in}{1.245207in}}%
\pgfpathlineto{\pgfqpoint{1.690382in}{1.238185in}}%
\pgfpathlineto{\pgfqpoint{1.688692in}{1.232783in}}%
\pgfpathlineto{\pgfqpoint{1.686661in}{1.227405in}}%
\pgfpathlineto{\pgfqpoint{1.684290in}{1.222058in}}%
\pgfpathlineto{\pgfqpoint{1.681580in}{1.216748in}}%
\pgfpathlineto{\pgfqpoint{1.678369in}{1.224001in}}%
\pgfpathlineto{\pgfqpoint{1.675156in}{1.231284in}}%
\pgfpathlineto{\pgfqpoint{1.671942in}{1.238593in}}%
\pgfpathlineto{\pgfqpoint{1.668726in}{1.245926in}}%
\pgfpathlineto{\pgfqpoint{1.671342in}{1.251006in}}%
\pgfpathlineto{\pgfqpoint{1.673631in}{1.256122in}}%
\pgfpathlineto{\pgfqpoint{1.675594in}{1.261267in}}%
\pgfpathlineto{\pgfqpoint{1.677229in}{1.266437in}}%
\pgfpathclose%
\pgfusepath{fill}%
\end{pgfscope}%
\begin{pgfscope}%
\pgfpathrectangle{\pgfqpoint{0.329460in}{0.284240in}}{\pgfqpoint{1.989680in}{1.989680in}}%
\pgfusepath{clip}%
\pgfsetbuttcap%
\pgfsetroundjoin%
\definecolor{currentfill}{rgb}{0.412913,0.803041,0.357269}%
\pgfsetfillcolor{currentfill}%
\pgfsetlinewidth{0.000000pt}%
\definecolor{currentstroke}{rgb}{0.000000,0.000000,0.000000}%
\pgfsetstrokecolor{currentstroke}%
\pgfsetdash{}{0pt}%
\pgfpathmoveto{\pgfqpoint{1.486421in}{1.635647in}}%
\pgfpathlineto{\pgfqpoint{1.488957in}{1.631265in}}%
\pgfpathlineto{\pgfqpoint{1.491490in}{1.626811in}}%
\pgfpathlineto{\pgfqpoint{1.494020in}{1.622287in}}%
\pgfpathlineto{\pgfqpoint{1.496547in}{1.617693in}}%
\pgfpathlineto{\pgfqpoint{1.500527in}{1.615483in}}%
\pgfpathlineto{\pgfqpoint{1.504363in}{1.613214in}}%
\pgfpathlineto{\pgfqpoint{1.508051in}{1.610887in}}%
\pgfpathlineto{\pgfqpoint{1.511586in}{1.608505in}}%
\pgfpathlineto{\pgfqpoint{1.508793in}{1.613265in}}%
\pgfpathlineto{\pgfqpoint{1.505997in}{1.617957in}}%
\pgfpathlineto{\pgfqpoint{1.503198in}{1.622577in}}%
\pgfpathlineto{\pgfqpoint{1.500396in}{1.627125in}}%
\pgfpathlineto{\pgfqpoint{1.497112in}{1.629334in}}%
\pgfpathlineto{\pgfqpoint{1.493685in}{1.631492in}}%
\pgfpathlineto{\pgfqpoint{1.490121in}{1.633597in}}%
\pgfpathlineto{\pgfqpoint{1.486421in}{1.635647in}}%
\pgfpathclose%
\pgfusepath{fill}%
\end{pgfscope}%
\begin{pgfscope}%
\pgfpathrectangle{\pgfqpoint{0.329460in}{0.284240in}}{\pgfqpoint{1.989680in}{1.989680in}}%
\pgfusepath{clip}%
\pgfsetbuttcap%
\pgfsetroundjoin%
\definecolor{currentfill}{rgb}{0.172719,0.448791,0.557885}%
\pgfsetfillcolor{currentfill}%
\pgfsetlinewidth{0.000000pt}%
\definecolor{currentstroke}{rgb}{0.000000,0.000000,0.000000}%
\pgfsetstrokecolor{currentstroke}%
\pgfsetdash{}{0pt}%
\pgfpathmoveto{\pgfqpoint{2.023365in}{1.255618in}}%
\pgfpathlineto{\pgfqpoint{2.027291in}{1.268989in}}%
\pgfpathlineto{\pgfqpoint{2.031238in}{1.282818in}}%
\pgfpathlineto{\pgfqpoint{2.035209in}{1.297113in}}%
\pgfpathlineto{\pgfqpoint{2.035166in}{1.286289in}}%
\pgfpathlineto{\pgfqpoint{2.034437in}{1.275452in}}%
\pgfpathlineto{\pgfqpoint{2.033019in}{1.264613in}}%
\pgfpathlineto{\pgfqpoint{2.030911in}{1.253784in}}%
\pgfpathlineto{\pgfqpoint{2.026938in}{1.239644in}}%
\pgfpathlineto{\pgfqpoint{2.022988in}{1.225973in}}%
\pgfpathlineto{\pgfqpoint{2.019060in}{1.212761in}}%
\pgfpathlineto{\pgfqpoint{2.021152in}{1.223472in}}%
\pgfpathlineto{\pgfqpoint{2.022566in}{1.234192in}}%
\pgfpathlineto{\pgfqpoint{2.023302in}{1.244911in}}%
\pgfpathlineto{\pgfqpoint{2.023365in}{1.255618in}}%
\pgfpathclose%
\pgfusepath{fill}%
\end{pgfscope}%
\begin{pgfscope}%
\pgfpathrectangle{\pgfqpoint{0.329460in}{0.284240in}}{\pgfqpoint{1.989680in}{1.989680in}}%
\pgfusepath{clip}%
\pgfsetbuttcap%
\pgfsetroundjoin%
\definecolor{currentfill}{rgb}{0.147607,0.511733,0.557049}%
\pgfsetfillcolor{currentfill}%
\pgfsetlinewidth{0.000000pt}%
\definecolor{currentstroke}{rgb}{0.000000,0.000000,0.000000}%
\pgfsetstrokecolor{currentstroke}%
\pgfsetdash{}{0pt}%
\pgfpathmoveto{\pgfqpoint{1.052881in}{1.319599in}}%
\pgfpathlineto{\pgfqpoint{1.049586in}{1.312351in}}%
\pgfpathlineto{\pgfqpoint{1.046294in}{1.305107in}}%
\pgfpathlineto{\pgfqpoint{1.043004in}{1.297868in}}%
\pgfpathlineto{\pgfqpoint{1.039716in}{1.290636in}}%
\pgfpathlineto{\pgfqpoint{1.038172in}{1.295576in}}%
\pgfpathlineto{\pgfqpoint{1.036943in}{1.300534in}}%
\pgfpathlineto{\pgfqpoint{1.036031in}{1.305505in}}%
\pgfpathlineto{\pgfqpoint{1.035434in}{1.310484in}}%
\pgfpathlineto{\pgfqpoint{1.038757in}{1.317484in}}%
\pgfpathlineto{\pgfqpoint{1.042082in}{1.324492in}}%
\pgfpathlineto{\pgfqpoint{1.045409in}{1.331505in}}%
\pgfpathlineto{\pgfqpoint{1.048739in}{1.338521in}}%
\pgfpathlineto{\pgfqpoint{1.049321in}{1.333774in}}%
\pgfpathlineto{\pgfqpoint{1.050205in}{1.329035in}}%
\pgfpathlineto{\pgfqpoint{1.051392in}{1.324308in}}%
\pgfpathlineto{\pgfqpoint{1.052881in}{1.319599in}}%
\pgfpathclose%
\pgfusepath{fill}%
\end{pgfscope}%
\begin{pgfscope}%
\pgfpathrectangle{\pgfqpoint{0.329460in}{0.284240in}}{\pgfqpoint{1.989680in}{1.989680in}}%
\pgfusepath{clip}%
\pgfsetbuttcap%
\pgfsetroundjoin%
\definecolor{currentfill}{rgb}{0.248629,0.278775,0.534556}%
\pgfsetfillcolor{currentfill}%
\pgfsetlinewidth{0.000000pt}%
\definecolor{currentstroke}{rgb}{0.000000,0.000000,0.000000}%
\pgfsetstrokecolor{currentstroke}%
\pgfsetdash{}{0pt}%
\pgfpathmoveto{\pgfqpoint{1.002331in}{1.104407in}}%
\pgfpathlineto{\pgfqpoint{0.999290in}{1.097512in}}%
\pgfpathlineto{\pgfqpoint{0.996250in}{1.090687in}}%
\pgfpathlineto{\pgfqpoint{0.993210in}{1.083934in}}%
\pgfpathlineto{\pgfqpoint{0.990169in}{1.077257in}}%
\pgfpathlineto{\pgfqpoint{0.985208in}{1.083205in}}%
\pgfpathlineto{\pgfqpoint{0.980625in}{1.089225in}}%
\pgfpathlineto{\pgfqpoint{0.976424in}{1.095311in}}%
\pgfpathlineto{\pgfqpoint{0.972609in}{1.101457in}}%
\pgfpathlineto{\pgfqpoint{0.975786in}{1.107906in}}%
\pgfpathlineto{\pgfqpoint{0.978962in}{1.114431in}}%
\pgfpathlineto{\pgfqpoint{0.982139in}{1.121028in}}%
\pgfpathlineto{\pgfqpoint{0.985316in}{1.127696in}}%
\pgfpathlineto{\pgfqpoint{0.989015in}{1.121781in}}%
\pgfpathlineto{\pgfqpoint{0.993086in}{1.115924in}}%
\pgfpathlineto{\pgfqpoint{0.997526in}{1.110130in}}%
\pgfpathlineto{\pgfqpoint{1.002331in}{1.104407in}}%
\pgfpathclose%
\pgfusepath{fill}%
\end{pgfscope}%
\begin{pgfscope}%
\pgfpathrectangle{\pgfqpoint{0.329460in}{0.284240in}}{\pgfqpoint{1.989680in}{1.989680in}}%
\pgfusepath{clip}%
\pgfsetbuttcap%
\pgfsetroundjoin%
\definecolor{currentfill}{rgb}{0.267004,0.004874,0.329415}%
\pgfsetfillcolor{currentfill}%
\pgfsetlinewidth{0.000000pt}%
\definecolor{currentstroke}{rgb}{0.000000,0.000000,0.000000}%
\pgfsetstrokecolor{currentstroke}%
\pgfsetdash{}{0pt}%
\pgfpathmoveto{\pgfqpoint{1.828522in}{0.922857in}}%
\pgfpathlineto{\pgfqpoint{1.831700in}{0.921463in}}%
\pgfpathlineto{\pgfqpoint{1.834884in}{0.920284in}}%
\pgfpathlineto{\pgfqpoint{1.838074in}{0.919322in}}%
\pgfpathlineto{\pgfqpoint{1.841271in}{0.918584in}}%
\pgfpathlineto{\pgfqpoint{1.834713in}{0.910414in}}%
\pgfpathlineto{\pgfqpoint{1.827648in}{0.902348in}}%
\pgfpathlineto{\pgfqpoint{1.820081in}{0.894395in}}%
\pgfpathlineto{\pgfqpoint{1.812018in}{0.886565in}}%
\pgfpathlineto{\pgfqpoint{1.808998in}{0.887516in}}%
\pgfpathlineto{\pgfqpoint{1.805984in}{0.888692in}}%
\pgfpathlineto{\pgfqpoint{1.802976in}{0.890086in}}%
\pgfpathlineto{\pgfqpoint{1.799974in}{0.891695in}}%
\pgfpathlineto{\pgfqpoint{1.807841in}{0.899315in}}%
\pgfpathlineto{\pgfqpoint{1.815225in}{0.907055in}}%
\pgfpathlineto{\pgfqpoint{1.822121in}{0.914905in}}%
\pgfpathlineto{\pgfqpoint{1.828522in}{0.922857in}}%
\pgfpathclose%
\pgfusepath{fill}%
\end{pgfscope}%
\begin{pgfscope}%
\pgfpathrectangle{\pgfqpoint{0.329460in}{0.284240in}}{\pgfqpoint{1.989680in}{1.989680in}}%
\pgfusepath{clip}%
\pgfsetbuttcap%
\pgfsetroundjoin%
\definecolor{currentfill}{rgb}{0.134692,0.658636,0.517649}%
\pgfsetfillcolor{currentfill}%
\pgfsetlinewidth{0.000000pt}%
\definecolor{currentstroke}{rgb}{0.000000,0.000000,0.000000}%
\pgfsetstrokecolor{currentstroke}%
\pgfsetdash{}{0pt}%
\pgfpathmoveto{\pgfqpoint{1.102800in}{1.464525in}}%
\pgfpathlineto{\pgfqpoint{1.099443in}{1.458015in}}%
\pgfpathlineto{\pgfqpoint{1.096089in}{1.451471in}}%
\pgfpathlineto{\pgfqpoint{1.092737in}{1.444894in}}%
\pgfpathlineto{\pgfqpoint{1.089389in}{1.438288in}}%
\pgfpathlineto{\pgfqpoint{1.090165in}{1.442326in}}%
\pgfpathlineto{\pgfqpoint{1.091199in}{1.446347in}}%
\pgfpathlineto{\pgfqpoint{1.092490in}{1.450348in}}%
\pgfpathlineto{\pgfqpoint{1.094037in}{1.454324in}}%
\pgfpathlineto{\pgfqpoint{1.097317in}{1.460709in}}%
\pgfpathlineto{\pgfqpoint{1.100600in}{1.467064in}}%
\pgfpathlineto{\pgfqpoint{1.103887in}{1.473387in}}%
\pgfpathlineto{\pgfqpoint{1.107177in}{1.479675in}}%
\pgfpathlineto{\pgfqpoint{1.105718in}{1.475918in}}%
\pgfpathlineto{\pgfqpoint{1.104501in}{1.472138in}}%
\pgfpathlineto{\pgfqpoint{1.103528in}{1.468339in}}%
\pgfpathlineto{\pgfqpoint{1.102800in}{1.464525in}}%
\pgfpathclose%
\pgfusepath{fill}%
\end{pgfscope}%
\begin{pgfscope}%
\pgfpathrectangle{\pgfqpoint{0.329460in}{0.284240in}}{\pgfqpoint{1.989680in}{1.989680in}}%
\pgfusepath{clip}%
\pgfsetbuttcap%
\pgfsetroundjoin%
\definecolor{currentfill}{rgb}{0.231674,0.318106,0.544834}%
\pgfsetfillcolor{currentfill}%
\pgfsetlinewidth{0.000000pt}%
\definecolor{currentstroke}{rgb}{0.000000,0.000000,0.000000}%
\pgfsetstrokecolor{currentstroke}%
\pgfsetdash{}{0pt}%
\pgfpathmoveto{\pgfqpoint{1.707228in}{1.160099in}}%
\pgfpathlineto{\pgfqpoint{1.710430in}{1.153234in}}%
\pgfpathlineto{\pgfqpoint{1.713631in}{1.146427in}}%
\pgfpathlineto{\pgfqpoint{1.716832in}{1.139680in}}%
\pgfpathlineto{\pgfqpoint{1.720033in}{1.132997in}}%
\pgfpathlineto{\pgfqpoint{1.716667in}{1.127036in}}%
\pgfpathlineto{\pgfqpoint{1.712926in}{1.121127in}}%
\pgfpathlineto{\pgfqpoint{1.708814in}{1.115277in}}%
\pgfpathlineto{\pgfqpoint{1.704334in}{1.109491in}}%
\pgfpathlineto{\pgfqpoint{1.701258in}{1.116403in}}%
\pgfpathlineto{\pgfqpoint{1.698183in}{1.123378in}}%
\pgfpathlineto{\pgfqpoint{1.695107in}{1.130414in}}%
\pgfpathlineto{\pgfqpoint{1.692030in}{1.137508in}}%
\pgfpathlineto{\pgfqpoint{1.696366in}{1.143068in}}%
\pgfpathlineto{\pgfqpoint{1.700346in}{1.148691in}}%
\pgfpathlineto{\pgfqpoint{1.703968in}{1.154370in}}%
\pgfpathlineto{\pgfqpoint{1.707228in}{1.160099in}}%
\pgfpathclose%
\pgfusepath{fill}%
\end{pgfscope}%
\begin{pgfscope}%
\pgfpathrectangle{\pgfqpoint{0.329460in}{0.284240in}}{\pgfqpoint{1.989680in}{1.989680in}}%
\pgfusepath{clip}%
\pgfsetbuttcap%
\pgfsetroundjoin%
\definecolor{currentfill}{rgb}{0.344074,0.780029,0.397381}%
\pgfsetfillcolor{currentfill}%
\pgfsetlinewidth{0.000000pt}%
\definecolor{currentstroke}{rgb}{0.000000,0.000000,0.000000}%
\pgfsetstrokecolor{currentstroke}%
\pgfsetdash{}{0pt}%
\pgfpathmoveto{\pgfqpoint{1.175809in}{1.596142in}}%
\pgfpathlineto{\pgfqpoint{1.172752in}{1.591089in}}%
\pgfpathlineto{\pgfqpoint{1.169699in}{1.585970in}}%
\pgfpathlineto{\pgfqpoint{1.166649in}{1.580788in}}%
\pgfpathlineto{\pgfqpoint{1.163602in}{1.575543in}}%
\pgfpathlineto{\pgfqpoint{1.166541in}{1.578353in}}%
\pgfpathlineto{\pgfqpoint{1.169660in}{1.581116in}}%
\pgfpathlineto{\pgfqpoint{1.172956in}{1.583830in}}%
\pgfpathlineto{\pgfqpoint{1.176424in}{1.586491in}}%
\pgfpathlineto{\pgfqpoint{1.179258in}{1.591549in}}%
\pgfpathlineto{\pgfqpoint{1.182094in}{1.596544in}}%
\pgfpathlineto{\pgfqpoint{1.184934in}{1.601477in}}%
\pgfpathlineto{\pgfqpoint{1.187777in}{1.606343in}}%
\pgfpathlineto{\pgfqpoint{1.184539in}{1.603863in}}%
\pgfpathlineto{\pgfqpoint{1.181462in}{1.601335in}}%
\pgfpathlineto{\pgfqpoint{1.178552in}{1.598760in}}%
\pgfpathlineto{\pgfqpoint{1.175809in}{1.596142in}}%
\pgfpathclose%
\pgfusepath{fill}%
\end{pgfscope}%
\begin{pgfscope}%
\pgfpathrectangle{\pgfqpoint{0.329460in}{0.284240in}}{\pgfqpoint{1.989680in}{1.989680in}}%
\pgfusepath{clip}%
\pgfsetbuttcap%
\pgfsetroundjoin%
\definecolor{currentfill}{rgb}{0.412913,0.803041,0.357269}%
\pgfsetfillcolor{currentfill}%
\pgfsetlinewidth{0.000000pt}%
\definecolor{currentstroke}{rgb}{0.000000,0.000000,0.000000}%
\pgfsetstrokecolor{currentstroke}%
\pgfsetdash{}{0pt}%
\pgfpathmoveto{\pgfqpoint{1.199181in}{1.625121in}}%
\pgfpathlineto{\pgfqpoint{1.196325in}{1.620533in}}%
\pgfpathlineto{\pgfqpoint{1.193473in}{1.615873in}}%
\pgfpathlineto{\pgfqpoint{1.190623in}{1.611143in}}%
\pgfpathlineto{\pgfqpoint{1.187777in}{1.606343in}}%
\pgfpathlineto{\pgfqpoint{1.191174in}{1.608772in}}%
\pgfpathlineto{\pgfqpoint{1.194727in}{1.611148in}}%
\pgfpathlineto{\pgfqpoint{1.198431in}{1.613469in}}%
\pgfpathlineto{\pgfqpoint{1.202283in}{1.615732in}}%
\pgfpathlineto{\pgfqpoint{1.204873in}{1.620361in}}%
\pgfpathlineto{\pgfqpoint{1.207465in}{1.624921in}}%
\pgfpathlineto{\pgfqpoint{1.210060in}{1.629410in}}%
\pgfpathlineto{\pgfqpoint{1.212659in}{1.633827in}}%
\pgfpathlineto{\pgfqpoint{1.209079in}{1.631729in}}%
\pgfpathlineto{\pgfqpoint{1.205637in}{1.629577in}}%
\pgfpathlineto{\pgfqpoint{1.202337in}{1.627373in}}%
\pgfpathlineto{\pgfqpoint{1.199181in}{1.625121in}}%
\pgfpathclose%
\pgfusepath{fill}%
\end{pgfscope}%
\begin{pgfscope}%
\pgfpathrectangle{\pgfqpoint{0.329460in}{0.284240in}}{\pgfqpoint{1.989680in}{1.989680in}}%
\pgfusepath{clip}%
\pgfsetbuttcap%
\pgfsetroundjoin%
\definecolor{currentfill}{rgb}{0.283072,0.130895,0.449241}%
\pgfsetfillcolor{currentfill}%
\pgfsetlinewidth{0.000000pt}%
\definecolor{currentstroke}{rgb}{0.000000,0.000000,0.000000}%
\pgfsetstrokecolor{currentstroke}%
\pgfsetdash{}{0pt}%
\pgfpathmoveto{\pgfqpoint{1.753591in}{1.010106in}}%
\pgfpathlineto{\pgfqpoint{1.756678in}{1.004760in}}%
\pgfpathlineto{\pgfqpoint{1.759768in}{0.999536in}}%
\pgfpathlineto{\pgfqpoint{1.762859in}{0.994438in}}%
\pgfpathlineto{\pgfqpoint{1.765953in}{0.989469in}}%
\pgfpathlineto{\pgfqpoint{1.760326in}{0.982630in}}%
\pgfpathlineto{\pgfqpoint{1.754272in}{0.975881in}}%
\pgfpathlineto{\pgfqpoint{1.747797in}{0.969229in}}%
\pgfpathlineto{\pgfqpoint{1.740905in}{0.962682in}}%
\pgfpathlineto{\pgfqpoint{1.737987in}{0.967872in}}%
\pgfpathlineto{\pgfqpoint{1.735070in}{0.973191in}}%
\pgfpathlineto{\pgfqpoint{1.732156in}{0.978635in}}%
\pgfpathlineto{\pgfqpoint{1.729244in}{0.984202in}}%
\pgfpathlineto{\pgfqpoint{1.735940in}{0.990533in}}%
\pgfpathlineto{\pgfqpoint{1.742234in}{0.996965in}}%
\pgfpathlineto{\pgfqpoint{1.748119in}{1.003491in}}%
\pgfpathlineto{\pgfqpoint{1.753591in}{1.010106in}}%
\pgfpathclose%
\pgfusepath{fill}%
\end{pgfscope}%
\begin{pgfscope}%
\pgfpathrectangle{\pgfqpoint{0.329460in}{0.284240in}}{\pgfqpoint{1.989680in}{1.989680in}}%
\pgfusepath{clip}%
\pgfsetbuttcap%
\pgfsetroundjoin%
\definecolor{currentfill}{rgb}{0.220124,0.725509,0.466226}%
\pgfsetfillcolor{currentfill}%
\pgfsetlinewidth{0.000000pt}%
\definecolor{currentstroke}{rgb}{0.000000,0.000000,0.000000}%
\pgfsetstrokecolor{currentstroke}%
\pgfsetdash{}{0pt}%
\pgfpathmoveto{\pgfqpoint{1.559424in}{1.544349in}}%
\pgfpathlineto{\pgfqpoint{1.562589in}{1.538668in}}%
\pgfpathlineto{\pgfqpoint{1.565750in}{1.532936in}}%
\pgfpathlineto{\pgfqpoint{1.568908in}{1.527155in}}%
\pgfpathlineto{\pgfqpoint{1.572062in}{1.521328in}}%
\pgfpathlineto{\pgfqpoint{1.574526in}{1.517942in}}%
\pgfpathlineto{\pgfqpoint{1.576772in}{1.514518in}}%
\pgfpathlineto{\pgfqpoint{1.578796in}{1.511059in}}%
\pgfpathlineto{\pgfqpoint{1.580597in}{1.507570in}}%
\pgfpathlineto{\pgfqpoint{1.577313in}{1.513607in}}%
\pgfpathlineto{\pgfqpoint{1.574026in}{1.519598in}}%
\pgfpathlineto{\pgfqpoint{1.570735in}{1.525539in}}%
\pgfpathlineto{\pgfqpoint{1.567442in}{1.531430in}}%
\pgfpathlineto{\pgfqpoint{1.565752in}{1.534706in}}%
\pgfpathlineto{\pgfqpoint{1.563851in}{1.537954in}}%
\pgfpathlineto{\pgfqpoint{1.561741in}{1.541169in}}%
\pgfpathlineto{\pgfqpoint{1.559424in}{1.544349in}}%
\pgfpathclose%
\pgfusepath{fill}%
\end{pgfscope}%
\begin{pgfscope}%
\pgfpathrectangle{\pgfqpoint{0.329460in}{0.284240in}}{\pgfqpoint{1.989680in}{1.989680in}}%
\pgfusepath{clip}%
\pgfsetbuttcap%
\pgfsetroundjoin%
\definecolor{currentfill}{rgb}{0.565498,0.842430,0.262877}%
\pgfsetfillcolor{currentfill}%
\pgfsetlinewidth{0.000000pt}%
\definecolor{currentstroke}{rgb}{0.000000,0.000000,0.000000}%
\pgfsetstrokecolor{currentstroke}%
\pgfsetdash{}{0pt}%
\pgfpathmoveto{\pgfqpoint{1.420689in}{1.684519in}}%
\pgfpathlineto{\pgfqpoint{1.422233in}{1.681153in}}%
\pgfpathlineto{\pgfqpoint{1.423775in}{1.677703in}}%
\pgfpathlineto{\pgfqpoint{1.425316in}{1.674169in}}%
\pgfpathlineto{\pgfqpoint{1.426854in}{1.670554in}}%
\pgfpathlineto{\pgfqpoint{1.431473in}{1.669395in}}%
\pgfpathlineto{\pgfqpoint{1.436015in}{1.668168in}}%
\pgfpathlineto{\pgfqpoint{1.440475in}{1.666873in}}%
\pgfpathlineto{\pgfqpoint{1.444849in}{1.665513in}}%
\pgfpathlineto{\pgfqpoint{1.442943in}{1.669234in}}%
\pgfpathlineto{\pgfqpoint{1.441035in}{1.672873in}}%
\pgfpathlineto{\pgfqpoint{1.439124in}{1.676429in}}%
\pgfpathlineto{\pgfqpoint{1.437212in}{1.679901in}}%
\pgfpathlineto{\pgfqpoint{1.433196in}{1.681147in}}%
\pgfpathlineto{\pgfqpoint{1.429101in}{1.682333in}}%
\pgfpathlineto{\pgfqpoint{1.424931in}{1.683458in}}%
\pgfpathlineto{\pgfqpoint{1.420689in}{1.684519in}}%
\pgfpathclose%
\pgfusepath{fill}%
\end{pgfscope}%
\begin{pgfscope}%
\pgfpathrectangle{\pgfqpoint{0.329460in}{0.284240in}}{\pgfqpoint{1.989680in}{1.989680in}}%
\pgfusepath{clip}%
\pgfsetbuttcap%
\pgfsetroundjoin%
\definecolor{currentfill}{rgb}{0.281477,0.755203,0.432552}%
\pgfsetfillcolor{currentfill}%
\pgfsetlinewidth{0.000000pt}%
\definecolor{currentstroke}{rgb}{0.000000,0.000000,0.000000}%
\pgfsetstrokecolor{currentstroke}%
\pgfsetdash{}{0pt}%
\pgfpathmoveto{\pgfqpoint{1.153704in}{1.563886in}}%
\pgfpathlineto{\pgfqpoint{1.150494in}{1.558382in}}%
\pgfpathlineto{\pgfqpoint{1.147286in}{1.552818in}}%
\pgfpathlineto{\pgfqpoint{1.144082in}{1.547199in}}%
\pgfpathlineto{\pgfqpoint{1.140882in}{1.541524in}}%
\pgfpathlineto{\pgfqpoint{1.143221in}{1.544700in}}%
\pgfpathlineto{\pgfqpoint{1.145764in}{1.547837in}}%
\pgfpathlineto{\pgfqpoint{1.148507in}{1.550932in}}%
\pgfpathlineto{\pgfqpoint{1.151448in}{1.553983in}}%
\pgfpathlineto{\pgfqpoint{1.154482in}{1.559457in}}%
\pgfpathlineto{\pgfqpoint{1.157519in}{1.564876in}}%
\pgfpathlineto{\pgfqpoint{1.160559in}{1.570239in}}%
\pgfpathlineto{\pgfqpoint{1.163602in}{1.575543in}}%
\pgfpathlineto{\pgfqpoint{1.160846in}{1.572689in}}%
\pgfpathlineto{\pgfqpoint{1.158276in}{1.569792in}}%
\pgfpathlineto{\pgfqpoint{1.155894in}{1.566857in}}%
\pgfpathlineto{\pgfqpoint{1.153704in}{1.563886in}}%
\pgfpathclose%
\pgfusepath{fill}%
\end{pgfscope}%
\begin{pgfscope}%
\pgfpathrectangle{\pgfqpoint{0.329460in}{0.284240in}}{\pgfqpoint{1.989680in}{1.989680in}}%
\pgfusepath{clip}%
\pgfsetbuttcap%
\pgfsetroundjoin%
\definecolor{currentfill}{rgb}{0.260571,0.246922,0.522828}%
\pgfsetfillcolor{currentfill}%
\pgfsetlinewidth{0.000000pt}%
\definecolor{currentstroke}{rgb}{0.000000,0.000000,0.000000}%
\pgfsetstrokecolor{currentstroke}%
\pgfsetdash{}{0pt}%
\pgfpathmoveto{\pgfqpoint{0.747897in}{1.038252in}}%
\pgfpathlineto{\pgfqpoint{0.744281in}{1.046264in}}%
\pgfpathlineto{\pgfqpoint{0.740649in}{1.054656in}}%
\pgfpathlineto{\pgfqpoint{0.737001in}{1.063433in}}%
\pgfpathlineto{\pgfqpoint{0.733335in}{1.072603in}}%
\pgfpathlineto{\pgfqpoint{0.728122in}{1.082759in}}%
\pgfpathlineto{\pgfqpoint{0.723555in}{1.092979in}}%
\pgfpathlineto{\pgfqpoint{0.719636in}{1.103252in}}%
\pgfpathlineto{\pgfqpoint{0.716365in}{1.113567in}}%
\pgfpathlineto{\pgfqpoint{0.720108in}{1.104213in}}%
\pgfpathlineto{\pgfqpoint{0.723834in}{1.095250in}}%
\pgfpathlineto{\pgfqpoint{0.727542in}{1.086671in}}%
\pgfpathlineto{\pgfqpoint{0.731235in}{1.078469in}}%
\pgfpathlineto{\pgfqpoint{0.734452in}{1.068341in}}%
\pgfpathlineto{\pgfqpoint{0.738303in}{1.058255in}}%
\pgfpathlineto{\pgfqpoint{0.742785in}{1.048222in}}%
\pgfpathlineto{\pgfqpoint{0.747897in}{1.038252in}}%
\pgfpathclose%
\pgfusepath{fill}%
\end{pgfscope}%
\begin{pgfscope}%
\pgfpathrectangle{\pgfqpoint{0.329460in}{0.284240in}}{\pgfqpoint{1.989680in}{1.989680in}}%
\pgfusepath{clip}%
\pgfsetbuttcap%
\pgfsetroundjoin%
\definecolor{currentfill}{rgb}{0.487026,0.823929,0.312321}%
\pgfsetfillcolor{currentfill}%
\pgfsetlinewidth{0.000000pt}%
\definecolor{currentstroke}{rgb}{0.000000,0.000000,0.000000}%
\pgfsetstrokecolor{currentstroke}%
\pgfsetdash{}{0pt}%
\pgfpathmoveto{\pgfqpoint{1.461400in}{1.659438in}}%
\pgfpathlineto{\pgfqpoint{1.463642in}{1.655509in}}%
\pgfpathlineto{\pgfqpoint{1.465882in}{1.651502in}}%
\pgfpathlineto{\pgfqpoint{1.468119in}{1.647417in}}%
\pgfpathlineto{\pgfqpoint{1.470353in}{1.643257in}}%
\pgfpathlineto{\pgfqpoint{1.474553in}{1.641447in}}%
\pgfpathlineto{\pgfqpoint{1.478634in}{1.639573in}}%
\pgfpathlineto{\pgfqpoint{1.482591in}{1.637639in}}%
\pgfpathlineto{\pgfqpoint{1.486421in}{1.635647in}}%
\pgfpathlineto{\pgfqpoint{1.483882in}{1.639955in}}%
\pgfpathlineto{\pgfqpoint{1.481341in}{1.644187in}}%
\pgfpathlineto{\pgfqpoint{1.478796in}{1.648342in}}%
\pgfpathlineto{\pgfqpoint{1.476249in}{1.652419in}}%
\pgfpathlineto{\pgfqpoint{1.472710in}{1.654256in}}%
\pgfpathlineto{\pgfqpoint{1.469053in}{1.656040in}}%
\pgfpathlineto{\pgfqpoint{1.465282in}{1.657767in}}%
\pgfpathlineto{\pgfqpoint{1.461400in}{1.659438in}}%
\pgfpathclose%
\pgfusepath{fill}%
\end{pgfscope}%
\begin{pgfscope}%
\pgfpathrectangle{\pgfqpoint{0.329460in}{0.284240in}}{\pgfqpoint{1.989680in}{1.989680in}}%
\pgfusepath{clip}%
\pgfsetbuttcap%
\pgfsetroundjoin%
\definecolor{currentfill}{rgb}{0.274952,0.037752,0.364543}%
\pgfsetfillcolor{currentfill}%
\pgfsetlinewidth{0.000000pt}%
\definecolor{currentstroke}{rgb}{0.000000,0.000000,0.000000}%
\pgfsetstrokecolor{currentstroke}%
\pgfsetdash{}{0pt}%
\pgfpathmoveto{\pgfqpoint{0.944863in}{0.920138in}}%
\pgfpathlineto{\pgfqpoint{0.941964in}{0.916227in}}%
\pgfpathlineto{\pgfqpoint{0.939061in}{0.912481in}}%
\pgfpathlineto{\pgfqpoint{0.936155in}{0.908905in}}%
\pgfpathlineto{\pgfqpoint{0.933245in}{0.905504in}}%
\pgfpathlineto{\pgfqpoint{0.925370in}{0.912590in}}%
\pgfpathlineto{\pgfqpoint{0.917944in}{0.919797in}}%
\pgfpathlineto{\pgfqpoint{0.910975in}{0.927117in}}%
\pgfpathlineto{\pgfqpoint{0.904469in}{0.934542in}}%
\pgfpathlineto{\pgfqpoint{0.907566in}{0.937728in}}%
\pgfpathlineto{\pgfqpoint{0.910659in}{0.941087in}}%
\pgfpathlineto{\pgfqpoint{0.913748in}{0.944615in}}%
\pgfpathlineto{\pgfqpoint{0.916834in}{0.948310in}}%
\pgfpathlineto{\pgfqpoint{0.923174in}{0.941106in}}%
\pgfpathlineto{\pgfqpoint{0.929963in}{0.934005in}}%
\pgfpathlineto{\pgfqpoint{0.937194in}{0.927013in}}%
\pgfpathlineto{\pgfqpoint{0.944863in}{0.920138in}}%
\pgfpathclose%
\pgfusepath{fill}%
\end{pgfscope}%
\begin{pgfscope}%
\pgfpathrectangle{\pgfqpoint{0.329460in}{0.284240in}}{\pgfqpoint{1.989680in}{1.989680in}}%
\pgfusepath{clip}%
\pgfsetbuttcap%
\pgfsetroundjoin%
\definecolor{currentfill}{rgb}{0.271305,0.019942,0.347269}%
\pgfsetfillcolor{currentfill}%
\pgfsetlinewidth{0.000000pt}%
\definecolor{currentstroke}{rgb}{0.000000,0.000000,0.000000}%
\pgfsetstrokecolor{currentstroke}%
\pgfsetdash{}{0pt}%
\pgfpathmoveto{\pgfqpoint{0.933245in}{0.905504in}}%
\pgfpathlineto{\pgfqpoint{0.930330in}{0.902280in}}%
\pgfpathlineto{\pgfqpoint{0.927411in}{0.899238in}}%
\pgfpathlineto{\pgfqpoint{0.924487in}{0.896381in}}%
\pgfpathlineto{\pgfqpoint{0.921559in}{0.893714in}}%
\pgfpathlineto{\pgfqpoint{0.913477in}{0.901012in}}%
\pgfpathlineto{\pgfqpoint{0.905858in}{0.908433in}}%
\pgfpathlineto{\pgfqpoint{0.898708in}{0.915970in}}%
\pgfpathlineto{\pgfqpoint{0.892034in}{0.923614in}}%
\pgfpathlineto{\pgfqpoint{0.895150in}{0.926066in}}%
\pgfpathlineto{\pgfqpoint{0.898261in}{0.928707in}}%
\pgfpathlineto{\pgfqpoint{0.901367in}{0.931534in}}%
\pgfpathlineto{\pgfqpoint{0.904469in}{0.934542in}}%
\pgfpathlineto{\pgfqpoint{0.910975in}{0.927117in}}%
\pgfpathlineto{\pgfqpoint{0.917944in}{0.919797in}}%
\pgfpathlineto{\pgfqpoint{0.925370in}{0.912590in}}%
\pgfpathlineto{\pgfqpoint{0.933245in}{0.905504in}}%
\pgfpathclose%
\pgfusepath{fill}%
\end{pgfscope}%
\begin{pgfscope}%
\pgfpathrectangle{\pgfqpoint{0.329460in}{0.284240in}}{\pgfqpoint{1.989680in}{1.989680in}}%
\pgfusepath{clip}%
\pgfsetbuttcap%
\pgfsetroundjoin%
\definecolor{currentfill}{rgb}{0.565498,0.842430,0.262877}%
\pgfsetfillcolor{currentfill}%
\pgfsetlinewidth{0.000000pt}%
\definecolor{currentstroke}{rgb}{0.000000,0.000000,0.000000}%
\pgfsetstrokecolor{currentstroke}%
\pgfsetdash{}{0pt}%
\pgfpathmoveto{\pgfqpoint{1.261664in}{1.678743in}}%
\pgfpathlineto{\pgfqpoint{1.259673in}{1.675245in}}%
\pgfpathlineto{\pgfqpoint{1.257684in}{1.671662in}}%
\pgfpathlineto{\pgfqpoint{1.255698in}{1.667997in}}%
\pgfpathlineto{\pgfqpoint{1.253714in}{1.664249in}}%
\pgfpathlineto{\pgfqpoint{1.258008in}{1.665667in}}%
\pgfpathlineto{\pgfqpoint{1.262392in}{1.667020in}}%
\pgfpathlineto{\pgfqpoint{1.266861in}{1.668307in}}%
\pgfpathlineto{\pgfqpoint{1.271412in}{1.669527in}}%
\pgfpathlineto{\pgfqpoint{1.273034in}{1.673164in}}%
\pgfpathlineto{\pgfqpoint{1.274658in}{1.676719in}}%
\pgfpathlineto{\pgfqpoint{1.276284in}{1.680191in}}%
\pgfpathlineto{\pgfqpoint{1.277912in}{1.683579in}}%
\pgfpathlineto{\pgfqpoint{1.273734in}{1.682461in}}%
\pgfpathlineto{\pgfqpoint{1.269630in}{1.681282in}}%
\pgfpathlineto{\pgfqpoint{1.265606in}{1.680042in}}%
\pgfpathlineto{\pgfqpoint{1.261664in}{1.678743in}}%
\pgfpathclose%
\pgfusepath{fill}%
\end{pgfscope}%
\begin{pgfscope}%
\pgfpathrectangle{\pgfqpoint{0.329460in}{0.284240in}}{\pgfqpoint{1.989680in}{1.989680in}}%
\pgfusepath{clip}%
\pgfsetbuttcap%
\pgfsetroundjoin%
\definecolor{currentfill}{rgb}{0.267004,0.004874,0.329415}%
\pgfsetfillcolor{currentfill}%
\pgfsetlinewidth{0.000000pt}%
\definecolor{currentstroke}{rgb}{0.000000,0.000000,0.000000}%
\pgfsetstrokecolor{currentstroke}%
\pgfsetdash{}{0pt}%
\pgfpathmoveto{\pgfqpoint{1.841271in}{0.918584in}}%
\pgfpathlineto{\pgfqpoint{1.844474in}{0.918074in}}%
\pgfpathlineto{\pgfqpoint{1.847685in}{0.917795in}}%
\pgfpathlineto{\pgfqpoint{1.850903in}{0.917752in}}%
\pgfpathlineto{\pgfqpoint{1.854129in}{0.917950in}}%
\pgfpathlineto{\pgfqpoint{1.847416in}{0.909565in}}%
\pgfpathlineto{\pgfqpoint{1.840181in}{0.901285in}}%
\pgfpathlineto{\pgfqpoint{1.832430in}{0.893122in}}%
\pgfpathlineto{\pgfqpoint{1.824169in}{0.885083in}}%
\pgfpathlineto{\pgfqpoint{1.821120in}{0.885096in}}%
\pgfpathlineto{\pgfqpoint{1.818079in}{0.885350in}}%
\pgfpathlineto{\pgfqpoint{1.815045in}{0.885841in}}%
\pgfpathlineto{\pgfqpoint{1.812018in}{0.886565in}}%
\pgfpathlineto{\pgfqpoint{1.820081in}{0.894395in}}%
\pgfpathlineto{\pgfqpoint{1.827648in}{0.902348in}}%
\pgfpathlineto{\pgfqpoint{1.834713in}{0.910414in}}%
\pgfpathlineto{\pgfqpoint{1.841271in}{0.918584in}}%
\pgfpathclose%
\pgfusepath{fill}%
\end{pgfscope}%
\begin{pgfscope}%
\pgfpathrectangle{\pgfqpoint{0.329460in}{0.284240in}}{\pgfqpoint{1.989680in}{1.989680in}}%
\pgfusepath{clip}%
\pgfsetbuttcap%
\pgfsetroundjoin%
\definecolor{currentfill}{rgb}{0.279566,0.067836,0.391917}%
\pgfsetfillcolor{currentfill}%
\pgfsetlinewidth{0.000000pt}%
\definecolor{currentstroke}{rgb}{0.000000,0.000000,0.000000}%
\pgfsetstrokecolor{currentstroke}%
\pgfsetdash{}{0pt}%
\pgfpathmoveto{\pgfqpoint{0.956425in}{0.937369in}}%
\pgfpathlineto{\pgfqpoint{0.953539in}{0.932831in}}%
\pgfpathlineto{\pgfqpoint{0.950650in}{0.928444in}}%
\pgfpathlineto{\pgfqpoint{0.947758in}{0.924212in}}%
\pgfpathlineto{\pgfqpoint{0.944863in}{0.920138in}}%
\pgfpathlineto{\pgfqpoint{0.937194in}{0.927013in}}%
\pgfpathlineto{\pgfqpoint{0.929963in}{0.934005in}}%
\pgfpathlineto{\pgfqpoint{0.923174in}{0.941106in}}%
\pgfpathlineto{\pgfqpoint{0.916834in}{0.948310in}}%
\pgfpathlineto{\pgfqpoint{0.919916in}{0.952167in}}%
\pgfpathlineto{\pgfqpoint{0.922994in}{0.956182in}}%
\pgfpathlineto{\pgfqpoint{0.926069in}{0.960351in}}%
\pgfpathlineto{\pgfqpoint{0.929142in}{0.964671in}}%
\pgfpathlineto{\pgfqpoint{0.935315in}{0.957689in}}%
\pgfpathlineto{\pgfqpoint{0.941924in}{0.950806in}}%
\pgfpathlineto{\pgfqpoint{0.948962in}{0.944030in}}%
\pgfpathlineto{\pgfqpoint{0.956425in}{0.937369in}}%
\pgfpathclose%
\pgfusepath{fill}%
\end{pgfscope}%
\begin{pgfscope}%
\pgfpathrectangle{\pgfqpoint{0.329460in}{0.284240in}}{\pgfqpoint{1.989680in}{1.989680in}}%
\pgfusepath{clip}%
\pgfsetbuttcap%
\pgfsetroundjoin%
\definecolor{currentfill}{rgb}{0.120081,0.622161,0.534946}%
\pgfsetfillcolor{currentfill}%
\pgfsetlinewidth{0.000000pt}%
\definecolor{currentstroke}{rgb}{0.000000,0.000000,0.000000}%
\pgfsetstrokecolor{currentstroke}%
\pgfsetdash{}{0pt}%
\pgfpathmoveto{\pgfqpoint{1.612309in}{1.441878in}}%
\pgfpathlineto{\pgfqpoint{1.615644in}{1.435293in}}%
\pgfpathlineto{\pgfqpoint{1.618976in}{1.428683in}}%
\pgfpathlineto{\pgfqpoint{1.622304in}{1.422049in}}%
\pgfpathlineto{\pgfqpoint{1.625630in}{1.415394in}}%
\pgfpathlineto{\pgfqpoint{1.626425in}{1.411130in}}%
\pgfpathlineto{\pgfqpoint{1.626946in}{1.406853in}}%
\pgfpathlineto{\pgfqpoint{1.627193in}{1.402566in}}%
\pgfpathlineto{\pgfqpoint{1.627165in}{1.398275in}}%
\pgfpathlineto{\pgfqpoint{1.623811in}{1.405156in}}%
\pgfpathlineto{\pgfqpoint{1.620455in}{1.412016in}}%
\pgfpathlineto{\pgfqpoint{1.617096in}{1.418852in}}%
\pgfpathlineto{\pgfqpoint{1.613734in}{1.425662in}}%
\pgfpathlineto{\pgfqpoint{1.613769in}{1.429727in}}%
\pgfpathlineto{\pgfqpoint{1.613543in}{1.433787in}}%
\pgfpathlineto{\pgfqpoint{1.613056in}{1.437838in}}%
\pgfpathlineto{\pgfqpoint{1.612309in}{1.441878in}}%
\pgfpathclose%
\pgfusepath{fill}%
\end{pgfscope}%
\begin{pgfscope}%
\pgfpathrectangle{\pgfqpoint{0.329460in}{0.284240in}}{\pgfqpoint{1.989680in}{1.989680in}}%
\pgfusepath{clip}%
\pgfsetbuttcap%
\pgfsetroundjoin%
\definecolor{currentfill}{rgb}{0.133743,0.548535,0.553541}%
\pgfsetfillcolor{currentfill}%
\pgfsetlinewidth{0.000000pt}%
\definecolor{currentstroke}{rgb}{0.000000,0.000000,0.000000}%
\pgfsetstrokecolor{currentstroke}%
\pgfsetdash{}{0pt}%
\pgfpathmoveto{\pgfqpoint{1.640553in}{1.370585in}}%
\pgfpathlineto{\pgfqpoint{1.643893in}{1.363633in}}%
\pgfpathlineto{\pgfqpoint{1.647231in}{1.356673in}}%
\pgfpathlineto{\pgfqpoint{1.650566in}{1.349710in}}%
\pgfpathlineto{\pgfqpoint{1.653899in}{1.342744in}}%
\pgfpathlineto{\pgfqpoint{1.653586in}{1.337994in}}%
\pgfpathlineto{\pgfqpoint{1.652971in}{1.333247in}}%
\pgfpathlineto{\pgfqpoint{1.652053in}{1.328509in}}%
\pgfpathlineto{\pgfqpoint{1.650833in}{1.323784in}}%
\pgfpathlineto{\pgfqpoint{1.647524in}{1.330980in}}%
\pgfpathlineto{\pgfqpoint{1.644213in}{1.338174in}}%
\pgfpathlineto{\pgfqpoint{1.640899in}{1.345363in}}%
\pgfpathlineto{\pgfqpoint{1.637583in}{1.352544in}}%
\pgfpathlineto{\pgfqpoint{1.638759in}{1.357040in}}%
\pgfpathlineto{\pgfqpoint{1.639646in}{1.361548in}}%
\pgfpathlineto{\pgfqpoint{1.640244in}{1.366064in}}%
\pgfpathlineto{\pgfqpoint{1.640553in}{1.370585in}}%
\pgfpathclose%
\pgfusepath{fill}%
\end{pgfscope}%
\begin{pgfscope}%
\pgfpathrectangle{\pgfqpoint{0.329460in}{0.284240in}}{\pgfqpoint{1.989680in}{1.989680in}}%
\pgfusepath{clip}%
\pgfsetbuttcap%
\pgfsetroundjoin%
\definecolor{currentfill}{rgb}{0.636902,0.856542,0.216620}%
\pgfsetfillcolor{currentfill}%
\pgfsetlinewidth{0.000000pt}%
\definecolor{currentstroke}{rgb}{0.000000,0.000000,0.000000}%
\pgfsetstrokecolor{currentstroke}%
\pgfsetdash{}{0pt}%
\pgfpathmoveto{\pgfqpoint{1.334024in}{1.703784in}}%
\pgfpathlineto{\pgfqpoint{1.333604in}{1.700938in}}%
\pgfpathlineto{\pgfqpoint{1.333185in}{1.698001in}}%
\pgfpathlineto{\pgfqpoint{1.332766in}{1.694976in}}%
\pgfpathlineto{\pgfqpoint{1.332347in}{1.691863in}}%
\pgfpathlineto{\pgfqpoint{1.337108in}{1.692107in}}%
\pgfpathlineto{\pgfqpoint{1.341882in}{1.692280in}}%
\pgfpathlineto{\pgfqpoint{1.346666in}{1.692383in}}%
\pgfpathlineto{\pgfqpoint{1.351454in}{1.692414in}}%
\pgfpathlineto{\pgfqpoint{1.351448in}{1.695514in}}%
\pgfpathlineto{\pgfqpoint{1.351442in}{1.698527in}}%
\pgfpathlineto{\pgfqpoint{1.351436in}{1.701451in}}%
\pgfpathlineto{\pgfqpoint{1.351430in}{1.704284in}}%
\pgfpathlineto{\pgfqpoint{1.347068in}{1.704256in}}%
\pgfpathlineto{\pgfqpoint{1.342711in}{1.704162in}}%
\pgfpathlineto{\pgfqpoint{1.338361in}{1.704005in}}%
\pgfpathlineto{\pgfqpoint{1.334024in}{1.703784in}}%
\pgfpathclose%
\pgfusepath{fill}%
\end{pgfscope}%
\begin{pgfscope}%
\pgfpathrectangle{\pgfqpoint{0.329460in}{0.284240in}}{\pgfqpoint{1.989680in}{1.989680in}}%
\pgfusepath{clip}%
\pgfsetbuttcap%
\pgfsetroundjoin%
\definecolor{currentfill}{rgb}{0.636902,0.856542,0.216620}%
\pgfsetfillcolor{currentfill}%
\pgfsetlinewidth{0.000000pt}%
\definecolor{currentstroke}{rgb}{0.000000,0.000000,0.000000}%
\pgfsetstrokecolor{currentstroke}%
\pgfsetdash{}{0pt}%
\pgfpathmoveto{\pgfqpoint{1.351430in}{1.704284in}}%
\pgfpathlineto{\pgfqpoint{1.351436in}{1.701451in}}%
\pgfpathlineto{\pgfqpoint{1.351442in}{1.698527in}}%
\pgfpathlineto{\pgfqpoint{1.351448in}{1.695514in}}%
\pgfpathlineto{\pgfqpoint{1.351454in}{1.692414in}}%
\pgfpathlineto{\pgfqpoint{1.356241in}{1.692375in}}%
\pgfpathlineto{\pgfqpoint{1.361024in}{1.692265in}}%
\pgfpathlineto{\pgfqpoint{1.365797in}{1.692084in}}%
\pgfpathlineto{\pgfqpoint{1.370556in}{1.691832in}}%
\pgfpathlineto{\pgfqpoint{1.370126in}{1.694945in}}%
\pgfpathlineto{\pgfqpoint{1.369695in}{1.697971in}}%
\pgfpathlineto{\pgfqpoint{1.369264in}{1.700908in}}%
\pgfpathlineto{\pgfqpoint{1.368832in}{1.703755in}}%
\pgfpathlineto{\pgfqpoint{1.364496in}{1.703984in}}%
\pgfpathlineto{\pgfqpoint{1.360148in}{1.704148in}}%
\pgfpathlineto{\pgfqpoint{1.355791in}{1.704248in}}%
\pgfpathlineto{\pgfqpoint{1.351430in}{1.704284in}}%
\pgfpathclose%
\pgfusepath{fill}%
\end{pgfscope}%
\begin{pgfscope}%
\pgfpathrectangle{\pgfqpoint{0.329460in}{0.284240in}}{\pgfqpoint{1.989680in}{1.989680in}}%
\pgfusepath{clip}%
\pgfsetbuttcap%
\pgfsetroundjoin%
\definecolor{currentfill}{rgb}{0.179019,0.433756,0.557430}%
\pgfsetfillcolor{currentfill}%
\pgfsetlinewidth{0.000000pt}%
\definecolor{currentstroke}{rgb}{0.000000,0.000000,0.000000}%
\pgfsetstrokecolor{currentstroke}%
\pgfsetdash{}{0pt}%
\pgfpathmoveto{\pgfqpoint{1.036245in}{1.241443in}}%
\pgfpathlineto{\pgfqpoint{1.033053in}{1.234060in}}%
\pgfpathlineto{\pgfqpoint{1.029863in}{1.226700in}}%
\pgfpathlineto{\pgfqpoint{1.026674in}{1.219366in}}%
\pgfpathlineto{\pgfqpoint{1.023487in}{1.212062in}}%
\pgfpathlineto{\pgfqpoint{1.020477in}{1.217336in}}%
\pgfpathlineto{\pgfqpoint{1.017805in}{1.222651in}}%
\pgfpathlineto{\pgfqpoint{1.015471in}{1.228001in}}%
\pgfpathlineto{\pgfqpoint{1.013478in}{1.233382in}}%
\pgfpathlineto{\pgfqpoint{1.016752in}{1.240455in}}%
\pgfpathlineto{\pgfqpoint{1.020027in}{1.247558in}}%
\pgfpathlineto{\pgfqpoint{1.023304in}{1.254687in}}%
\pgfpathlineto{\pgfqpoint{1.026583in}{1.261841in}}%
\pgfpathlineto{\pgfqpoint{1.028510in}{1.256692in}}%
\pgfpathlineto{\pgfqpoint{1.030763in}{1.251573in}}%
\pgfpathlineto{\pgfqpoint{1.033342in}{1.246488in}}%
\pgfpathlineto{\pgfqpoint{1.036245in}{1.241443in}}%
\pgfpathclose%
\pgfusepath{fill}%
\end{pgfscope}%
\begin{pgfscope}%
\pgfpathrectangle{\pgfqpoint{0.329460in}{0.284240in}}{\pgfqpoint{1.989680in}{1.989680in}}%
\pgfusepath{clip}%
\pgfsetbuttcap%
\pgfsetroundjoin%
\definecolor{currentfill}{rgb}{0.282327,0.094955,0.417331}%
\pgfsetfillcolor{currentfill}%
\pgfsetlinewidth{0.000000pt}%
\definecolor{currentstroke}{rgb}{0.000000,0.000000,0.000000}%
\pgfsetstrokecolor{currentstroke}%
\pgfsetdash{}{0pt}%
\pgfpathmoveto{\pgfqpoint{0.815205in}{0.932824in}}%
\pgfpathlineto{\pgfqpoint{0.811890in}{0.936546in}}%
\pgfpathlineto{\pgfqpoint{0.808564in}{0.940578in}}%
\pgfpathlineto{\pgfqpoint{0.805226in}{0.944927in}}%
\pgfpathlineto{\pgfqpoint{0.801875in}{0.949597in}}%
\pgfpathlineto{\pgfqpoint{0.794594in}{0.958822in}}%
\pgfpathlineto{\pgfqpoint{0.787899in}{0.968150in}}%
\pgfpathlineto{\pgfqpoint{0.781794in}{0.977571in}}%
\pgfpathlineto{\pgfqpoint{0.776283in}{0.987075in}}%
\pgfpathlineto{\pgfqpoint{0.779770in}{0.982204in}}%
\pgfpathlineto{\pgfqpoint{0.783244in}{0.977654in}}%
\pgfpathlineto{\pgfqpoint{0.786706in}{0.973418in}}%
\pgfpathlineto{\pgfqpoint{0.790156in}{0.969491in}}%
\pgfpathlineto{\pgfqpoint{0.795554in}{0.960192in}}%
\pgfpathlineto{\pgfqpoint{0.801530in}{0.950974in}}%
\pgfpathlineto{\pgfqpoint{0.808082in}{0.941848in}}%
\pgfpathlineto{\pgfqpoint{0.815205in}{0.932824in}}%
\pgfpathclose%
\pgfusepath{fill}%
\end{pgfscope}%
\begin{pgfscope}%
\pgfpathrectangle{\pgfqpoint{0.329460in}{0.284240in}}{\pgfqpoint{1.989680in}{1.989680in}}%
\pgfusepath{clip}%
\pgfsetbuttcap%
\pgfsetroundjoin%
\definecolor{currentfill}{rgb}{0.268510,0.009605,0.335427}%
\pgfsetfillcolor{currentfill}%
\pgfsetlinewidth{0.000000pt}%
\definecolor{currentstroke}{rgb}{0.000000,0.000000,0.000000}%
\pgfsetstrokecolor{currentstroke}%
\pgfsetdash{}{0pt}%
\pgfpathmoveto{\pgfqpoint{0.921559in}{0.893714in}}%
\pgfpathlineto{\pgfqpoint{0.918625in}{0.891242in}}%
\pgfpathlineto{\pgfqpoint{0.915687in}{0.888967in}}%
\pgfpathlineto{\pgfqpoint{0.912743in}{0.886895in}}%
\pgfpathlineto{\pgfqpoint{0.909793in}{0.885029in}}%
\pgfpathlineto{\pgfqpoint{0.901503in}{0.892536in}}%
\pgfpathlineto{\pgfqpoint{0.893690in}{0.900169in}}%
\pgfpathlineto{\pgfqpoint{0.886360in}{0.907922in}}%
\pgfpathlineto{\pgfqpoint{0.879519in}{0.915784in}}%
\pgfpathlineto{\pgfqpoint{0.882656in}{0.917437in}}%
\pgfpathlineto{\pgfqpoint{0.885788in}{0.919295in}}%
\pgfpathlineto{\pgfqpoint{0.888914in}{0.921356in}}%
\pgfpathlineto{\pgfqpoint{0.892034in}{0.923614in}}%
\pgfpathlineto{\pgfqpoint{0.898708in}{0.915970in}}%
\pgfpathlineto{\pgfqpoint{0.905858in}{0.908433in}}%
\pgfpathlineto{\pgfqpoint{0.913477in}{0.901012in}}%
\pgfpathlineto{\pgfqpoint{0.921559in}{0.893714in}}%
\pgfpathclose%
\pgfusepath{fill}%
\end{pgfscope}%
\begin{pgfscope}%
\pgfpathrectangle{\pgfqpoint{0.329460in}{0.284240in}}{\pgfqpoint{1.989680in}{1.989680in}}%
\pgfusepath{clip}%
\pgfsetbuttcap%
\pgfsetroundjoin%
\definecolor{currentfill}{rgb}{0.487026,0.823929,0.312321}%
\pgfsetfillcolor{currentfill}%
\pgfsetlinewidth{0.000000pt}%
\definecolor{currentstroke}{rgb}{0.000000,0.000000,0.000000}%
\pgfsetstrokecolor{currentstroke}%
\pgfsetdash{}{0pt}%
\pgfpathmoveto{\pgfqpoint{1.223082in}{1.650741in}}%
\pgfpathlineto{\pgfqpoint{1.220472in}{1.646629in}}%
\pgfpathlineto{\pgfqpoint{1.217865in}{1.642439in}}%
\pgfpathlineto{\pgfqpoint{1.215260in}{1.638171in}}%
\pgfpathlineto{\pgfqpoint{1.212659in}{1.633827in}}%
\pgfpathlineto{\pgfqpoint{1.216373in}{1.635871in}}%
\pgfpathlineto{\pgfqpoint{1.220218in}{1.637857in}}%
\pgfpathlineto{\pgfqpoint{1.224189in}{1.639785in}}%
\pgfpathlineto{\pgfqpoint{1.228283in}{1.641651in}}%
\pgfpathlineto{\pgfqpoint{1.230588in}{1.645842in}}%
\pgfpathlineto{\pgfqpoint{1.232896in}{1.649957in}}%
\pgfpathlineto{\pgfqpoint{1.235206in}{1.653996in}}%
\pgfpathlineto{\pgfqpoint{1.237519in}{1.657956in}}%
\pgfpathlineto{\pgfqpoint{1.233736in}{1.656235in}}%
\pgfpathlineto{\pgfqpoint{1.230066in}{1.654457in}}%
\pgfpathlineto{\pgfqpoint{1.226514in}{1.652625in}}%
\pgfpathlineto{\pgfqpoint{1.223082in}{1.650741in}}%
\pgfpathclose%
\pgfusepath{fill}%
\end{pgfscope}%
\begin{pgfscope}%
\pgfpathrectangle{\pgfqpoint{0.329460in}{0.284240in}}{\pgfqpoint{1.989680in}{1.989680in}}%
\pgfusepath{clip}%
\pgfsetbuttcap%
\pgfsetroundjoin%
\definecolor{currentfill}{rgb}{0.280255,0.165693,0.476498}%
\pgfsetfillcolor{currentfill}%
\pgfsetlinewidth{0.000000pt}%
\definecolor{currentstroke}{rgb}{0.000000,0.000000,0.000000}%
\pgfsetstrokecolor{currentstroke}%
\pgfsetdash{}{0pt}%
\pgfpathmoveto{\pgfqpoint{1.741256in}{1.032640in}}%
\pgfpathlineto{\pgfqpoint{1.744338in}{1.026840in}}%
\pgfpathlineto{\pgfqpoint{1.747420in}{1.021149in}}%
\pgfpathlineto{\pgfqpoint{1.750505in}{1.015570in}}%
\pgfpathlineto{\pgfqpoint{1.753591in}{1.010106in}}%
\pgfpathlineto{\pgfqpoint{1.748119in}{1.003491in}}%
\pgfpathlineto{\pgfqpoint{1.742234in}{0.996965in}}%
\pgfpathlineto{\pgfqpoint{1.735940in}{0.990533in}}%
\pgfpathlineto{\pgfqpoint{1.729244in}{0.984202in}}%
\pgfpathlineto{\pgfqpoint{1.726333in}{0.989887in}}%
\pgfpathlineto{\pgfqpoint{1.723424in}{0.995688in}}%
\pgfpathlineto{\pgfqpoint{1.720516in}{1.001600in}}%
\pgfpathlineto{\pgfqpoint{1.717610in}{1.007620in}}%
\pgfpathlineto{\pgfqpoint{1.724112in}{1.013734in}}%
\pgfpathlineto{\pgfqpoint{1.730223in}{1.019947in}}%
\pgfpathlineto{\pgfqpoint{1.735940in}{1.026251in}}%
\pgfpathlineto{\pgfqpoint{1.741256in}{1.032640in}}%
\pgfpathclose%
\pgfusepath{fill}%
\end{pgfscope}%
\begin{pgfscope}%
\pgfpathrectangle{\pgfqpoint{0.329460in}{0.284240in}}{\pgfqpoint{1.989680in}{1.989680in}}%
\pgfusepath{clip}%
\pgfsetbuttcap%
\pgfsetroundjoin%
\definecolor{currentfill}{rgb}{0.220124,0.725509,0.466226}%
\pgfsetfillcolor{currentfill}%
\pgfsetlinewidth{0.000000pt}%
\definecolor{currentstroke}{rgb}{0.000000,0.000000,0.000000}%
\pgfsetstrokecolor{currentstroke}%
\pgfsetdash{}{0pt}%
\pgfpathmoveto{\pgfqpoint{1.133609in}{1.528496in}}%
\pgfpathlineto{\pgfqpoint{1.130294in}{1.522557in}}%
\pgfpathlineto{\pgfqpoint{1.126982in}{1.516568in}}%
\pgfpathlineto{\pgfqpoint{1.123673in}{1.510530in}}%
\pgfpathlineto{\pgfqpoint{1.120368in}{1.504444in}}%
\pgfpathlineto{\pgfqpoint{1.121968in}{1.507959in}}%
\pgfpathlineto{\pgfqpoint{1.123793in}{1.511445in}}%
\pgfpathlineto{\pgfqpoint{1.125842in}{1.514900in}}%
\pgfpathlineto{\pgfqpoint{1.128112in}{1.518320in}}%
\pgfpathlineto{\pgfqpoint{1.131299in}{1.524193in}}%
\pgfpathlineto{\pgfqpoint{1.134490in}{1.530019in}}%
\pgfpathlineto{\pgfqpoint{1.137684in}{1.535797in}}%
\pgfpathlineto{\pgfqpoint{1.140882in}{1.541524in}}%
\pgfpathlineto{\pgfqpoint{1.138749in}{1.538313in}}%
\pgfpathlineto{\pgfqpoint{1.136824in}{1.535069in}}%
\pgfpathlineto{\pgfqpoint{1.135110in}{1.531795in}}%
\pgfpathlineto{\pgfqpoint{1.133609in}{1.528496in}}%
\pgfpathclose%
\pgfusepath{fill}%
\end{pgfscope}%
\begin{pgfscope}%
\pgfpathrectangle{\pgfqpoint{0.329460in}{0.284240in}}{\pgfqpoint{1.989680in}{1.989680in}}%
\pgfusepath{clip}%
\pgfsetbuttcap%
\pgfsetroundjoin%
\definecolor{currentfill}{rgb}{0.282327,0.094955,0.417331}%
\pgfsetfillcolor{currentfill}%
\pgfsetlinewidth{0.000000pt}%
\definecolor{currentstroke}{rgb}{0.000000,0.000000,0.000000}%
\pgfsetstrokecolor{currentstroke}%
\pgfsetdash{}{0pt}%
\pgfpathmoveto{\pgfqpoint{0.967940in}{0.956955in}}%
\pgfpathlineto{\pgfqpoint{0.965065in}{0.951851in}}%
\pgfpathlineto{\pgfqpoint{0.962187in}{0.946882in}}%
\pgfpathlineto{\pgfqpoint{0.959307in}{0.942054in}}%
\pgfpathlineto{\pgfqpoint{0.956425in}{0.937369in}}%
\pgfpathlineto{\pgfqpoint{0.948962in}{0.944030in}}%
\pgfpathlineto{\pgfqpoint{0.941924in}{0.950806in}}%
\pgfpathlineto{\pgfqpoint{0.935315in}{0.957689in}}%
\pgfpathlineto{\pgfqpoint{0.929142in}{0.964671in}}%
\pgfpathlineto{\pgfqpoint{0.932211in}{0.969138in}}%
\pgfpathlineto{\pgfqpoint{0.935278in}{0.973749in}}%
\pgfpathlineto{\pgfqpoint{0.938342in}{0.978499in}}%
\pgfpathlineto{\pgfqpoint{0.941403in}{0.983385in}}%
\pgfpathlineto{\pgfqpoint{0.947410in}{0.976626in}}%
\pgfpathlineto{\pgfqpoint{0.953838in}{0.969963in}}%
\pgfpathlineto{\pgfqpoint{0.960684in}{0.963404in}}%
\pgfpathlineto{\pgfqpoint{0.967940in}{0.956955in}}%
\pgfpathclose%
\pgfusepath{fill}%
\end{pgfscope}%
\begin{pgfscope}%
\pgfpathrectangle{\pgfqpoint{0.329460in}{0.284240in}}{\pgfqpoint{1.989680in}{1.989680in}}%
\pgfusepath{clip}%
\pgfsetbuttcap%
\pgfsetroundjoin%
\definecolor{currentfill}{rgb}{0.636902,0.856542,0.216620}%
\pgfsetfillcolor{currentfill}%
\pgfsetlinewidth{0.000000pt}%
\definecolor{currentstroke}{rgb}{0.000000,0.000000,0.000000}%
\pgfsetstrokecolor{currentstroke}%
\pgfsetdash{}{0pt}%
\pgfpathmoveto{\pgfqpoint{1.316883in}{1.702260in}}%
\pgfpathlineto{\pgfqpoint{1.316043in}{1.699376in}}%
\pgfpathlineto{\pgfqpoint{1.315204in}{1.696402in}}%
\pgfpathlineto{\pgfqpoint{1.314366in}{1.693339in}}%
\pgfpathlineto{\pgfqpoint{1.313530in}{1.690188in}}%
\pgfpathlineto{\pgfqpoint{1.318191in}{1.690711in}}%
\pgfpathlineto{\pgfqpoint{1.322884in}{1.691165in}}%
\pgfpathlineto{\pgfqpoint{1.327604in}{1.691549in}}%
\pgfpathlineto{\pgfqpoint{1.332347in}{1.691863in}}%
\pgfpathlineto{\pgfqpoint{1.332766in}{1.694976in}}%
\pgfpathlineto{\pgfqpoint{1.333185in}{1.698001in}}%
\pgfpathlineto{\pgfqpoint{1.333604in}{1.700938in}}%
\pgfpathlineto{\pgfqpoint{1.334024in}{1.703784in}}%
\pgfpathlineto{\pgfqpoint{1.329704in}{1.703498in}}%
\pgfpathlineto{\pgfqpoint{1.325404in}{1.703149in}}%
\pgfpathlineto{\pgfqpoint{1.321129in}{1.702736in}}%
\pgfpathlineto{\pgfqpoint{1.316883in}{1.702260in}}%
\pgfpathclose%
\pgfusepath{fill}%
\end{pgfscope}%
\begin{pgfscope}%
\pgfpathrectangle{\pgfqpoint{0.329460in}{0.284240in}}{\pgfqpoint{1.989680in}{1.989680in}}%
\pgfusepath{clip}%
\pgfsetbuttcap%
\pgfsetroundjoin%
\definecolor{currentfill}{rgb}{0.636902,0.856542,0.216620}%
\pgfsetfillcolor{currentfill}%
\pgfsetlinewidth{0.000000pt}%
\definecolor{currentstroke}{rgb}{0.000000,0.000000,0.000000}%
\pgfsetstrokecolor{currentstroke}%
\pgfsetdash{}{0pt}%
\pgfpathmoveto{\pgfqpoint{1.368832in}{1.703755in}}%
\pgfpathlineto{\pgfqpoint{1.369264in}{1.700908in}}%
\pgfpathlineto{\pgfqpoint{1.369695in}{1.697971in}}%
\pgfpathlineto{\pgfqpoint{1.370126in}{1.694945in}}%
\pgfpathlineto{\pgfqpoint{1.370556in}{1.691832in}}%
\pgfpathlineto{\pgfqpoint{1.375297in}{1.691510in}}%
\pgfpathlineto{\pgfqpoint{1.380014in}{1.691118in}}%
\pgfpathlineto{\pgfqpoint{1.384704in}{1.690657in}}%
\pgfpathlineto{\pgfqpoint{1.389361in}{1.690126in}}%
\pgfpathlineto{\pgfqpoint{1.388513in}{1.693278in}}%
\pgfpathlineto{\pgfqpoint{1.387664in}{1.696342in}}%
\pgfpathlineto{\pgfqpoint{1.386813in}{1.699318in}}%
\pgfpathlineto{\pgfqpoint{1.385962in}{1.702204in}}%
\pgfpathlineto{\pgfqpoint{1.381719in}{1.702686in}}%
\pgfpathlineto{\pgfqpoint{1.377447in}{1.703106in}}%
\pgfpathlineto{\pgfqpoint{1.373150in}{1.703462in}}%
\pgfpathlineto{\pgfqpoint{1.368832in}{1.703755in}}%
\pgfpathclose%
\pgfusepath{fill}%
\end{pgfscope}%
\begin{pgfscope}%
\pgfpathrectangle{\pgfqpoint{0.329460in}{0.284240in}}{\pgfqpoint{1.989680in}{1.989680in}}%
\pgfusepath{clip}%
\pgfsetbuttcap%
\pgfsetroundjoin%
\definecolor{currentfill}{rgb}{0.231674,0.318106,0.544834}%
\pgfsetfillcolor{currentfill}%
\pgfsetlinewidth{0.000000pt}%
\definecolor{currentstroke}{rgb}{0.000000,0.000000,0.000000}%
\pgfsetstrokecolor{currentstroke}%
\pgfsetdash{}{0pt}%
\pgfpathmoveto{\pgfqpoint{1.014494in}{1.132623in}}%
\pgfpathlineto{\pgfqpoint{1.011453in}{1.125480in}}%
\pgfpathlineto{\pgfqpoint{1.008412in}{1.118394in}}%
\pgfpathlineto{\pgfqpoint{1.005371in}{1.111369in}}%
\pgfpathlineto{\pgfqpoint{1.002331in}{1.104407in}}%
\pgfpathlineto{\pgfqpoint{0.997526in}{1.110130in}}%
\pgfpathlineto{\pgfqpoint{0.993086in}{1.115924in}}%
\pgfpathlineto{\pgfqpoint{0.989015in}{1.121781in}}%
\pgfpathlineto{\pgfqpoint{0.985316in}{1.127696in}}%
\pgfpathlineto{\pgfqpoint{0.988493in}{1.134430in}}%
\pgfpathlineto{\pgfqpoint{0.991671in}{1.141229in}}%
\pgfpathlineto{\pgfqpoint{0.994849in}{1.148087in}}%
\pgfpathlineto{\pgfqpoint{0.998027in}{1.155004in}}%
\pgfpathlineto{\pgfqpoint{1.001609in}{1.149319in}}%
\pgfpathlineto{\pgfqpoint{1.005549in}{1.143690in}}%
\pgfpathlineto{\pgfqpoint{1.009845in}{1.138123in}}%
\pgfpathlineto{\pgfqpoint{1.014494in}{1.132623in}}%
\pgfpathclose%
\pgfusepath{fill}%
\end{pgfscope}%
\begin{pgfscope}%
\pgfpathrectangle{\pgfqpoint{0.329460in}{0.284240in}}{\pgfqpoint{1.989680in}{1.989680in}}%
\pgfusepath{clip}%
\pgfsetbuttcap%
\pgfsetroundjoin%
\definecolor{currentfill}{rgb}{0.282884,0.135920,0.453427}%
\pgfsetfillcolor{currentfill}%
\pgfsetlinewidth{0.000000pt}%
\definecolor{currentstroke}{rgb}{0.000000,0.000000,0.000000}%
\pgfsetstrokecolor{currentstroke}%
\pgfsetdash{}{0pt}%
\pgfpathmoveto{\pgfqpoint{1.930489in}{0.995584in}}%
\pgfpathlineto{\pgfqpoint{1.934011in}{1.000826in}}%
\pgfpathlineto{\pgfqpoint{1.937546in}{1.006398in}}%
\pgfpathlineto{\pgfqpoint{1.941094in}{1.012308in}}%
\pgfpathlineto{\pgfqpoint{1.944656in}{1.018559in}}%
\pgfpathlineto{\pgfqpoint{1.939575in}{1.008791in}}%
\pgfpathlineto{\pgfqpoint{1.933882in}{0.999096in}}%
\pgfpathlineto{\pgfqpoint{1.927581in}{0.989487in}}%
\pgfpathlineto{\pgfqpoint{1.920674in}{0.979973in}}%
\pgfpathlineto{\pgfqpoint{1.917235in}{0.973917in}}%
\pgfpathlineto{\pgfqpoint{1.913811in}{0.968206in}}%
\pgfpathlineto{\pgfqpoint{1.910399in}{0.962832in}}%
\pgfpathlineto{\pgfqpoint{1.907001in}{0.957791in}}%
\pgfpathlineto{\pgfqpoint{1.913761in}{0.967108in}}%
\pgfpathlineto{\pgfqpoint{1.919932in}{0.976520in}}%
\pgfpathlineto{\pgfqpoint{1.925509in}{0.986015in}}%
\pgfpathlineto{\pgfqpoint{1.930489in}{0.995584in}}%
\pgfpathclose%
\pgfusepath{fill}%
\end{pgfscope}%
\begin{pgfscope}%
\pgfpathrectangle{\pgfqpoint{0.329460in}{0.284240in}}{\pgfqpoint{1.989680in}{1.989680in}}%
\pgfusepath{clip}%
\pgfsetbuttcap%
\pgfsetroundjoin%
\definecolor{currentfill}{rgb}{0.166383,0.690856,0.496502}%
\pgfsetfillcolor{currentfill}%
\pgfsetlinewidth{0.000000pt}%
\definecolor{currentstroke}{rgb}{0.000000,0.000000,0.000000}%
\pgfsetstrokecolor{currentstroke}%
\pgfsetdash{}{0pt}%
\pgfpathmoveto{\pgfqpoint{1.580597in}{1.507570in}}%
\pgfpathlineto{\pgfqpoint{1.583877in}{1.501488in}}%
\pgfpathlineto{\pgfqpoint{1.587154in}{1.495363in}}%
\pgfpathlineto{\pgfqpoint{1.590429in}{1.489197in}}%
\pgfpathlineto{\pgfqpoint{1.593700in}{1.482993in}}%
\pgfpathlineto{\pgfqpoint{1.595372in}{1.479259in}}%
\pgfpathlineto{\pgfqpoint{1.596805in}{1.475499in}}%
\pgfpathlineto{\pgfqpoint{1.597995in}{1.471717in}}%
\pgfpathlineto{\pgfqpoint{1.598940in}{1.467916in}}%
\pgfpathlineto{\pgfqpoint{1.595590in}{1.474339in}}%
\pgfpathlineto{\pgfqpoint{1.592237in}{1.480724in}}%
\pgfpathlineto{\pgfqpoint{1.588882in}{1.487067in}}%
\pgfpathlineto{\pgfqpoint{1.585522in}{1.493368in}}%
\pgfpathlineto{\pgfqpoint{1.584636in}{1.496948in}}%
\pgfpathlineto{\pgfqpoint{1.583518in}{1.500511in}}%
\pgfpathlineto{\pgfqpoint{1.582171in}{1.504052in}}%
\pgfpathlineto{\pgfqpoint{1.580597in}{1.507570in}}%
\pgfpathclose%
\pgfusepath{fill}%
\end{pgfscope}%
\begin{pgfscope}%
\pgfpathrectangle{\pgfqpoint{0.329460in}{0.284240in}}{\pgfqpoint{1.989680in}{1.989680in}}%
\pgfusepath{clip}%
\pgfsetbuttcap%
\pgfsetroundjoin%
\definecolor{currentfill}{rgb}{0.267004,0.004874,0.329415}%
\pgfsetfillcolor{currentfill}%
\pgfsetlinewidth{0.000000pt}%
\definecolor{currentstroke}{rgb}{0.000000,0.000000,0.000000}%
\pgfsetstrokecolor{currentstroke}%
\pgfsetdash{}{0pt}%
\pgfpathmoveto{\pgfqpoint{0.909793in}{0.885029in}}%
\pgfpathlineto{\pgfqpoint{0.906838in}{0.883373in}}%
\pgfpathlineto{\pgfqpoint{0.903877in}{0.881933in}}%
\pgfpathlineto{\pgfqpoint{0.900909in}{0.880712in}}%
\pgfpathlineto{\pgfqpoint{0.897935in}{0.879714in}}%
\pgfpathlineto{\pgfqpoint{0.889437in}{0.887429in}}%
\pgfpathlineto{\pgfqpoint{0.881429in}{0.895273in}}%
\pgfpathlineto{\pgfqpoint{0.873917in}{0.903239in}}%
\pgfpathlineto{\pgfqpoint{0.866908in}{0.911317in}}%
\pgfpathlineto{\pgfqpoint{0.870070in}{0.912103in}}%
\pgfpathlineto{\pgfqpoint{0.873226in}{0.913113in}}%
\pgfpathlineto{\pgfqpoint{0.876376in}{0.914341in}}%
\pgfpathlineto{\pgfqpoint{0.879519in}{0.915784in}}%
\pgfpathlineto{\pgfqpoint{0.886360in}{0.907922in}}%
\pgfpathlineto{\pgfqpoint{0.893690in}{0.900169in}}%
\pgfpathlineto{\pgfqpoint{0.901503in}{0.892536in}}%
\pgfpathlineto{\pgfqpoint{0.909793in}{0.885029in}}%
\pgfpathclose%
\pgfusepath{fill}%
\end{pgfscope}%
\begin{pgfscope}%
\pgfpathrectangle{\pgfqpoint{0.329460in}{0.284240in}}{\pgfqpoint{1.989680in}{1.989680in}}%
\pgfusepath{clip}%
\pgfsetbuttcap%
\pgfsetroundjoin%
\definecolor{currentfill}{rgb}{0.212395,0.359683,0.551710}%
\pgfsetfillcolor{currentfill}%
\pgfsetlinewidth{0.000000pt}%
\definecolor{currentstroke}{rgb}{0.000000,0.000000,0.000000}%
\pgfsetstrokecolor{currentstroke}%
\pgfsetdash{}{0pt}%
\pgfpathmoveto{\pgfqpoint{1.694412in}{1.188078in}}%
\pgfpathlineto{\pgfqpoint{1.697618in}{1.181012in}}%
\pgfpathlineto{\pgfqpoint{1.700822in}{1.173991in}}%
\pgfpathlineto{\pgfqpoint{1.704025in}{1.167019in}}%
\pgfpathlineto{\pgfqpoint{1.707228in}{1.160099in}}%
\pgfpathlineto{\pgfqpoint{1.703968in}{1.154370in}}%
\pgfpathlineto{\pgfqpoint{1.700346in}{1.148691in}}%
\pgfpathlineto{\pgfqpoint{1.696366in}{1.143068in}}%
\pgfpathlineto{\pgfqpoint{1.692030in}{1.137508in}}%
\pgfpathlineto{\pgfqpoint{1.688954in}{1.144656in}}%
\pgfpathlineto{\pgfqpoint{1.685876in}{1.151856in}}%
\pgfpathlineto{\pgfqpoint{1.682798in}{1.159105in}}%
\pgfpathlineto{\pgfqpoint{1.679720in}{1.166399in}}%
\pgfpathlineto{\pgfqpoint{1.683909in}{1.171734in}}%
\pgfpathlineto{\pgfqpoint{1.687756in}{1.177129in}}%
\pgfpathlineto{\pgfqpoint{1.691258in}{1.182579in}}%
\pgfpathlineto{\pgfqpoint{1.694412in}{1.188078in}}%
\pgfpathclose%
\pgfusepath{fill}%
\end{pgfscope}%
\begin{pgfscope}%
\pgfpathrectangle{\pgfqpoint{0.329460in}{0.284240in}}{\pgfqpoint{1.989680in}{1.989680in}}%
\pgfusepath{clip}%
\pgfsetbuttcap%
\pgfsetroundjoin%
\definecolor{currentfill}{rgb}{0.636902,0.856542,0.216620}%
\pgfsetfillcolor{currentfill}%
\pgfsetlinewidth{0.000000pt}%
\definecolor{currentstroke}{rgb}{0.000000,0.000000,0.000000}%
\pgfsetstrokecolor{currentstroke}%
\pgfsetdash{}{0pt}%
\pgfpathmoveto{\pgfqpoint{1.385962in}{1.702204in}}%
\pgfpathlineto{\pgfqpoint{1.386813in}{1.699318in}}%
\pgfpathlineto{\pgfqpoint{1.387664in}{1.696342in}}%
\pgfpathlineto{\pgfqpoint{1.388513in}{1.693278in}}%
\pgfpathlineto{\pgfqpoint{1.389361in}{1.690126in}}%
\pgfpathlineto{\pgfqpoint{1.393982in}{1.689526in}}%
\pgfpathlineto{\pgfqpoint{1.398562in}{1.688858in}}%
\pgfpathlineto{\pgfqpoint{1.403096in}{1.688123in}}%
\pgfpathlineto{\pgfqpoint{1.401942in}{1.691320in}}%
\pgfpathlineto{\pgfqpoint{1.400787in}{1.694431in}}%
\pgfpathlineto{\pgfqpoint{1.399630in}{1.697452in}}%
\pgfpathlineto{\pgfqpoint{1.398472in}{1.700383in}}%
\pgfpathlineto{\pgfqpoint{1.394342in}{1.701052in}}%
\pgfpathlineto{\pgfqpoint{1.390171in}{1.701659in}}%
\pgfpathlineto{\pgfqpoint{1.385962in}{1.702204in}}%
\pgfpathclose%
\pgfusepath{fill}%
\end{pgfscope}%
\begin{pgfscope}%
\pgfpathrectangle{\pgfqpoint{0.329460in}{0.284240in}}{\pgfqpoint{1.989680in}{1.989680in}}%
\pgfusepath{clip}%
\pgfsetbuttcap%
\pgfsetroundjoin%
\definecolor{currentfill}{rgb}{0.163625,0.471133,0.558148}%
\pgfsetfillcolor{currentfill}%
\pgfsetlinewidth{0.000000pt}%
\definecolor{currentstroke}{rgb}{0.000000,0.000000,0.000000}%
\pgfsetstrokecolor{currentstroke}%
\pgfsetdash{}{0pt}%
\pgfpathmoveto{\pgfqpoint{1.664047in}{1.295026in}}%
\pgfpathlineto{\pgfqpoint{1.667346in}{1.287857in}}%
\pgfpathlineto{\pgfqpoint{1.670642in}{1.280700in}}%
\pgfpathlineto{\pgfqpoint{1.673936in}{1.273559in}}%
\pgfpathlineto{\pgfqpoint{1.677229in}{1.266437in}}%
\pgfpathlineto{\pgfqpoint{1.675594in}{1.261267in}}%
\pgfpathlineto{\pgfqpoint{1.673631in}{1.256122in}}%
\pgfpathlineto{\pgfqpoint{1.671342in}{1.251006in}}%
\pgfpathlineto{\pgfqpoint{1.668726in}{1.245926in}}%
\pgfpathlineto{\pgfqpoint{1.665509in}{1.253279in}}%
\pgfpathlineto{\pgfqpoint{1.662291in}{1.260651in}}%
\pgfpathlineto{\pgfqpoint{1.659070in}{1.268038in}}%
\pgfpathlineto{\pgfqpoint{1.655848in}{1.275437in}}%
\pgfpathlineto{\pgfqpoint{1.658367in}{1.280289in}}%
\pgfpathlineto{\pgfqpoint{1.660574in}{1.285174in}}%
\pgfpathlineto{\pgfqpoint{1.662468in}{1.290089in}}%
\pgfpathlineto{\pgfqpoint{1.664047in}{1.295026in}}%
\pgfpathclose%
\pgfusepath{fill}%
\end{pgfscope}%
\begin{pgfscope}%
\pgfpathrectangle{\pgfqpoint{0.329460in}{0.284240in}}{\pgfqpoint{1.989680in}{1.989680in}}%
\pgfusepath{clip}%
\pgfsetbuttcap%
\pgfsetroundjoin%
\definecolor{currentfill}{rgb}{0.283072,0.130895,0.449241}%
\pgfsetfillcolor{currentfill}%
\pgfsetlinewidth{0.000000pt}%
\definecolor{currentstroke}{rgb}{0.000000,0.000000,0.000000}%
\pgfsetstrokecolor{currentstroke}%
\pgfsetdash{}{0pt}%
\pgfpathmoveto{\pgfqpoint{0.979418in}{0.978665in}}%
\pgfpathlineto{\pgfqpoint{0.976551in}{0.973051in}}%
\pgfpathlineto{\pgfqpoint{0.973683in}{0.967559in}}%
\pgfpathlineto{\pgfqpoint{0.970812in}{0.962193in}}%
\pgfpathlineto{\pgfqpoint{0.967940in}{0.956955in}}%
\pgfpathlineto{\pgfqpoint{0.960684in}{0.963404in}}%
\pgfpathlineto{\pgfqpoint{0.953838in}{0.969963in}}%
\pgfpathlineto{\pgfqpoint{0.947410in}{0.976626in}}%
\pgfpathlineto{\pgfqpoint{0.941403in}{0.983385in}}%
\pgfpathlineto{\pgfqpoint{0.944462in}{0.988405in}}%
\pgfpathlineto{\pgfqpoint{0.947519in}{0.993553in}}%
\pgfpathlineto{\pgfqpoint{0.950575in}{0.998826in}}%
\pgfpathlineto{\pgfqpoint{0.953628in}{1.004222in}}%
\pgfpathlineto{\pgfqpoint{0.959467in}{0.997685in}}%
\pgfpathlineto{\pgfqpoint{0.965716in}{0.991242in}}%
\pgfpathlineto{\pgfqpoint{0.972368in}{0.984900in}}%
\pgfpathlineto{\pgfqpoint{0.979418in}{0.978665in}}%
\pgfpathclose%
\pgfusepath{fill}%
\end{pgfscope}%
\begin{pgfscope}%
\pgfpathrectangle{\pgfqpoint{0.329460in}{0.284240in}}{\pgfqpoint{1.989680in}{1.989680in}}%
\pgfusepath{clip}%
\pgfsetbuttcap%
\pgfsetroundjoin%
\definecolor{currentfill}{rgb}{0.636902,0.856542,0.216620}%
\pgfsetfillcolor{currentfill}%
\pgfsetlinewidth{0.000000pt}%
\definecolor{currentstroke}{rgb}{0.000000,0.000000,0.000000}%
\pgfsetstrokecolor{currentstroke}%
\pgfsetdash{}{0pt}%
\pgfpathmoveto{\pgfqpoint{1.300271in}{1.699738in}}%
\pgfpathlineto{\pgfqpoint{1.299024in}{1.696790in}}%
\pgfpathlineto{\pgfqpoint{1.297778in}{1.693753in}}%
\pgfpathlineto{\pgfqpoint{1.296534in}{1.690627in}}%
\pgfpathlineto{\pgfqpoint{1.295291in}{1.687413in}}%
\pgfpathlineto{\pgfqpoint{1.299781in}{1.688208in}}%
\pgfpathlineto{\pgfqpoint{1.304320in}{1.688936in}}%
\pgfpathlineto{\pgfqpoint{1.308905in}{1.689596in}}%
\pgfpathlineto{\pgfqpoint{1.313530in}{1.690188in}}%
\pgfpathlineto{\pgfqpoint{1.314366in}{1.693339in}}%
\pgfpathlineto{\pgfqpoint{1.315204in}{1.696402in}}%
\pgfpathlineto{\pgfqpoint{1.316043in}{1.699376in}}%
\pgfpathlineto{\pgfqpoint{1.316883in}{1.702260in}}%
\pgfpathlineto{\pgfqpoint{1.312670in}{1.701722in}}%
\pgfpathlineto{\pgfqpoint{1.308495in}{1.701122in}}%
\pgfpathlineto{\pgfqpoint{1.304360in}{1.700460in}}%
\pgfpathlineto{\pgfqpoint{1.300271in}{1.699738in}}%
\pgfpathclose%
\pgfusepath{fill}%
\end{pgfscope}%
\begin{pgfscope}%
\pgfpathrectangle{\pgfqpoint{0.329460in}{0.284240in}}{\pgfqpoint{1.989680in}{1.989680in}}%
\pgfusepath{clip}%
\pgfsetbuttcap%
\pgfsetroundjoin%
\definecolor{currentfill}{rgb}{0.120081,0.622161,0.534946}%
\pgfsetfillcolor{currentfill}%
\pgfsetlinewidth{0.000000pt}%
\definecolor{currentstroke}{rgb}{0.000000,0.000000,0.000000}%
\pgfsetstrokecolor{currentstroke}%
\pgfsetdash{}{0pt}%
\pgfpathmoveto{\pgfqpoint{1.088893in}{1.422050in}}%
\pgfpathlineto{\pgfqpoint{1.085533in}{1.415189in}}%
\pgfpathlineto{\pgfqpoint{1.082175in}{1.408302in}}%
\pgfpathlineto{\pgfqpoint{1.078819in}{1.401392in}}%
\pgfpathlineto{\pgfqpoint{1.075467in}{1.394460in}}%
\pgfpathlineto{\pgfqpoint{1.075194in}{1.398752in}}%
\pgfpathlineto{\pgfqpoint{1.075196in}{1.403043in}}%
\pgfpathlineto{\pgfqpoint{1.075474in}{1.407329in}}%
\pgfpathlineto{\pgfqpoint{1.076025in}{1.411605in}}%
\pgfpathlineto{\pgfqpoint{1.079362in}{1.418309in}}%
\pgfpathlineto{\pgfqpoint{1.082701in}{1.424993in}}%
\pgfpathlineto{\pgfqpoint{1.086044in}{1.431653in}}%
\pgfpathlineto{\pgfqpoint{1.089389in}{1.438288in}}%
\pgfpathlineto{\pgfqpoint{1.088873in}{1.434237in}}%
\pgfpathlineto{\pgfqpoint{1.088618in}{1.430178in}}%
\pgfpathlineto{\pgfqpoint{1.088625in}{1.426114in}}%
\pgfpathlineto{\pgfqpoint{1.088893in}{1.422050in}}%
\pgfpathclose%
\pgfusepath{fill}%
\end{pgfscope}%
\begin{pgfscope}%
\pgfpathrectangle{\pgfqpoint{0.329460in}{0.284240in}}{\pgfqpoint{1.989680in}{1.989680in}}%
\pgfusepath{clip}%
\pgfsetbuttcap%
\pgfsetroundjoin%
\definecolor{currentfill}{rgb}{0.565498,0.842430,0.262877}%
\pgfsetfillcolor{currentfill}%
\pgfsetlinewidth{0.000000pt}%
\definecolor{currentstroke}{rgb}{0.000000,0.000000,0.000000}%
\pgfsetstrokecolor{currentstroke}%
\pgfsetdash{}{0pt}%
\pgfpathmoveto{\pgfqpoint{1.437212in}{1.679901in}}%
\pgfpathlineto{\pgfqpoint{1.439124in}{1.676429in}}%
\pgfpathlineto{\pgfqpoint{1.441035in}{1.672873in}}%
\pgfpathlineto{\pgfqpoint{1.442943in}{1.669234in}}%
\pgfpathlineto{\pgfqpoint{1.444849in}{1.665513in}}%
\pgfpathlineto{\pgfqpoint{1.449132in}{1.664088in}}%
\pgfpathlineto{\pgfqpoint{1.453322in}{1.662599in}}%
\pgfpathlineto{\pgfqpoint{1.457412in}{1.661049in}}%
\pgfpathlineto{\pgfqpoint{1.461400in}{1.659438in}}%
\pgfpathlineto{\pgfqpoint{1.459155in}{1.663286in}}%
\pgfpathlineto{\pgfqpoint{1.456908in}{1.667053in}}%
\pgfpathlineto{\pgfqpoint{1.454658in}{1.670736in}}%
\pgfpathlineto{\pgfqpoint{1.452405in}{1.674335in}}%
\pgfpathlineto{\pgfqpoint{1.448745in}{1.675811in}}%
\pgfpathlineto{\pgfqpoint{1.444990in}{1.677231in}}%
\pgfpathlineto{\pgfqpoint{1.441144in}{1.678595in}}%
\pgfpathlineto{\pgfqpoint{1.437212in}{1.679901in}}%
\pgfpathclose%
\pgfusepath{fill}%
\end{pgfscope}%
\begin{pgfscope}%
\pgfpathrectangle{\pgfqpoint{0.329460in}{0.284240in}}{\pgfqpoint{1.989680in}{1.989680in}}%
\pgfusepath{clip}%
\pgfsetbuttcap%
\pgfsetroundjoin%
\definecolor{currentfill}{rgb}{0.133743,0.548535,0.553541}%
\pgfsetfillcolor{currentfill}%
\pgfsetlinewidth{0.000000pt}%
\definecolor{currentstroke}{rgb}{0.000000,0.000000,0.000000}%
\pgfsetstrokecolor{currentstroke}%
\pgfsetdash{}{0pt}%
\pgfpathmoveto{\pgfqpoint{1.066080in}{1.348563in}}%
\pgfpathlineto{\pgfqpoint{1.062777in}{1.341330in}}%
\pgfpathlineto{\pgfqpoint{1.059476in}{1.334091in}}%
\pgfpathlineto{\pgfqpoint{1.056177in}{1.326846in}}%
\pgfpathlineto{\pgfqpoint{1.052881in}{1.319599in}}%
\pgfpathlineto{\pgfqpoint{1.051392in}{1.324308in}}%
\pgfpathlineto{\pgfqpoint{1.050205in}{1.329035in}}%
\pgfpathlineto{\pgfqpoint{1.049321in}{1.333774in}}%
\pgfpathlineto{\pgfqpoint{1.048739in}{1.338521in}}%
\pgfpathlineto{\pgfqpoint{1.052071in}{1.345538in}}%
\pgfpathlineto{\pgfqpoint{1.055406in}{1.352553in}}%
\pgfpathlineto{\pgfqpoint{1.058743in}{1.359563in}}%
\pgfpathlineto{\pgfqpoint{1.062082in}{1.366567in}}%
\pgfpathlineto{\pgfqpoint{1.062648in}{1.362049in}}%
\pgfpathlineto{\pgfqpoint{1.063503in}{1.357540in}}%
\pgfpathlineto{\pgfqpoint{1.064647in}{1.353043in}}%
\pgfpathlineto{\pgfqpoint{1.066080in}{1.348563in}}%
\pgfpathclose%
\pgfusepath{fill}%
\end{pgfscope}%
\begin{pgfscope}%
\pgfpathrectangle{\pgfqpoint{0.329460in}{0.284240in}}{\pgfqpoint{1.989680in}{1.989680in}}%
\pgfusepath{clip}%
\pgfsetbuttcap%
\pgfsetroundjoin%
\definecolor{currentfill}{rgb}{0.268510,0.009605,0.335427}%
\pgfsetfillcolor{currentfill}%
\pgfsetlinewidth{0.000000pt}%
\definecolor{currentstroke}{rgb}{0.000000,0.000000,0.000000}%
\pgfsetstrokecolor{currentstroke}%
\pgfsetdash{}{0pt}%
\pgfpathmoveto{\pgfqpoint{1.854129in}{0.917950in}}%
\pgfpathlineto{\pgfqpoint{1.857362in}{0.918394in}}%
\pgfpathlineto{\pgfqpoint{1.860604in}{0.919088in}}%
\pgfpathlineto{\pgfqpoint{1.863854in}{0.920036in}}%
\pgfpathlineto{\pgfqpoint{1.867113in}{0.921244in}}%
\pgfpathlineto{\pgfqpoint{1.860244in}{0.912646in}}%
\pgfpathlineto{\pgfqpoint{1.852838in}{0.904157in}}%
\pgfpathlineto{\pgfqpoint{1.844902in}{0.895785in}}%
\pgfpathlineto{\pgfqpoint{1.836442in}{0.887542in}}%
\pgfpathlineto{\pgfqpoint{1.833362in}{0.886541in}}%
\pgfpathlineto{\pgfqpoint{1.830289in}{0.885801in}}%
\pgfpathlineto{\pgfqpoint{1.827225in}{0.885317in}}%
\pgfpathlineto{\pgfqpoint{1.824169in}{0.885083in}}%
\pgfpathlineto{\pgfqpoint{1.832430in}{0.893122in}}%
\pgfpathlineto{\pgfqpoint{1.840181in}{0.901285in}}%
\pgfpathlineto{\pgfqpoint{1.847416in}{0.909565in}}%
\pgfpathlineto{\pgfqpoint{1.854129in}{0.917950in}}%
\pgfpathclose%
\pgfusepath{fill}%
\end{pgfscope}%
\begin{pgfscope}%
\pgfpathrectangle{\pgfqpoint{0.329460in}{0.284240in}}{\pgfqpoint{1.989680in}{1.989680in}}%
\pgfusepath{clip}%
\pgfsetbuttcap%
\pgfsetroundjoin%
\definecolor{currentfill}{rgb}{0.233603,0.313828,0.543914}%
\pgfsetfillcolor{currentfill}%
\pgfsetlinewidth{0.000000pt}%
\definecolor{currentstroke}{rgb}{0.000000,0.000000,0.000000}%
\pgfsetstrokecolor{currentstroke}%
\pgfsetdash{}{0pt}%
\pgfpathmoveto{\pgfqpoint{1.988373in}{1.122762in}}%
\pgfpathlineto{\pgfqpoint{1.992142in}{1.132553in}}%
\pgfpathlineto{\pgfqpoint{1.995929in}{1.142747in}}%
\pgfpathlineto{\pgfqpoint{1.999735in}{1.153350in}}%
\pgfpathlineto{\pgfqpoint{2.003560in}{1.164371in}}%
\pgfpathlineto{\pgfqpoint{2.000827in}{1.153849in}}%
\pgfpathlineto{\pgfqpoint{1.997429in}{1.143360in}}%
\pgfpathlineto{\pgfqpoint{1.993367in}{1.132915in}}%
\pgfpathlineto{\pgfqpoint{1.988641in}{1.122523in}}%
\pgfpathlineto{\pgfqpoint{1.984879in}{1.111677in}}%
\pgfpathlineto{\pgfqpoint{1.981136in}{1.101251in}}%
\pgfpathlineto{\pgfqpoint{1.977412in}{1.091236in}}%
\pgfpathlineto{\pgfqpoint{1.973706in}{1.081627in}}%
\pgfpathlineto{\pgfqpoint{1.978345in}{1.091841in}}%
\pgfpathlineto{\pgfqpoint{1.982336in}{1.102108in}}%
\pgfpathlineto{\pgfqpoint{1.985679in}{1.112419in}}%
\pgfpathlineto{\pgfqpoint{1.988373in}{1.122762in}}%
\pgfpathclose%
\pgfusepath{fill}%
\end{pgfscope}%
\begin{pgfscope}%
\pgfpathrectangle{\pgfqpoint{0.329460in}{0.284240in}}{\pgfqpoint{1.989680in}{1.989680in}}%
\pgfusepath{clip}%
\pgfsetbuttcap%
\pgfsetroundjoin%
\definecolor{currentfill}{rgb}{0.274128,0.199721,0.498911}%
\pgfsetfillcolor{currentfill}%
\pgfsetlinewidth{0.000000pt}%
\definecolor{currentstroke}{rgb}{0.000000,0.000000,0.000000}%
\pgfsetstrokecolor{currentstroke}%
\pgfsetdash{}{0pt}%
\pgfpathmoveto{\pgfqpoint{1.728940in}{1.056855in}}%
\pgfpathlineto{\pgfqpoint{1.732017in}{1.050655in}}%
\pgfpathlineto{\pgfqpoint{1.735096in}{1.044551in}}%
\pgfpathlineto{\pgfqpoint{1.738175in}{1.038545in}}%
\pgfpathlineto{\pgfqpoint{1.741256in}{1.032640in}}%
\pgfpathlineto{\pgfqpoint{1.735940in}{1.026251in}}%
\pgfpathlineto{\pgfqpoint{1.730223in}{1.019947in}}%
\pgfpathlineto{\pgfqpoint{1.724112in}{1.013734in}}%
\pgfpathlineto{\pgfqpoint{1.717610in}{1.007620in}}%
\pgfpathlineto{\pgfqpoint{1.714705in}{1.013746in}}%
\pgfpathlineto{\pgfqpoint{1.711802in}{1.019973in}}%
\pgfpathlineto{\pgfqpoint{1.708899in}{1.026299in}}%
\pgfpathlineto{\pgfqpoint{1.705997in}{1.032719in}}%
\pgfpathlineto{\pgfqpoint{1.712304in}{1.038617in}}%
\pgfpathlineto{\pgfqpoint{1.718233in}{1.044610in}}%
\pgfpathlineto{\pgfqpoint{1.723780in}{1.050691in}}%
\pgfpathlineto{\pgfqpoint{1.728940in}{1.056855in}}%
\pgfpathclose%
\pgfusepath{fill}%
\end{pgfscope}%
\begin{pgfscope}%
\pgfpathrectangle{\pgfqpoint{0.329460in}{0.284240in}}{\pgfqpoint{1.989680in}{1.989680in}}%
\pgfusepath{clip}%
\pgfsetbuttcap%
\pgfsetroundjoin%
\definecolor{currentfill}{rgb}{0.565498,0.842430,0.262877}%
\pgfsetfillcolor{currentfill}%
\pgfsetlinewidth{0.000000pt}%
\definecolor{currentstroke}{rgb}{0.000000,0.000000,0.000000}%
\pgfsetstrokecolor{currentstroke}%
\pgfsetdash{}{0pt}%
\pgfpathmoveto{\pgfqpoint{1.246799in}{1.672977in}}%
\pgfpathlineto{\pgfqpoint{1.244475in}{1.669348in}}%
\pgfpathlineto{\pgfqpoint{1.242154in}{1.665633in}}%
\pgfpathlineto{\pgfqpoint{1.239835in}{1.661835in}}%
\pgfpathlineto{\pgfqpoint{1.237519in}{1.657956in}}%
\pgfpathlineto{\pgfqpoint{1.241413in}{1.659620in}}%
\pgfpathlineto{\pgfqpoint{1.245412in}{1.661224in}}%
\pgfpathlineto{\pgfqpoint{1.249514in}{1.662768in}}%
\pgfpathlineto{\pgfqpoint{1.253714in}{1.664249in}}%
\pgfpathlineto{\pgfqpoint{1.255698in}{1.667997in}}%
\pgfpathlineto{\pgfqpoint{1.257684in}{1.671662in}}%
\pgfpathlineto{\pgfqpoint{1.259673in}{1.675245in}}%
\pgfpathlineto{\pgfqpoint{1.261664in}{1.678743in}}%
\pgfpathlineto{\pgfqpoint{1.257808in}{1.677385in}}%
\pgfpathlineto{\pgfqpoint{1.254043in}{1.675971in}}%
\pgfpathlineto{\pgfqpoint{1.250372in}{1.674501in}}%
\pgfpathlineto{\pgfqpoint{1.246799in}{1.672977in}}%
\pgfpathclose%
\pgfusepath{fill}%
\end{pgfscope}%
\begin{pgfscope}%
\pgfpathrectangle{\pgfqpoint{0.329460in}{0.284240in}}{\pgfqpoint{1.989680in}{1.989680in}}%
\pgfusepath{clip}%
\pgfsetbuttcap%
\pgfsetroundjoin%
\definecolor{currentfill}{rgb}{0.172719,0.448791,0.557885}%
\pgfsetfillcolor{currentfill}%
\pgfsetlinewidth{0.000000pt}%
\definecolor{currentstroke}{rgb}{0.000000,0.000000,0.000000}%
\pgfsetstrokecolor{currentstroke}%
\pgfsetdash{}{0pt}%
\pgfpathmoveto{\pgfqpoint{0.685745in}{1.203258in}}%
\pgfpathlineto{\pgfqpoint{0.681826in}{1.216433in}}%
\pgfpathlineto{\pgfqpoint{0.677884in}{1.230070in}}%
\pgfpathlineto{\pgfqpoint{0.673920in}{1.244175in}}%
\pgfpathlineto{\pgfqpoint{0.671196in}{1.254987in}}%
\pgfpathlineto{\pgfqpoint{0.669164in}{1.265817in}}%
\pgfpathlineto{\pgfqpoint{0.667823in}{1.276656in}}%
\pgfpathlineto{\pgfqpoint{0.667171in}{1.287492in}}%
\pgfpathlineto{\pgfqpoint{0.671147in}{1.273232in}}%
\pgfpathlineto{\pgfqpoint{0.675101in}{1.259438in}}%
\pgfpathlineto{\pgfqpoint{0.679033in}{1.246102in}}%
\pgfpathlineto{\pgfqpoint{0.679694in}{1.235383in}}%
\pgfpathlineto{\pgfqpoint{0.681032in}{1.224663in}}%
\pgfpathlineto{\pgfqpoint{0.683049in}{1.213950in}}%
\pgfpathlineto{\pgfqpoint{0.685745in}{1.203258in}}%
\pgfpathclose%
\pgfusepath{fill}%
\end{pgfscope}%
\begin{pgfscope}%
\pgfpathrectangle{\pgfqpoint{0.329460in}{0.284240in}}{\pgfqpoint{1.989680in}{1.989680in}}%
\pgfusepath{clip}%
\pgfsetbuttcap%
\pgfsetroundjoin%
\definecolor{currentfill}{rgb}{0.412913,0.803041,0.357269}%
\pgfsetfillcolor{currentfill}%
\pgfsetlinewidth{0.000000pt}%
\definecolor{currentstroke}{rgb}{0.000000,0.000000,0.000000}%
\pgfsetstrokecolor{currentstroke}%
\pgfsetdash{}{0pt}%
\pgfpathmoveto{\pgfqpoint{1.500396in}{1.627125in}}%
\pgfpathlineto{\pgfqpoint{1.503198in}{1.622577in}}%
\pgfpathlineto{\pgfqpoint{1.505997in}{1.617957in}}%
\pgfpathlineto{\pgfqpoint{1.508793in}{1.613265in}}%
\pgfpathlineto{\pgfqpoint{1.511586in}{1.608505in}}%
\pgfpathlineto{\pgfqpoint{1.514966in}{1.606070in}}%
\pgfpathlineto{\pgfqpoint{1.518186in}{1.603585in}}%
\pgfpathlineto{\pgfqpoint{1.521244in}{1.601051in}}%
\pgfpathlineto{\pgfqpoint{1.524137in}{1.598471in}}%
\pgfpathlineto{\pgfqpoint{1.521120in}{1.603414in}}%
\pgfpathlineto{\pgfqpoint{1.518101in}{1.608288in}}%
\pgfpathlineto{\pgfqpoint{1.515078in}{1.613092in}}%
\pgfpathlineto{\pgfqpoint{1.512053in}{1.617822in}}%
\pgfpathlineto{\pgfqpoint{1.509367in}{1.620214in}}%
\pgfpathlineto{\pgfqpoint{1.506527in}{1.622563in}}%
\pgfpathlineto{\pgfqpoint{1.503536in}{1.624868in}}%
\pgfpathlineto{\pgfqpoint{1.500396in}{1.627125in}}%
\pgfpathclose%
\pgfusepath{fill}%
\end{pgfscope}%
\begin{pgfscope}%
\pgfpathrectangle{\pgfqpoint{0.329460in}{0.284240in}}{\pgfqpoint{1.989680in}{1.989680in}}%
\pgfusepath{clip}%
\pgfsetbuttcap%
\pgfsetroundjoin%
\definecolor{currentfill}{rgb}{0.344074,0.780029,0.397381}%
\pgfsetfillcolor{currentfill}%
\pgfsetlinewidth{0.000000pt}%
\definecolor{currentstroke}{rgb}{0.000000,0.000000,0.000000}%
\pgfsetstrokecolor{currentstroke}%
\pgfsetdash{}{0pt}%
\pgfpathmoveto{\pgfqpoint{1.524137in}{1.598471in}}%
\pgfpathlineto{\pgfqpoint{1.527150in}{1.593461in}}%
\pgfpathlineto{\pgfqpoint{1.530159in}{1.588385in}}%
\pgfpathlineto{\pgfqpoint{1.533166in}{1.583245in}}%
\pgfpathlineto{\pgfqpoint{1.536169in}{1.578043in}}%
\pgfpathlineto{\pgfqpoint{1.539088in}{1.575228in}}%
\pgfpathlineto{\pgfqpoint{1.541824in}{1.572369in}}%
\pgfpathlineto{\pgfqpoint{1.544373in}{1.569468in}}%
\pgfpathlineto{\pgfqpoint{1.546734in}{1.566529in}}%
\pgfpathlineto{\pgfqpoint{1.543553in}{1.571928in}}%
\pgfpathlineto{\pgfqpoint{1.540369in}{1.577265in}}%
\pgfpathlineto{\pgfqpoint{1.537181in}{1.582537in}}%
\pgfpathlineto{\pgfqpoint{1.533991in}{1.587744in}}%
\pgfpathlineto{\pgfqpoint{1.531790in}{1.590482in}}%
\pgfpathlineto{\pgfqpoint{1.529412in}{1.593184in}}%
\pgfpathlineto{\pgfqpoint{1.526860in}{1.595848in}}%
\pgfpathlineto{\pgfqpoint{1.524137in}{1.598471in}}%
\pgfpathclose%
\pgfusepath{fill}%
\end{pgfscope}%
\begin{pgfscope}%
\pgfpathrectangle{\pgfqpoint{0.329460in}{0.284240in}}{\pgfqpoint{1.989680in}{1.989680in}}%
\pgfusepath{clip}%
\pgfsetbuttcap%
\pgfsetroundjoin%
\definecolor{currentfill}{rgb}{0.166383,0.690856,0.496502}%
\pgfsetfillcolor{currentfill}%
\pgfsetlinewidth{0.000000pt}%
\definecolor{currentstroke}{rgb}{0.000000,0.000000,0.000000}%
\pgfsetstrokecolor{currentstroke}%
\pgfsetdash{}{0pt}%
\pgfpathmoveto{\pgfqpoint{1.116260in}{1.490173in}}%
\pgfpathlineto{\pgfqpoint{1.112890in}{1.483824in}}%
\pgfpathlineto{\pgfqpoint{1.109524in}{1.477431in}}%
\pgfpathlineto{\pgfqpoint{1.106161in}{1.470997in}}%
\pgfpathlineto{\pgfqpoint{1.102800in}{1.464525in}}%
\pgfpathlineto{\pgfqpoint{1.103528in}{1.468339in}}%
\pgfpathlineto{\pgfqpoint{1.104501in}{1.472138in}}%
\pgfpathlineto{\pgfqpoint{1.105718in}{1.475918in}}%
\pgfpathlineto{\pgfqpoint{1.107177in}{1.479675in}}%
\pgfpathlineto{\pgfqpoint{1.110470in}{1.485928in}}%
\pgfpathlineto{\pgfqpoint{1.113766in}{1.492141in}}%
\pgfpathlineto{\pgfqpoint{1.117065in}{1.498314in}}%
\pgfpathlineto{\pgfqpoint{1.120368in}{1.504444in}}%
\pgfpathlineto{\pgfqpoint{1.118995in}{1.500905in}}%
\pgfpathlineto{\pgfqpoint{1.117852in}{1.497345in}}%
\pgfpathlineto{\pgfqpoint{1.116940in}{1.493766in}}%
\pgfpathlineto{\pgfqpoint{1.116260in}{1.490173in}}%
\pgfpathclose%
\pgfusepath{fill}%
\end{pgfscope}%
\begin{pgfscope}%
\pgfpathrectangle{\pgfqpoint{0.329460in}{0.284240in}}{\pgfqpoint{1.989680in}{1.989680in}}%
\pgfusepath{clip}%
\pgfsetbuttcap%
\pgfsetroundjoin%
\definecolor{currentfill}{rgb}{0.636902,0.856542,0.216620}%
\pgfsetfillcolor{currentfill}%
\pgfsetlinewidth{0.000000pt}%
\definecolor{currentstroke}{rgb}{0.000000,0.000000,0.000000}%
\pgfsetstrokecolor{currentstroke}%
\pgfsetdash{}{0pt}%
\pgfpathmoveto{\pgfqpoint{1.398472in}{1.700383in}}%
\pgfpathlineto{\pgfqpoint{1.399630in}{1.697452in}}%
\pgfpathlineto{\pgfqpoint{1.400787in}{1.694431in}}%
\pgfpathlineto{\pgfqpoint{1.401942in}{1.691320in}}%
\pgfpathlineto{\pgfqpoint{1.403096in}{1.688123in}}%
\pgfpathlineto{\pgfqpoint{1.407580in}{1.687320in}}%
\pgfpathlineto{\pgfqpoint{1.412010in}{1.686452in}}%
\pgfpathlineto{\pgfqpoint{1.416381in}{1.685518in}}%
\pgfpathlineto{\pgfqpoint{1.420689in}{1.684519in}}%
\pgfpathlineto{\pgfqpoint{1.419143in}{1.687799in}}%
\pgfpathlineto{\pgfqpoint{1.417596in}{1.690992in}}%
\pgfpathlineto{\pgfqpoint{1.416046in}{1.694095in}}%
\pgfpathlineto{\pgfqpoint{1.414495in}{1.697108in}}%
\pgfpathlineto{\pgfqpoint{1.410571in}{1.698015in}}%
\pgfpathlineto{\pgfqpoint{1.406590in}{1.698864in}}%
\pgfpathlineto{\pgfqpoint{1.402556in}{1.699654in}}%
\pgfpathlineto{\pgfqpoint{1.398472in}{1.700383in}}%
\pgfpathclose%
\pgfusepath{fill}%
\end{pgfscope}%
\begin{pgfscope}%
\pgfpathrectangle{\pgfqpoint{0.329460in}{0.284240in}}{\pgfqpoint{1.989680in}{1.989680in}}%
\pgfusepath{clip}%
\pgfsetbuttcap%
\pgfsetroundjoin%
\definecolor{currentfill}{rgb}{0.267004,0.004874,0.329415}%
\pgfsetfillcolor{currentfill}%
\pgfsetlinewidth{0.000000pt}%
\definecolor{currentstroke}{rgb}{0.000000,0.000000,0.000000}%
\pgfsetstrokecolor{currentstroke}%
\pgfsetdash{}{0pt}%
\pgfpathmoveto{\pgfqpoint{0.897935in}{0.879714in}}%
\pgfpathlineto{\pgfqpoint{0.894955in}{0.878945in}}%
\pgfpathlineto{\pgfqpoint{0.891968in}{0.878408in}}%
\pgfpathlineto{\pgfqpoint{0.888973in}{0.878108in}}%
\pgfpathlineto{\pgfqpoint{0.885971in}{0.878050in}}%
\pgfpathlineto{\pgfqpoint{0.877263in}{0.885970in}}%
\pgfpathlineto{\pgfqpoint{0.869059in}{0.894023in}}%
\pgfpathlineto{\pgfqpoint{0.861365in}{0.902200in}}%
\pgfpathlineto{\pgfqpoint{0.854188in}{0.910491in}}%
\pgfpathlineto{\pgfqpoint{0.857379in}{0.910341in}}%
\pgfpathlineto{\pgfqpoint{0.860562in}{0.910431in}}%
\pgfpathlineto{\pgfqpoint{0.863739in}{0.910758in}}%
\pgfpathlineto{\pgfqpoint{0.866908in}{0.911317in}}%
\pgfpathlineto{\pgfqpoint{0.873917in}{0.903239in}}%
\pgfpathlineto{\pgfqpoint{0.881429in}{0.895273in}}%
\pgfpathlineto{\pgfqpoint{0.889437in}{0.887429in}}%
\pgfpathlineto{\pgfqpoint{0.897935in}{0.879714in}}%
\pgfpathclose%
\pgfusepath{fill}%
\end{pgfscope}%
\begin{pgfscope}%
\pgfpathrectangle{\pgfqpoint{0.329460in}{0.284240in}}{\pgfqpoint{1.989680in}{1.989680in}}%
\pgfusepath{clip}%
\pgfsetbuttcap%
\pgfsetroundjoin%
\definecolor{currentfill}{rgb}{0.280255,0.165693,0.476498}%
\pgfsetfillcolor{currentfill}%
\pgfsetlinewidth{0.000000pt}%
\definecolor{currentstroke}{rgb}{0.000000,0.000000,0.000000}%
\pgfsetstrokecolor{currentstroke}%
\pgfsetdash{}{0pt}%
\pgfpathmoveto{\pgfqpoint{0.990867in}{1.002274in}}%
\pgfpathlineto{\pgfqpoint{0.988007in}{0.996206in}}%
\pgfpathlineto{\pgfqpoint{0.985145in}{0.990246in}}%
\pgfpathlineto{\pgfqpoint{0.982282in}{0.984398in}}%
\pgfpathlineto{\pgfqpoint{0.979418in}{0.978665in}}%
\pgfpathlineto{\pgfqpoint{0.972368in}{0.984900in}}%
\pgfpathlineto{\pgfqpoint{0.965716in}{0.991242in}}%
\pgfpathlineto{\pgfqpoint{0.959467in}{0.997685in}}%
\pgfpathlineto{\pgfqpoint{0.953628in}{1.004222in}}%
\pgfpathlineto{\pgfqpoint{0.956679in}{1.009737in}}%
\pgfpathlineto{\pgfqpoint{0.959729in}{1.015366in}}%
\pgfpathlineto{\pgfqpoint{0.962778in}{1.021107in}}%
\pgfpathlineto{\pgfqpoint{0.965825in}{1.026957in}}%
\pgfpathlineto{\pgfqpoint{0.971497in}{1.020643in}}%
\pgfpathlineto{\pgfqpoint{0.977565in}{1.014420in}}%
\pgfpathlineto{\pgfqpoint{0.984024in}{1.008295in}}%
\pgfpathlineto{\pgfqpoint{0.990867in}{1.002274in}}%
\pgfpathclose%
\pgfusepath{fill}%
\end{pgfscope}%
\begin{pgfscope}%
\pgfpathrectangle{\pgfqpoint{0.329460in}{0.284240in}}{\pgfqpoint{1.989680in}{1.989680in}}%
\pgfusepath{clip}%
\pgfsetbuttcap%
\pgfsetroundjoin%
\definecolor{currentfill}{rgb}{0.636902,0.856542,0.216620}%
\pgfsetfillcolor{currentfill}%
\pgfsetlinewidth{0.000000pt}%
\definecolor{currentstroke}{rgb}{0.000000,0.000000,0.000000}%
\pgfsetstrokecolor{currentstroke}%
\pgfsetdash{}{0pt}%
\pgfpathmoveto{\pgfqpoint{1.284444in}{1.696253in}}%
\pgfpathlineto{\pgfqpoint{1.282809in}{1.693218in}}%
\pgfpathlineto{\pgfqpoint{1.281175in}{1.690094in}}%
\pgfpathlineto{\pgfqpoint{1.279542in}{1.686880in}}%
\pgfpathlineto{\pgfqpoint{1.277912in}{1.683579in}}%
\pgfpathlineto{\pgfqpoint{1.282161in}{1.684634in}}%
\pgfpathlineto{\pgfqpoint{1.286477in}{1.685625in}}%
\pgfpathlineto{\pgfqpoint{1.290855in}{1.686551in}}%
\pgfpathlineto{\pgfqpoint{1.295291in}{1.687413in}}%
\pgfpathlineto{\pgfqpoint{1.296534in}{1.690627in}}%
\pgfpathlineto{\pgfqpoint{1.297778in}{1.693753in}}%
\pgfpathlineto{\pgfqpoint{1.299024in}{1.696790in}}%
\pgfpathlineto{\pgfqpoint{1.300271in}{1.699738in}}%
\pgfpathlineto{\pgfqpoint{1.296231in}{1.698955in}}%
\pgfpathlineto{\pgfqpoint{1.292244in}{1.698113in}}%
\pgfpathlineto{\pgfqpoint{1.288314in}{1.697212in}}%
\pgfpathlineto{\pgfqpoint{1.284444in}{1.696253in}}%
\pgfpathclose%
\pgfusepath{fill}%
\end{pgfscope}%
\begin{pgfscope}%
\pgfpathrectangle{\pgfqpoint{0.329460in}{0.284240in}}{\pgfqpoint{1.989680in}{1.989680in}}%
\pgfusepath{clip}%
\pgfsetbuttcap%
\pgfsetroundjoin%
\definecolor{currentfill}{rgb}{0.487026,0.823929,0.312321}%
\pgfsetfillcolor{currentfill}%
\pgfsetlinewidth{0.000000pt}%
\definecolor{currentstroke}{rgb}{0.000000,0.000000,0.000000}%
\pgfsetstrokecolor{currentstroke}%
\pgfsetdash{}{0pt}%
\pgfpathmoveto{\pgfqpoint{1.476249in}{1.652419in}}%
\pgfpathlineto{\pgfqpoint{1.478796in}{1.648342in}}%
\pgfpathlineto{\pgfqpoint{1.481341in}{1.644187in}}%
\pgfpathlineto{\pgfqpoint{1.483882in}{1.639955in}}%
\pgfpathlineto{\pgfqpoint{1.486421in}{1.635647in}}%
\pgfpathlineto{\pgfqpoint{1.490121in}{1.633597in}}%
\pgfpathlineto{\pgfqpoint{1.493685in}{1.631492in}}%
\pgfpathlineto{\pgfqpoint{1.497112in}{1.629334in}}%
\pgfpathlineto{\pgfqpoint{1.500396in}{1.627125in}}%
\pgfpathlineto{\pgfqpoint{1.497591in}{1.631600in}}%
\pgfpathlineto{\pgfqpoint{1.494783in}{1.635998in}}%
\pgfpathlineto{\pgfqpoint{1.491972in}{1.640319in}}%
\pgfpathlineto{\pgfqpoint{1.489158in}{1.644562in}}%
\pgfpathlineto{\pgfqpoint{1.486124in}{1.646598in}}%
\pgfpathlineto{\pgfqpoint{1.482959in}{1.648588in}}%
\pgfpathlineto{\pgfqpoint{1.479666in}{1.650529in}}%
\pgfpathlineto{\pgfqpoint{1.476249in}{1.652419in}}%
\pgfpathclose%
\pgfusepath{fill}%
\end{pgfscope}%
\begin{pgfscope}%
\pgfpathrectangle{\pgfqpoint{0.329460in}{0.284240in}}{\pgfqpoint{1.989680in}{1.989680in}}%
\pgfusepath{clip}%
\pgfsetbuttcap%
\pgfsetroundjoin%
\definecolor{currentfill}{rgb}{0.163625,0.471133,0.558148}%
\pgfsetfillcolor{currentfill}%
\pgfsetlinewidth{0.000000pt}%
\definecolor{currentstroke}{rgb}{0.000000,0.000000,0.000000}%
\pgfsetstrokecolor{currentstroke}%
\pgfsetdash{}{0pt}%
\pgfpathmoveto{\pgfqpoint{1.049027in}{1.271157in}}%
\pgfpathlineto{\pgfqpoint{1.045829in}{1.263707in}}%
\pgfpathlineto{\pgfqpoint{1.042633in}{1.256269in}}%
\pgfpathlineto{\pgfqpoint{1.039438in}{1.248847in}}%
\pgfpathlineto{\pgfqpoint{1.036245in}{1.241443in}}%
\pgfpathlineto{\pgfqpoint{1.033342in}{1.246488in}}%
\pgfpathlineto{\pgfqpoint{1.030763in}{1.251573in}}%
\pgfpathlineto{\pgfqpoint{1.028510in}{1.256692in}}%
\pgfpathlineto{\pgfqpoint{1.026583in}{1.261841in}}%
\pgfpathlineto{\pgfqpoint{1.029864in}{1.269015in}}%
\pgfpathlineto{\pgfqpoint{1.033146in}{1.276207in}}%
\pgfpathlineto{\pgfqpoint{1.036430in}{1.283415in}}%
\pgfpathlineto{\pgfqpoint{1.039716in}{1.290636in}}%
\pgfpathlineto{\pgfqpoint{1.041575in}{1.285719in}}%
\pgfpathlineto{\pgfqpoint{1.043747in}{1.280830in}}%
\pgfpathlineto{\pgfqpoint{1.046232in}{1.275975in}}%
\pgfpathlineto{\pgfqpoint{1.049027in}{1.271157in}}%
\pgfpathclose%
\pgfusepath{fill}%
\end{pgfscope}%
\begin{pgfscope}%
\pgfpathrectangle{\pgfqpoint{0.329460in}{0.284240in}}{\pgfqpoint{1.989680in}{1.989680in}}%
\pgfusepath{clip}%
\pgfsetbuttcap%
\pgfsetroundjoin%
\definecolor{currentfill}{rgb}{0.412913,0.803041,0.357269}%
\pgfsetfillcolor{currentfill}%
\pgfsetlinewidth{0.000000pt}%
\definecolor{currentstroke}{rgb}{0.000000,0.000000,0.000000}%
\pgfsetstrokecolor{currentstroke}%
\pgfsetdash{}{0pt}%
\pgfpathmoveto{\pgfqpoint{1.188067in}{1.615663in}}%
\pgfpathlineto{\pgfqpoint{1.184998in}{1.610890in}}%
\pgfpathlineto{\pgfqpoint{1.181932in}{1.606044in}}%
\pgfpathlineto{\pgfqpoint{1.178869in}{1.601127in}}%
\pgfpathlineto{\pgfqpoint{1.175809in}{1.596142in}}%
\pgfpathlineto{\pgfqpoint{1.178552in}{1.598760in}}%
\pgfpathlineto{\pgfqpoint{1.181462in}{1.601335in}}%
\pgfpathlineto{\pgfqpoint{1.184539in}{1.603863in}}%
\pgfpathlineto{\pgfqpoint{1.187777in}{1.606343in}}%
\pgfpathlineto{\pgfqpoint{1.190623in}{1.611143in}}%
\pgfpathlineto{\pgfqpoint{1.193473in}{1.615873in}}%
\pgfpathlineto{\pgfqpoint{1.196325in}{1.620533in}}%
\pgfpathlineto{\pgfqpoint{1.199181in}{1.625121in}}%
\pgfpathlineto{\pgfqpoint{1.196173in}{1.622821in}}%
\pgfpathlineto{\pgfqpoint{1.193316in}{1.620477in}}%
\pgfpathlineto{\pgfqpoint{1.190613in}{1.618090in}}%
\pgfpathlineto{\pgfqpoint{1.188067in}{1.615663in}}%
\pgfpathclose%
\pgfusepath{fill}%
\end{pgfscope}%
\begin{pgfscope}%
\pgfpathrectangle{\pgfqpoint{0.329460in}{0.284240in}}{\pgfqpoint{1.989680in}{1.989680in}}%
\pgfusepath{clip}%
\pgfsetbuttcap%
\pgfsetroundjoin%
\definecolor{currentfill}{rgb}{0.212395,0.359683,0.551710}%
\pgfsetfillcolor{currentfill}%
\pgfsetlinewidth{0.000000pt}%
\definecolor{currentstroke}{rgb}{0.000000,0.000000,0.000000}%
\pgfsetstrokecolor{currentstroke}%
\pgfsetdash{}{0pt}%
\pgfpathmoveto{\pgfqpoint{1.026664in}{1.161712in}}%
\pgfpathlineto{\pgfqpoint{1.023620in}{1.154368in}}%
\pgfpathlineto{\pgfqpoint{1.020578in}{1.147070in}}%
\pgfpathlineto{\pgfqpoint{1.017535in}{1.139821in}}%
\pgfpathlineto{\pgfqpoint{1.014494in}{1.132623in}}%
\pgfpathlineto{\pgfqpoint{1.009845in}{1.138123in}}%
\pgfpathlineto{\pgfqpoint{1.005549in}{1.143690in}}%
\pgfpathlineto{\pgfqpoint{1.001609in}{1.149319in}}%
\pgfpathlineto{\pgfqpoint{0.998027in}{1.155004in}}%
\pgfpathlineto{\pgfqpoint{1.001207in}{1.161975in}}%
\pgfpathlineto{\pgfqpoint{1.004387in}{1.168998in}}%
\pgfpathlineto{\pgfqpoint{1.007568in}{1.176070in}}%
\pgfpathlineto{\pgfqpoint{1.010749in}{1.183188in}}%
\pgfpathlineto{\pgfqpoint{1.014213in}{1.177732in}}%
\pgfpathlineto{\pgfqpoint{1.018022in}{1.172331in}}%
\pgfpathlineto{\pgfqpoint{1.022173in}{1.166989in}}%
\pgfpathlineto{\pgfqpoint{1.026664in}{1.161712in}}%
\pgfpathclose%
\pgfusepath{fill}%
\end{pgfscope}%
\begin{pgfscope}%
\pgfpathrectangle{\pgfqpoint{0.329460in}{0.284240in}}{\pgfqpoint{1.989680in}{1.989680in}}%
\pgfusepath{clip}%
\pgfsetbuttcap%
\pgfsetroundjoin%
\definecolor{currentfill}{rgb}{0.281477,0.755203,0.432552}%
\pgfsetfillcolor{currentfill}%
\pgfsetlinewidth{0.000000pt}%
\definecolor{currentstroke}{rgb}{0.000000,0.000000,0.000000}%
\pgfsetstrokecolor{currentstroke}%
\pgfsetdash{}{0pt}%
\pgfpathmoveto{\pgfqpoint{1.546734in}{1.566529in}}%
\pgfpathlineto{\pgfqpoint{1.549911in}{1.561070in}}%
\pgfpathlineto{\pgfqpoint{1.553086in}{1.555552in}}%
\pgfpathlineto{\pgfqpoint{1.556257in}{1.549978in}}%
\pgfpathlineto{\pgfqpoint{1.559424in}{1.544349in}}%
\pgfpathlineto{\pgfqpoint{1.561741in}{1.541169in}}%
\pgfpathlineto{\pgfqpoint{1.563851in}{1.537954in}}%
\pgfpathlineto{\pgfqpoint{1.565752in}{1.534706in}}%
\pgfpathlineto{\pgfqpoint{1.567442in}{1.531430in}}%
\pgfpathlineto{\pgfqpoint{1.564145in}{1.537268in}}%
\pgfpathlineto{\pgfqpoint{1.560845in}{1.543050in}}%
\pgfpathlineto{\pgfqpoint{1.557542in}{1.548776in}}%
\pgfpathlineto{\pgfqpoint{1.554236in}{1.554444in}}%
\pgfpathlineto{\pgfqpoint{1.552656in}{1.557509in}}%
\pgfpathlineto{\pgfqpoint{1.550877in}{1.560546in}}%
\pgfpathlineto{\pgfqpoint{1.548902in}{1.563554in}}%
\pgfpathlineto{\pgfqpoint{1.546734in}{1.566529in}}%
\pgfpathclose%
\pgfusepath{fill}%
\end{pgfscope}%
\begin{pgfscope}%
\pgfpathrectangle{\pgfqpoint{0.329460in}{0.284240in}}{\pgfqpoint{1.989680in}{1.989680in}}%
\pgfusepath{clip}%
\pgfsetbuttcap%
\pgfsetroundjoin%
\definecolor{currentfill}{rgb}{0.344074,0.780029,0.397381}%
\pgfsetfillcolor{currentfill}%
\pgfsetlinewidth{0.000000pt}%
\definecolor{currentstroke}{rgb}{0.000000,0.000000,0.000000}%
\pgfsetstrokecolor{currentstroke}%
\pgfsetdash{}{0pt}%
\pgfpathmoveto{\pgfqpoint{1.166579in}{1.585283in}}%
\pgfpathlineto{\pgfqpoint{1.163355in}{1.580031in}}%
\pgfpathlineto{\pgfqpoint{1.160135in}{1.574713in}}%
\pgfpathlineto{\pgfqpoint{1.156918in}{1.569331in}}%
\pgfpathlineto{\pgfqpoint{1.153704in}{1.563886in}}%
\pgfpathlineto{\pgfqpoint{1.155894in}{1.566857in}}%
\pgfpathlineto{\pgfqpoint{1.158276in}{1.569792in}}%
\pgfpathlineto{\pgfqpoint{1.160846in}{1.572689in}}%
\pgfpathlineto{\pgfqpoint{1.163602in}{1.575543in}}%
\pgfpathlineto{\pgfqpoint{1.166649in}{1.580788in}}%
\pgfpathlineto{\pgfqpoint{1.169699in}{1.585970in}}%
\pgfpathlineto{\pgfqpoint{1.172752in}{1.591089in}}%
\pgfpathlineto{\pgfqpoint{1.175809in}{1.596142in}}%
\pgfpathlineto{\pgfqpoint{1.173238in}{1.593482in}}%
\pgfpathlineto{\pgfqpoint{1.170841in}{1.590784in}}%
\pgfpathlineto{\pgfqpoint{1.168620in}{1.588050in}}%
\pgfpathlineto{\pgfqpoint{1.166579in}{1.585283in}}%
\pgfpathclose%
\pgfusepath{fill}%
\end{pgfscope}%
\begin{pgfscope}%
\pgfpathrectangle{\pgfqpoint{0.329460in}{0.284240in}}{\pgfqpoint{1.989680in}{1.989680in}}%
\pgfusepath{clip}%
\pgfsetbuttcap%
\pgfsetroundjoin%
\definecolor{currentfill}{rgb}{0.122606,0.585371,0.546557}%
\pgfsetfillcolor{currentfill}%
\pgfsetlinewidth{0.000000pt}%
\definecolor{currentstroke}{rgb}{0.000000,0.000000,0.000000}%
\pgfsetstrokecolor{currentstroke}%
\pgfsetdash{}{0pt}%
\pgfpathmoveto{\pgfqpoint{1.627165in}{1.398275in}}%
\pgfpathlineto{\pgfqpoint{1.630516in}{1.391375in}}%
\pgfpathlineto{\pgfqpoint{1.633864in}{1.384458in}}%
\pgfpathlineto{\pgfqpoint{1.637210in}{1.377527in}}%
\pgfpathlineto{\pgfqpoint{1.640553in}{1.370585in}}%
\pgfpathlineto{\pgfqpoint{1.640244in}{1.366064in}}%
\pgfpathlineto{\pgfqpoint{1.639646in}{1.361548in}}%
\pgfpathlineto{\pgfqpoint{1.638759in}{1.357040in}}%
\pgfpathlineto{\pgfqpoint{1.637583in}{1.352544in}}%
\pgfpathlineto{\pgfqpoint{1.634265in}{1.359716in}}%
\pgfpathlineto{\pgfqpoint{1.630944in}{1.366876in}}%
\pgfpathlineto{\pgfqpoint{1.627621in}{1.374021in}}%
\pgfpathlineto{\pgfqpoint{1.624295in}{1.381149in}}%
\pgfpathlineto{\pgfqpoint{1.625426in}{1.385416in}}%
\pgfpathlineto{\pgfqpoint{1.626282in}{1.389696in}}%
\pgfpathlineto{\pgfqpoint{1.626861in}{1.393984in}}%
\pgfpathlineto{\pgfqpoint{1.627165in}{1.398275in}}%
\pgfpathclose%
\pgfusepath{fill}%
\end{pgfscope}%
\begin{pgfscope}%
\pgfpathrectangle{\pgfqpoint{0.329460in}{0.284240in}}{\pgfqpoint{1.989680in}{1.989680in}}%
\pgfusepath{clip}%
\pgfsetbuttcap%
\pgfsetroundjoin%
\definecolor{currentfill}{rgb}{0.134692,0.658636,0.517649}%
\pgfsetfillcolor{currentfill}%
\pgfsetlinewidth{0.000000pt}%
\definecolor{currentstroke}{rgb}{0.000000,0.000000,0.000000}%
\pgfsetstrokecolor{currentstroke}%
\pgfsetdash{}{0pt}%
\pgfpathmoveto{\pgfqpoint{1.598940in}{1.467916in}}%
\pgfpathlineto{\pgfqpoint{1.602287in}{1.461456in}}%
\pgfpathlineto{\pgfqpoint{1.605631in}{1.454962in}}%
\pgfpathlineto{\pgfqpoint{1.608972in}{1.448435in}}%
\pgfpathlineto{\pgfqpoint{1.612309in}{1.441878in}}%
\pgfpathlineto{\pgfqpoint{1.613056in}{1.437838in}}%
\pgfpathlineto{\pgfqpoint{1.613543in}{1.433787in}}%
\pgfpathlineto{\pgfqpoint{1.613769in}{1.429727in}}%
\pgfpathlineto{\pgfqpoint{1.613734in}{1.425662in}}%
\pgfpathlineto{\pgfqpoint{1.610369in}{1.432444in}}%
\pgfpathlineto{\pgfqpoint{1.607001in}{1.439196in}}%
\pgfpathlineto{\pgfqpoint{1.603631in}{1.445915in}}%
\pgfpathlineto{\pgfqpoint{1.600258in}{1.452599in}}%
\pgfpathlineto{\pgfqpoint{1.600300in}{1.456438in}}%
\pgfpathlineto{\pgfqpoint{1.600094in}{1.460272in}}%
\pgfpathlineto{\pgfqpoint{1.599640in}{1.464100in}}%
\pgfpathlineto{\pgfqpoint{1.598940in}{1.467916in}}%
\pgfpathclose%
\pgfusepath{fill}%
\end{pgfscope}%
\begin{pgfscope}%
\pgfpathrectangle{\pgfqpoint{0.329460in}{0.284240in}}{\pgfqpoint{1.989680in}{1.989680in}}%
\pgfusepath{clip}%
\pgfsetbuttcap%
\pgfsetroundjoin%
\definecolor{currentfill}{rgb}{0.487026,0.823929,0.312321}%
\pgfsetfillcolor{currentfill}%
\pgfsetlinewidth{0.000000pt}%
\definecolor{currentstroke}{rgb}{0.000000,0.000000,0.000000}%
\pgfsetstrokecolor{currentstroke}%
\pgfsetdash{}{0pt}%
\pgfpathmoveto{\pgfqpoint{1.210634in}{1.642714in}}%
\pgfpathlineto{\pgfqpoint{1.207766in}{1.638432in}}%
\pgfpathlineto{\pgfqpoint{1.204901in}{1.634072in}}%
\pgfpathlineto{\pgfqpoint{1.202039in}{1.629634in}}%
\pgfpathlineto{\pgfqpoint{1.199181in}{1.625121in}}%
\pgfpathlineto{\pgfqpoint{1.202337in}{1.627373in}}%
\pgfpathlineto{\pgfqpoint{1.205637in}{1.629577in}}%
\pgfpathlineto{\pgfqpoint{1.209079in}{1.631729in}}%
\pgfpathlineto{\pgfqpoint{1.212659in}{1.633827in}}%
\pgfpathlineto{\pgfqpoint{1.215260in}{1.638171in}}%
\pgfpathlineto{\pgfqpoint{1.217865in}{1.642439in}}%
\pgfpathlineto{\pgfqpoint{1.220472in}{1.646629in}}%
\pgfpathlineto{\pgfqpoint{1.223082in}{1.650741in}}%
\pgfpathlineto{\pgfqpoint{1.219775in}{1.648806in}}%
\pgfpathlineto{\pgfqpoint{1.216596in}{1.646822in}}%
\pgfpathlineto{\pgfqpoint{1.213548in}{1.644790in}}%
\pgfpathlineto{\pgfqpoint{1.210634in}{1.642714in}}%
\pgfpathclose%
\pgfusepath{fill}%
\end{pgfscope}%
\begin{pgfscope}%
\pgfpathrectangle{\pgfqpoint{0.329460in}{0.284240in}}{\pgfqpoint{1.989680in}{1.989680in}}%
\pgfusepath{clip}%
\pgfsetbuttcap%
\pgfsetroundjoin%
\definecolor{currentfill}{rgb}{0.263663,0.237631,0.518762}%
\pgfsetfillcolor{currentfill}%
\pgfsetlinewidth{0.000000pt}%
\definecolor{currentstroke}{rgb}{0.000000,0.000000,0.000000}%
\pgfsetstrokecolor{currentstroke}%
\pgfsetdash{}{0pt}%
\pgfpathmoveto{\pgfqpoint{1.716635in}{1.082540in}}%
\pgfpathlineto{\pgfqpoint{1.719710in}{1.075993in}}%
\pgfpathlineto{\pgfqpoint{1.722786in}{1.069527in}}%
\pgfpathlineto{\pgfqpoint{1.725863in}{1.063147in}}%
\pgfpathlineto{\pgfqpoint{1.728940in}{1.056855in}}%
\pgfpathlineto{\pgfqpoint{1.723780in}{1.050691in}}%
\pgfpathlineto{\pgfqpoint{1.718233in}{1.044610in}}%
\pgfpathlineto{\pgfqpoint{1.712304in}{1.038617in}}%
\pgfpathlineto{\pgfqpoint{1.705997in}{1.032719in}}%
\pgfpathlineto{\pgfqpoint{1.703096in}{1.039231in}}%
\pgfpathlineto{\pgfqpoint{1.700196in}{1.045832in}}%
\pgfpathlineto{\pgfqpoint{1.697297in}{1.052518in}}%
\pgfpathlineto{\pgfqpoint{1.694398in}{1.059286in}}%
\pgfpathlineto{\pgfqpoint{1.700508in}{1.064968in}}%
\pgfpathlineto{\pgfqpoint{1.706255in}{1.070741in}}%
\pgfpathlineto{\pgfqpoint{1.711631in}{1.076601in}}%
\pgfpathlineto{\pgfqpoint{1.716635in}{1.082540in}}%
\pgfpathclose%
\pgfusepath{fill}%
\end{pgfscope}%
\begin{pgfscope}%
\pgfpathrectangle{\pgfqpoint{0.329460in}{0.284240in}}{\pgfqpoint{1.989680in}{1.989680in}}%
\pgfusepath{clip}%
\pgfsetbuttcap%
\pgfsetroundjoin%
\definecolor{currentfill}{rgb}{0.195860,0.395433,0.555276}%
\pgfsetfillcolor{currentfill}%
\pgfsetlinewidth{0.000000pt}%
\definecolor{currentstroke}{rgb}{0.000000,0.000000,0.000000}%
\pgfsetstrokecolor{currentstroke}%
\pgfsetdash{}{0pt}%
\pgfpathmoveto{\pgfqpoint{1.681580in}{1.216748in}}%
\pgfpathlineto{\pgfqpoint{1.684790in}{1.209526in}}%
\pgfpathlineto{\pgfqpoint{1.687998in}{1.202338in}}%
\pgfpathlineto{\pgfqpoint{1.691206in}{1.195188in}}%
\pgfpathlineto{\pgfqpoint{1.694412in}{1.188078in}}%
\pgfpathlineto{\pgfqpoint{1.691258in}{1.182579in}}%
\pgfpathlineto{\pgfqpoint{1.687756in}{1.177129in}}%
\pgfpathlineto{\pgfqpoint{1.683909in}{1.171734in}}%
\pgfpathlineto{\pgfqpoint{1.679720in}{1.166399in}}%
\pgfpathlineto{\pgfqpoint{1.676640in}{1.173736in}}%
\pgfpathlineto{\pgfqpoint{1.673560in}{1.181113in}}%
\pgfpathlineto{\pgfqpoint{1.670478in}{1.188527in}}%
\pgfpathlineto{\pgfqpoint{1.667396in}{1.195976in}}%
\pgfpathlineto{\pgfqpoint{1.671438in}{1.201087in}}%
\pgfpathlineto{\pgfqpoint{1.675152in}{1.206257in}}%
\pgfpathlineto{\pgfqpoint{1.678533in}{1.211478in}}%
\pgfpathlineto{\pgfqpoint{1.681580in}{1.216748in}}%
\pgfpathclose%
\pgfusepath{fill}%
\end{pgfscope}%
\begin{pgfscope}%
\pgfpathrectangle{\pgfqpoint{0.329460in}{0.284240in}}{\pgfqpoint{1.989680in}{1.989680in}}%
\pgfusepath{clip}%
\pgfsetbuttcap%
\pgfsetroundjoin%
\definecolor{currentfill}{rgb}{0.274128,0.199721,0.498911}%
\pgfsetfillcolor{currentfill}%
\pgfsetlinewidth{0.000000pt}%
\definecolor{currentstroke}{rgb}{0.000000,0.000000,0.000000}%
\pgfsetstrokecolor{currentstroke}%
\pgfsetdash{}{0pt}%
\pgfpathmoveto{\pgfqpoint{1.002296in}{1.027562in}}%
\pgfpathlineto{\pgfqpoint{0.999440in}{1.021094in}}%
\pgfpathlineto{\pgfqpoint{0.996583in}{1.014721in}}%
\pgfpathlineto{\pgfqpoint{0.993726in}{1.008447in}}%
\pgfpathlineto{\pgfqpoint{0.990867in}{1.002274in}}%
\pgfpathlineto{\pgfqpoint{0.984024in}{1.008295in}}%
\pgfpathlineto{\pgfqpoint{0.977565in}{1.014420in}}%
\pgfpathlineto{\pgfqpoint{0.971497in}{1.020643in}}%
\pgfpathlineto{\pgfqpoint{0.965825in}{1.026957in}}%
\pgfpathlineto{\pgfqpoint{0.968871in}{1.032912in}}%
\pgfpathlineto{\pgfqpoint{0.971916in}{1.038968in}}%
\pgfpathlineto{\pgfqpoint{0.974960in}{1.045122in}}%
\pgfpathlineto{\pgfqpoint{0.978003in}{1.051372in}}%
\pgfpathlineto{\pgfqpoint{0.983507in}{1.045281in}}%
\pgfpathlineto{\pgfqpoint{0.989394in}{1.039278in}}%
\pgfpathlineto{\pgfqpoint{0.995659in}{1.033369in}}%
\pgfpathlineto{\pgfqpoint{1.002296in}{1.027562in}}%
\pgfpathclose%
\pgfusepath{fill}%
\end{pgfscope}%
\begin{pgfscope}%
\pgfpathrectangle{\pgfqpoint{0.329460in}{0.284240in}}{\pgfqpoint{1.989680in}{1.989680in}}%
\pgfusepath{clip}%
\pgfsetbuttcap%
\pgfsetroundjoin%
\definecolor{currentfill}{rgb}{0.636902,0.856542,0.216620}%
\pgfsetfillcolor{currentfill}%
\pgfsetlinewidth{0.000000pt}%
\definecolor{currentstroke}{rgb}{0.000000,0.000000,0.000000}%
\pgfsetstrokecolor{currentstroke}%
\pgfsetdash{}{0pt}%
\pgfpathmoveto{\pgfqpoint{1.414495in}{1.697108in}}%
\pgfpathlineto{\pgfqpoint{1.416046in}{1.694095in}}%
\pgfpathlineto{\pgfqpoint{1.417596in}{1.690992in}}%
\pgfpathlineto{\pgfqpoint{1.419143in}{1.687799in}}%
\pgfpathlineto{\pgfqpoint{1.420689in}{1.684519in}}%
\pgfpathlineto{\pgfqpoint{1.424931in}{1.683458in}}%
\pgfpathlineto{\pgfqpoint{1.429101in}{1.682333in}}%
\pgfpathlineto{\pgfqpoint{1.433196in}{1.681147in}}%
\pgfpathlineto{\pgfqpoint{1.437212in}{1.679901in}}%
\pgfpathlineto{\pgfqpoint{1.435297in}{1.683286in}}%
\pgfpathlineto{\pgfqpoint{1.433380in}{1.686583in}}%
\pgfpathlineto{\pgfqpoint{1.431460in}{1.689792in}}%
\pgfpathlineto{\pgfqpoint{1.429539in}{1.692910in}}%
\pgfpathlineto{\pgfqpoint{1.425883in}{1.694043in}}%
\pgfpathlineto{\pgfqpoint{1.422154in}{1.695121in}}%
\pgfpathlineto{\pgfqpoint{1.418357in}{1.696143in}}%
\pgfpathlineto{\pgfqpoint{1.414495in}{1.697108in}}%
\pgfpathclose%
\pgfusepath{fill}%
\end{pgfscope}%
\begin{pgfscope}%
\pgfpathrectangle{\pgfqpoint{0.329460in}{0.284240in}}{\pgfqpoint{1.989680in}{1.989680in}}%
\pgfusepath{clip}%
\pgfsetbuttcap%
\pgfsetroundjoin%
\definecolor{currentfill}{rgb}{0.272594,0.025563,0.353093}%
\pgfsetfillcolor{currentfill}%
\pgfsetlinewidth{0.000000pt}%
\definecolor{currentstroke}{rgb}{0.000000,0.000000,0.000000}%
\pgfsetstrokecolor{currentstroke}%
\pgfsetdash{}{0pt}%
\pgfpathmoveto{\pgfqpoint{1.867113in}{0.921244in}}%
\pgfpathlineto{\pgfqpoint{1.870381in}{0.922717in}}%
\pgfpathlineto{\pgfqpoint{1.873658in}{0.924458in}}%
\pgfpathlineto{\pgfqpoint{1.876944in}{0.926474in}}%
\pgfpathlineto{\pgfqpoint{1.880241in}{0.928769in}}%
\pgfpathlineto{\pgfqpoint{1.873214in}{0.919961in}}%
\pgfpathlineto{\pgfqpoint{1.865636in}{0.911265in}}%
\pgfpathlineto{\pgfqpoint{1.857514in}{0.902689in}}%
\pgfpathlineto{\pgfqpoint{1.848854in}{0.894243in}}%
\pgfpathlineto{\pgfqpoint{1.845737in}{0.892153in}}%
\pgfpathlineto{\pgfqpoint{1.842630in}{0.890342in}}%
\pgfpathlineto{\pgfqpoint{1.839532in}{0.888807in}}%
\pgfpathlineto{\pgfqpoint{1.836442in}{0.887542in}}%
\pgfpathlineto{\pgfqpoint{1.844902in}{0.895785in}}%
\pgfpathlineto{\pgfqpoint{1.852838in}{0.904157in}}%
\pgfpathlineto{\pgfqpoint{1.860244in}{0.912646in}}%
\pgfpathlineto{\pgfqpoint{1.867113in}{0.921244in}}%
\pgfpathclose%
\pgfusepath{fill}%
\end{pgfscope}%
\begin{pgfscope}%
\pgfpathrectangle{\pgfqpoint{0.329460in}{0.284240in}}{\pgfqpoint{1.989680in}{1.989680in}}%
\pgfusepath{clip}%
\pgfsetbuttcap%
\pgfsetroundjoin%
\definecolor{currentfill}{rgb}{0.565498,0.842430,0.262877}%
\pgfsetfillcolor{currentfill}%
\pgfsetlinewidth{0.000000pt}%
\definecolor{currentstroke}{rgb}{0.000000,0.000000,0.000000}%
\pgfsetstrokecolor{currentstroke}%
\pgfsetdash{}{0pt}%
\pgfpathmoveto{\pgfqpoint{1.452405in}{1.674335in}}%
\pgfpathlineto{\pgfqpoint{1.454658in}{1.670736in}}%
\pgfpathlineto{\pgfqpoint{1.456908in}{1.667053in}}%
\pgfpathlineto{\pgfqpoint{1.459155in}{1.663286in}}%
\pgfpathlineto{\pgfqpoint{1.461400in}{1.659438in}}%
\pgfpathlineto{\pgfqpoint{1.465282in}{1.657767in}}%
\pgfpathlineto{\pgfqpoint{1.469053in}{1.656040in}}%
\pgfpathlineto{\pgfqpoint{1.472710in}{1.654256in}}%
\pgfpathlineto{\pgfqpoint{1.476249in}{1.652419in}}%
\pgfpathlineto{\pgfqpoint{1.473698in}{1.656415in}}%
\pgfpathlineto{\pgfqpoint{1.471145in}{1.660329in}}%
\pgfpathlineto{\pgfqpoint{1.468589in}{1.664160in}}%
\pgfpathlineto{\pgfqpoint{1.466030in}{1.667906in}}%
\pgfpathlineto{\pgfqpoint{1.462783in}{1.669589in}}%
\pgfpathlineto{\pgfqpoint{1.459428in}{1.671222in}}%
\pgfpathlineto{\pgfqpoint{1.455967in}{1.672805in}}%
\pgfpathlineto{\pgfqpoint{1.452405in}{1.674335in}}%
\pgfpathclose%
\pgfusepath{fill}%
\end{pgfscope}%
\begin{pgfscope}%
\pgfpathrectangle{\pgfqpoint{0.329460in}{0.284240in}}{\pgfqpoint{1.989680in}{1.989680in}}%
\pgfusepath{clip}%
\pgfsetbuttcap%
\pgfsetroundjoin%
\definecolor{currentfill}{rgb}{0.282884,0.135920,0.453427}%
\pgfsetfillcolor{currentfill}%
\pgfsetlinewidth{0.000000pt}%
\definecolor{currentstroke}{rgb}{0.000000,0.000000,0.000000}%
\pgfsetstrokecolor{currentstroke}%
\pgfsetdash{}{0pt}%
\pgfpathmoveto{\pgfqpoint{0.801875in}{0.949597in}}%
\pgfpathlineto{\pgfqpoint{0.798512in}{0.954594in}}%
\pgfpathlineto{\pgfqpoint{0.795136in}{0.959924in}}%
\pgfpathlineto{\pgfqpoint{0.791747in}{0.965592in}}%
\pgfpathlineto{\pgfqpoint{0.788344in}{0.971605in}}%
\pgfpathlineto{\pgfqpoint{0.780904in}{0.981025in}}%
\pgfpathlineto{\pgfqpoint{0.774064in}{0.990550in}}%
\pgfpathlineto{\pgfqpoint{0.767830in}{1.000169in}}%
\pgfpathlineto{\pgfqpoint{0.762205in}{1.009872in}}%
\pgfpathlineto{\pgfqpoint{0.765745in}{1.003665in}}%
\pgfpathlineto{\pgfqpoint{0.769271in}{0.997800in}}%
\pgfpathlineto{\pgfqpoint{0.772784in}{0.992272in}}%
\pgfpathlineto{\pgfqpoint{0.776283in}{0.987075in}}%
\pgfpathlineto{\pgfqpoint{0.781794in}{0.977571in}}%
\pgfpathlineto{\pgfqpoint{0.787899in}{0.968150in}}%
\pgfpathlineto{\pgfqpoint{0.794594in}{0.958822in}}%
\pgfpathlineto{\pgfqpoint{0.801875in}{0.949597in}}%
\pgfpathclose%
\pgfusepath{fill}%
\end{pgfscope}%
\begin{pgfscope}%
\pgfpathrectangle{\pgfqpoint{0.329460in}{0.284240in}}{\pgfqpoint{1.989680in}{1.989680in}}%
\pgfusepath{clip}%
\pgfsetbuttcap%
\pgfsetroundjoin%
\definecolor{currentfill}{rgb}{0.147607,0.511733,0.557049}%
\pgfsetfillcolor{currentfill}%
\pgfsetlinewidth{0.000000pt}%
\definecolor{currentstroke}{rgb}{0.000000,0.000000,0.000000}%
\pgfsetstrokecolor{currentstroke}%
\pgfsetdash{}{0pt}%
\pgfpathmoveto{\pgfqpoint{1.650833in}{1.323784in}}%
\pgfpathlineto{\pgfqpoint{1.654140in}{1.316588in}}%
\pgfpathlineto{\pgfqpoint{1.657444in}{1.309394in}}%
\pgfpathlineto{\pgfqpoint{1.660747in}{1.302206in}}%
\pgfpathlineto{\pgfqpoint{1.664047in}{1.295026in}}%
\pgfpathlineto{\pgfqpoint{1.662468in}{1.290089in}}%
\pgfpathlineto{\pgfqpoint{1.660574in}{1.285174in}}%
\pgfpathlineto{\pgfqpoint{1.658367in}{1.280289in}}%
\pgfpathlineto{\pgfqpoint{1.655848in}{1.275437in}}%
\pgfpathlineto{\pgfqpoint{1.652624in}{1.282847in}}%
\pgfpathlineto{\pgfqpoint{1.649398in}{1.290264in}}%
\pgfpathlineto{\pgfqpoint{1.646171in}{1.297687in}}%
\pgfpathlineto{\pgfqpoint{1.642941in}{1.305111in}}%
\pgfpathlineto{\pgfqpoint{1.645363in}{1.309736in}}%
\pgfpathlineto{\pgfqpoint{1.647487in}{1.314392in}}%
\pgfpathlineto{\pgfqpoint{1.649310in}{1.319077in}}%
\pgfpathlineto{\pgfqpoint{1.650833in}{1.323784in}}%
\pgfpathclose%
\pgfusepath{fill}%
\end{pgfscope}%
\begin{pgfscope}%
\pgfpathrectangle{\pgfqpoint{0.329460in}{0.284240in}}{\pgfqpoint{1.989680in}{1.989680in}}%
\pgfusepath{clip}%
\pgfsetbuttcap%
\pgfsetroundjoin%
\definecolor{currentfill}{rgb}{0.281477,0.755203,0.432552}%
\pgfsetfillcolor{currentfill}%
\pgfsetlinewidth{0.000000pt}%
\definecolor{currentstroke}{rgb}{0.000000,0.000000,0.000000}%
\pgfsetstrokecolor{currentstroke}%
\pgfsetdash{}{0pt}%
\pgfpathmoveto{\pgfqpoint{1.146902in}{1.551699in}}%
\pgfpathlineto{\pgfqpoint{1.143574in}{1.545985in}}%
\pgfpathlineto{\pgfqpoint{1.140250in}{1.540211in}}%
\pgfpathlineto{\pgfqpoint{1.136928in}{1.534381in}}%
\pgfpathlineto{\pgfqpoint{1.133609in}{1.528496in}}%
\pgfpathlineto{\pgfqpoint{1.135110in}{1.531795in}}%
\pgfpathlineto{\pgfqpoint{1.136824in}{1.535069in}}%
\pgfpathlineto{\pgfqpoint{1.138749in}{1.538313in}}%
\pgfpathlineto{\pgfqpoint{1.140882in}{1.541524in}}%
\pgfpathlineto{\pgfqpoint{1.144082in}{1.547199in}}%
\pgfpathlineto{\pgfqpoint{1.147286in}{1.552818in}}%
\pgfpathlineto{\pgfqpoint{1.150494in}{1.558382in}}%
\pgfpathlineto{\pgfqpoint{1.153704in}{1.563886in}}%
\pgfpathlineto{\pgfqpoint{1.151708in}{1.560882in}}%
\pgfpathlineto{\pgfqpoint{1.149907in}{1.557847in}}%
\pgfpathlineto{\pgfqpoint{1.148305in}{1.554786in}}%
\pgfpathlineto{\pgfqpoint{1.146902in}{1.551699in}}%
\pgfpathclose%
\pgfusepath{fill}%
\end{pgfscope}%
\begin{pgfscope}%
\pgfpathrectangle{\pgfqpoint{0.329460in}{0.284240in}}{\pgfqpoint{1.989680in}{1.989680in}}%
\pgfusepath{clip}%
\pgfsetbuttcap%
\pgfsetroundjoin%
\definecolor{currentfill}{rgb}{0.268510,0.009605,0.335427}%
\pgfsetfillcolor{currentfill}%
\pgfsetlinewidth{0.000000pt}%
\definecolor{currentstroke}{rgb}{0.000000,0.000000,0.000000}%
\pgfsetstrokecolor{currentstroke}%
\pgfsetdash{}{0pt}%
\pgfpathmoveto{\pgfqpoint{0.885971in}{0.878050in}}%
\pgfpathlineto{\pgfqpoint{0.882962in}{0.878239in}}%
\pgfpathlineto{\pgfqpoint{0.879945in}{0.878678in}}%
\pgfpathlineto{\pgfqpoint{0.876920in}{0.879373in}}%
\pgfpathlineto{\pgfqpoint{0.873887in}{0.880328in}}%
\pgfpathlineto{\pgfqpoint{0.864967in}{0.888451in}}%
\pgfpathlineto{\pgfqpoint{0.856565in}{0.896709in}}%
\pgfpathlineto{\pgfqpoint{0.848688in}{0.905094in}}%
\pgfpathlineto{\pgfqpoint{0.841342in}{0.913596in}}%
\pgfpathlineto{\pgfqpoint{0.844566in}{0.912435in}}%
\pgfpathlineto{\pgfqpoint{0.847781in}{0.911534in}}%
\pgfpathlineto{\pgfqpoint{0.850989in}{0.910887in}}%
\pgfpathlineto{\pgfqpoint{0.854188in}{0.910491in}}%
\pgfpathlineto{\pgfqpoint{0.861365in}{0.902200in}}%
\pgfpathlineto{\pgfqpoint{0.869059in}{0.894023in}}%
\pgfpathlineto{\pgfqpoint{0.877263in}{0.885970in}}%
\pgfpathlineto{\pgfqpoint{0.885971in}{0.878050in}}%
\pgfpathclose%
\pgfusepath{fill}%
\end{pgfscope}%
\begin{pgfscope}%
\pgfpathrectangle{\pgfqpoint{0.329460in}{0.284240in}}{\pgfqpoint{1.989680in}{1.989680in}}%
\pgfusepath{clip}%
\pgfsetbuttcap%
\pgfsetroundjoin%
\definecolor{currentfill}{rgb}{0.636902,0.856542,0.216620}%
\pgfsetfillcolor{currentfill}%
\pgfsetlinewidth{0.000000pt}%
\definecolor{currentstroke}{rgb}{0.000000,0.000000,0.000000}%
\pgfsetstrokecolor{currentstroke}%
\pgfsetdash{}{0pt}%
\pgfpathmoveto{\pgfqpoint{1.269650in}{1.691858in}}%
\pgfpathlineto{\pgfqpoint{1.267650in}{1.688713in}}%
\pgfpathlineto{\pgfqpoint{1.265652in}{1.685478in}}%
\pgfpathlineto{\pgfqpoint{1.263657in}{1.682154in}}%
\pgfpathlineto{\pgfqpoint{1.261664in}{1.678743in}}%
\pgfpathlineto{\pgfqpoint{1.265606in}{1.680042in}}%
\pgfpathlineto{\pgfqpoint{1.269630in}{1.681282in}}%
\pgfpathlineto{\pgfqpoint{1.273734in}{1.682461in}}%
\pgfpathlineto{\pgfqpoint{1.277912in}{1.683579in}}%
\pgfpathlineto{\pgfqpoint{1.279542in}{1.686880in}}%
\pgfpathlineto{\pgfqpoint{1.281175in}{1.690094in}}%
\pgfpathlineto{\pgfqpoint{1.282809in}{1.693218in}}%
\pgfpathlineto{\pgfqpoint{1.284444in}{1.696253in}}%
\pgfpathlineto{\pgfqpoint{1.280640in}{1.695237in}}%
\pgfpathlineto{\pgfqpoint{1.276903in}{1.694165in}}%
\pgfpathlineto{\pgfqpoint{1.273239in}{1.693038in}}%
\pgfpathlineto{\pgfqpoint{1.269650in}{1.691858in}}%
\pgfpathclose%
\pgfusepath{fill}%
\end{pgfscope}%
\begin{pgfscope}%
\pgfpathrectangle{\pgfqpoint{0.329460in}{0.284240in}}{\pgfqpoint{1.989680in}{1.989680in}}%
\pgfusepath{clip}%
\pgfsetbuttcap%
\pgfsetroundjoin%
\definecolor{currentfill}{rgb}{0.565498,0.842430,0.262877}%
\pgfsetfillcolor{currentfill}%
\pgfsetlinewidth{0.000000pt}%
\definecolor{currentstroke}{rgb}{0.000000,0.000000,0.000000}%
\pgfsetstrokecolor{currentstroke}%
\pgfsetdash{}{0pt}%
\pgfpathmoveto{\pgfqpoint{1.233552in}{1.666369in}}%
\pgfpathlineto{\pgfqpoint{1.230930in}{1.662588in}}%
\pgfpathlineto{\pgfqpoint{1.228312in}{1.658722in}}%
\pgfpathlineto{\pgfqpoint{1.225696in}{1.654773in}}%
\pgfpathlineto{\pgfqpoint{1.223082in}{1.650741in}}%
\pgfpathlineto{\pgfqpoint{1.226514in}{1.652625in}}%
\pgfpathlineto{\pgfqpoint{1.230066in}{1.654457in}}%
\pgfpathlineto{\pgfqpoint{1.233736in}{1.656235in}}%
\pgfpathlineto{\pgfqpoint{1.237519in}{1.657956in}}%
\pgfpathlineto{\pgfqpoint{1.239835in}{1.661835in}}%
\pgfpathlineto{\pgfqpoint{1.242154in}{1.665633in}}%
\pgfpathlineto{\pgfqpoint{1.244475in}{1.669348in}}%
\pgfpathlineto{\pgfqpoint{1.246799in}{1.672977in}}%
\pgfpathlineto{\pgfqpoint{1.243327in}{1.671401in}}%
\pgfpathlineto{\pgfqpoint{1.239959in}{1.669773in}}%
\pgfpathlineto{\pgfqpoint{1.236700in}{1.668095in}}%
\pgfpathlineto{\pgfqpoint{1.233552in}{1.666369in}}%
\pgfpathclose%
\pgfusepath{fill}%
\end{pgfscope}%
\begin{pgfscope}%
\pgfpathrectangle{\pgfqpoint{0.329460in}{0.284240in}}{\pgfqpoint{1.989680in}{1.989680in}}%
\pgfusepath{clip}%
\pgfsetbuttcap%
\pgfsetroundjoin%
\definecolor{currentfill}{rgb}{0.220124,0.725509,0.466226}%
\pgfsetfillcolor{currentfill}%
\pgfsetlinewidth{0.000000pt}%
\definecolor{currentstroke}{rgb}{0.000000,0.000000,0.000000}%
\pgfsetstrokecolor{currentstroke}%
\pgfsetdash{}{0pt}%
\pgfpathmoveto{\pgfqpoint{1.567442in}{1.531430in}}%
\pgfpathlineto{\pgfqpoint{1.570735in}{1.525539in}}%
\pgfpathlineto{\pgfqpoint{1.574026in}{1.519598in}}%
\pgfpathlineto{\pgfqpoint{1.577313in}{1.513607in}}%
\pgfpathlineto{\pgfqpoint{1.580597in}{1.507570in}}%
\pgfpathlineto{\pgfqpoint{1.582171in}{1.504052in}}%
\pgfpathlineto{\pgfqpoint{1.583518in}{1.500511in}}%
\pgfpathlineto{\pgfqpoint{1.584636in}{1.496948in}}%
\pgfpathlineto{\pgfqpoint{1.585522in}{1.493368in}}%
\pgfpathlineto{\pgfqpoint{1.582160in}{1.499623in}}%
\pgfpathlineto{\pgfqpoint{1.578795in}{1.505831in}}%
\pgfpathlineto{\pgfqpoint{1.575427in}{1.511990in}}%
\pgfpathlineto{\pgfqpoint{1.572056in}{1.518097in}}%
\pgfpathlineto{\pgfqpoint{1.571227in}{1.521458in}}%
\pgfpathlineto{\pgfqpoint{1.570182in}{1.524802in}}%
\pgfpathlineto{\pgfqpoint{1.568919in}{1.528127in}}%
\pgfpathlineto{\pgfqpoint{1.567442in}{1.531430in}}%
\pgfpathclose%
\pgfusepath{fill}%
\end{pgfscope}%
\begin{pgfscope}%
\pgfpathrectangle{\pgfqpoint{0.329460in}{0.284240in}}{\pgfqpoint{1.989680in}{1.989680in}}%
\pgfusepath{clip}%
\pgfsetbuttcap%
\pgfsetroundjoin%
\definecolor{currentfill}{rgb}{0.122606,0.585371,0.546557}%
\pgfsetfillcolor{currentfill}%
\pgfsetlinewidth{0.000000pt}%
\definecolor{currentstroke}{rgb}{0.000000,0.000000,0.000000}%
\pgfsetstrokecolor{currentstroke}%
\pgfsetdash{}{0pt}%
\pgfpathmoveto{\pgfqpoint{1.079316in}{1.377370in}}%
\pgfpathlineto{\pgfqpoint{1.076003in}{1.370192in}}%
\pgfpathlineto{\pgfqpoint{1.072693in}{1.362996in}}%
\pgfpathlineto{\pgfqpoint{1.069385in}{1.355786in}}%
\pgfpathlineto{\pgfqpoint{1.066080in}{1.348563in}}%
\pgfpathlineto{\pgfqpoint{1.064647in}{1.353043in}}%
\pgfpathlineto{\pgfqpoint{1.063503in}{1.357540in}}%
\pgfpathlineto{\pgfqpoint{1.062648in}{1.362049in}}%
\pgfpathlineto{\pgfqpoint{1.062082in}{1.366567in}}%
\pgfpathlineto{\pgfqpoint{1.065425in}{1.373560in}}%
\pgfpathlineto{\pgfqpoint{1.068769in}{1.380542in}}%
\pgfpathlineto{\pgfqpoint{1.072117in}{1.387510in}}%
\pgfpathlineto{\pgfqpoint{1.075467in}{1.394460in}}%
\pgfpathlineto{\pgfqpoint{1.076015in}{1.390172in}}%
\pgfpathlineto{\pgfqpoint{1.076840in}{1.385891in}}%
\pgfpathlineto{\pgfqpoint{1.077940in}{1.381623in}}%
\pgfpathlineto{\pgfqpoint{1.079316in}{1.377370in}}%
\pgfpathclose%
\pgfusepath{fill}%
\end{pgfscope}%
\begin{pgfscope}%
\pgfpathrectangle{\pgfqpoint{0.329460in}{0.284240in}}{\pgfqpoint{1.989680in}{1.989680in}}%
\pgfusepath{clip}%
\pgfsetbuttcap%
\pgfsetroundjoin%
\definecolor{currentfill}{rgb}{0.134692,0.658636,0.517649}%
\pgfsetfillcolor{currentfill}%
\pgfsetlinewidth{0.000000pt}%
\definecolor{currentstroke}{rgb}{0.000000,0.000000,0.000000}%
\pgfsetstrokecolor{currentstroke}%
\pgfsetdash{}{0pt}%
\pgfpathmoveto{\pgfqpoint{1.102364in}{1.449187in}}%
\pgfpathlineto{\pgfqpoint{1.098992in}{1.442453in}}%
\pgfpathlineto{\pgfqpoint{1.095623in}{1.435684in}}%
\pgfpathlineto{\pgfqpoint{1.092257in}{1.428882in}}%
\pgfpathlineto{\pgfqpoint{1.088893in}{1.422050in}}%
\pgfpathlineto{\pgfqpoint{1.088625in}{1.426114in}}%
\pgfpathlineto{\pgfqpoint{1.088618in}{1.430178in}}%
\pgfpathlineto{\pgfqpoint{1.088873in}{1.434237in}}%
\pgfpathlineto{\pgfqpoint{1.089389in}{1.438288in}}%
\pgfpathlineto{\pgfqpoint{1.092737in}{1.444894in}}%
\pgfpathlineto{\pgfqpoint{1.096089in}{1.451471in}}%
\pgfpathlineto{\pgfqpoint{1.099443in}{1.458015in}}%
\pgfpathlineto{\pgfqpoint{1.102800in}{1.464525in}}%
\pgfpathlineto{\pgfqpoint{1.102319in}{1.460698in}}%
\pgfpathlineto{\pgfqpoint{1.102086in}{1.456864in}}%
\pgfpathlineto{\pgfqpoint{1.102101in}{1.453025in}}%
\pgfpathlineto{\pgfqpoint{1.102364in}{1.449187in}}%
\pgfpathclose%
\pgfusepath{fill}%
\end{pgfscope}%
\begin{pgfscope}%
\pgfpathrectangle{\pgfqpoint{0.329460in}{0.284240in}}{\pgfqpoint{1.989680in}{1.989680in}}%
\pgfusepath{clip}%
\pgfsetbuttcap%
\pgfsetroundjoin%
\definecolor{currentfill}{rgb}{0.699415,0.867117,0.175971}%
\pgfsetfillcolor{currentfill}%
\pgfsetlinewidth{0.000000pt}%
\definecolor{currentstroke}{rgb}{0.000000,0.000000,0.000000}%
\pgfsetstrokecolor{currentstroke}%
\pgfsetdash{}{0pt}%
\pgfpathmoveto{\pgfqpoint{1.335710in}{1.714238in}}%
\pgfpathlineto{\pgfqpoint{1.335288in}{1.711766in}}%
\pgfpathlineto{\pgfqpoint{1.334866in}{1.709199in}}%
\pgfpathlineto{\pgfqpoint{1.334445in}{1.706538in}}%
\pgfpathlineto{\pgfqpoint{1.334024in}{1.703784in}}%
\pgfpathlineto{\pgfqpoint{1.338361in}{1.704005in}}%
\pgfpathlineto{\pgfqpoint{1.342711in}{1.704162in}}%
\pgfpathlineto{\pgfqpoint{1.347068in}{1.704256in}}%
\pgfpathlineto{\pgfqpoint{1.351430in}{1.704284in}}%
\pgfpathlineto{\pgfqpoint{1.351424in}{1.707026in}}%
\pgfpathlineto{\pgfqpoint{1.351418in}{1.709675in}}%
\pgfpathlineto{\pgfqpoint{1.351412in}{1.712229in}}%
\pgfpathlineto{\pgfqpoint{1.351406in}{1.714688in}}%
\pgfpathlineto{\pgfqpoint{1.347473in}{1.714663in}}%
\pgfpathlineto{\pgfqpoint{1.343543in}{1.714579in}}%
\pgfpathlineto{\pgfqpoint{1.339621in}{1.714437in}}%
\pgfpathlineto{\pgfqpoint{1.335710in}{1.714238in}}%
\pgfpathclose%
\pgfusepath{fill}%
\end{pgfscope}%
\begin{pgfscope}%
\pgfpathrectangle{\pgfqpoint{0.329460in}{0.284240in}}{\pgfqpoint{1.989680in}{1.989680in}}%
\pgfusepath{clip}%
\pgfsetbuttcap%
\pgfsetroundjoin%
\definecolor{currentfill}{rgb}{0.699415,0.867117,0.175971}%
\pgfsetfillcolor{currentfill}%
\pgfsetlinewidth{0.000000pt}%
\definecolor{currentstroke}{rgb}{0.000000,0.000000,0.000000}%
\pgfsetstrokecolor{currentstroke}%
\pgfsetdash{}{0pt}%
\pgfpathmoveto{\pgfqpoint{1.351406in}{1.714688in}}%
\pgfpathlineto{\pgfqpoint{1.351412in}{1.712229in}}%
\pgfpathlineto{\pgfqpoint{1.351418in}{1.709675in}}%
\pgfpathlineto{\pgfqpoint{1.351424in}{1.707026in}}%
\pgfpathlineto{\pgfqpoint{1.351430in}{1.704284in}}%
\pgfpathlineto{\pgfqpoint{1.355791in}{1.704248in}}%
\pgfpathlineto{\pgfqpoint{1.360148in}{1.704148in}}%
\pgfpathlineto{\pgfqpoint{1.364496in}{1.703984in}}%
\pgfpathlineto{\pgfqpoint{1.368832in}{1.703755in}}%
\pgfpathlineto{\pgfqpoint{1.368399in}{1.706510in}}%
\pgfpathlineto{\pgfqpoint{1.367966in}{1.709172in}}%
\pgfpathlineto{\pgfqpoint{1.367533in}{1.711740in}}%
\pgfpathlineto{\pgfqpoint{1.367099in}{1.714212in}}%
\pgfpathlineto{\pgfqpoint{1.363189in}{1.714418in}}%
\pgfpathlineto{\pgfqpoint{1.359268in}{1.714566in}}%
\pgfpathlineto{\pgfqpoint{1.355339in}{1.714656in}}%
\pgfpathlineto{\pgfqpoint{1.351406in}{1.714688in}}%
\pgfpathclose%
\pgfusepath{fill}%
\end{pgfscope}%
\begin{pgfscope}%
\pgfpathrectangle{\pgfqpoint{0.329460in}{0.284240in}}{\pgfqpoint{1.989680in}{1.989680in}}%
\pgfusepath{clip}%
\pgfsetbuttcap%
\pgfsetroundjoin%
\definecolor{currentfill}{rgb}{0.233603,0.313828,0.543914}%
\pgfsetfillcolor{currentfill}%
\pgfsetlinewidth{0.000000pt}%
\definecolor{currentstroke}{rgb}{0.000000,0.000000,0.000000}%
\pgfsetstrokecolor{currentstroke}%
\pgfsetdash{}{0pt}%
\pgfpathmoveto{\pgfqpoint{0.733335in}{1.072603in}}%
\pgfpathlineto{\pgfqpoint{0.729651in}{1.082172in}}%
\pgfpathlineto{\pgfqpoint{0.725949in}{1.092146in}}%
\pgfpathlineto{\pgfqpoint{0.722229in}{1.102534in}}%
\pgfpathlineto{\pgfqpoint{0.718490in}{1.113341in}}%
\pgfpathlineto{\pgfqpoint{0.713176in}{1.123675in}}%
\pgfpathlineto{\pgfqpoint{0.708524in}{1.134073in}}%
\pgfpathlineto{\pgfqpoint{0.704536in}{1.144524in}}%
\pgfpathlineto{\pgfqpoint{0.701212in}{1.155017in}}%
\pgfpathlineto{\pgfqpoint{0.705028in}{1.144036in}}%
\pgfpathlineto{\pgfqpoint{0.708826in}{1.133471in}}%
\pgfpathlineto{\pgfqpoint{0.712605in}{1.123317in}}%
\pgfpathlineto{\pgfqpoint{0.716365in}{1.113567in}}%
\pgfpathlineto{\pgfqpoint{0.719636in}{1.103252in}}%
\pgfpathlineto{\pgfqpoint{0.723555in}{1.092979in}}%
\pgfpathlineto{\pgfqpoint{0.728122in}{1.082759in}}%
\pgfpathlineto{\pgfqpoint{0.733335in}{1.072603in}}%
\pgfpathclose%
\pgfusepath{fill}%
\end{pgfscope}%
\begin{pgfscope}%
\pgfpathrectangle{\pgfqpoint{0.329460in}{0.284240in}}{\pgfqpoint{1.989680in}{1.989680in}}%
\pgfusepath{clip}%
\pgfsetbuttcap%
\pgfsetroundjoin%
\definecolor{currentfill}{rgb}{0.248629,0.278775,0.534556}%
\pgfsetfillcolor{currentfill}%
\pgfsetlinewidth{0.000000pt}%
\definecolor{currentstroke}{rgb}{0.000000,0.000000,0.000000}%
\pgfsetstrokecolor{currentstroke}%
\pgfsetdash{}{0pt}%
\pgfpathmoveto{\pgfqpoint{1.704334in}{1.109491in}}%
\pgfpathlineto{\pgfqpoint{1.707409in}{1.102646in}}%
\pgfpathlineto{\pgfqpoint{1.710484in}{1.095870in}}%
\pgfpathlineto{\pgfqpoint{1.713559in}{1.089167in}}%
\pgfpathlineto{\pgfqpoint{1.716635in}{1.082540in}}%
\pgfpathlineto{\pgfqpoint{1.711631in}{1.076601in}}%
\pgfpathlineto{\pgfqpoint{1.706255in}{1.070741in}}%
\pgfpathlineto{\pgfqpoint{1.700508in}{1.064968in}}%
\pgfpathlineto{\pgfqpoint{1.694398in}{1.059286in}}%
\pgfpathlineto{\pgfqpoint{1.691499in}{1.066133in}}%
\pgfpathlineto{\pgfqpoint{1.688601in}{1.073056in}}%
\pgfpathlineto{\pgfqpoint{1.685703in}{1.080051in}}%
\pgfpathlineto{\pgfqpoint{1.682805in}{1.087116in}}%
\pgfpathlineto{\pgfqpoint{1.688719in}{1.092582in}}%
\pgfpathlineto{\pgfqpoint{1.694282in}{1.098138in}}%
\pgfpathlineto{\pgfqpoint{1.699488in}{1.103776in}}%
\pgfpathlineto{\pgfqpoint{1.704334in}{1.109491in}}%
\pgfpathclose%
\pgfusepath{fill}%
\end{pgfscope}%
\begin{pgfscope}%
\pgfpathrectangle{\pgfqpoint{0.329460in}{0.284240in}}{\pgfqpoint{1.989680in}{1.989680in}}%
\pgfusepath{clip}%
\pgfsetbuttcap%
\pgfsetroundjoin%
\definecolor{currentfill}{rgb}{0.699415,0.867117,0.175971}%
\pgfsetfillcolor{currentfill}%
\pgfsetlinewidth{0.000000pt}%
\definecolor{currentstroke}{rgb}{0.000000,0.000000,0.000000}%
\pgfsetstrokecolor{currentstroke}%
\pgfsetdash{}{0pt}%
\pgfpathmoveto{\pgfqpoint{1.320253in}{1.712867in}}%
\pgfpathlineto{\pgfqpoint{1.319409in}{1.710357in}}%
\pgfpathlineto{\pgfqpoint{1.318566in}{1.707752in}}%
\pgfpathlineto{\pgfqpoint{1.317724in}{1.705053in}}%
\pgfpathlineto{\pgfqpoint{1.316883in}{1.702260in}}%
\pgfpathlineto{\pgfqpoint{1.321129in}{1.702736in}}%
\pgfpathlineto{\pgfqpoint{1.325404in}{1.703149in}}%
\pgfpathlineto{\pgfqpoint{1.329704in}{1.703498in}}%
\pgfpathlineto{\pgfqpoint{1.334024in}{1.703784in}}%
\pgfpathlineto{\pgfqpoint{1.334445in}{1.706538in}}%
\pgfpathlineto{\pgfqpoint{1.334866in}{1.709199in}}%
\pgfpathlineto{\pgfqpoint{1.335288in}{1.711766in}}%
\pgfpathlineto{\pgfqpoint{1.335710in}{1.714238in}}%
\pgfpathlineto{\pgfqpoint{1.331814in}{1.713981in}}%
\pgfpathlineto{\pgfqpoint{1.327936in}{1.713666in}}%
\pgfpathlineto{\pgfqpoint{1.324081in}{1.713295in}}%
\pgfpathlineto{\pgfqpoint{1.320253in}{1.712867in}}%
\pgfpathclose%
\pgfusepath{fill}%
\end{pgfscope}%
\begin{pgfscope}%
\pgfpathrectangle{\pgfqpoint{0.329460in}{0.284240in}}{\pgfqpoint{1.989680in}{1.989680in}}%
\pgfusepath{clip}%
\pgfsetbuttcap%
\pgfsetroundjoin%
\definecolor{currentfill}{rgb}{0.699415,0.867117,0.175971}%
\pgfsetfillcolor{currentfill}%
\pgfsetlinewidth{0.000000pt}%
\definecolor{currentstroke}{rgb}{0.000000,0.000000,0.000000}%
\pgfsetstrokecolor{currentstroke}%
\pgfsetdash{}{0pt}%
\pgfpathmoveto{\pgfqpoint{1.367099in}{1.714212in}}%
\pgfpathlineto{\pgfqpoint{1.367533in}{1.711740in}}%
\pgfpathlineto{\pgfqpoint{1.367966in}{1.709172in}}%
\pgfpathlineto{\pgfqpoint{1.368399in}{1.706510in}}%
\pgfpathlineto{\pgfqpoint{1.368832in}{1.703755in}}%
\pgfpathlineto{\pgfqpoint{1.373150in}{1.703462in}}%
\pgfpathlineto{\pgfqpoint{1.377447in}{1.703106in}}%
\pgfpathlineto{\pgfqpoint{1.381719in}{1.702686in}}%
\pgfpathlineto{\pgfqpoint{1.385962in}{1.702204in}}%
\pgfpathlineto{\pgfqpoint{1.385109in}{1.704998in}}%
\pgfpathlineto{\pgfqpoint{1.384256in}{1.707698in}}%
\pgfpathlineto{\pgfqpoint{1.383401in}{1.710305in}}%
\pgfpathlineto{\pgfqpoint{1.382546in}{1.712816in}}%
\pgfpathlineto{\pgfqpoint{1.378720in}{1.713250in}}%
\pgfpathlineto{\pgfqpoint{1.374868in}{1.713628in}}%
\pgfpathlineto{\pgfqpoint{1.370993in}{1.713949in}}%
\pgfpathlineto{\pgfqpoint{1.367099in}{1.714212in}}%
\pgfpathclose%
\pgfusepath{fill}%
\end{pgfscope}%
\begin{pgfscope}%
\pgfpathrectangle{\pgfqpoint{0.329460in}{0.284240in}}{\pgfqpoint{1.989680in}{1.989680in}}%
\pgfusepath{clip}%
\pgfsetbuttcap%
\pgfsetroundjoin%
\definecolor{currentfill}{rgb}{0.263663,0.237631,0.518762}%
\pgfsetfillcolor{currentfill}%
\pgfsetlinewidth{0.000000pt}%
\definecolor{currentstroke}{rgb}{0.000000,0.000000,0.000000}%
\pgfsetstrokecolor{currentstroke}%
\pgfsetdash{}{0pt}%
\pgfpathmoveto{\pgfqpoint{1.013710in}{1.054318in}}%
\pgfpathlineto{\pgfqpoint{1.010858in}{1.047503in}}%
\pgfpathlineto{\pgfqpoint{1.008004in}{1.040769in}}%
\pgfpathlineto{\pgfqpoint{1.005150in}{1.034121in}}%
\pgfpathlineto{\pgfqpoint{1.002296in}{1.027562in}}%
\pgfpathlineto{\pgfqpoint{0.995659in}{1.033369in}}%
\pgfpathlineto{\pgfqpoint{0.989394in}{1.039278in}}%
\pgfpathlineto{\pgfqpoint{0.983507in}{1.045281in}}%
\pgfpathlineto{\pgfqpoint{0.978003in}{1.051372in}}%
\pgfpathlineto{\pgfqpoint{0.981045in}{1.057714in}}%
\pgfpathlineto{\pgfqpoint{0.984087in}{1.064144in}}%
\pgfpathlineto{\pgfqpoint{0.987128in}{1.070659in}}%
\pgfpathlineto{\pgfqpoint{0.990169in}{1.077257in}}%
\pgfpathlineto{\pgfqpoint{0.995505in}{1.071388in}}%
\pgfpathlineto{\pgfqpoint{1.001210in}{1.065605in}}%
\pgfpathlineto{\pgfqpoint{1.007281in}{1.059913in}}%
\pgfpathlineto{\pgfqpoint{1.013710in}{1.054318in}}%
\pgfpathclose%
\pgfusepath{fill}%
\end{pgfscope}%
\begin{pgfscope}%
\pgfpathrectangle{\pgfqpoint{0.329460in}{0.284240in}}{\pgfqpoint{1.989680in}{1.989680in}}%
\pgfusepath{clip}%
\pgfsetbuttcap%
\pgfsetroundjoin%
\definecolor{currentfill}{rgb}{0.195860,0.395433,0.555276}%
\pgfsetfillcolor{currentfill}%
\pgfsetlinewidth{0.000000pt}%
\definecolor{currentstroke}{rgb}{0.000000,0.000000,0.000000}%
\pgfsetstrokecolor{currentstroke}%
\pgfsetdash{}{0pt}%
\pgfpathmoveto{\pgfqpoint{1.038846in}{1.191486in}}%
\pgfpathlineto{\pgfqpoint{1.035799in}{1.183988in}}%
\pgfpathlineto{\pgfqpoint{1.032753in}{1.176525in}}%
\pgfpathlineto{\pgfqpoint{1.029708in}{1.169098in}}%
\pgfpathlineto{\pgfqpoint{1.026664in}{1.161712in}}%
\pgfpathlineto{\pgfqpoint{1.022173in}{1.166989in}}%
\pgfpathlineto{\pgfqpoint{1.018022in}{1.172331in}}%
\pgfpathlineto{\pgfqpoint{1.014213in}{1.177732in}}%
\pgfpathlineto{\pgfqpoint{1.010749in}{1.183188in}}%
\pgfpathlineto{\pgfqpoint{1.013932in}{1.190349in}}%
\pgfpathlineto{\pgfqpoint{1.017116in}{1.197550in}}%
\pgfpathlineto{\pgfqpoint{1.020301in}{1.204789in}}%
\pgfpathlineto{\pgfqpoint{1.023487in}{1.212062in}}%
\pgfpathlineto{\pgfqpoint{1.026831in}{1.206834in}}%
\pgfpathlineto{\pgfqpoint{1.030508in}{1.201659in}}%
\pgfpathlineto{\pgfqpoint{1.034514in}{1.196541in}}%
\pgfpathlineto{\pgfqpoint{1.038846in}{1.191486in}}%
\pgfpathclose%
\pgfusepath{fill}%
\end{pgfscope}%
\begin{pgfscope}%
\pgfpathrectangle{\pgfqpoint{0.329460in}{0.284240in}}{\pgfqpoint{1.989680in}{1.989680in}}%
\pgfusepath{clip}%
\pgfsetbuttcap%
\pgfsetroundjoin%
\definecolor{currentfill}{rgb}{0.276194,0.190074,0.493001}%
\pgfsetfillcolor{currentfill}%
\pgfsetlinewidth{0.000000pt}%
\definecolor{currentstroke}{rgb}{0.000000,0.000000,0.000000}%
\pgfsetstrokecolor{currentstroke}%
\pgfsetdash{}{0pt}%
\pgfpathmoveto{\pgfqpoint{1.944656in}{1.018559in}}%
\pgfpathlineto{\pgfqpoint{1.948233in}{1.025160in}}%
\pgfpathlineto{\pgfqpoint{1.951825in}{1.032114in}}%
\pgfpathlineto{\pgfqpoint{1.955431in}{1.039429in}}%
\pgfpathlineto{\pgfqpoint{1.959053in}{1.047111in}}%
\pgfpathlineto{\pgfqpoint{1.953871in}{1.037148in}}%
\pgfpathlineto{\pgfqpoint{1.948061in}{1.027261in}}%
\pgfpathlineto{\pgfqpoint{1.941628in}{1.017460in}}%
\pgfpathlineto{\pgfqpoint{1.934573in}{1.007756in}}%
\pgfpathlineto{\pgfqpoint{1.931076in}{1.000264in}}%
\pgfpathlineto{\pgfqpoint{1.927594in}{0.993141in}}%
\pgfpathlineto{\pgfqpoint{1.924127in}{0.986379in}}%
\pgfpathlineto{\pgfqpoint{1.920674in}{0.979973in}}%
\pgfpathlineto{\pgfqpoint{1.927581in}{0.989487in}}%
\pgfpathlineto{\pgfqpoint{1.933882in}{0.999096in}}%
\pgfpathlineto{\pgfqpoint{1.939575in}{1.008791in}}%
\pgfpathlineto{\pgfqpoint{1.944656in}{1.018559in}}%
\pgfpathclose%
\pgfusepath{fill}%
\end{pgfscope}%
\begin{pgfscope}%
\pgfpathrectangle{\pgfqpoint{0.329460in}{0.284240in}}{\pgfqpoint{1.989680in}{1.989680in}}%
\pgfusepath{clip}%
\pgfsetbuttcap%
\pgfsetroundjoin%
\definecolor{currentfill}{rgb}{0.147607,0.511733,0.557049}%
\pgfsetfillcolor{currentfill}%
\pgfsetlinewidth{0.000000pt}%
\definecolor{currentstroke}{rgb}{0.000000,0.000000,0.000000}%
\pgfsetstrokecolor{currentstroke}%
\pgfsetdash{}{0pt}%
\pgfpathmoveto{\pgfqpoint{1.061836in}{1.301032in}}%
\pgfpathlineto{\pgfqpoint{1.058631in}{1.293557in}}%
\pgfpathlineto{\pgfqpoint{1.055428in}{1.286085in}}%
\pgfpathlineto{\pgfqpoint{1.052226in}{1.278617in}}%
\pgfpathlineto{\pgfqpoint{1.049027in}{1.271157in}}%
\pgfpathlineto{\pgfqpoint{1.046232in}{1.275975in}}%
\pgfpathlineto{\pgfqpoint{1.043747in}{1.280830in}}%
\pgfpathlineto{\pgfqpoint{1.041575in}{1.285719in}}%
\pgfpathlineto{\pgfqpoint{1.039716in}{1.290636in}}%
\pgfpathlineto{\pgfqpoint{1.043004in}{1.297868in}}%
\pgfpathlineto{\pgfqpoint{1.046294in}{1.305107in}}%
\pgfpathlineto{\pgfqpoint{1.049586in}{1.312351in}}%
\pgfpathlineto{\pgfqpoint{1.052881in}{1.319599in}}%
\pgfpathlineto{\pgfqpoint{1.054671in}{1.314911in}}%
\pgfpathlineto{\pgfqpoint{1.056761in}{1.310251in}}%
\pgfpathlineto{\pgfqpoint{1.059150in}{1.305623in}}%
\pgfpathlineto{\pgfqpoint{1.061836in}{1.301032in}}%
\pgfpathclose%
\pgfusepath{fill}%
\end{pgfscope}%
\begin{pgfscope}%
\pgfpathrectangle{\pgfqpoint{0.329460in}{0.284240in}}{\pgfqpoint{1.989680in}{1.989680in}}%
\pgfusepath{clip}%
\pgfsetbuttcap%
\pgfsetroundjoin%
\definecolor{currentfill}{rgb}{0.220124,0.725509,0.466226}%
\pgfsetfillcolor{currentfill}%
\pgfsetlinewidth{0.000000pt}%
\definecolor{currentstroke}{rgb}{0.000000,0.000000,0.000000}%
\pgfsetstrokecolor{currentstroke}%
\pgfsetdash{}{0pt}%
\pgfpathmoveto{\pgfqpoint{1.129767in}{1.515099in}}%
\pgfpathlineto{\pgfqpoint{1.126386in}{1.508942in}}%
\pgfpathlineto{\pgfqpoint{1.123007in}{1.502735in}}%
\pgfpathlineto{\pgfqpoint{1.119632in}{1.496478in}}%
\pgfpathlineto{\pgfqpoint{1.116260in}{1.490173in}}%
\pgfpathlineto{\pgfqpoint{1.116940in}{1.493766in}}%
\pgfpathlineto{\pgfqpoint{1.117852in}{1.497345in}}%
\pgfpathlineto{\pgfqpoint{1.118995in}{1.500905in}}%
\pgfpathlineto{\pgfqpoint{1.120368in}{1.504444in}}%
\pgfpathlineto{\pgfqpoint{1.123673in}{1.510530in}}%
\pgfpathlineto{\pgfqpoint{1.126982in}{1.516568in}}%
\pgfpathlineto{\pgfqpoint{1.130294in}{1.522557in}}%
\pgfpathlineto{\pgfqpoint{1.133609in}{1.528496in}}%
\pgfpathlineto{\pgfqpoint{1.132323in}{1.525173in}}%
\pgfpathlineto{\pgfqpoint{1.131253in}{1.521830in}}%
\pgfpathlineto{\pgfqpoint{1.130401in}{1.518471in}}%
\pgfpathlineto{\pgfqpoint{1.129767in}{1.515099in}}%
\pgfpathclose%
\pgfusepath{fill}%
\end{pgfscope}%
\begin{pgfscope}%
\pgfpathrectangle{\pgfqpoint{0.329460in}{0.284240in}}{\pgfqpoint{1.989680in}{1.989680in}}%
\pgfusepath{clip}%
\pgfsetbuttcap%
\pgfsetroundjoin%
\definecolor{currentfill}{rgb}{0.636902,0.856542,0.216620}%
\pgfsetfillcolor{currentfill}%
\pgfsetlinewidth{0.000000pt}%
\definecolor{currentstroke}{rgb}{0.000000,0.000000,0.000000}%
\pgfsetstrokecolor{currentstroke}%
\pgfsetdash{}{0pt}%
\pgfpathmoveto{\pgfqpoint{1.429539in}{1.692910in}}%
\pgfpathlineto{\pgfqpoint{1.431460in}{1.689792in}}%
\pgfpathlineto{\pgfqpoint{1.433380in}{1.686583in}}%
\pgfpathlineto{\pgfqpoint{1.435297in}{1.683286in}}%
\pgfpathlineto{\pgfqpoint{1.437212in}{1.679901in}}%
\pgfpathlineto{\pgfqpoint{1.441144in}{1.678595in}}%
\pgfpathlineto{\pgfqpoint{1.444990in}{1.677231in}}%
\pgfpathlineto{\pgfqpoint{1.448745in}{1.675811in}}%
\pgfpathlineto{\pgfqpoint{1.452405in}{1.674335in}}%
\pgfpathlineto{\pgfqpoint{1.450150in}{1.677847in}}%
\pgfpathlineto{\pgfqpoint{1.447892in}{1.681272in}}%
\pgfpathlineto{\pgfqpoint{1.445632in}{1.684607in}}%
\pgfpathlineto{\pgfqpoint{1.443369in}{1.687852in}}%
\pgfpathlineto{\pgfqpoint{1.440038in}{1.689193in}}%
\pgfpathlineto{\pgfqpoint{1.436620in}{1.690484in}}%
\pgfpathlineto{\pgfqpoint{1.433119in}{1.691723in}}%
\pgfpathlineto{\pgfqpoint{1.429539in}{1.692910in}}%
\pgfpathclose%
\pgfusepath{fill}%
\end{pgfscope}%
\begin{pgfscope}%
\pgfpathrectangle{\pgfqpoint{0.329460in}{0.284240in}}{\pgfqpoint{1.989680in}{1.989680in}}%
\pgfusepath{clip}%
\pgfsetbuttcap%
\pgfsetroundjoin%
\definecolor{currentfill}{rgb}{0.699415,0.867117,0.175971}%
\pgfsetfillcolor{currentfill}%
\pgfsetlinewidth{0.000000pt}%
\definecolor{currentstroke}{rgb}{0.000000,0.000000,0.000000}%
\pgfsetstrokecolor{currentstroke}%
\pgfsetdash{}{0pt}%
\pgfpathmoveto{\pgfqpoint{1.382546in}{1.712816in}}%
\pgfpathlineto{\pgfqpoint{1.383401in}{1.710305in}}%
\pgfpathlineto{\pgfqpoint{1.384256in}{1.707698in}}%
\pgfpathlineto{\pgfqpoint{1.385109in}{1.704998in}}%
\pgfpathlineto{\pgfqpoint{1.385962in}{1.702204in}}%
\pgfpathlineto{\pgfqpoint{1.390171in}{1.701659in}}%
\pgfpathlineto{\pgfqpoint{1.394342in}{1.701052in}}%
\pgfpathlineto{\pgfqpoint{1.398472in}{1.700383in}}%
\pgfpathlineto{\pgfqpoint{1.397312in}{1.703222in}}%
\pgfpathlineto{\pgfqpoint{1.396151in}{1.705969in}}%
\pgfpathlineto{\pgfqpoint{1.394989in}{1.708621in}}%
\pgfpathlineto{\pgfqpoint{1.393826in}{1.711177in}}%
\pgfpathlineto{\pgfqpoint{1.390102in}{1.711779in}}%
\pgfpathlineto{\pgfqpoint{1.386341in}{1.712325in}}%
\pgfpathlineto{\pgfqpoint{1.382546in}{1.712816in}}%
\pgfpathclose%
\pgfusepath{fill}%
\end{pgfscope}%
\begin{pgfscope}%
\pgfpathrectangle{\pgfqpoint{0.329460in}{0.284240in}}{\pgfqpoint{1.989680in}{1.989680in}}%
\pgfusepath{clip}%
\pgfsetbuttcap%
\pgfsetroundjoin%
\definecolor{currentfill}{rgb}{0.699415,0.867117,0.175971}%
\pgfsetfillcolor{currentfill}%
\pgfsetlinewidth{0.000000pt}%
\definecolor{currentstroke}{rgb}{0.000000,0.000000,0.000000}%
\pgfsetstrokecolor{currentstroke}%
\pgfsetdash{}{0pt}%
\pgfpathmoveto{\pgfqpoint{1.305274in}{1.710596in}}%
\pgfpathlineto{\pgfqpoint{1.304021in}{1.708024in}}%
\pgfpathlineto{\pgfqpoint{1.302770in}{1.705355in}}%
\pgfpathlineto{\pgfqpoint{1.301519in}{1.702593in}}%
\pgfpathlineto{\pgfqpoint{1.300271in}{1.699738in}}%
\pgfpathlineto{\pgfqpoint{1.304360in}{1.700460in}}%
\pgfpathlineto{\pgfqpoint{1.308495in}{1.701122in}}%
\pgfpathlineto{\pgfqpoint{1.312670in}{1.701722in}}%
\pgfpathlineto{\pgfqpoint{1.316883in}{1.702260in}}%
\pgfpathlineto{\pgfqpoint{1.317724in}{1.705053in}}%
\pgfpathlineto{\pgfqpoint{1.318566in}{1.707752in}}%
\pgfpathlineto{\pgfqpoint{1.319409in}{1.710357in}}%
\pgfpathlineto{\pgfqpoint{1.320253in}{1.712867in}}%
\pgfpathlineto{\pgfqpoint{1.316454in}{1.712382in}}%
\pgfpathlineto{\pgfqpoint{1.312689in}{1.711842in}}%
\pgfpathlineto{\pgfqpoint{1.308961in}{1.711246in}}%
\pgfpathlineto{\pgfqpoint{1.305274in}{1.710596in}}%
\pgfpathclose%
\pgfusepath{fill}%
\end{pgfscope}%
\begin{pgfscope}%
\pgfpathrectangle{\pgfqpoint{0.329460in}{0.284240in}}{\pgfqpoint{1.989680in}{1.989680in}}%
\pgfusepath{clip}%
\pgfsetbuttcap%
\pgfsetroundjoin%
\definecolor{currentfill}{rgb}{0.412913,0.803041,0.357269}%
\pgfsetfillcolor{currentfill}%
\pgfsetlinewidth{0.000000pt}%
\definecolor{currentstroke}{rgb}{0.000000,0.000000,0.000000}%
\pgfsetstrokecolor{currentstroke}%
\pgfsetdash{}{0pt}%
\pgfpathmoveto{\pgfqpoint{1.512053in}{1.617822in}}%
\pgfpathlineto{\pgfqpoint{1.515078in}{1.613092in}}%
\pgfpathlineto{\pgfqpoint{1.518101in}{1.608288in}}%
\pgfpathlineto{\pgfqpoint{1.521120in}{1.603414in}}%
\pgfpathlineto{\pgfqpoint{1.524137in}{1.598471in}}%
\pgfpathlineto{\pgfqpoint{1.526860in}{1.595848in}}%
\pgfpathlineto{\pgfqpoint{1.529412in}{1.593184in}}%
\pgfpathlineto{\pgfqpoint{1.531790in}{1.590482in}}%
\pgfpathlineto{\pgfqpoint{1.533991in}{1.587744in}}%
\pgfpathlineto{\pgfqpoint{1.530797in}{1.592884in}}%
\pgfpathlineto{\pgfqpoint{1.527600in}{1.597954in}}%
\pgfpathlineto{\pgfqpoint{1.524400in}{1.602952in}}%
\pgfpathlineto{\pgfqpoint{1.521197in}{1.607878in}}%
\pgfpathlineto{\pgfqpoint{1.519155in}{1.610416in}}%
\pgfpathlineto{\pgfqpoint{1.516949in}{1.612921in}}%
\pgfpathlineto{\pgfqpoint{1.514581in}{1.615390in}}%
\pgfpathlineto{\pgfqpoint{1.512053in}{1.617822in}}%
\pgfpathclose%
\pgfusepath{fill}%
\end{pgfscope}%
\begin{pgfscope}%
\pgfpathrectangle{\pgfqpoint{0.329460in}{0.284240in}}{\pgfqpoint{1.989680in}{1.989680in}}%
\pgfusepath{clip}%
\pgfsetbuttcap%
\pgfsetroundjoin%
\definecolor{currentfill}{rgb}{0.487026,0.823929,0.312321}%
\pgfsetfillcolor{currentfill}%
\pgfsetlinewidth{0.000000pt}%
\definecolor{currentstroke}{rgb}{0.000000,0.000000,0.000000}%
\pgfsetstrokecolor{currentstroke}%
\pgfsetdash{}{0pt}%
\pgfpathmoveto{\pgfqpoint{1.489158in}{1.644562in}}%
\pgfpathlineto{\pgfqpoint{1.491972in}{1.640319in}}%
\pgfpathlineto{\pgfqpoint{1.494783in}{1.635998in}}%
\pgfpathlineto{\pgfqpoint{1.497591in}{1.631600in}}%
\pgfpathlineto{\pgfqpoint{1.500396in}{1.627125in}}%
\pgfpathlineto{\pgfqpoint{1.503536in}{1.624868in}}%
\pgfpathlineto{\pgfqpoint{1.506527in}{1.622563in}}%
\pgfpathlineto{\pgfqpoint{1.509367in}{1.620214in}}%
\pgfpathlineto{\pgfqpoint{1.512053in}{1.617822in}}%
\pgfpathlineto{\pgfqpoint{1.509024in}{1.622478in}}%
\pgfpathlineto{\pgfqpoint{1.505991in}{1.627059in}}%
\pgfpathlineto{\pgfqpoint{1.502956in}{1.631562in}}%
\pgfpathlineto{\pgfqpoint{1.499918in}{1.635986in}}%
\pgfpathlineto{\pgfqpoint{1.497440in}{1.638190in}}%
\pgfpathlineto{\pgfqpoint{1.494818in}{1.640356in}}%
\pgfpathlineto{\pgfqpoint{1.492057in}{1.642480in}}%
\pgfpathlineto{\pgfqpoint{1.489158in}{1.644562in}}%
\pgfpathclose%
\pgfusepath{fill}%
\end{pgfscope}%
\begin{pgfscope}%
\pgfpathrectangle{\pgfqpoint{0.329460in}{0.284240in}}{\pgfqpoint{1.989680in}{1.989680in}}%
\pgfusepath{clip}%
\pgfsetbuttcap%
\pgfsetroundjoin%
\definecolor{currentfill}{rgb}{0.179019,0.433756,0.557430}%
\pgfsetfillcolor{currentfill}%
\pgfsetlinewidth{0.000000pt}%
\definecolor{currentstroke}{rgb}{0.000000,0.000000,0.000000}%
\pgfsetstrokecolor{currentstroke}%
\pgfsetdash{}{0pt}%
\pgfpathmoveto{\pgfqpoint{1.668726in}{1.245926in}}%
\pgfpathlineto{\pgfqpoint{1.671942in}{1.238593in}}%
\pgfpathlineto{\pgfqpoint{1.675156in}{1.231284in}}%
\pgfpathlineto{\pgfqpoint{1.678369in}{1.224001in}}%
\pgfpathlineto{\pgfqpoint{1.681580in}{1.216748in}}%
\pgfpathlineto{\pgfqpoint{1.678533in}{1.211478in}}%
\pgfpathlineto{\pgfqpoint{1.675152in}{1.206257in}}%
\pgfpathlineto{\pgfqpoint{1.671438in}{1.201087in}}%
\pgfpathlineto{\pgfqpoint{1.667396in}{1.195976in}}%
\pgfpathlineto{\pgfqpoint{1.664313in}{1.203456in}}%
\pgfpathlineto{\pgfqpoint{1.661228in}{1.210965in}}%
\pgfpathlineto{\pgfqpoint{1.658142in}{1.218500in}}%
\pgfpathlineto{\pgfqpoint{1.655055in}{1.226057in}}%
\pgfpathlineto{\pgfqpoint{1.658950in}{1.230946in}}%
\pgfpathlineto{\pgfqpoint{1.662528in}{1.235890in}}%
\pgfpathlineto{\pgfqpoint{1.665788in}{1.240885in}}%
\pgfpathlineto{\pgfqpoint{1.668726in}{1.245926in}}%
\pgfpathclose%
\pgfusepath{fill}%
\end{pgfscope}%
\begin{pgfscope}%
\pgfpathrectangle{\pgfqpoint{0.329460in}{0.284240in}}{\pgfqpoint{1.989680in}{1.989680in}}%
\pgfusepath{clip}%
\pgfsetbuttcap%
\pgfsetroundjoin%
\definecolor{currentfill}{rgb}{0.636902,0.856542,0.216620}%
\pgfsetfillcolor{currentfill}%
\pgfsetlinewidth{0.000000pt}%
\definecolor{currentstroke}{rgb}{0.000000,0.000000,0.000000}%
\pgfsetstrokecolor{currentstroke}%
\pgfsetdash{}{0pt}%
\pgfpathmoveto{\pgfqpoint{1.256120in}{1.686619in}}%
\pgfpathlineto{\pgfqpoint{1.253786in}{1.683343in}}%
\pgfpathlineto{\pgfqpoint{1.251454in}{1.679977in}}%
\pgfpathlineto{\pgfqpoint{1.249125in}{1.676521in}}%
\pgfpathlineto{\pgfqpoint{1.246799in}{1.672977in}}%
\pgfpathlineto{\pgfqpoint{1.250372in}{1.674501in}}%
\pgfpathlineto{\pgfqpoint{1.254043in}{1.675971in}}%
\pgfpathlineto{\pgfqpoint{1.257808in}{1.677385in}}%
\pgfpathlineto{\pgfqpoint{1.261664in}{1.678743in}}%
\pgfpathlineto{\pgfqpoint{1.263657in}{1.682154in}}%
\pgfpathlineto{\pgfqpoint{1.265652in}{1.685478in}}%
\pgfpathlineto{\pgfqpoint{1.267650in}{1.688713in}}%
\pgfpathlineto{\pgfqpoint{1.269650in}{1.691858in}}%
\pgfpathlineto{\pgfqpoint{1.266140in}{1.690624in}}%
\pgfpathlineto{\pgfqpoint{1.262713in}{1.689339in}}%
\pgfpathlineto{\pgfqpoint{1.259372in}{1.688004in}}%
\pgfpathlineto{\pgfqpoint{1.256120in}{1.686619in}}%
\pgfpathclose%
\pgfusepath{fill}%
\end{pgfscope}%
\begin{pgfscope}%
\pgfpathrectangle{\pgfqpoint{0.329460in}{0.284240in}}{\pgfqpoint{1.989680in}{1.989680in}}%
\pgfusepath{clip}%
\pgfsetbuttcap%
\pgfsetroundjoin%
\definecolor{currentfill}{rgb}{0.120081,0.622161,0.534946}%
\pgfsetfillcolor{currentfill}%
\pgfsetlinewidth{0.000000pt}%
\definecolor{currentstroke}{rgb}{0.000000,0.000000,0.000000}%
\pgfsetstrokecolor{currentstroke}%
\pgfsetdash{}{0pt}%
\pgfpathmoveto{\pgfqpoint{1.613734in}{1.425662in}}%
\pgfpathlineto{\pgfqpoint{1.617096in}{1.418852in}}%
\pgfpathlineto{\pgfqpoint{1.620455in}{1.412016in}}%
\pgfpathlineto{\pgfqpoint{1.623811in}{1.405156in}}%
\pgfpathlineto{\pgfqpoint{1.627165in}{1.398275in}}%
\pgfpathlineto{\pgfqpoint{1.626861in}{1.393984in}}%
\pgfpathlineto{\pgfqpoint{1.626282in}{1.389696in}}%
\pgfpathlineto{\pgfqpoint{1.625426in}{1.385416in}}%
\pgfpathlineto{\pgfqpoint{1.624295in}{1.381149in}}%
\pgfpathlineto{\pgfqpoint{1.620967in}{1.388258in}}%
\pgfpathlineto{\pgfqpoint{1.617637in}{1.395345in}}%
\pgfpathlineto{\pgfqpoint{1.614304in}{1.402408in}}%
\pgfpathlineto{\pgfqpoint{1.610968in}{1.409445in}}%
\pgfpathlineto{\pgfqpoint{1.612053in}{1.413485in}}%
\pgfpathlineto{\pgfqpoint{1.612876in}{1.417538in}}%
\pgfpathlineto{\pgfqpoint{1.613436in}{1.421598in}}%
\pgfpathlineto{\pgfqpoint{1.613734in}{1.425662in}}%
\pgfpathclose%
\pgfusepath{fill}%
\end{pgfscope}%
\begin{pgfscope}%
\pgfpathrectangle{\pgfqpoint{0.329460in}{0.284240in}}{\pgfqpoint{1.989680in}{1.989680in}}%
\pgfusepath{clip}%
\pgfsetbuttcap%
\pgfsetroundjoin%
\definecolor{currentfill}{rgb}{0.272594,0.025563,0.353093}%
\pgfsetfillcolor{currentfill}%
\pgfsetlinewidth{0.000000pt}%
\definecolor{currentstroke}{rgb}{0.000000,0.000000,0.000000}%
\pgfsetstrokecolor{currentstroke}%
\pgfsetdash{}{0pt}%
\pgfpathmoveto{\pgfqpoint{0.873887in}{0.880328in}}%
\pgfpathlineto{\pgfqpoint{0.870845in}{0.881549in}}%
\pgfpathlineto{\pgfqpoint{0.867794in}{0.883040in}}%
\pgfpathlineto{\pgfqpoint{0.864734in}{0.884806in}}%
\pgfpathlineto{\pgfqpoint{0.861664in}{0.886853in}}%
\pgfpathlineto{\pgfqpoint{0.852532in}{0.895175in}}%
\pgfpathlineto{\pgfqpoint{0.843932in}{0.903636in}}%
\pgfpathlineto{\pgfqpoint{0.835870in}{0.912226in}}%
\pgfpathlineto{\pgfqpoint{0.828353in}{0.920935in}}%
\pgfpathlineto{\pgfqpoint{0.831615in}{0.918686in}}%
\pgfpathlineto{\pgfqpoint{0.834867in}{0.916717in}}%
\pgfpathlineto{\pgfqpoint{0.838109in}{0.915022in}}%
\pgfpathlineto{\pgfqpoint{0.841342in}{0.913596in}}%
\pgfpathlineto{\pgfqpoint{0.848688in}{0.905094in}}%
\pgfpathlineto{\pgfqpoint{0.856565in}{0.896709in}}%
\pgfpathlineto{\pgfqpoint{0.864967in}{0.888451in}}%
\pgfpathlineto{\pgfqpoint{0.873887in}{0.880328in}}%
\pgfpathclose%
\pgfusepath{fill}%
\end{pgfscope}%
\begin{pgfscope}%
\pgfpathrectangle{\pgfqpoint{0.329460in}{0.284240in}}{\pgfqpoint{1.989680in}{1.989680in}}%
\pgfusepath{clip}%
\pgfsetbuttcap%
\pgfsetroundjoin%
\definecolor{currentfill}{rgb}{0.166383,0.690856,0.496502}%
\pgfsetfillcolor{currentfill}%
\pgfsetlinewidth{0.000000pt}%
\definecolor{currentstroke}{rgb}{0.000000,0.000000,0.000000}%
\pgfsetstrokecolor{currentstroke}%
\pgfsetdash{}{0pt}%
\pgfpathmoveto{\pgfqpoint{1.585522in}{1.493368in}}%
\pgfpathlineto{\pgfqpoint{1.588882in}{1.487067in}}%
\pgfpathlineto{\pgfqpoint{1.592237in}{1.480724in}}%
\pgfpathlineto{\pgfqpoint{1.595590in}{1.474339in}}%
\pgfpathlineto{\pgfqpoint{1.598940in}{1.467916in}}%
\pgfpathlineto{\pgfqpoint{1.599640in}{1.464100in}}%
\pgfpathlineto{\pgfqpoint{1.600094in}{1.460272in}}%
\pgfpathlineto{\pgfqpoint{1.600300in}{1.456438in}}%
\pgfpathlineto{\pgfqpoint{1.600258in}{1.452599in}}%
\pgfpathlineto{\pgfqpoint{1.596881in}{1.459246in}}%
\pgfpathlineto{\pgfqpoint{1.593503in}{1.465854in}}%
\pgfpathlineto{\pgfqpoint{1.590121in}{1.472420in}}%
\pgfpathlineto{\pgfqpoint{1.586736in}{1.478943in}}%
\pgfpathlineto{\pgfqpoint{1.586784in}{1.482558in}}%
\pgfpathlineto{\pgfqpoint{1.586598in}{1.486169in}}%
\pgfpathlineto{\pgfqpoint{1.586177in}{1.489773in}}%
\pgfpathlineto{\pgfqpoint{1.585522in}{1.493368in}}%
\pgfpathclose%
\pgfusepath{fill}%
\end{pgfscope}%
\begin{pgfscope}%
\pgfpathrectangle{\pgfqpoint{0.329460in}{0.284240in}}{\pgfqpoint{1.989680in}{1.989680in}}%
\pgfusepath{clip}%
\pgfsetbuttcap%
\pgfsetroundjoin%
\definecolor{currentfill}{rgb}{0.344074,0.780029,0.397381}%
\pgfsetfillcolor{currentfill}%
\pgfsetlinewidth{0.000000pt}%
\definecolor{currentstroke}{rgb}{0.000000,0.000000,0.000000}%
\pgfsetstrokecolor{currentstroke}%
\pgfsetdash{}{0pt}%
\pgfpathmoveto{\pgfqpoint{1.533991in}{1.587744in}}%
\pgfpathlineto{\pgfqpoint{1.537181in}{1.582537in}}%
\pgfpathlineto{\pgfqpoint{1.540369in}{1.577265in}}%
\pgfpathlineto{\pgfqpoint{1.543553in}{1.571928in}}%
\pgfpathlineto{\pgfqpoint{1.546734in}{1.566529in}}%
\pgfpathlineto{\pgfqpoint{1.548902in}{1.563554in}}%
\pgfpathlineto{\pgfqpoint{1.550877in}{1.560546in}}%
\pgfpathlineto{\pgfqpoint{1.552656in}{1.557509in}}%
\pgfpathlineto{\pgfqpoint{1.554236in}{1.554444in}}%
\pgfpathlineto{\pgfqpoint{1.550927in}{1.560051in}}%
\pgfpathlineto{\pgfqpoint{1.547614in}{1.565595in}}%
\pgfpathlineto{\pgfqpoint{1.544298in}{1.571075in}}%
\pgfpathlineto{\pgfqpoint{1.540979in}{1.576488in}}%
\pgfpathlineto{\pgfqpoint{1.539508in}{1.579343in}}%
\pgfpathlineto{\pgfqpoint{1.537852in}{1.582172in}}%
\pgfpathlineto{\pgfqpoint{1.536012in}{1.584973in}}%
\pgfpathlineto{\pgfqpoint{1.533991in}{1.587744in}}%
\pgfpathclose%
\pgfusepath{fill}%
\end{pgfscope}%
\begin{pgfscope}%
\pgfpathrectangle{\pgfqpoint{0.329460in}{0.284240in}}{\pgfqpoint{1.989680in}{1.989680in}}%
\pgfusepath{clip}%
\pgfsetbuttcap%
\pgfsetroundjoin%
\definecolor{currentfill}{rgb}{0.277941,0.056324,0.381191}%
\pgfsetfillcolor{currentfill}%
\pgfsetlinewidth{0.000000pt}%
\definecolor{currentstroke}{rgb}{0.000000,0.000000,0.000000}%
\pgfsetstrokecolor{currentstroke}%
\pgfsetdash{}{0pt}%
\pgfpathmoveto{\pgfqpoint{1.880241in}{0.928769in}}%
\pgfpathlineto{\pgfqpoint{1.883547in}{0.931348in}}%
\pgfpathlineto{\pgfqpoint{1.886864in}{0.934216in}}%
\pgfpathlineto{\pgfqpoint{1.890191in}{0.937378in}}%
\pgfpathlineto{\pgfqpoint{1.893530in}{0.940840in}}%
\pgfpathlineto{\pgfqpoint{1.886344in}{0.931828in}}%
\pgfpathlineto{\pgfqpoint{1.878594in}{0.922928in}}%
\pgfpathlineto{\pgfqpoint{1.870284in}{0.914152in}}%
\pgfpathlineto{\pgfqpoint{1.861422in}{0.905508in}}%
\pgfpathlineto{\pgfqpoint{1.858265in}{0.902246in}}%
\pgfpathlineto{\pgfqpoint{1.855118in}{0.899285in}}%
\pgfpathlineto{\pgfqpoint{1.851981in}{0.896619in}}%
\pgfpathlineto{\pgfqpoint{1.848854in}{0.894243in}}%
\pgfpathlineto{\pgfqpoint{1.857514in}{0.902689in}}%
\pgfpathlineto{\pgfqpoint{1.865636in}{0.911265in}}%
\pgfpathlineto{\pgfqpoint{1.873214in}{0.919961in}}%
\pgfpathlineto{\pgfqpoint{1.880241in}{0.928769in}}%
\pgfpathclose%
\pgfusepath{fill}%
\end{pgfscope}%
\begin{pgfscope}%
\pgfpathrectangle{\pgfqpoint{0.329460in}{0.284240in}}{\pgfqpoint{1.989680in}{1.989680in}}%
\pgfusepath{clip}%
\pgfsetbuttcap%
\pgfsetroundjoin%
\definecolor{currentfill}{rgb}{0.565498,0.842430,0.262877}%
\pgfsetfillcolor{currentfill}%
\pgfsetlinewidth{0.000000pt}%
\definecolor{currentstroke}{rgb}{0.000000,0.000000,0.000000}%
\pgfsetstrokecolor{currentstroke}%
\pgfsetdash{}{0pt}%
\pgfpathmoveto{\pgfqpoint{1.466030in}{1.667906in}}%
\pgfpathlineto{\pgfqpoint{1.468589in}{1.664160in}}%
\pgfpathlineto{\pgfqpoint{1.471145in}{1.660329in}}%
\pgfpathlineto{\pgfqpoint{1.473698in}{1.656415in}}%
\pgfpathlineto{\pgfqpoint{1.476249in}{1.652419in}}%
\pgfpathlineto{\pgfqpoint{1.479666in}{1.650529in}}%
\pgfpathlineto{\pgfqpoint{1.482959in}{1.648588in}}%
\pgfpathlineto{\pgfqpoint{1.486124in}{1.646598in}}%
\pgfpathlineto{\pgfqpoint{1.489158in}{1.644562in}}%
\pgfpathlineto{\pgfqpoint{1.486340in}{1.648724in}}%
\pgfpathlineto{\pgfqpoint{1.483520in}{1.652803in}}%
\pgfpathlineto{\pgfqpoint{1.480697in}{1.656799in}}%
\pgfpathlineto{\pgfqpoint{1.477871in}{1.660711in}}%
\pgfpathlineto{\pgfqpoint{1.475089in}{1.662575in}}%
\pgfpathlineto{\pgfqpoint{1.472186in}{1.664397in}}%
\pgfpathlineto{\pgfqpoint{1.469166in}{1.666175in}}%
\pgfpathlineto{\pgfqpoint{1.466030in}{1.667906in}}%
\pgfpathclose%
\pgfusepath{fill}%
\end{pgfscope}%
\begin{pgfscope}%
\pgfpathrectangle{\pgfqpoint{0.329460in}{0.284240in}}{\pgfqpoint{1.989680in}{1.989680in}}%
\pgfusepath{clip}%
\pgfsetbuttcap%
\pgfsetroundjoin%
\definecolor{currentfill}{rgb}{0.699415,0.867117,0.175971}%
\pgfsetfillcolor{currentfill}%
\pgfsetlinewidth{0.000000pt}%
\definecolor{currentstroke}{rgb}{0.000000,0.000000,0.000000}%
\pgfsetstrokecolor{currentstroke}%
\pgfsetdash{}{0pt}%
\pgfpathmoveto{\pgfqpoint{1.393826in}{1.711177in}}%
\pgfpathlineto{\pgfqpoint{1.394989in}{1.708621in}}%
\pgfpathlineto{\pgfqpoint{1.396151in}{1.705969in}}%
\pgfpathlineto{\pgfqpoint{1.397312in}{1.703222in}}%
\pgfpathlineto{\pgfqpoint{1.398472in}{1.700383in}}%
\pgfpathlineto{\pgfqpoint{1.402556in}{1.699654in}}%
\pgfpathlineto{\pgfqpoint{1.406590in}{1.698864in}}%
\pgfpathlineto{\pgfqpoint{1.410571in}{1.698015in}}%
\pgfpathlineto{\pgfqpoint{1.414495in}{1.697108in}}%
\pgfpathlineto{\pgfqpoint{1.412941in}{1.700029in}}%
\pgfpathlineto{\pgfqpoint{1.411386in}{1.702857in}}%
\pgfpathlineto{\pgfqpoint{1.409829in}{1.705591in}}%
\pgfpathlineto{\pgfqpoint{1.408271in}{1.708229in}}%
\pgfpathlineto{\pgfqpoint{1.404734in}{1.709046in}}%
\pgfpathlineto{\pgfqpoint{1.401145in}{1.709810in}}%
\pgfpathlineto{\pgfqpoint{1.397508in}{1.710520in}}%
\pgfpathlineto{\pgfqpoint{1.393826in}{1.711177in}}%
\pgfpathclose%
\pgfusepath{fill}%
\end{pgfscope}%
\begin{pgfscope}%
\pgfpathrectangle{\pgfqpoint{0.329460in}{0.284240in}}{\pgfqpoint{1.989680in}{1.989680in}}%
\pgfusepath{clip}%
\pgfsetbuttcap%
\pgfsetroundjoin%
\definecolor{currentfill}{rgb}{0.487026,0.823929,0.312321}%
\pgfsetfillcolor{currentfill}%
\pgfsetlinewidth{0.000000pt}%
\definecolor{currentstroke}{rgb}{0.000000,0.000000,0.000000}%
\pgfsetstrokecolor{currentstroke}%
\pgfsetdash{}{0pt}%
\pgfpathmoveto{\pgfqpoint{1.200376in}{1.633995in}}%
\pgfpathlineto{\pgfqpoint{1.197294in}{1.629529in}}%
\pgfpathlineto{\pgfqpoint{1.194216in}{1.624984in}}%
\pgfpathlineto{\pgfqpoint{1.191140in}{1.620361in}}%
\pgfpathlineto{\pgfqpoint{1.188067in}{1.615663in}}%
\pgfpathlineto{\pgfqpoint{1.190613in}{1.618090in}}%
\pgfpathlineto{\pgfqpoint{1.193316in}{1.620477in}}%
\pgfpathlineto{\pgfqpoint{1.196173in}{1.622821in}}%
\pgfpathlineto{\pgfqpoint{1.199181in}{1.625121in}}%
\pgfpathlineto{\pgfqpoint{1.202039in}{1.629634in}}%
\pgfpathlineto{\pgfqpoint{1.204901in}{1.634072in}}%
\pgfpathlineto{\pgfqpoint{1.207766in}{1.638432in}}%
\pgfpathlineto{\pgfqpoint{1.210634in}{1.642714in}}%
\pgfpathlineto{\pgfqpoint{1.207857in}{1.640594in}}%
\pgfpathlineto{\pgfqpoint{1.205220in}{1.638433in}}%
\pgfpathlineto{\pgfqpoint{1.202725in}{1.636232in}}%
\pgfpathlineto{\pgfqpoint{1.200376in}{1.633995in}}%
\pgfpathclose%
\pgfusepath{fill}%
\end{pgfscope}%
\begin{pgfscope}%
\pgfpathrectangle{\pgfqpoint{0.329460in}{0.284240in}}{\pgfqpoint{1.989680in}{1.989680in}}%
\pgfusepath{clip}%
\pgfsetbuttcap%
\pgfsetroundjoin%
\definecolor{currentfill}{rgb}{0.412913,0.803041,0.357269}%
\pgfsetfillcolor{currentfill}%
\pgfsetlinewidth{0.000000pt}%
\definecolor{currentstroke}{rgb}{0.000000,0.000000,0.000000}%
\pgfsetstrokecolor{currentstroke}%
\pgfsetdash{}{0pt}%
\pgfpathmoveto{\pgfqpoint{1.179504in}{1.605597in}}%
\pgfpathlineto{\pgfqpoint{1.176268in}{1.600626in}}%
\pgfpathlineto{\pgfqpoint{1.173035in}{1.595582in}}%
\pgfpathlineto{\pgfqpoint{1.169805in}{1.590467in}}%
\pgfpathlineto{\pgfqpoint{1.166579in}{1.585283in}}%
\pgfpathlineto{\pgfqpoint{1.168620in}{1.588050in}}%
\pgfpathlineto{\pgfqpoint{1.170841in}{1.590784in}}%
\pgfpathlineto{\pgfqpoint{1.173238in}{1.593482in}}%
\pgfpathlineto{\pgfqpoint{1.175809in}{1.596142in}}%
\pgfpathlineto{\pgfqpoint{1.178869in}{1.601127in}}%
\pgfpathlineto{\pgfqpoint{1.181932in}{1.606044in}}%
\pgfpathlineto{\pgfqpoint{1.184998in}{1.610890in}}%
\pgfpathlineto{\pgfqpoint{1.188067in}{1.615663in}}%
\pgfpathlineto{\pgfqpoint{1.185681in}{1.613197in}}%
\pgfpathlineto{\pgfqpoint{1.183457in}{1.610696in}}%
\pgfpathlineto{\pgfqpoint{1.181397in}{1.608162in}}%
\pgfpathlineto{\pgfqpoint{1.179504in}{1.605597in}}%
\pgfpathclose%
\pgfusepath{fill}%
\end{pgfscope}%
\begin{pgfscope}%
\pgfpathrectangle{\pgfqpoint{0.329460in}{0.284240in}}{\pgfqpoint{1.989680in}{1.989680in}}%
\pgfusepath{clip}%
\pgfsetbuttcap%
\pgfsetroundjoin%
\definecolor{currentfill}{rgb}{0.133743,0.548535,0.553541}%
\pgfsetfillcolor{currentfill}%
\pgfsetlinewidth{0.000000pt}%
\definecolor{currentstroke}{rgb}{0.000000,0.000000,0.000000}%
\pgfsetstrokecolor{currentstroke}%
\pgfsetdash{}{0pt}%
\pgfpathmoveto{\pgfqpoint{1.637583in}{1.352544in}}%
\pgfpathlineto{\pgfqpoint{1.640899in}{1.345363in}}%
\pgfpathlineto{\pgfqpoint{1.644213in}{1.338174in}}%
\pgfpathlineto{\pgfqpoint{1.647524in}{1.330980in}}%
\pgfpathlineto{\pgfqpoint{1.650833in}{1.323784in}}%
\pgfpathlineto{\pgfqpoint{1.649310in}{1.319077in}}%
\pgfpathlineto{\pgfqpoint{1.647487in}{1.314392in}}%
\pgfpathlineto{\pgfqpoint{1.645363in}{1.309736in}}%
\pgfpathlineto{\pgfqpoint{1.642941in}{1.305111in}}%
\pgfpathlineto{\pgfqpoint{1.639710in}{1.312536in}}%
\pgfpathlineto{\pgfqpoint{1.636476in}{1.319957in}}%
\pgfpathlineto{\pgfqpoint{1.633241in}{1.327374in}}%
\pgfpathlineto{\pgfqpoint{1.630003in}{1.334783in}}%
\pgfpathlineto{\pgfqpoint{1.632327in}{1.339181in}}%
\pgfpathlineto{\pgfqpoint{1.634366in}{1.343611in}}%
\pgfpathlineto{\pgfqpoint{1.636118in}{1.348066in}}%
\pgfpathlineto{\pgfqpoint{1.637583in}{1.352544in}}%
\pgfpathclose%
\pgfusepath{fill}%
\end{pgfscope}%
\begin{pgfscope}%
\pgfpathrectangle{\pgfqpoint{0.329460in}{0.284240in}}{\pgfqpoint{1.989680in}{1.989680in}}%
\pgfusepath{clip}%
\pgfsetbuttcap%
\pgfsetroundjoin%
\definecolor{currentfill}{rgb}{0.248629,0.278775,0.534556}%
\pgfsetfillcolor{currentfill}%
\pgfsetlinewidth{0.000000pt}%
\definecolor{currentstroke}{rgb}{0.000000,0.000000,0.000000}%
\pgfsetstrokecolor{currentstroke}%
\pgfsetdash{}{0pt}%
\pgfpathmoveto{\pgfqpoint{1.025118in}{1.082337in}}%
\pgfpathlineto{\pgfqpoint{1.022266in}{1.075225in}}%
\pgfpathlineto{\pgfqpoint{1.019415in}{1.068182in}}%
\pgfpathlineto{\pgfqpoint{1.016563in}{1.061212in}}%
\pgfpathlineto{\pgfqpoint{1.013710in}{1.054318in}}%
\pgfpathlineto{\pgfqpoint{1.007281in}{1.059913in}}%
\pgfpathlineto{\pgfqpoint{1.001210in}{1.065605in}}%
\pgfpathlineto{\pgfqpoint{0.995505in}{1.071388in}}%
\pgfpathlineto{\pgfqpoint{0.990169in}{1.077257in}}%
\pgfpathlineto{\pgfqpoint{0.993210in}{1.083934in}}%
\pgfpathlineto{\pgfqpoint{0.996250in}{1.090687in}}%
\pgfpathlineto{\pgfqpoint{0.999290in}{1.097512in}}%
\pgfpathlineto{\pgfqpoint{1.002331in}{1.104407in}}%
\pgfpathlineto{\pgfqpoint{1.007497in}{1.098760in}}%
\pgfpathlineto{\pgfqpoint{1.013021in}{1.093195in}}%
\pgfpathlineto{\pgfqpoint{1.018896in}{1.087719in}}%
\pgfpathlineto{\pgfqpoint{1.025118in}{1.082337in}}%
\pgfpathclose%
\pgfusepath{fill}%
\end{pgfscope}%
\begin{pgfscope}%
\pgfpathrectangle{\pgfqpoint{0.329460in}{0.284240in}}{\pgfqpoint{1.989680in}{1.989680in}}%
\pgfusepath{clip}%
\pgfsetbuttcap%
\pgfsetroundjoin%
\definecolor{currentfill}{rgb}{0.231674,0.318106,0.544834}%
\pgfsetfillcolor{currentfill}%
\pgfsetlinewidth{0.000000pt}%
\definecolor{currentstroke}{rgb}{0.000000,0.000000,0.000000}%
\pgfsetstrokecolor{currentstroke}%
\pgfsetdash{}{0pt}%
\pgfpathmoveto{\pgfqpoint{1.692030in}{1.137508in}}%
\pgfpathlineto{\pgfqpoint{1.695107in}{1.130414in}}%
\pgfpathlineto{\pgfqpoint{1.698183in}{1.123378in}}%
\pgfpathlineto{\pgfqpoint{1.701258in}{1.116403in}}%
\pgfpathlineto{\pgfqpoint{1.704334in}{1.109491in}}%
\pgfpathlineto{\pgfqpoint{1.699488in}{1.103776in}}%
\pgfpathlineto{\pgfqpoint{1.694282in}{1.098138in}}%
\pgfpathlineto{\pgfqpoint{1.688719in}{1.092582in}}%
\pgfpathlineto{\pgfqpoint{1.682805in}{1.087116in}}%
\pgfpathlineto{\pgfqpoint{1.679907in}{1.094248in}}%
\pgfpathlineto{\pgfqpoint{1.677009in}{1.101442in}}%
\pgfpathlineto{\pgfqpoint{1.674111in}{1.108698in}}%
\pgfpathlineto{\pgfqpoint{1.671213in}{1.116010in}}%
\pgfpathlineto{\pgfqpoint{1.676930in}{1.121262in}}%
\pgfpathlineto{\pgfqpoint{1.682308in}{1.126599in}}%
\pgfpathlineto{\pgfqpoint{1.687343in}{1.132016in}}%
\pgfpathlineto{\pgfqpoint{1.692030in}{1.137508in}}%
\pgfpathclose%
\pgfusepath{fill}%
\end{pgfscope}%
\begin{pgfscope}%
\pgfpathrectangle{\pgfqpoint{0.329460in}{0.284240in}}{\pgfqpoint{1.989680in}{1.989680in}}%
\pgfusepath{clip}%
\pgfsetbuttcap%
\pgfsetroundjoin%
\definecolor{currentfill}{rgb}{0.699415,0.867117,0.175971}%
\pgfsetfillcolor{currentfill}%
\pgfsetlinewidth{0.000000pt}%
\definecolor{currentstroke}{rgb}{0.000000,0.000000,0.000000}%
\pgfsetstrokecolor{currentstroke}%
\pgfsetdash{}{0pt}%
\pgfpathmoveto{\pgfqpoint{1.291007in}{1.707460in}}%
\pgfpathlineto{\pgfqpoint{1.289363in}{1.704800in}}%
\pgfpathlineto{\pgfqpoint{1.287722in}{1.702045in}}%
\pgfpathlineto{\pgfqpoint{1.286082in}{1.699195in}}%
\pgfpathlineto{\pgfqpoint{1.284444in}{1.696253in}}%
\pgfpathlineto{\pgfqpoint{1.288314in}{1.697212in}}%
\pgfpathlineto{\pgfqpoint{1.292244in}{1.698113in}}%
\pgfpathlineto{\pgfqpoint{1.296231in}{1.698955in}}%
\pgfpathlineto{\pgfqpoint{1.300271in}{1.699738in}}%
\pgfpathlineto{\pgfqpoint{1.301519in}{1.702593in}}%
\pgfpathlineto{\pgfqpoint{1.302770in}{1.705355in}}%
\pgfpathlineto{\pgfqpoint{1.304021in}{1.708024in}}%
\pgfpathlineto{\pgfqpoint{1.305274in}{1.710596in}}%
\pgfpathlineto{\pgfqpoint{1.301632in}{1.709891in}}%
\pgfpathlineto{\pgfqpoint{1.298037in}{1.709133in}}%
\pgfpathlineto{\pgfqpoint{1.294494in}{1.708323in}}%
\pgfpathlineto{\pgfqpoint{1.291007in}{1.707460in}}%
\pgfpathclose%
\pgfusepath{fill}%
\end{pgfscope}%
\begin{pgfscope}%
\pgfpathrectangle{\pgfqpoint{0.329460in}{0.284240in}}{\pgfqpoint{1.989680in}{1.989680in}}%
\pgfusepath{clip}%
\pgfsetbuttcap%
\pgfsetroundjoin%
\definecolor{currentfill}{rgb}{0.274952,0.037752,0.364543}%
\pgfsetfillcolor{currentfill}%
\pgfsetlinewidth{0.000000pt}%
\definecolor{currentstroke}{rgb}{0.000000,0.000000,0.000000}%
\pgfsetstrokecolor{currentstroke}%
\pgfsetdash{}{0pt}%
\pgfpathmoveto{\pgfqpoint{1.764350in}{0.926243in}}%
\pgfpathlineto{\pgfqpoint{1.767295in}{0.922378in}}%
\pgfpathlineto{\pgfqpoint{1.770244in}{0.918680in}}%
\pgfpathlineto{\pgfqpoint{1.773196in}{0.915151in}}%
\pgfpathlineto{\pgfqpoint{1.776153in}{0.911797in}}%
\pgfpathlineto{\pgfqpoint{1.768228in}{0.904724in}}%
\pgfpathlineto{\pgfqpoint{1.759862in}{0.897781in}}%
\pgfpathlineto{\pgfqpoint{1.751060in}{0.890975in}}%
\pgfpathlineto{\pgfqpoint{1.741833in}{0.884314in}}%
\pgfpathlineto{\pgfqpoint{1.739104in}{0.887875in}}%
\pgfpathlineto{\pgfqpoint{1.736379in}{0.891609in}}%
\pgfpathlineto{\pgfqpoint{1.733657in}{0.895514in}}%
\pgfpathlineto{\pgfqpoint{1.730939in}{0.899585in}}%
\pgfpathlineto{\pgfqpoint{1.739921in}{0.906045in}}%
\pgfpathlineto{\pgfqpoint{1.748488in}{0.912647in}}%
\pgfpathlineto{\pgfqpoint{1.756634in}{0.919382in}}%
\pgfpathlineto{\pgfqpoint{1.764350in}{0.926243in}}%
\pgfpathclose%
\pgfusepath{fill}%
\end{pgfscope}%
\begin{pgfscope}%
\pgfpathrectangle{\pgfqpoint{0.329460in}{0.284240in}}{\pgfqpoint{1.989680in}{1.989680in}}%
\pgfusepath{clip}%
\pgfsetbuttcap%
\pgfsetroundjoin%
\definecolor{currentfill}{rgb}{0.279566,0.067836,0.391917}%
\pgfsetfillcolor{currentfill}%
\pgfsetlinewidth{0.000000pt}%
\definecolor{currentstroke}{rgb}{0.000000,0.000000,0.000000}%
\pgfsetstrokecolor{currentstroke}%
\pgfsetdash{}{0pt}%
\pgfpathmoveto{\pgfqpoint{1.752604in}{0.943284in}}%
\pgfpathlineto{\pgfqpoint{1.755536in}{0.938794in}}%
\pgfpathlineto{\pgfqpoint{1.758471in}{0.934454in}}%
\pgfpathlineto{\pgfqpoint{1.761409in}{0.930269in}}%
\pgfpathlineto{\pgfqpoint{1.764350in}{0.926243in}}%
\pgfpathlineto{\pgfqpoint{1.756634in}{0.919382in}}%
\pgfpathlineto{\pgfqpoint{1.748488in}{0.912647in}}%
\pgfpathlineto{\pgfqpoint{1.739921in}{0.906045in}}%
\pgfpathlineto{\pgfqpoint{1.730939in}{0.899585in}}%
\pgfpathlineto{\pgfqpoint{1.728225in}{0.903818in}}%
\pgfpathlineto{\pgfqpoint{1.725514in}{0.908210in}}%
\pgfpathlineto{\pgfqpoint{1.722806in}{0.912757in}}%
\pgfpathlineto{\pgfqpoint{1.720101in}{0.917455in}}%
\pgfpathlineto{\pgfqpoint{1.728837in}{0.923714in}}%
\pgfpathlineto{\pgfqpoint{1.737171in}{0.930110in}}%
\pgfpathlineto{\pgfqpoint{1.745096in}{0.936636in}}%
\pgfpathlineto{\pgfqpoint{1.752604in}{0.943284in}}%
\pgfpathclose%
\pgfusepath{fill}%
\end{pgfscope}%
\begin{pgfscope}%
\pgfpathrectangle{\pgfqpoint{0.329460in}{0.284240in}}{\pgfqpoint{1.989680in}{1.989680in}}%
\pgfusepath{clip}%
\pgfsetbuttcap%
\pgfsetroundjoin%
\definecolor{currentfill}{rgb}{0.565498,0.842430,0.262877}%
\pgfsetfillcolor{currentfill}%
\pgfsetlinewidth{0.000000pt}%
\definecolor{currentstroke}{rgb}{0.000000,0.000000,0.000000}%
\pgfsetstrokecolor{currentstroke}%
\pgfsetdash{}{0pt}%
\pgfpathmoveto{\pgfqpoint{1.222136in}{1.659018in}}%
\pgfpathlineto{\pgfqpoint{1.219256in}{1.655068in}}%
\pgfpathlineto{\pgfqpoint{1.216379in}{1.651033in}}%
\pgfpathlineto{\pgfqpoint{1.213505in}{1.646914in}}%
\pgfpathlineto{\pgfqpoint{1.210634in}{1.642714in}}%
\pgfpathlineto{\pgfqpoint{1.213548in}{1.644790in}}%
\pgfpathlineto{\pgfqpoint{1.216596in}{1.646822in}}%
\pgfpathlineto{\pgfqpoint{1.219775in}{1.648806in}}%
\pgfpathlineto{\pgfqpoint{1.223082in}{1.650741in}}%
\pgfpathlineto{\pgfqpoint{1.225696in}{1.654773in}}%
\pgfpathlineto{\pgfqpoint{1.228312in}{1.658722in}}%
\pgfpathlineto{\pgfqpoint{1.230930in}{1.662588in}}%
\pgfpathlineto{\pgfqpoint{1.233552in}{1.666369in}}%
\pgfpathlineto{\pgfqpoint{1.230519in}{1.664597in}}%
\pgfpathlineto{\pgfqpoint{1.227603in}{1.662780in}}%
\pgfpathlineto{\pgfqpoint{1.224808in}{1.660920in}}%
\pgfpathlineto{\pgfqpoint{1.222136in}{1.659018in}}%
\pgfpathclose%
\pgfusepath{fill}%
\end{pgfscope}%
\begin{pgfscope}%
\pgfpathrectangle{\pgfqpoint{0.329460in}{0.284240in}}{\pgfqpoint{1.989680in}{1.989680in}}%
\pgfusepath{clip}%
\pgfsetbuttcap%
\pgfsetroundjoin%
\definecolor{currentfill}{rgb}{0.271305,0.019942,0.347269}%
\pgfsetfillcolor{currentfill}%
\pgfsetlinewidth{0.000000pt}%
\definecolor{currentstroke}{rgb}{0.000000,0.000000,0.000000}%
\pgfsetstrokecolor{currentstroke}%
\pgfsetdash{}{0pt}%
\pgfpathmoveto{\pgfqpoint{1.776153in}{0.911797in}}%
\pgfpathlineto{\pgfqpoint{1.779113in}{0.908620in}}%
\pgfpathlineto{\pgfqpoint{1.782079in}{0.905624in}}%
\pgfpathlineto{\pgfqpoint{1.785049in}{0.902815in}}%
\pgfpathlineto{\pgfqpoint{1.788023in}{0.900195in}}%
\pgfpathlineto{\pgfqpoint{1.779890in}{0.892912in}}%
\pgfpathlineto{\pgfqpoint{1.771302in}{0.885761in}}%
\pgfpathlineto{\pgfqpoint{1.762266in}{0.878752in}}%
\pgfpathlineto{\pgfqpoint{1.752791in}{0.871892in}}%
\pgfpathlineto{\pgfqpoint{1.750045in}{0.874716in}}%
\pgfpathlineto{\pgfqpoint{1.747303in}{0.877731in}}%
\pgfpathlineto{\pgfqpoint{1.744566in}{0.880932in}}%
\pgfpathlineto{\pgfqpoint{1.741833in}{0.884314in}}%
\pgfpathlineto{\pgfqpoint{1.751060in}{0.890975in}}%
\pgfpathlineto{\pgfqpoint{1.759862in}{0.897781in}}%
\pgfpathlineto{\pgfqpoint{1.768228in}{0.904724in}}%
\pgfpathlineto{\pgfqpoint{1.776153in}{0.911797in}}%
\pgfpathclose%
\pgfusepath{fill}%
\end{pgfscope}%
\begin{pgfscope}%
\pgfpathrectangle{\pgfqpoint{0.329460in}{0.284240in}}{\pgfqpoint{1.989680in}{1.989680in}}%
\pgfusepath{clip}%
\pgfsetbuttcap%
\pgfsetroundjoin%
\definecolor{currentfill}{rgb}{0.282327,0.094955,0.417331}%
\pgfsetfillcolor{currentfill}%
\pgfsetlinewidth{0.000000pt}%
\definecolor{currentstroke}{rgb}{0.000000,0.000000,0.000000}%
\pgfsetstrokecolor{currentstroke}%
\pgfsetdash{}{0pt}%
\pgfpathmoveto{\pgfqpoint{1.740905in}{0.962682in}}%
\pgfpathlineto{\pgfqpoint{1.743826in}{0.957624in}}%
\pgfpathlineto{\pgfqpoint{1.746750in}{0.952703in}}%
\pgfpathlineto{\pgfqpoint{1.749676in}{0.947922in}}%
\pgfpathlineto{\pgfqpoint{1.752604in}{0.943284in}}%
\pgfpathlineto{\pgfqpoint{1.745096in}{0.936636in}}%
\pgfpathlineto{\pgfqpoint{1.737171in}{0.930110in}}%
\pgfpathlineto{\pgfqpoint{1.728837in}{0.923714in}}%
\pgfpathlineto{\pgfqpoint{1.720101in}{0.917455in}}%
\pgfpathlineto{\pgfqpoint{1.717399in}{0.922299in}}%
\pgfpathlineto{\pgfqpoint{1.714700in}{0.927288in}}%
\pgfpathlineto{\pgfqpoint{1.712003in}{0.932417in}}%
\pgfpathlineto{\pgfqpoint{1.709308in}{0.937681in}}%
\pgfpathlineto{\pgfqpoint{1.717798in}{0.943739in}}%
\pgfpathlineto{\pgfqpoint{1.725900in}{0.949930in}}%
\pgfpathlineto{\pgfqpoint{1.733604in}{0.956246in}}%
\pgfpathlineto{\pgfqpoint{1.740905in}{0.962682in}}%
\pgfpathclose%
\pgfusepath{fill}%
\end{pgfscope}%
\begin{pgfscope}%
\pgfpathrectangle{\pgfqpoint{0.329460in}{0.284240in}}{\pgfqpoint{1.989680in}{1.989680in}}%
\pgfusepath{clip}%
\pgfsetbuttcap%
\pgfsetroundjoin%
\definecolor{currentfill}{rgb}{0.344074,0.780029,0.397381}%
\pgfsetfillcolor{currentfill}%
\pgfsetlinewidth{0.000000pt}%
\definecolor{currentstroke}{rgb}{0.000000,0.000000,0.000000}%
\pgfsetstrokecolor{currentstroke}%
\pgfsetdash{}{0pt}%
\pgfpathmoveto{\pgfqpoint{1.160246in}{1.573933in}}%
\pgfpathlineto{\pgfqpoint{1.156905in}{1.568472in}}%
\pgfpathlineto{\pgfqpoint{1.153568in}{1.562945in}}%
\pgfpathlineto{\pgfqpoint{1.150234in}{1.557353in}}%
\pgfpathlineto{\pgfqpoint{1.146902in}{1.551699in}}%
\pgfpathlineto{\pgfqpoint{1.148305in}{1.554786in}}%
\pgfpathlineto{\pgfqpoint{1.149907in}{1.557847in}}%
\pgfpathlineto{\pgfqpoint{1.151708in}{1.560882in}}%
\pgfpathlineto{\pgfqpoint{1.153704in}{1.563886in}}%
\pgfpathlineto{\pgfqpoint{1.156918in}{1.569331in}}%
\pgfpathlineto{\pgfqpoint{1.160135in}{1.574713in}}%
\pgfpathlineto{\pgfqpoint{1.163355in}{1.580031in}}%
\pgfpathlineto{\pgfqpoint{1.166579in}{1.585283in}}%
\pgfpathlineto{\pgfqpoint{1.164718in}{1.582484in}}%
\pgfpathlineto{\pgfqpoint{1.163042in}{1.579658in}}%
\pgfpathlineto{\pgfqpoint{1.161550in}{1.576807in}}%
\pgfpathlineto{\pgfqpoint{1.160246in}{1.573933in}}%
\pgfpathclose%
\pgfusepath{fill}%
\end{pgfscope}%
\begin{pgfscope}%
\pgfpathrectangle{\pgfqpoint{0.329460in}{0.284240in}}{\pgfqpoint{1.989680in}{1.989680in}}%
\pgfusepath{clip}%
\pgfsetbuttcap%
\pgfsetroundjoin%
\definecolor{currentfill}{rgb}{0.179019,0.433756,0.557430}%
\pgfsetfillcolor{currentfill}%
\pgfsetlinewidth{0.000000pt}%
\definecolor{currentstroke}{rgb}{0.000000,0.000000,0.000000}%
\pgfsetstrokecolor{currentstroke}%
\pgfsetdash{}{0pt}%
\pgfpathmoveto{\pgfqpoint{1.051044in}{1.221763in}}%
\pgfpathlineto{\pgfqpoint{1.047992in}{1.214156in}}%
\pgfpathlineto{\pgfqpoint{1.044942in}{1.206573in}}%
\pgfpathlineto{\pgfqpoint{1.041893in}{1.199015in}}%
\pgfpathlineto{\pgfqpoint{1.038846in}{1.191486in}}%
\pgfpathlineto{\pgfqpoint{1.034514in}{1.196541in}}%
\pgfpathlineto{\pgfqpoint{1.030508in}{1.201659in}}%
\pgfpathlineto{\pgfqpoint{1.026831in}{1.206834in}}%
\pgfpathlineto{\pgfqpoint{1.023487in}{1.212062in}}%
\pgfpathlineto{\pgfqpoint{1.026674in}{1.219366in}}%
\pgfpathlineto{\pgfqpoint{1.029863in}{1.226700in}}%
\pgfpathlineto{\pgfqpoint{1.033053in}{1.234060in}}%
\pgfpathlineto{\pgfqpoint{1.036245in}{1.241443in}}%
\pgfpathlineto{\pgfqpoint{1.039469in}{1.236443in}}%
\pgfpathlineto{\pgfqpoint{1.043012in}{1.231493in}}%
\pgfpathlineto{\pgfqpoint{1.046872in}{1.226598in}}%
\pgfpathlineto{\pgfqpoint{1.051044in}{1.221763in}}%
\pgfpathclose%
\pgfusepath{fill}%
\end{pgfscope}%
\begin{pgfscope}%
\pgfpathrectangle{\pgfqpoint{0.329460in}{0.284240in}}{\pgfqpoint{1.989680in}{1.989680in}}%
\pgfusepath{clip}%
\pgfsetbuttcap%
\pgfsetroundjoin%
\definecolor{currentfill}{rgb}{0.120081,0.622161,0.534946}%
\pgfsetfillcolor{currentfill}%
\pgfsetlinewidth{0.000000pt}%
\definecolor{currentstroke}{rgb}{0.000000,0.000000,0.000000}%
\pgfsetstrokecolor{currentstroke}%
\pgfsetdash{}{0pt}%
\pgfpathmoveto{\pgfqpoint{1.092590in}{1.405867in}}%
\pgfpathlineto{\pgfqpoint{1.089268in}{1.398780in}}%
\pgfpathlineto{\pgfqpoint{1.085948in}{1.391667in}}%
\pgfpathlineto{\pgfqpoint{1.082631in}{1.384530in}}%
\pgfpathlineto{\pgfqpoint{1.079316in}{1.377370in}}%
\pgfpathlineto{\pgfqpoint{1.077940in}{1.381623in}}%
\pgfpathlineto{\pgfqpoint{1.076840in}{1.385891in}}%
\pgfpathlineto{\pgfqpoint{1.076015in}{1.390172in}}%
\pgfpathlineto{\pgfqpoint{1.075467in}{1.394460in}}%
\pgfpathlineto{\pgfqpoint{1.078819in}{1.401392in}}%
\pgfpathlineto{\pgfqpoint{1.082175in}{1.408302in}}%
\pgfpathlineto{\pgfqpoint{1.085533in}{1.415189in}}%
\pgfpathlineto{\pgfqpoint{1.088893in}{1.422050in}}%
\pgfpathlineto{\pgfqpoint{1.089424in}{1.417989in}}%
\pgfpathlineto{\pgfqpoint{1.090218in}{1.413935in}}%
\pgfpathlineto{\pgfqpoint{1.091273in}{1.409893in}}%
\pgfpathlineto{\pgfqpoint{1.092590in}{1.405867in}}%
\pgfpathclose%
\pgfusepath{fill}%
\end{pgfscope}%
\begin{pgfscope}%
\pgfpathrectangle{\pgfqpoint{0.329460in}{0.284240in}}{\pgfqpoint{1.989680in}{1.989680in}}%
\pgfusepath{clip}%
\pgfsetbuttcap%
\pgfsetroundjoin%
\definecolor{currentfill}{rgb}{0.166383,0.690856,0.496502}%
\pgfsetfillcolor{currentfill}%
\pgfsetlinewidth{0.000000pt}%
\definecolor{currentstroke}{rgb}{0.000000,0.000000,0.000000}%
\pgfsetstrokecolor{currentstroke}%
\pgfsetdash{}{0pt}%
\pgfpathmoveto{\pgfqpoint{1.115879in}{1.475730in}}%
\pgfpathlineto{\pgfqpoint{1.112496in}{1.469158in}}%
\pgfpathlineto{\pgfqpoint{1.109116in}{1.462541in}}%
\pgfpathlineto{\pgfqpoint{1.105739in}{1.455884in}}%
\pgfpathlineto{\pgfqpoint{1.102364in}{1.449187in}}%
\pgfpathlineto{\pgfqpoint{1.102101in}{1.453025in}}%
\pgfpathlineto{\pgfqpoint{1.102086in}{1.456864in}}%
\pgfpathlineto{\pgfqpoint{1.102319in}{1.460698in}}%
\pgfpathlineto{\pgfqpoint{1.102800in}{1.464525in}}%
\pgfpathlineto{\pgfqpoint{1.106161in}{1.470997in}}%
\pgfpathlineto{\pgfqpoint{1.109524in}{1.477431in}}%
\pgfpathlineto{\pgfqpoint{1.112890in}{1.483824in}}%
\pgfpathlineto{\pgfqpoint{1.116260in}{1.490173in}}%
\pgfpathlineto{\pgfqpoint{1.115813in}{1.486570in}}%
\pgfpathlineto{\pgfqpoint{1.115600in}{1.482959in}}%
\pgfpathlineto{\pgfqpoint{1.115622in}{1.479345in}}%
\pgfpathlineto{\pgfqpoint{1.115879in}{1.475730in}}%
\pgfpathclose%
\pgfusepath{fill}%
\end{pgfscope}%
\begin{pgfscope}%
\pgfpathrectangle{\pgfqpoint{0.329460in}{0.284240in}}{\pgfqpoint{1.989680in}{1.989680in}}%
\pgfusepath{clip}%
\pgfsetbuttcap%
\pgfsetroundjoin%
\definecolor{currentfill}{rgb}{0.281477,0.755203,0.432552}%
\pgfsetfillcolor{currentfill}%
\pgfsetlinewidth{0.000000pt}%
\definecolor{currentstroke}{rgb}{0.000000,0.000000,0.000000}%
\pgfsetstrokecolor{currentstroke}%
\pgfsetdash{}{0pt}%
\pgfpathmoveto{\pgfqpoint{1.554236in}{1.554444in}}%
\pgfpathlineto{\pgfqpoint{1.557542in}{1.548776in}}%
\pgfpathlineto{\pgfqpoint{1.560845in}{1.543050in}}%
\pgfpathlineto{\pgfqpoint{1.564145in}{1.537268in}}%
\pgfpathlineto{\pgfqpoint{1.567442in}{1.531430in}}%
\pgfpathlineto{\pgfqpoint{1.568919in}{1.528127in}}%
\pgfpathlineto{\pgfqpoint{1.570182in}{1.524802in}}%
\pgfpathlineto{\pgfqpoint{1.571227in}{1.521458in}}%
\pgfpathlineto{\pgfqpoint{1.572056in}{1.518097in}}%
\pgfpathlineto{\pgfqpoint{1.568681in}{1.524151in}}%
\pgfpathlineto{\pgfqpoint{1.565304in}{1.530150in}}%
\pgfpathlineto{\pgfqpoint{1.561923in}{1.536092in}}%
\pgfpathlineto{\pgfqpoint{1.558540in}{1.541975in}}%
\pgfpathlineto{\pgfqpoint{1.557769in}{1.545118in}}%
\pgfpathlineto{\pgfqpoint{1.556794in}{1.548245in}}%
\pgfpathlineto{\pgfqpoint{1.555616in}{1.551355in}}%
\pgfpathlineto{\pgfqpoint{1.554236in}{1.554444in}}%
\pgfpathclose%
\pgfusepath{fill}%
\end{pgfscope}%
\begin{pgfscope}%
\pgfpathrectangle{\pgfqpoint{0.329460in}{0.284240in}}{\pgfqpoint{1.989680in}{1.989680in}}%
\pgfusepath{clip}%
\pgfsetbuttcap%
\pgfsetroundjoin%
\definecolor{currentfill}{rgb}{0.636902,0.856542,0.216620}%
\pgfsetfillcolor{currentfill}%
\pgfsetlinewidth{0.000000pt}%
\definecolor{currentstroke}{rgb}{0.000000,0.000000,0.000000}%
\pgfsetstrokecolor{currentstroke}%
\pgfsetdash{}{0pt}%
\pgfpathmoveto{\pgfqpoint{1.443369in}{1.687852in}}%
\pgfpathlineto{\pgfqpoint{1.445632in}{1.684607in}}%
\pgfpathlineto{\pgfqpoint{1.447892in}{1.681272in}}%
\pgfpathlineto{\pgfqpoint{1.450150in}{1.677847in}}%
\pgfpathlineto{\pgfqpoint{1.452405in}{1.674335in}}%
\pgfpathlineto{\pgfqpoint{1.455967in}{1.672805in}}%
\pgfpathlineto{\pgfqpoint{1.459428in}{1.671222in}}%
\pgfpathlineto{\pgfqpoint{1.462783in}{1.669589in}}%
\pgfpathlineto{\pgfqpoint{1.466030in}{1.667906in}}%
\pgfpathlineto{\pgfqpoint{1.463469in}{1.671565in}}%
\pgfpathlineto{\pgfqpoint{1.460905in}{1.675137in}}%
\pgfpathlineto{\pgfqpoint{1.458337in}{1.678620in}}%
\pgfpathlineto{\pgfqpoint{1.455768in}{1.682011in}}%
\pgfpathlineto{\pgfqpoint{1.452814in}{1.683540in}}%
\pgfpathlineto{\pgfqpoint{1.449760in}{1.685025in}}%
\pgfpathlineto{\pgfqpoint{1.446611in}{1.686462in}}%
\pgfpathlineto{\pgfqpoint{1.443369in}{1.687852in}}%
\pgfpathclose%
\pgfusepath{fill}%
\end{pgfscope}%
\begin{pgfscope}%
\pgfpathrectangle{\pgfqpoint{0.329460in}{0.284240in}}{\pgfqpoint{1.989680in}{1.989680in}}%
\pgfusepath{clip}%
\pgfsetbuttcap%
\pgfsetroundjoin%
\definecolor{currentfill}{rgb}{0.201239,0.383670,0.554294}%
\pgfsetfillcolor{currentfill}%
\pgfsetlinewidth{0.000000pt}%
\definecolor{currentstroke}{rgb}{0.000000,0.000000,0.000000}%
\pgfsetstrokecolor{currentstroke}%
\pgfsetdash{}{0pt}%
\pgfpathmoveto{\pgfqpoint{2.003560in}{1.164371in}}%
\pgfpathlineto{\pgfqpoint{2.007405in}{1.175815in}}%
\pgfpathlineto{\pgfqpoint{2.011269in}{1.187690in}}%
\pgfpathlineto{\pgfqpoint{2.015154in}{1.200003in}}%
\pgfpathlineto{\pgfqpoint{2.019060in}{1.212761in}}%
\pgfpathlineto{\pgfqpoint{2.016288in}{1.202072in}}%
\pgfpathlineto{\pgfqpoint{2.012836in}{1.191414in}}%
\pgfpathlineto{\pgfqpoint{2.008702in}{1.180799in}}%
\pgfpathlineto{\pgfqpoint{2.003889in}{1.170239in}}%
\pgfpathlineto{\pgfqpoint{2.000046in}{1.157645in}}%
\pgfpathlineto{\pgfqpoint{1.996224in}{1.145500in}}%
\pgfpathlineto{\pgfqpoint{1.992423in}{1.133795in}}%
\pgfpathlineto{\pgfqpoint{1.988641in}{1.122523in}}%
\pgfpathlineto{\pgfqpoint{1.993367in}{1.132915in}}%
\pgfpathlineto{\pgfqpoint{1.997429in}{1.143360in}}%
\pgfpathlineto{\pgfqpoint{2.000827in}{1.153849in}}%
\pgfpathlineto{\pgfqpoint{2.003560in}{1.164371in}}%
\pgfpathclose%
\pgfusepath{fill}%
\end{pgfscope}%
\begin{pgfscope}%
\pgfpathrectangle{\pgfqpoint{0.329460in}{0.284240in}}{\pgfqpoint{1.989680in}{1.989680in}}%
\pgfusepath{clip}%
\pgfsetbuttcap%
\pgfsetroundjoin%
\definecolor{currentfill}{rgb}{0.699415,0.867117,0.175971}%
\pgfsetfillcolor{currentfill}%
\pgfsetlinewidth{0.000000pt}%
\definecolor{currentstroke}{rgb}{0.000000,0.000000,0.000000}%
\pgfsetstrokecolor{currentstroke}%
\pgfsetdash{}{0pt}%
\pgfpathmoveto{\pgfqpoint{1.408271in}{1.708229in}}%
\pgfpathlineto{\pgfqpoint{1.409829in}{1.705591in}}%
\pgfpathlineto{\pgfqpoint{1.411386in}{1.702857in}}%
\pgfpathlineto{\pgfqpoint{1.412941in}{1.700029in}}%
\pgfpathlineto{\pgfqpoint{1.414495in}{1.697108in}}%
\pgfpathlineto{\pgfqpoint{1.418357in}{1.696143in}}%
\pgfpathlineto{\pgfqpoint{1.422154in}{1.695121in}}%
\pgfpathlineto{\pgfqpoint{1.425883in}{1.694043in}}%
\pgfpathlineto{\pgfqpoint{1.429539in}{1.692910in}}%
\pgfpathlineto{\pgfqpoint{1.427615in}{1.695936in}}%
\pgfpathlineto{\pgfqpoint{1.425689in}{1.698870in}}%
\pgfpathlineto{\pgfqpoint{1.423761in}{1.701708in}}%
\pgfpathlineto{\pgfqpoint{1.421831in}{1.704451in}}%
\pgfpathlineto{\pgfqpoint{1.418536in}{1.705471in}}%
\pgfpathlineto{\pgfqpoint{1.415175in}{1.706441in}}%
\pgfpathlineto{\pgfqpoint{1.411753in}{1.707361in}}%
\pgfpathlineto{\pgfqpoint{1.408271in}{1.708229in}}%
\pgfpathclose%
\pgfusepath{fill}%
\end{pgfscope}%
\begin{pgfscope}%
\pgfpathrectangle{\pgfqpoint{0.329460in}{0.284240in}}{\pgfqpoint{1.989680in}{1.989680in}}%
\pgfusepath{clip}%
\pgfsetbuttcap%
\pgfsetroundjoin%
\definecolor{currentfill}{rgb}{0.268510,0.009605,0.335427}%
\pgfsetfillcolor{currentfill}%
\pgfsetlinewidth{0.000000pt}%
\definecolor{currentstroke}{rgb}{0.000000,0.000000,0.000000}%
\pgfsetstrokecolor{currentstroke}%
\pgfsetdash{}{0pt}%
\pgfpathmoveto{\pgfqpoint{1.788023in}{0.900195in}}%
\pgfpathlineto{\pgfqpoint{1.791003in}{0.897769in}}%
\pgfpathlineto{\pgfqpoint{1.793988in}{0.895541in}}%
\pgfpathlineto{\pgfqpoint{1.796978in}{0.893515in}}%
\pgfpathlineto{\pgfqpoint{1.799974in}{0.891695in}}%
\pgfpathlineto{\pgfqpoint{1.791632in}{0.884203in}}%
\pgfpathlineto{\pgfqpoint{1.782821in}{0.876847in}}%
\pgfpathlineto{\pgfqpoint{1.773550in}{0.869635in}}%
\pgfpathlineto{\pgfqpoint{1.763827in}{0.862577in}}%
\pgfpathlineto{\pgfqpoint{1.761060in}{0.864600in}}%
\pgfpathlineto{\pgfqpoint{1.758299in}{0.866829in}}%
\pgfpathlineto{\pgfqpoint{1.755543in}{0.869261in}}%
\pgfpathlineto{\pgfqpoint{1.752791in}{0.871892in}}%
\pgfpathlineto{\pgfqpoint{1.762266in}{0.878752in}}%
\pgfpathlineto{\pgfqpoint{1.771302in}{0.885761in}}%
\pgfpathlineto{\pgfqpoint{1.779890in}{0.892912in}}%
\pgfpathlineto{\pgfqpoint{1.788023in}{0.900195in}}%
\pgfpathclose%
\pgfusepath{fill}%
\end{pgfscope}%
\begin{pgfscope}%
\pgfpathrectangle{\pgfqpoint{0.329460in}{0.284240in}}{\pgfqpoint{1.989680in}{1.989680in}}%
\pgfusepath{clip}%
\pgfsetbuttcap%
\pgfsetroundjoin%
\definecolor{currentfill}{rgb}{0.283072,0.130895,0.449241}%
\pgfsetfillcolor{currentfill}%
\pgfsetlinewidth{0.000000pt}%
\definecolor{currentstroke}{rgb}{0.000000,0.000000,0.000000}%
\pgfsetstrokecolor{currentstroke}%
\pgfsetdash{}{0pt}%
\pgfpathmoveto{\pgfqpoint{1.729244in}{0.984202in}}%
\pgfpathlineto{\pgfqpoint{1.732156in}{0.978635in}}%
\pgfpathlineto{\pgfqpoint{1.735070in}{0.973191in}}%
\pgfpathlineto{\pgfqpoint{1.737987in}{0.967872in}}%
\pgfpathlineto{\pgfqpoint{1.740905in}{0.962682in}}%
\pgfpathlineto{\pgfqpoint{1.733604in}{0.956246in}}%
\pgfpathlineto{\pgfqpoint{1.725900in}{0.949930in}}%
\pgfpathlineto{\pgfqpoint{1.717798in}{0.943739in}}%
\pgfpathlineto{\pgfqpoint{1.709308in}{0.937681in}}%
\pgfpathlineto{\pgfqpoint{1.706616in}{0.943079in}}%
\pgfpathlineto{\pgfqpoint{1.703926in}{0.948606in}}%
\pgfpathlineto{\pgfqpoint{1.701238in}{0.954258in}}%
\pgfpathlineto{\pgfqpoint{1.698552in}{0.960032in}}%
\pgfpathlineto{\pgfqpoint{1.706797in}{0.965888in}}%
\pgfpathlineto{\pgfqpoint{1.714666in}{0.971873in}}%
\pgfpathlineto{\pgfqpoint{1.722150in}{0.977980in}}%
\pgfpathlineto{\pgfqpoint{1.729244in}{0.984202in}}%
\pgfpathclose%
\pgfusepath{fill}%
\end{pgfscope}%
\begin{pgfscope}%
\pgfpathrectangle{\pgfqpoint{0.329460in}{0.284240in}}{\pgfqpoint{1.989680in}{1.989680in}}%
\pgfusepath{clip}%
\pgfsetbuttcap%
\pgfsetroundjoin%
\definecolor{currentfill}{rgb}{0.133743,0.548535,0.553541}%
\pgfsetfillcolor{currentfill}%
\pgfsetlinewidth{0.000000pt}%
\definecolor{currentstroke}{rgb}{0.000000,0.000000,0.000000}%
\pgfsetstrokecolor{currentstroke}%
\pgfsetdash{}{0pt}%
\pgfpathmoveto{\pgfqpoint{1.074676in}{1.330903in}}%
\pgfpathlineto{\pgfqpoint{1.071463in}{1.323444in}}%
\pgfpathlineto{\pgfqpoint{1.068252in}{1.315978in}}%
\pgfpathlineto{\pgfqpoint{1.065043in}{1.308506in}}%
\pgfpathlineto{\pgfqpoint{1.061836in}{1.301032in}}%
\pgfpathlineto{\pgfqpoint{1.059150in}{1.305623in}}%
\pgfpathlineto{\pgfqpoint{1.056761in}{1.310251in}}%
\pgfpathlineto{\pgfqpoint{1.054671in}{1.314911in}}%
\pgfpathlineto{\pgfqpoint{1.052881in}{1.319599in}}%
\pgfpathlineto{\pgfqpoint{1.056177in}{1.326846in}}%
\pgfpathlineto{\pgfqpoint{1.059476in}{1.334091in}}%
\pgfpathlineto{\pgfqpoint{1.062777in}{1.341330in}}%
\pgfpathlineto{\pgfqpoint{1.066080in}{1.348563in}}%
\pgfpathlineto{\pgfqpoint{1.067800in}{1.344104in}}%
\pgfpathlineto{\pgfqpoint{1.069807in}{1.339672in}}%
\pgfpathlineto{\pgfqpoint{1.072100in}{1.335270in}}%
\pgfpathlineto{\pgfqpoint{1.074676in}{1.330903in}}%
\pgfpathclose%
\pgfusepath{fill}%
\end{pgfscope}%
\begin{pgfscope}%
\pgfpathrectangle{\pgfqpoint{0.329460in}{0.284240in}}{\pgfqpoint{1.989680in}{1.989680in}}%
\pgfusepath{clip}%
\pgfsetbuttcap%
\pgfsetroundjoin%
\definecolor{currentfill}{rgb}{0.276194,0.190074,0.493001}%
\pgfsetfillcolor{currentfill}%
\pgfsetlinewidth{0.000000pt}%
\definecolor{currentstroke}{rgb}{0.000000,0.000000,0.000000}%
\pgfsetstrokecolor{currentstroke}%
\pgfsetdash{}{0pt}%
\pgfpathmoveto{\pgfqpoint{0.788344in}{0.971605in}}%
\pgfpathlineto{\pgfqpoint{0.784928in}{0.977968in}}%
\pgfpathlineto{\pgfqpoint{0.781496in}{0.984688in}}%
\pgfpathlineto{\pgfqpoint{0.778050in}{0.991770in}}%
\pgfpathlineto{\pgfqpoint{0.774589in}{0.999220in}}%
\pgfpathlineto{\pgfqpoint{0.766988in}{1.008829in}}%
\pgfpathlineto{\pgfqpoint{0.760002in}{1.018545in}}%
\pgfpathlineto{\pgfqpoint{0.753637in}{1.028356in}}%
\pgfpathlineto{\pgfqpoint{0.747897in}{1.038252in}}%
\pgfpathlineto{\pgfqpoint{0.751497in}{1.030613in}}%
\pgfpathlineto{\pgfqpoint{0.755081in}{1.023341in}}%
\pgfpathlineto{\pgfqpoint{0.758650in}{1.016429in}}%
\pgfpathlineto{\pgfqpoint{0.762205in}{1.009872in}}%
\pgfpathlineto{\pgfqpoint{0.767830in}{1.000169in}}%
\pgfpathlineto{\pgfqpoint{0.774064in}{0.990550in}}%
\pgfpathlineto{\pgfqpoint{0.780904in}{0.981025in}}%
\pgfpathlineto{\pgfqpoint{0.788344in}{0.971605in}}%
\pgfpathclose%
\pgfusepath{fill}%
\end{pgfscope}%
\begin{pgfscope}%
\pgfpathrectangle{\pgfqpoint{0.329460in}{0.284240in}}{\pgfqpoint{1.989680in}{1.989680in}}%
\pgfusepath{clip}%
\pgfsetbuttcap%
\pgfsetroundjoin%
\definecolor{currentfill}{rgb}{0.699415,0.867117,0.175971}%
\pgfsetfillcolor{currentfill}%
\pgfsetlinewidth{0.000000pt}%
\definecolor{currentstroke}{rgb}{0.000000,0.000000,0.000000}%
\pgfsetstrokecolor{currentstroke}%
\pgfsetdash{}{0pt}%
\pgfpathmoveto{\pgfqpoint{1.277672in}{1.703505in}}%
\pgfpathlineto{\pgfqpoint{1.275664in}{1.700735in}}%
\pgfpathlineto{\pgfqpoint{1.273657in}{1.697870in}}%
\pgfpathlineto{\pgfqpoint{1.271652in}{1.694911in}}%
\pgfpathlineto{\pgfqpoint{1.269650in}{1.691858in}}%
\pgfpathlineto{\pgfqpoint{1.273239in}{1.693038in}}%
\pgfpathlineto{\pgfqpoint{1.276903in}{1.694165in}}%
\pgfpathlineto{\pgfqpoint{1.280640in}{1.695237in}}%
\pgfpathlineto{\pgfqpoint{1.284444in}{1.696253in}}%
\pgfpathlineto{\pgfqpoint{1.286082in}{1.699195in}}%
\pgfpathlineto{\pgfqpoint{1.287722in}{1.702045in}}%
\pgfpathlineto{\pgfqpoint{1.289363in}{1.704800in}}%
\pgfpathlineto{\pgfqpoint{1.291007in}{1.707460in}}%
\pgfpathlineto{\pgfqpoint{1.287577in}{1.706546in}}%
\pgfpathlineto{\pgfqpoint{1.284209in}{1.705581in}}%
\pgfpathlineto{\pgfqpoint{1.280907in}{1.704567in}}%
\pgfpathlineto{\pgfqpoint{1.277672in}{1.703505in}}%
\pgfpathclose%
\pgfusepath{fill}%
\end{pgfscope}%
\begin{pgfscope}%
\pgfpathrectangle{\pgfqpoint{0.329460in}{0.284240in}}{\pgfqpoint{1.989680in}{1.989680in}}%
\pgfusepath{clip}%
\pgfsetbuttcap%
\pgfsetroundjoin%
\definecolor{currentfill}{rgb}{0.163625,0.471133,0.558148}%
\pgfsetfillcolor{currentfill}%
\pgfsetlinewidth{0.000000pt}%
\definecolor{currentstroke}{rgb}{0.000000,0.000000,0.000000}%
\pgfsetstrokecolor{currentstroke}%
\pgfsetdash{}{0pt}%
\pgfpathmoveto{\pgfqpoint{1.655848in}{1.275437in}}%
\pgfpathlineto{\pgfqpoint{1.659070in}{1.268038in}}%
\pgfpathlineto{\pgfqpoint{1.662291in}{1.260651in}}%
\pgfpathlineto{\pgfqpoint{1.665509in}{1.253279in}}%
\pgfpathlineto{\pgfqpoint{1.668726in}{1.245926in}}%
\pgfpathlineto{\pgfqpoint{1.665788in}{1.240885in}}%
\pgfpathlineto{\pgfqpoint{1.662528in}{1.235890in}}%
\pgfpathlineto{\pgfqpoint{1.658950in}{1.230946in}}%
\pgfpathlineto{\pgfqpoint{1.655055in}{1.226057in}}%
\pgfpathlineto{\pgfqpoint{1.651967in}{1.233636in}}%
\pgfpathlineto{\pgfqpoint{1.648877in}{1.241232in}}%
\pgfpathlineto{\pgfqpoint{1.645786in}{1.248843in}}%
\pgfpathlineto{\pgfqpoint{1.642694in}{1.256467in}}%
\pgfpathlineto{\pgfqpoint{1.646439in}{1.261134in}}%
\pgfpathlineto{\pgfqpoint{1.649882in}{1.265855in}}%
\pgfpathlineto{\pgfqpoint{1.653019in}{1.270624in}}%
\pgfpathlineto{\pgfqpoint{1.655848in}{1.275437in}}%
\pgfpathclose%
\pgfusepath{fill}%
\end{pgfscope}%
\begin{pgfscope}%
\pgfpathrectangle{\pgfqpoint{0.329460in}{0.284240in}}{\pgfqpoint{1.989680in}{1.989680in}}%
\pgfusepath{clip}%
\pgfsetbuttcap%
\pgfsetroundjoin%
\definecolor{currentfill}{rgb}{0.636902,0.856542,0.216620}%
\pgfsetfillcolor{currentfill}%
\pgfsetlinewidth{0.000000pt}%
\definecolor{currentstroke}{rgb}{0.000000,0.000000,0.000000}%
\pgfsetstrokecolor{currentstroke}%
\pgfsetdash{}{0pt}%
\pgfpathmoveto{\pgfqpoint{1.244067in}{1.680616in}}%
\pgfpathlineto{\pgfqpoint{1.241434in}{1.677189in}}%
\pgfpathlineto{\pgfqpoint{1.238804in}{1.673671in}}%
\pgfpathlineto{\pgfqpoint{1.236177in}{1.670064in}}%
\pgfpathlineto{\pgfqpoint{1.233552in}{1.666369in}}%
\pgfpathlineto{\pgfqpoint{1.236700in}{1.668095in}}%
\pgfpathlineto{\pgfqpoint{1.239959in}{1.669773in}}%
\pgfpathlineto{\pgfqpoint{1.243327in}{1.671401in}}%
\pgfpathlineto{\pgfqpoint{1.246799in}{1.672977in}}%
\pgfpathlineto{\pgfqpoint{1.249125in}{1.676521in}}%
\pgfpathlineto{\pgfqpoint{1.251454in}{1.679977in}}%
\pgfpathlineto{\pgfqpoint{1.253786in}{1.683343in}}%
\pgfpathlineto{\pgfqpoint{1.256120in}{1.686619in}}%
\pgfpathlineto{\pgfqpoint{1.252960in}{1.685187in}}%
\pgfpathlineto{\pgfqpoint{1.249896in}{1.683708in}}%
\pgfpathlineto{\pgfqpoint{1.246931in}{1.682184in}}%
\pgfpathlineto{\pgfqpoint{1.244067in}{1.680616in}}%
\pgfpathclose%
\pgfusepath{fill}%
\end{pgfscope}%
\begin{pgfscope}%
\pgfpathrectangle{\pgfqpoint{0.329460in}{0.284240in}}{\pgfqpoint{1.989680in}{1.989680in}}%
\pgfusepath{clip}%
\pgfsetbuttcap%
\pgfsetroundjoin%
\definecolor{currentfill}{rgb}{0.231674,0.318106,0.544834}%
\pgfsetfillcolor{currentfill}%
\pgfsetlinewidth{0.000000pt}%
\definecolor{currentstroke}{rgb}{0.000000,0.000000,0.000000}%
\pgfsetstrokecolor{currentstroke}%
\pgfsetdash{}{0pt}%
\pgfpathmoveto{\pgfqpoint{1.036524in}{1.111419in}}%
\pgfpathlineto{\pgfqpoint{1.033673in}{1.104059in}}%
\pgfpathlineto{\pgfqpoint{1.030821in}{1.096757in}}%
\pgfpathlineto{\pgfqpoint{1.027970in}{1.089515in}}%
\pgfpathlineto{\pgfqpoint{1.025118in}{1.082337in}}%
\pgfpathlineto{\pgfqpoint{1.018896in}{1.087719in}}%
\pgfpathlineto{\pgfqpoint{1.013021in}{1.093195in}}%
\pgfpathlineto{\pgfqpoint{1.007497in}{1.098760in}}%
\pgfpathlineto{\pgfqpoint{1.002331in}{1.104407in}}%
\pgfpathlineto{\pgfqpoint{1.005371in}{1.111369in}}%
\pgfpathlineto{\pgfqpoint{1.008412in}{1.118394in}}%
\pgfpathlineto{\pgfqpoint{1.011453in}{1.125480in}}%
\pgfpathlineto{\pgfqpoint{1.014494in}{1.132623in}}%
\pgfpathlineto{\pgfqpoint{1.019490in}{1.127197in}}%
\pgfpathlineto{\pgfqpoint{1.024831in}{1.121851in}}%
\pgfpathlineto{\pgfqpoint{1.030510in}{1.116589in}}%
\pgfpathlineto{\pgfqpoint{1.036524in}{1.111419in}}%
\pgfpathclose%
\pgfusepath{fill}%
\end{pgfscope}%
\begin{pgfscope}%
\pgfpathrectangle{\pgfqpoint{0.329460in}{0.284240in}}{\pgfqpoint{1.989680in}{1.989680in}}%
\pgfusepath{clip}%
\pgfsetbuttcap%
\pgfsetroundjoin%
\definecolor{currentfill}{rgb}{0.212395,0.359683,0.551710}%
\pgfsetfillcolor{currentfill}%
\pgfsetlinewidth{0.000000pt}%
\definecolor{currentstroke}{rgb}{0.000000,0.000000,0.000000}%
\pgfsetstrokecolor{currentstroke}%
\pgfsetdash{}{0pt}%
\pgfpathmoveto{\pgfqpoint{1.679720in}{1.166399in}}%
\pgfpathlineto{\pgfqpoint{1.682798in}{1.159105in}}%
\pgfpathlineto{\pgfqpoint{1.685876in}{1.151856in}}%
\pgfpathlineto{\pgfqpoint{1.688954in}{1.144656in}}%
\pgfpathlineto{\pgfqpoint{1.692030in}{1.137508in}}%
\pgfpathlineto{\pgfqpoint{1.687343in}{1.132016in}}%
\pgfpathlineto{\pgfqpoint{1.682308in}{1.126599in}}%
\pgfpathlineto{\pgfqpoint{1.676930in}{1.121262in}}%
\pgfpathlineto{\pgfqpoint{1.671213in}{1.116010in}}%
\pgfpathlineto{\pgfqpoint{1.668314in}{1.123377in}}%
\pgfpathlineto{\pgfqpoint{1.665416in}{1.130795in}}%
\pgfpathlineto{\pgfqpoint{1.662516in}{1.138262in}}%
\pgfpathlineto{\pgfqpoint{1.659616in}{1.145774in}}%
\pgfpathlineto{\pgfqpoint{1.665136in}{1.150812in}}%
\pgfpathlineto{\pgfqpoint{1.670329in}{1.155932in}}%
\pgfpathlineto{\pgfqpoint{1.675192in}{1.161130in}}%
\pgfpathlineto{\pgfqpoint{1.679720in}{1.166399in}}%
\pgfpathclose%
\pgfusepath{fill}%
\end{pgfscope}%
\begin{pgfscope}%
\pgfpathrectangle{\pgfqpoint{0.329460in}{0.284240in}}{\pgfqpoint{1.989680in}{1.989680in}}%
\pgfusepath{clip}%
\pgfsetbuttcap%
\pgfsetroundjoin%
\definecolor{currentfill}{rgb}{0.281477,0.755203,0.432552}%
\pgfsetfillcolor{currentfill}%
\pgfsetlinewidth{0.000000pt}%
\definecolor{currentstroke}{rgb}{0.000000,0.000000,0.000000}%
\pgfsetstrokecolor{currentstroke}%
\pgfsetdash{}{0pt}%
\pgfpathmoveto{\pgfqpoint{1.143323in}{1.539172in}}%
\pgfpathlineto{\pgfqpoint{1.139930in}{1.533240in}}%
\pgfpathlineto{\pgfqpoint{1.136539in}{1.527250in}}%
\pgfpathlineto{\pgfqpoint{1.133152in}{1.521202in}}%
\pgfpathlineto{\pgfqpoint{1.129767in}{1.515099in}}%
\pgfpathlineto{\pgfqpoint{1.130401in}{1.518471in}}%
\pgfpathlineto{\pgfqpoint{1.131253in}{1.521830in}}%
\pgfpathlineto{\pgfqpoint{1.132323in}{1.525173in}}%
\pgfpathlineto{\pgfqpoint{1.133609in}{1.528496in}}%
\pgfpathlineto{\pgfqpoint{1.136928in}{1.534381in}}%
\pgfpathlineto{\pgfqpoint{1.140250in}{1.540211in}}%
\pgfpathlineto{\pgfqpoint{1.143574in}{1.545985in}}%
\pgfpathlineto{\pgfqpoint{1.146902in}{1.551699in}}%
\pgfpathlineto{\pgfqpoint{1.145702in}{1.548592in}}%
\pgfpathlineto{\pgfqpoint{1.144704in}{1.545466in}}%
\pgfpathlineto{\pgfqpoint{1.143911in}{1.542325in}}%
\pgfpathlineto{\pgfqpoint{1.143323in}{1.539172in}}%
\pgfpathclose%
\pgfusepath{fill}%
\end{pgfscope}%
\begin{pgfscope}%
\pgfpathrectangle{\pgfqpoint{0.329460in}{0.284240in}}{\pgfqpoint{1.989680in}{1.989680in}}%
\pgfusepath{clip}%
\pgfsetbuttcap%
\pgfsetroundjoin%
\definecolor{currentfill}{rgb}{0.280255,0.165693,0.476498}%
\pgfsetfillcolor{currentfill}%
\pgfsetlinewidth{0.000000pt}%
\definecolor{currentstroke}{rgb}{0.000000,0.000000,0.000000}%
\pgfsetstrokecolor{currentstroke}%
\pgfsetdash{}{0pt}%
\pgfpathmoveto{\pgfqpoint{1.717610in}{1.007620in}}%
\pgfpathlineto{\pgfqpoint{1.720516in}{1.001600in}}%
\pgfpathlineto{\pgfqpoint{1.723424in}{0.995688in}}%
\pgfpathlineto{\pgfqpoint{1.726333in}{0.989887in}}%
\pgfpathlineto{\pgfqpoint{1.729244in}{0.984202in}}%
\pgfpathlineto{\pgfqpoint{1.722150in}{0.977980in}}%
\pgfpathlineto{\pgfqpoint{1.714666in}{0.971873in}}%
\pgfpathlineto{\pgfqpoint{1.706797in}{0.965888in}}%
\pgfpathlineto{\pgfqpoint{1.698552in}{0.960032in}}%
\pgfpathlineto{\pgfqpoint{1.695868in}{0.965925in}}%
\pgfpathlineto{\pgfqpoint{1.693186in}{0.971933in}}%
\pgfpathlineto{\pgfqpoint{1.690505in}{0.978053in}}%
\pgfpathlineto{\pgfqpoint{1.687825in}{0.984281in}}%
\pgfpathlineto{\pgfqpoint{1.695825in}{0.989935in}}%
\pgfpathlineto{\pgfqpoint{1.703460in}{0.995714in}}%
\pgfpathlineto{\pgfqpoint{1.710724in}{1.001611in}}%
\pgfpathlineto{\pgfqpoint{1.717610in}{1.007620in}}%
\pgfpathclose%
\pgfusepath{fill}%
\end{pgfscope}%
\begin{pgfscope}%
\pgfpathrectangle{\pgfqpoint{0.329460in}{0.284240in}}{\pgfqpoint{1.989680in}{1.989680in}}%
\pgfusepath{clip}%
\pgfsetbuttcap%
\pgfsetroundjoin%
\definecolor{currentfill}{rgb}{0.279566,0.067836,0.391917}%
\pgfsetfillcolor{currentfill}%
\pgfsetlinewidth{0.000000pt}%
\definecolor{currentstroke}{rgb}{0.000000,0.000000,0.000000}%
\pgfsetstrokecolor{currentstroke}%
\pgfsetdash{}{0pt}%
\pgfpathmoveto{\pgfqpoint{0.990369in}{0.912011in}}%
\pgfpathlineto{\pgfqpoint{0.987721in}{0.907270in}}%
\pgfpathlineto{\pgfqpoint{0.985071in}{0.902679in}}%
\pgfpathlineto{\pgfqpoint{0.982417in}{0.898244in}}%
\pgfpathlineto{\pgfqpoint{0.979760in}{0.893967in}}%
\pgfpathlineto{\pgfqpoint{0.970418in}{0.900296in}}%
\pgfpathlineto{\pgfqpoint{0.961482in}{0.906772in}}%
\pgfpathlineto{\pgfqpoint{0.952961in}{0.913389in}}%
\pgfpathlineto{\pgfqpoint{0.944863in}{0.920138in}}%
\pgfpathlineto{\pgfqpoint{0.947758in}{0.924212in}}%
\pgfpathlineto{\pgfqpoint{0.950650in}{0.928444in}}%
\pgfpathlineto{\pgfqpoint{0.953539in}{0.932831in}}%
\pgfpathlineto{\pgfqpoint{0.956425in}{0.937369in}}%
\pgfpathlineto{\pgfqpoint{0.964304in}{0.930829in}}%
\pgfpathlineto{\pgfqpoint{0.972593in}{0.924418in}}%
\pgfpathlineto{\pgfqpoint{0.981284in}{0.918143in}}%
\pgfpathlineto{\pgfqpoint{0.990369in}{0.912011in}}%
\pgfpathclose%
\pgfusepath{fill}%
\end{pgfscope}%
\begin{pgfscope}%
\pgfpathrectangle{\pgfqpoint{0.329460in}{0.284240in}}{\pgfqpoint{1.989680in}{1.989680in}}%
\pgfusepath{clip}%
\pgfsetbuttcap%
\pgfsetroundjoin%
\definecolor{currentfill}{rgb}{0.762373,0.876424,0.137064}%
\pgfsetfillcolor{currentfill}%
\pgfsetlinewidth{0.000000pt}%
\definecolor{currentstroke}{rgb}{0.000000,0.000000,0.000000}%
\pgfsetstrokecolor{currentstroke}%
\pgfsetdash{}{0pt}%
\pgfpathmoveto{\pgfqpoint{1.337403in}{1.723143in}}%
\pgfpathlineto{\pgfqpoint{1.336979in}{1.721066in}}%
\pgfpathlineto{\pgfqpoint{1.336556in}{1.718889in}}%
\pgfpathlineto{\pgfqpoint{1.336132in}{1.716612in}}%
\pgfpathlineto{\pgfqpoint{1.335710in}{1.714238in}}%
\pgfpathlineto{\pgfqpoint{1.339621in}{1.714437in}}%
\pgfpathlineto{\pgfqpoint{1.343543in}{1.714579in}}%
\pgfpathlineto{\pgfqpoint{1.347473in}{1.714663in}}%
\pgfpathlineto{\pgfqpoint{1.351406in}{1.714688in}}%
\pgfpathlineto{\pgfqpoint{1.351400in}{1.717051in}}%
\pgfpathlineto{\pgfqpoint{1.351394in}{1.719315in}}%
\pgfpathlineto{\pgfqpoint{1.351388in}{1.721480in}}%
\pgfpathlineto{\pgfqpoint{1.351382in}{1.723544in}}%
\pgfpathlineto{\pgfqpoint{1.347879in}{1.723521in}}%
\pgfpathlineto{\pgfqpoint{1.344380in}{1.723447in}}%
\pgfpathlineto{\pgfqpoint{1.340886in}{1.723321in}}%
\pgfpathlineto{\pgfqpoint{1.337403in}{1.723143in}}%
\pgfpathclose%
\pgfusepath{fill}%
\end{pgfscope}%
\begin{pgfscope}%
\pgfpathrectangle{\pgfqpoint{0.329460in}{0.284240in}}{\pgfqpoint{1.989680in}{1.989680in}}%
\pgfusepath{clip}%
\pgfsetbuttcap%
\pgfsetroundjoin%
\definecolor{currentfill}{rgb}{0.762373,0.876424,0.137064}%
\pgfsetfillcolor{currentfill}%
\pgfsetlinewidth{0.000000pt}%
\definecolor{currentstroke}{rgb}{0.000000,0.000000,0.000000}%
\pgfsetstrokecolor{currentstroke}%
\pgfsetdash{}{0pt}%
\pgfpathmoveto{\pgfqpoint{1.351382in}{1.723544in}}%
\pgfpathlineto{\pgfqpoint{1.351388in}{1.721480in}}%
\pgfpathlineto{\pgfqpoint{1.351394in}{1.719315in}}%
\pgfpathlineto{\pgfqpoint{1.351400in}{1.717051in}}%
\pgfpathlineto{\pgfqpoint{1.351406in}{1.714688in}}%
\pgfpathlineto{\pgfqpoint{1.355339in}{1.714656in}}%
\pgfpathlineto{\pgfqpoint{1.359268in}{1.714566in}}%
\pgfpathlineto{\pgfqpoint{1.363189in}{1.714418in}}%
\pgfpathlineto{\pgfqpoint{1.367099in}{1.714212in}}%
\pgfpathlineto{\pgfqpoint{1.366665in}{1.716587in}}%
\pgfpathlineto{\pgfqpoint{1.366230in}{1.718865in}}%
\pgfpathlineto{\pgfqpoint{1.365794in}{1.721043in}}%
\pgfpathlineto{\pgfqpoint{1.365358in}{1.723120in}}%
\pgfpathlineto{\pgfqpoint{1.361876in}{1.723304in}}%
\pgfpathlineto{\pgfqpoint{1.358384in}{1.723435in}}%
\pgfpathlineto{\pgfqpoint{1.354885in}{1.723515in}}%
\pgfpathlineto{\pgfqpoint{1.351382in}{1.723544in}}%
\pgfpathclose%
\pgfusepath{fill}%
\end{pgfscope}%
\begin{pgfscope}%
\pgfpathrectangle{\pgfqpoint{0.329460in}{0.284240in}}{\pgfqpoint{1.989680in}{1.989680in}}%
\pgfusepath{clip}%
\pgfsetbuttcap%
\pgfsetroundjoin%
\definecolor{currentfill}{rgb}{0.267004,0.004874,0.329415}%
\pgfsetfillcolor{currentfill}%
\pgfsetlinewidth{0.000000pt}%
\definecolor{currentstroke}{rgb}{0.000000,0.000000,0.000000}%
\pgfsetstrokecolor{currentstroke}%
\pgfsetdash{}{0pt}%
\pgfpathmoveto{\pgfqpoint{1.799974in}{0.891695in}}%
\pgfpathlineto{\pgfqpoint{1.802976in}{0.890086in}}%
\pgfpathlineto{\pgfqpoint{1.805984in}{0.888692in}}%
\pgfpathlineto{\pgfqpoint{1.808998in}{0.887516in}}%
\pgfpathlineto{\pgfqpoint{1.812018in}{0.886565in}}%
\pgfpathlineto{\pgfqpoint{1.803466in}{0.878865in}}%
\pgfpathlineto{\pgfqpoint{1.794431in}{0.871305in}}%
\pgfpathlineto{\pgfqpoint{1.784924in}{0.863894in}}%
\pgfpathlineto{\pgfqpoint{1.774951in}{0.856639in}}%
\pgfpathlineto{\pgfqpoint{1.772161in}{0.857791in}}%
\pgfpathlineto{\pgfqpoint{1.769377in}{0.859168in}}%
\pgfpathlineto{\pgfqpoint{1.766599in}{0.860765in}}%
\pgfpathlineto{\pgfqpoint{1.763827in}{0.862577in}}%
\pgfpathlineto{\pgfqpoint{1.773550in}{0.869635in}}%
\pgfpathlineto{\pgfqpoint{1.782821in}{0.876847in}}%
\pgfpathlineto{\pgfqpoint{1.791632in}{0.884203in}}%
\pgfpathlineto{\pgfqpoint{1.799974in}{0.891695in}}%
\pgfpathclose%
\pgfusepath{fill}%
\end{pgfscope}%
\begin{pgfscope}%
\pgfpathrectangle{\pgfqpoint{0.329460in}{0.284240in}}{\pgfqpoint{1.989680in}{1.989680in}}%
\pgfusepath{clip}%
\pgfsetbuttcap%
\pgfsetroundjoin%
\definecolor{currentfill}{rgb}{0.274952,0.037752,0.364543}%
\pgfsetfillcolor{currentfill}%
\pgfsetlinewidth{0.000000pt}%
\definecolor{currentstroke}{rgb}{0.000000,0.000000,0.000000}%
\pgfsetstrokecolor{currentstroke}%
\pgfsetdash{}{0pt}%
\pgfpathmoveto{\pgfqpoint{0.979760in}{0.893967in}}%
\pgfpathlineto{\pgfqpoint{0.977099in}{0.889852in}}%
\pgfpathlineto{\pgfqpoint{0.974435in}{0.885904in}}%
\pgfpathlineto{\pgfqpoint{0.971767in}{0.882125in}}%
\pgfpathlineto{\pgfqpoint{0.969096in}{0.878521in}}%
\pgfpathlineto{\pgfqpoint{0.959496in}{0.885047in}}%
\pgfpathlineto{\pgfqpoint{0.950316in}{0.891724in}}%
\pgfpathlineto{\pgfqpoint{0.941562in}{0.898546in}}%
\pgfpathlineto{\pgfqpoint{0.933245in}{0.905504in}}%
\pgfpathlineto{\pgfqpoint{0.936155in}{0.908905in}}%
\pgfpathlineto{\pgfqpoint{0.939061in}{0.912481in}}%
\pgfpathlineto{\pgfqpoint{0.941964in}{0.916227in}}%
\pgfpathlineto{\pgfqpoint{0.944863in}{0.920138in}}%
\pgfpathlineto{\pgfqpoint{0.952961in}{0.913389in}}%
\pgfpathlineto{\pgfqpoint{0.961482in}{0.906772in}}%
\pgfpathlineto{\pgfqpoint{0.970418in}{0.900296in}}%
\pgfpathlineto{\pgfqpoint{0.979760in}{0.893967in}}%
\pgfpathclose%
\pgfusepath{fill}%
\end{pgfscope}%
\begin{pgfscope}%
\pgfpathrectangle{\pgfqpoint{0.329460in}{0.284240in}}{\pgfqpoint{1.989680in}{1.989680in}}%
\pgfusepath{clip}%
\pgfsetbuttcap%
\pgfsetroundjoin%
\definecolor{currentfill}{rgb}{0.134692,0.658636,0.517649}%
\pgfsetfillcolor{currentfill}%
\pgfsetlinewidth{0.000000pt}%
\definecolor{currentstroke}{rgb}{0.000000,0.000000,0.000000}%
\pgfsetstrokecolor{currentstroke}%
\pgfsetdash{}{0pt}%
\pgfpathmoveto{\pgfqpoint{1.600258in}{1.452599in}}%
\pgfpathlineto{\pgfqpoint{1.603631in}{1.445915in}}%
\pgfpathlineto{\pgfqpoint{1.607001in}{1.439196in}}%
\pgfpathlineto{\pgfqpoint{1.610369in}{1.432444in}}%
\pgfpathlineto{\pgfqpoint{1.613734in}{1.425662in}}%
\pgfpathlineto{\pgfqpoint{1.613436in}{1.421598in}}%
\pgfpathlineto{\pgfqpoint{1.612876in}{1.417538in}}%
\pgfpathlineto{\pgfqpoint{1.612053in}{1.413485in}}%
\pgfpathlineto{\pgfqpoint{1.610968in}{1.409445in}}%
\pgfpathlineto{\pgfqpoint{1.607630in}{1.416453in}}%
\pgfpathlineto{\pgfqpoint{1.604290in}{1.423431in}}%
\pgfpathlineto{\pgfqpoint{1.600947in}{1.430375in}}%
\pgfpathlineto{\pgfqpoint{1.597601in}{1.437284in}}%
\pgfpathlineto{\pgfqpoint{1.598638in}{1.441099in}}%
\pgfpathlineto{\pgfqpoint{1.599427in}{1.444926in}}%
\pgfpathlineto{\pgfqpoint{1.599967in}{1.448760in}}%
\pgfpathlineto{\pgfqpoint{1.600258in}{1.452599in}}%
\pgfpathclose%
\pgfusepath{fill}%
\end{pgfscope}%
\begin{pgfscope}%
\pgfpathrectangle{\pgfqpoint{0.329460in}{0.284240in}}{\pgfqpoint{1.989680in}{1.989680in}}%
\pgfusepath{clip}%
\pgfsetbuttcap%
\pgfsetroundjoin%
\definecolor{currentfill}{rgb}{0.277941,0.056324,0.381191}%
\pgfsetfillcolor{currentfill}%
\pgfsetlinewidth{0.000000pt}%
\definecolor{currentstroke}{rgb}{0.000000,0.000000,0.000000}%
\pgfsetstrokecolor{currentstroke}%
\pgfsetdash{}{0pt}%
\pgfpathmoveto{\pgfqpoint{0.861664in}{0.886853in}}%
\pgfpathlineto{\pgfqpoint{0.858586in}{0.889185in}}%
\pgfpathlineto{\pgfqpoint{0.855497in}{0.891807in}}%
\pgfpathlineto{\pgfqpoint{0.852398in}{0.894725in}}%
\pgfpathlineto{\pgfqpoint{0.849288in}{0.897943in}}%
\pgfpathlineto{\pgfqpoint{0.839941in}{0.906461in}}%
\pgfpathlineto{\pgfqpoint{0.831140in}{0.915121in}}%
\pgfpathlineto{\pgfqpoint{0.822893in}{0.923911in}}%
\pgfpathlineto{\pgfqpoint{0.815205in}{0.932824in}}%
\pgfpathlineto{\pgfqpoint{0.818508in}{0.929407in}}%
\pgfpathlineto{\pgfqpoint{0.821800in}{0.926290in}}%
\pgfpathlineto{\pgfqpoint{0.825082in}{0.923468in}}%
\pgfpathlineto{\pgfqpoint{0.828353in}{0.920935in}}%
\pgfpathlineto{\pgfqpoint{0.835870in}{0.912226in}}%
\pgfpathlineto{\pgfqpoint{0.843932in}{0.903636in}}%
\pgfpathlineto{\pgfqpoint{0.852532in}{0.895175in}}%
\pgfpathlineto{\pgfqpoint{0.861664in}{0.886853in}}%
\pgfpathclose%
\pgfusepath{fill}%
\end{pgfscope}%
\begin{pgfscope}%
\pgfpathrectangle{\pgfqpoint{0.329460in}{0.284240in}}{\pgfqpoint{1.989680in}{1.989680in}}%
\pgfusepath{clip}%
\pgfsetbuttcap%
\pgfsetroundjoin%
\definecolor{currentfill}{rgb}{0.282327,0.094955,0.417331}%
\pgfsetfillcolor{currentfill}%
\pgfsetlinewidth{0.000000pt}%
\definecolor{currentstroke}{rgb}{0.000000,0.000000,0.000000}%
\pgfsetstrokecolor{currentstroke}%
\pgfsetdash{}{0pt}%
\pgfpathmoveto{\pgfqpoint{1.000933in}{0.932414in}}%
\pgfpathlineto{\pgfqpoint{0.998296in}{0.927105in}}%
\pgfpathlineto{\pgfqpoint{0.995656in}{0.921933in}}%
\pgfpathlineto{\pgfqpoint{0.993014in}{0.916900in}}%
\pgfpathlineto{\pgfqpoint{0.990369in}{0.912011in}}%
\pgfpathlineto{\pgfqpoint{0.981284in}{0.918143in}}%
\pgfpathlineto{\pgfqpoint{0.972593in}{0.924418in}}%
\pgfpathlineto{\pgfqpoint{0.964304in}{0.930829in}}%
\pgfpathlineto{\pgfqpoint{0.956425in}{0.937369in}}%
\pgfpathlineto{\pgfqpoint{0.959307in}{0.942054in}}%
\pgfpathlineto{\pgfqpoint{0.962187in}{0.946882in}}%
\pgfpathlineto{\pgfqpoint{0.965065in}{0.951851in}}%
\pgfpathlineto{\pgfqpoint{0.967940in}{0.956955in}}%
\pgfpathlineto{\pgfqpoint{0.975600in}{0.950625in}}%
\pgfpathlineto{\pgfqpoint{0.983657in}{0.944420in}}%
\pgfpathlineto{\pgfqpoint{0.992104in}{0.938348in}}%
\pgfpathlineto{\pgfqpoint{1.000933in}{0.932414in}}%
\pgfpathclose%
\pgfusepath{fill}%
\end{pgfscope}%
\begin{pgfscope}%
\pgfpathrectangle{\pgfqpoint{0.329460in}{0.284240in}}{\pgfqpoint{1.989680in}{1.989680in}}%
\pgfusepath{clip}%
\pgfsetbuttcap%
\pgfsetroundjoin%
\definecolor{currentfill}{rgb}{0.762373,0.876424,0.137064}%
\pgfsetfillcolor{currentfill}%
\pgfsetlinewidth{0.000000pt}%
\definecolor{currentstroke}{rgb}{0.000000,0.000000,0.000000}%
\pgfsetstrokecolor{currentstroke}%
\pgfsetdash{}{0pt}%
\pgfpathmoveto{\pgfqpoint{1.323638in}{1.721924in}}%
\pgfpathlineto{\pgfqpoint{1.322790in}{1.719809in}}%
\pgfpathlineto{\pgfqpoint{1.321944in}{1.717594in}}%
\pgfpathlineto{\pgfqpoint{1.321098in}{1.715279in}}%
\pgfpathlineto{\pgfqpoint{1.320253in}{1.712867in}}%
\pgfpathlineto{\pgfqpoint{1.324081in}{1.713295in}}%
\pgfpathlineto{\pgfqpoint{1.327936in}{1.713666in}}%
\pgfpathlineto{\pgfqpoint{1.331814in}{1.713981in}}%
\pgfpathlineto{\pgfqpoint{1.335710in}{1.714238in}}%
\pgfpathlineto{\pgfqpoint{1.336132in}{1.716612in}}%
\pgfpathlineto{\pgfqpoint{1.336556in}{1.718889in}}%
\pgfpathlineto{\pgfqpoint{1.336979in}{1.721066in}}%
\pgfpathlineto{\pgfqpoint{1.337403in}{1.723143in}}%
\pgfpathlineto{\pgfqpoint{1.333933in}{1.722915in}}%
\pgfpathlineto{\pgfqpoint{1.330480in}{1.722635in}}%
\pgfpathlineto{\pgfqpoint{1.327048in}{1.722305in}}%
\pgfpathlineto{\pgfqpoint{1.323638in}{1.721924in}}%
\pgfpathclose%
\pgfusepath{fill}%
\end{pgfscope}%
\begin{pgfscope}%
\pgfpathrectangle{\pgfqpoint{0.329460in}{0.284240in}}{\pgfqpoint{1.989680in}{1.989680in}}%
\pgfusepath{clip}%
\pgfsetbuttcap%
\pgfsetroundjoin%
\definecolor{currentfill}{rgb}{0.762373,0.876424,0.137064}%
\pgfsetfillcolor{currentfill}%
\pgfsetlinewidth{0.000000pt}%
\definecolor{currentstroke}{rgb}{0.000000,0.000000,0.000000}%
\pgfsetstrokecolor{currentstroke}%
\pgfsetdash{}{0pt}%
\pgfpathmoveto{\pgfqpoint{1.365358in}{1.723120in}}%
\pgfpathlineto{\pgfqpoint{1.365794in}{1.721043in}}%
\pgfpathlineto{\pgfqpoint{1.366230in}{1.718865in}}%
\pgfpathlineto{\pgfqpoint{1.366665in}{1.716587in}}%
\pgfpathlineto{\pgfqpoint{1.367099in}{1.714212in}}%
\pgfpathlineto{\pgfqpoint{1.370993in}{1.713949in}}%
\pgfpathlineto{\pgfqpoint{1.374868in}{1.713628in}}%
\pgfpathlineto{\pgfqpoint{1.378720in}{1.713250in}}%
\pgfpathlineto{\pgfqpoint{1.382546in}{1.712816in}}%
\pgfpathlineto{\pgfqpoint{1.381690in}{1.715230in}}%
\pgfpathlineto{\pgfqpoint{1.380832in}{1.717546in}}%
\pgfpathlineto{\pgfqpoint{1.379974in}{1.719763in}}%
\pgfpathlineto{\pgfqpoint{1.379114in}{1.721879in}}%
\pgfpathlineto{\pgfqpoint{1.375708in}{1.722265in}}%
\pgfpathlineto{\pgfqpoint{1.372277in}{1.722601in}}%
\pgfpathlineto{\pgfqpoint{1.368826in}{1.722886in}}%
\pgfpathlineto{\pgfqpoint{1.365358in}{1.723120in}}%
\pgfpathclose%
\pgfusepath{fill}%
\end{pgfscope}%
\begin{pgfscope}%
\pgfpathrectangle{\pgfqpoint{0.329460in}{0.284240in}}{\pgfqpoint{1.989680in}{1.989680in}}%
\pgfusepath{clip}%
\pgfsetbuttcap%
\pgfsetroundjoin%
\definecolor{currentfill}{rgb}{0.487026,0.823929,0.312321}%
\pgfsetfillcolor{currentfill}%
\pgfsetlinewidth{0.000000pt}%
\definecolor{currentstroke}{rgb}{0.000000,0.000000,0.000000}%
\pgfsetstrokecolor{currentstroke}%
\pgfsetdash{}{0pt}%
\pgfpathmoveto{\pgfqpoint{1.499918in}{1.635986in}}%
\pgfpathlineto{\pgfqpoint{1.502956in}{1.631562in}}%
\pgfpathlineto{\pgfqpoint{1.505991in}{1.627059in}}%
\pgfpathlineto{\pgfqpoint{1.509024in}{1.622478in}}%
\pgfpathlineto{\pgfqpoint{1.512053in}{1.617822in}}%
\pgfpathlineto{\pgfqpoint{1.514581in}{1.615390in}}%
\pgfpathlineto{\pgfqpoint{1.516949in}{1.612921in}}%
\pgfpathlineto{\pgfqpoint{1.519155in}{1.610416in}}%
\pgfpathlineto{\pgfqpoint{1.521197in}{1.607878in}}%
\pgfpathlineto{\pgfqpoint{1.517990in}{1.612730in}}%
\pgfpathlineto{\pgfqpoint{1.514781in}{1.617506in}}%
\pgfpathlineto{\pgfqpoint{1.511568in}{1.622203in}}%
\pgfpathlineto{\pgfqpoint{1.508352in}{1.626822in}}%
\pgfpathlineto{\pgfqpoint{1.506470in}{1.629160in}}%
\pgfpathlineto{\pgfqpoint{1.504436in}{1.631469in}}%
\pgfpathlineto{\pgfqpoint{1.502251in}{1.633744in}}%
\pgfpathlineto{\pgfqpoint{1.499918in}{1.635986in}}%
\pgfpathclose%
\pgfusepath{fill}%
\end{pgfscope}%
\begin{pgfscope}%
\pgfpathrectangle{\pgfqpoint{0.329460in}{0.284240in}}{\pgfqpoint{1.989680in}{1.989680in}}%
\pgfusepath{clip}%
\pgfsetbuttcap%
\pgfsetroundjoin%
\definecolor{currentfill}{rgb}{0.271305,0.019942,0.347269}%
\pgfsetfillcolor{currentfill}%
\pgfsetlinewidth{0.000000pt}%
\definecolor{currentstroke}{rgb}{0.000000,0.000000,0.000000}%
\pgfsetstrokecolor{currentstroke}%
\pgfsetdash{}{0pt}%
\pgfpathmoveto{\pgfqpoint{0.969096in}{0.878521in}}%
\pgfpathlineto{\pgfqpoint{0.966420in}{0.875095in}}%
\pgfpathlineto{\pgfqpoint{0.963740in}{0.871851in}}%
\pgfpathlineto{\pgfqpoint{0.961056in}{0.868793in}}%
\pgfpathlineto{\pgfqpoint{0.958367in}{0.865925in}}%
\pgfpathlineto{\pgfqpoint{0.948509in}{0.872646in}}%
\pgfpathlineto{\pgfqpoint{0.939083in}{0.879524in}}%
\pgfpathlineto{\pgfqpoint{0.930097in}{0.886549in}}%
\pgfpathlineto{\pgfqpoint{0.921559in}{0.893714in}}%
\pgfpathlineto{\pgfqpoint{0.924487in}{0.896381in}}%
\pgfpathlineto{\pgfqpoint{0.927411in}{0.899238in}}%
\pgfpathlineto{\pgfqpoint{0.930330in}{0.902280in}}%
\pgfpathlineto{\pgfqpoint{0.933245in}{0.905504in}}%
\pgfpathlineto{\pgfqpoint{0.941562in}{0.898546in}}%
\pgfpathlineto{\pgfqpoint{0.950316in}{0.891724in}}%
\pgfpathlineto{\pgfqpoint{0.959496in}{0.885047in}}%
\pgfpathlineto{\pgfqpoint{0.969096in}{0.878521in}}%
\pgfpathclose%
\pgfusepath{fill}%
\end{pgfscope}%
\begin{pgfscope}%
\pgfpathrectangle{\pgfqpoint{0.329460in}{0.284240in}}{\pgfqpoint{1.989680in}{1.989680in}}%
\pgfusepath{clip}%
\pgfsetbuttcap%
\pgfsetroundjoin%
\definecolor{currentfill}{rgb}{0.122606,0.585371,0.546557}%
\pgfsetfillcolor{currentfill}%
\pgfsetlinewidth{0.000000pt}%
\definecolor{currentstroke}{rgb}{0.000000,0.000000,0.000000}%
\pgfsetstrokecolor{currentstroke}%
\pgfsetdash{}{0pt}%
\pgfpathmoveto{\pgfqpoint{1.624295in}{1.381149in}}%
\pgfpathlineto{\pgfqpoint{1.627621in}{1.374021in}}%
\pgfpathlineto{\pgfqpoint{1.630944in}{1.366876in}}%
\pgfpathlineto{\pgfqpoint{1.634265in}{1.359716in}}%
\pgfpathlineto{\pgfqpoint{1.637583in}{1.352544in}}%
\pgfpathlineto{\pgfqpoint{1.636118in}{1.348066in}}%
\pgfpathlineto{\pgfqpoint{1.634366in}{1.343611in}}%
\pgfpathlineto{\pgfqpoint{1.632327in}{1.339181in}}%
\pgfpathlineto{\pgfqpoint{1.630003in}{1.334783in}}%
\pgfpathlineto{\pgfqpoint{1.626763in}{1.342182in}}%
\pgfpathlineto{\pgfqpoint{1.623522in}{1.349568in}}%
\pgfpathlineto{\pgfqpoint{1.620278in}{1.356939in}}%
\pgfpathlineto{\pgfqpoint{1.617032in}{1.364293in}}%
\pgfpathlineto{\pgfqpoint{1.619256in}{1.368467in}}%
\pgfpathlineto{\pgfqpoint{1.621209in}{1.372670in}}%
\pgfpathlineto{\pgfqpoint{1.622890in}{1.376899in}}%
\pgfpathlineto{\pgfqpoint{1.624295in}{1.381149in}}%
\pgfpathclose%
\pgfusepath{fill}%
\end{pgfscope}%
\begin{pgfscope}%
\pgfpathrectangle{\pgfqpoint{0.329460in}{0.284240in}}{\pgfqpoint{1.989680in}{1.989680in}}%
\pgfusepath{clip}%
\pgfsetbuttcap%
\pgfsetroundjoin%
\definecolor{currentfill}{rgb}{0.565498,0.842430,0.262877}%
\pgfsetfillcolor{currentfill}%
\pgfsetlinewidth{0.000000pt}%
\definecolor{currentstroke}{rgb}{0.000000,0.000000,0.000000}%
\pgfsetstrokecolor{currentstroke}%
\pgfsetdash{}{0pt}%
\pgfpathmoveto{\pgfqpoint{1.477871in}{1.660711in}}%
\pgfpathlineto{\pgfqpoint{1.480697in}{1.656799in}}%
\pgfpathlineto{\pgfqpoint{1.483520in}{1.652803in}}%
\pgfpathlineto{\pgfqpoint{1.486340in}{1.648724in}}%
\pgfpathlineto{\pgfqpoint{1.489158in}{1.644562in}}%
\pgfpathlineto{\pgfqpoint{1.492057in}{1.642480in}}%
\pgfpathlineto{\pgfqpoint{1.494818in}{1.640356in}}%
\pgfpathlineto{\pgfqpoint{1.497440in}{1.638190in}}%
\pgfpathlineto{\pgfqpoint{1.499918in}{1.635986in}}%
\pgfpathlineto{\pgfqpoint{1.496877in}{1.640329in}}%
\pgfpathlineto{\pgfqpoint{1.493832in}{1.644590in}}%
\pgfpathlineto{\pgfqpoint{1.490785in}{1.648767in}}%
\pgfpathlineto{\pgfqpoint{1.487734in}{1.652859in}}%
\pgfpathlineto{\pgfqpoint{1.485463in}{1.654877in}}%
\pgfpathlineto{\pgfqpoint{1.483060in}{1.656859in}}%
\pgfpathlineto{\pgfqpoint{1.480529in}{1.658805in}}%
\pgfpathlineto{\pgfqpoint{1.477871in}{1.660711in}}%
\pgfpathclose%
\pgfusepath{fill}%
\end{pgfscope}%
\begin{pgfscope}%
\pgfpathrectangle{\pgfqpoint{0.329460in}{0.284240in}}{\pgfqpoint{1.989680in}{1.989680in}}%
\pgfusepath{clip}%
\pgfsetbuttcap%
\pgfsetroundjoin%
\definecolor{currentfill}{rgb}{0.220124,0.725509,0.466226}%
\pgfsetfillcolor{currentfill}%
\pgfsetlinewidth{0.000000pt}%
\definecolor{currentstroke}{rgb}{0.000000,0.000000,0.000000}%
\pgfsetstrokecolor{currentstroke}%
\pgfsetdash{}{0pt}%
\pgfpathmoveto{\pgfqpoint{1.572056in}{1.518097in}}%
\pgfpathlineto{\pgfqpoint{1.575427in}{1.511990in}}%
\pgfpathlineto{\pgfqpoint{1.578795in}{1.505831in}}%
\pgfpathlineto{\pgfqpoint{1.582160in}{1.499623in}}%
\pgfpathlineto{\pgfqpoint{1.585522in}{1.493368in}}%
\pgfpathlineto{\pgfqpoint{1.586177in}{1.489773in}}%
\pgfpathlineto{\pgfqpoint{1.586598in}{1.486169in}}%
\pgfpathlineto{\pgfqpoint{1.586784in}{1.482558in}}%
\pgfpathlineto{\pgfqpoint{1.586736in}{1.478943in}}%
\pgfpathlineto{\pgfqpoint{1.583349in}{1.485420in}}%
\pgfpathlineto{\pgfqpoint{1.579958in}{1.491850in}}%
\pgfpathlineto{\pgfqpoint{1.576565in}{1.498230in}}%
\pgfpathlineto{\pgfqpoint{1.573169in}{1.504559in}}%
\pgfpathlineto{\pgfqpoint{1.573222in}{1.507951in}}%
\pgfpathlineto{\pgfqpoint{1.573054in}{1.511340in}}%
\pgfpathlineto{\pgfqpoint{1.572665in}{1.514723in}}%
\pgfpathlineto{\pgfqpoint{1.572056in}{1.518097in}}%
\pgfpathclose%
\pgfusepath{fill}%
\end{pgfscope}%
\begin{pgfscope}%
\pgfpathrectangle{\pgfqpoint{0.329460in}{0.284240in}}{\pgfqpoint{1.989680in}{1.989680in}}%
\pgfusepath{clip}%
\pgfsetbuttcap%
\pgfsetroundjoin%
\definecolor{currentfill}{rgb}{0.699415,0.867117,0.175971}%
\pgfsetfillcolor{currentfill}%
\pgfsetlinewidth{0.000000pt}%
\definecolor{currentstroke}{rgb}{0.000000,0.000000,0.000000}%
\pgfsetstrokecolor{currentstroke}%
\pgfsetdash{}{0pt}%
\pgfpathmoveto{\pgfqpoint{1.421831in}{1.704451in}}%
\pgfpathlineto{\pgfqpoint{1.423761in}{1.701708in}}%
\pgfpathlineto{\pgfqpoint{1.425689in}{1.698870in}}%
\pgfpathlineto{\pgfqpoint{1.427615in}{1.695936in}}%
\pgfpathlineto{\pgfqpoint{1.429539in}{1.692910in}}%
\pgfpathlineto{\pgfqpoint{1.433119in}{1.691723in}}%
\pgfpathlineto{\pgfqpoint{1.436620in}{1.690484in}}%
\pgfpathlineto{\pgfqpoint{1.440038in}{1.689193in}}%
\pgfpathlineto{\pgfqpoint{1.443369in}{1.687852in}}%
\pgfpathlineto{\pgfqpoint{1.441104in}{1.691006in}}%
\pgfpathlineto{\pgfqpoint{1.438836in}{1.694066in}}%
\pgfpathlineto{\pgfqpoint{1.436566in}{1.697031in}}%
\pgfpathlineto{\pgfqpoint{1.434294in}{1.699901in}}%
\pgfpathlineto{\pgfqpoint{1.431292in}{1.701107in}}%
\pgfpathlineto{\pgfqpoint{1.428212in}{1.702269in}}%
\pgfpathlineto{\pgfqpoint{1.425058in}{1.703384in}}%
\pgfpathlineto{\pgfqpoint{1.421831in}{1.704451in}}%
\pgfpathclose%
\pgfusepath{fill}%
\end{pgfscope}%
\begin{pgfscope}%
\pgfpathrectangle{\pgfqpoint{0.329460in}{0.284240in}}{\pgfqpoint{1.989680in}{1.989680in}}%
\pgfusepath{clip}%
\pgfsetbuttcap%
\pgfsetroundjoin%
\definecolor{currentfill}{rgb}{0.282327,0.094955,0.417331}%
\pgfsetfillcolor{currentfill}%
\pgfsetlinewidth{0.000000pt}%
\definecolor{currentstroke}{rgb}{0.000000,0.000000,0.000000}%
\pgfsetstrokecolor{currentstroke}%
\pgfsetdash{}{0pt}%
\pgfpathmoveto{\pgfqpoint{1.893530in}{0.940840in}}%
\pgfpathlineto{\pgfqpoint{1.896880in}{0.944607in}}%
\pgfpathlineto{\pgfqpoint{1.900241in}{0.948684in}}%
\pgfpathlineto{\pgfqpoint{1.903615in}{0.953077in}}%
\pgfpathlineto{\pgfqpoint{1.907001in}{0.957791in}}%
\pgfpathlineto{\pgfqpoint{1.899655in}{0.948579in}}%
\pgfpathlineto{\pgfqpoint{1.891730in}{0.939481in}}%
\pgfpathlineto{\pgfqpoint{1.883231in}{0.930508in}}%
\pgfpathlineto{\pgfqpoint{1.874165in}{0.921670in}}%
\pgfpathlineto{\pgfqpoint{1.870962in}{0.917151in}}%
\pgfpathlineto{\pgfqpoint{1.867771in}{0.912955in}}%
\pgfpathlineto{\pgfqpoint{1.864591in}{0.909076in}}%
\pgfpathlineto{\pgfqpoint{1.861422in}{0.905508in}}%
\pgfpathlineto{\pgfqpoint{1.870284in}{0.914152in}}%
\pgfpathlineto{\pgfqpoint{1.878594in}{0.922928in}}%
\pgfpathlineto{\pgfqpoint{1.886344in}{0.931828in}}%
\pgfpathlineto{\pgfqpoint{1.893530in}{0.940840in}}%
\pgfpathclose%
\pgfusepath{fill}%
\end{pgfscope}%
\begin{pgfscope}%
\pgfpathrectangle{\pgfqpoint{0.329460in}{0.284240in}}{\pgfqpoint{1.989680in}{1.989680in}}%
\pgfusepath{clip}%
\pgfsetbuttcap%
\pgfsetroundjoin%
\definecolor{currentfill}{rgb}{0.163625,0.471133,0.558148}%
\pgfsetfillcolor{currentfill}%
\pgfsetlinewidth{0.000000pt}%
\definecolor{currentstroke}{rgb}{0.000000,0.000000,0.000000}%
\pgfsetstrokecolor{currentstroke}%
\pgfsetdash{}{0pt}%
\pgfpathmoveto{\pgfqpoint{1.063262in}{1.252367in}}%
\pgfpathlineto{\pgfqpoint{1.060205in}{1.244695in}}%
\pgfpathlineto{\pgfqpoint{1.057150in}{1.237035in}}%
\pgfpathlineto{\pgfqpoint{1.054096in}{1.229390in}}%
\pgfpathlineto{\pgfqpoint{1.051044in}{1.221763in}}%
\pgfpathlineto{\pgfqpoint{1.046872in}{1.226598in}}%
\pgfpathlineto{\pgfqpoint{1.043012in}{1.231493in}}%
\pgfpathlineto{\pgfqpoint{1.039469in}{1.236443in}}%
\pgfpathlineto{\pgfqpoint{1.036245in}{1.241443in}}%
\pgfpathlineto{\pgfqpoint{1.039438in}{1.248847in}}%
\pgfpathlineto{\pgfqpoint{1.042633in}{1.256269in}}%
\pgfpathlineto{\pgfqpoint{1.045829in}{1.263707in}}%
\pgfpathlineto{\pgfqpoint{1.049027in}{1.271157in}}%
\pgfpathlineto{\pgfqpoint{1.052130in}{1.266382in}}%
\pgfpathlineto{\pgfqpoint{1.055539in}{1.261656in}}%
\pgfpathlineto{\pgfqpoint{1.059250in}{1.256983in}}%
\pgfpathlineto{\pgfqpoint{1.063262in}{1.252367in}}%
\pgfpathclose%
\pgfusepath{fill}%
\end{pgfscope}%
\begin{pgfscope}%
\pgfpathrectangle{\pgfqpoint{0.329460in}{0.284240in}}{\pgfqpoint{1.989680in}{1.989680in}}%
\pgfusepath{clip}%
\pgfsetbuttcap%
\pgfsetroundjoin%
\definecolor{currentfill}{rgb}{0.412913,0.803041,0.357269}%
\pgfsetfillcolor{currentfill}%
\pgfsetlinewidth{0.000000pt}%
\definecolor{currentstroke}{rgb}{0.000000,0.000000,0.000000}%
\pgfsetstrokecolor{currentstroke}%
\pgfsetdash{}{0pt}%
\pgfpathmoveto{\pgfqpoint{1.521197in}{1.607878in}}%
\pgfpathlineto{\pgfqpoint{1.524400in}{1.602952in}}%
\pgfpathlineto{\pgfqpoint{1.527600in}{1.597954in}}%
\pgfpathlineto{\pgfqpoint{1.530797in}{1.592884in}}%
\pgfpathlineto{\pgfqpoint{1.533991in}{1.587744in}}%
\pgfpathlineto{\pgfqpoint{1.536012in}{1.584973in}}%
\pgfpathlineto{\pgfqpoint{1.537852in}{1.582172in}}%
\pgfpathlineto{\pgfqpoint{1.539508in}{1.579343in}}%
\pgfpathlineto{\pgfqpoint{1.540979in}{1.576488in}}%
\pgfpathlineto{\pgfqpoint{1.537657in}{1.581834in}}%
\pgfpathlineto{\pgfqpoint{1.534332in}{1.587111in}}%
\pgfpathlineto{\pgfqpoint{1.531004in}{1.592316in}}%
\pgfpathlineto{\pgfqpoint{1.527672in}{1.597447in}}%
\pgfpathlineto{\pgfqpoint{1.526311in}{1.600092in}}%
\pgfpathlineto{\pgfqpoint{1.524776in}{1.602714in}}%
\pgfpathlineto{\pgfqpoint{1.523071in}{1.605310in}}%
\pgfpathlineto{\pgfqpoint{1.521197in}{1.607878in}}%
\pgfpathclose%
\pgfusepath{fill}%
\end{pgfscope}%
\begin{pgfscope}%
\pgfpathrectangle{\pgfqpoint{0.329460in}{0.284240in}}{\pgfqpoint{1.989680in}{1.989680in}}%
\pgfusepath{clip}%
\pgfsetbuttcap%
\pgfsetroundjoin%
\definecolor{currentfill}{rgb}{0.283072,0.130895,0.449241}%
\pgfsetfillcolor{currentfill}%
\pgfsetlinewidth{0.000000pt}%
\definecolor{currentstroke}{rgb}{0.000000,0.000000,0.000000}%
\pgfsetstrokecolor{currentstroke}%
\pgfsetdash{}{0pt}%
\pgfpathmoveto{\pgfqpoint{1.011461in}{0.954940in}}%
\pgfpathlineto{\pgfqpoint{1.008832in}{0.949122in}}%
\pgfpathlineto{\pgfqpoint{1.006201in}{0.943426in}}%
\pgfpathlineto{\pgfqpoint{1.003569in}{0.937855in}}%
\pgfpathlineto{\pgfqpoint{1.000933in}{0.932414in}}%
\pgfpathlineto{\pgfqpoint{0.992104in}{0.938348in}}%
\pgfpathlineto{\pgfqpoint{0.983657in}{0.944420in}}%
\pgfpathlineto{\pgfqpoint{0.975600in}{0.950625in}}%
\pgfpathlineto{\pgfqpoint{0.967940in}{0.956955in}}%
\pgfpathlineto{\pgfqpoint{0.970812in}{0.962193in}}%
\pgfpathlineto{\pgfqpoint{0.973683in}{0.967559in}}%
\pgfpathlineto{\pgfqpoint{0.976551in}{0.973051in}}%
\pgfpathlineto{\pgfqpoint{0.979418in}{0.978665in}}%
\pgfpathlineto{\pgfqpoint{0.986859in}{0.972545in}}%
\pgfpathlineto{\pgfqpoint{0.994685in}{0.966547in}}%
\pgfpathlineto{\pgfqpoint{1.002888in}{0.960676in}}%
\pgfpathlineto{\pgfqpoint{1.011461in}{0.954940in}}%
\pgfpathclose%
\pgfusepath{fill}%
\end{pgfscope}%
\begin{pgfscope}%
\pgfpathrectangle{\pgfqpoint{0.329460in}{0.284240in}}{\pgfqpoint{1.989680in}{1.989680in}}%
\pgfusepath{clip}%
\pgfsetbuttcap%
\pgfsetroundjoin%
\definecolor{currentfill}{rgb}{0.762373,0.876424,0.137064}%
\pgfsetfillcolor{currentfill}%
\pgfsetlinewidth{0.000000pt}%
\definecolor{currentstroke}{rgb}{0.000000,0.000000,0.000000}%
\pgfsetstrokecolor{currentstroke}%
\pgfsetdash{}{0pt}%
\pgfpathmoveto{\pgfqpoint{1.379114in}{1.721879in}}%
\pgfpathlineto{\pgfqpoint{1.379974in}{1.719763in}}%
\pgfpathlineto{\pgfqpoint{1.380832in}{1.717546in}}%
\pgfpathlineto{\pgfqpoint{1.381690in}{1.715230in}}%
\pgfpathlineto{\pgfqpoint{1.382546in}{1.712816in}}%
\pgfpathlineto{\pgfqpoint{1.386341in}{1.712325in}}%
\pgfpathlineto{\pgfqpoint{1.390102in}{1.711779in}}%
\pgfpathlineto{\pgfqpoint{1.393826in}{1.711177in}}%
\pgfpathlineto{\pgfqpoint{1.392661in}{1.713636in}}%
\pgfpathlineto{\pgfqpoint{1.391495in}{1.715998in}}%
\pgfpathlineto{\pgfqpoint{1.390327in}{1.718260in}}%
\pgfpathlineto{\pgfqpoint{1.389159in}{1.720422in}}%
\pgfpathlineto{\pgfqpoint{1.385843in}{1.720957in}}%
\pgfpathlineto{\pgfqpoint{1.382494in}{1.721443in}}%
\pgfpathlineto{\pgfqpoint{1.379114in}{1.721879in}}%
\pgfpathclose%
\pgfusepath{fill}%
\end{pgfscope}%
\begin{pgfscope}%
\pgfpathrectangle{\pgfqpoint{0.329460in}{0.284240in}}{\pgfqpoint{1.989680in}{1.989680in}}%
\pgfusepath{clip}%
\pgfsetbuttcap%
\pgfsetroundjoin%
\definecolor{currentfill}{rgb}{0.274128,0.199721,0.498911}%
\pgfsetfillcolor{currentfill}%
\pgfsetlinewidth{0.000000pt}%
\definecolor{currentstroke}{rgb}{0.000000,0.000000,0.000000}%
\pgfsetstrokecolor{currentstroke}%
\pgfsetdash{}{0pt}%
\pgfpathmoveto{\pgfqpoint{1.705997in}{1.032719in}}%
\pgfpathlineto{\pgfqpoint{1.708899in}{1.026299in}}%
\pgfpathlineto{\pgfqpoint{1.711802in}{1.019973in}}%
\pgfpathlineto{\pgfqpoint{1.714705in}{1.013746in}}%
\pgfpathlineto{\pgfqpoint{1.717610in}{1.007620in}}%
\pgfpathlineto{\pgfqpoint{1.710724in}{1.001611in}}%
\pgfpathlineto{\pgfqpoint{1.703460in}{0.995714in}}%
\pgfpathlineto{\pgfqpoint{1.695825in}{0.989935in}}%
\pgfpathlineto{\pgfqpoint{1.687825in}{0.984281in}}%
\pgfpathlineto{\pgfqpoint{1.685147in}{0.990614in}}%
\pgfpathlineto{\pgfqpoint{1.682470in}{0.997049in}}%
\pgfpathlineto{\pgfqpoint{1.679794in}{1.003581in}}%
\pgfpathlineto{\pgfqpoint{1.677119in}{1.010209in}}%
\pgfpathlineto{\pgfqpoint{1.684874in}{1.015662in}}%
\pgfpathlineto{\pgfqpoint{1.692276in}{1.021235in}}%
\pgfpathlineto{\pgfqpoint{1.699319in}{1.026923in}}%
\pgfpathlineto{\pgfqpoint{1.705997in}{1.032719in}}%
\pgfpathclose%
\pgfusepath{fill}%
\end{pgfscope}%
\begin{pgfscope}%
\pgfpathrectangle{\pgfqpoint{0.329460in}{0.284240in}}{\pgfqpoint{1.989680in}{1.989680in}}%
\pgfusepath{clip}%
\pgfsetbuttcap%
\pgfsetroundjoin%
\definecolor{currentfill}{rgb}{0.762373,0.876424,0.137064}%
\pgfsetfillcolor{currentfill}%
\pgfsetlinewidth{0.000000pt}%
\definecolor{currentstroke}{rgb}{0.000000,0.000000,0.000000}%
\pgfsetstrokecolor{currentstroke}%
\pgfsetdash{}{0pt}%
\pgfpathmoveto{\pgfqpoint{1.310300in}{1.719905in}}%
\pgfpathlineto{\pgfqpoint{1.309042in}{1.717727in}}%
\pgfpathlineto{\pgfqpoint{1.307785in}{1.715449in}}%
\pgfpathlineto{\pgfqpoint{1.306529in}{1.713072in}}%
\pgfpathlineto{\pgfqpoint{1.305274in}{1.710596in}}%
\pgfpathlineto{\pgfqpoint{1.308961in}{1.711246in}}%
\pgfpathlineto{\pgfqpoint{1.312689in}{1.711842in}}%
\pgfpathlineto{\pgfqpoint{1.316454in}{1.712382in}}%
\pgfpathlineto{\pgfqpoint{1.320253in}{1.712867in}}%
\pgfpathlineto{\pgfqpoint{1.321098in}{1.715279in}}%
\pgfpathlineto{\pgfqpoint{1.321944in}{1.717594in}}%
\pgfpathlineto{\pgfqpoint{1.322790in}{1.719809in}}%
\pgfpathlineto{\pgfqpoint{1.323638in}{1.721924in}}%
\pgfpathlineto{\pgfqpoint{1.320255in}{1.721494in}}%
\pgfpathlineto{\pgfqpoint{1.316903in}{1.721013in}}%
\pgfpathlineto{\pgfqpoint{1.313583in}{1.720484in}}%
\pgfpathlineto{\pgfqpoint{1.310300in}{1.719905in}}%
\pgfpathclose%
\pgfusepath{fill}%
\end{pgfscope}%
\begin{pgfscope}%
\pgfpathrectangle{\pgfqpoint{0.329460in}{0.284240in}}{\pgfqpoint{1.989680in}{1.989680in}}%
\pgfusepath{clip}%
\pgfsetbuttcap%
\pgfsetroundjoin%
\definecolor{currentfill}{rgb}{0.699415,0.867117,0.175971}%
\pgfsetfillcolor{currentfill}%
\pgfsetlinewidth{0.000000pt}%
\definecolor{currentstroke}{rgb}{0.000000,0.000000,0.000000}%
\pgfsetstrokecolor{currentstroke}%
\pgfsetdash{}{0pt}%
\pgfpathmoveto{\pgfqpoint{1.265481in}{1.698791in}}%
\pgfpathlineto{\pgfqpoint{1.263137in}{1.695891in}}%
\pgfpathlineto{\pgfqpoint{1.260795in}{1.692894in}}%
\pgfpathlineto{\pgfqpoint{1.258456in}{1.689803in}}%
\pgfpathlineto{\pgfqpoint{1.256120in}{1.686619in}}%
\pgfpathlineto{\pgfqpoint{1.259372in}{1.688004in}}%
\pgfpathlineto{\pgfqpoint{1.262713in}{1.689339in}}%
\pgfpathlineto{\pgfqpoint{1.266140in}{1.690624in}}%
\pgfpathlineto{\pgfqpoint{1.269650in}{1.691858in}}%
\pgfpathlineto{\pgfqpoint{1.271652in}{1.694911in}}%
\pgfpathlineto{\pgfqpoint{1.273657in}{1.697870in}}%
\pgfpathlineto{\pgfqpoint{1.275664in}{1.700735in}}%
\pgfpathlineto{\pgfqpoint{1.277672in}{1.703505in}}%
\pgfpathlineto{\pgfqpoint{1.274510in}{1.702395in}}%
\pgfpathlineto{\pgfqpoint{1.271421in}{1.701239in}}%
\pgfpathlineto{\pgfqpoint{1.268411in}{1.700037in}}%
\pgfpathlineto{\pgfqpoint{1.265481in}{1.698791in}}%
\pgfpathclose%
\pgfusepath{fill}%
\end{pgfscope}%
\begin{pgfscope}%
\pgfpathrectangle{\pgfqpoint{0.329460in}{0.284240in}}{\pgfqpoint{1.989680in}{1.989680in}}%
\pgfusepath{clip}%
\pgfsetbuttcap%
\pgfsetroundjoin%
\definecolor{currentfill}{rgb}{0.268510,0.009605,0.335427}%
\pgfsetfillcolor{currentfill}%
\pgfsetlinewidth{0.000000pt}%
\definecolor{currentstroke}{rgb}{0.000000,0.000000,0.000000}%
\pgfsetstrokecolor{currentstroke}%
\pgfsetdash{}{0pt}%
\pgfpathmoveto{\pgfqpoint{0.958367in}{0.865925in}}%
\pgfpathlineto{\pgfqpoint{0.955674in}{0.863251in}}%
\pgfpathlineto{\pgfqpoint{0.952976in}{0.860776in}}%
\pgfpathlineto{\pgfqpoint{0.950272in}{0.858503in}}%
\pgfpathlineto{\pgfqpoint{0.947564in}{0.856437in}}%
\pgfpathlineto{\pgfqpoint{0.937446in}{0.863353in}}%
\pgfpathlineto{\pgfqpoint{0.927773in}{0.870429in}}%
\pgfpathlineto{\pgfqpoint{0.918552in}{0.877657in}}%
\pgfpathlineto{\pgfqpoint{0.909793in}{0.885029in}}%
\pgfpathlineto{\pgfqpoint{0.912743in}{0.886895in}}%
\pgfpathlineto{\pgfqpoint{0.915687in}{0.888967in}}%
\pgfpathlineto{\pgfqpoint{0.918625in}{0.891242in}}%
\pgfpathlineto{\pgfqpoint{0.921559in}{0.893714in}}%
\pgfpathlineto{\pgfqpoint{0.930097in}{0.886549in}}%
\pgfpathlineto{\pgfqpoint{0.939083in}{0.879524in}}%
\pgfpathlineto{\pgfqpoint{0.948509in}{0.872646in}}%
\pgfpathlineto{\pgfqpoint{0.958367in}{0.865925in}}%
\pgfpathclose%
\pgfusepath{fill}%
\end{pgfscope}%
\begin{pgfscope}%
\pgfpathrectangle{\pgfqpoint{0.329460in}{0.284240in}}{\pgfqpoint{1.989680in}{1.989680in}}%
\pgfusepath{clip}%
\pgfsetbuttcap%
\pgfsetroundjoin%
\definecolor{currentfill}{rgb}{0.487026,0.823929,0.312321}%
\pgfsetfillcolor{currentfill}%
\pgfsetlinewidth{0.000000pt}%
\definecolor{currentstroke}{rgb}{0.000000,0.000000,0.000000}%
\pgfsetstrokecolor{currentstroke}%
\pgfsetdash{}{0pt}%
\pgfpathmoveto{\pgfqpoint{1.192480in}{1.624719in}}%
\pgfpathlineto{\pgfqpoint{1.189231in}{1.620056in}}%
\pgfpathlineto{\pgfqpoint{1.185986in}{1.615314in}}%
\pgfpathlineto{\pgfqpoint{1.182743in}{1.610493in}}%
\pgfpathlineto{\pgfqpoint{1.179504in}{1.605597in}}%
\pgfpathlineto{\pgfqpoint{1.181397in}{1.608162in}}%
\pgfpathlineto{\pgfqpoint{1.183457in}{1.610696in}}%
\pgfpathlineto{\pgfqpoint{1.185681in}{1.613197in}}%
\pgfpathlineto{\pgfqpoint{1.188067in}{1.615663in}}%
\pgfpathlineto{\pgfqpoint{1.191140in}{1.620361in}}%
\pgfpathlineto{\pgfqpoint{1.194216in}{1.624984in}}%
\pgfpathlineto{\pgfqpoint{1.197294in}{1.629529in}}%
\pgfpathlineto{\pgfqpoint{1.200376in}{1.633995in}}%
\pgfpathlineto{\pgfqpoint{1.198175in}{1.631723in}}%
\pgfpathlineto{\pgfqpoint{1.196123in}{1.629418in}}%
\pgfpathlineto{\pgfqpoint{1.194224in}{1.627083in}}%
\pgfpathlineto{\pgfqpoint{1.192480in}{1.624719in}}%
\pgfpathclose%
\pgfusepath{fill}%
\end{pgfscope}%
\begin{pgfscope}%
\pgfpathrectangle{\pgfqpoint{0.329460in}{0.284240in}}{\pgfqpoint{1.989680in}{1.989680in}}%
\pgfusepath{clip}%
\pgfsetbuttcap%
\pgfsetroundjoin%
\definecolor{currentfill}{rgb}{0.260571,0.246922,0.522828}%
\pgfsetfillcolor{currentfill}%
\pgfsetlinewidth{0.000000pt}%
\definecolor{currentstroke}{rgb}{0.000000,0.000000,0.000000}%
\pgfsetstrokecolor{currentstroke}%
\pgfsetdash{}{0pt}%
\pgfpathmoveto{\pgfqpoint{1.959053in}{1.047111in}}%
\pgfpathlineto{\pgfqpoint{1.962692in}{1.055165in}}%
\pgfpathlineto{\pgfqpoint{1.966346in}{1.063598in}}%
\pgfpathlineto{\pgfqpoint{1.970017in}{1.072417in}}%
\pgfpathlineto{\pgfqpoint{1.973706in}{1.081627in}}%
\pgfpathlineto{\pgfqpoint{1.968422in}{1.071479in}}%
\pgfpathlineto{\pgfqpoint{1.962494in}{1.061406in}}%
\pgfpathlineto{\pgfqpoint{1.955927in}{1.051421in}}%
\pgfpathlineto{\pgfqpoint{1.948723in}{1.041533in}}%
\pgfpathlineto{\pgfqpoint{1.945161in}{1.032505in}}%
\pgfpathlineto{\pgfqpoint{1.941615in}{1.023870in}}%
\pgfpathlineto{\pgfqpoint{1.938086in}{1.015623in}}%
\pgfpathlineto{\pgfqpoint{1.934573in}{1.007756in}}%
\pgfpathlineto{\pgfqpoint{1.941628in}{1.017460in}}%
\pgfpathlineto{\pgfqpoint{1.948061in}{1.027261in}}%
\pgfpathlineto{\pgfqpoint{1.953871in}{1.037148in}}%
\pgfpathlineto{\pgfqpoint{1.959053in}{1.047111in}}%
\pgfpathclose%
\pgfusepath{fill}%
\end{pgfscope}%
\begin{pgfscope}%
\pgfpathrectangle{\pgfqpoint{0.329460in}{0.284240in}}{\pgfqpoint{1.989680in}{1.989680in}}%
\pgfusepath{clip}%
\pgfsetbuttcap%
\pgfsetroundjoin%
\definecolor{currentfill}{rgb}{0.565498,0.842430,0.262877}%
\pgfsetfillcolor{currentfill}%
\pgfsetlinewidth{0.000000pt}%
\definecolor{currentstroke}{rgb}{0.000000,0.000000,0.000000}%
\pgfsetstrokecolor{currentstroke}%
\pgfsetdash{}{0pt}%
\pgfpathmoveto{\pgfqpoint{1.212734in}{1.651037in}}%
\pgfpathlineto{\pgfqpoint{1.209640in}{1.646903in}}%
\pgfpathlineto{\pgfqpoint{1.206549in}{1.642683in}}%
\pgfpathlineto{\pgfqpoint{1.203461in}{1.638380in}}%
\pgfpathlineto{\pgfqpoint{1.200376in}{1.633995in}}%
\pgfpathlineto{\pgfqpoint{1.202725in}{1.636232in}}%
\pgfpathlineto{\pgfqpoint{1.205220in}{1.638433in}}%
\pgfpathlineto{\pgfqpoint{1.207857in}{1.640594in}}%
\pgfpathlineto{\pgfqpoint{1.210634in}{1.642714in}}%
\pgfpathlineto{\pgfqpoint{1.213505in}{1.646914in}}%
\pgfpathlineto{\pgfqpoint{1.216379in}{1.651033in}}%
\pgfpathlineto{\pgfqpoint{1.219256in}{1.655068in}}%
\pgfpathlineto{\pgfqpoint{1.222136in}{1.659018in}}%
\pgfpathlineto{\pgfqpoint{1.219590in}{1.657078in}}%
\pgfpathlineto{\pgfqpoint{1.217172in}{1.655099in}}%
\pgfpathlineto{\pgfqpoint{1.214886in}{1.653085in}}%
\pgfpathlineto{\pgfqpoint{1.212734in}{1.651037in}}%
\pgfpathclose%
\pgfusepath{fill}%
\end{pgfscope}%
\begin{pgfscope}%
\pgfpathrectangle{\pgfqpoint{0.329460in}{0.284240in}}{\pgfqpoint{1.989680in}{1.989680in}}%
\pgfusepath{clip}%
\pgfsetbuttcap%
\pgfsetroundjoin%
\definecolor{currentfill}{rgb}{0.636902,0.856542,0.216620}%
\pgfsetfillcolor{currentfill}%
\pgfsetlinewidth{0.000000pt}%
\definecolor{currentstroke}{rgb}{0.000000,0.000000,0.000000}%
\pgfsetstrokecolor{currentstroke}%
\pgfsetdash{}{0pt}%
\pgfpathmoveto{\pgfqpoint{1.455768in}{1.682011in}}%
\pgfpathlineto{\pgfqpoint{1.458337in}{1.678620in}}%
\pgfpathlineto{\pgfqpoint{1.460905in}{1.675137in}}%
\pgfpathlineto{\pgfqpoint{1.463469in}{1.671565in}}%
\pgfpathlineto{\pgfqpoint{1.466030in}{1.667906in}}%
\pgfpathlineto{\pgfqpoint{1.469166in}{1.666175in}}%
\pgfpathlineto{\pgfqpoint{1.472186in}{1.664397in}}%
\pgfpathlineto{\pgfqpoint{1.475089in}{1.662575in}}%
\pgfpathlineto{\pgfqpoint{1.477871in}{1.660711in}}%
\pgfpathlineto{\pgfqpoint{1.475042in}{1.664535in}}%
\pgfpathlineto{\pgfqpoint{1.472210in}{1.668272in}}%
\pgfpathlineto{\pgfqpoint{1.469375in}{1.671919in}}%
\pgfpathlineto{\pgfqpoint{1.466537in}{1.675476in}}%
\pgfpathlineto{\pgfqpoint{1.464008in}{1.677170in}}%
\pgfpathlineto{\pgfqpoint{1.461368in}{1.678824in}}%
\pgfpathlineto{\pgfqpoint{1.458620in}{1.680439in}}%
\pgfpathlineto{\pgfqpoint{1.455768in}{1.682011in}}%
\pgfpathclose%
\pgfusepath{fill}%
\end{pgfscope}%
\begin{pgfscope}%
\pgfpathrectangle{\pgfqpoint{0.329460in}{0.284240in}}{\pgfqpoint{1.989680in}{1.989680in}}%
\pgfusepath{clip}%
\pgfsetbuttcap%
\pgfsetroundjoin%
\definecolor{currentfill}{rgb}{0.134692,0.658636,0.517649}%
\pgfsetfillcolor{currentfill}%
\pgfsetlinewidth{0.000000pt}%
\definecolor{currentstroke}{rgb}{0.000000,0.000000,0.000000}%
\pgfsetstrokecolor{currentstroke}%
\pgfsetdash{}{0pt}%
\pgfpathmoveto{\pgfqpoint{1.105905in}{1.433906in}}%
\pgfpathlineto{\pgfqpoint{1.102572in}{1.426947in}}%
\pgfpathlineto{\pgfqpoint{1.099242in}{1.419953in}}%
\pgfpathlineto{\pgfqpoint{1.095915in}{1.412926in}}%
\pgfpathlineto{\pgfqpoint{1.092590in}{1.405867in}}%
\pgfpathlineto{\pgfqpoint{1.091273in}{1.409893in}}%
\pgfpathlineto{\pgfqpoint{1.090218in}{1.413935in}}%
\pgfpathlineto{\pgfqpoint{1.089424in}{1.417989in}}%
\pgfpathlineto{\pgfqpoint{1.088893in}{1.422050in}}%
\pgfpathlineto{\pgfqpoint{1.092257in}{1.428882in}}%
\pgfpathlineto{\pgfqpoint{1.095623in}{1.435684in}}%
\pgfpathlineto{\pgfqpoint{1.098992in}{1.442453in}}%
\pgfpathlineto{\pgfqpoint{1.102364in}{1.449187in}}%
\pgfpathlineto{\pgfqpoint{1.102876in}{1.445352in}}%
\pgfpathlineto{\pgfqpoint{1.103637in}{1.441524in}}%
\pgfpathlineto{\pgfqpoint{1.104647in}{1.437707in}}%
\pgfpathlineto{\pgfqpoint{1.105905in}{1.433906in}}%
\pgfpathclose%
\pgfusepath{fill}%
\end{pgfscope}%
\begin{pgfscope}%
\pgfpathrectangle{\pgfqpoint{0.329460in}{0.284240in}}{\pgfqpoint{1.989680in}{1.989680in}}%
\pgfusepath{clip}%
\pgfsetbuttcap%
\pgfsetroundjoin%
\definecolor{currentfill}{rgb}{0.212395,0.359683,0.551710}%
\pgfsetfillcolor{currentfill}%
\pgfsetlinewidth{0.000000pt}%
\definecolor{currentstroke}{rgb}{0.000000,0.000000,0.000000}%
\pgfsetstrokecolor{currentstroke}%
\pgfsetdash{}{0pt}%
\pgfpathmoveto{\pgfqpoint{1.047934in}{1.141369in}}%
\pgfpathlineto{\pgfqpoint{1.045081in}{1.133811in}}%
\pgfpathlineto{\pgfqpoint{1.042229in}{1.126297in}}%
\pgfpathlineto{\pgfqpoint{1.039376in}{1.118832in}}%
\pgfpathlineto{\pgfqpoint{1.036524in}{1.111419in}}%
\pgfpathlineto{\pgfqpoint{1.030510in}{1.116589in}}%
\pgfpathlineto{\pgfqpoint{1.024831in}{1.121851in}}%
\pgfpathlineto{\pgfqpoint{1.019490in}{1.127197in}}%
\pgfpathlineto{\pgfqpoint{1.014494in}{1.132623in}}%
\pgfpathlineto{\pgfqpoint{1.017535in}{1.139821in}}%
\pgfpathlineto{\pgfqpoint{1.020578in}{1.147070in}}%
\pgfpathlineto{\pgfqpoint{1.023620in}{1.154368in}}%
\pgfpathlineto{\pgfqpoint{1.026664in}{1.161712in}}%
\pgfpathlineto{\pgfqpoint{1.031489in}{1.156506in}}%
\pgfpathlineto{\pgfqpoint{1.036646in}{1.151377in}}%
\pgfpathlineto{\pgfqpoint{1.042129in}{1.146329in}}%
\pgfpathlineto{\pgfqpoint{1.047934in}{1.141369in}}%
\pgfpathclose%
\pgfusepath{fill}%
\end{pgfscope}%
\begin{pgfscope}%
\pgfpathrectangle{\pgfqpoint{0.329460in}{0.284240in}}{\pgfqpoint{1.989680in}{1.989680in}}%
\pgfusepath{clip}%
\pgfsetbuttcap%
\pgfsetroundjoin%
\definecolor{currentfill}{rgb}{0.267004,0.004874,0.329415}%
\pgfsetfillcolor{currentfill}%
\pgfsetlinewidth{0.000000pt}%
\definecolor{currentstroke}{rgb}{0.000000,0.000000,0.000000}%
\pgfsetstrokecolor{currentstroke}%
\pgfsetdash{}{0pt}%
\pgfpathmoveto{\pgfqpoint{1.812018in}{0.886565in}}%
\pgfpathlineto{\pgfqpoint{1.815045in}{0.885841in}}%
\pgfpathlineto{\pgfqpoint{1.818079in}{0.885350in}}%
\pgfpathlineto{\pgfqpoint{1.821120in}{0.885096in}}%
\pgfpathlineto{\pgfqpoint{1.824169in}{0.885083in}}%
\pgfpathlineto{\pgfqpoint{1.815405in}{0.877179in}}%
\pgfpathlineto{\pgfqpoint{1.806146in}{0.869417in}}%
\pgfpathlineto{\pgfqpoint{1.796400in}{0.861807in}}%
\pgfpathlineto{\pgfqpoint{1.786176in}{0.854358in}}%
\pgfpathlineto{\pgfqpoint{1.783359in}{0.854569in}}%
\pgfpathlineto{\pgfqpoint{1.780550in}{0.855023in}}%
\pgfpathlineto{\pgfqpoint{1.777747in}{0.855714in}}%
\pgfpathlineto{\pgfqpoint{1.774951in}{0.856639in}}%
\pgfpathlineto{\pgfqpoint{1.784924in}{0.863894in}}%
\pgfpathlineto{\pgfqpoint{1.794431in}{0.871305in}}%
\pgfpathlineto{\pgfqpoint{1.803466in}{0.878865in}}%
\pgfpathlineto{\pgfqpoint{1.812018in}{0.886565in}}%
\pgfpathclose%
\pgfusepath{fill}%
\end{pgfscope}%
\begin{pgfscope}%
\pgfpathrectangle{\pgfqpoint{0.329460in}{0.284240in}}{\pgfqpoint{1.989680in}{1.989680in}}%
\pgfusepath{clip}%
\pgfsetbuttcap%
\pgfsetroundjoin%
\definecolor{currentfill}{rgb}{0.280255,0.165693,0.476498}%
\pgfsetfillcolor{currentfill}%
\pgfsetlinewidth{0.000000pt}%
\definecolor{currentstroke}{rgb}{0.000000,0.000000,0.000000}%
\pgfsetstrokecolor{currentstroke}%
\pgfsetdash{}{0pt}%
\pgfpathmoveto{\pgfqpoint{1.021961in}{0.979365in}}%
\pgfpathlineto{\pgfqpoint{1.019338in}{0.973093in}}%
\pgfpathlineto{\pgfqpoint{1.016714in}{0.966929in}}%
\pgfpathlineto{\pgfqpoint{1.014089in}{0.960877in}}%
\pgfpathlineto{\pgfqpoint{1.011461in}{0.954940in}}%
\pgfpathlineto{\pgfqpoint{1.002888in}{0.960676in}}%
\pgfpathlineto{\pgfqpoint{0.994685in}{0.966547in}}%
\pgfpathlineto{\pgfqpoint{0.986859in}{0.972545in}}%
\pgfpathlineto{\pgfqpoint{0.979418in}{0.978665in}}%
\pgfpathlineto{\pgfqpoint{0.982282in}{0.984398in}}%
\pgfpathlineto{\pgfqpoint{0.985145in}{0.990246in}}%
\pgfpathlineto{\pgfqpoint{0.988007in}{0.996206in}}%
\pgfpathlineto{\pgfqpoint{0.990867in}{1.002274in}}%
\pgfpathlineto{\pgfqpoint{0.998089in}{0.996364in}}%
\pgfpathlineto{\pgfqpoint{1.005684in}{0.990571in}}%
\pgfpathlineto{\pgfqpoint{1.013643in}{0.984903in}}%
\pgfpathlineto{\pgfqpoint{1.021961in}{0.979365in}}%
\pgfpathclose%
\pgfusepath{fill}%
\end{pgfscope}%
\begin{pgfscope}%
\pgfpathrectangle{\pgfqpoint{0.329460in}{0.284240in}}{\pgfqpoint{1.989680in}{1.989680in}}%
\pgfusepath{clip}%
\pgfsetbuttcap%
\pgfsetroundjoin%
\definecolor{currentfill}{rgb}{0.220124,0.725509,0.466226}%
\pgfsetfillcolor{currentfill}%
\pgfsetlinewidth{0.000000pt}%
\definecolor{currentstroke}{rgb}{0.000000,0.000000,0.000000}%
\pgfsetstrokecolor{currentstroke}%
\pgfsetdash{}{0pt}%
\pgfpathmoveto{\pgfqpoint{1.129439in}{1.501544in}}%
\pgfpathlineto{\pgfqpoint{1.126045in}{1.495166in}}%
\pgfpathlineto{\pgfqpoint{1.122654in}{1.488737in}}%
\pgfpathlineto{\pgfqpoint{1.119265in}{1.482257in}}%
\pgfpathlineto{\pgfqpoint{1.115879in}{1.475730in}}%
\pgfpathlineto{\pgfqpoint{1.115622in}{1.479345in}}%
\pgfpathlineto{\pgfqpoint{1.115600in}{1.482959in}}%
\pgfpathlineto{\pgfqpoint{1.115813in}{1.486570in}}%
\pgfpathlineto{\pgfqpoint{1.116260in}{1.490173in}}%
\pgfpathlineto{\pgfqpoint{1.119632in}{1.496478in}}%
\pgfpathlineto{\pgfqpoint{1.123007in}{1.502735in}}%
\pgfpathlineto{\pgfqpoint{1.126386in}{1.508942in}}%
\pgfpathlineto{\pgfqpoint{1.129767in}{1.515099in}}%
\pgfpathlineto{\pgfqpoint{1.129354in}{1.511717in}}%
\pgfpathlineto{\pgfqpoint{1.129161in}{1.508328in}}%
\pgfpathlineto{\pgfqpoint{1.129189in}{1.504936in}}%
\pgfpathlineto{\pgfqpoint{1.129439in}{1.501544in}}%
\pgfpathclose%
\pgfusepath{fill}%
\end{pgfscope}%
\begin{pgfscope}%
\pgfpathrectangle{\pgfqpoint{0.329460in}{0.284240in}}{\pgfqpoint{1.989680in}{1.989680in}}%
\pgfusepath{clip}%
\pgfsetbuttcap%
\pgfsetroundjoin%
\definecolor{currentfill}{rgb}{0.147607,0.511733,0.557049}%
\pgfsetfillcolor{currentfill}%
\pgfsetlinewidth{0.000000pt}%
\definecolor{currentstroke}{rgb}{0.000000,0.000000,0.000000}%
\pgfsetstrokecolor{currentstroke}%
\pgfsetdash{}{0pt}%
\pgfpathmoveto{\pgfqpoint{1.642941in}{1.305111in}}%
\pgfpathlineto{\pgfqpoint{1.646171in}{1.297687in}}%
\pgfpathlineto{\pgfqpoint{1.649398in}{1.290264in}}%
\pgfpathlineto{\pgfqpoint{1.652624in}{1.282847in}}%
\pgfpathlineto{\pgfqpoint{1.655848in}{1.275437in}}%
\pgfpathlineto{\pgfqpoint{1.653019in}{1.270624in}}%
\pgfpathlineto{\pgfqpoint{1.649882in}{1.265855in}}%
\pgfpathlineto{\pgfqpoint{1.646439in}{1.261134in}}%
\pgfpathlineto{\pgfqpoint{1.642694in}{1.256467in}}%
\pgfpathlineto{\pgfqpoint{1.639600in}{1.264100in}}%
\pgfpathlineto{\pgfqpoint{1.636504in}{1.271741in}}%
\pgfpathlineto{\pgfqpoint{1.633407in}{1.279386in}}%
\pgfpathlineto{\pgfqpoint{1.630308in}{1.287033in}}%
\pgfpathlineto{\pgfqpoint{1.633903in}{1.291480in}}%
\pgfpathlineto{\pgfqpoint{1.637209in}{1.295979in}}%
\pgfpathlineto{\pgfqpoint{1.640222in}{1.300524in}}%
\pgfpathlineto{\pgfqpoint{1.642941in}{1.305111in}}%
\pgfpathclose%
\pgfusepath{fill}%
\end{pgfscope}%
\begin{pgfscope}%
\pgfpathrectangle{\pgfqpoint{0.329460in}{0.284240in}}{\pgfqpoint{1.989680in}{1.989680in}}%
\pgfusepath{clip}%
\pgfsetbuttcap%
\pgfsetroundjoin%
\definecolor{currentfill}{rgb}{0.762373,0.876424,0.137064}%
\pgfsetfillcolor{currentfill}%
\pgfsetlinewidth{0.000000pt}%
\definecolor{currentstroke}{rgb}{0.000000,0.000000,0.000000}%
\pgfsetstrokecolor{currentstroke}%
\pgfsetdash{}{0pt}%
\pgfpathmoveto{\pgfqpoint{1.389159in}{1.720422in}}%
\pgfpathlineto{\pgfqpoint{1.390327in}{1.718260in}}%
\pgfpathlineto{\pgfqpoint{1.391495in}{1.715998in}}%
\pgfpathlineto{\pgfqpoint{1.392661in}{1.713636in}}%
\pgfpathlineto{\pgfqpoint{1.393826in}{1.711177in}}%
\pgfpathlineto{\pgfqpoint{1.397508in}{1.710520in}}%
\pgfpathlineto{\pgfqpoint{1.401145in}{1.709810in}}%
\pgfpathlineto{\pgfqpoint{1.404734in}{1.709046in}}%
\pgfpathlineto{\pgfqpoint{1.408271in}{1.708229in}}%
\pgfpathlineto{\pgfqpoint{1.406711in}{1.710770in}}%
\pgfpathlineto{\pgfqpoint{1.405149in}{1.713214in}}%
\pgfpathlineto{\pgfqpoint{1.403585in}{1.715558in}}%
\pgfpathlineto{\pgfqpoint{1.402020in}{1.717801in}}%
\pgfpathlineto{\pgfqpoint{1.398871in}{1.718527in}}%
\pgfpathlineto{\pgfqpoint{1.395676in}{1.719206in}}%
\pgfpathlineto{\pgfqpoint{1.392437in}{1.719838in}}%
\pgfpathlineto{\pgfqpoint{1.389159in}{1.720422in}}%
\pgfpathclose%
\pgfusepath{fill}%
\end{pgfscope}%
\begin{pgfscope}%
\pgfpathrectangle{\pgfqpoint{0.329460in}{0.284240in}}{\pgfqpoint{1.989680in}{1.989680in}}%
\pgfusepath{clip}%
\pgfsetbuttcap%
\pgfsetroundjoin%
\definecolor{currentfill}{rgb}{0.412913,0.803041,0.357269}%
\pgfsetfillcolor{currentfill}%
\pgfsetlinewidth{0.000000pt}%
\definecolor{currentstroke}{rgb}{0.000000,0.000000,0.000000}%
\pgfsetstrokecolor{currentstroke}%
\pgfsetdash{}{0pt}%
\pgfpathmoveto{\pgfqpoint{1.173638in}{1.595079in}}%
\pgfpathlineto{\pgfqpoint{1.170286in}{1.589901in}}%
\pgfpathlineto{\pgfqpoint{1.166936in}{1.584649in}}%
\pgfpathlineto{\pgfqpoint{1.163589in}{1.579326in}}%
\pgfpathlineto{\pgfqpoint{1.160246in}{1.573933in}}%
\pgfpathlineto{\pgfqpoint{1.161550in}{1.576807in}}%
\pgfpathlineto{\pgfqpoint{1.163042in}{1.579658in}}%
\pgfpathlineto{\pgfqpoint{1.164718in}{1.582484in}}%
\pgfpathlineto{\pgfqpoint{1.166579in}{1.585283in}}%
\pgfpathlineto{\pgfqpoint{1.169805in}{1.590467in}}%
\pgfpathlineto{\pgfqpoint{1.173035in}{1.595582in}}%
\pgfpathlineto{\pgfqpoint{1.176268in}{1.600626in}}%
\pgfpathlineto{\pgfqpoint{1.179504in}{1.605597in}}%
\pgfpathlineto{\pgfqpoint{1.177780in}{1.603004in}}%
\pgfpathlineto{\pgfqpoint{1.176226in}{1.600385in}}%
\pgfpathlineto{\pgfqpoint{1.174845in}{1.597742in}}%
\pgfpathlineto{\pgfqpoint{1.173638in}{1.595079in}}%
\pgfpathclose%
\pgfusepath{fill}%
\end{pgfscope}%
\begin{pgfscope}%
\pgfpathrectangle{\pgfqpoint{0.329460in}{0.284240in}}{\pgfqpoint{1.989680in}{1.989680in}}%
\pgfusepath{clip}%
\pgfsetbuttcap%
\pgfsetroundjoin%
\definecolor{currentfill}{rgb}{0.195860,0.395433,0.555276}%
\pgfsetfillcolor{currentfill}%
\pgfsetlinewidth{0.000000pt}%
\definecolor{currentstroke}{rgb}{0.000000,0.000000,0.000000}%
\pgfsetstrokecolor{currentstroke}%
\pgfsetdash{}{0pt}%
\pgfpathmoveto{\pgfqpoint{1.667396in}{1.195976in}}%
\pgfpathlineto{\pgfqpoint{1.670478in}{1.188527in}}%
\pgfpathlineto{\pgfqpoint{1.673560in}{1.181113in}}%
\pgfpathlineto{\pgfqpoint{1.676640in}{1.173736in}}%
\pgfpathlineto{\pgfqpoint{1.679720in}{1.166399in}}%
\pgfpathlineto{\pgfqpoint{1.675192in}{1.161130in}}%
\pgfpathlineto{\pgfqpoint{1.670329in}{1.155932in}}%
\pgfpathlineto{\pgfqpoint{1.665136in}{1.150812in}}%
\pgfpathlineto{\pgfqpoint{1.659616in}{1.145774in}}%
\pgfpathlineto{\pgfqpoint{1.656716in}{1.153328in}}%
\pgfpathlineto{\pgfqpoint{1.653815in}{1.160923in}}%
\pgfpathlineto{\pgfqpoint{1.650913in}{1.168554in}}%
\pgfpathlineto{\pgfqpoint{1.648011in}{1.176219in}}%
\pgfpathlineto{\pgfqpoint{1.653331in}{1.181045in}}%
\pgfpathlineto{\pgfqpoint{1.658339in}{1.185949in}}%
\pgfpathlineto{\pgfqpoint{1.663028in}{1.190928in}}%
\pgfpathlineto{\pgfqpoint{1.667396in}{1.195976in}}%
\pgfpathclose%
\pgfusepath{fill}%
\end{pgfscope}%
\begin{pgfscope}%
\pgfpathrectangle{\pgfqpoint{0.329460in}{0.284240in}}{\pgfqpoint{1.989680in}{1.989680in}}%
\pgfusepath{clip}%
\pgfsetbuttcap%
\pgfsetroundjoin%
\definecolor{currentfill}{rgb}{0.122606,0.585371,0.546557}%
\pgfsetfillcolor{currentfill}%
\pgfsetlinewidth{0.000000pt}%
\definecolor{currentstroke}{rgb}{0.000000,0.000000,0.000000}%
\pgfsetstrokecolor{currentstroke}%
\pgfsetdash{}{0pt}%
\pgfpathmoveto{\pgfqpoint{1.087548in}{1.360611in}}%
\pgfpathlineto{\pgfqpoint{1.084327in}{1.353208in}}%
\pgfpathlineto{\pgfqpoint{1.081108in}{1.345787in}}%
\pgfpathlineto{\pgfqpoint{1.077891in}{1.338352in}}%
\pgfpathlineto{\pgfqpoint{1.074676in}{1.330903in}}%
\pgfpathlineto{\pgfqpoint{1.072100in}{1.335270in}}%
\pgfpathlineto{\pgfqpoint{1.069807in}{1.339672in}}%
\pgfpathlineto{\pgfqpoint{1.067800in}{1.344104in}}%
\pgfpathlineto{\pgfqpoint{1.066080in}{1.348563in}}%
\pgfpathlineto{\pgfqpoint{1.069385in}{1.355786in}}%
\pgfpathlineto{\pgfqpoint{1.072693in}{1.362996in}}%
\pgfpathlineto{\pgfqpoint{1.076003in}{1.370192in}}%
\pgfpathlineto{\pgfqpoint{1.079316in}{1.377370in}}%
\pgfpathlineto{\pgfqpoint{1.080965in}{1.373139in}}%
\pgfpathlineto{\pgfqpoint{1.082888in}{1.368932in}}%
\pgfpathlineto{\pgfqpoint{1.085083in}{1.364755in}}%
\pgfpathlineto{\pgfqpoint{1.087548in}{1.360611in}}%
\pgfpathclose%
\pgfusepath{fill}%
\end{pgfscope}%
\begin{pgfscope}%
\pgfpathrectangle{\pgfqpoint{0.329460in}{0.284240in}}{\pgfqpoint{1.989680in}{1.989680in}}%
\pgfusepath{clip}%
\pgfsetbuttcap%
\pgfsetroundjoin%
\definecolor{currentfill}{rgb}{0.636902,0.856542,0.216620}%
\pgfsetfillcolor{currentfill}%
\pgfsetlinewidth{0.000000pt}%
\definecolor{currentstroke}{rgb}{0.000000,0.000000,0.000000}%
\pgfsetstrokecolor{currentstroke}%
\pgfsetdash{}{0pt}%
\pgfpathmoveto{\pgfqpoint{1.233684in}{1.673939in}}%
\pgfpathlineto{\pgfqpoint{1.230793in}{1.670344in}}%
\pgfpathlineto{\pgfqpoint{1.227904in}{1.666657in}}%
\pgfpathlineto{\pgfqpoint{1.225018in}{1.662882in}}%
\pgfpathlineto{\pgfqpoint{1.222136in}{1.659018in}}%
\pgfpathlineto{\pgfqpoint{1.224808in}{1.660920in}}%
\pgfpathlineto{\pgfqpoint{1.227603in}{1.662780in}}%
\pgfpathlineto{\pgfqpoint{1.230519in}{1.664597in}}%
\pgfpathlineto{\pgfqpoint{1.233552in}{1.666369in}}%
\pgfpathlineto{\pgfqpoint{1.236177in}{1.670064in}}%
\pgfpathlineto{\pgfqpoint{1.238804in}{1.673671in}}%
\pgfpathlineto{\pgfqpoint{1.241434in}{1.677189in}}%
\pgfpathlineto{\pgfqpoint{1.244067in}{1.680616in}}%
\pgfpathlineto{\pgfqpoint{1.241307in}{1.679006in}}%
\pgfpathlineto{\pgfqpoint{1.238655in}{1.677355in}}%
\pgfpathlineto{\pgfqpoint{1.236113in}{1.675666in}}%
\pgfpathlineto{\pgfqpoint{1.233684in}{1.673939in}}%
\pgfpathclose%
\pgfusepath{fill}%
\end{pgfscope}%
\begin{pgfscope}%
\pgfpathrectangle{\pgfqpoint{0.329460in}{0.284240in}}{\pgfqpoint{1.989680in}{1.989680in}}%
\pgfusepath{clip}%
\pgfsetbuttcap%
\pgfsetroundjoin%
\definecolor{currentfill}{rgb}{0.344074,0.780029,0.397381}%
\pgfsetfillcolor{currentfill}%
\pgfsetlinewidth{0.000000pt}%
\definecolor{currentstroke}{rgb}{0.000000,0.000000,0.000000}%
\pgfsetstrokecolor{currentstroke}%
\pgfsetdash{}{0pt}%
\pgfpathmoveto{\pgfqpoint{1.540979in}{1.576488in}}%
\pgfpathlineto{\pgfqpoint{1.544298in}{1.571075in}}%
\pgfpathlineto{\pgfqpoint{1.547614in}{1.565595in}}%
\pgfpathlineto{\pgfqpoint{1.550927in}{1.560051in}}%
\pgfpathlineto{\pgfqpoint{1.554236in}{1.554444in}}%
\pgfpathlineto{\pgfqpoint{1.555616in}{1.551355in}}%
\pgfpathlineto{\pgfqpoint{1.556794in}{1.548245in}}%
\pgfpathlineto{\pgfqpoint{1.557769in}{1.545118in}}%
\pgfpathlineto{\pgfqpoint{1.558540in}{1.541975in}}%
\pgfpathlineto{\pgfqpoint{1.555153in}{1.547797in}}%
\pgfpathlineto{\pgfqpoint{1.551763in}{1.553556in}}%
\pgfpathlineto{\pgfqpoint{1.548371in}{1.559250in}}%
\pgfpathlineto{\pgfqpoint{1.544975in}{1.564878in}}%
\pgfpathlineto{\pgfqpoint{1.544262in}{1.567804in}}%
\pgfpathlineto{\pgfqpoint{1.543357in}{1.570716in}}%
\pgfpathlineto{\pgfqpoint{1.542263in}{1.573612in}}%
\pgfpathlineto{\pgfqpoint{1.540979in}{1.576488in}}%
\pgfpathclose%
\pgfusepath{fill}%
\end{pgfscope}%
\begin{pgfscope}%
\pgfpathrectangle{\pgfqpoint{0.329460in}{0.284240in}}{\pgfqpoint{1.989680in}{1.989680in}}%
\pgfusepath{clip}%
\pgfsetbuttcap%
\pgfsetroundjoin%
\definecolor{currentfill}{rgb}{0.762373,0.876424,0.137064}%
\pgfsetfillcolor{currentfill}%
\pgfsetlinewidth{0.000000pt}%
\definecolor{currentstroke}{rgb}{0.000000,0.000000,0.000000}%
\pgfsetstrokecolor{currentstroke}%
\pgfsetdash{}{0pt}%
\pgfpathmoveto{\pgfqpoint{1.297598in}{1.717117in}}%
\pgfpathlineto{\pgfqpoint{1.295947in}{1.714852in}}%
\pgfpathlineto{\pgfqpoint{1.294299in}{1.712487in}}%
\pgfpathlineto{\pgfqpoint{1.292652in}{1.710022in}}%
\pgfpathlineto{\pgfqpoint{1.291007in}{1.707460in}}%
\pgfpathlineto{\pgfqpoint{1.294494in}{1.708323in}}%
\pgfpathlineto{\pgfqpoint{1.298037in}{1.709133in}}%
\pgfpathlineto{\pgfqpoint{1.301632in}{1.709891in}}%
\pgfpathlineto{\pgfqpoint{1.305274in}{1.710596in}}%
\pgfpathlineto{\pgfqpoint{1.306529in}{1.713072in}}%
\pgfpathlineto{\pgfqpoint{1.307785in}{1.715449in}}%
\pgfpathlineto{\pgfqpoint{1.309042in}{1.717727in}}%
\pgfpathlineto{\pgfqpoint{1.310300in}{1.719905in}}%
\pgfpathlineto{\pgfqpoint{1.307057in}{1.719279in}}%
\pgfpathlineto{\pgfqpoint{1.303857in}{1.718605in}}%
\pgfpathlineto{\pgfqpoint{1.300703in}{1.717884in}}%
\pgfpathlineto{\pgfqpoint{1.297598in}{1.717117in}}%
\pgfpathclose%
\pgfusepath{fill}%
\end{pgfscope}%
\begin{pgfscope}%
\pgfpathrectangle{\pgfqpoint{0.329460in}{0.284240in}}{\pgfqpoint{1.989680in}{1.989680in}}%
\pgfusepath{clip}%
\pgfsetbuttcap%
\pgfsetroundjoin%
\definecolor{currentfill}{rgb}{0.201239,0.383670,0.554294}%
\pgfsetfillcolor{currentfill}%
\pgfsetlinewidth{0.000000pt}%
\definecolor{currentstroke}{rgb}{0.000000,0.000000,0.000000}%
\pgfsetstrokecolor{currentstroke}%
\pgfsetdash{}{0pt}%
\pgfpathmoveto{\pgfqpoint{0.718490in}{1.113341in}}%
\pgfpathlineto{\pgfqpoint{0.714731in}{1.124574in}}%
\pgfpathlineto{\pgfqpoint{0.710952in}{1.136241in}}%
\pgfpathlineto{\pgfqpoint{0.707153in}{1.148350in}}%
\pgfpathlineto{\pgfqpoint{0.703333in}{1.160907in}}%
\pgfpathlineto{\pgfqpoint{0.697917in}{1.171409in}}%
\pgfpathlineto{\pgfqpoint{0.693180in}{1.181976in}}%
\pgfpathlineto{\pgfqpoint{0.689122in}{1.192596in}}%
\pgfpathlineto{\pgfqpoint{0.685745in}{1.203258in}}%
\pgfpathlineto{\pgfqpoint{0.689643in}{1.190536in}}%
\pgfpathlineto{\pgfqpoint{0.693520in}{1.178260in}}%
\pgfpathlineto{\pgfqpoint{0.697376in}{1.166423in}}%
\pgfpathlineto{\pgfqpoint{0.701212in}{1.155017in}}%
\pgfpathlineto{\pgfqpoint{0.704536in}{1.144524in}}%
\pgfpathlineto{\pgfqpoint{0.708524in}{1.134073in}}%
\pgfpathlineto{\pgfqpoint{0.713176in}{1.123675in}}%
\pgfpathlineto{\pgfqpoint{0.718490in}{1.113341in}}%
\pgfpathclose%
\pgfusepath{fill}%
\end{pgfscope}%
\begin{pgfscope}%
\pgfpathrectangle{\pgfqpoint{0.329460in}{0.284240in}}{\pgfqpoint{1.989680in}{1.989680in}}%
\pgfusepath{clip}%
\pgfsetbuttcap%
\pgfsetroundjoin%
\definecolor{currentfill}{rgb}{0.267004,0.004874,0.329415}%
\pgfsetfillcolor{currentfill}%
\pgfsetlinewidth{0.000000pt}%
\definecolor{currentstroke}{rgb}{0.000000,0.000000,0.000000}%
\pgfsetstrokecolor{currentstroke}%
\pgfsetdash{}{0pt}%
\pgfpathmoveto{\pgfqpoint{0.947564in}{0.856437in}}%
\pgfpathlineto{\pgfqpoint{0.944849in}{0.854582in}}%
\pgfpathlineto{\pgfqpoint{0.942130in}{0.852943in}}%
\pgfpathlineto{\pgfqpoint{0.939404in}{0.851523in}}%
\pgfpathlineto{\pgfqpoint{0.936673in}{0.850328in}}%
\pgfpathlineto{\pgfqpoint{0.926294in}{0.857437in}}%
\pgfpathlineto{\pgfqpoint{0.916372in}{0.864710in}}%
\pgfpathlineto{\pgfqpoint{0.906916in}{0.872138in}}%
\pgfpathlineto{\pgfqpoint{0.897935in}{0.879714in}}%
\pgfpathlineto{\pgfqpoint{0.900909in}{0.880712in}}%
\pgfpathlineto{\pgfqpoint{0.903877in}{0.881933in}}%
\pgfpathlineto{\pgfqpoint{0.906838in}{0.883373in}}%
\pgfpathlineto{\pgfqpoint{0.909793in}{0.885029in}}%
\pgfpathlineto{\pgfqpoint{0.918552in}{0.877657in}}%
\pgfpathlineto{\pgfqpoint{0.927773in}{0.870429in}}%
\pgfpathlineto{\pgfqpoint{0.937446in}{0.863353in}}%
\pgfpathlineto{\pgfqpoint{0.947564in}{0.856437in}}%
\pgfpathclose%
\pgfusepath{fill}%
\end{pgfscope}%
\begin{pgfscope}%
\pgfpathrectangle{\pgfqpoint{0.329460in}{0.284240in}}{\pgfqpoint{1.989680in}{1.989680in}}%
\pgfusepath{clip}%
\pgfsetbuttcap%
\pgfsetroundjoin%
\definecolor{currentfill}{rgb}{0.263663,0.237631,0.518762}%
\pgfsetfillcolor{currentfill}%
\pgfsetlinewidth{0.000000pt}%
\definecolor{currentstroke}{rgb}{0.000000,0.000000,0.000000}%
\pgfsetstrokecolor{currentstroke}%
\pgfsetdash{}{0pt}%
\pgfpathmoveto{\pgfqpoint{1.694398in}{1.059286in}}%
\pgfpathlineto{\pgfqpoint{1.697297in}{1.052518in}}%
\pgfpathlineto{\pgfqpoint{1.700196in}{1.045832in}}%
\pgfpathlineto{\pgfqpoint{1.703096in}{1.039231in}}%
\pgfpathlineto{\pgfqpoint{1.705997in}{1.032719in}}%
\pgfpathlineto{\pgfqpoint{1.699319in}{1.026923in}}%
\pgfpathlineto{\pgfqpoint{1.692276in}{1.021235in}}%
\pgfpathlineto{\pgfqpoint{1.684874in}{1.015662in}}%
\pgfpathlineto{\pgfqpoint{1.677119in}{1.010209in}}%
\pgfpathlineto{\pgfqpoint{1.674445in}{1.016928in}}%
\pgfpathlineto{\pgfqpoint{1.671772in}{1.023736in}}%
\pgfpathlineto{\pgfqpoint{1.669100in}{1.030629in}}%
\pgfpathlineto{\pgfqpoint{1.666428in}{1.037604in}}%
\pgfpathlineto{\pgfqpoint{1.673937in}{1.042856in}}%
\pgfpathlineto{\pgfqpoint{1.681106in}{1.048224in}}%
\pgfpathlineto{\pgfqpoint{1.687928in}{1.053703in}}%
\pgfpathlineto{\pgfqpoint{1.694398in}{1.059286in}}%
\pgfpathclose%
\pgfusepath{fill}%
\end{pgfscope}%
\begin{pgfscope}%
\pgfpathrectangle{\pgfqpoint{0.329460in}{0.284240in}}{\pgfqpoint{1.989680in}{1.989680in}}%
\pgfusepath{clip}%
\pgfsetbuttcap%
\pgfsetroundjoin%
\definecolor{currentfill}{rgb}{0.699415,0.867117,0.175971}%
\pgfsetfillcolor{currentfill}%
\pgfsetlinewidth{0.000000pt}%
\definecolor{currentstroke}{rgb}{0.000000,0.000000,0.000000}%
\pgfsetstrokecolor{currentstroke}%
\pgfsetdash{}{0pt}%
\pgfpathmoveto{\pgfqpoint{1.434294in}{1.699901in}}%
\pgfpathlineto{\pgfqpoint{1.436566in}{1.697031in}}%
\pgfpathlineto{\pgfqpoint{1.438836in}{1.694066in}}%
\pgfpathlineto{\pgfqpoint{1.441104in}{1.691006in}}%
\pgfpathlineto{\pgfqpoint{1.443369in}{1.687852in}}%
\pgfpathlineto{\pgfqpoint{1.446611in}{1.686462in}}%
\pgfpathlineto{\pgfqpoint{1.449760in}{1.685025in}}%
\pgfpathlineto{\pgfqpoint{1.452814in}{1.683540in}}%
\pgfpathlineto{\pgfqpoint{1.455768in}{1.682011in}}%
\pgfpathlineto{\pgfqpoint{1.453195in}{1.685311in}}%
\pgfpathlineto{\pgfqpoint{1.450620in}{1.688518in}}%
\pgfpathlineto{\pgfqpoint{1.448043in}{1.691630in}}%
\pgfpathlineto{\pgfqpoint{1.445463in}{1.694646in}}%
\pgfpathlineto{\pgfqpoint{1.442802in}{1.696022in}}%
\pgfpathlineto{\pgfqpoint{1.440052in}{1.697357in}}%
\pgfpathlineto{\pgfqpoint{1.437215in}{1.698650in}}%
\pgfpathlineto{\pgfqpoint{1.434294in}{1.699901in}}%
\pgfpathclose%
\pgfusepath{fill}%
\end{pgfscope}%
\begin{pgfscope}%
\pgfpathrectangle{\pgfqpoint{0.329460in}{0.284240in}}{\pgfqpoint{1.989680in}{1.989680in}}%
\pgfusepath{clip}%
\pgfsetbuttcap%
\pgfsetroundjoin%
\definecolor{currentfill}{rgb}{0.274128,0.199721,0.498911}%
\pgfsetfillcolor{currentfill}%
\pgfsetlinewidth{0.000000pt}%
\definecolor{currentstroke}{rgb}{0.000000,0.000000,0.000000}%
\pgfsetstrokecolor{currentstroke}%
\pgfsetdash{}{0pt}%
\pgfpathmoveto{\pgfqpoint{1.032439in}{1.005468in}}%
\pgfpathlineto{\pgfqpoint{1.029821in}{0.998797in}}%
\pgfpathlineto{\pgfqpoint{1.027202in}{0.992220in}}%
\pgfpathlineto{\pgfqpoint{1.024582in}{0.985742in}}%
\pgfpathlineto{\pgfqpoint{1.021961in}{0.979365in}}%
\pgfpathlineto{\pgfqpoint{1.013643in}{0.984903in}}%
\pgfpathlineto{\pgfqpoint{1.005684in}{0.990571in}}%
\pgfpathlineto{\pgfqpoint{0.998089in}{0.996364in}}%
\pgfpathlineto{\pgfqpoint{0.990867in}{1.002274in}}%
\pgfpathlineto{\pgfqpoint{0.993726in}{1.008447in}}%
\pgfpathlineto{\pgfqpoint{0.996583in}{1.014721in}}%
\pgfpathlineto{\pgfqpoint{0.999440in}{1.021094in}}%
\pgfpathlineto{\pgfqpoint{1.002296in}{1.027562in}}%
\pgfpathlineto{\pgfqpoint{1.009299in}{1.021862in}}%
\pgfpathlineto{\pgfqpoint{1.016661in}{1.016275in}}%
\pgfpathlineto{\pgfqpoint{1.024377in}{1.010809in}}%
\pgfpathlineto{\pgfqpoint{1.032439in}{1.005468in}}%
\pgfpathclose%
\pgfusepath{fill}%
\end{pgfscope}%
\begin{pgfscope}%
\pgfpathrectangle{\pgfqpoint{0.329460in}{0.284240in}}{\pgfqpoint{1.989680in}{1.989680in}}%
\pgfusepath{clip}%
\pgfsetbuttcap%
\pgfsetroundjoin%
\definecolor{currentfill}{rgb}{0.344074,0.780029,0.397381}%
\pgfsetfillcolor{currentfill}%
\pgfsetlinewidth{0.000000pt}%
\definecolor{currentstroke}{rgb}{0.000000,0.000000,0.000000}%
\pgfsetstrokecolor{currentstroke}%
\pgfsetdash{}{0pt}%
\pgfpathmoveto{\pgfqpoint{1.156927in}{1.562268in}}%
\pgfpathlineto{\pgfqpoint{1.153522in}{1.556592in}}%
\pgfpathlineto{\pgfqpoint{1.150119in}{1.550849in}}%
\pgfpathlineto{\pgfqpoint{1.146720in}{1.545042in}}%
\pgfpathlineto{\pgfqpoint{1.143323in}{1.539172in}}%
\pgfpathlineto{\pgfqpoint{1.143911in}{1.542325in}}%
\pgfpathlineto{\pgfqpoint{1.144704in}{1.545466in}}%
\pgfpathlineto{\pgfqpoint{1.145702in}{1.548592in}}%
\pgfpathlineto{\pgfqpoint{1.146902in}{1.551699in}}%
\pgfpathlineto{\pgfqpoint{1.150234in}{1.557353in}}%
\pgfpathlineto{\pgfqpoint{1.153568in}{1.562945in}}%
\pgfpathlineto{\pgfqpoint{1.156905in}{1.568472in}}%
\pgfpathlineto{\pgfqpoint{1.160246in}{1.573933in}}%
\pgfpathlineto{\pgfqpoint{1.159130in}{1.571039in}}%
\pgfpathlineto{\pgfqpoint{1.158204in}{1.568129in}}%
\pgfpathlineto{\pgfqpoint{1.157470in}{1.565204in}}%
\pgfpathlineto{\pgfqpoint{1.156927in}{1.562268in}}%
\pgfpathclose%
\pgfusepath{fill}%
\end{pgfscope}%
\begin{pgfscope}%
\pgfpathrectangle{\pgfqpoint{0.329460in}{0.284240in}}{\pgfqpoint{1.989680in}{1.989680in}}%
\pgfusepath{clip}%
\pgfsetbuttcap%
\pgfsetroundjoin%
\definecolor{currentfill}{rgb}{0.166383,0.690856,0.496502}%
\pgfsetfillcolor{currentfill}%
\pgfsetlinewidth{0.000000pt}%
\definecolor{currentstroke}{rgb}{0.000000,0.000000,0.000000}%
\pgfsetstrokecolor{currentstroke}%
\pgfsetdash{}{0pt}%
\pgfpathmoveto{\pgfqpoint{1.586736in}{1.478943in}}%
\pgfpathlineto{\pgfqpoint{1.590121in}{1.472420in}}%
\pgfpathlineto{\pgfqpoint{1.593503in}{1.465854in}}%
\pgfpathlineto{\pgfqpoint{1.596881in}{1.459246in}}%
\pgfpathlineto{\pgfqpoint{1.600258in}{1.452599in}}%
\pgfpathlineto{\pgfqpoint{1.599967in}{1.448760in}}%
\pgfpathlineto{\pgfqpoint{1.599427in}{1.444926in}}%
\pgfpathlineto{\pgfqpoint{1.598638in}{1.441099in}}%
\pgfpathlineto{\pgfqpoint{1.597601in}{1.437284in}}%
\pgfpathlineto{\pgfqpoint{1.594252in}{1.444156in}}%
\pgfpathlineto{\pgfqpoint{1.590901in}{1.450988in}}%
\pgfpathlineto{\pgfqpoint{1.587548in}{1.457778in}}%
\pgfpathlineto{\pgfqpoint{1.584192in}{1.464525in}}%
\pgfpathlineto{\pgfqpoint{1.585180in}{1.468116in}}%
\pgfpathlineto{\pgfqpoint{1.585934in}{1.471719in}}%
\pgfpathlineto{\pgfqpoint{1.586453in}{1.475329in}}%
\pgfpathlineto{\pgfqpoint{1.586736in}{1.478943in}}%
\pgfpathclose%
\pgfusepath{fill}%
\end{pgfscope}%
\begin{pgfscope}%
\pgfpathrectangle{\pgfqpoint{0.329460in}{0.284240in}}{\pgfqpoint{1.989680in}{1.989680in}}%
\pgfusepath{clip}%
\pgfsetbuttcap%
\pgfsetroundjoin%
\definecolor{currentfill}{rgb}{0.762373,0.876424,0.137064}%
\pgfsetfillcolor{currentfill}%
\pgfsetlinewidth{0.000000pt}%
\definecolor{currentstroke}{rgb}{0.000000,0.000000,0.000000}%
\pgfsetstrokecolor{currentstroke}%
\pgfsetdash{}{0pt}%
\pgfpathmoveto{\pgfqpoint{1.402020in}{1.717801in}}%
\pgfpathlineto{\pgfqpoint{1.403585in}{1.715558in}}%
\pgfpathlineto{\pgfqpoint{1.405149in}{1.713214in}}%
\pgfpathlineto{\pgfqpoint{1.406711in}{1.710770in}}%
\pgfpathlineto{\pgfqpoint{1.408271in}{1.708229in}}%
\pgfpathlineto{\pgfqpoint{1.411753in}{1.707361in}}%
\pgfpathlineto{\pgfqpoint{1.415175in}{1.706441in}}%
\pgfpathlineto{\pgfqpoint{1.418536in}{1.705471in}}%
\pgfpathlineto{\pgfqpoint{1.421831in}{1.704451in}}%
\pgfpathlineto{\pgfqpoint{1.419899in}{1.707098in}}%
\pgfpathlineto{\pgfqpoint{1.417965in}{1.709646in}}%
\pgfpathlineto{\pgfqpoint{1.416029in}{1.712094in}}%
\pgfpathlineto{\pgfqpoint{1.414090in}{1.714443in}}%
\pgfpathlineto{\pgfqpoint{1.411157in}{1.715349in}}%
\pgfpathlineto{\pgfqpoint{1.408166in}{1.716211in}}%
\pgfpathlineto{\pgfqpoint{1.405119in}{1.717029in}}%
\pgfpathlineto{\pgfqpoint{1.402020in}{1.717801in}}%
\pgfpathclose%
\pgfusepath{fill}%
\end{pgfscope}%
\begin{pgfscope}%
\pgfpathrectangle{\pgfqpoint{0.329460in}{0.284240in}}{\pgfqpoint{1.989680in}{1.989680in}}%
\pgfusepath{clip}%
\pgfsetbuttcap%
\pgfsetroundjoin%
\definecolor{currentfill}{rgb}{0.699415,0.867117,0.175971}%
\pgfsetfillcolor{currentfill}%
\pgfsetlinewidth{0.000000pt}%
\definecolor{currentstroke}{rgb}{0.000000,0.000000,0.000000}%
\pgfsetstrokecolor{currentstroke}%
\pgfsetdash{}{0pt}%
\pgfpathmoveto{\pgfqpoint{1.254624in}{1.693391in}}%
\pgfpathlineto{\pgfqpoint{1.251981in}{1.690340in}}%
\pgfpathlineto{\pgfqpoint{1.249340in}{1.687193in}}%
\pgfpathlineto{\pgfqpoint{1.246702in}{1.683951in}}%
\pgfpathlineto{\pgfqpoint{1.244067in}{1.680616in}}%
\pgfpathlineto{\pgfqpoint{1.246931in}{1.682184in}}%
\pgfpathlineto{\pgfqpoint{1.249896in}{1.683708in}}%
\pgfpathlineto{\pgfqpoint{1.252960in}{1.685187in}}%
\pgfpathlineto{\pgfqpoint{1.256120in}{1.686619in}}%
\pgfpathlineto{\pgfqpoint{1.258456in}{1.689803in}}%
\pgfpathlineto{\pgfqpoint{1.260795in}{1.692894in}}%
\pgfpathlineto{\pgfqpoint{1.263137in}{1.695891in}}%
\pgfpathlineto{\pgfqpoint{1.265481in}{1.698791in}}%
\pgfpathlineto{\pgfqpoint{1.262634in}{1.697503in}}%
\pgfpathlineto{\pgfqpoint{1.259874in}{1.696172in}}%
\pgfpathlineto{\pgfqpoint{1.257204in}{1.694801in}}%
\pgfpathlineto{\pgfqpoint{1.254624in}{1.693391in}}%
\pgfpathclose%
\pgfusepath{fill}%
\end{pgfscope}%
\begin{pgfscope}%
\pgfpathrectangle{\pgfqpoint{0.329460in}{0.284240in}}{\pgfqpoint{1.989680in}{1.989680in}}%
\pgfusepath{clip}%
\pgfsetbuttcap%
\pgfsetroundjoin%
\definecolor{currentfill}{rgb}{0.120081,0.622161,0.534946}%
\pgfsetfillcolor{currentfill}%
\pgfsetlinewidth{0.000000pt}%
\definecolor{currentstroke}{rgb}{0.000000,0.000000,0.000000}%
\pgfsetstrokecolor{currentstroke}%
\pgfsetdash{}{0pt}%
\pgfpathmoveto{\pgfqpoint{1.610968in}{1.409445in}}%
\pgfpathlineto{\pgfqpoint{1.614304in}{1.402408in}}%
\pgfpathlineto{\pgfqpoint{1.617637in}{1.395345in}}%
\pgfpathlineto{\pgfqpoint{1.620967in}{1.388258in}}%
\pgfpathlineto{\pgfqpoint{1.624295in}{1.381149in}}%
\pgfpathlineto{\pgfqpoint{1.622890in}{1.376899in}}%
\pgfpathlineto{\pgfqpoint{1.621209in}{1.372670in}}%
\pgfpathlineto{\pgfqpoint{1.619256in}{1.368467in}}%
\pgfpathlineto{\pgfqpoint{1.617032in}{1.364293in}}%
\pgfpathlineto{\pgfqpoint{1.613783in}{1.371627in}}%
\pgfpathlineto{\pgfqpoint{1.610533in}{1.378939in}}%
\pgfpathlineto{\pgfqpoint{1.607280in}{1.386227in}}%
\pgfpathlineto{\pgfqpoint{1.604025in}{1.393487in}}%
\pgfpathlineto{\pgfqpoint{1.606150in}{1.397438in}}%
\pgfpathlineto{\pgfqpoint{1.608016in}{1.401418in}}%
\pgfpathlineto{\pgfqpoint{1.609622in}{1.405421in}}%
\pgfpathlineto{\pgfqpoint{1.610968in}{1.409445in}}%
\pgfpathclose%
\pgfusepath{fill}%
\end{pgfscope}%
\begin{pgfscope}%
\pgfpathrectangle{\pgfqpoint{0.329460in}{0.284240in}}{\pgfqpoint{1.989680in}{1.989680in}}%
\pgfusepath{clip}%
\pgfsetbuttcap%
\pgfsetroundjoin%
\definecolor{currentfill}{rgb}{0.147607,0.511733,0.557049}%
\pgfsetfillcolor{currentfill}%
\pgfsetlinewidth{0.000000pt}%
\definecolor{currentstroke}{rgb}{0.000000,0.000000,0.000000}%
\pgfsetstrokecolor{currentstroke}%
\pgfsetdash{}{0pt}%
\pgfpathmoveto{\pgfqpoint{1.075503in}{1.283127in}}%
\pgfpathlineto{\pgfqpoint{1.072440in}{1.275432in}}%
\pgfpathlineto{\pgfqpoint{1.069379in}{1.267738in}}%
\pgfpathlineto{\pgfqpoint{1.066320in}{1.260049in}}%
\pgfpathlineto{\pgfqpoint{1.063262in}{1.252367in}}%
\pgfpathlineto{\pgfqpoint{1.059250in}{1.256983in}}%
\pgfpathlineto{\pgfqpoint{1.055539in}{1.261656in}}%
\pgfpathlineto{\pgfqpoint{1.052130in}{1.266382in}}%
\pgfpathlineto{\pgfqpoint{1.049027in}{1.271157in}}%
\pgfpathlineto{\pgfqpoint{1.052226in}{1.278617in}}%
\pgfpathlineto{\pgfqpoint{1.055428in}{1.286085in}}%
\pgfpathlineto{\pgfqpoint{1.058631in}{1.293557in}}%
\pgfpathlineto{\pgfqpoint{1.061836in}{1.301032in}}%
\pgfpathlineto{\pgfqpoint{1.064817in}{1.296482in}}%
\pgfpathlineto{\pgfqpoint{1.068090in}{1.291978in}}%
\pgfpathlineto{\pgfqpoint{1.071653in}{1.287525in}}%
\pgfpathlineto{\pgfqpoint{1.075503in}{1.283127in}}%
\pgfpathclose%
\pgfusepath{fill}%
\end{pgfscope}%
\begin{pgfscope}%
\pgfpathrectangle{\pgfqpoint{0.329460in}{0.284240in}}{\pgfqpoint{1.989680in}{1.989680in}}%
\pgfusepath{clip}%
\pgfsetbuttcap%
\pgfsetroundjoin%
\definecolor{currentfill}{rgb}{0.762373,0.876424,0.137064}%
\pgfsetfillcolor{currentfill}%
\pgfsetlinewidth{0.000000pt}%
\definecolor{currentstroke}{rgb}{0.000000,0.000000,0.000000}%
\pgfsetstrokecolor{currentstroke}%
\pgfsetdash{}{0pt}%
\pgfpathmoveto{\pgfqpoint{1.285729in}{1.713601in}}%
\pgfpathlineto{\pgfqpoint{1.283712in}{1.711227in}}%
\pgfpathlineto{\pgfqpoint{1.281697in}{1.708752in}}%
\pgfpathlineto{\pgfqpoint{1.279683in}{1.706177in}}%
\pgfpathlineto{\pgfqpoint{1.277672in}{1.703505in}}%
\pgfpathlineto{\pgfqpoint{1.280907in}{1.704567in}}%
\pgfpathlineto{\pgfqpoint{1.284209in}{1.705581in}}%
\pgfpathlineto{\pgfqpoint{1.287577in}{1.706546in}}%
\pgfpathlineto{\pgfqpoint{1.291007in}{1.707460in}}%
\pgfpathlineto{\pgfqpoint{1.292652in}{1.710022in}}%
\pgfpathlineto{\pgfqpoint{1.294299in}{1.712487in}}%
\pgfpathlineto{\pgfqpoint{1.295947in}{1.714852in}}%
\pgfpathlineto{\pgfqpoint{1.297598in}{1.717117in}}%
\pgfpathlineto{\pgfqpoint{1.294545in}{1.716304in}}%
\pgfpathlineto{\pgfqpoint{1.291547in}{1.715447in}}%
\pgfpathlineto{\pgfqpoint{1.288608in}{1.714546in}}%
\pgfpathlineto{\pgfqpoint{1.285729in}{1.713601in}}%
\pgfpathclose%
\pgfusepath{fill}%
\end{pgfscope}%
\begin{pgfscope}%
\pgfpathrectangle{\pgfqpoint{0.329460in}{0.284240in}}{\pgfqpoint{1.989680in}{1.989680in}}%
\pgfusepath{clip}%
\pgfsetbuttcap%
\pgfsetroundjoin%
\definecolor{currentfill}{rgb}{0.195860,0.395433,0.555276}%
\pgfsetfillcolor{currentfill}%
\pgfsetlinewidth{0.000000pt}%
\definecolor{currentstroke}{rgb}{0.000000,0.000000,0.000000}%
\pgfsetstrokecolor{currentstroke}%
\pgfsetdash{}{0pt}%
\pgfpathmoveto{\pgfqpoint{1.059353in}{1.172001in}}%
\pgfpathlineto{\pgfqpoint{1.056497in}{1.164289in}}%
\pgfpathlineto{\pgfqpoint{1.053642in}{1.156612in}}%
\pgfpathlineto{\pgfqpoint{1.050788in}{1.148971in}}%
\pgfpathlineto{\pgfqpoint{1.047934in}{1.141369in}}%
\pgfpathlineto{\pgfqpoint{1.042129in}{1.146329in}}%
\pgfpathlineto{\pgfqpoint{1.036646in}{1.151377in}}%
\pgfpathlineto{\pgfqpoint{1.031489in}{1.156506in}}%
\pgfpathlineto{\pgfqpoint{1.026664in}{1.161712in}}%
\pgfpathlineto{\pgfqpoint{1.029708in}{1.169098in}}%
\pgfpathlineto{\pgfqpoint{1.032753in}{1.176525in}}%
\pgfpathlineto{\pgfqpoint{1.035799in}{1.183988in}}%
\pgfpathlineto{\pgfqpoint{1.038846in}{1.191486in}}%
\pgfpathlineto{\pgfqpoint{1.043500in}{1.186499in}}%
\pgfpathlineto{\pgfqpoint{1.048472in}{1.181586in}}%
\pgfpathlineto{\pgfqpoint{1.053758in}{1.176751in}}%
\pgfpathlineto{\pgfqpoint{1.059353in}{1.172001in}}%
\pgfpathclose%
\pgfusepath{fill}%
\end{pgfscope}%
\begin{pgfscope}%
\pgfpathrectangle{\pgfqpoint{0.329460in}{0.284240in}}{\pgfqpoint{1.989680in}{1.989680in}}%
\pgfusepath{clip}%
\pgfsetbuttcap%
\pgfsetroundjoin%
\definecolor{currentfill}{rgb}{0.268510,0.009605,0.335427}%
\pgfsetfillcolor{currentfill}%
\pgfsetlinewidth{0.000000pt}%
\definecolor{currentstroke}{rgb}{0.000000,0.000000,0.000000}%
\pgfsetstrokecolor{currentstroke}%
\pgfsetdash{}{0pt}%
\pgfpathmoveto{\pgfqpoint{1.824169in}{0.885083in}}%
\pgfpathlineto{\pgfqpoint{1.827225in}{0.885317in}}%
\pgfpathlineto{\pgfqpoint{1.830289in}{0.885801in}}%
\pgfpathlineto{\pgfqpoint{1.833362in}{0.886541in}}%
\pgfpathlineto{\pgfqpoint{1.836442in}{0.887542in}}%
\pgfpathlineto{\pgfqpoint{1.827466in}{0.879435in}}%
\pgfpathlineto{\pgfqpoint{1.817981in}{0.871474in}}%
\pgfpathlineto{\pgfqpoint{1.807995in}{0.863668in}}%
\pgfpathlineto{\pgfqpoint{1.797517in}{0.856026in}}%
\pgfpathlineto{\pgfqpoint{1.794670in}{0.855222in}}%
\pgfpathlineto{\pgfqpoint{1.791831in}{0.854679in}}%
\pgfpathlineto{\pgfqpoint{1.789000in}{0.854393in}}%
\pgfpathlineto{\pgfqpoint{1.786176in}{0.854358in}}%
\pgfpathlineto{\pgfqpoint{1.796400in}{0.861807in}}%
\pgfpathlineto{\pgfqpoint{1.806146in}{0.869417in}}%
\pgfpathlineto{\pgfqpoint{1.815405in}{0.877179in}}%
\pgfpathlineto{\pgfqpoint{1.824169in}{0.885083in}}%
\pgfpathclose%
\pgfusepath{fill}%
\end{pgfscope}%
\begin{pgfscope}%
\pgfpathrectangle{\pgfqpoint{0.329460in}{0.284240in}}{\pgfqpoint{1.989680in}{1.989680in}}%
\pgfusepath{clip}%
\pgfsetbuttcap%
\pgfsetroundjoin%
\definecolor{currentfill}{rgb}{0.282327,0.094955,0.417331}%
\pgfsetfillcolor{currentfill}%
\pgfsetlinewidth{0.000000pt}%
\definecolor{currentstroke}{rgb}{0.000000,0.000000,0.000000}%
\pgfsetstrokecolor{currentstroke}%
\pgfsetdash{}{0pt}%
\pgfpathmoveto{\pgfqpoint{0.849288in}{0.897943in}}%
\pgfpathlineto{\pgfqpoint{0.846168in}{0.901468in}}%
\pgfpathlineto{\pgfqpoint{0.843036in}{0.905305in}}%
\pgfpathlineto{\pgfqpoint{0.839893in}{0.909459in}}%
\pgfpathlineto{\pgfqpoint{0.836739in}{0.913935in}}%
\pgfpathlineto{\pgfqpoint{0.827175in}{0.922645in}}%
\pgfpathlineto{\pgfqpoint{0.818172in}{0.931498in}}%
\pgfpathlineto{\pgfqpoint{0.809736in}{0.940486in}}%
\pgfpathlineto{\pgfqpoint{0.801875in}{0.949597in}}%
\pgfpathlineto{\pgfqpoint{0.805226in}{0.944927in}}%
\pgfpathlineto{\pgfqpoint{0.808564in}{0.940578in}}%
\pgfpathlineto{\pgfqpoint{0.811890in}{0.936546in}}%
\pgfpathlineto{\pgfqpoint{0.815205in}{0.932824in}}%
\pgfpathlineto{\pgfqpoint{0.822893in}{0.923911in}}%
\pgfpathlineto{\pgfqpoint{0.831140in}{0.915121in}}%
\pgfpathlineto{\pgfqpoint{0.839941in}{0.906461in}}%
\pgfpathlineto{\pgfqpoint{0.849288in}{0.897943in}}%
\pgfpathclose%
\pgfusepath{fill}%
\end{pgfscope}%
\begin{pgfscope}%
\pgfpathrectangle{\pgfqpoint{0.329460in}{0.284240in}}{\pgfqpoint{1.989680in}{1.989680in}}%
\pgfusepath{clip}%
\pgfsetbuttcap%
\pgfsetroundjoin%
\definecolor{currentfill}{rgb}{0.281477,0.755203,0.432552}%
\pgfsetfillcolor{currentfill}%
\pgfsetlinewidth{0.000000pt}%
\definecolor{currentstroke}{rgb}{0.000000,0.000000,0.000000}%
\pgfsetstrokecolor{currentstroke}%
\pgfsetdash{}{0pt}%
\pgfpathmoveto{\pgfqpoint{1.558540in}{1.541975in}}%
\pgfpathlineto{\pgfqpoint{1.561923in}{1.536092in}}%
\pgfpathlineto{\pgfqpoint{1.565304in}{1.530150in}}%
\pgfpathlineto{\pgfqpoint{1.568681in}{1.524151in}}%
\pgfpathlineto{\pgfqpoint{1.572056in}{1.518097in}}%
\pgfpathlineto{\pgfqpoint{1.572665in}{1.514723in}}%
\pgfpathlineto{\pgfqpoint{1.573054in}{1.511340in}}%
\pgfpathlineto{\pgfqpoint{1.573222in}{1.507951in}}%
\pgfpathlineto{\pgfqpoint{1.573169in}{1.504559in}}%
\pgfpathlineto{\pgfqpoint{1.569770in}{1.510834in}}%
\pgfpathlineto{\pgfqpoint{1.566368in}{1.517053in}}%
\pgfpathlineto{\pgfqpoint{1.562963in}{1.523215in}}%
\pgfpathlineto{\pgfqpoint{1.559556in}{1.529318in}}%
\pgfpathlineto{\pgfqpoint{1.559613in}{1.532489in}}%
\pgfpathlineto{\pgfqpoint{1.559462in}{1.535657in}}%
\pgfpathlineto{\pgfqpoint{1.559104in}{1.538821in}}%
\pgfpathlineto{\pgfqpoint{1.558540in}{1.541975in}}%
\pgfpathclose%
\pgfusepath{fill}%
\end{pgfscope}%
\begin{pgfscope}%
\pgfpathrectangle{\pgfqpoint{0.329460in}{0.284240in}}{\pgfqpoint{1.989680in}{1.989680in}}%
\pgfusepath{clip}%
\pgfsetbuttcap%
\pgfsetroundjoin%
\definecolor{currentfill}{rgb}{0.248629,0.278775,0.534556}%
\pgfsetfillcolor{currentfill}%
\pgfsetlinewidth{0.000000pt}%
\definecolor{currentstroke}{rgb}{0.000000,0.000000,0.000000}%
\pgfsetstrokecolor{currentstroke}%
\pgfsetdash{}{0pt}%
\pgfpathmoveto{\pgfqpoint{1.682805in}{1.087116in}}%
\pgfpathlineto{\pgfqpoint{1.685703in}{1.080051in}}%
\pgfpathlineto{\pgfqpoint{1.688601in}{1.073056in}}%
\pgfpathlineto{\pgfqpoint{1.691499in}{1.066133in}}%
\pgfpathlineto{\pgfqpoint{1.694398in}{1.059286in}}%
\pgfpathlineto{\pgfqpoint{1.687928in}{1.053703in}}%
\pgfpathlineto{\pgfqpoint{1.681106in}{1.048224in}}%
\pgfpathlineto{\pgfqpoint{1.673937in}{1.042856in}}%
\pgfpathlineto{\pgfqpoint{1.666428in}{1.037604in}}%
\pgfpathlineto{\pgfqpoint{1.663757in}{1.044657in}}%
\pgfpathlineto{\pgfqpoint{1.661086in}{1.051786in}}%
\pgfpathlineto{\pgfqpoint{1.658416in}{1.058988in}}%
\pgfpathlineto{\pgfqpoint{1.655745in}{1.066259in}}%
\pgfpathlineto{\pgfqpoint{1.663009in}{1.071311in}}%
\pgfpathlineto{\pgfqpoint{1.669944in}{1.076475in}}%
\pgfpathlineto{\pgfqpoint{1.676545in}{1.081745in}}%
\pgfpathlineto{\pgfqpoint{1.682805in}{1.087116in}}%
\pgfpathclose%
\pgfusepath{fill}%
\end{pgfscope}%
\begin{pgfscope}%
\pgfpathrectangle{\pgfqpoint{0.329460in}{0.284240in}}{\pgfqpoint{1.989680in}{1.989680in}}%
\pgfusepath{clip}%
\pgfsetbuttcap%
\pgfsetroundjoin%
\definecolor{currentfill}{rgb}{0.267004,0.004874,0.329415}%
\pgfsetfillcolor{currentfill}%
\pgfsetlinewidth{0.000000pt}%
\definecolor{currentstroke}{rgb}{0.000000,0.000000,0.000000}%
\pgfsetstrokecolor{currentstroke}%
\pgfsetdash{}{0pt}%
\pgfpathmoveto{\pgfqpoint{0.936673in}{0.850328in}}%
\pgfpathlineto{\pgfqpoint{0.933935in}{0.849361in}}%
\pgfpathlineto{\pgfqpoint{0.931190in}{0.848627in}}%
\pgfpathlineto{\pgfqpoint{0.928439in}{0.848131in}}%
\pgfpathlineto{\pgfqpoint{0.925681in}{0.847877in}}%
\pgfpathlineto{\pgfqpoint{0.915040in}{0.855177in}}%
\pgfpathlineto{\pgfqpoint{0.904868in}{0.862645in}}%
\pgfpathlineto{\pgfqpoint{0.895176in}{0.870272in}}%
\pgfpathlineto{\pgfqpoint{0.885971in}{0.878050in}}%
\pgfpathlineto{\pgfqpoint{0.888973in}{0.878108in}}%
\pgfpathlineto{\pgfqpoint{0.891968in}{0.878408in}}%
\pgfpathlineto{\pgfqpoint{0.894955in}{0.878945in}}%
\pgfpathlineto{\pgfqpoint{0.897935in}{0.879714in}}%
\pgfpathlineto{\pgfqpoint{0.906916in}{0.872138in}}%
\pgfpathlineto{\pgfqpoint{0.916372in}{0.864710in}}%
\pgfpathlineto{\pgfqpoint{0.926294in}{0.857437in}}%
\pgfpathlineto{\pgfqpoint{0.936673in}{0.850328in}}%
\pgfpathclose%
\pgfusepath{fill}%
\end{pgfscope}%
\begin{pgfscope}%
\pgfpathrectangle{\pgfqpoint{0.329460in}{0.284240in}}{\pgfqpoint{1.989680in}{1.989680in}}%
\pgfusepath{clip}%
\pgfsetbuttcap%
\pgfsetroundjoin%
\definecolor{currentfill}{rgb}{0.565498,0.842430,0.262877}%
\pgfsetfillcolor{currentfill}%
\pgfsetlinewidth{0.000000pt}%
\definecolor{currentstroke}{rgb}{0.000000,0.000000,0.000000}%
\pgfsetstrokecolor{currentstroke}%
\pgfsetdash{}{0pt}%
\pgfpathmoveto{\pgfqpoint{1.487734in}{1.652859in}}%
\pgfpathlineto{\pgfqpoint{1.490785in}{1.648767in}}%
\pgfpathlineto{\pgfqpoint{1.493832in}{1.644590in}}%
\pgfpathlineto{\pgfqpoint{1.496877in}{1.640329in}}%
\pgfpathlineto{\pgfqpoint{1.499918in}{1.635986in}}%
\pgfpathlineto{\pgfqpoint{1.502251in}{1.633744in}}%
\pgfpathlineto{\pgfqpoint{1.504436in}{1.631469in}}%
\pgfpathlineto{\pgfqpoint{1.506470in}{1.629160in}}%
\pgfpathlineto{\pgfqpoint{1.508352in}{1.626822in}}%
\pgfpathlineto{\pgfqpoint{1.505134in}{1.631359in}}%
\pgfpathlineto{\pgfqpoint{1.501912in}{1.635814in}}%
\pgfpathlineto{\pgfqpoint{1.498687in}{1.640185in}}%
\pgfpathlineto{\pgfqpoint{1.495459in}{1.644471in}}%
\pgfpathlineto{\pgfqpoint{1.493736in}{1.646611in}}%
\pgfpathlineto{\pgfqpoint{1.491873in}{1.648724in}}%
\pgfpathlineto{\pgfqpoint{1.489872in}{1.650807in}}%
\pgfpathlineto{\pgfqpoint{1.487734in}{1.652859in}}%
\pgfpathclose%
\pgfusepath{fill}%
\end{pgfscope}%
\begin{pgfscope}%
\pgfpathrectangle{\pgfqpoint{0.329460in}{0.284240in}}{\pgfqpoint{1.989680in}{1.989680in}}%
\pgfusepath{clip}%
\pgfsetbuttcap%
\pgfsetroundjoin%
\definecolor{currentfill}{rgb}{0.263663,0.237631,0.518762}%
\pgfsetfillcolor{currentfill}%
\pgfsetlinewidth{0.000000pt}%
\definecolor{currentstroke}{rgb}{0.000000,0.000000,0.000000}%
\pgfsetstrokecolor{currentstroke}%
\pgfsetdash{}{0pt}%
\pgfpathmoveto{\pgfqpoint{1.042902in}{1.033038in}}%
\pgfpathlineto{\pgfqpoint{1.040287in}{1.026019in}}%
\pgfpathlineto{\pgfqpoint{1.037672in}{1.019083in}}%
\pgfpathlineto{\pgfqpoint{1.035055in}{1.012231in}}%
\pgfpathlineto{\pgfqpoint{1.032439in}{1.005468in}}%
\pgfpathlineto{\pgfqpoint{1.024377in}{1.010809in}}%
\pgfpathlineto{\pgfqpoint{1.016661in}{1.016275in}}%
\pgfpathlineto{\pgfqpoint{1.009299in}{1.021862in}}%
\pgfpathlineto{\pgfqpoint{1.002296in}{1.027562in}}%
\pgfpathlineto{\pgfqpoint{1.005150in}{1.034121in}}%
\pgfpathlineto{\pgfqpoint{1.008004in}{1.040769in}}%
\pgfpathlineto{\pgfqpoint{1.010858in}{1.047503in}}%
\pgfpathlineto{\pgfqpoint{1.013710in}{1.054318in}}%
\pgfpathlineto{\pgfqpoint{1.020494in}{1.048827in}}%
\pgfpathlineto{\pgfqpoint{1.027625in}{1.043447in}}%
\pgfpathlineto{\pgfqpoint{1.035096in}{1.038181in}}%
\pgfpathlineto{\pgfqpoint{1.042902in}{1.033038in}}%
\pgfpathclose%
\pgfusepath{fill}%
\end{pgfscope}%
\begin{pgfscope}%
\pgfpathrectangle{\pgfqpoint{0.329460in}{0.284240in}}{\pgfqpoint{1.989680in}{1.989680in}}%
\pgfusepath{clip}%
\pgfsetbuttcap%
\pgfsetroundjoin%
\definecolor{currentfill}{rgb}{0.179019,0.433756,0.557430}%
\pgfsetfillcolor{currentfill}%
\pgfsetlinewidth{0.000000pt}%
\definecolor{currentstroke}{rgb}{0.000000,0.000000,0.000000}%
\pgfsetstrokecolor{currentstroke}%
\pgfsetdash{}{0pt}%
\pgfpathmoveto{\pgfqpoint{1.655055in}{1.226057in}}%
\pgfpathlineto{\pgfqpoint{1.658142in}{1.218500in}}%
\pgfpathlineto{\pgfqpoint{1.661228in}{1.210965in}}%
\pgfpathlineto{\pgfqpoint{1.664313in}{1.203456in}}%
\pgfpathlineto{\pgfqpoint{1.667396in}{1.195976in}}%
\pgfpathlineto{\pgfqpoint{1.663028in}{1.190928in}}%
\pgfpathlineto{\pgfqpoint{1.658339in}{1.185949in}}%
\pgfpathlineto{\pgfqpoint{1.653331in}{1.181045in}}%
\pgfpathlineto{\pgfqpoint{1.648011in}{1.176219in}}%
\pgfpathlineto{\pgfqpoint{1.645107in}{1.183916in}}%
\pgfpathlineto{\pgfqpoint{1.642203in}{1.191641in}}%
\pgfpathlineto{\pgfqpoint{1.639298in}{1.199391in}}%
\pgfpathlineto{\pgfqpoint{1.636392in}{1.207165in}}%
\pgfpathlineto{\pgfqpoint{1.641513in}{1.211778in}}%
\pgfpathlineto{\pgfqpoint{1.646333in}{1.216469in}}%
\pgfpathlineto{\pgfqpoint{1.650849in}{1.221230in}}%
\pgfpathlineto{\pgfqpoint{1.655055in}{1.226057in}}%
\pgfpathclose%
\pgfusepath{fill}%
\end{pgfscope}%
\begin{pgfscope}%
\pgfpathrectangle{\pgfqpoint{0.329460in}{0.284240in}}{\pgfqpoint{1.989680in}{1.989680in}}%
\pgfusepath{clip}%
\pgfsetbuttcap%
\pgfsetroundjoin%
\definecolor{currentfill}{rgb}{0.636902,0.856542,0.216620}%
\pgfsetfillcolor{currentfill}%
\pgfsetlinewidth{0.000000pt}%
\definecolor{currentstroke}{rgb}{0.000000,0.000000,0.000000}%
\pgfsetstrokecolor{currentstroke}%
\pgfsetdash{}{0pt}%
\pgfpathmoveto{\pgfqpoint{1.466537in}{1.675476in}}%
\pgfpathlineto{\pgfqpoint{1.469375in}{1.671919in}}%
\pgfpathlineto{\pgfqpoint{1.472210in}{1.668272in}}%
\pgfpathlineto{\pgfqpoint{1.475042in}{1.664535in}}%
\pgfpathlineto{\pgfqpoint{1.477871in}{1.660711in}}%
\pgfpathlineto{\pgfqpoint{1.480529in}{1.658805in}}%
\pgfpathlineto{\pgfqpoint{1.483060in}{1.656859in}}%
\pgfpathlineto{\pgfqpoint{1.485463in}{1.654877in}}%
\pgfpathlineto{\pgfqpoint{1.487734in}{1.652859in}}%
\pgfpathlineto{\pgfqpoint{1.484681in}{1.656864in}}%
\pgfpathlineto{\pgfqpoint{1.481625in}{1.660781in}}%
\pgfpathlineto{\pgfqpoint{1.478566in}{1.664609in}}%
\pgfpathlineto{\pgfqpoint{1.475504in}{1.668345in}}%
\pgfpathlineto{\pgfqpoint{1.473440in}{1.670178in}}%
\pgfpathlineto{\pgfqpoint{1.471256in}{1.671978in}}%
\pgfpathlineto{\pgfqpoint{1.468954in}{1.673745in}}%
\pgfpathlineto{\pgfqpoint{1.466537in}{1.675476in}}%
\pgfpathclose%
\pgfusepath{fill}%
\end{pgfscope}%
\begin{pgfscope}%
\pgfpathrectangle{\pgfqpoint{0.329460in}{0.284240in}}{\pgfqpoint{1.989680in}{1.989680in}}%
\pgfusepath{clip}%
\pgfsetbuttcap%
\pgfsetroundjoin%
\definecolor{currentfill}{rgb}{0.487026,0.823929,0.312321}%
\pgfsetfillcolor{currentfill}%
\pgfsetlinewidth{0.000000pt}%
\definecolor{currentstroke}{rgb}{0.000000,0.000000,0.000000}%
\pgfsetstrokecolor{currentstroke}%
\pgfsetdash{}{0pt}%
\pgfpathmoveto{\pgfqpoint{1.508352in}{1.626822in}}%
\pgfpathlineto{\pgfqpoint{1.511568in}{1.622203in}}%
\pgfpathlineto{\pgfqpoint{1.514781in}{1.617506in}}%
\pgfpathlineto{\pgfqpoint{1.517990in}{1.612730in}}%
\pgfpathlineto{\pgfqpoint{1.521197in}{1.607878in}}%
\pgfpathlineto{\pgfqpoint{1.523071in}{1.605310in}}%
\pgfpathlineto{\pgfqpoint{1.524776in}{1.602714in}}%
\pgfpathlineto{\pgfqpoint{1.526311in}{1.600092in}}%
\pgfpathlineto{\pgfqpoint{1.527672in}{1.597447in}}%
\pgfpathlineto{\pgfqpoint{1.524338in}{1.602504in}}%
\pgfpathlineto{\pgfqpoint{1.521001in}{1.607485in}}%
\pgfpathlineto{\pgfqpoint{1.517660in}{1.612388in}}%
\pgfpathlineto{\pgfqpoint{1.514317in}{1.617211in}}%
\pgfpathlineto{\pgfqpoint{1.513064in}{1.619647in}}%
\pgfpathlineto{\pgfqpoint{1.511651in}{1.622063in}}%
\pgfpathlineto{\pgfqpoint{1.510080in}{1.624455in}}%
\pgfpathlineto{\pgfqpoint{1.508352in}{1.626822in}}%
\pgfpathclose%
\pgfusepath{fill}%
\end{pgfscope}%
\begin{pgfscope}%
\pgfpathrectangle{\pgfqpoint{0.329460in}{0.284240in}}{\pgfqpoint{1.989680in}{1.989680in}}%
\pgfusepath{clip}%
\pgfsetbuttcap%
\pgfsetroundjoin%
\definecolor{currentfill}{rgb}{0.166383,0.690856,0.496502}%
\pgfsetfillcolor{currentfill}%
\pgfsetlinewidth{0.000000pt}%
\definecolor{currentstroke}{rgb}{0.000000,0.000000,0.000000}%
\pgfsetstrokecolor{currentstroke}%
\pgfsetdash{}{0pt}%
\pgfpathmoveto{\pgfqpoint{1.119259in}{1.461345in}}%
\pgfpathlineto{\pgfqpoint{1.115917in}{1.454549in}}%
\pgfpathlineto{\pgfqpoint{1.112577in}{1.447709in}}%
\pgfpathlineto{\pgfqpoint{1.109239in}{1.440827in}}%
\pgfpathlineto{\pgfqpoint{1.105905in}{1.433906in}}%
\pgfpathlineto{\pgfqpoint{1.104647in}{1.437707in}}%
\pgfpathlineto{\pgfqpoint{1.103637in}{1.441524in}}%
\pgfpathlineto{\pgfqpoint{1.102876in}{1.445352in}}%
\pgfpathlineto{\pgfqpoint{1.102364in}{1.449187in}}%
\pgfpathlineto{\pgfqpoint{1.105739in}{1.455884in}}%
\pgfpathlineto{\pgfqpoint{1.109116in}{1.462541in}}%
\pgfpathlineto{\pgfqpoint{1.112496in}{1.469158in}}%
\pgfpathlineto{\pgfqpoint{1.115879in}{1.475730in}}%
\pgfpathlineto{\pgfqpoint{1.116372in}{1.472119in}}%
\pgfpathlineto{\pgfqpoint{1.117099in}{1.468516in}}%
\pgfpathlineto{\pgfqpoint{1.118062in}{1.464923in}}%
\pgfpathlineto{\pgfqpoint{1.119259in}{1.461345in}}%
\pgfpathclose%
\pgfusepath{fill}%
\end{pgfscope}%
\begin{pgfscope}%
\pgfpathrectangle{\pgfqpoint{0.329460in}{0.284240in}}{\pgfqpoint{1.989680in}{1.989680in}}%
\pgfusepath{clip}%
\pgfsetbuttcap%
\pgfsetroundjoin%
\definecolor{currentfill}{rgb}{0.133743,0.548535,0.553541}%
\pgfsetfillcolor{currentfill}%
\pgfsetlinewidth{0.000000pt}%
\definecolor{currentstroke}{rgb}{0.000000,0.000000,0.000000}%
\pgfsetstrokecolor{currentstroke}%
\pgfsetdash{}{0pt}%
\pgfpathmoveto{\pgfqpoint{1.630003in}{1.334783in}}%
\pgfpathlineto{\pgfqpoint{1.633241in}{1.327374in}}%
\pgfpathlineto{\pgfqpoint{1.636476in}{1.319957in}}%
\pgfpathlineto{\pgfqpoint{1.639710in}{1.312536in}}%
\pgfpathlineto{\pgfqpoint{1.642941in}{1.305111in}}%
\pgfpathlineto{\pgfqpoint{1.640222in}{1.300524in}}%
\pgfpathlineto{\pgfqpoint{1.637209in}{1.295979in}}%
\pgfpathlineto{\pgfqpoint{1.633903in}{1.291480in}}%
\pgfpathlineto{\pgfqpoint{1.630308in}{1.287033in}}%
\pgfpathlineto{\pgfqpoint{1.627208in}{1.294680in}}%
\pgfpathlineto{\pgfqpoint{1.624106in}{1.302323in}}%
\pgfpathlineto{\pgfqpoint{1.621002in}{1.309961in}}%
\pgfpathlineto{\pgfqpoint{1.617896in}{1.317591in}}%
\pgfpathlineto{\pgfqpoint{1.621340in}{1.321820in}}%
\pgfpathlineto{\pgfqpoint{1.624507in}{1.326098in}}%
\pgfpathlineto{\pgfqpoint{1.627396in}{1.330420in}}%
\pgfpathlineto{\pgfqpoint{1.630003in}{1.334783in}}%
\pgfpathclose%
\pgfusepath{fill}%
\end{pgfscope}%
\begin{pgfscope}%
\pgfpathrectangle{\pgfqpoint{0.329460in}{0.284240in}}{\pgfqpoint{1.989680in}{1.989680in}}%
\pgfusepath{clip}%
\pgfsetbuttcap%
\pgfsetroundjoin%
\definecolor{currentfill}{rgb}{0.814576,0.883393,0.110347}%
\pgfsetfillcolor{currentfill}%
\pgfsetlinewidth{0.000000pt}%
\definecolor{currentstroke}{rgb}{0.000000,0.000000,0.000000}%
\pgfsetstrokecolor{currentstroke}%
\pgfsetdash{}{0pt}%
\pgfpathmoveto{\pgfqpoint{1.339104in}{1.730425in}}%
\pgfpathlineto{\pgfqpoint{1.338678in}{1.728761in}}%
\pgfpathlineto{\pgfqpoint{1.338253in}{1.726992in}}%
\pgfpathlineto{\pgfqpoint{1.337828in}{1.725119in}}%
\pgfpathlineto{\pgfqpoint{1.337403in}{1.723143in}}%
\pgfpathlineto{\pgfqpoint{1.340886in}{1.723321in}}%
\pgfpathlineto{\pgfqpoint{1.344380in}{1.723447in}}%
\pgfpathlineto{\pgfqpoint{1.347879in}{1.723521in}}%
\pgfpathlineto{\pgfqpoint{1.351382in}{1.723544in}}%
\pgfpathlineto{\pgfqpoint{1.351376in}{1.725507in}}%
\pgfpathlineto{\pgfqpoint{1.351370in}{1.727367in}}%
\pgfpathlineto{\pgfqpoint{1.351364in}{1.729124in}}%
\pgfpathlineto{\pgfqpoint{1.351358in}{1.730776in}}%
\pgfpathlineto{\pgfqpoint{1.348287in}{1.730756in}}%
\pgfpathlineto{\pgfqpoint{1.345220in}{1.730690in}}%
\pgfpathlineto{\pgfqpoint{1.342158in}{1.730580in}}%
\pgfpathlineto{\pgfqpoint{1.339104in}{1.730425in}}%
\pgfpathclose%
\pgfusepath{fill}%
\end{pgfscope}%
\begin{pgfscope}%
\pgfpathrectangle{\pgfqpoint{0.329460in}{0.284240in}}{\pgfqpoint{1.989680in}{1.989680in}}%
\pgfusepath{clip}%
\pgfsetbuttcap%
\pgfsetroundjoin%
\definecolor{currentfill}{rgb}{0.814576,0.883393,0.110347}%
\pgfsetfillcolor{currentfill}%
\pgfsetlinewidth{0.000000pt}%
\definecolor{currentstroke}{rgb}{0.000000,0.000000,0.000000}%
\pgfsetstrokecolor{currentstroke}%
\pgfsetdash{}{0pt}%
\pgfpathmoveto{\pgfqpoint{1.351358in}{1.730776in}}%
\pgfpathlineto{\pgfqpoint{1.351364in}{1.729124in}}%
\pgfpathlineto{\pgfqpoint{1.351370in}{1.727367in}}%
\pgfpathlineto{\pgfqpoint{1.351376in}{1.725507in}}%
\pgfpathlineto{\pgfqpoint{1.351382in}{1.723544in}}%
\pgfpathlineto{\pgfqpoint{1.354885in}{1.723515in}}%
\pgfpathlineto{\pgfqpoint{1.358384in}{1.723435in}}%
\pgfpathlineto{\pgfqpoint{1.361876in}{1.723304in}}%
\pgfpathlineto{\pgfqpoint{1.365358in}{1.723120in}}%
\pgfpathlineto{\pgfqpoint{1.364922in}{1.725097in}}%
\pgfpathlineto{\pgfqpoint{1.364485in}{1.726970in}}%
\pgfpathlineto{\pgfqpoint{1.364047in}{1.728740in}}%
\pgfpathlineto{\pgfqpoint{1.363609in}{1.730405in}}%
\pgfpathlineto{\pgfqpoint{1.360557in}{1.730565in}}%
\pgfpathlineto{\pgfqpoint{1.357496in}{1.730680in}}%
\pgfpathlineto{\pgfqpoint{1.354429in}{1.730751in}}%
\pgfpathlineto{\pgfqpoint{1.351358in}{1.730776in}}%
\pgfpathclose%
\pgfusepath{fill}%
\end{pgfscope}%
\begin{pgfscope}%
\pgfpathrectangle{\pgfqpoint{0.329460in}{0.284240in}}{\pgfqpoint{1.989680in}{1.989680in}}%
\pgfusepath{clip}%
\pgfsetbuttcap%
\pgfsetroundjoin%
\definecolor{currentfill}{rgb}{0.260571,0.246922,0.522828}%
\pgfsetfillcolor{currentfill}%
\pgfsetlinewidth{0.000000pt}%
\definecolor{currentstroke}{rgb}{0.000000,0.000000,0.000000}%
\pgfsetstrokecolor{currentstroke}%
\pgfsetdash{}{0pt}%
\pgfpathmoveto{\pgfqpoint{0.774589in}{0.999220in}}%
\pgfpathlineto{\pgfqpoint{0.771112in}{1.007046in}}%
\pgfpathlineto{\pgfqpoint{0.767620in}{1.015252in}}%
\pgfpathlineto{\pgfqpoint{0.764111in}{1.023847in}}%
\pgfpathlineto{\pgfqpoint{0.760585in}{1.032836in}}%
\pgfpathlineto{\pgfqpoint{0.752820in}{1.042627in}}%
\pgfpathlineto{\pgfqpoint{0.745687in}{1.052526in}}%
\pgfpathlineto{\pgfqpoint{0.739191in}{1.062522in}}%
\pgfpathlineto{\pgfqpoint{0.733335in}{1.072603in}}%
\pgfpathlineto{\pgfqpoint{0.737001in}{1.063433in}}%
\pgfpathlineto{\pgfqpoint{0.740649in}{1.054656in}}%
\pgfpathlineto{\pgfqpoint{0.744281in}{1.046264in}}%
\pgfpathlineto{\pgfqpoint{0.747897in}{1.038252in}}%
\pgfpathlineto{\pgfqpoint{0.753637in}{1.028356in}}%
\pgfpathlineto{\pgfqpoint{0.760002in}{1.018545in}}%
\pgfpathlineto{\pgfqpoint{0.766988in}{1.008829in}}%
\pgfpathlineto{\pgfqpoint{0.774589in}{0.999220in}}%
\pgfpathclose%
\pgfusepath{fill}%
\end{pgfscope}%
\begin{pgfscope}%
\pgfpathrectangle{\pgfqpoint{0.329460in}{0.284240in}}{\pgfqpoint{1.989680in}{1.989680in}}%
\pgfusepath{clip}%
\pgfsetbuttcap%
\pgfsetroundjoin%
\definecolor{currentfill}{rgb}{0.120081,0.622161,0.534946}%
\pgfsetfillcolor{currentfill}%
\pgfsetlinewidth{0.000000pt}%
\definecolor{currentstroke}{rgb}{0.000000,0.000000,0.000000}%
\pgfsetstrokecolor{currentstroke}%
\pgfsetdash{}{0pt}%
\pgfpathmoveto{\pgfqpoint{1.100454in}{1.390003in}}%
\pgfpathlineto{\pgfqpoint{1.097224in}{1.382693in}}%
\pgfpathlineto{\pgfqpoint{1.093996in}{1.375356in}}%
\pgfpathlineto{\pgfqpoint{1.090771in}{1.367995in}}%
\pgfpathlineto{\pgfqpoint{1.087548in}{1.360611in}}%
\pgfpathlineto{\pgfqpoint{1.085083in}{1.364755in}}%
\pgfpathlineto{\pgfqpoint{1.082888in}{1.368932in}}%
\pgfpathlineto{\pgfqpoint{1.080965in}{1.373139in}}%
\pgfpathlineto{\pgfqpoint{1.079316in}{1.377370in}}%
\pgfpathlineto{\pgfqpoint{1.082631in}{1.384530in}}%
\pgfpathlineto{\pgfqpoint{1.085948in}{1.391667in}}%
\pgfpathlineto{\pgfqpoint{1.089268in}{1.398780in}}%
\pgfpathlineto{\pgfqpoint{1.092590in}{1.405867in}}%
\pgfpathlineto{\pgfqpoint{1.094168in}{1.401861in}}%
\pgfpathlineto{\pgfqpoint{1.096005in}{1.397879in}}%
\pgfpathlineto{\pgfqpoint{1.098101in}{1.393925in}}%
\pgfpathlineto{\pgfqpoint{1.100454in}{1.390003in}}%
\pgfpathclose%
\pgfusepath{fill}%
\end{pgfscope}%
\begin{pgfscope}%
\pgfpathrectangle{\pgfqpoint{0.329460in}{0.284240in}}{\pgfqpoint{1.989680in}{1.989680in}}%
\pgfusepath{clip}%
\pgfsetbuttcap%
\pgfsetroundjoin%
\definecolor{currentfill}{rgb}{0.762373,0.876424,0.137064}%
\pgfsetfillcolor{currentfill}%
\pgfsetlinewidth{0.000000pt}%
\definecolor{currentstroke}{rgb}{0.000000,0.000000,0.000000}%
\pgfsetstrokecolor{currentstroke}%
\pgfsetdash{}{0pt}%
\pgfpathmoveto{\pgfqpoint{1.414090in}{1.714443in}}%
\pgfpathlineto{\pgfqpoint{1.416029in}{1.712094in}}%
\pgfpathlineto{\pgfqpoint{1.417965in}{1.709646in}}%
\pgfpathlineto{\pgfqpoint{1.419899in}{1.707098in}}%
\pgfpathlineto{\pgfqpoint{1.421831in}{1.704451in}}%
\pgfpathlineto{\pgfqpoint{1.425058in}{1.703384in}}%
\pgfpathlineto{\pgfqpoint{1.428212in}{1.702269in}}%
\pgfpathlineto{\pgfqpoint{1.431292in}{1.701107in}}%
\pgfpathlineto{\pgfqpoint{1.434294in}{1.699901in}}%
\pgfpathlineto{\pgfqpoint{1.432019in}{1.702674in}}%
\pgfpathlineto{\pgfqpoint{1.429742in}{1.705348in}}%
\pgfpathlineto{\pgfqpoint{1.427463in}{1.707923in}}%
\pgfpathlineto{\pgfqpoint{1.425181in}{1.710398in}}%
\pgfpathlineto{\pgfqpoint{1.422510in}{1.711470in}}%
\pgfpathlineto{\pgfqpoint{1.419770in}{1.712503in}}%
\pgfpathlineto{\pgfqpoint{1.416962in}{1.713494in}}%
\pgfpathlineto{\pgfqpoint{1.414090in}{1.714443in}}%
\pgfpathclose%
\pgfusepath{fill}%
\end{pgfscope}%
\begin{pgfscope}%
\pgfpathrectangle{\pgfqpoint{0.329460in}{0.284240in}}{\pgfqpoint{1.989680in}{1.989680in}}%
\pgfusepath{clip}%
\pgfsetbuttcap%
\pgfsetroundjoin%
\definecolor{currentfill}{rgb}{0.814576,0.883393,0.110347}%
\pgfsetfillcolor{currentfill}%
\pgfsetlinewidth{0.000000pt}%
\definecolor{currentstroke}{rgb}{0.000000,0.000000,0.000000}%
\pgfsetstrokecolor{currentstroke}%
\pgfsetdash{}{0pt}%
\pgfpathmoveto{\pgfqpoint{1.327038in}{1.729358in}}%
\pgfpathlineto{\pgfqpoint{1.326187in}{1.727655in}}%
\pgfpathlineto{\pgfqpoint{1.325336in}{1.725848in}}%
\pgfpathlineto{\pgfqpoint{1.324487in}{1.723938in}}%
\pgfpathlineto{\pgfqpoint{1.323638in}{1.721924in}}%
\pgfpathlineto{\pgfqpoint{1.327048in}{1.722305in}}%
\pgfpathlineto{\pgfqpoint{1.330480in}{1.722635in}}%
\pgfpathlineto{\pgfqpoint{1.333933in}{1.722915in}}%
\pgfpathlineto{\pgfqpoint{1.337403in}{1.723143in}}%
\pgfpathlineto{\pgfqpoint{1.337828in}{1.725119in}}%
\pgfpathlineto{\pgfqpoint{1.338253in}{1.726992in}}%
\pgfpathlineto{\pgfqpoint{1.338678in}{1.728761in}}%
\pgfpathlineto{\pgfqpoint{1.339104in}{1.730425in}}%
\pgfpathlineto{\pgfqpoint{1.336063in}{1.730225in}}%
\pgfpathlineto{\pgfqpoint{1.333036in}{1.729980in}}%
\pgfpathlineto{\pgfqpoint{1.330027in}{1.729691in}}%
\pgfpathlineto{\pgfqpoint{1.327038in}{1.729358in}}%
\pgfpathclose%
\pgfusepath{fill}%
\end{pgfscope}%
\begin{pgfscope}%
\pgfpathrectangle{\pgfqpoint{0.329460in}{0.284240in}}{\pgfqpoint{1.989680in}{1.989680in}}%
\pgfusepath{clip}%
\pgfsetbuttcap%
\pgfsetroundjoin%
\definecolor{currentfill}{rgb}{0.565498,0.842430,0.262877}%
\pgfsetfillcolor{currentfill}%
\pgfsetlinewidth{0.000000pt}%
\definecolor{currentstroke}{rgb}{0.000000,0.000000,0.000000}%
\pgfsetstrokecolor{currentstroke}%
\pgfsetdash{}{0pt}%
\pgfpathmoveto{\pgfqpoint{1.205503in}{1.642547in}}%
\pgfpathlineto{\pgfqpoint{1.202243in}{1.638217in}}%
\pgfpathlineto{\pgfqpoint{1.198986in}{1.633801in}}%
\pgfpathlineto{\pgfqpoint{1.195731in}{1.629301in}}%
\pgfpathlineto{\pgfqpoint{1.192480in}{1.624719in}}%
\pgfpathlineto{\pgfqpoint{1.194224in}{1.627083in}}%
\pgfpathlineto{\pgfqpoint{1.196123in}{1.629418in}}%
\pgfpathlineto{\pgfqpoint{1.198175in}{1.631723in}}%
\pgfpathlineto{\pgfqpoint{1.200376in}{1.633995in}}%
\pgfpathlineto{\pgfqpoint{1.203461in}{1.638380in}}%
\pgfpathlineto{\pgfqpoint{1.206549in}{1.642683in}}%
\pgfpathlineto{\pgfqpoint{1.209640in}{1.646903in}}%
\pgfpathlineto{\pgfqpoint{1.212734in}{1.651037in}}%
\pgfpathlineto{\pgfqpoint{1.210717in}{1.648957in}}%
\pgfpathlineto{\pgfqpoint{1.208839in}{1.646847in}}%
\pgfpathlineto{\pgfqpoint{1.207100in}{1.644710in}}%
\pgfpathlineto{\pgfqpoint{1.205503in}{1.642547in}}%
\pgfpathclose%
\pgfusepath{fill}%
\end{pgfscope}%
\begin{pgfscope}%
\pgfpathrectangle{\pgfqpoint{0.329460in}{0.284240in}}{\pgfqpoint{1.989680in}{1.989680in}}%
\pgfusepath{clip}%
\pgfsetbuttcap%
\pgfsetroundjoin%
\definecolor{currentfill}{rgb}{0.814576,0.883393,0.110347}%
\pgfsetfillcolor{currentfill}%
\pgfsetlinewidth{0.000000pt}%
\definecolor{currentstroke}{rgb}{0.000000,0.000000,0.000000}%
\pgfsetstrokecolor{currentstroke}%
\pgfsetdash{}{0pt}%
\pgfpathmoveto{\pgfqpoint{1.363609in}{1.730405in}}%
\pgfpathlineto{\pgfqpoint{1.364047in}{1.728740in}}%
\pgfpathlineto{\pgfqpoint{1.364485in}{1.726970in}}%
\pgfpathlineto{\pgfqpoint{1.364922in}{1.725097in}}%
\pgfpathlineto{\pgfqpoint{1.365358in}{1.723120in}}%
\pgfpathlineto{\pgfqpoint{1.368826in}{1.722886in}}%
\pgfpathlineto{\pgfqpoint{1.372277in}{1.722601in}}%
\pgfpathlineto{\pgfqpoint{1.375708in}{1.722265in}}%
\pgfpathlineto{\pgfqpoint{1.379114in}{1.721879in}}%
\pgfpathlineto{\pgfqpoint{1.378254in}{1.723894in}}%
\pgfpathlineto{\pgfqpoint{1.377393in}{1.725806in}}%
\pgfpathlineto{\pgfqpoint{1.376531in}{1.727614in}}%
\pgfpathlineto{\pgfqpoint{1.375668in}{1.729318in}}%
\pgfpathlineto{\pgfqpoint{1.372681in}{1.729656in}}%
\pgfpathlineto{\pgfqpoint{1.369675in}{1.729950in}}%
\pgfpathlineto{\pgfqpoint{1.366650in}{1.730200in}}%
\pgfpathlineto{\pgfqpoint{1.363609in}{1.730405in}}%
\pgfpathclose%
\pgfusepath{fill}%
\end{pgfscope}%
\begin{pgfscope}%
\pgfpathrectangle{\pgfqpoint{0.329460in}{0.284240in}}{\pgfqpoint{1.989680in}{1.989680in}}%
\pgfusepath{clip}%
\pgfsetbuttcap%
\pgfsetroundjoin%
\definecolor{currentfill}{rgb}{0.281477,0.755203,0.432552}%
\pgfsetfillcolor{currentfill}%
\pgfsetlinewidth{0.000000pt}%
\definecolor{currentstroke}{rgb}{0.000000,0.000000,0.000000}%
\pgfsetstrokecolor{currentstroke}%
\pgfsetdash{}{0pt}%
\pgfpathmoveto{\pgfqpoint{1.143044in}{1.526500in}}%
\pgfpathlineto{\pgfqpoint{1.139639in}{1.520348in}}%
\pgfpathlineto{\pgfqpoint{1.136236in}{1.514137in}}%
\pgfpathlineto{\pgfqpoint{1.132837in}{1.507869in}}%
\pgfpathlineto{\pgfqpoint{1.129439in}{1.501544in}}%
\pgfpathlineto{\pgfqpoint{1.129189in}{1.504936in}}%
\pgfpathlineto{\pgfqpoint{1.129161in}{1.508328in}}%
\pgfpathlineto{\pgfqpoint{1.129354in}{1.511717in}}%
\pgfpathlineto{\pgfqpoint{1.129767in}{1.515099in}}%
\pgfpathlineto{\pgfqpoint{1.133152in}{1.521202in}}%
\pgfpathlineto{\pgfqpoint{1.136539in}{1.527250in}}%
\pgfpathlineto{\pgfqpoint{1.139930in}{1.533240in}}%
\pgfpathlineto{\pgfqpoint{1.143323in}{1.539172in}}%
\pgfpathlineto{\pgfqpoint{1.142942in}{1.536009in}}%
\pgfpathlineto{\pgfqpoint{1.142769in}{1.532841in}}%
\pgfpathlineto{\pgfqpoint{1.142802in}{1.529670in}}%
\pgfpathlineto{\pgfqpoint{1.143044in}{1.526500in}}%
\pgfpathclose%
\pgfusepath{fill}%
\end{pgfscope}%
\begin{pgfscope}%
\pgfpathrectangle{\pgfqpoint{0.329460in}{0.284240in}}{\pgfqpoint{1.989680in}{1.989680in}}%
\pgfusepath{clip}%
\pgfsetbuttcap%
\pgfsetroundjoin%
\definecolor{currentfill}{rgb}{0.636902,0.856542,0.216620}%
\pgfsetfillcolor{currentfill}%
\pgfsetlinewidth{0.000000pt}%
\definecolor{currentstroke}{rgb}{0.000000,0.000000,0.000000}%
\pgfsetstrokecolor{currentstroke}%
\pgfsetdash{}{0pt}%
\pgfpathmoveto{\pgfqpoint{1.225139in}{1.666691in}}%
\pgfpathlineto{\pgfqpoint{1.222033in}{1.662912in}}%
\pgfpathlineto{\pgfqpoint{1.218931in}{1.659043in}}%
\pgfpathlineto{\pgfqpoint{1.215831in}{1.655084in}}%
\pgfpathlineto{\pgfqpoint{1.212734in}{1.651037in}}%
\pgfpathlineto{\pgfqpoint{1.214886in}{1.653085in}}%
\pgfpathlineto{\pgfqpoint{1.217172in}{1.655099in}}%
\pgfpathlineto{\pgfqpoint{1.219590in}{1.657078in}}%
\pgfpathlineto{\pgfqpoint{1.222136in}{1.659018in}}%
\pgfpathlineto{\pgfqpoint{1.225018in}{1.662882in}}%
\pgfpathlineto{\pgfqpoint{1.227904in}{1.666657in}}%
\pgfpathlineto{\pgfqpoint{1.230793in}{1.670344in}}%
\pgfpathlineto{\pgfqpoint{1.233684in}{1.673939in}}%
\pgfpathlineto{\pgfqpoint{1.231369in}{1.672176in}}%
\pgfpathlineto{\pgfqpoint{1.229172in}{1.670379in}}%
\pgfpathlineto{\pgfqpoint{1.227095in}{1.668550in}}%
\pgfpathlineto{\pgfqpoint{1.225139in}{1.666691in}}%
\pgfpathclose%
\pgfusepath{fill}%
\end{pgfscope}%
\begin{pgfscope}%
\pgfpathrectangle{\pgfqpoint{0.329460in}{0.284240in}}{\pgfqpoint{1.989680in}{1.989680in}}%
\pgfusepath{clip}%
\pgfsetbuttcap%
\pgfsetroundjoin%
\definecolor{currentfill}{rgb}{0.282884,0.135920,0.453427}%
\pgfsetfillcolor{currentfill}%
\pgfsetlinewidth{0.000000pt}%
\definecolor{currentstroke}{rgb}{0.000000,0.000000,0.000000}%
\pgfsetstrokecolor{currentstroke}%
\pgfsetdash{}{0pt}%
\pgfpathmoveto{\pgfqpoint{1.907001in}{0.957791in}}%
\pgfpathlineto{\pgfqpoint{1.910399in}{0.962832in}}%
\pgfpathlineto{\pgfqpoint{1.913811in}{0.968206in}}%
\pgfpathlineto{\pgfqpoint{1.917235in}{0.973917in}}%
\pgfpathlineto{\pgfqpoint{1.920674in}{0.979973in}}%
\pgfpathlineto{\pgfqpoint{1.913167in}{0.970566in}}%
\pgfpathlineto{\pgfqpoint{1.905066in}{0.961274in}}%
\pgfpathlineto{\pgfqpoint{1.896375in}{0.952110in}}%
\pgfpathlineto{\pgfqpoint{1.887103in}{0.943083in}}%
\pgfpathlineto{\pgfqpoint{1.883849in}{0.937218in}}%
\pgfpathlineto{\pgfqpoint{1.880608in}{0.931697in}}%
\pgfpathlineto{\pgfqpoint{1.877381in}{0.926517in}}%
\pgfpathlineto{\pgfqpoint{1.874165in}{0.921670in}}%
\pgfpathlineto{\pgfqpoint{1.883231in}{0.930508in}}%
\pgfpathlineto{\pgfqpoint{1.891730in}{0.939481in}}%
\pgfpathlineto{\pgfqpoint{1.899655in}{0.948579in}}%
\pgfpathlineto{\pgfqpoint{1.907001in}{0.957791in}}%
\pgfpathclose%
\pgfusepath{fill}%
\end{pgfscope}%
\begin{pgfscope}%
\pgfpathrectangle{\pgfqpoint{0.329460in}{0.284240in}}{\pgfqpoint{1.989680in}{1.989680in}}%
\pgfusepath{clip}%
\pgfsetbuttcap%
\pgfsetroundjoin%
\definecolor{currentfill}{rgb}{0.762373,0.876424,0.137064}%
\pgfsetfillcolor{currentfill}%
\pgfsetlinewidth{0.000000pt}%
\definecolor{currentstroke}{rgb}{0.000000,0.000000,0.000000}%
\pgfsetstrokecolor{currentstroke}%
\pgfsetdash{}{0pt}%
\pgfpathmoveto{\pgfqpoint{1.274880in}{1.709412in}}%
\pgfpathlineto{\pgfqpoint{1.272527in}{1.706907in}}%
\pgfpathlineto{\pgfqpoint{1.270176in}{1.704300in}}%
\pgfpathlineto{\pgfqpoint{1.267827in}{1.701595in}}%
\pgfpathlineto{\pgfqpoint{1.265481in}{1.698791in}}%
\pgfpathlineto{\pgfqpoint{1.268411in}{1.700037in}}%
\pgfpathlineto{\pgfqpoint{1.271421in}{1.701239in}}%
\pgfpathlineto{\pgfqpoint{1.274510in}{1.702395in}}%
\pgfpathlineto{\pgfqpoint{1.277672in}{1.703505in}}%
\pgfpathlineto{\pgfqpoint{1.279683in}{1.706177in}}%
\pgfpathlineto{\pgfqpoint{1.281697in}{1.708752in}}%
\pgfpathlineto{\pgfqpoint{1.283712in}{1.711227in}}%
\pgfpathlineto{\pgfqpoint{1.285729in}{1.713601in}}%
\pgfpathlineto{\pgfqpoint{1.282914in}{1.712615in}}%
\pgfpathlineto{\pgfqpoint{1.280166in}{1.711587in}}%
\pgfpathlineto{\pgfqpoint{1.277487in}{1.710519in}}%
\pgfpathlineto{\pgfqpoint{1.274880in}{1.709412in}}%
\pgfpathclose%
\pgfusepath{fill}%
\end{pgfscope}%
\begin{pgfscope}%
\pgfpathrectangle{\pgfqpoint{0.329460in}{0.284240in}}{\pgfqpoint{1.989680in}{1.989680in}}%
\pgfusepath{clip}%
\pgfsetbuttcap%
\pgfsetroundjoin%
\definecolor{currentfill}{rgb}{0.172719,0.448791,0.557885}%
\pgfsetfillcolor{currentfill}%
\pgfsetlinewidth{0.000000pt}%
\definecolor{currentstroke}{rgb}{0.000000,0.000000,0.000000}%
\pgfsetstrokecolor{currentstroke}%
\pgfsetdash{}{0pt}%
\pgfpathmoveto{\pgfqpoint{2.019060in}{1.212761in}}%
\pgfpathlineto{\pgfqpoint{2.022988in}{1.225973in}}%
\pgfpathlineto{\pgfqpoint{2.026938in}{1.239644in}}%
\pgfpathlineto{\pgfqpoint{2.030911in}{1.253784in}}%
\pgfpathlineto{\pgfqpoint{2.028110in}{1.242975in}}%
\pgfpathlineto{\pgfqpoint{2.024616in}{1.232198in}}%
\pgfpathlineto{\pgfqpoint{2.020429in}{1.221464in}}%
\pgfpathlineto{\pgfqpoint{2.015550in}{1.210784in}}%
\pgfpathlineto{\pgfqpoint{2.011641in}{1.196800in}}%
\pgfpathlineto{\pgfqpoint{2.007754in}{1.183288in}}%
\pgfpathlineto{\pgfqpoint{2.003889in}{1.170239in}}%
\pgfpathlineto{\pgfqpoint{2.008702in}{1.180799in}}%
\pgfpathlineto{\pgfqpoint{2.012836in}{1.191414in}}%
\pgfpathlineto{\pgfqpoint{2.016288in}{1.202072in}}%
\pgfpathlineto{\pgfqpoint{2.019060in}{1.212761in}}%
\pgfpathclose%
\pgfusepath{fill}%
\end{pgfscope}%
\begin{pgfscope}%
\pgfpathrectangle{\pgfqpoint{0.329460in}{0.284240in}}{\pgfqpoint{1.989680in}{1.989680in}}%
\pgfusepath{clip}%
\pgfsetbuttcap%
\pgfsetroundjoin%
\definecolor{currentfill}{rgb}{0.699415,0.867117,0.175971}%
\pgfsetfillcolor{currentfill}%
\pgfsetlinewidth{0.000000pt}%
\definecolor{currentstroke}{rgb}{0.000000,0.000000,0.000000}%
\pgfsetstrokecolor{currentstroke}%
\pgfsetdash{}{0pt}%
\pgfpathmoveto{\pgfqpoint{1.445463in}{1.694646in}}%
\pgfpathlineto{\pgfqpoint{1.448043in}{1.691630in}}%
\pgfpathlineto{\pgfqpoint{1.450620in}{1.688518in}}%
\pgfpathlineto{\pgfqpoint{1.453195in}{1.685311in}}%
\pgfpathlineto{\pgfqpoint{1.455768in}{1.682011in}}%
\pgfpathlineto{\pgfqpoint{1.458620in}{1.680439in}}%
\pgfpathlineto{\pgfqpoint{1.461368in}{1.678824in}}%
\pgfpathlineto{\pgfqpoint{1.464008in}{1.677170in}}%
\pgfpathlineto{\pgfqpoint{1.466537in}{1.675476in}}%
\pgfpathlineto{\pgfqpoint{1.463697in}{1.678940in}}%
\pgfpathlineto{\pgfqpoint{1.460854in}{1.682311in}}%
\pgfpathlineto{\pgfqpoint{1.458008in}{1.685588in}}%
\pgfpathlineto{\pgfqpoint{1.455160in}{1.688768in}}%
\pgfpathlineto{\pgfqpoint{1.452882in}{1.690291in}}%
\pgfpathlineto{\pgfqpoint{1.450505in}{1.691780in}}%
\pgfpathlineto{\pgfqpoint{1.448031in}{1.693232in}}%
\pgfpathlineto{\pgfqpoint{1.445463in}{1.694646in}}%
\pgfpathclose%
\pgfusepath{fill}%
\end{pgfscope}%
\begin{pgfscope}%
\pgfpathrectangle{\pgfqpoint{0.329460in}{0.284240in}}{\pgfqpoint{1.989680in}{1.989680in}}%
\pgfusepath{clip}%
\pgfsetbuttcap%
\pgfsetroundjoin%
\definecolor{currentfill}{rgb}{0.487026,0.823929,0.312321}%
\pgfsetfillcolor{currentfill}%
\pgfsetlinewidth{0.000000pt}%
\definecolor{currentstroke}{rgb}{0.000000,0.000000,0.000000}%
\pgfsetstrokecolor{currentstroke}%
\pgfsetdash{}{0pt}%
\pgfpathmoveto{\pgfqpoint{1.187079in}{1.615029in}}%
\pgfpathlineto{\pgfqpoint{1.183715in}{1.610160in}}%
\pgfpathlineto{\pgfqpoint{1.180353in}{1.605210in}}%
\pgfpathlineto{\pgfqpoint{1.176994in}{1.600183in}}%
\pgfpathlineto{\pgfqpoint{1.173638in}{1.595079in}}%
\pgfpathlineto{\pgfqpoint{1.174845in}{1.597742in}}%
\pgfpathlineto{\pgfqpoint{1.176226in}{1.600385in}}%
\pgfpathlineto{\pgfqpoint{1.177780in}{1.603004in}}%
\pgfpathlineto{\pgfqpoint{1.179504in}{1.605597in}}%
\pgfpathlineto{\pgfqpoint{1.182743in}{1.610493in}}%
\pgfpathlineto{\pgfqpoint{1.185986in}{1.615314in}}%
\pgfpathlineto{\pgfqpoint{1.189231in}{1.620056in}}%
\pgfpathlineto{\pgfqpoint{1.192480in}{1.624719in}}%
\pgfpathlineto{\pgfqpoint{1.190891in}{1.622330in}}%
\pgfpathlineto{\pgfqpoint{1.189460in}{1.619917in}}%
\pgfpathlineto{\pgfqpoint{1.188189in}{1.617482in}}%
\pgfpathlineto{\pgfqpoint{1.187079in}{1.615029in}}%
\pgfpathclose%
\pgfusepath{fill}%
\end{pgfscope}%
\begin{pgfscope}%
\pgfpathrectangle{\pgfqpoint{0.329460in}{0.284240in}}{\pgfqpoint{1.989680in}{1.989680in}}%
\pgfusepath{clip}%
\pgfsetbuttcap%
\pgfsetroundjoin%
\definecolor{currentfill}{rgb}{0.412913,0.803041,0.357269}%
\pgfsetfillcolor{currentfill}%
\pgfsetlinewidth{0.000000pt}%
\definecolor{currentstroke}{rgb}{0.000000,0.000000,0.000000}%
\pgfsetstrokecolor{currentstroke}%
\pgfsetdash{}{0pt}%
\pgfpathmoveto{\pgfqpoint{1.527672in}{1.597447in}}%
\pgfpathlineto{\pgfqpoint{1.531004in}{1.592316in}}%
\pgfpathlineto{\pgfqpoint{1.534332in}{1.587111in}}%
\pgfpathlineto{\pgfqpoint{1.537657in}{1.581834in}}%
\pgfpathlineto{\pgfqpoint{1.540979in}{1.576488in}}%
\pgfpathlineto{\pgfqpoint{1.542263in}{1.573612in}}%
\pgfpathlineto{\pgfqpoint{1.543357in}{1.570716in}}%
\pgfpathlineto{\pgfqpoint{1.544262in}{1.567804in}}%
\pgfpathlineto{\pgfqpoint{1.544975in}{1.564878in}}%
\pgfpathlineto{\pgfqpoint{1.541577in}{1.570438in}}%
\pgfpathlineto{\pgfqpoint{1.538175in}{1.575928in}}%
\pgfpathlineto{\pgfqpoint{1.534771in}{1.581346in}}%
\pgfpathlineto{\pgfqpoint{1.531364in}{1.586691in}}%
\pgfpathlineto{\pgfqpoint{1.530707in}{1.589401in}}%
\pgfpathlineto{\pgfqpoint{1.529872in}{1.592099in}}%
\pgfpathlineto{\pgfqpoint{1.528860in}{1.594782in}}%
\pgfpathlineto{\pgfqpoint{1.527672in}{1.597447in}}%
\pgfpathclose%
\pgfusepath{fill}%
\end{pgfscope}%
\begin{pgfscope}%
\pgfpathrectangle{\pgfqpoint{0.329460in}{0.284240in}}{\pgfqpoint{1.989680in}{1.989680in}}%
\pgfusepath{clip}%
\pgfsetbuttcap%
\pgfsetroundjoin%
\definecolor{currentfill}{rgb}{0.814576,0.883393,0.110347}%
\pgfsetfillcolor{currentfill}%
\pgfsetlinewidth{0.000000pt}%
\definecolor{currentstroke}{rgb}{0.000000,0.000000,0.000000}%
\pgfsetstrokecolor{currentstroke}%
\pgfsetdash{}{0pt}%
\pgfpathmoveto{\pgfqpoint{1.375668in}{1.729318in}}%
\pgfpathlineto{\pgfqpoint{1.376531in}{1.727614in}}%
\pgfpathlineto{\pgfqpoint{1.377393in}{1.725806in}}%
\pgfpathlineto{\pgfqpoint{1.378254in}{1.723894in}}%
\pgfpathlineto{\pgfqpoint{1.379114in}{1.721879in}}%
\pgfpathlineto{\pgfqpoint{1.382494in}{1.721443in}}%
\pgfpathlineto{\pgfqpoint{1.385843in}{1.720957in}}%
\pgfpathlineto{\pgfqpoint{1.389159in}{1.720422in}}%
\pgfpathlineto{\pgfqpoint{1.387988in}{1.722482in}}%
\pgfpathlineto{\pgfqpoint{1.386817in}{1.724439in}}%
\pgfpathlineto{\pgfqpoint{1.385645in}{1.726293in}}%
\pgfpathlineto{\pgfqpoint{1.384471in}{1.728042in}}%
\pgfpathlineto{\pgfqpoint{1.381565in}{1.728511in}}%
\pgfpathlineto{\pgfqpoint{1.378630in}{1.728936in}}%
\pgfpathlineto{\pgfqpoint{1.375668in}{1.729318in}}%
\pgfpathclose%
\pgfusepath{fill}%
\end{pgfscope}%
\begin{pgfscope}%
\pgfpathrectangle{\pgfqpoint{0.329460in}{0.284240in}}{\pgfqpoint{1.989680in}{1.989680in}}%
\pgfusepath{clip}%
\pgfsetbuttcap%
\pgfsetroundjoin%
\definecolor{currentfill}{rgb}{0.814576,0.883393,0.110347}%
\pgfsetfillcolor{currentfill}%
\pgfsetlinewidth{0.000000pt}%
\definecolor{currentstroke}{rgb}{0.000000,0.000000,0.000000}%
\pgfsetstrokecolor{currentstroke}%
\pgfsetdash{}{0pt}%
\pgfpathmoveto{\pgfqpoint{1.315348in}{1.727590in}}%
\pgfpathlineto{\pgfqpoint{1.314084in}{1.725825in}}%
\pgfpathlineto{\pgfqpoint{1.312822in}{1.723955in}}%
\pgfpathlineto{\pgfqpoint{1.311560in}{1.721981in}}%
\pgfpathlineto{\pgfqpoint{1.310300in}{1.719905in}}%
\pgfpathlineto{\pgfqpoint{1.313583in}{1.720484in}}%
\pgfpathlineto{\pgfqpoint{1.316903in}{1.721013in}}%
\pgfpathlineto{\pgfqpoint{1.320255in}{1.721494in}}%
\pgfpathlineto{\pgfqpoint{1.323638in}{1.721924in}}%
\pgfpathlineto{\pgfqpoint{1.324487in}{1.723938in}}%
\pgfpathlineto{\pgfqpoint{1.325336in}{1.725848in}}%
\pgfpathlineto{\pgfqpoint{1.326187in}{1.727655in}}%
\pgfpathlineto{\pgfqpoint{1.327038in}{1.729358in}}%
\pgfpathlineto{\pgfqpoint{1.324073in}{1.728981in}}%
\pgfpathlineto{\pgfqpoint{1.321135in}{1.728560in}}%
\pgfpathlineto{\pgfqpoint{1.318225in}{1.728096in}}%
\pgfpathlineto{\pgfqpoint{1.315348in}{1.727590in}}%
\pgfpathclose%
\pgfusepath{fill}%
\end{pgfscope}%
\begin{pgfscope}%
\pgfpathrectangle{\pgfqpoint{0.329460in}{0.284240in}}{\pgfqpoint{1.989680in}{1.989680in}}%
\pgfusepath{clip}%
\pgfsetbuttcap%
\pgfsetroundjoin%
\definecolor{currentfill}{rgb}{0.231674,0.318106,0.544834}%
\pgfsetfillcolor{currentfill}%
\pgfsetlinewidth{0.000000pt}%
\definecolor{currentstroke}{rgb}{0.000000,0.000000,0.000000}%
\pgfsetstrokecolor{currentstroke}%
\pgfsetdash{}{0pt}%
\pgfpathmoveto{\pgfqpoint{1.671213in}{1.116010in}}%
\pgfpathlineto{\pgfqpoint{1.674111in}{1.108698in}}%
\pgfpathlineto{\pgfqpoint{1.677009in}{1.101442in}}%
\pgfpathlineto{\pgfqpoint{1.679907in}{1.094248in}}%
\pgfpathlineto{\pgfqpoint{1.682805in}{1.087116in}}%
\pgfpathlineto{\pgfqpoint{1.676545in}{1.081745in}}%
\pgfpathlineto{\pgfqpoint{1.669944in}{1.076475in}}%
\pgfpathlineto{\pgfqpoint{1.663009in}{1.071311in}}%
\pgfpathlineto{\pgfqpoint{1.655745in}{1.066259in}}%
\pgfpathlineto{\pgfqpoint{1.653075in}{1.073596in}}%
\pgfpathlineto{\pgfqpoint{1.650406in}{1.080996in}}%
\pgfpathlineto{\pgfqpoint{1.647736in}{1.088457in}}%
\pgfpathlineto{\pgfqpoint{1.645066in}{1.095975in}}%
\pgfpathlineto{\pgfqpoint{1.652083in}{1.100827in}}%
\pgfpathlineto{\pgfqpoint{1.658784in}{1.105787in}}%
\pgfpathlineto{\pgfqpoint{1.665162in}{1.110850in}}%
\pgfpathlineto{\pgfqpoint{1.671213in}{1.116010in}}%
\pgfpathclose%
\pgfusepath{fill}%
\end{pgfscope}%
\begin{pgfscope}%
\pgfpathrectangle{\pgfqpoint{0.329460in}{0.284240in}}{\pgfqpoint{1.989680in}{1.989680in}}%
\pgfusepath{clip}%
\pgfsetbuttcap%
\pgfsetroundjoin%
\definecolor{currentfill}{rgb}{0.248629,0.278775,0.534556}%
\pgfsetfillcolor{currentfill}%
\pgfsetlinewidth{0.000000pt}%
\definecolor{currentstroke}{rgb}{0.000000,0.000000,0.000000}%
\pgfsetstrokecolor{currentstroke}%
\pgfsetdash{}{0pt}%
\pgfpathmoveto{\pgfqpoint{1.053356in}{1.061867in}}%
\pgfpathlineto{\pgfqpoint{1.050743in}{1.054553in}}%
\pgfpathlineto{\pgfqpoint{1.048129in}{1.047308in}}%
\pgfpathlineto{\pgfqpoint{1.045516in}{1.040135in}}%
\pgfpathlineto{\pgfqpoint{1.042902in}{1.033038in}}%
\pgfpathlineto{\pgfqpoint{1.035096in}{1.038181in}}%
\pgfpathlineto{\pgfqpoint{1.027625in}{1.043447in}}%
\pgfpathlineto{\pgfqpoint{1.020494in}{1.048827in}}%
\pgfpathlineto{\pgfqpoint{1.013710in}{1.054318in}}%
\pgfpathlineto{\pgfqpoint{1.016563in}{1.061212in}}%
\pgfpathlineto{\pgfqpoint{1.019415in}{1.068182in}}%
\pgfpathlineto{\pgfqpoint{1.022266in}{1.075225in}}%
\pgfpathlineto{\pgfqpoint{1.025118in}{1.082337in}}%
\pgfpathlineto{\pgfqpoint{1.031681in}{1.077055in}}%
\pgfpathlineto{\pgfqpoint{1.038580in}{1.071879in}}%
\pgfpathlineto{\pgfqpoint{1.045807in}{1.066815in}}%
\pgfpathlineto{\pgfqpoint{1.053356in}{1.061867in}}%
\pgfpathclose%
\pgfusepath{fill}%
\end{pgfscope}%
\begin{pgfscope}%
\pgfpathrectangle{\pgfqpoint{0.329460in}{0.284240in}}{\pgfqpoint{1.989680in}{1.989680in}}%
\pgfusepath{clip}%
\pgfsetbuttcap%
\pgfsetroundjoin%
\definecolor{currentfill}{rgb}{0.699415,0.867117,0.175971}%
\pgfsetfillcolor{currentfill}%
\pgfsetlinewidth{0.000000pt}%
\definecolor{currentstroke}{rgb}{0.000000,0.000000,0.000000}%
\pgfsetstrokecolor{currentstroke}%
\pgfsetdash{}{0pt}%
\pgfpathmoveto{\pgfqpoint{1.245277in}{1.687386in}}%
\pgfpathlineto{\pgfqpoint{1.242374in}{1.684167in}}%
\pgfpathlineto{\pgfqpoint{1.239475in}{1.680852in}}%
\pgfpathlineto{\pgfqpoint{1.236578in}{1.677442in}}%
\pgfpathlineto{\pgfqpoint{1.233684in}{1.673939in}}%
\pgfpathlineto{\pgfqpoint{1.236113in}{1.675666in}}%
\pgfpathlineto{\pgfqpoint{1.238655in}{1.677355in}}%
\pgfpathlineto{\pgfqpoint{1.241307in}{1.679006in}}%
\pgfpathlineto{\pgfqpoint{1.244067in}{1.680616in}}%
\pgfpathlineto{\pgfqpoint{1.246702in}{1.683951in}}%
\pgfpathlineto{\pgfqpoint{1.249340in}{1.687193in}}%
\pgfpathlineto{\pgfqpoint{1.251981in}{1.690340in}}%
\pgfpathlineto{\pgfqpoint{1.254624in}{1.693391in}}%
\pgfpathlineto{\pgfqpoint{1.252140in}{1.691943in}}%
\pgfpathlineto{\pgfqpoint{1.249752in}{1.690459in}}%
\pgfpathlineto{\pgfqpoint{1.247464in}{1.688939in}}%
\pgfpathlineto{\pgfqpoint{1.245277in}{1.687386in}}%
\pgfpathclose%
\pgfusepath{fill}%
\end{pgfscope}%
\begin{pgfscope}%
\pgfpathrectangle{\pgfqpoint{0.329460in}{0.284240in}}{\pgfqpoint{1.989680in}{1.989680in}}%
\pgfusepath{clip}%
\pgfsetbuttcap%
\pgfsetroundjoin%
\definecolor{currentfill}{rgb}{0.179019,0.433756,0.557430}%
\pgfsetfillcolor{currentfill}%
\pgfsetlinewidth{0.000000pt}%
\definecolor{currentstroke}{rgb}{0.000000,0.000000,0.000000}%
\pgfsetstrokecolor{currentstroke}%
\pgfsetdash{}{0pt}%
\pgfpathmoveto{\pgfqpoint{1.070784in}{1.203131in}}%
\pgfpathlineto{\pgfqpoint{1.067925in}{1.195312in}}%
\pgfpathlineto{\pgfqpoint{1.065067in}{1.187515in}}%
\pgfpathlineto{\pgfqpoint{1.062209in}{1.179744in}}%
\pgfpathlineto{\pgfqpoint{1.059353in}{1.172001in}}%
\pgfpathlineto{\pgfqpoint{1.053758in}{1.176751in}}%
\pgfpathlineto{\pgfqpoint{1.048472in}{1.181586in}}%
\pgfpathlineto{\pgfqpoint{1.043500in}{1.186499in}}%
\pgfpathlineto{\pgfqpoint{1.038846in}{1.191486in}}%
\pgfpathlineto{\pgfqpoint{1.041893in}{1.199015in}}%
\pgfpathlineto{\pgfqpoint{1.044942in}{1.206573in}}%
\pgfpathlineto{\pgfqpoint{1.047992in}{1.214156in}}%
\pgfpathlineto{\pgfqpoint{1.051044in}{1.221763in}}%
\pgfpathlineto{\pgfqpoint{1.055525in}{1.216994in}}%
\pgfpathlineto{\pgfqpoint{1.060312in}{1.212296in}}%
\pgfpathlineto{\pgfqpoint{1.065400in}{1.207673in}}%
\pgfpathlineto{\pgfqpoint{1.070784in}{1.203131in}}%
\pgfpathclose%
\pgfusepath{fill}%
\end{pgfscope}%
\begin{pgfscope}%
\pgfpathrectangle{\pgfqpoint{0.329460in}{0.284240in}}{\pgfqpoint{1.989680in}{1.989680in}}%
\pgfusepath{clip}%
\pgfsetbuttcap%
\pgfsetroundjoin%
\definecolor{currentfill}{rgb}{0.133743,0.548535,0.553541}%
\pgfsetfillcolor{currentfill}%
\pgfsetlinewidth{0.000000pt}%
\definecolor{currentstroke}{rgb}{0.000000,0.000000,0.000000}%
\pgfsetstrokecolor{currentstroke}%
\pgfsetdash{}{0pt}%
\pgfpathmoveto{\pgfqpoint{1.087770in}{1.313877in}}%
\pgfpathlineto{\pgfqpoint{1.084700in}{1.306200in}}%
\pgfpathlineto{\pgfqpoint{1.081633in}{1.298514in}}%
\pgfpathlineto{\pgfqpoint{1.078567in}{1.290822in}}%
\pgfpathlineto{\pgfqpoint{1.075503in}{1.283127in}}%
\pgfpathlineto{\pgfqpoint{1.071653in}{1.287525in}}%
\pgfpathlineto{\pgfqpoint{1.068090in}{1.291978in}}%
\pgfpathlineto{\pgfqpoint{1.064817in}{1.296482in}}%
\pgfpathlineto{\pgfqpoint{1.061836in}{1.301032in}}%
\pgfpathlineto{\pgfqpoint{1.065043in}{1.308506in}}%
\pgfpathlineto{\pgfqpoint{1.068252in}{1.315978in}}%
\pgfpathlineto{\pgfqpoint{1.071463in}{1.323444in}}%
\pgfpathlineto{\pgfqpoint{1.074676in}{1.330903in}}%
\pgfpathlineto{\pgfqpoint{1.077533in}{1.326576in}}%
\pgfpathlineto{\pgfqpoint{1.080670in}{1.322293in}}%
\pgfpathlineto{\pgfqpoint{1.084083in}{1.318059in}}%
\pgfpathlineto{\pgfqpoint{1.087770in}{1.313877in}}%
\pgfpathclose%
\pgfusepath{fill}%
\end{pgfscope}%
\begin{pgfscope}%
\pgfpathrectangle{\pgfqpoint{0.329460in}{0.284240in}}{\pgfqpoint{1.989680in}{1.989680in}}%
\pgfusepath{clip}%
\pgfsetbuttcap%
\pgfsetroundjoin%
\definecolor{currentfill}{rgb}{0.814576,0.883393,0.110347}%
\pgfsetfillcolor{currentfill}%
\pgfsetlinewidth{0.000000pt}%
\definecolor{currentstroke}{rgb}{0.000000,0.000000,0.000000}%
\pgfsetstrokecolor{currentstroke}%
\pgfsetdash{}{0pt}%
\pgfpathmoveto{\pgfqpoint{1.384471in}{1.728042in}}%
\pgfpathlineto{\pgfqpoint{1.385645in}{1.726293in}}%
\pgfpathlineto{\pgfqpoint{1.386817in}{1.724439in}}%
\pgfpathlineto{\pgfqpoint{1.387988in}{1.722482in}}%
\pgfpathlineto{\pgfqpoint{1.389159in}{1.720422in}}%
\pgfpathlineto{\pgfqpoint{1.392437in}{1.719838in}}%
\pgfpathlineto{\pgfqpoint{1.395676in}{1.719206in}}%
\pgfpathlineto{\pgfqpoint{1.398871in}{1.718527in}}%
\pgfpathlineto{\pgfqpoint{1.402020in}{1.717801in}}%
\pgfpathlineto{\pgfqpoint{1.400453in}{1.719943in}}%
\pgfpathlineto{\pgfqpoint{1.398884in}{1.721982in}}%
\pgfpathlineto{\pgfqpoint{1.397314in}{1.723917in}}%
\pgfpathlineto{\pgfqpoint{1.395742in}{1.725748in}}%
\pgfpathlineto{\pgfqpoint{1.392983in}{1.726384in}}%
\pgfpathlineto{\pgfqpoint{1.390183in}{1.726978in}}%
\pgfpathlineto{\pgfqpoint{1.387344in}{1.727531in}}%
\pgfpathlineto{\pgfqpoint{1.384471in}{1.728042in}}%
\pgfpathclose%
\pgfusepath{fill}%
\end{pgfscope}%
\begin{pgfscope}%
\pgfpathrectangle{\pgfqpoint{0.329460in}{0.284240in}}{\pgfqpoint{1.989680in}{1.989680in}}%
\pgfusepath{clip}%
\pgfsetbuttcap%
\pgfsetroundjoin%
\definecolor{currentfill}{rgb}{0.220124,0.725509,0.466226}%
\pgfsetfillcolor{currentfill}%
\pgfsetlinewidth{0.000000pt}%
\definecolor{currentstroke}{rgb}{0.000000,0.000000,0.000000}%
\pgfsetstrokecolor{currentstroke}%
\pgfsetdash{}{0pt}%
\pgfpathmoveto{\pgfqpoint{1.573169in}{1.504559in}}%
\pgfpathlineto{\pgfqpoint{1.576565in}{1.498230in}}%
\pgfpathlineto{\pgfqpoint{1.579958in}{1.491850in}}%
\pgfpathlineto{\pgfqpoint{1.583349in}{1.485420in}}%
\pgfpathlineto{\pgfqpoint{1.586736in}{1.478943in}}%
\pgfpathlineto{\pgfqpoint{1.586453in}{1.475329in}}%
\pgfpathlineto{\pgfqpoint{1.585934in}{1.471719in}}%
\pgfpathlineto{\pgfqpoint{1.585180in}{1.468116in}}%
\pgfpathlineto{\pgfqpoint{1.584192in}{1.464525in}}%
\pgfpathlineto{\pgfqpoint{1.580833in}{1.471225in}}%
\pgfpathlineto{\pgfqpoint{1.577471in}{1.477878in}}%
\pgfpathlineto{\pgfqpoint{1.574107in}{1.484480in}}%
\pgfpathlineto{\pgfqpoint{1.570741in}{1.491031in}}%
\pgfpathlineto{\pgfqpoint{1.571680in}{1.494400in}}%
\pgfpathlineto{\pgfqpoint{1.572398in}{1.497780in}}%
\pgfpathlineto{\pgfqpoint{1.572894in}{1.501168in}}%
\pgfpathlineto{\pgfqpoint{1.573169in}{1.504559in}}%
\pgfpathclose%
\pgfusepath{fill}%
\end{pgfscope}%
\begin{pgfscope}%
\pgfpathrectangle{\pgfqpoint{0.329460in}{0.284240in}}{\pgfqpoint{1.989680in}{1.989680in}}%
\pgfusepath{clip}%
\pgfsetbuttcap%
\pgfsetroundjoin%
\definecolor{currentfill}{rgb}{0.268510,0.009605,0.335427}%
\pgfsetfillcolor{currentfill}%
\pgfsetlinewidth{0.000000pt}%
\definecolor{currentstroke}{rgb}{0.000000,0.000000,0.000000}%
\pgfsetstrokecolor{currentstroke}%
\pgfsetdash{}{0pt}%
\pgfpathmoveto{\pgfqpoint{0.925681in}{0.847877in}}%
\pgfpathlineto{\pgfqpoint{0.922916in}{0.847870in}}%
\pgfpathlineto{\pgfqpoint{0.920144in}{0.848115in}}%
\pgfpathlineto{\pgfqpoint{0.917364in}{0.848616in}}%
\pgfpathlineto{\pgfqpoint{0.914576in}{0.849378in}}%
\pgfpathlineto{\pgfqpoint{0.903670in}{0.856867in}}%
\pgfpathlineto{\pgfqpoint{0.893246in}{0.864527in}}%
\pgfpathlineto{\pgfqpoint{0.883316in}{0.872351in}}%
\pgfpathlineto{\pgfqpoint{0.873887in}{0.880328in}}%
\pgfpathlineto{\pgfqpoint{0.876920in}{0.879373in}}%
\pgfpathlineto{\pgfqpoint{0.879945in}{0.878678in}}%
\pgfpathlineto{\pgfqpoint{0.882962in}{0.878239in}}%
\pgfpathlineto{\pgfqpoint{0.885971in}{0.878050in}}%
\pgfpathlineto{\pgfqpoint{0.895176in}{0.870272in}}%
\pgfpathlineto{\pgfqpoint{0.904868in}{0.862645in}}%
\pgfpathlineto{\pgfqpoint{0.915040in}{0.855177in}}%
\pgfpathlineto{\pgfqpoint{0.925681in}{0.847877in}}%
\pgfpathclose%
\pgfusepath{fill}%
\end{pgfscope}%
\begin{pgfscope}%
\pgfpathrectangle{\pgfqpoint{0.329460in}{0.284240in}}{\pgfqpoint{1.989680in}{1.989680in}}%
\pgfusepath{clip}%
\pgfsetbuttcap%
\pgfsetroundjoin%
\definecolor{currentfill}{rgb}{0.134692,0.658636,0.517649}%
\pgfsetfillcolor{currentfill}%
\pgfsetlinewidth{0.000000pt}%
\definecolor{currentstroke}{rgb}{0.000000,0.000000,0.000000}%
\pgfsetstrokecolor{currentstroke}%
\pgfsetdash{}{0pt}%
\pgfpathmoveto{\pgfqpoint{1.597601in}{1.437284in}}%
\pgfpathlineto{\pgfqpoint{1.600947in}{1.430375in}}%
\pgfpathlineto{\pgfqpoint{1.604290in}{1.423431in}}%
\pgfpathlineto{\pgfqpoint{1.607630in}{1.416453in}}%
\pgfpathlineto{\pgfqpoint{1.610968in}{1.409445in}}%
\pgfpathlineto{\pgfqpoint{1.609622in}{1.405421in}}%
\pgfpathlineto{\pgfqpoint{1.608016in}{1.401418in}}%
\pgfpathlineto{\pgfqpoint{1.606150in}{1.397438in}}%
\pgfpathlineto{\pgfqpoint{1.604025in}{1.393487in}}%
\pgfpathlineto{\pgfqpoint{1.600768in}{1.400719in}}%
\pgfpathlineto{\pgfqpoint{1.597508in}{1.407920in}}%
\pgfpathlineto{\pgfqpoint{1.594247in}{1.415087in}}%
\pgfpathlineto{\pgfqpoint{1.590982in}{1.422219in}}%
\pgfpathlineto{\pgfqpoint{1.593005in}{1.425948in}}%
\pgfpathlineto{\pgfqpoint{1.594783in}{1.429705in}}%
\pgfpathlineto{\pgfqpoint{1.596316in}{1.433485in}}%
\pgfpathlineto{\pgfqpoint{1.597601in}{1.437284in}}%
\pgfpathclose%
\pgfusepath{fill}%
\end{pgfscope}%
\begin{pgfscope}%
\pgfpathrectangle{\pgfqpoint{0.329460in}{0.284240in}}{\pgfqpoint{1.989680in}{1.989680in}}%
\pgfusepath{clip}%
\pgfsetbuttcap%
\pgfsetroundjoin%
\definecolor{currentfill}{rgb}{0.412913,0.803041,0.357269}%
\pgfsetfillcolor{currentfill}%
\pgfsetlinewidth{0.000000pt}%
\definecolor{currentstroke}{rgb}{0.000000,0.000000,0.000000}%
\pgfsetstrokecolor{currentstroke}%
\pgfsetdash{}{0pt}%
\pgfpathmoveto{\pgfqpoint{1.170578in}{1.584273in}}%
\pgfpathlineto{\pgfqpoint{1.167161in}{1.578881in}}%
\pgfpathlineto{\pgfqpoint{1.163747in}{1.573414in}}%
\pgfpathlineto{\pgfqpoint{1.160335in}{1.567876in}}%
\pgfpathlineto{\pgfqpoint{1.156927in}{1.562268in}}%
\pgfpathlineto{\pgfqpoint{1.157470in}{1.565204in}}%
\pgfpathlineto{\pgfqpoint{1.158204in}{1.568129in}}%
\pgfpathlineto{\pgfqpoint{1.159130in}{1.571039in}}%
\pgfpathlineto{\pgfqpoint{1.160246in}{1.573933in}}%
\pgfpathlineto{\pgfqpoint{1.163589in}{1.579326in}}%
\pgfpathlineto{\pgfqpoint{1.166936in}{1.584649in}}%
\pgfpathlineto{\pgfqpoint{1.170286in}{1.589901in}}%
\pgfpathlineto{\pgfqpoint{1.173638in}{1.595079in}}%
\pgfpathlineto{\pgfqpoint{1.172607in}{1.592398in}}%
\pgfpathlineto{\pgfqpoint{1.171752in}{1.589702in}}%
\pgfpathlineto{\pgfqpoint{1.171076in}{1.586993in}}%
\pgfpathlineto{\pgfqpoint{1.170578in}{1.584273in}}%
\pgfpathclose%
\pgfusepath{fill}%
\end{pgfscope}%
\begin{pgfscope}%
\pgfpathrectangle{\pgfqpoint{0.329460in}{0.284240in}}{\pgfqpoint{1.989680in}{1.989680in}}%
\pgfusepath{clip}%
\pgfsetbuttcap%
\pgfsetroundjoin%
\definecolor{currentfill}{rgb}{0.272594,0.025563,0.353093}%
\pgfsetfillcolor{currentfill}%
\pgfsetlinewidth{0.000000pt}%
\definecolor{currentstroke}{rgb}{0.000000,0.000000,0.000000}%
\pgfsetstrokecolor{currentstroke}%
\pgfsetdash{}{0pt}%
\pgfpathmoveto{\pgfqpoint{1.836442in}{0.887542in}}%
\pgfpathlineto{\pgfqpoint{1.839532in}{0.888807in}}%
\pgfpathlineto{\pgfqpoint{1.842630in}{0.890342in}}%
\pgfpathlineto{\pgfqpoint{1.845737in}{0.892153in}}%
\pgfpathlineto{\pgfqpoint{1.848854in}{0.894243in}}%
\pgfpathlineto{\pgfqpoint{1.839664in}{0.885937in}}%
\pgfpathlineto{\pgfqpoint{1.829950in}{0.877780in}}%
\pgfpathlineto{\pgfqpoint{1.819722in}{0.869781in}}%
\pgfpathlineto{\pgfqpoint{1.808989in}{0.861951in}}%
\pgfpathlineto{\pgfqpoint{1.806108in}{0.860053in}}%
\pgfpathlineto{\pgfqpoint{1.803236in}{0.858437in}}%
\pgfpathlineto{\pgfqpoint{1.800372in}{0.857096in}}%
\pgfpathlineto{\pgfqpoint{1.797517in}{0.856026in}}%
\pgfpathlineto{\pgfqpoint{1.807995in}{0.863668in}}%
\pgfpathlineto{\pgfqpoint{1.817981in}{0.871474in}}%
\pgfpathlineto{\pgfqpoint{1.827466in}{0.879435in}}%
\pgfpathlineto{\pgfqpoint{1.836442in}{0.887542in}}%
\pgfpathclose%
\pgfusepath{fill}%
\end{pgfscope}%
\begin{pgfscope}%
\pgfpathrectangle{\pgfqpoint{0.329460in}{0.284240in}}{\pgfqpoint{1.989680in}{1.989680in}}%
\pgfusepath{clip}%
\pgfsetbuttcap%
\pgfsetroundjoin%
\definecolor{currentfill}{rgb}{0.814576,0.883393,0.110347}%
\pgfsetfillcolor{currentfill}%
\pgfsetlinewidth{0.000000pt}%
\definecolor{currentstroke}{rgb}{0.000000,0.000000,0.000000}%
\pgfsetstrokecolor{currentstroke}%
\pgfsetdash{}{0pt}%
\pgfpathmoveto{\pgfqpoint{1.304217in}{1.725149in}}%
\pgfpathlineto{\pgfqpoint{1.302560in}{1.723297in}}%
\pgfpathlineto{\pgfqpoint{1.300904in}{1.721341in}}%
\pgfpathlineto{\pgfqpoint{1.299250in}{1.719280in}}%
\pgfpathlineto{\pgfqpoint{1.297598in}{1.717117in}}%
\pgfpathlineto{\pgfqpoint{1.300703in}{1.717884in}}%
\pgfpathlineto{\pgfqpoint{1.303857in}{1.718605in}}%
\pgfpathlineto{\pgfqpoint{1.307057in}{1.719279in}}%
\pgfpathlineto{\pgfqpoint{1.310300in}{1.719905in}}%
\pgfpathlineto{\pgfqpoint{1.311560in}{1.721981in}}%
\pgfpathlineto{\pgfqpoint{1.312822in}{1.723955in}}%
\pgfpathlineto{\pgfqpoint{1.314084in}{1.725825in}}%
\pgfpathlineto{\pgfqpoint{1.315348in}{1.727590in}}%
\pgfpathlineto{\pgfqpoint{1.312506in}{1.727042in}}%
\pgfpathlineto{\pgfqpoint{1.309702in}{1.726452in}}%
\pgfpathlineto{\pgfqpoint{1.306938in}{1.725821in}}%
\pgfpathlineto{\pgfqpoint{1.304217in}{1.725149in}}%
\pgfpathclose%
\pgfusepath{fill}%
\end{pgfscope}%
\begin{pgfscope}%
\pgfpathrectangle{\pgfqpoint{0.329460in}{0.284240in}}{\pgfqpoint{1.989680in}{1.989680in}}%
\pgfusepath{clip}%
\pgfsetbuttcap%
\pgfsetroundjoin%
\definecolor{currentfill}{rgb}{0.163625,0.471133,0.558148}%
\pgfsetfillcolor{currentfill}%
\pgfsetlinewidth{0.000000pt}%
\definecolor{currentstroke}{rgb}{0.000000,0.000000,0.000000}%
\pgfsetstrokecolor{currentstroke}%
\pgfsetdash{}{0pt}%
\pgfpathmoveto{\pgfqpoint{1.642694in}{1.256467in}}%
\pgfpathlineto{\pgfqpoint{1.645786in}{1.248843in}}%
\pgfpathlineto{\pgfqpoint{1.648877in}{1.241232in}}%
\pgfpathlineto{\pgfqpoint{1.651967in}{1.233636in}}%
\pgfpathlineto{\pgfqpoint{1.655055in}{1.226057in}}%
\pgfpathlineto{\pgfqpoint{1.650849in}{1.221230in}}%
\pgfpathlineto{\pgfqpoint{1.646333in}{1.216469in}}%
\pgfpathlineto{\pgfqpoint{1.641513in}{1.211778in}}%
\pgfpathlineto{\pgfqpoint{1.636392in}{1.207165in}}%
\pgfpathlineto{\pgfqpoint{1.633485in}{1.214958in}}%
\pgfpathlineto{\pgfqpoint{1.630576in}{1.222769in}}%
\pgfpathlineto{\pgfqpoint{1.627667in}{1.230595in}}%
\pgfpathlineto{\pgfqpoint{1.624756in}{1.238432in}}%
\pgfpathlineto{\pgfqpoint{1.629677in}{1.242836in}}%
\pgfpathlineto{\pgfqpoint{1.634309in}{1.247313in}}%
\pgfpathlineto{\pgfqpoint{1.638649in}{1.251858in}}%
\pgfpathlineto{\pgfqpoint{1.642694in}{1.256467in}}%
\pgfpathclose%
\pgfusepath{fill}%
\end{pgfscope}%
\begin{pgfscope}%
\pgfpathrectangle{\pgfqpoint{0.329460in}{0.284240in}}{\pgfqpoint{1.989680in}{1.989680in}}%
\pgfusepath{clip}%
\pgfsetbuttcap%
\pgfsetroundjoin%
\definecolor{currentfill}{rgb}{0.762373,0.876424,0.137064}%
\pgfsetfillcolor{currentfill}%
\pgfsetlinewidth{0.000000pt}%
\definecolor{currentstroke}{rgb}{0.000000,0.000000,0.000000}%
\pgfsetstrokecolor{currentstroke}%
\pgfsetdash{}{0pt}%
\pgfpathmoveto{\pgfqpoint{1.425181in}{1.710398in}}%
\pgfpathlineto{\pgfqpoint{1.427463in}{1.707923in}}%
\pgfpathlineto{\pgfqpoint{1.429742in}{1.705348in}}%
\pgfpathlineto{\pgfqpoint{1.432019in}{1.702674in}}%
\pgfpathlineto{\pgfqpoint{1.434294in}{1.699901in}}%
\pgfpathlineto{\pgfqpoint{1.437215in}{1.698650in}}%
\pgfpathlineto{\pgfqpoint{1.440052in}{1.697357in}}%
\pgfpathlineto{\pgfqpoint{1.442802in}{1.696022in}}%
\pgfpathlineto{\pgfqpoint{1.445463in}{1.694646in}}%
\pgfpathlineto{\pgfqpoint{1.442880in}{1.697566in}}%
\pgfpathlineto{\pgfqpoint{1.440295in}{1.700386in}}%
\pgfpathlineto{\pgfqpoint{1.437707in}{1.703108in}}%
\pgfpathlineto{\pgfqpoint{1.435117in}{1.705729in}}%
\pgfpathlineto{\pgfqpoint{1.432750in}{1.706951in}}%
\pgfpathlineto{\pgfqpoint{1.430304in}{1.708137in}}%
\pgfpathlineto{\pgfqpoint{1.427780in}{1.709287in}}%
\pgfpathlineto{\pgfqpoint{1.425181in}{1.710398in}}%
\pgfpathclose%
\pgfusepath{fill}%
\end{pgfscope}%
\begin{pgfscope}%
\pgfpathrectangle{\pgfqpoint{0.329460in}{0.284240in}}{\pgfqpoint{1.989680in}{1.989680in}}%
\pgfusepath{clip}%
\pgfsetbuttcap%
\pgfsetroundjoin%
\definecolor{currentfill}{rgb}{0.344074,0.780029,0.397381}%
\pgfsetfillcolor{currentfill}%
\pgfsetlinewidth{0.000000pt}%
\definecolor{currentstroke}{rgb}{0.000000,0.000000,0.000000}%
\pgfsetstrokecolor{currentstroke}%
\pgfsetdash{}{0pt}%
\pgfpathmoveto{\pgfqpoint{1.544975in}{1.564878in}}%
\pgfpathlineto{\pgfqpoint{1.548371in}{1.559250in}}%
\pgfpathlineto{\pgfqpoint{1.551763in}{1.553556in}}%
\pgfpathlineto{\pgfqpoint{1.555153in}{1.547797in}}%
\pgfpathlineto{\pgfqpoint{1.558540in}{1.541975in}}%
\pgfpathlineto{\pgfqpoint{1.559104in}{1.538821in}}%
\pgfpathlineto{\pgfqpoint{1.559462in}{1.535657in}}%
\pgfpathlineto{\pgfqpoint{1.559613in}{1.532489in}}%
\pgfpathlineto{\pgfqpoint{1.559556in}{1.529318in}}%
\pgfpathlineto{\pgfqpoint{1.556146in}{1.535359in}}%
\pgfpathlineto{\pgfqpoint{1.552733in}{1.541337in}}%
\pgfpathlineto{\pgfqpoint{1.549317in}{1.547250in}}%
\pgfpathlineto{\pgfqpoint{1.545898in}{1.553096in}}%
\pgfpathlineto{\pgfqpoint{1.545958in}{1.556048in}}%
\pgfpathlineto{\pgfqpoint{1.545824in}{1.558997in}}%
\pgfpathlineto{\pgfqpoint{1.545496in}{1.561942in}}%
\pgfpathlineto{\pgfqpoint{1.544975in}{1.564878in}}%
\pgfpathclose%
\pgfusepath{fill}%
\end{pgfscope}%
\begin{pgfscope}%
\pgfpathrectangle{\pgfqpoint{0.329460in}{0.284240in}}{\pgfqpoint{1.989680in}{1.989680in}}%
\pgfusepath{clip}%
\pgfsetbuttcap%
\pgfsetroundjoin%
\definecolor{currentfill}{rgb}{0.279566,0.067836,0.391917}%
\pgfsetfillcolor{currentfill}%
\pgfsetlinewidth{0.000000pt}%
\definecolor{currentstroke}{rgb}{0.000000,0.000000,0.000000}%
\pgfsetstrokecolor{currentstroke}%
\pgfsetdash{}{0pt}%
\pgfpathmoveto{\pgfqpoint{1.720101in}{0.917455in}}%
\pgfpathlineto{\pgfqpoint{1.722806in}{0.912757in}}%
\pgfpathlineto{\pgfqpoint{1.725514in}{0.908210in}}%
\pgfpathlineto{\pgfqpoint{1.728225in}{0.903818in}}%
\pgfpathlineto{\pgfqpoint{1.730939in}{0.899585in}}%
\pgfpathlineto{\pgfqpoint{1.721553in}{0.893273in}}%
\pgfpathlineto{\pgfqpoint{1.711769in}{0.887116in}}%
\pgfpathlineto{\pgfqpoint{1.701599in}{0.881122in}}%
\pgfpathlineto{\pgfqpoint{1.691051in}{0.875298in}}%
\pgfpathlineto{\pgfqpoint{1.688613in}{0.879721in}}%
\pgfpathlineto{\pgfqpoint{1.686178in}{0.884302in}}%
\pgfpathlineto{\pgfqpoint{1.683746in}{0.889039in}}%
\pgfpathlineto{\pgfqpoint{1.681317in}{0.893927in}}%
\pgfpathlineto{\pgfqpoint{1.691571in}{0.899569in}}%
\pgfpathlineto{\pgfqpoint{1.701459in}{0.905375in}}%
\pgfpathlineto{\pgfqpoint{1.710973in}{0.911339in}}%
\pgfpathlineto{\pgfqpoint{1.720101in}{0.917455in}}%
\pgfpathclose%
\pgfusepath{fill}%
\end{pgfscope}%
\begin{pgfscope}%
\pgfpathrectangle{\pgfqpoint{0.329460in}{0.284240in}}{\pgfqpoint{1.989680in}{1.989680in}}%
\pgfusepath{clip}%
\pgfsetbuttcap%
\pgfsetroundjoin%
\definecolor{currentfill}{rgb}{0.122606,0.585371,0.546557}%
\pgfsetfillcolor{currentfill}%
\pgfsetlinewidth{0.000000pt}%
\definecolor{currentstroke}{rgb}{0.000000,0.000000,0.000000}%
\pgfsetstrokecolor{currentstroke}%
\pgfsetdash{}{0pt}%
\pgfpathmoveto{\pgfqpoint{1.617032in}{1.364293in}}%
\pgfpathlineto{\pgfqpoint{1.620278in}{1.356939in}}%
\pgfpathlineto{\pgfqpoint{1.623522in}{1.349568in}}%
\pgfpathlineto{\pgfqpoint{1.626763in}{1.342182in}}%
\pgfpathlineto{\pgfqpoint{1.630003in}{1.334783in}}%
\pgfpathlineto{\pgfqpoint{1.627396in}{1.330420in}}%
\pgfpathlineto{\pgfqpoint{1.624507in}{1.326098in}}%
\pgfpathlineto{\pgfqpoint{1.621340in}{1.321820in}}%
\pgfpathlineto{\pgfqpoint{1.617896in}{1.317591in}}%
\pgfpathlineto{\pgfqpoint{1.614789in}{1.325211in}}%
\pgfpathlineto{\pgfqpoint{1.611679in}{1.332817in}}%
\pgfpathlineto{\pgfqpoint{1.608568in}{1.340408in}}%
\pgfpathlineto{\pgfqpoint{1.605455in}{1.347981in}}%
\pgfpathlineto{\pgfqpoint{1.608746in}{1.351993in}}%
\pgfpathlineto{\pgfqpoint{1.611774in}{1.356052in}}%
\pgfpathlineto{\pgfqpoint{1.614537in}{1.360153in}}%
\pgfpathlineto{\pgfqpoint{1.617032in}{1.364293in}}%
\pgfpathclose%
\pgfusepath{fill}%
\end{pgfscope}%
\begin{pgfscope}%
\pgfpathrectangle{\pgfqpoint{0.329460in}{0.284240in}}{\pgfqpoint{1.989680in}{1.989680in}}%
\pgfusepath{clip}%
\pgfsetbuttcap%
\pgfsetroundjoin%
\definecolor{currentfill}{rgb}{0.282327,0.094955,0.417331}%
\pgfsetfillcolor{currentfill}%
\pgfsetlinewidth{0.000000pt}%
\definecolor{currentstroke}{rgb}{0.000000,0.000000,0.000000}%
\pgfsetstrokecolor{currentstroke}%
\pgfsetdash{}{0pt}%
\pgfpathmoveto{\pgfqpoint{1.709308in}{0.937681in}}%
\pgfpathlineto{\pgfqpoint{1.712003in}{0.932417in}}%
\pgfpathlineto{\pgfqpoint{1.714700in}{0.927288in}}%
\pgfpathlineto{\pgfqpoint{1.717399in}{0.922299in}}%
\pgfpathlineto{\pgfqpoint{1.720101in}{0.917455in}}%
\pgfpathlineto{\pgfqpoint{1.710973in}{0.911339in}}%
\pgfpathlineto{\pgfqpoint{1.701459in}{0.905375in}}%
\pgfpathlineto{\pgfqpoint{1.691571in}{0.899569in}}%
\pgfpathlineto{\pgfqpoint{1.681317in}{0.893927in}}%
\pgfpathlineto{\pgfqpoint{1.678890in}{0.898962in}}%
\pgfpathlineto{\pgfqpoint{1.676466in}{0.904140in}}%
\pgfpathlineto{\pgfqpoint{1.674045in}{0.909459in}}%
\pgfpathlineto{\pgfqpoint{1.671625in}{0.914914in}}%
\pgfpathlineto{\pgfqpoint{1.681587in}{0.920373in}}%
\pgfpathlineto{\pgfqpoint{1.691194in}{0.925992in}}%
\pgfpathlineto{\pgfqpoint{1.700437in}{0.931763in}}%
\pgfpathlineto{\pgfqpoint{1.709308in}{0.937681in}}%
\pgfpathclose%
\pgfusepath{fill}%
\end{pgfscope}%
\begin{pgfscope}%
\pgfpathrectangle{\pgfqpoint{0.329460in}{0.284240in}}{\pgfqpoint{1.989680in}{1.989680in}}%
\pgfusepath{clip}%
\pgfsetbuttcap%
\pgfsetroundjoin%
\definecolor{currentfill}{rgb}{0.762373,0.876424,0.137064}%
\pgfsetfillcolor{currentfill}%
\pgfsetlinewidth{0.000000pt}%
\definecolor{currentstroke}{rgb}{0.000000,0.000000,0.000000}%
\pgfsetstrokecolor{currentstroke}%
\pgfsetdash{}{0pt}%
\pgfpathmoveto{\pgfqpoint{1.265223in}{1.704613in}}%
\pgfpathlineto{\pgfqpoint{1.262570in}{1.701957in}}%
\pgfpathlineto{\pgfqpoint{1.259919in}{1.699201in}}%
\pgfpathlineto{\pgfqpoint{1.257270in}{1.696345in}}%
\pgfpathlineto{\pgfqpoint{1.254624in}{1.693391in}}%
\pgfpathlineto{\pgfqpoint{1.257204in}{1.694801in}}%
\pgfpathlineto{\pgfqpoint{1.259874in}{1.696172in}}%
\pgfpathlineto{\pgfqpoint{1.262634in}{1.697503in}}%
\pgfpathlineto{\pgfqpoint{1.265481in}{1.698791in}}%
\pgfpathlineto{\pgfqpoint{1.267827in}{1.701595in}}%
\pgfpathlineto{\pgfqpoint{1.270176in}{1.704300in}}%
\pgfpathlineto{\pgfqpoint{1.272527in}{1.706907in}}%
\pgfpathlineto{\pgfqpoint{1.274880in}{1.709412in}}%
\pgfpathlineto{\pgfqpoint{1.272348in}{1.708267in}}%
\pgfpathlineto{\pgfqpoint{1.269893in}{1.707084in}}%
\pgfpathlineto{\pgfqpoint{1.267517in}{1.705866in}}%
\pgfpathlineto{\pgfqpoint{1.265223in}{1.704613in}}%
\pgfpathclose%
\pgfusepath{fill}%
\end{pgfscope}%
\begin{pgfscope}%
\pgfpathrectangle{\pgfqpoint{0.329460in}{0.284240in}}{\pgfqpoint{1.989680in}{1.989680in}}%
\pgfusepath{clip}%
\pgfsetbuttcap%
\pgfsetroundjoin%
\definecolor{currentfill}{rgb}{0.274952,0.037752,0.364543}%
\pgfsetfillcolor{currentfill}%
\pgfsetlinewidth{0.000000pt}%
\definecolor{currentstroke}{rgb}{0.000000,0.000000,0.000000}%
\pgfsetstrokecolor{currentstroke}%
\pgfsetdash{}{0pt}%
\pgfpathmoveto{\pgfqpoint{1.730939in}{0.899585in}}%
\pgfpathlineto{\pgfqpoint{1.733657in}{0.895514in}}%
\pgfpathlineto{\pgfqpoint{1.736379in}{0.891609in}}%
\pgfpathlineto{\pgfqpoint{1.739104in}{0.887875in}}%
\pgfpathlineto{\pgfqpoint{1.741833in}{0.884314in}}%
\pgfpathlineto{\pgfqpoint{1.732187in}{0.877806in}}%
\pgfpathlineto{\pgfqpoint{1.722133in}{0.871458in}}%
\pgfpathlineto{\pgfqpoint{1.711680in}{0.865277in}}%
\pgfpathlineto{\pgfqpoint{1.700837in}{0.859270in}}%
\pgfpathlineto{\pgfqpoint{1.698385in}{0.863020in}}%
\pgfpathlineto{\pgfqpoint{1.695937in}{0.866943in}}%
\pgfpathlineto{\pgfqpoint{1.693493in}{0.871037in}}%
\pgfpathlineto{\pgfqpoint{1.691051in}{0.875298in}}%
\pgfpathlineto{\pgfqpoint{1.701599in}{0.881122in}}%
\pgfpathlineto{\pgfqpoint{1.711769in}{0.887116in}}%
\pgfpathlineto{\pgfqpoint{1.721553in}{0.893273in}}%
\pgfpathlineto{\pgfqpoint{1.730939in}{0.899585in}}%
\pgfpathclose%
\pgfusepath{fill}%
\end{pgfscope}%
\begin{pgfscope}%
\pgfpathrectangle{\pgfqpoint{0.329460in}{0.284240in}}{\pgfqpoint{1.989680in}{1.989680in}}%
\pgfusepath{clip}%
\pgfsetbuttcap%
\pgfsetroundjoin%
\definecolor{currentfill}{rgb}{0.231674,0.318106,0.544834}%
\pgfsetfillcolor{currentfill}%
\pgfsetlinewidth{0.000000pt}%
\definecolor{currentstroke}{rgb}{0.000000,0.000000,0.000000}%
\pgfsetstrokecolor{currentstroke}%
\pgfsetdash{}{0pt}%
\pgfpathmoveto{\pgfqpoint{1.063806in}{1.091757in}}%
\pgfpathlineto{\pgfqpoint{1.061194in}{1.084196in}}%
\pgfpathlineto{\pgfqpoint{1.058581in}{1.076692in}}%
\pgfpathlineto{\pgfqpoint{1.055968in}{1.069248in}}%
\pgfpathlineto{\pgfqpoint{1.053356in}{1.061867in}}%
\pgfpathlineto{\pgfqpoint{1.045807in}{1.066815in}}%
\pgfpathlineto{\pgfqpoint{1.038580in}{1.071879in}}%
\pgfpathlineto{\pgfqpoint{1.031681in}{1.077055in}}%
\pgfpathlineto{\pgfqpoint{1.025118in}{1.082337in}}%
\pgfpathlineto{\pgfqpoint{1.027970in}{1.089515in}}%
\pgfpathlineto{\pgfqpoint{1.030821in}{1.096757in}}%
\pgfpathlineto{\pgfqpoint{1.033673in}{1.104059in}}%
\pgfpathlineto{\pgfqpoint{1.036524in}{1.111419in}}%
\pgfpathlineto{\pgfqpoint{1.042867in}{1.106345in}}%
\pgfpathlineto{\pgfqpoint{1.049532in}{1.101373in}}%
\pgfpathlineto{\pgfqpoint{1.056514in}{1.096508in}}%
\pgfpathlineto{\pgfqpoint{1.063806in}{1.091757in}}%
\pgfpathclose%
\pgfusepath{fill}%
\end{pgfscope}%
\begin{pgfscope}%
\pgfpathrectangle{\pgfqpoint{0.329460in}{0.284240in}}{\pgfqpoint{1.989680in}{1.989680in}}%
\pgfusepath{clip}%
\pgfsetbuttcap%
\pgfsetroundjoin%
\definecolor{currentfill}{rgb}{0.814576,0.883393,0.110347}%
\pgfsetfillcolor{currentfill}%
\pgfsetlinewidth{0.000000pt}%
\definecolor{currentstroke}{rgb}{0.000000,0.000000,0.000000}%
\pgfsetstrokecolor{currentstroke}%
\pgfsetdash{}{0pt}%
\pgfpathmoveto{\pgfqpoint{1.395742in}{1.725748in}}%
\pgfpathlineto{\pgfqpoint{1.397314in}{1.723917in}}%
\pgfpathlineto{\pgfqpoint{1.398884in}{1.721982in}}%
\pgfpathlineto{\pgfqpoint{1.400453in}{1.719943in}}%
\pgfpathlineto{\pgfqpoint{1.402020in}{1.717801in}}%
\pgfpathlineto{\pgfqpoint{1.405119in}{1.717029in}}%
\pgfpathlineto{\pgfqpoint{1.408166in}{1.716211in}}%
\pgfpathlineto{\pgfqpoint{1.411157in}{1.715349in}}%
\pgfpathlineto{\pgfqpoint{1.414090in}{1.714443in}}%
\pgfpathlineto{\pgfqpoint{1.412150in}{1.716689in}}%
\pgfpathlineto{\pgfqpoint{1.410208in}{1.718833in}}%
\pgfpathlineto{\pgfqpoint{1.408264in}{1.720874in}}%
\pgfpathlineto{\pgfqpoint{1.406318in}{1.722809in}}%
\pgfpathlineto{\pgfqpoint{1.403749in}{1.723602in}}%
\pgfpathlineto{\pgfqpoint{1.401128in}{1.724357in}}%
\pgfpathlineto{\pgfqpoint{1.398458in}{1.725072in}}%
\pgfpathlineto{\pgfqpoint{1.395742in}{1.725748in}}%
\pgfpathclose%
\pgfusepath{fill}%
\end{pgfscope}%
\begin{pgfscope}%
\pgfpathrectangle{\pgfqpoint{0.329460in}{0.284240in}}{\pgfqpoint{1.989680in}{1.989680in}}%
\pgfusepath{clip}%
\pgfsetbuttcap%
\pgfsetroundjoin%
\definecolor{currentfill}{rgb}{0.220124,0.725509,0.466226}%
\pgfsetfillcolor{currentfill}%
\pgfsetlinewidth{0.000000pt}%
\definecolor{currentstroke}{rgb}{0.000000,0.000000,0.000000}%
\pgfsetstrokecolor{currentstroke}%
\pgfsetdash{}{0pt}%
\pgfpathmoveto{\pgfqpoint{1.132655in}{1.488048in}}%
\pgfpathlineto{\pgfqpoint{1.129302in}{1.481448in}}%
\pgfpathlineto{\pgfqpoint{1.125952in}{1.474796in}}%
\pgfpathlineto{\pgfqpoint{1.122604in}{1.468095in}}%
\pgfpathlineto{\pgfqpoint{1.119259in}{1.461345in}}%
\pgfpathlineto{\pgfqpoint{1.118062in}{1.464923in}}%
\pgfpathlineto{\pgfqpoint{1.117099in}{1.468516in}}%
\pgfpathlineto{\pgfqpoint{1.116372in}{1.472119in}}%
\pgfpathlineto{\pgfqpoint{1.115879in}{1.475730in}}%
\pgfpathlineto{\pgfqpoint{1.119265in}{1.482257in}}%
\pgfpathlineto{\pgfqpoint{1.122654in}{1.488737in}}%
\pgfpathlineto{\pgfqpoint{1.126045in}{1.495166in}}%
\pgfpathlineto{\pgfqpoint{1.129439in}{1.501544in}}%
\pgfpathlineto{\pgfqpoint{1.129911in}{1.498156in}}%
\pgfpathlineto{\pgfqpoint{1.130605in}{1.494775in}}%
\pgfpathlineto{\pgfqpoint{1.131519in}{1.491404in}}%
\pgfpathlineto{\pgfqpoint{1.132655in}{1.488048in}}%
\pgfpathclose%
\pgfusepath{fill}%
\end{pgfscope}%
\begin{pgfscope}%
\pgfpathrectangle{\pgfqpoint{0.329460in}{0.284240in}}{\pgfqpoint{1.989680in}{1.989680in}}%
\pgfusepath{clip}%
\pgfsetbuttcap%
\pgfsetroundjoin%
\definecolor{currentfill}{rgb}{0.212395,0.359683,0.551710}%
\pgfsetfillcolor{currentfill}%
\pgfsetlinewidth{0.000000pt}%
\definecolor{currentstroke}{rgb}{0.000000,0.000000,0.000000}%
\pgfsetstrokecolor{currentstroke}%
\pgfsetdash{}{0pt}%
\pgfpathmoveto{\pgfqpoint{1.659616in}{1.145774in}}%
\pgfpathlineto{\pgfqpoint{1.662516in}{1.138262in}}%
\pgfpathlineto{\pgfqpoint{1.665416in}{1.130795in}}%
\pgfpathlineto{\pgfqpoint{1.668314in}{1.123377in}}%
\pgfpathlineto{\pgfqpoint{1.671213in}{1.116010in}}%
\pgfpathlineto{\pgfqpoint{1.665162in}{1.110850in}}%
\pgfpathlineto{\pgfqpoint{1.658784in}{1.105787in}}%
\pgfpathlineto{\pgfqpoint{1.652083in}{1.100827in}}%
\pgfpathlineto{\pgfqpoint{1.645066in}{1.095975in}}%
\pgfpathlineto{\pgfqpoint{1.642396in}{1.103546in}}%
\pgfpathlineto{\pgfqpoint{1.639726in}{1.111169in}}%
\pgfpathlineto{\pgfqpoint{1.637055in}{1.118840in}}%
\pgfpathlineto{\pgfqpoint{1.634385in}{1.126556in}}%
\pgfpathlineto{\pgfqpoint{1.641154in}{1.131210in}}%
\pgfpathlineto{\pgfqpoint{1.647620in}{1.135968in}}%
\pgfpathlineto{\pgfqpoint{1.653776in}{1.140824in}}%
\pgfpathlineto{\pgfqpoint{1.659616in}{1.145774in}}%
\pgfpathclose%
\pgfusepath{fill}%
\end{pgfscope}%
\begin{pgfscope}%
\pgfpathrectangle{\pgfqpoint{0.329460in}{0.284240in}}{\pgfqpoint{1.989680in}{1.989680in}}%
\pgfusepath{clip}%
\pgfsetbuttcap%
\pgfsetroundjoin%
\definecolor{currentfill}{rgb}{0.233603,0.313828,0.543914}%
\pgfsetfillcolor{currentfill}%
\pgfsetlinewidth{0.000000pt}%
\definecolor{currentstroke}{rgb}{0.000000,0.000000,0.000000}%
\pgfsetstrokecolor{currentstroke}%
\pgfsetdash{}{0pt}%
\pgfpathmoveto{\pgfqpoint{1.973706in}{1.081627in}}%
\pgfpathlineto{\pgfqpoint{1.977412in}{1.091236in}}%
\pgfpathlineto{\pgfqpoint{1.981136in}{1.101251in}}%
\pgfpathlineto{\pgfqpoint{1.984879in}{1.111677in}}%
\pgfpathlineto{\pgfqpoint{1.988641in}{1.122523in}}%
\pgfpathlineto{\pgfqpoint{1.983254in}{1.112197in}}%
\pgfpathlineto{\pgfqpoint{1.977208in}{1.101947in}}%
\pgfpathlineto{\pgfqpoint{1.970505in}{1.091785in}}%
\pgfpathlineto{\pgfqpoint{1.963151in}{1.081722in}}%
\pgfpathlineto{\pgfqpoint{1.959516in}{1.071050in}}%
\pgfpathlineto{\pgfqpoint{1.955901in}{1.060799in}}%
\pgfpathlineto{\pgfqpoint{1.952303in}{1.050963in}}%
\pgfpathlineto{\pgfqpoint{1.948723in}{1.041533in}}%
\pgfpathlineto{\pgfqpoint{1.955927in}{1.051421in}}%
\pgfpathlineto{\pgfqpoint{1.962494in}{1.061406in}}%
\pgfpathlineto{\pgfqpoint{1.968422in}{1.071479in}}%
\pgfpathlineto{\pgfqpoint{1.973706in}{1.081627in}}%
\pgfpathclose%
\pgfusepath{fill}%
\end{pgfscope}%
\begin{pgfscope}%
\pgfpathrectangle{\pgfqpoint{0.329460in}{0.284240in}}{\pgfqpoint{1.989680in}{1.989680in}}%
\pgfusepath{clip}%
\pgfsetbuttcap%
\pgfsetroundjoin%
\definecolor{currentfill}{rgb}{0.636902,0.856542,0.216620}%
\pgfsetfillcolor{currentfill}%
\pgfsetlinewidth{0.000000pt}%
\definecolor{currentstroke}{rgb}{0.000000,0.000000,0.000000}%
\pgfsetstrokecolor{currentstroke}%
\pgfsetdash{}{0pt}%
\pgfpathmoveto{\pgfqpoint{1.475504in}{1.668345in}}%
\pgfpathlineto{\pgfqpoint{1.478566in}{1.664609in}}%
\pgfpathlineto{\pgfqpoint{1.481625in}{1.660781in}}%
\pgfpathlineto{\pgfqpoint{1.484681in}{1.656864in}}%
\pgfpathlineto{\pgfqpoint{1.487734in}{1.652859in}}%
\pgfpathlineto{\pgfqpoint{1.489872in}{1.650807in}}%
\pgfpathlineto{\pgfqpoint{1.491873in}{1.648724in}}%
\pgfpathlineto{\pgfqpoint{1.493736in}{1.646611in}}%
\pgfpathlineto{\pgfqpoint{1.495459in}{1.644471in}}%
\pgfpathlineto{\pgfqpoint{1.492229in}{1.648669in}}%
\pgfpathlineto{\pgfqpoint{1.488995in}{1.652780in}}%
\pgfpathlineto{\pgfqpoint{1.485759in}{1.656800in}}%
\pgfpathlineto{\pgfqpoint{1.482520in}{1.660730in}}%
\pgfpathlineto{\pgfqpoint{1.480956in}{1.662673in}}%
\pgfpathlineto{\pgfqpoint{1.479264in}{1.664591in}}%
\pgfpathlineto{\pgfqpoint{1.477446in}{1.666482in}}%
\pgfpathlineto{\pgfqpoint{1.475504in}{1.668345in}}%
\pgfpathclose%
\pgfusepath{fill}%
\end{pgfscope}%
\begin{pgfscope}%
\pgfpathrectangle{\pgfqpoint{0.329460in}{0.284240in}}{\pgfqpoint{1.989680in}{1.989680in}}%
\pgfusepath{clip}%
\pgfsetbuttcap%
\pgfsetroundjoin%
\definecolor{currentfill}{rgb}{0.283072,0.130895,0.449241}%
\pgfsetfillcolor{currentfill}%
\pgfsetlinewidth{0.000000pt}%
\definecolor{currentstroke}{rgb}{0.000000,0.000000,0.000000}%
\pgfsetstrokecolor{currentstroke}%
\pgfsetdash{}{0pt}%
\pgfpathmoveto{\pgfqpoint{1.698552in}{0.960032in}}%
\pgfpathlineto{\pgfqpoint{1.701238in}{0.954258in}}%
\pgfpathlineto{\pgfqpoint{1.703926in}{0.948606in}}%
\pgfpathlineto{\pgfqpoint{1.706616in}{0.943079in}}%
\pgfpathlineto{\pgfqpoint{1.709308in}{0.937681in}}%
\pgfpathlineto{\pgfqpoint{1.700437in}{0.931763in}}%
\pgfpathlineto{\pgfqpoint{1.691194in}{0.925992in}}%
\pgfpathlineto{\pgfqpoint{1.681587in}{0.920373in}}%
\pgfpathlineto{\pgfqpoint{1.671625in}{0.914914in}}%
\pgfpathlineto{\pgfqpoint{1.669208in}{0.920502in}}%
\pgfpathlineto{\pgfqpoint{1.666793in}{0.926219in}}%
\pgfpathlineto{\pgfqpoint{1.664380in}{0.932062in}}%
\pgfpathlineto{\pgfqpoint{1.661969in}{0.938026in}}%
\pgfpathlineto{\pgfqpoint{1.671638in}{0.943302in}}%
\pgfpathlineto{\pgfqpoint{1.680964in}{0.948733in}}%
\pgfpathlineto{\pgfqpoint{1.689939in}{0.954312in}}%
\pgfpathlineto{\pgfqpoint{1.698552in}{0.960032in}}%
\pgfpathclose%
\pgfusepath{fill}%
\end{pgfscope}%
\begin{pgfscope}%
\pgfpathrectangle{\pgfqpoint{0.329460in}{0.284240in}}{\pgfqpoint{1.989680in}{1.989680in}}%
\pgfusepath{clip}%
\pgfsetbuttcap%
\pgfsetroundjoin%
\definecolor{currentfill}{rgb}{0.134692,0.658636,0.517649}%
\pgfsetfillcolor{currentfill}%
\pgfsetlinewidth{0.000000pt}%
\definecolor{currentstroke}{rgb}{0.000000,0.000000,0.000000}%
\pgfsetstrokecolor{currentstroke}%
\pgfsetdash{}{0pt}%
\pgfpathmoveto{\pgfqpoint{1.113395in}{1.418929in}}%
\pgfpathlineto{\pgfqpoint{1.110156in}{1.411749in}}%
\pgfpathlineto{\pgfqpoint{1.106920in}{1.404533in}}%
\pgfpathlineto{\pgfqpoint{1.103686in}{1.397284in}}%
\pgfpathlineto{\pgfqpoint{1.100454in}{1.390003in}}%
\pgfpathlineto{\pgfqpoint{1.098101in}{1.393925in}}%
\pgfpathlineto{\pgfqpoint{1.096005in}{1.397879in}}%
\pgfpathlineto{\pgfqpoint{1.094168in}{1.401861in}}%
\pgfpathlineto{\pgfqpoint{1.092590in}{1.405867in}}%
\pgfpathlineto{\pgfqpoint{1.095915in}{1.412926in}}%
\pgfpathlineto{\pgfqpoint{1.099242in}{1.419953in}}%
\pgfpathlineto{\pgfqpoint{1.102572in}{1.426947in}}%
\pgfpathlineto{\pgfqpoint{1.105905in}{1.433906in}}%
\pgfpathlineto{\pgfqpoint{1.107409in}{1.430124in}}%
\pgfpathlineto{\pgfqpoint{1.109160in}{1.426364in}}%
\pgfpathlineto{\pgfqpoint{1.111156in}{1.422632in}}%
\pgfpathlineto{\pgfqpoint{1.113395in}{1.418929in}}%
\pgfpathclose%
\pgfusepath{fill}%
\end{pgfscope}%
\begin{pgfscope}%
\pgfpathrectangle{\pgfqpoint{0.329460in}{0.284240in}}{\pgfqpoint{1.989680in}{1.989680in}}%
\pgfusepath{clip}%
\pgfsetbuttcap%
\pgfsetroundjoin%
\definecolor{currentfill}{rgb}{0.814576,0.883393,0.110347}%
\pgfsetfillcolor{currentfill}%
\pgfsetlinewidth{0.000000pt}%
\definecolor{currentstroke}{rgb}{0.000000,0.000000,0.000000}%
\pgfsetstrokecolor{currentstroke}%
\pgfsetdash{}{0pt}%
\pgfpathmoveto{\pgfqpoint{1.293818in}{1.722073in}}%
\pgfpathlineto{\pgfqpoint{1.291793in}{1.720111in}}%
\pgfpathlineto{\pgfqpoint{1.289770in}{1.718044in}}%
\pgfpathlineto{\pgfqpoint{1.287748in}{1.715874in}}%
\pgfpathlineto{\pgfqpoint{1.285729in}{1.713601in}}%
\pgfpathlineto{\pgfqpoint{1.288608in}{1.714546in}}%
\pgfpathlineto{\pgfqpoint{1.291547in}{1.715447in}}%
\pgfpathlineto{\pgfqpoint{1.294545in}{1.716304in}}%
\pgfpathlineto{\pgfqpoint{1.297598in}{1.717117in}}%
\pgfpathlineto{\pgfqpoint{1.299250in}{1.719280in}}%
\pgfpathlineto{\pgfqpoint{1.300904in}{1.721341in}}%
\pgfpathlineto{\pgfqpoint{1.302560in}{1.723297in}}%
\pgfpathlineto{\pgfqpoint{1.304217in}{1.725149in}}%
\pgfpathlineto{\pgfqpoint{1.301542in}{1.724438in}}%
\pgfpathlineto{\pgfqpoint{1.298915in}{1.723688in}}%
\pgfpathlineto{\pgfqpoint{1.296340in}{1.722899in}}%
\pgfpathlineto{\pgfqpoint{1.293818in}{1.722073in}}%
\pgfpathclose%
\pgfusepath{fill}%
\end{pgfscope}%
\begin{pgfscope}%
\pgfpathrectangle{\pgfqpoint{0.329460in}{0.284240in}}{\pgfqpoint{1.989680in}{1.989680in}}%
\pgfusepath{clip}%
\pgfsetbuttcap%
\pgfsetroundjoin%
\definecolor{currentfill}{rgb}{0.565498,0.842430,0.262877}%
\pgfsetfillcolor{currentfill}%
\pgfsetlinewidth{0.000000pt}%
\definecolor{currentstroke}{rgb}{0.000000,0.000000,0.000000}%
\pgfsetstrokecolor{currentstroke}%
\pgfsetdash{}{0pt}%
\pgfpathmoveto{\pgfqpoint{1.495459in}{1.644471in}}%
\pgfpathlineto{\pgfqpoint{1.498687in}{1.640185in}}%
\pgfpathlineto{\pgfqpoint{1.501912in}{1.635814in}}%
\pgfpathlineto{\pgfqpoint{1.505134in}{1.631359in}}%
\pgfpathlineto{\pgfqpoint{1.508352in}{1.626822in}}%
\pgfpathlineto{\pgfqpoint{1.510080in}{1.624455in}}%
\pgfpathlineto{\pgfqpoint{1.511651in}{1.622063in}}%
\pgfpathlineto{\pgfqpoint{1.513064in}{1.619647in}}%
\pgfpathlineto{\pgfqpoint{1.514317in}{1.617211in}}%
\pgfpathlineto{\pgfqpoint{1.510971in}{1.621953in}}%
\pgfpathlineto{\pgfqpoint{1.507622in}{1.626612in}}%
\pgfpathlineto{\pgfqpoint{1.504270in}{1.631187in}}%
\pgfpathlineto{\pgfqpoint{1.500915in}{1.635676in}}%
\pgfpathlineto{\pgfqpoint{1.499770in}{1.637905in}}%
\pgfpathlineto{\pgfqpoint{1.498478in}{1.640116in}}%
\pgfpathlineto{\pgfqpoint{1.497040in}{1.642305in}}%
\pgfpathlineto{\pgfqpoint{1.495459in}{1.644471in}}%
\pgfpathclose%
\pgfusepath{fill}%
\end{pgfscope}%
\begin{pgfscope}%
\pgfpathrectangle{\pgfqpoint{0.329460in}{0.284240in}}{\pgfqpoint{1.989680in}{1.989680in}}%
\pgfusepath{clip}%
\pgfsetbuttcap%
\pgfsetroundjoin%
\definecolor{currentfill}{rgb}{0.271305,0.019942,0.347269}%
\pgfsetfillcolor{currentfill}%
\pgfsetlinewidth{0.000000pt}%
\definecolor{currentstroke}{rgb}{0.000000,0.000000,0.000000}%
\pgfsetstrokecolor{currentstroke}%
\pgfsetdash{}{0pt}%
\pgfpathmoveto{\pgfqpoint{1.741833in}{0.884314in}}%
\pgfpathlineto{\pgfqpoint{1.744566in}{0.880932in}}%
\pgfpathlineto{\pgfqpoint{1.747303in}{0.877731in}}%
\pgfpathlineto{\pgfqpoint{1.750045in}{0.874716in}}%
\pgfpathlineto{\pgfqpoint{1.752791in}{0.871892in}}%
\pgfpathlineto{\pgfqpoint{1.742886in}{0.865188in}}%
\pgfpathlineto{\pgfqpoint{1.732560in}{0.858649in}}%
\pgfpathlineto{\pgfqpoint{1.721822in}{0.852282in}}%
\pgfpathlineto{\pgfqpoint{1.710684in}{0.846094in}}%
\pgfpathlineto{\pgfqpoint{1.708216in}{0.849107in}}%
\pgfpathlineto{\pgfqpoint{1.705752in}{0.852309in}}%
\pgfpathlineto{\pgfqpoint{1.703293in}{0.855699in}}%
\pgfpathlineto{\pgfqpoint{1.700837in}{0.859270in}}%
\pgfpathlineto{\pgfqpoint{1.711680in}{0.865277in}}%
\pgfpathlineto{\pgfqpoint{1.722133in}{0.871458in}}%
\pgfpathlineto{\pgfqpoint{1.732187in}{0.877806in}}%
\pgfpathlineto{\pgfqpoint{1.741833in}{0.884314in}}%
\pgfpathclose%
\pgfusepath{fill}%
\end{pgfscope}%
\begin{pgfscope}%
\pgfpathrectangle{\pgfqpoint{0.329460in}{0.284240in}}{\pgfqpoint{1.989680in}{1.989680in}}%
\pgfusepath{clip}%
\pgfsetbuttcap%
\pgfsetroundjoin%
\definecolor{currentfill}{rgb}{0.699415,0.867117,0.175971}%
\pgfsetfillcolor{currentfill}%
\pgfsetlinewidth{0.000000pt}%
\definecolor{currentstroke}{rgb}{0.000000,0.000000,0.000000}%
\pgfsetstrokecolor{currentstroke}%
\pgfsetdash{}{0pt}%
\pgfpathmoveto{\pgfqpoint{1.455160in}{1.688768in}}%
\pgfpathlineto{\pgfqpoint{1.458008in}{1.685588in}}%
\pgfpathlineto{\pgfqpoint{1.460854in}{1.682311in}}%
\pgfpathlineto{\pgfqpoint{1.463697in}{1.678940in}}%
\pgfpathlineto{\pgfqpoint{1.466537in}{1.675476in}}%
\pgfpathlineto{\pgfqpoint{1.468954in}{1.673745in}}%
\pgfpathlineto{\pgfqpoint{1.471256in}{1.671978in}}%
\pgfpathlineto{\pgfqpoint{1.473440in}{1.670178in}}%
\pgfpathlineto{\pgfqpoint{1.475504in}{1.668345in}}%
\pgfpathlineto{\pgfqpoint{1.472439in}{1.671990in}}%
\pgfpathlineto{\pgfqpoint{1.469371in}{1.675540in}}%
\pgfpathlineto{\pgfqpoint{1.466301in}{1.678996in}}%
\pgfpathlineto{\pgfqpoint{1.463228in}{1.682356in}}%
\pgfpathlineto{\pgfqpoint{1.461371in}{1.684004in}}%
\pgfpathlineto{\pgfqpoint{1.459406in}{1.685623in}}%
\pgfpathlineto{\pgfqpoint{1.457335in}{1.687211in}}%
\pgfpathlineto{\pgfqpoint{1.455160in}{1.688768in}}%
\pgfpathclose%
\pgfusepath{fill}%
\end{pgfscope}%
\begin{pgfscope}%
\pgfpathrectangle{\pgfqpoint{0.329460in}{0.284240in}}{\pgfqpoint{1.989680in}{1.989680in}}%
\pgfusepath{clip}%
\pgfsetbuttcap%
\pgfsetroundjoin%
\definecolor{currentfill}{rgb}{0.282884,0.135920,0.453427}%
\pgfsetfillcolor{currentfill}%
\pgfsetlinewidth{0.000000pt}%
\definecolor{currentstroke}{rgb}{0.000000,0.000000,0.000000}%
\pgfsetstrokecolor{currentstroke}%
\pgfsetdash{}{0pt}%
\pgfpathmoveto{\pgfqpoint{0.836739in}{0.913935in}}%
\pgfpathlineto{\pgfqpoint{0.833572in}{0.918740in}}%
\pgfpathlineto{\pgfqpoint{0.830393in}{0.923879in}}%
\pgfpathlineto{\pgfqpoint{0.827201in}{0.929358in}}%
\pgfpathlineto{\pgfqpoint{0.823996in}{0.935183in}}%
\pgfpathlineto{\pgfqpoint{0.814213in}{0.944079in}}%
\pgfpathlineto{\pgfqpoint{0.805005in}{0.953122in}}%
\pgfpathlineto{\pgfqpoint{0.796380in}{0.962301in}}%
\pgfpathlineto{\pgfqpoint{0.788344in}{0.971605in}}%
\pgfpathlineto{\pgfqpoint{0.791747in}{0.965592in}}%
\pgfpathlineto{\pgfqpoint{0.795136in}{0.959924in}}%
\pgfpathlineto{\pgfqpoint{0.798512in}{0.954594in}}%
\pgfpathlineto{\pgfqpoint{0.801875in}{0.949597in}}%
\pgfpathlineto{\pgfqpoint{0.809736in}{0.940486in}}%
\pgfpathlineto{\pgfqpoint{0.818172in}{0.931498in}}%
\pgfpathlineto{\pgfqpoint{0.827175in}{0.922645in}}%
\pgfpathlineto{\pgfqpoint{0.836739in}{0.913935in}}%
\pgfpathclose%
\pgfusepath{fill}%
\end{pgfscope}%
\begin{pgfscope}%
\pgfpathrectangle{\pgfqpoint{0.329460in}{0.284240in}}{\pgfqpoint{1.989680in}{1.989680in}}%
\pgfusepath{clip}%
\pgfsetbuttcap%
\pgfsetroundjoin%
\definecolor{currentfill}{rgb}{0.344074,0.780029,0.397381}%
\pgfsetfillcolor{currentfill}%
\pgfsetlinewidth{0.000000pt}%
\definecolor{currentstroke}{rgb}{0.000000,0.000000,0.000000}%
\pgfsetstrokecolor{currentstroke}%
\pgfsetdash{}{0pt}%
\pgfpathmoveto{\pgfqpoint{1.156693in}{1.550473in}}%
\pgfpathlineto{\pgfqpoint{1.153277in}{1.544578in}}%
\pgfpathlineto{\pgfqpoint{1.149863in}{1.538617in}}%
\pgfpathlineto{\pgfqpoint{1.146453in}{1.532590in}}%
\pgfpathlineto{\pgfqpoint{1.143044in}{1.526500in}}%
\pgfpathlineto{\pgfqpoint{1.142802in}{1.529670in}}%
\pgfpathlineto{\pgfqpoint{1.142769in}{1.532841in}}%
\pgfpathlineto{\pgfqpoint{1.142942in}{1.536009in}}%
\pgfpathlineto{\pgfqpoint{1.143323in}{1.539172in}}%
\pgfpathlineto{\pgfqpoint{1.146720in}{1.545042in}}%
\pgfpathlineto{\pgfqpoint{1.150119in}{1.550849in}}%
\pgfpathlineto{\pgfqpoint{1.153522in}{1.556592in}}%
\pgfpathlineto{\pgfqpoint{1.156927in}{1.562268in}}%
\pgfpathlineto{\pgfqpoint{1.156578in}{1.559325in}}%
\pgfpathlineto{\pgfqpoint{1.156422in}{1.556376in}}%
\pgfpathlineto{\pgfqpoint{1.156461in}{1.553424in}}%
\pgfpathlineto{\pgfqpoint{1.156693in}{1.550473in}}%
\pgfpathclose%
\pgfusepath{fill}%
\end{pgfscope}%
\begin{pgfscope}%
\pgfpathrectangle{\pgfqpoint{0.329460in}{0.284240in}}{\pgfqpoint{1.989680in}{1.989680in}}%
\pgfusepath{clip}%
\pgfsetbuttcap%
\pgfsetroundjoin%
\definecolor{currentfill}{rgb}{0.163625,0.471133,0.558148}%
\pgfsetfillcolor{currentfill}%
\pgfsetlinewidth{0.000000pt}%
\definecolor{currentstroke}{rgb}{0.000000,0.000000,0.000000}%
\pgfsetstrokecolor{currentstroke}%
\pgfsetdash{}{0pt}%
\pgfpathmoveto{\pgfqpoint{1.082231in}{1.234583in}}%
\pgfpathlineto{\pgfqpoint{1.079367in}{1.226699in}}%
\pgfpathlineto{\pgfqpoint{1.076505in}{1.218828in}}%
\pgfpathlineto{\pgfqpoint{1.073644in}{1.210971in}}%
\pgfpathlineto{\pgfqpoint{1.070784in}{1.203131in}}%
\pgfpathlineto{\pgfqpoint{1.065400in}{1.207673in}}%
\pgfpathlineto{\pgfqpoint{1.060312in}{1.212296in}}%
\pgfpathlineto{\pgfqpoint{1.055525in}{1.216994in}}%
\pgfpathlineto{\pgfqpoint{1.051044in}{1.221763in}}%
\pgfpathlineto{\pgfqpoint{1.054096in}{1.229390in}}%
\pgfpathlineto{\pgfqpoint{1.057150in}{1.237035in}}%
\pgfpathlineto{\pgfqpoint{1.060205in}{1.244695in}}%
\pgfpathlineto{\pgfqpoint{1.063262in}{1.252367in}}%
\pgfpathlineto{\pgfqpoint{1.067569in}{1.247815in}}%
\pgfpathlineto{\pgfqpoint{1.072170in}{1.243330in}}%
\pgfpathlineto{\pgfqpoint{1.077058in}{1.238918in}}%
\pgfpathlineto{\pgfqpoint{1.082231in}{1.234583in}}%
\pgfpathclose%
\pgfusepath{fill}%
\end{pgfscope}%
\begin{pgfscope}%
\pgfpathrectangle{\pgfqpoint{0.329460in}{0.284240in}}{\pgfqpoint{1.989680in}{1.989680in}}%
\pgfusepath{clip}%
\pgfsetbuttcap%
\pgfsetroundjoin%
\definecolor{currentfill}{rgb}{0.280255,0.165693,0.476498}%
\pgfsetfillcolor{currentfill}%
\pgfsetlinewidth{0.000000pt}%
\definecolor{currentstroke}{rgb}{0.000000,0.000000,0.000000}%
\pgfsetstrokecolor{currentstroke}%
\pgfsetdash{}{0pt}%
\pgfpathmoveto{\pgfqpoint{1.687825in}{0.984281in}}%
\pgfpathlineto{\pgfqpoint{1.690505in}{0.978053in}}%
\pgfpathlineto{\pgfqpoint{1.693186in}{0.971933in}}%
\pgfpathlineto{\pgfqpoint{1.695868in}{0.965925in}}%
\pgfpathlineto{\pgfqpoint{1.698552in}{0.960032in}}%
\pgfpathlineto{\pgfqpoint{1.689939in}{0.954312in}}%
\pgfpathlineto{\pgfqpoint{1.680964in}{0.948733in}}%
\pgfpathlineto{\pgfqpoint{1.671638in}{0.943302in}}%
\pgfpathlineto{\pgfqpoint{1.661969in}{0.938026in}}%
\pgfpathlineto{\pgfqpoint{1.659559in}{0.944109in}}%
\pgfpathlineto{\pgfqpoint{1.657151in}{0.950308in}}%
\pgfpathlineto{\pgfqpoint{1.654745in}{0.956618in}}%
\pgfpathlineto{\pgfqpoint{1.652340in}{0.963036in}}%
\pgfpathlineto{\pgfqpoint{1.661717in}{0.968129in}}%
\pgfpathlineto{\pgfqpoint{1.670763in}{0.973372in}}%
\pgfpathlineto{\pgfqpoint{1.679469in}{0.978758in}}%
\pgfpathlineto{\pgfqpoint{1.687825in}{0.984281in}}%
\pgfpathclose%
\pgfusepath{fill}%
\end{pgfscope}%
\begin{pgfscope}%
\pgfpathrectangle{\pgfqpoint{0.329460in}{0.284240in}}{\pgfqpoint{1.989680in}{1.989680in}}%
\pgfusepath{clip}%
\pgfsetbuttcap%
\pgfsetroundjoin%
\definecolor{currentfill}{rgb}{0.636902,0.856542,0.216620}%
\pgfsetfillcolor{currentfill}%
\pgfsetlinewidth{0.000000pt}%
\definecolor{currentstroke}{rgb}{0.000000,0.000000,0.000000}%
\pgfsetstrokecolor{currentstroke}%
\pgfsetdash{}{0pt}%
\pgfpathmoveto{\pgfqpoint{1.218573in}{1.658983in}}%
\pgfpathlineto{\pgfqpoint{1.215302in}{1.655009in}}%
\pgfpathlineto{\pgfqpoint{1.212033in}{1.650944in}}%
\pgfpathlineto{\pgfqpoint{1.208767in}{1.646790in}}%
\pgfpathlineto{\pgfqpoint{1.205503in}{1.642547in}}%
\pgfpathlineto{\pgfqpoint{1.207100in}{1.644710in}}%
\pgfpathlineto{\pgfqpoint{1.208839in}{1.646847in}}%
\pgfpathlineto{\pgfqpoint{1.210717in}{1.648957in}}%
\pgfpathlineto{\pgfqpoint{1.212734in}{1.651037in}}%
\pgfpathlineto{\pgfqpoint{1.215831in}{1.655084in}}%
\pgfpathlineto{\pgfqpoint{1.218931in}{1.659043in}}%
\pgfpathlineto{\pgfqpoint{1.222033in}{1.662912in}}%
\pgfpathlineto{\pgfqpoint{1.225139in}{1.666691in}}%
\pgfpathlineto{\pgfqpoint{1.223307in}{1.664802in}}%
\pgfpathlineto{\pgfqpoint{1.221601in}{1.662887in}}%
\pgfpathlineto{\pgfqpoint{1.220023in}{1.660947in}}%
\pgfpathlineto{\pgfqpoint{1.218573in}{1.658983in}}%
\pgfpathclose%
\pgfusepath{fill}%
\end{pgfscope}%
\begin{pgfscope}%
\pgfpathrectangle{\pgfqpoint{0.329460in}{0.284240in}}{\pgfqpoint{1.989680in}{1.989680in}}%
\pgfusepath{clip}%
\pgfsetbuttcap%
\pgfsetroundjoin%
\definecolor{currentfill}{rgb}{0.487026,0.823929,0.312321}%
\pgfsetfillcolor{currentfill}%
\pgfsetlinewidth{0.000000pt}%
\definecolor{currentstroke}{rgb}{0.000000,0.000000,0.000000}%
\pgfsetstrokecolor{currentstroke}%
\pgfsetdash{}{0pt}%
\pgfpathmoveto{\pgfqpoint{1.514317in}{1.617211in}}%
\pgfpathlineto{\pgfqpoint{1.517660in}{1.612388in}}%
\pgfpathlineto{\pgfqpoint{1.521001in}{1.607485in}}%
\pgfpathlineto{\pgfqpoint{1.524338in}{1.602504in}}%
\pgfpathlineto{\pgfqpoint{1.527672in}{1.597447in}}%
\pgfpathlineto{\pgfqpoint{1.528860in}{1.594782in}}%
\pgfpathlineto{\pgfqpoint{1.529872in}{1.592099in}}%
\pgfpathlineto{\pgfqpoint{1.530707in}{1.589401in}}%
\pgfpathlineto{\pgfqpoint{1.531364in}{1.586691in}}%
\pgfpathlineto{\pgfqpoint{1.527954in}{1.591961in}}%
\pgfpathlineto{\pgfqpoint{1.524541in}{1.597154in}}%
\pgfpathlineto{\pgfqpoint{1.521125in}{1.602268in}}%
\pgfpathlineto{\pgfqpoint{1.517706in}{1.607303in}}%
\pgfpathlineto{\pgfqpoint{1.517105in}{1.609799in}}%
\pgfpathlineto{\pgfqpoint{1.516339in}{1.612284in}}%
\pgfpathlineto{\pgfqpoint{1.515409in}{1.614756in}}%
\pgfpathlineto{\pgfqpoint{1.514317in}{1.617211in}}%
\pgfpathclose%
\pgfusepath{fill}%
\end{pgfscope}%
\begin{pgfscope}%
\pgfpathrectangle{\pgfqpoint{0.329460in}{0.284240in}}{\pgfqpoint{1.989680in}{1.989680in}}%
\pgfusepath{clip}%
\pgfsetbuttcap%
\pgfsetroundjoin%
\definecolor{currentfill}{rgb}{0.699415,0.867117,0.175971}%
\pgfsetfillcolor{currentfill}%
\pgfsetlinewidth{0.000000pt}%
\definecolor{currentstroke}{rgb}{0.000000,0.000000,0.000000}%
\pgfsetstrokecolor{currentstroke}%
\pgfsetdash{}{0pt}%
\pgfpathmoveto{\pgfqpoint{1.237589in}{1.680869in}}%
\pgfpathlineto{\pgfqpoint{1.234472in}{1.677467in}}%
\pgfpathlineto{\pgfqpoint{1.231358in}{1.673970in}}%
\pgfpathlineto{\pgfqpoint{1.228247in}{1.670377in}}%
\pgfpathlineto{\pgfqpoint{1.225139in}{1.666691in}}%
\pgfpathlineto{\pgfqpoint{1.227095in}{1.668550in}}%
\pgfpathlineto{\pgfqpoint{1.229172in}{1.670379in}}%
\pgfpathlineto{\pgfqpoint{1.231369in}{1.672176in}}%
\pgfpathlineto{\pgfqpoint{1.233684in}{1.673939in}}%
\pgfpathlineto{\pgfqpoint{1.236578in}{1.677442in}}%
\pgfpathlineto{\pgfqpoint{1.239475in}{1.680852in}}%
\pgfpathlineto{\pgfqpoint{1.242374in}{1.684167in}}%
\pgfpathlineto{\pgfqpoint{1.245277in}{1.687386in}}%
\pgfpathlineto{\pgfqpoint{1.243194in}{1.685801in}}%
\pgfpathlineto{\pgfqpoint{1.241217in}{1.684185in}}%
\pgfpathlineto{\pgfqpoint{1.239348in}{1.682541in}}%
\pgfpathlineto{\pgfqpoint{1.237589in}{1.680869in}}%
\pgfpathclose%
\pgfusepath{fill}%
\end{pgfscope}%
\begin{pgfscope}%
\pgfpathrectangle{\pgfqpoint{0.329460in}{0.284240in}}{\pgfqpoint{1.989680in}{1.989680in}}%
\pgfusepath{clip}%
\pgfsetbuttcap%
\pgfsetroundjoin%
\definecolor{currentfill}{rgb}{0.565498,0.842430,0.262877}%
\pgfsetfillcolor{currentfill}%
\pgfsetlinewidth{0.000000pt}%
\definecolor{currentstroke}{rgb}{0.000000,0.000000,0.000000}%
\pgfsetstrokecolor{currentstroke}%
\pgfsetdash{}{0pt}%
\pgfpathmoveto{\pgfqpoint{1.200567in}{1.633680in}}%
\pgfpathlineto{\pgfqpoint{1.197191in}{1.629144in}}%
\pgfpathlineto{\pgfqpoint{1.193817in}{1.624523in}}%
\pgfpathlineto{\pgfqpoint{1.190447in}{1.619818in}}%
\pgfpathlineto{\pgfqpoint{1.187079in}{1.615029in}}%
\pgfpathlineto{\pgfqpoint{1.188189in}{1.617482in}}%
\pgfpathlineto{\pgfqpoint{1.189460in}{1.619917in}}%
\pgfpathlineto{\pgfqpoint{1.190891in}{1.622330in}}%
\pgfpathlineto{\pgfqpoint{1.192480in}{1.624719in}}%
\pgfpathlineto{\pgfqpoint{1.195731in}{1.629301in}}%
\pgfpathlineto{\pgfqpoint{1.198986in}{1.633801in}}%
\pgfpathlineto{\pgfqpoint{1.202243in}{1.638217in}}%
\pgfpathlineto{\pgfqpoint{1.205503in}{1.642547in}}%
\pgfpathlineto{\pgfqpoint{1.204050in}{1.640360in}}%
\pgfpathlineto{\pgfqpoint{1.202742in}{1.638152in}}%
\pgfpathlineto{\pgfqpoint{1.201580in}{1.635924in}}%
\pgfpathlineto{\pgfqpoint{1.200567in}{1.633680in}}%
\pgfpathclose%
\pgfusepath{fill}%
\end{pgfscope}%
\begin{pgfscope}%
\pgfpathrectangle{\pgfqpoint{0.329460in}{0.284240in}}{\pgfqpoint{1.989680in}{1.989680in}}%
\pgfusepath{clip}%
\pgfsetbuttcap%
\pgfsetroundjoin%
\definecolor{currentfill}{rgb}{0.122606,0.585371,0.546557}%
\pgfsetfillcolor{currentfill}%
\pgfsetlinewidth{0.000000pt}%
\definecolor{currentstroke}{rgb}{0.000000,0.000000,0.000000}%
\pgfsetstrokecolor{currentstroke}%
\pgfsetdash{}{0pt}%
\pgfpathmoveto{\pgfqpoint{1.100064in}{1.344458in}}%
\pgfpathlineto{\pgfqpoint{1.096988in}{1.336837in}}%
\pgfpathlineto{\pgfqpoint{1.093913in}{1.329199in}}%
\pgfpathlineto{\pgfqpoint{1.090841in}{1.321545in}}%
\pgfpathlineto{\pgfqpoint{1.087770in}{1.313877in}}%
\pgfpathlineto{\pgfqpoint{1.084083in}{1.318059in}}%
\pgfpathlineto{\pgfqpoint{1.080670in}{1.322293in}}%
\pgfpathlineto{\pgfqpoint{1.077533in}{1.326576in}}%
\pgfpathlineto{\pgfqpoint{1.074676in}{1.330903in}}%
\pgfpathlineto{\pgfqpoint{1.077891in}{1.338352in}}%
\pgfpathlineto{\pgfqpoint{1.081108in}{1.345787in}}%
\pgfpathlineto{\pgfqpoint{1.084327in}{1.353208in}}%
\pgfpathlineto{\pgfqpoint{1.087548in}{1.360611in}}%
\pgfpathlineto{\pgfqpoint{1.090281in}{1.356506in}}%
\pgfpathlineto{\pgfqpoint{1.093279in}{1.352442in}}%
\pgfpathlineto{\pgfqpoint{1.096541in}{1.348425in}}%
\pgfpathlineto{\pgfqpoint{1.100064in}{1.344458in}}%
\pgfpathclose%
\pgfusepath{fill}%
\end{pgfscope}%
\begin{pgfscope}%
\pgfpathrectangle{\pgfqpoint{0.329460in}{0.284240in}}{\pgfqpoint{1.989680in}{1.989680in}}%
\pgfusepath{clip}%
\pgfsetbuttcap%
\pgfsetroundjoin%
\definecolor{currentfill}{rgb}{0.814576,0.883393,0.110347}%
\pgfsetfillcolor{currentfill}%
\pgfsetlinewidth{0.000000pt}%
\definecolor{currentstroke}{rgb}{0.000000,0.000000,0.000000}%
\pgfsetstrokecolor{currentstroke}%
\pgfsetdash{}{0pt}%
\pgfpathmoveto{\pgfqpoint{1.406318in}{1.722809in}}%
\pgfpathlineto{\pgfqpoint{1.408264in}{1.720874in}}%
\pgfpathlineto{\pgfqpoint{1.410208in}{1.718833in}}%
\pgfpathlineto{\pgfqpoint{1.412150in}{1.716689in}}%
\pgfpathlineto{\pgfqpoint{1.414090in}{1.714443in}}%
\pgfpathlineto{\pgfqpoint{1.416962in}{1.713494in}}%
\pgfpathlineto{\pgfqpoint{1.419770in}{1.712503in}}%
\pgfpathlineto{\pgfqpoint{1.422510in}{1.711470in}}%
\pgfpathlineto{\pgfqpoint{1.425181in}{1.710398in}}%
\pgfpathlineto{\pgfqpoint{1.422897in}{1.712771in}}%
\pgfpathlineto{\pgfqpoint{1.420611in}{1.715041in}}%
\pgfpathlineto{\pgfqpoint{1.418323in}{1.717208in}}%
\pgfpathlineto{\pgfqpoint{1.416033in}{1.719269in}}%
\pgfpathlineto{\pgfqpoint{1.413694in}{1.720208in}}%
\pgfpathlineto{\pgfqpoint{1.411293in}{1.721111in}}%
\pgfpathlineto{\pgfqpoint{1.408834in}{1.721978in}}%
\pgfpathlineto{\pgfqpoint{1.406318in}{1.722809in}}%
\pgfpathclose%
\pgfusepath{fill}%
\end{pgfscope}%
\begin{pgfscope}%
\pgfpathrectangle{\pgfqpoint{0.329460in}{0.284240in}}{\pgfqpoint{1.989680in}{1.989680in}}%
\pgfusepath{clip}%
\pgfsetbuttcap%
\pgfsetroundjoin%
\definecolor{currentfill}{rgb}{0.268510,0.009605,0.335427}%
\pgfsetfillcolor{currentfill}%
\pgfsetlinewidth{0.000000pt}%
\definecolor{currentstroke}{rgb}{0.000000,0.000000,0.000000}%
\pgfsetstrokecolor{currentstroke}%
\pgfsetdash{}{0pt}%
\pgfpathmoveto{\pgfqpoint{1.752791in}{0.871892in}}%
\pgfpathlineto{\pgfqpoint{1.755543in}{0.869261in}}%
\pgfpathlineto{\pgfqpoint{1.758299in}{0.866829in}}%
\pgfpathlineto{\pgfqpoint{1.761060in}{0.864600in}}%
\pgfpathlineto{\pgfqpoint{1.763827in}{0.862577in}}%
\pgfpathlineto{\pgfqpoint{1.753660in}{0.855679in}}%
\pgfpathlineto{\pgfqpoint{1.743061in}{0.848950in}}%
\pgfpathlineto{\pgfqpoint{1.732037in}{0.842398in}}%
\pgfpathlineto{\pgfqpoint{1.720602in}{0.836030in}}%
\pgfpathlineto{\pgfqpoint{1.718115in}{0.838240in}}%
\pgfpathlineto{\pgfqpoint{1.715633in}{0.840656in}}%
\pgfpathlineto{\pgfqpoint{1.713156in}{0.843276in}}%
\pgfpathlineto{\pgfqpoint{1.710684in}{0.846094in}}%
\pgfpathlineto{\pgfqpoint{1.721822in}{0.852282in}}%
\pgfpathlineto{\pgfqpoint{1.732560in}{0.858649in}}%
\pgfpathlineto{\pgfqpoint{1.742886in}{0.865188in}}%
\pgfpathlineto{\pgfqpoint{1.752791in}{0.871892in}}%
\pgfpathclose%
\pgfusepath{fill}%
\end{pgfscope}%
\begin{pgfscope}%
\pgfpathrectangle{\pgfqpoint{0.329460in}{0.284240in}}{\pgfqpoint{1.989680in}{1.989680in}}%
\pgfusepath{clip}%
\pgfsetbuttcap%
\pgfsetroundjoin%
\definecolor{currentfill}{rgb}{0.855810,0.888601,0.097452}%
\pgfsetfillcolor{currentfill}%
\pgfsetlinewidth{0.000000pt}%
\definecolor{currentstroke}{rgb}{0.000000,0.000000,0.000000}%
\pgfsetstrokecolor{currentstroke}%
\pgfsetdash{}{0pt}%
\pgfpathmoveto{\pgfqpoint{1.340812in}{1.736015in}}%
\pgfpathlineto{\pgfqpoint{1.340385in}{1.734779in}}%
\pgfpathlineto{\pgfqpoint{1.339957in}{1.733435in}}%
\pgfpathlineto{\pgfqpoint{1.339531in}{1.731983in}}%
\pgfpathlineto{\pgfqpoint{1.339104in}{1.730425in}}%
\pgfpathlineto{\pgfqpoint{1.342158in}{1.730580in}}%
\pgfpathlineto{\pgfqpoint{1.345220in}{1.730690in}}%
\pgfpathlineto{\pgfqpoint{1.348287in}{1.730756in}}%
\pgfpathlineto{\pgfqpoint{1.351358in}{1.730776in}}%
\pgfpathlineto{\pgfqpoint{1.351352in}{1.732322in}}%
\pgfpathlineto{\pgfqpoint{1.351346in}{1.733761in}}%
\pgfpathlineto{\pgfqpoint{1.351340in}{1.735093in}}%
\pgfpathlineto{\pgfqpoint{1.351334in}{1.736316in}}%
\pgfpathlineto{\pgfqpoint{1.348697in}{1.736299in}}%
\pgfpathlineto{\pgfqpoint{1.346063in}{1.736243in}}%
\pgfpathlineto{\pgfqpoint{1.343434in}{1.736148in}}%
\pgfpathlineto{\pgfqpoint{1.340812in}{1.736015in}}%
\pgfpathclose%
\pgfusepath{fill}%
\end{pgfscope}%
\begin{pgfscope}%
\pgfpathrectangle{\pgfqpoint{0.329460in}{0.284240in}}{\pgfqpoint{1.989680in}{1.989680in}}%
\pgfusepath{clip}%
\pgfsetbuttcap%
\pgfsetroundjoin%
\definecolor{currentfill}{rgb}{0.855810,0.888601,0.097452}%
\pgfsetfillcolor{currentfill}%
\pgfsetlinewidth{0.000000pt}%
\definecolor{currentstroke}{rgb}{0.000000,0.000000,0.000000}%
\pgfsetstrokecolor{currentstroke}%
\pgfsetdash{}{0pt}%
\pgfpathmoveto{\pgfqpoint{1.351334in}{1.736316in}}%
\pgfpathlineto{\pgfqpoint{1.351340in}{1.735093in}}%
\pgfpathlineto{\pgfqpoint{1.351346in}{1.733761in}}%
\pgfpathlineto{\pgfqpoint{1.351352in}{1.732322in}}%
\pgfpathlineto{\pgfqpoint{1.351358in}{1.730776in}}%
\pgfpathlineto{\pgfqpoint{1.354429in}{1.730751in}}%
\pgfpathlineto{\pgfqpoint{1.357496in}{1.730680in}}%
\pgfpathlineto{\pgfqpoint{1.360557in}{1.730565in}}%
\pgfpathlineto{\pgfqpoint{1.363609in}{1.730405in}}%
\pgfpathlineto{\pgfqpoint{1.363171in}{1.731964in}}%
\pgfpathlineto{\pgfqpoint{1.362732in}{1.733417in}}%
\pgfpathlineto{\pgfqpoint{1.362293in}{1.734762in}}%
\pgfpathlineto{\pgfqpoint{1.361853in}{1.735998in}}%
\pgfpathlineto{\pgfqpoint{1.359233in}{1.736135in}}%
\pgfpathlineto{\pgfqpoint{1.356604in}{1.736234in}}%
\pgfpathlineto{\pgfqpoint{1.353971in}{1.736294in}}%
\pgfpathlineto{\pgfqpoint{1.351334in}{1.736316in}}%
\pgfpathclose%
\pgfusepath{fill}%
\end{pgfscope}%
\begin{pgfscope}%
\pgfpathrectangle{\pgfqpoint{0.329460in}{0.284240in}}{\pgfqpoint{1.989680in}{1.989680in}}%
\pgfusepath{clip}%
\pgfsetbuttcap%
\pgfsetroundjoin%
\definecolor{currentfill}{rgb}{0.147607,0.511733,0.557049}%
\pgfsetfillcolor{currentfill}%
\pgfsetlinewidth{0.000000pt}%
\definecolor{currentstroke}{rgb}{0.000000,0.000000,0.000000}%
\pgfsetstrokecolor{currentstroke}%
\pgfsetdash{}{0pt}%
\pgfpathmoveto{\pgfqpoint{1.630308in}{1.287033in}}%
\pgfpathlineto{\pgfqpoint{1.633407in}{1.279386in}}%
\pgfpathlineto{\pgfqpoint{1.636504in}{1.271741in}}%
\pgfpathlineto{\pgfqpoint{1.639600in}{1.264100in}}%
\pgfpathlineto{\pgfqpoint{1.642694in}{1.256467in}}%
\pgfpathlineto{\pgfqpoint{1.638649in}{1.251858in}}%
\pgfpathlineto{\pgfqpoint{1.634309in}{1.247313in}}%
\pgfpathlineto{\pgfqpoint{1.629677in}{1.242836in}}%
\pgfpathlineto{\pgfqpoint{1.624756in}{1.238432in}}%
\pgfpathlineto{\pgfqpoint{1.621844in}{1.246279in}}%
\pgfpathlineto{\pgfqpoint{1.618931in}{1.254133in}}%
\pgfpathlineto{\pgfqpoint{1.616017in}{1.261992in}}%
\pgfpathlineto{\pgfqpoint{1.613101in}{1.269851in}}%
\pgfpathlineto{\pgfqpoint{1.617819in}{1.274047in}}%
\pgfpathlineto{\pgfqpoint{1.622263in}{1.278312in}}%
\pgfpathlineto{\pgfqpoint{1.626427in}{1.282642in}}%
\pgfpathlineto{\pgfqpoint{1.630308in}{1.287033in}}%
\pgfpathclose%
\pgfusepath{fill}%
\end{pgfscope}%
\begin{pgfscope}%
\pgfpathrectangle{\pgfqpoint{0.329460in}{0.284240in}}{\pgfqpoint{1.989680in}{1.989680in}}%
\pgfusepath{clip}%
\pgfsetbuttcap%
\pgfsetroundjoin%
\definecolor{currentfill}{rgb}{0.172719,0.448791,0.557885}%
\pgfsetfillcolor{currentfill}%
\pgfsetlinewidth{0.000000pt}%
\definecolor{currentstroke}{rgb}{0.000000,0.000000,0.000000}%
\pgfsetstrokecolor{currentstroke}%
\pgfsetdash{}{0pt}%
\pgfpathmoveto{\pgfqpoint{0.703333in}{1.160907in}}%
\pgfpathlineto{\pgfqpoint{0.699492in}{1.173920in}}%
\pgfpathlineto{\pgfqpoint{0.695628in}{1.187397in}}%
\pgfpathlineto{\pgfqpoint{0.691742in}{1.201345in}}%
\pgfpathlineto{\pgfqpoint{0.686249in}{1.211968in}}%
\pgfpathlineto{\pgfqpoint{0.681447in}{1.222654in}}%
\pgfpathlineto{\pgfqpoint{0.677337in}{1.233394in}}%
\pgfpathlineto{\pgfqpoint{0.673920in}{1.244175in}}%
\pgfpathlineto{\pgfqpoint{0.677884in}{1.230070in}}%
\pgfpathlineto{\pgfqpoint{0.681826in}{1.216433in}}%
\pgfpathlineto{\pgfqpoint{0.685745in}{1.203258in}}%
\pgfpathlineto{\pgfqpoint{0.689122in}{1.192596in}}%
\pgfpathlineto{\pgfqpoint{0.693180in}{1.181976in}}%
\pgfpathlineto{\pgfqpoint{0.697917in}{1.171409in}}%
\pgfpathlineto{\pgfqpoint{0.703333in}{1.160907in}}%
\pgfpathclose%
\pgfusepath{fill}%
\end{pgfscope}%
\begin{pgfscope}%
\pgfpathrectangle{\pgfqpoint{0.329460in}{0.284240in}}{\pgfqpoint{1.989680in}{1.989680in}}%
\pgfusepath{clip}%
\pgfsetbuttcap%
\pgfsetroundjoin%
\definecolor{currentfill}{rgb}{0.762373,0.876424,0.137064}%
\pgfsetfillcolor{currentfill}%
\pgfsetlinewidth{0.000000pt}%
\definecolor{currentstroke}{rgb}{0.000000,0.000000,0.000000}%
\pgfsetstrokecolor{currentstroke}%
\pgfsetdash{}{0pt}%
\pgfpathmoveto{\pgfqpoint{1.435117in}{1.705729in}}%
\pgfpathlineto{\pgfqpoint{1.437707in}{1.703108in}}%
\pgfpathlineto{\pgfqpoint{1.440295in}{1.700386in}}%
\pgfpathlineto{\pgfqpoint{1.442880in}{1.697566in}}%
\pgfpathlineto{\pgfqpoint{1.445463in}{1.694646in}}%
\pgfpathlineto{\pgfqpoint{1.448031in}{1.693232in}}%
\pgfpathlineto{\pgfqpoint{1.450505in}{1.691780in}}%
\pgfpathlineto{\pgfqpoint{1.452882in}{1.690291in}}%
\pgfpathlineto{\pgfqpoint{1.455160in}{1.688768in}}%
\pgfpathlineto{\pgfqpoint{1.452308in}{1.691851in}}%
\pgfpathlineto{\pgfqpoint{1.449455in}{1.694836in}}%
\pgfpathlineto{\pgfqpoint{1.446598in}{1.697721in}}%
\pgfpathlineto{\pgfqpoint{1.443740in}{1.700506in}}%
\pgfpathlineto{\pgfqpoint{1.441715in}{1.701859in}}%
\pgfpathlineto{\pgfqpoint{1.439602in}{1.703182in}}%
\pgfpathlineto{\pgfqpoint{1.437402in}{1.704472in}}%
\pgfpathlineto{\pgfqpoint{1.435117in}{1.705729in}}%
\pgfpathclose%
\pgfusepath{fill}%
\end{pgfscope}%
\begin{pgfscope}%
\pgfpathrectangle{\pgfqpoint{0.329460in}{0.284240in}}{\pgfqpoint{1.989680in}{1.989680in}}%
\pgfusepath{clip}%
\pgfsetbuttcap%
\pgfsetroundjoin%
\definecolor{currentfill}{rgb}{0.272594,0.025563,0.353093}%
\pgfsetfillcolor{currentfill}%
\pgfsetlinewidth{0.000000pt}%
\definecolor{currentstroke}{rgb}{0.000000,0.000000,0.000000}%
\pgfsetstrokecolor{currentstroke}%
\pgfsetdash{}{0pt}%
\pgfpathmoveto{\pgfqpoint{0.914576in}{0.849378in}}%
\pgfpathlineto{\pgfqpoint{0.911781in}{0.850407in}}%
\pgfpathlineto{\pgfqpoint{0.908977in}{0.851706in}}%
\pgfpathlineto{\pgfqpoint{0.906164in}{0.853281in}}%
\pgfpathlineto{\pgfqpoint{0.903343in}{0.855138in}}%
\pgfpathlineto{\pgfqpoint{0.892169in}{0.862812in}}%
\pgfpathlineto{\pgfqpoint{0.881492in}{0.870662in}}%
\pgfpathlineto{\pgfqpoint{0.871320in}{0.878679in}}%
\pgfpathlineto{\pgfqpoint{0.861664in}{0.886853in}}%
\pgfpathlineto{\pgfqpoint{0.864734in}{0.884806in}}%
\pgfpathlineto{\pgfqpoint{0.867794in}{0.883040in}}%
\pgfpathlineto{\pgfqpoint{0.870845in}{0.881549in}}%
\pgfpathlineto{\pgfqpoint{0.873887in}{0.880328in}}%
\pgfpathlineto{\pgfqpoint{0.883316in}{0.872351in}}%
\pgfpathlineto{\pgfqpoint{0.893246in}{0.864527in}}%
\pgfpathlineto{\pgfqpoint{0.903670in}{0.856867in}}%
\pgfpathlineto{\pgfqpoint{0.914576in}{0.849378in}}%
\pgfpathclose%
\pgfusepath{fill}%
\end{pgfscope}%
\begin{pgfscope}%
\pgfpathrectangle{\pgfqpoint{0.329460in}{0.284240in}}{\pgfqpoint{1.989680in}{1.989680in}}%
\pgfusepath{clip}%
\pgfsetbuttcap%
\pgfsetroundjoin%
\definecolor{currentfill}{rgb}{0.279566,0.067836,0.391917}%
\pgfsetfillcolor{currentfill}%
\pgfsetlinewidth{0.000000pt}%
\definecolor{currentstroke}{rgb}{0.000000,0.000000,0.000000}%
\pgfsetstrokecolor{currentstroke}%
\pgfsetdash{}{0pt}%
\pgfpathmoveto{\pgfqpoint{1.030472in}{0.889055in}}%
\pgfpathlineto{\pgfqpoint{1.028110in}{0.884128in}}%
\pgfpathlineto{\pgfqpoint{1.025746in}{0.879352in}}%
\pgfpathlineto{\pgfqpoint{1.023378in}{0.874731in}}%
\pgfpathlineto{\pgfqpoint{1.021008in}{0.870268in}}%
\pgfpathlineto{\pgfqpoint{1.010134in}{0.875936in}}%
\pgfpathlineto{\pgfqpoint{0.999627in}{0.881780in}}%
\pgfpathlineto{\pgfqpoint{0.989499in}{0.887793in}}%
\pgfpathlineto{\pgfqpoint{0.979760in}{0.893967in}}%
\pgfpathlineto{\pgfqpoint{0.982417in}{0.898244in}}%
\pgfpathlineto{\pgfqpoint{0.985071in}{0.902679in}}%
\pgfpathlineto{\pgfqpoint{0.987721in}{0.907270in}}%
\pgfpathlineto{\pgfqpoint{0.990369in}{0.912011in}}%
\pgfpathlineto{\pgfqpoint{0.999840in}{0.906030in}}%
\pgfpathlineto{\pgfqpoint{1.009687in}{0.900206in}}%
\pgfpathlineto{\pgfqpoint{1.019901in}{0.894545in}}%
\pgfpathlineto{\pgfqpoint{1.030472in}{0.889055in}}%
\pgfpathclose%
\pgfusepath{fill}%
\end{pgfscope}%
\begin{pgfscope}%
\pgfpathrectangle{\pgfqpoint{0.329460in}{0.284240in}}{\pgfqpoint{1.989680in}{1.989680in}}%
\pgfusepath{clip}%
\pgfsetbuttcap%
\pgfsetroundjoin%
\definecolor{currentfill}{rgb}{0.282327,0.094955,0.417331}%
\pgfsetfillcolor{currentfill}%
\pgfsetlinewidth{0.000000pt}%
\definecolor{currentstroke}{rgb}{0.000000,0.000000,0.000000}%
\pgfsetstrokecolor{currentstroke}%
\pgfsetdash{}{0pt}%
\pgfpathmoveto{\pgfqpoint{1.039894in}{0.910201in}}%
\pgfpathlineto{\pgfqpoint{1.037542in}{0.904706in}}%
\pgfpathlineto{\pgfqpoint{1.035188in}{0.899348in}}%
\pgfpathlineto{\pgfqpoint{1.032831in}{0.894129in}}%
\pgfpathlineto{\pgfqpoint{1.030472in}{0.889055in}}%
\pgfpathlineto{\pgfqpoint{1.019901in}{0.894545in}}%
\pgfpathlineto{\pgfqpoint{1.009687in}{0.900206in}}%
\pgfpathlineto{\pgfqpoint{0.999840in}{0.906030in}}%
\pgfpathlineto{\pgfqpoint{0.990369in}{0.912011in}}%
\pgfpathlineto{\pgfqpoint{0.993014in}{0.916900in}}%
\pgfpathlineto{\pgfqpoint{0.995656in}{0.921933in}}%
\pgfpathlineto{\pgfqpoint{0.998296in}{0.927105in}}%
\pgfpathlineto{\pgfqpoint{1.000933in}{0.932414in}}%
\pgfpathlineto{\pgfqpoint{1.010136in}{0.926626in}}%
\pgfpathlineto{\pgfqpoint{1.019703in}{0.920990in}}%
\pgfpathlineto{\pgfqpoint{1.029626in}{0.915513in}}%
\pgfpathlineto{\pgfqpoint{1.039894in}{0.910201in}}%
\pgfpathclose%
\pgfusepath{fill}%
\end{pgfscope}%
\begin{pgfscope}%
\pgfpathrectangle{\pgfqpoint{0.329460in}{0.284240in}}{\pgfqpoint{1.989680in}{1.989680in}}%
\pgfusepath{clip}%
\pgfsetbuttcap%
\pgfsetroundjoin%
\definecolor{currentfill}{rgb}{0.166383,0.690856,0.496502}%
\pgfsetfillcolor{currentfill}%
\pgfsetlinewidth{0.000000pt}%
\definecolor{currentstroke}{rgb}{0.000000,0.000000,0.000000}%
\pgfsetstrokecolor{currentstroke}%
\pgfsetdash{}{0pt}%
\pgfpathmoveto{\pgfqpoint{1.584192in}{1.464525in}}%
\pgfpathlineto{\pgfqpoint{1.587548in}{1.457778in}}%
\pgfpathlineto{\pgfqpoint{1.590901in}{1.450988in}}%
\pgfpathlineto{\pgfqpoint{1.594252in}{1.444156in}}%
\pgfpathlineto{\pgfqpoint{1.597601in}{1.437284in}}%
\pgfpathlineto{\pgfqpoint{1.596316in}{1.433485in}}%
\pgfpathlineto{\pgfqpoint{1.594783in}{1.429705in}}%
\pgfpathlineto{\pgfqpoint{1.593005in}{1.425948in}}%
\pgfpathlineto{\pgfqpoint{1.590982in}{1.422219in}}%
\pgfpathlineto{\pgfqpoint{1.587716in}{1.429312in}}%
\pgfpathlineto{\pgfqpoint{1.584447in}{1.436366in}}%
\pgfpathlineto{\pgfqpoint{1.581176in}{1.443377in}}%
\pgfpathlineto{\pgfqpoint{1.577903in}{1.450345in}}%
\pgfpathlineto{\pgfqpoint{1.579823in}{1.453855in}}%
\pgfpathlineto{\pgfqpoint{1.581512in}{1.457390in}}%
\pgfpathlineto{\pgfqpoint{1.582968in}{1.460948in}}%
\pgfpathlineto{\pgfqpoint{1.584192in}{1.464525in}}%
\pgfpathclose%
\pgfusepath{fill}%
\end{pgfscope}%
\begin{pgfscope}%
\pgfpathrectangle{\pgfqpoint{0.329460in}{0.284240in}}{\pgfqpoint{1.989680in}{1.989680in}}%
\pgfusepath{clip}%
\pgfsetbuttcap%
\pgfsetroundjoin%
\definecolor{currentfill}{rgb}{0.212395,0.359683,0.551710}%
\pgfsetfillcolor{currentfill}%
\pgfsetlinewidth{0.000000pt}%
\definecolor{currentstroke}{rgb}{0.000000,0.000000,0.000000}%
\pgfsetstrokecolor{currentstroke}%
\pgfsetdash{}{0pt}%
\pgfpathmoveto{\pgfqpoint{1.074258in}{1.122511in}}%
\pgfpathlineto{\pgfqpoint{1.071644in}{1.114752in}}%
\pgfpathlineto{\pgfqpoint{1.069031in}{1.107038in}}%
\pgfpathlineto{\pgfqpoint{1.066419in}{1.099372in}}%
\pgfpathlineto{\pgfqpoint{1.063806in}{1.091757in}}%
\pgfpathlineto{\pgfqpoint{1.056514in}{1.096508in}}%
\pgfpathlineto{\pgfqpoint{1.049532in}{1.101373in}}%
\pgfpathlineto{\pgfqpoint{1.042867in}{1.106345in}}%
\pgfpathlineto{\pgfqpoint{1.036524in}{1.111419in}}%
\pgfpathlineto{\pgfqpoint{1.039376in}{1.118832in}}%
\pgfpathlineto{\pgfqpoint{1.042229in}{1.126297in}}%
\pgfpathlineto{\pgfqpoint{1.045081in}{1.133811in}}%
\pgfpathlineto{\pgfqpoint{1.047934in}{1.141369in}}%
\pgfpathlineto{\pgfqpoint{1.054055in}{1.136502in}}%
\pgfpathlineto{\pgfqpoint{1.060487in}{1.131734in}}%
\pgfpathlineto{\pgfqpoint{1.067223in}{1.127068in}}%
\pgfpathlineto{\pgfqpoint{1.074258in}{1.122511in}}%
\pgfpathclose%
\pgfusepath{fill}%
\end{pgfscope}%
\begin{pgfscope}%
\pgfpathrectangle{\pgfqpoint{0.329460in}{0.284240in}}{\pgfqpoint{1.989680in}{1.989680in}}%
\pgfusepath{clip}%
\pgfsetbuttcap%
\pgfsetroundjoin%
\definecolor{currentfill}{rgb}{0.281477,0.755203,0.432552}%
\pgfsetfillcolor{currentfill}%
\pgfsetlinewidth{0.000000pt}%
\definecolor{currentstroke}{rgb}{0.000000,0.000000,0.000000}%
\pgfsetstrokecolor{currentstroke}%
\pgfsetdash{}{0pt}%
\pgfpathmoveto{\pgfqpoint{1.559556in}{1.529318in}}%
\pgfpathlineto{\pgfqpoint{1.562963in}{1.523215in}}%
\pgfpathlineto{\pgfqpoint{1.566368in}{1.517053in}}%
\pgfpathlineto{\pgfqpoint{1.569770in}{1.510834in}}%
\pgfpathlineto{\pgfqpoint{1.573169in}{1.504559in}}%
\pgfpathlineto{\pgfqpoint{1.572894in}{1.501168in}}%
\pgfpathlineto{\pgfqpoint{1.572398in}{1.497780in}}%
\pgfpathlineto{\pgfqpoint{1.571680in}{1.494400in}}%
\pgfpathlineto{\pgfqpoint{1.570741in}{1.491031in}}%
\pgfpathlineto{\pgfqpoint{1.567371in}{1.497527in}}%
\pgfpathlineto{\pgfqpoint{1.564000in}{1.503968in}}%
\pgfpathlineto{\pgfqpoint{1.560625in}{1.510350in}}%
\pgfpathlineto{\pgfqpoint{1.557248in}{1.516673in}}%
\pgfpathlineto{\pgfqpoint{1.558137in}{1.519822in}}%
\pgfpathlineto{\pgfqpoint{1.558818in}{1.522981in}}%
\pgfpathlineto{\pgfqpoint{1.559291in}{1.526148in}}%
\pgfpathlineto{\pgfqpoint{1.559556in}{1.529318in}}%
\pgfpathclose%
\pgfusepath{fill}%
\end{pgfscope}%
\begin{pgfscope}%
\pgfpathrectangle{\pgfqpoint{0.329460in}{0.284240in}}{\pgfqpoint{1.989680in}{1.989680in}}%
\pgfusepath{clip}%
\pgfsetbuttcap%
\pgfsetroundjoin%
\definecolor{currentfill}{rgb}{0.855810,0.888601,0.097452}%
\pgfsetfillcolor{currentfill}%
\pgfsetlinewidth{0.000000pt}%
\definecolor{currentstroke}{rgb}{0.000000,0.000000,0.000000}%
\pgfsetstrokecolor{currentstroke}%
\pgfsetdash{}{0pt}%
\pgfpathmoveto{\pgfqpoint{1.330453in}{1.735099in}}%
\pgfpathlineto{\pgfqpoint{1.329598in}{1.733826in}}%
\pgfpathlineto{\pgfqpoint{1.328744in}{1.732444in}}%
\pgfpathlineto{\pgfqpoint{1.327891in}{1.730954in}}%
\pgfpathlineto{\pgfqpoint{1.327038in}{1.729358in}}%
\pgfpathlineto{\pgfqpoint{1.330027in}{1.729691in}}%
\pgfpathlineto{\pgfqpoint{1.333036in}{1.729980in}}%
\pgfpathlineto{\pgfqpoint{1.336063in}{1.730225in}}%
\pgfpathlineto{\pgfqpoint{1.339104in}{1.730425in}}%
\pgfpathlineto{\pgfqpoint{1.339531in}{1.731983in}}%
\pgfpathlineto{\pgfqpoint{1.339957in}{1.733435in}}%
\pgfpathlineto{\pgfqpoint{1.340385in}{1.734779in}}%
\pgfpathlineto{\pgfqpoint{1.340812in}{1.736015in}}%
\pgfpathlineto{\pgfqpoint{1.338201in}{1.735843in}}%
\pgfpathlineto{\pgfqpoint{1.335602in}{1.735633in}}%
\pgfpathlineto{\pgfqpoint{1.333018in}{1.735385in}}%
\pgfpathlineto{\pgfqpoint{1.330453in}{1.735099in}}%
\pgfpathclose%
\pgfusepath{fill}%
\end{pgfscope}%
\begin{pgfscope}%
\pgfpathrectangle{\pgfqpoint{0.329460in}{0.284240in}}{\pgfqpoint{1.989680in}{1.989680in}}%
\pgfusepath{clip}%
\pgfsetbuttcap%
\pgfsetroundjoin%
\definecolor{currentfill}{rgb}{0.274128,0.199721,0.498911}%
\pgfsetfillcolor{currentfill}%
\pgfsetlinewidth{0.000000pt}%
\definecolor{currentstroke}{rgb}{0.000000,0.000000,0.000000}%
\pgfsetstrokecolor{currentstroke}%
\pgfsetdash{}{0pt}%
\pgfpathmoveto{\pgfqpoint{1.677119in}{1.010209in}}%
\pgfpathlineto{\pgfqpoint{1.679794in}{1.003581in}}%
\pgfpathlineto{\pgfqpoint{1.682470in}{0.997049in}}%
\pgfpathlineto{\pgfqpoint{1.685147in}{0.990614in}}%
\pgfpathlineto{\pgfqpoint{1.687825in}{0.984281in}}%
\pgfpathlineto{\pgfqpoint{1.679469in}{0.978758in}}%
\pgfpathlineto{\pgfqpoint{1.670763in}{0.973372in}}%
\pgfpathlineto{\pgfqpoint{1.661717in}{0.968129in}}%
\pgfpathlineto{\pgfqpoint{1.652340in}{0.963036in}}%
\pgfpathlineto{\pgfqpoint{1.649936in}{0.969559in}}%
\pgfpathlineto{\pgfqpoint{1.647534in}{0.976183in}}%
\pgfpathlineto{\pgfqpoint{1.645133in}{0.982906in}}%
\pgfpathlineto{\pgfqpoint{1.642733in}{0.989723in}}%
\pgfpathlineto{\pgfqpoint{1.651818in}{0.994634in}}%
\pgfpathlineto{\pgfqpoint{1.660583in}{0.999689in}}%
\pgfpathlineto{\pgfqpoint{1.669020in}{1.004883in}}%
\pgfpathlineto{\pgfqpoint{1.677119in}{1.010209in}}%
\pgfpathclose%
\pgfusepath{fill}%
\end{pgfscope}%
\begin{pgfscope}%
\pgfpathrectangle{\pgfqpoint{0.329460in}{0.284240in}}{\pgfqpoint{1.989680in}{1.989680in}}%
\pgfusepath{clip}%
\pgfsetbuttcap%
\pgfsetroundjoin%
\definecolor{currentfill}{rgb}{0.855810,0.888601,0.097452}%
\pgfsetfillcolor{currentfill}%
\pgfsetlinewidth{0.000000pt}%
\definecolor{currentstroke}{rgb}{0.000000,0.000000,0.000000}%
\pgfsetstrokecolor{currentstroke}%
\pgfsetdash{}{0pt}%
\pgfpathmoveto{\pgfqpoint{1.361853in}{1.735998in}}%
\pgfpathlineto{\pgfqpoint{1.362293in}{1.734762in}}%
\pgfpathlineto{\pgfqpoint{1.362732in}{1.733417in}}%
\pgfpathlineto{\pgfqpoint{1.363171in}{1.731964in}}%
\pgfpathlineto{\pgfqpoint{1.363609in}{1.730405in}}%
\pgfpathlineto{\pgfqpoint{1.366650in}{1.730200in}}%
\pgfpathlineto{\pgfqpoint{1.369675in}{1.729950in}}%
\pgfpathlineto{\pgfqpoint{1.372681in}{1.729656in}}%
\pgfpathlineto{\pgfqpoint{1.375668in}{1.729318in}}%
\pgfpathlineto{\pgfqpoint{1.374804in}{1.730916in}}%
\pgfpathlineto{\pgfqpoint{1.373939in}{1.732407in}}%
\pgfpathlineto{\pgfqpoint{1.373073in}{1.733790in}}%
\pgfpathlineto{\pgfqpoint{1.372206in}{1.735065in}}%
\pgfpathlineto{\pgfqpoint{1.369643in}{1.735355in}}%
\pgfpathlineto{\pgfqpoint{1.367061in}{1.735608in}}%
\pgfpathlineto{\pgfqpoint{1.364464in}{1.735822in}}%
\pgfpathlineto{\pgfqpoint{1.361853in}{1.735998in}}%
\pgfpathclose%
\pgfusepath{fill}%
\end{pgfscope}%
\begin{pgfscope}%
\pgfpathrectangle{\pgfqpoint{0.329460in}{0.284240in}}{\pgfqpoint{1.989680in}{1.989680in}}%
\pgfusepath{clip}%
\pgfsetbuttcap%
\pgfsetroundjoin%
\definecolor{currentfill}{rgb}{0.814576,0.883393,0.110347}%
\pgfsetfillcolor{currentfill}%
\pgfsetlinewidth{0.000000pt}%
\definecolor{currentstroke}{rgb}{0.000000,0.000000,0.000000}%
\pgfsetstrokecolor{currentstroke}%
\pgfsetdash{}{0pt}%
\pgfpathmoveto{\pgfqpoint{1.284316in}{1.718407in}}%
\pgfpathlineto{\pgfqpoint{1.281954in}{1.716314in}}%
\pgfpathlineto{\pgfqpoint{1.279594in}{1.714117in}}%
\pgfpathlineto{\pgfqpoint{1.277236in}{1.711816in}}%
\pgfpathlineto{\pgfqpoint{1.274880in}{1.709412in}}%
\pgfpathlineto{\pgfqpoint{1.277487in}{1.710519in}}%
\pgfpathlineto{\pgfqpoint{1.280166in}{1.711587in}}%
\pgfpathlineto{\pgfqpoint{1.282914in}{1.712615in}}%
\pgfpathlineto{\pgfqpoint{1.285729in}{1.713601in}}%
\pgfpathlineto{\pgfqpoint{1.287748in}{1.715874in}}%
\pgfpathlineto{\pgfqpoint{1.289770in}{1.718044in}}%
\pgfpathlineto{\pgfqpoint{1.291793in}{1.720111in}}%
\pgfpathlineto{\pgfqpoint{1.293818in}{1.722073in}}%
\pgfpathlineto{\pgfqpoint{1.291352in}{1.721209in}}%
\pgfpathlineto{\pgfqpoint{1.288945in}{1.720310in}}%
\pgfpathlineto{\pgfqpoint{1.286599in}{1.719375in}}%
\pgfpathlineto{\pgfqpoint{1.284316in}{1.718407in}}%
\pgfpathclose%
\pgfusepath{fill}%
\end{pgfscope}%
\begin{pgfscope}%
\pgfpathrectangle{\pgfqpoint{0.329460in}{0.284240in}}{\pgfqpoint{1.989680in}{1.989680in}}%
\pgfusepath{clip}%
\pgfsetbuttcap%
\pgfsetroundjoin%
\definecolor{currentfill}{rgb}{0.274952,0.037752,0.364543}%
\pgfsetfillcolor{currentfill}%
\pgfsetlinewidth{0.000000pt}%
\definecolor{currentstroke}{rgb}{0.000000,0.000000,0.000000}%
\pgfsetstrokecolor{currentstroke}%
\pgfsetdash{}{0pt}%
\pgfpathmoveto{\pgfqpoint{1.021008in}{0.870268in}}%
\pgfpathlineto{\pgfqpoint{1.018634in}{0.865968in}}%
\pgfpathlineto{\pgfqpoint{1.016257in}{0.861835in}}%
\pgfpathlineto{\pgfqpoint{1.013877in}{0.857872in}}%
\pgfpathlineto{\pgfqpoint{1.011493in}{0.854083in}}%
\pgfpathlineto{\pgfqpoint{1.000314in}{0.859929in}}%
\pgfpathlineto{\pgfqpoint{0.989515in}{0.865955in}}%
\pgfpathlineto{\pgfqpoint{0.979105in}{0.872155in}}%
\pgfpathlineto{\pgfqpoint{0.969096in}{0.878521in}}%
\pgfpathlineto{\pgfqpoint{0.971767in}{0.882125in}}%
\pgfpathlineto{\pgfqpoint{0.974435in}{0.885904in}}%
\pgfpathlineto{\pgfqpoint{0.977099in}{0.889852in}}%
\pgfpathlineto{\pgfqpoint{0.979760in}{0.893967in}}%
\pgfpathlineto{\pgfqpoint{0.989499in}{0.887793in}}%
\pgfpathlineto{\pgfqpoint{0.999627in}{0.881780in}}%
\pgfpathlineto{\pgfqpoint{1.010134in}{0.875936in}}%
\pgfpathlineto{\pgfqpoint{1.021008in}{0.870268in}}%
\pgfpathclose%
\pgfusepath{fill}%
\end{pgfscope}%
\begin{pgfscope}%
\pgfpathrectangle{\pgfqpoint{0.329460in}{0.284240in}}{\pgfqpoint{1.989680in}{1.989680in}}%
\pgfusepath{clip}%
\pgfsetbuttcap%
\pgfsetroundjoin%
\definecolor{currentfill}{rgb}{0.195860,0.395433,0.555276}%
\pgfsetfillcolor{currentfill}%
\pgfsetlinewidth{0.000000pt}%
\definecolor{currentstroke}{rgb}{0.000000,0.000000,0.000000}%
\pgfsetstrokecolor{currentstroke}%
\pgfsetdash{}{0pt}%
\pgfpathmoveto{\pgfqpoint{1.648011in}{1.176219in}}%
\pgfpathlineto{\pgfqpoint{1.650913in}{1.168554in}}%
\pgfpathlineto{\pgfqpoint{1.653815in}{1.160923in}}%
\pgfpathlineto{\pgfqpoint{1.656716in}{1.153328in}}%
\pgfpathlineto{\pgfqpoint{1.659616in}{1.145774in}}%
\pgfpathlineto{\pgfqpoint{1.653776in}{1.140824in}}%
\pgfpathlineto{\pgfqpoint{1.647620in}{1.135968in}}%
\pgfpathlineto{\pgfqpoint{1.641154in}{1.131210in}}%
\pgfpathlineto{\pgfqpoint{1.634385in}{1.126556in}}%
\pgfpathlineto{\pgfqpoint{1.631714in}{1.134315in}}%
\pgfpathlineto{\pgfqpoint{1.629042in}{1.142113in}}%
\pgfpathlineto{\pgfqpoint{1.626370in}{1.149947in}}%
\pgfpathlineto{\pgfqpoint{1.623698in}{1.157816in}}%
\pgfpathlineto{\pgfqpoint{1.630219in}{1.162272in}}%
\pgfpathlineto{\pgfqpoint{1.636449in}{1.166828in}}%
\pgfpathlineto{\pgfqpoint{1.642382in}{1.171479in}}%
\pgfpathlineto{\pgfqpoint{1.648011in}{1.176219in}}%
\pgfpathclose%
\pgfusepath{fill}%
\end{pgfscope}%
\begin{pgfscope}%
\pgfpathrectangle{\pgfqpoint{0.329460in}{0.284240in}}{\pgfqpoint{1.989680in}{1.989680in}}%
\pgfusepath{clip}%
\pgfsetbuttcap%
\pgfsetroundjoin%
\definecolor{currentfill}{rgb}{0.283072,0.130895,0.449241}%
\pgfsetfillcolor{currentfill}%
\pgfsetlinewidth{0.000000pt}%
\definecolor{currentstroke}{rgb}{0.000000,0.000000,0.000000}%
\pgfsetstrokecolor{currentstroke}%
\pgfsetdash{}{0pt}%
\pgfpathmoveto{\pgfqpoint{1.049281in}{0.933471in}}%
\pgfpathlineto{\pgfqpoint{1.046937in}{0.927467in}}%
\pgfpathlineto{\pgfqpoint{1.044591in}{0.921584in}}%
\pgfpathlineto{\pgfqpoint{1.042244in}{0.915828in}}%
\pgfpathlineto{\pgfqpoint{1.039894in}{0.910201in}}%
\pgfpathlineto{\pgfqpoint{1.029626in}{0.915513in}}%
\pgfpathlineto{\pgfqpoint{1.019703in}{0.920990in}}%
\pgfpathlineto{\pgfqpoint{1.010136in}{0.926626in}}%
\pgfpathlineto{\pgfqpoint{1.000933in}{0.932414in}}%
\pgfpathlineto{\pgfqpoint{1.003569in}{0.937855in}}%
\pgfpathlineto{\pgfqpoint{1.006201in}{0.943426in}}%
\pgfpathlineto{\pgfqpoint{1.008832in}{0.949122in}}%
\pgfpathlineto{\pgfqpoint{1.011461in}{0.954940in}}%
\pgfpathlineto{\pgfqpoint{1.020396in}{0.949346in}}%
\pgfpathlineto{\pgfqpoint{1.029684in}{0.943898in}}%
\pgfpathlineto{\pgfqpoint{1.039315in}{0.938605in}}%
\pgfpathlineto{\pgfqpoint{1.049281in}{0.933471in}}%
\pgfpathclose%
\pgfusepath{fill}%
\end{pgfscope}%
\begin{pgfscope}%
\pgfpathrectangle{\pgfqpoint{0.329460in}{0.284240in}}{\pgfqpoint{1.989680in}{1.989680in}}%
\pgfusepath{clip}%
\pgfsetbuttcap%
\pgfsetroundjoin%
\definecolor{currentfill}{rgb}{0.762373,0.876424,0.137064}%
\pgfsetfillcolor{currentfill}%
\pgfsetlinewidth{0.000000pt}%
\definecolor{currentstroke}{rgb}{0.000000,0.000000,0.000000}%
\pgfsetstrokecolor{currentstroke}%
\pgfsetdash{}{0pt}%
\pgfpathmoveto{\pgfqpoint{1.256912in}{1.699278in}}%
\pgfpathlineto{\pgfqpoint{1.254000in}{1.696455in}}%
\pgfpathlineto{\pgfqpoint{1.251089in}{1.693531in}}%
\pgfpathlineto{\pgfqpoint{1.248182in}{1.690508in}}%
\pgfpathlineto{\pgfqpoint{1.245277in}{1.687386in}}%
\pgfpathlineto{\pgfqpoint{1.247464in}{1.688939in}}%
\pgfpathlineto{\pgfqpoint{1.249752in}{1.690459in}}%
\pgfpathlineto{\pgfqpoint{1.252140in}{1.691943in}}%
\pgfpathlineto{\pgfqpoint{1.254624in}{1.693391in}}%
\pgfpathlineto{\pgfqpoint{1.257270in}{1.696345in}}%
\pgfpathlineto{\pgfqpoint{1.259919in}{1.699201in}}%
\pgfpathlineto{\pgfqpoint{1.262570in}{1.701957in}}%
\pgfpathlineto{\pgfqpoint{1.265223in}{1.704613in}}%
\pgfpathlineto{\pgfqpoint{1.263014in}{1.703327in}}%
\pgfpathlineto{\pgfqpoint{1.260891in}{1.702008in}}%
\pgfpathlineto{\pgfqpoint{1.258856in}{1.700658in}}%
\pgfpathlineto{\pgfqpoint{1.256912in}{1.699278in}}%
\pgfpathclose%
\pgfusepath{fill}%
\end{pgfscope}%
\begin{pgfscope}%
\pgfpathrectangle{\pgfqpoint{0.329460in}{0.284240in}}{\pgfqpoint{1.989680in}{1.989680in}}%
\pgfusepath{clip}%
\pgfsetbuttcap%
\pgfsetroundjoin%
\definecolor{currentfill}{rgb}{0.487026,0.823929,0.312321}%
\pgfsetfillcolor{currentfill}%
\pgfsetlinewidth{0.000000pt}%
\definecolor{currentstroke}{rgb}{0.000000,0.000000,0.000000}%
\pgfsetstrokecolor{currentstroke}%
\pgfsetdash{}{0pt}%
\pgfpathmoveto{\pgfqpoint{1.184273in}{1.605076in}}%
\pgfpathlineto{\pgfqpoint{1.180845in}{1.599994in}}%
\pgfpathlineto{\pgfqpoint{1.177420in}{1.594832in}}%
\pgfpathlineto{\pgfqpoint{1.173997in}{1.589591in}}%
\pgfpathlineto{\pgfqpoint{1.170578in}{1.584273in}}%
\pgfpathlineto{\pgfqpoint{1.171076in}{1.586993in}}%
\pgfpathlineto{\pgfqpoint{1.171752in}{1.589702in}}%
\pgfpathlineto{\pgfqpoint{1.172607in}{1.592398in}}%
\pgfpathlineto{\pgfqpoint{1.173638in}{1.595079in}}%
\pgfpathlineto{\pgfqpoint{1.176994in}{1.600183in}}%
\pgfpathlineto{\pgfqpoint{1.180353in}{1.605210in}}%
\pgfpathlineto{\pgfqpoint{1.183715in}{1.610160in}}%
\pgfpathlineto{\pgfqpoint{1.187079in}{1.615029in}}%
\pgfpathlineto{\pgfqpoint{1.186132in}{1.612560in}}%
\pgfpathlineto{\pgfqpoint{1.185347in}{1.610076in}}%
\pgfpathlineto{\pgfqpoint{1.184728in}{1.607581in}}%
\pgfpathlineto{\pgfqpoint{1.184273in}{1.605076in}}%
\pgfpathclose%
\pgfusepath{fill}%
\end{pgfscope}%
\begin{pgfscope}%
\pgfpathrectangle{\pgfqpoint{0.329460in}{0.284240in}}{\pgfqpoint{1.989680in}{1.989680in}}%
\pgfusepath{clip}%
\pgfsetbuttcap%
\pgfsetroundjoin%
\definecolor{currentfill}{rgb}{0.855810,0.888601,0.097452}%
\pgfsetfillcolor{currentfill}%
\pgfsetlinewidth{0.000000pt}%
\definecolor{currentstroke}{rgb}{0.000000,0.000000,0.000000}%
\pgfsetstrokecolor{currentstroke}%
\pgfsetdash{}{0pt}%
\pgfpathmoveto{\pgfqpoint{1.372206in}{1.735065in}}%
\pgfpathlineto{\pgfqpoint{1.373073in}{1.733790in}}%
\pgfpathlineto{\pgfqpoint{1.373939in}{1.732407in}}%
\pgfpathlineto{\pgfqpoint{1.374804in}{1.730916in}}%
\pgfpathlineto{\pgfqpoint{1.375668in}{1.729318in}}%
\pgfpathlineto{\pgfqpoint{1.378630in}{1.728936in}}%
\pgfpathlineto{\pgfqpoint{1.381565in}{1.728511in}}%
\pgfpathlineto{\pgfqpoint{1.384471in}{1.728042in}}%
\pgfpathlineto{\pgfqpoint{1.383296in}{1.729685in}}%
\pgfpathlineto{\pgfqpoint{1.382120in}{1.731222in}}%
\pgfpathlineto{\pgfqpoint{1.380943in}{1.732651in}}%
\pgfpathlineto{\pgfqpoint{1.379764in}{1.733971in}}%
\pgfpathlineto{\pgfqpoint{1.377270in}{1.734373in}}%
\pgfpathlineto{\pgfqpoint{1.374750in}{1.734738in}}%
\pgfpathlineto{\pgfqpoint{1.372206in}{1.735065in}}%
\pgfpathclose%
\pgfusepath{fill}%
\end{pgfscope}%
\begin{pgfscope}%
\pgfpathrectangle{\pgfqpoint{0.329460in}{0.284240in}}{\pgfqpoint{1.989680in}{1.989680in}}%
\pgfusepath{clip}%
\pgfsetbuttcap%
\pgfsetroundjoin%
\definecolor{currentfill}{rgb}{0.277941,0.056324,0.381191}%
\pgfsetfillcolor{currentfill}%
\pgfsetlinewidth{0.000000pt}%
\definecolor{currentstroke}{rgb}{0.000000,0.000000,0.000000}%
\pgfsetstrokecolor{currentstroke}%
\pgfsetdash{}{0pt}%
\pgfpathmoveto{\pgfqpoint{1.848854in}{0.894243in}}%
\pgfpathlineto{\pgfqpoint{1.851981in}{0.896619in}}%
\pgfpathlineto{\pgfqpoint{1.855118in}{0.899285in}}%
\pgfpathlineto{\pgfqpoint{1.858265in}{0.902246in}}%
\pgfpathlineto{\pgfqpoint{1.861422in}{0.905508in}}%
\pgfpathlineto{\pgfqpoint{1.852015in}{0.897006in}}%
\pgfpathlineto{\pgfqpoint{1.842071in}{0.888656in}}%
\pgfpathlineto{\pgfqpoint{1.831599in}{0.880468in}}%
\pgfpathlineto{\pgfqpoint{1.820608in}{0.872452in}}%
\pgfpathlineto{\pgfqpoint{1.817689in}{0.869379in}}%
\pgfpathlineto{\pgfqpoint{1.814779in}{0.866608in}}%
\pgfpathlineto{\pgfqpoint{1.811879in}{0.864134in}}%
\pgfpathlineto{\pgfqpoint{1.808989in}{0.861951in}}%
\pgfpathlineto{\pgfqpoint{1.819722in}{0.869781in}}%
\pgfpathlineto{\pgfqpoint{1.829950in}{0.877780in}}%
\pgfpathlineto{\pgfqpoint{1.839664in}{0.885937in}}%
\pgfpathlineto{\pgfqpoint{1.848854in}{0.894243in}}%
\pgfpathclose%
\pgfusepath{fill}%
\end{pgfscope}%
\begin{pgfscope}%
\pgfpathrectangle{\pgfqpoint{0.329460in}{0.284240in}}{\pgfqpoint{1.989680in}{1.989680in}}%
\pgfusepath{clip}%
\pgfsetbuttcap%
\pgfsetroundjoin%
\definecolor{currentfill}{rgb}{0.855810,0.888601,0.097452}%
\pgfsetfillcolor{currentfill}%
\pgfsetlinewidth{0.000000pt}%
\definecolor{currentstroke}{rgb}{0.000000,0.000000,0.000000}%
\pgfsetstrokecolor{currentstroke}%
\pgfsetdash{}{0pt}%
\pgfpathmoveto{\pgfqpoint{1.320417in}{1.733583in}}%
\pgfpathlineto{\pgfqpoint{1.319148in}{1.732247in}}%
\pgfpathlineto{\pgfqpoint{1.317880in}{1.730802in}}%
\pgfpathlineto{\pgfqpoint{1.316614in}{1.729249in}}%
\pgfpathlineto{\pgfqpoint{1.315348in}{1.727590in}}%
\pgfpathlineto{\pgfqpoint{1.318225in}{1.728096in}}%
\pgfpathlineto{\pgfqpoint{1.321135in}{1.728560in}}%
\pgfpathlineto{\pgfqpoint{1.324073in}{1.728981in}}%
\pgfpathlineto{\pgfqpoint{1.327038in}{1.729358in}}%
\pgfpathlineto{\pgfqpoint{1.327891in}{1.730954in}}%
\pgfpathlineto{\pgfqpoint{1.328744in}{1.732444in}}%
\pgfpathlineto{\pgfqpoint{1.329598in}{1.733826in}}%
\pgfpathlineto{\pgfqpoint{1.330453in}{1.735099in}}%
\pgfpathlineto{\pgfqpoint{1.327907in}{1.734776in}}%
\pgfpathlineto{\pgfqpoint{1.325384in}{1.734415in}}%
\pgfpathlineto{\pgfqpoint{1.322887in}{1.734018in}}%
\pgfpathlineto{\pgfqpoint{1.320417in}{1.733583in}}%
\pgfpathclose%
\pgfusepath{fill}%
\end{pgfscope}%
\begin{pgfscope}%
\pgfpathrectangle{\pgfqpoint{0.329460in}{0.284240in}}{\pgfqpoint{1.989680in}{1.989680in}}%
\pgfusepath{clip}%
\pgfsetbuttcap%
\pgfsetroundjoin%
\definecolor{currentfill}{rgb}{0.120081,0.622161,0.534946}%
\pgfsetfillcolor{currentfill}%
\pgfsetlinewidth{0.000000pt}%
\definecolor{currentstroke}{rgb}{0.000000,0.000000,0.000000}%
\pgfsetstrokecolor{currentstroke}%
\pgfsetdash{}{0pt}%
\pgfpathmoveto{\pgfqpoint{1.604025in}{1.393487in}}%
\pgfpathlineto{\pgfqpoint{1.607280in}{1.386227in}}%
\pgfpathlineto{\pgfqpoint{1.610533in}{1.378939in}}%
\pgfpathlineto{\pgfqpoint{1.613783in}{1.371627in}}%
\pgfpathlineto{\pgfqpoint{1.617032in}{1.364293in}}%
\pgfpathlineto{\pgfqpoint{1.614537in}{1.360153in}}%
\pgfpathlineto{\pgfqpoint{1.611774in}{1.356052in}}%
\pgfpathlineto{\pgfqpoint{1.608746in}{1.351993in}}%
\pgfpathlineto{\pgfqpoint{1.605455in}{1.347981in}}%
\pgfpathlineto{\pgfqpoint{1.602340in}{1.355535in}}%
\pgfpathlineto{\pgfqpoint{1.599223in}{1.363065in}}%
\pgfpathlineto{\pgfqpoint{1.596104in}{1.370571in}}%
\pgfpathlineto{\pgfqpoint{1.592984in}{1.378050in}}%
\pgfpathlineto{\pgfqpoint{1.596121in}{1.381846in}}%
\pgfpathlineto{\pgfqpoint{1.599009in}{1.385687in}}%
\pgfpathlineto{\pgfqpoint{1.601644in}{1.389569in}}%
\pgfpathlineto{\pgfqpoint{1.604025in}{1.393487in}}%
\pgfpathclose%
\pgfusepath{fill}%
\end{pgfscope}%
\begin{pgfscope}%
\pgfpathrectangle{\pgfqpoint{0.329460in}{0.284240in}}{\pgfqpoint{1.989680in}{1.989680in}}%
\pgfusepath{clip}%
\pgfsetbuttcap%
\pgfsetroundjoin%
\definecolor{currentfill}{rgb}{0.271305,0.019942,0.347269}%
\pgfsetfillcolor{currentfill}%
\pgfsetlinewidth{0.000000pt}%
\definecolor{currentstroke}{rgb}{0.000000,0.000000,0.000000}%
\pgfsetstrokecolor{currentstroke}%
\pgfsetdash{}{0pt}%
\pgfpathmoveto{\pgfqpoint{1.011493in}{0.854083in}}%
\pgfpathlineto{\pgfqpoint{1.009106in}{0.850472in}}%
\pgfpathlineto{\pgfqpoint{1.006714in}{0.847044in}}%
\pgfpathlineto{\pgfqpoint{1.004319in}{0.843802in}}%
\pgfpathlineto{\pgfqpoint{1.001919in}{0.840750in}}%
\pgfpathlineto{\pgfqpoint{0.990434in}{0.846772in}}%
\pgfpathlineto{\pgfqpoint{0.979340in}{0.852981in}}%
\pgfpathlineto{\pgfqpoint{0.968648in}{0.859367in}}%
\pgfpathlineto{\pgfqpoint{0.958367in}{0.865925in}}%
\pgfpathlineto{\pgfqpoint{0.961056in}{0.868793in}}%
\pgfpathlineto{\pgfqpoint{0.963740in}{0.871851in}}%
\pgfpathlineto{\pgfqpoint{0.966420in}{0.875095in}}%
\pgfpathlineto{\pgfqpoint{0.969096in}{0.878521in}}%
\pgfpathlineto{\pgfqpoint{0.979105in}{0.872155in}}%
\pgfpathlineto{\pgfqpoint{0.989515in}{0.865955in}}%
\pgfpathlineto{\pgfqpoint{1.000314in}{0.859929in}}%
\pgfpathlineto{\pgfqpoint{1.011493in}{0.854083in}}%
\pgfpathclose%
\pgfusepath{fill}%
\end{pgfscope}%
\begin{pgfscope}%
\pgfpathrectangle{\pgfqpoint{0.329460in}{0.284240in}}{\pgfqpoint{1.989680in}{1.989680in}}%
\pgfusepath{clip}%
\pgfsetbuttcap%
\pgfsetroundjoin%
\definecolor{currentfill}{rgb}{0.412913,0.803041,0.357269}%
\pgfsetfillcolor{currentfill}%
\pgfsetlinewidth{0.000000pt}%
\definecolor{currentstroke}{rgb}{0.000000,0.000000,0.000000}%
\pgfsetstrokecolor{currentstroke}%
\pgfsetdash{}{0pt}%
\pgfpathmoveto{\pgfqpoint{1.531364in}{1.586691in}}%
\pgfpathlineto{\pgfqpoint{1.534771in}{1.581346in}}%
\pgfpathlineto{\pgfqpoint{1.538175in}{1.575928in}}%
\pgfpathlineto{\pgfqpoint{1.541577in}{1.570438in}}%
\pgfpathlineto{\pgfqpoint{1.544975in}{1.564878in}}%
\pgfpathlineto{\pgfqpoint{1.545496in}{1.561942in}}%
\pgfpathlineto{\pgfqpoint{1.545824in}{1.558997in}}%
\pgfpathlineto{\pgfqpoint{1.545958in}{1.556048in}}%
\pgfpathlineto{\pgfqpoint{1.545898in}{1.553096in}}%
\pgfpathlineto{\pgfqpoint{1.542477in}{1.558874in}}%
\pgfpathlineto{\pgfqpoint{1.539053in}{1.564581in}}%
\pgfpathlineto{\pgfqpoint{1.535626in}{1.570216in}}%
\pgfpathlineto{\pgfqpoint{1.532197in}{1.575778in}}%
\pgfpathlineto{\pgfqpoint{1.532259in}{1.578511in}}%
\pgfpathlineto{\pgfqpoint{1.532140in}{1.581243in}}%
\pgfpathlineto{\pgfqpoint{1.531842in}{1.583971in}}%
\pgfpathlineto{\pgfqpoint{1.531364in}{1.586691in}}%
\pgfpathclose%
\pgfusepath{fill}%
\end{pgfscope}%
\begin{pgfscope}%
\pgfpathrectangle{\pgfqpoint{0.329460in}{0.284240in}}{\pgfqpoint{1.989680in}{1.989680in}}%
\pgfusepath{clip}%
\pgfsetbuttcap%
\pgfsetroundjoin%
\definecolor{currentfill}{rgb}{0.280255,0.165693,0.476498}%
\pgfsetfillcolor{currentfill}%
\pgfsetlinewidth{0.000000pt}%
\definecolor{currentstroke}{rgb}{0.000000,0.000000,0.000000}%
\pgfsetstrokecolor{currentstroke}%
\pgfsetdash{}{0pt}%
\pgfpathmoveto{\pgfqpoint{1.058641in}{0.958638in}}%
\pgfpathlineto{\pgfqpoint{1.056304in}{0.952181in}}%
\pgfpathlineto{\pgfqpoint{1.053964in}{0.945831in}}%
\pgfpathlineto{\pgfqpoint{1.051624in}{0.939593in}}%
\pgfpathlineto{\pgfqpoint{1.049281in}{0.933471in}}%
\pgfpathlineto{\pgfqpoint{1.039315in}{0.938605in}}%
\pgfpathlineto{\pgfqpoint{1.029684in}{0.943898in}}%
\pgfpathlineto{\pgfqpoint{1.020396in}{0.949346in}}%
\pgfpathlineto{\pgfqpoint{1.011461in}{0.954940in}}%
\pgfpathlineto{\pgfqpoint{1.014089in}{0.960877in}}%
\pgfpathlineto{\pgfqpoint{1.016714in}{0.966929in}}%
\pgfpathlineto{\pgfqpoint{1.019338in}{0.973093in}}%
\pgfpathlineto{\pgfqpoint{1.021961in}{0.979365in}}%
\pgfpathlineto{\pgfqpoint{1.030628in}{0.973963in}}%
\pgfpathlineto{\pgfqpoint{1.039636in}{0.968704in}}%
\pgfpathlineto{\pgfqpoint{1.048977in}{0.963594in}}%
\pgfpathlineto{\pgfqpoint{1.058641in}{0.958638in}}%
\pgfpathclose%
\pgfusepath{fill}%
\end{pgfscope}%
\begin{pgfscope}%
\pgfpathrectangle{\pgfqpoint{0.329460in}{0.284240in}}{\pgfqpoint{1.989680in}{1.989680in}}%
\pgfusepath{clip}%
\pgfsetbuttcap%
\pgfsetroundjoin%
\definecolor{currentfill}{rgb}{0.276194,0.190074,0.493001}%
\pgfsetfillcolor{currentfill}%
\pgfsetlinewidth{0.000000pt}%
\definecolor{currentstroke}{rgb}{0.000000,0.000000,0.000000}%
\pgfsetstrokecolor{currentstroke}%
\pgfsetdash{}{0pt}%
\pgfpathmoveto{\pgfqpoint{1.920674in}{0.979973in}}%
\pgfpathlineto{\pgfqpoint{1.924127in}{0.986379in}}%
\pgfpathlineto{\pgfqpoint{1.927594in}{0.993141in}}%
\pgfpathlineto{\pgfqpoint{1.931076in}{1.000264in}}%
\pgfpathlineto{\pgfqpoint{1.934573in}{1.007756in}}%
\pgfpathlineto{\pgfqpoint{1.926904in}{0.998160in}}%
\pgfpathlineto{\pgfqpoint{1.918624in}{0.988681in}}%
\pgfpathlineto{\pgfqpoint{1.909740in}{0.979332in}}%
\pgfpathlineto{\pgfqpoint{1.900259in}{0.970121in}}%
\pgfpathlineto{\pgfqpoint{1.896948in}{0.962813in}}%
\pgfpathlineto{\pgfqpoint{1.893653in}{0.955875in}}%
\pgfpathlineto{\pgfqpoint{1.890371in}{0.949300in}}%
\pgfpathlineto{\pgfqpoint{1.887103in}{0.943083in}}%
\pgfpathlineto{\pgfqpoint{1.896375in}{0.952110in}}%
\pgfpathlineto{\pgfqpoint{1.905066in}{0.961274in}}%
\pgfpathlineto{\pgfqpoint{1.913167in}{0.970566in}}%
\pgfpathlineto{\pgfqpoint{1.920674in}{0.979973in}}%
\pgfpathclose%
\pgfusepath{fill}%
\end{pgfscope}%
\begin{pgfscope}%
\pgfpathrectangle{\pgfqpoint{0.329460in}{0.284240in}}{\pgfqpoint{1.989680in}{1.989680in}}%
\pgfusepath{clip}%
\pgfsetbuttcap%
\pgfsetroundjoin%
\definecolor{currentfill}{rgb}{0.263663,0.237631,0.518762}%
\pgfsetfillcolor{currentfill}%
\pgfsetlinewidth{0.000000pt}%
\definecolor{currentstroke}{rgb}{0.000000,0.000000,0.000000}%
\pgfsetstrokecolor{currentstroke}%
\pgfsetdash{}{0pt}%
\pgfpathmoveto{\pgfqpoint{1.666428in}{1.037604in}}%
\pgfpathlineto{\pgfqpoint{1.669100in}{1.030629in}}%
\pgfpathlineto{\pgfqpoint{1.671772in}{1.023736in}}%
\pgfpathlineto{\pgfqpoint{1.674445in}{1.016928in}}%
\pgfpathlineto{\pgfqpoint{1.677119in}{1.010209in}}%
\pgfpathlineto{\pgfqpoint{1.669020in}{1.004883in}}%
\pgfpathlineto{\pgfqpoint{1.660583in}{0.999689in}}%
\pgfpathlineto{\pgfqpoint{1.651818in}{0.994634in}}%
\pgfpathlineto{\pgfqpoint{1.642733in}{0.989723in}}%
\pgfpathlineto{\pgfqpoint{1.640333in}{0.996632in}}%
\pgfpathlineto{\pgfqpoint{1.637935in}{1.003629in}}%
\pgfpathlineto{\pgfqpoint{1.635537in}{1.010711in}}%
\pgfpathlineto{\pgfqpoint{1.633140in}{1.017875in}}%
\pgfpathlineto{\pgfqpoint{1.641934in}{1.022604in}}%
\pgfpathlineto{\pgfqpoint{1.650419in}{1.027472in}}%
\pgfpathlineto{\pgfqpoint{1.658586in}{1.032474in}}%
\pgfpathlineto{\pgfqpoint{1.666428in}{1.037604in}}%
\pgfpathclose%
\pgfusepath{fill}%
\end{pgfscope}%
\begin{pgfscope}%
\pgfpathrectangle{\pgfqpoint{0.329460in}{0.284240in}}{\pgfqpoint{1.989680in}{1.989680in}}%
\pgfusepath{clip}%
\pgfsetbuttcap%
\pgfsetroundjoin%
\definecolor{currentfill}{rgb}{0.267004,0.004874,0.329415}%
\pgfsetfillcolor{currentfill}%
\pgfsetlinewidth{0.000000pt}%
\definecolor{currentstroke}{rgb}{0.000000,0.000000,0.000000}%
\pgfsetstrokecolor{currentstroke}%
\pgfsetdash{}{0pt}%
\pgfpathmoveto{\pgfqpoint{1.763827in}{0.862577in}}%
\pgfpathlineto{\pgfqpoint{1.766599in}{0.860765in}}%
\pgfpathlineto{\pgfqpoint{1.769377in}{0.859168in}}%
\pgfpathlineto{\pgfqpoint{1.772161in}{0.857791in}}%
\pgfpathlineto{\pgfqpoint{1.774951in}{0.856639in}}%
\pgfpathlineto{\pgfqpoint{1.764522in}{0.849548in}}%
\pgfpathlineto{\pgfqpoint{1.753647in}{0.842631in}}%
\pgfpathlineto{\pgfqpoint{1.742336in}{0.835896in}}%
\pgfpathlineto{\pgfqpoint{1.730601in}{0.829349in}}%
\pgfpathlineto{\pgfqpoint{1.728093in}{0.830687in}}%
\pgfpathlineto{\pgfqpoint{1.725590in}{0.832249in}}%
\pgfpathlineto{\pgfqpoint{1.723093in}{0.834032in}}%
\pgfpathlineto{\pgfqpoint{1.720602in}{0.836030in}}%
\pgfpathlineto{\pgfqpoint{1.732037in}{0.842398in}}%
\pgfpathlineto{\pgfqpoint{1.743061in}{0.848950in}}%
\pgfpathlineto{\pgfqpoint{1.753660in}{0.855679in}}%
\pgfpathlineto{\pgfqpoint{1.763827in}{0.862577in}}%
\pgfpathclose%
\pgfusepath{fill}%
\end{pgfscope}%
\begin{pgfscope}%
\pgfpathrectangle{\pgfqpoint{0.329460in}{0.284240in}}{\pgfqpoint{1.989680in}{1.989680in}}%
\pgfusepath{clip}%
\pgfsetbuttcap%
\pgfsetroundjoin%
\definecolor{currentfill}{rgb}{0.281477,0.755203,0.432552}%
\pgfsetfillcolor{currentfill}%
\pgfsetlinewidth{0.000000pt}%
\definecolor{currentstroke}{rgb}{0.000000,0.000000,0.000000}%
\pgfsetstrokecolor{currentstroke}%
\pgfsetdash{}{0pt}%
\pgfpathmoveto{\pgfqpoint{1.146091in}{1.513885in}}%
\pgfpathlineto{\pgfqpoint{1.142728in}{1.507514in}}%
\pgfpathlineto{\pgfqpoint{1.139368in}{1.501083in}}%
\pgfpathlineto{\pgfqpoint{1.136010in}{1.494593in}}%
\pgfpathlineto{\pgfqpoint{1.132655in}{1.488048in}}%
\pgfpathlineto{\pgfqpoint{1.131519in}{1.491404in}}%
\pgfpathlineto{\pgfqpoint{1.130605in}{1.494775in}}%
\pgfpathlineto{\pgfqpoint{1.129911in}{1.498156in}}%
\pgfpathlineto{\pgfqpoint{1.129439in}{1.501544in}}%
\pgfpathlineto{\pgfqpoint{1.132837in}{1.507869in}}%
\pgfpathlineto{\pgfqpoint{1.136236in}{1.514137in}}%
\pgfpathlineto{\pgfqpoint{1.139639in}{1.520348in}}%
\pgfpathlineto{\pgfqpoint{1.143044in}{1.526500in}}%
\pgfpathlineto{\pgfqpoint{1.143494in}{1.523333in}}%
\pgfpathlineto{\pgfqpoint{1.144152in}{1.520173in}}%
\pgfpathlineto{\pgfqpoint{1.145018in}{1.517022in}}%
\pgfpathlineto{\pgfqpoint{1.146091in}{1.513885in}}%
\pgfpathclose%
\pgfusepath{fill}%
\end{pgfscope}%
\begin{pgfscope}%
\pgfpathrectangle{\pgfqpoint{0.329460in}{0.284240in}}{\pgfqpoint{1.989680in}{1.989680in}}%
\pgfusepath{clip}%
\pgfsetbuttcap%
\pgfsetroundjoin%
\definecolor{currentfill}{rgb}{0.855810,0.888601,0.097452}%
\pgfsetfillcolor{currentfill}%
\pgfsetlinewidth{0.000000pt}%
\definecolor{currentstroke}{rgb}{0.000000,0.000000,0.000000}%
\pgfsetstrokecolor{currentstroke}%
\pgfsetdash{}{0pt}%
\pgfpathmoveto{\pgfqpoint{1.379764in}{1.733971in}}%
\pgfpathlineto{\pgfqpoint{1.380943in}{1.732651in}}%
\pgfpathlineto{\pgfqpoint{1.382120in}{1.731222in}}%
\pgfpathlineto{\pgfqpoint{1.383296in}{1.729685in}}%
\pgfpathlineto{\pgfqpoint{1.384471in}{1.728042in}}%
\pgfpathlineto{\pgfqpoint{1.387344in}{1.727531in}}%
\pgfpathlineto{\pgfqpoint{1.390183in}{1.726978in}}%
\pgfpathlineto{\pgfqpoint{1.392983in}{1.726384in}}%
\pgfpathlineto{\pgfqpoint{1.395742in}{1.725748in}}%
\pgfpathlineto{\pgfqpoint{1.394169in}{1.727473in}}%
\pgfpathlineto{\pgfqpoint{1.392594in}{1.729091in}}%
\pgfpathlineto{\pgfqpoint{1.391017in}{1.730602in}}%
\pgfpathlineto{\pgfqpoint{1.389439in}{1.732004in}}%
\pgfpathlineto{\pgfqpoint{1.387071in}{1.732549in}}%
\pgfpathlineto{\pgfqpoint{1.384667in}{1.733059in}}%
\pgfpathlineto{\pgfqpoint{1.382231in}{1.733533in}}%
\pgfpathlineto{\pgfqpoint{1.379764in}{1.733971in}}%
\pgfpathclose%
\pgfusepath{fill}%
\end{pgfscope}%
\begin{pgfscope}%
\pgfpathrectangle{\pgfqpoint{0.329460in}{0.284240in}}{\pgfqpoint{1.989680in}{1.989680in}}%
\pgfusepath{clip}%
\pgfsetbuttcap%
\pgfsetroundjoin%
\definecolor{currentfill}{rgb}{0.166383,0.690856,0.496502}%
\pgfsetfillcolor{currentfill}%
\pgfsetlinewidth{0.000000pt}%
\definecolor{currentstroke}{rgb}{0.000000,0.000000,0.000000}%
\pgfsetstrokecolor{currentstroke}%
\pgfsetdash{}{0pt}%
\pgfpathmoveto{\pgfqpoint{1.126372in}{1.447249in}}%
\pgfpathlineto{\pgfqpoint{1.123124in}{1.440234in}}%
\pgfpathlineto{\pgfqpoint{1.119879in}{1.433174in}}%
\pgfpathlineto{\pgfqpoint{1.116636in}{1.426072in}}%
\pgfpathlineto{\pgfqpoint{1.113395in}{1.418929in}}%
\pgfpathlineto{\pgfqpoint{1.111156in}{1.422632in}}%
\pgfpathlineto{\pgfqpoint{1.109160in}{1.426364in}}%
\pgfpathlineto{\pgfqpoint{1.107409in}{1.430124in}}%
\pgfpathlineto{\pgfqpoint{1.105905in}{1.433906in}}%
\pgfpathlineto{\pgfqpoint{1.109239in}{1.440827in}}%
\pgfpathlineto{\pgfqpoint{1.112577in}{1.447709in}}%
\pgfpathlineto{\pgfqpoint{1.115917in}{1.454549in}}%
\pgfpathlineto{\pgfqpoint{1.119259in}{1.461345in}}%
\pgfpathlineto{\pgfqpoint{1.120690in}{1.457785in}}%
\pgfpathlineto{\pgfqpoint{1.122353in}{1.454246in}}%
\pgfpathlineto{\pgfqpoint{1.124248in}{1.450733in}}%
\pgfpathlineto{\pgfqpoint{1.126372in}{1.447249in}}%
\pgfpathclose%
\pgfusepath{fill}%
\end{pgfscope}%
\begin{pgfscope}%
\pgfpathrectangle{\pgfqpoint{0.329460in}{0.284240in}}{\pgfqpoint{1.989680in}{1.989680in}}%
\pgfusepath{clip}%
\pgfsetbuttcap%
\pgfsetroundjoin%
\definecolor{currentfill}{rgb}{0.147607,0.511733,0.557049}%
\pgfsetfillcolor{currentfill}%
\pgfsetlinewidth{0.000000pt}%
\definecolor{currentstroke}{rgb}{0.000000,0.000000,0.000000}%
\pgfsetstrokecolor{currentstroke}%
\pgfsetdash{}{0pt}%
\pgfpathmoveto{\pgfqpoint{1.093696in}{1.266185in}}%
\pgfpathlineto{\pgfqpoint{1.090828in}{1.258279in}}%
\pgfpathlineto{\pgfqpoint{1.087961in}{1.250375in}}%
\pgfpathlineto{\pgfqpoint{1.085095in}{1.242476in}}%
\pgfpathlineto{\pgfqpoint{1.082231in}{1.234583in}}%
\pgfpathlineto{\pgfqpoint{1.077058in}{1.238918in}}%
\pgfpathlineto{\pgfqpoint{1.072170in}{1.243330in}}%
\pgfpathlineto{\pgfqpoint{1.067569in}{1.247815in}}%
\pgfpathlineto{\pgfqpoint{1.063262in}{1.252367in}}%
\pgfpathlineto{\pgfqpoint{1.066320in}{1.260049in}}%
\pgfpathlineto{\pgfqpoint{1.069379in}{1.267738in}}%
\pgfpathlineto{\pgfqpoint{1.072440in}{1.275432in}}%
\pgfpathlineto{\pgfqpoint{1.075503in}{1.283127in}}%
\pgfpathlineto{\pgfqpoint{1.079636in}{1.278790in}}%
\pgfpathlineto{\pgfqpoint{1.084048in}{1.274517in}}%
\pgfpathlineto{\pgfqpoint{1.088737in}{1.270314in}}%
\pgfpathlineto{\pgfqpoint{1.093696in}{1.266185in}}%
\pgfpathclose%
\pgfusepath{fill}%
\end{pgfscope}%
\begin{pgfscope}%
\pgfpathrectangle{\pgfqpoint{0.329460in}{0.284240in}}{\pgfqpoint{1.989680in}{1.989680in}}%
\pgfusepath{clip}%
\pgfsetbuttcap%
\pgfsetroundjoin%
\definecolor{currentfill}{rgb}{0.855810,0.888601,0.097452}%
\pgfsetfillcolor{currentfill}%
\pgfsetlinewidth{0.000000pt}%
\definecolor{currentstroke}{rgb}{0.000000,0.000000,0.000000}%
\pgfsetstrokecolor{currentstroke}%
\pgfsetdash{}{0pt}%
\pgfpathmoveto{\pgfqpoint{1.310862in}{1.731490in}}%
\pgfpathlineto{\pgfqpoint{1.309198in}{1.730067in}}%
\pgfpathlineto{\pgfqpoint{1.307536in}{1.728535in}}%
\pgfpathlineto{\pgfqpoint{1.305876in}{1.726896in}}%
\pgfpathlineto{\pgfqpoint{1.304217in}{1.725149in}}%
\pgfpathlineto{\pgfqpoint{1.306938in}{1.725821in}}%
\pgfpathlineto{\pgfqpoint{1.309702in}{1.726452in}}%
\pgfpathlineto{\pgfqpoint{1.312506in}{1.727042in}}%
\pgfpathlineto{\pgfqpoint{1.315348in}{1.727590in}}%
\pgfpathlineto{\pgfqpoint{1.316614in}{1.729249in}}%
\pgfpathlineto{\pgfqpoint{1.317880in}{1.730802in}}%
\pgfpathlineto{\pgfqpoint{1.319148in}{1.732247in}}%
\pgfpathlineto{\pgfqpoint{1.320417in}{1.733583in}}%
\pgfpathlineto{\pgfqpoint{1.317977in}{1.733113in}}%
\pgfpathlineto{\pgfqpoint{1.315570in}{1.732607in}}%
\pgfpathlineto{\pgfqpoint{1.313197in}{1.732066in}}%
\pgfpathlineto{\pgfqpoint{1.310862in}{1.731490in}}%
\pgfpathclose%
\pgfusepath{fill}%
\end{pgfscope}%
\begin{pgfscope}%
\pgfpathrectangle{\pgfqpoint{0.329460in}{0.284240in}}{\pgfqpoint{1.989680in}{1.989680in}}%
\pgfusepath{clip}%
\pgfsetbuttcap%
\pgfsetroundjoin%
\definecolor{currentfill}{rgb}{0.814576,0.883393,0.110347}%
\pgfsetfillcolor{currentfill}%
\pgfsetlinewidth{0.000000pt}%
\definecolor{currentstroke}{rgb}{0.000000,0.000000,0.000000}%
\pgfsetstrokecolor{currentstroke}%
\pgfsetdash{}{0pt}%
\pgfpathmoveto{\pgfqpoint{1.416033in}{1.719269in}}%
\pgfpathlineto{\pgfqpoint{1.418323in}{1.717208in}}%
\pgfpathlineto{\pgfqpoint{1.420611in}{1.715041in}}%
\pgfpathlineto{\pgfqpoint{1.422897in}{1.712771in}}%
\pgfpathlineto{\pgfqpoint{1.425181in}{1.710398in}}%
\pgfpathlineto{\pgfqpoint{1.427780in}{1.709287in}}%
\pgfpathlineto{\pgfqpoint{1.430304in}{1.708137in}}%
\pgfpathlineto{\pgfqpoint{1.432750in}{1.706951in}}%
\pgfpathlineto{\pgfqpoint{1.435117in}{1.705729in}}%
\pgfpathlineto{\pgfqpoint{1.432525in}{1.708248in}}%
\pgfpathlineto{\pgfqpoint{1.429930in}{1.710664in}}%
\pgfpathlineto{\pgfqpoint{1.427333in}{1.712976in}}%
\pgfpathlineto{\pgfqpoint{1.424733in}{1.715184in}}%
\pgfpathlineto{\pgfqpoint{1.422661in}{1.716253in}}%
\pgfpathlineto{\pgfqpoint{1.420519in}{1.717291in}}%
\pgfpathlineto{\pgfqpoint{1.418309in}{1.718297in}}%
\pgfpathlineto{\pgfqpoint{1.416033in}{1.719269in}}%
\pgfpathclose%
\pgfusepath{fill}%
\end{pgfscope}%
\begin{pgfscope}%
\pgfpathrectangle{\pgfqpoint{0.329460in}{0.284240in}}{\pgfqpoint{1.989680in}{1.989680in}}%
\pgfusepath{clip}%
\pgfsetbuttcap%
\pgfsetroundjoin%
\definecolor{currentfill}{rgb}{0.274128,0.199721,0.498911}%
\pgfsetfillcolor{currentfill}%
\pgfsetlinewidth{0.000000pt}%
\definecolor{currentstroke}{rgb}{0.000000,0.000000,0.000000}%
\pgfsetstrokecolor{currentstroke}%
\pgfsetdash{}{0pt}%
\pgfpathmoveto{\pgfqpoint{1.067980in}{0.985483in}}%
\pgfpathlineto{\pgfqpoint{1.065647in}{0.978627in}}%
\pgfpathlineto{\pgfqpoint{1.063313in}{0.971865in}}%
\pgfpathlineto{\pgfqpoint{1.060978in}{0.965201in}}%
\pgfpathlineto{\pgfqpoint{1.058641in}{0.958638in}}%
\pgfpathlineto{\pgfqpoint{1.048977in}{0.963594in}}%
\pgfpathlineto{\pgfqpoint{1.039636in}{0.968704in}}%
\pgfpathlineto{\pgfqpoint{1.030628in}{0.973963in}}%
\pgfpathlineto{\pgfqpoint{1.021961in}{0.979365in}}%
\pgfpathlineto{\pgfqpoint{1.024582in}{0.985742in}}%
\pgfpathlineto{\pgfqpoint{1.027202in}{0.992220in}}%
\pgfpathlineto{\pgfqpoint{1.029821in}{0.998797in}}%
\pgfpathlineto{\pgfqpoint{1.032439in}{1.005468in}}%
\pgfpathlineto{\pgfqpoint{1.040838in}{1.000260in}}%
\pgfpathlineto{\pgfqpoint{1.049567in}{0.995189in}}%
\pgfpathlineto{\pgfqpoint{1.058618in}{0.990262in}}%
\pgfpathlineto{\pgfqpoint{1.067980in}{0.985483in}}%
\pgfpathclose%
\pgfusepath{fill}%
\end{pgfscope}%
\begin{pgfscope}%
\pgfpathrectangle{\pgfqpoint{0.329460in}{0.284240in}}{\pgfqpoint{1.989680in}{1.989680in}}%
\pgfusepath{clip}%
\pgfsetbuttcap%
\pgfsetroundjoin%
\definecolor{currentfill}{rgb}{0.233603,0.313828,0.543914}%
\pgfsetfillcolor{currentfill}%
\pgfsetlinewidth{0.000000pt}%
\definecolor{currentstroke}{rgb}{0.000000,0.000000,0.000000}%
\pgfsetstrokecolor{currentstroke}%
\pgfsetdash{}{0pt}%
\pgfpathmoveto{\pgfqpoint{0.760585in}{1.032836in}}%
\pgfpathlineto{\pgfqpoint{0.757042in}{1.042225in}}%
\pgfpathlineto{\pgfqpoint{0.753481in}{1.052023in}}%
\pgfpathlineto{\pgfqpoint{0.749902in}{1.062235in}}%
\pgfpathlineto{\pgfqpoint{0.746305in}{1.072869in}}%
\pgfpathlineto{\pgfqpoint{0.738375in}{1.082835in}}%
\pgfpathlineto{\pgfqpoint{0.731093in}{1.092910in}}%
\pgfpathlineto{\pgfqpoint{0.724463in}{1.103082in}}%
\pgfpathlineto{\pgfqpoint{0.718490in}{1.113341in}}%
\pgfpathlineto{\pgfqpoint{0.722229in}{1.102534in}}%
\pgfpathlineto{\pgfqpoint{0.725949in}{1.092146in}}%
\pgfpathlineto{\pgfqpoint{0.729651in}{1.082172in}}%
\pgfpathlineto{\pgfqpoint{0.733335in}{1.072603in}}%
\pgfpathlineto{\pgfqpoint{0.739191in}{1.062522in}}%
\pgfpathlineto{\pgfqpoint{0.745687in}{1.052526in}}%
\pgfpathlineto{\pgfqpoint{0.752820in}{1.042627in}}%
\pgfpathlineto{\pgfqpoint{0.760585in}{1.032836in}}%
\pgfpathclose%
\pgfusepath{fill}%
\end{pgfscope}%
\begin{pgfscope}%
\pgfpathrectangle{\pgfqpoint{0.329460in}{0.284240in}}{\pgfqpoint{1.989680in}{1.989680in}}%
\pgfusepath{clip}%
\pgfsetbuttcap%
\pgfsetroundjoin%
\definecolor{currentfill}{rgb}{0.699415,0.867117,0.175971}%
\pgfsetfillcolor{currentfill}%
\pgfsetlinewidth{0.000000pt}%
\definecolor{currentstroke}{rgb}{0.000000,0.000000,0.000000}%
\pgfsetstrokecolor{currentstroke}%
\pgfsetdash{}{0pt}%
\pgfpathmoveto{\pgfqpoint{1.463228in}{1.682356in}}%
\pgfpathlineto{\pgfqpoint{1.466301in}{1.678996in}}%
\pgfpathlineto{\pgfqpoint{1.469371in}{1.675540in}}%
\pgfpathlineto{\pgfqpoint{1.472439in}{1.671990in}}%
\pgfpathlineto{\pgfqpoint{1.475504in}{1.668345in}}%
\pgfpathlineto{\pgfqpoint{1.477446in}{1.666482in}}%
\pgfpathlineto{\pgfqpoint{1.479264in}{1.664591in}}%
\pgfpathlineto{\pgfqpoint{1.480956in}{1.662673in}}%
\pgfpathlineto{\pgfqpoint{1.482520in}{1.660730in}}%
\pgfpathlineto{\pgfqpoint{1.479278in}{1.664567in}}%
\pgfpathlineto{\pgfqpoint{1.476034in}{1.668310in}}%
\pgfpathlineto{\pgfqpoint{1.472786in}{1.671958in}}%
\pgfpathlineto{\pgfqpoint{1.469536in}{1.675510in}}%
\pgfpathlineto{\pgfqpoint{1.468131in}{1.677257in}}%
\pgfpathlineto{\pgfqpoint{1.466610in}{1.678981in}}%
\pgfpathlineto{\pgfqpoint{1.464975in}{1.680681in}}%
\pgfpathlineto{\pgfqpoint{1.463228in}{1.682356in}}%
\pgfpathclose%
\pgfusepath{fill}%
\end{pgfscope}%
\begin{pgfscope}%
\pgfpathrectangle{\pgfqpoint{0.329460in}{0.284240in}}{\pgfqpoint{1.989680in}{1.989680in}}%
\pgfusepath{clip}%
\pgfsetbuttcap%
\pgfsetroundjoin%
\definecolor{currentfill}{rgb}{0.636902,0.856542,0.216620}%
\pgfsetfillcolor{currentfill}%
\pgfsetlinewidth{0.000000pt}%
\definecolor{currentstroke}{rgb}{0.000000,0.000000,0.000000}%
\pgfsetstrokecolor{currentstroke}%
\pgfsetdash{}{0pt}%
\pgfpathmoveto{\pgfqpoint{1.482520in}{1.660730in}}%
\pgfpathlineto{\pgfqpoint{1.485759in}{1.656800in}}%
\pgfpathlineto{\pgfqpoint{1.488995in}{1.652780in}}%
\pgfpathlineto{\pgfqpoint{1.492229in}{1.648669in}}%
\pgfpathlineto{\pgfqpoint{1.495459in}{1.644471in}}%
\pgfpathlineto{\pgfqpoint{1.497040in}{1.642305in}}%
\pgfpathlineto{\pgfqpoint{1.498478in}{1.640116in}}%
\pgfpathlineto{\pgfqpoint{1.499770in}{1.637905in}}%
\pgfpathlineto{\pgfqpoint{1.500915in}{1.635676in}}%
\pgfpathlineto{\pgfqpoint{1.497557in}{1.640078in}}%
\pgfpathlineto{\pgfqpoint{1.494197in}{1.644391in}}%
\pgfpathlineto{\pgfqpoint{1.490834in}{1.648615in}}%
\pgfpathlineto{\pgfqpoint{1.487468in}{1.652747in}}%
\pgfpathlineto{\pgfqpoint{1.486430in}{1.654770in}}%
\pgfpathlineto{\pgfqpoint{1.485259in}{1.656776in}}%
\pgfpathlineto{\pgfqpoint{1.483955in}{1.658764in}}%
\pgfpathlineto{\pgfqpoint{1.482520in}{1.660730in}}%
\pgfpathclose%
\pgfusepath{fill}%
\end{pgfscope}%
\begin{pgfscope}%
\pgfpathrectangle{\pgfqpoint{0.329460in}{0.284240in}}{\pgfqpoint{1.989680in}{1.989680in}}%
\pgfusepath{clip}%
\pgfsetbuttcap%
\pgfsetroundjoin%
\definecolor{currentfill}{rgb}{0.814576,0.883393,0.110347}%
\pgfsetfillcolor{currentfill}%
\pgfsetlinewidth{0.000000pt}%
\definecolor{currentstroke}{rgb}{0.000000,0.000000,0.000000}%
\pgfsetstrokecolor{currentstroke}%
\pgfsetdash{}{0pt}%
\pgfpathmoveto{\pgfqpoint{1.275861in}{1.714208in}}%
\pgfpathlineto{\pgfqpoint{1.273198in}{1.711966in}}%
\pgfpathlineto{\pgfqpoint{1.270537in}{1.709618in}}%
\pgfpathlineto{\pgfqpoint{1.267879in}{1.707167in}}%
\pgfpathlineto{\pgfqpoint{1.265223in}{1.704613in}}%
\pgfpathlineto{\pgfqpoint{1.267517in}{1.705866in}}%
\pgfpathlineto{\pgfqpoint{1.269893in}{1.707084in}}%
\pgfpathlineto{\pgfqpoint{1.272348in}{1.708267in}}%
\pgfpathlineto{\pgfqpoint{1.274880in}{1.709412in}}%
\pgfpathlineto{\pgfqpoint{1.277236in}{1.711816in}}%
\pgfpathlineto{\pgfqpoint{1.279594in}{1.714117in}}%
\pgfpathlineto{\pgfqpoint{1.281954in}{1.716314in}}%
\pgfpathlineto{\pgfqpoint{1.284316in}{1.718407in}}%
\pgfpathlineto{\pgfqpoint{1.282098in}{1.717405in}}%
\pgfpathlineto{\pgfqpoint{1.279949in}{1.716370in}}%
\pgfpathlineto{\pgfqpoint{1.277869in}{1.715304in}}%
\pgfpathlineto{\pgfqpoint{1.275861in}{1.714208in}}%
\pgfpathclose%
\pgfusepath{fill}%
\end{pgfscope}%
\begin{pgfscope}%
\pgfpathrectangle{\pgfqpoint{0.329460in}{0.284240in}}{\pgfqpoint{1.989680in}{1.989680in}}%
\pgfusepath{clip}%
\pgfsetbuttcap%
\pgfsetroundjoin%
\definecolor{currentfill}{rgb}{0.195860,0.395433,0.555276}%
\pgfsetfillcolor{currentfill}%
\pgfsetlinewidth{0.000000pt}%
\definecolor{currentstroke}{rgb}{0.000000,0.000000,0.000000}%
\pgfsetstrokecolor{currentstroke}%
\pgfsetdash{}{0pt}%
\pgfpathmoveto{\pgfqpoint{1.084715in}{1.153942in}}%
\pgfpathlineto{\pgfqpoint{1.082100in}{1.146031in}}%
\pgfpathlineto{\pgfqpoint{1.079485in}{1.138154in}}%
\pgfpathlineto{\pgfqpoint{1.076871in}{1.130313in}}%
\pgfpathlineto{\pgfqpoint{1.074258in}{1.122511in}}%
\pgfpathlineto{\pgfqpoint{1.067223in}{1.127068in}}%
\pgfpathlineto{\pgfqpoint{1.060487in}{1.131734in}}%
\pgfpathlineto{\pgfqpoint{1.054055in}{1.136502in}}%
\pgfpathlineto{\pgfqpoint{1.047934in}{1.141369in}}%
\pgfpathlineto{\pgfqpoint{1.050788in}{1.148971in}}%
\pgfpathlineto{\pgfqpoint{1.053642in}{1.156612in}}%
\pgfpathlineto{\pgfqpoint{1.056497in}{1.164289in}}%
\pgfpathlineto{\pgfqpoint{1.059353in}{1.172001in}}%
\pgfpathlineto{\pgfqpoint{1.065252in}{1.167340in}}%
\pgfpathlineto{\pgfqpoint{1.071449in}{1.162773in}}%
\pgfpathlineto{\pgfqpoint{1.077939in}{1.158306in}}%
\pgfpathlineto{\pgfqpoint{1.084715in}{1.153942in}}%
\pgfpathclose%
\pgfusepath{fill}%
\end{pgfscope}%
\begin{pgfscope}%
\pgfpathrectangle{\pgfqpoint{0.329460in}{0.284240in}}{\pgfqpoint{1.989680in}{1.989680in}}%
\pgfusepath{clip}%
\pgfsetbuttcap%
\pgfsetroundjoin%
\definecolor{currentfill}{rgb}{0.268510,0.009605,0.335427}%
\pgfsetfillcolor{currentfill}%
\pgfsetlinewidth{0.000000pt}%
\definecolor{currentstroke}{rgb}{0.000000,0.000000,0.000000}%
\pgfsetstrokecolor{currentstroke}%
\pgfsetdash{}{0pt}%
\pgfpathmoveto{\pgfqpoint{1.001919in}{0.840750in}}%
\pgfpathlineto{\pgfqpoint{0.999515in}{0.837893in}}%
\pgfpathlineto{\pgfqpoint{0.997107in}{0.835235in}}%
\pgfpathlineto{\pgfqpoint{0.994694in}{0.832779in}}%
\pgfpathlineto{\pgfqpoint{0.992276in}{0.830531in}}%
\pgfpathlineto{\pgfqpoint{0.980483in}{0.836728in}}%
\pgfpathlineto{\pgfqpoint{0.969092in}{0.843117in}}%
\pgfpathlineto{\pgfqpoint{0.958116in}{0.849689in}}%
\pgfpathlineto{\pgfqpoint{0.947564in}{0.856437in}}%
\pgfpathlineto{\pgfqpoint{0.950272in}{0.858503in}}%
\pgfpathlineto{\pgfqpoint{0.952976in}{0.860776in}}%
\pgfpathlineto{\pgfqpoint{0.955674in}{0.863251in}}%
\pgfpathlineto{\pgfqpoint{0.958367in}{0.865925in}}%
\pgfpathlineto{\pgfqpoint{0.968648in}{0.859367in}}%
\pgfpathlineto{\pgfqpoint{0.979340in}{0.852981in}}%
\pgfpathlineto{\pgfqpoint{0.990434in}{0.846772in}}%
\pgfpathlineto{\pgfqpoint{1.001919in}{0.840750in}}%
\pgfpathclose%
\pgfusepath{fill}%
\end{pgfscope}%
\begin{pgfscope}%
\pgfpathrectangle{\pgfqpoint{0.329460in}{0.284240in}}{\pgfqpoint{1.989680in}{1.989680in}}%
\pgfusepath{clip}%
\pgfsetbuttcap%
\pgfsetroundjoin%
\definecolor{currentfill}{rgb}{0.412913,0.803041,0.357269}%
\pgfsetfillcolor{currentfill}%
\pgfsetlinewidth{0.000000pt}%
\definecolor{currentstroke}{rgb}{0.000000,0.000000,0.000000}%
\pgfsetstrokecolor{currentstroke}%
\pgfsetdash{}{0pt}%
\pgfpathmoveto{\pgfqpoint{1.170385in}{1.573349in}}%
\pgfpathlineto{\pgfqpoint{1.166958in}{1.567739in}}%
\pgfpathlineto{\pgfqpoint{1.163534in}{1.562055in}}%
\pgfpathlineto{\pgfqpoint{1.160112in}{1.556300in}}%
\pgfpathlineto{\pgfqpoint{1.156693in}{1.550473in}}%
\pgfpathlineto{\pgfqpoint{1.156461in}{1.553424in}}%
\pgfpathlineto{\pgfqpoint{1.156422in}{1.556376in}}%
\pgfpathlineto{\pgfqpoint{1.156578in}{1.559325in}}%
\pgfpathlineto{\pgfqpoint{1.156927in}{1.562268in}}%
\pgfpathlineto{\pgfqpoint{1.160335in}{1.567876in}}%
\pgfpathlineto{\pgfqpoint{1.163747in}{1.573414in}}%
\pgfpathlineto{\pgfqpoint{1.167161in}{1.578881in}}%
\pgfpathlineto{\pgfqpoint{1.170578in}{1.584273in}}%
\pgfpathlineto{\pgfqpoint{1.170259in}{1.581547in}}%
\pgfpathlineto{\pgfqpoint{1.170121in}{1.578815in}}%
\pgfpathlineto{\pgfqpoint{1.170163in}{1.576082in}}%
\pgfpathlineto{\pgfqpoint{1.170385in}{1.573349in}}%
\pgfpathclose%
\pgfusepath{fill}%
\end{pgfscope}%
\begin{pgfscope}%
\pgfpathrectangle{\pgfqpoint{0.329460in}{0.284240in}}{\pgfqpoint{1.989680in}{1.989680in}}%
\pgfusepath{clip}%
\pgfsetbuttcap%
\pgfsetroundjoin%
\definecolor{currentfill}{rgb}{0.120081,0.622161,0.534946}%
\pgfsetfillcolor{currentfill}%
\pgfsetlinewidth{0.000000pt}%
\definecolor{currentstroke}{rgb}{0.000000,0.000000,0.000000}%
\pgfsetstrokecolor{currentstroke}%
\pgfsetdash{}{0pt}%
\pgfpathmoveto{\pgfqpoint{1.112388in}{1.374716in}}%
\pgfpathlineto{\pgfqpoint{1.109304in}{1.367190in}}%
\pgfpathlineto{\pgfqpoint{1.106222in}{1.359637in}}%
\pgfpathlineto{\pgfqpoint{1.103142in}{1.352059in}}%
\pgfpathlineto{\pgfqpoint{1.100064in}{1.344458in}}%
\pgfpathlineto{\pgfqpoint{1.096541in}{1.348425in}}%
\pgfpathlineto{\pgfqpoint{1.093279in}{1.352442in}}%
\pgfpathlineto{\pgfqpoint{1.090281in}{1.356506in}}%
\pgfpathlineto{\pgfqpoint{1.087548in}{1.360611in}}%
\pgfpathlineto{\pgfqpoint{1.090771in}{1.367995in}}%
\pgfpathlineto{\pgfqpoint{1.093996in}{1.375356in}}%
\pgfpathlineto{\pgfqpoint{1.097224in}{1.382693in}}%
\pgfpathlineto{\pgfqpoint{1.100454in}{1.390003in}}%
\pgfpathlineto{\pgfqpoint{1.103061in}{1.386117in}}%
\pgfpathlineto{\pgfqpoint{1.105921in}{1.382271in}}%
\pgfpathlineto{\pgfqpoint{1.109031in}{1.378469in}}%
\pgfpathlineto{\pgfqpoint{1.112388in}{1.374716in}}%
\pgfpathclose%
\pgfusepath{fill}%
\end{pgfscope}%
\begin{pgfscope}%
\pgfpathrectangle{\pgfqpoint{0.329460in}{0.284240in}}{\pgfqpoint{1.989680in}{1.989680in}}%
\pgfusepath{clip}%
\pgfsetbuttcap%
\pgfsetroundjoin%
\definecolor{currentfill}{rgb}{0.133743,0.548535,0.553541}%
\pgfsetfillcolor{currentfill}%
\pgfsetlinewidth{0.000000pt}%
\definecolor{currentstroke}{rgb}{0.000000,0.000000,0.000000}%
\pgfsetstrokecolor{currentstroke}%
\pgfsetdash{}{0pt}%
\pgfpathmoveto{\pgfqpoint{1.617896in}{1.317591in}}%
\pgfpathlineto{\pgfqpoint{1.621002in}{1.309961in}}%
\pgfpathlineto{\pgfqpoint{1.624106in}{1.302323in}}%
\pgfpathlineto{\pgfqpoint{1.627208in}{1.294680in}}%
\pgfpathlineto{\pgfqpoint{1.630308in}{1.287033in}}%
\pgfpathlineto{\pgfqpoint{1.626427in}{1.282642in}}%
\pgfpathlineto{\pgfqpoint{1.622263in}{1.278312in}}%
\pgfpathlineto{\pgfqpoint{1.617819in}{1.274047in}}%
\pgfpathlineto{\pgfqpoint{1.613101in}{1.269851in}}%
\pgfpathlineto{\pgfqpoint{1.610183in}{1.277710in}}%
\pgfpathlineto{\pgfqpoint{1.607265in}{1.285566in}}%
\pgfpathlineto{\pgfqpoint{1.604345in}{1.293415in}}%
\pgfpathlineto{\pgfqpoint{1.601423in}{1.301256in}}%
\pgfpathlineto{\pgfqpoint{1.605939in}{1.305244in}}%
\pgfpathlineto{\pgfqpoint{1.610192in}{1.309299in}}%
\pgfpathlineto{\pgfqpoint{1.614179in}{1.313416in}}%
\pgfpathlineto{\pgfqpoint{1.617896in}{1.317591in}}%
\pgfpathclose%
\pgfusepath{fill}%
\end{pgfscope}%
\begin{pgfscope}%
\pgfpathrectangle{\pgfqpoint{0.329460in}{0.284240in}}{\pgfqpoint{1.989680in}{1.989680in}}%
\pgfusepath{clip}%
\pgfsetbuttcap%
\pgfsetroundjoin%
\definecolor{currentfill}{rgb}{0.855810,0.888601,0.097452}%
\pgfsetfillcolor{currentfill}%
\pgfsetlinewidth{0.000000pt}%
\definecolor{currentstroke}{rgb}{0.000000,0.000000,0.000000}%
\pgfsetstrokecolor{currentstroke}%
\pgfsetdash{}{0pt}%
\pgfpathmoveto{\pgfqpoint{1.389439in}{1.732004in}}%
\pgfpathlineto{\pgfqpoint{1.391017in}{1.730602in}}%
\pgfpathlineto{\pgfqpoint{1.392594in}{1.729091in}}%
\pgfpathlineto{\pgfqpoint{1.394169in}{1.727473in}}%
\pgfpathlineto{\pgfqpoint{1.395742in}{1.725748in}}%
\pgfpathlineto{\pgfqpoint{1.398458in}{1.725072in}}%
\pgfpathlineto{\pgfqpoint{1.401128in}{1.724357in}}%
\pgfpathlineto{\pgfqpoint{1.403749in}{1.723602in}}%
\pgfpathlineto{\pgfqpoint{1.406318in}{1.722809in}}%
\pgfpathlineto{\pgfqpoint{1.404370in}{1.724639in}}%
\pgfpathlineto{\pgfqpoint{1.402421in}{1.726361in}}%
\pgfpathlineto{\pgfqpoint{1.400469in}{1.727977in}}%
\pgfpathlineto{\pgfqpoint{1.398516in}{1.729483in}}%
\pgfpathlineto{\pgfqpoint{1.396311in}{1.730163in}}%
\pgfpathlineto{\pgfqpoint{1.394062in}{1.730810in}}%
\pgfpathlineto{\pgfqpoint{1.391770in}{1.731424in}}%
\pgfpathlineto{\pgfqpoint{1.389439in}{1.732004in}}%
\pgfpathclose%
\pgfusepath{fill}%
\end{pgfscope}%
\begin{pgfscope}%
\pgfpathrectangle{\pgfqpoint{0.329460in}{0.284240in}}{\pgfqpoint{1.989680in}{1.989680in}}%
\pgfusepath{clip}%
\pgfsetbuttcap%
\pgfsetroundjoin%
\definecolor{currentfill}{rgb}{0.179019,0.433756,0.557430}%
\pgfsetfillcolor{currentfill}%
\pgfsetlinewidth{0.000000pt}%
\definecolor{currentstroke}{rgb}{0.000000,0.000000,0.000000}%
\pgfsetstrokecolor{currentstroke}%
\pgfsetdash{}{0pt}%
\pgfpathmoveto{\pgfqpoint{1.636392in}{1.207165in}}%
\pgfpathlineto{\pgfqpoint{1.639298in}{1.199391in}}%
\pgfpathlineto{\pgfqpoint{1.642203in}{1.191641in}}%
\pgfpathlineto{\pgfqpoint{1.645107in}{1.183916in}}%
\pgfpathlineto{\pgfqpoint{1.648011in}{1.176219in}}%
\pgfpathlineto{\pgfqpoint{1.642382in}{1.171479in}}%
\pgfpathlineto{\pgfqpoint{1.636449in}{1.166828in}}%
\pgfpathlineto{\pgfqpoint{1.630219in}{1.162272in}}%
\pgfpathlineto{\pgfqpoint{1.623698in}{1.157816in}}%
\pgfpathlineto{\pgfqpoint{1.621024in}{1.165715in}}%
\pgfpathlineto{\pgfqpoint{1.618351in}{1.173642in}}%
\pgfpathlineto{\pgfqpoint{1.615676in}{1.181594in}}%
\pgfpathlineto{\pgfqpoint{1.613001in}{1.189569in}}%
\pgfpathlineto{\pgfqpoint{1.619274in}{1.193829in}}%
\pgfpathlineto{\pgfqpoint{1.625267in}{1.198185in}}%
\pgfpathlineto{\pgfqpoint{1.630975in}{1.202632in}}%
\pgfpathlineto{\pgfqpoint{1.636392in}{1.207165in}}%
\pgfpathclose%
\pgfusepath{fill}%
\end{pgfscope}%
\begin{pgfscope}%
\pgfpathrectangle{\pgfqpoint{0.329460in}{0.284240in}}{\pgfqpoint{1.989680in}{1.989680in}}%
\pgfusepath{clip}%
\pgfsetbuttcap%
\pgfsetroundjoin%
\definecolor{currentfill}{rgb}{0.762373,0.876424,0.137064}%
\pgfsetfillcolor{currentfill}%
\pgfsetlinewidth{0.000000pt}%
\definecolor{currentstroke}{rgb}{0.000000,0.000000,0.000000}%
\pgfsetstrokecolor{currentstroke}%
\pgfsetdash{}{0pt}%
\pgfpathmoveto{\pgfqpoint{1.443740in}{1.700506in}}%
\pgfpathlineto{\pgfqpoint{1.446598in}{1.697721in}}%
\pgfpathlineto{\pgfqpoint{1.449455in}{1.694836in}}%
\pgfpathlineto{\pgfqpoint{1.452308in}{1.691851in}}%
\pgfpathlineto{\pgfqpoint{1.455160in}{1.688768in}}%
\pgfpathlineto{\pgfqpoint{1.457335in}{1.687211in}}%
\pgfpathlineto{\pgfqpoint{1.459406in}{1.685623in}}%
\pgfpathlineto{\pgfqpoint{1.461371in}{1.684004in}}%
\pgfpathlineto{\pgfqpoint{1.463228in}{1.682356in}}%
\pgfpathlineto{\pgfqpoint{1.460152in}{1.685619in}}%
\pgfpathlineto{\pgfqpoint{1.457074in}{1.688783in}}%
\pgfpathlineto{\pgfqpoint{1.453993in}{1.691847in}}%
\pgfpathlineto{\pgfqpoint{1.450910in}{1.694811in}}%
\pgfpathlineto{\pgfqpoint{1.449260in}{1.696274in}}%
\pgfpathlineto{\pgfqpoint{1.447514in}{1.697712in}}%
\pgfpathlineto{\pgfqpoint{1.445673in}{1.699123in}}%
\pgfpathlineto{\pgfqpoint{1.443740in}{1.700506in}}%
\pgfpathclose%
\pgfusepath{fill}%
\end{pgfscope}%
\begin{pgfscope}%
\pgfpathrectangle{\pgfqpoint{0.329460in}{0.284240in}}{\pgfqpoint{1.989680in}{1.989680in}}%
\pgfusepath{clip}%
\pgfsetbuttcap%
\pgfsetroundjoin%
\definecolor{currentfill}{rgb}{0.699415,0.867117,0.175971}%
\pgfsetfillcolor{currentfill}%
\pgfsetlinewidth{0.000000pt}%
\definecolor{currentstroke}{rgb}{0.000000,0.000000,0.000000}%
\pgfsetstrokecolor{currentstroke}%
\pgfsetdash{}{0pt}%
\pgfpathmoveto{\pgfqpoint{1.231687in}{1.673940in}}%
\pgfpathlineto{\pgfqpoint{1.228405in}{1.670344in}}%
\pgfpathlineto{\pgfqpoint{1.225125in}{1.666652in}}%
\pgfpathlineto{\pgfqpoint{1.221848in}{1.662864in}}%
\pgfpathlineto{\pgfqpoint{1.218573in}{1.658983in}}%
\pgfpathlineto{\pgfqpoint{1.220023in}{1.660947in}}%
\pgfpathlineto{\pgfqpoint{1.221601in}{1.662887in}}%
\pgfpathlineto{\pgfqpoint{1.223307in}{1.664802in}}%
\pgfpathlineto{\pgfqpoint{1.225139in}{1.666691in}}%
\pgfpathlineto{\pgfqpoint{1.228247in}{1.670377in}}%
\pgfpathlineto{\pgfqpoint{1.231358in}{1.673970in}}%
\pgfpathlineto{\pgfqpoint{1.234472in}{1.677467in}}%
\pgfpathlineto{\pgfqpoint{1.237589in}{1.680869in}}%
\pgfpathlineto{\pgfqpoint{1.235942in}{1.679171in}}%
\pgfpathlineto{\pgfqpoint{1.234408in}{1.677449in}}%
\pgfpathlineto{\pgfqpoint{1.232989in}{1.675705in}}%
\pgfpathlineto{\pgfqpoint{1.231687in}{1.673940in}}%
\pgfpathclose%
\pgfusepath{fill}%
\end{pgfscope}%
\begin{pgfscope}%
\pgfpathrectangle{\pgfqpoint{0.329460in}{0.284240in}}{\pgfqpoint{1.989680in}{1.989680in}}%
\pgfusepath{clip}%
\pgfsetbuttcap%
\pgfsetroundjoin%
\definecolor{currentfill}{rgb}{0.248629,0.278775,0.534556}%
\pgfsetfillcolor{currentfill}%
\pgfsetlinewidth{0.000000pt}%
\definecolor{currentstroke}{rgb}{0.000000,0.000000,0.000000}%
\pgfsetstrokecolor{currentstroke}%
\pgfsetdash{}{0pt}%
\pgfpathmoveto{\pgfqpoint{1.655745in}{1.066259in}}%
\pgfpathlineto{\pgfqpoint{1.658416in}{1.058988in}}%
\pgfpathlineto{\pgfqpoint{1.661086in}{1.051786in}}%
\pgfpathlineto{\pgfqpoint{1.663757in}{1.044657in}}%
\pgfpathlineto{\pgfqpoint{1.666428in}{1.037604in}}%
\pgfpathlineto{\pgfqpoint{1.658586in}{1.032474in}}%
\pgfpathlineto{\pgfqpoint{1.650419in}{1.027472in}}%
\pgfpathlineto{\pgfqpoint{1.641934in}{1.022604in}}%
\pgfpathlineto{\pgfqpoint{1.633140in}{1.017875in}}%
\pgfpathlineto{\pgfqpoint{1.630744in}{1.025118in}}%
\pgfpathlineto{\pgfqpoint{1.628348in}{1.032436in}}%
\pgfpathlineto{\pgfqpoint{1.625952in}{1.039826in}}%
\pgfpathlineto{\pgfqpoint{1.623557in}{1.047285in}}%
\pgfpathlineto{\pgfqpoint{1.632059in}{1.051833in}}%
\pgfpathlineto{\pgfqpoint{1.640263in}{1.056515in}}%
\pgfpathlineto{\pgfqpoint{1.648161in}{1.061325in}}%
\pgfpathlineto{\pgfqpoint{1.655745in}{1.066259in}}%
\pgfpathclose%
\pgfusepath{fill}%
\end{pgfscope}%
\begin{pgfscope}%
\pgfpathrectangle{\pgfqpoint{0.329460in}{0.284240in}}{\pgfqpoint{1.989680in}{1.989680in}}%
\pgfusepath{clip}%
\pgfsetbuttcap%
\pgfsetroundjoin%
\definecolor{currentfill}{rgb}{0.855810,0.888601,0.097452}%
\pgfsetfillcolor{currentfill}%
\pgfsetlinewidth{0.000000pt}%
\definecolor{currentstroke}{rgb}{0.000000,0.000000,0.000000}%
\pgfsetstrokecolor{currentstroke}%
\pgfsetdash{}{0pt}%
\pgfpathmoveto{\pgfqpoint{1.301938in}{1.728852in}}%
\pgfpathlineto{\pgfqpoint{1.299905in}{1.727319in}}%
\pgfpathlineto{\pgfqpoint{1.297874in}{1.725677in}}%
\pgfpathlineto{\pgfqpoint{1.295845in}{1.723928in}}%
\pgfpathlineto{\pgfqpoint{1.293818in}{1.722073in}}%
\pgfpathlineto{\pgfqpoint{1.296340in}{1.722899in}}%
\pgfpathlineto{\pgfqpoint{1.298915in}{1.723688in}}%
\pgfpathlineto{\pgfqpoint{1.301542in}{1.724438in}}%
\pgfpathlineto{\pgfqpoint{1.304217in}{1.725149in}}%
\pgfpathlineto{\pgfqpoint{1.305876in}{1.726896in}}%
\pgfpathlineto{\pgfqpoint{1.307536in}{1.728535in}}%
\pgfpathlineto{\pgfqpoint{1.309198in}{1.730067in}}%
\pgfpathlineto{\pgfqpoint{1.310862in}{1.731490in}}%
\pgfpathlineto{\pgfqpoint{1.308566in}{1.730880in}}%
\pgfpathlineto{\pgfqpoint{1.306312in}{1.730237in}}%
\pgfpathlineto{\pgfqpoint{1.304102in}{1.729560in}}%
\pgfpathlineto{\pgfqpoint{1.301938in}{1.728852in}}%
\pgfpathclose%
\pgfusepath{fill}%
\end{pgfscope}%
\begin{pgfscope}%
\pgfpathrectangle{\pgfqpoint{0.329460in}{0.284240in}}{\pgfqpoint{1.989680in}{1.989680in}}%
\pgfusepath{clip}%
\pgfsetbuttcap%
\pgfsetroundjoin%
\definecolor{currentfill}{rgb}{0.565498,0.842430,0.262877}%
\pgfsetfillcolor{currentfill}%
\pgfsetlinewidth{0.000000pt}%
\definecolor{currentstroke}{rgb}{0.000000,0.000000,0.000000}%
\pgfsetstrokecolor{currentstroke}%
\pgfsetdash{}{0pt}%
\pgfpathmoveto{\pgfqpoint{1.500915in}{1.635676in}}%
\pgfpathlineto{\pgfqpoint{1.504270in}{1.631187in}}%
\pgfpathlineto{\pgfqpoint{1.507622in}{1.626612in}}%
\pgfpathlineto{\pgfqpoint{1.510971in}{1.621953in}}%
\pgfpathlineto{\pgfqpoint{1.514317in}{1.617211in}}%
\pgfpathlineto{\pgfqpoint{1.515409in}{1.614756in}}%
\pgfpathlineto{\pgfqpoint{1.516339in}{1.612284in}}%
\pgfpathlineto{\pgfqpoint{1.517105in}{1.609799in}}%
\pgfpathlineto{\pgfqpoint{1.517706in}{1.607303in}}%
\pgfpathlineto{\pgfqpoint{1.514285in}{1.612256in}}%
\pgfpathlineto{\pgfqpoint{1.510861in}{1.617126in}}%
\pgfpathlineto{\pgfqpoint{1.507434in}{1.621912in}}%
\pgfpathlineto{\pgfqpoint{1.504004in}{1.626612in}}%
\pgfpathlineto{\pgfqpoint{1.503458in}{1.628895in}}%
\pgfpathlineto{\pgfqpoint{1.502760in}{1.631169in}}%
\pgfpathlineto{\pgfqpoint{1.501912in}{1.633430in}}%
\pgfpathlineto{\pgfqpoint{1.500915in}{1.635676in}}%
\pgfpathclose%
\pgfusepath{fill}%
\end{pgfscope}%
\begin{pgfscope}%
\pgfpathrectangle{\pgfqpoint{0.329460in}{0.284240in}}{\pgfqpoint{1.989680in}{1.989680in}}%
\pgfusepath{clip}%
\pgfsetbuttcap%
\pgfsetroundjoin%
\definecolor{currentfill}{rgb}{0.636902,0.856542,0.216620}%
\pgfsetfillcolor{currentfill}%
\pgfsetlinewidth{0.000000pt}%
\definecolor{currentstroke}{rgb}{0.000000,0.000000,0.000000}%
\pgfsetstrokecolor{currentstroke}%
\pgfsetdash{}{0pt}%
\pgfpathmoveto{\pgfqpoint{1.214098in}{1.650935in}}%
\pgfpathlineto{\pgfqpoint{1.210711in}{1.646757in}}%
\pgfpathlineto{\pgfqpoint{1.207327in}{1.642488in}}%
\pgfpathlineto{\pgfqpoint{1.203945in}{1.638128in}}%
\pgfpathlineto{\pgfqpoint{1.200567in}{1.633680in}}%
\pgfpathlineto{\pgfqpoint{1.201580in}{1.635924in}}%
\pgfpathlineto{\pgfqpoint{1.202742in}{1.638152in}}%
\pgfpathlineto{\pgfqpoint{1.204050in}{1.640360in}}%
\pgfpathlineto{\pgfqpoint{1.205503in}{1.642547in}}%
\pgfpathlineto{\pgfqpoint{1.208767in}{1.646790in}}%
\pgfpathlineto{\pgfqpoint{1.212033in}{1.650944in}}%
\pgfpathlineto{\pgfqpoint{1.215302in}{1.655009in}}%
\pgfpathlineto{\pgfqpoint{1.218573in}{1.658983in}}%
\pgfpathlineto{\pgfqpoint{1.217255in}{1.656998in}}%
\pgfpathlineto{\pgfqpoint{1.216069in}{1.654994in}}%
\pgfpathlineto{\pgfqpoint{1.215016in}{1.652972in}}%
\pgfpathlineto{\pgfqpoint{1.214098in}{1.650935in}}%
\pgfpathclose%
\pgfusepath{fill}%
\end{pgfscope}%
\begin{pgfscope}%
\pgfpathrectangle{\pgfqpoint{0.329460in}{0.284240in}}{\pgfqpoint{1.989680in}{1.989680in}}%
\pgfusepath{clip}%
\pgfsetbuttcap%
\pgfsetroundjoin%
\definecolor{currentfill}{rgb}{0.220124,0.725509,0.466226}%
\pgfsetfillcolor{currentfill}%
\pgfsetlinewidth{0.000000pt}%
\definecolor{currentstroke}{rgb}{0.000000,0.000000,0.000000}%
\pgfsetstrokecolor{currentstroke}%
\pgfsetdash{}{0pt}%
\pgfpathmoveto{\pgfqpoint{1.570741in}{1.491031in}}%
\pgfpathlineto{\pgfqpoint{1.574107in}{1.484480in}}%
\pgfpathlineto{\pgfqpoint{1.577471in}{1.477878in}}%
\pgfpathlineto{\pgfqpoint{1.580833in}{1.471225in}}%
\pgfpathlineto{\pgfqpoint{1.584192in}{1.464525in}}%
\pgfpathlineto{\pgfqpoint{1.582968in}{1.460948in}}%
\pgfpathlineto{\pgfqpoint{1.581512in}{1.457390in}}%
\pgfpathlineto{\pgfqpoint{1.579823in}{1.453855in}}%
\pgfpathlineto{\pgfqpoint{1.577903in}{1.450345in}}%
\pgfpathlineto{\pgfqpoint{1.574627in}{1.457265in}}%
\pgfpathlineto{\pgfqpoint{1.571349in}{1.464138in}}%
\pgfpathlineto{\pgfqpoint{1.568069in}{1.470960in}}%
\pgfpathlineto{\pgfqpoint{1.564786in}{1.477730in}}%
\pgfpathlineto{\pgfqpoint{1.566602in}{1.481022in}}%
\pgfpathlineto{\pgfqpoint{1.568201in}{1.484338in}}%
\pgfpathlineto{\pgfqpoint{1.569581in}{1.487676in}}%
\pgfpathlineto{\pgfqpoint{1.570741in}{1.491031in}}%
\pgfpathclose%
\pgfusepath{fill}%
\end{pgfscope}%
\begin{pgfscope}%
\pgfpathrectangle{\pgfqpoint{0.329460in}{0.284240in}}{\pgfqpoint{1.989680in}{1.989680in}}%
\pgfusepath{clip}%
\pgfsetbuttcap%
\pgfsetroundjoin%
\definecolor{currentfill}{rgb}{0.263663,0.237631,0.518762}%
\pgfsetfillcolor{currentfill}%
\pgfsetlinewidth{0.000000pt}%
\definecolor{currentstroke}{rgb}{0.000000,0.000000,0.000000}%
\pgfsetstrokecolor{currentstroke}%
\pgfsetdash{}{0pt}%
\pgfpathmoveto{\pgfqpoint{1.077304in}{1.013793in}}%
\pgfpathlineto{\pgfqpoint{1.074974in}{1.006590in}}%
\pgfpathlineto{\pgfqpoint{1.072644in}{0.999468in}}%
\pgfpathlineto{\pgfqpoint{1.070313in}{0.992432in}}%
\pgfpathlineto{\pgfqpoint{1.067980in}{0.985483in}}%
\pgfpathlineto{\pgfqpoint{1.058618in}{0.990262in}}%
\pgfpathlineto{\pgfqpoint{1.049567in}{0.995189in}}%
\pgfpathlineto{\pgfqpoint{1.040838in}{1.000260in}}%
\pgfpathlineto{\pgfqpoint{1.032439in}{1.005468in}}%
\pgfpathlineto{\pgfqpoint{1.035055in}{1.012231in}}%
\pgfpathlineto{\pgfqpoint{1.037672in}{1.019083in}}%
\pgfpathlineto{\pgfqpoint{1.040287in}{1.026019in}}%
\pgfpathlineto{\pgfqpoint{1.042902in}{1.033038in}}%
\pgfpathlineto{\pgfqpoint{1.051033in}{1.028022in}}%
\pgfpathlineto{\pgfqpoint{1.059483in}{1.023138in}}%
\pgfpathlineto{\pgfqpoint{1.068243in}{1.018394in}}%
\pgfpathlineto{\pgfqpoint{1.077304in}{1.013793in}}%
\pgfpathclose%
\pgfusepath{fill}%
\end{pgfscope}%
\begin{pgfscope}%
\pgfpathrectangle{\pgfqpoint{0.329460in}{0.284240in}}{\pgfqpoint{1.989680in}{1.989680in}}%
\pgfusepath{clip}%
\pgfsetbuttcap%
\pgfsetroundjoin%
\definecolor{currentfill}{rgb}{0.344074,0.780029,0.397381}%
\pgfsetfillcolor{currentfill}%
\pgfsetlinewidth{0.000000pt}%
\definecolor{currentstroke}{rgb}{0.000000,0.000000,0.000000}%
\pgfsetstrokecolor{currentstroke}%
\pgfsetdash{}{0pt}%
\pgfpathmoveto{\pgfqpoint{1.545898in}{1.553096in}}%
\pgfpathlineto{\pgfqpoint{1.549317in}{1.547250in}}%
\pgfpathlineto{\pgfqpoint{1.552733in}{1.541337in}}%
\pgfpathlineto{\pgfqpoint{1.556146in}{1.535359in}}%
\pgfpathlineto{\pgfqpoint{1.559556in}{1.529318in}}%
\pgfpathlineto{\pgfqpoint{1.559291in}{1.526148in}}%
\pgfpathlineto{\pgfqpoint{1.558818in}{1.522981in}}%
\pgfpathlineto{\pgfqpoint{1.558137in}{1.519822in}}%
\pgfpathlineto{\pgfqpoint{1.557248in}{1.516673in}}%
\pgfpathlineto{\pgfqpoint{1.553869in}{1.522934in}}%
\pgfpathlineto{\pgfqpoint{1.550487in}{1.529132in}}%
\pgfpathlineto{\pgfqpoint{1.547102in}{1.535264in}}%
\pgfpathlineto{\pgfqpoint{1.543715in}{1.541329in}}%
\pgfpathlineto{\pgfqpoint{1.544552in}{1.544259in}}%
\pgfpathlineto{\pgfqpoint{1.545195in}{1.547199in}}%
\pgfpathlineto{\pgfqpoint{1.545644in}{1.550146in}}%
\pgfpathlineto{\pgfqpoint{1.545898in}{1.553096in}}%
\pgfpathclose%
\pgfusepath{fill}%
\end{pgfscope}%
\begin{pgfscope}%
\pgfpathrectangle{\pgfqpoint{0.329460in}{0.284240in}}{\pgfqpoint{1.989680in}{1.989680in}}%
\pgfusepath{clip}%
\pgfsetbuttcap%
\pgfsetroundjoin%
\definecolor{currentfill}{rgb}{0.762373,0.876424,0.137064}%
\pgfsetfillcolor{currentfill}%
\pgfsetlinewidth{0.000000pt}%
\definecolor{currentstroke}{rgb}{0.000000,0.000000,0.000000}%
\pgfsetstrokecolor{currentstroke}%
\pgfsetdash{}{0pt}%
\pgfpathmoveto{\pgfqpoint{1.250081in}{1.693490in}}%
\pgfpathlineto{\pgfqpoint{1.246954in}{1.690485in}}%
\pgfpathlineto{\pgfqpoint{1.243830in}{1.687379in}}%
\pgfpathlineto{\pgfqpoint{1.240708in}{1.684173in}}%
\pgfpathlineto{\pgfqpoint{1.237589in}{1.680869in}}%
\pgfpathlineto{\pgfqpoint{1.239348in}{1.682541in}}%
\pgfpathlineto{\pgfqpoint{1.241217in}{1.684185in}}%
\pgfpathlineto{\pgfqpoint{1.243194in}{1.685801in}}%
\pgfpathlineto{\pgfqpoint{1.245277in}{1.687386in}}%
\pgfpathlineto{\pgfqpoint{1.248182in}{1.690508in}}%
\pgfpathlineto{\pgfqpoint{1.251089in}{1.693531in}}%
\pgfpathlineto{\pgfqpoint{1.254000in}{1.696455in}}%
\pgfpathlineto{\pgfqpoint{1.256912in}{1.699278in}}%
\pgfpathlineto{\pgfqpoint{1.255061in}{1.697870in}}%
\pgfpathlineto{\pgfqpoint{1.253304in}{1.696435in}}%
\pgfpathlineto{\pgfqpoint{1.251644in}{1.694974in}}%
\pgfpathlineto{\pgfqpoint{1.250081in}{1.693490in}}%
\pgfpathclose%
\pgfusepath{fill}%
\end{pgfscope}%
\begin{pgfscope}%
\pgfpathrectangle{\pgfqpoint{0.329460in}{0.284240in}}{\pgfqpoint{1.989680in}{1.989680in}}%
\pgfusepath{clip}%
\pgfsetbuttcap%
\pgfsetroundjoin%
\definecolor{currentfill}{rgb}{0.267004,0.004874,0.329415}%
\pgfsetfillcolor{currentfill}%
\pgfsetlinewidth{0.000000pt}%
\definecolor{currentstroke}{rgb}{0.000000,0.000000,0.000000}%
\pgfsetstrokecolor{currentstroke}%
\pgfsetdash{}{0pt}%
\pgfpathmoveto{\pgfqpoint{1.774951in}{0.856639in}}%
\pgfpathlineto{\pgfqpoint{1.777747in}{0.855714in}}%
\pgfpathlineto{\pgfqpoint{1.780550in}{0.855023in}}%
\pgfpathlineto{\pgfqpoint{1.783359in}{0.854569in}}%
\pgfpathlineto{\pgfqpoint{1.786176in}{0.854358in}}%
\pgfpathlineto{\pgfqpoint{1.775483in}{0.847077in}}%
\pgfpathlineto{\pgfqpoint{1.764331in}{0.839974in}}%
\pgfpathlineto{\pgfqpoint{1.752731in}{0.833056in}}%
\pgfpathlineto{\pgfqpoint{1.740694in}{0.826332in}}%
\pgfpathlineto{\pgfqpoint{1.738161in}{0.826727in}}%
\pgfpathlineto{\pgfqpoint{1.735635in}{0.827365in}}%
\pgfpathlineto{\pgfqpoint{1.733115in}{0.828240in}}%
\pgfpathlineto{\pgfqpoint{1.730601in}{0.829349in}}%
\pgfpathlineto{\pgfqpoint{1.742336in}{0.835896in}}%
\pgfpathlineto{\pgfqpoint{1.753647in}{0.842631in}}%
\pgfpathlineto{\pgfqpoint{1.764522in}{0.849548in}}%
\pgfpathlineto{\pgfqpoint{1.774951in}{0.856639in}}%
\pgfpathclose%
\pgfusepath{fill}%
\end{pgfscope}%
\begin{pgfscope}%
\pgfpathrectangle{\pgfqpoint{0.329460in}{0.284240in}}{\pgfqpoint{1.989680in}{1.989680in}}%
\pgfusepath{clip}%
\pgfsetbuttcap%
\pgfsetroundjoin%
\definecolor{currentfill}{rgb}{0.277941,0.056324,0.381191}%
\pgfsetfillcolor{currentfill}%
\pgfsetlinewidth{0.000000pt}%
\definecolor{currentstroke}{rgb}{0.000000,0.000000,0.000000}%
\pgfsetstrokecolor{currentstroke}%
\pgfsetdash{}{0pt}%
\pgfpathmoveto{\pgfqpoint{0.903343in}{0.855138in}}%
\pgfpathlineto{\pgfqpoint{0.900512in}{0.857280in}}%
\pgfpathlineto{\pgfqpoint{0.897673in}{0.859714in}}%
\pgfpathlineto{\pgfqpoint{0.894824in}{0.862445in}}%
\pgfpathlineto{\pgfqpoint{0.891965in}{0.865477in}}%
\pgfpathlineto{\pgfqpoint{0.880521in}{0.873334in}}%
\pgfpathlineto{\pgfqpoint{0.869587in}{0.881370in}}%
\pgfpathlineto{\pgfqpoint{0.859173in}{0.889576in}}%
\pgfpathlineto{\pgfqpoint{0.849288in}{0.897943in}}%
\pgfpathlineto{\pgfqpoint{0.852398in}{0.894725in}}%
\pgfpathlineto{\pgfqpoint{0.855497in}{0.891807in}}%
\pgfpathlineto{\pgfqpoint{0.858586in}{0.889185in}}%
\pgfpathlineto{\pgfqpoint{0.861664in}{0.886853in}}%
\pgfpathlineto{\pgfqpoint{0.871320in}{0.878679in}}%
\pgfpathlineto{\pgfqpoint{0.881492in}{0.870662in}}%
\pgfpathlineto{\pgfqpoint{0.892169in}{0.862812in}}%
\pgfpathlineto{\pgfqpoint{0.903343in}{0.855138in}}%
\pgfpathclose%
\pgfusepath{fill}%
\end{pgfscope}%
\begin{pgfscope}%
\pgfpathrectangle{\pgfqpoint{0.329460in}{0.284240in}}{\pgfqpoint{1.989680in}{1.989680in}}%
\pgfusepath{clip}%
\pgfsetbuttcap%
\pgfsetroundjoin%
\definecolor{currentfill}{rgb}{0.134692,0.658636,0.517649}%
\pgfsetfillcolor{currentfill}%
\pgfsetlinewidth{0.000000pt}%
\definecolor{currentstroke}{rgb}{0.000000,0.000000,0.000000}%
\pgfsetstrokecolor{currentstroke}%
\pgfsetdash{}{0pt}%
\pgfpathmoveto{\pgfqpoint{1.590982in}{1.422219in}}%
\pgfpathlineto{\pgfqpoint{1.594247in}{1.415087in}}%
\pgfpathlineto{\pgfqpoint{1.597508in}{1.407920in}}%
\pgfpathlineto{\pgfqpoint{1.600768in}{1.400719in}}%
\pgfpathlineto{\pgfqpoint{1.604025in}{1.393487in}}%
\pgfpathlineto{\pgfqpoint{1.601644in}{1.389569in}}%
\pgfpathlineto{\pgfqpoint{1.599009in}{1.385687in}}%
\pgfpathlineto{\pgfqpoint{1.596121in}{1.381846in}}%
\pgfpathlineto{\pgfqpoint{1.592984in}{1.378050in}}%
\pgfpathlineto{\pgfqpoint{1.589861in}{1.385499in}}%
\pgfpathlineto{\pgfqpoint{1.586736in}{1.392917in}}%
\pgfpathlineto{\pgfqpoint{1.583609in}{1.400300in}}%
\pgfpathlineto{\pgfqpoint{1.580481in}{1.407648in}}%
\pgfpathlineto{\pgfqpoint{1.583463in}{1.411231in}}%
\pgfpathlineto{\pgfqpoint{1.586210in}{1.414856in}}%
\pgfpathlineto{\pgfqpoint{1.588717in}{1.418520in}}%
\pgfpathlineto{\pgfqpoint{1.590982in}{1.422219in}}%
\pgfpathclose%
\pgfusepath{fill}%
\end{pgfscope}%
\begin{pgfscope}%
\pgfpathrectangle{\pgfqpoint{0.329460in}{0.284240in}}{\pgfqpoint{1.989680in}{1.989680in}}%
\pgfusepath{clip}%
\pgfsetbuttcap%
\pgfsetroundjoin%
\definecolor{currentfill}{rgb}{0.267004,0.004874,0.329415}%
\pgfsetfillcolor{currentfill}%
\pgfsetlinewidth{0.000000pt}%
\definecolor{currentstroke}{rgb}{0.000000,0.000000,0.000000}%
\pgfsetstrokecolor{currentstroke}%
\pgfsetdash{}{0pt}%
\pgfpathmoveto{\pgfqpoint{0.992276in}{0.830531in}}%
\pgfpathlineto{\pgfqpoint{0.989853in}{0.828494in}}%
\pgfpathlineto{\pgfqpoint{0.987425in}{0.826672in}}%
\pgfpathlineto{\pgfqpoint{0.984992in}{0.825071in}}%
\pgfpathlineto{\pgfqpoint{0.982553in}{0.823695in}}%
\pgfpathlineto{\pgfqpoint{0.970450in}{0.830067in}}%
\pgfpathlineto{\pgfqpoint{0.958761in}{0.836635in}}%
\pgfpathlineto{\pgfqpoint{0.947498in}{0.843391in}}%
\pgfpathlineto{\pgfqpoint{0.936673in}{0.850328in}}%
\pgfpathlineto{\pgfqpoint{0.939404in}{0.851523in}}%
\pgfpathlineto{\pgfqpoint{0.942130in}{0.852943in}}%
\pgfpathlineto{\pgfqpoint{0.944849in}{0.854582in}}%
\pgfpathlineto{\pgfqpoint{0.947564in}{0.856437in}}%
\pgfpathlineto{\pgfqpoint{0.958116in}{0.849689in}}%
\pgfpathlineto{\pgfqpoint{0.969092in}{0.843117in}}%
\pgfpathlineto{\pgfqpoint{0.980483in}{0.836728in}}%
\pgfpathlineto{\pgfqpoint{0.992276in}{0.830531in}}%
\pgfpathclose%
\pgfusepath{fill}%
\end{pgfscope}%
\begin{pgfscope}%
\pgfpathrectangle{\pgfqpoint{0.329460in}{0.284240in}}{\pgfqpoint{1.989680in}{1.989680in}}%
\pgfusepath{clip}%
\pgfsetbuttcap%
\pgfsetroundjoin%
\definecolor{currentfill}{rgb}{0.814576,0.883393,0.110347}%
\pgfsetfillcolor{currentfill}%
\pgfsetlinewidth{0.000000pt}%
\definecolor{currentstroke}{rgb}{0.000000,0.000000,0.000000}%
\pgfsetstrokecolor{currentstroke}%
\pgfsetdash{}{0pt}%
\pgfpathmoveto{\pgfqpoint{1.424733in}{1.715184in}}%
\pgfpathlineto{\pgfqpoint{1.427333in}{1.712976in}}%
\pgfpathlineto{\pgfqpoint{1.429930in}{1.710664in}}%
\pgfpathlineto{\pgfqpoint{1.432525in}{1.708248in}}%
\pgfpathlineto{\pgfqpoint{1.435117in}{1.705729in}}%
\pgfpathlineto{\pgfqpoint{1.437402in}{1.704472in}}%
\pgfpathlineto{\pgfqpoint{1.439602in}{1.703182in}}%
\pgfpathlineto{\pgfqpoint{1.441715in}{1.701859in}}%
\pgfpathlineto{\pgfqpoint{1.443740in}{1.700506in}}%
\pgfpathlineto{\pgfqpoint{1.440878in}{1.703189in}}%
\pgfpathlineto{\pgfqpoint{1.438014in}{1.705769in}}%
\pgfpathlineto{\pgfqpoint{1.435148in}{1.708245in}}%
\pgfpathlineto{\pgfqpoint{1.432280in}{1.710615in}}%
\pgfpathlineto{\pgfqpoint{1.430508in}{1.711799in}}%
\pgfpathlineto{\pgfqpoint{1.428659in}{1.712956in}}%
\pgfpathlineto{\pgfqpoint{1.426733in}{1.714085in}}%
\pgfpathlineto{\pgfqpoint{1.424733in}{1.715184in}}%
\pgfpathclose%
\pgfusepath{fill}%
\end{pgfscope}%
\begin{pgfscope}%
\pgfpathrectangle{\pgfqpoint{0.329460in}{0.284240in}}{\pgfqpoint{1.989680in}{1.989680in}}%
\pgfusepath{clip}%
\pgfsetbuttcap%
\pgfsetroundjoin%
\definecolor{currentfill}{rgb}{0.565498,0.842430,0.262877}%
\pgfsetfillcolor{currentfill}%
\pgfsetlinewidth{0.000000pt}%
\definecolor{currentstroke}{rgb}{0.000000,0.000000,0.000000}%
\pgfsetstrokecolor{currentstroke}%
\pgfsetdash{}{0pt}%
\pgfpathmoveto{\pgfqpoint{1.198012in}{1.624575in}}%
\pgfpathlineto{\pgfqpoint{1.194574in}{1.619828in}}%
\pgfpathlineto{\pgfqpoint{1.191137in}{1.614995in}}%
\pgfpathlineto{\pgfqpoint{1.187704in}{1.610077in}}%
\pgfpathlineto{\pgfqpoint{1.184273in}{1.605076in}}%
\pgfpathlineto{\pgfqpoint{1.184728in}{1.607581in}}%
\pgfpathlineto{\pgfqpoint{1.185347in}{1.610076in}}%
\pgfpathlineto{\pgfqpoint{1.186132in}{1.612560in}}%
\pgfpathlineto{\pgfqpoint{1.187079in}{1.615029in}}%
\pgfpathlineto{\pgfqpoint{1.190447in}{1.619818in}}%
\pgfpathlineto{\pgfqpoint{1.193817in}{1.624523in}}%
\pgfpathlineto{\pgfqpoint{1.197191in}{1.629144in}}%
\pgfpathlineto{\pgfqpoint{1.200567in}{1.633680in}}%
\pgfpathlineto{\pgfqpoint{1.199702in}{1.631420in}}%
\pgfpathlineto{\pgfqpoint{1.198988in}{1.629148in}}%
\pgfpathlineto{\pgfqpoint{1.198424in}{1.626866in}}%
\pgfpathlineto{\pgfqpoint{1.198012in}{1.624575in}}%
\pgfpathclose%
\pgfusepath{fill}%
\end{pgfscope}%
\begin{pgfscope}%
\pgfpathrectangle{\pgfqpoint{0.329460in}{0.284240in}}{\pgfqpoint{1.989680in}{1.989680in}}%
\pgfusepath{clip}%
\pgfsetbuttcap%
\pgfsetroundjoin%
\definecolor{currentfill}{rgb}{0.855810,0.888601,0.097452}%
\pgfsetfillcolor{currentfill}%
\pgfsetlinewidth{0.000000pt}%
\definecolor{currentstroke}{rgb}{0.000000,0.000000,0.000000}%
\pgfsetstrokecolor{currentstroke}%
\pgfsetdash{}{0pt}%
\pgfpathmoveto{\pgfqpoint{1.398516in}{1.729483in}}%
\pgfpathlineto{\pgfqpoint{1.400469in}{1.727977in}}%
\pgfpathlineto{\pgfqpoint{1.402421in}{1.726361in}}%
\pgfpathlineto{\pgfqpoint{1.404370in}{1.724639in}}%
\pgfpathlineto{\pgfqpoint{1.406318in}{1.722809in}}%
\pgfpathlineto{\pgfqpoint{1.408834in}{1.721978in}}%
\pgfpathlineto{\pgfqpoint{1.411293in}{1.721111in}}%
\pgfpathlineto{\pgfqpoint{1.413694in}{1.720208in}}%
\pgfpathlineto{\pgfqpoint{1.416033in}{1.719269in}}%
\pgfpathlineto{\pgfqpoint{1.413741in}{1.721225in}}%
\pgfpathlineto{\pgfqpoint{1.411446in}{1.723074in}}%
\pgfpathlineto{\pgfqpoint{1.409150in}{1.724816in}}%
\pgfpathlineto{\pgfqpoint{1.406851in}{1.726448in}}%
\pgfpathlineto{\pgfqpoint{1.404844in}{1.727253in}}%
\pgfpathlineto{\pgfqpoint{1.402785in}{1.728027in}}%
\pgfpathlineto{\pgfqpoint{1.400675in}{1.728771in}}%
\pgfpathlineto{\pgfqpoint{1.398516in}{1.729483in}}%
\pgfpathclose%
\pgfusepath{fill}%
\end{pgfscope}%
\begin{pgfscope}%
\pgfpathrectangle{\pgfqpoint{0.329460in}{0.284240in}}{\pgfqpoint{1.989680in}{1.989680in}}%
\pgfusepath{clip}%
\pgfsetbuttcap%
\pgfsetroundjoin%
\definecolor{currentfill}{rgb}{0.133743,0.548535,0.553541}%
\pgfsetfillcolor{currentfill}%
\pgfsetlinewidth{0.000000pt}%
\definecolor{currentstroke}{rgb}{0.000000,0.000000,0.000000}%
\pgfsetstrokecolor{currentstroke}%
\pgfsetdash{}{0pt}%
\pgfpathmoveto{\pgfqpoint{1.105183in}{1.297770in}}%
\pgfpathlineto{\pgfqpoint{1.102310in}{1.289884in}}%
\pgfpathlineto{\pgfqpoint{1.099437in}{1.281990in}}%
\pgfpathlineto{\pgfqpoint{1.096566in}{1.274089in}}%
\pgfpathlineto{\pgfqpoint{1.093696in}{1.266185in}}%
\pgfpathlineto{\pgfqpoint{1.088737in}{1.270314in}}%
\pgfpathlineto{\pgfqpoint{1.084048in}{1.274517in}}%
\pgfpathlineto{\pgfqpoint{1.079636in}{1.278790in}}%
\pgfpathlineto{\pgfqpoint{1.075503in}{1.283127in}}%
\pgfpathlineto{\pgfqpoint{1.078567in}{1.290822in}}%
\pgfpathlineto{\pgfqpoint{1.081633in}{1.298514in}}%
\pgfpathlineto{\pgfqpoint{1.084700in}{1.306200in}}%
\pgfpathlineto{\pgfqpoint{1.087770in}{1.313877in}}%
\pgfpathlineto{\pgfqpoint{1.091727in}{1.309754in}}%
\pgfpathlineto{\pgfqpoint{1.095951in}{1.305692in}}%
\pgfpathlineto{\pgfqpoint{1.100438in}{1.301696in}}%
\pgfpathlineto{\pgfqpoint{1.105183in}{1.297770in}}%
\pgfpathclose%
\pgfusepath{fill}%
\end{pgfscope}%
\begin{pgfscope}%
\pgfpathrectangle{\pgfqpoint{0.329460in}{0.284240in}}{\pgfqpoint{1.989680in}{1.989680in}}%
\pgfusepath{clip}%
\pgfsetbuttcap%
\pgfsetroundjoin%
\definecolor{currentfill}{rgb}{0.487026,0.823929,0.312321}%
\pgfsetfillcolor{currentfill}%
\pgfsetlinewidth{0.000000pt}%
\definecolor{currentstroke}{rgb}{0.000000,0.000000,0.000000}%
\pgfsetstrokecolor{currentstroke}%
\pgfsetdash{}{0pt}%
\pgfpathmoveto{\pgfqpoint{1.517706in}{1.607303in}}%
\pgfpathlineto{\pgfqpoint{1.521125in}{1.602268in}}%
\pgfpathlineto{\pgfqpoint{1.524541in}{1.597154in}}%
\pgfpathlineto{\pgfqpoint{1.527954in}{1.591961in}}%
\pgfpathlineto{\pgfqpoint{1.531364in}{1.586691in}}%
\pgfpathlineto{\pgfqpoint{1.531842in}{1.583971in}}%
\pgfpathlineto{\pgfqpoint{1.532140in}{1.581243in}}%
\pgfpathlineto{\pgfqpoint{1.532259in}{1.578511in}}%
\pgfpathlineto{\pgfqpoint{1.532197in}{1.575778in}}%
\pgfpathlineto{\pgfqpoint{1.528765in}{1.581264in}}%
\pgfpathlineto{\pgfqpoint{1.525330in}{1.586673in}}%
\pgfpathlineto{\pgfqpoint{1.521892in}{1.592004in}}%
\pgfpathlineto{\pgfqpoint{1.518453in}{1.597254in}}%
\pgfpathlineto{\pgfqpoint{1.518516in}{1.599770in}}%
\pgfpathlineto{\pgfqpoint{1.518412in}{1.602286in}}%
\pgfpathlineto{\pgfqpoint{1.518142in}{1.604798in}}%
\pgfpathlineto{\pgfqpoint{1.517706in}{1.607303in}}%
\pgfpathclose%
\pgfusepath{fill}%
\end{pgfscope}%
\begin{pgfscope}%
\pgfpathrectangle{\pgfqpoint{0.329460in}{0.284240in}}{\pgfqpoint{1.989680in}{1.989680in}}%
\pgfusepath{clip}%
\pgfsetbuttcap%
\pgfsetroundjoin%
\definecolor{currentfill}{rgb}{0.179019,0.433756,0.557430}%
\pgfsetfillcolor{currentfill}%
\pgfsetlinewidth{0.000000pt}%
\definecolor{currentstroke}{rgb}{0.000000,0.000000,0.000000}%
\pgfsetstrokecolor{currentstroke}%
\pgfsetdash{}{0pt}%
\pgfpathmoveto{\pgfqpoint{1.095181in}{1.185867in}}%
\pgfpathlineto{\pgfqpoint{1.092563in}{1.177849in}}%
\pgfpathlineto{\pgfqpoint{1.089946in}{1.169854in}}%
\pgfpathlineto{\pgfqpoint{1.087330in}{1.161884in}}%
\pgfpathlineto{\pgfqpoint{1.084715in}{1.153942in}}%
\pgfpathlineto{\pgfqpoint{1.077939in}{1.158306in}}%
\pgfpathlineto{\pgfqpoint{1.071449in}{1.162773in}}%
\pgfpathlineto{\pgfqpoint{1.065252in}{1.167340in}}%
\pgfpathlineto{\pgfqpoint{1.059353in}{1.172001in}}%
\pgfpathlineto{\pgfqpoint{1.062209in}{1.179744in}}%
\pgfpathlineto{\pgfqpoint{1.065067in}{1.187515in}}%
\pgfpathlineto{\pgfqpoint{1.067925in}{1.195312in}}%
\pgfpathlineto{\pgfqpoint{1.070784in}{1.203131in}}%
\pgfpathlineto{\pgfqpoint{1.076459in}{1.198675in}}%
\pgfpathlineto{\pgfqpoint{1.082421in}{1.194309in}}%
\pgfpathlineto{\pgfqpoint{1.088664in}{1.190038in}}%
\pgfpathlineto{\pgfqpoint{1.095181in}{1.185867in}}%
\pgfpathclose%
\pgfusepath{fill}%
\end{pgfscope}%
\begin{pgfscope}%
\pgfpathrectangle{\pgfqpoint{0.329460in}{0.284240in}}{\pgfqpoint{1.989680in}{1.989680in}}%
\pgfusepath{clip}%
\pgfsetbuttcap%
\pgfsetroundjoin%
\definecolor{currentfill}{rgb}{0.855810,0.888601,0.097452}%
\pgfsetfillcolor{currentfill}%
\pgfsetlinewidth{0.000000pt}%
\definecolor{currentstroke}{rgb}{0.000000,0.000000,0.000000}%
\pgfsetstrokecolor{currentstroke}%
\pgfsetdash{}{0pt}%
\pgfpathmoveto{\pgfqpoint{1.293786in}{1.725709in}}%
\pgfpathlineto{\pgfqpoint{1.291415in}{1.724045in}}%
\pgfpathlineto{\pgfqpoint{1.289047in}{1.722273in}}%
\pgfpathlineto{\pgfqpoint{1.286680in}{1.720393in}}%
\pgfpathlineto{\pgfqpoint{1.284316in}{1.718407in}}%
\pgfpathlineto{\pgfqpoint{1.286599in}{1.719375in}}%
\pgfpathlineto{\pgfqpoint{1.288945in}{1.720310in}}%
\pgfpathlineto{\pgfqpoint{1.291352in}{1.721209in}}%
\pgfpathlineto{\pgfqpoint{1.293818in}{1.722073in}}%
\pgfpathlineto{\pgfqpoint{1.295845in}{1.723928in}}%
\pgfpathlineto{\pgfqpoint{1.297874in}{1.725677in}}%
\pgfpathlineto{\pgfqpoint{1.299905in}{1.727319in}}%
\pgfpathlineto{\pgfqpoint{1.301938in}{1.728852in}}%
\pgfpathlineto{\pgfqpoint{1.299822in}{1.728111in}}%
\pgfpathlineto{\pgfqpoint{1.297757in}{1.727340in}}%
\pgfpathlineto{\pgfqpoint{1.295744in}{1.726539in}}%
\pgfpathlineto{\pgfqpoint{1.293786in}{1.725709in}}%
\pgfpathclose%
\pgfusepath{fill}%
\end{pgfscope}%
\begin{pgfscope}%
\pgfpathrectangle{\pgfqpoint{0.329460in}{0.284240in}}{\pgfqpoint{1.989680in}{1.989680in}}%
\pgfusepath{clip}%
\pgfsetbuttcap%
\pgfsetroundjoin%
\definecolor{currentfill}{rgb}{0.814576,0.883393,0.110347}%
\pgfsetfillcolor{currentfill}%
\pgfsetlinewidth{0.000000pt}%
\definecolor{currentstroke}{rgb}{0.000000,0.000000,0.000000}%
\pgfsetstrokecolor{currentstroke}%
\pgfsetdash{}{0pt}%
\pgfpathmoveto{\pgfqpoint{1.268588in}{1.709542in}}%
\pgfpathlineto{\pgfqpoint{1.265665in}{1.707132in}}%
\pgfpathlineto{\pgfqpoint{1.262745in}{1.704618in}}%
\pgfpathlineto{\pgfqpoint{1.259827in}{1.701999in}}%
\pgfpathlineto{\pgfqpoint{1.256912in}{1.699278in}}%
\pgfpathlineto{\pgfqpoint{1.258856in}{1.700658in}}%
\pgfpathlineto{\pgfqpoint{1.260891in}{1.702008in}}%
\pgfpathlineto{\pgfqpoint{1.263014in}{1.703327in}}%
\pgfpathlineto{\pgfqpoint{1.265223in}{1.704613in}}%
\pgfpathlineto{\pgfqpoint{1.267879in}{1.707167in}}%
\pgfpathlineto{\pgfqpoint{1.270537in}{1.709618in}}%
\pgfpathlineto{\pgfqpoint{1.273198in}{1.711966in}}%
\pgfpathlineto{\pgfqpoint{1.275861in}{1.714208in}}%
\pgfpathlineto{\pgfqpoint{1.273927in}{1.713083in}}%
\pgfpathlineto{\pgfqpoint{1.272069in}{1.711929in}}%
\pgfpathlineto{\pgfqpoint{1.270288in}{1.710748in}}%
\pgfpathlineto{\pgfqpoint{1.268588in}{1.709542in}}%
\pgfpathclose%
\pgfusepath{fill}%
\end{pgfscope}%
\begin{pgfscope}%
\pgfpathrectangle{\pgfqpoint{0.329460in}{0.284240in}}{\pgfqpoint{1.989680in}{1.989680in}}%
\pgfusepath{clip}%
\pgfsetbuttcap%
\pgfsetroundjoin%
\definecolor{currentfill}{rgb}{0.231674,0.318106,0.544834}%
\pgfsetfillcolor{currentfill}%
\pgfsetlinewidth{0.000000pt}%
\definecolor{currentstroke}{rgb}{0.000000,0.000000,0.000000}%
\pgfsetstrokecolor{currentstroke}%
\pgfsetdash{}{0pt}%
\pgfpathmoveto{\pgfqpoint{1.645066in}{1.095975in}}%
\pgfpathlineto{\pgfqpoint{1.647736in}{1.088457in}}%
\pgfpathlineto{\pgfqpoint{1.650406in}{1.080996in}}%
\pgfpathlineto{\pgfqpoint{1.653075in}{1.073596in}}%
\pgfpathlineto{\pgfqpoint{1.655745in}{1.066259in}}%
\pgfpathlineto{\pgfqpoint{1.648161in}{1.061325in}}%
\pgfpathlineto{\pgfqpoint{1.640263in}{1.056515in}}%
\pgfpathlineto{\pgfqpoint{1.632059in}{1.051833in}}%
\pgfpathlineto{\pgfqpoint{1.623557in}{1.047285in}}%
\pgfpathlineto{\pgfqpoint{1.621163in}{1.054811in}}%
\pgfpathlineto{\pgfqpoint{1.618768in}{1.062399in}}%
\pgfpathlineto{\pgfqpoint{1.616374in}{1.070047in}}%
\pgfpathlineto{\pgfqpoint{1.613980in}{1.077753in}}%
\pgfpathlineto{\pgfqpoint{1.622189in}{1.082120in}}%
\pgfpathlineto{\pgfqpoint{1.630112in}{1.086616in}}%
\pgfpathlineto{\pgfqpoint{1.637740in}{1.091236in}}%
\pgfpathlineto{\pgfqpoint{1.645066in}{1.095975in}}%
\pgfpathclose%
\pgfusepath{fill}%
\end{pgfscope}%
\begin{pgfscope}%
\pgfpathrectangle{\pgfqpoint{0.329460in}{0.284240in}}{\pgfqpoint{1.989680in}{1.989680in}}%
\pgfusepath{clip}%
\pgfsetbuttcap%
\pgfsetroundjoin%
\definecolor{currentfill}{rgb}{0.220124,0.725509,0.466226}%
\pgfsetfillcolor{currentfill}%
\pgfsetlinewidth{0.000000pt}%
\definecolor{currentstroke}{rgb}{0.000000,0.000000,0.000000}%
\pgfsetstrokecolor{currentstroke}%
\pgfsetdash{}{0pt}%
\pgfpathmoveto{\pgfqpoint{1.139385in}{1.474826in}}%
\pgfpathlineto{\pgfqpoint{1.136128in}{1.468009in}}%
\pgfpathlineto{\pgfqpoint{1.132874in}{1.461139in}}%
\pgfpathlineto{\pgfqpoint{1.129622in}{1.454218in}}%
\pgfpathlineto{\pgfqpoint{1.126372in}{1.447249in}}%
\pgfpathlineto{\pgfqpoint{1.124248in}{1.450733in}}%
\pgfpathlineto{\pgfqpoint{1.122353in}{1.454246in}}%
\pgfpathlineto{\pgfqpoint{1.120690in}{1.457785in}}%
\pgfpathlineto{\pgfqpoint{1.119259in}{1.461345in}}%
\pgfpathlineto{\pgfqpoint{1.122604in}{1.468095in}}%
\pgfpathlineto{\pgfqpoint{1.125952in}{1.474796in}}%
\pgfpathlineto{\pgfqpoint{1.129302in}{1.481448in}}%
\pgfpathlineto{\pgfqpoint{1.132655in}{1.488048in}}%
\pgfpathlineto{\pgfqpoint{1.134010in}{1.484708in}}%
\pgfpathlineto{\pgfqpoint{1.135584in}{1.481389in}}%
\pgfpathlineto{\pgfqpoint{1.137377in}{1.478094in}}%
\pgfpathlineto{\pgfqpoint{1.139385in}{1.474826in}}%
\pgfpathclose%
\pgfusepath{fill}%
\end{pgfscope}%
\begin{pgfscope}%
\pgfpathrectangle{\pgfqpoint{0.329460in}{0.284240in}}{\pgfqpoint{1.989680in}{1.989680in}}%
\pgfusepath{clip}%
\pgfsetbuttcap%
\pgfsetroundjoin%
\definecolor{currentfill}{rgb}{0.896320,0.893616,0.096335}%
\pgfsetfillcolor{currentfill}%
\pgfsetlinewidth{0.000000pt}%
\definecolor{currentstroke}{rgb}{0.000000,0.000000,0.000000}%
\pgfsetstrokecolor{currentstroke}%
\pgfsetdash{}{0pt}%
\pgfpathmoveto{\pgfqpoint{1.342527in}{1.739854in}}%
\pgfpathlineto{\pgfqpoint{1.342098in}{1.739061in}}%
\pgfpathlineto{\pgfqpoint{1.341669in}{1.738157in}}%
\pgfpathlineto{\pgfqpoint{1.341240in}{1.737141in}}%
\pgfpathlineto{\pgfqpoint{1.340812in}{1.736015in}}%
\pgfpathlineto{\pgfqpoint{1.343434in}{1.736148in}}%
\pgfpathlineto{\pgfqpoint{1.346063in}{1.736243in}}%
\pgfpathlineto{\pgfqpoint{1.348697in}{1.736299in}}%
\pgfpathlineto{\pgfqpoint{1.351334in}{1.736316in}}%
\pgfpathlineto{\pgfqpoint{1.351328in}{1.737429in}}%
\pgfpathlineto{\pgfqpoint{1.351322in}{1.738433in}}%
\pgfpathlineto{\pgfqpoint{1.351316in}{1.739325in}}%
\pgfpathlineto{\pgfqpoint{1.351310in}{1.740105in}}%
\pgfpathlineto{\pgfqpoint{1.349109in}{1.740090in}}%
\pgfpathlineto{\pgfqpoint{1.346910in}{1.740044in}}%
\pgfpathlineto{\pgfqpoint{1.344716in}{1.739965in}}%
\pgfpathlineto{\pgfqpoint{1.342527in}{1.739854in}}%
\pgfpathclose%
\pgfusepath{fill}%
\end{pgfscope}%
\begin{pgfscope}%
\pgfpathrectangle{\pgfqpoint{0.329460in}{0.284240in}}{\pgfqpoint{1.989680in}{1.989680in}}%
\pgfusepath{clip}%
\pgfsetbuttcap%
\pgfsetroundjoin%
\definecolor{currentfill}{rgb}{0.896320,0.893616,0.096335}%
\pgfsetfillcolor{currentfill}%
\pgfsetlinewidth{0.000000pt}%
\definecolor{currentstroke}{rgb}{0.000000,0.000000,0.000000}%
\pgfsetstrokecolor{currentstroke}%
\pgfsetdash{}{0pt}%
\pgfpathmoveto{\pgfqpoint{1.351310in}{1.740105in}}%
\pgfpathlineto{\pgfqpoint{1.351316in}{1.739325in}}%
\pgfpathlineto{\pgfqpoint{1.351322in}{1.738433in}}%
\pgfpathlineto{\pgfqpoint{1.351328in}{1.737429in}}%
\pgfpathlineto{\pgfqpoint{1.351334in}{1.736316in}}%
\pgfpathlineto{\pgfqpoint{1.353971in}{1.736294in}}%
\pgfpathlineto{\pgfqpoint{1.356604in}{1.736234in}}%
\pgfpathlineto{\pgfqpoint{1.359233in}{1.736135in}}%
\pgfpathlineto{\pgfqpoint{1.361853in}{1.735998in}}%
\pgfpathlineto{\pgfqpoint{1.361413in}{1.737124in}}%
\pgfpathlineto{\pgfqpoint{1.360973in}{1.738141in}}%
\pgfpathlineto{\pgfqpoint{1.360532in}{1.739046in}}%
\pgfpathlineto{\pgfqpoint{1.360091in}{1.739839in}}%
\pgfpathlineto{\pgfqpoint{1.357903in}{1.739954in}}%
\pgfpathlineto{\pgfqpoint{1.355709in}{1.740037in}}%
\pgfpathlineto{\pgfqpoint{1.353511in}{1.740087in}}%
\pgfpathlineto{\pgfqpoint{1.351310in}{1.740105in}}%
\pgfpathclose%
\pgfusepath{fill}%
\end{pgfscope}%
\begin{pgfscope}%
\pgfpathrectangle{\pgfqpoint{0.329460in}{0.284240in}}{\pgfqpoint{1.989680in}{1.989680in}}%
\pgfusepath{clip}%
\pgfsetbuttcap%
\pgfsetroundjoin%
\definecolor{currentfill}{rgb}{0.344074,0.780029,0.397381}%
\pgfsetfillcolor{currentfill}%
\pgfsetlinewidth{0.000000pt}%
\definecolor{currentstroke}{rgb}{0.000000,0.000000,0.000000}%
\pgfsetstrokecolor{currentstroke}%
\pgfsetdash{}{0pt}%
\pgfpathmoveto{\pgfqpoint{1.159566in}{1.538735in}}%
\pgfpathlineto{\pgfqpoint{1.156194in}{1.532622in}}%
\pgfpathlineto{\pgfqpoint{1.152823in}{1.526441in}}%
\pgfpathlineto{\pgfqpoint{1.149456in}{1.520195in}}%
\pgfpathlineto{\pgfqpoint{1.146091in}{1.513885in}}%
\pgfpathlineto{\pgfqpoint{1.145018in}{1.517022in}}%
\pgfpathlineto{\pgfqpoint{1.144152in}{1.520173in}}%
\pgfpathlineto{\pgfqpoint{1.143494in}{1.523333in}}%
\pgfpathlineto{\pgfqpoint{1.143044in}{1.526500in}}%
\pgfpathlineto{\pgfqpoint{1.146453in}{1.532590in}}%
\pgfpathlineto{\pgfqpoint{1.149863in}{1.538617in}}%
\pgfpathlineto{\pgfqpoint{1.153277in}{1.544578in}}%
\pgfpathlineto{\pgfqpoint{1.156693in}{1.550473in}}%
\pgfpathlineto{\pgfqpoint{1.157121in}{1.547526in}}%
\pgfpathlineto{\pgfqpoint{1.157742in}{1.544585in}}%
\pgfpathlineto{\pgfqpoint{1.158557in}{1.541654in}}%
\pgfpathlineto{\pgfqpoint{1.159566in}{1.538735in}}%
\pgfpathclose%
\pgfusepath{fill}%
\end{pgfscope}%
\begin{pgfscope}%
\pgfpathrectangle{\pgfqpoint{0.329460in}{0.284240in}}{\pgfqpoint{1.989680in}{1.989680in}}%
\pgfusepath{clip}%
\pgfsetbuttcap%
\pgfsetroundjoin%
\definecolor{currentfill}{rgb}{0.276194,0.190074,0.493001}%
\pgfsetfillcolor{currentfill}%
\pgfsetlinewidth{0.000000pt}%
\definecolor{currentstroke}{rgb}{0.000000,0.000000,0.000000}%
\pgfsetstrokecolor{currentstroke}%
\pgfsetdash{}{0pt}%
\pgfpathmoveto{\pgfqpoint{0.823996in}{0.935183in}}%
\pgfpathlineto{\pgfqpoint{0.820778in}{0.941359in}}%
\pgfpathlineto{\pgfqpoint{0.817545in}{0.947894in}}%
\pgfpathlineto{\pgfqpoint{0.814299in}{0.954792in}}%
\pgfpathlineto{\pgfqpoint{0.811038in}{0.962060in}}%
\pgfpathlineto{\pgfqpoint{0.801033in}{0.971138in}}%
\pgfpathlineto{\pgfqpoint{0.791619in}{0.980364in}}%
\pgfpathlineto{\pgfqpoint{0.782802in}{0.989728in}}%
\pgfpathlineto{\pgfqpoint{0.774589in}{0.999220in}}%
\pgfpathlineto{\pgfqpoint{0.778050in}{0.991770in}}%
\pgfpathlineto{\pgfqpoint{0.781496in}{0.984688in}}%
\pgfpathlineto{\pgfqpoint{0.784928in}{0.977968in}}%
\pgfpathlineto{\pgfqpoint{0.788344in}{0.971605in}}%
\pgfpathlineto{\pgfqpoint{0.796380in}{0.962301in}}%
\pgfpathlineto{\pgfqpoint{0.805005in}{0.953122in}}%
\pgfpathlineto{\pgfqpoint{0.814213in}{0.944079in}}%
\pgfpathlineto{\pgfqpoint{0.823996in}{0.935183in}}%
\pgfpathclose%
\pgfusepath{fill}%
\end{pgfscope}%
\begin{pgfscope}%
\pgfpathrectangle{\pgfqpoint{0.329460in}{0.284240in}}{\pgfqpoint{1.989680in}{1.989680in}}%
\pgfusepath{clip}%
\pgfsetbuttcap%
\pgfsetroundjoin%
\definecolor{currentfill}{rgb}{0.248629,0.278775,0.534556}%
\pgfsetfillcolor{currentfill}%
\pgfsetlinewidth{0.000000pt}%
\definecolor{currentstroke}{rgb}{0.000000,0.000000,0.000000}%
\pgfsetstrokecolor{currentstroke}%
\pgfsetdash{}{0pt}%
\pgfpathmoveto{\pgfqpoint{1.086618in}{1.043360in}}%
\pgfpathlineto{\pgfqpoint{1.084291in}{1.035861in}}%
\pgfpathlineto{\pgfqpoint{1.081962in}{1.028432in}}%
\pgfpathlineto{\pgfqpoint{1.079634in}{1.021075in}}%
\pgfpathlineto{\pgfqpoint{1.077304in}{1.013793in}}%
\pgfpathlineto{\pgfqpoint{1.068243in}{1.018394in}}%
\pgfpathlineto{\pgfqpoint{1.059483in}{1.023138in}}%
\pgfpathlineto{\pgfqpoint{1.051033in}{1.028022in}}%
\pgfpathlineto{\pgfqpoint{1.042902in}{1.033038in}}%
\pgfpathlineto{\pgfqpoint{1.045516in}{1.040135in}}%
\pgfpathlineto{\pgfqpoint{1.048129in}{1.047308in}}%
\pgfpathlineto{\pgfqpoint{1.050743in}{1.054553in}}%
\pgfpathlineto{\pgfqpoint{1.053356in}{1.061867in}}%
\pgfpathlineto{\pgfqpoint{1.061219in}{1.057043in}}%
\pgfpathlineto{\pgfqpoint{1.069389in}{1.052347in}}%
\pgfpathlineto{\pgfqpoint{1.077859in}{1.047784in}}%
\pgfpathlineto{\pgfqpoint{1.086618in}{1.043360in}}%
\pgfpathclose%
\pgfusepath{fill}%
\end{pgfscope}%
\begin{pgfscope}%
\pgfpathrectangle{\pgfqpoint{0.329460in}{0.284240in}}{\pgfqpoint{1.989680in}{1.989680in}}%
\pgfusepath{clip}%
\pgfsetbuttcap%
\pgfsetroundjoin%
\definecolor{currentfill}{rgb}{0.896320,0.893616,0.096335}%
\pgfsetfillcolor{currentfill}%
\pgfsetlinewidth{0.000000pt}%
\definecolor{currentstroke}{rgb}{0.000000,0.000000,0.000000}%
\pgfsetstrokecolor{currentstroke}%
\pgfsetdash{}{0pt}%
\pgfpathmoveto{\pgfqpoint{1.333880in}{1.739090in}}%
\pgfpathlineto{\pgfqpoint{1.333022in}{1.738260in}}%
\pgfpathlineto{\pgfqpoint{1.332165in}{1.737317in}}%
\pgfpathlineto{\pgfqpoint{1.331308in}{1.736263in}}%
\pgfpathlineto{\pgfqpoint{1.330453in}{1.735099in}}%
\pgfpathlineto{\pgfqpoint{1.333018in}{1.735385in}}%
\pgfpathlineto{\pgfqpoint{1.335602in}{1.735633in}}%
\pgfpathlineto{\pgfqpoint{1.338201in}{1.735843in}}%
\pgfpathlineto{\pgfqpoint{1.340812in}{1.736015in}}%
\pgfpathlineto{\pgfqpoint{1.341240in}{1.737141in}}%
\pgfpathlineto{\pgfqpoint{1.341669in}{1.738157in}}%
\pgfpathlineto{\pgfqpoint{1.342098in}{1.739061in}}%
\pgfpathlineto{\pgfqpoint{1.342527in}{1.739854in}}%
\pgfpathlineto{\pgfqpoint{1.340347in}{1.739711in}}%
\pgfpathlineto{\pgfqpoint{1.338178in}{1.739535in}}%
\pgfpathlineto{\pgfqpoint{1.336022in}{1.739329in}}%
\pgfpathlineto{\pgfqpoint{1.333880in}{1.739090in}}%
\pgfpathclose%
\pgfusepath{fill}%
\end{pgfscope}%
\begin{pgfscope}%
\pgfpathrectangle{\pgfqpoint{0.329460in}{0.284240in}}{\pgfqpoint{1.989680in}{1.989680in}}%
\pgfusepath{clip}%
\pgfsetbuttcap%
\pgfsetroundjoin%
\definecolor{currentfill}{rgb}{0.896320,0.893616,0.096335}%
\pgfsetfillcolor{currentfill}%
\pgfsetlinewidth{0.000000pt}%
\definecolor{currentstroke}{rgb}{0.000000,0.000000,0.000000}%
\pgfsetstrokecolor{currentstroke}%
\pgfsetdash{}{0pt}%
\pgfpathmoveto{\pgfqpoint{1.360091in}{1.739839in}}%
\pgfpathlineto{\pgfqpoint{1.360532in}{1.739046in}}%
\pgfpathlineto{\pgfqpoint{1.360973in}{1.738141in}}%
\pgfpathlineto{\pgfqpoint{1.361413in}{1.737124in}}%
\pgfpathlineto{\pgfqpoint{1.361853in}{1.735998in}}%
\pgfpathlineto{\pgfqpoint{1.364464in}{1.735822in}}%
\pgfpathlineto{\pgfqpoint{1.367061in}{1.735608in}}%
\pgfpathlineto{\pgfqpoint{1.369643in}{1.735355in}}%
\pgfpathlineto{\pgfqpoint{1.372206in}{1.735065in}}%
\pgfpathlineto{\pgfqpoint{1.371339in}{1.736231in}}%
\pgfpathlineto{\pgfqpoint{1.370471in}{1.737286in}}%
\pgfpathlineto{\pgfqpoint{1.369602in}{1.738230in}}%
\pgfpathlineto{\pgfqpoint{1.368732in}{1.739062in}}%
\pgfpathlineto{\pgfqpoint{1.366592in}{1.739304in}}%
\pgfpathlineto{\pgfqpoint{1.364437in}{1.739514in}}%
\pgfpathlineto{\pgfqpoint{1.362269in}{1.739693in}}%
\pgfpathlineto{\pgfqpoint{1.360091in}{1.739839in}}%
\pgfpathclose%
\pgfusepath{fill}%
\end{pgfscope}%
\begin{pgfscope}%
\pgfpathrectangle{\pgfqpoint{0.329460in}{0.284240in}}{\pgfqpoint{1.989680in}{1.989680in}}%
\pgfusepath{clip}%
\pgfsetbuttcap%
\pgfsetroundjoin%
\definecolor{currentfill}{rgb}{0.163625,0.471133,0.558148}%
\pgfsetfillcolor{currentfill}%
\pgfsetlinewidth{0.000000pt}%
\definecolor{currentstroke}{rgb}{0.000000,0.000000,0.000000}%
\pgfsetstrokecolor{currentstroke}%
\pgfsetdash{}{0pt}%
\pgfpathmoveto{\pgfqpoint{1.624756in}{1.238432in}}%
\pgfpathlineto{\pgfqpoint{1.627667in}{1.230595in}}%
\pgfpathlineto{\pgfqpoint{1.630576in}{1.222769in}}%
\pgfpathlineto{\pgfqpoint{1.633485in}{1.214958in}}%
\pgfpathlineto{\pgfqpoint{1.636392in}{1.207165in}}%
\pgfpathlineto{\pgfqpoint{1.630975in}{1.202632in}}%
\pgfpathlineto{\pgfqpoint{1.625267in}{1.198185in}}%
\pgfpathlineto{\pgfqpoint{1.619274in}{1.193829in}}%
\pgfpathlineto{\pgfqpoint{1.613001in}{1.189569in}}%
\pgfpathlineto{\pgfqpoint{1.610325in}{1.197564in}}%
\pgfpathlineto{\pgfqpoint{1.607647in}{1.205576in}}%
\pgfpathlineto{\pgfqpoint{1.604970in}{1.213603in}}%
\pgfpathlineto{\pgfqpoint{1.602291in}{1.221641in}}%
\pgfpathlineto{\pgfqpoint{1.608314in}{1.225706in}}%
\pgfpathlineto{\pgfqpoint{1.614070in}{1.229863in}}%
\pgfpathlineto{\pgfqpoint{1.619552in}{1.234106in}}%
\pgfpathlineto{\pgfqpoint{1.624756in}{1.238432in}}%
\pgfpathclose%
\pgfusepath{fill}%
\end{pgfscope}%
\begin{pgfscope}%
\pgfpathrectangle{\pgfqpoint{0.329460in}{0.284240in}}{\pgfqpoint{1.989680in}{1.989680in}}%
\pgfusepath{clip}%
\pgfsetbuttcap%
\pgfsetroundjoin%
\definecolor{currentfill}{rgb}{0.282327,0.094955,0.417331}%
\pgfsetfillcolor{currentfill}%
\pgfsetlinewidth{0.000000pt}%
\definecolor{currentstroke}{rgb}{0.000000,0.000000,0.000000}%
\pgfsetstrokecolor{currentstroke}%
\pgfsetdash{}{0pt}%
\pgfpathmoveto{\pgfqpoint{1.861422in}{0.905508in}}%
\pgfpathlineto{\pgfqpoint{1.864591in}{0.909076in}}%
\pgfpathlineto{\pgfqpoint{1.867771in}{0.912955in}}%
\pgfpathlineto{\pgfqpoint{1.870962in}{0.917151in}}%
\pgfpathlineto{\pgfqpoint{1.874165in}{0.921670in}}%
\pgfpathlineto{\pgfqpoint{1.864540in}{0.912977in}}%
\pgfpathlineto{\pgfqpoint{1.854363in}{0.904439in}}%
\pgfpathlineto{\pgfqpoint{1.843644in}{0.896066in}}%
\pgfpathlineto{\pgfqpoint{1.832392in}{0.887867in}}%
\pgfpathlineto{\pgfqpoint{1.829429in}{0.883533in}}%
\pgfpathlineto{\pgfqpoint{1.826478in}{0.879523in}}%
\pgfpathlineto{\pgfqpoint{1.823538in}{0.875831in}}%
\pgfpathlineto{\pgfqpoint{1.820608in}{0.872452in}}%
\pgfpathlineto{\pgfqpoint{1.831599in}{0.880468in}}%
\pgfpathlineto{\pgfqpoint{1.842071in}{0.888656in}}%
\pgfpathlineto{\pgfqpoint{1.852015in}{0.897006in}}%
\pgfpathlineto{\pgfqpoint{1.861422in}{0.905508in}}%
\pgfpathclose%
\pgfusepath{fill}%
\end{pgfscope}%
\begin{pgfscope}%
\pgfpathrectangle{\pgfqpoint{0.329460in}{0.284240in}}{\pgfqpoint{1.989680in}{1.989680in}}%
\pgfusepath{clip}%
\pgfsetbuttcap%
\pgfsetroundjoin%
\definecolor{currentfill}{rgb}{0.122606,0.585371,0.546557}%
\pgfsetfillcolor{currentfill}%
\pgfsetlinewidth{0.000000pt}%
\definecolor{currentstroke}{rgb}{0.000000,0.000000,0.000000}%
\pgfsetstrokecolor{currentstroke}%
\pgfsetdash{}{0pt}%
\pgfpathmoveto{\pgfqpoint{1.605455in}{1.347981in}}%
\pgfpathlineto{\pgfqpoint{1.608568in}{1.340408in}}%
\pgfpathlineto{\pgfqpoint{1.611679in}{1.332817in}}%
\pgfpathlineto{\pgfqpoint{1.614789in}{1.325211in}}%
\pgfpathlineto{\pgfqpoint{1.617896in}{1.317591in}}%
\pgfpathlineto{\pgfqpoint{1.614179in}{1.313416in}}%
\pgfpathlineto{\pgfqpoint{1.610192in}{1.309299in}}%
\pgfpathlineto{\pgfqpoint{1.605939in}{1.305244in}}%
\pgfpathlineto{\pgfqpoint{1.601423in}{1.301256in}}%
\pgfpathlineto{\pgfqpoint{1.598500in}{1.309086in}}%
\pgfpathlineto{\pgfqpoint{1.595575in}{1.316903in}}%
\pgfpathlineto{\pgfqpoint{1.592649in}{1.324704in}}%
\pgfpathlineto{\pgfqpoint{1.589721in}{1.332487in}}%
\pgfpathlineto{\pgfqpoint{1.594033in}{1.336269in}}%
\pgfpathlineto{\pgfqpoint{1.598095in}{1.340115in}}%
\pgfpathlineto{\pgfqpoint{1.601904in}{1.344021in}}%
\pgfpathlineto{\pgfqpoint{1.605455in}{1.347981in}}%
\pgfpathclose%
\pgfusepath{fill}%
\end{pgfscope}%
\begin{pgfscope}%
\pgfpathrectangle{\pgfqpoint{0.329460in}{0.284240in}}{\pgfqpoint{1.989680in}{1.989680in}}%
\pgfusepath{clip}%
\pgfsetbuttcap%
\pgfsetroundjoin%
\definecolor{currentfill}{rgb}{0.896320,0.893616,0.096335}%
\pgfsetfillcolor{currentfill}%
\pgfsetlinewidth{0.000000pt}%
\definecolor{currentstroke}{rgb}{0.000000,0.000000,0.000000}%
\pgfsetstrokecolor{currentstroke}%
\pgfsetdash{}{0pt}%
\pgfpathmoveto{\pgfqpoint{1.368732in}{1.739062in}}%
\pgfpathlineto{\pgfqpoint{1.369602in}{1.738230in}}%
\pgfpathlineto{\pgfqpoint{1.370471in}{1.737286in}}%
\pgfpathlineto{\pgfqpoint{1.371339in}{1.736231in}}%
\pgfpathlineto{\pgfqpoint{1.372206in}{1.735065in}}%
\pgfpathlineto{\pgfqpoint{1.374750in}{1.734738in}}%
\pgfpathlineto{\pgfqpoint{1.377270in}{1.734373in}}%
\pgfpathlineto{\pgfqpoint{1.379764in}{1.733971in}}%
\pgfpathlineto{\pgfqpoint{1.378585in}{1.735182in}}%
\pgfpathlineto{\pgfqpoint{1.377404in}{1.736282in}}%
\pgfpathlineto{\pgfqpoint{1.376222in}{1.737272in}}%
\pgfpathlineto{\pgfqpoint{1.375039in}{1.738149in}}%
\pgfpathlineto{\pgfqpoint{1.372958in}{1.738484in}}%
\pgfpathlineto{\pgfqpoint{1.370854in}{1.738788in}}%
\pgfpathlineto{\pgfqpoint{1.368732in}{1.739062in}}%
\pgfpathclose%
\pgfusepath{fill}%
\end{pgfscope}%
\begin{pgfscope}%
\pgfpathrectangle{\pgfqpoint{0.329460in}{0.284240in}}{\pgfqpoint{1.989680in}{1.989680in}}%
\pgfusepath{clip}%
\pgfsetbuttcap%
\pgfsetroundjoin%
\definecolor{currentfill}{rgb}{0.134692,0.658636,0.517649}%
\pgfsetfillcolor{currentfill}%
\pgfsetlinewidth{0.000000pt}%
\definecolor{currentstroke}{rgb}{0.000000,0.000000,0.000000}%
\pgfsetstrokecolor{currentstroke}%
\pgfsetdash{}{0pt}%
\pgfpathmoveto{\pgfqpoint{1.124742in}{1.404502in}}%
\pgfpathlineto{\pgfqpoint{1.121651in}{1.397107in}}%
\pgfpathlineto{\pgfqpoint{1.118561in}{1.389677in}}%
\pgfpathlineto{\pgfqpoint{1.115474in}{1.382212in}}%
\pgfpathlineto{\pgfqpoint{1.112388in}{1.374716in}}%
\pgfpathlineto{\pgfqpoint{1.109031in}{1.378469in}}%
\pgfpathlineto{\pgfqpoint{1.105921in}{1.382271in}}%
\pgfpathlineto{\pgfqpoint{1.103061in}{1.386117in}}%
\pgfpathlineto{\pgfqpoint{1.100454in}{1.390003in}}%
\pgfpathlineto{\pgfqpoint{1.103686in}{1.397284in}}%
\pgfpathlineto{\pgfqpoint{1.106920in}{1.404533in}}%
\pgfpathlineto{\pgfqpoint{1.110156in}{1.411749in}}%
\pgfpathlineto{\pgfqpoint{1.113395in}{1.418929in}}%
\pgfpathlineto{\pgfqpoint{1.115875in}{1.415261in}}%
\pgfpathlineto{\pgfqpoint{1.118595in}{1.411632in}}%
\pgfpathlineto{\pgfqpoint{1.121551in}{1.408044in}}%
\pgfpathlineto{\pgfqpoint{1.124742in}{1.404502in}}%
\pgfpathclose%
\pgfusepath{fill}%
\end{pgfscope}%
\begin{pgfscope}%
\pgfpathrectangle{\pgfqpoint{0.329460in}{0.284240in}}{\pgfqpoint{1.989680in}{1.989680in}}%
\pgfusepath{clip}%
\pgfsetbuttcap%
\pgfsetroundjoin%
\definecolor{currentfill}{rgb}{0.896320,0.893616,0.096335}%
\pgfsetfillcolor{currentfill}%
\pgfsetlinewidth{0.000000pt}%
\definecolor{currentstroke}{rgb}{0.000000,0.000000,0.000000}%
\pgfsetstrokecolor{currentstroke}%
\pgfsetdash{}{0pt}%
\pgfpathmoveto{\pgfqpoint{1.325505in}{1.737826in}}%
\pgfpathlineto{\pgfqpoint{1.324231in}{1.736932in}}%
\pgfpathlineto{\pgfqpoint{1.322958in}{1.735927in}}%
\pgfpathlineto{\pgfqpoint{1.321687in}{1.734810in}}%
\pgfpathlineto{\pgfqpoint{1.320417in}{1.733583in}}%
\pgfpathlineto{\pgfqpoint{1.322887in}{1.734018in}}%
\pgfpathlineto{\pgfqpoint{1.325384in}{1.734415in}}%
\pgfpathlineto{\pgfqpoint{1.327907in}{1.734776in}}%
\pgfpathlineto{\pgfqpoint{1.330453in}{1.735099in}}%
\pgfpathlineto{\pgfqpoint{1.331308in}{1.736263in}}%
\pgfpathlineto{\pgfqpoint{1.332165in}{1.737317in}}%
\pgfpathlineto{\pgfqpoint{1.333022in}{1.738260in}}%
\pgfpathlineto{\pgfqpoint{1.333880in}{1.739090in}}%
\pgfpathlineto{\pgfqpoint{1.331756in}{1.738820in}}%
\pgfpathlineto{\pgfqpoint{1.329650in}{1.738519in}}%
\pgfpathlineto{\pgfqpoint{1.327566in}{1.738188in}}%
\pgfpathlineto{\pgfqpoint{1.325505in}{1.737826in}}%
\pgfpathclose%
\pgfusepath{fill}%
\end{pgfscope}%
\begin{pgfscope}%
\pgfpathrectangle{\pgfqpoint{0.329460in}{0.284240in}}{\pgfqpoint{1.989680in}{1.989680in}}%
\pgfusepath{clip}%
\pgfsetbuttcap%
\pgfsetroundjoin%
\definecolor{currentfill}{rgb}{0.487026,0.823929,0.312321}%
\pgfsetfillcolor{currentfill}%
\pgfsetlinewidth{0.000000pt}%
\definecolor{currentstroke}{rgb}{0.000000,0.000000,0.000000}%
\pgfsetstrokecolor{currentstroke}%
\pgfsetdash{}{0pt}%
\pgfpathmoveto{\pgfqpoint{1.184119in}{1.595017in}}%
\pgfpathlineto{\pgfqpoint{1.180682in}{1.589719in}}%
\pgfpathlineto{\pgfqpoint{1.177247in}{1.584341in}}%
\pgfpathlineto{\pgfqpoint{1.173815in}{1.578883in}}%
\pgfpathlineto{\pgfqpoint{1.170385in}{1.573349in}}%
\pgfpathlineto{\pgfqpoint{1.170163in}{1.576082in}}%
\pgfpathlineto{\pgfqpoint{1.170121in}{1.578815in}}%
\pgfpathlineto{\pgfqpoint{1.170259in}{1.581547in}}%
\pgfpathlineto{\pgfqpoint{1.170578in}{1.584273in}}%
\pgfpathlineto{\pgfqpoint{1.173997in}{1.589591in}}%
\pgfpathlineto{\pgfqpoint{1.177420in}{1.594832in}}%
\pgfpathlineto{\pgfqpoint{1.180845in}{1.599994in}}%
\pgfpathlineto{\pgfqpoint{1.184273in}{1.605076in}}%
\pgfpathlineto{\pgfqpoint{1.183985in}{1.602565in}}%
\pgfpathlineto{\pgfqpoint{1.183863in}{1.600050in}}%
\pgfpathlineto{\pgfqpoint{1.183907in}{1.597533in}}%
\pgfpathlineto{\pgfqpoint{1.184119in}{1.595017in}}%
\pgfpathclose%
\pgfusepath{fill}%
\end{pgfscope}%
\begin{pgfscope}%
\pgfpathrectangle{\pgfqpoint{0.329460in}{0.284240in}}{\pgfqpoint{1.989680in}{1.989680in}}%
\pgfusepath{clip}%
\pgfsetbuttcap%
\pgfsetroundjoin%
\definecolor{currentfill}{rgb}{0.699415,0.867117,0.175971}%
\pgfsetfillcolor{currentfill}%
\pgfsetlinewidth{0.000000pt}%
\definecolor{currentstroke}{rgb}{0.000000,0.000000,0.000000}%
\pgfsetstrokecolor{currentstroke}%
\pgfsetdash{}{0pt}%
\pgfpathmoveto{\pgfqpoint{1.469536in}{1.675510in}}%
\pgfpathlineto{\pgfqpoint{1.472786in}{1.671958in}}%
\pgfpathlineto{\pgfqpoint{1.476034in}{1.668310in}}%
\pgfpathlineto{\pgfqpoint{1.479278in}{1.664567in}}%
\pgfpathlineto{\pgfqpoint{1.482520in}{1.660730in}}%
\pgfpathlineto{\pgfqpoint{1.483955in}{1.658764in}}%
\pgfpathlineto{\pgfqpoint{1.485259in}{1.656776in}}%
\pgfpathlineto{\pgfqpoint{1.486430in}{1.654770in}}%
\pgfpathlineto{\pgfqpoint{1.487468in}{1.652747in}}%
\pgfpathlineto{\pgfqpoint{1.484099in}{1.656786in}}%
\pgfpathlineto{\pgfqpoint{1.480728in}{1.660732in}}%
\pgfpathlineto{\pgfqpoint{1.477355in}{1.664582in}}%
\pgfpathlineto{\pgfqpoint{1.473978in}{1.668336in}}%
\pgfpathlineto{\pgfqpoint{1.473048in}{1.670154in}}%
\pgfpathlineto{\pgfqpoint{1.471996in}{1.671957in}}%
\pgfpathlineto{\pgfqpoint{1.470825in}{1.673743in}}%
\pgfpathlineto{\pgfqpoint{1.469536in}{1.675510in}}%
\pgfpathclose%
\pgfusepath{fill}%
\end{pgfscope}%
\begin{pgfscope}%
\pgfpathrectangle{\pgfqpoint{0.329460in}{0.284240in}}{\pgfqpoint{1.989680in}{1.989680in}}%
\pgfusepath{clip}%
\pgfsetbuttcap%
\pgfsetroundjoin%
\definecolor{currentfill}{rgb}{0.762373,0.876424,0.137064}%
\pgfsetfillcolor{currentfill}%
\pgfsetlinewidth{0.000000pt}%
\definecolor{currentstroke}{rgb}{0.000000,0.000000,0.000000}%
\pgfsetstrokecolor{currentstroke}%
\pgfsetdash{}{0pt}%
\pgfpathmoveto{\pgfqpoint{1.450910in}{1.694811in}}%
\pgfpathlineto{\pgfqpoint{1.453993in}{1.691847in}}%
\pgfpathlineto{\pgfqpoint{1.457074in}{1.688783in}}%
\pgfpathlineto{\pgfqpoint{1.460152in}{1.685619in}}%
\pgfpathlineto{\pgfqpoint{1.463228in}{1.682356in}}%
\pgfpathlineto{\pgfqpoint{1.464975in}{1.680681in}}%
\pgfpathlineto{\pgfqpoint{1.466610in}{1.678981in}}%
\pgfpathlineto{\pgfqpoint{1.468131in}{1.677257in}}%
\pgfpathlineto{\pgfqpoint{1.469536in}{1.675510in}}%
\pgfpathlineto{\pgfqpoint{1.466284in}{1.678964in}}%
\pgfpathlineto{\pgfqpoint{1.463029in}{1.682320in}}%
\pgfpathlineto{\pgfqpoint{1.459771in}{1.685576in}}%
\pgfpathlineto{\pgfqpoint{1.456511in}{1.688731in}}%
\pgfpathlineto{\pgfqpoint{1.455263in}{1.690282in}}%
\pgfpathlineto{\pgfqpoint{1.453913in}{1.691813in}}%
\pgfpathlineto{\pgfqpoint{1.452461in}{1.693323in}}%
\pgfpathlineto{\pgfqpoint{1.450910in}{1.694811in}}%
\pgfpathclose%
\pgfusepath{fill}%
\end{pgfscope}%
\begin{pgfscope}%
\pgfpathrectangle{\pgfqpoint{0.329460in}{0.284240in}}{\pgfqpoint{1.989680in}{1.989680in}}%
\pgfusepath{clip}%
\pgfsetbuttcap%
\pgfsetroundjoin%
\definecolor{currentfill}{rgb}{0.855810,0.888601,0.097452}%
\pgfsetfillcolor{currentfill}%
\pgfsetlinewidth{0.000000pt}%
\definecolor{currentstroke}{rgb}{0.000000,0.000000,0.000000}%
\pgfsetstrokecolor{currentstroke}%
\pgfsetdash{}{0pt}%
\pgfpathmoveto{\pgfqpoint{1.406851in}{1.726448in}}%
\pgfpathlineto{\pgfqpoint{1.409150in}{1.724816in}}%
\pgfpathlineto{\pgfqpoint{1.411446in}{1.723074in}}%
\pgfpathlineto{\pgfqpoint{1.413741in}{1.721225in}}%
\pgfpathlineto{\pgfqpoint{1.416033in}{1.719269in}}%
\pgfpathlineto{\pgfqpoint{1.418309in}{1.718297in}}%
\pgfpathlineto{\pgfqpoint{1.420519in}{1.717291in}}%
\pgfpathlineto{\pgfqpoint{1.422661in}{1.716253in}}%
\pgfpathlineto{\pgfqpoint{1.424733in}{1.715184in}}%
\pgfpathlineto{\pgfqpoint{1.422131in}{1.717286in}}%
\pgfpathlineto{\pgfqpoint{1.419527in}{1.719281in}}%
\pgfpathlineto{\pgfqpoint{1.416921in}{1.721168in}}%
\pgfpathlineto{\pgfqpoint{1.414313in}{1.722946in}}%
\pgfpathlineto{\pgfqpoint{1.412536in}{1.723863in}}%
\pgfpathlineto{\pgfqpoint{1.410699in}{1.724752in}}%
\pgfpathlineto{\pgfqpoint{1.408803in}{1.725615in}}%
\pgfpathlineto{\pgfqpoint{1.406851in}{1.726448in}}%
\pgfpathclose%
\pgfusepath{fill}%
\end{pgfscope}%
\begin{pgfscope}%
\pgfpathrectangle{\pgfqpoint{0.329460in}{0.284240in}}{\pgfqpoint{1.989680in}{1.989680in}}%
\pgfusepath{clip}%
\pgfsetbuttcap%
\pgfsetroundjoin%
\definecolor{currentfill}{rgb}{0.896320,0.893616,0.096335}%
\pgfsetfillcolor{currentfill}%
\pgfsetlinewidth{0.000000pt}%
\definecolor{currentstroke}{rgb}{0.000000,0.000000,0.000000}%
\pgfsetstrokecolor{currentstroke}%
\pgfsetdash{}{0pt}%
\pgfpathmoveto{\pgfqpoint{1.375039in}{1.738149in}}%
\pgfpathlineto{\pgfqpoint{1.376222in}{1.737272in}}%
\pgfpathlineto{\pgfqpoint{1.377404in}{1.736282in}}%
\pgfpathlineto{\pgfqpoint{1.378585in}{1.735182in}}%
\pgfpathlineto{\pgfqpoint{1.379764in}{1.733971in}}%
\pgfpathlineto{\pgfqpoint{1.382231in}{1.733533in}}%
\pgfpathlineto{\pgfqpoint{1.384667in}{1.733059in}}%
\pgfpathlineto{\pgfqpoint{1.387071in}{1.732549in}}%
\pgfpathlineto{\pgfqpoint{1.389439in}{1.732004in}}%
\pgfpathlineto{\pgfqpoint{1.387860in}{1.733296in}}%
\pgfpathlineto{\pgfqpoint{1.386279in}{1.734478in}}%
\pgfpathlineto{\pgfqpoint{1.384697in}{1.735549in}}%
\pgfpathlineto{\pgfqpoint{1.383113in}{1.736508in}}%
\pgfpathlineto{\pgfqpoint{1.381137in}{1.736963in}}%
\pgfpathlineto{\pgfqpoint{1.379131in}{1.737388in}}%
\pgfpathlineto{\pgfqpoint{1.377098in}{1.737784in}}%
\pgfpathlineto{\pgfqpoint{1.375039in}{1.738149in}}%
\pgfpathclose%
\pgfusepath{fill}%
\end{pgfscope}%
\begin{pgfscope}%
\pgfpathrectangle{\pgfqpoint{0.329460in}{0.284240in}}{\pgfqpoint{1.989680in}{1.989680in}}%
\pgfusepath{clip}%
\pgfsetbuttcap%
\pgfsetroundjoin%
\definecolor{currentfill}{rgb}{0.281477,0.755203,0.432552}%
\pgfsetfillcolor{currentfill}%
\pgfsetlinewidth{0.000000pt}%
\definecolor{currentstroke}{rgb}{0.000000,0.000000,0.000000}%
\pgfsetstrokecolor{currentstroke}%
\pgfsetdash{}{0pt}%
\pgfpathmoveto{\pgfqpoint{1.557248in}{1.516673in}}%
\pgfpathlineto{\pgfqpoint{1.560625in}{1.510350in}}%
\pgfpathlineto{\pgfqpoint{1.564000in}{1.503968in}}%
\pgfpathlineto{\pgfqpoint{1.567371in}{1.497527in}}%
\pgfpathlineto{\pgfqpoint{1.570741in}{1.491031in}}%
\pgfpathlineto{\pgfqpoint{1.569581in}{1.487676in}}%
\pgfpathlineto{\pgfqpoint{1.568201in}{1.484338in}}%
\pgfpathlineto{\pgfqpoint{1.566602in}{1.481022in}}%
\pgfpathlineto{\pgfqpoint{1.564786in}{1.477730in}}%
\pgfpathlineto{\pgfqpoint{1.561501in}{1.484445in}}%
\pgfpathlineto{\pgfqpoint{1.558214in}{1.491104in}}%
\pgfpathlineto{\pgfqpoint{1.554924in}{1.497704in}}%
\pgfpathlineto{\pgfqpoint{1.551632in}{1.504244in}}%
\pgfpathlineto{\pgfqpoint{1.553344in}{1.507320in}}%
\pgfpathlineto{\pgfqpoint{1.554851in}{1.510419in}}%
\pgfpathlineto{\pgfqpoint{1.556153in}{1.513538in}}%
\pgfpathlineto{\pgfqpoint{1.557248in}{1.516673in}}%
\pgfpathclose%
\pgfusepath{fill}%
\end{pgfscope}%
\begin{pgfscope}%
\pgfpathrectangle{\pgfqpoint{0.329460in}{0.284240in}}{\pgfqpoint{1.989680in}{1.989680in}}%
\pgfusepath{clip}%
\pgfsetbuttcap%
\pgfsetroundjoin%
\definecolor{currentfill}{rgb}{0.267004,0.004874,0.329415}%
\pgfsetfillcolor{currentfill}%
\pgfsetlinewidth{0.000000pt}%
\definecolor{currentstroke}{rgb}{0.000000,0.000000,0.000000}%
\pgfsetstrokecolor{currentstroke}%
\pgfsetdash{}{0pt}%
\pgfpathmoveto{\pgfqpoint{0.982553in}{0.823695in}}%
\pgfpathlineto{\pgfqpoint{0.980108in}{0.822547in}}%
\pgfpathlineto{\pgfqpoint{0.977657in}{0.821634in}}%
\pgfpathlineto{\pgfqpoint{0.975201in}{0.820958in}}%
\pgfpathlineto{\pgfqpoint{0.972738in}{0.820525in}}%
\pgfpathlineto{\pgfqpoint{0.960323in}{0.827070in}}%
\pgfpathlineto{\pgfqpoint{0.948334in}{0.833815in}}%
\pgfpathlineto{\pgfqpoint{0.936783in}{0.840754in}}%
\pgfpathlineto{\pgfqpoint{0.925681in}{0.847877in}}%
\pgfpathlineto{\pgfqpoint{0.928439in}{0.848131in}}%
\pgfpathlineto{\pgfqpoint{0.931190in}{0.848627in}}%
\pgfpathlineto{\pgfqpoint{0.933935in}{0.849361in}}%
\pgfpathlineto{\pgfqpoint{0.936673in}{0.850328in}}%
\pgfpathlineto{\pgfqpoint{0.947498in}{0.843391in}}%
\pgfpathlineto{\pgfqpoint{0.958761in}{0.836635in}}%
\pgfpathlineto{\pgfqpoint{0.970450in}{0.830067in}}%
\pgfpathlineto{\pgfqpoint{0.982553in}{0.823695in}}%
\pgfpathclose%
\pgfusepath{fill}%
\end{pgfscope}%
\begin{pgfscope}%
\pgfpathrectangle{\pgfqpoint{0.329460in}{0.284240in}}{\pgfqpoint{1.989680in}{1.989680in}}%
\pgfusepath{clip}%
\pgfsetbuttcap%
\pgfsetroundjoin%
\definecolor{currentfill}{rgb}{0.896320,0.893616,0.096335}%
\pgfsetfillcolor{currentfill}%
\pgfsetlinewidth{0.000000pt}%
\definecolor{currentstroke}{rgb}{0.000000,0.000000,0.000000}%
\pgfsetstrokecolor{currentstroke}%
\pgfsetdash{}{0pt}%
\pgfpathmoveto{\pgfqpoint{1.317532in}{1.736080in}}%
\pgfpathlineto{\pgfqpoint{1.315862in}{1.735099in}}%
\pgfpathlineto{\pgfqpoint{1.314194in}{1.734007in}}%
\pgfpathlineto{\pgfqpoint{1.312527in}{1.732804in}}%
\pgfpathlineto{\pgfqpoint{1.310862in}{1.731490in}}%
\pgfpathlineto{\pgfqpoint{1.313197in}{1.732066in}}%
\pgfpathlineto{\pgfqpoint{1.315570in}{1.732607in}}%
\pgfpathlineto{\pgfqpoint{1.317977in}{1.733113in}}%
\pgfpathlineto{\pgfqpoint{1.320417in}{1.733583in}}%
\pgfpathlineto{\pgfqpoint{1.321687in}{1.734810in}}%
\pgfpathlineto{\pgfqpoint{1.322958in}{1.735927in}}%
\pgfpathlineto{\pgfqpoint{1.324231in}{1.736932in}}%
\pgfpathlineto{\pgfqpoint{1.325505in}{1.737826in}}%
\pgfpathlineto{\pgfqpoint{1.323469in}{1.737433in}}%
\pgfpathlineto{\pgfqpoint{1.321460in}{1.737011in}}%
\pgfpathlineto{\pgfqpoint{1.319480in}{1.736560in}}%
\pgfpathlineto{\pgfqpoint{1.317532in}{1.736080in}}%
\pgfpathclose%
\pgfusepath{fill}%
\end{pgfscope}%
\begin{pgfscope}%
\pgfpathrectangle{\pgfqpoint{0.329460in}{0.284240in}}{\pgfqpoint{1.989680in}{1.989680in}}%
\pgfusepath{clip}%
\pgfsetbuttcap%
\pgfsetroundjoin%
\definecolor{currentfill}{rgb}{0.855810,0.888601,0.097452}%
\pgfsetfillcolor{currentfill}%
\pgfsetlinewidth{0.000000pt}%
\definecolor{currentstroke}{rgb}{0.000000,0.000000,0.000000}%
\pgfsetstrokecolor{currentstroke}%
\pgfsetdash{}{0pt}%
\pgfpathmoveto{\pgfqpoint{1.286535in}{1.722110in}}%
\pgfpathlineto{\pgfqpoint{1.283863in}{1.720296in}}%
\pgfpathlineto{\pgfqpoint{1.281193in}{1.718374in}}%
\pgfpathlineto{\pgfqpoint{1.278526in}{1.716345in}}%
\pgfpathlineto{\pgfqpoint{1.275861in}{1.714208in}}%
\pgfpathlineto{\pgfqpoint{1.277869in}{1.715304in}}%
\pgfpathlineto{\pgfqpoint{1.279949in}{1.716370in}}%
\pgfpathlineto{\pgfqpoint{1.282098in}{1.717405in}}%
\pgfpathlineto{\pgfqpoint{1.284316in}{1.718407in}}%
\pgfpathlineto{\pgfqpoint{1.286680in}{1.720393in}}%
\pgfpathlineto{\pgfqpoint{1.289047in}{1.722273in}}%
\pgfpathlineto{\pgfqpoint{1.291415in}{1.724045in}}%
\pgfpathlineto{\pgfqpoint{1.293786in}{1.725709in}}%
\pgfpathlineto{\pgfqpoint{1.291884in}{1.724850in}}%
\pgfpathlineto{\pgfqpoint{1.290040in}{1.723963in}}%
\pgfpathlineto{\pgfqpoint{1.288257in}{1.723049in}}%
\pgfpathlineto{\pgfqpoint{1.286535in}{1.722110in}}%
\pgfpathclose%
\pgfusepath{fill}%
\end{pgfscope}%
\begin{pgfscope}%
\pgfpathrectangle{\pgfqpoint{0.329460in}{0.284240in}}{\pgfqpoint{1.989680in}{1.989680in}}%
\pgfusepath{clip}%
\pgfsetbuttcap%
\pgfsetroundjoin%
\definecolor{currentfill}{rgb}{0.636902,0.856542,0.216620}%
\pgfsetfillcolor{currentfill}%
\pgfsetlinewidth{0.000000pt}%
\definecolor{currentstroke}{rgb}{0.000000,0.000000,0.000000}%
\pgfsetstrokecolor{currentstroke}%
\pgfsetdash{}{0pt}%
\pgfpathmoveto{\pgfqpoint{1.487468in}{1.652747in}}%
\pgfpathlineto{\pgfqpoint{1.490834in}{1.648615in}}%
\pgfpathlineto{\pgfqpoint{1.494197in}{1.644391in}}%
\pgfpathlineto{\pgfqpoint{1.497557in}{1.640078in}}%
\pgfpathlineto{\pgfqpoint{1.500915in}{1.635676in}}%
\pgfpathlineto{\pgfqpoint{1.501912in}{1.633430in}}%
\pgfpathlineto{\pgfqpoint{1.502760in}{1.631169in}}%
\pgfpathlineto{\pgfqpoint{1.503458in}{1.628895in}}%
\pgfpathlineto{\pgfqpoint{1.504004in}{1.626612in}}%
\pgfpathlineto{\pgfqpoint{1.500572in}{1.631224in}}%
\pgfpathlineto{\pgfqpoint{1.497138in}{1.635747in}}%
\pgfpathlineto{\pgfqpoint{1.493700in}{1.640180in}}%
\pgfpathlineto{\pgfqpoint{1.490261in}{1.644522in}}%
\pgfpathlineto{\pgfqpoint{1.489768in}{1.646594in}}%
\pgfpathlineto{\pgfqpoint{1.489138in}{1.648657in}}%
\pgfpathlineto{\pgfqpoint{1.488371in}{1.650708in}}%
\pgfpathlineto{\pgfqpoint{1.487468in}{1.652747in}}%
\pgfpathclose%
\pgfusepath{fill}%
\end{pgfscope}%
\begin{pgfscope}%
\pgfpathrectangle{\pgfqpoint{0.329460in}{0.284240in}}{\pgfqpoint{1.989680in}{1.989680in}}%
\pgfusepath{clip}%
\pgfsetbuttcap%
\pgfsetroundjoin%
\definecolor{currentfill}{rgb}{0.231674,0.318106,0.544834}%
\pgfsetfillcolor{currentfill}%
\pgfsetlinewidth{0.000000pt}%
\definecolor{currentstroke}{rgb}{0.000000,0.000000,0.000000}%
\pgfsetstrokecolor{currentstroke}%
\pgfsetdash{}{0pt}%
\pgfpathmoveto{\pgfqpoint{1.095927in}{1.073983in}}%
\pgfpathlineto{\pgfqpoint{1.093600in}{1.066239in}}%
\pgfpathlineto{\pgfqpoint{1.091273in}{1.058551in}}%
\pgfpathlineto{\pgfqpoint{1.088946in}{1.050924in}}%
\pgfpathlineto{\pgfqpoint{1.086618in}{1.043360in}}%
\pgfpathlineto{\pgfqpoint{1.077859in}{1.047784in}}%
\pgfpathlineto{\pgfqpoint{1.069389in}{1.052347in}}%
\pgfpathlineto{\pgfqpoint{1.061219in}{1.057043in}}%
\pgfpathlineto{\pgfqpoint{1.053356in}{1.061867in}}%
\pgfpathlineto{\pgfqpoint{1.055968in}{1.069248in}}%
\pgfpathlineto{\pgfqpoint{1.058581in}{1.076692in}}%
\pgfpathlineto{\pgfqpoint{1.061194in}{1.084196in}}%
\pgfpathlineto{\pgfqpoint{1.063806in}{1.091757in}}%
\pgfpathlineto{\pgfqpoint{1.071401in}{1.087123in}}%
\pgfpathlineto{\pgfqpoint{1.079291in}{1.082613in}}%
\pgfpathlineto{\pgfqpoint{1.087469in}{1.078231in}}%
\pgfpathlineto{\pgfqpoint{1.095927in}{1.073983in}}%
\pgfpathclose%
\pgfusepath{fill}%
\end{pgfscope}%
\begin{pgfscope}%
\pgfpathrectangle{\pgfqpoint{0.329460in}{0.284240in}}{\pgfqpoint{1.989680in}{1.989680in}}%
\pgfusepath{clip}%
\pgfsetbuttcap%
\pgfsetroundjoin%
\definecolor{currentfill}{rgb}{0.212395,0.359683,0.551710}%
\pgfsetfillcolor{currentfill}%
\pgfsetlinewidth{0.000000pt}%
\definecolor{currentstroke}{rgb}{0.000000,0.000000,0.000000}%
\pgfsetstrokecolor{currentstroke}%
\pgfsetdash{}{0pt}%
\pgfpathmoveto{\pgfqpoint{1.634385in}{1.126556in}}%
\pgfpathlineto{\pgfqpoint{1.637055in}{1.118840in}}%
\pgfpathlineto{\pgfqpoint{1.639726in}{1.111169in}}%
\pgfpathlineto{\pgfqpoint{1.642396in}{1.103546in}}%
\pgfpathlineto{\pgfqpoint{1.645066in}{1.095975in}}%
\pgfpathlineto{\pgfqpoint{1.637740in}{1.091236in}}%
\pgfpathlineto{\pgfqpoint{1.630112in}{1.086616in}}%
\pgfpathlineto{\pgfqpoint{1.622189in}{1.082120in}}%
\pgfpathlineto{\pgfqpoint{1.613980in}{1.077753in}}%
\pgfpathlineto{\pgfqpoint{1.611585in}{1.085512in}}%
\pgfpathlineto{\pgfqpoint{1.609191in}{1.093321in}}%
\pgfpathlineto{\pgfqpoint{1.606797in}{1.101179in}}%
\pgfpathlineto{\pgfqpoint{1.604402in}{1.109082in}}%
\pgfpathlineto{\pgfqpoint{1.612319in}{1.113270in}}%
\pgfpathlineto{\pgfqpoint{1.619960in}{1.117581in}}%
\pgfpathlineto{\pgfqpoint{1.627318in}{1.122012in}}%
\pgfpathlineto{\pgfqpoint{1.634385in}{1.126556in}}%
\pgfpathclose%
\pgfusepath{fill}%
\end{pgfscope}%
\begin{pgfscope}%
\pgfpathrectangle{\pgfqpoint{0.329460in}{0.284240in}}{\pgfqpoint{1.989680in}{1.989680in}}%
\pgfusepath{clip}%
\pgfsetbuttcap%
\pgfsetroundjoin%
\definecolor{currentfill}{rgb}{0.201239,0.383670,0.554294}%
\pgfsetfillcolor{currentfill}%
\pgfsetlinewidth{0.000000pt}%
\definecolor{currentstroke}{rgb}{0.000000,0.000000,0.000000}%
\pgfsetstrokecolor{currentstroke}%
\pgfsetdash{}{0pt}%
\pgfpathmoveto{\pgfqpoint{1.988641in}{1.122523in}}%
\pgfpathlineto{\pgfqpoint{1.992423in}{1.133795in}}%
\pgfpathlineto{\pgfqpoint{1.996224in}{1.145500in}}%
\pgfpathlineto{\pgfqpoint{2.000046in}{1.157645in}}%
\pgfpathlineto{\pgfqpoint{2.003889in}{1.170239in}}%
\pgfpathlineto{\pgfqpoint{1.998398in}{1.159744in}}%
\pgfpathlineto{\pgfqpoint{1.992232in}{1.149327in}}%
\pgfpathlineto{\pgfqpoint{1.985393in}{1.138997in}}%
\pgfpathlineto{\pgfqpoint{1.977885in}{1.128768in}}%
\pgfpathlineto{\pgfqpoint{1.974171in}{1.116338in}}%
\pgfpathlineto{\pgfqpoint{1.970478in}{1.104359in}}%
\pgfpathlineto{\pgfqpoint{1.966805in}{1.092823in}}%
\pgfpathlineto{\pgfqpoint{1.963151in}{1.081722in}}%
\pgfpathlineto{\pgfqpoint{1.970505in}{1.091785in}}%
\pgfpathlineto{\pgfqpoint{1.977208in}{1.101947in}}%
\pgfpathlineto{\pgfqpoint{1.983254in}{1.112197in}}%
\pgfpathlineto{\pgfqpoint{1.988641in}{1.122523in}}%
\pgfpathclose%
\pgfusepath{fill}%
\end{pgfscope}%
\begin{pgfscope}%
\pgfpathrectangle{\pgfqpoint{0.329460in}{0.284240in}}{\pgfqpoint{1.989680in}{1.989680in}}%
\pgfusepath{clip}%
\pgfsetbuttcap%
\pgfsetroundjoin%
\definecolor{currentfill}{rgb}{0.268510,0.009605,0.335427}%
\pgfsetfillcolor{currentfill}%
\pgfsetlinewidth{0.000000pt}%
\definecolor{currentstroke}{rgb}{0.000000,0.000000,0.000000}%
\pgfsetstrokecolor{currentstroke}%
\pgfsetdash{}{0pt}%
\pgfpathmoveto{\pgfqpoint{1.786176in}{0.854358in}}%
\pgfpathlineto{\pgfqpoint{1.789000in}{0.854393in}}%
\pgfpathlineto{\pgfqpoint{1.791831in}{0.854679in}}%
\pgfpathlineto{\pgfqpoint{1.794670in}{0.855222in}}%
\pgfpathlineto{\pgfqpoint{1.797517in}{0.856026in}}%
\pgfpathlineto{\pgfqpoint{1.786558in}{0.848557in}}%
\pgfpathlineto{\pgfqpoint{1.775126in}{0.841270in}}%
\pgfpathlineto{\pgfqpoint{1.763234in}{0.834173in}}%
\pgfpathlineto{\pgfqpoint{1.750893in}{0.827275in}}%
\pgfpathlineto{\pgfqpoint{1.748333in}{0.826651in}}%
\pgfpathlineto{\pgfqpoint{1.745779in}{0.826290in}}%
\pgfpathlineto{\pgfqpoint{1.743233in}{0.826185in}}%
\pgfpathlineto{\pgfqpoint{1.740694in}{0.826332in}}%
\pgfpathlineto{\pgfqpoint{1.752731in}{0.833056in}}%
\pgfpathlineto{\pgfqpoint{1.764331in}{0.839974in}}%
\pgfpathlineto{\pgfqpoint{1.775483in}{0.847077in}}%
\pgfpathlineto{\pgfqpoint{1.786176in}{0.854358in}}%
\pgfpathclose%
\pgfusepath{fill}%
\end{pgfscope}%
\begin{pgfscope}%
\pgfpathrectangle{\pgfqpoint{0.329460in}{0.284240in}}{\pgfqpoint{1.989680in}{1.989680in}}%
\pgfusepath{clip}%
\pgfsetbuttcap%
\pgfsetroundjoin%
\definecolor{currentfill}{rgb}{0.762373,0.876424,0.137064}%
\pgfsetfillcolor{currentfill}%
\pgfsetlinewidth{0.000000pt}%
\definecolor{currentstroke}{rgb}{0.000000,0.000000,0.000000}%
\pgfsetstrokecolor{currentstroke}%
\pgfsetdash{}{0pt}%
\pgfpathmoveto{\pgfqpoint{1.244843in}{1.687337in}}%
\pgfpathlineto{\pgfqpoint{1.241550in}{1.684138in}}%
\pgfpathlineto{\pgfqpoint{1.238260in}{1.680838in}}%
\pgfpathlineto{\pgfqpoint{1.234972in}{1.677439in}}%
\pgfpathlineto{\pgfqpoint{1.231687in}{1.673940in}}%
\pgfpathlineto{\pgfqpoint{1.232989in}{1.675705in}}%
\pgfpathlineto{\pgfqpoint{1.234408in}{1.677449in}}%
\pgfpathlineto{\pgfqpoint{1.235942in}{1.679171in}}%
\pgfpathlineto{\pgfqpoint{1.237589in}{1.680869in}}%
\pgfpathlineto{\pgfqpoint{1.240708in}{1.684173in}}%
\pgfpathlineto{\pgfqpoint{1.243830in}{1.687379in}}%
\pgfpathlineto{\pgfqpoint{1.246954in}{1.690485in}}%
\pgfpathlineto{\pgfqpoint{1.250081in}{1.693490in}}%
\pgfpathlineto{\pgfqpoint{1.248619in}{1.691982in}}%
\pgfpathlineto{\pgfqpoint{1.247257in}{1.690453in}}%
\pgfpathlineto{\pgfqpoint{1.245998in}{1.688904in}}%
\pgfpathlineto{\pgfqpoint{1.244843in}{1.687337in}}%
\pgfpathclose%
\pgfusepath{fill}%
\end{pgfscope}%
\begin{pgfscope}%
\pgfpathrectangle{\pgfqpoint{0.329460in}{0.284240in}}{\pgfqpoint{1.989680in}{1.989680in}}%
\pgfusepath{clip}%
\pgfsetbuttcap%
\pgfsetroundjoin%
\definecolor{currentfill}{rgb}{0.814576,0.883393,0.110347}%
\pgfsetfillcolor{currentfill}%
\pgfsetlinewidth{0.000000pt}%
\definecolor{currentstroke}{rgb}{0.000000,0.000000,0.000000}%
\pgfsetstrokecolor{currentstroke}%
\pgfsetdash{}{0pt}%
\pgfpathmoveto{\pgfqpoint{1.432280in}{1.710615in}}%
\pgfpathlineto{\pgfqpoint{1.435148in}{1.708245in}}%
\pgfpathlineto{\pgfqpoint{1.438014in}{1.705769in}}%
\pgfpathlineto{\pgfqpoint{1.440878in}{1.703189in}}%
\pgfpathlineto{\pgfqpoint{1.443740in}{1.700506in}}%
\pgfpathlineto{\pgfqpoint{1.445673in}{1.699123in}}%
\pgfpathlineto{\pgfqpoint{1.447514in}{1.697712in}}%
\pgfpathlineto{\pgfqpoint{1.449260in}{1.696274in}}%
\pgfpathlineto{\pgfqpoint{1.450910in}{1.694811in}}%
\pgfpathlineto{\pgfqpoint{1.447824in}{1.697672in}}%
\pgfpathlineto{\pgfqpoint{1.444735in}{1.700431in}}%
\pgfpathlineto{\pgfqpoint{1.441645in}{1.703085in}}%
\pgfpathlineto{\pgfqpoint{1.438552in}{1.705635in}}%
\pgfpathlineto{\pgfqpoint{1.437109in}{1.706914in}}%
\pgfpathlineto{\pgfqpoint{1.435582in}{1.708172in}}%
\pgfpathlineto{\pgfqpoint{1.433971in}{1.709406in}}%
\pgfpathlineto{\pgfqpoint{1.432280in}{1.710615in}}%
\pgfpathclose%
\pgfusepath{fill}%
\end{pgfscope}%
\begin{pgfscope}%
\pgfpathrectangle{\pgfqpoint{0.329460in}{0.284240in}}{\pgfqpoint{1.989680in}{1.989680in}}%
\pgfusepath{clip}%
\pgfsetbuttcap%
\pgfsetroundjoin%
\definecolor{currentfill}{rgb}{0.412913,0.803041,0.357269}%
\pgfsetfillcolor{currentfill}%
\pgfsetlinewidth{0.000000pt}%
\definecolor{currentstroke}{rgb}{0.000000,0.000000,0.000000}%
\pgfsetstrokecolor{currentstroke}%
\pgfsetdash{}{0pt}%
\pgfpathmoveto{\pgfqpoint{1.532197in}{1.575778in}}%
\pgfpathlineto{\pgfqpoint{1.535626in}{1.570216in}}%
\pgfpathlineto{\pgfqpoint{1.539053in}{1.564581in}}%
\pgfpathlineto{\pgfqpoint{1.542477in}{1.558874in}}%
\pgfpathlineto{\pgfqpoint{1.545898in}{1.553096in}}%
\pgfpathlineto{\pgfqpoint{1.545644in}{1.550146in}}%
\pgfpathlineto{\pgfqpoint{1.545195in}{1.547199in}}%
\pgfpathlineto{\pgfqpoint{1.544552in}{1.544259in}}%
\pgfpathlineto{\pgfqpoint{1.543715in}{1.541329in}}%
\pgfpathlineto{\pgfqpoint{1.540326in}{1.547325in}}%
\pgfpathlineto{\pgfqpoint{1.536934in}{1.553250in}}%
\pgfpathlineto{\pgfqpoint{1.533539in}{1.559103in}}%
\pgfpathlineto{\pgfqpoint{1.530142in}{1.564882in}}%
\pgfpathlineto{\pgfqpoint{1.530926in}{1.567595in}}%
\pgfpathlineto{\pgfqpoint{1.531530in}{1.570317in}}%
\pgfpathlineto{\pgfqpoint{1.531954in}{1.573046in}}%
\pgfpathlineto{\pgfqpoint{1.532197in}{1.575778in}}%
\pgfpathclose%
\pgfusepath{fill}%
\end{pgfscope}%
\begin{pgfscope}%
\pgfpathrectangle{\pgfqpoint{0.329460in}{0.284240in}}{\pgfqpoint{1.989680in}{1.989680in}}%
\pgfusepath{clip}%
\pgfsetbuttcap%
\pgfsetroundjoin%
\definecolor{currentfill}{rgb}{0.699415,0.867117,0.175971}%
\pgfsetfillcolor{currentfill}%
\pgfsetlinewidth{0.000000pt}%
\definecolor{currentstroke}{rgb}{0.000000,0.000000,0.000000}%
\pgfsetstrokecolor{currentstroke}%
\pgfsetdash{}{0pt}%
\pgfpathmoveto{\pgfqpoint{1.227671in}{1.666708in}}%
\pgfpathlineto{\pgfqpoint{1.224274in}{1.662908in}}%
\pgfpathlineto{\pgfqpoint{1.220880in}{1.659012in}}%
\pgfpathlineto{\pgfqpoint{1.217488in}{1.655021in}}%
\pgfpathlineto{\pgfqpoint{1.214098in}{1.650935in}}%
\pgfpathlineto{\pgfqpoint{1.215016in}{1.652972in}}%
\pgfpathlineto{\pgfqpoint{1.216069in}{1.654994in}}%
\pgfpathlineto{\pgfqpoint{1.217255in}{1.656998in}}%
\pgfpathlineto{\pgfqpoint{1.218573in}{1.658983in}}%
\pgfpathlineto{\pgfqpoint{1.221848in}{1.662864in}}%
\pgfpathlineto{\pgfqpoint{1.225125in}{1.666652in}}%
\pgfpathlineto{\pgfqpoint{1.228405in}{1.670344in}}%
\pgfpathlineto{\pgfqpoint{1.231687in}{1.673940in}}%
\pgfpathlineto{\pgfqpoint{1.230503in}{1.672156in}}%
\pgfpathlineto{\pgfqpoint{1.229438in}{1.670355in}}%
\pgfpathlineto{\pgfqpoint{1.228494in}{1.668538in}}%
\pgfpathlineto{\pgfqpoint{1.227671in}{1.666708in}}%
\pgfpathclose%
\pgfusepath{fill}%
\end{pgfscope}%
\begin{pgfscope}%
\pgfpathrectangle{\pgfqpoint{0.329460in}{0.284240in}}{\pgfqpoint{1.989680in}{1.989680in}}%
\pgfusepath{clip}%
\pgfsetbuttcap%
\pgfsetroundjoin%
\definecolor{currentfill}{rgb}{0.166383,0.690856,0.496502}%
\pgfsetfillcolor{currentfill}%
\pgfsetlinewidth{0.000000pt}%
\definecolor{currentstroke}{rgb}{0.000000,0.000000,0.000000}%
\pgfsetstrokecolor{currentstroke}%
\pgfsetdash{}{0pt}%
\pgfpathmoveto{\pgfqpoint{1.577903in}{1.450345in}}%
\pgfpathlineto{\pgfqpoint{1.581176in}{1.443377in}}%
\pgfpathlineto{\pgfqpoint{1.584447in}{1.436366in}}%
\pgfpathlineto{\pgfqpoint{1.587716in}{1.429312in}}%
\pgfpathlineto{\pgfqpoint{1.590982in}{1.422219in}}%
\pgfpathlineto{\pgfqpoint{1.588717in}{1.418520in}}%
\pgfpathlineto{\pgfqpoint{1.586210in}{1.414856in}}%
\pgfpathlineto{\pgfqpoint{1.583463in}{1.411231in}}%
\pgfpathlineto{\pgfqpoint{1.580481in}{1.407648in}}%
\pgfpathlineto{\pgfqpoint{1.577350in}{1.414957in}}%
\pgfpathlineto{\pgfqpoint{1.574217in}{1.422226in}}%
\pgfpathlineto{\pgfqpoint{1.571082in}{1.429452in}}%
\pgfpathlineto{\pgfqpoint{1.567946in}{1.436634in}}%
\pgfpathlineto{\pgfqpoint{1.570772in}{1.440005in}}%
\pgfpathlineto{\pgfqpoint{1.573376in}{1.443416in}}%
\pgfpathlineto{\pgfqpoint{1.575753in}{1.446864in}}%
\pgfpathlineto{\pgfqpoint{1.577903in}{1.450345in}}%
\pgfpathclose%
\pgfusepath{fill}%
\end{pgfscope}%
\begin{pgfscope}%
\pgfpathrectangle{\pgfqpoint{0.329460in}{0.284240in}}{\pgfqpoint{1.989680in}{1.989680in}}%
\pgfusepath{clip}%
\pgfsetbuttcap%
\pgfsetroundjoin%
\definecolor{currentfill}{rgb}{0.163625,0.471133,0.558148}%
\pgfsetfillcolor{currentfill}%
\pgfsetlinewidth{0.000000pt}%
\definecolor{currentstroke}{rgb}{0.000000,0.000000,0.000000}%
\pgfsetstrokecolor{currentstroke}%
\pgfsetdash{}{0pt}%
\pgfpathmoveto{\pgfqpoint{1.105659in}{1.218107in}}%
\pgfpathlineto{\pgfqpoint{1.103038in}{1.210027in}}%
\pgfpathlineto{\pgfqpoint{1.100418in}{1.201958in}}%
\pgfpathlineto{\pgfqpoint{1.097799in}{1.193904in}}%
\pgfpathlineto{\pgfqpoint{1.095181in}{1.185867in}}%
\pgfpathlineto{\pgfqpoint{1.088664in}{1.190038in}}%
\pgfpathlineto{\pgfqpoint{1.082421in}{1.194309in}}%
\pgfpathlineto{\pgfqpoint{1.076459in}{1.198675in}}%
\pgfpathlineto{\pgfqpoint{1.070784in}{1.203131in}}%
\pgfpathlineto{\pgfqpoint{1.073644in}{1.210971in}}%
\pgfpathlineto{\pgfqpoint{1.076505in}{1.218828in}}%
\pgfpathlineto{\pgfqpoint{1.079367in}{1.226699in}}%
\pgfpathlineto{\pgfqpoint{1.082231in}{1.234583in}}%
\pgfpathlineto{\pgfqpoint{1.087682in}{1.230330in}}%
\pgfpathlineto{\pgfqpoint{1.093408in}{1.226163in}}%
\pgfpathlineto{\pgfqpoint{1.099402in}{1.222088in}}%
\pgfpathlineto{\pgfqpoint{1.105659in}{1.218107in}}%
\pgfpathclose%
\pgfusepath{fill}%
\end{pgfscope}%
\begin{pgfscope}%
\pgfpathrectangle{\pgfqpoint{0.329460in}{0.284240in}}{\pgfqpoint{1.989680in}{1.989680in}}%
\pgfusepath{clip}%
\pgfsetbuttcap%
\pgfsetroundjoin%
\definecolor{currentfill}{rgb}{0.122606,0.585371,0.546557}%
\pgfsetfillcolor{currentfill}%
\pgfsetlinewidth{0.000000pt}%
\definecolor{currentstroke}{rgb}{0.000000,0.000000,0.000000}%
\pgfsetstrokecolor{currentstroke}%
\pgfsetdash{}{0pt}%
\pgfpathmoveto{\pgfqpoint{1.116694in}{1.329181in}}%
\pgfpathlineto{\pgfqpoint{1.113814in}{1.321353in}}%
\pgfpathlineto{\pgfqpoint{1.110936in}{1.313507in}}%
\pgfpathlineto{\pgfqpoint{1.108059in}{1.305646in}}%
\pgfpathlineto{\pgfqpoint{1.105183in}{1.297770in}}%
\pgfpathlineto{\pgfqpoint{1.100438in}{1.301696in}}%
\pgfpathlineto{\pgfqpoint{1.095951in}{1.305692in}}%
\pgfpathlineto{\pgfqpoint{1.091727in}{1.309754in}}%
\pgfpathlineto{\pgfqpoint{1.087770in}{1.313877in}}%
\pgfpathlineto{\pgfqpoint{1.090841in}{1.321545in}}%
\pgfpathlineto{\pgfqpoint{1.093913in}{1.329199in}}%
\pgfpathlineto{\pgfqpoint{1.096988in}{1.336837in}}%
\pgfpathlineto{\pgfqpoint{1.100064in}{1.344458in}}%
\pgfpathlineto{\pgfqpoint{1.103844in}{1.340546in}}%
\pgfpathlineto{\pgfqpoint{1.107878in}{1.336693in}}%
\pgfpathlineto{\pgfqpoint{1.112163in}{1.332904in}}%
\pgfpathlineto{\pgfqpoint{1.116694in}{1.329181in}}%
\pgfpathclose%
\pgfusepath{fill}%
\end{pgfscope}%
\begin{pgfscope}%
\pgfpathrectangle{\pgfqpoint{0.329460in}{0.284240in}}{\pgfqpoint{1.989680in}{1.989680in}}%
\pgfusepath{clip}%
\pgfsetbuttcap%
\pgfsetroundjoin%
\definecolor{currentfill}{rgb}{0.896320,0.893616,0.096335}%
\pgfsetfillcolor{currentfill}%
\pgfsetlinewidth{0.000000pt}%
\definecolor{currentstroke}{rgb}{0.000000,0.000000,0.000000}%
\pgfsetstrokecolor{currentstroke}%
\pgfsetdash{}{0pt}%
\pgfpathmoveto{\pgfqpoint{1.383113in}{1.736508in}}%
\pgfpathlineto{\pgfqpoint{1.384697in}{1.735549in}}%
\pgfpathlineto{\pgfqpoint{1.386279in}{1.734478in}}%
\pgfpathlineto{\pgfqpoint{1.387860in}{1.733296in}}%
\pgfpathlineto{\pgfqpoint{1.389439in}{1.732004in}}%
\pgfpathlineto{\pgfqpoint{1.391770in}{1.731424in}}%
\pgfpathlineto{\pgfqpoint{1.394062in}{1.730810in}}%
\pgfpathlineto{\pgfqpoint{1.396311in}{1.730163in}}%
\pgfpathlineto{\pgfqpoint{1.398516in}{1.729483in}}%
\pgfpathlineto{\pgfqpoint{1.396561in}{1.730880in}}%
\pgfpathlineto{\pgfqpoint{1.394604in}{1.732167in}}%
\pgfpathlineto{\pgfqpoint{1.392646in}{1.733343in}}%
\pgfpathlineto{\pgfqpoint{1.390686in}{1.734406in}}%
\pgfpathlineto{\pgfqpoint{1.388846in}{1.734973in}}%
\pgfpathlineto{\pgfqpoint{1.386970in}{1.735513in}}%
\pgfpathlineto{\pgfqpoint{1.385058in}{1.736025in}}%
\pgfpathlineto{\pgfqpoint{1.383113in}{1.736508in}}%
\pgfpathclose%
\pgfusepath{fill}%
\end{pgfscope}%
\begin{pgfscope}%
\pgfpathrectangle{\pgfqpoint{0.329460in}{0.284240in}}{\pgfqpoint{1.989680in}{1.989680in}}%
\pgfusepath{clip}%
\pgfsetbuttcap%
\pgfsetroundjoin%
\definecolor{currentfill}{rgb}{0.814576,0.883393,0.110347}%
\pgfsetfillcolor{currentfill}%
\pgfsetlinewidth{0.000000pt}%
\definecolor{currentstroke}{rgb}{0.000000,0.000000,0.000000}%
\pgfsetstrokecolor{currentstroke}%
\pgfsetdash{}{0pt}%
\pgfpathmoveto{\pgfqpoint{1.262614in}{1.704479in}}%
\pgfpathlineto{\pgfqpoint{1.259477in}{1.701889in}}%
\pgfpathlineto{\pgfqpoint{1.256343in}{1.699193in}}%
\pgfpathlineto{\pgfqpoint{1.253211in}{1.696393in}}%
\pgfpathlineto{\pgfqpoint{1.250081in}{1.693490in}}%
\pgfpathlineto{\pgfqpoint{1.251644in}{1.694974in}}%
\pgfpathlineto{\pgfqpoint{1.253304in}{1.696435in}}%
\pgfpathlineto{\pgfqpoint{1.255061in}{1.697870in}}%
\pgfpathlineto{\pgfqpoint{1.256912in}{1.699278in}}%
\pgfpathlineto{\pgfqpoint{1.259827in}{1.701999in}}%
\pgfpathlineto{\pgfqpoint{1.262745in}{1.704618in}}%
\pgfpathlineto{\pgfqpoint{1.265665in}{1.707132in}}%
\pgfpathlineto{\pgfqpoint{1.268588in}{1.709542in}}%
\pgfpathlineto{\pgfqpoint{1.266968in}{1.708310in}}%
\pgfpathlineto{\pgfqpoint{1.265432in}{1.707055in}}%
\pgfpathlineto{\pgfqpoint{1.263980in}{1.705778in}}%
\pgfpathlineto{\pgfqpoint{1.262614in}{1.704479in}}%
\pgfpathclose%
\pgfusepath{fill}%
\end{pgfscope}%
\begin{pgfscope}%
\pgfpathrectangle{\pgfqpoint{0.329460in}{0.284240in}}{\pgfqpoint{1.989680in}{1.989680in}}%
\pgfusepath{clip}%
\pgfsetbuttcap%
\pgfsetroundjoin%
\definecolor{currentfill}{rgb}{0.282327,0.094955,0.417331}%
\pgfsetfillcolor{currentfill}%
\pgfsetlinewidth{0.000000pt}%
\definecolor{currentstroke}{rgb}{0.000000,0.000000,0.000000}%
\pgfsetstrokecolor{currentstroke}%
\pgfsetdash{}{0pt}%
\pgfpathmoveto{\pgfqpoint{1.671625in}{0.914914in}}%
\pgfpathlineto{\pgfqpoint{1.674045in}{0.909459in}}%
\pgfpathlineto{\pgfqpoint{1.676466in}{0.904140in}}%
\pgfpathlineto{\pgfqpoint{1.678890in}{0.898962in}}%
\pgfpathlineto{\pgfqpoint{1.681317in}{0.893927in}}%
\pgfpathlineto{\pgfqpoint{1.670707in}{0.888456in}}%
\pgfpathlineto{\pgfqpoint{1.659752in}{0.883162in}}%
\pgfpathlineto{\pgfqpoint{1.648464in}{0.878052in}}%
\pgfpathlineto{\pgfqpoint{1.636853in}{0.873131in}}%
\pgfpathlineto{\pgfqpoint{1.634746in}{0.878335in}}%
\pgfpathlineto{\pgfqpoint{1.632642in}{0.883683in}}%
\pgfpathlineto{\pgfqpoint{1.630541in}{0.889171in}}%
\pgfpathlineto{\pgfqpoint{1.628441in}{0.894795in}}%
\pgfpathlineto{\pgfqpoint{1.639716in}{0.899555in}}%
\pgfpathlineto{\pgfqpoint{1.650680in}{0.904499in}}%
\pgfpathlineto{\pgfqpoint{1.661320in}{0.909621in}}%
\pgfpathlineto{\pgfqpoint{1.671625in}{0.914914in}}%
\pgfpathclose%
\pgfusepath{fill}%
\end{pgfscope}%
\begin{pgfscope}%
\pgfpathrectangle{\pgfqpoint{0.329460in}{0.284240in}}{\pgfqpoint{1.989680in}{1.989680in}}%
\pgfusepath{clip}%
\pgfsetbuttcap%
\pgfsetroundjoin%
\definecolor{currentfill}{rgb}{0.636902,0.856542,0.216620}%
\pgfsetfillcolor{currentfill}%
\pgfsetlinewidth{0.000000pt}%
\definecolor{currentstroke}{rgb}{0.000000,0.000000,0.000000}%
\pgfsetstrokecolor{currentstroke}%
\pgfsetdash{}{0pt}%
\pgfpathmoveto{\pgfqpoint{1.211793in}{1.642674in}}%
\pgfpathlineto{\pgfqpoint{1.208344in}{1.638285in}}%
\pgfpathlineto{\pgfqpoint{1.204897in}{1.633805in}}%
\pgfpathlineto{\pgfqpoint{1.201454in}{1.629235in}}%
\pgfpathlineto{\pgfqpoint{1.198012in}{1.624575in}}%
\pgfpathlineto{\pgfqpoint{1.198424in}{1.626866in}}%
\pgfpathlineto{\pgfqpoint{1.198988in}{1.629148in}}%
\pgfpathlineto{\pgfqpoint{1.199702in}{1.631420in}}%
\pgfpathlineto{\pgfqpoint{1.200567in}{1.633680in}}%
\pgfpathlineto{\pgfqpoint{1.203945in}{1.638128in}}%
\pgfpathlineto{\pgfqpoint{1.207327in}{1.642488in}}%
\pgfpathlineto{\pgfqpoint{1.210711in}{1.646757in}}%
\pgfpathlineto{\pgfqpoint{1.214098in}{1.650935in}}%
\pgfpathlineto{\pgfqpoint{1.213316in}{1.648885in}}%
\pgfpathlineto{\pgfqpoint{1.212670in}{1.646823in}}%
\pgfpathlineto{\pgfqpoint{1.212162in}{1.644752in}}%
\pgfpathlineto{\pgfqpoint{1.211793in}{1.642674in}}%
\pgfpathclose%
\pgfusepath{fill}%
\end{pgfscope}%
\begin{pgfscope}%
\pgfpathrectangle{\pgfqpoint{0.329460in}{0.284240in}}{\pgfqpoint{1.989680in}{1.989680in}}%
\pgfusepath{clip}%
\pgfsetbuttcap%
\pgfsetroundjoin%
\definecolor{currentfill}{rgb}{0.279566,0.067836,0.391917}%
\pgfsetfillcolor{currentfill}%
\pgfsetlinewidth{0.000000pt}%
\definecolor{currentstroke}{rgb}{0.000000,0.000000,0.000000}%
\pgfsetstrokecolor{currentstroke}%
\pgfsetdash{}{0pt}%
\pgfpathmoveto{\pgfqpoint{1.681317in}{0.893927in}}%
\pgfpathlineto{\pgfqpoint{1.683746in}{0.889039in}}%
\pgfpathlineto{\pgfqpoint{1.686178in}{0.884302in}}%
\pgfpathlineto{\pgfqpoint{1.688613in}{0.879721in}}%
\pgfpathlineto{\pgfqpoint{1.691051in}{0.875298in}}%
\pgfpathlineto{\pgfqpoint{1.680137in}{0.869650in}}%
\pgfpathlineto{\pgfqpoint{1.668866in}{0.864184in}}%
\pgfpathlineto{\pgfqpoint{1.657251in}{0.858908in}}%
\pgfpathlineto{\pgfqpoint{1.645303in}{0.853827in}}%
\pgfpathlineto{\pgfqpoint{1.643187in}{0.858418in}}%
\pgfpathlineto{\pgfqpoint{1.641073in}{0.863169in}}%
\pgfpathlineto{\pgfqpoint{1.638961in}{0.868074in}}%
\pgfpathlineto{\pgfqpoint{1.636853in}{0.873131in}}%
\pgfpathlineto{\pgfqpoint{1.648464in}{0.878052in}}%
\pgfpathlineto{\pgfqpoint{1.659752in}{0.883162in}}%
\pgfpathlineto{\pgfqpoint{1.670707in}{0.888456in}}%
\pgfpathlineto{\pgfqpoint{1.681317in}{0.893927in}}%
\pgfpathclose%
\pgfusepath{fill}%
\end{pgfscope}%
\begin{pgfscope}%
\pgfpathrectangle{\pgfqpoint{0.329460in}{0.284240in}}{\pgfqpoint{1.989680in}{1.989680in}}%
\pgfusepath{clip}%
\pgfsetbuttcap%
\pgfsetroundjoin%
\definecolor{currentfill}{rgb}{0.896320,0.893616,0.096335}%
\pgfsetfillcolor{currentfill}%
\pgfsetlinewidth{0.000000pt}%
\definecolor{currentstroke}{rgb}{0.000000,0.000000,0.000000}%
\pgfsetstrokecolor{currentstroke}%
\pgfsetdash{}{0pt}%
\pgfpathmoveto{\pgfqpoint{1.310087in}{1.733880in}}%
\pgfpathlineto{\pgfqpoint{1.308047in}{1.732790in}}%
\pgfpathlineto{\pgfqpoint{1.306009in}{1.731588in}}%
\pgfpathlineto{\pgfqpoint{1.303973in}{1.730275in}}%
\pgfpathlineto{\pgfqpoint{1.301938in}{1.728852in}}%
\pgfpathlineto{\pgfqpoint{1.304102in}{1.729560in}}%
\pgfpathlineto{\pgfqpoint{1.306312in}{1.730237in}}%
\pgfpathlineto{\pgfqpoint{1.308566in}{1.730880in}}%
\pgfpathlineto{\pgfqpoint{1.310862in}{1.731490in}}%
\pgfpathlineto{\pgfqpoint{1.312527in}{1.732804in}}%
\pgfpathlineto{\pgfqpoint{1.314194in}{1.734007in}}%
\pgfpathlineto{\pgfqpoint{1.315862in}{1.735099in}}%
\pgfpathlineto{\pgfqpoint{1.317532in}{1.736080in}}%
\pgfpathlineto{\pgfqpoint{1.315616in}{1.735571in}}%
\pgfpathlineto{\pgfqpoint{1.313736in}{1.735035in}}%
\pgfpathlineto{\pgfqpoint{1.311892in}{1.734471in}}%
\pgfpathlineto{\pgfqpoint{1.310087in}{1.733880in}}%
\pgfpathclose%
\pgfusepath{fill}%
\end{pgfscope}%
\begin{pgfscope}%
\pgfpathrectangle{\pgfqpoint{0.329460in}{0.284240in}}{\pgfqpoint{1.989680in}{1.989680in}}%
\pgfusepath{clip}%
\pgfsetbuttcap%
\pgfsetroundjoin%
\definecolor{currentfill}{rgb}{0.283072,0.130895,0.449241}%
\pgfsetfillcolor{currentfill}%
\pgfsetlinewidth{0.000000pt}%
\definecolor{currentstroke}{rgb}{0.000000,0.000000,0.000000}%
\pgfsetstrokecolor{currentstroke}%
\pgfsetdash{}{0pt}%
\pgfpathmoveto{\pgfqpoint{1.661969in}{0.938026in}}%
\pgfpathlineto{\pgfqpoint{1.664380in}{0.932062in}}%
\pgfpathlineto{\pgfqpoint{1.666793in}{0.926219in}}%
\pgfpathlineto{\pgfqpoint{1.669208in}{0.920502in}}%
\pgfpathlineto{\pgfqpoint{1.671625in}{0.914914in}}%
\pgfpathlineto{\pgfqpoint{1.661320in}{0.909621in}}%
\pgfpathlineto{\pgfqpoint{1.650680in}{0.904499in}}%
\pgfpathlineto{\pgfqpoint{1.639716in}{0.899555in}}%
\pgfpathlineto{\pgfqpoint{1.628441in}{0.894795in}}%
\pgfpathlineto{\pgfqpoint{1.626343in}{0.900552in}}%
\pgfpathlineto{\pgfqpoint{1.624247in}{0.906438in}}%
\pgfpathlineto{\pgfqpoint{1.622153in}{0.912450in}}%
\pgfpathlineto{\pgfqpoint{1.620061in}{0.918583in}}%
\pgfpathlineto{\pgfqpoint{1.631002in}{0.923183in}}%
\pgfpathlineto{\pgfqpoint{1.641641in}{0.927961in}}%
\pgfpathlineto{\pgfqpoint{1.651966in}{0.932910in}}%
\pgfpathlineto{\pgfqpoint{1.661969in}{0.938026in}}%
\pgfpathclose%
\pgfusepath{fill}%
\end{pgfscope}%
\begin{pgfscope}%
\pgfpathrectangle{\pgfqpoint{0.329460in}{0.284240in}}{\pgfqpoint{1.989680in}{1.989680in}}%
\pgfusepath{clip}%
\pgfsetbuttcap%
\pgfsetroundjoin%
\definecolor{currentfill}{rgb}{0.147607,0.511733,0.557049}%
\pgfsetfillcolor{currentfill}%
\pgfsetlinewidth{0.000000pt}%
\definecolor{currentstroke}{rgb}{0.000000,0.000000,0.000000}%
\pgfsetstrokecolor{currentstroke}%
\pgfsetdash{}{0pt}%
\pgfpathmoveto{\pgfqpoint{1.613101in}{1.269851in}}%
\pgfpathlineto{\pgfqpoint{1.616017in}{1.261992in}}%
\pgfpathlineto{\pgfqpoint{1.618931in}{1.254133in}}%
\pgfpathlineto{\pgfqpoint{1.621844in}{1.246279in}}%
\pgfpathlineto{\pgfqpoint{1.624756in}{1.238432in}}%
\pgfpathlineto{\pgfqpoint{1.619552in}{1.234106in}}%
\pgfpathlineto{\pgfqpoint{1.614070in}{1.229863in}}%
\pgfpathlineto{\pgfqpoint{1.608314in}{1.225706in}}%
\pgfpathlineto{\pgfqpoint{1.602291in}{1.221641in}}%
\pgfpathlineto{\pgfqpoint{1.599611in}{1.229688in}}%
\pgfpathlineto{\pgfqpoint{1.596930in}{1.237741in}}%
\pgfpathlineto{\pgfqpoint{1.594248in}{1.245799in}}%
\pgfpathlineto{\pgfqpoint{1.591565in}{1.253857in}}%
\pgfpathlineto{\pgfqpoint{1.597338in}{1.257729in}}%
\pgfpathlineto{\pgfqpoint{1.602855in}{1.261688in}}%
\pgfpathlineto{\pgfqpoint{1.608111in}{1.265730in}}%
\pgfpathlineto{\pgfqpoint{1.613101in}{1.269851in}}%
\pgfpathclose%
\pgfusepath{fill}%
\end{pgfscope}%
\begin{pgfscope}%
\pgfpathrectangle{\pgfqpoint{0.329460in}{0.284240in}}{\pgfqpoint{1.989680in}{1.989680in}}%
\pgfusepath{clip}%
\pgfsetbuttcap%
\pgfsetroundjoin%
\definecolor{currentfill}{rgb}{0.281477,0.755203,0.432552}%
\pgfsetfillcolor{currentfill}%
\pgfsetlinewidth{0.000000pt}%
\definecolor{currentstroke}{rgb}{0.000000,0.000000,0.000000}%
\pgfsetstrokecolor{currentstroke}%
\pgfsetdash{}{0pt}%
\pgfpathmoveto{\pgfqpoint{1.152434in}{1.501532in}}%
\pgfpathlineto{\pgfqpoint{1.149168in}{1.494944in}}%
\pgfpathlineto{\pgfqpoint{1.145905in}{1.488296in}}%
\pgfpathlineto{\pgfqpoint{1.142644in}{1.481590in}}%
\pgfpathlineto{\pgfqpoint{1.139385in}{1.474826in}}%
\pgfpathlineto{\pgfqpoint{1.137377in}{1.478094in}}%
\pgfpathlineto{\pgfqpoint{1.135584in}{1.481389in}}%
\pgfpathlineto{\pgfqpoint{1.134010in}{1.484708in}}%
\pgfpathlineto{\pgfqpoint{1.132655in}{1.488048in}}%
\pgfpathlineto{\pgfqpoint{1.136010in}{1.494593in}}%
\pgfpathlineto{\pgfqpoint{1.139368in}{1.501083in}}%
\pgfpathlineto{\pgfqpoint{1.142728in}{1.507514in}}%
\pgfpathlineto{\pgfqpoint{1.146091in}{1.513885in}}%
\pgfpathlineto{\pgfqpoint{1.147370in}{1.510765in}}%
\pgfpathlineto{\pgfqpoint{1.148854in}{1.507663in}}%
\pgfpathlineto{\pgfqpoint{1.150543in}{1.504585in}}%
\pgfpathlineto{\pgfqpoint{1.152434in}{1.501532in}}%
\pgfpathclose%
\pgfusepath{fill}%
\end{pgfscope}%
\begin{pgfscope}%
\pgfpathrectangle{\pgfqpoint{0.329460in}{0.284240in}}{\pgfqpoint{1.989680in}{1.989680in}}%
\pgfusepath{clip}%
\pgfsetbuttcap%
\pgfsetroundjoin%
\definecolor{currentfill}{rgb}{0.565498,0.842430,0.262877}%
\pgfsetfillcolor{currentfill}%
\pgfsetlinewidth{0.000000pt}%
\definecolor{currentstroke}{rgb}{0.000000,0.000000,0.000000}%
\pgfsetstrokecolor{currentstroke}%
\pgfsetdash{}{0pt}%
\pgfpathmoveto{\pgfqpoint{1.504004in}{1.626612in}}%
\pgfpathlineto{\pgfqpoint{1.507434in}{1.621912in}}%
\pgfpathlineto{\pgfqpoint{1.510861in}{1.617126in}}%
\pgfpathlineto{\pgfqpoint{1.514285in}{1.612256in}}%
\pgfpathlineto{\pgfqpoint{1.517706in}{1.607303in}}%
\pgfpathlineto{\pgfqpoint{1.518142in}{1.604798in}}%
\pgfpathlineto{\pgfqpoint{1.518412in}{1.602286in}}%
\pgfpathlineto{\pgfqpoint{1.518516in}{1.599770in}}%
\pgfpathlineto{\pgfqpoint{1.518453in}{1.597254in}}%
\pgfpathlineto{\pgfqpoint{1.515010in}{1.602422in}}%
\pgfpathlineto{\pgfqpoint{1.511565in}{1.607507in}}%
\pgfpathlineto{\pgfqpoint{1.508118in}{1.612507in}}%
\pgfpathlineto{\pgfqpoint{1.504668in}{1.617421in}}%
\pgfpathlineto{\pgfqpoint{1.504731in}{1.619722in}}%
\pgfpathlineto{\pgfqpoint{1.504642in}{1.622023in}}%
\pgfpathlineto{\pgfqpoint{1.504399in}{1.624320in}}%
\pgfpathlineto{\pgfqpoint{1.504004in}{1.626612in}}%
\pgfpathclose%
\pgfusepath{fill}%
\end{pgfscope}%
\begin{pgfscope}%
\pgfpathrectangle{\pgfqpoint{0.329460in}{0.284240in}}{\pgfqpoint{1.989680in}{1.989680in}}%
\pgfusepath{clip}%
\pgfsetbuttcap%
\pgfsetroundjoin%
\definecolor{currentfill}{rgb}{0.412913,0.803041,0.357269}%
\pgfsetfillcolor{currentfill}%
\pgfsetlinewidth{0.000000pt}%
\definecolor{currentstroke}{rgb}{0.000000,0.000000,0.000000}%
\pgfsetstrokecolor{currentstroke}%
\pgfsetdash{}{0pt}%
\pgfpathmoveto{\pgfqpoint{1.173081in}{1.562480in}}%
\pgfpathlineto{\pgfqpoint{1.169699in}{1.556654in}}%
\pgfpathlineto{\pgfqpoint{1.166319in}{1.550753in}}%
\pgfpathlineto{\pgfqpoint{1.162941in}{1.544779in}}%
\pgfpathlineto{\pgfqpoint{1.159566in}{1.538735in}}%
\pgfpathlineto{\pgfqpoint{1.158557in}{1.541654in}}%
\pgfpathlineto{\pgfqpoint{1.157742in}{1.544585in}}%
\pgfpathlineto{\pgfqpoint{1.157121in}{1.547526in}}%
\pgfpathlineto{\pgfqpoint{1.156693in}{1.550473in}}%
\pgfpathlineto{\pgfqpoint{1.160112in}{1.556300in}}%
\pgfpathlineto{\pgfqpoint{1.163534in}{1.562055in}}%
\pgfpathlineto{\pgfqpoint{1.166958in}{1.567739in}}%
\pgfpathlineto{\pgfqpoint{1.170385in}{1.573349in}}%
\pgfpathlineto{\pgfqpoint{1.170789in}{1.570620in}}%
\pgfpathlineto{\pgfqpoint{1.171373in}{1.567897in}}%
\pgfpathlineto{\pgfqpoint{1.172137in}{1.565183in}}%
\pgfpathlineto{\pgfqpoint{1.173081in}{1.562480in}}%
\pgfpathclose%
\pgfusepath{fill}%
\end{pgfscope}%
\begin{pgfscope}%
\pgfpathrectangle{\pgfqpoint{0.329460in}{0.284240in}}{\pgfqpoint{1.989680in}{1.989680in}}%
\pgfusepath{clip}%
\pgfsetbuttcap%
\pgfsetroundjoin%
\definecolor{currentfill}{rgb}{0.855810,0.888601,0.097452}%
\pgfsetfillcolor{currentfill}%
\pgfsetlinewidth{0.000000pt}%
\definecolor{currentstroke}{rgb}{0.000000,0.000000,0.000000}%
\pgfsetstrokecolor{currentstroke}%
\pgfsetdash{}{0pt}%
\pgfpathmoveto{\pgfqpoint{1.414313in}{1.722946in}}%
\pgfpathlineto{\pgfqpoint{1.416921in}{1.721168in}}%
\pgfpathlineto{\pgfqpoint{1.419527in}{1.719281in}}%
\pgfpathlineto{\pgfqpoint{1.422131in}{1.717286in}}%
\pgfpathlineto{\pgfqpoint{1.424733in}{1.715184in}}%
\pgfpathlineto{\pgfqpoint{1.426733in}{1.714085in}}%
\pgfpathlineto{\pgfqpoint{1.428659in}{1.712956in}}%
\pgfpathlineto{\pgfqpoint{1.430508in}{1.711799in}}%
\pgfpathlineto{\pgfqpoint{1.432280in}{1.710615in}}%
\pgfpathlineto{\pgfqpoint{1.429409in}{1.712880in}}%
\pgfpathlineto{\pgfqpoint{1.426536in}{1.715039in}}%
\pgfpathlineto{\pgfqpoint{1.423660in}{1.717089in}}%
\pgfpathlineto{\pgfqpoint{1.420783in}{1.719030in}}%
\pgfpathlineto{\pgfqpoint{1.419264in}{1.720045in}}%
\pgfpathlineto{\pgfqpoint{1.417679in}{1.721036in}}%
\pgfpathlineto{\pgfqpoint{1.416028in}{1.722004in}}%
\pgfpathlineto{\pgfqpoint{1.414313in}{1.722946in}}%
\pgfpathclose%
\pgfusepath{fill}%
\end{pgfscope}%
\begin{pgfscope}%
\pgfpathrectangle{\pgfqpoint{0.329460in}{0.284240in}}{\pgfqpoint{1.989680in}{1.989680in}}%
\pgfusepath{clip}%
\pgfsetbuttcap%
\pgfsetroundjoin%
\definecolor{currentfill}{rgb}{0.274952,0.037752,0.364543}%
\pgfsetfillcolor{currentfill}%
\pgfsetlinewidth{0.000000pt}%
\definecolor{currentstroke}{rgb}{0.000000,0.000000,0.000000}%
\pgfsetstrokecolor{currentstroke}%
\pgfsetdash{}{0pt}%
\pgfpathmoveto{\pgfqpoint{1.691051in}{0.875298in}}%
\pgfpathlineto{\pgfqpoint{1.693493in}{0.871037in}}%
\pgfpathlineto{\pgfqpoint{1.695937in}{0.866943in}}%
\pgfpathlineto{\pgfqpoint{1.698385in}{0.863020in}}%
\pgfpathlineto{\pgfqpoint{1.700837in}{0.859270in}}%
\pgfpathlineto{\pgfqpoint{1.689617in}{0.853445in}}%
\pgfpathlineto{\pgfqpoint{1.678029in}{0.847808in}}%
\pgfpathlineto{\pgfqpoint{1.666086in}{0.842366in}}%
\pgfpathlineto{\pgfqpoint{1.653800in}{0.837125in}}%
\pgfpathlineto{\pgfqpoint{1.651671in}{0.841043in}}%
\pgfpathlineto{\pgfqpoint{1.649545in}{0.845135in}}%
\pgfpathlineto{\pgfqpoint{1.647423in}{0.849398in}}%
\pgfpathlineto{\pgfqpoint{1.645303in}{0.853827in}}%
\pgfpathlineto{\pgfqpoint{1.657251in}{0.858908in}}%
\pgfpathlineto{\pgfqpoint{1.668866in}{0.864184in}}%
\pgfpathlineto{\pgfqpoint{1.680137in}{0.869650in}}%
\pgfpathlineto{\pgfqpoint{1.691051in}{0.875298in}}%
\pgfpathclose%
\pgfusepath{fill}%
\end{pgfscope}%
\begin{pgfscope}%
\pgfpathrectangle{\pgfqpoint{0.329460in}{0.284240in}}{\pgfqpoint{1.989680in}{1.989680in}}%
\pgfusepath{clip}%
\pgfsetbuttcap%
\pgfsetroundjoin%
\definecolor{currentfill}{rgb}{0.280255,0.165693,0.476498}%
\pgfsetfillcolor{currentfill}%
\pgfsetlinewidth{0.000000pt}%
\definecolor{currentstroke}{rgb}{0.000000,0.000000,0.000000}%
\pgfsetstrokecolor{currentstroke}%
\pgfsetdash{}{0pt}%
\pgfpathmoveto{\pgfqpoint{1.652340in}{0.963036in}}%
\pgfpathlineto{\pgfqpoint{1.654745in}{0.956618in}}%
\pgfpathlineto{\pgfqpoint{1.657151in}{0.950308in}}%
\pgfpathlineto{\pgfqpoint{1.659559in}{0.944109in}}%
\pgfpathlineto{\pgfqpoint{1.661969in}{0.938026in}}%
\pgfpathlineto{\pgfqpoint{1.651966in}{0.932910in}}%
\pgfpathlineto{\pgfqpoint{1.641641in}{0.927961in}}%
\pgfpathlineto{\pgfqpoint{1.631002in}{0.923183in}}%
\pgfpathlineto{\pgfqpoint{1.620061in}{0.918583in}}%
\pgfpathlineto{\pgfqpoint{1.617970in}{0.924835in}}%
\pgfpathlineto{\pgfqpoint{1.615881in}{0.931203in}}%
\pgfpathlineto{\pgfqpoint{1.613793in}{0.937682in}}%
\pgfpathlineto{\pgfqpoint{1.611707in}{0.944269in}}%
\pgfpathlineto{\pgfqpoint{1.622314in}{0.948708in}}%
\pgfpathlineto{\pgfqpoint{1.632628in}{0.953320in}}%
\pgfpathlineto{\pgfqpoint{1.642640in}{0.958097in}}%
\pgfpathlineto{\pgfqpoint{1.652340in}{0.963036in}}%
\pgfpathclose%
\pgfusepath{fill}%
\end{pgfscope}%
\begin{pgfscope}%
\pgfpathrectangle{\pgfqpoint{0.329460in}{0.284240in}}{\pgfqpoint{1.989680in}{1.989680in}}%
\pgfusepath{clip}%
\pgfsetbuttcap%
\pgfsetroundjoin%
\definecolor{currentfill}{rgb}{0.260571,0.246922,0.522828}%
\pgfsetfillcolor{currentfill}%
\pgfsetlinewidth{0.000000pt}%
\definecolor{currentstroke}{rgb}{0.000000,0.000000,0.000000}%
\pgfsetstrokecolor{currentstroke}%
\pgfsetdash{}{0pt}%
\pgfpathmoveto{\pgfqpoint{1.934573in}{1.007756in}}%
\pgfpathlineto{\pgfqpoint{1.938086in}{1.015623in}}%
\pgfpathlineto{\pgfqpoint{1.941615in}{1.023870in}}%
\pgfpathlineto{\pgfqpoint{1.945161in}{1.032505in}}%
\pgfpathlineto{\pgfqpoint{1.948723in}{1.041533in}}%
\pgfpathlineto{\pgfqpoint{1.940889in}{1.031755in}}%
\pgfpathlineto{\pgfqpoint{1.932428in}{1.022096in}}%
\pgfpathlineto{\pgfqpoint{1.923348in}{1.012567in}}%
\pgfpathlineto{\pgfqpoint{1.913656in}{1.003180in}}%
\pgfpathlineto{\pgfqpoint{1.910283in}{0.994329in}}%
\pgfpathlineto{\pgfqpoint{1.906926in}{0.985873in}}%
\pgfpathlineto{\pgfqpoint{1.903585in}{0.977806in}}%
\pgfpathlineto{\pgfqpoint{1.900259in}{0.970121in}}%
\pgfpathlineto{\pgfqpoint{1.909740in}{0.979332in}}%
\pgfpathlineto{\pgfqpoint{1.918624in}{0.988681in}}%
\pgfpathlineto{\pgfqpoint{1.926904in}{0.998160in}}%
\pgfpathlineto{\pgfqpoint{1.934573in}{1.007756in}}%
\pgfpathclose%
\pgfusepath{fill}%
\end{pgfscope}%
\begin{pgfscope}%
\pgfpathrectangle{\pgfqpoint{0.329460in}{0.284240in}}{\pgfqpoint{1.989680in}{1.989680in}}%
\pgfusepath{clip}%
\pgfsetbuttcap%
\pgfsetroundjoin%
\definecolor{currentfill}{rgb}{0.120081,0.622161,0.534946}%
\pgfsetfillcolor{currentfill}%
\pgfsetlinewidth{0.000000pt}%
\definecolor{currentstroke}{rgb}{0.000000,0.000000,0.000000}%
\pgfsetstrokecolor{currentstroke}%
\pgfsetdash{}{0pt}%
\pgfpathmoveto{\pgfqpoint{1.592984in}{1.378050in}}%
\pgfpathlineto{\pgfqpoint{1.596104in}{1.370571in}}%
\pgfpathlineto{\pgfqpoint{1.599223in}{1.363065in}}%
\pgfpathlineto{\pgfqpoint{1.602340in}{1.355535in}}%
\pgfpathlineto{\pgfqpoint{1.605455in}{1.347981in}}%
\pgfpathlineto{\pgfqpoint{1.601904in}{1.344021in}}%
\pgfpathlineto{\pgfqpoint{1.598095in}{1.340115in}}%
\pgfpathlineto{\pgfqpoint{1.594033in}{1.336269in}}%
\pgfpathlineto{\pgfqpoint{1.589721in}{1.332487in}}%
\pgfpathlineto{\pgfqpoint{1.586792in}{1.340249in}}%
\pgfpathlineto{\pgfqpoint{1.583860in}{1.347988in}}%
\pgfpathlineto{\pgfqpoint{1.580928in}{1.355702in}}%
\pgfpathlineto{\pgfqpoint{1.577993in}{1.363388in}}%
\pgfpathlineto{\pgfqpoint{1.582100in}{1.366967in}}%
\pgfpathlineto{\pgfqpoint{1.585970in}{1.370606in}}%
\pgfpathlineto{\pgfqpoint{1.589599in}{1.374302in}}%
\pgfpathlineto{\pgfqpoint{1.592984in}{1.378050in}}%
\pgfpathclose%
\pgfusepath{fill}%
\end{pgfscope}%
\begin{pgfscope}%
\pgfpathrectangle{\pgfqpoint{0.329460in}{0.284240in}}{\pgfqpoint{1.989680in}{1.989680in}}%
\pgfusepath{clip}%
\pgfsetbuttcap%
\pgfsetroundjoin%
\definecolor{currentfill}{rgb}{0.212395,0.359683,0.551710}%
\pgfsetfillcolor{currentfill}%
\pgfsetlinewidth{0.000000pt}%
\definecolor{currentstroke}{rgb}{0.000000,0.000000,0.000000}%
\pgfsetstrokecolor{currentstroke}%
\pgfsetdash{}{0pt}%
\pgfpathmoveto{\pgfqpoint{1.105236in}{1.105467in}}%
\pgfpathlineto{\pgfqpoint{1.102908in}{1.097526in}}%
\pgfpathlineto{\pgfqpoint{1.100581in}{1.089629in}}%
\pgfpathlineto{\pgfqpoint{1.098254in}{1.081781in}}%
\pgfpathlineto{\pgfqpoint{1.095927in}{1.073983in}}%
\pgfpathlineto{\pgfqpoint{1.087469in}{1.078231in}}%
\pgfpathlineto{\pgfqpoint{1.079291in}{1.082613in}}%
\pgfpathlineto{\pgfqpoint{1.071401in}{1.087123in}}%
\pgfpathlineto{\pgfqpoint{1.063806in}{1.091757in}}%
\pgfpathlineto{\pgfqpoint{1.066419in}{1.099372in}}%
\pgfpathlineto{\pgfqpoint{1.069031in}{1.107038in}}%
\pgfpathlineto{\pgfqpoint{1.071644in}{1.114752in}}%
\pgfpathlineto{\pgfqpoint{1.074258in}{1.122511in}}%
\pgfpathlineto{\pgfqpoint{1.081583in}{1.118068in}}%
\pgfpathlineto{\pgfqpoint{1.089193in}{1.113743in}}%
\pgfpathlineto{\pgfqpoint{1.097080in}{1.109541in}}%
\pgfpathlineto{\pgfqpoint{1.105236in}{1.105467in}}%
\pgfpathclose%
\pgfusepath{fill}%
\end{pgfscope}%
\begin{pgfscope}%
\pgfpathrectangle{\pgfqpoint{0.329460in}{0.284240in}}{\pgfqpoint{1.989680in}{1.989680in}}%
\pgfusepath{clip}%
\pgfsetbuttcap%
\pgfsetroundjoin%
\definecolor{currentfill}{rgb}{0.166383,0.690856,0.496502}%
\pgfsetfillcolor{currentfill}%
\pgfsetlinewidth{0.000000pt}%
\definecolor{currentstroke}{rgb}{0.000000,0.000000,0.000000}%
\pgfsetstrokecolor{currentstroke}%
\pgfsetdash{}{0pt}%
\pgfpathmoveto{\pgfqpoint{1.137128in}{1.433674in}}%
\pgfpathlineto{\pgfqpoint{1.134029in}{1.426446in}}%
\pgfpathlineto{\pgfqpoint{1.130931in}{1.419173in}}%
\pgfpathlineto{\pgfqpoint{1.127836in}{1.411858in}}%
\pgfpathlineto{\pgfqpoint{1.124742in}{1.404502in}}%
\pgfpathlineto{\pgfqpoint{1.121551in}{1.408044in}}%
\pgfpathlineto{\pgfqpoint{1.118595in}{1.411632in}}%
\pgfpathlineto{\pgfqpoint{1.115875in}{1.415261in}}%
\pgfpathlineto{\pgfqpoint{1.113395in}{1.418929in}}%
\pgfpathlineto{\pgfqpoint{1.116636in}{1.426072in}}%
\pgfpathlineto{\pgfqpoint{1.119879in}{1.433174in}}%
\pgfpathlineto{\pgfqpoint{1.123124in}{1.440234in}}%
\pgfpathlineto{\pgfqpoint{1.126372in}{1.447249in}}%
\pgfpathlineto{\pgfqpoint{1.128724in}{1.443798in}}%
\pgfpathlineto{\pgfqpoint{1.131302in}{1.440382in}}%
\pgfpathlineto{\pgfqpoint{1.134105in}{1.437006in}}%
\pgfpathlineto{\pgfqpoint{1.137128in}{1.433674in}}%
\pgfpathclose%
\pgfusepath{fill}%
\end{pgfscope}%
\begin{pgfscope}%
\pgfpathrectangle{\pgfqpoint{0.329460in}{0.284240in}}{\pgfqpoint{1.989680in}{1.989680in}}%
\pgfusepath{clip}%
\pgfsetbuttcap%
\pgfsetroundjoin%
\definecolor{currentfill}{rgb}{0.896320,0.893616,0.096335}%
\pgfsetfillcolor{currentfill}%
\pgfsetlinewidth{0.000000pt}%
\definecolor{currentstroke}{rgb}{0.000000,0.000000,0.000000}%
\pgfsetstrokecolor{currentstroke}%
\pgfsetdash{}{0pt}%
\pgfpathmoveto{\pgfqpoint{1.390686in}{1.734406in}}%
\pgfpathlineto{\pgfqpoint{1.392646in}{1.733343in}}%
\pgfpathlineto{\pgfqpoint{1.394604in}{1.732167in}}%
\pgfpathlineto{\pgfqpoint{1.396561in}{1.730880in}}%
\pgfpathlineto{\pgfqpoint{1.398516in}{1.729483in}}%
\pgfpathlineto{\pgfqpoint{1.400675in}{1.728771in}}%
\pgfpathlineto{\pgfqpoint{1.402785in}{1.728027in}}%
\pgfpathlineto{\pgfqpoint{1.404844in}{1.727253in}}%
\pgfpathlineto{\pgfqpoint{1.406851in}{1.726448in}}%
\pgfpathlineto{\pgfqpoint{1.404551in}{1.727971in}}%
\pgfpathlineto{\pgfqpoint{1.402248in}{1.729384in}}%
\pgfpathlineto{\pgfqpoint{1.399944in}{1.730686in}}%
\pgfpathlineto{\pgfqpoint{1.397638in}{1.731876in}}%
\pgfpathlineto{\pgfqpoint{1.395964in}{1.732547in}}%
\pgfpathlineto{\pgfqpoint{1.394246in}{1.733192in}}%
\pgfpathlineto{\pgfqpoint{1.392486in}{1.733812in}}%
\pgfpathlineto{\pgfqpoint{1.390686in}{1.734406in}}%
\pgfpathclose%
\pgfusepath{fill}%
\end{pgfscope}%
\begin{pgfscope}%
\pgfpathrectangle{\pgfqpoint{0.329460in}{0.284240in}}{\pgfqpoint{1.989680in}{1.989680in}}%
\pgfusepath{clip}%
\pgfsetbuttcap%
\pgfsetroundjoin%
\definecolor{currentfill}{rgb}{0.855810,0.888601,0.097452}%
\pgfsetfillcolor{currentfill}%
\pgfsetlinewidth{0.000000pt}%
\definecolor{currentstroke}{rgb}{0.000000,0.000000,0.000000}%
\pgfsetstrokecolor{currentstroke}%
\pgfsetdash{}{0pt}%
\pgfpathmoveto{\pgfqpoint{1.280300in}{1.718110in}}%
\pgfpathlineto{\pgfqpoint{1.277369in}{1.716130in}}%
\pgfpathlineto{\pgfqpoint{1.274440in}{1.714042in}}%
\pgfpathlineto{\pgfqpoint{1.271512in}{1.711845in}}%
\pgfpathlineto{\pgfqpoint{1.268588in}{1.709542in}}%
\pgfpathlineto{\pgfqpoint{1.270288in}{1.710748in}}%
\pgfpathlineto{\pgfqpoint{1.272069in}{1.711929in}}%
\pgfpathlineto{\pgfqpoint{1.273927in}{1.713083in}}%
\pgfpathlineto{\pgfqpoint{1.275861in}{1.714208in}}%
\pgfpathlineto{\pgfqpoint{1.278526in}{1.716345in}}%
\pgfpathlineto{\pgfqpoint{1.281193in}{1.718374in}}%
\pgfpathlineto{\pgfqpoint{1.283863in}{1.720296in}}%
\pgfpathlineto{\pgfqpoint{1.286535in}{1.722110in}}%
\pgfpathlineto{\pgfqpoint{1.284877in}{1.721145in}}%
\pgfpathlineto{\pgfqpoint{1.283284in}{1.720156in}}%
\pgfpathlineto{\pgfqpoint{1.281758in}{1.719144in}}%
\pgfpathlineto{\pgfqpoint{1.280300in}{1.718110in}}%
\pgfpathclose%
\pgfusepath{fill}%
\end{pgfscope}%
\begin{pgfscope}%
\pgfpathrectangle{\pgfqpoint{0.329460in}{0.284240in}}{\pgfqpoint{1.989680in}{1.989680in}}%
\pgfusepath{clip}%
\pgfsetbuttcap%
\pgfsetroundjoin%
\definecolor{currentfill}{rgb}{0.195860,0.395433,0.555276}%
\pgfsetfillcolor{currentfill}%
\pgfsetlinewidth{0.000000pt}%
\definecolor{currentstroke}{rgb}{0.000000,0.000000,0.000000}%
\pgfsetstrokecolor{currentstroke}%
\pgfsetdash{}{0pt}%
\pgfpathmoveto{\pgfqpoint{1.623698in}{1.157816in}}%
\pgfpathlineto{\pgfqpoint{1.626370in}{1.149947in}}%
\pgfpathlineto{\pgfqpoint{1.629042in}{1.142113in}}%
\pgfpathlineto{\pgfqpoint{1.631714in}{1.134315in}}%
\pgfpathlineto{\pgfqpoint{1.634385in}{1.126556in}}%
\pgfpathlineto{\pgfqpoint{1.627318in}{1.122012in}}%
\pgfpathlineto{\pgfqpoint{1.619960in}{1.117581in}}%
\pgfpathlineto{\pgfqpoint{1.612319in}{1.113270in}}%
\pgfpathlineto{\pgfqpoint{1.604402in}{1.109082in}}%
\pgfpathlineto{\pgfqpoint{1.602007in}{1.117027in}}%
\pgfpathlineto{\pgfqpoint{1.599612in}{1.125011in}}%
\pgfpathlineto{\pgfqpoint{1.597217in}{1.133031in}}%
\pgfpathlineto{\pgfqpoint{1.594821in}{1.141085in}}%
\pgfpathlineto{\pgfqpoint{1.602445in}{1.145094in}}%
\pgfpathlineto{\pgfqpoint{1.609804in}{1.149222in}}%
\pgfpathlineto{\pgfqpoint{1.616890in}{1.153464in}}%
\pgfpathlineto{\pgfqpoint{1.623698in}{1.157816in}}%
\pgfpathclose%
\pgfusepath{fill}%
\end{pgfscope}%
\begin{pgfscope}%
\pgfpathrectangle{\pgfqpoint{0.329460in}{0.284240in}}{\pgfqpoint{1.989680in}{1.989680in}}%
\pgfusepath{clip}%
\pgfsetbuttcap%
\pgfsetroundjoin%
\definecolor{currentfill}{rgb}{0.282327,0.094955,0.417331}%
\pgfsetfillcolor{currentfill}%
\pgfsetlinewidth{0.000000pt}%
\definecolor{currentstroke}{rgb}{0.000000,0.000000,0.000000}%
\pgfsetstrokecolor{currentstroke}%
\pgfsetdash{}{0pt}%
\pgfpathmoveto{\pgfqpoint{0.891965in}{0.865477in}}%
\pgfpathlineto{\pgfqpoint{0.889095in}{0.868816in}}%
\pgfpathlineto{\pgfqpoint{0.886216in}{0.872469in}}%
\pgfpathlineto{\pgfqpoint{0.883326in}{0.876439in}}%
\pgfpathlineto{\pgfqpoint{0.880424in}{0.880733in}}%
\pgfpathlineto{\pgfqpoint{0.868707in}{0.888769in}}%
\pgfpathlineto{\pgfqpoint{0.857514in}{0.896988in}}%
\pgfpathlineto{\pgfqpoint{0.846854in}{0.905380in}}%
\pgfpathlineto{\pgfqpoint{0.836739in}{0.913935in}}%
\pgfpathlineto{\pgfqpoint{0.839893in}{0.909459in}}%
\pgfpathlineto{\pgfqpoint{0.843036in}{0.905305in}}%
\pgfpathlineto{\pgfqpoint{0.846168in}{0.901468in}}%
\pgfpathlineto{\pgfqpoint{0.849288in}{0.897943in}}%
\pgfpathlineto{\pgfqpoint{0.859173in}{0.889576in}}%
\pgfpathlineto{\pgfqpoint{0.869587in}{0.881370in}}%
\pgfpathlineto{\pgfqpoint{0.880521in}{0.873334in}}%
\pgfpathlineto{\pgfqpoint{0.891965in}{0.865477in}}%
\pgfpathclose%
\pgfusepath{fill}%
\end{pgfscope}%
\begin{pgfscope}%
\pgfpathrectangle{\pgfqpoint{0.329460in}{0.284240in}}{\pgfqpoint{1.989680in}{1.989680in}}%
\pgfusepath{clip}%
\pgfsetbuttcap%
\pgfsetroundjoin%
\definecolor{currentfill}{rgb}{0.565498,0.842430,0.262877}%
\pgfsetfillcolor{currentfill}%
\pgfsetlinewidth{0.000000pt}%
\definecolor{currentstroke}{rgb}{0.000000,0.000000,0.000000}%
\pgfsetstrokecolor{currentstroke}%
\pgfsetdash{}{0pt}%
\pgfpathmoveto{\pgfqpoint{1.197892in}{1.615376in}}%
\pgfpathlineto{\pgfqpoint{1.194445in}{1.610414in}}%
\pgfpathlineto{\pgfqpoint{1.191001in}{1.605366in}}%
\pgfpathlineto{\pgfqpoint{1.187559in}{1.600234in}}%
\pgfpathlineto{\pgfqpoint{1.184119in}{1.595017in}}%
\pgfpathlineto{\pgfqpoint{1.183907in}{1.597533in}}%
\pgfpathlineto{\pgfqpoint{1.183863in}{1.600050in}}%
\pgfpathlineto{\pgfqpoint{1.183985in}{1.602565in}}%
\pgfpathlineto{\pgfqpoint{1.184273in}{1.605076in}}%
\pgfpathlineto{\pgfqpoint{1.187704in}{1.610077in}}%
\pgfpathlineto{\pgfqpoint{1.191137in}{1.614995in}}%
\pgfpathlineto{\pgfqpoint{1.194574in}{1.619828in}}%
\pgfpathlineto{\pgfqpoint{1.198012in}{1.624575in}}%
\pgfpathlineto{\pgfqpoint{1.197753in}{1.622278in}}%
\pgfpathlineto{\pgfqpoint{1.197646in}{1.619978in}}%
\pgfpathlineto{\pgfqpoint{1.197693in}{1.617676in}}%
\pgfpathlineto{\pgfqpoint{1.197892in}{1.615376in}}%
\pgfpathclose%
\pgfusepath{fill}%
\end{pgfscope}%
\begin{pgfscope}%
\pgfpathrectangle{\pgfqpoint{0.329460in}{0.284240in}}{\pgfqpoint{1.989680in}{1.989680in}}%
\pgfusepath{clip}%
\pgfsetbuttcap%
\pgfsetroundjoin%
\definecolor{currentfill}{rgb}{0.896320,0.893616,0.096335}%
\pgfsetfillcolor{currentfill}%
\pgfsetlinewidth{0.000000pt}%
\definecolor{currentstroke}{rgb}{0.000000,0.000000,0.000000}%
\pgfsetstrokecolor{currentstroke}%
\pgfsetdash{}{0pt}%
\pgfpathmoveto{\pgfqpoint{1.303288in}{1.731259in}}%
\pgfpathlineto{\pgfqpoint{1.300910in}{1.730039in}}%
\pgfpathlineto{\pgfqpoint{1.298533in}{1.728706in}}%
\pgfpathlineto{\pgfqpoint{1.296159in}{1.727263in}}%
\pgfpathlineto{\pgfqpoint{1.293786in}{1.725709in}}%
\pgfpathlineto{\pgfqpoint{1.295744in}{1.726539in}}%
\pgfpathlineto{\pgfqpoint{1.297757in}{1.727340in}}%
\pgfpathlineto{\pgfqpoint{1.299822in}{1.728111in}}%
\pgfpathlineto{\pgfqpoint{1.301938in}{1.728852in}}%
\pgfpathlineto{\pgfqpoint{1.303973in}{1.730275in}}%
\pgfpathlineto{\pgfqpoint{1.306009in}{1.731588in}}%
\pgfpathlineto{\pgfqpoint{1.308047in}{1.732790in}}%
\pgfpathlineto{\pgfqpoint{1.310087in}{1.733880in}}%
\pgfpathlineto{\pgfqpoint{1.308322in}{1.733263in}}%
\pgfpathlineto{\pgfqpoint{1.306600in}{1.732620in}}%
\pgfpathlineto{\pgfqpoint{1.304921in}{1.731952in}}%
\pgfpathlineto{\pgfqpoint{1.303288in}{1.731259in}}%
\pgfpathclose%
\pgfusepath{fill}%
\end{pgfscope}%
\begin{pgfscope}%
\pgfpathrectangle{\pgfqpoint{0.329460in}{0.284240in}}{\pgfqpoint{1.989680in}{1.989680in}}%
\pgfusepath{clip}%
\pgfsetbuttcap%
\pgfsetroundjoin%
\definecolor{currentfill}{rgb}{0.274128,0.199721,0.498911}%
\pgfsetfillcolor{currentfill}%
\pgfsetlinewidth{0.000000pt}%
\definecolor{currentstroke}{rgb}{0.000000,0.000000,0.000000}%
\pgfsetstrokecolor{currentstroke}%
\pgfsetdash{}{0pt}%
\pgfpathmoveto{\pgfqpoint{1.642733in}{0.989723in}}%
\pgfpathlineto{\pgfqpoint{1.645133in}{0.982906in}}%
\pgfpathlineto{\pgfqpoint{1.647534in}{0.976183in}}%
\pgfpathlineto{\pgfqpoint{1.649936in}{0.969559in}}%
\pgfpathlineto{\pgfqpoint{1.652340in}{0.963036in}}%
\pgfpathlineto{\pgfqpoint{1.642640in}{0.958097in}}%
\pgfpathlineto{\pgfqpoint{1.632628in}{0.953320in}}%
\pgfpathlineto{\pgfqpoint{1.622314in}{0.948708in}}%
\pgfpathlineto{\pgfqpoint{1.611707in}{0.944269in}}%
\pgfpathlineto{\pgfqpoint{1.609622in}{0.950960in}}%
\pgfpathlineto{\pgfqpoint{1.607538in}{0.957753in}}%
\pgfpathlineto{\pgfqpoint{1.605455in}{0.964645in}}%
\pgfpathlineto{\pgfqpoint{1.603373in}{0.971630in}}%
\pgfpathlineto{\pgfqpoint{1.613646in}{0.975910in}}%
\pgfpathlineto{\pgfqpoint{1.623637in}{0.980356in}}%
\pgfpathlineto{\pgfqpoint{1.633336in}{0.984962in}}%
\pgfpathlineto{\pgfqpoint{1.642733in}{0.989723in}}%
\pgfpathclose%
\pgfusepath{fill}%
\end{pgfscope}%
\begin{pgfscope}%
\pgfpathrectangle{\pgfqpoint{0.329460in}{0.284240in}}{\pgfqpoint{1.989680in}{1.989680in}}%
\pgfusepath{clip}%
\pgfsetbuttcap%
\pgfsetroundjoin%
\definecolor{currentfill}{rgb}{0.271305,0.019942,0.347269}%
\pgfsetfillcolor{currentfill}%
\pgfsetlinewidth{0.000000pt}%
\definecolor{currentstroke}{rgb}{0.000000,0.000000,0.000000}%
\pgfsetstrokecolor{currentstroke}%
\pgfsetdash{}{0pt}%
\pgfpathmoveto{\pgfqpoint{1.700837in}{0.859270in}}%
\pgfpathlineto{\pgfqpoint{1.703293in}{0.855699in}}%
\pgfpathlineto{\pgfqpoint{1.705752in}{0.852309in}}%
\pgfpathlineto{\pgfqpoint{1.708216in}{0.849107in}}%
\pgfpathlineto{\pgfqpoint{1.710684in}{0.846094in}}%
\pgfpathlineto{\pgfqpoint{1.699156in}{0.840093in}}%
\pgfpathlineto{\pgfqpoint{1.687250in}{0.834285in}}%
\pgfpathlineto{\pgfqpoint{1.674977in}{0.828678in}}%
\pgfpathlineto{\pgfqpoint{1.662351in}{0.823278in}}%
\pgfpathlineto{\pgfqpoint{1.660208in}{0.826458in}}%
\pgfpathlineto{\pgfqpoint{1.658068in}{0.829829in}}%
\pgfpathlineto{\pgfqpoint{1.655932in}{0.833386in}}%
\pgfpathlineto{\pgfqpoint{1.653800in}{0.837125in}}%
\pgfpathlineto{\pgfqpoint{1.666086in}{0.842366in}}%
\pgfpathlineto{\pgfqpoint{1.678029in}{0.847808in}}%
\pgfpathlineto{\pgfqpoint{1.689617in}{0.853445in}}%
\pgfpathlineto{\pgfqpoint{1.700837in}{0.859270in}}%
\pgfpathclose%
\pgfusepath{fill}%
\end{pgfscope}%
\begin{pgfscope}%
\pgfpathrectangle{\pgfqpoint{0.329460in}{0.284240in}}{\pgfqpoint{1.989680in}{1.989680in}}%
\pgfusepath{clip}%
\pgfsetbuttcap%
\pgfsetroundjoin%
\definecolor{currentfill}{rgb}{0.762373,0.876424,0.137064}%
\pgfsetfillcolor{currentfill}%
\pgfsetlinewidth{0.000000pt}%
\definecolor{currentstroke}{rgb}{0.000000,0.000000,0.000000}%
\pgfsetstrokecolor{currentstroke}%
\pgfsetdash{}{0pt}%
\pgfpathmoveto{\pgfqpoint{1.456511in}{1.688731in}}%
\pgfpathlineto{\pgfqpoint{1.459771in}{1.685576in}}%
\pgfpathlineto{\pgfqpoint{1.463029in}{1.682320in}}%
\pgfpathlineto{\pgfqpoint{1.466284in}{1.678964in}}%
\pgfpathlineto{\pgfqpoint{1.469536in}{1.675510in}}%
\pgfpathlineto{\pgfqpoint{1.470825in}{1.673743in}}%
\pgfpathlineto{\pgfqpoint{1.471996in}{1.671957in}}%
\pgfpathlineto{\pgfqpoint{1.473048in}{1.670154in}}%
\pgfpathlineto{\pgfqpoint{1.473978in}{1.668336in}}%
\pgfpathlineto{\pgfqpoint{1.470600in}{1.671992in}}%
\pgfpathlineto{\pgfqpoint{1.467219in}{1.675549in}}%
\pgfpathlineto{\pgfqpoint{1.463835in}{1.679006in}}%
\pgfpathlineto{\pgfqpoint{1.460449in}{1.682361in}}%
\pgfpathlineto{\pgfqpoint{1.459625in}{1.683975in}}%
\pgfpathlineto{\pgfqpoint{1.458693in}{1.685576in}}%
\pgfpathlineto{\pgfqpoint{1.457654in}{1.687162in}}%
\pgfpathlineto{\pgfqpoint{1.456511in}{1.688731in}}%
\pgfpathclose%
\pgfusepath{fill}%
\end{pgfscope}%
\begin{pgfscope}%
\pgfpathrectangle{\pgfqpoint{0.329460in}{0.284240in}}{\pgfqpoint{1.989680in}{1.989680in}}%
\pgfusepath{clip}%
\pgfsetbuttcap%
\pgfsetroundjoin%
\definecolor{currentfill}{rgb}{0.282327,0.094955,0.417331}%
\pgfsetfillcolor{currentfill}%
\pgfsetlinewidth{0.000000pt}%
\definecolor{currentstroke}{rgb}{0.000000,0.000000,0.000000}%
\pgfsetstrokecolor{currentstroke}%
\pgfsetdash{}{0pt}%
\pgfpathmoveto{\pgfqpoint{1.084210in}{0.890722in}}%
\pgfpathlineto{\pgfqpoint{1.082187in}{0.885063in}}%
\pgfpathlineto{\pgfqpoint{1.080161in}{0.879541in}}%
\pgfpathlineto{\pgfqpoint{1.078134in}{0.874159in}}%
\pgfpathlineto{\pgfqpoint{1.076105in}{0.868921in}}%
\pgfpathlineto{\pgfqpoint{1.064217in}{0.873668in}}%
\pgfpathlineto{\pgfqpoint{1.052641in}{0.878610in}}%
\pgfpathlineto{\pgfqpoint{1.041389in}{0.883741in}}%
\pgfpathlineto{\pgfqpoint{1.030472in}{0.889055in}}%
\pgfpathlineto{\pgfqpoint{1.032831in}{0.894129in}}%
\pgfpathlineto{\pgfqpoint{1.035188in}{0.899348in}}%
\pgfpathlineto{\pgfqpoint{1.037542in}{0.904706in}}%
\pgfpathlineto{\pgfqpoint{1.039894in}{0.910201in}}%
\pgfpathlineto{\pgfqpoint{1.050497in}{0.905060in}}%
\pgfpathlineto{\pgfqpoint{1.061425in}{0.900096in}}%
\pgfpathlineto{\pgfqpoint{1.072666in}{0.895315in}}%
\pgfpathlineto{\pgfqpoint{1.084210in}{0.890722in}}%
\pgfpathclose%
\pgfusepath{fill}%
\end{pgfscope}%
\begin{pgfscope}%
\pgfpathrectangle{\pgfqpoint{0.329460in}{0.284240in}}{\pgfqpoint{1.989680in}{1.989680in}}%
\pgfusepath{clip}%
\pgfsetbuttcap%
\pgfsetroundjoin%
\definecolor{currentfill}{rgb}{0.268510,0.009605,0.335427}%
\pgfsetfillcolor{currentfill}%
\pgfsetlinewidth{0.000000pt}%
\definecolor{currentstroke}{rgb}{0.000000,0.000000,0.000000}%
\pgfsetstrokecolor{currentstroke}%
\pgfsetdash{}{0pt}%
\pgfpathmoveto{\pgfqpoint{0.972738in}{0.820525in}}%
\pgfpathlineto{\pgfqpoint{0.970269in}{0.820340in}}%
\pgfpathlineto{\pgfqpoint{0.967793in}{0.820406in}}%
\pgfpathlineto{\pgfqpoint{0.965310in}{0.820730in}}%
\pgfpathlineto{\pgfqpoint{0.962820in}{0.821316in}}%
\pgfpathlineto{\pgfqpoint{0.950089in}{0.828031in}}%
\pgfpathlineto{\pgfqpoint{0.937797in}{0.834952in}}%
\pgfpathlineto{\pgfqpoint{0.925956in}{0.842071in}}%
\pgfpathlineto{\pgfqpoint{0.914576in}{0.849378in}}%
\pgfpathlineto{\pgfqpoint{0.917364in}{0.848616in}}%
\pgfpathlineto{\pgfqpoint{0.920144in}{0.848115in}}%
\pgfpathlineto{\pgfqpoint{0.922916in}{0.847870in}}%
\pgfpathlineto{\pgfqpoint{0.925681in}{0.847877in}}%
\pgfpathlineto{\pgfqpoint{0.936783in}{0.840754in}}%
\pgfpathlineto{\pgfqpoint{0.948334in}{0.833815in}}%
\pgfpathlineto{\pgfqpoint{0.960323in}{0.827070in}}%
\pgfpathlineto{\pgfqpoint{0.972738in}{0.820525in}}%
\pgfpathclose%
\pgfusepath{fill}%
\end{pgfscope}%
\begin{pgfscope}%
\pgfpathrectangle{\pgfqpoint{0.329460in}{0.284240in}}{\pgfqpoint{1.989680in}{1.989680in}}%
\pgfusepath{clip}%
\pgfsetbuttcap%
\pgfsetroundjoin%
\definecolor{currentfill}{rgb}{0.344074,0.780029,0.397381}%
\pgfsetfillcolor{currentfill}%
\pgfsetlinewidth{0.000000pt}%
\definecolor{currentstroke}{rgb}{0.000000,0.000000,0.000000}%
\pgfsetstrokecolor{currentstroke}%
\pgfsetdash{}{0pt}%
\pgfpathmoveto{\pgfqpoint{1.543715in}{1.541329in}}%
\pgfpathlineto{\pgfqpoint{1.547102in}{1.535264in}}%
\pgfpathlineto{\pgfqpoint{1.550487in}{1.529132in}}%
\pgfpathlineto{\pgfqpoint{1.553869in}{1.522934in}}%
\pgfpathlineto{\pgfqpoint{1.557248in}{1.516673in}}%
\pgfpathlineto{\pgfqpoint{1.556153in}{1.513538in}}%
\pgfpathlineto{\pgfqpoint{1.554851in}{1.510419in}}%
\pgfpathlineto{\pgfqpoint{1.553344in}{1.507320in}}%
\pgfpathlineto{\pgfqpoint{1.551632in}{1.504244in}}%
\pgfpathlineto{\pgfqpoint{1.548338in}{1.510722in}}%
\pgfpathlineto{\pgfqpoint{1.545042in}{1.517136in}}%
\pgfpathlineto{\pgfqpoint{1.541743in}{1.523485in}}%
\pgfpathlineto{\pgfqpoint{1.538443in}{1.529766in}}%
\pgfpathlineto{\pgfqpoint{1.540048in}{1.532627in}}%
\pgfpathlineto{\pgfqpoint{1.541462in}{1.535510in}}%
\pgfpathlineto{\pgfqpoint{1.542685in}{1.538412in}}%
\pgfpathlineto{\pgfqpoint{1.543715in}{1.541329in}}%
\pgfpathclose%
\pgfusepath{fill}%
\end{pgfscope}%
\begin{pgfscope}%
\pgfpathrectangle{\pgfqpoint{0.329460in}{0.284240in}}{\pgfqpoint{1.989680in}{1.989680in}}%
\pgfusepath{clip}%
\pgfsetbuttcap%
\pgfsetroundjoin%
\definecolor{currentfill}{rgb}{0.147607,0.511733,0.557049}%
\pgfsetfillcolor{currentfill}%
\pgfsetlinewidth{0.000000pt}%
\definecolor{currentstroke}{rgb}{0.000000,0.000000,0.000000}%
\pgfsetstrokecolor{currentstroke}%
\pgfsetdash{}{0pt}%
\pgfpathmoveto{\pgfqpoint{1.116152in}{1.250493in}}%
\pgfpathlineto{\pgfqpoint{1.113528in}{1.242392in}}%
\pgfpathlineto{\pgfqpoint{1.110904in}{1.234292in}}%
\pgfpathlineto{\pgfqpoint{1.108281in}{1.226197in}}%
\pgfpathlineto{\pgfqpoint{1.105659in}{1.218107in}}%
\pgfpathlineto{\pgfqpoint{1.099402in}{1.222088in}}%
\pgfpathlineto{\pgfqpoint{1.093408in}{1.226163in}}%
\pgfpathlineto{\pgfqpoint{1.087682in}{1.230330in}}%
\pgfpathlineto{\pgfqpoint{1.082231in}{1.234583in}}%
\pgfpathlineto{\pgfqpoint{1.085095in}{1.242476in}}%
\pgfpathlineto{\pgfqpoint{1.087961in}{1.250375in}}%
\pgfpathlineto{\pgfqpoint{1.090828in}{1.258279in}}%
\pgfpathlineto{\pgfqpoint{1.093696in}{1.266185in}}%
\pgfpathlineto{\pgfqpoint{1.098923in}{1.262133in}}%
\pgfpathlineto{\pgfqpoint{1.104411in}{1.258165in}}%
\pgfpathlineto{\pgfqpoint{1.110156in}{1.254283in}}%
\pgfpathlineto{\pgfqpoint{1.116152in}{1.250493in}}%
\pgfpathclose%
\pgfusepath{fill}%
\end{pgfscope}%
\begin{pgfscope}%
\pgfpathrectangle{\pgfqpoint{0.329460in}{0.284240in}}{\pgfqpoint{1.989680in}{1.989680in}}%
\pgfusepath{clip}%
\pgfsetbuttcap%
\pgfsetroundjoin%
\definecolor{currentfill}{rgb}{0.935904,0.898570,0.108131}%
\pgfsetfillcolor{currentfill}%
\pgfsetlinewidth{0.000000pt}%
\definecolor{currentstroke}{rgb}{0.000000,0.000000,0.000000}%
\pgfsetstrokecolor{currentstroke}%
\pgfsetdash{}{0pt}%
\pgfpathmoveto{\pgfqpoint{1.344249in}{1.741890in}}%
\pgfpathlineto{\pgfqpoint{1.343818in}{1.741552in}}%
\pgfpathlineto{\pgfqpoint{1.343387in}{1.741100in}}%
\pgfpathlineto{\pgfqpoint{1.342957in}{1.740534in}}%
\pgfpathlineto{\pgfqpoint{1.342527in}{1.739854in}}%
\pgfpathlineto{\pgfqpoint{1.344716in}{1.739965in}}%
\pgfpathlineto{\pgfqpoint{1.346910in}{1.740044in}}%
\pgfpathlineto{\pgfqpoint{1.349109in}{1.740090in}}%
\pgfpathlineto{\pgfqpoint{1.351310in}{1.740105in}}%
\pgfpathlineto{\pgfqpoint{1.351304in}{1.740772in}}%
\pgfpathlineto{\pgfqpoint{1.351298in}{1.741326in}}%
\pgfpathlineto{\pgfqpoint{1.351292in}{1.741766in}}%
\pgfpathlineto{\pgfqpoint{1.351286in}{1.742091in}}%
\pgfpathlineto{\pgfqpoint{1.349522in}{1.742080in}}%
\pgfpathlineto{\pgfqpoint{1.347760in}{1.742042in}}%
\pgfpathlineto{\pgfqpoint{1.346002in}{1.741979in}}%
\pgfpathlineto{\pgfqpoint{1.344249in}{1.741890in}}%
\pgfpathclose%
\pgfusepath{fill}%
\end{pgfscope}%
\begin{pgfscope}%
\pgfpathrectangle{\pgfqpoint{0.329460in}{0.284240in}}{\pgfqpoint{1.989680in}{1.989680in}}%
\pgfusepath{clip}%
\pgfsetbuttcap%
\pgfsetroundjoin%
\definecolor{currentfill}{rgb}{0.814576,0.883393,0.110347}%
\pgfsetfillcolor{currentfill}%
\pgfsetlinewidth{0.000000pt}%
\definecolor{currentstroke}{rgb}{0.000000,0.000000,0.000000}%
\pgfsetstrokecolor{currentstroke}%
\pgfsetdash{}{0pt}%
\pgfpathmoveto{\pgfqpoint{1.438552in}{1.705635in}}%
\pgfpathlineto{\pgfqpoint{1.441645in}{1.703085in}}%
\pgfpathlineto{\pgfqpoint{1.444735in}{1.700431in}}%
\pgfpathlineto{\pgfqpoint{1.447824in}{1.697672in}}%
\pgfpathlineto{\pgfqpoint{1.450910in}{1.694811in}}%
\pgfpathlineto{\pgfqpoint{1.452461in}{1.693323in}}%
\pgfpathlineto{\pgfqpoint{1.453913in}{1.691813in}}%
\pgfpathlineto{\pgfqpoint{1.455263in}{1.690282in}}%
\pgfpathlineto{\pgfqpoint{1.456511in}{1.688731in}}%
\pgfpathlineto{\pgfqpoint{1.453248in}{1.691784in}}%
\pgfpathlineto{\pgfqpoint{1.449983in}{1.694733in}}%
\pgfpathlineto{\pgfqpoint{1.446716in}{1.697579in}}%
\pgfpathlineto{\pgfqpoint{1.443446in}{1.700319in}}%
\pgfpathlineto{\pgfqpoint{1.442357in}{1.701675in}}%
\pgfpathlineto{\pgfqpoint{1.441177in}{1.703014in}}%
\pgfpathlineto{\pgfqpoint{1.439908in}{1.704334in}}%
\pgfpathlineto{\pgfqpoint{1.438552in}{1.705635in}}%
\pgfpathclose%
\pgfusepath{fill}%
\end{pgfscope}%
\begin{pgfscope}%
\pgfpathrectangle{\pgfqpoint{0.329460in}{0.284240in}}{\pgfqpoint{1.989680in}{1.989680in}}%
\pgfusepath{clip}%
\pgfsetbuttcap%
\pgfsetroundjoin%
\definecolor{currentfill}{rgb}{0.935904,0.898570,0.108131}%
\pgfsetfillcolor{currentfill}%
\pgfsetlinewidth{0.000000pt}%
\definecolor{currentstroke}{rgb}{0.000000,0.000000,0.000000}%
\pgfsetstrokecolor{currentstroke}%
\pgfsetdash{}{0pt}%
\pgfpathmoveto{\pgfqpoint{1.351286in}{1.742091in}}%
\pgfpathlineto{\pgfqpoint{1.351292in}{1.741766in}}%
\pgfpathlineto{\pgfqpoint{1.351298in}{1.741326in}}%
\pgfpathlineto{\pgfqpoint{1.351304in}{1.740772in}}%
\pgfpathlineto{\pgfqpoint{1.351310in}{1.740105in}}%
\pgfpathlineto{\pgfqpoint{1.353511in}{1.740087in}}%
\pgfpathlineto{\pgfqpoint{1.355709in}{1.740037in}}%
\pgfpathlineto{\pgfqpoint{1.357903in}{1.739954in}}%
\pgfpathlineto{\pgfqpoint{1.360091in}{1.739839in}}%
\pgfpathlineto{\pgfqpoint{1.359649in}{1.740520in}}%
\pgfpathlineto{\pgfqpoint{1.359207in}{1.741087in}}%
\pgfpathlineto{\pgfqpoint{1.358764in}{1.741540in}}%
\pgfpathlineto{\pgfqpoint{1.358321in}{1.741878in}}%
\pgfpathlineto{\pgfqpoint{1.356568in}{1.741970in}}%
\pgfpathlineto{\pgfqpoint{1.354810in}{1.742036in}}%
\pgfpathlineto{\pgfqpoint{1.353049in}{1.742077in}}%
\pgfpathlineto{\pgfqpoint{1.351286in}{1.742091in}}%
\pgfpathclose%
\pgfusepath{fill}%
\end{pgfscope}%
\begin{pgfscope}%
\pgfpathrectangle{\pgfqpoint{0.329460in}{0.284240in}}{\pgfqpoint{1.989680in}{1.989680in}}%
\pgfusepath{clip}%
\pgfsetbuttcap%
\pgfsetroundjoin%
\definecolor{currentfill}{rgb}{0.283072,0.130895,0.449241}%
\pgfsetfillcolor{currentfill}%
\pgfsetlinewidth{0.000000pt}%
\definecolor{currentstroke}{rgb}{0.000000,0.000000,0.000000}%
\pgfsetstrokecolor{currentstroke}%
\pgfsetdash{}{0pt}%
\pgfpathmoveto{\pgfqpoint{1.092284in}{0.914648in}}%
\pgfpathlineto{\pgfqpoint{1.090268in}{0.908480in}}%
\pgfpathlineto{\pgfqpoint{1.088250in}{0.902434in}}%
\pgfpathlineto{\pgfqpoint{1.086231in}{0.896513in}}%
\pgfpathlineto{\pgfqpoint{1.084210in}{0.890722in}}%
\pgfpathlineto{\pgfqpoint{1.072666in}{0.895315in}}%
\pgfpathlineto{\pgfqpoint{1.061425in}{0.900096in}}%
\pgfpathlineto{\pgfqpoint{1.050497in}{0.905060in}}%
\pgfpathlineto{\pgfqpoint{1.039894in}{0.910201in}}%
\pgfpathlineto{\pgfqpoint{1.042244in}{0.915828in}}%
\pgfpathlineto{\pgfqpoint{1.044591in}{0.921584in}}%
\pgfpathlineto{\pgfqpoint{1.046937in}{0.927467in}}%
\pgfpathlineto{\pgfqpoint{1.049281in}{0.933471in}}%
\pgfpathlineto{\pgfqpoint{1.059572in}{0.928502in}}%
\pgfpathlineto{\pgfqpoint{1.070176in}{0.923705in}}%
\pgfpathlineto{\pgfqpoint{1.081084in}{0.919085in}}%
\pgfpathlineto{\pgfqpoint{1.092284in}{0.914648in}}%
\pgfpathclose%
\pgfusepath{fill}%
\end{pgfscope}%
\begin{pgfscope}%
\pgfpathrectangle{\pgfqpoint{0.329460in}{0.284240in}}{\pgfqpoint{1.989680in}{1.989680in}}%
\pgfusepath{clip}%
\pgfsetbuttcap%
\pgfsetroundjoin%
\definecolor{currentfill}{rgb}{0.279566,0.067836,0.391917}%
\pgfsetfillcolor{currentfill}%
\pgfsetlinewidth{0.000000pt}%
\definecolor{currentstroke}{rgb}{0.000000,0.000000,0.000000}%
\pgfsetstrokecolor{currentstroke}%
\pgfsetdash{}{0pt}%
\pgfpathmoveto{\pgfqpoint{1.076105in}{0.868921in}}%
\pgfpathlineto{\pgfqpoint{1.074073in}{0.863830in}}%
\pgfpathlineto{\pgfqpoint{1.072038in}{0.858890in}}%
\pgfpathlineto{\pgfqpoint{1.070001in}{0.854105in}}%
\pgfpathlineto{\pgfqpoint{1.067962in}{0.849479in}}%
\pgfpathlineto{\pgfqpoint{1.055728in}{0.854381in}}%
\pgfpathlineto{\pgfqpoint{1.043817in}{0.859484in}}%
\pgfpathlineto{\pgfqpoint{1.032239in}{0.864782in}}%
\pgfpathlineto{\pgfqpoint{1.021008in}{0.870268in}}%
\pgfpathlineto{\pgfqpoint{1.023378in}{0.874731in}}%
\pgfpathlineto{\pgfqpoint{1.025746in}{0.879352in}}%
\pgfpathlineto{\pgfqpoint{1.028110in}{0.884128in}}%
\pgfpathlineto{\pgfqpoint{1.030472in}{0.889055in}}%
\pgfpathlineto{\pgfqpoint{1.041389in}{0.883741in}}%
\pgfpathlineto{\pgfqpoint{1.052641in}{0.878610in}}%
\pgfpathlineto{\pgfqpoint{1.064217in}{0.873668in}}%
\pgfpathlineto{\pgfqpoint{1.076105in}{0.868921in}}%
\pgfpathclose%
\pgfusepath{fill}%
\end{pgfscope}%
\begin{pgfscope}%
\pgfpathrectangle{\pgfqpoint{0.329460in}{0.284240in}}{\pgfqpoint{1.989680in}{1.989680in}}%
\pgfusepath{clip}%
\pgfsetbuttcap%
\pgfsetroundjoin%
\definecolor{currentfill}{rgb}{0.935904,0.898570,0.108131}%
\pgfsetfillcolor{currentfill}%
\pgfsetlinewidth{0.000000pt}%
\definecolor{currentstroke}{rgb}{0.000000,0.000000,0.000000}%
\pgfsetstrokecolor{currentstroke}%
\pgfsetdash{}{0pt}%
\pgfpathmoveto{\pgfqpoint{1.337321in}{1.741278in}}%
\pgfpathlineto{\pgfqpoint{1.336459in}{1.740903in}}%
\pgfpathlineto{\pgfqpoint{1.335599in}{1.740412in}}%
\pgfpathlineto{\pgfqpoint{1.334739in}{1.739808in}}%
\pgfpathlineto{\pgfqpoint{1.333880in}{1.739090in}}%
\pgfpathlineto{\pgfqpoint{1.336022in}{1.739329in}}%
\pgfpathlineto{\pgfqpoint{1.338178in}{1.739535in}}%
\pgfpathlineto{\pgfqpoint{1.340347in}{1.739711in}}%
\pgfpathlineto{\pgfqpoint{1.342527in}{1.739854in}}%
\pgfpathlineto{\pgfqpoint{1.342957in}{1.740534in}}%
\pgfpathlineto{\pgfqpoint{1.343387in}{1.741100in}}%
\pgfpathlineto{\pgfqpoint{1.343818in}{1.741552in}}%
\pgfpathlineto{\pgfqpoint{1.344249in}{1.741890in}}%
\pgfpathlineto{\pgfqpoint{1.342502in}{1.741775in}}%
\pgfpathlineto{\pgfqpoint{1.340764in}{1.741635in}}%
\pgfpathlineto{\pgfqpoint{1.339036in}{1.741469in}}%
\pgfpathlineto{\pgfqpoint{1.337321in}{1.741278in}}%
\pgfpathclose%
\pgfusepath{fill}%
\end{pgfscope}%
\begin{pgfscope}%
\pgfpathrectangle{\pgfqpoint{0.329460in}{0.284240in}}{\pgfqpoint{1.989680in}{1.989680in}}%
\pgfusepath{clip}%
\pgfsetbuttcap%
\pgfsetroundjoin%
\definecolor{currentfill}{rgb}{0.935904,0.898570,0.108131}%
\pgfsetfillcolor{currentfill}%
\pgfsetlinewidth{0.000000pt}%
\definecolor{currentstroke}{rgb}{0.000000,0.000000,0.000000}%
\pgfsetstrokecolor{currentstroke}%
\pgfsetdash{}{0pt}%
\pgfpathmoveto{\pgfqpoint{1.358321in}{1.741878in}}%
\pgfpathlineto{\pgfqpoint{1.358764in}{1.741540in}}%
\pgfpathlineto{\pgfqpoint{1.359207in}{1.741087in}}%
\pgfpathlineto{\pgfqpoint{1.359649in}{1.740520in}}%
\pgfpathlineto{\pgfqpoint{1.360091in}{1.739839in}}%
\pgfpathlineto{\pgfqpoint{1.362269in}{1.739693in}}%
\pgfpathlineto{\pgfqpoint{1.364437in}{1.739514in}}%
\pgfpathlineto{\pgfqpoint{1.366592in}{1.739304in}}%
\pgfpathlineto{\pgfqpoint{1.368732in}{1.739062in}}%
\pgfpathlineto{\pgfqpoint{1.367861in}{1.739781in}}%
\pgfpathlineto{\pgfqpoint{1.366990in}{1.740387in}}%
\pgfpathlineto{\pgfqpoint{1.366117in}{1.740879in}}%
\pgfpathlineto{\pgfqpoint{1.365244in}{1.741256in}}%
\pgfpathlineto{\pgfqpoint{1.363530in}{1.741449in}}%
\pgfpathlineto{\pgfqpoint{1.361804in}{1.741618in}}%
\pgfpathlineto{\pgfqpoint{1.360067in}{1.741761in}}%
\pgfpathlineto{\pgfqpoint{1.358321in}{1.741878in}}%
\pgfpathclose%
\pgfusepath{fill}%
\end{pgfscope}%
\begin{pgfscope}%
\pgfpathrectangle{\pgfqpoint{0.329460in}{0.284240in}}{\pgfqpoint{1.989680in}{1.989680in}}%
\pgfusepath{clip}%
\pgfsetbuttcap%
\pgfsetroundjoin%
\definecolor{currentfill}{rgb}{0.699415,0.867117,0.175971}%
\pgfsetfillcolor{currentfill}%
\pgfsetlinewidth{0.000000pt}%
\definecolor{currentstroke}{rgb}{0.000000,0.000000,0.000000}%
\pgfsetstrokecolor{currentstroke}%
\pgfsetdash{}{0pt}%
\pgfpathmoveto{\pgfqpoint{1.473978in}{1.668336in}}%
\pgfpathlineto{\pgfqpoint{1.477355in}{1.664582in}}%
\pgfpathlineto{\pgfqpoint{1.480728in}{1.660732in}}%
\pgfpathlineto{\pgfqpoint{1.484099in}{1.656786in}}%
\pgfpathlineto{\pgfqpoint{1.487468in}{1.652747in}}%
\pgfpathlineto{\pgfqpoint{1.488371in}{1.650708in}}%
\pgfpathlineto{\pgfqpoint{1.489138in}{1.648657in}}%
\pgfpathlineto{\pgfqpoint{1.489768in}{1.646594in}}%
\pgfpathlineto{\pgfqpoint{1.490261in}{1.644522in}}%
\pgfpathlineto{\pgfqpoint{1.486818in}{1.648770in}}%
\pgfpathlineto{\pgfqpoint{1.483374in}{1.652925in}}%
\pgfpathlineto{\pgfqpoint{1.479927in}{1.656983in}}%
\pgfpathlineto{\pgfqpoint{1.476477in}{1.660945in}}%
\pgfpathlineto{\pgfqpoint{1.476038in}{1.662807in}}%
\pgfpathlineto{\pgfqpoint{1.475474in}{1.664660in}}%
\pgfpathlineto{\pgfqpoint{1.474788in}{1.666504in}}%
\pgfpathlineto{\pgfqpoint{1.473978in}{1.668336in}}%
\pgfpathclose%
\pgfusepath{fill}%
\end{pgfscope}%
\begin{pgfscope}%
\pgfpathrectangle{\pgfqpoint{0.329460in}{0.284240in}}{\pgfqpoint{1.989680in}{1.989680in}}%
\pgfusepath{clip}%
\pgfsetbuttcap%
\pgfsetroundjoin%
\definecolor{currentfill}{rgb}{0.487026,0.823929,0.312321}%
\pgfsetfillcolor{currentfill}%
\pgfsetlinewidth{0.000000pt}%
\definecolor{currentstroke}{rgb}{0.000000,0.000000,0.000000}%
\pgfsetstrokecolor{currentstroke}%
\pgfsetdash{}{0pt}%
\pgfpathmoveto{\pgfqpoint{1.518453in}{1.597254in}}%
\pgfpathlineto{\pgfqpoint{1.521892in}{1.592004in}}%
\pgfpathlineto{\pgfqpoint{1.525330in}{1.586673in}}%
\pgfpathlineto{\pgfqpoint{1.528765in}{1.581264in}}%
\pgfpathlineto{\pgfqpoint{1.532197in}{1.575778in}}%
\pgfpathlineto{\pgfqpoint{1.531954in}{1.573046in}}%
\pgfpathlineto{\pgfqpoint{1.531530in}{1.570317in}}%
\pgfpathlineto{\pgfqpoint{1.530926in}{1.567595in}}%
\pgfpathlineto{\pgfqpoint{1.530142in}{1.564882in}}%
\pgfpathlineto{\pgfqpoint{1.526743in}{1.570585in}}%
\pgfpathlineto{\pgfqpoint{1.523341in}{1.576210in}}%
\pgfpathlineto{\pgfqpoint{1.519937in}{1.581757in}}%
\pgfpathlineto{\pgfqpoint{1.516531in}{1.587223in}}%
\pgfpathlineto{\pgfqpoint{1.517261in}{1.589720in}}%
\pgfpathlineto{\pgfqpoint{1.517825in}{1.592226in}}%
\pgfpathlineto{\pgfqpoint{1.518222in}{1.594738in}}%
\pgfpathlineto{\pgfqpoint{1.518453in}{1.597254in}}%
\pgfpathclose%
\pgfusepath{fill}%
\end{pgfscope}%
\begin{pgfscope}%
\pgfpathrectangle{\pgfqpoint{0.329460in}{0.284240in}}{\pgfqpoint{1.989680in}{1.989680in}}%
\pgfusepath{clip}%
\pgfsetbuttcap%
\pgfsetroundjoin%
\definecolor{currentfill}{rgb}{0.120081,0.622161,0.534946}%
\pgfsetfillcolor{currentfill}%
\pgfsetlinewidth{0.000000pt}%
\definecolor{currentstroke}{rgb}{0.000000,0.000000,0.000000}%
\pgfsetstrokecolor{currentstroke}%
\pgfsetdash{}{0pt}%
\pgfpathmoveto{\pgfqpoint{1.128229in}{1.360261in}}%
\pgfpathlineto{\pgfqpoint{1.125342in}{1.352530in}}%
\pgfpathlineto{\pgfqpoint{1.122458in}{1.344771in}}%
\pgfpathlineto{\pgfqpoint{1.119575in}{1.336988in}}%
\pgfpathlineto{\pgfqpoint{1.116694in}{1.329181in}}%
\pgfpathlineto{\pgfqpoint{1.112163in}{1.332904in}}%
\pgfpathlineto{\pgfqpoint{1.107878in}{1.336693in}}%
\pgfpathlineto{\pgfqpoint{1.103844in}{1.340546in}}%
\pgfpathlineto{\pgfqpoint{1.100064in}{1.344458in}}%
\pgfpathlineto{\pgfqpoint{1.103142in}{1.352059in}}%
\pgfpathlineto{\pgfqpoint{1.106222in}{1.359637in}}%
\pgfpathlineto{\pgfqpoint{1.109304in}{1.367190in}}%
\pgfpathlineto{\pgfqpoint{1.112388in}{1.374716in}}%
\pgfpathlineto{\pgfqpoint{1.115990in}{1.371014in}}%
\pgfpathlineto{\pgfqpoint{1.119833in}{1.367369in}}%
\pgfpathlineto{\pgfqpoint{1.123914in}{1.363783in}}%
\pgfpathlineto{\pgfqpoint{1.128229in}{1.360261in}}%
\pgfpathclose%
\pgfusepath{fill}%
\end{pgfscope}%
\begin{pgfscope}%
\pgfpathrectangle{\pgfqpoint{0.329460in}{0.284240in}}{\pgfqpoint{1.989680in}{1.989680in}}%
\pgfusepath{clip}%
\pgfsetbuttcap%
\pgfsetroundjoin%
\definecolor{currentfill}{rgb}{0.220124,0.725509,0.466226}%
\pgfsetfillcolor{currentfill}%
\pgfsetlinewidth{0.000000pt}%
\definecolor{currentstroke}{rgb}{0.000000,0.000000,0.000000}%
\pgfsetstrokecolor{currentstroke}%
\pgfsetdash{}{0pt}%
\pgfpathmoveto{\pgfqpoint{1.564786in}{1.477730in}}%
\pgfpathlineto{\pgfqpoint{1.568069in}{1.470960in}}%
\pgfpathlineto{\pgfqpoint{1.571349in}{1.464138in}}%
\pgfpathlineto{\pgfqpoint{1.574627in}{1.457265in}}%
\pgfpathlineto{\pgfqpoint{1.577903in}{1.450345in}}%
\pgfpathlineto{\pgfqpoint{1.575753in}{1.446864in}}%
\pgfpathlineto{\pgfqpoint{1.573376in}{1.443416in}}%
\pgfpathlineto{\pgfqpoint{1.570772in}{1.440005in}}%
\pgfpathlineto{\pgfqpoint{1.567946in}{1.436634in}}%
\pgfpathlineto{\pgfqpoint{1.564807in}{1.443769in}}%
\pgfpathlineto{\pgfqpoint{1.561666in}{1.450855in}}%
\pgfpathlineto{\pgfqpoint{1.558523in}{1.457890in}}%
\pgfpathlineto{\pgfqpoint{1.555378in}{1.464872in}}%
\pgfpathlineto{\pgfqpoint{1.558048in}{1.468033in}}%
\pgfpathlineto{\pgfqpoint{1.560507in}{1.471232in}}%
\pgfpathlineto{\pgfqpoint{1.562754in}{1.474465in}}%
\pgfpathlineto{\pgfqpoint{1.564786in}{1.477730in}}%
\pgfpathclose%
\pgfusepath{fill}%
\end{pgfscope}%
\begin{pgfscope}%
\pgfpathrectangle{\pgfqpoint{0.329460in}{0.284240in}}{\pgfqpoint{1.989680in}{1.989680in}}%
\pgfusepath{clip}%
\pgfsetbuttcap%
\pgfsetroundjoin%
\definecolor{currentfill}{rgb}{0.280255,0.165693,0.476498}%
\pgfsetfillcolor{currentfill}%
\pgfsetlinewidth{0.000000pt}%
\definecolor{currentstroke}{rgb}{0.000000,0.000000,0.000000}%
\pgfsetstrokecolor{currentstroke}%
\pgfsetdash{}{0pt}%
\pgfpathmoveto{\pgfqpoint{1.100333in}{0.940470in}}%
\pgfpathlineto{\pgfqpoint{1.098323in}{0.933849in}}%
\pgfpathlineto{\pgfqpoint{1.096311in}{0.927336in}}%
\pgfpathlineto{\pgfqpoint{1.094298in}{0.920934in}}%
\pgfpathlineto{\pgfqpoint{1.092284in}{0.914648in}}%
\pgfpathlineto{\pgfqpoint{1.081084in}{0.919085in}}%
\pgfpathlineto{\pgfqpoint{1.070176in}{0.923705in}}%
\pgfpathlineto{\pgfqpoint{1.059572in}{0.928502in}}%
\pgfpathlineto{\pgfqpoint{1.049281in}{0.933471in}}%
\pgfpathlineto{\pgfqpoint{1.051624in}{0.939593in}}%
\pgfpathlineto{\pgfqpoint{1.053964in}{0.945831in}}%
\pgfpathlineto{\pgfqpoint{1.056304in}{0.952181in}}%
\pgfpathlineto{\pgfqpoint{1.058641in}{0.958638in}}%
\pgfpathlineto{\pgfqpoint{1.068619in}{0.953843in}}%
\pgfpathlineto{\pgfqpoint{1.078901in}{0.949212in}}%
\pgfpathlineto{\pgfqpoint{1.089476in}{0.944753in}}%
\pgfpathlineto{\pgfqpoint{1.100333in}{0.940470in}}%
\pgfpathclose%
\pgfusepath{fill}%
\end{pgfscope}%
\begin{pgfscope}%
\pgfpathrectangle{\pgfqpoint{0.329460in}{0.284240in}}{\pgfqpoint{1.989680in}{1.989680in}}%
\pgfusepath{clip}%
\pgfsetbuttcap%
\pgfsetroundjoin%
\definecolor{currentfill}{rgb}{0.263663,0.237631,0.518762}%
\pgfsetfillcolor{currentfill}%
\pgfsetlinewidth{0.000000pt}%
\definecolor{currentstroke}{rgb}{0.000000,0.000000,0.000000}%
\pgfsetstrokecolor{currentstroke}%
\pgfsetdash{}{0pt}%
\pgfpathmoveto{\pgfqpoint{1.633140in}{1.017875in}}%
\pgfpathlineto{\pgfqpoint{1.635537in}{1.010711in}}%
\pgfpathlineto{\pgfqpoint{1.637935in}{1.003629in}}%
\pgfpathlineto{\pgfqpoint{1.640333in}{0.996632in}}%
\pgfpathlineto{\pgfqpoint{1.642733in}{0.989723in}}%
\pgfpathlineto{\pgfqpoint{1.633336in}{0.984962in}}%
\pgfpathlineto{\pgfqpoint{1.623637in}{0.980356in}}%
\pgfpathlineto{\pgfqpoint{1.613646in}{0.975910in}}%
\pgfpathlineto{\pgfqpoint{1.603373in}{0.971630in}}%
\pgfpathlineto{\pgfqpoint{1.601291in}{0.978708in}}%
\pgfpathlineto{\pgfqpoint{1.599211in}{0.985873in}}%
\pgfpathlineto{\pgfqpoint{1.597132in}{0.993123in}}%
\pgfpathlineto{\pgfqpoint{1.595053in}{1.000455in}}%
\pgfpathlineto{\pgfqpoint{1.604993in}{1.004576in}}%
\pgfpathlineto{\pgfqpoint{1.614660in}{1.008856in}}%
\pgfpathlineto{\pgfqpoint{1.624046in}{1.013291in}}%
\pgfpathlineto{\pgfqpoint{1.633140in}{1.017875in}}%
\pgfpathclose%
\pgfusepath{fill}%
\end{pgfscope}%
\begin{pgfscope}%
\pgfpathrectangle{\pgfqpoint{0.329460in}{0.284240in}}{\pgfqpoint{1.989680in}{1.989680in}}%
\pgfusepath{clip}%
\pgfsetbuttcap%
\pgfsetroundjoin%
\definecolor{currentfill}{rgb}{0.762373,0.876424,0.137064}%
\pgfsetfillcolor{currentfill}%
\pgfsetlinewidth{0.000000pt}%
\definecolor{currentstroke}{rgb}{0.000000,0.000000,0.000000}%
\pgfsetstrokecolor{currentstroke}%
\pgfsetdash{}{0pt}%
\pgfpathmoveto{\pgfqpoint{1.241284in}{1.680917in}}%
\pgfpathlineto{\pgfqpoint{1.237877in}{1.677515in}}%
\pgfpathlineto{\pgfqpoint{1.234473in}{1.674012in}}%
\pgfpathlineto{\pgfqpoint{1.231071in}{1.670410in}}%
\pgfpathlineto{\pgfqpoint{1.227671in}{1.666708in}}%
\pgfpathlineto{\pgfqpoint{1.228494in}{1.668538in}}%
\pgfpathlineto{\pgfqpoint{1.229438in}{1.670355in}}%
\pgfpathlineto{\pgfqpoint{1.230503in}{1.672156in}}%
\pgfpathlineto{\pgfqpoint{1.231687in}{1.673940in}}%
\pgfpathlineto{\pgfqpoint{1.234972in}{1.677439in}}%
\pgfpathlineto{\pgfqpoint{1.238260in}{1.680838in}}%
\pgfpathlineto{\pgfqpoint{1.241550in}{1.684138in}}%
\pgfpathlineto{\pgfqpoint{1.244843in}{1.687337in}}%
\pgfpathlineto{\pgfqpoint{1.243793in}{1.685753in}}%
\pgfpathlineto{\pgfqpoint{1.242849in}{1.684154in}}%
\pgfpathlineto{\pgfqpoint{1.242012in}{1.682541in}}%
\pgfpathlineto{\pgfqpoint{1.241284in}{1.680917in}}%
\pgfpathclose%
\pgfusepath{fill}%
\end{pgfscope}%
\begin{pgfscope}%
\pgfpathrectangle{\pgfqpoint{0.329460in}{0.284240in}}{\pgfqpoint{1.989680in}{1.989680in}}%
\pgfusepath{clip}%
\pgfsetbuttcap%
\pgfsetroundjoin%
\definecolor{currentfill}{rgb}{0.896320,0.893616,0.096335}%
\pgfsetfillcolor{currentfill}%
\pgfsetlinewidth{0.000000pt}%
\definecolor{currentstroke}{rgb}{0.000000,0.000000,0.000000}%
\pgfsetstrokecolor{currentstroke}%
\pgfsetdash{}{0pt}%
\pgfpathmoveto{\pgfqpoint{1.397638in}{1.731876in}}%
\pgfpathlineto{\pgfqpoint{1.399944in}{1.730686in}}%
\pgfpathlineto{\pgfqpoint{1.402248in}{1.729384in}}%
\pgfpathlineto{\pgfqpoint{1.404551in}{1.727971in}}%
\pgfpathlineto{\pgfqpoint{1.406851in}{1.726448in}}%
\pgfpathlineto{\pgfqpoint{1.408803in}{1.725615in}}%
\pgfpathlineto{\pgfqpoint{1.410699in}{1.724752in}}%
\pgfpathlineto{\pgfqpoint{1.412536in}{1.723863in}}%
\pgfpathlineto{\pgfqpoint{1.414313in}{1.722946in}}%
\pgfpathlineto{\pgfqpoint{1.411703in}{1.724615in}}%
\pgfpathlineto{\pgfqpoint{1.409090in}{1.726173in}}%
\pgfpathlineto{\pgfqpoint{1.406476in}{1.727621in}}%
\pgfpathlineto{\pgfqpoint{1.403859in}{1.728956in}}%
\pgfpathlineto{\pgfqpoint{1.402378in}{1.729720in}}%
\pgfpathlineto{\pgfqpoint{1.400846in}{1.730462in}}%
\pgfpathlineto{\pgfqpoint{1.399266in}{1.731181in}}%
\pgfpathlineto{\pgfqpoint{1.397638in}{1.731876in}}%
\pgfpathclose%
\pgfusepath{fill}%
\end{pgfscope}%
\begin{pgfscope}%
\pgfpathrectangle{\pgfqpoint{0.329460in}{0.284240in}}{\pgfqpoint{1.989680in}{1.989680in}}%
\pgfusepath{clip}%
\pgfsetbuttcap%
\pgfsetroundjoin%
\definecolor{currentfill}{rgb}{0.935904,0.898570,0.108131}%
\pgfsetfillcolor{currentfill}%
\pgfsetlinewidth{0.000000pt}%
\definecolor{currentstroke}{rgb}{0.000000,0.000000,0.000000}%
\pgfsetstrokecolor{currentstroke}%
\pgfsetdash{}{0pt}%
\pgfpathmoveto{\pgfqpoint{1.365244in}{1.741256in}}%
\pgfpathlineto{\pgfqpoint{1.366117in}{1.740879in}}%
\pgfpathlineto{\pgfqpoint{1.366990in}{1.740387in}}%
\pgfpathlineto{\pgfqpoint{1.367861in}{1.739781in}}%
\pgfpathlineto{\pgfqpoint{1.368732in}{1.739062in}}%
\pgfpathlineto{\pgfqpoint{1.370854in}{1.738788in}}%
\pgfpathlineto{\pgfqpoint{1.372958in}{1.738484in}}%
\pgfpathlineto{\pgfqpoint{1.375039in}{1.738149in}}%
\pgfpathlineto{\pgfqpoint{1.373856in}{1.738914in}}%
\pgfpathlineto{\pgfqpoint{1.372671in}{1.739565in}}%
\pgfpathlineto{\pgfqpoint{1.371485in}{1.740102in}}%
\pgfpathlineto{\pgfqpoint{1.370298in}{1.740525in}}%
\pgfpathlineto{\pgfqpoint{1.368630in}{1.740793in}}%
\pgfpathlineto{\pgfqpoint{1.366945in}{1.741037in}}%
\pgfpathlineto{\pgfqpoint{1.365244in}{1.741256in}}%
\pgfpathclose%
\pgfusepath{fill}%
\end{pgfscope}%
\begin{pgfscope}%
\pgfpathrectangle{\pgfqpoint{0.329460in}{0.284240in}}{\pgfqpoint{1.989680in}{1.989680in}}%
\pgfusepath{clip}%
\pgfsetbuttcap%
\pgfsetroundjoin%
\definecolor{currentfill}{rgb}{0.814576,0.883393,0.110347}%
\pgfsetfillcolor{currentfill}%
\pgfsetlinewidth{0.000000pt}%
\definecolor{currentstroke}{rgb}{0.000000,0.000000,0.000000}%
\pgfsetstrokecolor{currentstroke}%
\pgfsetdash{}{0pt}%
\pgfpathmoveto{\pgfqpoint{1.258036in}{1.699100in}}%
\pgfpathlineto{\pgfqpoint{1.254735in}{1.696316in}}%
\pgfpathlineto{\pgfqpoint{1.251435in}{1.693427in}}%
\pgfpathlineto{\pgfqpoint{1.248138in}{1.690434in}}%
\pgfpathlineto{\pgfqpoint{1.244843in}{1.687337in}}%
\pgfpathlineto{\pgfqpoint{1.245998in}{1.688904in}}%
\pgfpathlineto{\pgfqpoint{1.247257in}{1.690453in}}%
\pgfpathlineto{\pgfqpoint{1.248619in}{1.691982in}}%
\pgfpathlineto{\pgfqpoint{1.250081in}{1.693490in}}%
\pgfpathlineto{\pgfqpoint{1.253211in}{1.696393in}}%
\pgfpathlineto{\pgfqpoint{1.256343in}{1.699193in}}%
\pgfpathlineto{\pgfqpoint{1.259477in}{1.701889in}}%
\pgfpathlineto{\pgfqpoint{1.262614in}{1.704479in}}%
\pgfpathlineto{\pgfqpoint{1.261335in}{1.703161in}}%
\pgfpathlineto{\pgfqpoint{1.260145in}{1.701824in}}%
\pgfpathlineto{\pgfqpoint{1.259045in}{1.700470in}}%
\pgfpathlineto{\pgfqpoint{1.258036in}{1.699100in}}%
\pgfpathclose%
\pgfusepath{fill}%
\end{pgfscope}%
\begin{pgfscope}%
\pgfpathrectangle{\pgfqpoint{0.329460in}{0.284240in}}{\pgfqpoint{1.989680in}{1.989680in}}%
\pgfusepath{clip}%
\pgfsetbuttcap%
\pgfsetroundjoin%
\definecolor{currentfill}{rgb}{0.935904,0.898570,0.108131}%
\pgfsetfillcolor{currentfill}%
\pgfsetlinewidth{0.000000pt}%
\definecolor{currentstroke}{rgb}{0.000000,0.000000,0.000000}%
\pgfsetstrokecolor{currentstroke}%
\pgfsetdash{}{0pt}%
\pgfpathmoveto{\pgfqpoint{1.330611in}{1.740266in}}%
\pgfpathlineto{\pgfqpoint{1.329332in}{1.739827in}}%
\pgfpathlineto{\pgfqpoint{1.328055in}{1.739274in}}%
\pgfpathlineto{\pgfqpoint{1.326779in}{1.738606in}}%
\pgfpathlineto{\pgfqpoint{1.325505in}{1.737826in}}%
\pgfpathlineto{\pgfqpoint{1.327566in}{1.738188in}}%
\pgfpathlineto{\pgfqpoint{1.329650in}{1.738519in}}%
\pgfpathlineto{\pgfqpoint{1.331756in}{1.738820in}}%
\pgfpathlineto{\pgfqpoint{1.333880in}{1.739090in}}%
\pgfpathlineto{\pgfqpoint{1.334739in}{1.739808in}}%
\pgfpathlineto{\pgfqpoint{1.335599in}{1.740412in}}%
\pgfpathlineto{\pgfqpoint{1.336459in}{1.740903in}}%
\pgfpathlineto{\pgfqpoint{1.337321in}{1.741278in}}%
\pgfpathlineto{\pgfqpoint{1.335618in}{1.741062in}}%
\pgfpathlineto{\pgfqpoint{1.333932in}{1.740821in}}%
\pgfpathlineto{\pgfqpoint{1.332262in}{1.740556in}}%
\pgfpathlineto{\pgfqpoint{1.330611in}{1.740266in}}%
\pgfpathclose%
\pgfusepath{fill}%
\end{pgfscope}%
\begin{pgfscope}%
\pgfpathrectangle{\pgfqpoint{0.329460in}{0.284240in}}{\pgfqpoint{1.989680in}{1.989680in}}%
\pgfusepath{clip}%
\pgfsetbuttcap%
\pgfsetroundjoin%
\definecolor{currentfill}{rgb}{0.274952,0.037752,0.364543}%
\pgfsetfillcolor{currentfill}%
\pgfsetlinewidth{0.000000pt}%
\definecolor{currentstroke}{rgb}{0.000000,0.000000,0.000000}%
\pgfsetstrokecolor{currentstroke}%
\pgfsetdash{}{0pt}%
\pgfpathmoveto{\pgfqpoint{1.067962in}{0.849479in}}%
\pgfpathlineto{\pgfqpoint{1.065919in}{0.845016in}}%
\pgfpathlineto{\pgfqpoint{1.063874in}{0.840719in}}%
\pgfpathlineto{\pgfqpoint{1.061826in}{0.836593in}}%
\pgfpathlineto{\pgfqpoint{1.059774in}{0.832641in}}%
\pgfpathlineto{\pgfqpoint{1.047193in}{0.837697in}}%
\pgfpathlineto{\pgfqpoint{1.034945in}{0.842961in}}%
\pgfpathlineto{\pgfqpoint{1.023041in}{0.848425in}}%
\pgfpathlineto{\pgfqpoint{1.011493in}{0.854083in}}%
\pgfpathlineto{\pgfqpoint{1.013877in}{0.857872in}}%
\pgfpathlineto{\pgfqpoint{1.016257in}{0.861835in}}%
\pgfpathlineto{\pgfqpoint{1.018634in}{0.865968in}}%
\pgfpathlineto{\pgfqpoint{1.021008in}{0.870268in}}%
\pgfpathlineto{\pgfqpoint{1.032239in}{0.864782in}}%
\pgfpathlineto{\pgfqpoint{1.043817in}{0.859484in}}%
\pgfpathlineto{\pgfqpoint{1.055728in}{0.854381in}}%
\pgfpathlineto{\pgfqpoint{1.067962in}{0.849479in}}%
\pgfpathclose%
\pgfusepath{fill}%
\end{pgfscope}%
\begin{pgfscope}%
\pgfpathrectangle{\pgfqpoint{0.329460in}{0.284240in}}{\pgfqpoint{1.989680in}{1.989680in}}%
\pgfusepath{clip}%
\pgfsetbuttcap%
\pgfsetroundjoin%
\definecolor{currentfill}{rgb}{0.272594,0.025563,0.353093}%
\pgfsetfillcolor{currentfill}%
\pgfsetlinewidth{0.000000pt}%
\definecolor{currentstroke}{rgb}{0.000000,0.000000,0.000000}%
\pgfsetstrokecolor{currentstroke}%
\pgfsetdash{}{0pt}%
\pgfpathmoveto{\pgfqpoint{1.797517in}{0.856026in}}%
\pgfpathlineto{\pgfqpoint{1.800372in}{0.857096in}}%
\pgfpathlineto{\pgfqpoint{1.803236in}{0.858437in}}%
\pgfpathlineto{\pgfqpoint{1.806108in}{0.860053in}}%
\pgfpathlineto{\pgfqpoint{1.808989in}{0.861951in}}%
\pgfpathlineto{\pgfqpoint{1.797761in}{0.854296in}}%
\pgfpathlineto{\pgfqpoint{1.786047in}{0.846828in}}%
\pgfpathlineto{\pgfqpoint{1.773861in}{0.839554in}}%
\pgfpathlineto{\pgfqpoint{1.761212in}{0.832483in}}%
\pgfpathlineto{\pgfqpoint{1.758620in}{0.830763in}}%
\pgfpathlineto{\pgfqpoint{1.756037in}{0.829326in}}%
\pgfpathlineto{\pgfqpoint{1.753461in}{0.828164in}}%
\pgfpathlineto{\pgfqpoint{1.750893in}{0.827275in}}%
\pgfpathlineto{\pgfqpoint{1.763234in}{0.834173in}}%
\pgfpathlineto{\pgfqpoint{1.775126in}{0.841270in}}%
\pgfpathlineto{\pgfqpoint{1.786558in}{0.848557in}}%
\pgfpathlineto{\pgfqpoint{1.797517in}{0.856026in}}%
\pgfpathclose%
\pgfusepath{fill}%
\end{pgfscope}%
\begin{pgfscope}%
\pgfpathrectangle{\pgfqpoint{0.329460in}{0.284240in}}{\pgfqpoint{1.989680in}{1.989680in}}%
\pgfusepath{clip}%
\pgfsetbuttcap%
\pgfsetroundjoin%
\definecolor{currentfill}{rgb}{0.133743,0.548535,0.553541}%
\pgfsetfillcolor{currentfill}%
\pgfsetlinewidth{0.000000pt}%
\definecolor{currentstroke}{rgb}{0.000000,0.000000,0.000000}%
\pgfsetstrokecolor{currentstroke}%
\pgfsetdash{}{0pt}%
\pgfpathmoveto{\pgfqpoint{1.601423in}{1.301256in}}%
\pgfpathlineto{\pgfqpoint{1.604345in}{1.293415in}}%
\pgfpathlineto{\pgfqpoint{1.607265in}{1.285566in}}%
\pgfpathlineto{\pgfqpoint{1.610183in}{1.277710in}}%
\pgfpathlineto{\pgfqpoint{1.613101in}{1.269851in}}%
\pgfpathlineto{\pgfqpoint{1.608111in}{1.265730in}}%
\pgfpathlineto{\pgfqpoint{1.602855in}{1.261688in}}%
\pgfpathlineto{\pgfqpoint{1.597338in}{1.257729in}}%
\pgfpathlineto{\pgfqpoint{1.591565in}{1.253857in}}%
\pgfpathlineto{\pgfqpoint{1.588881in}{1.261915in}}%
\pgfpathlineto{\pgfqpoint{1.586195in}{1.269968in}}%
\pgfpathlineto{\pgfqpoint{1.583509in}{1.278015in}}%
\pgfpathlineto{\pgfqpoint{1.580821in}{1.286054in}}%
\pgfpathlineto{\pgfqpoint{1.586343in}{1.289734in}}%
\pgfpathlineto{\pgfqpoint{1.591620in}{1.293497in}}%
\pgfpathlineto{\pgfqpoint{1.596649in}{1.297339in}}%
\pgfpathlineto{\pgfqpoint{1.601423in}{1.301256in}}%
\pgfpathclose%
\pgfusepath{fill}%
\end{pgfscope}%
\begin{pgfscope}%
\pgfpathrectangle{\pgfqpoint{0.329460in}{0.284240in}}{\pgfqpoint{1.989680in}{1.989680in}}%
\pgfusepath{clip}%
\pgfsetbuttcap%
\pgfsetroundjoin%
\definecolor{currentfill}{rgb}{0.855810,0.888601,0.097452}%
\pgfsetfillcolor{currentfill}%
\pgfsetlinewidth{0.000000pt}%
\definecolor{currentstroke}{rgb}{0.000000,0.000000,0.000000}%
\pgfsetstrokecolor{currentstroke}%
\pgfsetdash{}{0pt}%
\pgfpathmoveto{\pgfqpoint{1.420783in}{1.719030in}}%
\pgfpathlineto{\pgfqpoint{1.423660in}{1.717089in}}%
\pgfpathlineto{\pgfqpoint{1.426536in}{1.715039in}}%
\pgfpathlineto{\pgfqpoint{1.429409in}{1.712880in}}%
\pgfpathlineto{\pgfqpoint{1.432280in}{1.710615in}}%
\pgfpathlineto{\pgfqpoint{1.433971in}{1.709406in}}%
\pgfpathlineto{\pgfqpoint{1.435582in}{1.708172in}}%
\pgfpathlineto{\pgfqpoint{1.437109in}{1.706914in}}%
\pgfpathlineto{\pgfqpoint{1.438552in}{1.705635in}}%
\pgfpathlineto{\pgfqpoint{1.435456in}{1.708078in}}%
\pgfpathlineto{\pgfqpoint{1.432358in}{1.710414in}}%
\pgfpathlineto{\pgfqpoint{1.429258in}{1.712643in}}%
\pgfpathlineto{\pgfqpoint{1.426156in}{1.714763in}}%
\pgfpathlineto{\pgfqpoint{1.424921in}{1.715859in}}%
\pgfpathlineto{\pgfqpoint{1.423612in}{1.716937in}}%
\pgfpathlineto{\pgfqpoint{1.422232in}{1.717994in}}%
\pgfpathlineto{\pgfqpoint{1.420783in}{1.719030in}}%
\pgfpathclose%
\pgfusepath{fill}%
\end{pgfscope}%
\begin{pgfscope}%
\pgfpathrectangle{\pgfqpoint{0.329460in}{0.284240in}}{\pgfqpoint{1.989680in}{1.989680in}}%
\pgfusepath{clip}%
\pgfsetbuttcap%
\pgfsetroundjoin%
\definecolor{currentfill}{rgb}{0.195860,0.395433,0.555276}%
\pgfsetfillcolor{currentfill}%
\pgfsetlinewidth{0.000000pt}%
\definecolor{currentstroke}{rgb}{0.000000,0.000000,0.000000}%
\pgfsetstrokecolor{currentstroke}%
\pgfsetdash{}{0pt}%
\pgfpathmoveto{\pgfqpoint{1.114547in}{1.137624in}}%
\pgfpathlineto{\pgfqpoint{1.112218in}{1.129532in}}%
\pgfpathlineto{\pgfqpoint{1.109891in}{1.121473in}}%
\pgfpathlineto{\pgfqpoint{1.107563in}{1.113451in}}%
\pgfpathlineto{\pgfqpoint{1.105236in}{1.105467in}}%
\pgfpathlineto{\pgfqpoint{1.097080in}{1.109541in}}%
\pgfpathlineto{\pgfqpoint{1.089193in}{1.113743in}}%
\pgfpathlineto{\pgfqpoint{1.081583in}{1.118068in}}%
\pgfpathlineto{\pgfqpoint{1.074258in}{1.122511in}}%
\pgfpathlineto{\pgfqpoint{1.076871in}{1.130313in}}%
\pgfpathlineto{\pgfqpoint{1.079485in}{1.138154in}}%
\pgfpathlineto{\pgfqpoint{1.082100in}{1.146031in}}%
\pgfpathlineto{\pgfqpoint{1.084715in}{1.153942in}}%
\pgfpathlineto{\pgfqpoint{1.091771in}{1.149688in}}%
\pgfpathlineto{\pgfqpoint{1.099099in}{1.145547in}}%
\pgfpathlineto{\pgfqpoint{1.106694in}{1.141524in}}%
\pgfpathlineto{\pgfqpoint{1.114547in}{1.137624in}}%
\pgfpathclose%
\pgfusepath{fill}%
\end{pgfscope}%
\begin{pgfscope}%
\pgfpathrectangle{\pgfqpoint{0.329460in}{0.284240in}}{\pgfqpoint{1.989680in}{1.989680in}}%
\pgfusepath{clip}%
\pgfsetbuttcap%
\pgfsetroundjoin%
\definecolor{currentfill}{rgb}{0.896320,0.893616,0.096335}%
\pgfsetfillcolor{currentfill}%
\pgfsetlinewidth{0.000000pt}%
\definecolor{currentstroke}{rgb}{0.000000,0.000000,0.000000}%
\pgfsetstrokecolor{currentstroke}%
\pgfsetdash{}{0pt}%
\pgfpathmoveto{\pgfqpoint{1.297243in}{1.728259in}}%
\pgfpathlineto{\pgfqpoint{1.294563in}{1.726889in}}%
\pgfpathlineto{\pgfqpoint{1.291885in}{1.725407in}}%
\pgfpathlineto{\pgfqpoint{1.289209in}{1.723813in}}%
\pgfpathlineto{\pgfqpoint{1.286535in}{1.722110in}}%
\pgfpathlineto{\pgfqpoint{1.288257in}{1.723049in}}%
\pgfpathlineto{\pgfqpoint{1.290040in}{1.723963in}}%
\pgfpathlineto{\pgfqpoint{1.291884in}{1.724850in}}%
\pgfpathlineto{\pgfqpoint{1.293786in}{1.725709in}}%
\pgfpathlineto{\pgfqpoint{1.296159in}{1.727263in}}%
\pgfpathlineto{\pgfqpoint{1.298533in}{1.728706in}}%
\pgfpathlineto{\pgfqpoint{1.300910in}{1.730039in}}%
\pgfpathlineto{\pgfqpoint{1.303288in}{1.731259in}}%
\pgfpathlineto{\pgfqpoint{1.301702in}{1.730543in}}%
\pgfpathlineto{\pgfqpoint{1.300165in}{1.729804in}}%
\pgfpathlineto{\pgfqpoint{1.298678in}{1.729042in}}%
\pgfpathlineto{\pgfqpoint{1.297243in}{1.728259in}}%
\pgfpathclose%
\pgfusepath{fill}%
\end{pgfscope}%
\begin{pgfscope}%
\pgfpathrectangle{\pgfqpoint{0.329460in}{0.284240in}}{\pgfqpoint{1.989680in}{1.989680in}}%
\pgfusepath{clip}%
\pgfsetbuttcap%
\pgfsetroundjoin%
\definecolor{currentfill}{rgb}{0.268510,0.009605,0.335427}%
\pgfsetfillcolor{currentfill}%
\pgfsetlinewidth{0.000000pt}%
\definecolor{currentstroke}{rgb}{0.000000,0.000000,0.000000}%
\pgfsetstrokecolor{currentstroke}%
\pgfsetdash{}{0pt}%
\pgfpathmoveto{\pgfqpoint{1.710684in}{0.846094in}}%
\pgfpathlineto{\pgfqpoint{1.713156in}{0.843276in}}%
\pgfpathlineto{\pgfqpoint{1.715633in}{0.840656in}}%
\pgfpathlineto{\pgfqpoint{1.718115in}{0.838240in}}%
\pgfpathlineto{\pgfqpoint{1.720602in}{0.836030in}}%
\pgfpathlineto{\pgfqpoint{1.708764in}{0.829854in}}%
\pgfpathlineto{\pgfqpoint{1.696538in}{0.823877in}}%
\pgfpathlineto{\pgfqpoint{1.683934in}{0.818106in}}%
\pgfpathlineto{\pgfqpoint{1.670966in}{0.812548in}}%
\pgfpathlineto{\pgfqpoint{1.668806in}{0.814923in}}%
\pgfpathlineto{\pgfqpoint{1.666650in}{0.817507in}}%
\pgfpathlineto{\pgfqpoint{1.664499in}{0.820293in}}%
\pgfpathlineto{\pgfqpoint{1.662351in}{0.823278in}}%
\pgfpathlineto{\pgfqpoint{1.674977in}{0.828678in}}%
\pgfpathlineto{\pgfqpoint{1.687250in}{0.834285in}}%
\pgfpathlineto{\pgfqpoint{1.699156in}{0.840093in}}%
\pgfpathlineto{\pgfqpoint{1.710684in}{0.846094in}}%
\pgfpathclose%
\pgfusepath{fill}%
\end{pgfscope}%
\begin{pgfscope}%
\pgfpathrectangle{\pgfqpoint{0.329460in}{0.284240in}}{\pgfqpoint{1.989680in}{1.989680in}}%
\pgfusepath{clip}%
\pgfsetbuttcap%
\pgfsetroundjoin%
\definecolor{currentfill}{rgb}{0.699415,0.867117,0.175971}%
\pgfsetfillcolor{currentfill}%
\pgfsetlinewidth{0.000000pt}%
\definecolor{currentstroke}{rgb}{0.000000,0.000000,0.000000}%
\pgfsetstrokecolor{currentstroke}%
\pgfsetdash{}{0pt}%
\pgfpathmoveto{\pgfqpoint{1.225612in}{1.659285in}}%
\pgfpathlineto{\pgfqpoint{1.222153in}{1.655277in}}%
\pgfpathlineto{\pgfqpoint{1.218697in}{1.651171in}}%
\pgfpathlineto{\pgfqpoint{1.215244in}{1.646970in}}%
\pgfpathlineto{\pgfqpoint{1.211793in}{1.642674in}}%
\pgfpathlineto{\pgfqpoint{1.212162in}{1.644752in}}%
\pgfpathlineto{\pgfqpoint{1.212670in}{1.646823in}}%
\pgfpathlineto{\pgfqpoint{1.213316in}{1.648885in}}%
\pgfpathlineto{\pgfqpoint{1.214098in}{1.650935in}}%
\pgfpathlineto{\pgfqpoint{1.217488in}{1.655021in}}%
\pgfpathlineto{\pgfqpoint{1.220880in}{1.659012in}}%
\pgfpathlineto{\pgfqpoint{1.224274in}{1.662908in}}%
\pgfpathlineto{\pgfqpoint{1.227671in}{1.666708in}}%
\pgfpathlineto{\pgfqpoint{1.226971in}{1.664866in}}%
\pgfpathlineto{\pgfqpoint{1.226394in}{1.663013in}}%
\pgfpathlineto{\pgfqpoint{1.225940in}{1.661153in}}%
\pgfpathlineto{\pgfqpoint{1.225612in}{1.659285in}}%
\pgfpathclose%
\pgfusepath{fill}%
\end{pgfscope}%
\begin{pgfscope}%
\pgfpathrectangle{\pgfqpoint{0.329460in}{0.284240in}}{\pgfqpoint{1.989680in}{1.989680in}}%
\pgfusepath{clip}%
\pgfsetbuttcap%
\pgfsetroundjoin%
\definecolor{currentfill}{rgb}{0.935904,0.898570,0.108131}%
\pgfsetfillcolor{currentfill}%
\pgfsetlinewidth{0.000000pt}%
\definecolor{currentstroke}{rgb}{0.000000,0.000000,0.000000}%
\pgfsetstrokecolor{currentstroke}%
\pgfsetdash{}{0pt}%
\pgfpathmoveto{\pgfqpoint{1.370298in}{1.740525in}}%
\pgfpathlineto{\pgfqpoint{1.371485in}{1.740102in}}%
\pgfpathlineto{\pgfqpoint{1.372671in}{1.739565in}}%
\pgfpathlineto{\pgfqpoint{1.373856in}{1.738914in}}%
\pgfpathlineto{\pgfqpoint{1.375039in}{1.738149in}}%
\pgfpathlineto{\pgfqpoint{1.377098in}{1.737784in}}%
\pgfpathlineto{\pgfqpoint{1.379131in}{1.737388in}}%
\pgfpathlineto{\pgfqpoint{1.381137in}{1.736963in}}%
\pgfpathlineto{\pgfqpoint{1.383113in}{1.736508in}}%
\pgfpathlineto{\pgfqpoint{1.381528in}{1.737354in}}%
\pgfpathlineto{\pgfqpoint{1.379942in}{1.738087in}}%
\pgfpathlineto{\pgfqpoint{1.378354in}{1.738706in}}%
\pgfpathlineto{\pgfqpoint{1.376765in}{1.739211in}}%
\pgfpathlineto{\pgfqpoint{1.375182in}{1.739575in}}%
\pgfpathlineto{\pgfqpoint{1.373575in}{1.739915in}}%
\pgfpathlineto{\pgfqpoint{1.371947in}{1.740232in}}%
\pgfpathlineto{\pgfqpoint{1.370298in}{1.740525in}}%
\pgfpathclose%
\pgfusepath{fill}%
\end{pgfscope}%
\begin{pgfscope}%
\pgfpathrectangle{\pgfqpoint{0.329460in}{0.284240in}}{\pgfqpoint{1.989680in}{1.989680in}}%
\pgfusepath{clip}%
\pgfsetbuttcap%
\pgfsetroundjoin%
\definecolor{currentfill}{rgb}{0.344074,0.780029,0.397381}%
\pgfsetfillcolor{currentfill}%
\pgfsetlinewidth{0.000000pt}%
\definecolor{currentstroke}{rgb}{0.000000,0.000000,0.000000}%
\pgfsetstrokecolor{currentstroke}%
\pgfsetdash{}{0pt}%
\pgfpathmoveto{\pgfqpoint{1.165518in}{1.527243in}}%
\pgfpathlineto{\pgfqpoint{1.162244in}{1.520915in}}%
\pgfpathlineto{\pgfqpoint{1.158972in}{1.514519in}}%
\pgfpathlineto{\pgfqpoint{1.155702in}{1.508058in}}%
\pgfpathlineto{\pgfqpoint{1.152434in}{1.501532in}}%
\pgfpathlineto{\pgfqpoint{1.150543in}{1.504585in}}%
\pgfpathlineto{\pgfqpoint{1.148854in}{1.507663in}}%
\pgfpathlineto{\pgfqpoint{1.147370in}{1.510765in}}%
\pgfpathlineto{\pgfqpoint{1.146091in}{1.513885in}}%
\pgfpathlineto{\pgfqpoint{1.149456in}{1.520195in}}%
\pgfpathlineto{\pgfqpoint{1.152823in}{1.526441in}}%
\pgfpathlineto{\pgfqpoint{1.156194in}{1.532622in}}%
\pgfpathlineto{\pgfqpoint{1.159566in}{1.538735in}}%
\pgfpathlineto{\pgfqpoint{1.160768in}{1.535832in}}%
\pgfpathlineto{\pgfqpoint{1.162161in}{1.532946in}}%
\pgfpathlineto{\pgfqpoint{1.163745in}{1.530082in}}%
\pgfpathlineto{\pgfqpoint{1.165518in}{1.527243in}}%
\pgfpathclose%
\pgfusepath{fill}%
\end{pgfscope}%
\begin{pgfscope}%
\pgfpathrectangle{\pgfqpoint{0.329460in}{0.284240in}}{\pgfqpoint{1.989680in}{1.989680in}}%
\pgfusepath{clip}%
\pgfsetbuttcap%
\pgfsetroundjoin%
\definecolor{currentfill}{rgb}{0.274128,0.199721,0.498911}%
\pgfsetfillcolor{currentfill}%
\pgfsetlinewidth{0.000000pt}%
\definecolor{currentstroke}{rgb}{0.000000,0.000000,0.000000}%
\pgfsetstrokecolor{currentstroke}%
\pgfsetdash{}{0pt}%
\pgfpathmoveto{\pgfqpoint{1.108362in}{0.967969in}}%
\pgfpathlineto{\pgfqpoint{1.106357in}{0.960949in}}%
\pgfpathlineto{\pgfqpoint{1.104350in}{0.954024in}}%
\pgfpathlineto{\pgfqpoint{1.102342in}{0.947196in}}%
\pgfpathlineto{\pgfqpoint{1.100333in}{0.940470in}}%
\pgfpathlineto{\pgfqpoint{1.089476in}{0.944753in}}%
\pgfpathlineto{\pgfqpoint{1.078901in}{0.949212in}}%
\pgfpathlineto{\pgfqpoint{1.068619in}{0.953843in}}%
\pgfpathlineto{\pgfqpoint{1.058641in}{0.958638in}}%
\pgfpathlineto{\pgfqpoint{1.060978in}{0.965201in}}%
\pgfpathlineto{\pgfqpoint{1.063313in}{0.971865in}}%
\pgfpathlineto{\pgfqpoint{1.065647in}{0.978627in}}%
\pgfpathlineto{\pgfqpoint{1.067980in}{0.985483in}}%
\pgfpathlineto{\pgfqpoint{1.077646in}{0.980860in}}%
\pgfpathlineto{\pgfqpoint{1.087605in}{0.976396in}}%
\pgfpathlineto{\pgfqpoint{1.097848in}{0.972098in}}%
\pgfpathlineto{\pgfqpoint{1.108362in}{0.967969in}}%
\pgfpathclose%
\pgfusepath{fill}%
\end{pgfscope}%
\begin{pgfscope}%
\pgfpathrectangle{\pgfqpoint{0.329460in}{0.284240in}}{\pgfqpoint{1.989680in}{1.989680in}}%
\pgfusepath{clip}%
\pgfsetbuttcap%
\pgfsetroundjoin%
\definecolor{currentfill}{rgb}{0.179019,0.433756,0.557430}%
\pgfsetfillcolor{currentfill}%
\pgfsetlinewidth{0.000000pt}%
\definecolor{currentstroke}{rgb}{0.000000,0.000000,0.000000}%
\pgfsetstrokecolor{currentstroke}%
\pgfsetdash{}{0pt}%
\pgfpathmoveto{\pgfqpoint{1.613001in}{1.189569in}}%
\pgfpathlineto{\pgfqpoint{1.615676in}{1.181594in}}%
\pgfpathlineto{\pgfqpoint{1.618351in}{1.173642in}}%
\pgfpathlineto{\pgfqpoint{1.621024in}{1.165715in}}%
\pgfpathlineto{\pgfqpoint{1.623698in}{1.157816in}}%
\pgfpathlineto{\pgfqpoint{1.616890in}{1.153464in}}%
\pgfpathlineto{\pgfqpoint{1.609804in}{1.149222in}}%
\pgfpathlineto{\pgfqpoint{1.602445in}{1.145094in}}%
\pgfpathlineto{\pgfqpoint{1.594821in}{1.141085in}}%
\pgfpathlineto{\pgfqpoint{1.592425in}{1.149169in}}%
\pgfpathlineto{\pgfqpoint{1.590028in}{1.157281in}}%
\pgfpathlineto{\pgfqpoint{1.587631in}{1.165418in}}%
\pgfpathlineto{\pgfqpoint{1.585234in}{1.173577in}}%
\pgfpathlineto{\pgfqpoint{1.592563in}{1.177409in}}%
\pgfpathlineto{\pgfqpoint{1.599639in}{1.181355in}}%
\pgfpathlineto{\pgfqpoint{1.606454in}{1.185410in}}%
\pgfpathlineto{\pgfqpoint{1.613001in}{1.189569in}}%
\pgfpathclose%
\pgfusepath{fill}%
\end{pgfscope}%
\begin{pgfscope}%
\pgfpathrectangle{\pgfqpoint{0.329460in}{0.284240in}}{\pgfqpoint{1.989680in}{1.989680in}}%
\pgfusepath{clip}%
\pgfsetbuttcap%
\pgfsetroundjoin%
\definecolor{currentfill}{rgb}{0.201239,0.383670,0.554294}%
\pgfsetfillcolor{currentfill}%
\pgfsetlinewidth{0.000000pt}%
\definecolor{currentstroke}{rgb}{0.000000,0.000000,0.000000}%
\pgfsetstrokecolor{currentstroke}%
\pgfsetdash{}{0pt}%
\pgfpathmoveto{\pgfqpoint{0.746305in}{1.072869in}}%
\pgfpathlineto{\pgfqpoint{0.742689in}{1.083932in}}%
\pgfpathlineto{\pgfqpoint{0.739053in}{1.095431in}}%
\pgfpathlineto{\pgfqpoint{0.735397in}{1.107374in}}%
\pgfpathlineto{\pgfqpoint{0.731721in}{1.119768in}}%
\pgfpathlineto{\pgfqpoint{0.723623in}{1.129899in}}%
\pgfpathlineto{\pgfqpoint{0.716189in}{1.140140in}}%
\pgfpathlineto{\pgfqpoint{0.709425in}{1.150480in}}%
\pgfpathlineto{\pgfqpoint{0.703333in}{1.160907in}}%
\pgfpathlineto{\pgfqpoint{0.707153in}{1.148350in}}%
\pgfpathlineto{\pgfqpoint{0.710952in}{1.136241in}}%
\pgfpathlineto{\pgfqpoint{0.714731in}{1.124574in}}%
\pgfpathlineto{\pgfqpoint{0.718490in}{1.113341in}}%
\pgfpathlineto{\pgfqpoint{0.724463in}{1.103082in}}%
\pgfpathlineto{\pgfqpoint{0.731093in}{1.092910in}}%
\pgfpathlineto{\pgfqpoint{0.738375in}{1.082835in}}%
\pgfpathlineto{\pgfqpoint{0.746305in}{1.072869in}}%
\pgfpathclose%
\pgfusepath{fill}%
\end{pgfscope}%
\begin{pgfscope}%
\pgfpathrectangle{\pgfqpoint{0.329460in}{0.284240in}}{\pgfqpoint{1.989680in}{1.989680in}}%
\pgfusepath{clip}%
\pgfsetbuttcap%
\pgfsetroundjoin%
\definecolor{currentfill}{rgb}{0.855810,0.888601,0.097452}%
\pgfsetfillcolor{currentfill}%
\pgfsetlinewidth{0.000000pt}%
\definecolor{currentstroke}{rgb}{0.000000,0.000000,0.000000}%
\pgfsetstrokecolor{currentstroke}%
\pgfsetdash{}{0pt}%
\pgfpathmoveto{\pgfqpoint{1.275183in}{1.713773in}}%
\pgfpathlineto{\pgfqpoint{1.272037in}{1.711612in}}%
\pgfpathlineto{\pgfqpoint{1.268894in}{1.709342in}}%
\pgfpathlineto{\pgfqpoint{1.265753in}{1.706964in}}%
\pgfpathlineto{\pgfqpoint{1.262614in}{1.704479in}}%
\pgfpathlineto{\pgfqpoint{1.263980in}{1.705778in}}%
\pgfpathlineto{\pgfqpoint{1.265432in}{1.707055in}}%
\pgfpathlineto{\pgfqpoint{1.266968in}{1.708310in}}%
\pgfpathlineto{\pgfqpoint{1.268588in}{1.709542in}}%
\pgfpathlineto{\pgfqpoint{1.271512in}{1.711845in}}%
\pgfpathlineto{\pgfqpoint{1.274440in}{1.714042in}}%
\pgfpathlineto{\pgfqpoint{1.277369in}{1.716130in}}%
\pgfpathlineto{\pgfqpoint{1.280300in}{1.718110in}}%
\pgfpathlineto{\pgfqpoint{1.278913in}{1.717055in}}%
\pgfpathlineto{\pgfqpoint{1.277596in}{1.715980in}}%
\pgfpathlineto{\pgfqpoint{1.276353in}{1.714885in}}%
\pgfpathlineto{\pgfqpoint{1.275183in}{1.713773in}}%
\pgfpathclose%
\pgfusepath{fill}%
\end{pgfscope}%
\begin{pgfscope}%
\pgfpathrectangle{\pgfqpoint{0.329460in}{0.284240in}}{\pgfqpoint{1.989680in}{1.989680in}}%
\pgfusepath{clip}%
\pgfsetbuttcap%
\pgfsetroundjoin%
\definecolor{currentfill}{rgb}{0.636902,0.856542,0.216620}%
\pgfsetfillcolor{currentfill}%
\pgfsetlinewidth{0.000000pt}%
\definecolor{currentstroke}{rgb}{0.000000,0.000000,0.000000}%
\pgfsetstrokecolor{currentstroke}%
\pgfsetdash{}{0pt}%
\pgfpathmoveto{\pgfqpoint{1.490261in}{1.644522in}}%
\pgfpathlineto{\pgfqpoint{1.493700in}{1.640180in}}%
\pgfpathlineto{\pgfqpoint{1.497138in}{1.635747in}}%
\pgfpathlineto{\pgfqpoint{1.500572in}{1.631224in}}%
\pgfpathlineto{\pgfqpoint{1.504004in}{1.626612in}}%
\pgfpathlineto{\pgfqpoint{1.504399in}{1.624320in}}%
\pgfpathlineto{\pgfqpoint{1.504642in}{1.622023in}}%
\pgfpathlineto{\pgfqpoint{1.504731in}{1.619722in}}%
\pgfpathlineto{\pgfqpoint{1.504668in}{1.617421in}}%
\pgfpathlineto{\pgfqpoint{1.501216in}{1.622247in}}%
\pgfpathlineto{\pgfqpoint{1.497761in}{1.626983in}}%
\pgfpathlineto{\pgfqpoint{1.494304in}{1.631630in}}%
\pgfpathlineto{\pgfqpoint{1.490845in}{1.636184in}}%
\pgfpathlineto{\pgfqpoint{1.490908in}{1.638272in}}%
\pgfpathlineto{\pgfqpoint{1.490831in}{1.640359in}}%
\pgfpathlineto{\pgfqpoint{1.490615in}{1.642443in}}%
\pgfpathlineto{\pgfqpoint{1.490261in}{1.644522in}}%
\pgfpathclose%
\pgfusepath{fill}%
\end{pgfscope}%
\begin{pgfscope}%
\pgfpathrectangle{\pgfqpoint{0.329460in}{0.284240in}}{\pgfqpoint{1.989680in}{1.989680in}}%
\pgfusepath{clip}%
\pgfsetbuttcap%
\pgfsetroundjoin%
\definecolor{currentfill}{rgb}{0.935904,0.898570,0.108131}%
\pgfsetfillcolor{currentfill}%
\pgfsetlinewidth{0.000000pt}%
\definecolor{currentstroke}{rgb}{0.000000,0.000000,0.000000}%
\pgfsetstrokecolor{currentstroke}%
\pgfsetdash{}{0pt}%
\pgfpathmoveto{\pgfqpoint{1.324224in}{1.738868in}}%
\pgfpathlineto{\pgfqpoint{1.322549in}{1.738342in}}%
\pgfpathlineto{\pgfqpoint{1.320875in}{1.737702in}}%
\pgfpathlineto{\pgfqpoint{1.319203in}{1.736948in}}%
\pgfpathlineto{\pgfqpoint{1.317532in}{1.736080in}}%
\pgfpathlineto{\pgfqpoint{1.319480in}{1.736560in}}%
\pgfpathlineto{\pgfqpoint{1.321460in}{1.737011in}}%
\pgfpathlineto{\pgfqpoint{1.323469in}{1.737433in}}%
\pgfpathlineto{\pgfqpoint{1.325505in}{1.737826in}}%
\pgfpathlineto{\pgfqpoint{1.326779in}{1.738606in}}%
\pgfpathlineto{\pgfqpoint{1.328055in}{1.739274in}}%
\pgfpathlineto{\pgfqpoint{1.329332in}{1.739827in}}%
\pgfpathlineto{\pgfqpoint{1.330611in}{1.740266in}}%
\pgfpathlineto{\pgfqpoint{1.328980in}{1.739951in}}%
\pgfpathlineto{\pgfqpoint{1.327371in}{1.739614in}}%
\pgfpathlineto{\pgfqpoint{1.325785in}{1.739252in}}%
\pgfpathlineto{\pgfqpoint{1.324224in}{1.738868in}}%
\pgfpathclose%
\pgfusepath{fill}%
\end{pgfscope}%
\begin{pgfscope}%
\pgfpathrectangle{\pgfqpoint{0.329460in}{0.284240in}}{\pgfqpoint{1.989680in}{1.989680in}}%
\pgfusepath{clip}%
\pgfsetbuttcap%
\pgfsetroundjoin%
\definecolor{currentfill}{rgb}{0.487026,0.823929,0.312321}%
\pgfsetfillcolor{currentfill}%
\pgfsetlinewidth{0.000000pt}%
\definecolor{currentstroke}{rgb}{0.000000,0.000000,0.000000}%
\pgfsetstrokecolor{currentstroke}%
\pgfsetdash{}{0pt}%
\pgfpathmoveto{\pgfqpoint{1.186632in}{1.585012in}}%
\pgfpathlineto{\pgfqpoint{1.183241in}{1.579499in}}%
\pgfpathlineto{\pgfqpoint{1.179852in}{1.573905in}}%
\pgfpathlineto{\pgfqpoint{1.176465in}{1.568231in}}%
\pgfpathlineto{\pgfqpoint{1.173081in}{1.562480in}}%
\pgfpathlineto{\pgfqpoint{1.172137in}{1.565183in}}%
\pgfpathlineto{\pgfqpoint{1.171373in}{1.567897in}}%
\pgfpathlineto{\pgfqpoint{1.170789in}{1.570620in}}%
\pgfpathlineto{\pgfqpoint{1.170385in}{1.573349in}}%
\pgfpathlineto{\pgfqpoint{1.173815in}{1.578883in}}%
\pgfpathlineto{\pgfqpoint{1.177247in}{1.584341in}}%
\pgfpathlineto{\pgfqpoint{1.180682in}{1.589719in}}%
\pgfpathlineto{\pgfqpoint{1.184119in}{1.595017in}}%
\pgfpathlineto{\pgfqpoint{1.184497in}{1.592505in}}%
\pgfpathlineto{\pgfqpoint{1.185043in}{1.589998in}}%
\pgfpathlineto{\pgfqpoint{1.185755in}{1.587500in}}%
\pgfpathlineto{\pgfqpoint{1.186632in}{1.585012in}}%
\pgfpathclose%
\pgfusepath{fill}%
\end{pgfscope}%
\begin{pgfscope}%
\pgfpathrectangle{\pgfqpoint{0.329460in}{0.284240in}}{\pgfqpoint{1.989680in}{1.989680in}}%
\pgfusepath{clip}%
\pgfsetbuttcap%
\pgfsetroundjoin%
\definecolor{currentfill}{rgb}{0.134692,0.658636,0.517649}%
\pgfsetfillcolor{currentfill}%
\pgfsetlinewidth{0.000000pt}%
\definecolor{currentstroke}{rgb}{0.000000,0.000000,0.000000}%
\pgfsetstrokecolor{currentstroke}%
\pgfsetdash{}{0pt}%
\pgfpathmoveto{\pgfqpoint{1.580481in}{1.407648in}}%
\pgfpathlineto{\pgfqpoint{1.583609in}{1.400300in}}%
\pgfpathlineto{\pgfqpoint{1.586736in}{1.392917in}}%
\pgfpathlineto{\pgfqpoint{1.589861in}{1.385499in}}%
\pgfpathlineto{\pgfqpoint{1.592984in}{1.378050in}}%
\pgfpathlineto{\pgfqpoint{1.589599in}{1.374302in}}%
\pgfpathlineto{\pgfqpoint{1.585970in}{1.370606in}}%
\pgfpathlineto{\pgfqpoint{1.582100in}{1.366967in}}%
\pgfpathlineto{\pgfqpoint{1.577993in}{1.363388in}}%
\pgfpathlineto{\pgfqpoint{1.575057in}{1.371045in}}%
\pgfpathlineto{\pgfqpoint{1.572120in}{1.378670in}}%
\pgfpathlineto{\pgfqpoint{1.569180in}{1.386260in}}%
\pgfpathlineto{\pgfqpoint{1.566239in}{1.393813in}}%
\pgfpathlineto{\pgfqpoint{1.570140in}{1.397190in}}%
\pgfpathlineto{\pgfqpoint{1.573816in}{1.400624in}}%
\pgfpathlineto{\pgfqpoint{1.577264in}{1.404111in}}%
\pgfpathlineto{\pgfqpoint{1.580481in}{1.407648in}}%
\pgfpathclose%
\pgfusepath{fill}%
\end{pgfscope}%
\begin{pgfscope}%
\pgfpathrectangle{\pgfqpoint{0.329460in}{0.284240in}}{\pgfqpoint{1.989680in}{1.989680in}}%
\pgfusepath{clip}%
\pgfsetbuttcap%
\pgfsetroundjoin%
\definecolor{currentfill}{rgb}{0.282884,0.135920,0.453427}%
\pgfsetfillcolor{currentfill}%
\pgfsetlinewidth{0.000000pt}%
\definecolor{currentstroke}{rgb}{0.000000,0.000000,0.000000}%
\pgfsetstrokecolor{currentstroke}%
\pgfsetdash{}{0pt}%
\pgfpathmoveto{\pgfqpoint{1.874165in}{0.921670in}}%
\pgfpathlineto{\pgfqpoint{1.877381in}{0.926517in}}%
\pgfpathlineto{\pgfqpoint{1.880608in}{0.931697in}}%
\pgfpathlineto{\pgfqpoint{1.883849in}{0.937218in}}%
\pgfpathlineto{\pgfqpoint{1.887103in}{0.943083in}}%
\pgfpathlineto{\pgfqpoint{1.877257in}{0.934204in}}%
\pgfpathlineto{\pgfqpoint{1.866845in}{0.925482in}}%
\pgfpathlineto{\pgfqpoint{1.855876in}{0.916928in}}%
\pgfpathlineto{\pgfqpoint{1.844359in}{0.908552in}}%
\pgfpathlineto{\pgfqpoint{1.841349in}{0.902867in}}%
\pgfpathlineto{\pgfqpoint{1.838351in}{0.897528in}}%
\pgfpathlineto{\pgfqpoint{1.835366in}{0.892530in}}%
\pgfpathlineto{\pgfqpoint{1.832392in}{0.887867in}}%
\pgfpathlineto{\pgfqpoint{1.843644in}{0.896066in}}%
\pgfpathlineto{\pgfqpoint{1.854363in}{0.904439in}}%
\pgfpathlineto{\pgfqpoint{1.864540in}{0.912977in}}%
\pgfpathlineto{\pgfqpoint{1.874165in}{0.921670in}}%
\pgfpathclose%
\pgfusepath{fill}%
\end{pgfscope}%
\begin{pgfscope}%
\pgfpathrectangle{\pgfqpoint{0.329460in}{0.284240in}}{\pgfqpoint{1.989680in}{1.989680in}}%
\pgfusepath{clip}%
\pgfsetbuttcap%
\pgfsetroundjoin%
\definecolor{currentfill}{rgb}{0.220124,0.725509,0.466226}%
\pgfsetfillcolor{currentfill}%
\pgfsetlinewidth{0.000000pt}%
\definecolor{currentstroke}{rgb}{0.000000,0.000000,0.000000}%
\pgfsetstrokecolor{currentstroke}%
\pgfsetdash{}{0pt}%
\pgfpathmoveto{\pgfqpoint{1.149544in}{1.462097in}}%
\pgfpathlineto{\pgfqpoint{1.146437in}{1.455069in}}%
\pgfpathlineto{\pgfqpoint{1.143332in}{1.447987in}}%
\pgfpathlineto{\pgfqpoint{1.140229in}{1.440855in}}%
\pgfpathlineto{\pgfqpoint{1.137128in}{1.433674in}}%
\pgfpathlineto{\pgfqpoint{1.134105in}{1.437006in}}%
\pgfpathlineto{\pgfqpoint{1.131302in}{1.440382in}}%
\pgfpathlineto{\pgfqpoint{1.128724in}{1.443798in}}%
\pgfpathlineto{\pgfqpoint{1.126372in}{1.447249in}}%
\pgfpathlineto{\pgfqpoint{1.129622in}{1.454218in}}%
\pgfpathlineto{\pgfqpoint{1.132874in}{1.461139in}}%
\pgfpathlineto{\pgfqpoint{1.136128in}{1.468009in}}%
\pgfpathlineto{\pgfqpoint{1.139385in}{1.474826in}}%
\pgfpathlineto{\pgfqpoint{1.141608in}{1.471590in}}%
\pgfpathlineto{\pgfqpoint{1.144044in}{1.468387in}}%
\pgfpathlineto{\pgfqpoint{1.146690in}{1.465221in}}%
\pgfpathlineto{\pgfqpoint{1.149544in}{1.462097in}}%
\pgfpathclose%
\pgfusepath{fill}%
\end{pgfscope}%
\begin{pgfscope}%
\pgfpathrectangle{\pgfqpoint{0.329460in}{0.284240in}}{\pgfqpoint{1.989680in}{1.989680in}}%
\pgfusepath{clip}%
\pgfsetbuttcap%
\pgfsetroundjoin%
\definecolor{currentfill}{rgb}{0.248629,0.278775,0.534556}%
\pgfsetfillcolor{currentfill}%
\pgfsetlinewidth{0.000000pt}%
\definecolor{currentstroke}{rgb}{0.000000,0.000000,0.000000}%
\pgfsetstrokecolor{currentstroke}%
\pgfsetdash{}{0pt}%
\pgfpathmoveto{\pgfqpoint{1.623557in}{1.047285in}}%
\pgfpathlineto{\pgfqpoint{1.625952in}{1.039826in}}%
\pgfpathlineto{\pgfqpoint{1.628348in}{1.032436in}}%
\pgfpathlineto{\pgfqpoint{1.630744in}{1.025118in}}%
\pgfpathlineto{\pgfqpoint{1.633140in}{1.017875in}}%
\pgfpathlineto{\pgfqpoint{1.624046in}{1.013291in}}%
\pgfpathlineto{\pgfqpoint{1.614660in}{1.008856in}}%
\pgfpathlineto{\pgfqpoint{1.604993in}{1.004576in}}%
\pgfpathlineto{\pgfqpoint{1.595053in}{1.000455in}}%
\pgfpathlineto{\pgfqpoint{1.592975in}{1.007865in}}%
\pgfpathlineto{\pgfqpoint{1.590897in}{1.015351in}}%
\pgfpathlineto{\pgfqpoint{1.588820in}{1.022908in}}%
\pgfpathlineto{\pgfqpoint{1.586743in}{1.030535in}}%
\pgfpathlineto{\pgfqpoint{1.596349in}{1.034497in}}%
\pgfpathlineto{\pgfqpoint{1.605693in}{1.038612in}}%
\pgfpathlineto{\pgfqpoint{1.614766in}{1.042877in}}%
\pgfpathlineto{\pgfqpoint{1.623557in}{1.047285in}}%
\pgfpathclose%
\pgfusepath{fill}%
\end{pgfscope}%
\begin{pgfscope}%
\pgfpathrectangle{\pgfqpoint{0.329460in}{0.284240in}}{\pgfqpoint{1.989680in}{1.989680in}}%
\pgfusepath{clip}%
\pgfsetbuttcap%
\pgfsetroundjoin%
\definecolor{currentfill}{rgb}{0.271305,0.019942,0.347269}%
\pgfsetfillcolor{currentfill}%
\pgfsetlinewidth{0.000000pt}%
\definecolor{currentstroke}{rgb}{0.000000,0.000000,0.000000}%
\pgfsetstrokecolor{currentstroke}%
\pgfsetdash{}{0pt}%
\pgfpathmoveto{\pgfqpoint{1.059774in}{0.832641in}}%
\pgfpathlineto{\pgfqpoint{1.057719in}{0.828867in}}%
\pgfpathlineto{\pgfqpoint{1.055661in}{0.825276in}}%
\pgfpathlineto{\pgfqpoint{1.053599in}{0.821872in}}%
\pgfpathlineto{\pgfqpoint{1.051533in}{0.818658in}}%
\pgfpathlineto{\pgfqpoint{1.038604in}{0.823868in}}%
\pgfpathlineto{\pgfqpoint{1.026016in}{0.829291in}}%
\pgfpathlineto{\pgfqpoint{1.013784in}{0.834921in}}%
\pgfpathlineto{\pgfqpoint{1.001919in}{0.840750in}}%
\pgfpathlineto{\pgfqpoint{1.004319in}{0.843802in}}%
\pgfpathlineto{\pgfqpoint{1.006714in}{0.847044in}}%
\pgfpathlineto{\pgfqpoint{1.009106in}{0.850472in}}%
\pgfpathlineto{\pgfqpoint{1.011493in}{0.854083in}}%
\pgfpathlineto{\pgfqpoint{1.023041in}{0.848425in}}%
\pgfpathlineto{\pgfqpoint{1.034945in}{0.842961in}}%
\pgfpathlineto{\pgfqpoint{1.047193in}{0.837697in}}%
\pgfpathlineto{\pgfqpoint{1.059774in}{0.832641in}}%
\pgfpathclose%
\pgfusepath{fill}%
\end{pgfscope}%
\begin{pgfscope}%
\pgfpathrectangle{\pgfqpoint{0.329460in}{0.284240in}}{\pgfqpoint{1.989680in}{1.989680in}}%
\pgfusepath{clip}%
\pgfsetbuttcap%
\pgfsetroundjoin%
\definecolor{currentfill}{rgb}{0.935904,0.898570,0.108131}%
\pgfsetfillcolor{currentfill}%
\pgfsetlinewidth{0.000000pt}%
\definecolor{currentstroke}{rgb}{0.000000,0.000000,0.000000}%
\pgfsetstrokecolor{currentstroke}%
\pgfsetdash{}{0pt}%
\pgfpathmoveto{\pgfqpoint{1.376765in}{1.739211in}}%
\pgfpathlineto{\pgfqpoint{1.378354in}{1.738706in}}%
\pgfpathlineto{\pgfqpoint{1.379942in}{1.738087in}}%
\pgfpathlineto{\pgfqpoint{1.381528in}{1.737354in}}%
\pgfpathlineto{\pgfqpoint{1.383113in}{1.736508in}}%
\pgfpathlineto{\pgfqpoint{1.385058in}{1.736025in}}%
\pgfpathlineto{\pgfqpoint{1.386970in}{1.735513in}}%
\pgfpathlineto{\pgfqpoint{1.388846in}{1.734973in}}%
\pgfpathlineto{\pgfqpoint{1.390686in}{1.734406in}}%
\pgfpathlineto{\pgfqpoint{1.388724in}{1.735357in}}%
\pgfpathlineto{\pgfqpoint{1.386761in}{1.736195in}}%
\pgfpathlineto{\pgfqpoint{1.384796in}{1.736919in}}%
\pgfpathlineto{\pgfqpoint{1.382829in}{1.737527in}}%
\pgfpathlineto{\pgfqpoint{1.381356in}{1.737981in}}%
\pgfpathlineto{\pgfqpoint{1.379854in}{1.738414in}}%
\pgfpathlineto{\pgfqpoint{1.378323in}{1.738823in}}%
\pgfpathlineto{\pgfqpoint{1.376765in}{1.739211in}}%
\pgfpathclose%
\pgfusepath{fill}%
\end{pgfscope}%
\begin{pgfscope}%
\pgfpathrectangle{\pgfqpoint{0.329460in}{0.284240in}}{\pgfqpoint{1.989680in}{1.989680in}}%
\pgfusepath{clip}%
\pgfsetbuttcap%
\pgfsetroundjoin%
\definecolor{currentfill}{rgb}{0.263663,0.237631,0.518762}%
\pgfsetfillcolor{currentfill}%
\pgfsetlinewidth{0.000000pt}%
\definecolor{currentstroke}{rgb}{0.000000,0.000000,0.000000}%
\pgfsetstrokecolor{currentstroke}%
\pgfsetdash{}{0pt}%
\pgfpathmoveto{\pgfqpoint{1.116378in}{0.996930in}}%
\pgfpathlineto{\pgfqpoint{1.114375in}{0.989564in}}%
\pgfpathlineto{\pgfqpoint{1.112372in}{0.982280in}}%
\pgfpathlineto{\pgfqpoint{1.110367in}{0.975080in}}%
\pgfpathlineto{\pgfqpoint{1.108362in}{0.967969in}}%
\pgfpathlineto{\pgfqpoint{1.097848in}{0.972098in}}%
\pgfpathlineto{\pgfqpoint{1.087605in}{0.976396in}}%
\pgfpathlineto{\pgfqpoint{1.077646in}{0.980860in}}%
\pgfpathlineto{\pgfqpoint{1.067980in}{0.985483in}}%
\pgfpathlineto{\pgfqpoint{1.070313in}{0.992432in}}%
\pgfpathlineto{\pgfqpoint{1.072644in}{0.999468in}}%
\pgfpathlineto{\pgfqpoint{1.074974in}{1.006590in}}%
\pgfpathlineto{\pgfqpoint{1.077304in}{1.013793in}}%
\pgfpathlineto{\pgfqpoint{1.086658in}{1.009341in}}%
\pgfpathlineto{\pgfqpoint{1.096295in}{1.005043in}}%
\pgfpathlineto{\pgfqpoint{1.106205in}{1.000905in}}%
\pgfpathlineto{\pgfqpoint{1.116378in}{0.996930in}}%
\pgfpathclose%
\pgfusepath{fill}%
\end{pgfscope}%
\begin{pgfscope}%
\pgfpathrectangle{\pgfqpoint{0.329460in}{0.284240in}}{\pgfqpoint{1.989680in}{1.989680in}}%
\pgfusepath{clip}%
\pgfsetbuttcap%
\pgfsetroundjoin%
\definecolor{currentfill}{rgb}{0.935904,0.898570,0.108131}%
\pgfsetfillcolor{currentfill}%
\pgfsetlinewidth{0.000000pt}%
\definecolor{currentstroke}{rgb}{0.000000,0.000000,0.000000}%
\pgfsetstrokecolor{currentstroke}%
\pgfsetdash{}{0pt}%
\pgfpathmoveto{\pgfqpoint{1.318262in}{1.737106in}}%
\pgfpathlineto{\pgfqpoint{1.316216in}{1.736471in}}%
\pgfpathlineto{\pgfqpoint{1.314171in}{1.735721in}}%
\pgfpathlineto{\pgfqpoint{1.312128in}{1.734857in}}%
\pgfpathlineto{\pgfqpoint{1.310087in}{1.733880in}}%
\pgfpathlineto{\pgfqpoint{1.311892in}{1.734471in}}%
\pgfpathlineto{\pgfqpoint{1.313736in}{1.735035in}}%
\pgfpathlineto{\pgfqpoint{1.315616in}{1.735571in}}%
\pgfpathlineto{\pgfqpoint{1.317532in}{1.736080in}}%
\pgfpathlineto{\pgfqpoint{1.319203in}{1.736948in}}%
\pgfpathlineto{\pgfqpoint{1.320875in}{1.737702in}}%
\pgfpathlineto{\pgfqpoint{1.322549in}{1.738342in}}%
\pgfpathlineto{\pgfqpoint{1.324224in}{1.738868in}}%
\pgfpathlineto{\pgfqpoint{1.322690in}{1.738460in}}%
\pgfpathlineto{\pgfqpoint{1.321184in}{1.738031in}}%
\pgfpathlineto{\pgfqpoint{1.319708in}{1.737579in}}%
\pgfpathlineto{\pgfqpoint{1.318262in}{1.737106in}}%
\pgfpathclose%
\pgfusepath{fill}%
\end{pgfscope}%
\begin{pgfscope}%
\pgfpathrectangle{\pgfqpoint{0.329460in}{0.284240in}}{\pgfqpoint{1.989680in}{1.989680in}}%
\pgfusepath{clip}%
\pgfsetbuttcap%
\pgfsetroundjoin%
\definecolor{currentfill}{rgb}{0.896320,0.893616,0.096335}%
\pgfsetfillcolor{currentfill}%
\pgfsetlinewidth{0.000000pt}%
\definecolor{currentstroke}{rgb}{0.000000,0.000000,0.000000}%
\pgfsetstrokecolor{currentstroke}%
\pgfsetdash{}{0pt}%
\pgfpathmoveto{\pgfqpoint{1.403859in}{1.728956in}}%
\pgfpathlineto{\pgfqpoint{1.406476in}{1.727621in}}%
\pgfpathlineto{\pgfqpoint{1.409090in}{1.726173in}}%
\pgfpathlineto{\pgfqpoint{1.411703in}{1.724615in}}%
\pgfpathlineto{\pgfqpoint{1.414313in}{1.722946in}}%
\pgfpathlineto{\pgfqpoint{1.416028in}{1.722004in}}%
\pgfpathlineto{\pgfqpoint{1.417679in}{1.721036in}}%
\pgfpathlineto{\pgfqpoint{1.419264in}{1.720045in}}%
\pgfpathlineto{\pgfqpoint{1.420783in}{1.719030in}}%
\pgfpathlineto{\pgfqpoint{1.417903in}{1.720862in}}%
\pgfpathlineto{\pgfqpoint{1.415021in}{1.722584in}}%
\pgfpathlineto{\pgfqpoint{1.412137in}{1.724195in}}%
\pgfpathlineto{\pgfqpoint{1.409251in}{1.725693in}}%
\pgfpathlineto{\pgfqpoint{1.407986in}{1.726538in}}%
\pgfpathlineto{\pgfqpoint{1.406665in}{1.727365in}}%
\pgfpathlineto{\pgfqpoint{1.405289in}{1.728171in}}%
\pgfpathlineto{\pgfqpoint{1.403859in}{1.728956in}}%
\pgfpathclose%
\pgfusepath{fill}%
\end{pgfscope}%
\begin{pgfscope}%
\pgfpathrectangle{\pgfqpoint{0.329460in}{0.284240in}}{\pgfqpoint{1.989680in}{1.989680in}}%
\pgfusepath{clip}%
\pgfsetbuttcap%
\pgfsetroundjoin%
\definecolor{currentfill}{rgb}{0.636902,0.856542,0.216620}%
\pgfsetfillcolor{currentfill}%
\pgfsetlinewidth{0.000000pt}%
\definecolor{currentstroke}{rgb}{0.000000,0.000000,0.000000}%
\pgfsetstrokecolor{currentstroke}%
\pgfsetdash{}{0pt}%
\pgfpathmoveto{\pgfqpoint{1.211703in}{1.634329in}}%
\pgfpathlineto{\pgfqpoint{1.208247in}{1.629727in}}%
\pgfpathlineto{\pgfqpoint{1.204793in}{1.625034in}}%
\pgfpathlineto{\pgfqpoint{1.201341in}{1.620249in}}%
\pgfpathlineto{\pgfqpoint{1.197892in}{1.615376in}}%
\pgfpathlineto{\pgfqpoint{1.197693in}{1.617676in}}%
\pgfpathlineto{\pgfqpoint{1.197646in}{1.619978in}}%
\pgfpathlineto{\pgfqpoint{1.197753in}{1.622278in}}%
\pgfpathlineto{\pgfqpoint{1.198012in}{1.624575in}}%
\pgfpathlineto{\pgfqpoint{1.201454in}{1.629235in}}%
\pgfpathlineto{\pgfqpoint{1.204897in}{1.633805in}}%
\pgfpathlineto{\pgfqpoint{1.208344in}{1.638285in}}%
\pgfpathlineto{\pgfqpoint{1.211793in}{1.642674in}}%
\pgfpathlineto{\pgfqpoint{1.211561in}{1.640591in}}%
\pgfpathlineto{\pgfqpoint{1.211469in}{1.638504in}}%
\pgfpathlineto{\pgfqpoint{1.211516in}{1.636416in}}%
\pgfpathlineto{\pgfqpoint{1.211703in}{1.634329in}}%
\pgfpathclose%
\pgfusepath{fill}%
\end{pgfscope}%
\begin{pgfscope}%
\pgfpathrectangle{\pgfqpoint{0.329460in}{0.284240in}}{\pgfqpoint{1.989680in}{1.989680in}}%
\pgfusepath{clip}%
\pgfsetbuttcap%
\pgfsetroundjoin%
\definecolor{currentfill}{rgb}{0.133743,0.548535,0.553541}%
\pgfsetfillcolor{currentfill}%
\pgfsetlinewidth{0.000000pt}%
\definecolor{currentstroke}{rgb}{0.000000,0.000000,0.000000}%
\pgfsetstrokecolor{currentstroke}%
\pgfsetdash{}{0pt}%
\pgfpathmoveto{\pgfqpoint{1.126663in}{1.282856in}}%
\pgfpathlineto{\pgfqpoint{1.124033in}{1.274776in}}%
\pgfpathlineto{\pgfqpoint{1.121405in}{1.266687in}}%
\pgfpathlineto{\pgfqpoint{1.118778in}{1.258592in}}%
\pgfpathlineto{\pgfqpoint{1.116152in}{1.250493in}}%
\pgfpathlineto{\pgfqpoint{1.110156in}{1.254283in}}%
\pgfpathlineto{\pgfqpoint{1.104411in}{1.258165in}}%
\pgfpathlineto{\pgfqpoint{1.098923in}{1.262133in}}%
\pgfpathlineto{\pgfqpoint{1.093696in}{1.266185in}}%
\pgfpathlineto{\pgfqpoint{1.096566in}{1.274089in}}%
\pgfpathlineto{\pgfqpoint{1.099437in}{1.281990in}}%
\pgfpathlineto{\pgfqpoint{1.102310in}{1.289884in}}%
\pgfpathlineto{\pgfqpoint{1.105183in}{1.297770in}}%
\pgfpathlineto{\pgfqpoint{1.110184in}{1.293920in}}%
\pgfpathlineto{\pgfqpoint{1.115434in}{1.290148in}}%
\pgfpathlineto{\pgfqpoint{1.120929in}{1.286458in}}%
\pgfpathlineto{\pgfqpoint{1.126663in}{1.282856in}}%
\pgfpathclose%
\pgfusepath{fill}%
\end{pgfscope}%
\begin{pgfscope}%
\pgfpathrectangle{\pgfqpoint{0.329460in}{0.284240in}}{\pgfqpoint{1.989680in}{1.989680in}}%
\pgfusepath{clip}%
\pgfsetbuttcap%
\pgfsetroundjoin%
\definecolor{currentfill}{rgb}{0.260571,0.246922,0.522828}%
\pgfsetfillcolor{currentfill}%
\pgfsetlinewidth{0.000000pt}%
\definecolor{currentstroke}{rgb}{0.000000,0.000000,0.000000}%
\pgfsetstrokecolor{currentstroke}%
\pgfsetdash{}{0pt}%
\pgfpathmoveto{\pgfqpoint{0.811038in}{0.962060in}}%
\pgfpathlineto{\pgfqpoint{0.807763in}{0.969705in}}%
\pgfpathlineto{\pgfqpoint{0.804472in}{0.977733in}}%
\pgfpathlineto{\pgfqpoint{0.801165in}{0.986150in}}%
\pgfpathlineto{\pgfqpoint{0.797843in}{0.994964in}}%
\pgfpathlineto{\pgfqpoint{0.787612in}{1.004216in}}%
\pgfpathlineto{\pgfqpoint{0.777988in}{1.013619in}}%
\pgfpathlineto{\pgfqpoint{0.768976in}{1.023163in}}%
\pgfpathlineto{\pgfqpoint{0.760585in}{1.032836in}}%
\pgfpathlineto{\pgfqpoint{0.764111in}{1.023847in}}%
\pgfpathlineto{\pgfqpoint{0.767620in}{1.015252in}}%
\pgfpathlineto{\pgfqpoint{0.771112in}{1.007046in}}%
\pgfpathlineto{\pgfqpoint{0.774589in}{0.999220in}}%
\pgfpathlineto{\pgfqpoint{0.782802in}{0.989728in}}%
\pgfpathlineto{\pgfqpoint{0.791619in}{0.980364in}}%
\pgfpathlineto{\pgfqpoint{0.801033in}{0.971138in}}%
\pgfpathlineto{\pgfqpoint{0.811038in}{0.962060in}}%
\pgfpathclose%
\pgfusepath{fill}%
\end{pgfscope}%
\begin{pgfscope}%
\pgfpathrectangle{\pgfqpoint{0.329460in}{0.284240in}}{\pgfqpoint{1.989680in}{1.989680in}}%
\pgfusepath{clip}%
\pgfsetbuttcap%
\pgfsetroundjoin%
\definecolor{currentfill}{rgb}{0.896320,0.893616,0.096335}%
\pgfsetfillcolor{currentfill}%
\pgfsetlinewidth{0.000000pt}%
\definecolor{currentstroke}{rgb}{0.000000,0.000000,0.000000}%
\pgfsetstrokecolor{currentstroke}%
\pgfsetdash{}{0pt}%
\pgfpathmoveto{\pgfqpoint{1.292048in}{1.724926in}}%
\pgfpathlineto{\pgfqpoint{1.289108in}{1.723389in}}%
\pgfpathlineto{\pgfqpoint{1.286170in}{1.721741in}}%
\pgfpathlineto{\pgfqpoint{1.283234in}{1.719981in}}%
\pgfpathlineto{\pgfqpoint{1.280300in}{1.718110in}}%
\pgfpathlineto{\pgfqpoint{1.281758in}{1.719144in}}%
\pgfpathlineto{\pgfqpoint{1.283284in}{1.720156in}}%
\pgfpathlineto{\pgfqpoint{1.284877in}{1.721145in}}%
\pgfpathlineto{\pgfqpoint{1.286535in}{1.722110in}}%
\pgfpathlineto{\pgfqpoint{1.289209in}{1.723813in}}%
\pgfpathlineto{\pgfqpoint{1.291885in}{1.725407in}}%
\pgfpathlineto{\pgfqpoint{1.294563in}{1.726889in}}%
\pgfpathlineto{\pgfqpoint{1.297243in}{1.728259in}}%
\pgfpathlineto{\pgfqpoint{1.295861in}{1.727455in}}%
\pgfpathlineto{\pgfqpoint{1.294533in}{1.726631in}}%
\pgfpathlineto{\pgfqpoint{1.293262in}{1.725788in}}%
\pgfpathlineto{\pgfqpoint{1.292048in}{1.724926in}}%
\pgfpathclose%
\pgfusepath{fill}%
\end{pgfscope}%
\begin{pgfscope}%
\pgfpathrectangle{\pgfqpoint{0.329460in}{0.284240in}}{\pgfqpoint{1.989680in}{1.989680in}}%
\pgfusepath{clip}%
\pgfsetbuttcap%
\pgfsetroundjoin%
\definecolor{currentfill}{rgb}{0.412913,0.803041,0.357269}%
\pgfsetfillcolor{currentfill}%
\pgfsetlinewidth{0.000000pt}%
\definecolor{currentstroke}{rgb}{0.000000,0.000000,0.000000}%
\pgfsetstrokecolor{currentstroke}%
\pgfsetdash{}{0pt}%
\pgfpathmoveto{\pgfqpoint{1.530142in}{1.564882in}}%
\pgfpathlineto{\pgfqpoint{1.533539in}{1.559103in}}%
\pgfpathlineto{\pgfqpoint{1.536934in}{1.553250in}}%
\pgfpathlineto{\pgfqpoint{1.540326in}{1.547325in}}%
\pgfpathlineto{\pgfqpoint{1.543715in}{1.541329in}}%
\pgfpathlineto{\pgfqpoint{1.542685in}{1.538412in}}%
\pgfpathlineto{\pgfqpoint{1.541462in}{1.535510in}}%
\pgfpathlineto{\pgfqpoint{1.540048in}{1.532627in}}%
\pgfpathlineto{\pgfqpoint{1.538443in}{1.529766in}}%
\pgfpathlineto{\pgfqpoint{1.535140in}{1.535977in}}%
\pgfpathlineto{\pgfqpoint{1.531834in}{1.542117in}}%
\pgfpathlineto{\pgfqpoint{1.528527in}{1.548185in}}%
\pgfpathlineto{\pgfqpoint{1.525218in}{1.554178in}}%
\pgfpathlineto{\pgfqpoint{1.526715in}{1.556826in}}%
\pgfpathlineto{\pgfqpoint{1.528036in}{1.559495in}}%
\pgfpathlineto{\pgfqpoint{1.529179in}{1.562181in}}%
\pgfpathlineto{\pgfqpoint{1.530142in}{1.564882in}}%
\pgfpathclose%
\pgfusepath{fill}%
\end{pgfscope}%
\begin{pgfscope}%
\pgfpathrectangle{\pgfqpoint{0.329460in}{0.284240in}}{\pgfqpoint{1.989680in}{1.989680in}}%
\pgfusepath{clip}%
\pgfsetbuttcap%
\pgfsetroundjoin%
\definecolor{currentfill}{rgb}{0.179019,0.433756,0.557430}%
\pgfsetfillcolor{currentfill}%
\pgfsetlinewidth{0.000000pt}%
\definecolor{currentstroke}{rgb}{0.000000,0.000000,0.000000}%
\pgfsetstrokecolor{currentstroke}%
\pgfsetdash{}{0pt}%
\pgfpathmoveto{\pgfqpoint{1.123864in}{1.170270in}}%
\pgfpathlineto{\pgfqpoint{1.121534in}{1.162073in}}%
\pgfpathlineto{\pgfqpoint{1.119204in}{1.153897in}}%
\pgfpathlineto{\pgfqpoint{1.116875in}{1.145747in}}%
\pgfpathlineto{\pgfqpoint{1.114547in}{1.137624in}}%
\pgfpathlineto{\pgfqpoint{1.106694in}{1.141524in}}%
\pgfpathlineto{\pgfqpoint{1.099099in}{1.145547in}}%
\pgfpathlineto{\pgfqpoint{1.091771in}{1.149688in}}%
\pgfpathlineto{\pgfqpoint{1.084715in}{1.153942in}}%
\pgfpathlineto{\pgfqpoint{1.087330in}{1.161884in}}%
\pgfpathlineto{\pgfqpoint{1.089946in}{1.169854in}}%
\pgfpathlineto{\pgfqpoint{1.092563in}{1.177849in}}%
\pgfpathlineto{\pgfqpoint{1.095181in}{1.185867in}}%
\pgfpathlineto{\pgfqpoint{1.101966in}{1.181800in}}%
\pgfpathlineto{\pgfqpoint{1.109013in}{1.177842in}}%
\pgfpathlineto{\pgfqpoint{1.116315in}{1.173997in}}%
\pgfpathlineto{\pgfqpoint{1.123864in}{1.170270in}}%
\pgfpathclose%
\pgfusepath{fill}%
\end{pgfscope}%
\begin{pgfscope}%
\pgfpathrectangle{\pgfqpoint{0.329460in}{0.284240in}}{\pgfqpoint{1.989680in}{1.989680in}}%
\pgfusepath{clip}%
\pgfsetbuttcap%
\pgfsetroundjoin%
\definecolor{currentfill}{rgb}{0.814576,0.883393,0.110347}%
\pgfsetfillcolor{currentfill}%
\pgfsetlinewidth{0.000000pt}%
\definecolor{currentstroke}{rgb}{0.000000,0.000000,0.000000}%
\pgfsetstrokecolor{currentstroke}%
\pgfsetdash{}{0pt}%
\pgfpathmoveto{\pgfqpoint{1.443446in}{1.700319in}}%
\pgfpathlineto{\pgfqpoint{1.446716in}{1.697579in}}%
\pgfpathlineto{\pgfqpoint{1.449983in}{1.694733in}}%
\pgfpathlineto{\pgfqpoint{1.453248in}{1.691784in}}%
\pgfpathlineto{\pgfqpoint{1.456511in}{1.688731in}}%
\pgfpathlineto{\pgfqpoint{1.457654in}{1.687162in}}%
\pgfpathlineto{\pgfqpoint{1.458693in}{1.685576in}}%
\pgfpathlineto{\pgfqpoint{1.459625in}{1.683975in}}%
\pgfpathlineto{\pgfqpoint{1.460449in}{1.682361in}}%
\pgfpathlineto{\pgfqpoint{1.457061in}{1.685615in}}%
\pgfpathlineto{\pgfqpoint{1.453671in}{1.688765in}}%
\pgfpathlineto{\pgfqpoint{1.450278in}{1.691811in}}%
\pgfpathlineto{\pgfqpoint{1.446883in}{1.694751in}}%
\pgfpathlineto{\pgfqpoint{1.446164in}{1.696162in}}%
\pgfpathlineto{\pgfqpoint{1.445351in}{1.697561in}}%
\pgfpathlineto{\pgfqpoint{1.444445in}{1.698947in}}%
\pgfpathlineto{\pgfqpoint{1.443446in}{1.700319in}}%
\pgfpathclose%
\pgfusepath{fill}%
\end{pgfscope}%
\begin{pgfscope}%
\pgfpathrectangle{\pgfqpoint{0.329460in}{0.284240in}}{\pgfqpoint{1.989680in}{1.989680in}}%
\pgfusepath{clip}%
\pgfsetbuttcap%
\pgfsetroundjoin%
\definecolor{currentfill}{rgb}{0.134692,0.658636,0.517649}%
\pgfsetfillcolor{currentfill}%
\pgfsetlinewidth{0.000000pt}%
\definecolor{currentstroke}{rgb}{0.000000,0.000000,0.000000}%
\pgfsetstrokecolor{currentstroke}%
\pgfsetdash{}{0pt}%
\pgfpathmoveto{\pgfqpoint{1.139789in}{1.390863in}}%
\pgfpathlineto{\pgfqpoint{1.136897in}{1.383265in}}%
\pgfpathlineto{\pgfqpoint{1.134006in}{1.375631in}}%
\pgfpathlineto{\pgfqpoint{1.131116in}{1.367962in}}%
\pgfpathlineto{\pgfqpoint{1.128229in}{1.360261in}}%
\pgfpathlineto{\pgfqpoint{1.123914in}{1.363783in}}%
\pgfpathlineto{\pgfqpoint{1.119833in}{1.367369in}}%
\pgfpathlineto{\pgfqpoint{1.115990in}{1.371014in}}%
\pgfpathlineto{\pgfqpoint{1.112388in}{1.374716in}}%
\pgfpathlineto{\pgfqpoint{1.115474in}{1.382212in}}%
\pgfpathlineto{\pgfqpoint{1.118561in}{1.389677in}}%
\pgfpathlineto{\pgfqpoint{1.121651in}{1.397107in}}%
\pgfpathlineto{\pgfqpoint{1.124742in}{1.404502in}}%
\pgfpathlineto{\pgfqpoint{1.128165in}{1.401009in}}%
\pgfpathlineto{\pgfqpoint{1.131816in}{1.397569in}}%
\pgfpathlineto{\pgfqpoint{1.135692in}{1.394186in}}%
\pgfpathlineto{\pgfqpoint{1.139789in}{1.390863in}}%
\pgfpathclose%
\pgfusepath{fill}%
\end{pgfscope}%
\begin{pgfscope}%
\pgfpathrectangle{\pgfqpoint{0.329460in}{0.284240in}}{\pgfqpoint{1.989680in}{1.989680in}}%
\pgfusepath{clip}%
\pgfsetbuttcap%
\pgfsetroundjoin%
\definecolor{currentfill}{rgb}{0.281477,0.755203,0.432552}%
\pgfsetfillcolor{currentfill}%
\pgfsetlinewidth{0.000000pt}%
\definecolor{currentstroke}{rgb}{0.000000,0.000000,0.000000}%
\pgfsetstrokecolor{currentstroke}%
\pgfsetdash{}{0pt}%
\pgfpathmoveto{\pgfqpoint{1.551632in}{1.504244in}}%
\pgfpathlineto{\pgfqpoint{1.554924in}{1.497704in}}%
\pgfpathlineto{\pgfqpoint{1.558214in}{1.491104in}}%
\pgfpathlineto{\pgfqpoint{1.561501in}{1.484445in}}%
\pgfpathlineto{\pgfqpoint{1.564786in}{1.477730in}}%
\pgfpathlineto{\pgfqpoint{1.562754in}{1.474465in}}%
\pgfpathlineto{\pgfqpoint{1.560507in}{1.471232in}}%
\pgfpathlineto{\pgfqpoint{1.558048in}{1.468033in}}%
\pgfpathlineto{\pgfqpoint{1.555378in}{1.464872in}}%
\pgfpathlineto{\pgfqpoint{1.552231in}{1.471800in}}%
\pgfpathlineto{\pgfqpoint{1.549082in}{1.478670in}}%
\pgfpathlineto{\pgfqpoint{1.545932in}{1.485482in}}%
\pgfpathlineto{\pgfqpoint{1.542779in}{1.492233in}}%
\pgfpathlineto{\pgfqpoint{1.545290in}{1.495186in}}%
\pgfpathlineto{\pgfqpoint{1.547604in}{1.498174in}}%
\pgfpathlineto{\pgfqpoint{1.549719in}{1.501194in}}%
\pgfpathlineto{\pgfqpoint{1.551632in}{1.504244in}}%
\pgfpathclose%
\pgfusepath{fill}%
\end{pgfscope}%
\begin{pgfscope}%
\pgfpathrectangle{\pgfqpoint{0.329460in}{0.284240in}}{\pgfqpoint{1.989680in}{1.989680in}}%
\pgfusepath{clip}%
\pgfsetbuttcap%
\pgfsetroundjoin%
\definecolor{currentfill}{rgb}{0.565498,0.842430,0.262877}%
\pgfsetfillcolor{currentfill}%
\pgfsetlinewidth{0.000000pt}%
\definecolor{currentstroke}{rgb}{0.000000,0.000000,0.000000}%
\pgfsetstrokecolor{currentstroke}%
\pgfsetdash{}{0pt}%
\pgfpathmoveto{\pgfqpoint{1.504668in}{1.617421in}}%
\pgfpathlineto{\pgfqpoint{1.508118in}{1.612507in}}%
\pgfpathlineto{\pgfqpoint{1.511565in}{1.607507in}}%
\pgfpathlineto{\pgfqpoint{1.515010in}{1.602422in}}%
\pgfpathlineto{\pgfqpoint{1.518453in}{1.597254in}}%
\pgfpathlineto{\pgfqpoint{1.518222in}{1.594738in}}%
\pgfpathlineto{\pgfqpoint{1.517825in}{1.592226in}}%
\pgfpathlineto{\pgfqpoint{1.517261in}{1.589720in}}%
\pgfpathlineto{\pgfqpoint{1.516531in}{1.587223in}}%
\pgfpathlineto{\pgfqpoint{1.513123in}{1.592606in}}%
\pgfpathlineto{\pgfqpoint{1.509712in}{1.597906in}}%
\pgfpathlineto{\pgfqpoint{1.506299in}{1.603121in}}%
\pgfpathlineto{\pgfqpoint{1.502884in}{1.608249in}}%
\pgfpathlineto{\pgfqpoint{1.503559in}{1.610532in}}%
\pgfpathlineto{\pgfqpoint{1.504082in}{1.612823in}}%
\pgfpathlineto{\pgfqpoint{1.504451in}{1.615120in}}%
\pgfpathlineto{\pgfqpoint{1.504668in}{1.617421in}}%
\pgfpathclose%
\pgfusepath{fill}%
\end{pgfscope}%
\begin{pgfscope}%
\pgfpathrectangle{\pgfqpoint{0.329460in}{0.284240in}}{\pgfqpoint{1.989680in}{1.989680in}}%
\pgfusepath{clip}%
\pgfsetbuttcap%
\pgfsetroundjoin%
\definecolor{currentfill}{rgb}{0.267004,0.004874,0.329415}%
\pgfsetfillcolor{currentfill}%
\pgfsetlinewidth{0.000000pt}%
\definecolor{currentstroke}{rgb}{0.000000,0.000000,0.000000}%
\pgfsetstrokecolor{currentstroke}%
\pgfsetdash{}{0pt}%
\pgfpathmoveto{\pgfqpoint{1.720602in}{0.836030in}}%
\pgfpathlineto{\pgfqpoint{1.723093in}{0.834032in}}%
\pgfpathlineto{\pgfqpoint{1.725590in}{0.832249in}}%
\pgfpathlineto{\pgfqpoint{1.728093in}{0.830687in}}%
\pgfpathlineto{\pgfqpoint{1.730601in}{0.829349in}}%
\pgfpathlineto{\pgfqpoint{1.718453in}{0.822999in}}%
\pgfpathlineto{\pgfqpoint{1.705903in}{0.816854in}}%
\pgfpathlineto{\pgfqpoint{1.692966in}{0.810920in}}%
\pgfpathlineto{\pgfqpoint{1.679653in}{0.805204in}}%
\pgfpathlineto{\pgfqpoint{1.677474in}{0.806707in}}%
\pgfpathlineto{\pgfqpoint{1.675300in}{0.808435in}}%
\pgfpathlineto{\pgfqpoint{1.673130in}{0.810383in}}%
\pgfpathlineto{\pgfqpoint{1.670966in}{0.812548in}}%
\pgfpathlineto{\pgfqpoint{1.683934in}{0.818106in}}%
\pgfpathlineto{\pgfqpoint{1.696538in}{0.823877in}}%
\pgfpathlineto{\pgfqpoint{1.708764in}{0.829854in}}%
\pgfpathlineto{\pgfqpoint{1.720602in}{0.836030in}}%
\pgfpathclose%
\pgfusepath{fill}%
\end{pgfscope}%
\begin{pgfscope}%
\pgfpathrectangle{\pgfqpoint{0.329460in}{0.284240in}}{\pgfqpoint{1.989680in}{1.989680in}}%
\pgfusepath{clip}%
\pgfsetbuttcap%
\pgfsetroundjoin%
\definecolor{currentfill}{rgb}{0.935904,0.898570,0.108131}%
\pgfsetfillcolor{currentfill}%
\pgfsetlinewidth{0.000000pt}%
\definecolor{currentstroke}{rgb}{0.000000,0.000000,0.000000}%
\pgfsetstrokecolor{currentstroke}%
\pgfsetdash{}{0pt}%
\pgfpathmoveto{\pgfqpoint{1.382829in}{1.737527in}}%
\pgfpathlineto{\pgfqpoint{1.384796in}{1.736919in}}%
\pgfpathlineto{\pgfqpoint{1.386761in}{1.736195in}}%
\pgfpathlineto{\pgfqpoint{1.388724in}{1.735357in}}%
\pgfpathlineto{\pgfqpoint{1.390686in}{1.734406in}}%
\pgfpathlineto{\pgfqpoint{1.392486in}{1.733812in}}%
\pgfpathlineto{\pgfqpoint{1.394246in}{1.733192in}}%
\pgfpathlineto{\pgfqpoint{1.395964in}{1.732547in}}%
\pgfpathlineto{\pgfqpoint{1.397638in}{1.731876in}}%
\pgfpathlineto{\pgfqpoint{1.395330in}{1.732953in}}%
\pgfpathlineto{\pgfqpoint{1.393020in}{1.733917in}}%
\pgfpathlineto{\pgfqpoint{1.390709in}{1.734767in}}%
\pgfpathlineto{\pgfqpoint{1.388396in}{1.735502in}}%
\pgfpathlineto{\pgfqpoint{1.387056in}{1.736039in}}%
\pgfpathlineto{\pgfqpoint{1.385681in}{1.736556in}}%
\pgfpathlineto{\pgfqpoint{1.384271in}{1.737052in}}%
\pgfpathlineto{\pgfqpoint{1.382829in}{1.737527in}}%
\pgfpathclose%
\pgfusepath{fill}%
\end{pgfscope}%
\begin{pgfscope}%
\pgfpathrectangle{\pgfqpoint{0.329460in}{0.284240in}}{\pgfqpoint{1.989680in}{1.989680in}}%
\pgfusepath{clip}%
\pgfsetbuttcap%
\pgfsetroundjoin%
\definecolor{currentfill}{rgb}{0.231674,0.318106,0.544834}%
\pgfsetfillcolor{currentfill}%
\pgfsetlinewidth{0.000000pt}%
\definecolor{currentstroke}{rgb}{0.000000,0.000000,0.000000}%
\pgfsetstrokecolor{currentstroke}%
\pgfsetdash{}{0pt}%
\pgfpathmoveto{\pgfqpoint{1.613980in}{1.077753in}}%
\pgfpathlineto{\pgfqpoint{1.616374in}{1.070047in}}%
\pgfpathlineto{\pgfqpoint{1.618768in}{1.062399in}}%
\pgfpathlineto{\pgfqpoint{1.621163in}{1.054811in}}%
\pgfpathlineto{\pgfqpoint{1.623557in}{1.047285in}}%
\pgfpathlineto{\pgfqpoint{1.614766in}{1.042877in}}%
\pgfpathlineto{\pgfqpoint{1.605693in}{1.038612in}}%
\pgfpathlineto{\pgfqpoint{1.596349in}{1.034497in}}%
\pgfpathlineto{\pgfqpoint{1.586743in}{1.030535in}}%
\pgfpathlineto{\pgfqpoint{1.584667in}{1.038227in}}%
\pgfpathlineto{\pgfqpoint{1.582591in}{1.045982in}}%
\pgfpathlineto{\pgfqpoint{1.580515in}{1.053797in}}%
\pgfpathlineto{\pgfqpoint{1.578439in}{1.061668in}}%
\pgfpathlineto{\pgfqpoint{1.587712in}{1.065472in}}%
\pgfpathlineto{\pgfqpoint{1.596732in}{1.069424in}}%
\pgfpathlineto{\pgfqpoint{1.605491in}{1.073519in}}%
\pgfpathlineto{\pgfqpoint{1.613980in}{1.077753in}}%
\pgfpathclose%
\pgfusepath{fill}%
\end{pgfscope}%
\begin{pgfscope}%
\pgfpathrectangle{\pgfqpoint{0.329460in}{0.284240in}}{\pgfqpoint{1.989680in}{1.989680in}}%
\pgfusepath{clip}%
\pgfsetbuttcap%
\pgfsetroundjoin%
\definecolor{currentfill}{rgb}{0.122606,0.585371,0.546557}%
\pgfsetfillcolor{currentfill}%
\pgfsetlinewidth{0.000000pt}%
\definecolor{currentstroke}{rgb}{0.000000,0.000000,0.000000}%
\pgfsetstrokecolor{currentstroke}%
\pgfsetdash{}{0pt}%
\pgfpathmoveto{\pgfqpoint{1.589721in}{1.332487in}}%
\pgfpathlineto{\pgfqpoint{1.592649in}{1.324704in}}%
\pgfpathlineto{\pgfqpoint{1.595575in}{1.316903in}}%
\pgfpathlineto{\pgfqpoint{1.598500in}{1.309086in}}%
\pgfpathlineto{\pgfqpoint{1.601423in}{1.301256in}}%
\pgfpathlineto{\pgfqpoint{1.596649in}{1.297339in}}%
\pgfpathlineto{\pgfqpoint{1.591620in}{1.293497in}}%
\pgfpathlineto{\pgfqpoint{1.586343in}{1.289734in}}%
\pgfpathlineto{\pgfqpoint{1.580821in}{1.286054in}}%
\pgfpathlineto{\pgfqpoint{1.578132in}{1.294081in}}%
\pgfpathlineto{\pgfqpoint{1.575442in}{1.302094in}}%
\pgfpathlineto{\pgfqpoint{1.572750in}{1.310091in}}%
\pgfpathlineto{\pgfqpoint{1.570058in}{1.318070in}}%
\pgfpathlineto{\pgfqpoint{1.575327in}{1.321559in}}%
\pgfpathlineto{\pgfqpoint{1.580363in}{1.325127in}}%
\pgfpathlineto{\pgfqpoint{1.585163in}{1.328771in}}%
\pgfpathlineto{\pgfqpoint{1.589721in}{1.332487in}}%
\pgfpathclose%
\pgfusepath{fill}%
\end{pgfscope}%
\begin{pgfscope}%
\pgfpathrectangle{\pgfqpoint{0.329460in}{0.284240in}}{\pgfqpoint{1.989680in}{1.989680in}}%
\pgfusepath{clip}%
\pgfsetbuttcap%
\pgfsetroundjoin%
\definecolor{currentfill}{rgb}{0.762373,0.876424,0.137064}%
\pgfsetfillcolor{currentfill}%
\pgfsetlinewidth{0.000000pt}%
\definecolor{currentstroke}{rgb}{0.000000,0.000000,0.000000}%
\pgfsetstrokecolor{currentstroke}%
\pgfsetdash{}{0pt}%
\pgfpathmoveto{\pgfqpoint{1.460449in}{1.682361in}}%
\pgfpathlineto{\pgfqpoint{1.463835in}{1.679006in}}%
\pgfpathlineto{\pgfqpoint{1.467219in}{1.675549in}}%
\pgfpathlineto{\pgfqpoint{1.470600in}{1.671992in}}%
\pgfpathlineto{\pgfqpoint{1.473978in}{1.668336in}}%
\pgfpathlineto{\pgfqpoint{1.474788in}{1.666504in}}%
\pgfpathlineto{\pgfqpoint{1.475474in}{1.664660in}}%
\pgfpathlineto{\pgfqpoint{1.476038in}{1.662807in}}%
\pgfpathlineto{\pgfqpoint{1.476477in}{1.660945in}}%
\pgfpathlineto{\pgfqpoint{1.473026in}{1.664809in}}%
\pgfpathlineto{\pgfqpoint{1.469572in}{1.668574in}}%
\pgfpathlineto{\pgfqpoint{1.466116in}{1.672239in}}%
\pgfpathlineto{\pgfqpoint{1.462657in}{1.675802in}}%
\pgfpathlineto{\pgfqpoint{1.462270in}{1.677454in}}%
\pgfpathlineto{\pgfqpoint{1.461773in}{1.679099in}}%
\pgfpathlineto{\pgfqpoint{1.461166in}{1.680735in}}%
\pgfpathlineto{\pgfqpoint{1.460449in}{1.682361in}}%
\pgfpathclose%
\pgfusepath{fill}%
\end{pgfscope}%
\begin{pgfscope}%
\pgfpathrectangle{\pgfqpoint{0.329460in}{0.284240in}}{\pgfqpoint{1.989680in}{1.989680in}}%
\pgfusepath{clip}%
\pgfsetbuttcap%
\pgfsetroundjoin%
\definecolor{currentfill}{rgb}{0.855810,0.888601,0.097452}%
\pgfsetfillcolor{currentfill}%
\pgfsetlinewidth{0.000000pt}%
\definecolor{currentstroke}{rgb}{0.000000,0.000000,0.000000}%
\pgfsetstrokecolor{currentstroke}%
\pgfsetdash{}{0pt}%
\pgfpathmoveto{\pgfqpoint{1.426156in}{1.714763in}}%
\pgfpathlineto{\pgfqpoint{1.429258in}{1.712643in}}%
\pgfpathlineto{\pgfqpoint{1.432358in}{1.710414in}}%
\pgfpathlineto{\pgfqpoint{1.435456in}{1.708078in}}%
\pgfpathlineto{\pgfqpoint{1.438552in}{1.705635in}}%
\pgfpathlineto{\pgfqpoint{1.439908in}{1.704334in}}%
\pgfpathlineto{\pgfqpoint{1.441177in}{1.703014in}}%
\pgfpathlineto{\pgfqpoint{1.442357in}{1.701675in}}%
\pgfpathlineto{\pgfqpoint{1.443446in}{1.700319in}}%
\pgfpathlineto{\pgfqpoint{1.440175in}{1.702953in}}%
\pgfpathlineto{\pgfqpoint{1.436901in}{1.705480in}}%
\pgfpathlineto{\pgfqpoint{1.433624in}{1.707899in}}%
\pgfpathlineto{\pgfqpoint{1.430346in}{1.710209in}}%
\pgfpathlineto{\pgfqpoint{1.429414in}{1.711370in}}%
\pgfpathlineto{\pgfqpoint{1.428404in}{1.712517in}}%
\pgfpathlineto{\pgfqpoint{1.427318in}{1.713648in}}%
\pgfpathlineto{\pgfqpoint{1.426156in}{1.714763in}}%
\pgfpathclose%
\pgfusepath{fill}%
\end{pgfscope}%
\begin{pgfscope}%
\pgfpathrectangle{\pgfqpoint{0.329460in}{0.284240in}}{\pgfqpoint{1.989680in}{1.989680in}}%
\pgfusepath{clip}%
\pgfsetbuttcap%
\pgfsetroundjoin%
\definecolor{currentfill}{rgb}{0.268510,0.009605,0.335427}%
\pgfsetfillcolor{currentfill}%
\pgfsetlinewidth{0.000000pt}%
\definecolor{currentstroke}{rgb}{0.000000,0.000000,0.000000}%
\pgfsetstrokecolor{currentstroke}%
\pgfsetdash{}{0pt}%
\pgfpathmoveto{\pgfqpoint{1.051533in}{0.818658in}}%
\pgfpathlineto{\pgfqpoint{1.049464in}{0.815638in}}%
\pgfpathlineto{\pgfqpoint{1.047391in}{0.812818in}}%
\pgfpathlineto{\pgfqpoint{1.045313in}{0.810201in}}%
\pgfpathlineto{\pgfqpoint{1.043232in}{0.807791in}}%
\pgfpathlineto{\pgfqpoint{1.029951in}{0.813154in}}%
\pgfpathlineto{\pgfqpoint{1.017023in}{0.818737in}}%
\pgfpathlineto{\pgfqpoint{1.004460in}{0.824531in}}%
\pgfpathlineto{\pgfqpoint{0.992276in}{0.830531in}}%
\pgfpathlineto{\pgfqpoint{0.994694in}{0.832779in}}%
\pgfpathlineto{\pgfqpoint{0.997107in}{0.835235in}}%
\pgfpathlineto{\pgfqpoint{0.999515in}{0.837893in}}%
\pgfpathlineto{\pgfqpoint{1.001919in}{0.840750in}}%
\pgfpathlineto{\pgfqpoint{1.013784in}{0.834921in}}%
\pgfpathlineto{\pgfqpoint{1.026016in}{0.829291in}}%
\pgfpathlineto{\pgfqpoint{1.038604in}{0.823868in}}%
\pgfpathlineto{\pgfqpoint{1.051533in}{0.818658in}}%
\pgfpathclose%
\pgfusepath{fill}%
\end{pgfscope}%
\begin{pgfscope}%
\pgfpathrectangle{\pgfqpoint{0.329460in}{0.284240in}}{\pgfqpoint{1.989680in}{1.989680in}}%
\pgfusepath{clip}%
\pgfsetbuttcap%
\pgfsetroundjoin%
\definecolor{currentfill}{rgb}{0.935904,0.898570,0.108131}%
\pgfsetfillcolor{currentfill}%
\pgfsetlinewidth{0.000000pt}%
\definecolor{currentstroke}{rgb}{0.000000,0.000000,0.000000}%
\pgfsetstrokecolor{currentstroke}%
\pgfsetdash{}{0pt}%
\pgfpathmoveto{\pgfqpoint{1.312819in}{1.735008in}}%
\pgfpathlineto{\pgfqpoint{1.310434in}{1.734242in}}%
\pgfpathlineto{\pgfqpoint{1.308050in}{1.733362in}}%
\pgfpathlineto{\pgfqpoint{1.305668in}{1.732367in}}%
\pgfpathlineto{\pgfqpoint{1.303288in}{1.731259in}}%
\pgfpathlineto{\pgfqpoint{1.304921in}{1.731952in}}%
\pgfpathlineto{\pgfqpoint{1.306600in}{1.732620in}}%
\pgfpathlineto{\pgfqpoint{1.308322in}{1.733263in}}%
\pgfpathlineto{\pgfqpoint{1.310087in}{1.733880in}}%
\pgfpathlineto{\pgfqpoint{1.312128in}{1.734857in}}%
\pgfpathlineto{\pgfqpoint{1.314171in}{1.735721in}}%
\pgfpathlineto{\pgfqpoint{1.316216in}{1.736471in}}%
\pgfpathlineto{\pgfqpoint{1.318262in}{1.737106in}}%
\pgfpathlineto{\pgfqpoint{1.316850in}{1.736612in}}%
\pgfpathlineto{\pgfqpoint{1.315471in}{1.736097in}}%
\pgfpathlineto{\pgfqpoint{1.314127in}{1.735562in}}%
\pgfpathlineto{\pgfqpoint{1.312819in}{1.735008in}}%
\pgfpathclose%
\pgfusepath{fill}%
\end{pgfscope}%
\begin{pgfscope}%
\pgfpathrectangle{\pgfqpoint{0.329460in}{0.284240in}}{\pgfqpoint{1.989680in}{1.989680in}}%
\pgfusepath{clip}%
\pgfsetbuttcap%
\pgfsetroundjoin%
\definecolor{currentfill}{rgb}{0.163625,0.471133,0.558148}%
\pgfsetfillcolor{currentfill}%
\pgfsetlinewidth{0.000000pt}%
\definecolor{currentstroke}{rgb}{0.000000,0.000000,0.000000}%
\pgfsetstrokecolor{currentstroke}%
\pgfsetdash{}{0pt}%
\pgfpathmoveto{\pgfqpoint{1.602291in}{1.221641in}}%
\pgfpathlineto{\pgfqpoint{1.604970in}{1.213603in}}%
\pgfpathlineto{\pgfqpoint{1.607647in}{1.205576in}}%
\pgfpathlineto{\pgfqpoint{1.610325in}{1.197564in}}%
\pgfpathlineto{\pgfqpoint{1.613001in}{1.189569in}}%
\pgfpathlineto{\pgfqpoint{1.606454in}{1.185410in}}%
\pgfpathlineto{\pgfqpoint{1.599639in}{1.181355in}}%
\pgfpathlineto{\pgfqpoint{1.592563in}{1.177409in}}%
\pgfpathlineto{\pgfqpoint{1.585234in}{1.173577in}}%
\pgfpathlineto{\pgfqpoint{1.582835in}{1.181756in}}%
\pgfpathlineto{\pgfqpoint{1.580436in}{1.189951in}}%
\pgfpathlineto{\pgfqpoint{1.578037in}{1.198161in}}%
\pgfpathlineto{\pgfqpoint{1.575636in}{1.206382in}}%
\pgfpathlineto{\pgfqpoint{1.582671in}{1.210038in}}%
\pgfpathlineto{\pgfqpoint{1.589463in}{1.213802in}}%
\pgfpathlineto{\pgfqpoint{1.596005in}{1.217671in}}%
\pgfpathlineto{\pgfqpoint{1.602291in}{1.221641in}}%
\pgfpathclose%
\pgfusepath{fill}%
\end{pgfscope}%
\begin{pgfscope}%
\pgfpathrectangle{\pgfqpoint{0.329460in}{0.284240in}}{\pgfqpoint{1.989680in}{1.989680in}}%
\pgfusepath{clip}%
\pgfsetbuttcap%
\pgfsetroundjoin%
\definecolor{currentfill}{rgb}{0.248629,0.278775,0.534556}%
\pgfsetfillcolor{currentfill}%
\pgfsetlinewidth{0.000000pt}%
\definecolor{currentstroke}{rgb}{0.000000,0.000000,0.000000}%
\pgfsetstrokecolor{currentstroke}%
\pgfsetdash{}{0pt}%
\pgfpathmoveto{\pgfqpoint{1.124383in}{1.027146in}}%
\pgfpathlineto{\pgfqpoint{1.122382in}{1.019485in}}%
\pgfpathlineto{\pgfqpoint{1.120381in}{1.011894in}}%
\pgfpathlineto{\pgfqpoint{1.118380in}{1.004374in}}%
\pgfpathlineto{\pgfqpoint{1.116378in}{0.996930in}}%
\pgfpathlineto{\pgfqpoint{1.106205in}{1.000905in}}%
\pgfpathlineto{\pgfqpoint{1.096295in}{1.005043in}}%
\pgfpathlineto{\pgfqpoint{1.086658in}{1.009341in}}%
\pgfpathlineto{\pgfqpoint{1.077304in}{1.013793in}}%
\pgfpathlineto{\pgfqpoint{1.079634in}{1.021075in}}%
\pgfpathlineto{\pgfqpoint{1.081962in}{1.028432in}}%
\pgfpathlineto{\pgfqpoint{1.084291in}{1.035861in}}%
\pgfpathlineto{\pgfqpoint{1.086618in}{1.043360in}}%
\pgfpathlineto{\pgfqpoint{1.095660in}{1.039079in}}%
\pgfpathlineto{\pgfqpoint{1.104974in}{1.034946in}}%
\pgfpathlineto{\pgfqpoint{1.114552in}{1.030967in}}%
\pgfpathlineto{\pgfqpoint{1.124383in}{1.027146in}}%
\pgfpathclose%
\pgfusepath{fill}%
\end{pgfscope}%
\begin{pgfscope}%
\pgfpathrectangle{\pgfqpoint{0.329460in}{0.284240in}}{\pgfqpoint{1.989680in}{1.989680in}}%
\pgfusepath{clip}%
\pgfsetbuttcap%
\pgfsetroundjoin%
\definecolor{currentfill}{rgb}{0.272594,0.025563,0.353093}%
\pgfsetfillcolor{currentfill}%
\pgfsetlinewidth{0.000000pt}%
\definecolor{currentstroke}{rgb}{0.000000,0.000000,0.000000}%
\pgfsetstrokecolor{currentstroke}%
\pgfsetdash{}{0pt}%
\pgfpathmoveto{\pgfqpoint{0.962820in}{0.821316in}}%
\pgfpathlineto{\pgfqpoint{0.960322in}{0.822168in}}%
\pgfpathlineto{\pgfqpoint{0.957817in}{0.823292in}}%
\pgfpathlineto{\pgfqpoint{0.955305in}{0.824693in}}%
\pgfpathlineto{\pgfqpoint{0.952784in}{0.826375in}}%
\pgfpathlineto{\pgfqpoint{0.939735in}{0.833258in}}%
\pgfpathlineto{\pgfqpoint{0.927137in}{0.840352in}}%
\pgfpathlineto{\pgfqpoint{0.915003in}{0.847648in}}%
\pgfpathlineto{\pgfqpoint{0.903343in}{0.855138in}}%
\pgfpathlineto{\pgfqpoint{0.906164in}{0.853281in}}%
\pgfpathlineto{\pgfqpoint{0.908977in}{0.851706in}}%
\pgfpathlineto{\pgfqpoint{0.911781in}{0.850407in}}%
\pgfpathlineto{\pgfqpoint{0.914576in}{0.849378in}}%
\pgfpathlineto{\pgfqpoint{0.925956in}{0.842071in}}%
\pgfpathlineto{\pgfqpoint{0.937797in}{0.834952in}}%
\pgfpathlineto{\pgfqpoint{0.950089in}{0.828031in}}%
\pgfpathlineto{\pgfqpoint{0.962820in}{0.821316in}}%
\pgfpathclose%
\pgfusepath{fill}%
\end{pgfscope}%
\begin{pgfscope}%
\pgfpathrectangle{\pgfqpoint{0.329460in}{0.284240in}}{\pgfqpoint{1.989680in}{1.989680in}}%
\pgfusepath{clip}%
\pgfsetbuttcap%
\pgfsetroundjoin%
\definecolor{currentfill}{rgb}{0.814576,0.883393,0.110347}%
\pgfsetfillcolor{currentfill}%
\pgfsetlinewidth{0.000000pt}%
\definecolor{currentstroke}{rgb}{0.000000,0.000000,0.000000}%
\pgfsetstrokecolor{currentstroke}%
\pgfsetdash{}{0pt}%
\pgfpathmoveto{\pgfqpoint{1.254932in}{1.693489in}}%
\pgfpathlineto{\pgfqpoint{1.251517in}{1.690503in}}%
\pgfpathlineto{\pgfqpoint{1.248104in}{1.687411in}}%
\pgfpathlineto{\pgfqpoint{1.244693in}{1.684216in}}%
\pgfpathlineto{\pgfqpoint{1.241284in}{1.680917in}}%
\pgfpathlineto{\pgfqpoint{1.242012in}{1.682541in}}%
\pgfpathlineto{\pgfqpoint{1.242849in}{1.684154in}}%
\pgfpathlineto{\pgfqpoint{1.243793in}{1.685753in}}%
\pgfpathlineto{\pgfqpoint{1.244843in}{1.687337in}}%
\pgfpathlineto{\pgfqpoint{1.248138in}{1.690434in}}%
\pgfpathlineto{\pgfqpoint{1.251435in}{1.693427in}}%
\pgfpathlineto{\pgfqpoint{1.254735in}{1.696316in}}%
\pgfpathlineto{\pgfqpoint{1.258036in}{1.699100in}}%
\pgfpathlineto{\pgfqpoint{1.257120in}{1.697716in}}%
\pgfpathlineto{\pgfqpoint{1.256296in}{1.696318in}}%
\pgfpathlineto{\pgfqpoint{1.255567in}{1.694909in}}%
\pgfpathlineto{\pgfqpoint{1.254932in}{1.693489in}}%
\pgfpathclose%
\pgfusepath{fill}%
\end{pgfscope}%
\begin{pgfscope}%
\pgfpathrectangle{\pgfqpoint{0.329460in}{0.284240in}}{\pgfqpoint{1.989680in}{1.989680in}}%
\pgfusepath{clip}%
\pgfsetbuttcap%
\pgfsetroundjoin%
\definecolor{currentfill}{rgb}{0.855810,0.888601,0.097452}%
\pgfsetfillcolor{currentfill}%
\pgfsetlinewidth{0.000000pt}%
\definecolor{currentstroke}{rgb}{0.000000,0.000000,0.000000}%
\pgfsetstrokecolor{currentstroke}%
\pgfsetdash{}{0pt}%
\pgfpathmoveto{\pgfqpoint{1.271266in}{1.709165in}}%
\pgfpathlineto{\pgfqpoint{1.267955in}{1.706811in}}%
\pgfpathlineto{\pgfqpoint{1.264647in}{1.704349in}}%
\pgfpathlineto{\pgfqpoint{1.261341in}{1.701778in}}%
\pgfpathlineto{\pgfqpoint{1.258036in}{1.699100in}}%
\pgfpathlineto{\pgfqpoint{1.259045in}{1.700470in}}%
\pgfpathlineto{\pgfqpoint{1.260145in}{1.701824in}}%
\pgfpathlineto{\pgfqpoint{1.261335in}{1.703161in}}%
\pgfpathlineto{\pgfqpoint{1.262614in}{1.704479in}}%
\pgfpathlineto{\pgfqpoint{1.265753in}{1.706964in}}%
\pgfpathlineto{\pgfqpoint{1.268894in}{1.709342in}}%
\pgfpathlineto{\pgfqpoint{1.272037in}{1.711612in}}%
\pgfpathlineto{\pgfqpoint{1.275183in}{1.713773in}}%
\pgfpathlineto{\pgfqpoint{1.274088in}{1.712643in}}%
\pgfpathlineto{\pgfqpoint{1.273070in}{1.711498in}}%
\pgfpathlineto{\pgfqpoint{1.272129in}{1.710338in}}%
\pgfpathlineto{\pgfqpoint{1.271266in}{1.709165in}}%
\pgfpathclose%
\pgfusepath{fill}%
\end{pgfscope}%
\begin{pgfscope}%
\pgfpathrectangle{\pgfqpoint{0.329460in}{0.284240in}}{\pgfqpoint{1.989680in}{1.989680in}}%
\pgfusepath{clip}%
\pgfsetbuttcap%
\pgfsetroundjoin%
\definecolor{currentfill}{rgb}{0.412913,0.803041,0.357269}%
\pgfsetfillcolor{currentfill}%
\pgfsetlinewidth{0.000000pt}%
\definecolor{currentstroke}{rgb}{0.000000,0.000000,0.000000}%
\pgfsetstrokecolor{currentstroke}%
\pgfsetdash{}{0pt}%
\pgfpathmoveto{\pgfqpoint{1.178636in}{1.551842in}}%
\pgfpathlineto{\pgfqpoint{1.175353in}{1.545803in}}%
\pgfpathlineto{\pgfqpoint{1.172073in}{1.539688in}}%
\pgfpathlineto{\pgfqpoint{1.168794in}{1.533501in}}%
\pgfpathlineto{\pgfqpoint{1.165518in}{1.527243in}}%
\pgfpathlineto{\pgfqpoint{1.163745in}{1.530082in}}%
\pgfpathlineto{\pgfqpoint{1.162161in}{1.532946in}}%
\pgfpathlineto{\pgfqpoint{1.160768in}{1.535832in}}%
\pgfpathlineto{\pgfqpoint{1.159566in}{1.538735in}}%
\pgfpathlineto{\pgfqpoint{1.162941in}{1.544779in}}%
\pgfpathlineto{\pgfqpoint{1.166319in}{1.550753in}}%
\pgfpathlineto{\pgfqpoint{1.169699in}{1.556654in}}%
\pgfpathlineto{\pgfqpoint{1.173081in}{1.562480in}}%
\pgfpathlineto{\pgfqpoint{1.174203in}{1.559793in}}%
\pgfpathlineto{\pgfqpoint{1.175504in}{1.557122in}}%
\pgfpathlineto{\pgfqpoint{1.176982in}{1.554471in}}%
\pgfpathlineto{\pgfqpoint{1.178636in}{1.551842in}}%
\pgfpathclose%
\pgfusepath{fill}%
\end{pgfscope}%
\begin{pgfscope}%
\pgfpathrectangle{\pgfqpoint{0.329460in}{0.284240in}}{\pgfqpoint{1.989680in}{1.989680in}}%
\pgfusepath{clip}%
\pgfsetbuttcap%
\pgfsetroundjoin%
\definecolor{currentfill}{rgb}{0.762373,0.876424,0.137064}%
\pgfsetfillcolor{currentfill}%
\pgfsetlinewidth{0.000000pt}%
\definecolor{currentstroke}{rgb}{0.000000,0.000000,0.000000}%
\pgfsetstrokecolor{currentstroke}%
\pgfsetdash{}{0pt}%
\pgfpathmoveto{\pgfqpoint{1.239467in}{1.674329in}}%
\pgfpathlineto{\pgfqpoint{1.236000in}{1.670719in}}%
\pgfpathlineto{\pgfqpoint{1.232535in}{1.667008in}}%
\pgfpathlineto{\pgfqpoint{1.229072in}{1.663196in}}%
\pgfpathlineto{\pgfqpoint{1.225612in}{1.659285in}}%
\pgfpathlineto{\pgfqpoint{1.225940in}{1.661153in}}%
\pgfpathlineto{\pgfqpoint{1.226394in}{1.663013in}}%
\pgfpathlineto{\pgfqpoint{1.226971in}{1.664866in}}%
\pgfpathlineto{\pgfqpoint{1.227671in}{1.666708in}}%
\pgfpathlineto{\pgfqpoint{1.231071in}{1.670410in}}%
\pgfpathlineto{\pgfqpoint{1.234473in}{1.674012in}}%
\pgfpathlineto{\pgfqpoint{1.237877in}{1.677515in}}%
\pgfpathlineto{\pgfqpoint{1.241284in}{1.680917in}}%
\pgfpathlineto{\pgfqpoint{1.240664in}{1.679281in}}%
\pgfpathlineto{\pgfqpoint{1.240155in}{1.677637in}}%
\pgfpathlineto{\pgfqpoint{1.239755in}{1.675986in}}%
\pgfpathlineto{\pgfqpoint{1.239467in}{1.674329in}}%
\pgfpathclose%
\pgfusepath{fill}%
\end{pgfscope}%
\begin{pgfscope}%
\pgfpathrectangle{\pgfqpoint{0.329460in}{0.284240in}}{\pgfqpoint{1.989680in}{1.989680in}}%
\pgfusepath{clip}%
\pgfsetbuttcap%
\pgfsetroundjoin%
\definecolor{currentfill}{rgb}{0.166383,0.690856,0.496502}%
\pgfsetfillcolor{currentfill}%
\pgfsetlinewidth{0.000000pt}%
\definecolor{currentstroke}{rgb}{0.000000,0.000000,0.000000}%
\pgfsetstrokecolor{currentstroke}%
\pgfsetdash{}{0pt}%
\pgfpathmoveto{\pgfqpoint{1.567946in}{1.436634in}}%
\pgfpathlineto{\pgfqpoint{1.571082in}{1.429452in}}%
\pgfpathlineto{\pgfqpoint{1.574217in}{1.422226in}}%
\pgfpathlineto{\pgfqpoint{1.577350in}{1.414957in}}%
\pgfpathlineto{\pgfqpoint{1.580481in}{1.407648in}}%
\pgfpathlineto{\pgfqpoint{1.577264in}{1.404111in}}%
\pgfpathlineto{\pgfqpoint{1.573816in}{1.400624in}}%
\pgfpathlineto{\pgfqpoint{1.570140in}{1.397190in}}%
\pgfpathlineto{\pgfqpoint{1.566239in}{1.393813in}}%
\pgfpathlineto{\pgfqpoint{1.563296in}{1.401328in}}%
\pgfpathlineto{\pgfqpoint{1.560352in}{1.408802in}}%
\pgfpathlineto{\pgfqpoint{1.557405in}{1.416234in}}%
\pgfpathlineto{\pgfqpoint{1.554457in}{1.423620in}}%
\pgfpathlineto{\pgfqpoint{1.558151in}{1.426796in}}%
\pgfpathlineto{\pgfqpoint{1.561632in}{1.430026in}}%
\pgfpathlineto{\pgfqpoint{1.564898in}{1.433306in}}%
\pgfpathlineto{\pgfqpoint{1.567946in}{1.436634in}}%
\pgfpathclose%
\pgfusepath{fill}%
\end{pgfscope}%
\begin{pgfscope}%
\pgfpathrectangle{\pgfqpoint{0.329460in}{0.284240in}}{\pgfqpoint{1.989680in}{1.989680in}}%
\pgfusepath{clip}%
\pgfsetbuttcap%
\pgfsetroundjoin%
\definecolor{currentfill}{rgb}{0.565498,0.842430,0.262877}%
\pgfsetfillcolor{currentfill}%
\pgfsetlinewidth{0.000000pt}%
\definecolor{currentstroke}{rgb}{0.000000,0.000000,0.000000}%
\pgfsetstrokecolor{currentstroke}%
\pgfsetdash{}{0pt}%
\pgfpathmoveto{\pgfqpoint{1.200219in}{1.606228in}}%
\pgfpathlineto{\pgfqpoint{1.196819in}{1.601053in}}%
\pgfpathlineto{\pgfqpoint{1.193422in}{1.595791in}}%
\pgfpathlineto{\pgfqpoint{1.190026in}{1.590443in}}%
\pgfpathlineto{\pgfqpoint{1.186632in}{1.585012in}}%
\pgfpathlineto{\pgfqpoint{1.185755in}{1.587500in}}%
\pgfpathlineto{\pgfqpoint{1.185043in}{1.589998in}}%
\pgfpathlineto{\pgfqpoint{1.184497in}{1.592505in}}%
\pgfpathlineto{\pgfqpoint{1.184119in}{1.595017in}}%
\pgfpathlineto{\pgfqpoint{1.187559in}{1.600234in}}%
\pgfpathlineto{\pgfqpoint{1.191001in}{1.605366in}}%
\pgfpathlineto{\pgfqpoint{1.194445in}{1.610414in}}%
\pgfpathlineto{\pgfqpoint{1.197892in}{1.615376in}}%
\pgfpathlineto{\pgfqpoint{1.198245in}{1.613078in}}%
\pgfpathlineto{\pgfqpoint{1.198751in}{1.610786in}}%
\pgfpathlineto{\pgfqpoint{1.199409in}{1.608502in}}%
\pgfpathlineto{\pgfqpoint{1.200219in}{1.606228in}}%
\pgfpathclose%
\pgfusepath{fill}%
\end{pgfscope}%
\begin{pgfscope}%
\pgfpathrectangle{\pgfqpoint{0.329460in}{0.284240in}}{\pgfqpoint{1.989680in}{1.989680in}}%
\pgfusepath{clip}%
\pgfsetbuttcap%
\pgfsetroundjoin%
\definecolor{currentfill}{rgb}{0.699415,0.867117,0.175971}%
\pgfsetfillcolor{currentfill}%
\pgfsetlinewidth{0.000000pt}%
\definecolor{currentstroke}{rgb}{0.000000,0.000000,0.000000}%
\pgfsetstrokecolor{currentstroke}%
\pgfsetdash{}{0pt}%
\pgfpathmoveto{\pgfqpoint{1.476477in}{1.660945in}}%
\pgfpathlineto{\pgfqpoint{1.479927in}{1.656983in}}%
\pgfpathlineto{\pgfqpoint{1.483374in}{1.652925in}}%
\pgfpathlineto{\pgfqpoint{1.486818in}{1.648770in}}%
\pgfpathlineto{\pgfqpoint{1.490261in}{1.644522in}}%
\pgfpathlineto{\pgfqpoint{1.490615in}{1.642443in}}%
\pgfpathlineto{\pgfqpoint{1.490831in}{1.640359in}}%
\pgfpathlineto{\pgfqpoint{1.490908in}{1.638272in}}%
\pgfpathlineto{\pgfqpoint{1.490845in}{1.636184in}}%
\pgfpathlineto{\pgfqpoint{1.487383in}{1.640645in}}%
\pgfpathlineto{\pgfqpoint{1.483920in}{1.645012in}}%
\pgfpathlineto{\pgfqpoint{1.480454in}{1.649282in}}%
\pgfpathlineto{\pgfqpoint{1.476986in}{1.653456in}}%
\pgfpathlineto{\pgfqpoint{1.477047in}{1.655331in}}%
\pgfpathlineto{\pgfqpoint{1.476982in}{1.657206in}}%
\pgfpathlineto{\pgfqpoint{1.476792in}{1.659078in}}%
\pgfpathlineto{\pgfqpoint{1.476477in}{1.660945in}}%
\pgfpathclose%
\pgfusepath{fill}%
\end{pgfscope}%
\begin{pgfscope}%
\pgfpathrectangle{\pgfqpoint{0.329460in}{0.284240in}}{\pgfqpoint{1.989680in}{1.989680in}}%
\pgfusepath{clip}%
\pgfsetbuttcap%
\pgfsetroundjoin%
\definecolor{currentfill}{rgb}{0.896320,0.893616,0.096335}%
\pgfsetfillcolor{currentfill}%
\pgfsetlinewidth{0.000000pt}%
\definecolor{currentstroke}{rgb}{0.000000,0.000000,0.000000}%
\pgfsetstrokecolor{currentstroke}%
\pgfsetdash{}{0pt}%
\pgfpathmoveto{\pgfqpoint{1.409251in}{1.725693in}}%
\pgfpathlineto{\pgfqpoint{1.412137in}{1.724195in}}%
\pgfpathlineto{\pgfqpoint{1.415021in}{1.722584in}}%
\pgfpathlineto{\pgfqpoint{1.417903in}{1.720862in}}%
\pgfpathlineto{\pgfqpoint{1.420783in}{1.719030in}}%
\pgfpathlineto{\pgfqpoint{1.422232in}{1.717994in}}%
\pgfpathlineto{\pgfqpoint{1.423612in}{1.716937in}}%
\pgfpathlineto{\pgfqpoint{1.424921in}{1.715859in}}%
\pgfpathlineto{\pgfqpoint{1.426156in}{1.714763in}}%
\pgfpathlineto{\pgfqpoint{1.423052in}{1.716773in}}%
\pgfpathlineto{\pgfqpoint{1.419945in}{1.718672in}}%
\pgfpathlineto{\pgfqpoint{1.416837in}{1.720460in}}%
\pgfpathlineto{\pgfqpoint{1.413727in}{1.722137in}}%
\pgfpathlineto{\pgfqpoint{1.412698in}{1.723050in}}%
\pgfpathlineto{\pgfqpoint{1.411608in}{1.723948in}}%
\pgfpathlineto{\pgfqpoint{1.410459in}{1.724829in}}%
\pgfpathlineto{\pgfqpoint{1.409251in}{1.725693in}}%
\pgfpathclose%
\pgfusepath{fill}%
\end{pgfscope}%
\begin{pgfscope}%
\pgfpathrectangle{\pgfqpoint{0.329460in}{0.284240in}}{\pgfqpoint{1.989680in}{1.989680in}}%
\pgfusepath{clip}%
\pgfsetbuttcap%
\pgfsetroundjoin%
\definecolor{currentfill}{rgb}{0.935904,0.898570,0.108131}%
\pgfsetfillcolor{currentfill}%
\pgfsetlinewidth{0.000000pt}%
\definecolor{currentstroke}{rgb}{0.000000,0.000000,0.000000}%
\pgfsetstrokecolor{currentstroke}%
\pgfsetdash{}{0pt}%
\pgfpathmoveto{\pgfqpoint{1.388396in}{1.735502in}}%
\pgfpathlineto{\pgfqpoint{1.390709in}{1.734767in}}%
\pgfpathlineto{\pgfqpoint{1.393020in}{1.733917in}}%
\pgfpathlineto{\pgfqpoint{1.395330in}{1.732953in}}%
\pgfpathlineto{\pgfqpoint{1.397638in}{1.731876in}}%
\pgfpathlineto{\pgfqpoint{1.399266in}{1.731181in}}%
\pgfpathlineto{\pgfqpoint{1.400846in}{1.730462in}}%
\pgfpathlineto{\pgfqpoint{1.402378in}{1.729720in}}%
\pgfpathlineto{\pgfqpoint{1.403859in}{1.728956in}}%
\pgfpathlineto{\pgfqpoint{1.401241in}{1.730179in}}%
\pgfpathlineto{\pgfqpoint{1.398621in}{1.731289in}}%
\pgfpathlineto{\pgfqpoint{1.395999in}{1.732284in}}%
\pgfpathlineto{\pgfqpoint{1.393375in}{1.733165in}}%
\pgfpathlineto{\pgfqpoint{1.392190in}{1.733776in}}%
\pgfpathlineto{\pgfqpoint{1.390964in}{1.734370in}}%
\pgfpathlineto{\pgfqpoint{1.389699in}{1.734945in}}%
\pgfpathlineto{\pgfqpoint{1.388396in}{1.735502in}}%
\pgfpathclose%
\pgfusepath{fill}%
\end{pgfscope}%
\begin{pgfscope}%
\pgfpathrectangle{\pgfqpoint{0.329460in}{0.284240in}}{\pgfqpoint{1.989680in}{1.989680in}}%
\pgfusepath{clip}%
\pgfsetbuttcap%
\pgfsetroundjoin%
\definecolor{currentfill}{rgb}{0.281477,0.755203,0.432552}%
\pgfsetfillcolor{currentfill}%
\pgfsetlinewidth{0.000000pt}%
\definecolor{currentstroke}{rgb}{0.000000,0.000000,0.000000}%
\pgfsetstrokecolor{currentstroke}%
\pgfsetdash{}{0pt}%
\pgfpathmoveto{\pgfqpoint{1.161992in}{1.489641in}}%
\pgfpathlineto{\pgfqpoint{1.158877in}{1.482844in}}%
\pgfpathlineto{\pgfqpoint{1.155764in}{1.475987in}}%
\pgfpathlineto{\pgfqpoint{1.152653in}{1.469070in}}%
\pgfpathlineto{\pgfqpoint{1.149544in}{1.462097in}}%
\pgfpathlineto{\pgfqpoint{1.146690in}{1.465221in}}%
\pgfpathlineto{\pgfqpoint{1.144044in}{1.468387in}}%
\pgfpathlineto{\pgfqpoint{1.141608in}{1.471590in}}%
\pgfpathlineto{\pgfqpoint{1.139385in}{1.474826in}}%
\pgfpathlineto{\pgfqpoint{1.142644in}{1.481590in}}%
\pgfpathlineto{\pgfqpoint{1.145905in}{1.488296in}}%
\pgfpathlineto{\pgfqpoint{1.149168in}{1.494944in}}%
\pgfpathlineto{\pgfqpoint{1.152434in}{1.501532in}}%
\pgfpathlineto{\pgfqpoint{1.154526in}{1.498508in}}%
\pgfpathlineto{\pgfqpoint{1.156818in}{1.495516in}}%
\pgfpathlineto{\pgfqpoint{1.159308in}{1.492559in}}%
\pgfpathlineto{\pgfqpoint{1.161992in}{1.489641in}}%
\pgfpathclose%
\pgfusepath{fill}%
\end{pgfscope}%
\begin{pgfscope}%
\pgfpathrectangle{\pgfqpoint{0.329460in}{0.284240in}}{\pgfqpoint{1.989680in}{1.989680in}}%
\pgfusepath{clip}%
\pgfsetbuttcap%
\pgfsetroundjoin%
\definecolor{currentfill}{rgb}{0.955300,0.901065,0.118128}%
\pgfsetfillcolor{currentfill}%
\pgfsetlinewidth{0.000000pt}%
\definecolor{currentstroke}{rgb}{0.000000,0.000000,0.000000}%
\pgfsetstrokecolor{currentstroke}%
\pgfsetdash{}{0pt}%
\pgfpathmoveto{\pgfqpoint{1.345976in}{1.742080in}}%
\pgfpathlineto{\pgfqpoint{1.345543in}{1.742208in}}%
\pgfpathlineto{\pgfqpoint{1.345111in}{1.742218in}}%
\pgfpathlineto{\pgfqpoint{1.344680in}{1.742112in}}%
\pgfpathlineto{\pgfqpoint{1.344249in}{1.741890in}}%
\pgfpathlineto{\pgfqpoint{1.346002in}{1.741979in}}%
\pgfpathlineto{\pgfqpoint{1.347760in}{1.742042in}}%
\pgfpathlineto{\pgfqpoint{1.349522in}{1.742080in}}%
\pgfpathlineto{\pgfqpoint{1.351286in}{1.742091in}}%
\pgfpathlineto{\pgfqpoint{1.351279in}{1.742301in}}%
\pgfpathlineto{\pgfqpoint{1.351273in}{1.742394in}}%
\pgfpathlineto{\pgfqpoint{1.351267in}{1.742371in}}%
\pgfpathlineto{\pgfqpoint{1.351261in}{1.742231in}}%
\pgfpathlineto{\pgfqpoint{1.349937in}{1.742223in}}%
\pgfpathlineto{\pgfqpoint{1.348613in}{1.742195in}}%
\pgfpathlineto{\pgfqpoint{1.347293in}{1.742147in}}%
\pgfpathlineto{\pgfqpoint{1.345976in}{1.742080in}}%
\pgfpathclose%
\pgfusepath{fill}%
\end{pgfscope}%
\begin{pgfscope}%
\pgfpathrectangle{\pgfqpoint{0.329460in}{0.284240in}}{\pgfqpoint{1.989680in}{1.989680in}}%
\pgfusepath{clip}%
\pgfsetbuttcap%
\pgfsetroundjoin%
\definecolor{currentfill}{rgb}{0.955300,0.901065,0.118128}%
\pgfsetfillcolor{currentfill}%
\pgfsetlinewidth{0.000000pt}%
\definecolor{currentstroke}{rgb}{0.000000,0.000000,0.000000}%
\pgfsetstrokecolor{currentstroke}%
\pgfsetdash{}{0pt}%
\pgfpathmoveto{\pgfqpoint{1.351261in}{1.742231in}}%
\pgfpathlineto{\pgfqpoint{1.351267in}{1.742371in}}%
\pgfpathlineto{\pgfqpoint{1.351273in}{1.742394in}}%
\pgfpathlineto{\pgfqpoint{1.351279in}{1.742301in}}%
\pgfpathlineto{\pgfqpoint{1.351286in}{1.742091in}}%
\pgfpathlineto{\pgfqpoint{1.353049in}{1.742077in}}%
\pgfpathlineto{\pgfqpoint{1.354810in}{1.742036in}}%
\pgfpathlineto{\pgfqpoint{1.356568in}{1.741970in}}%
\pgfpathlineto{\pgfqpoint{1.358321in}{1.741878in}}%
\pgfpathlineto{\pgfqpoint{1.357878in}{1.742101in}}%
\pgfpathlineto{\pgfqpoint{1.357434in}{1.742208in}}%
\pgfpathlineto{\pgfqpoint{1.356990in}{1.742198in}}%
\pgfpathlineto{\pgfqpoint{1.356546in}{1.742072in}}%
\pgfpathlineto{\pgfqpoint{1.355229in}{1.742141in}}%
\pgfpathlineto{\pgfqpoint{1.353909in}{1.742190in}}%
\pgfpathlineto{\pgfqpoint{1.352586in}{1.742220in}}%
\pgfpathlineto{\pgfqpoint{1.351261in}{1.742231in}}%
\pgfpathclose%
\pgfusepath{fill}%
\end{pgfscope}%
\begin{pgfscope}%
\pgfpathrectangle{\pgfqpoint{0.329460in}{0.284240in}}{\pgfqpoint{1.989680in}{1.989680in}}%
\pgfusepath{clip}%
\pgfsetbuttcap%
\pgfsetroundjoin%
\definecolor{currentfill}{rgb}{0.955300,0.901065,0.118128}%
\pgfsetfillcolor{currentfill}%
\pgfsetlinewidth{0.000000pt}%
\definecolor{currentstroke}{rgb}{0.000000,0.000000,0.000000}%
\pgfsetstrokecolor{currentstroke}%
\pgfsetdash{}{0pt}%
\pgfpathmoveto{\pgfqpoint{1.340772in}{1.741621in}}%
\pgfpathlineto{\pgfqpoint{1.339908in}{1.741710in}}%
\pgfpathlineto{\pgfqpoint{1.339045in}{1.741683in}}%
\pgfpathlineto{\pgfqpoint{1.338182in}{1.741538in}}%
\pgfpathlineto{\pgfqpoint{1.337321in}{1.741278in}}%
\pgfpathlineto{\pgfqpoint{1.339036in}{1.741469in}}%
\pgfpathlineto{\pgfqpoint{1.340764in}{1.741635in}}%
\pgfpathlineto{\pgfqpoint{1.342502in}{1.741775in}}%
\pgfpathlineto{\pgfqpoint{1.344249in}{1.741890in}}%
\pgfpathlineto{\pgfqpoint{1.344680in}{1.742112in}}%
\pgfpathlineto{\pgfqpoint{1.345111in}{1.742218in}}%
\pgfpathlineto{\pgfqpoint{1.345543in}{1.742208in}}%
\pgfpathlineto{\pgfqpoint{1.345976in}{1.742080in}}%
\pgfpathlineto{\pgfqpoint{1.344664in}{1.741994in}}%
\pgfpathlineto{\pgfqpoint{1.343359in}{1.741889in}}%
\pgfpathlineto{\pgfqpoint{1.342061in}{1.741764in}}%
\pgfpathlineto{\pgfqpoint{1.340772in}{1.741621in}}%
\pgfpathclose%
\pgfusepath{fill}%
\end{pgfscope}%
\begin{pgfscope}%
\pgfpathrectangle{\pgfqpoint{0.329460in}{0.284240in}}{\pgfqpoint{1.989680in}{1.989680in}}%
\pgfusepath{clip}%
\pgfsetbuttcap%
\pgfsetroundjoin%
\definecolor{currentfill}{rgb}{0.955300,0.901065,0.118128}%
\pgfsetfillcolor{currentfill}%
\pgfsetlinewidth{0.000000pt}%
\definecolor{currentstroke}{rgb}{0.000000,0.000000,0.000000}%
\pgfsetstrokecolor{currentstroke}%
\pgfsetdash{}{0pt}%
\pgfpathmoveto{\pgfqpoint{1.356546in}{1.742072in}}%
\pgfpathlineto{\pgfqpoint{1.356990in}{1.742198in}}%
\pgfpathlineto{\pgfqpoint{1.357434in}{1.742208in}}%
\pgfpathlineto{\pgfqpoint{1.357878in}{1.742101in}}%
\pgfpathlineto{\pgfqpoint{1.358321in}{1.741878in}}%
\pgfpathlineto{\pgfqpoint{1.360067in}{1.741761in}}%
\pgfpathlineto{\pgfqpoint{1.361804in}{1.741618in}}%
\pgfpathlineto{\pgfqpoint{1.363530in}{1.741449in}}%
\pgfpathlineto{\pgfqpoint{1.365244in}{1.741256in}}%
\pgfpathlineto{\pgfqpoint{1.364371in}{1.741517in}}%
\pgfpathlineto{\pgfqpoint{1.363496in}{1.741663in}}%
\pgfpathlineto{\pgfqpoint{1.362621in}{1.741692in}}%
\pgfpathlineto{\pgfqpoint{1.361745in}{1.741604in}}%
\pgfpathlineto{\pgfqpoint{1.360458in}{1.741749in}}%
\pgfpathlineto{\pgfqpoint{1.359161in}{1.741876in}}%
\pgfpathlineto{\pgfqpoint{1.357857in}{1.741983in}}%
\pgfpathlineto{\pgfqpoint{1.356546in}{1.742072in}}%
\pgfpathclose%
\pgfusepath{fill}%
\end{pgfscope}%
\begin{pgfscope}%
\pgfpathrectangle{\pgfqpoint{0.329460in}{0.284240in}}{\pgfqpoint{1.989680in}{1.989680in}}%
\pgfusepath{clip}%
\pgfsetbuttcap%
\pgfsetroundjoin%
\definecolor{currentfill}{rgb}{0.935904,0.898570,0.108131}%
\pgfsetfillcolor{currentfill}%
\pgfsetlinewidth{0.000000pt}%
\definecolor{currentstroke}{rgb}{0.000000,0.000000,0.000000}%
\pgfsetstrokecolor{currentstroke}%
\pgfsetdash{}{0pt}%
\pgfpathmoveto{\pgfqpoint{1.307982in}{1.732607in}}%
\pgfpathlineto{\pgfqpoint{1.305294in}{1.731691in}}%
\pgfpathlineto{\pgfqpoint{1.302608in}{1.730661in}}%
\pgfpathlineto{\pgfqpoint{1.299925in}{1.729517in}}%
\pgfpathlineto{\pgfqpoint{1.297243in}{1.728259in}}%
\pgfpathlineto{\pgfqpoint{1.298678in}{1.729042in}}%
\pgfpathlineto{\pgfqpoint{1.300165in}{1.729804in}}%
\pgfpathlineto{\pgfqpoint{1.301702in}{1.730543in}}%
\pgfpathlineto{\pgfqpoint{1.303288in}{1.731259in}}%
\pgfpathlineto{\pgfqpoint{1.305668in}{1.732367in}}%
\pgfpathlineto{\pgfqpoint{1.308050in}{1.733362in}}%
\pgfpathlineto{\pgfqpoint{1.310434in}{1.734242in}}%
\pgfpathlineto{\pgfqpoint{1.312819in}{1.735008in}}%
\pgfpathlineto{\pgfqpoint{1.311550in}{1.734435in}}%
\pgfpathlineto{\pgfqpoint{1.310320in}{1.733843in}}%
\pgfpathlineto{\pgfqpoint{1.309130in}{1.733234in}}%
\pgfpathlineto{\pgfqpoint{1.307982in}{1.732607in}}%
\pgfpathclose%
\pgfusepath{fill}%
\end{pgfscope}%
\begin{pgfscope}%
\pgfpathrectangle{\pgfqpoint{0.329460in}{0.284240in}}{\pgfqpoint{1.989680in}{1.989680in}}%
\pgfusepath{clip}%
\pgfsetbuttcap%
\pgfsetroundjoin%
\definecolor{currentfill}{rgb}{0.896320,0.893616,0.096335}%
\pgfsetfillcolor{currentfill}%
\pgfsetlinewidth{0.000000pt}%
\definecolor{currentstroke}{rgb}{0.000000,0.000000,0.000000}%
\pgfsetstrokecolor{currentstroke}%
\pgfsetdash{}{0pt}%
\pgfpathmoveto{\pgfqpoint{1.287786in}{1.721312in}}%
\pgfpathlineto{\pgfqpoint{1.284632in}{1.719595in}}%
\pgfpathlineto{\pgfqpoint{1.281480in}{1.717765in}}%
\pgfpathlineto{\pgfqpoint{1.278331in}{1.715824in}}%
\pgfpathlineto{\pgfqpoint{1.275183in}{1.713773in}}%
\pgfpathlineto{\pgfqpoint{1.276353in}{1.714885in}}%
\pgfpathlineto{\pgfqpoint{1.277596in}{1.715980in}}%
\pgfpathlineto{\pgfqpoint{1.278913in}{1.717055in}}%
\pgfpathlineto{\pgfqpoint{1.280300in}{1.718110in}}%
\pgfpathlineto{\pgfqpoint{1.283234in}{1.719981in}}%
\pgfpathlineto{\pgfqpoint{1.286170in}{1.721741in}}%
\pgfpathlineto{\pgfqpoint{1.289108in}{1.723389in}}%
\pgfpathlineto{\pgfqpoint{1.292048in}{1.724926in}}%
\pgfpathlineto{\pgfqpoint{1.290892in}{1.724047in}}%
\pgfpathlineto{\pgfqpoint{1.289795in}{1.723151in}}%
\pgfpathlineto{\pgfqpoint{1.288760in}{1.722239in}}%
\pgfpathlineto{\pgfqpoint{1.287786in}{1.721312in}}%
\pgfpathclose%
\pgfusepath{fill}%
\end{pgfscope}%
\begin{pgfscope}%
\pgfpathrectangle{\pgfqpoint{0.329460in}{0.284240in}}{\pgfqpoint{1.989680in}{1.989680in}}%
\pgfusepath{clip}%
\pgfsetbuttcap%
\pgfsetroundjoin%
\definecolor{currentfill}{rgb}{0.122606,0.585371,0.546557}%
\pgfsetfillcolor{currentfill}%
\pgfsetlinewidth{0.000000pt}%
\definecolor{currentstroke}{rgb}{0.000000,0.000000,0.000000}%
\pgfsetstrokecolor{currentstroke}%
\pgfsetdash{}{0pt}%
\pgfpathmoveto{\pgfqpoint{1.137192in}{1.315037in}}%
\pgfpathlineto{\pgfqpoint{1.134558in}{1.307018in}}%
\pgfpathlineto{\pgfqpoint{1.131925in}{1.298979in}}%
\pgfpathlineto{\pgfqpoint{1.129293in}{1.290925in}}%
\pgfpathlineto{\pgfqpoint{1.126663in}{1.282856in}}%
\pgfpathlineto{\pgfqpoint{1.120929in}{1.286458in}}%
\pgfpathlineto{\pgfqpoint{1.115434in}{1.290148in}}%
\pgfpathlineto{\pgfqpoint{1.110184in}{1.293920in}}%
\pgfpathlineto{\pgfqpoint{1.105183in}{1.297770in}}%
\pgfpathlineto{\pgfqpoint{1.108059in}{1.305646in}}%
\pgfpathlineto{\pgfqpoint{1.110936in}{1.313507in}}%
\pgfpathlineto{\pgfqpoint{1.113814in}{1.321353in}}%
\pgfpathlineto{\pgfqpoint{1.116694in}{1.329181in}}%
\pgfpathlineto{\pgfqpoint{1.121467in}{1.325529in}}%
\pgfpathlineto{\pgfqpoint{1.126477in}{1.321952in}}%
\pgfpathlineto{\pgfqpoint{1.131721in}{1.318453in}}%
\pgfpathlineto{\pgfqpoint{1.137192in}{1.315037in}}%
\pgfpathclose%
\pgfusepath{fill}%
\end{pgfscope}%
\begin{pgfscope}%
\pgfpathrectangle{\pgfqpoint{0.329460in}{0.284240in}}{\pgfqpoint{1.989680in}{1.989680in}}%
\pgfusepath{clip}%
\pgfsetbuttcap%
\pgfsetroundjoin%
\definecolor{currentfill}{rgb}{0.282884,0.135920,0.453427}%
\pgfsetfillcolor{currentfill}%
\pgfsetlinewidth{0.000000pt}%
\definecolor{currentstroke}{rgb}{0.000000,0.000000,0.000000}%
\pgfsetstrokecolor{currentstroke}%
\pgfsetdash{}{0pt}%
\pgfpathmoveto{\pgfqpoint{0.880424in}{0.880733in}}%
\pgfpathlineto{\pgfqpoint{0.877512in}{0.885357in}}%
\pgfpathlineto{\pgfqpoint{0.874588in}{0.890316in}}%
\pgfpathlineto{\pgfqpoint{0.871652in}{0.895617in}}%
\pgfpathlineto{\pgfqpoint{0.868704in}{0.901264in}}%
\pgfpathlineto{\pgfqpoint{0.856710in}{0.909474in}}%
\pgfpathlineto{\pgfqpoint{0.845254in}{0.917870in}}%
\pgfpathlineto{\pgfqpoint{0.834346in}{0.926443in}}%
\pgfpathlineto{\pgfqpoint{0.823996in}{0.935183in}}%
\pgfpathlineto{\pgfqpoint{0.827201in}{0.929358in}}%
\pgfpathlineto{\pgfqpoint{0.830393in}{0.923879in}}%
\pgfpathlineto{\pgfqpoint{0.833572in}{0.918740in}}%
\pgfpathlineto{\pgfqpoint{0.836739in}{0.913935in}}%
\pgfpathlineto{\pgfqpoint{0.846854in}{0.905380in}}%
\pgfpathlineto{\pgfqpoint{0.857514in}{0.896988in}}%
\pgfpathlineto{\pgfqpoint{0.868707in}{0.888769in}}%
\pgfpathlineto{\pgfqpoint{0.880424in}{0.880733in}}%
\pgfpathclose%
\pgfusepath{fill}%
\end{pgfscope}%
\begin{pgfscope}%
\pgfpathrectangle{\pgfqpoint{0.329460in}{0.284240in}}{\pgfqpoint{1.989680in}{1.989680in}}%
\pgfusepath{clip}%
\pgfsetbuttcap%
\pgfsetroundjoin%
\definecolor{currentfill}{rgb}{0.212395,0.359683,0.551710}%
\pgfsetfillcolor{currentfill}%
\pgfsetlinewidth{0.000000pt}%
\definecolor{currentstroke}{rgb}{0.000000,0.000000,0.000000}%
\pgfsetstrokecolor{currentstroke}%
\pgfsetdash{}{0pt}%
\pgfpathmoveto{\pgfqpoint{1.604402in}{1.109082in}}%
\pgfpathlineto{\pgfqpoint{1.606797in}{1.101179in}}%
\pgfpathlineto{\pgfqpoint{1.609191in}{1.093321in}}%
\pgfpathlineto{\pgfqpoint{1.611585in}{1.085512in}}%
\pgfpathlineto{\pgfqpoint{1.613980in}{1.077753in}}%
\pgfpathlineto{\pgfqpoint{1.605491in}{1.073519in}}%
\pgfpathlineto{\pgfqpoint{1.596732in}{1.069424in}}%
\pgfpathlineto{\pgfqpoint{1.587712in}{1.065472in}}%
\pgfpathlineto{\pgfqpoint{1.578439in}{1.061668in}}%
\pgfpathlineto{\pgfqpoint{1.576364in}{1.069594in}}%
\pgfpathlineto{\pgfqpoint{1.574288in}{1.077569in}}%
\pgfpathlineto{\pgfqpoint{1.572213in}{1.085593in}}%
\pgfpathlineto{\pgfqpoint{1.570137in}{1.093661in}}%
\pgfpathlineto{\pgfqpoint{1.579076in}{1.097308in}}%
\pgfpathlineto{\pgfqpoint{1.587772in}{1.101097in}}%
\pgfpathlineto{\pgfqpoint{1.596217in}{1.105023in}}%
\pgfpathlineto{\pgfqpoint{1.604402in}{1.109082in}}%
\pgfpathclose%
\pgfusepath{fill}%
\end{pgfscope}%
\begin{pgfscope}%
\pgfpathrectangle{\pgfqpoint{0.329460in}{0.284240in}}{\pgfqpoint{1.989680in}{1.989680in}}%
\pgfusepath{clip}%
\pgfsetbuttcap%
\pgfsetroundjoin%
\definecolor{currentfill}{rgb}{0.231674,0.318106,0.544834}%
\pgfsetfillcolor{currentfill}%
\pgfsetlinewidth{0.000000pt}%
\definecolor{currentstroke}{rgb}{0.000000,0.000000,0.000000}%
\pgfsetstrokecolor{currentstroke}%
\pgfsetdash{}{0pt}%
\pgfpathmoveto{\pgfqpoint{1.132383in}{1.058415in}}%
\pgfpathlineto{\pgfqpoint{1.130383in}{1.050509in}}%
\pgfpathlineto{\pgfqpoint{1.128383in}{1.042661in}}%
\pgfpathlineto{\pgfqpoint{1.126383in}{1.034872in}}%
\pgfpathlineto{\pgfqpoint{1.124383in}{1.027146in}}%
\pgfpathlineto{\pgfqpoint{1.114552in}{1.030967in}}%
\pgfpathlineto{\pgfqpoint{1.104974in}{1.034946in}}%
\pgfpathlineto{\pgfqpoint{1.095660in}{1.039079in}}%
\pgfpathlineto{\pgfqpoint{1.086618in}{1.043360in}}%
\pgfpathlineto{\pgfqpoint{1.088946in}{1.050924in}}%
\pgfpathlineto{\pgfqpoint{1.091273in}{1.058551in}}%
\pgfpathlineto{\pgfqpoint{1.093600in}{1.066239in}}%
\pgfpathlineto{\pgfqpoint{1.095927in}{1.073983in}}%
\pgfpathlineto{\pgfqpoint{1.104657in}{1.069872in}}%
\pgfpathlineto{\pgfqpoint{1.113648in}{1.065904in}}%
\pgfpathlineto{\pgfqpoint{1.122893in}{1.062084in}}%
\pgfpathlineto{\pgfqpoint{1.132383in}{1.058415in}}%
\pgfpathclose%
\pgfusepath{fill}%
\end{pgfscope}%
\begin{pgfscope}%
\pgfpathrectangle{\pgfqpoint{0.329460in}{0.284240in}}{\pgfqpoint{1.989680in}{1.989680in}}%
\pgfusepath{clip}%
\pgfsetbuttcap%
\pgfsetroundjoin%
\definecolor{currentfill}{rgb}{0.955300,0.901065,0.118128}%
\pgfsetfillcolor{currentfill}%
\pgfsetlinewidth{0.000000pt}%
\definecolor{currentstroke}{rgb}{0.000000,0.000000,0.000000}%
\pgfsetstrokecolor{currentstroke}%
\pgfsetdash{}{0pt}%
\pgfpathmoveto{\pgfqpoint{1.361745in}{1.741604in}}%
\pgfpathlineto{\pgfqpoint{1.362621in}{1.741692in}}%
\pgfpathlineto{\pgfqpoint{1.363496in}{1.741663in}}%
\pgfpathlineto{\pgfqpoint{1.364371in}{1.741517in}}%
\pgfpathlineto{\pgfqpoint{1.365244in}{1.741256in}}%
\pgfpathlineto{\pgfqpoint{1.366945in}{1.741037in}}%
\pgfpathlineto{\pgfqpoint{1.368630in}{1.740793in}}%
\pgfpathlineto{\pgfqpoint{1.370298in}{1.740525in}}%
\pgfpathlineto{\pgfqpoint{1.369110in}{1.740832in}}%
\pgfpathlineto{\pgfqpoint{1.367921in}{1.741023in}}%
\pgfpathlineto{\pgfqpoint{1.366731in}{1.741097in}}%
\pgfpathlineto{\pgfqpoint{1.365540in}{1.741055in}}%
\pgfpathlineto{\pgfqpoint{1.364288in}{1.741256in}}%
\pgfpathlineto{\pgfqpoint{1.363022in}{1.741439in}}%
\pgfpathlineto{\pgfqpoint{1.361745in}{1.741604in}}%
\pgfpathclose%
\pgfusepath{fill}%
\end{pgfscope}%
\begin{pgfscope}%
\pgfpathrectangle{\pgfqpoint{0.329460in}{0.284240in}}{\pgfqpoint{1.989680in}{1.989680in}}%
\pgfusepath{clip}%
\pgfsetbuttcap%
\pgfsetroundjoin%
\definecolor{currentfill}{rgb}{0.163625,0.471133,0.558148}%
\pgfsetfillcolor{currentfill}%
\pgfsetlinewidth{0.000000pt}%
\definecolor{currentstroke}{rgb}{0.000000,0.000000,0.000000}%
\pgfsetstrokecolor{currentstroke}%
\pgfsetdash{}{0pt}%
\pgfpathmoveto{\pgfqpoint{1.133190in}{1.203227in}}%
\pgfpathlineto{\pgfqpoint{1.130858in}{1.194968in}}%
\pgfpathlineto{\pgfqpoint{1.128526in}{1.186720in}}%
\pgfpathlineto{\pgfqpoint{1.126195in}{1.178487in}}%
\pgfpathlineto{\pgfqpoint{1.123864in}{1.170270in}}%
\pgfpathlineto{\pgfqpoint{1.116315in}{1.173997in}}%
\pgfpathlineto{\pgfqpoint{1.109013in}{1.177842in}}%
\pgfpathlineto{\pgfqpoint{1.101966in}{1.181800in}}%
\pgfpathlineto{\pgfqpoint{1.095181in}{1.185867in}}%
\pgfpathlineto{\pgfqpoint{1.097799in}{1.193904in}}%
\pgfpathlineto{\pgfqpoint{1.100418in}{1.201958in}}%
\pgfpathlineto{\pgfqpoint{1.103038in}{1.210027in}}%
\pgfpathlineto{\pgfqpoint{1.105659in}{1.218107in}}%
\pgfpathlineto{\pgfqpoint{1.112173in}{1.214227in}}%
\pgfpathlineto{\pgfqpoint{1.118937in}{1.210451in}}%
\pgfpathlineto{\pgfqpoint{1.125945in}{1.206783in}}%
\pgfpathlineto{\pgfqpoint{1.133190in}{1.203227in}}%
\pgfpathclose%
\pgfusepath{fill}%
\end{pgfscope}%
\begin{pgfscope}%
\pgfpathrectangle{\pgfqpoint{0.329460in}{0.284240in}}{\pgfqpoint{1.989680in}{1.989680in}}%
\pgfusepath{clip}%
\pgfsetbuttcap%
\pgfsetroundjoin%
\definecolor{currentfill}{rgb}{0.955300,0.901065,0.118128}%
\pgfsetfillcolor{currentfill}%
\pgfsetlinewidth{0.000000pt}%
\definecolor{currentstroke}{rgb}{0.000000,0.000000,0.000000}%
\pgfsetstrokecolor{currentstroke}%
\pgfsetdash{}{0pt}%
\pgfpathmoveto{\pgfqpoint{1.335733in}{1.740860in}}%
\pgfpathlineto{\pgfqpoint{1.334451in}{1.740887in}}%
\pgfpathlineto{\pgfqpoint{1.333170in}{1.740796in}}%
\pgfpathlineto{\pgfqpoint{1.331890in}{1.740589in}}%
\pgfpathlineto{\pgfqpoint{1.330611in}{1.740266in}}%
\pgfpathlineto{\pgfqpoint{1.332262in}{1.740556in}}%
\pgfpathlineto{\pgfqpoint{1.333932in}{1.740821in}}%
\pgfpathlineto{\pgfqpoint{1.335618in}{1.741062in}}%
\pgfpathlineto{\pgfqpoint{1.337321in}{1.741278in}}%
\pgfpathlineto{\pgfqpoint{1.338182in}{1.741538in}}%
\pgfpathlineto{\pgfqpoint{1.339045in}{1.741683in}}%
\pgfpathlineto{\pgfqpoint{1.339908in}{1.741710in}}%
\pgfpathlineto{\pgfqpoint{1.340772in}{1.741621in}}%
\pgfpathlineto{\pgfqpoint{1.339494in}{1.741459in}}%
\pgfpathlineto{\pgfqpoint{1.338227in}{1.741278in}}%
\pgfpathlineto{\pgfqpoint{1.336973in}{1.741078in}}%
\pgfpathlineto{\pgfqpoint{1.335733in}{1.740860in}}%
\pgfpathclose%
\pgfusepath{fill}%
\end{pgfscope}%
\begin{pgfscope}%
\pgfpathrectangle{\pgfqpoint{0.329460in}{0.284240in}}{\pgfqpoint{1.989680in}{1.989680in}}%
\pgfusepath{clip}%
\pgfsetbuttcap%
\pgfsetroundjoin%
\definecolor{currentfill}{rgb}{0.277941,0.056324,0.381191}%
\pgfsetfillcolor{currentfill}%
\pgfsetlinewidth{0.000000pt}%
\definecolor{currentstroke}{rgb}{0.000000,0.000000,0.000000}%
\pgfsetstrokecolor{currentstroke}%
\pgfsetdash{}{0pt}%
\pgfpathmoveto{\pgfqpoint{1.808989in}{0.861951in}}%
\pgfpathlineto{\pgfqpoint{1.811879in}{0.864134in}}%
\pgfpathlineto{\pgfqpoint{1.814779in}{0.866608in}}%
\pgfpathlineto{\pgfqpoint{1.817689in}{0.869379in}}%
\pgfpathlineto{\pgfqpoint{1.820608in}{0.872452in}}%
\pgfpathlineto{\pgfqpoint{1.809108in}{0.864615in}}%
\pgfpathlineto{\pgfqpoint{1.797110in}{0.856969in}}%
\pgfpathlineto{\pgfqpoint{1.784625in}{0.849521in}}%
\pgfpathlineto{\pgfqpoint{1.771666in}{0.842280in}}%
\pgfpathlineto{\pgfqpoint{1.769039in}{0.839383in}}%
\pgfpathlineto{\pgfqpoint{1.766421in}{0.836787in}}%
\pgfpathlineto{\pgfqpoint{1.763812in}{0.834489in}}%
\pgfpathlineto{\pgfqpoint{1.761212in}{0.832483in}}%
\pgfpathlineto{\pgfqpoint{1.773861in}{0.839554in}}%
\pgfpathlineto{\pgfqpoint{1.786047in}{0.846828in}}%
\pgfpathlineto{\pgfqpoint{1.797761in}{0.854296in}}%
\pgfpathlineto{\pgfqpoint{1.808989in}{0.861951in}}%
\pgfpathclose%
\pgfusepath{fill}%
\end{pgfscope}%
\begin{pgfscope}%
\pgfpathrectangle{\pgfqpoint{0.329460in}{0.284240in}}{\pgfqpoint{1.989680in}{1.989680in}}%
\pgfusepath{clip}%
\pgfsetbuttcap%
\pgfsetroundjoin%
\definecolor{currentfill}{rgb}{0.699415,0.867117,0.175971}%
\pgfsetfillcolor{currentfill}%
\pgfsetlinewidth{0.000000pt}%
\definecolor{currentstroke}{rgb}{0.000000,0.000000,0.000000}%
\pgfsetstrokecolor{currentstroke}%
\pgfsetdash{}{0pt}%
\pgfpathmoveto{\pgfqpoint{1.225549in}{1.651790in}}%
\pgfpathlineto{\pgfqpoint{1.222084in}{1.647569in}}%
\pgfpathlineto{\pgfqpoint{1.218622in}{1.643251in}}%
\pgfpathlineto{\pgfqpoint{1.215161in}{1.638838in}}%
\pgfpathlineto{\pgfqpoint{1.211703in}{1.634329in}}%
\pgfpathlineto{\pgfqpoint{1.211516in}{1.636416in}}%
\pgfpathlineto{\pgfqpoint{1.211469in}{1.638504in}}%
\pgfpathlineto{\pgfqpoint{1.211561in}{1.640591in}}%
\pgfpathlineto{\pgfqpoint{1.211793in}{1.642674in}}%
\pgfpathlineto{\pgfqpoint{1.215244in}{1.646970in}}%
\pgfpathlineto{\pgfqpoint{1.218697in}{1.651171in}}%
\pgfpathlineto{\pgfqpoint{1.222153in}{1.655277in}}%
\pgfpathlineto{\pgfqpoint{1.225612in}{1.659285in}}%
\pgfpathlineto{\pgfqpoint{1.225408in}{1.657414in}}%
\pgfpathlineto{\pgfqpoint{1.225329in}{1.655540in}}%
\pgfpathlineto{\pgfqpoint{1.225376in}{1.653664in}}%
\pgfpathlineto{\pgfqpoint{1.225549in}{1.651790in}}%
\pgfpathclose%
\pgfusepath{fill}%
\end{pgfscope}%
\begin{pgfscope}%
\pgfpathrectangle{\pgfqpoint{0.329460in}{0.284240in}}{\pgfqpoint{1.989680in}{1.989680in}}%
\pgfusepath{clip}%
\pgfsetbuttcap%
\pgfsetroundjoin%
\definecolor{currentfill}{rgb}{0.487026,0.823929,0.312321}%
\pgfsetfillcolor{currentfill}%
\pgfsetlinewidth{0.000000pt}%
\definecolor{currentstroke}{rgb}{0.000000,0.000000,0.000000}%
\pgfsetstrokecolor{currentstroke}%
\pgfsetdash{}{0pt}%
\pgfpathmoveto{\pgfqpoint{1.516531in}{1.587223in}}%
\pgfpathlineto{\pgfqpoint{1.519937in}{1.581757in}}%
\pgfpathlineto{\pgfqpoint{1.523341in}{1.576210in}}%
\pgfpathlineto{\pgfqpoint{1.526743in}{1.570585in}}%
\pgfpathlineto{\pgfqpoint{1.530142in}{1.564882in}}%
\pgfpathlineto{\pgfqpoint{1.529179in}{1.562181in}}%
\pgfpathlineto{\pgfqpoint{1.528036in}{1.559495in}}%
\pgfpathlineto{\pgfqpoint{1.526715in}{1.556826in}}%
\pgfpathlineto{\pgfqpoint{1.525218in}{1.554178in}}%
\pgfpathlineto{\pgfqpoint{1.521906in}{1.560094in}}%
\pgfpathlineto{\pgfqpoint{1.518593in}{1.565933in}}%
\pgfpathlineto{\pgfqpoint{1.515277in}{1.571693in}}%
\pgfpathlineto{\pgfqpoint{1.511959in}{1.577371in}}%
\pgfpathlineto{\pgfqpoint{1.513348in}{1.579808in}}%
\pgfpathlineto{\pgfqpoint{1.514574in}{1.582264in}}%
\pgfpathlineto{\pgfqpoint{1.515635in}{1.584737in}}%
\pgfpathlineto{\pgfqpoint{1.516531in}{1.587223in}}%
\pgfpathclose%
\pgfusepath{fill}%
\end{pgfscope}%
\begin{pgfscope}%
\pgfpathrectangle{\pgfqpoint{0.329460in}{0.284240in}}{\pgfqpoint{1.989680in}{1.989680in}}%
\pgfusepath{clip}%
\pgfsetbuttcap%
\pgfsetroundjoin%
\definecolor{currentfill}{rgb}{0.955300,0.901065,0.118128}%
\pgfsetfillcolor{currentfill}%
\pgfsetlinewidth{0.000000pt}%
\definecolor{currentstroke}{rgb}{0.000000,0.000000,0.000000}%
\pgfsetstrokecolor{currentstroke}%
\pgfsetdash{}{0pt}%
\pgfpathmoveto{\pgfqpoint{1.365540in}{1.741055in}}%
\pgfpathlineto{\pgfqpoint{1.366731in}{1.741097in}}%
\pgfpathlineto{\pgfqpoint{1.367921in}{1.741023in}}%
\pgfpathlineto{\pgfqpoint{1.369110in}{1.740832in}}%
\pgfpathlineto{\pgfqpoint{1.370298in}{1.740525in}}%
\pgfpathlineto{\pgfqpoint{1.371947in}{1.740232in}}%
\pgfpathlineto{\pgfqpoint{1.373575in}{1.739915in}}%
\pgfpathlineto{\pgfqpoint{1.375182in}{1.739575in}}%
\pgfpathlineto{\pgfqpoint{1.376765in}{1.739211in}}%
\pgfpathlineto{\pgfqpoint{1.375175in}{1.739599in}}%
\pgfpathlineto{\pgfqpoint{1.373583in}{1.739872in}}%
\pgfpathlineto{\pgfqpoint{1.371990in}{1.740029in}}%
\pgfpathlineto{\pgfqpoint{1.370396in}{1.740068in}}%
\pgfpathlineto{\pgfqpoint{1.369208in}{1.740341in}}%
\pgfpathlineto{\pgfqpoint{1.368002in}{1.740597in}}%
\pgfpathlineto{\pgfqpoint{1.366779in}{1.740835in}}%
\pgfpathlineto{\pgfqpoint{1.365540in}{1.741055in}}%
\pgfpathclose%
\pgfusepath{fill}%
\end{pgfscope}%
\begin{pgfscope}%
\pgfpathrectangle{\pgfqpoint{0.329460in}{0.284240in}}{\pgfqpoint{1.989680in}{1.989680in}}%
\pgfusepath{clip}%
\pgfsetbuttcap%
\pgfsetroundjoin%
\definecolor{currentfill}{rgb}{0.267004,0.004874,0.329415}%
\pgfsetfillcolor{currentfill}%
\pgfsetlinewidth{0.000000pt}%
\definecolor{currentstroke}{rgb}{0.000000,0.000000,0.000000}%
\pgfsetstrokecolor{currentstroke}%
\pgfsetdash{}{0pt}%
\pgfpathmoveto{\pgfqpoint{1.043232in}{0.807791in}}%
\pgfpathlineto{\pgfqpoint{1.041146in}{0.805593in}}%
\pgfpathlineto{\pgfqpoint{1.039055in}{0.803611in}}%
\pgfpathlineto{\pgfqpoint{1.036960in}{0.801850in}}%
\pgfpathlineto{\pgfqpoint{1.034860in}{0.800313in}}%
\pgfpathlineto{\pgfqpoint{1.021225in}{0.805828in}}%
\pgfpathlineto{\pgfqpoint{1.007953in}{0.811568in}}%
\pgfpathlineto{\pgfqpoint{0.995058in}{0.817526in}}%
\pgfpathlineto{\pgfqpoint{0.982553in}{0.823695in}}%
\pgfpathlineto{\pgfqpoint{0.984992in}{0.825071in}}%
\pgfpathlineto{\pgfqpoint{0.987425in}{0.826672in}}%
\pgfpathlineto{\pgfqpoint{0.989853in}{0.828494in}}%
\pgfpathlineto{\pgfqpoint{0.992276in}{0.830531in}}%
\pgfpathlineto{\pgfqpoint{1.004460in}{0.824531in}}%
\pgfpathlineto{\pgfqpoint{1.017023in}{0.818737in}}%
\pgfpathlineto{\pgfqpoint{1.029951in}{0.813154in}}%
\pgfpathlineto{\pgfqpoint{1.043232in}{0.807791in}}%
\pgfpathclose%
\pgfusepath{fill}%
\end{pgfscope}%
\begin{pgfscope}%
\pgfpathrectangle{\pgfqpoint{0.329460in}{0.284240in}}{\pgfqpoint{1.989680in}{1.989680in}}%
\pgfusepath{clip}%
\pgfsetbuttcap%
\pgfsetroundjoin%
\definecolor{currentfill}{rgb}{0.172719,0.448791,0.557885}%
\pgfsetfillcolor{currentfill}%
\pgfsetlinewidth{0.000000pt}%
\definecolor{currentstroke}{rgb}{0.000000,0.000000,0.000000}%
\pgfsetstrokecolor{currentstroke}%
\pgfsetdash{}{0pt}%
\pgfpathmoveto{\pgfqpoint{2.003889in}{1.170239in}}%
\pgfpathlineto{\pgfqpoint{2.007754in}{1.183288in}}%
\pgfpathlineto{\pgfqpoint{2.011641in}{1.196800in}}%
\pgfpathlineto{\pgfqpoint{2.015550in}{1.210784in}}%
\pgfpathlineto{\pgfqpoint{2.009980in}{1.200170in}}%
\pgfpathlineto{\pgfqpoint{2.003722in}{1.189633in}}%
\pgfpathlineto{\pgfqpoint{1.996779in}{1.179185in}}%
\pgfpathlineto{\pgfqpoint{1.989155in}{1.168837in}}%
\pgfpathlineto{\pgfqpoint{1.985377in}{1.155009in}}%
\pgfpathlineto{\pgfqpoint{1.981620in}{1.141655in}}%
\pgfpathlineto{\pgfqpoint{1.977885in}{1.128768in}}%
\pgfpathlineto{\pgfqpoint{1.985393in}{1.138997in}}%
\pgfpathlineto{\pgfqpoint{1.992232in}{1.149327in}}%
\pgfpathlineto{\pgfqpoint{1.998398in}{1.159744in}}%
\pgfpathlineto{\pgfqpoint{2.003889in}{1.170239in}}%
\pgfpathclose%
\pgfusepath{fill}%
\end{pgfscope}%
\begin{pgfscope}%
\pgfpathrectangle{\pgfqpoint{0.329460in}{0.284240in}}{\pgfqpoint{1.989680in}{1.989680in}}%
\pgfusepath{clip}%
\pgfsetbuttcap%
\pgfsetroundjoin%
\definecolor{currentfill}{rgb}{0.166383,0.690856,0.496502}%
\pgfsetfillcolor{currentfill}%
\pgfsetlinewidth{0.000000pt}%
\definecolor{currentstroke}{rgb}{0.000000,0.000000,0.000000}%
\pgfsetstrokecolor{currentstroke}%
\pgfsetdash{}{0pt}%
\pgfpathmoveto{\pgfqpoint{1.151376in}{1.420844in}}%
\pgfpathlineto{\pgfqpoint{1.148477in}{1.413414in}}%
\pgfpathlineto{\pgfqpoint{1.145579in}{1.405940in}}%
\pgfpathlineto{\pgfqpoint{1.142683in}{1.398422in}}%
\pgfpathlineto{\pgfqpoint{1.139789in}{1.390863in}}%
\pgfpathlineto{\pgfqpoint{1.135692in}{1.394186in}}%
\pgfpathlineto{\pgfqpoint{1.131816in}{1.397569in}}%
\pgfpathlineto{\pgfqpoint{1.128165in}{1.401009in}}%
\pgfpathlineto{\pgfqpoint{1.124742in}{1.404502in}}%
\pgfpathlineto{\pgfqpoint{1.127836in}{1.411858in}}%
\pgfpathlineto{\pgfqpoint{1.130931in}{1.419173in}}%
\pgfpathlineto{\pgfqpoint{1.134029in}{1.426446in}}%
\pgfpathlineto{\pgfqpoint{1.137128in}{1.433674in}}%
\pgfpathlineto{\pgfqpoint{1.140370in}{1.430388in}}%
\pgfpathlineto{\pgfqpoint{1.143827in}{1.427152in}}%
\pgfpathlineto{\pgfqpoint{1.147497in}{1.423970in}}%
\pgfpathlineto{\pgfqpoint{1.151376in}{1.420844in}}%
\pgfpathclose%
\pgfusepath{fill}%
\end{pgfscope}%
\begin{pgfscope}%
\pgfpathrectangle{\pgfqpoint{0.329460in}{0.284240in}}{\pgfqpoint{1.989680in}{1.989680in}}%
\pgfusepath{clip}%
\pgfsetbuttcap%
\pgfsetroundjoin%
\definecolor{currentfill}{rgb}{0.955300,0.901065,0.118128}%
\pgfsetfillcolor{currentfill}%
\pgfsetlinewidth{0.000000pt}%
\definecolor{currentstroke}{rgb}{0.000000,0.000000,0.000000}%
\pgfsetstrokecolor{currentstroke}%
\pgfsetdash{}{0pt}%
\pgfpathmoveto{\pgfqpoint{1.330938in}{1.739810in}}%
\pgfpathlineto{\pgfqpoint{1.329258in}{1.739750in}}%
\pgfpathlineto{\pgfqpoint{1.327579in}{1.739572in}}%
\pgfpathlineto{\pgfqpoint{1.325901in}{1.739278in}}%
\pgfpathlineto{\pgfqpoint{1.324224in}{1.738868in}}%
\pgfpathlineto{\pgfqpoint{1.325785in}{1.739252in}}%
\pgfpathlineto{\pgfqpoint{1.327371in}{1.739614in}}%
\pgfpathlineto{\pgfqpoint{1.328980in}{1.739951in}}%
\pgfpathlineto{\pgfqpoint{1.330611in}{1.740266in}}%
\pgfpathlineto{\pgfqpoint{1.331890in}{1.740589in}}%
\pgfpathlineto{\pgfqpoint{1.333170in}{1.740796in}}%
\pgfpathlineto{\pgfqpoint{1.334451in}{1.740887in}}%
\pgfpathlineto{\pgfqpoint{1.335733in}{1.740860in}}%
\pgfpathlineto{\pgfqpoint{1.334509in}{1.740624in}}%
\pgfpathlineto{\pgfqpoint{1.333300in}{1.740371in}}%
\pgfpathlineto{\pgfqpoint{1.332110in}{1.740099in}}%
\pgfpathlineto{\pgfqpoint{1.330938in}{1.739810in}}%
\pgfpathclose%
\pgfusepath{fill}%
\end{pgfscope}%
\begin{pgfscope}%
\pgfpathrectangle{\pgfqpoint{0.329460in}{0.284240in}}{\pgfqpoint{1.989680in}{1.989680in}}%
\pgfusepath{clip}%
\pgfsetbuttcap%
\pgfsetroundjoin%
\definecolor{currentfill}{rgb}{0.147607,0.511733,0.557049}%
\pgfsetfillcolor{currentfill}%
\pgfsetlinewidth{0.000000pt}%
\definecolor{currentstroke}{rgb}{0.000000,0.000000,0.000000}%
\pgfsetstrokecolor{currentstroke}%
\pgfsetdash{}{0pt}%
\pgfpathmoveto{\pgfqpoint{1.591565in}{1.253857in}}%
\pgfpathlineto{\pgfqpoint{1.594248in}{1.245799in}}%
\pgfpathlineto{\pgfqpoint{1.596930in}{1.237741in}}%
\pgfpathlineto{\pgfqpoint{1.599611in}{1.229688in}}%
\pgfpathlineto{\pgfqpoint{1.602291in}{1.221641in}}%
\pgfpathlineto{\pgfqpoint{1.596005in}{1.217671in}}%
\pgfpathlineto{\pgfqpoint{1.589463in}{1.213802in}}%
\pgfpathlineto{\pgfqpoint{1.582671in}{1.210038in}}%
\pgfpathlineto{\pgfqpoint{1.575636in}{1.206382in}}%
\pgfpathlineto{\pgfqpoint{1.573235in}{1.214612in}}%
\pgfpathlineto{\pgfqpoint{1.570833in}{1.222847in}}%
\pgfpathlineto{\pgfqpoint{1.568431in}{1.231087in}}%
\pgfpathlineto{\pgfqpoint{1.566027in}{1.239327in}}%
\pgfpathlineto{\pgfqpoint{1.572766in}{1.242808in}}%
\pgfpathlineto{\pgfqpoint{1.579273in}{1.246393in}}%
\pgfpathlineto{\pgfqpoint{1.585541in}{1.250077in}}%
\pgfpathlineto{\pgfqpoint{1.591565in}{1.253857in}}%
\pgfpathclose%
\pgfusepath{fill}%
\end{pgfscope}%
\begin{pgfscope}%
\pgfpathrectangle{\pgfqpoint{0.329460in}{0.284240in}}{\pgfqpoint{1.989680in}{1.989680in}}%
\pgfusepath{clip}%
\pgfsetbuttcap%
\pgfsetroundjoin%
\definecolor{currentfill}{rgb}{0.120081,0.622161,0.534946}%
\pgfsetfillcolor{currentfill}%
\pgfsetlinewidth{0.000000pt}%
\definecolor{currentstroke}{rgb}{0.000000,0.000000,0.000000}%
\pgfsetstrokecolor{currentstroke}%
\pgfsetdash{}{0pt}%
\pgfpathmoveto{\pgfqpoint{1.577993in}{1.363388in}}%
\pgfpathlineto{\pgfqpoint{1.580928in}{1.355702in}}%
\pgfpathlineto{\pgfqpoint{1.583860in}{1.347988in}}%
\pgfpathlineto{\pgfqpoint{1.586792in}{1.340249in}}%
\pgfpathlineto{\pgfqpoint{1.589721in}{1.332487in}}%
\pgfpathlineto{\pgfqpoint{1.585163in}{1.328771in}}%
\pgfpathlineto{\pgfqpoint{1.580363in}{1.325127in}}%
\pgfpathlineto{\pgfqpoint{1.575327in}{1.321559in}}%
\pgfpathlineto{\pgfqpoint{1.570058in}{1.318070in}}%
\pgfpathlineto{\pgfqpoint{1.567363in}{1.326027in}}%
\pgfpathlineto{\pgfqpoint{1.564668in}{1.333961in}}%
\pgfpathlineto{\pgfqpoint{1.561971in}{1.341870in}}%
\pgfpathlineto{\pgfqpoint{1.559273in}{1.349750in}}%
\pgfpathlineto{\pgfqpoint{1.564288in}{1.353051in}}%
\pgfpathlineto{\pgfqpoint{1.569083in}{1.356427in}}%
\pgfpathlineto{\pgfqpoint{1.573653in}{1.359874in}}%
\pgfpathlineto{\pgfqpoint{1.577993in}{1.363388in}}%
\pgfpathclose%
\pgfusepath{fill}%
\end{pgfscope}%
\begin{pgfscope}%
\pgfpathrectangle{\pgfqpoint{0.329460in}{0.284240in}}{\pgfqpoint{1.989680in}{1.989680in}}%
\pgfusepath{clip}%
\pgfsetbuttcap%
\pgfsetroundjoin%
\definecolor{currentfill}{rgb}{0.344074,0.780029,0.397381}%
\pgfsetfillcolor{currentfill}%
\pgfsetlinewidth{0.000000pt}%
\definecolor{currentstroke}{rgb}{0.000000,0.000000,0.000000}%
\pgfsetstrokecolor{currentstroke}%
\pgfsetdash{}{0pt}%
\pgfpathmoveto{\pgfqpoint{1.538443in}{1.529766in}}%
\pgfpathlineto{\pgfqpoint{1.541743in}{1.523485in}}%
\pgfpathlineto{\pgfqpoint{1.545042in}{1.517136in}}%
\pgfpathlineto{\pgfqpoint{1.548338in}{1.510722in}}%
\pgfpathlineto{\pgfqpoint{1.551632in}{1.504244in}}%
\pgfpathlineto{\pgfqpoint{1.549719in}{1.501194in}}%
\pgfpathlineto{\pgfqpoint{1.547604in}{1.498174in}}%
\pgfpathlineto{\pgfqpoint{1.545290in}{1.495186in}}%
\pgfpathlineto{\pgfqpoint{1.542779in}{1.492233in}}%
\pgfpathlineto{\pgfqpoint{1.539624in}{1.498922in}}%
\pgfpathlineto{\pgfqpoint{1.536467in}{1.505546in}}%
\pgfpathlineto{\pgfqpoint{1.533309in}{1.512104in}}%
\pgfpathlineto{\pgfqpoint{1.530148in}{1.518594in}}%
\pgfpathlineto{\pgfqpoint{1.532499in}{1.521340in}}%
\pgfpathlineto{\pgfqpoint{1.534667in}{1.524119in}}%
\pgfpathlineto{\pgfqpoint{1.536649in}{1.526929in}}%
\pgfpathlineto{\pgfqpoint{1.538443in}{1.529766in}}%
\pgfpathclose%
\pgfusepath{fill}%
\end{pgfscope}%
\begin{pgfscope}%
\pgfpathrectangle{\pgfqpoint{0.329460in}{0.284240in}}{\pgfqpoint{1.989680in}{1.989680in}}%
\pgfusepath{clip}%
\pgfsetbuttcap%
\pgfsetroundjoin%
\definecolor{currentfill}{rgb}{0.267004,0.004874,0.329415}%
\pgfsetfillcolor{currentfill}%
\pgfsetlinewidth{0.000000pt}%
\definecolor{currentstroke}{rgb}{0.000000,0.000000,0.000000}%
\pgfsetstrokecolor{currentstroke}%
\pgfsetdash{}{0pt}%
\pgfpathmoveto{\pgfqpoint{1.730601in}{0.829349in}}%
\pgfpathlineto{\pgfqpoint{1.733115in}{0.828240in}}%
\pgfpathlineto{\pgfqpoint{1.735635in}{0.827365in}}%
\pgfpathlineto{\pgfqpoint{1.738161in}{0.826727in}}%
\pgfpathlineto{\pgfqpoint{1.740694in}{0.826332in}}%
\pgfpathlineto{\pgfqpoint{1.728232in}{0.819811in}}%
\pgfpathlineto{\pgfqpoint{1.715357in}{0.813498in}}%
\pgfpathlineto{\pgfqpoint{1.702083in}{0.807403in}}%
\pgfpathlineto{\pgfqpoint{1.688423in}{0.801532in}}%
\pgfpathlineto{\pgfqpoint{1.686222in}{0.802090in}}%
\pgfpathlineto{\pgfqpoint{1.684027in}{0.802891in}}%
\pgfpathlineto{\pgfqpoint{1.681837in}{0.803931in}}%
\pgfpathlineto{\pgfqpoint{1.679653in}{0.805204in}}%
\pgfpathlineto{\pgfqpoint{1.692966in}{0.810920in}}%
\pgfpathlineto{\pgfqpoint{1.705903in}{0.816854in}}%
\pgfpathlineto{\pgfqpoint{1.718453in}{0.822999in}}%
\pgfpathlineto{\pgfqpoint{1.730601in}{0.829349in}}%
\pgfpathclose%
\pgfusepath{fill}%
\end{pgfscope}%
\begin{pgfscope}%
\pgfpathrectangle{\pgfqpoint{0.329460in}{0.284240in}}{\pgfqpoint{1.989680in}{1.989680in}}%
\pgfusepath{clip}%
\pgfsetbuttcap%
\pgfsetroundjoin%
\definecolor{currentfill}{rgb}{0.636902,0.856542,0.216620}%
\pgfsetfillcolor{currentfill}%
\pgfsetlinewidth{0.000000pt}%
\definecolor{currentstroke}{rgb}{0.000000,0.000000,0.000000}%
\pgfsetstrokecolor{currentstroke}%
\pgfsetdash{}{0pt}%
\pgfpathmoveto{\pgfqpoint{1.490845in}{1.636184in}}%
\pgfpathlineto{\pgfqpoint{1.494304in}{1.631630in}}%
\pgfpathlineto{\pgfqpoint{1.497761in}{1.626983in}}%
\pgfpathlineto{\pgfqpoint{1.501216in}{1.622247in}}%
\pgfpathlineto{\pgfqpoint{1.504668in}{1.617421in}}%
\pgfpathlineto{\pgfqpoint{1.504451in}{1.615120in}}%
\pgfpathlineto{\pgfqpoint{1.504082in}{1.612823in}}%
\pgfpathlineto{\pgfqpoint{1.503559in}{1.610532in}}%
\pgfpathlineto{\pgfqpoint{1.502884in}{1.608249in}}%
\pgfpathlineto{\pgfqpoint{1.499466in}{1.613289in}}%
\pgfpathlineto{\pgfqpoint{1.496047in}{1.618239in}}%
\pgfpathlineto{\pgfqpoint{1.492626in}{1.623099in}}%
\pgfpathlineto{\pgfqpoint{1.489202in}{1.627866in}}%
\pgfpathlineto{\pgfqpoint{1.489821in}{1.629936in}}%
\pgfpathlineto{\pgfqpoint{1.490302in}{1.632014in}}%
\pgfpathlineto{\pgfqpoint{1.490643in}{1.634098in}}%
\pgfpathlineto{\pgfqpoint{1.490845in}{1.636184in}}%
\pgfpathclose%
\pgfusepath{fill}%
\end{pgfscope}%
\begin{pgfscope}%
\pgfpathrectangle{\pgfqpoint{0.329460in}{0.284240in}}{\pgfqpoint{1.989680in}{1.989680in}}%
\pgfusepath{clip}%
\pgfsetbuttcap%
\pgfsetroundjoin%
\definecolor{currentfill}{rgb}{0.935904,0.898570,0.108131}%
\pgfsetfillcolor{currentfill}%
\pgfsetlinewidth{0.000000pt}%
\definecolor{currentstroke}{rgb}{0.000000,0.000000,0.000000}%
\pgfsetstrokecolor{currentstroke}%
\pgfsetdash{}{0pt}%
\pgfpathmoveto{\pgfqpoint{1.393375in}{1.733165in}}%
\pgfpathlineto{\pgfqpoint{1.395999in}{1.732284in}}%
\pgfpathlineto{\pgfqpoint{1.398621in}{1.731289in}}%
\pgfpathlineto{\pgfqpoint{1.401241in}{1.730179in}}%
\pgfpathlineto{\pgfqpoint{1.403859in}{1.728956in}}%
\pgfpathlineto{\pgfqpoint{1.405289in}{1.728171in}}%
\pgfpathlineto{\pgfqpoint{1.406665in}{1.727365in}}%
\pgfpathlineto{\pgfqpoint{1.407986in}{1.726538in}}%
\pgfpathlineto{\pgfqpoint{1.409251in}{1.725693in}}%
\pgfpathlineto{\pgfqpoint{1.406363in}{1.727079in}}%
\pgfpathlineto{\pgfqpoint{1.403473in}{1.728351in}}%
\pgfpathlineto{\pgfqpoint{1.400582in}{1.729509in}}%
\pgfpathlineto{\pgfqpoint{1.397688in}{1.730553in}}%
\pgfpathlineto{\pgfqpoint{1.396676in}{1.731230in}}%
\pgfpathlineto{\pgfqpoint{1.395619in}{1.731891in}}%
\pgfpathlineto{\pgfqpoint{1.394519in}{1.732536in}}%
\pgfpathlineto{\pgfqpoint{1.393375in}{1.733165in}}%
\pgfpathclose%
\pgfusepath{fill}%
\end{pgfscope}%
\begin{pgfscope}%
\pgfpathrectangle{\pgfqpoint{0.329460in}{0.284240in}}{\pgfqpoint{1.989680in}{1.989680in}}%
\pgfusepath{clip}%
\pgfsetbuttcap%
\pgfsetroundjoin%
\definecolor{currentfill}{rgb}{0.855810,0.888601,0.097452}%
\pgfsetfillcolor{currentfill}%
\pgfsetlinewidth{0.000000pt}%
\definecolor{currentstroke}{rgb}{0.000000,0.000000,0.000000}%
\pgfsetstrokecolor{currentstroke}%
\pgfsetdash{}{0pt}%
\pgfpathmoveto{\pgfqpoint{1.430346in}{1.710209in}}%
\pgfpathlineto{\pgfqpoint{1.433624in}{1.707899in}}%
\pgfpathlineto{\pgfqpoint{1.436901in}{1.705480in}}%
\pgfpathlineto{\pgfqpoint{1.440175in}{1.702953in}}%
\pgfpathlineto{\pgfqpoint{1.443446in}{1.700319in}}%
\pgfpathlineto{\pgfqpoint{1.444445in}{1.698947in}}%
\pgfpathlineto{\pgfqpoint{1.445351in}{1.697561in}}%
\pgfpathlineto{\pgfqpoint{1.446164in}{1.696162in}}%
\pgfpathlineto{\pgfqpoint{1.446883in}{1.694751in}}%
\pgfpathlineto{\pgfqpoint{1.443487in}{1.697585in}}%
\pgfpathlineto{\pgfqpoint{1.440088in}{1.700312in}}%
\pgfpathlineto{\pgfqpoint{1.436687in}{1.702931in}}%
\pgfpathlineto{\pgfqpoint{1.433284in}{1.705440in}}%
\pgfpathlineto{\pgfqpoint{1.432670in}{1.706649in}}%
\pgfpathlineto{\pgfqpoint{1.431975in}{1.707847in}}%
\pgfpathlineto{\pgfqpoint{1.431200in}{1.709034in}}%
\pgfpathlineto{\pgfqpoint{1.430346in}{1.710209in}}%
\pgfpathclose%
\pgfusepath{fill}%
\end{pgfscope}%
\begin{pgfscope}%
\pgfpathrectangle{\pgfqpoint{0.329460in}{0.284240in}}{\pgfqpoint{1.989680in}{1.989680in}}%
\pgfusepath{clip}%
\pgfsetbuttcap%
\pgfsetroundjoin%
\definecolor{currentfill}{rgb}{0.814576,0.883393,0.110347}%
\pgfsetfillcolor{currentfill}%
\pgfsetlinewidth{0.000000pt}%
\definecolor{currentstroke}{rgb}{0.000000,0.000000,0.000000}%
\pgfsetstrokecolor{currentstroke}%
\pgfsetdash{}{0pt}%
\pgfpathmoveto{\pgfqpoint{1.446883in}{1.694751in}}%
\pgfpathlineto{\pgfqpoint{1.450278in}{1.691811in}}%
\pgfpathlineto{\pgfqpoint{1.453671in}{1.688765in}}%
\pgfpathlineto{\pgfqpoint{1.457061in}{1.685615in}}%
\pgfpathlineto{\pgfqpoint{1.460449in}{1.682361in}}%
\pgfpathlineto{\pgfqpoint{1.461166in}{1.680735in}}%
\pgfpathlineto{\pgfqpoint{1.461773in}{1.679099in}}%
\pgfpathlineto{\pgfqpoint{1.462270in}{1.677454in}}%
\pgfpathlineto{\pgfqpoint{1.462657in}{1.675802in}}%
\pgfpathlineto{\pgfqpoint{1.459197in}{1.679263in}}%
\pgfpathlineto{\pgfqpoint{1.455735in}{1.682620in}}%
\pgfpathlineto{\pgfqpoint{1.452270in}{1.685872in}}%
\pgfpathlineto{\pgfqpoint{1.448804in}{1.689019in}}%
\pgfpathlineto{\pgfqpoint{1.448468in}{1.690463in}}%
\pgfpathlineto{\pgfqpoint{1.448036in}{1.691900in}}%
\pgfpathlineto{\pgfqpoint{1.447507in}{1.693330in}}%
\pgfpathlineto{\pgfqpoint{1.446883in}{1.694751in}}%
\pgfpathclose%
\pgfusepath{fill}%
\end{pgfscope}%
\begin{pgfscope}%
\pgfpathrectangle{\pgfqpoint{0.329460in}{0.284240in}}{\pgfqpoint{1.989680in}{1.989680in}}%
\pgfusepath{clip}%
\pgfsetbuttcap%
\pgfsetroundjoin%
\definecolor{currentfill}{rgb}{0.955300,0.901065,0.118128}%
\pgfsetfillcolor{currentfill}%
\pgfsetlinewidth{0.000000pt}%
\definecolor{currentstroke}{rgb}{0.000000,0.000000,0.000000}%
\pgfsetstrokecolor{currentstroke}%
\pgfsetdash{}{0pt}%
\pgfpathmoveto{\pgfqpoint{1.370396in}{1.740068in}}%
\pgfpathlineto{\pgfqpoint{1.371990in}{1.740029in}}%
\pgfpathlineto{\pgfqpoint{1.373583in}{1.739872in}}%
\pgfpathlineto{\pgfqpoint{1.375175in}{1.739599in}}%
\pgfpathlineto{\pgfqpoint{1.376765in}{1.739211in}}%
\pgfpathlineto{\pgfqpoint{1.378323in}{1.738823in}}%
\pgfpathlineto{\pgfqpoint{1.379854in}{1.738414in}}%
\pgfpathlineto{\pgfqpoint{1.381356in}{1.737981in}}%
\pgfpathlineto{\pgfqpoint{1.382829in}{1.737527in}}%
\pgfpathlineto{\pgfqpoint{1.380862in}{1.738021in}}%
\pgfpathlineto{\pgfqpoint{1.378892in}{1.738399in}}%
\pgfpathlineto{\pgfqpoint{1.376921in}{1.738660in}}%
\pgfpathlineto{\pgfqpoint{1.374949in}{1.738804in}}%
\pgfpathlineto{\pgfqpoint{1.373844in}{1.739145in}}%
\pgfpathlineto{\pgfqpoint{1.372715in}{1.739470in}}%
\pgfpathlineto{\pgfqpoint{1.371566in}{1.739777in}}%
\pgfpathlineto{\pgfqpoint{1.370396in}{1.740068in}}%
\pgfpathclose%
\pgfusepath{fill}%
\end{pgfscope}%
\begin{pgfscope}%
\pgfpathrectangle{\pgfqpoint{0.329460in}{0.284240in}}{\pgfqpoint{1.989680in}{1.989680in}}%
\pgfusepath{clip}%
\pgfsetbuttcap%
\pgfsetroundjoin%
\definecolor{currentfill}{rgb}{0.935904,0.898570,0.108131}%
\pgfsetfillcolor{currentfill}%
\pgfsetlinewidth{0.000000pt}%
\definecolor{currentstroke}{rgb}{0.000000,0.000000,0.000000}%
\pgfsetstrokecolor{currentstroke}%
\pgfsetdash{}{0pt}%
\pgfpathmoveto{\pgfqpoint{1.303826in}{1.729940in}}%
\pgfpathlineto{\pgfqpoint{1.300879in}{1.728858in}}%
\pgfpathlineto{\pgfqpoint{1.297933in}{1.727661in}}%
\pgfpathlineto{\pgfqpoint{1.294989in}{1.726350in}}%
\pgfpathlineto{\pgfqpoint{1.292048in}{1.724926in}}%
\pgfpathlineto{\pgfqpoint{1.293262in}{1.725788in}}%
\pgfpathlineto{\pgfqpoint{1.294533in}{1.726631in}}%
\pgfpathlineto{\pgfqpoint{1.295861in}{1.727455in}}%
\pgfpathlineto{\pgfqpoint{1.297243in}{1.728259in}}%
\pgfpathlineto{\pgfqpoint{1.299925in}{1.729517in}}%
\pgfpathlineto{\pgfqpoint{1.302608in}{1.730661in}}%
\pgfpathlineto{\pgfqpoint{1.305294in}{1.731691in}}%
\pgfpathlineto{\pgfqpoint{1.307982in}{1.732607in}}%
\pgfpathlineto{\pgfqpoint{1.306876in}{1.731963in}}%
\pgfpathlineto{\pgfqpoint{1.305814in}{1.731304in}}%
\pgfpathlineto{\pgfqpoint{1.304797in}{1.730629in}}%
\pgfpathlineto{\pgfqpoint{1.303826in}{1.729940in}}%
\pgfpathclose%
\pgfusepath{fill}%
\end{pgfscope}%
\begin{pgfscope}%
\pgfpathrectangle{\pgfqpoint{0.329460in}{0.284240in}}{\pgfqpoint{1.989680in}{1.989680in}}%
\pgfusepath{clip}%
\pgfsetbuttcap%
\pgfsetroundjoin%
\definecolor{currentfill}{rgb}{0.955300,0.901065,0.118128}%
\pgfsetfillcolor{currentfill}%
\pgfsetlinewidth{0.000000pt}%
\definecolor{currentstroke}{rgb}{0.000000,0.000000,0.000000}%
\pgfsetstrokecolor{currentstroke}%
\pgfsetdash{}{0pt}%
\pgfpathmoveto{\pgfqpoint{1.326463in}{1.738488in}}%
\pgfpathlineto{\pgfqpoint{1.324410in}{1.738317in}}%
\pgfpathlineto{\pgfqpoint{1.322360in}{1.738030in}}%
\pgfpathlineto{\pgfqpoint{1.320310in}{1.737626in}}%
\pgfpathlineto{\pgfqpoint{1.318262in}{1.737106in}}%
\pgfpathlineto{\pgfqpoint{1.319708in}{1.737579in}}%
\pgfpathlineto{\pgfqpoint{1.321184in}{1.738031in}}%
\pgfpathlineto{\pgfqpoint{1.322690in}{1.738460in}}%
\pgfpathlineto{\pgfqpoint{1.324224in}{1.738868in}}%
\pgfpathlineto{\pgfqpoint{1.325901in}{1.739278in}}%
\pgfpathlineto{\pgfqpoint{1.327579in}{1.739572in}}%
\pgfpathlineto{\pgfqpoint{1.329258in}{1.739750in}}%
\pgfpathlineto{\pgfqpoint{1.330938in}{1.739810in}}%
\pgfpathlineto{\pgfqpoint{1.329786in}{1.739505in}}%
\pgfpathlineto{\pgfqpoint{1.328656in}{1.739182in}}%
\pgfpathlineto{\pgfqpoint{1.327548in}{1.738843in}}%
\pgfpathlineto{\pgfqpoint{1.326463in}{1.738488in}}%
\pgfpathclose%
\pgfusepath{fill}%
\end{pgfscope}%
\begin{pgfscope}%
\pgfpathrectangle{\pgfqpoint{0.329460in}{0.284240in}}{\pgfqpoint{1.989680in}{1.989680in}}%
\pgfusepath{clip}%
\pgfsetbuttcap%
\pgfsetroundjoin%
\definecolor{currentfill}{rgb}{0.283072,0.130895,0.449241}%
\pgfsetfillcolor{currentfill}%
\pgfsetlinewidth{0.000000pt}%
\definecolor{currentstroke}{rgb}{0.000000,0.000000,0.000000}%
\pgfsetstrokecolor{currentstroke}%
\pgfsetdash{}{0pt}%
\pgfpathmoveto{\pgfqpoint{1.620061in}{0.918583in}}%
\pgfpathlineto{\pgfqpoint{1.622153in}{0.912450in}}%
\pgfpathlineto{\pgfqpoint{1.624247in}{0.906438in}}%
\pgfpathlineto{\pgfqpoint{1.626343in}{0.900552in}}%
\pgfpathlineto{\pgfqpoint{1.628441in}{0.894795in}}%
\pgfpathlineto{\pgfqpoint{1.616865in}{0.890223in}}%
\pgfpathlineto{\pgfqpoint{1.605000in}{0.885847in}}%
\pgfpathlineto{\pgfqpoint{1.592859in}{0.881669in}}%
\pgfpathlineto{\pgfqpoint{1.580454in}{0.877696in}}%
\pgfpathlineto{\pgfqpoint{1.578715in}{0.883598in}}%
\pgfpathlineto{\pgfqpoint{1.576977in}{0.889628in}}%
\pgfpathlineto{\pgfqpoint{1.575241in}{0.895784in}}%
\pgfpathlineto{\pgfqpoint{1.573507in}{0.902062in}}%
\pgfpathlineto{\pgfqpoint{1.585540in}{0.905901in}}%
\pgfpathlineto{\pgfqpoint{1.597318in}{0.909937in}}%
\pgfpathlineto{\pgfqpoint{1.608829in}{0.914166in}}%
\pgfpathlineto{\pgfqpoint{1.620061in}{0.918583in}}%
\pgfpathclose%
\pgfusepath{fill}%
\end{pgfscope}%
\begin{pgfscope}%
\pgfpathrectangle{\pgfqpoint{0.329460in}{0.284240in}}{\pgfqpoint{1.989680in}{1.989680in}}%
\pgfusepath{clip}%
\pgfsetbuttcap%
\pgfsetroundjoin%
\definecolor{currentfill}{rgb}{0.282327,0.094955,0.417331}%
\pgfsetfillcolor{currentfill}%
\pgfsetlinewidth{0.000000pt}%
\definecolor{currentstroke}{rgb}{0.000000,0.000000,0.000000}%
\pgfsetstrokecolor{currentstroke}%
\pgfsetdash{}{0pt}%
\pgfpathmoveto{\pgfqpoint{1.628441in}{0.894795in}}%
\pgfpathlineto{\pgfqpoint{1.630541in}{0.889171in}}%
\pgfpathlineto{\pgfqpoint{1.632642in}{0.883683in}}%
\pgfpathlineto{\pgfqpoint{1.634746in}{0.878335in}}%
\pgfpathlineto{\pgfqpoint{1.636853in}{0.873131in}}%
\pgfpathlineto{\pgfqpoint{1.624931in}{0.868405in}}%
\pgfpathlineto{\pgfqpoint{1.612711in}{0.863881in}}%
\pgfpathlineto{\pgfqpoint{1.600206in}{0.859562in}}%
\pgfpathlineto{\pgfqpoint{1.587428in}{0.855455in}}%
\pgfpathlineto{\pgfqpoint{1.585682in}{0.860803in}}%
\pgfpathlineto{\pgfqpoint{1.583937in}{0.866295in}}%
\pgfpathlineto{\pgfqpoint{1.582195in}{0.871928in}}%
\pgfpathlineto{\pgfqpoint{1.580454in}{0.877696in}}%
\pgfpathlineto{\pgfqpoint{1.592859in}{0.881669in}}%
\pgfpathlineto{\pgfqpoint{1.605000in}{0.885847in}}%
\pgfpathlineto{\pgfqpoint{1.616865in}{0.890223in}}%
\pgfpathlineto{\pgfqpoint{1.628441in}{0.894795in}}%
\pgfpathclose%
\pgfusepath{fill}%
\end{pgfscope}%
\begin{pgfscope}%
\pgfpathrectangle{\pgfqpoint{0.329460in}{0.284240in}}{\pgfqpoint{1.989680in}{1.989680in}}%
\pgfusepath{clip}%
\pgfsetbuttcap%
\pgfsetroundjoin%
\definecolor{currentfill}{rgb}{0.896320,0.893616,0.096335}%
\pgfsetfillcolor{currentfill}%
\pgfsetlinewidth{0.000000pt}%
\definecolor{currentstroke}{rgb}{0.000000,0.000000,0.000000}%
\pgfsetstrokecolor{currentstroke}%
\pgfsetdash{}{0pt}%
\pgfpathmoveto{\pgfqpoint{1.413727in}{1.722137in}}%
\pgfpathlineto{\pgfqpoint{1.416837in}{1.720460in}}%
\pgfpathlineto{\pgfqpoint{1.419945in}{1.718672in}}%
\pgfpathlineto{\pgfqpoint{1.423052in}{1.716773in}}%
\pgfpathlineto{\pgfqpoint{1.426156in}{1.714763in}}%
\pgfpathlineto{\pgfqpoint{1.427318in}{1.713648in}}%
\pgfpathlineto{\pgfqpoint{1.428404in}{1.712517in}}%
\pgfpathlineto{\pgfqpoint{1.429414in}{1.711370in}}%
\pgfpathlineto{\pgfqpoint{1.430346in}{1.710209in}}%
\pgfpathlineto{\pgfqpoint{1.427066in}{1.712409in}}%
\pgfpathlineto{\pgfqpoint{1.423784in}{1.714499in}}%
\pgfpathlineto{\pgfqpoint{1.420500in}{1.716477in}}%
\pgfpathlineto{\pgfqpoint{1.417213in}{1.718343in}}%
\pgfpathlineto{\pgfqpoint{1.416438in}{1.719311in}}%
\pgfpathlineto{\pgfqpoint{1.415598in}{1.720266in}}%
\pgfpathlineto{\pgfqpoint{1.414694in}{1.721208in}}%
\pgfpathlineto{\pgfqpoint{1.413727in}{1.722137in}}%
\pgfpathclose%
\pgfusepath{fill}%
\end{pgfscope}%
\begin{pgfscope}%
\pgfpathrectangle{\pgfqpoint{0.329460in}{0.284240in}}{\pgfqpoint{1.989680in}{1.989680in}}%
\pgfusepath{clip}%
\pgfsetbuttcap%
\pgfsetroundjoin%
\definecolor{currentfill}{rgb}{0.212395,0.359683,0.551710}%
\pgfsetfillcolor{currentfill}%
\pgfsetlinewidth{0.000000pt}%
\definecolor{currentstroke}{rgb}{0.000000,0.000000,0.000000}%
\pgfsetstrokecolor{currentstroke}%
\pgfsetdash{}{0pt}%
\pgfpathmoveto{\pgfqpoint{1.140380in}{1.090541in}}%
\pgfpathlineto{\pgfqpoint{1.138381in}{1.082440in}}%
\pgfpathlineto{\pgfqpoint{1.136381in}{1.074383in}}%
\pgfpathlineto{\pgfqpoint{1.134382in}{1.066373in}}%
\pgfpathlineto{\pgfqpoint{1.132383in}{1.058415in}}%
\pgfpathlineto{\pgfqpoint{1.122893in}{1.062084in}}%
\pgfpathlineto{\pgfqpoint{1.113648in}{1.065904in}}%
\pgfpathlineto{\pgfqpoint{1.104657in}{1.069872in}}%
\pgfpathlineto{\pgfqpoint{1.095927in}{1.073983in}}%
\pgfpathlineto{\pgfqpoint{1.098254in}{1.081781in}}%
\pgfpathlineto{\pgfqpoint{1.100581in}{1.089629in}}%
\pgfpathlineto{\pgfqpoint{1.102908in}{1.097526in}}%
\pgfpathlineto{\pgfqpoint{1.105236in}{1.105467in}}%
\pgfpathlineto{\pgfqpoint{1.113652in}{1.101526in}}%
\pgfpathlineto{\pgfqpoint{1.122321in}{1.097722in}}%
\pgfpathlineto{\pgfqpoint{1.131233in}{1.094059in}}%
\pgfpathlineto{\pgfqpoint{1.140380in}{1.090541in}}%
\pgfpathclose%
\pgfusepath{fill}%
\end{pgfscope}%
\begin{pgfscope}%
\pgfpathrectangle{\pgfqpoint{0.329460in}{0.284240in}}{\pgfqpoint{1.989680in}{1.989680in}}%
\pgfusepath{clip}%
\pgfsetbuttcap%
\pgfsetroundjoin%
\definecolor{currentfill}{rgb}{0.855810,0.888601,0.097452}%
\pgfsetfillcolor{currentfill}%
\pgfsetlinewidth{0.000000pt}%
\definecolor{currentstroke}{rgb}{0.000000,0.000000,0.000000}%
\pgfsetstrokecolor{currentstroke}%
\pgfsetdash{}{0pt}%
\pgfpathmoveto{\pgfqpoint{1.268614in}{1.704359in}}%
\pgfpathlineto{\pgfqpoint{1.265190in}{1.701804in}}%
\pgfpathlineto{\pgfqpoint{1.261769in}{1.699140in}}%
\pgfpathlineto{\pgfqpoint{1.258350in}{1.696368in}}%
\pgfpathlineto{\pgfqpoint{1.254932in}{1.693489in}}%
\pgfpathlineto{\pgfqpoint{1.255567in}{1.694909in}}%
\pgfpathlineto{\pgfqpoint{1.256296in}{1.696318in}}%
\pgfpathlineto{\pgfqpoint{1.257120in}{1.697716in}}%
\pgfpathlineto{\pgfqpoint{1.258036in}{1.699100in}}%
\pgfpathlineto{\pgfqpoint{1.261341in}{1.701778in}}%
\pgfpathlineto{\pgfqpoint{1.264647in}{1.704349in}}%
\pgfpathlineto{\pgfqpoint{1.267955in}{1.706811in}}%
\pgfpathlineto{\pgfqpoint{1.271266in}{1.709165in}}%
\pgfpathlineto{\pgfqpoint{1.270482in}{1.707979in}}%
\pgfpathlineto{\pgfqpoint{1.269778in}{1.706782in}}%
\pgfpathlineto{\pgfqpoint{1.269155in}{1.705575in}}%
\pgfpathlineto{\pgfqpoint{1.268614in}{1.704359in}}%
\pgfpathclose%
\pgfusepath{fill}%
\end{pgfscope}%
\begin{pgfscope}%
\pgfpathrectangle{\pgfqpoint{0.329460in}{0.284240in}}{\pgfqpoint{1.989680in}{1.989680in}}%
\pgfusepath{clip}%
\pgfsetbuttcap%
\pgfsetroundjoin%
\definecolor{currentfill}{rgb}{0.280255,0.165693,0.476498}%
\pgfsetfillcolor{currentfill}%
\pgfsetlinewidth{0.000000pt}%
\definecolor{currentstroke}{rgb}{0.000000,0.000000,0.000000}%
\pgfsetstrokecolor{currentstroke}%
\pgfsetdash{}{0pt}%
\pgfpathmoveto{\pgfqpoint{1.611707in}{0.944269in}}%
\pgfpathlineto{\pgfqpoint{1.613793in}{0.937682in}}%
\pgfpathlineto{\pgfqpoint{1.615881in}{0.931203in}}%
\pgfpathlineto{\pgfqpoint{1.617970in}{0.924835in}}%
\pgfpathlineto{\pgfqpoint{1.620061in}{0.918583in}}%
\pgfpathlineto{\pgfqpoint{1.608829in}{0.914166in}}%
\pgfpathlineto{\pgfqpoint{1.597318in}{0.909937in}}%
\pgfpathlineto{\pgfqpoint{1.585540in}{0.905901in}}%
\pgfpathlineto{\pgfqpoint{1.573507in}{0.902062in}}%
\pgfpathlineto{\pgfqpoint{1.571774in}{0.908459in}}%
\pgfpathlineto{\pgfqpoint{1.570042in}{0.914970in}}%
\pgfpathlineto{\pgfqpoint{1.568312in}{0.921593in}}%
\pgfpathlineto{\pgfqpoint{1.566583in}{0.928324in}}%
\pgfpathlineto{\pgfqpoint{1.578245in}{0.932029in}}%
\pgfpathlineto{\pgfqpoint{1.589661in}{0.935924in}}%
\pgfpathlineto{\pgfqpoint{1.600819in}{0.940005in}}%
\pgfpathlineto{\pgfqpoint{1.611707in}{0.944269in}}%
\pgfpathclose%
\pgfusepath{fill}%
\end{pgfscope}%
\begin{pgfscope}%
\pgfpathrectangle{\pgfqpoint{0.329460in}{0.284240in}}{\pgfqpoint{1.989680in}{1.989680in}}%
\pgfusepath{clip}%
\pgfsetbuttcap%
\pgfsetroundjoin%
\definecolor{currentfill}{rgb}{0.487026,0.823929,0.312321}%
\pgfsetfillcolor{currentfill}%
\pgfsetlinewidth{0.000000pt}%
\definecolor{currentstroke}{rgb}{0.000000,0.000000,0.000000}%
\pgfsetstrokecolor{currentstroke}%
\pgfsetdash{}{0pt}%
\pgfpathmoveto{\pgfqpoint{1.191787in}{1.575222in}}%
\pgfpathlineto{\pgfqpoint{1.188496in}{1.569497in}}%
\pgfpathlineto{\pgfqpoint{1.185207in}{1.563691in}}%
\pgfpathlineto{\pgfqpoint{1.181921in}{1.557806in}}%
\pgfpathlineto{\pgfqpoint{1.178636in}{1.551842in}}%
\pgfpathlineto{\pgfqpoint{1.176982in}{1.554471in}}%
\pgfpathlineto{\pgfqpoint{1.175504in}{1.557122in}}%
\pgfpathlineto{\pgfqpoint{1.174203in}{1.559793in}}%
\pgfpathlineto{\pgfqpoint{1.173081in}{1.562480in}}%
\pgfpathlineto{\pgfqpoint{1.176465in}{1.568231in}}%
\pgfpathlineto{\pgfqpoint{1.179852in}{1.573905in}}%
\pgfpathlineto{\pgfqpoint{1.183241in}{1.579499in}}%
\pgfpathlineto{\pgfqpoint{1.186632in}{1.585012in}}%
\pgfpathlineto{\pgfqpoint{1.187675in}{1.582538in}}%
\pgfpathlineto{\pgfqpoint{1.188883in}{1.580080in}}%
\pgfpathlineto{\pgfqpoint{1.190254in}{1.577641in}}%
\pgfpathlineto{\pgfqpoint{1.191787in}{1.575222in}}%
\pgfpathclose%
\pgfusepath{fill}%
\end{pgfscope}%
\begin{pgfscope}%
\pgfpathrectangle{\pgfqpoint{0.329460in}{0.284240in}}{\pgfqpoint{1.989680in}{1.989680in}}%
\pgfusepath{clip}%
\pgfsetbuttcap%
\pgfsetroundjoin%
\definecolor{currentfill}{rgb}{0.195860,0.395433,0.555276}%
\pgfsetfillcolor{currentfill}%
\pgfsetlinewidth{0.000000pt}%
\definecolor{currentstroke}{rgb}{0.000000,0.000000,0.000000}%
\pgfsetstrokecolor{currentstroke}%
\pgfsetdash{}{0pt}%
\pgfpathmoveto{\pgfqpoint{1.594821in}{1.141085in}}%
\pgfpathlineto{\pgfqpoint{1.597217in}{1.133031in}}%
\pgfpathlineto{\pgfqpoint{1.599612in}{1.125011in}}%
\pgfpathlineto{\pgfqpoint{1.602007in}{1.117027in}}%
\pgfpathlineto{\pgfqpoint{1.604402in}{1.109082in}}%
\pgfpathlineto{\pgfqpoint{1.596217in}{1.105023in}}%
\pgfpathlineto{\pgfqpoint{1.587772in}{1.101097in}}%
\pgfpathlineto{\pgfqpoint{1.579076in}{1.097308in}}%
\pgfpathlineto{\pgfqpoint{1.570137in}{1.093661in}}%
\pgfpathlineto{\pgfqpoint{1.568061in}{1.101771in}}%
\pgfpathlineto{\pgfqpoint{1.565986in}{1.109920in}}%
\pgfpathlineto{\pgfqpoint{1.563910in}{1.118105in}}%
\pgfpathlineto{\pgfqpoint{1.561834in}{1.126323in}}%
\pgfpathlineto{\pgfqpoint{1.570438in}{1.129813in}}%
\pgfpathlineto{\pgfqpoint{1.578810in}{1.133440in}}%
\pgfpathlineto{\pgfqpoint{1.586940in}{1.137199in}}%
\pgfpathlineto{\pgfqpoint{1.594821in}{1.141085in}}%
\pgfpathclose%
\pgfusepath{fill}%
\end{pgfscope}%
\begin{pgfscope}%
\pgfpathrectangle{\pgfqpoint{0.329460in}{0.284240in}}{\pgfqpoint{1.989680in}{1.989680in}}%
\pgfusepath{clip}%
\pgfsetbuttcap%
\pgfsetroundjoin%
\definecolor{currentfill}{rgb}{0.233603,0.313828,0.543914}%
\pgfsetfillcolor{currentfill}%
\pgfsetlinewidth{0.000000pt}%
\definecolor{currentstroke}{rgb}{0.000000,0.000000,0.000000}%
\pgfsetstrokecolor{currentstroke}%
\pgfsetdash{}{0pt}%
\pgfpathmoveto{\pgfqpoint{1.948723in}{1.041533in}}%
\pgfpathlineto{\pgfqpoint{1.952303in}{1.050963in}}%
\pgfpathlineto{\pgfqpoint{1.955901in}{1.060799in}}%
\pgfpathlineto{\pgfqpoint{1.959516in}{1.071050in}}%
\pgfpathlineto{\pgfqpoint{1.963151in}{1.081722in}}%
\pgfpathlineto{\pgfqpoint{1.955149in}{1.071769in}}%
\pgfpathlineto{\pgfqpoint{1.946505in}{1.061937in}}%
\pgfpathlineto{\pgfqpoint{1.937226in}{1.052238in}}%
\pgfpathlineto{\pgfqpoint{1.927319in}{1.042681in}}%
\pgfpathlineto{\pgfqpoint{1.923877in}{1.032178in}}%
\pgfpathlineto{\pgfqpoint{1.920452in}{1.022098in}}%
\pgfpathlineto{\pgfqpoint{1.917045in}{1.012435in}}%
\pgfpathlineto{\pgfqpoint{1.913656in}{1.003180in}}%
\pgfpathlineto{\pgfqpoint{1.923348in}{1.012567in}}%
\pgfpathlineto{\pgfqpoint{1.932428in}{1.022096in}}%
\pgfpathlineto{\pgfqpoint{1.940889in}{1.031755in}}%
\pgfpathlineto{\pgfqpoint{1.948723in}{1.041533in}}%
\pgfpathclose%
\pgfusepath{fill}%
\end{pgfscope}%
\begin{pgfscope}%
\pgfpathrectangle{\pgfqpoint{0.329460in}{0.284240in}}{\pgfqpoint{1.989680in}{1.989680in}}%
\pgfusepath{clip}%
\pgfsetbuttcap%
\pgfsetroundjoin%
\definecolor{currentfill}{rgb}{0.220124,0.725509,0.466226}%
\pgfsetfillcolor{currentfill}%
\pgfsetlinewidth{0.000000pt}%
\definecolor{currentstroke}{rgb}{0.000000,0.000000,0.000000}%
\pgfsetstrokecolor{currentstroke}%
\pgfsetdash{}{0pt}%
\pgfpathmoveto{\pgfqpoint{1.555378in}{1.464872in}}%
\pgfpathlineto{\pgfqpoint{1.558523in}{1.457890in}}%
\pgfpathlineto{\pgfqpoint{1.561666in}{1.450855in}}%
\pgfpathlineto{\pgfqpoint{1.564807in}{1.443769in}}%
\pgfpathlineto{\pgfqpoint{1.567946in}{1.436634in}}%
\pgfpathlineto{\pgfqpoint{1.564898in}{1.433306in}}%
\pgfpathlineto{\pgfqpoint{1.561632in}{1.430026in}}%
\pgfpathlineto{\pgfqpoint{1.558151in}{1.426796in}}%
\pgfpathlineto{\pgfqpoint{1.554457in}{1.423620in}}%
\pgfpathlineto{\pgfqpoint{1.551508in}{1.430958in}}%
\pgfpathlineto{\pgfqpoint{1.548556in}{1.438248in}}%
\pgfpathlineto{\pgfqpoint{1.545603in}{1.445486in}}%
\pgfpathlineto{\pgfqpoint{1.542648in}{1.452671in}}%
\pgfpathlineto{\pgfqpoint{1.546133in}{1.455648in}}%
\pgfpathlineto{\pgfqpoint{1.549418in}{1.458677in}}%
\pgfpathlineto{\pgfqpoint{1.552501in}{1.461752in}}%
\pgfpathlineto{\pgfqpoint{1.555378in}{1.464872in}}%
\pgfpathclose%
\pgfusepath{fill}%
\end{pgfscope}%
\begin{pgfscope}%
\pgfpathrectangle{\pgfqpoint{0.329460in}{0.284240in}}{\pgfqpoint{1.989680in}{1.989680in}}%
\pgfusepath{clip}%
\pgfsetbuttcap%
\pgfsetroundjoin%
\definecolor{currentfill}{rgb}{0.814576,0.883393,0.110347}%
\pgfsetfillcolor{currentfill}%
\pgfsetlinewidth{0.000000pt}%
\definecolor{currentstroke}{rgb}{0.000000,0.000000,0.000000}%
\pgfsetstrokecolor{currentstroke}%
\pgfsetdash{}{0pt}%
\pgfpathmoveto{\pgfqpoint{1.253355in}{1.687732in}}%
\pgfpathlineto{\pgfqpoint{1.249880in}{1.684539in}}%
\pgfpathlineto{\pgfqpoint{1.246407in}{1.681240in}}%
\pgfpathlineto{\pgfqpoint{1.242936in}{1.677836in}}%
\pgfpathlineto{\pgfqpoint{1.239467in}{1.674329in}}%
\pgfpathlineto{\pgfqpoint{1.239755in}{1.675986in}}%
\pgfpathlineto{\pgfqpoint{1.240155in}{1.677637in}}%
\pgfpathlineto{\pgfqpoint{1.240664in}{1.679281in}}%
\pgfpathlineto{\pgfqpoint{1.241284in}{1.680917in}}%
\pgfpathlineto{\pgfqpoint{1.244693in}{1.684216in}}%
\pgfpathlineto{\pgfqpoint{1.248104in}{1.687411in}}%
\pgfpathlineto{\pgfqpoint{1.251517in}{1.690503in}}%
\pgfpathlineto{\pgfqpoint{1.254932in}{1.693489in}}%
\pgfpathlineto{\pgfqpoint{1.254393in}{1.692060in}}%
\pgfpathlineto{\pgfqpoint{1.253950in}{1.690623in}}%
\pgfpathlineto{\pgfqpoint{1.253604in}{1.689180in}}%
\pgfpathlineto{\pgfqpoint{1.253355in}{1.687732in}}%
\pgfpathclose%
\pgfusepath{fill}%
\end{pgfscope}%
\begin{pgfscope}%
\pgfpathrectangle{\pgfqpoint{0.329460in}{0.284240in}}{\pgfqpoint{1.989680in}{1.989680in}}%
\pgfusepath{clip}%
\pgfsetbuttcap%
\pgfsetroundjoin%
\definecolor{currentfill}{rgb}{0.279566,0.067836,0.391917}%
\pgfsetfillcolor{currentfill}%
\pgfsetlinewidth{0.000000pt}%
\definecolor{currentstroke}{rgb}{0.000000,0.000000,0.000000}%
\pgfsetstrokecolor{currentstroke}%
\pgfsetdash{}{0pt}%
\pgfpathmoveto{\pgfqpoint{1.636853in}{0.873131in}}%
\pgfpathlineto{\pgfqpoint{1.638961in}{0.868074in}}%
\pgfpathlineto{\pgfqpoint{1.641073in}{0.863169in}}%
\pgfpathlineto{\pgfqpoint{1.643187in}{0.858418in}}%
\pgfpathlineto{\pgfqpoint{1.645303in}{0.853827in}}%
\pgfpathlineto{\pgfqpoint{1.633035in}{0.848947in}}%
\pgfpathlineto{\pgfqpoint{1.620459in}{0.844275in}}%
\pgfpathlineto{\pgfqpoint{1.607588in}{0.839815in}}%
\pgfpathlineto{\pgfqpoint{1.594436in}{0.835573in}}%
\pgfpathlineto{\pgfqpoint{1.592680in}{0.840309in}}%
\pgfpathlineto{\pgfqpoint{1.590927in}{0.845204in}}%
\pgfpathlineto{\pgfqpoint{1.589177in}{0.850254in}}%
\pgfpathlineto{\pgfqpoint{1.587428in}{0.855455in}}%
\pgfpathlineto{\pgfqpoint{1.600206in}{0.859562in}}%
\pgfpathlineto{\pgfqpoint{1.612711in}{0.863881in}}%
\pgfpathlineto{\pgfqpoint{1.624931in}{0.868405in}}%
\pgfpathlineto{\pgfqpoint{1.636853in}{0.873131in}}%
\pgfpathclose%
\pgfusepath{fill}%
\end{pgfscope}%
\begin{pgfscope}%
\pgfpathrectangle{\pgfqpoint{0.329460in}{0.284240in}}{\pgfqpoint{1.989680in}{1.989680in}}%
\pgfusepath{clip}%
\pgfsetbuttcap%
\pgfsetroundjoin%
\definecolor{currentfill}{rgb}{0.636902,0.856542,0.216620}%
\pgfsetfillcolor{currentfill}%
\pgfsetlinewidth{0.000000pt}%
\definecolor{currentstroke}{rgb}{0.000000,0.000000,0.000000}%
\pgfsetstrokecolor{currentstroke}%
\pgfsetdash{}{0pt}%
\pgfpathmoveto{\pgfqpoint{1.213840in}{1.626034in}}%
\pgfpathlineto{\pgfqpoint{1.210432in}{1.621220in}}%
\pgfpathlineto{\pgfqpoint{1.207026in}{1.616313in}}%
\pgfpathlineto{\pgfqpoint{1.203621in}{1.611316in}}%
\pgfpathlineto{\pgfqpoint{1.200219in}{1.606228in}}%
\pgfpathlineto{\pgfqpoint{1.199409in}{1.608502in}}%
\pgfpathlineto{\pgfqpoint{1.198751in}{1.610786in}}%
\pgfpathlineto{\pgfqpoint{1.198245in}{1.613078in}}%
\pgfpathlineto{\pgfqpoint{1.197892in}{1.615376in}}%
\pgfpathlineto{\pgfqpoint{1.201341in}{1.620249in}}%
\pgfpathlineto{\pgfqpoint{1.204793in}{1.625034in}}%
\pgfpathlineto{\pgfqpoint{1.208247in}{1.629727in}}%
\pgfpathlineto{\pgfqpoint{1.211703in}{1.634329in}}%
\pgfpathlineto{\pgfqpoint{1.212029in}{1.632246in}}%
\pgfpathlineto{\pgfqpoint{1.212494in}{1.630167in}}%
\pgfpathlineto{\pgfqpoint{1.213097in}{1.628096in}}%
\pgfpathlineto{\pgfqpoint{1.213840in}{1.626034in}}%
\pgfpathclose%
\pgfusepath{fill}%
\end{pgfscope}%
\begin{pgfscope}%
\pgfpathrectangle{\pgfqpoint{0.329460in}{0.284240in}}{\pgfqpoint{1.989680in}{1.989680in}}%
\pgfusepath{clip}%
\pgfsetbuttcap%
\pgfsetroundjoin%
\definecolor{currentfill}{rgb}{0.896320,0.893616,0.096335}%
\pgfsetfillcolor{currentfill}%
\pgfsetlinewidth{0.000000pt}%
\definecolor{currentstroke}{rgb}{0.000000,0.000000,0.000000}%
\pgfsetstrokecolor{currentstroke}%
\pgfsetdash{}{0pt}%
\pgfpathmoveto{\pgfqpoint{1.284527in}{1.717474in}}%
\pgfpathlineto{\pgfqpoint{1.281209in}{1.715564in}}%
\pgfpathlineto{\pgfqpoint{1.277893in}{1.713542in}}%
\pgfpathlineto{\pgfqpoint{1.274578in}{1.711409in}}%
\pgfpathlineto{\pgfqpoint{1.271266in}{1.709165in}}%
\pgfpathlineto{\pgfqpoint{1.272129in}{1.710338in}}%
\pgfpathlineto{\pgfqpoint{1.273070in}{1.711498in}}%
\pgfpathlineto{\pgfqpoint{1.274088in}{1.712643in}}%
\pgfpathlineto{\pgfqpoint{1.275183in}{1.713773in}}%
\pgfpathlineto{\pgfqpoint{1.278331in}{1.715824in}}%
\pgfpathlineto{\pgfqpoint{1.281480in}{1.717765in}}%
\pgfpathlineto{\pgfqpoint{1.284632in}{1.719595in}}%
\pgfpathlineto{\pgfqpoint{1.287786in}{1.721312in}}%
\pgfpathlineto{\pgfqpoint{1.286875in}{1.720371in}}%
\pgfpathlineto{\pgfqpoint{1.286027in}{1.719417in}}%
\pgfpathlineto{\pgfqpoint{1.285245in}{1.718451in}}%
\pgfpathlineto{\pgfqpoint{1.284527in}{1.717474in}}%
\pgfpathclose%
\pgfusepath{fill}%
\end{pgfscope}%
\begin{pgfscope}%
\pgfpathrectangle{\pgfqpoint{0.329460in}{0.284240in}}{\pgfqpoint{1.989680in}{1.989680in}}%
\pgfusepath{clip}%
\pgfsetbuttcap%
\pgfsetroundjoin%
\definecolor{currentfill}{rgb}{0.762373,0.876424,0.137064}%
\pgfsetfillcolor{currentfill}%
\pgfsetlinewidth{0.000000pt}%
\definecolor{currentstroke}{rgb}{0.000000,0.000000,0.000000}%
\pgfsetstrokecolor{currentstroke}%
\pgfsetdash{}{0pt}%
\pgfpathmoveto{\pgfqpoint{1.462657in}{1.675802in}}%
\pgfpathlineto{\pgfqpoint{1.466116in}{1.672239in}}%
\pgfpathlineto{\pgfqpoint{1.469572in}{1.668574in}}%
\pgfpathlineto{\pgfqpoint{1.473026in}{1.664809in}}%
\pgfpathlineto{\pgfqpoint{1.476477in}{1.660945in}}%
\pgfpathlineto{\pgfqpoint{1.476792in}{1.659078in}}%
\pgfpathlineto{\pgfqpoint{1.476982in}{1.657206in}}%
\pgfpathlineto{\pgfqpoint{1.477047in}{1.655331in}}%
\pgfpathlineto{\pgfqpoint{1.476986in}{1.653456in}}%
\pgfpathlineto{\pgfqpoint{1.473516in}{1.657531in}}%
\pgfpathlineto{\pgfqpoint{1.470044in}{1.661508in}}%
\pgfpathlineto{\pgfqpoint{1.466570in}{1.665383in}}%
\pgfpathlineto{\pgfqpoint{1.463094in}{1.669157in}}%
\pgfpathlineto{\pgfqpoint{1.463152in}{1.670820in}}%
\pgfpathlineto{\pgfqpoint{1.463099in}{1.672484in}}%
\pgfpathlineto{\pgfqpoint{1.462934in}{1.674145in}}%
\pgfpathlineto{\pgfqpoint{1.462657in}{1.675802in}}%
\pgfpathclose%
\pgfusepath{fill}%
\end{pgfscope}%
\begin{pgfscope}%
\pgfpathrectangle{\pgfqpoint{0.329460in}{0.284240in}}{\pgfqpoint{1.989680in}{1.989680in}}%
\pgfusepath{clip}%
\pgfsetbuttcap%
\pgfsetroundjoin%
\definecolor{currentfill}{rgb}{0.955300,0.901065,0.118128}%
\pgfsetfillcolor{currentfill}%
\pgfsetlinewidth{0.000000pt}%
\definecolor{currentstroke}{rgb}{0.000000,0.000000,0.000000}%
\pgfsetstrokecolor{currentstroke}%
\pgfsetdash{}{0pt}%
\pgfpathmoveto{\pgfqpoint{1.374949in}{1.738804in}}%
\pgfpathlineto{\pgfqpoint{1.376921in}{1.738660in}}%
\pgfpathlineto{\pgfqpoint{1.378892in}{1.738399in}}%
\pgfpathlineto{\pgfqpoint{1.380862in}{1.738021in}}%
\pgfpathlineto{\pgfqpoint{1.382829in}{1.737527in}}%
\pgfpathlineto{\pgfqpoint{1.384271in}{1.737052in}}%
\pgfpathlineto{\pgfqpoint{1.385681in}{1.736556in}}%
\pgfpathlineto{\pgfqpoint{1.387056in}{1.736039in}}%
\pgfpathlineto{\pgfqpoint{1.388396in}{1.735502in}}%
\pgfpathlineto{\pgfqpoint{1.386081in}{1.736121in}}%
\pgfpathlineto{\pgfqpoint{1.383764in}{1.736625in}}%
\pgfpathlineto{\pgfqpoint{1.381446in}{1.737013in}}%
\pgfpathlineto{\pgfqpoint{1.379127in}{1.737283in}}%
\pgfpathlineto{\pgfqpoint{1.378121in}{1.737687in}}%
\pgfpathlineto{\pgfqpoint{1.377089in}{1.738075in}}%
\pgfpathlineto{\pgfqpoint{1.376032in}{1.738447in}}%
\pgfpathlineto{\pgfqpoint{1.374949in}{1.738804in}}%
\pgfpathclose%
\pgfusepath{fill}%
\end{pgfscope}%
\begin{pgfscope}%
\pgfpathrectangle{\pgfqpoint{0.329460in}{0.284240in}}{\pgfqpoint{1.989680in}{1.989680in}}%
\pgfusepath{clip}%
\pgfsetbuttcap%
\pgfsetroundjoin%
\definecolor{currentfill}{rgb}{0.344074,0.780029,0.397381}%
\pgfsetfillcolor{currentfill}%
\pgfsetlinewidth{0.000000pt}%
\definecolor{currentstroke}{rgb}{0.000000,0.000000,0.000000}%
\pgfsetstrokecolor{currentstroke}%
\pgfsetdash{}{0pt}%
\pgfpathmoveto{\pgfqpoint{1.174470in}{1.516183in}}%
\pgfpathlineto{\pgfqpoint{1.171348in}{1.509648in}}%
\pgfpathlineto{\pgfqpoint{1.168227in}{1.503045in}}%
\pgfpathlineto{\pgfqpoint{1.165109in}{1.496375in}}%
\pgfpathlineto{\pgfqpoint{1.161992in}{1.489641in}}%
\pgfpathlineto{\pgfqpoint{1.159308in}{1.492559in}}%
\pgfpathlineto{\pgfqpoint{1.156818in}{1.495516in}}%
\pgfpathlineto{\pgfqpoint{1.154526in}{1.498508in}}%
\pgfpathlineto{\pgfqpoint{1.152434in}{1.501532in}}%
\pgfpathlineto{\pgfqpoint{1.155702in}{1.508058in}}%
\pgfpathlineto{\pgfqpoint{1.158972in}{1.514519in}}%
\pgfpathlineto{\pgfqpoint{1.162244in}{1.520915in}}%
\pgfpathlineto{\pgfqpoint{1.165518in}{1.527243in}}%
\pgfpathlineto{\pgfqpoint{1.167479in}{1.524430in}}%
\pgfpathlineto{\pgfqpoint{1.169626in}{1.521647in}}%
\pgfpathlineto{\pgfqpoint{1.171957in}{1.518897in}}%
\pgfpathlineto{\pgfqpoint{1.174470in}{1.516183in}}%
\pgfpathclose%
\pgfusepath{fill}%
\end{pgfscope}%
\begin{pgfscope}%
\pgfpathrectangle{\pgfqpoint{0.329460in}{0.284240in}}{\pgfqpoint{1.989680in}{1.989680in}}%
\pgfusepath{clip}%
\pgfsetbuttcap%
\pgfsetroundjoin%
\definecolor{currentfill}{rgb}{0.274128,0.199721,0.498911}%
\pgfsetfillcolor{currentfill}%
\pgfsetlinewidth{0.000000pt}%
\definecolor{currentstroke}{rgb}{0.000000,0.000000,0.000000}%
\pgfsetstrokecolor{currentstroke}%
\pgfsetdash{}{0pt}%
\pgfpathmoveto{\pgfqpoint{1.603373in}{0.971630in}}%
\pgfpathlineto{\pgfqpoint{1.605455in}{0.964645in}}%
\pgfpathlineto{\pgfqpoint{1.607538in}{0.957753in}}%
\pgfpathlineto{\pgfqpoint{1.609622in}{0.950960in}}%
\pgfpathlineto{\pgfqpoint{1.611707in}{0.944269in}}%
\pgfpathlineto{\pgfqpoint{1.600819in}{0.940005in}}%
\pgfpathlineto{\pgfqpoint{1.589661in}{0.935924in}}%
\pgfpathlineto{\pgfqpoint{1.578245in}{0.932029in}}%
\pgfpathlineto{\pgfqpoint{1.566583in}{0.928324in}}%
\pgfpathlineto{\pgfqpoint{1.564854in}{0.935160in}}%
\pgfpathlineto{\pgfqpoint{1.563127in}{0.942097in}}%
\pgfpathlineto{\pgfqpoint{1.561401in}{0.949132in}}%
\pgfpathlineto{\pgfqpoint{1.559676in}{0.956262in}}%
\pgfpathlineto{\pgfqpoint{1.570968in}{0.959832in}}%
\pgfpathlineto{\pgfqpoint{1.582023in}{0.963587in}}%
\pgfpathlineto{\pgfqpoint{1.592828in}{0.967521in}}%
\pgfpathlineto{\pgfqpoint{1.603373in}{0.971630in}}%
\pgfpathclose%
\pgfusepath{fill}%
\end{pgfscope}%
\begin{pgfscope}%
\pgfpathrectangle{\pgfqpoint{0.329460in}{0.284240in}}{\pgfqpoint{1.989680in}{1.989680in}}%
\pgfusepath{clip}%
\pgfsetbuttcap%
\pgfsetroundjoin%
\definecolor{currentfill}{rgb}{0.955300,0.901065,0.118128}%
\pgfsetfillcolor{currentfill}%
\pgfsetlinewidth{0.000000pt}%
\definecolor{currentstroke}{rgb}{0.000000,0.000000,0.000000}%
\pgfsetstrokecolor{currentstroke}%
\pgfsetdash{}{0pt}%
\pgfpathmoveto{\pgfqpoint{1.322378in}{1.736913in}}%
\pgfpathlineto{\pgfqpoint{1.319986in}{1.736612in}}%
\pgfpathlineto{\pgfqpoint{1.317595in}{1.736193in}}%
\pgfpathlineto{\pgfqpoint{1.315207in}{1.735659in}}%
\pgfpathlineto{\pgfqpoint{1.312819in}{1.735008in}}%
\pgfpathlineto{\pgfqpoint{1.314127in}{1.735562in}}%
\pgfpathlineto{\pgfqpoint{1.315471in}{1.736097in}}%
\pgfpathlineto{\pgfqpoint{1.316850in}{1.736612in}}%
\pgfpathlineto{\pgfqpoint{1.318262in}{1.737106in}}%
\pgfpathlineto{\pgfqpoint{1.320310in}{1.737626in}}%
\pgfpathlineto{\pgfqpoint{1.322360in}{1.738030in}}%
\pgfpathlineto{\pgfqpoint{1.324410in}{1.738317in}}%
\pgfpathlineto{\pgfqpoint{1.326463in}{1.738488in}}%
\pgfpathlineto{\pgfqpoint{1.325402in}{1.738117in}}%
\pgfpathlineto{\pgfqpoint{1.324367in}{1.737730in}}%
\pgfpathlineto{\pgfqpoint{1.323359in}{1.737329in}}%
\pgfpathlineto{\pgfqpoint{1.322378in}{1.736913in}}%
\pgfpathclose%
\pgfusepath{fill}%
\end{pgfscope}%
\begin{pgfscope}%
\pgfpathrectangle{\pgfqpoint{0.329460in}{0.284240in}}{\pgfqpoint{1.989680in}{1.989680in}}%
\pgfusepath{clip}%
\pgfsetbuttcap%
\pgfsetroundjoin%
\definecolor{currentfill}{rgb}{0.147607,0.511733,0.557049}%
\pgfsetfillcolor{currentfill}%
\pgfsetlinewidth{0.000000pt}%
\definecolor{currentstroke}{rgb}{0.000000,0.000000,0.000000}%
\pgfsetstrokecolor{currentstroke}%
\pgfsetdash{}{0pt}%
\pgfpathmoveto{\pgfqpoint{1.142528in}{1.236323in}}%
\pgfpathlineto{\pgfqpoint{1.140192in}{1.228045in}}%
\pgfpathlineto{\pgfqpoint{1.137858in}{1.219768in}}%
\pgfpathlineto{\pgfqpoint{1.135524in}{1.211494in}}%
\pgfpathlineto{\pgfqpoint{1.133190in}{1.203227in}}%
\pgfpathlineto{\pgfqpoint{1.125945in}{1.206783in}}%
\pgfpathlineto{\pgfqpoint{1.118937in}{1.210451in}}%
\pgfpathlineto{\pgfqpoint{1.112173in}{1.214227in}}%
\pgfpathlineto{\pgfqpoint{1.105659in}{1.218107in}}%
\pgfpathlineto{\pgfqpoint{1.108281in}{1.226197in}}%
\pgfpathlineto{\pgfqpoint{1.110904in}{1.234292in}}%
\pgfpathlineto{\pgfqpoint{1.113528in}{1.242392in}}%
\pgfpathlineto{\pgfqpoint{1.116152in}{1.250493in}}%
\pgfpathlineto{\pgfqpoint{1.122394in}{1.246797in}}%
\pgfpathlineto{\pgfqpoint{1.128875in}{1.243201in}}%
\pgfpathlineto{\pgfqpoint{1.135588in}{1.239708in}}%
\pgfpathlineto{\pgfqpoint{1.142528in}{1.236323in}}%
\pgfpathclose%
\pgfusepath{fill}%
\end{pgfscope}%
\begin{pgfscope}%
\pgfpathrectangle{\pgfqpoint{0.329460in}{0.284240in}}{\pgfqpoint{1.989680in}{1.989680in}}%
\pgfusepath{clip}%
\pgfsetbuttcap%
\pgfsetroundjoin%
\definecolor{currentfill}{rgb}{0.120081,0.622161,0.534946}%
\pgfsetfillcolor{currentfill}%
\pgfsetlinewidth{0.000000pt}%
\definecolor{currentstroke}{rgb}{0.000000,0.000000,0.000000}%
\pgfsetstrokecolor{currentstroke}%
\pgfsetdash{}{0pt}%
\pgfpathmoveto{\pgfqpoint{1.147742in}{1.346882in}}%
\pgfpathlineto{\pgfqpoint{1.145102in}{1.338961in}}%
\pgfpathlineto{\pgfqpoint{1.142464in}{1.331011in}}%
\pgfpathlineto{\pgfqpoint{1.139827in}{1.323036in}}%
\pgfpathlineto{\pgfqpoint{1.137192in}{1.315037in}}%
\pgfpathlineto{\pgfqpoint{1.131721in}{1.318453in}}%
\pgfpathlineto{\pgfqpoint{1.126477in}{1.321952in}}%
\pgfpathlineto{\pgfqpoint{1.121467in}{1.325529in}}%
\pgfpathlineto{\pgfqpoint{1.116694in}{1.329181in}}%
\pgfpathlineto{\pgfqpoint{1.119575in}{1.336988in}}%
\pgfpathlineto{\pgfqpoint{1.122458in}{1.344771in}}%
\pgfpathlineto{\pgfqpoint{1.125342in}{1.352530in}}%
\pgfpathlineto{\pgfqpoint{1.128229in}{1.360261in}}%
\pgfpathlineto{\pgfqpoint{1.132773in}{1.356806in}}%
\pgfpathlineto{\pgfqpoint{1.137543in}{1.353422in}}%
\pgfpathlineto{\pgfqpoint{1.142534in}{1.350113in}}%
\pgfpathlineto{\pgfqpoint{1.147742in}{1.346882in}}%
\pgfpathclose%
\pgfusepath{fill}%
\end{pgfscope}%
\begin{pgfscope}%
\pgfpathrectangle{\pgfqpoint{0.329460in}{0.284240in}}{\pgfqpoint{1.989680in}{1.989680in}}%
\pgfusepath{clip}%
\pgfsetbuttcap%
\pgfsetroundjoin%
\definecolor{currentfill}{rgb}{0.935904,0.898570,0.108131}%
\pgfsetfillcolor{currentfill}%
\pgfsetlinewidth{0.000000pt}%
\definecolor{currentstroke}{rgb}{0.000000,0.000000,0.000000}%
\pgfsetstrokecolor{currentstroke}%
\pgfsetdash{}{0pt}%
\pgfpathmoveto{\pgfqpoint{1.397688in}{1.730553in}}%
\pgfpathlineto{\pgfqpoint{1.400582in}{1.729509in}}%
\pgfpathlineto{\pgfqpoint{1.403473in}{1.728351in}}%
\pgfpathlineto{\pgfqpoint{1.406363in}{1.727079in}}%
\pgfpathlineto{\pgfqpoint{1.409251in}{1.725693in}}%
\pgfpathlineto{\pgfqpoint{1.410459in}{1.724829in}}%
\pgfpathlineto{\pgfqpoint{1.411608in}{1.723948in}}%
\pgfpathlineto{\pgfqpoint{1.412698in}{1.723050in}}%
\pgfpathlineto{\pgfqpoint{1.413727in}{1.722137in}}%
\pgfpathlineto{\pgfqpoint{1.410614in}{1.723700in}}%
\pgfpathlineto{\pgfqpoint{1.407500in}{1.725150in}}%
\pgfpathlineto{\pgfqpoint{1.404384in}{1.726486in}}%
\pgfpathlineto{\pgfqpoint{1.401266in}{1.727708in}}%
\pgfpathlineto{\pgfqpoint{1.400444in}{1.728439in}}%
\pgfpathlineto{\pgfqpoint{1.399573in}{1.729157in}}%
\pgfpathlineto{\pgfqpoint{1.398654in}{1.729862in}}%
\pgfpathlineto{\pgfqpoint{1.397688in}{1.730553in}}%
\pgfpathclose%
\pgfusepath{fill}%
\end{pgfscope}%
\begin{pgfscope}%
\pgfpathrectangle{\pgfqpoint{0.329460in}{0.284240in}}{\pgfqpoint{1.989680in}{1.989680in}}%
\pgfusepath{clip}%
\pgfsetbuttcap%
\pgfsetroundjoin%
\definecolor{currentfill}{rgb}{0.274952,0.037752,0.364543}%
\pgfsetfillcolor{currentfill}%
\pgfsetlinewidth{0.000000pt}%
\definecolor{currentstroke}{rgb}{0.000000,0.000000,0.000000}%
\pgfsetstrokecolor{currentstroke}%
\pgfsetdash{}{0pt}%
\pgfpathmoveto{\pgfqpoint{1.645303in}{0.853827in}}%
\pgfpathlineto{\pgfqpoint{1.647423in}{0.849398in}}%
\pgfpathlineto{\pgfqpoint{1.649545in}{0.845135in}}%
\pgfpathlineto{\pgfqpoint{1.651671in}{0.841043in}}%
\pgfpathlineto{\pgfqpoint{1.653800in}{0.837125in}}%
\pgfpathlineto{\pgfqpoint{1.641183in}{0.832092in}}%
\pgfpathlineto{\pgfqpoint{1.628249in}{0.827272in}}%
\pgfpathlineto{\pgfqpoint{1.615011in}{0.822672in}}%
\pgfpathlineto{\pgfqpoint{1.601483in}{0.818296in}}%
\pgfpathlineto{\pgfqpoint{1.599717in}{0.822357in}}%
\pgfpathlineto{\pgfqpoint{1.597954in}{0.826593in}}%
\pgfpathlineto{\pgfqpoint{1.596194in}{0.831000in}}%
\pgfpathlineto{\pgfqpoint{1.594436in}{0.835573in}}%
\pgfpathlineto{\pgfqpoint{1.607588in}{0.839815in}}%
\pgfpathlineto{\pgfqpoint{1.620459in}{0.844275in}}%
\pgfpathlineto{\pgfqpoint{1.633035in}{0.848947in}}%
\pgfpathlineto{\pgfqpoint{1.645303in}{0.853827in}}%
\pgfpathclose%
\pgfusepath{fill}%
\end{pgfscope}%
\begin{pgfscope}%
\pgfpathrectangle{\pgfqpoint{0.329460in}{0.284240in}}{\pgfqpoint{1.989680in}{1.989680in}}%
\pgfusepath{clip}%
\pgfsetbuttcap%
\pgfsetroundjoin%
\definecolor{currentfill}{rgb}{0.762373,0.876424,0.137064}%
\pgfsetfillcolor{currentfill}%
\pgfsetlinewidth{0.000000pt}%
\definecolor{currentstroke}{rgb}{0.000000,0.000000,0.000000}%
\pgfsetstrokecolor{currentstroke}%
\pgfsetdash{}{0pt}%
\pgfpathmoveto{\pgfqpoint{1.239426in}{1.667679in}}%
\pgfpathlineto{\pgfqpoint{1.235954in}{1.663858in}}%
\pgfpathlineto{\pgfqpoint{1.232484in}{1.659936in}}%
\pgfpathlineto{\pgfqpoint{1.229015in}{1.655913in}}%
\pgfpathlineto{\pgfqpoint{1.225549in}{1.651790in}}%
\pgfpathlineto{\pgfqpoint{1.225376in}{1.653664in}}%
\pgfpathlineto{\pgfqpoint{1.225329in}{1.655540in}}%
\pgfpathlineto{\pgfqpoint{1.225408in}{1.657414in}}%
\pgfpathlineto{\pgfqpoint{1.225612in}{1.659285in}}%
\pgfpathlineto{\pgfqpoint{1.229072in}{1.663196in}}%
\pgfpathlineto{\pgfqpoint{1.232535in}{1.667008in}}%
\pgfpathlineto{\pgfqpoint{1.236000in}{1.670719in}}%
\pgfpathlineto{\pgfqpoint{1.239467in}{1.674329in}}%
\pgfpathlineto{\pgfqpoint{1.239289in}{1.672668in}}%
\pgfpathlineto{\pgfqpoint{1.239223in}{1.671005in}}%
\pgfpathlineto{\pgfqpoint{1.239269in}{1.669342in}}%
\pgfpathlineto{\pgfqpoint{1.239426in}{1.667679in}}%
\pgfpathclose%
\pgfusepath{fill}%
\end{pgfscope}%
\begin{pgfscope}%
\pgfpathrectangle{\pgfqpoint{0.329460in}{0.284240in}}{\pgfqpoint{1.989680in}{1.989680in}}%
\pgfusepath{clip}%
\pgfsetbuttcap%
\pgfsetroundjoin%
\definecolor{currentfill}{rgb}{0.935904,0.898570,0.108131}%
\pgfsetfillcolor{currentfill}%
\pgfsetlinewidth{0.000000pt}%
\definecolor{currentstroke}{rgb}{0.000000,0.000000,0.000000}%
\pgfsetstrokecolor{currentstroke}%
\pgfsetdash{}{0pt}%
\pgfpathmoveto{\pgfqpoint{1.300420in}{1.727048in}}%
\pgfpathlineto{\pgfqpoint{1.297258in}{1.725786in}}%
\pgfpathlineto{\pgfqpoint{1.294099in}{1.724408in}}%
\pgfpathlineto{\pgfqpoint{1.290942in}{1.722917in}}%
\pgfpathlineto{\pgfqpoint{1.287786in}{1.721312in}}%
\pgfpathlineto{\pgfqpoint{1.288760in}{1.722239in}}%
\pgfpathlineto{\pgfqpoint{1.289795in}{1.723151in}}%
\pgfpathlineto{\pgfqpoint{1.290892in}{1.724047in}}%
\pgfpathlineto{\pgfqpoint{1.292048in}{1.724926in}}%
\pgfpathlineto{\pgfqpoint{1.294989in}{1.726350in}}%
\pgfpathlineto{\pgfqpoint{1.297933in}{1.727661in}}%
\pgfpathlineto{\pgfqpoint{1.300879in}{1.728858in}}%
\pgfpathlineto{\pgfqpoint{1.303826in}{1.729940in}}%
\pgfpathlineto{\pgfqpoint{1.302902in}{1.729236in}}%
\pgfpathlineto{\pgfqpoint{1.302025in}{1.728519in}}%
\pgfpathlineto{\pgfqpoint{1.301198in}{1.727790in}}%
\pgfpathlineto{\pgfqpoint{1.300420in}{1.727048in}}%
\pgfpathclose%
\pgfusepath{fill}%
\end{pgfscope}%
\begin{pgfscope}%
\pgfpathrectangle{\pgfqpoint{0.329460in}{0.284240in}}{\pgfqpoint{1.989680in}{1.989680in}}%
\pgfusepath{clip}%
\pgfsetbuttcap%
\pgfsetroundjoin%
\definecolor{currentfill}{rgb}{0.133743,0.548535,0.553541}%
\pgfsetfillcolor{currentfill}%
\pgfsetlinewidth{0.000000pt}%
\definecolor{currentstroke}{rgb}{0.000000,0.000000,0.000000}%
\pgfsetstrokecolor{currentstroke}%
\pgfsetdash{}{0pt}%
\pgfpathmoveto{\pgfqpoint{1.580821in}{1.286054in}}%
\pgfpathlineto{\pgfqpoint{1.583509in}{1.278015in}}%
\pgfpathlineto{\pgfqpoint{1.586195in}{1.269968in}}%
\pgfpathlineto{\pgfqpoint{1.588881in}{1.261915in}}%
\pgfpathlineto{\pgfqpoint{1.591565in}{1.253857in}}%
\pgfpathlineto{\pgfqpoint{1.585541in}{1.250077in}}%
\pgfpathlineto{\pgfqpoint{1.579273in}{1.246393in}}%
\pgfpathlineto{\pgfqpoint{1.572766in}{1.242808in}}%
\pgfpathlineto{\pgfqpoint{1.566027in}{1.239327in}}%
\pgfpathlineto{\pgfqpoint{1.563622in}{1.247565in}}%
\pgfpathlineto{\pgfqpoint{1.561217in}{1.255800in}}%
\pgfpathlineto{\pgfqpoint{1.558811in}{1.264027in}}%
\pgfpathlineto{\pgfqpoint{1.556403in}{1.272245in}}%
\pgfpathlineto{\pgfqpoint{1.562846in}{1.275553in}}%
\pgfpathlineto{\pgfqpoint{1.569067in}{1.278960in}}%
\pgfpathlineto{\pgfqpoint{1.575061in}{1.282461in}}%
\pgfpathlineto{\pgfqpoint{1.580821in}{1.286054in}}%
\pgfpathclose%
\pgfusepath{fill}%
\end{pgfscope}%
\begin{pgfscope}%
\pgfpathrectangle{\pgfqpoint{0.329460in}{0.284240in}}{\pgfqpoint{1.989680in}{1.989680in}}%
\pgfusepath{clip}%
\pgfsetbuttcap%
\pgfsetroundjoin%
\definecolor{currentfill}{rgb}{0.565498,0.842430,0.262877}%
\pgfsetfillcolor{currentfill}%
\pgfsetlinewidth{0.000000pt}%
\definecolor{currentstroke}{rgb}{0.000000,0.000000,0.000000}%
\pgfsetstrokecolor{currentstroke}%
\pgfsetdash{}{0pt}%
\pgfpathmoveto{\pgfqpoint{1.502884in}{1.608249in}}%
\pgfpathlineto{\pgfqpoint{1.506299in}{1.603121in}}%
\pgfpathlineto{\pgfqpoint{1.509712in}{1.597906in}}%
\pgfpathlineto{\pgfqpoint{1.513123in}{1.592606in}}%
\pgfpathlineto{\pgfqpoint{1.516531in}{1.587223in}}%
\pgfpathlineto{\pgfqpoint{1.515635in}{1.584737in}}%
\pgfpathlineto{\pgfqpoint{1.514574in}{1.582264in}}%
\pgfpathlineto{\pgfqpoint{1.513348in}{1.579808in}}%
\pgfpathlineto{\pgfqpoint{1.511959in}{1.577371in}}%
\pgfpathlineto{\pgfqpoint{1.508639in}{1.582967in}}%
\pgfpathlineto{\pgfqpoint{1.505318in}{1.588479in}}%
\pgfpathlineto{\pgfqpoint{1.501994in}{1.593905in}}%
\pgfpathlineto{\pgfqpoint{1.498669in}{1.599244in}}%
\pgfpathlineto{\pgfqpoint{1.499948in}{1.601472in}}%
\pgfpathlineto{\pgfqpoint{1.501077in}{1.603716in}}%
\pgfpathlineto{\pgfqpoint{1.502056in}{1.605976in}}%
\pgfpathlineto{\pgfqpoint{1.502884in}{1.608249in}}%
\pgfpathclose%
\pgfusepath{fill}%
\end{pgfscope}%
\begin{pgfscope}%
\pgfpathrectangle{\pgfqpoint{0.329460in}{0.284240in}}{\pgfqpoint{1.989680in}{1.989680in}}%
\pgfusepath{clip}%
\pgfsetbuttcap%
\pgfsetroundjoin%
\definecolor{currentfill}{rgb}{0.283072,0.130895,0.449241}%
\pgfsetfillcolor{currentfill}%
\pgfsetlinewidth{0.000000pt}%
\definecolor{currentstroke}{rgb}{0.000000,0.000000,0.000000}%
\pgfsetstrokecolor{currentstroke}%
\pgfsetdash{}{0pt}%
\pgfpathmoveto{\pgfqpoint{1.139768in}{0.898820in}}%
\pgfpathlineto{\pgfqpoint{1.138118in}{0.892514in}}%
\pgfpathlineto{\pgfqpoint{1.136466in}{0.886329in}}%
\pgfpathlineto{\pgfqpoint{1.134813in}{0.880270in}}%
\pgfpathlineto{\pgfqpoint{1.133159in}{0.874340in}}%
\pgfpathlineto{\pgfqpoint{1.120530in}{0.878127in}}%
\pgfpathlineto{\pgfqpoint{1.108154in}{0.882123in}}%
\pgfpathlineto{\pgfqpoint{1.096043in}{0.886323in}}%
\pgfpathlineto{\pgfqpoint{1.084210in}{0.890722in}}%
\pgfpathlineto{\pgfqpoint{1.086231in}{0.896513in}}%
\pgfpathlineto{\pgfqpoint{1.088250in}{0.902434in}}%
\pgfpathlineto{\pgfqpoint{1.090268in}{0.908480in}}%
\pgfpathlineto{\pgfqpoint{1.092284in}{0.914648in}}%
\pgfpathlineto{\pgfqpoint{1.103764in}{0.910397in}}%
\pgfpathlineto{\pgfqpoint{1.115513in}{0.906339in}}%
\pgfpathlineto{\pgfqpoint{1.127519in}{0.902479in}}%
\pgfpathlineto{\pgfqpoint{1.139768in}{0.898820in}}%
\pgfpathclose%
\pgfusepath{fill}%
\end{pgfscope}%
\begin{pgfscope}%
\pgfpathrectangle{\pgfqpoint{0.329460in}{0.284240in}}{\pgfqpoint{1.989680in}{1.989680in}}%
\pgfusepath{clip}%
\pgfsetbuttcap%
\pgfsetroundjoin%
\definecolor{currentfill}{rgb}{0.955300,0.901065,0.118128}%
\pgfsetfillcolor{currentfill}%
\pgfsetlinewidth{0.000000pt}%
\definecolor{currentstroke}{rgb}{0.000000,0.000000,0.000000}%
\pgfsetstrokecolor{currentstroke}%
\pgfsetdash{}{0pt}%
\pgfpathmoveto{\pgfqpoint{1.379127in}{1.737283in}}%
\pgfpathlineto{\pgfqpoint{1.381446in}{1.737013in}}%
\pgfpathlineto{\pgfqpoint{1.383764in}{1.736625in}}%
\pgfpathlineto{\pgfqpoint{1.386081in}{1.736121in}}%
\pgfpathlineto{\pgfqpoint{1.388396in}{1.735502in}}%
\pgfpathlineto{\pgfqpoint{1.389699in}{1.734945in}}%
\pgfpathlineto{\pgfqpoint{1.390964in}{1.734370in}}%
\pgfpathlineto{\pgfqpoint{1.392190in}{1.733776in}}%
\pgfpathlineto{\pgfqpoint{1.393375in}{1.733165in}}%
\pgfpathlineto{\pgfqpoint{1.390749in}{1.733930in}}%
\pgfpathlineto{\pgfqpoint{1.388122in}{1.734580in}}%
\pgfpathlineto{\pgfqpoint{1.385493in}{1.735113in}}%
\pgfpathlineto{\pgfqpoint{1.382863in}{1.735529in}}%
\pgfpathlineto{\pgfqpoint{1.381973in}{1.735988in}}%
\pgfpathlineto{\pgfqpoint{1.381054in}{1.736434in}}%
\pgfpathlineto{\pgfqpoint{1.380105in}{1.736866in}}%
\pgfpathlineto{\pgfqpoint{1.379127in}{1.737283in}}%
\pgfpathclose%
\pgfusepath{fill}%
\end{pgfscope}%
\begin{pgfscope}%
\pgfpathrectangle{\pgfqpoint{0.329460in}{0.284240in}}{\pgfqpoint{1.989680in}{1.989680in}}%
\pgfusepath{clip}%
\pgfsetbuttcap%
\pgfsetroundjoin%
\definecolor{currentfill}{rgb}{0.263663,0.237631,0.518762}%
\pgfsetfillcolor{currentfill}%
\pgfsetlinewidth{0.000000pt}%
\definecolor{currentstroke}{rgb}{0.000000,0.000000,0.000000}%
\pgfsetstrokecolor{currentstroke}%
\pgfsetdash{}{0pt}%
\pgfpathmoveto{\pgfqpoint{1.595053in}{1.000455in}}%
\pgfpathlineto{\pgfqpoint{1.597132in}{0.993123in}}%
\pgfpathlineto{\pgfqpoint{1.599211in}{0.985873in}}%
\pgfpathlineto{\pgfqpoint{1.601291in}{0.978708in}}%
\pgfpathlineto{\pgfqpoint{1.603373in}{0.971630in}}%
\pgfpathlineto{\pgfqpoint{1.592828in}{0.967521in}}%
\pgfpathlineto{\pgfqpoint{1.582023in}{0.963587in}}%
\pgfpathlineto{\pgfqpoint{1.570968in}{0.959832in}}%
\pgfpathlineto{\pgfqpoint{1.559676in}{0.956262in}}%
\pgfpathlineto{\pgfqpoint{1.557951in}{0.963483in}}%
\pgfpathlineto{\pgfqpoint{1.556227in}{0.970791in}}%
\pgfpathlineto{\pgfqpoint{1.554504in}{0.978185in}}%
\pgfpathlineto{\pgfqpoint{1.552782in}{0.985660in}}%
\pgfpathlineto{\pgfqpoint{1.563705in}{0.989097in}}%
\pgfpathlineto{\pgfqpoint{1.574398in}{0.992711in}}%
\pgfpathlineto{\pgfqpoint{1.584851in}{0.996498in}}%
\pgfpathlineto{\pgfqpoint{1.595053in}{1.000455in}}%
\pgfpathclose%
\pgfusepath{fill}%
\end{pgfscope}%
\begin{pgfscope}%
\pgfpathrectangle{\pgfqpoint{0.329460in}{0.284240in}}{\pgfqpoint{1.989680in}{1.989680in}}%
\pgfusepath{clip}%
\pgfsetbuttcap%
\pgfsetroundjoin%
\definecolor{currentfill}{rgb}{0.276194,0.190074,0.493001}%
\pgfsetfillcolor{currentfill}%
\pgfsetlinewidth{0.000000pt}%
\definecolor{currentstroke}{rgb}{0.000000,0.000000,0.000000}%
\pgfsetstrokecolor{currentstroke}%
\pgfsetdash{}{0pt}%
\pgfpathmoveto{\pgfqpoint{1.887103in}{0.943083in}}%
\pgfpathlineto{\pgfqpoint{1.890371in}{0.949300in}}%
\pgfpathlineto{\pgfqpoint{1.893653in}{0.955875in}}%
\pgfpathlineto{\pgfqpoint{1.896948in}{0.962813in}}%
\pgfpathlineto{\pgfqpoint{1.900259in}{0.970121in}}%
\pgfpathlineto{\pgfqpoint{1.890189in}{0.961061in}}%
\pgfpathlineto{\pgfqpoint{1.879538in}{0.952161in}}%
\pgfpathlineto{\pgfqpoint{1.868315in}{0.943432in}}%
\pgfpathlineto{\pgfqpoint{1.856531in}{0.934884in}}%
\pgfpathlineto{\pgfqpoint{1.853467in}{0.927750in}}%
\pgfpathlineto{\pgfqpoint{1.850418in}{0.920988in}}%
\pgfpathlineto{\pgfqpoint{1.847382in}{0.914591in}}%
\pgfpathlineto{\pgfqpoint{1.844359in}{0.908552in}}%
\pgfpathlineto{\pgfqpoint{1.855876in}{0.916928in}}%
\pgfpathlineto{\pgfqpoint{1.866845in}{0.925482in}}%
\pgfpathlineto{\pgfqpoint{1.877257in}{0.934204in}}%
\pgfpathlineto{\pgfqpoint{1.887103in}{0.943083in}}%
\pgfpathclose%
\pgfusepath{fill}%
\end{pgfscope}%
\begin{pgfscope}%
\pgfpathrectangle{\pgfqpoint{0.329460in}{0.284240in}}{\pgfqpoint{1.989680in}{1.989680in}}%
\pgfusepath{clip}%
\pgfsetbuttcap%
\pgfsetroundjoin%
\definecolor{currentfill}{rgb}{0.277941,0.056324,0.381191}%
\pgfsetfillcolor{currentfill}%
\pgfsetlinewidth{0.000000pt}%
\definecolor{currentstroke}{rgb}{0.000000,0.000000,0.000000}%
\pgfsetstrokecolor{currentstroke}%
\pgfsetdash{}{0pt}%
\pgfpathmoveto{\pgfqpoint{0.952784in}{0.826375in}}%
\pgfpathlineto{\pgfqpoint{0.950255in}{0.828344in}}%
\pgfpathlineto{\pgfqpoint{0.947718in}{0.830605in}}%
\pgfpathlineto{\pgfqpoint{0.945172in}{0.833164in}}%
\pgfpathlineto{\pgfqpoint{0.942617in}{0.836026in}}%
\pgfpathlineto{\pgfqpoint{0.929246in}{0.843074in}}%
\pgfpathlineto{\pgfqpoint{0.916339in}{0.850338in}}%
\pgfpathlineto{\pgfqpoint{0.903908in}{0.857809in}}%
\pgfpathlineto{\pgfqpoint{0.891965in}{0.865477in}}%
\pgfpathlineto{\pgfqpoint{0.894824in}{0.862445in}}%
\pgfpathlineto{\pgfqpoint{0.897673in}{0.859714in}}%
\pgfpathlineto{\pgfqpoint{0.900512in}{0.857280in}}%
\pgfpathlineto{\pgfqpoint{0.903343in}{0.855138in}}%
\pgfpathlineto{\pgfqpoint{0.915003in}{0.847648in}}%
\pgfpathlineto{\pgfqpoint{0.927137in}{0.840352in}}%
\pgfpathlineto{\pgfqpoint{0.939735in}{0.833258in}}%
\pgfpathlineto{\pgfqpoint{0.952784in}{0.826375in}}%
\pgfpathclose%
\pgfusepath{fill}%
\end{pgfscope}%
\begin{pgfscope}%
\pgfpathrectangle{\pgfqpoint{0.329460in}{0.284240in}}{\pgfqpoint{1.989680in}{1.989680in}}%
\pgfusepath{clip}%
\pgfsetbuttcap%
\pgfsetroundjoin%
\definecolor{currentfill}{rgb}{0.282327,0.094955,0.417331}%
\pgfsetfillcolor{currentfill}%
\pgfsetlinewidth{0.000000pt}%
\definecolor{currentstroke}{rgb}{0.000000,0.000000,0.000000}%
\pgfsetstrokecolor{currentstroke}%
\pgfsetdash{}{0pt}%
\pgfpathmoveto{\pgfqpoint{1.133159in}{0.874340in}}%
\pgfpathlineto{\pgfqpoint{1.131502in}{0.868543in}}%
\pgfpathlineto{\pgfqpoint{1.129844in}{0.862883in}}%
\pgfpathlineto{\pgfqpoint{1.128184in}{0.857362in}}%
\pgfpathlineto{\pgfqpoint{1.126523in}{0.851985in}}%
\pgfpathlineto{\pgfqpoint{1.113514in}{0.855900in}}%
\pgfpathlineto{\pgfqpoint{1.100766in}{0.860031in}}%
\pgfpathlineto{\pgfqpoint{1.088292in}{0.864373in}}%
\pgfpathlineto{\pgfqpoint{1.076105in}{0.868921in}}%
\pgfpathlineto{\pgfqpoint{1.078134in}{0.874159in}}%
\pgfpathlineto{\pgfqpoint{1.080161in}{0.879541in}}%
\pgfpathlineto{\pgfqpoint{1.082187in}{0.885063in}}%
\pgfpathlineto{\pgfqpoint{1.084210in}{0.890722in}}%
\pgfpathlineto{\pgfqpoint{1.096043in}{0.886323in}}%
\pgfpathlineto{\pgfqpoint{1.108154in}{0.882123in}}%
\pgfpathlineto{\pgfqpoint{1.120530in}{0.878127in}}%
\pgfpathlineto{\pgfqpoint{1.133159in}{0.874340in}}%
\pgfpathclose%
\pgfusepath{fill}%
\end{pgfscope}%
\begin{pgfscope}%
\pgfpathrectangle{\pgfqpoint{0.329460in}{0.284240in}}{\pgfqpoint{1.989680in}{1.989680in}}%
\pgfusepath{clip}%
\pgfsetbuttcap%
\pgfsetroundjoin%
\definecolor{currentfill}{rgb}{0.220124,0.725509,0.466226}%
\pgfsetfillcolor{currentfill}%
\pgfsetlinewidth{0.000000pt}%
\definecolor{currentstroke}{rgb}{0.000000,0.000000,0.000000}%
\pgfsetstrokecolor{currentstroke}%
\pgfsetdash{}{0pt}%
\pgfpathmoveto{\pgfqpoint{1.162989in}{1.450069in}}%
\pgfpathlineto{\pgfqpoint{1.160083in}{1.442841in}}%
\pgfpathlineto{\pgfqpoint{1.157179in}{1.435559in}}%
\pgfpathlineto{\pgfqpoint{1.154277in}{1.428226in}}%
\pgfpathlineto{\pgfqpoint{1.151376in}{1.420844in}}%
\pgfpathlineto{\pgfqpoint{1.147497in}{1.423970in}}%
\pgfpathlineto{\pgfqpoint{1.143827in}{1.427152in}}%
\pgfpathlineto{\pgfqpoint{1.140370in}{1.430388in}}%
\pgfpathlineto{\pgfqpoint{1.137128in}{1.433674in}}%
\pgfpathlineto{\pgfqpoint{1.140229in}{1.440855in}}%
\pgfpathlineto{\pgfqpoint{1.143332in}{1.447987in}}%
\pgfpathlineto{\pgfqpoint{1.146437in}{1.455069in}}%
\pgfpathlineto{\pgfqpoint{1.149544in}{1.462097in}}%
\pgfpathlineto{\pgfqpoint{1.152605in}{1.459016in}}%
\pgfpathlineto{\pgfqpoint{1.155867in}{1.455982in}}%
\pgfpathlineto{\pgfqpoint{1.159330in}{1.452999in}}%
\pgfpathlineto{\pgfqpoint{1.162989in}{1.450069in}}%
\pgfpathclose%
\pgfusepath{fill}%
\end{pgfscope}%
\begin{pgfscope}%
\pgfpathrectangle{\pgfqpoint{0.329460in}{0.284240in}}{\pgfqpoint{1.989680in}{1.989680in}}%
\pgfusepath{clip}%
\pgfsetbuttcap%
\pgfsetroundjoin%
\definecolor{currentfill}{rgb}{0.134692,0.658636,0.517649}%
\pgfsetfillcolor{currentfill}%
\pgfsetlinewidth{0.000000pt}%
\definecolor{currentstroke}{rgb}{0.000000,0.000000,0.000000}%
\pgfsetstrokecolor{currentstroke}%
\pgfsetdash{}{0pt}%
\pgfpathmoveto{\pgfqpoint{1.566239in}{1.393813in}}%
\pgfpathlineto{\pgfqpoint{1.569180in}{1.386260in}}%
\pgfpathlineto{\pgfqpoint{1.572120in}{1.378670in}}%
\pgfpathlineto{\pgfqpoint{1.575057in}{1.371045in}}%
\pgfpathlineto{\pgfqpoint{1.577993in}{1.363388in}}%
\pgfpathlineto{\pgfqpoint{1.573653in}{1.359874in}}%
\pgfpathlineto{\pgfqpoint{1.569083in}{1.356427in}}%
\pgfpathlineto{\pgfqpoint{1.564288in}{1.353051in}}%
\pgfpathlineto{\pgfqpoint{1.559273in}{1.349750in}}%
\pgfpathlineto{\pgfqpoint{1.556573in}{1.357601in}}%
\pgfpathlineto{\pgfqpoint{1.553872in}{1.365418in}}%
\pgfpathlineto{\pgfqpoint{1.551170in}{1.373201in}}%
\pgfpathlineto{\pgfqpoint{1.548466in}{1.380948in}}%
\pgfpathlineto{\pgfqpoint{1.553226in}{1.384061in}}%
\pgfpathlineto{\pgfqpoint{1.557778in}{1.387245in}}%
\pgfpathlineto{\pgfqpoint{1.562117in}{1.390497in}}%
\pgfpathlineto{\pgfqpoint{1.566239in}{1.393813in}}%
\pgfpathclose%
\pgfusepath{fill}%
\end{pgfscope}%
\begin{pgfscope}%
\pgfpathrectangle{\pgfqpoint{0.329460in}{0.284240in}}{\pgfqpoint{1.989680in}{1.989680in}}%
\pgfusepath{clip}%
\pgfsetbuttcap%
\pgfsetroundjoin%
\definecolor{currentfill}{rgb}{0.267004,0.004874,0.329415}%
\pgfsetfillcolor{currentfill}%
\pgfsetlinewidth{0.000000pt}%
\definecolor{currentstroke}{rgb}{0.000000,0.000000,0.000000}%
\pgfsetstrokecolor{currentstroke}%
\pgfsetdash{}{0pt}%
\pgfpathmoveto{\pgfqpoint{1.034860in}{0.800313in}}%
\pgfpathlineto{\pgfqpoint{1.032755in}{0.799006in}}%
\pgfpathlineto{\pgfqpoint{1.030644in}{0.797933in}}%
\pgfpathlineto{\pgfqpoint{1.028529in}{0.797099in}}%
\pgfpathlineto{\pgfqpoint{1.026408in}{0.796507in}}%
\pgfpathlineto{\pgfqpoint{1.012416in}{0.802173in}}%
\pgfpathlineto{\pgfqpoint{0.998798in}{0.808069in}}%
\pgfpathlineto{\pgfqpoint{0.985567in}{0.814189in}}%
\pgfpathlineto{\pgfqpoint{0.972738in}{0.820525in}}%
\pgfpathlineto{\pgfqpoint{0.975201in}{0.820958in}}%
\pgfpathlineto{\pgfqpoint{0.977657in}{0.821634in}}%
\pgfpathlineto{\pgfqpoint{0.980108in}{0.822547in}}%
\pgfpathlineto{\pgfqpoint{0.982553in}{0.823695in}}%
\pgfpathlineto{\pgfqpoint{0.995058in}{0.817526in}}%
\pgfpathlineto{\pgfqpoint{1.007953in}{0.811568in}}%
\pgfpathlineto{\pgfqpoint{1.021225in}{0.805828in}}%
\pgfpathlineto{\pgfqpoint{1.034860in}{0.800313in}}%
\pgfpathclose%
\pgfusepath{fill}%
\end{pgfscope}%
\begin{pgfscope}%
\pgfpathrectangle{\pgfqpoint{0.329460in}{0.284240in}}{\pgfqpoint{1.989680in}{1.989680in}}%
\pgfusepath{clip}%
\pgfsetbuttcap%
\pgfsetroundjoin%
\definecolor{currentfill}{rgb}{0.195860,0.395433,0.555276}%
\pgfsetfillcolor{currentfill}%
\pgfsetlinewidth{0.000000pt}%
\definecolor{currentstroke}{rgb}{0.000000,0.000000,0.000000}%
\pgfsetstrokecolor{currentstroke}%
\pgfsetdash{}{0pt}%
\pgfpathmoveto{\pgfqpoint{1.148378in}{1.123337in}}%
\pgfpathlineto{\pgfqpoint{1.146379in}{1.115085in}}%
\pgfpathlineto{\pgfqpoint{1.144379in}{1.106867in}}%
\pgfpathlineto{\pgfqpoint{1.142379in}{1.098685in}}%
\pgfpathlineto{\pgfqpoint{1.140380in}{1.090541in}}%
\pgfpathlineto{\pgfqpoint{1.131233in}{1.094059in}}%
\pgfpathlineto{\pgfqpoint{1.122321in}{1.097722in}}%
\pgfpathlineto{\pgfqpoint{1.113652in}{1.101526in}}%
\pgfpathlineto{\pgfqpoint{1.105236in}{1.105467in}}%
\pgfpathlineto{\pgfqpoint{1.107563in}{1.113451in}}%
\pgfpathlineto{\pgfqpoint{1.109891in}{1.121473in}}%
\pgfpathlineto{\pgfqpoint{1.112218in}{1.129532in}}%
\pgfpathlineto{\pgfqpoint{1.114547in}{1.137624in}}%
\pgfpathlineto{\pgfqpoint{1.122650in}{1.133851in}}%
\pgfpathlineto{\pgfqpoint{1.130995in}{1.130210in}}%
\pgfpathlineto{\pgfqpoint{1.139574in}{1.126704in}}%
\pgfpathlineto{\pgfqpoint{1.148378in}{1.123337in}}%
\pgfpathclose%
\pgfusepath{fill}%
\end{pgfscope}%
\begin{pgfscope}%
\pgfpathrectangle{\pgfqpoint{0.329460in}{0.284240in}}{\pgfqpoint{1.989680in}{1.989680in}}%
\pgfusepath{clip}%
\pgfsetbuttcap%
\pgfsetroundjoin%
\definecolor{currentfill}{rgb}{0.280255,0.165693,0.476498}%
\pgfsetfillcolor{currentfill}%
\pgfsetlinewidth{0.000000pt}%
\definecolor{currentstroke}{rgb}{0.000000,0.000000,0.000000}%
\pgfsetstrokecolor{currentstroke}%
\pgfsetdash{}{0pt}%
\pgfpathmoveto{\pgfqpoint{1.146356in}{0.925196in}}%
\pgfpathlineto{\pgfqpoint{1.144711in}{0.918436in}}%
\pgfpathlineto{\pgfqpoint{1.143065in}{0.911785in}}%
\pgfpathlineto{\pgfqpoint{1.141417in}{0.905245in}}%
\pgfpathlineto{\pgfqpoint{1.139768in}{0.898820in}}%
\pgfpathlineto{\pgfqpoint{1.127519in}{0.902479in}}%
\pgfpathlineto{\pgfqpoint{1.115513in}{0.906339in}}%
\pgfpathlineto{\pgfqpoint{1.103764in}{0.910397in}}%
\pgfpathlineto{\pgfqpoint{1.092284in}{0.914648in}}%
\pgfpathlineto{\pgfqpoint{1.094298in}{0.920934in}}%
\pgfpathlineto{\pgfqpoint{1.096311in}{0.927336in}}%
\pgfpathlineto{\pgfqpoint{1.098323in}{0.933849in}}%
\pgfpathlineto{\pgfqpoint{1.100333in}{0.940470in}}%
\pgfpathlineto{\pgfqpoint{1.111461in}{0.936368in}}%
\pgfpathlineto{\pgfqpoint{1.122849in}{0.932452in}}%
\pgfpathlineto{\pgfqpoint{1.134485in}{0.928727in}}%
\pgfpathlineto{\pgfqpoint{1.146356in}{0.925196in}}%
\pgfpathclose%
\pgfusepath{fill}%
\end{pgfscope}%
\begin{pgfscope}%
\pgfpathrectangle{\pgfqpoint{0.329460in}{0.284240in}}{\pgfqpoint{1.989680in}{1.989680in}}%
\pgfusepath{clip}%
\pgfsetbuttcap%
\pgfsetroundjoin%
\definecolor{currentfill}{rgb}{0.955300,0.901065,0.118128}%
\pgfsetfillcolor{currentfill}%
\pgfsetlinewidth{0.000000pt}%
\definecolor{currentstroke}{rgb}{0.000000,0.000000,0.000000}%
\pgfsetstrokecolor{currentstroke}%
\pgfsetdash{}{0pt}%
\pgfpathmoveto{\pgfqpoint{1.318749in}{1.735111in}}%
\pgfpathlineto{\pgfqpoint{1.316054in}{1.734659in}}%
\pgfpathlineto{\pgfqpoint{1.313362in}{1.734091in}}%
\pgfpathlineto{\pgfqpoint{1.310671in}{1.733407in}}%
\pgfpathlineto{\pgfqpoint{1.307982in}{1.732607in}}%
\pgfpathlineto{\pgfqpoint{1.309130in}{1.733234in}}%
\pgfpathlineto{\pgfqpoint{1.310320in}{1.733843in}}%
\pgfpathlineto{\pgfqpoint{1.311550in}{1.734435in}}%
\pgfpathlineto{\pgfqpoint{1.312819in}{1.735008in}}%
\pgfpathlineto{\pgfqpoint{1.315207in}{1.735659in}}%
\pgfpathlineto{\pgfqpoint{1.317595in}{1.736193in}}%
\pgfpathlineto{\pgfqpoint{1.319986in}{1.736612in}}%
\pgfpathlineto{\pgfqpoint{1.322378in}{1.736913in}}%
\pgfpathlineto{\pgfqpoint{1.321425in}{1.736483in}}%
\pgfpathlineto{\pgfqpoint{1.320502in}{1.736039in}}%
\pgfpathlineto{\pgfqpoint{1.319610in}{1.735581in}}%
\pgfpathlineto{\pgfqpoint{1.318749in}{1.735111in}}%
\pgfpathclose%
\pgfusepath{fill}%
\end{pgfscope}%
\begin{pgfscope}%
\pgfpathrectangle{\pgfqpoint{0.329460in}{0.284240in}}{\pgfqpoint{1.989680in}{1.989680in}}%
\pgfusepath{clip}%
\pgfsetbuttcap%
\pgfsetroundjoin%
\definecolor{currentfill}{rgb}{0.412913,0.803041,0.357269}%
\pgfsetfillcolor{currentfill}%
\pgfsetlinewidth{0.000000pt}%
\definecolor{currentstroke}{rgb}{0.000000,0.000000,0.000000}%
\pgfsetstrokecolor{currentstroke}%
\pgfsetdash{}{0pt}%
\pgfpathmoveto{\pgfqpoint{1.525218in}{1.554178in}}%
\pgfpathlineto{\pgfqpoint{1.528527in}{1.548185in}}%
\pgfpathlineto{\pgfqpoint{1.531834in}{1.542117in}}%
\pgfpathlineto{\pgfqpoint{1.535140in}{1.535977in}}%
\pgfpathlineto{\pgfqpoint{1.538443in}{1.529766in}}%
\pgfpathlineto{\pgfqpoint{1.536649in}{1.526929in}}%
\pgfpathlineto{\pgfqpoint{1.534667in}{1.524119in}}%
\pgfpathlineto{\pgfqpoint{1.532499in}{1.521340in}}%
\pgfpathlineto{\pgfqpoint{1.530148in}{1.518594in}}%
\pgfpathlineto{\pgfqpoint{1.526986in}{1.525014in}}%
\pgfpathlineto{\pgfqpoint{1.523821in}{1.531363in}}%
\pgfpathlineto{\pgfqpoint{1.520655in}{1.537638in}}%
\pgfpathlineto{\pgfqpoint{1.517487in}{1.543839in}}%
\pgfpathlineto{\pgfqpoint{1.519678in}{1.546380in}}%
\pgfpathlineto{\pgfqpoint{1.521697in}{1.548952in}}%
\pgfpathlineto{\pgfqpoint{1.523545in}{1.551552in}}%
\pgfpathlineto{\pgfqpoint{1.525218in}{1.554178in}}%
\pgfpathclose%
\pgfusepath{fill}%
\end{pgfscope}%
\begin{pgfscope}%
\pgfpathrectangle{\pgfqpoint{0.329460in}{0.284240in}}{\pgfqpoint{1.989680in}{1.989680in}}%
\pgfusepath{clip}%
\pgfsetbuttcap%
\pgfsetroundjoin%
\definecolor{currentfill}{rgb}{0.699415,0.867117,0.175971}%
\pgfsetfillcolor{currentfill}%
\pgfsetlinewidth{0.000000pt}%
\definecolor{currentstroke}{rgb}{0.000000,0.000000,0.000000}%
\pgfsetstrokecolor{currentstroke}%
\pgfsetdash{}{0pt}%
\pgfpathmoveto{\pgfqpoint{1.476986in}{1.653456in}}%
\pgfpathlineto{\pgfqpoint{1.480454in}{1.649282in}}%
\pgfpathlineto{\pgfqpoint{1.483920in}{1.645012in}}%
\pgfpathlineto{\pgfqpoint{1.487383in}{1.640645in}}%
\pgfpathlineto{\pgfqpoint{1.490845in}{1.636184in}}%
\pgfpathlineto{\pgfqpoint{1.490643in}{1.634098in}}%
\pgfpathlineto{\pgfqpoint{1.490302in}{1.632014in}}%
\pgfpathlineto{\pgfqpoint{1.489821in}{1.629936in}}%
\pgfpathlineto{\pgfqpoint{1.489202in}{1.627866in}}%
\pgfpathlineto{\pgfqpoint{1.485777in}{1.632540in}}%
\pgfpathlineto{\pgfqpoint{1.482349in}{1.637119in}}%
\pgfpathlineto{\pgfqpoint{1.478920in}{1.641601in}}%
\pgfpathlineto{\pgfqpoint{1.475489in}{1.645987in}}%
\pgfpathlineto{\pgfqpoint{1.476051in}{1.647846in}}%
\pgfpathlineto{\pgfqpoint{1.476488in}{1.649711in}}%
\pgfpathlineto{\pgfqpoint{1.476800in}{1.651582in}}%
\pgfpathlineto{\pgfqpoint{1.476986in}{1.653456in}}%
\pgfpathclose%
\pgfusepath{fill}%
\end{pgfscope}%
\begin{pgfscope}%
\pgfpathrectangle{\pgfqpoint{0.329460in}{0.284240in}}{\pgfqpoint{1.989680in}{1.989680in}}%
\pgfusepath{clip}%
\pgfsetbuttcap%
\pgfsetroundjoin%
\definecolor{currentfill}{rgb}{0.179019,0.433756,0.557430}%
\pgfsetfillcolor{currentfill}%
\pgfsetlinewidth{0.000000pt}%
\definecolor{currentstroke}{rgb}{0.000000,0.000000,0.000000}%
\pgfsetstrokecolor{currentstroke}%
\pgfsetdash{}{0pt}%
\pgfpathmoveto{\pgfqpoint{1.585234in}{1.173577in}}%
\pgfpathlineto{\pgfqpoint{1.587631in}{1.165418in}}%
\pgfpathlineto{\pgfqpoint{1.590028in}{1.157281in}}%
\pgfpathlineto{\pgfqpoint{1.592425in}{1.149169in}}%
\pgfpathlineto{\pgfqpoint{1.594821in}{1.141085in}}%
\pgfpathlineto{\pgfqpoint{1.586940in}{1.137199in}}%
\pgfpathlineto{\pgfqpoint{1.578810in}{1.133440in}}%
\pgfpathlineto{\pgfqpoint{1.570438in}{1.129813in}}%
\pgfpathlineto{\pgfqpoint{1.561834in}{1.126323in}}%
\pgfpathlineto{\pgfqpoint{1.559757in}{1.134571in}}%
\pgfpathlineto{\pgfqpoint{1.557680in}{1.142847in}}%
\pgfpathlineto{\pgfqpoint{1.555603in}{1.151147in}}%
\pgfpathlineto{\pgfqpoint{1.553526in}{1.159470in}}%
\pgfpathlineto{\pgfqpoint{1.561795in}{1.162805in}}%
\pgfpathlineto{\pgfqpoint{1.569842in}{1.166271in}}%
\pgfpathlineto{\pgfqpoint{1.577657in}{1.169863in}}%
\pgfpathlineto{\pgfqpoint{1.585234in}{1.173577in}}%
\pgfpathclose%
\pgfusepath{fill}%
\end{pgfscope}%
\begin{pgfscope}%
\pgfpathrectangle{\pgfqpoint{0.329460in}{0.284240in}}{\pgfqpoint{1.989680in}{1.989680in}}%
\pgfusepath{clip}%
\pgfsetbuttcap%
\pgfsetroundjoin%
\definecolor{currentfill}{rgb}{0.279566,0.067836,0.391917}%
\pgfsetfillcolor{currentfill}%
\pgfsetlinewidth{0.000000pt}%
\definecolor{currentstroke}{rgb}{0.000000,0.000000,0.000000}%
\pgfsetstrokecolor{currentstroke}%
\pgfsetdash{}{0pt}%
\pgfpathmoveto{\pgfqpoint{1.126523in}{0.851985in}}%
\pgfpathlineto{\pgfqpoint{1.124859in}{0.846755in}}%
\pgfpathlineto{\pgfqpoint{1.123193in}{0.841677in}}%
\pgfpathlineto{\pgfqpoint{1.121525in}{0.836754in}}%
\pgfpathlineto{\pgfqpoint{1.119855in}{0.831990in}}%
\pgfpathlineto{\pgfqpoint{1.106465in}{0.836034in}}%
\pgfpathlineto{\pgfqpoint{1.093343in}{0.840300in}}%
\pgfpathlineto{\pgfqpoint{1.080504in}{0.844783in}}%
\pgfpathlineto{\pgfqpoint{1.067962in}{0.849479in}}%
\pgfpathlineto{\pgfqpoint{1.070001in}{0.854105in}}%
\pgfpathlineto{\pgfqpoint{1.072038in}{0.858890in}}%
\pgfpathlineto{\pgfqpoint{1.074073in}{0.863830in}}%
\pgfpathlineto{\pgfqpoint{1.076105in}{0.868921in}}%
\pgfpathlineto{\pgfqpoint{1.088292in}{0.864373in}}%
\pgfpathlineto{\pgfqpoint{1.100766in}{0.860031in}}%
\pgfpathlineto{\pgfqpoint{1.113514in}{0.855900in}}%
\pgfpathlineto{\pgfqpoint{1.126523in}{0.851985in}}%
\pgfpathclose%
\pgfusepath{fill}%
\end{pgfscope}%
\begin{pgfscope}%
\pgfpathrectangle{\pgfqpoint{0.329460in}{0.284240in}}{\pgfqpoint{1.989680in}{1.989680in}}%
\pgfusepath{clip}%
\pgfsetbuttcap%
\pgfsetroundjoin%
\definecolor{currentfill}{rgb}{0.896320,0.893616,0.096335}%
\pgfsetfillcolor{currentfill}%
\pgfsetlinewidth{0.000000pt}%
\definecolor{currentstroke}{rgb}{0.000000,0.000000,0.000000}%
\pgfsetstrokecolor{currentstroke}%
\pgfsetdash{}{0pt}%
\pgfpathmoveto{\pgfqpoint{1.417213in}{1.718343in}}%
\pgfpathlineto{\pgfqpoint{1.420500in}{1.716477in}}%
\pgfpathlineto{\pgfqpoint{1.423784in}{1.714499in}}%
\pgfpathlineto{\pgfqpoint{1.427066in}{1.712409in}}%
\pgfpathlineto{\pgfqpoint{1.430346in}{1.710209in}}%
\pgfpathlineto{\pgfqpoint{1.431200in}{1.709034in}}%
\pgfpathlineto{\pgfqpoint{1.431975in}{1.707847in}}%
\pgfpathlineto{\pgfqpoint{1.432670in}{1.706649in}}%
\pgfpathlineto{\pgfqpoint{1.433284in}{1.705440in}}%
\pgfpathlineto{\pgfqpoint{1.429879in}{1.707840in}}%
\pgfpathlineto{\pgfqpoint{1.426473in}{1.710129in}}%
\pgfpathlineto{\pgfqpoint{1.423065in}{1.712307in}}%
\pgfpathlineto{\pgfqpoint{1.419655in}{1.714372in}}%
\pgfpathlineto{\pgfqpoint{1.419145in}{1.715378in}}%
\pgfpathlineto{\pgfqpoint{1.418568in}{1.716376in}}%
\pgfpathlineto{\pgfqpoint{1.417924in}{1.717365in}}%
\pgfpathlineto{\pgfqpoint{1.417213in}{1.718343in}}%
\pgfpathclose%
\pgfusepath{fill}%
\end{pgfscope}%
\begin{pgfscope}%
\pgfpathrectangle{\pgfqpoint{0.329460in}{0.284240in}}{\pgfqpoint{1.989680in}{1.989680in}}%
\pgfusepath{clip}%
\pgfsetbuttcap%
\pgfsetroundjoin%
\definecolor{currentfill}{rgb}{0.855810,0.888601,0.097452}%
\pgfsetfillcolor{currentfill}%
\pgfsetlinewidth{0.000000pt}%
\definecolor{currentstroke}{rgb}{0.000000,0.000000,0.000000}%
\pgfsetstrokecolor{currentstroke}%
\pgfsetdash{}{0pt}%
\pgfpathmoveto{\pgfqpoint{1.433284in}{1.705440in}}%
\pgfpathlineto{\pgfqpoint{1.436687in}{1.702931in}}%
\pgfpathlineto{\pgfqpoint{1.440088in}{1.700312in}}%
\pgfpathlineto{\pgfqpoint{1.443487in}{1.697585in}}%
\pgfpathlineto{\pgfqpoint{1.446883in}{1.694751in}}%
\pgfpathlineto{\pgfqpoint{1.447507in}{1.693330in}}%
\pgfpathlineto{\pgfqpoint{1.448036in}{1.691900in}}%
\pgfpathlineto{\pgfqpoint{1.448468in}{1.690463in}}%
\pgfpathlineto{\pgfqpoint{1.448804in}{1.689019in}}%
\pgfpathlineto{\pgfqpoint{1.445335in}{1.692060in}}%
\pgfpathlineto{\pgfqpoint{1.441865in}{1.694993in}}%
\pgfpathlineto{\pgfqpoint{1.438393in}{1.697817in}}%
\pgfpathlineto{\pgfqpoint{1.434920in}{1.700533in}}%
\pgfpathlineto{\pgfqpoint{1.434635in}{1.701769in}}%
\pgfpathlineto{\pgfqpoint{1.434267in}{1.702999in}}%
\pgfpathlineto{\pgfqpoint{1.433817in}{1.704224in}}%
\pgfpathlineto{\pgfqpoint{1.433284in}{1.705440in}}%
\pgfpathclose%
\pgfusepath{fill}%
\end{pgfscope}%
\begin{pgfscope}%
\pgfpathrectangle{\pgfqpoint{0.329460in}{0.284240in}}{\pgfqpoint{1.989680in}{1.989680in}}%
\pgfusepath{clip}%
\pgfsetbuttcap%
\pgfsetroundjoin%
\definecolor{currentfill}{rgb}{0.274128,0.199721,0.498911}%
\pgfsetfillcolor{currentfill}%
\pgfsetlinewidth{0.000000pt}%
\definecolor{currentstroke}{rgb}{0.000000,0.000000,0.000000}%
\pgfsetstrokecolor{currentstroke}%
\pgfsetdash{}{0pt}%
\pgfpathmoveto{\pgfqpoint{1.152927in}{0.953246in}}%
\pgfpathlineto{\pgfqpoint{1.151286in}{0.946088in}}%
\pgfpathlineto{\pgfqpoint{1.149644in}{0.939025in}}%
\pgfpathlineto{\pgfqpoint{1.148000in}{0.932060in}}%
\pgfpathlineto{\pgfqpoint{1.146356in}{0.925196in}}%
\pgfpathlineto{\pgfqpoint{1.134485in}{0.928727in}}%
\pgfpathlineto{\pgfqpoint{1.122849in}{0.932452in}}%
\pgfpathlineto{\pgfqpoint{1.111461in}{0.936368in}}%
\pgfpathlineto{\pgfqpoint{1.100333in}{0.940470in}}%
\pgfpathlineto{\pgfqpoint{1.102342in}{0.947196in}}%
\pgfpathlineto{\pgfqpoint{1.104350in}{0.954024in}}%
\pgfpathlineto{\pgfqpoint{1.106357in}{0.960949in}}%
\pgfpathlineto{\pgfqpoint{1.108362in}{0.967969in}}%
\pgfpathlineto{\pgfqpoint{1.119139in}{0.964015in}}%
\pgfpathlineto{\pgfqpoint{1.130167in}{0.960240in}}%
\pgfpathlineto{\pgfqpoint{1.141433in}{0.956649in}}%
\pgfpathlineto{\pgfqpoint{1.152927in}{0.953246in}}%
\pgfpathclose%
\pgfusepath{fill}%
\end{pgfscope}%
\begin{pgfscope}%
\pgfpathrectangle{\pgfqpoint{0.329460in}{0.284240in}}{\pgfqpoint{1.989680in}{1.989680in}}%
\pgfusepath{clip}%
\pgfsetbuttcap%
\pgfsetroundjoin%
\definecolor{currentfill}{rgb}{0.974417,0.903590,0.130215}%
\pgfsetfillcolor{currentfill}%
\pgfsetlinewidth{0.000000pt}%
\definecolor{currentstroke}{rgb}{0.000000,0.000000,0.000000}%
\pgfsetstrokecolor{currentstroke}%
\pgfsetdash{}{0pt}%
\pgfpathmoveto{\pgfqpoint{1.347708in}{1.740390in}}%
\pgfpathlineto{\pgfqpoint{1.347275in}{1.740990in}}%
\pgfpathlineto{\pgfqpoint{1.346841in}{1.741472in}}%
\pgfpathlineto{\pgfqpoint{1.346408in}{1.741835in}}%
\pgfpathlineto{\pgfqpoint{1.345976in}{1.742080in}}%
\pgfpathlineto{\pgfqpoint{1.347293in}{1.742147in}}%
\pgfpathlineto{\pgfqpoint{1.348613in}{1.742195in}}%
\pgfpathlineto{\pgfqpoint{1.349937in}{1.742223in}}%
\pgfpathlineto{\pgfqpoint{1.351261in}{1.742231in}}%
\pgfpathlineto{\pgfqpoint{1.351255in}{1.741974in}}%
\pgfpathlineto{\pgfqpoint{1.351249in}{1.741598in}}%
\pgfpathlineto{\pgfqpoint{1.351243in}{1.741104in}}%
\pgfpathlineto{\pgfqpoint{1.351237in}{1.740491in}}%
\pgfpathlineto{\pgfqpoint{1.350352in}{1.740485in}}%
\pgfpathlineto{\pgfqpoint{1.349469in}{1.740466in}}%
\pgfpathlineto{\pgfqpoint{1.348587in}{1.740434in}}%
\pgfpathlineto{\pgfqpoint{1.347708in}{1.740390in}}%
\pgfpathclose%
\pgfusepath{fill}%
\end{pgfscope}%
\begin{pgfscope}%
\pgfpathrectangle{\pgfqpoint{0.329460in}{0.284240in}}{\pgfqpoint{1.989680in}{1.989680in}}%
\pgfusepath{clip}%
\pgfsetbuttcap%
\pgfsetroundjoin%
\definecolor{currentfill}{rgb}{0.974417,0.903590,0.130215}%
\pgfsetfillcolor{currentfill}%
\pgfsetlinewidth{0.000000pt}%
\definecolor{currentstroke}{rgb}{0.000000,0.000000,0.000000}%
\pgfsetstrokecolor{currentstroke}%
\pgfsetdash{}{0pt}%
\pgfpathmoveto{\pgfqpoint{1.351237in}{1.740491in}}%
\pgfpathlineto{\pgfqpoint{1.351243in}{1.741104in}}%
\pgfpathlineto{\pgfqpoint{1.351249in}{1.741598in}}%
\pgfpathlineto{\pgfqpoint{1.351255in}{1.741974in}}%
\pgfpathlineto{\pgfqpoint{1.351261in}{1.742231in}}%
\pgfpathlineto{\pgfqpoint{1.352586in}{1.742220in}}%
\pgfpathlineto{\pgfqpoint{1.353909in}{1.742190in}}%
\pgfpathlineto{\pgfqpoint{1.355229in}{1.742141in}}%
\pgfpathlineto{\pgfqpoint{1.356546in}{1.742072in}}%
\pgfpathlineto{\pgfqpoint{1.356101in}{1.741827in}}%
\pgfpathlineto{\pgfqpoint{1.355656in}{1.741465in}}%
\pgfpathlineto{\pgfqpoint{1.355210in}{1.740984in}}%
\pgfpathlineto{\pgfqpoint{1.354764in}{1.740384in}}%
\pgfpathlineto{\pgfqpoint{1.353886in}{1.740430in}}%
\pgfpathlineto{\pgfqpoint{1.353004in}{1.740463in}}%
\pgfpathlineto{\pgfqpoint{1.352121in}{1.740483in}}%
\pgfpathlineto{\pgfqpoint{1.351237in}{1.740491in}}%
\pgfpathclose%
\pgfusepath{fill}%
\end{pgfscope}%
\begin{pgfscope}%
\pgfpathrectangle{\pgfqpoint{0.329460in}{0.284240in}}{\pgfqpoint{1.989680in}{1.989680in}}%
\pgfusepath{clip}%
\pgfsetbuttcap%
\pgfsetroundjoin%
\definecolor{currentfill}{rgb}{0.974417,0.903590,0.130215}%
\pgfsetfillcolor{currentfill}%
\pgfsetlinewidth{0.000000pt}%
\definecolor{currentstroke}{rgb}{0.000000,0.000000,0.000000}%
\pgfsetstrokecolor{currentstroke}%
\pgfsetdash{}{0pt}%
\pgfpathmoveto{\pgfqpoint{1.344235in}{1.740083in}}%
\pgfpathlineto{\pgfqpoint{1.343368in}{1.740645in}}%
\pgfpathlineto{\pgfqpoint{1.342502in}{1.741089in}}%
\pgfpathlineto{\pgfqpoint{1.341637in}{1.741414in}}%
\pgfpathlineto{\pgfqpoint{1.340772in}{1.741621in}}%
\pgfpathlineto{\pgfqpoint{1.342061in}{1.741764in}}%
\pgfpathlineto{\pgfqpoint{1.343359in}{1.741889in}}%
\pgfpathlineto{\pgfqpoint{1.344664in}{1.741994in}}%
\pgfpathlineto{\pgfqpoint{1.345976in}{1.742080in}}%
\pgfpathlineto{\pgfqpoint{1.346408in}{1.741835in}}%
\pgfpathlineto{\pgfqpoint{1.346841in}{1.741472in}}%
\pgfpathlineto{\pgfqpoint{1.347275in}{1.740990in}}%
\pgfpathlineto{\pgfqpoint{1.347708in}{1.740390in}}%
\pgfpathlineto{\pgfqpoint{1.346833in}{1.740332in}}%
\pgfpathlineto{\pgfqpoint{1.345961in}{1.740262in}}%
\pgfpathlineto{\pgfqpoint{1.345095in}{1.740179in}}%
\pgfpathlineto{\pgfqpoint{1.344235in}{1.740083in}}%
\pgfpathclose%
\pgfusepath{fill}%
\end{pgfscope}%
\begin{pgfscope}%
\pgfpathrectangle{\pgfqpoint{0.329460in}{0.284240in}}{\pgfqpoint{1.989680in}{1.989680in}}%
\pgfusepath{clip}%
\pgfsetbuttcap%
\pgfsetroundjoin%
\definecolor{currentfill}{rgb}{0.974417,0.903590,0.130215}%
\pgfsetfillcolor{currentfill}%
\pgfsetlinewidth{0.000000pt}%
\definecolor{currentstroke}{rgb}{0.000000,0.000000,0.000000}%
\pgfsetstrokecolor{currentstroke}%
\pgfsetdash{}{0pt}%
\pgfpathmoveto{\pgfqpoint{1.354764in}{1.740384in}}%
\pgfpathlineto{\pgfqpoint{1.355210in}{1.740984in}}%
\pgfpathlineto{\pgfqpoint{1.355656in}{1.741465in}}%
\pgfpathlineto{\pgfqpoint{1.356101in}{1.741827in}}%
\pgfpathlineto{\pgfqpoint{1.356546in}{1.742072in}}%
\pgfpathlineto{\pgfqpoint{1.357857in}{1.741983in}}%
\pgfpathlineto{\pgfqpoint{1.359161in}{1.741876in}}%
\pgfpathlineto{\pgfqpoint{1.360458in}{1.741749in}}%
\pgfpathlineto{\pgfqpoint{1.361745in}{1.741604in}}%
\pgfpathlineto{\pgfqpoint{1.360869in}{1.741398in}}%
\pgfpathlineto{\pgfqpoint{1.359992in}{1.741075in}}%
\pgfpathlineto{\pgfqpoint{1.359114in}{1.740632in}}%
\pgfpathlineto{\pgfqpoint{1.358236in}{1.740072in}}%
\pgfpathlineto{\pgfqpoint{1.357376in}{1.740169in}}%
\pgfpathlineto{\pgfqpoint{1.356511in}{1.740253in}}%
\pgfpathlineto{\pgfqpoint{1.355640in}{1.740325in}}%
\pgfpathlineto{\pgfqpoint{1.354764in}{1.740384in}}%
\pgfpathclose%
\pgfusepath{fill}%
\end{pgfscope}%
\begin{pgfscope}%
\pgfpathrectangle{\pgfqpoint{0.329460in}{0.284240in}}{\pgfqpoint{1.989680in}{1.989680in}}%
\pgfusepath{clip}%
\pgfsetbuttcap%
\pgfsetroundjoin%
\definecolor{currentfill}{rgb}{0.271305,0.019942,0.347269}%
\pgfsetfillcolor{currentfill}%
\pgfsetlinewidth{0.000000pt}%
\definecolor{currentstroke}{rgb}{0.000000,0.000000,0.000000}%
\pgfsetstrokecolor{currentstroke}%
\pgfsetdash{}{0pt}%
\pgfpathmoveto{\pgfqpoint{1.653800in}{0.837125in}}%
\pgfpathlineto{\pgfqpoint{1.655932in}{0.833386in}}%
\pgfpathlineto{\pgfqpoint{1.658068in}{0.829829in}}%
\pgfpathlineto{\pgfqpoint{1.660208in}{0.826458in}}%
\pgfpathlineto{\pgfqpoint{1.662351in}{0.823278in}}%
\pgfpathlineto{\pgfqpoint{1.649384in}{0.818092in}}%
\pgfpathlineto{\pgfqpoint{1.636090in}{0.813125in}}%
\pgfpathlineto{\pgfqpoint{1.622483in}{0.808385in}}%
\pgfpathlineto{\pgfqpoint{1.608576in}{0.803875in}}%
\pgfpathlineto{\pgfqpoint{1.606798in}{0.807198in}}%
\pgfpathlineto{\pgfqpoint{1.605023in}{0.810712in}}%
\pgfpathlineto{\pgfqpoint{1.603252in}{0.814413in}}%
\pgfpathlineto{\pgfqpoint{1.601483in}{0.818296in}}%
\pgfpathlineto{\pgfqpoint{1.615011in}{0.822672in}}%
\pgfpathlineto{\pgfqpoint{1.628249in}{0.827272in}}%
\pgfpathlineto{\pgfqpoint{1.641183in}{0.832092in}}%
\pgfpathlineto{\pgfqpoint{1.653800in}{0.837125in}}%
\pgfpathclose%
\pgfusepath{fill}%
\end{pgfscope}%
\begin{pgfscope}%
\pgfpathrectangle{\pgfqpoint{0.329460in}{0.284240in}}{\pgfqpoint{1.989680in}{1.989680in}}%
\pgfusepath{clip}%
\pgfsetbuttcap%
\pgfsetroundjoin%
\definecolor{currentfill}{rgb}{0.565498,0.842430,0.262877}%
\pgfsetfillcolor{currentfill}%
\pgfsetlinewidth{0.000000pt}%
\definecolor{currentstroke}{rgb}{0.000000,0.000000,0.000000}%
\pgfsetstrokecolor{currentstroke}%
\pgfsetdash{}{0pt}%
\pgfpathmoveto{\pgfqpoint{1.204968in}{1.597280in}}%
\pgfpathlineto{\pgfqpoint{1.201670in}{1.591895in}}%
\pgfpathlineto{\pgfqpoint{1.198374in}{1.586422in}}%
\pgfpathlineto{\pgfqpoint{1.195079in}{1.580864in}}%
\pgfpathlineto{\pgfqpoint{1.191787in}{1.575222in}}%
\pgfpathlineto{\pgfqpoint{1.190254in}{1.577641in}}%
\pgfpathlineto{\pgfqpoint{1.188883in}{1.580080in}}%
\pgfpathlineto{\pgfqpoint{1.187675in}{1.582538in}}%
\pgfpathlineto{\pgfqpoint{1.186632in}{1.585012in}}%
\pgfpathlineto{\pgfqpoint{1.190026in}{1.590443in}}%
\pgfpathlineto{\pgfqpoint{1.193422in}{1.595791in}}%
\pgfpathlineto{\pgfqpoint{1.196819in}{1.601053in}}%
\pgfpathlineto{\pgfqpoint{1.200219in}{1.606228in}}%
\pgfpathlineto{\pgfqpoint{1.201181in}{1.603967in}}%
\pgfpathlineto{\pgfqpoint{1.202294in}{1.601720in}}%
\pgfpathlineto{\pgfqpoint{1.203557in}{1.599490in}}%
\pgfpathlineto{\pgfqpoint{1.204968in}{1.597280in}}%
\pgfpathclose%
\pgfusepath{fill}%
\end{pgfscope}%
\begin{pgfscope}%
\pgfpathrectangle{\pgfqpoint{0.329460in}{0.284240in}}{\pgfqpoint{1.989680in}{1.989680in}}%
\pgfusepath{clip}%
\pgfsetbuttcap%
\pgfsetroundjoin%
\definecolor{currentfill}{rgb}{0.248629,0.278775,0.534556}%
\pgfsetfillcolor{currentfill}%
\pgfsetlinewidth{0.000000pt}%
\definecolor{currentstroke}{rgb}{0.000000,0.000000,0.000000}%
\pgfsetstrokecolor{currentstroke}%
\pgfsetdash{}{0pt}%
\pgfpathmoveto{\pgfqpoint{1.586743in}{1.030535in}}%
\pgfpathlineto{\pgfqpoint{1.588820in}{1.022908in}}%
\pgfpathlineto{\pgfqpoint{1.590897in}{1.015351in}}%
\pgfpathlineto{\pgfqpoint{1.592975in}{1.007865in}}%
\pgfpathlineto{\pgfqpoint{1.595053in}{1.000455in}}%
\pgfpathlineto{\pgfqpoint{1.584851in}{0.996498in}}%
\pgfpathlineto{\pgfqpoint{1.574398in}{0.992711in}}%
\pgfpathlineto{\pgfqpoint{1.563705in}{0.989097in}}%
\pgfpathlineto{\pgfqpoint{1.552782in}{0.985660in}}%
\pgfpathlineto{\pgfqpoint{1.551060in}{0.993213in}}%
\pgfpathlineto{\pgfqpoint{1.549339in}{1.000842in}}%
\pgfpathlineto{\pgfqpoint{1.547618in}{1.008542in}}%
\pgfpathlineto{\pgfqpoint{1.545898in}{1.016311in}}%
\pgfpathlineto{\pgfqpoint{1.556451in}{1.019615in}}%
\pgfpathlineto{\pgfqpoint{1.566784in}{1.023089in}}%
\pgfpathlineto{\pgfqpoint{1.576885in}{1.026731in}}%
\pgfpathlineto{\pgfqpoint{1.586743in}{1.030535in}}%
\pgfpathclose%
\pgfusepath{fill}%
\end{pgfscope}%
\begin{pgfscope}%
\pgfpathrectangle{\pgfqpoint{0.329460in}{0.284240in}}{\pgfqpoint{1.989680in}{1.989680in}}%
\pgfusepath{clip}%
\pgfsetbuttcap%
\pgfsetroundjoin%
\definecolor{currentfill}{rgb}{0.268510,0.009605,0.335427}%
\pgfsetfillcolor{currentfill}%
\pgfsetlinewidth{0.000000pt}%
\definecolor{currentstroke}{rgb}{0.000000,0.000000,0.000000}%
\pgfsetstrokecolor{currentstroke}%
\pgfsetdash{}{0pt}%
\pgfpathmoveto{\pgfqpoint{1.740694in}{0.826332in}}%
\pgfpathlineto{\pgfqpoint{1.743233in}{0.826185in}}%
\pgfpathlineto{\pgfqpoint{1.745779in}{0.826290in}}%
\pgfpathlineto{\pgfqpoint{1.748333in}{0.826651in}}%
\pgfpathlineto{\pgfqpoint{1.750893in}{0.827275in}}%
\pgfpathlineto{\pgfqpoint{1.738115in}{0.820583in}}%
\pgfpathlineto{\pgfqpoint{1.724912in}{0.814105in}}%
\pgfpathlineto{\pgfqpoint{1.711298in}{0.807851in}}%
\pgfpathlineto{\pgfqpoint{1.697287in}{0.801826in}}%
\pgfpathlineto{\pgfqpoint{1.695061in}{0.801364in}}%
\pgfpathlineto{\pgfqpoint{1.692842in}{0.801164in}}%
\pgfpathlineto{\pgfqpoint{1.690630in}{0.801222in}}%
\pgfpathlineto{\pgfqpoint{1.688423in}{0.801532in}}%
\pgfpathlineto{\pgfqpoint{1.702083in}{0.807403in}}%
\pgfpathlineto{\pgfqpoint{1.715357in}{0.813498in}}%
\pgfpathlineto{\pgfqpoint{1.728232in}{0.819811in}}%
\pgfpathlineto{\pgfqpoint{1.740694in}{0.826332in}}%
\pgfpathclose%
\pgfusepath{fill}%
\end{pgfscope}%
\begin{pgfscope}%
\pgfpathrectangle{\pgfqpoint{0.329460in}{0.284240in}}{\pgfqpoint{1.989680in}{1.989680in}}%
\pgfusepath{clip}%
\pgfsetbuttcap%
\pgfsetroundjoin%
\definecolor{currentfill}{rgb}{0.896320,0.893616,0.096335}%
\pgfsetfillcolor{currentfill}%
\pgfsetlinewidth{0.000000pt}%
\definecolor{currentstroke}{rgb}{0.000000,0.000000,0.000000}%
\pgfsetstrokecolor{currentstroke}%
\pgfsetdash{}{0pt}%
\pgfpathmoveto{\pgfqpoint{1.282325in}{1.713472in}}%
\pgfpathlineto{\pgfqpoint{1.278894in}{1.711361in}}%
\pgfpathlineto{\pgfqpoint{1.275466in}{1.709139in}}%
\pgfpathlineto{\pgfqpoint{1.272039in}{1.706804in}}%
\pgfpathlineto{\pgfqpoint{1.268614in}{1.704359in}}%
\pgfpathlineto{\pgfqpoint{1.269155in}{1.705575in}}%
\pgfpathlineto{\pgfqpoint{1.269778in}{1.706782in}}%
\pgfpathlineto{\pgfqpoint{1.270482in}{1.707979in}}%
\pgfpathlineto{\pgfqpoint{1.271266in}{1.709165in}}%
\pgfpathlineto{\pgfqpoint{1.274578in}{1.711409in}}%
\pgfpathlineto{\pgfqpoint{1.277893in}{1.713542in}}%
\pgfpathlineto{\pgfqpoint{1.281209in}{1.715564in}}%
\pgfpathlineto{\pgfqpoint{1.284527in}{1.717474in}}%
\pgfpathlineto{\pgfqpoint{1.283876in}{1.716486in}}%
\pgfpathlineto{\pgfqpoint{1.283291in}{1.715489in}}%
\pgfpathlineto{\pgfqpoint{1.282774in}{1.714484in}}%
\pgfpathlineto{\pgfqpoint{1.282325in}{1.713472in}}%
\pgfpathclose%
\pgfusepath{fill}%
\end{pgfscope}%
\begin{pgfscope}%
\pgfpathrectangle{\pgfqpoint{0.329460in}{0.284240in}}{\pgfqpoint{1.989680in}{1.989680in}}%
\pgfusepath{clip}%
\pgfsetbuttcap%
\pgfsetroundjoin%
\definecolor{currentfill}{rgb}{0.955300,0.901065,0.118128}%
\pgfsetfillcolor{currentfill}%
\pgfsetlinewidth{0.000000pt}%
\definecolor{currentstroke}{rgb}{0.000000,0.000000,0.000000}%
\pgfsetstrokecolor{currentstroke}%
\pgfsetdash{}{0pt}%
\pgfpathmoveto{\pgfqpoint{1.382863in}{1.735529in}}%
\pgfpathlineto{\pgfqpoint{1.385493in}{1.735113in}}%
\pgfpathlineto{\pgfqpoint{1.388122in}{1.734580in}}%
\pgfpathlineto{\pgfqpoint{1.390749in}{1.733930in}}%
\pgfpathlineto{\pgfqpoint{1.393375in}{1.733165in}}%
\pgfpathlineto{\pgfqpoint{1.394519in}{1.732536in}}%
\pgfpathlineto{\pgfqpoint{1.395619in}{1.731891in}}%
\pgfpathlineto{\pgfqpoint{1.396676in}{1.731230in}}%
\pgfpathlineto{\pgfqpoint{1.397688in}{1.730553in}}%
\pgfpathlineto{\pgfqpoint{1.394793in}{1.731482in}}%
\pgfpathlineto{\pgfqpoint{1.391896in}{1.732294in}}%
\pgfpathlineto{\pgfqpoint{1.388997in}{1.732990in}}%
\pgfpathlineto{\pgfqpoint{1.386097in}{1.733570in}}%
\pgfpathlineto{\pgfqpoint{1.385338in}{1.734077in}}%
\pgfpathlineto{\pgfqpoint{1.384546in}{1.734573in}}%
\pgfpathlineto{\pgfqpoint{1.383720in}{1.735058in}}%
\pgfpathlineto{\pgfqpoint{1.382863in}{1.735529in}}%
\pgfpathclose%
\pgfusepath{fill}%
\end{pgfscope}%
\begin{pgfscope}%
\pgfpathrectangle{\pgfqpoint{0.329460in}{0.284240in}}{\pgfqpoint{1.989680in}{1.989680in}}%
\pgfusepath{clip}%
\pgfsetbuttcap%
\pgfsetroundjoin%
\definecolor{currentfill}{rgb}{0.974417,0.903590,0.130215}%
\pgfsetfillcolor{currentfill}%
\pgfsetlinewidth{0.000000pt}%
\definecolor{currentstroke}{rgb}{0.000000,0.000000,0.000000}%
\pgfsetstrokecolor{currentstroke}%
\pgfsetdash{}{0pt}%
\pgfpathmoveto{\pgfqpoint{1.358236in}{1.740072in}}%
\pgfpathlineto{\pgfqpoint{1.359114in}{1.740632in}}%
\pgfpathlineto{\pgfqpoint{1.359992in}{1.741075in}}%
\pgfpathlineto{\pgfqpoint{1.360869in}{1.741398in}}%
\pgfpathlineto{\pgfqpoint{1.361745in}{1.741604in}}%
\pgfpathlineto{\pgfqpoint{1.363022in}{1.741439in}}%
\pgfpathlineto{\pgfqpoint{1.364288in}{1.741256in}}%
\pgfpathlineto{\pgfqpoint{1.365540in}{1.741055in}}%
\pgfpathlineto{\pgfqpoint{1.364349in}{1.740895in}}%
\pgfpathlineto{\pgfqpoint{1.363156in}{1.740617in}}%
\pgfpathlineto{\pgfqpoint{1.361963in}{1.740220in}}%
\pgfpathlineto{\pgfqpoint{1.360769in}{1.739705in}}%
\pgfpathlineto{\pgfqpoint{1.359933in}{1.739840in}}%
\pgfpathlineto{\pgfqpoint{1.359088in}{1.739962in}}%
\pgfpathlineto{\pgfqpoint{1.358236in}{1.740072in}}%
\pgfpathclose%
\pgfusepath{fill}%
\end{pgfscope}%
\begin{pgfscope}%
\pgfpathrectangle{\pgfqpoint{0.329460in}{0.284240in}}{\pgfqpoint{1.989680in}{1.989680in}}%
\pgfusepath{clip}%
\pgfsetbuttcap%
\pgfsetroundjoin%
\definecolor{currentfill}{rgb}{0.281477,0.755203,0.432552}%
\pgfsetfillcolor{currentfill}%
\pgfsetlinewidth{0.000000pt}%
\definecolor{currentstroke}{rgb}{0.000000,0.000000,0.000000}%
\pgfsetstrokecolor{currentstroke}%
\pgfsetdash{}{0pt}%
\pgfpathmoveto{\pgfqpoint{1.542779in}{1.492233in}}%
\pgfpathlineto{\pgfqpoint{1.545932in}{1.485482in}}%
\pgfpathlineto{\pgfqpoint{1.549082in}{1.478670in}}%
\pgfpathlineto{\pgfqpoint{1.552231in}{1.471800in}}%
\pgfpathlineto{\pgfqpoint{1.555378in}{1.464872in}}%
\pgfpathlineto{\pgfqpoint{1.552501in}{1.461752in}}%
\pgfpathlineto{\pgfqpoint{1.549418in}{1.458677in}}%
\pgfpathlineto{\pgfqpoint{1.546133in}{1.455648in}}%
\pgfpathlineto{\pgfqpoint{1.542648in}{1.452671in}}%
\pgfpathlineto{\pgfqpoint{1.539692in}{1.459800in}}%
\pgfpathlineto{\pgfqpoint{1.536733in}{1.466873in}}%
\pgfpathlineto{\pgfqpoint{1.533773in}{1.473886in}}%
\pgfpathlineto{\pgfqpoint{1.530812in}{1.480838in}}%
\pgfpathlineto{\pgfqpoint{1.534087in}{1.483618in}}%
\pgfpathlineto{\pgfqpoint{1.537175in}{1.486446in}}%
\pgfpathlineto{\pgfqpoint{1.540073in}{1.489319in}}%
\pgfpathlineto{\pgfqpoint{1.542779in}{1.492233in}}%
\pgfpathclose%
\pgfusepath{fill}%
\end{pgfscope}%
\begin{pgfscope}%
\pgfpathrectangle{\pgfqpoint{0.329460in}{0.284240in}}{\pgfqpoint{1.989680in}{1.989680in}}%
\pgfusepath{clip}%
\pgfsetbuttcap%
\pgfsetroundjoin%
\definecolor{currentfill}{rgb}{0.974417,0.903590,0.130215}%
\pgfsetfillcolor{currentfill}%
\pgfsetlinewidth{0.000000pt}%
\definecolor{currentstroke}{rgb}{0.000000,0.000000,0.000000}%
\pgfsetstrokecolor{currentstroke}%
\pgfsetdash{}{0pt}%
\pgfpathmoveto{\pgfqpoint{1.340871in}{1.739575in}}%
\pgfpathlineto{\pgfqpoint{1.339585in}{1.740074in}}%
\pgfpathlineto{\pgfqpoint{1.338300in}{1.740455in}}%
\pgfpathlineto{\pgfqpoint{1.337016in}{1.740716in}}%
\pgfpathlineto{\pgfqpoint{1.335733in}{1.740860in}}%
\pgfpathlineto{\pgfqpoint{1.336973in}{1.741078in}}%
\pgfpathlineto{\pgfqpoint{1.338227in}{1.741278in}}%
\pgfpathlineto{\pgfqpoint{1.339494in}{1.741459in}}%
\pgfpathlineto{\pgfqpoint{1.340772in}{1.741621in}}%
\pgfpathlineto{\pgfqpoint{1.341637in}{1.741414in}}%
\pgfpathlineto{\pgfqpoint{1.342502in}{1.741089in}}%
\pgfpathlineto{\pgfqpoint{1.343368in}{1.740645in}}%
\pgfpathlineto{\pgfqpoint{1.344235in}{1.740083in}}%
\pgfpathlineto{\pgfqpoint{1.343381in}{1.739975in}}%
\pgfpathlineto{\pgfqpoint{1.342536in}{1.739854in}}%
\pgfpathlineto{\pgfqpoint{1.341699in}{1.739721in}}%
\pgfpathlineto{\pgfqpoint{1.340871in}{1.739575in}}%
\pgfpathclose%
\pgfusepath{fill}%
\end{pgfscope}%
\begin{pgfscope}%
\pgfpathrectangle{\pgfqpoint{0.329460in}{0.284240in}}{\pgfqpoint{1.989680in}{1.989680in}}%
\pgfusepath{clip}%
\pgfsetbuttcap%
\pgfsetroundjoin%
\definecolor{currentfill}{rgb}{0.935904,0.898570,0.108131}%
\pgfsetfillcolor{currentfill}%
\pgfsetlinewidth{0.000000pt}%
\definecolor{currentstroke}{rgb}{0.000000,0.000000,0.000000}%
\pgfsetstrokecolor{currentstroke}%
\pgfsetdash{}{0pt}%
\pgfpathmoveto{\pgfqpoint{1.401266in}{1.727708in}}%
\pgfpathlineto{\pgfqpoint{1.404384in}{1.726486in}}%
\pgfpathlineto{\pgfqpoint{1.407500in}{1.725150in}}%
\pgfpathlineto{\pgfqpoint{1.410614in}{1.723700in}}%
\pgfpathlineto{\pgfqpoint{1.413727in}{1.722137in}}%
\pgfpathlineto{\pgfqpoint{1.414694in}{1.721208in}}%
\pgfpathlineto{\pgfqpoint{1.415598in}{1.720266in}}%
\pgfpathlineto{\pgfqpoint{1.416438in}{1.719311in}}%
\pgfpathlineto{\pgfqpoint{1.417213in}{1.718343in}}%
\pgfpathlineto{\pgfqpoint{1.413926in}{1.720097in}}%
\pgfpathlineto{\pgfqpoint{1.410636in}{1.721737in}}%
\pgfpathlineto{\pgfqpoint{1.407345in}{1.723262in}}%
\pgfpathlineto{\pgfqpoint{1.404052in}{1.724673in}}%
\pgfpathlineto{\pgfqpoint{1.403432in}{1.725447in}}%
\pgfpathlineto{\pgfqpoint{1.402761in}{1.726211in}}%
\pgfpathlineto{\pgfqpoint{1.402039in}{1.726965in}}%
\pgfpathlineto{\pgfqpoint{1.401266in}{1.727708in}}%
\pgfpathclose%
\pgfusepath{fill}%
\end{pgfscope}%
\begin{pgfscope}%
\pgfpathrectangle{\pgfqpoint{0.329460in}{0.284240in}}{\pgfqpoint{1.989680in}{1.989680in}}%
\pgfusepath{clip}%
\pgfsetbuttcap%
\pgfsetroundjoin%
\definecolor{currentfill}{rgb}{0.699415,0.867117,0.175971}%
\pgfsetfillcolor{currentfill}%
\pgfsetlinewidth{0.000000pt}%
\definecolor{currentstroke}{rgb}{0.000000,0.000000,0.000000}%
\pgfsetstrokecolor{currentstroke}%
\pgfsetdash{}{0pt}%
\pgfpathmoveto{\pgfqpoint{1.227491in}{1.644342in}}%
\pgfpathlineto{\pgfqpoint{1.224075in}{1.639910in}}%
\pgfpathlineto{\pgfqpoint{1.220661in}{1.635380in}}%
\pgfpathlineto{\pgfqpoint{1.217250in}{1.630754in}}%
\pgfpathlineto{\pgfqpoint{1.213840in}{1.626034in}}%
\pgfpathlineto{\pgfqpoint{1.213097in}{1.628096in}}%
\pgfpathlineto{\pgfqpoint{1.212494in}{1.630167in}}%
\pgfpathlineto{\pgfqpoint{1.212029in}{1.632246in}}%
\pgfpathlineto{\pgfqpoint{1.211703in}{1.634329in}}%
\pgfpathlineto{\pgfqpoint{1.215161in}{1.638838in}}%
\pgfpathlineto{\pgfqpoint{1.218622in}{1.643251in}}%
\pgfpathlineto{\pgfqpoint{1.222084in}{1.647569in}}%
\pgfpathlineto{\pgfqpoint{1.225549in}{1.651790in}}%
\pgfpathlineto{\pgfqpoint{1.225846in}{1.649919in}}%
\pgfpathlineto{\pgfqpoint{1.226270in}{1.648053in}}%
\pgfpathlineto{\pgfqpoint{1.226818in}{1.646193in}}%
\pgfpathlineto{\pgfqpoint{1.227491in}{1.644342in}}%
\pgfpathclose%
\pgfusepath{fill}%
\end{pgfscope}%
\begin{pgfscope}%
\pgfpathrectangle{\pgfqpoint{0.329460in}{0.284240in}}{\pgfqpoint{1.989680in}{1.989680in}}%
\pgfusepath{clip}%
\pgfsetbuttcap%
\pgfsetroundjoin%
\definecolor{currentfill}{rgb}{0.855810,0.888601,0.097452}%
\pgfsetfillcolor{currentfill}%
\pgfsetlinewidth{0.000000pt}%
\definecolor{currentstroke}{rgb}{0.000000,0.000000,0.000000}%
\pgfsetstrokecolor{currentstroke}%
\pgfsetdash{}{0pt}%
\pgfpathmoveto{\pgfqpoint{1.267272in}{1.699431in}}%
\pgfpathlineto{\pgfqpoint{1.263790in}{1.696670in}}%
\pgfpathlineto{\pgfqpoint{1.260310in}{1.693799in}}%
\pgfpathlineto{\pgfqpoint{1.256831in}{1.690819in}}%
\pgfpathlineto{\pgfqpoint{1.253355in}{1.687732in}}%
\pgfpathlineto{\pgfqpoint{1.253604in}{1.689180in}}%
\pgfpathlineto{\pgfqpoint{1.253950in}{1.690623in}}%
\pgfpathlineto{\pgfqpoint{1.254393in}{1.692060in}}%
\pgfpathlineto{\pgfqpoint{1.254932in}{1.693489in}}%
\pgfpathlineto{\pgfqpoint{1.258350in}{1.696368in}}%
\pgfpathlineto{\pgfqpoint{1.261769in}{1.699140in}}%
\pgfpathlineto{\pgfqpoint{1.265190in}{1.701804in}}%
\pgfpathlineto{\pgfqpoint{1.268614in}{1.704359in}}%
\pgfpathlineto{\pgfqpoint{1.268154in}{1.703136in}}%
\pgfpathlineto{\pgfqpoint{1.267777in}{1.701906in}}%
\pgfpathlineto{\pgfqpoint{1.267483in}{1.700670in}}%
\pgfpathlineto{\pgfqpoint{1.267272in}{1.699431in}}%
\pgfpathclose%
\pgfusepath{fill}%
\end{pgfscope}%
\begin{pgfscope}%
\pgfpathrectangle{\pgfqpoint{0.329460in}{0.284240in}}{\pgfqpoint{1.989680in}{1.989680in}}%
\pgfusepath{clip}%
\pgfsetbuttcap%
\pgfsetroundjoin%
\definecolor{currentfill}{rgb}{0.814576,0.883393,0.110347}%
\pgfsetfillcolor{currentfill}%
\pgfsetlinewidth{0.000000pt}%
\definecolor{currentstroke}{rgb}{0.000000,0.000000,0.000000}%
\pgfsetstrokecolor{currentstroke}%
\pgfsetdash{}{0pt}%
\pgfpathmoveto{\pgfqpoint{1.448804in}{1.689019in}}%
\pgfpathlineto{\pgfqpoint{1.452270in}{1.685872in}}%
\pgfpathlineto{\pgfqpoint{1.455735in}{1.682620in}}%
\pgfpathlineto{\pgfqpoint{1.459197in}{1.679263in}}%
\pgfpathlineto{\pgfqpoint{1.462657in}{1.675802in}}%
\pgfpathlineto{\pgfqpoint{1.462934in}{1.674145in}}%
\pgfpathlineto{\pgfqpoint{1.463099in}{1.672484in}}%
\pgfpathlineto{\pgfqpoint{1.463152in}{1.670820in}}%
\pgfpathlineto{\pgfqpoint{1.463094in}{1.669157in}}%
\pgfpathlineto{\pgfqpoint{1.459616in}{1.672828in}}%
\pgfpathlineto{\pgfqpoint{1.456137in}{1.676395in}}%
\pgfpathlineto{\pgfqpoint{1.452655in}{1.679857in}}%
\pgfpathlineto{\pgfqpoint{1.449172in}{1.683214in}}%
\pgfpathlineto{\pgfqpoint{1.449227in}{1.684667in}}%
\pgfpathlineto{\pgfqpoint{1.449183in}{1.686120in}}%
\pgfpathlineto{\pgfqpoint{1.449042in}{1.687571in}}%
\pgfpathlineto{\pgfqpoint{1.448804in}{1.689019in}}%
\pgfpathclose%
\pgfusepath{fill}%
\end{pgfscope}%
\begin{pgfscope}%
\pgfpathrectangle{\pgfqpoint{0.329460in}{0.284240in}}{\pgfqpoint{1.989680in}{1.989680in}}%
\pgfusepath{clip}%
\pgfsetbuttcap%
\pgfsetroundjoin%
\definecolor{currentfill}{rgb}{0.955300,0.901065,0.118128}%
\pgfsetfillcolor{currentfill}%
\pgfsetlinewidth{0.000000pt}%
\definecolor{currentstroke}{rgb}{0.000000,0.000000,0.000000}%
\pgfsetstrokecolor{currentstroke}%
\pgfsetdash{}{0pt}%
\pgfpathmoveto{\pgfqpoint{1.315633in}{1.733109in}}%
\pgfpathlineto{\pgfqpoint{1.312679in}{1.732492in}}%
\pgfpathlineto{\pgfqpoint{1.309726in}{1.731757in}}%
\pgfpathlineto{\pgfqpoint{1.306775in}{1.730906in}}%
\pgfpathlineto{\pgfqpoint{1.303826in}{1.729940in}}%
\pgfpathlineto{\pgfqpoint{1.304797in}{1.730629in}}%
\pgfpathlineto{\pgfqpoint{1.305814in}{1.731304in}}%
\pgfpathlineto{\pgfqpoint{1.306876in}{1.731963in}}%
\pgfpathlineto{\pgfqpoint{1.307982in}{1.732607in}}%
\pgfpathlineto{\pgfqpoint{1.310671in}{1.733407in}}%
\pgfpathlineto{\pgfqpoint{1.313362in}{1.734091in}}%
\pgfpathlineto{\pgfqpoint{1.316054in}{1.734659in}}%
\pgfpathlineto{\pgfqpoint{1.318749in}{1.735111in}}%
\pgfpathlineto{\pgfqpoint{1.317919in}{1.734628in}}%
\pgfpathlineto{\pgfqpoint{1.317123in}{1.734133in}}%
\pgfpathlineto{\pgfqpoint{1.316361in}{1.733627in}}%
\pgfpathlineto{\pgfqpoint{1.315633in}{1.733109in}}%
\pgfpathclose%
\pgfusepath{fill}%
\end{pgfscope}%
\begin{pgfscope}%
\pgfpathrectangle{\pgfqpoint{0.329460in}{0.284240in}}{\pgfqpoint{1.989680in}{1.989680in}}%
\pgfusepath{clip}%
\pgfsetbuttcap%
\pgfsetroundjoin%
\definecolor{currentfill}{rgb}{0.133743,0.548535,0.553541}%
\pgfsetfillcolor{currentfill}%
\pgfsetlinewidth{0.000000pt}%
\definecolor{currentstroke}{rgb}{0.000000,0.000000,0.000000}%
\pgfsetstrokecolor{currentstroke}%
\pgfsetdash{}{0pt}%
\pgfpathmoveto{\pgfqpoint{1.151879in}{1.269391in}}%
\pgfpathlineto{\pgfqpoint{1.149540in}{1.261135in}}%
\pgfpathlineto{\pgfqpoint{1.147202in}{1.252870in}}%
\pgfpathlineto{\pgfqpoint{1.144864in}{1.244599in}}%
\pgfpathlineto{\pgfqpoint{1.142528in}{1.236323in}}%
\pgfpathlineto{\pgfqpoint{1.135588in}{1.239708in}}%
\pgfpathlineto{\pgfqpoint{1.128875in}{1.243201in}}%
\pgfpathlineto{\pgfqpoint{1.122394in}{1.246797in}}%
\pgfpathlineto{\pgfqpoint{1.116152in}{1.250493in}}%
\pgfpathlineto{\pgfqpoint{1.118778in}{1.258592in}}%
\pgfpathlineto{\pgfqpoint{1.121405in}{1.266687in}}%
\pgfpathlineto{\pgfqpoint{1.124033in}{1.274776in}}%
\pgfpathlineto{\pgfqpoint{1.126663in}{1.282856in}}%
\pgfpathlineto{\pgfqpoint{1.132631in}{1.279344in}}%
\pgfpathlineto{\pgfqpoint{1.138827in}{1.275927in}}%
\pgfpathlineto{\pgfqpoint{1.145245in}{1.272608in}}%
\pgfpathlineto{\pgfqpoint{1.151879in}{1.269391in}}%
\pgfpathclose%
\pgfusepath{fill}%
\end{pgfscope}%
\begin{pgfscope}%
\pgfpathrectangle{\pgfqpoint{0.329460in}{0.284240in}}{\pgfqpoint{1.989680in}{1.989680in}}%
\pgfusepath{clip}%
\pgfsetbuttcap%
\pgfsetroundjoin%
\definecolor{currentfill}{rgb}{0.274952,0.037752,0.364543}%
\pgfsetfillcolor{currentfill}%
\pgfsetlinewidth{0.000000pt}%
\definecolor{currentstroke}{rgb}{0.000000,0.000000,0.000000}%
\pgfsetstrokecolor{currentstroke}%
\pgfsetdash{}{0pt}%
\pgfpathmoveto{\pgfqpoint{1.119855in}{0.831990in}}%
\pgfpathlineto{\pgfqpoint{1.118182in}{0.827388in}}%
\pgfpathlineto{\pgfqpoint{1.116507in}{0.822953in}}%
\pgfpathlineto{\pgfqpoint{1.114829in}{0.818689in}}%
\pgfpathlineto{\pgfqpoint{1.113149in}{0.814599in}}%
\pgfpathlineto{\pgfqpoint{1.099375in}{0.818771in}}%
\pgfpathlineto{\pgfqpoint{1.085879in}{0.823172in}}%
\pgfpathlineto{\pgfqpoint{1.072674in}{0.827797in}}%
\pgfpathlineto{\pgfqpoint{1.059774in}{0.832641in}}%
\pgfpathlineto{\pgfqpoint{1.061826in}{0.836593in}}%
\pgfpathlineto{\pgfqpoint{1.063874in}{0.840719in}}%
\pgfpathlineto{\pgfqpoint{1.065919in}{0.845016in}}%
\pgfpathlineto{\pgfqpoint{1.067962in}{0.849479in}}%
\pgfpathlineto{\pgfqpoint{1.080504in}{0.844783in}}%
\pgfpathlineto{\pgfqpoint{1.093343in}{0.840300in}}%
\pgfpathlineto{\pgfqpoint{1.106465in}{0.836034in}}%
\pgfpathlineto{\pgfqpoint{1.119855in}{0.831990in}}%
\pgfpathclose%
\pgfusepath{fill}%
\end{pgfscope}%
\begin{pgfscope}%
\pgfpathrectangle{\pgfqpoint{0.329460in}{0.284240in}}{\pgfqpoint{1.989680in}{1.989680in}}%
\pgfusepath{clip}%
\pgfsetbuttcap%
\pgfsetroundjoin%
\definecolor{currentfill}{rgb}{0.974417,0.903590,0.130215}%
\pgfsetfillcolor{currentfill}%
\pgfsetlinewidth{0.000000pt}%
\definecolor{currentstroke}{rgb}{0.000000,0.000000,0.000000}%
\pgfsetstrokecolor{currentstroke}%
\pgfsetdash{}{0pt}%
\pgfpathmoveto{\pgfqpoint{1.360769in}{1.739705in}}%
\pgfpathlineto{\pgfqpoint{1.361963in}{1.740220in}}%
\pgfpathlineto{\pgfqpoint{1.363156in}{1.740617in}}%
\pgfpathlineto{\pgfqpoint{1.364349in}{1.740895in}}%
\pgfpathlineto{\pgfqpoint{1.365540in}{1.741055in}}%
\pgfpathlineto{\pgfqpoint{1.366779in}{1.740835in}}%
\pgfpathlineto{\pgfqpoint{1.368002in}{1.740597in}}%
\pgfpathlineto{\pgfqpoint{1.369208in}{1.740341in}}%
\pgfpathlineto{\pgfqpoint{1.370396in}{1.740068in}}%
\pgfpathlineto{\pgfqpoint{1.368801in}{1.739990in}}%
\pgfpathlineto{\pgfqpoint{1.367205in}{1.739794in}}%
\pgfpathlineto{\pgfqpoint{1.365608in}{1.739479in}}%
\pgfpathlineto{\pgfqpoint{1.364010in}{1.739046in}}%
\pgfpathlineto{\pgfqpoint{1.363216in}{1.739229in}}%
\pgfpathlineto{\pgfqpoint{1.362411in}{1.739399in}}%
\pgfpathlineto{\pgfqpoint{1.361595in}{1.739558in}}%
\pgfpathlineto{\pgfqpoint{1.360769in}{1.739705in}}%
\pgfpathclose%
\pgfusepath{fill}%
\end{pgfscope}%
\begin{pgfscope}%
\pgfpathrectangle{\pgfqpoint{0.329460in}{0.284240in}}{\pgfqpoint{1.989680in}{1.989680in}}%
\pgfusepath{clip}%
\pgfsetbuttcap%
\pgfsetroundjoin%
\definecolor{currentfill}{rgb}{0.412913,0.803041,0.357269}%
\pgfsetfillcolor{currentfill}%
\pgfsetlinewidth{0.000000pt}%
\definecolor{currentstroke}{rgb}{0.000000,0.000000,0.000000}%
\pgfsetstrokecolor{currentstroke}%
\pgfsetdash{}{0pt}%
\pgfpathmoveto{\pgfqpoint{1.186977in}{1.541608in}}%
\pgfpathlineto{\pgfqpoint{1.183847in}{1.535363in}}%
\pgfpathlineto{\pgfqpoint{1.180720in}{1.529042in}}%
\pgfpathlineto{\pgfqpoint{1.177594in}{1.522649in}}%
\pgfpathlineto{\pgfqpoint{1.174470in}{1.516183in}}%
\pgfpathlineto{\pgfqpoint{1.171957in}{1.518897in}}%
\pgfpathlineto{\pgfqpoint{1.169626in}{1.521647in}}%
\pgfpathlineto{\pgfqpoint{1.167479in}{1.524430in}}%
\pgfpathlineto{\pgfqpoint{1.165518in}{1.527243in}}%
\pgfpathlineto{\pgfqpoint{1.168794in}{1.533501in}}%
\pgfpathlineto{\pgfqpoint{1.172073in}{1.539688in}}%
\pgfpathlineto{\pgfqpoint{1.175353in}{1.545803in}}%
\pgfpathlineto{\pgfqpoint{1.178636in}{1.551842in}}%
\pgfpathlineto{\pgfqpoint{1.180464in}{1.549239in}}%
\pgfpathlineto{\pgfqpoint{1.182465in}{1.546664in}}%
\pgfpathlineto{\pgfqpoint{1.184636in}{1.544119in}}%
\pgfpathlineto{\pgfqpoint{1.186977in}{1.541608in}}%
\pgfpathclose%
\pgfusepath{fill}%
\end{pgfscope}%
\begin{pgfscope}%
\pgfpathrectangle{\pgfqpoint{0.329460in}{0.284240in}}{\pgfqpoint{1.989680in}{1.989680in}}%
\pgfusepath{clip}%
\pgfsetbuttcap%
\pgfsetroundjoin%
\definecolor{currentfill}{rgb}{0.263663,0.237631,0.518762}%
\pgfsetfillcolor{currentfill}%
\pgfsetlinewidth{0.000000pt}%
\definecolor{currentstroke}{rgb}{0.000000,0.000000,0.000000}%
\pgfsetstrokecolor{currentstroke}%
\pgfsetdash{}{0pt}%
\pgfpathmoveto{\pgfqpoint{1.159486in}{0.982757in}}%
\pgfpathlineto{\pgfqpoint{1.157847in}{0.975254in}}%
\pgfpathlineto{\pgfqpoint{1.156208in}{0.967832in}}%
\pgfpathlineto{\pgfqpoint{1.154568in}{0.960495in}}%
\pgfpathlineto{\pgfqpoint{1.152927in}{0.953246in}}%
\pgfpathlineto{\pgfqpoint{1.141433in}{0.956649in}}%
\pgfpathlineto{\pgfqpoint{1.130167in}{0.960240in}}%
\pgfpathlineto{\pgfqpoint{1.119139in}{0.964015in}}%
\pgfpathlineto{\pgfqpoint{1.108362in}{0.967969in}}%
\pgfpathlineto{\pgfqpoint{1.110367in}{0.975080in}}%
\pgfpathlineto{\pgfqpoint{1.112372in}{0.982280in}}%
\pgfpathlineto{\pgfqpoint{1.114375in}{0.989564in}}%
\pgfpathlineto{\pgfqpoint{1.116378in}{0.996930in}}%
\pgfpathlineto{\pgfqpoint{1.126803in}{0.993123in}}%
\pgfpathlineto{\pgfqpoint{1.137471in}{0.989490in}}%
\pgfpathlineto{\pgfqpoint{1.148369in}{0.986033in}}%
\pgfpathlineto{\pgfqpoint{1.159486in}{0.982757in}}%
\pgfpathclose%
\pgfusepath{fill}%
\end{pgfscope}%
\begin{pgfscope}%
\pgfpathrectangle{\pgfqpoint{0.329460in}{0.284240in}}{\pgfqpoint{1.989680in}{1.989680in}}%
\pgfusepath{clip}%
\pgfsetbuttcap%
\pgfsetroundjoin%
\definecolor{currentfill}{rgb}{0.974417,0.903590,0.130215}%
\pgfsetfillcolor{currentfill}%
\pgfsetlinewidth{0.000000pt}%
\definecolor{currentstroke}{rgb}{0.000000,0.000000,0.000000}%
\pgfsetstrokecolor{currentstroke}%
\pgfsetdash{}{0pt}%
\pgfpathmoveto{\pgfqpoint{1.337671in}{1.738874in}}%
\pgfpathlineto{\pgfqpoint{1.335986in}{1.739286in}}%
\pgfpathlineto{\pgfqpoint{1.334302in}{1.739579in}}%
\pgfpathlineto{\pgfqpoint{1.332620in}{1.739754in}}%
\pgfpathlineto{\pgfqpoint{1.330938in}{1.739810in}}%
\pgfpathlineto{\pgfqpoint{1.332110in}{1.740099in}}%
\pgfpathlineto{\pgfqpoint{1.333300in}{1.740371in}}%
\pgfpathlineto{\pgfqpoint{1.334509in}{1.740624in}}%
\pgfpathlineto{\pgfqpoint{1.335733in}{1.740860in}}%
\pgfpathlineto{\pgfqpoint{1.337016in}{1.740716in}}%
\pgfpathlineto{\pgfqpoint{1.338300in}{1.740455in}}%
\pgfpathlineto{\pgfqpoint{1.339585in}{1.740074in}}%
\pgfpathlineto{\pgfqpoint{1.340871in}{1.739575in}}%
\pgfpathlineto{\pgfqpoint{1.340054in}{1.739418in}}%
\pgfpathlineto{\pgfqpoint{1.339248in}{1.739248in}}%
\pgfpathlineto{\pgfqpoint{1.338453in}{1.739067in}}%
\pgfpathlineto{\pgfqpoint{1.337671in}{1.738874in}}%
\pgfpathclose%
\pgfusepath{fill}%
\end{pgfscope}%
\begin{pgfscope}%
\pgfpathrectangle{\pgfqpoint{0.329460in}{0.284240in}}{\pgfqpoint{1.989680in}{1.989680in}}%
\pgfusepath{clip}%
\pgfsetbuttcap%
\pgfsetroundjoin%
\definecolor{currentfill}{rgb}{0.935904,0.898570,0.108131}%
\pgfsetfillcolor{currentfill}%
\pgfsetlinewidth{0.000000pt}%
\definecolor{currentstroke}{rgb}{0.000000,0.000000,0.000000}%
\pgfsetstrokecolor{currentstroke}%
\pgfsetdash{}{0pt}%
\pgfpathmoveto{\pgfqpoint{1.297817in}{1.723978in}}%
\pgfpathlineto{\pgfqpoint{1.294492in}{1.722524in}}%
\pgfpathlineto{\pgfqpoint{1.291169in}{1.720954in}}%
\pgfpathlineto{\pgfqpoint{1.287847in}{1.719271in}}%
\pgfpathlineto{\pgfqpoint{1.284527in}{1.717474in}}%
\pgfpathlineto{\pgfqpoint{1.285245in}{1.718451in}}%
\pgfpathlineto{\pgfqpoint{1.286027in}{1.719417in}}%
\pgfpathlineto{\pgfqpoint{1.286875in}{1.720371in}}%
\pgfpathlineto{\pgfqpoint{1.287786in}{1.721312in}}%
\pgfpathlineto{\pgfqpoint{1.290942in}{1.722917in}}%
\pgfpathlineto{\pgfqpoint{1.294099in}{1.724408in}}%
\pgfpathlineto{\pgfqpoint{1.297258in}{1.725786in}}%
\pgfpathlineto{\pgfqpoint{1.300420in}{1.727048in}}%
\pgfpathlineto{\pgfqpoint{1.299692in}{1.726296in}}%
\pgfpathlineto{\pgfqpoint{1.299015in}{1.725533in}}%
\pgfpathlineto{\pgfqpoint{1.298390in}{1.724760in}}%
\pgfpathlineto{\pgfqpoint{1.297817in}{1.723978in}}%
\pgfpathclose%
\pgfusepath{fill}%
\end{pgfscope}%
\begin{pgfscope}%
\pgfpathrectangle{\pgfqpoint{0.329460in}{0.284240in}}{\pgfqpoint{1.989680in}{1.989680in}}%
\pgfusepath{clip}%
\pgfsetbuttcap%
\pgfsetroundjoin%
\definecolor{currentfill}{rgb}{0.134692,0.658636,0.517649}%
\pgfsetfillcolor{currentfill}%
\pgfsetlinewidth{0.000000pt}%
\definecolor{currentstroke}{rgb}{0.000000,0.000000,0.000000}%
\pgfsetstrokecolor{currentstroke}%
\pgfsetdash{}{0pt}%
\pgfpathmoveto{\pgfqpoint{1.158312in}{1.378242in}}%
\pgfpathlineto{\pgfqpoint{1.155667in}{1.370456in}}%
\pgfpathlineto{\pgfqpoint{1.153024in}{1.362632in}}%
\pgfpathlineto{\pgfqpoint{1.150382in}{1.354773in}}%
\pgfpathlineto{\pgfqpoint{1.147742in}{1.346882in}}%
\pgfpathlineto{\pgfqpoint{1.142534in}{1.350113in}}%
\pgfpathlineto{\pgfqpoint{1.137543in}{1.353422in}}%
\pgfpathlineto{\pgfqpoint{1.132773in}{1.356806in}}%
\pgfpathlineto{\pgfqpoint{1.128229in}{1.360261in}}%
\pgfpathlineto{\pgfqpoint{1.131116in}{1.367962in}}%
\pgfpathlineto{\pgfqpoint{1.134006in}{1.375631in}}%
\pgfpathlineto{\pgfqpoint{1.136897in}{1.383265in}}%
\pgfpathlineto{\pgfqpoint{1.139789in}{1.390863in}}%
\pgfpathlineto{\pgfqpoint{1.144104in}{1.387603in}}%
\pgfpathlineto{\pgfqpoint{1.148632in}{1.384411in}}%
\pgfpathlineto{\pgfqpoint{1.153370in}{1.381290in}}%
\pgfpathlineto{\pgfqpoint{1.158312in}{1.378242in}}%
\pgfpathclose%
\pgfusepath{fill}%
\end{pgfscope}%
\begin{pgfscope}%
\pgfpathrectangle{\pgfqpoint{0.329460in}{0.284240in}}{\pgfqpoint{1.989680in}{1.989680in}}%
\pgfusepath{clip}%
\pgfsetbuttcap%
\pgfsetroundjoin%
\definecolor{currentfill}{rgb}{0.172719,0.448791,0.557885}%
\pgfsetfillcolor{currentfill}%
\pgfsetlinewidth{0.000000pt}%
\definecolor{currentstroke}{rgb}{0.000000,0.000000,0.000000}%
\pgfsetstrokecolor{currentstroke}%
\pgfsetdash{}{0pt}%
\pgfpathmoveto{\pgfqpoint{0.731721in}{1.119768in}}%
\pgfpathlineto{\pgfqpoint{0.728024in}{1.132620in}}%
\pgfpathlineto{\pgfqpoint{0.724305in}{1.145939in}}%
\pgfpathlineto{\pgfqpoint{0.720565in}{1.159732in}}%
\pgfpathlineto{\pgfqpoint{0.712339in}{1.169981in}}%
\pgfpathlineto{\pgfqpoint{0.704791in}{1.180341in}}%
\pgfpathlineto{\pgfqpoint{0.697923in}{1.190799in}}%
\pgfpathlineto{\pgfqpoint{0.691742in}{1.201345in}}%
\pgfpathlineto{\pgfqpoint{0.695628in}{1.187397in}}%
\pgfpathlineto{\pgfqpoint{0.699492in}{1.173920in}}%
\pgfpathlineto{\pgfqpoint{0.703333in}{1.160907in}}%
\pgfpathlineto{\pgfqpoint{0.709425in}{1.150480in}}%
\pgfpathlineto{\pgfqpoint{0.716189in}{1.140140in}}%
\pgfpathlineto{\pgfqpoint{0.723623in}{1.129899in}}%
\pgfpathlineto{\pgfqpoint{0.731721in}{1.119768in}}%
\pgfpathclose%
\pgfusepath{fill}%
\end{pgfscope}%
\begin{pgfscope}%
\pgfpathrectangle{\pgfqpoint{0.329460in}{0.284240in}}{\pgfqpoint{1.989680in}{1.989680in}}%
\pgfusepath{clip}%
\pgfsetbuttcap%
\pgfsetroundjoin%
\definecolor{currentfill}{rgb}{0.974417,0.903590,0.130215}%
\pgfsetfillcolor{currentfill}%
\pgfsetlinewidth{0.000000pt}%
\definecolor{currentstroke}{rgb}{0.000000,0.000000,0.000000}%
\pgfsetstrokecolor{currentstroke}%
\pgfsetdash{}{0pt}%
\pgfpathmoveto{\pgfqpoint{1.364010in}{1.739046in}}%
\pgfpathlineto{\pgfqpoint{1.365608in}{1.739479in}}%
\pgfpathlineto{\pgfqpoint{1.367205in}{1.739794in}}%
\pgfpathlineto{\pgfqpoint{1.368801in}{1.739990in}}%
\pgfpathlineto{\pgfqpoint{1.370396in}{1.740068in}}%
\pgfpathlineto{\pgfqpoint{1.371566in}{1.739777in}}%
\pgfpathlineto{\pgfqpoint{1.372715in}{1.739470in}}%
\pgfpathlineto{\pgfqpoint{1.373844in}{1.739145in}}%
\pgfpathlineto{\pgfqpoint{1.374949in}{1.738804in}}%
\pgfpathlineto{\pgfqpoint{1.372976in}{1.738831in}}%
\pgfpathlineto{\pgfqpoint{1.371001in}{1.738740in}}%
\pgfpathlineto{\pgfqpoint{1.369025in}{1.738531in}}%
\pgfpathlineto{\pgfqpoint{1.367047in}{1.738203in}}%
\pgfpathlineto{\pgfqpoint{1.366310in}{1.738430in}}%
\pgfpathlineto{\pgfqpoint{1.365557in}{1.738647in}}%
\pgfpathlineto{\pgfqpoint{1.364790in}{1.738852in}}%
\pgfpathlineto{\pgfqpoint{1.364010in}{1.739046in}}%
\pgfpathclose%
\pgfusepath{fill}%
\end{pgfscope}%
\begin{pgfscope}%
\pgfpathrectangle{\pgfqpoint{0.329460in}{0.284240in}}{\pgfqpoint{1.989680in}{1.989680in}}%
\pgfusepath{clip}%
\pgfsetbuttcap%
\pgfsetroundjoin%
\definecolor{currentfill}{rgb}{0.814576,0.883393,0.110347}%
\pgfsetfillcolor{currentfill}%
\pgfsetlinewidth{0.000000pt}%
\definecolor{currentstroke}{rgb}{0.000000,0.000000,0.000000}%
\pgfsetstrokecolor{currentstroke}%
\pgfsetdash{}{0pt}%
\pgfpathmoveto{\pgfqpoint{1.253333in}{1.681923in}}%
\pgfpathlineto{\pgfqpoint{1.249854in}{1.678520in}}%
\pgfpathlineto{\pgfqpoint{1.246376in}{1.675011in}}%
\pgfpathlineto{\pgfqpoint{1.242900in}{1.671397in}}%
\pgfpathlineto{\pgfqpoint{1.239426in}{1.667679in}}%
\pgfpathlineto{\pgfqpoint{1.239269in}{1.669342in}}%
\pgfpathlineto{\pgfqpoint{1.239223in}{1.671005in}}%
\pgfpathlineto{\pgfqpoint{1.239289in}{1.672668in}}%
\pgfpathlineto{\pgfqpoint{1.239467in}{1.674329in}}%
\pgfpathlineto{\pgfqpoint{1.242936in}{1.677836in}}%
\pgfpathlineto{\pgfqpoint{1.246407in}{1.681240in}}%
\pgfpathlineto{\pgfqpoint{1.249880in}{1.684539in}}%
\pgfpathlineto{\pgfqpoint{1.253355in}{1.687732in}}%
\pgfpathlineto{\pgfqpoint{1.253203in}{1.686282in}}%
\pgfpathlineto{\pgfqpoint{1.253148in}{1.684829in}}%
\pgfpathlineto{\pgfqpoint{1.253192in}{1.683375in}}%
\pgfpathlineto{\pgfqpoint{1.253333in}{1.681923in}}%
\pgfpathclose%
\pgfusepath{fill}%
\end{pgfscope}%
\begin{pgfscope}%
\pgfpathrectangle{\pgfqpoint{0.329460in}{0.284240in}}{\pgfqpoint{1.989680in}{1.989680in}}%
\pgfusepath{clip}%
\pgfsetbuttcap%
\pgfsetroundjoin%
\definecolor{currentfill}{rgb}{0.179019,0.433756,0.557430}%
\pgfsetfillcolor{currentfill}%
\pgfsetlinewidth{0.000000pt}%
\definecolor{currentstroke}{rgb}{0.000000,0.000000,0.000000}%
\pgfsetstrokecolor{currentstroke}%
\pgfsetdash{}{0pt}%
\pgfpathmoveto{\pgfqpoint{1.156381in}{1.156616in}}%
\pgfpathlineto{\pgfqpoint{1.154380in}{1.148261in}}%
\pgfpathlineto{\pgfqpoint{1.152379in}{1.139927in}}%
\pgfpathlineto{\pgfqpoint{1.150379in}{1.131618in}}%
\pgfpathlineto{\pgfqpoint{1.148378in}{1.123337in}}%
\pgfpathlineto{\pgfqpoint{1.139574in}{1.126704in}}%
\pgfpathlineto{\pgfqpoint{1.130995in}{1.130210in}}%
\pgfpathlineto{\pgfqpoint{1.122650in}{1.133851in}}%
\pgfpathlineto{\pgfqpoint{1.114547in}{1.137624in}}%
\pgfpathlineto{\pgfqpoint{1.116875in}{1.145747in}}%
\pgfpathlineto{\pgfqpoint{1.119204in}{1.153897in}}%
\pgfpathlineto{\pgfqpoint{1.121534in}{1.162073in}}%
\pgfpathlineto{\pgfqpoint{1.123864in}{1.170270in}}%
\pgfpathlineto{\pgfqpoint{1.131653in}{1.166664in}}%
\pgfpathlineto{\pgfqpoint{1.139675in}{1.163184in}}%
\pgfpathlineto{\pgfqpoint{1.147920in}{1.159834in}}%
\pgfpathlineto{\pgfqpoint{1.156381in}{1.156616in}}%
\pgfpathclose%
\pgfusepath{fill}%
\end{pgfscope}%
\begin{pgfscope}%
\pgfpathrectangle{\pgfqpoint{0.329460in}{0.284240in}}{\pgfqpoint{1.989680in}{1.989680in}}%
\pgfusepath{clip}%
\pgfsetbuttcap%
\pgfsetroundjoin%
\definecolor{currentfill}{rgb}{0.974417,0.903590,0.130215}%
\pgfsetfillcolor{currentfill}%
\pgfsetlinewidth{0.000000pt}%
\definecolor{currentstroke}{rgb}{0.000000,0.000000,0.000000}%
\pgfsetstrokecolor{currentstroke}%
\pgfsetdash{}{0pt}%
\pgfpathmoveto{\pgfqpoint{1.334685in}{1.737991in}}%
\pgfpathlineto{\pgfqpoint{1.332628in}{1.738293in}}%
\pgfpathlineto{\pgfqpoint{1.330571in}{1.738476in}}%
\pgfpathlineto{\pgfqpoint{1.328516in}{1.738541in}}%
\pgfpathlineto{\pgfqpoint{1.326463in}{1.738488in}}%
\pgfpathlineto{\pgfqpoint{1.327548in}{1.738843in}}%
\pgfpathlineto{\pgfqpoint{1.328656in}{1.739182in}}%
\pgfpathlineto{\pgfqpoint{1.329786in}{1.739505in}}%
\pgfpathlineto{\pgfqpoint{1.330938in}{1.739810in}}%
\pgfpathlineto{\pgfqpoint{1.332620in}{1.739754in}}%
\pgfpathlineto{\pgfqpoint{1.334302in}{1.739579in}}%
\pgfpathlineto{\pgfqpoint{1.335986in}{1.739286in}}%
\pgfpathlineto{\pgfqpoint{1.337671in}{1.738874in}}%
\pgfpathlineto{\pgfqpoint{1.336903in}{1.738670in}}%
\pgfpathlineto{\pgfqpoint{1.336148in}{1.738455in}}%
\pgfpathlineto{\pgfqpoint{1.335409in}{1.738228in}}%
\pgfpathlineto{\pgfqpoint{1.334685in}{1.737991in}}%
\pgfpathclose%
\pgfusepath{fill}%
\end{pgfscope}%
\begin{pgfscope}%
\pgfpathrectangle{\pgfqpoint{0.329460in}{0.284240in}}{\pgfqpoint{1.989680in}{1.989680in}}%
\pgfusepath{clip}%
\pgfsetbuttcap%
\pgfsetroundjoin%
\definecolor{currentfill}{rgb}{0.122606,0.585371,0.546557}%
\pgfsetfillcolor{currentfill}%
\pgfsetlinewidth{0.000000pt}%
\definecolor{currentstroke}{rgb}{0.000000,0.000000,0.000000}%
\pgfsetstrokecolor{currentstroke}%
\pgfsetdash{}{0pt}%
\pgfpathmoveto{\pgfqpoint{1.570058in}{1.318070in}}%
\pgfpathlineto{\pgfqpoint{1.572750in}{1.310091in}}%
\pgfpathlineto{\pgfqpoint{1.575442in}{1.302094in}}%
\pgfpathlineto{\pgfqpoint{1.578132in}{1.294081in}}%
\pgfpathlineto{\pgfqpoint{1.580821in}{1.286054in}}%
\pgfpathlineto{\pgfqpoint{1.575061in}{1.282461in}}%
\pgfpathlineto{\pgfqpoint{1.569067in}{1.278960in}}%
\pgfpathlineto{\pgfqpoint{1.562846in}{1.275553in}}%
\pgfpathlineto{\pgfqpoint{1.556403in}{1.272245in}}%
\pgfpathlineto{\pgfqpoint{1.553995in}{1.280452in}}%
\pgfpathlineto{\pgfqpoint{1.551586in}{1.288645in}}%
\pgfpathlineto{\pgfqpoint{1.549175in}{1.296821in}}%
\pgfpathlineto{\pgfqpoint{1.546764in}{1.304977in}}%
\pgfpathlineto{\pgfqpoint{1.552909in}{1.308114in}}%
\pgfpathlineto{\pgfqpoint{1.558843in}{1.311343in}}%
\pgfpathlineto{\pgfqpoint{1.564561in}{1.314663in}}%
\pgfpathlineto{\pgfqpoint{1.570058in}{1.318070in}}%
\pgfpathclose%
\pgfusepath{fill}%
\end{pgfscope}%
\begin{pgfscope}%
\pgfpathrectangle{\pgfqpoint{0.329460in}{0.284240in}}{\pgfqpoint{1.989680in}{1.989680in}}%
\pgfusepath{clip}%
\pgfsetbuttcap%
\pgfsetroundjoin%
\definecolor{currentfill}{rgb}{0.636902,0.856542,0.216620}%
\pgfsetfillcolor{currentfill}%
\pgfsetlinewidth{0.000000pt}%
\definecolor{currentstroke}{rgb}{0.000000,0.000000,0.000000}%
\pgfsetstrokecolor{currentstroke}%
\pgfsetdash{}{0pt}%
\pgfpathmoveto{\pgfqpoint{1.489202in}{1.627866in}}%
\pgfpathlineto{\pgfqpoint{1.492626in}{1.623099in}}%
\pgfpathlineto{\pgfqpoint{1.496047in}{1.618239in}}%
\pgfpathlineto{\pgfqpoint{1.499466in}{1.613289in}}%
\pgfpathlineto{\pgfqpoint{1.502884in}{1.608249in}}%
\pgfpathlineto{\pgfqpoint{1.502056in}{1.605976in}}%
\pgfpathlineto{\pgfqpoint{1.501077in}{1.603716in}}%
\pgfpathlineto{\pgfqpoint{1.499948in}{1.601472in}}%
\pgfpathlineto{\pgfqpoint{1.498669in}{1.599244in}}%
\pgfpathlineto{\pgfqpoint{1.495342in}{1.604494in}}%
\pgfpathlineto{\pgfqpoint{1.492012in}{1.609655in}}%
\pgfpathlineto{\pgfqpoint{1.488682in}{1.614725in}}%
\pgfpathlineto{\pgfqpoint{1.485349in}{1.619701in}}%
\pgfpathlineto{\pgfqpoint{1.486517in}{1.621721in}}%
\pgfpathlineto{\pgfqpoint{1.487549in}{1.623756in}}%
\pgfpathlineto{\pgfqpoint{1.488445in}{1.625805in}}%
\pgfpathlineto{\pgfqpoint{1.489202in}{1.627866in}}%
\pgfpathclose%
\pgfusepath{fill}%
\end{pgfscope}%
\begin{pgfscope}%
\pgfpathrectangle{\pgfqpoint{0.329460in}{0.284240in}}{\pgfqpoint{1.989680in}{1.989680in}}%
\pgfusepath{clip}%
\pgfsetbuttcap%
\pgfsetroundjoin%
\definecolor{currentfill}{rgb}{0.282327,0.094955,0.417331}%
\pgfsetfillcolor{currentfill}%
\pgfsetlinewidth{0.000000pt}%
\definecolor{currentstroke}{rgb}{0.000000,0.000000,0.000000}%
\pgfsetstrokecolor{currentstroke}%
\pgfsetdash{}{0pt}%
\pgfpathmoveto{\pgfqpoint{1.820608in}{0.872452in}}%
\pgfpathlineto{\pgfqpoint{1.823538in}{0.875831in}}%
\pgfpathlineto{\pgfqpoint{1.826478in}{0.879523in}}%
\pgfpathlineto{\pgfqpoint{1.829429in}{0.883533in}}%
\pgfpathlineto{\pgfqpoint{1.832392in}{0.887867in}}%
\pgfpathlineto{\pgfqpoint{1.820617in}{0.879852in}}%
\pgfpathlineto{\pgfqpoint{1.808331in}{0.872031in}}%
\pgfpathlineto{\pgfqpoint{1.795544in}{0.864413in}}%
\pgfpathlineto{\pgfqpoint{1.782270in}{0.857007in}}%
\pgfpathlineto{\pgfqpoint{1.779604in}{0.852844in}}%
\pgfpathlineto{\pgfqpoint{1.776948in}{0.849005in}}%
\pgfpathlineto{\pgfqpoint{1.774302in}{0.845486in}}%
\pgfpathlineto{\pgfqpoint{1.771666in}{0.842280in}}%
\pgfpathlineto{\pgfqpoint{1.784625in}{0.849521in}}%
\pgfpathlineto{\pgfqpoint{1.797110in}{0.856969in}}%
\pgfpathlineto{\pgfqpoint{1.809108in}{0.864615in}}%
\pgfpathlineto{\pgfqpoint{1.820608in}{0.872452in}}%
\pgfpathclose%
\pgfusepath{fill}%
\end{pgfscope}%
\begin{pgfscope}%
\pgfpathrectangle{\pgfqpoint{0.329460in}{0.284240in}}{\pgfqpoint{1.989680in}{1.989680in}}%
\pgfusepath{clip}%
\pgfsetbuttcap%
\pgfsetroundjoin%
\definecolor{currentfill}{rgb}{0.231674,0.318106,0.544834}%
\pgfsetfillcolor{currentfill}%
\pgfsetlinewidth{0.000000pt}%
\definecolor{currentstroke}{rgb}{0.000000,0.000000,0.000000}%
\pgfsetstrokecolor{currentstroke}%
\pgfsetdash{}{0pt}%
\pgfpathmoveto{\pgfqpoint{1.578439in}{1.061668in}}%
\pgfpathlineto{\pgfqpoint{1.580515in}{1.053797in}}%
\pgfpathlineto{\pgfqpoint{1.582591in}{1.045982in}}%
\pgfpathlineto{\pgfqpoint{1.584667in}{1.038227in}}%
\pgfpathlineto{\pgfqpoint{1.586743in}{1.030535in}}%
\pgfpathlineto{\pgfqpoint{1.576885in}{1.026731in}}%
\pgfpathlineto{\pgfqpoint{1.566784in}{1.023089in}}%
\pgfpathlineto{\pgfqpoint{1.556451in}{1.019615in}}%
\pgfpathlineto{\pgfqpoint{1.545898in}{1.016311in}}%
\pgfpathlineto{\pgfqpoint{1.544178in}{1.024146in}}%
\pgfpathlineto{\pgfqpoint{1.542458in}{1.032043in}}%
\pgfpathlineto{\pgfqpoint{1.540738in}{1.040000in}}%
\pgfpathlineto{\pgfqpoint{1.539019in}{1.048013in}}%
\pgfpathlineto{\pgfqpoint{1.549203in}{1.051185in}}%
\pgfpathlineto{\pgfqpoint{1.559175in}{1.054521in}}%
\pgfpathlineto{\pgfqpoint{1.568924in}{1.058016in}}%
\pgfpathlineto{\pgfqpoint{1.578439in}{1.061668in}}%
\pgfpathclose%
\pgfusepath{fill}%
\end{pgfscope}%
\begin{pgfscope}%
\pgfpathrectangle{\pgfqpoint{0.329460in}{0.284240in}}{\pgfqpoint{1.989680in}{1.989680in}}%
\pgfusepath{clip}%
\pgfsetbuttcap%
\pgfsetroundjoin%
\definecolor{currentfill}{rgb}{0.955300,0.901065,0.118128}%
\pgfsetfillcolor{currentfill}%
\pgfsetlinewidth{0.000000pt}%
\definecolor{currentstroke}{rgb}{0.000000,0.000000,0.000000}%
\pgfsetstrokecolor{currentstroke}%
\pgfsetdash{}{0pt}%
\pgfpathmoveto{\pgfqpoint{1.386097in}{1.733570in}}%
\pgfpathlineto{\pgfqpoint{1.388997in}{1.732990in}}%
\pgfpathlineto{\pgfqpoint{1.391896in}{1.732294in}}%
\pgfpathlineto{\pgfqpoint{1.394793in}{1.731482in}}%
\pgfpathlineto{\pgfqpoint{1.397688in}{1.730553in}}%
\pgfpathlineto{\pgfqpoint{1.398654in}{1.729862in}}%
\pgfpathlineto{\pgfqpoint{1.399573in}{1.729157in}}%
\pgfpathlineto{\pgfqpoint{1.400444in}{1.728439in}}%
\pgfpathlineto{\pgfqpoint{1.401266in}{1.727708in}}%
\pgfpathlineto{\pgfqpoint{1.398147in}{1.728814in}}%
\pgfpathlineto{\pgfqpoint{1.395026in}{1.729804in}}%
\pgfpathlineto{\pgfqpoint{1.391903in}{1.730678in}}%
\pgfpathlineto{\pgfqpoint{1.388779in}{1.731435in}}%
\pgfpathlineto{\pgfqpoint{1.388163in}{1.731983in}}%
\pgfpathlineto{\pgfqpoint{1.387510in}{1.732522in}}%
\pgfpathlineto{\pgfqpoint{1.386821in}{1.733051in}}%
\pgfpathlineto{\pgfqpoint{1.386097in}{1.733570in}}%
\pgfpathclose%
\pgfusepath{fill}%
\end{pgfscope}%
\begin{pgfscope}%
\pgfpathrectangle{\pgfqpoint{0.329460in}{0.284240in}}{\pgfqpoint{1.989680in}{1.989680in}}%
\pgfusepath{clip}%
\pgfsetbuttcap%
\pgfsetroundjoin%
\definecolor{currentfill}{rgb}{0.163625,0.471133,0.558148}%
\pgfsetfillcolor{currentfill}%
\pgfsetlinewidth{0.000000pt}%
\definecolor{currentstroke}{rgb}{0.000000,0.000000,0.000000}%
\pgfsetstrokecolor{currentstroke}%
\pgfsetdash{}{0pt}%
\pgfpathmoveto{\pgfqpoint{1.575636in}{1.206382in}}%
\pgfpathlineto{\pgfqpoint{1.578037in}{1.198161in}}%
\pgfpathlineto{\pgfqpoint{1.580436in}{1.189951in}}%
\pgfpathlineto{\pgfqpoint{1.582835in}{1.181756in}}%
\pgfpathlineto{\pgfqpoint{1.585234in}{1.173577in}}%
\pgfpathlineto{\pgfqpoint{1.577657in}{1.169863in}}%
\pgfpathlineto{\pgfqpoint{1.569842in}{1.166271in}}%
\pgfpathlineto{\pgfqpoint{1.561795in}{1.162805in}}%
\pgfpathlineto{\pgfqpoint{1.553526in}{1.159470in}}%
\pgfpathlineto{\pgfqpoint{1.551448in}{1.167811in}}%
\pgfpathlineto{\pgfqpoint{1.549369in}{1.176169in}}%
\pgfpathlineto{\pgfqpoint{1.547290in}{1.184541in}}%
\pgfpathlineto{\pgfqpoint{1.545211in}{1.192924in}}%
\pgfpathlineto{\pgfqpoint{1.553145in}{1.196106in}}%
\pgfpathlineto{\pgfqpoint{1.560866in}{1.199412in}}%
\pgfpathlineto{\pgfqpoint{1.568366in}{1.202839in}}%
\pgfpathlineto{\pgfqpoint{1.575636in}{1.206382in}}%
\pgfpathclose%
\pgfusepath{fill}%
\end{pgfscope}%
\begin{pgfscope}%
\pgfpathrectangle{\pgfqpoint{0.329460in}{0.284240in}}{\pgfqpoint{1.989680in}{1.989680in}}%
\pgfusepath{clip}%
\pgfsetbuttcap%
\pgfsetroundjoin%
\definecolor{currentfill}{rgb}{0.281477,0.755203,0.432552}%
\pgfsetfillcolor{currentfill}%
\pgfsetlinewidth{0.000000pt}%
\definecolor{currentstroke}{rgb}{0.000000,0.000000,0.000000}%
\pgfsetstrokecolor{currentstroke}%
\pgfsetdash{}{0pt}%
\pgfpathmoveto{\pgfqpoint{1.174628in}{1.478408in}}%
\pgfpathlineto{\pgfqpoint{1.171716in}{1.471414in}}%
\pgfpathlineto{\pgfqpoint{1.168806in}{1.464357in}}%
\pgfpathlineto{\pgfqpoint{1.165897in}{1.457242in}}%
\pgfpathlineto{\pgfqpoint{1.162989in}{1.450069in}}%
\pgfpathlineto{\pgfqpoint{1.159330in}{1.452999in}}%
\pgfpathlineto{\pgfqpoint{1.155867in}{1.455982in}}%
\pgfpathlineto{\pgfqpoint{1.152605in}{1.459016in}}%
\pgfpathlineto{\pgfqpoint{1.149544in}{1.462097in}}%
\pgfpathlineto{\pgfqpoint{1.152653in}{1.469070in}}%
\pgfpathlineto{\pgfqpoint{1.155764in}{1.475987in}}%
\pgfpathlineto{\pgfqpoint{1.158877in}{1.482844in}}%
\pgfpathlineto{\pgfqpoint{1.161992in}{1.489641in}}%
\pgfpathlineto{\pgfqpoint{1.164869in}{1.486764in}}%
\pgfpathlineto{\pgfqpoint{1.167936in}{1.483930in}}%
\pgfpathlineto{\pgfqpoint{1.171190in}{1.481144in}}%
\pgfpathlineto{\pgfqpoint{1.174628in}{1.478408in}}%
\pgfpathclose%
\pgfusepath{fill}%
\end{pgfscope}%
\begin{pgfscope}%
\pgfpathrectangle{\pgfqpoint{0.329460in}{0.284240in}}{\pgfqpoint{1.989680in}{1.989680in}}%
\pgfusepath{clip}%
\pgfsetbuttcap%
\pgfsetroundjoin%
\definecolor{currentfill}{rgb}{0.762373,0.876424,0.137064}%
\pgfsetfillcolor{currentfill}%
\pgfsetlinewidth{0.000000pt}%
\definecolor{currentstroke}{rgb}{0.000000,0.000000,0.000000}%
\pgfsetstrokecolor{currentstroke}%
\pgfsetdash{}{0pt}%
\pgfpathmoveto{\pgfqpoint{1.463094in}{1.669157in}}%
\pgfpathlineto{\pgfqpoint{1.466570in}{1.665383in}}%
\pgfpathlineto{\pgfqpoint{1.470044in}{1.661508in}}%
\pgfpathlineto{\pgfqpoint{1.473516in}{1.657531in}}%
\pgfpathlineto{\pgfqpoint{1.476986in}{1.653456in}}%
\pgfpathlineto{\pgfqpoint{1.476800in}{1.651582in}}%
\pgfpathlineto{\pgfqpoint{1.476488in}{1.649711in}}%
\pgfpathlineto{\pgfqpoint{1.476051in}{1.647846in}}%
\pgfpathlineto{\pgfqpoint{1.475489in}{1.645987in}}%
\pgfpathlineto{\pgfqpoint{1.472056in}{1.650273in}}%
\pgfpathlineto{\pgfqpoint{1.468621in}{1.654460in}}%
\pgfpathlineto{\pgfqpoint{1.465185in}{1.658547in}}%
\pgfpathlineto{\pgfqpoint{1.461747in}{1.662531in}}%
\pgfpathlineto{\pgfqpoint{1.462250in}{1.664180in}}%
\pgfpathlineto{\pgfqpoint{1.462643in}{1.665835in}}%
\pgfpathlineto{\pgfqpoint{1.462924in}{1.667494in}}%
\pgfpathlineto{\pgfqpoint{1.463094in}{1.669157in}}%
\pgfpathclose%
\pgfusepath{fill}%
\end{pgfscope}%
\begin{pgfscope}%
\pgfpathrectangle{\pgfqpoint{0.329460in}{0.284240in}}{\pgfqpoint{1.989680in}{1.989680in}}%
\pgfusepath{clip}%
\pgfsetbuttcap%
\pgfsetroundjoin%
\definecolor{currentfill}{rgb}{0.166383,0.690856,0.496502}%
\pgfsetfillcolor{currentfill}%
\pgfsetlinewidth{0.000000pt}%
\definecolor{currentstroke}{rgb}{0.000000,0.000000,0.000000}%
\pgfsetstrokecolor{currentstroke}%
\pgfsetdash{}{0pt}%
\pgfpathmoveto{\pgfqpoint{1.554457in}{1.423620in}}%
\pgfpathlineto{\pgfqpoint{1.557405in}{1.416234in}}%
\pgfpathlineto{\pgfqpoint{1.560352in}{1.408802in}}%
\pgfpathlineto{\pgfqpoint{1.563296in}{1.401328in}}%
\pgfpathlineto{\pgfqpoint{1.566239in}{1.393813in}}%
\pgfpathlineto{\pgfqpoint{1.562117in}{1.390497in}}%
\pgfpathlineto{\pgfqpoint{1.557778in}{1.387245in}}%
\pgfpathlineto{\pgfqpoint{1.553226in}{1.384061in}}%
\pgfpathlineto{\pgfqpoint{1.548466in}{1.380948in}}%
\pgfpathlineto{\pgfqpoint{1.545761in}{1.388655in}}%
\pgfpathlineto{\pgfqpoint{1.543054in}{1.396320in}}%
\pgfpathlineto{\pgfqpoint{1.540346in}{1.403942in}}%
\pgfpathlineto{\pgfqpoint{1.537637in}{1.411519in}}%
\pgfpathlineto{\pgfqpoint{1.542141in}{1.414447in}}%
\pgfpathlineto{\pgfqpoint{1.546449in}{1.417442in}}%
\pgfpathlineto{\pgfqpoint{1.550555in}{1.420500in}}%
\pgfpathlineto{\pgfqpoint{1.554457in}{1.423620in}}%
\pgfpathclose%
\pgfusepath{fill}%
\end{pgfscope}%
\begin{pgfscope}%
\pgfpathrectangle{\pgfqpoint{0.329460in}{0.284240in}}{\pgfqpoint{1.989680in}{1.989680in}}%
\pgfusepath{clip}%
\pgfsetbuttcap%
\pgfsetroundjoin%
\definecolor{currentfill}{rgb}{0.248629,0.278775,0.534556}%
\pgfsetfillcolor{currentfill}%
\pgfsetlinewidth{0.000000pt}%
\definecolor{currentstroke}{rgb}{0.000000,0.000000,0.000000}%
\pgfsetstrokecolor{currentstroke}%
\pgfsetdash{}{0pt}%
\pgfpathmoveto{\pgfqpoint{1.166035in}{1.013521in}}%
\pgfpathlineto{\pgfqpoint{1.164398in}{1.005723in}}%
\pgfpathlineto{\pgfqpoint{1.162761in}{0.997995in}}%
\pgfpathlineto{\pgfqpoint{1.161124in}{0.990339in}}%
\pgfpathlineto{\pgfqpoint{1.159486in}{0.982757in}}%
\pgfpathlineto{\pgfqpoint{1.148369in}{0.986033in}}%
\pgfpathlineto{\pgfqpoint{1.137471in}{0.989490in}}%
\pgfpathlineto{\pgfqpoint{1.126803in}{0.993123in}}%
\pgfpathlineto{\pgfqpoint{1.116378in}{0.996930in}}%
\pgfpathlineto{\pgfqpoint{1.118380in}{1.004374in}}%
\pgfpathlineto{\pgfqpoint{1.120381in}{1.011894in}}%
\pgfpathlineto{\pgfqpoint{1.122382in}{1.019485in}}%
\pgfpathlineto{\pgfqpoint{1.124383in}{1.027146in}}%
\pgfpathlineto{\pgfqpoint{1.134457in}{1.023486in}}%
\pgfpathlineto{\pgfqpoint{1.144765in}{1.019993in}}%
\pgfpathlineto{\pgfqpoint{1.155294in}{1.016670in}}%
\pgfpathlineto{\pgfqpoint{1.166035in}{1.013521in}}%
\pgfpathclose%
\pgfusepath{fill}%
\end{pgfscope}%
\begin{pgfscope}%
\pgfpathrectangle{\pgfqpoint{0.329460in}{0.284240in}}{\pgfqpoint{1.989680in}{1.989680in}}%
\pgfusepath{clip}%
\pgfsetbuttcap%
\pgfsetroundjoin%
\definecolor{currentfill}{rgb}{0.974417,0.903590,0.130215}%
\pgfsetfillcolor{currentfill}%
\pgfsetlinewidth{0.000000pt}%
\definecolor{currentstroke}{rgb}{0.000000,0.000000,0.000000}%
\pgfsetstrokecolor{currentstroke}%
\pgfsetdash{}{0pt}%
\pgfpathmoveto{\pgfqpoint{1.367047in}{1.738203in}}%
\pgfpathlineto{\pgfqpoint{1.369025in}{1.738531in}}%
\pgfpathlineto{\pgfqpoint{1.371001in}{1.738740in}}%
\pgfpathlineto{\pgfqpoint{1.372976in}{1.738831in}}%
\pgfpathlineto{\pgfqpoint{1.374949in}{1.738804in}}%
\pgfpathlineto{\pgfqpoint{1.376032in}{1.738447in}}%
\pgfpathlineto{\pgfqpoint{1.377089in}{1.738075in}}%
\pgfpathlineto{\pgfqpoint{1.378121in}{1.737687in}}%
\pgfpathlineto{\pgfqpoint{1.379127in}{1.737283in}}%
\pgfpathlineto{\pgfqpoint{1.376806in}{1.737437in}}%
\pgfpathlineto{\pgfqpoint{1.374483in}{1.737472in}}%
\pgfpathlineto{\pgfqpoint{1.372159in}{1.737389in}}%
\pgfpathlineto{\pgfqpoint{1.369834in}{1.737187in}}%
\pgfpathlineto{\pgfqpoint{1.369163in}{1.737457in}}%
\pgfpathlineto{\pgfqpoint{1.368475in}{1.737716in}}%
\pgfpathlineto{\pgfqpoint{1.367769in}{1.737964in}}%
\pgfpathlineto{\pgfqpoint{1.367047in}{1.738203in}}%
\pgfpathclose%
\pgfusepath{fill}%
\end{pgfscope}%
\begin{pgfscope}%
\pgfpathrectangle{\pgfqpoint{0.329460in}{0.284240in}}{\pgfqpoint{1.989680in}{1.989680in}}%
\pgfusepath{clip}%
\pgfsetbuttcap%
\pgfsetroundjoin%
\definecolor{currentfill}{rgb}{0.487026,0.823929,0.312321}%
\pgfsetfillcolor{currentfill}%
\pgfsetlinewidth{0.000000pt}%
\definecolor{currentstroke}{rgb}{0.000000,0.000000,0.000000}%
\pgfsetstrokecolor{currentstroke}%
\pgfsetdash{}{0pt}%
\pgfpathmoveto{\pgfqpoint{1.511959in}{1.577371in}}%
\pgfpathlineto{\pgfqpoint{1.515277in}{1.571693in}}%
\pgfpathlineto{\pgfqpoint{1.518593in}{1.565933in}}%
\pgfpathlineto{\pgfqpoint{1.521906in}{1.560094in}}%
\pgfpathlineto{\pgfqpoint{1.525218in}{1.554178in}}%
\pgfpathlineto{\pgfqpoint{1.523545in}{1.551552in}}%
\pgfpathlineto{\pgfqpoint{1.521697in}{1.548952in}}%
\pgfpathlineto{\pgfqpoint{1.519678in}{1.546380in}}%
\pgfpathlineto{\pgfqpoint{1.517487in}{1.543839in}}%
\pgfpathlineto{\pgfqpoint{1.514317in}{1.549963in}}%
\pgfpathlineto{\pgfqpoint{1.511146in}{1.556008in}}%
\pgfpathlineto{\pgfqpoint{1.507972in}{1.561974in}}%
\pgfpathlineto{\pgfqpoint{1.504797in}{1.567858in}}%
\pgfpathlineto{\pgfqpoint{1.506826in}{1.570196in}}%
\pgfpathlineto{\pgfqpoint{1.508696in}{1.572562in}}%
\pgfpathlineto{\pgfqpoint{1.510408in}{1.574955in}}%
\pgfpathlineto{\pgfqpoint{1.511959in}{1.577371in}}%
\pgfpathclose%
\pgfusepath{fill}%
\end{pgfscope}%
\begin{pgfscope}%
\pgfpathrectangle{\pgfqpoint{0.329460in}{0.284240in}}{\pgfqpoint{1.989680in}{1.989680in}}%
\pgfusepath{clip}%
\pgfsetbuttcap%
\pgfsetroundjoin%
\definecolor{currentfill}{rgb}{0.955300,0.901065,0.118128}%
\pgfsetfillcolor{currentfill}%
\pgfsetlinewidth{0.000000pt}%
\definecolor{currentstroke}{rgb}{0.000000,0.000000,0.000000}%
\pgfsetstrokecolor{currentstroke}%
\pgfsetdash{}{0pt}%
\pgfpathmoveto{\pgfqpoint{1.313080in}{1.730940in}}%
\pgfpathlineto{\pgfqpoint{1.309913in}{1.730142in}}%
\pgfpathlineto{\pgfqpoint{1.306747in}{1.729227in}}%
\pgfpathlineto{\pgfqpoint{1.303582in}{1.728195in}}%
\pgfpathlineto{\pgfqpoint{1.300420in}{1.727048in}}%
\pgfpathlineto{\pgfqpoint{1.301198in}{1.727790in}}%
\pgfpathlineto{\pgfqpoint{1.302025in}{1.728519in}}%
\pgfpathlineto{\pgfqpoint{1.302902in}{1.729236in}}%
\pgfpathlineto{\pgfqpoint{1.303826in}{1.729940in}}%
\pgfpathlineto{\pgfqpoint{1.306775in}{1.730906in}}%
\pgfpathlineto{\pgfqpoint{1.309726in}{1.731757in}}%
\pgfpathlineto{\pgfqpoint{1.312679in}{1.732492in}}%
\pgfpathlineto{\pgfqpoint{1.315633in}{1.733109in}}%
\pgfpathlineto{\pgfqpoint{1.314940in}{1.732581in}}%
\pgfpathlineto{\pgfqpoint{1.314283in}{1.732044in}}%
\pgfpathlineto{\pgfqpoint{1.313663in}{1.731496in}}%
\pgfpathlineto{\pgfqpoint{1.313080in}{1.730940in}}%
\pgfpathclose%
\pgfusepath{fill}%
\end{pgfscope}%
\begin{pgfscope}%
\pgfpathrectangle{\pgfqpoint{0.329460in}{0.284240in}}{\pgfqpoint{1.989680in}{1.989680in}}%
\pgfusepath{clip}%
\pgfsetbuttcap%
\pgfsetroundjoin%
\definecolor{currentfill}{rgb}{0.233603,0.313828,0.543914}%
\pgfsetfillcolor{currentfill}%
\pgfsetlinewidth{0.000000pt}%
\definecolor{currentstroke}{rgb}{0.000000,0.000000,0.000000}%
\pgfsetstrokecolor{currentstroke}%
\pgfsetdash{}{0pt}%
\pgfpathmoveto{\pgfqpoint{0.797843in}{0.994964in}}%
\pgfpathlineto{\pgfqpoint{0.794504in}{1.004180in}}%
\pgfpathlineto{\pgfqpoint{0.791148in}{1.013807in}}%
\pgfpathlineto{\pgfqpoint{0.787775in}{1.023849in}}%
\pgfpathlineto{\pgfqpoint{0.784384in}{1.034316in}}%
\pgfpathlineto{\pgfqpoint{0.773925in}{1.043736in}}%
\pgfpathlineto{\pgfqpoint{0.764087in}{1.053309in}}%
\pgfpathlineto{\pgfqpoint{0.754878in}{1.063023in}}%
\pgfpathlineto{\pgfqpoint{0.746305in}{1.072869in}}%
\pgfpathlineto{\pgfqpoint{0.749902in}{1.062235in}}%
\pgfpathlineto{\pgfqpoint{0.753481in}{1.052023in}}%
\pgfpathlineto{\pgfqpoint{0.757042in}{1.042225in}}%
\pgfpathlineto{\pgfqpoint{0.760585in}{1.032836in}}%
\pgfpathlineto{\pgfqpoint{0.768976in}{1.023163in}}%
\pgfpathlineto{\pgfqpoint{0.777988in}{1.013619in}}%
\pgfpathlineto{\pgfqpoint{0.787612in}{1.004216in}}%
\pgfpathlineto{\pgfqpoint{0.797843in}{0.994964in}}%
\pgfpathclose%
\pgfusepath{fill}%
\end{pgfscope}%
\begin{pgfscope}%
\pgfpathrectangle{\pgfqpoint{0.329460in}{0.284240in}}{\pgfqpoint{1.989680in}{1.989680in}}%
\pgfusepath{clip}%
\pgfsetbuttcap%
\pgfsetroundjoin%
\definecolor{currentfill}{rgb}{0.268510,0.009605,0.335427}%
\pgfsetfillcolor{currentfill}%
\pgfsetlinewidth{0.000000pt}%
\definecolor{currentstroke}{rgb}{0.000000,0.000000,0.000000}%
\pgfsetstrokecolor{currentstroke}%
\pgfsetdash{}{0pt}%
\pgfpathmoveto{\pgfqpoint{1.662351in}{0.823278in}}%
\pgfpathlineto{\pgfqpoint{1.664499in}{0.820293in}}%
\pgfpathlineto{\pgfqpoint{1.666650in}{0.817507in}}%
\pgfpathlineto{\pgfqpoint{1.668806in}{0.814923in}}%
\pgfpathlineto{\pgfqpoint{1.670966in}{0.812548in}}%
\pgfpathlineto{\pgfqpoint{1.657646in}{0.807209in}}%
\pgfpathlineto{\pgfqpoint{1.643990in}{0.802096in}}%
\pgfpathlineto{\pgfqpoint{1.630011in}{0.797216in}}%
\pgfpathlineto{\pgfqpoint{1.615723in}{0.792574in}}%
\pgfpathlineto{\pgfqpoint{1.613931in}{0.795092in}}%
\pgfpathlineto{\pgfqpoint{1.612143in}{0.797818in}}%
\pgfpathlineto{\pgfqpoint{1.610358in}{0.800747in}}%
\pgfpathlineto{\pgfqpoint{1.608576in}{0.803875in}}%
\pgfpathlineto{\pgfqpoint{1.622483in}{0.808385in}}%
\pgfpathlineto{\pgfqpoint{1.636090in}{0.813125in}}%
\pgfpathlineto{\pgfqpoint{1.649384in}{0.818092in}}%
\pgfpathlineto{\pgfqpoint{1.662351in}{0.823278in}}%
\pgfpathclose%
\pgfusepath{fill}%
\end{pgfscope}%
\begin{pgfscope}%
\pgfpathrectangle{\pgfqpoint{0.329460in}{0.284240in}}{\pgfqpoint{1.989680in}{1.989680in}}%
\pgfusepath{clip}%
\pgfsetbuttcap%
\pgfsetroundjoin%
\definecolor{currentfill}{rgb}{0.974417,0.903590,0.130215}%
\pgfsetfillcolor{currentfill}%
\pgfsetlinewidth{0.000000pt}%
\definecolor{currentstroke}{rgb}{0.000000,0.000000,0.000000}%
\pgfsetstrokecolor{currentstroke}%
\pgfsetdash{}{0pt}%
\pgfpathmoveto{\pgfqpoint{1.331960in}{1.736940in}}%
\pgfpathlineto{\pgfqpoint{1.329563in}{1.737111in}}%
\pgfpathlineto{\pgfqpoint{1.327166in}{1.737163in}}%
\pgfpathlineto{\pgfqpoint{1.324771in}{1.737097in}}%
\pgfpathlineto{\pgfqpoint{1.322378in}{1.736913in}}%
\pgfpathlineto{\pgfqpoint{1.323359in}{1.737329in}}%
\pgfpathlineto{\pgfqpoint{1.324367in}{1.737730in}}%
\pgfpathlineto{\pgfqpoint{1.325402in}{1.738117in}}%
\pgfpathlineto{\pgfqpoint{1.326463in}{1.738488in}}%
\pgfpathlineto{\pgfqpoint{1.328516in}{1.738541in}}%
\pgfpathlineto{\pgfqpoint{1.330571in}{1.738476in}}%
\pgfpathlineto{\pgfqpoint{1.332628in}{1.738293in}}%
\pgfpathlineto{\pgfqpoint{1.334685in}{1.737991in}}%
\pgfpathlineto{\pgfqpoint{1.333978in}{1.737744in}}%
\pgfpathlineto{\pgfqpoint{1.333287in}{1.737486in}}%
\pgfpathlineto{\pgfqpoint{1.332615in}{1.737218in}}%
\pgfpathlineto{\pgfqpoint{1.331960in}{1.736940in}}%
\pgfpathclose%
\pgfusepath{fill}%
\end{pgfscope}%
\begin{pgfscope}%
\pgfpathrectangle{\pgfqpoint{0.329460in}{0.284240in}}{\pgfqpoint{1.989680in}{1.989680in}}%
\pgfusepath{clip}%
\pgfsetbuttcap%
\pgfsetroundjoin%
\definecolor{currentfill}{rgb}{0.271305,0.019942,0.347269}%
\pgfsetfillcolor{currentfill}%
\pgfsetlinewidth{0.000000pt}%
\definecolor{currentstroke}{rgb}{0.000000,0.000000,0.000000}%
\pgfsetstrokecolor{currentstroke}%
\pgfsetdash{}{0pt}%
\pgfpathmoveto{\pgfqpoint{1.113149in}{0.814599in}}%
\pgfpathlineto{\pgfqpoint{1.111466in}{0.810688in}}%
\pgfpathlineto{\pgfqpoint{1.109780in}{0.806959in}}%
\pgfpathlineto{\pgfqpoint{1.108091in}{0.803417in}}%
\pgfpathlineto{\pgfqpoint{1.106399in}{0.800065in}}%
\pgfpathlineto{\pgfqpoint{1.092239in}{0.804365in}}%
\pgfpathlineto{\pgfqpoint{1.078365in}{0.808900in}}%
\pgfpathlineto{\pgfqpoint{1.064792in}{0.813666in}}%
\pgfpathlineto{\pgfqpoint{1.051533in}{0.818658in}}%
\pgfpathlineto{\pgfqpoint{1.053599in}{0.821872in}}%
\pgfpathlineto{\pgfqpoint{1.055661in}{0.825276in}}%
\pgfpathlineto{\pgfqpoint{1.057719in}{0.828867in}}%
\pgfpathlineto{\pgfqpoint{1.059774in}{0.832641in}}%
\pgfpathlineto{\pgfqpoint{1.072674in}{0.827797in}}%
\pgfpathlineto{\pgfqpoint{1.085879in}{0.823172in}}%
\pgfpathlineto{\pgfqpoint{1.099375in}{0.818771in}}%
\pgfpathlineto{\pgfqpoint{1.113149in}{0.814599in}}%
\pgfpathclose%
\pgfusepath{fill}%
\end{pgfscope}%
\begin{pgfscope}%
\pgfpathrectangle{\pgfqpoint{0.329460in}{0.284240in}}{\pgfqpoint{1.989680in}{1.989680in}}%
\pgfusepath{clip}%
\pgfsetbuttcap%
\pgfsetroundjoin%
\definecolor{currentfill}{rgb}{0.896320,0.893616,0.096335}%
\pgfsetfillcolor{currentfill}%
\pgfsetlinewidth{0.000000pt}%
\definecolor{currentstroke}{rgb}{0.000000,0.000000,0.000000}%
\pgfsetstrokecolor{currentstroke}%
\pgfsetdash{}{0pt}%
\pgfpathmoveto{\pgfqpoint{1.419655in}{1.714372in}}%
\pgfpathlineto{\pgfqpoint{1.423065in}{1.712307in}}%
\pgfpathlineto{\pgfqpoint{1.426473in}{1.710129in}}%
\pgfpathlineto{\pgfqpoint{1.429879in}{1.707840in}}%
\pgfpathlineto{\pgfqpoint{1.433284in}{1.705440in}}%
\pgfpathlineto{\pgfqpoint{1.433817in}{1.704224in}}%
\pgfpathlineto{\pgfqpoint{1.434267in}{1.702999in}}%
\pgfpathlineto{\pgfqpoint{1.434635in}{1.701769in}}%
\pgfpathlineto{\pgfqpoint{1.434920in}{1.700533in}}%
\pgfpathlineto{\pgfqpoint{1.431444in}{1.703138in}}%
\pgfpathlineto{\pgfqpoint{1.427967in}{1.705633in}}%
\pgfpathlineto{\pgfqpoint{1.424489in}{1.708016in}}%
\pgfpathlineto{\pgfqpoint{1.421009in}{1.710286in}}%
\pgfpathlineto{\pgfqpoint{1.420774in}{1.711315in}}%
\pgfpathlineto{\pgfqpoint{1.420469in}{1.712340in}}%
\pgfpathlineto{\pgfqpoint{1.420096in}{1.713359in}}%
\pgfpathlineto{\pgfqpoint{1.419655in}{1.714372in}}%
\pgfpathclose%
\pgfusepath{fill}%
\end{pgfscope}%
\begin{pgfscope}%
\pgfpathrectangle{\pgfqpoint{0.329460in}{0.284240in}}{\pgfqpoint{1.989680in}{1.989680in}}%
\pgfusepath{clip}%
\pgfsetbuttcap%
\pgfsetroundjoin%
\definecolor{currentfill}{rgb}{0.935904,0.898570,0.108131}%
\pgfsetfillcolor{currentfill}%
\pgfsetlinewidth{0.000000pt}%
\definecolor{currentstroke}{rgb}{0.000000,0.000000,0.000000}%
\pgfsetstrokecolor{currentstroke}%
\pgfsetdash{}{0pt}%
\pgfpathmoveto{\pgfqpoint{1.404052in}{1.724673in}}%
\pgfpathlineto{\pgfqpoint{1.407345in}{1.723262in}}%
\pgfpathlineto{\pgfqpoint{1.410636in}{1.721737in}}%
\pgfpathlineto{\pgfqpoint{1.413926in}{1.720097in}}%
\pgfpathlineto{\pgfqpoint{1.417213in}{1.718343in}}%
\pgfpathlineto{\pgfqpoint{1.417924in}{1.717365in}}%
\pgfpathlineto{\pgfqpoint{1.418568in}{1.716376in}}%
\pgfpathlineto{\pgfqpoint{1.419145in}{1.715378in}}%
\pgfpathlineto{\pgfqpoint{1.419655in}{1.714372in}}%
\pgfpathlineto{\pgfqpoint{1.416243in}{1.716325in}}%
\pgfpathlineto{\pgfqpoint{1.412830in}{1.718164in}}%
\pgfpathlineto{\pgfqpoint{1.409415in}{1.719888in}}%
\pgfpathlineto{\pgfqpoint{1.405999in}{1.721498in}}%
\pgfpathlineto{\pgfqpoint{1.405592in}{1.722302in}}%
\pgfpathlineto{\pgfqpoint{1.405132in}{1.723100in}}%
\pgfpathlineto{\pgfqpoint{1.404618in}{1.723891in}}%
\pgfpathlineto{\pgfqpoint{1.404052in}{1.724673in}}%
\pgfpathclose%
\pgfusepath{fill}%
\end{pgfscope}%
\begin{pgfscope}%
\pgfpathrectangle{\pgfqpoint{0.329460in}{0.284240in}}{\pgfqpoint{1.989680in}{1.989680in}}%
\pgfusepath{clip}%
\pgfsetbuttcap%
\pgfsetroundjoin%
\definecolor{currentfill}{rgb}{0.636902,0.856542,0.216620}%
\pgfsetfillcolor{currentfill}%
\pgfsetlinewidth{0.000000pt}%
\definecolor{currentstroke}{rgb}{0.000000,0.000000,0.000000}%
\pgfsetstrokecolor{currentstroke}%
\pgfsetdash{}{0pt}%
\pgfpathmoveto{\pgfqpoint{1.218178in}{1.617921in}}%
\pgfpathlineto{\pgfqpoint{1.214873in}{1.612899in}}%
\pgfpathlineto{\pgfqpoint{1.211570in}{1.607783in}}%
\pgfpathlineto{\pgfqpoint{1.208268in}{1.602577in}}%
\pgfpathlineto{\pgfqpoint{1.204968in}{1.597280in}}%
\pgfpathlineto{\pgfqpoint{1.203557in}{1.599490in}}%
\pgfpathlineto{\pgfqpoint{1.202294in}{1.601720in}}%
\pgfpathlineto{\pgfqpoint{1.201181in}{1.603967in}}%
\pgfpathlineto{\pgfqpoint{1.200219in}{1.606228in}}%
\pgfpathlineto{\pgfqpoint{1.203621in}{1.611316in}}%
\pgfpathlineto{\pgfqpoint{1.207026in}{1.616313in}}%
\pgfpathlineto{\pgfqpoint{1.210432in}{1.621220in}}%
\pgfpathlineto{\pgfqpoint{1.213840in}{1.626034in}}%
\pgfpathlineto{\pgfqpoint{1.214720in}{1.623983in}}%
\pgfpathlineto{\pgfqpoint{1.215737in}{1.621946in}}%
\pgfpathlineto{\pgfqpoint{1.216890in}{1.619925in}}%
\pgfpathlineto{\pgfqpoint{1.218178in}{1.617921in}}%
\pgfpathclose%
\pgfusepath{fill}%
\end{pgfscope}%
\begin{pgfscope}%
\pgfpathrectangle{\pgfqpoint{0.329460in}{0.284240in}}{\pgfqpoint{1.989680in}{1.989680in}}%
\pgfusepath{clip}%
\pgfsetbuttcap%
\pgfsetroundjoin%
\definecolor{currentfill}{rgb}{0.974417,0.903590,0.130215}%
\pgfsetfillcolor{currentfill}%
\pgfsetlinewidth{0.000000pt}%
\definecolor{currentstroke}{rgb}{0.000000,0.000000,0.000000}%
\pgfsetstrokecolor{currentstroke}%
\pgfsetdash{}{0pt}%
\pgfpathmoveto{\pgfqpoint{1.369834in}{1.737187in}}%
\pgfpathlineto{\pgfqpoint{1.372159in}{1.737389in}}%
\pgfpathlineto{\pgfqpoint{1.374483in}{1.737472in}}%
\pgfpathlineto{\pgfqpoint{1.376806in}{1.737437in}}%
\pgfpathlineto{\pgfqpoint{1.379127in}{1.737283in}}%
\pgfpathlineto{\pgfqpoint{1.380105in}{1.736866in}}%
\pgfpathlineto{\pgfqpoint{1.381054in}{1.736434in}}%
\pgfpathlineto{\pgfqpoint{1.381973in}{1.735988in}}%
\pgfpathlineto{\pgfqpoint{1.382863in}{1.735529in}}%
\pgfpathlineto{\pgfqpoint{1.380230in}{1.735828in}}%
\pgfpathlineto{\pgfqpoint{1.377597in}{1.736010in}}%
\pgfpathlineto{\pgfqpoint{1.374962in}{1.736073in}}%
\pgfpathlineto{\pgfqpoint{1.372325in}{1.736017in}}%
\pgfpathlineto{\pgfqpoint{1.371732in}{1.736323in}}%
\pgfpathlineto{\pgfqpoint{1.371119in}{1.736621in}}%
\pgfpathlineto{\pgfqpoint{1.370486in}{1.736909in}}%
\pgfpathlineto{\pgfqpoint{1.369834in}{1.737187in}}%
\pgfpathclose%
\pgfusepath{fill}%
\end{pgfscope}%
\begin{pgfscope}%
\pgfpathrectangle{\pgfqpoint{0.329460in}{0.284240in}}{\pgfqpoint{1.989680in}{1.989680in}}%
\pgfusepath{clip}%
\pgfsetbuttcap%
\pgfsetroundjoin%
\definecolor{currentfill}{rgb}{0.762373,0.876424,0.137064}%
\pgfsetfillcolor{currentfill}%
\pgfsetlinewidth{0.000000pt}%
\definecolor{currentstroke}{rgb}{0.000000,0.000000,0.000000}%
\pgfsetstrokecolor{currentstroke}%
\pgfsetdash{}{0pt}%
\pgfpathmoveto{\pgfqpoint{1.241169in}{1.661072in}}%
\pgfpathlineto{\pgfqpoint{1.237747in}{1.657041in}}%
\pgfpathlineto{\pgfqpoint{1.234327in}{1.652909in}}%
\pgfpathlineto{\pgfqpoint{1.230908in}{1.648675in}}%
\pgfpathlineto{\pgfqpoint{1.227491in}{1.644342in}}%
\pgfpathlineto{\pgfqpoint{1.226818in}{1.646193in}}%
\pgfpathlineto{\pgfqpoint{1.226270in}{1.648053in}}%
\pgfpathlineto{\pgfqpoint{1.225846in}{1.649919in}}%
\pgfpathlineto{\pgfqpoint{1.225549in}{1.651790in}}%
\pgfpathlineto{\pgfqpoint{1.229015in}{1.655913in}}%
\pgfpathlineto{\pgfqpoint{1.232484in}{1.659936in}}%
\pgfpathlineto{\pgfqpoint{1.235954in}{1.663858in}}%
\pgfpathlineto{\pgfqpoint{1.239426in}{1.667679in}}%
\pgfpathlineto{\pgfqpoint{1.239695in}{1.666019in}}%
\pgfpathlineto{\pgfqpoint{1.240076in}{1.664363in}}%
\pgfpathlineto{\pgfqpoint{1.240567in}{1.662714in}}%
\pgfpathlineto{\pgfqpoint{1.241169in}{1.661072in}}%
\pgfpathclose%
\pgfusepath{fill}%
\end{pgfscope}%
\begin{pgfscope}%
\pgfpathrectangle{\pgfqpoint{0.329460in}{0.284240in}}{\pgfqpoint{1.989680in}{1.989680in}}%
\pgfusepath{clip}%
\pgfsetbuttcap%
\pgfsetroundjoin%
\definecolor{currentfill}{rgb}{0.855810,0.888601,0.097452}%
\pgfsetfillcolor{currentfill}%
\pgfsetlinewidth{0.000000pt}%
\definecolor{currentstroke}{rgb}{0.000000,0.000000,0.000000}%
\pgfsetstrokecolor{currentstroke}%
\pgfsetdash{}{0pt}%
\pgfpathmoveto{\pgfqpoint{1.434920in}{1.700533in}}%
\pgfpathlineto{\pgfqpoint{1.438393in}{1.697817in}}%
\pgfpathlineto{\pgfqpoint{1.441865in}{1.694993in}}%
\pgfpathlineto{\pgfqpoint{1.445335in}{1.692060in}}%
\pgfpathlineto{\pgfqpoint{1.448804in}{1.689019in}}%
\pgfpathlineto{\pgfqpoint{1.449042in}{1.687571in}}%
\pgfpathlineto{\pgfqpoint{1.449183in}{1.686120in}}%
\pgfpathlineto{\pgfqpoint{1.449227in}{1.684667in}}%
\pgfpathlineto{\pgfqpoint{1.449172in}{1.683214in}}%
\pgfpathlineto{\pgfqpoint{1.445688in}{1.686464in}}%
\pgfpathlineto{\pgfqpoint{1.442201in}{1.689606in}}%
\pgfpathlineto{\pgfqpoint{1.438713in}{1.692640in}}%
\pgfpathlineto{\pgfqpoint{1.435224in}{1.695564in}}%
\pgfpathlineto{\pgfqpoint{1.435274in}{1.696807in}}%
\pgfpathlineto{\pgfqpoint{1.435239in}{1.698051in}}%
\pgfpathlineto{\pgfqpoint{1.435121in}{1.699293in}}%
\pgfpathlineto{\pgfqpoint{1.434920in}{1.700533in}}%
\pgfpathclose%
\pgfusepath{fill}%
\end{pgfscope}%
\begin{pgfscope}%
\pgfpathrectangle{\pgfqpoint{0.329460in}{0.284240in}}{\pgfqpoint{1.989680in}{1.989680in}}%
\pgfusepath{clip}%
\pgfsetbuttcap%
\pgfsetroundjoin%
\definecolor{currentfill}{rgb}{0.896320,0.893616,0.096335}%
\pgfsetfillcolor{currentfill}%
\pgfsetlinewidth{0.000000pt}%
\definecolor{currentstroke}{rgb}{0.000000,0.000000,0.000000}%
\pgfsetstrokecolor{currentstroke}%
\pgfsetdash{}{0pt}%
\pgfpathmoveto{\pgfqpoint{1.281216in}{1.709369in}}%
\pgfpathlineto{\pgfqpoint{1.277728in}{1.707052in}}%
\pgfpathlineto{\pgfqpoint{1.274241in}{1.704623in}}%
\pgfpathlineto{\pgfqpoint{1.270756in}{1.702083in}}%
\pgfpathlineto{\pgfqpoint{1.267272in}{1.699431in}}%
\pgfpathlineto{\pgfqpoint{1.267483in}{1.700670in}}%
\pgfpathlineto{\pgfqpoint{1.267777in}{1.701906in}}%
\pgfpathlineto{\pgfqpoint{1.268154in}{1.703136in}}%
\pgfpathlineto{\pgfqpoint{1.268614in}{1.704359in}}%
\pgfpathlineto{\pgfqpoint{1.272039in}{1.706804in}}%
\pgfpathlineto{\pgfqpoint{1.275466in}{1.709139in}}%
\pgfpathlineto{\pgfqpoint{1.278894in}{1.711361in}}%
\pgfpathlineto{\pgfqpoint{1.282325in}{1.713472in}}%
\pgfpathlineto{\pgfqpoint{1.281944in}{1.712453in}}%
\pgfpathlineto{\pgfqpoint{1.281632in}{1.711429in}}%
\pgfpathlineto{\pgfqpoint{1.281389in}{1.710401in}}%
\pgfpathlineto{\pgfqpoint{1.281216in}{1.709369in}}%
\pgfpathclose%
\pgfusepath{fill}%
\end{pgfscope}%
\begin{pgfscope}%
\pgfpathrectangle{\pgfqpoint{0.329460in}{0.284240in}}{\pgfqpoint{1.989680in}{1.989680in}}%
\pgfusepath{clip}%
\pgfsetbuttcap%
\pgfsetroundjoin%
\definecolor{currentfill}{rgb}{0.935904,0.898570,0.108131}%
\pgfsetfillcolor{currentfill}%
\pgfsetlinewidth{0.000000pt}%
\definecolor{currentstroke}{rgb}{0.000000,0.000000,0.000000}%
\pgfsetstrokecolor{currentstroke}%
\pgfsetdash{}{0pt}%
\pgfpathmoveto{\pgfqpoint{1.296061in}{1.720778in}}%
\pgfpathlineto{\pgfqpoint{1.292625in}{1.719123in}}%
\pgfpathlineto{\pgfqpoint{1.289190in}{1.717353in}}%
\pgfpathlineto{\pgfqpoint{1.285757in}{1.715469in}}%
\pgfpathlineto{\pgfqpoint{1.282325in}{1.713472in}}%
\pgfpathlineto{\pgfqpoint{1.282774in}{1.714484in}}%
\pgfpathlineto{\pgfqpoint{1.283291in}{1.715489in}}%
\pgfpathlineto{\pgfqpoint{1.283876in}{1.716486in}}%
\pgfpathlineto{\pgfqpoint{1.284527in}{1.717474in}}%
\pgfpathlineto{\pgfqpoint{1.287847in}{1.719271in}}%
\pgfpathlineto{\pgfqpoint{1.291169in}{1.720954in}}%
\pgfpathlineto{\pgfqpoint{1.294492in}{1.722524in}}%
\pgfpathlineto{\pgfqpoint{1.297817in}{1.723978in}}%
\pgfpathlineto{\pgfqpoint{1.297297in}{1.723188in}}%
\pgfpathlineto{\pgfqpoint{1.296831in}{1.722391in}}%
\pgfpathlineto{\pgfqpoint{1.296419in}{1.721588in}}%
\pgfpathlineto{\pgfqpoint{1.296061in}{1.720778in}}%
\pgfpathclose%
\pgfusepath{fill}%
\end{pgfscope}%
\begin{pgfscope}%
\pgfpathrectangle{\pgfqpoint{0.329460in}{0.284240in}}{\pgfqpoint{1.989680in}{1.989680in}}%
\pgfusepath{clip}%
\pgfsetbuttcap%
\pgfsetroundjoin%
\definecolor{currentfill}{rgb}{0.344074,0.780029,0.397381}%
\pgfsetfillcolor{currentfill}%
\pgfsetlinewidth{0.000000pt}%
\definecolor{currentstroke}{rgb}{0.000000,0.000000,0.000000}%
\pgfsetstrokecolor{currentstroke}%
\pgfsetdash{}{0pt}%
\pgfpathmoveto{\pgfqpoint{1.530148in}{1.518594in}}%
\pgfpathlineto{\pgfqpoint{1.533309in}{1.512104in}}%
\pgfpathlineto{\pgfqpoint{1.536467in}{1.505546in}}%
\pgfpathlineto{\pgfqpoint{1.539624in}{1.498922in}}%
\pgfpathlineto{\pgfqpoint{1.542779in}{1.492233in}}%
\pgfpathlineto{\pgfqpoint{1.540073in}{1.489319in}}%
\pgfpathlineto{\pgfqpoint{1.537175in}{1.486446in}}%
\pgfpathlineto{\pgfqpoint{1.534087in}{1.483618in}}%
\pgfpathlineto{\pgfqpoint{1.530812in}{1.480838in}}%
\pgfpathlineto{\pgfqpoint{1.527849in}{1.487727in}}%
\pgfpathlineto{\pgfqpoint{1.524884in}{1.494551in}}%
\pgfpathlineto{\pgfqpoint{1.521917in}{1.501309in}}%
\pgfpathlineto{\pgfqpoint{1.518949in}{1.507998in}}%
\pgfpathlineto{\pgfqpoint{1.522013in}{1.510583in}}%
\pgfpathlineto{\pgfqpoint{1.524902in}{1.513213in}}%
\pgfpathlineto{\pgfqpoint{1.527615in}{1.515884in}}%
\pgfpathlineto{\pgfqpoint{1.530148in}{1.518594in}}%
\pgfpathclose%
\pgfusepath{fill}%
\end{pgfscope}%
\begin{pgfscope}%
\pgfpathrectangle{\pgfqpoint{0.329460in}{0.284240in}}{\pgfqpoint{1.989680in}{1.989680in}}%
\pgfusepath{clip}%
\pgfsetbuttcap%
\pgfsetroundjoin%
\definecolor{currentfill}{rgb}{0.974417,0.903590,0.130215}%
\pgfsetfillcolor{currentfill}%
\pgfsetlinewidth{0.000000pt}%
\definecolor{currentstroke}{rgb}{0.000000,0.000000,0.000000}%
\pgfsetstrokecolor{currentstroke}%
\pgfsetdash{}{0pt}%
\pgfpathmoveto{\pgfqpoint{1.329541in}{1.735738in}}%
\pgfpathlineto{\pgfqpoint{1.326840in}{1.735758in}}%
\pgfpathlineto{\pgfqpoint{1.324142in}{1.735660in}}%
\pgfpathlineto{\pgfqpoint{1.321444in}{1.735444in}}%
\pgfpathlineto{\pgfqpoint{1.318749in}{1.735111in}}%
\pgfpathlineto{\pgfqpoint{1.319610in}{1.735581in}}%
\pgfpathlineto{\pgfqpoint{1.320502in}{1.736039in}}%
\pgfpathlineto{\pgfqpoint{1.321425in}{1.736483in}}%
\pgfpathlineto{\pgfqpoint{1.322378in}{1.736913in}}%
\pgfpathlineto{\pgfqpoint{1.324771in}{1.737097in}}%
\pgfpathlineto{\pgfqpoint{1.327166in}{1.737163in}}%
\pgfpathlineto{\pgfqpoint{1.329563in}{1.737111in}}%
\pgfpathlineto{\pgfqpoint{1.331960in}{1.736940in}}%
\pgfpathlineto{\pgfqpoint{1.331325in}{1.736653in}}%
\pgfpathlineto{\pgfqpoint{1.330710in}{1.736357in}}%
\pgfpathlineto{\pgfqpoint{1.330115in}{1.736051in}}%
\pgfpathlineto{\pgfqpoint{1.329541in}{1.735738in}}%
\pgfpathclose%
\pgfusepath{fill}%
\end{pgfscope}%
\begin{pgfscope}%
\pgfpathrectangle{\pgfqpoint{0.329460in}{0.284240in}}{\pgfqpoint{1.989680in}{1.989680in}}%
\pgfusepath{clip}%
\pgfsetbuttcap%
\pgfsetroundjoin%
\definecolor{currentfill}{rgb}{0.276194,0.190074,0.493001}%
\pgfsetfillcolor{currentfill}%
\pgfsetlinewidth{0.000000pt}%
\definecolor{currentstroke}{rgb}{0.000000,0.000000,0.000000}%
\pgfsetstrokecolor{currentstroke}%
\pgfsetdash{}{0pt}%
\pgfpathmoveto{\pgfqpoint{0.868704in}{0.901264in}}%
\pgfpathlineto{\pgfqpoint{0.865743in}{0.907264in}}%
\pgfpathlineto{\pgfqpoint{0.862769in}{0.913624in}}%
\pgfpathlineto{\pgfqpoint{0.859783in}{0.920349in}}%
\pgfpathlineto{\pgfqpoint{0.856783in}{0.927445in}}%
\pgfpathlineto{\pgfqpoint{0.844508in}{0.935824in}}%
\pgfpathlineto{\pgfqpoint{0.832785in}{0.944393in}}%
\pgfpathlineto{\pgfqpoint{0.821625in}{0.953142in}}%
\pgfpathlineto{\pgfqpoint{0.811038in}{0.962060in}}%
\pgfpathlineto{\pgfqpoint{0.814299in}{0.954792in}}%
\pgfpathlineto{\pgfqpoint{0.817545in}{0.947894in}}%
\pgfpathlineto{\pgfqpoint{0.820778in}{0.941359in}}%
\pgfpathlineto{\pgfqpoint{0.823996in}{0.935183in}}%
\pgfpathlineto{\pgfqpoint{0.834346in}{0.926443in}}%
\pgfpathlineto{\pgfqpoint{0.845254in}{0.917870in}}%
\pgfpathlineto{\pgfqpoint{0.856710in}{0.909474in}}%
\pgfpathlineto{\pgfqpoint{0.868704in}{0.901264in}}%
\pgfpathclose%
\pgfusepath{fill}%
\end{pgfscope}%
\begin{pgfscope}%
\pgfpathrectangle{\pgfqpoint{0.329460in}{0.284240in}}{\pgfqpoint{1.989680in}{1.989680in}}%
\pgfusepath{clip}%
\pgfsetbuttcap%
\pgfsetroundjoin%
\definecolor{currentfill}{rgb}{0.122606,0.585371,0.546557}%
\pgfsetfillcolor{currentfill}%
\pgfsetlinewidth{0.000000pt}%
\definecolor{currentstroke}{rgb}{0.000000,0.000000,0.000000}%
\pgfsetstrokecolor{currentstroke}%
\pgfsetdash{}{0pt}%
\pgfpathmoveto{\pgfqpoint{1.161245in}{1.302271in}}%
\pgfpathlineto{\pgfqpoint{1.158902in}{1.294077in}}%
\pgfpathlineto{\pgfqpoint{1.156560in}{1.285864in}}%
\pgfpathlineto{\pgfqpoint{1.154219in}{1.277635in}}%
\pgfpathlineto{\pgfqpoint{1.151879in}{1.269391in}}%
\pgfpathlineto{\pgfqpoint{1.145245in}{1.272608in}}%
\pgfpathlineto{\pgfqpoint{1.138827in}{1.275927in}}%
\pgfpathlineto{\pgfqpoint{1.132631in}{1.279344in}}%
\pgfpathlineto{\pgfqpoint{1.126663in}{1.282856in}}%
\pgfpathlineto{\pgfqpoint{1.129293in}{1.290925in}}%
\pgfpathlineto{\pgfqpoint{1.131925in}{1.298979in}}%
\pgfpathlineto{\pgfqpoint{1.134558in}{1.307018in}}%
\pgfpathlineto{\pgfqpoint{1.137192in}{1.315037in}}%
\pgfpathlineto{\pgfqpoint{1.142886in}{1.311708in}}%
\pgfpathlineto{\pgfqpoint{1.148796in}{1.308468in}}%
\pgfpathlineto{\pgfqpoint{1.154918in}{1.305321in}}%
\pgfpathlineto{\pgfqpoint{1.161245in}{1.302271in}}%
\pgfpathclose%
\pgfusepath{fill}%
\end{pgfscope}%
\begin{pgfscope}%
\pgfpathrectangle{\pgfqpoint{0.329460in}{0.284240in}}{\pgfqpoint{1.989680in}{1.989680in}}%
\pgfusepath{clip}%
\pgfsetbuttcap%
\pgfsetroundjoin%
\definecolor{currentfill}{rgb}{0.268510,0.009605,0.335427}%
\pgfsetfillcolor{currentfill}%
\pgfsetlinewidth{0.000000pt}%
\definecolor{currentstroke}{rgb}{0.000000,0.000000,0.000000}%
\pgfsetstrokecolor{currentstroke}%
\pgfsetdash{}{0pt}%
\pgfpathmoveto{\pgfqpoint{1.026408in}{0.796507in}}%
\pgfpathlineto{\pgfqpoint{1.024281in}{0.796164in}}%
\pgfpathlineto{\pgfqpoint{1.022148in}{0.796073in}}%
\pgfpathlineto{\pgfqpoint{1.020009in}{0.796240in}}%
\pgfpathlineto{\pgfqpoint{1.017864in}{0.796669in}}%
\pgfpathlineto{\pgfqpoint{1.003512in}{0.802484in}}%
\pgfpathlineto{\pgfqpoint{0.989545in}{0.808534in}}%
\pgfpathlineto{\pgfqpoint{0.975976in}{0.814814in}}%
\pgfpathlineto{\pgfqpoint{0.962820in}{0.821316in}}%
\pgfpathlineto{\pgfqpoint{0.965310in}{0.820730in}}%
\pgfpathlineto{\pgfqpoint{0.967793in}{0.820406in}}%
\pgfpathlineto{\pgfqpoint{0.970269in}{0.820340in}}%
\pgfpathlineto{\pgfqpoint{0.972738in}{0.820525in}}%
\pgfpathlineto{\pgfqpoint{0.985567in}{0.814189in}}%
\pgfpathlineto{\pgfqpoint{0.998798in}{0.808069in}}%
\pgfpathlineto{\pgfqpoint{1.012416in}{0.802173in}}%
\pgfpathlineto{\pgfqpoint{1.026408in}{0.796507in}}%
\pgfpathclose%
\pgfusepath{fill}%
\end{pgfscope}%
\begin{pgfscope}%
\pgfpathrectangle{\pgfqpoint{0.329460in}{0.284240in}}{\pgfqpoint{1.989680in}{1.989680in}}%
\pgfusepath{clip}%
\pgfsetbuttcap%
\pgfsetroundjoin%
\definecolor{currentfill}{rgb}{0.487026,0.823929,0.312321}%
\pgfsetfillcolor{currentfill}%
\pgfsetlinewidth{0.000000pt}%
\definecolor{currentstroke}{rgb}{0.000000,0.000000,0.000000}%
\pgfsetstrokecolor{currentstroke}%
\pgfsetdash{}{0pt}%
\pgfpathmoveto{\pgfqpoint{1.199512in}{1.565806in}}%
\pgfpathlineto{\pgfqpoint{1.196375in}{1.559877in}}%
\pgfpathlineto{\pgfqpoint{1.193241in}{1.553867in}}%
\pgfpathlineto{\pgfqpoint{1.190108in}{1.547777in}}%
\pgfpathlineto{\pgfqpoint{1.186977in}{1.541608in}}%
\pgfpathlineto{\pgfqpoint{1.184636in}{1.544119in}}%
\pgfpathlineto{\pgfqpoint{1.182465in}{1.546664in}}%
\pgfpathlineto{\pgfqpoint{1.180464in}{1.549239in}}%
\pgfpathlineto{\pgfqpoint{1.178636in}{1.551842in}}%
\pgfpathlineto{\pgfqpoint{1.181921in}{1.557806in}}%
\pgfpathlineto{\pgfqpoint{1.185207in}{1.563691in}}%
\pgfpathlineto{\pgfqpoint{1.188496in}{1.569497in}}%
\pgfpathlineto{\pgfqpoint{1.191787in}{1.575222in}}%
\pgfpathlineto{\pgfqpoint{1.193481in}{1.572827in}}%
\pgfpathlineto{\pgfqpoint{1.195334in}{1.570458in}}%
\pgfpathlineto{\pgfqpoint{1.197345in}{1.568116in}}%
\pgfpathlineto{\pgfqpoint{1.199512in}{1.565806in}}%
\pgfpathclose%
\pgfusepath{fill}%
\end{pgfscope}%
\begin{pgfscope}%
\pgfpathrectangle{\pgfqpoint{0.329460in}{0.284240in}}{\pgfqpoint{1.989680in}{1.989680in}}%
\pgfusepath{clip}%
\pgfsetbuttcap%
\pgfsetroundjoin%
\definecolor{currentfill}{rgb}{0.212395,0.359683,0.551710}%
\pgfsetfillcolor{currentfill}%
\pgfsetlinewidth{0.000000pt}%
\definecolor{currentstroke}{rgb}{0.000000,0.000000,0.000000}%
\pgfsetstrokecolor{currentstroke}%
\pgfsetdash{}{0pt}%
\pgfpathmoveto{\pgfqpoint{1.570137in}{1.093661in}}%
\pgfpathlineto{\pgfqpoint{1.572213in}{1.085593in}}%
\pgfpathlineto{\pgfqpoint{1.574288in}{1.077569in}}%
\pgfpathlineto{\pgfqpoint{1.576364in}{1.069594in}}%
\pgfpathlineto{\pgfqpoint{1.578439in}{1.061668in}}%
\pgfpathlineto{\pgfqpoint{1.568924in}{1.058016in}}%
\pgfpathlineto{\pgfqpoint{1.559175in}{1.054521in}}%
\pgfpathlineto{\pgfqpoint{1.549203in}{1.051185in}}%
\pgfpathlineto{\pgfqpoint{1.539019in}{1.048013in}}%
\pgfpathlineto{\pgfqpoint{1.537300in}{1.056080in}}%
\pgfpathlineto{\pgfqpoint{1.535581in}{1.064197in}}%
\pgfpathlineto{\pgfqpoint{1.533862in}{1.072362in}}%
\pgfpathlineto{\pgfqpoint{1.532143in}{1.080571in}}%
\pgfpathlineto{\pgfqpoint{1.541958in}{1.083611in}}%
\pgfpathlineto{\pgfqpoint{1.551569in}{1.086809in}}%
\pgfpathlineto{\pgfqpoint{1.560965in}{1.090160in}}%
\pgfpathlineto{\pgfqpoint{1.570137in}{1.093661in}}%
\pgfpathclose%
\pgfusepath{fill}%
\end{pgfscope}%
\begin{pgfscope}%
\pgfpathrectangle{\pgfqpoint{0.329460in}{0.284240in}}{\pgfqpoint{1.989680in}{1.989680in}}%
\pgfusepath{clip}%
\pgfsetbuttcap%
\pgfsetroundjoin%
\definecolor{currentfill}{rgb}{0.955300,0.901065,0.118128}%
\pgfsetfillcolor{currentfill}%
\pgfsetlinewidth{0.000000pt}%
\definecolor{currentstroke}{rgb}{0.000000,0.000000,0.000000}%
\pgfsetstrokecolor{currentstroke}%
\pgfsetdash{}{0pt}%
\pgfpathmoveto{\pgfqpoint{1.388779in}{1.731435in}}%
\pgfpathlineto{\pgfqpoint{1.391903in}{1.730678in}}%
\pgfpathlineto{\pgfqpoint{1.395026in}{1.729804in}}%
\pgfpathlineto{\pgfqpoint{1.398147in}{1.728814in}}%
\pgfpathlineto{\pgfqpoint{1.401266in}{1.727708in}}%
\pgfpathlineto{\pgfqpoint{1.402039in}{1.726965in}}%
\pgfpathlineto{\pgfqpoint{1.402761in}{1.726211in}}%
\pgfpathlineto{\pgfqpoint{1.403432in}{1.725447in}}%
\pgfpathlineto{\pgfqpoint{1.404052in}{1.724673in}}%
\pgfpathlineto{\pgfqpoint{1.400757in}{1.725969in}}%
\pgfpathlineto{\pgfqpoint{1.397461in}{1.727149in}}%
\pgfpathlineto{\pgfqpoint{1.394163in}{1.728212in}}%
\pgfpathlineto{\pgfqpoint{1.390864in}{1.729159in}}%
\pgfpathlineto{\pgfqpoint{1.390401in}{1.729739in}}%
\pgfpathlineto{\pgfqpoint{1.389898in}{1.730312in}}%
\pgfpathlineto{\pgfqpoint{1.389357in}{1.730878in}}%
\pgfpathlineto{\pgfqpoint{1.388779in}{1.731435in}}%
\pgfpathclose%
\pgfusepath{fill}%
\end{pgfscope}%
\begin{pgfscope}%
\pgfpathrectangle{\pgfqpoint{0.329460in}{0.284240in}}{\pgfqpoint{1.989680in}{1.989680in}}%
\pgfusepath{clip}%
\pgfsetbuttcap%
\pgfsetroundjoin%
\definecolor{currentfill}{rgb}{0.231674,0.318106,0.544834}%
\pgfsetfillcolor{currentfill}%
\pgfsetlinewidth{0.000000pt}%
\definecolor{currentstroke}{rgb}{0.000000,0.000000,0.000000}%
\pgfsetstrokecolor{currentstroke}%
\pgfsetdash{}{0pt}%
\pgfpathmoveto{\pgfqpoint{1.172579in}{1.045334in}}%
\pgfpathlineto{\pgfqpoint{1.170943in}{1.037293in}}%
\pgfpathlineto{\pgfqpoint{1.169308in}{1.029309in}}%
\pgfpathlineto{\pgfqpoint{1.167672in}{1.021383in}}%
\pgfpathlineto{\pgfqpoint{1.166035in}{1.013521in}}%
\pgfpathlineto{\pgfqpoint{1.155294in}{1.016670in}}%
\pgfpathlineto{\pgfqpoint{1.144765in}{1.019993in}}%
\pgfpathlineto{\pgfqpoint{1.134457in}{1.023486in}}%
\pgfpathlineto{\pgfqpoint{1.124383in}{1.027146in}}%
\pgfpathlineto{\pgfqpoint{1.126383in}{1.034872in}}%
\pgfpathlineto{\pgfqpoint{1.128383in}{1.042661in}}%
\pgfpathlineto{\pgfqpoint{1.130383in}{1.050509in}}%
\pgfpathlineto{\pgfqpoint{1.132383in}{1.058415in}}%
\pgfpathlineto{\pgfqpoint{1.142106in}{1.054901in}}%
\pgfpathlineto{\pgfqpoint{1.152053in}{1.051547in}}%
\pgfpathlineto{\pgfqpoint{1.162214in}{1.048357in}}%
\pgfpathlineto{\pgfqpoint{1.172579in}{1.045334in}}%
\pgfpathclose%
\pgfusepath{fill}%
\end{pgfscope}%
\begin{pgfscope}%
\pgfpathrectangle{\pgfqpoint{0.329460in}{0.284240in}}{\pgfqpoint{1.989680in}{1.989680in}}%
\pgfusepath{clip}%
\pgfsetbuttcap%
\pgfsetroundjoin%
\definecolor{currentfill}{rgb}{0.163625,0.471133,0.558148}%
\pgfsetfillcolor{currentfill}%
\pgfsetlinewidth{0.000000pt}%
\definecolor{currentstroke}{rgb}{0.000000,0.000000,0.000000}%
\pgfsetstrokecolor{currentstroke}%
\pgfsetdash{}{0pt}%
\pgfpathmoveto{\pgfqpoint{1.164390in}{1.190203in}}%
\pgfpathlineto{\pgfqpoint{1.162387in}{1.181787in}}%
\pgfpathlineto{\pgfqpoint{1.160384in}{1.173382in}}%
\pgfpathlineto{\pgfqpoint{1.158382in}{1.164991in}}%
\pgfpathlineto{\pgfqpoint{1.156381in}{1.156616in}}%
\pgfpathlineto{\pgfqpoint{1.147920in}{1.159834in}}%
\pgfpathlineto{\pgfqpoint{1.139675in}{1.163184in}}%
\pgfpathlineto{\pgfqpoint{1.131653in}{1.166664in}}%
\pgfpathlineto{\pgfqpoint{1.123864in}{1.170270in}}%
\pgfpathlineto{\pgfqpoint{1.126195in}{1.178487in}}%
\pgfpathlineto{\pgfqpoint{1.128526in}{1.186720in}}%
\pgfpathlineto{\pgfqpoint{1.130858in}{1.194968in}}%
\pgfpathlineto{\pgfqpoint{1.133190in}{1.203227in}}%
\pgfpathlineto{\pgfqpoint{1.140665in}{1.199787in}}%
\pgfpathlineto{\pgfqpoint{1.148361in}{1.196467in}}%
\pgfpathlineto{\pgfqpoint{1.156272in}{1.193271in}}%
\pgfpathlineto{\pgfqpoint{1.164390in}{1.190203in}}%
\pgfpathclose%
\pgfusepath{fill}%
\end{pgfscope}%
\begin{pgfscope}%
\pgfpathrectangle{\pgfqpoint{0.329460in}{0.284240in}}{\pgfqpoint{1.989680in}{1.989680in}}%
\pgfusepath{clip}%
\pgfsetbuttcap%
\pgfsetroundjoin%
\definecolor{currentfill}{rgb}{0.166383,0.690856,0.496502}%
\pgfsetfillcolor{currentfill}%
\pgfsetlinewidth{0.000000pt}%
\definecolor{currentstroke}{rgb}{0.000000,0.000000,0.000000}%
\pgfsetstrokecolor{currentstroke}%
\pgfsetdash{}{0pt}%
\pgfpathmoveto{\pgfqpoint{1.168904in}{1.408975in}}%
\pgfpathlineto{\pgfqpoint{1.166254in}{1.401358in}}%
\pgfpathlineto{\pgfqpoint{1.163605in}{1.393696in}}%
\pgfpathlineto{\pgfqpoint{1.160958in}{1.385990in}}%
\pgfpathlineto{\pgfqpoint{1.158312in}{1.378242in}}%
\pgfpathlineto{\pgfqpoint{1.153370in}{1.381290in}}%
\pgfpathlineto{\pgfqpoint{1.148632in}{1.384411in}}%
\pgfpathlineto{\pgfqpoint{1.144104in}{1.387603in}}%
\pgfpathlineto{\pgfqpoint{1.139789in}{1.390863in}}%
\pgfpathlineto{\pgfqpoint{1.142683in}{1.398422in}}%
\pgfpathlineto{\pgfqpoint{1.145579in}{1.405940in}}%
\pgfpathlineto{\pgfqpoint{1.148477in}{1.413414in}}%
\pgfpathlineto{\pgfqpoint{1.151376in}{1.420844in}}%
\pgfpathlineto{\pgfqpoint{1.155460in}{1.417779in}}%
\pgfpathlineto{\pgfqpoint{1.159746in}{1.414777in}}%
\pgfpathlineto{\pgfqpoint{1.164228in}{1.411841in}}%
\pgfpathlineto{\pgfqpoint{1.168904in}{1.408975in}}%
\pgfpathclose%
\pgfusepath{fill}%
\end{pgfscope}%
\begin{pgfscope}%
\pgfpathrectangle{\pgfqpoint{0.329460in}{0.284240in}}{\pgfqpoint{1.989680in}{1.989680in}}%
\pgfusepath{clip}%
\pgfsetbuttcap%
\pgfsetroundjoin%
\definecolor{currentfill}{rgb}{0.955300,0.901065,0.118128}%
\pgfsetfillcolor{currentfill}%
\pgfsetlinewidth{0.000000pt}%
\definecolor{currentstroke}{rgb}{0.000000,0.000000,0.000000}%
\pgfsetstrokecolor{currentstroke}%
\pgfsetdash{}{0pt}%
\pgfpathmoveto{\pgfqpoint{1.311132in}{1.728638in}}%
\pgfpathlineto{\pgfqpoint{1.307801in}{1.727648in}}%
\pgfpathlineto{\pgfqpoint{1.304472in}{1.726541in}}%
\pgfpathlineto{\pgfqpoint{1.301144in}{1.725317in}}%
\pgfpathlineto{\pgfqpoint{1.297817in}{1.723978in}}%
\pgfpathlineto{\pgfqpoint{1.298390in}{1.724760in}}%
\pgfpathlineto{\pgfqpoint{1.299015in}{1.725533in}}%
\pgfpathlineto{\pgfqpoint{1.299692in}{1.726296in}}%
\pgfpathlineto{\pgfqpoint{1.300420in}{1.727048in}}%
\pgfpathlineto{\pgfqpoint{1.303582in}{1.728195in}}%
\pgfpathlineto{\pgfqpoint{1.306747in}{1.729227in}}%
\pgfpathlineto{\pgfqpoint{1.309913in}{1.730142in}}%
\pgfpathlineto{\pgfqpoint{1.313080in}{1.730940in}}%
\pgfpathlineto{\pgfqpoint{1.312535in}{1.730376in}}%
\pgfpathlineto{\pgfqpoint{1.312028in}{1.729803in}}%
\pgfpathlineto{\pgfqpoint{1.311560in}{1.729224in}}%
\pgfpathlineto{\pgfqpoint{1.311132in}{1.728638in}}%
\pgfpathclose%
\pgfusepath{fill}%
\end{pgfscope}%
\begin{pgfscope}%
\pgfpathrectangle{\pgfqpoint{0.329460in}{0.284240in}}{\pgfqpoint{1.989680in}{1.989680in}}%
\pgfusepath{clip}%
\pgfsetbuttcap%
\pgfsetroundjoin%
\definecolor{currentfill}{rgb}{0.855810,0.888601,0.097452}%
\pgfsetfillcolor{currentfill}%
\pgfsetlinewidth{0.000000pt}%
\definecolor{currentstroke}{rgb}{0.000000,0.000000,0.000000}%
\pgfsetstrokecolor{currentstroke}%
\pgfsetdash{}{0pt}%
\pgfpathmoveto{\pgfqpoint{1.267266in}{1.694459in}}%
\pgfpathlineto{\pgfqpoint{1.263780in}{1.691488in}}%
\pgfpathlineto{\pgfqpoint{1.260296in}{1.688408in}}%
\pgfpathlineto{\pgfqpoint{1.256814in}{1.685220in}}%
\pgfpathlineto{\pgfqpoint{1.253333in}{1.681923in}}%
\pgfpathlineto{\pgfqpoint{1.253192in}{1.683375in}}%
\pgfpathlineto{\pgfqpoint{1.253148in}{1.684829in}}%
\pgfpathlineto{\pgfqpoint{1.253203in}{1.686282in}}%
\pgfpathlineto{\pgfqpoint{1.253355in}{1.687732in}}%
\pgfpathlineto{\pgfqpoint{1.256831in}{1.690819in}}%
\pgfpathlineto{\pgfqpoint{1.260310in}{1.693799in}}%
\pgfpathlineto{\pgfqpoint{1.263790in}{1.696670in}}%
\pgfpathlineto{\pgfqpoint{1.267272in}{1.699431in}}%
\pgfpathlineto{\pgfqpoint{1.267145in}{1.698189in}}%
\pgfpathlineto{\pgfqpoint{1.267101in}{1.696946in}}%
\pgfpathlineto{\pgfqpoint{1.267141in}{1.695702in}}%
\pgfpathlineto{\pgfqpoint{1.267266in}{1.694459in}}%
\pgfpathclose%
\pgfusepath{fill}%
\end{pgfscope}%
\begin{pgfscope}%
\pgfpathrectangle{\pgfqpoint{0.329460in}{0.284240in}}{\pgfqpoint{1.989680in}{1.989680in}}%
\pgfusepath{clip}%
\pgfsetbuttcap%
\pgfsetroundjoin%
\definecolor{currentfill}{rgb}{0.974417,0.903590,0.130215}%
\pgfsetfillcolor{currentfill}%
\pgfsetlinewidth{0.000000pt}%
\definecolor{currentstroke}{rgb}{0.000000,0.000000,0.000000}%
\pgfsetstrokecolor{currentstroke}%
\pgfsetdash{}{0pt}%
\pgfpathmoveto{\pgfqpoint{1.372325in}{1.736017in}}%
\pgfpathlineto{\pgfqpoint{1.374962in}{1.736073in}}%
\pgfpathlineto{\pgfqpoint{1.377597in}{1.736010in}}%
\pgfpathlineto{\pgfqpoint{1.380230in}{1.735828in}}%
\pgfpathlineto{\pgfqpoint{1.382863in}{1.735529in}}%
\pgfpathlineto{\pgfqpoint{1.383720in}{1.735058in}}%
\pgfpathlineto{\pgfqpoint{1.384546in}{1.734573in}}%
\pgfpathlineto{\pgfqpoint{1.385338in}{1.734077in}}%
\pgfpathlineto{\pgfqpoint{1.386097in}{1.733570in}}%
\pgfpathlineto{\pgfqpoint{1.383195in}{1.734032in}}%
\pgfpathlineto{\pgfqpoint{1.380292in}{1.734376in}}%
\pgfpathlineto{\pgfqpoint{1.377387in}{1.734602in}}%
\pgfpathlineto{\pgfqpoint{1.374481in}{1.734709in}}%
\pgfpathlineto{\pgfqpoint{1.373975in}{1.735048in}}%
\pgfpathlineto{\pgfqpoint{1.373447in}{1.735379in}}%
\pgfpathlineto{\pgfqpoint{1.372897in}{1.735702in}}%
\pgfpathlineto{\pgfqpoint{1.372325in}{1.736017in}}%
\pgfpathclose%
\pgfusepath{fill}%
\end{pgfscope}%
\begin{pgfscope}%
\pgfpathrectangle{\pgfqpoint{0.329460in}{0.284240in}}{\pgfqpoint{1.989680in}{1.989680in}}%
\pgfusepath{clip}%
\pgfsetbuttcap%
\pgfsetroundjoin%
\definecolor{currentfill}{rgb}{0.974417,0.903590,0.130215}%
\pgfsetfillcolor{currentfill}%
\pgfsetlinewidth{0.000000pt}%
\definecolor{currentstroke}{rgb}{0.000000,0.000000,0.000000}%
\pgfsetstrokecolor{currentstroke}%
\pgfsetdash{}{0pt}%
\pgfpathmoveto{\pgfqpoint{1.327464in}{1.734402in}}%
\pgfpathlineto{\pgfqpoint{1.324504in}{1.734256in}}%
\pgfpathlineto{\pgfqpoint{1.321545in}{1.733992in}}%
\pgfpathlineto{\pgfqpoint{1.318588in}{1.733609in}}%
\pgfpathlineto{\pgfqpoint{1.315633in}{1.733109in}}%
\pgfpathlineto{\pgfqpoint{1.316361in}{1.733627in}}%
\pgfpathlineto{\pgfqpoint{1.317123in}{1.734133in}}%
\pgfpathlineto{\pgfqpoint{1.317919in}{1.734628in}}%
\pgfpathlineto{\pgfqpoint{1.318749in}{1.735111in}}%
\pgfpathlineto{\pgfqpoint{1.321444in}{1.735444in}}%
\pgfpathlineto{\pgfqpoint{1.324142in}{1.735660in}}%
\pgfpathlineto{\pgfqpoint{1.326840in}{1.735758in}}%
\pgfpathlineto{\pgfqpoint{1.329541in}{1.735738in}}%
\pgfpathlineto{\pgfqpoint{1.328988in}{1.735415in}}%
\pgfpathlineto{\pgfqpoint{1.328457in}{1.735085in}}%
\pgfpathlineto{\pgfqpoint{1.327949in}{1.734747in}}%
\pgfpathlineto{\pgfqpoint{1.327464in}{1.734402in}}%
\pgfpathclose%
\pgfusepath{fill}%
\end{pgfscope}%
\begin{pgfscope}%
\pgfpathrectangle{\pgfqpoint{0.329460in}{0.284240in}}{\pgfqpoint{1.989680in}{1.989680in}}%
\pgfusepath{clip}%
\pgfsetbuttcap%
\pgfsetroundjoin%
\definecolor{currentfill}{rgb}{0.120081,0.622161,0.534946}%
\pgfsetfillcolor{currentfill}%
\pgfsetlinewidth{0.000000pt}%
\definecolor{currentstroke}{rgb}{0.000000,0.000000,0.000000}%
\pgfsetstrokecolor{currentstroke}%
\pgfsetdash{}{0pt}%
\pgfpathmoveto{\pgfqpoint{1.559273in}{1.349750in}}%
\pgfpathlineto{\pgfqpoint{1.561971in}{1.341870in}}%
\pgfpathlineto{\pgfqpoint{1.564668in}{1.333961in}}%
\pgfpathlineto{\pgfqpoint{1.567363in}{1.326027in}}%
\pgfpathlineto{\pgfqpoint{1.570058in}{1.318070in}}%
\pgfpathlineto{\pgfqpoint{1.564561in}{1.314663in}}%
\pgfpathlineto{\pgfqpoint{1.558843in}{1.311343in}}%
\pgfpathlineto{\pgfqpoint{1.552909in}{1.308114in}}%
\pgfpathlineto{\pgfqpoint{1.546764in}{1.304977in}}%
\pgfpathlineto{\pgfqpoint{1.544351in}{1.313113in}}%
\pgfpathlineto{\pgfqpoint{1.541938in}{1.321225in}}%
\pgfpathlineto{\pgfqpoint{1.539523in}{1.329311in}}%
\pgfpathlineto{\pgfqpoint{1.537108in}{1.337368in}}%
\pgfpathlineto{\pgfqpoint{1.542954in}{1.340334in}}%
\pgfpathlineto{\pgfqpoint{1.548600in}{1.343388in}}%
\pgfpathlineto{\pgfqpoint{1.554042in}{1.346528in}}%
\pgfpathlineto{\pgfqpoint{1.559273in}{1.349750in}}%
\pgfpathclose%
\pgfusepath{fill}%
\end{pgfscope}%
\begin{pgfscope}%
\pgfpathrectangle{\pgfqpoint{0.329460in}{0.284240in}}{\pgfqpoint{1.989680in}{1.989680in}}%
\pgfusepath{clip}%
\pgfsetbuttcap%
\pgfsetroundjoin%
\definecolor{currentfill}{rgb}{0.699415,0.867117,0.175971}%
\pgfsetfillcolor{currentfill}%
\pgfsetlinewidth{0.000000pt}%
\definecolor{currentstroke}{rgb}{0.000000,0.000000,0.000000}%
\pgfsetstrokecolor{currentstroke}%
\pgfsetdash{}{0pt}%
\pgfpathmoveto{\pgfqpoint{1.475489in}{1.645987in}}%
\pgfpathlineto{\pgfqpoint{1.478920in}{1.641601in}}%
\pgfpathlineto{\pgfqpoint{1.482349in}{1.637119in}}%
\pgfpathlineto{\pgfqpoint{1.485777in}{1.632540in}}%
\pgfpathlineto{\pgfqpoint{1.489202in}{1.627866in}}%
\pgfpathlineto{\pgfqpoint{1.488445in}{1.625805in}}%
\pgfpathlineto{\pgfqpoint{1.487549in}{1.623756in}}%
\pgfpathlineto{\pgfqpoint{1.486517in}{1.621721in}}%
\pgfpathlineto{\pgfqpoint{1.485349in}{1.619701in}}%
\pgfpathlineto{\pgfqpoint{1.482015in}{1.624584in}}%
\pgfpathlineto{\pgfqpoint{1.478679in}{1.629372in}}%
\pgfpathlineto{\pgfqpoint{1.475341in}{1.634063in}}%
\pgfpathlineto{\pgfqpoint{1.472002in}{1.638657in}}%
\pgfpathlineto{\pgfqpoint{1.473058in}{1.640470in}}%
\pgfpathlineto{\pgfqpoint{1.473992in}{1.642297in}}%
\pgfpathlineto{\pgfqpoint{1.474802in}{1.644136in}}%
\pgfpathlineto{\pgfqpoint{1.475489in}{1.645987in}}%
\pgfpathclose%
\pgfusepath{fill}%
\end{pgfscope}%
\begin{pgfscope}%
\pgfpathrectangle{\pgfqpoint{0.329460in}{0.284240in}}{\pgfqpoint{1.989680in}{1.989680in}}%
\pgfusepath{clip}%
\pgfsetbuttcap%
\pgfsetroundjoin%
\definecolor{currentfill}{rgb}{0.147607,0.511733,0.557049}%
\pgfsetfillcolor{currentfill}%
\pgfsetlinewidth{0.000000pt}%
\definecolor{currentstroke}{rgb}{0.000000,0.000000,0.000000}%
\pgfsetstrokecolor{currentstroke}%
\pgfsetdash{}{0pt}%
\pgfpathmoveto{\pgfqpoint{1.566027in}{1.239327in}}%
\pgfpathlineto{\pgfqpoint{1.568431in}{1.231087in}}%
\pgfpathlineto{\pgfqpoint{1.570833in}{1.222847in}}%
\pgfpathlineto{\pgfqpoint{1.573235in}{1.214612in}}%
\pgfpathlineto{\pgfqpoint{1.575636in}{1.206382in}}%
\pgfpathlineto{\pgfqpoint{1.568366in}{1.202839in}}%
\pgfpathlineto{\pgfqpoint{1.560866in}{1.199412in}}%
\pgfpathlineto{\pgfqpoint{1.553145in}{1.196106in}}%
\pgfpathlineto{\pgfqpoint{1.545211in}{1.192924in}}%
\pgfpathlineto{\pgfqpoint{1.543131in}{1.201315in}}%
\pgfpathlineto{\pgfqpoint{1.541050in}{1.209712in}}%
\pgfpathlineto{\pgfqpoint{1.538969in}{1.218113in}}%
\pgfpathlineto{\pgfqpoint{1.536887in}{1.226513in}}%
\pgfpathlineto{\pgfqpoint{1.544485in}{1.229543in}}%
\pgfpathlineto{\pgfqpoint{1.551879in}{1.232691in}}%
\pgfpathlineto{\pgfqpoint{1.559062in}{1.235953in}}%
\pgfpathlineto{\pgfqpoint{1.566027in}{1.239327in}}%
\pgfpathclose%
\pgfusepath{fill}%
\end{pgfscope}%
\begin{pgfscope}%
\pgfpathrectangle{\pgfqpoint{0.329460in}{0.284240in}}{\pgfqpoint{1.989680in}{1.989680in}}%
\pgfusepath{clip}%
\pgfsetbuttcap%
\pgfsetroundjoin%
\definecolor{currentfill}{rgb}{0.814576,0.883393,0.110347}%
\pgfsetfillcolor{currentfill}%
\pgfsetlinewidth{0.000000pt}%
\definecolor{currentstroke}{rgb}{0.000000,0.000000,0.000000}%
\pgfsetstrokecolor{currentstroke}%
\pgfsetdash{}{0pt}%
\pgfpathmoveto{\pgfqpoint{1.449172in}{1.683214in}}%
\pgfpathlineto{\pgfqpoint{1.452655in}{1.679857in}}%
\pgfpathlineto{\pgfqpoint{1.456137in}{1.676395in}}%
\pgfpathlineto{\pgfqpoint{1.459616in}{1.672828in}}%
\pgfpathlineto{\pgfqpoint{1.463094in}{1.669157in}}%
\pgfpathlineto{\pgfqpoint{1.462924in}{1.667494in}}%
\pgfpathlineto{\pgfqpoint{1.462643in}{1.665835in}}%
\pgfpathlineto{\pgfqpoint{1.462250in}{1.664180in}}%
\pgfpathlineto{\pgfqpoint{1.461747in}{1.662531in}}%
\pgfpathlineto{\pgfqpoint{1.458307in}{1.666412in}}%
\pgfpathlineto{\pgfqpoint{1.454866in}{1.670189in}}%
\pgfpathlineto{\pgfqpoint{1.451423in}{1.673861in}}%
\pgfpathlineto{\pgfqpoint{1.447979in}{1.677427in}}%
\pgfpathlineto{\pgfqpoint{1.448423in}{1.678867in}}%
\pgfpathlineto{\pgfqpoint{1.448771in}{1.680312in}}%
\pgfpathlineto{\pgfqpoint{1.449020in}{1.681762in}}%
\pgfpathlineto{\pgfqpoint{1.449172in}{1.683214in}}%
\pgfpathclose%
\pgfusepath{fill}%
\end{pgfscope}%
\begin{pgfscope}%
\pgfpathrectangle{\pgfqpoint{0.329460in}{0.284240in}}{\pgfqpoint{1.989680in}{1.989680in}}%
\pgfusepath{clip}%
\pgfsetbuttcap%
\pgfsetroundjoin%
\definecolor{currentfill}{rgb}{0.268510,0.009605,0.335427}%
\pgfsetfillcolor{currentfill}%
\pgfsetlinewidth{0.000000pt}%
\definecolor{currentstroke}{rgb}{0.000000,0.000000,0.000000}%
\pgfsetstrokecolor{currentstroke}%
\pgfsetdash{}{0pt}%
\pgfpathmoveto{\pgfqpoint{1.106399in}{0.800065in}}%
\pgfpathlineto{\pgfqpoint{1.104704in}{0.796909in}}%
\pgfpathlineto{\pgfqpoint{1.103006in}{0.793952in}}%
\pgfpathlineto{\pgfqpoint{1.101304in}{0.791198in}}%
\pgfpathlineto{\pgfqpoint{1.099598in}{0.788651in}}%
\pgfpathlineto{\pgfqpoint{1.085050in}{0.793077in}}%
\pgfpathlineto{\pgfqpoint{1.070796in}{0.797747in}}%
\pgfpathlineto{\pgfqpoint{1.056852in}{0.802653in}}%
\pgfpathlineto{\pgfqpoint{1.043232in}{0.807791in}}%
\pgfpathlineto{\pgfqpoint{1.045313in}{0.810201in}}%
\pgfpathlineto{\pgfqpoint{1.047391in}{0.812818in}}%
\pgfpathlineto{\pgfqpoint{1.049464in}{0.815638in}}%
\pgfpathlineto{\pgfqpoint{1.051533in}{0.818658in}}%
\pgfpathlineto{\pgfqpoint{1.064792in}{0.813666in}}%
\pgfpathlineto{\pgfqpoint{1.078365in}{0.808900in}}%
\pgfpathlineto{\pgfqpoint{1.092239in}{0.804365in}}%
\pgfpathlineto{\pgfqpoint{1.106399in}{0.800065in}}%
\pgfpathclose%
\pgfusepath{fill}%
\end{pgfscope}%
\begin{pgfscope}%
\pgfpathrectangle{\pgfqpoint{0.329460in}{0.284240in}}{\pgfqpoint{1.989680in}{1.989680in}}%
\pgfusepath{clip}%
\pgfsetbuttcap%
\pgfsetroundjoin%
\definecolor{currentfill}{rgb}{0.344074,0.780029,0.397381}%
\pgfsetfillcolor{currentfill}%
\pgfsetlinewidth{0.000000pt}%
\definecolor{currentstroke}{rgb}{0.000000,0.000000,0.000000}%
\pgfsetstrokecolor{currentstroke}%
\pgfsetdash{}{0pt}%
\pgfpathmoveto{\pgfqpoint{1.186293in}{1.505739in}}%
\pgfpathlineto{\pgfqpoint{1.183375in}{1.499007in}}%
\pgfpathlineto{\pgfqpoint{1.180458in}{1.492207in}}%
\pgfpathlineto{\pgfqpoint{1.177542in}{1.485340in}}%
\pgfpathlineto{\pgfqpoint{1.174628in}{1.478408in}}%
\pgfpathlineto{\pgfqpoint{1.171190in}{1.481144in}}%
\pgfpathlineto{\pgfqpoint{1.167936in}{1.483930in}}%
\pgfpathlineto{\pgfqpoint{1.164869in}{1.486764in}}%
\pgfpathlineto{\pgfqpoint{1.161992in}{1.489641in}}%
\pgfpathlineto{\pgfqpoint{1.165109in}{1.496375in}}%
\pgfpathlineto{\pgfqpoint{1.168227in}{1.503045in}}%
\pgfpathlineto{\pgfqpoint{1.171348in}{1.509648in}}%
\pgfpathlineto{\pgfqpoint{1.174470in}{1.516183in}}%
\pgfpathlineto{\pgfqpoint{1.177163in}{1.513508in}}%
\pgfpathlineto{\pgfqpoint{1.180033in}{1.510873in}}%
\pgfpathlineto{\pgfqpoint{1.183077in}{1.508283in}}%
\pgfpathlineto{\pgfqpoint{1.186293in}{1.505739in}}%
\pgfpathclose%
\pgfusepath{fill}%
\end{pgfscope}%
\begin{pgfscope}%
\pgfpathrectangle{\pgfqpoint{0.329460in}{0.284240in}}{\pgfqpoint{1.989680in}{1.989680in}}%
\pgfusepath{clip}%
\pgfsetbuttcap%
\pgfsetroundjoin%
\definecolor{currentfill}{rgb}{0.565498,0.842430,0.262877}%
\pgfsetfillcolor{currentfill}%
\pgfsetlinewidth{0.000000pt}%
\definecolor{currentstroke}{rgb}{0.000000,0.000000,0.000000}%
\pgfsetstrokecolor{currentstroke}%
\pgfsetdash{}{0pt}%
\pgfpathmoveto{\pgfqpoint{1.498669in}{1.599244in}}%
\pgfpathlineto{\pgfqpoint{1.501994in}{1.593905in}}%
\pgfpathlineto{\pgfqpoint{1.505318in}{1.588479in}}%
\pgfpathlineto{\pgfqpoint{1.508639in}{1.582967in}}%
\pgfpathlineto{\pgfqpoint{1.511959in}{1.577371in}}%
\pgfpathlineto{\pgfqpoint{1.510408in}{1.574955in}}%
\pgfpathlineto{\pgfqpoint{1.508696in}{1.572562in}}%
\pgfpathlineto{\pgfqpoint{1.506826in}{1.570196in}}%
\pgfpathlineto{\pgfqpoint{1.504797in}{1.567858in}}%
\pgfpathlineto{\pgfqpoint{1.501620in}{1.573660in}}%
\pgfpathlineto{\pgfqpoint{1.498442in}{1.579376in}}%
\pgfpathlineto{\pgfqpoint{1.495262in}{1.585007in}}%
\pgfpathlineto{\pgfqpoint{1.492080in}{1.590551in}}%
\pgfpathlineto{\pgfqpoint{1.493945in}{1.592687in}}%
\pgfpathlineto{\pgfqpoint{1.495666in}{1.594849in}}%
\pgfpathlineto{\pgfqpoint{1.497241in}{1.597036in}}%
\pgfpathlineto{\pgfqpoint{1.498669in}{1.599244in}}%
\pgfpathclose%
\pgfusepath{fill}%
\end{pgfscope}%
\begin{pgfscope}%
\pgfpathrectangle{\pgfqpoint{0.329460in}{0.284240in}}{\pgfqpoint{1.989680in}{1.989680in}}%
\pgfusepath{clip}%
\pgfsetbuttcap%
\pgfsetroundjoin%
\definecolor{currentfill}{rgb}{0.220124,0.725509,0.466226}%
\pgfsetfillcolor{currentfill}%
\pgfsetlinewidth{0.000000pt}%
\definecolor{currentstroke}{rgb}{0.000000,0.000000,0.000000}%
\pgfsetstrokecolor{currentstroke}%
\pgfsetdash{}{0pt}%
\pgfpathmoveto{\pgfqpoint{1.542648in}{1.452671in}}%
\pgfpathlineto{\pgfqpoint{1.545603in}{1.445486in}}%
\pgfpathlineto{\pgfqpoint{1.548556in}{1.438248in}}%
\pgfpathlineto{\pgfqpoint{1.551508in}{1.430958in}}%
\pgfpathlineto{\pgfqpoint{1.554457in}{1.423620in}}%
\pgfpathlineto{\pgfqpoint{1.550555in}{1.420500in}}%
\pgfpathlineto{\pgfqpoint{1.546449in}{1.417442in}}%
\pgfpathlineto{\pgfqpoint{1.542141in}{1.414447in}}%
\pgfpathlineto{\pgfqpoint{1.537637in}{1.411519in}}%
\pgfpathlineto{\pgfqpoint{1.534926in}{1.419048in}}%
\pgfpathlineto{\pgfqpoint{1.532213in}{1.426528in}}%
\pgfpathlineto{\pgfqpoint{1.529500in}{1.433955in}}%
\pgfpathlineto{\pgfqpoint{1.526784in}{1.441329in}}%
\pgfpathlineto{\pgfqpoint{1.531032in}{1.444073in}}%
\pgfpathlineto{\pgfqpoint{1.535094in}{1.446880in}}%
\pgfpathlineto{\pgfqpoint{1.538967in}{1.449747in}}%
\pgfpathlineto{\pgfqpoint{1.542648in}{1.452671in}}%
\pgfpathclose%
\pgfusepath{fill}%
\end{pgfscope}%
\begin{pgfscope}%
\pgfpathrectangle{\pgfqpoint{0.329460in}{0.284240in}}{\pgfqpoint{1.989680in}{1.989680in}}%
\pgfusepath{clip}%
\pgfsetbuttcap%
\pgfsetroundjoin%
\definecolor{currentfill}{rgb}{0.993248,0.906157,0.143936}%
\pgfsetfillcolor{currentfill}%
\pgfsetlinewidth{0.000000pt}%
\definecolor{currentstroke}{rgb}{0.000000,0.000000,0.000000}%
\pgfsetstrokecolor{currentstroke}%
\pgfsetdash{}{0pt}%
\pgfpathmoveto{\pgfqpoint{1.349446in}{1.736792in}}%
\pgfpathlineto{\pgfqpoint{1.349011in}{1.737872in}}%
\pgfpathlineto{\pgfqpoint{1.348576in}{1.738831in}}%
\pgfpathlineto{\pgfqpoint{1.348142in}{1.739670in}}%
\pgfpathlineto{\pgfqpoint{1.347708in}{1.740390in}}%
\pgfpathlineto{\pgfqpoint{1.348587in}{1.740434in}}%
\pgfpathlineto{\pgfqpoint{1.349469in}{1.740466in}}%
\pgfpathlineto{\pgfqpoint{1.350352in}{1.740485in}}%
\pgfpathlineto{\pgfqpoint{1.351237in}{1.740491in}}%
\pgfpathlineto{\pgfqpoint{1.351231in}{1.739758in}}%
\pgfpathlineto{\pgfqpoint{1.351224in}{1.738906in}}%
\pgfpathlineto{\pgfqpoint{1.351218in}{1.737935in}}%
\pgfpathlineto{\pgfqpoint{1.351212in}{1.736843in}}%
\pgfpathlineto{\pgfqpoint{1.350769in}{1.736840in}}%
\pgfpathlineto{\pgfqpoint{1.350327in}{1.736831in}}%
\pgfpathlineto{\pgfqpoint{1.349886in}{1.736815in}}%
\pgfpathlineto{\pgfqpoint{1.349446in}{1.736792in}}%
\pgfpathclose%
\pgfusepath{fill}%
\end{pgfscope}%
\begin{pgfscope}%
\pgfpathrectangle{\pgfqpoint{0.329460in}{0.284240in}}{\pgfqpoint{1.989680in}{1.989680in}}%
\pgfusepath{clip}%
\pgfsetbuttcap%
\pgfsetroundjoin%
\definecolor{currentfill}{rgb}{0.993248,0.906157,0.143936}%
\pgfsetfillcolor{currentfill}%
\pgfsetlinewidth{0.000000pt}%
\definecolor{currentstroke}{rgb}{0.000000,0.000000,0.000000}%
\pgfsetstrokecolor{currentstroke}%
\pgfsetdash{}{0pt}%
\pgfpathmoveto{\pgfqpoint{1.351212in}{1.736843in}}%
\pgfpathlineto{\pgfqpoint{1.351218in}{1.737935in}}%
\pgfpathlineto{\pgfqpoint{1.351224in}{1.738906in}}%
\pgfpathlineto{\pgfqpoint{1.351231in}{1.739758in}}%
\pgfpathlineto{\pgfqpoint{1.351237in}{1.740491in}}%
\pgfpathlineto{\pgfqpoint{1.352121in}{1.740483in}}%
\pgfpathlineto{\pgfqpoint{1.353004in}{1.740463in}}%
\pgfpathlineto{\pgfqpoint{1.353886in}{1.740430in}}%
\pgfpathlineto{\pgfqpoint{1.354764in}{1.740384in}}%
\pgfpathlineto{\pgfqpoint{1.354318in}{1.739665in}}%
\pgfpathlineto{\pgfqpoint{1.353872in}{1.738826in}}%
\pgfpathlineto{\pgfqpoint{1.353425in}{1.737868in}}%
\pgfpathlineto{\pgfqpoint{1.352978in}{1.736789in}}%
\pgfpathlineto{\pgfqpoint{1.352538in}{1.736812in}}%
\pgfpathlineto{\pgfqpoint{1.352097in}{1.736829in}}%
\pgfpathlineto{\pgfqpoint{1.351655in}{1.736839in}}%
\pgfpathlineto{\pgfqpoint{1.351212in}{1.736843in}}%
\pgfpathclose%
\pgfusepath{fill}%
\end{pgfscope}%
\begin{pgfscope}%
\pgfpathrectangle{\pgfqpoint{0.329460in}{0.284240in}}{\pgfqpoint{1.989680in}{1.989680in}}%
\pgfusepath{clip}%
\pgfsetbuttcap%
\pgfsetroundjoin%
\definecolor{currentfill}{rgb}{0.993248,0.906157,0.143936}%
\pgfsetfillcolor{currentfill}%
\pgfsetlinewidth{0.000000pt}%
\definecolor{currentstroke}{rgb}{0.000000,0.000000,0.000000}%
\pgfsetstrokecolor{currentstroke}%
\pgfsetdash{}{0pt}%
\pgfpathmoveto{\pgfqpoint{1.347707in}{1.736639in}}%
\pgfpathlineto{\pgfqpoint{1.346838in}{1.737680in}}%
\pgfpathlineto{\pgfqpoint{1.345970in}{1.738600in}}%
\pgfpathlineto{\pgfqpoint{1.345102in}{1.739401in}}%
\pgfpathlineto{\pgfqpoint{1.344235in}{1.740083in}}%
\pgfpathlineto{\pgfqpoint{1.345095in}{1.740179in}}%
\pgfpathlineto{\pgfqpoint{1.345961in}{1.740262in}}%
\pgfpathlineto{\pgfqpoint{1.346833in}{1.740332in}}%
\pgfpathlineto{\pgfqpoint{1.347708in}{1.740390in}}%
\pgfpathlineto{\pgfqpoint{1.348142in}{1.739670in}}%
\pgfpathlineto{\pgfqpoint{1.348576in}{1.738831in}}%
\pgfpathlineto{\pgfqpoint{1.349011in}{1.737872in}}%
\pgfpathlineto{\pgfqpoint{1.349446in}{1.736792in}}%
\pgfpathlineto{\pgfqpoint{1.349007in}{1.736763in}}%
\pgfpathlineto{\pgfqpoint{1.348571in}{1.736728in}}%
\pgfpathlineto{\pgfqpoint{1.348137in}{1.736687in}}%
\pgfpathlineto{\pgfqpoint{1.347707in}{1.736639in}}%
\pgfpathclose%
\pgfusepath{fill}%
\end{pgfscope}%
\begin{pgfscope}%
\pgfpathrectangle{\pgfqpoint{0.329460in}{0.284240in}}{\pgfqpoint{1.989680in}{1.989680in}}%
\pgfusepath{clip}%
\pgfsetbuttcap%
\pgfsetroundjoin%
\definecolor{currentfill}{rgb}{0.993248,0.906157,0.143936}%
\pgfsetfillcolor{currentfill}%
\pgfsetlinewidth{0.000000pt}%
\definecolor{currentstroke}{rgb}{0.000000,0.000000,0.000000}%
\pgfsetstrokecolor{currentstroke}%
\pgfsetdash{}{0pt}%
\pgfpathmoveto{\pgfqpoint{1.352978in}{1.736789in}}%
\pgfpathlineto{\pgfqpoint{1.353425in}{1.737868in}}%
\pgfpathlineto{\pgfqpoint{1.353872in}{1.738826in}}%
\pgfpathlineto{\pgfqpoint{1.354318in}{1.739665in}}%
\pgfpathlineto{\pgfqpoint{1.354764in}{1.740384in}}%
\pgfpathlineto{\pgfqpoint{1.355640in}{1.740325in}}%
\pgfpathlineto{\pgfqpoint{1.356511in}{1.740253in}}%
\pgfpathlineto{\pgfqpoint{1.357376in}{1.740169in}}%
\pgfpathlineto{\pgfqpoint{1.358236in}{1.740072in}}%
\pgfpathlineto{\pgfqpoint{1.357357in}{1.739391in}}%
\pgfpathlineto{\pgfqpoint{1.356477in}{1.738592in}}%
\pgfpathlineto{\pgfqpoint{1.355597in}{1.737672in}}%
\pgfpathlineto{\pgfqpoint{1.354716in}{1.736633in}}%
\pgfpathlineto{\pgfqpoint{1.354286in}{1.736682in}}%
\pgfpathlineto{\pgfqpoint{1.353852in}{1.736724in}}%
\pgfpathlineto{\pgfqpoint{1.353416in}{1.736760in}}%
\pgfpathlineto{\pgfqpoint{1.352978in}{1.736789in}}%
\pgfpathclose%
\pgfusepath{fill}%
\end{pgfscope}%
\begin{pgfscope}%
\pgfpathrectangle{\pgfqpoint{0.329460in}{0.284240in}}{\pgfqpoint{1.989680in}{1.989680in}}%
\pgfusepath{clip}%
\pgfsetbuttcap%
\pgfsetroundjoin%
\definecolor{currentfill}{rgb}{0.935904,0.898570,0.108131}%
\pgfsetfillcolor{currentfill}%
\pgfsetlinewidth{0.000000pt}%
\definecolor{currentstroke}{rgb}{0.000000,0.000000,0.000000}%
\pgfsetstrokecolor{currentstroke}%
\pgfsetdash{}{0pt}%
\pgfpathmoveto{\pgfqpoint{1.405999in}{1.721498in}}%
\pgfpathlineto{\pgfqpoint{1.409415in}{1.719888in}}%
\pgfpathlineto{\pgfqpoint{1.412830in}{1.718164in}}%
\pgfpathlineto{\pgfqpoint{1.416243in}{1.716325in}}%
\pgfpathlineto{\pgfqpoint{1.419655in}{1.714372in}}%
\pgfpathlineto{\pgfqpoint{1.420096in}{1.713359in}}%
\pgfpathlineto{\pgfqpoint{1.420469in}{1.712340in}}%
\pgfpathlineto{\pgfqpoint{1.420774in}{1.711315in}}%
\pgfpathlineto{\pgfqpoint{1.421009in}{1.710286in}}%
\pgfpathlineto{\pgfqpoint{1.417527in}{1.712444in}}%
\pgfpathlineto{\pgfqpoint{1.414045in}{1.714487in}}%
\pgfpathlineto{\pgfqpoint{1.410560in}{1.716417in}}%
\pgfpathlineto{\pgfqpoint{1.407075in}{1.718231in}}%
\pgfpathlineto{\pgfqpoint{1.406889in}{1.719054in}}%
\pgfpathlineto{\pgfqpoint{1.406647in}{1.719873in}}%
\pgfpathlineto{\pgfqpoint{1.406350in}{1.720688in}}%
\pgfpathlineto{\pgfqpoint{1.405999in}{1.721498in}}%
\pgfpathclose%
\pgfusepath{fill}%
\end{pgfscope}%
\begin{pgfscope}%
\pgfpathrectangle{\pgfqpoint{0.329460in}{0.284240in}}{\pgfqpoint{1.989680in}{1.989680in}}%
\pgfusepath{clip}%
\pgfsetbuttcap%
\pgfsetroundjoin%
\definecolor{currentfill}{rgb}{0.974417,0.903590,0.130215}%
\pgfsetfillcolor{currentfill}%
\pgfsetlinewidth{0.000000pt}%
\definecolor{currentstroke}{rgb}{0.000000,0.000000,0.000000}%
\pgfsetstrokecolor{currentstroke}%
\pgfsetdash{}{0pt}%
\pgfpathmoveto{\pgfqpoint{1.374481in}{1.734709in}}%
\pgfpathlineto{\pgfqpoint{1.377387in}{1.734602in}}%
\pgfpathlineto{\pgfqpoint{1.380292in}{1.734376in}}%
\pgfpathlineto{\pgfqpoint{1.383195in}{1.734032in}}%
\pgfpathlineto{\pgfqpoint{1.386097in}{1.733570in}}%
\pgfpathlineto{\pgfqpoint{1.386821in}{1.733051in}}%
\pgfpathlineto{\pgfqpoint{1.387510in}{1.732522in}}%
\pgfpathlineto{\pgfqpoint{1.388163in}{1.731983in}}%
\pgfpathlineto{\pgfqpoint{1.388779in}{1.731435in}}%
\pgfpathlineto{\pgfqpoint{1.385653in}{1.732075in}}%
\pgfpathlineto{\pgfqpoint{1.382526in}{1.732596in}}%
\pgfpathlineto{\pgfqpoint{1.379397in}{1.733000in}}%
\pgfpathlineto{\pgfqpoint{1.376267in}{1.733285in}}%
\pgfpathlineto{\pgfqpoint{1.375857in}{1.733651in}}%
\pgfpathlineto{\pgfqpoint{1.375422in}{1.734011in}}%
\pgfpathlineto{\pgfqpoint{1.374963in}{1.734363in}}%
\pgfpathlineto{\pgfqpoint{1.374481in}{1.734709in}}%
\pgfpathclose%
\pgfusepath{fill}%
\end{pgfscope}%
\begin{pgfscope}%
\pgfpathrectangle{\pgfqpoint{0.329460in}{0.284240in}}{\pgfqpoint{1.989680in}{1.989680in}}%
\pgfusepath{clip}%
\pgfsetbuttcap%
\pgfsetroundjoin%
\definecolor{currentfill}{rgb}{0.993248,0.906157,0.143936}%
\pgfsetfillcolor{currentfill}%
\pgfsetlinewidth{0.000000pt}%
\definecolor{currentstroke}{rgb}{0.000000,0.000000,0.000000}%
\pgfsetstrokecolor{currentstroke}%
\pgfsetdash{}{0pt}%
\pgfpathmoveto{\pgfqpoint{1.354716in}{1.736633in}}%
\pgfpathlineto{\pgfqpoint{1.355597in}{1.737672in}}%
\pgfpathlineto{\pgfqpoint{1.356477in}{1.738592in}}%
\pgfpathlineto{\pgfqpoint{1.357357in}{1.739391in}}%
\pgfpathlineto{\pgfqpoint{1.358236in}{1.740072in}}%
\pgfpathlineto{\pgfqpoint{1.359088in}{1.739962in}}%
\pgfpathlineto{\pgfqpoint{1.359933in}{1.739840in}}%
\pgfpathlineto{\pgfqpoint{1.360769in}{1.739705in}}%
\pgfpathlineto{\pgfqpoint{1.359574in}{1.739071in}}%
\pgfpathlineto{\pgfqpoint{1.358378in}{1.738317in}}%
\pgfpathlineto{\pgfqpoint{1.357181in}{1.737443in}}%
\pgfpathlineto{\pgfqpoint{1.355984in}{1.736449in}}%
\pgfpathlineto{\pgfqpoint{1.355565in}{1.736517in}}%
\pgfpathlineto{\pgfqpoint{1.355143in}{1.736578in}}%
\pgfpathlineto{\pgfqpoint{1.354716in}{1.736633in}}%
\pgfpathclose%
\pgfusepath{fill}%
\end{pgfscope}%
\begin{pgfscope}%
\pgfpathrectangle{\pgfqpoint{0.329460in}{0.284240in}}{\pgfqpoint{1.989680in}{1.989680in}}%
\pgfusepath{clip}%
\pgfsetbuttcap%
\pgfsetroundjoin%
\definecolor{currentfill}{rgb}{0.993248,0.906157,0.143936}%
\pgfsetfillcolor{currentfill}%
\pgfsetlinewidth{0.000000pt}%
\definecolor{currentstroke}{rgb}{0.000000,0.000000,0.000000}%
\pgfsetstrokecolor{currentstroke}%
\pgfsetdash{}{0pt}%
\pgfpathmoveto{\pgfqpoint{1.346023in}{1.736384in}}%
\pgfpathlineto{\pgfqpoint{1.344734in}{1.737362in}}%
\pgfpathlineto{\pgfqpoint{1.343446in}{1.738219in}}%
\pgfpathlineto{\pgfqpoint{1.342158in}{1.738957in}}%
\pgfpathlineto{\pgfqpoint{1.340871in}{1.739575in}}%
\pgfpathlineto{\pgfqpoint{1.341699in}{1.739721in}}%
\pgfpathlineto{\pgfqpoint{1.342536in}{1.739854in}}%
\pgfpathlineto{\pgfqpoint{1.343381in}{1.739975in}}%
\pgfpathlineto{\pgfqpoint{1.344235in}{1.740083in}}%
\pgfpathlineto{\pgfqpoint{1.345102in}{1.739401in}}%
\pgfpathlineto{\pgfqpoint{1.345970in}{1.738600in}}%
\pgfpathlineto{\pgfqpoint{1.346838in}{1.737680in}}%
\pgfpathlineto{\pgfqpoint{1.347707in}{1.736639in}}%
\pgfpathlineto{\pgfqpoint{1.347280in}{1.736584in}}%
\pgfpathlineto{\pgfqpoint{1.346856in}{1.736524in}}%
\pgfpathlineto{\pgfqpoint{1.346437in}{1.736457in}}%
\pgfpathlineto{\pgfqpoint{1.346023in}{1.736384in}}%
\pgfpathclose%
\pgfusepath{fill}%
\end{pgfscope}%
\begin{pgfscope}%
\pgfpathrectangle{\pgfqpoint{0.329460in}{0.284240in}}{\pgfqpoint{1.989680in}{1.989680in}}%
\pgfusepath{clip}%
\pgfsetbuttcap%
\pgfsetroundjoin%
\definecolor{currentfill}{rgb}{0.267004,0.004874,0.329415}%
\pgfsetfillcolor{currentfill}%
\pgfsetlinewidth{0.000000pt}%
\definecolor{currentstroke}{rgb}{0.000000,0.000000,0.000000}%
\pgfsetstrokecolor{currentstroke}%
\pgfsetdash{}{0pt}%
\pgfpathmoveto{\pgfqpoint{1.670966in}{0.812548in}}%
\pgfpathlineto{\pgfqpoint{1.673130in}{0.810383in}}%
\pgfpathlineto{\pgfqpoint{1.675300in}{0.808435in}}%
\pgfpathlineto{\pgfqpoint{1.677474in}{0.806707in}}%
\pgfpathlineto{\pgfqpoint{1.679653in}{0.805204in}}%
\pgfpathlineto{\pgfqpoint{1.665979in}{0.799715in}}%
\pgfpathlineto{\pgfqpoint{1.651957in}{0.794457in}}%
\pgfpathlineto{\pgfqpoint{1.637603in}{0.789438in}}%
\pgfpathlineto{\pgfqpoint{1.622932in}{0.784664in}}%
\pgfpathlineto{\pgfqpoint{1.621123in}{0.786308in}}%
\pgfpathlineto{\pgfqpoint{1.619319in}{0.788177in}}%
\pgfpathlineto{\pgfqpoint{1.617519in}{0.790267in}}%
\pgfpathlineto{\pgfqpoint{1.615723in}{0.792574in}}%
\pgfpathlineto{\pgfqpoint{1.630011in}{0.797216in}}%
\pgfpathlineto{\pgfqpoint{1.643990in}{0.802096in}}%
\pgfpathlineto{\pgfqpoint{1.657646in}{0.807209in}}%
\pgfpathlineto{\pgfqpoint{1.670966in}{0.812548in}}%
\pgfpathclose%
\pgfusepath{fill}%
\end{pgfscope}%
\begin{pgfscope}%
\pgfpathrectangle{\pgfqpoint{0.329460in}{0.284240in}}{\pgfqpoint{1.989680in}{1.989680in}}%
\pgfusepath{clip}%
\pgfsetbuttcap%
\pgfsetroundjoin%
\definecolor{currentfill}{rgb}{0.212395,0.359683,0.551710}%
\pgfsetfillcolor{currentfill}%
\pgfsetlinewidth{0.000000pt}%
\definecolor{currentstroke}{rgb}{0.000000,0.000000,0.000000}%
\pgfsetstrokecolor{currentstroke}%
\pgfsetdash{}{0pt}%
\pgfpathmoveto{\pgfqpoint{1.179120in}{1.078003in}}%
\pgfpathlineto{\pgfqpoint{1.177485in}{1.069766in}}%
\pgfpathlineto{\pgfqpoint{1.175849in}{1.061574in}}%
\pgfpathlineto{\pgfqpoint{1.174214in}{1.053429in}}%
\pgfpathlineto{\pgfqpoint{1.172579in}{1.045334in}}%
\pgfpathlineto{\pgfqpoint{1.162214in}{1.048357in}}%
\pgfpathlineto{\pgfqpoint{1.152053in}{1.051547in}}%
\pgfpathlineto{\pgfqpoint{1.142106in}{1.054901in}}%
\pgfpathlineto{\pgfqpoint{1.132383in}{1.058415in}}%
\pgfpathlineto{\pgfqpoint{1.134382in}{1.066373in}}%
\pgfpathlineto{\pgfqpoint{1.136381in}{1.074383in}}%
\pgfpathlineto{\pgfqpoint{1.138381in}{1.082440in}}%
\pgfpathlineto{\pgfqpoint{1.140380in}{1.090541in}}%
\pgfpathlineto{\pgfqpoint{1.149752in}{1.087173in}}%
\pgfpathlineto{\pgfqpoint{1.159339in}{1.083959in}}%
\pgfpathlineto{\pgfqpoint{1.169132in}{1.080901in}}%
\pgfpathlineto{\pgfqpoint{1.179120in}{1.078003in}}%
\pgfpathclose%
\pgfusepath{fill}%
\end{pgfscope}%
\begin{pgfscope}%
\pgfpathrectangle{\pgfqpoint{0.329460in}{0.284240in}}{\pgfqpoint{1.989680in}{1.989680in}}%
\pgfusepath{clip}%
\pgfsetbuttcap%
\pgfsetroundjoin%
\definecolor{currentfill}{rgb}{0.272594,0.025563,0.353093}%
\pgfsetfillcolor{currentfill}%
\pgfsetlinewidth{0.000000pt}%
\definecolor{currentstroke}{rgb}{0.000000,0.000000,0.000000}%
\pgfsetstrokecolor{currentstroke}%
\pgfsetdash{}{0pt}%
\pgfpathmoveto{\pgfqpoint{1.750893in}{0.827275in}}%
\pgfpathlineto{\pgfqpoint{1.753461in}{0.828164in}}%
\pgfpathlineto{\pgfqpoint{1.756037in}{0.829326in}}%
\pgfpathlineto{\pgfqpoint{1.758620in}{0.830763in}}%
\pgfpathlineto{\pgfqpoint{1.761212in}{0.832483in}}%
\pgfpathlineto{\pgfqpoint{1.748114in}{0.825623in}}%
\pgfpathlineto{\pgfqpoint{1.734580in}{0.818984in}}%
\pgfpathlineto{\pgfqpoint{1.720622in}{0.812572in}}%
\pgfpathlineto{\pgfqpoint{1.706257in}{0.806395in}}%
\pgfpathlineto{\pgfqpoint{1.704004in}{0.804834in}}%
\pgfpathlineto{\pgfqpoint{1.701758in}{0.803556in}}%
\pgfpathlineto{\pgfqpoint{1.699519in}{0.802555in}}%
\pgfpathlineto{\pgfqpoint{1.697287in}{0.801826in}}%
\pgfpathlineto{\pgfqpoint{1.711298in}{0.807851in}}%
\pgfpathlineto{\pgfqpoint{1.724912in}{0.814105in}}%
\pgfpathlineto{\pgfqpoint{1.738115in}{0.820583in}}%
\pgfpathlineto{\pgfqpoint{1.750893in}{0.827275in}}%
\pgfpathclose%
\pgfusepath{fill}%
\end{pgfscope}%
\begin{pgfscope}%
\pgfpathrectangle{\pgfqpoint{0.329460in}{0.284240in}}{\pgfqpoint{1.989680in}{1.989680in}}%
\pgfusepath{clip}%
\pgfsetbuttcap%
\pgfsetroundjoin%
\definecolor{currentfill}{rgb}{0.955300,0.901065,0.118128}%
\pgfsetfillcolor{currentfill}%
\pgfsetlinewidth{0.000000pt}%
\definecolor{currentstroke}{rgb}{0.000000,0.000000,0.000000}%
\pgfsetstrokecolor{currentstroke}%
\pgfsetdash{}{0pt}%
\pgfpathmoveto{\pgfqpoint{1.390864in}{1.729159in}}%
\pgfpathlineto{\pgfqpoint{1.394163in}{1.728212in}}%
\pgfpathlineto{\pgfqpoint{1.397461in}{1.727149in}}%
\pgfpathlineto{\pgfqpoint{1.400757in}{1.725969in}}%
\pgfpathlineto{\pgfqpoint{1.404052in}{1.724673in}}%
\pgfpathlineto{\pgfqpoint{1.404618in}{1.723891in}}%
\pgfpathlineto{\pgfqpoint{1.405132in}{1.723100in}}%
\pgfpathlineto{\pgfqpoint{1.405592in}{1.722302in}}%
\pgfpathlineto{\pgfqpoint{1.405999in}{1.721498in}}%
\pgfpathlineto{\pgfqpoint{1.402581in}{1.722992in}}%
\pgfpathlineto{\pgfqpoint{1.399162in}{1.724371in}}%
\pgfpathlineto{\pgfqpoint{1.395741in}{1.725633in}}%
\pgfpathlineto{\pgfqpoint{1.392320in}{1.726778in}}%
\pgfpathlineto{\pgfqpoint{1.392017in}{1.727381in}}%
\pgfpathlineto{\pgfqpoint{1.391673in}{1.727979in}}%
\pgfpathlineto{\pgfqpoint{1.391288in}{1.728572in}}%
\pgfpathlineto{\pgfqpoint{1.390864in}{1.729159in}}%
\pgfpathclose%
\pgfusepath{fill}%
\end{pgfscope}%
\begin{pgfscope}%
\pgfpathrectangle{\pgfqpoint{0.329460in}{0.284240in}}{\pgfqpoint{1.989680in}{1.989680in}}%
\pgfusepath{clip}%
\pgfsetbuttcap%
\pgfsetroundjoin%
\definecolor{currentfill}{rgb}{0.699415,0.867117,0.175971}%
\pgfsetfillcolor{currentfill}%
\pgfsetlinewidth{0.000000pt}%
\definecolor{currentstroke}{rgb}{0.000000,0.000000,0.000000}%
\pgfsetstrokecolor{currentstroke}%
\pgfsetdash{}{0pt}%
\pgfpathmoveto{\pgfqpoint{1.231414in}{1.637059in}}%
\pgfpathlineto{\pgfqpoint{1.228103in}{1.632420in}}%
\pgfpathlineto{\pgfqpoint{1.224793in}{1.627683in}}%
\pgfpathlineto{\pgfqpoint{1.221485in}{1.622850in}}%
\pgfpathlineto{\pgfqpoint{1.218178in}{1.617921in}}%
\pgfpathlineto{\pgfqpoint{1.216890in}{1.619925in}}%
\pgfpathlineto{\pgfqpoint{1.215737in}{1.621946in}}%
\pgfpathlineto{\pgfqpoint{1.214720in}{1.623983in}}%
\pgfpathlineto{\pgfqpoint{1.213840in}{1.626034in}}%
\pgfpathlineto{\pgfqpoint{1.217250in}{1.630754in}}%
\pgfpathlineto{\pgfqpoint{1.220661in}{1.635380in}}%
\pgfpathlineto{\pgfqpoint{1.224075in}{1.639910in}}%
\pgfpathlineto{\pgfqpoint{1.227491in}{1.644342in}}%
\pgfpathlineto{\pgfqpoint{1.228287in}{1.642501in}}%
\pgfpathlineto{\pgfqpoint{1.229207in}{1.640672in}}%
\pgfpathlineto{\pgfqpoint{1.230250in}{1.638858in}}%
\pgfpathlineto{\pgfqpoint{1.231414in}{1.637059in}}%
\pgfpathclose%
\pgfusepath{fill}%
\end{pgfscope}%
\begin{pgfscope}%
\pgfpathrectangle{\pgfqpoint{0.329460in}{0.284240in}}{\pgfqpoint{1.989680in}{1.989680in}}%
\pgfusepath{clip}%
\pgfsetbuttcap%
\pgfsetroundjoin%
\definecolor{currentfill}{rgb}{0.974417,0.903590,0.130215}%
\pgfsetfillcolor{currentfill}%
\pgfsetlinewidth{0.000000pt}%
\definecolor{currentstroke}{rgb}{0.000000,0.000000,0.000000}%
\pgfsetstrokecolor{currentstroke}%
\pgfsetdash{}{0pt}%
\pgfpathmoveto{\pgfqpoint{1.325764in}{1.732955in}}%
\pgfpathlineto{\pgfqpoint{1.322591in}{1.732629in}}%
\pgfpathlineto{\pgfqpoint{1.319419in}{1.732184in}}%
\pgfpathlineto{\pgfqpoint{1.316249in}{1.731621in}}%
\pgfpathlineto{\pgfqpoint{1.313080in}{1.730940in}}%
\pgfpathlineto{\pgfqpoint{1.313663in}{1.731496in}}%
\pgfpathlineto{\pgfqpoint{1.314283in}{1.732044in}}%
\pgfpathlineto{\pgfqpoint{1.314940in}{1.732581in}}%
\pgfpathlineto{\pgfqpoint{1.315633in}{1.733109in}}%
\pgfpathlineto{\pgfqpoint{1.318588in}{1.733609in}}%
\pgfpathlineto{\pgfqpoint{1.321545in}{1.733992in}}%
\pgfpathlineto{\pgfqpoint{1.324504in}{1.734256in}}%
\pgfpathlineto{\pgfqpoint{1.327464in}{1.734402in}}%
\pgfpathlineto{\pgfqpoint{1.327003in}{1.734050in}}%
\pgfpathlineto{\pgfqpoint{1.326565in}{1.733691in}}%
\pgfpathlineto{\pgfqpoint{1.326152in}{1.733326in}}%
\pgfpathlineto{\pgfqpoint{1.325764in}{1.732955in}}%
\pgfpathclose%
\pgfusepath{fill}%
\end{pgfscope}%
\begin{pgfscope}%
\pgfpathrectangle{\pgfqpoint{0.329460in}{0.284240in}}{\pgfqpoint{1.989680in}{1.989680in}}%
\pgfusepath{clip}%
\pgfsetbuttcap%
\pgfsetroundjoin%
\definecolor{currentfill}{rgb}{0.993248,0.906157,0.143936}%
\pgfsetfillcolor{currentfill}%
\pgfsetlinewidth{0.000000pt}%
\definecolor{currentstroke}{rgb}{0.000000,0.000000,0.000000}%
\pgfsetstrokecolor{currentstroke}%
\pgfsetdash{}{0pt}%
\pgfpathmoveto{\pgfqpoint{1.355984in}{1.736449in}}%
\pgfpathlineto{\pgfqpoint{1.357181in}{1.737443in}}%
\pgfpathlineto{\pgfqpoint{1.358378in}{1.738317in}}%
\pgfpathlineto{\pgfqpoint{1.359574in}{1.739071in}}%
\pgfpathlineto{\pgfqpoint{1.360769in}{1.739705in}}%
\pgfpathlineto{\pgfqpoint{1.361595in}{1.739558in}}%
\pgfpathlineto{\pgfqpoint{1.362411in}{1.739399in}}%
\pgfpathlineto{\pgfqpoint{1.363216in}{1.739229in}}%
\pgfpathlineto{\pgfqpoint{1.364010in}{1.739046in}}%
\pgfpathlineto{\pgfqpoint{1.362410in}{1.738494in}}%
\pgfpathlineto{\pgfqpoint{1.360810in}{1.737822in}}%
\pgfpathlineto{\pgfqpoint{1.359208in}{1.737031in}}%
\pgfpathlineto{\pgfqpoint{1.357606in}{1.736119in}}%
\pgfpathlineto{\pgfqpoint{1.357209in}{1.736211in}}%
\pgfpathlineto{\pgfqpoint{1.356806in}{1.736296in}}%
\pgfpathlineto{\pgfqpoint{1.356398in}{1.736376in}}%
\pgfpathlineto{\pgfqpoint{1.355984in}{1.736449in}}%
\pgfpathclose%
\pgfusepath{fill}%
\end{pgfscope}%
\begin{pgfscope}%
\pgfpathrectangle{\pgfqpoint{0.329460in}{0.284240in}}{\pgfqpoint{1.989680in}{1.989680in}}%
\pgfusepath{clip}%
\pgfsetbuttcap%
\pgfsetroundjoin%
\definecolor{currentfill}{rgb}{0.814576,0.883393,0.110347}%
\pgfsetfillcolor{currentfill}%
\pgfsetlinewidth{0.000000pt}%
\definecolor{currentstroke}{rgb}{0.000000,0.000000,0.000000}%
\pgfsetstrokecolor{currentstroke}%
\pgfsetdash{}{0pt}%
\pgfpathmoveto{\pgfqpoint{1.254873in}{1.676153in}}%
\pgfpathlineto{\pgfqpoint{1.251445in}{1.672541in}}%
\pgfpathlineto{\pgfqpoint{1.248018in}{1.668823in}}%
\pgfpathlineto{\pgfqpoint{1.244593in}{1.664999in}}%
\pgfpathlineto{\pgfqpoint{1.241169in}{1.661072in}}%
\pgfpathlineto{\pgfqpoint{1.240567in}{1.662714in}}%
\pgfpathlineto{\pgfqpoint{1.240076in}{1.664363in}}%
\pgfpathlineto{\pgfqpoint{1.239695in}{1.666019in}}%
\pgfpathlineto{\pgfqpoint{1.239426in}{1.667679in}}%
\pgfpathlineto{\pgfqpoint{1.242900in}{1.671397in}}%
\pgfpathlineto{\pgfqpoint{1.246376in}{1.675011in}}%
\pgfpathlineto{\pgfqpoint{1.249854in}{1.678520in}}%
\pgfpathlineto{\pgfqpoint{1.253333in}{1.681923in}}%
\pgfpathlineto{\pgfqpoint{1.253572in}{1.680473in}}%
\pgfpathlineto{\pgfqpoint{1.253908in}{1.679027in}}%
\pgfpathlineto{\pgfqpoint{1.254342in}{1.677587in}}%
\pgfpathlineto{\pgfqpoint{1.254873in}{1.676153in}}%
\pgfpathclose%
\pgfusepath{fill}%
\end{pgfscope}%
\begin{pgfscope}%
\pgfpathrectangle{\pgfqpoint{0.329460in}{0.284240in}}{\pgfqpoint{1.989680in}{1.989680in}}%
\pgfusepath{clip}%
\pgfsetbuttcap%
\pgfsetroundjoin%
\definecolor{currentfill}{rgb}{0.195860,0.395433,0.555276}%
\pgfsetfillcolor{currentfill}%
\pgfsetlinewidth{0.000000pt}%
\definecolor{currentstroke}{rgb}{0.000000,0.000000,0.000000}%
\pgfsetstrokecolor{currentstroke}%
\pgfsetdash{}{0pt}%
\pgfpathmoveto{\pgfqpoint{1.561834in}{1.126323in}}%
\pgfpathlineto{\pgfqpoint{1.563910in}{1.118105in}}%
\pgfpathlineto{\pgfqpoint{1.565986in}{1.109920in}}%
\pgfpathlineto{\pgfqpoint{1.568061in}{1.101771in}}%
\pgfpathlineto{\pgfqpoint{1.570137in}{1.093661in}}%
\pgfpathlineto{\pgfqpoint{1.560965in}{1.090160in}}%
\pgfpathlineto{\pgfqpoint{1.551569in}{1.086809in}}%
\pgfpathlineto{\pgfqpoint{1.541958in}{1.083611in}}%
\pgfpathlineto{\pgfqpoint{1.532143in}{1.080571in}}%
\pgfpathlineto{\pgfqpoint{1.530424in}{1.088821in}}%
\pgfpathlineto{\pgfqpoint{1.528705in}{1.097111in}}%
\pgfpathlineto{\pgfqpoint{1.526986in}{1.105436in}}%
\pgfpathlineto{\pgfqpoint{1.525267in}{1.113794in}}%
\pgfpathlineto{\pgfqpoint{1.534712in}{1.116703in}}%
\pgfpathlineto{\pgfqpoint{1.543961in}{1.119764in}}%
\pgfpathlineto{\pgfqpoint{1.553005in}{1.122971in}}%
\pgfpathlineto{\pgfqpoint{1.561834in}{1.126323in}}%
\pgfpathclose%
\pgfusepath{fill}%
\end{pgfscope}%
\begin{pgfscope}%
\pgfpathrectangle{\pgfqpoint{0.329460in}{0.284240in}}{\pgfqpoint{1.989680in}{1.989680in}}%
\pgfusepath{clip}%
\pgfsetbuttcap%
\pgfsetroundjoin%
\definecolor{currentfill}{rgb}{0.283072,0.130895,0.449241}%
\pgfsetfillcolor{currentfill}%
\pgfsetlinewidth{0.000000pt}%
\definecolor{currentstroke}{rgb}{0.000000,0.000000,0.000000}%
\pgfsetstrokecolor{currentstroke}%
\pgfsetdash{}{0pt}%
\pgfpathmoveto{\pgfqpoint{1.573507in}{0.902062in}}%
\pgfpathlineto{\pgfqpoint{1.575241in}{0.895784in}}%
\pgfpathlineto{\pgfqpoint{1.576977in}{0.889628in}}%
\pgfpathlineto{\pgfqpoint{1.578715in}{0.883598in}}%
\pgfpathlineto{\pgfqpoint{1.580454in}{0.877696in}}%
\pgfpathlineto{\pgfqpoint{1.567798in}{0.873933in}}%
\pgfpathlineto{\pgfqpoint{1.554906in}{0.870383in}}%
\pgfpathlineto{\pgfqpoint{1.541790in}{0.867051in}}%
\pgfpathlineto{\pgfqpoint{1.528465in}{0.863940in}}%
\pgfpathlineto{\pgfqpoint{1.527118in}{0.869959in}}%
\pgfpathlineto{\pgfqpoint{1.525772in}{0.876106in}}%
\pgfpathlineto{\pgfqpoint{1.524427in}{0.882378in}}%
\pgfpathlineto{\pgfqpoint{1.523083in}{0.888773in}}%
\pgfpathlineto{\pgfqpoint{1.536006in}{0.891778in}}%
\pgfpathlineto{\pgfqpoint{1.548727in}{0.894997in}}%
\pgfpathlineto{\pgfqpoint{1.561232in}{0.898426in}}%
\pgfpathlineto{\pgfqpoint{1.573507in}{0.902062in}}%
\pgfpathclose%
\pgfusepath{fill}%
\end{pgfscope}%
\begin{pgfscope}%
\pgfpathrectangle{\pgfqpoint{0.329460in}{0.284240in}}{\pgfqpoint{1.989680in}{1.989680in}}%
\pgfusepath{clip}%
\pgfsetbuttcap%
\pgfsetroundjoin%
\definecolor{currentfill}{rgb}{0.993248,0.906157,0.143936}%
\pgfsetfillcolor{currentfill}%
\pgfsetlinewidth{0.000000pt}%
\definecolor{currentstroke}{rgb}{0.000000,0.000000,0.000000}%
\pgfsetstrokecolor{currentstroke}%
\pgfsetdash{}{0pt}%
\pgfpathmoveto{\pgfqpoint{1.344422in}{1.736033in}}%
\pgfpathlineto{\pgfqpoint{1.342733in}{1.736923in}}%
\pgfpathlineto{\pgfqpoint{1.341044in}{1.737693in}}%
\pgfpathlineto{\pgfqpoint{1.339357in}{1.738343in}}%
\pgfpathlineto{\pgfqpoint{1.337671in}{1.738874in}}%
\pgfpathlineto{\pgfqpoint{1.338453in}{1.739067in}}%
\pgfpathlineto{\pgfqpoint{1.339248in}{1.739248in}}%
\pgfpathlineto{\pgfqpoint{1.340054in}{1.739418in}}%
\pgfpathlineto{\pgfqpoint{1.340871in}{1.739575in}}%
\pgfpathlineto{\pgfqpoint{1.342158in}{1.738957in}}%
\pgfpathlineto{\pgfqpoint{1.343446in}{1.738219in}}%
\pgfpathlineto{\pgfqpoint{1.344734in}{1.737362in}}%
\pgfpathlineto{\pgfqpoint{1.346023in}{1.736384in}}%
\pgfpathlineto{\pgfqpoint{1.345614in}{1.736305in}}%
\pgfpathlineto{\pgfqpoint{1.345211in}{1.736220in}}%
\pgfpathlineto{\pgfqpoint{1.344813in}{1.736130in}}%
\pgfpathlineto{\pgfqpoint{1.344422in}{1.736033in}}%
\pgfpathclose%
\pgfusepath{fill}%
\end{pgfscope}%
\begin{pgfscope}%
\pgfpathrectangle{\pgfqpoint{0.329460in}{0.284240in}}{\pgfqpoint{1.989680in}{1.989680in}}%
\pgfusepath{clip}%
\pgfsetbuttcap%
\pgfsetroundjoin%
\definecolor{currentfill}{rgb}{0.896320,0.893616,0.096335}%
\pgfsetfillcolor{currentfill}%
\pgfsetlinewidth{0.000000pt}%
\definecolor{currentstroke}{rgb}{0.000000,0.000000,0.000000}%
\pgfsetstrokecolor{currentstroke}%
\pgfsetdash{}{0pt}%
\pgfpathmoveto{\pgfqpoint{1.421009in}{1.710286in}}%
\pgfpathlineto{\pgfqpoint{1.424489in}{1.708016in}}%
\pgfpathlineto{\pgfqpoint{1.427967in}{1.705633in}}%
\pgfpathlineto{\pgfqpoint{1.431444in}{1.703138in}}%
\pgfpathlineto{\pgfqpoint{1.434920in}{1.700533in}}%
\pgfpathlineto{\pgfqpoint{1.435121in}{1.699293in}}%
\pgfpathlineto{\pgfqpoint{1.435239in}{1.698051in}}%
\pgfpathlineto{\pgfqpoint{1.435274in}{1.696807in}}%
\pgfpathlineto{\pgfqpoint{1.435224in}{1.695564in}}%
\pgfpathlineto{\pgfqpoint{1.431733in}{1.698378in}}%
\pgfpathlineto{\pgfqpoint{1.428241in}{1.701080in}}%
\pgfpathlineto{\pgfqpoint{1.424747in}{1.703672in}}%
\pgfpathlineto{\pgfqpoint{1.421253in}{1.706150in}}%
\pgfpathlineto{\pgfqpoint{1.421296in}{1.707185in}}%
\pgfpathlineto{\pgfqpoint{1.421270in}{1.708220in}}%
\pgfpathlineto{\pgfqpoint{1.421174in}{1.709254in}}%
\pgfpathlineto{\pgfqpoint{1.421009in}{1.710286in}}%
\pgfpathclose%
\pgfusepath{fill}%
\end{pgfscope}%
\begin{pgfscope}%
\pgfpathrectangle{\pgfqpoint{0.329460in}{0.284240in}}{\pgfqpoint{1.989680in}{1.989680in}}%
\pgfusepath{clip}%
\pgfsetbuttcap%
\pgfsetroundjoin%
\definecolor{currentfill}{rgb}{0.280255,0.165693,0.476498}%
\pgfsetfillcolor{currentfill}%
\pgfsetlinewidth{0.000000pt}%
\definecolor{currentstroke}{rgb}{0.000000,0.000000,0.000000}%
\pgfsetstrokecolor{currentstroke}%
\pgfsetdash{}{0pt}%
\pgfpathmoveto{\pgfqpoint{1.566583in}{0.928324in}}%
\pgfpathlineto{\pgfqpoint{1.568312in}{0.921593in}}%
\pgfpathlineto{\pgfqpoint{1.570042in}{0.914970in}}%
\pgfpathlineto{\pgfqpoint{1.571774in}{0.908459in}}%
\pgfpathlineto{\pgfqpoint{1.573507in}{0.902062in}}%
\pgfpathlineto{\pgfqpoint{1.561232in}{0.898426in}}%
\pgfpathlineto{\pgfqpoint{1.548727in}{0.894997in}}%
\pgfpathlineto{\pgfqpoint{1.536006in}{0.891778in}}%
\pgfpathlineto{\pgfqpoint{1.523083in}{0.888773in}}%
\pgfpathlineto{\pgfqpoint{1.521740in}{0.895286in}}%
\pgfpathlineto{\pgfqpoint{1.520399in}{0.901914in}}%
\pgfpathlineto{\pgfqpoint{1.519058in}{0.908654in}}%
\pgfpathlineto{\pgfqpoint{1.517719in}{0.915501in}}%
\pgfpathlineto{\pgfqpoint{1.530241in}{0.918400in}}%
\pgfpathlineto{\pgfqpoint{1.542568in}{0.921506in}}%
\pgfpathlineto{\pgfqpoint{1.554686in}{0.924816in}}%
\pgfpathlineto{\pgfqpoint{1.566583in}{0.928324in}}%
\pgfpathclose%
\pgfusepath{fill}%
\end{pgfscope}%
\begin{pgfscope}%
\pgfpathrectangle{\pgfqpoint{0.329460in}{0.284240in}}{\pgfqpoint{1.989680in}{1.989680in}}%
\pgfusepath{clip}%
\pgfsetbuttcap%
\pgfsetroundjoin%
\definecolor{currentfill}{rgb}{0.935904,0.898570,0.108131}%
\pgfsetfillcolor{currentfill}%
\pgfsetlinewidth{0.000000pt}%
\definecolor{currentstroke}{rgb}{0.000000,0.000000,0.000000}%
\pgfsetstrokecolor{currentstroke}%
\pgfsetdash{}{0pt}%
\pgfpathmoveto{\pgfqpoint{1.295181in}{1.717498in}}%
\pgfpathlineto{\pgfqpoint{1.291688in}{1.715638in}}%
\pgfpathlineto{\pgfqpoint{1.288196in}{1.713662in}}%
\pgfpathlineto{\pgfqpoint{1.284705in}{1.711573in}}%
\pgfpathlineto{\pgfqpoint{1.281216in}{1.709369in}}%
\pgfpathlineto{\pgfqpoint{1.281389in}{1.710401in}}%
\pgfpathlineto{\pgfqpoint{1.281632in}{1.711429in}}%
\pgfpathlineto{\pgfqpoint{1.281944in}{1.712453in}}%
\pgfpathlineto{\pgfqpoint{1.282325in}{1.713472in}}%
\pgfpathlineto{\pgfqpoint{1.285757in}{1.715469in}}%
\pgfpathlineto{\pgfqpoint{1.289190in}{1.717353in}}%
\pgfpathlineto{\pgfqpoint{1.292625in}{1.719123in}}%
\pgfpathlineto{\pgfqpoint{1.296061in}{1.720778in}}%
\pgfpathlineto{\pgfqpoint{1.295758in}{1.719964in}}%
\pgfpathlineto{\pgfqpoint{1.295511in}{1.719145in}}%
\pgfpathlineto{\pgfqpoint{1.295318in}{1.718323in}}%
\pgfpathlineto{\pgfqpoint{1.295181in}{1.717498in}}%
\pgfpathclose%
\pgfusepath{fill}%
\end{pgfscope}%
\begin{pgfscope}%
\pgfpathrectangle{\pgfqpoint{0.329460in}{0.284240in}}{\pgfqpoint{1.989680in}{1.989680in}}%
\pgfusepath{clip}%
\pgfsetbuttcap%
\pgfsetroundjoin%
\definecolor{currentfill}{rgb}{0.282327,0.094955,0.417331}%
\pgfsetfillcolor{currentfill}%
\pgfsetlinewidth{0.000000pt}%
\definecolor{currentstroke}{rgb}{0.000000,0.000000,0.000000}%
\pgfsetstrokecolor{currentstroke}%
\pgfsetdash{}{0pt}%
\pgfpathmoveto{\pgfqpoint{0.942617in}{0.836026in}}%
\pgfpathlineto{\pgfqpoint{0.940053in}{0.839195in}}%
\pgfpathlineto{\pgfqpoint{0.937479in}{0.842679in}}%
\pgfpathlineto{\pgfqpoint{0.934896in}{0.846481in}}%
\pgfpathlineto{\pgfqpoint{0.932303in}{0.850608in}}%
\pgfpathlineto{\pgfqpoint{0.918607in}{0.857819in}}%
\pgfpathlineto{\pgfqpoint{0.905386in}{0.865249in}}%
\pgfpathlineto{\pgfqpoint{0.892655in}{0.872890in}}%
\pgfpathlineto{\pgfqpoint{0.880424in}{0.880733in}}%
\pgfpathlineto{\pgfqpoint{0.883326in}{0.876439in}}%
\pgfpathlineto{\pgfqpoint{0.886216in}{0.872469in}}%
\pgfpathlineto{\pgfqpoint{0.889095in}{0.868816in}}%
\pgfpathlineto{\pgfqpoint{0.891965in}{0.865477in}}%
\pgfpathlineto{\pgfqpoint{0.903908in}{0.857809in}}%
\pgfpathlineto{\pgfqpoint{0.916339in}{0.850338in}}%
\pgfpathlineto{\pgfqpoint{0.929246in}{0.843074in}}%
\pgfpathlineto{\pgfqpoint{0.942617in}{0.836026in}}%
\pgfpathclose%
\pgfusepath{fill}%
\end{pgfscope}%
\begin{pgfscope}%
\pgfpathrectangle{\pgfqpoint{0.329460in}{0.284240in}}{\pgfqpoint{1.989680in}{1.989680in}}%
\pgfusepath{clip}%
\pgfsetbuttcap%
\pgfsetroundjoin%
\definecolor{currentfill}{rgb}{0.955300,0.901065,0.118128}%
\pgfsetfillcolor{currentfill}%
\pgfsetlinewidth{0.000000pt}%
\definecolor{currentstroke}{rgb}{0.000000,0.000000,0.000000}%
\pgfsetstrokecolor{currentstroke}%
\pgfsetdash{}{0pt}%
\pgfpathmoveto{\pgfqpoint{1.309820in}{1.726238in}}%
\pgfpathlineto{\pgfqpoint{1.306378in}{1.725048in}}%
\pgfpathlineto{\pgfqpoint{1.302938in}{1.723741in}}%
\pgfpathlineto{\pgfqpoint{1.299499in}{1.722318in}}%
\pgfpathlineto{\pgfqpoint{1.296061in}{1.720778in}}%
\pgfpathlineto{\pgfqpoint{1.296419in}{1.721588in}}%
\pgfpathlineto{\pgfqpoint{1.296831in}{1.722391in}}%
\pgfpathlineto{\pgfqpoint{1.297297in}{1.723188in}}%
\pgfpathlineto{\pgfqpoint{1.297817in}{1.723978in}}%
\pgfpathlineto{\pgfqpoint{1.301144in}{1.725317in}}%
\pgfpathlineto{\pgfqpoint{1.304472in}{1.726541in}}%
\pgfpathlineto{\pgfqpoint{1.307801in}{1.727648in}}%
\pgfpathlineto{\pgfqpoint{1.311132in}{1.728638in}}%
\pgfpathlineto{\pgfqpoint{1.310743in}{1.728045in}}%
\pgfpathlineto{\pgfqpoint{1.310395in}{1.727448in}}%
\pgfpathlineto{\pgfqpoint{1.310087in}{1.726845in}}%
\pgfpathlineto{\pgfqpoint{1.309820in}{1.726238in}}%
\pgfpathclose%
\pgfusepath{fill}%
\end{pgfscope}%
\begin{pgfscope}%
\pgfpathrectangle{\pgfqpoint{0.329460in}{0.284240in}}{\pgfqpoint{1.989680in}{1.989680in}}%
\pgfusepath{clip}%
\pgfsetbuttcap%
\pgfsetroundjoin%
\definecolor{currentfill}{rgb}{0.993248,0.906157,0.143936}%
\pgfsetfillcolor{currentfill}%
\pgfsetlinewidth{0.000000pt}%
\definecolor{currentstroke}{rgb}{0.000000,0.000000,0.000000}%
\pgfsetstrokecolor{currentstroke}%
\pgfsetdash{}{0pt}%
\pgfpathmoveto{\pgfqpoint{1.357606in}{1.736119in}}%
\pgfpathlineto{\pgfqpoint{1.359208in}{1.737031in}}%
\pgfpathlineto{\pgfqpoint{1.360810in}{1.737822in}}%
\pgfpathlineto{\pgfqpoint{1.362410in}{1.738494in}}%
\pgfpathlineto{\pgfqpoint{1.364010in}{1.739046in}}%
\pgfpathlineto{\pgfqpoint{1.364790in}{1.738852in}}%
\pgfpathlineto{\pgfqpoint{1.365557in}{1.738647in}}%
\pgfpathlineto{\pgfqpoint{1.366310in}{1.738430in}}%
\pgfpathlineto{\pgfqpoint{1.367047in}{1.738203in}}%
\pgfpathlineto{\pgfqpoint{1.365069in}{1.737755in}}%
\pgfpathlineto{\pgfqpoint{1.363089in}{1.737189in}}%
\pgfpathlineto{\pgfqpoint{1.361108in}{1.736503in}}%
\pgfpathlineto{\pgfqpoint{1.359126in}{1.735697in}}%
\pgfpathlineto{\pgfqpoint{1.358757in}{1.735811in}}%
\pgfpathlineto{\pgfqpoint{1.358380in}{1.735919in}}%
\pgfpathlineto{\pgfqpoint{1.357997in}{1.736022in}}%
\pgfpathlineto{\pgfqpoint{1.357606in}{1.736119in}}%
\pgfpathclose%
\pgfusepath{fill}%
\end{pgfscope}%
\begin{pgfscope}%
\pgfpathrectangle{\pgfqpoint{0.329460in}{0.284240in}}{\pgfqpoint{1.989680in}{1.989680in}}%
\pgfusepath{clip}%
\pgfsetbuttcap%
\pgfsetroundjoin%
\definecolor{currentfill}{rgb}{0.282327,0.094955,0.417331}%
\pgfsetfillcolor{currentfill}%
\pgfsetlinewidth{0.000000pt}%
\definecolor{currentstroke}{rgb}{0.000000,0.000000,0.000000}%
\pgfsetstrokecolor{currentstroke}%
\pgfsetdash{}{0pt}%
\pgfpathmoveto{\pgfqpoint{1.580454in}{0.877696in}}%
\pgfpathlineto{\pgfqpoint{1.582195in}{0.871928in}}%
\pgfpathlineto{\pgfqpoint{1.583937in}{0.866295in}}%
\pgfpathlineto{\pgfqpoint{1.585682in}{0.860803in}}%
\pgfpathlineto{\pgfqpoint{1.587428in}{0.855455in}}%
\pgfpathlineto{\pgfqpoint{1.574392in}{0.851563in}}%
\pgfpathlineto{\pgfqpoint{1.561110in}{0.847893in}}%
\pgfpathlineto{\pgfqpoint{1.547598in}{0.844448in}}%
\pgfpathlineto{\pgfqpoint{1.533870in}{0.841232in}}%
\pgfpathlineto{\pgfqpoint{1.532516in}{0.846697in}}%
\pgfpathlineto{\pgfqpoint{1.531164in}{0.852306in}}%
\pgfpathlineto{\pgfqpoint{1.529814in}{0.858055in}}%
\pgfpathlineto{\pgfqpoint{1.528465in}{0.863940in}}%
\pgfpathlineto{\pgfqpoint{1.541790in}{0.867051in}}%
\pgfpathlineto{\pgfqpoint{1.554906in}{0.870383in}}%
\pgfpathlineto{\pgfqpoint{1.567798in}{0.873933in}}%
\pgfpathlineto{\pgfqpoint{1.580454in}{0.877696in}}%
\pgfpathclose%
\pgfusepath{fill}%
\end{pgfscope}%
\begin{pgfscope}%
\pgfpathrectangle{\pgfqpoint{0.329460in}{0.284240in}}{\pgfqpoint{1.989680in}{1.989680in}}%
\pgfusepath{clip}%
\pgfsetbuttcap%
\pgfsetroundjoin%
\definecolor{currentfill}{rgb}{0.412913,0.803041,0.357269}%
\pgfsetfillcolor{currentfill}%
\pgfsetlinewidth{0.000000pt}%
\definecolor{currentstroke}{rgb}{0.000000,0.000000,0.000000}%
\pgfsetstrokecolor{currentstroke}%
\pgfsetdash{}{0pt}%
\pgfpathmoveto{\pgfqpoint{1.517487in}{1.543839in}}%
\pgfpathlineto{\pgfqpoint{1.520655in}{1.537638in}}%
\pgfpathlineto{\pgfqpoint{1.523821in}{1.531363in}}%
\pgfpathlineto{\pgfqpoint{1.526986in}{1.525014in}}%
\pgfpathlineto{\pgfqpoint{1.530148in}{1.518594in}}%
\pgfpathlineto{\pgfqpoint{1.527615in}{1.515884in}}%
\pgfpathlineto{\pgfqpoint{1.524902in}{1.513213in}}%
\pgfpathlineto{\pgfqpoint{1.522013in}{1.510583in}}%
\pgfpathlineto{\pgfqpoint{1.518949in}{1.507998in}}%
\pgfpathlineto{\pgfqpoint{1.515979in}{1.514617in}}%
\pgfpathlineto{\pgfqpoint{1.513008in}{1.521164in}}%
\pgfpathlineto{\pgfqpoint{1.510035in}{1.527637in}}%
\pgfpathlineto{\pgfqpoint{1.507061in}{1.534035in}}%
\pgfpathlineto{\pgfqpoint{1.509912in}{1.536427in}}%
\pgfpathlineto{\pgfqpoint{1.512602in}{1.538860in}}%
\pgfpathlineto{\pgfqpoint{1.515128in}{1.541331in}}%
\pgfpathlineto{\pgfqpoint{1.517487in}{1.543839in}}%
\pgfpathclose%
\pgfusepath{fill}%
\end{pgfscope}%
\begin{pgfscope}%
\pgfpathrectangle{\pgfqpoint{0.329460in}{0.284240in}}{\pgfqpoint{1.989680in}{1.989680in}}%
\pgfusepath{clip}%
\pgfsetbuttcap%
\pgfsetroundjoin%
\definecolor{currentfill}{rgb}{0.993248,0.906157,0.143936}%
\pgfsetfillcolor{currentfill}%
\pgfsetlinewidth{0.000000pt}%
\definecolor{currentstroke}{rgb}{0.000000,0.000000,0.000000}%
\pgfsetstrokecolor{currentstroke}%
\pgfsetdash{}{0pt}%
\pgfpathmoveto{\pgfqpoint{1.342928in}{1.735591in}}%
\pgfpathlineto{\pgfqpoint{1.340865in}{1.736371in}}%
\pgfpathlineto{\pgfqpoint{1.338804in}{1.737030in}}%
\pgfpathlineto{\pgfqpoint{1.336744in}{1.737571in}}%
\pgfpathlineto{\pgfqpoint{1.334685in}{1.737991in}}%
\pgfpathlineto{\pgfqpoint{1.335409in}{1.738228in}}%
\pgfpathlineto{\pgfqpoint{1.336148in}{1.738455in}}%
\pgfpathlineto{\pgfqpoint{1.336903in}{1.738670in}}%
\pgfpathlineto{\pgfqpoint{1.337671in}{1.738874in}}%
\pgfpathlineto{\pgfqpoint{1.339357in}{1.738343in}}%
\pgfpathlineto{\pgfqpoint{1.341044in}{1.737693in}}%
\pgfpathlineto{\pgfqpoint{1.342733in}{1.736923in}}%
\pgfpathlineto{\pgfqpoint{1.344422in}{1.736033in}}%
\pgfpathlineto{\pgfqpoint{1.344037in}{1.735931in}}%
\pgfpathlineto{\pgfqpoint{1.343660in}{1.735823in}}%
\pgfpathlineto{\pgfqpoint{1.343290in}{1.735710in}}%
\pgfpathlineto{\pgfqpoint{1.342928in}{1.735591in}}%
\pgfpathclose%
\pgfusepath{fill}%
\end{pgfscope}%
\begin{pgfscope}%
\pgfpathrectangle{\pgfqpoint{0.329460in}{0.284240in}}{\pgfqpoint{1.989680in}{1.989680in}}%
\pgfusepath{clip}%
\pgfsetbuttcap%
\pgfsetroundjoin%
\definecolor{currentfill}{rgb}{0.120081,0.622161,0.534946}%
\pgfsetfillcolor{currentfill}%
\pgfsetlinewidth{0.000000pt}%
\definecolor{currentstroke}{rgb}{0.000000,0.000000,0.000000}%
\pgfsetstrokecolor{currentstroke}%
\pgfsetdash{}{0pt}%
\pgfpathmoveto{\pgfqpoint{1.170627in}{1.334809in}}%
\pgfpathlineto{\pgfqpoint{1.168280in}{1.326715in}}%
\pgfpathlineto{\pgfqpoint{1.165934in}{1.318592in}}%
\pgfpathlineto{\pgfqpoint{1.163589in}{1.310444in}}%
\pgfpathlineto{\pgfqpoint{1.161245in}{1.302271in}}%
\pgfpathlineto{\pgfqpoint{1.154918in}{1.305321in}}%
\pgfpathlineto{\pgfqpoint{1.148796in}{1.308468in}}%
\pgfpathlineto{\pgfqpoint{1.142886in}{1.311708in}}%
\pgfpathlineto{\pgfqpoint{1.137192in}{1.315037in}}%
\pgfpathlineto{\pgfqpoint{1.139827in}{1.323036in}}%
\pgfpathlineto{\pgfqpoint{1.142464in}{1.331011in}}%
\pgfpathlineto{\pgfqpoint{1.145102in}{1.338961in}}%
\pgfpathlineto{\pgfqpoint{1.147742in}{1.346882in}}%
\pgfpathlineto{\pgfqpoint{1.153160in}{1.343733in}}%
\pgfpathlineto{\pgfqpoint{1.158784in}{1.340669in}}%
\pgfpathlineto{\pgfqpoint{1.164608in}{1.337693in}}%
\pgfpathlineto{\pgfqpoint{1.170627in}{1.334809in}}%
\pgfpathclose%
\pgfusepath{fill}%
\end{pgfscope}%
\begin{pgfscope}%
\pgfpathrectangle{\pgfqpoint{0.329460in}{0.284240in}}{\pgfqpoint{1.989680in}{1.989680in}}%
\pgfusepath{clip}%
\pgfsetbuttcap%
\pgfsetroundjoin%
\definecolor{currentfill}{rgb}{0.274128,0.199721,0.498911}%
\pgfsetfillcolor{currentfill}%
\pgfsetlinewidth{0.000000pt}%
\definecolor{currentstroke}{rgb}{0.000000,0.000000,0.000000}%
\pgfsetstrokecolor{currentstroke}%
\pgfsetdash{}{0pt}%
\pgfpathmoveto{\pgfqpoint{1.559676in}{0.956262in}}%
\pgfpathlineto{\pgfqpoint{1.561401in}{0.949132in}}%
\pgfpathlineto{\pgfqpoint{1.563127in}{0.942097in}}%
\pgfpathlineto{\pgfqpoint{1.564854in}{0.935160in}}%
\pgfpathlineto{\pgfqpoint{1.566583in}{0.928324in}}%
\pgfpathlineto{\pgfqpoint{1.554686in}{0.924816in}}%
\pgfpathlineto{\pgfqpoint{1.542568in}{0.921506in}}%
\pgfpathlineto{\pgfqpoint{1.530241in}{0.918400in}}%
\pgfpathlineto{\pgfqpoint{1.517719in}{0.915501in}}%
\pgfpathlineto{\pgfqpoint{1.516380in}{0.922453in}}%
\pgfpathlineto{\pgfqpoint{1.515042in}{0.929506in}}%
\pgfpathlineto{\pgfqpoint{1.513705in}{0.936658in}}%
\pgfpathlineto{\pgfqpoint{1.512368in}{0.943903in}}%
\pgfpathlineto{\pgfqpoint{1.524491in}{0.946697in}}%
\pgfpathlineto{\pgfqpoint{1.536425in}{0.949690in}}%
\pgfpathlineto{\pgfqpoint{1.548157in}{0.952880in}}%
\pgfpathlineto{\pgfqpoint{1.559676in}{0.956262in}}%
\pgfpathclose%
\pgfusepath{fill}%
\end{pgfscope}%
\begin{pgfscope}%
\pgfpathrectangle{\pgfqpoint{0.329460in}{0.284240in}}{\pgfqpoint{1.989680in}{1.989680in}}%
\pgfusepath{clip}%
\pgfsetbuttcap%
\pgfsetroundjoin%
\definecolor{currentfill}{rgb}{0.565498,0.842430,0.262877}%
\pgfsetfillcolor{currentfill}%
\pgfsetlinewidth{0.000000pt}%
\definecolor{currentstroke}{rgb}{0.000000,0.000000,0.000000}%
\pgfsetstrokecolor{currentstroke}%
\pgfsetdash{}{0pt}%
\pgfpathmoveto{\pgfqpoint{1.212073in}{1.588676in}}%
\pgfpathlineto{\pgfqpoint{1.208930in}{1.583088in}}%
\pgfpathlineto{\pgfqpoint{1.205789in}{1.577413in}}%
\pgfpathlineto{\pgfqpoint{1.202650in}{1.571652in}}%
\pgfpathlineto{\pgfqpoint{1.199512in}{1.565806in}}%
\pgfpathlineto{\pgfqpoint{1.197345in}{1.568116in}}%
\pgfpathlineto{\pgfqpoint{1.195334in}{1.570458in}}%
\pgfpathlineto{\pgfqpoint{1.193481in}{1.572827in}}%
\pgfpathlineto{\pgfqpoint{1.191787in}{1.575222in}}%
\pgfpathlineto{\pgfqpoint{1.195079in}{1.580864in}}%
\pgfpathlineto{\pgfqpoint{1.198374in}{1.586422in}}%
\pgfpathlineto{\pgfqpoint{1.201670in}{1.591895in}}%
\pgfpathlineto{\pgfqpoint{1.204968in}{1.597280in}}%
\pgfpathlineto{\pgfqpoint{1.206527in}{1.595091in}}%
\pgfpathlineto{\pgfqpoint{1.208231in}{1.592926in}}%
\pgfpathlineto{\pgfqpoint{1.210081in}{1.590787in}}%
\pgfpathlineto{\pgfqpoint{1.212073in}{1.588676in}}%
\pgfpathclose%
\pgfusepath{fill}%
\end{pgfscope}%
\begin{pgfscope}%
\pgfpathrectangle{\pgfqpoint{0.329460in}{0.284240in}}{\pgfqpoint{1.989680in}{1.989680in}}%
\pgfusepath{clip}%
\pgfsetbuttcap%
\pgfsetroundjoin%
\definecolor{currentfill}{rgb}{0.147607,0.511733,0.557049}%
\pgfsetfillcolor{currentfill}%
\pgfsetlinewidth{0.000000pt}%
\definecolor{currentstroke}{rgb}{0.000000,0.000000,0.000000}%
\pgfsetstrokecolor{currentstroke}%
\pgfsetdash{}{0pt}%
\pgfpathmoveto{\pgfqpoint{1.172407in}{1.223923in}}%
\pgfpathlineto{\pgfqpoint{1.170402in}{1.215489in}}%
\pgfpathlineto{\pgfqpoint{1.168397in}{1.207056in}}%
\pgfpathlineto{\pgfqpoint{1.166393in}{1.198627in}}%
\pgfpathlineto{\pgfqpoint{1.164390in}{1.190203in}}%
\pgfpathlineto{\pgfqpoint{1.156272in}{1.193271in}}%
\pgfpathlineto{\pgfqpoint{1.148361in}{1.196467in}}%
\pgfpathlineto{\pgfqpoint{1.140665in}{1.199787in}}%
\pgfpathlineto{\pgfqpoint{1.133190in}{1.203227in}}%
\pgfpathlineto{\pgfqpoint{1.135524in}{1.211494in}}%
\pgfpathlineto{\pgfqpoint{1.137858in}{1.219768in}}%
\pgfpathlineto{\pgfqpoint{1.140192in}{1.228045in}}%
\pgfpathlineto{\pgfqpoint{1.142528in}{1.236323in}}%
\pgfpathlineto{\pgfqpoint{1.149687in}{1.233047in}}%
\pgfpathlineto{\pgfqpoint{1.157058in}{1.229887in}}%
\pgfpathlineto{\pgfqpoint{1.164634in}{1.226844in}}%
\pgfpathlineto{\pgfqpoint{1.172407in}{1.223923in}}%
\pgfpathclose%
\pgfusepath{fill}%
\end{pgfscope}%
\begin{pgfscope}%
\pgfpathrectangle{\pgfqpoint{0.329460in}{0.284240in}}{\pgfqpoint{1.989680in}{1.989680in}}%
\pgfusepath{clip}%
\pgfsetbuttcap%
\pgfsetroundjoin%
\definecolor{currentfill}{rgb}{0.896320,0.893616,0.096335}%
\pgfsetfillcolor{currentfill}%
\pgfsetlinewidth{0.000000pt}%
\definecolor{currentstroke}{rgb}{0.000000,0.000000,0.000000}%
\pgfsetstrokecolor{currentstroke}%
\pgfsetdash{}{0pt}%
\pgfpathmoveto{\pgfqpoint{1.281220in}{1.705231in}}%
\pgfpathlineto{\pgfqpoint{1.277730in}{1.702706in}}%
\pgfpathlineto{\pgfqpoint{1.274240in}{1.700069in}}%
\pgfpathlineto{\pgfqpoint{1.270752in}{1.697319in}}%
\pgfpathlineto{\pgfqpoint{1.267266in}{1.694459in}}%
\pgfpathlineto{\pgfqpoint{1.267141in}{1.695702in}}%
\pgfpathlineto{\pgfqpoint{1.267101in}{1.696946in}}%
\pgfpathlineto{\pgfqpoint{1.267145in}{1.698189in}}%
\pgfpathlineto{\pgfqpoint{1.267272in}{1.699431in}}%
\pgfpathlineto{\pgfqpoint{1.270756in}{1.702083in}}%
\pgfpathlineto{\pgfqpoint{1.274241in}{1.704623in}}%
\pgfpathlineto{\pgfqpoint{1.277728in}{1.707052in}}%
\pgfpathlineto{\pgfqpoint{1.281216in}{1.709369in}}%
\pgfpathlineto{\pgfqpoint{1.281112in}{1.708335in}}%
\pgfpathlineto{\pgfqpoint{1.281078in}{1.707300in}}%
\pgfpathlineto{\pgfqpoint{1.281114in}{1.706265in}}%
\pgfpathlineto{\pgfqpoint{1.281220in}{1.705231in}}%
\pgfpathclose%
\pgfusepath{fill}%
\end{pgfscope}%
\begin{pgfscope}%
\pgfpathrectangle{\pgfqpoint{0.329460in}{0.284240in}}{\pgfqpoint{1.989680in}{1.989680in}}%
\pgfusepath{clip}%
\pgfsetbuttcap%
\pgfsetroundjoin%
\definecolor{currentfill}{rgb}{0.993248,0.906157,0.143936}%
\pgfsetfillcolor{currentfill}%
\pgfsetlinewidth{0.000000pt}%
\definecolor{currentstroke}{rgb}{0.000000,0.000000,0.000000}%
\pgfsetstrokecolor{currentstroke}%
\pgfsetdash{}{0pt}%
\pgfpathmoveto{\pgfqpoint{1.359126in}{1.735697in}}%
\pgfpathlineto{\pgfqpoint{1.361108in}{1.736503in}}%
\pgfpathlineto{\pgfqpoint{1.363089in}{1.737189in}}%
\pgfpathlineto{\pgfqpoint{1.365069in}{1.737755in}}%
\pgfpathlineto{\pgfqpoint{1.367047in}{1.738203in}}%
\pgfpathlineto{\pgfqpoint{1.367769in}{1.737964in}}%
\pgfpathlineto{\pgfqpoint{1.368475in}{1.737716in}}%
\pgfpathlineto{\pgfqpoint{1.369163in}{1.737457in}}%
\pgfpathlineto{\pgfqpoint{1.369834in}{1.737187in}}%
\pgfpathlineto{\pgfqpoint{1.367507in}{1.736867in}}%
\pgfpathlineto{\pgfqpoint{1.365180in}{1.736427in}}%
\pgfpathlineto{\pgfqpoint{1.362850in}{1.735868in}}%
\pgfpathlineto{\pgfqpoint{1.360520in}{1.735189in}}%
\pgfpathlineto{\pgfqpoint{1.360185in}{1.735323in}}%
\pgfpathlineto{\pgfqpoint{1.359840in}{1.735453in}}%
\pgfpathlineto{\pgfqpoint{1.359487in}{1.735578in}}%
\pgfpathlineto{\pgfqpoint{1.359126in}{1.735697in}}%
\pgfpathclose%
\pgfusepath{fill}%
\end{pgfscope}%
\begin{pgfscope}%
\pgfpathrectangle{\pgfqpoint{0.329460in}{0.284240in}}{\pgfqpoint{1.989680in}{1.989680in}}%
\pgfusepath{clip}%
\pgfsetbuttcap%
\pgfsetroundjoin%
\definecolor{currentfill}{rgb}{0.993248,0.906157,0.143936}%
\pgfsetfillcolor{currentfill}%
\pgfsetlinewidth{0.000000pt}%
\definecolor{currentstroke}{rgb}{0.000000,0.000000,0.000000}%
\pgfsetstrokecolor{currentstroke}%
\pgfsetdash{}{0pt}%
\pgfpathmoveto{\pgfqpoint{1.341565in}{1.735065in}}%
\pgfpathlineto{\pgfqpoint{1.339162in}{1.735713in}}%
\pgfpathlineto{\pgfqpoint{1.336760in}{1.736242in}}%
\pgfpathlineto{\pgfqpoint{1.334360in}{1.736651in}}%
\pgfpathlineto{\pgfqpoint{1.331960in}{1.736940in}}%
\pgfpathlineto{\pgfqpoint{1.332615in}{1.737218in}}%
\pgfpathlineto{\pgfqpoint{1.333287in}{1.737486in}}%
\pgfpathlineto{\pgfqpoint{1.333978in}{1.737744in}}%
\pgfpathlineto{\pgfqpoint{1.334685in}{1.737991in}}%
\pgfpathlineto{\pgfqpoint{1.336744in}{1.737571in}}%
\pgfpathlineto{\pgfqpoint{1.338804in}{1.737030in}}%
\pgfpathlineto{\pgfqpoint{1.340865in}{1.736371in}}%
\pgfpathlineto{\pgfqpoint{1.342928in}{1.735591in}}%
\pgfpathlineto{\pgfqpoint{1.342574in}{1.735467in}}%
\pgfpathlineto{\pgfqpoint{1.342228in}{1.735338in}}%
\pgfpathlineto{\pgfqpoint{1.341892in}{1.735204in}}%
\pgfpathlineto{\pgfqpoint{1.341565in}{1.735065in}}%
\pgfpathclose%
\pgfusepath{fill}%
\end{pgfscope}%
\begin{pgfscope}%
\pgfpathrectangle{\pgfqpoint{0.329460in}{0.284240in}}{\pgfqpoint{1.989680in}{1.989680in}}%
\pgfusepath{clip}%
\pgfsetbuttcap%
\pgfsetroundjoin%
\definecolor{currentfill}{rgb}{0.974417,0.903590,0.130215}%
\pgfsetfillcolor{currentfill}%
\pgfsetlinewidth{0.000000pt}%
\definecolor{currentstroke}{rgb}{0.000000,0.000000,0.000000}%
\pgfsetstrokecolor{currentstroke}%
\pgfsetdash{}{0pt}%
\pgfpathmoveto{\pgfqpoint{1.376267in}{1.733285in}}%
\pgfpathlineto{\pgfqpoint{1.379397in}{1.733000in}}%
\pgfpathlineto{\pgfqpoint{1.382526in}{1.732596in}}%
\pgfpathlineto{\pgfqpoint{1.385653in}{1.732075in}}%
\pgfpathlineto{\pgfqpoint{1.388779in}{1.731435in}}%
\pgfpathlineto{\pgfqpoint{1.389357in}{1.730878in}}%
\pgfpathlineto{\pgfqpoint{1.389898in}{1.730312in}}%
\pgfpathlineto{\pgfqpoint{1.390401in}{1.729739in}}%
\pgfpathlineto{\pgfqpoint{1.390864in}{1.729159in}}%
\pgfpathlineto{\pgfqpoint{1.387564in}{1.729988in}}%
\pgfpathlineto{\pgfqpoint{1.384262in}{1.730700in}}%
\pgfpathlineto{\pgfqpoint{1.380959in}{1.731293in}}%
\pgfpathlineto{\pgfqpoint{1.377655in}{1.731768in}}%
\pgfpathlineto{\pgfqpoint{1.377347in}{1.732155in}}%
\pgfpathlineto{\pgfqpoint{1.377013in}{1.732537in}}%
\pgfpathlineto{\pgfqpoint{1.376653in}{1.732914in}}%
\pgfpathlineto{\pgfqpoint{1.376267in}{1.733285in}}%
\pgfpathclose%
\pgfusepath{fill}%
\end{pgfscope}%
\begin{pgfscope}%
\pgfpathrectangle{\pgfqpoint{0.329460in}{0.284240in}}{\pgfqpoint{1.989680in}{1.989680in}}%
\pgfusepath{clip}%
\pgfsetbuttcap%
\pgfsetroundjoin%
\definecolor{currentfill}{rgb}{0.220124,0.725509,0.466226}%
\pgfsetfillcolor{currentfill}%
\pgfsetlinewidth{0.000000pt}%
\definecolor{currentstroke}{rgb}{0.000000,0.000000,0.000000}%
\pgfsetstrokecolor{currentstroke}%
\pgfsetdash{}{0pt}%
\pgfpathmoveto{\pgfqpoint{1.179518in}{1.438945in}}%
\pgfpathlineto{\pgfqpoint{1.176863in}{1.431531in}}%
\pgfpathlineto{\pgfqpoint{1.174208in}{1.424064in}}%
\pgfpathlineto{\pgfqpoint{1.171556in}{1.416544in}}%
\pgfpathlineto{\pgfqpoint{1.168904in}{1.408975in}}%
\pgfpathlineto{\pgfqpoint{1.164228in}{1.411841in}}%
\pgfpathlineto{\pgfqpoint{1.159746in}{1.414777in}}%
\pgfpathlineto{\pgfqpoint{1.155460in}{1.417779in}}%
\pgfpathlineto{\pgfqpoint{1.151376in}{1.420844in}}%
\pgfpathlineto{\pgfqpoint{1.154277in}{1.428226in}}%
\pgfpathlineto{\pgfqpoint{1.157179in}{1.435559in}}%
\pgfpathlineto{\pgfqpoint{1.160083in}{1.442841in}}%
\pgfpathlineto{\pgfqpoint{1.162989in}{1.450069in}}%
\pgfpathlineto{\pgfqpoint{1.166841in}{1.447196in}}%
\pgfpathlineto{\pgfqpoint{1.170883in}{1.444382in}}%
\pgfpathlineto{\pgfqpoint{1.175110in}{1.441631in}}%
\pgfpathlineto{\pgfqpoint{1.179518in}{1.438945in}}%
\pgfpathclose%
\pgfusepath{fill}%
\end{pgfscope}%
\begin{pgfscope}%
\pgfpathrectangle{\pgfqpoint{0.329460in}{0.284240in}}{\pgfqpoint{1.989680in}{1.989680in}}%
\pgfusepath{clip}%
\pgfsetbuttcap%
\pgfsetroundjoin%
\definecolor{currentfill}{rgb}{0.762373,0.876424,0.137064}%
\pgfsetfillcolor{currentfill}%
\pgfsetlinewidth{0.000000pt}%
\definecolor{currentstroke}{rgb}{0.000000,0.000000,0.000000}%
\pgfsetstrokecolor{currentstroke}%
\pgfsetdash{}{0pt}%
\pgfpathmoveto{\pgfqpoint{1.461747in}{1.662531in}}%
\pgfpathlineto{\pgfqpoint{1.465185in}{1.658547in}}%
\pgfpathlineto{\pgfqpoint{1.468621in}{1.654460in}}%
\pgfpathlineto{\pgfqpoint{1.472056in}{1.650273in}}%
\pgfpathlineto{\pgfqpoint{1.475489in}{1.645987in}}%
\pgfpathlineto{\pgfqpoint{1.474802in}{1.644136in}}%
\pgfpathlineto{\pgfqpoint{1.473992in}{1.642297in}}%
\pgfpathlineto{\pgfqpoint{1.473058in}{1.640470in}}%
\pgfpathlineto{\pgfqpoint{1.472002in}{1.638657in}}%
\pgfpathlineto{\pgfqpoint{1.468661in}{1.643152in}}%
\pgfpathlineto{\pgfqpoint{1.465319in}{1.647546in}}%
\pgfpathlineto{\pgfqpoint{1.461975in}{1.651840in}}%
\pgfpathlineto{\pgfqpoint{1.458630in}{1.656031in}}%
\pgfpathlineto{\pgfqpoint{1.459573in}{1.657638in}}%
\pgfpathlineto{\pgfqpoint{1.460408in}{1.659259in}}%
\pgfpathlineto{\pgfqpoint{1.461132in}{1.660890in}}%
\pgfpathlineto{\pgfqpoint{1.461747in}{1.662531in}}%
\pgfpathclose%
\pgfusepath{fill}%
\end{pgfscope}%
\begin{pgfscope}%
\pgfpathrectangle{\pgfqpoint{0.329460in}{0.284240in}}{\pgfqpoint{1.989680in}{1.989680in}}%
\pgfusepath{clip}%
\pgfsetbuttcap%
\pgfsetroundjoin%
\definecolor{currentfill}{rgb}{0.279566,0.067836,0.391917}%
\pgfsetfillcolor{currentfill}%
\pgfsetlinewidth{0.000000pt}%
\definecolor{currentstroke}{rgb}{0.000000,0.000000,0.000000}%
\pgfsetstrokecolor{currentstroke}%
\pgfsetdash{}{0pt}%
\pgfpathmoveto{\pgfqpoint{1.587428in}{0.855455in}}%
\pgfpathlineto{\pgfqpoint{1.589177in}{0.850254in}}%
\pgfpathlineto{\pgfqpoint{1.590927in}{0.845204in}}%
\pgfpathlineto{\pgfqpoint{1.592680in}{0.840309in}}%
\pgfpathlineto{\pgfqpoint{1.594436in}{0.835573in}}%
\pgfpathlineto{\pgfqpoint{1.581017in}{0.831554in}}%
\pgfpathlineto{\pgfqpoint{1.567345in}{0.827764in}}%
\pgfpathlineto{\pgfqpoint{1.553434in}{0.824205in}}%
\pgfpathlineto{\pgfqpoint{1.539301in}{0.820884in}}%
\pgfpathlineto{\pgfqpoint{1.537940in}{0.825736in}}%
\pgfpathlineto{\pgfqpoint{1.536581in}{0.830748in}}%
\pgfpathlineto{\pgfqpoint{1.535225in}{0.835914in}}%
\pgfpathlineto{\pgfqpoint{1.533870in}{0.841232in}}%
\pgfpathlineto{\pgfqpoint{1.547598in}{0.844448in}}%
\pgfpathlineto{\pgfqpoint{1.561110in}{0.847893in}}%
\pgfpathlineto{\pgfqpoint{1.574392in}{0.851563in}}%
\pgfpathlineto{\pgfqpoint{1.587428in}{0.855455in}}%
\pgfpathclose%
\pgfusepath{fill}%
\end{pgfscope}%
\begin{pgfscope}%
\pgfpathrectangle{\pgfqpoint{0.329460in}{0.284240in}}{\pgfqpoint{1.989680in}{1.989680in}}%
\pgfusepath{clip}%
\pgfsetbuttcap%
\pgfsetroundjoin%
\definecolor{currentfill}{rgb}{0.974417,0.903590,0.130215}%
\pgfsetfillcolor{currentfill}%
\pgfsetlinewidth{0.000000pt}%
\definecolor{currentstroke}{rgb}{0.000000,0.000000,0.000000}%
\pgfsetstrokecolor{currentstroke}%
\pgfsetdash{}{0pt}%
\pgfpathmoveto{\pgfqpoint{1.324468in}{1.731420in}}%
\pgfpathlineto{\pgfqpoint{1.321132in}{1.730902in}}%
\pgfpathlineto{\pgfqpoint{1.317798in}{1.730265in}}%
\pgfpathlineto{\pgfqpoint{1.314464in}{1.729510in}}%
\pgfpathlineto{\pgfqpoint{1.311132in}{1.728638in}}%
\pgfpathlineto{\pgfqpoint{1.311560in}{1.729224in}}%
\pgfpathlineto{\pgfqpoint{1.312028in}{1.729803in}}%
\pgfpathlineto{\pgfqpoint{1.312535in}{1.730376in}}%
\pgfpathlineto{\pgfqpoint{1.313080in}{1.730940in}}%
\pgfpathlineto{\pgfqpoint{1.316249in}{1.731621in}}%
\pgfpathlineto{\pgfqpoint{1.319419in}{1.732184in}}%
\pgfpathlineto{\pgfqpoint{1.322591in}{1.732629in}}%
\pgfpathlineto{\pgfqpoint{1.325764in}{1.732955in}}%
\pgfpathlineto{\pgfqpoint{1.325401in}{1.732579in}}%
\pgfpathlineto{\pgfqpoint{1.325064in}{1.732197in}}%
\pgfpathlineto{\pgfqpoint{1.324753in}{1.731811in}}%
\pgfpathlineto{\pgfqpoint{1.324468in}{1.731420in}}%
\pgfpathclose%
\pgfusepath{fill}%
\end{pgfscope}%
\begin{pgfscope}%
\pgfpathrectangle{\pgfqpoint{0.329460in}{0.284240in}}{\pgfqpoint{1.989680in}{1.989680in}}%
\pgfusepath{clip}%
\pgfsetbuttcap%
\pgfsetroundjoin%
\definecolor{currentfill}{rgb}{0.855810,0.888601,0.097452}%
\pgfsetfillcolor{currentfill}%
\pgfsetlinewidth{0.000000pt}%
\definecolor{currentstroke}{rgb}{0.000000,0.000000,0.000000}%
\pgfsetstrokecolor{currentstroke}%
\pgfsetdash{}{0pt}%
\pgfpathmoveto{\pgfqpoint{1.435224in}{1.695564in}}%
\pgfpathlineto{\pgfqpoint{1.438713in}{1.692640in}}%
\pgfpathlineto{\pgfqpoint{1.442201in}{1.689606in}}%
\pgfpathlineto{\pgfqpoint{1.445688in}{1.686464in}}%
\pgfpathlineto{\pgfqpoint{1.449172in}{1.683214in}}%
\pgfpathlineto{\pgfqpoint{1.449020in}{1.681762in}}%
\pgfpathlineto{\pgfqpoint{1.448771in}{1.680312in}}%
\pgfpathlineto{\pgfqpoint{1.448423in}{1.678867in}}%
\pgfpathlineto{\pgfqpoint{1.447979in}{1.677427in}}%
\pgfpathlineto{\pgfqpoint{1.444533in}{1.680886in}}%
\pgfpathlineto{\pgfqpoint{1.441086in}{1.684237in}}%
\pgfpathlineto{\pgfqpoint{1.437638in}{1.687479in}}%
\pgfpathlineto{\pgfqpoint{1.434189in}{1.690612in}}%
\pgfpathlineto{\pgfqpoint{1.434573in}{1.691844in}}%
\pgfpathlineto{\pgfqpoint{1.434873in}{1.693081in}}%
\pgfpathlineto{\pgfqpoint{1.435091in}{1.694321in}}%
\pgfpathlineto{\pgfqpoint{1.435224in}{1.695564in}}%
\pgfpathclose%
\pgfusepath{fill}%
\end{pgfscope}%
\begin{pgfscope}%
\pgfpathrectangle{\pgfqpoint{0.329460in}{0.284240in}}{\pgfqpoint{1.989680in}{1.989680in}}%
\pgfusepath{clip}%
\pgfsetbuttcap%
\pgfsetroundjoin%
\definecolor{currentfill}{rgb}{0.993248,0.906157,0.143936}%
\pgfsetfillcolor{currentfill}%
\pgfsetlinewidth{0.000000pt}%
\definecolor{currentstroke}{rgb}{0.000000,0.000000,0.000000}%
\pgfsetstrokecolor{currentstroke}%
\pgfsetdash{}{0pt}%
\pgfpathmoveto{\pgfqpoint{1.360520in}{1.735189in}}%
\pgfpathlineto{\pgfqpoint{1.362850in}{1.735868in}}%
\pgfpathlineto{\pgfqpoint{1.365180in}{1.736427in}}%
\pgfpathlineto{\pgfqpoint{1.367507in}{1.736867in}}%
\pgfpathlineto{\pgfqpoint{1.369834in}{1.737187in}}%
\pgfpathlineto{\pgfqpoint{1.370486in}{1.736909in}}%
\pgfpathlineto{\pgfqpoint{1.371119in}{1.736621in}}%
\pgfpathlineto{\pgfqpoint{1.371732in}{1.736323in}}%
\pgfpathlineto{\pgfqpoint{1.372325in}{1.736017in}}%
\pgfpathlineto{\pgfqpoint{1.369687in}{1.735842in}}%
\pgfpathlineto{\pgfqpoint{1.367048in}{1.735549in}}%
\pgfpathlineto{\pgfqpoint{1.364408in}{1.735135in}}%
\pgfpathlineto{\pgfqpoint{1.361766in}{1.734603in}}%
\pgfpathlineto{\pgfqpoint{1.361469in}{1.734756in}}%
\pgfpathlineto{\pgfqpoint{1.361163in}{1.734905in}}%
\pgfpathlineto{\pgfqpoint{1.360846in}{1.735049in}}%
\pgfpathlineto{\pgfqpoint{1.360520in}{1.735189in}}%
\pgfpathclose%
\pgfusepath{fill}%
\end{pgfscope}%
\begin{pgfscope}%
\pgfpathrectangle{\pgfqpoint{0.329460in}{0.284240in}}{\pgfqpoint{1.989680in}{1.989680in}}%
\pgfusepath{clip}%
\pgfsetbuttcap%
\pgfsetroundjoin%
\definecolor{currentfill}{rgb}{0.263663,0.237631,0.518762}%
\pgfsetfillcolor{currentfill}%
\pgfsetlinewidth{0.000000pt}%
\definecolor{currentstroke}{rgb}{0.000000,0.000000,0.000000}%
\pgfsetstrokecolor{currentstroke}%
\pgfsetdash{}{0pt}%
\pgfpathmoveto{\pgfqpoint{1.552782in}{0.985660in}}%
\pgfpathlineto{\pgfqpoint{1.554504in}{0.978185in}}%
\pgfpathlineto{\pgfqpoint{1.556227in}{0.970791in}}%
\pgfpathlineto{\pgfqpoint{1.557951in}{0.963483in}}%
\pgfpathlineto{\pgfqpoint{1.559676in}{0.956262in}}%
\pgfpathlineto{\pgfqpoint{1.548157in}{0.952880in}}%
\pgfpathlineto{\pgfqpoint{1.536425in}{0.949690in}}%
\pgfpathlineto{\pgfqpoint{1.524491in}{0.946697in}}%
\pgfpathlineto{\pgfqpoint{1.512368in}{0.943903in}}%
\pgfpathlineto{\pgfqpoint{1.511033in}{0.951240in}}%
\pgfpathlineto{\pgfqpoint{1.509698in}{0.958664in}}%
\pgfpathlineto{\pgfqpoint{1.508363in}{0.966174in}}%
\pgfpathlineto{\pgfqpoint{1.507029in}{0.973764in}}%
\pgfpathlineto{\pgfqpoint{1.518753in}{0.976453in}}%
\pgfpathlineto{\pgfqpoint{1.530294in}{0.979335in}}%
\pgfpathlineto{\pgfqpoint{1.541641in}{0.982405in}}%
\pgfpathlineto{\pgfqpoint{1.552782in}{0.985660in}}%
\pgfpathclose%
\pgfusepath{fill}%
\end{pgfscope}%
\begin{pgfscope}%
\pgfpathrectangle{\pgfqpoint{0.329460in}{0.284240in}}{\pgfqpoint{1.989680in}{1.989680in}}%
\pgfusepath{clip}%
\pgfsetbuttcap%
\pgfsetroundjoin%
\definecolor{currentfill}{rgb}{0.133743,0.548535,0.553541}%
\pgfsetfillcolor{currentfill}%
\pgfsetlinewidth{0.000000pt}%
\definecolor{currentstroke}{rgb}{0.000000,0.000000,0.000000}%
\pgfsetstrokecolor{currentstroke}%
\pgfsetdash{}{0pt}%
\pgfpathmoveto{\pgfqpoint{1.556403in}{1.272245in}}%
\pgfpathlineto{\pgfqpoint{1.558811in}{1.264027in}}%
\pgfpathlineto{\pgfqpoint{1.561217in}{1.255800in}}%
\pgfpathlineto{\pgfqpoint{1.563622in}{1.247565in}}%
\pgfpathlineto{\pgfqpoint{1.566027in}{1.239327in}}%
\pgfpathlineto{\pgfqpoint{1.559062in}{1.235953in}}%
\pgfpathlineto{\pgfqpoint{1.551879in}{1.232691in}}%
\pgfpathlineto{\pgfqpoint{1.544485in}{1.229543in}}%
\pgfpathlineto{\pgfqpoint{1.536887in}{1.226513in}}%
\pgfpathlineto{\pgfqpoint{1.534805in}{1.234912in}}%
\pgfpathlineto{\pgfqpoint{1.532721in}{1.243307in}}%
\pgfpathlineto{\pgfqpoint{1.530637in}{1.251694in}}%
\pgfpathlineto{\pgfqpoint{1.528553in}{1.260071in}}%
\pgfpathlineto{\pgfqpoint{1.535814in}{1.262950in}}%
\pgfpathlineto{\pgfqpoint{1.542880in}{1.265940in}}%
\pgfpathlineto{\pgfqpoint{1.549746in}{1.269040in}}%
\pgfpathlineto{\pgfqpoint{1.556403in}{1.272245in}}%
\pgfpathclose%
\pgfusepath{fill}%
\end{pgfscope}%
\begin{pgfscope}%
\pgfpathrectangle{\pgfqpoint{0.329460in}{0.284240in}}{\pgfqpoint{1.989680in}{1.989680in}}%
\pgfusepath{clip}%
\pgfsetbuttcap%
\pgfsetroundjoin%
\definecolor{currentfill}{rgb}{0.993248,0.906157,0.143936}%
\pgfsetfillcolor{currentfill}%
\pgfsetlinewidth{0.000000pt}%
\definecolor{currentstroke}{rgb}{0.000000,0.000000,0.000000}%
\pgfsetstrokecolor{currentstroke}%
\pgfsetdash{}{0pt}%
\pgfpathmoveto{\pgfqpoint{1.340355in}{1.734463in}}%
\pgfpathlineto{\pgfqpoint{1.337649in}{1.734961in}}%
\pgfpathlineto{\pgfqpoint{1.334945in}{1.735339in}}%
\pgfpathlineto{\pgfqpoint{1.332242in}{1.735598in}}%
\pgfpathlineto{\pgfqpoint{1.329541in}{1.735738in}}%
\pgfpathlineto{\pgfqpoint{1.330115in}{1.736051in}}%
\pgfpathlineto{\pgfqpoint{1.330710in}{1.736357in}}%
\pgfpathlineto{\pgfqpoint{1.331325in}{1.736653in}}%
\pgfpathlineto{\pgfqpoint{1.331960in}{1.736940in}}%
\pgfpathlineto{\pgfqpoint{1.334360in}{1.736651in}}%
\pgfpathlineto{\pgfqpoint{1.336760in}{1.736242in}}%
\pgfpathlineto{\pgfqpoint{1.339162in}{1.735713in}}%
\pgfpathlineto{\pgfqpoint{1.341565in}{1.735065in}}%
\pgfpathlineto{\pgfqpoint{1.341247in}{1.734921in}}%
\pgfpathlineto{\pgfqpoint{1.340939in}{1.734773in}}%
\pgfpathlineto{\pgfqpoint{1.340642in}{1.734620in}}%
\pgfpathlineto{\pgfqpoint{1.340355in}{1.734463in}}%
\pgfpathclose%
\pgfusepath{fill}%
\end{pgfscope}%
\begin{pgfscope}%
\pgfpathrectangle{\pgfqpoint{0.329460in}{0.284240in}}{\pgfqpoint{1.989680in}{1.989680in}}%
\pgfusepath{clip}%
\pgfsetbuttcap%
\pgfsetroundjoin%
\definecolor{currentfill}{rgb}{0.134692,0.658636,0.517649}%
\pgfsetfillcolor{currentfill}%
\pgfsetlinewidth{0.000000pt}%
\definecolor{currentstroke}{rgb}{0.000000,0.000000,0.000000}%
\pgfsetstrokecolor{currentstroke}%
\pgfsetdash{}{0pt}%
\pgfpathmoveto{\pgfqpoint{1.548466in}{1.380948in}}%
\pgfpathlineto{\pgfqpoint{1.551170in}{1.373201in}}%
\pgfpathlineto{\pgfqpoint{1.553872in}{1.365418in}}%
\pgfpathlineto{\pgfqpoint{1.556573in}{1.357601in}}%
\pgfpathlineto{\pgfqpoint{1.559273in}{1.349750in}}%
\pgfpathlineto{\pgfqpoint{1.554042in}{1.346528in}}%
\pgfpathlineto{\pgfqpoint{1.548600in}{1.343388in}}%
\pgfpathlineto{\pgfqpoint{1.542954in}{1.340334in}}%
\pgfpathlineto{\pgfqpoint{1.537108in}{1.337368in}}%
\pgfpathlineto{\pgfqpoint{1.534691in}{1.345395in}}%
\pgfpathlineto{\pgfqpoint{1.532273in}{1.353389in}}%
\pgfpathlineto{\pgfqpoint{1.529854in}{1.361348in}}%
\pgfpathlineto{\pgfqpoint{1.527433in}{1.369269in}}%
\pgfpathlineto{\pgfqpoint{1.532980in}{1.372066in}}%
\pgfpathlineto{\pgfqpoint{1.538338in}{1.374947in}}%
\pgfpathlineto{\pgfqpoint{1.543502in}{1.377908in}}%
\pgfpathlineto{\pgfqpoint{1.548466in}{1.380948in}}%
\pgfpathclose%
\pgfusepath{fill}%
\end{pgfscope}%
\begin{pgfscope}%
\pgfpathrectangle{\pgfqpoint{0.329460in}{0.284240in}}{\pgfqpoint{1.989680in}{1.989680in}}%
\pgfusepath{clip}%
\pgfsetbuttcap%
\pgfsetroundjoin%
\definecolor{currentfill}{rgb}{0.195860,0.395433,0.555276}%
\pgfsetfillcolor{currentfill}%
\pgfsetlinewidth{0.000000pt}%
\definecolor{currentstroke}{rgb}{0.000000,0.000000,0.000000}%
\pgfsetstrokecolor{currentstroke}%
\pgfsetdash{}{0pt}%
\pgfpathmoveto{\pgfqpoint{1.185661in}{1.111337in}}%
\pgfpathlineto{\pgfqpoint{1.184025in}{1.102951in}}%
\pgfpathlineto{\pgfqpoint{1.182390in}{1.094598in}}%
\pgfpathlineto{\pgfqpoint{1.180755in}{1.086281in}}%
\pgfpathlineto{\pgfqpoint{1.179120in}{1.078003in}}%
\pgfpathlineto{\pgfqpoint{1.169132in}{1.080901in}}%
\pgfpathlineto{\pgfqpoint{1.159339in}{1.083959in}}%
\pgfpathlineto{\pgfqpoint{1.149752in}{1.087173in}}%
\pgfpathlineto{\pgfqpoint{1.140380in}{1.090541in}}%
\pgfpathlineto{\pgfqpoint{1.142379in}{1.098685in}}%
\pgfpathlineto{\pgfqpoint{1.144379in}{1.106867in}}%
\pgfpathlineto{\pgfqpoint{1.146379in}{1.115085in}}%
\pgfpathlineto{\pgfqpoint{1.148378in}{1.123337in}}%
\pgfpathlineto{\pgfqpoint{1.157399in}{1.120113in}}%
\pgfpathlineto{\pgfqpoint{1.166625in}{1.117036in}}%
\pgfpathlineto{\pgfqpoint{1.176049in}{1.114110in}}%
\pgfpathlineto{\pgfqpoint{1.185661in}{1.111337in}}%
\pgfpathclose%
\pgfusepath{fill}%
\end{pgfscope}%
\begin{pgfscope}%
\pgfpathrectangle{\pgfqpoint{0.329460in}{0.284240in}}{\pgfqpoint{1.989680in}{1.989680in}}%
\pgfusepath{clip}%
\pgfsetbuttcap%
\pgfsetroundjoin%
\definecolor{currentfill}{rgb}{0.283072,0.130895,0.449241}%
\pgfsetfillcolor{currentfill}%
\pgfsetlinewidth{0.000000pt}%
\definecolor{currentstroke}{rgb}{0.000000,0.000000,0.000000}%
\pgfsetstrokecolor{currentstroke}%
\pgfsetdash{}{0pt}%
\pgfpathmoveto{\pgfqpoint{1.190938in}{0.886285in}}%
\pgfpathlineto{\pgfqpoint{1.189685in}{0.879869in}}%
\pgfpathlineto{\pgfqpoint{1.188431in}{0.873574in}}%
\pgfpathlineto{\pgfqpoint{1.187175in}{0.867405in}}%
\pgfpathlineto{\pgfqpoint{1.185918in}{0.861365in}}%
\pgfpathlineto{\pgfqpoint{1.172419in}{0.864275in}}%
\pgfpathlineto{\pgfqpoint{1.159117in}{0.867410in}}%
\pgfpathlineto{\pgfqpoint{1.146025in}{0.870766in}}%
\pgfpathlineto{\pgfqpoint{1.133159in}{0.874340in}}%
\pgfpathlineto{\pgfqpoint{1.134813in}{0.880270in}}%
\pgfpathlineto{\pgfqpoint{1.136466in}{0.886329in}}%
\pgfpathlineto{\pgfqpoint{1.138118in}{0.892514in}}%
\pgfpathlineto{\pgfqpoint{1.139768in}{0.898820in}}%
\pgfpathlineto{\pgfqpoint{1.152248in}{0.895367in}}%
\pgfpathlineto{\pgfqpoint{1.164945in}{0.892125in}}%
\pgfpathlineto{\pgfqpoint{1.177847in}{0.889096in}}%
\pgfpathlineto{\pgfqpoint{1.190938in}{0.886285in}}%
\pgfpathclose%
\pgfusepath{fill}%
\end{pgfscope}%
\begin{pgfscope}%
\pgfpathrectangle{\pgfqpoint{0.329460in}{0.284240in}}{\pgfqpoint{1.989680in}{1.989680in}}%
\pgfusepath{clip}%
\pgfsetbuttcap%
\pgfsetroundjoin%
\definecolor{currentfill}{rgb}{0.636902,0.856542,0.216620}%
\pgfsetfillcolor{currentfill}%
\pgfsetlinewidth{0.000000pt}%
\definecolor{currentstroke}{rgb}{0.000000,0.000000,0.000000}%
\pgfsetstrokecolor{currentstroke}%
\pgfsetdash{}{0pt}%
\pgfpathmoveto{\pgfqpoint{1.485349in}{1.619701in}}%
\pgfpathlineto{\pgfqpoint{1.488682in}{1.614725in}}%
\pgfpathlineto{\pgfqpoint{1.492012in}{1.609655in}}%
\pgfpathlineto{\pgfqpoint{1.495342in}{1.604494in}}%
\pgfpathlineto{\pgfqpoint{1.498669in}{1.599244in}}%
\pgfpathlineto{\pgfqpoint{1.497241in}{1.597036in}}%
\pgfpathlineto{\pgfqpoint{1.495666in}{1.594849in}}%
\pgfpathlineto{\pgfqpoint{1.493945in}{1.592687in}}%
\pgfpathlineto{\pgfqpoint{1.492080in}{1.590551in}}%
\pgfpathlineto{\pgfqpoint{1.488897in}{1.596005in}}%
\pgfpathlineto{\pgfqpoint{1.485712in}{1.601370in}}%
\pgfpathlineto{\pgfqpoint{1.482526in}{1.606642in}}%
\pgfpathlineto{\pgfqpoint{1.479338in}{1.611822in}}%
\pgfpathlineto{\pgfqpoint{1.481039in}{1.613758in}}%
\pgfpathlineto{\pgfqpoint{1.482609in}{1.615718in}}%
\pgfpathlineto{\pgfqpoint{1.484046in}{1.617700in}}%
\pgfpathlineto{\pgfqpoint{1.485349in}{1.619701in}}%
\pgfpathclose%
\pgfusepath{fill}%
\end{pgfscope}%
\begin{pgfscope}%
\pgfpathrectangle{\pgfqpoint{0.329460in}{0.284240in}}{\pgfqpoint{1.989680in}{1.989680in}}%
\pgfusepath{clip}%
\pgfsetbuttcap%
\pgfsetroundjoin%
\definecolor{currentfill}{rgb}{0.280255,0.165693,0.476498}%
\pgfsetfillcolor{currentfill}%
\pgfsetlinewidth{0.000000pt}%
\definecolor{currentstroke}{rgb}{0.000000,0.000000,0.000000}%
\pgfsetstrokecolor{currentstroke}%
\pgfsetdash{}{0pt}%
\pgfpathmoveto{\pgfqpoint{1.195941in}{0.913101in}}%
\pgfpathlineto{\pgfqpoint{1.194691in}{0.906232in}}%
\pgfpathlineto{\pgfqpoint{1.193441in}{0.899470in}}%
\pgfpathlineto{\pgfqpoint{1.192190in}{0.892820in}}%
\pgfpathlineto{\pgfqpoint{1.190938in}{0.886285in}}%
\pgfpathlineto{\pgfqpoint{1.177847in}{0.889096in}}%
\pgfpathlineto{\pgfqpoint{1.164945in}{0.892125in}}%
\pgfpathlineto{\pgfqpoint{1.152248in}{0.895367in}}%
\pgfpathlineto{\pgfqpoint{1.139768in}{0.898820in}}%
\pgfpathlineto{\pgfqpoint{1.141417in}{0.905245in}}%
\pgfpathlineto{\pgfqpoint{1.143065in}{0.911785in}}%
\pgfpathlineto{\pgfqpoint{1.144711in}{0.918436in}}%
\pgfpathlineto{\pgfqpoint{1.146356in}{0.925196in}}%
\pgfpathlineto{\pgfqpoint{1.158450in}{0.921864in}}%
\pgfpathlineto{\pgfqpoint{1.170755in}{0.918735in}}%
\pgfpathlineto{\pgfqpoint{1.183256in}{0.915813in}}%
\pgfpathlineto{\pgfqpoint{1.195941in}{0.913101in}}%
\pgfpathclose%
\pgfusepath{fill}%
\end{pgfscope}%
\begin{pgfscope}%
\pgfpathrectangle{\pgfqpoint{0.329460in}{0.284240in}}{\pgfqpoint{1.989680in}{1.989680in}}%
\pgfusepath{clip}%
\pgfsetbuttcap%
\pgfsetroundjoin%
\definecolor{currentfill}{rgb}{0.955300,0.901065,0.118128}%
\pgfsetfillcolor{currentfill}%
\pgfsetlinewidth{0.000000pt}%
\definecolor{currentstroke}{rgb}{0.000000,0.000000,0.000000}%
\pgfsetstrokecolor{currentstroke}%
\pgfsetdash{}{0pt}%
\pgfpathmoveto{\pgfqpoint{1.392320in}{1.726778in}}%
\pgfpathlineto{\pgfqpoint{1.395741in}{1.725633in}}%
\pgfpathlineto{\pgfqpoint{1.399162in}{1.724371in}}%
\pgfpathlineto{\pgfqpoint{1.402581in}{1.722992in}}%
\pgfpathlineto{\pgfqpoint{1.405999in}{1.721498in}}%
\pgfpathlineto{\pgfqpoint{1.406350in}{1.720688in}}%
\pgfpathlineto{\pgfqpoint{1.406647in}{1.719873in}}%
\pgfpathlineto{\pgfqpoint{1.406889in}{1.719054in}}%
\pgfpathlineto{\pgfqpoint{1.407075in}{1.718231in}}%
\pgfpathlineto{\pgfqpoint{1.403588in}{1.719930in}}%
\pgfpathlineto{\pgfqpoint{1.400100in}{1.721513in}}%
\pgfpathlineto{\pgfqpoint{1.396612in}{1.722980in}}%
\pgfpathlineto{\pgfqpoint{1.393122in}{1.724329in}}%
\pgfpathlineto{\pgfqpoint{1.392984in}{1.724945in}}%
\pgfpathlineto{\pgfqpoint{1.392804in}{1.725559in}}%
\pgfpathlineto{\pgfqpoint{1.392582in}{1.726170in}}%
\pgfpathlineto{\pgfqpoint{1.392320in}{1.726778in}}%
\pgfpathclose%
\pgfusepath{fill}%
\end{pgfscope}%
\begin{pgfscope}%
\pgfpathrectangle{\pgfqpoint{0.329460in}{0.284240in}}{\pgfqpoint{1.989680in}{1.989680in}}%
\pgfusepath{clip}%
\pgfsetbuttcap%
\pgfsetroundjoin%
\definecolor{currentfill}{rgb}{0.412913,0.803041,0.357269}%
\pgfsetfillcolor{currentfill}%
\pgfsetlinewidth{0.000000pt}%
\definecolor{currentstroke}{rgb}{0.000000,0.000000,0.000000}%
\pgfsetstrokecolor{currentstroke}%
\pgfsetdash{}{0pt}%
\pgfpathmoveto{\pgfqpoint{1.197982in}{1.531945in}}%
\pgfpathlineto{\pgfqpoint{1.195058in}{1.525505in}}%
\pgfpathlineto{\pgfqpoint{1.192135in}{1.518990in}}%
\pgfpathlineto{\pgfqpoint{1.189213in}{1.512401in}}%
\pgfpathlineto{\pgfqpoint{1.186293in}{1.505739in}}%
\pgfpathlineto{\pgfqpoint{1.183077in}{1.508283in}}%
\pgfpathlineto{\pgfqpoint{1.180033in}{1.510873in}}%
\pgfpathlineto{\pgfqpoint{1.177163in}{1.513508in}}%
\pgfpathlineto{\pgfqpoint{1.174470in}{1.516183in}}%
\pgfpathlineto{\pgfqpoint{1.177594in}{1.522649in}}%
\pgfpathlineto{\pgfqpoint{1.180720in}{1.529042in}}%
\pgfpathlineto{\pgfqpoint{1.183847in}{1.535363in}}%
\pgfpathlineto{\pgfqpoint{1.186977in}{1.541608in}}%
\pgfpathlineto{\pgfqpoint{1.189484in}{1.539132in}}%
\pgfpathlineto{\pgfqpoint{1.192156in}{1.536695in}}%
\pgfpathlineto{\pgfqpoint{1.194989in}{1.534298in}}%
\pgfpathlineto{\pgfqpoint{1.197982in}{1.531945in}}%
\pgfpathclose%
\pgfusepath{fill}%
\end{pgfscope}%
\begin{pgfscope}%
\pgfpathrectangle{\pgfqpoint{0.329460in}{0.284240in}}{\pgfqpoint{1.989680in}{1.989680in}}%
\pgfusepath{clip}%
\pgfsetbuttcap%
\pgfsetroundjoin%
\definecolor{currentfill}{rgb}{0.993248,0.906157,0.143936}%
\pgfsetfillcolor{currentfill}%
\pgfsetlinewidth{0.000000pt}%
\definecolor{currentstroke}{rgb}{0.000000,0.000000,0.000000}%
\pgfsetstrokecolor{currentstroke}%
\pgfsetdash{}{0pt}%
\pgfpathmoveto{\pgfqpoint{1.361766in}{1.734603in}}%
\pgfpathlineto{\pgfqpoint{1.364408in}{1.735135in}}%
\pgfpathlineto{\pgfqpoint{1.367048in}{1.735549in}}%
\pgfpathlineto{\pgfqpoint{1.369687in}{1.735842in}}%
\pgfpathlineto{\pgfqpoint{1.372325in}{1.736017in}}%
\pgfpathlineto{\pgfqpoint{1.372897in}{1.735702in}}%
\pgfpathlineto{\pgfqpoint{1.373447in}{1.735379in}}%
\pgfpathlineto{\pgfqpoint{1.373975in}{1.735048in}}%
\pgfpathlineto{\pgfqpoint{1.374481in}{1.734709in}}%
\pgfpathlineto{\pgfqpoint{1.371574in}{1.734698in}}%
\pgfpathlineto{\pgfqpoint{1.368665in}{1.734568in}}%
\pgfpathlineto{\pgfqpoint{1.365755in}{1.734318in}}%
\pgfpathlineto{\pgfqpoint{1.362843in}{1.733948in}}%
\pgfpathlineto{\pgfqpoint{1.362591in}{1.734118in}}%
\pgfpathlineto{\pgfqpoint{1.362327in}{1.734283in}}%
\pgfpathlineto{\pgfqpoint{1.362052in}{1.734445in}}%
\pgfpathlineto{\pgfqpoint{1.361766in}{1.734603in}}%
\pgfpathclose%
\pgfusepath{fill}%
\end{pgfscope}%
\begin{pgfscope}%
\pgfpathrectangle{\pgfqpoint{0.329460in}{0.284240in}}{\pgfqpoint{1.989680in}{1.989680in}}%
\pgfusepath{clip}%
\pgfsetbuttcap%
\pgfsetroundjoin%
\definecolor{currentfill}{rgb}{0.855810,0.888601,0.097452}%
\pgfsetfillcolor{currentfill}%
\pgfsetlinewidth{0.000000pt}%
\definecolor{currentstroke}{rgb}{0.000000,0.000000,0.000000}%
\pgfsetstrokecolor{currentstroke}%
\pgfsetdash{}{0pt}%
\pgfpathmoveto{\pgfqpoint{1.268598in}{1.689522in}}%
\pgfpathlineto{\pgfqpoint{1.265165in}{1.686343in}}%
\pgfpathlineto{\pgfqpoint{1.261733in}{1.683055in}}%
\pgfpathlineto{\pgfqpoint{1.258302in}{1.679658in}}%
\pgfpathlineto{\pgfqpoint{1.254873in}{1.676153in}}%
\pgfpathlineto{\pgfqpoint{1.254342in}{1.677587in}}%
\pgfpathlineto{\pgfqpoint{1.253908in}{1.679027in}}%
\pgfpathlineto{\pgfqpoint{1.253572in}{1.680473in}}%
\pgfpathlineto{\pgfqpoint{1.253333in}{1.681923in}}%
\pgfpathlineto{\pgfqpoint{1.256814in}{1.685220in}}%
\pgfpathlineto{\pgfqpoint{1.260296in}{1.688408in}}%
\pgfpathlineto{\pgfqpoint{1.263780in}{1.691488in}}%
\pgfpathlineto{\pgfqpoint{1.267266in}{1.694459in}}%
\pgfpathlineto{\pgfqpoint{1.267473in}{1.693218in}}%
\pgfpathlineto{\pgfqpoint{1.267765in}{1.691981in}}%
\pgfpathlineto{\pgfqpoint{1.268140in}{1.690748in}}%
\pgfpathlineto{\pgfqpoint{1.268598in}{1.689522in}}%
\pgfpathclose%
\pgfusepath{fill}%
\end{pgfscope}%
\begin{pgfscope}%
\pgfpathrectangle{\pgfqpoint{0.329460in}{0.284240in}}{\pgfqpoint{1.989680in}{1.989680in}}%
\pgfusepath{clip}%
\pgfsetbuttcap%
\pgfsetroundjoin%
\definecolor{currentfill}{rgb}{0.935904,0.898570,0.108131}%
\pgfsetfillcolor{currentfill}%
\pgfsetlinewidth{0.000000pt}%
\definecolor{currentstroke}{rgb}{0.000000,0.000000,0.000000}%
\pgfsetstrokecolor{currentstroke}%
\pgfsetdash{}{0pt}%
\pgfpathmoveto{\pgfqpoint{1.407075in}{1.718231in}}%
\pgfpathlineto{\pgfqpoint{1.410560in}{1.716417in}}%
\pgfpathlineto{\pgfqpoint{1.414045in}{1.714487in}}%
\pgfpathlineto{\pgfqpoint{1.417527in}{1.712444in}}%
\pgfpathlineto{\pgfqpoint{1.421009in}{1.710286in}}%
\pgfpathlineto{\pgfqpoint{1.421174in}{1.709254in}}%
\pgfpathlineto{\pgfqpoint{1.421270in}{1.708220in}}%
\pgfpathlineto{\pgfqpoint{1.421296in}{1.707185in}}%
\pgfpathlineto{\pgfqpoint{1.421253in}{1.706150in}}%
\pgfpathlineto{\pgfqpoint{1.417757in}{1.708515in}}%
\pgfpathlineto{\pgfqpoint{1.414260in}{1.710767in}}%
\pgfpathlineto{\pgfqpoint{1.410761in}{1.712904in}}%
\pgfpathlineto{\pgfqpoint{1.407262in}{1.714926in}}%
\pgfpathlineto{\pgfqpoint{1.407299in}{1.715753in}}%
\pgfpathlineto{\pgfqpoint{1.407280in}{1.716580in}}%
\pgfpathlineto{\pgfqpoint{1.407205in}{1.717407in}}%
\pgfpathlineto{\pgfqpoint{1.407075in}{1.718231in}}%
\pgfpathclose%
\pgfusepath{fill}%
\end{pgfscope}%
\begin{pgfscope}%
\pgfpathrectangle{\pgfqpoint{0.329460in}{0.284240in}}{\pgfqpoint{1.989680in}{1.989680in}}%
\pgfusepath{clip}%
\pgfsetbuttcap%
\pgfsetroundjoin%
\definecolor{currentfill}{rgb}{0.179019,0.433756,0.557430}%
\pgfsetfillcolor{currentfill}%
\pgfsetlinewidth{0.000000pt}%
\definecolor{currentstroke}{rgb}{0.000000,0.000000,0.000000}%
\pgfsetstrokecolor{currentstroke}%
\pgfsetdash{}{0pt}%
\pgfpathmoveto{\pgfqpoint{1.553526in}{1.159470in}}%
\pgfpathlineto{\pgfqpoint{1.555603in}{1.151147in}}%
\pgfpathlineto{\pgfqpoint{1.557680in}{1.142847in}}%
\pgfpathlineto{\pgfqpoint{1.559757in}{1.134571in}}%
\pgfpathlineto{\pgfqpoint{1.561834in}{1.126323in}}%
\pgfpathlineto{\pgfqpoint{1.553005in}{1.122971in}}%
\pgfpathlineto{\pgfqpoint{1.543961in}{1.119764in}}%
\pgfpathlineto{\pgfqpoint{1.534712in}{1.116703in}}%
\pgfpathlineto{\pgfqpoint{1.525267in}{1.113794in}}%
\pgfpathlineto{\pgfqpoint{1.523548in}{1.122182in}}%
\pgfpathlineto{\pgfqpoint{1.521828in}{1.130597in}}%
\pgfpathlineto{\pgfqpoint{1.520108in}{1.139037in}}%
\pgfpathlineto{\pgfqpoint{1.518388in}{1.147498in}}%
\pgfpathlineto{\pgfqpoint{1.527463in}{1.150278in}}%
\pgfpathlineto{\pgfqpoint{1.536351in}{1.153203in}}%
\pgfpathlineto{\pgfqpoint{1.545041in}{1.156267in}}%
\pgfpathlineto{\pgfqpoint{1.553526in}{1.159470in}}%
\pgfpathclose%
\pgfusepath{fill}%
\end{pgfscope}%
\begin{pgfscope}%
\pgfpathrectangle{\pgfqpoint{0.329460in}{0.284240in}}{\pgfqpoint{1.989680in}{1.989680in}}%
\pgfusepath{clip}%
\pgfsetbuttcap%
\pgfsetroundjoin%
\definecolor{currentfill}{rgb}{0.281477,0.755203,0.432552}%
\pgfsetfillcolor{currentfill}%
\pgfsetlinewidth{0.000000pt}%
\definecolor{currentstroke}{rgb}{0.000000,0.000000,0.000000}%
\pgfsetstrokecolor{currentstroke}%
\pgfsetdash{}{0pt}%
\pgfpathmoveto{\pgfqpoint{1.530812in}{1.480838in}}%
\pgfpathlineto{\pgfqpoint{1.533773in}{1.473886in}}%
\pgfpathlineto{\pgfqpoint{1.536733in}{1.466873in}}%
\pgfpathlineto{\pgfqpoint{1.539692in}{1.459800in}}%
\pgfpathlineto{\pgfqpoint{1.542648in}{1.452671in}}%
\pgfpathlineto{\pgfqpoint{1.538967in}{1.449747in}}%
\pgfpathlineto{\pgfqpoint{1.535094in}{1.446880in}}%
\pgfpathlineto{\pgfqpoint{1.531032in}{1.444073in}}%
\pgfpathlineto{\pgfqpoint{1.526784in}{1.441329in}}%
\pgfpathlineto{\pgfqpoint{1.524068in}{1.448647in}}%
\pgfpathlineto{\pgfqpoint{1.521350in}{1.455908in}}%
\pgfpathlineto{\pgfqpoint{1.518631in}{1.463109in}}%
\pgfpathlineto{\pgfqpoint{1.515910in}{1.470248in}}%
\pgfpathlineto{\pgfqpoint{1.519899in}{1.472810in}}%
\pgfpathlineto{\pgfqpoint{1.523715in}{1.475431in}}%
\pgfpathlineto{\pgfqpoint{1.527353in}{1.478108in}}%
\pgfpathlineto{\pgfqpoint{1.530812in}{1.480838in}}%
\pgfpathclose%
\pgfusepath{fill}%
\end{pgfscope}%
\begin{pgfscope}%
\pgfpathrectangle{\pgfqpoint{0.329460in}{0.284240in}}{\pgfqpoint{1.989680in}{1.989680in}}%
\pgfusepath{clip}%
\pgfsetbuttcap%
\pgfsetroundjoin%
\definecolor{currentfill}{rgb}{0.282327,0.094955,0.417331}%
\pgfsetfillcolor{currentfill}%
\pgfsetlinewidth{0.000000pt}%
\definecolor{currentstroke}{rgb}{0.000000,0.000000,0.000000}%
\pgfsetstrokecolor{currentstroke}%
\pgfsetdash{}{0pt}%
\pgfpathmoveto{\pgfqpoint{1.185918in}{0.861365in}}%
\pgfpathlineto{\pgfqpoint{1.184660in}{0.855458in}}%
\pgfpathlineto{\pgfqpoint{1.183401in}{0.849687in}}%
\pgfpathlineto{\pgfqpoint{1.182140in}{0.844056in}}%
\pgfpathlineto{\pgfqpoint{1.180878in}{0.838569in}}%
\pgfpathlineto{\pgfqpoint{1.166970in}{0.841578in}}%
\pgfpathlineto{\pgfqpoint{1.153265in}{0.844819in}}%
\pgfpathlineto{\pgfqpoint{1.139777in}{0.848290in}}%
\pgfpathlineto{\pgfqpoint{1.126523in}{0.851985in}}%
\pgfpathlineto{\pgfqpoint{1.128184in}{0.857362in}}%
\pgfpathlineto{\pgfqpoint{1.129844in}{0.862883in}}%
\pgfpathlineto{\pgfqpoint{1.131502in}{0.868543in}}%
\pgfpathlineto{\pgfqpoint{1.133159in}{0.874340in}}%
\pgfpathlineto{\pgfqpoint{1.146025in}{0.870766in}}%
\pgfpathlineto{\pgfqpoint{1.159117in}{0.867410in}}%
\pgfpathlineto{\pgfqpoint{1.172419in}{0.864275in}}%
\pgfpathlineto{\pgfqpoint{1.185918in}{0.861365in}}%
\pgfpathclose%
\pgfusepath{fill}%
\end{pgfscope}%
\begin{pgfscope}%
\pgfpathrectangle{\pgfqpoint{0.329460in}{0.284240in}}{\pgfqpoint{1.989680in}{1.989680in}}%
\pgfusepath{clip}%
\pgfsetbuttcap%
\pgfsetroundjoin%
\definecolor{currentfill}{rgb}{0.993248,0.906157,0.143936}%
\pgfsetfillcolor{currentfill}%
\pgfsetlinewidth{0.000000pt}%
\definecolor{currentstroke}{rgb}{0.000000,0.000000,0.000000}%
\pgfsetstrokecolor{currentstroke}%
\pgfsetdash{}{0pt}%
\pgfpathmoveto{\pgfqpoint{1.339317in}{1.733794in}}%
\pgfpathlineto{\pgfqpoint{1.336352in}{1.734126in}}%
\pgfpathlineto{\pgfqpoint{1.333388in}{1.734337in}}%
\pgfpathlineto{\pgfqpoint{1.330425in}{1.734429in}}%
\pgfpathlineto{\pgfqpoint{1.327464in}{1.734402in}}%
\pgfpathlineto{\pgfqpoint{1.327949in}{1.734747in}}%
\pgfpathlineto{\pgfqpoint{1.328457in}{1.735085in}}%
\pgfpathlineto{\pgfqpoint{1.328988in}{1.735415in}}%
\pgfpathlineto{\pgfqpoint{1.329541in}{1.735738in}}%
\pgfpathlineto{\pgfqpoint{1.332242in}{1.735598in}}%
\pgfpathlineto{\pgfqpoint{1.334945in}{1.735339in}}%
\pgfpathlineto{\pgfqpoint{1.337649in}{1.734961in}}%
\pgfpathlineto{\pgfqpoint{1.340355in}{1.734463in}}%
\pgfpathlineto{\pgfqpoint{1.340078in}{1.734301in}}%
\pgfpathlineto{\pgfqpoint{1.339813in}{1.734136in}}%
\pgfpathlineto{\pgfqpoint{1.339559in}{1.733967in}}%
\pgfpathlineto{\pgfqpoint{1.339317in}{1.733794in}}%
\pgfpathclose%
\pgfusepath{fill}%
\end{pgfscope}%
\begin{pgfscope}%
\pgfpathrectangle{\pgfqpoint{0.329460in}{0.284240in}}{\pgfqpoint{1.989680in}{1.989680in}}%
\pgfusepath{clip}%
\pgfsetbuttcap%
\pgfsetroundjoin%
\definecolor{currentfill}{rgb}{0.762373,0.876424,0.137064}%
\pgfsetfillcolor{currentfill}%
\pgfsetlinewidth{0.000000pt}%
\definecolor{currentstroke}{rgb}{0.000000,0.000000,0.000000}%
\pgfsetstrokecolor{currentstroke}%
\pgfsetdash{}{0pt}%
\pgfpathmoveto{\pgfqpoint{1.244674in}{1.654614in}}%
\pgfpathlineto{\pgfqpoint{1.241357in}{1.650378in}}%
\pgfpathlineto{\pgfqpoint{1.238041in}{1.646039in}}%
\pgfpathlineto{\pgfqpoint{1.234727in}{1.641599in}}%
\pgfpathlineto{\pgfqpoint{1.231414in}{1.637059in}}%
\pgfpathlineto{\pgfqpoint{1.230250in}{1.638858in}}%
\pgfpathlineto{\pgfqpoint{1.229207in}{1.640672in}}%
\pgfpathlineto{\pgfqpoint{1.228287in}{1.642501in}}%
\pgfpathlineto{\pgfqpoint{1.227491in}{1.644342in}}%
\pgfpathlineto{\pgfqpoint{1.230908in}{1.648675in}}%
\pgfpathlineto{\pgfqpoint{1.234327in}{1.652909in}}%
\pgfpathlineto{\pgfqpoint{1.237747in}{1.657041in}}%
\pgfpathlineto{\pgfqpoint{1.241169in}{1.661072in}}%
\pgfpathlineto{\pgfqpoint{1.241882in}{1.659439in}}%
\pgfpathlineto{\pgfqpoint{1.242704in}{1.657818in}}%
\pgfpathlineto{\pgfqpoint{1.243635in}{1.656209in}}%
\pgfpathlineto{\pgfqpoint{1.244674in}{1.654614in}}%
\pgfpathclose%
\pgfusepath{fill}%
\end{pgfscope}%
\begin{pgfscope}%
\pgfpathrectangle{\pgfqpoint{0.329460in}{0.284240in}}{\pgfqpoint{1.989680in}{1.989680in}}%
\pgfusepath{clip}%
\pgfsetbuttcap%
\pgfsetroundjoin%
\definecolor{currentfill}{rgb}{0.274128,0.199721,0.498911}%
\pgfsetfillcolor{currentfill}%
\pgfsetlinewidth{0.000000pt}%
\definecolor{currentstroke}{rgb}{0.000000,0.000000,0.000000}%
\pgfsetstrokecolor{currentstroke}%
\pgfsetdash{}{0pt}%
\pgfpathmoveto{\pgfqpoint{1.200930in}{0.941590in}}%
\pgfpathlineto{\pgfqpoint{1.199684in}{0.934323in}}%
\pgfpathlineto{\pgfqpoint{1.198437in}{0.927150in}}%
\pgfpathlineto{\pgfqpoint{1.197189in}{0.920075in}}%
\pgfpathlineto{\pgfqpoint{1.195941in}{0.913101in}}%
\pgfpathlineto{\pgfqpoint{1.183256in}{0.915813in}}%
\pgfpathlineto{\pgfqpoint{1.170755in}{0.918735in}}%
\pgfpathlineto{\pgfqpoint{1.158450in}{0.921864in}}%
\pgfpathlineto{\pgfqpoint{1.146356in}{0.925196in}}%
\pgfpathlineto{\pgfqpoint{1.148000in}{0.932060in}}%
\pgfpathlineto{\pgfqpoint{1.149644in}{0.939025in}}%
\pgfpathlineto{\pgfqpoint{1.151286in}{0.946088in}}%
\pgfpathlineto{\pgfqpoint{1.152927in}{0.953246in}}%
\pgfpathlineto{\pgfqpoint{1.164637in}{0.950035in}}%
\pgfpathlineto{\pgfqpoint{1.176549in}{0.947020in}}%
\pgfpathlineto{\pgfqpoint{1.188651in}{0.944204in}}%
\pgfpathlineto{\pgfqpoint{1.200930in}{0.941590in}}%
\pgfpathclose%
\pgfusepath{fill}%
\end{pgfscope}%
\begin{pgfscope}%
\pgfpathrectangle{\pgfqpoint{0.329460in}{0.284240in}}{\pgfqpoint{1.989680in}{1.989680in}}%
\pgfusepath{clip}%
\pgfsetbuttcap%
\pgfsetroundjoin%
\definecolor{currentfill}{rgb}{0.955300,0.901065,0.118128}%
\pgfsetfillcolor{currentfill}%
\pgfsetlinewidth{0.000000pt}%
\definecolor{currentstroke}{rgb}{0.000000,0.000000,0.000000}%
\pgfsetstrokecolor{currentstroke}%
\pgfsetdash{}{0pt}%
\pgfpathmoveto{\pgfqpoint{1.309166in}{1.723780in}}%
\pgfpathlineto{\pgfqpoint{1.305668in}{1.722384in}}%
\pgfpathlineto{\pgfqpoint{1.302171in}{1.720872in}}%
\pgfpathlineto{\pgfqpoint{1.298676in}{1.719243in}}%
\pgfpathlineto{\pgfqpoint{1.295181in}{1.717498in}}%
\pgfpathlineto{\pgfqpoint{1.295318in}{1.718323in}}%
\pgfpathlineto{\pgfqpoint{1.295511in}{1.719145in}}%
\pgfpathlineto{\pgfqpoint{1.295758in}{1.719964in}}%
\pgfpathlineto{\pgfqpoint{1.296061in}{1.720778in}}%
\pgfpathlineto{\pgfqpoint{1.299499in}{1.722318in}}%
\pgfpathlineto{\pgfqpoint{1.302938in}{1.723741in}}%
\pgfpathlineto{\pgfqpoint{1.306378in}{1.725048in}}%
\pgfpathlineto{\pgfqpoint{1.309820in}{1.726238in}}%
\pgfpathlineto{\pgfqpoint{1.309594in}{1.725627in}}%
\pgfpathlineto{\pgfqpoint{1.309410in}{1.725014in}}%
\pgfpathlineto{\pgfqpoint{1.309267in}{1.724398in}}%
\pgfpathlineto{\pgfqpoint{1.309166in}{1.723780in}}%
\pgfpathclose%
\pgfusepath{fill}%
\end{pgfscope}%
\begin{pgfscope}%
\pgfpathrectangle{\pgfqpoint{0.329460in}{0.284240in}}{\pgfqpoint{1.989680in}{1.989680in}}%
\pgfusepath{clip}%
\pgfsetbuttcap%
\pgfsetroundjoin%
\definecolor{currentfill}{rgb}{0.974417,0.903590,0.130215}%
\pgfsetfillcolor{currentfill}%
\pgfsetlinewidth{0.000000pt}%
\definecolor{currentstroke}{rgb}{0.000000,0.000000,0.000000}%
\pgfsetstrokecolor{currentstroke}%
\pgfsetdash{}{0pt}%
\pgfpathmoveto{\pgfqpoint{1.377655in}{1.731768in}}%
\pgfpathlineto{\pgfqpoint{1.380959in}{1.731293in}}%
\pgfpathlineto{\pgfqpoint{1.384262in}{1.730700in}}%
\pgfpathlineto{\pgfqpoint{1.387564in}{1.729988in}}%
\pgfpathlineto{\pgfqpoint{1.390864in}{1.729159in}}%
\pgfpathlineto{\pgfqpoint{1.391288in}{1.728572in}}%
\pgfpathlineto{\pgfqpoint{1.391673in}{1.727979in}}%
\pgfpathlineto{\pgfqpoint{1.392017in}{1.727381in}}%
\pgfpathlineto{\pgfqpoint{1.392320in}{1.726778in}}%
\pgfpathlineto{\pgfqpoint{1.388897in}{1.727805in}}%
\pgfpathlineto{\pgfqpoint{1.385473in}{1.728715in}}%
\pgfpathlineto{\pgfqpoint{1.382048in}{1.729507in}}%
\pgfpathlineto{\pgfqpoint{1.378622in}{1.730180in}}%
\pgfpathlineto{\pgfqpoint{1.378421in}{1.730582in}}%
\pgfpathlineto{\pgfqpoint{1.378193in}{1.730981in}}%
\pgfpathlineto{\pgfqpoint{1.377937in}{1.731376in}}%
\pgfpathlineto{\pgfqpoint{1.377655in}{1.731768in}}%
\pgfpathclose%
\pgfusepath{fill}%
\end{pgfscope}%
\begin{pgfscope}%
\pgfpathrectangle{\pgfqpoint{0.329460in}{0.284240in}}{\pgfqpoint{1.989680in}{1.989680in}}%
\pgfusepath{clip}%
\pgfsetbuttcap%
\pgfsetroundjoin%
\definecolor{currentfill}{rgb}{0.267004,0.004874,0.329415}%
\pgfsetfillcolor{currentfill}%
\pgfsetlinewidth{0.000000pt}%
\definecolor{currentstroke}{rgb}{0.000000,0.000000,0.000000}%
\pgfsetstrokecolor{currentstroke}%
\pgfsetdash{}{0pt}%
\pgfpathmoveto{\pgfqpoint{1.099598in}{0.788651in}}%
\pgfpathlineto{\pgfqpoint{1.097889in}{0.786317in}}%
\pgfpathlineto{\pgfqpoint{1.096176in}{0.784199in}}%
\pgfpathlineto{\pgfqpoint{1.094459in}{0.782302in}}%
\pgfpathlineto{\pgfqpoint{1.092739in}{0.780630in}}%
\pgfpathlineto{\pgfqpoint{1.077798in}{0.785182in}}%
\pgfpathlineto{\pgfqpoint{1.063161in}{0.789984in}}%
\pgfpathlineto{\pgfqpoint{1.048843in}{0.795030in}}%
\pgfpathlineto{\pgfqpoint{1.034860in}{0.800313in}}%
\pgfpathlineto{\pgfqpoint{1.036960in}{0.801850in}}%
\pgfpathlineto{\pgfqpoint{1.039055in}{0.803611in}}%
\pgfpathlineto{\pgfqpoint{1.041146in}{0.805593in}}%
\pgfpathlineto{\pgfqpoint{1.043232in}{0.807791in}}%
\pgfpathlineto{\pgfqpoint{1.056852in}{0.802653in}}%
\pgfpathlineto{\pgfqpoint{1.070796in}{0.797747in}}%
\pgfpathlineto{\pgfqpoint{1.085050in}{0.793077in}}%
\pgfpathlineto{\pgfqpoint{1.099598in}{0.788651in}}%
\pgfpathclose%
\pgfusepath{fill}%
\end{pgfscope}%
\begin{pgfscope}%
\pgfpathrectangle{\pgfqpoint{0.329460in}{0.284240in}}{\pgfqpoint{1.989680in}{1.989680in}}%
\pgfusepath{clip}%
\pgfsetbuttcap%
\pgfsetroundjoin%
\definecolor{currentfill}{rgb}{0.248629,0.278775,0.534556}%
\pgfsetfillcolor{currentfill}%
\pgfsetlinewidth{0.000000pt}%
\definecolor{currentstroke}{rgb}{0.000000,0.000000,0.000000}%
\pgfsetstrokecolor{currentstroke}%
\pgfsetdash{}{0pt}%
\pgfpathmoveto{\pgfqpoint{1.545898in}{1.016311in}}%
\pgfpathlineto{\pgfqpoint{1.547618in}{1.008542in}}%
\pgfpathlineto{\pgfqpoint{1.549339in}{1.000842in}}%
\pgfpathlineto{\pgfqpoint{1.551060in}{0.993213in}}%
\pgfpathlineto{\pgfqpoint{1.552782in}{0.985660in}}%
\pgfpathlineto{\pgfqpoint{1.541641in}{0.982405in}}%
\pgfpathlineto{\pgfqpoint{1.530294in}{0.979335in}}%
\pgfpathlineto{\pgfqpoint{1.518753in}{0.976453in}}%
\pgfpathlineto{\pgfqpoint{1.507029in}{0.973764in}}%
\pgfpathlineto{\pgfqpoint{1.505696in}{0.981433in}}%
\pgfpathlineto{\pgfqpoint{1.504363in}{0.989177in}}%
\pgfpathlineto{\pgfqpoint{1.503030in}{0.996992in}}%
\pgfpathlineto{\pgfqpoint{1.501698in}{1.004876in}}%
\pgfpathlineto{\pgfqpoint{1.513022in}{1.007461in}}%
\pgfpathlineto{\pgfqpoint{1.524172in}{1.010230in}}%
\pgfpathlineto{\pgfqpoint{1.535134in}{1.013182in}}%
\pgfpathlineto{\pgfqpoint{1.545898in}{1.016311in}}%
\pgfpathclose%
\pgfusepath{fill}%
\end{pgfscope}%
\begin{pgfscope}%
\pgfpathrectangle{\pgfqpoint{0.329460in}{0.284240in}}{\pgfqpoint{1.989680in}{1.989680in}}%
\pgfusepath{clip}%
\pgfsetbuttcap%
\pgfsetroundjoin%
\definecolor{currentfill}{rgb}{0.274952,0.037752,0.364543}%
\pgfsetfillcolor{currentfill}%
\pgfsetlinewidth{0.000000pt}%
\definecolor{currentstroke}{rgb}{0.000000,0.000000,0.000000}%
\pgfsetstrokecolor{currentstroke}%
\pgfsetdash{}{0pt}%
\pgfpathmoveto{\pgfqpoint{1.594436in}{0.835573in}}%
\pgfpathlineto{\pgfqpoint{1.596194in}{0.831000in}}%
\pgfpathlineto{\pgfqpoint{1.597954in}{0.826593in}}%
\pgfpathlineto{\pgfqpoint{1.599717in}{0.822357in}}%
\pgfpathlineto{\pgfqpoint{1.601483in}{0.818296in}}%
\pgfpathlineto{\pgfqpoint{1.587679in}{0.814150in}}%
\pgfpathlineto{\pgfqpoint{1.573614in}{0.810239in}}%
\pgfpathlineto{\pgfqpoint{1.559304in}{0.806568in}}%
\pgfpathlineto{\pgfqpoint{1.544763in}{0.803141in}}%
\pgfpathlineto{\pgfqpoint{1.543394in}{0.807319in}}%
\pgfpathlineto{\pgfqpoint{1.542027in}{0.811671in}}%
\pgfpathlineto{\pgfqpoint{1.540663in}{0.816194in}}%
\pgfpathlineto{\pgfqpoint{1.539301in}{0.820884in}}%
\pgfpathlineto{\pgfqpoint{1.553434in}{0.824205in}}%
\pgfpathlineto{\pgfqpoint{1.567345in}{0.827764in}}%
\pgfpathlineto{\pgfqpoint{1.581017in}{0.831554in}}%
\pgfpathlineto{\pgfqpoint{1.594436in}{0.835573in}}%
\pgfpathclose%
\pgfusepath{fill}%
\end{pgfscope}%
\begin{pgfscope}%
\pgfpathrectangle{\pgfqpoint{0.329460in}{0.284240in}}{\pgfqpoint{1.989680in}{1.989680in}}%
\pgfusepath{clip}%
\pgfsetbuttcap%
\pgfsetroundjoin%
\definecolor{currentfill}{rgb}{0.974417,0.903590,0.130215}%
\pgfsetfillcolor{currentfill}%
\pgfsetlinewidth{0.000000pt}%
\definecolor{currentstroke}{rgb}{0.000000,0.000000,0.000000}%
\pgfsetstrokecolor{currentstroke}%
\pgfsetdash{}{0pt}%
\pgfpathmoveto{\pgfqpoint{1.323597in}{1.729820in}}%
\pgfpathlineto{\pgfqpoint{1.320151in}{1.729102in}}%
\pgfpathlineto{\pgfqpoint{1.316706in}{1.728266in}}%
\pgfpathlineto{\pgfqpoint{1.313263in}{1.727311in}}%
\pgfpathlineto{\pgfqpoint{1.309820in}{1.726238in}}%
\pgfpathlineto{\pgfqpoint{1.310087in}{1.726845in}}%
\pgfpathlineto{\pgfqpoint{1.310395in}{1.727448in}}%
\pgfpathlineto{\pgfqpoint{1.310743in}{1.728045in}}%
\pgfpathlineto{\pgfqpoint{1.311132in}{1.728638in}}%
\pgfpathlineto{\pgfqpoint{1.314464in}{1.729510in}}%
\pgfpathlineto{\pgfqpoint{1.317798in}{1.730265in}}%
\pgfpathlineto{\pgfqpoint{1.321132in}{1.730902in}}%
\pgfpathlineto{\pgfqpoint{1.324468in}{1.731420in}}%
\pgfpathlineto{\pgfqpoint{1.324210in}{1.731025in}}%
\pgfpathlineto{\pgfqpoint{1.323978in}{1.730627in}}%
\pgfpathlineto{\pgfqpoint{1.323774in}{1.730225in}}%
\pgfpathlineto{\pgfqpoint{1.323597in}{1.729820in}}%
\pgfpathclose%
\pgfusepath{fill}%
\end{pgfscope}%
\begin{pgfscope}%
\pgfpathrectangle{\pgfqpoint{0.329460in}{0.284240in}}{\pgfqpoint{1.989680in}{1.989680in}}%
\pgfusepath{clip}%
\pgfsetbuttcap%
\pgfsetroundjoin%
\definecolor{currentfill}{rgb}{0.935904,0.898570,0.108131}%
\pgfsetfillcolor{currentfill}%
\pgfsetlinewidth{0.000000pt}%
\definecolor{currentstroke}{rgb}{0.000000,0.000000,0.000000}%
\pgfsetstrokecolor{currentstroke}%
\pgfsetdash{}{0pt}%
\pgfpathmoveto{\pgfqpoint{1.295193in}{1.714191in}}%
\pgfpathlineto{\pgfqpoint{1.291698in}{1.712123in}}%
\pgfpathlineto{\pgfqpoint{1.288204in}{1.709940in}}%
\pgfpathlineto{\pgfqpoint{1.284712in}{1.707642in}}%
\pgfpathlineto{\pgfqpoint{1.281220in}{1.705231in}}%
\pgfpathlineto{\pgfqpoint{1.281114in}{1.706265in}}%
\pgfpathlineto{\pgfqpoint{1.281078in}{1.707300in}}%
\pgfpathlineto{\pgfqpoint{1.281112in}{1.708335in}}%
\pgfpathlineto{\pgfqpoint{1.281216in}{1.709369in}}%
\pgfpathlineto{\pgfqpoint{1.284705in}{1.711573in}}%
\pgfpathlineto{\pgfqpoint{1.288196in}{1.713662in}}%
\pgfpathlineto{\pgfqpoint{1.291688in}{1.715638in}}%
\pgfpathlineto{\pgfqpoint{1.295181in}{1.717498in}}%
\pgfpathlineto{\pgfqpoint{1.295100in}{1.716672in}}%
\pgfpathlineto{\pgfqpoint{1.295075in}{1.715845in}}%
\pgfpathlineto{\pgfqpoint{1.295106in}{1.715018in}}%
\pgfpathlineto{\pgfqpoint{1.295193in}{1.714191in}}%
\pgfpathclose%
\pgfusepath{fill}%
\end{pgfscope}%
\begin{pgfscope}%
\pgfpathrectangle{\pgfqpoint{0.329460in}{0.284240in}}{\pgfqpoint{1.989680in}{1.989680in}}%
\pgfusepath{clip}%
\pgfsetbuttcap%
\pgfsetroundjoin%
\definecolor{currentfill}{rgb}{0.993248,0.906157,0.143936}%
\pgfsetfillcolor{currentfill}%
\pgfsetlinewidth{0.000000pt}%
\definecolor{currentstroke}{rgb}{0.000000,0.000000,0.000000}%
\pgfsetstrokecolor{currentstroke}%
\pgfsetdash{}{0pt}%
\pgfpathmoveto{\pgfqpoint{1.362843in}{1.733948in}}%
\pgfpathlineto{\pgfqpoint{1.365755in}{1.734318in}}%
\pgfpathlineto{\pgfqpoint{1.368665in}{1.734568in}}%
\pgfpathlineto{\pgfqpoint{1.371574in}{1.734698in}}%
\pgfpathlineto{\pgfqpoint{1.374481in}{1.734709in}}%
\pgfpathlineto{\pgfqpoint{1.374963in}{1.734363in}}%
\pgfpathlineto{\pgfqpoint{1.375422in}{1.734011in}}%
\pgfpathlineto{\pgfqpoint{1.375857in}{1.733651in}}%
\pgfpathlineto{\pgfqpoint{1.376267in}{1.733285in}}%
\pgfpathlineto{\pgfqpoint{1.373136in}{1.733452in}}%
\pgfpathlineto{\pgfqpoint{1.370004in}{1.733499in}}%
\pgfpathlineto{\pgfqpoint{1.366870in}{1.733427in}}%
\pgfpathlineto{\pgfqpoint{1.363736in}{1.733236in}}%
\pgfpathlineto{\pgfqpoint{1.363531in}{1.733419in}}%
\pgfpathlineto{\pgfqpoint{1.363314in}{1.733598in}}%
\pgfpathlineto{\pgfqpoint{1.363084in}{1.733775in}}%
\pgfpathlineto{\pgfqpoint{1.362843in}{1.733948in}}%
\pgfpathclose%
\pgfusepath{fill}%
\end{pgfscope}%
\begin{pgfscope}%
\pgfpathrectangle{\pgfqpoint{0.329460in}{0.284240in}}{\pgfqpoint{1.989680in}{1.989680in}}%
\pgfusepath{clip}%
\pgfsetbuttcap%
\pgfsetroundjoin%
\definecolor{currentfill}{rgb}{0.993248,0.906157,0.143936}%
\pgfsetfillcolor{currentfill}%
\pgfsetlinewidth{0.000000pt}%
\definecolor{currentstroke}{rgb}{0.000000,0.000000,0.000000}%
\pgfsetstrokecolor{currentstroke}%
\pgfsetdash{}{0pt}%
\pgfpathmoveto{\pgfqpoint{1.338468in}{1.733070in}}%
\pgfpathlineto{\pgfqpoint{1.335290in}{1.733221in}}%
\pgfpathlineto{\pgfqpoint{1.332114in}{1.733252in}}%
\pgfpathlineto{\pgfqpoint{1.328938in}{1.733163in}}%
\pgfpathlineto{\pgfqpoint{1.325764in}{1.732955in}}%
\pgfpathlineto{\pgfqpoint{1.326152in}{1.733326in}}%
\pgfpathlineto{\pgfqpoint{1.326565in}{1.733691in}}%
\pgfpathlineto{\pgfqpoint{1.327003in}{1.734050in}}%
\pgfpathlineto{\pgfqpoint{1.327464in}{1.734402in}}%
\pgfpathlineto{\pgfqpoint{1.330425in}{1.734429in}}%
\pgfpathlineto{\pgfqpoint{1.333388in}{1.734337in}}%
\pgfpathlineto{\pgfqpoint{1.336352in}{1.734126in}}%
\pgfpathlineto{\pgfqpoint{1.339317in}{1.733794in}}%
\pgfpathlineto{\pgfqpoint{1.339086in}{1.733618in}}%
\pgfpathlineto{\pgfqpoint{1.338868in}{1.733439in}}%
\pgfpathlineto{\pgfqpoint{1.338662in}{1.733256in}}%
\pgfpathlineto{\pgfqpoint{1.338468in}{1.733070in}}%
\pgfpathclose%
\pgfusepath{fill}%
\end{pgfscope}%
\begin{pgfscope}%
\pgfpathrectangle{\pgfqpoint{0.329460in}{0.284240in}}{\pgfqpoint{1.989680in}{1.989680in}}%
\pgfusepath{clip}%
\pgfsetbuttcap%
\pgfsetroundjoin%
\definecolor{currentfill}{rgb}{0.487026,0.823929,0.312321}%
\pgfsetfillcolor{currentfill}%
\pgfsetlinewidth{0.000000pt}%
\definecolor{currentstroke}{rgb}{0.000000,0.000000,0.000000}%
\pgfsetstrokecolor{currentstroke}%
\pgfsetdash{}{0pt}%
\pgfpathmoveto{\pgfqpoint{1.504797in}{1.567858in}}%
\pgfpathlineto{\pgfqpoint{1.507972in}{1.561974in}}%
\pgfpathlineto{\pgfqpoint{1.511146in}{1.556008in}}%
\pgfpathlineto{\pgfqpoint{1.514317in}{1.549963in}}%
\pgfpathlineto{\pgfqpoint{1.517487in}{1.543839in}}%
\pgfpathlineto{\pgfqpoint{1.515128in}{1.541331in}}%
\pgfpathlineto{\pgfqpoint{1.512602in}{1.538860in}}%
\pgfpathlineto{\pgfqpoint{1.509912in}{1.536427in}}%
\pgfpathlineto{\pgfqpoint{1.507061in}{1.534035in}}%
\pgfpathlineto{\pgfqpoint{1.504085in}{1.540356in}}%
\pgfpathlineto{\pgfqpoint{1.501108in}{1.546598in}}%
\pgfpathlineto{\pgfqpoint{1.498129in}{1.552760in}}%
\pgfpathlineto{\pgfqpoint{1.495149in}{1.558840in}}%
\pgfpathlineto{\pgfqpoint{1.497787in}{1.561040in}}%
\pgfpathlineto{\pgfqpoint{1.500276in}{1.563278in}}%
\pgfpathlineto{\pgfqpoint{1.502613in}{1.565551in}}%
\pgfpathlineto{\pgfqpoint{1.504797in}{1.567858in}}%
\pgfpathclose%
\pgfusepath{fill}%
\end{pgfscope}%
\begin{pgfscope}%
\pgfpathrectangle{\pgfqpoint{0.329460in}{0.284240in}}{\pgfqpoint{1.989680in}{1.989680in}}%
\pgfusepath{clip}%
\pgfsetbuttcap%
\pgfsetroundjoin%
\definecolor{currentfill}{rgb}{0.263663,0.237631,0.518762}%
\pgfsetfillcolor{currentfill}%
\pgfsetlinewidth{0.000000pt}%
\definecolor{currentstroke}{rgb}{0.000000,0.000000,0.000000}%
\pgfsetstrokecolor{currentstroke}%
\pgfsetdash{}{0pt}%
\pgfpathmoveto{\pgfqpoint{1.205909in}{0.971538in}}%
\pgfpathlineto{\pgfqpoint{1.204665in}{0.963926in}}%
\pgfpathlineto{\pgfqpoint{1.203421in}{0.956395in}}%
\pgfpathlineto{\pgfqpoint{1.202176in}{0.948948in}}%
\pgfpathlineto{\pgfqpoint{1.200930in}{0.941590in}}%
\pgfpathlineto{\pgfqpoint{1.188651in}{0.944204in}}%
\pgfpathlineto{\pgfqpoint{1.176549in}{0.947020in}}%
\pgfpathlineto{\pgfqpoint{1.164637in}{0.950035in}}%
\pgfpathlineto{\pgfqpoint{1.152927in}{0.953246in}}%
\pgfpathlineto{\pgfqpoint{1.154568in}{0.960495in}}%
\pgfpathlineto{\pgfqpoint{1.156208in}{0.967832in}}%
\pgfpathlineto{\pgfqpoint{1.157847in}{0.975254in}}%
\pgfpathlineto{\pgfqpoint{1.159486in}{0.982757in}}%
\pgfpathlineto{\pgfqpoint{1.170811in}{0.979666in}}%
\pgfpathlineto{\pgfqpoint{1.182331in}{0.976764in}}%
\pgfpathlineto{\pgfqpoint{1.194035in}{0.974053in}}%
\pgfpathlineto{\pgfqpoint{1.205909in}{0.971538in}}%
\pgfpathclose%
\pgfusepath{fill}%
\end{pgfscope}%
\begin{pgfscope}%
\pgfpathrectangle{\pgfqpoint{0.329460in}{0.284240in}}{\pgfqpoint{1.989680in}{1.989680in}}%
\pgfusepath{clip}%
\pgfsetbuttcap%
\pgfsetroundjoin%
\definecolor{currentfill}{rgb}{0.636902,0.856542,0.216620}%
\pgfsetfillcolor{currentfill}%
\pgfsetlinewidth{0.000000pt}%
\definecolor{currentstroke}{rgb}{0.000000,0.000000,0.000000}%
\pgfsetstrokecolor{currentstroke}%
\pgfsetdash{}{0pt}%
\pgfpathmoveto{\pgfqpoint{1.224657in}{1.610123in}}%
\pgfpathlineto{\pgfqpoint{1.221509in}{1.604899in}}%
\pgfpathlineto{\pgfqpoint{1.218362in}{1.599583in}}%
\pgfpathlineto{\pgfqpoint{1.215217in}{1.594174in}}%
\pgfpathlineto{\pgfqpoint{1.212073in}{1.588676in}}%
\pgfpathlineto{\pgfqpoint{1.210081in}{1.590787in}}%
\pgfpathlineto{\pgfqpoint{1.208231in}{1.592926in}}%
\pgfpathlineto{\pgfqpoint{1.206527in}{1.595091in}}%
\pgfpathlineto{\pgfqpoint{1.204968in}{1.597280in}}%
\pgfpathlineto{\pgfqpoint{1.208268in}{1.602577in}}%
\pgfpathlineto{\pgfqpoint{1.211570in}{1.607783in}}%
\pgfpathlineto{\pgfqpoint{1.214873in}{1.612899in}}%
\pgfpathlineto{\pgfqpoint{1.218178in}{1.617921in}}%
\pgfpathlineto{\pgfqpoint{1.219600in}{1.615937in}}%
\pgfpathlineto{\pgfqpoint{1.221155in}{1.613975in}}%
\pgfpathlineto{\pgfqpoint{1.222841in}{1.612036in}}%
\pgfpathlineto{\pgfqpoint{1.224657in}{1.610123in}}%
\pgfpathclose%
\pgfusepath{fill}%
\end{pgfscope}%
\begin{pgfscope}%
\pgfpathrectangle{\pgfqpoint{0.329460in}{0.284240in}}{\pgfqpoint{1.989680in}{1.989680in}}%
\pgfusepath{clip}%
\pgfsetbuttcap%
\pgfsetroundjoin%
\definecolor{currentfill}{rgb}{0.279566,0.067836,0.391917}%
\pgfsetfillcolor{currentfill}%
\pgfsetlinewidth{0.000000pt}%
\definecolor{currentstroke}{rgb}{0.000000,0.000000,0.000000}%
\pgfsetstrokecolor{currentstroke}%
\pgfsetdash{}{0pt}%
\pgfpathmoveto{\pgfqpoint{1.180878in}{0.838569in}}%
\pgfpathlineto{\pgfqpoint{1.179614in}{0.833230in}}%
\pgfpathlineto{\pgfqpoint{1.178349in}{0.828041in}}%
\pgfpathlineto{\pgfqpoint{1.177082in}{0.823008in}}%
\pgfpathlineto{\pgfqpoint{1.175813in}{0.818134in}}%
\pgfpathlineto{\pgfqpoint{1.161494in}{0.821241in}}%
\pgfpathlineto{\pgfqpoint{1.147384in}{0.824589in}}%
\pgfpathlineto{\pgfqpoint{1.133499in}{0.828173in}}%
\pgfpathlineto{\pgfqpoint{1.119855in}{0.831990in}}%
\pgfpathlineto{\pgfqpoint{1.121525in}{0.836754in}}%
\pgfpathlineto{\pgfqpoint{1.123193in}{0.841677in}}%
\pgfpathlineto{\pgfqpoint{1.124859in}{0.846755in}}%
\pgfpathlineto{\pgfqpoint{1.126523in}{0.851985in}}%
\pgfpathlineto{\pgfqpoint{1.139777in}{0.848290in}}%
\pgfpathlineto{\pgfqpoint{1.153265in}{0.844819in}}%
\pgfpathlineto{\pgfqpoint{1.166970in}{0.841578in}}%
\pgfpathlineto{\pgfqpoint{1.180878in}{0.838569in}}%
\pgfpathclose%
\pgfusepath{fill}%
\end{pgfscope}%
\begin{pgfscope}%
\pgfpathrectangle{\pgfqpoint{0.329460in}{0.284240in}}{\pgfqpoint{1.989680in}{1.989680in}}%
\pgfusepath{clip}%
\pgfsetbuttcap%
\pgfsetroundjoin%
\definecolor{currentfill}{rgb}{0.260571,0.246922,0.522828}%
\pgfsetfillcolor{currentfill}%
\pgfsetlinewidth{0.000000pt}%
\definecolor{currentstroke}{rgb}{0.000000,0.000000,0.000000}%
\pgfsetstrokecolor{currentstroke}%
\pgfsetdash{}{0pt}%
\pgfpathmoveto{\pgfqpoint{1.900259in}{0.970121in}}%
\pgfpathlineto{\pgfqpoint{1.903585in}{0.977806in}}%
\pgfpathlineto{\pgfqpoint{1.906926in}{0.985873in}}%
\pgfpathlineto{\pgfqpoint{1.910283in}{0.994329in}}%
\pgfpathlineto{\pgfqpoint{1.913656in}{1.003180in}}%
\pgfpathlineto{\pgfqpoint{1.903358in}{0.993946in}}%
\pgfpathlineto{\pgfqpoint{1.892465in}{0.984874in}}%
\pgfpathlineto{\pgfqpoint{1.880985in}{0.975975in}}%
\pgfpathlineto{\pgfqpoint{1.868928in}{0.967261in}}%
\pgfpathlineto{\pgfqpoint{1.865806in}{0.958577in}}%
\pgfpathlineto{\pgfqpoint{1.862700in}{0.950291in}}%
\pgfpathlineto{\pgfqpoint{1.859608in}{0.942395in}}%
\pgfpathlineto{\pgfqpoint{1.856531in}{0.934884in}}%
\pgfpathlineto{\pgfqpoint{1.868315in}{0.943432in}}%
\pgfpathlineto{\pgfqpoint{1.879538in}{0.952161in}}%
\pgfpathlineto{\pgfqpoint{1.890189in}{0.961061in}}%
\pgfpathlineto{\pgfqpoint{1.900259in}{0.970121in}}%
\pgfpathclose%
\pgfusepath{fill}%
\end{pgfscope}%
\begin{pgfscope}%
\pgfpathrectangle{\pgfqpoint{0.329460in}{0.284240in}}{\pgfqpoint{1.989680in}{1.989680in}}%
\pgfusepath{clip}%
\pgfsetbuttcap%
\pgfsetroundjoin%
\definecolor{currentfill}{rgb}{0.896320,0.893616,0.096335}%
\pgfsetfillcolor{currentfill}%
\pgfsetlinewidth{0.000000pt}%
\definecolor{currentstroke}{rgb}{0.000000,0.000000,0.000000}%
\pgfsetstrokecolor{currentstroke}%
\pgfsetdash{}{0pt}%
\pgfpathmoveto{\pgfqpoint{1.421253in}{1.706150in}}%
\pgfpathlineto{\pgfqpoint{1.424747in}{1.703672in}}%
\pgfpathlineto{\pgfqpoint{1.428241in}{1.701080in}}%
\pgfpathlineto{\pgfqpoint{1.431733in}{1.698378in}}%
\pgfpathlineto{\pgfqpoint{1.435224in}{1.695564in}}%
\pgfpathlineto{\pgfqpoint{1.435091in}{1.694321in}}%
\pgfpathlineto{\pgfqpoint{1.434873in}{1.693081in}}%
\pgfpathlineto{\pgfqpoint{1.434573in}{1.691844in}}%
\pgfpathlineto{\pgfqpoint{1.434189in}{1.690612in}}%
\pgfpathlineto{\pgfqpoint{1.430738in}{1.693634in}}%
\pgfpathlineto{\pgfqpoint{1.427286in}{1.696545in}}%
\pgfpathlineto{\pgfqpoint{1.423833in}{1.699344in}}%
\pgfpathlineto{\pgfqpoint{1.420380in}{1.702030in}}%
\pgfpathlineto{\pgfqpoint{1.420702in}{1.703055in}}%
\pgfpathlineto{\pgfqpoint{1.420955in}{1.704084in}}%
\pgfpathlineto{\pgfqpoint{1.421139in}{1.705116in}}%
\pgfpathlineto{\pgfqpoint{1.421253in}{1.706150in}}%
\pgfpathclose%
\pgfusepath{fill}%
\end{pgfscope}%
\begin{pgfscope}%
\pgfpathrectangle{\pgfqpoint{0.329460in}{0.284240in}}{\pgfqpoint{1.989680in}{1.989680in}}%
\pgfusepath{clip}%
\pgfsetbuttcap%
\pgfsetroundjoin%
\definecolor{currentfill}{rgb}{0.272594,0.025563,0.353093}%
\pgfsetfillcolor{currentfill}%
\pgfsetlinewidth{0.000000pt}%
\definecolor{currentstroke}{rgb}{0.000000,0.000000,0.000000}%
\pgfsetstrokecolor{currentstroke}%
\pgfsetdash{}{0pt}%
\pgfpathmoveto{\pgfqpoint{1.017864in}{0.796669in}}%
\pgfpathlineto{\pgfqpoint{1.015713in}{0.797366in}}%
\pgfpathlineto{\pgfqpoint{1.013555in}{0.798334in}}%
\pgfpathlineto{\pgfqpoint{1.011390in}{0.799580in}}%
\pgfpathlineto{\pgfqpoint{1.009219in}{0.801108in}}%
\pgfpathlineto{\pgfqpoint{0.994502in}{0.807069in}}%
\pgfpathlineto{\pgfqpoint{0.980181in}{0.813273in}}%
\pgfpathlineto{\pgfqpoint{0.966271in}{0.819710in}}%
\pgfpathlineto{\pgfqpoint{0.952784in}{0.826375in}}%
\pgfpathlineto{\pgfqpoint{0.955305in}{0.824693in}}%
\pgfpathlineto{\pgfqpoint{0.957817in}{0.823292in}}%
\pgfpathlineto{\pgfqpoint{0.960322in}{0.822168in}}%
\pgfpathlineto{\pgfqpoint{0.962820in}{0.821316in}}%
\pgfpathlineto{\pgfqpoint{0.975976in}{0.814814in}}%
\pgfpathlineto{\pgfqpoint{0.989545in}{0.808534in}}%
\pgfpathlineto{\pgfqpoint{1.003512in}{0.802484in}}%
\pgfpathlineto{\pgfqpoint{1.017864in}{0.796669in}}%
\pgfpathclose%
\pgfusepath{fill}%
\end{pgfscope}%
\begin{pgfscope}%
\pgfpathrectangle{\pgfqpoint{0.329460in}{0.284240in}}{\pgfqpoint{1.989680in}{1.989680in}}%
\pgfusepath{clip}%
\pgfsetbuttcap%
\pgfsetroundjoin%
\definecolor{currentfill}{rgb}{0.133743,0.548535,0.553541}%
\pgfsetfillcolor{currentfill}%
\pgfsetlinewidth{0.000000pt}%
\definecolor{currentstroke}{rgb}{0.000000,0.000000,0.000000}%
\pgfsetstrokecolor{currentstroke}%
\pgfsetdash{}{0pt}%
\pgfpathmoveto{\pgfqpoint{1.180434in}{1.257610in}}%
\pgfpathlineto{\pgfqpoint{1.178426in}{1.249200in}}%
\pgfpathlineto{\pgfqpoint{1.176419in}{1.240781in}}%
\pgfpathlineto{\pgfqpoint{1.174412in}{1.232354in}}%
\pgfpathlineto{\pgfqpoint{1.172407in}{1.223923in}}%
\pgfpathlineto{\pgfqpoint{1.164634in}{1.226844in}}%
\pgfpathlineto{\pgfqpoint{1.157058in}{1.229887in}}%
\pgfpathlineto{\pgfqpoint{1.149687in}{1.233047in}}%
\pgfpathlineto{\pgfqpoint{1.142528in}{1.236323in}}%
\pgfpathlineto{\pgfqpoint{1.144864in}{1.244599in}}%
\pgfpathlineto{\pgfqpoint{1.147202in}{1.252870in}}%
\pgfpathlineto{\pgfqpoint{1.149540in}{1.261135in}}%
\pgfpathlineto{\pgfqpoint{1.151879in}{1.269391in}}%
\pgfpathlineto{\pgfqpoint{1.158722in}{1.266279in}}%
\pgfpathlineto{\pgfqpoint{1.165767in}{1.263276in}}%
\pgfpathlineto{\pgfqpoint{1.173006in}{1.260386in}}%
\pgfpathlineto{\pgfqpoint{1.180434in}{1.257610in}}%
\pgfpathclose%
\pgfusepath{fill}%
\end{pgfscope}%
\begin{pgfscope}%
\pgfpathrectangle{\pgfqpoint{0.329460in}{0.284240in}}{\pgfqpoint{1.989680in}{1.989680in}}%
\pgfusepath{clip}%
\pgfsetbuttcap%
\pgfsetroundjoin%
\definecolor{currentfill}{rgb}{0.814576,0.883393,0.110347}%
\pgfsetfillcolor{currentfill}%
\pgfsetlinewidth{0.000000pt}%
\definecolor{currentstroke}{rgb}{0.000000,0.000000,0.000000}%
\pgfsetstrokecolor{currentstroke}%
\pgfsetdash{}{0pt}%
\pgfpathmoveto{\pgfqpoint{1.447979in}{1.677427in}}%
\pgfpathlineto{\pgfqpoint{1.451423in}{1.673861in}}%
\pgfpathlineto{\pgfqpoint{1.454866in}{1.670189in}}%
\pgfpathlineto{\pgfqpoint{1.458307in}{1.666412in}}%
\pgfpathlineto{\pgfqpoint{1.461747in}{1.662531in}}%
\pgfpathlineto{\pgfqpoint{1.461132in}{1.660890in}}%
\pgfpathlineto{\pgfqpoint{1.460408in}{1.659259in}}%
\pgfpathlineto{\pgfqpoint{1.459573in}{1.657638in}}%
\pgfpathlineto{\pgfqpoint{1.458630in}{1.656031in}}%
\pgfpathlineto{\pgfqpoint{1.455284in}{1.660119in}}%
\pgfpathlineto{\pgfqpoint{1.451936in}{1.664102in}}%
\pgfpathlineto{\pgfqpoint{1.448588in}{1.667980in}}%
\pgfpathlineto{\pgfqpoint{1.445238in}{1.671752in}}%
\pgfpathlineto{\pgfqpoint{1.446066in}{1.673155in}}%
\pgfpathlineto{\pgfqpoint{1.446800in}{1.674570in}}%
\pgfpathlineto{\pgfqpoint{1.447438in}{1.675994in}}%
\pgfpathlineto{\pgfqpoint{1.447979in}{1.677427in}}%
\pgfpathclose%
\pgfusepath{fill}%
\end{pgfscope}%
\begin{pgfscope}%
\pgfpathrectangle{\pgfqpoint{0.329460in}{0.284240in}}{\pgfqpoint{1.989680in}{1.989680in}}%
\pgfusepath{clip}%
\pgfsetbuttcap%
\pgfsetroundjoin%
\definecolor{currentfill}{rgb}{0.134692,0.658636,0.517649}%
\pgfsetfillcolor{currentfill}%
\pgfsetlinewidth{0.000000pt}%
\definecolor{currentstroke}{rgb}{0.000000,0.000000,0.000000}%
\pgfsetstrokecolor{currentstroke}%
\pgfsetdash{}{0pt}%
\pgfpathmoveto{\pgfqpoint{1.180026in}{1.366856in}}%
\pgfpathlineto{\pgfqpoint{1.177675in}{1.358898in}}%
\pgfpathlineto{\pgfqpoint{1.175324in}{1.350903in}}%
\pgfpathlineto{\pgfqpoint{1.172975in}{1.342873in}}%
\pgfpathlineto{\pgfqpoint{1.170627in}{1.334809in}}%
\pgfpathlineto{\pgfqpoint{1.164608in}{1.337693in}}%
\pgfpathlineto{\pgfqpoint{1.158784in}{1.340669in}}%
\pgfpathlineto{\pgfqpoint{1.153160in}{1.343733in}}%
\pgfpathlineto{\pgfqpoint{1.147742in}{1.346882in}}%
\pgfpathlineto{\pgfqpoint{1.150382in}{1.354773in}}%
\pgfpathlineto{\pgfqpoint{1.153024in}{1.362632in}}%
\pgfpathlineto{\pgfqpoint{1.155667in}{1.370456in}}%
\pgfpathlineto{\pgfqpoint{1.158312in}{1.378242in}}%
\pgfpathlineto{\pgfqpoint{1.163454in}{1.375272in}}%
\pgfpathlineto{\pgfqpoint{1.168790in}{1.372382in}}%
\pgfpathlineto{\pgfqpoint{1.174316in}{1.369576in}}%
\pgfpathlineto{\pgfqpoint{1.180026in}{1.366856in}}%
\pgfpathclose%
\pgfusepath{fill}%
\end{pgfscope}%
\begin{pgfscope}%
\pgfpathrectangle{\pgfqpoint{0.329460in}{0.284240in}}{\pgfqpoint{1.989680in}{1.989680in}}%
\pgfusepath{clip}%
\pgfsetbuttcap%
\pgfsetroundjoin%
\definecolor{currentfill}{rgb}{0.993248,0.906157,0.143936}%
\pgfsetfillcolor{currentfill}%
\pgfsetlinewidth{0.000000pt}%
\definecolor{currentstroke}{rgb}{0.000000,0.000000,0.000000}%
\pgfsetstrokecolor{currentstroke}%
\pgfsetdash{}{0pt}%
\pgfpathmoveto{\pgfqpoint{1.363736in}{1.733236in}}%
\pgfpathlineto{\pgfqpoint{1.366870in}{1.733427in}}%
\pgfpathlineto{\pgfqpoint{1.370004in}{1.733499in}}%
\pgfpathlineto{\pgfqpoint{1.373136in}{1.733452in}}%
\pgfpathlineto{\pgfqpoint{1.376267in}{1.733285in}}%
\pgfpathlineto{\pgfqpoint{1.376653in}{1.732914in}}%
\pgfpathlineto{\pgfqpoint{1.377013in}{1.732537in}}%
\pgfpathlineto{\pgfqpoint{1.377347in}{1.732155in}}%
\pgfpathlineto{\pgfqpoint{1.377655in}{1.731768in}}%
\pgfpathlineto{\pgfqpoint{1.374350in}{1.732124in}}%
\pgfpathlineto{\pgfqpoint{1.371044in}{1.732361in}}%
\pgfpathlineto{\pgfqpoint{1.367737in}{1.732478in}}%
\pgfpathlineto{\pgfqpoint{1.364428in}{1.732476in}}%
\pgfpathlineto{\pgfqpoint{1.364275in}{1.732670in}}%
\pgfpathlineto{\pgfqpoint{1.364108in}{1.732861in}}%
\pgfpathlineto{\pgfqpoint{1.363928in}{1.733050in}}%
\pgfpathlineto{\pgfqpoint{1.363736in}{1.733236in}}%
\pgfpathclose%
\pgfusepath{fill}%
\end{pgfscope}%
\begin{pgfscope}%
\pgfpathrectangle{\pgfqpoint{0.329460in}{0.284240in}}{\pgfqpoint{1.989680in}{1.989680in}}%
\pgfusepath{clip}%
\pgfsetbuttcap%
\pgfsetroundjoin%
\definecolor{currentfill}{rgb}{0.201239,0.383670,0.554294}%
\pgfsetfillcolor{currentfill}%
\pgfsetlinewidth{0.000000pt}%
\definecolor{currentstroke}{rgb}{0.000000,0.000000,0.000000}%
\pgfsetstrokecolor{currentstroke}%
\pgfsetdash{}{0pt}%
\pgfpathmoveto{\pgfqpoint{1.963151in}{1.081722in}}%
\pgfpathlineto{\pgfqpoint{1.966805in}{1.092823in}}%
\pgfpathlineto{\pgfqpoint{1.970478in}{1.104359in}}%
\pgfpathlineto{\pgfqpoint{1.974171in}{1.116338in}}%
\pgfpathlineto{\pgfqpoint{1.977885in}{1.128768in}}%
\pgfpathlineto{\pgfqpoint{1.969714in}{1.118649in}}%
\pgfpathlineto{\pgfqpoint{1.960884in}{1.108653in}}%
\pgfpathlineto{\pgfqpoint{1.951402in}{1.098791in}}%
\pgfpathlineto{\pgfqpoint{1.941276in}{1.089073in}}%
\pgfpathlineto{\pgfqpoint{1.937758in}{1.076804in}}%
\pgfpathlineto{\pgfqpoint{1.934259in}{1.064987in}}%
\pgfpathlineto{\pgfqpoint{1.930780in}{1.053615in}}%
\pgfpathlineto{\pgfqpoint{1.927319in}{1.042681in}}%
\pgfpathlineto{\pgfqpoint{1.937226in}{1.052238in}}%
\pgfpathlineto{\pgfqpoint{1.946505in}{1.061937in}}%
\pgfpathlineto{\pgfqpoint{1.955149in}{1.071769in}}%
\pgfpathlineto{\pgfqpoint{1.963151in}{1.081722in}}%
\pgfpathclose%
\pgfusepath{fill}%
\end{pgfscope}%
\begin{pgfscope}%
\pgfpathrectangle{\pgfqpoint{0.329460in}{0.284240in}}{\pgfqpoint{1.989680in}{1.989680in}}%
\pgfusepath{clip}%
\pgfsetbuttcap%
\pgfsetroundjoin%
\definecolor{currentfill}{rgb}{0.267004,0.004874,0.329415}%
\pgfsetfillcolor{currentfill}%
\pgfsetlinewidth{0.000000pt}%
\definecolor{currentstroke}{rgb}{0.000000,0.000000,0.000000}%
\pgfsetstrokecolor{currentstroke}%
\pgfsetdash{}{0pt}%
\pgfpathmoveto{\pgfqpoint{1.679653in}{0.805204in}}%
\pgfpathlineto{\pgfqpoint{1.681837in}{0.803931in}}%
\pgfpathlineto{\pgfqpoint{1.684027in}{0.802891in}}%
\pgfpathlineto{\pgfqpoint{1.686222in}{0.802090in}}%
\pgfpathlineto{\pgfqpoint{1.688423in}{0.801532in}}%
\pgfpathlineto{\pgfqpoint{1.674391in}{0.795892in}}%
\pgfpathlineto{\pgfqpoint{1.660001in}{0.790491in}}%
\pgfpathlineto{\pgfqpoint{1.645269in}{0.785335in}}%
\pgfpathlineto{\pgfqpoint{1.630210in}{0.780429in}}%
\pgfpathlineto{\pgfqpoint{1.628383in}{0.781127in}}%
\pgfpathlineto{\pgfqpoint{1.626561in}{0.782069in}}%
\pgfpathlineto{\pgfqpoint{1.624744in}{0.783249in}}%
\pgfpathlineto{\pgfqpoint{1.622932in}{0.784664in}}%
\pgfpathlineto{\pgfqpoint{1.637603in}{0.789438in}}%
\pgfpathlineto{\pgfqpoint{1.651957in}{0.794457in}}%
\pgfpathlineto{\pgfqpoint{1.665979in}{0.799715in}}%
\pgfpathlineto{\pgfqpoint{1.679653in}{0.805204in}}%
\pgfpathclose%
\pgfusepath{fill}%
\end{pgfscope}%
\begin{pgfscope}%
\pgfpathrectangle{\pgfqpoint{0.329460in}{0.284240in}}{\pgfqpoint{1.989680in}{1.989680in}}%
\pgfusepath{clip}%
\pgfsetbuttcap%
\pgfsetroundjoin%
\definecolor{currentfill}{rgb}{0.974417,0.903590,0.130215}%
\pgfsetfillcolor{currentfill}%
\pgfsetlinewidth{0.000000pt}%
\definecolor{currentstroke}{rgb}{0.000000,0.000000,0.000000}%
\pgfsetstrokecolor{currentstroke}%
\pgfsetdash{}{0pt}%
\pgfpathmoveto{\pgfqpoint{1.378622in}{1.730180in}}%
\pgfpathlineto{\pgfqpoint{1.382048in}{1.729507in}}%
\pgfpathlineto{\pgfqpoint{1.385473in}{1.728715in}}%
\pgfpathlineto{\pgfqpoint{1.388897in}{1.727805in}}%
\pgfpathlineto{\pgfqpoint{1.392320in}{1.726778in}}%
\pgfpathlineto{\pgfqpoint{1.392582in}{1.726170in}}%
\pgfpathlineto{\pgfqpoint{1.392804in}{1.725559in}}%
\pgfpathlineto{\pgfqpoint{1.392984in}{1.724945in}}%
\pgfpathlineto{\pgfqpoint{1.393122in}{1.724329in}}%
\pgfpathlineto{\pgfqpoint{1.389631in}{1.725561in}}%
\pgfpathlineto{\pgfqpoint{1.386139in}{1.726675in}}%
\pgfpathlineto{\pgfqpoint{1.382647in}{1.727671in}}%
\pgfpathlineto{\pgfqpoint{1.379153in}{1.728548in}}%
\pgfpathlineto{\pgfqpoint{1.379062in}{1.728959in}}%
\pgfpathlineto{\pgfqpoint{1.378943in}{1.729368in}}%
\pgfpathlineto{\pgfqpoint{1.378797in}{1.729775in}}%
\pgfpathlineto{\pgfqpoint{1.378622in}{1.730180in}}%
\pgfpathclose%
\pgfusepath{fill}%
\end{pgfscope}%
\begin{pgfscope}%
\pgfpathrectangle{\pgfqpoint{0.329460in}{0.284240in}}{\pgfqpoint{1.989680in}{1.989680in}}%
\pgfusepath{clip}%
\pgfsetbuttcap%
\pgfsetroundjoin%
\definecolor{currentfill}{rgb}{0.993248,0.906157,0.143936}%
\pgfsetfillcolor{currentfill}%
\pgfsetlinewidth{0.000000pt}%
\definecolor{currentstroke}{rgb}{0.000000,0.000000,0.000000}%
\pgfsetstrokecolor{currentstroke}%
\pgfsetdash{}{0pt}%
\pgfpathmoveto{\pgfqpoint{1.337821in}{1.732302in}}%
\pgfpathlineto{\pgfqpoint{1.334481in}{1.732261in}}%
\pgfpathlineto{\pgfqpoint{1.331143in}{1.732100in}}%
\pgfpathlineto{\pgfqpoint{1.327805in}{1.731819in}}%
\pgfpathlineto{\pgfqpoint{1.324468in}{1.731420in}}%
\pgfpathlineto{\pgfqpoint{1.324753in}{1.731811in}}%
\pgfpathlineto{\pgfqpoint{1.325064in}{1.732197in}}%
\pgfpathlineto{\pgfqpoint{1.325401in}{1.732579in}}%
\pgfpathlineto{\pgfqpoint{1.325764in}{1.732955in}}%
\pgfpathlineto{\pgfqpoint{1.328938in}{1.733163in}}%
\pgfpathlineto{\pgfqpoint{1.332114in}{1.733252in}}%
\pgfpathlineto{\pgfqpoint{1.335290in}{1.733221in}}%
\pgfpathlineto{\pgfqpoint{1.338468in}{1.733070in}}%
\pgfpathlineto{\pgfqpoint{1.338287in}{1.732882in}}%
\pgfpathlineto{\pgfqpoint{1.338118in}{1.732691in}}%
\pgfpathlineto{\pgfqpoint{1.337963in}{1.732498in}}%
\pgfpathlineto{\pgfqpoint{1.337821in}{1.732302in}}%
\pgfpathclose%
\pgfusepath{fill}%
\end{pgfscope}%
\begin{pgfscope}%
\pgfpathrectangle{\pgfqpoint{0.329460in}{0.284240in}}{\pgfqpoint{1.989680in}{1.989680in}}%
\pgfusepath{clip}%
\pgfsetbuttcap%
\pgfsetroundjoin%
\definecolor{currentfill}{rgb}{0.281477,0.755203,0.432552}%
\pgfsetfillcolor{currentfill}%
\pgfsetlinewidth{0.000000pt}%
\definecolor{currentstroke}{rgb}{0.000000,0.000000,0.000000}%
\pgfsetstrokecolor{currentstroke}%
\pgfsetdash{}{0pt}%
\pgfpathmoveto{\pgfqpoint{1.190153in}{1.468022in}}%
\pgfpathlineto{\pgfqpoint{1.187493in}{1.460843in}}%
\pgfpathlineto{\pgfqpoint{1.184833in}{1.453603in}}%
\pgfpathlineto{\pgfqpoint{1.182175in}{1.446303in}}%
\pgfpathlineto{\pgfqpoint{1.179518in}{1.438945in}}%
\pgfpathlineto{\pgfqpoint{1.175110in}{1.441631in}}%
\pgfpathlineto{\pgfqpoint{1.170883in}{1.444382in}}%
\pgfpathlineto{\pgfqpoint{1.166841in}{1.447196in}}%
\pgfpathlineto{\pgfqpoint{1.162989in}{1.450069in}}%
\pgfpathlineto{\pgfqpoint{1.165897in}{1.457242in}}%
\pgfpathlineto{\pgfqpoint{1.168806in}{1.464357in}}%
\pgfpathlineto{\pgfqpoint{1.171716in}{1.471414in}}%
\pgfpathlineto{\pgfqpoint{1.174628in}{1.478408in}}%
\pgfpathlineto{\pgfqpoint{1.178247in}{1.475725in}}%
\pgfpathlineto{\pgfqpoint{1.182044in}{1.473098in}}%
\pgfpathlineto{\pgfqpoint{1.186014in}{1.470530in}}%
\pgfpathlineto{\pgfqpoint{1.190153in}{1.468022in}}%
\pgfpathclose%
\pgfusepath{fill}%
\end{pgfscope}%
\begin{pgfscope}%
\pgfpathrectangle{\pgfqpoint{0.329460in}{0.284240in}}{\pgfqpoint{1.989680in}{1.989680in}}%
\pgfusepath{clip}%
\pgfsetbuttcap%
\pgfsetroundjoin%
\definecolor{currentfill}{rgb}{0.179019,0.433756,0.557430}%
\pgfsetfillcolor{currentfill}%
\pgfsetlinewidth{0.000000pt}%
\definecolor{currentstroke}{rgb}{0.000000,0.000000,0.000000}%
\pgfsetstrokecolor{currentstroke}%
\pgfsetdash{}{0pt}%
\pgfpathmoveto{\pgfqpoint{1.192204in}{1.145151in}}%
\pgfpathlineto{\pgfqpoint{1.190568in}{1.136662in}}%
\pgfpathlineto{\pgfqpoint{1.188932in}{1.128195in}}%
\pgfpathlineto{\pgfqpoint{1.187296in}{1.119752in}}%
\pgfpathlineto{\pgfqpoint{1.185661in}{1.111337in}}%
\pgfpathlineto{\pgfqpoint{1.176049in}{1.114110in}}%
\pgfpathlineto{\pgfqpoint{1.166625in}{1.117036in}}%
\pgfpathlineto{\pgfqpoint{1.157399in}{1.120113in}}%
\pgfpathlineto{\pgfqpoint{1.148378in}{1.123337in}}%
\pgfpathlineto{\pgfqpoint{1.150379in}{1.131618in}}%
\pgfpathlineto{\pgfqpoint{1.152379in}{1.139927in}}%
\pgfpathlineto{\pgfqpoint{1.154380in}{1.148261in}}%
\pgfpathlineto{\pgfqpoint{1.156381in}{1.156616in}}%
\pgfpathlineto{\pgfqpoint{1.165049in}{1.153536in}}%
\pgfpathlineto{\pgfqpoint{1.173915in}{1.150596in}}%
\pgfpathlineto{\pgfqpoint{1.182970in}{1.147800in}}%
\pgfpathlineto{\pgfqpoint{1.192204in}{1.145151in}}%
\pgfpathclose%
\pgfusepath{fill}%
\end{pgfscope}%
\begin{pgfscope}%
\pgfpathrectangle{\pgfqpoint{0.329460in}{0.284240in}}{\pgfqpoint{1.989680in}{1.989680in}}%
\pgfusepath{clip}%
\pgfsetbuttcap%
\pgfsetroundjoin%
\definecolor{currentfill}{rgb}{0.955300,0.901065,0.118128}%
\pgfsetfillcolor{currentfill}%
\pgfsetlinewidth{0.000000pt}%
\definecolor{currentstroke}{rgb}{0.000000,0.000000,0.000000}%
\pgfsetstrokecolor{currentstroke}%
\pgfsetdash{}{0pt}%
\pgfpathmoveto{\pgfqpoint{1.393122in}{1.724329in}}%
\pgfpathlineto{\pgfqpoint{1.396612in}{1.722980in}}%
\pgfpathlineto{\pgfqpoint{1.400100in}{1.721513in}}%
\pgfpathlineto{\pgfqpoint{1.403588in}{1.719930in}}%
\pgfpathlineto{\pgfqpoint{1.407075in}{1.718231in}}%
\pgfpathlineto{\pgfqpoint{1.407205in}{1.717407in}}%
\pgfpathlineto{\pgfqpoint{1.407280in}{1.716580in}}%
\pgfpathlineto{\pgfqpoint{1.407299in}{1.715753in}}%
\pgfpathlineto{\pgfqpoint{1.407262in}{1.714926in}}%
\pgfpathlineto{\pgfqpoint{1.403762in}{1.716832in}}%
\pgfpathlineto{\pgfqpoint{1.400261in}{1.718622in}}%
\pgfpathlineto{\pgfqpoint{1.396759in}{1.720295in}}%
\pgfpathlineto{\pgfqpoint{1.393256in}{1.721852in}}%
\pgfpathlineto{\pgfqpoint{1.393285in}{1.722471in}}%
\pgfpathlineto{\pgfqpoint{1.393273in}{1.723091in}}%
\pgfpathlineto{\pgfqpoint{1.393218in}{1.723711in}}%
\pgfpathlineto{\pgfqpoint{1.393122in}{1.724329in}}%
\pgfpathclose%
\pgfusepath{fill}%
\end{pgfscope}%
\begin{pgfscope}%
\pgfpathrectangle{\pgfqpoint{0.329460in}{0.284240in}}{\pgfqpoint{1.989680in}{1.989680in}}%
\pgfusepath{clip}%
\pgfsetbuttcap%
\pgfsetroundjoin%
\definecolor{currentfill}{rgb}{0.896320,0.893616,0.096335}%
\pgfsetfillcolor{currentfill}%
\pgfsetlinewidth{0.000000pt}%
\definecolor{currentstroke}{rgb}{0.000000,0.000000,0.000000}%
\pgfsetstrokecolor{currentstroke}%
\pgfsetdash{}{0pt}%
\pgfpathmoveto{\pgfqpoint{1.282340in}{1.701123in}}%
\pgfpathlineto{\pgfqpoint{1.278903in}{1.698391in}}%
\pgfpathlineto{\pgfqpoint{1.275467in}{1.695546in}}%
\pgfpathlineto{\pgfqpoint{1.272032in}{1.692590in}}%
\pgfpathlineto{\pgfqpoint{1.268598in}{1.689522in}}%
\pgfpathlineto{\pgfqpoint{1.268140in}{1.690748in}}%
\pgfpathlineto{\pgfqpoint{1.267765in}{1.691981in}}%
\pgfpathlineto{\pgfqpoint{1.267473in}{1.693218in}}%
\pgfpathlineto{\pgfqpoint{1.267266in}{1.694459in}}%
\pgfpathlineto{\pgfqpoint{1.270752in}{1.697319in}}%
\pgfpathlineto{\pgfqpoint{1.274240in}{1.700069in}}%
\pgfpathlineto{\pgfqpoint{1.277730in}{1.702706in}}%
\pgfpathlineto{\pgfqpoint{1.281220in}{1.705231in}}%
\pgfpathlineto{\pgfqpoint{1.281396in}{1.704198in}}%
\pgfpathlineto{\pgfqpoint{1.281641in}{1.703169in}}%
\pgfpathlineto{\pgfqpoint{1.281956in}{1.702143in}}%
\pgfpathlineto{\pgfqpoint{1.282340in}{1.701123in}}%
\pgfpathclose%
\pgfusepath{fill}%
\end{pgfscope}%
\begin{pgfscope}%
\pgfpathrectangle{\pgfqpoint{0.329460in}{0.284240in}}{\pgfqpoint{1.989680in}{1.989680in}}%
\pgfusepath{clip}%
\pgfsetbuttcap%
\pgfsetroundjoin%
\definecolor{currentfill}{rgb}{0.231674,0.318106,0.544834}%
\pgfsetfillcolor{currentfill}%
\pgfsetlinewidth{0.000000pt}%
\definecolor{currentstroke}{rgb}{0.000000,0.000000,0.000000}%
\pgfsetstrokecolor{currentstroke}%
\pgfsetdash{}{0pt}%
\pgfpathmoveto{\pgfqpoint{1.539019in}{1.048013in}}%
\pgfpathlineto{\pgfqpoint{1.540738in}{1.040000in}}%
\pgfpathlineto{\pgfqpoint{1.542458in}{1.032043in}}%
\pgfpathlineto{\pgfqpoint{1.544178in}{1.024146in}}%
\pgfpathlineto{\pgfqpoint{1.545898in}{1.016311in}}%
\pgfpathlineto{\pgfqpoint{1.535134in}{1.013182in}}%
\pgfpathlineto{\pgfqpoint{1.524172in}{1.010230in}}%
\pgfpathlineto{\pgfqpoint{1.513022in}{1.007461in}}%
\pgfpathlineto{\pgfqpoint{1.501698in}{1.004876in}}%
\pgfpathlineto{\pgfqpoint{1.500366in}{1.012826in}}%
\pgfpathlineto{\pgfqpoint{1.499034in}{1.020838in}}%
\pgfpathlineto{\pgfqpoint{1.497703in}{1.028909in}}%
\pgfpathlineto{\pgfqpoint{1.496372in}{1.037037in}}%
\pgfpathlineto{\pgfqpoint{1.507298in}{1.039518in}}%
\pgfpathlineto{\pgfqpoint{1.518055in}{1.042176in}}%
\pgfpathlineto{\pgfqpoint{1.528633in}{1.045009in}}%
\pgfpathlineto{\pgfqpoint{1.539019in}{1.048013in}}%
\pgfpathclose%
\pgfusepath{fill}%
\end{pgfscope}%
\begin{pgfscope}%
\pgfpathrectangle{\pgfqpoint{0.329460in}{0.284240in}}{\pgfqpoint{1.989680in}{1.989680in}}%
\pgfusepath{clip}%
\pgfsetbuttcap%
\pgfsetroundjoin%
\definecolor{currentfill}{rgb}{0.699415,0.867117,0.175971}%
\pgfsetfillcolor{currentfill}%
\pgfsetlinewidth{0.000000pt}%
\definecolor{currentstroke}{rgb}{0.000000,0.000000,0.000000}%
\pgfsetstrokecolor{currentstroke}%
\pgfsetdash{}{0pt}%
\pgfpathmoveto{\pgfqpoint{1.472002in}{1.638657in}}%
\pgfpathlineto{\pgfqpoint{1.475341in}{1.634063in}}%
\pgfpathlineto{\pgfqpoint{1.478679in}{1.629372in}}%
\pgfpathlineto{\pgfqpoint{1.482015in}{1.624584in}}%
\pgfpathlineto{\pgfqpoint{1.485349in}{1.619701in}}%
\pgfpathlineto{\pgfqpoint{1.484046in}{1.617700in}}%
\pgfpathlineto{\pgfqpoint{1.482609in}{1.615718in}}%
\pgfpathlineto{\pgfqpoint{1.481039in}{1.613758in}}%
\pgfpathlineto{\pgfqpoint{1.479338in}{1.611822in}}%
\pgfpathlineto{\pgfqpoint{1.476149in}{1.616907in}}%
\pgfpathlineto{\pgfqpoint{1.472959in}{1.621897in}}%
\pgfpathlineto{\pgfqpoint{1.469767in}{1.626790in}}%
\pgfpathlineto{\pgfqpoint{1.466574in}{1.631585in}}%
\pgfpathlineto{\pgfqpoint{1.468109in}{1.633322in}}%
\pgfpathlineto{\pgfqpoint{1.469526in}{1.635081in}}%
\pgfpathlineto{\pgfqpoint{1.470824in}{1.636860in}}%
\pgfpathlineto{\pgfqpoint{1.472002in}{1.638657in}}%
\pgfpathclose%
\pgfusepath{fill}%
\end{pgfscope}%
\begin{pgfscope}%
\pgfpathrectangle{\pgfqpoint{0.329460in}{0.284240in}}{\pgfqpoint{1.989680in}{1.989680in}}%
\pgfusepath{clip}%
\pgfsetbuttcap%
\pgfsetroundjoin%
\definecolor{currentfill}{rgb}{0.974417,0.903590,0.130215}%
\pgfsetfillcolor{currentfill}%
\pgfsetlinewidth{0.000000pt}%
\definecolor{currentstroke}{rgb}{0.000000,0.000000,0.000000}%
\pgfsetstrokecolor{currentstroke}%
\pgfsetdash{}{0pt}%
\pgfpathmoveto{\pgfqpoint{1.323164in}{1.728182in}}%
\pgfpathlineto{\pgfqpoint{1.319663in}{1.727259in}}%
\pgfpathlineto{\pgfqpoint{1.316163in}{1.726217in}}%
\pgfpathlineto{\pgfqpoint{1.312664in}{1.725057in}}%
\pgfpathlineto{\pgfqpoint{1.309166in}{1.723780in}}%
\pgfpathlineto{\pgfqpoint{1.309267in}{1.724398in}}%
\pgfpathlineto{\pgfqpoint{1.309410in}{1.725014in}}%
\pgfpathlineto{\pgfqpoint{1.309594in}{1.725627in}}%
\pgfpathlineto{\pgfqpoint{1.309820in}{1.726238in}}%
\pgfpathlineto{\pgfqpoint{1.313263in}{1.727311in}}%
\pgfpathlineto{\pgfqpoint{1.316706in}{1.728266in}}%
\pgfpathlineto{\pgfqpoint{1.320151in}{1.729102in}}%
\pgfpathlineto{\pgfqpoint{1.323597in}{1.729820in}}%
\pgfpathlineto{\pgfqpoint{1.323447in}{1.729413in}}%
\pgfpathlineto{\pgfqpoint{1.323325in}{1.729004in}}%
\pgfpathlineto{\pgfqpoint{1.323231in}{1.728594in}}%
\pgfpathlineto{\pgfqpoint{1.323164in}{1.728182in}}%
\pgfpathclose%
\pgfusepath{fill}%
\end{pgfscope}%
\begin{pgfscope}%
\pgfpathrectangle{\pgfqpoint{0.329460in}{0.284240in}}{\pgfqpoint{1.989680in}{1.989680in}}%
\pgfusepath{clip}%
\pgfsetbuttcap%
\pgfsetroundjoin%
\definecolor{currentfill}{rgb}{0.122606,0.585371,0.546557}%
\pgfsetfillcolor{currentfill}%
\pgfsetlinewidth{0.000000pt}%
\definecolor{currentstroke}{rgb}{0.000000,0.000000,0.000000}%
\pgfsetstrokecolor{currentstroke}%
\pgfsetdash{}{0pt}%
\pgfpathmoveto{\pgfqpoint{1.546764in}{1.304977in}}%
\pgfpathlineto{\pgfqpoint{1.549175in}{1.296821in}}%
\pgfpathlineto{\pgfqpoint{1.551586in}{1.288645in}}%
\pgfpathlineto{\pgfqpoint{1.553995in}{1.280452in}}%
\pgfpathlineto{\pgfqpoint{1.556403in}{1.272245in}}%
\pgfpathlineto{\pgfqpoint{1.549746in}{1.269040in}}%
\pgfpathlineto{\pgfqpoint{1.542880in}{1.265940in}}%
\pgfpathlineto{\pgfqpoint{1.535814in}{1.262950in}}%
\pgfpathlineto{\pgfqpoint{1.528553in}{1.260071in}}%
\pgfpathlineto{\pgfqpoint{1.526467in}{1.268437in}}%
\pgfpathlineto{\pgfqpoint{1.524381in}{1.276788in}}%
\pgfpathlineto{\pgfqpoint{1.522294in}{1.285122in}}%
\pgfpathlineto{\pgfqpoint{1.520206in}{1.293437in}}%
\pgfpathlineto{\pgfqpoint{1.527129in}{1.296165in}}%
\pgfpathlineto{\pgfqpoint{1.533868in}{1.299000in}}%
\pgfpathlineto{\pgfqpoint{1.540415in}{1.301939in}}%
\pgfpathlineto{\pgfqpoint{1.546764in}{1.304977in}}%
\pgfpathclose%
\pgfusepath{fill}%
\end{pgfscope}%
\begin{pgfscope}%
\pgfpathrectangle{\pgfqpoint{0.329460in}{0.284240in}}{\pgfqpoint{1.989680in}{1.989680in}}%
\pgfusepath{clip}%
\pgfsetbuttcap%
\pgfsetroundjoin%
\definecolor{currentfill}{rgb}{0.248629,0.278775,0.534556}%
\pgfsetfillcolor{currentfill}%
\pgfsetlinewidth{0.000000pt}%
\definecolor{currentstroke}{rgb}{0.000000,0.000000,0.000000}%
\pgfsetstrokecolor{currentstroke}%
\pgfsetdash{}{0pt}%
\pgfpathmoveto{\pgfqpoint{1.210881in}{1.002736in}}%
\pgfpathlineto{\pgfqpoint{1.209638in}{0.994830in}}%
\pgfpathlineto{\pgfqpoint{1.208396in}{0.986993in}}%
\pgfpathlineto{\pgfqpoint{1.207153in}{0.979228in}}%
\pgfpathlineto{\pgfqpoint{1.205909in}{0.971538in}}%
\pgfpathlineto{\pgfqpoint{1.194035in}{0.974053in}}%
\pgfpathlineto{\pgfqpoint{1.182331in}{0.976764in}}%
\pgfpathlineto{\pgfqpoint{1.170811in}{0.979666in}}%
\pgfpathlineto{\pgfqpoint{1.159486in}{0.982757in}}%
\pgfpathlineto{\pgfqpoint{1.161124in}{0.990339in}}%
\pgfpathlineto{\pgfqpoint{1.162761in}{0.997995in}}%
\pgfpathlineto{\pgfqpoint{1.164398in}{1.005723in}}%
\pgfpathlineto{\pgfqpoint{1.166035in}{1.013521in}}%
\pgfpathlineto{\pgfqpoint{1.176976in}{1.010549in}}%
\pgfpathlineto{\pgfqpoint{1.188105in}{1.007759in}}%
\pgfpathlineto{\pgfqpoint{1.199411in}{1.005154in}}%
\pgfpathlineto{\pgfqpoint{1.210881in}{1.002736in}}%
\pgfpathclose%
\pgfusepath{fill}%
\end{pgfscope}%
\begin{pgfscope}%
\pgfpathrectangle{\pgfqpoint{0.329460in}{0.284240in}}{\pgfqpoint{1.989680in}{1.989680in}}%
\pgfusepath{clip}%
\pgfsetbuttcap%
\pgfsetroundjoin%
\definecolor{currentfill}{rgb}{0.282884,0.135920,0.453427}%
\pgfsetfillcolor{currentfill}%
\pgfsetlinewidth{0.000000pt}%
\definecolor{currentstroke}{rgb}{0.000000,0.000000,0.000000}%
\pgfsetstrokecolor{currentstroke}%
\pgfsetdash{}{0pt}%
\pgfpathmoveto{\pgfqpoint{1.832392in}{0.887867in}}%
\pgfpathlineto{\pgfqpoint{1.835366in}{0.892530in}}%
\pgfpathlineto{\pgfqpoint{1.838351in}{0.897528in}}%
\pgfpathlineto{\pgfqpoint{1.841349in}{0.902867in}}%
\pgfpathlineto{\pgfqpoint{1.844359in}{0.908552in}}%
\pgfpathlineto{\pgfqpoint{1.832306in}{0.900364in}}%
\pgfpathlineto{\pgfqpoint{1.819727in}{0.892372in}}%
\pgfpathlineto{\pgfqpoint{1.806635in}{0.884588in}}%
\pgfpathlineto{\pgfqpoint{1.793042in}{0.877019in}}%
\pgfpathlineto{\pgfqpoint{1.790332in}{0.871500in}}%
\pgfpathlineto{\pgfqpoint{1.787634in}{0.866329in}}%
\pgfpathlineto{\pgfqpoint{1.784947in}{0.861500in}}%
\pgfpathlineto{\pgfqpoint{1.782270in}{0.857007in}}%
\pgfpathlineto{\pgfqpoint{1.795544in}{0.864413in}}%
\pgfpathlineto{\pgfqpoint{1.808331in}{0.872031in}}%
\pgfpathlineto{\pgfqpoint{1.820617in}{0.879852in}}%
\pgfpathlineto{\pgfqpoint{1.832392in}{0.887867in}}%
\pgfpathclose%
\pgfusepath{fill}%
\end{pgfscope}%
\begin{pgfscope}%
\pgfpathrectangle{\pgfqpoint{0.329460in}{0.284240in}}{\pgfqpoint{1.989680in}{1.989680in}}%
\pgfusepath{clip}%
\pgfsetbuttcap%
\pgfsetroundjoin%
\definecolor{currentfill}{rgb}{0.814576,0.883393,0.110347}%
\pgfsetfillcolor{currentfill}%
\pgfsetlinewidth{0.000000pt}%
\definecolor{currentstroke}{rgb}{0.000000,0.000000,0.000000}%
\pgfsetstrokecolor{currentstroke}%
\pgfsetdash{}{0pt}%
\pgfpathmoveto{\pgfqpoint{1.257953in}{1.670515in}}%
\pgfpathlineto{\pgfqpoint{1.254632in}{1.666698in}}%
\pgfpathlineto{\pgfqpoint{1.251311in}{1.662775in}}%
\pgfpathlineto{\pgfqpoint{1.247992in}{1.658747in}}%
\pgfpathlineto{\pgfqpoint{1.244674in}{1.654614in}}%
\pgfpathlineto{\pgfqpoint{1.243635in}{1.656209in}}%
\pgfpathlineto{\pgfqpoint{1.242704in}{1.657818in}}%
\pgfpathlineto{\pgfqpoint{1.241882in}{1.659439in}}%
\pgfpathlineto{\pgfqpoint{1.241169in}{1.661072in}}%
\pgfpathlineto{\pgfqpoint{1.244593in}{1.664999in}}%
\pgfpathlineto{\pgfqpoint{1.248018in}{1.668823in}}%
\pgfpathlineto{\pgfqpoint{1.251445in}{1.672541in}}%
\pgfpathlineto{\pgfqpoint{1.254873in}{1.676153in}}%
\pgfpathlineto{\pgfqpoint{1.255500in}{1.674728in}}%
\pgfpathlineto{\pgfqpoint{1.256223in}{1.673312in}}%
\pgfpathlineto{\pgfqpoint{1.257041in}{1.671907in}}%
\pgfpathlineto{\pgfqpoint{1.257953in}{1.670515in}}%
\pgfpathclose%
\pgfusepath{fill}%
\end{pgfscope}%
\begin{pgfscope}%
\pgfpathrectangle{\pgfqpoint{0.329460in}{0.284240in}}{\pgfqpoint{1.989680in}{1.989680in}}%
\pgfusepath{clip}%
\pgfsetbuttcap%
\pgfsetroundjoin%
\definecolor{currentfill}{rgb}{0.166383,0.690856,0.496502}%
\pgfsetfillcolor{currentfill}%
\pgfsetlinewidth{0.000000pt}%
\definecolor{currentstroke}{rgb}{0.000000,0.000000,0.000000}%
\pgfsetstrokecolor{currentstroke}%
\pgfsetdash{}{0pt}%
\pgfpathmoveto{\pgfqpoint{1.537637in}{1.411519in}}%
\pgfpathlineto{\pgfqpoint{1.540346in}{1.403942in}}%
\pgfpathlineto{\pgfqpoint{1.543054in}{1.396320in}}%
\pgfpathlineto{\pgfqpoint{1.545761in}{1.388655in}}%
\pgfpathlineto{\pgfqpoint{1.548466in}{1.380948in}}%
\pgfpathlineto{\pgfqpoint{1.543502in}{1.377908in}}%
\pgfpathlineto{\pgfqpoint{1.538338in}{1.374947in}}%
\pgfpathlineto{\pgfqpoint{1.532980in}{1.372066in}}%
\pgfpathlineto{\pgfqpoint{1.527433in}{1.369269in}}%
\pgfpathlineto{\pgfqpoint{1.525012in}{1.377151in}}%
\pgfpathlineto{\pgfqpoint{1.522590in}{1.384991in}}%
\pgfpathlineto{\pgfqpoint{1.520166in}{1.392788in}}%
\pgfpathlineto{\pgfqpoint{1.517741in}{1.400538in}}%
\pgfpathlineto{\pgfqpoint{1.522987in}{1.403168in}}%
\pgfpathlineto{\pgfqpoint{1.528055in}{1.405876in}}%
\pgfpathlineto{\pgfqpoint{1.532940in}{1.408661in}}%
\pgfpathlineto{\pgfqpoint{1.537637in}{1.411519in}}%
\pgfpathclose%
\pgfusepath{fill}%
\end{pgfscope}%
\begin{pgfscope}%
\pgfpathrectangle{\pgfqpoint{0.329460in}{0.284240in}}{\pgfqpoint{1.989680in}{1.989680in}}%
\pgfusepath{clip}%
\pgfsetbuttcap%
\pgfsetroundjoin%
\definecolor{currentfill}{rgb}{0.955300,0.901065,0.118128}%
\pgfsetfillcolor{currentfill}%
\pgfsetlinewidth{0.000000pt}%
\definecolor{currentstroke}{rgb}{0.000000,0.000000,0.000000}%
\pgfsetstrokecolor{currentstroke}%
\pgfsetdash{}{0pt}%
\pgfpathmoveto{\pgfqpoint{1.309180in}{1.721301in}}%
\pgfpathlineto{\pgfqpoint{1.305682in}{1.719699in}}%
\pgfpathlineto{\pgfqpoint{1.302185in}{1.717979in}}%
\pgfpathlineto{\pgfqpoint{1.298689in}{1.716143in}}%
\pgfpathlineto{\pgfqpoint{1.295193in}{1.714191in}}%
\pgfpathlineto{\pgfqpoint{1.295106in}{1.715018in}}%
\pgfpathlineto{\pgfqpoint{1.295075in}{1.715845in}}%
\pgfpathlineto{\pgfqpoint{1.295100in}{1.716672in}}%
\pgfpathlineto{\pgfqpoint{1.295181in}{1.717498in}}%
\pgfpathlineto{\pgfqpoint{1.298676in}{1.719243in}}%
\pgfpathlineto{\pgfqpoint{1.302171in}{1.720872in}}%
\pgfpathlineto{\pgfqpoint{1.305668in}{1.722384in}}%
\pgfpathlineto{\pgfqpoint{1.309166in}{1.723780in}}%
\pgfpathlineto{\pgfqpoint{1.309106in}{1.723160in}}%
\pgfpathlineto{\pgfqpoint{1.309089in}{1.722540in}}%
\pgfpathlineto{\pgfqpoint{1.309114in}{1.721920in}}%
\pgfpathlineto{\pgfqpoint{1.309180in}{1.721301in}}%
\pgfpathclose%
\pgfusepath{fill}%
\end{pgfscope}%
\begin{pgfscope}%
\pgfpathrectangle{\pgfqpoint{0.329460in}{0.284240in}}{\pgfqpoint{1.989680in}{1.989680in}}%
\pgfusepath{clip}%
\pgfsetbuttcap%
\pgfsetroundjoin%
\definecolor{currentfill}{rgb}{0.993248,0.906157,0.143936}%
\pgfsetfillcolor{currentfill}%
\pgfsetlinewidth{0.000000pt}%
\definecolor{currentstroke}{rgb}{0.000000,0.000000,0.000000}%
\pgfsetstrokecolor{currentstroke}%
\pgfsetdash{}{0pt}%
\pgfpathmoveto{\pgfqpoint{1.351188in}{1.731270in}}%
\pgfpathlineto{\pgfqpoint{1.350752in}{1.732832in}}%
\pgfpathlineto{\pgfqpoint{1.350316in}{1.734273in}}%
\pgfpathlineto{\pgfqpoint{1.349881in}{1.735593in}}%
\pgfpathlineto{\pgfqpoint{1.349446in}{1.736792in}}%
\pgfpathlineto{\pgfqpoint{1.349886in}{1.736815in}}%
\pgfpathlineto{\pgfqpoint{1.350327in}{1.736831in}}%
\pgfpathlineto{\pgfqpoint{1.350769in}{1.736840in}}%
\pgfpathlineto{\pgfqpoint{1.351212in}{1.736843in}}%
\pgfpathlineto{\pgfqpoint{1.351206in}{1.735631in}}%
\pgfpathlineto{\pgfqpoint{1.351200in}{1.734298in}}%
\pgfpathlineto{\pgfqpoint{1.351194in}{1.732844in}}%
\pgfpathlineto{\pgfqpoint{1.351188in}{1.731270in}}%
\pgfpathlineto{\pgfqpoint{1.351188in}{1.731270in}}%
\pgfpathlineto{\pgfqpoint{1.351188in}{1.731270in}}%
\pgfpathlineto{\pgfqpoint{1.351188in}{1.731270in}}%
\pgfpathlineto{\pgfqpoint{1.351188in}{1.731270in}}%
\pgfpathclose%
\pgfusepath{fill}%
\end{pgfscope}%
\begin{pgfscope}%
\pgfpathrectangle{\pgfqpoint{0.329460in}{0.284240in}}{\pgfqpoint{1.989680in}{1.989680in}}%
\pgfusepath{clip}%
\pgfsetbuttcap%
\pgfsetroundjoin%
\definecolor{currentfill}{rgb}{0.993248,0.906157,0.143936}%
\pgfsetfillcolor{currentfill}%
\pgfsetlinewidth{0.000000pt}%
\definecolor{currentstroke}{rgb}{0.000000,0.000000,0.000000}%
\pgfsetstrokecolor{currentstroke}%
\pgfsetdash{}{0pt}%
\pgfpathmoveto{\pgfqpoint{1.351188in}{1.731270in}}%
\pgfpathlineto{\pgfqpoint{1.351194in}{1.732844in}}%
\pgfpathlineto{\pgfqpoint{1.351200in}{1.734298in}}%
\pgfpathlineto{\pgfqpoint{1.351206in}{1.735631in}}%
\pgfpathlineto{\pgfqpoint{1.351212in}{1.736843in}}%
\pgfpathlineto{\pgfqpoint{1.351655in}{1.736839in}}%
\pgfpathlineto{\pgfqpoint{1.352097in}{1.736829in}}%
\pgfpathlineto{\pgfqpoint{1.352538in}{1.736812in}}%
\pgfpathlineto{\pgfqpoint{1.352978in}{1.736789in}}%
\pgfpathlineto{\pgfqpoint{1.352531in}{1.735591in}}%
\pgfpathlineto{\pgfqpoint{1.352083in}{1.734271in}}%
\pgfpathlineto{\pgfqpoint{1.351636in}{1.732831in}}%
\pgfpathlineto{\pgfqpoint{1.351188in}{1.731270in}}%
\pgfpathlineto{\pgfqpoint{1.351188in}{1.731270in}}%
\pgfpathlineto{\pgfqpoint{1.351188in}{1.731270in}}%
\pgfpathlineto{\pgfqpoint{1.351188in}{1.731270in}}%
\pgfpathlineto{\pgfqpoint{1.351188in}{1.731270in}}%
\pgfpathclose%
\pgfusepath{fill}%
\end{pgfscope}%
\begin{pgfscope}%
\pgfpathrectangle{\pgfqpoint{0.329460in}{0.284240in}}{\pgfqpoint{1.989680in}{1.989680in}}%
\pgfusepath{clip}%
\pgfsetbuttcap%
\pgfsetroundjoin%
\definecolor{currentfill}{rgb}{0.993248,0.906157,0.143936}%
\pgfsetfillcolor{currentfill}%
\pgfsetlinewidth{0.000000pt}%
\definecolor{currentstroke}{rgb}{0.000000,0.000000,0.000000}%
\pgfsetstrokecolor{currentstroke}%
\pgfsetdash{}{0pt}%
\pgfpathmoveto{\pgfqpoint{1.364428in}{1.732476in}}%
\pgfpathlineto{\pgfqpoint{1.367737in}{1.732478in}}%
\pgfpathlineto{\pgfqpoint{1.371044in}{1.732361in}}%
\pgfpathlineto{\pgfqpoint{1.374350in}{1.732124in}}%
\pgfpathlineto{\pgfqpoint{1.377655in}{1.731768in}}%
\pgfpathlineto{\pgfqpoint{1.377937in}{1.731376in}}%
\pgfpathlineto{\pgfqpoint{1.378193in}{1.730981in}}%
\pgfpathlineto{\pgfqpoint{1.378421in}{1.730582in}}%
\pgfpathlineto{\pgfqpoint{1.378622in}{1.730180in}}%
\pgfpathlineto{\pgfqpoint{1.375196in}{1.730735in}}%
\pgfpathlineto{\pgfqpoint{1.371768in}{1.731170in}}%
\pgfpathlineto{\pgfqpoint{1.368340in}{1.731486in}}%
\pgfpathlineto{\pgfqpoint{1.364910in}{1.731682in}}%
\pgfpathlineto{\pgfqpoint{1.364810in}{1.731883in}}%
\pgfpathlineto{\pgfqpoint{1.364696in}{1.732083in}}%
\pgfpathlineto{\pgfqpoint{1.364569in}{1.732281in}}%
\pgfpathlineto{\pgfqpoint{1.364428in}{1.732476in}}%
\pgfpathclose%
\pgfusepath{fill}%
\end{pgfscope}%
\begin{pgfscope}%
\pgfpathrectangle{\pgfqpoint{0.329460in}{0.284240in}}{\pgfqpoint{1.989680in}{1.989680in}}%
\pgfusepath{clip}%
\pgfsetbuttcap%
\pgfsetroundjoin%
\definecolor{currentfill}{rgb}{0.163625,0.471133,0.558148}%
\pgfsetfillcolor{currentfill}%
\pgfsetlinewidth{0.000000pt}%
\definecolor{currentstroke}{rgb}{0.000000,0.000000,0.000000}%
\pgfsetstrokecolor{currentstroke}%
\pgfsetdash{}{0pt}%
\pgfpathmoveto{\pgfqpoint{1.545211in}{1.192924in}}%
\pgfpathlineto{\pgfqpoint{1.547290in}{1.184541in}}%
\pgfpathlineto{\pgfqpoint{1.549369in}{1.176169in}}%
\pgfpathlineto{\pgfqpoint{1.551448in}{1.167811in}}%
\pgfpathlineto{\pgfqpoint{1.553526in}{1.159470in}}%
\pgfpathlineto{\pgfqpoint{1.545041in}{1.156267in}}%
\pgfpathlineto{\pgfqpoint{1.536351in}{1.153203in}}%
\pgfpathlineto{\pgfqpoint{1.527463in}{1.150278in}}%
\pgfpathlineto{\pgfqpoint{1.518388in}{1.147498in}}%
\pgfpathlineto{\pgfqpoint{1.516668in}{1.155979in}}%
\pgfpathlineto{\pgfqpoint{1.514947in}{1.164475in}}%
\pgfpathlineto{\pgfqpoint{1.513226in}{1.172985in}}%
\pgfpathlineto{\pgfqpoint{1.511505in}{1.181506in}}%
\pgfpathlineto{\pgfqpoint{1.520210in}{1.184157in}}%
\pgfpathlineto{\pgfqpoint{1.528735in}{1.186946in}}%
\pgfpathlineto{\pgfqpoint{1.537071in}{1.189870in}}%
\pgfpathlineto{\pgfqpoint{1.545211in}{1.192924in}}%
\pgfpathclose%
\pgfusepath{fill}%
\end{pgfscope}%
\begin{pgfscope}%
\pgfpathrectangle{\pgfqpoint{0.329460in}{0.284240in}}{\pgfqpoint{1.989680in}{1.989680in}}%
\pgfusepath{clip}%
\pgfsetbuttcap%
\pgfsetroundjoin%
\definecolor{currentfill}{rgb}{0.993248,0.906157,0.143936}%
\pgfsetfillcolor{currentfill}%
\pgfsetlinewidth{0.000000pt}%
\definecolor{currentstroke}{rgb}{0.000000,0.000000,0.000000}%
\pgfsetstrokecolor{currentstroke}%
\pgfsetdash{}{0pt}%
\pgfpathmoveto{\pgfqpoint{1.351188in}{1.731270in}}%
\pgfpathlineto{\pgfqpoint{1.350317in}{1.732793in}}%
\pgfpathlineto{\pgfqpoint{1.349446in}{1.734196in}}%
\pgfpathlineto{\pgfqpoint{1.348576in}{1.735477in}}%
\pgfpathlineto{\pgfqpoint{1.347707in}{1.736639in}}%
\pgfpathlineto{\pgfqpoint{1.348137in}{1.736687in}}%
\pgfpathlineto{\pgfqpoint{1.348571in}{1.736728in}}%
\pgfpathlineto{\pgfqpoint{1.349007in}{1.736763in}}%
\pgfpathlineto{\pgfqpoint{1.349446in}{1.736792in}}%
\pgfpathlineto{\pgfqpoint{1.349881in}{1.735593in}}%
\pgfpathlineto{\pgfqpoint{1.350316in}{1.734273in}}%
\pgfpathlineto{\pgfqpoint{1.350752in}{1.732832in}}%
\pgfpathlineto{\pgfqpoint{1.351188in}{1.731270in}}%
\pgfpathlineto{\pgfqpoint{1.351188in}{1.731270in}}%
\pgfpathlineto{\pgfqpoint{1.351188in}{1.731270in}}%
\pgfpathlineto{\pgfqpoint{1.351188in}{1.731270in}}%
\pgfpathlineto{\pgfqpoint{1.351188in}{1.731270in}}%
\pgfpathclose%
\pgfusepath{fill}%
\end{pgfscope}%
\begin{pgfscope}%
\pgfpathrectangle{\pgfqpoint{0.329460in}{0.284240in}}{\pgfqpoint{1.989680in}{1.989680in}}%
\pgfusepath{clip}%
\pgfsetbuttcap%
\pgfsetroundjoin%
\definecolor{currentfill}{rgb}{0.993248,0.906157,0.143936}%
\pgfsetfillcolor{currentfill}%
\pgfsetlinewidth{0.000000pt}%
\definecolor{currentstroke}{rgb}{0.000000,0.000000,0.000000}%
\pgfsetstrokecolor{currentstroke}%
\pgfsetdash{}{0pt}%
\pgfpathmoveto{\pgfqpoint{1.351188in}{1.731270in}}%
\pgfpathlineto{\pgfqpoint{1.351636in}{1.732831in}}%
\pgfpathlineto{\pgfqpoint{1.352083in}{1.734271in}}%
\pgfpathlineto{\pgfqpoint{1.352531in}{1.735591in}}%
\pgfpathlineto{\pgfqpoint{1.352978in}{1.736789in}}%
\pgfpathlineto{\pgfqpoint{1.353416in}{1.736760in}}%
\pgfpathlineto{\pgfqpoint{1.353852in}{1.736724in}}%
\pgfpathlineto{\pgfqpoint{1.354286in}{1.736682in}}%
\pgfpathlineto{\pgfqpoint{1.354716in}{1.736633in}}%
\pgfpathlineto{\pgfqpoint{1.353835in}{1.735473in}}%
\pgfpathlineto{\pgfqpoint{1.352953in}{1.734193in}}%
\pgfpathlineto{\pgfqpoint{1.352070in}{1.732792in}}%
\pgfpathlineto{\pgfqpoint{1.351188in}{1.731270in}}%
\pgfpathlineto{\pgfqpoint{1.351188in}{1.731270in}}%
\pgfpathlineto{\pgfqpoint{1.351188in}{1.731270in}}%
\pgfpathlineto{\pgfqpoint{1.351188in}{1.731270in}}%
\pgfpathlineto{\pgfqpoint{1.351188in}{1.731270in}}%
\pgfpathclose%
\pgfusepath{fill}%
\end{pgfscope}%
\begin{pgfscope}%
\pgfpathrectangle{\pgfqpoint{0.329460in}{0.284240in}}{\pgfqpoint{1.989680in}{1.989680in}}%
\pgfusepath{clip}%
\pgfsetbuttcap%
\pgfsetroundjoin%
\definecolor{currentfill}{rgb}{0.487026,0.823929,0.312321}%
\pgfsetfillcolor{currentfill}%
\pgfsetlinewidth{0.000000pt}%
\definecolor{currentstroke}{rgb}{0.000000,0.000000,0.000000}%
\pgfsetstrokecolor{currentstroke}%
\pgfsetdash{}{0pt}%
\pgfpathmoveto{\pgfqpoint{1.209694in}{1.556918in}}%
\pgfpathlineto{\pgfqpoint{1.206764in}{1.550796in}}%
\pgfpathlineto{\pgfqpoint{1.203836in}{1.544593in}}%
\pgfpathlineto{\pgfqpoint{1.200908in}{1.538308in}}%
\pgfpathlineto{\pgfqpoint{1.197982in}{1.531945in}}%
\pgfpathlineto{\pgfqpoint{1.194989in}{1.534298in}}%
\pgfpathlineto{\pgfqpoint{1.192156in}{1.536695in}}%
\pgfpathlineto{\pgfqpoint{1.189484in}{1.539132in}}%
\pgfpathlineto{\pgfqpoint{1.186977in}{1.541608in}}%
\pgfpathlineto{\pgfqpoint{1.190108in}{1.547777in}}%
\pgfpathlineto{\pgfqpoint{1.193241in}{1.553867in}}%
\pgfpathlineto{\pgfqpoint{1.196375in}{1.559877in}}%
\pgfpathlineto{\pgfqpoint{1.199512in}{1.565806in}}%
\pgfpathlineto{\pgfqpoint{1.201832in}{1.563529in}}%
\pgfpathlineto{\pgfqpoint{1.204305in}{1.561287in}}%
\pgfpathlineto{\pgfqpoint{1.206926in}{1.559082in}}%
\pgfpathlineto{\pgfqpoint{1.209694in}{1.556918in}}%
\pgfpathclose%
\pgfusepath{fill}%
\end{pgfscope}%
\begin{pgfscope}%
\pgfpathrectangle{\pgfqpoint{0.329460in}{0.284240in}}{\pgfqpoint{1.989680in}{1.989680in}}%
\pgfusepath{clip}%
\pgfsetbuttcap%
\pgfsetroundjoin%
\definecolor{currentfill}{rgb}{0.993248,0.906157,0.143936}%
\pgfsetfillcolor{currentfill}%
\pgfsetlinewidth{0.000000pt}%
\definecolor{currentstroke}{rgb}{0.000000,0.000000,0.000000}%
\pgfsetstrokecolor{currentstroke}%
\pgfsetdash{}{0pt}%
\pgfpathmoveto{\pgfqpoint{1.351188in}{1.731270in}}%
\pgfpathlineto{\pgfqpoint{1.352070in}{1.732792in}}%
\pgfpathlineto{\pgfqpoint{1.352953in}{1.734193in}}%
\pgfpathlineto{\pgfqpoint{1.353835in}{1.735473in}}%
\pgfpathlineto{\pgfqpoint{1.354716in}{1.736633in}}%
\pgfpathlineto{\pgfqpoint{1.355143in}{1.736578in}}%
\pgfpathlineto{\pgfqpoint{1.355565in}{1.736517in}}%
\pgfpathlineto{\pgfqpoint{1.355984in}{1.736449in}}%
\pgfpathlineto{\pgfqpoint{1.354786in}{1.735335in}}%
\pgfpathlineto{\pgfqpoint{1.353587in}{1.734101in}}%
\pgfpathlineto{\pgfqpoint{1.352388in}{1.732746in}}%
\pgfpathlineto{\pgfqpoint{1.351188in}{1.731270in}}%
\pgfpathlineto{\pgfqpoint{1.351188in}{1.731270in}}%
\pgfpathlineto{\pgfqpoint{1.351188in}{1.731270in}}%
\pgfpathlineto{\pgfqpoint{1.351188in}{1.731270in}}%
\pgfpathclose%
\pgfusepath{fill}%
\end{pgfscope}%
\begin{pgfscope}%
\pgfpathrectangle{\pgfqpoint{0.329460in}{0.284240in}}{\pgfqpoint{1.989680in}{1.989680in}}%
\pgfusepath{clip}%
\pgfsetbuttcap%
\pgfsetroundjoin%
\definecolor{currentfill}{rgb}{0.993248,0.906157,0.143936}%
\pgfsetfillcolor{currentfill}%
\pgfsetlinewidth{0.000000pt}%
\definecolor{currentstroke}{rgb}{0.000000,0.000000,0.000000}%
\pgfsetstrokecolor{currentstroke}%
\pgfsetdash{}{0pt}%
\pgfpathmoveto{\pgfqpoint{1.351188in}{1.731270in}}%
\pgfpathlineto{\pgfqpoint{1.349895in}{1.732730in}}%
\pgfpathlineto{\pgfqpoint{1.348604in}{1.734068in}}%
\pgfpathlineto{\pgfqpoint{1.347313in}{1.735287in}}%
\pgfpathlineto{\pgfqpoint{1.346023in}{1.736384in}}%
\pgfpathlineto{\pgfqpoint{1.346437in}{1.736457in}}%
\pgfpathlineto{\pgfqpoint{1.346856in}{1.736524in}}%
\pgfpathlineto{\pgfqpoint{1.347280in}{1.736584in}}%
\pgfpathlineto{\pgfqpoint{1.347707in}{1.736639in}}%
\pgfpathlineto{\pgfqpoint{1.348576in}{1.735477in}}%
\pgfpathlineto{\pgfqpoint{1.349446in}{1.734196in}}%
\pgfpathlineto{\pgfqpoint{1.350317in}{1.732793in}}%
\pgfpathlineto{\pgfqpoint{1.351188in}{1.731270in}}%
\pgfpathlineto{\pgfqpoint{1.351188in}{1.731270in}}%
\pgfpathlineto{\pgfqpoint{1.351188in}{1.731270in}}%
\pgfpathlineto{\pgfqpoint{1.351188in}{1.731270in}}%
\pgfpathlineto{\pgfqpoint{1.351188in}{1.731270in}}%
\pgfpathclose%
\pgfusepath{fill}%
\end{pgfscope}%
\begin{pgfscope}%
\pgfpathrectangle{\pgfqpoint{0.329460in}{0.284240in}}{\pgfqpoint{1.989680in}{1.989680in}}%
\pgfusepath{clip}%
\pgfsetbuttcap%
\pgfsetroundjoin%
\definecolor{currentfill}{rgb}{0.993248,0.906157,0.143936}%
\pgfsetfillcolor{currentfill}%
\pgfsetlinewidth{0.000000pt}%
\definecolor{currentstroke}{rgb}{0.000000,0.000000,0.000000}%
\pgfsetstrokecolor{currentstroke}%
\pgfsetdash{}{0pt}%
\pgfpathmoveto{\pgfqpoint{1.351188in}{1.731270in}}%
\pgfpathlineto{\pgfqpoint{1.352388in}{1.732746in}}%
\pgfpathlineto{\pgfqpoint{1.353587in}{1.734101in}}%
\pgfpathlineto{\pgfqpoint{1.354786in}{1.735335in}}%
\pgfpathlineto{\pgfqpoint{1.355984in}{1.736449in}}%
\pgfpathlineto{\pgfqpoint{1.356398in}{1.736376in}}%
\pgfpathlineto{\pgfqpoint{1.356806in}{1.736296in}}%
\pgfpathlineto{\pgfqpoint{1.357209in}{1.736211in}}%
\pgfpathlineto{\pgfqpoint{1.357606in}{1.736119in}}%
\pgfpathlineto{\pgfqpoint{1.356003in}{1.735088in}}%
\pgfpathlineto{\pgfqpoint{1.354398in}{1.733936in}}%
\pgfpathlineto{\pgfqpoint{1.352793in}{1.732663in}}%
\pgfpathlineto{\pgfqpoint{1.351188in}{1.731270in}}%
\pgfpathlineto{\pgfqpoint{1.351188in}{1.731270in}}%
\pgfpathlineto{\pgfqpoint{1.351188in}{1.731270in}}%
\pgfpathlineto{\pgfqpoint{1.351188in}{1.731270in}}%
\pgfpathlineto{\pgfqpoint{1.351188in}{1.731270in}}%
\pgfpathclose%
\pgfusepath{fill}%
\end{pgfscope}%
\begin{pgfscope}%
\pgfpathrectangle{\pgfqpoint{0.329460in}{0.284240in}}{\pgfqpoint{1.989680in}{1.989680in}}%
\pgfusepath{clip}%
\pgfsetbuttcap%
\pgfsetroundjoin%
\definecolor{currentfill}{rgb}{0.993248,0.906157,0.143936}%
\pgfsetfillcolor{currentfill}%
\pgfsetlinewidth{0.000000pt}%
\definecolor{currentstroke}{rgb}{0.000000,0.000000,0.000000}%
\pgfsetstrokecolor{currentstroke}%
\pgfsetdash{}{0pt}%
\pgfpathmoveto{\pgfqpoint{1.337387in}{1.731502in}}%
\pgfpathlineto{\pgfqpoint{1.333938in}{1.731261in}}%
\pgfpathlineto{\pgfqpoint{1.330490in}{1.730900in}}%
\pgfpathlineto{\pgfqpoint{1.327043in}{1.730420in}}%
\pgfpathlineto{\pgfqpoint{1.323597in}{1.729820in}}%
\pgfpathlineto{\pgfqpoint{1.323774in}{1.730225in}}%
\pgfpathlineto{\pgfqpoint{1.323978in}{1.730627in}}%
\pgfpathlineto{\pgfqpoint{1.324210in}{1.731025in}}%
\pgfpathlineto{\pgfqpoint{1.324468in}{1.731420in}}%
\pgfpathlineto{\pgfqpoint{1.327805in}{1.731819in}}%
\pgfpathlineto{\pgfqpoint{1.331143in}{1.732100in}}%
\pgfpathlineto{\pgfqpoint{1.334481in}{1.732261in}}%
\pgfpathlineto{\pgfqpoint{1.337821in}{1.732302in}}%
\pgfpathlineto{\pgfqpoint{1.337692in}{1.732105in}}%
\pgfpathlineto{\pgfqpoint{1.337577in}{1.731906in}}%
\pgfpathlineto{\pgfqpoint{1.337475in}{1.731705in}}%
\pgfpathlineto{\pgfqpoint{1.337387in}{1.731502in}}%
\pgfpathclose%
\pgfusepath{fill}%
\end{pgfscope}%
\begin{pgfscope}%
\pgfpathrectangle{\pgfqpoint{0.329460in}{0.284240in}}{\pgfqpoint{1.989680in}{1.989680in}}%
\pgfusepath{clip}%
\pgfsetbuttcap%
\pgfsetroundjoin%
\definecolor{currentfill}{rgb}{0.993248,0.906157,0.143936}%
\pgfsetfillcolor{currentfill}%
\pgfsetlinewidth{0.000000pt}%
\definecolor{currentstroke}{rgb}{0.000000,0.000000,0.000000}%
\pgfsetstrokecolor{currentstroke}%
\pgfsetdash{}{0pt}%
\pgfpathmoveto{\pgfqpoint{1.351188in}{1.731270in}}%
\pgfpathlineto{\pgfqpoint{1.349495in}{1.732642in}}%
\pgfpathlineto{\pgfqpoint{1.347803in}{1.733893in}}%
\pgfpathlineto{\pgfqpoint{1.346112in}{1.735023in}}%
\pgfpathlineto{\pgfqpoint{1.344422in}{1.736033in}}%
\pgfpathlineto{\pgfqpoint{1.344813in}{1.736130in}}%
\pgfpathlineto{\pgfqpoint{1.345211in}{1.736220in}}%
\pgfpathlineto{\pgfqpoint{1.345614in}{1.736305in}}%
\pgfpathlineto{\pgfqpoint{1.346023in}{1.736384in}}%
\pgfpathlineto{\pgfqpoint{1.347313in}{1.735287in}}%
\pgfpathlineto{\pgfqpoint{1.348604in}{1.734068in}}%
\pgfpathlineto{\pgfqpoint{1.349895in}{1.732730in}}%
\pgfpathlineto{\pgfqpoint{1.351188in}{1.731270in}}%
\pgfpathlineto{\pgfqpoint{1.351188in}{1.731270in}}%
\pgfpathlineto{\pgfqpoint{1.351188in}{1.731270in}}%
\pgfpathlineto{\pgfqpoint{1.351188in}{1.731270in}}%
\pgfpathlineto{\pgfqpoint{1.351188in}{1.731270in}}%
\pgfpathclose%
\pgfusepath{fill}%
\end{pgfscope}%
\begin{pgfscope}%
\pgfpathrectangle{\pgfqpoint{0.329460in}{0.284240in}}{\pgfqpoint{1.989680in}{1.989680in}}%
\pgfusepath{clip}%
\pgfsetbuttcap%
\pgfsetroundjoin%
\definecolor{currentfill}{rgb}{0.274952,0.037752,0.364543}%
\pgfsetfillcolor{currentfill}%
\pgfsetlinewidth{0.000000pt}%
\definecolor{currentstroke}{rgb}{0.000000,0.000000,0.000000}%
\pgfsetstrokecolor{currentstroke}%
\pgfsetdash{}{0pt}%
\pgfpathmoveto{\pgfqpoint{1.175813in}{0.818134in}}%
\pgfpathlineto{\pgfqpoint{1.174542in}{0.813422in}}%
\pgfpathlineto{\pgfqpoint{1.173270in}{0.808877in}}%
\pgfpathlineto{\pgfqpoint{1.171995in}{0.804503in}}%
\pgfpathlineto{\pgfqpoint{1.170719in}{0.800303in}}%
\pgfpathlineto{\pgfqpoint{1.155986in}{0.803510in}}%
\pgfpathlineto{\pgfqpoint{1.141470in}{0.806964in}}%
\pgfpathlineto{\pgfqpoint{1.127186in}{0.810662in}}%
\pgfpathlineto{\pgfqpoint{1.113149in}{0.814599in}}%
\pgfpathlineto{\pgfqpoint{1.114829in}{0.818689in}}%
\pgfpathlineto{\pgfqpoint{1.116507in}{0.822953in}}%
\pgfpathlineto{\pgfqpoint{1.118182in}{0.827388in}}%
\pgfpathlineto{\pgfqpoint{1.119855in}{0.831990in}}%
\pgfpathlineto{\pgfqpoint{1.133499in}{0.828173in}}%
\pgfpathlineto{\pgfqpoint{1.147384in}{0.824589in}}%
\pgfpathlineto{\pgfqpoint{1.161494in}{0.821241in}}%
\pgfpathlineto{\pgfqpoint{1.175813in}{0.818134in}}%
\pgfpathclose%
\pgfusepath{fill}%
\end{pgfscope}%
\begin{pgfscope}%
\pgfpathrectangle{\pgfqpoint{0.329460in}{0.284240in}}{\pgfqpoint{1.989680in}{1.989680in}}%
\pgfusepath{clip}%
\pgfsetbuttcap%
\pgfsetroundjoin%
\definecolor{currentfill}{rgb}{0.271305,0.019942,0.347269}%
\pgfsetfillcolor{currentfill}%
\pgfsetlinewidth{0.000000pt}%
\definecolor{currentstroke}{rgb}{0.000000,0.000000,0.000000}%
\pgfsetstrokecolor{currentstroke}%
\pgfsetdash{}{0pt}%
\pgfpathmoveto{\pgfqpoint{1.601483in}{0.818296in}}%
\pgfpathlineto{\pgfqpoint{1.603252in}{0.814413in}}%
\pgfpathlineto{\pgfqpoint{1.605023in}{0.810712in}}%
\pgfpathlineto{\pgfqpoint{1.606798in}{0.807198in}}%
\pgfpathlineto{\pgfqpoint{1.608576in}{0.803875in}}%
\pgfpathlineto{\pgfqpoint{1.594386in}{0.799602in}}%
\pgfpathlineto{\pgfqpoint{1.579926in}{0.795572in}}%
\pgfpathlineto{\pgfqpoint{1.565212in}{0.791788in}}%
\pgfpathlineto{\pgfqpoint{1.550261in}{0.788256in}}%
\pgfpathlineto{\pgfqpoint{1.548883in}{0.791695in}}%
\pgfpathlineto{\pgfqpoint{1.547507in}{0.795325in}}%
\pgfpathlineto{\pgfqpoint{1.546134in}{0.799142in}}%
\pgfpathlineto{\pgfqpoint{1.544763in}{0.803141in}}%
\pgfpathlineto{\pgfqpoint{1.559304in}{0.806568in}}%
\pgfpathlineto{\pgfqpoint{1.573614in}{0.810239in}}%
\pgfpathlineto{\pgfqpoint{1.587679in}{0.814150in}}%
\pgfpathlineto{\pgfqpoint{1.601483in}{0.818296in}}%
\pgfpathclose%
\pgfusepath{fill}%
\end{pgfscope}%
\begin{pgfscope}%
\pgfpathrectangle{\pgfqpoint{0.329460in}{0.284240in}}{\pgfqpoint{1.989680in}{1.989680in}}%
\pgfusepath{clip}%
\pgfsetbuttcap%
\pgfsetroundjoin%
\definecolor{currentfill}{rgb}{0.993248,0.906157,0.143936}%
\pgfsetfillcolor{currentfill}%
\pgfsetlinewidth{0.000000pt}%
\definecolor{currentstroke}{rgb}{0.000000,0.000000,0.000000}%
\pgfsetstrokecolor{currentstroke}%
\pgfsetdash{}{0pt}%
\pgfpathmoveto{\pgfqpoint{1.351188in}{1.731270in}}%
\pgfpathlineto{\pgfqpoint{1.352793in}{1.732663in}}%
\pgfpathlineto{\pgfqpoint{1.354398in}{1.733936in}}%
\pgfpathlineto{\pgfqpoint{1.356003in}{1.735088in}}%
\pgfpathlineto{\pgfqpoint{1.357606in}{1.736119in}}%
\pgfpathlineto{\pgfqpoint{1.357997in}{1.736022in}}%
\pgfpathlineto{\pgfqpoint{1.358380in}{1.735919in}}%
\pgfpathlineto{\pgfqpoint{1.358757in}{1.735811in}}%
\pgfpathlineto{\pgfqpoint{1.359126in}{1.735697in}}%
\pgfpathlineto{\pgfqpoint{1.357143in}{1.734771in}}%
\pgfpathlineto{\pgfqpoint{1.355159in}{1.733724in}}%
\pgfpathlineto{\pgfqpoint{1.353174in}{1.732558in}}%
\pgfpathlineto{\pgfqpoint{1.351188in}{1.731270in}}%
\pgfpathlineto{\pgfqpoint{1.351188in}{1.731270in}}%
\pgfpathlineto{\pgfqpoint{1.351188in}{1.731270in}}%
\pgfpathlineto{\pgfqpoint{1.351188in}{1.731270in}}%
\pgfpathlineto{\pgfqpoint{1.351188in}{1.731270in}}%
\pgfpathclose%
\pgfusepath{fill}%
\end{pgfscope}%
\begin{pgfscope}%
\pgfpathrectangle{\pgfqpoint{0.329460in}{0.284240in}}{\pgfqpoint{1.989680in}{1.989680in}}%
\pgfusepath{clip}%
\pgfsetbuttcap%
\pgfsetroundjoin%
\definecolor{currentfill}{rgb}{0.993248,0.906157,0.143936}%
\pgfsetfillcolor{currentfill}%
\pgfsetlinewidth{0.000000pt}%
\definecolor{currentstroke}{rgb}{0.000000,0.000000,0.000000}%
\pgfsetstrokecolor{currentstroke}%
\pgfsetdash{}{0pt}%
\pgfpathmoveto{\pgfqpoint{1.351188in}{1.731270in}}%
\pgfpathlineto{\pgfqpoint{1.349121in}{1.732531in}}%
\pgfpathlineto{\pgfqpoint{1.347056in}{1.733671in}}%
\pgfpathlineto{\pgfqpoint{1.344991in}{1.734691in}}%
\pgfpathlineto{\pgfqpoint{1.342928in}{1.735591in}}%
\pgfpathlineto{\pgfqpoint{1.343290in}{1.735710in}}%
\pgfpathlineto{\pgfqpoint{1.343660in}{1.735823in}}%
\pgfpathlineto{\pgfqpoint{1.344037in}{1.735931in}}%
\pgfpathlineto{\pgfqpoint{1.344422in}{1.736033in}}%
\pgfpathlineto{\pgfqpoint{1.346112in}{1.735023in}}%
\pgfpathlineto{\pgfqpoint{1.347803in}{1.733893in}}%
\pgfpathlineto{\pgfqpoint{1.349495in}{1.732642in}}%
\pgfpathlineto{\pgfqpoint{1.351188in}{1.731270in}}%
\pgfpathlineto{\pgfqpoint{1.351188in}{1.731270in}}%
\pgfpathlineto{\pgfqpoint{1.351188in}{1.731270in}}%
\pgfpathlineto{\pgfqpoint{1.351188in}{1.731270in}}%
\pgfpathlineto{\pgfqpoint{1.351188in}{1.731270in}}%
\pgfpathclose%
\pgfusepath{fill}%
\end{pgfscope}%
\begin{pgfscope}%
\pgfpathrectangle{\pgfqpoint{0.329460in}{0.284240in}}{\pgfqpoint{1.989680in}{1.989680in}}%
\pgfusepath{clip}%
\pgfsetbuttcap%
\pgfsetroundjoin%
\definecolor{currentfill}{rgb}{0.935904,0.898570,0.108131}%
\pgfsetfillcolor{currentfill}%
\pgfsetlinewidth{0.000000pt}%
\definecolor{currentstroke}{rgb}{0.000000,0.000000,0.000000}%
\pgfsetstrokecolor{currentstroke}%
\pgfsetdash{}{0pt}%
\pgfpathmoveto{\pgfqpoint{1.407262in}{1.714926in}}%
\pgfpathlineto{\pgfqpoint{1.410761in}{1.712904in}}%
\pgfpathlineto{\pgfqpoint{1.414260in}{1.710767in}}%
\pgfpathlineto{\pgfqpoint{1.417757in}{1.708515in}}%
\pgfpathlineto{\pgfqpoint{1.421253in}{1.706150in}}%
\pgfpathlineto{\pgfqpoint{1.421139in}{1.705116in}}%
\pgfpathlineto{\pgfqpoint{1.420955in}{1.704084in}}%
\pgfpathlineto{\pgfqpoint{1.420702in}{1.703055in}}%
\pgfpathlineto{\pgfqpoint{1.420380in}{1.702030in}}%
\pgfpathlineto{\pgfqpoint{1.416925in}{1.704602in}}%
\pgfpathlineto{\pgfqpoint{1.413469in}{1.707061in}}%
\pgfpathlineto{\pgfqpoint{1.410013in}{1.709405in}}%
\pgfpathlineto{\pgfqpoint{1.406556in}{1.711633in}}%
\pgfpathlineto{\pgfqpoint{1.406816in}{1.712452in}}%
\pgfpathlineto{\pgfqpoint{1.407020in}{1.713275in}}%
\pgfpathlineto{\pgfqpoint{1.407169in}{1.714099in}}%
\pgfpathlineto{\pgfqpoint{1.407262in}{1.714926in}}%
\pgfpathclose%
\pgfusepath{fill}%
\end{pgfscope}%
\begin{pgfscope}%
\pgfpathrectangle{\pgfqpoint{0.329460in}{0.284240in}}{\pgfqpoint{1.989680in}{1.989680in}}%
\pgfusepath{clip}%
\pgfsetbuttcap%
\pgfsetroundjoin%
\definecolor{currentfill}{rgb}{0.344074,0.780029,0.397381}%
\pgfsetfillcolor{currentfill}%
\pgfsetlinewidth{0.000000pt}%
\definecolor{currentstroke}{rgb}{0.000000,0.000000,0.000000}%
\pgfsetstrokecolor{currentstroke}%
\pgfsetdash{}{0pt}%
\pgfpathmoveto{\pgfqpoint{1.518949in}{1.507998in}}%
\pgfpathlineto{\pgfqpoint{1.521917in}{1.501309in}}%
\pgfpathlineto{\pgfqpoint{1.524884in}{1.494551in}}%
\pgfpathlineto{\pgfqpoint{1.527849in}{1.487727in}}%
\pgfpathlineto{\pgfqpoint{1.530812in}{1.480838in}}%
\pgfpathlineto{\pgfqpoint{1.527353in}{1.478108in}}%
\pgfpathlineto{\pgfqpoint{1.523715in}{1.475431in}}%
\pgfpathlineto{\pgfqpoint{1.519899in}{1.472810in}}%
\pgfpathlineto{\pgfqpoint{1.515910in}{1.470248in}}%
\pgfpathlineto{\pgfqpoint{1.513188in}{1.477324in}}%
\pgfpathlineto{\pgfqpoint{1.510464in}{1.484335in}}%
\pgfpathlineto{\pgfqpoint{1.507740in}{1.491278in}}%
\pgfpathlineto{\pgfqpoint{1.505013in}{1.498153in}}%
\pgfpathlineto{\pgfqpoint{1.508743in}{1.500535in}}%
\pgfpathlineto{\pgfqpoint{1.512311in}{1.502971in}}%
\pgfpathlineto{\pgfqpoint{1.515714in}{1.505460in}}%
\pgfpathlineto{\pgfqpoint{1.518949in}{1.507998in}}%
\pgfpathclose%
\pgfusepath{fill}%
\end{pgfscope}%
\begin{pgfscope}%
\pgfpathrectangle{\pgfqpoint{0.329460in}{0.284240in}}{\pgfqpoint{1.989680in}{1.989680in}}%
\pgfusepath{clip}%
\pgfsetbuttcap%
\pgfsetroundjoin%
\definecolor{currentfill}{rgb}{0.993248,0.906157,0.143936}%
\pgfsetfillcolor{currentfill}%
\pgfsetlinewidth{0.000000pt}%
\definecolor{currentstroke}{rgb}{0.000000,0.000000,0.000000}%
\pgfsetstrokecolor{currentstroke}%
\pgfsetdash{}{0pt}%
\pgfpathmoveto{\pgfqpoint{1.351188in}{1.731270in}}%
\pgfpathlineto{\pgfqpoint{1.353174in}{1.732558in}}%
\pgfpathlineto{\pgfqpoint{1.355159in}{1.733724in}}%
\pgfpathlineto{\pgfqpoint{1.357143in}{1.734771in}}%
\pgfpathlineto{\pgfqpoint{1.359126in}{1.735697in}}%
\pgfpathlineto{\pgfqpoint{1.359487in}{1.735578in}}%
\pgfpathlineto{\pgfqpoint{1.359840in}{1.735453in}}%
\pgfpathlineto{\pgfqpoint{1.360185in}{1.735323in}}%
\pgfpathlineto{\pgfqpoint{1.360520in}{1.735189in}}%
\pgfpathlineto{\pgfqpoint{1.358189in}{1.734389in}}%
\pgfpathlineto{\pgfqpoint{1.355856in}{1.733470in}}%
\pgfpathlineto{\pgfqpoint{1.353522in}{1.732430in}}%
\pgfpathlineto{\pgfqpoint{1.351188in}{1.731270in}}%
\pgfpathlineto{\pgfqpoint{1.351188in}{1.731270in}}%
\pgfpathlineto{\pgfqpoint{1.351188in}{1.731270in}}%
\pgfpathlineto{\pgfqpoint{1.351188in}{1.731270in}}%
\pgfpathlineto{\pgfqpoint{1.351188in}{1.731270in}}%
\pgfpathclose%
\pgfusepath{fill}%
\end{pgfscope}%
\begin{pgfscope}%
\pgfpathrectangle{\pgfqpoint{0.329460in}{0.284240in}}{\pgfqpoint{1.989680in}{1.989680in}}%
\pgfusepath{clip}%
\pgfsetbuttcap%
\pgfsetroundjoin%
\definecolor{currentfill}{rgb}{0.993248,0.906157,0.143936}%
\pgfsetfillcolor{currentfill}%
\pgfsetlinewidth{0.000000pt}%
\definecolor{currentstroke}{rgb}{0.000000,0.000000,0.000000}%
\pgfsetstrokecolor{currentstroke}%
\pgfsetdash{}{0pt}%
\pgfpathmoveto{\pgfqpoint{1.351188in}{1.731270in}}%
\pgfpathlineto{\pgfqpoint{1.348780in}{1.732399in}}%
\pgfpathlineto{\pgfqpoint{1.346374in}{1.733408in}}%
\pgfpathlineto{\pgfqpoint{1.343969in}{1.734296in}}%
\pgfpathlineto{\pgfqpoint{1.341565in}{1.735065in}}%
\pgfpathlineto{\pgfqpoint{1.341892in}{1.735204in}}%
\pgfpathlineto{\pgfqpoint{1.342228in}{1.735338in}}%
\pgfpathlineto{\pgfqpoint{1.342574in}{1.735467in}}%
\pgfpathlineto{\pgfqpoint{1.342928in}{1.735591in}}%
\pgfpathlineto{\pgfqpoint{1.344991in}{1.734691in}}%
\pgfpathlineto{\pgfqpoint{1.347056in}{1.733671in}}%
\pgfpathlineto{\pgfqpoint{1.349121in}{1.732531in}}%
\pgfpathlineto{\pgfqpoint{1.351188in}{1.731270in}}%
\pgfpathlineto{\pgfqpoint{1.351188in}{1.731270in}}%
\pgfpathlineto{\pgfqpoint{1.351188in}{1.731270in}}%
\pgfpathlineto{\pgfqpoint{1.351188in}{1.731270in}}%
\pgfpathlineto{\pgfqpoint{1.351188in}{1.731270in}}%
\pgfpathclose%
\pgfusepath{fill}%
\end{pgfscope}%
\begin{pgfscope}%
\pgfpathrectangle{\pgfqpoint{0.329460in}{0.284240in}}{\pgfqpoint{1.989680in}{1.989680in}}%
\pgfusepath{clip}%
\pgfsetbuttcap%
\pgfsetroundjoin%
\definecolor{currentfill}{rgb}{0.855810,0.888601,0.097452}%
\pgfsetfillcolor{currentfill}%
\pgfsetlinewidth{0.000000pt}%
\definecolor{currentstroke}{rgb}{0.000000,0.000000,0.000000}%
\pgfsetstrokecolor{currentstroke}%
\pgfsetdash{}{0pt}%
\pgfpathmoveto{\pgfqpoint{1.434189in}{1.690612in}}%
\pgfpathlineto{\pgfqpoint{1.437638in}{1.687479in}}%
\pgfpathlineto{\pgfqpoint{1.441086in}{1.684237in}}%
\pgfpathlineto{\pgfqpoint{1.444533in}{1.680886in}}%
\pgfpathlineto{\pgfqpoint{1.447979in}{1.677427in}}%
\pgfpathlineto{\pgfqpoint{1.447438in}{1.675994in}}%
\pgfpathlineto{\pgfqpoint{1.446800in}{1.674570in}}%
\pgfpathlineto{\pgfqpoint{1.446066in}{1.673155in}}%
\pgfpathlineto{\pgfqpoint{1.445238in}{1.671752in}}%
\pgfpathlineto{\pgfqpoint{1.441886in}{1.675416in}}%
\pgfpathlineto{\pgfqpoint{1.438534in}{1.678972in}}%
\pgfpathlineto{\pgfqpoint{1.435181in}{1.682419in}}%
\pgfpathlineto{\pgfqpoint{1.431827in}{1.685757in}}%
\pgfpathlineto{\pgfqpoint{1.432540in}{1.686957in}}%
\pgfpathlineto{\pgfqpoint{1.433172in}{1.688167in}}%
\pgfpathlineto{\pgfqpoint{1.433722in}{1.689386in}}%
\pgfpathlineto{\pgfqpoint{1.434189in}{1.690612in}}%
\pgfpathclose%
\pgfusepath{fill}%
\end{pgfscope}%
\begin{pgfscope}%
\pgfpathrectangle{\pgfqpoint{0.329460in}{0.284240in}}{\pgfqpoint{1.989680in}{1.989680in}}%
\pgfusepath{clip}%
\pgfsetbuttcap%
\pgfsetroundjoin%
\definecolor{currentfill}{rgb}{0.993248,0.906157,0.143936}%
\pgfsetfillcolor{currentfill}%
\pgfsetlinewidth{0.000000pt}%
\definecolor{currentstroke}{rgb}{0.000000,0.000000,0.000000}%
\pgfsetstrokecolor{currentstroke}%
\pgfsetdash{}{0pt}%
\pgfpathmoveto{\pgfqpoint{1.351188in}{1.731270in}}%
\pgfpathlineto{\pgfqpoint{1.353522in}{1.732430in}}%
\pgfpathlineto{\pgfqpoint{1.355856in}{1.733470in}}%
\pgfpathlineto{\pgfqpoint{1.358189in}{1.734389in}}%
\pgfpathlineto{\pgfqpoint{1.360520in}{1.735189in}}%
\pgfpathlineto{\pgfqpoint{1.360846in}{1.735049in}}%
\pgfpathlineto{\pgfqpoint{1.361163in}{1.734905in}}%
\pgfpathlineto{\pgfqpoint{1.361469in}{1.734756in}}%
\pgfpathlineto{\pgfqpoint{1.361766in}{1.734603in}}%
\pgfpathlineto{\pgfqpoint{1.359123in}{1.733950in}}%
\pgfpathlineto{\pgfqpoint{1.356479in}{1.733177in}}%
\pgfpathlineto{\pgfqpoint{1.353834in}{1.732284in}}%
\pgfpathlineto{\pgfqpoint{1.351188in}{1.731270in}}%
\pgfpathlineto{\pgfqpoint{1.351188in}{1.731270in}}%
\pgfpathlineto{\pgfqpoint{1.351188in}{1.731270in}}%
\pgfpathlineto{\pgfqpoint{1.351188in}{1.731270in}}%
\pgfpathlineto{\pgfqpoint{1.351188in}{1.731270in}}%
\pgfpathclose%
\pgfusepath{fill}%
\end{pgfscope}%
\begin{pgfscope}%
\pgfpathrectangle{\pgfqpoint{0.329460in}{0.284240in}}{\pgfqpoint{1.989680in}{1.989680in}}%
\pgfusepath{clip}%
\pgfsetbuttcap%
\pgfsetroundjoin%
\definecolor{currentfill}{rgb}{0.993248,0.906157,0.143936}%
\pgfsetfillcolor{currentfill}%
\pgfsetlinewidth{0.000000pt}%
\definecolor{currentstroke}{rgb}{0.000000,0.000000,0.000000}%
\pgfsetstrokecolor{currentstroke}%
\pgfsetdash{}{0pt}%
\pgfpathmoveto{\pgfqpoint{1.351188in}{1.731270in}}%
\pgfpathlineto{\pgfqpoint{1.348478in}{1.732249in}}%
\pgfpathlineto{\pgfqpoint{1.345769in}{1.733107in}}%
\pgfpathlineto{\pgfqpoint{1.343061in}{1.733845in}}%
\pgfpathlineto{\pgfqpoint{1.340355in}{1.734463in}}%
\pgfpathlineto{\pgfqpoint{1.340642in}{1.734620in}}%
\pgfpathlineto{\pgfqpoint{1.340939in}{1.734773in}}%
\pgfpathlineto{\pgfqpoint{1.341247in}{1.734921in}}%
\pgfpathlineto{\pgfqpoint{1.341565in}{1.735065in}}%
\pgfpathlineto{\pgfqpoint{1.343969in}{1.734296in}}%
\pgfpathlineto{\pgfqpoint{1.346374in}{1.733408in}}%
\pgfpathlineto{\pgfqpoint{1.348780in}{1.732399in}}%
\pgfpathlineto{\pgfqpoint{1.351188in}{1.731270in}}%
\pgfpathlineto{\pgfqpoint{1.351188in}{1.731270in}}%
\pgfpathlineto{\pgfqpoint{1.351188in}{1.731270in}}%
\pgfpathlineto{\pgfqpoint{1.351188in}{1.731270in}}%
\pgfpathlineto{\pgfqpoint{1.351188in}{1.731270in}}%
\pgfpathclose%
\pgfusepath{fill}%
\end{pgfscope}%
\begin{pgfscope}%
\pgfpathrectangle{\pgfqpoint{0.329460in}{0.284240in}}{\pgfqpoint{1.989680in}{1.989680in}}%
\pgfusepath{clip}%
\pgfsetbuttcap%
\pgfsetroundjoin%
\definecolor{currentfill}{rgb}{0.699415,0.867117,0.175971}%
\pgfsetfillcolor{currentfill}%
\pgfsetlinewidth{0.000000pt}%
\definecolor{currentstroke}{rgb}{0.000000,0.000000,0.000000}%
\pgfsetstrokecolor{currentstroke}%
\pgfsetdash{}{0pt}%
\pgfpathmoveto{\pgfqpoint{1.237264in}{1.630060in}}%
\pgfpathlineto{\pgfqpoint{1.234110in}{1.625222in}}%
\pgfpathlineto{\pgfqpoint{1.230958in}{1.620285in}}%
\pgfpathlineto{\pgfqpoint{1.227807in}{1.615252in}}%
\pgfpathlineto{\pgfqpoint{1.224657in}{1.610123in}}%
\pgfpathlineto{\pgfqpoint{1.222841in}{1.612036in}}%
\pgfpathlineto{\pgfqpoint{1.221155in}{1.613975in}}%
\pgfpathlineto{\pgfqpoint{1.219600in}{1.615937in}}%
\pgfpathlineto{\pgfqpoint{1.218178in}{1.617921in}}%
\pgfpathlineto{\pgfqpoint{1.221485in}{1.622850in}}%
\pgfpathlineto{\pgfqpoint{1.224793in}{1.627683in}}%
\pgfpathlineto{\pgfqpoint{1.228103in}{1.632420in}}%
\pgfpathlineto{\pgfqpoint{1.231414in}{1.637059in}}%
\pgfpathlineto{\pgfqpoint{1.232699in}{1.635278in}}%
\pgfpathlineto{\pgfqpoint{1.234103in}{1.633517in}}%
\pgfpathlineto{\pgfqpoint{1.235625in}{1.631777in}}%
\pgfpathlineto{\pgfqpoint{1.237264in}{1.630060in}}%
\pgfpathclose%
\pgfusepath{fill}%
\end{pgfscope}%
\begin{pgfscope}%
\pgfpathrectangle{\pgfqpoint{0.329460in}{0.284240in}}{\pgfqpoint{1.989680in}{1.989680in}}%
\pgfusepath{clip}%
\pgfsetbuttcap%
\pgfsetroundjoin%
\definecolor{currentfill}{rgb}{0.974417,0.903590,0.130215}%
\pgfsetfillcolor{currentfill}%
\pgfsetlinewidth{0.000000pt}%
\definecolor{currentstroke}{rgb}{0.000000,0.000000,0.000000}%
\pgfsetstrokecolor{currentstroke}%
\pgfsetdash{}{0pt}%
\pgfpathmoveto{\pgfqpoint{1.379153in}{1.728548in}}%
\pgfpathlineto{\pgfqpoint{1.382647in}{1.727671in}}%
\pgfpathlineto{\pgfqpoint{1.386139in}{1.726675in}}%
\pgfpathlineto{\pgfqpoint{1.389631in}{1.725561in}}%
\pgfpathlineto{\pgfqpoint{1.393122in}{1.724329in}}%
\pgfpathlineto{\pgfqpoint{1.393218in}{1.723711in}}%
\pgfpathlineto{\pgfqpoint{1.393273in}{1.723091in}}%
\pgfpathlineto{\pgfqpoint{1.393285in}{1.722471in}}%
\pgfpathlineto{\pgfqpoint{1.393256in}{1.721852in}}%
\pgfpathlineto{\pgfqpoint{1.389753in}{1.723290in}}%
\pgfpathlineto{\pgfqpoint{1.386249in}{1.724611in}}%
\pgfpathlineto{\pgfqpoint{1.382744in}{1.725813in}}%
\pgfpathlineto{\pgfqpoint{1.379239in}{1.726897in}}%
\pgfpathlineto{\pgfqpoint{1.379260in}{1.727310in}}%
\pgfpathlineto{\pgfqpoint{1.379252in}{1.727723in}}%
\pgfpathlineto{\pgfqpoint{1.379217in}{1.728136in}}%
\pgfpathlineto{\pgfqpoint{1.379153in}{1.728548in}}%
\pgfpathclose%
\pgfusepath{fill}%
\end{pgfscope}%
\begin{pgfscope}%
\pgfpathrectangle{\pgfqpoint{0.329460in}{0.284240in}}{\pgfqpoint{1.989680in}{1.989680in}}%
\pgfusepath{clip}%
\pgfsetbuttcap%
\pgfsetroundjoin%
\definecolor{currentfill}{rgb}{0.993248,0.906157,0.143936}%
\pgfsetfillcolor{currentfill}%
\pgfsetlinewidth{0.000000pt}%
\definecolor{currentstroke}{rgb}{0.000000,0.000000,0.000000}%
\pgfsetstrokecolor{currentstroke}%
\pgfsetdash{}{0pt}%
\pgfpathmoveto{\pgfqpoint{1.364910in}{1.731682in}}%
\pgfpathlineto{\pgfqpoint{1.368340in}{1.731486in}}%
\pgfpathlineto{\pgfqpoint{1.371768in}{1.731170in}}%
\pgfpathlineto{\pgfqpoint{1.375196in}{1.730735in}}%
\pgfpathlineto{\pgfqpoint{1.378622in}{1.730180in}}%
\pgfpathlineto{\pgfqpoint{1.378797in}{1.729775in}}%
\pgfpathlineto{\pgfqpoint{1.378943in}{1.729368in}}%
\pgfpathlineto{\pgfqpoint{1.379062in}{1.728959in}}%
\pgfpathlineto{\pgfqpoint{1.379153in}{1.728548in}}%
\pgfpathlineto{\pgfqpoint{1.375659in}{1.729306in}}%
\pgfpathlineto{\pgfqpoint{1.372165in}{1.729946in}}%
\pgfpathlineto{\pgfqpoint{1.368670in}{1.730466in}}%
\pgfpathlineto{\pgfqpoint{1.365174in}{1.730866in}}%
\pgfpathlineto{\pgfqpoint{1.365129in}{1.731072in}}%
\pgfpathlineto{\pgfqpoint{1.365070in}{1.731276in}}%
\pgfpathlineto{\pgfqpoint{1.364997in}{1.731480in}}%
\pgfpathlineto{\pgfqpoint{1.364910in}{1.731682in}}%
\pgfpathclose%
\pgfusepath{fill}%
\end{pgfscope}%
\begin{pgfscope}%
\pgfpathrectangle{\pgfqpoint{0.329460in}{0.284240in}}{\pgfqpoint{1.989680in}{1.989680in}}%
\pgfusepath{clip}%
\pgfsetbuttcap%
\pgfsetroundjoin%
\definecolor{currentfill}{rgb}{0.993248,0.906157,0.143936}%
\pgfsetfillcolor{currentfill}%
\pgfsetlinewidth{0.000000pt}%
\definecolor{currentstroke}{rgb}{0.000000,0.000000,0.000000}%
\pgfsetstrokecolor{currentstroke}%
\pgfsetdash{}{0pt}%
\pgfpathmoveto{\pgfqpoint{1.351188in}{1.731270in}}%
\pgfpathlineto{\pgfqpoint{1.353834in}{1.732284in}}%
\pgfpathlineto{\pgfqpoint{1.356479in}{1.733177in}}%
\pgfpathlineto{\pgfqpoint{1.359123in}{1.733950in}}%
\pgfpathlineto{\pgfqpoint{1.361766in}{1.734603in}}%
\pgfpathlineto{\pgfqpoint{1.362052in}{1.734445in}}%
\pgfpathlineto{\pgfqpoint{1.362327in}{1.734283in}}%
\pgfpathlineto{\pgfqpoint{1.362591in}{1.734118in}}%
\pgfpathlineto{\pgfqpoint{1.362843in}{1.733948in}}%
\pgfpathlineto{\pgfqpoint{1.359931in}{1.733459in}}%
\pgfpathlineto{\pgfqpoint{1.357018in}{1.732849in}}%
\pgfpathlineto{\pgfqpoint{1.354103in}{1.732120in}}%
\pgfpathlineto{\pgfqpoint{1.351188in}{1.731270in}}%
\pgfpathlineto{\pgfqpoint{1.351188in}{1.731270in}}%
\pgfpathlineto{\pgfqpoint{1.351188in}{1.731270in}}%
\pgfpathlineto{\pgfqpoint{1.351188in}{1.731270in}}%
\pgfpathlineto{\pgfqpoint{1.351188in}{1.731270in}}%
\pgfpathclose%
\pgfusepath{fill}%
\end{pgfscope}%
\begin{pgfscope}%
\pgfpathrectangle{\pgfqpoint{0.329460in}{0.284240in}}{\pgfqpoint{1.989680in}{1.989680in}}%
\pgfusepath{clip}%
\pgfsetbuttcap%
\pgfsetroundjoin%
\definecolor{currentfill}{rgb}{0.993248,0.906157,0.143936}%
\pgfsetfillcolor{currentfill}%
\pgfsetlinewidth{0.000000pt}%
\definecolor{currentstroke}{rgb}{0.000000,0.000000,0.000000}%
\pgfsetstrokecolor{currentstroke}%
\pgfsetdash{}{0pt}%
\pgfpathmoveto{\pgfqpoint{1.351188in}{1.731270in}}%
\pgfpathlineto{\pgfqpoint{1.348218in}{1.732081in}}%
\pgfpathlineto{\pgfqpoint{1.345250in}{1.732772in}}%
\pgfpathlineto{\pgfqpoint{1.342283in}{1.733343in}}%
\pgfpathlineto{\pgfqpoint{1.339317in}{1.733794in}}%
\pgfpathlineto{\pgfqpoint{1.339559in}{1.733967in}}%
\pgfpathlineto{\pgfqpoint{1.339813in}{1.734136in}}%
\pgfpathlineto{\pgfqpoint{1.340078in}{1.734301in}}%
\pgfpathlineto{\pgfqpoint{1.340355in}{1.734463in}}%
\pgfpathlineto{\pgfqpoint{1.343061in}{1.733845in}}%
\pgfpathlineto{\pgfqpoint{1.345769in}{1.733107in}}%
\pgfpathlineto{\pgfqpoint{1.348478in}{1.732249in}}%
\pgfpathlineto{\pgfqpoint{1.351188in}{1.731270in}}%
\pgfpathlineto{\pgfqpoint{1.351188in}{1.731270in}}%
\pgfpathlineto{\pgfqpoint{1.351188in}{1.731270in}}%
\pgfpathlineto{\pgfqpoint{1.351188in}{1.731270in}}%
\pgfpathlineto{\pgfqpoint{1.351188in}{1.731270in}}%
\pgfpathclose%
\pgfusepath{fill}%
\end{pgfscope}%
\begin{pgfscope}%
\pgfpathrectangle{\pgfqpoint{0.329460in}{0.284240in}}{\pgfqpoint{1.989680in}{1.989680in}}%
\pgfusepath{clip}%
\pgfsetbuttcap%
\pgfsetroundjoin%
\definecolor{currentfill}{rgb}{0.935904,0.898570,0.108131}%
\pgfsetfillcolor{currentfill}%
\pgfsetlinewidth{0.000000pt}%
\definecolor{currentstroke}{rgb}{0.000000,0.000000,0.000000}%
\pgfsetstrokecolor{currentstroke}%
\pgfsetdash{}{0pt}%
\pgfpathmoveto{\pgfqpoint{1.296097in}{1.710909in}}%
\pgfpathlineto{\pgfqpoint{1.292657in}{1.708635in}}%
\pgfpathlineto{\pgfqpoint{1.289217in}{1.706245in}}%
\pgfpathlineto{\pgfqpoint{1.285778in}{1.703741in}}%
\pgfpathlineto{\pgfqpoint{1.282340in}{1.701123in}}%
\pgfpathlineto{\pgfqpoint{1.281956in}{1.702143in}}%
\pgfpathlineto{\pgfqpoint{1.281641in}{1.703169in}}%
\pgfpathlineto{\pgfqpoint{1.281396in}{1.704198in}}%
\pgfpathlineto{\pgfqpoint{1.281220in}{1.705231in}}%
\pgfpathlineto{\pgfqpoint{1.284712in}{1.707642in}}%
\pgfpathlineto{\pgfqpoint{1.288204in}{1.709940in}}%
\pgfpathlineto{\pgfqpoint{1.291698in}{1.712123in}}%
\pgfpathlineto{\pgfqpoint{1.295193in}{1.714191in}}%
\pgfpathlineto{\pgfqpoint{1.295336in}{1.713366in}}%
\pgfpathlineto{\pgfqpoint{1.295534in}{1.712544in}}%
\pgfpathlineto{\pgfqpoint{1.295788in}{1.711724in}}%
\pgfpathlineto{\pgfqpoint{1.296097in}{1.710909in}}%
\pgfpathclose%
\pgfusepath{fill}%
\end{pgfscope}%
\begin{pgfscope}%
\pgfpathrectangle{\pgfqpoint{0.329460in}{0.284240in}}{\pgfqpoint{1.989680in}{1.989680in}}%
\pgfusepath{clip}%
\pgfsetbuttcap%
\pgfsetroundjoin%
\definecolor{currentfill}{rgb}{0.565498,0.842430,0.262877}%
\pgfsetfillcolor{currentfill}%
\pgfsetlinewidth{0.000000pt}%
\definecolor{currentstroke}{rgb}{0.000000,0.000000,0.000000}%
\pgfsetstrokecolor{currentstroke}%
\pgfsetdash{}{0pt}%
\pgfpathmoveto{\pgfqpoint{1.492080in}{1.590551in}}%
\pgfpathlineto{\pgfqpoint{1.495262in}{1.585007in}}%
\pgfpathlineto{\pgfqpoint{1.498442in}{1.579376in}}%
\pgfpathlineto{\pgfqpoint{1.501620in}{1.573660in}}%
\pgfpathlineto{\pgfqpoint{1.504797in}{1.567858in}}%
\pgfpathlineto{\pgfqpoint{1.502613in}{1.565551in}}%
\pgfpathlineto{\pgfqpoint{1.500276in}{1.563278in}}%
\pgfpathlineto{\pgfqpoint{1.497787in}{1.561040in}}%
\pgfpathlineto{\pgfqpoint{1.495149in}{1.558840in}}%
\pgfpathlineto{\pgfqpoint{1.492167in}{1.564837in}}%
\pgfpathlineto{\pgfqpoint{1.489184in}{1.570749in}}%
\pgfpathlineto{\pgfqpoint{1.486200in}{1.576574in}}%
\pgfpathlineto{\pgfqpoint{1.483214in}{1.582312in}}%
\pgfpathlineto{\pgfqpoint{1.485637in}{1.584321in}}%
\pgfpathlineto{\pgfqpoint{1.487924in}{1.586366in}}%
\pgfpathlineto{\pgfqpoint{1.490073in}{1.588443in}}%
\pgfpathlineto{\pgfqpoint{1.492080in}{1.590551in}}%
\pgfpathclose%
\pgfusepath{fill}%
\end{pgfscope}%
\begin{pgfscope}%
\pgfpathrectangle{\pgfqpoint{0.329460in}{0.284240in}}{\pgfqpoint{1.989680in}{1.989680in}}%
\pgfusepath{clip}%
\pgfsetbuttcap%
\pgfsetroundjoin%
\definecolor{currentfill}{rgb}{0.993248,0.906157,0.143936}%
\pgfsetfillcolor{currentfill}%
\pgfsetlinewidth{0.000000pt}%
\definecolor{currentstroke}{rgb}{0.000000,0.000000,0.000000}%
\pgfsetstrokecolor{currentstroke}%
\pgfsetdash{}{0pt}%
\pgfpathmoveto{\pgfqpoint{1.337173in}{1.730683in}}%
\pgfpathlineto{\pgfqpoint{1.333670in}{1.730237in}}%
\pgfpathlineto{\pgfqpoint{1.330167in}{1.729671in}}%
\pgfpathlineto{\pgfqpoint{1.326665in}{1.728986in}}%
\pgfpathlineto{\pgfqpoint{1.323164in}{1.728182in}}%
\pgfpathlineto{\pgfqpoint{1.323231in}{1.728594in}}%
\pgfpathlineto{\pgfqpoint{1.323325in}{1.729004in}}%
\pgfpathlineto{\pgfqpoint{1.323447in}{1.729413in}}%
\pgfpathlineto{\pgfqpoint{1.323597in}{1.729820in}}%
\pgfpathlineto{\pgfqpoint{1.327043in}{1.730420in}}%
\pgfpathlineto{\pgfqpoint{1.330490in}{1.730900in}}%
\pgfpathlineto{\pgfqpoint{1.333938in}{1.731261in}}%
\pgfpathlineto{\pgfqpoint{1.337387in}{1.731502in}}%
\pgfpathlineto{\pgfqpoint{1.337313in}{1.731299in}}%
\pgfpathlineto{\pgfqpoint{1.337252in}{1.731094in}}%
\pgfpathlineto{\pgfqpoint{1.337206in}{1.730889in}}%
\pgfpathlineto{\pgfqpoint{1.337173in}{1.730683in}}%
\pgfpathclose%
\pgfusepath{fill}%
\end{pgfscope}%
\begin{pgfscope}%
\pgfpathrectangle{\pgfqpoint{0.329460in}{0.284240in}}{\pgfqpoint{1.989680in}{1.989680in}}%
\pgfusepath{clip}%
\pgfsetbuttcap%
\pgfsetroundjoin%
\definecolor{currentfill}{rgb}{0.231674,0.318106,0.544834}%
\pgfsetfillcolor{currentfill}%
\pgfsetlinewidth{0.000000pt}%
\definecolor{currentstroke}{rgb}{0.000000,0.000000,0.000000}%
\pgfsetstrokecolor{currentstroke}%
\pgfsetdash{}{0pt}%
\pgfpathmoveto{\pgfqpoint{1.215847in}{1.034983in}}%
\pgfpathlineto{\pgfqpoint{1.214606in}{1.026834in}}%
\pgfpathlineto{\pgfqpoint{1.213364in}{1.018741in}}%
\pgfpathlineto{\pgfqpoint{1.212123in}{1.010707in}}%
\pgfpathlineto{\pgfqpoint{1.210881in}{1.002736in}}%
\pgfpathlineto{\pgfqpoint{1.199411in}{1.005154in}}%
\pgfpathlineto{\pgfqpoint{1.188105in}{1.007759in}}%
\pgfpathlineto{\pgfqpoint{1.176976in}{1.010549in}}%
\pgfpathlineto{\pgfqpoint{1.166035in}{1.013521in}}%
\pgfpathlineto{\pgfqpoint{1.167672in}{1.021383in}}%
\pgfpathlineto{\pgfqpoint{1.169308in}{1.029309in}}%
\pgfpathlineto{\pgfqpoint{1.170943in}{1.037293in}}%
\pgfpathlineto{\pgfqpoint{1.172579in}{1.045334in}}%
\pgfpathlineto{\pgfqpoint{1.183136in}{1.042482in}}%
\pgfpathlineto{\pgfqpoint{1.193874in}{1.039804in}}%
\pgfpathlineto{\pgfqpoint{1.204781in}{1.037303in}}%
\pgfpathlineto{\pgfqpoint{1.215847in}{1.034983in}}%
\pgfpathclose%
\pgfusepath{fill}%
\end{pgfscope}%
\begin{pgfscope}%
\pgfpathrectangle{\pgfqpoint{0.329460in}{0.284240in}}{\pgfqpoint{1.989680in}{1.989680in}}%
\pgfusepath{clip}%
\pgfsetbuttcap%
\pgfsetroundjoin%
\definecolor{currentfill}{rgb}{0.974417,0.903590,0.130215}%
\pgfsetfillcolor{currentfill}%
\pgfsetlinewidth{0.000000pt}%
\definecolor{currentstroke}{rgb}{0.000000,0.000000,0.000000}%
\pgfsetstrokecolor{currentstroke}%
\pgfsetdash{}{0pt}%
\pgfpathmoveto{\pgfqpoint{1.323178in}{1.726531in}}%
\pgfpathlineto{\pgfqpoint{1.319678in}{1.725401in}}%
\pgfpathlineto{\pgfqpoint{1.316178in}{1.724152in}}%
\pgfpathlineto{\pgfqpoint{1.312679in}{1.722786in}}%
\pgfpathlineto{\pgfqpoint{1.309180in}{1.721301in}}%
\pgfpathlineto{\pgfqpoint{1.309114in}{1.721920in}}%
\pgfpathlineto{\pgfqpoint{1.309089in}{1.722540in}}%
\pgfpathlineto{\pgfqpoint{1.309106in}{1.723160in}}%
\pgfpathlineto{\pgfqpoint{1.309166in}{1.723780in}}%
\pgfpathlineto{\pgfqpoint{1.312664in}{1.725057in}}%
\pgfpathlineto{\pgfqpoint{1.316163in}{1.726217in}}%
\pgfpathlineto{\pgfqpoint{1.319663in}{1.727259in}}%
\pgfpathlineto{\pgfqpoint{1.323164in}{1.728182in}}%
\pgfpathlineto{\pgfqpoint{1.323126in}{1.727769in}}%
\pgfpathlineto{\pgfqpoint{1.323115in}{1.727356in}}%
\pgfpathlineto{\pgfqpoint{1.323132in}{1.726943in}}%
\pgfpathlineto{\pgfqpoint{1.323178in}{1.726531in}}%
\pgfpathclose%
\pgfusepath{fill}%
\end{pgfscope}%
\begin{pgfscope}%
\pgfpathrectangle{\pgfqpoint{0.329460in}{0.284240in}}{\pgfqpoint{1.989680in}{1.989680in}}%
\pgfusepath{clip}%
\pgfsetbuttcap%
\pgfsetroundjoin%
\definecolor{currentfill}{rgb}{0.993248,0.906157,0.143936}%
\pgfsetfillcolor{currentfill}%
\pgfsetlinewidth{0.000000pt}%
\definecolor{currentstroke}{rgb}{0.000000,0.000000,0.000000}%
\pgfsetstrokecolor{currentstroke}%
\pgfsetdash{}{0pt}%
\pgfpathmoveto{\pgfqpoint{1.351188in}{1.731270in}}%
\pgfpathlineto{\pgfqpoint{1.354103in}{1.732120in}}%
\pgfpathlineto{\pgfqpoint{1.357018in}{1.732849in}}%
\pgfpathlineto{\pgfqpoint{1.359931in}{1.733459in}}%
\pgfpathlineto{\pgfqpoint{1.362843in}{1.733948in}}%
\pgfpathlineto{\pgfqpoint{1.363084in}{1.733775in}}%
\pgfpathlineto{\pgfqpoint{1.363314in}{1.733598in}}%
\pgfpathlineto{\pgfqpoint{1.363531in}{1.733419in}}%
\pgfpathlineto{\pgfqpoint{1.363736in}{1.733236in}}%
\pgfpathlineto{\pgfqpoint{1.360600in}{1.732924in}}%
\pgfpathlineto{\pgfqpoint{1.357463in}{1.732493in}}%
\pgfpathlineto{\pgfqpoint{1.354326in}{1.731942in}}%
\pgfpathlineto{\pgfqpoint{1.351188in}{1.731270in}}%
\pgfpathlineto{\pgfqpoint{1.351188in}{1.731270in}}%
\pgfpathlineto{\pgfqpoint{1.351188in}{1.731270in}}%
\pgfpathlineto{\pgfqpoint{1.351188in}{1.731270in}}%
\pgfpathlineto{\pgfqpoint{1.351188in}{1.731270in}}%
\pgfpathclose%
\pgfusepath{fill}%
\end{pgfscope}%
\begin{pgfscope}%
\pgfpathrectangle{\pgfqpoint{0.329460in}{0.284240in}}{\pgfqpoint{1.989680in}{1.989680in}}%
\pgfusepath{clip}%
\pgfsetbuttcap%
\pgfsetroundjoin%
\definecolor{currentfill}{rgb}{0.212395,0.359683,0.551710}%
\pgfsetfillcolor{currentfill}%
\pgfsetlinewidth{0.000000pt}%
\definecolor{currentstroke}{rgb}{0.000000,0.000000,0.000000}%
\pgfsetstrokecolor{currentstroke}%
\pgfsetdash{}{0pt}%
\pgfpathmoveto{\pgfqpoint{1.532143in}{1.080571in}}%
\pgfpathlineto{\pgfqpoint{1.533862in}{1.072362in}}%
\pgfpathlineto{\pgfqpoint{1.535581in}{1.064197in}}%
\pgfpathlineto{\pgfqpoint{1.537300in}{1.056080in}}%
\pgfpathlineto{\pgfqpoint{1.539019in}{1.048013in}}%
\pgfpathlineto{\pgfqpoint{1.528633in}{1.045009in}}%
\pgfpathlineto{\pgfqpoint{1.518055in}{1.042176in}}%
\pgfpathlineto{\pgfqpoint{1.507298in}{1.039518in}}%
\pgfpathlineto{\pgfqpoint{1.496372in}{1.037037in}}%
\pgfpathlineto{\pgfqpoint{1.495041in}{1.045218in}}%
\pgfpathlineto{\pgfqpoint{1.493710in}{1.053449in}}%
\pgfpathlineto{\pgfqpoint{1.492379in}{1.061727in}}%
\pgfpathlineto{\pgfqpoint{1.491048in}{1.070050in}}%
\pgfpathlineto{\pgfqpoint{1.501576in}{1.072428in}}%
\pgfpathlineto{\pgfqpoint{1.511941in}{1.074976in}}%
\pgfpathlineto{\pgfqpoint{1.522134in}{1.077691in}}%
\pgfpathlineto{\pgfqpoint{1.532143in}{1.080571in}}%
\pgfpathclose%
\pgfusepath{fill}%
\end{pgfscope}%
\begin{pgfscope}%
\pgfpathrectangle{\pgfqpoint{0.329460in}{0.284240in}}{\pgfqpoint{1.989680in}{1.989680in}}%
\pgfusepath{clip}%
\pgfsetbuttcap%
\pgfsetroundjoin%
\definecolor{currentfill}{rgb}{0.993248,0.906157,0.143936}%
\pgfsetfillcolor{currentfill}%
\pgfsetlinewidth{0.000000pt}%
\definecolor{currentstroke}{rgb}{0.000000,0.000000,0.000000}%
\pgfsetstrokecolor{currentstroke}%
\pgfsetdash{}{0pt}%
\pgfpathmoveto{\pgfqpoint{1.351188in}{1.731270in}}%
\pgfpathlineto{\pgfqpoint{1.348006in}{1.731900in}}%
\pgfpathlineto{\pgfqpoint{1.344826in}{1.732410in}}%
\pgfpathlineto{\pgfqpoint{1.341646in}{1.732800in}}%
\pgfpathlineto{\pgfqpoint{1.338468in}{1.733070in}}%
\pgfpathlineto{\pgfqpoint{1.338662in}{1.733256in}}%
\pgfpathlineto{\pgfqpoint{1.338868in}{1.733439in}}%
\pgfpathlineto{\pgfqpoint{1.339086in}{1.733618in}}%
\pgfpathlineto{\pgfqpoint{1.339317in}{1.733794in}}%
\pgfpathlineto{\pgfqpoint{1.342283in}{1.733343in}}%
\pgfpathlineto{\pgfqpoint{1.345250in}{1.732772in}}%
\pgfpathlineto{\pgfqpoint{1.348218in}{1.732081in}}%
\pgfpathlineto{\pgfqpoint{1.351188in}{1.731270in}}%
\pgfpathlineto{\pgfqpoint{1.351188in}{1.731270in}}%
\pgfpathlineto{\pgfqpoint{1.351188in}{1.731270in}}%
\pgfpathlineto{\pgfqpoint{1.351188in}{1.731270in}}%
\pgfpathlineto{\pgfqpoint{1.351188in}{1.731270in}}%
\pgfpathclose%
\pgfusepath{fill}%
\end{pgfscope}%
\begin{pgfscope}%
\pgfpathrectangle{\pgfqpoint{0.329460in}{0.284240in}}{\pgfqpoint{1.989680in}{1.989680in}}%
\pgfusepath{clip}%
\pgfsetbuttcap%
\pgfsetroundjoin%
\definecolor{currentfill}{rgb}{0.993248,0.906157,0.143936}%
\pgfsetfillcolor{currentfill}%
\pgfsetlinewidth{0.000000pt}%
\definecolor{currentstroke}{rgb}{0.000000,0.000000,0.000000}%
\pgfsetstrokecolor{currentstroke}%
\pgfsetdash{}{0pt}%
\pgfpathmoveto{\pgfqpoint{1.351188in}{1.731270in}}%
\pgfpathlineto{\pgfqpoint{1.354326in}{1.731942in}}%
\pgfpathlineto{\pgfqpoint{1.357463in}{1.732493in}}%
\pgfpathlineto{\pgfqpoint{1.360600in}{1.732924in}}%
\pgfpathlineto{\pgfqpoint{1.363736in}{1.733236in}}%
\pgfpathlineto{\pgfqpoint{1.363928in}{1.733050in}}%
\pgfpathlineto{\pgfqpoint{1.364108in}{1.732861in}}%
\pgfpathlineto{\pgfqpoint{1.364275in}{1.732670in}}%
\pgfpathlineto{\pgfqpoint{1.364428in}{1.732476in}}%
\pgfpathlineto{\pgfqpoint{1.361119in}{1.732355in}}%
\pgfpathlineto{\pgfqpoint{1.357809in}{1.732113in}}%
\pgfpathlineto{\pgfqpoint{1.354499in}{1.731752in}}%
\pgfpathlineto{\pgfqpoint{1.351188in}{1.731270in}}%
\pgfpathlineto{\pgfqpoint{1.351188in}{1.731270in}}%
\pgfpathlineto{\pgfqpoint{1.351188in}{1.731270in}}%
\pgfpathlineto{\pgfqpoint{1.351188in}{1.731270in}}%
\pgfpathlineto{\pgfqpoint{1.351188in}{1.731270in}}%
\pgfpathclose%
\pgfusepath{fill}%
\end{pgfscope}%
\begin{pgfscope}%
\pgfpathrectangle{\pgfqpoint{0.329460in}{0.284240in}}{\pgfqpoint{1.989680in}{1.989680in}}%
\pgfusepath{clip}%
\pgfsetbuttcap%
\pgfsetroundjoin%
\definecolor{currentfill}{rgb}{0.122606,0.585371,0.546557}%
\pgfsetfillcolor{currentfill}%
\pgfsetlinewidth{0.000000pt}%
\definecolor{currentstroke}{rgb}{0.000000,0.000000,0.000000}%
\pgfsetstrokecolor{currentstroke}%
\pgfsetdash{}{0pt}%
\pgfpathmoveto{\pgfqpoint{1.188472in}{1.291104in}}%
\pgfpathlineto{\pgfqpoint{1.186462in}{1.282757in}}%
\pgfpathlineto{\pgfqpoint{1.184452in}{1.274391in}}%
\pgfpathlineto{\pgfqpoint{1.182442in}{1.266008in}}%
\pgfpathlineto{\pgfqpoint{1.180434in}{1.257610in}}%
\pgfpathlineto{\pgfqpoint{1.173006in}{1.260386in}}%
\pgfpathlineto{\pgfqpoint{1.165767in}{1.263276in}}%
\pgfpathlineto{\pgfqpoint{1.158722in}{1.266279in}}%
\pgfpathlineto{\pgfqpoint{1.151879in}{1.269391in}}%
\pgfpathlineto{\pgfqpoint{1.154219in}{1.277635in}}%
\pgfpathlineto{\pgfqpoint{1.156560in}{1.285864in}}%
\pgfpathlineto{\pgfqpoint{1.158902in}{1.294077in}}%
\pgfpathlineto{\pgfqpoint{1.161245in}{1.302271in}}%
\pgfpathlineto{\pgfqpoint{1.167771in}{1.299322in}}%
\pgfpathlineto{\pgfqpoint{1.174488in}{1.296475in}}%
\pgfpathlineto{\pgfqpoint{1.181391in}{1.293735in}}%
\pgfpathlineto{\pgfqpoint{1.188472in}{1.291104in}}%
\pgfpathclose%
\pgfusepath{fill}%
\end{pgfscope}%
\begin{pgfscope}%
\pgfpathrectangle{\pgfqpoint{0.329460in}{0.284240in}}{\pgfqpoint{1.989680in}{1.989680in}}%
\pgfusepath{clip}%
\pgfsetbuttcap%
\pgfsetroundjoin%
\definecolor{currentfill}{rgb}{0.855810,0.888601,0.097452}%
\pgfsetfillcolor{currentfill}%
\pgfsetlinewidth{0.000000pt}%
\definecolor{currentstroke}{rgb}{0.000000,0.000000,0.000000}%
\pgfsetstrokecolor{currentstroke}%
\pgfsetdash{}{0pt}%
\pgfpathmoveto{\pgfqpoint{1.271250in}{1.684699in}}%
\pgfpathlineto{\pgfqpoint{1.267925in}{1.681317in}}%
\pgfpathlineto{\pgfqpoint{1.264600in}{1.677825in}}%
\pgfpathlineto{\pgfqpoint{1.261276in}{1.674224in}}%
\pgfpathlineto{\pgfqpoint{1.257953in}{1.670515in}}%
\pgfpathlineto{\pgfqpoint{1.257041in}{1.671907in}}%
\pgfpathlineto{\pgfqpoint{1.256223in}{1.673312in}}%
\pgfpathlineto{\pgfqpoint{1.255500in}{1.674728in}}%
\pgfpathlineto{\pgfqpoint{1.254873in}{1.676153in}}%
\pgfpathlineto{\pgfqpoint{1.258302in}{1.679658in}}%
\pgfpathlineto{\pgfqpoint{1.261733in}{1.683055in}}%
\pgfpathlineto{\pgfqpoint{1.265165in}{1.686343in}}%
\pgfpathlineto{\pgfqpoint{1.268598in}{1.689522in}}%
\pgfpathlineto{\pgfqpoint{1.269138in}{1.688302in}}%
\pgfpathlineto{\pgfqpoint{1.269761in}{1.687091in}}%
\pgfpathlineto{\pgfqpoint{1.270465in}{1.685890in}}%
\pgfpathlineto{\pgfqpoint{1.271250in}{1.684699in}}%
\pgfpathclose%
\pgfusepath{fill}%
\end{pgfscope}%
\begin{pgfscope}%
\pgfpathrectangle{\pgfqpoint{0.329460in}{0.284240in}}{\pgfqpoint{1.989680in}{1.989680in}}%
\pgfusepath{clip}%
\pgfsetbuttcap%
\pgfsetroundjoin%
\definecolor{currentfill}{rgb}{0.993248,0.906157,0.143936}%
\pgfsetfillcolor{currentfill}%
\pgfsetlinewidth{0.000000pt}%
\definecolor{currentstroke}{rgb}{0.000000,0.000000,0.000000}%
\pgfsetstrokecolor{currentstroke}%
\pgfsetdash{}{0pt}%
\pgfpathmoveto{\pgfqpoint{1.351188in}{1.731270in}}%
\pgfpathlineto{\pgfqpoint{1.347845in}{1.731708in}}%
\pgfpathlineto{\pgfqpoint{1.344503in}{1.732026in}}%
\pgfpathlineto{\pgfqpoint{1.341162in}{1.732224in}}%
\pgfpathlineto{\pgfqpoint{1.337821in}{1.732302in}}%
\pgfpathlineto{\pgfqpoint{1.337963in}{1.732498in}}%
\pgfpathlineto{\pgfqpoint{1.338118in}{1.732691in}}%
\pgfpathlineto{\pgfqpoint{1.338287in}{1.732882in}}%
\pgfpathlineto{\pgfqpoint{1.338468in}{1.733070in}}%
\pgfpathlineto{\pgfqpoint{1.341646in}{1.732800in}}%
\pgfpathlineto{\pgfqpoint{1.344826in}{1.732410in}}%
\pgfpathlineto{\pgfqpoint{1.348006in}{1.731900in}}%
\pgfpathlineto{\pgfqpoint{1.351188in}{1.731270in}}%
\pgfpathlineto{\pgfqpoint{1.351188in}{1.731270in}}%
\pgfpathlineto{\pgfqpoint{1.351188in}{1.731270in}}%
\pgfpathlineto{\pgfqpoint{1.351188in}{1.731270in}}%
\pgfpathlineto{\pgfqpoint{1.351188in}{1.731270in}}%
\pgfpathclose%
\pgfusepath{fill}%
\end{pgfscope}%
\begin{pgfscope}%
\pgfpathrectangle{\pgfqpoint{0.329460in}{0.284240in}}{\pgfqpoint{1.989680in}{1.989680in}}%
\pgfusepath{clip}%
\pgfsetbuttcap%
\pgfsetroundjoin%
\definecolor{currentfill}{rgb}{0.955300,0.901065,0.118128}%
\pgfsetfillcolor{currentfill}%
\pgfsetlinewidth{0.000000pt}%
\definecolor{currentstroke}{rgb}{0.000000,0.000000,0.000000}%
\pgfsetstrokecolor{currentstroke}%
\pgfsetdash{}{0pt}%
\pgfpathmoveto{\pgfqpoint{1.393256in}{1.721852in}}%
\pgfpathlineto{\pgfqpoint{1.396759in}{1.720295in}}%
\pgfpathlineto{\pgfqpoint{1.400261in}{1.718622in}}%
\pgfpathlineto{\pgfqpoint{1.403762in}{1.716832in}}%
\pgfpathlineto{\pgfqpoint{1.407262in}{1.714926in}}%
\pgfpathlineto{\pgfqpoint{1.407169in}{1.714099in}}%
\pgfpathlineto{\pgfqpoint{1.407020in}{1.713275in}}%
\pgfpathlineto{\pgfqpoint{1.406816in}{1.712452in}}%
\pgfpathlineto{\pgfqpoint{1.406556in}{1.711633in}}%
\pgfpathlineto{\pgfqpoint{1.403098in}{1.713746in}}%
\pgfpathlineto{\pgfqpoint{1.399639in}{1.715743in}}%
\pgfpathlineto{\pgfqpoint{1.396180in}{1.717623in}}%
\pgfpathlineto{\pgfqpoint{1.392720in}{1.719385in}}%
\pgfpathlineto{\pgfqpoint{1.392917in}{1.719998in}}%
\pgfpathlineto{\pgfqpoint{1.393072in}{1.720614in}}%
\pgfpathlineto{\pgfqpoint{1.393185in}{1.721232in}}%
\pgfpathlineto{\pgfqpoint{1.393256in}{1.721852in}}%
\pgfpathclose%
\pgfusepath{fill}%
\end{pgfscope}%
\begin{pgfscope}%
\pgfpathrectangle{\pgfqpoint{0.329460in}{0.284240in}}{\pgfqpoint{1.989680in}{1.989680in}}%
\pgfusepath{clip}%
\pgfsetbuttcap%
\pgfsetroundjoin%
\definecolor{currentfill}{rgb}{0.762373,0.876424,0.137064}%
\pgfsetfillcolor{currentfill}%
\pgfsetlinewidth{0.000000pt}%
\definecolor{currentstroke}{rgb}{0.000000,0.000000,0.000000}%
\pgfsetstrokecolor{currentstroke}%
\pgfsetdash{}{0pt}%
\pgfpathmoveto{\pgfqpoint{1.458630in}{1.656031in}}%
\pgfpathlineto{\pgfqpoint{1.461975in}{1.651840in}}%
\pgfpathlineto{\pgfqpoint{1.465319in}{1.647546in}}%
\pgfpathlineto{\pgfqpoint{1.468661in}{1.643152in}}%
\pgfpathlineto{\pgfqpoint{1.472002in}{1.638657in}}%
\pgfpathlineto{\pgfqpoint{1.470824in}{1.636860in}}%
\pgfpathlineto{\pgfqpoint{1.469526in}{1.635081in}}%
\pgfpathlineto{\pgfqpoint{1.468109in}{1.633322in}}%
\pgfpathlineto{\pgfqpoint{1.466574in}{1.631585in}}%
\pgfpathlineto{\pgfqpoint{1.463380in}{1.636281in}}%
\pgfpathlineto{\pgfqpoint{1.460184in}{1.640876in}}%
\pgfpathlineto{\pgfqpoint{1.456987in}{1.645370in}}%
\pgfpathlineto{\pgfqpoint{1.453789in}{1.649761in}}%
\pgfpathlineto{\pgfqpoint{1.455158in}{1.651301in}}%
\pgfpathlineto{\pgfqpoint{1.456422in}{1.652861in}}%
\pgfpathlineto{\pgfqpoint{1.457579in}{1.654438in}}%
\pgfpathlineto{\pgfqpoint{1.458630in}{1.656031in}}%
\pgfpathclose%
\pgfusepath{fill}%
\end{pgfscope}%
\begin{pgfscope}%
\pgfpathrectangle{\pgfqpoint{0.329460in}{0.284240in}}{\pgfqpoint{1.989680in}{1.989680in}}%
\pgfusepath{clip}%
\pgfsetbuttcap%
\pgfsetroundjoin%
\definecolor{currentfill}{rgb}{0.166383,0.690856,0.496502}%
\pgfsetfillcolor{currentfill}%
\pgfsetlinewidth{0.000000pt}%
\definecolor{currentstroke}{rgb}{0.000000,0.000000,0.000000}%
\pgfsetstrokecolor{currentstroke}%
\pgfsetdash{}{0pt}%
\pgfpathmoveto{\pgfqpoint{1.189442in}{1.398269in}}%
\pgfpathlineto{\pgfqpoint{1.187086in}{1.390483in}}%
\pgfpathlineto{\pgfqpoint{1.184732in}{1.382650in}}%
\pgfpathlineto{\pgfqpoint{1.182378in}{1.374774in}}%
\pgfpathlineto{\pgfqpoint{1.180026in}{1.366856in}}%
\pgfpathlineto{\pgfqpoint{1.174316in}{1.369576in}}%
\pgfpathlineto{\pgfqpoint{1.168790in}{1.372382in}}%
\pgfpathlineto{\pgfqpoint{1.163454in}{1.375272in}}%
\pgfpathlineto{\pgfqpoint{1.158312in}{1.378242in}}%
\pgfpathlineto{\pgfqpoint{1.160958in}{1.385990in}}%
\pgfpathlineto{\pgfqpoint{1.163605in}{1.393696in}}%
\pgfpathlineto{\pgfqpoint{1.166254in}{1.401358in}}%
\pgfpathlineto{\pgfqpoint{1.168904in}{1.408975in}}%
\pgfpathlineto{\pgfqpoint{1.173768in}{1.406182in}}%
\pgfpathlineto{\pgfqpoint{1.178816in}{1.403465in}}%
\pgfpathlineto{\pgfqpoint{1.184042in}{1.400826in}}%
\pgfpathlineto{\pgfqpoint{1.189442in}{1.398269in}}%
\pgfpathclose%
\pgfusepath{fill}%
\end{pgfscope}%
\begin{pgfscope}%
\pgfpathrectangle{\pgfqpoint{0.329460in}{0.284240in}}{\pgfqpoint{1.989680in}{1.989680in}}%
\pgfusepath{clip}%
\pgfsetbuttcap%
\pgfsetroundjoin%
\definecolor{currentfill}{rgb}{0.163625,0.471133,0.558148}%
\pgfsetfillcolor{currentfill}%
\pgfsetlinewidth{0.000000pt}%
\definecolor{currentstroke}{rgb}{0.000000,0.000000,0.000000}%
\pgfsetstrokecolor{currentstroke}%
\pgfsetdash{}{0pt}%
\pgfpathmoveto{\pgfqpoint{1.198751in}{1.179267in}}%
\pgfpathlineto{\pgfqpoint{1.197114in}{1.170719in}}%
\pgfpathlineto{\pgfqpoint{1.195477in}{1.162182in}}%
\pgfpathlineto{\pgfqpoint{1.193840in}{1.153658in}}%
\pgfpathlineto{\pgfqpoint{1.192204in}{1.145151in}}%
\pgfpathlineto{\pgfqpoint{1.182970in}{1.147800in}}%
\pgfpathlineto{\pgfqpoint{1.173915in}{1.150596in}}%
\pgfpathlineto{\pgfqpoint{1.165049in}{1.153536in}}%
\pgfpathlineto{\pgfqpoint{1.156381in}{1.156616in}}%
\pgfpathlineto{\pgfqpoint{1.158382in}{1.164991in}}%
\pgfpathlineto{\pgfqpoint{1.160384in}{1.173382in}}%
\pgfpathlineto{\pgfqpoint{1.162387in}{1.181787in}}%
\pgfpathlineto{\pgfqpoint{1.164390in}{1.190203in}}%
\pgfpathlineto{\pgfqpoint{1.172705in}{1.187265in}}%
\pgfpathlineto{\pgfqpoint{1.181209in}{1.184461in}}%
\pgfpathlineto{\pgfqpoint{1.189894in}{1.181794in}}%
\pgfpathlineto{\pgfqpoint{1.198751in}{1.179267in}}%
\pgfpathclose%
\pgfusepath{fill}%
\end{pgfscope}%
\begin{pgfscope}%
\pgfpathrectangle{\pgfqpoint{0.329460in}{0.284240in}}{\pgfqpoint{1.989680in}{1.989680in}}%
\pgfusepath{clip}%
\pgfsetbuttcap%
\pgfsetroundjoin%
\definecolor{currentfill}{rgb}{0.993248,0.906157,0.143936}%
\pgfsetfillcolor{currentfill}%
\pgfsetlinewidth{0.000000pt}%
\definecolor{currentstroke}{rgb}{0.000000,0.000000,0.000000}%
\pgfsetstrokecolor{currentstroke}%
\pgfsetdash{}{0pt}%
\pgfpathmoveto{\pgfqpoint{1.351188in}{1.731270in}}%
\pgfpathlineto{\pgfqpoint{1.354499in}{1.731752in}}%
\pgfpathlineto{\pgfqpoint{1.357809in}{1.732113in}}%
\pgfpathlineto{\pgfqpoint{1.361119in}{1.732355in}}%
\pgfpathlineto{\pgfqpoint{1.364428in}{1.732476in}}%
\pgfpathlineto{\pgfqpoint{1.364569in}{1.732281in}}%
\pgfpathlineto{\pgfqpoint{1.364696in}{1.732083in}}%
\pgfpathlineto{\pgfqpoint{1.364810in}{1.731883in}}%
\pgfpathlineto{\pgfqpoint{1.364910in}{1.731682in}}%
\pgfpathlineto{\pgfqpoint{1.361480in}{1.731759in}}%
\pgfpathlineto{\pgfqpoint{1.358050in}{1.731716in}}%
\pgfpathlineto{\pgfqpoint{1.354619in}{1.731553in}}%
\pgfpathlineto{\pgfqpoint{1.351188in}{1.731270in}}%
\pgfpathlineto{\pgfqpoint{1.351188in}{1.731270in}}%
\pgfpathlineto{\pgfqpoint{1.351188in}{1.731270in}}%
\pgfpathlineto{\pgfqpoint{1.351188in}{1.731270in}}%
\pgfpathlineto{\pgfqpoint{1.351188in}{1.731270in}}%
\pgfpathclose%
\pgfusepath{fill}%
\end{pgfscope}%
\begin{pgfscope}%
\pgfpathrectangle{\pgfqpoint{0.329460in}{0.284240in}}{\pgfqpoint{1.989680in}{1.989680in}}%
\pgfusepath{clip}%
\pgfsetbuttcap%
\pgfsetroundjoin%
\definecolor{currentfill}{rgb}{0.993248,0.906157,0.143936}%
\pgfsetfillcolor{currentfill}%
\pgfsetlinewidth{0.000000pt}%
\definecolor{currentstroke}{rgb}{0.000000,0.000000,0.000000}%
\pgfsetstrokecolor{currentstroke}%
\pgfsetdash{}{0pt}%
\pgfpathmoveto{\pgfqpoint{1.365174in}{1.730866in}}%
\pgfpathlineto{\pgfqpoint{1.368670in}{1.730466in}}%
\pgfpathlineto{\pgfqpoint{1.372165in}{1.729946in}}%
\pgfpathlineto{\pgfqpoint{1.375659in}{1.729306in}}%
\pgfpathlineto{\pgfqpoint{1.379153in}{1.728548in}}%
\pgfpathlineto{\pgfqpoint{1.379217in}{1.728136in}}%
\pgfpathlineto{\pgfqpoint{1.379252in}{1.727723in}}%
\pgfpathlineto{\pgfqpoint{1.379260in}{1.727310in}}%
\pgfpathlineto{\pgfqpoint{1.379239in}{1.726897in}}%
\pgfpathlineto{\pgfqpoint{1.375733in}{1.727862in}}%
\pgfpathlineto{\pgfqpoint{1.372228in}{1.728708in}}%
\pgfpathlineto{\pgfqpoint{1.368721in}{1.729434in}}%
\pgfpathlineto{\pgfqpoint{1.365215in}{1.730041in}}%
\pgfpathlineto{\pgfqpoint{1.365226in}{1.730248in}}%
\pgfpathlineto{\pgfqpoint{1.365222in}{1.730454in}}%
\pgfpathlineto{\pgfqpoint{1.365205in}{1.730660in}}%
\pgfpathlineto{\pgfqpoint{1.365174in}{1.730866in}}%
\pgfpathclose%
\pgfusepath{fill}%
\end{pgfscope}%
\begin{pgfscope}%
\pgfpathrectangle{\pgfqpoint{0.329460in}{0.284240in}}{\pgfqpoint{1.989680in}{1.989680in}}%
\pgfusepath{clip}%
\pgfsetbuttcap%
\pgfsetroundjoin%
\definecolor{currentfill}{rgb}{0.993248,0.906157,0.143936}%
\pgfsetfillcolor{currentfill}%
\pgfsetlinewidth{0.000000pt}%
\definecolor{currentstroke}{rgb}{0.000000,0.000000,0.000000}%
\pgfsetstrokecolor{currentstroke}%
\pgfsetdash{}{0pt}%
\pgfpathmoveto{\pgfqpoint{1.351188in}{1.731270in}}%
\pgfpathlineto{\pgfqpoint{1.347737in}{1.731508in}}%
\pgfpathlineto{\pgfqpoint{1.344286in}{1.731626in}}%
\pgfpathlineto{\pgfqpoint{1.340836in}{1.731624in}}%
\pgfpathlineto{\pgfqpoint{1.337387in}{1.731502in}}%
\pgfpathlineto{\pgfqpoint{1.337475in}{1.731705in}}%
\pgfpathlineto{\pgfqpoint{1.337577in}{1.731906in}}%
\pgfpathlineto{\pgfqpoint{1.337692in}{1.732105in}}%
\pgfpathlineto{\pgfqpoint{1.337821in}{1.732302in}}%
\pgfpathlineto{\pgfqpoint{1.341162in}{1.732224in}}%
\pgfpathlineto{\pgfqpoint{1.344503in}{1.732026in}}%
\pgfpathlineto{\pgfqpoint{1.347845in}{1.731708in}}%
\pgfpathlineto{\pgfqpoint{1.351188in}{1.731270in}}%
\pgfpathlineto{\pgfqpoint{1.351188in}{1.731270in}}%
\pgfpathlineto{\pgfqpoint{1.351188in}{1.731270in}}%
\pgfpathlineto{\pgfqpoint{1.351188in}{1.731270in}}%
\pgfpathlineto{\pgfqpoint{1.351188in}{1.731270in}}%
\pgfpathclose%
\pgfusepath{fill}%
\end{pgfscope}%
\begin{pgfscope}%
\pgfpathrectangle{\pgfqpoint{0.329460in}{0.284240in}}{\pgfqpoint{1.989680in}{1.989680in}}%
\pgfusepath{clip}%
\pgfsetbuttcap%
\pgfsetroundjoin%
\definecolor{currentfill}{rgb}{0.277941,0.056324,0.381191}%
\pgfsetfillcolor{currentfill}%
\pgfsetlinewidth{0.000000pt}%
\definecolor{currentstroke}{rgb}{0.000000,0.000000,0.000000}%
\pgfsetstrokecolor{currentstroke}%
\pgfsetdash{}{0pt}%
\pgfpathmoveto{\pgfqpoint{1.761212in}{0.832483in}}%
\pgfpathlineto{\pgfqpoint{1.763812in}{0.834489in}}%
\pgfpathlineto{\pgfqpoint{1.766421in}{0.836787in}}%
\pgfpathlineto{\pgfqpoint{1.769039in}{0.839383in}}%
\pgfpathlineto{\pgfqpoint{1.771666in}{0.842280in}}%
\pgfpathlineto{\pgfqpoint{1.758244in}{0.835256in}}%
\pgfpathlineto{\pgfqpoint{1.744375in}{0.828457in}}%
\pgfpathlineto{\pgfqpoint{1.730070in}{0.821890in}}%
\pgfpathlineto{\pgfqpoint{1.715346in}{0.815564in}}%
\pgfpathlineto{\pgfqpoint{1.713062in}{0.812822in}}%
\pgfpathlineto{\pgfqpoint{1.710786in}{0.810383in}}%
\pgfpathlineto{\pgfqpoint{1.708517in}{0.808243in}}%
\pgfpathlineto{\pgfqpoint{1.706257in}{0.806395in}}%
\pgfpathlineto{\pgfqpoint{1.720622in}{0.812572in}}%
\pgfpathlineto{\pgfqpoint{1.734580in}{0.818984in}}%
\pgfpathlineto{\pgfqpoint{1.748114in}{0.825623in}}%
\pgfpathlineto{\pgfqpoint{1.761212in}{0.832483in}}%
\pgfpathclose%
\pgfusepath{fill}%
\end{pgfscope}%
\begin{pgfscope}%
\pgfpathrectangle{\pgfqpoint{0.329460in}{0.284240in}}{\pgfqpoint{1.989680in}{1.989680in}}%
\pgfusepath{clip}%
\pgfsetbuttcap%
\pgfsetroundjoin%
\definecolor{currentfill}{rgb}{0.344074,0.780029,0.397381}%
\pgfsetfillcolor{currentfill}%
\pgfsetlinewidth{0.000000pt}%
\definecolor{currentstroke}{rgb}{0.000000,0.000000,0.000000}%
\pgfsetstrokecolor{currentstroke}%
\pgfsetdash{}{0pt}%
\pgfpathmoveto{\pgfqpoint{1.200809in}{1.496084in}}%
\pgfpathlineto{\pgfqpoint{1.198144in}{1.489170in}}%
\pgfpathlineto{\pgfqpoint{1.195479in}{1.482187in}}%
\pgfpathlineto{\pgfqpoint{1.192815in}{1.475138in}}%
\pgfpathlineto{\pgfqpoint{1.190153in}{1.468022in}}%
\pgfpathlineto{\pgfqpoint{1.186014in}{1.470530in}}%
\pgfpathlineto{\pgfqpoint{1.182044in}{1.473098in}}%
\pgfpathlineto{\pgfqpoint{1.178247in}{1.475725in}}%
\pgfpathlineto{\pgfqpoint{1.174628in}{1.478408in}}%
\pgfpathlineto{\pgfqpoint{1.177542in}{1.485340in}}%
\pgfpathlineto{\pgfqpoint{1.180458in}{1.492207in}}%
\pgfpathlineto{\pgfqpoint{1.183375in}{1.499007in}}%
\pgfpathlineto{\pgfqpoint{1.186293in}{1.505739in}}%
\pgfpathlineto{\pgfqpoint{1.189678in}{1.503245in}}%
\pgfpathlineto{\pgfqpoint{1.193228in}{1.500803in}}%
\pgfpathlineto{\pgfqpoint{1.196939in}{1.498415in}}%
\pgfpathlineto{\pgfqpoint{1.200809in}{1.496084in}}%
\pgfpathclose%
\pgfusepath{fill}%
\end{pgfscope}%
\begin{pgfscope}%
\pgfpathrectangle{\pgfqpoint{0.329460in}{0.284240in}}{\pgfqpoint{1.989680in}{1.989680in}}%
\pgfusepath{clip}%
\pgfsetbuttcap%
\pgfsetroundjoin%
\definecolor{currentfill}{rgb}{0.993248,0.906157,0.143936}%
\pgfsetfillcolor{currentfill}%
\pgfsetlinewidth{0.000000pt}%
\definecolor{currentstroke}{rgb}{0.000000,0.000000,0.000000}%
\pgfsetstrokecolor{currentstroke}%
\pgfsetdash{}{0pt}%
\pgfpathmoveto{\pgfqpoint{1.337182in}{1.729858in}}%
\pgfpathlineto{\pgfqpoint{1.333680in}{1.729205in}}%
\pgfpathlineto{\pgfqpoint{1.330179in}{1.728433in}}%
\pgfpathlineto{\pgfqpoint{1.326678in}{1.727541in}}%
\pgfpathlineto{\pgfqpoint{1.323178in}{1.726531in}}%
\pgfpathlineto{\pgfqpoint{1.323132in}{1.726943in}}%
\pgfpathlineto{\pgfqpoint{1.323115in}{1.727356in}}%
\pgfpathlineto{\pgfqpoint{1.323126in}{1.727769in}}%
\pgfpathlineto{\pgfqpoint{1.323164in}{1.728182in}}%
\pgfpathlineto{\pgfqpoint{1.326665in}{1.728986in}}%
\pgfpathlineto{\pgfqpoint{1.330167in}{1.729671in}}%
\pgfpathlineto{\pgfqpoint{1.333670in}{1.730237in}}%
\pgfpathlineto{\pgfqpoint{1.337173in}{1.730683in}}%
\pgfpathlineto{\pgfqpoint{1.337154in}{1.730477in}}%
\pgfpathlineto{\pgfqpoint{1.337149in}{1.730270in}}%
\pgfpathlineto{\pgfqpoint{1.337158in}{1.730064in}}%
\pgfpathlineto{\pgfqpoint{1.337182in}{1.729858in}}%
\pgfpathclose%
\pgfusepath{fill}%
\end{pgfscope}%
\begin{pgfscope}%
\pgfpathrectangle{\pgfqpoint{0.329460in}{0.284240in}}{\pgfqpoint{1.989680in}{1.989680in}}%
\pgfusepath{clip}%
\pgfsetbuttcap%
\pgfsetroundjoin%
\definecolor{currentfill}{rgb}{0.993248,0.906157,0.143936}%
\pgfsetfillcolor{currentfill}%
\pgfsetlinewidth{0.000000pt}%
\definecolor{currentstroke}{rgb}{0.000000,0.000000,0.000000}%
\pgfsetstrokecolor{currentstroke}%
\pgfsetdash{}{0pt}%
\pgfpathmoveto{\pgfqpoint{1.351188in}{1.731270in}}%
\pgfpathlineto{\pgfqpoint{1.354619in}{1.731553in}}%
\pgfpathlineto{\pgfqpoint{1.358050in}{1.731716in}}%
\pgfpathlineto{\pgfqpoint{1.361480in}{1.731759in}}%
\pgfpathlineto{\pgfqpoint{1.364910in}{1.731682in}}%
\pgfpathlineto{\pgfqpoint{1.364997in}{1.731480in}}%
\pgfpathlineto{\pgfqpoint{1.365070in}{1.731276in}}%
\pgfpathlineto{\pgfqpoint{1.365129in}{1.731072in}}%
\pgfpathlineto{\pgfqpoint{1.365174in}{1.730866in}}%
\pgfpathlineto{\pgfqpoint{1.361678in}{1.731147in}}%
\pgfpathlineto{\pgfqpoint{1.358181in}{1.731308in}}%
\pgfpathlineto{\pgfqpoint{1.354685in}{1.731349in}}%
\pgfpathlineto{\pgfqpoint{1.351188in}{1.731270in}}%
\pgfpathlineto{\pgfqpoint{1.351188in}{1.731270in}}%
\pgfpathlineto{\pgfqpoint{1.351188in}{1.731270in}}%
\pgfpathlineto{\pgfqpoint{1.351188in}{1.731270in}}%
\pgfpathlineto{\pgfqpoint{1.351188in}{1.731270in}}%
\pgfpathclose%
\pgfusepath{fill}%
\end{pgfscope}%
\begin{pgfscope}%
\pgfpathrectangle{\pgfqpoint{0.329460in}{0.284240in}}{\pgfqpoint{1.989680in}{1.989680in}}%
\pgfusepath{clip}%
\pgfsetbuttcap%
\pgfsetroundjoin%
\definecolor{currentfill}{rgb}{0.955300,0.901065,0.118128}%
\pgfsetfillcolor{currentfill}%
\pgfsetlinewidth{0.000000pt}%
\definecolor{currentstroke}{rgb}{0.000000,0.000000,0.000000}%
\pgfsetstrokecolor{currentstroke}%
\pgfsetdash{}{0pt}%
\pgfpathmoveto{\pgfqpoint{1.309864in}{1.718842in}}%
\pgfpathlineto{\pgfqpoint{1.306422in}{1.717034in}}%
\pgfpathlineto{\pgfqpoint{1.302980in}{1.715109in}}%
\pgfpathlineto{\pgfqpoint{1.299538in}{1.713067in}}%
\pgfpathlineto{\pgfqpoint{1.296097in}{1.710909in}}%
\pgfpathlineto{\pgfqpoint{1.295788in}{1.711724in}}%
\pgfpathlineto{\pgfqpoint{1.295534in}{1.712544in}}%
\pgfpathlineto{\pgfqpoint{1.295336in}{1.713366in}}%
\pgfpathlineto{\pgfqpoint{1.295193in}{1.714191in}}%
\pgfpathlineto{\pgfqpoint{1.298689in}{1.716143in}}%
\pgfpathlineto{\pgfqpoint{1.302185in}{1.717979in}}%
\pgfpathlineto{\pgfqpoint{1.305682in}{1.719699in}}%
\pgfpathlineto{\pgfqpoint{1.309180in}{1.721301in}}%
\pgfpathlineto{\pgfqpoint{1.309289in}{1.720683in}}%
\pgfpathlineto{\pgfqpoint{1.309439in}{1.720067in}}%
\pgfpathlineto{\pgfqpoint{1.309631in}{1.719453in}}%
\pgfpathlineto{\pgfqpoint{1.309864in}{1.718842in}}%
\pgfpathclose%
\pgfusepath{fill}%
\end{pgfscope}%
\begin{pgfscope}%
\pgfpathrectangle{\pgfqpoint{0.329460in}{0.284240in}}{\pgfqpoint{1.989680in}{1.989680in}}%
\pgfusepath{clip}%
\pgfsetbuttcap%
\pgfsetroundjoin%
\definecolor{currentfill}{rgb}{0.896320,0.893616,0.096335}%
\pgfsetfillcolor{currentfill}%
\pgfsetlinewidth{0.000000pt}%
\definecolor{currentstroke}{rgb}{0.000000,0.000000,0.000000}%
\pgfsetstrokecolor{currentstroke}%
\pgfsetdash{}{0pt}%
\pgfpathmoveto{\pgfqpoint{1.420380in}{1.702030in}}%
\pgfpathlineto{\pgfqpoint{1.423833in}{1.699344in}}%
\pgfpathlineto{\pgfqpoint{1.427286in}{1.696545in}}%
\pgfpathlineto{\pgfqpoint{1.430738in}{1.693634in}}%
\pgfpathlineto{\pgfqpoint{1.434189in}{1.690612in}}%
\pgfpathlineto{\pgfqpoint{1.433722in}{1.689386in}}%
\pgfpathlineto{\pgfqpoint{1.433172in}{1.688167in}}%
\pgfpathlineto{\pgfqpoint{1.432540in}{1.686957in}}%
\pgfpathlineto{\pgfqpoint{1.431827in}{1.685757in}}%
\pgfpathlineto{\pgfqpoint{1.428472in}{1.688983in}}%
\pgfpathlineto{\pgfqpoint{1.425116in}{1.692098in}}%
\pgfpathlineto{\pgfqpoint{1.421759in}{1.695101in}}%
\pgfpathlineto{\pgfqpoint{1.418401in}{1.697991in}}%
\pgfpathlineto{\pgfqpoint{1.418998in}{1.698989in}}%
\pgfpathlineto{\pgfqpoint{1.419527in}{1.699996in}}%
\pgfpathlineto{\pgfqpoint{1.419988in}{1.701010in}}%
\pgfpathlineto{\pgfqpoint{1.420380in}{1.702030in}}%
\pgfpathclose%
\pgfusepath{fill}%
\end{pgfscope}%
\begin{pgfscope}%
\pgfpathrectangle{\pgfqpoint{0.329460in}{0.284240in}}{\pgfqpoint{1.989680in}{1.989680in}}%
\pgfusepath{clip}%
\pgfsetbuttcap%
\pgfsetroundjoin%
\definecolor{currentfill}{rgb}{0.993248,0.906157,0.143936}%
\pgfsetfillcolor{currentfill}%
\pgfsetlinewidth{0.000000pt}%
\definecolor{currentstroke}{rgb}{0.000000,0.000000,0.000000}%
\pgfsetstrokecolor{currentstroke}%
\pgfsetdash{}{0pt}%
\pgfpathmoveto{\pgfqpoint{1.351188in}{1.731270in}}%
\pgfpathlineto{\pgfqpoint{1.347684in}{1.731303in}}%
\pgfpathlineto{\pgfqpoint{1.344180in}{1.731216in}}%
\pgfpathlineto{\pgfqpoint{1.340676in}{1.731010in}}%
\pgfpathlineto{\pgfqpoint{1.337173in}{1.730683in}}%
\pgfpathlineto{\pgfqpoint{1.337206in}{1.730889in}}%
\pgfpathlineto{\pgfqpoint{1.337252in}{1.731094in}}%
\pgfpathlineto{\pgfqpoint{1.337313in}{1.731299in}}%
\pgfpathlineto{\pgfqpoint{1.337387in}{1.731502in}}%
\pgfpathlineto{\pgfqpoint{1.340836in}{1.731624in}}%
\pgfpathlineto{\pgfqpoint{1.344286in}{1.731626in}}%
\pgfpathlineto{\pgfqpoint{1.347737in}{1.731508in}}%
\pgfpathlineto{\pgfqpoint{1.351188in}{1.731270in}}%
\pgfpathlineto{\pgfqpoint{1.351188in}{1.731270in}}%
\pgfpathlineto{\pgfqpoint{1.351188in}{1.731270in}}%
\pgfpathlineto{\pgfqpoint{1.351188in}{1.731270in}}%
\pgfpathlineto{\pgfqpoint{1.351188in}{1.731270in}}%
\pgfpathclose%
\pgfusepath{fill}%
\end{pgfscope}%
\begin{pgfscope}%
\pgfpathrectangle{\pgfqpoint{0.329460in}{0.284240in}}{\pgfqpoint{1.989680in}{1.989680in}}%
\pgfusepath{clip}%
\pgfsetbuttcap%
\pgfsetroundjoin%
\definecolor{currentfill}{rgb}{0.267004,0.004874,0.329415}%
\pgfsetfillcolor{currentfill}%
\pgfsetlinewidth{0.000000pt}%
\definecolor{currentstroke}{rgb}{0.000000,0.000000,0.000000}%
\pgfsetstrokecolor{currentstroke}%
\pgfsetdash{}{0pt}%
\pgfpathmoveto{\pgfqpoint{1.092739in}{0.780630in}}%
\pgfpathlineto{\pgfqpoint{1.091014in}{0.779187in}}%
\pgfpathlineto{\pgfqpoint{1.089284in}{0.777979in}}%
\pgfpathlineto{\pgfqpoint{1.087550in}{0.777010in}}%
\pgfpathlineto{\pgfqpoint{1.085812in}{0.776285in}}%
\pgfpathlineto{\pgfqpoint{1.070476in}{0.780962in}}%
\pgfpathlineto{\pgfqpoint{1.055453in}{0.785895in}}%
\pgfpathlineto{\pgfqpoint{1.040758in}{0.791079in}}%
\pgfpathlineto{\pgfqpoint{1.026408in}{0.796507in}}%
\pgfpathlineto{\pgfqpoint{1.028529in}{0.797099in}}%
\pgfpathlineto{\pgfqpoint{1.030644in}{0.797933in}}%
\pgfpathlineto{\pgfqpoint{1.032755in}{0.799006in}}%
\pgfpathlineto{\pgfqpoint{1.034860in}{0.800313in}}%
\pgfpathlineto{\pgfqpoint{1.048843in}{0.795030in}}%
\pgfpathlineto{\pgfqpoint{1.063161in}{0.789984in}}%
\pgfpathlineto{\pgfqpoint{1.077798in}{0.785182in}}%
\pgfpathlineto{\pgfqpoint{1.092739in}{0.780630in}}%
\pgfpathclose%
\pgfusepath{fill}%
\end{pgfscope}%
\begin{pgfscope}%
\pgfpathrectangle{\pgfqpoint{0.329460in}{0.284240in}}{\pgfqpoint{1.989680in}{1.989680in}}%
\pgfusepath{clip}%
\pgfsetbuttcap%
\pgfsetroundjoin%
\definecolor{currentfill}{rgb}{0.993248,0.906157,0.143936}%
\pgfsetfillcolor{currentfill}%
\pgfsetlinewidth{0.000000pt}%
\definecolor{currentstroke}{rgb}{0.000000,0.000000,0.000000}%
\pgfsetstrokecolor{currentstroke}%
\pgfsetdash{}{0pt}%
\pgfpathmoveto{\pgfqpoint{1.351188in}{1.731270in}}%
\pgfpathlineto{\pgfqpoint{1.354685in}{1.731349in}}%
\pgfpathlineto{\pgfqpoint{1.358181in}{1.731308in}}%
\pgfpathlineto{\pgfqpoint{1.361678in}{1.731147in}}%
\pgfpathlineto{\pgfqpoint{1.365174in}{1.730866in}}%
\pgfpathlineto{\pgfqpoint{1.365205in}{1.730660in}}%
\pgfpathlineto{\pgfqpoint{1.365222in}{1.730454in}}%
\pgfpathlineto{\pgfqpoint{1.365226in}{1.730248in}}%
\pgfpathlineto{\pgfqpoint{1.365215in}{1.730041in}}%
\pgfpathlineto{\pgfqpoint{1.361708in}{1.730528in}}%
\pgfpathlineto{\pgfqpoint{1.358201in}{1.730895in}}%
\pgfpathlineto{\pgfqpoint{1.354694in}{1.731143in}}%
\pgfpathlineto{\pgfqpoint{1.351188in}{1.731270in}}%
\pgfpathlineto{\pgfqpoint{1.351188in}{1.731270in}}%
\pgfpathlineto{\pgfqpoint{1.351188in}{1.731270in}}%
\pgfpathlineto{\pgfqpoint{1.351188in}{1.731270in}}%
\pgfpathlineto{\pgfqpoint{1.351188in}{1.731270in}}%
\pgfpathclose%
\pgfusepath{fill}%
\end{pgfscope}%
\begin{pgfscope}%
\pgfpathrectangle{\pgfqpoint{0.329460in}{0.284240in}}{\pgfqpoint{1.989680in}{1.989680in}}%
\pgfusepath{clip}%
\pgfsetbuttcap%
\pgfsetroundjoin%
\definecolor{currentfill}{rgb}{0.120081,0.622161,0.534946}%
\pgfsetfillcolor{currentfill}%
\pgfsetlinewidth{0.000000pt}%
\definecolor{currentstroke}{rgb}{0.000000,0.000000,0.000000}%
\pgfsetstrokecolor{currentstroke}%
\pgfsetdash{}{0pt}%
\pgfpathmoveto{\pgfqpoint{1.537108in}{1.337368in}}%
\pgfpathlineto{\pgfqpoint{1.539523in}{1.329311in}}%
\pgfpathlineto{\pgfqpoint{1.541938in}{1.321225in}}%
\pgfpathlineto{\pgfqpoint{1.544351in}{1.313113in}}%
\pgfpathlineto{\pgfqpoint{1.546764in}{1.304977in}}%
\pgfpathlineto{\pgfqpoint{1.540415in}{1.301939in}}%
\pgfpathlineto{\pgfqpoint{1.533868in}{1.299000in}}%
\pgfpathlineto{\pgfqpoint{1.527129in}{1.296165in}}%
\pgfpathlineto{\pgfqpoint{1.520206in}{1.293437in}}%
\pgfpathlineto{\pgfqpoint{1.518117in}{1.301730in}}%
\pgfpathlineto{\pgfqpoint{1.516028in}{1.309999in}}%
\pgfpathlineto{\pgfqpoint{1.513937in}{1.318242in}}%
\pgfpathlineto{\pgfqpoint{1.511846in}{1.326456in}}%
\pgfpathlineto{\pgfqpoint{1.518430in}{1.329035in}}%
\pgfpathlineto{\pgfqpoint{1.524840in}{1.331716in}}%
\pgfpathlineto{\pgfqpoint{1.531068in}{1.334494in}}%
\pgfpathlineto{\pgfqpoint{1.537108in}{1.337368in}}%
\pgfpathclose%
\pgfusepath{fill}%
\end{pgfscope}%
\begin{pgfscope}%
\pgfpathrectangle{\pgfqpoint{0.329460in}{0.284240in}}{\pgfqpoint{1.989680in}{1.989680in}}%
\pgfusepath{clip}%
\pgfsetbuttcap%
\pgfsetroundjoin%
\definecolor{currentfill}{rgb}{0.974417,0.903590,0.130215}%
\pgfsetfillcolor{currentfill}%
\pgfsetlinewidth{0.000000pt}%
\definecolor{currentstroke}{rgb}{0.000000,0.000000,0.000000}%
\pgfsetstrokecolor{currentstroke}%
\pgfsetdash{}{0pt}%
\pgfpathmoveto{\pgfqpoint{1.379239in}{1.726897in}}%
\pgfpathlineto{\pgfqpoint{1.382744in}{1.725813in}}%
\pgfpathlineto{\pgfqpoint{1.386249in}{1.724611in}}%
\pgfpathlineto{\pgfqpoint{1.389753in}{1.723290in}}%
\pgfpathlineto{\pgfqpoint{1.393256in}{1.721852in}}%
\pgfpathlineto{\pgfqpoint{1.393185in}{1.721232in}}%
\pgfpathlineto{\pgfqpoint{1.393072in}{1.720614in}}%
\pgfpathlineto{\pgfqpoint{1.392917in}{1.719998in}}%
\pgfpathlineto{\pgfqpoint{1.392720in}{1.719385in}}%
\pgfpathlineto{\pgfqpoint{1.389260in}{1.721030in}}%
\pgfpathlineto{\pgfqpoint{1.385800in}{1.722556in}}%
\pgfpathlineto{\pgfqpoint{1.382339in}{1.723965in}}%
\pgfpathlineto{\pgfqpoint{1.378878in}{1.725254in}}%
\pgfpathlineto{\pgfqpoint{1.379010in}{1.725663in}}%
\pgfpathlineto{\pgfqpoint{1.379114in}{1.726073in}}%
\pgfpathlineto{\pgfqpoint{1.379191in}{1.726485in}}%
\pgfpathlineto{\pgfqpoint{1.379239in}{1.726897in}}%
\pgfpathclose%
\pgfusepath{fill}%
\end{pgfscope}%
\begin{pgfscope}%
\pgfpathrectangle{\pgfqpoint{0.329460in}{0.284240in}}{\pgfqpoint{1.989680in}{1.989680in}}%
\pgfusepath{clip}%
\pgfsetbuttcap%
\pgfsetroundjoin%
\definecolor{currentfill}{rgb}{0.565498,0.842430,0.262877}%
\pgfsetfillcolor{currentfill}%
\pgfsetlinewidth{0.000000pt}%
\definecolor{currentstroke}{rgb}{0.000000,0.000000,0.000000}%
\pgfsetstrokecolor{currentstroke}%
\pgfsetdash{}{0pt}%
\pgfpathmoveto{\pgfqpoint{1.221428in}{1.580556in}}%
\pgfpathlineto{\pgfqpoint{1.218493in}{1.574777in}}%
\pgfpathlineto{\pgfqpoint{1.215559in}{1.568910in}}%
\pgfpathlineto{\pgfqpoint{1.212626in}{1.562957in}}%
\pgfpathlineto{\pgfqpoint{1.209694in}{1.556918in}}%
\pgfpathlineto{\pgfqpoint{1.206926in}{1.559082in}}%
\pgfpathlineto{\pgfqpoint{1.204305in}{1.561287in}}%
\pgfpathlineto{\pgfqpoint{1.201832in}{1.563529in}}%
\pgfpathlineto{\pgfqpoint{1.199512in}{1.565806in}}%
\pgfpathlineto{\pgfqpoint{1.202650in}{1.571652in}}%
\pgfpathlineto{\pgfqpoint{1.205789in}{1.577413in}}%
\pgfpathlineto{\pgfqpoint{1.208930in}{1.583088in}}%
\pgfpathlineto{\pgfqpoint{1.212073in}{1.588676in}}%
\pgfpathlineto{\pgfqpoint{1.214205in}{1.586595in}}%
\pgfpathlineto{\pgfqpoint{1.216477in}{1.584547in}}%
\pgfpathlineto{\pgfqpoint{1.218885in}{1.582533in}}%
\pgfpathlineto{\pgfqpoint{1.221428in}{1.580556in}}%
\pgfpathclose%
\pgfusepath{fill}%
\end{pgfscope}%
\begin{pgfscope}%
\pgfpathrectangle{\pgfqpoint{0.329460in}{0.284240in}}{\pgfqpoint{1.989680in}{1.989680in}}%
\pgfusepath{clip}%
\pgfsetbuttcap%
\pgfsetroundjoin%
\definecolor{currentfill}{rgb}{0.993248,0.906157,0.143936}%
\pgfsetfillcolor{currentfill}%
\pgfsetlinewidth{0.000000pt}%
\definecolor{currentstroke}{rgb}{0.000000,0.000000,0.000000}%
\pgfsetstrokecolor{currentstroke}%
\pgfsetdash{}{0pt}%
\pgfpathmoveto{\pgfqpoint{1.351188in}{1.731270in}}%
\pgfpathlineto{\pgfqpoint{1.347686in}{1.731097in}}%
\pgfpathlineto{\pgfqpoint{1.344185in}{1.730804in}}%
\pgfpathlineto{\pgfqpoint{1.340683in}{1.730391in}}%
\pgfpathlineto{\pgfqpoint{1.337182in}{1.729858in}}%
\pgfpathlineto{\pgfqpoint{1.337158in}{1.730064in}}%
\pgfpathlineto{\pgfqpoint{1.337149in}{1.730270in}}%
\pgfpathlineto{\pgfqpoint{1.337154in}{1.730477in}}%
\pgfpathlineto{\pgfqpoint{1.337173in}{1.730683in}}%
\pgfpathlineto{\pgfqpoint{1.340676in}{1.731010in}}%
\pgfpathlineto{\pgfqpoint{1.344180in}{1.731216in}}%
\pgfpathlineto{\pgfqpoint{1.347684in}{1.731303in}}%
\pgfpathlineto{\pgfqpoint{1.351188in}{1.731270in}}%
\pgfpathlineto{\pgfqpoint{1.351188in}{1.731270in}}%
\pgfpathlineto{\pgfqpoint{1.351188in}{1.731270in}}%
\pgfpathlineto{\pgfqpoint{1.351188in}{1.731270in}}%
\pgfpathlineto{\pgfqpoint{1.351188in}{1.731270in}}%
\pgfpathclose%
\pgfusepath{fill}%
\end{pgfscope}%
\begin{pgfscope}%
\pgfpathrectangle{\pgfqpoint{0.329460in}{0.284240in}}{\pgfqpoint{1.989680in}{1.989680in}}%
\pgfusepath{clip}%
\pgfsetbuttcap%
\pgfsetroundjoin%
\definecolor{currentfill}{rgb}{0.271305,0.019942,0.347269}%
\pgfsetfillcolor{currentfill}%
\pgfsetlinewidth{0.000000pt}%
\definecolor{currentstroke}{rgb}{0.000000,0.000000,0.000000}%
\pgfsetstrokecolor{currentstroke}%
\pgfsetdash{}{0pt}%
\pgfpathmoveto{\pgfqpoint{1.170719in}{0.800303in}}%
\pgfpathlineto{\pgfqpoint{1.169440in}{0.796282in}}%
\pgfpathlineto{\pgfqpoint{1.168159in}{0.792444in}}%
\pgfpathlineto{\pgfqpoint{1.166876in}{0.788792in}}%
\pgfpathlineto{\pgfqpoint{1.165590in}{0.785331in}}%
\pgfpathlineto{\pgfqpoint{1.150441in}{0.788636in}}%
\pgfpathlineto{\pgfqpoint{1.135516in}{0.792196in}}%
\pgfpathlineto{\pgfqpoint{1.120830in}{0.796008in}}%
\pgfpathlineto{\pgfqpoint{1.106399in}{0.800065in}}%
\pgfpathlineto{\pgfqpoint{1.108091in}{0.803417in}}%
\pgfpathlineto{\pgfqpoint{1.109780in}{0.806959in}}%
\pgfpathlineto{\pgfqpoint{1.111466in}{0.810688in}}%
\pgfpathlineto{\pgfqpoint{1.113149in}{0.814599in}}%
\pgfpathlineto{\pgfqpoint{1.127186in}{0.810662in}}%
\pgfpathlineto{\pgfqpoint{1.141470in}{0.806964in}}%
\pgfpathlineto{\pgfqpoint{1.155986in}{0.803510in}}%
\pgfpathlineto{\pgfqpoint{1.170719in}{0.800303in}}%
\pgfpathclose%
\pgfusepath{fill}%
\end{pgfscope}%
\begin{pgfscope}%
\pgfpathrectangle{\pgfqpoint{0.329460in}{0.284240in}}{\pgfqpoint{1.989680in}{1.989680in}}%
\pgfusepath{clip}%
\pgfsetbuttcap%
\pgfsetroundjoin%
\definecolor{currentfill}{rgb}{0.147607,0.511733,0.557049}%
\pgfsetfillcolor{currentfill}%
\pgfsetlinewidth{0.000000pt}%
\definecolor{currentstroke}{rgb}{0.000000,0.000000,0.000000}%
\pgfsetstrokecolor{currentstroke}%
\pgfsetdash{}{0pt}%
\pgfpathmoveto{\pgfqpoint{1.536887in}{1.226513in}}%
\pgfpathlineto{\pgfqpoint{1.538969in}{1.218113in}}%
\pgfpathlineto{\pgfqpoint{1.541050in}{1.209712in}}%
\pgfpathlineto{\pgfqpoint{1.543131in}{1.201315in}}%
\pgfpathlineto{\pgfqpoint{1.545211in}{1.192924in}}%
\pgfpathlineto{\pgfqpoint{1.537071in}{1.189870in}}%
\pgfpathlineto{\pgfqpoint{1.528735in}{1.186946in}}%
\pgfpathlineto{\pgfqpoint{1.520210in}{1.184157in}}%
\pgfpathlineto{\pgfqpoint{1.511505in}{1.181506in}}%
\pgfpathlineto{\pgfqpoint{1.509783in}{1.190035in}}%
\pgfpathlineto{\pgfqpoint{1.508061in}{1.198569in}}%
\pgfpathlineto{\pgfqpoint{1.506338in}{1.207107in}}%
\pgfpathlineto{\pgfqpoint{1.504615in}{1.215644in}}%
\pgfpathlineto{\pgfqpoint{1.512949in}{1.218168in}}%
\pgfpathlineto{\pgfqpoint{1.521111in}{1.220823in}}%
\pgfpathlineto{\pgfqpoint{1.529093in}{1.223606in}}%
\pgfpathlineto{\pgfqpoint{1.536887in}{1.226513in}}%
\pgfpathclose%
\pgfusepath{fill}%
\end{pgfscope}%
\begin{pgfscope}%
\pgfpathrectangle{\pgfqpoint{0.329460in}{0.284240in}}{\pgfqpoint{1.989680in}{1.989680in}}%
\pgfusepath{clip}%
\pgfsetbuttcap%
\pgfsetroundjoin%
\definecolor{currentfill}{rgb}{0.220124,0.725509,0.466226}%
\pgfsetfillcolor{currentfill}%
\pgfsetlinewidth{0.000000pt}%
\definecolor{currentstroke}{rgb}{0.000000,0.000000,0.000000}%
\pgfsetstrokecolor{currentstroke}%
\pgfsetdash{}{0pt}%
\pgfpathmoveto{\pgfqpoint{1.526784in}{1.441329in}}%
\pgfpathlineto{\pgfqpoint{1.529500in}{1.433955in}}%
\pgfpathlineto{\pgfqpoint{1.532213in}{1.426528in}}%
\pgfpathlineto{\pgfqpoint{1.534926in}{1.419048in}}%
\pgfpathlineto{\pgfqpoint{1.537637in}{1.411519in}}%
\pgfpathlineto{\pgfqpoint{1.532940in}{1.408661in}}%
\pgfpathlineto{\pgfqpoint{1.528055in}{1.405876in}}%
\pgfpathlineto{\pgfqpoint{1.522987in}{1.403168in}}%
\pgfpathlineto{\pgfqpoint{1.517741in}{1.400538in}}%
\pgfpathlineto{\pgfqpoint{1.515315in}{1.408240in}}%
\pgfpathlineto{\pgfqpoint{1.512888in}{1.415893in}}%
\pgfpathlineto{\pgfqpoint{1.510460in}{1.423493in}}%
\pgfpathlineto{\pgfqpoint{1.508031in}{1.431039in}}%
\pgfpathlineto{\pgfqpoint{1.512975in}{1.433503in}}%
\pgfpathlineto{\pgfqpoint{1.517751in}{1.436041in}}%
\pgfpathlineto{\pgfqpoint{1.522356in}{1.438651in}}%
\pgfpathlineto{\pgfqpoint{1.526784in}{1.441329in}}%
\pgfpathclose%
\pgfusepath{fill}%
\end{pgfscope}%
\begin{pgfscope}%
\pgfpathrectangle{\pgfqpoint{0.329460in}{0.284240in}}{\pgfqpoint{1.989680in}{1.989680in}}%
\pgfusepath{clip}%
\pgfsetbuttcap%
\pgfsetroundjoin%
\definecolor{currentfill}{rgb}{0.993248,0.906157,0.143936}%
\pgfsetfillcolor{currentfill}%
\pgfsetlinewidth{0.000000pt}%
\definecolor{currentstroke}{rgb}{0.000000,0.000000,0.000000}%
\pgfsetstrokecolor{currentstroke}%
\pgfsetdash{}{0pt}%
\pgfpathmoveto{\pgfqpoint{1.351188in}{1.731270in}}%
\pgfpathlineto{\pgfqpoint{1.354694in}{1.731143in}}%
\pgfpathlineto{\pgfqpoint{1.358201in}{1.730895in}}%
\pgfpathlineto{\pgfqpoint{1.361708in}{1.730528in}}%
\pgfpathlineto{\pgfqpoint{1.365215in}{1.730041in}}%
\pgfpathlineto{\pgfqpoint{1.365190in}{1.729835in}}%
\pgfpathlineto{\pgfqpoint{1.365151in}{1.729629in}}%
\pgfpathlineto{\pgfqpoint{1.365099in}{1.729424in}}%
\pgfpathlineto{\pgfqpoint{1.365032in}{1.729220in}}%
\pgfpathlineto{\pgfqpoint{1.361571in}{1.729912in}}%
\pgfpathlineto{\pgfqpoint{1.358109in}{1.730485in}}%
\pgfpathlineto{\pgfqpoint{1.354648in}{1.730938in}}%
\pgfpathlineto{\pgfqpoint{1.351188in}{1.731270in}}%
\pgfpathlineto{\pgfqpoint{1.351188in}{1.731270in}}%
\pgfpathlineto{\pgfqpoint{1.351188in}{1.731270in}}%
\pgfpathlineto{\pgfqpoint{1.351188in}{1.731270in}}%
\pgfpathlineto{\pgfqpoint{1.351188in}{1.731270in}}%
\pgfpathclose%
\pgfusepath{fill}%
\end{pgfscope}%
\begin{pgfscope}%
\pgfpathrectangle{\pgfqpoint{0.329460in}{0.284240in}}{\pgfqpoint{1.989680in}{1.989680in}}%
\pgfusepath{clip}%
\pgfsetbuttcap%
\pgfsetroundjoin%
\definecolor{currentfill}{rgb}{0.762373,0.876424,0.137064}%
\pgfsetfillcolor{currentfill}%
\pgfsetlinewidth{0.000000pt}%
\definecolor{currentstroke}{rgb}{0.000000,0.000000,0.000000}%
\pgfsetstrokecolor{currentstroke}%
\pgfsetdash{}{0pt}%
\pgfpathmoveto{\pgfqpoint{1.249889in}{1.648409in}}%
\pgfpathlineto{\pgfqpoint{1.246731in}{1.643975in}}%
\pgfpathlineto{\pgfqpoint{1.243574in}{1.639438in}}%
\pgfpathlineto{\pgfqpoint{1.240418in}{1.634799in}}%
\pgfpathlineto{\pgfqpoint{1.237264in}{1.630060in}}%
\pgfpathlineto{\pgfqpoint{1.235625in}{1.631777in}}%
\pgfpathlineto{\pgfqpoint{1.234103in}{1.633517in}}%
\pgfpathlineto{\pgfqpoint{1.232699in}{1.635278in}}%
\pgfpathlineto{\pgfqpoint{1.231414in}{1.637059in}}%
\pgfpathlineto{\pgfqpoint{1.234727in}{1.641599in}}%
\pgfpathlineto{\pgfqpoint{1.238041in}{1.646039in}}%
\pgfpathlineto{\pgfqpoint{1.241357in}{1.650378in}}%
\pgfpathlineto{\pgfqpoint{1.244674in}{1.654614in}}%
\pgfpathlineto{\pgfqpoint{1.245820in}{1.653035in}}%
\pgfpathlineto{\pgfqpoint{1.247072in}{1.651473in}}%
\pgfpathlineto{\pgfqpoint{1.248428in}{1.649931in}}%
\pgfpathlineto{\pgfqpoint{1.249889in}{1.648409in}}%
\pgfpathclose%
\pgfusepath{fill}%
\end{pgfscope}%
\begin{pgfscope}%
\pgfpathrectangle{\pgfqpoint{0.329460in}{0.284240in}}{\pgfqpoint{1.989680in}{1.989680in}}%
\pgfusepath{clip}%
\pgfsetbuttcap%
\pgfsetroundjoin%
\definecolor{currentfill}{rgb}{0.974417,0.903590,0.130215}%
\pgfsetfillcolor{currentfill}%
\pgfsetlinewidth{0.000000pt}%
\definecolor{currentstroke}{rgb}{0.000000,0.000000,0.000000}%
\pgfsetstrokecolor{currentstroke}%
\pgfsetdash{}{0pt}%
\pgfpathmoveto{\pgfqpoint{1.323638in}{1.724893in}}%
\pgfpathlineto{\pgfqpoint{1.320194in}{1.723558in}}%
\pgfpathlineto{\pgfqpoint{1.316751in}{1.722104in}}%
\pgfpathlineto{\pgfqpoint{1.313307in}{1.720532in}}%
\pgfpathlineto{\pgfqpoint{1.309864in}{1.718842in}}%
\pgfpathlineto{\pgfqpoint{1.309631in}{1.719453in}}%
\pgfpathlineto{\pgfqpoint{1.309439in}{1.720067in}}%
\pgfpathlineto{\pgfqpoint{1.309289in}{1.720683in}}%
\pgfpathlineto{\pgfqpoint{1.309180in}{1.721301in}}%
\pgfpathlineto{\pgfqpoint{1.312679in}{1.722786in}}%
\pgfpathlineto{\pgfqpoint{1.316178in}{1.724152in}}%
\pgfpathlineto{\pgfqpoint{1.319678in}{1.725401in}}%
\pgfpathlineto{\pgfqpoint{1.323178in}{1.726531in}}%
\pgfpathlineto{\pgfqpoint{1.323251in}{1.726119in}}%
\pgfpathlineto{\pgfqpoint{1.323352in}{1.725708in}}%
\pgfpathlineto{\pgfqpoint{1.323481in}{1.725300in}}%
\pgfpathlineto{\pgfqpoint{1.323638in}{1.724893in}}%
\pgfpathclose%
\pgfusepath{fill}%
\end{pgfscope}%
\begin{pgfscope}%
\pgfpathrectangle{\pgfqpoint{0.329460in}{0.284240in}}{\pgfqpoint{1.989680in}{1.989680in}}%
\pgfusepath{clip}%
\pgfsetbuttcap%
\pgfsetroundjoin%
\definecolor{currentfill}{rgb}{0.993248,0.906157,0.143936}%
\pgfsetfillcolor{currentfill}%
\pgfsetlinewidth{0.000000pt}%
\definecolor{currentstroke}{rgb}{0.000000,0.000000,0.000000}%
\pgfsetstrokecolor{currentstroke}%
\pgfsetdash{}{0pt}%
\pgfpathmoveto{\pgfqpoint{1.351188in}{1.731270in}}%
\pgfpathlineto{\pgfqpoint{1.347745in}{1.730892in}}%
\pgfpathlineto{\pgfqpoint{1.344301in}{1.730395in}}%
\pgfpathlineto{\pgfqpoint{1.340858in}{1.729777in}}%
\pgfpathlineto{\pgfqpoint{1.337414in}{1.729039in}}%
\pgfpathlineto{\pgfqpoint{1.337335in}{1.729243in}}%
\pgfpathlineto{\pgfqpoint{1.337270in}{1.729447in}}%
\pgfpathlineto{\pgfqpoint{1.337219in}{1.729652in}}%
\pgfpathlineto{\pgfqpoint{1.337182in}{1.729858in}}%
\pgfpathlineto{\pgfqpoint{1.340683in}{1.730391in}}%
\pgfpathlineto{\pgfqpoint{1.344185in}{1.730804in}}%
\pgfpathlineto{\pgfqpoint{1.347686in}{1.731097in}}%
\pgfpathlineto{\pgfqpoint{1.351188in}{1.731270in}}%
\pgfpathlineto{\pgfqpoint{1.351188in}{1.731270in}}%
\pgfpathlineto{\pgfqpoint{1.351188in}{1.731270in}}%
\pgfpathlineto{\pgfqpoint{1.351188in}{1.731270in}}%
\pgfpathlineto{\pgfqpoint{1.351188in}{1.731270in}}%
\pgfpathclose%
\pgfusepath{fill}%
\end{pgfscope}%
\begin{pgfscope}%
\pgfpathrectangle{\pgfqpoint{0.329460in}{0.284240in}}{\pgfqpoint{1.989680in}{1.989680in}}%
\pgfusepath{clip}%
\pgfsetbuttcap%
\pgfsetroundjoin%
\definecolor{currentfill}{rgb}{0.993248,0.906157,0.143936}%
\pgfsetfillcolor{currentfill}%
\pgfsetlinewidth{0.000000pt}%
\definecolor{currentstroke}{rgb}{0.000000,0.000000,0.000000}%
\pgfsetstrokecolor{currentstroke}%
\pgfsetdash{}{0pt}%
\pgfpathmoveto{\pgfqpoint{1.365215in}{1.730041in}}%
\pgfpathlineto{\pgfqpoint{1.368721in}{1.729434in}}%
\pgfpathlineto{\pgfqpoint{1.372228in}{1.728708in}}%
\pgfpathlineto{\pgfqpoint{1.375733in}{1.727862in}}%
\pgfpathlineto{\pgfqpoint{1.379239in}{1.726897in}}%
\pgfpathlineto{\pgfqpoint{1.379191in}{1.726485in}}%
\pgfpathlineto{\pgfqpoint{1.379114in}{1.726073in}}%
\pgfpathlineto{\pgfqpoint{1.379010in}{1.725663in}}%
\pgfpathlineto{\pgfqpoint{1.378878in}{1.725254in}}%
\pgfpathlineto{\pgfqpoint{1.375416in}{1.726425in}}%
\pgfpathlineto{\pgfqpoint{1.371955in}{1.727476in}}%
\pgfpathlineto{\pgfqpoint{1.368494in}{1.728408in}}%
\pgfpathlineto{\pgfqpoint{1.365032in}{1.729220in}}%
\pgfpathlineto{\pgfqpoint{1.365099in}{1.729424in}}%
\pgfpathlineto{\pgfqpoint{1.365151in}{1.729629in}}%
\pgfpathlineto{\pgfqpoint{1.365190in}{1.729835in}}%
\pgfpathlineto{\pgfqpoint{1.365215in}{1.730041in}}%
\pgfpathclose%
\pgfusepath{fill}%
\end{pgfscope}%
\begin{pgfscope}%
\pgfpathrectangle{\pgfqpoint{0.329460in}{0.284240in}}{\pgfqpoint{1.989680in}{1.989680in}}%
\pgfusepath{clip}%
\pgfsetbuttcap%
\pgfsetroundjoin%
\definecolor{currentfill}{rgb}{0.212395,0.359683,0.551710}%
\pgfsetfillcolor{currentfill}%
\pgfsetlinewidth{0.000000pt}%
\definecolor{currentstroke}{rgb}{0.000000,0.000000,0.000000}%
\pgfsetstrokecolor{currentstroke}%
\pgfsetdash{}{0pt}%
\pgfpathmoveto{\pgfqpoint{1.220811in}{1.068081in}}%
\pgfpathlineto{\pgfqpoint{1.219570in}{1.059737in}}%
\pgfpathlineto{\pgfqpoint{1.218330in}{1.051438in}}%
\pgfpathlineto{\pgfqpoint{1.217089in}{1.043185in}}%
\pgfpathlineto{\pgfqpoint{1.215847in}{1.034983in}}%
\pgfpathlineto{\pgfqpoint{1.204781in}{1.037303in}}%
\pgfpathlineto{\pgfqpoint{1.193874in}{1.039804in}}%
\pgfpathlineto{\pgfqpoint{1.183136in}{1.042482in}}%
\pgfpathlineto{\pgfqpoint{1.172579in}{1.045334in}}%
\pgfpathlineto{\pgfqpoint{1.174214in}{1.053429in}}%
\pgfpathlineto{\pgfqpoint{1.175849in}{1.061574in}}%
\pgfpathlineto{\pgfqpoint{1.177485in}{1.069766in}}%
\pgfpathlineto{\pgfqpoint{1.179120in}{1.078003in}}%
\pgfpathlineto{\pgfqpoint{1.189293in}{1.075269in}}%
\pgfpathlineto{\pgfqpoint{1.199639in}{1.072703in}}%
\pgfpathlineto{\pgfqpoint{1.210149in}{1.070306in}}%
\pgfpathlineto{\pgfqpoint{1.220811in}{1.068081in}}%
\pgfpathclose%
\pgfusepath{fill}%
\end{pgfscope}%
\begin{pgfscope}%
\pgfpathrectangle{\pgfqpoint{0.329460in}{0.284240in}}{\pgfqpoint{1.989680in}{1.989680in}}%
\pgfusepath{clip}%
\pgfsetbuttcap%
\pgfsetroundjoin%
\definecolor{currentfill}{rgb}{0.896320,0.893616,0.096335}%
\pgfsetfillcolor{currentfill}%
\pgfsetlinewidth{0.000000pt}%
\definecolor{currentstroke}{rgb}{0.000000,0.000000,0.000000}%
\pgfsetstrokecolor{currentstroke}%
\pgfsetdash{}{0pt}%
\pgfpathmoveto{\pgfqpoint{1.284561in}{1.697111in}}%
\pgfpathlineto{\pgfqpoint{1.281232in}{1.694177in}}%
\pgfpathlineto{\pgfqpoint{1.277904in}{1.691129in}}%
\pgfpathlineto{\pgfqpoint{1.274577in}{1.687970in}}%
\pgfpathlineto{\pgfqpoint{1.271250in}{1.684699in}}%
\pgfpathlineto{\pgfqpoint{1.270465in}{1.685890in}}%
\pgfpathlineto{\pgfqpoint{1.269761in}{1.687091in}}%
\pgfpathlineto{\pgfqpoint{1.269138in}{1.688302in}}%
\pgfpathlineto{\pgfqpoint{1.268598in}{1.689522in}}%
\pgfpathlineto{\pgfqpoint{1.272032in}{1.692590in}}%
\pgfpathlineto{\pgfqpoint{1.275467in}{1.695546in}}%
\pgfpathlineto{\pgfqpoint{1.278903in}{1.698391in}}%
\pgfpathlineto{\pgfqpoint{1.282340in}{1.701123in}}%
\pgfpathlineto{\pgfqpoint{1.282793in}{1.700108in}}%
\pgfpathlineto{\pgfqpoint{1.283315in}{1.699101in}}%
\pgfpathlineto{\pgfqpoint{1.283904in}{1.698101in}}%
\pgfpathlineto{\pgfqpoint{1.284561in}{1.697111in}}%
\pgfpathclose%
\pgfusepath{fill}%
\end{pgfscope}%
\begin{pgfscope}%
\pgfpathrectangle{\pgfqpoint{0.329460in}{0.284240in}}{\pgfqpoint{1.989680in}{1.989680in}}%
\pgfusepath{clip}%
\pgfsetbuttcap%
\pgfsetroundjoin%
\definecolor{currentfill}{rgb}{0.280255,0.165693,0.476498}%
\pgfsetfillcolor{currentfill}%
\pgfsetlinewidth{0.000000pt}%
\definecolor{currentstroke}{rgb}{0.000000,0.000000,0.000000}%
\pgfsetstrokecolor{currentstroke}%
\pgfsetdash{}{0pt}%
\pgfpathmoveto{\pgfqpoint{1.517719in}{0.915501in}}%
\pgfpathlineto{\pgfqpoint{1.519058in}{0.908654in}}%
\pgfpathlineto{\pgfqpoint{1.520399in}{0.901914in}}%
\pgfpathlineto{\pgfqpoint{1.521740in}{0.895286in}}%
\pgfpathlineto{\pgfqpoint{1.523083in}{0.888773in}}%
\pgfpathlineto{\pgfqpoint{1.509971in}{0.885987in}}%
\pgfpathlineto{\pgfqpoint{1.496686in}{0.883422in}}%
\pgfpathlineto{\pgfqpoint{1.483240in}{0.881081in}}%
\pgfpathlineto{\pgfqpoint{1.469649in}{0.878968in}}%
\pgfpathlineto{\pgfqpoint{1.468722in}{0.885567in}}%
\pgfpathlineto{\pgfqpoint{1.467796in}{0.892281in}}%
\pgfpathlineto{\pgfqpoint{1.466871in}{0.899107in}}%
\pgfpathlineto{\pgfqpoint{1.465947in}{0.906041in}}%
\pgfpathlineto{\pgfqpoint{1.479114in}{0.908080in}}%
\pgfpathlineto{\pgfqpoint{1.492141in}{0.910338in}}%
\pgfpathlineto{\pgfqpoint{1.505014in}{0.912813in}}%
\pgfpathlineto{\pgfqpoint{1.517719in}{0.915501in}}%
\pgfpathclose%
\pgfusepath{fill}%
\end{pgfscope}%
\begin{pgfscope}%
\pgfpathrectangle{\pgfqpoint{0.329460in}{0.284240in}}{\pgfqpoint{1.989680in}{1.989680in}}%
\pgfusepath{clip}%
\pgfsetbuttcap%
\pgfsetroundjoin%
\definecolor{currentfill}{rgb}{0.412913,0.803041,0.357269}%
\pgfsetfillcolor{currentfill}%
\pgfsetlinewidth{0.000000pt}%
\definecolor{currentstroke}{rgb}{0.000000,0.000000,0.000000}%
\pgfsetstrokecolor{currentstroke}%
\pgfsetdash{}{0pt}%
\pgfpathmoveto{\pgfqpoint{1.507061in}{1.534035in}}%
\pgfpathlineto{\pgfqpoint{1.510035in}{1.527637in}}%
\pgfpathlineto{\pgfqpoint{1.513008in}{1.521164in}}%
\pgfpathlineto{\pgfqpoint{1.515979in}{1.514617in}}%
\pgfpathlineto{\pgfqpoint{1.518949in}{1.507998in}}%
\pgfpathlineto{\pgfqpoint{1.515714in}{1.505460in}}%
\pgfpathlineto{\pgfqpoint{1.512311in}{1.502971in}}%
\pgfpathlineto{\pgfqpoint{1.508743in}{1.500535in}}%
\pgfpathlineto{\pgfqpoint{1.505013in}{1.498153in}}%
\pgfpathlineto{\pgfqpoint{1.502286in}{1.504957in}}%
\pgfpathlineto{\pgfqpoint{1.499557in}{1.511689in}}%
\pgfpathlineto{\pgfqpoint{1.496827in}{1.518346in}}%
\pgfpathlineto{\pgfqpoint{1.494096in}{1.524928in}}%
\pgfpathlineto{\pgfqpoint{1.497565in}{1.527131in}}%
\pgfpathlineto{\pgfqpoint{1.500884in}{1.529385in}}%
\pgfpathlineto{\pgfqpoint{1.504051in}{1.531687in}}%
\pgfpathlineto{\pgfqpoint{1.507061in}{1.534035in}}%
\pgfpathclose%
\pgfusepath{fill}%
\end{pgfscope}%
\begin{pgfscope}%
\pgfpathrectangle{\pgfqpoint{0.329460in}{0.284240in}}{\pgfqpoint{1.989680in}{1.989680in}}%
\pgfusepath{clip}%
\pgfsetbuttcap%
\pgfsetroundjoin%
\definecolor{currentfill}{rgb}{0.993248,0.906157,0.143936}%
\pgfsetfillcolor{currentfill}%
\pgfsetlinewidth{0.000000pt}%
\definecolor{currentstroke}{rgb}{0.000000,0.000000,0.000000}%
\pgfsetstrokecolor{currentstroke}%
\pgfsetdash{}{0pt}%
\pgfpathmoveto{\pgfqpoint{1.337414in}{1.729039in}}%
\pgfpathlineto{\pgfqpoint{1.333970in}{1.728182in}}%
\pgfpathlineto{\pgfqpoint{1.330526in}{1.727205in}}%
\pgfpathlineto{\pgfqpoint{1.327082in}{1.726109in}}%
\pgfpathlineto{\pgfqpoint{1.323638in}{1.724893in}}%
\pgfpathlineto{\pgfqpoint{1.323481in}{1.725300in}}%
\pgfpathlineto{\pgfqpoint{1.323352in}{1.725708in}}%
\pgfpathlineto{\pgfqpoint{1.323251in}{1.726119in}}%
\pgfpathlineto{\pgfqpoint{1.323178in}{1.726531in}}%
\pgfpathlineto{\pgfqpoint{1.326678in}{1.727541in}}%
\pgfpathlineto{\pgfqpoint{1.330179in}{1.728433in}}%
\pgfpathlineto{\pgfqpoint{1.333680in}{1.729205in}}%
\pgfpathlineto{\pgfqpoint{1.337182in}{1.729858in}}%
\pgfpathlineto{\pgfqpoint{1.337219in}{1.729652in}}%
\pgfpathlineto{\pgfqpoint{1.337270in}{1.729447in}}%
\pgfpathlineto{\pgfqpoint{1.337335in}{1.729243in}}%
\pgfpathlineto{\pgfqpoint{1.337414in}{1.729039in}}%
\pgfpathclose%
\pgfusepath{fill}%
\end{pgfscope}%
\begin{pgfscope}%
\pgfpathrectangle{\pgfqpoint{0.329460in}{0.284240in}}{\pgfqpoint{1.989680in}{1.989680in}}%
\pgfusepath{clip}%
\pgfsetbuttcap%
\pgfsetroundjoin%
\definecolor{currentfill}{rgb}{0.993248,0.906157,0.143936}%
\pgfsetfillcolor{currentfill}%
\pgfsetlinewidth{0.000000pt}%
\definecolor{currentstroke}{rgb}{0.000000,0.000000,0.000000}%
\pgfsetstrokecolor{currentstroke}%
\pgfsetdash{}{0pt}%
\pgfpathmoveto{\pgfqpoint{1.351188in}{1.731270in}}%
\pgfpathlineto{\pgfqpoint{1.354648in}{1.730938in}}%
\pgfpathlineto{\pgfqpoint{1.358109in}{1.730485in}}%
\pgfpathlineto{\pgfqpoint{1.361571in}{1.729912in}}%
\pgfpathlineto{\pgfqpoint{1.365032in}{1.729220in}}%
\pgfpathlineto{\pgfqpoint{1.364952in}{1.729017in}}%
\pgfpathlineto{\pgfqpoint{1.364858in}{1.728815in}}%
\pgfpathlineto{\pgfqpoint{1.364750in}{1.728615in}}%
\pgfpathlineto{\pgfqpoint{1.364629in}{1.728416in}}%
\pgfpathlineto{\pgfqpoint{1.361268in}{1.729310in}}%
\pgfpathlineto{\pgfqpoint{1.357907in}{1.730083in}}%
\pgfpathlineto{\pgfqpoint{1.354547in}{1.730737in}}%
\pgfpathlineto{\pgfqpoint{1.351188in}{1.731270in}}%
\pgfpathlineto{\pgfqpoint{1.351188in}{1.731270in}}%
\pgfpathlineto{\pgfqpoint{1.351188in}{1.731270in}}%
\pgfpathlineto{\pgfqpoint{1.351188in}{1.731270in}}%
\pgfpathlineto{\pgfqpoint{1.351188in}{1.731270in}}%
\pgfpathclose%
\pgfusepath{fill}%
\end{pgfscope}%
\begin{pgfscope}%
\pgfpathrectangle{\pgfqpoint{0.329460in}{0.284240in}}{\pgfqpoint{1.989680in}{1.989680in}}%
\pgfusepath{clip}%
\pgfsetbuttcap%
\pgfsetroundjoin%
\definecolor{currentfill}{rgb}{0.283072,0.130895,0.449241}%
\pgfsetfillcolor{currentfill}%
\pgfsetlinewidth{0.000000pt}%
\definecolor{currentstroke}{rgb}{0.000000,0.000000,0.000000}%
\pgfsetstrokecolor{currentstroke}%
\pgfsetdash{}{0pt}%
\pgfpathmoveto{\pgfqpoint{1.523083in}{0.888773in}}%
\pgfpathlineto{\pgfqpoint{1.524427in}{0.882378in}}%
\pgfpathlineto{\pgfqpoint{1.525772in}{0.876106in}}%
\pgfpathlineto{\pgfqpoint{1.527118in}{0.869959in}}%
\pgfpathlineto{\pgfqpoint{1.528465in}{0.863940in}}%
\pgfpathlineto{\pgfqpoint{1.514945in}{0.861056in}}%
\pgfpathlineto{\pgfqpoint{1.501245in}{0.858401in}}%
\pgfpathlineto{\pgfqpoint{1.487380in}{0.855978in}}%
\pgfpathlineto{\pgfqpoint{1.473364in}{0.853790in}}%
\pgfpathlineto{\pgfqpoint{1.472434in}{0.859894in}}%
\pgfpathlineto{\pgfqpoint{1.471505in}{0.866128in}}%
\pgfpathlineto{\pgfqpoint{1.470576in}{0.872487in}}%
\pgfpathlineto{\pgfqpoint{1.469649in}{0.878968in}}%
\pgfpathlineto{\pgfqpoint{1.483240in}{0.881081in}}%
\pgfpathlineto{\pgfqpoint{1.496686in}{0.883422in}}%
\pgfpathlineto{\pgfqpoint{1.509971in}{0.885987in}}%
\pgfpathlineto{\pgfqpoint{1.523083in}{0.888773in}}%
\pgfpathclose%
\pgfusepath{fill}%
\end{pgfscope}%
\begin{pgfscope}%
\pgfpathrectangle{\pgfqpoint{0.329460in}{0.284240in}}{\pgfqpoint{1.989680in}{1.989680in}}%
\pgfusepath{clip}%
\pgfsetbuttcap%
\pgfsetroundjoin%
\definecolor{currentfill}{rgb}{0.993248,0.906157,0.143936}%
\pgfsetfillcolor{currentfill}%
\pgfsetlinewidth{0.000000pt}%
\definecolor{currentstroke}{rgb}{0.000000,0.000000,0.000000}%
\pgfsetstrokecolor{currentstroke}%
\pgfsetdash{}{0pt}%
\pgfpathmoveto{\pgfqpoint{1.351188in}{1.731270in}}%
\pgfpathlineto{\pgfqpoint{1.347858in}{1.730693in}}%
\pgfpathlineto{\pgfqpoint{1.344527in}{1.729996in}}%
\pgfpathlineto{\pgfqpoint{1.341197in}{1.729178in}}%
\pgfpathlineto{\pgfqpoint{1.337865in}{1.728241in}}%
\pgfpathlineto{\pgfqpoint{1.337732in}{1.728438in}}%
\pgfpathlineto{\pgfqpoint{1.337612in}{1.728637in}}%
\pgfpathlineto{\pgfqpoint{1.337506in}{1.728838in}}%
\pgfpathlineto{\pgfqpoint{1.337414in}{1.729039in}}%
\pgfpathlineto{\pgfqpoint{1.340858in}{1.729777in}}%
\pgfpathlineto{\pgfqpoint{1.344301in}{1.730395in}}%
\pgfpathlineto{\pgfqpoint{1.347745in}{1.730892in}}%
\pgfpathlineto{\pgfqpoint{1.351188in}{1.731270in}}%
\pgfpathlineto{\pgfqpoint{1.351188in}{1.731270in}}%
\pgfpathlineto{\pgfqpoint{1.351188in}{1.731270in}}%
\pgfpathlineto{\pgfqpoint{1.351188in}{1.731270in}}%
\pgfpathlineto{\pgfqpoint{1.351188in}{1.731270in}}%
\pgfpathclose%
\pgfusepath{fill}%
\end{pgfscope}%
\begin{pgfscope}%
\pgfpathrectangle{\pgfqpoint{0.329460in}{0.284240in}}{\pgfqpoint{1.989680in}{1.989680in}}%
\pgfusepath{clip}%
\pgfsetbuttcap%
\pgfsetroundjoin%
\definecolor{currentfill}{rgb}{0.935904,0.898570,0.108131}%
\pgfsetfillcolor{currentfill}%
\pgfsetlinewidth{0.000000pt}%
\definecolor{currentstroke}{rgb}{0.000000,0.000000,0.000000}%
\pgfsetstrokecolor{currentstroke}%
\pgfsetdash{}{0pt}%
\pgfpathmoveto{\pgfqpoint{1.406556in}{1.711633in}}%
\pgfpathlineto{\pgfqpoint{1.410013in}{1.709405in}}%
\pgfpathlineto{\pgfqpoint{1.413469in}{1.707061in}}%
\pgfpathlineto{\pgfqpoint{1.416925in}{1.704602in}}%
\pgfpathlineto{\pgfqpoint{1.420380in}{1.702030in}}%
\pgfpathlineto{\pgfqpoint{1.419988in}{1.701010in}}%
\pgfpathlineto{\pgfqpoint{1.419527in}{1.699996in}}%
\pgfpathlineto{\pgfqpoint{1.418998in}{1.698989in}}%
\pgfpathlineto{\pgfqpoint{1.418401in}{1.697991in}}%
\pgfpathlineto{\pgfqpoint{1.415043in}{1.700767in}}%
\pgfpathlineto{\pgfqpoint{1.411684in}{1.703429in}}%
\pgfpathlineto{\pgfqpoint{1.408325in}{1.705976in}}%
\pgfpathlineto{\pgfqpoint{1.404965in}{1.708407in}}%
\pgfpathlineto{\pgfqpoint{1.405445in}{1.709205in}}%
\pgfpathlineto{\pgfqpoint{1.405870in}{1.710009in}}%
\pgfpathlineto{\pgfqpoint{1.406240in}{1.710819in}}%
\pgfpathlineto{\pgfqpoint{1.406556in}{1.711633in}}%
\pgfpathclose%
\pgfusepath{fill}%
\end{pgfscope}%
\begin{pgfscope}%
\pgfpathrectangle{\pgfqpoint{0.329460in}{0.284240in}}{\pgfqpoint{1.989680in}{1.989680in}}%
\pgfusepath{clip}%
\pgfsetbuttcap%
\pgfsetroundjoin%
\definecolor{currentfill}{rgb}{0.195860,0.395433,0.555276}%
\pgfsetfillcolor{currentfill}%
\pgfsetlinewidth{0.000000pt}%
\definecolor{currentstroke}{rgb}{0.000000,0.000000,0.000000}%
\pgfsetstrokecolor{currentstroke}%
\pgfsetdash{}{0pt}%
\pgfpathmoveto{\pgfqpoint{1.525267in}{1.113794in}}%
\pgfpathlineto{\pgfqpoint{1.526986in}{1.105436in}}%
\pgfpathlineto{\pgfqpoint{1.528705in}{1.097111in}}%
\pgfpathlineto{\pgfqpoint{1.530424in}{1.088821in}}%
\pgfpathlineto{\pgfqpoint{1.532143in}{1.080571in}}%
\pgfpathlineto{\pgfqpoint{1.522134in}{1.077691in}}%
\pgfpathlineto{\pgfqpoint{1.511941in}{1.074976in}}%
\pgfpathlineto{\pgfqpoint{1.501576in}{1.072428in}}%
\pgfpathlineto{\pgfqpoint{1.491048in}{1.070050in}}%
\pgfpathlineto{\pgfqpoint{1.489718in}{1.078414in}}%
\pgfpathlineto{\pgfqpoint{1.488387in}{1.086817in}}%
\pgfpathlineto{\pgfqpoint{1.487056in}{1.095255in}}%
\pgfpathlineto{\pgfqpoint{1.485725in}{1.103726in}}%
\pgfpathlineto{\pgfqpoint{1.495854in}{1.106001in}}%
\pgfpathlineto{\pgfqpoint{1.505828in}{1.108440in}}%
\pgfpathlineto{\pgfqpoint{1.515635in}{1.111038in}}%
\pgfpathlineto{\pgfqpoint{1.525267in}{1.113794in}}%
\pgfpathclose%
\pgfusepath{fill}%
\end{pgfscope}%
\begin{pgfscope}%
\pgfpathrectangle{\pgfqpoint{0.329460in}{0.284240in}}{\pgfqpoint{1.989680in}{1.989680in}}%
\pgfusepath{clip}%
\pgfsetbuttcap%
\pgfsetroundjoin%
\definecolor{currentfill}{rgb}{0.274128,0.199721,0.498911}%
\pgfsetfillcolor{currentfill}%
\pgfsetlinewidth{0.000000pt}%
\definecolor{currentstroke}{rgb}{0.000000,0.000000,0.000000}%
\pgfsetstrokecolor{currentstroke}%
\pgfsetdash{}{0pt}%
\pgfpathmoveto{\pgfqpoint{1.512368in}{0.943903in}}%
\pgfpathlineto{\pgfqpoint{1.513705in}{0.936658in}}%
\pgfpathlineto{\pgfqpoint{1.515042in}{0.929506in}}%
\pgfpathlineto{\pgfqpoint{1.516380in}{0.922453in}}%
\pgfpathlineto{\pgfqpoint{1.517719in}{0.915501in}}%
\pgfpathlineto{\pgfqpoint{1.505014in}{0.912813in}}%
\pgfpathlineto{\pgfqpoint{1.492141in}{0.910338in}}%
\pgfpathlineto{\pgfqpoint{1.479114in}{0.908080in}}%
\pgfpathlineto{\pgfqpoint{1.465947in}{0.906041in}}%
\pgfpathlineto{\pgfqpoint{1.465023in}{0.913079in}}%
\pgfpathlineto{\pgfqpoint{1.464100in}{0.920218in}}%
\pgfpathlineto{\pgfqpoint{1.463177in}{0.927455in}}%
\pgfpathlineto{\pgfqpoint{1.462255in}{0.934786in}}%
\pgfpathlineto{\pgfqpoint{1.475000in}{0.936751in}}%
\pgfpathlineto{\pgfqpoint{1.487610in}{0.938927in}}%
\pgfpathlineto{\pgfqpoint{1.500070in}{0.941312in}}%
\pgfpathlineto{\pgfqpoint{1.512368in}{0.943903in}}%
\pgfpathclose%
\pgfusepath{fill}%
\end{pgfscope}%
\begin{pgfscope}%
\pgfpathrectangle{\pgfqpoint{0.329460in}{0.284240in}}{\pgfqpoint{1.989680in}{1.989680in}}%
\pgfusepath{clip}%
\pgfsetbuttcap%
\pgfsetroundjoin%
\definecolor{currentfill}{rgb}{0.636902,0.856542,0.216620}%
\pgfsetfillcolor{currentfill}%
\pgfsetlinewidth{0.000000pt}%
\definecolor{currentstroke}{rgb}{0.000000,0.000000,0.000000}%
\pgfsetstrokecolor{currentstroke}%
\pgfsetdash{}{0pt}%
\pgfpathmoveto{\pgfqpoint{1.479338in}{1.611822in}}%
\pgfpathlineto{\pgfqpoint{1.482526in}{1.606642in}}%
\pgfpathlineto{\pgfqpoint{1.485712in}{1.601370in}}%
\pgfpathlineto{\pgfqpoint{1.488897in}{1.596005in}}%
\pgfpathlineto{\pgfqpoint{1.492080in}{1.590551in}}%
\pgfpathlineto{\pgfqpoint{1.490073in}{1.588443in}}%
\pgfpathlineto{\pgfqpoint{1.487924in}{1.586366in}}%
\pgfpathlineto{\pgfqpoint{1.485637in}{1.584321in}}%
\pgfpathlineto{\pgfqpoint{1.483214in}{1.582312in}}%
\pgfpathlineto{\pgfqpoint{1.480227in}{1.587960in}}%
\pgfpathlineto{\pgfqpoint{1.477239in}{1.593518in}}%
\pgfpathlineto{\pgfqpoint{1.474249in}{1.598984in}}%
\pgfpathlineto{\pgfqpoint{1.471259in}{1.604356in}}%
\pgfpathlineto{\pgfqpoint{1.473466in}{1.606177in}}%
\pgfpathlineto{\pgfqpoint{1.475550in}{1.608029in}}%
\pgfpathlineto{\pgfqpoint{1.477508in}{1.609912in}}%
\pgfpathlineto{\pgfqpoint{1.479338in}{1.611822in}}%
\pgfpathclose%
\pgfusepath{fill}%
\end{pgfscope}%
\begin{pgfscope}%
\pgfpathrectangle{\pgfqpoint{0.329460in}{0.284240in}}{\pgfqpoint{1.989680in}{1.989680in}}%
\pgfusepath{clip}%
\pgfsetbuttcap%
\pgfsetroundjoin%
\definecolor{currentfill}{rgb}{0.268510,0.009605,0.335427}%
\pgfsetfillcolor{currentfill}%
\pgfsetlinewidth{0.000000pt}%
\definecolor{currentstroke}{rgb}{0.000000,0.000000,0.000000}%
\pgfsetstrokecolor{currentstroke}%
\pgfsetdash{}{0pt}%
\pgfpathmoveto{\pgfqpoint{1.608576in}{0.803875in}}%
\pgfpathlineto{\pgfqpoint{1.610358in}{0.800747in}}%
\pgfpathlineto{\pgfqpoint{1.612143in}{0.797818in}}%
\pgfpathlineto{\pgfqpoint{1.613931in}{0.795092in}}%
\pgfpathlineto{\pgfqpoint{1.615723in}{0.792574in}}%
\pgfpathlineto{\pgfqpoint{1.601143in}{0.788175in}}%
\pgfpathlineto{\pgfqpoint{1.586285in}{0.784025in}}%
\pgfpathlineto{\pgfqpoint{1.571166in}{0.780130in}}%
\pgfpathlineto{\pgfqpoint{1.555802in}{0.776493in}}%
\pgfpathlineto{\pgfqpoint{1.554413in}{0.779126in}}%
\pgfpathlineto{\pgfqpoint{1.553026in}{0.781968in}}%
\pgfpathlineto{\pgfqpoint{1.551642in}{0.785012in}}%
\pgfpathlineto{\pgfqpoint{1.550261in}{0.788256in}}%
\pgfpathlineto{\pgfqpoint{1.565212in}{0.791788in}}%
\pgfpathlineto{\pgfqpoint{1.579926in}{0.795572in}}%
\pgfpathlineto{\pgfqpoint{1.594386in}{0.799602in}}%
\pgfpathlineto{\pgfqpoint{1.608576in}{0.803875in}}%
\pgfpathclose%
\pgfusepath{fill}%
\end{pgfscope}%
\begin{pgfscope}%
\pgfpathrectangle{\pgfqpoint{0.329460in}{0.284240in}}{\pgfqpoint{1.989680in}{1.989680in}}%
\pgfusepath{clip}%
\pgfsetbuttcap%
\pgfsetroundjoin%
\definecolor{currentfill}{rgb}{0.814576,0.883393,0.110347}%
\pgfsetfillcolor{currentfill}%
\pgfsetlinewidth{0.000000pt}%
\definecolor{currentstroke}{rgb}{0.000000,0.000000,0.000000}%
\pgfsetstrokecolor{currentstroke}%
\pgfsetdash{}{0pt}%
\pgfpathmoveto{\pgfqpoint{1.445238in}{1.671752in}}%
\pgfpathlineto{\pgfqpoint{1.448588in}{1.667980in}}%
\pgfpathlineto{\pgfqpoint{1.451936in}{1.664102in}}%
\pgfpathlineto{\pgfqpoint{1.455284in}{1.660119in}}%
\pgfpathlineto{\pgfqpoint{1.458630in}{1.656031in}}%
\pgfpathlineto{\pgfqpoint{1.457579in}{1.654438in}}%
\pgfpathlineto{\pgfqpoint{1.456422in}{1.652861in}}%
\pgfpathlineto{\pgfqpoint{1.455158in}{1.651301in}}%
\pgfpathlineto{\pgfqpoint{1.453789in}{1.649761in}}%
\pgfpathlineto{\pgfqpoint{1.450591in}{1.654048in}}%
\pgfpathlineto{\pgfqpoint{1.447391in}{1.658231in}}%
\pgfpathlineto{\pgfqpoint{1.444190in}{1.662308in}}%
\pgfpathlineto{\pgfqpoint{1.440988in}{1.666279in}}%
\pgfpathlineto{\pgfqpoint{1.442189in}{1.667623in}}%
\pgfpathlineto{\pgfqpoint{1.443298in}{1.668984in}}%
\pgfpathlineto{\pgfqpoint{1.444315in}{1.670361in}}%
\pgfpathlineto{\pgfqpoint{1.445238in}{1.671752in}}%
\pgfpathclose%
\pgfusepath{fill}%
\end{pgfscope}%
\begin{pgfscope}%
\pgfpathrectangle{\pgfqpoint{0.329460in}{0.284240in}}{\pgfqpoint{1.989680in}{1.989680in}}%
\pgfusepath{clip}%
\pgfsetbuttcap%
\pgfsetroundjoin%
\definecolor{currentfill}{rgb}{0.993248,0.906157,0.143936}%
\pgfsetfillcolor{currentfill}%
\pgfsetlinewidth{0.000000pt}%
\definecolor{currentstroke}{rgb}{0.000000,0.000000,0.000000}%
\pgfsetstrokecolor{currentstroke}%
\pgfsetdash{}{0pt}%
\pgfpathmoveto{\pgfqpoint{1.351188in}{1.731270in}}%
\pgfpathlineto{\pgfqpoint{1.354547in}{1.730737in}}%
\pgfpathlineto{\pgfqpoint{1.357907in}{1.730083in}}%
\pgfpathlineto{\pgfqpoint{1.361268in}{1.729310in}}%
\pgfpathlineto{\pgfqpoint{1.364629in}{1.728416in}}%
\pgfpathlineto{\pgfqpoint{1.364494in}{1.728219in}}%
\pgfpathlineto{\pgfqpoint{1.364346in}{1.728025in}}%
\pgfpathlineto{\pgfqpoint{1.364185in}{1.727832in}}%
\pgfpathlineto{\pgfqpoint{1.364011in}{1.727642in}}%
\pgfpathlineto{\pgfqpoint{1.360804in}{1.728729in}}%
\pgfpathlineto{\pgfqpoint{1.357598in}{1.729696in}}%
\pgfpathlineto{\pgfqpoint{1.354393in}{1.730543in}}%
\pgfpathlineto{\pgfqpoint{1.351188in}{1.731270in}}%
\pgfpathlineto{\pgfqpoint{1.351188in}{1.731270in}}%
\pgfpathlineto{\pgfqpoint{1.351188in}{1.731270in}}%
\pgfpathlineto{\pgfqpoint{1.351188in}{1.731270in}}%
\pgfpathlineto{\pgfqpoint{1.351188in}{1.731270in}}%
\pgfpathclose%
\pgfusepath{fill}%
\end{pgfscope}%
\begin{pgfscope}%
\pgfpathrectangle{\pgfqpoint{0.329460in}{0.284240in}}{\pgfqpoint{1.989680in}{1.989680in}}%
\pgfusepath{clip}%
\pgfsetbuttcap%
\pgfsetroundjoin%
\definecolor{currentfill}{rgb}{0.993248,0.906157,0.143936}%
\pgfsetfillcolor{currentfill}%
\pgfsetlinewidth{0.000000pt}%
\definecolor{currentstroke}{rgb}{0.000000,0.000000,0.000000}%
\pgfsetstrokecolor{currentstroke}%
\pgfsetdash{}{0pt}%
\pgfpathmoveto{\pgfqpoint{1.351188in}{1.731270in}}%
\pgfpathlineto{\pgfqpoint{1.348024in}{1.730502in}}%
\pgfpathlineto{\pgfqpoint{1.344860in}{1.729613in}}%
\pgfpathlineto{\pgfqpoint{1.341695in}{1.728604in}}%
\pgfpathlineto{\pgfqpoint{1.338529in}{1.727476in}}%
\pgfpathlineto{\pgfqpoint{1.338344in}{1.727663in}}%
\pgfpathlineto{\pgfqpoint{1.338171in}{1.727853in}}%
\pgfpathlineto{\pgfqpoint{1.338012in}{1.728046in}}%
\pgfpathlineto{\pgfqpoint{1.337865in}{1.728241in}}%
\pgfpathlineto{\pgfqpoint{1.341197in}{1.729178in}}%
\pgfpathlineto{\pgfqpoint{1.344527in}{1.729996in}}%
\pgfpathlineto{\pgfqpoint{1.347858in}{1.730693in}}%
\pgfpathlineto{\pgfqpoint{1.351188in}{1.731270in}}%
\pgfpathlineto{\pgfqpoint{1.351188in}{1.731270in}}%
\pgfpathlineto{\pgfqpoint{1.351188in}{1.731270in}}%
\pgfpathlineto{\pgfqpoint{1.351188in}{1.731270in}}%
\pgfpathlineto{\pgfqpoint{1.351188in}{1.731270in}}%
\pgfpathclose%
\pgfusepath{fill}%
\end{pgfscope}%
\begin{pgfscope}%
\pgfpathrectangle{\pgfqpoint{0.329460in}{0.284240in}}{\pgfqpoint{1.989680in}{1.989680in}}%
\pgfusepath{clip}%
\pgfsetbuttcap%
\pgfsetroundjoin%
\definecolor{currentfill}{rgb}{0.993248,0.906157,0.143936}%
\pgfsetfillcolor{currentfill}%
\pgfsetlinewidth{0.000000pt}%
\definecolor{currentstroke}{rgb}{0.000000,0.000000,0.000000}%
\pgfsetstrokecolor{currentstroke}%
\pgfsetdash{}{0pt}%
\pgfpathmoveto{\pgfqpoint{1.351188in}{1.731270in}}%
\pgfpathlineto{\pgfqpoint{1.354393in}{1.730543in}}%
\pgfpathlineto{\pgfqpoint{1.357598in}{1.729696in}}%
\pgfpathlineto{\pgfqpoint{1.360804in}{1.728729in}}%
\pgfpathlineto{\pgfqpoint{1.364011in}{1.727642in}}%
\pgfpathlineto{\pgfqpoint{1.363824in}{1.727455in}}%
\pgfpathlineto{\pgfqpoint{1.363625in}{1.727270in}}%
\pgfpathlineto{\pgfqpoint{1.363413in}{1.727089in}}%
\pgfpathlineto{\pgfqpoint{1.363189in}{1.726911in}}%
\pgfpathlineto{\pgfqpoint{1.360188in}{1.728181in}}%
\pgfpathlineto{\pgfqpoint{1.357187in}{1.729331in}}%
\pgfpathlineto{\pgfqpoint{1.354187in}{1.730361in}}%
\pgfpathlineto{\pgfqpoint{1.351188in}{1.731270in}}%
\pgfpathlineto{\pgfqpoint{1.351188in}{1.731270in}}%
\pgfpathlineto{\pgfqpoint{1.351188in}{1.731270in}}%
\pgfpathlineto{\pgfqpoint{1.351188in}{1.731270in}}%
\pgfpathlineto{\pgfqpoint{1.351188in}{1.731270in}}%
\pgfpathclose%
\pgfusepath{fill}%
\end{pgfscope}%
\begin{pgfscope}%
\pgfpathrectangle{\pgfqpoint{0.329460in}{0.284240in}}{\pgfqpoint{1.989680in}{1.989680in}}%
\pgfusepath{clip}%
\pgfsetbuttcap%
\pgfsetroundjoin%
\definecolor{currentfill}{rgb}{0.993248,0.906157,0.143936}%
\pgfsetfillcolor{currentfill}%
\pgfsetlinewidth{0.000000pt}%
\definecolor{currentstroke}{rgb}{0.000000,0.000000,0.000000}%
\pgfsetstrokecolor{currentstroke}%
\pgfsetdash{}{0pt}%
\pgfpathmoveto{\pgfqpoint{1.351188in}{1.731270in}}%
\pgfpathlineto{\pgfqpoint{1.348241in}{1.730322in}}%
\pgfpathlineto{\pgfqpoint{1.345293in}{1.729253in}}%
\pgfpathlineto{\pgfqpoint{1.342345in}{1.728064in}}%
\pgfpathlineto{\pgfqpoint{1.339395in}{1.726755in}}%
\pgfpathlineto{\pgfqpoint{1.339161in}{1.726930in}}%
\pgfpathlineto{\pgfqpoint{1.338938in}{1.727109in}}%
\pgfpathlineto{\pgfqpoint{1.338727in}{1.727291in}}%
\pgfpathlineto{\pgfqpoint{1.338529in}{1.727476in}}%
\pgfpathlineto{\pgfqpoint{1.341695in}{1.728604in}}%
\pgfpathlineto{\pgfqpoint{1.344860in}{1.729613in}}%
\pgfpathlineto{\pgfqpoint{1.348024in}{1.730502in}}%
\pgfpathlineto{\pgfqpoint{1.351188in}{1.731270in}}%
\pgfpathlineto{\pgfqpoint{1.351188in}{1.731270in}}%
\pgfpathlineto{\pgfqpoint{1.351188in}{1.731270in}}%
\pgfpathlineto{\pgfqpoint{1.351188in}{1.731270in}}%
\pgfpathlineto{\pgfqpoint{1.351188in}{1.731270in}}%
\pgfpathclose%
\pgfusepath{fill}%
\end{pgfscope}%
\begin{pgfscope}%
\pgfpathrectangle{\pgfqpoint{0.329460in}{0.284240in}}{\pgfqpoint{1.989680in}{1.989680in}}%
\pgfusepath{clip}%
\pgfsetbuttcap%
\pgfsetroundjoin%
\definecolor{currentfill}{rgb}{0.282327,0.094955,0.417331}%
\pgfsetfillcolor{currentfill}%
\pgfsetlinewidth{0.000000pt}%
\definecolor{currentstroke}{rgb}{0.000000,0.000000,0.000000}%
\pgfsetstrokecolor{currentstroke}%
\pgfsetdash{}{0pt}%
\pgfpathmoveto{\pgfqpoint{1.528465in}{0.863940in}}%
\pgfpathlineto{\pgfqpoint{1.529814in}{0.858055in}}%
\pgfpathlineto{\pgfqpoint{1.531164in}{0.852306in}}%
\pgfpathlineto{\pgfqpoint{1.532516in}{0.846697in}}%
\pgfpathlineto{\pgfqpoint{1.533870in}{0.841232in}}%
\pgfpathlineto{\pgfqpoint{1.519940in}{0.838249in}}%
\pgfpathlineto{\pgfqpoint{1.505824in}{0.835504in}}%
\pgfpathlineto{\pgfqpoint{1.491537in}{0.832998in}}%
\pgfpathlineto{\pgfqpoint{1.477094in}{0.830736in}}%
\pgfpathlineto{\pgfqpoint{1.476160in}{0.836287in}}%
\pgfpathlineto{\pgfqpoint{1.475227in}{0.841983in}}%
\pgfpathlineto{\pgfqpoint{1.474295in}{0.847818in}}%
\pgfpathlineto{\pgfqpoint{1.473364in}{0.853790in}}%
\pgfpathlineto{\pgfqpoint{1.487380in}{0.855978in}}%
\pgfpathlineto{\pgfqpoint{1.501245in}{0.858401in}}%
\pgfpathlineto{\pgfqpoint{1.514945in}{0.861056in}}%
\pgfpathlineto{\pgfqpoint{1.528465in}{0.863940in}}%
\pgfpathclose%
\pgfusepath{fill}%
\end{pgfscope}%
\begin{pgfscope}%
\pgfpathrectangle{\pgfqpoint{0.329460in}{0.284240in}}{\pgfqpoint{1.989680in}{1.989680in}}%
\pgfusepath{clip}%
\pgfsetbuttcap%
\pgfsetroundjoin%
\definecolor{currentfill}{rgb}{0.993248,0.906157,0.143936}%
\pgfsetfillcolor{currentfill}%
\pgfsetlinewidth{0.000000pt}%
\definecolor{currentstroke}{rgb}{0.000000,0.000000,0.000000}%
\pgfsetstrokecolor{currentstroke}%
\pgfsetdash{}{0pt}%
\pgfpathmoveto{\pgfqpoint{1.365032in}{1.729220in}}%
\pgfpathlineto{\pgfqpoint{1.368494in}{1.728408in}}%
\pgfpathlineto{\pgfqpoint{1.371955in}{1.727476in}}%
\pgfpathlineto{\pgfqpoint{1.375416in}{1.726425in}}%
\pgfpathlineto{\pgfqpoint{1.378878in}{1.725254in}}%
\pgfpathlineto{\pgfqpoint{1.378718in}{1.724848in}}%
\pgfpathlineto{\pgfqpoint{1.378531in}{1.724444in}}%
\pgfpathlineto{\pgfqpoint{1.378316in}{1.724043in}}%
\pgfpathlineto{\pgfqpoint{1.378075in}{1.723645in}}%
\pgfpathlineto{\pgfqpoint{1.374713in}{1.725017in}}%
\pgfpathlineto{\pgfqpoint{1.371352in}{1.726270in}}%
\pgfpathlineto{\pgfqpoint{1.367990in}{1.727403in}}%
\pgfpathlineto{\pgfqpoint{1.364629in}{1.728416in}}%
\pgfpathlineto{\pgfqpoint{1.364750in}{1.728615in}}%
\pgfpathlineto{\pgfqpoint{1.364858in}{1.728815in}}%
\pgfpathlineto{\pgfqpoint{1.364952in}{1.729017in}}%
\pgfpathlineto{\pgfqpoint{1.365032in}{1.729220in}}%
\pgfpathclose%
\pgfusepath{fill}%
\end{pgfscope}%
\begin{pgfscope}%
\pgfpathrectangle{\pgfqpoint{0.329460in}{0.284240in}}{\pgfqpoint{1.989680in}{1.989680in}}%
\pgfusepath{clip}%
\pgfsetbuttcap%
\pgfsetroundjoin%
\definecolor{currentfill}{rgb}{0.935904,0.898570,0.108131}%
\pgfsetfillcolor{currentfill}%
\pgfsetlinewidth{0.000000pt}%
\definecolor{currentstroke}{rgb}{0.000000,0.000000,0.000000}%
\pgfsetstrokecolor{currentstroke}%
\pgfsetdash{}{0pt}%
\pgfpathmoveto{\pgfqpoint{1.297881in}{1.707705in}}%
\pgfpathlineto{\pgfqpoint{1.294550in}{1.705229in}}%
\pgfpathlineto{\pgfqpoint{1.291220in}{1.702638in}}%
\pgfpathlineto{\pgfqpoint{1.287890in}{1.699931in}}%
\pgfpathlineto{\pgfqpoint{1.284561in}{1.697111in}}%
\pgfpathlineto{\pgfqpoint{1.283904in}{1.698101in}}%
\pgfpathlineto{\pgfqpoint{1.283315in}{1.699101in}}%
\pgfpathlineto{\pgfqpoint{1.282793in}{1.700108in}}%
\pgfpathlineto{\pgfqpoint{1.282340in}{1.701123in}}%
\pgfpathlineto{\pgfqpoint{1.285778in}{1.703741in}}%
\pgfpathlineto{\pgfqpoint{1.289217in}{1.706245in}}%
\pgfpathlineto{\pgfqpoint{1.292657in}{1.708635in}}%
\pgfpathlineto{\pgfqpoint{1.296097in}{1.710909in}}%
\pgfpathlineto{\pgfqpoint{1.296462in}{1.710099in}}%
\pgfpathlineto{\pgfqpoint{1.296881in}{1.709294in}}%
\pgfpathlineto{\pgfqpoint{1.297354in}{1.708496in}}%
\pgfpathlineto{\pgfqpoint{1.297881in}{1.707705in}}%
\pgfpathclose%
\pgfusepath{fill}%
\end{pgfscope}%
\begin{pgfscope}%
\pgfpathrectangle{\pgfqpoint{0.329460in}{0.284240in}}{\pgfqpoint{1.989680in}{1.989680in}}%
\pgfusepath{clip}%
\pgfsetbuttcap%
\pgfsetroundjoin%
\definecolor{currentfill}{rgb}{0.955300,0.901065,0.118128}%
\pgfsetfillcolor{currentfill}%
\pgfsetlinewidth{0.000000pt}%
\definecolor{currentstroke}{rgb}{0.000000,0.000000,0.000000}%
\pgfsetstrokecolor{currentstroke}%
\pgfsetdash{}{0pt}%
\pgfpathmoveto{\pgfqpoint{1.392720in}{1.719385in}}%
\pgfpathlineto{\pgfqpoint{1.396180in}{1.717623in}}%
\pgfpathlineto{\pgfqpoint{1.399639in}{1.715743in}}%
\pgfpathlineto{\pgfqpoint{1.403098in}{1.713746in}}%
\pgfpathlineto{\pgfqpoint{1.406556in}{1.711633in}}%
\pgfpathlineto{\pgfqpoint{1.406240in}{1.710819in}}%
\pgfpathlineto{\pgfqpoint{1.405870in}{1.710009in}}%
\pgfpathlineto{\pgfqpoint{1.405445in}{1.709205in}}%
\pgfpathlineto{\pgfqpoint{1.404965in}{1.708407in}}%
\pgfpathlineto{\pgfqpoint{1.401605in}{1.710723in}}%
\pgfpathlineto{\pgfqpoint{1.398244in}{1.712922in}}%
\pgfpathlineto{\pgfqpoint{1.394883in}{1.715004in}}%
\pgfpathlineto{\pgfqpoint{1.391522in}{1.716968in}}%
\pgfpathlineto{\pgfqpoint{1.391883in}{1.717566in}}%
\pgfpathlineto{\pgfqpoint{1.392203in}{1.718168in}}%
\pgfpathlineto{\pgfqpoint{1.392482in}{1.718775in}}%
\pgfpathlineto{\pgfqpoint{1.392720in}{1.719385in}}%
\pgfpathclose%
\pgfusepath{fill}%
\end{pgfscope}%
\begin{pgfscope}%
\pgfpathrectangle{\pgfqpoint{0.329460in}{0.284240in}}{\pgfqpoint{1.989680in}{1.989680in}}%
\pgfusepath{clip}%
\pgfsetbuttcap%
\pgfsetroundjoin%
\definecolor{currentfill}{rgb}{0.993248,0.906157,0.143936}%
\pgfsetfillcolor{currentfill}%
\pgfsetlinewidth{0.000000pt}%
\definecolor{currentstroke}{rgb}{0.000000,0.000000,0.000000}%
\pgfsetstrokecolor{currentstroke}%
\pgfsetdash{}{0pt}%
\pgfpathmoveto{\pgfqpoint{1.351188in}{1.731270in}}%
\pgfpathlineto{\pgfqpoint{1.354187in}{1.730361in}}%
\pgfpathlineto{\pgfqpoint{1.357187in}{1.729331in}}%
\pgfpathlineto{\pgfqpoint{1.360188in}{1.728181in}}%
\pgfpathlineto{\pgfqpoint{1.363189in}{1.726911in}}%
\pgfpathlineto{\pgfqpoint{1.362953in}{1.726736in}}%
\pgfpathlineto{\pgfqpoint{1.362705in}{1.726564in}}%
\pgfpathlineto{\pgfqpoint{1.362446in}{1.726397in}}%
\pgfpathlineto{\pgfqpoint{1.362175in}{1.726233in}}%
\pgfpathlineto{\pgfqpoint{1.359427in}{1.727673in}}%
\pgfpathlineto{\pgfqpoint{1.356680in}{1.728992in}}%
\pgfpathlineto{\pgfqpoint{1.353933in}{1.730191in}}%
\pgfpathlineto{\pgfqpoint{1.351188in}{1.731270in}}%
\pgfpathlineto{\pgfqpoint{1.351188in}{1.731270in}}%
\pgfpathlineto{\pgfqpoint{1.351188in}{1.731270in}}%
\pgfpathlineto{\pgfqpoint{1.351188in}{1.731270in}}%
\pgfpathlineto{\pgfqpoint{1.351188in}{1.731270in}}%
\pgfpathclose%
\pgfusepath{fill}%
\end{pgfscope}%
\begin{pgfscope}%
\pgfpathrectangle{\pgfqpoint{0.329460in}{0.284240in}}{\pgfqpoint{1.989680in}{1.989680in}}%
\pgfusepath{clip}%
\pgfsetbuttcap%
\pgfsetroundjoin%
\definecolor{currentfill}{rgb}{0.263663,0.237631,0.518762}%
\pgfsetfillcolor{currentfill}%
\pgfsetlinewidth{0.000000pt}%
\definecolor{currentstroke}{rgb}{0.000000,0.000000,0.000000}%
\pgfsetstrokecolor{currentstroke}%
\pgfsetdash{}{0pt}%
\pgfpathmoveto{\pgfqpoint{1.507029in}{0.973764in}}%
\pgfpathlineto{\pgfqpoint{1.508363in}{0.966174in}}%
\pgfpathlineto{\pgfqpoint{1.509698in}{0.958664in}}%
\pgfpathlineto{\pgfqpoint{1.511033in}{0.951240in}}%
\pgfpathlineto{\pgfqpoint{1.512368in}{0.943903in}}%
\pgfpathlineto{\pgfqpoint{1.500070in}{0.941312in}}%
\pgfpathlineto{\pgfqpoint{1.487610in}{0.938927in}}%
\pgfpathlineto{\pgfqpoint{1.475000in}{0.936751in}}%
\pgfpathlineto{\pgfqpoint{1.462255in}{0.934786in}}%
\pgfpathlineto{\pgfqpoint{1.461333in}{0.942209in}}%
\pgfpathlineto{\pgfqpoint{1.460412in}{0.949719in}}%
\pgfpathlineto{\pgfqpoint{1.459491in}{0.957314in}}%
\pgfpathlineto{\pgfqpoint{1.458571in}{0.964990in}}%
\pgfpathlineto{\pgfqpoint{1.470894in}{0.966881in}}%
\pgfpathlineto{\pgfqpoint{1.483087in}{0.968975in}}%
\pgfpathlineto{\pgfqpoint{1.495137in}{0.971270in}}%
\pgfpathlineto{\pgfqpoint{1.507029in}{0.973764in}}%
\pgfpathclose%
\pgfusepath{fill}%
\end{pgfscope}%
\begin{pgfscope}%
\pgfpathrectangle{\pgfqpoint{0.329460in}{0.284240in}}{\pgfqpoint{1.989680in}{1.989680in}}%
\pgfusepath{clip}%
\pgfsetbuttcap%
\pgfsetroundjoin%
\definecolor{currentfill}{rgb}{0.993248,0.906157,0.143936}%
\pgfsetfillcolor{currentfill}%
\pgfsetlinewidth{0.000000pt}%
\definecolor{currentstroke}{rgb}{0.000000,0.000000,0.000000}%
\pgfsetstrokecolor{currentstroke}%
\pgfsetdash{}{0pt}%
\pgfpathmoveto{\pgfqpoint{1.337865in}{1.728241in}}%
\pgfpathlineto{\pgfqpoint{1.334534in}{1.727184in}}%
\pgfpathlineto{\pgfqpoint{1.331202in}{1.726007in}}%
\pgfpathlineto{\pgfqpoint{1.327870in}{1.724710in}}%
\pgfpathlineto{\pgfqpoint{1.324538in}{1.723295in}}%
\pgfpathlineto{\pgfqpoint{1.324272in}{1.723689in}}%
\pgfpathlineto{\pgfqpoint{1.324034in}{1.724087in}}%
\pgfpathlineto{\pgfqpoint{1.323822in}{1.724488in}}%
\pgfpathlineto{\pgfqpoint{1.323638in}{1.724893in}}%
\pgfpathlineto{\pgfqpoint{1.327082in}{1.726109in}}%
\pgfpathlineto{\pgfqpoint{1.330526in}{1.727205in}}%
\pgfpathlineto{\pgfqpoint{1.333970in}{1.728182in}}%
\pgfpathlineto{\pgfqpoint{1.337414in}{1.729039in}}%
\pgfpathlineto{\pgfqpoint{1.337506in}{1.728838in}}%
\pgfpathlineto{\pgfqpoint{1.337612in}{1.728637in}}%
\pgfpathlineto{\pgfqpoint{1.337732in}{1.728438in}}%
\pgfpathlineto{\pgfqpoint{1.337865in}{1.728241in}}%
\pgfpathclose%
\pgfusepath{fill}%
\end{pgfscope}%
\begin{pgfscope}%
\pgfpathrectangle{\pgfqpoint{0.329460in}{0.284240in}}{\pgfqpoint{1.989680in}{1.989680in}}%
\pgfusepath{clip}%
\pgfsetbuttcap%
\pgfsetroundjoin%
\definecolor{currentfill}{rgb}{0.260571,0.246922,0.522828}%
\pgfsetfillcolor{currentfill}%
\pgfsetlinewidth{0.000000pt}%
\definecolor{currentstroke}{rgb}{0.000000,0.000000,0.000000}%
\pgfsetstrokecolor{currentstroke}%
\pgfsetdash{}{0pt}%
\pgfpathmoveto{\pgfqpoint{0.856783in}{0.927445in}}%
\pgfpathlineto{\pgfqpoint{0.853768in}{0.934920in}}%
\pgfpathlineto{\pgfqpoint{0.850740in}{0.942779in}}%
\pgfpathlineto{\pgfqpoint{0.847697in}{0.951029in}}%
\pgfpathlineto{\pgfqpoint{0.844639in}{0.959677in}}%
\pgfpathlineto{\pgfqpoint{0.832079in}{0.968219in}}%
\pgfpathlineto{\pgfqpoint{0.820086in}{0.976955in}}%
\pgfpathlineto{\pgfqpoint{0.808670in}{0.985873in}}%
\pgfpathlineto{\pgfqpoint{0.797843in}{0.994964in}}%
\pgfpathlineto{\pgfqpoint{0.801165in}{0.986150in}}%
\pgfpathlineto{\pgfqpoint{0.804472in}{0.977733in}}%
\pgfpathlineto{\pgfqpoint{0.807763in}{0.969705in}}%
\pgfpathlineto{\pgfqpoint{0.811038in}{0.962060in}}%
\pgfpathlineto{\pgfqpoint{0.821625in}{0.953142in}}%
\pgfpathlineto{\pgfqpoint{0.832785in}{0.944393in}}%
\pgfpathlineto{\pgfqpoint{0.844508in}{0.935824in}}%
\pgfpathlineto{\pgfqpoint{0.856783in}{0.927445in}}%
\pgfpathclose%
\pgfusepath{fill}%
\end{pgfscope}%
\begin{pgfscope}%
\pgfpathrectangle{\pgfqpoint{0.329460in}{0.284240in}}{\pgfqpoint{1.989680in}{1.989680in}}%
\pgfusepath{clip}%
\pgfsetbuttcap%
\pgfsetroundjoin%
\definecolor{currentfill}{rgb}{0.993248,0.906157,0.143936}%
\pgfsetfillcolor{currentfill}%
\pgfsetlinewidth{0.000000pt}%
\definecolor{currentstroke}{rgb}{0.000000,0.000000,0.000000}%
\pgfsetstrokecolor{currentstroke}%
\pgfsetdash{}{0pt}%
\pgfpathmoveto{\pgfqpoint{1.351188in}{1.731270in}}%
\pgfpathlineto{\pgfqpoint{1.348504in}{1.730156in}}%
\pgfpathlineto{\pgfqpoint{1.345820in}{1.728921in}}%
\pgfpathlineto{\pgfqpoint{1.343135in}{1.727566in}}%
\pgfpathlineto{\pgfqpoint{1.340449in}{1.726091in}}%
\pgfpathlineto{\pgfqpoint{1.340169in}{1.726251in}}%
\pgfpathlineto{\pgfqpoint{1.339900in}{1.726415in}}%
\pgfpathlineto{\pgfqpoint{1.339642in}{1.726583in}}%
\pgfpathlineto{\pgfqpoint{1.339395in}{1.726755in}}%
\pgfpathlineto{\pgfqpoint{1.342345in}{1.728064in}}%
\pgfpathlineto{\pgfqpoint{1.345293in}{1.729253in}}%
\pgfpathlineto{\pgfqpoint{1.348241in}{1.730322in}}%
\pgfpathlineto{\pgfqpoint{1.351188in}{1.731270in}}%
\pgfpathlineto{\pgfqpoint{1.351188in}{1.731270in}}%
\pgfpathlineto{\pgfqpoint{1.351188in}{1.731270in}}%
\pgfpathlineto{\pgfqpoint{1.351188in}{1.731270in}}%
\pgfpathlineto{\pgfqpoint{1.351188in}{1.731270in}}%
\pgfpathclose%
\pgfusepath{fill}%
\end{pgfscope}%
\begin{pgfscope}%
\pgfpathrectangle{\pgfqpoint{0.329460in}{0.284240in}}{\pgfqpoint{1.989680in}{1.989680in}}%
\pgfusepath{clip}%
\pgfsetbuttcap%
\pgfsetroundjoin%
\definecolor{currentfill}{rgb}{0.974417,0.903590,0.130215}%
\pgfsetfillcolor{currentfill}%
\pgfsetlinewidth{0.000000pt}%
\definecolor{currentstroke}{rgb}{0.000000,0.000000,0.000000}%
\pgfsetstrokecolor{currentstroke}%
\pgfsetdash{}{0pt}%
\pgfpathmoveto{\pgfqpoint{1.378878in}{1.725254in}}%
\pgfpathlineto{\pgfqpoint{1.382339in}{1.723965in}}%
\pgfpathlineto{\pgfqpoint{1.385800in}{1.722556in}}%
\pgfpathlineto{\pgfqpoint{1.389260in}{1.721030in}}%
\pgfpathlineto{\pgfqpoint{1.392720in}{1.719385in}}%
\pgfpathlineto{\pgfqpoint{1.392482in}{1.718775in}}%
\pgfpathlineto{\pgfqpoint{1.392203in}{1.718168in}}%
\pgfpathlineto{\pgfqpoint{1.391883in}{1.717566in}}%
\pgfpathlineto{\pgfqpoint{1.391522in}{1.716968in}}%
\pgfpathlineto{\pgfqpoint{1.388160in}{1.718815in}}%
\pgfpathlineto{\pgfqpoint{1.384798in}{1.720544in}}%
\pgfpathlineto{\pgfqpoint{1.381437in}{1.722154in}}%
\pgfpathlineto{\pgfqpoint{1.378075in}{1.723645in}}%
\pgfpathlineto{\pgfqpoint{1.378316in}{1.724043in}}%
\pgfpathlineto{\pgfqpoint{1.378531in}{1.724444in}}%
\pgfpathlineto{\pgfqpoint{1.378718in}{1.724848in}}%
\pgfpathlineto{\pgfqpoint{1.378878in}{1.725254in}}%
\pgfpathclose%
\pgfusepath{fill}%
\end{pgfscope}%
\begin{pgfscope}%
\pgfpathrectangle{\pgfqpoint{0.329460in}{0.284240in}}{\pgfqpoint{1.989680in}{1.989680in}}%
\pgfusepath{clip}%
\pgfsetbuttcap%
\pgfsetroundjoin%
\definecolor{currentfill}{rgb}{0.993248,0.906157,0.143936}%
\pgfsetfillcolor{currentfill}%
\pgfsetlinewidth{0.000000pt}%
\definecolor{currentstroke}{rgb}{0.000000,0.000000,0.000000}%
\pgfsetstrokecolor{currentstroke}%
\pgfsetdash{}{0pt}%
\pgfpathmoveto{\pgfqpoint{1.351188in}{1.731270in}}%
\pgfpathlineto{\pgfqpoint{1.353933in}{1.730191in}}%
\pgfpathlineto{\pgfqpoint{1.356680in}{1.728992in}}%
\pgfpathlineto{\pgfqpoint{1.359427in}{1.727673in}}%
\pgfpathlineto{\pgfqpoint{1.362175in}{1.726233in}}%
\pgfpathlineto{\pgfqpoint{1.361894in}{1.726073in}}%
\pgfpathlineto{\pgfqpoint{1.361602in}{1.725918in}}%
\pgfpathlineto{\pgfqpoint{1.361299in}{1.725767in}}%
\pgfpathlineto{\pgfqpoint{1.360986in}{1.725620in}}%
\pgfpathlineto{\pgfqpoint{1.358535in}{1.727213in}}%
\pgfpathlineto{\pgfqpoint{1.356085in}{1.728686in}}%
\pgfpathlineto{\pgfqpoint{1.353636in}{1.730038in}}%
\pgfpathlineto{\pgfqpoint{1.351188in}{1.731270in}}%
\pgfpathlineto{\pgfqpoint{1.351188in}{1.731270in}}%
\pgfpathlineto{\pgfqpoint{1.351188in}{1.731270in}}%
\pgfpathlineto{\pgfqpoint{1.351188in}{1.731270in}}%
\pgfpathlineto{\pgfqpoint{1.351188in}{1.731270in}}%
\pgfpathclose%
\pgfusepath{fill}%
\end{pgfscope}%
\begin{pgfscope}%
\pgfpathrectangle{\pgfqpoint{0.329460in}{0.284240in}}{\pgfqpoint{1.989680in}{1.989680in}}%
\pgfusepath{clip}%
\pgfsetbuttcap%
\pgfsetroundjoin%
\definecolor{currentfill}{rgb}{0.120081,0.622161,0.534946}%
\pgfsetfillcolor{currentfill}%
\pgfsetlinewidth{0.000000pt}%
\definecolor{currentstroke}{rgb}{0.000000,0.000000,0.000000}%
\pgfsetstrokecolor{currentstroke}%
\pgfsetdash{}{0pt}%
\pgfpathmoveto{\pgfqpoint{1.196523in}{1.324250in}}%
\pgfpathlineto{\pgfqpoint{1.194509in}{1.316005in}}%
\pgfpathlineto{\pgfqpoint{1.192496in}{1.307730in}}%
\pgfpathlineto{\pgfqpoint{1.190484in}{1.299429in}}%
\pgfpathlineto{\pgfqpoint{1.188472in}{1.291104in}}%
\pgfpathlineto{\pgfqpoint{1.181391in}{1.293735in}}%
\pgfpathlineto{\pgfqpoint{1.174488in}{1.296475in}}%
\pgfpathlineto{\pgfqpoint{1.167771in}{1.299322in}}%
\pgfpathlineto{\pgfqpoint{1.161245in}{1.302271in}}%
\pgfpathlineto{\pgfqpoint{1.163589in}{1.310444in}}%
\pgfpathlineto{\pgfqpoint{1.165934in}{1.318592in}}%
\pgfpathlineto{\pgfqpoint{1.168280in}{1.326715in}}%
\pgfpathlineto{\pgfqpoint{1.170627in}{1.334809in}}%
\pgfpathlineto{\pgfqpoint{1.176834in}{1.332020in}}%
\pgfpathlineto{\pgfqpoint{1.183224in}{1.329328in}}%
\pgfpathlineto{\pgfqpoint{1.189789in}{1.326737in}}%
\pgfpathlineto{\pgfqpoint{1.196523in}{1.324250in}}%
\pgfpathclose%
\pgfusepath{fill}%
\end{pgfscope}%
\begin{pgfscope}%
\pgfpathrectangle{\pgfqpoint{0.329460in}{0.284240in}}{\pgfqpoint{1.989680in}{1.989680in}}%
\pgfusepath{clip}%
\pgfsetbuttcap%
\pgfsetroundjoin%
\definecolor{currentfill}{rgb}{0.147607,0.511733,0.557049}%
\pgfsetfillcolor{currentfill}%
\pgfsetlinewidth{0.000000pt}%
\definecolor{currentstroke}{rgb}{0.000000,0.000000,0.000000}%
\pgfsetstrokecolor{currentstroke}%
\pgfsetdash{}{0pt}%
\pgfpathmoveto{\pgfqpoint{1.205304in}{1.213513in}}%
\pgfpathlineto{\pgfqpoint{1.203665in}{1.204949in}}%
\pgfpathlineto{\pgfqpoint{1.202027in}{1.196384in}}%
\pgfpathlineto{\pgfqpoint{1.200389in}{1.187823in}}%
\pgfpathlineto{\pgfqpoint{1.198751in}{1.179267in}}%
\pgfpathlineto{\pgfqpoint{1.189894in}{1.181794in}}%
\pgfpathlineto{\pgfqpoint{1.181209in}{1.184461in}}%
\pgfpathlineto{\pgfqpoint{1.172705in}{1.187265in}}%
\pgfpathlineto{\pgfqpoint{1.164390in}{1.190203in}}%
\pgfpathlineto{\pgfqpoint{1.166393in}{1.198627in}}%
\pgfpathlineto{\pgfqpoint{1.168397in}{1.207056in}}%
\pgfpathlineto{\pgfqpoint{1.170402in}{1.215489in}}%
\pgfpathlineto{\pgfqpoint{1.172407in}{1.223923in}}%
\pgfpathlineto{\pgfqpoint{1.180368in}{1.221126in}}%
\pgfpathlineto{\pgfqpoint{1.188511in}{1.218457in}}%
\pgfpathlineto{\pgfqpoint{1.196826in}{1.215918in}}%
\pgfpathlineto{\pgfqpoint{1.205304in}{1.213513in}}%
\pgfpathclose%
\pgfusepath{fill}%
\end{pgfscope}%
\begin{pgfscope}%
\pgfpathrectangle{\pgfqpoint{0.329460in}{0.284240in}}{\pgfqpoint{1.989680in}{1.989680in}}%
\pgfusepath{clip}%
\pgfsetbuttcap%
\pgfsetroundjoin%
\definecolor{currentfill}{rgb}{0.993248,0.906157,0.143936}%
\pgfsetfillcolor{currentfill}%
\pgfsetlinewidth{0.000000pt}%
\definecolor{currentstroke}{rgb}{0.000000,0.000000,0.000000}%
\pgfsetstrokecolor{currentstroke}%
\pgfsetdash{}{0pt}%
\pgfpathmoveto{\pgfqpoint{1.351188in}{1.731270in}}%
\pgfpathlineto{\pgfqpoint{1.348811in}{1.730007in}}%
\pgfpathlineto{\pgfqpoint{1.346433in}{1.728623in}}%
\pgfpathlineto{\pgfqpoint{1.344054in}{1.727119in}}%
\pgfpathlineto{\pgfqpoint{1.341675in}{1.725494in}}%
\pgfpathlineto{\pgfqpoint{1.341354in}{1.725636in}}%
\pgfpathlineto{\pgfqpoint{1.341042in}{1.725783in}}%
\pgfpathlineto{\pgfqpoint{1.340740in}{1.725935in}}%
\pgfpathlineto{\pgfqpoint{1.340449in}{1.726091in}}%
\pgfpathlineto{\pgfqpoint{1.343135in}{1.727566in}}%
\pgfpathlineto{\pgfqpoint{1.345820in}{1.728921in}}%
\pgfpathlineto{\pgfqpoint{1.348504in}{1.730156in}}%
\pgfpathlineto{\pgfqpoint{1.351188in}{1.731270in}}%
\pgfpathlineto{\pgfqpoint{1.351188in}{1.731270in}}%
\pgfpathlineto{\pgfqpoint{1.351188in}{1.731270in}}%
\pgfpathlineto{\pgfqpoint{1.351188in}{1.731270in}}%
\pgfpathlineto{\pgfqpoint{1.351188in}{1.731270in}}%
\pgfpathclose%
\pgfusepath{fill}%
\end{pgfscope}%
\begin{pgfscope}%
\pgfpathrectangle{\pgfqpoint{0.329460in}{0.284240in}}{\pgfqpoint{1.989680in}{1.989680in}}%
\pgfusepath{clip}%
\pgfsetbuttcap%
\pgfsetroundjoin%
\definecolor{currentfill}{rgb}{0.955300,0.901065,0.118128}%
\pgfsetfillcolor{currentfill}%
\pgfsetlinewidth{0.000000pt}%
\definecolor{currentstroke}{rgb}{0.000000,0.000000,0.000000}%
\pgfsetstrokecolor{currentstroke}%
\pgfsetdash{}{0pt}%
\pgfpathmoveto{\pgfqpoint{1.311208in}{1.716442in}}%
\pgfpathlineto{\pgfqpoint{1.307876in}{1.714434in}}%
\pgfpathlineto{\pgfqpoint{1.304544in}{1.712307in}}%
\pgfpathlineto{\pgfqpoint{1.301213in}{1.710064in}}%
\pgfpathlineto{\pgfqpoint{1.297881in}{1.707705in}}%
\pgfpathlineto{\pgfqpoint{1.297354in}{1.708496in}}%
\pgfpathlineto{\pgfqpoint{1.296881in}{1.709294in}}%
\pgfpathlineto{\pgfqpoint{1.296462in}{1.710099in}}%
\pgfpathlineto{\pgfqpoint{1.296097in}{1.710909in}}%
\pgfpathlineto{\pgfqpoint{1.299538in}{1.713067in}}%
\pgfpathlineto{\pgfqpoint{1.302980in}{1.715109in}}%
\pgfpathlineto{\pgfqpoint{1.306422in}{1.717034in}}%
\pgfpathlineto{\pgfqpoint{1.309864in}{1.718842in}}%
\pgfpathlineto{\pgfqpoint{1.310139in}{1.718235in}}%
\pgfpathlineto{\pgfqpoint{1.310455in}{1.717632in}}%
\pgfpathlineto{\pgfqpoint{1.310811in}{1.717035in}}%
\pgfpathlineto{\pgfqpoint{1.311208in}{1.716442in}}%
\pgfpathclose%
\pgfusepath{fill}%
\end{pgfscope}%
\begin{pgfscope}%
\pgfpathrectangle{\pgfqpoint{0.329460in}{0.284240in}}{\pgfqpoint{1.989680in}{1.989680in}}%
\pgfusepath{clip}%
\pgfsetbuttcap%
\pgfsetroundjoin%
\definecolor{currentfill}{rgb}{0.814576,0.883393,0.110347}%
\pgfsetfillcolor{currentfill}%
\pgfsetlinewidth{0.000000pt}%
\definecolor{currentstroke}{rgb}{0.000000,0.000000,0.000000}%
\pgfsetstrokecolor{currentstroke}%
\pgfsetdash{}{0pt}%
\pgfpathmoveto{\pgfqpoint{1.262530in}{1.665099in}}%
\pgfpathlineto{\pgfqpoint{1.259368in}{1.661086in}}%
\pgfpathlineto{\pgfqpoint{1.256208in}{1.656966in}}%
\pgfpathlineto{\pgfqpoint{1.253048in}{1.652740in}}%
\pgfpathlineto{\pgfqpoint{1.249889in}{1.648409in}}%
\pgfpathlineto{\pgfqpoint{1.248428in}{1.649931in}}%
\pgfpathlineto{\pgfqpoint{1.247072in}{1.651473in}}%
\pgfpathlineto{\pgfqpoint{1.245820in}{1.653035in}}%
\pgfpathlineto{\pgfqpoint{1.244674in}{1.654614in}}%
\pgfpathlineto{\pgfqpoint{1.247992in}{1.658747in}}%
\pgfpathlineto{\pgfqpoint{1.251311in}{1.662775in}}%
\pgfpathlineto{\pgfqpoint{1.254632in}{1.666698in}}%
\pgfpathlineto{\pgfqpoint{1.257953in}{1.670515in}}%
\pgfpathlineto{\pgfqpoint{1.258960in}{1.669137in}}%
\pgfpathlineto{\pgfqpoint{1.260059in}{1.667774in}}%
\pgfpathlineto{\pgfqpoint{1.261249in}{1.666427in}}%
\pgfpathlineto{\pgfqpoint{1.262530in}{1.665099in}}%
\pgfpathclose%
\pgfusepath{fill}%
\end{pgfscope}%
\begin{pgfscope}%
\pgfpathrectangle{\pgfqpoint{0.329460in}{0.284240in}}{\pgfqpoint{1.989680in}{1.989680in}}%
\pgfusepath{clip}%
\pgfsetbuttcap%
\pgfsetroundjoin%
\definecolor{currentfill}{rgb}{0.280255,0.165693,0.476498}%
\pgfsetfillcolor{currentfill}%
\pgfsetlinewidth{0.000000pt}%
\definecolor{currentstroke}{rgb}{0.000000,0.000000,0.000000}%
\pgfsetstrokecolor{currentstroke}%
\pgfsetdash{}{0pt}%
\pgfpathmoveto{\pgfqpoint{1.248239in}{0.904414in}}%
\pgfpathlineto{\pgfqpoint{1.247409in}{0.897466in}}%
\pgfpathlineto{\pgfqpoint{1.246579in}{0.890626in}}%
\pgfpathlineto{\pgfqpoint{1.245748in}{0.883896in}}%
\pgfpathlineto{\pgfqpoint{1.244917in}{0.877282in}}%
\pgfpathlineto{\pgfqpoint{1.231209in}{0.879191in}}%
\pgfpathlineto{\pgfqpoint{1.217634in}{0.881330in}}%
\pgfpathlineto{\pgfqpoint{1.204205in}{0.883696in}}%
\pgfpathlineto{\pgfqpoint{1.190938in}{0.886285in}}%
\pgfpathlineto{\pgfqpoint{1.192190in}{0.892820in}}%
\pgfpathlineto{\pgfqpoint{1.193441in}{0.899470in}}%
\pgfpathlineto{\pgfqpoint{1.194691in}{0.906232in}}%
\pgfpathlineto{\pgfqpoint{1.195941in}{0.913101in}}%
\pgfpathlineto{\pgfqpoint{1.208795in}{0.910602in}}%
\pgfpathlineto{\pgfqpoint{1.221806in}{0.908320in}}%
\pgfpathlineto{\pgfqpoint{1.234959in}{0.906256in}}%
\pgfpathlineto{\pgfqpoint{1.248239in}{0.904414in}}%
\pgfpathclose%
\pgfusepath{fill}%
\end{pgfscope}%
\begin{pgfscope}%
\pgfpathrectangle{\pgfqpoint{0.329460in}{0.284240in}}{\pgfqpoint{1.989680in}{1.989680in}}%
\pgfusepath{clip}%
\pgfsetbuttcap%
\pgfsetroundjoin%
\definecolor{currentfill}{rgb}{0.974417,0.903590,0.130215}%
\pgfsetfillcolor{currentfill}%
\pgfsetlinewidth{0.000000pt}%
\definecolor{currentstroke}{rgb}{0.000000,0.000000,0.000000}%
\pgfsetstrokecolor{currentstroke}%
\pgfsetdash{}{0pt}%
\pgfpathmoveto{\pgfqpoint{1.324538in}{1.723295in}}%
\pgfpathlineto{\pgfqpoint{1.321205in}{1.721760in}}%
\pgfpathlineto{\pgfqpoint{1.317873in}{1.720106in}}%
\pgfpathlineto{\pgfqpoint{1.314540in}{1.718333in}}%
\pgfpathlineto{\pgfqpoint{1.311208in}{1.716442in}}%
\pgfpathlineto{\pgfqpoint{1.310811in}{1.717035in}}%
\pgfpathlineto{\pgfqpoint{1.310455in}{1.717632in}}%
\pgfpathlineto{\pgfqpoint{1.310139in}{1.718235in}}%
\pgfpathlineto{\pgfqpoint{1.309864in}{1.718842in}}%
\pgfpathlineto{\pgfqpoint{1.313307in}{1.720532in}}%
\pgfpathlineto{\pgfqpoint{1.316751in}{1.722104in}}%
\pgfpathlineto{\pgfqpoint{1.320194in}{1.723558in}}%
\pgfpathlineto{\pgfqpoint{1.323638in}{1.724893in}}%
\pgfpathlineto{\pgfqpoint{1.323822in}{1.724488in}}%
\pgfpathlineto{\pgfqpoint{1.324034in}{1.724087in}}%
\pgfpathlineto{\pgfqpoint{1.324272in}{1.723689in}}%
\pgfpathlineto{\pgfqpoint{1.324538in}{1.723295in}}%
\pgfpathclose%
\pgfusepath{fill}%
\end{pgfscope}%
\begin{pgfscope}%
\pgfpathrectangle{\pgfqpoint{0.329460in}{0.284240in}}{\pgfqpoint{1.989680in}{1.989680in}}%
\pgfusepath{clip}%
\pgfsetbuttcap%
\pgfsetroundjoin%
\definecolor{currentfill}{rgb}{0.220124,0.725509,0.466226}%
\pgfsetfillcolor{currentfill}%
\pgfsetlinewidth{0.000000pt}%
\definecolor{currentstroke}{rgb}{0.000000,0.000000,0.000000}%
\pgfsetstrokecolor{currentstroke}%
\pgfsetdash{}{0pt}%
\pgfpathmoveto{\pgfqpoint{1.198876in}{1.428913in}}%
\pgfpathlineto{\pgfqpoint{1.196516in}{1.421331in}}%
\pgfpathlineto{\pgfqpoint{1.194157in}{1.413695in}}%
\pgfpathlineto{\pgfqpoint{1.191799in}{1.406007in}}%
\pgfpathlineto{\pgfqpoint{1.189442in}{1.398269in}}%
\pgfpathlineto{\pgfqpoint{1.184042in}{1.400826in}}%
\pgfpathlineto{\pgfqpoint{1.178816in}{1.403465in}}%
\pgfpathlineto{\pgfqpoint{1.173768in}{1.406182in}}%
\pgfpathlineto{\pgfqpoint{1.168904in}{1.408975in}}%
\pgfpathlineto{\pgfqpoint{1.171556in}{1.416544in}}%
\pgfpathlineto{\pgfqpoint{1.174208in}{1.424064in}}%
\pgfpathlineto{\pgfqpoint{1.176863in}{1.431531in}}%
\pgfpathlineto{\pgfqpoint{1.179518in}{1.438945in}}%
\pgfpathlineto{\pgfqpoint{1.184103in}{1.436328in}}%
\pgfpathlineto{\pgfqpoint{1.188861in}{1.433781in}}%
\pgfpathlineto{\pgfqpoint{1.193787in}{1.431309in}}%
\pgfpathlineto{\pgfqpoint{1.198876in}{1.428913in}}%
\pgfpathclose%
\pgfusepath{fill}%
\end{pgfscope}%
\begin{pgfscope}%
\pgfpathrectangle{\pgfqpoint{0.329460in}{0.284240in}}{\pgfqpoint{1.989680in}{1.989680in}}%
\pgfusepath{clip}%
\pgfsetbuttcap%
\pgfsetroundjoin%
\definecolor{currentfill}{rgb}{0.282884,0.135920,0.453427}%
\pgfsetfillcolor{currentfill}%
\pgfsetlinewidth{0.000000pt}%
\definecolor{currentstroke}{rgb}{0.000000,0.000000,0.000000}%
\pgfsetstrokecolor{currentstroke}%
\pgfsetdash{}{0pt}%
\pgfpathmoveto{\pgfqpoint{0.932303in}{0.850608in}}%
\pgfpathlineto{\pgfqpoint{0.929700in}{0.855066in}}%
\pgfpathlineto{\pgfqpoint{0.927086in}{0.859860in}}%
\pgfpathlineto{\pgfqpoint{0.924461in}{0.864996in}}%
\pgfpathlineto{\pgfqpoint{0.921826in}{0.870481in}}%
\pgfpathlineto{\pgfqpoint{0.907798in}{0.877849in}}%
\pgfpathlineto{\pgfqpoint{0.894260in}{0.885442in}}%
\pgfpathlineto{\pgfqpoint{0.881225in}{0.893250in}}%
\pgfpathlineto{\pgfqpoint{0.868704in}{0.901264in}}%
\pgfpathlineto{\pgfqpoint{0.871652in}{0.895617in}}%
\pgfpathlineto{\pgfqpoint{0.874588in}{0.890316in}}%
\pgfpathlineto{\pgfqpoint{0.877512in}{0.885357in}}%
\pgfpathlineto{\pgfqpoint{0.880424in}{0.880733in}}%
\pgfpathlineto{\pgfqpoint{0.892655in}{0.872890in}}%
\pgfpathlineto{\pgfqpoint{0.905386in}{0.865249in}}%
\pgfpathlineto{\pgfqpoint{0.918607in}{0.857819in}}%
\pgfpathlineto{\pgfqpoint{0.932303in}{0.850608in}}%
\pgfpathclose%
\pgfusepath{fill}%
\end{pgfscope}%
\begin{pgfscope}%
\pgfpathrectangle{\pgfqpoint{0.329460in}{0.284240in}}{\pgfqpoint{1.989680in}{1.989680in}}%
\pgfusepath{clip}%
\pgfsetbuttcap%
\pgfsetroundjoin%
\definecolor{currentfill}{rgb}{0.993248,0.906157,0.143936}%
\pgfsetfillcolor{currentfill}%
\pgfsetlinewidth{0.000000pt}%
\definecolor{currentstroke}{rgb}{0.000000,0.000000,0.000000}%
\pgfsetstrokecolor{currentstroke}%
\pgfsetdash{}{0pt}%
\pgfpathmoveto{\pgfqpoint{1.351188in}{1.731270in}}%
\pgfpathlineto{\pgfqpoint{1.353636in}{1.730038in}}%
\pgfpathlineto{\pgfqpoint{1.356085in}{1.728686in}}%
\pgfpathlineto{\pgfqpoint{1.358535in}{1.727213in}}%
\pgfpathlineto{\pgfqpoint{1.360986in}{1.725620in}}%
\pgfpathlineto{\pgfqpoint{1.360664in}{1.725478in}}%
\pgfpathlineto{\pgfqpoint{1.360332in}{1.725341in}}%
\pgfpathlineto{\pgfqpoint{1.359991in}{1.725209in}}%
\pgfpathlineto{\pgfqpoint{1.359641in}{1.725082in}}%
\pgfpathlineto{\pgfqpoint{1.357526in}{1.726810in}}%
\pgfpathlineto{\pgfqpoint{1.355412in}{1.728417in}}%
\pgfpathlineto{\pgfqpoint{1.353300in}{1.729904in}}%
\pgfpathlineto{\pgfqpoint{1.351188in}{1.731270in}}%
\pgfpathlineto{\pgfqpoint{1.351188in}{1.731270in}}%
\pgfpathlineto{\pgfqpoint{1.351188in}{1.731270in}}%
\pgfpathlineto{\pgfqpoint{1.351188in}{1.731270in}}%
\pgfpathlineto{\pgfqpoint{1.351188in}{1.731270in}}%
\pgfpathclose%
\pgfusepath{fill}%
\end{pgfscope}%
\begin{pgfscope}%
\pgfpathrectangle{\pgfqpoint{0.329460in}{0.284240in}}{\pgfqpoint{1.989680in}{1.989680in}}%
\pgfusepath{clip}%
\pgfsetbuttcap%
\pgfsetroundjoin%
\definecolor{currentfill}{rgb}{0.993248,0.906157,0.143936}%
\pgfsetfillcolor{currentfill}%
\pgfsetlinewidth{0.000000pt}%
\definecolor{currentstroke}{rgb}{0.000000,0.000000,0.000000}%
\pgfsetstrokecolor{currentstroke}%
\pgfsetdash{}{0pt}%
\pgfpathmoveto{\pgfqpoint{1.351188in}{1.731270in}}%
\pgfpathlineto{\pgfqpoint{1.349155in}{1.729877in}}%
\pgfpathlineto{\pgfqpoint{1.347122in}{1.728363in}}%
\pgfpathlineto{\pgfqpoint{1.345088in}{1.726729in}}%
\pgfpathlineto{\pgfqpoint{1.343052in}{1.724974in}}%
\pgfpathlineto{\pgfqpoint{1.342695in}{1.725096in}}%
\pgfpathlineto{\pgfqpoint{1.342346in}{1.725224in}}%
\pgfpathlineto{\pgfqpoint{1.342006in}{1.725356in}}%
\pgfpathlineto{\pgfqpoint{1.341675in}{1.725494in}}%
\pgfpathlineto{\pgfqpoint{1.344054in}{1.727119in}}%
\pgfpathlineto{\pgfqpoint{1.346433in}{1.728623in}}%
\pgfpathlineto{\pgfqpoint{1.348811in}{1.730007in}}%
\pgfpathlineto{\pgfqpoint{1.351188in}{1.731270in}}%
\pgfpathlineto{\pgfqpoint{1.351188in}{1.731270in}}%
\pgfpathlineto{\pgfqpoint{1.351188in}{1.731270in}}%
\pgfpathlineto{\pgfqpoint{1.351188in}{1.731270in}}%
\pgfpathlineto{\pgfqpoint{1.351188in}{1.731270in}}%
\pgfpathclose%
\pgfusepath{fill}%
\end{pgfscope}%
\begin{pgfscope}%
\pgfpathrectangle{\pgfqpoint{0.329460in}{0.284240in}}{\pgfqpoint{1.989680in}{1.989680in}}%
\pgfusepath{clip}%
\pgfsetbuttcap%
\pgfsetroundjoin%
\definecolor{currentfill}{rgb}{0.283072,0.130895,0.449241}%
\pgfsetfillcolor{currentfill}%
\pgfsetlinewidth{0.000000pt}%
\definecolor{currentstroke}{rgb}{0.000000,0.000000,0.000000}%
\pgfsetstrokecolor{currentstroke}%
\pgfsetdash{}{0pt}%
\pgfpathmoveto{\pgfqpoint{1.244917in}{0.877282in}}%
\pgfpathlineto{\pgfqpoint{1.244085in}{0.870786in}}%
\pgfpathlineto{\pgfqpoint{1.243252in}{0.864413in}}%
\pgfpathlineto{\pgfqpoint{1.242418in}{0.858164in}}%
\pgfpathlineto{\pgfqpoint{1.241583in}{0.852045in}}%
\pgfpathlineto{\pgfqpoint{1.227447in}{0.854021in}}%
\pgfpathlineto{\pgfqpoint{1.213447in}{0.856235in}}%
\pgfpathlineto{\pgfqpoint{1.199599in}{0.858684in}}%
\pgfpathlineto{\pgfqpoint{1.185918in}{0.861365in}}%
\pgfpathlineto{\pgfqpoint{1.187175in}{0.867405in}}%
\pgfpathlineto{\pgfqpoint{1.188431in}{0.873574in}}%
\pgfpathlineto{\pgfqpoint{1.189685in}{0.879869in}}%
\pgfpathlineto{\pgfqpoint{1.190938in}{0.886285in}}%
\pgfpathlineto{\pgfqpoint{1.204205in}{0.883696in}}%
\pgfpathlineto{\pgfqpoint{1.217634in}{0.881330in}}%
\pgfpathlineto{\pgfqpoint{1.231209in}{0.879191in}}%
\pgfpathlineto{\pgfqpoint{1.244917in}{0.877282in}}%
\pgfpathclose%
\pgfusepath{fill}%
\end{pgfscope}%
\begin{pgfscope}%
\pgfpathrectangle{\pgfqpoint{0.329460in}{0.284240in}}{\pgfqpoint{1.989680in}{1.989680in}}%
\pgfusepath{clip}%
\pgfsetbuttcap%
\pgfsetroundjoin%
\definecolor{currentfill}{rgb}{0.268510,0.009605,0.335427}%
\pgfsetfillcolor{currentfill}%
\pgfsetlinewidth{0.000000pt}%
\definecolor{currentstroke}{rgb}{0.000000,0.000000,0.000000}%
\pgfsetstrokecolor{currentstroke}%
\pgfsetdash{}{0pt}%
\pgfpathmoveto{\pgfqpoint{1.688423in}{0.801532in}}%
\pgfpathlineto{\pgfqpoint{1.690630in}{0.801222in}}%
\pgfpathlineto{\pgfqpoint{1.692842in}{0.801164in}}%
\pgfpathlineto{\pgfqpoint{1.695061in}{0.801364in}}%
\pgfpathlineto{\pgfqpoint{1.697287in}{0.801826in}}%
\pgfpathlineto{\pgfqpoint{1.682893in}{0.796038in}}%
\pgfpathlineto{\pgfqpoint{1.668131in}{0.790495in}}%
\pgfpathlineto{\pgfqpoint{1.653018in}{0.785202in}}%
\pgfpathlineto{\pgfqpoint{1.637568in}{0.780168in}}%
\pgfpathlineto{\pgfqpoint{1.635720in}{0.779844in}}%
\pgfpathlineto{\pgfqpoint{1.633878in}{0.779783in}}%
\pgfpathlineto{\pgfqpoint{1.632042in}{0.779980in}}%
\pgfpathlineto{\pgfqpoint{1.630210in}{0.780429in}}%
\pgfpathlineto{\pgfqpoint{1.645269in}{0.785335in}}%
\pgfpathlineto{\pgfqpoint{1.660001in}{0.790491in}}%
\pgfpathlineto{\pgfqpoint{1.674391in}{0.795892in}}%
\pgfpathlineto{\pgfqpoint{1.688423in}{0.801532in}}%
\pgfpathclose%
\pgfusepath{fill}%
\end{pgfscope}%
\begin{pgfscope}%
\pgfpathrectangle{\pgfqpoint{0.329460in}{0.284240in}}{\pgfqpoint{1.989680in}{1.989680in}}%
\pgfusepath{clip}%
\pgfsetbuttcap%
\pgfsetroundjoin%
\definecolor{currentfill}{rgb}{0.412913,0.803041,0.357269}%
\pgfsetfillcolor{currentfill}%
\pgfsetlinewidth{0.000000pt}%
\definecolor{currentstroke}{rgb}{0.000000,0.000000,0.000000}%
\pgfsetstrokecolor{currentstroke}%
\pgfsetdash{}{0pt}%
\pgfpathmoveto{\pgfqpoint{1.211485in}{1.523015in}}%
\pgfpathlineto{\pgfqpoint{1.208815in}{1.516394in}}%
\pgfpathlineto{\pgfqpoint{1.206145in}{1.509698in}}%
\pgfpathlineto{\pgfqpoint{1.203477in}{1.502927in}}%
\pgfpathlineto{\pgfqpoint{1.200809in}{1.496084in}}%
\pgfpathlineto{\pgfqpoint{1.196939in}{1.498415in}}%
\pgfpathlineto{\pgfqpoint{1.193228in}{1.500803in}}%
\pgfpathlineto{\pgfqpoint{1.189678in}{1.503245in}}%
\pgfpathlineto{\pgfqpoint{1.186293in}{1.505739in}}%
\pgfpathlineto{\pgfqpoint{1.189213in}{1.512401in}}%
\pgfpathlineto{\pgfqpoint{1.192135in}{1.518990in}}%
\pgfpathlineto{\pgfqpoint{1.195058in}{1.525505in}}%
\pgfpathlineto{\pgfqpoint{1.197982in}{1.531945in}}%
\pgfpathlineto{\pgfqpoint{1.201131in}{1.529638in}}%
\pgfpathlineto{\pgfqpoint{1.204434in}{1.527379in}}%
\pgfpathlineto{\pgfqpoint{1.207886in}{1.525170in}}%
\pgfpathlineto{\pgfqpoint{1.211485in}{1.523015in}}%
\pgfpathclose%
\pgfusepath{fill}%
\end{pgfscope}%
\begin{pgfscope}%
\pgfpathrectangle{\pgfqpoint{0.329460in}{0.284240in}}{\pgfqpoint{1.989680in}{1.989680in}}%
\pgfusepath{clip}%
\pgfsetbuttcap%
\pgfsetroundjoin%
\definecolor{currentfill}{rgb}{0.636902,0.856542,0.216620}%
\pgfsetfillcolor{currentfill}%
\pgfsetlinewidth{0.000000pt}%
\definecolor{currentstroke}{rgb}{0.000000,0.000000,0.000000}%
\pgfsetstrokecolor{currentstroke}%
\pgfsetdash{}{0pt}%
\pgfpathmoveto{\pgfqpoint{1.233181in}{1.602765in}}%
\pgfpathlineto{\pgfqpoint{1.230241in}{1.597352in}}%
\pgfpathlineto{\pgfqpoint{1.227302in}{1.591845in}}%
\pgfpathlineto{\pgfqpoint{1.224365in}{1.586246in}}%
\pgfpathlineto{\pgfqpoint{1.221428in}{1.580556in}}%
\pgfpathlineto{\pgfqpoint{1.218885in}{1.582533in}}%
\pgfpathlineto{\pgfqpoint{1.216477in}{1.584547in}}%
\pgfpathlineto{\pgfqpoint{1.214205in}{1.586595in}}%
\pgfpathlineto{\pgfqpoint{1.212073in}{1.588676in}}%
\pgfpathlineto{\pgfqpoint{1.215217in}{1.594174in}}%
\pgfpathlineto{\pgfqpoint{1.218362in}{1.599583in}}%
\pgfpathlineto{\pgfqpoint{1.221509in}{1.604899in}}%
\pgfpathlineto{\pgfqpoint{1.224657in}{1.610123in}}%
\pgfpathlineto{\pgfqpoint{1.226601in}{1.608237in}}%
\pgfpathlineto{\pgfqpoint{1.228671in}{1.606381in}}%
\pgfpathlineto{\pgfqpoint{1.230865in}{1.604557in}}%
\pgfpathlineto{\pgfqpoint{1.233181in}{1.602765in}}%
\pgfpathclose%
\pgfusepath{fill}%
\end{pgfscope}%
\begin{pgfscope}%
\pgfpathrectangle{\pgfqpoint{0.329460in}{0.284240in}}{\pgfqpoint{1.989680in}{1.989680in}}%
\pgfusepath{clip}%
\pgfsetbuttcap%
\pgfsetroundjoin%
\definecolor{currentfill}{rgb}{0.274128,0.199721,0.498911}%
\pgfsetfillcolor{currentfill}%
\pgfsetlinewidth{0.000000pt}%
\definecolor{currentstroke}{rgb}{0.000000,0.000000,0.000000}%
\pgfsetstrokecolor{currentstroke}%
\pgfsetdash{}{0pt}%
\pgfpathmoveto{\pgfqpoint{1.251552in}{0.933219in}}%
\pgfpathlineto{\pgfqpoint{1.250724in}{0.925873in}}%
\pgfpathlineto{\pgfqpoint{1.249896in}{0.918621in}}%
\pgfpathlineto{\pgfqpoint{1.249068in}{0.911467in}}%
\pgfpathlineto{\pgfqpoint{1.248239in}{0.904414in}}%
\pgfpathlineto{\pgfqpoint{1.234959in}{0.906256in}}%
\pgfpathlineto{\pgfqpoint{1.221806in}{0.908320in}}%
\pgfpathlineto{\pgfqpoint{1.208795in}{0.910602in}}%
\pgfpathlineto{\pgfqpoint{1.195941in}{0.913101in}}%
\pgfpathlineto{\pgfqpoint{1.197189in}{0.920075in}}%
\pgfpathlineto{\pgfqpoint{1.198437in}{0.927150in}}%
\pgfpathlineto{\pgfqpoint{1.199684in}{0.934323in}}%
\pgfpathlineto{\pgfqpoint{1.200930in}{0.941590in}}%
\pgfpathlineto{\pgfqpoint{1.213373in}{0.939182in}}%
\pgfpathlineto{\pgfqpoint{1.225967in}{0.936983in}}%
\pgfpathlineto{\pgfqpoint{1.238698in}{0.934994in}}%
\pgfpathlineto{\pgfqpoint{1.251552in}{0.933219in}}%
\pgfpathclose%
\pgfusepath{fill}%
\end{pgfscope}%
\begin{pgfscope}%
\pgfpathrectangle{\pgfqpoint{0.329460in}{0.284240in}}{\pgfqpoint{1.989680in}{1.989680in}}%
\pgfusepath{clip}%
\pgfsetbuttcap%
\pgfsetroundjoin%
\definecolor{currentfill}{rgb}{0.993248,0.906157,0.143936}%
\pgfsetfillcolor{currentfill}%
\pgfsetlinewidth{0.000000pt}%
\definecolor{currentstroke}{rgb}{0.000000,0.000000,0.000000}%
\pgfsetstrokecolor{currentstroke}%
\pgfsetdash{}{0pt}%
\pgfpathmoveto{\pgfqpoint{1.351188in}{1.731270in}}%
\pgfpathlineto{\pgfqpoint{1.353300in}{1.729904in}}%
\pgfpathlineto{\pgfqpoint{1.355412in}{1.728417in}}%
\pgfpathlineto{\pgfqpoint{1.357526in}{1.726810in}}%
\pgfpathlineto{\pgfqpoint{1.359641in}{1.725082in}}%
\pgfpathlineto{\pgfqpoint{1.359282in}{1.724960in}}%
\pgfpathlineto{\pgfqpoint{1.358916in}{1.724844in}}%
\pgfpathlineto{\pgfqpoint{1.358542in}{1.724733in}}%
\pgfpathlineto{\pgfqpoint{1.358160in}{1.724627in}}%
\pgfpathlineto{\pgfqpoint{1.356416in}{1.726469in}}%
\pgfpathlineto{\pgfqpoint{1.354672in}{1.728190in}}%
\pgfpathlineto{\pgfqpoint{1.352930in}{1.729791in}}%
\pgfpathlineto{\pgfqpoint{1.351188in}{1.731270in}}%
\pgfpathlineto{\pgfqpoint{1.351188in}{1.731270in}}%
\pgfpathlineto{\pgfqpoint{1.351188in}{1.731270in}}%
\pgfpathlineto{\pgfqpoint{1.351188in}{1.731270in}}%
\pgfpathlineto{\pgfqpoint{1.351188in}{1.731270in}}%
\pgfpathclose%
\pgfusepath{fill}%
\end{pgfscope}%
\begin{pgfscope}%
\pgfpathrectangle{\pgfqpoint{0.329460in}{0.284240in}}{\pgfqpoint{1.989680in}{1.989680in}}%
\pgfusepath{clip}%
\pgfsetbuttcap%
\pgfsetroundjoin%
\definecolor{currentfill}{rgb}{0.855810,0.888601,0.097452}%
\pgfsetfillcolor{currentfill}%
\pgfsetlinewidth{0.000000pt}%
\definecolor{currentstroke}{rgb}{0.000000,0.000000,0.000000}%
\pgfsetstrokecolor{currentstroke}%
\pgfsetdash{}{0pt}%
\pgfpathmoveto{\pgfqpoint{1.431827in}{1.685757in}}%
\pgfpathlineto{\pgfqpoint{1.435181in}{1.682419in}}%
\pgfpathlineto{\pgfqpoint{1.438534in}{1.678972in}}%
\pgfpathlineto{\pgfqpoint{1.441886in}{1.675416in}}%
\pgfpathlineto{\pgfqpoint{1.445238in}{1.671752in}}%
\pgfpathlineto{\pgfqpoint{1.444315in}{1.670361in}}%
\pgfpathlineto{\pgfqpoint{1.443298in}{1.668984in}}%
\pgfpathlineto{\pgfqpoint{1.442189in}{1.667623in}}%
\pgfpathlineto{\pgfqpoint{1.440988in}{1.666279in}}%
\pgfpathlineto{\pgfqpoint{1.437786in}{1.670142in}}%
\pgfpathlineto{\pgfqpoint{1.434582in}{1.673896in}}%
\pgfpathlineto{\pgfqpoint{1.431378in}{1.677541in}}%
\pgfpathlineto{\pgfqpoint{1.428173in}{1.681076in}}%
\pgfpathlineto{\pgfqpoint{1.429205in}{1.682226in}}%
\pgfpathlineto{\pgfqpoint{1.430158in}{1.683390in}}%
\pgfpathlineto{\pgfqpoint{1.431033in}{1.684567in}}%
\pgfpathlineto{\pgfqpoint{1.431827in}{1.685757in}}%
\pgfpathclose%
\pgfusepath{fill}%
\end{pgfscope}%
\begin{pgfscope}%
\pgfpathrectangle{\pgfqpoint{0.329460in}{0.284240in}}{\pgfqpoint{1.989680in}{1.989680in}}%
\pgfusepath{clip}%
\pgfsetbuttcap%
\pgfsetroundjoin%
\definecolor{currentfill}{rgb}{0.993248,0.906157,0.143936}%
\pgfsetfillcolor{currentfill}%
\pgfsetlinewidth{0.000000pt}%
\definecolor{currentstroke}{rgb}{0.000000,0.000000,0.000000}%
\pgfsetstrokecolor{currentstroke}%
\pgfsetdash{}{0pt}%
\pgfpathmoveto{\pgfqpoint{1.351188in}{1.731270in}}%
\pgfpathlineto{\pgfqpoint{1.349532in}{1.729768in}}%
\pgfpathlineto{\pgfqpoint{1.347875in}{1.728146in}}%
\pgfpathlineto{\pgfqpoint{1.346218in}{1.726402in}}%
\pgfpathlineto{\pgfqpoint{1.344560in}{1.724538in}}%
\pgfpathlineto{\pgfqpoint{1.344172in}{1.724639in}}%
\pgfpathlineto{\pgfqpoint{1.343791in}{1.724745in}}%
\pgfpathlineto{\pgfqpoint{1.343418in}{1.724857in}}%
\pgfpathlineto{\pgfqpoint{1.343052in}{1.724974in}}%
\pgfpathlineto{\pgfqpoint{1.345088in}{1.726729in}}%
\pgfpathlineto{\pgfqpoint{1.347122in}{1.728363in}}%
\pgfpathlineto{\pgfqpoint{1.349155in}{1.729877in}}%
\pgfpathlineto{\pgfqpoint{1.351188in}{1.731270in}}%
\pgfpathlineto{\pgfqpoint{1.351188in}{1.731270in}}%
\pgfpathlineto{\pgfqpoint{1.351188in}{1.731270in}}%
\pgfpathlineto{\pgfqpoint{1.351188in}{1.731270in}}%
\pgfpathlineto{\pgfqpoint{1.351188in}{1.731270in}}%
\pgfpathclose%
\pgfusepath{fill}%
\end{pgfscope}%
\begin{pgfscope}%
\pgfpathrectangle{\pgfqpoint{0.329460in}{0.284240in}}{\pgfqpoint{1.989680in}{1.989680in}}%
\pgfusepath{clip}%
\pgfsetbuttcap%
\pgfsetroundjoin%
\definecolor{currentfill}{rgb}{0.993248,0.906157,0.143936}%
\pgfsetfillcolor{currentfill}%
\pgfsetlinewidth{0.000000pt}%
\definecolor{currentstroke}{rgb}{0.000000,0.000000,0.000000}%
\pgfsetstrokecolor{currentstroke}%
\pgfsetdash{}{0pt}%
\pgfpathmoveto{\pgfqpoint{1.364629in}{1.728416in}}%
\pgfpathlineto{\pgfqpoint{1.367990in}{1.727403in}}%
\pgfpathlineto{\pgfqpoint{1.371352in}{1.726270in}}%
\pgfpathlineto{\pgfqpoint{1.374713in}{1.725017in}}%
\pgfpathlineto{\pgfqpoint{1.378075in}{1.723645in}}%
\pgfpathlineto{\pgfqpoint{1.377806in}{1.723251in}}%
\pgfpathlineto{\pgfqpoint{1.377511in}{1.722861in}}%
\pgfpathlineto{\pgfqpoint{1.377190in}{1.722476in}}%
\pgfpathlineto{\pgfqpoint{1.376843in}{1.722095in}}%
\pgfpathlineto{\pgfqpoint{1.373634in}{1.723661in}}%
\pgfpathlineto{\pgfqpoint{1.370426in}{1.725108in}}%
\pgfpathlineto{\pgfqpoint{1.367219in}{1.726435in}}%
\pgfpathlineto{\pgfqpoint{1.364011in}{1.727642in}}%
\pgfpathlineto{\pgfqpoint{1.364185in}{1.727832in}}%
\pgfpathlineto{\pgfqpoint{1.364346in}{1.728025in}}%
\pgfpathlineto{\pgfqpoint{1.364494in}{1.728219in}}%
\pgfpathlineto{\pgfqpoint{1.364629in}{1.728416in}}%
\pgfpathclose%
\pgfusepath{fill}%
\end{pgfscope}%
\begin{pgfscope}%
\pgfpathrectangle{\pgfqpoint{0.329460in}{0.284240in}}{\pgfqpoint{1.989680in}{1.989680in}}%
\pgfusepath{clip}%
\pgfsetbuttcap%
\pgfsetroundjoin%
\definecolor{currentfill}{rgb}{0.993248,0.906157,0.143936}%
\pgfsetfillcolor{currentfill}%
\pgfsetlinewidth{0.000000pt}%
\definecolor{currentstroke}{rgb}{0.000000,0.000000,0.000000}%
\pgfsetstrokecolor{currentstroke}%
\pgfsetdash{}{0pt}%
\pgfpathmoveto{\pgfqpoint{1.351188in}{1.731270in}}%
\pgfpathlineto{\pgfqpoint{1.352930in}{1.729791in}}%
\pgfpathlineto{\pgfqpoint{1.354672in}{1.728190in}}%
\pgfpathlineto{\pgfqpoint{1.356416in}{1.726469in}}%
\pgfpathlineto{\pgfqpoint{1.358160in}{1.724627in}}%
\pgfpathlineto{\pgfqpoint{1.357772in}{1.724528in}}%
\pgfpathlineto{\pgfqpoint{1.357377in}{1.724434in}}%
\pgfpathlineto{\pgfqpoint{1.356975in}{1.724345in}}%
\pgfpathlineto{\pgfqpoint{1.356568in}{1.724263in}}%
\pgfpathlineto{\pgfqpoint{1.355222in}{1.726196in}}%
\pgfpathlineto{\pgfqpoint{1.353877in}{1.728008in}}%
\pgfpathlineto{\pgfqpoint{1.352532in}{1.729700in}}%
\pgfpathlineto{\pgfqpoint{1.351188in}{1.731270in}}%
\pgfpathlineto{\pgfqpoint{1.351188in}{1.731270in}}%
\pgfpathlineto{\pgfqpoint{1.351188in}{1.731270in}}%
\pgfpathlineto{\pgfqpoint{1.351188in}{1.731270in}}%
\pgfpathlineto{\pgfqpoint{1.351188in}{1.731270in}}%
\pgfpathclose%
\pgfusepath{fill}%
\end{pgfscope}%
\begin{pgfscope}%
\pgfpathrectangle{\pgfqpoint{0.329460in}{0.284240in}}{\pgfqpoint{1.989680in}{1.989680in}}%
\pgfusepath{clip}%
\pgfsetbuttcap%
\pgfsetroundjoin%
\definecolor{currentfill}{rgb}{0.195860,0.395433,0.555276}%
\pgfsetfillcolor{currentfill}%
\pgfsetlinewidth{0.000000pt}%
\definecolor{currentstroke}{rgb}{0.000000,0.000000,0.000000}%
\pgfsetstrokecolor{currentstroke}%
\pgfsetdash{}{0pt}%
\pgfpathmoveto{\pgfqpoint{1.225775in}{1.101842in}}%
\pgfpathlineto{\pgfqpoint{1.224534in}{1.093350in}}%
\pgfpathlineto{\pgfqpoint{1.223293in}{1.084891in}}%
\pgfpathlineto{\pgfqpoint{1.222052in}{1.076467in}}%
\pgfpathlineto{\pgfqpoint{1.220811in}{1.068081in}}%
\pgfpathlineto{\pgfqpoint{1.210149in}{1.070306in}}%
\pgfpathlineto{\pgfqpoint{1.199639in}{1.072703in}}%
\pgfpathlineto{\pgfqpoint{1.189293in}{1.075269in}}%
\pgfpathlineto{\pgfqpoint{1.179120in}{1.078003in}}%
\pgfpathlineto{\pgfqpoint{1.180755in}{1.086281in}}%
\pgfpathlineto{\pgfqpoint{1.182390in}{1.094598in}}%
\pgfpathlineto{\pgfqpoint{1.184025in}{1.102951in}}%
\pgfpathlineto{\pgfqpoint{1.185661in}{1.111337in}}%
\pgfpathlineto{\pgfqpoint{1.195449in}{1.108720in}}%
\pgfpathlineto{\pgfqpoint{1.205405in}{1.106264in}}%
\pgfpathlineto{\pgfqpoint{1.215517in}{1.103970in}}%
\pgfpathlineto{\pgfqpoint{1.225775in}{1.101842in}}%
\pgfpathclose%
\pgfusepath{fill}%
\end{pgfscope}%
\begin{pgfscope}%
\pgfpathrectangle{\pgfqpoint{0.329460in}{0.284240in}}{\pgfqpoint{1.989680in}{1.989680in}}%
\pgfusepath{clip}%
\pgfsetbuttcap%
\pgfsetroundjoin%
\definecolor{currentfill}{rgb}{0.993248,0.906157,0.143936}%
\pgfsetfillcolor{currentfill}%
\pgfsetlinewidth{0.000000pt}%
\definecolor{currentstroke}{rgb}{0.000000,0.000000,0.000000}%
\pgfsetstrokecolor{currentstroke}%
\pgfsetdash{}{0pt}%
\pgfpathmoveto{\pgfqpoint{1.351188in}{1.731270in}}%
\pgfpathlineto{\pgfqpoint{1.349935in}{1.729683in}}%
\pgfpathlineto{\pgfqpoint{1.348682in}{1.727974in}}%
\pgfpathlineto{\pgfqpoint{1.347428in}{1.726145in}}%
\pgfpathlineto{\pgfqpoint{1.346173in}{1.724195in}}%
\pgfpathlineto{\pgfqpoint{1.345761in}{1.724272in}}%
\pgfpathlineto{\pgfqpoint{1.345355in}{1.724355in}}%
\pgfpathlineto{\pgfqpoint{1.344954in}{1.724444in}}%
\pgfpathlineto{\pgfqpoint{1.344560in}{1.724538in}}%
\pgfpathlineto{\pgfqpoint{1.346218in}{1.726402in}}%
\pgfpathlineto{\pgfqpoint{1.347875in}{1.728146in}}%
\pgfpathlineto{\pgfqpoint{1.349532in}{1.729768in}}%
\pgfpathlineto{\pgfqpoint{1.351188in}{1.731270in}}%
\pgfpathlineto{\pgfqpoint{1.351188in}{1.731270in}}%
\pgfpathlineto{\pgfqpoint{1.351188in}{1.731270in}}%
\pgfpathlineto{\pgfqpoint{1.351188in}{1.731270in}}%
\pgfpathlineto{\pgfqpoint{1.351188in}{1.731270in}}%
\pgfpathclose%
\pgfusepath{fill}%
\end{pgfscope}%
\begin{pgfscope}%
\pgfpathrectangle{\pgfqpoint{0.329460in}{0.284240in}}{\pgfqpoint{1.989680in}{1.989680in}}%
\pgfusepath{clip}%
\pgfsetbuttcap%
\pgfsetroundjoin%
\definecolor{currentfill}{rgb}{0.993248,0.906157,0.143936}%
\pgfsetfillcolor{currentfill}%
\pgfsetlinewidth{0.000000pt}%
\definecolor{currentstroke}{rgb}{0.000000,0.000000,0.000000}%
\pgfsetstrokecolor{currentstroke}%
\pgfsetdash{}{0pt}%
\pgfpathmoveto{\pgfqpoint{1.351188in}{1.731270in}}%
\pgfpathlineto{\pgfqpoint{1.352532in}{1.729700in}}%
\pgfpathlineto{\pgfqpoint{1.353877in}{1.728008in}}%
\pgfpathlineto{\pgfqpoint{1.355222in}{1.726196in}}%
\pgfpathlineto{\pgfqpoint{1.356568in}{1.724263in}}%
\pgfpathlineto{\pgfqpoint{1.356156in}{1.724187in}}%
\pgfpathlineto{\pgfqpoint{1.355738in}{1.724117in}}%
\pgfpathlineto{\pgfqpoint{1.355316in}{1.724053in}}%
\pgfpathlineto{\pgfqpoint{1.354890in}{1.723995in}}%
\pgfpathlineto{\pgfqpoint{1.353964in}{1.725995in}}%
\pgfpathlineto{\pgfqpoint{1.353038in}{1.727875in}}%
\pgfpathlineto{\pgfqpoint{1.352113in}{1.729633in}}%
\pgfpathlineto{\pgfqpoint{1.351188in}{1.731270in}}%
\pgfpathlineto{\pgfqpoint{1.351188in}{1.731270in}}%
\pgfpathlineto{\pgfqpoint{1.351188in}{1.731270in}}%
\pgfpathlineto{\pgfqpoint{1.351188in}{1.731270in}}%
\pgfpathlineto{\pgfqpoint{1.351188in}{1.731270in}}%
\pgfpathclose%
\pgfusepath{fill}%
\end{pgfscope}%
\begin{pgfscope}%
\pgfpathrectangle{\pgfqpoint{0.329460in}{0.284240in}}{\pgfqpoint{1.989680in}{1.989680in}}%
\pgfusepath{clip}%
\pgfsetbuttcap%
\pgfsetroundjoin%
\definecolor{currentfill}{rgb}{0.993248,0.906157,0.143936}%
\pgfsetfillcolor{currentfill}%
\pgfsetlinewidth{0.000000pt}%
\definecolor{currentstroke}{rgb}{0.000000,0.000000,0.000000}%
\pgfsetstrokecolor{currentstroke}%
\pgfsetdash{}{0pt}%
\pgfpathmoveto{\pgfqpoint{1.338529in}{1.727476in}}%
\pgfpathlineto{\pgfqpoint{1.335363in}{1.726227in}}%
\pgfpathlineto{\pgfqpoint{1.332197in}{1.724858in}}%
\pgfpathlineto{\pgfqpoint{1.329030in}{1.723370in}}%
\pgfpathlineto{\pgfqpoint{1.325862in}{1.721762in}}%
\pgfpathlineto{\pgfqpoint{1.325492in}{1.722137in}}%
\pgfpathlineto{\pgfqpoint{1.325148in}{1.722518in}}%
\pgfpathlineto{\pgfqpoint{1.324830in}{1.722904in}}%
\pgfpathlineto{\pgfqpoint{1.324538in}{1.723295in}}%
\pgfpathlineto{\pgfqpoint{1.327870in}{1.724710in}}%
\pgfpathlineto{\pgfqpoint{1.331202in}{1.726007in}}%
\pgfpathlineto{\pgfqpoint{1.334534in}{1.727184in}}%
\pgfpathlineto{\pgfqpoint{1.337865in}{1.728241in}}%
\pgfpathlineto{\pgfqpoint{1.338012in}{1.728046in}}%
\pgfpathlineto{\pgfqpoint{1.338171in}{1.727853in}}%
\pgfpathlineto{\pgfqpoint{1.338344in}{1.727663in}}%
\pgfpathlineto{\pgfqpoint{1.338529in}{1.727476in}}%
\pgfpathclose%
\pgfusepath{fill}%
\end{pgfscope}%
\begin{pgfscope}%
\pgfpathrectangle{\pgfqpoint{0.329460in}{0.284240in}}{\pgfqpoint{1.989680in}{1.989680in}}%
\pgfusepath{clip}%
\pgfsetbuttcap%
\pgfsetroundjoin%
\definecolor{currentfill}{rgb}{0.993248,0.906157,0.143936}%
\pgfsetfillcolor{currentfill}%
\pgfsetlinewidth{0.000000pt}%
\definecolor{currentstroke}{rgb}{0.000000,0.000000,0.000000}%
\pgfsetstrokecolor{currentstroke}%
\pgfsetdash{}{0pt}%
\pgfpathmoveto{\pgfqpoint{1.351188in}{1.731270in}}%
\pgfpathlineto{\pgfqpoint{1.350358in}{1.729621in}}%
\pgfpathlineto{\pgfqpoint{1.349528in}{1.727852in}}%
\pgfpathlineto{\pgfqpoint{1.348697in}{1.725961in}}%
\pgfpathlineto{\pgfqpoint{1.347867in}{1.723949in}}%
\pgfpathlineto{\pgfqpoint{1.347437in}{1.724001in}}%
\pgfpathlineto{\pgfqpoint{1.347012in}{1.724060in}}%
\pgfpathlineto{\pgfqpoint{1.346590in}{1.724124in}}%
\pgfpathlineto{\pgfqpoint{1.346173in}{1.724195in}}%
\pgfpathlineto{\pgfqpoint{1.347428in}{1.726145in}}%
\pgfpathlineto{\pgfqpoint{1.348682in}{1.727974in}}%
\pgfpathlineto{\pgfqpoint{1.349935in}{1.729683in}}%
\pgfpathlineto{\pgfqpoint{1.351188in}{1.731270in}}%
\pgfpathlineto{\pgfqpoint{1.351188in}{1.731270in}}%
\pgfpathlineto{\pgfqpoint{1.351188in}{1.731270in}}%
\pgfpathlineto{\pgfqpoint{1.351188in}{1.731270in}}%
\pgfpathlineto{\pgfqpoint{1.351188in}{1.731270in}}%
\pgfpathclose%
\pgfusepath{fill}%
\end{pgfscope}%
\begin{pgfscope}%
\pgfpathrectangle{\pgfqpoint{0.329460in}{0.284240in}}{\pgfqpoint{1.989680in}{1.989680in}}%
\pgfusepath{clip}%
\pgfsetbuttcap%
\pgfsetroundjoin%
\definecolor{currentfill}{rgb}{0.279566,0.067836,0.391917}%
\pgfsetfillcolor{currentfill}%
\pgfsetlinewidth{0.000000pt}%
\definecolor{currentstroke}{rgb}{0.000000,0.000000,0.000000}%
\pgfsetstrokecolor{currentstroke}%
\pgfsetdash{}{0pt}%
\pgfpathmoveto{\pgfqpoint{1.533870in}{0.841232in}}%
\pgfpathlineto{\pgfqpoint{1.535225in}{0.835914in}}%
\pgfpathlineto{\pgfqpoint{1.536581in}{0.830748in}}%
\pgfpathlineto{\pgfqpoint{1.537940in}{0.825736in}}%
\pgfpathlineto{\pgfqpoint{1.539301in}{0.820884in}}%
\pgfpathlineto{\pgfqpoint{1.524959in}{0.817803in}}%
\pgfpathlineto{\pgfqpoint{1.510425in}{0.814967in}}%
\pgfpathlineto{\pgfqpoint{1.495714in}{0.812379in}}%
\pgfpathlineto{\pgfqpoint{1.480843in}{0.810042in}}%
\pgfpathlineto{\pgfqpoint{1.479904in}{0.814981in}}%
\pgfpathlineto{\pgfqpoint{1.478966in}{0.820079in}}%
\pgfpathlineto{\pgfqpoint{1.478030in}{0.825332in}}%
\pgfpathlineto{\pgfqpoint{1.477094in}{0.830736in}}%
\pgfpathlineto{\pgfqpoint{1.491537in}{0.832998in}}%
\pgfpathlineto{\pgfqpoint{1.505824in}{0.835504in}}%
\pgfpathlineto{\pgfqpoint{1.519940in}{0.838249in}}%
\pgfpathlineto{\pgfqpoint{1.533870in}{0.841232in}}%
\pgfpathclose%
\pgfusepath{fill}%
\end{pgfscope}%
\begin{pgfscope}%
\pgfpathrectangle{\pgfqpoint{0.329460in}{0.284240in}}{\pgfqpoint{1.989680in}{1.989680in}}%
\pgfusepath{clip}%
\pgfsetbuttcap%
\pgfsetroundjoin%
\definecolor{currentfill}{rgb}{0.248629,0.278775,0.534556}%
\pgfsetfillcolor{currentfill}%
\pgfsetlinewidth{0.000000pt}%
\definecolor{currentstroke}{rgb}{0.000000,0.000000,0.000000}%
\pgfsetstrokecolor{currentstroke}%
\pgfsetdash{}{0pt}%
\pgfpathmoveto{\pgfqpoint{1.501698in}{1.004876in}}%
\pgfpathlineto{\pgfqpoint{1.503030in}{0.996992in}}%
\pgfpathlineto{\pgfqpoint{1.504363in}{0.989177in}}%
\pgfpathlineto{\pgfqpoint{1.505696in}{0.981433in}}%
\pgfpathlineto{\pgfqpoint{1.507029in}{0.973764in}}%
\pgfpathlineto{\pgfqpoint{1.495137in}{0.971270in}}%
\pgfpathlineto{\pgfqpoint{1.483087in}{0.968975in}}%
\pgfpathlineto{\pgfqpoint{1.470894in}{0.966881in}}%
\pgfpathlineto{\pgfqpoint{1.458571in}{0.964990in}}%
\pgfpathlineto{\pgfqpoint{1.457651in}{0.972744in}}%
\pgfpathlineto{\pgfqpoint{1.456731in}{0.980573in}}%
\pgfpathlineto{\pgfqpoint{1.455812in}{0.988474in}}%
\pgfpathlineto{\pgfqpoint{1.454893in}{0.996443in}}%
\pgfpathlineto{\pgfqpoint{1.466795in}{0.998260in}}%
\pgfpathlineto{\pgfqpoint{1.478572in}{1.000273in}}%
\pgfpathlineto{\pgfqpoint{1.490210in}{1.002479in}}%
\pgfpathlineto{\pgfqpoint{1.501698in}{1.004876in}}%
\pgfpathclose%
\pgfusepath{fill}%
\end{pgfscope}%
\begin{pgfscope}%
\pgfpathrectangle{\pgfqpoint{0.329460in}{0.284240in}}{\pgfqpoint{1.989680in}{1.989680in}}%
\pgfusepath{clip}%
\pgfsetbuttcap%
\pgfsetroundjoin%
\definecolor{currentfill}{rgb}{0.993248,0.906157,0.143936}%
\pgfsetfillcolor{currentfill}%
\pgfsetlinewidth{0.000000pt}%
\definecolor{currentstroke}{rgb}{0.000000,0.000000,0.000000}%
\pgfsetstrokecolor{currentstroke}%
\pgfsetdash{}{0pt}%
\pgfpathmoveto{\pgfqpoint{1.351188in}{1.731270in}}%
\pgfpathlineto{\pgfqpoint{1.352113in}{1.729633in}}%
\pgfpathlineto{\pgfqpoint{1.353038in}{1.727875in}}%
\pgfpathlineto{\pgfqpoint{1.353964in}{1.725995in}}%
\pgfpathlineto{\pgfqpoint{1.354890in}{1.723995in}}%
\pgfpathlineto{\pgfqpoint{1.354461in}{1.723944in}}%
\pgfpathlineto{\pgfqpoint{1.354028in}{1.723899in}}%
\pgfpathlineto{\pgfqpoint{1.353592in}{1.723860in}}%
\pgfpathlineto{\pgfqpoint{1.353153in}{1.723828in}}%
\pgfpathlineto{\pgfqpoint{1.352661in}{1.725870in}}%
\pgfpathlineto{\pgfqpoint{1.352170in}{1.727791in}}%
\pgfpathlineto{\pgfqpoint{1.351679in}{1.729591in}}%
\pgfpathlineto{\pgfqpoint{1.351188in}{1.731270in}}%
\pgfpathlineto{\pgfqpoint{1.351188in}{1.731270in}}%
\pgfpathlineto{\pgfqpoint{1.351188in}{1.731270in}}%
\pgfpathlineto{\pgfqpoint{1.351188in}{1.731270in}}%
\pgfpathlineto{\pgfqpoint{1.351188in}{1.731270in}}%
\pgfpathclose%
\pgfusepath{fill}%
\end{pgfscope}%
\begin{pgfscope}%
\pgfpathrectangle{\pgfqpoint{0.329460in}{0.284240in}}{\pgfqpoint{1.989680in}{1.989680in}}%
\pgfusepath{clip}%
\pgfsetbuttcap%
\pgfsetroundjoin%
\definecolor{currentfill}{rgb}{0.993248,0.906157,0.143936}%
\pgfsetfillcolor{currentfill}%
\pgfsetlinewidth{0.000000pt}%
\definecolor{currentstroke}{rgb}{0.000000,0.000000,0.000000}%
\pgfsetstrokecolor{currentstroke}%
\pgfsetdash{}{0pt}%
\pgfpathmoveto{\pgfqpoint{1.351188in}{1.731270in}}%
\pgfpathlineto{\pgfqpoint{1.350794in}{1.729585in}}%
\pgfpathlineto{\pgfqpoint{1.350401in}{1.727779in}}%
\pgfpathlineto{\pgfqpoint{1.350007in}{1.725853in}}%
\pgfpathlineto{\pgfqpoint{1.349613in}{1.723805in}}%
\pgfpathlineto{\pgfqpoint{1.349173in}{1.723831in}}%
\pgfpathlineto{\pgfqpoint{1.348735in}{1.723864in}}%
\pgfpathlineto{\pgfqpoint{1.348299in}{1.723904in}}%
\pgfpathlineto{\pgfqpoint{1.347867in}{1.723949in}}%
\pgfpathlineto{\pgfqpoint{1.348697in}{1.725961in}}%
\pgfpathlineto{\pgfqpoint{1.349528in}{1.727852in}}%
\pgfpathlineto{\pgfqpoint{1.350358in}{1.729621in}}%
\pgfpathlineto{\pgfqpoint{1.351188in}{1.731270in}}%
\pgfpathlineto{\pgfqpoint{1.351188in}{1.731270in}}%
\pgfpathlineto{\pgfqpoint{1.351188in}{1.731270in}}%
\pgfpathlineto{\pgfqpoint{1.351188in}{1.731270in}}%
\pgfpathlineto{\pgfqpoint{1.351188in}{1.731270in}}%
\pgfpathclose%
\pgfusepath{fill}%
\end{pgfscope}%
\begin{pgfscope}%
\pgfpathrectangle{\pgfqpoint{0.329460in}{0.284240in}}{\pgfqpoint{1.989680in}{1.989680in}}%
\pgfusepath{clip}%
\pgfsetbuttcap%
\pgfsetroundjoin%
\definecolor{currentfill}{rgb}{0.993248,0.906157,0.143936}%
\pgfsetfillcolor{currentfill}%
\pgfsetlinewidth{0.000000pt}%
\definecolor{currentstroke}{rgb}{0.000000,0.000000,0.000000}%
\pgfsetstrokecolor{currentstroke}%
\pgfsetdash{}{0pt}%
\pgfpathmoveto{\pgfqpoint{1.351188in}{1.731270in}}%
\pgfpathlineto{\pgfqpoint{1.351679in}{1.729591in}}%
\pgfpathlineto{\pgfqpoint{1.352170in}{1.727791in}}%
\pgfpathlineto{\pgfqpoint{1.352661in}{1.725870in}}%
\pgfpathlineto{\pgfqpoint{1.353153in}{1.723828in}}%
\pgfpathlineto{\pgfqpoint{1.352713in}{1.723802in}}%
\pgfpathlineto{\pgfqpoint{1.352271in}{1.723783in}}%
\pgfpathlineto{\pgfqpoint{1.351828in}{1.723771in}}%
\pgfpathlineto{\pgfqpoint{1.351385in}{1.723764in}}%
\pgfpathlineto{\pgfqpoint{1.351335in}{1.725822in}}%
\pgfpathlineto{\pgfqpoint{1.351286in}{1.727759in}}%
\pgfpathlineto{\pgfqpoint{1.351237in}{1.729575in}}%
\pgfpathlineto{\pgfqpoint{1.351188in}{1.731270in}}%
\pgfpathlineto{\pgfqpoint{1.351188in}{1.731270in}}%
\pgfpathlineto{\pgfqpoint{1.351188in}{1.731270in}}%
\pgfpathlineto{\pgfqpoint{1.351188in}{1.731270in}}%
\pgfpathlineto{\pgfqpoint{1.351188in}{1.731270in}}%
\pgfpathclose%
\pgfusepath{fill}%
\end{pgfscope}%
\begin{pgfscope}%
\pgfpathrectangle{\pgfqpoint{0.329460in}{0.284240in}}{\pgfqpoint{1.989680in}{1.989680in}}%
\pgfusepath{clip}%
\pgfsetbuttcap%
\pgfsetroundjoin%
\definecolor{currentfill}{rgb}{0.993248,0.906157,0.143936}%
\pgfsetfillcolor{currentfill}%
\pgfsetlinewidth{0.000000pt}%
\definecolor{currentstroke}{rgb}{0.000000,0.000000,0.000000}%
\pgfsetstrokecolor{currentstroke}%
\pgfsetdash{}{0pt}%
\pgfpathmoveto{\pgfqpoint{1.351188in}{1.731270in}}%
\pgfpathlineto{\pgfqpoint{1.351237in}{1.729575in}}%
\pgfpathlineto{\pgfqpoint{1.351286in}{1.727759in}}%
\pgfpathlineto{\pgfqpoint{1.351335in}{1.725822in}}%
\pgfpathlineto{\pgfqpoint{1.351385in}{1.723764in}}%
\pgfpathlineto{\pgfqpoint{1.350941in}{1.723765in}}%
\pgfpathlineto{\pgfqpoint{1.350498in}{1.723772in}}%
\pgfpathlineto{\pgfqpoint{1.350055in}{1.723785in}}%
\pgfpathlineto{\pgfqpoint{1.349613in}{1.723805in}}%
\pgfpathlineto{\pgfqpoint{1.350007in}{1.725853in}}%
\pgfpathlineto{\pgfqpoint{1.350401in}{1.727779in}}%
\pgfpathlineto{\pgfqpoint{1.350794in}{1.729585in}}%
\pgfpathlineto{\pgfqpoint{1.351188in}{1.731270in}}%
\pgfpathlineto{\pgfqpoint{1.351188in}{1.731270in}}%
\pgfpathlineto{\pgfqpoint{1.351188in}{1.731270in}}%
\pgfpathlineto{\pgfqpoint{1.351188in}{1.731270in}}%
\pgfpathlineto{\pgfqpoint{1.351188in}{1.731270in}}%
\pgfpathclose%
\pgfusepath{fill}%
\end{pgfscope}%
\begin{pgfscope}%
\pgfpathrectangle{\pgfqpoint{0.329460in}{0.284240in}}{\pgfqpoint{1.989680in}{1.989680in}}%
\pgfusepath{clip}%
\pgfsetbuttcap%
\pgfsetroundjoin%
\definecolor{currentfill}{rgb}{0.133743,0.548535,0.553541}%
\pgfsetfillcolor{currentfill}%
\pgfsetlinewidth{0.000000pt}%
\definecolor{currentstroke}{rgb}{0.000000,0.000000,0.000000}%
\pgfsetstrokecolor{currentstroke}%
\pgfsetdash{}{0pt}%
\pgfpathmoveto{\pgfqpoint{1.528553in}{1.260071in}}%
\pgfpathlineto{\pgfqpoint{1.530637in}{1.251694in}}%
\pgfpathlineto{\pgfqpoint{1.532721in}{1.243307in}}%
\pgfpathlineto{\pgfqpoint{1.534805in}{1.234912in}}%
\pgfpathlineto{\pgfqpoint{1.536887in}{1.226513in}}%
\pgfpathlineto{\pgfqpoint{1.529093in}{1.223606in}}%
\pgfpathlineto{\pgfqpoint{1.521111in}{1.220823in}}%
\pgfpathlineto{\pgfqpoint{1.512949in}{1.218168in}}%
\pgfpathlineto{\pgfqpoint{1.504615in}{1.215644in}}%
\pgfpathlineto{\pgfqpoint{1.502892in}{1.224180in}}%
\pgfpathlineto{\pgfqpoint{1.501168in}{1.232710in}}%
\pgfpathlineto{\pgfqpoint{1.499443in}{1.241233in}}%
\pgfpathlineto{\pgfqpoint{1.497718in}{1.249746in}}%
\pgfpathlineto{\pgfqpoint{1.505679in}{1.252144in}}%
\pgfpathlineto{\pgfqpoint{1.513478in}{1.254665in}}%
\pgfpathlineto{\pgfqpoint{1.521105in}{1.257309in}}%
\pgfpathlineto{\pgfqpoint{1.528553in}{1.260071in}}%
\pgfpathclose%
\pgfusepath{fill}%
\end{pgfscope}%
\begin{pgfscope}%
\pgfpathrectangle{\pgfqpoint{0.329460in}{0.284240in}}{\pgfqpoint{1.989680in}{1.989680in}}%
\pgfusepath{clip}%
\pgfsetbuttcap%
\pgfsetroundjoin%
\definecolor{currentfill}{rgb}{0.699415,0.867117,0.175971}%
\pgfsetfillcolor{currentfill}%
\pgfsetlinewidth{0.000000pt}%
\definecolor{currentstroke}{rgb}{0.000000,0.000000,0.000000}%
\pgfsetstrokecolor{currentstroke}%
\pgfsetdash{}{0pt}%
\pgfpathmoveto{\pgfqpoint{1.466574in}{1.631585in}}%
\pgfpathlineto{\pgfqpoint{1.469767in}{1.626790in}}%
\pgfpathlineto{\pgfqpoint{1.472959in}{1.621897in}}%
\pgfpathlineto{\pgfqpoint{1.476149in}{1.616907in}}%
\pgfpathlineto{\pgfqpoint{1.479338in}{1.611822in}}%
\pgfpathlineto{\pgfqpoint{1.477508in}{1.609912in}}%
\pgfpathlineto{\pgfqpoint{1.475550in}{1.608029in}}%
\pgfpathlineto{\pgfqpoint{1.473466in}{1.606177in}}%
\pgfpathlineto{\pgfqpoint{1.471259in}{1.604356in}}%
\pgfpathlineto{\pgfqpoint{1.468267in}{1.609633in}}%
\pgfpathlineto{\pgfqpoint{1.465274in}{1.614815in}}%
\pgfpathlineto{\pgfqpoint{1.462280in}{1.619900in}}%
\pgfpathlineto{\pgfqpoint{1.459285in}{1.624886in}}%
\pgfpathlineto{\pgfqpoint{1.461276in}{1.626520in}}%
\pgfpathlineto{\pgfqpoint{1.463156in}{1.628182in}}%
\pgfpathlineto{\pgfqpoint{1.464922in}{1.629871in}}%
\pgfpathlineto{\pgfqpoint{1.466574in}{1.631585in}}%
\pgfpathclose%
\pgfusepath{fill}%
\end{pgfscope}%
\begin{pgfscope}%
\pgfpathrectangle{\pgfqpoint{0.329460in}{0.284240in}}{\pgfqpoint{1.989680in}{1.989680in}}%
\pgfusepath{clip}%
\pgfsetbuttcap%
\pgfsetroundjoin%
\definecolor{currentfill}{rgb}{0.134692,0.658636,0.517649}%
\pgfsetfillcolor{currentfill}%
\pgfsetlinewidth{0.000000pt}%
\definecolor{currentstroke}{rgb}{0.000000,0.000000,0.000000}%
\pgfsetstrokecolor{currentstroke}%
\pgfsetdash{}{0pt}%
\pgfpathmoveto{\pgfqpoint{1.527433in}{1.369269in}}%
\pgfpathlineto{\pgfqpoint{1.529854in}{1.361348in}}%
\pgfpathlineto{\pgfqpoint{1.532273in}{1.353389in}}%
\pgfpathlineto{\pgfqpoint{1.534691in}{1.345395in}}%
\pgfpathlineto{\pgfqpoint{1.537108in}{1.337368in}}%
\pgfpathlineto{\pgfqpoint{1.531068in}{1.334494in}}%
\pgfpathlineto{\pgfqpoint{1.524840in}{1.331716in}}%
\pgfpathlineto{\pgfqpoint{1.518430in}{1.329035in}}%
\pgfpathlineto{\pgfqpoint{1.511846in}{1.326456in}}%
\pgfpathlineto{\pgfqpoint{1.509754in}{1.334639in}}%
\pgfpathlineto{\pgfqpoint{1.507661in}{1.342789in}}%
\pgfpathlineto{\pgfqpoint{1.505567in}{1.350903in}}%
\pgfpathlineto{\pgfqpoint{1.503472in}{1.358979in}}%
\pgfpathlineto{\pgfqpoint{1.509717in}{1.361411in}}%
\pgfpathlineto{\pgfqpoint{1.515796in}{1.363939in}}%
\pgfpathlineto{\pgfqpoint{1.521704in}{1.366559in}}%
\pgfpathlineto{\pgfqpoint{1.527433in}{1.369269in}}%
\pgfpathclose%
\pgfusepath{fill}%
\end{pgfscope}%
\begin{pgfscope}%
\pgfpathrectangle{\pgfqpoint{0.329460in}{0.284240in}}{\pgfqpoint{1.989680in}{1.989680in}}%
\pgfusepath{clip}%
\pgfsetbuttcap%
\pgfsetroundjoin%
\definecolor{currentfill}{rgb}{0.282327,0.094955,0.417331}%
\pgfsetfillcolor{currentfill}%
\pgfsetlinewidth{0.000000pt}%
\definecolor{currentstroke}{rgb}{0.000000,0.000000,0.000000}%
\pgfsetstrokecolor{currentstroke}%
\pgfsetdash{}{0pt}%
\pgfpathmoveto{\pgfqpoint{1.241583in}{0.852045in}}%
\pgfpathlineto{\pgfqpoint{1.240748in}{0.846058in}}%
\pgfpathlineto{\pgfqpoint{1.239912in}{0.840208in}}%
\pgfpathlineto{\pgfqpoint{1.239074in}{0.834498in}}%
\pgfpathlineto{\pgfqpoint{1.238236in}{0.828931in}}%
\pgfpathlineto{\pgfqpoint{1.223669in}{0.830975in}}%
\pgfpathlineto{\pgfqpoint{1.209243in}{0.833265in}}%
\pgfpathlineto{\pgfqpoint{1.194974in}{0.835797in}}%
\pgfpathlineto{\pgfqpoint{1.180878in}{0.838569in}}%
\pgfpathlineto{\pgfqpoint{1.182140in}{0.844056in}}%
\pgfpathlineto{\pgfqpoint{1.183401in}{0.849687in}}%
\pgfpathlineto{\pgfqpoint{1.184660in}{0.855458in}}%
\pgfpathlineto{\pgfqpoint{1.185918in}{0.861365in}}%
\pgfpathlineto{\pgfqpoint{1.199599in}{0.858684in}}%
\pgfpathlineto{\pgfqpoint{1.213447in}{0.856235in}}%
\pgfpathlineto{\pgfqpoint{1.227447in}{0.854021in}}%
\pgfpathlineto{\pgfqpoint{1.241583in}{0.852045in}}%
\pgfpathclose%
\pgfusepath{fill}%
\end{pgfscope}%
\begin{pgfscope}%
\pgfpathrectangle{\pgfqpoint{0.329460in}{0.284240in}}{\pgfqpoint{1.989680in}{1.989680in}}%
\pgfusepath{clip}%
\pgfsetbuttcap%
\pgfsetroundjoin%
\definecolor{currentfill}{rgb}{0.201239,0.383670,0.554294}%
\pgfsetfillcolor{currentfill}%
\pgfsetlinewidth{0.000000pt}%
\definecolor{currentstroke}{rgb}{0.000000,0.000000,0.000000}%
\pgfsetstrokecolor{currentstroke}%
\pgfsetdash{}{0pt}%
\pgfpathmoveto{\pgfqpoint{0.784384in}{1.034316in}}%
\pgfpathlineto{\pgfqpoint{0.780975in}{1.045214in}}%
\pgfpathlineto{\pgfqpoint{0.777547in}{1.056550in}}%
\pgfpathlineto{\pgfqpoint{0.774100in}{1.068331in}}%
\pgfpathlineto{\pgfqpoint{0.770634in}{1.080567in}}%
\pgfpathlineto{\pgfqpoint{0.759942in}{1.090146in}}%
\pgfpathlineto{\pgfqpoint{0.749887in}{1.099880in}}%
\pgfpathlineto{\pgfqpoint{0.740478in}{1.109757in}}%
\pgfpathlineto{\pgfqpoint{0.731721in}{1.119768in}}%
\pgfpathlineto{\pgfqpoint{0.735397in}{1.107374in}}%
\pgfpathlineto{\pgfqpoint{0.739053in}{1.095431in}}%
\pgfpathlineto{\pgfqpoint{0.742689in}{1.083932in}}%
\pgfpathlineto{\pgfqpoint{0.746305in}{1.072869in}}%
\pgfpathlineto{\pgfqpoint{0.754878in}{1.063023in}}%
\pgfpathlineto{\pgfqpoint{0.764087in}{1.053309in}}%
\pgfpathlineto{\pgfqpoint{0.773925in}{1.043736in}}%
\pgfpathlineto{\pgfqpoint{0.784384in}{1.034316in}}%
\pgfpathclose%
\pgfusepath{fill}%
\end{pgfscope}%
\begin{pgfscope}%
\pgfpathrectangle{\pgfqpoint{0.329460in}{0.284240in}}{\pgfqpoint{1.989680in}{1.989680in}}%
\pgfusepath{clip}%
\pgfsetbuttcap%
\pgfsetroundjoin%
\definecolor{currentfill}{rgb}{0.263663,0.237631,0.518762}%
\pgfsetfillcolor{currentfill}%
\pgfsetlinewidth{0.000000pt}%
\definecolor{currentstroke}{rgb}{0.000000,0.000000,0.000000}%
\pgfsetstrokecolor{currentstroke}%
\pgfsetdash{}{0pt}%
\pgfpathmoveto{\pgfqpoint{1.254857in}{0.963482in}}%
\pgfpathlineto{\pgfqpoint{1.254031in}{0.955791in}}%
\pgfpathlineto{\pgfqpoint{1.253205in}{0.948182in}}%
\pgfpathlineto{\pgfqpoint{1.252379in}{0.940657in}}%
\pgfpathlineto{\pgfqpoint{1.251552in}{0.933219in}}%
\pgfpathlineto{\pgfqpoint{1.238698in}{0.934994in}}%
\pgfpathlineto{\pgfqpoint{1.225967in}{0.936983in}}%
\pgfpathlineto{\pgfqpoint{1.213373in}{0.939182in}}%
\pgfpathlineto{\pgfqpoint{1.200930in}{0.941590in}}%
\pgfpathlineto{\pgfqpoint{1.202176in}{0.948948in}}%
\pgfpathlineto{\pgfqpoint{1.203421in}{0.956395in}}%
\pgfpathlineto{\pgfqpoint{1.204665in}{0.963926in}}%
\pgfpathlineto{\pgfqpoint{1.205909in}{0.971538in}}%
\pgfpathlineto{\pgfqpoint{1.217942in}{0.969220in}}%
\pgfpathlineto{\pgfqpoint{1.230119in}{0.967103in}}%
\pgfpathlineto{\pgfqpoint{1.242429in}{0.965190in}}%
\pgfpathlineto{\pgfqpoint{1.254857in}{0.963482in}}%
\pgfpathclose%
\pgfusepath{fill}%
\end{pgfscope}%
\begin{pgfscope}%
\pgfpathrectangle{\pgfqpoint{0.329460in}{0.284240in}}{\pgfqpoint{1.989680in}{1.989680in}}%
\pgfusepath{clip}%
\pgfsetbuttcap%
\pgfsetroundjoin%
\definecolor{currentfill}{rgb}{0.281477,0.755203,0.432552}%
\pgfsetfillcolor{currentfill}%
\pgfsetlinewidth{0.000000pt}%
\definecolor{currentstroke}{rgb}{0.000000,0.000000,0.000000}%
\pgfsetstrokecolor{currentstroke}%
\pgfsetdash{}{0pt}%
\pgfpathmoveto{\pgfqpoint{1.515910in}{1.470248in}}%
\pgfpathlineto{\pgfqpoint{1.518631in}{1.463109in}}%
\pgfpathlineto{\pgfqpoint{1.521350in}{1.455908in}}%
\pgfpathlineto{\pgfqpoint{1.524068in}{1.448647in}}%
\pgfpathlineto{\pgfqpoint{1.526784in}{1.441329in}}%
\pgfpathlineto{\pgfqpoint{1.522356in}{1.438651in}}%
\pgfpathlineto{\pgfqpoint{1.517751in}{1.436041in}}%
\pgfpathlineto{\pgfqpoint{1.512975in}{1.433503in}}%
\pgfpathlineto{\pgfqpoint{1.508031in}{1.431039in}}%
\pgfpathlineto{\pgfqpoint{1.505600in}{1.438528in}}%
\pgfpathlineto{\pgfqpoint{1.503169in}{1.445960in}}%
\pgfpathlineto{\pgfqpoint{1.500736in}{1.453332in}}%
\pgfpathlineto{\pgfqpoint{1.498302in}{1.460642in}}%
\pgfpathlineto{\pgfqpoint{1.502943in}{1.462942in}}%
\pgfpathlineto{\pgfqpoint{1.507428in}{1.465311in}}%
\pgfpathlineto{\pgfqpoint{1.511751in}{1.467748in}}%
\pgfpathlineto{\pgfqpoint{1.515910in}{1.470248in}}%
\pgfpathclose%
\pgfusepath{fill}%
\end{pgfscope}%
\begin{pgfscope}%
\pgfpathrectangle{\pgfqpoint{0.329460in}{0.284240in}}{\pgfqpoint{1.989680in}{1.989680in}}%
\pgfusepath{clip}%
\pgfsetbuttcap%
\pgfsetroundjoin%
\definecolor{currentfill}{rgb}{0.179019,0.433756,0.557430}%
\pgfsetfillcolor{currentfill}%
\pgfsetlinewidth{0.000000pt}%
\definecolor{currentstroke}{rgb}{0.000000,0.000000,0.000000}%
\pgfsetstrokecolor{currentstroke}%
\pgfsetdash{}{0pt}%
\pgfpathmoveto{\pgfqpoint{1.518388in}{1.147498in}}%
\pgfpathlineto{\pgfqpoint{1.520108in}{1.139037in}}%
\pgfpathlineto{\pgfqpoint{1.521828in}{1.130597in}}%
\pgfpathlineto{\pgfqpoint{1.523548in}{1.122182in}}%
\pgfpathlineto{\pgfqpoint{1.525267in}{1.113794in}}%
\pgfpathlineto{\pgfqpoint{1.515635in}{1.111038in}}%
\pgfpathlineto{\pgfqpoint{1.505828in}{1.108440in}}%
\pgfpathlineto{\pgfqpoint{1.495854in}{1.106001in}}%
\pgfpathlineto{\pgfqpoint{1.485725in}{1.103726in}}%
\pgfpathlineto{\pgfqpoint{1.484395in}{1.112226in}}%
\pgfpathlineto{\pgfqpoint{1.483064in}{1.120754in}}%
\pgfpathlineto{\pgfqpoint{1.481733in}{1.129306in}}%
\pgfpathlineto{\pgfqpoint{1.480401in}{1.137880in}}%
\pgfpathlineto{\pgfqpoint{1.490131in}{1.140053in}}%
\pgfpathlineto{\pgfqpoint{1.499712in}{1.142383in}}%
\pgfpathlineto{\pgfqpoint{1.509135in}{1.144866in}}%
\pgfpathlineto{\pgfqpoint{1.518388in}{1.147498in}}%
\pgfpathclose%
\pgfusepath{fill}%
\end{pgfscope}%
\begin{pgfscope}%
\pgfpathrectangle{\pgfqpoint{0.329460in}{0.284240in}}{\pgfqpoint{1.989680in}{1.989680in}}%
\pgfusepath{clip}%
\pgfsetbuttcap%
\pgfsetroundjoin%
\definecolor{currentfill}{rgb}{0.487026,0.823929,0.312321}%
\pgfsetfillcolor{currentfill}%
\pgfsetlinewidth{0.000000pt}%
\definecolor{currentstroke}{rgb}{0.000000,0.000000,0.000000}%
\pgfsetstrokecolor{currentstroke}%
\pgfsetdash{}{0pt}%
\pgfpathmoveto{\pgfqpoint{1.495149in}{1.558840in}}%
\pgfpathlineto{\pgfqpoint{1.498129in}{1.552760in}}%
\pgfpathlineto{\pgfqpoint{1.501108in}{1.546598in}}%
\pgfpathlineto{\pgfqpoint{1.504085in}{1.540356in}}%
\pgfpathlineto{\pgfqpoint{1.507061in}{1.534035in}}%
\pgfpathlineto{\pgfqpoint{1.504051in}{1.531687in}}%
\pgfpathlineto{\pgfqpoint{1.500884in}{1.529385in}}%
\pgfpathlineto{\pgfqpoint{1.497565in}{1.527131in}}%
\pgfpathlineto{\pgfqpoint{1.494096in}{1.524928in}}%
\pgfpathlineto{\pgfqpoint{1.491364in}{1.531433in}}%
\pgfpathlineto{\pgfqpoint{1.488630in}{1.537858in}}%
\pgfpathlineto{\pgfqpoint{1.485895in}{1.544203in}}%
\pgfpathlineto{\pgfqpoint{1.483159in}{1.550465in}}%
\pgfpathlineto{\pgfqpoint{1.486367in}{1.552491in}}%
\pgfpathlineto{\pgfqpoint{1.489436in}{1.554563in}}%
\pgfpathlineto{\pgfqpoint{1.492364in}{1.556680in}}%
\pgfpathlineto{\pgfqpoint{1.495149in}{1.558840in}}%
\pgfpathclose%
\pgfusepath{fill}%
\end{pgfscope}%
\begin{pgfscope}%
\pgfpathrectangle{\pgfqpoint{0.329460in}{0.284240in}}{\pgfqpoint{1.989680in}{1.989680in}}%
\pgfusepath{clip}%
\pgfsetbuttcap%
\pgfsetroundjoin%
\definecolor{currentfill}{rgb}{0.855810,0.888601,0.097452}%
\pgfsetfillcolor{currentfill}%
\pgfsetlinewidth{0.000000pt}%
\definecolor{currentstroke}{rgb}{0.000000,0.000000,0.000000}%
\pgfsetstrokecolor{currentstroke}%
\pgfsetdash{}{0pt}%
\pgfpathmoveto{\pgfqpoint{1.275184in}{1.680067in}}%
\pgfpathlineto{\pgfqpoint{1.272020in}{1.676490in}}%
\pgfpathlineto{\pgfqpoint{1.268856in}{1.672802in}}%
\pgfpathlineto{\pgfqpoint{1.265693in}{1.669005in}}%
\pgfpathlineto{\pgfqpoint{1.262530in}{1.665099in}}%
\pgfpathlineto{\pgfqpoint{1.261249in}{1.666427in}}%
\pgfpathlineto{\pgfqpoint{1.260059in}{1.667774in}}%
\pgfpathlineto{\pgfqpoint{1.258960in}{1.669137in}}%
\pgfpathlineto{\pgfqpoint{1.257953in}{1.670515in}}%
\pgfpathlineto{\pgfqpoint{1.261276in}{1.674224in}}%
\pgfpathlineto{\pgfqpoint{1.264600in}{1.677825in}}%
\pgfpathlineto{\pgfqpoint{1.267925in}{1.681317in}}%
\pgfpathlineto{\pgfqpoint{1.271250in}{1.684699in}}%
\pgfpathlineto{\pgfqpoint{1.272116in}{1.683520in}}%
\pgfpathlineto{\pgfqpoint{1.273060in}{1.682354in}}%
\pgfpathlineto{\pgfqpoint{1.274084in}{1.681203in}}%
\pgfpathlineto{\pgfqpoint{1.275184in}{1.680067in}}%
\pgfpathclose%
\pgfusepath{fill}%
\end{pgfscope}%
\begin{pgfscope}%
\pgfpathrectangle{\pgfqpoint{0.329460in}{0.284240in}}{\pgfqpoint{1.989680in}{1.989680in}}%
\pgfusepath{clip}%
\pgfsetbuttcap%
\pgfsetroundjoin%
\definecolor{currentfill}{rgb}{0.896320,0.893616,0.096335}%
\pgfsetfillcolor{currentfill}%
\pgfsetlinewidth{0.000000pt}%
\definecolor{currentstroke}{rgb}{0.000000,0.000000,0.000000}%
\pgfsetstrokecolor{currentstroke}%
\pgfsetdash{}{0pt}%
\pgfpathmoveto{\pgfqpoint{1.418401in}{1.697991in}}%
\pgfpathlineto{\pgfqpoint{1.421759in}{1.695101in}}%
\pgfpathlineto{\pgfqpoint{1.425116in}{1.692098in}}%
\pgfpathlineto{\pgfqpoint{1.428472in}{1.688983in}}%
\pgfpathlineto{\pgfqpoint{1.431827in}{1.685757in}}%
\pgfpathlineto{\pgfqpoint{1.431033in}{1.684567in}}%
\pgfpathlineto{\pgfqpoint{1.430158in}{1.683390in}}%
\pgfpathlineto{\pgfqpoint{1.429205in}{1.682226in}}%
\pgfpathlineto{\pgfqpoint{1.428173in}{1.681076in}}%
\pgfpathlineto{\pgfqpoint{1.424967in}{1.684500in}}%
\pgfpathlineto{\pgfqpoint{1.421761in}{1.687812in}}%
\pgfpathlineto{\pgfqpoint{1.418555in}{1.691012in}}%
\pgfpathlineto{\pgfqpoint{1.415347in}{1.694098in}}%
\pgfpathlineto{\pgfqpoint{1.416209in}{1.695054in}}%
\pgfpathlineto{\pgfqpoint{1.417006in}{1.696022in}}%
\pgfpathlineto{\pgfqpoint{1.417737in}{1.697001in}}%
\pgfpathlineto{\pgfqpoint{1.418401in}{1.697991in}}%
\pgfpathclose%
\pgfusepath{fill}%
\end{pgfscope}%
\begin{pgfscope}%
\pgfpathrectangle{\pgfqpoint{0.329460in}{0.284240in}}{\pgfqpoint{1.989680in}{1.989680in}}%
\pgfusepath{clip}%
\pgfsetbuttcap%
\pgfsetroundjoin%
\definecolor{currentfill}{rgb}{0.268510,0.009605,0.335427}%
\pgfsetfillcolor{currentfill}%
\pgfsetlinewidth{0.000000pt}%
\definecolor{currentstroke}{rgb}{0.000000,0.000000,0.000000}%
\pgfsetstrokecolor{currentstroke}%
\pgfsetdash{}{0pt}%
\pgfpathmoveto{\pgfqpoint{1.165590in}{0.785331in}}%
\pgfpathlineto{\pgfqpoint{1.164302in}{0.782066in}}%
\pgfpathlineto{\pgfqpoint{1.163011in}{0.778999in}}%
\pgfpathlineto{\pgfqpoint{1.161718in}{0.776136in}}%
\pgfpathlineto{\pgfqpoint{1.160422in}{0.773481in}}%
\pgfpathlineto{\pgfqpoint{1.144854in}{0.776884in}}%
\pgfpathlineto{\pgfqpoint{1.129517in}{0.780550in}}%
\pgfpathlineto{\pgfqpoint{1.114426in}{0.784474in}}%
\pgfpathlineto{\pgfqpoint{1.099598in}{0.788651in}}%
\pgfpathlineto{\pgfqpoint{1.101304in}{0.791198in}}%
\pgfpathlineto{\pgfqpoint{1.103006in}{0.793952in}}%
\pgfpathlineto{\pgfqpoint{1.104704in}{0.796909in}}%
\pgfpathlineto{\pgfqpoint{1.106399in}{0.800065in}}%
\pgfpathlineto{\pgfqpoint{1.120830in}{0.796008in}}%
\pgfpathlineto{\pgfqpoint{1.135516in}{0.792196in}}%
\pgfpathlineto{\pgfqpoint{1.150441in}{0.788636in}}%
\pgfpathlineto{\pgfqpoint{1.165590in}{0.785331in}}%
\pgfpathclose%
\pgfusepath{fill}%
\end{pgfscope}%
\begin{pgfscope}%
\pgfpathrectangle{\pgfqpoint{0.329460in}{0.284240in}}{\pgfqpoint{1.989680in}{1.989680in}}%
\pgfusepath{clip}%
\pgfsetbuttcap%
\pgfsetroundjoin%
\definecolor{currentfill}{rgb}{0.974417,0.903590,0.130215}%
\pgfsetfillcolor{currentfill}%
\pgfsetlinewidth{0.000000pt}%
\definecolor{currentstroke}{rgb}{0.000000,0.000000,0.000000}%
\pgfsetstrokecolor{currentstroke}%
\pgfsetdash{}{0pt}%
\pgfpathmoveto{\pgfqpoint{1.378075in}{1.723645in}}%
\pgfpathlineto{\pgfqpoint{1.381437in}{1.722154in}}%
\pgfpathlineto{\pgfqpoint{1.384798in}{1.720544in}}%
\pgfpathlineto{\pgfqpoint{1.388160in}{1.718815in}}%
\pgfpathlineto{\pgfqpoint{1.391522in}{1.716968in}}%
\pgfpathlineto{\pgfqpoint{1.391120in}{1.716377in}}%
\pgfpathlineto{\pgfqpoint{1.390679in}{1.715791in}}%
\pgfpathlineto{\pgfqpoint{1.390198in}{1.715212in}}%
\pgfpathlineto{\pgfqpoint{1.389679in}{1.714641in}}%
\pgfpathlineto{\pgfqpoint{1.386470in}{1.716682in}}%
\pgfpathlineto{\pgfqpoint{1.383260in}{1.718605in}}%
\pgfpathlineto{\pgfqpoint{1.380052in}{1.720410in}}%
\pgfpathlineto{\pgfqpoint{1.376843in}{1.722095in}}%
\pgfpathlineto{\pgfqpoint{1.377190in}{1.722476in}}%
\pgfpathlineto{\pgfqpoint{1.377511in}{1.722861in}}%
\pgfpathlineto{\pgfqpoint{1.377806in}{1.723251in}}%
\pgfpathlineto{\pgfqpoint{1.378075in}{1.723645in}}%
\pgfpathclose%
\pgfusepath{fill}%
\end{pgfscope}%
\begin{pgfscope}%
\pgfpathrectangle{\pgfqpoint{0.329460in}{0.284240in}}{\pgfqpoint{1.989680in}{1.989680in}}%
\pgfusepath{clip}%
\pgfsetbuttcap%
\pgfsetroundjoin%
\definecolor{currentfill}{rgb}{0.993248,0.906157,0.143936}%
\pgfsetfillcolor{currentfill}%
\pgfsetlinewidth{0.000000pt}%
\definecolor{currentstroke}{rgb}{0.000000,0.000000,0.000000}%
\pgfsetstrokecolor{currentstroke}%
\pgfsetdash{}{0pt}%
\pgfpathmoveto{\pgfqpoint{1.364011in}{1.727642in}}%
\pgfpathlineto{\pgfqpoint{1.367219in}{1.726435in}}%
\pgfpathlineto{\pgfqpoint{1.370426in}{1.725108in}}%
\pgfpathlineto{\pgfqpoint{1.373634in}{1.723661in}}%
\pgfpathlineto{\pgfqpoint{1.376843in}{1.722095in}}%
\pgfpathlineto{\pgfqpoint{1.376470in}{1.721720in}}%
\pgfpathlineto{\pgfqpoint{1.376072in}{1.721351in}}%
\pgfpathlineto{\pgfqpoint{1.375649in}{1.720987in}}%
\pgfpathlineto{\pgfqpoint{1.375201in}{1.720630in}}%
\pgfpathlineto{\pgfqpoint{1.372197in}{1.722380in}}%
\pgfpathlineto{\pgfqpoint{1.369194in}{1.724010in}}%
\pgfpathlineto{\pgfqpoint{1.366191in}{1.725520in}}%
\pgfpathlineto{\pgfqpoint{1.363189in}{1.726911in}}%
\pgfpathlineto{\pgfqpoint{1.363413in}{1.727089in}}%
\pgfpathlineto{\pgfqpoint{1.363625in}{1.727270in}}%
\pgfpathlineto{\pgfqpoint{1.363824in}{1.727455in}}%
\pgfpathlineto{\pgfqpoint{1.364011in}{1.727642in}}%
\pgfpathclose%
\pgfusepath{fill}%
\end{pgfscope}%
\begin{pgfscope}%
\pgfpathrectangle{\pgfqpoint{0.329460in}{0.284240in}}{\pgfqpoint{1.989680in}{1.989680in}}%
\pgfusepath{clip}%
\pgfsetbuttcap%
\pgfsetroundjoin%
\definecolor{currentfill}{rgb}{0.993248,0.906157,0.143936}%
\pgfsetfillcolor{currentfill}%
\pgfsetlinewidth{0.000000pt}%
\definecolor{currentstroke}{rgb}{0.000000,0.000000,0.000000}%
\pgfsetstrokecolor{currentstroke}%
\pgfsetdash{}{0pt}%
\pgfpathmoveto{\pgfqpoint{1.339395in}{1.726755in}}%
\pgfpathlineto{\pgfqpoint{1.336445in}{1.725326in}}%
\pgfpathlineto{\pgfqpoint{1.333495in}{1.723776in}}%
\pgfpathlineto{\pgfqpoint{1.330544in}{1.722107in}}%
\pgfpathlineto{\pgfqpoint{1.327592in}{1.720318in}}%
\pgfpathlineto{\pgfqpoint{1.327123in}{1.720669in}}%
\pgfpathlineto{\pgfqpoint{1.326678in}{1.721027in}}%
\pgfpathlineto{\pgfqpoint{1.326258in}{1.721391in}}%
\pgfpathlineto{\pgfqpoint{1.325862in}{1.721762in}}%
\pgfpathlineto{\pgfqpoint{1.329030in}{1.723370in}}%
\pgfpathlineto{\pgfqpoint{1.332197in}{1.724858in}}%
\pgfpathlineto{\pgfqpoint{1.335363in}{1.726227in}}%
\pgfpathlineto{\pgfqpoint{1.338529in}{1.727476in}}%
\pgfpathlineto{\pgfqpoint{1.338727in}{1.727291in}}%
\pgfpathlineto{\pgfqpoint{1.338938in}{1.727109in}}%
\pgfpathlineto{\pgfqpoint{1.339161in}{1.726930in}}%
\pgfpathlineto{\pgfqpoint{1.339395in}{1.726755in}}%
\pgfpathclose%
\pgfusepath{fill}%
\end{pgfscope}%
\begin{pgfscope}%
\pgfpathrectangle{\pgfqpoint{0.329460in}{0.284240in}}{\pgfqpoint{1.989680in}{1.989680in}}%
\pgfusepath{clip}%
\pgfsetbuttcap%
\pgfsetroundjoin%
\definecolor{currentfill}{rgb}{0.277941,0.056324,0.381191}%
\pgfsetfillcolor{currentfill}%
\pgfsetlinewidth{0.000000pt}%
\definecolor{currentstroke}{rgb}{0.000000,0.000000,0.000000}%
\pgfsetstrokecolor{currentstroke}%
\pgfsetdash{}{0pt}%
\pgfpathmoveto{\pgfqpoint{1.009219in}{0.801108in}}%
\pgfpathlineto{\pgfqpoint{1.007040in}{0.802924in}}%
\pgfpathlineto{\pgfqpoint{1.004854in}{0.805033in}}%
\pgfpathlineto{\pgfqpoint{1.002660in}{0.807439in}}%
\pgfpathlineto{\pgfqpoint{1.000458in}{0.810149in}}%
\pgfpathlineto{\pgfqpoint{0.985373in}{0.816255in}}%
\pgfpathlineto{\pgfqpoint{0.970695in}{0.822608in}}%
\pgfpathlineto{\pgfqpoint{0.956438in}{0.829201in}}%
\pgfpathlineto{\pgfqpoint{0.942617in}{0.836026in}}%
\pgfpathlineto{\pgfqpoint{0.945172in}{0.833164in}}%
\pgfpathlineto{\pgfqpoint{0.947718in}{0.830605in}}%
\pgfpathlineto{\pgfqpoint{0.950255in}{0.828344in}}%
\pgfpathlineto{\pgfqpoint{0.952784in}{0.826375in}}%
\pgfpathlineto{\pgfqpoint{0.966271in}{0.819710in}}%
\pgfpathlineto{\pgfqpoint{0.980181in}{0.813273in}}%
\pgfpathlineto{\pgfqpoint{0.994502in}{0.807069in}}%
\pgfpathlineto{\pgfqpoint{1.009219in}{0.801108in}}%
\pgfpathclose%
\pgfusepath{fill}%
\end{pgfscope}%
\begin{pgfscope}%
\pgfpathrectangle{\pgfqpoint{0.329460in}{0.284240in}}{\pgfqpoint{1.989680in}{1.989680in}}%
\pgfusepath{clip}%
\pgfsetbuttcap%
\pgfsetroundjoin%
\definecolor{currentfill}{rgb}{0.974417,0.903590,0.130215}%
\pgfsetfillcolor{currentfill}%
\pgfsetlinewidth{0.000000pt}%
\definecolor{currentstroke}{rgb}{0.000000,0.000000,0.000000}%
\pgfsetstrokecolor{currentstroke}%
\pgfsetdash{}{0pt}%
\pgfpathmoveto{\pgfqpoint{1.325862in}{1.721762in}}%
\pgfpathlineto{\pgfqpoint{1.322695in}{1.720034in}}%
\pgfpathlineto{\pgfqpoint{1.319527in}{1.718188in}}%
\pgfpathlineto{\pgfqpoint{1.316359in}{1.716223in}}%
\pgfpathlineto{\pgfqpoint{1.313191in}{1.714139in}}%
\pgfpathlineto{\pgfqpoint{1.312637in}{1.714704in}}%
\pgfpathlineto{\pgfqpoint{1.312121in}{1.715276in}}%
\pgfpathlineto{\pgfqpoint{1.311645in}{1.715856in}}%
\pgfpathlineto{\pgfqpoint{1.311208in}{1.716442in}}%
\pgfpathlineto{\pgfqpoint{1.314540in}{1.718333in}}%
\pgfpathlineto{\pgfqpoint{1.317873in}{1.720106in}}%
\pgfpathlineto{\pgfqpoint{1.321205in}{1.721760in}}%
\pgfpathlineto{\pgfqpoint{1.324538in}{1.723295in}}%
\pgfpathlineto{\pgfqpoint{1.324830in}{1.722904in}}%
\pgfpathlineto{\pgfqpoint{1.325148in}{1.722518in}}%
\pgfpathlineto{\pgfqpoint{1.325492in}{1.722137in}}%
\pgfpathlineto{\pgfqpoint{1.325862in}{1.721762in}}%
\pgfpathclose%
\pgfusepath{fill}%
\end{pgfscope}%
\begin{pgfscope}%
\pgfpathrectangle{\pgfqpoint{0.329460in}{0.284240in}}{\pgfqpoint{1.989680in}{1.989680in}}%
\pgfusepath{clip}%
\pgfsetbuttcap%
\pgfsetroundjoin%
\definecolor{currentfill}{rgb}{0.935904,0.898570,0.108131}%
\pgfsetfillcolor{currentfill}%
\pgfsetlinewidth{0.000000pt}%
\definecolor{currentstroke}{rgb}{0.000000,0.000000,0.000000}%
\pgfsetstrokecolor{currentstroke}%
\pgfsetdash{}{0pt}%
\pgfpathmoveto{\pgfqpoint{1.404965in}{1.708407in}}%
\pgfpathlineto{\pgfqpoint{1.408325in}{1.705976in}}%
\pgfpathlineto{\pgfqpoint{1.411684in}{1.703429in}}%
\pgfpathlineto{\pgfqpoint{1.415043in}{1.700767in}}%
\pgfpathlineto{\pgfqpoint{1.418401in}{1.697991in}}%
\pgfpathlineto{\pgfqpoint{1.417737in}{1.697001in}}%
\pgfpathlineto{\pgfqpoint{1.417006in}{1.696022in}}%
\pgfpathlineto{\pgfqpoint{1.416209in}{1.695054in}}%
\pgfpathlineto{\pgfqpoint{1.415347in}{1.694098in}}%
\pgfpathlineto{\pgfqpoint{1.412140in}{1.697071in}}%
\pgfpathlineto{\pgfqpoint{1.408932in}{1.699929in}}%
\pgfpathlineto{\pgfqpoint{1.405723in}{1.702672in}}%
\pgfpathlineto{\pgfqpoint{1.402515in}{1.705299in}}%
\pgfpathlineto{\pgfqpoint{1.403206in}{1.706062in}}%
\pgfpathlineto{\pgfqpoint{1.403845in}{1.706835in}}%
\pgfpathlineto{\pgfqpoint{1.404432in}{1.707617in}}%
\pgfpathlineto{\pgfqpoint{1.404965in}{1.708407in}}%
\pgfpathclose%
\pgfusepath{fill}%
\end{pgfscope}%
\begin{pgfscope}%
\pgfpathrectangle{\pgfqpoint{0.329460in}{0.284240in}}{\pgfqpoint{1.989680in}{1.989680in}}%
\pgfusepath{clip}%
\pgfsetbuttcap%
\pgfsetroundjoin%
\definecolor{currentfill}{rgb}{0.955300,0.901065,0.118128}%
\pgfsetfillcolor{currentfill}%
\pgfsetlinewidth{0.000000pt}%
\definecolor{currentstroke}{rgb}{0.000000,0.000000,0.000000}%
\pgfsetstrokecolor{currentstroke}%
\pgfsetdash{}{0pt}%
\pgfpathmoveto{\pgfqpoint{1.391522in}{1.716968in}}%
\pgfpathlineto{\pgfqpoint{1.394883in}{1.715004in}}%
\pgfpathlineto{\pgfqpoint{1.398244in}{1.712922in}}%
\pgfpathlineto{\pgfqpoint{1.401605in}{1.710723in}}%
\pgfpathlineto{\pgfqpoint{1.404965in}{1.708407in}}%
\pgfpathlineto{\pgfqpoint{1.404432in}{1.707617in}}%
\pgfpathlineto{\pgfqpoint{1.403845in}{1.706835in}}%
\pgfpathlineto{\pgfqpoint{1.403206in}{1.706062in}}%
\pgfpathlineto{\pgfqpoint{1.402515in}{1.705299in}}%
\pgfpathlineto{\pgfqpoint{1.399306in}{1.707810in}}%
\pgfpathlineto{\pgfqpoint{1.396097in}{1.710204in}}%
\pgfpathlineto{\pgfqpoint{1.392888in}{1.712481in}}%
\pgfpathlineto{\pgfqpoint{1.389679in}{1.714641in}}%
\pgfpathlineto{\pgfqpoint{1.390198in}{1.715212in}}%
\pgfpathlineto{\pgfqpoint{1.390679in}{1.715791in}}%
\pgfpathlineto{\pgfqpoint{1.391120in}{1.716377in}}%
\pgfpathlineto{\pgfqpoint{1.391522in}{1.716968in}}%
\pgfpathclose%
\pgfusepath{fill}%
\end{pgfscope}%
\begin{pgfscope}%
\pgfpathrectangle{\pgfqpoint{0.329460in}{0.284240in}}{\pgfqpoint{1.989680in}{1.989680in}}%
\pgfusepath{clip}%
\pgfsetbuttcap%
\pgfsetroundjoin%
\definecolor{currentfill}{rgb}{0.248629,0.278775,0.534556}%
\pgfsetfillcolor{currentfill}%
\pgfsetlinewidth{0.000000pt}%
\definecolor{currentstroke}{rgb}{0.000000,0.000000,0.000000}%
\pgfsetstrokecolor{currentstroke}%
\pgfsetdash{}{0pt}%
\pgfpathmoveto{\pgfqpoint{1.258158in}{0.994993in}}%
\pgfpathlineto{\pgfqpoint{1.257333in}{0.987009in}}%
\pgfpathlineto{\pgfqpoint{1.256508in}{0.979094in}}%
\pgfpathlineto{\pgfqpoint{1.255683in}{0.971251in}}%
\pgfpathlineto{\pgfqpoint{1.254857in}{0.963482in}}%
\pgfpathlineto{\pgfqpoint{1.242429in}{0.965190in}}%
\pgfpathlineto{\pgfqpoint{1.230119in}{0.967103in}}%
\pgfpathlineto{\pgfqpoint{1.217942in}{0.969220in}}%
\pgfpathlineto{\pgfqpoint{1.205909in}{0.971538in}}%
\pgfpathlineto{\pgfqpoint{1.207153in}{0.979228in}}%
\pgfpathlineto{\pgfqpoint{1.208396in}{0.986993in}}%
\pgfpathlineto{\pgfqpoint{1.209638in}{0.994830in}}%
\pgfpathlineto{\pgfqpoint{1.210881in}{1.002736in}}%
\pgfpathlineto{\pgfqpoint{1.222503in}{1.000509in}}%
\pgfpathlineto{\pgfqpoint{1.234265in}{0.998474in}}%
\pgfpathlineto{\pgfqpoint{1.246154in}{0.996635in}}%
\pgfpathlineto{\pgfqpoint{1.258158in}{0.994993in}}%
\pgfpathclose%
\pgfusepath{fill}%
\end{pgfscope}%
\begin{pgfscope}%
\pgfpathrectangle{\pgfqpoint{0.329460in}{0.284240in}}{\pgfqpoint{1.989680in}{1.989680in}}%
\pgfusepath{clip}%
\pgfsetbuttcap%
\pgfsetroundjoin%
\definecolor{currentfill}{rgb}{0.231674,0.318106,0.544834}%
\pgfsetfillcolor{currentfill}%
\pgfsetlinewidth{0.000000pt}%
\definecolor{currentstroke}{rgb}{0.000000,0.000000,0.000000}%
\pgfsetstrokecolor{currentstroke}%
\pgfsetdash{}{0pt}%
\pgfpathmoveto{\pgfqpoint{1.496372in}{1.037037in}}%
\pgfpathlineto{\pgfqpoint{1.497703in}{1.028909in}}%
\pgfpathlineto{\pgfqpoint{1.499034in}{1.020838in}}%
\pgfpathlineto{\pgfqpoint{1.500366in}{1.012826in}}%
\pgfpathlineto{\pgfqpoint{1.501698in}{1.004876in}}%
\pgfpathlineto{\pgfqpoint{1.490210in}{1.002479in}}%
\pgfpathlineto{\pgfqpoint{1.478572in}{1.000273in}}%
\pgfpathlineto{\pgfqpoint{1.466795in}{0.998260in}}%
\pgfpathlineto{\pgfqpoint{1.454893in}{0.996443in}}%
\pgfpathlineto{\pgfqpoint{1.453974in}{1.004477in}}%
\pgfpathlineto{\pgfqpoint{1.453055in}{1.012574in}}%
\pgfpathlineto{\pgfqpoint{1.452136in}{1.020730in}}%
\pgfpathlineto{\pgfqpoint{1.451218in}{1.028942in}}%
\pgfpathlineto{\pgfqpoint{1.462700in}{1.030686in}}%
\pgfpathlineto{\pgfqpoint{1.474061in}{1.032619in}}%
\pgfpathlineto{\pgfqpoint{1.485289in}{1.034736in}}%
\pgfpathlineto{\pgfqpoint{1.496372in}{1.037037in}}%
\pgfpathclose%
\pgfusepath{fill}%
\end{pgfscope}%
\begin{pgfscope}%
\pgfpathrectangle{\pgfqpoint{0.329460in}{0.284240in}}{\pgfqpoint{1.989680in}{1.989680in}}%
\pgfusepath{clip}%
\pgfsetbuttcap%
\pgfsetroundjoin%
\definecolor{currentfill}{rgb}{0.896320,0.893616,0.096335}%
\pgfsetfillcolor{currentfill}%
\pgfsetlinewidth{0.000000pt}%
\definecolor{currentstroke}{rgb}{0.000000,0.000000,0.000000}%
\pgfsetstrokecolor{currentstroke}%
\pgfsetdash{}{0pt}%
\pgfpathmoveto{\pgfqpoint{1.287848in}{1.693259in}}%
\pgfpathlineto{\pgfqpoint{1.284681in}{1.690131in}}%
\pgfpathlineto{\pgfqpoint{1.281515in}{1.686888in}}%
\pgfpathlineto{\pgfqpoint{1.278349in}{1.683534in}}%
\pgfpathlineto{\pgfqpoint{1.275184in}{1.680067in}}%
\pgfpathlineto{\pgfqpoint{1.274084in}{1.681203in}}%
\pgfpathlineto{\pgfqpoint{1.273060in}{1.682354in}}%
\pgfpathlineto{\pgfqpoint{1.272116in}{1.683520in}}%
\pgfpathlineto{\pgfqpoint{1.271250in}{1.684699in}}%
\pgfpathlineto{\pgfqpoint{1.274577in}{1.687970in}}%
\pgfpathlineto{\pgfqpoint{1.277904in}{1.691129in}}%
\pgfpathlineto{\pgfqpoint{1.281232in}{1.694177in}}%
\pgfpathlineto{\pgfqpoint{1.284561in}{1.697111in}}%
\pgfpathlineto{\pgfqpoint{1.285284in}{1.696130in}}%
\pgfpathlineto{\pgfqpoint{1.286074in}{1.695161in}}%
\pgfpathlineto{\pgfqpoint{1.286929in}{1.694204in}}%
\pgfpathlineto{\pgfqpoint{1.287848in}{1.693259in}}%
\pgfpathclose%
\pgfusepath{fill}%
\end{pgfscope}%
\begin{pgfscope}%
\pgfpathrectangle{\pgfqpoint{0.329460in}{0.284240in}}{\pgfqpoint{1.989680in}{1.989680in}}%
\pgfusepath{clip}%
\pgfsetbuttcap%
\pgfsetroundjoin%
\definecolor{currentfill}{rgb}{0.699415,0.867117,0.175971}%
\pgfsetfillcolor{currentfill}%
\pgfsetlinewidth{0.000000pt}%
\definecolor{currentstroke}{rgb}{0.000000,0.000000,0.000000}%
\pgfsetstrokecolor{currentstroke}%
\pgfsetdash{}{0pt}%
\pgfpathmoveto{\pgfqpoint{1.244951in}{1.623459in}}%
\pgfpathlineto{\pgfqpoint{1.242007in}{1.618432in}}%
\pgfpathlineto{\pgfqpoint{1.239064in}{1.613307in}}%
\pgfpathlineto{\pgfqpoint{1.236122in}{1.608084in}}%
\pgfpathlineto{\pgfqpoint{1.233181in}{1.602765in}}%
\pgfpathlineto{\pgfqpoint{1.230865in}{1.604557in}}%
\pgfpathlineto{\pgfqpoint{1.228671in}{1.606381in}}%
\pgfpathlineto{\pgfqpoint{1.226601in}{1.608237in}}%
\pgfpathlineto{\pgfqpoint{1.224657in}{1.610123in}}%
\pgfpathlineto{\pgfqpoint{1.227807in}{1.615252in}}%
\pgfpathlineto{\pgfqpoint{1.230958in}{1.620285in}}%
\pgfpathlineto{\pgfqpoint{1.234110in}{1.625222in}}%
\pgfpathlineto{\pgfqpoint{1.237264in}{1.630060in}}%
\pgfpathlineto{\pgfqpoint{1.239017in}{1.628368in}}%
\pgfpathlineto{\pgfqpoint{1.240884in}{1.626703in}}%
\pgfpathlineto{\pgfqpoint{1.242863in}{1.625066in}}%
\pgfpathlineto{\pgfqpoint{1.244951in}{1.623459in}}%
\pgfpathclose%
\pgfusepath{fill}%
\end{pgfscope}%
\begin{pgfscope}%
\pgfpathrectangle{\pgfqpoint{0.329460in}{0.284240in}}{\pgfqpoint{1.989680in}{1.989680in}}%
\pgfusepath{clip}%
\pgfsetbuttcap%
\pgfsetroundjoin%
\definecolor{currentfill}{rgb}{0.993248,0.906157,0.143936}%
\pgfsetfillcolor{currentfill}%
\pgfsetlinewidth{0.000000pt}%
\definecolor{currentstroke}{rgb}{0.000000,0.000000,0.000000}%
\pgfsetstrokecolor{currentstroke}%
\pgfsetdash{}{0pt}%
\pgfpathmoveto{\pgfqpoint{1.363189in}{1.726911in}}%
\pgfpathlineto{\pgfqpoint{1.366191in}{1.725520in}}%
\pgfpathlineto{\pgfqpoint{1.369194in}{1.724010in}}%
\pgfpathlineto{\pgfqpoint{1.372197in}{1.722380in}}%
\pgfpathlineto{\pgfqpoint{1.375201in}{1.720630in}}%
\pgfpathlineto{\pgfqpoint{1.374730in}{1.720280in}}%
\pgfpathlineto{\pgfqpoint{1.374234in}{1.719936in}}%
\pgfpathlineto{\pgfqpoint{1.373716in}{1.719601in}}%
\pgfpathlineto{\pgfqpoint{1.373175in}{1.719273in}}%
\pgfpathlineto{\pgfqpoint{1.370424in}{1.721193in}}%
\pgfpathlineto{\pgfqpoint{1.367674in}{1.722993in}}%
\pgfpathlineto{\pgfqpoint{1.364924in}{1.724673in}}%
\pgfpathlineto{\pgfqpoint{1.362175in}{1.726233in}}%
\pgfpathlineto{\pgfqpoint{1.362446in}{1.726397in}}%
\pgfpathlineto{\pgfqpoint{1.362705in}{1.726564in}}%
\pgfpathlineto{\pgfqpoint{1.362953in}{1.726736in}}%
\pgfpathlineto{\pgfqpoint{1.363189in}{1.726911in}}%
\pgfpathclose%
\pgfusepath{fill}%
\end{pgfscope}%
\begin{pgfscope}%
\pgfpathrectangle{\pgfqpoint{0.329460in}{0.284240in}}{\pgfqpoint{1.989680in}{1.989680in}}%
\pgfusepath{clip}%
\pgfsetbuttcap%
\pgfsetroundjoin%
\definecolor{currentfill}{rgb}{0.955300,0.901065,0.118128}%
\pgfsetfillcolor{currentfill}%
\pgfsetlinewidth{0.000000pt}%
\definecolor{currentstroke}{rgb}{0.000000,0.000000,0.000000}%
\pgfsetstrokecolor{currentstroke}%
\pgfsetdash{}{0pt}%
\pgfpathmoveto{\pgfqpoint{1.313191in}{1.714139in}}%
\pgfpathlineto{\pgfqpoint{1.310023in}{1.711938in}}%
\pgfpathlineto{\pgfqpoint{1.306854in}{1.709619in}}%
\pgfpathlineto{\pgfqpoint{1.303686in}{1.707182in}}%
\pgfpathlineto{\pgfqpoint{1.300518in}{1.704629in}}%
\pgfpathlineto{\pgfqpoint{1.299781in}{1.705383in}}%
\pgfpathlineto{\pgfqpoint{1.299095in}{1.706148in}}%
\pgfpathlineto{\pgfqpoint{1.298462in}{1.706922in}}%
\pgfpathlineto{\pgfqpoint{1.297881in}{1.707705in}}%
\pgfpathlineto{\pgfqpoint{1.301213in}{1.710064in}}%
\pgfpathlineto{\pgfqpoint{1.304544in}{1.712307in}}%
\pgfpathlineto{\pgfqpoint{1.307876in}{1.714434in}}%
\pgfpathlineto{\pgfqpoint{1.311208in}{1.716442in}}%
\pgfpathlineto{\pgfqpoint{1.311645in}{1.715856in}}%
\pgfpathlineto{\pgfqpoint{1.312121in}{1.715276in}}%
\pgfpathlineto{\pgfqpoint{1.312637in}{1.714704in}}%
\pgfpathlineto{\pgfqpoint{1.313191in}{1.714139in}}%
\pgfpathclose%
\pgfusepath{fill}%
\end{pgfscope}%
\begin{pgfscope}%
\pgfpathrectangle{\pgfqpoint{0.329460in}{0.284240in}}{\pgfqpoint{1.989680in}{1.989680in}}%
\pgfusepath{clip}%
\pgfsetbuttcap%
\pgfsetroundjoin%
\definecolor{currentfill}{rgb}{0.279566,0.067836,0.391917}%
\pgfsetfillcolor{currentfill}%
\pgfsetlinewidth{0.000000pt}%
\definecolor{currentstroke}{rgb}{0.000000,0.000000,0.000000}%
\pgfsetstrokecolor{currentstroke}%
\pgfsetdash{}{0pt}%
\pgfpathmoveto{\pgfqpoint{1.238236in}{0.828931in}}%
\pgfpathlineto{\pgfqpoint{1.237397in}{0.823512in}}%
\pgfpathlineto{\pgfqpoint{1.236556in}{0.818244in}}%
\pgfpathlineto{\pgfqpoint{1.235715in}{0.813132in}}%
\pgfpathlineto{\pgfqpoint{1.234872in}{0.808178in}}%
\pgfpathlineto{\pgfqpoint{1.219872in}{0.810289in}}%
\pgfpathlineto{\pgfqpoint{1.205018in}{0.812654in}}%
\pgfpathlineto{\pgfqpoint{1.190326in}{0.815270in}}%
\pgfpathlineto{\pgfqpoint{1.175813in}{0.818134in}}%
\pgfpathlineto{\pgfqpoint{1.177082in}{0.823008in}}%
\pgfpathlineto{\pgfqpoint{1.178349in}{0.828041in}}%
\pgfpathlineto{\pgfqpoint{1.179614in}{0.833230in}}%
\pgfpathlineto{\pgfqpoint{1.180878in}{0.838569in}}%
\pgfpathlineto{\pgfqpoint{1.194974in}{0.835797in}}%
\pgfpathlineto{\pgfqpoint{1.209243in}{0.833265in}}%
\pgfpathlineto{\pgfqpoint{1.223669in}{0.830975in}}%
\pgfpathlineto{\pgfqpoint{1.238236in}{0.828931in}}%
\pgfpathclose%
\pgfusepath{fill}%
\end{pgfscope}%
\begin{pgfscope}%
\pgfpathrectangle{\pgfqpoint{0.329460in}{0.284240in}}{\pgfqpoint{1.989680in}{1.989680in}}%
\pgfusepath{clip}%
\pgfsetbuttcap%
\pgfsetroundjoin%
\definecolor{currentfill}{rgb}{0.993248,0.906157,0.143936}%
\pgfsetfillcolor{currentfill}%
\pgfsetlinewidth{0.000000pt}%
\definecolor{currentstroke}{rgb}{0.000000,0.000000,0.000000}%
\pgfsetstrokecolor{currentstroke}%
\pgfsetdash{}{0pt}%
\pgfpathmoveto{\pgfqpoint{1.340449in}{1.726091in}}%
\pgfpathlineto{\pgfqpoint{1.337763in}{1.724495in}}%
\pgfpathlineto{\pgfqpoint{1.335075in}{1.722779in}}%
\pgfpathlineto{\pgfqpoint{1.332387in}{1.720944in}}%
\pgfpathlineto{\pgfqpoint{1.329699in}{1.718988in}}%
\pgfpathlineto{\pgfqpoint{1.329139in}{1.719309in}}%
\pgfpathlineto{\pgfqpoint{1.328600in}{1.719638in}}%
\pgfpathlineto{\pgfqpoint{1.328085in}{1.719974in}}%
\pgfpathlineto{\pgfqpoint{1.327592in}{1.720318in}}%
\pgfpathlineto{\pgfqpoint{1.330544in}{1.722107in}}%
\pgfpathlineto{\pgfqpoint{1.333495in}{1.723776in}}%
\pgfpathlineto{\pgfqpoint{1.336445in}{1.725326in}}%
\pgfpathlineto{\pgfqpoint{1.339395in}{1.726755in}}%
\pgfpathlineto{\pgfqpoint{1.339642in}{1.726583in}}%
\pgfpathlineto{\pgfqpoint{1.339900in}{1.726415in}}%
\pgfpathlineto{\pgfqpoint{1.340169in}{1.726251in}}%
\pgfpathlineto{\pgfqpoint{1.340449in}{1.726091in}}%
\pgfpathclose%
\pgfusepath{fill}%
\end{pgfscope}%
\begin{pgfscope}%
\pgfpathrectangle{\pgfqpoint{0.329460in}{0.284240in}}{\pgfqpoint{1.989680in}{1.989680in}}%
\pgfusepath{clip}%
\pgfsetbuttcap%
\pgfsetroundjoin%
\definecolor{currentfill}{rgb}{0.935904,0.898570,0.108131}%
\pgfsetfillcolor{currentfill}%
\pgfsetlinewidth{0.000000pt}%
\definecolor{currentstroke}{rgb}{0.000000,0.000000,0.000000}%
\pgfsetstrokecolor{currentstroke}%
\pgfsetdash{}{0pt}%
\pgfpathmoveto{\pgfqpoint{1.300518in}{1.704629in}}%
\pgfpathlineto{\pgfqpoint{1.297350in}{1.701960in}}%
\pgfpathlineto{\pgfqpoint{1.294183in}{1.699175in}}%
\pgfpathlineto{\pgfqpoint{1.291015in}{1.696274in}}%
\pgfpathlineto{\pgfqpoint{1.287848in}{1.693259in}}%
\pgfpathlineto{\pgfqpoint{1.286929in}{1.694204in}}%
\pgfpathlineto{\pgfqpoint{1.286074in}{1.695161in}}%
\pgfpathlineto{\pgfqpoint{1.285284in}{1.696130in}}%
\pgfpathlineto{\pgfqpoint{1.284561in}{1.697111in}}%
\pgfpathlineto{\pgfqpoint{1.287890in}{1.699931in}}%
\pgfpathlineto{\pgfqpoint{1.291220in}{1.702638in}}%
\pgfpathlineto{\pgfqpoint{1.294550in}{1.705229in}}%
\pgfpathlineto{\pgfqpoint{1.297881in}{1.707705in}}%
\pgfpathlineto{\pgfqpoint{1.298462in}{1.706922in}}%
\pgfpathlineto{\pgfqpoint{1.299095in}{1.706148in}}%
\pgfpathlineto{\pgfqpoint{1.299781in}{1.705383in}}%
\pgfpathlineto{\pgfqpoint{1.300518in}{1.704629in}}%
\pgfpathclose%
\pgfusepath{fill}%
\end{pgfscope}%
\begin{pgfscope}%
\pgfpathrectangle{\pgfqpoint{0.329460in}{0.284240in}}{\pgfqpoint{1.989680in}{1.989680in}}%
\pgfusepath{clip}%
\pgfsetbuttcap%
\pgfsetroundjoin%
\definecolor{currentfill}{rgb}{0.762373,0.876424,0.137064}%
\pgfsetfillcolor{currentfill}%
\pgfsetlinewidth{0.000000pt}%
\definecolor{currentstroke}{rgb}{0.000000,0.000000,0.000000}%
\pgfsetstrokecolor{currentstroke}%
\pgfsetdash{}{0pt}%
\pgfpathmoveto{\pgfqpoint{1.453789in}{1.649761in}}%
\pgfpathlineto{\pgfqpoint{1.456987in}{1.645370in}}%
\pgfpathlineto{\pgfqpoint{1.460184in}{1.640876in}}%
\pgfpathlineto{\pgfqpoint{1.463380in}{1.636281in}}%
\pgfpathlineto{\pgfqpoint{1.466574in}{1.631585in}}%
\pgfpathlineto{\pgfqpoint{1.464922in}{1.629871in}}%
\pgfpathlineto{\pgfqpoint{1.463156in}{1.628182in}}%
\pgfpathlineto{\pgfqpoint{1.461276in}{1.626520in}}%
\pgfpathlineto{\pgfqpoint{1.459285in}{1.624886in}}%
\pgfpathlineto{\pgfqpoint{1.456290in}{1.629772in}}%
\pgfpathlineto{\pgfqpoint{1.453293in}{1.634558in}}%
\pgfpathlineto{\pgfqpoint{1.450295in}{1.639242in}}%
\pgfpathlineto{\pgfqpoint{1.447296in}{1.643823in}}%
\pgfpathlineto{\pgfqpoint{1.449070in}{1.645271in}}%
\pgfpathlineto{\pgfqpoint{1.450744in}{1.646744in}}%
\pgfpathlineto{\pgfqpoint{1.452318in}{1.648241in}}%
\pgfpathlineto{\pgfqpoint{1.453789in}{1.649761in}}%
\pgfpathclose%
\pgfusepath{fill}%
\end{pgfscope}%
\begin{pgfscope}%
\pgfpathrectangle{\pgfqpoint{0.329460in}{0.284240in}}{\pgfqpoint{1.989680in}{1.989680in}}%
\pgfusepath{clip}%
\pgfsetbuttcap%
\pgfsetroundjoin%
\definecolor{currentfill}{rgb}{0.133743,0.548535,0.553541}%
\pgfsetfillcolor{currentfill}%
\pgfsetlinewidth{0.000000pt}%
\definecolor{currentstroke}{rgb}{0.000000,0.000000,0.000000}%
\pgfsetstrokecolor{currentstroke}%
\pgfsetdash{}{0pt}%
\pgfpathmoveto{\pgfqpoint{1.211865in}{1.247722in}}%
\pgfpathlineto{\pgfqpoint{1.210224in}{1.239182in}}%
\pgfpathlineto{\pgfqpoint{1.208584in}{1.230632in}}%
\pgfpathlineto{\pgfqpoint{1.206944in}{1.222075in}}%
\pgfpathlineto{\pgfqpoint{1.205304in}{1.213513in}}%
\pgfpathlineto{\pgfqpoint{1.196826in}{1.215918in}}%
\pgfpathlineto{\pgfqpoint{1.188511in}{1.218457in}}%
\pgfpathlineto{\pgfqpoint{1.180368in}{1.221126in}}%
\pgfpathlineto{\pgfqpoint{1.172407in}{1.223923in}}%
\pgfpathlineto{\pgfqpoint{1.174412in}{1.232354in}}%
\pgfpathlineto{\pgfqpoint{1.176419in}{1.240781in}}%
\pgfpathlineto{\pgfqpoint{1.178426in}{1.249200in}}%
\pgfpathlineto{\pgfqpoint{1.180434in}{1.257610in}}%
\pgfpathlineto{\pgfqpoint{1.188041in}{1.254953in}}%
\pgfpathlineto{\pgfqpoint{1.195821in}{1.252418in}}%
\pgfpathlineto{\pgfqpoint{1.203765in}{1.250006in}}%
\pgfpathlineto{\pgfqpoint{1.211865in}{1.247722in}}%
\pgfpathclose%
\pgfusepath{fill}%
\end{pgfscope}%
\begin{pgfscope}%
\pgfpathrectangle{\pgfqpoint{0.329460in}{0.284240in}}{\pgfqpoint{1.989680in}{1.989680in}}%
\pgfusepath{clip}%
\pgfsetbuttcap%
\pgfsetroundjoin%
\definecolor{currentfill}{rgb}{0.274952,0.037752,0.364543}%
\pgfsetfillcolor{currentfill}%
\pgfsetlinewidth{0.000000pt}%
\definecolor{currentstroke}{rgb}{0.000000,0.000000,0.000000}%
\pgfsetstrokecolor{currentstroke}%
\pgfsetdash{}{0pt}%
\pgfpathmoveto{\pgfqpoint{1.539301in}{0.820884in}}%
\pgfpathlineto{\pgfqpoint{1.540663in}{0.816194in}}%
\pgfpathlineto{\pgfqpoint{1.542027in}{0.811671in}}%
\pgfpathlineto{\pgfqpoint{1.543394in}{0.807319in}}%
\pgfpathlineto{\pgfqpoint{1.544763in}{0.803141in}}%
\pgfpathlineto{\pgfqpoint{1.530007in}{0.799962in}}%
\pgfpathlineto{\pgfqpoint{1.515053in}{0.797036in}}%
\pgfpathlineto{\pgfqpoint{1.499916in}{0.794366in}}%
\pgfpathlineto{\pgfqpoint{1.484614in}{0.791954in}}%
\pgfpathlineto{\pgfqpoint{1.483669in}{0.796218in}}%
\pgfpathlineto{\pgfqpoint{1.482726in}{0.800657in}}%
\pgfpathlineto{\pgfqpoint{1.481784in}{0.805266in}}%
\pgfpathlineto{\pgfqpoint{1.480843in}{0.810042in}}%
\pgfpathlineto{\pgfqpoint{1.495714in}{0.812379in}}%
\pgfpathlineto{\pgfqpoint{1.510425in}{0.814967in}}%
\pgfpathlineto{\pgfqpoint{1.524959in}{0.817803in}}%
\pgfpathlineto{\pgfqpoint{1.539301in}{0.820884in}}%
\pgfpathclose%
\pgfusepath{fill}%
\end{pgfscope}%
\begin{pgfscope}%
\pgfpathrectangle{\pgfqpoint{0.329460in}{0.284240in}}{\pgfqpoint{1.989680in}{1.989680in}}%
\pgfusepath{clip}%
\pgfsetbuttcap%
\pgfsetroundjoin%
\definecolor{currentfill}{rgb}{0.487026,0.823929,0.312321}%
\pgfsetfillcolor{currentfill}%
\pgfsetlinewidth{0.000000pt}%
\definecolor{currentstroke}{rgb}{0.000000,0.000000,0.000000}%
\pgfsetstrokecolor{currentstroke}%
\pgfsetdash{}{0pt}%
\pgfpathmoveto{\pgfqpoint{1.222180in}{1.548706in}}%
\pgfpathlineto{\pgfqpoint{1.219505in}{1.542405in}}%
\pgfpathlineto{\pgfqpoint{1.216831in}{1.536022in}}%
\pgfpathlineto{\pgfqpoint{1.214157in}{1.529558in}}%
\pgfpathlineto{\pgfqpoint{1.211485in}{1.523015in}}%
\pgfpathlineto{\pgfqpoint{1.207886in}{1.525170in}}%
\pgfpathlineto{\pgfqpoint{1.204434in}{1.527379in}}%
\pgfpathlineto{\pgfqpoint{1.201131in}{1.529638in}}%
\pgfpathlineto{\pgfqpoint{1.197982in}{1.531945in}}%
\pgfpathlineto{\pgfqpoint{1.200908in}{1.538308in}}%
\pgfpathlineto{\pgfqpoint{1.203836in}{1.544593in}}%
\pgfpathlineto{\pgfqpoint{1.206764in}{1.550796in}}%
\pgfpathlineto{\pgfqpoint{1.209694in}{1.556918in}}%
\pgfpathlineto{\pgfqpoint{1.212607in}{1.554796in}}%
\pgfpathlineto{\pgfqpoint{1.215661in}{1.552719in}}%
\pgfpathlineto{\pgfqpoint{1.218853in}{1.550688in}}%
\pgfpathlineto{\pgfqpoint{1.222180in}{1.548706in}}%
\pgfpathclose%
\pgfusepath{fill}%
\end{pgfscope}%
\begin{pgfscope}%
\pgfpathrectangle{\pgfqpoint{0.329460in}{0.284240in}}{\pgfqpoint{1.989680in}{1.989680in}}%
\pgfusepath{clip}%
\pgfsetbuttcap%
\pgfsetroundjoin%
\definecolor{currentfill}{rgb}{0.179019,0.433756,0.557430}%
\pgfsetfillcolor{currentfill}%
\pgfsetlinewidth{0.000000pt}%
\definecolor{currentstroke}{rgb}{0.000000,0.000000,0.000000}%
\pgfsetstrokecolor{currentstroke}%
\pgfsetdash{}{0pt}%
\pgfpathmoveto{\pgfqpoint{1.230739in}{1.136080in}}%
\pgfpathlineto{\pgfqpoint{1.229498in}{1.127486in}}%
\pgfpathlineto{\pgfqpoint{1.228257in}{1.118913in}}%
\pgfpathlineto{\pgfqpoint{1.227016in}{1.110364in}}%
\pgfpathlineto{\pgfqpoint{1.225775in}{1.101842in}}%
\pgfpathlineto{\pgfqpoint{1.215517in}{1.103970in}}%
\pgfpathlineto{\pgfqpoint{1.205405in}{1.106264in}}%
\pgfpathlineto{\pgfqpoint{1.195449in}{1.108720in}}%
\pgfpathlineto{\pgfqpoint{1.185661in}{1.111337in}}%
\pgfpathlineto{\pgfqpoint{1.187296in}{1.119752in}}%
\pgfpathlineto{\pgfqpoint{1.188932in}{1.128195in}}%
\pgfpathlineto{\pgfqpoint{1.190568in}{1.136662in}}%
\pgfpathlineto{\pgfqpoint{1.192204in}{1.145151in}}%
\pgfpathlineto{\pgfqpoint{1.201608in}{1.142651in}}%
\pgfpathlineto{\pgfqpoint{1.211172in}{1.140305in}}%
\pgfpathlineto{\pgfqpoint{1.220886in}{1.138113in}}%
\pgfpathlineto{\pgfqpoint{1.230739in}{1.136080in}}%
\pgfpathclose%
\pgfusepath{fill}%
\end{pgfscope}%
\begin{pgfscope}%
\pgfpathrectangle{\pgfqpoint{0.329460in}{0.284240in}}{\pgfqpoint{1.989680in}{1.989680in}}%
\pgfusepath{clip}%
\pgfsetbuttcap%
\pgfsetroundjoin%
\definecolor{currentfill}{rgb}{0.134692,0.658636,0.517649}%
\pgfsetfillcolor{currentfill}%
\pgfsetlinewidth{0.000000pt}%
\definecolor{currentstroke}{rgb}{0.000000,0.000000,0.000000}%
\pgfsetstrokecolor{currentstroke}%
\pgfsetdash{}{0pt}%
\pgfpathmoveto{\pgfqpoint{1.204587in}{1.356899in}}%
\pgfpathlineto{\pgfqpoint{1.202570in}{1.348792in}}%
\pgfpathlineto{\pgfqpoint{1.200554in}{1.340646in}}%
\pgfpathlineto{\pgfqpoint{1.198538in}{1.332465in}}%
\pgfpathlineto{\pgfqpoint{1.196523in}{1.324250in}}%
\pgfpathlineto{\pgfqpoint{1.189789in}{1.326737in}}%
\pgfpathlineto{\pgfqpoint{1.183224in}{1.329328in}}%
\pgfpathlineto{\pgfqpoint{1.176834in}{1.332020in}}%
\pgfpathlineto{\pgfqpoint{1.170627in}{1.334809in}}%
\pgfpathlineto{\pgfqpoint{1.172975in}{1.342873in}}%
\pgfpathlineto{\pgfqpoint{1.175324in}{1.350903in}}%
\pgfpathlineto{\pgfqpoint{1.177675in}{1.358898in}}%
\pgfpathlineto{\pgfqpoint{1.180026in}{1.366856in}}%
\pgfpathlineto{\pgfqpoint{1.185914in}{1.364225in}}%
\pgfpathlineto{\pgfqpoint{1.191974in}{1.361687in}}%
\pgfpathlineto{\pgfqpoint{1.198201in}{1.359245in}}%
\pgfpathlineto{\pgfqpoint{1.204587in}{1.356899in}}%
\pgfpathclose%
\pgfusepath{fill}%
\end{pgfscope}%
\begin{pgfscope}%
\pgfpathrectangle{\pgfqpoint{0.329460in}{0.284240in}}{\pgfqpoint{1.989680in}{1.989680in}}%
\pgfusepath{clip}%
\pgfsetbuttcap%
\pgfsetroundjoin%
\definecolor{currentfill}{rgb}{0.281477,0.755203,0.432552}%
\pgfsetfillcolor{currentfill}%
\pgfsetlinewidth{0.000000pt}%
\definecolor{currentstroke}{rgb}{0.000000,0.000000,0.000000}%
\pgfsetstrokecolor{currentstroke}%
\pgfsetdash{}{0pt}%
\pgfpathmoveto{\pgfqpoint{1.208326in}{1.458657in}}%
\pgfpathlineto{\pgfqpoint{1.205962in}{1.451312in}}%
\pgfpathlineto{\pgfqpoint{1.203599in}{1.443905in}}%
\pgfpathlineto{\pgfqpoint{1.201237in}{1.436438in}}%
\pgfpathlineto{\pgfqpoint{1.198876in}{1.428913in}}%
\pgfpathlineto{\pgfqpoint{1.193787in}{1.431309in}}%
\pgfpathlineto{\pgfqpoint{1.188861in}{1.433781in}}%
\pgfpathlineto{\pgfqpoint{1.184103in}{1.436328in}}%
\pgfpathlineto{\pgfqpoint{1.179518in}{1.438945in}}%
\pgfpathlineto{\pgfqpoint{1.182175in}{1.446303in}}%
\pgfpathlineto{\pgfqpoint{1.184833in}{1.453603in}}%
\pgfpathlineto{\pgfqpoint{1.187493in}{1.460843in}}%
\pgfpathlineto{\pgfqpoint{1.190153in}{1.468022in}}%
\pgfpathlineto{\pgfqpoint{1.194459in}{1.465579in}}%
\pgfpathlineto{\pgfqpoint{1.198926in}{1.463202in}}%
\pgfpathlineto{\pgfqpoint{1.203550in}{1.460894in}}%
\pgfpathlineto{\pgfqpoint{1.208326in}{1.458657in}}%
\pgfpathclose%
\pgfusepath{fill}%
\end{pgfscope}%
\begin{pgfscope}%
\pgfpathrectangle{\pgfqpoint{0.329460in}{0.284240in}}{\pgfqpoint{1.989680in}{1.989680in}}%
\pgfusepath{clip}%
\pgfsetbuttcap%
\pgfsetroundjoin%
\definecolor{currentfill}{rgb}{0.974417,0.903590,0.130215}%
\pgfsetfillcolor{currentfill}%
\pgfsetlinewidth{0.000000pt}%
\definecolor{currentstroke}{rgb}{0.000000,0.000000,0.000000}%
\pgfsetstrokecolor{currentstroke}%
\pgfsetdash{}{0pt}%
\pgfpathmoveto{\pgfqpoint{1.376843in}{1.722095in}}%
\pgfpathlineto{\pgfqpoint{1.380052in}{1.720410in}}%
\pgfpathlineto{\pgfqpoint{1.383260in}{1.718605in}}%
\pgfpathlineto{\pgfqpoint{1.386470in}{1.716682in}}%
\pgfpathlineto{\pgfqpoint{1.389679in}{1.714641in}}%
\pgfpathlineto{\pgfqpoint{1.389120in}{1.714077in}}%
\pgfpathlineto{\pgfqpoint{1.388524in}{1.713522in}}%
\pgfpathlineto{\pgfqpoint{1.387890in}{1.712976in}}%
\pgfpathlineto{\pgfqpoint{1.387220in}{1.712439in}}%
\pgfpathlineto{\pgfqpoint{1.384215in}{1.714665in}}%
\pgfpathlineto{\pgfqpoint{1.381210in}{1.716772in}}%
\pgfpathlineto{\pgfqpoint{1.378205in}{1.718761in}}%
\pgfpathlineto{\pgfqpoint{1.375201in}{1.720630in}}%
\pgfpathlineto{\pgfqpoint{1.375649in}{1.720987in}}%
\pgfpathlineto{\pgfqpoint{1.376072in}{1.721351in}}%
\pgfpathlineto{\pgfqpoint{1.376470in}{1.721720in}}%
\pgfpathlineto{\pgfqpoint{1.376843in}{1.722095in}}%
\pgfpathclose%
\pgfusepath{fill}%
\end{pgfscope}%
\begin{pgfscope}%
\pgfpathrectangle{\pgfqpoint{0.329460in}{0.284240in}}{\pgfqpoint{1.989680in}{1.989680in}}%
\pgfusepath{clip}%
\pgfsetbuttcap%
\pgfsetroundjoin%
\definecolor{currentfill}{rgb}{0.267004,0.004874,0.329415}%
\pgfsetfillcolor{currentfill}%
\pgfsetlinewidth{0.000000pt}%
\definecolor{currentstroke}{rgb}{0.000000,0.000000,0.000000}%
\pgfsetstrokecolor{currentstroke}%
\pgfsetdash{}{0pt}%
\pgfpathmoveto{\pgfqpoint{1.615723in}{0.792574in}}%
\pgfpathlineto{\pgfqpoint{1.617519in}{0.790267in}}%
\pgfpathlineto{\pgfqpoint{1.619319in}{0.788177in}}%
\pgfpathlineto{\pgfqpoint{1.621123in}{0.786308in}}%
\pgfpathlineto{\pgfqpoint{1.622932in}{0.784664in}}%
\pgfpathlineto{\pgfqpoint{1.607958in}{0.780140in}}%
\pgfpathlineto{\pgfqpoint{1.592700in}{0.775872in}}%
\pgfpathlineto{\pgfqpoint{1.577172in}{0.771865in}}%
\pgfpathlineto{\pgfqpoint{1.561392in}{0.768124in}}%
\pgfpathlineto{\pgfqpoint{1.559989in}{0.769883in}}%
\pgfpathlineto{\pgfqpoint{1.558590in}{0.771867in}}%
\pgfpathlineto{\pgfqpoint{1.557195in}{0.774072in}}%
\pgfpathlineto{\pgfqpoint{1.555802in}{0.776493in}}%
\pgfpathlineto{\pgfqpoint{1.571166in}{0.780130in}}%
\pgfpathlineto{\pgfqpoint{1.586285in}{0.784025in}}%
\pgfpathlineto{\pgfqpoint{1.601143in}{0.788175in}}%
\pgfpathlineto{\pgfqpoint{1.615723in}{0.792574in}}%
\pgfpathclose%
\pgfusepath{fill}%
\end{pgfscope}%
\begin{pgfscope}%
\pgfpathrectangle{\pgfqpoint{0.329460in}{0.284240in}}{\pgfqpoint{1.989680in}{1.989680in}}%
\pgfusepath{clip}%
\pgfsetbuttcap%
\pgfsetroundjoin%
\definecolor{currentfill}{rgb}{0.993248,0.906157,0.143936}%
\pgfsetfillcolor{currentfill}%
\pgfsetlinewidth{0.000000pt}%
\definecolor{currentstroke}{rgb}{0.000000,0.000000,0.000000}%
\pgfsetstrokecolor{currentstroke}%
\pgfsetdash{}{0pt}%
\pgfpathmoveto{\pgfqpoint{1.362175in}{1.726233in}}%
\pgfpathlineto{\pgfqpoint{1.364924in}{1.724673in}}%
\pgfpathlineto{\pgfqpoint{1.367674in}{1.722993in}}%
\pgfpathlineto{\pgfqpoint{1.370424in}{1.721193in}}%
\pgfpathlineto{\pgfqpoint{1.373175in}{1.719273in}}%
\pgfpathlineto{\pgfqpoint{1.372613in}{1.718953in}}%
\pgfpathlineto{\pgfqpoint{1.372028in}{1.718641in}}%
\pgfpathlineto{\pgfqpoint{1.371423in}{1.718339in}}%
\pgfpathlineto{\pgfqpoint{1.370798in}{1.718045in}}%
\pgfpathlineto{\pgfqpoint{1.368344in}{1.720119in}}%
\pgfpathlineto{\pgfqpoint{1.365891in}{1.722073in}}%
\pgfpathlineto{\pgfqpoint{1.363438in}{1.723907in}}%
\pgfpathlineto{\pgfqpoint{1.360986in}{1.725620in}}%
\pgfpathlineto{\pgfqpoint{1.361299in}{1.725767in}}%
\pgfpathlineto{\pgfqpoint{1.361602in}{1.725918in}}%
\pgfpathlineto{\pgfqpoint{1.361894in}{1.726073in}}%
\pgfpathlineto{\pgfqpoint{1.362175in}{1.726233in}}%
\pgfpathclose%
\pgfusepath{fill}%
\end{pgfscope}%
\begin{pgfscope}%
\pgfpathrectangle{\pgfqpoint{0.329460in}{0.284240in}}{\pgfqpoint{1.989680in}{1.989680in}}%
\pgfusepath{clip}%
\pgfsetbuttcap%
\pgfsetroundjoin%
\definecolor{currentfill}{rgb}{0.993248,0.906157,0.143936}%
\pgfsetfillcolor{currentfill}%
\pgfsetlinewidth{0.000000pt}%
\definecolor{currentstroke}{rgb}{0.000000,0.000000,0.000000}%
\pgfsetstrokecolor{currentstroke}%
\pgfsetdash{}{0pt}%
\pgfpathmoveto{\pgfqpoint{1.341675in}{1.725494in}}%
\pgfpathlineto{\pgfqpoint{1.339295in}{1.723749in}}%
\pgfpathlineto{\pgfqpoint{1.336914in}{1.721883in}}%
\pgfpathlineto{\pgfqpoint{1.334532in}{1.719897in}}%
\pgfpathlineto{\pgfqpoint{1.332149in}{1.717792in}}%
\pgfpathlineto{\pgfqpoint{1.331507in}{1.718077in}}%
\pgfpathlineto{\pgfqpoint{1.330883in}{1.718372in}}%
\pgfpathlineto{\pgfqpoint{1.330281in}{1.718675in}}%
\pgfpathlineto{\pgfqpoint{1.329699in}{1.718988in}}%
\pgfpathlineto{\pgfqpoint{1.332387in}{1.720944in}}%
\pgfpathlineto{\pgfqpoint{1.335075in}{1.722779in}}%
\pgfpathlineto{\pgfqpoint{1.337763in}{1.724495in}}%
\pgfpathlineto{\pgfqpoint{1.340449in}{1.726091in}}%
\pgfpathlineto{\pgfqpoint{1.340740in}{1.725935in}}%
\pgfpathlineto{\pgfqpoint{1.341042in}{1.725783in}}%
\pgfpathlineto{\pgfqpoint{1.341354in}{1.725636in}}%
\pgfpathlineto{\pgfqpoint{1.341675in}{1.725494in}}%
\pgfpathclose%
\pgfusepath{fill}%
\end{pgfscope}%
\begin{pgfscope}%
\pgfpathrectangle{\pgfqpoint{0.329460in}{0.284240in}}{\pgfqpoint{1.989680in}{1.989680in}}%
\pgfusepath{clip}%
\pgfsetbuttcap%
\pgfsetroundjoin%
\definecolor{currentfill}{rgb}{0.565498,0.842430,0.262877}%
\pgfsetfillcolor{currentfill}%
\pgfsetlinewidth{0.000000pt}%
\definecolor{currentstroke}{rgb}{0.000000,0.000000,0.000000}%
\pgfsetstrokecolor{currentstroke}%
\pgfsetdash{}{0pt}%
\pgfpathmoveto{\pgfqpoint{1.483214in}{1.582312in}}%
\pgfpathlineto{\pgfqpoint{1.486200in}{1.576574in}}%
\pgfpathlineto{\pgfqpoint{1.489184in}{1.570749in}}%
\pgfpathlineto{\pgfqpoint{1.492167in}{1.564837in}}%
\pgfpathlineto{\pgfqpoint{1.495149in}{1.558840in}}%
\pgfpathlineto{\pgfqpoint{1.492364in}{1.556680in}}%
\pgfpathlineto{\pgfqpoint{1.489436in}{1.554563in}}%
\pgfpathlineto{\pgfqpoint{1.486367in}{1.552491in}}%
\pgfpathlineto{\pgfqpoint{1.483159in}{1.550465in}}%
\pgfpathlineto{\pgfqpoint{1.480422in}{1.556644in}}%
\pgfpathlineto{\pgfqpoint{1.477684in}{1.562737in}}%
\pgfpathlineto{\pgfqpoint{1.474945in}{1.568744in}}%
\pgfpathlineto{\pgfqpoint{1.472204in}{1.574662in}}%
\pgfpathlineto{\pgfqpoint{1.475149in}{1.576512in}}%
\pgfpathlineto{\pgfqpoint{1.477967in}{1.578405in}}%
\pgfpathlineto{\pgfqpoint{1.480656in}{1.580339in}}%
\pgfpathlineto{\pgfqpoint{1.483214in}{1.582312in}}%
\pgfpathclose%
\pgfusepath{fill}%
\end{pgfscope}%
\begin{pgfscope}%
\pgfpathrectangle{\pgfqpoint{0.329460in}{0.284240in}}{\pgfqpoint{1.989680in}{1.989680in}}%
\pgfusepath{clip}%
\pgfsetbuttcap%
\pgfsetroundjoin%
\definecolor{currentfill}{rgb}{0.974417,0.903590,0.130215}%
\pgfsetfillcolor{currentfill}%
\pgfsetlinewidth{0.000000pt}%
\definecolor{currentstroke}{rgb}{0.000000,0.000000,0.000000}%
\pgfsetstrokecolor{currentstroke}%
\pgfsetdash{}{0pt}%
\pgfpathmoveto{\pgfqpoint{1.327592in}{1.720318in}}%
\pgfpathlineto{\pgfqpoint{1.324640in}{1.718410in}}%
\pgfpathlineto{\pgfqpoint{1.321687in}{1.716382in}}%
\pgfpathlineto{\pgfqpoint{1.318735in}{1.714236in}}%
\pgfpathlineto{\pgfqpoint{1.315781in}{1.711971in}}%
\pgfpathlineto{\pgfqpoint{1.315079in}{1.712498in}}%
\pgfpathlineto{\pgfqpoint{1.314412in}{1.713036in}}%
\pgfpathlineto{\pgfqpoint{1.313783in}{1.713583in}}%
\pgfpathlineto{\pgfqpoint{1.313191in}{1.714139in}}%
\pgfpathlineto{\pgfqpoint{1.316359in}{1.716223in}}%
\pgfpathlineto{\pgfqpoint{1.319527in}{1.718188in}}%
\pgfpathlineto{\pgfqpoint{1.322695in}{1.720034in}}%
\pgfpathlineto{\pgfqpoint{1.325862in}{1.721762in}}%
\pgfpathlineto{\pgfqpoint{1.326258in}{1.721391in}}%
\pgfpathlineto{\pgfqpoint{1.326678in}{1.721027in}}%
\pgfpathlineto{\pgfqpoint{1.327123in}{1.720669in}}%
\pgfpathlineto{\pgfqpoint{1.327592in}{1.720318in}}%
\pgfpathclose%
\pgfusepath{fill}%
\end{pgfscope}%
\begin{pgfscope}%
\pgfpathrectangle{\pgfqpoint{0.329460in}{0.284240in}}{\pgfqpoint{1.989680in}{1.989680in}}%
\pgfusepath{clip}%
\pgfsetbuttcap%
\pgfsetroundjoin%
\definecolor{currentfill}{rgb}{0.163625,0.471133,0.558148}%
\pgfsetfillcolor{currentfill}%
\pgfsetlinewidth{0.000000pt}%
\definecolor{currentstroke}{rgb}{0.000000,0.000000,0.000000}%
\pgfsetstrokecolor{currentstroke}%
\pgfsetdash{}{0pt}%
\pgfpathmoveto{\pgfqpoint{1.511505in}{1.181506in}}%
\pgfpathlineto{\pgfqpoint{1.513226in}{1.172985in}}%
\pgfpathlineto{\pgfqpoint{1.514947in}{1.164475in}}%
\pgfpathlineto{\pgfqpoint{1.516668in}{1.155979in}}%
\pgfpathlineto{\pgfqpoint{1.518388in}{1.147498in}}%
\pgfpathlineto{\pgfqpoint{1.509135in}{1.144866in}}%
\pgfpathlineto{\pgfqpoint{1.499712in}{1.142383in}}%
\pgfpathlineto{\pgfqpoint{1.490131in}{1.140053in}}%
\pgfpathlineto{\pgfqpoint{1.480401in}{1.137880in}}%
\pgfpathlineto{\pgfqpoint{1.479070in}{1.146472in}}%
\pgfpathlineto{\pgfqpoint{1.477738in}{1.155080in}}%
\pgfpathlineto{\pgfqpoint{1.476406in}{1.163701in}}%
\pgfpathlineto{\pgfqpoint{1.475074in}{1.172333in}}%
\pgfpathlineto{\pgfqpoint{1.484405in}{1.174406in}}%
\pgfpathlineto{\pgfqpoint{1.493593in}{1.176628in}}%
\pgfpathlineto{\pgfqpoint{1.502630in}{1.178995in}}%
\pgfpathlineto{\pgfqpoint{1.511505in}{1.181506in}}%
\pgfpathclose%
\pgfusepath{fill}%
\end{pgfscope}%
\begin{pgfscope}%
\pgfpathrectangle{\pgfqpoint{0.329460in}{0.284240in}}{\pgfqpoint{1.989680in}{1.989680in}}%
\pgfusepath{clip}%
\pgfsetbuttcap%
\pgfsetroundjoin%
\definecolor{currentfill}{rgb}{0.231674,0.318106,0.544834}%
\pgfsetfillcolor{currentfill}%
\pgfsetlinewidth{0.000000pt}%
\definecolor{currentstroke}{rgb}{0.000000,0.000000,0.000000}%
\pgfsetstrokecolor{currentstroke}%
\pgfsetdash{}{0pt}%
\pgfpathmoveto{\pgfqpoint{1.261455in}{1.027551in}}%
\pgfpathlineto{\pgfqpoint{1.260631in}{1.019324in}}%
\pgfpathlineto{\pgfqpoint{1.259807in}{1.011154in}}%
\pgfpathlineto{\pgfqpoint{1.258982in}{1.003042in}}%
\pgfpathlineto{\pgfqpoint{1.258158in}{0.994993in}}%
\pgfpathlineto{\pgfqpoint{1.246154in}{0.996635in}}%
\pgfpathlineto{\pgfqpoint{1.234265in}{0.998474in}}%
\pgfpathlineto{\pgfqpoint{1.222503in}{1.000509in}}%
\pgfpathlineto{\pgfqpoint{1.210881in}{1.002736in}}%
\pgfpathlineto{\pgfqpoint{1.212123in}{1.010707in}}%
\pgfpathlineto{\pgfqpoint{1.213364in}{1.018741in}}%
\pgfpathlineto{\pgfqpoint{1.214606in}{1.026834in}}%
\pgfpathlineto{\pgfqpoint{1.215847in}{1.034983in}}%
\pgfpathlineto{\pgfqpoint{1.227060in}{1.032845in}}%
\pgfpathlineto{\pgfqpoint{1.238407in}{1.030892in}}%
\pgfpathlineto{\pgfqpoint{1.249876in}{1.029127in}}%
\pgfpathlineto{\pgfqpoint{1.261455in}{1.027551in}}%
\pgfpathclose%
\pgfusepath{fill}%
\end{pgfscope}%
\begin{pgfscope}%
\pgfpathrectangle{\pgfqpoint{0.329460in}{0.284240in}}{\pgfqpoint{1.989680in}{1.989680in}}%
\pgfusepath{clip}%
\pgfsetbuttcap%
\pgfsetroundjoin%
\definecolor{currentfill}{rgb}{0.122606,0.585371,0.546557}%
\pgfsetfillcolor{currentfill}%
\pgfsetlinewidth{0.000000pt}%
\definecolor{currentstroke}{rgb}{0.000000,0.000000,0.000000}%
\pgfsetstrokecolor{currentstroke}%
\pgfsetdash{}{0pt}%
\pgfpathmoveto{\pgfqpoint{1.520206in}{1.293437in}}%
\pgfpathlineto{\pgfqpoint{1.522294in}{1.285122in}}%
\pgfpathlineto{\pgfqpoint{1.524381in}{1.276788in}}%
\pgfpathlineto{\pgfqpoint{1.526467in}{1.268437in}}%
\pgfpathlineto{\pgfqpoint{1.528553in}{1.260071in}}%
\pgfpathlineto{\pgfqpoint{1.521105in}{1.257309in}}%
\pgfpathlineto{\pgfqpoint{1.513478in}{1.254665in}}%
\pgfpathlineto{\pgfqpoint{1.505679in}{1.252144in}}%
\pgfpathlineto{\pgfqpoint{1.497718in}{1.249746in}}%
\pgfpathlineto{\pgfqpoint{1.495992in}{1.258247in}}%
\pgfpathlineto{\pgfqpoint{1.494266in}{1.266733in}}%
\pgfpathlineto{\pgfqpoint{1.492539in}{1.275202in}}%
\pgfpathlineto{\pgfqpoint{1.490811in}{1.283651in}}%
\pgfpathlineto{\pgfqpoint{1.498400in}{1.285923in}}%
\pgfpathlineto{\pgfqpoint{1.505834in}{1.288313in}}%
\pgfpathlineto{\pgfqpoint{1.513105in}{1.290819in}}%
\pgfpathlineto{\pgfqpoint{1.520206in}{1.293437in}}%
\pgfpathclose%
\pgfusepath{fill}%
\end{pgfscope}%
\begin{pgfscope}%
\pgfpathrectangle{\pgfqpoint{0.329460in}{0.284240in}}{\pgfqpoint{1.989680in}{1.989680in}}%
\pgfusepath{clip}%
\pgfsetbuttcap%
\pgfsetroundjoin%
\definecolor{currentfill}{rgb}{0.268510,0.009605,0.335427}%
\pgfsetfillcolor{currentfill}%
\pgfsetlinewidth{0.000000pt}%
\definecolor{currentstroke}{rgb}{0.000000,0.000000,0.000000}%
\pgfsetstrokecolor{currentstroke}%
\pgfsetdash{}{0pt}%
\pgfpathmoveto{\pgfqpoint{1.085812in}{0.776285in}}%
\pgfpathlineto{\pgfqpoint{1.084069in}{0.775808in}}%
\pgfpathlineto{\pgfqpoint{1.082321in}{0.775583in}}%
\pgfpathlineto{\pgfqpoint{1.080568in}{0.775617in}}%
\pgfpathlineto{\pgfqpoint{1.078810in}{0.775914in}}%
\pgfpathlineto{\pgfqpoint{1.063075in}{0.780714in}}%
\pgfpathlineto{\pgfqpoint{1.047661in}{0.785778in}}%
\pgfpathlineto{\pgfqpoint{1.032586in}{0.791098in}}%
\pgfpathlineto{\pgfqpoint{1.017864in}{0.796669in}}%
\pgfpathlineto{\pgfqpoint{1.020009in}{0.796240in}}%
\pgfpathlineto{\pgfqpoint{1.022148in}{0.796073in}}%
\pgfpathlineto{\pgfqpoint{1.024281in}{0.796164in}}%
\pgfpathlineto{\pgfqpoint{1.026408in}{0.796507in}}%
\pgfpathlineto{\pgfqpoint{1.040758in}{0.791079in}}%
\pgfpathlineto{\pgfqpoint{1.055453in}{0.785895in}}%
\pgfpathlineto{\pgfqpoint{1.070476in}{0.780962in}}%
\pgfpathlineto{\pgfqpoint{1.085812in}{0.776285in}}%
\pgfpathclose%
\pgfusepath{fill}%
\end{pgfscope}%
\begin{pgfscope}%
\pgfpathrectangle{\pgfqpoint{0.329460in}{0.284240in}}{\pgfqpoint{1.989680in}{1.989680in}}%
\pgfusepath{clip}%
\pgfsetbuttcap%
\pgfsetroundjoin%
\definecolor{currentfill}{rgb}{0.344074,0.780029,0.397381}%
\pgfsetfillcolor{currentfill}%
\pgfsetlinewidth{0.000000pt}%
\definecolor{currentstroke}{rgb}{0.000000,0.000000,0.000000}%
\pgfsetstrokecolor{currentstroke}%
\pgfsetdash{}{0pt}%
\pgfpathmoveto{\pgfqpoint{1.505013in}{1.498153in}}%
\pgfpathlineto{\pgfqpoint{1.507740in}{1.491278in}}%
\pgfpathlineto{\pgfqpoint{1.510464in}{1.484335in}}%
\pgfpathlineto{\pgfqpoint{1.513188in}{1.477324in}}%
\pgfpathlineto{\pgfqpoint{1.515910in}{1.470248in}}%
\pgfpathlineto{\pgfqpoint{1.511751in}{1.467748in}}%
\pgfpathlineto{\pgfqpoint{1.507428in}{1.465311in}}%
\pgfpathlineto{\pgfqpoint{1.502943in}{1.462942in}}%
\pgfpathlineto{\pgfqpoint{1.498302in}{1.460642in}}%
\pgfpathlineto{\pgfqpoint{1.495867in}{1.467888in}}%
\pgfpathlineto{\pgfqpoint{1.493431in}{1.475068in}}%
\pgfpathlineto{\pgfqpoint{1.490994in}{1.482181in}}%
\pgfpathlineto{\pgfqpoint{1.488556in}{1.489225in}}%
\pgfpathlineto{\pgfqpoint{1.492893in}{1.491362in}}%
\pgfpathlineto{\pgfqpoint{1.497085in}{1.493564in}}%
\pgfpathlineto{\pgfqpoint{1.501126in}{1.495829in}}%
\pgfpathlineto{\pgfqpoint{1.505013in}{1.498153in}}%
\pgfpathclose%
\pgfusepath{fill}%
\end{pgfscope}%
\begin{pgfscope}%
\pgfpathrectangle{\pgfqpoint{0.329460in}{0.284240in}}{\pgfqpoint{1.989680in}{1.989680in}}%
\pgfusepath{clip}%
\pgfsetbuttcap%
\pgfsetroundjoin%
\definecolor{currentfill}{rgb}{0.166383,0.690856,0.496502}%
\pgfsetfillcolor{currentfill}%
\pgfsetlinewidth{0.000000pt}%
\definecolor{currentstroke}{rgb}{0.000000,0.000000,0.000000}%
\pgfsetstrokecolor{currentstroke}%
\pgfsetdash{}{0pt}%
\pgfpathmoveto{\pgfqpoint{1.517741in}{1.400538in}}%
\pgfpathlineto{\pgfqpoint{1.520166in}{1.392788in}}%
\pgfpathlineto{\pgfqpoint{1.522590in}{1.384991in}}%
\pgfpathlineto{\pgfqpoint{1.525012in}{1.377151in}}%
\pgfpathlineto{\pgfqpoint{1.527433in}{1.369269in}}%
\pgfpathlineto{\pgfqpoint{1.521704in}{1.366559in}}%
\pgfpathlineto{\pgfqpoint{1.515796in}{1.363939in}}%
\pgfpathlineto{\pgfqpoint{1.509717in}{1.361411in}}%
\pgfpathlineto{\pgfqpoint{1.503472in}{1.358979in}}%
\pgfpathlineto{\pgfqpoint{1.501377in}{1.367016in}}%
\pgfpathlineto{\pgfqpoint{1.499280in}{1.375010in}}%
\pgfpathlineto{\pgfqpoint{1.497183in}{1.382960in}}%
\pgfpathlineto{\pgfqpoint{1.495085in}{1.390864in}}%
\pgfpathlineto{\pgfqpoint{1.500989in}{1.393150in}}%
\pgfpathlineto{\pgfqpoint{1.506736in}{1.395527in}}%
\pgfpathlineto{\pgfqpoint{1.512323in}{1.397990in}}%
\pgfpathlineto{\pgfqpoint{1.517741in}{1.400538in}}%
\pgfpathclose%
\pgfusepath{fill}%
\end{pgfscope}%
\begin{pgfscope}%
\pgfpathrectangle{\pgfqpoint{0.329460in}{0.284240in}}{\pgfqpoint{1.989680in}{1.989680in}}%
\pgfusepath{clip}%
\pgfsetbuttcap%
\pgfsetroundjoin%
\definecolor{currentfill}{rgb}{0.762373,0.876424,0.137064}%
\pgfsetfillcolor{currentfill}%
\pgfsetlinewidth{0.000000pt}%
\definecolor{currentstroke}{rgb}{0.000000,0.000000,0.000000}%
\pgfsetstrokecolor{currentstroke}%
\pgfsetdash{}{0pt}%
\pgfpathmoveto{\pgfqpoint{1.256736in}{1.642559in}}%
\pgfpathlineto{\pgfqpoint{1.253789in}{1.637937in}}%
\pgfpathlineto{\pgfqpoint{1.250842in}{1.633213in}}%
\pgfpathlineto{\pgfqpoint{1.247896in}{1.628386in}}%
\pgfpathlineto{\pgfqpoint{1.244951in}{1.623459in}}%
\pgfpathlineto{\pgfqpoint{1.242863in}{1.625066in}}%
\pgfpathlineto{\pgfqpoint{1.240884in}{1.626703in}}%
\pgfpathlineto{\pgfqpoint{1.239017in}{1.628368in}}%
\pgfpathlineto{\pgfqpoint{1.237264in}{1.630060in}}%
\pgfpathlineto{\pgfqpoint{1.240418in}{1.634799in}}%
\pgfpathlineto{\pgfqpoint{1.243574in}{1.639438in}}%
\pgfpathlineto{\pgfqpoint{1.246731in}{1.643975in}}%
\pgfpathlineto{\pgfqpoint{1.249889in}{1.648409in}}%
\pgfpathlineto{\pgfqpoint{1.251451in}{1.646910in}}%
\pgfpathlineto{\pgfqpoint{1.253115in}{1.645434in}}%
\pgfpathlineto{\pgfqpoint{1.254877in}{1.643983in}}%
\pgfpathlineto{\pgfqpoint{1.256736in}{1.642559in}}%
\pgfpathclose%
\pgfusepath{fill}%
\end{pgfscope}%
\begin{pgfscope}%
\pgfpathrectangle{\pgfqpoint{0.329460in}{0.284240in}}{\pgfqpoint{1.989680in}{1.989680in}}%
\pgfusepath{clip}%
\pgfsetbuttcap%
\pgfsetroundjoin%
\definecolor{currentfill}{rgb}{0.814576,0.883393,0.110347}%
\pgfsetfillcolor{currentfill}%
\pgfsetlinewidth{0.000000pt}%
\definecolor{currentstroke}{rgb}{0.000000,0.000000,0.000000}%
\pgfsetstrokecolor{currentstroke}%
\pgfsetdash{}{0pt}%
\pgfpathmoveto{\pgfqpoint{1.440988in}{1.666279in}}%
\pgfpathlineto{\pgfqpoint{1.444190in}{1.662308in}}%
\pgfpathlineto{\pgfqpoint{1.447391in}{1.658231in}}%
\pgfpathlineto{\pgfqpoint{1.450591in}{1.654048in}}%
\pgfpathlineto{\pgfqpoint{1.453789in}{1.649761in}}%
\pgfpathlineto{\pgfqpoint{1.452318in}{1.648241in}}%
\pgfpathlineto{\pgfqpoint{1.450744in}{1.646744in}}%
\pgfpathlineto{\pgfqpoint{1.449070in}{1.645271in}}%
\pgfpathlineto{\pgfqpoint{1.447296in}{1.643823in}}%
\pgfpathlineto{\pgfqpoint{1.444297in}{1.648300in}}%
\pgfpathlineto{\pgfqpoint{1.441297in}{1.652672in}}%
\pgfpathlineto{\pgfqpoint{1.438296in}{1.656938in}}%
\pgfpathlineto{\pgfqpoint{1.435295in}{1.661097in}}%
\pgfpathlineto{\pgfqpoint{1.436849in}{1.662361in}}%
\pgfpathlineto{\pgfqpoint{1.438317in}{1.663646in}}%
\pgfpathlineto{\pgfqpoint{1.439697in}{1.664953in}}%
\pgfpathlineto{\pgfqpoint{1.440988in}{1.666279in}}%
\pgfpathclose%
\pgfusepath{fill}%
\end{pgfscope}%
\begin{pgfscope}%
\pgfpathrectangle{\pgfqpoint{0.329460in}{0.284240in}}{\pgfqpoint{1.989680in}{1.989680in}}%
\pgfusepath{clip}%
\pgfsetbuttcap%
\pgfsetroundjoin%
\definecolor{currentfill}{rgb}{0.212395,0.359683,0.551710}%
\pgfsetfillcolor{currentfill}%
\pgfsetlinewidth{0.000000pt}%
\definecolor{currentstroke}{rgb}{0.000000,0.000000,0.000000}%
\pgfsetstrokecolor{currentstroke}%
\pgfsetdash{}{0pt}%
\pgfpathmoveto{\pgfqpoint{1.491048in}{1.070050in}}%
\pgfpathlineto{\pgfqpoint{1.492379in}{1.061727in}}%
\pgfpathlineto{\pgfqpoint{1.493710in}{1.053449in}}%
\pgfpathlineto{\pgfqpoint{1.495041in}{1.045218in}}%
\pgfpathlineto{\pgfqpoint{1.496372in}{1.037037in}}%
\pgfpathlineto{\pgfqpoint{1.485289in}{1.034736in}}%
\pgfpathlineto{\pgfqpoint{1.474061in}{1.032619in}}%
\pgfpathlineto{\pgfqpoint{1.462700in}{1.030686in}}%
\pgfpathlineto{\pgfqpoint{1.451218in}{1.028942in}}%
\pgfpathlineto{\pgfqpoint{1.450300in}{1.037208in}}%
\pgfpathlineto{\pgfqpoint{1.449382in}{1.045523in}}%
\pgfpathlineto{\pgfqpoint{1.448464in}{1.053886in}}%
\pgfpathlineto{\pgfqpoint{1.447546in}{1.062292in}}%
\pgfpathlineto{\pgfqpoint{1.458607in}{1.063964in}}%
\pgfpathlineto{\pgfqpoint{1.469553in}{1.065816in}}%
\pgfpathlineto{\pgfqpoint{1.480370in}{1.067845in}}%
\pgfpathlineto{\pgfqpoint{1.491048in}{1.070050in}}%
\pgfpathclose%
\pgfusepath{fill}%
\end{pgfscope}%
\begin{pgfscope}%
\pgfpathrectangle{\pgfqpoint{0.329460in}{0.284240in}}{\pgfqpoint{1.989680in}{1.989680in}}%
\pgfusepath{clip}%
\pgfsetbuttcap%
\pgfsetroundjoin%
\definecolor{currentfill}{rgb}{0.993248,0.906157,0.143936}%
\pgfsetfillcolor{currentfill}%
\pgfsetlinewidth{0.000000pt}%
\definecolor{currentstroke}{rgb}{0.000000,0.000000,0.000000}%
\pgfsetstrokecolor{currentstroke}%
\pgfsetdash{}{0pt}%
\pgfpathmoveto{\pgfqpoint{1.360986in}{1.725620in}}%
\pgfpathlineto{\pgfqpoint{1.363438in}{1.723907in}}%
\pgfpathlineto{\pgfqpoint{1.365891in}{1.722073in}}%
\pgfpathlineto{\pgfqpoint{1.368344in}{1.720119in}}%
\pgfpathlineto{\pgfqpoint{1.370798in}{1.718045in}}%
\pgfpathlineto{\pgfqpoint{1.370153in}{1.717761in}}%
\pgfpathlineto{\pgfqpoint{1.369489in}{1.717486in}}%
\pgfpathlineto{\pgfqpoint{1.368807in}{1.717221in}}%
\pgfpathlineto{\pgfqpoint{1.368107in}{1.716966in}}%
\pgfpathlineto{\pgfqpoint{1.365989in}{1.719176in}}%
\pgfpathlineto{\pgfqpoint{1.363872in}{1.721265in}}%
\pgfpathlineto{\pgfqpoint{1.361756in}{1.723234in}}%
\pgfpathlineto{\pgfqpoint{1.359641in}{1.725082in}}%
\pgfpathlineto{\pgfqpoint{1.359991in}{1.725209in}}%
\pgfpathlineto{\pgfqpoint{1.360332in}{1.725341in}}%
\pgfpathlineto{\pgfqpoint{1.360664in}{1.725478in}}%
\pgfpathlineto{\pgfqpoint{1.360986in}{1.725620in}}%
\pgfpathclose%
\pgfusepath{fill}%
\end{pgfscope}%
\begin{pgfscope}%
\pgfpathrectangle{\pgfqpoint{0.329460in}{0.284240in}}{\pgfqpoint{1.989680in}{1.989680in}}%
\pgfusepath{clip}%
\pgfsetbuttcap%
\pgfsetroundjoin%
\definecolor{currentfill}{rgb}{0.993248,0.906157,0.143936}%
\pgfsetfillcolor{currentfill}%
\pgfsetlinewidth{0.000000pt}%
\definecolor{currentstroke}{rgb}{0.000000,0.000000,0.000000}%
\pgfsetstrokecolor{currentstroke}%
\pgfsetdash{}{0pt}%
\pgfpathmoveto{\pgfqpoint{1.343052in}{1.724974in}}%
\pgfpathlineto{\pgfqpoint{1.341017in}{1.723098in}}%
\pgfpathlineto{\pgfqpoint{1.338980in}{1.721102in}}%
\pgfpathlineto{\pgfqpoint{1.336943in}{1.718986in}}%
\pgfpathlineto{\pgfqpoint{1.334905in}{1.716749in}}%
\pgfpathlineto{\pgfqpoint{1.334190in}{1.716994in}}%
\pgfpathlineto{\pgfqpoint{1.333492in}{1.717250in}}%
\pgfpathlineto{\pgfqpoint{1.332811in}{1.717516in}}%
\pgfpathlineto{\pgfqpoint{1.332149in}{1.717792in}}%
\pgfpathlineto{\pgfqpoint{1.334532in}{1.719897in}}%
\pgfpathlineto{\pgfqpoint{1.336914in}{1.721883in}}%
\pgfpathlineto{\pgfqpoint{1.339295in}{1.723749in}}%
\pgfpathlineto{\pgfqpoint{1.341675in}{1.725494in}}%
\pgfpathlineto{\pgfqpoint{1.342006in}{1.725356in}}%
\pgfpathlineto{\pgfqpoint{1.342346in}{1.725224in}}%
\pgfpathlineto{\pgfqpoint{1.342695in}{1.725096in}}%
\pgfpathlineto{\pgfqpoint{1.343052in}{1.724974in}}%
\pgfpathclose%
\pgfusepath{fill}%
\end{pgfscope}%
\begin{pgfscope}%
\pgfpathrectangle{\pgfqpoint{0.329460in}{0.284240in}}{\pgfqpoint{1.989680in}{1.989680in}}%
\pgfusepath{clip}%
\pgfsetbuttcap%
\pgfsetroundjoin%
\definecolor{currentfill}{rgb}{0.955300,0.901065,0.118128}%
\pgfsetfillcolor{currentfill}%
\pgfsetlinewidth{0.000000pt}%
\definecolor{currentstroke}{rgb}{0.000000,0.000000,0.000000}%
\pgfsetstrokecolor{currentstroke}%
\pgfsetdash{}{0pt}%
\pgfpathmoveto{\pgfqpoint{1.389679in}{1.714641in}}%
\pgfpathlineto{\pgfqpoint{1.392888in}{1.712481in}}%
\pgfpathlineto{\pgfqpoint{1.396097in}{1.710204in}}%
\pgfpathlineto{\pgfqpoint{1.399306in}{1.707810in}}%
\pgfpathlineto{\pgfqpoint{1.402515in}{1.705299in}}%
\pgfpathlineto{\pgfqpoint{1.401772in}{1.704546in}}%
\pgfpathlineto{\pgfqpoint{1.400978in}{1.703804in}}%
\pgfpathlineto{\pgfqpoint{1.400135in}{1.703075in}}%
\pgfpathlineto{\pgfqpoint{1.399242in}{1.702358in}}%
\pgfpathlineto{\pgfqpoint{1.396237in}{1.705054in}}%
\pgfpathlineto{\pgfqpoint{1.393231in}{1.707633in}}%
\pgfpathlineto{\pgfqpoint{1.390225in}{1.710095in}}%
\pgfpathlineto{\pgfqpoint{1.387220in}{1.712439in}}%
\pgfpathlineto{\pgfqpoint{1.387890in}{1.712976in}}%
\pgfpathlineto{\pgfqpoint{1.388524in}{1.713522in}}%
\pgfpathlineto{\pgfqpoint{1.389120in}{1.714077in}}%
\pgfpathlineto{\pgfqpoint{1.389679in}{1.714641in}}%
\pgfpathclose%
\pgfusepath{fill}%
\end{pgfscope}%
\begin{pgfscope}%
\pgfpathrectangle{\pgfqpoint{0.329460in}{0.284240in}}{\pgfqpoint{1.989680in}{1.989680in}}%
\pgfusepath{clip}%
\pgfsetbuttcap%
\pgfsetroundjoin%
\definecolor{currentfill}{rgb}{0.280255,0.165693,0.476498}%
\pgfsetfillcolor{currentfill}%
\pgfsetlinewidth{0.000000pt}%
\definecolor{currentstroke}{rgb}{0.000000,0.000000,0.000000}%
\pgfsetstrokecolor{currentstroke}%
\pgfsetdash{}{0pt}%
\pgfpathmoveto{\pgfqpoint{1.465947in}{0.906041in}}%
\pgfpathlineto{\pgfqpoint{1.466871in}{0.899107in}}%
\pgfpathlineto{\pgfqpoint{1.467796in}{0.892281in}}%
\pgfpathlineto{\pgfqpoint{1.468722in}{0.885567in}}%
\pgfpathlineto{\pgfqpoint{1.469649in}{0.878968in}}%
\pgfpathlineto{\pgfqpoint{1.455928in}{0.877084in}}%
\pgfpathlineto{\pgfqpoint{1.442091in}{0.875433in}}%
\pgfpathlineto{\pgfqpoint{1.428154in}{0.874016in}}%
\pgfpathlineto{\pgfqpoint{1.414132in}{0.872834in}}%
\pgfpathlineto{\pgfqpoint{1.413639in}{0.879488in}}%
\pgfpathlineto{\pgfqpoint{1.413147in}{0.886256in}}%
\pgfpathlineto{\pgfqpoint{1.412655in}{0.893136in}}%
\pgfpathlineto{\pgfqpoint{1.412163in}{0.900123in}}%
\pgfpathlineto{\pgfqpoint{1.425747in}{0.901263in}}%
\pgfpathlineto{\pgfqpoint{1.439248in}{0.902630in}}%
\pgfpathlineto{\pgfqpoint{1.452653in}{0.904224in}}%
\pgfpathlineto{\pgfqpoint{1.465947in}{0.906041in}}%
\pgfpathclose%
\pgfusepath{fill}%
\end{pgfscope}%
\begin{pgfscope}%
\pgfpathrectangle{\pgfqpoint{0.329460in}{0.284240in}}{\pgfqpoint{1.989680in}{1.989680in}}%
\pgfusepath{clip}%
\pgfsetbuttcap%
\pgfsetroundjoin%
\definecolor{currentfill}{rgb}{0.274952,0.037752,0.364543}%
\pgfsetfillcolor{currentfill}%
\pgfsetlinewidth{0.000000pt}%
\definecolor{currentstroke}{rgb}{0.000000,0.000000,0.000000}%
\pgfsetstrokecolor{currentstroke}%
\pgfsetdash{}{0pt}%
\pgfpathmoveto{\pgfqpoint{1.234872in}{0.808178in}}%
\pgfpathlineto{\pgfqpoint{1.234028in}{0.803387in}}%
\pgfpathlineto{\pgfqpoint{1.233182in}{0.798763in}}%
\pgfpathlineto{\pgfqpoint{1.232336in}{0.794310in}}%
\pgfpathlineto{\pgfqpoint{1.231488in}{0.790031in}}%
\pgfpathlineto{\pgfqpoint{1.216053in}{0.792209in}}%
\pgfpathlineto{\pgfqpoint{1.200768in}{0.794650in}}%
\pgfpathlineto{\pgfqpoint{1.185652in}{0.797349in}}%
\pgfpathlineto{\pgfqpoint{1.170719in}{0.800303in}}%
\pgfpathlineto{\pgfqpoint{1.171995in}{0.804503in}}%
\pgfpathlineto{\pgfqpoint{1.173270in}{0.808877in}}%
\pgfpathlineto{\pgfqpoint{1.174542in}{0.813422in}}%
\pgfpathlineto{\pgfqpoint{1.175813in}{0.818134in}}%
\pgfpathlineto{\pgfqpoint{1.190326in}{0.815270in}}%
\pgfpathlineto{\pgfqpoint{1.205018in}{0.812654in}}%
\pgfpathlineto{\pgfqpoint{1.219872in}{0.810289in}}%
\pgfpathlineto{\pgfqpoint{1.234872in}{0.808178in}}%
\pgfpathclose%
\pgfusepath{fill}%
\end{pgfscope}%
\begin{pgfscope}%
\pgfpathrectangle{\pgfqpoint{0.329460in}{0.284240in}}{\pgfqpoint{1.989680in}{1.989680in}}%
\pgfusepath{clip}%
\pgfsetbuttcap%
\pgfsetroundjoin%
\definecolor{currentfill}{rgb}{0.993248,0.906157,0.143936}%
\pgfsetfillcolor{currentfill}%
\pgfsetlinewidth{0.000000pt}%
\definecolor{currentstroke}{rgb}{0.000000,0.000000,0.000000}%
\pgfsetstrokecolor{currentstroke}%
\pgfsetdash{}{0pt}%
\pgfpathmoveto{\pgfqpoint{1.359641in}{1.725082in}}%
\pgfpathlineto{\pgfqpoint{1.361756in}{1.723234in}}%
\pgfpathlineto{\pgfqpoint{1.363872in}{1.721265in}}%
\pgfpathlineto{\pgfqpoint{1.365989in}{1.719176in}}%
\pgfpathlineto{\pgfqpoint{1.368107in}{1.716966in}}%
\pgfpathlineto{\pgfqpoint{1.367390in}{1.716722in}}%
\pgfpathlineto{\pgfqpoint{1.366657in}{1.716489in}}%
\pgfpathlineto{\pgfqpoint{1.365908in}{1.716266in}}%
\pgfpathlineto{\pgfqpoint{1.365145in}{1.716055in}}%
\pgfpathlineto{\pgfqpoint{1.363398in}{1.718379in}}%
\pgfpathlineto{\pgfqpoint{1.361651in}{1.720582in}}%
\pgfpathlineto{\pgfqpoint{1.359905in}{1.722665in}}%
\pgfpathlineto{\pgfqpoint{1.358160in}{1.724627in}}%
\pgfpathlineto{\pgfqpoint{1.358542in}{1.724733in}}%
\pgfpathlineto{\pgfqpoint{1.358916in}{1.724844in}}%
\pgfpathlineto{\pgfqpoint{1.359282in}{1.724960in}}%
\pgfpathlineto{\pgfqpoint{1.359641in}{1.725082in}}%
\pgfpathclose%
\pgfusepath{fill}%
\end{pgfscope}%
\begin{pgfscope}%
\pgfpathrectangle{\pgfqpoint{0.329460in}{0.284240in}}{\pgfqpoint{1.989680in}{1.989680in}}%
\pgfusepath{clip}%
\pgfsetbuttcap%
\pgfsetroundjoin%
\definecolor{currentfill}{rgb}{0.974417,0.903590,0.130215}%
\pgfsetfillcolor{currentfill}%
\pgfsetlinewidth{0.000000pt}%
\definecolor{currentstroke}{rgb}{0.000000,0.000000,0.000000}%
\pgfsetstrokecolor{currentstroke}%
\pgfsetdash{}{0pt}%
\pgfpathmoveto{\pgfqpoint{1.375201in}{1.720630in}}%
\pgfpathlineto{\pgfqpoint{1.378205in}{1.718761in}}%
\pgfpathlineto{\pgfqpoint{1.381210in}{1.716772in}}%
\pgfpathlineto{\pgfqpoint{1.384215in}{1.714665in}}%
\pgfpathlineto{\pgfqpoint{1.387220in}{1.712439in}}%
\pgfpathlineto{\pgfqpoint{1.386513in}{1.711913in}}%
\pgfpathlineto{\pgfqpoint{1.385772in}{1.711397in}}%
\pgfpathlineto{\pgfqpoint{1.384995in}{1.710892in}}%
\pgfpathlineto{\pgfqpoint{1.384184in}{1.710399in}}%
\pgfpathlineto{\pgfqpoint{1.381431in}{1.712796in}}%
\pgfpathlineto{\pgfqpoint{1.378679in}{1.715074in}}%
\pgfpathlineto{\pgfqpoint{1.375927in}{1.717233in}}%
\pgfpathlineto{\pgfqpoint{1.373175in}{1.719273in}}%
\pgfpathlineto{\pgfqpoint{1.373716in}{1.719601in}}%
\pgfpathlineto{\pgfqpoint{1.374234in}{1.719936in}}%
\pgfpathlineto{\pgfqpoint{1.374730in}{1.720280in}}%
\pgfpathlineto{\pgfqpoint{1.375201in}{1.720630in}}%
\pgfpathclose%
\pgfusepath{fill}%
\end{pgfscope}%
\begin{pgfscope}%
\pgfpathrectangle{\pgfqpoint{0.329460in}{0.284240in}}{\pgfqpoint{1.989680in}{1.989680in}}%
\pgfusepath{clip}%
\pgfsetbuttcap%
\pgfsetroundjoin%
\definecolor{currentfill}{rgb}{0.274128,0.199721,0.498911}%
\pgfsetfillcolor{currentfill}%
\pgfsetlinewidth{0.000000pt}%
\definecolor{currentstroke}{rgb}{0.000000,0.000000,0.000000}%
\pgfsetstrokecolor{currentstroke}%
\pgfsetdash{}{0pt}%
\pgfpathmoveto{\pgfqpoint{1.462255in}{0.934786in}}%
\pgfpathlineto{\pgfqpoint{1.463177in}{0.927455in}}%
\pgfpathlineto{\pgfqpoint{1.464100in}{0.920218in}}%
\pgfpathlineto{\pgfqpoint{1.465023in}{0.913079in}}%
\pgfpathlineto{\pgfqpoint{1.465947in}{0.906041in}}%
\pgfpathlineto{\pgfqpoint{1.452653in}{0.904224in}}%
\pgfpathlineto{\pgfqpoint{1.439248in}{0.902630in}}%
\pgfpathlineto{\pgfqpoint{1.425747in}{0.901263in}}%
\pgfpathlineto{\pgfqpoint{1.412163in}{0.900123in}}%
\pgfpathlineto{\pgfqpoint{1.411672in}{0.907215in}}%
\pgfpathlineto{\pgfqpoint{1.411181in}{0.914409in}}%
\pgfpathlineto{\pgfqpoint{1.410690in}{0.921699in}}%
\pgfpathlineto{\pgfqpoint{1.410200in}{0.929085in}}%
\pgfpathlineto{\pgfqpoint{1.423346in}{0.930183in}}%
\pgfpathlineto{\pgfqpoint{1.436414in}{0.931500in}}%
\pgfpathlineto{\pgfqpoint{1.449388in}{0.933035in}}%
\pgfpathlineto{\pgfqpoint{1.462255in}{0.934786in}}%
\pgfpathclose%
\pgfusepath{fill}%
\end{pgfscope}%
\begin{pgfscope}%
\pgfpathrectangle{\pgfqpoint{0.329460in}{0.284240in}}{\pgfqpoint{1.989680in}{1.989680in}}%
\pgfusepath{clip}%
\pgfsetbuttcap%
\pgfsetroundjoin%
\definecolor{currentfill}{rgb}{0.855810,0.888601,0.097452}%
\pgfsetfillcolor{currentfill}%
\pgfsetlinewidth{0.000000pt}%
\definecolor{currentstroke}{rgb}{0.000000,0.000000,0.000000}%
\pgfsetstrokecolor{currentstroke}%
\pgfsetdash{}{0pt}%
\pgfpathmoveto{\pgfqpoint{1.428173in}{1.681076in}}%
\pgfpathlineto{\pgfqpoint{1.431378in}{1.677541in}}%
\pgfpathlineto{\pgfqpoint{1.434582in}{1.673896in}}%
\pgfpathlineto{\pgfqpoint{1.437786in}{1.670142in}}%
\pgfpathlineto{\pgfqpoint{1.440988in}{1.666279in}}%
\pgfpathlineto{\pgfqpoint{1.439697in}{1.664953in}}%
\pgfpathlineto{\pgfqpoint{1.438317in}{1.663646in}}%
\pgfpathlineto{\pgfqpoint{1.436849in}{1.662361in}}%
\pgfpathlineto{\pgfqpoint{1.435295in}{1.661097in}}%
\pgfpathlineto{\pgfqpoint{1.432293in}{1.665149in}}%
\pgfpathlineto{\pgfqpoint{1.429290in}{1.669091in}}%
\pgfpathlineto{\pgfqpoint{1.426287in}{1.672924in}}%
\pgfpathlineto{\pgfqpoint{1.423283in}{1.676646in}}%
\pgfpathlineto{\pgfqpoint{1.424618in}{1.677726in}}%
\pgfpathlineto{\pgfqpoint{1.425878in}{1.678825in}}%
\pgfpathlineto{\pgfqpoint{1.427064in}{1.679942in}}%
\pgfpathlineto{\pgfqpoint{1.428173in}{1.681076in}}%
\pgfpathclose%
\pgfusepath{fill}%
\end{pgfscope}%
\begin{pgfscope}%
\pgfpathrectangle{\pgfqpoint{0.329460in}{0.284240in}}{\pgfqpoint{1.989680in}{1.989680in}}%
\pgfusepath{clip}%
\pgfsetbuttcap%
\pgfsetroundjoin%
\definecolor{currentfill}{rgb}{0.955300,0.901065,0.118128}%
\pgfsetfillcolor{currentfill}%
\pgfsetlinewidth{0.000000pt}%
\definecolor{currentstroke}{rgb}{0.000000,0.000000,0.000000}%
\pgfsetstrokecolor{currentstroke}%
\pgfsetdash{}{0pt}%
\pgfpathmoveto{\pgfqpoint{1.315781in}{1.711971in}}%
\pgfpathlineto{\pgfqpoint{1.312828in}{1.709588in}}%
\pgfpathlineto{\pgfqpoint{1.309874in}{1.707086in}}%
\pgfpathlineto{\pgfqpoint{1.306921in}{1.704468in}}%
\pgfpathlineto{\pgfqpoint{1.303967in}{1.701732in}}%
\pgfpathlineto{\pgfqpoint{1.303031in}{1.702437in}}%
\pgfpathlineto{\pgfqpoint{1.302144in}{1.703155in}}%
\pgfpathlineto{\pgfqpoint{1.301306in}{1.703886in}}%
\pgfpathlineto{\pgfqpoint{1.300518in}{1.704629in}}%
\pgfpathlineto{\pgfqpoint{1.303686in}{1.707182in}}%
\pgfpathlineto{\pgfqpoint{1.306854in}{1.709619in}}%
\pgfpathlineto{\pgfqpoint{1.310023in}{1.711938in}}%
\pgfpathlineto{\pgfqpoint{1.313191in}{1.714139in}}%
\pgfpathlineto{\pgfqpoint{1.313783in}{1.713583in}}%
\pgfpathlineto{\pgfqpoint{1.314412in}{1.713036in}}%
\pgfpathlineto{\pgfqpoint{1.315079in}{1.712498in}}%
\pgfpathlineto{\pgfqpoint{1.315781in}{1.711971in}}%
\pgfpathclose%
\pgfusepath{fill}%
\end{pgfscope}%
\begin{pgfscope}%
\pgfpathrectangle{\pgfqpoint{0.329460in}{0.284240in}}{\pgfqpoint{1.989680in}{1.989680in}}%
\pgfusepath{clip}%
\pgfsetbuttcap%
\pgfsetroundjoin%
\definecolor{currentfill}{rgb}{0.993248,0.906157,0.143936}%
\pgfsetfillcolor{currentfill}%
\pgfsetlinewidth{0.000000pt}%
\definecolor{currentstroke}{rgb}{0.000000,0.000000,0.000000}%
\pgfsetstrokecolor{currentstroke}%
\pgfsetdash{}{0pt}%
\pgfpathmoveto{\pgfqpoint{1.344560in}{1.724538in}}%
\pgfpathlineto{\pgfqpoint{1.342901in}{1.722554in}}%
\pgfpathlineto{\pgfqpoint{1.341242in}{1.720448in}}%
\pgfpathlineto{\pgfqpoint{1.339582in}{1.718223in}}%
\pgfpathlineto{\pgfqpoint{1.337921in}{1.715876in}}%
\pgfpathlineto{\pgfqpoint{1.337145in}{1.716078in}}%
\pgfpathlineto{\pgfqpoint{1.336383in}{1.716290in}}%
\pgfpathlineto{\pgfqpoint{1.335636in}{1.716514in}}%
\pgfpathlineto{\pgfqpoint{1.334905in}{1.716749in}}%
\pgfpathlineto{\pgfqpoint{1.336943in}{1.718986in}}%
\pgfpathlineto{\pgfqpoint{1.338980in}{1.721102in}}%
\pgfpathlineto{\pgfqpoint{1.341017in}{1.723098in}}%
\pgfpathlineto{\pgfqpoint{1.343052in}{1.724974in}}%
\pgfpathlineto{\pgfqpoint{1.343418in}{1.724857in}}%
\pgfpathlineto{\pgfqpoint{1.343791in}{1.724745in}}%
\pgfpathlineto{\pgfqpoint{1.344172in}{1.724639in}}%
\pgfpathlineto{\pgfqpoint{1.344560in}{1.724538in}}%
\pgfpathclose%
\pgfusepath{fill}%
\end{pgfscope}%
\begin{pgfscope}%
\pgfpathrectangle{\pgfqpoint{0.329460in}{0.284240in}}{\pgfqpoint{1.989680in}{1.989680in}}%
\pgfusepath{clip}%
\pgfsetbuttcap%
\pgfsetroundjoin%
\definecolor{currentfill}{rgb}{0.283072,0.130895,0.449241}%
\pgfsetfillcolor{currentfill}%
\pgfsetlinewidth{0.000000pt}%
\definecolor{currentstroke}{rgb}{0.000000,0.000000,0.000000}%
\pgfsetstrokecolor{currentstroke}%
\pgfsetdash{}{0pt}%
\pgfpathmoveto{\pgfqpoint{1.469649in}{0.878968in}}%
\pgfpathlineto{\pgfqpoint{1.470576in}{0.872487in}}%
\pgfpathlineto{\pgfqpoint{1.471505in}{0.866128in}}%
\pgfpathlineto{\pgfqpoint{1.472434in}{0.859894in}}%
\pgfpathlineto{\pgfqpoint{1.473364in}{0.853790in}}%
\pgfpathlineto{\pgfqpoint{1.459213in}{0.851840in}}%
\pgfpathlineto{\pgfqpoint{1.444943in}{0.850130in}}%
\pgfpathlineto{\pgfqpoint{1.430569in}{0.848663in}}%
\pgfpathlineto{\pgfqpoint{1.416108in}{0.847440in}}%
\pgfpathlineto{\pgfqpoint{1.415613in}{0.853599in}}%
\pgfpathlineto{\pgfqpoint{1.415119in}{0.859886in}}%
\pgfpathlineto{\pgfqpoint{1.414625in}{0.866299in}}%
\pgfpathlineto{\pgfqpoint{1.414132in}{0.872834in}}%
\pgfpathlineto{\pgfqpoint{1.428154in}{0.874016in}}%
\pgfpathlineto{\pgfqpoint{1.442091in}{0.875433in}}%
\pgfpathlineto{\pgfqpoint{1.455928in}{0.877084in}}%
\pgfpathlineto{\pgfqpoint{1.469649in}{0.878968in}}%
\pgfpathclose%
\pgfusepath{fill}%
\end{pgfscope}%
\begin{pgfscope}%
\pgfpathrectangle{\pgfqpoint{0.329460in}{0.284240in}}{\pgfqpoint{1.989680in}{1.989680in}}%
\pgfusepath{clip}%
\pgfsetbuttcap%
\pgfsetroundjoin%
\definecolor{currentfill}{rgb}{0.935904,0.898570,0.108131}%
\pgfsetfillcolor{currentfill}%
\pgfsetlinewidth{0.000000pt}%
\definecolor{currentstroke}{rgb}{0.000000,0.000000,0.000000}%
\pgfsetstrokecolor{currentstroke}%
\pgfsetdash{}{0pt}%
\pgfpathmoveto{\pgfqpoint{1.402515in}{1.705299in}}%
\pgfpathlineto{\pgfqpoint{1.405723in}{1.702672in}}%
\pgfpathlineto{\pgfqpoint{1.408932in}{1.699929in}}%
\pgfpathlineto{\pgfqpoint{1.412140in}{1.697071in}}%
\pgfpathlineto{\pgfqpoint{1.415347in}{1.694098in}}%
\pgfpathlineto{\pgfqpoint{1.414421in}{1.693155in}}%
\pgfpathlineto{\pgfqpoint{1.413431in}{1.692226in}}%
\pgfpathlineto{\pgfqpoint{1.412378in}{1.691313in}}%
\pgfpathlineto{\pgfqpoint{1.411264in}{1.690415in}}%
\pgfpathlineto{\pgfqpoint{1.408259in}{1.693573in}}%
\pgfpathlineto{\pgfqpoint{1.405254in}{1.696617in}}%
\pgfpathlineto{\pgfqpoint{1.402248in}{1.699546in}}%
\pgfpathlineto{\pgfqpoint{1.399242in}{1.702358in}}%
\pgfpathlineto{\pgfqpoint{1.400135in}{1.703075in}}%
\pgfpathlineto{\pgfqpoint{1.400978in}{1.703804in}}%
\pgfpathlineto{\pgfqpoint{1.401772in}{1.704546in}}%
\pgfpathlineto{\pgfqpoint{1.402515in}{1.705299in}}%
\pgfpathclose%
\pgfusepath{fill}%
\end{pgfscope}%
\begin{pgfscope}%
\pgfpathrectangle{\pgfqpoint{0.329460in}{0.284240in}}{\pgfqpoint{1.989680in}{1.989680in}}%
\pgfusepath{clip}%
\pgfsetbuttcap%
\pgfsetroundjoin%
\definecolor{currentfill}{rgb}{0.565498,0.842430,0.262877}%
\pgfsetfillcolor{currentfill}%
\pgfsetlinewidth{0.000000pt}%
\definecolor{currentstroke}{rgb}{0.000000,0.000000,0.000000}%
\pgfsetstrokecolor{currentstroke}%
\pgfsetdash{}{0pt}%
\pgfpathmoveto{\pgfqpoint{1.232892in}{1.573056in}}%
\pgfpathlineto{\pgfqpoint{1.230212in}{1.567099in}}%
\pgfpathlineto{\pgfqpoint{1.227534in}{1.561054in}}%
\pgfpathlineto{\pgfqpoint{1.224857in}{1.554923in}}%
\pgfpathlineto{\pgfqpoint{1.222180in}{1.548706in}}%
\pgfpathlineto{\pgfqpoint{1.218853in}{1.550688in}}%
\pgfpathlineto{\pgfqpoint{1.215661in}{1.552719in}}%
\pgfpathlineto{\pgfqpoint{1.212607in}{1.554796in}}%
\pgfpathlineto{\pgfqpoint{1.209694in}{1.556918in}}%
\pgfpathlineto{\pgfqpoint{1.212626in}{1.562957in}}%
\pgfpathlineto{\pgfqpoint{1.215559in}{1.568910in}}%
\pgfpathlineto{\pgfqpoint{1.218493in}{1.574777in}}%
\pgfpathlineto{\pgfqpoint{1.221428in}{1.580556in}}%
\pgfpathlineto{\pgfqpoint{1.224103in}{1.578618in}}%
\pgfpathlineto{\pgfqpoint{1.226907in}{1.576721in}}%
\pgfpathlineto{\pgfqpoint{1.229837in}{1.574866in}}%
\pgfpathlineto{\pgfqpoint{1.232892in}{1.573056in}}%
\pgfpathclose%
\pgfusepath{fill}%
\end{pgfscope}%
\begin{pgfscope}%
\pgfpathrectangle{\pgfqpoint{0.329460in}{0.284240in}}{\pgfqpoint{1.989680in}{1.989680in}}%
\pgfusepath{clip}%
\pgfsetbuttcap%
\pgfsetroundjoin%
\definecolor{currentfill}{rgb}{0.814576,0.883393,0.110347}%
\pgfsetfillcolor{currentfill}%
\pgfsetlinewidth{0.000000pt}%
\definecolor{currentstroke}{rgb}{0.000000,0.000000,0.000000}%
\pgfsetstrokecolor{currentstroke}%
\pgfsetdash{}{0pt}%
\pgfpathmoveto{\pgfqpoint{1.268533in}{1.659994in}}%
\pgfpathlineto{\pgfqpoint{1.265583in}{1.655795in}}%
\pgfpathlineto{\pgfqpoint{1.262634in}{1.651489in}}%
\pgfpathlineto{\pgfqpoint{1.259685in}{1.647076in}}%
\pgfpathlineto{\pgfqpoint{1.256736in}{1.642559in}}%
\pgfpathlineto{\pgfqpoint{1.254877in}{1.643983in}}%
\pgfpathlineto{\pgfqpoint{1.253115in}{1.645434in}}%
\pgfpathlineto{\pgfqpoint{1.251451in}{1.646910in}}%
\pgfpathlineto{\pgfqpoint{1.249889in}{1.648409in}}%
\pgfpathlineto{\pgfqpoint{1.253048in}{1.652740in}}%
\pgfpathlineto{\pgfqpoint{1.256208in}{1.656966in}}%
\pgfpathlineto{\pgfqpoint{1.259368in}{1.661086in}}%
\pgfpathlineto{\pgfqpoint{1.262530in}{1.665099in}}%
\pgfpathlineto{\pgfqpoint{1.263900in}{1.663791in}}%
\pgfpathlineto{\pgfqpoint{1.265359in}{1.662503in}}%
\pgfpathlineto{\pgfqpoint{1.266904in}{1.661237in}}%
\pgfpathlineto{\pgfqpoint{1.268533in}{1.659994in}}%
\pgfpathclose%
\pgfusepath{fill}%
\end{pgfscope}%
\begin{pgfscope}%
\pgfpathrectangle{\pgfqpoint{0.329460in}{0.284240in}}{\pgfqpoint{1.989680in}{1.989680in}}%
\pgfusepath{clip}%
\pgfsetbuttcap%
\pgfsetroundjoin%
\definecolor{currentfill}{rgb}{0.974417,0.903590,0.130215}%
\pgfsetfillcolor{currentfill}%
\pgfsetlinewidth{0.000000pt}%
\definecolor{currentstroke}{rgb}{0.000000,0.000000,0.000000}%
\pgfsetstrokecolor{currentstroke}%
\pgfsetdash{}{0pt}%
\pgfpathmoveto{\pgfqpoint{1.329699in}{1.718988in}}%
\pgfpathlineto{\pgfqpoint{1.327010in}{1.716913in}}%
\pgfpathlineto{\pgfqpoint{1.324320in}{1.714718in}}%
\pgfpathlineto{\pgfqpoint{1.321630in}{1.712404in}}%
\pgfpathlineto{\pgfqpoint{1.318939in}{1.709971in}}%
\pgfpathlineto{\pgfqpoint{1.318099in}{1.710454in}}%
\pgfpathlineto{\pgfqpoint{1.317292in}{1.710948in}}%
\pgfpathlineto{\pgfqpoint{1.316519in}{1.711454in}}%
\pgfpathlineto{\pgfqpoint{1.315781in}{1.711971in}}%
\pgfpathlineto{\pgfqpoint{1.318735in}{1.714236in}}%
\pgfpathlineto{\pgfqpoint{1.321687in}{1.716382in}}%
\pgfpathlineto{\pgfqpoint{1.324640in}{1.718410in}}%
\pgfpathlineto{\pgfqpoint{1.327592in}{1.720318in}}%
\pgfpathlineto{\pgfqpoint{1.328085in}{1.719974in}}%
\pgfpathlineto{\pgfqpoint{1.328600in}{1.719638in}}%
\pgfpathlineto{\pgfqpoint{1.329139in}{1.719309in}}%
\pgfpathlineto{\pgfqpoint{1.329699in}{1.718988in}}%
\pgfpathclose%
\pgfusepath{fill}%
\end{pgfscope}%
\begin{pgfscope}%
\pgfpathrectangle{\pgfqpoint{0.329460in}{0.284240in}}{\pgfqpoint{1.989680in}{1.989680in}}%
\pgfusepath{clip}%
\pgfsetbuttcap%
\pgfsetroundjoin%
\definecolor{currentfill}{rgb}{0.896320,0.893616,0.096335}%
\pgfsetfillcolor{currentfill}%
\pgfsetlinewidth{0.000000pt}%
\definecolor{currentstroke}{rgb}{0.000000,0.000000,0.000000}%
\pgfsetstrokecolor{currentstroke}%
\pgfsetdash{}{0pt}%
\pgfpathmoveto{\pgfqpoint{1.415347in}{1.694098in}}%
\pgfpathlineto{\pgfqpoint{1.418555in}{1.691012in}}%
\pgfpathlineto{\pgfqpoint{1.421761in}{1.687812in}}%
\pgfpathlineto{\pgfqpoint{1.424967in}{1.684500in}}%
\pgfpathlineto{\pgfqpoint{1.428173in}{1.681076in}}%
\pgfpathlineto{\pgfqpoint{1.427064in}{1.679942in}}%
\pgfpathlineto{\pgfqpoint{1.425878in}{1.678825in}}%
\pgfpathlineto{\pgfqpoint{1.424618in}{1.677726in}}%
\pgfpathlineto{\pgfqpoint{1.423283in}{1.676646in}}%
\pgfpathlineto{\pgfqpoint{1.420279in}{1.680257in}}%
\pgfpathlineto{\pgfqpoint{1.417274in}{1.683756in}}%
\pgfpathlineto{\pgfqpoint{1.414269in}{1.687142in}}%
\pgfpathlineto{\pgfqpoint{1.411264in}{1.690415in}}%
\pgfpathlineto{\pgfqpoint{1.412378in}{1.691313in}}%
\pgfpathlineto{\pgfqpoint{1.413431in}{1.692226in}}%
\pgfpathlineto{\pgfqpoint{1.414421in}{1.693155in}}%
\pgfpathlineto{\pgfqpoint{1.415347in}{1.694098in}}%
\pgfpathclose%
\pgfusepath{fill}%
\end{pgfscope}%
\begin{pgfscope}%
\pgfpathrectangle{\pgfqpoint{0.329460in}{0.284240in}}{\pgfqpoint{1.989680in}{1.989680in}}%
\pgfusepath{clip}%
\pgfsetbuttcap%
\pgfsetroundjoin%
\definecolor{currentfill}{rgb}{0.993248,0.906157,0.143936}%
\pgfsetfillcolor{currentfill}%
\pgfsetlinewidth{0.000000pt}%
\definecolor{currentstroke}{rgb}{0.000000,0.000000,0.000000}%
\pgfsetstrokecolor{currentstroke}%
\pgfsetdash{}{0pt}%
\pgfpathmoveto{\pgfqpoint{1.358160in}{1.724627in}}%
\pgfpathlineto{\pgfqpoint{1.359905in}{1.722665in}}%
\pgfpathlineto{\pgfqpoint{1.361651in}{1.720582in}}%
\pgfpathlineto{\pgfqpoint{1.363398in}{1.718379in}}%
\pgfpathlineto{\pgfqpoint{1.365145in}{1.716055in}}%
\pgfpathlineto{\pgfqpoint{1.364367in}{1.715855in}}%
\pgfpathlineto{\pgfqpoint{1.363576in}{1.715666in}}%
\pgfpathlineto{\pgfqpoint{1.362773in}{1.715489in}}%
\pgfpathlineto{\pgfqpoint{1.361959in}{1.715325in}}%
\pgfpathlineto{\pgfqpoint{1.360610in}{1.717740in}}%
\pgfpathlineto{\pgfqpoint{1.359262in}{1.720035in}}%
\pgfpathlineto{\pgfqpoint{1.357915in}{1.722209in}}%
\pgfpathlineto{\pgfqpoint{1.356568in}{1.724263in}}%
\pgfpathlineto{\pgfqpoint{1.356975in}{1.724345in}}%
\pgfpathlineto{\pgfqpoint{1.357377in}{1.724434in}}%
\pgfpathlineto{\pgfqpoint{1.357772in}{1.724528in}}%
\pgfpathlineto{\pgfqpoint{1.358160in}{1.724627in}}%
\pgfpathclose%
\pgfusepath{fill}%
\end{pgfscope}%
\begin{pgfscope}%
\pgfpathrectangle{\pgfqpoint{0.329460in}{0.284240in}}{\pgfqpoint{1.989680in}{1.989680in}}%
\pgfusepath{clip}%
\pgfsetbuttcap%
\pgfsetroundjoin%
\definecolor{currentfill}{rgb}{0.163625,0.471133,0.558148}%
\pgfsetfillcolor{currentfill}%
\pgfsetlinewidth{0.000000pt}%
\definecolor{currentstroke}{rgb}{0.000000,0.000000,0.000000}%
\pgfsetstrokecolor{currentstroke}%
\pgfsetdash{}{0pt}%
\pgfpathmoveto{\pgfqpoint{1.235706in}{1.170617in}}%
\pgfpathlineto{\pgfqpoint{1.234464in}{1.161964in}}%
\pgfpathlineto{\pgfqpoint{1.233222in}{1.153322in}}%
\pgfpathlineto{\pgfqpoint{1.231981in}{1.144693in}}%
\pgfpathlineto{\pgfqpoint{1.230739in}{1.136080in}}%
\pgfpathlineto{\pgfqpoint{1.220886in}{1.138113in}}%
\pgfpathlineto{\pgfqpoint{1.211172in}{1.140305in}}%
\pgfpathlineto{\pgfqpoint{1.201608in}{1.142651in}}%
\pgfpathlineto{\pgfqpoint{1.192204in}{1.145151in}}%
\pgfpathlineto{\pgfqpoint{1.193840in}{1.153658in}}%
\pgfpathlineto{\pgfqpoint{1.195477in}{1.162182in}}%
\pgfpathlineto{\pgfqpoint{1.197114in}{1.170719in}}%
\pgfpathlineto{\pgfqpoint{1.198751in}{1.179267in}}%
\pgfpathlineto{\pgfqpoint{1.207770in}{1.176883in}}%
\pgfpathlineto{\pgfqpoint{1.216942in}{1.174646in}}%
\pgfpathlineto{\pgfqpoint{1.226258in}{1.172556in}}%
\pgfpathlineto{\pgfqpoint{1.235706in}{1.170617in}}%
\pgfpathclose%
\pgfusepath{fill}%
\end{pgfscope}%
\begin{pgfscope}%
\pgfpathrectangle{\pgfqpoint{0.329460in}{0.284240in}}{\pgfqpoint{1.989680in}{1.989680in}}%
\pgfusepath{clip}%
\pgfsetbuttcap%
\pgfsetroundjoin%
\definecolor{currentfill}{rgb}{0.263663,0.237631,0.518762}%
\pgfsetfillcolor{currentfill}%
\pgfsetlinewidth{0.000000pt}%
\definecolor{currentstroke}{rgb}{0.000000,0.000000,0.000000}%
\pgfsetstrokecolor{currentstroke}%
\pgfsetdash{}{0pt}%
\pgfpathmoveto{\pgfqpoint{1.458571in}{0.964990in}}%
\pgfpathlineto{\pgfqpoint{1.459491in}{0.957314in}}%
\pgfpathlineto{\pgfqpoint{1.460412in}{0.949719in}}%
\pgfpathlineto{\pgfqpoint{1.461333in}{0.942209in}}%
\pgfpathlineto{\pgfqpoint{1.462255in}{0.934786in}}%
\pgfpathlineto{\pgfqpoint{1.449388in}{0.933035in}}%
\pgfpathlineto{\pgfqpoint{1.436414in}{0.931500in}}%
\pgfpathlineto{\pgfqpoint{1.423346in}{0.930183in}}%
\pgfpathlineto{\pgfqpoint{1.410200in}{0.929085in}}%
\pgfpathlineto{\pgfqpoint{1.409709in}{0.936561in}}%
\pgfpathlineto{\pgfqpoint{1.409220in}{0.944125in}}%
\pgfpathlineto{\pgfqpoint{1.408730in}{0.951773in}}%
\pgfpathlineto{\pgfqpoint{1.408241in}{0.959503in}}%
\pgfpathlineto{\pgfqpoint{1.420951in}{0.960560in}}%
\pgfpathlineto{\pgfqpoint{1.433586in}{0.961827in}}%
\pgfpathlineto{\pgfqpoint{1.446130in}{0.963305in}}%
\pgfpathlineto{\pgfqpoint{1.458571in}{0.964990in}}%
\pgfpathclose%
\pgfusepath{fill}%
\end{pgfscope}%
\begin{pgfscope}%
\pgfpathrectangle{\pgfqpoint{0.329460in}{0.284240in}}{\pgfqpoint{1.989680in}{1.989680in}}%
\pgfusepath{clip}%
\pgfsetbuttcap%
\pgfsetroundjoin%
\definecolor{currentfill}{rgb}{0.212395,0.359683,0.551710}%
\pgfsetfillcolor{currentfill}%
\pgfsetlinewidth{0.000000pt}%
\definecolor{currentstroke}{rgb}{0.000000,0.000000,0.000000}%
\pgfsetstrokecolor{currentstroke}%
\pgfsetdash{}{0pt}%
\pgfpathmoveto{\pgfqpoint{1.264750in}{1.060959in}}%
\pgfpathlineto{\pgfqpoint{1.263926in}{1.052538in}}%
\pgfpathlineto{\pgfqpoint{1.263102in}{1.044161in}}%
\pgfpathlineto{\pgfqpoint{1.262279in}{1.035831in}}%
\pgfpathlineto{\pgfqpoint{1.261455in}{1.027551in}}%
\pgfpathlineto{\pgfqpoint{1.249876in}{1.029127in}}%
\pgfpathlineto{\pgfqpoint{1.238407in}{1.030892in}}%
\pgfpathlineto{\pgfqpoint{1.227060in}{1.032845in}}%
\pgfpathlineto{\pgfqpoint{1.215847in}{1.034983in}}%
\pgfpathlineto{\pgfqpoint{1.217089in}{1.043185in}}%
\pgfpathlineto{\pgfqpoint{1.218330in}{1.051438in}}%
\pgfpathlineto{\pgfqpoint{1.219570in}{1.059737in}}%
\pgfpathlineto{\pgfqpoint{1.220811in}{1.068081in}}%
\pgfpathlineto{\pgfqpoint{1.231614in}{1.066032in}}%
\pgfpathlineto{\pgfqpoint{1.242546in}{1.064161in}}%
\pgfpathlineto{\pgfqpoint{1.253595in}{1.062469in}}%
\pgfpathlineto{\pgfqpoint{1.264750in}{1.060959in}}%
\pgfpathclose%
\pgfusepath{fill}%
\end{pgfscope}%
\begin{pgfscope}%
\pgfpathrectangle{\pgfqpoint{0.329460in}{0.284240in}}{\pgfqpoint{1.989680in}{1.989680in}}%
\pgfusepath{clip}%
\pgfsetbuttcap%
\pgfsetroundjoin%
\definecolor{currentfill}{rgb}{0.993248,0.906157,0.143936}%
\pgfsetfillcolor{currentfill}%
\pgfsetlinewidth{0.000000pt}%
\definecolor{currentstroke}{rgb}{0.000000,0.000000,0.000000}%
\pgfsetstrokecolor{currentstroke}%
\pgfsetdash{}{0pt}%
\pgfpathmoveto{\pgfqpoint{1.346173in}{1.724195in}}%
\pgfpathlineto{\pgfqpoint{1.344918in}{1.722124in}}%
\pgfpathlineto{\pgfqpoint{1.343662in}{1.719933in}}%
\pgfpathlineto{\pgfqpoint{1.342406in}{1.717621in}}%
\pgfpathlineto{\pgfqpoint{1.341150in}{1.715188in}}%
\pgfpathlineto{\pgfqpoint{1.340325in}{1.715342in}}%
\pgfpathlineto{\pgfqpoint{1.339512in}{1.715509in}}%
\pgfpathlineto{\pgfqpoint{1.338710in}{1.715687in}}%
\pgfpathlineto{\pgfqpoint{1.337921in}{1.715876in}}%
\pgfpathlineto{\pgfqpoint{1.339582in}{1.718223in}}%
\pgfpathlineto{\pgfqpoint{1.341242in}{1.720448in}}%
\pgfpathlineto{\pgfqpoint{1.342901in}{1.722554in}}%
\pgfpathlineto{\pgfqpoint{1.344560in}{1.724538in}}%
\pgfpathlineto{\pgfqpoint{1.344954in}{1.724444in}}%
\pgfpathlineto{\pgfqpoint{1.345355in}{1.724355in}}%
\pgfpathlineto{\pgfqpoint{1.345761in}{1.724272in}}%
\pgfpathlineto{\pgfqpoint{1.346173in}{1.724195in}}%
\pgfpathclose%
\pgfusepath{fill}%
\end{pgfscope}%
\begin{pgfscope}%
\pgfpathrectangle{\pgfqpoint{0.329460in}{0.284240in}}{\pgfqpoint{1.989680in}{1.989680in}}%
\pgfusepath{clip}%
\pgfsetbuttcap%
\pgfsetroundjoin%
\definecolor{currentfill}{rgb}{0.636902,0.856542,0.216620}%
\pgfsetfillcolor{currentfill}%
\pgfsetlinewidth{0.000000pt}%
\definecolor{currentstroke}{rgb}{0.000000,0.000000,0.000000}%
\pgfsetstrokecolor{currentstroke}%
\pgfsetdash{}{0pt}%
\pgfpathmoveto{\pgfqpoint{1.471259in}{1.604356in}}%
\pgfpathlineto{\pgfqpoint{1.474249in}{1.598984in}}%
\pgfpathlineto{\pgfqpoint{1.477239in}{1.593518in}}%
\pgfpathlineto{\pgfqpoint{1.480227in}{1.587960in}}%
\pgfpathlineto{\pgfqpoint{1.483214in}{1.582312in}}%
\pgfpathlineto{\pgfqpoint{1.480656in}{1.580339in}}%
\pgfpathlineto{\pgfqpoint{1.477967in}{1.578405in}}%
\pgfpathlineto{\pgfqpoint{1.475149in}{1.576512in}}%
\pgfpathlineto{\pgfqpoint{1.472204in}{1.574662in}}%
\pgfpathlineto{\pgfqpoint{1.469463in}{1.580491in}}%
\pgfpathlineto{\pgfqpoint{1.466721in}{1.586229in}}%
\pgfpathlineto{\pgfqpoint{1.463977in}{1.591874in}}%
\pgfpathlineto{\pgfqpoint{1.461233in}{1.597426in}}%
\pgfpathlineto{\pgfqpoint{1.463914in}{1.599102in}}%
\pgfpathlineto{\pgfqpoint{1.466480in}{1.600816in}}%
\pgfpathlineto{\pgfqpoint{1.468929in}{1.602568in}}%
\pgfpathlineto{\pgfqpoint{1.471259in}{1.604356in}}%
\pgfpathclose%
\pgfusepath{fill}%
\end{pgfscope}%
\begin{pgfscope}%
\pgfpathrectangle{\pgfqpoint{0.329460in}{0.284240in}}{\pgfqpoint{1.989680in}{1.989680in}}%
\pgfusepath{clip}%
\pgfsetbuttcap%
\pgfsetroundjoin%
\definecolor{currentfill}{rgb}{0.122606,0.585371,0.546557}%
\pgfsetfillcolor{currentfill}%
\pgfsetlinewidth{0.000000pt}%
\definecolor{currentstroke}{rgb}{0.000000,0.000000,0.000000}%
\pgfsetstrokecolor{currentstroke}%
\pgfsetdash{}{0pt}%
\pgfpathmoveto{\pgfqpoint{1.218433in}{1.281733in}}%
\pgfpathlineto{\pgfqpoint{1.216790in}{1.273257in}}%
\pgfpathlineto{\pgfqpoint{1.215148in}{1.264762in}}%
\pgfpathlineto{\pgfqpoint{1.213506in}{1.256249in}}%
\pgfpathlineto{\pgfqpoint{1.211865in}{1.247722in}}%
\pgfpathlineto{\pgfqpoint{1.203765in}{1.250006in}}%
\pgfpathlineto{\pgfqpoint{1.195821in}{1.252418in}}%
\pgfpathlineto{\pgfqpoint{1.188041in}{1.254953in}}%
\pgfpathlineto{\pgfqpoint{1.180434in}{1.257610in}}%
\pgfpathlineto{\pgfqpoint{1.182442in}{1.266008in}}%
\pgfpathlineto{\pgfqpoint{1.184452in}{1.274391in}}%
\pgfpathlineto{\pgfqpoint{1.186462in}{1.282757in}}%
\pgfpathlineto{\pgfqpoint{1.188472in}{1.291104in}}%
\pgfpathlineto{\pgfqpoint{1.195725in}{1.288586in}}%
\pgfpathlineto{\pgfqpoint{1.203141in}{1.286183in}}%
\pgfpathlineto{\pgfqpoint{1.210713in}{1.283898in}}%
\pgfpathlineto{\pgfqpoint{1.218433in}{1.281733in}}%
\pgfpathclose%
\pgfusepath{fill}%
\end{pgfscope}%
\begin{pgfscope}%
\pgfpathrectangle{\pgfqpoint{0.329460in}{0.284240in}}{\pgfqpoint{1.989680in}{1.989680in}}%
\pgfusepath{clip}%
\pgfsetbuttcap%
\pgfsetroundjoin%
\definecolor{currentfill}{rgb}{0.280255,0.165693,0.476498}%
\pgfsetfillcolor{currentfill}%
\pgfsetlinewidth{0.000000pt}%
\definecolor{currentstroke}{rgb}{0.000000,0.000000,0.000000}%
\pgfsetstrokecolor{currentstroke}%
\pgfsetdash{}{0pt}%
\pgfpathmoveto{\pgfqpoint{1.302343in}{0.899302in}}%
\pgfpathlineto{\pgfqpoint{1.301950in}{0.892307in}}%
\pgfpathlineto{\pgfqpoint{1.301555in}{0.885420in}}%
\pgfpathlineto{\pgfqpoint{1.301161in}{0.878644in}}%
\pgfpathlineto{\pgfqpoint{1.300766in}{0.871983in}}%
\pgfpathlineto{\pgfqpoint{1.286682in}{0.872954in}}%
\pgfpathlineto{\pgfqpoint{1.272668in}{0.874162in}}%
\pgfpathlineto{\pgfqpoint{1.258742in}{0.875605in}}%
\pgfpathlineto{\pgfqpoint{1.244917in}{0.877282in}}%
\pgfpathlineto{\pgfqpoint{1.245748in}{0.883896in}}%
\pgfpathlineto{\pgfqpoint{1.246579in}{0.890626in}}%
\pgfpathlineto{\pgfqpoint{1.247409in}{0.897466in}}%
\pgfpathlineto{\pgfqpoint{1.248239in}{0.904414in}}%
\pgfpathlineto{\pgfqpoint{1.261632in}{0.902796in}}%
\pgfpathlineto{\pgfqpoint{1.275124in}{0.901404in}}%
\pgfpathlineto{\pgfqpoint{1.288699in}{0.900239in}}%
\pgfpathlineto{\pgfqpoint{1.302343in}{0.899302in}}%
\pgfpathclose%
\pgfusepath{fill}%
\end{pgfscope}%
\begin{pgfscope}%
\pgfpathrectangle{\pgfqpoint{0.329460in}{0.284240in}}{\pgfqpoint{1.989680in}{1.989680in}}%
\pgfusepath{clip}%
\pgfsetbuttcap%
\pgfsetroundjoin%
\definecolor{currentfill}{rgb}{0.344074,0.780029,0.397381}%
\pgfsetfillcolor{currentfill}%
\pgfsetlinewidth{0.000000pt}%
\definecolor{currentstroke}{rgb}{0.000000,0.000000,0.000000}%
\pgfsetstrokecolor{currentstroke}%
\pgfsetdash{}{0pt}%
\pgfpathmoveto{\pgfqpoint{1.217794in}{1.487380in}}%
\pgfpathlineto{\pgfqpoint{1.215425in}{1.480302in}}%
\pgfpathlineto{\pgfqpoint{1.213058in}{1.473154in}}%
\pgfpathlineto{\pgfqpoint{1.210692in}{1.465939in}}%
\pgfpathlineto{\pgfqpoint{1.208326in}{1.458657in}}%
\pgfpathlineto{\pgfqpoint{1.203550in}{1.460894in}}%
\pgfpathlineto{\pgfqpoint{1.198926in}{1.463202in}}%
\pgfpathlineto{\pgfqpoint{1.194459in}{1.465579in}}%
\pgfpathlineto{\pgfqpoint{1.190153in}{1.468022in}}%
\pgfpathlineto{\pgfqpoint{1.192815in}{1.475138in}}%
\pgfpathlineto{\pgfqpoint{1.195479in}{1.482187in}}%
\pgfpathlineto{\pgfqpoint{1.198144in}{1.489170in}}%
\pgfpathlineto{\pgfqpoint{1.200809in}{1.496084in}}%
\pgfpathlineto{\pgfqpoint{1.204834in}{1.493813in}}%
\pgfpathlineto{\pgfqpoint{1.209009in}{1.491604in}}%
\pgfpathlineto{\pgfqpoint{1.213330in}{1.489459in}}%
\pgfpathlineto{\pgfqpoint{1.217794in}{1.487380in}}%
\pgfpathclose%
\pgfusepath{fill}%
\end{pgfscope}%
\begin{pgfscope}%
\pgfpathrectangle{\pgfqpoint{0.329460in}{0.284240in}}{\pgfqpoint{1.989680in}{1.989680in}}%
\pgfusepath{clip}%
\pgfsetbuttcap%
\pgfsetroundjoin%
\definecolor{currentfill}{rgb}{0.935904,0.898570,0.108131}%
\pgfsetfillcolor{currentfill}%
\pgfsetlinewidth{0.000000pt}%
\definecolor{currentstroke}{rgb}{0.000000,0.000000,0.000000}%
\pgfsetstrokecolor{currentstroke}%
\pgfsetdash{}{0pt}%
\pgfpathmoveto{\pgfqpoint{1.303967in}{1.701732in}}%
\pgfpathlineto{\pgfqpoint{1.301013in}{1.698880in}}%
\pgfpathlineto{\pgfqpoint{1.298059in}{1.695912in}}%
\pgfpathlineto{\pgfqpoint{1.295105in}{1.692829in}}%
\pgfpathlineto{\pgfqpoint{1.292152in}{1.689630in}}%
\pgfpathlineto{\pgfqpoint{1.290984in}{1.690514in}}%
\pgfpathlineto{\pgfqpoint{1.289877in}{1.691413in}}%
\pgfpathlineto{\pgfqpoint{1.288831in}{1.692329in}}%
\pgfpathlineto{\pgfqpoint{1.287848in}{1.693259in}}%
\pgfpathlineto{\pgfqpoint{1.291015in}{1.696274in}}%
\pgfpathlineto{\pgfqpoint{1.294183in}{1.699175in}}%
\pgfpathlineto{\pgfqpoint{1.297350in}{1.701960in}}%
\pgfpathlineto{\pgfqpoint{1.300518in}{1.704629in}}%
\pgfpathlineto{\pgfqpoint{1.301306in}{1.703886in}}%
\pgfpathlineto{\pgfqpoint{1.302144in}{1.703155in}}%
\pgfpathlineto{\pgfqpoint{1.303031in}{1.702437in}}%
\pgfpathlineto{\pgfqpoint{1.303967in}{1.701732in}}%
\pgfpathclose%
\pgfusepath{fill}%
\end{pgfscope}%
\begin{pgfscope}%
\pgfpathrectangle{\pgfqpoint{0.329460in}{0.284240in}}{\pgfqpoint{1.989680in}{1.989680in}}%
\pgfusepath{clip}%
\pgfsetbuttcap%
\pgfsetroundjoin%
\definecolor{currentfill}{rgb}{0.274128,0.199721,0.498911}%
\pgfsetfillcolor{currentfill}%
\pgfsetlinewidth{0.000000pt}%
\definecolor{currentstroke}{rgb}{0.000000,0.000000,0.000000}%
\pgfsetstrokecolor{currentstroke}%
\pgfsetdash{}{0pt}%
\pgfpathmoveto{\pgfqpoint{1.303916in}{0.928293in}}%
\pgfpathlineto{\pgfqpoint{1.303524in}{0.920901in}}%
\pgfpathlineto{\pgfqpoint{1.303130in}{0.913602in}}%
\pgfpathlineto{\pgfqpoint{1.302737in}{0.906402in}}%
\pgfpathlineto{\pgfqpoint{1.302343in}{0.899302in}}%
\pgfpathlineto{\pgfqpoint{1.288699in}{0.900239in}}%
\pgfpathlineto{\pgfqpoint{1.275124in}{0.901404in}}%
\pgfpathlineto{\pgfqpoint{1.261632in}{0.902796in}}%
\pgfpathlineto{\pgfqpoint{1.248239in}{0.904414in}}%
\pgfpathlineto{\pgfqpoint{1.249068in}{0.911467in}}%
\pgfpathlineto{\pgfqpoint{1.249896in}{0.918621in}}%
\pgfpathlineto{\pgfqpoint{1.250724in}{0.925873in}}%
\pgfpathlineto{\pgfqpoint{1.251552in}{0.933219in}}%
\pgfpathlineto{\pgfqpoint{1.264515in}{0.931660in}}%
\pgfpathlineto{\pgfqpoint{1.277573in}{0.930318in}}%
\pgfpathlineto{\pgfqpoint{1.290711in}{0.929196in}}%
\pgfpathlineto{\pgfqpoint{1.303916in}{0.928293in}}%
\pgfpathclose%
\pgfusepath{fill}%
\end{pgfscope}%
\begin{pgfscope}%
\pgfpathrectangle{\pgfqpoint{0.329460in}{0.284240in}}{\pgfqpoint{1.989680in}{1.989680in}}%
\pgfusepath{clip}%
\pgfsetbuttcap%
\pgfsetroundjoin%
\definecolor{currentfill}{rgb}{0.271305,0.019942,0.347269}%
\pgfsetfillcolor{currentfill}%
\pgfsetlinewidth{0.000000pt}%
\definecolor{currentstroke}{rgb}{0.000000,0.000000,0.000000}%
\pgfsetstrokecolor{currentstroke}%
\pgfsetdash{}{0pt}%
\pgfpathmoveto{\pgfqpoint{1.544763in}{0.803141in}}%
\pgfpathlineto{\pgfqpoint{1.546134in}{0.799142in}}%
\pgfpathlineto{\pgfqpoint{1.547507in}{0.795325in}}%
\pgfpathlineto{\pgfqpoint{1.548883in}{0.791695in}}%
\pgfpathlineto{\pgfqpoint{1.550261in}{0.788256in}}%
\pgfpathlineto{\pgfqpoint{1.535089in}{0.784980in}}%
\pgfpathlineto{\pgfqpoint{1.519712in}{0.781964in}}%
\pgfpathlineto{\pgfqpoint{1.504147in}{0.779211in}}%
\pgfpathlineto{\pgfqpoint{1.488411in}{0.776726in}}%
\pgfpathlineto{\pgfqpoint{1.487459in}{0.780251in}}%
\pgfpathlineto{\pgfqpoint{1.486509in}{0.783966in}}%
\pgfpathlineto{\pgfqpoint{1.485561in}{0.787869in}}%
\pgfpathlineto{\pgfqpoint{1.484614in}{0.791954in}}%
\pgfpathlineto{\pgfqpoint{1.499916in}{0.794366in}}%
\pgfpathlineto{\pgfqpoint{1.515053in}{0.797036in}}%
\pgfpathlineto{\pgfqpoint{1.530007in}{0.799962in}}%
\pgfpathlineto{\pgfqpoint{1.544763in}{0.803141in}}%
\pgfpathclose%
\pgfusepath{fill}%
\end{pgfscope}%
\begin{pgfscope}%
\pgfpathrectangle{\pgfqpoint{0.329460in}{0.284240in}}{\pgfqpoint{1.989680in}{1.989680in}}%
\pgfusepath{clip}%
\pgfsetbuttcap%
\pgfsetroundjoin%
\definecolor{currentfill}{rgb}{0.993248,0.906157,0.143936}%
\pgfsetfillcolor{currentfill}%
\pgfsetlinewidth{0.000000pt}%
\definecolor{currentstroke}{rgb}{0.000000,0.000000,0.000000}%
\pgfsetstrokecolor{currentstroke}%
\pgfsetdash{}{0pt}%
\pgfpathmoveto{\pgfqpoint{1.356568in}{1.724263in}}%
\pgfpathlineto{\pgfqpoint{1.357915in}{1.722209in}}%
\pgfpathlineto{\pgfqpoint{1.359262in}{1.720035in}}%
\pgfpathlineto{\pgfqpoint{1.360610in}{1.717740in}}%
\pgfpathlineto{\pgfqpoint{1.361959in}{1.715325in}}%
\pgfpathlineto{\pgfqpoint{1.361133in}{1.715172in}}%
\pgfpathlineto{\pgfqpoint{1.360298in}{1.715031in}}%
\pgfpathlineto{\pgfqpoint{1.359453in}{1.714903in}}%
\pgfpathlineto{\pgfqpoint{1.358600in}{1.714788in}}%
\pgfpathlineto{\pgfqpoint{1.357672in}{1.717270in}}%
\pgfpathlineto{\pgfqpoint{1.356744in}{1.719633in}}%
\pgfpathlineto{\pgfqpoint{1.355817in}{1.721874in}}%
\pgfpathlineto{\pgfqpoint{1.354890in}{1.723995in}}%
\pgfpathlineto{\pgfqpoint{1.355316in}{1.724053in}}%
\pgfpathlineto{\pgfqpoint{1.355738in}{1.724117in}}%
\pgfpathlineto{\pgfqpoint{1.356156in}{1.724187in}}%
\pgfpathlineto{\pgfqpoint{1.356568in}{1.724263in}}%
\pgfpathclose%
\pgfusepath{fill}%
\end{pgfscope}%
\begin{pgfscope}%
\pgfpathrectangle{\pgfqpoint{0.329460in}{0.284240in}}{\pgfqpoint{1.989680in}{1.989680in}}%
\pgfusepath{clip}%
\pgfsetbuttcap%
\pgfsetroundjoin%
\definecolor{currentfill}{rgb}{0.855810,0.888601,0.097452}%
\pgfsetfillcolor{currentfill}%
\pgfsetlinewidth{0.000000pt}%
\definecolor{currentstroke}{rgb}{0.000000,0.000000,0.000000}%
\pgfsetstrokecolor{currentstroke}%
\pgfsetdash{}{0pt}%
\pgfpathmoveto{\pgfqpoint{1.280340in}{1.675703in}}%
\pgfpathlineto{\pgfqpoint{1.277387in}{1.671940in}}%
\pgfpathlineto{\pgfqpoint{1.274436in}{1.668068in}}%
\pgfpathlineto{\pgfqpoint{1.271484in}{1.664085in}}%
\pgfpathlineto{\pgfqpoint{1.268533in}{1.659994in}}%
\pgfpathlineto{\pgfqpoint{1.266904in}{1.661237in}}%
\pgfpathlineto{\pgfqpoint{1.265359in}{1.662503in}}%
\pgfpathlineto{\pgfqpoint{1.263900in}{1.663791in}}%
\pgfpathlineto{\pgfqpoint{1.262530in}{1.665099in}}%
\pgfpathlineto{\pgfqpoint{1.265693in}{1.669005in}}%
\pgfpathlineto{\pgfqpoint{1.268856in}{1.672802in}}%
\pgfpathlineto{\pgfqpoint{1.272020in}{1.676490in}}%
\pgfpathlineto{\pgfqpoint{1.275184in}{1.680067in}}%
\pgfpathlineto{\pgfqpoint{1.276361in}{1.678948in}}%
\pgfpathlineto{\pgfqpoint{1.277614in}{1.677847in}}%
\pgfpathlineto{\pgfqpoint{1.278940in}{1.676765in}}%
\pgfpathlineto{\pgfqpoint{1.280340in}{1.675703in}}%
\pgfpathclose%
\pgfusepath{fill}%
\end{pgfscope}%
\begin{pgfscope}%
\pgfpathrectangle{\pgfqpoint{0.329460in}{0.284240in}}{\pgfqpoint{1.989680in}{1.989680in}}%
\pgfusepath{clip}%
\pgfsetbuttcap%
\pgfsetroundjoin%
\definecolor{currentfill}{rgb}{0.166383,0.690856,0.496502}%
\pgfsetfillcolor{currentfill}%
\pgfsetlinewidth{0.000000pt}%
\definecolor{currentstroke}{rgb}{0.000000,0.000000,0.000000}%
\pgfsetstrokecolor{currentstroke}%
\pgfsetdash{}{0pt}%
\pgfpathmoveto{\pgfqpoint{1.212665in}{1.388909in}}%
\pgfpathlineto{\pgfqpoint{1.210644in}{1.380974in}}%
\pgfpathlineto{\pgfqpoint{1.208624in}{1.372993in}}%
\pgfpathlineto{\pgfqpoint{1.206605in}{1.364967in}}%
\pgfpathlineto{\pgfqpoint{1.204587in}{1.356899in}}%
\pgfpathlineto{\pgfqpoint{1.198201in}{1.359245in}}%
\pgfpathlineto{\pgfqpoint{1.191974in}{1.361687in}}%
\pgfpathlineto{\pgfqpoint{1.185914in}{1.364225in}}%
\pgfpathlineto{\pgfqpoint{1.180026in}{1.366856in}}%
\pgfpathlineto{\pgfqpoint{1.182378in}{1.374774in}}%
\pgfpathlineto{\pgfqpoint{1.184732in}{1.382650in}}%
\pgfpathlineto{\pgfqpoint{1.187086in}{1.390483in}}%
\pgfpathlineto{\pgfqpoint{1.189442in}{1.398269in}}%
\pgfpathlineto{\pgfqpoint{1.195010in}{1.395796in}}%
\pgfpathlineto{\pgfqpoint{1.200740in}{1.393410in}}%
\pgfpathlineto{\pgfqpoint{1.206627in}{1.391114in}}%
\pgfpathlineto{\pgfqpoint{1.212665in}{1.388909in}}%
\pgfpathclose%
\pgfusepath{fill}%
\end{pgfscope}%
\begin{pgfscope}%
\pgfpathrectangle{\pgfqpoint{0.329460in}{0.284240in}}{\pgfqpoint{1.989680in}{1.989680in}}%
\pgfusepath{clip}%
\pgfsetbuttcap%
\pgfsetroundjoin%
\definecolor{currentfill}{rgb}{0.993248,0.906157,0.143936}%
\pgfsetfillcolor{currentfill}%
\pgfsetlinewidth{0.000000pt}%
\definecolor{currentstroke}{rgb}{0.000000,0.000000,0.000000}%
\pgfsetstrokecolor{currentstroke}%
\pgfsetdash{}{0pt}%
\pgfpathmoveto{\pgfqpoint{1.347867in}{1.723949in}}%
\pgfpathlineto{\pgfqpoint{1.347035in}{1.721817in}}%
\pgfpathlineto{\pgfqpoint{1.346204in}{1.719564in}}%
\pgfpathlineto{\pgfqpoint{1.345372in}{1.717190in}}%
\pgfpathlineto{\pgfqpoint{1.344539in}{1.714695in}}%
\pgfpathlineto{\pgfqpoint{1.343680in}{1.714800in}}%
\pgfpathlineto{\pgfqpoint{1.342828in}{1.714917in}}%
\pgfpathlineto{\pgfqpoint{1.341984in}{1.715046in}}%
\pgfpathlineto{\pgfqpoint{1.341150in}{1.715188in}}%
\pgfpathlineto{\pgfqpoint{1.342406in}{1.717621in}}%
\pgfpathlineto{\pgfqpoint{1.343662in}{1.719933in}}%
\pgfpathlineto{\pgfqpoint{1.344918in}{1.722124in}}%
\pgfpathlineto{\pgfqpoint{1.346173in}{1.724195in}}%
\pgfpathlineto{\pgfqpoint{1.346590in}{1.724124in}}%
\pgfpathlineto{\pgfqpoint{1.347012in}{1.724060in}}%
\pgfpathlineto{\pgfqpoint{1.347437in}{1.724001in}}%
\pgfpathlineto{\pgfqpoint{1.347867in}{1.723949in}}%
\pgfpathclose%
\pgfusepath{fill}%
\end{pgfscope}%
\begin{pgfscope}%
\pgfpathrectangle{\pgfqpoint{0.329460in}{0.284240in}}{\pgfqpoint{1.989680in}{1.989680in}}%
\pgfusepath{clip}%
\pgfsetbuttcap%
\pgfsetroundjoin%
\definecolor{currentfill}{rgb}{0.282327,0.094955,0.417331}%
\pgfsetfillcolor{currentfill}%
\pgfsetlinewidth{0.000000pt}%
\definecolor{currentstroke}{rgb}{0.000000,0.000000,0.000000}%
\pgfsetstrokecolor{currentstroke}%
\pgfsetdash{}{0pt}%
\pgfpathmoveto{\pgfqpoint{1.473364in}{0.853790in}}%
\pgfpathlineto{\pgfqpoint{1.474295in}{0.847818in}}%
\pgfpathlineto{\pgfqpoint{1.475227in}{0.841983in}}%
\pgfpathlineto{\pgfqpoint{1.476160in}{0.836287in}}%
\pgfpathlineto{\pgfqpoint{1.477094in}{0.830736in}}%
\pgfpathlineto{\pgfqpoint{1.462513in}{0.828719in}}%
\pgfpathlineto{\pgfqpoint{1.447807in}{0.826951in}}%
\pgfpathlineto{\pgfqpoint{1.432995in}{0.825434in}}%
\pgfpathlineto{\pgfqpoint{1.418092in}{0.824169in}}%
\pgfpathlineto{\pgfqpoint{1.417595in}{0.829775in}}%
\pgfpathlineto{\pgfqpoint{1.417099in}{0.835524in}}%
\pgfpathlineto{\pgfqpoint{1.416603in}{0.841414in}}%
\pgfpathlineto{\pgfqpoint{1.416108in}{0.847440in}}%
\pgfpathlineto{\pgfqpoint{1.430569in}{0.848663in}}%
\pgfpathlineto{\pgfqpoint{1.444943in}{0.850130in}}%
\pgfpathlineto{\pgfqpoint{1.459213in}{0.851840in}}%
\pgfpathlineto{\pgfqpoint{1.473364in}{0.853790in}}%
\pgfpathclose%
\pgfusepath{fill}%
\end{pgfscope}%
\begin{pgfscope}%
\pgfpathrectangle{\pgfqpoint{0.329460in}{0.284240in}}{\pgfqpoint{1.989680in}{1.989680in}}%
\pgfusepath{clip}%
\pgfsetbuttcap%
\pgfsetroundjoin%
\definecolor{currentfill}{rgb}{0.896320,0.893616,0.096335}%
\pgfsetfillcolor{currentfill}%
\pgfsetlinewidth{0.000000pt}%
\definecolor{currentstroke}{rgb}{0.000000,0.000000,0.000000}%
\pgfsetstrokecolor{currentstroke}%
\pgfsetdash{}{0pt}%
\pgfpathmoveto{\pgfqpoint{1.292152in}{1.689630in}}%
\pgfpathlineto{\pgfqpoint{1.289198in}{1.686318in}}%
\pgfpathlineto{\pgfqpoint{1.286245in}{1.682892in}}%
\pgfpathlineto{\pgfqpoint{1.283292in}{1.679353in}}%
\pgfpathlineto{\pgfqpoint{1.280340in}{1.675703in}}%
\pgfpathlineto{\pgfqpoint{1.278940in}{1.676765in}}%
\pgfpathlineto{\pgfqpoint{1.277614in}{1.677847in}}%
\pgfpathlineto{\pgfqpoint{1.276361in}{1.678948in}}%
\pgfpathlineto{\pgfqpoint{1.275184in}{1.680067in}}%
\pgfpathlineto{\pgfqpoint{1.278349in}{1.683534in}}%
\pgfpathlineto{\pgfqpoint{1.281515in}{1.686888in}}%
\pgfpathlineto{\pgfqpoint{1.284681in}{1.690131in}}%
\pgfpathlineto{\pgfqpoint{1.287848in}{1.693259in}}%
\pgfpathlineto{\pgfqpoint{1.288831in}{1.692329in}}%
\pgfpathlineto{\pgfqpoint{1.289877in}{1.691413in}}%
\pgfpathlineto{\pgfqpoint{1.290984in}{1.690514in}}%
\pgfpathlineto{\pgfqpoint{1.292152in}{1.689630in}}%
\pgfpathclose%
\pgfusepath{fill}%
\end{pgfscope}%
\begin{pgfscope}%
\pgfpathrectangle{\pgfqpoint{0.329460in}{0.284240in}}{\pgfqpoint{1.989680in}{1.989680in}}%
\pgfusepath{clip}%
\pgfsetbuttcap%
\pgfsetroundjoin%
\definecolor{currentfill}{rgb}{0.267004,0.004874,0.329415}%
\pgfsetfillcolor{currentfill}%
\pgfsetlinewidth{0.000000pt}%
\definecolor{currentstroke}{rgb}{0.000000,0.000000,0.000000}%
\pgfsetstrokecolor{currentstroke}%
\pgfsetdash{}{0pt}%
\pgfpathmoveto{\pgfqpoint{1.160422in}{0.773481in}}%
\pgfpathlineto{\pgfqpoint{1.159123in}{0.771038in}}%
\pgfpathlineto{\pgfqpoint{1.157821in}{0.768812in}}%
\pgfpathlineto{\pgfqpoint{1.156516in}{0.766807in}}%
\pgfpathlineto{\pgfqpoint{1.155208in}{0.765027in}}%
\pgfpathlineto{\pgfqpoint{1.139218in}{0.768527in}}%
\pgfpathlineto{\pgfqpoint{1.123465in}{0.772297in}}%
\pgfpathlineto{\pgfqpoint{1.107966in}{0.776333in}}%
\pgfpathlineto{\pgfqpoint{1.092739in}{0.780630in}}%
\pgfpathlineto{\pgfqpoint{1.094459in}{0.782302in}}%
\pgfpathlineto{\pgfqpoint{1.096176in}{0.784199in}}%
\pgfpathlineto{\pgfqpoint{1.097889in}{0.786317in}}%
\pgfpathlineto{\pgfqpoint{1.099598in}{0.788651in}}%
\pgfpathlineto{\pgfqpoint{1.114426in}{0.784474in}}%
\pgfpathlineto{\pgfqpoint{1.129517in}{0.780550in}}%
\pgfpathlineto{\pgfqpoint{1.144854in}{0.776884in}}%
\pgfpathlineto{\pgfqpoint{1.160422in}{0.773481in}}%
\pgfpathclose%
\pgfusepath{fill}%
\end{pgfscope}%
\begin{pgfscope}%
\pgfpathrectangle{\pgfqpoint{0.329460in}{0.284240in}}{\pgfqpoint{1.989680in}{1.989680in}}%
\pgfusepath{clip}%
\pgfsetbuttcap%
\pgfsetroundjoin%
\definecolor{currentfill}{rgb}{0.283072,0.130895,0.449241}%
\pgfsetfillcolor{currentfill}%
\pgfsetlinewidth{0.000000pt}%
\definecolor{currentstroke}{rgb}{0.000000,0.000000,0.000000}%
\pgfsetstrokecolor{currentstroke}%
\pgfsetdash{}{0pt}%
\pgfpathmoveto{\pgfqpoint{1.300766in}{0.871983in}}%
\pgfpathlineto{\pgfqpoint{1.300371in}{0.865441in}}%
\pgfpathlineto{\pgfqpoint{1.299975in}{0.859020in}}%
\pgfpathlineto{\pgfqpoint{1.299580in}{0.852725in}}%
\pgfpathlineto{\pgfqpoint{1.299183in}{0.846559in}}%
\pgfpathlineto{\pgfqpoint{1.284657in}{0.847564in}}%
\pgfpathlineto{\pgfqpoint{1.270204in}{0.848814in}}%
\pgfpathlineto{\pgfqpoint{1.255841in}{0.850308in}}%
\pgfpathlineto{\pgfqpoint{1.241583in}{0.852045in}}%
\pgfpathlineto{\pgfqpoint{1.242418in}{0.858164in}}%
\pgfpathlineto{\pgfqpoint{1.243252in}{0.864413in}}%
\pgfpathlineto{\pgfqpoint{1.244085in}{0.870786in}}%
\pgfpathlineto{\pgfqpoint{1.244917in}{0.877282in}}%
\pgfpathlineto{\pgfqpoint{1.258742in}{0.875605in}}%
\pgfpathlineto{\pgfqpoint{1.272668in}{0.874162in}}%
\pgfpathlineto{\pgfqpoint{1.286682in}{0.872954in}}%
\pgfpathlineto{\pgfqpoint{1.300766in}{0.871983in}}%
\pgfpathclose%
\pgfusepath{fill}%
\end{pgfscope}%
\begin{pgfscope}%
\pgfpathrectangle{\pgfqpoint{0.329460in}{0.284240in}}{\pgfqpoint{1.989680in}{1.989680in}}%
\pgfusepath{clip}%
\pgfsetbuttcap%
\pgfsetroundjoin%
\definecolor{currentfill}{rgb}{0.195860,0.395433,0.555276}%
\pgfsetfillcolor{currentfill}%
\pgfsetlinewidth{0.000000pt}%
\definecolor{currentstroke}{rgb}{0.000000,0.000000,0.000000}%
\pgfsetstrokecolor{currentstroke}%
\pgfsetdash{}{0pt}%
\pgfpathmoveto{\pgfqpoint{1.485725in}{1.103726in}}%
\pgfpathlineto{\pgfqpoint{1.487056in}{1.095255in}}%
\pgfpathlineto{\pgfqpoint{1.488387in}{1.086817in}}%
\pgfpathlineto{\pgfqpoint{1.489718in}{1.078414in}}%
\pgfpathlineto{\pgfqpoint{1.491048in}{1.070050in}}%
\pgfpathlineto{\pgfqpoint{1.480370in}{1.067845in}}%
\pgfpathlineto{\pgfqpoint{1.469553in}{1.065816in}}%
\pgfpathlineto{\pgfqpoint{1.458607in}{1.063964in}}%
\pgfpathlineto{\pgfqpoint{1.447546in}{1.062292in}}%
\pgfpathlineto{\pgfqpoint{1.446628in}{1.070740in}}%
\pgfpathlineto{\pgfqpoint{1.445710in}{1.079227in}}%
\pgfpathlineto{\pgfqpoint{1.444792in}{1.087748in}}%
\pgfpathlineto{\pgfqpoint{1.443874in}{1.096303in}}%
\pgfpathlineto{\pgfqpoint{1.454515in}{1.097902in}}%
\pgfpathlineto{\pgfqpoint{1.465045in}{1.099674in}}%
\pgfpathlineto{\pgfqpoint{1.475452in}{1.101616in}}%
\pgfpathlineto{\pgfqpoint{1.485725in}{1.103726in}}%
\pgfpathclose%
\pgfusepath{fill}%
\end{pgfscope}%
\begin{pgfscope}%
\pgfpathrectangle{\pgfqpoint{0.329460in}{0.284240in}}{\pgfqpoint{1.989680in}{1.989680in}}%
\pgfusepath{clip}%
\pgfsetbuttcap%
\pgfsetroundjoin%
\definecolor{currentfill}{rgb}{0.993248,0.906157,0.143936}%
\pgfsetfillcolor{currentfill}%
\pgfsetlinewidth{0.000000pt}%
\definecolor{currentstroke}{rgb}{0.000000,0.000000,0.000000}%
\pgfsetstrokecolor{currentstroke}%
\pgfsetdash{}{0pt}%
\pgfpathmoveto{\pgfqpoint{1.354890in}{1.723995in}}%
\pgfpathlineto{\pgfqpoint{1.355817in}{1.721874in}}%
\pgfpathlineto{\pgfqpoint{1.356744in}{1.719633in}}%
\pgfpathlineto{\pgfqpoint{1.357672in}{1.717270in}}%
\pgfpathlineto{\pgfqpoint{1.358600in}{1.714788in}}%
\pgfpathlineto{\pgfqpoint{1.357740in}{1.714685in}}%
\pgfpathlineto{\pgfqpoint{1.356873in}{1.714594in}}%
\pgfpathlineto{\pgfqpoint{1.356000in}{1.714517in}}%
\pgfpathlineto{\pgfqpoint{1.355123in}{1.714452in}}%
\pgfpathlineto{\pgfqpoint{1.354630in}{1.716977in}}%
\pgfpathlineto{\pgfqpoint{1.354138in}{1.719382in}}%
\pgfpathlineto{\pgfqpoint{1.353645in}{1.721665in}}%
\pgfpathlineto{\pgfqpoint{1.353153in}{1.723828in}}%
\pgfpathlineto{\pgfqpoint{1.353592in}{1.723860in}}%
\pgfpathlineto{\pgfqpoint{1.354028in}{1.723899in}}%
\pgfpathlineto{\pgfqpoint{1.354461in}{1.723944in}}%
\pgfpathlineto{\pgfqpoint{1.354890in}{1.723995in}}%
\pgfpathclose%
\pgfusepath{fill}%
\end{pgfscope}%
\begin{pgfscope}%
\pgfpathrectangle{\pgfqpoint{0.329460in}{0.284240in}}{\pgfqpoint{1.989680in}{1.989680in}}%
\pgfusepath{clip}%
\pgfsetbuttcap%
\pgfsetroundjoin%
\definecolor{currentfill}{rgb}{0.993248,0.906157,0.143936}%
\pgfsetfillcolor{currentfill}%
\pgfsetlinewidth{0.000000pt}%
\definecolor{currentstroke}{rgb}{0.000000,0.000000,0.000000}%
\pgfsetstrokecolor{currentstroke}%
\pgfsetdash{}{0pt}%
\pgfpathmoveto{\pgfqpoint{1.349613in}{1.723805in}}%
\pgfpathlineto{\pgfqpoint{1.349219in}{1.721636in}}%
\pgfpathlineto{\pgfqpoint{1.348825in}{1.719347in}}%
\pgfpathlineto{\pgfqpoint{1.348430in}{1.716937in}}%
\pgfpathlineto{\pgfqpoint{1.348036in}{1.714406in}}%
\pgfpathlineto{\pgfqpoint{1.347155in}{1.714459in}}%
\pgfpathlineto{\pgfqpoint{1.346278in}{1.714525in}}%
\pgfpathlineto{\pgfqpoint{1.345406in}{1.714604in}}%
\pgfpathlineto{\pgfqpoint{1.344539in}{1.714695in}}%
\pgfpathlineto{\pgfqpoint{1.345372in}{1.717190in}}%
\pgfpathlineto{\pgfqpoint{1.346204in}{1.719564in}}%
\pgfpathlineto{\pgfqpoint{1.347035in}{1.721817in}}%
\pgfpathlineto{\pgfqpoint{1.347867in}{1.723949in}}%
\pgfpathlineto{\pgfqpoint{1.348299in}{1.723904in}}%
\pgfpathlineto{\pgfqpoint{1.348735in}{1.723864in}}%
\pgfpathlineto{\pgfqpoint{1.349173in}{1.723831in}}%
\pgfpathlineto{\pgfqpoint{1.349613in}{1.723805in}}%
\pgfpathclose%
\pgfusepath{fill}%
\end{pgfscope}%
\begin{pgfscope}%
\pgfpathrectangle{\pgfqpoint{0.329460in}{0.284240in}}{\pgfqpoint{1.989680in}{1.989680in}}%
\pgfusepath{clip}%
\pgfsetbuttcap%
\pgfsetroundjoin%
\definecolor{currentfill}{rgb}{0.993248,0.906157,0.143936}%
\pgfsetfillcolor{currentfill}%
\pgfsetlinewidth{0.000000pt}%
\definecolor{currentstroke}{rgb}{0.000000,0.000000,0.000000}%
\pgfsetstrokecolor{currentstroke}%
\pgfsetdash{}{0pt}%
\pgfpathmoveto{\pgfqpoint{1.353153in}{1.723828in}}%
\pgfpathlineto{\pgfqpoint{1.353645in}{1.721665in}}%
\pgfpathlineto{\pgfqpoint{1.354138in}{1.719382in}}%
\pgfpathlineto{\pgfqpoint{1.354630in}{1.716977in}}%
\pgfpathlineto{\pgfqpoint{1.355123in}{1.714452in}}%
\pgfpathlineto{\pgfqpoint{1.354241in}{1.714401in}}%
\pgfpathlineto{\pgfqpoint{1.353357in}{1.714362in}}%
\pgfpathlineto{\pgfqpoint{1.352470in}{1.714337in}}%
\pgfpathlineto{\pgfqpoint{1.351582in}{1.714325in}}%
\pgfpathlineto{\pgfqpoint{1.351533in}{1.716865in}}%
\pgfpathlineto{\pgfqpoint{1.351484in}{1.719286in}}%
\pgfpathlineto{\pgfqpoint{1.351434in}{1.721586in}}%
\pgfpathlineto{\pgfqpoint{1.351385in}{1.723764in}}%
\pgfpathlineto{\pgfqpoint{1.351828in}{1.723771in}}%
\pgfpathlineto{\pgfqpoint{1.352271in}{1.723783in}}%
\pgfpathlineto{\pgfqpoint{1.352713in}{1.723802in}}%
\pgfpathlineto{\pgfqpoint{1.353153in}{1.723828in}}%
\pgfpathclose%
\pgfusepath{fill}%
\end{pgfscope}%
\begin{pgfscope}%
\pgfpathrectangle{\pgfqpoint{0.329460in}{0.284240in}}{\pgfqpoint{1.989680in}{1.989680in}}%
\pgfusepath{clip}%
\pgfsetbuttcap%
\pgfsetroundjoin%
\definecolor{currentfill}{rgb}{0.993248,0.906157,0.143936}%
\pgfsetfillcolor{currentfill}%
\pgfsetlinewidth{0.000000pt}%
\definecolor{currentstroke}{rgb}{0.000000,0.000000,0.000000}%
\pgfsetstrokecolor{currentstroke}%
\pgfsetdash{}{0pt}%
\pgfpathmoveto{\pgfqpoint{1.351385in}{1.723764in}}%
\pgfpathlineto{\pgfqpoint{1.351434in}{1.721586in}}%
\pgfpathlineto{\pgfqpoint{1.351484in}{1.719286in}}%
\pgfpathlineto{\pgfqpoint{1.351533in}{1.716865in}}%
\pgfpathlineto{\pgfqpoint{1.351582in}{1.714325in}}%
\pgfpathlineto{\pgfqpoint{1.350694in}{1.714325in}}%
\pgfpathlineto{\pgfqpoint{1.349806in}{1.714339in}}%
\pgfpathlineto{\pgfqpoint{1.348920in}{1.714366in}}%
\pgfpathlineto{\pgfqpoint{1.348036in}{1.714406in}}%
\pgfpathlineto{\pgfqpoint{1.348430in}{1.716937in}}%
\pgfpathlineto{\pgfqpoint{1.348825in}{1.719347in}}%
\pgfpathlineto{\pgfqpoint{1.349219in}{1.721636in}}%
\pgfpathlineto{\pgfqpoint{1.349613in}{1.723805in}}%
\pgfpathlineto{\pgfqpoint{1.350055in}{1.723785in}}%
\pgfpathlineto{\pgfqpoint{1.350498in}{1.723772in}}%
\pgfpathlineto{\pgfqpoint{1.350941in}{1.723765in}}%
\pgfpathlineto{\pgfqpoint{1.351385in}{1.723764in}}%
\pgfpathclose%
\pgfusepath{fill}%
\end{pgfscope}%
\begin{pgfscope}%
\pgfpathrectangle{\pgfqpoint{0.329460in}{0.284240in}}{\pgfqpoint{1.989680in}{1.989680in}}%
\pgfusepath{clip}%
\pgfsetbuttcap%
\pgfsetroundjoin%
\definecolor{currentfill}{rgb}{0.974417,0.903590,0.130215}%
\pgfsetfillcolor{currentfill}%
\pgfsetlinewidth{0.000000pt}%
\definecolor{currentstroke}{rgb}{0.000000,0.000000,0.000000}%
\pgfsetstrokecolor{currentstroke}%
\pgfsetdash{}{0pt}%
\pgfpathmoveto{\pgfqpoint{1.373175in}{1.719273in}}%
\pgfpathlineto{\pgfqpoint{1.375927in}{1.717233in}}%
\pgfpathlineto{\pgfqpoint{1.378679in}{1.715074in}}%
\pgfpathlineto{\pgfqpoint{1.381431in}{1.712796in}}%
\pgfpathlineto{\pgfqpoint{1.384184in}{1.710399in}}%
\pgfpathlineto{\pgfqpoint{1.383341in}{1.709919in}}%
\pgfpathlineto{\pgfqpoint{1.382465in}{1.709450in}}%
\pgfpathlineto{\pgfqpoint{1.381558in}{1.708995in}}%
\pgfpathlineto{\pgfqpoint{1.380620in}{1.708554in}}%
\pgfpathlineto{\pgfqpoint{1.378164in}{1.711105in}}%
\pgfpathlineto{\pgfqpoint{1.375708in}{1.713538in}}%
\pgfpathlineto{\pgfqpoint{1.373253in}{1.715851in}}%
\pgfpathlineto{\pgfqpoint{1.370798in}{1.718045in}}%
\pgfpathlineto{\pgfqpoint{1.371423in}{1.718339in}}%
\pgfpathlineto{\pgfqpoint{1.372028in}{1.718641in}}%
\pgfpathlineto{\pgfqpoint{1.372613in}{1.718953in}}%
\pgfpathlineto{\pgfqpoint{1.373175in}{1.719273in}}%
\pgfpathclose%
\pgfusepath{fill}%
\end{pgfscope}%
\begin{pgfscope}%
\pgfpathrectangle{\pgfqpoint{0.329460in}{0.284240in}}{\pgfqpoint{1.989680in}{1.989680in}}%
\pgfusepath{clip}%
\pgfsetbuttcap%
\pgfsetroundjoin%
\definecolor{currentfill}{rgb}{0.147607,0.511733,0.557049}%
\pgfsetfillcolor{currentfill}%
\pgfsetlinewidth{0.000000pt}%
\definecolor{currentstroke}{rgb}{0.000000,0.000000,0.000000}%
\pgfsetstrokecolor{currentstroke}%
\pgfsetdash{}{0pt}%
\pgfpathmoveto{\pgfqpoint{1.504615in}{1.215644in}}%
\pgfpathlineto{\pgfqpoint{1.506338in}{1.207107in}}%
\pgfpathlineto{\pgfqpoint{1.508061in}{1.198569in}}%
\pgfpathlineto{\pgfqpoint{1.509783in}{1.190035in}}%
\pgfpathlineto{\pgfqpoint{1.511505in}{1.181506in}}%
\pgfpathlineto{\pgfqpoint{1.502630in}{1.178995in}}%
\pgfpathlineto{\pgfqpoint{1.493593in}{1.176628in}}%
\pgfpathlineto{\pgfqpoint{1.484405in}{1.174406in}}%
\pgfpathlineto{\pgfqpoint{1.475074in}{1.172333in}}%
\pgfpathlineto{\pgfqpoint{1.473742in}{1.180973in}}%
\pgfpathlineto{\pgfqpoint{1.472409in}{1.189618in}}%
\pgfpathlineto{\pgfqpoint{1.471076in}{1.198266in}}%
\pgfpathlineto{\pgfqpoint{1.469743in}{1.206913in}}%
\pgfpathlineto{\pgfqpoint{1.478674in}{1.208886in}}%
\pgfpathlineto{\pgfqpoint{1.487469in}{1.211001in}}%
\pgfpathlineto{\pgfqpoint{1.496119in}{1.213254in}}%
\pgfpathlineto{\pgfqpoint{1.504615in}{1.215644in}}%
\pgfpathclose%
\pgfusepath{fill}%
\end{pgfscope}%
\begin{pgfscope}%
\pgfpathrectangle{\pgfqpoint{0.329460in}{0.284240in}}{\pgfqpoint{1.989680in}{1.989680in}}%
\pgfusepath{clip}%
\pgfsetbuttcap%
\pgfsetroundjoin%
\definecolor{currentfill}{rgb}{0.263663,0.237631,0.518762}%
\pgfsetfillcolor{currentfill}%
\pgfsetlinewidth{0.000000pt}%
\definecolor{currentstroke}{rgb}{0.000000,0.000000,0.000000}%
\pgfsetstrokecolor{currentstroke}%
\pgfsetdash{}{0pt}%
\pgfpathmoveto{\pgfqpoint{1.305486in}{0.958741in}}%
\pgfpathlineto{\pgfqpoint{1.305094in}{0.951004in}}%
\pgfpathlineto{\pgfqpoint{1.304702in}{0.943348in}}%
\pgfpathlineto{\pgfqpoint{1.304309in}{0.935777in}}%
\pgfpathlineto{\pgfqpoint{1.303916in}{0.928293in}}%
\pgfpathlineto{\pgfqpoint{1.290711in}{0.929196in}}%
\pgfpathlineto{\pgfqpoint{1.277573in}{0.930318in}}%
\pgfpathlineto{\pgfqpoint{1.264515in}{0.931660in}}%
\pgfpathlineto{\pgfqpoint{1.251552in}{0.933219in}}%
\pgfpathlineto{\pgfqpoint{1.252379in}{0.940657in}}%
\pgfpathlineto{\pgfqpoint{1.253205in}{0.948182in}}%
\pgfpathlineto{\pgfqpoint{1.254031in}{0.955791in}}%
\pgfpathlineto{\pgfqpoint{1.254857in}{0.963482in}}%
\pgfpathlineto{\pgfqpoint{1.267391in}{0.961981in}}%
\pgfpathlineto{\pgfqpoint{1.280016in}{0.960690in}}%
\pgfpathlineto{\pgfqpoint{1.292719in}{0.959610in}}%
\pgfpathlineto{\pgfqpoint{1.305486in}{0.958741in}}%
\pgfpathclose%
\pgfusepath{fill}%
\end{pgfscope}%
\begin{pgfscope}%
\pgfpathrectangle{\pgfqpoint{0.329460in}{0.284240in}}{\pgfqpoint{1.989680in}{1.989680in}}%
\pgfusepath{clip}%
\pgfsetbuttcap%
\pgfsetroundjoin%
\definecolor{currentfill}{rgb}{0.412913,0.803041,0.357269}%
\pgfsetfillcolor{currentfill}%
\pgfsetlinewidth{0.000000pt}%
\definecolor{currentstroke}{rgb}{0.000000,0.000000,0.000000}%
\pgfsetstrokecolor{currentstroke}%
\pgfsetdash{}{0pt}%
\pgfpathmoveto{\pgfqpoint{1.494096in}{1.524928in}}%
\pgfpathlineto{\pgfqpoint{1.496827in}{1.518346in}}%
\pgfpathlineto{\pgfqpoint{1.499557in}{1.511689in}}%
\pgfpathlineto{\pgfqpoint{1.502286in}{1.504957in}}%
\pgfpathlineto{\pgfqpoint{1.505013in}{1.498153in}}%
\pgfpathlineto{\pgfqpoint{1.501126in}{1.495829in}}%
\pgfpathlineto{\pgfqpoint{1.497085in}{1.493564in}}%
\pgfpathlineto{\pgfqpoint{1.492893in}{1.491362in}}%
\pgfpathlineto{\pgfqpoint{1.488556in}{1.489225in}}%
\pgfpathlineto{\pgfqpoint{1.486117in}{1.496197in}}%
\pgfpathlineto{\pgfqpoint{1.483677in}{1.503097in}}%
\pgfpathlineto{\pgfqpoint{1.481235in}{1.509922in}}%
\pgfpathlineto{\pgfqpoint{1.478793in}{1.516671in}}%
\pgfpathlineto{\pgfqpoint{1.482826in}{1.518648in}}%
\pgfpathlineto{\pgfqpoint{1.486723in}{1.520685in}}%
\pgfpathlineto{\pgfqpoint{1.490481in}{1.522779in}}%
\pgfpathlineto{\pgfqpoint{1.494096in}{1.524928in}}%
\pgfpathclose%
\pgfusepath{fill}%
\end{pgfscope}%
\begin{pgfscope}%
\pgfpathrectangle{\pgfqpoint{0.329460in}{0.284240in}}{\pgfqpoint{1.989680in}{1.989680in}}%
\pgfusepath{clip}%
\pgfsetbuttcap%
\pgfsetroundjoin%
\definecolor{currentfill}{rgb}{0.248629,0.278775,0.534556}%
\pgfsetfillcolor{currentfill}%
\pgfsetlinewidth{0.000000pt}%
\definecolor{currentstroke}{rgb}{0.000000,0.000000,0.000000}%
\pgfsetstrokecolor{currentstroke}%
\pgfsetdash{}{0pt}%
\pgfpathmoveto{\pgfqpoint{1.454893in}{0.996443in}}%
\pgfpathlineto{\pgfqpoint{1.455812in}{0.988474in}}%
\pgfpathlineto{\pgfqpoint{1.456731in}{0.980573in}}%
\pgfpathlineto{\pgfqpoint{1.457651in}{0.972744in}}%
\pgfpathlineto{\pgfqpoint{1.458571in}{0.964990in}}%
\pgfpathlineto{\pgfqpoint{1.446130in}{0.963305in}}%
\pgfpathlineto{\pgfqpoint{1.433586in}{0.961827in}}%
\pgfpathlineto{\pgfqpoint{1.420951in}{0.960560in}}%
\pgfpathlineto{\pgfqpoint{1.408241in}{0.959503in}}%
\pgfpathlineto{\pgfqpoint{1.407751in}{0.967310in}}%
\pgfpathlineto{\pgfqpoint{1.407262in}{0.975193in}}%
\pgfpathlineto{\pgfqpoint{1.406773in}{0.983147in}}%
\pgfpathlineto{\pgfqpoint{1.406285in}{0.991169in}}%
\pgfpathlineto{\pgfqpoint{1.418560in}{0.992185in}}%
\pgfpathlineto{\pgfqpoint{1.430762in}{0.993403in}}%
\pgfpathlineto{\pgfqpoint{1.442877in}{0.994823in}}%
\pgfpathlineto{\pgfqpoint{1.454893in}{0.996443in}}%
\pgfpathclose%
\pgfusepath{fill}%
\end{pgfscope}%
\begin{pgfscope}%
\pgfpathrectangle{\pgfqpoint{0.329460in}{0.284240in}}{\pgfqpoint{1.989680in}{1.989680in}}%
\pgfusepath{clip}%
\pgfsetbuttcap%
\pgfsetroundjoin%
\definecolor{currentfill}{rgb}{0.974417,0.903590,0.130215}%
\pgfsetfillcolor{currentfill}%
\pgfsetlinewidth{0.000000pt}%
\definecolor{currentstroke}{rgb}{0.000000,0.000000,0.000000}%
\pgfsetstrokecolor{currentstroke}%
\pgfsetdash{}{0pt}%
\pgfpathmoveto{\pgfqpoint{1.332149in}{1.717792in}}%
\pgfpathlineto{\pgfqpoint{1.329766in}{1.715566in}}%
\pgfpathlineto{\pgfqpoint{1.327383in}{1.713221in}}%
\pgfpathlineto{\pgfqpoint{1.324998in}{1.710757in}}%
\pgfpathlineto{\pgfqpoint{1.322614in}{1.708173in}}%
\pgfpathlineto{\pgfqpoint{1.321650in}{1.708602in}}%
\pgfpathlineto{\pgfqpoint{1.320715in}{1.709045in}}%
\pgfpathlineto{\pgfqpoint{1.319811in}{1.709502in}}%
\pgfpathlineto{\pgfqpoint{1.318939in}{1.709971in}}%
\pgfpathlineto{\pgfqpoint{1.321630in}{1.712404in}}%
\pgfpathlineto{\pgfqpoint{1.324320in}{1.714718in}}%
\pgfpathlineto{\pgfqpoint{1.327010in}{1.716913in}}%
\pgfpathlineto{\pgfqpoint{1.329699in}{1.718988in}}%
\pgfpathlineto{\pgfqpoint{1.330281in}{1.718675in}}%
\pgfpathlineto{\pgfqpoint{1.330883in}{1.718372in}}%
\pgfpathlineto{\pgfqpoint{1.331507in}{1.718077in}}%
\pgfpathlineto{\pgfqpoint{1.332149in}{1.717792in}}%
\pgfpathclose%
\pgfusepath{fill}%
\end{pgfscope}%
\begin{pgfscope}%
\pgfpathrectangle{\pgfqpoint{0.329460in}{0.284240in}}{\pgfqpoint{1.989680in}{1.989680in}}%
\pgfusepath{clip}%
\pgfsetbuttcap%
\pgfsetroundjoin%
\definecolor{currentfill}{rgb}{0.120081,0.622161,0.534946}%
\pgfsetfillcolor{currentfill}%
\pgfsetlinewidth{0.000000pt}%
\definecolor{currentstroke}{rgb}{0.000000,0.000000,0.000000}%
\pgfsetstrokecolor{currentstroke}%
\pgfsetdash{}{0pt}%
\pgfpathmoveto{\pgfqpoint{1.511846in}{1.326456in}}%
\pgfpathlineto{\pgfqpoint{1.513937in}{1.318242in}}%
\pgfpathlineto{\pgfqpoint{1.516028in}{1.309999in}}%
\pgfpathlineto{\pgfqpoint{1.518117in}{1.301730in}}%
\pgfpathlineto{\pgfqpoint{1.520206in}{1.293437in}}%
\pgfpathlineto{\pgfqpoint{1.513105in}{1.290819in}}%
\pgfpathlineto{\pgfqpoint{1.505834in}{1.288313in}}%
\pgfpathlineto{\pgfqpoint{1.498400in}{1.285923in}}%
\pgfpathlineto{\pgfqpoint{1.490811in}{1.283651in}}%
\pgfpathlineto{\pgfqpoint{1.489083in}{1.292078in}}%
\pgfpathlineto{\pgfqpoint{1.487355in}{1.300481in}}%
\pgfpathlineto{\pgfqpoint{1.485625in}{1.308857in}}%
\pgfpathlineto{\pgfqpoint{1.483895in}{1.317204in}}%
\pgfpathlineto{\pgfqpoint{1.491111in}{1.319352in}}%
\pgfpathlineto{\pgfqpoint{1.498179in}{1.321612in}}%
\pgfpathlineto{\pgfqpoint{1.505093in}{1.323980in}}%
\pgfpathlineto{\pgfqpoint{1.511846in}{1.326456in}}%
\pgfpathclose%
\pgfusepath{fill}%
\end{pgfscope}%
\begin{pgfscope}%
\pgfpathrectangle{\pgfqpoint{0.329460in}{0.284240in}}{\pgfqpoint{1.989680in}{1.989680in}}%
\pgfusepath{clip}%
\pgfsetbuttcap%
\pgfsetroundjoin%
\definecolor{currentfill}{rgb}{0.955300,0.901065,0.118128}%
\pgfsetfillcolor{currentfill}%
\pgfsetlinewidth{0.000000pt}%
\definecolor{currentstroke}{rgb}{0.000000,0.000000,0.000000}%
\pgfsetstrokecolor{currentstroke}%
\pgfsetdash{}{0pt}%
\pgfpathmoveto{\pgfqpoint{1.387220in}{1.712439in}}%
\pgfpathlineto{\pgfqpoint{1.390225in}{1.710095in}}%
\pgfpathlineto{\pgfqpoint{1.393231in}{1.707633in}}%
\pgfpathlineto{\pgfqpoint{1.396237in}{1.705054in}}%
\pgfpathlineto{\pgfqpoint{1.399242in}{1.702358in}}%
\pgfpathlineto{\pgfqpoint{1.398301in}{1.701655in}}%
\pgfpathlineto{\pgfqpoint{1.397313in}{1.700966in}}%
\pgfpathlineto{\pgfqpoint{1.396279in}{1.700291in}}%
\pgfpathlineto{\pgfqpoint{1.395199in}{1.699632in}}%
\pgfpathlineto{\pgfqpoint{1.392445in}{1.702500in}}%
\pgfpathlineto{\pgfqpoint{1.389691in}{1.705251in}}%
\pgfpathlineto{\pgfqpoint{1.386938in}{1.707884in}}%
\pgfpathlineto{\pgfqpoint{1.384184in}{1.710399in}}%
\pgfpathlineto{\pgfqpoint{1.384995in}{1.710892in}}%
\pgfpathlineto{\pgfqpoint{1.385772in}{1.711397in}}%
\pgfpathlineto{\pgfqpoint{1.386513in}{1.711913in}}%
\pgfpathlineto{\pgfqpoint{1.387220in}{1.712439in}}%
\pgfpathclose%
\pgfusepath{fill}%
\end{pgfscope}%
\begin{pgfscope}%
\pgfpathrectangle{\pgfqpoint{0.329460in}{0.284240in}}{\pgfqpoint{1.989680in}{1.989680in}}%
\pgfusepath{clip}%
\pgfsetbuttcap%
\pgfsetroundjoin%
\definecolor{currentfill}{rgb}{0.220124,0.725509,0.466226}%
\pgfsetfillcolor{currentfill}%
\pgfsetlinewidth{0.000000pt}%
\definecolor{currentstroke}{rgb}{0.000000,0.000000,0.000000}%
\pgfsetstrokecolor{currentstroke}%
\pgfsetdash{}{0pt}%
\pgfpathmoveto{\pgfqpoint{1.508031in}{1.431039in}}%
\pgfpathlineto{\pgfqpoint{1.510460in}{1.423493in}}%
\pgfpathlineto{\pgfqpoint{1.512888in}{1.415893in}}%
\pgfpathlineto{\pgfqpoint{1.515315in}{1.408240in}}%
\pgfpathlineto{\pgfqpoint{1.517741in}{1.400538in}}%
\pgfpathlineto{\pgfqpoint{1.512323in}{1.397990in}}%
\pgfpathlineto{\pgfqpoint{1.506736in}{1.395527in}}%
\pgfpathlineto{\pgfqpoint{1.500989in}{1.393150in}}%
\pgfpathlineto{\pgfqpoint{1.495085in}{1.390864in}}%
\pgfpathlineto{\pgfqpoint{1.492985in}{1.398720in}}%
\pgfpathlineto{\pgfqpoint{1.490885in}{1.406525in}}%
\pgfpathlineto{\pgfqpoint{1.488784in}{1.414277in}}%
\pgfpathlineto{\pgfqpoint{1.486682in}{1.421975in}}%
\pgfpathlineto{\pgfqpoint{1.492245in}{1.424117in}}%
\pgfpathlineto{\pgfqpoint{1.497660in}{1.426343in}}%
\pgfpathlineto{\pgfqpoint{1.502924in}{1.428651in}}%
\pgfpathlineto{\pgfqpoint{1.508031in}{1.431039in}}%
\pgfpathclose%
\pgfusepath{fill}%
\end{pgfscope}%
\begin{pgfscope}%
\pgfpathrectangle{\pgfqpoint{0.329460in}{0.284240in}}{\pgfqpoint{1.989680in}{1.989680in}}%
\pgfusepath{clip}%
\pgfsetbuttcap%
\pgfsetroundjoin%
\definecolor{currentfill}{rgb}{0.276194,0.190074,0.493001}%
\pgfsetfillcolor{currentfill}%
\pgfsetlinewidth{0.000000pt}%
\definecolor{currentstroke}{rgb}{0.000000,0.000000,0.000000}%
\pgfsetstrokecolor{currentstroke}%
\pgfsetdash{}{0pt}%
\pgfpathmoveto{\pgfqpoint{1.844359in}{0.908552in}}%
\pgfpathlineto{\pgfqpoint{1.847382in}{0.914591in}}%
\pgfpathlineto{\pgfqpoint{1.850418in}{0.920988in}}%
\pgfpathlineto{\pgfqpoint{1.853467in}{0.927750in}}%
\pgfpathlineto{\pgfqpoint{1.856531in}{0.934884in}}%
\pgfpathlineto{\pgfqpoint{1.844195in}{0.926526in}}%
\pgfpathlineto{\pgfqpoint{1.831320in}{0.918370in}}%
\pgfpathlineto{\pgfqpoint{1.817918in}{0.910424in}}%
\pgfpathlineto{\pgfqpoint{1.804000in}{0.902698in}}%
\pgfpathlineto{\pgfqpoint{1.801242in}{0.895725in}}%
\pgfpathlineto{\pgfqpoint{1.798497in}{0.889126in}}%
\pgfpathlineto{\pgfqpoint{1.795763in}{0.882893in}}%
\pgfpathlineto{\pgfqpoint{1.793042in}{0.877019in}}%
\pgfpathlineto{\pgfqpoint{1.806635in}{0.884588in}}%
\pgfpathlineto{\pgfqpoint{1.819727in}{0.892372in}}%
\pgfpathlineto{\pgfqpoint{1.832306in}{0.900364in}}%
\pgfpathlineto{\pgfqpoint{1.844359in}{0.908552in}}%
\pgfpathclose%
\pgfusepath{fill}%
\end{pgfscope}%
\begin{pgfscope}%
\pgfpathrectangle{\pgfqpoint{0.329460in}{0.284240in}}{\pgfqpoint{1.989680in}{1.989680in}}%
\pgfusepath{clip}%
\pgfsetbuttcap%
\pgfsetroundjoin%
\definecolor{currentfill}{rgb}{0.282327,0.094955,0.417331}%
\pgfsetfillcolor{currentfill}%
\pgfsetlinewidth{0.000000pt}%
\definecolor{currentstroke}{rgb}{0.000000,0.000000,0.000000}%
\pgfsetstrokecolor{currentstroke}%
\pgfsetdash{}{0pt}%
\pgfpathmoveto{\pgfqpoint{1.771666in}{0.842280in}}%
\pgfpathlineto{\pgfqpoint{1.774302in}{0.845486in}}%
\pgfpathlineto{\pgfqpoint{1.776948in}{0.849005in}}%
\pgfpathlineto{\pgfqpoint{1.779604in}{0.852844in}}%
\pgfpathlineto{\pgfqpoint{1.782270in}{0.857007in}}%
\pgfpathlineto{\pgfqpoint{1.768521in}{0.849821in}}%
\pgfpathlineto{\pgfqpoint{1.754312in}{0.842865in}}%
\pgfpathlineto{\pgfqpoint{1.739655in}{0.836146in}}%
\pgfpathlineto{\pgfqpoint{1.724567in}{0.829674in}}%
\pgfpathlineto{\pgfqpoint{1.722249in}{0.825664in}}%
\pgfpathlineto{\pgfqpoint{1.719939in}{0.821979in}}%
\pgfpathlineto{\pgfqpoint{1.717638in}{0.818614in}}%
\pgfpathlineto{\pgfqpoint{1.715346in}{0.815564in}}%
\pgfpathlineto{\pgfqpoint{1.730070in}{0.821890in}}%
\pgfpathlineto{\pgfqpoint{1.744375in}{0.828457in}}%
\pgfpathlineto{\pgfqpoint{1.758244in}{0.835256in}}%
\pgfpathlineto{\pgfqpoint{1.771666in}{0.842280in}}%
\pgfpathclose%
\pgfusepath{fill}%
\end{pgfscope}%
\begin{pgfscope}%
\pgfpathrectangle{\pgfqpoint{0.329460in}{0.284240in}}{\pgfqpoint{1.989680in}{1.989680in}}%
\pgfusepath{clip}%
\pgfsetbuttcap%
\pgfsetroundjoin%
\definecolor{currentfill}{rgb}{0.636902,0.856542,0.216620}%
\pgfsetfillcolor{currentfill}%
\pgfsetlinewidth{0.000000pt}%
\definecolor{currentstroke}{rgb}{0.000000,0.000000,0.000000}%
\pgfsetstrokecolor{currentstroke}%
\pgfsetdash{}{0pt}%
\pgfpathmoveto{\pgfqpoint{1.243619in}{1.595970in}}%
\pgfpathlineto{\pgfqpoint{1.240936in}{1.590381in}}%
\pgfpathlineto{\pgfqpoint{1.238254in}{1.584698in}}%
\pgfpathlineto{\pgfqpoint{1.235572in}{1.578922in}}%
\pgfpathlineto{\pgfqpoint{1.232892in}{1.573056in}}%
\pgfpathlineto{\pgfqpoint{1.229837in}{1.574866in}}%
\pgfpathlineto{\pgfqpoint{1.226907in}{1.576721in}}%
\pgfpathlineto{\pgfqpoint{1.224103in}{1.578618in}}%
\pgfpathlineto{\pgfqpoint{1.221428in}{1.580556in}}%
\pgfpathlineto{\pgfqpoint{1.224365in}{1.586246in}}%
\pgfpathlineto{\pgfqpoint{1.227302in}{1.591845in}}%
\pgfpathlineto{\pgfqpoint{1.230241in}{1.597352in}}%
\pgfpathlineto{\pgfqpoint{1.233181in}{1.602765in}}%
\pgfpathlineto{\pgfqpoint{1.235617in}{1.601009in}}%
\pgfpathlineto{\pgfqpoint{1.238170in}{1.599290in}}%
\pgfpathlineto{\pgfqpoint{1.240838in}{1.597610in}}%
\pgfpathlineto{\pgfqpoint{1.243619in}{1.595970in}}%
\pgfpathclose%
\pgfusepath{fill}%
\end{pgfscope}%
\begin{pgfscope}%
\pgfpathrectangle{\pgfqpoint{0.329460in}{0.284240in}}{\pgfqpoint{1.989680in}{1.989680in}}%
\pgfusepath{clip}%
\pgfsetbuttcap%
\pgfsetroundjoin%
\definecolor{currentfill}{rgb}{0.282327,0.094955,0.417331}%
\pgfsetfillcolor{currentfill}%
\pgfsetlinewidth{0.000000pt}%
\definecolor{currentstroke}{rgb}{0.000000,0.000000,0.000000}%
\pgfsetstrokecolor{currentstroke}%
\pgfsetdash{}{0pt}%
\pgfpathmoveto{\pgfqpoint{1.299183in}{0.846559in}}%
\pgfpathlineto{\pgfqpoint{1.298786in}{0.840525in}}%
\pgfpathlineto{\pgfqpoint{1.298389in}{0.834628in}}%
\pgfpathlineto{\pgfqpoint{1.297992in}{0.828871in}}%
\pgfpathlineto{\pgfqpoint{1.297594in}{0.823258in}}%
\pgfpathlineto{\pgfqpoint{1.282623in}{0.824297in}}%
\pgfpathlineto{\pgfqpoint{1.267730in}{0.825590in}}%
\pgfpathlineto{\pgfqpoint{1.252928in}{0.827135in}}%
\pgfpathlineto{\pgfqpoint{1.238236in}{0.828931in}}%
\pgfpathlineto{\pgfqpoint{1.239074in}{0.834498in}}%
\pgfpathlineto{\pgfqpoint{1.239912in}{0.840208in}}%
\pgfpathlineto{\pgfqpoint{1.240748in}{0.846058in}}%
\pgfpathlineto{\pgfqpoint{1.241583in}{0.852045in}}%
\pgfpathlineto{\pgfqpoint{1.255841in}{0.850308in}}%
\pgfpathlineto{\pgfqpoint{1.270204in}{0.848814in}}%
\pgfpathlineto{\pgfqpoint{1.284657in}{0.847564in}}%
\pgfpathlineto{\pgfqpoint{1.299183in}{0.846559in}}%
\pgfpathclose%
\pgfusepath{fill}%
\end{pgfscope}%
\begin{pgfscope}%
\pgfpathrectangle{\pgfqpoint{0.329460in}{0.284240in}}{\pgfqpoint{1.989680in}{1.989680in}}%
\pgfusepath{clip}%
\pgfsetbuttcap%
\pgfsetroundjoin%
\definecolor{currentfill}{rgb}{0.699415,0.867117,0.175971}%
\pgfsetfillcolor{currentfill}%
\pgfsetlinewidth{0.000000pt}%
\definecolor{currentstroke}{rgb}{0.000000,0.000000,0.000000}%
\pgfsetstrokecolor{currentstroke}%
\pgfsetdash{}{0pt}%
\pgfpathmoveto{\pgfqpoint{1.459285in}{1.624886in}}%
\pgfpathlineto{\pgfqpoint{1.462280in}{1.619900in}}%
\pgfpathlineto{\pgfqpoint{1.465274in}{1.614815in}}%
\pgfpathlineto{\pgfqpoint{1.468267in}{1.609633in}}%
\pgfpathlineto{\pgfqpoint{1.471259in}{1.604356in}}%
\pgfpathlineto{\pgfqpoint{1.468929in}{1.602568in}}%
\pgfpathlineto{\pgfqpoint{1.466480in}{1.600816in}}%
\pgfpathlineto{\pgfqpoint{1.463914in}{1.599102in}}%
\pgfpathlineto{\pgfqpoint{1.461233in}{1.597426in}}%
\pgfpathlineto{\pgfqpoint{1.458488in}{1.602882in}}%
\pgfpathlineto{\pgfqpoint{1.455742in}{1.608243in}}%
\pgfpathlineto{\pgfqpoint{1.452996in}{1.613505in}}%
\pgfpathlineto{\pgfqpoint{1.450248in}{1.618669in}}%
\pgfpathlineto{\pgfqpoint{1.452664in}{1.620173in}}%
\pgfpathlineto{\pgfqpoint{1.454977in}{1.621711in}}%
\pgfpathlineto{\pgfqpoint{1.457185in}{1.623282in}}%
\pgfpathlineto{\pgfqpoint{1.459285in}{1.624886in}}%
\pgfpathclose%
\pgfusepath{fill}%
\end{pgfscope}%
\begin{pgfscope}%
\pgfpathrectangle{\pgfqpoint{0.329460in}{0.284240in}}{\pgfqpoint{1.989680in}{1.989680in}}%
\pgfusepath{clip}%
\pgfsetbuttcap%
\pgfsetroundjoin%
\definecolor{currentfill}{rgb}{0.955300,0.901065,0.118128}%
\pgfsetfillcolor{currentfill}%
\pgfsetlinewidth{0.000000pt}%
\definecolor{currentstroke}{rgb}{0.000000,0.000000,0.000000}%
\pgfsetstrokecolor{currentstroke}%
\pgfsetdash{}{0pt}%
\pgfpathmoveto{\pgfqpoint{1.318939in}{1.709971in}}%
\pgfpathlineto{\pgfqpoint{1.316248in}{1.707420in}}%
\pgfpathlineto{\pgfqpoint{1.313557in}{1.704751in}}%
\pgfpathlineto{\pgfqpoint{1.310865in}{1.701964in}}%
\pgfpathlineto{\pgfqpoint{1.308173in}{1.699061in}}%
\pgfpathlineto{\pgfqpoint{1.307054in}{1.699705in}}%
\pgfpathlineto{\pgfqpoint{1.305979in}{1.700365in}}%
\pgfpathlineto{\pgfqpoint{1.304950in}{1.701041in}}%
\pgfpathlineto{\pgfqpoint{1.303967in}{1.701732in}}%
\pgfpathlineto{\pgfqpoint{1.306921in}{1.704468in}}%
\pgfpathlineto{\pgfqpoint{1.309874in}{1.707086in}}%
\pgfpathlineto{\pgfqpoint{1.312828in}{1.709588in}}%
\pgfpathlineto{\pgfqpoint{1.315781in}{1.711971in}}%
\pgfpathlineto{\pgfqpoint{1.316519in}{1.711454in}}%
\pgfpathlineto{\pgfqpoint{1.317292in}{1.710948in}}%
\pgfpathlineto{\pgfqpoint{1.318099in}{1.710454in}}%
\pgfpathlineto{\pgfqpoint{1.318939in}{1.709971in}}%
\pgfpathclose%
\pgfusepath{fill}%
\end{pgfscope}%
\begin{pgfscope}%
\pgfpathrectangle{\pgfqpoint{0.329460in}{0.284240in}}{\pgfqpoint{1.989680in}{1.989680in}}%
\pgfusepath{clip}%
\pgfsetbuttcap%
\pgfsetroundjoin%
\definecolor{currentfill}{rgb}{0.248629,0.278775,0.534556}%
\pgfsetfillcolor{currentfill}%
\pgfsetlinewidth{0.000000pt}%
\definecolor{currentstroke}{rgb}{0.000000,0.000000,0.000000}%
\pgfsetstrokecolor{currentstroke}%
\pgfsetdash{}{0pt}%
\pgfpathmoveto{\pgfqpoint{1.307053in}{0.990437in}}%
\pgfpathlineto{\pgfqpoint{1.306661in}{0.982407in}}%
\pgfpathlineto{\pgfqpoint{1.306270in}{0.974446in}}%
\pgfpathlineto{\pgfqpoint{1.305878in}{0.966556in}}%
\pgfpathlineto{\pgfqpoint{1.305486in}{0.958741in}}%
\pgfpathlineto{\pgfqpoint{1.292719in}{0.959610in}}%
\pgfpathlineto{\pgfqpoint{1.280016in}{0.960690in}}%
\pgfpathlineto{\pgfqpoint{1.267391in}{0.961981in}}%
\pgfpathlineto{\pgfqpoint{1.254857in}{0.963482in}}%
\pgfpathlineto{\pgfqpoint{1.255683in}{0.971251in}}%
\pgfpathlineto{\pgfqpoint{1.256508in}{0.979094in}}%
\pgfpathlineto{\pgfqpoint{1.257333in}{0.987009in}}%
\pgfpathlineto{\pgfqpoint{1.258158in}{0.994993in}}%
\pgfpathlineto{\pgfqpoint{1.270263in}{0.993551in}}%
\pgfpathlineto{\pgfqpoint{1.282455in}{0.992310in}}%
\pgfpathlineto{\pgfqpoint{1.294723in}{0.991272in}}%
\pgfpathlineto{\pgfqpoint{1.307053in}{0.990437in}}%
\pgfpathclose%
\pgfusepath{fill}%
\end{pgfscope}%
\begin{pgfscope}%
\pgfpathrectangle{\pgfqpoint{0.329460in}{0.284240in}}{\pgfqpoint{1.989680in}{1.989680in}}%
\pgfusepath{clip}%
\pgfsetbuttcap%
\pgfsetroundjoin%
\definecolor{currentfill}{rgb}{0.195860,0.395433,0.555276}%
\pgfsetfillcolor{currentfill}%
\pgfsetlinewidth{0.000000pt}%
\definecolor{currentstroke}{rgb}{0.000000,0.000000,0.000000}%
\pgfsetstrokecolor{currentstroke}%
\pgfsetdash{}{0pt}%
\pgfpathmoveto{\pgfqpoint{1.268044in}{1.095027in}}%
\pgfpathlineto{\pgfqpoint{1.267220in}{1.086459in}}%
\pgfpathlineto{\pgfqpoint{1.266397in}{1.077922in}}%
\pgfpathlineto{\pgfqpoint{1.265573in}{1.069422in}}%
\pgfpathlineto{\pgfqpoint{1.264750in}{1.060959in}}%
\pgfpathlineto{\pgfqpoint{1.253595in}{1.062469in}}%
\pgfpathlineto{\pgfqpoint{1.242546in}{1.064161in}}%
\pgfpathlineto{\pgfqpoint{1.231614in}{1.066032in}}%
\pgfpathlineto{\pgfqpoint{1.220811in}{1.068081in}}%
\pgfpathlineto{\pgfqpoint{1.222052in}{1.076467in}}%
\pgfpathlineto{\pgfqpoint{1.223293in}{1.084891in}}%
\pgfpathlineto{\pgfqpoint{1.224534in}{1.093350in}}%
\pgfpathlineto{\pgfqpoint{1.225775in}{1.101842in}}%
\pgfpathlineto{\pgfqpoint{1.236168in}{1.099881in}}%
\pgfpathlineto{\pgfqpoint{1.246684in}{1.098091in}}%
\pgfpathlineto{\pgfqpoint{1.257313in}{1.096472in}}%
\pgfpathlineto{\pgfqpoint{1.268044in}{1.095027in}}%
\pgfpathclose%
\pgfusepath{fill}%
\end{pgfscope}%
\begin{pgfscope}%
\pgfpathrectangle{\pgfqpoint{0.329460in}{0.284240in}}{\pgfqpoint{1.989680in}{1.989680in}}%
\pgfusepath{clip}%
\pgfsetbuttcap%
\pgfsetroundjoin%
\definecolor{currentfill}{rgb}{0.279566,0.067836,0.391917}%
\pgfsetfillcolor{currentfill}%
\pgfsetlinewidth{0.000000pt}%
\definecolor{currentstroke}{rgb}{0.000000,0.000000,0.000000}%
\pgfsetstrokecolor{currentstroke}%
\pgfsetdash{}{0pt}%
\pgfpathmoveto{\pgfqpoint{1.477094in}{0.830736in}}%
\pgfpathlineto{\pgfqpoint{1.478030in}{0.825332in}}%
\pgfpathlineto{\pgfqpoint{1.478966in}{0.820079in}}%
\pgfpathlineto{\pgfqpoint{1.479904in}{0.814981in}}%
\pgfpathlineto{\pgfqpoint{1.480843in}{0.810042in}}%
\pgfpathlineto{\pgfqpoint{1.465828in}{0.807959in}}%
\pgfpathlineto{\pgfqpoint{1.450686in}{0.806133in}}%
\pgfpathlineto{\pgfqpoint{1.435433in}{0.804566in}}%
\pgfpathlineto{\pgfqpoint{1.420086in}{0.803259in}}%
\pgfpathlineto{\pgfqpoint{1.419586in}{0.808252in}}%
\pgfpathlineto{\pgfqpoint{1.419088in}{0.813404in}}%
\pgfpathlineto{\pgfqpoint{1.418589in}{0.818711in}}%
\pgfpathlineto{\pgfqpoint{1.418092in}{0.824169in}}%
\pgfpathlineto{\pgfqpoint{1.432995in}{0.825434in}}%
\pgfpathlineto{\pgfqpoint{1.447807in}{0.826951in}}%
\pgfpathlineto{\pgfqpoint{1.462513in}{0.828719in}}%
\pgfpathlineto{\pgfqpoint{1.477094in}{0.830736in}}%
\pgfpathclose%
\pgfusepath{fill}%
\end{pgfscope}%
\begin{pgfscope}%
\pgfpathrectangle{\pgfqpoint{0.329460in}{0.284240in}}{\pgfqpoint{1.989680in}{1.989680in}}%
\pgfusepath{clip}%
\pgfsetbuttcap%
\pgfsetroundjoin%
\definecolor{currentfill}{rgb}{0.974417,0.903590,0.130215}%
\pgfsetfillcolor{currentfill}%
\pgfsetlinewidth{0.000000pt}%
\definecolor{currentstroke}{rgb}{0.000000,0.000000,0.000000}%
\pgfsetstrokecolor{currentstroke}%
\pgfsetdash{}{0pt}%
\pgfpathmoveto{\pgfqpoint{1.370798in}{1.718045in}}%
\pgfpathlineto{\pgfqpoint{1.373253in}{1.715851in}}%
\pgfpathlineto{\pgfqpoint{1.375708in}{1.713538in}}%
\pgfpathlineto{\pgfqpoint{1.378164in}{1.711105in}}%
\pgfpathlineto{\pgfqpoint{1.380620in}{1.708554in}}%
\pgfpathlineto{\pgfqpoint{1.379652in}{1.708126in}}%
\pgfpathlineto{\pgfqpoint{1.378657in}{1.707714in}}%
\pgfpathlineto{\pgfqpoint{1.377633in}{1.707315in}}%
\pgfpathlineto{\pgfqpoint{1.376583in}{1.706933in}}%
\pgfpathlineto{\pgfqpoint{1.374463in}{1.709620in}}%
\pgfpathlineto{\pgfqpoint{1.372344in}{1.712188in}}%
\pgfpathlineto{\pgfqpoint{1.370225in}{1.714637in}}%
\pgfpathlineto{\pgfqpoint{1.368107in}{1.716966in}}%
\pgfpathlineto{\pgfqpoint{1.368807in}{1.717221in}}%
\pgfpathlineto{\pgfqpoint{1.369489in}{1.717486in}}%
\pgfpathlineto{\pgfqpoint{1.370153in}{1.717761in}}%
\pgfpathlineto{\pgfqpoint{1.370798in}{1.718045in}}%
\pgfpathclose%
\pgfusepath{fill}%
\end{pgfscope}%
\begin{pgfscope}%
\pgfpathrectangle{\pgfqpoint{0.329460in}{0.284240in}}{\pgfqpoint{1.989680in}{1.989680in}}%
\pgfusepath{clip}%
\pgfsetbuttcap%
\pgfsetroundjoin%
\definecolor{currentfill}{rgb}{0.271305,0.019942,0.347269}%
\pgfsetfillcolor{currentfill}%
\pgfsetlinewidth{0.000000pt}%
\definecolor{currentstroke}{rgb}{0.000000,0.000000,0.000000}%
\pgfsetstrokecolor{currentstroke}%
\pgfsetdash{}{0pt}%
\pgfpathmoveto{\pgfqpoint{1.231488in}{0.790031in}}%
\pgfpathlineto{\pgfqpoint{1.230638in}{0.785931in}}%
\pgfpathlineto{\pgfqpoint{1.229787in}{0.782013in}}%
\pgfpathlineto{\pgfqpoint{1.228935in}{0.778283in}}%
\pgfpathlineto{\pgfqpoint{1.228081in}{0.774743in}}%
\pgfpathlineto{\pgfqpoint{1.212208in}{0.776989in}}%
\pgfpathlineto{\pgfqpoint{1.196490in}{0.779504in}}%
\pgfpathlineto{\pgfqpoint{1.180945in}{0.782286in}}%
\pgfpathlineto{\pgfqpoint{1.165590in}{0.785331in}}%
\pgfpathlineto{\pgfqpoint{1.166876in}{0.788792in}}%
\pgfpathlineto{\pgfqpoint{1.168159in}{0.792444in}}%
\pgfpathlineto{\pgfqpoint{1.169440in}{0.796282in}}%
\pgfpathlineto{\pgfqpoint{1.170719in}{0.800303in}}%
\pgfpathlineto{\pgfqpoint{1.185652in}{0.797349in}}%
\pgfpathlineto{\pgfqpoint{1.200768in}{0.794650in}}%
\pgfpathlineto{\pgfqpoint{1.216053in}{0.792209in}}%
\pgfpathlineto{\pgfqpoint{1.231488in}{0.790031in}}%
\pgfpathclose%
\pgfusepath{fill}%
\end{pgfscope}%
\begin{pgfscope}%
\pgfpathrectangle{\pgfqpoint{0.329460in}{0.284240in}}{\pgfqpoint{1.989680in}{1.989680in}}%
\pgfusepath{clip}%
\pgfsetbuttcap%
\pgfsetroundjoin%
\definecolor{currentfill}{rgb}{0.272594,0.025563,0.353093}%
\pgfsetfillcolor{currentfill}%
\pgfsetlinewidth{0.000000pt}%
\definecolor{currentstroke}{rgb}{0.000000,0.000000,0.000000}%
\pgfsetstrokecolor{currentstroke}%
\pgfsetdash{}{0pt}%
\pgfpathmoveto{\pgfqpoint{1.697287in}{0.801826in}}%
\pgfpathlineto{\pgfqpoint{1.699519in}{0.802555in}}%
\pgfpathlineto{\pgfqpoint{1.701758in}{0.803556in}}%
\pgfpathlineto{\pgfqpoint{1.704004in}{0.804834in}}%
\pgfpathlineto{\pgfqpoint{1.706257in}{0.806395in}}%
\pgfpathlineto{\pgfqpoint{1.691498in}{0.800461in}}%
\pgfpathlineto{\pgfqpoint{1.676360in}{0.794778in}}%
\pgfpathlineto{\pgfqpoint{1.660860in}{0.789351in}}%
\pgfpathlineto{\pgfqpoint{1.645014in}{0.784189in}}%
\pgfpathlineto{\pgfqpoint{1.643144in}{0.782764in}}%
\pgfpathlineto{\pgfqpoint{1.641279in}{0.781623in}}%
\pgfpathlineto{\pgfqpoint{1.639420in}{0.780759in}}%
\pgfpathlineto{\pgfqpoint{1.637568in}{0.780168in}}%
\pgfpathlineto{\pgfqpoint{1.653018in}{0.785202in}}%
\pgfpathlineto{\pgfqpoint{1.668131in}{0.790495in}}%
\pgfpathlineto{\pgfqpoint{1.682893in}{0.796038in}}%
\pgfpathlineto{\pgfqpoint{1.697287in}{0.801826in}}%
\pgfpathclose%
\pgfusepath{fill}%
\end{pgfscope}%
\begin{pgfscope}%
\pgfpathrectangle{\pgfqpoint{0.329460in}{0.284240in}}{\pgfqpoint{1.989680in}{1.989680in}}%
\pgfusepath{clip}%
\pgfsetbuttcap%
\pgfsetroundjoin%
\definecolor{currentfill}{rgb}{0.231674,0.318106,0.544834}%
\pgfsetfillcolor{currentfill}%
\pgfsetlinewidth{0.000000pt}%
\definecolor{currentstroke}{rgb}{0.000000,0.000000,0.000000}%
\pgfsetstrokecolor{currentstroke}%
\pgfsetdash{}{0pt}%
\pgfpathmoveto{\pgfqpoint{1.451218in}{1.028942in}}%
\pgfpathlineto{\pgfqpoint{1.452136in}{1.020730in}}%
\pgfpathlineto{\pgfqpoint{1.453055in}{1.012574in}}%
\pgfpathlineto{\pgfqpoint{1.453974in}{1.004477in}}%
\pgfpathlineto{\pgfqpoint{1.454893in}{0.996443in}}%
\pgfpathlineto{\pgfqpoint{1.442877in}{0.994823in}}%
\pgfpathlineto{\pgfqpoint{1.430762in}{0.993403in}}%
\pgfpathlineto{\pgfqpoint{1.418560in}{0.992185in}}%
\pgfpathlineto{\pgfqpoint{1.406285in}{0.991169in}}%
\pgfpathlineto{\pgfqpoint{1.405796in}{0.999257in}}%
\pgfpathlineto{\pgfqpoint{1.405308in}{1.007407in}}%
\pgfpathlineto{\pgfqpoint{1.404819in}{1.015616in}}%
\pgfpathlineto{\pgfqpoint{1.404331in}{1.023881in}}%
\pgfpathlineto{\pgfqpoint{1.416171in}{1.024856in}}%
\pgfpathlineto{\pgfqpoint{1.427941in}{1.026025in}}%
\pgfpathlineto{\pgfqpoint{1.439628in}{1.027388in}}%
\pgfpathlineto{\pgfqpoint{1.451218in}{1.028942in}}%
\pgfpathclose%
\pgfusepath{fill}%
\end{pgfscope}%
\begin{pgfscope}%
\pgfpathrectangle{\pgfqpoint{0.329460in}{0.284240in}}{\pgfqpoint{1.989680in}{1.989680in}}%
\pgfusepath{clip}%
\pgfsetbuttcap%
\pgfsetroundjoin%
\definecolor{currentfill}{rgb}{0.974417,0.903590,0.130215}%
\pgfsetfillcolor{currentfill}%
\pgfsetlinewidth{0.000000pt}%
\definecolor{currentstroke}{rgb}{0.000000,0.000000,0.000000}%
\pgfsetstrokecolor{currentstroke}%
\pgfsetdash{}{0pt}%
\pgfpathmoveto{\pgfqpoint{1.334905in}{1.716749in}}%
\pgfpathlineto{\pgfqpoint{1.332866in}{1.714392in}}%
\pgfpathlineto{\pgfqpoint{1.330827in}{1.711916in}}%
\pgfpathlineto{\pgfqpoint{1.328787in}{1.709320in}}%
\pgfpathlineto{\pgfqpoint{1.326747in}{1.706606in}}%
\pgfpathlineto{\pgfqpoint{1.325674in}{1.706974in}}%
\pgfpathlineto{\pgfqpoint{1.324627in}{1.707359in}}%
\pgfpathlineto{\pgfqpoint{1.323607in}{1.707759in}}%
\pgfpathlineto{\pgfqpoint{1.322614in}{1.708173in}}%
\pgfpathlineto{\pgfqpoint{1.324998in}{1.710757in}}%
\pgfpathlineto{\pgfqpoint{1.327383in}{1.713221in}}%
\pgfpathlineto{\pgfqpoint{1.329766in}{1.715566in}}%
\pgfpathlineto{\pgfqpoint{1.332149in}{1.717792in}}%
\pgfpathlineto{\pgfqpoint{1.332811in}{1.717516in}}%
\pgfpathlineto{\pgfqpoint{1.333492in}{1.717250in}}%
\pgfpathlineto{\pgfqpoint{1.334190in}{1.716994in}}%
\pgfpathlineto{\pgfqpoint{1.334905in}{1.716749in}}%
\pgfpathclose%
\pgfusepath{fill}%
\end{pgfscope}%
\begin{pgfscope}%
\pgfpathrectangle{\pgfqpoint{0.329460in}{0.284240in}}{\pgfqpoint{1.989680in}{1.989680in}}%
\pgfusepath{clip}%
\pgfsetbuttcap%
\pgfsetroundjoin%
\definecolor{currentfill}{rgb}{0.147607,0.511733,0.557049}%
\pgfsetfillcolor{currentfill}%
\pgfsetlinewidth{0.000000pt}%
\definecolor{currentstroke}{rgb}{0.000000,0.000000,0.000000}%
\pgfsetstrokecolor{currentstroke}%
\pgfsetdash{}{0pt}%
\pgfpathmoveto{\pgfqpoint{1.240677in}{1.205280in}}%
\pgfpathlineto{\pgfqpoint{1.239434in}{1.196612in}}%
\pgfpathlineto{\pgfqpoint{1.238191in}{1.187943in}}%
\pgfpathlineto{\pgfqpoint{1.236949in}{1.179278in}}%
\pgfpathlineto{\pgfqpoint{1.235706in}{1.170617in}}%
\pgfpathlineto{\pgfqpoint{1.226258in}{1.172556in}}%
\pgfpathlineto{\pgfqpoint{1.216942in}{1.174646in}}%
\pgfpathlineto{\pgfqpoint{1.207770in}{1.176883in}}%
\pgfpathlineto{\pgfqpoint{1.198751in}{1.179267in}}%
\pgfpathlineto{\pgfqpoint{1.200389in}{1.187823in}}%
\pgfpathlineto{\pgfqpoint{1.202027in}{1.196384in}}%
\pgfpathlineto{\pgfqpoint{1.203665in}{1.204949in}}%
\pgfpathlineto{\pgfqpoint{1.205304in}{1.213513in}}%
\pgfpathlineto{\pgfqpoint{1.213938in}{1.211244in}}%
\pgfpathlineto{\pgfqpoint{1.222717in}{1.209114in}}%
\pgfpathlineto{\pgfqpoint{1.231634in}{1.207126in}}%
\pgfpathlineto{\pgfqpoint{1.240677in}{1.205280in}}%
\pgfpathclose%
\pgfusepath{fill}%
\end{pgfscope}%
\begin{pgfscope}%
\pgfpathrectangle{\pgfqpoint{0.329460in}{0.284240in}}{\pgfqpoint{1.989680in}{1.989680in}}%
\pgfusepath{clip}%
\pgfsetbuttcap%
\pgfsetroundjoin%
\definecolor{currentfill}{rgb}{0.935904,0.898570,0.108131}%
\pgfsetfillcolor{currentfill}%
\pgfsetlinewidth{0.000000pt}%
\definecolor{currentstroke}{rgb}{0.000000,0.000000,0.000000}%
\pgfsetstrokecolor{currentstroke}%
\pgfsetdash{}{0pt}%
\pgfpathmoveto{\pgfqpoint{1.399242in}{1.702358in}}%
\pgfpathlineto{\pgfqpoint{1.402248in}{1.699546in}}%
\pgfpathlineto{\pgfqpoint{1.405254in}{1.696617in}}%
\pgfpathlineto{\pgfqpoint{1.408259in}{1.693573in}}%
\pgfpathlineto{\pgfqpoint{1.411264in}{1.690415in}}%
\pgfpathlineto{\pgfqpoint{1.410090in}{1.689533in}}%
\pgfpathlineto{\pgfqpoint{1.408856in}{1.688670in}}%
\pgfpathlineto{\pgfqpoint{1.407564in}{1.687825in}}%
\pgfpathlineto{\pgfqpoint{1.406216in}{1.687000in}}%
\pgfpathlineto{\pgfqpoint{1.403462in}{1.690331in}}%
\pgfpathlineto{\pgfqpoint{1.400707in}{1.693547in}}%
\pgfpathlineto{\pgfqpoint{1.397953in}{1.696648in}}%
\pgfpathlineto{\pgfqpoint{1.395199in}{1.699632in}}%
\pgfpathlineto{\pgfqpoint{1.396279in}{1.700291in}}%
\pgfpathlineto{\pgfqpoint{1.397313in}{1.700966in}}%
\pgfpathlineto{\pgfqpoint{1.398301in}{1.701655in}}%
\pgfpathlineto{\pgfqpoint{1.399242in}{1.702358in}}%
\pgfpathclose%
\pgfusepath{fill}%
\end{pgfscope}%
\begin{pgfscope}%
\pgfpathrectangle{\pgfqpoint{0.329460in}{0.284240in}}{\pgfqpoint{1.989680in}{1.989680in}}%
\pgfusepath{clip}%
\pgfsetbuttcap%
\pgfsetroundjoin%
\definecolor{currentfill}{rgb}{0.280255,0.165693,0.476498}%
\pgfsetfillcolor{currentfill}%
\pgfsetlinewidth{0.000000pt}%
\definecolor{currentstroke}{rgb}{0.000000,0.000000,0.000000}%
\pgfsetstrokecolor{currentstroke}%
\pgfsetdash{}{0pt}%
\pgfpathmoveto{\pgfqpoint{1.412163in}{0.900123in}}%
\pgfpathlineto{\pgfqpoint{1.412655in}{0.893136in}}%
\pgfpathlineto{\pgfqpoint{1.413147in}{0.886256in}}%
\pgfpathlineto{\pgfqpoint{1.413639in}{0.879488in}}%
\pgfpathlineto{\pgfqpoint{1.414132in}{0.872834in}}%
\pgfpathlineto{\pgfqpoint{1.400040in}{0.871890in}}%
\pgfpathlineto{\pgfqpoint{1.385895in}{0.871185in}}%
\pgfpathlineto{\pgfqpoint{1.371711in}{0.870718in}}%
\pgfpathlineto{\pgfqpoint{1.357505in}{0.870491in}}%
\pgfpathlineto{\pgfqpoint{1.357455in}{0.877165in}}%
\pgfpathlineto{\pgfqpoint{1.357406in}{0.883954in}}%
\pgfpathlineto{\pgfqpoint{1.357356in}{0.890855in}}%
\pgfpathlineto{\pgfqpoint{1.357307in}{0.897863in}}%
\pgfpathlineto{\pgfqpoint{1.371069in}{0.898082in}}%
\pgfpathlineto{\pgfqpoint{1.384809in}{0.898532in}}%
\pgfpathlineto{\pgfqpoint{1.398512in}{0.899213in}}%
\pgfpathlineto{\pgfqpoint{1.412163in}{0.900123in}}%
\pgfpathclose%
\pgfusepath{fill}%
\end{pgfscope}%
\begin{pgfscope}%
\pgfpathrectangle{\pgfqpoint{0.329460in}{0.284240in}}{\pgfqpoint{1.989680in}{1.989680in}}%
\pgfusepath{clip}%
\pgfsetbuttcap%
\pgfsetroundjoin%
\definecolor{currentfill}{rgb}{0.274128,0.199721,0.498911}%
\pgfsetfillcolor{currentfill}%
\pgfsetlinewidth{0.000000pt}%
\definecolor{currentstroke}{rgb}{0.000000,0.000000,0.000000}%
\pgfsetstrokecolor{currentstroke}%
\pgfsetdash{}{0pt}%
\pgfpathmoveto{\pgfqpoint{1.410200in}{0.929085in}}%
\pgfpathlineto{\pgfqpoint{1.410690in}{0.921699in}}%
\pgfpathlineto{\pgfqpoint{1.411181in}{0.914409in}}%
\pgfpathlineto{\pgfqpoint{1.411672in}{0.907215in}}%
\pgfpathlineto{\pgfqpoint{1.412163in}{0.900123in}}%
\pgfpathlineto{\pgfqpoint{1.398512in}{0.899213in}}%
\pgfpathlineto{\pgfqpoint{1.384809in}{0.898532in}}%
\pgfpathlineto{\pgfqpoint{1.371069in}{0.898082in}}%
\pgfpathlineto{\pgfqpoint{1.357307in}{0.897863in}}%
\pgfpathlineto{\pgfqpoint{1.357258in}{0.904975in}}%
\pgfpathlineto{\pgfqpoint{1.357208in}{0.912189in}}%
\pgfpathlineto{\pgfqpoint{1.357159in}{0.919501in}}%
\pgfpathlineto{\pgfqpoint{1.357110in}{0.926906in}}%
\pgfpathlineto{\pgfqpoint{1.370429in}{0.927117in}}%
\pgfpathlineto{\pgfqpoint{1.383726in}{0.927551in}}%
\pgfpathlineto{\pgfqpoint{1.396988in}{0.928207in}}%
\pgfpathlineto{\pgfqpoint{1.410200in}{0.929085in}}%
\pgfpathclose%
\pgfusepath{fill}%
\end{pgfscope}%
\begin{pgfscope}%
\pgfpathrectangle{\pgfqpoint{0.329460in}{0.284240in}}{\pgfqpoint{1.989680in}{1.989680in}}%
\pgfusepath{clip}%
\pgfsetbuttcap%
\pgfsetroundjoin%
\definecolor{currentfill}{rgb}{0.412913,0.803041,0.357269}%
\pgfsetfillcolor{currentfill}%
\pgfsetlinewidth{0.000000pt}%
\definecolor{currentstroke}{rgb}{0.000000,0.000000,0.000000}%
\pgfsetstrokecolor{currentstroke}%
\pgfsetdash{}{0pt}%
\pgfpathmoveto{\pgfqpoint{1.227277in}{1.514966in}}%
\pgfpathlineto{\pgfqpoint{1.224904in}{1.508182in}}%
\pgfpathlineto{\pgfqpoint{1.222533in}{1.501322in}}%
\pgfpathlineto{\pgfqpoint{1.220163in}{1.494388in}}%
\pgfpathlineto{\pgfqpoint{1.217794in}{1.487380in}}%
\pgfpathlineto{\pgfqpoint{1.213330in}{1.489459in}}%
\pgfpathlineto{\pgfqpoint{1.209009in}{1.491604in}}%
\pgfpathlineto{\pgfqpoint{1.204834in}{1.493813in}}%
\pgfpathlineto{\pgfqpoint{1.200809in}{1.496084in}}%
\pgfpathlineto{\pgfqpoint{1.203477in}{1.502927in}}%
\pgfpathlineto{\pgfqpoint{1.206145in}{1.509698in}}%
\pgfpathlineto{\pgfqpoint{1.208815in}{1.516394in}}%
\pgfpathlineto{\pgfqpoint{1.211485in}{1.523015in}}%
\pgfpathlineto{\pgfqpoint{1.215228in}{1.520915in}}%
\pgfpathlineto{\pgfqpoint{1.219110in}{1.518871in}}%
\pgfpathlineto{\pgfqpoint{1.223127in}{1.516888in}}%
\pgfpathlineto{\pgfqpoint{1.227277in}{1.514966in}}%
\pgfpathclose%
\pgfusepath{fill}%
\end{pgfscope}%
\begin{pgfscope}%
\pgfpathrectangle{\pgfqpoint{0.329460in}{0.284240in}}{\pgfqpoint{1.989680in}{1.989680in}}%
\pgfusepath{clip}%
\pgfsetbuttcap%
\pgfsetroundjoin%
\definecolor{currentfill}{rgb}{0.762373,0.876424,0.137064}%
\pgfsetfillcolor{currentfill}%
\pgfsetlinewidth{0.000000pt}%
\definecolor{currentstroke}{rgb}{0.000000,0.000000,0.000000}%
\pgfsetstrokecolor{currentstroke}%
\pgfsetdash{}{0pt}%
\pgfpathmoveto{\pgfqpoint{1.447296in}{1.643823in}}%
\pgfpathlineto{\pgfqpoint{1.450295in}{1.639242in}}%
\pgfpathlineto{\pgfqpoint{1.453293in}{1.634558in}}%
\pgfpathlineto{\pgfqpoint{1.456290in}{1.629772in}}%
\pgfpathlineto{\pgfqpoint{1.459285in}{1.624886in}}%
\pgfpathlineto{\pgfqpoint{1.457185in}{1.623282in}}%
\pgfpathlineto{\pgfqpoint{1.454977in}{1.621711in}}%
\pgfpathlineto{\pgfqpoint{1.452664in}{1.620173in}}%
\pgfpathlineto{\pgfqpoint{1.450248in}{1.618669in}}%
\pgfpathlineto{\pgfqpoint{1.447500in}{1.623733in}}%
\pgfpathlineto{\pgfqpoint{1.444751in}{1.628696in}}%
\pgfpathlineto{\pgfqpoint{1.442001in}{1.633557in}}%
\pgfpathlineto{\pgfqpoint{1.439251in}{1.638315in}}%
\pgfpathlineto{\pgfqpoint{1.441402in}{1.639646in}}%
\pgfpathlineto{\pgfqpoint{1.443461in}{1.641009in}}%
\pgfpathlineto{\pgfqpoint{1.445426in}{1.642402in}}%
\pgfpathlineto{\pgfqpoint{1.447296in}{1.643823in}}%
\pgfpathclose%
\pgfusepath{fill}%
\end{pgfscope}%
\begin{pgfscope}%
\pgfpathrectangle{\pgfqpoint{0.329460in}{0.284240in}}{\pgfqpoint{1.989680in}{1.989680in}}%
\pgfusepath{clip}%
\pgfsetbuttcap%
\pgfsetroundjoin%
\definecolor{currentfill}{rgb}{0.179019,0.433756,0.557430}%
\pgfsetfillcolor{currentfill}%
\pgfsetlinewidth{0.000000pt}%
\definecolor{currentstroke}{rgb}{0.000000,0.000000,0.000000}%
\pgfsetstrokecolor{currentstroke}%
\pgfsetdash{}{0pt}%
\pgfpathmoveto{\pgfqpoint{1.480401in}{1.137880in}}%
\pgfpathlineto{\pgfqpoint{1.481733in}{1.129306in}}%
\pgfpathlineto{\pgfqpoint{1.483064in}{1.120754in}}%
\pgfpathlineto{\pgfqpoint{1.484395in}{1.112226in}}%
\pgfpathlineto{\pgfqpoint{1.485725in}{1.103726in}}%
\pgfpathlineto{\pgfqpoint{1.475452in}{1.101616in}}%
\pgfpathlineto{\pgfqpoint{1.465045in}{1.099674in}}%
\pgfpathlineto{\pgfqpoint{1.454515in}{1.097902in}}%
\pgfpathlineto{\pgfqpoint{1.443874in}{1.096303in}}%
\pgfpathlineto{\pgfqpoint{1.442956in}{1.104887in}}%
\pgfpathlineto{\pgfqpoint{1.442038in}{1.113498in}}%
\pgfpathlineto{\pgfqpoint{1.441120in}{1.122132in}}%
\pgfpathlineto{\pgfqpoint{1.440202in}{1.130789in}}%
\pgfpathlineto{\pgfqpoint{1.450423in}{1.132317in}}%
\pgfpathlineto{\pgfqpoint{1.460536in}{1.134009in}}%
\pgfpathlineto{\pgfqpoint{1.470533in}{1.135864in}}%
\pgfpathlineto{\pgfqpoint{1.480401in}{1.137880in}}%
\pgfpathclose%
\pgfusepath{fill}%
\end{pgfscope}%
\begin{pgfscope}%
\pgfpathrectangle{\pgfqpoint{0.329460in}{0.284240in}}{\pgfqpoint{1.989680in}{1.989680in}}%
\pgfusepath{clip}%
\pgfsetbuttcap%
\pgfsetroundjoin%
\definecolor{currentfill}{rgb}{0.233603,0.313828,0.543914}%
\pgfsetfillcolor{currentfill}%
\pgfsetlinewidth{0.000000pt}%
\definecolor{currentstroke}{rgb}{0.000000,0.000000,0.000000}%
\pgfsetstrokecolor{currentstroke}%
\pgfsetdash{}{0pt}%
\pgfpathmoveto{\pgfqpoint{1.913656in}{1.003180in}}%
\pgfpathlineto{\pgfqpoint{1.917045in}{1.012435in}}%
\pgfpathlineto{\pgfqpoint{1.920452in}{1.022098in}}%
\pgfpathlineto{\pgfqpoint{1.923877in}{1.032178in}}%
\pgfpathlineto{\pgfqpoint{1.927319in}{1.042681in}}%
\pgfpathlineto{\pgfqpoint{1.916791in}{1.033279in}}%
\pgfpathlineto{\pgfqpoint{1.905652in}{1.024043in}}%
\pgfpathlineto{\pgfqpoint{1.893910in}{1.014982in}}%
\pgfpathlineto{\pgfqpoint{1.881576in}{1.006107in}}%
\pgfpathlineto{\pgfqpoint{1.878389in}{0.995765in}}%
\pgfpathlineto{\pgfqpoint{1.875219in}{0.985848in}}%
\pgfpathlineto{\pgfqpoint{1.872066in}{0.976349in}}%
\pgfpathlineto{\pgfqpoint{1.868928in}{0.967261in}}%
\pgfpathlineto{\pgfqpoint{1.880985in}{0.975975in}}%
\pgfpathlineto{\pgfqpoint{1.892465in}{0.984874in}}%
\pgfpathlineto{\pgfqpoint{1.903358in}{0.993946in}}%
\pgfpathlineto{\pgfqpoint{1.913656in}{1.003180in}}%
\pgfpathclose%
\pgfusepath{fill}%
\end{pgfscope}%
\begin{pgfscope}%
\pgfpathrectangle{\pgfqpoint{0.329460in}{0.284240in}}{\pgfqpoint{1.989680in}{1.989680in}}%
\pgfusepath{clip}%
\pgfsetbuttcap%
\pgfsetroundjoin%
\definecolor{currentfill}{rgb}{0.280255,0.165693,0.476498}%
\pgfsetfillcolor{currentfill}%
\pgfsetlinewidth{0.000000pt}%
\definecolor{currentstroke}{rgb}{0.000000,0.000000,0.000000}%
\pgfsetstrokecolor{currentstroke}%
\pgfsetdash{}{0pt}%
\pgfpathmoveto{\pgfqpoint{1.357307in}{0.897863in}}%
\pgfpathlineto{\pgfqpoint{1.357356in}{0.890855in}}%
\pgfpathlineto{\pgfqpoint{1.357406in}{0.883954in}}%
\pgfpathlineto{\pgfqpoint{1.357455in}{0.877165in}}%
\pgfpathlineto{\pgfqpoint{1.357505in}{0.870491in}}%
\pgfpathlineto{\pgfqpoint{1.343291in}{0.870505in}}%
\pgfpathlineto{\pgfqpoint{1.329087in}{0.870758in}}%
\pgfpathlineto{\pgfqpoint{1.314906in}{0.871251in}}%
\pgfpathlineto{\pgfqpoint{1.300766in}{0.871983in}}%
\pgfpathlineto{\pgfqpoint{1.301161in}{0.878644in}}%
\pgfpathlineto{\pgfqpoint{1.301555in}{0.885420in}}%
\pgfpathlineto{\pgfqpoint{1.301950in}{0.892307in}}%
\pgfpathlineto{\pgfqpoint{1.302343in}{0.899302in}}%
\pgfpathlineto{\pgfqpoint{1.316041in}{0.898596in}}%
\pgfpathlineto{\pgfqpoint{1.329778in}{0.898120in}}%
\pgfpathlineto{\pgfqpoint{1.343538in}{0.897876in}}%
\pgfpathlineto{\pgfqpoint{1.357307in}{0.897863in}}%
\pgfpathclose%
\pgfusepath{fill}%
\end{pgfscope}%
\begin{pgfscope}%
\pgfpathrectangle{\pgfqpoint{0.329460in}{0.284240in}}{\pgfqpoint{1.989680in}{1.989680in}}%
\pgfusepath{clip}%
\pgfsetbuttcap%
\pgfsetroundjoin%
\definecolor{currentfill}{rgb}{0.274128,0.199721,0.498911}%
\pgfsetfillcolor{currentfill}%
\pgfsetlinewidth{0.000000pt}%
\definecolor{currentstroke}{rgb}{0.000000,0.000000,0.000000}%
\pgfsetstrokecolor{currentstroke}%
\pgfsetdash{}{0pt}%
\pgfpathmoveto{\pgfqpoint{1.357110in}{0.926906in}}%
\pgfpathlineto{\pgfqpoint{1.357159in}{0.919501in}}%
\pgfpathlineto{\pgfqpoint{1.357208in}{0.912189in}}%
\pgfpathlineto{\pgfqpoint{1.357258in}{0.904975in}}%
\pgfpathlineto{\pgfqpoint{1.357307in}{0.897863in}}%
\pgfpathlineto{\pgfqpoint{1.343538in}{0.897876in}}%
\pgfpathlineto{\pgfqpoint{1.329778in}{0.898120in}}%
\pgfpathlineto{\pgfqpoint{1.316041in}{0.898596in}}%
\pgfpathlineto{\pgfqpoint{1.302343in}{0.899302in}}%
\pgfpathlineto{\pgfqpoint{1.302737in}{0.906402in}}%
\pgfpathlineto{\pgfqpoint{1.303130in}{0.913602in}}%
\pgfpathlineto{\pgfqpoint{1.303524in}{0.920901in}}%
\pgfpathlineto{\pgfqpoint{1.303916in}{0.928293in}}%
\pgfpathlineto{\pgfqpoint{1.317173in}{0.927613in}}%
\pgfpathlineto{\pgfqpoint{1.330468in}{0.927154in}}%
\pgfpathlineto{\pgfqpoint{1.343785in}{0.926919in}}%
\pgfpathlineto{\pgfqpoint{1.357110in}{0.926906in}}%
\pgfpathclose%
\pgfusepath{fill}%
\end{pgfscope}%
\begin{pgfscope}%
\pgfpathrectangle{\pgfqpoint{0.329460in}{0.284240in}}{\pgfqpoint{1.989680in}{1.989680in}}%
\pgfusepath{clip}%
\pgfsetbuttcap%
\pgfsetroundjoin%
\definecolor{currentfill}{rgb}{0.699415,0.867117,0.175971}%
\pgfsetfillcolor{currentfill}%
\pgfsetlinewidth{0.000000pt}%
\definecolor{currentstroke}{rgb}{0.000000,0.000000,0.000000}%
\pgfsetstrokecolor{currentstroke}%
\pgfsetdash{}{0pt}%
\pgfpathmoveto{\pgfqpoint{1.254359in}{1.617364in}}%
\pgfpathlineto{\pgfqpoint{1.251673in}{1.612163in}}%
\pgfpathlineto{\pgfqpoint{1.248988in}{1.606862in}}%
\pgfpathlineto{\pgfqpoint{1.246303in}{1.601465in}}%
\pgfpathlineto{\pgfqpoint{1.243619in}{1.595970in}}%
\pgfpathlineto{\pgfqpoint{1.240838in}{1.597610in}}%
\pgfpathlineto{\pgfqpoint{1.238170in}{1.599290in}}%
\pgfpathlineto{\pgfqpoint{1.235617in}{1.601009in}}%
\pgfpathlineto{\pgfqpoint{1.233181in}{1.602765in}}%
\pgfpathlineto{\pgfqpoint{1.236122in}{1.608084in}}%
\pgfpathlineto{\pgfqpoint{1.239064in}{1.613307in}}%
\pgfpathlineto{\pgfqpoint{1.242007in}{1.618432in}}%
\pgfpathlineto{\pgfqpoint{1.244951in}{1.623459in}}%
\pgfpathlineto{\pgfqpoint{1.247147in}{1.621884in}}%
\pgfpathlineto{\pgfqpoint{1.249449in}{1.620342in}}%
\pgfpathlineto{\pgfqpoint{1.251854in}{1.618835in}}%
\pgfpathlineto{\pgfqpoint{1.254359in}{1.617364in}}%
\pgfpathclose%
\pgfusepath{fill}%
\end{pgfscope}%
\begin{pgfscope}%
\pgfpathrectangle{\pgfqpoint{0.329460in}{0.284240in}}{\pgfqpoint{1.989680in}{1.989680in}}%
\pgfusepath{clip}%
\pgfsetbuttcap%
\pgfsetroundjoin%
\definecolor{currentfill}{rgb}{0.120081,0.622161,0.534946}%
\pgfsetfillcolor{currentfill}%
\pgfsetlinewidth{0.000000pt}%
\definecolor{currentstroke}{rgb}{0.000000,0.000000,0.000000}%
\pgfsetstrokecolor{currentstroke}%
\pgfsetdash{}{0pt}%
\pgfpathmoveto{\pgfqpoint{1.225011in}{1.315390in}}%
\pgfpathlineto{\pgfqpoint{1.223366in}{1.307017in}}%
\pgfpathlineto{\pgfqpoint{1.221721in}{1.298615in}}%
\pgfpathlineto{\pgfqpoint{1.220077in}{1.290186in}}%
\pgfpathlineto{\pgfqpoint{1.218433in}{1.281733in}}%
\pgfpathlineto{\pgfqpoint{1.210713in}{1.283898in}}%
\pgfpathlineto{\pgfqpoint{1.203141in}{1.286183in}}%
\pgfpathlineto{\pgfqpoint{1.195725in}{1.288586in}}%
\pgfpathlineto{\pgfqpoint{1.188472in}{1.291104in}}%
\pgfpathlineto{\pgfqpoint{1.190484in}{1.299429in}}%
\pgfpathlineto{\pgfqpoint{1.192496in}{1.307730in}}%
\pgfpathlineto{\pgfqpoint{1.194509in}{1.316005in}}%
\pgfpathlineto{\pgfqpoint{1.196523in}{1.324250in}}%
\pgfpathlineto{\pgfqpoint{1.203420in}{1.321869in}}%
\pgfpathlineto{\pgfqpoint{1.210472in}{1.319597in}}%
\pgfpathlineto{\pgfqpoint{1.217671in}{1.317437in}}%
\pgfpathlineto{\pgfqpoint{1.225011in}{1.315390in}}%
\pgfpathclose%
\pgfusepath{fill}%
\end{pgfscope}%
\begin{pgfscope}%
\pgfpathrectangle{\pgfqpoint{0.329460in}{0.284240in}}{\pgfqpoint{1.989680in}{1.989680in}}%
\pgfusepath{clip}%
\pgfsetbuttcap%
\pgfsetroundjoin%
\definecolor{currentfill}{rgb}{0.283072,0.130895,0.449241}%
\pgfsetfillcolor{currentfill}%
\pgfsetlinewidth{0.000000pt}%
\definecolor{currentstroke}{rgb}{0.000000,0.000000,0.000000}%
\pgfsetstrokecolor{currentstroke}%
\pgfsetdash{}{0pt}%
\pgfpathmoveto{\pgfqpoint{1.414132in}{0.872834in}}%
\pgfpathlineto{\pgfqpoint{1.414625in}{0.866299in}}%
\pgfpathlineto{\pgfqpoint{1.415119in}{0.859886in}}%
\pgfpathlineto{\pgfqpoint{1.415613in}{0.853599in}}%
\pgfpathlineto{\pgfqpoint{1.416108in}{0.847440in}}%
\pgfpathlineto{\pgfqpoint{1.401574in}{0.846462in}}%
\pgfpathlineto{\pgfqpoint{1.386985in}{0.845732in}}%
\pgfpathlineto{\pgfqpoint{1.372356in}{0.845249in}}%
\pgfpathlineto{\pgfqpoint{1.357703in}{0.845014in}}%
\pgfpathlineto{\pgfqpoint{1.357653in}{0.851193in}}%
\pgfpathlineto{\pgfqpoint{1.357604in}{0.857502in}}%
\pgfpathlineto{\pgfqpoint{1.357554in}{0.863935in}}%
\pgfpathlineto{\pgfqpoint{1.357505in}{0.870491in}}%
\pgfpathlineto{\pgfqpoint{1.371711in}{0.870718in}}%
\pgfpathlineto{\pgfqpoint{1.385895in}{0.871185in}}%
\pgfpathlineto{\pgfqpoint{1.400040in}{0.871890in}}%
\pgfpathlineto{\pgfqpoint{1.414132in}{0.872834in}}%
\pgfpathclose%
\pgfusepath{fill}%
\end{pgfscope}%
\begin{pgfscope}%
\pgfpathrectangle{\pgfqpoint{0.329460in}{0.284240in}}{\pgfqpoint{1.989680in}{1.989680in}}%
\pgfusepath{clip}%
\pgfsetbuttcap%
\pgfsetroundjoin%
\definecolor{currentfill}{rgb}{0.220124,0.725509,0.466226}%
\pgfsetfillcolor{currentfill}%
\pgfsetlinewidth{0.000000pt}%
\definecolor{currentstroke}{rgb}{0.000000,0.000000,0.000000}%
\pgfsetstrokecolor{currentstroke}%
\pgfsetdash{}{0pt}%
\pgfpathmoveto{\pgfqpoint{1.220755in}{1.420144in}}%
\pgfpathlineto{\pgfqpoint{1.218731in}{1.412415in}}%
\pgfpathlineto{\pgfqpoint{1.216708in}{1.404632in}}%
\pgfpathlineto{\pgfqpoint{1.214686in}{1.396796in}}%
\pgfpathlineto{\pgfqpoint{1.212665in}{1.388909in}}%
\pgfpathlineto{\pgfqpoint{1.206627in}{1.391114in}}%
\pgfpathlineto{\pgfqpoint{1.200740in}{1.393410in}}%
\pgfpathlineto{\pgfqpoint{1.195010in}{1.395796in}}%
\pgfpathlineto{\pgfqpoint{1.189442in}{1.398269in}}%
\pgfpathlineto{\pgfqpoint{1.191799in}{1.406007in}}%
\pgfpathlineto{\pgfqpoint{1.194157in}{1.413695in}}%
\pgfpathlineto{\pgfqpoint{1.196516in}{1.421331in}}%
\pgfpathlineto{\pgfqpoint{1.198876in}{1.428913in}}%
\pgfpathlineto{\pgfqpoint{1.204122in}{1.426596in}}%
\pgfpathlineto{\pgfqpoint{1.209521in}{1.424360in}}%
\pgfpathlineto{\pgfqpoint{1.215068in}{1.422209in}}%
\pgfpathlineto{\pgfqpoint{1.220755in}{1.420144in}}%
\pgfpathclose%
\pgfusepath{fill}%
\end{pgfscope}%
\begin{pgfscope}%
\pgfpathrectangle{\pgfqpoint{0.329460in}{0.284240in}}{\pgfqpoint{1.989680in}{1.989680in}}%
\pgfusepath{clip}%
\pgfsetbuttcap%
\pgfsetroundjoin%
\definecolor{currentfill}{rgb}{0.263663,0.237631,0.518762}%
\pgfsetfillcolor{currentfill}%
\pgfsetlinewidth{0.000000pt}%
\definecolor{currentstroke}{rgb}{0.000000,0.000000,0.000000}%
\pgfsetstrokecolor{currentstroke}%
\pgfsetdash{}{0pt}%
\pgfpathmoveto{\pgfqpoint{1.408241in}{0.959503in}}%
\pgfpathlineto{\pgfqpoint{1.408730in}{0.951773in}}%
\pgfpathlineto{\pgfqpoint{1.409220in}{0.944125in}}%
\pgfpathlineto{\pgfqpoint{1.409709in}{0.936561in}}%
\pgfpathlineto{\pgfqpoint{1.410200in}{0.929085in}}%
\pgfpathlineto{\pgfqpoint{1.396988in}{0.928207in}}%
\pgfpathlineto{\pgfqpoint{1.383726in}{0.927551in}}%
\pgfpathlineto{\pgfqpoint{1.370429in}{0.927117in}}%
\pgfpathlineto{\pgfqpoint{1.357110in}{0.926906in}}%
\pgfpathlineto{\pgfqpoint{1.357061in}{0.934403in}}%
\pgfpathlineto{\pgfqpoint{1.357012in}{0.941988in}}%
\pgfpathlineto{\pgfqpoint{1.356962in}{0.949656in}}%
\pgfpathlineto{\pgfqpoint{1.356913in}{0.957406in}}%
\pgfpathlineto{\pgfqpoint{1.369790in}{0.957609in}}%
\pgfpathlineto{\pgfqpoint{1.382646in}{0.958027in}}%
\pgfpathlineto{\pgfqpoint{1.395467in}{0.958658in}}%
\pgfpathlineto{\pgfqpoint{1.408241in}{0.959503in}}%
\pgfpathclose%
\pgfusepath{fill}%
\end{pgfscope}%
\begin{pgfscope}%
\pgfpathrectangle{\pgfqpoint{0.329460in}{0.284240in}}{\pgfqpoint{1.989680in}{1.989680in}}%
\pgfusepath{clip}%
\pgfsetbuttcap%
\pgfsetroundjoin%
\definecolor{currentfill}{rgb}{0.487026,0.823929,0.312321}%
\pgfsetfillcolor{currentfill}%
\pgfsetlinewidth{0.000000pt}%
\definecolor{currentstroke}{rgb}{0.000000,0.000000,0.000000}%
\pgfsetstrokecolor{currentstroke}%
\pgfsetdash{}{0pt}%
\pgfpathmoveto{\pgfqpoint{1.483159in}{1.550465in}}%
\pgfpathlineto{\pgfqpoint{1.485895in}{1.544203in}}%
\pgfpathlineto{\pgfqpoint{1.488630in}{1.537858in}}%
\pgfpathlineto{\pgfqpoint{1.491364in}{1.531433in}}%
\pgfpathlineto{\pgfqpoint{1.494096in}{1.524928in}}%
\pgfpathlineto{\pgfqpoint{1.490481in}{1.522779in}}%
\pgfpathlineto{\pgfqpoint{1.486723in}{1.520685in}}%
\pgfpathlineto{\pgfqpoint{1.482826in}{1.518648in}}%
\pgfpathlineto{\pgfqpoint{1.478793in}{1.516671in}}%
\pgfpathlineto{\pgfqpoint{1.476350in}{1.523343in}}%
\pgfpathlineto{\pgfqpoint{1.473906in}{1.529935in}}%
\pgfpathlineto{\pgfqpoint{1.471461in}{1.536445in}}%
\pgfpathlineto{\pgfqpoint{1.469015in}{1.542874in}}%
\pgfpathlineto{\pgfqpoint{1.472742in}{1.544691in}}%
\pgfpathlineto{\pgfqpoint{1.476343in}{1.546563in}}%
\pgfpathlineto{\pgfqpoint{1.479817in}{1.548489in}}%
\pgfpathlineto{\pgfqpoint{1.483159in}{1.550465in}}%
\pgfpathclose%
\pgfusepath{fill}%
\end{pgfscope}%
\begin{pgfscope}%
\pgfpathrectangle{\pgfqpoint{0.329460in}{0.284240in}}{\pgfqpoint{1.989680in}{1.989680in}}%
\pgfusepath{clip}%
\pgfsetbuttcap%
\pgfsetroundjoin%
\definecolor{currentfill}{rgb}{0.935904,0.898570,0.108131}%
\pgfsetfillcolor{currentfill}%
\pgfsetlinewidth{0.000000pt}%
\definecolor{currentstroke}{rgb}{0.000000,0.000000,0.000000}%
\pgfsetstrokecolor{currentstroke}%
\pgfsetdash{}{0pt}%
\pgfpathmoveto{\pgfqpoint{1.308173in}{1.699061in}}%
\pgfpathlineto{\pgfqpoint{1.305481in}{1.696040in}}%
\pgfpathlineto{\pgfqpoint{1.302789in}{1.692903in}}%
\pgfpathlineto{\pgfqpoint{1.300097in}{1.689651in}}%
\pgfpathlineto{\pgfqpoint{1.297404in}{1.686283in}}%
\pgfpathlineto{\pgfqpoint{1.296007in}{1.687090in}}%
\pgfpathlineto{\pgfqpoint{1.294664in}{1.687918in}}%
\pgfpathlineto{\pgfqpoint{1.293379in}{1.688765in}}%
\pgfpathlineto{\pgfqpoint{1.292152in}{1.689630in}}%
\pgfpathlineto{\pgfqpoint{1.295105in}{1.692829in}}%
\pgfpathlineto{\pgfqpoint{1.298059in}{1.695912in}}%
\pgfpathlineto{\pgfqpoint{1.301013in}{1.698880in}}%
\pgfpathlineto{\pgfqpoint{1.303967in}{1.701732in}}%
\pgfpathlineto{\pgfqpoint{1.304950in}{1.701041in}}%
\pgfpathlineto{\pgfqpoint{1.305979in}{1.700365in}}%
\pgfpathlineto{\pgfqpoint{1.307054in}{1.699705in}}%
\pgfpathlineto{\pgfqpoint{1.308173in}{1.699061in}}%
\pgfpathclose%
\pgfusepath{fill}%
\end{pgfscope}%
\begin{pgfscope}%
\pgfpathrectangle{\pgfqpoint{0.329460in}{0.284240in}}{\pgfqpoint{1.989680in}{1.989680in}}%
\pgfusepath{clip}%
\pgfsetbuttcap%
\pgfsetroundjoin%
\definecolor{currentfill}{rgb}{0.231674,0.318106,0.544834}%
\pgfsetfillcolor{currentfill}%
\pgfsetlinewidth{0.000000pt}%
\definecolor{currentstroke}{rgb}{0.000000,0.000000,0.000000}%
\pgfsetstrokecolor{currentstroke}%
\pgfsetdash{}{0pt}%
\pgfpathmoveto{\pgfqpoint{1.308618in}{1.023179in}}%
\pgfpathlineto{\pgfqpoint{1.308227in}{1.014906in}}%
\pgfpathlineto{\pgfqpoint{1.307836in}{1.006690in}}%
\pgfpathlineto{\pgfqpoint{1.307444in}{0.998532in}}%
\pgfpathlineto{\pgfqpoint{1.307053in}{0.990437in}}%
\pgfpathlineto{\pgfqpoint{1.294723in}{0.991272in}}%
\pgfpathlineto{\pgfqpoint{1.282455in}{0.992310in}}%
\pgfpathlineto{\pgfqpoint{1.270263in}{0.993551in}}%
\pgfpathlineto{\pgfqpoint{1.258158in}{0.994993in}}%
\pgfpathlineto{\pgfqpoint{1.258982in}{1.003042in}}%
\pgfpathlineto{\pgfqpoint{1.259807in}{1.011154in}}%
\pgfpathlineto{\pgfqpoint{1.260631in}{1.019324in}}%
\pgfpathlineto{\pgfqpoint{1.261455in}{1.027551in}}%
\pgfpathlineto{\pgfqpoint{1.273131in}{1.026167in}}%
\pgfpathlineto{\pgfqpoint{1.284892in}{1.024976in}}%
\pgfpathlineto{\pgfqpoint{1.296725in}{1.023980in}}%
\pgfpathlineto{\pgfqpoint{1.308618in}{1.023179in}}%
\pgfpathclose%
\pgfusepath{fill}%
\end{pgfscope}%
\begin{pgfscope}%
\pgfpathrectangle{\pgfqpoint{0.329460in}{0.284240in}}{\pgfqpoint{1.989680in}{1.989680in}}%
\pgfusepath{clip}%
\pgfsetbuttcap%
\pgfsetroundjoin%
\definecolor{currentfill}{rgb}{0.279566,0.067836,0.391917}%
\pgfsetfillcolor{currentfill}%
\pgfsetlinewidth{0.000000pt}%
\definecolor{currentstroke}{rgb}{0.000000,0.000000,0.000000}%
\pgfsetstrokecolor{currentstroke}%
\pgfsetdash{}{0pt}%
\pgfpathmoveto{\pgfqpoint{1.297594in}{0.823258in}}%
\pgfpathlineto{\pgfqpoint{1.297195in}{0.817792in}}%
\pgfpathlineto{\pgfqpoint{1.296796in}{0.812478in}}%
\pgfpathlineto{\pgfqpoint{1.296396in}{0.807318in}}%
\pgfpathlineto{\pgfqpoint{1.295996in}{0.802318in}}%
\pgfpathlineto{\pgfqpoint{1.280580in}{0.803391in}}%
\pgfpathlineto{\pgfqpoint{1.265242in}{0.804727in}}%
\pgfpathlineto{\pgfqpoint{1.250001in}{0.806323in}}%
\pgfpathlineto{\pgfqpoint{1.234872in}{0.808178in}}%
\pgfpathlineto{\pgfqpoint{1.235715in}{0.813132in}}%
\pgfpathlineto{\pgfqpoint{1.236556in}{0.818244in}}%
\pgfpathlineto{\pgfqpoint{1.237397in}{0.823512in}}%
\pgfpathlineto{\pgfqpoint{1.238236in}{0.828931in}}%
\pgfpathlineto{\pgfqpoint{1.252928in}{0.827135in}}%
\pgfpathlineto{\pgfqpoint{1.267730in}{0.825590in}}%
\pgfpathlineto{\pgfqpoint{1.282623in}{0.824297in}}%
\pgfpathlineto{\pgfqpoint{1.297594in}{0.823258in}}%
\pgfpathclose%
\pgfusepath{fill}%
\end{pgfscope}%
\begin{pgfscope}%
\pgfpathrectangle{\pgfqpoint{0.329460in}{0.284240in}}{\pgfqpoint{1.989680in}{1.989680in}}%
\pgfusepath{clip}%
\pgfsetbuttcap%
\pgfsetroundjoin%
\definecolor{currentfill}{rgb}{0.955300,0.901065,0.118128}%
\pgfsetfillcolor{currentfill}%
\pgfsetlinewidth{0.000000pt}%
\definecolor{currentstroke}{rgb}{0.000000,0.000000,0.000000}%
\pgfsetstrokecolor{currentstroke}%
\pgfsetdash{}{0pt}%
\pgfpathmoveto{\pgfqpoint{1.384184in}{1.710399in}}%
\pgfpathlineto{\pgfqpoint{1.386938in}{1.707884in}}%
\pgfpathlineto{\pgfqpoint{1.389691in}{1.705251in}}%
\pgfpathlineto{\pgfqpoint{1.392445in}{1.702500in}}%
\pgfpathlineto{\pgfqpoint{1.395199in}{1.699632in}}%
\pgfpathlineto{\pgfqpoint{1.394075in}{1.698990in}}%
\pgfpathlineto{\pgfqpoint{1.392908in}{1.698364in}}%
\pgfpathlineto{\pgfqpoint{1.391699in}{1.697756in}}%
\pgfpathlineto{\pgfqpoint{1.390449in}{1.697166in}}%
\pgfpathlineto{\pgfqpoint{1.387991in}{1.700190in}}%
\pgfpathlineto{\pgfqpoint{1.385534in}{1.703096in}}%
\pgfpathlineto{\pgfqpoint{1.383076in}{1.705884in}}%
\pgfpathlineto{\pgfqpoint{1.380620in}{1.708554in}}%
\pgfpathlineto{\pgfqpoint{1.381558in}{1.708995in}}%
\pgfpathlineto{\pgfqpoint{1.382465in}{1.709450in}}%
\pgfpathlineto{\pgfqpoint{1.383341in}{1.709919in}}%
\pgfpathlineto{\pgfqpoint{1.384184in}{1.710399in}}%
\pgfpathclose%
\pgfusepath{fill}%
\end{pgfscope}%
\begin{pgfscope}%
\pgfpathrectangle{\pgfqpoint{0.329460in}{0.284240in}}{\pgfqpoint{1.989680in}{1.989680in}}%
\pgfusepath{clip}%
\pgfsetbuttcap%
\pgfsetroundjoin%
\definecolor{currentfill}{rgb}{0.974417,0.903590,0.130215}%
\pgfsetfillcolor{currentfill}%
\pgfsetlinewidth{0.000000pt}%
\definecolor{currentstroke}{rgb}{0.000000,0.000000,0.000000}%
\pgfsetstrokecolor{currentstroke}%
\pgfsetdash{}{0pt}%
\pgfpathmoveto{\pgfqpoint{1.368107in}{1.716966in}}%
\pgfpathlineto{\pgfqpoint{1.370225in}{1.714637in}}%
\pgfpathlineto{\pgfqpoint{1.372344in}{1.712188in}}%
\pgfpathlineto{\pgfqpoint{1.374463in}{1.709620in}}%
\pgfpathlineto{\pgfqpoint{1.376583in}{1.706933in}}%
\pgfpathlineto{\pgfqpoint{1.375507in}{1.706566in}}%
\pgfpathlineto{\pgfqpoint{1.374407in}{1.706214in}}%
\pgfpathlineto{\pgfqpoint{1.373284in}{1.705880in}}%
\pgfpathlineto{\pgfqpoint{1.372138in}{1.705562in}}%
\pgfpathlineto{\pgfqpoint{1.370389in}{1.708364in}}%
\pgfpathlineto{\pgfqpoint{1.368640in}{1.711047in}}%
\pgfpathlineto{\pgfqpoint{1.366892in}{1.713611in}}%
\pgfpathlineto{\pgfqpoint{1.365145in}{1.716055in}}%
\pgfpathlineto{\pgfqpoint{1.365908in}{1.716266in}}%
\pgfpathlineto{\pgfqpoint{1.366657in}{1.716489in}}%
\pgfpathlineto{\pgfqpoint{1.367390in}{1.716722in}}%
\pgfpathlineto{\pgfqpoint{1.368107in}{1.716966in}}%
\pgfpathclose%
\pgfusepath{fill}%
\end{pgfscope}%
\begin{pgfscope}%
\pgfpathrectangle{\pgfqpoint{0.329460in}{0.284240in}}{\pgfqpoint{1.989680in}{1.989680in}}%
\pgfusepath{clip}%
\pgfsetbuttcap%
\pgfsetroundjoin%
\definecolor{currentfill}{rgb}{0.896320,0.893616,0.096335}%
\pgfsetfillcolor{currentfill}%
\pgfsetlinewidth{0.000000pt}%
\definecolor{currentstroke}{rgb}{0.000000,0.000000,0.000000}%
\pgfsetstrokecolor{currentstroke}%
\pgfsetdash{}{0pt}%
\pgfpathmoveto{\pgfqpoint{1.411264in}{1.690415in}}%
\pgfpathlineto{\pgfqpoint{1.414269in}{1.687142in}}%
\pgfpathlineto{\pgfqpoint{1.417274in}{1.683756in}}%
\pgfpathlineto{\pgfqpoint{1.420279in}{1.680257in}}%
\pgfpathlineto{\pgfqpoint{1.423283in}{1.676646in}}%
\pgfpathlineto{\pgfqpoint{1.421876in}{1.675586in}}%
\pgfpathlineto{\pgfqpoint{1.420397in}{1.674547in}}%
\pgfpathlineto{\pgfqpoint{1.418849in}{1.673531in}}%
\pgfpathlineto{\pgfqpoint{1.417232in}{1.672538in}}%
\pgfpathlineto{\pgfqpoint{1.414478in}{1.676322in}}%
\pgfpathlineto{\pgfqpoint{1.411724in}{1.679995in}}%
\pgfpathlineto{\pgfqpoint{1.408970in}{1.683554in}}%
\pgfpathlineto{\pgfqpoint{1.406216in}{1.687000in}}%
\pgfpathlineto{\pgfqpoint{1.407564in}{1.687825in}}%
\pgfpathlineto{\pgfqpoint{1.408856in}{1.688670in}}%
\pgfpathlineto{\pgfqpoint{1.410090in}{1.689533in}}%
\pgfpathlineto{\pgfqpoint{1.411264in}{1.690415in}}%
\pgfpathclose%
\pgfusepath{fill}%
\end{pgfscope}%
\begin{pgfscope}%
\pgfpathrectangle{\pgfqpoint{0.329460in}{0.284240in}}{\pgfqpoint{1.989680in}{1.989680in}}%
\pgfusepath{clip}%
\pgfsetbuttcap%
\pgfsetroundjoin%
\definecolor{currentfill}{rgb}{0.267004,0.004874,0.329415}%
\pgfsetfillcolor{currentfill}%
\pgfsetlinewidth{0.000000pt}%
\definecolor{currentstroke}{rgb}{0.000000,0.000000,0.000000}%
\pgfsetstrokecolor{currentstroke}%
\pgfsetdash{}{0pt}%
\pgfpathmoveto{\pgfqpoint{1.622932in}{0.784664in}}%
\pgfpathlineto{\pgfqpoint{1.624744in}{0.783249in}}%
\pgfpathlineto{\pgfqpoint{1.626561in}{0.782069in}}%
\pgfpathlineto{\pgfqpoint{1.628383in}{0.781127in}}%
\pgfpathlineto{\pgfqpoint{1.630210in}{0.780429in}}%
\pgfpathlineto{\pgfqpoint{1.614841in}{0.775781in}}%
\pgfpathlineto{\pgfqpoint{1.599177in}{0.771396in}}%
\pgfpathlineto{\pgfqpoint{1.583236in}{0.767279in}}%
\pgfpathlineto{\pgfqpoint{1.567036in}{0.763435in}}%
\pgfpathlineto{\pgfqpoint{1.565619in}{0.764247in}}%
\pgfpathlineto{\pgfqpoint{1.564206in}{0.765302in}}%
\pgfpathlineto{\pgfqpoint{1.562797in}{0.766596in}}%
\pgfpathlineto{\pgfqpoint{1.561392in}{0.768124in}}%
\pgfpathlineto{\pgfqpoint{1.577172in}{0.771865in}}%
\pgfpathlineto{\pgfqpoint{1.592700in}{0.775872in}}%
\pgfpathlineto{\pgfqpoint{1.607958in}{0.780140in}}%
\pgfpathlineto{\pgfqpoint{1.622932in}{0.784664in}}%
\pgfpathclose%
\pgfusepath{fill}%
\end{pgfscope}%
\begin{pgfscope}%
\pgfpathrectangle{\pgfqpoint{0.329460in}{0.284240in}}{\pgfqpoint{1.989680in}{1.989680in}}%
\pgfusepath{clip}%
\pgfsetbuttcap%
\pgfsetroundjoin%
\definecolor{currentfill}{rgb}{0.283072,0.130895,0.449241}%
\pgfsetfillcolor{currentfill}%
\pgfsetlinewidth{0.000000pt}%
\definecolor{currentstroke}{rgb}{0.000000,0.000000,0.000000}%
\pgfsetstrokecolor{currentstroke}%
\pgfsetdash{}{0pt}%
\pgfpathmoveto{\pgfqpoint{1.357505in}{0.870491in}}%
\pgfpathlineto{\pgfqpoint{1.357554in}{0.863935in}}%
\pgfpathlineto{\pgfqpoint{1.357604in}{0.857502in}}%
\pgfpathlineto{\pgfqpoint{1.357653in}{0.851193in}}%
\pgfpathlineto{\pgfqpoint{1.357703in}{0.845014in}}%
\pgfpathlineto{\pgfqpoint{1.343043in}{0.845028in}}%
\pgfpathlineto{\pgfqpoint{1.328393in}{0.845290in}}%
\pgfpathlineto{\pgfqpoint{1.313767in}{0.845801in}}%
\pgfpathlineto{\pgfqpoint{1.299183in}{0.846559in}}%
\pgfpathlineto{\pgfqpoint{1.299580in}{0.852725in}}%
\pgfpathlineto{\pgfqpoint{1.299975in}{0.859020in}}%
\pgfpathlineto{\pgfqpoint{1.300371in}{0.865441in}}%
\pgfpathlineto{\pgfqpoint{1.300766in}{0.871983in}}%
\pgfpathlineto{\pgfqpoint{1.314906in}{0.871251in}}%
\pgfpathlineto{\pgfqpoint{1.329087in}{0.870758in}}%
\pgfpathlineto{\pgfqpoint{1.343291in}{0.870505in}}%
\pgfpathlineto{\pgfqpoint{1.357505in}{0.870491in}}%
\pgfpathclose%
\pgfusepath{fill}%
\end{pgfscope}%
\begin{pgfscope}%
\pgfpathrectangle{\pgfqpoint{0.329460in}{0.284240in}}{\pgfqpoint{1.989680in}{1.989680in}}%
\pgfusepath{clip}%
\pgfsetbuttcap%
\pgfsetroundjoin%
\definecolor{currentfill}{rgb}{0.263663,0.237631,0.518762}%
\pgfsetfillcolor{currentfill}%
\pgfsetlinewidth{0.000000pt}%
\definecolor{currentstroke}{rgb}{0.000000,0.000000,0.000000}%
\pgfsetstrokecolor{currentstroke}%
\pgfsetdash{}{0pt}%
\pgfpathmoveto{\pgfqpoint{1.356913in}{0.957406in}}%
\pgfpathlineto{\pgfqpoint{1.356962in}{0.949656in}}%
\pgfpathlineto{\pgfqpoint{1.357012in}{0.941988in}}%
\pgfpathlineto{\pgfqpoint{1.357061in}{0.934403in}}%
\pgfpathlineto{\pgfqpoint{1.357110in}{0.926906in}}%
\pgfpathlineto{\pgfqpoint{1.343785in}{0.926919in}}%
\pgfpathlineto{\pgfqpoint{1.330468in}{0.927154in}}%
\pgfpathlineto{\pgfqpoint{1.317173in}{0.927613in}}%
\pgfpathlineto{\pgfqpoint{1.303916in}{0.928293in}}%
\pgfpathlineto{\pgfqpoint{1.304309in}{0.935777in}}%
\pgfpathlineto{\pgfqpoint{1.304702in}{0.943348in}}%
\pgfpathlineto{\pgfqpoint{1.305094in}{0.951004in}}%
\pgfpathlineto{\pgfqpoint{1.305486in}{0.958741in}}%
\pgfpathlineto{\pgfqpoint{1.318303in}{0.958086in}}%
\pgfpathlineto{\pgfqpoint{1.331156in}{0.957645in}}%
\pgfpathlineto{\pgfqpoint{1.344031in}{0.957418in}}%
\pgfpathlineto{\pgfqpoint{1.356913in}{0.957406in}}%
\pgfpathclose%
\pgfusepath{fill}%
\end{pgfscope}%
\begin{pgfscope}%
\pgfpathrectangle{\pgfqpoint{0.329460in}{0.284240in}}{\pgfqpoint{1.989680in}{1.989680in}}%
\pgfusepath{clip}%
\pgfsetbuttcap%
\pgfsetroundjoin%
\definecolor{currentfill}{rgb}{0.814576,0.883393,0.110347}%
\pgfsetfillcolor{currentfill}%
\pgfsetlinewidth{0.000000pt}%
\definecolor{currentstroke}{rgb}{0.000000,0.000000,0.000000}%
\pgfsetstrokecolor{currentstroke}%
\pgfsetdash{}{0pt}%
\pgfpathmoveto{\pgfqpoint{1.435295in}{1.661097in}}%
\pgfpathlineto{\pgfqpoint{1.438296in}{1.656938in}}%
\pgfpathlineto{\pgfqpoint{1.441297in}{1.652672in}}%
\pgfpathlineto{\pgfqpoint{1.444297in}{1.648300in}}%
\pgfpathlineto{\pgfqpoint{1.447296in}{1.643823in}}%
\pgfpathlineto{\pgfqpoint{1.445426in}{1.642402in}}%
\pgfpathlineto{\pgfqpoint{1.443461in}{1.641009in}}%
\pgfpathlineto{\pgfqpoint{1.441402in}{1.639646in}}%
\pgfpathlineto{\pgfqpoint{1.439251in}{1.638315in}}%
\pgfpathlineto{\pgfqpoint{1.436500in}{1.642968in}}%
\pgfpathlineto{\pgfqpoint{1.433749in}{1.647516in}}%
\pgfpathlineto{\pgfqpoint{1.430997in}{1.651957in}}%
\pgfpathlineto{\pgfqpoint{1.428245in}{1.656291in}}%
\pgfpathlineto{\pgfqpoint{1.430129in}{1.657453in}}%
\pgfpathlineto{\pgfqpoint{1.431933in}{1.658642in}}%
\pgfpathlineto{\pgfqpoint{1.433655in}{1.659857in}}%
\pgfpathlineto{\pgfqpoint{1.435295in}{1.661097in}}%
\pgfpathclose%
\pgfusepath{fill}%
\end{pgfscope}%
\begin{pgfscope}%
\pgfpathrectangle{\pgfqpoint{0.329460in}{0.284240in}}{\pgfqpoint{1.989680in}{1.989680in}}%
\pgfusepath{clip}%
\pgfsetbuttcap%
\pgfsetroundjoin%
\definecolor{currentfill}{rgb}{0.133743,0.548535,0.553541}%
\pgfsetfillcolor{currentfill}%
\pgfsetlinewidth{0.000000pt}%
\definecolor{currentstroke}{rgb}{0.000000,0.000000,0.000000}%
\pgfsetstrokecolor{currentstroke}%
\pgfsetdash{}{0pt}%
\pgfpathmoveto{\pgfqpoint{1.497718in}{1.249746in}}%
\pgfpathlineto{\pgfqpoint{1.499443in}{1.241233in}}%
\pgfpathlineto{\pgfqpoint{1.501168in}{1.232710in}}%
\pgfpathlineto{\pgfqpoint{1.502892in}{1.224180in}}%
\pgfpathlineto{\pgfqpoint{1.504615in}{1.215644in}}%
\pgfpathlineto{\pgfqpoint{1.496119in}{1.213254in}}%
\pgfpathlineto{\pgfqpoint{1.487469in}{1.211001in}}%
\pgfpathlineto{\pgfqpoint{1.478674in}{1.208886in}}%
\pgfpathlineto{\pgfqpoint{1.469743in}{1.206913in}}%
\pgfpathlineto{\pgfqpoint{1.468409in}{1.215559in}}%
\pgfpathlineto{\pgfqpoint{1.467075in}{1.224199in}}%
\pgfpathlineto{\pgfqpoint{1.465741in}{1.232831in}}%
\pgfpathlineto{\pgfqpoint{1.464406in}{1.241453in}}%
\pgfpathlineto{\pgfqpoint{1.472936in}{1.243327in}}%
\pgfpathlineto{\pgfqpoint{1.481338in}{1.245336in}}%
\pgfpathlineto{\pgfqpoint{1.489601in}{1.247476in}}%
\pgfpathlineto{\pgfqpoint{1.497718in}{1.249746in}}%
\pgfpathclose%
\pgfusepath{fill}%
\end{pgfscope}%
\begin{pgfscope}%
\pgfpathrectangle{\pgfqpoint{0.329460in}{0.284240in}}{\pgfqpoint{1.989680in}{1.989680in}}%
\pgfusepath{clip}%
\pgfsetbuttcap%
\pgfsetroundjoin%
\definecolor{currentfill}{rgb}{0.974417,0.903590,0.130215}%
\pgfsetfillcolor{currentfill}%
\pgfsetlinewidth{0.000000pt}%
\definecolor{currentstroke}{rgb}{0.000000,0.000000,0.000000}%
\pgfsetstrokecolor{currentstroke}%
\pgfsetdash{}{0pt}%
\pgfpathmoveto{\pgfqpoint{1.337921in}{1.715876in}}%
\pgfpathlineto{\pgfqpoint{1.336260in}{1.713410in}}%
\pgfpathlineto{\pgfqpoint{1.334598in}{1.710824in}}%
\pgfpathlineto{\pgfqpoint{1.332936in}{1.708119in}}%
\pgfpathlineto{\pgfqpoint{1.331273in}{1.705294in}}%
\pgfpathlineto{\pgfqpoint{1.330108in}{1.705596in}}%
\pgfpathlineto{\pgfqpoint{1.328965in}{1.705916in}}%
\pgfpathlineto{\pgfqpoint{1.327844in}{1.706253in}}%
\pgfpathlineto{\pgfqpoint{1.326747in}{1.706606in}}%
\pgfpathlineto{\pgfqpoint{1.328787in}{1.709320in}}%
\pgfpathlineto{\pgfqpoint{1.330827in}{1.711916in}}%
\pgfpathlineto{\pgfqpoint{1.332866in}{1.714392in}}%
\pgfpathlineto{\pgfqpoint{1.334905in}{1.716749in}}%
\pgfpathlineto{\pgfqpoint{1.335636in}{1.716514in}}%
\pgfpathlineto{\pgfqpoint{1.336383in}{1.716290in}}%
\pgfpathlineto{\pgfqpoint{1.337145in}{1.716078in}}%
\pgfpathlineto{\pgfqpoint{1.337921in}{1.715876in}}%
\pgfpathclose%
\pgfusepath{fill}%
\end{pgfscope}%
\begin{pgfscope}%
\pgfpathrectangle{\pgfqpoint{0.329460in}{0.284240in}}{\pgfqpoint{1.989680in}{1.989680in}}%
\pgfusepath{clip}%
\pgfsetbuttcap%
\pgfsetroundjoin%
\definecolor{currentfill}{rgb}{0.855810,0.888601,0.097452}%
\pgfsetfillcolor{currentfill}%
\pgfsetlinewidth{0.000000pt}%
\definecolor{currentstroke}{rgb}{0.000000,0.000000,0.000000}%
\pgfsetstrokecolor{currentstroke}%
\pgfsetdash{}{0pt}%
\pgfpathmoveto{\pgfqpoint{1.423283in}{1.676646in}}%
\pgfpathlineto{\pgfqpoint{1.426287in}{1.672924in}}%
\pgfpathlineto{\pgfqpoint{1.429290in}{1.669091in}}%
\pgfpathlineto{\pgfqpoint{1.432293in}{1.665149in}}%
\pgfpathlineto{\pgfqpoint{1.435295in}{1.661097in}}%
\pgfpathlineto{\pgfqpoint{1.433655in}{1.659857in}}%
\pgfpathlineto{\pgfqpoint{1.431933in}{1.658642in}}%
\pgfpathlineto{\pgfqpoint{1.430129in}{1.657453in}}%
\pgfpathlineto{\pgfqpoint{1.428245in}{1.656291in}}%
\pgfpathlineto{\pgfqpoint{1.425492in}{1.660517in}}%
\pgfpathlineto{\pgfqpoint{1.422739in}{1.664634in}}%
\pgfpathlineto{\pgfqpoint{1.419986in}{1.668641in}}%
\pgfpathlineto{\pgfqpoint{1.417232in}{1.672538in}}%
\pgfpathlineto{\pgfqpoint{1.418849in}{1.673531in}}%
\pgfpathlineto{\pgfqpoint{1.420397in}{1.674547in}}%
\pgfpathlineto{\pgfqpoint{1.421876in}{1.675586in}}%
\pgfpathlineto{\pgfqpoint{1.423283in}{1.676646in}}%
\pgfpathclose%
\pgfusepath{fill}%
\end{pgfscope}%
\begin{pgfscope}%
\pgfpathrectangle{\pgfqpoint{0.329460in}{0.284240in}}{\pgfqpoint{1.989680in}{1.989680in}}%
\pgfusepath{clip}%
\pgfsetbuttcap%
\pgfsetroundjoin%
\definecolor{currentfill}{rgb}{0.955300,0.901065,0.118128}%
\pgfsetfillcolor{currentfill}%
\pgfsetlinewidth{0.000000pt}%
\definecolor{currentstroke}{rgb}{0.000000,0.000000,0.000000}%
\pgfsetstrokecolor{currentstroke}%
\pgfsetdash{}{0pt}%
\pgfpathmoveto{\pgfqpoint{1.322614in}{1.708173in}}%
\pgfpathlineto{\pgfqpoint{1.320229in}{1.705471in}}%
\pgfpathlineto{\pgfqpoint{1.317843in}{1.702651in}}%
\pgfpathlineto{\pgfqpoint{1.315457in}{1.699713in}}%
\pgfpathlineto{\pgfqpoint{1.313071in}{1.696657in}}%
\pgfpathlineto{\pgfqpoint{1.311786in}{1.697231in}}%
\pgfpathlineto{\pgfqpoint{1.310540in}{1.697823in}}%
\pgfpathlineto{\pgfqpoint{1.309336in}{1.698433in}}%
\pgfpathlineto{\pgfqpoint{1.308173in}{1.699061in}}%
\pgfpathlineto{\pgfqpoint{1.310865in}{1.701964in}}%
\pgfpathlineto{\pgfqpoint{1.313557in}{1.704751in}}%
\pgfpathlineto{\pgfqpoint{1.316248in}{1.707420in}}%
\pgfpathlineto{\pgfqpoint{1.318939in}{1.709971in}}%
\pgfpathlineto{\pgfqpoint{1.319811in}{1.709502in}}%
\pgfpathlineto{\pgfqpoint{1.320715in}{1.709045in}}%
\pgfpathlineto{\pgfqpoint{1.321650in}{1.708602in}}%
\pgfpathlineto{\pgfqpoint{1.322614in}{1.708173in}}%
\pgfpathclose%
\pgfusepath{fill}%
\end{pgfscope}%
\begin{pgfscope}%
\pgfpathrectangle{\pgfqpoint{0.329460in}{0.284240in}}{\pgfqpoint{1.989680in}{1.989680in}}%
\pgfusepath{clip}%
\pgfsetbuttcap%
\pgfsetroundjoin%
\definecolor{currentfill}{rgb}{0.281477,0.755203,0.432552}%
\pgfsetfillcolor{currentfill}%
\pgfsetlinewidth{0.000000pt}%
\definecolor{currentstroke}{rgb}{0.000000,0.000000,0.000000}%
\pgfsetstrokecolor{currentstroke}%
\pgfsetdash{}{0pt}%
\pgfpathmoveto{\pgfqpoint{1.498302in}{1.460642in}}%
\pgfpathlineto{\pgfqpoint{1.500736in}{1.453332in}}%
\pgfpathlineto{\pgfqpoint{1.503169in}{1.445960in}}%
\pgfpathlineto{\pgfqpoint{1.505600in}{1.438528in}}%
\pgfpathlineto{\pgfqpoint{1.508031in}{1.431039in}}%
\pgfpathlineto{\pgfqpoint{1.502924in}{1.428651in}}%
\pgfpathlineto{\pgfqpoint{1.497660in}{1.426343in}}%
\pgfpathlineto{\pgfqpoint{1.492245in}{1.424117in}}%
\pgfpathlineto{\pgfqpoint{1.486682in}{1.421975in}}%
\pgfpathlineto{\pgfqpoint{1.484580in}{1.429617in}}%
\pgfpathlineto{\pgfqpoint{1.482476in}{1.437200in}}%
\pgfpathlineto{\pgfqpoint{1.480372in}{1.444722in}}%
\pgfpathlineto{\pgfqpoint{1.478266in}{1.452183in}}%
\pgfpathlineto{\pgfqpoint{1.483486in}{1.454182in}}%
\pgfpathlineto{\pgfqpoint{1.488569in}{1.456259in}}%
\pgfpathlineto{\pgfqpoint{1.493509in}{1.458413in}}%
\pgfpathlineto{\pgfqpoint{1.498302in}{1.460642in}}%
\pgfpathclose%
\pgfusepath{fill}%
\end{pgfscope}%
\begin{pgfscope}%
\pgfpathrectangle{\pgfqpoint{0.329460in}{0.284240in}}{\pgfqpoint{1.989680in}{1.989680in}}%
\pgfusepath{clip}%
\pgfsetbuttcap%
\pgfsetroundjoin%
\definecolor{currentfill}{rgb}{0.212395,0.359683,0.551710}%
\pgfsetfillcolor{currentfill}%
\pgfsetlinewidth{0.000000pt}%
\definecolor{currentstroke}{rgb}{0.000000,0.000000,0.000000}%
\pgfsetstrokecolor{currentstroke}%
\pgfsetdash{}{0pt}%
\pgfpathmoveto{\pgfqpoint{1.447546in}{1.062292in}}%
\pgfpathlineto{\pgfqpoint{1.448464in}{1.053886in}}%
\pgfpathlineto{\pgfqpoint{1.449382in}{1.045523in}}%
\pgfpathlineto{\pgfqpoint{1.450300in}{1.037208in}}%
\pgfpathlineto{\pgfqpoint{1.451218in}{1.028942in}}%
\pgfpathlineto{\pgfqpoint{1.439628in}{1.027388in}}%
\pgfpathlineto{\pgfqpoint{1.427941in}{1.026025in}}%
\pgfpathlineto{\pgfqpoint{1.416171in}{1.024856in}}%
\pgfpathlineto{\pgfqpoint{1.404331in}{1.023881in}}%
\pgfpathlineto{\pgfqpoint{1.403843in}{1.032199in}}%
\pgfpathlineto{\pgfqpoint{1.403355in}{1.040567in}}%
\pgfpathlineto{\pgfqpoint{1.402867in}{1.048983in}}%
\pgfpathlineto{\pgfqpoint{1.402379in}{1.057442in}}%
\pgfpathlineto{\pgfqpoint{1.413784in}{1.058376in}}%
\pgfpathlineto{\pgfqpoint{1.425122in}{1.059497in}}%
\pgfpathlineto{\pgfqpoint{1.436380in}{1.060803in}}%
\pgfpathlineto{\pgfqpoint{1.447546in}{1.062292in}}%
\pgfpathclose%
\pgfusepath{fill}%
\end{pgfscope}%
\begin{pgfscope}%
\pgfpathrectangle{\pgfqpoint{0.329460in}{0.284240in}}{\pgfqpoint{1.989680in}{1.989680in}}%
\pgfusepath{clip}%
\pgfsetbuttcap%
\pgfsetroundjoin%
\definecolor{currentfill}{rgb}{0.134692,0.658636,0.517649}%
\pgfsetfillcolor{currentfill}%
\pgfsetlinewidth{0.000000pt}%
\definecolor{currentstroke}{rgb}{0.000000,0.000000,0.000000}%
\pgfsetstrokecolor{currentstroke}%
\pgfsetdash{}{0pt}%
\pgfpathmoveto{\pgfqpoint{1.503472in}{1.358979in}}%
\pgfpathlineto{\pgfqpoint{1.505567in}{1.350903in}}%
\pgfpathlineto{\pgfqpoint{1.507661in}{1.342789in}}%
\pgfpathlineto{\pgfqpoint{1.509754in}{1.334639in}}%
\pgfpathlineto{\pgfqpoint{1.511846in}{1.326456in}}%
\pgfpathlineto{\pgfqpoint{1.505093in}{1.323980in}}%
\pgfpathlineto{\pgfqpoint{1.498179in}{1.321612in}}%
\pgfpathlineto{\pgfqpoint{1.491111in}{1.319352in}}%
\pgfpathlineto{\pgfqpoint{1.483895in}{1.317204in}}%
\pgfpathlineto{\pgfqpoint{1.482164in}{1.325520in}}%
\pgfpathlineto{\pgfqpoint{1.480433in}{1.333802in}}%
\pgfpathlineto{\pgfqpoint{1.478701in}{1.342048in}}%
\pgfpathlineto{\pgfqpoint{1.476969in}{1.350256in}}%
\pgfpathlineto{\pgfqpoint{1.483810in}{1.352281in}}%
\pgfpathlineto{\pgfqpoint{1.490512in}{1.354411in}}%
\pgfpathlineto{\pgfqpoint{1.497069in}{1.356645in}}%
\pgfpathlineto{\pgfqpoint{1.503472in}{1.358979in}}%
\pgfpathclose%
\pgfusepath{fill}%
\end{pgfscope}%
\begin{pgfscope}%
\pgfpathrectangle{\pgfqpoint{0.329460in}{0.284240in}}{\pgfqpoint{1.989680in}{1.989680in}}%
\pgfusepath{clip}%
\pgfsetbuttcap%
\pgfsetroundjoin%
\definecolor{currentfill}{rgb}{0.762373,0.876424,0.137064}%
\pgfsetfillcolor{currentfill}%
\pgfsetlinewidth{0.000000pt}%
\definecolor{currentstroke}{rgb}{0.000000,0.000000,0.000000}%
\pgfsetstrokecolor{currentstroke}%
\pgfsetdash{}{0pt}%
\pgfpathmoveto{\pgfqpoint{1.265110in}{1.637158in}}%
\pgfpathlineto{\pgfqpoint{1.262422in}{1.632363in}}%
\pgfpathlineto{\pgfqpoint{1.259734in}{1.627465in}}%
\pgfpathlineto{\pgfqpoint{1.257046in}{1.622465in}}%
\pgfpathlineto{\pgfqpoint{1.254359in}{1.617364in}}%
\pgfpathlineto{\pgfqpoint{1.251854in}{1.618835in}}%
\pgfpathlineto{\pgfqpoint{1.249449in}{1.620342in}}%
\pgfpathlineto{\pgfqpoint{1.247147in}{1.621884in}}%
\pgfpathlineto{\pgfqpoint{1.244951in}{1.623459in}}%
\pgfpathlineto{\pgfqpoint{1.247896in}{1.628386in}}%
\pgfpathlineto{\pgfqpoint{1.250842in}{1.633213in}}%
\pgfpathlineto{\pgfqpoint{1.253789in}{1.637937in}}%
\pgfpathlineto{\pgfqpoint{1.256736in}{1.642559in}}%
\pgfpathlineto{\pgfqpoint{1.258692in}{1.641163in}}%
\pgfpathlineto{\pgfqpoint{1.260740in}{1.639796in}}%
\pgfpathlineto{\pgfqpoint{1.262881in}{1.638461in}}%
\pgfpathlineto{\pgfqpoint{1.265110in}{1.637158in}}%
\pgfpathclose%
\pgfusepath{fill}%
\end{pgfscope}%
\begin{pgfscope}%
\pgfpathrectangle{\pgfqpoint{0.329460in}{0.284240in}}{\pgfqpoint{1.989680in}{1.989680in}}%
\pgfusepath{clip}%
\pgfsetbuttcap%
\pgfsetroundjoin%
\definecolor{currentfill}{rgb}{0.248629,0.278775,0.534556}%
\pgfsetfillcolor{currentfill}%
\pgfsetlinewidth{0.000000pt}%
\definecolor{currentstroke}{rgb}{0.000000,0.000000,0.000000}%
\pgfsetstrokecolor{currentstroke}%
\pgfsetdash{}{0pt}%
\pgfpathmoveto{\pgfqpoint{1.406285in}{0.991169in}}%
\pgfpathlineto{\pgfqpoint{1.406773in}{0.983147in}}%
\pgfpathlineto{\pgfqpoint{1.407262in}{0.975193in}}%
\pgfpathlineto{\pgfqpoint{1.407751in}{0.967310in}}%
\pgfpathlineto{\pgfqpoint{1.408241in}{0.959503in}}%
\pgfpathlineto{\pgfqpoint{1.395467in}{0.958658in}}%
\pgfpathlineto{\pgfqpoint{1.382646in}{0.958027in}}%
\pgfpathlineto{\pgfqpoint{1.369790in}{0.957609in}}%
\pgfpathlineto{\pgfqpoint{1.356913in}{0.957406in}}%
\pgfpathlineto{\pgfqpoint{1.356864in}{0.965235in}}%
\pgfpathlineto{\pgfqpoint{1.356815in}{0.973137in}}%
\pgfpathlineto{\pgfqpoint{1.356766in}{0.981112in}}%
\pgfpathlineto{\pgfqpoint{1.356717in}{0.989154in}}%
\pgfpathlineto{\pgfqpoint{1.369152in}{0.989349in}}%
\pgfpathlineto{\pgfqpoint{1.381567in}{0.989750in}}%
\pgfpathlineto{\pgfqpoint{1.393949in}{0.990357in}}%
\pgfpathlineto{\pgfqpoint{1.406285in}{0.991169in}}%
\pgfpathclose%
\pgfusepath{fill}%
\end{pgfscope}%
\begin{pgfscope}%
\pgfpathrectangle{\pgfqpoint{0.329460in}{0.284240in}}{\pgfqpoint{1.989680in}{1.989680in}}%
\pgfusepath{clip}%
\pgfsetbuttcap%
\pgfsetroundjoin%
\definecolor{currentfill}{rgb}{0.896320,0.893616,0.096335}%
\pgfsetfillcolor{currentfill}%
\pgfsetlinewidth{0.000000pt}%
\definecolor{currentstroke}{rgb}{0.000000,0.000000,0.000000}%
\pgfsetstrokecolor{currentstroke}%
\pgfsetdash{}{0pt}%
\pgfpathmoveto{\pgfqpoint{1.297404in}{1.686283in}}%
\pgfpathlineto{\pgfqpoint{1.294712in}{1.682801in}}%
\pgfpathlineto{\pgfqpoint{1.292020in}{1.679205in}}%
\pgfpathlineto{\pgfqpoint{1.289328in}{1.675496in}}%
\pgfpathlineto{\pgfqpoint{1.286636in}{1.671675in}}%
\pgfpathlineto{\pgfqpoint{1.284960in}{1.672647in}}%
\pgfpathlineto{\pgfqpoint{1.283351in}{1.673643in}}%
\pgfpathlineto{\pgfqpoint{1.281810in}{1.674661in}}%
\pgfpathlineto{\pgfqpoint{1.280340in}{1.675703in}}%
\pgfpathlineto{\pgfqpoint{1.283292in}{1.679353in}}%
\pgfpathlineto{\pgfqpoint{1.286245in}{1.682892in}}%
\pgfpathlineto{\pgfqpoint{1.289198in}{1.686318in}}%
\pgfpathlineto{\pgfqpoint{1.292152in}{1.689630in}}%
\pgfpathlineto{\pgfqpoint{1.293379in}{1.688765in}}%
\pgfpathlineto{\pgfqpoint{1.294664in}{1.687918in}}%
\pgfpathlineto{\pgfqpoint{1.296007in}{1.687090in}}%
\pgfpathlineto{\pgfqpoint{1.297404in}{1.686283in}}%
\pgfpathclose%
\pgfusepath{fill}%
\end{pgfscope}%
\begin{pgfscope}%
\pgfpathrectangle{\pgfqpoint{0.329460in}{0.284240in}}{\pgfqpoint{1.989680in}{1.989680in}}%
\pgfusepath{clip}%
\pgfsetbuttcap%
\pgfsetroundjoin%
\definecolor{currentfill}{rgb}{0.179019,0.433756,0.557430}%
\pgfsetfillcolor{currentfill}%
\pgfsetlinewidth{0.000000pt}%
\definecolor{currentstroke}{rgb}{0.000000,0.000000,0.000000}%
\pgfsetstrokecolor{currentstroke}%
\pgfsetdash{}{0pt}%
\pgfpathmoveto{\pgfqpoint{1.271338in}{1.129570in}}%
\pgfpathlineto{\pgfqpoint{1.270515in}{1.120900in}}%
\pgfpathlineto{\pgfqpoint{1.269691in}{1.112251in}}%
\pgfpathlineto{\pgfqpoint{1.268867in}{1.103625in}}%
\pgfpathlineto{\pgfqpoint{1.268044in}{1.095027in}}%
\pgfpathlineto{\pgfqpoint{1.257313in}{1.096472in}}%
\pgfpathlineto{\pgfqpoint{1.246684in}{1.098091in}}%
\pgfpathlineto{\pgfqpoint{1.236168in}{1.099881in}}%
\pgfpathlineto{\pgfqpoint{1.225775in}{1.101842in}}%
\pgfpathlineto{\pgfqpoint{1.227016in}{1.110364in}}%
\pgfpathlineto{\pgfqpoint{1.228257in}{1.118913in}}%
\pgfpathlineto{\pgfqpoint{1.229498in}{1.127486in}}%
\pgfpathlineto{\pgfqpoint{1.230739in}{1.136080in}}%
\pgfpathlineto{\pgfqpoint{1.240722in}{1.134207in}}%
\pgfpathlineto{\pgfqpoint{1.250823in}{1.132496in}}%
\pgfpathlineto{\pgfqpoint{1.261032in}{1.130950in}}%
\pgfpathlineto{\pgfqpoint{1.271338in}{1.129570in}}%
\pgfpathclose%
\pgfusepath{fill}%
\end{pgfscope}%
\begin{pgfscope}%
\pgfpathrectangle{\pgfqpoint{0.329460in}{0.284240in}}{\pgfqpoint{1.989680in}{1.989680in}}%
\pgfusepath{clip}%
\pgfsetbuttcap%
\pgfsetroundjoin%
\definecolor{currentfill}{rgb}{0.282327,0.094955,0.417331}%
\pgfsetfillcolor{currentfill}%
\pgfsetlinewidth{0.000000pt}%
\definecolor{currentstroke}{rgb}{0.000000,0.000000,0.000000}%
\pgfsetstrokecolor{currentstroke}%
\pgfsetdash{}{0pt}%
\pgfpathmoveto{\pgfqpoint{1.416108in}{0.847440in}}%
\pgfpathlineto{\pgfqpoint{1.416603in}{0.841414in}}%
\pgfpathlineto{\pgfqpoint{1.417099in}{0.835524in}}%
\pgfpathlineto{\pgfqpoint{1.417595in}{0.829775in}}%
\pgfpathlineto{\pgfqpoint{1.418092in}{0.824169in}}%
\pgfpathlineto{\pgfqpoint{1.403114in}{0.823158in}}%
\pgfpathlineto{\pgfqpoint{1.388079in}{0.822402in}}%
\pgfpathlineto{\pgfqpoint{1.373003in}{0.821903in}}%
\pgfpathlineto{\pgfqpoint{1.357902in}{0.821660in}}%
\pgfpathlineto{\pgfqpoint{1.357852in}{0.827287in}}%
\pgfpathlineto{\pgfqpoint{1.357803in}{0.833057in}}%
\pgfpathlineto{\pgfqpoint{1.357753in}{0.838967in}}%
\pgfpathlineto{\pgfqpoint{1.357703in}{0.845014in}}%
\pgfpathlineto{\pgfqpoint{1.372356in}{0.845249in}}%
\pgfpathlineto{\pgfqpoint{1.386985in}{0.845732in}}%
\pgfpathlineto{\pgfqpoint{1.401574in}{0.846462in}}%
\pgfpathlineto{\pgfqpoint{1.416108in}{0.847440in}}%
\pgfpathclose%
\pgfusepath{fill}%
\end{pgfscope}%
\begin{pgfscope}%
\pgfpathrectangle{\pgfqpoint{0.329460in}{0.284240in}}{\pgfqpoint{1.989680in}{1.989680in}}%
\pgfusepath{clip}%
\pgfsetbuttcap%
\pgfsetroundjoin%
\definecolor{currentfill}{rgb}{0.268510,0.009605,0.335427}%
\pgfsetfillcolor{currentfill}%
\pgfsetlinewidth{0.000000pt}%
\definecolor{currentstroke}{rgb}{0.000000,0.000000,0.000000}%
\pgfsetstrokecolor{currentstroke}%
\pgfsetdash{}{0pt}%
\pgfpathmoveto{\pgfqpoint{1.550261in}{0.788256in}}%
\pgfpathlineto{\pgfqpoint{1.551642in}{0.785012in}}%
\pgfpathlineto{\pgfqpoint{1.553026in}{0.781968in}}%
\pgfpathlineto{\pgfqpoint{1.554413in}{0.779126in}}%
\pgfpathlineto{\pgfqpoint{1.555802in}{0.776493in}}%
\pgfpathlineto{\pgfqpoint{1.540210in}{0.773120in}}%
\pgfpathlineto{\pgfqpoint{1.524407in}{0.770014in}}%
\pgfpathlineto{\pgfqpoint{1.508410in}{0.767180in}}%
\pgfpathlineto{\pgfqpoint{1.492237in}{0.764620in}}%
\pgfpathlineto{\pgfqpoint{1.491278in}{0.767339in}}%
\pgfpathlineto{\pgfqpoint{1.490320in}{0.770266in}}%
\pgfpathlineto{\pgfqpoint{1.489365in}{0.773396in}}%
\pgfpathlineto{\pgfqpoint{1.488411in}{0.776726in}}%
\pgfpathlineto{\pgfqpoint{1.504147in}{0.779211in}}%
\pgfpathlineto{\pgfqpoint{1.519712in}{0.781964in}}%
\pgfpathlineto{\pgfqpoint{1.535089in}{0.784980in}}%
\pgfpathlineto{\pgfqpoint{1.550261in}{0.788256in}}%
\pgfpathclose%
\pgfusepath{fill}%
\end{pgfscope}%
\begin{pgfscope}%
\pgfpathrectangle{\pgfqpoint{0.329460in}{0.284240in}}{\pgfqpoint{1.989680in}{1.989680in}}%
\pgfusepath{clip}%
\pgfsetbuttcap%
\pgfsetroundjoin%
\definecolor{currentfill}{rgb}{0.974417,0.903590,0.130215}%
\pgfsetfillcolor{currentfill}%
\pgfsetlinewidth{0.000000pt}%
\definecolor{currentstroke}{rgb}{0.000000,0.000000,0.000000}%
\pgfsetstrokecolor{currentstroke}%
\pgfsetdash{}{0pt}%
\pgfpathmoveto{\pgfqpoint{1.365145in}{1.716055in}}%
\pgfpathlineto{\pgfqpoint{1.366892in}{1.713611in}}%
\pgfpathlineto{\pgfqpoint{1.368640in}{1.711047in}}%
\pgfpathlineto{\pgfqpoint{1.370389in}{1.708364in}}%
\pgfpathlineto{\pgfqpoint{1.372138in}{1.705562in}}%
\pgfpathlineto{\pgfqpoint{1.370972in}{1.705261in}}%
\pgfpathlineto{\pgfqpoint{1.369785in}{1.704977in}}%
\pgfpathlineto{\pgfqpoint{1.368580in}{1.704712in}}%
\pgfpathlineto{\pgfqpoint{1.367357in}{1.704464in}}%
\pgfpathlineto{\pgfqpoint{1.366007in}{1.707358in}}%
\pgfpathlineto{\pgfqpoint{1.364657in}{1.710134in}}%
\pgfpathlineto{\pgfqpoint{1.363308in}{1.712789in}}%
\pgfpathlineto{\pgfqpoint{1.361959in}{1.715325in}}%
\pgfpathlineto{\pgfqpoint{1.362773in}{1.715489in}}%
\pgfpathlineto{\pgfqpoint{1.363576in}{1.715666in}}%
\pgfpathlineto{\pgfqpoint{1.364367in}{1.715855in}}%
\pgfpathlineto{\pgfqpoint{1.365145in}{1.716055in}}%
\pgfpathclose%
\pgfusepath{fill}%
\end{pgfscope}%
\begin{pgfscope}%
\pgfpathrectangle{\pgfqpoint{0.329460in}{0.284240in}}{\pgfqpoint{1.989680in}{1.989680in}}%
\pgfusepath{clip}%
\pgfsetbuttcap%
\pgfsetroundjoin%
\definecolor{currentfill}{rgb}{0.172719,0.448791,0.557885}%
\pgfsetfillcolor{currentfill}%
\pgfsetlinewidth{0.000000pt}%
\definecolor{currentstroke}{rgb}{0.000000,0.000000,0.000000}%
\pgfsetstrokecolor{currentstroke}%
\pgfsetdash{}{0pt}%
\pgfpathmoveto{\pgfqpoint{1.977885in}{1.128768in}}%
\pgfpathlineto{\pgfqpoint{1.981620in}{1.141655in}}%
\pgfpathlineto{\pgfqpoint{1.985377in}{1.155009in}}%
\pgfpathlineto{\pgfqpoint{1.989155in}{1.168837in}}%
\pgfpathlineto{\pgfqpoint{1.980855in}{1.158600in}}%
\pgfpathlineto{\pgfqpoint{1.971884in}{1.148488in}}%
\pgfpathlineto{\pgfqpoint{1.962248in}{1.138510in}}%
\pgfpathlineto{\pgfqpoint{1.951956in}{1.128678in}}%
\pgfpathlineto{\pgfqpoint{1.948375in}{1.115002in}}%
\pgfpathlineto{\pgfqpoint{1.944815in}{1.101804in}}%
\pgfpathlineto{\pgfqpoint{1.941276in}{1.089073in}}%
\pgfpathlineto{\pgfqpoint{1.951402in}{1.098791in}}%
\pgfpathlineto{\pgfqpoint{1.960884in}{1.108653in}}%
\pgfpathlineto{\pgfqpoint{1.969714in}{1.118649in}}%
\pgfpathlineto{\pgfqpoint{1.977885in}{1.128768in}}%
\pgfpathclose%
\pgfusepath{fill}%
\end{pgfscope}%
\begin{pgfscope}%
\pgfpathrectangle{\pgfqpoint{0.329460in}{0.284240in}}{\pgfqpoint{1.989680in}{1.989680in}}%
\pgfusepath{clip}%
\pgfsetbuttcap%
\pgfsetroundjoin%
\definecolor{currentfill}{rgb}{0.248629,0.278775,0.534556}%
\pgfsetfillcolor{currentfill}%
\pgfsetlinewidth{0.000000pt}%
\definecolor{currentstroke}{rgb}{0.000000,0.000000,0.000000}%
\pgfsetstrokecolor{currentstroke}%
\pgfsetdash{}{0pt}%
\pgfpathmoveto{\pgfqpoint{1.356717in}{0.989154in}}%
\pgfpathlineto{\pgfqpoint{1.356766in}{0.981112in}}%
\pgfpathlineto{\pgfqpoint{1.356815in}{0.973137in}}%
\pgfpathlineto{\pgfqpoint{1.356864in}{0.965235in}}%
\pgfpathlineto{\pgfqpoint{1.356913in}{0.957406in}}%
\pgfpathlineto{\pgfqpoint{1.344031in}{0.957418in}}%
\pgfpathlineto{\pgfqpoint{1.331156in}{0.957645in}}%
\pgfpathlineto{\pgfqpoint{1.318303in}{0.958086in}}%
\pgfpathlineto{\pgfqpoint{1.305486in}{0.958741in}}%
\pgfpathlineto{\pgfqpoint{1.305878in}{0.966556in}}%
\pgfpathlineto{\pgfqpoint{1.306270in}{0.974446in}}%
\pgfpathlineto{\pgfqpoint{1.306661in}{0.982407in}}%
\pgfpathlineto{\pgfqpoint{1.307053in}{0.990437in}}%
\pgfpathlineto{\pgfqpoint{1.319430in}{0.989808in}}%
\pgfpathlineto{\pgfqpoint{1.331843in}{0.989384in}}%
\pgfpathlineto{\pgfqpoint{1.344276in}{0.989166in}}%
\pgfpathlineto{\pgfqpoint{1.356717in}{0.989154in}}%
\pgfpathclose%
\pgfusepath{fill}%
\end{pgfscope}%
\begin{pgfscope}%
\pgfpathrectangle{\pgfqpoint{0.329460in}{0.284240in}}{\pgfqpoint{1.989680in}{1.989680in}}%
\pgfusepath{clip}%
\pgfsetbuttcap%
\pgfsetroundjoin%
\definecolor{currentfill}{rgb}{0.974417,0.903590,0.130215}%
\pgfsetfillcolor{currentfill}%
\pgfsetlinewidth{0.000000pt}%
\definecolor{currentstroke}{rgb}{0.000000,0.000000,0.000000}%
\pgfsetstrokecolor{currentstroke}%
\pgfsetdash{}{0pt}%
\pgfpathmoveto{\pgfqpoint{1.341150in}{1.715188in}}%
\pgfpathlineto{\pgfqpoint{1.339893in}{1.712636in}}%
\pgfpathlineto{\pgfqpoint{1.338635in}{1.709963in}}%
\pgfpathlineto{\pgfqpoint{1.337377in}{1.707170in}}%
\pgfpathlineto{\pgfqpoint{1.336119in}{1.704258in}}%
\pgfpathlineto{\pgfqpoint{1.334881in}{1.704490in}}%
\pgfpathlineto{\pgfqpoint{1.333660in}{1.704740in}}%
\pgfpathlineto{\pgfqpoint{1.332457in}{1.705008in}}%
\pgfpathlineto{\pgfqpoint{1.331273in}{1.705294in}}%
\pgfpathlineto{\pgfqpoint{1.332936in}{1.708119in}}%
\pgfpathlineto{\pgfqpoint{1.334598in}{1.710824in}}%
\pgfpathlineto{\pgfqpoint{1.336260in}{1.713410in}}%
\pgfpathlineto{\pgfqpoint{1.337921in}{1.715876in}}%
\pgfpathlineto{\pgfqpoint{1.338710in}{1.715687in}}%
\pgfpathlineto{\pgfqpoint{1.339512in}{1.715509in}}%
\pgfpathlineto{\pgfqpoint{1.340325in}{1.715342in}}%
\pgfpathlineto{\pgfqpoint{1.341150in}{1.715188in}}%
\pgfpathclose%
\pgfusepath{fill}%
\end{pgfscope}%
\begin{pgfscope}%
\pgfpathrectangle{\pgfqpoint{0.329460in}{0.284240in}}{\pgfqpoint{1.989680in}{1.989680in}}%
\pgfusepath{clip}%
\pgfsetbuttcap%
\pgfsetroundjoin%
\definecolor{currentfill}{rgb}{0.282327,0.094955,0.417331}%
\pgfsetfillcolor{currentfill}%
\pgfsetlinewidth{0.000000pt}%
\definecolor{currentstroke}{rgb}{0.000000,0.000000,0.000000}%
\pgfsetstrokecolor{currentstroke}%
\pgfsetdash{}{0pt}%
\pgfpathmoveto{\pgfqpoint{1.357703in}{0.845014in}}%
\pgfpathlineto{\pgfqpoint{1.357753in}{0.838967in}}%
\pgfpathlineto{\pgfqpoint{1.357803in}{0.833057in}}%
\pgfpathlineto{\pgfqpoint{1.357852in}{0.827287in}}%
\pgfpathlineto{\pgfqpoint{1.357902in}{0.821660in}}%
\pgfpathlineto{\pgfqpoint{1.342794in}{0.821674in}}%
\pgfpathlineto{\pgfqpoint{1.327696in}{0.821946in}}%
\pgfpathlineto{\pgfqpoint{1.312623in}{0.822474in}}%
\pgfpathlineto{\pgfqpoint{1.297594in}{0.823258in}}%
\pgfpathlineto{\pgfqpoint{1.297992in}{0.828871in}}%
\pgfpathlineto{\pgfqpoint{1.298389in}{0.834628in}}%
\pgfpathlineto{\pgfqpoint{1.298786in}{0.840525in}}%
\pgfpathlineto{\pgfqpoint{1.299183in}{0.846559in}}%
\pgfpathlineto{\pgfqpoint{1.313767in}{0.845801in}}%
\pgfpathlineto{\pgfqpoint{1.328393in}{0.845290in}}%
\pgfpathlineto{\pgfqpoint{1.343043in}{0.845028in}}%
\pgfpathlineto{\pgfqpoint{1.357703in}{0.845014in}}%
\pgfpathclose%
\pgfusepath{fill}%
\end{pgfscope}%
\begin{pgfscope}%
\pgfpathrectangle{\pgfqpoint{0.329460in}{0.284240in}}{\pgfqpoint{1.989680in}{1.989680in}}%
\pgfusepath{clip}%
\pgfsetbuttcap%
\pgfsetroundjoin%
\definecolor{currentfill}{rgb}{0.814576,0.883393,0.110347}%
\pgfsetfillcolor{currentfill}%
\pgfsetlinewidth{0.000000pt}%
\definecolor{currentstroke}{rgb}{0.000000,0.000000,0.000000}%
\pgfsetstrokecolor{currentstroke}%
\pgfsetdash{}{0pt}%
\pgfpathmoveto{\pgfqpoint{1.275870in}{1.655282in}}%
\pgfpathlineto{\pgfqpoint{1.273180in}{1.650911in}}%
\pgfpathlineto{\pgfqpoint{1.270489in}{1.646433in}}%
\pgfpathlineto{\pgfqpoint{1.267800in}{1.641848in}}%
\pgfpathlineto{\pgfqpoint{1.265110in}{1.637158in}}%
\pgfpathlineto{\pgfqpoint{1.262881in}{1.638461in}}%
\pgfpathlineto{\pgfqpoint{1.260740in}{1.639796in}}%
\pgfpathlineto{\pgfqpoint{1.258692in}{1.641163in}}%
\pgfpathlineto{\pgfqpoint{1.256736in}{1.642559in}}%
\pgfpathlineto{\pgfqpoint{1.259685in}{1.647076in}}%
\pgfpathlineto{\pgfqpoint{1.262634in}{1.651489in}}%
\pgfpathlineto{\pgfqpoint{1.265583in}{1.655795in}}%
\pgfpathlineto{\pgfqpoint{1.268533in}{1.659994in}}%
\pgfpathlineto{\pgfqpoint{1.270247in}{1.658776in}}%
\pgfpathlineto{\pgfqpoint{1.272042in}{1.657584in}}%
\pgfpathlineto{\pgfqpoint{1.273917in}{1.656419in}}%
\pgfpathlineto{\pgfqpoint{1.275870in}{1.655282in}}%
\pgfpathclose%
\pgfusepath{fill}%
\end{pgfscope}%
\begin{pgfscope}%
\pgfpathrectangle{\pgfqpoint{0.329460in}{0.284240in}}{\pgfqpoint{1.989680in}{1.989680in}}%
\pgfusepath{clip}%
\pgfsetbuttcap%
\pgfsetroundjoin%
\definecolor{currentfill}{rgb}{0.274952,0.037752,0.364543}%
\pgfsetfillcolor{currentfill}%
\pgfsetlinewidth{0.000000pt}%
\definecolor{currentstroke}{rgb}{0.000000,0.000000,0.000000}%
\pgfsetstrokecolor{currentstroke}%
\pgfsetdash{}{0pt}%
\pgfpathmoveto{\pgfqpoint{1.480843in}{0.810042in}}%
\pgfpathlineto{\pgfqpoint{1.481784in}{0.805266in}}%
\pgfpathlineto{\pgfqpoint{1.482726in}{0.800657in}}%
\pgfpathlineto{\pgfqpoint{1.483669in}{0.796218in}}%
\pgfpathlineto{\pgfqpoint{1.484614in}{0.791954in}}%
\pgfpathlineto{\pgfqpoint{1.469164in}{0.789805in}}%
\pgfpathlineto{\pgfqpoint{1.453582in}{0.787921in}}%
\pgfpathlineto{\pgfqpoint{1.437885in}{0.786303in}}%
\pgfpathlineto{\pgfqpoint{1.422092in}{0.784955in}}%
\pgfpathlineto{\pgfqpoint{1.421589in}{0.789273in}}%
\pgfpathlineto{\pgfqpoint{1.421087in}{0.793765in}}%
\pgfpathlineto{\pgfqpoint{1.420586in}{0.798429in}}%
\pgfpathlineto{\pgfqpoint{1.420086in}{0.803259in}}%
\pgfpathlineto{\pgfqpoint{1.435433in}{0.804566in}}%
\pgfpathlineto{\pgfqpoint{1.450686in}{0.806133in}}%
\pgfpathlineto{\pgfqpoint{1.465828in}{0.807959in}}%
\pgfpathlineto{\pgfqpoint{1.480843in}{0.810042in}}%
\pgfpathclose%
\pgfusepath{fill}%
\end{pgfscope}%
\begin{pgfscope}%
\pgfpathrectangle{\pgfqpoint{0.329460in}{0.284240in}}{\pgfqpoint{1.989680in}{1.989680in}}%
\pgfusepath{clip}%
\pgfsetbuttcap%
\pgfsetroundjoin%
\definecolor{currentfill}{rgb}{0.855810,0.888601,0.097452}%
\pgfsetfillcolor{currentfill}%
\pgfsetlinewidth{0.000000pt}%
\definecolor{currentstroke}{rgb}{0.000000,0.000000,0.000000}%
\pgfsetstrokecolor{currentstroke}%
\pgfsetdash{}{0pt}%
\pgfpathmoveto{\pgfqpoint{1.286636in}{1.671675in}}%
\pgfpathlineto{\pgfqpoint{1.283944in}{1.667742in}}%
\pgfpathlineto{\pgfqpoint{1.281252in}{1.663699in}}%
\pgfpathlineto{\pgfqpoint{1.278561in}{1.659545in}}%
\pgfpathlineto{\pgfqpoint{1.275870in}{1.655282in}}%
\pgfpathlineto{\pgfqpoint{1.273917in}{1.656419in}}%
\pgfpathlineto{\pgfqpoint{1.272042in}{1.657584in}}%
\pgfpathlineto{\pgfqpoint{1.270247in}{1.658776in}}%
\pgfpathlineto{\pgfqpoint{1.268533in}{1.659994in}}%
\pgfpathlineto{\pgfqpoint{1.271484in}{1.664085in}}%
\pgfpathlineto{\pgfqpoint{1.274436in}{1.668068in}}%
\pgfpathlineto{\pgfqpoint{1.277387in}{1.671940in}}%
\pgfpathlineto{\pgfqpoint{1.280340in}{1.675703in}}%
\pgfpathlineto{\pgfqpoint{1.281810in}{1.674661in}}%
\pgfpathlineto{\pgfqpoint{1.283351in}{1.673643in}}%
\pgfpathlineto{\pgfqpoint{1.284960in}{1.672647in}}%
\pgfpathlineto{\pgfqpoint{1.286636in}{1.671675in}}%
\pgfpathclose%
\pgfusepath{fill}%
\end{pgfscope}%
\begin{pgfscope}%
\pgfpathrectangle{\pgfqpoint{0.329460in}{0.284240in}}{\pgfqpoint{1.989680in}{1.989680in}}%
\pgfusepath{clip}%
\pgfsetbuttcap%
\pgfsetroundjoin%
\definecolor{currentfill}{rgb}{0.487026,0.823929,0.312321}%
\pgfsetfillcolor{currentfill}%
\pgfsetlinewidth{0.000000pt}%
\definecolor{currentstroke}{rgb}{0.000000,0.000000,0.000000}%
\pgfsetstrokecolor{currentstroke}%
\pgfsetdash{}{0pt}%
\pgfpathmoveto{\pgfqpoint{1.236774in}{1.541306in}}%
\pgfpathlineto{\pgfqpoint{1.234398in}{1.534843in}}%
\pgfpathlineto{\pgfqpoint{1.232024in}{1.528298in}}%
\pgfpathlineto{\pgfqpoint{1.229650in}{1.521672in}}%
\pgfpathlineto{\pgfqpoint{1.227277in}{1.514966in}}%
\pgfpathlineto{\pgfqpoint{1.223127in}{1.516888in}}%
\pgfpathlineto{\pgfqpoint{1.219110in}{1.518871in}}%
\pgfpathlineto{\pgfqpoint{1.215228in}{1.520915in}}%
\pgfpathlineto{\pgfqpoint{1.211485in}{1.523015in}}%
\pgfpathlineto{\pgfqpoint{1.214157in}{1.529558in}}%
\pgfpathlineto{\pgfqpoint{1.216831in}{1.536022in}}%
\pgfpathlineto{\pgfqpoint{1.219505in}{1.542405in}}%
\pgfpathlineto{\pgfqpoint{1.222180in}{1.548706in}}%
\pgfpathlineto{\pgfqpoint{1.225639in}{1.546775in}}%
\pgfpathlineto{\pgfqpoint{1.229227in}{1.544896in}}%
\pgfpathlineto{\pgfqpoint{1.232940in}{1.543073in}}%
\pgfpathlineto{\pgfqpoint{1.236774in}{1.541306in}}%
\pgfpathclose%
\pgfusepath{fill}%
\end{pgfscope}%
\begin{pgfscope}%
\pgfpathrectangle{\pgfqpoint{0.329460in}{0.284240in}}{\pgfqpoint{1.989680in}{1.989680in}}%
\pgfusepath{clip}%
\pgfsetbuttcap%
\pgfsetroundjoin%
\definecolor{currentfill}{rgb}{0.212395,0.359683,0.551710}%
\pgfsetfillcolor{currentfill}%
\pgfsetlinewidth{0.000000pt}%
\definecolor{currentstroke}{rgb}{0.000000,0.000000,0.000000}%
\pgfsetstrokecolor{currentstroke}%
\pgfsetdash{}{0pt}%
\pgfpathmoveto{\pgfqpoint{1.310182in}{1.056769in}}%
\pgfpathlineto{\pgfqpoint{1.309791in}{1.048303in}}%
\pgfpathlineto{\pgfqpoint{1.309400in}{1.039880in}}%
\pgfpathlineto{\pgfqpoint{1.309009in}{1.031504in}}%
\pgfpathlineto{\pgfqpoint{1.308618in}{1.023179in}}%
\pgfpathlineto{\pgfqpoint{1.296725in}{1.023980in}}%
\pgfpathlineto{\pgfqpoint{1.284892in}{1.024976in}}%
\pgfpathlineto{\pgfqpoint{1.273131in}{1.026167in}}%
\pgfpathlineto{\pgfqpoint{1.261455in}{1.027551in}}%
\pgfpathlineto{\pgfqpoint{1.262279in}{1.035831in}}%
\pgfpathlineto{\pgfqpoint{1.263102in}{1.044161in}}%
\pgfpathlineto{\pgfqpoint{1.263926in}{1.052538in}}%
\pgfpathlineto{\pgfqpoint{1.264750in}{1.060959in}}%
\pgfpathlineto{\pgfqpoint{1.275998in}{1.059633in}}%
\pgfpathlineto{\pgfqpoint{1.287327in}{1.058492in}}%
\pgfpathlineto{\pgfqpoint{1.298726in}{1.057537in}}%
\pgfpathlineto{\pgfqpoint{1.310182in}{1.056769in}}%
\pgfpathclose%
\pgfusepath{fill}%
\end{pgfscope}%
\begin{pgfscope}%
\pgfpathrectangle{\pgfqpoint{0.329460in}{0.284240in}}{\pgfqpoint{1.989680in}{1.989680in}}%
\pgfusepath{clip}%
\pgfsetbuttcap%
\pgfsetroundjoin%
\definecolor{currentfill}{rgb}{0.565498,0.842430,0.262877}%
\pgfsetfillcolor{currentfill}%
\pgfsetlinewidth{0.000000pt}%
\definecolor{currentstroke}{rgb}{0.000000,0.000000,0.000000}%
\pgfsetstrokecolor{currentstroke}%
\pgfsetdash{}{0pt}%
\pgfpathmoveto{\pgfqpoint{1.472204in}{1.574662in}}%
\pgfpathlineto{\pgfqpoint{1.474945in}{1.568744in}}%
\pgfpathlineto{\pgfqpoint{1.477684in}{1.562737in}}%
\pgfpathlineto{\pgfqpoint{1.480422in}{1.556644in}}%
\pgfpathlineto{\pgfqpoint{1.483159in}{1.550465in}}%
\pgfpathlineto{\pgfqpoint{1.479817in}{1.548489in}}%
\pgfpathlineto{\pgfqpoint{1.476343in}{1.546563in}}%
\pgfpathlineto{\pgfqpoint{1.472742in}{1.544691in}}%
\pgfpathlineto{\pgfqpoint{1.469015in}{1.542874in}}%
\pgfpathlineto{\pgfqpoint{1.466568in}{1.549218in}}%
\pgfpathlineto{\pgfqpoint{1.464121in}{1.555476in}}%
\pgfpathlineto{\pgfqpoint{1.461672in}{1.561647in}}%
\pgfpathlineto{\pgfqpoint{1.459223in}{1.567730in}}%
\pgfpathlineto{\pgfqpoint{1.462642in}{1.569389in}}%
\pgfpathlineto{\pgfqpoint{1.465948in}{1.571099in}}%
\pgfpathlineto{\pgfqpoint{1.469136in}{1.572857in}}%
\pgfpathlineto{\pgfqpoint{1.472204in}{1.574662in}}%
\pgfpathclose%
\pgfusepath{fill}%
\end{pgfscope}%
\begin{pgfscope}%
\pgfpathrectangle{\pgfqpoint{0.329460in}{0.284240in}}{\pgfqpoint{1.989680in}{1.989680in}}%
\pgfusepath{clip}%
\pgfsetbuttcap%
\pgfsetroundjoin%
\definecolor{currentfill}{rgb}{0.974417,0.903590,0.130215}%
\pgfsetfillcolor{currentfill}%
\pgfsetlinewidth{0.000000pt}%
\definecolor{currentstroke}{rgb}{0.000000,0.000000,0.000000}%
\pgfsetstrokecolor{currentstroke}%
\pgfsetdash{}{0pt}%
\pgfpathmoveto{\pgfqpoint{1.361959in}{1.715325in}}%
\pgfpathlineto{\pgfqpoint{1.363308in}{1.712789in}}%
\pgfpathlineto{\pgfqpoint{1.364657in}{1.710134in}}%
\pgfpathlineto{\pgfqpoint{1.366007in}{1.707358in}}%
\pgfpathlineto{\pgfqpoint{1.367357in}{1.704464in}}%
\pgfpathlineto{\pgfqpoint{1.366118in}{1.704234in}}%
\pgfpathlineto{\pgfqpoint{1.364864in}{1.704023in}}%
\pgfpathlineto{\pgfqpoint{1.363596in}{1.703830in}}%
\pgfpathlineto{\pgfqpoint{1.362316in}{1.703656in}}%
\pgfpathlineto{\pgfqpoint{1.361386in}{1.706618in}}%
\pgfpathlineto{\pgfqpoint{1.360457in}{1.709461in}}%
\pgfpathlineto{\pgfqpoint{1.359529in}{1.712185in}}%
\pgfpathlineto{\pgfqpoint{1.358600in}{1.714788in}}%
\pgfpathlineto{\pgfqpoint{1.359453in}{1.714903in}}%
\pgfpathlineto{\pgfqpoint{1.360298in}{1.715031in}}%
\pgfpathlineto{\pgfqpoint{1.361133in}{1.715172in}}%
\pgfpathlineto{\pgfqpoint{1.361959in}{1.715325in}}%
\pgfpathclose%
\pgfusepath{fill}%
\end{pgfscope}%
\begin{pgfscope}%
\pgfpathrectangle{\pgfqpoint{0.329460in}{0.284240in}}{\pgfqpoint{1.989680in}{1.989680in}}%
\pgfusepath{clip}%
\pgfsetbuttcap%
\pgfsetroundjoin%
\definecolor{currentfill}{rgb}{0.133743,0.548535,0.553541}%
\pgfsetfillcolor{currentfill}%
\pgfsetlinewidth{0.000000pt}%
\definecolor{currentstroke}{rgb}{0.000000,0.000000,0.000000}%
\pgfsetstrokecolor{currentstroke}%
\pgfsetdash{}{0pt}%
\pgfpathmoveto{\pgfqpoint{1.245653in}{1.239902in}}%
\pgfpathlineto{\pgfqpoint{1.244409in}{1.231260in}}%
\pgfpathlineto{\pgfqpoint{1.243165in}{1.222607in}}%
\pgfpathlineto{\pgfqpoint{1.241921in}{1.213946in}}%
\pgfpathlineto{\pgfqpoint{1.240677in}{1.205280in}}%
\pgfpathlineto{\pgfqpoint{1.231634in}{1.207126in}}%
\pgfpathlineto{\pgfqpoint{1.222717in}{1.209114in}}%
\pgfpathlineto{\pgfqpoint{1.213938in}{1.211244in}}%
\pgfpathlineto{\pgfqpoint{1.205304in}{1.213513in}}%
\pgfpathlineto{\pgfqpoint{1.206944in}{1.222075in}}%
\pgfpathlineto{\pgfqpoint{1.208584in}{1.230632in}}%
\pgfpathlineto{\pgfqpoint{1.210224in}{1.239182in}}%
\pgfpathlineto{\pgfqpoint{1.211865in}{1.247722in}}%
\pgfpathlineto{\pgfqpoint{1.220112in}{1.245567in}}%
\pgfpathlineto{\pgfqpoint{1.228499in}{1.243544in}}%
\pgfpathlineto{\pgfqpoint{1.237015in}{1.241655in}}%
\pgfpathlineto{\pgfqpoint{1.245653in}{1.239902in}}%
\pgfpathclose%
\pgfusepath{fill}%
\end{pgfscope}%
\begin{pgfscope}%
\pgfpathrectangle{\pgfqpoint{0.329460in}{0.284240in}}{\pgfqpoint{1.989680in}{1.989680in}}%
\pgfusepath{clip}%
\pgfsetbuttcap%
\pgfsetroundjoin%
\definecolor{currentfill}{rgb}{0.163625,0.471133,0.558148}%
\pgfsetfillcolor{currentfill}%
\pgfsetlinewidth{0.000000pt}%
\definecolor{currentstroke}{rgb}{0.000000,0.000000,0.000000}%
\pgfsetstrokecolor{currentstroke}%
\pgfsetdash{}{0pt}%
\pgfpathmoveto{\pgfqpoint{1.475074in}{1.172333in}}%
\pgfpathlineto{\pgfqpoint{1.476406in}{1.163701in}}%
\pgfpathlineto{\pgfqpoint{1.477738in}{1.155080in}}%
\pgfpathlineto{\pgfqpoint{1.479070in}{1.146472in}}%
\pgfpathlineto{\pgfqpoint{1.480401in}{1.137880in}}%
\pgfpathlineto{\pgfqpoint{1.470533in}{1.135864in}}%
\pgfpathlineto{\pgfqpoint{1.460536in}{1.134009in}}%
\pgfpathlineto{\pgfqpoint{1.450423in}{1.132317in}}%
\pgfpathlineto{\pgfqpoint{1.440202in}{1.130789in}}%
\pgfpathlineto{\pgfqpoint{1.439284in}{1.139463in}}%
\pgfpathlineto{\pgfqpoint{1.438366in}{1.148154in}}%
\pgfpathlineto{\pgfqpoint{1.437447in}{1.156858in}}%
\pgfpathlineto{\pgfqpoint{1.436529in}{1.165572in}}%
\pgfpathlineto{\pgfqpoint{1.446328in}{1.167028in}}%
\pgfpathlineto{\pgfqpoint{1.456026in}{1.168642in}}%
\pgfpathlineto{\pgfqpoint{1.465611in}{1.170411in}}%
\pgfpathlineto{\pgfqpoint{1.475074in}{1.172333in}}%
\pgfpathclose%
\pgfusepath{fill}%
\end{pgfscope}%
\begin{pgfscope}%
\pgfpathrectangle{\pgfqpoint{0.329460in}{0.284240in}}{\pgfqpoint{1.989680in}{1.989680in}}%
\pgfusepath{clip}%
\pgfsetbuttcap%
\pgfsetroundjoin%
\definecolor{currentfill}{rgb}{0.974417,0.903590,0.130215}%
\pgfsetfillcolor{currentfill}%
\pgfsetlinewidth{0.000000pt}%
\definecolor{currentstroke}{rgb}{0.000000,0.000000,0.000000}%
\pgfsetstrokecolor{currentstroke}%
\pgfsetdash{}{0pt}%
\pgfpathmoveto{\pgfqpoint{1.344539in}{1.714695in}}%
\pgfpathlineto{\pgfqpoint{1.343707in}{1.712081in}}%
\pgfpathlineto{\pgfqpoint{1.342874in}{1.709346in}}%
\pgfpathlineto{\pgfqpoint{1.342040in}{1.706491in}}%
\pgfpathlineto{\pgfqpoint{1.341207in}{1.703517in}}%
\pgfpathlineto{\pgfqpoint{1.339916in}{1.703674in}}%
\pgfpathlineto{\pgfqpoint{1.338637in}{1.703850in}}%
\pgfpathlineto{\pgfqpoint{1.337371in}{1.704045in}}%
\pgfpathlineto{\pgfqpoint{1.336119in}{1.704258in}}%
\pgfpathlineto{\pgfqpoint{1.337377in}{1.707170in}}%
\pgfpathlineto{\pgfqpoint{1.338635in}{1.709963in}}%
\pgfpathlineto{\pgfqpoint{1.339893in}{1.712636in}}%
\pgfpathlineto{\pgfqpoint{1.341150in}{1.715188in}}%
\pgfpathlineto{\pgfqpoint{1.341984in}{1.715046in}}%
\pgfpathlineto{\pgfqpoint{1.342828in}{1.714917in}}%
\pgfpathlineto{\pgfqpoint{1.343680in}{1.714800in}}%
\pgfpathlineto{\pgfqpoint{1.344539in}{1.714695in}}%
\pgfpathclose%
\pgfusepath{fill}%
\end{pgfscope}%
\begin{pgfscope}%
\pgfpathrectangle{\pgfqpoint{0.329460in}{0.284240in}}{\pgfqpoint{1.989680in}{1.989680in}}%
\pgfusepath{clip}%
\pgfsetbuttcap%
\pgfsetroundjoin%
\definecolor{currentfill}{rgb}{0.955300,0.901065,0.118128}%
\pgfsetfillcolor{currentfill}%
\pgfsetlinewidth{0.000000pt}%
\definecolor{currentstroke}{rgb}{0.000000,0.000000,0.000000}%
\pgfsetstrokecolor{currentstroke}%
\pgfsetdash{}{0pt}%
\pgfpathmoveto{\pgfqpoint{1.380620in}{1.708554in}}%
\pgfpathlineto{\pgfqpoint{1.383076in}{1.705884in}}%
\pgfpathlineto{\pgfqpoint{1.385534in}{1.703096in}}%
\pgfpathlineto{\pgfqpoint{1.387991in}{1.700190in}}%
\pgfpathlineto{\pgfqpoint{1.390449in}{1.697166in}}%
\pgfpathlineto{\pgfqpoint{1.389159in}{1.696595in}}%
\pgfpathlineto{\pgfqpoint{1.387832in}{1.696043in}}%
\pgfpathlineto{\pgfqpoint{1.386467in}{1.695511in}}%
\pgfpathlineto{\pgfqpoint{1.385067in}{1.694999in}}%
\pgfpathlineto{\pgfqpoint{1.382945in}{1.698159in}}%
\pgfpathlineto{\pgfqpoint{1.380824in}{1.701202in}}%
\pgfpathlineto{\pgfqpoint{1.378703in}{1.704126in}}%
\pgfpathlineto{\pgfqpoint{1.376583in}{1.706933in}}%
\pgfpathlineto{\pgfqpoint{1.377633in}{1.707315in}}%
\pgfpathlineto{\pgfqpoint{1.378657in}{1.707714in}}%
\pgfpathlineto{\pgfqpoint{1.379652in}{1.708126in}}%
\pgfpathlineto{\pgfqpoint{1.380620in}{1.708554in}}%
\pgfpathclose%
\pgfusepath{fill}%
\end{pgfscope}%
\begin{pgfscope}%
\pgfpathrectangle{\pgfqpoint{0.329460in}{0.284240in}}{\pgfqpoint{1.989680in}{1.989680in}}%
\pgfusepath{clip}%
\pgfsetbuttcap%
\pgfsetroundjoin%
\definecolor{currentfill}{rgb}{0.935904,0.898570,0.108131}%
\pgfsetfillcolor{currentfill}%
\pgfsetlinewidth{0.000000pt}%
\definecolor{currentstroke}{rgb}{0.000000,0.000000,0.000000}%
\pgfsetstrokecolor{currentstroke}%
\pgfsetdash{}{0pt}%
\pgfpathmoveto{\pgfqpoint{1.395199in}{1.699632in}}%
\pgfpathlineto{\pgfqpoint{1.397953in}{1.696648in}}%
\pgfpathlineto{\pgfqpoint{1.400707in}{1.693547in}}%
\pgfpathlineto{\pgfqpoint{1.403462in}{1.690331in}}%
\pgfpathlineto{\pgfqpoint{1.406216in}{1.687000in}}%
\pgfpathlineto{\pgfqpoint{1.404812in}{1.686195in}}%
\pgfpathlineto{\pgfqpoint{1.403354in}{1.685411in}}%
\pgfpathlineto{\pgfqpoint{1.401843in}{1.684648in}}%
\pgfpathlineto{\pgfqpoint{1.400282in}{1.683909in}}%
\pgfpathlineto{\pgfqpoint{1.397823in}{1.687397in}}%
\pgfpathlineto{\pgfqpoint{1.395365in}{1.690769in}}%
\pgfpathlineto{\pgfqpoint{1.392907in}{1.694026in}}%
\pgfpathlineto{\pgfqpoint{1.390449in}{1.697166in}}%
\pgfpathlineto{\pgfqpoint{1.391699in}{1.697756in}}%
\pgfpathlineto{\pgfqpoint{1.392908in}{1.698364in}}%
\pgfpathlineto{\pgfqpoint{1.394075in}{1.698990in}}%
\pgfpathlineto{\pgfqpoint{1.395199in}{1.699632in}}%
\pgfpathclose%
\pgfusepath{fill}%
\end{pgfscope}%
\begin{pgfscope}%
\pgfpathrectangle{\pgfqpoint{0.329460in}{0.284240in}}{\pgfqpoint{1.989680in}{1.989680in}}%
\pgfusepath{clip}%
\pgfsetbuttcap%
\pgfsetroundjoin%
\definecolor{currentfill}{rgb}{0.231674,0.318106,0.544834}%
\pgfsetfillcolor{currentfill}%
\pgfsetlinewidth{0.000000pt}%
\definecolor{currentstroke}{rgb}{0.000000,0.000000,0.000000}%
\pgfsetstrokecolor{currentstroke}%
\pgfsetdash{}{0pt}%
\pgfpathmoveto{\pgfqpoint{1.404331in}{1.023881in}}%
\pgfpathlineto{\pgfqpoint{1.404819in}{1.015616in}}%
\pgfpathlineto{\pgfqpoint{1.405308in}{1.007407in}}%
\pgfpathlineto{\pgfqpoint{1.405796in}{0.999257in}}%
\pgfpathlineto{\pgfqpoint{1.406285in}{0.991169in}}%
\pgfpathlineto{\pgfqpoint{1.393949in}{0.990357in}}%
\pgfpathlineto{\pgfqpoint{1.381567in}{0.989750in}}%
\pgfpathlineto{\pgfqpoint{1.369152in}{0.989349in}}%
\pgfpathlineto{\pgfqpoint{1.356717in}{0.989154in}}%
\pgfpathlineto{\pgfqpoint{1.356668in}{0.997263in}}%
\pgfpathlineto{\pgfqpoint{1.356619in}{1.005433in}}%
\pgfpathlineto{\pgfqpoint{1.356570in}{1.013662in}}%
\pgfpathlineto{\pgfqpoint{1.356521in}{1.021948in}}%
\pgfpathlineto{\pgfqpoint{1.368515in}{1.022135in}}%
\pgfpathlineto{\pgfqpoint{1.380490in}{1.022520in}}%
\pgfpathlineto{\pgfqpoint{1.392433in}{1.023102in}}%
\pgfpathlineto{\pgfqpoint{1.404331in}{1.023881in}}%
\pgfpathclose%
\pgfusepath{fill}%
\end{pgfscope}%
\begin{pgfscope}%
\pgfpathrectangle{\pgfqpoint{0.329460in}{0.284240in}}{\pgfqpoint{1.989680in}{1.989680in}}%
\pgfusepath{clip}%
\pgfsetbuttcap%
\pgfsetroundjoin%
\definecolor{currentfill}{rgb}{0.974417,0.903590,0.130215}%
\pgfsetfillcolor{currentfill}%
\pgfsetlinewidth{0.000000pt}%
\definecolor{currentstroke}{rgb}{0.000000,0.000000,0.000000}%
\pgfsetstrokecolor{currentstroke}%
\pgfsetdash{}{0pt}%
\pgfpathmoveto{\pgfqpoint{1.358600in}{1.714788in}}%
\pgfpathlineto{\pgfqpoint{1.359529in}{1.712185in}}%
\pgfpathlineto{\pgfqpoint{1.360457in}{1.709461in}}%
\pgfpathlineto{\pgfqpoint{1.361386in}{1.706618in}}%
\pgfpathlineto{\pgfqpoint{1.362316in}{1.703656in}}%
\pgfpathlineto{\pgfqpoint{1.361024in}{1.703501in}}%
\pgfpathlineto{\pgfqpoint{1.359723in}{1.703365in}}%
\pgfpathlineto{\pgfqpoint{1.358413in}{1.703249in}}%
\pgfpathlineto{\pgfqpoint{1.357096in}{1.703152in}}%
\pgfpathlineto{\pgfqpoint{1.356602in}{1.706157in}}%
\pgfpathlineto{\pgfqpoint{1.356109in}{1.709042in}}%
\pgfpathlineto{\pgfqpoint{1.355616in}{1.711807in}}%
\pgfpathlineto{\pgfqpoint{1.355123in}{1.714452in}}%
\pgfpathlineto{\pgfqpoint{1.356000in}{1.714517in}}%
\pgfpathlineto{\pgfqpoint{1.356873in}{1.714594in}}%
\pgfpathlineto{\pgfqpoint{1.357740in}{1.714685in}}%
\pgfpathlineto{\pgfqpoint{1.358600in}{1.714788in}}%
\pgfpathclose%
\pgfusepath{fill}%
\end{pgfscope}%
\begin{pgfscope}%
\pgfpathrectangle{\pgfqpoint{0.329460in}{0.284240in}}{\pgfqpoint{1.989680in}{1.989680in}}%
\pgfusepath{clip}%
\pgfsetbuttcap%
\pgfsetroundjoin%
\definecolor{currentfill}{rgb}{0.955300,0.901065,0.118128}%
\pgfsetfillcolor{currentfill}%
\pgfsetlinewidth{0.000000pt}%
\definecolor{currentstroke}{rgb}{0.000000,0.000000,0.000000}%
\pgfsetstrokecolor{currentstroke}%
\pgfsetdash{}{0pt}%
\pgfpathmoveto{\pgfqpoint{1.326747in}{1.706606in}}%
\pgfpathlineto{\pgfqpoint{1.324706in}{1.703772in}}%
\pgfpathlineto{\pgfqpoint{1.322665in}{1.700820in}}%
\pgfpathlineto{\pgfqpoint{1.320623in}{1.697749in}}%
\pgfpathlineto{\pgfqpoint{1.318581in}{1.694562in}}%
\pgfpathlineto{\pgfqpoint{1.317151in}{1.695055in}}%
\pgfpathlineto{\pgfqpoint{1.315755in}{1.695569in}}%
\pgfpathlineto{\pgfqpoint{1.314394in}{1.696103in}}%
\pgfpathlineto{\pgfqpoint{1.313071in}{1.696657in}}%
\pgfpathlineto{\pgfqpoint{1.315457in}{1.699713in}}%
\pgfpathlineto{\pgfqpoint{1.317843in}{1.702651in}}%
\pgfpathlineto{\pgfqpoint{1.320229in}{1.705471in}}%
\pgfpathlineto{\pgfqpoint{1.322614in}{1.708173in}}%
\pgfpathlineto{\pgfqpoint{1.323607in}{1.707759in}}%
\pgfpathlineto{\pgfqpoint{1.324627in}{1.707359in}}%
\pgfpathlineto{\pgfqpoint{1.325674in}{1.706974in}}%
\pgfpathlineto{\pgfqpoint{1.326747in}{1.706606in}}%
\pgfpathclose%
\pgfusepath{fill}%
\end{pgfscope}%
\begin{pgfscope}%
\pgfpathrectangle{\pgfqpoint{0.329460in}{0.284240in}}{\pgfqpoint{1.989680in}{1.989680in}}%
\pgfusepath{clip}%
\pgfsetbuttcap%
\pgfsetroundjoin%
\definecolor{currentfill}{rgb}{0.231674,0.318106,0.544834}%
\pgfsetfillcolor{currentfill}%
\pgfsetlinewidth{0.000000pt}%
\definecolor{currentstroke}{rgb}{0.000000,0.000000,0.000000}%
\pgfsetstrokecolor{currentstroke}%
\pgfsetdash{}{0pt}%
\pgfpathmoveto{\pgfqpoint{1.356521in}{1.021948in}}%
\pgfpathlineto{\pgfqpoint{1.356570in}{1.013662in}}%
\pgfpathlineto{\pgfqpoint{1.356619in}{1.005433in}}%
\pgfpathlineto{\pgfqpoint{1.356668in}{0.997263in}}%
\pgfpathlineto{\pgfqpoint{1.356717in}{0.989154in}}%
\pgfpathlineto{\pgfqpoint{1.344276in}{0.989166in}}%
\pgfpathlineto{\pgfqpoint{1.331843in}{0.989384in}}%
\pgfpathlineto{\pgfqpoint{1.319430in}{0.989808in}}%
\pgfpathlineto{\pgfqpoint{1.307053in}{0.990437in}}%
\pgfpathlineto{\pgfqpoint{1.307444in}{0.998532in}}%
\pgfpathlineto{\pgfqpoint{1.307836in}{1.006690in}}%
\pgfpathlineto{\pgfqpoint{1.308227in}{1.014906in}}%
\pgfpathlineto{\pgfqpoint{1.308618in}{1.023179in}}%
\pgfpathlineto{\pgfqpoint{1.320557in}{1.022575in}}%
\pgfpathlineto{\pgfqpoint{1.332529in}{1.022168in}}%
\pgfpathlineto{\pgfqpoint{1.344521in}{1.021959in}}%
\pgfpathlineto{\pgfqpoint{1.356521in}{1.021948in}}%
\pgfpathclose%
\pgfusepath{fill}%
\end{pgfscope}%
\begin{pgfscope}%
\pgfpathrectangle{\pgfqpoint{0.329460in}{0.284240in}}{\pgfqpoint{1.989680in}{1.989680in}}%
\pgfusepath{clip}%
\pgfsetbuttcap%
\pgfsetroundjoin%
\definecolor{currentfill}{rgb}{0.281477,0.755203,0.432552}%
\pgfsetfillcolor{currentfill}%
\pgfsetlinewidth{0.000000pt}%
\definecolor{currentstroke}{rgb}{0.000000,0.000000,0.000000}%
\pgfsetstrokecolor{currentstroke}%
\pgfsetdash{}{0pt}%
\pgfpathmoveto{\pgfqpoint{1.228859in}{1.450473in}}%
\pgfpathlineto{\pgfqpoint{1.226832in}{1.442983in}}%
\pgfpathlineto{\pgfqpoint{1.224806in}{1.435429in}}%
\pgfpathlineto{\pgfqpoint{1.222780in}{1.427816in}}%
\pgfpathlineto{\pgfqpoint{1.220755in}{1.420144in}}%
\pgfpathlineto{\pgfqpoint{1.215068in}{1.422209in}}%
\pgfpathlineto{\pgfqpoint{1.209521in}{1.424360in}}%
\pgfpathlineto{\pgfqpoint{1.204122in}{1.426596in}}%
\pgfpathlineto{\pgfqpoint{1.198876in}{1.428913in}}%
\pgfpathlineto{\pgfqpoint{1.201237in}{1.436438in}}%
\pgfpathlineto{\pgfqpoint{1.203599in}{1.443905in}}%
\pgfpathlineto{\pgfqpoint{1.205962in}{1.451312in}}%
\pgfpathlineto{\pgfqpoint{1.208326in}{1.458657in}}%
\pgfpathlineto{\pgfqpoint{1.213251in}{1.456495in}}%
\pgfpathlineto{\pgfqpoint{1.218318in}{1.454409in}}%
\pgfpathlineto{\pgfqpoint{1.223522in}{1.452401in}}%
\pgfpathlineto{\pgfqpoint{1.228859in}{1.450473in}}%
\pgfpathclose%
\pgfusepath{fill}%
\end{pgfscope}%
\begin{pgfscope}%
\pgfpathrectangle{\pgfqpoint{0.329460in}{0.284240in}}{\pgfqpoint{1.989680in}{1.989680in}}%
\pgfusepath{clip}%
\pgfsetbuttcap%
\pgfsetroundjoin%
\definecolor{currentfill}{rgb}{0.974417,0.903590,0.130215}%
\pgfsetfillcolor{currentfill}%
\pgfsetlinewidth{0.000000pt}%
\definecolor{currentstroke}{rgb}{0.000000,0.000000,0.000000}%
\pgfsetstrokecolor{currentstroke}%
\pgfsetdash{}{0pt}%
\pgfpathmoveto{\pgfqpoint{1.348036in}{1.714406in}}%
\pgfpathlineto{\pgfqpoint{1.347641in}{1.711755in}}%
\pgfpathlineto{\pgfqpoint{1.347246in}{1.708984in}}%
\pgfpathlineto{\pgfqpoint{1.346851in}{1.706093in}}%
\pgfpathlineto{\pgfqpoint{1.346455in}{1.703082in}}%
\pgfpathlineto{\pgfqpoint{1.345133in}{1.703162in}}%
\pgfpathlineto{\pgfqpoint{1.343816in}{1.703261in}}%
\pgfpathlineto{\pgfqpoint{1.342507in}{1.703379in}}%
\pgfpathlineto{\pgfqpoint{1.341207in}{1.703517in}}%
\pgfpathlineto{\pgfqpoint{1.342040in}{1.706491in}}%
\pgfpathlineto{\pgfqpoint{1.342874in}{1.709346in}}%
\pgfpathlineto{\pgfqpoint{1.343707in}{1.712081in}}%
\pgfpathlineto{\pgfqpoint{1.344539in}{1.714695in}}%
\pgfpathlineto{\pgfqpoint{1.345406in}{1.714604in}}%
\pgfpathlineto{\pgfqpoint{1.346278in}{1.714525in}}%
\pgfpathlineto{\pgfqpoint{1.347155in}{1.714459in}}%
\pgfpathlineto{\pgfqpoint{1.348036in}{1.714406in}}%
\pgfpathclose%
\pgfusepath{fill}%
\end{pgfscope}%
\begin{pgfscope}%
\pgfpathrectangle{\pgfqpoint{0.329460in}{0.284240in}}{\pgfqpoint{1.989680in}{1.989680in}}%
\pgfusepath{clip}%
\pgfsetbuttcap%
\pgfsetroundjoin%
\definecolor{currentfill}{rgb}{0.134692,0.658636,0.517649}%
\pgfsetfillcolor{currentfill}%
\pgfsetlinewidth{0.000000pt}%
\definecolor{currentstroke}{rgb}{0.000000,0.000000,0.000000}%
\pgfsetstrokecolor{currentstroke}%
\pgfsetdash{}{0pt}%
\pgfpathmoveto{\pgfqpoint{1.231599in}{1.348546in}}%
\pgfpathlineto{\pgfqpoint{1.229951in}{1.340312in}}%
\pgfpathlineto{\pgfqpoint{1.228304in}{1.332040in}}%
\pgfpathlineto{\pgfqpoint{1.226657in}{1.323732in}}%
\pgfpathlineto{\pgfqpoint{1.225011in}{1.315390in}}%
\pgfpathlineto{\pgfqpoint{1.217671in}{1.317437in}}%
\pgfpathlineto{\pgfqpoint{1.210472in}{1.319597in}}%
\pgfpathlineto{\pgfqpoint{1.203420in}{1.321869in}}%
\pgfpathlineto{\pgfqpoint{1.196523in}{1.324250in}}%
\pgfpathlineto{\pgfqpoint{1.198538in}{1.332465in}}%
\pgfpathlineto{\pgfqpoint{1.200554in}{1.340646in}}%
\pgfpathlineto{\pgfqpoint{1.202570in}{1.348792in}}%
\pgfpathlineto{\pgfqpoint{1.204587in}{1.356899in}}%
\pgfpathlineto{\pgfqpoint{1.211127in}{1.354655in}}%
\pgfpathlineto{\pgfqpoint{1.217813in}{1.352513in}}%
\pgfpathlineto{\pgfqpoint{1.224640in}{1.350476in}}%
\pgfpathlineto{\pgfqpoint{1.231599in}{1.348546in}}%
\pgfpathclose%
\pgfusepath{fill}%
\end{pgfscope}%
\begin{pgfscope}%
\pgfpathrectangle{\pgfqpoint{0.329460in}{0.284240in}}{\pgfqpoint{1.989680in}{1.989680in}}%
\pgfusepath{clip}%
\pgfsetbuttcap%
\pgfsetroundjoin%
\definecolor{currentfill}{rgb}{0.274952,0.037752,0.364543}%
\pgfsetfillcolor{currentfill}%
\pgfsetlinewidth{0.000000pt}%
\definecolor{currentstroke}{rgb}{0.000000,0.000000,0.000000}%
\pgfsetstrokecolor{currentstroke}%
\pgfsetdash{}{0pt}%
\pgfpathmoveto{\pgfqpoint{1.295996in}{0.802318in}}%
\pgfpathlineto{\pgfqpoint{1.295595in}{0.797480in}}%
\pgfpathlineto{\pgfqpoint{1.295194in}{0.792809in}}%
\pgfpathlineto{\pgfqpoint{1.294791in}{0.788309in}}%
\pgfpathlineto{\pgfqpoint{1.294389in}{0.783983in}}%
\pgfpathlineto{\pgfqpoint{1.278524in}{0.785091in}}%
\pgfpathlineto{\pgfqpoint{1.262741in}{0.786470in}}%
\pgfpathlineto{\pgfqpoint{1.247056in}{0.788117in}}%
\pgfpathlineto{\pgfqpoint{1.231488in}{0.790031in}}%
\pgfpathlineto{\pgfqpoint{1.232336in}{0.794310in}}%
\pgfpathlineto{\pgfqpoint{1.233182in}{0.798763in}}%
\pgfpathlineto{\pgfqpoint{1.234028in}{0.803387in}}%
\pgfpathlineto{\pgfqpoint{1.234872in}{0.808178in}}%
\pgfpathlineto{\pgfqpoint{1.250001in}{0.806323in}}%
\pgfpathlineto{\pgfqpoint{1.265242in}{0.804727in}}%
\pgfpathlineto{\pgfqpoint{1.280580in}{0.803391in}}%
\pgfpathlineto{\pgfqpoint{1.295996in}{0.802318in}}%
\pgfpathclose%
\pgfusepath{fill}%
\end{pgfscope}%
\begin{pgfscope}%
\pgfpathrectangle{\pgfqpoint{0.329460in}{0.284240in}}{\pgfqpoint{1.989680in}{1.989680in}}%
\pgfusepath{clip}%
\pgfsetbuttcap%
\pgfsetroundjoin%
\definecolor{currentfill}{rgb}{0.195860,0.395433,0.555276}%
\pgfsetfillcolor{currentfill}%
\pgfsetlinewidth{0.000000pt}%
\definecolor{currentstroke}{rgb}{0.000000,0.000000,0.000000}%
\pgfsetstrokecolor{currentstroke}%
\pgfsetdash{}{0pt}%
\pgfpathmoveto{\pgfqpoint{1.443874in}{1.096303in}}%
\pgfpathlineto{\pgfqpoint{1.444792in}{1.087748in}}%
\pgfpathlineto{\pgfqpoint{1.445710in}{1.079227in}}%
\pgfpathlineto{\pgfqpoint{1.446628in}{1.070740in}}%
\pgfpathlineto{\pgfqpoint{1.447546in}{1.062292in}}%
\pgfpathlineto{\pgfqpoint{1.436380in}{1.060803in}}%
\pgfpathlineto{\pgfqpoint{1.425122in}{1.059497in}}%
\pgfpathlineto{\pgfqpoint{1.413784in}{1.058376in}}%
\pgfpathlineto{\pgfqpoint{1.402379in}{1.057442in}}%
\pgfpathlineto{\pgfqpoint{1.401891in}{1.065943in}}%
\pgfpathlineto{\pgfqpoint{1.401403in}{1.074481in}}%
\pgfpathlineto{\pgfqpoint{1.400915in}{1.083055in}}%
\pgfpathlineto{\pgfqpoint{1.400427in}{1.091662in}}%
\pgfpathlineto{\pgfqpoint{1.411398in}{1.092556in}}%
\pgfpathlineto{\pgfqpoint{1.422304in}{1.093628in}}%
\pgfpathlineto{\pgfqpoint{1.433133in}{1.094877in}}%
\pgfpathlineto{\pgfqpoint{1.443874in}{1.096303in}}%
\pgfpathclose%
\pgfusepath{fill}%
\end{pgfscope}%
\begin{pgfscope}%
\pgfpathrectangle{\pgfqpoint{0.329460in}{0.284240in}}{\pgfqpoint{1.989680in}{1.989680in}}%
\pgfusepath{clip}%
\pgfsetbuttcap%
\pgfsetroundjoin%
\definecolor{currentfill}{rgb}{0.974417,0.903590,0.130215}%
\pgfsetfillcolor{currentfill}%
\pgfsetlinewidth{0.000000pt}%
\definecolor{currentstroke}{rgb}{0.000000,0.000000,0.000000}%
\pgfsetstrokecolor{currentstroke}%
\pgfsetdash{}{0pt}%
\pgfpathmoveto{\pgfqpoint{1.355123in}{1.714452in}}%
\pgfpathlineto{\pgfqpoint{1.355616in}{1.711807in}}%
\pgfpathlineto{\pgfqpoint{1.356109in}{1.709042in}}%
\pgfpathlineto{\pgfqpoint{1.356602in}{1.706157in}}%
\pgfpathlineto{\pgfqpoint{1.357096in}{1.703152in}}%
\pgfpathlineto{\pgfqpoint{1.355772in}{1.703074in}}%
\pgfpathlineto{\pgfqpoint{1.354445in}{1.703016in}}%
\pgfpathlineto{\pgfqpoint{1.353113in}{1.702978in}}%
\pgfpathlineto{\pgfqpoint{1.351780in}{1.702959in}}%
\pgfpathlineto{\pgfqpoint{1.351731in}{1.705980in}}%
\pgfpathlineto{\pgfqpoint{1.351681in}{1.708882in}}%
\pgfpathlineto{\pgfqpoint{1.351632in}{1.711663in}}%
\pgfpathlineto{\pgfqpoint{1.351582in}{1.714325in}}%
\pgfpathlineto{\pgfqpoint{1.352470in}{1.714337in}}%
\pgfpathlineto{\pgfqpoint{1.353357in}{1.714362in}}%
\pgfpathlineto{\pgfqpoint{1.354241in}{1.714401in}}%
\pgfpathlineto{\pgfqpoint{1.355123in}{1.714452in}}%
\pgfpathclose%
\pgfusepath{fill}%
\end{pgfscope}%
\begin{pgfscope}%
\pgfpathrectangle{\pgfqpoint{0.329460in}{0.284240in}}{\pgfqpoint{1.989680in}{1.989680in}}%
\pgfusepath{clip}%
\pgfsetbuttcap%
\pgfsetroundjoin%
\definecolor{currentfill}{rgb}{0.935904,0.898570,0.108131}%
\pgfsetfillcolor{currentfill}%
\pgfsetlinewidth{0.000000pt}%
\definecolor{currentstroke}{rgb}{0.000000,0.000000,0.000000}%
\pgfsetstrokecolor{currentstroke}%
\pgfsetdash{}{0pt}%
\pgfpathmoveto{\pgfqpoint{1.313071in}{1.696657in}}%
\pgfpathlineto{\pgfqpoint{1.310684in}{1.693485in}}%
\pgfpathlineto{\pgfqpoint{1.308297in}{1.690196in}}%
\pgfpathlineto{\pgfqpoint{1.305910in}{1.686791in}}%
\pgfpathlineto{\pgfqpoint{1.303523in}{1.683271in}}%
\pgfpathlineto{\pgfqpoint{1.301917in}{1.683990in}}%
\pgfpathlineto{\pgfqpoint{1.300361in}{1.684732in}}%
\pgfpathlineto{\pgfqpoint{1.298856in}{1.685497in}}%
\pgfpathlineto{\pgfqpoint{1.297404in}{1.686283in}}%
\pgfpathlineto{\pgfqpoint{1.300097in}{1.689651in}}%
\pgfpathlineto{\pgfqpoint{1.302789in}{1.692903in}}%
\pgfpathlineto{\pgfqpoint{1.305481in}{1.696040in}}%
\pgfpathlineto{\pgfqpoint{1.308173in}{1.699061in}}%
\pgfpathlineto{\pgfqpoint{1.309336in}{1.698433in}}%
\pgfpathlineto{\pgfqpoint{1.310540in}{1.697823in}}%
\pgfpathlineto{\pgfqpoint{1.311786in}{1.697231in}}%
\pgfpathlineto{\pgfqpoint{1.313071in}{1.696657in}}%
\pgfpathclose%
\pgfusepath{fill}%
\end{pgfscope}%
\begin{pgfscope}%
\pgfpathrectangle{\pgfqpoint{0.329460in}{0.284240in}}{\pgfqpoint{1.989680in}{1.989680in}}%
\pgfusepath{clip}%
\pgfsetbuttcap%
\pgfsetroundjoin%
\definecolor{currentfill}{rgb}{0.974417,0.903590,0.130215}%
\pgfsetfillcolor{currentfill}%
\pgfsetlinewidth{0.000000pt}%
\definecolor{currentstroke}{rgb}{0.000000,0.000000,0.000000}%
\pgfsetstrokecolor{currentstroke}%
\pgfsetdash{}{0pt}%
\pgfpathmoveto{\pgfqpoint{1.351582in}{1.714325in}}%
\pgfpathlineto{\pgfqpoint{1.351632in}{1.711663in}}%
\pgfpathlineto{\pgfqpoint{1.351681in}{1.708882in}}%
\pgfpathlineto{\pgfqpoint{1.351731in}{1.705980in}}%
\pgfpathlineto{\pgfqpoint{1.351780in}{1.702959in}}%
\pgfpathlineto{\pgfqpoint{1.350447in}{1.702960in}}%
\pgfpathlineto{\pgfqpoint{1.349114in}{1.702981in}}%
\pgfpathlineto{\pgfqpoint{1.347783in}{1.703022in}}%
\pgfpathlineto{\pgfqpoint{1.346455in}{1.703082in}}%
\pgfpathlineto{\pgfqpoint{1.346851in}{1.706093in}}%
\pgfpathlineto{\pgfqpoint{1.347246in}{1.708984in}}%
\pgfpathlineto{\pgfqpoint{1.347641in}{1.711755in}}%
\pgfpathlineto{\pgfqpoint{1.348036in}{1.714406in}}%
\pgfpathlineto{\pgfqpoint{1.348920in}{1.714366in}}%
\pgfpathlineto{\pgfqpoint{1.349806in}{1.714339in}}%
\pgfpathlineto{\pgfqpoint{1.350694in}{1.714325in}}%
\pgfpathlineto{\pgfqpoint{1.351582in}{1.714325in}}%
\pgfpathclose%
\pgfusepath{fill}%
\end{pgfscope}%
\begin{pgfscope}%
\pgfpathrectangle{\pgfqpoint{0.329460in}{0.284240in}}{\pgfqpoint{1.989680in}{1.989680in}}%
\pgfusepath{clip}%
\pgfsetbuttcap%
\pgfsetroundjoin%
\definecolor{currentfill}{rgb}{0.279566,0.067836,0.391917}%
\pgfsetfillcolor{currentfill}%
\pgfsetlinewidth{0.000000pt}%
\definecolor{currentstroke}{rgb}{0.000000,0.000000,0.000000}%
\pgfsetstrokecolor{currentstroke}%
\pgfsetdash{}{0pt}%
\pgfpathmoveto{\pgfqpoint{1.418092in}{0.824169in}}%
\pgfpathlineto{\pgfqpoint{1.418589in}{0.818711in}}%
\pgfpathlineto{\pgfqpoint{1.419088in}{0.813404in}}%
\pgfpathlineto{\pgfqpoint{1.419586in}{0.808252in}}%
\pgfpathlineto{\pgfqpoint{1.420086in}{0.803259in}}%
\pgfpathlineto{\pgfqpoint{1.404662in}{0.802215in}}%
\pgfpathlineto{\pgfqpoint{1.389179in}{0.801434in}}%
\pgfpathlineto{\pgfqpoint{1.373653in}{0.800918in}}%
\pgfpathlineto{\pgfqpoint{1.358103in}{0.800667in}}%
\pgfpathlineto{\pgfqpoint{1.358052in}{0.805681in}}%
\pgfpathlineto{\pgfqpoint{1.358002in}{0.810853in}}%
\pgfpathlineto{\pgfqpoint{1.357952in}{0.816181in}}%
\pgfpathlineto{\pgfqpoint{1.357902in}{0.821660in}}%
\pgfpathlineto{\pgfqpoint{1.373003in}{0.821903in}}%
\pgfpathlineto{\pgfqpoint{1.388079in}{0.822402in}}%
\pgfpathlineto{\pgfqpoint{1.403114in}{0.823158in}}%
\pgfpathlineto{\pgfqpoint{1.418092in}{0.824169in}}%
\pgfpathclose%
\pgfusepath{fill}%
\end{pgfscope}%
\begin{pgfscope}%
\pgfpathrectangle{\pgfqpoint{0.329460in}{0.284240in}}{\pgfqpoint{1.989680in}{1.989680in}}%
\pgfusepath{clip}%
\pgfsetbuttcap%
\pgfsetroundjoin%
\definecolor{currentfill}{rgb}{0.122606,0.585371,0.546557}%
\pgfsetfillcolor{currentfill}%
\pgfsetlinewidth{0.000000pt}%
\definecolor{currentstroke}{rgb}{0.000000,0.000000,0.000000}%
\pgfsetstrokecolor{currentstroke}%
\pgfsetdash{}{0pt}%
\pgfpathmoveto{\pgfqpoint{1.490811in}{1.283651in}}%
\pgfpathlineto{\pgfqpoint{1.492539in}{1.275202in}}%
\pgfpathlineto{\pgfqpoint{1.494266in}{1.266733in}}%
\pgfpathlineto{\pgfqpoint{1.495992in}{1.258247in}}%
\pgfpathlineto{\pgfqpoint{1.497718in}{1.249746in}}%
\pgfpathlineto{\pgfqpoint{1.489601in}{1.247476in}}%
\pgfpathlineto{\pgfqpoint{1.481338in}{1.245336in}}%
\pgfpathlineto{\pgfqpoint{1.472936in}{1.243327in}}%
\pgfpathlineto{\pgfqpoint{1.464406in}{1.241453in}}%
\pgfpathlineto{\pgfqpoint{1.463071in}{1.250063in}}%
\pgfpathlineto{\pgfqpoint{1.461735in}{1.258658in}}%
\pgfpathlineto{\pgfqpoint{1.460399in}{1.267235in}}%
\pgfpathlineto{\pgfqpoint{1.459063in}{1.275792in}}%
\pgfpathlineto{\pgfqpoint{1.467192in}{1.277568in}}%
\pgfpathlineto{\pgfqpoint{1.475199in}{1.279471in}}%
\pgfpathlineto{\pgfqpoint{1.483075in}{1.281500in}}%
\pgfpathlineto{\pgfqpoint{1.490811in}{1.283651in}}%
\pgfpathclose%
\pgfusepath{fill}%
\end{pgfscope}%
\begin{pgfscope}%
\pgfpathrectangle{\pgfqpoint{0.329460in}{0.284240in}}{\pgfqpoint{1.989680in}{1.989680in}}%
\pgfusepath{clip}%
\pgfsetbuttcap%
\pgfsetroundjoin%
\definecolor{currentfill}{rgb}{0.272594,0.025563,0.353093}%
\pgfsetfillcolor{currentfill}%
\pgfsetlinewidth{0.000000pt}%
\definecolor{currentstroke}{rgb}{0.000000,0.000000,0.000000}%
\pgfsetstrokecolor{currentstroke}%
\pgfsetdash{}{0pt}%
\pgfpathmoveto{\pgfqpoint{1.078810in}{0.775914in}}%
\pgfpathlineto{\pgfqpoint{1.077046in}{0.776478in}}%
\pgfpathlineto{\pgfqpoint{1.075277in}{0.777315in}}%
\pgfpathlineto{\pgfqpoint{1.073503in}{0.778429in}}%
\pgfpathlineto{\pgfqpoint{1.071723in}{0.779827in}}%
\pgfpathlineto{\pgfqpoint{1.055584in}{0.784749in}}%
\pgfpathlineto{\pgfqpoint{1.039775in}{0.789941in}}%
\pgfpathlineto{\pgfqpoint{1.024315in}{0.795397in}}%
\pgfpathlineto{\pgfqpoint{1.009219in}{0.801108in}}%
\pgfpathlineto{\pgfqpoint{1.011390in}{0.799580in}}%
\pgfpathlineto{\pgfqpoint{1.013555in}{0.798334in}}%
\pgfpathlineto{\pgfqpoint{1.015713in}{0.797366in}}%
\pgfpathlineto{\pgfqpoint{1.017864in}{0.796669in}}%
\pgfpathlineto{\pgfqpoint{1.032586in}{0.791098in}}%
\pgfpathlineto{\pgfqpoint{1.047661in}{0.785778in}}%
\pgfpathlineto{\pgfqpoint{1.063075in}{0.780714in}}%
\pgfpathlineto{\pgfqpoint{1.078810in}{0.775914in}}%
\pgfpathclose%
\pgfusepath{fill}%
\end{pgfscope}%
\begin{pgfscope}%
\pgfpathrectangle{\pgfqpoint{0.329460in}{0.284240in}}{\pgfqpoint{1.989680in}{1.989680in}}%
\pgfusepath{clip}%
\pgfsetbuttcap%
\pgfsetroundjoin%
\definecolor{currentfill}{rgb}{0.268510,0.009605,0.335427}%
\pgfsetfillcolor{currentfill}%
\pgfsetlinewidth{0.000000pt}%
\definecolor{currentstroke}{rgb}{0.000000,0.000000,0.000000}%
\pgfsetstrokecolor{currentstroke}%
\pgfsetdash{}{0pt}%
\pgfpathmoveto{\pgfqpoint{1.228081in}{0.774743in}}%
\pgfpathlineto{\pgfqpoint{1.227225in}{0.771399in}}%
\pgfpathlineto{\pgfqpoint{1.226368in}{0.768254in}}%
\pgfpathlineto{\pgfqpoint{1.225508in}{0.765312in}}%
\pgfpathlineto{\pgfqpoint{1.224647in}{0.762579in}}%
\pgfpathlineto{\pgfqpoint{1.208333in}{0.764891in}}%
\pgfpathlineto{\pgfqpoint{1.192179in}{0.767481in}}%
\pgfpathlineto{\pgfqpoint{1.176202in}{0.770346in}}%
\pgfpathlineto{\pgfqpoint{1.160422in}{0.773481in}}%
\pgfpathlineto{\pgfqpoint{1.161718in}{0.776136in}}%
\pgfpathlineto{\pgfqpoint{1.163011in}{0.778999in}}%
\pgfpathlineto{\pgfqpoint{1.164302in}{0.782066in}}%
\pgfpathlineto{\pgfqpoint{1.165590in}{0.785331in}}%
\pgfpathlineto{\pgfqpoint{1.180945in}{0.782286in}}%
\pgfpathlineto{\pgfqpoint{1.196490in}{0.779504in}}%
\pgfpathlineto{\pgfqpoint{1.212208in}{0.776989in}}%
\pgfpathlineto{\pgfqpoint{1.228081in}{0.774743in}}%
\pgfpathclose%
\pgfusepath{fill}%
\end{pgfscope}%
\begin{pgfscope}%
\pgfpathrectangle{\pgfqpoint{0.329460in}{0.284240in}}{\pgfqpoint{1.989680in}{1.989680in}}%
\pgfusepath{clip}%
\pgfsetbuttcap%
\pgfsetroundjoin%
\definecolor{currentfill}{rgb}{0.344074,0.780029,0.397381}%
\pgfsetfillcolor{currentfill}%
\pgfsetlinewidth{0.000000pt}%
\definecolor{currentstroke}{rgb}{0.000000,0.000000,0.000000}%
\pgfsetstrokecolor{currentstroke}%
\pgfsetdash{}{0pt}%
\pgfpathmoveto{\pgfqpoint{1.488556in}{1.489225in}}%
\pgfpathlineto{\pgfqpoint{1.490994in}{1.482181in}}%
\pgfpathlineto{\pgfqpoint{1.493431in}{1.475068in}}%
\pgfpathlineto{\pgfqpoint{1.495867in}{1.467888in}}%
\pgfpathlineto{\pgfqpoint{1.498302in}{1.460642in}}%
\pgfpathlineto{\pgfqpoint{1.493509in}{1.458413in}}%
\pgfpathlineto{\pgfqpoint{1.488569in}{1.456259in}}%
\pgfpathlineto{\pgfqpoint{1.483486in}{1.454182in}}%
\pgfpathlineto{\pgfqpoint{1.478266in}{1.452183in}}%
\pgfpathlineto{\pgfqpoint{1.476160in}{1.459579in}}%
\pgfpathlineto{\pgfqpoint{1.474053in}{1.466909in}}%
\pgfpathlineto{\pgfqpoint{1.471945in}{1.474171in}}%
\pgfpathlineto{\pgfqpoint{1.469837in}{1.481364in}}%
\pgfpathlineto{\pgfqpoint{1.474713in}{1.483221in}}%
\pgfpathlineto{\pgfqpoint{1.479461in}{1.485152in}}%
\pgfpathlineto{\pgfqpoint{1.484077in}{1.487154in}}%
\pgfpathlineto{\pgfqpoint{1.488556in}{1.489225in}}%
\pgfpathclose%
\pgfusepath{fill}%
\end{pgfscope}%
\begin{pgfscope}%
\pgfpathrectangle{\pgfqpoint{0.329460in}{0.284240in}}{\pgfqpoint{1.989680in}{1.989680in}}%
\pgfusepath{clip}%
\pgfsetbuttcap%
\pgfsetroundjoin%
\definecolor{currentfill}{rgb}{0.282327,0.094955,0.417331}%
\pgfsetfillcolor{currentfill}%
\pgfsetlinewidth{0.000000pt}%
\definecolor{currentstroke}{rgb}{0.000000,0.000000,0.000000}%
\pgfsetstrokecolor{currentstroke}%
\pgfsetdash{}{0pt}%
\pgfpathmoveto{\pgfqpoint{1.000458in}{0.810149in}}%
\pgfpathlineto{\pgfqpoint{0.998248in}{0.813168in}}%
\pgfpathlineto{\pgfqpoint{0.996031in}{0.816501in}}%
\pgfpathlineto{\pgfqpoint{0.993804in}{0.820155in}}%
\pgfpathlineto{\pgfqpoint{0.991569in}{0.824134in}}%
\pgfpathlineto{\pgfqpoint{0.976111in}{0.830381in}}%
\pgfpathlineto{\pgfqpoint{0.961070in}{0.836881in}}%
\pgfpathlineto{\pgfqpoint{0.946462in}{0.843626in}}%
\pgfpathlineto{\pgfqpoint{0.932303in}{0.850608in}}%
\pgfpathlineto{\pgfqpoint{0.934896in}{0.846481in}}%
\pgfpathlineto{\pgfqpoint{0.937479in}{0.842679in}}%
\pgfpathlineto{\pgfqpoint{0.940053in}{0.839195in}}%
\pgfpathlineto{\pgfqpoint{0.942617in}{0.836026in}}%
\pgfpathlineto{\pgfqpoint{0.956438in}{0.829201in}}%
\pgfpathlineto{\pgfqpoint{0.970695in}{0.822608in}}%
\pgfpathlineto{\pgfqpoint{0.985373in}{0.816255in}}%
\pgfpathlineto{\pgfqpoint{1.000458in}{0.810149in}}%
\pgfpathclose%
\pgfusepath{fill}%
\end{pgfscope}%
\begin{pgfscope}%
\pgfpathrectangle{\pgfqpoint{0.329460in}{0.284240in}}{\pgfqpoint{1.989680in}{1.989680in}}%
\pgfusepath{clip}%
\pgfsetbuttcap%
\pgfsetroundjoin%
\definecolor{currentfill}{rgb}{0.279566,0.067836,0.391917}%
\pgfsetfillcolor{currentfill}%
\pgfsetlinewidth{0.000000pt}%
\definecolor{currentstroke}{rgb}{0.000000,0.000000,0.000000}%
\pgfsetstrokecolor{currentstroke}%
\pgfsetdash{}{0pt}%
\pgfpathmoveto{\pgfqpoint{1.357902in}{0.821660in}}%
\pgfpathlineto{\pgfqpoint{1.357952in}{0.816181in}}%
\pgfpathlineto{\pgfqpoint{1.358002in}{0.810853in}}%
\pgfpathlineto{\pgfqpoint{1.358052in}{0.805681in}}%
\pgfpathlineto{\pgfqpoint{1.358103in}{0.800667in}}%
\pgfpathlineto{\pgfqpoint{1.342544in}{0.800682in}}%
\pgfpathlineto{\pgfqpoint{1.326995in}{0.800962in}}%
\pgfpathlineto{\pgfqpoint{1.311473in}{0.801508in}}%
\pgfpathlineto{\pgfqpoint{1.295996in}{0.802318in}}%
\pgfpathlineto{\pgfqpoint{1.296396in}{0.807318in}}%
\pgfpathlineto{\pgfqpoint{1.296796in}{0.812478in}}%
\pgfpathlineto{\pgfqpoint{1.297195in}{0.817792in}}%
\pgfpathlineto{\pgfqpoint{1.297594in}{0.823258in}}%
\pgfpathlineto{\pgfqpoint{1.312623in}{0.822474in}}%
\pgfpathlineto{\pgfqpoint{1.327696in}{0.821946in}}%
\pgfpathlineto{\pgfqpoint{1.342794in}{0.821674in}}%
\pgfpathlineto{\pgfqpoint{1.357902in}{0.821660in}}%
\pgfpathclose%
\pgfusepath{fill}%
\end{pgfscope}%
\begin{pgfscope}%
\pgfpathrectangle{\pgfqpoint{0.329460in}{0.284240in}}{\pgfqpoint{1.989680in}{1.989680in}}%
\pgfusepath{clip}%
\pgfsetbuttcap%
\pgfsetroundjoin%
\definecolor{currentfill}{rgb}{0.267004,0.004874,0.329415}%
\pgfsetfillcolor{currentfill}%
\pgfsetlinewidth{0.000000pt}%
\definecolor{currentstroke}{rgb}{0.000000,0.000000,0.000000}%
\pgfsetstrokecolor{currentstroke}%
\pgfsetdash{}{0pt}%
\pgfpathmoveto{\pgfqpoint{1.155208in}{0.765027in}}%
\pgfpathlineto{\pgfqpoint{1.153897in}{0.763477in}}%
\pgfpathlineto{\pgfqpoint{1.152583in}{0.762161in}}%
\pgfpathlineto{\pgfqpoint{1.151265in}{0.761085in}}%
\pgfpathlineto{\pgfqpoint{1.149944in}{0.760252in}}%
\pgfpathlineto{\pgfqpoint{1.133527in}{0.763849in}}%
\pgfpathlineto{\pgfqpoint{1.117354in}{0.767723in}}%
\pgfpathlineto{\pgfqpoint{1.101444in}{0.771870in}}%
\pgfpathlineto{\pgfqpoint{1.085812in}{0.776285in}}%
\pgfpathlineto{\pgfqpoint{1.087550in}{0.777010in}}%
\pgfpathlineto{\pgfqpoint{1.089284in}{0.777979in}}%
\pgfpathlineto{\pgfqpoint{1.091014in}{0.779187in}}%
\pgfpathlineto{\pgfqpoint{1.092739in}{0.780630in}}%
\pgfpathlineto{\pgfqpoint{1.107966in}{0.776333in}}%
\pgfpathlineto{\pgfqpoint{1.123465in}{0.772297in}}%
\pgfpathlineto{\pgfqpoint{1.139218in}{0.768527in}}%
\pgfpathlineto{\pgfqpoint{1.155208in}{0.765027in}}%
\pgfpathclose%
\pgfusepath{fill}%
\end{pgfscope}%
\begin{pgfscope}%
\pgfpathrectangle{\pgfqpoint{0.329460in}{0.284240in}}{\pgfqpoint{1.989680in}{1.989680in}}%
\pgfusepath{clip}%
\pgfsetbuttcap%
\pgfsetroundjoin%
\definecolor{currentfill}{rgb}{0.163625,0.471133,0.558148}%
\pgfsetfillcolor{currentfill}%
\pgfsetlinewidth{0.000000pt}%
\definecolor{currentstroke}{rgb}{0.000000,0.000000,0.000000}%
\pgfsetstrokecolor{currentstroke}%
\pgfsetdash{}{0pt}%
\pgfpathmoveto{\pgfqpoint{1.274634in}{1.164410in}}%
\pgfpathlineto{\pgfqpoint{1.273810in}{1.155682in}}%
\pgfpathlineto{\pgfqpoint{1.272986in}{1.146964in}}%
\pgfpathlineto{\pgfqpoint{1.272162in}{1.138259in}}%
\pgfpathlineto{\pgfqpoint{1.271338in}{1.129570in}}%
\pgfpathlineto{\pgfqpoint{1.261032in}{1.130950in}}%
\pgfpathlineto{\pgfqpoint{1.250823in}{1.132496in}}%
\pgfpathlineto{\pgfqpoint{1.240722in}{1.134207in}}%
\pgfpathlineto{\pgfqpoint{1.230739in}{1.136080in}}%
\pgfpathlineto{\pgfqpoint{1.231981in}{1.144693in}}%
\pgfpathlineto{\pgfqpoint{1.233222in}{1.153322in}}%
\pgfpathlineto{\pgfqpoint{1.234464in}{1.161964in}}%
\pgfpathlineto{\pgfqpoint{1.235706in}{1.170617in}}%
\pgfpathlineto{\pgfqpoint{1.245279in}{1.168831in}}%
\pgfpathlineto{\pgfqpoint{1.254964in}{1.167200in}}%
\pgfpathlineto{\pgfqpoint{1.264753in}{1.165726in}}%
\pgfpathlineto{\pgfqpoint{1.274634in}{1.164410in}}%
\pgfpathclose%
\pgfusepath{fill}%
\end{pgfscope}%
\begin{pgfscope}%
\pgfpathrectangle{\pgfqpoint{0.329460in}{0.284240in}}{\pgfqpoint{1.989680in}{1.989680in}}%
\pgfusepath{clip}%
\pgfsetbuttcap%
\pgfsetroundjoin%
\definecolor{currentfill}{rgb}{0.636902,0.856542,0.216620}%
\pgfsetfillcolor{currentfill}%
\pgfsetlinewidth{0.000000pt}%
\definecolor{currentstroke}{rgb}{0.000000,0.000000,0.000000}%
\pgfsetstrokecolor{currentstroke}%
\pgfsetdash{}{0pt}%
\pgfpathmoveto{\pgfqpoint{1.461233in}{1.597426in}}%
\pgfpathlineto{\pgfqpoint{1.463977in}{1.591874in}}%
\pgfpathlineto{\pgfqpoint{1.466721in}{1.586229in}}%
\pgfpathlineto{\pgfqpoint{1.469463in}{1.580491in}}%
\pgfpathlineto{\pgfqpoint{1.472204in}{1.574662in}}%
\pgfpathlineto{\pgfqpoint{1.469136in}{1.572857in}}%
\pgfpathlineto{\pgfqpoint{1.465948in}{1.571099in}}%
\pgfpathlineto{\pgfqpoint{1.462642in}{1.569389in}}%
\pgfpathlineto{\pgfqpoint{1.459223in}{1.567730in}}%
\pgfpathlineto{\pgfqpoint{1.456773in}{1.573723in}}%
\pgfpathlineto{\pgfqpoint{1.454322in}{1.579624in}}%
\pgfpathlineto{\pgfqpoint{1.451870in}{1.585432in}}%
\pgfpathlineto{\pgfqpoint{1.449418in}{1.591147in}}%
\pgfpathlineto{\pgfqpoint{1.452530in}{1.592650in}}%
\pgfpathlineto{\pgfqpoint{1.455538in}{1.594198in}}%
\pgfpathlineto{\pgfqpoint{1.458440in}{1.595791in}}%
\pgfpathlineto{\pgfqpoint{1.461233in}{1.597426in}}%
\pgfpathclose%
\pgfusepath{fill}%
\end{pgfscope}%
\begin{pgfscope}%
\pgfpathrectangle{\pgfqpoint{0.329460in}{0.284240in}}{\pgfqpoint{1.989680in}{1.989680in}}%
\pgfusepath{clip}%
\pgfsetbuttcap%
\pgfsetroundjoin%
\definecolor{currentfill}{rgb}{0.565498,0.842430,0.262877}%
\pgfsetfillcolor{currentfill}%
\pgfsetlinewidth{0.000000pt}%
\definecolor{currentstroke}{rgb}{0.000000,0.000000,0.000000}%
\pgfsetstrokecolor{currentstroke}%
\pgfsetdash{}{0pt}%
\pgfpathmoveto{\pgfqpoint{1.246285in}{1.566299in}}%
\pgfpathlineto{\pgfqpoint{1.243906in}{1.560182in}}%
\pgfpathlineto{\pgfqpoint{1.241528in}{1.553977in}}%
\pgfpathlineto{\pgfqpoint{1.239151in}{1.547684in}}%
\pgfpathlineto{\pgfqpoint{1.236774in}{1.541306in}}%
\pgfpathlineto{\pgfqpoint{1.232940in}{1.543073in}}%
\pgfpathlineto{\pgfqpoint{1.229227in}{1.544896in}}%
\pgfpathlineto{\pgfqpoint{1.225639in}{1.546775in}}%
\pgfpathlineto{\pgfqpoint{1.222180in}{1.548706in}}%
\pgfpathlineto{\pgfqpoint{1.224857in}{1.554923in}}%
\pgfpathlineto{\pgfqpoint{1.227534in}{1.561054in}}%
\pgfpathlineto{\pgfqpoint{1.230212in}{1.567099in}}%
\pgfpathlineto{\pgfqpoint{1.232892in}{1.573056in}}%
\pgfpathlineto{\pgfqpoint{1.236067in}{1.571292in}}%
\pgfpathlineto{\pgfqpoint{1.239360in}{1.569577in}}%
\pgfpathlineto{\pgfqpoint{1.242767in}{1.567912in}}%
\pgfpathlineto{\pgfqpoint{1.246285in}{1.566299in}}%
\pgfpathclose%
\pgfusepath{fill}%
\end{pgfscope}%
\begin{pgfscope}%
\pgfpathrectangle{\pgfqpoint{0.329460in}{0.284240in}}{\pgfqpoint{1.989680in}{1.989680in}}%
\pgfusepath{clip}%
\pgfsetbuttcap%
\pgfsetroundjoin%
\definecolor{currentfill}{rgb}{0.166383,0.690856,0.496502}%
\pgfsetfillcolor{currentfill}%
\pgfsetlinewidth{0.000000pt}%
\definecolor{currentstroke}{rgb}{0.000000,0.000000,0.000000}%
\pgfsetstrokecolor{currentstroke}%
\pgfsetdash{}{0pt}%
\pgfpathmoveto{\pgfqpoint{1.495085in}{1.390864in}}%
\pgfpathlineto{\pgfqpoint{1.497183in}{1.382960in}}%
\pgfpathlineto{\pgfqpoint{1.499280in}{1.375010in}}%
\pgfpathlineto{\pgfqpoint{1.501377in}{1.367016in}}%
\pgfpathlineto{\pgfqpoint{1.503472in}{1.358979in}}%
\pgfpathlineto{\pgfqpoint{1.497069in}{1.356645in}}%
\pgfpathlineto{\pgfqpoint{1.490512in}{1.354411in}}%
\pgfpathlineto{\pgfqpoint{1.483810in}{1.352281in}}%
\pgfpathlineto{\pgfqpoint{1.476969in}{1.350256in}}%
\pgfpathlineto{\pgfqpoint{1.475235in}{1.358424in}}%
\pgfpathlineto{\pgfqpoint{1.473501in}{1.366550in}}%
\pgfpathlineto{\pgfqpoint{1.471767in}{1.374631in}}%
\pgfpathlineto{\pgfqpoint{1.470031in}{1.382665in}}%
\pgfpathlineto{\pgfqpoint{1.476498in}{1.384568in}}%
\pgfpathlineto{\pgfqpoint{1.482833in}{1.386571in}}%
\pgfpathlineto{\pgfqpoint{1.489031in}{1.388670in}}%
\pgfpathlineto{\pgfqpoint{1.495085in}{1.390864in}}%
\pgfpathclose%
\pgfusepath{fill}%
\end{pgfscope}%
\begin{pgfscope}%
\pgfpathrectangle{\pgfqpoint{0.329460in}{0.284240in}}{\pgfqpoint{1.989680in}{1.989680in}}%
\pgfusepath{clip}%
\pgfsetbuttcap%
\pgfsetroundjoin%
\definecolor{currentfill}{rgb}{0.195860,0.395433,0.555276}%
\pgfsetfillcolor{currentfill}%
\pgfsetlinewidth{0.000000pt}%
\definecolor{currentstroke}{rgb}{0.000000,0.000000,0.000000}%
\pgfsetstrokecolor{currentstroke}%
\pgfsetdash{}{0pt}%
\pgfpathmoveto{\pgfqpoint{1.311746in}{1.091018in}}%
\pgfpathlineto{\pgfqpoint{1.311355in}{1.082404in}}%
\pgfpathlineto{\pgfqpoint{1.310964in}{1.073823in}}%
\pgfpathlineto{\pgfqpoint{1.310573in}{1.065277in}}%
\pgfpathlineto{\pgfqpoint{1.310182in}{1.056769in}}%
\pgfpathlineto{\pgfqpoint{1.298726in}{1.057537in}}%
\pgfpathlineto{\pgfqpoint{1.287327in}{1.058492in}}%
\pgfpathlineto{\pgfqpoint{1.275998in}{1.059633in}}%
\pgfpathlineto{\pgfqpoint{1.264750in}{1.060959in}}%
\pgfpathlineto{\pgfqpoint{1.265573in}{1.069422in}}%
\pgfpathlineto{\pgfqpoint{1.266397in}{1.077922in}}%
\pgfpathlineto{\pgfqpoint{1.267220in}{1.086459in}}%
\pgfpathlineto{\pgfqpoint{1.268044in}{1.095027in}}%
\pgfpathlineto{\pgfqpoint{1.278864in}{1.093758in}}%
\pgfpathlineto{\pgfqpoint{1.289762in}{1.092666in}}%
\pgfpathlineto{\pgfqpoint{1.300727in}{1.091752in}}%
\pgfpathlineto{\pgfqpoint{1.311746in}{1.091018in}}%
\pgfpathclose%
\pgfusepath{fill}%
\end{pgfscope}%
\begin{pgfscope}%
\pgfpathrectangle{\pgfqpoint{0.329460in}{0.284240in}}{\pgfqpoint{1.989680in}{1.989680in}}%
\pgfusepath{clip}%
\pgfsetbuttcap%
\pgfsetroundjoin%
\definecolor{currentfill}{rgb}{0.896320,0.893616,0.096335}%
\pgfsetfillcolor{currentfill}%
\pgfsetlinewidth{0.000000pt}%
\definecolor{currentstroke}{rgb}{0.000000,0.000000,0.000000}%
\pgfsetstrokecolor{currentstroke}%
\pgfsetdash{}{0pt}%
\pgfpathmoveto{\pgfqpoint{1.406216in}{1.687000in}}%
\pgfpathlineto{\pgfqpoint{1.408970in}{1.683554in}}%
\pgfpathlineto{\pgfqpoint{1.411724in}{1.679995in}}%
\pgfpathlineto{\pgfqpoint{1.414478in}{1.676322in}}%
\pgfpathlineto{\pgfqpoint{1.417232in}{1.672538in}}%
\pgfpathlineto{\pgfqpoint{1.415549in}{1.671569in}}%
\pgfpathlineto{\pgfqpoint{1.413801in}{1.670625in}}%
\pgfpathlineto{\pgfqpoint{1.411989in}{1.669708in}}%
\pgfpathlineto{\pgfqpoint{1.410116in}{1.668818in}}%
\pgfpathlineto{\pgfqpoint{1.407658in}{1.672761in}}%
\pgfpathlineto{\pgfqpoint{1.405199in}{1.676590in}}%
\pgfpathlineto{\pgfqpoint{1.402740in}{1.680307in}}%
\pgfpathlineto{\pgfqpoint{1.400282in}{1.683909in}}%
\pgfpathlineto{\pgfqpoint{1.401843in}{1.684648in}}%
\pgfpathlineto{\pgfqpoint{1.403354in}{1.685411in}}%
\pgfpathlineto{\pgfqpoint{1.404812in}{1.686195in}}%
\pgfpathlineto{\pgfqpoint{1.406216in}{1.687000in}}%
\pgfpathclose%
\pgfusepath{fill}%
\end{pgfscope}%
\begin{pgfscope}%
\pgfpathrectangle{\pgfqpoint{0.329460in}{0.284240in}}{\pgfqpoint{1.989680in}{1.989680in}}%
\pgfusepath{clip}%
\pgfsetbuttcap%
\pgfsetroundjoin%
\definecolor{currentfill}{rgb}{0.212395,0.359683,0.551710}%
\pgfsetfillcolor{currentfill}%
\pgfsetlinewidth{0.000000pt}%
\definecolor{currentstroke}{rgb}{0.000000,0.000000,0.000000}%
\pgfsetstrokecolor{currentstroke}%
\pgfsetdash{}{0pt}%
\pgfpathmoveto{\pgfqpoint{1.402379in}{1.057442in}}%
\pgfpathlineto{\pgfqpoint{1.402867in}{1.048983in}}%
\pgfpathlineto{\pgfqpoint{1.403355in}{1.040567in}}%
\pgfpathlineto{\pgfqpoint{1.403843in}{1.032199in}}%
\pgfpathlineto{\pgfqpoint{1.404331in}{1.023881in}}%
\pgfpathlineto{\pgfqpoint{1.392433in}{1.023102in}}%
\pgfpathlineto{\pgfqpoint{1.380490in}{1.022520in}}%
\pgfpathlineto{\pgfqpoint{1.368515in}{1.022135in}}%
\pgfpathlineto{\pgfqpoint{1.356521in}{1.021948in}}%
\pgfpathlineto{\pgfqpoint{1.356472in}{1.030286in}}%
\pgfpathlineto{\pgfqpoint{1.356423in}{1.038675in}}%
\pgfpathlineto{\pgfqpoint{1.356374in}{1.047110in}}%
\pgfpathlineto{\pgfqpoint{1.356325in}{1.055589in}}%
\pgfpathlineto{\pgfqpoint{1.367878in}{1.055769in}}%
\pgfpathlineto{\pgfqpoint{1.379413in}{1.056138in}}%
\pgfpathlineto{\pgfqpoint{1.390917in}{1.056696in}}%
\pgfpathlineto{\pgfqpoint{1.402379in}{1.057442in}}%
\pgfpathclose%
\pgfusepath{fill}%
\end{pgfscope}%
\begin{pgfscope}%
\pgfpathrectangle{\pgfqpoint{0.329460in}{0.284240in}}{\pgfqpoint{1.989680in}{1.989680in}}%
\pgfusepath{clip}%
\pgfsetbuttcap%
\pgfsetroundjoin%
\definecolor{currentfill}{rgb}{0.955300,0.901065,0.118128}%
\pgfsetfillcolor{currentfill}%
\pgfsetlinewidth{0.000000pt}%
\definecolor{currentstroke}{rgb}{0.000000,0.000000,0.000000}%
\pgfsetstrokecolor{currentstroke}%
\pgfsetdash{}{0pt}%
\pgfpathmoveto{\pgfqpoint{1.376583in}{1.706933in}}%
\pgfpathlineto{\pgfqpoint{1.378703in}{1.704126in}}%
\pgfpathlineto{\pgfqpoint{1.380824in}{1.701202in}}%
\pgfpathlineto{\pgfqpoint{1.382945in}{1.698159in}}%
\pgfpathlineto{\pgfqpoint{1.385067in}{1.694999in}}%
\pgfpathlineto{\pgfqpoint{1.383633in}{1.694508in}}%
\pgfpathlineto{\pgfqpoint{1.382166in}{1.694039in}}%
\pgfpathlineto{\pgfqpoint{1.380668in}{1.693591in}}%
\pgfpathlineto{\pgfqpoint{1.379140in}{1.693166in}}%
\pgfpathlineto{\pgfqpoint{1.377389in}{1.696442in}}%
\pgfpathlineto{\pgfqpoint{1.375638in}{1.699600in}}%
\pgfpathlineto{\pgfqpoint{1.373888in}{1.702640in}}%
\pgfpathlineto{\pgfqpoint{1.372138in}{1.705562in}}%
\pgfpathlineto{\pgfqpoint{1.373284in}{1.705880in}}%
\pgfpathlineto{\pgfqpoint{1.374407in}{1.706214in}}%
\pgfpathlineto{\pgfqpoint{1.375507in}{1.706566in}}%
\pgfpathlineto{\pgfqpoint{1.376583in}{1.706933in}}%
\pgfpathclose%
\pgfusepath{fill}%
\end{pgfscope}%
\begin{pgfscope}%
\pgfpathrectangle{\pgfqpoint{0.329460in}{0.284240in}}{\pgfqpoint{1.989680in}{1.989680in}}%
\pgfusepath{clip}%
\pgfsetbuttcap%
\pgfsetroundjoin%
\definecolor{currentfill}{rgb}{0.212395,0.359683,0.551710}%
\pgfsetfillcolor{currentfill}%
\pgfsetlinewidth{0.000000pt}%
\definecolor{currentstroke}{rgb}{0.000000,0.000000,0.000000}%
\pgfsetstrokecolor{currentstroke}%
\pgfsetdash{}{0pt}%
\pgfpathmoveto{\pgfqpoint{1.356325in}{1.055589in}}%
\pgfpathlineto{\pgfqpoint{1.356374in}{1.047110in}}%
\pgfpathlineto{\pgfqpoint{1.356423in}{1.038675in}}%
\pgfpathlineto{\pgfqpoint{1.356472in}{1.030286in}}%
\pgfpathlineto{\pgfqpoint{1.356521in}{1.021948in}}%
\pgfpathlineto{\pgfqpoint{1.344521in}{1.021959in}}%
\pgfpathlineto{\pgfqpoint{1.332529in}{1.022168in}}%
\pgfpathlineto{\pgfqpoint{1.320557in}{1.022575in}}%
\pgfpathlineto{\pgfqpoint{1.308618in}{1.023179in}}%
\pgfpathlineto{\pgfqpoint{1.309009in}{1.031504in}}%
\pgfpathlineto{\pgfqpoint{1.309400in}{1.039880in}}%
\pgfpathlineto{\pgfqpoint{1.309791in}{1.048303in}}%
\pgfpathlineto{\pgfqpoint{1.310182in}{1.056769in}}%
\pgfpathlineto{\pgfqpoint{1.321682in}{1.056190in}}%
\pgfpathlineto{\pgfqpoint{1.333214in}{1.055800in}}%
\pgfpathlineto{\pgfqpoint{1.344766in}{1.055600in}}%
\pgfpathlineto{\pgfqpoint{1.356325in}{1.055589in}}%
\pgfpathclose%
\pgfusepath{fill}%
\end{pgfscope}%
\begin{pgfscope}%
\pgfpathrectangle{\pgfqpoint{0.329460in}{0.284240in}}{\pgfqpoint{1.989680in}{1.989680in}}%
\pgfusepath{clip}%
\pgfsetbuttcap%
\pgfsetroundjoin%
\definecolor{currentfill}{rgb}{0.276194,0.190074,0.493001}%
\pgfsetfillcolor{currentfill}%
\pgfsetlinewidth{0.000000pt}%
\definecolor{currentstroke}{rgb}{0.000000,0.000000,0.000000}%
\pgfsetstrokecolor{currentstroke}%
\pgfsetdash{}{0pt}%
\pgfpathmoveto{\pgfqpoint{0.921826in}{0.870481in}}%
\pgfpathlineto{\pgfqpoint{0.919179in}{0.876319in}}%
\pgfpathlineto{\pgfqpoint{0.916520in}{0.882518in}}%
\pgfpathlineto{\pgfqpoint{0.913849in}{0.889084in}}%
\pgfpathlineto{\pgfqpoint{0.911166in}{0.896022in}}%
\pgfpathlineto{\pgfqpoint{0.896803in}{0.903545in}}%
\pgfpathlineto{\pgfqpoint{0.882943in}{0.911296in}}%
\pgfpathlineto{\pgfqpoint{0.869598in}{0.919266in}}%
\pgfpathlineto{\pgfqpoint{0.856783in}{0.927445in}}%
\pgfpathlineto{\pgfqpoint{0.859783in}{0.920349in}}%
\pgfpathlineto{\pgfqpoint{0.862769in}{0.913624in}}%
\pgfpathlineto{\pgfqpoint{0.865743in}{0.907264in}}%
\pgfpathlineto{\pgfqpoint{0.868704in}{0.901264in}}%
\pgfpathlineto{\pgfqpoint{0.881225in}{0.893250in}}%
\pgfpathlineto{\pgfqpoint{0.894260in}{0.885442in}}%
\pgfpathlineto{\pgfqpoint{0.907798in}{0.877849in}}%
\pgfpathlineto{\pgfqpoint{0.921826in}{0.870481in}}%
\pgfpathclose%
\pgfusepath{fill}%
\end{pgfscope}%
\begin{pgfscope}%
\pgfpathrectangle{\pgfqpoint{0.329460in}{0.284240in}}{\pgfqpoint{1.989680in}{1.989680in}}%
\pgfusepath{clip}%
\pgfsetbuttcap%
\pgfsetroundjoin%
\definecolor{currentfill}{rgb}{0.955300,0.901065,0.118128}%
\pgfsetfillcolor{currentfill}%
\pgfsetlinewidth{0.000000pt}%
\definecolor{currentstroke}{rgb}{0.000000,0.000000,0.000000}%
\pgfsetstrokecolor{currentstroke}%
\pgfsetdash{}{0pt}%
\pgfpathmoveto{\pgfqpoint{1.331273in}{1.705294in}}%
\pgfpathlineto{\pgfqpoint{1.329609in}{1.702350in}}%
\pgfpathlineto{\pgfqpoint{1.327946in}{1.699287in}}%
\pgfpathlineto{\pgfqpoint{1.326282in}{1.696106in}}%
\pgfpathlineto{\pgfqpoint{1.324617in}{1.692808in}}%
\pgfpathlineto{\pgfqpoint{1.323064in}{1.693212in}}%
\pgfpathlineto{\pgfqpoint{1.321540in}{1.693640in}}%
\pgfpathlineto{\pgfqpoint{1.320045in}{1.694090in}}%
\pgfpathlineto{\pgfqpoint{1.318581in}{1.694562in}}%
\pgfpathlineto{\pgfqpoint{1.320623in}{1.697749in}}%
\pgfpathlineto{\pgfqpoint{1.322665in}{1.700820in}}%
\pgfpathlineto{\pgfqpoint{1.324706in}{1.703772in}}%
\pgfpathlineto{\pgfqpoint{1.326747in}{1.706606in}}%
\pgfpathlineto{\pgfqpoint{1.327844in}{1.706253in}}%
\pgfpathlineto{\pgfqpoint{1.328965in}{1.705916in}}%
\pgfpathlineto{\pgfqpoint{1.330108in}{1.705596in}}%
\pgfpathlineto{\pgfqpoint{1.331273in}{1.705294in}}%
\pgfpathclose%
\pgfusepath{fill}%
\end{pgfscope}%
\begin{pgfscope}%
\pgfpathrectangle{\pgfqpoint{0.329460in}{0.284240in}}{\pgfqpoint{1.989680in}{1.989680in}}%
\pgfusepath{clip}%
\pgfsetbuttcap%
\pgfsetroundjoin%
\definecolor{currentfill}{rgb}{0.896320,0.893616,0.096335}%
\pgfsetfillcolor{currentfill}%
\pgfsetlinewidth{0.000000pt}%
\definecolor{currentstroke}{rgb}{0.000000,0.000000,0.000000}%
\pgfsetstrokecolor{currentstroke}%
\pgfsetdash{}{0pt}%
\pgfpathmoveto{\pgfqpoint{1.303523in}{1.683271in}}%
\pgfpathlineto{\pgfqpoint{1.301136in}{1.679637in}}%
\pgfpathlineto{\pgfqpoint{1.298748in}{1.675888in}}%
\pgfpathlineto{\pgfqpoint{1.296361in}{1.672026in}}%
\pgfpathlineto{\pgfqpoint{1.293974in}{1.668051in}}%
\pgfpathlineto{\pgfqpoint{1.292048in}{1.668916in}}%
\pgfpathlineto{\pgfqpoint{1.290181in}{1.669809in}}%
\pgfpathlineto{\pgfqpoint{1.288377in}{1.670729in}}%
\pgfpathlineto{\pgfqpoint{1.286636in}{1.671675in}}%
\pgfpathlineto{\pgfqpoint{1.289328in}{1.675496in}}%
\pgfpathlineto{\pgfqpoint{1.292020in}{1.679205in}}%
\pgfpathlineto{\pgfqpoint{1.294712in}{1.682801in}}%
\pgfpathlineto{\pgfqpoint{1.297404in}{1.686283in}}%
\pgfpathlineto{\pgfqpoint{1.298856in}{1.685497in}}%
\pgfpathlineto{\pgfqpoint{1.300361in}{1.684732in}}%
\pgfpathlineto{\pgfqpoint{1.301917in}{1.683990in}}%
\pgfpathlineto{\pgfqpoint{1.303523in}{1.683271in}}%
\pgfpathclose%
\pgfusepath{fill}%
\end{pgfscope}%
\begin{pgfscope}%
\pgfpathrectangle{\pgfqpoint{0.329460in}{0.284240in}}{\pgfqpoint{1.989680in}{1.989680in}}%
\pgfusepath{clip}%
\pgfsetbuttcap%
\pgfsetroundjoin%
\definecolor{currentfill}{rgb}{0.699415,0.867117,0.175971}%
\pgfsetfillcolor{currentfill}%
\pgfsetlinewidth{0.000000pt}%
\definecolor{currentstroke}{rgb}{0.000000,0.000000,0.000000}%
\pgfsetstrokecolor{currentstroke}%
\pgfsetdash{}{0pt}%
\pgfpathmoveto{\pgfqpoint{1.450248in}{1.618669in}}%
\pgfpathlineto{\pgfqpoint{1.452996in}{1.613505in}}%
\pgfpathlineto{\pgfqpoint{1.455742in}{1.608243in}}%
\pgfpathlineto{\pgfqpoint{1.458488in}{1.602882in}}%
\pgfpathlineto{\pgfqpoint{1.461233in}{1.597426in}}%
\pgfpathlineto{\pgfqpoint{1.458440in}{1.595791in}}%
\pgfpathlineto{\pgfqpoint{1.455538in}{1.594198in}}%
\pgfpathlineto{\pgfqpoint{1.452530in}{1.592650in}}%
\pgfpathlineto{\pgfqpoint{1.449418in}{1.591147in}}%
\pgfpathlineto{\pgfqpoint{1.446965in}{1.596766in}}%
\pgfpathlineto{\pgfqpoint{1.444512in}{1.602288in}}%
\pgfpathlineto{\pgfqpoint{1.442058in}{1.607712in}}%
\pgfpathlineto{\pgfqpoint{1.439603in}{1.613038in}}%
\pgfpathlineto{\pgfqpoint{1.442406in}{1.614386in}}%
\pgfpathlineto{\pgfqpoint{1.445117in}{1.615774in}}%
\pgfpathlineto{\pgfqpoint{1.447731in}{1.617203in}}%
\pgfpathlineto{\pgfqpoint{1.450248in}{1.618669in}}%
\pgfpathclose%
\pgfusepath{fill}%
\end{pgfscope}%
\begin{pgfscope}%
\pgfpathrectangle{\pgfqpoint{0.329460in}{0.284240in}}{\pgfqpoint{1.989680in}{1.989680in}}%
\pgfusepath{clip}%
\pgfsetbuttcap%
\pgfsetroundjoin%
\definecolor{currentfill}{rgb}{0.147607,0.511733,0.557049}%
\pgfsetfillcolor{currentfill}%
\pgfsetlinewidth{0.000000pt}%
\definecolor{currentstroke}{rgb}{0.000000,0.000000,0.000000}%
\pgfsetstrokecolor{currentstroke}%
\pgfsetdash{}{0pt}%
\pgfpathmoveto{\pgfqpoint{1.469743in}{1.206913in}}%
\pgfpathlineto{\pgfqpoint{1.471076in}{1.198266in}}%
\pgfpathlineto{\pgfqpoint{1.472409in}{1.189618in}}%
\pgfpathlineto{\pgfqpoint{1.473742in}{1.180973in}}%
\pgfpathlineto{\pgfqpoint{1.475074in}{1.172333in}}%
\pgfpathlineto{\pgfqpoint{1.465611in}{1.170411in}}%
\pgfpathlineto{\pgfqpoint{1.456026in}{1.168642in}}%
\pgfpathlineto{\pgfqpoint{1.446328in}{1.167028in}}%
\pgfpathlineto{\pgfqpoint{1.436529in}{1.165572in}}%
\pgfpathlineto{\pgfqpoint{1.435610in}{1.174293in}}%
\pgfpathlineto{\pgfqpoint{1.434691in}{1.183020in}}%
\pgfpathlineto{\pgfqpoint{1.433772in}{1.191749in}}%
\pgfpathlineto{\pgfqpoint{1.432852in}{1.200478in}}%
\pgfpathlineto{\pgfqpoint{1.442230in}{1.201865in}}%
\pgfpathlineto{\pgfqpoint{1.451511in}{1.203401in}}%
\pgfpathlineto{\pgfqpoint{1.460685in}{1.205084in}}%
\pgfpathlineto{\pgfqpoint{1.469743in}{1.206913in}}%
\pgfpathclose%
\pgfusepath{fill}%
\end{pgfscope}%
\begin{pgfscope}%
\pgfpathrectangle{\pgfqpoint{0.329460in}{0.284240in}}{\pgfqpoint{1.989680in}{1.989680in}}%
\pgfusepath{clip}%
\pgfsetbuttcap%
\pgfsetroundjoin%
\definecolor{currentfill}{rgb}{0.855810,0.888601,0.097452}%
\pgfsetfillcolor{currentfill}%
\pgfsetlinewidth{0.000000pt}%
\definecolor{currentstroke}{rgb}{0.000000,0.000000,0.000000}%
\pgfsetstrokecolor{currentstroke}%
\pgfsetdash{}{0pt}%
\pgfpathmoveto{\pgfqpoint{1.417232in}{1.672538in}}%
\pgfpathlineto{\pgfqpoint{1.419986in}{1.668641in}}%
\pgfpathlineto{\pgfqpoint{1.422739in}{1.664634in}}%
\pgfpathlineto{\pgfqpoint{1.425492in}{1.660517in}}%
\pgfpathlineto{\pgfqpoint{1.428245in}{1.656291in}}%
\pgfpathlineto{\pgfqpoint{1.426283in}{1.655158in}}%
\pgfpathlineto{\pgfqpoint{1.424245in}{1.654054in}}%
\pgfpathlineto{\pgfqpoint{1.422134in}{1.652981in}}%
\pgfpathlineto{\pgfqpoint{1.419950in}{1.651940in}}%
\pgfpathlineto{\pgfqpoint{1.417492in}{1.656324in}}%
\pgfpathlineto{\pgfqpoint{1.415033in}{1.660599in}}%
\pgfpathlineto{\pgfqpoint{1.412575in}{1.664764in}}%
\pgfpathlineto{\pgfqpoint{1.410116in}{1.668818in}}%
\pgfpathlineto{\pgfqpoint{1.411989in}{1.669708in}}%
\pgfpathlineto{\pgfqpoint{1.413801in}{1.670625in}}%
\pgfpathlineto{\pgfqpoint{1.415549in}{1.671569in}}%
\pgfpathlineto{\pgfqpoint{1.417232in}{1.672538in}}%
\pgfpathclose%
\pgfusepath{fill}%
\end{pgfscope}%
\begin{pgfscope}%
\pgfpathrectangle{\pgfqpoint{0.329460in}{0.284240in}}{\pgfqpoint{1.989680in}{1.989680in}}%
\pgfusepath{clip}%
\pgfsetbuttcap%
\pgfsetroundjoin%
\definecolor{currentfill}{rgb}{0.122606,0.585371,0.546557}%
\pgfsetfillcolor{currentfill}%
\pgfsetlinewidth{0.000000pt}%
\definecolor{currentstroke}{rgb}{0.000000,0.000000,0.000000}%
\pgfsetstrokecolor{currentstroke}%
\pgfsetdash{}{0pt}%
\pgfpathmoveto{\pgfqpoint{1.250635in}{1.274323in}}%
\pgfpathlineto{\pgfqpoint{1.249389in}{1.265745in}}%
\pgfpathlineto{\pgfqpoint{1.248143in}{1.257148in}}%
\pgfpathlineto{\pgfqpoint{1.246898in}{1.248533in}}%
\pgfpathlineto{\pgfqpoint{1.245653in}{1.239902in}}%
\pgfpathlineto{\pgfqpoint{1.237015in}{1.241655in}}%
\pgfpathlineto{\pgfqpoint{1.228499in}{1.243544in}}%
\pgfpathlineto{\pgfqpoint{1.220112in}{1.245567in}}%
\pgfpathlineto{\pgfqpoint{1.211865in}{1.247722in}}%
\pgfpathlineto{\pgfqpoint{1.213506in}{1.256249in}}%
\pgfpathlineto{\pgfqpoint{1.215148in}{1.264762in}}%
\pgfpathlineto{\pgfqpoint{1.216790in}{1.273257in}}%
\pgfpathlineto{\pgfqpoint{1.218433in}{1.281733in}}%
\pgfpathlineto{\pgfqpoint{1.226294in}{1.279690in}}%
\pgfpathlineto{\pgfqpoint{1.234287in}{1.277773in}}%
\pgfpathlineto{\pgfqpoint{1.242403in}{1.275983in}}%
\pgfpathlineto{\pgfqpoint{1.250635in}{1.274323in}}%
\pgfpathclose%
\pgfusepath{fill}%
\end{pgfscope}%
\begin{pgfscope}%
\pgfpathrectangle{\pgfqpoint{0.329460in}{0.284240in}}{\pgfqpoint{1.989680in}{1.989680in}}%
\pgfusepath{clip}%
\pgfsetbuttcap%
\pgfsetroundjoin%
\definecolor{currentfill}{rgb}{0.935904,0.898570,0.108131}%
\pgfsetfillcolor{currentfill}%
\pgfsetlinewidth{0.000000pt}%
\definecolor{currentstroke}{rgb}{0.000000,0.000000,0.000000}%
\pgfsetstrokecolor{currentstroke}%
\pgfsetdash{}{0pt}%
\pgfpathmoveto{\pgfqpoint{1.390449in}{1.697166in}}%
\pgfpathlineto{\pgfqpoint{1.392907in}{1.694026in}}%
\pgfpathlineto{\pgfqpoint{1.395365in}{1.690769in}}%
\pgfpathlineto{\pgfqpoint{1.397823in}{1.687397in}}%
\pgfpathlineto{\pgfqpoint{1.400282in}{1.683909in}}%
\pgfpathlineto{\pgfqpoint{1.398671in}{1.683193in}}%
\pgfpathlineto{\pgfqpoint{1.397011in}{1.682501in}}%
\pgfpathlineto{\pgfqpoint{1.395306in}{1.681834in}}%
\pgfpathlineto{\pgfqpoint{1.393556in}{1.681193in}}%
\pgfpathlineto{\pgfqpoint{1.391433in}{1.684818in}}%
\pgfpathlineto{\pgfqpoint{1.389311in}{1.688328in}}%
\pgfpathlineto{\pgfqpoint{1.387189in}{1.691722in}}%
\pgfpathlineto{\pgfqpoint{1.385067in}{1.694999in}}%
\pgfpathlineto{\pgfqpoint{1.386467in}{1.695511in}}%
\pgfpathlineto{\pgfqpoint{1.387832in}{1.696043in}}%
\pgfpathlineto{\pgfqpoint{1.389159in}{1.696595in}}%
\pgfpathlineto{\pgfqpoint{1.390449in}{1.697166in}}%
\pgfpathclose%
\pgfusepath{fill}%
\end{pgfscope}%
\begin{pgfscope}%
\pgfpathrectangle{\pgfqpoint{0.329460in}{0.284240in}}{\pgfqpoint{1.989680in}{1.989680in}}%
\pgfusepath{clip}%
\pgfsetbuttcap%
\pgfsetroundjoin%
\definecolor{currentfill}{rgb}{0.636902,0.856542,0.216620}%
\pgfsetfillcolor{currentfill}%
\pgfsetlinewidth{0.000000pt}%
\definecolor{currentstroke}{rgb}{0.000000,0.000000,0.000000}%
\pgfsetstrokecolor{currentstroke}%
\pgfsetdash{}{0pt}%
\pgfpathmoveto{\pgfqpoint{1.255807in}{1.589850in}}%
\pgfpathlineto{\pgfqpoint{1.253426in}{1.584102in}}%
\pgfpathlineto{\pgfqpoint{1.251045in}{1.578260in}}%
\pgfpathlineto{\pgfqpoint{1.248664in}{1.572325in}}%
\pgfpathlineto{\pgfqpoint{1.246285in}{1.566299in}}%
\pgfpathlineto{\pgfqpoint{1.242767in}{1.567912in}}%
\pgfpathlineto{\pgfqpoint{1.239360in}{1.569577in}}%
\pgfpathlineto{\pgfqpoint{1.236067in}{1.571292in}}%
\pgfpathlineto{\pgfqpoint{1.232892in}{1.573056in}}%
\pgfpathlineto{\pgfqpoint{1.235572in}{1.578922in}}%
\pgfpathlineto{\pgfqpoint{1.238254in}{1.584698in}}%
\pgfpathlineto{\pgfqpoint{1.240936in}{1.590381in}}%
\pgfpathlineto{\pgfqpoint{1.243619in}{1.595970in}}%
\pgfpathlineto{\pgfqpoint{1.246509in}{1.594373in}}%
\pgfpathlineto{\pgfqpoint{1.249506in}{1.592819in}}%
\pgfpathlineto{\pgfqpoint{1.252606in}{1.591311in}}%
\pgfpathlineto{\pgfqpoint{1.255807in}{1.589850in}}%
\pgfpathclose%
\pgfusepath{fill}%
\end{pgfscope}%
\begin{pgfscope}%
\pgfpathrectangle{\pgfqpoint{0.329460in}{0.284240in}}{\pgfqpoint{1.989680in}{1.989680in}}%
\pgfusepath{clip}%
\pgfsetbuttcap%
\pgfsetroundjoin%
\definecolor{currentfill}{rgb}{0.271305,0.019942,0.347269}%
\pgfsetfillcolor{currentfill}%
\pgfsetlinewidth{0.000000pt}%
\definecolor{currentstroke}{rgb}{0.000000,0.000000,0.000000}%
\pgfsetstrokecolor{currentstroke}%
\pgfsetdash{}{0pt}%
\pgfpathmoveto{\pgfqpoint{1.484614in}{0.791954in}}%
\pgfpathlineto{\pgfqpoint{1.485561in}{0.787869in}}%
\pgfpathlineto{\pgfqpoint{1.486509in}{0.783966in}}%
\pgfpathlineto{\pgfqpoint{1.487459in}{0.780251in}}%
\pgfpathlineto{\pgfqpoint{1.488411in}{0.776726in}}%
\pgfpathlineto{\pgfqpoint{1.472522in}{0.774510in}}%
\pgfpathlineto{\pgfqpoint{1.456497in}{0.772568in}}%
\pgfpathlineto{\pgfqpoint{1.440354in}{0.770900in}}%
\pgfpathlineto{\pgfqpoint{1.424112in}{0.769510in}}%
\pgfpathlineto{\pgfqpoint{1.423605in}{0.773089in}}%
\pgfpathlineto{\pgfqpoint{1.423100in}{0.776859in}}%
\pgfpathlineto{\pgfqpoint{1.422596in}{0.780815in}}%
\pgfpathlineto{\pgfqpoint{1.422092in}{0.784955in}}%
\pgfpathlineto{\pgfqpoint{1.437885in}{0.786303in}}%
\pgfpathlineto{\pgfqpoint{1.453582in}{0.787921in}}%
\pgfpathlineto{\pgfqpoint{1.469164in}{0.789805in}}%
\pgfpathlineto{\pgfqpoint{1.484614in}{0.791954in}}%
\pgfpathclose%
\pgfusepath{fill}%
\end{pgfscope}%
\begin{pgfscope}%
\pgfpathrectangle{\pgfqpoint{0.329460in}{0.284240in}}{\pgfqpoint{1.989680in}{1.989680in}}%
\pgfusepath{clip}%
\pgfsetbuttcap%
\pgfsetroundjoin%
\definecolor{currentfill}{rgb}{0.179019,0.433756,0.557430}%
\pgfsetfillcolor{currentfill}%
\pgfsetlinewidth{0.000000pt}%
\definecolor{currentstroke}{rgb}{0.000000,0.000000,0.000000}%
\pgfsetstrokecolor{currentstroke}%
\pgfsetdash{}{0pt}%
\pgfpathmoveto{\pgfqpoint{1.440202in}{1.130789in}}%
\pgfpathlineto{\pgfqpoint{1.441120in}{1.122132in}}%
\pgfpathlineto{\pgfqpoint{1.442038in}{1.113498in}}%
\pgfpathlineto{\pgfqpoint{1.442956in}{1.104887in}}%
\pgfpathlineto{\pgfqpoint{1.443874in}{1.096303in}}%
\pgfpathlineto{\pgfqpoint{1.433133in}{1.094877in}}%
\pgfpathlineto{\pgfqpoint{1.422304in}{1.093628in}}%
\pgfpathlineto{\pgfqpoint{1.411398in}{1.092556in}}%
\pgfpathlineto{\pgfqpoint{1.400427in}{1.091662in}}%
\pgfpathlineto{\pgfqpoint{1.399939in}{1.100298in}}%
\pgfpathlineto{\pgfqpoint{1.399451in}{1.108961in}}%
\pgfpathlineto{\pgfqpoint{1.398963in}{1.117648in}}%
\pgfpathlineto{\pgfqpoint{1.398474in}{1.126356in}}%
\pgfpathlineto{\pgfqpoint{1.409011in}{1.127210in}}%
\pgfpathlineto{\pgfqpoint{1.419485in}{1.128234in}}%
\pgfpathlineto{\pgfqpoint{1.429886in}{1.129427in}}%
\pgfpathlineto{\pgfqpoint{1.440202in}{1.130789in}}%
\pgfpathclose%
\pgfusepath{fill}%
\end{pgfscope}%
\begin{pgfscope}%
\pgfpathrectangle{\pgfqpoint{0.329460in}{0.284240in}}{\pgfqpoint{1.989680in}{1.989680in}}%
\pgfusepath{clip}%
\pgfsetbuttcap%
\pgfsetroundjoin%
\definecolor{currentfill}{rgb}{0.344074,0.780029,0.397381}%
\pgfsetfillcolor{currentfill}%
\pgfsetlinewidth{0.000000pt}%
\definecolor{currentstroke}{rgb}{0.000000,0.000000,0.000000}%
\pgfsetstrokecolor{currentstroke}%
\pgfsetdash{}{0pt}%
\pgfpathmoveto{\pgfqpoint{1.236976in}{1.479776in}}%
\pgfpathlineto{\pgfqpoint{1.234946in}{1.472553in}}%
\pgfpathlineto{\pgfqpoint{1.232916in}{1.465260in}}%
\pgfpathlineto{\pgfqpoint{1.230887in}{1.457900in}}%
\pgfpathlineto{\pgfqpoint{1.228859in}{1.450473in}}%
\pgfpathlineto{\pgfqpoint{1.223522in}{1.452401in}}%
\pgfpathlineto{\pgfqpoint{1.218318in}{1.454409in}}%
\pgfpathlineto{\pgfqpoint{1.213251in}{1.456495in}}%
\pgfpathlineto{\pgfqpoint{1.208326in}{1.458657in}}%
\pgfpathlineto{\pgfqpoint{1.210692in}{1.465939in}}%
\pgfpathlineto{\pgfqpoint{1.213058in}{1.473154in}}%
\pgfpathlineto{\pgfqpoint{1.215425in}{1.480302in}}%
\pgfpathlineto{\pgfqpoint{1.217794in}{1.487380in}}%
\pgfpathlineto{\pgfqpoint{1.222395in}{1.485371in}}%
\pgfpathlineto{\pgfqpoint{1.227128in}{1.483432in}}%
\pgfpathlineto{\pgfqpoint{1.231991in}{1.481566in}}%
\pgfpathlineto{\pgfqpoint{1.236976in}{1.479776in}}%
\pgfpathclose%
\pgfusepath{fill}%
\end{pgfscope}%
\begin{pgfscope}%
\pgfpathrectangle{\pgfqpoint{0.329460in}{0.284240in}}{\pgfqpoint{1.989680in}{1.989680in}}%
\pgfusepath{clip}%
\pgfsetbuttcap%
\pgfsetroundjoin%
\definecolor{currentfill}{rgb}{0.762373,0.876424,0.137064}%
\pgfsetfillcolor{currentfill}%
\pgfsetlinewidth{0.000000pt}%
\definecolor{currentstroke}{rgb}{0.000000,0.000000,0.000000}%
\pgfsetstrokecolor{currentstroke}%
\pgfsetdash{}{0pt}%
\pgfpathmoveto{\pgfqpoint{1.439251in}{1.638315in}}%
\pgfpathlineto{\pgfqpoint{1.442001in}{1.633557in}}%
\pgfpathlineto{\pgfqpoint{1.444751in}{1.628696in}}%
\pgfpathlineto{\pgfqpoint{1.447500in}{1.623733in}}%
\pgfpathlineto{\pgfqpoint{1.450248in}{1.618669in}}%
\pgfpathlineto{\pgfqpoint{1.447731in}{1.617203in}}%
\pgfpathlineto{\pgfqpoint{1.445117in}{1.615774in}}%
\pgfpathlineto{\pgfqpoint{1.442406in}{1.614386in}}%
\pgfpathlineto{\pgfqpoint{1.439603in}{1.613038in}}%
\pgfpathlineto{\pgfqpoint{1.437148in}{1.618263in}}%
\pgfpathlineto{\pgfqpoint{1.434692in}{1.623387in}}%
\pgfpathlineto{\pgfqpoint{1.432236in}{1.628408in}}%
\pgfpathlineto{\pgfqpoint{1.429780in}{1.633325in}}%
\pgfpathlineto{\pgfqpoint{1.432273in}{1.634519in}}%
\pgfpathlineto{\pgfqpoint{1.434685in}{1.635750in}}%
\pgfpathlineto{\pgfqpoint{1.437012in}{1.637015in}}%
\pgfpathlineto{\pgfqpoint{1.439251in}{1.638315in}}%
\pgfpathclose%
\pgfusepath{fill}%
\end{pgfscope}%
\begin{pgfscope}%
\pgfpathrectangle{\pgfqpoint{0.329460in}{0.284240in}}{\pgfqpoint{1.989680in}{1.989680in}}%
\pgfusepath{clip}%
\pgfsetbuttcap%
\pgfsetroundjoin%
\definecolor{currentfill}{rgb}{0.814576,0.883393,0.110347}%
\pgfsetfillcolor{currentfill}%
\pgfsetlinewidth{0.000000pt}%
\definecolor{currentstroke}{rgb}{0.000000,0.000000,0.000000}%
\pgfsetstrokecolor{currentstroke}%
\pgfsetdash{}{0pt}%
\pgfpathmoveto{\pgfqpoint{1.428245in}{1.656291in}}%
\pgfpathlineto{\pgfqpoint{1.430997in}{1.651957in}}%
\pgfpathlineto{\pgfqpoint{1.433749in}{1.647516in}}%
\pgfpathlineto{\pgfqpoint{1.436500in}{1.642968in}}%
\pgfpathlineto{\pgfqpoint{1.439251in}{1.638315in}}%
\pgfpathlineto{\pgfqpoint{1.437012in}{1.637015in}}%
\pgfpathlineto{\pgfqpoint{1.434685in}{1.635750in}}%
\pgfpathlineto{\pgfqpoint{1.432273in}{1.634519in}}%
\pgfpathlineto{\pgfqpoint{1.429780in}{1.633325in}}%
\pgfpathlineto{\pgfqpoint{1.427323in}{1.638138in}}%
\pgfpathlineto{\pgfqpoint{1.424865in}{1.642846in}}%
\pgfpathlineto{\pgfqpoint{1.422408in}{1.647446in}}%
\pgfpathlineto{\pgfqpoint{1.419950in}{1.651940in}}%
\pgfpathlineto{\pgfqpoint{1.422134in}{1.652981in}}%
\pgfpathlineto{\pgfqpoint{1.424245in}{1.654054in}}%
\pgfpathlineto{\pgfqpoint{1.426283in}{1.655158in}}%
\pgfpathlineto{\pgfqpoint{1.428245in}{1.656291in}}%
\pgfpathclose%
\pgfusepath{fill}%
\end{pgfscope}%
\begin{pgfscope}%
\pgfpathrectangle{\pgfqpoint{0.329460in}{0.284240in}}{\pgfqpoint{1.989680in}{1.989680in}}%
\pgfusepath{clip}%
\pgfsetbuttcap%
\pgfsetroundjoin%
\definecolor{currentfill}{rgb}{0.955300,0.901065,0.118128}%
\pgfsetfillcolor{currentfill}%
\pgfsetlinewidth{0.000000pt}%
\definecolor{currentstroke}{rgb}{0.000000,0.000000,0.000000}%
\pgfsetstrokecolor{currentstroke}%
\pgfsetdash{}{0pt}%
\pgfpathmoveto{\pgfqpoint{1.372138in}{1.705562in}}%
\pgfpathlineto{\pgfqpoint{1.373888in}{1.702640in}}%
\pgfpathlineto{\pgfqpoint{1.375638in}{1.699600in}}%
\pgfpathlineto{\pgfqpoint{1.377389in}{1.696442in}}%
\pgfpathlineto{\pgfqpoint{1.379140in}{1.693166in}}%
\pgfpathlineto{\pgfqpoint{1.377584in}{1.692764in}}%
\pgfpathlineto{\pgfqpoint{1.376001in}{1.692385in}}%
\pgfpathlineto{\pgfqpoint{1.374393in}{1.692029in}}%
\pgfpathlineto{\pgfqpoint{1.372762in}{1.691698in}}%
\pgfpathlineto{\pgfqpoint{1.371410in}{1.695067in}}%
\pgfpathlineto{\pgfqpoint{1.370059in}{1.698317in}}%
\pgfpathlineto{\pgfqpoint{1.368708in}{1.701450in}}%
\pgfpathlineto{\pgfqpoint{1.367357in}{1.704464in}}%
\pgfpathlineto{\pgfqpoint{1.368580in}{1.704712in}}%
\pgfpathlineto{\pgfqpoint{1.369785in}{1.704977in}}%
\pgfpathlineto{\pgfqpoint{1.370972in}{1.705261in}}%
\pgfpathlineto{\pgfqpoint{1.372138in}{1.705562in}}%
\pgfpathclose%
\pgfusepath{fill}%
\end{pgfscope}%
\begin{pgfscope}%
\pgfpathrectangle{\pgfqpoint{0.329460in}{0.284240in}}{\pgfqpoint{1.989680in}{1.989680in}}%
\pgfusepath{clip}%
\pgfsetbuttcap%
\pgfsetroundjoin%
\definecolor{currentfill}{rgb}{0.935904,0.898570,0.108131}%
\pgfsetfillcolor{currentfill}%
\pgfsetlinewidth{0.000000pt}%
\definecolor{currentstroke}{rgb}{0.000000,0.000000,0.000000}%
\pgfsetstrokecolor{currentstroke}%
\pgfsetdash{}{0pt}%
\pgfpathmoveto{\pgfqpoint{1.318581in}{1.694562in}}%
\pgfpathlineto{\pgfqpoint{1.316539in}{1.691257in}}%
\pgfpathlineto{\pgfqpoint{1.314496in}{1.687835in}}%
\pgfpathlineto{\pgfqpoint{1.312454in}{1.684298in}}%
\pgfpathlineto{\pgfqpoint{1.310410in}{1.680644in}}%
\pgfpathlineto{\pgfqpoint{1.308622in}{1.681263in}}%
\pgfpathlineto{\pgfqpoint{1.306877in}{1.681907in}}%
\pgfpathlineto{\pgfqpoint{1.305177in}{1.682577in}}%
\pgfpathlineto{\pgfqpoint{1.303523in}{1.683271in}}%
\pgfpathlineto{\pgfqpoint{1.305910in}{1.686791in}}%
\pgfpathlineto{\pgfqpoint{1.308297in}{1.690196in}}%
\pgfpathlineto{\pgfqpoint{1.310684in}{1.693485in}}%
\pgfpathlineto{\pgfqpoint{1.313071in}{1.696657in}}%
\pgfpathlineto{\pgfqpoint{1.314394in}{1.696103in}}%
\pgfpathlineto{\pgfqpoint{1.315755in}{1.695569in}}%
\pgfpathlineto{\pgfqpoint{1.317151in}{1.695055in}}%
\pgfpathlineto{\pgfqpoint{1.318581in}{1.694562in}}%
\pgfpathclose%
\pgfusepath{fill}%
\end{pgfscope}%
\begin{pgfscope}%
\pgfpathrectangle{\pgfqpoint{0.329460in}{0.284240in}}{\pgfqpoint{1.989680in}{1.989680in}}%
\pgfusepath{clip}%
\pgfsetbuttcap%
\pgfsetroundjoin%
\definecolor{currentfill}{rgb}{0.166383,0.690856,0.496502}%
\pgfsetfillcolor{currentfill}%
\pgfsetlinewidth{0.000000pt}%
\definecolor{currentstroke}{rgb}{0.000000,0.000000,0.000000}%
\pgfsetstrokecolor{currentstroke}%
\pgfsetdash{}{0pt}%
\pgfpathmoveto{\pgfqpoint{1.238196in}{1.381058in}}%
\pgfpathlineto{\pgfqpoint{1.236546in}{1.372998in}}%
\pgfpathlineto{\pgfqpoint{1.234896in}{1.364891in}}%
\pgfpathlineto{\pgfqpoint{1.233247in}{1.356740in}}%
\pgfpathlineto{\pgfqpoint{1.231599in}{1.348546in}}%
\pgfpathlineto{\pgfqpoint{1.224640in}{1.350476in}}%
\pgfpathlineto{\pgfqpoint{1.217813in}{1.352513in}}%
\pgfpathlineto{\pgfqpoint{1.211127in}{1.354655in}}%
\pgfpathlineto{\pgfqpoint{1.204587in}{1.356899in}}%
\pgfpathlineto{\pgfqpoint{1.206605in}{1.364967in}}%
\pgfpathlineto{\pgfqpoint{1.208624in}{1.372993in}}%
\pgfpathlineto{\pgfqpoint{1.210644in}{1.380974in}}%
\pgfpathlineto{\pgfqpoint{1.212665in}{1.388909in}}%
\pgfpathlineto{\pgfqpoint{1.218847in}{1.386799in}}%
\pgfpathlineto{\pgfqpoint{1.225167in}{1.384786in}}%
\pgfpathlineto{\pgfqpoint{1.231619in}{1.382871in}}%
\pgfpathlineto{\pgfqpoint{1.238196in}{1.381058in}}%
\pgfpathclose%
\pgfusepath{fill}%
\end{pgfscope}%
\begin{pgfscope}%
\pgfpathrectangle{\pgfqpoint{0.329460in}{0.284240in}}{\pgfqpoint{1.989680in}{1.989680in}}%
\pgfusepath{clip}%
\pgfsetbuttcap%
\pgfsetroundjoin%
\definecolor{currentfill}{rgb}{0.412913,0.803041,0.357269}%
\pgfsetfillcolor{currentfill}%
\pgfsetlinewidth{0.000000pt}%
\definecolor{currentstroke}{rgb}{0.000000,0.000000,0.000000}%
\pgfsetstrokecolor{currentstroke}%
\pgfsetdash{}{0pt}%
\pgfpathmoveto{\pgfqpoint{1.478793in}{1.516671in}}%
\pgfpathlineto{\pgfqpoint{1.481235in}{1.509922in}}%
\pgfpathlineto{\pgfqpoint{1.483677in}{1.503097in}}%
\pgfpathlineto{\pgfqpoint{1.486117in}{1.496197in}}%
\pgfpathlineto{\pgfqpoint{1.488556in}{1.489225in}}%
\pgfpathlineto{\pgfqpoint{1.484077in}{1.487154in}}%
\pgfpathlineto{\pgfqpoint{1.479461in}{1.485152in}}%
\pgfpathlineto{\pgfqpoint{1.474713in}{1.483221in}}%
\pgfpathlineto{\pgfqpoint{1.469837in}{1.481364in}}%
\pgfpathlineto{\pgfqpoint{1.467727in}{1.488485in}}%
\pgfpathlineto{\pgfqpoint{1.465617in}{1.495533in}}%
\pgfpathlineto{\pgfqpoint{1.463506in}{1.502506in}}%
\pgfpathlineto{\pgfqpoint{1.461394in}{1.509403in}}%
\pgfpathlineto{\pgfqpoint{1.465926in}{1.511120in}}%
\pgfpathlineto{\pgfqpoint{1.470339in}{1.512906in}}%
\pgfpathlineto{\pgfqpoint{1.474630in}{1.514757in}}%
\pgfpathlineto{\pgfqpoint{1.478793in}{1.516671in}}%
\pgfpathclose%
\pgfusepath{fill}%
\end{pgfscope}%
\begin{pgfscope}%
\pgfpathrectangle{\pgfqpoint{0.329460in}{0.284240in}}{\pgfqpoint{1.989680in}{1.989680in}}%
\pgfusepath{clip}%
\pgfsetbuttcap%
\pgfsetroundjoin%
\definecolor{currentfill}{rgb}{0.274952,0.037752,0.364543}%
\pgfsetfillcolor{currentfill}%
\pgfsetlinewidth{0.000000pt}%
\definecolor{currentstroke}{rgb}{0.000000,0.000000,0.000000}%
\pgfsetstrokecolor{currentstroke}%
\pgfsetdash{}{0pt}%
\pgfpathmoveto{\pgfqpoint{1.420086in}{0.803259in}}%
\pgfpathlineto{\pgfqpoint{1.420586in}{0.798429in}}%
\pgfpathlineto{\pgfqpoint{1.421087in}{0.793765in}}%
\pgfpathlineto{\pgfqpoint{1.421589in}{0.789273in}}%
\pgfpathlineto{\pgfqpoint{1.422092in}{0.784955in}}%
\pgfpathlineto{\pgfqpoint{1.406219in}{0.783877in}}%
\pgfpathlineto{\pgfqpoint{1.390285in}{0.783071in}}%
\pgfpathlineto{\pgfqpoint{1.374308in}{0.782539in}}%
\pgfpathlineto{\pgfqpoint{1.358304in}{0.782280in}}%
\pgfpathlineto{\pgfqpoint{1.358253in}{0.786619in}}%
\pgfpathlineto{\pgfqpoint{1.358203in}{0.791132in}}%
\pgfpathlineto{\pgfqpoint{1.358153in}{0.795816in}}%
\pgfpathlineto{\pgfqpoint{1.358103in}{0.800667in}}%
\pgfpathlineto{\pgfqpoint{1.373653in}{0.800918in}}%
\pgfpathlineto{\pgfqpoint{1.389179in}{0.801434in}}%
\pgfpathlineto{\pgfqpoint{1.404662in}{0.802215in}}%
\pgfpathlineto{\pgfqpoint{1.420086in}{0.803259in}}%
\pgfpathclose%
\pgfusepath{fill}%
\end{pgfscope}%
\begin{pgfscope}%
\pgfpathrectangle{\pgfqpoint{0.329460in}{0.284240in}}{\pgfqpoint{1.989680in}{1.989680in}}%
\pgfusepath{clip}%
\pgfsetbuttcap%
\pgfsetroundjoin%
\definecolor{currentfill}{rgb}{0.855810,0.888601,0.097452}%
\pgfsetfillcolor{currentfill}%
\pgfsetlinewidth{0.000000pt}%
\definecolor{currentstroke}{rgb}{0.000000,0.000000,0.000000}%
\pgfsetstrokecolor{currentstroke}%
\pgfsetdash{}{0pt}%
\pgfpathmoveto{\pgfqpoint{1.293974in}{1.668051in}}%
\pgfpathlineto{\pgfqpoint{1.291586in}{1.663965in}}%
\pgfpathlineto{\pgfqpoint{1.289199in}{1.659767in}}%
\pgfpathlineto{\pgfqpoint{1.286812in}{1.655459in}}%
\pgfpathlineto{\pgfqpoint{1.284425in}{1.651041in}}%
\pgfpathlineto{\pgfqpoint{1.282179in}{1.652054in}}%
\pgfpathlineto{\pgfqpoint{1.280003in}{1.653099in}}%
\pgfpathlineto{\pgfqpoint{1.277900in}{1.654175in}}%
\pgfpathlineto{\pgfqpoint{1.275870in}{1.655282in}}%
\pgfpathlineto{\pgfqpoint{1.278561in}{1.659545in}}%
\pgfpathlineto{\pgfqpoint{1.281252in}{1.663699in}}%
\pgfpathlineto{\pgfqpoint{1.283944in}{1.667742in}}%
\pgfpathlineto{\pgfqpoint{1.286636in}{1.671675in}}%
\pgfpathlineto{\pgfqpoint{1.288377in}{1.670729in}}%
\pgfpathlineto{\pgfqpoint{1.290181in}{1.669809in}}%
\pgfpathlineto{\pgfqpoint{1.292048in}{1.668916in}}%
\pgfpathlineto{\pgfqpoint{1.293974in}{1.668051in}}%
\pgfpathclose%
\pgfusepath{fill}%
\end{pgfscope}%
\begin{pgfscope}%
\pgfpathrectangle{\pgfqpoint{0.329460in}{0.284240in}}{\pgfqpoint{1.989680in}{1.989680in}}%
\pgfusepath{clip}%
\pgfsetbuttcap%
\pgfsetroundjoin%
\definecolor{currentfill}{rgb}{0.955300,0.901065,0.118128}%
\pgfsetfillcolor{currentfill}%
\pgfsetlinewidth{0.000000pt}%
\definecolor{currentstroke}{rgb}{0.000000,0.000000,0.000000}%
\pgfsetstrokecolor{currentstroke}%
\pgfsetdash{}{0pt}%
\pgfpathmoveto{\pgfqpoint{1.336119in}{1.704258in}}%
\pgfpathlineto{\pgfqpoint{1.334860in}{1.701227in}}%
\pgfpathlineto{\pgfqpoint{1.333601in}{1.698078in}}%
\pgfpathlineto{\pgfqpoint{1.332341in}{1.694810in}}%
\pgfpathlineto{\pgfqpoint{1.331081in}{1.691424in}}%
\pgfpathlineto{\pgfqpoint{1.329431in}{1.691734in}}%
\pgfpathlineto{\pgfqpoint{1.327802in}{1.692068in}}%
\pgfpathlineto{\pgfqpoint{1.326197in}{1.692426in}}%
\pgfpathlineto{\pgfqpoint{1.324617in}{1.692808in}}%
\pgfpathlineto{\pgfqpoint{1.326282in}{1.696106in}}%
\pgfpathlineto{\pgfqpoint{1.327946in}{1.699287in}}%
\pgfpathlineto{\pgfqpoint{1.329609in}{1.702350in}}%
\pgfpathlineto{\pgfqpoint{1.331273in}{1.705294in}}%
\pgfpathlineto{\pgfqpoint{1.332457in}{1.705008in}}%
\pgfpathlineto{\pgfqpoint{1.333660in}{1.704740in}}%
\pgfpathlineto{\pgfqpoint{1.334881in}{1.704490in}}%
\pgfpathlineto{\pgfqpoint{1.336119in}{1.704258in}}%
\pgfpathclose%
\pgfusepath{fill}%
\end{pgfscope}%
\begin{pgfscope}%
\pgfpathrectangle{\pgfqpoint{0.329460in}{0.284240in}}{\pgfqpoint{1.989680in}{1.989680in}}%
\pgfusepath{clip}%
\pgfsetbuttcap%
\pgfsetroundjoin%
\definecolor{currentfill}{rgb}{0.195860,0.395433,0.555276}%
\pgfsetfillcolor{currentfill}%
\pgfsetlinewidth{0.000000pt}%
\definecolor{currentstroke}{rgb}{0.000000,0.000000,0.000000}%
\pgfsetstrokecolor{currentstroke}%
\pgfsetdash{}{0pt}%
\pgfpathmoveto{\pgfqpoint{1.400427in}{1.091662in}}%
\pgfpathlineto{\pgfqpoint{1.400915in}{1.083055in}}%
\pgfpathlineto{\pgfqpoint{1.401403in}{1.074481in}}%
\pgfpathlineto{\pgfqpoint{1.401891in}{1.065943in}}%
\pgfpathlineto{\pgfqpoint{1.402379in}{1.057442in}}%
\pgfpathlineto{\pgfqpoint{1.390917in}{1.056696in}}%
\pgfpathlineto{\pgfqpoint{1.379413in}{1.056138in}}%
\pgfpathlineto{\pgfqpoint{1.367878in}{1.055769in}}%
\pgfpathlineto{\pgfqpoint{1.356325in}{1.055589in}}%
\pgfpathlineto{\pgfqpoint{1.356276in}{1.064110in}}%
\pgfpathlineto{\pgfqpoint{1.356227in}{1.072669in}}%
\pgfpathlineto{\pgfqpoint{1.356178in}{1.081263in}}%
\pgfpathlineto{\pgfqpoint{1.356129in}{1.089889in}}%
\pgfpathlineto{\pgfqpoint{1.367241in}{1.090061in}}%
\pgfpathlineto{\pgfqpoint{1.378336in}{1.090414in}}%
\pgfpathlineto{\pgfqpoint{1.389402in}{1.090948in}}%
\pgfpathlineto{\pgfqpoint{1.400427in}{1.091662in}}%
\pgfpathclose%
\pgfusepath{fill}%
\end{pgfscope}%
\begin{pgfscope}%
\pgfpathrectangle{\pgfqpoint{0.329460in}{0.284240in}}{\pgfqpoint{1.989680in}{1.989680in}}%
\pgfusepath{clip}%
\pgfsetbuttcap%
\pgfsetroundjoin%
\definecolor{currentfill}{rgb}{0.699415,0.867117,0.175971}%
\pgfsetfillcolor{currentfill}%
\pgfsetlinewidth{0.000000pt}%
\definecolor{currentstroke}{rgb}{0.000000,0.000000,0.000000}%
\pgfsetstrokecolor{currentstroke}%
\pgfsetdash{}{0pt}%
\pgfpathmoveto{\pgfqpoint{1.265339in}{1.611875in}}%
\pgfpathlineto{\pgfqpoint{1.262956in}{1.606517in}}%
\pgfpathlineto{\pgfqpoint{1.260572in}{1.601059in}}%
\pgfpathlineto{\pgfqpoint{1.258189in}{1.595503in}}%
\pgfpathlineto{\pgfqpoint{1.255807in}{1.589850in}}%
\pgfpathlineto{\pgfqpoint{1.252606in}{1.591311in}}%
\pgfpathlineto{\pgfqpoint{1.249506in}{1.592819in}}%
\pgfpathlineto{\pgfqpoint{1.246509in}{1.594373in}}%
\pgfpathlineto{\pgfqpoint{1.243619in}{1.595970in}}%
\pgfpathlineto{\pgfqpoint{1.246303in}{1.601465in}}%
\pgfpathlineto{\pgfqpoint{1.248988in}{1.606862in}}%
\pgfpathlineto{\pgfqpoint{1.251673in}{1.612163in}}%
\pgfpathlineto{\pgfqpoint{1.254359in}{1.617364in}}%
\pgfpathlineto{\pgfqpoint{1.256963in}{1.615931in}}%
\pgfpathlineto{\pgfqpoint{1.259663in}{1.614538in}}%
\pgfpathlineto{\pgfqpoint{1.262456in}{1.613185in}}%
\pgfpathlineto{\pgfqpoint{1.265339in}{1.611875in}}%
\pgfpathclose%
\pgfusepath{fill}%
\end{pgfscope}%
\begin{pgfscope}%
\pgfpathrectangle{\pgfqpoint{0.329460in}{0.284240in}}{\pgfqpoint{1.989680in}{1.989680in}}%
\pgfusepath{clip}%
\pgfsetbuttcap%
\pgfsetroundjoin%
\definecolor{currentfill}{rgb}{0.179019,0.433756,0.557430}%
\pgfsetfillcolor{currentfill}%
\pgfsetlinewidth{0.000000pt}%
\definecolor{currentstroke}{rgb}{0.000000,0.000000,0.000000}%
\pgfsetstrokecolor{currentstroke}%
\pgfsetdash{}{0pt}%
\pgfpathmoveto{\pgfqpoint{1.313310in}{1.125741in}}%
\pgfpathlineto{\pgfqpoint{1.312919in}{1.117026in}}%
\pgfpathlineto{\pgfqpoint{1.312528in}{1.108332in}}%
\pgfpathlineto{\pgfqpoint{1.312137in}{1.099662in}}%
\pgfpathlineto{\pgfqpoint{1.311746in}{1.091018in}}%
\pgfpathlineto{\pgfqpoint{1.300727in}{1.091752in}}%
\pgfpathlineto{\pgfqpoint{1.289762in}{1.092666in}}%
\pgfpathlineto{\pgfqpoint{1.278864in}{1.093758in}}%
\pgfpathlineto{\pgfqpoint{1.268044in}{1.095027in}}%
\pgfpathlineto{\pgfqpoint{1.268867in}{1.103625in}}%
\pgfpathlineto{\pgfqpoint{1.269691in}{1.112251in}}%
\pgfpathlineto{\pgfqpoint{1.270515in}{1.120900in}}%
\pgfpathlineto{\pgfqpoint{1.271338in}{1.129570in}}%
\pgfpathlineto{\pgfqpoint{1.281730in}{1.128358in}}%
\pgfpathlineto{\pgfqpoint{1.292197in}{1.127315in}}%
\pgfpathlineto{\pgfqpoint{1.302727in}{1.126442in}}%
\pgfpathlineto{\pgfqpoint{1.313310in}{1.125741in}}%
\pgfpathclose%
\pgfusepath{fill}%
\end{pgfscope}%
\begin{pgfscope}%
\pgfpathrectangle{\pgfqpoint{0.329460in}{0.284240in}}{\pgfqpoint{1.989680in}{1.989680in}}%
\pgfusepath{clip}%
\pgfsetbuttcap%
\pgfsetroundjoin%
\definecolor{currentfill}{rgb}{0.120081,0.622161,0.534946}%
\pgfsetfillcolor{currentfill}%
\pgfsetlinewidth{0.000000pt}%
\definecolor{currentstroke}{rgb}{0.000000,0.000000,0.000000}%
\pgfsetstrokecolor{currentstroke}%
\pgfsetdash{}{0pt}%
\pgfpathmoveto{\pgfqpoint{1.483895in}{1.317204in}}%
\pgfpathlineto{\pgfqpoint{1.485625in}{1.308857in}}%
\pgfpathlineto{\pgfqpoint{1.487355in}{1.300481in}}%
\pgfpathlineto{\pgfqpoint{1.489083in}{1.292078in}}%
\pgfpathlineto{\pgfqpoint{1.490811in}{1.283651in}}%
\pgfpathlineto{\pgfqpoint{1.483075in}{1.281500in}}%
\pgfpathlineto{\pgfqpoint{1.475199in}{1.279471in}}%
\pgfpathlineto{\pgfqpoint{1.467192in}{1.277568in}}%
\pgfpathlineto{\pgfqpoint{1.459063in}{1.275792in}}%
\pgfpathlineto{\pgfqpoint{1.457726in}{1.284328in}}%
\pgfpathlineto{\pgfqpoint{1.456389in}{1.292838in}}%
\pgfpathlineto{\pgfqpoint{1.455051in}{1.301321in}}%
\pgfpathlineto{\pgfqpoint{1.453713in}{1.309775in}}%
\pgfpathlineto{\pgfqpoint{1.461441in}{1.311454in}}%
\pgfpathlineto{\pgfqpoint{1.469052in}{1.313253in}}%
\pgfpathlineto{\pgfqpoint{1.476540in}{1.315170in}}%
\pgfpathlineto{\pgfqpoint{1.483895in}{1.317204in}}%
\pgfpathclose%
\pgfusepath{fill}%
\end{pgfscope}%
\begin{pgfscope}%
\pgfpathrectangle{\pgfqpoint{0.329460in}{0.284240in}}{\pgfqpoint{1.989680in}{1.989680in}}%
\pgfusepath{clip}%
\pgfsetbuttcap%
\pgfsetroundjoin%
\definecolor{currentfill}{rgb}{0.147607,0.511733,0.557049}%
\pgfsetfillcolor{currentfill}%
\pgfsetlinewidth{0.000000pt}%
\definecolor{currentstroke}{rgb}{0.000000,0.000000,0.000000}%
\pgfsetstrokecolor{currentstroke}%
\pgfsetdash{}{0pt}%
\pgfpathmoveto{\pgfqpoint{1.277933in}{1.199373in}}%
\pgfpathlineto{\pgfqpoint{1.277108in}{1.190630in}}%
\pgfpathlineto{\pgfqpoint{1.276283in}{1.181886in}}%
\pgfpathlineto{\pgfqpoint{1.275459in}{1.173146in}}%
\pgfpathlineto{\pgfqpoint{1.274634in}{1.164410in}}%
\pgfpathlineto{\pgfqpoint{1.264753in}{1.165726in}}%
\pgfpathlineto{\pgfqpoint{1.254964in}{1.167200in}}%
\pgfpathlineto{\pgfqpoint{1.245279in}{1.168831in}}%
\pgfpathlineto{\pgfqpoint{1.235706in}{1.170617in}}%
\pgfpathlineto{\pgfqpoint{1.236949in}{1.179278in}}%
\pgfpathlineto{\pgfqpoint{1.238191in}{1.187943in}}%
\pgfpathlineto{\pgfqpoint{1.239434in}{1.196612in}}%
\pgfpathlineto{\pgfqpoint{1.240677in}{1.205280in}}%
\pgfpathlineto{\pgfqpoint{1.249839in}{1.203580in}}%
\pgfpathlineto{\pgfqpoint{1.259108in}{1.202028in}}%
\pgfpathlineto{\pgfqpoint{1.268476in}{1.200625in}}%
\pgfpathlineto{\pgfqpoint{1.277933in}{1.199373in}}%
\pgfpathclose%
\pgfusepath{fill}%
\end{pgfscope}%
\begin{pgfscope}%
\pgfpathrectangle{\pgfqpoint{0.329460in}{0.284240in}}{\pgfqpoint{1.989680in}{1.989680in}}%
\pgfusepath{clip}%
\pgfsetbuttcap%
\pgfsetroundjoin%
\definecolor{currentfill}{rgb}{0.274952,0.037752,0.364543}%
\pgfsetfillcolor{currentfill}%
\pgfsetlinewidth{0.000000pt}%
\definecolor{currentstroke}{rgb}{0.000000,0.000000,0.000000}%
\pgfsetstrokecolor{currentstroke}%
\pgfsetdash{}{0pt}%
\pgfpathmoveto{\pgfqpoint{1.358103in}{0.800667in}}%
\pgfpathlineto{\pgfqpoint{1.358153in}{0.795816in}}%
\pgfpathlineto{\pgfqpoint{1.358203in}{0.791132in}}%
\pgfpathlineto{\pgfqpoint{1.358253in}{0.786619in}}%
\pgfpathlineto{\pgfqpoint{1.358304in}{0.782280in}}%
\pgfpathlineto{\pgfqpoint{1.342292in}{0.782295in}}%
\pgfpathlineto{\pgfqpoint{1.326291in}{0.782585in}}%
\pgfpathlineto{\pgfqpoint{1.310317in}{0.783147in}}%
\pgfpathlineto{\pgfqpoint{1.294389in}{0.783983in}}%
\pgfpathlineto{\pgfqpoint{1.294791in}{0.788309in}}%
\pgfpathlineto{\pgfqpoint{1.295194in}{0.792809in}}%
\pgfpathlineto{\pgfqpoint{1.295595in}{0.797480in}}%
\pgfpathlineto{\pgfqpoint{1.295996in}{0.802318in}}%
\pgfpathlineto{\pgfqpoint{1.311473in}{0.801508in}}%
\pgfpathlineto{\pgfqpoint{1.326995in}{0.800962in}}%
\pgfpathlineto{\pgfqpoint{1.342544in}{0.800682in}}%
\pgfpathlineto{\pgfqpoint{1.358103in}{0.800667in}}%
\pgfpathclose%
\pgfusepath{fill}%
\end{pgfscope}%
\begin{pgfscope}%
\pgfpathrectangle{\pgfqpoint{0.329460in}{0.284240in}}{\pgfqpoint{1.989680in}{1.989680in}}%
\pgfusepath{clip}%
\pgfsetbuttcap%
\pgfsetroundjoin%
\definecolor{currentfill}{rgb}{0.195860,0.395433,0.555276}%
\pgfsetfillcolor{currentfill}%
\pgfsetlinewidth{0.000000pt}%
\definecolor{currentstroke}{rgb}{0.000000,0.000000,0.000000}%
\pgfsetstrokecolor{currentstroke}%
\pgfsetdash{}{0pt}%
\pgfpathmoveto{\pgfqpoint{1.356129in}{1.089889in}}%
\pgfpathlineto{\pgfqpoint{1.356178in}{1.081263in}}%
\pgfpathlineto{\pgfqpoint{1.356227in}{1.072669in}}%
\pgfpathlineto{\pgfqpoint{1.356276in}{1.064110in}}%
\pgfpathlineto{\pgfqpoint{1.356325in}{1.055589in}}%
\pgfpathlineto{\pgfqpoint{1.344766in}{1.055600in}}%
\pgfpathlineto{\pgfqpoint{1.333214in}{1.055800in}}%
\pgfpathlineto{\pgfqpoint{1.321682in}{1.056190in}}%
\pgfpathlineto{\pgfqpoint{1.310182in}{1.056769in}}%
\pgfpathlineto{\pgfqpoint{1.310573in}{1.065277in}}%
\pgfpathlineto{\pgfqpoint{1.310964in}{1.073823in}}%
\pgfpathlineto{\pgfqpoint{1.311355in}{1.082404in}}%
\pgfpathlineto{\pgfqpoint{1.311746in}{1.091018in}}%
\pgfpathlineto{\pgfqpoint{1.322808in}{1.090464in}}%
\pgfpathlineto{\pgfqpoint{1.333900in}{1.090091in}}%
\pgfpathlineto{\pgfqpoint{1.345011in}{1.089900in}}%
\pgfpathlineto{\pgfqpoint{1.356129in}{1.089889in}}%
\pgfpathclose%
\pgfusepath{fill}%
\end{pgfscope}%
\begin{pgfscope}%
\pgfpathrectangle{\pgfqpoint{0.329460in}{0.284240in}}{\pgfqpoint{1.989680in}{1.989680in}}%
\pgfusepath{clip}%
\pgfsetbuttcap%
\pgfsetroundjoin%
\definecolor{currentfill}{rgb}{0.271305,0.019942,0.347269}%
\pgfsetfillcolor{currentfill}%
\pgfsetlinewidth{0.000000pt}%
\definecolor{currentstroke}{rgb}{0.000000,0.000000,0.000000}%
\pgfsetstrokecolor{currentstroke}%
\pgfsetdash{}{0pt}%
\pgfpathmoveto{\pgfqpoint{1.294389in}{0.783983in}}%
\pgfpathlineto{\pgfqpoint{1.293985in}{0.779837in}}%
\pgfpathlineto{\pgfqpoint{1.293581in}{0.775872in}}%
\pgfpathlineto{\pgfqpoint{1.293176in}{0.772095in}}%
\pgfpathlineto{\pgfqpoint{1.292771in}{0.768509in}}%
\pgfpathlineto{\pgfqpoint{1.276454in}{0.769651in}}%
\pgfpathlineto{\pgfqpoint{1.260222in}{0.771072in}}%
\pgfpathlineto{\pgfqpoint{1.244091in}{0.772770in}}%
\pgfpathlineto{\pgfqpoint{1.228081in}{0.774743in}}%
\pgfpathlineto{\pgfqpoint{1.228935in}{0.778283in}}%
\pgfpathlineto{\pgfqpoint{1.229787in}{0.782013in}}%
\pgfpathlineto{\pgfqpoint{1.230638in}{0.785931in}}%
\pgfpathlineto{\pgfqpoint{1.231488in}{0.790031in}}%
\pgfpathlineto{\pgfqpoint{1.247056in}{0.788117in}}%
\pgfpathlineto{\pgfqpoint{1.262741in}{0.786470in}}%
\pgfpathlineto{\pgfqpoint{1.278524in}{0.785091in}}%
\pgfpathlineto{\pgfqpoint{1.294389in}{0.783983in}}%
\pgfpathclose%
\pgfusepath{fill}%
\end{pgfscope}%
\begin{pgfscope}%
\pgfpathrectangle{\pgfqpoint{0.329460in}{0.284240in}}{\pgfqpoint{1.989680in}{1.989680in}}%
\pgfusepath{clip}%
\pgfsetbuttcap%
\pgfsetroundjoin%
\definecolor{currentfill}{rgb}{0.814576,0.883393,0.110347}%
\pgfsetfillcolor{currentfill}%
\pgfsetlinewidth{0.000000pt}%
\definecolor{currentstroke}{rgb}{0.000000,0.000000,0.000000}%
\pgfsetstrokecolor{currentstroke}%
\pgfsetdash{}{0pt}%
\pgfpathmoveto{\pgfqpoint{1.284425in}{1.651041in}}%
\pgfpathlineto{\pgfqpoint{1.282038in}{1.646516in}}%
\pgfpathlineto{\pgfqpoint{1.279652in}{1.641882in}}%
\pgfpathlineto{\pgfqpoint{1.277265in}{1.637142in}}%
\pgfpathlineto{\pgfqpoint{1.274879in}{1.632296in}}%
\pgfpathlineto{\pgfqpoint{1.272314in}{1.633456in}}%
\pgfpathlineto{\pgfqpoint{1.269830in}{1.634654in}}%
\pgfpathlineto{\pgfqpoint{1.267428in}{1.635889in}}%
\pgfpathlineto{\pgfqpoint{1.265110in}{1.637158in}}%
\pgfpathlineto{\pgfqpoint{1.267800in}{1.641848in}}%
\pgfpathlineto{\pgfqpoint{1.270489in}{1.646433in}}%
\pgfpathlineto{\pgfqpoint{1.273180in}{1.650911in}}%
\pgfpathlineto{\pgfqpoint{1.275870in}{1.655282in}}%
\pgfpathlineto{\pgfqpoint{1.277900in}{1.654175in}}%
\pgfpathlineto{\pgfqpoint{1.280003in}{1.653099in}}%
\pgfpathlineto{\pgfqpoint{1.282179in}{1.652054in}}%
\pgfpathlineto{\pgfqpoint{1.284425in}{1.651041in}}%
\pgfpathclose%
\pgfusepath{fill}%
\end{pgfscope}%
\begin{pgfscope}%
\pgfpathrectangle{\pgfqpoint{0.329460in}{0.284240in}}{\pgfqpoint{1.989680in}{1.989680in}}%
\pgfusepath{clip}%
\pgfsetbuttcap%
\pgfsetroundjoin%
\definecolor{currentfill}{rgb}{0.220124,0.725509,0.466226}%
\pgfsetfillcolor{currentfill}%
\pgfsetlinewidth{0.000000pt}%
\definecolor{currentstroke}{rgb}{0.000000,0.000000,0.000000}%
\pgfsetstrokecolor{currentstroke}%
\pgfsetdash{}{0pt}%
\pgfpathmoveto{\pgfqpoint{1.486682in}{1.421975in}}%
\pgfpathlineto{\pgfqpoint{1.488784in}{1.414277in}}%
\pgfpathlineto{\pgfqpoint{1.490885in}{1.406525in}}%
\pgfpathlineto{\pgfqpoint{1.492985in}{1.398720in}}%
\pgfpathlineto{\pgfqpoint{1.495085in}{1.390864in}}%
\pgfpathlineto{\pgfqpoint{1.489031in}{1.388670in}}%
\pgfpathlineto{\pgfqpoint{1.482833in}{1.386571in}}%
\pgfpathlineto{\pgfqpoint{1.476498in}{1.384568in}}%
\pgfpathlineto{\pgfqpoint{1.470031in}{1.382665in}}%
\pgfpathlineto{\pgfqpoint{1.468295in}{1.390651in}}%
\pgfpathlineto{\pgfqpoint{1.466559in}{1.398585in}}%
\pgfpathlineto{\pgfqpoint{1.464821in}{1.406467in}}%
\pgfpathlineto{\pgfqpoint{1.463083in}{1.414294in}}%
\pgfpathlineto{\pgfqpoint{1.469174in}{1.416077in}}%
\pgfpathlineto{\pgfqpoint{1.475141in}{1.417953in}}%
\pgfpathlineto{\pgfqpoint{1.480979in}{1.419920in}}%
\pgfpathlineto{\pgfqpoint{1.486682in}{1.421975in}}%
\pgfpathclose%
\pgfusepath{fill}%
\end{pgfscope}%
\begin{pgfscope}%
\pgfpathrectangle{\pgfqpoint{0.329460in}{0.284240in}}{\pgfqpoint{1.989680in}{1.989680in}}%
\pgfusepath{clip}%
\pgfsetbuttcap%
\pgfsetroundjoin%
\definecolor{currentfill}{rgb}{0.762373,0.876424,0.137064}%
\pgfsetfillcolor{currentfill}%
\pgfsetlinewidth{0.000000pt}%
\definecolor{currentstroke}{rgb}{0.000000,0.000000,0.000000}%
\pgfsetstrokecolor{currentstroke}%
\pgfsetdash{}{0pt}%
\pgfpathmoveto{\pgfqpoint{1.274879in}{1.632296in}}%
\pgfpathlineto{\pgfqpoint{1.272494in}{1.627345in}}%
\pgfpathlineto{\pgfqpoint{1.270109in}{1.622291in}}%
\pgfpathlineto{\pgfqpoint{1.267724in}{1.617134in}}%
\pgfpathlineto{\pgfqpoint{1.265339in}{1.611875in}}%
\pgfpathlineto{\pgfqpoint{1.262456in}{1.613185in}}%
\pgfpathlineto{\pgfqpoint{1.259663in}{1.614538in}}%
\pgfpathlineto{\pgfqpoint{1.256963in}{1.615931in}}%
\pgfpathlineto{\pgfqpoint{1.254359in}{1.617364in}}%
\pgfpathlineto{\pgfqpoint{1.257046in}{1.622465in}}%
\pgfpathlineto{\pgfqpoint{1.259734in}{1.627465in}}%
\pgfpathlineto{\pgfqpoint{1.262422in}{1.632363in}}%
\pgfpathlineto{\pgfqpoint{1.265110in}{1.637158in}}%
\pgfpathlineto{\pgfqpoint{1.267428in}{1.635889in}}%
\pgfpathlineto{\pgfqpoint{1.269830in}{1.634654in}}%
\pgfpathlineto{\pgfqpoint{1.272314in}{1.633456in}}%
\pgfpathlineto{\pgfqpoint{1.274879in}{1.632296in}}%
\pgfpathclose%
\pgfusepath{fill}%
\end{pgfscope}%
\begin{pgfscope}%
\pgfpathrectangle{\pgfqpoint{0.329460in}{0.284240in}}{\pgfqpoint{1.989680in}{1.989680in}}%
\pgfusepath{clip}%
\pgfsetbuttcap%
\pgfsetroundjoin%
\definecolor{currentfill}{rgb}{0.267004,0.004874,0.329415}%
\pgfsetfillcolor{currentfill}%
\pgfsetlinewidth{0.000000pt}%
\definecolor{currentstroke}{rgb}{0.000000,0.000000,0.000000}%
\pgfsetstrokecolor{currentstroke}%
\pgfsetdash{}{0pt}%
\pgfpathmoveto{\pgfqpoint{1.555802in}{0.776493in}}%
\pgfpathlineto{\pgfqpoint{1.557195in}{0.774072in}}%
\pgfpathlineto{\pgfqpoint{1.558590in}{0.771867in}}%
\pgfpathlineto{\pgfqpoint{1.559989in}{0.769883in}}%
\pgfpathlineto{\pgfqpoint{1.561392in}{0.768124in}}%
\pgfpathlineto{\pgfqpoint{1.545376in}{0.764655in}}%
\pgfpathlineto{\pgfqpoint{1.529143in}{0.761460in}}%
\pgfpathlineto{\pgfqpoint{1.512711in}{0.758544in}}%
\pgfpathlineto{\pgfqpoint{1.496097in}{0.755912in}}%
\pgfpathlineto{\pgfqpoint{1.495129in}{0.757755in}}%
\pgfpathlineto{\pgfqpoint{1.494163in}{0.759824in}}%
\pgfpathlineto{\pgfqpoint{1.493199in}{0.762114in}}%
\pgfpathlineto{\pgfqpoint{1.492237in}{0.764620in}}%
\pgfpathlineto{\pgfqpoint{1.508410in}{0.767180in}}%
\pgfpathlineto{\pgfqpoint{1.524407in}{0.770014in}}%
\pgfpathlineto{\pgfqpoint{1.540210in}{0.773120in}}%
\pgfpathlineto{\pgfqpoint{1.555802in}{0.776493in}}%
\pgfpathclose%
\pgfusepath{fill}%
\end{pgfscope}%
\begin{pgfscope}%
\pgfpathrectangle{\pgfqpoint{0.329460in}{0.284240in}}{\pgfqpoint{1.989680in}{1.989680in}}%
\pgfusepath{clip}%
\pgfsetbuttcap%
\pgfsetroundjoin%
\definecolor{currentfill}{rgb}{0.955300,0.901065,0.118128}%
\pgfsetfillcolor{currentfill}%
\pgfsetlinewidth{0.000000pt}%
\definecolor{currentstroke}{rgb}{0.000000,0.000000,0.000000}%
\pgfsetstrokecolor{currentstroke}%
\pgfsetdash{}{0pt}%
\pgfpathmoveto{\pgfqpoint{1.367357in}{1.704464in}}%
\pgfpathlineto{\pgfqpoint{1.368708in}{1.701450in}}%
\pgfpathlineto{\pgfqpoint{1.370059in}{1.698317in}}%
\pgfpathlineto{\pgfqpoint{1.371410in}{1.695067in}}%
\pgfpathlineto{\pgfqpoint{1.372762in}{1.691698in}}%
\pgfpathlineto{\pgfqpoint{1.371109in}{1.691391in}}%
\pgfpathlineto{\pgfqpoint{1.369436in}{1.691108in}}%
\pgfpathlineto{\pgfqpoint{1.367745in}{1.690850in}}%
\pgfpathlineto{\pgfqpoint{1.366036in}{1.690618in}}%
\pgfpathlineto{\pgfqpoint{1.365106in}{1.694055in}}%
\pgfpathlineto{\pgfqpoint{1.364176in}{1.697374in}}%
\pgfpathlineto{\pgfqpoint{1.363246in}{1.700574in}}%
\pgfpathlineto{\pgfqpoint{1.362316in}{1.703656in}}%
\pgfpathlineto{\pgfqpoint{1.363596in}{1.703830in}}%
\pgfpathlineto{\pgfqpoint{1.364864in}{1.704023in}}%
\pgfpathlineto{\pgfqpoint{1.366118in}{1.704234in}}%
\pgfpathlineto{\pgfqpoint{1.367357in}{1.704464in}}%
\pgfpathclose%
\pgfusepath{fill}%
\end{pgfscope}%
\begin{pgfscope}%
\pgfpathrectangle{\pgfqpoint{0.329460in}{0.284240in}}{\pgfqpoint{1.989680in}{1.989680in}}%
\pgfusepath{clip}%
\pgfsetbuttcap%
\pgfsetroundjoin%
\definecolor{currentfill}{rgb}{0.955300,0.901065,0.118128}%
\pgfsetfillcolor{currentfill}%
\pgfsetlinewidth{0.000000pt}%
\definecolor{currentstroke}{rgb}{0.000000,0.000000,0.000000}%
\pgfsetstrokecolor{currentstroke}%
\pgfsetdash{}{0pt}%
\pgfpathmoveto{\pgfqpoint{1.341207in}{1.703517in}}%
\pgfpathlineto{\pgfqpoint{1.340373in}{1.700424in}}%
\pgfpathlineto{\pgfqpoint{1.339539in}{1.697212in}}%
\pgfpathlineto{\pgfqpoint{1.338704in}{1.693881in}}%
\pgfpathlineto{\pgfqpoint{1.337870in}{1.690432in}}%
\pgfpathlineto{\pgfqpoint{1.336148in}{1.690642in}}%
\pgfpathlineto{\pgfqpoint{1.334442in}{1.690878in}}%
\pgfpathlineto{\pgfqpoint{1.332752in}{1.691138in}}%
\pgfpathlineto{\pgfqpoint{1.331081in}{1.691424in}}%
\pgfpathlineto{\pgfqpoint{1.332341in}{1.694810in}}%
\pgfpathlineto{\pgfqpoint{1.333601in}{1.698078in}}%
\pgfpathlineto{\pgfqpoint{1.334860in}{1.701227in}}%
\pgfpathlineto{\pgfqpoint{1.336119in}{1.704258in}}%
\pgfpathlineto{\pgfqpoint{1.337371in}{1.704045in}}%
\pgfpathlineto{\pgfqpoint{1.338637in}{1.703850in}}%
\pgfpathlineto{\pgfqpoint{1.339916in}{1.703674in}}%
\pgfpathlineto{\pgfqpoint{1.341207in}{1.703517in}}%
\pgfpathclose%
\pgfusepath{fill}%
\end{pgfscope}%
\begin{pgfscope}%
\pgfpathrectangle{\pgfqpoint{0.329460in}{0.284240in}}{\pgfqpoint{1.989680in}{1.989680in}}%
\pgfusepath{clip}%
\pgfsetbuttcap%
\pgfsetroundjoin%
\definecolor{currentfill}{rgb}{0.233603,0.313828,0.543914}%
\pgfsetfillcolor{currentfill}%
\pgfsetlinewidth{0.000000pt}%
\definecolor{currentstroke}{rgb}{0.000000,0.000000,0.000000}%
\pgfsetstrokecolor{currentstroke}%
\pgfsetdash{}{0pt}%
\pgfpathmoveto{\pgfqpoint{0.844639in}{0.959677in}}%
\pgfpathlineto{\pgfqpoint{0.841566in}{0.968729in}}%
\pgfpathlineto{\pgfqpoint{0.838477in}{0.978193in}}%
\pgfpathlineto{\pgfqpoint{0.835372in}{0.988076in}}%
\pgfpathlineto{\pgfqpoint{0.832250in}{0.998384in}}%
\pgfpathlineto{\pgfqpoint{0.819400in}{1.007084in}}%
\pgfpathlineto{\pgfqpoint{0.807131in}{1.015979in}}%
\pgfpathlineto{\pgfqpoint{0.795456in}{1.025060in}}%
\pgfpathlineto{\pgfqpoint{0.784384in}{1.034316in}}%
\pgfpathlineto{\pgfqpoint{0.787775in}{1.023849in}}%
\pgfpathlineto{\pgfqpoint{0.791148in}{1.013807in}}%
\pgfpathlineto{\pgfqpoint{0.794504in}{1.004180in}}%
\pgfpathlineto{\pgfqpoint{0.797843in}{0.994964in}}%
\pgfpathlineto{\pgfqpoint{0.808670in}{0.985873in}}%
\pgfpathlineto{\pgfqpoint{0.820086in}{0.976955in}}%
\pgfpathlineto{\pgfqpoint{0.832079in}{0.968219in}}%
\pgfpathlineto{\pgfqpoint{0.844639in}{0.959677in}}%
\pgfpathclose%
\pgfusepath{fill}%
\end{pgfscope}%
\begin{pgfscope}%
\pgfpathrectangle{\pgfqpoint{0.329460in}{0.284240in}}{\pgfqpoint{1.989680in}{1.989680in}}%
\pgfusepath{clip}%
\pgfsetbuttcap%
\pgfsetroundjoin%
\definecolor{currentfill}{rgb}{0.896320,0.893616,0.096335}%
\pgfsetfillcolor{currentfill}%
\pgfsetlinewidth{0.000000pt}%
\definecolor{currentstroke}{rgb}{0.000000,0.000000,0.000000}%
\pgfsetstrokecolor{currentstroke}%
\pgfsetdash{}{0pt}%
\pgfpathmoveto{\pgfqpoint{1.400282in}{1.683909in}}%
\pgfpathlineto{\pgfqpoint{1.402740in}{1.680307in}}%
\pgfpathlineto{\pgfqpoint{1.405199in}{1.676590in}}%
\pgfpathlineto{\pgfqpoint{1.407658in}{1.672761in}}%
\pgfpathlineto{\pgfqpoint{1.410116in}{1.668818in}}%
\pgfpathlineto{\pgfqpoint{1.408184in}{1.667957in}}%
\pgfpathlineto{\pgfqpoint{1.406194in}{1.667124in}}%
\pgfpathlineto{\pgfqpoint{1.404148in}{1.666321in}}%
\pgfpathlineto{\pgfqpoint{1.402048in}{1.665549in}}%
\pgfpathlineto{\pgfqpoint{1.399925in}{1.669630in}}%
\pgfpathlineto{\pgfqpoint{1.397802in}{1.673598in}}%
\pgfpathlineto{\pgfqpoint{1.395679in}{1.677452in}}%
\pgfpathlineto{\pgfqpoint{1.393556in}{1.681193in}}%
\pgfpathlineto{\pgfqpoint{1.395306in}{1.681834in}}%
\pgfpathlineto{\pgfqpoint{1.397011in}{1.682501in}}%
\pgfpathlineto{\pgfqpoint{1.398671in}{1.683193in}}%
\pgfpathlineto{\pgfqpoint{1.400282in}{1.683909in}}%
\pgfpathclose%
\pgfusepath{fill}%
\end{pgfscope}%
\begin{pgfscope}%
\pgfpathrectangle{\pgfqpoint{0.329460in}{0.284240in}}{\pgfqpoint{1.989680in}{1.989680in}}%
\pgfusepath{clip}%
\pgfsetbuttcap%
\pgfsetroundjoin%
\definecolor{currentfill}{rgb}{0.412913,0.803041,0.357269}%
\pgfsetfillcolor{currentfill}%
\pgfsetlinewidth{0.000000pt}%
\definecolor{currentstroke}{rgb}{0.000000,0.000000,0.000000}%
\pgfsetstrokecolor{currentstroke}%
\pgfsetdash{}{0pt}%
\pgfpathmoveto{\pgfqpoint{1.245105in}{1.507935in}}%
\pgfpathlineto{\pgfqpoint{1.243072in}{1.501008in}}%
\pgfpathlineto{\pgfqpoint{1.241039in}{1.494005in}}%
\pgfpathlineto{\pgfqpoint{1.239007in}{1.486927in}}%
\pgfpathlineto{\pgfqpoint{1.236976in}{1.479776in}}%
\pgfpathlineto{\pgfqpoint{1.231991in}{1.481566in}}%
\pgfpathlineto{\pgfqpoint{1.227128in}{1.483432in}}%
\pgfpathlineto{\pgfqpoint{1.222395in}{1.485371in}}%
\pgfpathlineto{\pgfqpoint{1.217794in}{1.487380in}}%
\pgfpathlineto{\pgfqpoint{1.220163in}{1.494388in}}%
\pgfpathlineto{\pgfqpoint{1.222533in}{1.501322in}}%
\pgfpathlineto{\pgfqpoint{1.224904in}{1.508182in}}%
\pgfpathlineto{\pgfqpoint{1.227277in}{1.514966in}}%
\pgfpathlineto{\pgfqpoint{1.231553in}{1.513108in}}%
\pgfpathlineto{\pgfqpoint{1.235953in}{1.511316in}}%
\pgfpathlineto{\pgfqpoint{1.240472in}{1.509591in}}%
\pgfpathlineto{\pgfqpoint{1.245105in}{1.507935in}}%
\pgfpathclose%
\pgfusepath{fill}%
\end{pgfscope}%
\begin{pgfscope}%
\pgfpathrectangle{\pgfqpoint{0.329460in}{0.284240in}}{\pgfqpoint{1.989680in}{1.989680in}}%
\pgfusepath{clip}%
\pgfsetbuttcap%
\pgfsetroundjoin%
\definecolor{currentfill}{rgb}{0.935904,0.898570,0.108131}%
\pgfsetfillcolor{currentfill}%
\pgfsetlinewidth{0.000000pt}%
\definecolor{currentstroke}{rgb}{0.000000,0.000000,0.000000}%
\pgfsetstrokecolor{currentstroke}%
\pgfsetdash{}{0pt}%
\pgfpathmoveto{\pgfqpoint{1.385067in}{1.694999in}}%
\pgfpathlineto{\pgfqpoint{1.387189in}{1.691722in}}%
\pgfpathlineto{\pgfqpoint{1.389311in}{1.688328in}}%
\pgfpathlineto{\pgfqpoint{1.391433in}{1.684818in}}%
\pgfpathlineto{\pgfqpoint{1.393556in}{1.681193in}}%
\pgfpathlineto{\pgfqpoint{1.391763in}{1.680577in}}%
\pgfpathlineto{\pgfqpoint{1.389930in}{1.679989in}}%
\pgfpathlineto{\pgfqpoint{1.388057in}{1.679428in}}%
\pgfpathlineto{\pgfqpoint{1.386146in}{1.678895in}}%
\pgfpathlineto{\pgfqpoint{1.384394in}{1.682637in}}%
\pgfpathlineto{\pgfqpoint{1.382643in}{1.686263in}}%
\pgfpathlineto{\pgfqpoint{1.380891in}{1.689773in}}%
\pgfpathlineto{\pgfqpoint{1.379140in}{1.693166in}}%
\pgfpathlineto{\pgfqpoint{1.380668in}{1.693591in}}%
\pgfpathlineto{\pgfqpoint{1.382166in}{1.694039in}}%
\pgfpathlineto{\pgfqpoint{1.383633in}{1.694508in}}%
\pgfpathlineto{\pgfqpoint{1.385067in}{1.694999in}}%
\pgfpathclose%
\pgfusepath{fill}%
\end{pgfscope}%
\begin{pgfscope}%
\pgfpathrectangle{\pgfqpoint{0.329460in}{0.284240in}}{\pgfqpoint{1.989680in}{1.989680in}}%
\pgfusepath{clip}%
\pgfsetbuttcap%
\pgfsetroundjoin%
\definecolor{currentfill}{rgb}{0.133743,0.548535,0.553541}%
\pgfsetfillcolor{currentfill}%
\pgfsetlinewidth{0.000000pt}%
\definecolor{currentstroke}{rgb}{0.000000,0.000000,0.000000}%
\pgfsetstrokecolor{currentstroke}%
\pgfsetdash{}{0pt}%
\pgfpathmoveto{\pgfqpoint{1.464406in}{1.241453in}}%
\pgfpathlineto{\pgfqpoint{1.465741in}{1.232831in}}%
\pgfpathlineto{\pgfqpoint{1.467075in}{1.224199in}}%
\pgfpathlineto{\pgfqpoint{1.468409in}{1.215559in}}%
\pgfpathlineto{\pgfqpoint{1.469743in}{1.206913in}}%
\pgfpathlineto{\pgfqpoint{1.460685in}{1.205084in}}%
\pgfpathlineto{\pgfqpoint{1.451511in}{1.203401in}}%
\pgfpathlineto{\pgfqpoint{1.442230in}{1.201865in}}%
\pgfpathlineto{\pgfqpoint{1.432852in}{1.200478in}}%
\pgfpathlineto{\pgfqpoint{1.431933in}{1.209205in}}%
\pgfpathlineto{\pgfqpoint{1.431013in}{1.217926in}}%
\pgfpathlineto{\pgfqpoint{1.430093in}{1.226639in}}%
\pgfpathlineto{\pgfqpoint{1.429172in}{1.235342in}}%
\pgfpathlineto{\pgfqpoint{1.438129in}{1.236659in}}%
\pgfpathlineto{\pgfqpoint{1.446993in}{1.238117in}}%
\pgfpathlineto{\pgfqpoint{1.455755in}{1.239716in}}%
\pgfpathlineto{\pgfqpoint{1.464406in}{1.241453in}}%
\pgfpathclose%
\pgfusepath{fill}%
\end{pgfscope}%
\begin{pgfscope}%
\pgfpathrectangle{\pgfqpoint{0.329460in}{0.284240in}}{\pgfqpoint{1.989680in}{1.989680in}}%
\pgfusepath{clip}%
\pgfsetbuttcap%
\pgfsetroundjoin%
\definecolor{currentfill}{rgb}{0.487026,0.823929,0.312321}%
\pgfsetfillcolor{currentfill}%
\pgfsetlinewidth{0.000000pt}%
\definecolor{currentstroke}{rgb}{0.000000,0.000000,0.000000}%
\pgfsetstrokecolor{currentstroke}%
\pgfsetdash{}{0pt}%
\pgfpathmoveto{\pgfqpoint{1.469015in}{1.542874in}}%
\pgfpathlineto{\pgfqpoint{1.471461in}{1.536445in}}%
\pgfpathlineto{\pgfqpoint{1.473906in}{1.529935in}}%
\pgfpathlineto{\pgfqpoint{1.476350in}{1.523343in}}%
\pgfpathlineto{\pgfqpoint{1.478793in}{1.516671in}}%
\pgfpathlineto{\pgfqpoint{1.474630in}{1.514757in}}%
\pgfpathlineto{\pgfqpoint{1.470339in}{1.512906in}}%
\pgfpathlineto{\pgfqpoint{1.465926in}{1.511120in}}%
\pgfpathlineto{\pgfqpoint{1.461394in}{1.509403in}}%
\pgfpathlineto{\pgfqpoint{1.459281in}{1.516222in}}%
\pgfpathlineto{\pgfqpoint{1.457168in}{1.522960in}}%
\pgfpathlineto{\pgfqpoint{1.455054in}{1.529618in}}%
\pgfpathlineto{\pgfqpoint{1.452940in}{1.536192in}}%
\pgfpathlineto{\pgfqpoint{1.457126in}{1.537771in}}%
\pgfpathlineto{\pgfqpoint{1.461203in}{1.539412in}}%
\pgfpathlineto{\pgfqpoint{1.465168in}{1.541113in}}%
\pgfpathlineto{\pgfqpoint{1.469015in}{1.542874in}}%
\pgfpathclose%
\pgfusepath{fill}%
\end{pgfscope}%
\begin{pgfscope}%
\pgfpathrectangle{\pgfqpoint{0.329460in}{0.284240in}}{\pgfqpoint{1.989680in}{1.989680in}}%
\pgfusepath{clip}%
\pgfsetbuttcap%
\pgfsetroundjoin%
\definecolor{currentfill}{rgb}{0.955300,0.901065,0.118128}%
\pgfsetfillcolor{currentfill}%
\pgfsetlinewidth{0.000000pt}%
\definecolor{currentstroke}{rgb}{0.000000,0.000000,0.000000}%
\pgfsetstrokecolor{currentstroke}%
\pgfsetdash{}{0pt}%
\pgfpathmoveto{\pgfqpoint{1.362316in}{1.703656in}}%
\pgfpathlineto{\pgfqpoint{1.363246in}{1.700574in}}%
\pgfpathlineto{\pgfqpoint{1.364176in}{1.697374in}}%
\pgfpathlineto{\pgfqpoint{1.365106in}{1.694055in}}%
\pgfpathlineto{\pgfqpoint{1.366036in}{1.690618in}}%
\pgfpathlineto{\pgfqpoint{1.364313in}{1.690411in}}%
\pgfpathlineto{\pgfqpoint{1.362577in}{1.690229in}}%
\pgfpathlineto{\pgfqpoint{1.360829in}{1.690073in}}%
\pgfpathlineto{\pgfqpoint{1.359071in}{1.689944in}}%
\pgfpathlineto{\pgfqpoint{1.358577in}{1.693423in}}%
\pgfpathlineto{\pgfqpoint{1.358083in}{1.696785in}}%
\pgfpathlineto{\pgfqpoint{1.357589in}{1.700028in}}%
\pgfpathlineto{\pgfqpoint{1.357096in}{1.703152in}}%
\pgfpathlineto{\pgfqpoint{1.358413in}{1.703249in}}%
\pgfpathlineto{\pgfqpoint{1.359723in}{1.703365in}}%
\pgfpathlineto{\pgfqpoint{1.361024in}{1.703501in}}%
\pgfpathlineto{\pgfqpoint{1.362316in}{1.703656in}}%
\pgfpathclose%
\pgfusepath{fill}%
\end{pgfscope}%
\begin{pgfscope}%
\pgfpathrectangle{\pgfqpoint{0.329460in}{0.284240in}}{\pgfqpoint{1.989680in}{1.989680in}}%
\pgfusepath{clip}%
\pgfsetbuttcap%
\pgfsetroundjoin%
\definecolor{currentfill}{rgb}{0.163625,0.471133,0.558148}%
\pgfsetfillcolor{currentfill}%
\pgfsetlinewidth{0.000000pt}%
\definecolor{currentstroke}{rgb}{0.000000,0.000000,0.000000}%
\pgfsetstrokecolor{currentstroke}%
\pgfsetdash{}{0pt}%
\pgfpathmoveto{\pgfqpoint{1.436529in}{1.165572in}}%
\pgfpathlineto{\pgfqpoint{1.437447in}{1.156858in}}%
\pgfpathlineto{\pgfqpoint{1.438366in}{1.148154in}}%
\pgfpathlineto{\pgfqpoint{1.439284in}{1.139463in}}%
\pgfpathlineto{\pgfqpoint{1.440202in}{1.130789in}}%
\pgfpathlineto{\pgfqpoint{1.429886in}{1.129427in}}%
\pgfpathlineto{\pgfqpoint{1.419485in}{1.128234in}}%
\pgfpathlineto{\pgfqpoint{1.409011in}{1.127210in}}%
\pgfpathlineto{\pgfqpoint{1.398474in}{1.126356in}}%
\pgfpathlineto{\pgfqpoint{1.397986in}{1.135082in}}%
\pgfpathlineto{\pgfqpoint{1.397498in}{1.143824in}}%
\pgfpathlineto{\pgfqpoint{1.397010in}{1.152580in}}%
\pgfpathlineto{\pgfqpoint{1.396522in}{1.161345in}}%
\pgfpathlineto{\pgfqpoint{1.406623in}{1.162159in}}%
\pgfpathlineto{\pgfqpoint{1.416666in}{1.163135in}}%
\pgfpathlineto{\pgfqpoint{1.426637in}{1.164273in}}%
\pgfpathlineto{\pgfqpoint{1.436529in}{1.165572in}}%
\pgfpathclose%
\pgfusepath{fill}%
\end{pgfscope}%
\begin{pgfscope}%
\pgfpathrectangle{\pgfqpoint{0.329460in}{0.284240in}}{\pgfqpoint{1.989680in}{1.989680in}}%
\pgfusepath{clip}%
\pgfsetbuttcap%
\pgfsetroundjoin%
\definecolor{currentfill}{rgb}{0.955300,0.901065,0.118128}%
\pgfsetfillcolor{currentfill}%
\pgfsetlinewidth{0.000000pt}%
\definecolor{currentstroke}{rgb}{0.000000,0.000000,0.000000}%
\pgfsetstrokecolor{currentstroke}%
\pgfsetdash{}{0pt}%
\pgfpathmoveto{\pgfqpoint{1.346455in}{1.703082in}}%
\pgfpathlineto{\pgfqpoint{1.346060in}{1.699952in}}%
\pgfpathlineto{\pgfqpoint{1.345665in}{1.696703in}}%
\pgfpathlineto{\pgfqpoint{1.345269in}{1.693336in}}%
\pgfpathlineto{\pgfqpoint{1.344873in}{1.689850in}}%
\pgfpathlineto{\pgfqpoint{1.343108in}{1.689957in}}%
\pgfpathlineto{\pgfqpoint{1.341352in}{1.690089in}}%
\pgfpathlineto{\pgfqpoint{1.339605in}{1.690248in}}%
\pgfpathlineto{\pgfqpoint{1.337870in}{1.690432in}}%
\pgfpathlineto{\pgfqpoint{1.338704in}{1.693881in}}%
\pgfpathlineto{\pgfqpoint{1.339539in}{1.697212in}}%
\pgfpathlineto{\pgfqpoint{1.340373in}{1.700424in}}%
\pgfpathlineto{\pgfqpoint{1.341207in}{1.703517in}}%
\pgfpathlineto{\pgfqpoint{1.342507in}{1.703379in}}%
\pgfpathlineto{\pgfqpoint{1.343816in}{1.703261in}}%
\pgfpathlineto{\pgfqpoint{1.345133in}{1.703162in}}%
\pgfpathlineto{\pgfqpoint{1.346455in}{1.703082in}}%
\pgfpathclose%
\pgfusepath{fill}%
\end{pgfscope}%
\begin{pgfscope}%
\pgfpathrectangle{\pgfqpoint{0.329460in}{0.284240in}}{\pgfqpoint{1.989680in}{1.989680in}}%
\pgfusepath{clip}%
\pgfsetbuttcap%
\pgfsetroundjoin%
\definecolor{currentfill}{rgb}{0.120081,0.622161,0.534946}%
\pgfsetfillcolor{currentfill}%
\pgfsetlinewidth{0.000000pt}%
\definecolor{currentstroke}{rgb}{0.000000,0.000000,0.000000}%
\pgfsetstrokecolor{currentstroke}%
\pgfsetdash{}{0pt}%
\pgfpathmoveto{\pgfqpoint{1.255623in}{1.308386in}}%
\pgfpathlineto{\pgfqpoint{1.254375in}{1.299912in}}%
\pgfpathlineto{\pgfqpoint{1.253128in}{1.291408in}}%
\pgfpathlineto{\pgfqpoint{1.251881in}{1.282878in}}%
\pgfpathlineto{\pgfqpoint{1.250635in}{1.274323in}}%
\pgfpathlineto{\pgfqpoint{1.242403in}{1.275983in}}%
\pgfpathlineto{\pgfqpoint{1.234287in}{1.277773in}}%
\pgfpathlineto{\pgfqpoint{1.226294in}{1.279690in}}%
\pgfpathlineto{\pgfqpoint{1.218433in}{1.281733in}}%
\pgfpathlineto{\pgfqpoint{1.220077in}{1.290186in}}%
\pgfpathlineto{\pgfqpoint{1.221721in}{1.298615in}}%
\pgfpathlineto{\pgfqpoint{1.223366in}{1.307017in}}%
\pgfpathlineto{\pgfqpoint{1.225011in}{1.315390in}}%
\pgfpathlineto{\pgfqpoint{1.232484in}{1.313460in}}%
\pgfpathlineto{\pgfqpoint{1.240083in}{1.311648in}}%
\pgfpathlineto{\pgfqpoint{1.247798in}{1.309956in}}%
\pgfpathlineto{\pgfqpoint{1.255623in}{1.308386in}}%
\pgfpathclose%
\pgfusepath{fill}%
\end{pgfscope}%
\begin{pgfscope}%
\pgfpathrectangle{\pgfqpoint{0.329460in}{0.284240in}}{\pgfqpoint{1.989680in}{1.989680in}}%
\pgfusepath{clip}%
\pgfsetbuttcap%
\pgfsetroundjoin%
\definecolor{currentfill}{rgb}{0.268510,0.009605,0.335427}%
\pgfsetfillcolor{currentfill}%
\pgfsetlinewidth{0.000000pt}%
\definecolor{currentstroke}{rgb}{0.000000,0.000000,0.000000}%
\pgfsetstrokecolor{currentstroke}%
\pgfsetdash{}{0pt}%
\pgfpathmoveto{\pgfqpoint{1.630210in}{0.780429in}}%
\pgfpathlineto{\pgfqpoint{1.632042in}{0.779980in}}%
\pgfpathlineto{\pgfqpoint{1.633878in}{0.779783in}}%
\pgfpathlineto{\pgfqpoint{1.635720in}{0.779844in}}%
\pgfpathlineto{\pgfqpoint{1.637568in}{0.780168in}}%
\pgfpathlineto{\pgfqpoint{1.621798in}{0.775397in}}%
\pgfpathlineto{\pgfqpoint{1.605725in}{0.770895in}}%
\pgfpathlineto{\pgfqpoint{1.589368in}{0.766669in}}%
\pgfpathlineto{\pgfqpoint{1.572742in}{0.762724in}}%
\pgfpathlineto{\pgfqpoint{1.571309in}{0.762512in}}%
\pgfpathlineto{\pgfqpoint{1.569881in}{0.762563in}}%
\pgfpathlineto{\pgfqpoint{1.568456in}{0.762873in}}%
\pgfpathlineto{\pgfqpoint{1.567036in}{0.763435in}}%
\pgfpathlineto{\pgfqpoint{1.583236in}{0.767279in}}%
\pgfpathlineto{\pgfqpoint{1.599177in}{0.771396in}}%
\pgfpathlineto{\pgfqpoint{1.614841in}{0.775781in}}%
\pgfpathlineto{\pgfqpoint{1.630210in}{0.780429in}}%
\pgfpathclose%
\pgfusepath{fill}%
\end{pgfscope}%
\begin{pgfscope}%
\pgfpathrectangle{\pgfqpoint{0.329460in}{0.284240in}}{\pgfqpoint{1.989680in}{1.989680in}}%
\pgfusepath{clip}%
\pgfsetbuttcap%
\pgfsetroundjoin%
\definecolor{currentfill}{rgb}{0.935904,0.898570,0.108131}%
\pgfsetfillcolor{currentfill}%
\pgfsetlinewidth{0.000000pt}%
\definecolor{currentstroke}{rgb}{0.000000,0.000000,0.000000}%
\pgfsetstrokecolor{currentstroke}%
\pgfsetdash{}{0pt}%
\pgfpathmoveto{\pgfqpoint{1.324617in}{1.692808in}}%
\pgfpathlineto{\pgfqpoint{1.322952in}{1.689392in}}%
\pgfpathlineto{\pgfqpoint{1.321287in}{1.685859in}}%
\pgfpathlineto{\pgfqpoint{1.319622in}{1.682210in}}%
\pgfpathlineto{\pgfqpoint{1.317956in}{1.678445in}}%
\pgfpathlineto{\pgfqpoint{1.316015in}{1.678953in}}%
\pgfpathlineto{\pgfqpoint{1.314108in}{1.679489in}}%
\pgfpathlineto{\pgfqpoint{1.312240in}{1.680053in}}%
\pgfpathlineto{\pgfqpoint{1.310410in}{1.680644in}}%
\pgfpathlineto{\pgfqpoint{1.312454in}{1.684298in}}%
\pgfpathlineto{\pgfqpoint{1.314496in}{1.687835in}}%
\pgfpathlineto{\pgfqpoint{1.316539in}{1.691257in}}%
\pgfpathlineto{\pgfqpoint{1.318581in}{1.694562in}}%
\pgfpathlineto{\pgfqpoint{1.320045in}{1.694090in}}%
\pgfpathlineto{\pgfqpoint{1.321540in}{1.693640in}}%
\pgfpathlineto{\pgfqpoint{1.323064in}{1.693212in}}%
\pgfpathlineto{\pgfqpoint{1.324617in}{1.692808in}}%
\pgfpathclose%
\pgfusepath{fill}%
\end{pgfscope}%
\begin{pgfscope}%
\pgfpathrectangle{\pgfqpoint{0.329460in}{0.284240in}}{\pgfqpoint{1.989680in}{1.989680in}}%
\pgfusepath{clip}%
\pgfsetbuttcap%
\pgfsetroundjoin%
\definecolor{currentfill}{rgb}{0.896320,0.893616,0.096335}%
\pgfsetfillcolor{currentfill}%
\pgfsetlinewidth{0.000000pt}%
\definecolor{currentstroke}{rgb}{0.000000,0.000000,0.000000}%
\pgfsetstrokecolor{currentstroke}%
\pgfsetdash{}{0pt}%
\pgfpathmoveto{\pgfqpoint{1.310410in}{1.680644in}}%
\pgfpathlineto{\pgfqpoint{1.308367in}{1.676876in}}%
\pgfpathlineto{\pgfqpoint{1.306324in}{1.672994in}}%
\pgfpathlineto{\pgfqpoint{1.304280in}{1.668998in}}%
\pgfpathlineto{\pgfqpoint{1.302236in}{1.664889in}}%
\pgfpathlineto{\pgfqpoint{1.300091in}{1.665633in}}%
\pgfpathlineto{\pgfqpoint{1.297997in}{1.666409in}}%
\pgfpathlineto{\pgfqpoint{1.295958in}{1.667215in}}%
\pgfpathlineto{\pgfqpoint{1.293974in}{1.668051in}}%
\pgfpathlineto{\pgfqpoint{1.296361in}{1.672026in}}%
\pgfpathlineto{\pgfqpoint{1.298748in}{1.675888in}}%
\pgfpathlineto{\pgfqpoint{1.301136in}{1.679637in}}%
\pgfpathlineto{\pgfqpoint{1.303523in}{1.683271in}}%
\pgfpathlineto{\pgfqpoint{1.305177in}{1.682577in}}%
\pgfpathlineto{\pgfqpoint{1.306877in}{1.681907in}}%
\pgfpathlineto{\pgfqpoint{1.308622in}{1.681263in}}%
\pgfpathlineto{\pgfqpoint{1.310410in}{1.680644in}}%
\pgfpathclose%
\pgfusepath{fill}%
\end{pgfscope}%
\begin{pgfscope}%
\pgfpathrectangle{\pgfqpoint{0.329460in}{0.284240in}}{\pgfqpoint{1.989680in}{1.989680in}}%
\pgfusepath{clip}%
\pgfsetbuttcap%
\pgfsetroundjoin%
\definecolor{currentfill}{rgb}{0.955300,0.901065,0.118128}%
\pgfsetfillcolor{currentfill}%
\pgfsetlinewidth{0.000000pt}%
\definecolor{currentstroke}{rgb}{0.000000,0.000000,0.000000}%
\pgfsetstrokecolor{currentstroke}%
\pgfsetdash{}{0pt}%
\pgfpathmoveto{\pgfqpoint{1.357096in}{1.703152in}}%
\pgfpathlineto{\pgfqpoint{1.357589in}{1.700028in}}%
\pgfpathlineto{\pgfqpoint{1.358083in}{1.696785in}}%
\pgfpathlineto{\pgfqpoint{1.358577in}{1.693423in}}%
\pgfpathlineto{\pgfqpoint{1.359071in}{1.689944in}}%
\pgfpathlineto{\pgfqpoint{1.357305in}{1.689840in}}%
\pgfpathlineto{\pgfqpoint{1.355534in}{1.689762in}}%
\pgfpathlineto{\pgfqpoint{1.353757in}{1.689711in}}%
\pgfpathlineto{\pgfqpoint{1.351979in}{1.689686in}}%
\pgfpathlineto{\pgfqpoint{1.351929in}{1.693182in}}%
\pgfpathlineto{\pgfqpoint{1.351879in}{1.696560in}}%
\pgfpathlineto{\pgfqpoint{1.351830in}{1.699819in}}%
\pgfpathlineto{\pgfqpoint{1.351780in}{1.702959in}}%
\pgfpathlineto{\pgfqpoint{1.353113in}{1.702978in}}%
\pgfpathlineto{\pgfqpoint{1.354445in}{1.703016in}}%
\pgfpathlineto{\pgfqpoint{1.355772in}{1.703074in}}%
\pgfpathlineto{\pgfqpoint{1.357096in}{1.703152in}}%
\pgfpathclose%
\pgfusepath{fill}%
\end{pgfscope}%
\begin{pgfscope}%
\pgfpathrectangle{\pgfqpoint{0.329460in}{0.284240in}}{\pgfqpoint{1.989680in}{1.989680in}}%
\pgfusepath{clip}%
\pgfsetbuttcap%
\pgfsetroundjoin%
\definecolor{currentfill}{rgb}{0.955300,0.901065,0.118128}%
\pgfsetfillcolor{currentfill}%
\pgfsetlinewidth{0.000000pt}%
\definecolor{currentstroke}{rgb}{0.000000,0.000000,0.000000}%
\pgfsetstrokecolor{currentstroke}%
\pgfsetdash{}{0pt}%
\pgfpathmoveto{\pgfqpoint{1.351780in}{1.702959in}}%
\pgfpathlineto{\pgfqpoint{1.351830in}{1.699819in}}%
\pgfpathlineto{\pgfqpoint{1.351879in}{1.696560in}}%
\pgfpathlineto{\pgfqpoint{1.351929in}{1.693182in}}%
\pgfpathlineto{\pgfqpoint{1.351979in}{1.689686in}}%
\pgfpathlineto{\pgfqpoint{1.350199in}{1.689688in}}%
\pgfpathlineto{\pgfqpoint{1.348420in}{1.689716in}}%
\pgfpathlineto{\pgfqpoint{1.346644in}{1.689770in}}%
\pgfpathlineto{\pgfqpoint{1.344873in}{1.689850in}}%
\pgfpathlineto{\pgfqpoint{1.345269in}{1.693336in}}%
\pgfpathlineto{\pgfqpoint{1.345665in}{1.696703in}}%
\pgfpathlineto{\pgfqpoint{1.346060in}{1.699952in}}%
\pgfpathlineto{\pgfqpoint{1.346455in}{1.703082in}}%
\pgfpathlineto{\pgfqpoint{1.347783in}{1.703022in}}%
\pgfpathlineto{\pgfqpoint{1.349114in}{1.702981in}}%
\pgfpathlineto{\pgfqpoint{1.350447in}{1.702960in}}%
\pgfpathlineto{\pgfqpoint{1.351780in}{1.702959in}}%
\pgfpathclose%
\pgfusepath{fill}%
\end{pgfscope}%
\begin{pgfscope}%
\pgfpathrectangle{\pgfqpoint{0.329460in}{0.284240in}}{\pgfqpoint{1.989680in}{1.989680in}}%
\pgfusepath{clip}%
\pgfsetbuttcap%
\pgfsetroundjoin%
\definecolor{currentfill}{rgb}{0.179019,0.433756,0.557430}%
\pgfsetfillcolor{currentfill}%
\pgfsetlinewidth{0.000000pt}%
\definecolor{currentstroke}{rgb}{0.000000,0.000000,0.000000}%
\pgfsetstrokecolor{currentstroke}%
\pgfsetdash{}{0pt}%
\pgfpathmoveto{\pgfqpoint{1.398474in}{1.126356in}}%
\pgfpathlineto{\pgfqpoint{1.398963in}{1.117648in}}%
\pgfpathlineto{\pgfqpoint{1.399451in}{1.108961in}}%
\pgfpathlineto{\pgfqpoint{1.399939in}{1.100298in}}%
\pgfpathlineto{\pgfqpoint{1.400427in}{1.091662in}}%
\pgfpathlineto{\pgfqpoint{1.389402in}{1.090948in}}%
\pgfpathlineto{\pgfqpoint{1.378336in}{1.090414in}}%
\pgfpathlineto{\pgfqpoint{1.367241in}{1.090061in}}%
\pgfpathlineto{\pgfqpoint{1.356129in}{1.089889in}}%
\pgfpathlineto{\pgfqpoint{1.356080in}{1.098545in}}%
\pgfpathlineto{\pgfqpoint{1.356031in}{1.107228in}}%
\pgfpathlineto{\pgfqpoint{1.355982in}{1.115935in}}%
\pgfpathlineto{\pgfqpoint{1.355933in}{1.124663in}}%
\pgfpathlineto{\pgfqpoint{1.366605in}{1.124827in}}%
\pgfpathlineto{\pgfqpoint{1.377260in}{1.125164in}}%
\pgfpathlineto{\pgfqpoint{1.387887in}{1.125674in}}%
\pgfpathlineto{\pgfqpoint{1.398474in}{1.126356in}}%
\pgfpathclose%
\pgfusepath{fill}%
\end{pgfscope}%
\begin{pgfscope}%
\pgfpathrectangle{\pgfqpoint{0.329460in}{0.284240in}}{\pgfqpoint{1.989680in}{1.989680in}}%
\pgfusepath{clip}%
\pgfsetbuttcap%
\pgfsetroundjoin%
\definecolor{currentfill}{rgb}{0.220124,0.725509,0.466226}%
\pgfsetfillcolor{currentfill}%
\pgfsetlinewidth{0.000000pt}%
\definecolor{currentstroke}{rgb}{0.000000,0.000000,0.000000}%
\pgfsetstrokecolor{currentstroke}%
\pgfsetdash{}{0pt}%
\pgfpathmoveto{\pgfqpoint{1.244804in}{1.412789in}}%
\pgfpathlineto{\pgfqpoint{1.243151in}{1.404937in}}%
\pgfpathlineto{\pgfqpoint{1.241499in}{1.397029in}}%
\pgfpathlineto{\pgfqpoint{1.239847in}{1.389069in}}%
\pgfpathlineto{\pgfqpoint{1.238196in}{1.381058in}}%
\pgfpathlineto{\pgfqpoint{1.231619in}{1.382871in}}%
\pgfpathlineto{\pgfqpoint{1.225167in}{1.384786in}}%
\pgfpathlineto{\pgfqpoint{1.218847in}{1.386799in}}%
\pgfpathlineto{\pgfqpoint{1.212665in}{1.388909in}}%
\pgfpathlineto{\pgfqpoint{1.214686in}{1.396796in}}%
\pgfpathlineto{\pgfqpoint{1.216708in}{1.404632in}}%
\pgfpathlineto{\pgfqpoint{1.218731in}{1.412415in}}%
\pgfpathlineto{\pgfqpoint{1.220755in}{1.420144in}}%
\pgfpathlineto{\pgfqpoint{1.226579in}{1.418167in}}%
\pgfpathlineto{\pgfqpoint{1.232532in}{1.416281in}}%
\pgfpathlineto{\pgfqpoint{1.238609in}{1.414488in}}%
\pgfpathlineto{\pgfqpoint{1.244804in}{1.412789in}}%
\pgfpathclose%
\pgfusepath{fill}%
\end{pgfscope}%
\begin{pgfscope}%
\pgfpathrectangle{\pgfqpoint{0.329460in}{0.284240in}}{\pgfqpoint{1.989680in}{1.989680in}}%
\pgfusepath{clip}%
\pgfsetbuttcap%
\pgfsetroundjoin%
\definecolor{currentfill}{rgb}{0.163625,0.471133,0.558148}%
\pgfsetfillcolor{currentfill}%
\pgfsetlinewidth{0.000000pt}%
\definecolor{currentstroke}{rgb}{0.000000,0.000000,0.000000}%
\pgfsetstrokecolor{currentstroke}%
\pgfsetdash{}{0pt}%
\pgfpathmoveto{\pgfqpoint{1.314874in}{1.160759in}}%
\pgfpathlineto{\pgfqpoint{1.314483in}{1.151986in}}%
\pgfpathlineto{\pgfqpoint{1.314092in}{1.143224in}}%
\pgfpathlineto{\pgfqpoint{1.313701in}{1.134475in}}%
\pgfpathlineto{\pgfqpoint{1.313310in}{1.125741in}}%
\pgfpathlineto{\pgfqpoint{1.302727in}{1.126442in}}%
\pgfpathlineto{\pgfqpoint{1.292197in}{1.127315in}}%
\pgfpathlineto{\pgfqpoint{1.281730in}{1.128358in}}%
\pgfpathlineto{\pgfqpoint{1.271338in}{1.129570in}}%
\pgfpathlineto{\pgfqpoint{1.272162in}{1.138259in}}%
\pgfpathlineto{\pgfqpoint{1.272986in}{1.146964in}}%
\pgfpathlineto{\pgfqpoint{1.273810in}{1.155682in}}%
\pgfpathlineto{\pgfqpoint{1.274634in}{1.164410in}}%
\pgfpathlineto{\pgfqpoint{1.284598in}{1.163254in}}%
\pgfpathlineto{\pgfqpoint{1.294633in}{1.162259in}}%
\pgfpathlineto{\pgfqpoint{1.304729in}{1.161427in}}%
\pgfpathlineto{\pgfqpoint{1.314874in}{1.160759in}}%
\pgfpathclose%
\pgfusepath{fill}%
\end{pgfscope}%
\begin{pgfscope}%
\pgfpathrectangle{\pgfqpoint{0.329460in}{0.284240in}}{\pgfqpoint{1.989680in}{1.989680in}}%
\pgfusepath{clip}%
\pgfsetbuttcap%
\pgfsetroundjoin%
\definecolor{currentfill}{rgb}{0.179019,0.433756,0.557430}%
\pgfsetfillcolor{currentfill}%
\pgfsetlinewidth{0.000000pt}%
\definecolor{currentstroke}{rgb}{0.000000,0.000000,0.000000}%
\pgfsetstrokecolor{currentstroke}%
\pgfsetdash{}{0pt}%
\pgfpathmoveto{\pgfqpoint{1.355933in}{1.124663in}}%
\pgfpathlineto{\pgfqpoint{1.355982in}{1.115935in}}%
\pgfpathlineto{\pgfqpoint{1.356031in}{1.107228in}}%
\pgfpathlineto{\pgfqpoint{1.356080in}{1.098545in}}%
\pgfpathlineto{\pgfqpoint{1.356129in}{1.089889in}}%
\pgfpathlineto{\pgfqpoint{1.345011in}{1.089900in}}%
\pgfpathlineto{\pgfqpoint{1.333900in}{1.090091in}}%
\pgfpathlineto{\pgfqpoint{1.322808in}{1.090464in}}%
\pgfpathlineto{\pgfqpoint{1.311746in}{1.091018in}}%
\pgfpathlineto{\pgfqpoint{1.312137in}{1.099662in}}%
\pgfpathlineto{\pgfqpoint{1.312528in}{1.108332in}}%
\pgfpathlineto{\pgfqpoint{1.312919in}{1.117026in}}%
\pgfpathlineto{\pgfqpoint{1.313310in}{1.125741in}}%
\pgfpathlineto{\pgfqpoint{1.323933in}{1.125212in}}%
\pgfpathlineto{\pgfqpoint{1.334585in}{1.124856in}}%
\pgfpathlineto{\pgfqpoint{1.345256in}{1.124672in}}%
\pgfpathlineto{\pgfqpoint{1.355933in}{1.124663in}}%
\pgfpathclose%
\pgfusepath{fill}%
\end{pgfscope}%
\begin{pgfscope}%
\pgfpathrectangle{\pgfqpoint{0.329460in}{0.284240in}}{\pgfqpoint{1.989680in}{1.989680in}}%
\pgfusepath{clip}%
\pgfsetbuttcap%
\pgfsetroundjoin%
\definecolor{currentfill}{rgb}{0.133743,0.548535,0.553541}%
\pgfsetfillcolor{currentfill}%
\pgfsetlinewidth{0.000000pt}%
\definecolor{currentstroke}{rgb}{0.000000,0.000000,0.000000}%
\pgfsetstrokecolor{currentstroke}%
\pgfsetdash{}{0pt}%
\pgfpathmoveto{\pgfqpoint{1.281234in}{1.234292in}}%
\pgfpathlineto{\pgfqpoint{1.280409in}{1.225575in}}%
\pgfpathlineto{\pgfqpoint{1.279583in}{1.216848in}}%
\pgfpathlineto{\pgfqpoint{1.278758in}{1.208113in}}%
\pgfpathlineto{\pgfqpoint{1.277933in}{1.199373in}}%
\pgfpathlineto{\pgfqpoint{1.268476in}{1.200625in}}%
\pgfpathlineto{\pgfqpoint{1.259108in}{1.202028in}}%
\pgfpathlineto{\pgfqpoint{1.249839in}{1.203580in}}%
\pgfpathlineto{\pgfqpoint{1.240677in}{1.205280in}}%
\pgfpathlineto{\pgfqpoint{1.241921in}{1.213946in}}%
\pgfpathlineto{\pgfqpoint{1.243165in}{1.222607in}}%
\pgfpathlineto{\pgfqpoint{1.244409in}{1.231260in}}%
\pgfpathlineto{\pgfqpoint{1.245653in}{1.239902in}}%
\pgfpathlineto{\pgfqpoint{1.254403in}{1.238288in}}%
\pgfpathlineto{\pgfqpoint{1.263257in}{1.236814in}}%
\pgfpathlineto{\pgfqpoint{1.272203in}{1.235481in}}%
\pgfpathlineto{\pgfqpoint{1.281234in}{1.234292in}}%
\pgfpathclose%
\pgfusepath{fill}%
\end{pgfscope}%
\begin{pgfscope}%
\pgfpathrectangle{\pgfqpoint{0.329460in}{0.284240in}}{\pgfqpoint{1.989680in}{1.989680in}}%
\pgfusepath{clip}%
\pgfsetbuttcap%
\pgfsetroundjoin%
\definecolor{currentfill}{rgb}{0.134692,0.658636,0.517649}%
\pgfsetfillcolor{currentfill}%
\pgfsetlinewidth{0.000000pt}%
\definecolor{currentstroke}{rgb}{0.000000,0.000000,0.000000}%
\pgfsetstrokecolor{currentstroke}%
\pgfsetdash{}{0pt}%
\pgfpathmoveto{\pgfqpoint{1.476969in}{1.350256in}}%
\pgfpathlineto{\pgfqpoint{1.478701in}{1.342048in}}%
\pgfpathlineto{\pgfqpoint{1.480433in}{1.333802in}}%
\pgfpathlineto{\pgfqpoint{1.482164in}{1.325520in}}%
\pgfpathlineto{\pgfqpoint{1.483895in}{1.317204in}}%
\pgfpathlineto{\pgfqpoint{1.476540in}{1.315170in}}%
\pgfpathlineto{\pgfqpoint{1.469052in}{1.313253in}}%
\pgfpathlineto{\pgfqpoint{1.461441in}{1.311454in}}%
\pgfpathlineto{\pgfqpoint{1.453713in}{1.309775in}}%
\pgfpathlineto{\pgfqpoint{1.452374in}{1.318198in}}%
\pgfpathlineto{\pgfqpoint{1.451035in}{1.326586in}}%
\pgfpathlineto{\pgfqpoint{1.449696in}{1.334939in}}%
\pgfpathlineto{\pgfqpoint{1.448355in}{1.343253in}}%
\pgfpathlineto{\pgfqpoint{1.455681in}{1.344835in}}%
\pgfpathlineto{\pgfqpoint{1.462897in}{1.346531in}}%
\pgfpathlineto{\pgfqpoint{1.469995in}{1.348339in}}%
\pgfpathlineto{\pgfqpoint{1.476969in}{1.350256in}}%
\pgfpathclose%
\pgfusepath{fill}%
\end{pgfscope}%
\begin{pgfscope}%
\pgfpathrectangle{\pgfqpoint{0.329460in}{0.284240in}}{\pgfqpoint{1.989680in}{1.989680in}}%
\pgfusepath{clip}%
\pgfsetbuttcap%
\pgfsetroundjoin%
\definecolor{currentfill}{rgb}{0.855810,0.888601,0.097452}%
\pgfsetfillcolor{currentfill}%
\pgfsetlinewidth{0.000000pt}%
\definecolor{currentstroke}{rgb}{0.000000,0.000000,0.000000}%
\pgfsetstrokecolor{currentstroke}%
\pgfsetdash{}{0pt}%
\pgfpathmoveto{\pgfqpoint{1.410116in}{1.668818in}}%
\pgfpathlineto{\pgfqpoint{1.412575in}{1.664764in}}%
\pgfpathlineto{\pgfqpoint{1.415033in}{1.660599in}}%
\pgfpathlineto{\pgfqpoint{1.417492in}{1.656324in}}%
\pgfpathlineto{\pgfqpoint{1.419950in}{1.651940in}}%
\pgfpathlineto{\pgfqpoint{1.417696in}{1.650931in}}%
\pgfpathlineto{\pgfqpoint{1.415376in}{1.649957in}}%
\pgfpathlineto{\pgfqpoint{1.412990in}{1.649017in}}%
\pgfpathlineto{\pgfqpoint{1.410541in}{1.648113in}}%
\pgfpathlineto{\pgfqpoint{1.408418in}{1.652638in}}%
\pgfpathlineto{\pgfqpoint{1.406295in}{1.657052in}}%
\pgfpathlineto{\pgfqpoint{1.404171in}{1.661356in}}%
\pgfpathlineto{\pgfqpoint{1.402048in}{1.665549in}}%
\pgfpathlineto{\pgfqpoint{1.404148in}{1.666321in}}%
\pgfpathlineto{\pgfqpoint{1.406194in}{1.667124in}}%
\pgfpathlineto{\pgfqpoint{1.408184in}{1.667957in}}%
\pgfpathlineto{\pgfqpoint{1.410116in}{1.668818in}}%
\pgfpathclose%
\pgfusepath{fill}%
\end{pgfscope}%
\begin{pgfscope}%
\pgfpathrectangle{\pgfqpoint{0.329460in}{0.284240in}}{\pgfqpoint{1.989680in}{1.989680in}}%
\pgfusepath{clip}%
\pgfsetbuttcap%
\pgfsetroundjoin%
\definecolor{currentfill}{rgb}{0.281477,0.755203,0.432552}%
\pgfsetfillcolor{currentfill}%
\pgfsetlinewidth{0.000000pt}%
\definecolor{currentstroke}{rgb}{0.000000,0.000000,0.000000}%
\pgfsetstrokecolor{currentstroke}%
\pgfsetdash{}{0pt}%
\pgfpathmoveto{\pgfqpoint{1.478266in}{1.452183in}}%
\pgfpathlineto{\pgfqpoint{1.480372in}{1.444722in}}%
\pgfpathlineto{\pgfqpoint{1.482476in}{1.437200in}}%
\pgfpathlineto{\pgfqpoint{1.484580in}{1.429617in}}%
\pgfpathlineto{\pgfqpoint{1.486682in}{1.421975in}}%
\pgfpathlineto{\pgfqpoint{1.480979in}{1.419920in}}%
\pgfpathlineto{\pgfqpoint{1.475141in}{1.417953in}}%
\pgfpathlineto{\pgfqpoint{1.469174in}{1.416077in}}%
\pgfpathlineto{\pgfqpoint{1.463083in}{1.414294in}}%
\pgfpathlineto{\pgfqpoint{1.461345in}{1.422065in}}%
\pgfpathlineto{\pgfqpoint{1.459605in}{1.429776in}}%
\pgfpathlineto{\pgfqpoint{1.457865in}{1.437427in}}%
\pgfpathlineto{\pgfqpoint{1.456125in}{1.445015in}}%
\pgfpathlineto{\pgfqpoint{1.461838in}{1.446679in}}%
\pgfpathlineto{\pgfqpoint{1.467437in}{1.448429in}}%
\pgfpathlineto{\pgfqpoint{1.472915in}{1.450264in}}%
\pgfpathlineto{\pgfqpoint{1.478266in}{1.452183in}}%
\pgfpathclose%
\pgfusepath{fill}%
\end{pgfscope}%
\begin{pgfscope}%
\pgfpathrectangle{\pgfqpoint{0.329460in}{0.284240in}}{\pgfqpoint{1.989680in}{1.989680in}}%
\pgfusepath{clip}%
\pgfsetbuttcap%
\pgfsetroundjoin%
\definecolor{currentfill}{rgb}{0.267004,0.004874,0.329415}%
\pgfsetfillcolor{currentfill}%
\pgfsetlinewidth{0.000000pt}%
\definecolor{currentstroke}{rgb}{0.000000,0.000000,0.000000}%
\pgfsetstrokecolor{currentstroke}%
\pgfsetdash{}{0pt}%
\pgfpathmoveto{\pgfqpoint{1.224647in}{0.762579in}}%
\pgfpathlineto{\pgfqpoint{1.223784in}{0.760057in}}%
\pgfpathlineto{\pgfqpoint{1.222919in}{0.757753in}}%
\pgfpathlineto{\pgfqpoint{1.222052in}{0.755669in}}%
\pgfpathlineto{\pgfqpoint{1.221183in}{0.753811in}}%
\pgfpathlineto{\pgfqpoint{1.204424in}{0.756190in}}%
\pgfpathlineto{\pgfqpoint{1.187829in}{0.758854in}}%
\pgfpathlineto{\pgfqpoint{1.171418in}{0.761801in}}%
\pgfpathlineto{\pgfqpoint{1.155208in}{0.765027in}}%
\pgfpathlineto{\pgfqpoint{1.156516in}{0.766807in}}%
\pgfpathlineto{\pgfqpoint{1.157821in}{0.768812in}}%
\pgfpathlineto{\pgfqpoint{1.159123in}{0.771038in}}%
\pgfpathlineto{\pgfqpoint{1.160422in}{0.773481in}}%
\pgfpathlineto{\pgfqpoint{1.176202in}{0.770346in}}%
\pgfpathlineto{\pgfqpoint{1.192179in}{0.767481in}}%
\pgfpathlineto{\pgfqpoint{1.208333in}{0.764891in}}%
\pgfpathlineto{\pgfqpoint{1.224647in}{0.762579in}}%
\pgfpathclose%
\pgfusepath{fill}%
\end{pgfscope}%
\begin{pgfscope}%
\pgfpathrectangle{\pgfqpoint{0.329460in}{0.284240in}}{\pgfqpoint{1.989680in}{1.989680in}}%
\pgfusepath{clip}%
\pgfsetbuttcap%
\pgfsetroundjoin%
\definecolor{currentfill}{rgb}{0.565498,0.842430,0.262877}%
\pgfsetfillcolor{currentfill}%
\pgfsetlinewidth{0.000000pt}%
\definecolor{currentstroke}{rgb}{0.000000,0.000000,0.000000}%
\pgfsetstrokecolor{currentstroke}%
\pgfsetdash{}{0pt}%
\pgfpathmoveto{\pgfqpoint{1.459223in}{1.567730in}}%
\pgfpathlineto{\pgfqpoint{1.461672in}{1.561647in}}%
\pgfpathlineto{\pgfqpoint{1.464121in}{1.555476in}}%
\pgfpathlineto{\pgfqpoint{1.466568in}{1.549218in}}%
\pgfpathlineto{\pgfqpoint{1.469015in}{1.542874in}}%
\pgfpathlineto{\pgfqpoint{1.465168in}{1.541113in}}%
\pgfpathlineto{\pgfqpoint{1.461203in}{1.539412in}}%
\pgfpathlineto{\pgfqpoint{1.457126in}{1.537771in}}%
\pgfpathlineto{\pgfqpoint{1.452940in}{1.536192in}}%
\pgfpathlineto{\pgfqpoint{1.450824in}{1.542682in}}%
\pgfpathlineto{\pgfqpoint{1.448708in}{1.549086in}}%
\pgfpathlineto{\pgfqpoint{1.446592in}{1.555402in}}%
\pgfpathlineto{\pgfqpoint{1.444475in}{1.561630in}}%
\pgfpathlineto{\pgfqpoint{1.448315in}{1.563071in}}%
\pgfpathlineto{\pgfqpoint{1.452056in}{1.564569in}}%
\pgfpathlineto{\pgfqpoint{1.455693in}{1.566123in}}%
\pgfpathlineto{\pgfqpoint{1.459223in}{1.567730in}}%
\pgfpathclose%
\pgfusepath{fill}%
\end{pgfscope}%
\begin{pgfscope}%
\pgfpathrectangle{\pgfqpoint{0.329460in}{0.284240in}}{\pgfqpoint{1.989680in}{1.989680in}}%
\pgfusepath{clip}%
\pgfsetbuttcap%
\pgfsetroundjoin%
\definecolor{currentfill}{rgb}{0.487026,0.823929,0.312321}%
\pgfsetfillcolor{currentfill}%
\pgfsetlinewidth{0.000000pt}%
\definecolor{currentstroke}{rgb}{0.000000,0.000000,0.000000}%
\pgfsetstrokecolor{currentstroke}%
\pgfsetdash{}{0pt}%
\pgfpathmoveto{\pgfqpoint{1.253245in}{1.534843in}}%
\pgfpathlineto{\pgfqpoint{1.251209in}{1.528239in}}%
\pgfpathlineto{\pgfqpoint{1.249173in}{1.521552in}}%
\pgfpathlineto{\pgfqpoint{1.247139in}{1.514783in}}%
\pgfpathlineto{\pgfqpoint{1.245105in}{1.507935in}}%
\pgfpathlineto{\pgfqpoint{1.240472in}{1.509591in}}%
\pgfpathlineto{\pgfqpoint{1.235953in}{1.511316in}}%
\pgfpathlineto{\pgfqpoint{1.231553in}{1.513108in}}%
\pgfpathlineto{\pgfqpoint{1.227277in}{1.514966in}}%
\pgfpathlineto{\pgfqpoint{1.229650in}{1.521672in}}%
\pgfpathlineto{\pgfqpoint{1.232024in}{1.528298in}}%
\pgfpathlineto{\pgfqpoint{1.234398in}{1.534843in}}%
\pgfpathlineto{\pgfqpoint{1.236774in}{1.541306in}}%
\pgfpathlineto{\pgfqpoint{1.240726in}{1.539598in}}%
\pgfpathlineto{\pgfqpoint{1.244791in}{1.537950in}}%
\pgfpathlineto{\pgfqpoint{1.248965in}{1.536364in}}%
\pgfpathlineto{\pgfqpoint{1.253245in}{1.534843in}}%
\pgfpathclose%
\pgfusepath{fill}%
\end{pgfscope}%
\begin{pgfscope}%
\pgfpathrectangle{\pgfqpoint{0.329460in}{0.284240in}}{\pgfqpoint{1.989680in}{1.989680in}}%
\pgfusepath{clip}%
\pgfsetbuttcap%
\pgfsetroundjoin%
\definecolor{currentfill}{rgb}{0.277941,0.056324,0.381191}%
\pgfsetfillcolor{currentfill}%
\pgfsetlinewidth{0.000000pt}%
\definecolor{currentstroke}{rgb}{0.000000,0.000000,0.000000}%
\pgfsetstrokecolor{currentstroke}%
\pgfsetdash{}{0pt}%
\pgfpathmoveto{\pgfqpoint{1.706257in}{0.806395in}}%
\pgfpathlineto{\pgfqpoint{1.708517in}{0.808243in}}%
\pgfpathlineto{\pgfqpoint{1.710786in}{0.810383in}}%
\pgfpathlineto{\pgfqpoint{1.713062in}{0.812822in}}%
\pgfpathlineto{\pgfqpoint{1.715346in}{0.815564in}}%
\pgfpathlineto{\pgfqpoint{1.700217in}{0.809486in}}%
\pgfpathlineto{\pgfqpoint{1.684698in}{0.803665in}}%
\pgfpathlineto{\pgfqpoint{1.668808in}{0.798107in}}%
\pgfpathlineto{\pgfqpoint{1.652561in}{0.792819in}}%
\pgfpathlineto{\pgfqpoint{1.650664in}{0.790211in}}%
\pgfpathlineto{\pgfqpoint{1.648775in}{0.787907in}}%
\pgfpathlineto{\pgfqpoint{1.646891in}{0.785901in}}%
\pgfpathlineto{\pgfqpoint{1.645014in}{0.784189in}}%
\pgfpathlineto{\pgfqpoint{1.660860in}{0.789351in}}%
\pgfpathlineto{\pgfqpoint{1.676360in}{0.794778in}}%
\pgfpathlineto{\pgfqpoint{1.691498in}{0.800461in}}%
\pgfpathlineto{\pgfqpoint{1.706257in}{0.806395in}}%
\pgfpathclose%
\pgfusepath{fill}%
\end{pgfscope}%
\begin{pgfscope}%
\pgfpathrectangle{\pgfqpoint{0.329460in}{0.284240in}}{\pgfqpoint{1.989680in}{1.989680in}}%
\pgfusepath{clip}%
\pgfsetbuttcap%
\pgfsetroundjoin%
\definecolor{currentfill}{rgb}{0.935904,0.898570,0.108131}%
\pgfsetfillcolor{currentfill}%
\pgfsetlinewidth{0.000000pt}%
\definecolor{currentstroke}{rgb}{0.000000,0.000000,0.000000}%
\pgfsetstrokecolor{currentstroke}%
\pgfsetdash{}{0pt}%
\pgfpathmoveto{\pgfqpoint{1.379140in}{1.693166in}}%
\pgfpathlineto{\pgfqpoint{1.380891in}{1.689773in}}%
\pgfpathlineto{\pgfqpoint{1.382643in}{1.686263in}}%
\pgfpathlineto{\pgfqpoint{1.384394in}{1.682637in}}%
\pgfpathlineto{\pgfqpoint{1.386146in}{1.678895in}}%
\pgfpathlineto{\pgfqpoint{1.384201in}{1.678391in}}%
\pgfpathlineto{\pgfqpoint{1.382222in}{1.677915in}}%
\pgfpathlineto{\pgfqpoint{1.380212in}{1.677470in}}%
\pgfpathlineto{\pgfqpoint{1.378172in}{1.677054in}}%
\pgfpathlineto{\pgfqpoint{1.376819in}{1.680889in}}%
\pgfpathlineto{\pgfqpoint{1.375466in}{1.684609in}}%
\pgfpathlineto{\pgfqpoint{1.374114in}{1.688212in}}%
\pgfpathlineto{\pgfqpoint{1.372762in}{1.691698in}}%
\pgfpathlineto{\pgfqpoint{1.374393in}{1.692029in}}%
\pgfpathlineto{\pgfqpoint{1.376001in}{1.692385in}}%
\pgfpathlineto{\pgfqpoint{1.377584in}{1.692764in}}%
\pgfpathlineto{\pgfqpoint{1.379140in}{1.693166in}}%
\pgfpathclose%
\pgfusepath{fill}%
\end{pgfscope}%
\begin{pgfscope}%
\pgfpathrectangle{\pgfqpoint{0.329460in}{0.284240in}}{\pgfqpoint{1.989680in}{1.989680in}}%
\pgfusepath{clip}%
\pgfsetbuttcap%
\pgfsetroundjoin%
\definecolor{currentfill}{rgb}{0.271305,0.019942,0.347269}%
\pgfsetfillcolor{currentfill}%
\pgfsetlinewidth{0.000000pt}%
\definecolor{currentstroke}{rgb}{0.000000,0.000000,0.000000}%
\pgfsetstrokecolor{currentstroke}%
\pgfsetdash{}{0pt}%
\pgfpathmoveto{\pgfqpoint{1.422092in}{0.784955in}}%
\pgfpathlineto{\pgfqpoint{1.422596in}{0.780815in}}%
\pgfpathlineto{\pgfqpoint{1.423100in}{0.776859in}}%
\pgfpathlineto{\pgfqpoint{1.423605in}{0.773089in}}%
\pgfpathlineto{\pgfqpoint{1.424112in}{0.769510in}}%
\pgfpathlineto{\pgfqpoint{1.407787in}{0.768400in}}%
\pgfpathlineto{\pgfqpoint{1.391399in}{0.767569in}}%
\pgfpathlineto{\pgfqpoint{1.374966in}{0.767020in}}%
\pgfpathlineto{\pgfqpoint{1.358507in}{0.766753in}}%
\pgfpathlineto{\pgfqpoint{1.358456in}{0.770353in}}%
\pgfpathlineto{\pgfqpoint{1.358405in}{0.774143in}}%
\pgfpathlineto{\pgfqpoint{1.358354in}{0.778120in}}%
\pgfpathlineto{\pgfqpoint{1.358304in}{0.782280in}}%
\pgfpathlineto{\pgfqpoint{1.374308in}{0.782539in}}%
\pgfpathlineto{\pgfqpoint{1.390285in}{0.783071in}}%
\pgfpathlineto{\pgfqpoint{1.406219in}{0.783877in}}%
\pgfpathlineto{\pgfqpoint{1.422092in}{0.784955in}}%
\pgfpathclose%
\pgfusepath{fill}%
\end{pgfscope}%
\begin{pgfscope}%
\pgfpathrectangle{\pgfqpoint{0.329460in}{0.284240in}}{\pgfqpoint{1.989680in}{1.989680in}}%
\pgfusepath{clip}%
\pgfsetbuttcap%
\pgfsetroundjoin%
\definecolor{currentfill}{rgb}{0.172719,0.448791,0.557885}%
\pgfsetfillcolor{currentfill}%
\pgfsetlinewidth{0.000000pt}%
\definecolor{currentstroke}{rgb}{0.000000,0.000000,0.000000}%
\pgfsetstrokecolor{currentstroke}%
\pgfsetdash{}{0pt}%
\pgfpathmoveto{\pgfqpoint{0.770634in}{1.080567in}}%
\pgfpathlineto{\pgfqpoint{0.767148in}{1.093263in}}%
\pgfpathlineto{\pgfqpoint{0.763641in}{1.106428in}}%
\pgfpathlineto{\pgfqpoint{0.760113in}{1.120070in}}%
\pgfpathlineto{\pgfqpoint{0.749244in}{1.129762in}}%
\pgfpathlineto{\pgfqpoint{0.739024in}{1.139611in}}%
\pgfpathlineto{\pgfqpoint{0.729462in}{1.149605in}}%
\pgfpathlineto{\pgfqpoint{0.720565in}{1.159732in}}%
\pgfpathlineto{\pgfqpoint{0.724305in}{1.145939in}}%
\pgfpathlineto{\pgfqpoint{0.728024in}{1.132620in}}%
\pgfpathlineto{\pgfqpoint{0.731721in}{1.119768in}}%
\pgfpathlineto{\pgfqpoint{0.740478in}{1.109757in}}%
\pgfpathlineto{\pgfqpoint{0.749887in}{1.099880in}}%
\pgfpathlineto{\pgfqpoint{0.759942in}{1.090146in}}%
\pgfpathlineto{\pgfqpoint{0.770634in}{1.080567in}}%
\pgfpathclose%
\pgfusepath{fill}%
\end{pgfscope}%
\begin{pgfscope}%
\pgfpathrectangle{\pgfqpoint{0.329460in}{0.284240in}}{\pgfqpoint{1.989680in}{1.989680in}}%
\pgfusepath{clip}%
\pgfsetbuttcap%
\pgfsetroundjoin%
\definecolor{currentfill}{rgb}{0.855810,0.888601,0.097452}%
\pgfsetfillcolor{currentfill}%
\pgfsetlinewidth{0.000000pt}%
\definecolor{currentstroke}{rgb}{0.000000,0.000000,0.000000}%
\pgfsetstrokecolor{currentstroke}%
\pgfsetdash{}{0pt}%
\pgfpathmoveto{\pgfqpoint{1.302236in}{1.664889in}}%
\pgfpathlineto{\pgfqpoint{1.300193in}{1.660669in}}%
\pgfpathlineto{\pgfqpoint{1.298149in}{1.656336in}}%
\pgfpathlineto{\pgfqpoint{1.296105in}{1.651894in}}%
\pgfpathlineto{\pgfqpoint{1.294062in}{1.647341in}}%
\pgfpathlineto{\pgfqpoint{1.291559in}{1.648212in}}%
\pgfpathlineto{\pgfqpoint{1.289117in}{1.649120in}}%
\pgfpathlineto{\pgfqpoint{1.286738in}{1.650063in}}%
\pgfpathlineto{\pgfqpoint{1.284425in}{1.651041in}}%
\pgfpathlineto{\pgfqpoint{1.286812in}{1.655459in}}%
\pgfpathlineto{\pgfqpoint{1.289199in}{1.659767in}}%
\pgfpathlineto{\pgfqpoint{1.291586in}{1.663965in}}%
\pgfpathlineto{\pgfqpoint{1.293974in}{1.668051in}}%
\pgfpathlineto{\pgfqpoint{1.295958in}{1.667215in}}%
\pgfpathlineto{\pgfqpoint{1.297997in}{1.666409in}}%
\pgfpathlineto{\pgfqpoint{1.300091in}{1.665633in}}%
\pgfpathlineto{\pgfqpoint{1.302236in}{1.664889in}}%
\pgfpathclose%
\pgfusepath{fill}%
\end{pgfscope}%
\begin{pgfscope}%
\pgfpathrectangle{\pgfqpoint{0.329460in}{0.284240in}}{\pgfqpoint{1.989680in}{1.989680in}}%
\pgfusepath{clip}%
\pgfsetbuttcap%
\pgfsetroundjoin%
\definecolor{currentfill}{rgb}{0.935904,0.898570,0.108131}%
\pgfsetfillcolor{currentfill}%
\pgfsetlinewidth{0.000000pt}%
\definecolor{currentstroke}{rgb}{0.000000,0.000000,0.000000}%
\pgfsetstrokecolor{currentstroke}%
\pgfsetdash{}{0pt}%
\pgfpathmoveto{\pgfqpoint{1.331081in}{1.691424in}}%
\pgfpathlineto{\pgfqpoint{1.329821in}{1.687920in}}%
\pgfpathlineto{\pgfqpoint{1.328561in}{1.684300in}}%
\pgfpathlineto{\pgfqpoint{1.327300in}{1.680563in}}%
\pgfpathlineto{\pgfqpoint{1.326039in}{1.676710in}}%
\pgfpathlineto{\pgfqpoint{1.323975in}{1.677099in}}%
\pgfpathlineto{\pgfqpoint{1.321939in}{1.677518in}}%
\pgfpathlineto{\pgfqpoint{1.319932in}{1.677967in}}%
\pgfpathlineto{\pgfqpoint{1.317956in}{1.678445in}}%
\pgfpathlineto{\pgfqpoint{1.319622in}{1.682210in}}%
\pgfpathlineto{\pgfqpoint{1.321287in}{1.685859in}}%
\pgfpathlineto{\pgfqpoint{1.322952in}{1.689392in}}%
\pgfpathlineto{\pgfqpoint{1.324617in}{1.692808in}}%
\pgfpathlineto{\pgfqpoint{1.326197in}{1.692426in}}%
\pgfpathlineto{\pgfqpoint{1.327802in}{1.692068in}}%
\pgfpathlineto{\pgfqpoint{1.329431in}{1.691734in}}%
\pgfpathlineto{\pgfqpoint{1.331081in}{1.691424in}}%
\pgfpathclose%
\pgfusepath{fill}%
\end{pgfscope}%
\begin{pgfscope}%
\pgfpathrectangle{\pgfqpoint{0.329460in}{0.284240in}}{\pgfqpoint{1.989680in}{1.989680in}}%
\pgfusepath{clip}%
\pgfsetbuttcap%
\pgfsetroundjoin%
\definecolor{currentfill}{rgb}{0.268510,0.009605,0.335427}%
\pgfsetfillcolor{currentfill}%
\pgfsetlinewidth{0.000000pt}%
\definecolor{currentstroke}{rgb}{0.000000,0.000000,0.000000}%
\pgfsetstrokecolor{currentstroke}%
\pgfsetdash{}{0pt}%
\pgfpathmoveto{\pgfqpoint{1.488411in}{0.776726in}}%
\pgfpathlineto{\pgfqpoint{1.489365in}{0.773396in}}%
\pgfpathlineto{\pgfqpoint{1.490320in}{0.770266in}}%
\pgfpathlineto{\pgfqpoint{1.491278in}{0.767339in}}%
\pgfpathlineto{\pgfqpoint{1.492237in}{0.764620in}}%
\pgfpathlineto{\pgfqpoint{1.475906in}{0.762339in}}%
\pgfpathlineto{\pgfqpoint{1.459435in}{0.760339in}}%
\pgfpathlineto{\pgfqpoint{1.442843in}{0.758622in}}%
\pgfpathlineto{\pgfqpoint{1.426147in}{0.757190in}}%
\pgfpathlineto{\pgfqpoint{1.425637in}{0.759963in}}%
\pgfpathlineto{\pgfqpoint{1.425127in}{0.762943in}}%
\pgfpathlineto{\pgfqpoint{1.424619in}{0.766127in}}%
\pgfpathlineto{\pgfqpoint{1.424112in}{0.769510in}}%
\pgfpathlineto{\pgfqpoint{1.440354in}{0.770900in}}%
\pgfpathlineto{\pgfqpoint{1.456497in}{0.772568in}}%
\pgfpathlineto{\pgfqpoint{1.472522in}{0.774510in}}%
\pgfpathlineto{\pgfqpoint{1.488411in}{0.776726in}}%
\pgfpathclose%
\pgfusepath{fill}%
\end{pgfscope}%
\begin{pgfscope}%
\pgfpathrectangle{\pgfqpoint{0.329460in}{0.284240in}}{\pgfqpoint{1.989680in}{1.989680in}}%
\pgfusepath{clip}%
\pgfsetbuttcap%
\pgfsetroundjoin%
\definecolor{currentfill}{rgb}{0.814576,0.883393,0.110347}%
\pgfsetfillcolor{currentfill}%
\pgfsetlinewidth{0.000000pt}%
\definecolor{currentstroke}{rgb}{0.000000,0.000000,0.000000}%
\pgfsetstrokecolor{currentstroke}%
\pgfsetdash{}{0pt}%
\pgfpathmoveto{\pgfqpoint{1.419950in}{1.651940in}}%
\pgfpathlineto{\pgfqpoint{1.422408in}{1.647446in}}%
\pgfpathlineto{\pgfqpoint{1.424865in}{1.642846in}}%
\pgfpathlineto{\pgfqpoint{1.427323in}{1.638138in}}%
\pgfpathlineto{\pgfqpoint{1.429780in}{1.633325in}}%
\pgfpathlineto{\pgfqpoint{1.427206in}{1.632169in}}%
\pgfpathlineto{\pgfqpoint{1.424555in}{1.631052in}}%
\pgfpathlineto{\pgfqpoint{1.421829in}{1.629974in}}%
\pgfpathlineto{\pgfqpoint{1.419032in}{1.628938in}}%
\pgfpathlineto{\pgfqpoint{1.416909in}{1.633892in}}%
\pgfpathlineto{\pgfqpoint{1.414787in}{1.638740in}}%
\pgfpathlineto{\pgfqpoint{1.412664in}{1.643480in}}%
\pgfpathlineto{\pgfqpoint{1.410541in}{1.648113in}}%
\pgfpathlineto{\pgfqpoint{1.412990in}{1.649017in}}%
\pgfpathlineto{\pgfqpoint{1.415376in}{1.649957in}}%
\pgfpathlineto{\pgfqpoint{1.417696in}{1.650931in}}%
\pgfpathlineto{\pgfqpoint{1.419950in}{1.651940in}}%
\pgfpathclose%
\pgfusepath{fill}%
\end{pgfscope}%
\begin{pgfscope}%
\pgfpathrectangle{\pgfqpoint{0.329460in}{0.284240in}}{\pgfqpoint{1.989680in}{1.989680in}}%
\pgfusepath{clip}%
\pgfsetbuttcap%
\pgfsetroundjoin%
\definecolor{currentfill}{rgb}{0.636902,0.856542,0.216620}%
\pgfsetfillcolor{currentfill}%
\pgfsetlinewidth{0.000000pt}%
\definecolor{currentstroke}{rgb}{0.000000,0.000000,0.000000}%
\pgfsetstrokecolor{currentstroke}%
\pgfsetdash{}{0pt}%
\pgfpathmoveto{\pgfqpoint{1.449418in}{1.591147in}}%
\pgfpathlineto{\pgfqpoint{1.451870in}{1.585432in}}%
\pgfpathlineto{\pgfqpoint{1.454322in}{1.579624in}}%
\pgfpathlineto{\pgfqpoint{1.456773in}{1.573723in}}%
\pgfpathlineto{\pgfqpoint{1.459223in}{1.567730in}}%
\pgfpathlineto{\pgfqpoint{1.455693in}{1.566123in}}%
\pgfpathlineto{\pgfqpoint{1.452056in}{1.564569in}}%
\pgfpathlineto{\pgfqpoint{1.448315in}{1.563071in}}%
\pgfpathlineto{\pgfqpoint{1.444475in}{1.561630in}}%
\pgfpathlineto{\pgfqpoint{1.442357in}{1.567767in}}%
\pgfpathlineto{\pgfqpoint{1.440239in}{1.573812in}}%
\pgfpathlineto{\pgfqpoint{1.438120in}{1.579765in}}%
\pgfpathlineto{\pgfqpoint{1.436001in}{1.585623in}}%
\pgfpathlineto{\pgfqpoint{1.439494in}{1.586927in}}%
\pgfpathlineto{\pgfqpoint{1.442897in}{1.588284in}}%
\pgfpathlineto{\pgfqpoint{1.446206in}{1.589691in}}%
\pgfpathlineto{\pgfqpoint{1.449418in}{1.591147in}}%
\pgfpathclose%
\pgfusepath{fill}%
\end{pgfscope}%
\begin{pgfscope}%
\pgfpathrectangle{\pgfqpoint{0.329460in}{0.284240in}}{\pgfqpoint{1.989680in}{1.989680in}}%
\pgfusepath{clip}%
\pgfsetbuttcap%
\pgfsetroundjoin%
\definecolor{currentfill}{rgb}{0.147607,0.511733,0.557049}%
\pgfsetfillcolor{currentfill}%
\pgfsetlinewidth{0.000000pt}%
\definecolor{currentstroke}{rgb}{0.000000,0.000000,0.000000}%
\pgfsetstrokecolor{currentstroke}%
\pgfsetdash{}{0pt}%
\pgfpathmoveto{\pgfqpoint{1.432852in}{1.200478in}}%
\pgfpathlineto{\pgfqpoint{1.433772in}{1.191749in}}%
\pgfpathlineto{\pgfqpoint{1.434691in}{1.183020in}}%
\pgfpathlineto{\pgfqpoint{1.435610in}{1.174293in}}%
\pgfpathlineto{\pgfqpoint{1.436529in}{1.165572in}}%
\pgfpathlineto{\pgfqpoint{1.426637in}{1.164273in}}%
\pgfpathlineto{\pgfqpoint{1.416666in}{1.163135in}}%
\pgfpathlineto{\pgfqpoint{1.406623in}{1.162159in}}%
\pgfpathlineto{\pgfqpoint{1.396522in}{1.161345in}}%
\pgfpathlineto{\pgfqpoint{1.396033in}{1.170118in}}%
\pgfpathlineto{\pgfqpoint{1.395545in}{1.178896in}}%
\pgfpathlineto{\pgfqpoint{1.395056in}{1.187676in}}%
\pgfpathlineto{\pgfqpoint{1.394567in}{1.196456in}}%
\pgfpathlineto{\pgfqpoint{1.404234in}{1.197231in}}%
\pgfpathlineto{\pgfqpoint{1.413844in}{1.198160in}}%
\pgfpathlineto{\pgfqpoint{1.423387in}{1.199243in}}%
\pgfpathlineto{\pgfqpoint{1.432852in}{1.200478in}}%
\pgfpathclose%
\pgfusepath{fill}%
\end{pgfscope}%
\begin{pgfscope}%
\pgfpathrectangle{\pgfqpoint{0.329460in}{0.284240in}}{\pgfqpoint{1.989680in}{1.989680in}}%
\pgfusepath{clip}%
\pgfsetbuttcap%
\pgfsetroundjoin%
\definecolor{currentfill}{rgb}{0.271305,0.019942,0.347269}%
\pgfsetfillcolor{currentfill}%
\pgfsetlinewidth{0.000000pt}%
\definecolor{currentstroke}{rgb}{0.000000,0.000000,0.000000}%
\pgfsetstrokecolor{currentstroke}%
\pgfsetdash{}{0pt}%
\pgfpathmoveto{\pgfqpoint{1.358304in}{0.782280in}}%
\pgfpathlineto{\pgfqpoint{1.358354in}{0.778120in}}%
\pgfpathlineto{\pgfqpoint{1.358405in}{0.774143in}}%
\pgfpathlineto{\pgfqpoint{1.358456in}{0.770353in}}%
\pgfpathlineto{\pgfqpoint{1.358507in}{0.766753in}}%
\pgfpathlineto{\pgfqpoint{1.342039in}{0.766769in}}%
\pgfpathlineto{\pgfqpoint{1.325581in}{0.767067in}}%
\pgfpathlineto{\pgfqpoint{1.309152in}{0.767647in}}%
\pgfpathlineto{\pgfqpoint{1.292771in}{0.768509in}}%
\pgfpathlineto{\pgfqpoint{1.293176in}{0.772095in}}%
\pgfpathlineto{\pgfqpoint{1.293581in}{0.775872in}}%
\pgfpathlineto{\pgfqpoint{1.293985in}{0.779837in}}%
\pgfpathlineto{\pgfqpoint{1.294389in}{0.783983in}}%
\pgfpathlineto{\pgfqpoint{1.310317in}{0.783147in}}%
\pgfpathlineto{\pgfqpoint{1.326291in}{0.782585in}}%
\pgfpathlineto{\pgfqpoint{1.342292in}{0.782295in}}%
\pgfpathlineto{\pgfqpoint{1.358304in}{0.782280in}}%
\pgfpathclose%
\pgfusepath{fill}%
\end{pgfscope}%
\begin{pgfscope}%
\pgfpathrectangle{\pgfqpoint{0.329460in}{0.284240in}}{\pgfqpoint{1.989680in}{1.989680in}}%
\pgfusepath{clip}%
\pgfsetbuttcap%
\pgfsetroundjoin%
\definecolor{currentfill}{rgb}{0.122606,0.585371,0.546557}%
\pgfsetfillcolor{currentfill}%
\pgfsetlinewidth{0.000000pt}%
\definecolor{currentstroke}{rgb}{0.000000,0.000000,0.000000}%
\pgfsetstrokecolor{currentstroke}%
\pgfsetdash{}{0pt}%
\pgfpathmoveto{\pgfqpoint{1.459063in}{1.275792in}}%
\pgfpathlineto{\pgfqpoint{1.460399in}{1.267235in}}%
\pgfpathlineto{\pgfqpoint{1.461735in}{1.258658in}}%
\pgfpathlineto{\pgfqpoint{1.463071in}{1.250063in}}%
\pgfpathlineto{\pgfqpoint{1.464406in}{1.241453in}}%
\pgfpathlineto{\pgfqpoint{1.455755in}{1.239716in}}%
\pgfpathlineto{\pgfqpoint{1.446993in}{1.238117in}}%
\pgfpathlineto{\pgfqpoint{1.438129in}{1.236659in}}%
\pgfpathlineto{\pgfqpoint{1.429172in}{1.235342in}}%
\pgfpathlineto{\pgfqpoint{1.428252in}{1.244032in}}%
\pgfpathlineto{\pgfqpoint{1.427331in}{1.252707in}}%
\pgfpathlineto{\pgfqpoint{1.426410in}{1.261365in}}%
\pgfpathlineto{\pgfqpoint{1.425489in}{1.270002in}}%
\pgfpathlineto{\pgfqpoint{1.434023in}{1.271249in}}%
\pgfpathlineto{\pgfqpoint{1.442469in}{1.272631in}}%
\pgfpathlineto{\pgfqpoint{1.450819in}{1.274146in}}%
\pgfpathlineto{\pgfqpoint{1.459063in}{1.275792in}}%
\pgfpathclose%
\pgfusepath{fill}%
\end{pgfscope}%
\begin{pgfscope}%
\pgfpathrectangle{\pgfqpoint{0.329460in}{0.284240in}}{\pgfqpoint{1.989680in}{1.989680in}}%
\pgfusepath{clip}%
\pgfsetbuttcap%
\pgfsetroundjoin%
\definecolor{currentfill}{rgb}{0.896320,0.893616,0.096335}%
\pgfsetfillcolor{currentfill}%
\pgfsetlinewidth{0.000000pt}%
\definecolor{currentstroke}{rgb}{0.000000,0.000000,0.000000}%
\pgfsetstrokecolor{currentstroke}%
\pgfsetdash{}{0pt}%
\pgfpathmoveto{\pgfqpoint{1.393556in}{1.681193in}}%
\pgfpathlineto{\pgfqpoint{1.395679in}{1.677452in}}%
\pgfpathlineto{\pgfqpoint{1.397802in}{1.673598in}}%
\pgfpathlineto{\pgfqpoint{1.399925in}{1.669630in}}%
\pgfpathlineto{\pgfqpoint{1.402048in}{1.665549in}}%
\pgfpathlineto{\pgfqpoint{1.399897in}{1.664809in}}%
\pgfpathlineto{\pgfqpoint{1.397697in}{1.664100in}}%
\pgfpathlineto{\pgfqpoint{1.395449in}{1.663425in}}%
\pgfpathlineto{\pgfqpoint{1.393157in}{1.662784in}}%
\pgfpathlineto{\pgfqpoint{1.391404in}{1.666982in}}%
\pgfpathlineto{\pgfqpoint{1.389651in}{1.671067in}}%
\pgfpathlineto{\pgfqpoint{1.387899in}{1.675038in}}%
\pgfpathlineto{\pgfqpoint{1.386146in}{1.678895in}}%
\pgfpathlineto{\pgfqpoint{1.388057in}{1.679428in}}%
\pgfpathlineto{\pgfqpoint{1.389930in}{1.679989in}}%
\pgfpathlineto{\pgfqpoint{1.391763in}{1.680577in}}%
\pgfpathlineto{\pgfqpoint{1.393556in}{1.681193in}}%
\pgfpathclose%
\pgfusepath{fill}%
\end{pgfscope}%
\begin{pgfscope}%
\pgfpathrectangle{\pgfqpoint{0.329460in}{0.284240in}}{\pgfqpoint{1.989680in}{1.989680in}}%
\pgfusepath{clip}%
\pgfsetbuttcap%
\pgfsetroundjoin%
\definecolor{currentfill}{rgb}{0.134692,0.658636,0.517649}%
\pgfsetfillcolor{currentfill}%
\pgfsetlinewidth{0.000000pt}%
\definecolor{currentstroke}{rgb}{0.000000,0.000000,0.000000}%
\pgfsetstrokecolor{currentstroke}%
\pgfsetdash{}{0pt}%
\pgfpathmoveto{\pgfqpoint{1.260617in}{1.341944in}}%
\pgfpathlineto{\pgfqpoint{1.259368in}{1.333610in}}%
\pgfpathlineto{\pgfqpoint{1.258119in}{1.325237in}}%
\pgfpathlineto{\pgfqpoint{1.256871in}{1.316829in}}%
\pgfpathlineto{\pgfqpoint{1.255623in}{1.308386in}}%
\pgfpathlineto{\pgfqpoint{1.247798in}{1.309956in}}%
\pgfpathlineto{\pgfqpoint{1.240083in}{1.311648in}}%
\pgfpathlineto{\pgfqpoint{1.232484in}{1.313460in}}%
\pgfpathlineto{\pgfqpoint{1.225011in}{1.315390in}}%
\pgfpathlineto{\pgfqpoint{1.226657in}{1.323732in}}%
\pgfpathlineto{\pgfqpoint{1.228304in}{1.332040in}}%
\pgfpathlineto{\pgfqpoint{1.229951in}{1.340312in}}%
\pgfpathlineto{\pgfqpoint{1.231599in}{1.348546in}}%
\pgfpathlineto{\pgfqpoint{1.238684in}{1.346727in}}%
\pgfpathlineto{\pgfqpoint{1.245887in}{1.345018in}}%
\pgfpathlineto{\pgfqpoint{1.253200in}{1.343423in}}%
\pgfpathlineto{\pgfqpoint{1.260617in}{1.341944in}}%
\pgfpathclose%
\pgfusepath{fill}%
\end{pgfscope}%
\begin{pgfscope}%
\pgfpathrectangle{\pgfqpoint{0.329460in}{0.284240in}}{\pgfqpoint{1.989680in}{1.989680in}}%
\pgfusepath{clip}%
\pgfsetbuttcap%
\pgfsetroundjoin%
\definecolor{currentfill}{rgb}{0.565498,0.842430,0.262877}%
\pgfsetfillcolor{currentfill}%
\pgfsetlinewidth{0.000000pt}%
\definecolor{currentstroke}{rgb}{0.000000,0.000000,0.000000}%
\pgfsetstrokecolor{currentstroke}%
\pgfsetdash{}{0pt}%
\pgfpathmoveto{\pgfqpoint{1.261394in}{1.560398in}}%
\pgfpathlineto{\pgfqpoint{1.259356in}{1.554141in}}%
\pgfpathlineto{\pgfqpoint{1.257318in}{1.547795in}}%
\pgfpathlineto{\pgfqpoint{1.255281in}{1.541362in}}%
\pgfpathlineto{\pgfqpoint{1.253245in}{1.534843in}}%
\pgfpathlineto{\pgfqpoint{1.248965in}{1.536364in}}%
\pgfpathlineto{\pgfqpoint{1.244791in}{1.537950in}}%
\pgfpathlineto{\pgfqpoint{1.240726in}{1.539598in}}%
\pgfpathlineto{\pgfqpoint{1.236774in}{1.541306in}}%
\pgfpathlineto{\pgfqpoint{1.239151in}{1.547684in}}%
\pgfpathlineto{\pgfqpoint{1.241528in}{1.553977in}}%
\pgfpathlineto{\pgfqpoint{1.243906in}{1.560182in}}%
\pgfpathlineto{\pgfqpoint{1.246285in}{1.566299in}}%
\pgfpathlineto{\pgfqpoint{1.249910in}{1.564739in}}%
\pgfpathlineto{\pgfqpoint{1.253640in}{1.563235in}}%
\pgfpathlineto{\pgfqpoint{1.257469in}{1.561787in}}%
\pgfpathlineto{\pgfqpoint{1.261394in}{1.560398in}}%
\pgfpathclose%
\pgfusepath{fill}%
\end{pgfscope}%
\begin{pgfscope}%
\pgfpathrectangle{\pgfqpoint{0.329460in}{0.284240in}}{\pgfqpoint{1.989680in}{1.989680in}}%
\pgfusepath{clip}%
\pgfsetbuttcap%
\pgfsetroundjoin%
\definecolor{currentfill}{rgb}{0.163625,0.471133,0.558148}%
\pgfsetfillcolor{currentfill}%
\pgfsetlinewidth{0.000000pt}%
\definecolor{currentstroke}{rgb}{0.000000,0.000000,0.000000}%
\pgfsetstrokecolor{currentstroke}%
\pgfsetdash{}{0pt}%
\pgfpathmoveto{\pgfqpoint{1.396522in}{1.161345in}}%
\pgfpathlineto{\pgfqpoint{1.397010in}{1.152580in}}%
\pgfpathlineto{\pgfqpoint{1.397498in}{1.143824in}}%
\pgfpathlineto{\pgfqpoint{1.397986in}{1.135082in}}%
\pgfpathlineto{\pgfqpoint{1.398474in}{1.126356in}}%
\pgfpathlineto{\pgfqpoint{1.387887in}{1.125674in}}%
\pgfpathlineto{\pgfqpoint{1.377260in}{1.125164in}}%
\pgfpathlineto{\pgfqpoint{1.366605in}{1.124827in}}%
\pgfpathlineto{\pgfqpoint{1.355933in}{1.124663in}}%
\pgfpathlineto{\pgfqpoint{1.355884in}{1.133409in}}%
\pgfpathlineto{\pgfqpoint{1.355835in}{1.142171in}}%
\pgfpathlineto{\pgfqpoint{1.355786in}{1.150946in}}%
\pgfpathlineto{\pgfqpoint{1.355737in}{1.159731in}}%
\pgfpathlineto{\pgfqpoint{1.365968in}{1.159887in}}%
\pgfpathlineto{\pgfqpoint{1.376183in}{1.160208in}}%
\pgfpathlineto{\pgfqpoint{1.386371in}{1.160694in}}%
\pgfpathlineto{\pgfqpoint{1.396522in}{1.161345in}}%
\pgfpathclose%
\pgfusepath{fill}%
\end{pgfscope}%
\begin{pgfscope}%
\pgfpathrectangle{\pgfqpoint{0.329460in}{0.284240in}}{\pgfqpoint{1.989680in}{1.989680in}}%
\pgfusepath{clip}%
\pgfsetbuttcap%
\pgfsetroundjoin%
\definecolor{currentfill}{rgb}{0.699415,0.867117,0.175971}%
\pgfsetfillcolor{currentfill}%
\pgfsetlinewidth{0.000000pt}%
\definecolor{currentstroke}{rgb}{0.000000,0.000000,0.000000}%
\pgfsetstrokecolor{currentstroke}%
\pgfsetdash{}{0pt}%
\pgfpathmoveto{\pgfqpoint{1.439603in}{1.613038in}}%
\pgfpathlineto{\pgfqpoint{1.442058in}{1.607712in}}%
\pgfpathlineto{\pgfqpoint{1.444512in}{1.602288in}}%
\pgfpathlineto{\pgfqpoint{1.446965in}{1.596766in}}%
\pgfpathlineto{\pgfqpoint{1.449418in}{1.591147in}}%
\pgfpathlineto{\pgfqpoint{1.446206in}{1.589691in}}%
\pgfpathlineto{\pgfqpoint{1.442897in}{1.588284in}}%
\pgfpathlineto{\pgfqpoint{1.439494in}{1.586927in}}%
\pgfpathlineto{\pgfqpoint{1.436001in}{1.585623in}}%
\pgfpathlineto{\pgfqpoint{1.433881in}{1.591385in}}%
\pgfpathlineto{\pgfqpoint{1.431761in}{1.597050in}}%
\pgfpathlineto{\pgfqpoint{1.429640in}{1.602617in}}%
\pgfpathlineto{\pgfqpoint{1.427519in}{1.608085in}}%
\pgfpathlineto{\pgfqpoint{1.430665in}{1.609255in}}%
\pgfpathlineto{\pgfqpoint{1.433729in}{1.610471in}}%
\pgfpathlineto{\pgfqpoint{1.436710in}{1.611732in}}%
\pgfpathlineto{\pgfqpoint{1.439603in}{1.613038in}}%
\pgfpathclose%
\pgfusepath{fill}%
\end{pgfscope}%
\begin{pgfscope}%
\pgfpathrectangle{\pgfqpoint{0.329460in}{0.284240in}}{\pgfqpoint{1.989680in}{1.989680in}}%
\pgfusepath{clip}%
\pgfsetbuttcap%
\pgfsetroundjoin%
\definecolor{currentfill}{rgb}{0.762373,0.876424,0.137064}%
\pgfsetfillcolor{currentfill}%
\pgfsetlinewidth{0.000000pt}%
\definecolor{currentstroke}{rgb}{0.000000,0.000000,0.000000}%
\pgfsetstrokecolor{currentstroke}%
\pgfsetdash{}{0pt}%
\pgfpathmoveto{\pgfqpoint{1.429780in}{1.633325in}}%
\pgfpathlineto{\pgfqpoint{1.432236in}{1.628408in}}%
\pgfpathlineto{\pgfqpoint{1.434692in}{1.623387in}}%
\pgfpathlineto{\pgfqpoint{1.437148in}{1.618263in}}%
\pgfpathlineto{\pgfqpoint{1.439603in}{1.613038in}}%
\pgfpathlineto{\pgfqpoint{1.436710in}{1.611732in}}%
\pgfpathlineto{\pgfqpoint{1.433729in}{1.610471in}}%
\pgfpathlineto{\pgfqpoint{1.430665in}{1.609255in}}%
\pgfpathlineto{\pgfqpoint{1.427519in}{1.608085in}}%
\pgfpathlineto{\pgfqpoint{1.425398in}{1.613452in}}%
\pgfpathlineto{\pgfqpoint{1.423276in}{1.618717in}}%
\pgfpathlineto{\pgfqpoint{1.421154in}{1.623879in}}%
\pgfpathlineto{\pgfqpoint{1.419032in}{1.628938in}}%
\pgfpathlineto{\pgfqpoint{1.421829in}{1.629974in}}%
\pgfpathlineto{\pgfqpoint{1.424555in}{1.631052in}}%
\pgfpathlineto{\pgfqpoint{1.427206in}{1.632169in}}%
\pgfpathlineto{\pgfqpoint{1.429780in}{1.633325in}}%
\pgfpathclose%
\pgfusepath{fill}%
\end{pgfscope}%
\begin{pgfscope}%
\pgfpathrectangle{\pgfqpoint{0.329460in}{0.284240in}}{\pgfqpoint{1.989680in}{1.989680in}}%
\pgfusepath{clip}%
\pgfsetbuttcap%
\pgfsetroundjoin%
\definecolor{currentfill}{rgb}{0.281477,0.755203,0.432552}%
\pgfsetfillcolor{currentfill}%
\pgfsetlinewidth{0.000000pt}%
\definecolor{currentstroke}{rgb}{0.000000,0.000000,0.000000}%
\pgfsetstrokecolor{currentstroke}%
\pgfsetdash{}{0pt}%
\pgfpathmoveto{\pgfqpoint{1.251421in}{1.443611in}}%
\pgfpathlineto{\pgfqpoint{1.249766in}{1.435998in}}%
\pgfpathlineto{\pgfqpoint{1.248111in}{1.428322in}}%
\pgfpathlineto{\pgfqpoint{1.246457in}{1.420585in}}%
\pgfpathlineto{\pgfqpoint{1.244804in}{1.412789in}}%
\pgfpathlineto{\pgfqpoint{1.238609in}{1.414488in}}%
\pgfpathlineto{\pgfqpoint{1.232532in}{1.416281in}}%
\pgfpathlineto{\pgfqpoint{1.226579in}{1.418167in}}%
\pgfpathlineto{\pgfqpoint{1.220755in}{1.420144in}}%
\pgfpathlineto{\pgfqpoint{1.222780in}{1.427816in}}%
\pgfpathlineto{\pgfqpoint{1.224806in}{1.435429in}}%
\pgfpathlineto{\pgfqpoint{1.226832in}{1.442983in}}%
\pgfpathlineto{\pgfqpoint{1.228859in}{1.450473in}}%
\pgfpathlineto{\pgfqpoint{1.234323in}{1.448629in}}%
\pgfpathlineto{\pgfqpoint{1.239909in}{1.446869in}}%
\pgfpathlineto{\pgfqpoint{1.245610in}{1.445196in}}%
\pgfpathlineto{\pgfqpoint{1.251421in}{1.443611in}}%
\pgfpathclose%
\pgfusepath{fill}%
\end{pgfscope}%
\begin{pgfscope}%
\pgfpathrectangle{\pgfqpoint{0.329460in}{0.284240in}}{\pgfqpoint{1.989680in}{1.989680in}}%
\pgfusepath{clip}%
\pgfsetbuttcap%
\pgfsetroundjoin%
\definecolor{currentfill}{rgb}{0.896320,0.893616,0.096335}%
\pgfsetfillcolor{currentfill}%
\pgfsetlinewidth{0.000000pt}%
\definecolor{currentstroke}{rgb}{0.000000,0.000000,0.000000}%
\pgfsetstrokecolor{currentstroke}%
\pgfsetdash{}{0pt}%
\pgfpathmoveto{\pgfqpoint{1.317956in}{1.678445in}}%
\pgfpathlineto{\pgfqpoint{1.316291in}{1.674565in}}%
\pgfpathlineto{\pgfqpoint{1.314624in}{1.670571in}}%
\pgfpathlineto{\pgfqpoint{1.312958in}{1.666463in}}%
\pgfpathlineto{\pgfqpoint{1.311292in}{1.662242in}}%
\pgfpathlineto{\pgfqpoint{1.308962in}{1.662853in}}%
\pgfpathlineto{\pgfqpoint{1.306674in}{1.663499in}}%
\pgfpathlineto{\pgfqpoint{1.304432in}{1.664177in}}%
\pgfpathlineto{\pgfqpoint{1.302236in}{1.664889in}}%
\pgfpathlineto{\pgfqpoint{1.304280in}{1.668998in}}%
\pgfpathlineto{\pgfqpoint{1.306324in}{1.672994in}}%
\pgfpathlineto{\pgfqpoint{1.308367in}{1.676876in}}%
\pgfpathlineto{\pgfqpoint{1.310410in}{1.680644in}}%
\pgfpathlineto{\pgfqpoint{1.312240in}{1.680053in}}%
\pgfpathlineto{\pgfqpoint{1.314108in}{1.679489in}}%
\pgfpathlineto{\pgfqpoint{1.316015in}{1.678953in}}%
\pgfpathlineto{\pgfqpoint{1.317956in}{1.678445in}}%
\pgfpathclose%
\pgfusepath{fill}%
\end{pgfscope}%
\begin{pgfscope}%
\pgfpathrectangle{\pgfqpoint{0.329460in}{0.284240in}}{\pgfqpoint{1.989680in}{1.989680in}}%
\pgfusepath{clip}%
\pgfsetbuttcap%
\pgfsetroundjoin%
\definecolor{currentfill}{rgb}{0.163625,0.471133,0.558148}%
\pgfsetfillcolor{currentfill}%
\pgfsetlinewidth{0.000000pt}%
\definecolor{currentstroke}{rgb}{0.000000,0.000000,0.000000}%
\pgfsetstrokecolor{currentstroke}%
\pgfsetdash{}{0pt}%
\pgfpathmoveto{\pgfqpoint{1.355737in}{1.159731in}}%
\pgfpathlineto{\pgfqpoint{1.355786in}{1.150946in}}%
\pgfpathlineto{\pgfqpoint{1.355835in}{1.142171in}}%
\pgfpathlineto{\pgfqpoint{1.355884in}{1.133409in}}%
\pgfpathlineto{\pgfqpoint{1.355933in}{1.124663in}}%
\pgfpathlineto{\pgfqpoint{1.345256in}{1.124672in}}%
\pgfpathlineto{\pgfqpoint{1.334585in}{1.124856in}}%
\pgfpathlineto{\pgfqpoint{1.323933in}{1.125212in}}%
\pgfpathlineto{\pgfqpoint{1.313310in}{1.125741in}}%
\pgfpathlineto{\pgfqpoint{1.313701in}{1.134475in}}%
\pgfpathlineto{\pgfqpoint{1.314092in}{1.143224in}}%
\pgfpathlineto{\pgfqpoint{1.314483in}{1.151986in}}%
\pgfpathlineto{\pgfqpoint{1.314874in}{1.160759in}}%
\pgfpathlineto{\pgfqpoint{1.325059in}{1.160254in}}%
\pgfpathlineto{\pgfqpoint{1.335271in}{1.159914in}}%
\pgfpathlineto{\pgfqpoint{1.345501in}{1.159740in}}%
\pgfpathlineto{\pgfqpoint{1.355737in}{1.159731in}}%
\pgfpathclose%
\pgfusepath{fill}%
\end{pgfscope}%
\begin{pgfscope}%
\pgfpathrectangle{\pgfqpoint{0.329460in}{0.284240in}}{\pgfqpoint{1.989680in}{1.989680in}}%
\pgfusepath{clip}%
\pgfsetbuttcap%
\pgfsetroundjoin%
\definecolor{currentfill}{rgb}{0.147607,0.511733,0.557049}%
\pgfsetfillcolor{currentfill}%
\pgfsetlinewidth{0.000000pt}%
\definecolor{currentstroke}{rgb}{0.000000,0.000000,0.000000}%
\pgfsetstrokecolor{currentstroke}%
\pgfsetdash{}{0pt}%
\pgfpathmoveto{\pgfqpoint{1.316440in}{1.195898in}}%
\pgfpathlineto{\pgfqpoint{1.316048in}{1.187111in}}%
\pgfpathlineto{\pgfqpoint{1.315657in}{1.178324in}}%
\pgfpathlineto{\pgfqpoint{1.315265in}{1.169539in}}%
\pgfpathlineto{\pgfqpoint{1.314874in}{1.160759in}}%
\pgfpathlineto{\pgfqpoint{1.304729in}{1.161427in}}%
\pgfpathlineto{\pgfqpoint{1.294633in}{1.162259in}}%
\pgfpathlineto{\pgfqpoint{1.284598in}{1.163254in}}%
\pgfpathlineto{\pgfqpoint{1.274634in}{1.164410in}}%
\pgfpathlineto{\pgfqpoint{1.275459in}{1.173146in}}%
\pgfpathlineto{\pgfqpoint{1.276283in}{1.181886in}}%
\pgfpathlineto{\pgfqpoint{1.277108in}{1.190630in}}%
\pgfpathlineto{\pgfqpoint{1.277933in}{1.199373in}}%
\pgfpathlineto{\pgfqpoint{1.287468in}{1.198273in}}%
\pgfpathlineto{\pgfqpoint{1.297070in}{1.197326in}}%
\pgfpathlineto{\pgfqpoint{1.306731in}{1.196535in}}%
\pgfpathlineto{\pgfqpoint{1.316440in}{1.195898in}}%
\pgfpathclose%
\pgfusepath{fill}%
\end{pgfscope}%
\begin{pgfscope}%
\pgfpathrectangle{\pgfqpoint{0.329460in}{0.284240in}}{\pgfqpoint{1.989680in}{1.989680in}}%
\pgfusepath{clip}%
\pgfsetbuttcap%
\pgfsetroundjoin%
\definecolor{currentfill}{rgb}{0.935904,0.898570,0.108131}%
\pgfsetfillcolor{currentfill}%
\pgfsetlinewidth{0.000000pt}%
\definecolor{currentstroke}{rgb}{0.000000,0.000000,0.000000}%
\pgfsetstrokecolor{currentstroke}%
\pgfsetdash{}{0pt}%
\pgfpathmoveto{\pgfqpoint{1.372762in}{1.691698in}}%
\pgfpathlineto{\pgfqpoint{1.374114in}{1.688212in}}%
\pgfpathlineto{\pgfqpoint{1.375466in}{1.684609in}}%
\pgfpathlineto{\pgfqpoint{1.376819in}{1.680889in}}%
\pgfpathlineto{\pgfqpoint{1.378172in}{1.677054in}}%
\pgfpathlineto{\pgfqpoint{1.376105in}{1.676669in}}%
\pgfpathlineto{\pgfqpoint{1.374012in}{1.676314in}}%
\pgfpathlineto{\pgfqpoint{1.371897in}{1.675991in}}%
\pgfpathlineto{\pgfqpoint{1.369761in}{1.675700in}}%
\pgfpathlineto{\pgfqpoint{1.368829in}{1.679604in}}%
\pgfpathlineto{\pgfqpoint{1.367898in}{1.683392in}}%
\pgfpathlineto{\pgfqpoint{1.366967in}{1.687063in}}%
\pgfpathlineto{\pgfqpoint{1.366036in}{1.690618in}}%
\pgfpathlineto{\pgfqpoint{1.367745in}{1.690850in}}%
\pgfpathlineto{\pgfqpoint{1.369436in}{1.691108in}}%
\pgfpathlineto{\pgfqpoint{1.371109in}{1.691391in}}%
\pgfpathlineto{\pgfqpoint{1.372762in}{1.691698in}}%
\pgfpathclose%
\pgfusepath{fill}%
\end{pgfscope}%
\begin{pgfscope}%
\pgfpathrectangle{\pgfqpoint{0.329460in}{0.284240in}}{\pgfqpoint{1.989680in}{1.989680in}}%
\pgfusepath{clip}%
\pgfsetbuttcap%
\pgfsetroundjoin%
\definecolor{currentfill}{rgb}{0.814576,0.883393,0.110347}%
\pgfsetfillcolor{currentfill}%
\pgfsetlinewidth{0.000000pt}%
\definecolor{currentstroke}{rgb}{0.000000,0.000000,0.000000}%
\pgfsetstrokecolor{currentstroke}%
\pgfsetdash{}{0pt}%
\pgfpathmoveto{\pgfqpoint{1.294062in}{1.647341in}}%
\pgfpathlineto{\pgfqpoint{1.292018in}{1.642680in}}%
\pgfpathlineto{\pgfqpoint{1.289975in}{1.637911in}}%
\pgfpathlineto{\pgfqpoint{1.287931in}{1.633034in}}%
\pgfpathlineto{\pgfqpoint{1.285888in}{1.628052in}}%
\pgfpathlineto{\pgfqpoint{1.283029in}{1.629051in}}%
\pgfpathlineto{\pgfqpoint{1.280239in}{1.630092in}}%
\pgfpathlineto{\pgfqpoint{1.277522in}{1.631174in}}%
\pgfpathlineto{\pgfqpoint{1.274879in}{1.632296in}}%
\pgfpathlineto{\pgfqpoint{1.277265in}{1.637142in}}%
\pgfpathlineto{\pgfqpoint{1.279652in}{1.641882in}}%
\pgfpathlineto{\pgfqpoint{1.282038in}{1.646516in}}%
\pgfpathlineto{\pgfqpoint{1.284425in}{1.651041in}}%
\pgfpathlineto{\pgfqpoint{1.286738in}{1.650063in}}%
\pgfpathlineto{\pgfqpoint{1.289117in}{1.649120in}}%
\pgfpathlineto{\pgfqpoint{1.291559in}{1.648212in}}%
\pgfpathlineto{\pgfqpoint{1.294062in}{1.647341in}}%
\pgfpathclose%
\pgfusepath{fill}%
\end{pgfscope}%
\begin{pgfscope}%
\pgfpathrectangle{\pgfqpoint{0.329460in}{0.284240in}}{\pgfqpoint{1.989680in}{1.989680in}}%
\pgfusepath{clip}%
\pgfsetbuttcap%
\pgfsetroundjoin%
\definecolor{currentfill}{rgb}{0.282884,0.135920,0.453427}%
\pgfsetfillcolor{currentfill}%
\pgfsetlinewidth{0.000000pt}%
\definecolor{currentstroke}{rgb}{0.000000,0.000000,0.000000}%
\pgfsetstrokecolor{currentstroke}%
\pgfsetdash{}{0pt}%
\pgfpathmoveto{\pgfqpoint{1.782270in}{0.857007in}}%
\pgfpathlineto{\pgfqpoint{1.784947in}{0.861500in}}%
\pgfpathlineto{\pgfqpoint{1.787634in}{0.866329in}}%
\pgfpathlineto{\pgfqpoint{1.790332in}{0.871500in}}%
\pgfpathlineto{\pgfqpoint{1.793042in}{0.877019in}}%
\pgfpathlineto{\pgfqpoint{1.778961in}{0.869676in}}%
\pgfpathlineto{\pgfqpoint{1.764407in}{0.862567in}}%
\pgfpathlineto{\pgfqpoint{1.749394in}{0.855700in}}%
\pgfpathlineto{\pgfqpoint{1.733937in}{0.849085in}}%
\pgfpathlineto{\pgfqpoint{1.731580in}{0.843715in}}%
\pgfpathlineto{\pgfqpoint{1.729233in}{0.838694in}}%
\pgfpathlineto{\pgfqpoint{1.726895in}{0.834015in}}%
\pgfpathlineto{\pgfqpoint{1.724567in}{0.829674in}}%
\pgfpathlineto{\pgfqpoint{1.739655in}{0.836146in}}%
\pgfpathlineto{\pgfqpoint{1.754312in}{0.842865in}}%
\pgfpathlineto{\pgfqpoint{1.768521in}{0.849821in}}%
\pgfpathlineto{\pgfqpoint{1.782270in}{0.857007in}}%
\pgfpathclose%
\pgfusepath{fill}%
\end{pgfscope}%
\begin{pgfscope}%
\pgfpathrectangle{\pgfqpoint{0.329460in}{0.284240in}}{\pgfqpoint{1.989680in}{1.989680in}}%
\pgfusepath{clip}%
\pgfsetbuttcap%
\pgfsetroundjoin%
\definecolor{currentfill}{rgb}{0.344074,0.780029,0.397381}%
\pgfsetfillcolor{currentfill}%
\pgfsetlinewidth{0.000000pt}%
\definecolor{currentstroke}{rgb}{0.000000,0.000000,0.000000}%
\pgfsetstrokecolor{currentstroke}%
\pgfsetdash{}{0pt}%
\pgfpathmoveto{\pgfqpoint{1.469837in}{1.481364in}}%
\pgfpathlineto{\pgfqpoint{1.471945in}{1.474171in}}%
\pgfpathlineto{\pgfqpoint{1.474053in}{1.466909in}}%
\pgfpathlineto{\pgfqpoint{1.476160in}{1.459579in}}%
\pgfpathlineto{\pgfqpoint{1.478266in}{1.452183in}}%
\pgfpathlineto{\pgfqpoint{1.472915in}{1.450264in}}%
\pgfpathlineto{\pgfqpoint{1.467437in}{1.448429in}}%
\pgfpathlineto{\pgfqpoint{1.461838in}{1.446679in}}%
\pgfpathlineto{\pgfqpoint{1.456125in}{1.445015in}}%
\pgfpathlineto{\pgfqpoint{1.454383in}{1.452539in}}%
\pgfpathlineto{\pgfqpoint{1.452642in}{1.459996in}}%
\pgfpathlineto{\pgfqpoint{1.450899in}{1.467385in}}%
\pgfpathlineto{\pgfqpoint{1.449156in}{1.474705in}}%
\pgfpathlineto{\pgfqpoint{1.454492in}{1.476250in}}%
\pgfpathlineto{\pgfqpoint{1.459721in}{1.477876in}}%
\pgfpathlineto{\pgfqpoint{1.464838in}{1.479581in}}%
\pgfpathlineto{\pgfqpoint{1.469837in}{1.481364in}}%
\pgfpathclose%
\pgfusepath{fill}%
\end{pgfscope}%
\begin{pgfscope}%
\pgfpathrectangle{\pgfqpoint{0.329460in}{0.284240in}}{\pgfqpoint{1.989680in}{1.989680in}}%
\pgfusepath{clip}%
\pgfsetbuttcap%
\pgfsetroundjoin%
\definecolor{currentfill}{rgb}{0.935904,0.898570,0.108131}%
\pgfsetfillcolor{currentfill}%
\pgfsetlinewidth{0.000000pt}%
\definecolor{currentstroke}{rgb}{0.000000,0.000000,0.000000}%
\pgfsetstrokecolor{currentstroke}%
\pgfsetdash{}{0pt}%
\pgfpathmoveto{\pgfqpoint{1.337870in}{1.690432in}}%
\pgfpathlineto{\pgfqpoint{1.337035in}{1.686866in}}%
\pgfpathlineto{\pgfqpoint{1.336200in}{1.683183in}}%
\pgfpathlineto{\pgfqpoint{1.335365in}{1.679383in}}%
\pgfpathlineto{\pgfqpoint{1.334530in}{1.675467in}}%
\pgfpathlineto{\pgfqpoint{1.332376in}{1.675730in}}%
\pgfpathlineto{\pgfqpoint{1.330242in}{1.676025in}}%
\pgfpathlineto{\pgfqpoint{1.328129in}{1.676352in}}%
\pgfpathlineto{\pgfqpoint{1.326039in}{1.676710in}}%
\pgfpathlineto{\pgfqpoint{1.327300in}{1.680563in}}%
\pgfpathlineto{\pgfqpoint{1.328561in}{1.684300in}}%
\pgfpathlineto{\pgfqpoint{1.329821in}{1.687920in}}%
\pgfpathlineto{\pgfqpoint{1.331081in}{1.691424in}}%
\pgfpathlineto{\pgfqpoint{1.332752in}{1.691138in}}%
\pgfpathlineto{\pgfqpoint{1.334442in}{1.690878in}}%
\pgfpathlineto{\pgfqpoint{1.336148in}{1.690642in}}%
\pgfpathlineto{\pgfqpoint{1.337870in}{1.690432in}}%
\pgfpathclose%
\pgfusepath{fill}%
\end{pgfscope}%
\begin{pgfscope}%
\pgfpathrectangle{\pgfqpoint{0.329460in}{0.284240in}}{\pgfqpoint{1.989680in}{1.989680in}}%
\pgfusepath{clip}%
\pgfsetbuttcap%
\pgfsetroundjoin%
\definecolor{currentfill}{rgb}{0.268510,0.009605,0.335427}%
\pgfsetfillcolor{currentfill}%
\pgfsetlinewidth{0.000000pt}%
\definecolor{currentstroke}{rgb}{0.000000,0.000000,0.000000}%
\pgfsetstrokecolor{currentstroke}%
\pgfsetdash{}{0pt}%
\pgfpathmoveto{\pgfqpoint{1.292771in}{0.768509in}}%
\pgfpathlineto{\pgfqpoint{1.292364in}{0.765118in}}%
\pgfpathlineto{\pgfqpoint{1.291957in}{0.761927in}}%
\pgfpathlineto{\pgfqpoint{1.291549in}{0.758939in}}%
\pgfpathlineto{\pgfqpoint{1.291140in}{0.756159in}}%
\pgfpathlineto{\pgfqpoint{1.274368in}{0.757335in}}%
\pgfpathlineto{\pgfqpoint{1.257683in}{0.758798in}}%
\pgfpathlineto{\pgfqpoint{1.241103in}{0.760547in}}%
\pgfpathlineto{\pgfqpoint{1.224647in}{0.762579in}}%
\pgfpathlineto{\pgfqpoint{1.225508in}{0.765312in}}%
\pgfpathlineto{\pgfqpoint{1.226368in}{0.768254in}}%
\pgfpathlineto{\pgfqpoint{1.227225in}{0.771399in}}%
\pgfpathlineto{\pgfqpoint{1.228081in}{0.774743in}}%
\pgfpathlineto{\pgfqpoint{1.244091in}{0.772770in}}%
\pgfpathlineto{\pgfqpoint{1.260222in}{0.771072in}}%
\pgfpathlineto{\pgfqpoint{1.276454in}{0.769651in}}%
\pgfpathlineto{\pgfqpoint{1.292771in}{0.768509in}}%
\pgfpathclose%
\pgfusepath{fill}%
\end{pgfscope}%
\begin{pgfscope}%
\pgfpathrectangle{\pgfqpoint{0.329460in}{0.284240in}}{\pgfqpoint{1.989680in}{1.989680in}}%
\pgfusepath{clip}%
\pgfsetbuttcap%
\pgfsetroundjoin%
\definecolor{currentfill}{rgb}{0.268510,0.009605,0.335427}%
\pgfsetfillcolor{currentfill}%
\pgfsetlinewidth{0.000000pt}%
\definecolor{currentstroke}{rgb}{0.000000,0.000000,0.000000}%
\pgfsetstrokecolor{currentstroke}%
\pgfsetdash{}{0pt}%
\pgfpathmoveto{\pgfqpoint{1.149944in}{0.760252in}}%
\pgfpathlineto{\pgfqpoint{1.148619in}{0.759668in}}%
\pgfpathlineto{\pgfqpoint{1.147290in}{0.759338in}}%
\pgfpathlineto{\pgfqpoint{1.145957in}{0.759265in}}%
\pgfpathlineto{\pgfqpoint{1.144621in}{0.759456in}}%
\pgfpathlineto{\pgfqpoint{1.127773in}{0.763148in}}%
\pgfpathlineto{\pgfqpoint{1.111177in}{0.767125in}}%
\pgfpathlineto{\pgfqpoint{1.094850in}{0.771382in}}%
\pgfpathlineto{\pgfqpoint{1.078810in}{0.775914in}}%
\pgfpathlineto{\pgfqpoint{1.080568in}{0.775617in}}%
\pgfpathlineto{\pgfqpoint{1.082321in}{0.775583in}}%
\pgfpathlineto{\pgfqpoint{1.084069in}{0.775808in}}%
\pgfpathlineto{\pgfqpoint{1.085812in}{0.776285in}}%
\pgfpathlineto{\pgfqpoint{1.101444in}{0.771870in}}%
\pgfpathlineto{\pgfqpoint{1.117354in}{0.767723in}}%
\pgfpathlineto{\pgfqpoint{1.133527in}{0.763849in}}%
\pgfpathlineto{\pgfqpoint{1.149944in}{0.760252in}}%
\pgfpathclose%
\pgfusepath{fill}%
\end{pgfscope}%
\begin{pgfscope}%
\pgfpathrectangle{\pgfqpoint{0.329460in}{0.284240in}}{\pgfqpoint{1.989680in}{1.989680in}}%
\pgfusepath{clip}%
\pgfsetbuttcap%
\pgfsetroundjoin%
\definecolor{currentfill}{rgb}{0.636902,0.856542,0.216620}%
\pgfsetfillcolor{currentfill}%
\pgfsetlinewidth{0.000000pt}%
\definecolor{currentstroke}{rgb}{0.000000,0.000000,0.000000}%
\pgfsetstrokecolor{currentstroke}%
\pgfsetdash{}{0pt}%
\pgfpathmoveto{\pgfqpoint{1.269553in}{1.584507in}}%
\pgfpathlineto{\pgfqpoint{1.267512in}{1.578620in}}%
\pgfpathlineto{\pgfqpoint{1.265473in}{1.572639in}}%
\pgfpathlineto{\pgfqpoint{1.263433in}{1.566564in}}%
\pgfpathlineto{\pgfqpoint{1.261394in}{1.560398in}}%
\pgfpathlineto{\pgfqpoint{1.257469in}{1.561787in}}%
\pgfpathlineto{\pgfqpoint{1.253640in}{1.563235in}}%
\pgfpathlineto{\pgfqpoint{1.249910in}{1.564739in}}%
\pgfpathlineto{\pgfqpoint{1.246285in}{1.566299in}}%
\pgfpathlineto{\pgfqpoint{1.248664in}{1.572325in}}%
\pgfpathlineto{\pgfqpoint{1.251045in}{1.578260in}}%
\pgfpathlineto{\pgfqpoint{1.253426in}{1.584102in}}%
\pgfpathlineto{\pgfqpoint{1.255807in}{1.589850in}}%
\pgfpathlineto{\pgfqpoint{1.259106in}{1.588438in}}%
\pgfpathlineto{\pgfqpoint{1.262498in}{1.587076in}}%
\pgfpathlineto{\pgfqpoint{1.265982in}{1.585765in}}%
\pgfpathlineto{\pgfqpoint{1.269553in}{1.584507in}}%
\pgfpathclose%
\pgfusepath{fill}%
\end{pgfscope}%
\begin{pgfscope}%
\pgfpathrectangle{\pgfqpoint{0.329460in}{0.284240in}}{\pgfqpoint{1.989680in}{1.989680in}}%
\pgfusepath{clip}%
\pgfsetbuttcap%
\pgfsetroundjoin%
\definecolor{currentfill}{rgb}{0.166383,0.690856,0.496502}%
\pgfsetfillcolor{currentfill}%
\pgfsetlinewidth{0.000000pt}%
\definecolor{currentstroke}{rgb}{0.000000,0.000000,0.000000}%
\pgfsetstrokecolor{currentstroke}%
\pgfsetdash{}{0pt}%
\pgfpathmoveto{\pgfqpoint{1.470031in}{1.382665in}}%
\pgfpathlineto{\pgfqpoint{1.471767in}{1.374631in}}%
\pgfpathlineto{\pgfqpoint{1.473501in}{1.366550in}}%
\pgfpathlineto{\pgfqpoint{1.475235in}{1.358424in}}%
\pgfpathlineto{\pgfqpoint{1.476969in}{1.350256in}}%
\pgfpathlineto{\pgfqpoint{1.469995in}{1.348339in}}%
\pgfpathlineto{\pgfqpoint{1.462897in}{1.346531in}}%
\pgfpathlineto{\pgfqpoint{1.455681in}{1.344835in}}%
\pgfpathlineto{\pgfqpoint{1.448355in}{1.343253in}}%
\pgfpathlineto{\pgfqpoint{1.447015in}{1.351527in}}%
\pgfpathlineto{\pgfqpoint{1.445674in}{1.359758in}}%
\pgfpathlineto{\pgfqpoint{1.444333in}{1.367944in}}%
\pgfpathlineto{\pgfqpoint{1.442991in}{1.376083in}}%
\pgfpathlineto{\pgfqpoint{1.449913in}{1.377570in}}%
\pgfpathlineto{\pgfqpoint{1.456732in}{1.379164in}}%
\pgfpathlineto{\pgfqpoint{1.463441in}{1.380863in}}%
\pgfpathlineto{\pgfqpoint{1.470031in}{1.382665in}}%
\pgfpathclose%
\pgfusepath{fill}%
\end{pgfscope}%
\begin{pgfscope}%
\pgfpathrectangle{\pgfqpoint{0.329460in}{0.284240in}}{\pgfqpoint{1.989680in}{1.989680in}}%
\pgfusepath{clip}%
\pgfsetbuttcap%
\pgfsetroundjoin%
\definecolor{currentfill}{rgb}{0.122606,0.585371,0.546557}%
\pgfsetfillcolor{currentfill}%
\pgfsetlinewidth{0.000000pt}%
\definecolor{currentstroke}{rgb}{0.000000,0.000000,0.000000}%
\pgfsetstrokecolor{currentstroke}%
\pgfsetdash{}{0pt}%
\pgfpathmoveto{\pgfqpoint{1.284539in}{1.269007in}}%
\pgfpathlineto{\pgfqpoint{1.283712in}{1.260356in}}%
\pgfpathlineto{\pgfqpoint{1.282886in}{1.251685in}}%
\pgfpathlineto{\pgfqpoint{1.282060in}{1.242996in}}%
\pgfpathlineto{\pgfqpoint{1.281234in}{1.234292in}}%
\pgfpathlineto{\pgfqpoint{1.272203in}{1.235481in}}%
\pgfpathlineto{\pgfqpoint{1.263257in}{1.236814in}}%
\pgfpathlineto{\pgfqpoint{1.254403in}{1.238288in}}%
\pgfpathlineto{\pgfqpoint{1.245653in}{1.239902in}}%
\pgfpathlineto{\pgfqpoint{1.246898in}{1.248533in}}%
\pgfpathlineto{\pgfqpoint{1.248143in}{1.257148in}}%
\pgfpathlineto{\pgfqpoint{1.249389in}{1.265745in}}%
\pgfpathlineto{\pgfqpoint{1.250635in}{1.274323in}}%
\pgfpathlineto{\pgfqpoint{1.258973in}{1.272793in}}%
\pgfpathlineto{\pgfqpoint{1.267409in}{1.271396in}}%
\pgfpathlineto{\pgfqpoint{1.275934in}{1.270134in}}%
\pgfpathlineto{\pgfqpoint{1.284539in}{1.269007in}}%
\pgfpathclose%
\pgfusepath{fill}%
\end{pgfscope}%
\begin{pgfscope}%
\pgfpathrectangle{\pgfqpoint{0.329460in}{0.284240in}}{\pgfqpoint{1.989680in}{1.989680in}}%
\pgfusepath{clip}%
\pgfsetbuttcap%
\pgfsetroundjoin%
\definecolor{currentfill}{rgb}{0.762373,0.876424,0.137064}%
\pgfsetfillcolor{currentfill}%
\pgfsetlinewidth{0.000000pt}%
\definecolor{currentstroke}{rgb}{0.000000,0.000000,0.000000}%
\pgfsetstrokecolor{currentstroke}%
\pgfsetdash{}{0pt}%
\pgfpathmoveto{\pgfqpoint{1.285888in}{1.628052in}}%
\pgfpathlineto{\pgfqpoint{1.283845in}{1.622965in}}%
\pgfpathlineto{\pgfqpoint{1.281802in}{1.617774in}}%
\pgfpathlineto{\pgfqpoint{1.279760in}{1.612480in}}%
\pgfpathlineto{\pgfqpoint{1.277718in}{1.607084in}}%
\pgfpathlineto{\pgfqpoint{1.274503in}{1.608212in}}%
\pgfpathlineto{\pgfqpoint{1.271366in}{1.609387in}}%
\pgfpathlineto{\pgfqpoint{1.268310in}{1.610609in}}%
\pgfpathlineto{\pgfqpoint{1.265339in}{1.611875in}}%
\pgfpathlineto{\pgfqpoint{1.267724in}{1.617134in}}%
\pgfpathlineto{\pgfqpoint{1.270109in}{1.622291in}}%
\pgfpathlineto{\pgfqpoint{1.272494in}{1.627345in}}%
\pgfpathlineto{\pgfqpoint{1.274879in}{1.632296in}}%
\pgfpathlineto{\pgfqpoint{1.277522in}{1.631174in}}%
\pgfpathlineto{\pgfqpoint{1.280239in}{1.630092in}}%
\pgfpathlineto{\pgfqpoint{1.283029in}{1.629051in}}%
\pgfpathlineto{\pgfqpoint{1.285888in}{1.628052in}}%
\pgfpathclose%
\pgfusepath{fill}%
\end{pgfscope}%
\begin{pgfscope}%
\pgfpathrectangle{\pgfqpoint{0.329460in}{0.284240in}}{\pgfqpoint{1.989680in}{1.989680in}}%
\pgfusepath{clip}%
\pgfsetbuttcap%
\pgfsetroundjoin%
\definecolor{currentfill}{rgb}{0.699415,0.867117,0.175971}%
\pgfsetfillcolor{currentfill}%
\pgfsetlinewidth{0.000000pt}%
\definecolor{currentstroke}{rgb}{0.000000,0.000000,0.000000}%
\pgfsetstrokecolor{currentstroke}%
\pgfsetdash{}{0pt}%
\pgfpathmoveto{\pgfqpoint{1.277718in}{1.607084in}}%
\pgfpathlineto{\pgfqpoint{1.275676in}{1.601588in}}%
\pgfpathlineto{\pgfqpoint{1.273634in}{1.595992in}}%
\pgfpathlineto{\pgfqpoint{1.271593in}{1.590298in}}%
\pgfpathlineto{\pgfqpoint{1.269553in}{1.584507in}}%
\pgfpathlineto{\pgfqpoint{1.265982in}{1.585765in}}%
\pgfpathlineto{\pgfqpoint{1.262498in}{1.587076in}}%
\pgfpathlineto{\pgfqpoint{1.259106in}{1.588438in}}%
\pgfpathlineto{\pgfqpoint{1.255807in}{1.589850in}}%
\pgfpathlineto{\pgfqpoint{1.258189in}{1.595503in}}%
\pgfpathlineto{\pgfqpoint{1.260572in}{1.601059in}}%
\pgfpathlineto{\pgfqpoint{1.262956in}{1.606517in}}%
\pgfpathlineto{\pgfqpoint{1.265339in}{1.611875in}}%
\pgfpathlineto{\pgfqpoint{1.268310in}{1.610609in}}%
\pgfpathlineto{\pgfqpoint{1.271366in}{1.609387in}}%
\pgfpathlineto{\pgfqpoint{1.274503in}{1.608212in}}%
\pgfpathlineto{\pgfqpoint{1.277718in}{1.607084in}}%
\pgfpathclose%
\pgfusepath{fill}%
\end{pgfscope}%
\begin{pgfscope}%
\pgfpathrectangle{\pgfqpoint{0.329460in}{0.284240in}}{\pgfqpoint{1.989680in}{1.989680in}}%
\pgfusepath{clip}%
\pgfsetbuttcap%
\pgfsetroundjoin%
\definecolor{currentfill}{rgb}{0.935904,0.898570,0.108131}%
\pgfsetfillcolor{currentfill}%
\pgfsetlinewidth{0.000000pt}%
\definecolor{currentstroke}{rgb}{0.000000,0.000000,0.000000}%
\pgfsetstrokecolor{currentstroke}%
\pgfsetdash{}{0pt}%
\pgfpathmoveto{\pgfqpoint{1.366036in}{1.690618in}}%
\pgfpathlineto{\pgfqpoint{1.366967in}{1.687063in}}%
\pgfpathlineto{\pgfqpoint{1.367898in}{1.683392in}}%
\pgfpathlineto{\pgfqpoint{1.368829in}{1.679604in}}%
\pgfpathlineto{\pgfqpoint{1.369761in}{1.675700in}}%
\pgfpathlineto{\pgfqpoint{1.367605in}{1.675440in}}%
\pgfpathlineto{\pgfqpoint{1.365433in}{1.675212in}}%
\pgfpathlineto{\pgfqpoint{1.363247in}{1.675017in}}%
\pgfpathlineto{\pgfqpoint{1.361049in}{1.674854in}}%
\pgfpathlineto{\pgfqpoint{1.360554in}{1.678801in}}%
\pgfpathlineto{\pgfqpoint{1.360060in}{1.682632in}}%
\pgfpathlineto{\pgfqpoint{1.359565in}{1.686346in}}%
\pgfpathlineto{\pgfqpoint{1.359071in}{1.689944in}}%
\pgfpathlineto{\pgfqpoint{1.360829in}{1.690073in}}%
\pgfpathlineto{\pgfqpoint{1.362577in}{1.690229in}}%
\pgfpathlineto{\pgfqpoint{1.364313in}{1.690411in}}%
\pgfpathlineto{\pgfqpoint{1.366036in}{1.690618in}}%
\pgfpathclose%
\pgfusepath{fill}%
\end{pgfscope}%
\begin{pgfscope}%
\pgfpathrectangle{\pgfqpoint{0.329460in}{0.284240in}}{\pgfqpoint{1.989680in}{1.989680in}}%
\pgfusepath{clip}%
\pgfsetbuttcap%
\pgfsetroundjoin%
\definecolor{currentfill}{rgb}{0.935904,0.898570,0.108131}%
\pgfsetfillcolor{currentfill}%
\pgfsetlinewidth{0.000000pt}%
\definecolor{currentstroke}{rgb}{0.000000,0.000000,0.000000}%
\pgfsetstrokecolor{currentstroke}%
\pgfsetdash{}{0pt}%
\pgfpathmoveto{\pgfqpoint{1.344873in}{1.689850in}}%
\pgfpathlineto{\pgfqpoint{1.344477in}{1.686247in}}%
\pgfpathlineto{\pgfqpoint{1.344081in}{1.682527in}}%
\pgfpathlineto{\pgfqpoint{1.343685in}{1.678690in}}%
\pgfpathlineto{\pgfqpoint{1.343289in}{1.674737in}}%
\pgfpathlineto{\pgfqpoint{1.341082in}{1.674870in}}%
\pgfpathlineto{\pgfqpoint{1.338884in}{1.675037in}}%
\pgfpathlineto{\pgfqpoint{1.336700in}{1.675236in}}%
\pgfpathlineto{\pgfqpoint{1.334530in}{1.675467in}}%
\pgfpathlineto{\pgfqpoint{1.335365in}{1.679383in}}%
\pgfpathlineto{\pgfqpoint{1.336200in}{1.683183in}}%
\pgfpathlineto{\pgfqpoint{1.337035in}{1.686866in}}%
\pgfpathlineto{\pgfqpoint{1.337870in}{1.690432in}}%
\pgfpathlineto{\pgfqpoint{1.339605in}{1.690248in}}%
\pgfpathlineto{\pgfqpoint{1.341352in}{1.690089in}}%
\pgfpathlineto{\pgfqpoint{1.343108in}{1.689957in}}%
\pgfpathlineto{\pgfqpoint{1.344873in}{1.689850in}}%
\pgfpathclose%
\pgfusepath{fill}%
\end{pgfscope}%
\begin{pgfscope}%
\pgfpathrectangle{\pgfqpoint{0.329460in}{0.284240in}}{\pgfqpoint{1.989680in}{1.989680in}}%
\pgfusepath{clip}%
\pgfsetbuttcap%
\pgfsetroundjoin%
\definecolor{currentfill}{rgb}{0.855810,0.888601,0.097452}%
\pgfsetfillcolor{currentfill}%
\pgfsetlinewidth{0.000000pt}%
\definecolor{currentstroke}{rgb}{0.000000,0.000000,0.000000}%
\pgfsetstrokecolor{currentstroke}%
\pgfsetdash{}{0pt}%
\pgfpathmoveto{\pgfqpoint{1.402048in}{1.665549in}}%
\pgfpathlineto{\pgfqpoint{1.404171in}{1.661356in}}%
\pgfpathlineto{\pgfqpoint{1.406295in}{1.657052in}}%
\pgfpathlineto{\pgfqpoint{1.408418in}{1.652638in}}%
\pgfpathlineto{\pgfqpoint{1.410541in}{1.648113in}}%
\pgfpathlineto{\pgfqpoint{1.408032in}{1.647247in}}%
\pgfpathlineto{\pgfqpoint{1.405465in}{1.646418in}}%
\pgfpathlineto{\pgfqpoint{1.402843in}{1.645627in}}%
\pgfpathlineto{\pgfqpoint{1.400169in}{1.644876in}}%
\pgfpathlineto{\pgfqpoint{1.398416in}{1.649519in}}%
\pgfpathlineto{\pgfqpoint{1.396663in}{1.654051in}}%
\pgfpathlineto{\pgfqpoint{1.394910in}{1.658473in}}%
\pgfpathlineto{\pgfqpoint{1.393157in}{1.662784in}}%
\pgfpathlineto{\pgfqpoint{1.395449in}{1.663425in}}%
\pgfpathlineto{\pgfqpoint{1.397697in}{1.664100in}}%
\pgfpathlineto{\pgfqpoint{1.399897in}{1.664809in}}%
\pgfpathlineto{\pgfqpoint{1.402048in}{1.665549in}}%
\pgfpathclose%
\pgfusepath{fill}%
\end{pgfscope}%
\begin{pgfscope}%
\pgfpathrectangle{\pgfqpoint{0.329460in}{0.284240in}}{\pgfqpoint{1.989680in}{1.989680in}}%
\pgfusepath{clip}%
\pgfsetbuttcap%
\pgfsetroundjoin%
\definecolor{currentfill}{rgb}{0.935904,0.898570,0.108131}%
\pgfsetfillcolor{currentfill}%
\pgfsetlinewidth{0.000000pt}%
\definecolor{currentstroke}{rgb}{0.000000,0.000000,0.000000}%
\pgfsetstrokecolor{currentstroke}%
\pgfsetdash{}{0pt}%
\pgfpathmoveto{\pgfqpoint{1.359071in}{1.689944in}}%
\pgfpathlineto{\pgfqpoint{1.359565in}{1.686346in}}%
\pgfpathlineto{\pgfqpoint{1.360060in}{1.682632in}}%
\pgfpathlineto{\pgfqpoint{1.360554in}{1.678801in}}%
\pgfpathlineto{\pgfqpoint{1.361049in}{1.674854in}}%
\pgfpathlineto{\pgfqpoint{1.358840in}{1.674724in}}%
\pgfpathlineto{\pgfqpoint{1.356624in}{1.674627in}}%
\pgfpathlineto{\pgfqpoint{1.354402in}{1.674563in}}%
\pgfpathlineto{\pgfqpoint{1.352177in}{1.674531in}}%
\pgfpathlineto{\pgfqpoint{1.352127in}{1.678495in}}%
\pgfpathlineto{\pgfqpoint{1.352078in}{1.682342in}}%
\pgfpathlineto{\pgfqpoint{1.352028in}{1.686073in}}%
\pgfpathlineto{\pgfqpoint{1.351979in}{1.689686in}}%
\pgfpathlineto{\pgfqpoint{1.353757in}{1.689711in}}%
\pgfpathlineto{\pgfqpoint{1.355534in}{1.689762in}}%
\pgfpathlineto{\pgfqpoint{1.357305in}{1.689840in}}%
\pgfpathlineto{\pgfqpoint{1.359071in}{1.689944in}}%
\pgfpathclose%
\pgfusepath{fill}%
\end{pgfscope}%
\begin{pgfscope}%
\pgfpathrectangle{\pgfqpoint{0.329460in}{0.284240in}}{\pgfqpoint{1.989680in}{1.989680in}}%
\pgfusepath{clip}%
\pgfsetbuttcap%
\pgfsetroundjoin%
\definecolor{currentfill}{rgb}{0.133743,0.548535,0.553541}%
\pgfsetfillcolor{currentfill}%
\pgfsetlinewidth{0.000000pt}%
\definecolor{currentstroke}{rgb}{0.000000,0.000000,0.000000}%
\pgfsetstrokecolor{currentstroke}%
\pgfsetdash{}{0pt}%
\pgfpathmoveto{\pgfqpoint{1.429172in}{1.235342in}}%
\pgfpathlineto{\pgfqpoint{1.430093in}{1.226639in}}%
\pgfpathlineto{\pgfqpoint{1.431013in}{1.217926in}}%
\pgfpathlineto{\pgfqpoint{1.431933in}{1.209205in}}%
\pgfpathlineto{\pgfqpoint{1.432852in}{1.200478in}}%
\pgfpathlineto{\pgfqpoint{1.423387in}{1.199243in}}%
\pgfpathlineto{\pgfqpoint{1.413844in}{1.198160in}}%
\pgfpathlineto{\pgfqpoint{1.404234in}{1.197231in}}%
\pgfpathlineto{\pgfqpoint{1.394567in}{1.196456in}}%
\pgfpathlineto{\pgfqpoint{1.394079in}{1.205233in}}%
\pgfpathlineto{\pgfqpoint{1.393590in}{1.214005in}}%
\pgfpathlineto{\pgfqpoint{1.393101in}{1.222769in}}%
\pgfpathlineto{\pgfqpoint{1.392612in}{1.231522in}}%
\pgfpathlineto{\pgfqpoint{1.401843in}{1.232258in}}%
\pgfpathlineto{\pgfqpoint{1.411020in}{1.233141in}}%
\pgfpathlineto{\pgfqpoint{1.420133in}{1.234169in}}%
\pgfpathlineto{\pgfqpoint{1.429172in}{1.235342in}}%
\pgfpathclose%
\pgfusepath{fill}%
\end{pgfscope}%
\begin{pgfscope}%
\pgfpathrectangle{\pgfqpoint{0.329460in}{0.284240in}}{\pgfqpoint{1.989680in}{1.989680in}}%
\pgfusepath{clip}%
\pgfsetbuttcap%
\pgfsetroundjoin%
\definecolor{currentfill}{rgb}{0.935904,0.898570,0.108131}%
\pgfsetfillcolor{currentfill}%
\pgfsetlinewidth{0.000000pt}%
\definecolor{currentstroke}{rgb}{0.000000,0.000000,0.000000}%
\pgfsetstrokecolor{currentstroke}%
\pgfsetdash{}{0pt}%
\pgfpathmoveto{\pgfqpoint{1.351979in}{1.689686in}}%
\pgfpathlineto{\pgfqpoint{1.352028in}{1.686073in}}%
\pgfpathlineto{\pgfqpoint{1.352078in}{1.682342in}}%
\pgfpathlineto{\pgfqpoint{1.352127in}{1.678495in}}%
\pgfpathlineto{\pgfqpoint{1.352177in}{1.674531in}}%
\pgfpathlineto{\pgfqpoint{1.349951in}{1.674533in}}%
\pgfpathlineto{\pgfqpoint{1.347726in}{1.674568in}}%
\pgfpathlineto{\pgfqpoint{1.345505in}{1.674636in}}%
\pgfpathlineto{\pgfqpoint{1.343289in}{1.674737in}}%
\pgfpathlineto{\pgfqpoint{1.343685in}{1.678690in}}%
\pgfpathlineto{\pgfqpoint{1.344081in}{1.682527in}}%
\pgfpathlineto{\pgfqpoint{1.344477in}{1.686247in}}%
\pgfpathlineto{\pgfqpoint{1.344873in}{1.689850in}}%
\pgfpathlineto{\pgfqpoint{1.346644in}{1.689770in}}%
\pgfpathlineto{\pgfqpoint{1.348420in}{1.689716in}}%
\pgfpathlineto{\pgfqpoint{1.350199in}{1.689688in}}%
\pgfpathlineto{\pgfqpoint{1.351979in}{1.689686in}}%
\pgfpathclose%
\pgfusepath{fill}%
\end{pgfscope}%
\begin{pgfscope}%
\pgfpathrectangle{\pgfqpoint{0.329460in}{0.284240in}}{\pgfqpoint{1.989680in}{1.989680in}}%
\pgfusepath{clip}%
\pgfsetbuttcap%
\pgfsetroundjoin%
\definecolor{currentfill}{rgb}{0.896320,0.893616,0.096335}%
\pgfsetfillcolor{currentfill}%
\pgfsetlinewidth{0.000000pt}%
\definecolor{currentstroke}{rgb}{0.000000,0.000000,0.000000}%
\pgfsetstrokecolor{currentstroke}%
\pgfsetdash{}{0pt}%
\pgfpathmoveto{\pgfqpoint{1.386146in}{1.678895in}}%
\pgfpathlineto{\pgfqpoint{1.387899in}{1.675038in}}%
\pgfpathlineto{\pgfqpoint{1.389651in}{1.671067in}}%
\pgfpathlineto{\pgfqpoint{1.391404in}{1.666982in}}%
\pgfpathlineto{\pgfqpoint{1.393157in}{1.662784in}}%
\pgfpathlineto{\pgfqpoint{1.390822in}{1.662176in}}%
\pgfpathlineto{\pgfqpoint{1.388446in}{1.661604in}}%
\pgfpathlineto{\pgfqpoint{1.386033in}{1.661068in}}%
\pgfpathlineto{\pgfqpoint{1.383585in}{1.660567in}}%
\pgfpathlineto{\pgfqpoint{1.382231in}{1.664859in}}%
\pgfpathlineto{\pgfqpoint{1.380878in}{1.669038in}}%
\pgfpathlineto{\pgfqpoint{1.379525in}{1.673103in}}%
\pgfpathlineto{\pgfqpoint{1.378172in}{1.677054in}}%
\pgfpathlineto{\pgfqpoint{1.380212in}{1.677470in}}%
\pgfpathlineto{\pgfqpoint{1.382222in}{1.677915in}}%
\pgfpathlineto{\pgfqpoint{1.384201in}{1.678391in}}%
\pgfpathlineto{\pgfqpoint{1.386146in}{1.678895in}}%
\pgfpathclose%
\pgfusepath{fill}%
\end{pgfscope}%
\begin{pgfscope}%
\pgfpathrectangle{\pgfqpoint{0.329460in}{0.284240in}}{\pgfqpoint{1.989680in}{1.989680in}}%
\pgfusepath{clip}%
\pgfsetbuttcap%
\pgfsetroundjoin%
\definecolor{currentfill}{rgb}{0.344074,0.780029,0.397381}%
\pgfsetfillcolor{currentfill}%
\pgfsetlinewidth{0.000000pt}%
\definecolor{currentstroke}{rgb}{0.000000,0.000000,0.000000}%
\pgfsetstrokecolor{currentstroke}%
\pgfsetdash{}{0pt}%
\pgfpathmoveto{\pgfqpoint{1.258048in}{1.473400in}}%
\pgfpathlineto{\pgfqpoint{1.256390in}{1.466056in}}%
\pgfpathlineto{\pgfqpoint{1.254733in}{1.458642in}}%
\pgfpathlineto{\pgfqpoint{1.253077in}{1.451160in}}%
\pgfpathlineto{\pgfqpoint{1.251421in}{1.443611in}}%
\pgfpathlineto{\pgfqpoint{1.245610in}{1.445196in}}%
\pgfpathlineto{\pgfqpoint{1.239909in}{1.446869in}}%
\pgfpathlineto{\pgfqpoint{1.234323in}{1.448629in}}%
\pgfpathlineto{\pgfqpoint{1.228859in}{1.450473in}}%
\pgfpathlineto{\pgfqpoint{1.230887in}{1.457900in}}%
\pgfpathlineto{\pgfqpoint{1.232916in}{1.465260in}}%
\pgfpathlineto{\pgfqpoint{1.234946in}{1.472553in}}%
\pgfpathlineto{\pgfqpoint{1.236976in}{1.479776in}}%
\pgfpathlineto{\pgfqpoint{1.242080in}{1.478062in}}%
\pgfpathlineto{\pgfqpoint{1.247297in}{1.476427in}}%
\pgfpathlineto{\pgfqpoint{1.252621in}{1.474872in}}%
\pgfpathlineto{\pgfqpoint{1.258048in}{1.473400in}}%
\pgfpathclose%
\pgfusepath{fill}%
\end{pgfscope}%
\begin{pgfscope}%
\pgfpathrectangle{\pgfqpoint{0.329460in}{0.284240in}}{\pgfqpoint{1.989680in}{1.989680in}}%
\pgfusepath{clip}%
\pgfsetbuttcap%
\pgfsetroundjoin%
\definecolor{currentfill}{rgb}{0.120081,0.622161,0.534946}%
\pgfsetfillcolor{currentfill}%
\pgfsetlinewidth{0.000000pt}%
\definecolor{currentstroke}{rgb}{0.000000,0.000000,0.000000}%
\pgfsetstrokecolor{currentstroke}%
\pgfsetdash{}{0pt}%
\pgfpathmoveto{\pgfqpoint{1.453713in}{1.309775in}}%
\pgfpathlineto{\pgfqpoint{1.455051in}{1.301321in}}%
\pgfpathlineto{\pgfqpoint{1.456389in}{1.292838in}}%
\pgfpathlineto{\pgfqpoint{1.457726in}{1.284328in}}%
\pgfpathlineto{\pgfqpoint{1.459063in}{1.275792in}}%
\pgfpathlineto{\pgfqpoint{1.450819in}{1.274146in}}%
\pgfpathlineto{\pgfqpoint{1.442469in}{1.272631in}}%
\pgfpathlineto{\pgfqpoint{1.434023in}{1.271249in}}%
\pgfpathlineto{\pgfqpoint{1.425489in}{1.270002in}}%
\pgfpathlineto{\pgfqpoint{1.424567in}{1.278616in}}%
\pgfpathlineto{\pgfqpoint{1.423645in}{1.287206in}}%
\pgfpathlineto{\pgfqpoint{1.422723in}{1.295769in}}%
\pgfpathlineto{\pgfqpoint{1.421801in}{1.304302in}}%
\pgfpathlineto{\pgfqpoint{1.429912in}{1.305481in}}%
\pgfpathlineto{\pgfqpoint{1.437940in}{1.306787in}}%
\pgfpathlineto{\pgfqpoint{1.445877in}{1.308219in}}%
\pgfpathlineto{\pgfqpoint{1.453713in}{1.309775in}}%
\pgfpathclose%
\pgfusepath{fill}%
\end{pgfscope}%
\begin{pgfscope}%
\pgfpathrectangle{\pgfqpoint{0.329460in}{0.284240in}}{\pgfqpoint{1.989680in}{1.989680in}}%
\pgfusepath{clip}%
\pgfsetbuttcap%
\pgfsetroundjoin%
\definecolor{currentfill}{rgb}{0.147607,0.511733,0.557049}%
\pgfsetfillcolor{currentfill}%
\pgfsetlinewidth{0.000000pt}%
\definecolor{currentstroke}{rgb}{0.000000,0.000000,0.000000}%
\pgfsetstrokecolor{currentstroke}%
\pgfsetdash{}{0pt}%
\pgfpathmoveto{\pgfqpoint{1.394567in}{1.196456in}}%
\pgfpathlineto{\pgfqpoint{1.395056in}{1.187676in}}%
\pgfpathlineto{\pgfqpoint{1.395545in}{1.178896in}}%
\pgfpathlineto{\pgfqpoint{1.396033in}{1.170118in}}%
\pgfpathlineto{\pgfqpoint{1.396522in}{1.161345in}}%
\pgfpathlineto{\pgfqpoint{1.386371in}{1.160694in}}%
\pgfpathlineto{\pgfqpoint{1.376183in}{1.160208in}}%
\pgfpathlineto{\pgfqpoint{1.365968in}{1.159887in}}%
\pgfpathlineto{\pgfqpoint{1.355737in}{1.159731in}}%
\pgfpathlineto{\pgfqpoint{1.355688in}{1.168523in}}%
\pgfpathlineto{\pgfqpoint{1.355639in}{1.177321in}}%
\pgfpathlineto{\pgfqpoint{1.355590in}{1.186121in}}%
\pgfpathlineto{\pgfqpoint{1.355541in}{1.194920in}}%
\pgfpathlineto{\pgfqpoint{1.365331in}{1.195069in}}%
\pgfpathlineto{\pgfqpoint{1.375105in}{1.195375in}}%
\pgfpathlineto{\pgfqpoint{1.384854in}{1.195837in}}%
\pgfpathlineto{\pgfqpoint{1.394567in}{1.196456in}}%
\pgfpathclose%
\pgfusepath{fill}%
\end{pgfscope}%
\begin{pgfscope}%
\pgfpathrectangle{\pgfqpoint{0.329460in}{0.284240in}}{\pgfqpoint{1.989680in}{1.989680in}}%
\pgfusepath{clip}%
\pgfsetbuttcap%
\pgfsetroundjoin%
\definecolor{currentfill}{rgb}{0.412913,0.803041,0.357269}%
\pgfsetfillcolor{currentfill}%
\pgfsetlinewidth{0.000000pt}%
\definecolor{currentstroke}{rgb}{0.000000,0.000000,0.000000}%
\pgfsetstrokecolor{currentstroke}%
\pgfsetdash{}{0pt}%
\pgfpathmoveto{\pgfqpoint{1.461394in}{1.509403in}}%
\pgfpathlineto{\pgfqpoint{1.463506in}{1.502506in}}%
\pgfpathlineto{\pgfqpoint{1.465617in}{1.495533in}}%
\pgfpathlineto{\pgfqpoint{1.467727in}{1.488485in}}%
\pgfpathlineto{\pgfqpoint{1.469837in}{1.481364in}}%
\pgfpathlineto{\pgfqpoint{1.464838in}{1.479581in}}%
\pgfpathlineto{\pgfqpoint{1.459721in}{1.477876in}}%
\pgfpathlineto{\pgfqpoint{1.454492in}{1.476250in}}%
\pgfpathlineto{\pgfqpoint{1.449156in}{1.474705in}}%
\pgfpathlineto{\pgfqpoint{1.447412in}{1.481952in}}%
\pgfpathlineto{\pgfqpoint{1.445668in}{1.489126in}}%
\pgfpathlineto{\pgfqpoint{1.443923in}{1.496225in}}%
\pgfpathlineto{\pgfqpoint{1.442178in}{1.503247in}}%
\pgfpathlineto{\pgfqpoint{1.447136in}{1.504676in}}%
\pgfpathlineto{\pgfqpoint{1.451994in}{1.506179in}}%
\pgfpathlineto{\pgfqpoint{1.456749in}{1.507755in}}%
\pgfpathlineto{\pgfqpoint{1.461394in}{1.509403in}}%
\pgfpathclose%
\pgfusepath{fill}%
\end{pgfscope}%
\begin{pgfscope}%
\pgfpathrectangle{\pgfqpoint{0.329460in}{0.284240in}}{\pgfqpoint{1.989680in}{1.989680in}}%
\pgfusepath{clip}%
\pgfsetbuttcap%
\pgfsetroundjoin%
\definecolor{currentfill}{rgb}{0.166383,0.690856,0.496502}%
\pgfsetfillcolor{currentfill}%
\pgfsetlinewidth{0.000000pt}%
\definecolor{currentstroke}{rgb}{0.000000,0.000000,0.000000}%
\pgfsetstrokecolor{currentstroke}%
\pgfsetdash{}{0pt}%
\pgfpathmoveto{\pgfqpoint{1.265619in}{1.374853in}}%
\pgfpathlineto{\pgfqpoint{1.264368in}{1.366694in}}%
\pgfpathlineto{\pgfqpoint{1.263117in}{1.358488in}}%
\pgfpathlineto{\pgfqpoint{1.261867in}{1.350237in}}%
\pgfpathlineto{\pgfqpoint{1.260617in}{1.341944in}}%
\pgfpathlineto{\pgfqpoint{1.253200in}{1.343423in}}%
\pgfpathlineto{\pgfqpoint{1.245887in}{1.345018in}}%
\pgfpathlineto{\pgfqpoint{1.238684in}{1.346727in}}%
\pgfpathlineto{\pgfqpoint{1.231599in}{1.348546in}}%
\pgfpathlineto{\pgfqpoint{1.233247in}{1.356740in}}%
\pgfpathlineto{\pgfqpoint{1.234896in}{1.364891in}}%
\pgfpathlineto{\pgfqpoint{1.236546in}{1.372998in}}%
\pgfpathlineto{\pgfqpoint{1.238196in}{1.381058in}}%
\pgfpathlineto{\pgfqpoint{1.244892in}{1.379348in}}%
\pgfpathlineto{\pgfqpoint{1.251699in}{1.377742in}}%
\pgfpathlineto{\pgfqpoint{1.258610in}{1.376243in}}%
\pgfpathlineto{\pgfqpoint{1.265619in}{1.374853in}}%
\pgfpathclose%
\pgfusepath{fill}%
\end{pgfscope}%
\begin{pgfscope}%
\pgfpathrectangle{\pgfqpoint{0.329460in}{0.284240in}}{\pgfqpoint{1.989680in}{1.989680in}}%
\pgfusepath{clip}%
\pgfsetbuttcap%
\pgfsetroundjoin%
\definecolor{currentfill}{rgb}{0.896320,0.893616,0.096335}%
\pgfsetfillcolor{currentfill}%
\pgfsetlinewidth{0.000000pt}%
\definecolor{currentstroke}{rgb}{0.000000,0.000000,0.000000}%
\pgfsetstrokecolor{currentstroke}%
\pgfsetdash{}{0pt}%
\pgfpathmoveto{\pgfqpoint{1.326039in}{1.676710in}}%
\pgfpathlineto{\pgfqpoint{1.324778in}{1.672742in}}%
\pgfpathlineto{\pgfqpoint{1.323517in}{1.668659in}}%
\pgfpathlineto{\pgfqpoint{1.322256in}{1.664463in}}%
\pgfpathlineto{\pgfqpoint{1.320994in}{1.660153in}}%
\pgfpathlineto{\pgfqpoint{1.318516in}{1.660621in}}%
\pgfpathlineto{\pgfqpoint{1.316072in}{1.661125in}}%
\pgfpathlineto{\pgfqpoint{1.313663in}{1.661666in}}%
\pgfpathlineto{\pgfqpoint{1.311292in}{1.662242in}}%
\pgfpathlineto{\pgfqpoint{1.312958in}{1.666463in}}%
\pgfpathlineto{\pgfqpoint{1.314624in}{1.670571in}}%
\pgfpathlineto{\pgfqpoint{1.316291in}{1.674565in}}%
\pgfpathlineto{\pgfqpoint{1.317956in}{1.678445in}}%
\pgfpathlineto{\pgfqpoint{1.319932in}{1.677967in}}%
\pgfpathlineto{\pgfqpoint{1.321939in}{1.677518in}}%
\pgfpathlineto{\pgfqpoint{1.323975in}{1.677099in}}%
\pgfpathlineto{\pgfqpoint{1.326039in}{1.676710in}}%
\pgfpathclose%
\pgfusepath{fill}%
\end{pgfscope}%
\begin{pgfscope}%
\pgfpathrectangle{\pgfqpoint{0.329460in}{0.284240in}}{\pgfqpoint{1.989680in}{1.989680in}}%
\pgfusepath{clip}%
\pgfsetbuttcap%
\pgfsetroundjoin%
\definecolor{currentfill}{rgb}{0.855810,0.888601,0.097452}%
\pgfsetfillcolor{currentfill}%
\pgfsetlinewidth{0.000000pt}%
\definecolor{currentstroke}{rgb}{0.000000,0.000000,0.000000}%
\pgfsetstrokecolor{currentstroke}%
\pgfsetdash{}{0pt}%
\pgfpathmoveto{\pgfqpoint{1.311292in}{1.662242in}}%
\pgfpathlineto{\pgfqpoint{1.309625in}{1.657909in}}%
\pgfpathlineto{\pgfqpoint{1.307959in}{1.653464in}}%
\pgfpathlineto{\pgfqpoint{1.306292in}{1.648908in}}%
\pgfpathlineto{\pgfqpoint{1.304626in}{1.644242in}}%
\pgfpathlineto{\pgfqpoint{1.301907in}{1.644958in}}%
\pgfpathlineto{\pgfqpoint{1.299238in}{1.645713in}}%
\pgfpathlineto{\pgfqpoint{1.296622in}{1.646508in}}%
\pgfpathlineto{\pgfqpoint{1.294062in}{1.647341in}}%
\pgfpathlineto{\pgfqpoint{1.296105in}{1.651894in}}%
\pgfpathlineto{\pgfqpoint{1.298149in}{1.656336in}}%
\pgfpathlineto{\pgfqpoint{1.300193in}{1.660669in}}%
\pgfpathlineto{\pgfqpoint{1.302236in}{1.664889in}}%
\pgfpathlineto{\pgfqpoint{1.304432in}{1.664177in}}%
\pgfpathlineto{\pgfqpoint{1.306674in}{1.663499in}}%
\pgfpathlineto{\pgfqpoint{1.308962in}{1.662853in}}%
\pgfpathlineto{\pgfqpoint{1.311292in}{1.662242in}}%
\pgfpathclose%
\pgfusepath{fill}%
\end{pgfscope}%
\begin{pgfscope}%
\pgfpathrectangle{\pgfqpoint{0.329460in}{0.284240in}}{\pgfqpoint{1.989680in}{1.989680in}}%
\pgfusepath{clip}%
\pgfsetbuttcap%
\pgfsetroundjoin%
\definecolor{currentfill}{rgb}{0.147607,0.511733,0.557049}%
\pgfsetfillcolor{currentfill}%
\pgfsetlinewidth{0.000000pt}%
\definecolor{currentstroke}{rgb}{0.000000,0.000000,0.000000}%
\pgfsetstrokecolor{currentstroke}%
\pgfsetdash{}{0pt}%
\pgfpathmoveto{\pgfqpoint{1.355541in}{1.194920in}}%
\pgfpathlineto{\pgfqpoint{1.355590in}{1.186121in}}%
\pgfpathlineto{\pgfqpoint{1.355639in}{1.177321in}}%
\pgfpathlineto{\pgfqpoint{1.355688in}{1.168523in}}%
\pgfpathlineto{\pgfqpoint{1.355737in}{1.159731in}}%
\pgfpathlineto{\pgfqpoint{1.345501in}{1.159740in}}%
\pgfpathlineto{\pgfqpoint{1.335271in}{1.159914in}}%
\pgfpathlineto{\pgfqpoint{1.325059in}{1.160254in}}%
\pgfpathlineto{\pgfqpoint{1.314874in}{1.160759in}}%
\pgfpathlineto{\pgfqpoint{1.315265in}{1.169539in}}%
\pgfpathlineto{\pgfqpoint{1.315657in}{1.178324in}}%
\pgfpathlineto{\pgfqpoint{1.316048in}{1.187111in}}%
\pgfpathlineto{\pgfqpoint{1.316440in}{1.195898in}}%
\pgfpathlineto{\pgfqpoint{1.326185in}{1.195418in}}%
\pgfpathlineto{\pgfqpoint{1.335958in}{1.195095in}}%
\pgfpathlineto{\pgfqpoint{1.345746in}{1.194929in}}%
\pgfpathlineto{\pgfqpoint{1.355541in}{1.194920in}}%
\pgfpathclose%
\pgfusepath{fill}%
\end{pgfscope}%
\begin{pgfscope}%
\pgfpathrectangle{\pgfqpoint{0.329460in}{0.284240in}}{\pgfqpoint{1.989680in}{1.989680in}}%
\pgfusepath{clip}%
\pgfsetbuttcap%
\pgfsetroundjoin%
\definecolor{currentfill}{rgb}{0.133743,0.548535,0.553541}%
\pgfsetfillcolor{currentfill}%
\pgfsetlinewidth{0.000000pt}%
\definecolor{currentstroke}{rgb}{0.000000,0.000000,0.000000}%
\pgfsetstrokecolor{currentstroke}%
\pgfsetdash{}{0pt}%
\pgfpathmoveto{\pgfqpoint{1.318006in}{1.230993in}}%
\pgfpathlineto{\pgfqpoint{1.317615in}{1.222232in}}%
\pgfpathlineto{\pgfqpoint{1.317223in}{1.213461in}}%
\pgfpathlineto{\pgfqpoint{1.316831in}{1.204682in}}%
\pgfpathlineto{\pgfqpoint{1.316440in}{1.195898in}}%
\pgfpathlineto{\pgfqpoint{1.306731in}{1.196535in}}%
\pgfpathlineto{\pgfqpoint{1.297070in}{1.197326in}}%
\pgfpathlineto{\pgfqpoint{1.287468in}{1.198273in}}%
\pgfpathlineto{\pgfqpoint{1.277933in}{1.199373in}}%
\pgfpathlineto{\pgfqpoint{1.278758in}{1.208113in}}%
\pgfpathlineto{\pgfqpoint{1.279583in}{1.216848in}}%
\pgfpathlineto{\pgfqpoint{1.280409in}{1.225575in}}%
\pgfpathlineto{\pgfqpoint{1.281234in}{1.234292in}}%
\pgfpathlineto{\pgfqpoint{1.290340in}{1.233248in}}%
\pgfpathlineto{\pgfqpoint{1.299510in}{1.232349in}}%
\pgfpathlineto{\pgfqpoint{1.308736in}{1.231597in}}%
\pgfpathlineto{\pgfqpoint{1.318006in}{1.230993in}}%
\pgfpathclose%
\pgfusepath{fill}%
\end{pgfscope}%
\begin{pgfscope}%
\pgfpathrectangle{\pgfqpoint{0.329460in}{0.284240in}}{\pgfqpoint{1.989680in}{1.989680in}}%
\pgfusepath{clip}%
\pgfsetbuttcap%
\pgfsetroundjoin%
\definecolor{currentfill}{rgb}{0.267004,0.004874,0.329415}%
\pgfsetfillcolor{currentfill}%
\pgfsetlinewidth{0.000000pt}%
\definecolor{currentstroke}{rgb}{0.000000,0.000000,0.000000}%
\pgfsetstrokecolor{currentstroke}%
\pgfsetdash{}{0pt}%
\pgfpathmoveto{\pgfqpoint{1.561392in}{0.768124in}}%
\pgfpathlineto{\pgfqpoint{1.562797in}{0.766596in}}%
\pgfpathlineto{\pgfqpoint{1.564206in}{0.765302in}}%
\pgfpathlineto{\pgfqpoint{1.565619in}{0.764247in}}%
\pgfpathlineto{\pgfqpoint{1.567036in}{0.763435in}}%
\pgfpathlineto{\pgfqpoint{1.550593in}{0.759870in}}%
\pgfpathlineto{\pgfqpoint{1.533927in}{0.756587in}}%
\pgfpathlineto{\pgfqpoint{1.517054in}{0.753591in}}%
\pgfpathlineto{\pgfqpoint{1.499995in}{0.750885in}}%
\pgfpathlineto{\pgfqpoint{1.499017in}{0.751781in}}%
\pgfpathlineto{\pgfqpoint{1.498041in}{0.752920in}}%
\pgfpathlineto{\pgfqpoint{1.497068in}{0.754298in}}%
\pgfpathlineto{\pgfqpoint{1.496097in}{0.755912in}}%
\pgfpathlineto{\pgfqpoint{1.512711in}{0.758544in}}%
\pgfpathlineto{\pgfqpoint{1.529143in}{0.761460in}}%
\pgfpathlineto{\pgfqpoint{1.545376in}{0.764655in}}%
\pgfpathlineto{\pgfqpoint{1.561392in}{0.768124in}}%
\pgfpathclose%
\pgfusepath{fill}%
\end{pgfscope}%
\begin{pgfscope}%
\pgfpathrectangle{\pgfqpoint{0.329460in}{0.284240in}}{\pgfqpoint{1.989680in}{1.989680in}}%
\pgfusepath{clip}%
\pgfsetbuttcap%
\pgfsetroundjoin%
\definecolor{currentfill}{rgb}{0.277941,0.056324,0.381191}%
\pgfsetfillcolor{currentfill}%
\pgfsetlinewidth{0.000000pt}%
\definecolor{currentstroke}{rgb}{0.000000,0.000000,0.000000}%
\pgfsetstrokecolor{currentstroke}%
\pgfsetdash{}{0pt}%
\pgfpathmoveto{\pgfqpoint{1.071723in}{0.779827in}}%
\pgfpathlineto{\pgfqpoint{1.069936in}{0.781512in}}%
\pgfpathlineto{\pgfqpoint{1.068144in}{0.783491in}}%
\pgfpathlineto{\pgfqpoint{1.066345in}{0.785768in}}%
\pgfpathlineto{\pgfqpoint{1.064540in}{0.788350in}}%
\pgfpathlineto{\pgfqpoint{1.047992in}{0.793393in}}%
\pgfpathlineto{\pgfqpoint{1.031784in}{0.798711in}}%
\pgfpathlineto{\pgfqpoint{1.015934in}{0.804299in}}%
\pgfpathlineto{\pgfqpoint{1.000458in}{0.810149in}}%
\pgfpathlineto{\pgfqpoint{1.002660in}{0.807439in}}%
\pgfpathlineto{\pgfqpoint{1.004854in}{0.805033in}}%
\pgfpathlineto{\pgfqpoint{1.007040in}{0.802924in}}%
\pgfpathlineto{\pgfqpoint{1.009219in}{0.801108in}}%
\pgfpathlineto{\pgfqpoint{1.024315in}{0.795397in}}%
\pgfpathlineto{\pgfqpoint{1.039775in}{0.789941in}}%
\pgfpathlineto{\pgfqpoint{1.055584in}{0.784749in}}%
\pgfpathlineto{\pgfqpoint{1.071723in}{0.779827in}}%
\pgfpathclose%
\pgfusepath{fill}%
\end{pgfscope}%
\begin{pgfscope}%
\pgfpathrectangle{\pgfqpoint{0.329460in}{0.284240in}}{\pgfqpoint{1.989680in}{1.989680in}}%
\pgfusepath{clip}%
\pgfsetbuttcap%
\pgfsetroundjoin%
\definecolor{currentfill}{rgb}{0.220124,0.725509,0.466226}%
\pgfsetfillcolor{currentfill}%
\pgfsetlinewidth{0.000000pt}%
\definecolor{currentstroke}{rgb}{0.000000,0.000000,0.000000}%
\pgfsetstrokecolor{currentstroke}%
\pgfsetdash{}{0pt}%
\pgfpathmoveto{\pgfqpoint{1.463083in}{1.414294in}}%
\pgfpathlineto{\pgfqpoint{1.464821in}{1.406467in}}%
\pgfpathlineto{\pgfqpoint{1.466559in}{1.398585in}}%
\pgfpathlineto{\pgfqpoint{1.468295in}{1.390651in}}%
\pgfpathlineto{\pgfqpoint{1.470031in}{1.382665in}}%
\pgfpathlineto{\pgfqpoint{1.463441in}{1.380863in}}%
\pgfpathlineto{\pgfqpoint{1.456732in}{1.379164in}}%
\pgfpathlineto{\pgfqpoint{1.449913in}{1.377570in}}%
\pgfpathlineto{\pgfqpoint{1.442991in}{1.376083in}}%
\pgfpathlineto{\pgfqpoint{1.441648in}{1.384174in}}%
\pgfpathlineto{\pgfqpoint{1.440305in}{1.392213in}}%
\pgfpathlineto{\pgfqpoint{1.438962in}{1.400199in}}%
\pgfpathlineto{\pgfqpoint{1.437618in}{1.408130in}}%
\pgfpathlineto{\pgfqpoint{1.444137in}{1.409523in}}%
\pgfpathlineto{\pgfqpoint{1.450558in}{1.411015in}}%
\pgfpathlineto{\pgfqpoint{1.456876in}{1.412607in}}%
\pgfpathlineto{\pgfqpoint{1.463083in}{1.414294in}}%
\pgfpathclose%
\pgfusepath{fill}%
\end{pgfscope}%
\begin{pgfscope}%
\pgfpathrectangle{\pgfqpoint{0.329460in}{0.284240in}}{\pgfqpoint{1.989680in}{1.989680in}}%
\pgfusepath{clip}%
\pgfsetbuttcap%
\pgfsetroundjoin%
\definecolor{currentfill}{rgb}{0.120081,0.622161,0.534946}%
\pgfsetfillcolor{currentfill}%
\pgfsetlinewidth{0.000000pt}%
\definecolor{currentstroke}{rgb}{0.000000,0.000000,0.000000}%
\pgfsetstrokecolor{currentstroke}%
\pgfsetdash{}{0pt}%
\pgfpathmoveto{\pgfqpoint{1.287848in}{1.303362in}}%
\pgfpathlineto{\pgfqpoint{1.287020in}{1.294815in}}%
\pgfpathlineto{\pgfqpoint{1.286193in}{1.286239in}}%
\pgfpathlineto{\pgfqpoint{1.285366in}{1.277635in}}%
\pgfpathlineto{\pgfqpoint{1.284539in}{1.269007in}}%
\pgfpathlineto{\pgfqpoint{1.275934in}{1.270134in}}%
\pgfpathlineto{\pgfqpoint{1.267409in}{1.271396in}}%
\pgfpathlineto{\pgfqpoint{1.258973in}{1.272793in}}%
\pgfpathlineto{\pgfqpoint{1.250635in}{1.274323in}}%
\pgfpathlineto{\pgfqpoint{1.251881in}{1.282878in}}%
\pgfpathlineto{\pgfqpoint{1.253128in}{1.291408in}}%
\pgfpathlineto{\pgfqpoint{1.254375in}{1.299912in}}%
\pgfpathlineto{\pgfqpoint{1.255623in}{1.308386in}}%
\pgfpathlineto{\pgfqpoint{1.263548in}{1.306940in}}%
\pgfpathlineto{\pgfqpoint{1.271567in}{1.305620in}}%
\pgfpathlineto{\pgfqpoint{1.279669in}{1.304427in}}%
\pgfpathlineto{\pgfqpoint{1.287848in}{1.303362in}}%
\pgfpathclose%
\pgfusepath{fill}%
\end{pgfscope}%
\begin{pgfscope}%
\pgfpathrectangle{\pgfqpoint{0.329460in}{0.284240in}}{\pgfqpoint{1.989680in}{1.989680in}}%
\pgfusepath{clip}%
\pgfsetbuttcap%
\pgfsetroundjoin%
\definecolor{currentfill}{rgb}{0.814576,0.883393,0.110347}%
\pgfsetfillcolor{currentfill}%
\pgfsetlinewidth{0.000000pt}%
\definecolor{currentstroke}{rgb}{0.000000,0.000000,0.000000}%
\pgfsetstrokecolor{currentstroke}%
\pgfsetdash{}{0pt}%
\pgfpathmoveto{\pgfqpoint{1.410541in}{1.648113in}}%
\pgfpathlineto{\pgfqpoint{1.412664in}{1.643480in}}%
\pgfpathlineto{\pgfqpoint{1.414787in}{1.638740in}}%
\pgfpathlineto{\pgfqpoint{1.416909in}{1.633892in}}%
\pgfpathlineto{\pgfqpoint{1.419032in}{1.628938in}}%
\pgfpathlineto{\pgfqpoint{1.416165in}{1.627944in}}%
\pgfpathlineto{\pgfqpoint{1.413232in}{1.626993in}}%
\pgfpathlineto{\pgfqpoint{1.410236in}{1.626087in}}%
\pgfpathlineto{\pgfqpoint{1.407180in}{1.625225in}}%
\pgfpathlineto{\pgfqpoint{1.405427in}{1.630298in}}%
\pgfpathlineto{\pgfqpoint{1.403675in}{1.635265in}}%
\pgfpathlineto{\pgfqpoint{1.401922in}{1.640125in}}%
\pgfpathlineto{\pgfqpoint{1.400169in}{1.644876in}}%
\pgfpathlineto{\pgfqpoint{1.402843in}{1.645627in}}%
\pgfpathlineto{\pgfqpoint{1.405465in}{1.646418in}}%
\pgfpathlineto{\pgfqpoint{1.408032in}{1.647247in}}%
\pgfpathlineto{\pgfqpoint{1.410541in}{1.648113in}}%
\pgfpathclose%
\pgfusepath{fill}%
\end{pgfscope}%
\begin{pgfscope}%
\pgfpathrectangle{\pgfqpoint{0.329460in}{0.284240in}}{\pgfqpoint{1.989680in}{1.989680in}}%
\pgfusepath{clip}%
\pgfsetbuttcap%
\pgfsetroundjoin%
\definecolor{currentfill}{rgb}{0.260571,0.246922,0.522828}%
\pgfsetfillcolor{currentfill}%
\pgfsetlinewidth{0.000000pt}%
\definecolor{currentstroke}{rgb}{0.000000,0.000000,0.000000}%
\pgfsetstrokecolor{currentstroke}%
\pgfsetdash{}{0pt}%
\pgfpathmoveto{\pgfqpoint{1.856531in}{0.934884in}}%
\pgfpathlineto{\pgfqpoint{1.859608in}{0.942395in}}%
\pgfpathlineto{\pgfqpoint{1.862700in}{0.950291in}}%
\pgfpathlineto{\pgfqpoint{1.865806in}{0.958577in}}%
\pgfpathlineto{\pgfqpoint{1.868928in}{0.967261in}}%
\pgfpathlineto{\pgfqpoint{1.856306in}{0.958740in}}%
\pgfpathlineto{\pgfqpoint{1.843130in}{0.950424in}}%
\pgfpathlineto{\pgfqpoint{1.829412in}{0.942321in}}%
\pgfpathlineto{\pgfqpoint{1.815165in}{0.934443in}}%
\pgfpathlineto{\pgfqpoint{1.812353in}{0.925915in}}%
\pgfpathlineto{\pgfqpoint{1.809556in}{0.917786in}}%
\pgfpathlineto{\pgfqpoint{1.806771in}{0.910049in}}%
\pgfpathlineto{\pgfqpoint{1.804000in}{0.902698in}}%
\pgfpathlineto{\pgfqpoint{1.817918in}{0.910424in}}%
\pgfpathlineto{\pgfqpoint{1.831320in}{0.918370in}}%
\pgfpathlineto{\pgfqpoint{1.844195in}{0.926526in}}%
\pgfpathlineto{\pgfqpoint{1.856531in}{0.934884in}}%
\pgfpathclose%
\pgfusepath{fill}%
\end{pgfscope}%
\begin{pgfscope}%
\pgfpathrectangle{\pgfqpoint{0.329460in}{0.284240in}}{\pgfqpoint{1.989680in}{1.989680in}}%
\pgfusepath{clip}%
\pgfsetbuttcap%
\pgfsetroundjoin%
\definecolor{currentfill}{rgb}{0.487026,0.823929,0.312321}%
\pgfsetfillcolor{currentfill}%
\pgfsetlinewidth{0.000000pt}%
\definecolor{currentstroke}{rgb}{0.000000,0.000000,0.000000}%
\pgfsetstrokecolor{currentstroke}%
\pgfsetdash{}{0pt}%
\pgfpathmoveto{\pgfqpoint{1.452940in}{1.536192in}}%
\pgfpathlineto{\pgfqpoint{1.455054in}{1.529618in}}%
\pgfpathlineto{\pgfqpoint{1.457168in}{1.522960in}}%
\pgfpathlineto{\pgfqpoint{1.459281in}{1.516222in}}%
\pgfpathlineto{\pgfqpoint{1.461394in}{1.509403in}}%
\pgfpathlineto{\pgfqpoint{1.456749in}{1.507755in}}%
\pgfpathlineto{\pgfqpoint{1.451994in}{1.506179in}}%
\pgfpathlineto{\pgfqpoint{1.447136in}{1.504676in}}%
\pgfpathlineto{\pgfqpoint{1.442178in}{1.503247in}}%
\pgfpathlineto{\pgfqpoint{1.440432in}{1.510191in}}%
\pgfpathlineto{\pgfqpoint{1.438685in}{1.517054in}}%
\pgfpathlineto{\pgfqpoint{1.436939in}{1.523836in}}%
\pgfpathlineto{\pgfqpoint{1.435191in}{1.530534in}}%
\pgfpathlineto{\pgfqpoint{1.439770in}{1.531847in}}%
\pgfpathlineto{\pgfqpoint{1.444257in}{1.533229in}}%
\pgfpathlineto{\pgfqpoint{1.448649in}{1.534678in}}%
\pgfpathlineto{\pgfqpoint{1.452940in}{1.536192in}}%
\pgfpathclose%
\pgfusepath{fill}%
\end{pgfscope}%
\begin{pgfscope}%
\pgfpathrectangle{\pgfqpoint{0.329460in}{0.284240in}}{\pgfqpoint{1.989680in}{1.989680in}}%
\pgfusepath{clip}%
\pgfsetbuttcap%
\pgfsetroundjoin%
\definecolor{currentfill}{rgb}{0.412913,0.803041,0.357269}%
\pgfsetfillcolor{currentfill}%
\pgfsetlinewidth{0.000000pt}%
\definecolor{currentstroke}{rgb}{0.000000,0.000000,0.000000}%
\pgfsetstrokecolor{currentstroke}%
\pgfsetdash{}{0pt}%
\pgfpathmoveto{\pgfqpoint{1.264683in}{1.502041in}}%
\pgfpathlineto{\pgfqpoint{1.263024in}{1.494994in}}%
\pgfpathlineto{\pgfqpoint{1.261365in}{1.487871in}}%
\pgfpathlineto{\pgfqpoint{1.259706in}{1.480672in}}%
\pgfpathlineto{\pgfqpoint{1.258048in}{1.473400in}}%
\pgfpathlineto{\pgfqpoint{1.252621in}{1.474872in}}%
\pgfpathlineto{\pgfqpoint{1.247297in}{1.476427in}}%
\pgfpathlineto{\pgfqpoint{1.242080in}{1.478062in}}%
\pgfpathlineto{\pgfqpoint{1.236976in}{1.479776in}}%
\pgfpathlineto{\pgfqpoint{1.239007in}{1.486927in}}%
\pgfpathlineto{\pgfqpoint{1.241039in}{1.494005in}}%
\pgfpathlineto{\pgfqpoint{1.243072in}{1.501008in}}%
\pgfpathlineto{\pgfqpoint{1.245105in}{1.507935in}}%
\pgfpathlineto{\pgfqpoint{1.249847in}{1.506351in}}%
\pgfpathlineto{\pgfqpoint{1.254695in}{1.504839in}}%
\pgfpathlineto{\pgfqpoint{1.259642in}{1.503402in}}%
\pgfpathlineto{\pgfqpoint{1.264683in}{1.502041in}}%
\pgfpathclose%
\pgfusepath{fill}%
\end{pgfscope}%
\begin{pgfscope}%
\pgfpathrectangle{\pgfqpoint{0.329460in}{0.284240in}}{\pgfqpoint{1.989680in}{1.989680in}}%
\pgfusepath{clip}%
\pgfsetbuttcap%
\pgfsetroundjoin%
\definecolor{currentfill}{rgb}{0.268510,0.009605,0.335427}%
\pgfsetfillcolor{currentfill}%
\pgfsetlinewidth{0.000000pt}%
\definecolor{currentstroke}{rgb}{0.000000,0.000000,0.000000}%
\pgfsetstrokecolor{currentstroke}%
\pgfsetdash{}{0pt}%
\pgfpathmoveto{\pgfqpoint{1.424112in}{0.769510in}}%
\pgfpathlineto{\pgfqpoint{1.424619in}{0.766127in}}%
\pgfpathlineto{\pgfqpoint{1.425127in}{0.762943in}}%
\pgfpathlineto{\pgfqpoint{1.425637in}{0.759963in}}%
\pgfpathlineto{\pgfqpoint{1.426147in}{0.757190in}}%
\pgfpathlineto{\pgfqpoint{1.409367in}{0.756046in}}%
\pgfpathlineto{\pgfqpoint{1.392522in}{0.755191in}}%
\pgfpathlineto{\pgfqpoint{1.375630in}{0.754626in}}%
\pgfpathlineto{\pgfqpoint{1.358711in}{0.754351in}}%
\pgfpathlineto{\pgfqpoint{1.358660in}{0.757144in}}%
\pgfpathlineto{\pgfqpoint{1.358609in}{0.760145in}}%
\pgfpathlineto{\pgfqpoint{1.358558in}{0.763349in}}%
\pgfpathlineto{\pgfqpoint{1.358507in}{0.766753in}}%
\pgfpathlineto{\pgfqpoint{1.374966in}{0.767020in}}%
\pgfpathlineto{\pgfqpoint{1.391399in}{0.767569in}}%
\pgfpathlineto{\pgfqpoint{1.407787in}{0.768400in}}%
\pgfpathlineto{\pgfqpoint{1.424112in}{0.769510in}}%
\pgfpathclose%
\pgfusepath{fill}%
\end{pgfscope}%
\begin{pgfscope}%
\pgfpathrectangle{\pgfqpoint{0.329460in}{0.284240in}}{\pgfqpoint{1.989680in}{1.989680in}}%
\pgfusepath{clip}%
\pgfsetbuttcap%
\pgfsetroundjoin%
\definecolor{currentfill}{rgb}{0.896320,0.893616,0.096335}%
\pgfsetfillcolor{currentfill}%
\pgfsetlinewidth{0.000000pt}%
\definecolor{currentstroke}{rgb}{0.000000,0.000000,0.000000}%
\pgfsetstrokecolor{currentstroke}%
\pgfsetdash{}{0pt}%
\pgfpathmoveto{\pgfqpoint{1.378172in}{1.677054in}}%
\pgfpathlineto{\pgfqpoint{1.379525in}{1.673103in}}%
\pgfpathlineto{\pgfqpoint{1.380878in}{1.669038in}}%
\pgfpathlineto{\pgfqpoint{1.382231in}{1.664859in}}%
\pgfpathlineto{\pgfqpoint{1.383585in}{1.660567in}}%
\pgfpathlineto{\pgfqpoint{1.381104in}{1.660103in}}%
\pgfpathlineto{\pgfqpoint{1.378592in}{1.659677in}}%
\pgfpathlineto{\pgfqpoint{1.376052in}{1.659287in}}%
\pgfpathlineto{\pgfqpoint{1.373487in}{1.658936in}}%
\pgfpathlineto{\pgfqpoint{1.372555in}{1.663298in}}%
\pgfpathlineto{\pgfqpoint{1.371624in}{1.667546in}}%
\pgfpathlineto{\pgfqpoint{1.370692in}{1.671680in}}%
\pgfpathlineto{\pgfqpoint{1.369761in}{1.675700in}}%
\pgfpathlineto{\pgfqpoint{1.371897in}{1.675991in}}%
\pgfpathlineto{\pgfqpoint{1.374012in}{1.676314in}}%
\pgfpathlineto{\pgfqpoint{1.376105in}{1.676669in}}%
\pgfpathlineto{\pgfqpoint{1.378172in}{1.677054in}}%
\pgfpathclose%
\pgfusepath{fill}%
\end{pgfscope}%
\begin{pgfscope}%
\pgfpathrectangle{\pgfqpoint{0.329460in}{0.284240in}}{\pgfqpoint{1.989680in}{1.989680in}}%
\pgfusepath{clip}%
\pgfsetbuttcap%
\pgfsetroundjoin%
\definecolor{currentfill}{rgb}{0.814576,0.883393,0.110347}%
\pgfsetfillcolor{currentfill}%
\pgfsetlinewidth{0.000000pt}%
\definecolor{currentstroke}{rgb}{0.000000,0.000000,0.000000}%
\pgfsetstrokecolor{currentstroke}%
\pgfsetdash{}{0pt}%
\pgfpathmoveto{\pgfqpoint{1.304626in}{1.644242in}}%
\pgfpathlineto{\pgfqpoint{1.302959in}{1.639467in}}%
\pgfpathlineto{\pgfqpoint{1.301293in}{1.634585in}}%
\pgfpathlineto{\pgfqpoint{1.299626in}{1.629594in}}%
\pgfpathlineto{\pgfqpoint{1.297960in}{1.624498in}}%
\pgfpathlineto{\pgfqpoint{1.294853in}{1.625319in}}%
\pgfpathlineto{\pgfqpoint{1.291803in}{1.626185in}}%
\pgfpathlineto{\pgfqpoint{1.288814in}{1.627097in}}%
\pgfpathlineto{\pgfqpoint{1.285888in}{1.628052in}}%
\pgfpathlineto{\pgfqpoint{1.287931in}{1.633034in}}%
\pgfpathlineto{\pgfqpoint{1.289975in}{1.637911in}}%
\pgfpathlineto{\pgfqpoint{1.292018in}{1.642680in}}%
\pgfpathlineto{\pgfqpoint{1.294062in}{1.647341in}}%
\pgfpathlineto{\pgfqpoint{1.296622in}{1.646508in}}%
\pgfpathlineto{\pgfqpoint{1.299238in}{1.645713in}}%
\pgfpathlineto{\pgfqpoint{1.301907in}{1.644958in}}%
\pgfpathlineto{\pgfqpoint{1.304626in}{1.644242in}}%
\pgfpathclose%
\pgfusepath{fill}%
\end{pgfscope}%
\begin{pgfscope}%
\pgfpathrectangle{\pgfqpoint{0.329460in}{0.284240in}}{\pgfqpoint{1.989680in}{1.989680in}}%
\pgfusepath{clip}%
\pgfsetbuttcap%
\pgfsetroundjoin%
\definecolor{currentfill}{rgb}{0.896320,0.893616,0.096335}%
\pgfsetfillcolor{currentfill}%
\pgfsetlinewidth{0.000000pt}%
\definecolor{currentstroke}{rgb}{0.000000,0.000000,0.000000}%
\pgfsetstrokecolor{currentstroke}%
\pgfsetdash{}{0pt}%
\pgfpathmoveto{\pgfqpoint{1.334530in}{1.675467in}}%
\pgfpathlineto{\pgfqpoint{1.333694in}{1.671436in}}%
\pgfpathlineto{\pgfqpoint{1.332858in}{1.667290in}}%
\pgfpathlineto{\pgfqpoint{1.332023in}{1.663030in}}%
\pgfpathlineto{\pgfqpoint{1.331187in}{1.658656in}}%
\pgfpathlineto{\pgfqpoint{1.328602in}{1.658973in}}%
\pgfpathlineto{\pgfqpoint{1.326039in}{1.659329in}}%
\pgfpathlineto{\pgfqpoint{1.323503in}{1.659722in}}%
\pgfpathlineto{\pgfqpoint{1.320994in}{1.660153in}}%
\pgfpathlineto{\pgfqpoint{1.322256in}{1.664463in}}%
\pgfpathlineto{\pgfqpoint{1.323517in}{1.668659in}}%
\pgfpathlineto{\pgfqpoint{1.324778in}{1.672742in}}%
\pgfpathlineto{\pgfqpoint{1.326039in}{1.676710in}}%
\pgfpathlineto{\pgfqpoint{1.328129in}{1.676352in}}%
\pgfpathlineto{\pgfqpoint{1.330242in}{1.676025in}}%
\pgfpathlineto{\pgfqpoint{1.332376in}{1.675730in}}%
\pgfpathlineto{\pgfqpoint{1.334530in}{1.675467in}}%
\pgfpathclose%
\pgfusepath{fill}%
\end{pgfscope}%
\begin{pgfscope}%
\pgfpathrectangle{\pgfqpoint{0.329460in}{0.284240in}}{\pgfqpoint{1.989680in}{1.989680in}}%
\pgfusepath{clip}%
\pgfsetbuttcap%
\pgfsetroundjoin%
\definecolor{currentfill}{rgb}{0.268510,0.009605,0.335427}%
\pgfsetfillcolor{currentfill}%
\pgfsetlinewidth{0.000000pt}%
\definecolor{currentstroke}{rgb}{0.000000,0.000000,0.000000}%
\pgfsetstrokecolor{currentstroke}%
\pgfsetdash{}{0pt}%
\pgfpathmoveto{\pgfqpoint{1.358507in}{0.766753in}}%
\pgfpathlineto{\pgfqpoint{1.358558in}{0.763349in}}%
\pgfpathlineto{\pgfqpoint{1.358609in}{0.760145in}}%
\pgfpathlineto{\pgfqpoint{1.358660in}{0.757144in}}%
\pgfpathlineto{\pgfqpoint{1.358711in}{0.754351in}}%
\pgfpathlineto{\pgfqpoint{1.341783in}{0.754367in}}%
\pgfpathlineto{\pgfqpoint{1.324866in}{0.754674in}}%
\pgfpathlineto{\pgfqpoint{1.307979in}{0.755272in}}%
\pgfpathlineto{\pgfqpoint{1.291140in}{0.756159in}}%
\pgfpathlineto{\pgfqpoint{1.291549in}{0.758939in}}%
\pgfpathlineto{\pgfqpoint{1.291957in}{0.761927in}}%
\pgfpathlineto{\pgfqpoint{1.292364in}{0.765118in}}%
\pgfpathlineto{\pgfqpoint{1.292771in}{0.768509in}}%
\pgfpathlineto{\pgfqpoint{1.309152in}{0.767647in}}%
\pgfpathlineto{\pgfqpoint{1.325581in}{0.767067in}}%
\pgfpathlineto{\pgfqpoint{1.342039in}{0.766769in}}%
\pgfpathlineto{\pgfqpoint{1.358507in}{0.766753in}}%
\pgfpathclose%
\pgfusepath{fill}%
\end{pgfscope}%
\begin{pgfscope}%
\pgfpathrectangle{\pgfqpoint{0.329460in}{0.284240in}}{\pgfqpoint{1.989680in}{1.989680in}}%
\pgfusepath{clip}%
\pgfsetbuttcap%
\pgfsetroundjoin%
\definecolor{currentfill}{rgb}{0.122606,0.585371,0.546557}%
\pgfsetfillcolor{currentfill}%
\pgfsetlinewidth{0.000000pt}%
\definecolor{currentstroke}{rgb}{0.000000,0.000000,0.000000}%
\pgfsetstrokecolor{currentstroke}%
\pgfsetdash{}{0pt}%
\pgfpathmoveto{\pgfqpoint{1.425489in}{1.270002in}}%
\pgfpathlineto{\pgfqpoint{1.426410in}{1.261365in}}%
\pgfpathlineto{\pgfqpoint{1.427331in}{1.252707in}}%
\pgfpathlineto{\pgfqpoint{1.428252in}{1.244032in}}%
\pgfpathlineto{\pgfqpoint{1.429172in}{1.235342in}}%
\pgfpathlineto{\pgfqpoint{1.420133in}{1.234169in}}%
\pgfpathlineto{\pgfqpoint{1.411020in}{1.233141in}}%
\pgfpathlineto{\pgfqpoint{1.401843in}{1.232258in}}%
\pgfpathlineto{\pgfqpoint{1.392612in}{1.231522in}}%
\pgfpathlineto{\pgfqpoint{1.392122in}{1.240263in}}%
\pgfpathlineto{\pgfqpoint{1.391633in}{1.248988in}}%
\pgfpathlineto{\pgfqpoint{1.391143in}{1.257695in}}%
\pgfpathlineto{\pgfqpoint{1.390654in}{1.266383in}}%
\pgfpathlineto{\pgfqpoint{1.399449in}{1.267080in}}%
\pgfpathlineto{\pgfqpoint{1.408192in}{1.267916in}}%
\pgfpathlineto{\pgfqpoint{1.416875in}{1.268890in}}%
\pgfpathlineto{\pgfqpoint{1.425489in}{1.270002in}}%
\pgfpathclose%
\pgfusepath{fill}%
\end{pgfscope}%
\begin{pgfscope}%
\pgfpathrectangle{\pgfqpoint{0.329460in}{0.284240in}}{\pgfqpoint{1.989680in}{1.989680in}}%
\pgfusepath{clip}%
\pgfsetbuttcap%
\pgfsetroundjoin%
\definecolor{currentfill}{rgb}{0.762373,0.876424,0.137064}%
\pgfsetfillcolor{currentfill}%
\pgfsetlinewidth{0.000000pt}%
\definecolor{currentstroke}{rgb}{0.000000,0.000000,0.000000}%
\pgfsetstrokecolor{currentstroke}%
\pgfsetdash{}{0pt}%
\pgfpathmoveto{\pgfqpoint{1.419032in}{1.628938in}}%
\pgfpathlineto{\pgfqpoint{1.421154in}{1.623879in}}%
\pgfpathlineto{\pgfqpoint{1.423276in}{1.618717in}}%
\pgfpathlineto{\pgfqpoint{1.425398in}{1.613452in}}%
\pgfpathlineto{\pgfqpoint{1.427519in}{1.608085in}}%
\pgfpathlineto{\pgfqpoint{1.424295in}{1.606962in}}%
\pgfpathlineto{\pgfqpoint{1.420997in}{1.605889in}}%
\pgfpathlineto{\pgfqpoint{1.417627in}{1.604865in}}%
\pgfpathlineto{\pgfqpoint{1.414190in}{1.603892in}}%
\pgfpathlineto{\pgfqpoint{1.412438in}{1.609379in}}%
\pgfpathlineto{\pgfqpoint{1.410685in}{1.614765in}}%
\pgfpathlineto{\pgfqpoint{1.408933in}{1.620047in}}%
\pgfpathlineto{\pgfqpoint{1.407180in}{1.625225in}}%
\pgfpathlineto{\pgfqpoint{1.410236in}{1.626087in}}%
\pgfpathlineto{\pgfqpoint{1.413232in}{1.626993in}}%
\pgfpathlineto{\pgfqpoint{1.416165in}{1.627944in}}%
\pgfpathlineto{\pgfqpoint{1.419032in}{1.628938in}}%
\pgfpathclose%
\pgfusepath{fill}%
\end{pgfscope}%
\begin{pgfscope}%
\pgfpathrectangle{\pgfqpoint{0.329460in}{0.284240in}}{\pgfqpoint{1.989680in}{1.989680in}}%
\pgfusepath{clip}%
\pgfsetbuttcap%
\pgfsetroundjoin%
\definecolor{currentfill}{rgb}{0.220124,0.725509,0.466226}%
\pgfsetfillcolor{currentfill}%
\pgfsetlinewidth{0.000000pt}%
\definecolor{currentstroke}{rgb}{0.000000,0.000000,0.000000}%
\pgfsetstrokecolor{currentstroke}%
\pgfsetdash{}{0pt}%
\pgfpathmoveto{\pgfqpoint{1.270628in}{1.406978in}}%
\pgfpathlineto{\pgfqpoint{1.269375in}{1.399027in}}%
\pgfpathlineto{\pgfqpoint{1.268123in}{1.391021in}}%
\pgfpathlineto{\pgfqpoint{1.266871in}{1.382963in}}%
\pgfpathlineto{\pgfqpoint{1.265619in}{1.374853in}}%
\pgfpathlineto{\pgfqpoint{1.258610in}{1.376243in}}%
\pgfpathlineto{\pgfqpoint{1.251699in}{1.377742in}}%
\pgfpathlineto{\pgfqpoint{1.244892in}{1.379348in}}%
\pgfpathlineto{\pgfqpoint{1.238196in}{1.381058in}}%
\pgfpathlineto{\pgfqpoint{1.239847in}{1.389069in}}%
\pgfpathlineto{\pgfqpoint{1.241499in}{1.397029in}}%
\pgfpathlineto{\pgfqpoint{1.243151in}{1.404937in}}%
\pgfpathlineto{\pgfqpoint{1.244804in}{1.412789in}}%
\pgfpathlineto{\pgfqpoint{1.251110in}{1.411187in}}%
\pgfpathlineto{\pgfqpoint{1.257520in}{1.409684in}}%
\pgfpathlineto{\pgfqpoint{1.264028in}{1.408280in}}%
\pgfpathlineto{\pgfqpoint{1.270628in}{1.406978in}}%
\pgfpathclose%
\pgfusepath{fill}%
\end{pgfscope}%
\begin{pgfscope}%
\pgfpathrectangle{\pgfqpoint{0.329460in}{0.284240in}}{\pgfqpoint{1.989680in}{1.989680in}}%
\pgfusepath{clip}%
\pgfsetbuttcap%
\pgfsetroundjoin%
\definecolor{currentfill}{rgb}{0.565498,0.842430,0.262877}%
\pgfsetfillcolor{currentfill}%
\pgfsetlinewidth{0.000000pt}%
\definecolor{currentstroke}{rgb}{0.000000,0.000000,0.000000}%
\pgfsetstrokecolor{currentstroke}%
\pgfsetdash{}{0pt}%
\pgfpathmoveto{\pgfqpoint{1.444475in}{1.561630in}}%
\pgfpathlineto{\pgfqpoint{1.446592in}{1.555402in}}%
\pgfpathlineto{\pgfqpoint{1.448708in}{1.549086in}}%
\pgfpathlineto{\pgfqpoint{1.450824in}{1.542682in}}%
\pgfpathlineto{\pgfqpoint{1.452940in}{1.536192in}}%
\pgfpathlineto{\pgfqpoint{1.448649in}{1.534678in}}%
\pgfpathlineto{\pgfqpoint{1.444257in}{1.533229in}}%
\pgfpathlineto{\pgfqpoint{1.439770in}{1.531847in}}%
\pgfpathlineto{\pgfqpoint{1.435191in}{1.530534in}}%
\pgfpathlineto{\pgfqpoint{1.433443in}{1.537148in}}%
\pgfpathlineto{\pgfqpoint{1.431695in}{1.543675in}}%
\pgfpathlineto{\pgfqpoint{1.429946in}{1.550115in}}%
\pgfpathlineto{\pgfqpoint{1.428197in}{1.556465in}}%
\pgfpathlineto{\pgfqpoint{1.432396in}{1.557664in}}%
\pgfpathlineto{\pgfqpoint{1.436511in}{1.558925in}}%
\pgfpathlineto{\pgfqpoint{1.440539in}{1.560247in}}%
\pgfpathlineto{\pgfqpoint{1.444475in}{1.561630in}}%
\pgfpathclose%
\pgfusepath{fill}%
\end{pgfscope}%
\begin{pgfscope}%
\pgfpathrectangle{\pgfqpoint{0.329460in}{0.284240in}}{\pgfqpoint{1.989680in}{1.989680in}}%
\pgfusepath{clip}%
\pgfsetbuttcap%
\pgfsetroundjoin%
\definecolor{currentfill}{rgb}{0.134692,0.658636,0.517649}%
\pgfsetfillcolor{currentfill}%
\pgfsetlinewidth{0.000000pt}%
\definecolor{currentstroke}{rgb}{0.000000,0.000000,0.000000}%
\pgfsetstrokecolor{currentstroke}%
\pgfsetdash{}{0pt}%
\pgfpathmoveto{\pgfqpoint{1.448355in}{1.343253in}}%
\pgfpathlineto{\pgfqpoint{1.449696in}{1.334939in}}%
\pgfpathlineto{\pgfqpoint{1.451035in}{1.326586in}}%
\pgfpathlineto{\pgfqpoint{1.452374in}{1.318198in}}%
\pgfpathlineto{\pgfqpoint{1.453713in}{1.309775in}}%
\pgfpathlineto{\pgfqpoint{1.445877in}{1.308219in}}%
\pgfpathlineto{\pgfqpoint{1.437940in}{1.306787in}}%
\pgfpathlineto{\pgfqpoint{1.429912in}{1.305481in}}%
\pgfpathlineto{\pgfqpoint{1.421801in}{1.304302in}}%
\pgfpathlineto{\pgfqpoint{1.420878in}{1.312803in}}%
\pgfpathlineto{\pgfqpoint{1.419955in}{1.321271in}}%
\pgfpathlineto{\pgfqpoint{1.419031in}{1.329702in}}%
\pgfpathlineto{\pgfqpoint{1.418108in}{1.338094in}}%
\pgfpathlineto{\pgfqpoint{1.425796in}{1.339205in}}%
\pgfpathlineto{\pgfqpoint{1.433405in}{1.340437in}}%
\pgfpathlineto{\pgfqpoint{1.440927in}{1.341786in}}%
\pgfpathlineto{\pgfqpoint{1.448355in}{1.343253in}}%
\pgfpathclose%
\pgfusepath{fill}%
\end{pgfscope}%
\begin{pgfscope}%
\pgfpathrectangle{\pgfqpoint{0.329460in}{0.284240in}}{\pgfqpoint{1.989680in}{1.989680in}}%
\pgfusepath{clip}%
\pgfsetbuttcap%
\pgfsetroundjoin%
\definecolor{currentfill}{rgb}{0.133743,0.548535,0.553541}%
\pgfsetfillcolor{currentfill}%
\pgfsetlinewidth{0.000000pt}%
\definecolor{currentstroke}{rgb}{0.000000,0.000000,0.000000}%
\pgfsetstrokecolor{currentstroke}%
\pgfsetdash{}{0pt}%
\pgfpathmoveto{\pgfqpoint{1.392612in}{1.231522in}}%
\pgfpathlineto{\pgfqpoint{1.393101in}{1.222769in}}%
\pgfpathlineto{\pgfqpoint{1.393590in}{1.214005in}}%
\pgfpathlineto{\pgfqpoint{1.394079in}{1.205233in}}%
\pgfpathlineto{\pgfqpoint{1.394567in}{1.196456in}}%
\pgfpathlineto{\pgfqpoint{1.384854in}{1.195837in}}%
\pgfpathlineto{\pgfqpoint{1.375105in}{1.195375in}}%
\pgfpathlineto{\pgfqpoint{1.365331in}{1.195069in}}%
\pgfpathlineto{\pgfqpoint{1.355541in}{1.194920in}}%
\pgfpathlineto{\pgfqpoint{1.355492in}{1.203717in}}%
\pgfpathlineto{\pgfqpoint{1.355443in}{1.212508in}}%
\pgfpathlineto{\pgfqpoint{1.355394in}{1.221291in}}%
\pgfpathlineto{\pgfqpoint{1.355344in}{1.230064in}}%
\pgfpathlineto{\pgfqpoint{1.364693in}{1.230205in}}%
\pgfpathlineto{\pgfqpoint{1.374027in}{1.230495in}}%
\pgfpathlineto{\pgfqpoint{1.383336in}{1.230935in}}%
\pgfpathlineto{\pgfqpoint{1.392612in}{1.231522in}}%
\pgfpathclose%
\pgfusepath{fill}%
\end{pgfscope}%
\begin{pgfscope}%
\pgfpathrectangle{\pgfqpoint{0.329460in}{0.284240in}}{\pgfqpoint{1.989680in}{1.989680in}}%
\pgfusepath{clip}%
\pgfsetbuttcap%
\pgfsetroundjoin%
\definecolor{currentfill}{rgb}{0.855810,0.888601,0.097452}%
\pgfsetfillcolor{currentfill}%
\pgfsetlinewidth{0.000000pt}%
\definecolor{currentstroke}{rgb}{0.000000,0.000000,0.000000}%
\pgfsetstrokecolor{currentstroke}%
\pgfsetdash{}{0pt}%
\pgfpathmoveto{\pgfqpoint{1.393157in}{1.662784in}}%
\pgfpathlineto{\pgfqpoint{1.394910in}{1.658473in}}%
\pgfpathlineto{\pgfqpoint{1.396663in}{1.654051in}}%
\pgfpathlineto{\pgfqpoint{1.398416in}{1.649519in}}%
\pgfpathlineto{\pgfqpoint{1.400169in}{1.644876in}}%
\pgfpathlineto{\pgfqpoint{1.397444in}{1.644165in}}%
\pgfpathlineto{\pgfqpoint{1.394673in}{1.643495in}}%
\pgfpathlineto{\pgfqpoint{1.391857in}{1.642867in}}%
\pgfpathlineto{\pgfqpoint{1.389000in}{1.642281in}}%
\pgfpathlineto{\pgfqpoint{1.387646in}{1.647019in}}%
\pgfpathlineto{\pgfqpoint{1.386292in}{1.651646in}}%
\pgfpathlineto{\pgfqpoint{1.384939in}{1.656162in}}%
\pgfpathlineto{\pgfqpoint{1.383585in}{1.660567in}}%
\pgfpathlineto{\pgfqpoint{1.386033in}{1.661068in}}%
\pgfpathlineto{\pgfqpoint{1.388446in}{1.661604in}}%
\pgfpathlineto{\pgfqpoint{1.390822in}{1.662176in}}%
\pgfpathlineto{\pgfqpoint{1.393157in}{1.662784in}}%
\pgfpathclose%
\pgfusepath{fill}%
\end{pgfscope}%
\begin{pgfscope}%
\pgfpathrectangle{\pgfqpoint{0.329460in}{0.284240in}}{\pgfqpoint{1.989680in}{1.989680in}}%
\pgfusepath{clip}%
\pgfsetbuttcap%
\pgfsetroundjoin%
\definecolor{currentfill}{rgb}{0.133743,0.548535,0.553541}%
\pgfsetfillcolor{currentfill}%
\pgfsetlinewidth{0.000000pt}%
\definecolor{currentstroke}{rgb}{0.000000,0.000000,0.000000}%
\pgfsetstrokecolor{currentstroke}%
\pgfsetdash{}{0pt}%
\pgfpathmoveto{\pgfqpoint{1.355344in}{1.230064in}}%
\pgfpathlineto{\pgfqpoint{1.355394in}{1.221291in}}%
\pgfpathlineto{\pgfqpoint{1.355443in}{1.212508in}}%
\pgfpathlineto{\pgfqpoint{1.355492in}{1.203717in}}%
\pgfpathlineto{\pgfqpoint{1.355541in}{1.194920in}}%
\pgfpathlineto{\pgfqpoint{1.345746in}{1.194929in}}%
\pgfpathlineto{\pgfqpoint{1.335958in}{1.195095in}}%
\pgfpathlineto{\pgfqpoint{1.326185in}{1.195418in}}%
\pgfpathlineto{\pgfqpoint{1.316440in}{1.195898in}}%
\pgfpathlineto{\pgfqpoint{1.316831in}{1.204682in}}%
\pgfpathlineto{\pgfqpoint{1.317223in}{1.213461in}}%
\pgfpathlineto{\pgfqpoint{1.317615in}{1.222232in}}%
\pgfpathlineto{\pgfqpoint{1.318006in}{1.230993in}}%
\pgfpathlineto{\pgfqpoint{1.327313in}{1.230537in}}%
\pgfpathlineto{\pgfqpoint{1.336644in}{1.230230in}}%
\pgfpathlineto{\pgfqpoint{1.345992in}{1.230072in}}%
\pgfpathlineto{\pgfqpoint{1.355344in}{1.230064in}}%
\pgfpathclose%
\pgfusepath{fill}%
\end{pgfscope}%
\begin{pgfscope}%
\pgfpathrectangle{\pgfqpoint{0.329460in}{0.284240in}}{\pgfqpoint{1.989680in}{1.989680in}}%
\pgfusepath{clip}%
\pgfsetbuttcap%
\pgfsetroundjoin%
\definecolor{currentfill}{rgb}{0.487026,0.823929,0.312321}%
\pgfsetfillcolor{currentfill}%
\pgfsetlinewidth{0.000000pt}%
\definecolor{currentstroke}{rgb}{0.000000,0.000000,0.000000}%
\pgfsetstrokecolor{currentstroke}%
\pgfsetdash{}{0pt}%
\pgfpathmoveto{\pgfqpoint{1.271327in}{1.529426in}}%
\pgfpathlineto{\pgfqpoint{1.269665in}{1.522703in}}%
\pgfpathlineto{\pgfqpoint{1.268004in}{1.515897in}}%
\pgfpathlineto{\pgfqpoint{1.266344in}{1.509009in}}%
\pgfpathlineto{\pgfqpoint{1.264683in}{1.502041in}}%
\pgfpathlineto{\pgfqpoint{1.259642in}{1.503402in}}%
\pgfpathlineto{\pgfqpoint{1.254695in}{1.504839in}}%
\pgfpathlineto{\pgfqpoint{1.249847in}{1.506351in}}%
\pgfpathlineto{\pgfqpoint{1.245105in}{1.507935in}}%
\pgfpathlineto{\pgfqpoint{1.247139in}{1.514783in}}%
\pgfpathlineto{\pgfqpoint{1.249173in}{1.521552in}}%
\pgfpathlineto{\pgfqpoint{1.251209in}{1.528239in}}%
\pgfpathlineto{\pgfqpoint{1.253245in}{1.534843in}}%
\pgfpathlineto{\pgfqpoint{1.257625in}{1.533387in}}%
\pgfpathlineto{\pgfqpoint{1.262102in}{1.531997in}}%
\pgfpathlineto{\pgfqpoint{1.266671in}{1.530677in}}%
\pgfpathlineto{\pgfqpoint{1.271327in}{1.529426in}}%
\pgfpathclose%
\pgfusepath{fill}%
\end{pgfscope}%
\begin{pgfscope}%
\pgfpathrectangle{\pgfqpoint{0.329460in}{0.284240in}}{\pgfqpoint{1.989680in}{1.989680in}}%
\pgfusepath{clip}%
\pgfsetbuttcap%
\pgfsetroundjoin%
\definecolor{currentfill}{rgb}{0.699415,0.867117,0.175971}%
\pgfsetfillcolor{currentfill}%
\pgfsetlinewidth{0.000000pt}%
\definecolor{currentstroke}{rgb}{0.000000,0.000000,0.000000}%
\pgfsetstrokecolor{currentstroke}%
\pgfsetdash{}{0pt}%
\pgfpathmoveto{\pgfqpoint{1.427519in}{1.608085in}}%
\pgfpathlineto{\pgfqpoint{1.429640in}{1.602617in}}%
\pgfpathlineto{\pgfqpoint{1.431761in}{1.597050in}}%
\pgfpathlineto{\pgfqpoint{1.433881in}{1.591385in}}%
\pgfpathlineto{\pgfqpoint{1.436001in}{1.585623in}}%
\pgfpathlineto{\pgfqpoint{1.432421in}{1.584371in}}%
\pgfpathlineto{\pgfqpoint{1.428757in}{1.583173in}}%
\pgfpathlineto{\pgfqpoint{1.425014in}{1.582031in}}%
\pgfpathlineto{\pgfqpoint{1.421196in}{1.580946in}}%
\pgfpathlineto{\pgfqpoint{1.419445in}{1.586830in}}%
\pgfpathlineto{\pgfqpoint{1.417693in}{1.592616in}}%
\pgfpathlineto{\pgfqpoint{1.415942in}{1.598304in}}%
\pgfpathlineto{\pgfqpoint{1.414190in}{1.603892in}}%
\pgfpathlineto{\pgfqpoint{1.417627in}{1.604865in}}%
\pgfpathlineto{\pgfqpoint{1.420997in}{1.605889in}}%
\pgfpathlineto{\pgfqpoint{1.424295in}{1.606962in}}%
\pgfpathlineto{\pgfqpoint{1.427519in}{1.608085in}}%
\pgfpathclose%
\pgfusepath{fill}%
\end{pgfscope}%
\begin{pgfscope}%
\pgfpathrectangle{\pgfqpoint{0.329460in}{0.284240in}}{\pgfqpoint{1.989680in}{1.989680in}}%
\pgfusepath{clip}%
\pgfsetbuttcap%
\pgfsetroundjoin%
\definecolor{currentfill}{rgb}{0.636902,0.856542,0.216620}%
\pgfsetfillcolor{currentfill}%
\pgfsetlinewidth{0.000000pt}%
\definecolor{currentstroke}{rgb}{0.000000,0.000000,0.000000}%
\pgfsetstrokecolor{currentstroke}%
\pgfsetdash{}{0pt}%
\pgfpathmoveto{\pgfqpoint{1.436001in}{1.585623in}}%
\pgfpathlineto{\pgfqpoint{1.438120in}{1.579765in}}%
\pgfpathlineto{\pgfqpoint{1.440239in}{1.573812in}}%
\pgfpathlineto{\pgfqpoint{1.442357in}{1.567767in}}%
\pgfpathlineto{\pgfqpoint{1.444475in}{1.561630in}}%
\pgfpathlineto{\pgfqpoint{1.440539in}{1.560247in}}%
\pgfpathlineto{\pgfqpoint{1.436511in}{1.558925in}}%
\pgfpathlineto{\pgfqpoint{1.432396in}{1.557664in}}%
\pgfpathlineto{\pgfqpoint{1.428197in}{1.556465in}}%
\pgfpathlineto{\pgfqpoint{1.426447in}{1.562725in}}%
\pgfpathlineto{\pgfqpoint{1.424697in}{1.568892in}}%
\pgfpathlineto{\pgfqpoint{1.422946in}{1.574967in}}%
\pgfpathlineto{\pgfqpoint{1.421196in}{1.580946in}}%
\pgfpathlineto{\pgfqpoint{1.425014in}{1.582031in}}%
\pgfpathlineto{\pgfqpoint{1.428757in}{1.583173in}}%
\pgfpathlineto{\pgfqpoint{1.432421in}{1.584371in}}%
\pgfpathlineto{\pgfqpoint{1.436001in}{1.585623in}}%
\pgfpathclose%
\pgfusepath{fill}%
\end{pgfscope}%
\begin{pgfscope}%
\pgfpathrectangle{\pgfqpoint{0.329460in}{0.284240in}}{\pgfqpoint{1.989680in}{1.989680in}}%
\pgfusepath{clip}%
\pgfsetbuttcap%
\pgfsetroundjoin%
\definecolor{currentfill}{rgb}{0.267004,0.004874,0.329415}%
\pgfsetfillcolor{currentfill}%
\pgfsetlinewidth{0.000000pt}%
\definecolor{currentstroke}{rgb}{0.000000,0.000000,0.000000}%
\pgfsetstrokecolor{currentstroke}%
\pgfsetdash{}{0pt}%
\pgfpathmoveto{\pgfqpoint{1.492237in}{0.764620in}}%
\pgfpathlineto{\pgfqpoint{1.493199in}{0.762114in}}%
\pgfpathlineto{\pgfqpoint{1.494163in}{0.759824in}}%
\pgfpathlineto{\pgfqpoint{1.495129in}{0.757755in}}%
\pgfpathlineto{\pgfqpoint{1.496097in}{0.755912in}}%
\pgfpathlineto{\pgfqpoint{1.479320in}{0.753565in}}%
\pgfpathlineto{\pgfqpoint{1.462399in}{0.751507in}}%
\pgfpathlineto{\pgfqpoint{1.445353in}{0.749741in}}%
\pgfpathlineto{\pgfqpoint{1.428201in}{0.748268in}}%
\pgfpathlineto{\pgfqpoint{1.427686in}{0.750165in}}%
\pgfpathlineto{\pgfqpoint{1.427172in}{0.752287in}}%
\pgfpathlineto{\pgfqpoint{1.426659in}{0.754630in}}%
\pgfpathlineto{\pgfqpoint{1.426147in}{0.757190in}}%
\pgfpathlineto{\pgfqpoint{1.442843in}{0.758622in}}%
\pgfpathlineto{\pgfqpoint{1.459435in}{0.760339in}}%
\pgfpathlineto{\pgfqpoint{1.475906in}{0.762339in}}%
\pgfpathlineto{\pgfqpoint{1.492237in}{0.764620in}}%
\pgfpathclose%
\pgfusepath{fill}%
\end{pgfscope}%
\begin{pgfscope}%
\pgfpathrectangle{\pgfqpoint{0.329460in}{0.284240in}}{\pgfqpoint{1.989680in}{1.989680in}}%
\pgfusepath{clip}%
\pgfsetbuttcap%
\pgfsetroundjoin%
\definecolor{currentfill}{rgb}{0.122606,0.585371,0.546557}%
\pgfsetfillcolor{currentfill}%
\pgfsetlinewidth{0.000000pt}%
\definecolor{currentstroke}{rgb}{0.000000,0.000000,0.000000}%
\pgfsetstrokecolor{currentstroke}%
\pgfsetdash{}{0pt}%
\pgfpathmoveto{\pgfqpoint{1.319575in}{1.265881in}}%
\pgfpathlineto{\pgfqpoint{1.319182in}{1.257187in}}%
\pgfpathlineto{\pgfqpoint{1.318790in}{1.248472in}}%
\pgfpathlineto{\pgfqpoint{1.318398in}{1.239740in}}%
\pgfpathlineto{\pgfqpoint{1.318006in}{1.230993in}}%
\pgfpathlineto{\pgfqpoint{1.308736in}{1.231597in}}%
\pgfpathlineto{\pgfqpoint{1.299510in}{1.232349in}}%
\pgfpathlineto{\pgfqpoint{1.290340in}{1.233248in}}%
\pgfpathlineto{\pgfqpoint{1.281234in}{1.234292in}}%
\pgfpathlineto{\pgfqpoint{1.282060in}{1.242996in}}%
\pgfpathlineto{\pgfqpoint{1.282886in}{1.251685in}}%
\pgfpathlineto{\pgfqpoint{1.283712in}{1.260356in}}%
\pgfpathlineto{\pgfqpoint{1.284539in}{1.269007in}}%
\pgfpathlineto{\pgfqpoint{1.293215in}{1.268017in}}%
\pgfpathlineto{\pgfqpoint{1.301952in}{1.267166in}}%
\pgfpathlineto{\pgfqpoint{1.310742in}{1.266453in}}%
\pgfpathlineto{\pgfqpoint{1.319575in}{1.265881in}}%
\pgfpathclose%
\pgfusepath{fill}%
\end{pgfscope}%
\begin{pgfscope}%
\pgfpathrectangle{\pgfqpoint{0.329460in}{0.284240in}}{\pgfqpoint{1.989680in}{1.989680in}}%
\pgfusepath{clip}%
\pgfsetbuttcap%
\pgfsetroundjoin%
\definecolor{currentfill}{rgb}{0.896320,0.893616,0.096335}%
\pgfsetfillcolor{currentfill}%
\pgfsetlinewidth{0.000000pt}%
\definecolor{currentstroke}{rgb}{0.000000,0.000000,0.000000}%
\pgfsetstrokecolor{currentstroke}%
\pgfsetdash{}{0pt}%
\pgfpathmoveto{\pgfqpoint{1.369761in}{1.675700in}}%
\pgfpathlineto{\pgfqpoint{1.370692in}{1.671680in}}%
\pgfpathlineto{\pgfqpoint{1.371624in}{1.667546in}}%
\pgfpathlineto{\pgfqpoint{1.372555in}{1.663298in}}%
\pgfpathlineto{\pgfqpoint{1.373487in}{1.658936in}}%
\pgfpathlineto{\pgfqpoint{1.370900in}{1.658623in}}%
\pgfpathlineto{\pgfqpoint{1.368292in}{1.658349in}}%
\pgfpathlineto{\pgfqpoint{1.365667in}{1.658114in}}%
\pgfpathlineto{\pgfqpoint{1.363028in}{1.657918in}}%
\pgfpathlineto{\pgfqpoint{1.362533in}{1.662323in}}%
\pgfpathlineto{\pgfqpoint{1.362038in}{1.666614in}}%
\pgfpathlineto{\pgfqpoint{1.361543in}{1.670792in}}%
\pgfpathlineto{\pgfqpoint{1.361049in}{1.674854in}}%
\pgfpathlineto{\pgfqpoint{1.363247in}{1.675017in}}%
\pgfpathlineto{\pgfqpoint{1.365433in}{1.675212in}}%
\pgfpathlineto{\pgfqpoint{1.367605in}{1.675440in}}%
\pgfpathlineto{\pgfqpoint{1.369761in}{1.675700in}}%
\pgfpathclose%
\pgfusepath{fill}%
\end{pgfscope}%
\begin{pgfscope}%
\pgfpathrectangle{\pgfqpoint{0.329460in}{0.284240in}}{\pgfqpoint{1.989680in}{1.989680in}}%
\pgfusepath{clip}%
\pgfsetbuttcap%
\pgfsetroundjoin%
\definecolor{currentfill}{rgb}{0.281477,0.755203,0.432552}%
\pgfsetfillcolor{currentfill}%
\pgfsetlinewidth{0.000000pt}%
\definecolor{currentstroke}{rgb}{0.000000,0.000000,0.000000}%
\pgfsetstrokecolor{currentstroke}%
\pgfsetdash{}{0pt}%
\pgfpathmoveto{\pgfqpoint{1.456125in}{1.445015in}}%
\pgfpathlineto{\pgfqpoint{1.457865in}{1.437427in}}%
\pgfpathlineto{\pgfqpoint{1.459605in}{1.429776in}}%
\pgfpathlineto{\pgfqpoint{1.461345in}{1.422065in}}%
\pgfpathlineto{\pgfqpoint{1.463083in}{1.414294in}}%
\pgfpathlineto{\pgfqpoint{1.456876in}{1.412607in}}%
\pgfpathlineto{\pgfqpoint{1.450558in}{1.411015in}}%
\pgfpathlineto{\pgfqpoint{1.444137in}{1.409523in}}%
\pgfpathlineto{\pgfqpoint{1.437618in}{1.408130in}}%
\pgfpathlineto{\pgfqpoint{1.436274in}{1.416004in}}%
\pgfpathlineto{\pgfqpoint{1.434929in}{1.423819in}}%
\pgfpathlineto{\pgfqpoint{1.433584in}{1.431573in}}%
\pgfpathlineto{\pgfqpoint{1.432238in}{1.439264in}}%
\pgfpathlineto{\pgfqpoint{1.438352in}{1.440563in}}%
\pgfpathlineto{\pgfqpoint{1.444376in}{1.441956in}}%
\pgfpathlineto{\pgfqpoint{1.450302in}{1.443440in}}%
\pgfpathlineto{\pgfqpoint{1.456125in}{1.445015in}}%
\pgfpathclose%
\pgfusepath{fill}%
\end{pgfscope}%
\begin{pgfscope}%
\pgfpathrectangle{\pgfqpoint{0.329460in}{0.284240in}}{\pgfqpoint{1.989680in}{1.989680in}}%
\pgfusepath{clip}%
\pgfsetbuttcap%
\pgfsetroundjoin%
\definecolor{currentfill}{rgb}{0.855810,0.888601,0.097452}%
\pgfsetfillcolor{currentfill}%
\pgfsetlinewidth{0.000000pt}%
\definecolor{currentstroke}{rgb}{0.000000,0.000000,0.000000}%
\pgfsetstrokecolor{currentstroke}%
\pgfsetdash{}{0pt}%
\pgfpathmoveto{\pgfqpoint{1.320994in}{1.660153in}}%
\pgfpathlineto{\pgfqpoint{1.319733in}{1.655731in}}%
\pgfpathlineto{\pgfqpoint{1.318471in}{1.651196in}}%
\pgfpathlineto{\pgfqpoint{1.317209in}{1.646551in}}%
\pgfpathlineto{\pgfqpoint{1.315947in}{1.641796in}}%
\pgfpathlineto{\pgfqpoint{1.313056in}{1.642344in}}%
\pgfpathlineto{\pgfqpoint{1.310203in}{1.642935in}}%
\pgfpathlineto{\pgfqpoint{1.307392in}{1.643568in}}%
\pgfpathlineto{\pgfqpoint{1.304626in}{1.644242in}}%
\pgfpathlineto{\pgfqpoint{1.306292in}{1.648908in}}%
\pgfpathlineto{\pgfqpoint{1.307959in}{1.653464in}}%
\pgfpathlineto{\pgfqpoint{1.309625in}{1.657909in}}%
\pgfpathlineto{\pgfqpoint{1.311292in}{1.662242in}}%
\pgfpathlineto{\pgfqpoint{1.313663in}{1.661666in}}%
\pgfpathlineto{\pgfqpoint{1.316072in}{1.661125in}}%
\pgfpathlineto{\pgfqpoint{1.318516in}{1.660621in}}%
\pgfpathlineto{\pgfqpoint{1.320994in}{1.660153in}}%
\pgfpathclose%
\pgfusepath{fill}%
\end{pgfscope}%
\begin{pgfscope}%
\pgfpathrectangle{\pgfqpoint{0.329460in}{0.284240in}}{\pgfqpoint{1.989680in}{1.989680in}}%
\pgfusepath{clip}%
\pgfsetbuttcap%
\pgfsetroundjoin%
\definecolor{currentfill}{rgb}{0.267004,0.004874,0.329415}%
\pgfsetfillcolor{currentfill}%
\pgfsetlinewidth{0.000000pt}%
\definecolor{currentstroke}{rgb}{0.000000,0.000000,0.000000}%
\pgfsetstrokecolor{currentstroke}%
\pgfsetdash{}{0pt}%
\pgfpathmoveto{\pgfqpoint{1.221183in}{0.753811in}}%
\pgfpathlineto{\pgfqpoint{1.220312in}{0.752184in}}%
\pgfpathlineto{\pgfqpoint{1.219439in}{0.750790in}}%
\pgfpathlineto{\pgfqpoint{1.218563in}{0.749637in}}%
\pgfpathlineto{\pgfqpoint{1.217685in}{0.748727in}}%
\pgfpathlineto{\pgfqpoint{1.200476in}{0.751172in}}%
\pgfpathlineto{\pgfqpoint{1.183437in}{0.753910in}}%
\pgfpathlineto{\pgfqpoint{1.166586in}{0.756938in}}%
\pgfpathlineto{\pgfqpoint{1.149944in}{0.760252in}}%
\pgfpathlineto{\pgfqpoint{1.151265in}{0.761085in}}%
\pgfpathlineto{\pgfqpoint{1.152583in}{0.762161in}}%
\pgfpathlineto{\pgfqpoint{1.153897in}{0.763477in}}%
\pgfpathlineto{\pgfqpoint{1.155208in}{0.765027in}}%
\pgfpathlineto{\pgfqpoint{1.171418in}{0.761801in}}%
\pgfpathlineto{\pgfqpoint{1.187829in}{0.758854in}}%
\pgfpathlineto{\pgfqpoint{1.204424in}{0.756190in}}%
\pgfpathlineto{\pgfqpoint{1.221183in}{0.753811in}}%
\pgfpathclose%
\pgfusepath{fill}%
\end{pgfscope}%
\begin{pgfscope}%
\pgfpathrectangle{\pgfqpoint{0.329460in}{0.284240in}}{\pgfqpoint{1.989680in}{1.989680in}}%
\pgfusepath{clip}%
\pgfsetbuttcap%
\pgfsetroundjoin%
\definecolor{currentfill}{rgb}{0.896320,0.893616,0.096335}%
\pgfsetfillcolor{currentfill}%
\pgfsetlinewidth{0.000000pt}%
\definecolor{currentstroke}{rgb}{0.000000,0.000000,0.000000}%
\pgfsetstrokecolor{currentstroke}%
\pgfsetdash{}{0pt}%
\pgfpathmoveto{\pgfqpoint{1.343289in}{1.674737in}}%
\pgfpathlineto{\pgfqpoint{1.342893in}{1.670668in}}%
\pgfpathlineto{\pgfqpoint{1.342497in}{1.666485in}}%
\pgfpathlineto{\pgfqpoint{1.342100in}{1.662188in}}%
\pgfpathlineto{\pgfqpoint{1.341704in}{1.657777in}}%
\pgfpathlineto{\pgfqpoint{1.339054in}{1.657938in}}%
\pgfpathlineto{\pgfqpoint{1.336415in}{1.658138in}}%
\pgfpathlineto{\pgfqpoint{1.333792in}{1.658378in}}%
\pgfpathlineto{\pgfqpoint{1.331187in}{1.658656in}}%
\pgfpathlineto{\pgfqpoint{1.332023in}{1.663030in}}%
\pgfpathlineto{\pgfqpoint{1.332858in}{1.667290in}}%
\pgfpathlineto{\pgfqpoint{1.333694in}{1.671436in}}%
\pgfpathlineto{\pgfqpoint{1.334530in}{1.675467in}}%
\pgfpathlineto{\pgfqpoint{1.336700in}{1.675236in}}%
\pgfpathlineto{\pgfqpoint{1.338884in}{1.675037in}}%
\pgfpathlineto{\pgfqpoint{1.341082in}{1.674870in}}%
\pgfpathlineto{\pgfqpoint{1.343289in}{1.674737in}}%
\pgfpathclose%
\pgfusepath{fill}%
\end{pgfscope}%
\begin{pgfscope}%
\pgfpathrectangle{\pgfqpoint{0.329460in}{0.284240in}}{\pgfqpoint{1.989680in}{1.989680in}}%
\pgfusepath{clip}%
\pgfsetbuttcap%
\pgfsetroundjoin%
\definecolor{currentfill}{rgb}{0.762373,0.876424,0.137064}%
\pgfsetfillcolor{currentfill}%
\pgfsetlinewidth{0.000000pt}%
\definecolor{currentstroke}{rgb}{0.000000,0.000000,0.000000}%
\pgfsetstrokecolor{currentstroke}%
\pgfsetdash{}{0pt}%
\pgfpathmoveto{\pgfqpoint{1.297960in}{1.624498in}}%
\pgfpathlineto{\pgfqpoint{1.296293in}{1.619297in}}%
\pgfpathlineto{\pgfqpoint{1.294627in}{1.613991in}}%
\pgfpathlineto{\pgfqpoint{1.292961in}{1.608582in}}%
\pgfpathlineto{\pgfqpoint{1.291295in}{1.603071in}}%
\pgfpathlineto{\pgfqpoint{1.287800in}{1.603998in}}%
\pgfpathlineto{\pgfqpoint{1.284370in}{1.604976in}}%
\pgfpathlineto{\pgfqpoint{1.281008in}{1.606005in}}%
\pgfpathlineto{\pgfqpoint{1.277718in}{1.607084in}}%
\pgfpathlineto{\pgfqpoint{1.279760in}{1.612480in}}%
\pgfpathlineto{\pgfqpoint{1.281802in}{1.617774in}}%
\pgfpathlineto{\pgfqpoint{1.283845in}{1.622965in}}%
\pgfpathlineto{\pgfqpoint{1.285888in}{1.628052in}}%
\pgfpathlineto{\pgfqpoint{1.288814in}{1.627097in}}%
\pgfpathlineto{\pgfqpoint{1.291803in}{1.626185in}}%
\pgfpathlineto{\pgfqpoint{1.294853in}{1.625319in}}%
\pgfpathlineto{\pgfqpoint{1.297960in}{1.624498in}}%
\pgfpathclose%
\pgfusepath{fill}%
\end{pgfscope}%
\begin{pgfscope}%
\pgfpathrectangle{\pgfqpoint{0.329460in}{0.284240in}}{\pgfqpoint{1.989680in}{1.989680in}}%
\pgfusepath{clip}%
\pgfsetbuttcap%
\pgfsetroundjoin%
\definecolor{currentfill}{rgb}{0.134692,0.658636,0.517649}%
\pgfsetfillcolor{currentfill}%
\pgfsetlinewidth{0.000000pt}%
\definecolor{currentstroke}{rgb}{0.000000,0.000000,0.000000}%
\pgfsetstrokecolor{currentstroke}%
\pgfsetdash{}{0pt}%
\pgfpathmoveto{\pgfqpoint{1.291161in}{1.337208in}}%
\pgfpathlineto{\pgfqpoint{1.290332in}{1.328802in}}%
\pgfpathlineto{\pgfqpoint{1.289504in}{1.320357in}}%
\pgfpathlineto{\pgfqpoint{1.288676in}{1.311877in}}%
\pgfpathlineto{\pgfqpoint{1.287848in}{1.303362in}}%
\pgfpathlineto{\pgfqpoint{1.279669in}{1.304427in}}%
\pgfpathlineto{\pgfqpoint{1.271567in}{1.305620in}}%
\pgfpathlineto{\pgfqpoint{1.263548in}{1.306940in}}%
\pgfpathlineto{\pgfqpoint{1.255623in}{1.308386in}}%
\pgfpathlineto{\pgfqpoint{1.256871in}{1.316829in}}%
\pgfpathlineto{\pgfqpoint{1.258119in}{1.325237in}}%
\pgfpathlineto{\pgfqpoint{1.259368in}{1.333610in}}%
\pgfpathlineto{\pgfqpoint{1.260617in}{1.341944in}}%
\pgfpathlineto{\pgfqpoint{1.268130in}{1.340581in}}%
\pgfpathlineto{\pgfqpoint{1.275730in}{1.339336in}}%
\pgfpathlineto{\pgfqpoint{1.283410in}{1.338211in}}%
\pgfpathlineto{\pgfqpoint{1.291161in}{1.337208in}}%
\pgfpathclose%
\pgfusepath{fill}%
\end{pgfscope}%
\begin{pgfscope}%
\pgfpathrectangle{\pgfqpoint{0.329460in}{0.284240in}}{\pgfqpoint{1.989680in}{1.989680in}}%
\pgfusepath{clip}%
\pgfsetbuttcap%
\pgfsetroundjoin%
\definecolor{currentfill}{rgb}{0.565498,0.842430,0.262877}%
\pgfsetfillcolor{currentfill}%
\pgfsetlinewidth{0.000000pt}%
\definecolor{currentstroke}{rgb}{0.000000,0.000000,0.000000}%
\pgfsetstrokecolor{currentstroke}%
\pgfsetdash{}{0pt}%
\pgfpathmoveto{\pgfqpoint{1.277978in}{1.555453in}}%
\pgfpathlineto{\pgfqpoint{1.276314in}{1.549079in}}%
\pgfpathlineto{\pgfqpoint{1.274651in}{1.542615in}}%
\pgfpathlineto{\pgfqpoint{1.272989in}{1.536064in}}%
\pgfpathlineto{\pgfqpoint{1.271327in}{1.529426in}}%
\pgfpathlineto{\pgfqpoint{1.266671in}{1.530677in}}%
\pgfpathlineto{\pgfqpoint{1.262102in}{1.531997in}}%
\pgfpathlineto{\pgfqpoint{1.257625in}{1.533387in}}%
\pgfpathlineto{\pgfqpoint{1.253245in}{1.534843in}}%
\pgfpathlineto{\pgfqpoint{1.255281in}{1.541362in}}%
\pgfpathlineto{\pgfqpoint{1.257318in}{1.547795in}}%
\pgfpathlineto{\pgfqpoint{1.259356in}{1.554141in}}%
\pgfpathlineto{\pgfqpoint{1.261394in}{1.560398in}}%
\pgfpathlineto{\pgfqpoint{1.265412in}{1.559069in}}%
\pgfpathlineto{\pgfqpoint{1.269518in}{1.557801in}}%
\pgfpathlineto{\pgfqpoint{1.273708in}{1.556595in}}%
\pgfpathlineto{\pgfqpoint{1.277978in}{1.555453in}}%
\pgfpathclose%
\pgfusepath{fill}%
\end{pgfscope}%
\begin{pgfscope}%
\pgfpathrectangle{\pgfqpoint{0.329460in}{0.284240in}}{\pgfqpoint{1.989680in}{1.989680in}}%
\pgfusepath{clip}%
\pgfsetbuttcap%
\pgfsetroundjoin%
\definecolor{currentfill}{rgb}{0.896320,0.893616,0.096335}%
\pgfsetfillcolor{currentfill}%
\pgfsetlinewidth{0.000000pt}%
\definecolor{currentstroke}{rgb}{0.000000,0.000000,0.000000}%
\pgfsetstrokecolor{currentstroke}%
\pgfsetdash{}{0pt}%
\pgfpathmoveto{\pgfqpoint{1.361049in}{1.674854in}}%
\pgfpathlineto{\pgfqpoint{1.361543in}{1.670792in}}%
\pgfpathlineto{\pgfqpoint{1.362038in}{1.666614in}}%
\pgfpathlineto{\pgfqpoint{1.362533in}{1.662323in}}%
\pgfpathlineto{\pgfqpoint{1.363028in}{1.657918in}}%
\pgfpathlineto{\pgfqpoint{1.360376in}{1.657762in}}%
\pgfpathlineto{\pgfqpoint{1.357715in}{1.657644in}}%
\pgfpathlineto{\pgfqpoint{1.355047in}{1.657567in}}%
\pgfpathlineto{\pgfqpoint{1.352376in}{1.657530in}}%
\pgfpathlineto{\pgfqpoint{1.352326in}{1.661951in}}%
\pgfpathlineto{\pgfqpoint{1.352276in}{1.666259in}}%
\pgfpathlineto{\pgfqpoint{1.352227in}{1.670452in}}%
\pgfpathlineto{\pgfqpoint{1.352177in}{1.674531in}}%
\pgfpathlineto{\pgfqpoint{1.354402in}{1.674563in}}%
\pgfpathlineto{\pgfqpoint{1.356624in}{1.674627in}}%
\pgfpathlineto{\pgfqpoint{1.358840in}{1.674724in}}%
\pgfpathlineto{\pgfqpoint{1.361049in}{1.674854in}}%
\pgfpathclose%
\pgfusepath{fill}%
\end{pgfscope}%
\begin{pgfscope}%
\pgfpathrectangle{\pgfqpoint{0.329460in}{0.284240in}}{\pgfqpoint{1.989680in}{1.989680in}}%
\pgfusepath{clip}%
\pgfsetbuttcap%
\pgfsetroundjoin%
\definecolor{currentfill}{rgb}{0.896320,0.893616,0.096335}%
\pgfsetfillcolor{currentfill}%
\pgfsetlinewidth{0.000000pt}%
\definecolor{currentstroke}{rgb}{0.000000,0.000000,0.000000}%
\pgfsetstrokecolor{currentstroke}%
\pgfsetdash{}{0pt}%
\pgfpathmoveto{\pgfqpoint{1.352177in}{1.674531in}}%
\pgfpathlineto{\pgfqpoint{1.352227in}{1.670452in}}%
\pgfpathlineto{\pgfqpoint{1.352276in}{1.666259in}}%
\pgfpathlineto{\pgfqpoint{1.352326in}{1.661951in}}%
\pgfpathlineto{\pgfqpoint{1.352376in}{1.657530in}}%
\pgfpathlineto{\pgfqpoint{1.349703in}{1.657532in}}%
\pgfpathlineto{\pgfqpoint{1.347031in}{1.657574in}}%
\pgfpathlineto{\pgfqpoint{1.344364in}{1.657656in}}%
\pgfpathlineto{\pgfqpoint{1.341704in}{1.657777in}}%
\pgfpathlineto{\pgfqpoint{1.342100in}{1.662188in}}%
\pgfpathlineto{\pgfqpoint{1.342497in}{1.666485in}}%
\pgfpathlineto{\pgfqpoint{1.342893in}{1.670668in}}%
\pgfpathlineto{\pgfqpoint{1.343289in}{1.674737in}}%
\pgfpathlineto{\pgfqpoint{1.345505in}{1.674636in}}%
\pgfpathlineto{\pgfqpoint{1.347726in}{1.674568in}}%
\pgfpathlineto{\pgfqpoint{1.349951in}{1.674533in}}%
\pgfpathlineto{\pgfqpoint{1.352177in}{1.674531in}}%
\pgfpathclose%
\pgfusepath{fill}%
\end{pgfscope}%
\begin{pgfscope}%
\pgfpathrectangle{\pgfqpoint{0.329460in}{0.284240in}}{\pgfqpoint{1.989680in}{1.989680in}}%
\pgfusepath{clip}%
\pgfsetbuttcap%
\pgfsetroundjoin%
\definecolor{currentfill}{rgb}{0.282884,0.135920,0.453427}%
\pgfsetfillcolor{currentfill}%
\pgfsetlinewidth{0.000000pt}%
\definecolor{currentstroke}{rgb}{0.000000,0.000000,0.000000}%
\pgfsetstrokecolor{currentstroke}%
\pgfsetdash{}{0pt}%
\pgfpathmoveto{\pgfqpoint{0.991569in}{0.824134in}}%
\pgfpathlineto{\pgfqpoint{0.989325in}{0.828444in}}%
\pgfpathlineto{\pgfqpoint{0.987072in}{0.833092in}}%
\pgfpathlineto{\pgfqpoint{0.984810in}{0.838082in}}%
\pgfpathlineto{\pgfqpoint{0.982537in}{0.843422in}}%
\pgfpathlineto{\pgfqpoint{0.966700in}{0.849807in}}%
\pgfpathlineto{\pgfqpoint{0.951291in}{0.856451in}}%
\pgfpathlineto{\pgfqpoint{0.936328in}{0.863345in}}%
\pgfpathlineto{\pgfqpoint{0.921826in}{0.870481in}}%
\pgfpathlineto{\pgfqpoint{0.924461in}{0.864996in}}%
\pgfpathlineto{\pgfqpoint{0.927086in}{0.859860in}}%
\pgfpathlineto{\pgfqpoint{0.929700in}{0.855066in}}%
\pgfpathlineto{\pgfqpoint{0.932303in}{0.850608in}}%
\pgfpathlineto{\pgfqpoint{0.946462in}{0.843626in}}%
\pgfpathlineto{\pgfqpoint{0.961070in}{0.836881in}}%
\pgfpathlineto{\pgfqpoint{0.976111in}{0.830381in}}%
\pgfpathlineto{\pgfqpoint{0.991569in}{0.824134in}}%
\pgfpathclose%
\pgfusepath{fill}%
\end{pgfscope}%
\begin{pgfscope}%
\pgfpathrectangle{\pgfqpoint{0.329460in}{0.284240in}}{\pgfqpoint{1.989680in}{1.989680in}}%
\pgfusepath{clip}%
\pgfsetbuttcap%
\pgfsetroundjoin%
\definecolor{currentfill}{rgb}{0.699415,0.867117,0.175971}%
\pgfsetfillcolor{currentfill}%
\pgfsetlinewidth{0.000000pt}%
\definecolor{currentstroke}{rgb}{0.000000,0.000000,0.000000}%
\pgfsetstrokecolor{currentstroke}%
\pgfsetdash{}{0pt}%
\pgfpathmoveto{\pgfqpoint{1.291295in}{1.603071in}}%
\pgfpathlineto{\pgfqpoint{1.289630in}{1.597459in}}%
\pgfpathlineto{\pgfqpoint{1.287964in}{1.591748in}}%
\pgfpathlineto{\pgfqpoint{1.286299in}{1.585938in}}%
\pgfpathlineto{\pgfqpoint{1.284634in}{1.580030in}}%
\pgfpathlineto{\pgfqpoint{1.280751in}{1.581064in}}%
\pgfpathlineto{\pgfqpoint{1.276941in}{1.582155in}}%
\pgfpathlineto{\pgfqpoint{1.273207in}{1.583303in}}%
\pgfpathlineto{\pgfqpoint{1.269553in}{1.584507in}}%
\pgfpathlineto{\pgfqpoint{1.271593in}{1.590298in}}%
\pgfpathlineto{\pgfqpoint{1.273634in}{1.595992in}}%
\pgfpathlineto{\pgfqpoint{1.275676in}{1.601588in}}%
\pgfpathlineto{\pgfqpoint{1.277718in}{1.607084in}}%
\pgfpathlineto{\pgfqpoint{1.281008in}{1.606005in}}%
\pgfpathlineto{\pgfqpoint{1.284370in}{1.604976in}}%
\pgfpathlineto{\pgfqpoint{1.287800in}{1.603998in}}%
\pgfpathlineto{\pgfqpoint{1.291295in}{1.603071in}}%
\pgfpathclose%
\pgfusepath{fill}%
\end{pgfscope}%
\begin{pgfscope}%
\pgfpathrectangle{\pgfqpoint{0.329460in}{0.284240in}}{\pgfqpoint{1.989680in}{1.989680in}}%
\pgfusepath{clip}%
\pgfsetbuttcap%
\pgfsetroundjoin%
\definecolor{currentfill}{rgb}{0.636902,0.856542,0.216620}%
\pgfsetfillcolor{currentfill}%
\pgfsetlinewidth{0.000000pt}%
\definecolor{currentstroke}{rgb}{0.000000,0.000000,0.000000}%
\pgfsetstrokecolor{currentstroke}%
\pgfsetdash{}{0pt}%
\pgfpathmoveto{\pgfqpoint{1.284634in}{1.580030in}}%
\pgfpathlineto{\pgfqpoint{1.282970in}{1.574027in}}%
\pgfpathlineto{\pgfqpoint{1.281305in}{1.567929in}}%
\pgfpathlineto{\pgfqpoint{1.279641in}{1.561737in}}%
\pgfpathlineto{\pgfqpoint{1.277978in}{1.555453in}}%
\pgfpathlineto{\pgfqpoint{1.273708in}{1.556595in}}%
\pgfpathlineto{\pgfqpoint{1.269518in}{1.557801in}}%
\pgfpathlineto{\pgfqpoint{1.265412in}{1.559069in}}%
\pgfpathlineto{\pgfqpoint{1.261394in}{1.560398in}}%
\pgfpathlineto{\pgfqpoint{1.263433in}{1.566564in}}%
\pgfpathlineto{\pgfqpoint{1.265473in}{1.572639in}}%
\pgfpathlineto{\pgfqpoint{1.267512in}{1.578620in}}%
\pgfpathlineto{\pgfqpoint{1.269553in}{1.584507in}}%
\pgfpathlineto{\pgfqpoint{1.273207in}{1.583303in}}%
\pgfpathlineto{\pgfqpoint{1.276941in}{1.582155in}}%
\pgfpathlineto{\pgfqpoint{1.280751in}{1.581064in}}%
\pgfpathlineto{\pgfqpoint{1.284634in}{1.580030in}}%
\pgfpathclose%
\pgfusepath{fill}%
\end{pgfscope}%
\begin{pgfscope}%
\pgfpathrectangle{\pgfqpoint{0.329460in}{0.284240in}}{\pgfqpoint{1.989680in}{1.989680in}}%
\pgfusepath{clip}%
\pgfsetbuttcap%
\pgfsetroundjoin%
\definecolor{currentfill}{rgb}{0.281477,0.755203,0.432552}%
\pgfsetfillcolor{currentfill}%
\pgfsetlinewidth{0.000000pt}%
\definecolor{currentstroke}{rgb}{0.000000,0.000000,0.000000}%
\pgfsetstrokecolor{currentstroke}%
\pgfsetdash{}{0pt}%
\pgfpathmoveto{\pgfqpoint{1.275643in}{1.438188in}}%
\pgfpathlineto{\pgfqpoint{1.274389in}{1.430478in}}%
\pgfpathlineto{\pgfqpoint{1.273135in}{1.422705in}}%
\pgfpathlineto{\pgfqpoint{1.271881in}{1.414871in}}%
\pgfpathlineto{\pgfqpoint{1.270628in}{1.406978in}}%
\pgfpathlineto{\pgfqpoint{1.264028in}{1.408280in}}%
\pgfpathlineto{\pgfqpoint{1.257520in}{1.409684in}}%
\pgfpathlineto{\pgfqpoint{1.251110in}{1.411187in}}%
\pgfpathlineto{\pgfqpoint{1.244804in}{1.412789in}}%
\pgfpathlineto{\pgfqpoint{1.246457in}{1.420585in}}%
\pgfpathlineto{\pgfqpoint{1.248111in}{1.428322in}}%
\pgfpathlineto{\pgfqpoint{1.249766in}{1.435998in}}%
\pgfpathlineto{\pgfqpoint{1.251421in}{1.443611in}}%
\pgfpathlineto{\pgfqpoint{1.257336in}{1.442116in}}%
\pgfpathlineto{\pgfqpoint{1.263349in}{1.440713in}}%
\pgfpathlineto{\pgfqpoint{1.269453in}{1.439403in}}%
\pgfpathlineto{\pgfqpoint{1.275643in}{1.438188in}}%
\pgfpathclose%
\pgfusepath{fill}%
\end{pgfscope}%
\begin{pgfscope}%
\pgfpathrectangle{\pgfqpoint{0.329460in}{0.284240in}}{\pgfqpoint{1.989680in}{1.989680in}}%
\pgfusepath{clip}%
\pgfsetbuttcap%
\pgfsetroundjoin%
\definecolor{currentfill}{rgb}{0.267004,0.004874,0.329415}%
\pgfsetfillcolor{currentfill}%
\pgfsetlinewidth{0.000000pt}%
\definecolor{currentstroke}{rgb}{0.000000,0.000000,0.000000}%
\pgfsetstrokecolor{currentstroke}%
\pgfsetdash{}{0pt}%
\pgfpathmoveto{\pgfqpoint{1.291140in}{0.756159in}}%
\pgfpathlineto{\pgfqpoint{1.290730in}{0.753592in}}%
\pgfpathlineto{\pgfqpoint{1.290319in}{0.751241in}}%
\pgfpathlineto{\pgfqpoint{1.289907in}{0.749111in}}%
\pgfpathlineto{\pgfqpoint{1.289494in}{0.747207in}}%
\pgfpathlineto{\pgfqpoint{1.272264in}{0.748417in}}%
\pgfpathlineto{\pgfqpoint{1.255122in}{0.749922in}}%
\pgfpathlineto{\pgfqpoint{1.238089in}{0.751721in}}%
\pgfpathlineto{\pgfqpoint{1.221183in}{0.753811in}}%
\pgfpathlineto{\pgfqpoint{1.222052in}{0.755669in}}%
\pgfpathlineto{\pgfqpoint{1.222919in}{0.757753in}}%
\pgfpathlineto{\pgfqpoint{1.223784in}{0.760057in}}%
\pgfpathlineto{\pgfqpoint{1.224647in}{0.762579in}}%
\pgfpathlineto{\pgfqpoint{1.241103in}{0.760547in}}%
\pgfpathlineto{\pgfqpoint{1.257683in}{0.758798in}}%
\pgfpathlineto{\pgfqpoint{1.274368in}{0.757335in}}%
\pgfpathlineto{\pgfqpoint{1.291140in}{0.756159in}}%
\pgfpathclose%
\pgfusepath{fill}%
\end{pgfscope}%
\begin{pgfscope}%
\pgfpathrectangle{\pgfqpoint{0.329460in}{0.284240in}}{\pgfqpoint{1.989680in}{1.989680in}}%
\pgfusepath{clip}%
\pgfsetbuttcap%
\pgfsetroundjoin%
\definecolor{currentfill}{rgb}{0.272594,0.025563,0.353093}%
\pgfsetfillcolor{currentfill}%
\pgfsetlinewidth{0.000000pt}%
\definecolor{currentstroke}{rgb}{0.000000,0.000000,0.000000}%
\pgfsetstrokecolor{currentstroke}%
\pgfsetdash{}{0pt}%
\pgfpathmoveto{\pgfqpoint{1.637568in}{0.780168in}}%
\pgfpathlineto{\pgfqpoint{1.639420in}{0.780759in}}%
\pgfpathlineto{\pgfqpoint{1.641279in}{0.781623in}}%
\pgfpathlineto{\pgfqpoint{1.643144in}{0.782764in}}%
\pgfpathlineto{\pgfqpoint{1.645014in}{0.784189in}}%
\pgfpathlineto{\pgfqpoint{1.628840in}{0.779297in}}%
\pgfpathlineto{\pgfqpoint{1.612353in}{0.774681in}}%
\pgfpathlineto{\pgfqpoint{1.595574in}{0.770347in}}%
\pgfpathlineto{\pgfqpoint{1.578519in}{0.766301in}}%
\pgfpathlineto{\pgfqpoint{1.577068in}{0.764987in}}%
\pgfpathlineto{\pgfqpoint{1.575621in}{0.763956in}}%
\pgfpathlineto{\pgfqpoint{1.574180in}{0.763203in}}%
\pgfpathlineto{\pgfqpoint{1.572742in}{0.762724in}}%
\pgfpathlineto{\pgfqpoint{1.589368in}{0.766669in}}%
\pgfpathlineto{\pgfqpoint{1.605725in}{0.770895in}}%
\pgfpathlineto{\pgfqpoint{1.621798in}{0.775397in}}%
\pgfpathlineto{\pgfqpoint{1.637568in}{0.780168in}}%
\pgfpathclose%
\pgfusepath{fill}%
\end{pgfscope}%
\begin{pgfscope}%
\pgfpathrectangle{\pgfqpoint{0.329460in}{0.284240in}}{\pgfqpoint{1.989680in}{1.989680in}}%
\pgfusepath{clip}%
\pgfsetbuttcap%
\pgfsetroundjoin%
\definecolor{currentfill}{rgb}{0.120081,0.622161,0.534946}%
\pgfsetfillcolor{currentfill}%
\pgfsetlinewidth{0.000000pt}%
\definecolor{currentstroke}{rgb}{0.000000,0.000000,0.000000}%
\pgfsetstrokecolor{currentstroke}%
\pgfsetdash{}{0pt}%
\pgfpathmoveto{\pgfqpoint{1.421801in}{1.304302in}}%
\pgfpathlineto{\pgfqpoint{1.422723in}{1.295769in}}%
\pgfpathlineto{\pgfqpoint{1.423645in}{1.287206in}}%
\pgfpathlineto{\pgfqpoint{1.424567in}{1.278616in}}%
\pgfpathlineto{\pgfqpoint{1.425489in}{1.270002in}}%
\pgfpathlineto{\pgfqpoint{1.416875in}{1.268890in}}%
\pgfpathlineto{\pgfqpoint{1.408192in}{1.267916in}}%
\pgfpathlineto{\pgfqpoint{1.399449in}{1.267080in}}%
\pgfpathlineto{\pgfqpoint{1.390654in}{1.266383in}}%
\pgfpathlineto{\pgfqpoint{1.390164in}{1.275047in}}%
\pgfpathlineto{\pgfqpoint{1.389674in}{1.283687in}}%
\pgfpathlineto{\pgfqpoint{1.389184in}{1.292299in}}%
\pgfpathlineto{\pgfqpoint{1.388694in}{1.300881in}}%
\pgfpathlineto{\pgfqpoint{1.397052in}{1.301540in}}%
\pgfpathlineto{\pgfqpoint{1.405362in}{1.302330in}}%
\pgfpathlineto{\pgfqpoint{1.413614in}{1.303251in}}%
\pgfpathlineto{\pgfqpoint{1.421801in}{1.304302in}}%
\pgfpathclose%
\pgfusepath{fill}%
\end{pgfscope}%
\begin{pgfscope}%
\pgfpathrectangle{\pgfqpoint{0.329460in}{0.284240in}}{\pgfqpoint{1.989680in}{1.989680in}}%
\pgfusepath{clip}%
\pgfsetbuttcap%
\pgfsetroundjoin%
\definecolor{currentfill}{rgb}{0.814576,0.883393,0.110347}%
\pgfsetfillcolor{currentfill}%
\pgfsetlinewidth{0.000000pt}%
\definecolor{currentstroke}{rgb}{0.000000,0.000000,0.000000}%
\pgfsetstrokecolor{currentstroke}%
\pgfsetdash{}{0pt}%
\pgfpathmoveto{\pgfqpoint{1.400169in}{1.644876in}}%
\pgfpathlineto{\pgfqpoint{1.401922in}{1.640125in}}%
\pgfpathlineto{\pgfqpoint{1.403675in}{1.635265in}}%
\pgfpathlineto{\pgfqpoint{1.405427in}{1.630298in}}%
\pgfpathlineto{\pgfqpoint{1.407180in}{1.625225in}}%
\pgfpathlineto{\pgfqpoint{1.404067in}{1.624410in}}%
\pgfpathlineto{\pgfqpoint{1.400899in}{1.623641in}}%
\pgfpathlineto{\pgfqpoint{1.397681in}{1.622921in}}%
\pgfpathlineto{\pgfqpoint{1.394416in}{1.622249in}}%
\pgfpathlineto{\pgfqpoint{1.393062in}{1.627417in}}%
\pgfpathlineto{\pgfqpoint{1.391708in}{1.632480in}}%
\pgfpathlineto{\pgfqpoint{1.390354in}{1.637435in}}%
\pgfpathlineto{\pgfqpoint{1.389000in}{1.642281in}}%
\pgfpathlineto{\pgfqpoint{1.391857in}{1.642867in}}%
\pgfpathlineto{\pgfqpoint{1.394673in}{1.643495in}}%
\pgfpathlineto{\pgfqpoint{1.397444in}{1.644165in}}%
\pgfpathlineto{\pgfqpoint{1.400169in}{1.644876in}}%
\pgfpathclose%
\pgfusepath{fill}%
\end{pgfscope}%
\begin{pgfscope}%
\pgfpathrectangle{\pgfqpoint{0.329460in}{0.284240in}}{\pgfqpoint{1.989680in}{1.989680in}}%
\pgfusepath{clip}%
\pgfsetbuttcap%
\pgfsetroundjoin%
\definecolor{currentfill}{rgb}{0.166383,0.690856,0.496502}%
\pgfsetfillcolor{currentfill}%
\pgfsetlinewidth{0.000000pt}%
\definecolor{currentstroke}{rgb}{0.000000,0.000000,0.000000}%
\pgfsetstrokecolor{currentstroke}%
\pgfsetdash{}{0pt}%
\pgfpathmoveto{\pgfqpoint{1.442991in}{1.376083in}}%
\pgfpathlineto{\pgfqpoint{1.444333in}{1.367944in}}%
\pgfpathlineto{\pgfqpoint{1.445674in}{1.359758in}}%
\pgfpathlineto{\pgfqpoint{1.447015in}{1.351527in}}%
\pgfpathlineto{\pgfqpoint{1.448355in}{1.343253in}}%
\pgfpathlineto{\pgfqpoint{1.440927in}{1.341786in}}%
\pgfpathlineto{\pgfqpoint{1.433405in}{1.340437in}}%
\pgfpathlineto{\pgfqpoint{1.425796in}{1.339205in}}%
\pgfpathlineto{\pgfqpoint{1.418108in}{1.338094in}}%
\pgfpathlineto{\pgfqpoint{1.417184in}{1.346446in}}%
\pgfpathlineto{\pgfqpoint{1.416259in}{1.354755in}}%
\pgfpathlineto{\pgfqpoint{1.415335in}{1.363019in}}%
\pgfpathlineto{\pgfqpoint{1.414410in}{1.371235in}}%
\pgfpathlineto{\pgfqpoint{1.421674in}{1.372280in}}%
\pgfpathlineto{\pgfqpoint{1.428864in}{1.373437in}}%
\pgfpathlineto{\pgfqpoint{1.435972in}{1.374705in}}%
\pgfpathlineto{\pgfqpoint{1.442991in}{1.376083in}}%
\pgfpathclose%
\pgfusepath{fill}%
\end{pgfscope}%
\begin{pgfscope}%
\pgfpathrectangle{\pgfqpoint{0.329460in}{0.284240in}}{\pgfqpoint{1.989680in}{1.989680in}}%
\pgfusepath{clip}%
\pgfsetbuttcap%
\pgfsetroundjoin%
\definecolor{currentfill}{rgb}{0.855810,0.888601,0.097452}%
\pgfsetfillcolor{currentfill}%
\pgfsetlinewidth{0.000000pt}%
\definecolor{currentstroke}{rgb}{0.000000,0.000000,0.000000}%
\pgfsetstrokecolor{currentstroke}%
\pgfsetdash{}{0pt}%
\pgfpathmoveto{\pgfqpoint{1.383585in}{1.660567in}}%
\pgfpathlineto{\pgfqpoint{1.384939in}{1.656162in}}%
\pgfpathlineto{\pgfqpoint{1.386292in}{1.651646in}}%
\pgfpathlineto{\pgfqpoint{1.387646in}{1.647019in}}%
\pgfpathlineto{\pgfqpoint{1.389000in}{1.642281in}}%
\pgfpathlineto{\pgfqpoint{1.386104in}{1.641738in}}%
\pgfpathlineto{\pgfqpoint{1.383173in}{1.641238in}}%
\pgfpathlineto{\pgfqpoint{1.380209in}{1.640783in}}%
\pgfpathlineto{\pgfqpoint{1.377216in}{1.640372in}}%
\pgfpathlineto{\pgfqpoint{1.376284in}{1.645179in}}%
\pgfpathlineto{\pgfqpoint{1.375352in}{1.649876in}}%
\pgfpathlineto{\pgfqpoint{1.374419in}{1.654462in}}%
\pgfpathlineto{\pgfqpoint{1.373487in}{1.658936in}}%
\pgfpathlineto{\pgfqpoint{1.376052in}{1.659287in}}%
\pgfpathlineto{\pgfqpoint{1.378592in}{1.659677in}}%
\pgfpathlineto{\pgfqpoint{1.381104in}{1.660103in}}%
\pgfpathlineto{\pgfqpoint{1.383585in}{1.660567in}}%
\pgfpathclose%
\pgfusepath{fill}%
\end{pgfscope}%
\begin{pgfscope}%
\pgfpathrectangle{\pgfqpoint{0.329460in}{0.284240in}}{\pgfqpoint{1.989680in}{1.989680in}}%
\pgfusepath{clip}%
\pgfsetbuttcap%
\pgfsetroundjoin%
\definecolor{currentfill}{rgb}{0.344074,0.780029,0.397381}%
\pgfsetfillcolor{currentfill}%
\pgfsetlinewidth{0.000000pt}%
\definecolor{currentstroke}{rgb}{0.000000,0.000000,0.000000}%
\pgfsetstrokecolor{currentstroke}%
\pgfsetdash{}{0pt}%
\pgfpathmoveto{\pgfqpoint{1.449156in}{1.474705in}}%
\pgfpathlineto{\pgfqpoint{1.450899in}{1.467385in}}%
\pgfpathlineto{\pgfqpoint{1.452642in}{1.459996in}}%
\pgfpathlineto{\pgfqpoint{1.454383in}{1.452539in}}%
\pgfpathlineto{\pgfqpoint{1.456125in}{1.445015in}}%
\pgfpathlineto{\pgfqpoint{1.450302in}{1.443440in}}%
\pgfpathlineto{\pgfqpoint{1.444376in}{1.441956in}}%
\pgfpathlineto{\pgfqpoint{1.438352in}{1.440563in}}%
\pgfpathlineto{\pgfqpoint{1.432238in}{1.439264in}}%
\pgfpathlineto{\pgfqpoint{1.430892in}{1.446890in}}%
\pgfpathlineto{\pgfqpoint{1.429545in}{1.454450in}}%
\pgfpathlineto{\pgfqpoint{1.428198in}{1.461941in}}%
\pgfpathlineto{\pgfqpoint{1.426851in}{1.469362in}}%
\pgfpathlineto{\pgfqpoint{1.432560in}{1.470569in}}%
\pgfpathlineto{\pgfqpoint{1.438184in}{1.471862in}}%
\pgfpathlineto{\pgfqpoint{1.443718in}{1.473242in}}%
\pgfpathlineto{\pgfqpoint{1.449156in}{1.474705in}}%
\pgfpathclose%
\pgfusepath{fill}%
\end{pgfscope}%
\begin{pgfscope}%
\pgfpathrectangle{\pgfqpoint{0.329460in}{0.284240in}}{\pgfqpoint{1.989680in}{1.989680in}}%
\pgfusepath{clip}%
\pgfsetbuttcap%
\pgfsetroundjoin%
\definecolor{currentfill}{rgb}{0.122606,0.585371,0.546557}%
\pgfsetfillcolor{currentfill}%
\pgfsetlinewidth{0.000000pt}%
\definecolor{currentstroke}{rgb}{0.000000,0.000000,0.000000}%
\pgfsetstrokecolor{currentstroke}%
\pgfsetdash{}{0pt}%
\pgfpathmoveto{\pgfqpoint{1.390654in}{1.266383in}}%
\pgfpathlineto{\pgfqpoint{1.391143in}{1.257695in}}%
\pgfpathlineto{\pgfqpoint{1.391633in}{1.248988in}}%
\pgfpathlineto{\pgfqpoint{1.392122in}{1.240263in}}%
\pgfpathlineto{\pgfqpoint{1.392612in}{1.231522in}}%
\pgfpathlineto{\pgfqpoint{1.383336in}{1.230935in}}%
\pgfpathlineto{\pgfqpoint{1.374027in}{1.230495in}}%
\pgfpathlineto{\pgfqpoint{1.364693in}{1.230205in}}%
\pgfpathlineto{\pgfqpoint{1.355344in}{1.230064in}}%
\pgfpathlineto{\pgfqpoint{1.355295in}{1.238824in}}%
\pgfpathlineto{\pgfqpoint{1.355246in}{1.247568in}}%
\pgfpathlineto{\pgfqpoint{1.355197in}{1.256294in}}%
\pgfpathlineto{\pgfqpoint{1.355148in}{1.265001in}}%
\pgfpathlineto{\pgfqpoint{1.364054in}{1.265134in}}%
\pgfpathlineto{\pgfqpoint{1.372947in}{1.265409in}}%
\pgfpathlineto{\pgfqpoint{1.381817in}{1.265826in}}%
\pgfpathlineto{\pgfqpoint{1.390654in}{1.266383in}}%
\pgfpathclose%
\pgfusepath{fill}%
\end{pgfscope}%
\begin{pgfscope}%
\pgfpathrectangle{\pgfqpoint{0.329460in}{0.284240in}}{\pgfqpoint{1.989680in}{1.989680in}}%
\pgfusepath{clip}%
\pgfsetbuttcap%
\pgfsetroundjoin%
\definecolor{currentfill}{rgb}{0.122606,0.585371,0.546557}%
\pgfsetfillcolor{currentfill}%
\pgfsetlinewidth{0.000000pt}%
\definecolor{currentstroke}{rgb}{0.000000,0.000000,0.000000}%
\pgfsetstrokecolor{currentstroke}%
\pgfsetdash{}{0pt}%
\pgfpathmoveto{\pgfqpoint{1.355148in}{1.265001in}}%
\pgfpathlineto{\pgfqpoint{1.355197in}{1.256294in}}%
\pgfpathlineto{\pgfqpoint{1.355246in}{1.247568in}}%
\pgfpathlineto{\pgfqpoint{1.355295in}{1.238824in}}%
\pgfpathlineto{\pgfqpoint{1.355344in}{1.230064in}}%
\pgfpathlineto{\pgfqpoint{1.345992in}{1.230072in}}%
\pgfpathlineto{\pgfqpoint{1.336644in}{1.230230in}}%
\pgfpathlineto{\pgfqpoint{1.327313in}{1.230537in}}%
\pgfpathlineto{\pgfqpoint{1.318006in}{1.230993in}}%
\pgfpathlineto{\pgfqpoint{1.318398in}{1.239740in}}%
\pgfpathlineto{\pgfqpoint{1.318790in}{1.248472in}}%
\pgfpathlineto{\pgfqpoint{1.319182in}{1.257187in}}%
\pgfpathlineto{\pgfqpoint{1.319575in}{1.265881in}}%
\pgfpathlineto{\pgfqpoint{1.328441in}{1.265449in}}%
\pgfpathlineto{\pgfqpoint{1.337332in}{1.265158in}}%
\pgfpathlineto{\pgfqpoint{1.346237in}{1.265008in}}%
\pgfpathlineto{\pgfqpoint{1.355148in}{1.265001in}}%
\pgfpathclose%
\pgfusepath{fill}%
\end{pgfscope}%
\begin{pgfscope}%
\pgfpathrectangle{\pgfqpoint{0.329460in}{0.284240in}}{\pgfqpoint{1.989680in}{1.989680in}}%
\pgfusepath{clip}%
\pgfsetbuttcap%
\pgfsetroundjoin%
\definecolor{currentfill}{rgb}{0.855810,0.888601,0.097452}%
\pgfsetfillcolor{currentfill}%
\pgfsetlinewidth{0.000000pt}%
\definecolor{currentstroke}{rgb}{0.000000,0.000000,0.000000}%
\pgfsetstrokecolor{currentstroke}%
\pgfsetdash{}{0pt}%
\pgfpathmoveto{\pgfqpoint{1.331187in}{1.658656in}}%
\pgfpathlineto{\pgfqpoint{1.330351in}{1.654170in}}%
\pgfpathlineto{\pgfqpoint{1.329515in}{1.649572in}}%
\pgfpathlineto{\pgfqpoint{1.328679in}{1.644863in}}%
\pgfpathlineto{\pgfqpoint{1.327842in}{1.640044in}}%
\pgfpathlineto{\pgfqpoint{1.324825in}{1.640415in}}%
\pgfpathlineto{\pgfqpoint{1.321835in}{1.640831in}}%
\pgfpathlineto{\pgfqpoint{1.318875in}{1.641292in}}%
\pgfpathlineto{\pgfqpoint{1.315947in}{1.641796in}}%
\pgfpathlineto{\pgfqpoint{1.317209in}{1.646551in}}%
\pgfpathlineto{\pgfqpoint{1.318471in}{1.651196in}}%
\pgfpathlineto{\pgfqpoint{1.319733in}{1.655731in}}%
\pgfpathlineto{\pgfqpoint{1.320994in}{1.660153in}}%
\pgfpathlineto{\pgfqpoint{1.323503in}{1.659722in}}%
\pgfpathlineto{\pgfqpoint{1.326039in}{1.659329in}}%
\pgfpathlineto{\pgfqpoint{1.328602in}{1.658973in}}%
\pgfpathlineto{\pgfqpoint{1.331187in}{1.658656in}}%
\pgfpathclose%
\pgfusepath{fill}%
\end{pgfscope}%
\begin{pgfscope}%
\pgfpathrectangle{\pgfqpoint{0.329460in}{0.284240in}}{\pgfqpoint{1.989680in}{1.989680in}}%
\pgfusepath{clip}%
\pgfsetbuttcap%
\pgfsetroundjoin%
\definecolor{currentfill}{rgb}{0.814576,0.883393,0.110347}%
\pgfsetfillcolor{currentfill}%
\pgfsetlinewidth{0.000000pt}%
\definecolor{currentstroke}{rgb}{0.000000,0.000000,0.000000}%
\pgfsetstrokecolor{currentstroke}%
\pgfsetdash{}{0pt}%
\pgfpathmoveto{\pgfqpoint{1.315947in}{1.641796in}}%
\pgfpathlineto{\pgfqpoint{1.314685in}{1.636932in}}%
\pgfpathlineto{\pgfqpoint{1.313423in}{1.631959in}}%
\pgfpathlineto{\pgfqpoint{1.312161in}{1.626879in}}%
\pgfpathlineto{\pgfqpoint{1.310900in}{1.621692in}}%
\pgfpathlineto{\pgfqpoint{1.307594in}{1.622321in}}%
\pgfpathlineto{\pgfqpoint{1.304334in}{1.622999in}}%
\pgfpathlineto{\pgfqpoint{1.301121in}{1.623725in}}%
\pgfpathlineto{\pgfqpoint{1.297960in}{1.624498in}}%
\pgfpathlineto{\pgfqpoint{1.299626in}{1.629594in}}%
\pgfpathlineto{\pgfqpoint{1.301293in}{1.634585in}}%
\pgfpathlineto{\pgfqpoint{1.302959in}{1.639467in}}%
\pgfpathlineto{\pgfqpoint{1.304626in}{1.644242in}}%
\pgfpathlineto{\pgfqpoint{1.307392in}{1.643568in}}%
\pgfpathlineto{\pgfqpoint{1.310203in}{1.642935in}}%
\pgfpathlineto{\pgfqpoint{1.313056in}{1.642344in}}%
\pgfpathlineto{\pgfqpoint{1.315947in}{1.641796in}}%
\pgfpathclose%
\pgfusepath{fill}%
\end{pgfscope}%
\begin{pgfscope}%
\pgfpathrectangle{\pgfqpoint{0.329460in}{0.284240in}}{\pgfqpoint{1.989680in}{1.989680in}}%
\pgfusepath{clip}%
\pgfsetbuttcap%
\pgfsetroundjoin%
\definecolor{currentfill}{rgb}{0.120081,0.622161,0.534946}%
\pgfsetfillcolor{currentfill}%
\pgfsetlinewidth{0.000000pt}%
\definecolor{currentstroke}{rgb}{0.000000,0.000000,0.000000}%
\pgfsetstrokecolor{currentstroke}%
\pgfsetdash{}{0pt}%
\pgfpathmoveto{\pgfqpoint{1.321145in}{1.300407in}}%
\pgfpathlineto{\pgfqpoint{1.320752in}{1.291818in}}%
\pgfpathlineto{\pgfqpoint{1.320360in}{1.283199in}}%
\pgfpathlineto{\pgfqpoint{1.319967in}{1.274552in}}%
\pgfpathlineto{\pgfqpoint{1.319575in}{1.265881in}}%
\pgfpathlineto{\pgfqpoint{1.310742in}{1.266453in}}%
\pgfpathlineto{\pgfqpoint{1.301952in}{1.267166in}}%
\pgfpathlineto{\pgfqpoint{1.293215in}{1.268017in}}%
\pgfpathlineto{\pgfqpoint{1.284539in}{1.269007in}}%
\pgfpathlineto{\pgfqpoint{1.285366in}{1.277635in}}%
\pgfpathlineto{\pgfqpoint{1.286193in}{1.286239in}}%
\pgfpathlineto{\pgfqpoint{1.287020in}{1.294815in}}%
\pgfpathlineto{\pgfqpoint{1.287848in}{1.303362in}}%
\pgfpathlineto{\pgfqpoint{1.296093in}{1.302426in}}%
\pgfpathlineto{\pgfqpoint{1.304397in}{1.301621in}}%
\pgfpathlineto{\pgfqpoint{1.312751in}{1.300948in}}%
\pgfpathlineto{\pgfqpoint{1.321145in}{1.300407in}}%
\pgfpathclose%
\pgfusepath{fill}%
\end{pgfscope}%
\begin{pgfscope}%
\pgfpathrectangle{\pgfqpoint{0.329460in}{0.284240in}}{\pgfqpoint{1.989680in}{1.989680in}}%
\pgfusepath{clip}%
\pgfsetbuttcap%
\pgfsetroundjoin%
\definecolor{currentfill}{rgb}{0.166383,0.690856,0.496502}%
\pgfsetfillcolor{currentfill}%
\pgfsetlinewidth{0.000000pt}%
\definecolor{currentstroke}{rgb}{0.000000,0.000000,0.000000}%
\pgfsetstrokecolor{currentstroke}%
\pgfsetdash{}{0pt}%
\pgfpathmoveto{\pgfqpoint{1.294478in}{1.370402in}}%
\pgfpathlineto{\pgfqpoint{1.293648in}{1.362172in}}%
\pgfpathlineto{\pgfqpoint{1.292819in}{1.353895in}}%
\pgfpathlineto{\pgfqpoint{1.291990in}{1.345573in}}%
\pgfpathlineto{\pgfqpoint{1.291161in}{1.337208in}}%
\pgfpathlineto{\pgfqpoint{1.283410in}{1.338211in}}%
\pgfpathlineto{\pgfqpoint{1.275730in}{1.339336in}}%
\pgfpathlineto{\pgfqpoint{1.268130in}{1.340581in}}%
\pgfpathlineto{\pgfqpoint{1.260617in}{1.341944in}}%
\pgfpathlineto{\pgfqpoint{1.261867in}{1.350237in}}%
\pgfpathlineto{\pgfqpoint{1.263117in}{1.358488in}}%
\pgfpathlineto{\pgfqpoint{1.264368in}{1.366694in}}%
\pgfpathlineto{\pgfqpoint{1.265619in}{1.374853in}}%
\pgfpathlineto{\pgfqpoint{1.272718in}{1.373572in}}%
\pgfpathlineto{\pgfqpoint{1.279899in}{1.372403in}}%
\pgfpathlineto{\pgfqpoint{1.287155in}{1.371346in}}%
\pgfpathlineto{\pgfqpoint{1.294478in}{1.370402in}}%
\pgfpathclose%
\pgfusepath{fill}%
\end{pgfscope}%
\begin{pgfscope}%
\pgfpathrectangle{\pgfqpoint{0.329460in}{0.284240in}}{\pgfqpoint{1.989680in}{1.989680in}}%
\pgfusepath{clip}%
\pgfsetbuttcap%
\pgfsetroundjoin%
\definecolor{currentfill}{rgb}{0.344074,0.780029,0.397381}%
\pgfsetfillcolor{currentfill}%
\pgfsetlinewidth{0.000000pt}%
\definecolor{currentstroke}{rgb}{0.000000,0.000000,0.000000}%
\pgfsetstrokecolor{currentstroke}%
\pgfsetdash{}{0pt}%
\pgfpathmoveto{\pgfqpoint{1.280665in}{1.468363in}}%
\pgfpathlineto{\pgfqpoint{1.279409in}{1.460923in}}%
\pgfpathlineto{\pgfqpoint{1.278153in}{1.453413in}}%
\pgfpathlineto{\pgfqpoint{1.276898in}{1.445834in}}%
\pgfpathlineto{\pgfqpoint{1.275643in}{1.438188in}}%
\pgfpathlineto{\pgfqpoint{1.269453in}{1.439403in}}%
\pgfpathlineto{\pgfqpoint{1.263349in}{1.440713in}}%
\pgfpathlineto{\pgfqpoint{1.257336in}{1.442116in}}%
\pgfpathlineto{\pgfqpoint{1.251421in}{1.443611in}}%
\pgfpathlineto{\pgfqpoint{1.253077in}{1.451160in}}%
\pgfpathlineto{\pgfqpoint{1.254733in}{1.458642in}}%
\pgfpathlineto{\pgfqpoint{1.256390in}{1.466056in}}%
\pgfpathlineto{\pgfqpoint{1.258048in}{1.473400in}}%
\pgfpathlineto{\pgfqpoint{1.263571in}{1.472011in}}%
\pgfpathlineto{\pgfqpoint{1.269186in}{1.470708in}}%
\pgfpathlineto{\pgfqpoint{1.274886in}{1.469491in}}%
\pgfpathlineto{\pgfqpoint{1.280665in}{1.468363in}}%
\pgfpathclose%
\pgfusepath{fill}%
\end{pgfscope}%
\begin{pgfscope}%
\pgfpathrectangle{\pgfqpoint{0.329460in}{0.284240in}}{\pgfqpoint{1.989680in}{1.989680in}}%
\pgfusepath{clip}%
\pgfsetbuttcap%
\pgfsetroundjoin%
\definecolor{currentfill}{rgb}{0.412913,0.803041,0.357269}%
\pgfsetfillcolor{currentfill}%
\pgfsetlinewidth{0.000000pt}%
\definecolor{currentstroke}{rgb}{0.000000,0.000000,0.000000}%
\pgfsetstrokecolor{currentstroke}%
\pgfsetdash{}{0pt}%
\pgfpathmoveto{\pgfqpoint{1.442178in}{1.503247in}}%
\pgfpathlineto{\pgfqpoint{1.443923in}{1.496225in}}%
\pgfpathlineto{\pgfqpoint{1.445668in}{1.489126in}}%
\pgfpathlineto{\pgfqpoint{1.447412in}{1.481952in}}%
\pgfpathlineto{\pgfqpoint{1.449156in}{1.474705in}}%
\pgfpathlineto{\pgfqpoint{1.443718in}{1.473242in}}%
\pgfpathlineto{\pgfqpoint{1.438184in}{1.471862in}}%
\pgfpathlineto{\pgfqpoint{1.432560in}{1.470569in}}%
\pgfpathlineto{\pgfqpoint{1.426851in}{1.469362in}}%
\pgfpathlineto{\pgfqpoint{1.425503in}{1.476711in}}%
\pgfpathlineto{\pgfqpoint{1.424155in}{1.483986in}}%
\pgfpathlineto{\pgfqpoint{1.422806in}{1.491186in}}%
\pgfpathlineto{\pgfqpoint{1.421457in}{1.498308in}}%
\pgfpathlineto{\pgfqpoint{1.426760in}{1.499424in}}%
\pgfpathlineto{\pgfqpoint{1.431985in}{1.500620in}}%
\pgfpathlineto{\pgfqpoint{1.437126in}{1.501895in}}%
\pgfpathlineto{\pgfqpoint{1.442178in}{1.503247in}}%
\pgfpathclose%
\pgfusepath{fill}%
\end{pgfscope}%
\begin{pgfscope}%
\pgfpathrectangle{\pgfqpoint{0.329460in}{0.284240in}}{\pgfqpoint{1.989680in}{1.989680in}}%
\pgfusepath{clip}%
\pgfsetbuttcap%
\pgfsetroundjoin%
\definecolor{currentfill}{rgb}{0.762373,0.876424,0.137064}%
\pgfsetfillcolor{currentfill}%
\pgfsetlinewidth{0.000000pt}%
\definecolor{currentstroke}{rgb}{0.000000,0.000000,0.000000}%
\pgfsetstrokecolor{currentstroke}%
\pgfsetdash{}{0pt}%
\pgfpathmoveto{\pgfqpoint{1.407180in}{1.625225in}}%
\pgfpathlineto{\pgfqpoint{1.408933in}{1.620047in}}%
\pgfpathlineto{\pgfqpoint{1.410685in}{1.614765in}}%
\pgfpathlineto{\pgfqpoint{1.412438in}{1.609379in}}%
\pgfpathlineto{\pgfqpoint{1.414190in}{1.603892in}}%
\pgfpathlineto{\pgfqpoint{1.410688in}{1.602971in}}%
\pgfpathlineto{\pgfqpoint{1.407125in}{1.602104in}}%
\pgfpathlineto{\pgfqpoint{1.403504in}{1.601290in}}%
\pgfpathlineto{\pgfqpoint{1.399830in}{1.600531in}}%
\pgfpathlineto{\pgfqpoint{1.398477in}{1.606115in}}%
\pgfpathlineto{\pgfqpoint{1.397123in}{1.611596in}}%
\pgfpathlineto{\pgfqpoint{1.395769in}{1.616975in}}%
\pgfpathlineto{\pgfqpoint{1.394416in}{1.622249in}}%
\pgfpathlineto{\pgfqpoint{1.397681in}{1.622921in}}%
\pgfpathlineto{\pgfqpoint{1.400899in}{1.623641in}}%
\pgfpathlineto{\pgfqpoint{1.404067in}{1.624410in}}%
\pgfpathlineto{\pgfqpoint{1.407180in}{1.625225in}}%
\pgfpathclose%
\pgfusepath{fill}%
\end{pgfscope}%
\begin{pgfscope}%
\pgfpathrectangle{\pgfqpoint{0.329460in}{0.284240in}}{\pgfqpoint{1.989680in}{1.989680in}}%
\pgfusepath{clip}%
\pgfsetbuttcap%
\pgfsetroundjoin%
\definecolor{currentfill}{rgb}{0.855810,0.888601,0.097452}%
\pgfsetfillcolor{currentfill}%
\pgfsetlinewidth{0.000000pt}%
\definecolor{currentstroke}{rgb}{0.000000,0.000000,0.000000}%
\pgfsetstrokecolor{currentstroke}%
\pgfsetdash{}{0pt}%
\pgfpathmoveto{\pgfqpoint{1.373487in}{1.658936in}}%
\pgfpathlineto{\pgfqpoint{1.374419in}{1.654462in}}%
\pgfpathlineto{\pgfqpoint{1.375352in}{1.649876in}}%
\pgfpathlineto{\pgfqpoint{1.376284in}{1.645179in}}%
\pgfpathlineto{\pgfqpoint{1.377216in}{1.640372in}}%
\pgfpathlineto{\pgfqpoint{1.374196in}{1.640005in}}%
\pgfpathlineto{\pgfqpoint{1.371153in}{1.639684in}}%
\pgfpathlineto{\pgfqpoint{1.368089in}{1.639409in}}%
\pgfpathlineto{\pgfqpoint{1.365008in}{1.639180in}}%
\pgfpathlineto{\pgfqpoint{1.364513in}{1.644030in}}%
\pgfpathlineto{\pgfqpoint{1.364018in}{1.648771in}}%
\pgfpathlineto{\pgfqpoint{1.363523in}{1.653401in}}%
\pgfpathlineto{\pgfqpoint{1.363028in}{1.657918in}}%
\pgfpathlineto{\pgfqpoint{1.365667in}{1.658114in}}%
\pgfpathlineto{\pgfqpoint{1.368292in}{1.658349in}}%
\pgfpathlineto{\pgfqpoint{1.370900in}{1.658623in}}%
\pgfpathlineto{\pgfqpoint{1.373487in}{1.658936in}}%
\pgfpathclose%
\pgfusepath{fill}%
\end{pgfscope}%
\begin{pgfscope}%
\pgfpathrectangle{\pgfqpoint{0.329460in}{0.284240in}}{\pgfqpoint{1.989680in}{1.989680in}}%
\pgfusepath{clip}%
\pgfsetbuttcap%
\pgfsetroundjoin%
\definecolor{currentfill}{rgb}{0.855810,0.888601,0.097452}%
\pgfsetfillcolor{currentfill}%
\pgfsetlinewidth{0.000000pt}%
\definecolor{currentstroke}{rgb}{0.000000,0.000000,0.000000}%
\pgfsetstrokecolor{currentstroke}%
\pgfsetdash{}{0pt}%
\pgfpathmoveto{\pgfqpoint{1.341704in}{1.657777in}}%
\pgfpathlineto{\pgfqpoint{1.341308in}{1.653253in}}%
\pgfpathlineto{\pgfqpoint{1.340911in}{1.648618in}}%
\pgfpathlineto{\pgfqpoint{1.340515in}{1.643871in}}%
\pgfpathlineto{\pgfqpoint{1.340118in}{1.639014in}}%
\pgfpathlineto{\pgfqpoint{1.337024in}{1.639203in}}%
\pgfpathlineto{\pgfqpoint{1.333945in}{1.639437in}}%
\pgfpathlineto{\pgfqpoint{1.330883in}{1.639718in}}%
\pgfpathlineto{\pgfqpoint{1.327842in}{1.640044in}}%
\pgfpathlineto{\pgfqpoint{1.328679in}{1.644863in}}%
\pgfpathlineto{\pgfqpoint{1.329515in}{1.649572in}}%
\pgfpathlineto{\pgfqpoint{1.330351in}{1.654170in}}%
\pgfpathlineto{\pgfqpoint{1.331187in}{1.658656in}}%
\pgfpathlineto{\pgfqpoint{1.333792in}{1.658378in}}%
\pgfpathlineto{\pgfqpoint{1.336415in}{1.658138in}}%
\pgfpathlineto{\pgfqpoint{1.339054in}{1.657938in}}%
\pgfpathlineto{\pgfqpoint{1.341704in}{1.657777in}}%
\pgfpathclose%
\pgfusepath{fill}%
\end{pgfscope}%
\begin{pgfscope}%
\pgfpathrectangle{\pgfqpoint{0.329460in}{0.284240in}}{\pgfqpoint{1.989680in}{1.989680in}}%
\pgfusepath{clip}%
\pgfsetbuttcap%
\pgfsetroundjoin%
\definecolor{currentfill}{rgb}{0.220124,0.725509,0.466226}%
\pgfsetfillcolor{currentfill}%
\pgfsetlinewidth{0.000000pt}%
\definecolor{currentstroke}{rgb}{0.000000,0.000000,0.000000}%
\pgfsetstrokecolor{currentstroke}%
\pgfsetdash{}{0pt}%
\pgfpathmoveto{\pgfqpoint{1.437618in}{1.408130in}}%
\pgfpathlineto{\pgfqpoint{1.438962in}{1.400199in}}%
\pgfpathlineto{\pgfqpoint{1.440305in}{1.392213in}}%
\pgfpathlineto{\pgfqpoint{1.441648in}{1.384174in}}%
\pgfpathlineto{\pgfqpoint{1.442991in}{1.376083in}}%
\pgfpathlineto{\pgfqpoint{1.435972in}{1.374705in}}%
\pgfpathlineto{\pgfqpoint{1.428864in}{1.373437in}}%
\pgfpathlineto{\pgfqpoint{1.421674in}{1.372280in}}%
\pgfpathlineto{\pgfqpoint{1.414410in}{1.371235in}}%
\pgfpathlineto{\pgfqpoint{1.413485in}{1.379403in}}%
\pgfpathlineto{\pgfqpoint{1.412559in}{1.387519in}}%
\pgfpathlineto{\pgfqpoint{1.411633in}{1.395582in}}%
\pgfpathlineto{\pgfqpoint{1.410707in}{1.403590in}}%
\pgfpathlineto{\pgfqpoint{1.417546in}{1.404568in}}%
\pgfpathlineto{\pgfqpoint{1.424316in}{1.405651in}}%
\pgfpathlineto{\pgfqpoint{1.431009in}{1.406839in}}%
\pgfpathlineto{\pgfqpoint{1.437618in}{1.408130in}}%
\pgfpathclose%
\pgfusepath{fill}%
\end{pgfscope}%
\begin{pgfscope}%
\pgfpathrectangle{\pgfqpoint{0.329460in}{0.284240in}}{\pgfqpoint{1.989680in}{1.989680in}}%
\pgfusepath{clip}%
\pgfsetbuttcap%
\pgfsetroundjoin%
\definecolor{currentfill}{rgb}{0.201239,0.383670,0.554294}%
\pgfsetfillcolor{currentfill}%
\pgfsetlinewidth{0.000000pt}%
\definecolor{currentstroke}{rgb}{0.000000,0.000000,0.000000}%
\pgfsetstrokecolor{currentstroke}%
\pgfsetdash{}{0pt}%
\pgfpathmoveto{\pgfqpoint{1.927319in}{1.042681in}}%
\pgfpathlineto{\pgfqpoint{1.930780in}{1.053615in}}%
\pgfpathlineto{\pgfqpoint{1.934259in}{1.064987in}}%
\pgfpathlineto{\pgfqpoint{1.937758in}{1.076804in}}%
\pgfpathlineto{\pgfqpoint{1.941276in}{1.089073in}}%
\pgfpathlineto{\pgfqpoint{1.930514in}{1.079512in}}%
\pgfpathlineto{\pgfqpoint{1.919124in}{1.070119in}}%
\pgfpathlineto{\pgfqpoint{1.907116in}{1.060903in}}%
\pgfpathlineto{\pgfqpoint{1.894500in}{1.051877in}}%
\pgfpathlineto{\pgfqpoint{1.891242in}{1.039759in}}%
\pgfpathlineto{\pgfqpoint{1.888002in}{1.028097in}}%
\pgfpathlineto{\pgfqpoint{1.884780in}{1.016882in}}%
\pgfpathlineto{\pgfqpoint{1.881576in}{1.006107in}}%
\pgfpathlineto{\pgfqpoint{1.893910in}{1.014982in}}%
\pgfpathlineto{\pgfqpoint{1.905652in}{1.024043in}}%
\pgfpathlineto{\pgfqpoint{1.916791in}{1.033279in}}%
\pgfpathlineto{\pgfqpoint{1.927319in}{1.042681in}}%
\pgfpathclose%
\pgfusepath{fill}%
\end{pgfscope}%
\begin{pgfscope}%
\pgfpathrectangle{\pgfqpoint{0.329460in}{0.284240in}}{\pgfqpoint{1.989680in}{1.989680in}}%
\pgfusepath{clip}%
\pgfsetbuttcap%
\pgfsetroundjoin%
\definecolor{currentfill}{rgb}{0.134692,0.658636,0.517649}%
\pgfsetfillcolor{currentfill}%
\pgfsetlinewidth{0.000000pt}%
\definecolor{currentstroke}{rgb}{0.000000,0.000000,0.000000}%
\pgfsetstrokecolor{currentstroke}%
\pgfsetdash{}{0pt}%
\pgfpathmoveto{\pgfqpoint{1.418108in}{1.338094in}}%
\pgfpathlineto{\pgfqpoint{1.419031in}{1.329702in}}%
\pgfpathlineto{\pgfqpoint{1.419955in}{1.321271in}}%
\pgfpathlineto{\pgfqpoint{1.420878in}{1.312803in}}%
\pgfpathlineto{\pgfqpoint{1.421801in}{1.304302in}}%
\pgfpathlineto{\pgfqpoint{1.413614in}{1.303251in}}%
\pgfpathlineto{\pgfqpoint{1.405362in}{1.302330in}}%
\pgfpathlineto{\pgfqpoint{1.397052in}{1.301540in}}%
\pgfpathlineto{\pgfqpoint{1.388694in}{1.300881in}}%
\pgfpathlineto{\pgfqpoint{1.388203in}{1.309432in}}%
\pgfpathlineto{\pgfqpoint{1.387713in}{1.317949in}}%
\pgfpathlineto{\pgfqpoint{1.387222in}{1.326429in}}%
\pgfpathlineto{\pgfqpoint{1.386731in}{1.334870in}}%
\pgfpathlineto{\pgfqpoint{1.394652in}{1.335491in}}%
\pgfpathlineto{\pgfqpoint{1.402528in}{1.336236in}}%
\pgfpathlineto{\pgfqpoint{1.410349in}{1.337104in}}%
\pgfpathlineto{\pgfqpoint{1.418108in}{1.338094in}}%
\pgfpathclose%
\pgfusepath{fill}%
\end{pgfscope}%
\begin{pgfscope}%
\pgfpathrectangle{\pgfqpoint{0.329460in}{0.284240in}}{\pgfqpoint{1.989680in}{1.989680in}}%
\pgfusepath{clip}%
\pgfsetbuttcap%
\pgfsetroundjoin%
\definecolor{currentfill}{rgb}{0.762373,0.876424,0.137064}%
\pgfsetfillcolor{currentfill}%
\pgfsetlinewidth{0.000000pt}%
\definecolor{currentstroke}{rgb}{0.000000,0.000000,0.000000}%
\pgfsetstrokecolor{currentstroke}%
\pgfsetdash{}{0pt}%
\pgfpathmoveto{\pgfqpoint{1.310900in}{1.621692in}}%
\pgfpathlineto{\pgfqpoint{1.309638in}{1.616400in}}%
\pgfpathlineto{\pgfqpoint{1.308376in}{1.611004in}}%
\pgfpathlineto{\pgfqpoint{1.307114in}{1.605504in}}%
\pgfpathlineto{\pgfqpoint{1.305853in}{1.599902in}}%
\pgfpathlineto{\pgfqpoint{1.302134in}{1.600612in}}%
\pgfpathlineto{\pgfqpoint{1.298466in}{1.601377in}}%
\pgfpathlineto{\pgfqpoint{1.294852in}{1.602197in}}%
\pgfpathlineto{\pgfqpoint{1.291295in}{1.603071in}}%
\pgfpathlineto{\pgfqpoint{1.292961in}{1.608582in}}%
\pgfpathlineto{\pgfqpoint{1.294627in}{1.613991in}}%
\pgfpathlineto{\pgfqpoint{1.296293in}{1.619297in}}%
\pgfpathlineto{\pgfqpoint{1.297960in}{1.624498in}}%
\pgfpathlineto{\pgfqpoint{1.301121in}{1.623725in}}%
\pgfpathlineto{\pgfqpoint{1.304334in}{1.622999in}}%
\pgfpathlineto{\pgfqpoint{1.307594in}{1.622321in}}%
\pgfpathlineto{\pgfqpoint{1.310900in}{1.621692in}}%
\pgfpathclose%
\pgfusepath{fill}%
\end{pgfscope}%
\begin{pgfscope}%
\pgfpathrectangle{\pgfqpoint{0.329460in}{0.284240in}}{\pgfqpoint{1.989680in}{1.989680in}}%
\pgfusepath{clip}%
\pgfsetbuttcap%
\pgfsetroundjoin%
\definecolor{currentfill}{rgb}{0.487026,0.823929,0.312321}%
\pgfsetfillcolor{currentfill}%
\pgfsetlinewidth{0.000000pt}%
\definecolor{currentstroke}{rgb}{0.000000,0.000000,0.000000}%
\pgfsetstrokecolor{currentstroke}%
\pgfsetdash{}{0pt}%
\pgfpathmoveto{\pgfqpoint{1.435191in}{1.530534in}}%
\pgfpathlineto{\pgfqpoint{1.436939in}{1.523836in}}%
\pgfpathlineto{\pgfqpoint{1.438685in}{1.517054in}}%
\pgfpathlineto{\pgfqpoint{1.440432in}{1.510191in}}%
\pgfpathlineto{\pgfqpoint{1.442178in}{1.503247in}}%
\pgfpathlineto{\pgfqpoint{1.437126in}{1.501895in}}%
\pgfpathlineto{\pgfqpoint{1.431985in}{1.500620in}}%
\pgfpathlineto{\pgfqpoint{1.426760in}{1.499424in}}%
\pgfpathlineto{\pgfqpoint{1.421457in}{1.498308in}}%
\pgfpathlineto{\pgfqpoint{1.420108in}{1.505352in}}%
\pgfpathlineto{\pgfqpoint{1.418758in}{1.512316in}}%
\pgfpathlineto{\pgfqpoint{1.417408in}{1.519198in}}%
\pgfpathlineto{\pgfqpoint{1.416057in}{1.525996in}}%
\pgfpathlineto{\pgfqpoint{1.420954in}{1.527021in}}%
\pgfpathlineto{\pgfqpoint{1.425778in}{1.528120in}}%
\pgfpathlineto{\pgfqpoint{1.430526in}{1.529291in}}%
\pgfpathlineto{\pgfqpoint{1.435191in}{1.530534in}}%
\pgfpathclose%
\pgfusepath{fill}%
\end{pgfscope}%
\begin{pgfscope}%
\pgfpathrectangle{\pgfqpoint{0.329460in}{0.284240in}}{\pgfqpoint{1.989680in}{1.989680in}}%
\pgfusepath{clip}%
\pgfsetbuttcap%
\pgfsetroundjoin%
\definecolor{currentfill}{rgb}{0.120081,0.622161,0.534946}%
\pgfsetfillcolor{currentfill}%
\pgfsetlinewidth{0.000000pt}%
\definecolor{currentstroke}{rgb}{0.000000,0.000000,0.000000}%
\pgfsetstrokecolor{currentstroke}%
\pgfsetdash{}{0pt}%
\pgfpathmoveto{\pgfqpoint{1.388694in}{1.300881in}}%
\pgfpathlineto{\pgfqpoint{1.389184in}{1.292299in}}%
\pgfpathlineto{\pgfqpoint{1.389674in}{1.283687in}}%
\pgfpathlineto{\pgfqpoint{1.390164in}{1.275047in}}%
\pgfpathlineto{\pgfqpoint{1.390654in}{1.266383in}}%
\pgfpathlineto{\pgfqpoint{1.381817in}{1.265826in}}%
\pgfpathlineto{\pgfqpoint{1.372947in}{1.265409in}}%
\pgfpathlineto{\pgfqpoint{1.364054in}{1.265134in}}%
\pgfpathlineto{\pgfqpoint{1.355148in}{1.265001in}}%
\pgfpathlineto{\pgfqpoint{1.355099in}{1.273684in}}%
\pgfpathlineto{\pgfqpoint{1.355050in}{1.282343in}}%
\pgfpathlineto{\pgfqpoint{1.355000in}{1.290974in}}%
\pgfpathlineto{\pgfqpoint{1.354951in}{1.299575in}}%
\pgfpathlineto{\pgfqpoint{1.363415in}{1.299702in}}%
\pgfpathlineto{\pgfqpoint{1.371866in}{1.299962in}}%
\pgfpathlineto{\pgfqpoint{1.380295in}{1.300355in}}%
\pgfpathlineto{\pgfqpoint{1.388694in}{1.300881in}}%
\pgfpathclose%
\pgfusepath{fill}%
\end{pgfscope}%
\begin{pgfscope}%
\pgfpathrectangle{\pgfqpoint{0.329460in}{0.284240in}}{\pgfqpoint{1.989680in}{1.989680in}}%
\pgfusepath{clip}%
\pgfsetbuttcap%
\pgfsetroundjoin%
\definecolor{currentfill}{rgb}{0.855810,0.888601,0.097452}%
\pgfsetfillcolor{currentfill}%
\pgfsetlinewidth{0.000000pt}%
\definecolor{currentstroke}{rgb}{0.000000,0.000000,0.000000}%
\pgfsetstrokecolor{currentstroke}%
\pgfsetdash{}{0pt}%
\pgfpathmoveto{\pgfqpoint{1.363028in}{1.657918in}}%
\pgfpathlineto{\pgfqpoint{1.363523in}{1.653401in}}%
\pgfpathlineto{\pgfqpoint{1.364018in}{1.648771in}}%
\pgfpathlineto{\pgfqpoint{1.364513in}{1.644030in}}%
\pgfpathlineto{\pgfqpoint{1.365008in}{1.639180in}}%
\pgfpathlineto{\pgfqpoint{1.361913in}{1.638996in}}%
\pgfpathlineto{\pgfqpoint{1.358806in}{1.638859in}}%
\pgfpathlineto{\pgfqpoint{1.355693in}{1.638769in}}%
\pgfpathlineto{\pgfqpoint{1.352574in}{1.638725in}}%
\pgfpathlineto{\pgfqpoint{1.352525in}{1.643592in}}%
\pgfpathlineto{\pgfqpoint{1.352475in}{1.648349in}}%
\pgfpathlineto{\pgfqpoint{1.352425in}{1.652995in}}%
\pgfpathlineto{\pgfqpoint{1.352376in}{1.657530in}}%
\pgfpathlineto{\pgfqpoint{1.355047in}{1.657567in}}%
\pgfpathlineto{\pgfqpoint{1.357715in}{1.657644in}}%
\pgfpathlineto{\pgfqpoint{1.360376in}{1.657762in}}%
\pgfpathlineto{\pgfqpoint{1.363028in}{1.657918in}}%
\pgfpathclose%
\pgfusepath{fill}%
\end{pgfscope}%
\begin{pgfscope}%
\pgfpathrectangle{\pgfqpoint{0.329460in}{0.284240in}}{\pgfqpoint{1.989680in}{1.989680in}}%
\pgfusepath{clip}%
\pgfsetbuttcap%
\pgfsetroundjoin%
\definecolor{currentfill}{rgb}{0.699415,0.867117,0.175971}%
\pgfsetfillcolor{currentfill}%
\pgfsetlinewidth{0.000000pt}%
\definecolor{currentstroke}{rgb}{0.000000,0.000000,0.000000}%
\pgfsetstrokecolor{currentstroke}%
\pgfsetdash{}{0pt}%
\pgfpathmoveto{\pgfqpoint{1.414190in}{1.603892in}}%
\pgfpathlineto{\pgfqpoint{1.415942in}{1.598304in}}%
\pgfpathlineto{\pgfqpoint{1.417693in}{1.592616in}}%
\pgfpathlineto{\pgfqpoint{1.419445in}{1.586830in}}%
\pgfpathlineto{\pgfqpoint{1.421196in}{1.580946in}}%
\pgfpathlineto{\pgfqpoint{1.417305in}{1.579919in}}%
\pgfpathlineto{\pgfqpoint{1.413347in}{1.578951in}}%
\pgfpathlineto{\pgfqpoint{1.409325in}{1.578043in}}%
\pgfpathlineto{\pgfqpoint{1.405243in}{1.577196in}}%
\pgfpathlineto{\pgfqpoint{1.403890in}{1.583177in}}%
\pgfpathlineto{\pgfqpoint{1.402537in}{1.589061in}}%
\pgfpathlineto{\pgfqpoint{1.401184in}{1.594846in}}%
\pgfpathlineto{\pgfqpoint{1.399830in}{1.600531in}}%
\pgfpathlineto{\pgfqpoint{1.403504in}{1.601290in}}%
\pgfpathlineto{\pgfqpoint{1.407125in}{1.602104in}}%
\pgfpathlineto{\pgfqpoint{1.410688in}{1.602971in}}%
\pgfpathlineto{\pgfqpoint{1.414190in}{1.603892in}}%
\pgfpathclose%
\pgfusepath{fill}%
\end{pgfscope}%
\begin{pgfscope}%
\pgfpathrectangle{\pgfqpoint{0.329460in}{0.284240in}}{\pgfqpoint{1.989680in}{1.989680in}}%
\pgfusepath{clip}%
\pgfsetbuttcap%
\pgfsetroundjoin%
\definecolor{currentfill}{rgb}{0.814576,0.883393,0.110347}%
\pgfsetfillcolor{currentfill}%
\pgfsetlinewidth{0.000000pt}%
\definecolor{currentstroke}{rgb}{0.000000,0.000000,0.000000}%
\pgfsetstrokecolor{currentstroke}%
\pgfsetdash{}{0pt}%
\pgfpathmoveto{\pgfqpoint{1.389000in}{1.642281in}}%
\pgfpathlineto{\pgfqpoint{1.390354in}{1.637435in}}%
\pgfpathlineto{\pgfqpoint{1.391708in}{1.632480in}}%
\pgfpathlineto{\pgfqpoint{1.393062in}{1.627417in}}%
\pgfpathlineto{\pgfqpoint{1.394416in}{1.622249in}}%
\pgfpathlineto{\pgfqpoint{1.391106in}{1.621626in}}%
\pgfpathlineto{\pgfqpoint{1.387755in}{1.621053in}}%
\pgfpathlineto{\pgfqpoint{1.384367in}{1.620530in}}%
\pgfpathlineto{\pgfqpoint{1.380945in}{1.620058in}}%
\pgfpathlineto{\pgfqpoint{1.380013in}{1.625297in}}%
\pgfpathlineto{\pgfqpoint{1.379080in}{1.630430in}}%
\pgfpathlineto{\pgfqpoint{1.378148in}{1.635455in}}%
\pgfpathlineto{\pgfqpoint{1.377216in}{1.640372in}}%
\pgfpathlineto{\pgfqpoint{1.380209in}{1.640783in}}%
\pgfpathlineto{\pgfqpoint{1.383173in}{1.641238in}}%
\pgfpathlineto{\pgfqpoint{1.386104in}{1.641738in}}%
\pgfpathlineto{\pgfqpoint{1.389000in}{1.642281in}}%
\pgfpathclose%
\pgfusepath{fill}%
\end{pgfscope}%
\begin{pgfscope}%
\pgfpathrectangle{\pgfqpoint{0.329460in}{0.284240in}}{\pgfqpoint{1.989680in}{1.989680in}}%
\pgfusepath{clip}%
\pgfsetbuttcap%
\pgfsetroundjoin%
\definecolor{currentfill}{rgb}{0.855810,0.888601,0.097452}%
\pgfsetfillcolor{currentfill}%
\pgfsetlinewidth{0.000000pt}%
\definecolor{currentstroke}{rgb}{0.000000,0.000000,0.000000}%
\pgfsetstrokecolor{currentstroke}%
\pgfsetdash{}{0pt}%
\pgfpathmoveto{\pgfqpoint{1.352376in}{1.657530in}}%
\pgfpathlineto{\pgfqpoint{1.352425in}{1.652995in}}%
\pgfpathlineto{\pgfqpoint{1.352475in}{1.648349in}}%
\pgfpathlineto{\pgfqpoint{1.352525in}{1.643592in}}%
\pgfpathlineto{\pgfqpoint{1.352574in}{1.638725in}}%
\pgfpathlineto{\pgfqpoint{1.349454in}{1.638727in}}%
\pgfpathlineto{\pgfqpoint{1.346336in}{1.638776in}}%
\pgfpathlineto{\pgfqpoint{1.343223in}{1.638872in}}%
\pgfpathlineto{\pgfqpoint{1.340118in}{1.639014in}}%
\pgfpathlineto{\pgfqpoint{1.340515in}{1.643871in}}%
\pgfpathlineto{\pgfqpoint{1.340911in}{1.648618in}}%
\pgfpathlineto{\pgfqpoint{1.341308in}{1.653253in}}%
\pgfpathlineto{\pgfqpoint{1.341704in}{1.657777in}}%
\pgfpathlineto{\pgfqpoint{1.344364in}{1.657656in}}%
\pgfpathlineto{\pgfqpoint{1.347031in}{1.657574in}}%
\pgfpathlineto{\pgfqpoint{1.349703in}{1.657532in}}%
\pgfpathlineto{\pgfqpoint{1.352376in}{1.657530in}}%
\pgfpathclose%
\pgfusepath{fill}%
\end{pgfscope}%
\begin{pgfscope}%
\pgfpathrectangle{\pgfqpoint{0.329460in}{0.284240in}}{\pgfqpoint{1.989680in}{1.989680in}}%
\pgfusepath{clip}%
\pgfsetbuttcap%
\pgfsetroundjoin%
\definecolor{currentfill}{rgb}{0.412913,0.803041,0.357269}%
\pgfsetfillcolor{currentfill}%
\pgfsetlinewidth{0.000000pt}%
\definecolor{currentstroke}{rgb}{0.000000,0.000000,0.000000}%
\pgfsetstrokecolor{currentstroke}%
\pgfsetdash{}{0pt}%
\pgfpathmoveto{\pgfqpoint{1.285693in}{1.497385in}}%
\pgfpathlineto{\pgfqpoint{1.284436in}{1.490244in}}%
\pgfpathlineto{\pgfqpoint{1.283179in}{1.483025in}}%
\pgfpathlineto{\pgfqpoint{1.281922in}{1.475731in}}%
\pgfpathlineto{\pgfqpoint{1.280665in}{1.468363in}}%
\pgfpathlineto{\pgfqpoint{1.274886in}{1.469491in}}%
\pgfpathlineto{\pgfqpoint{1.269186in}{1.470708in}}%
\pgfpathlineto{\pgfqpoint{1.263571in}{1.472011in}}%
\pgfpathlineto{\pgfqpoint{1.258048in}{1.473400in}}%
\pgfpathlineto{\pgfqpoint{1.259706in}{1.480672in}}%
\pgfpathlineto{\pgfqpoint{1.261365in}{1.487871in}}%
\pgfpathlineto{\pgfqpoint{1.263024in}{1.494994in}}%
\pgfpathlineto{\pgfqpoint{1.264683in}{1.502041in}}%
\pgfpathlineto{\pgfqpoint{1.269815in}{1.500758in}}%
\pgfpathlineto{\pgfqpoint{1.275030in}{1.499553in}}%
\pgfpathlineto{\pgfqpoint{1.280325in}{1.498428in}}%
\pgfpathlineto{\pgfqpoint{1.285693in}{1.497385in}}%
\pgfpathclose%
\pgfusepath{fill}%
\end{pgfscope}%
\begin{pgfscope}%
\pgfpathrectangle{\pgfqpoint{0.329460in}{0.284240in}}{\pgfqpoint{1.989680in}{1.989680in}}%
\pgfusepath{clip}%
\pgfsetbuttcap%
\pgfsetroundjoin%
\definecolor{currentfill}{rgb}{0.120081,0.622161,0.534946}%
\pgfsetfillcolor{currentfill}%
\pgfsetlinewidth{0.000000pt}%
\definecolor{currentstroke}{rgb}{0.000000,0.000000,0.000000}%
\pgfsetstrokecolor{currentstroke}%
\pgfsetdash{}{0pt}%
\pgfpathmoveto{\pgfqpoint{1.354951in}{1.299575in}}%
\pgfpathlineto{\pgfqpoint{1.355000in}{1.290974in}}%
\pgfpathlineto{\pgfqpoint{1.355050in}{1.282343in}}%
\pgfpathlineto{\pgfqpoint{1.355099in}{1.273684in}}%
\pgfpathlineto{\pgfqpoint{1.355148in}{1.265001in}}%
\pgfpathlineto{\pgfqpoint{1.346237in}{1.265008in}}%
\pgfpathlineto{\pgfqpoint{1.337332in}{1.265158in}}%
\pgfpathlineto{\pgfqpoint{1.328441in}{1.265449in}}%
\pgfpathlineto{\pgfqpoint{1.319575in}{1.265881in}}%
\pgfpathlineto{\pgfqpoint{1.319967in}{1.274552in}}%
\pgfpathlineto{\pgfqpoint{1.320360in}{1.283199in}}%
\pgfpathlineto{\pgfqpoint{1.320752in}{1.291818in}}%
\pgfpathlineto{\pgfqpoint{1.321145in}{1.300407in}}%
\pgfpathlineto{\pgfqpoint{1.329571in}{1.299999in}}%
\pgfpathlineto{\pgfqpoint{1.338020in}{1.299724in}}%
\pgfpathlineto{\pgfqpoint{1.346483in}{1.299583in}}%
\pgfpathlineto{\pgfqpoint{1.354951in}{1.299575in}}%
\pgfpathclose%
\pgfusepath{fill}%
\end{pgfscope}%
\begin{pgfscope}%
\pgfpathrectangle{\pgfqpoint{0.329460in}{0.284240in}}{\pgfqpoint{1.989680in}{1.989680in}}%
\pgfusepath{clip}%
\pgfsetbuttcap%
\pgfsetroundjoin%
\definecolor{currentfill}{rgb}{0.134692,0.658636,0.517649}%
\pgfsetfillcolor{currentfill}%
\pgfsetlinewidth{0.000000pt}%
\definecolor{currentstroke}{rgb}{0.000000,0.000000,0.000000}%
\pgfsetstrokecolor{currentstroke}%
\pgfsetdash{}{0pt}%
\pgfpathmoveto{\pgfqpoint{1.322717in}{1.334423in}}%
\pgfpathlineto{\pgfqpoint{1.322324in}{1.325975in}}%
\pgfpathlineto{\pgfqpoint{1.321931in}{1.317488in}}%
\pgfpathlineto{\pgfqpoint{1.321538in}{1.308965in}}%
\pgfpathlineto{\pgfqpoint{1.321145in}{1.300407in}}%
\pgfpathlineto{\pgfqpoint{1.312751in}{1.300948in}}%
\pgfpathlineto{\pgfqpoint{1.304397in}{1.301621in}}%
\pgfpathlineto{\pgfqpoint{1.296093in}{1.302426in}}%
\pgfpathlineto{\pgfqpoint{1.287848in}{1.303362in}}%
\pgfpathlineto{\pgfqpoint{1.288676in}{1.311877in}}%
\pgfpathlineto{\pgfqpoint{1.289504in}{1.320357in}}%
\pgfpathlineto{\pgfqpoint{1.290332in}{1.328802in}}%
\pgfpathlineto{\pgfqpoint{1.291161in}{1.337208in}}%
\pgfpathlineto{\pgfqpoint{1.298975in}{1.336326in}}%
\pgfpathlineto{\pgfqpoint{1.306845in}{1.335568in}}%
\pgfpathlineto{\pgfqpoint{1.314762in}{1.334933in}}%
\pgfpathlineto{\pgfqpoint{1.322717in}{1.334423in}}%
\pgfpathclose%
\pgfusepath{fill}%
\end{pgfscope}%
\begin{pgfscope}%
\pgfpathrectangle{\pgfqpoint{0.329460in}{0.284240in}}{\pgfqpoint{1.989680in}{1.989680in}}%
\pgfusepath{clip}%
\pgfsetbuttcap%
\pgfsetroundjoin%
\definecolor{currentfill}{rgb}{0.565498,0.842430,0.262877}%
\pgfsetfillcolor{currentfill}%
\pgfsetlinewidth{0.000000pt}%
\definecolor{currentstroke}{rgb}{0.000000,0.000000,0.000000}%
\pgfsetstrokecolor{currentstroke}%
\pgfsetdash{}{0pt}%
\pgfpathmoveto{\pgfqpoint{1.428197in}{1.556465in}}%
\pgfpathlineto{\pgfqpoint{1.429946in}{1.550115in}}%
\pgfpathlineto{\pgfqpoint{1.431695in}{1.543675in}}%
\pgfpathlineto{\pgfqpoint{1.433443in}{1.537148in}}%
\pgfpathlineto{\pgfqpoint{1.435191in}{1.530534in}}%
\pgfpathlineto{\pgfqpoint{1.430526in}{1.529291in}}%
\pgfpathlineto{\pgfqpoint{1.425778in}{1.528120in}}%
\pgfpathlineto{\pgfqpoint{1.420954in}{1.527021in}}%
\pgfpathlineto{\pgfqpoint{1.416057in}{1.525996in}}%
\pgfpathlineto{\pgfqpoint{1.414706in}{1.532709in}}%
\pgfpathlineto{\pgfqpoint{1.413355in}{1.539335in}}%
\pgfpathlineto{\pgfqpoint{1.412004in}{1.545874in}}%
\pgfpathlineto{\pgfqpoint{1.410652in}{1.552323in}}%
\pgfpathlineto{\pgfqpoint{1.415142in}{1.553258in}}%
\pgfpathlineto{\pgfqpoint{1.419565in}{1.554261in}}%
\pgfpathlineto{\pgfqpoint{1.423918in}{1.555331in}}%
\pgfpathlineto{\pgfqpoint{1.428197in}{1.556465in}}%
\pgfpathclose%
\pgfusepath{fill}%
\end{pgfscope}%
\begin{pgfscope}%
\pgfpathrectangle{\pgfqpoint{0.329460in}{0.284240in}}{\pgfqpoint{1.989680in}{1.989680in}}%
\pgfusepath{clip}%
\pgfsetbuttcap%
\pgfsetroundjoin%
\definecolor{currentfill}{rgb}{0.636902,0.856542,0.216620}%
\pgfsetfillcolor{currentfill}%
\pgfsetlinewidth{0.000000pt}%
\definecolor{currentstroke}{rgb}{0.000000,0.000000,0.000000}%
\pgfsetstrokecolor{currentstroke}%
\pgfsetdash{}{0pt}%
\pgfpathmoveto{\pgfqpoint{1.421196in}{1.580946in}}%
\pgfpathlineto{\pgfqpoint{1.422946in}{1.574967in}}%
\pgfpathlineto{\pgfqpoint{1.424697in}{1.568892in}}%
\pgfpathlineto{\pgfqpoint{1.426447in}{1.562725in}}%
\pgfpathlineto{\pgfqpoint{1.428197in}{1.556465in}}%
\pgfpathlineto{\pgfqpoint{1.423918in}{1.555331in}}%
\pgfpathlineto{\pgfqpoint{1.419565in}{1.554261in}}%
\pgfpathlineto{\pgfqpoint{1.415142in}{1.553258in}}%
\pgfpathlineto{\pgfqpoint{1.410652in}{1.552323in}}%
\pgfpathlineto{\pgfqpoint{1.409300in}{1.558681in}}%
\pgfpathlineto{\pgfqpoint{1.407948in}{1.564947in}}%
\pgfpathlineto{\pgfqpoint{1.406596in}{1.571119in}}%
\pgfpathlineto{\pgfqpoint{1.405243in}{1.577196in}}%
\pgfpathlineto{\pgfqpoint{1.409325in}{1.578043in}}%
\pgfpathlineto{\pgfqpoint{1.413347in}{1.578951in}}%
\pgfpathlineto{\pgfqpoint{1.417305in}{1.579919in}}%
\pgfpathlineto{\pgfqpoint{1.421196in}{1.580946in}}%
\pgfpathclose%
\pgfusepath{fill}%
\end{pgfscope}%
\begin{pgfscope}%
\pgfpathrectangle{\pgfqpoint{0.329460in}{0.284240in}}{\pgfqpoint{1.989680in}{1.989680in}}%
\pgfusepath{clip}%
\pgfsetbuttcap%
\pgfsetroundjoin%
\definecolor{currentfill}{rgb}{0.814576,0.883393,0.110347}%
\pgfsetfillcolor{currentfill}%
\pgfsetlinewidth{0.000000pt}%
\definecolor{currentstroke}{rgb}{0.000000,0.000000,0.000000}%
\pgfsetstrokecolor{currentstroke}%
\pgfsetdash{}{0pt}%
\pgfpathmoveto{\pgfqpoint{1.327842in}{1.640044in}}%
\pgfpathlineto{\pgfqpoint{1.327006in}{1.635115in}}%
\pgfpathlineto{\pgfqpoint{1.326170in}{1.630078in}}%
\pgfpathlineto{\pgfqpoint{1.325334in}{1.624933in}}%
\pgfpathlineto{\pgfqpoint{1.324497in}{1.619682in}}%
\pgfpathlineto{\pgfqpoint{1.321048in}{1.620108in}}%
\pgfpathlineto{\pgfqpoint{1.317630in}{1.620585in}}%
\pgfpathlineto{\pgfqpoint{1.314246in}{1.621114in}}%
\pgfpathlineto{\pgfqpoint{1.310900in}{1.621692in}}%
\pgfpathlineto{\pgfqpoint{1.312161in}{1.626879in}}%
\pgfpathlineto{\pgfqpoint{1.313423in}{1.631959in}}%
\pgfpathlineto{\pgfqpoint{1.314685in}{1.636932in}}%
\pgfpathlineto{\pgfqpoint{1.315947in}{1.641796in}}%
\pgfpathlineto{\pgfqpoint{1.318875in}{1.641292in}}%
\pgfpathlineto{\pgfqpoint{1.321835in}{1.640831in}}%
\pgfpathlineto{\pgfqpoint{1.324825in}{1.640415in}}%
\pgfpathlineto{\pgfqpoint{1.327842in}{1.640044in}}%
\pgfpathclose%
\pgfusepath{fill}%
\end{pgfscope}%
\begin{pgfscope}%
\pgfpathrectangle{\pgfqpoint{0.329460in}{0.284240in}}{\pgfqpoint{1.989680in}{1.989680in}}%
\pgfusepath{clip}%
\pgfsetbuttcap%
\pgfsetroundjoin%
\definecolor{currentfill}{rgb}{0.220124,0.725509,0.466226}%
\pgfsetfillcolor{currentfill}%
\pgfsetlinewidth{0.000000pt}%
\definecolor{currentstroke}{rgb}{0.000000,0.000000,0.000000}%
\pgfsetstrokecolor{currentstroke}%
\pgfsetdash{}{0pt}%
\pgfpathmoveto{\pgfqpoint{1.297800in}{1.402810in}}%
\pgfpathlineto{\pgfqpoint{1.296969in}{1.394789in}}%
\pgfpathlineto{\pgfqpoint{1.296138in}{1.386713in}}%
\pgfpathlineto{\pgfqpoint{1.295308in}{1.378583in}}%
\pgfpathlineto{\pgfqpoint{1.294478in}{1.370402in}}%
\pgfpathlineto{\pgfqpoint{1.287155in}{1.371346in}}%
\pgfpathlineto{\pgfqpoint{1.279899in}{1.372403in}}%
\pgfpathlineto{\pgfqpoint{1.272718in}{1.373572in}}%
\pgfpathlineto{\pgfqpoint{1.265619in}{1.374853in}}%
\pgfpathlineto{\pgfqpoint{1.266871in}{1.382963in}}%
\pgfpathlineto{\pgfqpoint{1.268123in}{1.391021in}}%
\pgfpathlineto{\pgfqpoint{1.269375in}{1.399027in}}%
\pgfpathlineto{\pgfqpoint{1.270628in}{1.406978in}}%
\pgfpathlineto{\pgfqpoint{1.277312in}{1.405778in}}%
\pgfpathlineto{\pgfqpoint{1.284073in}{1.404683in}}%
\pgfpathlineto{\pgfqpoint{1.290905in}{1.403693in}}%
\pgfpathlineto{\pgfqpoint{1.297800in}{1.402810in}}%
\pgfpathclose%
\pgfusepath{fill}%
\end{pgfscope}%
\begin{pgfscope}%
\pgfpathrectangle{\pgfqpoint{0.329460in}{0.284240in}}{\pgfqpoint{1.989680in}{1.989680in}}%
\pgfusepath{clip}%
\pgfsetbuttcap%
\pgfsetroundjoin%
\definecolor{currentfill}{rgb}{0.267004,0.004874,0.329415}%
\pgfsetfillcolor{currentfill}%
\pgfsetlinewidth{0.000000pt}%
\definecolor{currentstroke}{rgb}{0.000000,0.000000,0.000000}%
\pgfsetstrokecolor{currentstroke}%
\pgfsetdash{}{0pt}%
\pgfpathmoveto{\pgfqpoint{1.426147in}{0.757190in}}%
\pgfpathlineto{\pgfqpoint{1.426659in}{0.754630in}}%
\pgfpathlineto{\pgfqpoint{1.427172in}{0.752287in}}%
\pgfpathlineto{\pgfqpoint{1.427686in}{0.750165in}}%
\pgfpathlineto{\pgfqpoint{1.428201in}{0.748268in}}%
\pgfpathlineto{\pgfqpoint{1.410962in}{0.747091in}}%
\pgfpathlineto{\pgfqpoint{1.393655in}{0.746211in}}%
\pgfpathlineto{\pgfqpoint{1.376300in}{0.745630in}}%
\pgfpathlineto{\pgfqpoint{1.358917in}{0.745347in}}%
\pgfpathlineto{\pgfqpoint{1.358866in}{0.747264in}}%
\pgfpathlineto{\pgfqpoint{1.358814in}{0.749407in}}%
\pgfpathlineto{\pgfqpoint{1.358762in}{0.751770in}}%
\pgfpathlineto{\pgfqpoint{1.358711in}{0.754351in}}%
\pgfpathlineto{\pgfqpoint{1.375630in}{0.754626in}}%
\pgfpathlineto{\pgfqpoint{1.392522in}{0.755191in}}%
\pgfpathlineto{\pgfqpoint{1.409367in}{0.756046in}}%
\pgfpathlineto{\pgfqpoint{1.426147in}{0.757190in}}%
\pgfpathclose%
\pgfusepath{fill}%
\end{pgfscope}%
\begin{pgfscope}%
\pgfpathrectangle{\pgfqpoint{0.329460in}{0.284240in}}{\pgfqpoint{1.989680in}{1.989680in}}%
\pgfusepath{clip}%
\pgfsetbuttcap%
\pgfsetroundjoin%
\definecolor{currentfill}{rgb}{0.699415,0.867117,0.175971}%
\pgfsetfillcolor{currentfill}%
\pgfsetlinewidth{0.000000pt}%
\definecolor{currentstroke}{rgb}{0.000000,0.000000,0.000000}%
\pgfsetstrokecolor{currentstroke}%
\pgfsetdash{}{0pt}%
\pgfpathmoveto{\pgfqpoint{1.305853in}{1.599902in}}%
\pgfpathlineto{\pgfqpoint{1.304591in}{1.594199in}}%
\pgfpathlineto{\pgfqpoint{1.303330in}{1.588396in}}%
\pgfpathlineto{\pgfqpoint{1.302069in}{1.582495in}}%
\pgfpathlineto{\pgfqpoint{1.300807in}{1.576495in}}%
\pgfpathlineto{\pgfqpoint{1.296676in}{1.577287in}}%
\pgfpathlineto{\pgfqpoint{1.292600in}{1.578141in}}%
\pgfpathlineto{\pgfqpoint{1.288585in}{1.579056in}}%
\pgfpathlineto{\pgfqpoint{1.284634in}{1.580030in}}%
\pgfpathlineto{\pgfqpoint{1.286299in}{1.585938in}}%
\pgfpathlineto{\pgfqpoint{1.287964in}{1.591748in}}%
\pgfpathlineto{\pgfqpoint{1.289630in}{1.597459in}}%
\pgfpathlineto{\pgfqpoint{1.291295in}{1.603071in}}%
\pgfpathlineto{\pgfqpoint{1.294852in}{1.602197in}}%
\pgfpathlineto{\pgfqpoint{1.298466in}{1.601377in}}%
\pgfpathlineto{\pgfqpoint{1.302134in}{1.600612in}}%
\pgfpathlineto{\pgfqpoint{1.305853in}{1.599902in}}%
\pgfpathclose%
\pgfusepath{fill}%
\end{pgfscope}%
\begin{pgfscope}%
\pgfpathrectangle{\pgfqpoint{0.329460in}{0.284240in}}{\pgfqpoint{1.989680in}{1.989680in}}%
\pgfusepath{clip}%
\pgfsetbuttcap%
\pgfsetroundjoin%
\definecolor{currentfill}{rgb}{0.272594,0.025563,0.353093}%
\pgfsetfillcolor{currentfill}%
\pgfsetlinewidth{0.000000pt}%
\definecolor{currentstroke}{rgb}{0.000000,0.000000,0.000000}%
\pgfsetstrokecolor{currentstroke}%
\pgfsetdash{}{0pt}%
\pgfpathmoveto{\pgfqpoint{1.144621in}{0.759456in}}%
\pgfpathlineto{\pgfqpoint{1.143280in}{0.759915in}}%
\pgfpathlineto{\pgfqpoint{1.141935in}{0.760646in}}%
\pgfpathlineto{\pgfqpoint{1.140586in}{0.761656in}}%
\pgfpathlineto{\pgfqpoint{1.139232in}{0.762949in}}%
\pgfpathlineto{\pgfqpoint{1.121948in}{0.766736in}}%
\pgfpathlineto{\pgfqpoint{1.104923in}{0.770814in}}%
\pgfpathlineto{\pgfqpoint{1.088175in}{0.775180in}}%
\pgfpathlineto{\pgfqpoint{1.071723in}{0.779827in}}%
\pgfpathlineto{\pgfqpoint{1.073503in}{0.778429in}}%
\pgfpathlineto{\pgfqpoint{1.075277in}{0.777315in}}%
\pgfpathlineto{\pgfqpoint{1.077046in}{0.776478in}}%
\pgfpathlineto{\pgfqpoint{1.078810in}{0.775914in}}%
\pgfpathlineto{\pgfqpoint{1.094850in}{0.771382in}}%
\pgfpathlineto{\pgfqpoint{1.111177in}{0.767125in}}%
\pgfpathlineto{\pgfqpoint{1.127773in}{0.763148in}}%
\pgfpathlineto{\pgfqpoint{1.144621in}{0.759456in}}%
\pgfpathclose%
\pgfusepath{fill}%
\end{pgfscope}%
\begin{pgfscope}%
\pgfpathrectangle{\pgfqpoint{0.329460in}{0.284240in}}{\pgfqpoint{1.989680in}{1.989680in}}%
\pgfusepath{clip}%
\pgfsetbuttcap%
\pgfsetroundjoin%
\definecolor{currentfill}{rgb}{0.487026,0.823929,0.312321}%
\pgfsetfillcolor{currentfill}%
\pgfsetlinewidth{0.000000pt}%
\definecolor{currentstroke}{rgb}{0.000000,0.000000,0.000000}%
\pgfsetstrokecolor{currentstroke}%
\pgfsetdash{}{0pt}%
\pgfpathmoveto{\pgfqpoint{1.290727in}{1.525147in}}%
\pgfpathlineto{\pgfqpoint{1.289468in}{1.518331in}}%
\pgfpathlineto{\pgfqpoint{1.288209in}{1.511430in}}%
\pgfpathlineto{\pgfqpoint{1.286951in}{1.504448in}}%
\pgfpathlineto{\pgfqpoint{1.285693in}{1.497385in}}%
\pgfpathlineto{\pgfqpoint{1.280325in}{1.498428in}}%
\pgfpathlineto{\pgfqpoint{1.275030in}{1.499553in}}%
\pgfpathlineto{\pgfqpoint{1.269815in}{1.500758in}}%
\pgfpathlineto{\pgfqpoint{1.264683in}{1.502041in}}%
\pgfpathlineto{\pgfqpoint{1.266344in}{1.509009in}}%
\pgfpathlineto{\pgfqpoint{1.268004in}{1.515897in}}%
\pgfpathlineto{\pgfqpoint{1.269665in}{1.522703in}}%
\pgfpathlineto{\pgfqpoint{1.271327in}{1.529426in}}%
\pgfpathlineto{\pgfqpoint{1.276065in}{1.528246in}}%
\pgfpathlineto{\pgfqpoint{1.280882in}{1.527139in}}%
\pgfpathlineto{\pgfqpoint{1.285770in}{1.526106in}}%
\pgfpathlineto{\pgfqpoint{1.290727in}{1.525147in}}%
\pgfpathclose%
\pgfusepath{fill}%
\end{pgfscope}%
\begin{pgfscope}%
\pgfpathrectangle{\pgfqpoint{0.329460in}{0.284240in}}{\pgfqpoint{1.989680in}{1.989680in}}%
\pgfusepath{clip}%
\pgfsetbuttcap%
\pgfsetroundjoin%
\definecolor{currentfill}{rgb}{0.260571,0.246922,0.522828}%
\pgfsetfillcolor{currentfill}%
\pgfsetlinewidth{0.000000pt}%
\definecolor{currentstroke}{rgb}{0.000000,0.000000,0.000000}%
\pgfsetstrokecolor{currentstroke}%
\pgfsetdash{}{0pt}%
\pgfpathmoveto{\pgfqpoint{0.911166in}{0.896022in}}%
\pgfpathlineto{\pgfqpoint{0.908471in}{0.903340in}}%
\pgfpathlineto{\pgfqpoint{0.905762in}{0.911044in}}%
\pgfpathlineto{\pgfqpoint{0.903041in}{0.919141in}}%
\pgfpathlineto{\pgfqpoint{0.900306in}{0.927636in}}%
\pgfpathlineto{\pgfqpoint{0.885601in}{0.935307in}}%
\pgfpathlineto{\pgfqpoint{0.871413in}{0.943211in}}%
\pgfpathlineto{\pgfqpoint{0.857755in}{0.951337in}}%
\pgfpathlineto{\pgfqpoint{0.844639in}{0.959677in}}%
\pgfpathlineto{\pgfqpoint{0.847697in}{0.951029in}}%
\pgfpathlineto{\pgfqpoint{0.850740in}{0.942779in}}%
\pgfpathlineto{\pgfqpoint{0.853768in}{0.934920in}}%
\pgfpathlineto{\pgfqpoint{0.856783in}{0.927445in}}%
\pgfpathlineto{\pgfqpoint{0.869598in}{0.919266in}}%
\pgfpathlineto{\pgfqpoint{0.882943in}{0.911296in}}%
\pgfpathlineto{\pgfqpoint{0.896803in}{0.903545in}}%
\pgfpathlineto{\pgfqpoint{0.911166in}{0.896022in}}%
\pgfpathclose%
\pgfusepath{fill}%
\end{pgfscope}%
\begin{pgfscope}%
\pgfpathrectangle{\pgfqpoint{0.329460in}{0.284240in}}{\pgfqpoint{1.989680in}{1.989680in}}%
\pgfusepath{clip}%
\pgfsetbuttcap%
\pgfsetroundjoin%
\definecolor{currentfill}{rgb}{0.267004,0.004874,0.329415}%
\pgfsetfillcolor{currentfill}%
\pgfsetlinewidth{0.000000pt}%
\definecolor{currentstroke}{rgb}{0.000000,0.000000,0.000000}%
\pgfsetstrokecolor{currentstroke}%
\pgfsetdash{}{0pt}%
\pgfpathmoveto{\pgfqpoint{1.358711in}{0.754351in}}%
\pgfpathlineto{\pgfqpoint{1.358762in}{0.751770in}}%
\pgfpathlineto{\pgfqpoint{1.358814in}{0.749407in}}%
\pgfpathlineto{\pgfqpoint{1.358866in}{0.747264in}}%
\pgfpathlineto{\pgfqpoint{1.358917in}{0.745347in}}%
\pgfpathlineto{\pgfqpoint{1.341526in}{0.745364in}}%
\pgfpathlineto{\pgfqpoint{1.324145in}{0.745680in}}%
\pgfpathlineto{\pgfqpoint{1.306795in}{0.746294in}}%
\pgfpathlineto{\pgfqpoint{1.289494in}{0.747207in}}%
\pgfpathlineto{\pgfqpoint{1.289907in}{0.749111in}}%
\pgfpathlineto{\pgfqpoint{1.290319in}{0.751241in}}%
\pgfpathlineto{\pgfqpoint{1.290730in}{0.753592in}}%
\pgfpathlineto{\pgfqpoint{1.291140in}{0.756159in}}%
\pgfpathlineto{\pgfqpoint{1.307979in}{0.755272in}}%
\pgfpathlineto{\pgfqpoint{1.324866in}{0.754674in}}%
\pgfpathlineto{\pgfqpoint{1.341783in}{0.754367in}}%
\pgfpathlineto{\pgfqpoint{1.358711in}{0.754351in}}%
\pgfpathclose%
\pgfusepath{fill}%
\end{pgfscope}%
\begin{pgfscope}%
\pgfpathrectangle{\pgfqpoint{0.329460in}{0.284240in}}{\pgfqpoint{1.989680in}{1.989680in}}%
\pgfusepath{clip}%
\pgfsetbuttcap%
\pgfsetroundjoin%
\definecolor{currentfill}{rgb}{0.281477,0.755203,0.432552}%
\pgfsetfillcolor{currentfill}%
\pgfsetlinewidth{0.000000pt}%
\definecolor{currentstroke}{rgb}{0.000000,0.000000,0.000000}%
\pgfsetstrokecolor{currentstroke}%
\pgfsetdash{}{0pt}%
\pgfpathmoveto{\pgfqpoint{1.432238in}{1.439264in}}%
\pgfpathlineto{\pgfqpoint{1.433584in}{1.431573in}}%
\pgfpathlineto{\pgfqpoint{1.434929in}{1.423819in}}%
\pgfpathlineto{\pgfqpoint{1.436274in}{1.416004in}}%
\pgfpathlineto{\pgfqpoint{1.437618in}{1.408130in}}%
\pgfpathlineto{\pgfqpoint{1.431009in}{1.406839in}}%
\pgfpathlineto{\pgfqpoint{1.424316in}{1.405651in}}%
\pgfpathlineto{\pgfqpoint{1.417546in}{1.404568in}}%
\pgfpathlineto{\pgfqpoint{1.410707in}{1.403590in}}%
\pgfpathlineto{\pgfqpoint{1.409781in}{1.411540in}}%
\pgfpathlineto{\pgfqpoint{1.408854in}{1.419431in}}%
\pgfpathlineto{\pgfqpoint{1.407927in}{1.427261in}}%
\pgfpathlineto{\pgfqpoint{1.407000in}{1.435028in}}%
\pgfpathlineto{\pgfqpoint{1.413413in}{1.435940in}}%
\pgfpathlineto{\pgfqpoint{1.419762in}{1.436951in}}%
\pgfpathlineto{\pgfqpoint{1.426039in}{1.438059in}}%
\pgfpathlineto{\pgfqpoint{1.432238in}{1.439264in}}%
\pgfpathclose%
\pgfusepath{fill}%
\end{pgfscope}%
\begin{pgfscope}%
\pgfpathrectangle{\pgfqpoint{0.329460in}{0.284240in}}{\pgfqpoint{1.989680in}{1.989680in}}%
\pgfusepath{clip}%
\pgfsetbuttcap%
\pgfsetroundjoin%
\definecolor{currentfill}{rgb}{0.282327,0.094955,0.417331}%
\pgfsetfillcolor{currentfill}%
\pgfsetlinewidth{0.000000pt}%
\definecolor{currentstroke}{rgb}{0.000000,0.000000,0.000000}%
\pgfsetstrokecolor{currentstroke}%
\pgfsetdash{}{0pt}%
\pgfpathmoveto{\pgfqpoint{1.715346in}{0.815564in}}%
\pgfpathlineto{\pgfqpoint{1.717638in}{0.818614in}}%
\pgfpathlineto{\pgfqpoint{1.719939in}{0.821979in}}%
\pgfpathlineto{\pgfqpoint{1.722249in}{0.825664in}}%
\pgfpathlineto{\pgfqpoint{1.724567in}{0.829674in}}%
\pgfpathlineto{\pgfqpoint{1.709063in}{0.823456in}}%
\pgfpathlineto{\pgfqpoint{1.693159in}{0.817499in}}%
\pgfpathlineto{\pgfqpoint{1.676872in}{0.811811in}}%
\pgfpathlineto{\pgfqpoint{1.660219in}{0.806400in}}%
\pgfpathlineto{\pgfqpoint{1.658294in}{0.802521in}}%
\pgfpathlineto{\pgfqpoint{1.656376in}{0.798968in}}%
\pgfpathlineto{\pgfqpoint{1.654465in}{0.795736in}}%
\pgfpathlineto{\pgfqpoint{1.652561in}{0.792819in}}%
\pgfpathlineto{\pgfqpoint{1.668808in}{0.798107in}}%
\pgfpathlineto{\pgfqpoint{1.684698in}{0.803665in}}%
\pgfpathlineto{\pgfqpoint{1.700217in}{0.809486in}}%
\pgfpathlineto{\pgfqpoint{1.715346in}{0.815564in}}%
\pgfpathclose%
\pgfusepath{fill}%
\end{pgfscope}%
\begin{pgfscope}%
\pgfpathrectangle{\pgfqpoint{0.329460in}{0.284240in}}{\pgfqpoint{1.989680in}{1.989680in}}%
\pgfusepath{clip}%
\pgfsetbuttcap%
\pgfsetroundjoin%
\definecolor{currentfill}{rgb}{0.636902,0.856542,0.216620}%
\pgfsetfillcolor{currentfill}%
\pgfsetlinewidth{0.000000pt}%
\definecolor{currentstroke}{rgb}{0.000000,0.000000,0.000000}%
\pgfsetstrokecolor{currentstroke}%
\pgfsetdash{}{0pt}%
\pgfpathmoveto{\pgfqpoint{1.300807in}{1.576495in}}%
\pgfpathlineto{\pgfqpoint{1.299547in}{1.570400in}}%
\pgfpathlineto{\pgfqpoint{1.298286in}{1.564209in}}%
\pgfpathlineto{\pgfqpoint{1.297025in}{1.557925in}}%
\pgfpathlineto{\pgfqpoint{1.295765in}{1.551549in}}%
\pgfpathlineto{\pgfqpoint{1.291221in}{1.552423in}}%
\pgfpathlineto{\pgfqpoint{1.286739in}{1.553366in}}%
\pgfpathlineto{\pgfqpoint{1.282323in}{1.554377in}}%
\pgfpathlineto{\pgfqpoint{1.277978in}{1.555453in}}%
\pgfpathlineto{\pgfqpoint{1.279641in}{1.561737in}}%
\pgfpathlineto{\pgfqpoint{1.281305in}{1.567929in}}%
\pgfpathlineto{\pgfqpoint{1.282970in}{1.574027in}}%
\pgfpathlineto{\pgfqpoint{1.284634in}{1.580030in}}%
\pgfpathlineto{\pgfqpoint{1.288585in}{1.579056in}}%
\pgfpathlineto{\pgfqpoint{1.292600in}{1.578141in}}%
\pgfpathlineto{\pgfqpoint{1.296676in}{1.577287in}}%
\pgfpathlineto{\pgfqpoint{1.300807in}{1.576495in}}%
\pgfpathclose%
\pgfusepath{fill}%
\end{pgfscope}%
\begin{pgfscope}%
\pgfpathrectangle{\pgfqpoint{0.329460in}{0.284240in}}{\pgfqpoint{1.989680in}{1.989680in}}%
\pgfusepath{clip}%
\pgfsetbuttcap%
\pgfsetroundjoin%
\definecolor{currentfill}{rgb}{0.565498,0.842430,0.262877}%
\pgfsetfillcolor{currentfill}%
\pgfsetlinewidth{0.000000pt}%
\definecolor{currentstroke}{rgb}{0.000000,0.000000,0.000000}%
\pgfsetstrokecolor{currentstroke}%
\pgfsetdash{}{0pt}%
\pgfpathmoveto{\pgfqpoint{1.295765in}{1.551549in}}%
\pgfpathlineto{\pgfqpoint{1.294505in}{1.545081in}}%
\pgfpathlineto{\pgfqpoint{1.293246in}{1.538524in}}%
\pgfpathlineto{\pgfqpoint{1.291986in}{1.531879in}}%
\pgfpathlineto{\pgfqpoint{1.290727in}{1.525147in}}%
\pgfpathlineto{\pgfqpoint{1.285770in}{1.526106in}}%
\pgfpathlineto{\pgfqpoint{1.280882in}{1.527139in}}%
\pgfpathlineto{\pgfqpoint{1.276065in}{1.528246in}}%
\pgfpathlineto{\pgfqpoint{1.271327in}{1.529426in}}%
\pgfpathlineto{\pgfqpoint{1.272989in}{1.536064in}}%
\pgfpathlineto{\pgfqpoint{1.274651in}{1.542615in}}%
\pgfpathlineto{\pgfqpoint{1.276314in}{1.549079in}}%
\pgfpathlineto{\pgfqpoint{1.277978in}{1.555453in}}%
\pgfpathlineto{\pgfqpoint{1.282323in}{1.554377in}}%
\pgfpathlineto{\pgfqpoint{1.286739in}{1.553366in}}%
\pgfpathlineto{\pgfqpoint{1.291221in}{1.552423in}}%
\pgfpathlineto{\pgfqpoint{1.295765in}{1.551549in}}%
\pgfpathclose%
\pgfusepath{fill}%
\end{pgfscope}%
\begin{pgfscope}%
\pgfpathrectangle{\pgfqpoint{0.329460in}{0.284240in}}{\pgfqpoint{1.989680in}{1.989680in}}%
\pgfusepath{clip}%
\pgfsetbuttcap%
\pgfsetroundjoin%
\definecolor{currentfill}{rgb}{0.166383,0.690856,0.496502}%
\pgfsetfillcolor{currentfill}%
\pgfsetlinewidth{0.000000pt}%
\definecolor{currentstroke}{rgb}{0.000000,0.000000,0.000000}%
\pgfsetstrokecolor{currentstroke}%
\pgfsetdash{}{0pt}%
\pgfpathmoveto{\pgfqpoint{1.414410in}{1.371235in}}%
\pgfpathlineto{\pgfqpoint{1.415335in}{1.363019in}}%
\pgfpathlineto{\pgfqpoint{1.416259in}{1.354755in}}%
\pgfpathlineto{\pgfqpoint{1.417184in}{1.346446in}}%
\pgfpathlineto{\pgfqpoint{1.418108in}{1.338094in}}%
\pgfpathlineto{\pgfqpoint{1.410349in}{1.337104in}}%
\pgfpathlineto{\pgfqpoint{1.402528in}{1.336236in}}%
\pgfpathlineto{\pgfqpoint{1.394652in}{1.335491in}}%
\pgfpathlineto{\pgfqpoint{1.386731in}{1.334870in}}%
\pgfpathlineto{\pgfqpoint{1.386240in}{1.343271in}}%
\pgfpathlineto{\pgfqpoint{1.385749in}{1.351628in}}%
\pgfpathlineto{\pgfqpoint{1.385258in}{1.359941in}}%
\pgfpathlineto{\pgfqpoint{1.384766in}{1.368206in}}%
\pgfpathlineto{\pgfqpoint{1.392250in}{1.368789in}}%
\pgfpathlineto{\pgfqpoint{1.399690in}{1.369489in}}%
\pgfpathlineto{\pgfqpoint{1.407079in}{1.370305in}}%
\pgfpathlineto{\pgfqpoint{1.414410in}{1.371235in}}%
\pgfpathclose%
\pgfusepath{fill}%
\end{pgfscope}%
\begin{pgfscope}%
\pgfpathrectangle{\pgfqpoint{0.329460in}{0.284240in}}{\pgfqpoint{1.989680in}{1.989680in}}%
\pgfusepath{clip}%
\pgfsetbuttcap%
\pgfsetroundjoin%
\definecolor{currentfill}{rgb}{0.814576,0.883393,0.110347}%
\pgfsetfillcolor{currentfill}%
\pgfsetlinewidth{0.000000pt}%
\definecolor{currentstroke}{rgb}{0.000000,0.000000,0.000000}%
\pgfsetstrokecolor{currentstroke}%
\pgfsetdash{}{0pt}%
\pgfpathmoveto{\pgfqpoint{1.377216in}{1.640372in}}%
\pgfpathlineto{\pgfqpoint{1.378148in}{1.635455in}}%
\pgfpathlineto{\pgfqpoint{1.379080in}{1.630430in}}%
\pgfpathlineto{\pgfqpoint{1.380013in}{1.625297in}}%
\pgfpathlineto{\pgfqpoint{1.380945in}{1.620058in}}%
\pgfpathlineto{\pgfqpoint{1.377493in}{1.619638in}}%
\pgfpathlineto{\pgfqpoint{1.374013in}{1.619270in}}%
\pgfpathlineto{\pgfqpoint{1.370511in}{1.618954in}}%
\pgfpathlineto{\pgfqpoint{1.366988in}{1.618691in}}%
\pgfpathlineto{\pgfqpoint{1.366493in}{1.623974in}}%
\pgfpathlineto{\pgfqpoint{1.365998in}{1.629150in}}%
\pgfpathlineto{\pgfqpoint{1.365503in}{1.634219in}}%
\pgfpathlineto{\pgfqpoint{1.365008in}{1.639180in}}%
\pgfpathlineto{\pgfqpoint{1.368089in}{1.639409in}}%
\pgfpathlineto{\pgfqpoint{1.371153in}{1.639684in}}%
\pgfpathlineto{\pgfqpoint{1.374196in}{1.640005in}}%
\pgfpathlineto{\pgfqpoint{1.377216in}{1.640372in}}%
\pgfpathclose%
\pgfusepath{fill}%
\end{pgfscope}%
\begin{pgfscope}%
\pgfpathrectangle{\pgfqpoint{0.329460in}{0.284240in}}{\pgfqpoint{1.989680in}{1.989680in}}%
\pgfusepath{clip}%
\pgfsetbuttcap%
\pgfsetroundjoin%
\definecolor{currentfill}{rgb}{0.268510,0.009605,0.335427}%
\pgfsetfillcolor{currentfill}%
\pgfsetlinewidth{0.000000pt}%
\definecolor{currentstroke}{rgb}{0.000000,0.000000,0.000000}%
\pgfsetstrokecolor{currentstroke}%
\pgfsetdash{}{0pt}%
\pgfpathmoveto{\pgfqpoint{1.567036in}{0.763435in}}%
\pgfpathlineto{\pgfqpoint{1.568456in}{0.762873in}}%
\pgfpathlineto{\pgfqpoint{1.569881in}{0.762563in}}%
\pgfpathlineto{\pgfqpoint{1.571309in}{0.762512in}}%
\pgfpathlineto{\pgfqpoint{1.572742in}{0.762724in}}%
\pgfpathlineto{\pgfqpoint{1.555868in}{0.759064in}}%
\pgfpathlineto{\pgfqpoint{1.538763in}{0.755693in}}%
\pgfpathlineto{\pgfqpoint{1.521446in}{0.752618in}}%
\pgfpathlineto{\pgfqpoint{1.503937in}{0.749840in}}%
\pgfpathlineto{\pgfqpoint{1.502947in}{0.749711in}}%
\pgfpathlineto{\pgfqpoint{1.501960in}{0.749846in}}%
\pgfpathlineto{\pgfqpoint{1.500976in}{0.750239in}}%
\pgfpathlineto{\pgfqpoint{1.499995in}{0.750885in}}%
\pgfpathlineto{\pgfqpoint{1.517054in}{0.753591in}}%
\pgfpathlineto{\pgfqpoint{1.533927in}{0.756587in}}%
\pgfpathlineto{\pgfqpoint{1.550593in}{0.759870in}}%
\pgfpathlineto{\pgfqpoint{1.567036in}{0.763435in}}%
\pgfpathclose%
\pgfusepath{fill}%
\end{pgfscope}%
\begin{pgfscope}%
\pgfpathrectangle{\pgfqpoint{0.329460in}{0.284240in}}{\pgfqpoint{1.989680in}{1.989680in}}%
\pgfusepath{clip}%
\pgfsetbuttcap%
\pgfsetroundjoin%
\definecolor{currentfill}{rgb}{0.762373,0.876424,0.137064}%
\pgfsetfillcolor{currentfill}%
\pgfsetlinewidth{0.000000pt}%
\definecolor{currentstroke}{rgb}{0.000000,0.000000,0.000000}%
\pgfsetstrokecolor{currentstroke}%
\pgfsetdash{}{0pt}%
\pgfpathmoveto{\pgfqpoint{1.394416in}{1.622249in}}%
\pgfpathlineto{\pgfqpoint{1.395769in}{1.616975in}}%
\pgfpathlineto{\pgfqpoint{1.397123in}{1.611596in}}%
\pgfpathlineto{\pgfqpoint{1.398477in}{1.606115in}}%
\pgfpathlineto{\pgfqpoint{1.399830in}{1.600531in}}%
\pgfpathlineto{\pgfqpoint{1.396106in}{1.599827in}}%
\pgfpathlineto{\pgfqpoint{1.392337in}{1.599180in}}%
\pgfpathlineto{\pgfqpoint{1.388524in}{1.598589in}}%
\pgfpathlineto{\pgfqpoint{1.384674in}{1.598057in}}%
\pgfpathlineto{\pgfqpoint{1.383742in}{1.603712in}}%
\pgfpathlineto{\pgfqpoint{1.382810in}{1.609264in}}%
\pgfpathlineto{\pgfqpoint{1.381877in}{1.614714in}}%
\pgfpathlineto{\pgfqpoint{1.380945in}{1.620058in}}%
\pgfpathlineto{\pgfqpoint{1.384367in}{1.620530in}}%
\pgfpathlineto{\pgfqpoint{1.387755in}{1.621053in}}%
\pgfpathlineto{\pgfqpoint{1.391106in}{1.621626in}}%
\pgfpathlineto{\pgfqpoint{1.394416in}{1.622249in}}%
\pgfpathclose%
\pgfusepath{fill}%
\end{pgfscope}%
\begin{pgfscope}%
\pgfpathrectangle{\pgfqpoint{0.329460in}{0.284240in}}{\pgfqpoint{1.989680in}{1.989680in}}%
\pgfusepath{clip}%
\pgfsetbuttcap%
\pgfsetroundjoin%
\definecolor{currentfill}{rgb}{0.134692,0.658636,0.517649}%
\pgfsetfillcolor{currentfill}%
\pgfsetlinewidth{0.000000pt}%
\definecolor{currentstroke}{rgb}{0.000000,0.000000,0.000000}%
\pgfsetstrokecolor{currentstroke}%
\pgfsetdash{}{0pt}%
\pgfpathmoveto{\pgfqpoint{1.386731in}{1.334870in}}%
\pgfpathlineto{\pgfqpoint{1.387222in}{1.326429in}}%
\pgfpathlineto{\pgfqpoint{1.387713in}{1.317949in}}%
\pgfpathlineto{\pgfqpoint{1.388203in}{1.309432in}}%
\pgfpathlineto{\pgfqpoint{1.388694in}{1.300881in}}%
\pgfpathlineto{\pgfqpoint{1.380295in}{1.300355in}}%
\pgfpathlineto{\pgfqpoint{1.371866in}{1.299962in}}%
\pgfpathlineto{\pgfqpoint{1.363415in}{1.299702in}}%
\pgfpathlineto{\pgfqpoint{1.354951in}{1.299575in}}%
\pgfpathlineto{\pgfqpoint{1.354902in}{1.308145in}}%
\pgfpathlineto{\pgfqpoint{1.354853in}{1.316680in}}%
\pgfpathlineto{\pgfqpoint{1.354803in}{1.325179in}}%
\pgfpathlineto{\pgfqpoint{1.354754in}{1.333639in}}%
\pgfpathlineto{\pgfqpoint{1.362775in}{1.333758in}}%
\pgfpathlineto{\pgfqpoint{1.370784in}{1.334003in}}%
\pgfpathlineto{\pgfqpoint{1.378772in}{1.334374in}}%
\pgfpathlineto{\pgfqpoint{1.386731in}{1.334870in}}%
\pgfpathclose%
\pgfusepath{fill}%
\end{pgfscope}%
\begin{pgfscope}%
\pgfpathrectangle{\pgfqpoint{0.329460in}{0.284240in}}{\pgfqpoint{1.989680in}{1.989680in}}%
\pgfusepath{clip}%
\pgfsetbuttcap%
\pgfsetroundjoin%
\definecolor{currentfill}{rgb}{0.814576,0.883393,0.110347}%
\pgfsetfillcolor{currentfill}%
\pgfsetlinewidth{0.000000pt}%
\definecolor{currentstroke}{rgb}{0.000000,0.000000,0.000000}%
\pgfsetstrokecolor{currentstroke}%
\pgfsetdash{}{0pt}%
\pgfpathmoveto{\pgfqpoint{1.340118in}{1.639014in}}%
\pgfpathlineto{\pgfqpoint{1.339721in}{1.634048in}}%
\pgfpathlineto{\pgfqpoint{1.339325in}{1.628973in}}%
\pgfpathlineto{\pgfqpoint{1.338928in}{1.623790in}}%
\pgfpathlineto{\pgfqpoint{1.338532in}{1.618501in}}%
\pgfpathlineto{\pgfqpoint{1.334995in}{1.618717in}}%
\pgfpathlineto{\pgfqpoint{1.331474in}{1.618986in}}%
\pgfpathlineto{\pgfqpoint{1.327974in}{1.619308in}}%
\pgfpathlineto{\pgfqpoint{1.324497in}{1.619682in}}%
\pgfpathlineto{\pgfqpoint{1.325334in}{1.624933in}}%
\pgfpathlineto{\pgfqpoint{1.326170in}{1.630078in}}%
\pgfpathlineto{\pgfqpoint{1.327006in}{1.635115in}}%
\pgfpathlineto{\pgfqpoint{1.327842in}{1.640044in}}%
\pgfpathlineto{\pgfqpoint{1.330883in}{1.639718in}}%
\pgfpathlineto{\pgfqpoint{1.333945in}{1.639437in}}%
\pgfpathlineto{\pgfqpoint{1.337024in}{1.639203in}}%
\pgfpathlineto{\pgfqpoint{1.340118in}{1.639014in}}%
\pgfpathclose%
\pgfusepath{fill}%
\end{pgfscope}%
\begin{pgfscope}%
\pgfpathrectangle{\pgfqpoint{0.329460in}{0.284240in}}{\pgfqpoint{1.989680in}{1.989680in}}%
\pgfusepath{clip}%
\pgfsetbuttcap%
\pgfsetroundjoin%
\definecolor{currentfill}{rgb}{0.281477,0.755203,0.432552}%
\pgfsetfillcolor{currentfill}%
\pgfsetlinewidth{0.000000pt}%
\definecolor{currentstroke}{rgb}{0.000000,0.000000,0.000000}%
\pgfsetstrokecolor{currentstroke}%
\pgfsetdash{}{0pt}%
\pgfpathmoveto{\pgfqpoint{1.301126in}{1.434300in}}%
\pgfpathlineto{\pgfqpoint{1.300294in}{1.426520in}}%
\pgfpathlineto{\pgfqpoint{1.299462in}{1.418677in}}%
\pgfpathlineto{\pgfqpoint{1.298631in}{1.410773in}}%
\pgfpathlineto{\pgfqpoint{1.297800in}{1.402810in}}%
\pgfpathlineto{\pgfqpoint{1.290905in}{1.403693in}}%
\pgfpathlineto{\pgfqpoint{1.284073in}{1.404683in}}%
\pgfpathlineto{\pgfqpoint{1.277312in}{1.405778in}}%
\pgfpathlineto{\pgfqpoint{1.270628in}{1.406978in}}%
\pgfpathlineto{\pgfqpoint{1.271881in}{1.414871in}}%
\pgfpathlineto{\pgfqpoint{1.273135in}{1.422705in}}%
\pgfpathlineto{\pgfqpoint{1.274389in}{1.430478in}}%
\pgfpathlineto{\pgfqpoint{1.275643in}{1.438188in}}%
\pgfpathlineto{\pgfqpoint{1.281912in}{1.437069in}}%
\pgfpathlineto{\pgfqpoint{1.288253in}{1.436048in}}%
\pgfpathlineto{\pgfqpoint{1.294660in}{1.435124in}}%
\pgfpathlineto{\pgfqpoint{1.301126in}{1.434300in}}%
\pgfpathclose%
\pgfusepath{fill}%
\end{pgfscope}%
\begin{pgfscope}%
\pgfpathrectangle{\pgfqpoint{0.329460in}{0.284240in}}{\pgfqpoint{1.989680in}{1.989680in}}%
\pgfusepath{clip}%
\pgfsetbuttcap%
\pgfsetroundjoin%
\definecolor{currentfill}{rgb}{0.134692,0.658636,0.517649}%
\pgfsetfillcolor{currentfill}%
\pgfsetlinewidth{0.000000pt}%
\definecolor{currentstroke}{rgb}{0.000000,0.000000,0.000000}%
\pgfsetstrokecolor{currentstroke}%
\pgfsetdash{}{0pt}%
\pgfpathmoveto{\pgfqpoint{1.354754in}{1.333639in}}%
\pgfpathlineto{\pgfqpoint{1.354803in}{1.325179in}}%
\pgfpathlineto{\pgfqpoint{1.354853in}{1.316680in}}%
\pgfpathlineto{\pgfqpoint{1.354902in}{1.308145in}}%
\pgfpathlineto{\pgfqpoint{1.354951in}{1.299575in}}%
\pgfpathlineto{\pgfqpoint{1.346483in}{1.299583in}}%
\pgfpathlineto{\pgfqpoint{1.338020in}{1.299724in}}%
\pgfpathlineto{\pgfqpoint{1.329571in}{1.299999in}}%
\pgfpathlineto{\pgfqpoint{1.321145in}{1.300407in}}%
\pgfpathlineto{\pgfqpoint{1.321538in}{1.308965in}}%
\pgfpathlineto{\pgfqpoint{1.321931in}{1.317488in}}%
\pgfpathlineto{\pgfqpoint{1.322324in}{1.325975in}}%
\pgfpathlineto{\pgfqpoint{1.322717in}{1.334423in}}%
\pgfpathlineto{\pgfqpoint{1.330702in}{1.334038in}}%
\pgfpathlineto{\pgfqpoint{1.338709in}{1.333779in}}%
\pgfpathlineto{\pgfqpoint{1.346729in}{1.333646in}}%
\pgfpathlineto{\pgfqpoint{1.354754in}{1.333639in}}%
\pgfpathclose%
\pgfusepath{fill}%
\end{pgfscope}%
\begin{pgfscope}%
\pgfpathrectangle{\pgfqpoint{0.329460in}{0.284240in}}{\pgfqpoint{1.989680in}{1.989680in}}%
\pgfusepath{clip}%
\pgfsetbuttcap%
\pgfsetroundjoin%
\definecolor{currentfill}{rgb}{0.166383,0.690856,0.496502}%
\pgfsetfillcolor{currentfill}%
\pgfsetlinewidth{0.000000pt}%
\definecolor{currentstroke}{rgb}{0.000000,0.000000,0.000000}%
\pgfsetstrokecolor{currentstroke}%
\pgfsetdash{}{0pt}%
\pgfpathmoveto{\pgfqpoint{1.324291in}{1.367786in}}%
\pgfpathlineto{\pgfqpoint{1.323897in}{1.359514in}}%
\pgfpathlineto{\pgfqpoint{1.323504in}{1.351195in}}%
\pgfpathlineto{\pgfqpoint{1.323110in}{1.342830in}}%
\pgfpathlineto{\pgfqpoint{1.322717in}{1.334423in}}%
\pgfpathlineto{\pgfqpoint{1.314762in}{1.334933in}}%
\pgfpathlineto{\pgfqpoint{1.306845in}{1.335568in}}%
\pgfpathlineto{\pgfqpoint{1.298975in}{1.336326in}}%
\pgfpathlineto{\pgfqpoint{1.291161in}{1.337208in}}%
\pgfpathlineto{\pgfqpoint{1.291990in}{1.345573in}}%
\pgfpathlineto{\pgfqpoint{1.292819in}{1.353895in}}%
\pgfpathlineto{\pgfqpoint{1.293648in}{1.362172in}}%
\pgfpathlineto{\pgfqpoint{1.294478in}{1.370402in}}%
\pgfpathlineto{\pgfqpoint{1.301861in}{1.369574in}}%
\pgfpathlineto{\pgfqpoint{1.309296in}{1.368861in}}%
\pgfpathlineto{\pgfqpoint{1.316776in}{1.368265in}}%
\pgfpathlineto{\pgfqpoint{1.324291in}{1.367786in}}%
\pgfpathclose%
\pgfusepath{fill}%
\end{pgfscope}%
\begin{pgfscope}%
\pgfpathrectangle{\pgfqpoint{0.329460in}{0.284240in}}{\pgfqpoint{1.989680in}{1.989680in}}%
\pgfusepath{clip}%
\pgfsetbuttcap%
\pgfsetroundjoin%
\definecolor{currentfill}{rgb}{0.762373,0.876424,0.137064}%
\pgfsetfillcolor{currentfill}%
\pgfsetlinewidth{0.000000pt}%
\definecolor{currentstroke}{rgb}{0.000000,0.000000,0.000000}%
\pgfsetstrokecolor{currentstroke}%
\pgfsetdash{}{0pt}%
\pgfpathmoveto{\pgfqpoint{1.324497in}{1.619682in}}%
\pgfpathlineto{\pgfqpoint{1.323661in}{1.614325in}}%
\pgfpathlineto{\pgfqpoint{1.322825in}{1.608864in}}%
\pgfpathlineto{\pgfqpoint{1.321989in}{1.603299in}}%
\pgfpathlineto{\pgfqpoint{1.321153in}{1.597632in}}%
\pgfpathlineto{\pgfqpoint{1.317271in}{1.598113in}}%
\pgfpathlineto{\pgfqpoint{1.313425in}{1.598652in}}%
\pgfpathlineto{\pgfqpoint{1.309618in}{1.599249in}}%
\pgfpathlineto{\pgfqpoint{1.305853in}{1.599902in}}%
\pgfpathlineto{\pgfqpoint{1.307114in}{1.605504in}}%
\pgfpathlineto{\pgfqpoint{1.308376in}{1.611004in}}%
\pgfpathlineto{\pgfqpoint{1.309638in}{1.616400in}}%
\pgfpathlineto{\pgfqpoint{1.310900in}{1.621692in}}%
\pgfpathlineto{\pgfqpoint{1.314246in}{1.621114in}}%
\pgfpathlineto{\pgfqpoint{1.317630in}{1.620585in}}%
\pgfpathlineto{\pgfqpoint{1.321048in}{1.620108in}}%
\pgfpathlineto{\pgfqpoint{1.324497in}{1.619682in}}%
\pgfpathclose%
\pgfusepath{fill}%
\end{pgfscope}%
\begin{pgfscope}%
\pgfpathrectangle{\pgfqpoint{0.329460in}{0.284240in}}{\pgfqpoint{1.989680in}{1.989680in}}%
\pgfusepath{clip}%
\pgfsetbuttcap%
\pgfsetroundjoin%
\definecolor{currentfill}{rgb}{0.344074,0.780029,0.397381}%
\pgfsetfillcolor{currentfill}%
\pgfsetlinewidth{0.000000pt}%
\definecolor{currentstroke}{rgb}{0.000000,0.000000,0.000000}%
\pgfsetstrokecolor{currentstroke}%
\pgfsetdash{}{0pt}%
\pgfpathmoveto{\pgfqpoint{1.426851in}{1.469362in}}%
\pgfpathlineto{\pgfqpoint{1.428198in}{1.461941in}}%
\pgfpathlineto{\pgfqpoint{1.429545in}{1.454450in}}%
\pgfpathlineto{\pgfqpoint{1.430892in}{1.446890in}}%
\pgfpathlineto{\pgfqpoint{1.432238in}{1.439264in}}%
\pgfpathlineto{\pgfqpoint{1.426039in}{1.438059in}}%
\pgfpathlineto{\pgfqpoint{1.419762in}{1.436951in}}%
\pgfpathlineto{\pgfqpoint{1.413413in}{1.435940in}}%
\pgfpathlineto{\pgfqpoint{1.407000in}{1.435028in}}%
\pgfpathlineto{\pgfqpoint{1.406072in}{1.442730in}}%
\pgfpathlineto{\pgfqpoint{1.405144in}{1.450365in}}%
\pgfpathlineto{\pgfqpoint{1.404216in}{1.457931in}}%
\pgfpathlineto{\pgfqpoint{1.403288in}{1.465427in}}%
\pgfpathlineto{\pgfqpoint{1.409275in}{1.466275in}}%
\pgfpathlineto{\pgfqpoint{1.415203in}{1.467213in}}%
\pgfpathlineto{\pgfqpoint{1.421063in}{1.468243in}}%
\pgfpathlineto{\pgfqpoint{1.426851in}{1.469362in}}%
\pgfpathclose%
\pgfusepath{fill}%
\end{pgfscope}%
\begin{pgfscope}%
\pgfpathrectangle{\pgfqpoint{0.329460in}{0.284240in}}{\pgfqpoint{1.989680in}{1.989680in}}%
\pgfusepath{clip}%
\pgfsetbuttcap%
\pgfsetroundjoin%
\definecolor{currentfill}{rgb}{0.814576,0.883393,0.110347}%
\pgfsetfillcolor{currentfill}%
\pgfsetlinewidth{0.000000pt}%
\definecolor{currentstroke}{rgb}{0.000000,0.000000,0.000000}%
\pgfsetstrokecolor{currentstroke}%
\pgfsetdash{}{0pt}%
\pgfpathmoveto{\pgfqpoint{1.365008in}{1.639180in}}%
\pgfpathlineto{\pgfqpoint{1.365503in}{1.634219in}}%
\pgfpathlineto{\pgfqpoint{1.365998in}{1.629150in}}%
\pgfpathlineto{\pgfqpoint{1.366493in}{1.623974in}}%
\pgfpathlineto{\pgfqpoint{1.366988in}{1.618691in}}%
\pgfpathlineto{\pgfqpoint{1.363450in}{1.618480in}}%
\pgfpathlineto{\pgfqpoint{1.359898in}{1.618323in}}%
\pgfpathlineto{\pgfqpoint{1.356338in}{1.618219in}}%
\pgfpathlineto{\pgfqpoint{1.352773in}{1.618168in}}%
\pgfpathlineto{\pgfqpoint{1.352723in}{1.623469in}}%
\pgfpathlineto{\pgfqpoint{1.352674in}{1.628662in}}%
\pgfpathlineto{\pgfqpoint{1.352624in}{1.633747in}}%
\pgfpathlineto{\pgfqpoint{1.352574in}{1.638725in}}%
\pgfpathlineto{\pgfqpoint{1.355693in}{1.638769in}}%
\pgfpathlineto{\pgfqpoint{1.358806in}{1.638859in}}%
\pgfpathlineto{\pgfqpoint{1.361913in}{1.638996in}}%
\pgfpathlineto{\pgfqpoint{1.365008in}{1.639180in}}%
\pgfpathclose%
\pgfusepath{fill}%
\end{pgfscope}%
\begin{pgfscope}%
\pgfpathrectangle{\pgfqpoint{0.329460in}{0.284240in}}{\pgfqpoint{1.989680in}{1.989680in}}%
\pgfusepath{clip}%
\pgfsetbuttcap%
\pgfsetroundjoin%
\definecolor{currentfill}{rgb}{0.814576,0.883393,0.110347}%
\pgfsetfillcolor{currentfill}%
\pgfsetlinewidth{0.000000pt}%
\definecolor{currentstroke}{rgb}{0.000000,0.000000,0.000000}%
\pgfsetstrokecolor{currentstroke}%
\pgfsetdash{}{0pt}%
\pgfpathmoveto{\pgfqpoint{1.352574in}{1.638725in}}%
\pgfpathlineto{\pgfqpoint{1.352624in}{1.633747in}}%
\pgfpathlineto{\pgfqpoint{1.352674in}{1.628662in}}%
\pgfpathlineto{\pgfqpoint{1.352723in}{1.623469in}}%
\pgfpathlineto{\pgfqpoint{1.352773in}{1.618168in}}%
\pgfpathlineto{\pgfqpoint{1.349206in}{1.618171in}}%
\pgfpathlineto{\pgfqpoint{1.345641in}{1.618228in}}%
\pgfpathlineto{\pgfqpoint{1.342082in}{1.618338in}}%
\pgfpathlineto{\pgfqpoint{1.338532in}{1.618501in}}%
\pgfpathlineto{\pgfqpoint{1.338928in}{1.623790in}}%
\pgfpathlineto{\pgfqpoint{1.339325in}{1.628973in}}%
\pgfpathlineto{\pgfqpoint{1.339721in}{1.634048in}}%
\pgfpathlineto{\pgfqpoint{1.340118in}{1.639014in}}%
\pgfpathlineto{\pgfqpoint{1.343223in}{1.638872in}}%
\pgfpathlineto{\pgfqpoint{1.346336in}{1.638776in}}%
\pgfpathlineto{\pgfqpoint{1.349454in}{1.638727in}}%
\pgfpathlineto{\pgfqpoint{1.352574in}{1.638725in}}%
\pgfpathclose%
\pgfusepath{fill}%
\end{pgfscope}%
\begin{pgfscope}%
\pgfpathrectangle{\pgfqpoint{0.329460in}{0.284240in}}{\pgfqpoint{1.989680in}{1.989680in}}%
\pgfusepath{clip}%
\pgfsetbuttcap%
\pgfsetroundjoin%
\definecolor{currentfill}{rgb}{0.267004,0.004874,0.329415}%
\pgfsetfillcolor{currentfill}%
\pgfsetlinewidth{0.000000pt}%
\definecolor{currentstroke}{rgb}{0.000000,0.000000,0.000000}%
\pgfsetstrokecolor{currentstroke}%
\pgfsetdash{}{0pt}%
\pgfpathmoveto{\pgfqpoint{1.496097in}{0.755912in}}%
\pgfpathlineto{\pgfqpoint{1.497068in}{0.754298in}}%
\pgfpathlineto{\pgfqpoint{1.498041in}{0.752920in}}%
\pgfpathlineto{\pgfqpoint{1.499017in}{0.751781in}}%
\pgfpathlineto{\pgfqpoint{1.499995in}{0.750885in}}%
\pgfpathlineto{\pgfqpoint{1.482769in}{0.748474in}}%
\pgfpathlineto{\pgfqpoint{1.465393in}{0.746359in}}%
\pgfpathlineto{\pgfqpoint{1.447889in}{0.744544in}}%
\pgfpathlineto{\pgfqpoint{1.430275in}{0.743030in}}%
\pgfpathlineto{\pgfqpoint{1.429754in}{0.743978in}}%
\pgfpathlineto{\pgfqpoint{1.429235in}{0.745170in}}%
\pgfpathlineto{\pgfqpoint{1.428717in}{0.746602in}}%
\pgfpathlineto{\pgfqpoint{1.428201in}{0.748268in}}%
\pgfpathlineto{\pgfqpoint{1.445353in}{0.749741in}}%
\pgfpathlineto{\pgfqpoint{1.462399in}{0.751507in}}%
\pgfpathlineto{\pgfqpoint{1.479320in}{0.753565in}}%
\pgfpathlineto{\pgfqpoint{1.496097in}{0.755912in}}%
\pgfpathclose%
\pgfusepath{fill}%
\end{pgfscope}%
\begin{pgfscope}%
\pgfpathrectangle{\pgfqpoint{0.329460in}{0.284240in}}{\pgfqpoint{1.989680in}{1.989680in}}%
\pgfusepath{clip}%
\pgfsetbuttcap%
\pgfsetroundjoin%
\definecolor{currentfill}{rgb}{0.699415,0.867117,0.175971}%
\pgfsetfillcolor{currentfill}%
\pgfsetlinewidth{0.000000pt}%
\definecolor{currentstroke}{rgb}{0.000000,0.000000,0.000000}%
\pgfsetstrokecolor{currentstroke}%
\pgfsetdash{}{0pt}%
\pgfpathmoveto{\pgfqpoint{1.399830in}{1.600531in}}%
\pgfpathlineto{\pgfqpoint{1.401184in}{1.594846in}}%
\pgfpathlineto{\pgfqpoint{1.402537in}{1.589061in}}%
\pgfpathlineto{\pgfqpoint{1.403890in}{1.583177in}}%
\pgfpathlineto{\pgfqpoint{1.405243in}{1.577196in}}%
\pgfpathlineto{\pgfqpoint{1.401105in}{1.576411in}}%
\pgfpathlineto{\pgfqpoint{1.396916in}{1.575689in}}%
\pgfpathlineto{\pgfqpoint{1.392681in}{1.575030in}}%
\pgfpathlineto{\pgfqpoint{1.388402in}{1.574436in}}%
\pgfpathlineto{\pgfqpoint{1.387470in}{1.580489in}}%
\pgfpathlineto{\pgfqpoint{1.386538in}{1.586444in}}%
\pgfpathlineto{\pgfqpoint{1.385606in}{1.592300in}}%
\pgfpathlineto{\pgfqpoint{1.384674in}{1.598057in}}%
\pgfpathlineto{\pgfqpoint{1.388524in}{1.598589in}}%
\pgfpathlineto{\pgfqpoint{1.392337in}{1.599180in}}%
\pgfpathlineto{\pgfqpoint{1.396106in}{1.599827in}}%
\pgfpathlineto{\pgfqpoint{1.399830in}{1.600531in}}%
\pgfpathclose%
\pgfusepath{fill}%
\end{pgfscope}%
\begin{pgfscope}%
\pgfpathrectangle{\pgfqpoint{0.329460in}{0.284240in}}{\pgfqpoint{1.989680in}{1.989680in}}%
\pgfusepath{clip}%
\pgfsetbuttcap%
\pgfsetroundjoin%
\definecolor{currentfill}{rgb}{0.220124,0.725509,0.466226}%
\pgfsetfillcolor{currentfill}%
\pgfsetlinewidth{0.000000pt}%
\definecolor{currentstroke}{rgb}{0.000000,0.000000,0.000000}%
\pgfsetstrokecolor{currentstroke}%
\pgfsetdash{}{0pt}%
\pgfpathmoveto{\pgfqpoint{1.410707in}{1.403590in}}%
\pgfpathlineto{\pgfqpoint{1.411633in}{1.395582in}}%
\pgfpathlineto{\pgfqpoint{1.412559in}{1.387519in}}%
\pgfpathlineto{\pgfqpoint{1.413485in}{1.379403in}}%
\pgfpathlineto{\pgfqpoint{1.414410in}{1.371235in}}%
\pgfpathlineto{\pgfqpoint{1.407079in}{1.370305in}}%
\pgfpathlineto{\pgfqpoint{1.399690in}{1.369489in}}%
\pgfpathlineto{\pgfqpoint{1.392250in}{1.368789in}}%
\pgfpathlineto{\pgfqpoint{1.384766in}{1.368206in}}%
\pgfpathlineto{\pgfqpoint{1.384275in}{1.376422in}}%
\pgfpathlineto{\pgfqpoint{1.383783in}{1.384586in}}%
\pgfpathlineto{\pgfqpoint{1.383291in}{1.392697in}}%
\pgfpathlineto{\pgfqpoint{1.382799in}{1.400753in}}%
\pgfpathlineto{\pgfqpoint{1.389844in}{1.401299in}}%
\pgfpathlineto{\pgfqpoint{1.396849in}{1.401955in}}%
\pgfpathlineto{\pgfqpoint{1.403806in}{1.402718in}}%
\pgfpathlineto{\pgfqpoint{1.410707in}{1.403590in}}%
\pgfpathclose%
\pgfusepath{fill}%
\end{pgfscope}%
\begin{pgfscope}%
\pgfpathrectangle{\pgfqpoint{0.329460in}{0.284240in}}{\pgfqpoint{1.989680in}{1.989680in}}%
\pgfusepath{clip}%
\pgfsetbuttcap%
\pgfsetroundjoin%
\definecolor{currentfill}{rgb}{0.344074,0.780029,0.397381}%
\pgfsetfillcolor{currentfill}%
\pgfsetlinewidth{0.000000pt}%
\definecolor{currentstroke}{rgb}{0.000000,0.000000,0.000000}%
\pgfsetstrokecolor{currentstroke}%
\pgfsetdash{}{0pt}%
\pgfpathmoveto{\pgfqpoint{1.304456in}{1.464751in}}%
\pgfpathlineto{\pgfqpoint{1.303623in}{1.457242in}}%
\pgfpathlineto{\pgfqpoint{1.302790in}{1.449663in}}%
\pgfpathlineto{\pgfqpoint{1.301958in}{1.442015in}}%
\pgfpathlineto{\pgfqpoint{1.301126in}{1.434300in}}%
\pgfpathlineto{\pgfqpoint{1.294660in}{1.435124in}}%
\pgfpathlineto{\pgfqpoint{1.288253in}{1.436048in}}%
\pgfpathlineto{\pgfqpoint{1.281912in}{1.437069in}}%
\pgfpathlineto{\pgfqpoint{1.275643in}{1.438188in}}%
\pgfpathlineto{\pgfqpoint{1.276898in}{1.445834in}}%
\pgfpathlineto{\pgfqpoint{1.278153in}{1.453413in}}%
\pgfpathlineto{\pgfqpoint{1.279409in}{1.460923in}}%
\pgfpathlineto{\pgfqpoint{1.280665in}{1.468363in}}%
\pgfpathlineto{\pgfqpoint{1.286518in}{1.467323in}}%
\pgfpathlineto{\pgfqpoint{1.292438in}{1.466374in}}%
\pgfpathlineto{\pgfqpoint{1.298419in}{1.465517in}}%
\pgfpathlineto{\pgfqpoint{1.304456in}{1.464751in}}%
\pgfpathclose%
\pgfusepath{fill}%
\end{pgfscope}%
\begin{pgfscope}%
\pgfpathrectangle{\pgfqpoint{0.329460in}{0.284240in}}{\pgfqpoint{1.989680in}{1.989680in}}%
\pgfusepath{clip}%
\pgfsetbuttcap%
\pgfsetroundjoin%
\definecolor{currentfill}{rgb}{0.412913,0.803041,0.357269}%
\pgfsetfillcolor{currentfill}%
\pgfsetlinewidth{0.000000pt}%
\definecolor{currentstroke}{rgb}{0.000000,0.000000,0.000000}%
\pgfsetstrokecolor{currentstroke}%
\pgfsetdash{}{0pt}%
\pgfpathmoveto{\pgfqpoint{1.421457in}{1.498308in}}%
\pgfpathlineto{\pgfqpoint{1.422806in}{1.491186in}}%
\pgfpathlineto{\pgfqpoint{1.424155in}{1.483986in}}%
\pgfpathlineto{\pgfqpoint{1.425503in}{1.476711in}}%
\pgfpathlineto{\pgfqpoint{1.426851in}{1.469362in}}%
\pgfpathlineto{\pgfqpoint{1.421063in}{1.468243in}}%
\pgfpathlineto{\pgfqpoint{1.415203in}{1.467213in}}%
\pgfpathlineto{\pgfqpoint{1.409275in}{1.466275in}}%
\pgfpathlineto{\pgfqpoint{1.403288in}{1.465427in}}%
\pgfpathlineto{\pgfqpoint{1.402359in}{1.472851in}}%
\pgfpathlineto{\pgfqpoint{1.401430in}{1.480201in}}%
\pgfpathlineto{\pgfqpoint{1.400501in}{1.487475in}}%
\pgfpathlineto{\pgfqpoint{1.399571in}{1.494672in}}%
\pgfpathlineto{\pgfqpoint{1.405133in}{1.495455in}}%
\pgfpathlineto{\pgfqpoint{1.410638in}{1.496323in}}%
\pgfpathlineto{\pgfqpoint{1.416081in}{1.497274in}}%
\pgfpathlineto{\pgfqpoint{1.421457in}{1.498308in}}%
\pgfpathclose%
\pgfusepath{fill}%
\end{pgfscope}%
\begin{pgfscope}%
\pgfpathrectangle{\pgfqpoint{0.329460in}{0.284240in}}{\pgfqpoint{1.989680in}{1.989680in}}%
\pgfusepath{clip}%
\pgfsetbuttcap%
\pgfsetroundjoin%
\definecolor{currentfill}{rgb}{0.699415,0.867117,0.175971}%
\pgfsetfillcolor{currentfill}%
\pgfsetlinewidth{0.000000pt}%
\definecolor{currentstroke}{rgb}{0.000000,0.000000,0.000000}%
\pgfsetstrokecolor{currentstroke}%
\pgfsetdash{}{0pt}%
\pgfpathmoveto{\pgfqpoint{1.321153in}{1.597632in}}%
\pgfpathlineto{\pgfqpoint{1.320317in}{1.591863in}}%
\pgfpathlineto{\pgfqpoint{1.319481in}{1.585995in}}%
\pgfpathlineto{\pgfqpoint{1.318645in}{1.580027in}}%
\pgfpathlineto{\pgfqpoint{1.317809in}{1.573962in}}%
\pgfpathlineto{\pgfqpoint{1.313496in}{1.574499in}}%
\pgfpathlineto{\pgfqpoint{1.309222in}{1.575100in}}%
\pgfpathlineto{\pgfqpoint{1.304991in}{1.575766in}}%
\pgfpathlineto{\pgfqpoint{1.300807in}{1.576495in}}%
\pgfpathlineto{\pgfqpoint{1.302069in}{1.582495in}}%
\pgfpathlineto{\pgfqpoint{1.303330in}{1.588396in}}%
\pgfpathlineto{\pgfqpoint{1.304591in}{1.594199in}}%
\pgfpathlineto{\pgfqpoint{1.305853in}{1.599902in}}%
\pgfpathlineto{\pgfqpoint{1.309618in}{1.599249in}}%
\pgfpathlineto{\pgfqpoint{1.313425in}{1.598652in}}%
\pgfpathlineto{\pgfqpoint{1.317271in}{1.598113in}}%
\pgfpathlineto{\pgfqpoint{1.321153in}{1.597632in}}%
\pgfpathclose%
\pgfusepath{fill}%
\end{pgfscope}%
\begin{pgfscope}%
\pgfpathrectangle{\pgfqpoint{0.329460in}{0.284240in}}{\pgfqpoint{1.989680in}{1.989680in}}%
\pgfusepath{clip}%
\pgfsetbuttcap%
\pgfsetroundjoin%
\definecolor{currentfill}{rgb}{0.166383,0.690856,0.496502}%
\pgfsetfillcolor{currentfill}%
\pgfsetlinewidth{0.000000pt}%
\definecolor{currentstroke}{rgb}{0.000000,0.000000,0.000000}%
\pgfsetstrokecolor{currentstroke}%
\pgfsetdash{}{0pt}%
\pgfpathmoveto{\pgfqpoint{1.384766in}{1.368206in}}%
\pgfpathlineto{\pgfqpoint{1.385258in}{1.359941in}}%
\pgfpathlineto{\pgfqpoint{1.385749in}{1.351628in}}%
\pgfpathlineto{\pgfqpoint{1.386240in}{1.343271in}}%
\pgfpathlineto{\pgfqpoint{1.386731in}{1.334870in}}%
\pgfpathlineto{\pgfqpoint{1.378772in}{1.334374in}}%
\pgfpathlineto{\pgfqpoint{1.370784in}{1.334003in}}%
\pgfpathlineto{\pgfqpoint{1.362775in}{1.333758in}}%
\pgfpathlineto{\pgfqpoint{1.354754in}{1.333639in}}%
\pgfpathlineto{\pgfqpoint{1.354705in}{1.342058in}}%
\pgfpathlineto{\pgfqpoint{1.354656in}{1.350435in}}%
\pgfpathlineto{\pgfqpoint{1.354606in}{1.358766in}}%
\pgfpathlineto{\pgfqpoint{1.354557in}{1.367049in}}%
\pgfpathlineto{\pgfqpoint{1.362135in}{1.367161in}}%
\pgfpathlineto{\pgfqpoint{1.369701in}{1.367392in}}%
\pgfpathlineto{\pgfqpoint{1.377247in}{1.367740in}}%
\pgfpathlineto{\pgfqpoint{1.384766in}{1.368206in}}%
\pgfpathclose%
\pgfusepath{fill}%
\end{pgfscope}%
\begin{pgfscope}%
\pgfpathrectangle{\pgfqpoint{0.329460in}{0.284240in}}{\pgfqpoint{1.989680in}{1.989680in}}%
\pgfusepath{clip}%
\pgfsetbuttcap%
\pgfsetroundjoin%
\definecolor{currentfill}{rgb}{0.762373,0.876424,0.137064}%
\pgfsetfillcolor{currentfill}%
\pgfsetlinewidth{0.000000pt}%
\definecolor{currentstroke}{rgb}{0.000000,0.000000,0.000000}%
\pgfsetstrokecolor{currentstroke}%
\pgfsetdash{}{0pt}%
\pgfpathmoveto{\pgfqpoint{1.380945in}{1.620058in}}%
\pgfpathlineto{\pgfqpoint{1.381877in}{1.614714in}}%
\pgfpathlineto{\pgfqpoint{1.382810in}{1.609264in}}%
\pgfpathlineto{\pgfqpoint{1.383742in}{1.603712in}}%
\pgfpathlineto{\pgfqpoint{1.384674in}{1.598057in}}%
\pgfpathlineto{\pgfqpoint{1.380789in}{1.597582in}}%
\pgfpathlineto{\pgfqpoint{1.376874in}{1.597166in}}%
\pgfpathlineto{\pgfqpoint{1.372933in}{1.596809in}}%
\pgfpathlineto{\pgfqpoint{1.368969in}{1.596512in}}%
\pgfpathlineto{\pgfqpoint{1.368474in}{1.602211in}}%
\pgfpathlineto{\pgfqpoint{1.367979in}{1.607808in}}%
\pgfpathlineto{\pgfqpoint{1.367483in}{1.613302in}}%
\pgfpathlineto{\pgfqpoint{1.366988in}{1.618691in}}%
\pgfpathlineto{\pgfqpoint{1.370511in}{1.618954in}}%
\pgfpathlineto{\pgfqpoint{1.374013in}{1.619270in}}%
\pgfpathlineto{\pgfqpoint{1.377493in}{1.619638in}}%
\pgfpathlineto{\pgfqpoint{1.380945in}{1.620058in}}%
\pgfpathclose%
\pgfusepath{fill}%
\end{pgfscope}%
\begin{pgfscope}%
\pgfpathrectangle{\pgfqpoint{0.329460in}{0.284240in}}{\pgfqpoint{1.989680in}{1.989680in}}%
\pgfusepath{clip}%
\pgfsetbuttcap%
\pgfsetroundjoin%
\definecolor{currentfill}{rgb}{0.636902,0.856542,0.216620}%
\pgfsetfillcolor{currentfill}%
\pgfsetlinewidth{0.000000pt}%
\definecolor{currentstroke}{rgb}{0.000000,0.000000,0.000000}%
\pgfsetstrokecolor{currentstroke}%
\pgfsetdash{}{0pt}%
\pgfpathmoveto{\pgfqpoint{1.405243in}{1.577196in}}%
\pgfpathlineto{\pgfqpoint{1.406596in}{1.571119in}}%
\pgfpathlineto{\pgfqpoint{1.407948in}{1.564947in}}%
\pgfpathlineto{\pgfqpoint{1.409300in}{1.558681in}}%
\pgfpathlineto{\pgfqpoint{1.410652in}{1.552323in}}%
\pgfpathlineto{\pgfqpoint{1.406101in}{1.551456in}}%
\pgfpathlineto{\pgfqpoint{1.401494in}{1.550658in}}%
\pgfpathlineto{\pgfqpoint{1.396834in}{1.549930in}}%
\pgfpathlineto{\pgfqpoint{1.392128in}{1.549273in}}%
\pgfpathlineto{\pgfqpoint{1.391197in}{1.555704in}}%
\pgfpathlineto{\pgfqpoint{1.390265in}{1.562042in}}%
\pgfpathlineto{\pgfqpoint{1.389334in}{1.568287in}}%
\pgfpathlineto{\pgfqpoint{1.388402in}{1.574436in}}%
\pgfpathlineto{\pgfqpoint{1.392681in}{1.575030in}}%
\pgfpathlineto{\pgfqpoint{1.396916in}{1.575689in}}%
\pgfpathlineto{\pgfqpoint{1.401105in}{1.576411in}}%
\pgfpathlineto{\pgfqpoint{1.405243in}{1.577196in}}%
\pgfpathclose%
\pgfusepath{fill}%
\end{pgfscope}%
\begin{pgfscope}%
\pgfpathrectangle{\pgfqpoint{0.329460in}{0.284240in}}{\pgfqpoint{1.989680in}{1.989680in}}%
\pgfusepath{clip}%
\pgfsetbuttcap%
\pgfsetroundjoin%
\definecolor{currentfill}{rgb}{0.220124,0.725509,0.466226}%
\pgfsetfillcolor{currentfill}%
\pgfsetlinewidth{0.000000pt}%
\definecolor{currentstroke}{rgb}{0.000000,0.000000,0.000000}%
\pgfsetstrokecolor{currentstroke}%
\pgfsetdash{}{0pt}%
\pgfpathmoveto{\pgfqpoint{1.325867in}{1.400360in}}%
\pgfpathlineto{\pgfqpoint{1.325473in}{1.392297in}}%
\pgfpathlineto{\pgfqpoint{1.325079in}{1.384180in}}%
\pgfpathlineto{\pgfqpoint{1.324685in}{1.376008in}}%
\pgfpathlineto{\pgfqpoint{1.324291in}{1.367786in}}%
\pgfpathlineto{\pgfqpoint{1.316776in}{1.368265in}}%
\pgfpathlineto{\pgfqpoint{1.309296in}{1.368861in}}%
\pgfpathlineto{\pgfqpoint{1.301861in}{1.369574in}}%
\pgfpathlineto{\pgfqpoint{1.294478in}{1.370402in}}%
\pgfpathlineto{\pgfqpoint{1.295308in}{1.378583in}}%
\pgfpathlineto{\pgfqpoint{1.296138in}{1.386713in}}%
\pgfpathlineto{\pgfqpoint{1.296969in}{1.394789in}}%
\pgfpathlineto{\pgfqpoint{1.297800in}{1.402810in}}%
\pgfpathlineto{\pgfqpoint{1.304751in}{1.402034in}}%
\pgfpathlineto{\pgfqpoint{1.311751in}{1.401367in}}%
\pgfpathlineto{\pgfqpoint{1.318792in}{1.400808in}}%
\pgfpathlineto{\pgfqpoint{1.325867in}{1.400360in}}%
\pgfpathclose%
\pgfusepath{fill}%
\end{pgfscope}%
\begin{pgfscope}%
\pgfpathrectangle{\pgfqpoint{0.329460in}{0.284240in}}{\pgfqpoint{1.989680in}{1.989680in}}%
\pgfusepath{clip}%
\pgfsetbuttcap%
\pgfsetroundjoin%
\definecolor{currentfill}{rgb}{0.166383,0.690856,0.496502}%
\pgfsetfillcolor{currentfill}%
\pgfsetlinewidth{0.000000pt}%
\definecolor{currentstroke}{rgb}{0.000000,0.000000,0.000000}%
\pgfsetstrokecolor{currentstroke}%
\pgfsetdash{}{0pt}%
\pgfpathmoveto{\pgfqpoint{1.354557in}{1.367049in}}%
\pgfpathlineto{\pgfqpoint{1.354606in}{1.358766in}}%
\pgfpathlineto{\pgfqpoint{1.354656in}{1.350435in}}%
\pgfpathlineto{\pgfqpoint{1.354705in}{1.342058in}}%
\pgfpathlineto{\pgfqpoint{1.354754in}{1.333639in}}%
\pgfpathlineto{\pgfqpoint{1.346729in}{1.333646in}}%
\pgfpathlineto{\pgfqpoint{1.338709in}{1.333779in}}%
\pgfpathlineto{\pgfqpoint{1.330702in}{1.334038in}}%
\pgfpathlineto{\pgfqpoint{1.322717in}{1.334423in}}%
\pgfpathlineto{\pgfqpoint{1.323110in}{1.342830in}}%
\pgfpathlineto{\pgfqpoint{1.323504in}{1.351195in}}%
\pgfpathlineto{\pgfqpoint{1.323897in}{1.359514in}}%
\pgfpathlineto{\pgfqpoint{1.324291in}{1.367786in}}%
\pgfpathlineto{\pgfqpoint{1.331835in}{1.367424in}}%
\pgfpathlineto{\pgfqpoint{1.339399in}{1.367181in}}%
\pgfpathlineto{\pgfqpoint{1.346976in}{1.367056in}}%
\pgfpathlineto{\pgfqpoint{1.354557in}{1.367049in}}%
\pgfpathclose%
\pgfusepath{fill}%
\end{pgfscope}%
\begin{pgfscope}%
\pgfpathrectangle{\pgfqpoint{0.329460in}{0.284240in}}{\pgfqpoint{1.989680in}{1.989680in}}%
\pgfusepath{clip}%
\pgfsetbuttcap%
\pgfsetroundjoin%
\definecolor{currentfill}{rgb}{0.487026,0.823929,0.312321}%
\pgfsetfillcolor{currentfill}%
\pgfsetlinewidth{0.000000pt}%
\definecolor{currentstroke}{rgb}{0.000000,0.000000,0.000000}%
\pgfsetstrokecolor{currentstroke}%
\pgfsetdash{}{0pt}%
\pgfpathmoveto{\pgfqpoint{1.416057in}{1.525996in}}%
\pgfpathlineto{\pgfqpoint{1.417408in}{1.519198in}}%
\pgfpathlineto{\pgfqpoint{1.418758in}{1.512316in}}%
\pgfpathlineto{\pgfqpoint{1.420108in}{1.505352in}}%
\pgfpathlineto{\pgfqpoint{1.421457in}{1.498308in}}%
\pgfpathlineto{\pgfqpoint{1.416081in}{1.497274in}}%
\pgfpathlineto{\pgfqpoint{1.410638in}{1.496323in}}%
\pgfpathlineto{\pgfqpoint{1.405133in}{1.495455in}}%
\pgfpathlineto{\pgfqpoint{1.399571in}{1.494672in}}%
\pgfpathlineto{\pgfqpoint{1.398642in}{1.501790in}}%
\pgfpathlineto{\pgfqpoint{1.397712in}{1.508827in}}%
\pgfpathlineto{\pgfqpoint{1.396782in}{1.515783in}}%
\pgfpathlineto{\pgfqpoint{1.395851in}{1.522654in}}%
\pgfpathlineto{\pgfqpoint{1.400985in}{1.523374in}}%
\pgfpathlineto{\pgfqpoint{1.406068in}{1.524171in}}%
\pgfpathlineto{\pgfqpoint{1.411094in}{1.525046in}}%
\pgfpathlineto{\pgfqpoint{1.416057in}{1.525996in}}%
\pgfpathclose%
\pgfusepath{fill}%
\end{pgfscope}%
\begin{pgfscope}%
\pgfpathrectangle{\pgfqpoint{0.329460in}{0.284240in}}{\pgfqpoint{1.989680in}{1.989680in}}%
\pgfusepath{clip}%
\pgfsetbuttcap%
\pgfsetroundjoin%
\definecolor{currentfill}{rgb}{0.762373,0.876424,0.137064}%
\pgfsetfillcolor{currentfill}%
\pgfsetlinewidth{0.000000pt}%
\definecolor{currentstroke}{rgb}{0.000000,0.000000,0.000000}%
\pgfsetstrokecolor{currentstroke}%
\pgfsetdash{}{0pt}%
\pgfpathmoveto{\pgfqpoint{1.338532in}{1.618501in}}%
\pgfpathlineto{\pgfqpoint{1.338135in}{1.613106in}}%
\pgfpathlineto{\pgfqpoint{1.337738in}{1.607606in}}%
\pgfpathlineto{\pgfqpoint{1.337342in}{1.602003in}}%
\pgfpathlineto{\pgfqpoint{1.336945in}{1.596298in}}%
\pgfpathlineto{\pgfqpoint{1.332965in}{1.596542in}}%
\pgfpathlineto{\pgfqpoint{1.329003in}{1.596846in}}%
\pgfpathlineto{\pgfqpoint{1.325065in}{1.597209in}}%
\pgfpathlineto{\pgfqpoint{1.321153in}{1.597632in}}%
\pgfpathlineto{\pgfqpoint{1.321989in}{1.603299in}}%
\pgfpathlineto{\pgfqpoint{1.322825in}{1.608864in}}%
\pgfpathlineto{\pgfqpoint{1.323661in}{1.614325in}}%
\pgfpathlineto{\pgfqpoint{1.324497in}{1.619682in}}%
\pgfpathlineto{\pgfqpoint{1.327974in}{1.619308in}}%
\pgfpathlineto{\pgfqpoint{1.331474in}{1.618986in}}%
\pgfpathlineto{\pgfqpoint{1.334995in}{1.618717in}}%
\pgfpathlineto{\pgfqpoint{1.338532in}{1.618501in}}%
\pgfpathclose%
\pgfusepath{fill}%
\end{pgfscope}%
\begin{pgfscope}%
\pgfpathrectangle{\pgfqpoint{0.329460in}{0.284240in}}{\pgfqpoint{1.989680in}{1.989680in}}%
\pgfusepath{clip}%
\pgfsetbuttcap%
\pgfsetroundjoin%
\definecolor{currentfill}{rgb}{0.268510,0.009605,0.335427}%
\pgfsetfillcolor{currentfill}%
\pgfsetlinewidth{0.000000pt}%
\definecolor{currentstroke}{rgb}{0.000000,0.000000,0.000000}%
\pgfsetstrokecolor{currentstroke}%
\pgfsetdash{}{0pt}%
\pgfpathmoveto{\pgfqpoint{1.217685in}{0.748727in}}%
\pgfpathlineto{\pgfqpoint{1.216804in}{0.748066in}}%
\pgfpathlineto{\pgfqpoint{1.215921in}{0.747659in}}%
\pgfpathlineto{\pgfqpoint{1.215036in}{0.747510in}}%
\pgfpathlineto{\pgfqpoint{1.214147in}{0.747624in}}%
\pgfpathlineto{\pgfqpoint{1.196484in}{0.750134in}}%
\pgfpathlineto{\pgfqpoint{1.178995in}{0.752945in}}%
\pgfpathlineto{\pgfqpoint{1.161701in}{0.756053in}}%
\pgfpathlineto{\pgfqpoint{1.144621in}{0.759456in}}%
\pgfpathlineto{\pgfqpoint{1.145957in}{0.759265in}}%
\pgfpathlineto{\pgfqpoint{1.147290in}{0.759338in}}%
\pgfpathlineto{\pgfqpoint{1.148619in}{0.759668in}}%
\pgfpathlineto{\pgfqpoint{1.149944in}{0.760252in}}%
\pgfpathlineto{\pgfqpoint{1.166586in}{0.756938in}}%
\pgfpathlineto{\pgfqpoint{1.183437in}{0.753910in}}%
\pgfpathlineto{\pgfqpoint{1.200476in}{0.751172in}}%
\pgfpathlineto{\pgfqpoint{1.217685in}{0.748727in}}%
\pgfpathclose%
\pgfusepath{fill}%
\end{pgfscope}%
\begin{pgfscope}%
\pgfpathrectangle{\pgfqpoint{0.329460in}{0.284240in}}{\pgfqpoint{1.989680in}{1.989680in}}%
\pgfusepath{clip}%
\pgfsetbuttcap%
\pgfsetroundjoin%
\definecolor{currentfill}{rgb}{0.565498,0.842430,0.262877}%
\pgfsetfillcolor{currentfill}%
\pgfsetlinewidth{0.000000pt}%
\definecolor{currentstroke}{rgb}{0.000000,0.000000,0.000000}%
\pgfsetstrokecolor{currentstroke}%
\pgfsetdash{}{0pt}%
\pgfpathmoveto{\pgfqpoint{1.410652in}{1.552323in}}%
\pgfpathlineto{\pgfqpoint{1.412004in}{1.545874in}}%
\pgfpathlineto{\pgfqpoint{1.413355in}{1.539335in}}%
\pgfpathlineto{\pgfqpoint{1.414706in}{1.532709in}}%
\pgfpathlineto{\pgfqpoint{1.416057in}{1.525996in}}%
\pgfpathlineto{\pgfqpoint{1.411094in}{1.525046in}}%
\pgfpathlineto{\pgfqpoint{1.406068in}{1.524171in}}%
\pgfpathlineto{\pgfqpoint{1.400985in}{1.523374in}}%
\pgfpathlineto{\pgfqpoint{1.395851in}{1.522654in}}%
\pgfpathlineto{\pgfqpoint{1.394921in}{1.529441in}}%
\pgfpathlineto{\pgfqpoint{1.393990in}{1.536140in}}%
\pgfpathlineto{\pgfqpoint{1.393059in}{1.542752in}}%
\pgfpathlineto{\pgfqpoint{1.392128in}{1.549273in}}%
\pgfpathlineto{\pgfqpoint{1.396834in}{1.549930in}}%
\pgfpathlineto{\pgfqpoint{1.401494in}{1.550658in}}%
\pgfpathlineto{\pgfqpoint{1.406101in}{1.551456in}}%
\pgfpathlineto{\pgfqpoint{1.410652in}{1.552323in}}%
\pgfpathclose%
\pgfusepath{fill}%
\end{pgfscope}%
\begin{pgfscope}%
\pgfpathrectangle{\pgfqpoint{0.329460in}{0.284240in}}{\pgfqpoint{1.989680in}{1.989680in}}%
\pgfusepath{clip}%
\pgfsetbuttcap%
\pgfsetroundjoin%
\definecolor{currentfill}{rgb}{0.267004,0.004874,0.329415}%
\pgfsetfillcolor{currentfill}%
\pgfsetlinewidth{0.000000pt}%
\definecolor{currentstroke}{rgb}{0.000000,0.000000,0.000000}%
\pgfsetstrokecolor{currentstroke}%
\pgfsetdash{}{0pt}%
\pgfpathmoveto{\pgfqpoint{1.289494in}{0.747207in}}%
\pgfpathlineto{\pgfqpoint{1.289081in}{0.745534in}}%
\pgfpathlineto{\pgfqpoint{1.288666in}{0.744095in}}%
\pgfpathlineto{\pgfqpoint{1.288250in}{0.742895in}}%
\pgfpathlineto{\pgfqpoint{1.287833in}{0.741940in}}%
\pgfpathlineto{\pgfqpoint{1.270138in}{0.743183in}}%
\pgfpathlineto{\pgfqpoint{1.252536in}{0.744730in}}%
\pgfpathlineto{\pgfqpoint{1.235045in}{0.746579in}}%
\pgfpathlineto{\pgfqpoint{1.217685in}{0.748727in}}%
\pgfpathlineto{\pgfqpoint{1.218563in}{0.749637in}}%
\pgfpathlineto{\pgfqpoint{1.219439in}{0.750790in}}%
\pgfpathlineto{\pgfqpoint{1.220312in}{0.752184in}}%
\pgfpathlineto{\pgfqpoint{1.221183in}{0.753811in}}%
\pgfpathlineto{\pgfqpoint{1.238089in}{0.751721in}}%
\pgfpathlineto{\pgfqpoint{1.255122in}{0.749922in}}%
\pgfpathlineto{\pgfqpoint{1.272264in}{0.748417in}}%
\pgfpathlineto{\pgfqpoint{1.289494in}{0.747207in}}%
\pgfpathclose%
\pgfusepath{fill}%
\end{pgfscope}%
\begin{pgfscope}%
\pgfpathrectangle{\pgfqpoint{0.329460in}{0.284240in}}{\pgfqpoint{1.989680in}{1.989680in}}%
\pgfusepath{clip}%
\pgfsetbuttcap%
\pgfsetroundjoin%
\definecolor{currentfill}{rgb}{0.412913,0.803041,0.357269}%
\pgfsetfillcolor{currentfill}%
\pgfsetlinewidth{0.000000pt}%
\definecolor{currentstroke}{rgb}{0.000000,0.000000,0.000000}%
\pgfsetstrokecolor{currentstroke}%
\pgfsetdash{}{0pt}%
\pgfpathmoveto{\pgfqpoint{1.307789in}{1.494048in}}%
\pgfpathlineto{\pgfqpoint{1.306956in}{1.486838in}}%
\pgfpathlineto{\pgfqpoint{1.306122in}{1.479551in}}%
\pgfpathlineto{\pgfqpoint{1.305289in}{1.472188in}}%
\pgfpathlineto{\pgfqpoint{1.304456in}{1.464751in}}%
\pgfpathlineto{\pgfqpoint{1.298419in}{1.465517in}}%
\pgfpathlineto{\pgfqpoint{1.292438in}{1.466374in}}%
\pgfpathlineto{\pgfqpoint{1.286518in}{1.467323in}}%
\pgfpathlineto{\pgfqpoint{1.280665in}{1.468363in}}%
\pgfpathlineto{\pgfqpoint{1.281922in}{1.475731in}}%
\pgfpathlineto{\pgfqpoint{1.283179in}{1.483025in}}%
\pgfpathlineto{\pgfqpoint{1.284436in}{1.490244in}}%
\pgfpathlineto{\pgfqpoint{1.285693in}{1.497385in}}%
\pgfpathlineto{\pgfqpoint{1.291130in}{1.496425in}}%
\pgfpathlineto{\pgfqpoint{1.296628in}{1.495547in}}%
\pgfpathlineto{\pgfqpoint{1.302183in}{1.494755in}}%
\pgfpathlineto{\pgfqpoint{1.307789in}{1.494048in}}%
\pgfpathclose%
\pgfusepath{fill}%
\end{pgfscope}%
\begin{pgfscope}%
\pgfpathrectangle{\pgfqpoint{0.329460in}{0.284240in}}{\pgfqpoint{1.989680in}{1.989680in}}%
\pgfusepath{clip}%
\pgfsetbuttcap%
\pgfsetroundjoin%
\definecolor{currentfill}{rgb}{0.636902,0.856542,0.216620}%
\pgfsetfillcolor{currentfill}%
\pgfsetlinewidth{0.000000pt}%
\definecolor{currentstroke}{rgb}{0.000000,0.000000,0.000000}%
\pgfsetstrokecolor{currentstroke}%
\pgfsetdash{}{0pt}%
\pgfpathmoveto{\pgfqpoint{1.317809in}{1.573962in}}%
\pgfpathlineto{\pgfqpoint{1.316973in}{1.567800in}}%
\pgfpathlineto{\pgfqpoint{1.316137in}{1.561543in}}%
\pgfpathlineto{\pgfqpoint{1.315302in}{1.555193in}}%
\pgfpathlineto{\pgfqpoint{1.314466in}{1.548750in}}%
\pgfpathlineto{\pgfqpoint{1.309722in}{1.549343in}}%
\pgfpathlineto{\pgfqpoint{1.305021in}{1.550007in}}%
\pgfpathlineto{\pgfqpoint{1.300367in}{1.550743in}}%
\pgfpathlineto{\pgfqpoint{1.295765in}{1.551549in}}%
\pgfpathlineto{\pgfqpoint{1.297025in}{1.557925in}}%
\pgfpathlineto{\pgfqpoint{1.298286in}{1.564209in}}%
\pgfpathlineto{\pgfqpoint{1.299547in}{1.570400in}}%
\pgfpathlineto{\pgfqpoint{1.300807in}{1.576495in}}%
\pgfpathlineto{\pgfqpoint{1.304991in}{1.575766in}}%
\pgfpathlineto{\pgfqpoint{1.309222in}{1.575100in}}%
\pgfpathlineto{\pgfqpoint{1.313496in}{1.574499in}}%
\pgfpathlineto{\pgfqpoint{1.317809in}{1.573962in}}%
\pgfpathclose%
\pgfusepath{fill}%
\end{pgfscope}%
\begin{pgfscope}%
\pgfpathrectangle{\pgfqpoint{0.329460in}{0.284240in}}{\pgfqpoint{1.989680in}{1.989680in}}%
\pgfusepath{clip}%
\pgfsetbuttcap%
\pgfsetroundjoin%
\definecolor{currentfill}{rgb}{0.281477,0.755203,0.432552}%
\pgfsetfillcolor{currentfill}%
\pgfsetlinewidth{0.000000pt}%
\definecolor{currentstroke}{rgb}{0.000000,0.000000,0.000000}%
\pgfsetstrokecolor{currentstroke}%
\pgfsetdash{}{0pt}%
\pgfpathmoveto{\pgfqpoint{1.407000in}{1.435028in}}%
\pgfpathlineto{\pgfqpoint{1.407927in}{1.427261in}}%
\pgfpathlineto{\pgfqpoint{1.408854in}{1.419431in}}%
\pgfpathlineto{\pgfqpoint{1.409781in}{1.411540in}}%
\pgfpathlineto{\pgfqpoint{1.410707in}{1.403590in}}%
\pgfpathlineto{\pgfqpoint{1.403806in}{1.402718in}}%
\pgfpathlineto{\pgfqpoint{1.396849in}{1.401955in}}%
\pgfpathlineto{\pgfqpoint{1.389844in}{1.401299in}}%
\pgfpathlineto{\pgfqpoint{1.382799in}{1.400753in}}%
\pgfpathlineto{\pgfqpoint{1.382306in}{1.408751in}}%
\pgfpathlineto{\pgfqpoint{1.381814in}{1.416690in}}%
\pgfpathlineto{\pgfqpoint{1.381321in}{1.424567in}}%
\pgfpathlineto{\pgfqpoint{1.380829in}{1.432382in}}%
\pgfpathlineto{\pgfqpoint{1.387435in}{1.432891in}}%
\pgfpathlineto{\pgfqpoint{1.394004in}{1.433502in}}%
\pgfpathlineto{\pgfqpoint{1.400528in}{1.434215in}}%
\pgfpathlineto{\pgfqpoint{1.407000in}{1.435028in}}%
\pgfpathclose%
\pgfusepath{fill}%
\end{pgfscope}%
\begin{pgfscope}%
\pgfpathrectangle{\pgfqpoint{0.329460in}{0.284240in}}{\pgfqpoint{1.989680in}{1.989680in}}%
\pgfusepath{clip}%
\pgfsetbuttcap%
\pgfsetroundjoin%
\definecolor{currentfill}{rgb}{0.487026,0.823929,0.312321}%
\pgfsetfillcolor{currentfill}%
\pgfsetlinewidth{0.000000pt}%
\definecolor{currentstroke}{rgb}{0.000000,0.000000,0.000000}%
\pgfsetstrokecolor{currentstroke}%
\pgfsetdash{}{0pt}%
\pgfpathmoveto{\pgfqpoint{1.311127in}{1.522081in}}%
\pgfpathlineto{\pgfqpoint{1.310292in}{1.515196in}}%
\pgfpathlineto{\pgfqpoint{1.309458in}{1.508228in}}%
\pgfpathlineto{\pgfqpoint{1.308623in}{1.501178in}}%
\pgfpathlineto{\pgfqpoint{1.307789in}{1.494048in}}%
\pgfpathlineto{\pgfqpoint{1.302183in}{1.494755in}}%
\pgfpathlineto{\pgfqpoint{1.296628in}{1.495547in}}%
\pgfpathlineto{\pgfqpoint{1.291130in}{1.496425in}}%
\pgfpathlineto{\pgfqpoint{1.285693in}{1.497385in}}%
\pgfpathlineto{\pgfqpoint{1.286951in}{1.504448in}}%
\pgfpathlineto{\pgfqpoint{1.288209in}{1.511430in}}%
\pgfpathlineto{\pgfqpoint{1.289468in}{1.518331in}}%
\pgfpathlineto{\pgfqpoint{1.290727in}{1.525147in}}%
\pgfpathlineto{\pgfqpoint{1.295746in}{1.524265in}}%
\pgfpathlineto{\pgfqpoint{1.300822in}{1.523459in}}%
\pgfpathlineto{\pgfqpoint{1.305951in}{1.522730in}}%
\pgfpathlineto{\pgfqpoint{1.311127in}{1.522081in}}%
\pgfpathclose%
\pgfusepath{fill}%
\end{pgfscope}%
\begin{pgfscope}%
\pgfpathrectangle{\pgfqpoint{0.329460in}{0.284240in}}{\pgfqpoint{1.989680in}{1.989680in}}%
\pgfusepath{clip}%
\pgfsetbuttcap%
\pgfsetroundjoin%
\definecolor{currentfill}{rgb}{0.762373,0.876424,0.137064}%
\pgfsetfillcolor{currentfill}%
\pgfsetlinewidth{0.000000pt}%
\definecolor{currentstroke}{rgb}{0.000000,0.000000,0.000000}%
\pgfsetstrokecolor{currentstroke}%
\pgfsetdash{}{0pt}%
\pgfpathmoveto{\pgfqpoint{1.366988in}{1.618691in}}%
\pgfpathlineto{\pgfqpoint{1.367483in}{1.613302in}}%
\pgfpathlineto{\pgfqpoint{1.367979in}{1.607808in}}%
\pgfpathlineto{\pgfqpoint{1.368474in}{1.602211in}}%
\pgfpathlineto{\pgfqpoint{1.368969in}{1.596512in}}%
\pgfpathlineto{\pgfqpoint{1.364987in}{1.596274in}}%
\pgfpathlineto{\pgfqpoint{1.360990in}{1.596096in}}%
\pgfpathlineto{\pgfqpoint{1.356984in}{1.595979in}}%
\pgfpathlineto{\pgfqpoint{1.352972in}{1.595922in}}%
\pgfpathlineto{\pgfqpoint{1.352922in}{1.601638in}}%
\pgfpathlineto{\pgfqpoint{1.352872in}{1.607252in}}%
\pgfpathlineto{\pgfqpoint{1.352823in}{1.612763in}}%
\pgfpathlineto{\pgfqpoint{1.352773in}{1.618168in}}%
\pgfpathlineto{\pgfqpoint{1.356338in}{1.618219in}}%
\pgfpathlineto{\pgfqpoint{1.359898in}{1.618323in}}%
\pgfpathlineto{\pgfqpoint{1.363450in}{1.618480in}}%
\pgfpathlineto{\pgfqpoint{1.366988in}{1.618691in}}%
\pgfpathclose%
\pgfusepath{fill}%
\end{pgfscope}%
\begin{pgfscope}%
\pgfpathrectangle{\pgfqpoint{0.329460in}{0.284240in}}{\pgfqpoint{1.989680in}{1.989680in}}%
\pgfusepath{clip}%
\pgfsetbuttcap%
\pgfsetroundjoin%
\definecolor{currentfill}{rgb}{0.565498,0.842430,0.262877}%
\pgfsetfillcolor{currentfill}%
\pgfsetlinewidth{0.000000pt}%
\definecolor{currentstroke}{rgb}{0.000000,0.000000,0.000000}%
\pgfsetstrokecolor{currentstroke}%
\pgfsetdash{}{0pt}%
\pgfpathmoveto{\pgfqpoint{1.314466in}{1.548750in}}%
\pgfpathlineto{\pgfqpoint{1.313631in}{1.542216in}}%
\pgfpathlineto{\pgfqpoint{1.312796in}{1.535592in}}%
\pgfpathlineto{\pgfqpoint{1.311961in}{1.528880in}}%
\pgfpathlineto{\pgfqpoint{1.311127in}{1.522081in}}%
\pgfpathlineto{\pgfqpoint{1.305951in}{1.522730in}}%
\pgfpathlineto{\pgfqpoint{1.300822in}{1.523459in}}%
\pgfpathlineto{\pgfqpoint{1.295746in}{1.524265in}}%
\pgfpathlineto{\pgfqpoint{1.290727in}{1.525147in}}%
\pgfpathlineto{\pgfqpoint{1.291986in}{1.531879in}}%
\pgfpathlineto{\pgfqpoint{1.293246in}{1.538524in}}%
\pgfpathlineto{\pgfqpoint{1.294505in}{1.545081in}}%
\pgfpathlineto{\pgfqpoint{1.295765in}{1.551549in}}%
\pgfpathlineto{\pgfqpoint{1.300367in}{1.550743in}}%
\pgfpathlineto{\pgfqpoint{1.305021in}{1.550007in}}%
\pgfpathlineto{\pgfqpoint{1.309722in}{1.549343in}}%
\pgfpathlineto{\pgfqpoint{1.314466in}{1.548750in}}%
\pgfpathclose%
\pgfusepath{fill}%
\end{pgfscope}%
\begin{pgfscope}%
\pgfpathrectangle{\pgfqpoint{0.329460in}{0.284240in}}{\pgfqpoint{1.989680in}{1.989680in}}%
\pgfusepath{clip}%
\pgfsetbuttcap%
\pgfsetroundjoin%
\definecolor{currentfill}{rgb}{0.762373,0.876424,0.137064}%
\pgfsetfillcolor{currentfill}%
\pgfsetlinewidth{0.000000pt}%
\definecolor{currentstroke}{rgb}{0.000000,0.000000,0.000000}%
\pgfsetstrokecolor{currentstroke}%
\pgfsetdash{}{0pt}%
\pgfpathmoveto{\pgfqpoint{1.352773in}{1.618168in}}%
\pgfpathlineto{\pgfqpoint{1.352823in}{1.612763in}}%
\pgfpathlineto{\pgfqpoint{1.352872in}{1.607252in}}%
\pgfpathlineto{\pgfqpoint{1.352922in}{1.601638in}}%
\pgfpathlineto{\pgfqpoint{1.352972in}{1.595922in}}%
\pgfpathlineto{\pgfqpoint{1.348958in}{1.595925in}}%
\pgfpathlineto{\pgfqpoint{1.344946in}{1.595989in}}%
\pgfpathlineto{\pgfqpoint{1.340940in}{1.596113in}}%
\pgfpathlineto{\pgfqpoint{1.336945in}{1.596298in}}%
\pgfpathlineto{\pgfqpoint{1.337342in}{1.602003in}}%
\pgfpathlineto{\pgfqpoint{1.337738in}{1.607606in}}%
\pgfpathlineto{\pgfqpoint{1.338135in}{1.613106in}}%
\pgfpathlineto{\pgfqpoint{1.338532in}{1.618501in}}%
\pgfpathlineto{\pgfqpoint{1.342082in}{1.618338in}}%
\pgfpathlineto{\pgfqpoint{1.345641in}{1.618228in}}%
\pgfpathlineto{\pgfqpoint{1.349206in}{1.618171in}}%
\pgfpathlineto{\pgfqpoint{1.352773in}{1.618168in}}%
\pgfpathclose%
\pgfusepath{fill}%
\end{pgfscope}%
\begin{pgfscope}%
\pgfpathrectangle{\pgfqpoint{0.329460in}{0.284240in}}{\pgfqpoint{1.989680in}{1.989680in}}%
\pgfusepath{clip}%
\pgfsetbuttcap%
\pgfsetroundjoin%
\definecolor{currentfill}{rgb}{0.281477,0.755203,0.432552}%
\pgfsetfillcolor{currentfill}%
\pgfsetlinewidth{0.000000pt}%
\definecolor{currentstroke}{rgb}{0.000000,0.000000,0.000000}%
\pgfsetstrokecolor{currentstroke}%
\pgfsetdash{}{0pt}%
\pgfpathmoveto{\pgfqpoint{1.327445in}{1.432015in}}%
\pgfpathlineto{\pgfqpoint{1.327050in}{1.424194in}}%
\pgfpathlineto{\pgfqpoint{1.326656in}{1.416310in}}%
\pgfpathlineto{\pgfqpoint{1.326261in}{1.408364in}}%
\pgfpathlineto{\pgfqpoint{1.325867in}{1.400360in}}%
\pgfpathlineto{\pgfqpoint{1.318792in}{1.400808in}}%
\pgfpathlineto{\pgfqpoint{1.311751in}{1.401367in}}%
\pgfpathlineto{\pgfqpoint{1.304751in}{1.402034in}}%
\pgfpathlineto{\pgfqpoint{1.297800in}{1.402810in}}%
\pgfpathlineto{\pgfqpoint{1.298631in}{1.410773in}}%
\pgfpathlineto{\pgfqpoint{1.299462in}{1.418677in}}%
\pgfpathlineto{\pgfqpoint{1.300294in}{1.426520in}}%
\pgfpathlineto{\pgfqpoint{1.301126in}{1.434300in}}%
\pgfpathlineto{\pgfqpoint{1.307644in}{1.433577in}}%
\pgfpathlineto{\pgfqpoint{1.314208in}{1.432954in}}%
\pgfpathlineto{\pgfqpoint{1.320811in}{1.432433in}}%
\pgfpathlineto{\pgfqpoint{1.327445in}{1.432015in}}%
\pgfpathclose%
\pgfusepath{fill}%
\end{pgfscope}%
\begin{pgfscope}%
\pgfpathrectangle{\pgfqpoint{0.329460in}{0.284240in}}{\pgfqpoint{1.989680in}{1.989680in}}%
\pgfusepath{clip}%
\pgfsetbuttcap%
\pgfsetroundjoin%
\definecolor{currentfill}{rgb}{0.699415,0.867117,0.175971}%
\pgfsetfillcolor{currentfill}%
\pgfsetlinewidth{0.000000pt}%
\definecolor{currentstroke}{rgb}{0.000000,0.000000,0.000000}%
\pgfsetstrokecolor{currentstroke}%
\pgfsetdash{}{0pt}%
\pgfpathmoveto{\pgfqpoint{1.384674in}{1.598057in}}%
\pgfpathlineto{\pgfqpoint{1.385606in}{1.592300in}}%
\pgfpathlineto{\pgfqpoint{1.386538in}{1.586444in}}%
\pgfpathlineto{\pgfqpoint{1.387470in}{1.580489in}}%
\pgfpathlineto{\pgfqpoint{1.388402in}{1.574436in}}%
\pgfpathlineto{\pgfqpoint{1.384085in}{1.573906in}}%
\pgfpathlineto{\pgfqpoint{1.379734in}{1.573442in}}%
\pgfpathlineto{\pgfqpoint{1.375354in}{1.573044in}}%
\pgfpathlineto{\pgfqpoint{1.370949in}{1.572712in}}%
\pgfpathlineto{\pgfqpoint{1.370454in}{1.578810in}}%
\pgfpathlineto{\pgfqpoint{1.369959in}{1.584810in}}%
\pgfpathlineto{\pgfqpoint{1.369464in}{1.590711in}}%
\pgfpathlineto{\pgfqpoint{1.368969in}{1.596512in}}%
\pgfpathlineto{\pgfqpoint{1.372933in}{1.596809in}}%
\pgfpathlineto{\pgfqpoint{1.376874in}{1.597166in}}%
\pgfpathlineto{\pgfqpoint{1.380789in}{1.597582in}}%
\pgfpathlineto{\pgfqpoint{1.384674in}{1.598057in}}%
\pgfpathclose%
\pgfusepath{fill}%
\end{pgfscope}%
\begin{pgfscope}%
\pgfpathrectangle{\pgfqpoint{0.329460in}{0.284240in}}{\pgfqpoint{1.989680in}{1.989680in}}%
\pgfusepath{clip}%
\pgfsetbuttcap%
\pgfsetroundjoin%
\definecolor{currentfill}{rgb}{0.220124,0.725509,0.466226}%
\pgfsetfillcolor{currentfill}%
\pgfsetlinewidth{0.000000pt}%
\definecolor{currentstroke}{rgb}{0.000000,0.000000,0.000000}%
\pgfsetstrokecolor{currentstroke}%
\pgfsetdash{}{0pt}%
\pgfpathmoveto{\pgfqpoint{1.382799in}{1.400753in}}%
\pgfpathlineto{\pgfqpoint{1.383291in}{1.392697in}}%
\pgfpathlineto{\pgfqpoint{1.383783in}{1.384586in}}%
\pgfpathlineto{\pgfqpoint{1.384275in}{1.376422in}}%
\pgfpathlineto{\pgfqpoint{1.384766in}{1.368206in}}%
\pgfpathlineto{\pgfqpoint{1.377247in}{1.367740in}}%
\pgfpathlineto{\pgfqpoint{1.369701in}{1.367392in}}%
\pgfpathlineto{\pgfqpoint{1.362135in}{1.367161in}}%
\pgfpathlineto{\pgfqpoint{1.354557in}{1.367049in}}%
\pgfpathlineto{\pgfqpoint{1.354508in}{1.375284in}}%
\pgfpathlineto{\pgfqpoint{1.354458in}{1.383467in}}%
\pgfpathlineto{\pgfqpoint{1.354409in}{1.391596in}}%
\pgfpathlineto{\pgfqpoint{1.354360in}{1.399670in}}%
\pgfpathlineto{\pgfqpoint{1.361493in}{1.399775in}}%
\pgfpathlineto{\pgfqpoint{1.368616in}{1.399990in}}%
\pgfpathlineto{\pgfqpoint{1.375720in}{1.400316in}}%
\pgfpathlineto{\pgfqpoint{1.382799in}{1.400753in}}%
\pgfpathclose%
\pgfusepath{fill}%
\end{pgfscope}%
\begin{pgfscope}%
\pgfpathrectangle{\pgfqpoint{0.329460in}{0.284240in}}{\pgfqpoint{1.989680in}{1.989680in}}%
\pgfusepath{clip}%
\pgfsetbuttcap%
\pgfsetroundjoin%
\definecolor{currentfill}{rgb}{0.220124,0.725509,0.466226}%
\pgfsetfillcolor{currentfill}%
\pgfsetlinewidth{0.000000pt}%
\definecolor{currentstroke}{rgb}{0.000000,0.000000,0.000000}%
\pgfsetstrokecolor{currentstroke}%
\pgfsetdash{}{0pt}%
\pgfpathmoveto{\pgfqpoint{1.354360in}{1.399670in}}%
\pgfpathlineto{\pgfqpoint{1.354409in}{1.391596in}}%
\pgfpathlineto{\pgfqpoint{1.354458in}{1.383467in}}%
\pgfpathlineto{\pgfqpoint{1.354508in}{1.375284in}}%
\pgfpathlineto{\pgfqpoint{1.354557in}{1.367049in}}%
\pgfpathlineto{\pgfqpoint{1.346976in}{1.367056in}}%
\pgfpathlineto{\pgfqpoint{1.339399in}{1.367181in}}%
\pgfpathlineto{\pgfqpoint{1.331835in}{1.367424in}}%
\pgfpathlineto{\pgfqpoint{1.324291in}{1.367786in}}%
\pgfpathlineto{\pgfqpoint{1.324685in}{1.376008in}}%
\pgfpathlineto{\pgfqpoint{1.325079in}{1.384180in}}%
\pgfpathlineto{\pgfqpoint{1.325473in}{1.392297in}}%
\pgfpathlineto{\pgfqpoint{1.325867in}{1.400360in}}%
\pgfpathlineto{\pgfqpoint{1.332969in}{1.400021in}}%
\pgfpathlineto{\pgfqpoint{1.340090in}{1.399793in}}%
\pgfpathlineto{\pgfqpoint{1.347223in}{1.399676in}}%
\pgfpathlineto{\pgfqpoint{1.354360in}{1.399670in}}%
\pgfpathclose%
\pgfusepath{fill}%
\end{pgfscope}%
\begin{pgfscope}%
\pgfpathrectangle{\pgfqpoint{0.329460in}{0.284240in}}{\pgfqpoint{1.989680in}{1.989680in}}%
\pgfusepath{clip}%
\pgfsetbuttcap%
\pgfsetroundjoin%
\definecolor{currentfill}{rgb}{0.699415,0.867117,0.175971}%
\pgfsetfillcolor{currentfill}%
\pgfsetlinewidth{0.000000pt}%
\definecolor{currentstroke}{rgb}{0.000000,0.000000,0.000000}%
\pgfsetstrokecolor{currentstroke}%
\pgfsetdash{}{0pt}%
\pgfpathmoveto{\pgfqpoint{1.336945in}{1.596298in}}%
\pgfpathlineto{\pgfqpoint{1.336549in}{1.590490in}}%
\pgfpathlineto{\pgfqpoint{1.336152in}{1.584583in}}%
\pgfpathlineto{\pgfqpoint{1.335756in}{1.578577in}}%
\pgfpathlineto{\pgfqpoint{1.335359in}{1.572473in}}%
\pgfpathlineto{\pgfqpoint{1.330936in}{1.572746in}}%
\pgfpathlineto{\pgfqpoint{1.326533in}{1.573085in}}%
\pgfpathlineto{\pgfqpoint{1.322156in}{1.573490in}}%
\pgfpathlineto{\pgfqpoint{1.317809in}{1.573962in}}%
\pgfpathlineto{\pgfqpoint{1.318645in}{1.580027in}}%
\pgfpathlineto{\pgfqpoint{1.319481in}{1.585995in}}%
\pgfpathlineto{\pgfqpoint{1.320317in}{1.591863in}}%
\pgfpathlineto{\pgfqpoint{1.321153in}{1.597632in}}%
\pgfpathlineto{\pgfqpoint{1.325065in}{1.597209in}}%
\pgfpathlineto{\pgfqpoint{1.329003in}{1.596846in}}%
\pgfpathlineto{\pgfqpoint{1.332965in}{1.596542in}}%
\pgfpathlineto{\pgfqpoint{1.336945in}{1.596298in}}%
\pgfpathclose%
\pgfusepath{fill}%
\end{pgfscope}%
\begin{pgfscope}%
\pgfpathrectangle{\pgfqpoint{0.329460in}{0.284240in}}{\pgfqpoint{1.989680in}{1.989680in}}%
\pgfusepath{clip}%
\pgfsetbuttcap%
\pgfsetroundjoin%
\definecolor{currentfill}{rgb}{0.282327,0.094955,0.417331}%
\pgfsetfillcolor{currentfill}%
\pgfsetlinewidth{0.000000pt}%
\definecolor{currentstroke}{rgb}{0.000000,0.000000,0.000000}%
\pgfsetstrokecolor{currentstroke}%
\pgfsetdash{}{0pt}%
\pgfpathmoveto{\pgfqpoint{1.064540in}{0.788350in}}%
\pgfpathlineto{\pgfqpoint{1.062728in}{0.791241in}}%
\pgfpathlineto{\pgfqpoint{1.060909in}{0.794447in}}%
\pgfpathlineto{\pgfqpoint{1.059084in}{0.797974in}}%
\pgfpathlineto{\pgfqpoint{1.057251in}{0.801827in}}%
\pgfpathlineto{\pgfqpoint{1.040288in}{0.806987in}}%
\pgfpathlineto{\pgfqpoint{1.023675in}{0.812430in}}%
\pgfpathlineto{\pgfqpoint{1.007429in}{0.818148in}}%
\pgfpathlineto{\pgfqpoint{0.991569in}{0.824134in}}%
\pgfpathlineto{\pgfqpoint{0.993804in}{0.820155in}}%
\pgfpathlineto{\pgfqpoint{0.996031in}{0.816501in}}%
\pgfpathlineto{\pgfqpoint{0.998248in}{0.813168in}}%
\pgfpathlineto{\pgfqpoint{1.000458in}{0.810149in}}%
\pgfpathlineto{\pgfqpoint{1.015934in}{0.804299in}}%
\pgfpathlineto{\pgfqpoint{1.031784in}{0.798711in}}%
\pgfpathlineto{\pgfqpoint{1.047992in}{0.793393in}}%
\pgfpathlineto{\pgfqpoint{1.064540in}{0.788350in}}%
\pgfpathclose%
\pgfusepath{fill}%
\end{pgfscope}%
\begin{pgfscope}%
\pgfpathrectangle{\pgfqpoint{0.329460in}{0.284240in}}{\pgfqpoint{1.989680in}{1.989680in}}%
\pgfusepath{clip}%
\pgfsetbuttcap%
\pgfsetroundjoin%
\definecolor{currentfill}{rgb}{0.344074,0.780029,0.397381}%
\pgfsetfillcolor{currentfill}%
\pgfsetlinewidth{0.000000pt}%
\definecolor{currentstroke}{rgb}{0.000000,0.000000,0.000000}%
\pgfsetstrokecolor{currentstroke}%
\pgfsetdash{}{0pt}%
\pgfpathmoveto{\pgfqpoint{1.403288in}{1.465427in}}%
\pgfpathlineto{\pgfqpoint{1.404216in}{1.457931in}}%
\pgfpathlineto{\pgfqpoint{1.405144in}{1.450365in}}%
\pgfpathlineto{\pgfqpoint{1.406072in}{1.442730in}}%
\pgfpathlineto{\pgfqpoint{1.407000in}{1.435028in}}%
\pgfpathlineto{\pgfqpoint{1.400528in}{1.434215in}}%
\pgfpathlineto{\pgfqpoint{1.394004in}{1.433502in}}%
\pgfpathlineto{\pgfqpoint{1.387435in}{1.432891in}}%
\pgfpathlineto{\pgfqpoint{1.380829in}{1.432382in}}%
\pgfpathlineto{\pgfqpoint{1.380336in}{1.440131in}}%
\pgfpathlineto{\pgfqpoint{1.379843in}{1.447813in}}%
\pgfpathlineto{\pgfqpoint{1.379350in}{1.455426in}}%
\pgfpathlineto{\pgfqpoint{1.378857in}{1.462969in}}%
\pgfpathlineto{\pgfqpoint{1.385024in}{1.463442in}}%
\pgfpathlineto{\pgfqpoint{1.391155in}{1.464010in}}%
\pgfpathlineto{\pgfqpoint{1.397246in}{1.464672in}}%
\pgfpathlineto{\pgfqpoint{1.403288in}{1.465427in}}%
\pgfpathclose%
\pgfusepath{fill}%
\end{pgfscope}%
\begin{pgfscope}%
\pgfpathrectangle{\pgfqpoint{0.329460in}{0.284240in}}{\pgfqpoint{1.989680in}{1.989680in}}%
\pgfusepath{clip}%
\pgfsetbuttcap%
\pgfsetroundjoin%
\definecolor{currentfill}{rgb}{0.276194,0.190074,0.493001}%
\pgfsetfillcolor{currentfill}%
\pgfsetlinewidth{0.000000pt}%
\definecolor{currentstroke}{rgb}{0.000000,0.000000,0.000000}%
\pgfsetstrokecolor{currentstroke}%
\pgfsetdash{}{0pt}%
\pgfpathmoveto{\pgfqpoint{1.793042in}{0.877019in}}%
\pgfpathlineto{\pgfqpoint{1.795763in}{0.882893in}}%
\pgfpathlineto{\pgfqpoint{1.798497in}{0.889126in}}%
\pgfpathlineto{\pgfqpoint{1.801242in}{0.895725in}}%
\pgfpathlineto{\pgfqpoint{1.804000in}{0.902698in}}%
\pgfpathlineto{\pgfqpoint{1.789583in}{0.895201in}}%
\pgfpathlineto{\pgfqpoint{1.774678in}{0.887943in}}%
\pgfpathlineto{\pgfqpoint{1.759302in}{0.880933in}}%
\pgfpathlineto{\pgfqpoint{1.743470in}{0.874178in}}%
\pgfpathlineto{\pgfqpoint{1.741071in}{0.867351in}}%
\pgfpathlineto{\pgfqpoint{1.738682in}{0.860897in}}%
\pgfpathlineto{\pgfqpoint{1.736304in}{0.854810in}}%
\pgfpathlineto{\pgfqpoint{1.733937in}{0.849085in}}%
\pgfpathlineto{\pgfqpoint{1.749394in}{0.855700in}}%
\pgfpathlineto{\pgfqpoint{1.764407in}{0.862567in}}%
\pgfpathlineto{\pgfqpoint{1.778961in}{0.869676in}}%
\pgfpathlineto{\pgfqpoint{1.793042in}{0.877019in}}%
\pgfpathclose%
\pgfusepath{fill}%
\end{pgfscope}%
\begin{pgfscope}%
\pgfpathrectangle{\pgfqpoint{0.329460in}{0.284240in}}{\pgfqpoint{1.989680in}{1.989680in}}%
\pgfusepath{clip}%
\pgfsetbuttcap%
\pgfsetroundjoin%
\definecolor{currentfill}{rgb}{0.636902,0.856542,0.216620}%
\pgfsetfillcolor{currentfill}%
\pgfsetlinewidth{0.000000pt}%
\definecolor{currentstroke}{rgb}{0.000000,0.000000,0.000000}%
\pgfsetstrokecolor{currentstroke}%
\pgfsetdash{}{0pt}%
\pgfpathmoveto{\pgfqpoint{1.388402in}{1.574436in}}%
\pgfpathlineto{\pgfqpoint{1.389334in}{1.568287in}}%
\pgfpathlineto{\pgfqpoint{1.390265in}{1.562042in}}%
\pgfpathlineto{\pgfqpoint{1.391197in}{1.555704in}}%
\pgfpathlineto{\pgfqpoint{1.392128in}{1.549273in}}%
\pgfpathlineto{\pgfqpoint{1.387379in}{1.548688in}}%
\pgfpathlineto{\pgfqpoint{1.382593in}{1.548176in}}%
\pgfpathlineto{\pgfqpoint{1.377774in}{1.547736in}}%
\pgfpathlineto{\pgfqpoint{1.372928in}{1.547369in}}%
\pgfpathlineto{\pgfqpoint{1.372433in}{1.553845in}}%
\pgfpathlineto{\pgfqpoint{1.371939in}{1.560228in}}%
\pgfpathlineto{\pgfqpoint{1.371444in}{1.566518in}}%
\pgfpathlineto{\pgfqpoint{1.370949in}{1.572712in}}%
\pgfpathlineto{\pgfqpoint{1.375354in}{1.573044in}}%
\pgfpathlineto{\pgfqpoint{1.379734in}{1.573442in}}%
\pgfpathlineto{\pgfqpoint{1.384085in}{1.573906in}}%
\pgfpathlineto{\pgfqpoint{1.388402in}{1.574436in}}%
\pgfpathclose%
\pgfusepath{fill}%
\end{pgfscope}%
\begin{pgfscope}%
\pgfpathrectangle{\pgfqpoint{0.329460in}{0.284240in}}{\pgfqpoint{1.989680in}{1.989680in}}%
\pgfusepath{clip}%
\pgfsetbuttcap%
\pgfsetroundjoin%
\definecolor{currentfill}{rgb}{0.344074,0.780029,0.397381}%
\pgfsetfillcolor{currentfill}%
\pgfsetlinewidth{0.000000pt}%
\definecolor{currentstroke}{rgb}{0.000000,0.000000,0.000000}%
\pgfsetstrokecolor{currentstroke}%
\pgfsetdash{}{0pt}%
\pgfpathmoveto{\pgfqpoint{1.329025in}{1.462628in}}%
\pgfpathlineto{\pgfqpoint{1.328630in}{1.455079in}}%
\pgfpathlineto{\pgfqpoint{1.328235in}{1.447459in}}%
\pgfpathlineto{\pgfqpoint{1.327840in}{1.439770in}}%
\pgfpathlineto{\pgfqpoint{1.327445in}{1.432015in}}%
\pgfpathlineto{\pgfqpoint{1.320811in}{1.432433in}}%
\pgfpathlineto{\pgfqpoint{1.314208in}{1.432954in}}%
\pgfpathlineto{\pgfqpoint{1.307644in}{1.433577in}}%
\pgfpathlineto{\pgfqpoint{1.301126in}{1.434300in}}%
\pgfpathlineto{\pgfqpoint{1.301958in}{1.442015in}}%
\pgfpathlineto{\pgfqpoint{1.302790in}{1.449663in}}%
\pgfpathlineto{\pgfqpoint{1.303623in}{1.457242in}}%
\pgfpathlineto{\pgfqpoint{1.304456in}{1.464751in}}%
\pgfpathlineto{\pgfqpoint{1.310541in}{1.464079in}}%
\pgfpathlineto{\pgfqpoint{1.316668in}{1.463501in}}%
\pgfpathlineto{\pgfqpoint{1.322832in}{1.463017in}}%
\pgfpathlineto{\pgfqpoint{1.329025in}{1.462628in}}%
\pgfpathclose%
\pgfusepath{fill}%
\end{pgfscope}%
\begin{pgfscope}%
\pgfpathrectangle{\pgfqpoint{0.329460in}{0.284240in}}{\pgfqpoint{1.989680in}{1.989680in}}%
\pgfusepath{clip}%
\pgfsetbuttcap%
\pgfsetroundjoin%
\definecolor{currentfill}{rgb}{0.699415,0.867117,0.175971}%
\pgfsetfillcolor{currentfill}%
\pgfsetlinewidth{0.000000pt}%
\definecolor{currentstroke}{rgb}{0.000000,0.000000,0.000000}%
\pgfsetstrokecolor{currentstroke}%
\pgfsetdash{}{0pt}%
\pgfpathmoveto{\pgfqpoint{1.368969in}{1.596512in}}%
\pgfpathlineto{\pgfqpoint{1.369464in}{1.590711in}}%
\pgfpathlineto{\pgfqpoint{1.369959in}{1.584810in}}%
\pgfpathlineto{\pgfqpoint{1.370454in}{1.578810in}}%
\pgfpathlineto{\pgfqpoint{1.370949in}{1.572712in}}%
\pgfpathlineto{\pgfqpoint{1.366523in}{1.572447in}}%
\pgfpathlineto{\pgfqpoint{1.362082in}{1.572249in}}%
\pgfpathlineto{\pgfqpoint{1.357630in}{1.572118in}}%
\pgfpathlineto{\pgfqpoint{1.353170in}{1.572054in}}%
\pgfpathlineto{\pgfqpoint{1.353121in}{1.578169in}}%
\pgfpathlineto{\pgfqpoint{1.353071in}{1.584186in}}%
\pgfpathlineto{\pgfqpoint{1.353021in}{1.590104in}}%
\pgfpathlineto{\pgfqpoint{1.352972in}{1.595922in}}%
\pgfpathlineto{\pgfqpoint{1.356984in}{1.595979in}}%
\pgfpathlineto{\pgfqpoint{1.360990in}{1.596096in}}%
\pgfpathlineto{\pgfqpoint{1.364987in}{1.596274in}}%
\pgfpathlineto{\pgfqpoint{1.368969in}{1.596512in}}%
\pgfpathclose%
\pgfusepath{fill}%
\end{pgfscope}%
\begin{pgfscope}%
\pgfpathrectangle{\pgfqpoint{0.329460in}{0.284240in}}{\pgfqpoint{1.989680in}{1.989680in}}%
\pgfusepath{clip}%
\pgfsetbuttcap%
\pgfsetroundjoin%
\definecolor{currentfill}{rgb}{0.412913,0.803041,0.357269}%
\pgfsetfillcolor{currentfill}%
\pgfsetlinewidth{0.000000pt}%
\definecolor{currentstroke}{rgb}{0.000000,0.000000,0.000000}%
\pgfsetstrokecolor{currentstroke}%
\pgfsetdash{}{0pt}%
\pgfpathmoveto{\pgfqpoint{1.399571in}{1.494672in}}%
\pgfpathlineto{\pgfqpoint{1.400501in}{1.487475in}}%
\pgfpathlineto{\pgfqpoint{1.401430in}{1.480201in}}%
\pgfpathlineto{\pgfqpoint{1.402359in}{1.472851in}}%
\pgfpathlineto{\pgfqpoint{1.403288in}{1.465427in}}%
\pgfpathlineto{\pgfqpoint{1.397246in}{1.464672in}}%
\pgfpathlineto{\pgfqpoint{1.391155in}{1.464010in}}%
\pgfpathlineto{\pgfqpoint{1.385024in}{1.463442in}}%
\pgfpathlineto{\pgfqpoint{1.378857in}{1.462969in}}%
\pgfpathlineto{\pgfqpoint{1.378363in}{1.470440in}}%
\pgfpathlineto{\pgfqpoint{1.377870in}{1.477836in}}%
\pgfpathlineto{\pgfqpoint{1.377376in}{1.485157in}}%
\pgfpathlineto{\pgfqpoint{1.376882in}{1.492401in}}%
\pgfpathlineto{\pgfqpoint{1.382609in}{1.492838in}}%
\pgfpathlineto{\pgfqpoint{1.388304in}{1.493363in}}%
\pgfpathlineto{\pgfqpoint{1.393960in}{1.493974in}}%
\pgfpathlineto{\pgfqpoint{1.399571in}{1.494672in}}%
\pgfpathclose%
\pgfusepath{fill}%
\end{pgfscope}%
\begin{pgfscope}%
\pgfpathrectangle{\pgfqpoint{0.329460in}{0.284240in}}{\pgfqpoint{1.989680in}{1.989680in}}%
\pgfusepath{clip}%
\pgfsetbuttcap%
\pgfsetroundjoin%
\definecolor{currentfill}{rgb}{0.699415,0.867117,0.175971}%
\pgfsetfillcolor{currentfill}%
\pgfsetlinewidth{0.000000pt}%
\definecolor{currentstroke}{rgb}{0.000000,0.000000,0.000000}%
\pgfsetstrokecolor{currentstroke}%
\pgfsetdash{}{0pt}%
\pgfpathmoveto{\pgfqpoint{1.352972in}{1.595922in}}%
\pgfpathlineto{\pgfqpoint{1.353021in}{1.590104in}}%
\pgfpathlineto{\pgfqpoint{1.353071in}{1.584186in}}%
\pgfpathlineto{\pgfqpoint{1.353121in}{1.578169in}}%
\pgfpathlineto{\pgfqpoint{1.353170in}{1.572054in}}%
\pgfpathlineto{\pgfqpoint{1.348709in}{1.572058in}}%
\pgfpathlineto{\pgfqpoint{1.344250in}{1.572129in}}%
\pgfpathlineto{\pgfqpoint{1.339799in}{1.572267in}}%
\pgfpathlineto{\pgfqpoint{1.335359in}{1.572473in}}%
\pgfpathlineto{\pgfqpoint{1.335756in}{1.578577in}}%
\pgfpathlineto{\pgfqpoint{1.336152in}{1.584583in}}%
\pgfpathlineto{\pgfqpoint{1.336549in}{1.590490in}}%
\pgfpathlineto{\pgfqpoint{1.336945in}{1.596298in}}%
\pgfpathlineto{\pgfqpoint{1.340940in}{1.596113in}}%
\pgfpathlineto{\pgfqpoint{1.344946in}{1.595989in}}%
\pgfpathlineto{\pgfqpoint{1.348958in}{1.595925in}}%
\pgfpathlineto{\pgfqpoint{1.352972in}{1.595922in}}%
\pgfpathclose%
\pgfusepath{fill}%
\end{pgfscope}%
\begin{pgfscope}%
\pgfpathrectangle{\pgfqpoint{0.329460in}{0.284240in}}{\pgfqpoint{1.989680in}{1.989680in}}%
\pgfusepath{clip}%
\pgfsetbuttcap%
\pgfsetroundjoin%
\definecolor{currentfill}{rgb}{0.281477,0.755203,0.432552}%
\pgfsetfillcolor{currentfill}%
\pgfsetlinewidth{0.000000pt}%
\definecolor{currentstroke}{rgb}{0.000000,0.000000,0.000000}%
\pgfsetstrokecolor{currentstroke}%
\pgfsetdash{}{0pt}%
\pgfpathmoveto{\pgfqpoint{1.380829in}{1.432382in}}%
\pgfpathlineto{\pgfqpoint{1.381321in}{1.424567in}}%
\pgfpathlineto{\pgfqpoint{1.381814in}{1.416690in}}%
\pgfpathlineto{\pgfqpoint{1.382306in}{1.408751in}}%
\pgfpathlineto{\pgfqpoint{1.382799in}{1.400753in}}%
\pgfpathlineto{\pgfqpoint{1.375720in}{1.400316in}}%
\pgfpathlineto{\pgfqpoint{1.368616in}{1.399990in}}%
\pgfpathlineto{\pgfqpoint{1.361493in}{1.399775in}}%
\pgfpathlineto{\pgfqpoint{1.354360in}{1.399670in}}%
\pgfpathlineto{\pgfqpoint{1.354310in}{1.407686in}}%
\pgfpathlineto{\pgfqpoint{1.354261in}{1.415643in}}%
\pgfpathlineto{\pgfqpoint{1.354211in}{1.423539in}}%
\pgfpathlineto{\pgfqpoint{1.354162in}{1.431371in}}%
\pgfpathlineto{\pgfqpoint{1.360851in}{1.431469in}}%
\pgfpathlineto{\pgfqpoint{1.367530in}{1.431670in}}%
\pgfpathlineto{\pgfqpoint{1.374191in}{1.431974in}}%
\pgfpathlineto{\pgfqpoint{1.380829in}{1.432382in}}%
\pgfpathclose%
\pgfusepath{fill}%
\end{pgfscope}%
\begin{pgfscope}%
\pgfpathrectangle{\pgfqpoint{0.329460in}{0.284240in}}{\pgfqpoint{1.989680in}{1.989680in}}%
\pgfusepath{clip}%
\pgfsetbuttcap%
\pgfsetroundjoin%
\definecolor{currentfill}{rgb}{0.636902,0.856542,0.216620}%
\pgfsetfillcolor{currentfill}%
\pgfsetlinewidth{0.000000pt}%
\definecolor{currentstroke}{rgb}{0.000000,0.000000,0.000000}%
\pgfsetstrokecolor{currentstroke}%
\pgfsetdash{}{0pt}%
\pgfpathmoveto{\pgfqpoint{1.335359in}{1.572473in}}%
\pgfpathlineto{\pgfqpoint{1.334963in}{1.566273in}}%
\pgfpathlineto{\pgfqpoint{1.334566in}{1.559977in}}%
\pgfpathlineto{\pgfqpoint{1.334170in}{1.553587in}}%
\pgfpathlineto{\pgfqpoint{1.333774in}{1.547105in}}%
\pgfpathlineto{\pgfqpoint{1.328907in}{1.547406in}}%
\pgfpathlineto{\pgfqpoint{1.324064in}{1.547781in}}%
\pgfpathlineto{\pgfqpoint{1.319249in}{1.548229in}}%
\pgfpathlineto{\pgfqpoint{1.314466in}{1.548750in}}%
\pgfpathlineto{\pgfqpoint{1.315302in}{1.555193in}}%
\pgfpathlineto{\pgfqpoint{1.316137in}{1.561543in}}%
\pgfpathlineto{\pgfqpoint{1.316973in}{1.567800in}}%
\pgfpathlineto{\pgfqpoint{1.317809in}{1.573962in}}%
\pgfpathlineto{\pgfqpoint{1.322156in}{1.573490in}}%
\pgfpathlineto{\pgfqpoint{1.326533in}{1.573085in}}%
\pgfpathlineto{\pgfqpoint{1.330936in}{1.572746in}}%
\pgfpathlineto{\pgfqpoint{1.335359in}{1.572473in}}%
\pgfpathclose%
\pgfusepath{fill}%
\end{pgfscope}%
\begin{pgfscope}%
\pgfpathrectangle{\pgfqpoint{0.329460in}{0.284240in}}{\pgfqpoint{1.989680in}{1.989680in}}%
\pgfusepath{clip}%
\pgfsetbuttcap%
\pgfsetroundjoin%
\definecolor{currentfill}{rgb}{0.281477,0.755203,0.432552}%
\pgfsetfillcolor{currentfill}%
\pgfsetlinewidth{0.000000pt}%
\definecolor{currentstroke}{rgb}{0.000000,0.000000,0.000000}%
\pgfsetstrokecolor{currentstroke}%
\pgfsetdash{}{0pt}%
\pgfpathmoveto{\pgfqpoint{1.354162in}{1.431371in}}%
\pgfpathlineto{\pgfqpoint{1.354211in}{1.423539in}}%
\pgfpathlineto{\pgfqpoint{1.354261in}{1.415643in}}%
\pgfpathlineto{\pgfqpoint{1.354310in}{1.407686in}}%
\pgfpathlineto{\pgfqpoint{1.354360in}{1.399670in}}%
\pgfpathlineto{\pgfqpoint{1.347223in}{1.399676in}}%
\pgfpathlineto{\pgfqpoint{1.340090in}{1.399793in}}%
\pgfpathlineto{\pgfqpoint{1.332969in}{1.400021in}}%
\pgfpathlineto{\pgfqpoint{1.325867in}{1.400360in}}%
\pgfpathlineto{\pgfqpoint{1.326261in}{1.408364in}}%
\pgfpathlineto{\pgfqpoint{1.326656in}{1.416310in}}%
\pgfpathlineto{\pgfqpoint{1.327050in}{1.424194in}}%
\pgfpathlineto{\pgfqpoint{1.327445in}{1.432015in}}%
\pgfpathlineto{\pgfqpoint{1.334104in}{1.431699in}}%
\pgfpathlineto{\pgfqpoint{1.340782in}{1.431486in}}%
\pgfpathlineto{\pgfqpoint{1.347470in}{1.431377in}}%
\pgfpathlineto{\pgfqpoint{1.354162in}{1.431371in}}%
\pgfpathclose%
\pgfusepath{fill}%
\end{pgfscope}%
\begin{pgfscope}%
\pgfpathrectangle{\pgfqpoint{0.329460in}{0.284240in}}{\pgfqpoint{1.989680in}{1.989680in}}%
\pgfusepath{clip}%
\pgfsetbuttcap%
\pgfsetroundjoin%
\definecolor{currentfill}{rgb}{0.565498,0.842430,0.262877}%
\pgfsetfillcolor{currentfill}%
\pgfsetlinewidth{0.000000pt}%
\definecolor{currentstroke}{rgb}{0.000000,0.000000,0.000000}%
\pgfsetstrokecolor{currentstroke}%
\pgfsetdash{}{0pt}%
\pgfpathmoveto{\pgfqpoint{1.392128in}{1.549273in}}%
\pgfpathlineto{\pgfqpoint{1.393059in}{1.542752in}}%
\pgfpathlineto{\pgfqpoint{1.393990in}{1.536140in}}%
\pgfpathlineto{\pgfqpoint{1.394921in}{1.529441in}}%
\pgfpathlineto{\pgfqpoint{1.395851in}{1.522654in}}%
\pgfpathlineto{\pgfqpoint{1.390671in}{1.522013in}}%
\pgfpathlineto{\pgfqpoint{1.385450in}{1.521451in}}%
\pgfpathlineto{\pgfqpoint{1.380193in}{1.520969in}}%
\pgfpathlineto{\pgfqpoint{1.374906in}{1.520567in}}%
\pgfpathlineto{\pgfqpoint{1.374412in}{1.527400in}}%
\pgfpathlineto{\pgfqpoint{1.373917in}{1.534145in}}%
\pgfpathlineto{\pgfqpoint{1.373423in}{1.540802in}}%
\pgfpathlineto{\pgfqpoint{1.372928in}{1.547369in}}%
\pgfpathlineto{\pgfqpoint{1.377774in}{1.547736in}}%
\pgfpathlineto{\pgfqpoint{1.382593in}{1.548176in}}%
\pgfpathlineto{\pgfqpoint{1.387379in}{1.548688in}}%
\pgfpathlineto{\pgfqpoint{1.392128in}{1.549273in}}%
\pgfpathclose%
\pgfusepath{fill}%
\end{pgfscope}%
\begin{pgfscope}%
\pgfpathrectangle{\pgfqpoint{0.329460in}{0.284240in}}{\pgfqpoint{1.989680in}{1.989680in}}%
\pgfusepath{clip}%
\pgfsetbuttcap%
\pgfsetroundjoin%
\definecolor{currentfill}{rgb}{0.487026,0.823929,0.312321}%
\pgfsetfillcolor{currentfill}%
\pgfsetlinewidth{0.000000pt}%
\definecolor{currentstroke}{rgb}{0.000000,0.000000,0.000000}%
\pgfsetstrokecolor{currentstroke}%
\pgfsetdash{}{0pt}%
\pgfpathmoveto{\pgfqpoint{1.395851in}{1.522654in}}%
\pgfpathlineto{\pgfqpoint{1.396782in}{1.515783in}}%
\pgfpathlineto{\pgfqpoint{1.397712in}{1.508827in}}%
\pgfpathlineto{\pgfqpoint{1.398642in}{1.501790in}}%
\pgfpathlineto{\pgfqpoint{1.399571in}{1.494672in}}%
\pgfpathlineto{\pgfqpoint{1.393960in}{1.493974in}}%
\pgfpathlineto{\pgfqpoint{1.388304in}{1.493363in}}%
\pgfpathlineto{\pgfqpoint{1.382609in}{1.492838in}}%
\pgfpathlineto{\pgfqpoint{1.376882in}{1.492401in}}%
\pgfpathlineto{\pgfqpoint{1.376388in}{1.499565in}}%
\pgfpathlineto{\pgfqpoint{1.375894in}{1.506649in}}%
\pgfpathlineto{\pgfqpoint{1.375400in}{1.513650in}}%
\pgfpathlineto{\pgfqpoint{1.374906in}{1.520567in}}%
\pgfpathlineto{\pgfqpoint{1.380193in}{1.520969in}}%
\pgfpathlineto{\pgfqpoint{1.385450in}{1.521451in}}%
\pgfpathlineto{\pgfqpoint{1.390671in}{1.522013in}}%
\pgfpathlineto{\pgfqpoint{1.395851in}{1.522654in}}%
\pgfpathclose%
\pgfusepath{fill}%
\end{pgfscope}%
\begin{pgfscope}%
\pgfpathrectangle{\pgfqpoint{0.329460in}{0.284240in}}{\pgfqpoint{1.989680in}{1.989680in}}%
\pgfusepath{clip}%
\pgfsetbuttcap%
\pgfsetroundjoin%
\definecolor{currentfill}{rgb}{0.412913,0.803041,0.357269}%
\pgfsetfillcolor{currentfill}%
\pgfsetlinewidth{0.000000pt}%
\definecolor{currentstroke}{rgb}{0.000000,0.000000,0.000000}%
\pgfsetstrokecolor{currentstroke}%
\pgfsetdash{}{0pt}%
\pgfpathmoveto{\pgfqpoint{1.330607in}{1.492086in}}%
\pgfpathlineto{\pgfqpoint{1.330211in}{1.484836in}}%
\pgfpathlineto{\pgfqpoint{1.329816in}{1.477509in}}%
\pgfpathlineto{\pgfqpoint{1.329420in}{1.470105in}}%
\pgfpathlineto{\pgfqpoint{1.329025in}{1.462628in}}%
\pgfpathlineto{\pgfqpoint{1.322832in}{1.463017in}}%
\pgfpathlineto{\pgfqpoint{1.316668in}{1.463501in}}%
\pgfpathlineto{\pgfqpoint{1.310541in}{1.464079in}}%
\pgfpathlineto{\pgfqpoint{1.304456in}{1.464751in}}%
\pgfpathlineto{\pgfqpoint{1.305289in}{1.472188in}}%
\pgfpathlineto{\pgfqpoint{1.306122in}{1.479551in}}%
\pgfpathlineto{\pgfqpoint{1.306956in}{1.486838in}}%
\pgfpathlineto{\pgfqpoint{1.307789in}{1.494048in}}%
\pgfpathlineto{\pgfqpoint{1.313441in}{1.493426in}}%
\pgfpathlineto{\pgfqpoint{1.319131in}{1.492892in}}%
\pgfpathlineto{\pgfqpoint{1.324855in}{1.492445in}}%
\pgfpathlineto{\pgfqpoint{1.330607in}{1.492086in}}%
\pgfpathclose%
\pgfusepath{fill}%
\end{pgfscope}%
\begin{pgfscope}%
\pgfpathrectangle{\pgfqpoint{0.329460in}{0.284240in}}{\pgfqpoint{1.989680in}{1.989680in}}%
\pgfusepath{clip}%
\pgfsetbuttcap%
\pgfsetroundjoin%
\definecolor{currentfill}{rgb}{0.201239,0.383670,0.554294}%
\pgfsetfillcolor{currentfill}%
\pgfsetlinewidth{0.000000pt}%
\definecolor{currentstroke}{rgb}{0.000000,0.000000,0.000000}%
\pgfsetstrokecolor{currentstroke}%
\pgfsetdash{}{0pt}%
\pgfpathmoveto{\pgfqpoint{0.832250in}{0.998384in}}%
\pgfpathlineto{\pgfqpoint{0.829111in}{1.009125in}}%
\pgfpathlineto{\pgfqpoint{0.825955in}{1.020307in}}%
\pgfpathlineto{\pgfqpoint{0.822782in}{1.031936in}}%
\pgfpathlineto{\pgfqpoint{0.819590in}{1.044021in}}%
\pgfpathlineto{\pgfqpoint{0.806444in}{1.052870in}}%
\pgfpathlineto{\pgfqpoint{0.793895in}{1.061918in}}%
\pgfpathlineto{\pgfqpoint{0.781955in}{1.071154in}}%
\pgfpathlineto{\pgfqpoint{0.770634in}{1.080567in}}%
\pgfpathlineto{\pgfqpoint{0.774100in}{1.068331in}}%
\pgfpathlineto{\pgfqpoint{0.777547in}{1.056550in}}%
\pgfpathlineto{\pgfqpoint{0.780975in}{1.045214in}}%
\pgfpathlineto{\pgfqpoint{0.784384in}{1.034316in}}%
\pgfpathlineto{\pgfqpoint{0.795456in}{1.025060in}}%
\pgfpathlineto{\pgfqpoint{0.807131in}{1.015979in}}%
\pgfpathlineto{\pgfqpoint{0.819400in}{1.007084in}}%
\pgfpathlineto{\pgfqpoint{0.832250in}{0.998384in}}%
\pgfpathclose%
\pgfusepath{fill}%
\end{pgfscope}%
\begin{pgfscope}%
\pgfpathrectangle{\pgfqpoint{0.329460in}{0.284240in}}{\pgfqpoint{1.989680in}{1.989680in}}%
\pgfusepath{clip}%
\pgfsetbuttcap%
\pgfsetroundjoin%
\definecolor{currentfill}{rgb}{0.565498,0.842430,0.262877}%
\pgfsetfillcolor{currentfill}%
\pgfsetlinewidth{0.000000pt}%
\definecolor{currentstroke}{rgb}{0.000000,0.000000,0.000000}%
\pgfsetstrokecolor{currentstroke}%
\pgfsetdash{}{0pt}%
\pgfpathmoveto{\pgfqpoint{1.333774in}{1.547105in}}%
\pgfpathlineto{\pgfqpoint{1.333378in}{1.540531in}}%
\pgfpathlineto{\pgfqpoint{1.332982in}{1.533868in}}%
\pgfpathlineto{\pgfqpoint{1.332586in}{1.527117in}}%
\pgfpathlineto{\pgfqpoint{1.332190in}{1.520278in}}%
\pgfpathlineto{\pgfqpoint{1.326881in}{1.520608in}}%
\pgfpathlineto{\pgfqpoint{1.321597in}{1.521019in}}%
\pgfpathlineto{\pgfqpoint{1.316344in}{1.521510in}}%
\pgfpathlineto{\pgfqpoint{1.311127in}{1.522081in}}%
\pgfpathlineto{\pgfqpoint{1.311961in}{1.528880in}}%
\pgfpathlineto{\pgfqpoint{1.312796in}{1.535592in}}%
\pgfpathlineto{\pgfqpoint{1.313631in}{1.542216in}}%
\pgfpathlineto{\pgfqpoint{1.314466in}{1.548750in}}%
\pgfpathlineto{\pgfqpoint{1.319249in}{1.548229in}}%
\pgfpathlineto{\pgfqpoint{1.324064in}{1.547781in}}%
\pgfpathlineto{\pgfqpoint{1.328907in}{1.547406in}}%
\pgfpathlineto{\pgfqpoint{1.333774in}{1.547105in}}%
\pgfpathclose%
\pgfusepath{fill}%
\end{pgfscope}%
\begin{pgfscope}%
\pgfpathrectangle{\pgfqpoint{0.329460in}{0.284240in}}{\pgfqpoint{1.989680in}{1.989680in}}%
\pgfusepath{clip}%
\pgfsetbuttcap%
\pgfsetroundjoin%
\definecolor{currentfill}{rgb}{0.487026,0.823929,0.312321}%
\pgfsetfillcolor{currentfill}%
\pgfsetlinewidth{0.000000pt}%
\definecolor{currentstroke}{rgb}{0.000000,0.000000,0.000000}%
\pgfsetstrokecolor{currentstroke}%
\pgfsetdash{}{0pt}%
\pgfpathmoveto{\pgfqpoint{1.332190in}{1.520278in}}%
\pgfpathlineto{\pgfqpoint{1.331794in}{1.513354in}}%
\pgfpathlineto{\pgfqpoint{1.331398in}{1.506346in}}%
\pgfpathlineto{\pgfqpoint{1.331002in}{1.499256in}}%
\pgfpathlineto{\pgfqpoint{1.330607in}{1.492086in}}%
\pgfpathlineto{\pgfqpoint{1.324855in}{1.492445in}}%
\pgfpathlineto{\pgfqpoint{1.319131in}{1.492892in}}%
\pgfpathlineto{\pgfqpoint{1.313441in}{1.493426in}}%
\pgfpathlineto{\pgfqpoint{1.307789in}{1.494048in}}%
\pgfpathlineto{\pgfqpoint{1.308623in}{1.501178in}}%
\pgfpathlineto{\pgfqpoint{1.309458in}{1.508228in}}%
\pgfpathlineto{\pgfqpoint{1.310292in}{1.515196in}}%
\pgfpathlineto{\pgfqpoint{1.311127in}{1.522081in}}%
\pgfpathlineto{\pgfqpoint{1.316344in}{1.521510in}}%
\pgfpathlineto{\pgfqpoint{1.321597in}{1.521019in}}%
\pgfpathlineto{\pgfqpoint{1.326881in}{1.520608in}}%
\pgfpathlineto{\pgfqpoint{1.332190in}{1.520278in}}%
\pgfpathclose%
\pgfusepath{fill}%
\end{pgfscope}%
\begin{pgfscope}%
\pgfpathrectangle{\pgfqpoint{0.329460in}{0.284240in}}{\pgfqpoint{1.989680in}{1.989680in}}%
\pgfusepath{clip}%
\pgfsetbuttcap%
\pgfsetroundjoin%
\definecolor{currentfill}{rgb}{0.636902,0.856542,0.216620}%
\pgfsetfillcolor{currentfill}%
\pgfsetlinewidth{0.000000pt}%
\definecolor{currentstroke}{rgb}{0.000000,0.000000,0.000000}%
\pgfsetstrokecolor{currentstroke}%
\pgfsetdash{}{0pt}%
\pgfpathmoveto{\pgfqpoint{1.370949in}{1.572712in}}%
\pgfpathlineto{\pgfqpoint{1.371444in}{1.566518in}}%
\pgfpathlineto{\pgfqpoint{1.371939in}{1.560228in}}%
\pgfpathlineto{\pgfqpoint{1.372433in}{1.553845in}}%
\pgfpathlineto{\pgfqpoint{1.372928in}{1.547369in}}%
\pgfpathlineto{\pgfqpoint{1.368059in}{1.547076in}}%
\pgfpathlineto{\pgfqpoint{1.363173in}{1.546857in}}%
\pgfpathlineto{\pgfqpoint{1.358275in}{1.546712in}}%
\pgfpathlineto{\pgfqpoint{1.353369in}{1.546642in}}%
\pgfpathlineto{\pgfqpoint{1.353319in}{1.553135in}}%
\pgfpathlineto{\pgfqpoint{1.353270in}{1.559536in}}%
\pgfpathlineto{\pgfqpoint{1.353220in}{1.565843in}}%
\pgfpathlineto{\pgfqpoint{1.353170in}{1.572054in}}%
\pgfpathlineto{\pgfqpoint{1.357630in}{1.572118in}}%
\pgfpathlineto{\pgfqpoint{1.362082in}{1.572249in}}%
\pgfpathlineto{\pgfqpoint{1.366523in}{1.572447in}}%
\pgfpathlineto{\pgfqpoint{1.370949in}{1.572712in}}%
\pgfpathclose%
\pgfusepath{fill}%
\end{pgfscope}%
\begin{pgfscope}%
\pgfpathrectangle{\pgfqpoint{0.329460in}{0.284240in}}{\pgfqpoint{1.989680in}{1.989680in}}%
\pgfusepath{clip}%
\pgfsetbuttcap%
\pgfsetroundjoin%
\definecolor{currentfill}{rgb}{0.344074,0.780029,0.397381}%
\pgfsetfillcolor{currentfill}%
\pgfsetlinewidth{0.000000pt}%
\definecolor{currentstroke}{rgb}{0.000000,0.000000,0.000000}%
\pgfsetstrokecolor{currentstroke}%
\pgfsetdash{}{0pt}%
\pgfpathmoveto{\pgfqpoint{1.378857in}{1.462969in}}%
\pgfpathlineto{\pgfqpoint{1.379350in}{1.455426in}}%
\pgfpathlineto{\pgfqpoint{1.379843in}{1.447813in}}%
\pgfpathlineto{\pgfqpoint{1.380336in}{1.440131in}}%
\pgfpathlineto{\pgfqpoint{1.380829in}{1.432382in}}%
\pgfpathlineto{\pgfqpoint{1.374191in}{1.431974in}}%
\pgfpathlineto{\pgfqpoint{1.367530in}{1.431670in}}%
\pgfpathlineto{\pgfqpoint{1.360851in}{1.431469in}}%
\pgfpathlineto{\pgfqpoint{1.354162in}{1.431371in}}%
\pgfpathlineto{\pgfqpoint{1.354112in}{1.439138in}}%
\pgfpathlineto{\pgfqpoint{1.354063in}{1.446839in}}%
\pgfpathlineto{\pgfqpoint{1.354013in}{1.454470in}}%
\pgfpathlineto{\pgfqpoint{1.353964in}{1.462031in}}%
\pgfpathlineto{\pgfqpoint{1.360208in}{1.462122in}}%
\pgfpathlineto{\pgfqpoint{1.366442in}{1.462308in}}%
\pgfpathlineto{\pgfqpoint{1.372660in}{1.462591in}}%
\pgfpathlineto{\pgfqpoint{1.378857in}{1.462969in}}%
\pgfpathclose%
\pgfusepath{fill}%
\end{pgfscope}%
\begin{pgfscope}%
\pgfpathrectangle{\pgfqpoint{0.329460in}{0.284240in}}{\pgfqpoint{1.989680in}{1.989680in}}%
\pgfusepath{clip}%
\pgfsetbuttcap%
\pgfsetroundjoin%
\definecolor{currentfill}{rgb}{0.267004,0.004874,0.329415}%
\pgfsetfillcolor{currentfill}%
\pgfsetlinewidth{0.000000pt}%
\definecolor{currentstroke}{rgb}{0.000000,0.000000,0.000000}%
\pgfsetstrokecolor{currentstroke}%
\pgfsetdash{}{0pt}%
\pgfpathmoveto{\pgfqpoint{1.428201in}{0.748268in}}%
\pgfpathlineto{\pgfqpoint{1.428717in}{0.746602in}}%
\pgfpathlineto{\pgfqpoint{1.429235in}{0.745170in}}%
\pgfpathlineto{\pgfqpoint{1.429754in}{0.743978in}}%
\pgfpathlineto{\pgfqpoint{1.430275in}{0.743030in}}%
\pgfpathlineto{\pgfqpoint{1.412572in}{0.741821in}}%
\pgfpathlineto{\pgfqpoint{1.394799in}{0.740916in}}%
\pgfpathlineto{\pgfqpoint{1.376977in}{0.740319in}}%
\pgfpathlineto{\pgfqpoint{1.359126in}{0.740028in}}%
\pgfpathlineto{\pgfqpoint{1.359073in}{0.740996in}}%
\pgfpathlineto{\pgfqpoint{1.359021in}{0.742209in}}%
\pgfpathlineto{\pgfqpoint{1.358969in}{0.743661in}}%
\pgfpathlineto{\pgfqpoint{1.358917in}{0.745347in}}%
\pgfpathlineto{\pgfqpoint{1.376300in}{0.745630in}}%
\pgfpathlineto{\pgfqpoint{1.393655in}{0.746211in}}%
\pgfpathlineto{\pgfqpoint{1.410962in}{0.747091in}}%
\pgfpathlineto{\pgfqpoint{1.428201in}{0.748268in}}%
\pgfpathclose%
\pgfusepath{fill}%
\end{pgfscope}%
\begin{pgfscope}%
\pgfpathrectangle{\pgfqpoint{0.329460in}{0.284240in}}{\pgfqpoint{1.989680in}{1.989680in}}%
\pgfusepath{clip}%
\pgfsetbuttcap%
\pgfsetroundjoin%
\definecolor{currentfill}{rgb}{0.636902,0.856542,0.216620}%
\pgfsetfillcolor{currentfill}%
\pgfsetlinewidth{0.000000pt}%
\definecolor{currentstroke}{rgb}{0.000000,0.000000,0.000000}%
\pgfsetstrokecolor{currentstroke}%
\pgfsetdash{}{0pt}%
\pgfpathmoveto{\pgfqpoint{1.353170in}{1.572054in}}%
\pgfpathlineto{\pgfqpoint{1.353220in}{1.565843in}}%
\pgfpathlineto{\pgfqpoint{1.353270in}{1.559536in}}%
\pgfpathlineto{\pgfqpoint{1.353319in}{1.553135in}}%
\pgfpathlineto{\pgfqpoint{1.353369in}{1.546642in}}%
\pgfpathlineto{\pgfqpoint{1.348461in}{1.546646in}}%
\pgfpathlineto{\pgfqpoint{1.343556in}{1.546725in}}%
\pgfpathlineto{\pgfqpoint{1.338658in}{1.546878in}}%
\pgfpathlineto{\pgfqpoint{1.333774in}{1.547105in}}%
\pgfpathlineto{\pgfqpoint{1.334170in}{1.553587in}}%
\pgfpathlineto{\pgfqpoint{1.334566in}{1.559977in}}%
\pgfpathlineto{\pgfqpoint{1.334963in}{1.566273in}}%
\pgfpathlineto{\pgfqpoint{1.335359in}{1.572473in}}%
\pgfpathlineto{\pgfqpoint{1.339799in}{1.572267in}}%
\pgfpathlineto{\pgfqpoint{1.344250in}{1.572129in}}%
\pgfpathlineto{\pgfqpoint{1.348709in}{1.572058in}}%
\pgfpathlineto{\pgfqpoint{1.353170in}{1.572054in}}%
\pgfpathclose%
\pgfusepath{fill}%
\end{pgfscope}%
\begin{pgfscope}%
\pgfpathrectangle{\pgfqpoint{0.329460in}{0.284240in}}{\pgfqpoint{1.989680in}{1.989680in}}%
\pgfusepath{clip}%
\pgfsetbuttcap%
\pgfsetroundjoin%
\definecolor{currentfill}{rgb}{0.344074,0.780029,0.397381}%
\pgfsetfillcolor{currentfill}%
\pgfsetlinewidth{0.000000pt}%
\definecolor{currentstroke}{rgb}{0.000000,0.000000,0.000000}%
\pgfsetstrokecolor{currentstroke}%
\pgfsetdash{}{0pt}%
\pgfpathmoveto{\pgfqpoint{1.353964in}{1.462031in}}%
\pgfpathlineto{\pgfqpoint{1.354013in}{1.454470in}}%
\pgfpathlineto{\pgfqpoint{1.354063in}{1.446839in}}%
\pgfpathlineto{\pgfqpoint{1.354112in}{1.439138in}}%
\pgfpathlineto{\pgfqpoint{1.354162in}{1.431371in}}%
\pgfpathlineto{\pgfqpoint{1.347470in}{1.431377in}}%
\pgfpathlineto{\pgfqpoint{1.340782in}{1.431486in}}%
\pgfpathlineto{\pgfqpoint{1.334104in}{1.431699in}}%
\pgfpathlineto{\pgfqpoint{1.327445in}{1.432015in}}%
\pgfpathlineto{\pgfqpoint{1.327840in}{1.439770in}}%
\pgfpathlineto{\pgfqpoint{1.328235in}{1.447459in}}%
\pgfpathlineto{\pgfqpoint{1.328630in}{1.455079in}}%
\pgfpathlineto{\pgfqpoint{1.329025in}{1.462628in}}%
\pgfpathlineto{\pgfqpoint{1.335241in}{1.462335in}}%
\pgfpathlineto{\pgfqpoint{1.341474in}{1.462138in}}%
\pgfpathlineto{\pgfqpoint{1.347717in}{1.462036in}}%
\pgfpathlineto{\pgfqpoint{1.353964in}{1.462031in}}%
\pgfpathclose%
\pgfusepath{fill}%
\end{pgfscope}%
\begin{pgfscope}%
\pgfpathrectangle{\pgfqpoint{0.329460in}{0.284240in}}{\pgfqpoint{1.989680in}{1.989680in}}%
\pgfusepath{clip}%
\pgfsetbuttcap%
\pgfsetroundjoin%
\definecolor{currentfill}{rgb}{0.277941,0.056324,0.381191}%
\pgfsetfillcolor{currentfill}%
\pgfsetlinewidth{0.000000pt}%
\definecolor{currentstroke}{rgb}{0.000000,0.000000,0.000000}%
\pgfsetstrokecolor{currentstroke}%
\pgfsetdash{}{0pt}%
\pgfpathmoveto{\pgfqpoint{1.645014in}{0.784189in}}%
\pgfpathlineto{\pgfqpoint{1.646891in}{0.785901in}}%
\pgfpathlineto{\pgfqpoint{1.648775in}{0.787907in}}%
\pgfpathlineto{\pgfqpoint{1.650664in}{0.790211in}}%
\pgfpathlineto{\pgfqpoint{1.652561in}{0.792819in}}%
\pgfpathlineto{\pgfqpoint{1.635976in}{0.787807in}}%
\pgfpathlineto{\pgfqpoint{1.619071in}{0.783078in}}%
\pgfpathlineto{\pgfqpoint{1.601864in}{0.778639in}}%
\pgfpathlineto{\pgfqpoint{1.584373in}{0.774493in}}%
\pgfpathlineto{\pgfqpoint{1.582902in}{0.771994in}}%
\pgfpathlineto{\pgfqpoint{1.581436in}{0.769799in}}%
\pgfpathlineto{\pgfqpoint{1.579975in}{0.767903in}}%
\pgfpathlineto{\pgfqpoint{1.578519in}{0.766301in}}%
\pgfpathlineto{\pgfqpoint{1.595574in}{0.770347in}}%
\pgfpathlineto{\pgfqpoint{1.612353in}{0.774681in}}%
\pgfpathlineto{\pgfqpoint{1.628840in}{0.779297in}}%
\pgfpathlineto{\pgfqpoint{1.645014in}{0.784189in}}%
\pgfpathclose%
\pgfusepath{fill}%
\end{pgfscope}%
\begin{pgfscope}%
\pgfpathrectangle{\pgfqpoint{0.329460in}{0.284240in}}{\pgfqpoint{1.989680in}{1.989680in}}%
\pgfusepath{clip}%
\pgfsetbuttcap%
\pgfsetroundjoin%
\definecolor{currentfill}{rgb}{0.267004,0.004874,0.329415}%
\pgfsetfillcolor{currentfill}%
\pgfsetlinewidth{0.000000pt}%
\definecolor{currentstroke}{rgb}{0.000000,0.000000,0.000000}%
\pgfsetstrokecolor{currentstroke}%
\pgfsetdash{}{0pt}%
\pgfpathmoveto{\pgfqpoint{1.358917in}{0.745347in}}%
\pgfpathlineto{\pgfqpoint{1.358969in}{0.743661in}}%
\pgfpathlineto{\pgfqpoint{1.359021in}{0.742209in}}%
\pgfpathlineto{\pgfqpoint{1.359073in}{0.740996in}}%
\pgfpathlineto{\pgfqpoint{1.359126in}{0.740028in}}%
\pgfpathlineto{\pgfqpoint{1.341265in}{0.740045in}}%
\pgfpathlineto{\pgfqpoint{1.323416in}{0.740370in}}%
\pgfpathlineto{\pgfqpoint{1.305599in}{0.741002in}}%
\pgfpathlineto{\pgfqpoint{1.287833in}{0.741940in}}%
\pgfpathlineto{\pgfqpoint{1.288250in}{0.742895in}}%
\pgfpathlineto{\pgfqpoint{1.288666in}{0.744095in}}%
\pgfpathlineto{\pgfqpoint{1.289081in}{0.745534in}}%
\pgfpathlineto{\pgfqpoint{1.289494in}{0.747207in}}%
\pgfpathlineto{\pgfqpoint{1.306795in}{0.746294in}}%
\pgfpathlineto{\pgfqpoint{1.324145in}{0.745680in}}%
\pgfpathlineto{\pgfqpoint{1.341526in}{0.745364in}}%
\pgfpathlineto{\pgfqpoint{1.358917in}{0.745347in}}%
\pgfpathclose%
\pgfusepath{fill}%
\end{pgfscope}%
\begin{pgfscope}%
\pgfpathrectangle{\pgfqpoint{0.329460in}{0.284240in}}{\pgfqpoint{1.989680in}{1.989680in}}%
\pgfusepath{clip}%
\pgfsetbuttcap%
\pgfsetroundjoin%
\definecolor{currentfill}{rgb}{0.412913,0.803041,0.357269}%
\pgfsetfillcolor{currentfill}%
\pgfsetlinewidth{0.000000pt}%
\definecolor{currentstroke}{rgb}{0.000000,0.000000,0.000000}%
\pgfsetstrokecolor{currentstroke}%
\pgfsetdash{}{0pt}%
\pgfpathmoveto{\pgfqpoint{1.376882in}{1.492401in}}%
\pgfpathlineto{\pgfqpoint{1.377376in}{1.485157in}}%
\pgfpathlineto{\pgfqpoint{1.377870in}{1.477836in}}%
\pgfpathlineto{\pgfqpoint{1.378363in}{1.470440in}}%
\pgfpathlineto{\pgfqpoint{1.378857in}{1.462969in}}%
\pgfpathlineto{\pgfqpoint{1.372660in}{1.462591in}}%
\pgfpathlineto{\pgfqpoint{1.366442in}{1.462308in}}%
\pgfpathlineto{\pgfqpoint{1.360208in}{1.462122in}}%
\pgfpathlineto{\pgfqpoint{1.353964in}{1.462031in}}%
\pgfpathlineto{\pgfqpoint{1.353914in}{1.469519in}}%
\pgfpathlineto{\pgfqpoint{1.353865in}{1.476934in}}%
\pgfpathlineto{\pgfqpoint{1.353815in}{1.484272in}}%
\pgfpathlineto{\pgfqpoint{1.353766in}{1.491533in}}%
\pgfpathlineto{\pgfqpoint{1.359564in}{1.491617in}}%
\pgfpathlineto{\pgfqpoint{1.365353in}{1.491790in}}%
\pgfpathlineto{\pgfqpoint{1.371128in}{1.492051in}}%
\pgfpathlineto{\pgfqpoint{1.376882in}{1.492401in}}%
\pgfpathclose%
\pgfusepath{fill}%
\end{pgfscope}%
\begin{pgfscope}%
\pgfpathrectangle{\pgfqpoint{0.329460in}{0.284240in}}{\pgfqpoint{1.989680in}{1.989680in}}%
\pgfusepath{clip}%
\pgfsetbuttcap%
\pgfsetroundjoin%
\definecolor{currentfill}{rgb}{0.565498,0.842430,0.262877}%
\pgfsetfillcolor{currentfill}%
\pgfsetlinewidth{0.000000pt}%
\definecolor{currentstroke}{rgb}{0.000000,0.000000,0.000000}%
\pgfsetstrokecolor{currentstroke}%
\pgfsetdash{}{0pt}%
\pgfpathmoveto{\pgfqpoint{1.372928in}{1.547369in}}%
\pgfpathlineto{\pgfqpoint{1.373423in}{1.540802in}}%
\pgfpathlineto{\pgfqpoint{1.373917in}{1.534145in}}%
\pgfpathlineto{\pgfqpoint{1.374412in}{1.527400in}}%
\pgfpathlineto{\pgfqpoint{1.374906in}{1.520567in}}%
\pgfpathlineto{\pgfqpoint{1.369594in}{1.520246in}}%
\pgfpathlineto{\pgfqpoint{1.364264in}{1.520006in}}%
\pgfpathlineto{\pgfqpoint{1.358920in}{1.519848in}}%
\pgfpathlineto{\pgfqpoint{1.353567in}{1.519771in}}%
\pgfpathlineto{\pgfqpoint{1.353518in}{1.526620in}}%
\pgfpathlineto{\pgfqpoint{1.353468in}{1.533383in}}%
\pgfpathlineto{\pgfqpoint{1.353419in}{1.540057in}}%
\pgfpathlineto{\pgfqpoint{1.353369in}{1.546642in}}%
\pgfpathlineto{\pgfqpoint{1.358275in}{1.546712in}}%
\pgfpathlineto{\pgfqpoint{1.363173in}{1.546857in}}%
\pgfpathlineto{\pgfqpoint{1.368059in}{1.547076in}}%
\pgfpathlineto{\pgfqpoint{1.372928in}{1.547369in}}%
\pgfpathclose%
\pgfusepath{fill}%
\end{pgfscope}%
\begin{pgfscope}%
\pgfpathrectangle{\pgfqpoint{0.329460in}{0.284240in}}{\pgfqpoint{1.989680in}{1.989680in}}%
\pgfusepath{clip}%
\pgfsetbuttcap%
\pgfsetroundjoin%
\definecolor{currentfill}{rgb}{0.412913,0.803041,0.357269}%
\pgfsetfillcolor{currentfill}%
\pgfsetlinewidth{0.000000pt}%
\definecolor{currentstroke}{rgb}{0.000000,0.000000,0.000000}%
\pgfsetstrokecolor{currentstroke}%
\pgfsetdash{}{0pt}%
\pgfpathmoveto{\pgfqpoint{1.353766in}{1.491533in}}%
\pgfpathlineto{\pgfqpoint{1.353815in}{1.484272in}}%
\pgfpathlineto{\pgfqpoint{1.353865in}{1.476934in}}%
\pgfpathlineto{\pgfqpoint{1.353914in}{1.469519in}}%
\pgfpathlineto{\pgfqpoint{1.353964in}{1.462031in}}%
\pgfpathlineto{\pgfqpoint{1.347717in}{1.462036in}}%
\pgfpathlineto{\pgfqpoint{1.341474in}{1.462138in}}%
\pgfpathlineto{\pgfqpoint{1.335241in}{1.462335in}}%
\pgfpathlineto{\pgfqpoint{1.329025in}{1.462628in}}%
\pgfpathlineto{\pgfqpoint{1.329420in}{1.470105in}}%
\pgfpathlineto{\pgfqpoint{1.329816in}{1.477509in}}%
\pgfpathlineto{\pgfqpoint{1.330211in}{1.484836in}}%
\pgfpathlineto{\pgfqpoint{1.330607in}{1.492086in}}%
\pgfpathlineto{\pgfqpoint{1.336379in}{1.491815in}}%
\pgfpathlineto{\pgfqpoint{1.342167in}{1.491632in}}%
\pgfpathlineto{\pgfqpoint{1.347965in}{1.491538in}}%
\pgfpathlineto{\pgfqpoint{1.353766in}{1.491533in}}%
\pgfpathclose%
\pgfusepath{fill}%
\end{pgfscope}%
\begin{pgfscope}%
\pgfpathrectangle{\pgfqpoint{0.329460in}{0.284240in}}{\pgfqpoint{1.989680in}{1.989680in}}%
\pgfusepath{clip}%
\pgfsetbuttcap%
\pgfsetroundjoin%
\definecolor{currentfill}{rgb}{0.565498,0.842430,0.262877}%
\pgfsetfillcolor{currentfill}%
\pgfsetlinewidth{0.000000pt}%
\definecolor{currentstroke}{rgb}{0.000000,0.000000,0.000000}%
\pgfsetstrokecolor{currentstroke}%
\pgfsetdash{}{0pt}%
\pgfpathmoveto{\pgfqpoint{1.353369in}{1.546642in}}%
\pgfpathlineto{\pgfqpoint{1.353419in}{1.540057in}}%
\pgfpathlineto{\pgfqpoint{1.353468in}{1.533383in}}%
\pgfpathlineto{\pgfqpoint{1.353518in}{1.526620in}}%
\pgfpathlineto{\pgfqpoint{1.353567in}{1.519771in}}%
\pgfpathlineto{\pgfqpoint{1.348213in}{1.519775in}}%
\pgfpathlineto{\pgfqpoint{1.342861in}{1.519861in}}%
\pgfpathlineto{\pgfqpoint{1.337518in}{1.520029in}}%
\pgfpathlineto{\pgfqpoint{1.332190in}{1.520278in}}%
\pgfpathlineto{\pgfqpoint{1.332586in}{1.527117in}}%
\pgfpathlineto{\pgfqpoint{1.332982in}{1.533868in}}%
\pgfpathlineto{\pgfqpoint{1.333378in}{1.540531in}}%
\pgfpathlineto{\pgfqpoint{1.333774in}{1.547105in}}%
\pgfpathlineto{\pgfqpoint{1.338658in}{1.546878in}}%
\pgfpathlineto{\pgfqpoint{1.343556in}{1.546725in}}%
\pgfpathlineto{\pgfqpoint{1.348461in}{1.546646in}}%
\pgfpathlineto{\pgfqpoint{1.353369in}{1.546642in}}%
\pgfpathclose%
\pgfusepath{fill}%
\end{pgfscope}%
\begin{pgfscope}%
\pgfpathrectangle{\pgfqpoint{0.329460in}{0.284240in}}{\pgfqpoint{1.989680in}{1.989680in}}%
\pgfusepath{clip}%
\pgfsetbuttcap%
\pgfsetroundjoin%
\definecolor{currentfill}{rgb}{0.487026,0.823929,0.312321}%
\pgfsetfillcolor{currentfill}%
\pgfsetlinewidth{0.000000pt}%
\definecolor{currentstroke}{rgb}{0.000000,0.000000,0.000000}%
\pgfsetstrokecolor{currentstroke}%
\pgfsetdash{}{0pt}%
\pgfpathmoveto{\pgfqpoint{1.374906in}{1.520567in}}%
\pgfpathlineto{\pgfqpoint{1.375400in}{1.513650in}}%
\pgfpathlineto{\pgfqpoint{1.375894in}{1.506649in}}%
\pgfpathlineto{\pgfqpoint{1.376388in}{1.499565in}}%
\pgfpathlineto{\pgfqpoint{1.376882in}{1.492401in}}%
\pgfpathlineto{\pgfqpoint{1.371128in}{1.492051in}}%
\pgfpathlineto{\pgfqpoint{1.365353in}{1.491790in}}%
\pgfpathlineto{\pgfqpoint{1.359564in}{1.491617in}}%
\pgfpathlineto{\pgfqpoint{1.353766in}{1.491533in}}%
\pgfpathlineto{\pgfqpoint{1.353716in}{1.498715in}}%
\pgfpathlineto{\pgfqpoint{1.353667in}{1.505817in}}%
\pgfpathlineto{\pgfqpoint{1.353617in}{1.512836in}}%
\pgfpathlineto{\pgfqpoint{1.353567in}{1.519771in}}%
\pgfpathlineto{\pgfqpoint{1.358920in}{1.519848in}}%
\pgfpathlineto{\pgfqpoint{1.364264in}{1.520006in}}%
\pgfpathlineto{\pgfqpoint{1.369594in}{1.520246in}}%
\pgfpathlineto{\pgfqpoint{1.374906in}{1.520567in}}%
\pgfpathclose%
\pgfusepath{fill}%
\end{pgfscope}%
\begin{pgfscope}%
\pgfpathrectangle{\pgfqpoint{0.329460in}{0.284240in}}{\pgfqpoint{1.989680in}{1.989680in}}%
\pgfusepath{clip}%
\pgfsetbuttcap%
\pgfsetroundjoin%
\definecolor{currentfill}{rgb}{0.487026,0.823929,0.312321}%
\pgfsetfillcolor{currentfill}%
\pgfsetlinewidth{0.000000pt}%
\definecolor{currentstroke}{rgb}{0.000000,0.000000,0.000000}%
\pgfsetstrokecolor{currentstroke}%
\pgfsetdash{}{0pt}%
\pgfpathmoveto{\pgfqpoint{1.353567in}{1.519771in}}%
\pgfpathlineto{\pgfqpoint{1.353617in}{1.512836in}}%
\pgfpathlineto{\pgfqpoint{1.353667in}{1.505817in}}%
\pgfpathlineto{\pgfqpoint{1.353716in}{1.498715in}}%
\pgfpathlineto{\pgfqpoint{1.353766in}{1.491533in}}%
\pgfpathlineto{\pgfqpoint{1.347965in}{1.491538in}}%
\pgfpathlineto{\pgfqpoint{1.342167in}{1.491632in}}%
\pgfpathlineto{\pgfqpoint{1.336379in}{1.491815in}}%
\pgfpathlineto{\pgfqpoint{1.330607in}{1.492086in}}%
\pgfpathlineto{\pgfqpoint{1.331002in}{1.499256in}}%
\pgfpathlineto{\pgfqpoint{1.331398in}{1.506346in}}%
\pgfpathlineto{\pgfqpoint{1.331794in}{1.513354in}}%
\pgfpathlineto{\pgfqpoint{1.332190in}{1.520278in}}%
\pgfpathlineto{\pgfqpoint{1.337518in}{1.520029in}}%
\pgfpathlineto{\pgfqpoint{1.342861in}{1.519861in}}%
\pgfpathlineto{\pgfqpoint{1.348213in}{1.519775in}}%
\pgfpathlineto{\pgfqpoint{1.353567in}{1.519771in}}%
\pgfpathclose%
\pgfusepath{fill}%
\end{pgfscope}%
\begin{pgfscope}%
\pgfpathrectangle{\pgfqpoint{0.329460in}{0.284240in}}{\pgfqpoint{1.989680in}{1.989680in}}%
\pgfusepath{clip}%
\pgfsetbuttcap%
\pgfsetroundjoin%
\definecolor{currentfill}{rgb}{0.268510,0.009605,0.335427}%
\pgfsetfillcolor{currentfill}%
\pgfsetlinewidth{0.000000pt}%
\definecolor{currentstroke}{rgb}{0.000000,0.000000,0.000000}%
\pgfsetstrokecolor{currentstroke}%
\pgfsetdash{}{0pt}%
\pgfpathmoveto{\pgfqpoint{1.499995in}{0.750885in}}%
\pgfpathlineto{\pgfqpoint{1.500976in}{0.750239in}}%
\pgfpathlineto{\pgfqpoint{1.501960in}{0.749846in}}%
\pgfpathlineto{\pgfqpoint{1.502947in}{0.749711in}}%
\pgfpathlineto{\pgfqpoint{1.503937in}{0.749840in}}%
\pgfpathlineto{\pgfqpoint{1.486255in}{0.747364in}}%
\pgfpathlineto{\pgfqpoint{1.468420in}{0.745193in}}%
\pgfpathlineto{\pgfqpoint{1.450453in}{0.743329in}}%
\pgfpathlineto{\pgfqpoint{1.432372in}{0.741775in}}%
\pgfpathlineto{\pgfqpoint{1.431846in}{0.741699in}}%
\pgfpathlineto{\pgfqpoint{1.431321in}{0.741886in}}%
\pgfpathlineto{\pgfqpoint{1.430797in}{0.742331in}}%
\pgfpathlineto{\pgfqpoint{1.430275in}{0.743030in}}%
\pgfpathlineto{\pgfqpoint{1.447889in}{0.744544in}}%
\pgfpathlineto{\pgfqpoint{1.465393in}{0.746359in}}%
\pgfpathlineto{\pgfqpoint{1.482769in}{0.748474in}}%
\pgfpathlineto{\pgfqpoint{1.499995in}{0.750885in}}%
\pgfpathclose%
\pgfusepath{fill}%
\end{pgfscope}%
\begin{pgfscope}%
\pgfpathrectangle{\pgfqpoint{0.329460in}{0.284240in}}{\pgfqpoint{1.989680in}{1.989680in}}%
\pgfusepath{clip}%
\pgfsetbuttcap%
\pgfsetroundjoin%
\definecolor{currentfill}{rgb}{0.272594,0.025563,0.353093}%
\pgfsetfillcolor{currentfill}%
\pgfsetlinewidth{0.000000pt}%
\definecolor{currentstroke}{rgb}{0.000000,0.000000,0.000000}%
\pgfsetstrokecolor{currentstroke}%
\pgfsetdash{}{0pt}%
\pgfpathmoveto{\pgfqpoint{1.572742in}{0.762724in}}%
\pgfpathlineto{\pgfqpoint{1.574180in}{0.763203in}}%
\pgfpathlineto{\pgfqpoint{1.575621in}{0.763956in}}%
\pgfpathlineto{\pgfqpoint{1.577068in}{0.764987in}}%
\pgfpathlineto{\pgfqpoint{1.578519in}{0.766301in}}%
\pgfpathlineto{\pgfqpoint{1.561207in}{0.762547in}}%
\pgfpathlineto{\pgfqpoint{1.543658in}{0.759091in}}%
\pgfpathlineto{\pgfqpoint{1.525892in}{0.755936in}}%
\pgfpathlineto{\pgfqpoint{1.507927in}{0.753087in}}%
\pgfpathlineto{\pgfqpoint{1.506925in}{0.751855in}}%
\pgfpathlineto{\pgfqpoint{1.505926in}{0.750907in}}%
\pgfpathlineto{\pgfqpoint{1.504930in}{0.750237in}}%
\pgfpathlineto{\pgfqpoint{1.503937in}{0.749840in}}%
\pgfpathlineto{\pgfqpoint{1.521446in}{0.752618in}}%
\pgfpathlineto{\pgfqpoint{1.538763in}{0.755693in}}%
\pgfpathlineto{\pgfqpoint{1.555868in}{0.759064in}}%
\pgfpathlineto{\pgfqpoint{1.572742in}{0.762724in}}%
\pgfpathclose%
\pgfusepath{fill}%
\end{pgfscope}%
\begin{pgfscope}%
\pgfpathrectangle{\pgfqpoint{0.329460in}{0.284240in}}{\pgfqpoint{1.989680in}{1.989680in}}%
\pgfusepath{clip}%
\pgfsetbuttcap%
\pgfsetroundjoin%
\definecolor{currentfill}{rgb}{0.277941,0.056324,0.381191}%
\pgfsetfillcolor{currentfill}%
\pgfsetlinewidth{0.000000pt}%
\definecolor{currentstroke}{rgb}{0.000000,0.000000,0.000000}%
\pgfsetstrokecolor{currentstroke}%
\pgfsetdash{}{0pt}%
\pgfpathmoveto{\pgfqpoint{1.139232in}{0.762949in}}%
\pgfpathlineto{\pgfqpoint{1.137874in}{0.764531in}}%
\pgfpathlineto{\pgfqpoint{1.136511in}{0.766406in}}%
\pgfpathlineto{\pgfqpoint{1.135143in}{0.768581in}}%
\pgfpathlineto{\pgfqpoint{1.133771in}{0.771060in}}%
\pgfpathlineto{\pgfqpoint{1.116045in}{0.774939in}}%
\pgfpathlineto{\pgfqpoint{1.098585in}{0.779117in}}%
\pgfpathlineto{\pgfqpoint{1.081410in}{0.783590in}}%
\pgfpathlineto{\pgfqpoint{1.064540in}{0.788350in}}%
\pgfpathlineto{\pgfqpoint{1.066345in}{0.785768in}}%
\pgfpathlineto{\pgfqpoint{1.068144in}{0.783491in}}%
\pgfpathlineto{\pgfqpoint{1.069936in}{0.781512in}}%
\pgfpathlineto{\pgfqpoint{1.071723in}{0.779827in}}%
\pgfpathlineto{\pgfqpoint{1.088175in}{0.775180in}}%
\pgfpathlineto{\pgfqpoint{1.104923in}{0.770814in}}%
\pgfpathlineto{\pgfqpoint{1.121948in}{0.766736in}}%
\pgfpathlineto{\pgfqpoint{1.139232in}{0.762949in}}%
\pgfpathclose%
\pgfusepath{fill}%
\end{pgfscope}%
\begin{pgfscope}%
\pgfpathrectangle{\pgfqpoint{0.329460in}{0.284240in}}{\pgfqpoint{1.989680in}{1.989680in}}%
\pgfusepath{clip}%
\pgfsetbuttcap%
\pgfsetroundjoin%
\definecolor{currentfill}{rgb}{0.276194,0.190074,0.493001}%
\pgfsetfillcolor{currentfill}%
\pgfsetlinewidth{0.000000pt}%
\definecolor{currentstroke}{rgb}{0.000000,0.000000,0.000000}%
\pgfsetstrokecolor{currentstroke}%
\pgfsetdash{}{0pt}%
\pgfpathmoveto{\pgfqpoint{0.982537in}{0.843422in}}%
\pgfpathlineto{\pgfqpoint{0.980255in}{0.849117in}}%
\pgfpathlineto{\pgfqpoint{0.977963in}{0.855174in}}%
\pgfpathlineto{\pgfqpoint{0.975660in}{0.861598in}}%
\pgfpathlineto{\pgfqpoint{0.973347in}{0.868396in}}%
\pgfpathlineto{\pgfqpoint{0.957124in}{0.874916in}}%
\pgfpathlineto{\pgfqpoint{0.941342in}{0.881699in}}%
\pgfpathlineto{\pgfqpoint{0.926017in}{0.888738in}}%
\pgfpathlineto{\pgfqpoint{0.911166in}{0.896022in}}%
\pgfpathlineto{\pgfqpoint{0.913849in}{0.889084in}}%
\pgfpathlineto{\pgfqpoint{0.916520in}{0.882518in}}%
\pgfpathlineto{\pgfqpoint{0.919179in}{0.876319in}}%
\pgfpathlineto{\pgfqpoint{0.921826in}{0.870481in}}%
\pgfpathlineto{\pgfqpoint{0.936328in}{0.863345in}}%
\pgfpathlineto{\pgfqpoint{0.951291in}{0.856451in}}%
\pgfpathlineto{\pgfqpoint{0.966700in}{0.849807in}}%
\pgfpathlineto{\pgfqpoint{0.982537in}{0.843422in}}%
\pgfpathclose%
\pgfusepath{fill}%
\end{pgfscope}%
\begin{pgfscope}%
\pgfpathrectangle{\pgfqpoint{0.329460in}{0.284240in}}{\pgfqpoint{1.989680in}{1.989680in}}%
\pgfusepath{clip}%
\pgfsetbuttcap%
\pgfsetroundjoin%
\definecolor{currentfill}{rgb}{0.233603,0.313828,0.543914}%
\pgfsetfillcolor{currentfill}%
\pgfsetlinewidth{0.000000pt}%
\definecolor{currentstroke}{rgb}{0.000000,0.000000,0.000000}%
\pgfsetstrokecolor{currentstroke}%
\pgfsetdash{}{0pt}%
\pgfpathmoveto{\pgfqpoint{1.868928in}{0.967261in}}%
\pgfpathlineto{\pgfqpoint{1.872066in}{0.976349in}}%
\pgfpathlineto{\pgfqpoint{1.875219in}{0.985848in}}%
\pgfpathlineto{\pgfqpoint{1.878389in}{0.995765in}}%
\pgfpathlineto{\pgfqpoint{1.881576in}{1.006107in}}%
\pgfpathlineto{\pgfqpoint{1.868662in}{0.997430in}}%
\pgfpathlineto{\pgfqpoint{1.855179in}{0.988960in}}%
\pgfpathlineto{\pgfqpoint{1.841140in}{0.980708in}}%
\pgfpathlineto{\pgfqpoint{1.826558in}{0.972683in}}%
\pgfpathlineto{\pgfqpoint{1.823687in}{0.962490in}}%
\pgfpathlineto{\pgfqpoint{1.820832in}{0.952724in}}%
\pgfpathlineto{\pgfqpoint{1.817991in}{0.943377in}}%
\pgfpathlineto{\pgfqpoint{1.815165in}{0.934443in}}%
\pgfpathlineto{\pgfqpoint{1.829412in}{0.942321in}}%
\pgfpathlineto{\pgfqpoint{1.843130in}{0.950424in}}%
\pgfpathlineto{\pgfqpoint{1.856306in}{0.958740in}}%
\pgfpathlineto{\pgfqpoint{1.868928in}{0.967261in}}%
\pgfpathclose%
\pgfusepath{fill}%
\end{pgfscope}%
\begin{pgfscope}%
\pgfpathrectangle{\pgfqpoint{0.329460in}{0.284240in}}{\pgfqpoint{1.989680in}{1.989680in}}%
\pgfusepath{clip}%
\pgfsetbuttcap%
\pgfsetroundjoin%
\definecolor{currentfill}{rgb}{0.268510,0.009605,0.335427}%
\pgfsetfillcolor{currentfill}%
\pgfsetlinewidth{0.000000pt}%
\definecolor{currentstroke}{rgb}{0.000000,0.000000,0.000000}%
\pgfsetstrokecolor{currentstroke}%
\pgfsetdash{}{0pt}%
\pgfpathmoveto{\pgfqpoint{1.287833in}{0.741940in}}%
\pgfpathlineto{\pgfqpoint{1.287414in}{0.741234in}}%
\pgfpathlineto{\pgfqpoint{1.286995in}{0.740781in}}%
\pgfpathlineto{\pgfqpoint{1.286574in}{0.740587in}}%
\pgfpathlineto{\pgfqpoint{1.286152in}{0.740656in}}%
\pgfpathlineto{\pgfqpoint{1.267989in}{0.741932in}}%
\pgfpathlineto{\pgfqpoint{1.249920in}{0.743521in}}%
\pgfpathlineto{\pgfqpoint{1.231966in}{0.745419in}}%
\pgfpathlineto{\pgfqpoint{1.214147in}{0.747624in}}%
\pgfpathlineto{\pgfqpoint{1.215036in}{0.747510in}}%
\pgfpathlineto{\pgfqpoint{1.215921in}{0.747659in}}%
\pgfpathlineto{\pgfqpoint{1.216804in}{0.748066in}}%
\pgfpathlineto{\pgfqpoint{1.217685in}{0.748727in}}%
\pgfpathlineto{\pgfqpoint{1.235045in}{0.746579in}}%
\pgfpathlineto{\pgfqpoint{1.252536in}{0.744730in}}%
\pgfpathlineto{\pgfqpoint{1.270138in}{0.743183in}}%
\pgfpathlineto{\pgfqpoint{1.287833in}{0.741940in}}%
\pgfpathclose%
\pgfusepath{fill}%
\end{pgfscope}%
\begin{pgfscope}%
\pgfpathrectangle{\pgfqpoint{0.329460in}{0.284240in}}{\pgfqpoint{1.989680in}{1.989680in}}%
\pgfusepath{clip}%
\pgfsetbuttcap%
\pgfsetroundjoin%
\definecolor{currentfill}{rgb}{0.272594,0.025563,0.353093}%
\pgfsetfillcolor{currentfill}%
\pgfsetlinewidth{0.000000pt}%
\definecolor{currentstroke}{rgb}{0.000000,0.000000,0.000000}%
\pgfsetstrokecolor{currentstroke}%
\pgfsetdash{}{0pt}%
\pgfpathmoveto{\pgfqpoint{1.214147in}{0.747624in}}%
\pgfpathlineto{\pgfqpoint{1.213256in}{0.748007in}}%
\pgfpathlineto{\pgfqpoint{1.212363in}{0.748662in}}%
\pgfpathlineto{\pgfqpoint{1.211466in}{0.749597in}}%
\pgfpathlineto{\pgfqpoint{1.210566in}{0.750815in}}%
\pgfpathlineto{\pgfqpoint{1.192443in}{0.753389in}}%
\pgfpathlineto{\pgfqpoint{1.174499in}{0.756272in}}%
\pgfpathlineto{\pgfqpoint{1.156756in}{0.759460in}}%
\pgfpathlineto{\pgfqpoint{1.139232in}{0.762949in}}%
\pgfpathlineto{\pgfqpoint{1.140586in}{0.761656in}}%
\pgfpathlineto{\pgfqpoint{1.141935in}{0.760646in}}%
\pgfpathlineto{\pgfqpoint{1.143280in}{0.759915in}}%
\pgfpathlineto{\pgfqpoint{1.144621in}{0.759456in}}%
\pgfpathlineto{\pgfqpoint{1.161701in}{0.756053in}}%
\pgfpathlineto{\pgfqpoint{1.178995in}{0.752945in}}%
\pgfpathlineto{\pgfqpoint{1.196484in}{0.750134in}}%
\pgfpathlineto{\pgfqpoint{1.214147in}{0.747624in}}%
\pgfpathclose%
\pgfusepath{fill}%
\end{pgfscope}%
\begin{pgfscope}%
\pgfpathrectangle{\pgfqpoint{0.329460in}{0.284240in}}{\pgfqpoint{1.989680in}{1.989680in}}%
\pgfusepath{clip}%
\pgfsetbuttcap%
\pgfsetroundjoin%
\definecolor{currentfill}{rgb}{0.282884,0.135920,0.453427}%
\pgfsetfillcolor{currentfill}%
\pgfsetlinewidth{0.000000pt}%
\definecolor{currentstroke}{rgb}{0.000000,0.000000,0.000000}%
\pgfsetstrokecolor{currentstroke}%
\pgfsetdash{}{0pt}%
\pgfpathmoveto{\pgfqpoint{1.724567in}{0.829674in}}%
\pgfpathlineto{\pgfqpoint{1.726895in}{0.834015in}}%
\pgfpathlineto{\pgfqpoint{1.729233in}{0.838694in}}%
\pgfpathlineto{\pgfqpoint{1.731580in}{0.843715in}}%
\pgfpathlineto{\pgfqpoint{1.733937in}{0.849085in}}%
\pgfpathlineto{\pgfqpoint{1.718052in}{0.842729in}}%
\pgfpathlineto{\pgfqpoint{1.701757in}{0.836640in}}%
\pgfpathlineto{\pgfqpoint{1.685067in}{0.830826in}}%
\pgfpathlineto{\pgfqpoint{1.668002in}{0.825294in}}%
\pgfpathlineto{\pgfqpoint{1.666044in}{0.820052in}}%
\pgfpathlineto{\pgfqpoint{1.664094in}{0.815159in}}%
\pgfpathlineto{\pgfqpoint{1.662153in}{0.810611in}}%
\pgfpathlineto{\pgfqpoint{1.660219in}{0.806400in}}%
\pgfpathlineto{\pgfqpoint{1.676872in}{0.811811in}}%
\pgfpathlineto{\pgfqpoint{1.693159in}{0.817499in}}%
\pgfpathlineto{\pgfqpoint{1.709063in}{0.823456in}}%
\pgfpathlineto{\pgfqpoint{1.724567in}{0.829674in}}%
\pgfpathclose%
\pgfusepath{fill}%
\end{pgfscope}%
\begin{pgfscope}%
\pgfpathrectangle{\pgfqpoint{0.329460in}{0.284240in}}{\pgfqpoint{1.989680in}{1.989680in}}%
\pgfusepath{clip}%
\pgfsetbuttcap%
\pgfsetroundjoin%
\definecolor{currentfill}{rgb}{0.268510,0.009605,0.335427}%
\pgfsetfillcolor{currentfill}%
\pgfsetlinewidth{0.000000pt}%
\definecolor{currentstroke}{rgb}{0.000000,0.000000,0.000000}%
\pgfsetstrokecolor{currentstroke}%
\pgfsetdash{}{0pt}%
\pgfpathmoveto{\pgfqpoint{1.430275in}{0.743030in}}%
\pgfpathlineto{\pgfqpoint{1.430797in}{0.742331in}}%
\pgfpathlineto{\pgfqpoint{1.431321in}{0.741886in}}%
\pgfpathlineto{\pgfqpoint{1.431846in}{0.741699in}}%
\pgfpathlineto{\pgfqpoint{1.432372in}{0.741775in}}%
\pgfpathlineto{\pgfqpoint{1.414200in}{0.740533in}}%
\pgfpathlineto{\pgfqpoint{1.395956in}{0.739605in}}%
\pgfpathlineto{\pgfqpoint{1.377661in}{0.738991in}}%
\pgfpathlineto{\pgfqpoint{1.359336in}{0.738693in}}%
\pgfpathlineto{\pgfqpoint{1.359283in}{0.738636in}}%
\pgfpathlineto{\pgfqpoint{1.359231in}{0.738843in}}%
\pgfpathlineto{\pgfqpoint{1.359178in}{0.739309in}}%
\pgfpathlineto{\pgfqpoint{1.359126in}{0.740028in}}%
\pgfpathlineto{\pgfqpoint{1.376977in}{0.740319in}}%
\pgfpathlineto{\pgfqpoint{1.394799in}{0.740916in}}%
\pgfpathlineto{\pgfqpoint{1.412572in}{0.741821in}}%
\pgfpathlineto{\pgfqpoint{1.430275in}{0.743030in}}%
\pgfpathclose%
\pgfusepath{fill}%
\end{pgfscope}%
\begin{pgfscope}%
\pgfpathrectangle{\pgfqpoint{0.329460in}{0.284240in}}{\pgfqpoint{1.989680in}{1.989680in}}%
\pgfusepath{clip}%
\pgfsetbuttcap%
\pgfsetroundjoin%
\definecolor{currentfill}{rgb}{0.268510,0.009605,0.335427}%
\pgfsetfillcolor{currentfill}%
\pgfsetlinewidth{0.000000pt}%
\definecolor{currentstroke}{rgb}{0.000000,0.000000,0.000000}%
\pgfsetstrokecolor{currentstroke}%
\pgfsetdash{}{0pt}%
\pgfpathmoveto{\pgfqpoint{1.359126in}{0.740028in}}%
\pgfpathlineto{\pgfqpoint{1.359178in}{0.739309in}}%
\pgfpathlineto{\pgfqpoint{1.359231in}{0.738843in}}%
\pgfpathlineto{\pgfqpoint{1.359283in}{0.738636in}}%
\pgfpathlineto{\pgfqpoint{1.359336in}{0.738693in}}%
\pgfpathlineto{\pgfqpoint{1.341002in}{0.738710in}}%
\pgfpathlineto{\pgfqpoint{1.322679in}{0.739044in}}%
\pgfpathlineto{\pgfqpoint{1.304389in}{0.739693in}}%
\pgfpathlineto{\pgfqpoint{1.286152in}{0.740656in}}%
\pgfpathlineto{\pgfqpoint{1.286574in}{0.740587in}}%
\pgfpathlineto{\pgfqpoint{1.286995in}{0.740781in}}%
\pgfpathlineto{\pgfqpoint{1.287414in}{0.741234in}}%
\pgfpathlineto{\pgfqpoint{1.287833in}{0.741940in}}%
\pgfpathlineto{\pgfqpoint{1.305599in}{0.741002in}}%
\pgfpathlineto{\pgfqpoint{1.323416in}{0.740370in}}%
\pgfpathlineto{\pgfqpoint{1.341265in}{0.740045in}}%
\pgfpathlineto{\pgfqpoint{1.359126in}{0.740028in}}%
\pgfpathclose%
\pgfusepath{fill}%
\end{pgfscope}%
\begin{pgfscope}%
\pgfpathrectangle{\pgfqpoint{0.329460in}{0.284240in}}{\pgfqpoint{1.989680in}{1.989680in}}%
\pgfusepath{clip}%
\pgfsetbuttcap%
\pgfsetroundjoin%
\definecolor{currentfill}{rgb}{0.172719,0.448791,0.557885}%
\pgfsetfillcolor{currentfill}%
\pgfsetlinewidth{0.000000pt}%
\definecolor{currentstroke}{rgb}{0.000000,0.000000,0.000000}%
\pgfsetstrokecolor{currentstroke}%
\pgfsetdash{}{0pt}%
\pgfpathmoveto{\pgfqpoint{1.941276in}{1.089073in}}%
\pgfpathlineto{\pgfqpoint{1.944815in}{1.101804in}}%
\pgfpathlineto{\pgfqpoint{1.948375in}{1.115002in}}%
\pgfpathlineto{\pgfqpoint{1.951956in}{1.128678in}}%
\pgfpathlineto{\pgfqpoint{1.941014in}{1.119003in}}%
\pgfpathlineto{\pgfqpoint{1.929433in}{1.109498in}}%
\pgfpathlineto{\pgfqpoint{1.917221in}{1.100172in}}%
\pgfpathlineto{\pgfqpoint{1.904391in}{1.091037in}}%
\pgfpathlineto{\pgfqpoint{1.901074in}{1.077507in}}%
\pgfpathlineto{\pgfqpoint{1.897777in}{1.064457in}}%
\pgfpathlineto{\pgfqpoint{1.894500in}{1.051877in}}%
\pgfpathlineto{\pgfqpoint{1.907116in}{1.060903in}}%
\pgfpathlineto{\pgfqpoint{1.919124in}{1.070119in}}%
\pgfpathlineto{\pgfqpoint{1.930514in}{1.079512in}}%
\pgfpathlineto{\pgfqpoint{1.941276in}{1.089073in}}%
\pgfpathclose%
\pgfusepath{fill}%
\end{pgfscope}%
\begin{pgfscope}%
\pgfpathrectangle{\pgfqpoint{0.329460in}{0.284240in}}{\pgfqpoint{1.989680in}{1.989680in}}%
\pgfusepath{clip}%
\pgfsetbuttcap%
\pgfsetroundjoin%
\definecolor{currentfill}{rgb}{0.233603,0.313828,0.543914}%
\pgfsetfillcolor{currentfill}%
\pgfsetlinewidth{0.000000pt}%
\definecolor{currentstroke}{rgb}{0.000000,0.000000,0.000000}%
\pgfsetstrokecolor{currentstroke}%
\pgfsetdash{}{0pt}%
\pgfpathmoveto{\pgfqpoint{0.900306in}{0.927636in}}%
\pgfpathlineto{\pgfqpoint{0.897557in}{0.936538in}}%
\pgfpathlineto{\pgfqpoint{0.894794in}{0.945853in}}%
\pgfpathlineto{\pgfqpoint{0.892016in}{0.955587in}}%
\pgfpathlineto{\pgfqpoint{0.889223in}{0.965750in}}%
\pgfpathlineto{\pgfqpoint{0.874171in}{0.973564in}}%
\pgfpathlineto{\pgfqpoint{0.859649in}{0.981614in}}%
\pgfpathlineto{\pgfqpoint{0.845671in}{0.989891in}}%
\pgfpathlineto{\pgfqpoint{0.832250in}{0.998384in}}%
\pgfpathlineto{\pgfqpoint{0.835372in}{0.988076in}}%
\pgfpathlineto{\pgfqpoint{0.838477in}{0.978193in}}%
\pgfpathlineto{\pgfqpoint{0.841566in}{0.968729in}}%
\pgfpathlineto{\pgfqpoint{0.844639in}{0.959677in}}%
\pgfpathlineto{\pgfqpoint{0.857755in}{0.951337in}}%
\pgfpathlineto{\pgfqpoint{0.871413in}{0.943211in}}%
\pgfpathlineto{\pgfqpoint{0.885601in}{0.935307in}}%
\pgfpathlineto{\pgfqpoint{0.900306in}{0.927636in}}%
\pgfpathclose%
\pgfusepath{fill}%
\end{pgfscope}%
\begin{pgfscope}%
\pgfpathrectangle{\pgfqpoint{0.329460in}{0.284240in}}{\pgfqpoint{1.989680in}{1.989680in}}%
\pgfusepath{clip}%
\pgfsetbuttcap%
\pgfsetroundjoin%
\definecolor{currentfill}{rgb}{0.282884,0.135920,0.453427}%
\pgfsetfillcolor{currentfill}%
\pgfsetlinewidth{0.000000pt}%
\definecolor{currentstroke}{rgb}{0.000000,0.000000,0.000000}%
\pgfsetstrokecolor{currentstroke}%
\pgfsetdash{}{0pt}%
\pgfpathmoveto{\pgfqpoint{1.057251in}{0.801827in}}%
\pgfpathlineto{\pgfqpoint{1.055411in}{0.806012in}}%
\pgfpathlineto{\pgfqpoint{1.053563in}{0.810535in}}%
\pgfpathlineto{\pgfqpoint{1.051707in}{0.815402in}}%
\pgfpathlineto{\pgfqpoint{1.049843in}{0.820619in}}%
\pgfpathlineto{\pgfqpoint{1.032459in}{0.825895in}}%
\pgfpathlineto{\pgfqpoint{1.015435in}{0.831458in}}%
\pgfpathlineto{\pgfqpoint{0.998788in}{0.837303in}}%
\pgfpathlineto{\pgfqpoint{0.982537in}{0.843422in}}%
\pgfpathlineto{\pgfqpoint{0.984810in}{0.838082in}}%
\pgfpathlineto{\pgfqpoint{0.987072in}{0.833092in}}%
\pgfpathlineto{\pgfqpoint{0.989325in}{0.828444in}}%
\pgfpathlineto{\pgfqpoint{0.991569in}{0.824134in}}%
\pgfpathlineto{\pgfqpoint{1.007429in}{0.818148in}}%
\pgfpathlineto{\pgfqpoint{1.023675in}{0.812430in}}%
\pgfpathlineto{\pgfqpoint{1.040288in}{0.806987in}}%
\pgfpathlineto{\pgfqpoint{1.057251in}{0.801827in}}%
\pgfpathclose%
\pgfusepath{fill}%
\end{pgfscope}%
\begin{pgfscope}%
\pgfpathrectangle{\pgfqpoint{0.329460in}{0.284240in}}{\pgfqpoint{1.989680in}{1.989680in}}%
\pgfusepath{clip}%
\pgfsetbuttcap%
\pgfsetroundjoin%
\definecolor{currentfill}{rgb}{0.282327,0.094955,0.417331}%
\pgfsetfillcolor{currentfill}%
\pgfsetlinewidth{0.000000pt}%
\definecolor{currentstroke}{rgb}{0.000000,0.000000,0.000000}%
\pgfsetstrokecolor{currentstroke}%
\pgfsetdash{}{0pt}%
\pgfpathmoveto{\pgfqpoint{1.652561in}{0.792819in}}%
\pgfpathlineto{\pgfqpoint{1.654465in}{0.795736in}}%
\pgfpathlineto{\pgfqpoint{1.656376in}{0.798968in}}%
\pgfpathlineto{\pgfqpoint{1.658294in}{0.802521in}}%
\pgfpathlineto{\pgfqpoint{1.660219in}{0.806400in}}%
\pgfpathlineto{\pgfqpoint{1.643219in}{0.801271in}}%
\pgfpathlineto{\pgfqpoint{1.625888in}{0.796432in}}%
\pgfpathlineto{\pgfqpoint{1.608248in}{0.791888in}}%
\pgfpathlineto{\pgfqpoint{1.590316in}{0.787645in}}%
\pgfpathlineto{\pgfqpoint{1.588821in}{0.783873in}}%
\pgfpathlineto{\pgfqpoint{1.587333in}{0.780427in}}%
\pgfpathlineto{\pgfqpoint{1.585850in}{0.777302in}}%
\pgfpathlineto{\pgfqpoint{1.584373in}{0.774493in}}%
\pgfpathlineto{\pgfqpoint{1.601864in}{0.778639in}}%
\pgfpathlineto{\pgfqpoint{1.619071in}{0.783078in}}%
\pgfpathlineto{\pgfqpoint{1.635976in}{0.787807in}}%
\pgfpathlineto{\pgfqpoint{1.652561in}{0.792819in}}%
\pgfpathclose%
\pgfusepath{fill}%
\end{pgfscope}%
\begin{pgfscope}%
\pgfpathrectangle{\pgfqpoint{0.329460in}{0.284240in}}{\pgfqpoint{1.989680in}{1.989680in}}%
\pgfusepath{clip}%
\pgfsetbuttcap%
\pgfsetroundjoin%
\definecolor{currentfill}{rgb}{0.272594,0.025563,0.353093}%
\pgfsetfillcolor{currentfill}%
\pgfsetlinewidth{0.000000pt}%
\definecolor{currentstroke}{rgb}{0.000000,0.000000,0.000000}%
\pgfsetstrokecolor{currentstroke}%
\pgfsetdash{}{0pt}%
\pgfpathmoveto{\pgfqpoint{1.503937in}{0.749840in}}%
\pgfpathlineto{\pgfqpoint{1.504930in}{0.750237in}}%
\pgfpathlineto{\pgfqpoint{1.505926in}{0.750907in}}%
\pgfpathlineto{\pgfqpoint{1.506925in}{0.751855in}}%
\pgfpathlineto{\pgfqpoint{1.507927in}{0.753087in}}%
\pgfpathlineto{\pgfqpoint{1.489785in}{0.750548in}}%
\pgfpathlineto{\pgfqpoint{1.471485in}{0.748321in}}%
\pgfpathlineto{\pgfqpoint{1.453048in}{0.746409in}}%
\pgfpathlineto{\pgfqpoint{1.434496in}{0.744816in}}%
\pgfpathlineto{\pgfqpoint{1.433962in}{0.743635in}}%
\pgfpathlineto{\pgfqpoint{1.433431in}{0.742738in}}%
\pgfpathlineto{\pgfqpoint{1.432901in}{0.742120in}}%
\pgfpathlineto{\pgfqpoint{1.432372in}{0.741775in}}%
\pgfpathlineto{\pgfqpoint{1.450453in}{0.743329in}}%
\pgfpathlineto{\pgfqpoint{1.468420in}{0.745193in}}%
\pgfpathlineto{\pgfqpoint{1.486255in}{0.747364in}}%
\pgfpathlineto{\pgfqpoint{1.503937in}{0.749840in}}%
\pgfpathclose%
\pgfusepath{fill}%
\end{pgfscope}%
\begin{pgfscope}%
\pgfpathrectangle{\pgfqpoint{0.329460in}{0.284240in}}{\pgfqpoint{1.989680in}{1.989680in}}%
\pgfusepath{clip}%
\pgfsetbuttcap%
\pgfsetroundjoin%
\definecolor{currentfill}{rgb}{0.277941,0.056324,0.381191}%
\pgfsetfillcolor{currentfill}%
\pgfsetlinewidth{0.000000pt}%
\definecolor{currentstroke}{rgb}{0.000000,0.000000,0.000000}%
\pgfsetstrokecolor{currentstroke}%
\pgfsetdash{}{0pt}%
\pgfpathmoveto{\pgfqpoint{1.578519in}{0.766301in}}%
\pgfpathlineto{\pgfqpoint{1.579975in}{0.767903in}}%
\pgfpathlineto{\pgfqpoint{1.581436in}{0.769799in}}%
\pgfpathlineto{\pgfqpoint{1.582902in}{0.771994in}}%
\pgfpathlineto{\pgfqpoint{1.584373in}{0.774493in}}%
\pgfpathlineto{\pgfqpoint{1.566619in}{0.770647in}}%
\pgfpathlineto{\pgfqpoint{1.548621in}{0.767106in}}%
\pgfpathlineto{\pgfqpoint{1.530398in}{0.763874in}}%
\pgfpathlineto{\pgfqpoint{1.511972in}{0.760955in}}%
\pgfpathlineto{\pgfqpoint{1.510955in}{0.758537in}}%
\pgfpathlineto{\pgfqpoint{1.509942in}{0.756423in}}%
\pgfpathlineto{\pgfqpoint{1.508933in}{0.754608in}}%
\pgfpathlineto{\pgfqpoint{1.507927in}{0.753087in}}%
\pgfpathlineto{\pgfqpoint{1.525892in}{0.755936in}}%
\pgfpathlineto{\pgfqpoint{1.543658in}{0.759091in}}%
\pgfpathlineto{\pgfqpoint{1.561207in}{0.762547in}}%
\pgfpathlineto{\pgfqpoint{1.578519in}{0.766301in}}%
\pgfpathclose%
\pgfusepath{fill}%
\end{pgfscope}%
\begin{pgfscope}%
\pgfpathrectangle{\pgfqpoint{0.329460in}{0.284240in}}{\pgfqpoint{1.989680in}{1.989680in}}%
\pgfusepath{clip}%
\pgfsetbuttcap%
\pgfsetroundjoin%
\definecolor{currentfill}{rgb}{0.272594,0.025563,0.353093}%
\pgfsetfillcolor{currentfill}%
\pgfsetlinewidth{0.000000pt}%
\definecolor{currentstroke}{rgb}{0.000000,0.000000,0.000000}%
\pgfsetstrokecolor{currentstroke}%
\pgfsetdash{}{0pt}%
\pgfpathmoveto{\pgfqpoint{1.286152in}{0.740656in}}%
\pgfpathlineto{\pgfqpoint{1.285729in}{0.740993in}}%
\pgfpathlineto{\pgfqpoint{1.285304in}{0.741605in}}%
\pgfpathlineto{\pgfqpoint{1.284878in}{0.742494in}}%
\pgfpathlineto{\pgfqpoint{1.284451in}{0.743668in}}%
\pgfpathlineto{\pgfqpoint{1.265813in}{0.744977in}}%
\pgfpathlineto{\pgfqpoint{1.247272in}{0.746606in}}%
\pgfpathlineto{\pgfqpoint{1.228850in}{0.748553in}}%
\pgfpathlineto{\pgfqpoint{1.210566in}{0.750815in}}%
\pgfpathlineto{\pgfqpoint{1.211466in}{0.749597in}}%
\pgfpathlineto{\pgfqpoint{1.212363in}{0.748662in}}%
\pgfpathlineto{\pgfqpoint{1.213256in}{0.748007in}}%
\pgfpathlineto{\pgfqpoint{1.214147in}{0.747624in}}%
\pgfpathlineto{\pgfqpoint{1.231966in}{0.745419in}}%
\pgfpathlineto{\pgfqpoint{1.249920in}{0.743521in}}%
\pgfpathlineto{\pgfqpoint{1.267989in}{0.741932in}}%
\pgfpathlineto{\pgfqpoint{1.286152in}{0.740656in}}%
\pgfpathclose%
\pgfusepath{fill}%
\end{pgfscope}%
\begin{pgfscope}%
\pgfpathrectangle{\pgfqpoint{0.329460in}{0.284240in}}{\pgfqpoint{1.989680in}{1.989680in}}%
\pgfusepath{clip}%
\pgfsetbuttcap%
\pgfsetroundjoin%
\definecolor{currentfill}{rgb}{0.260571,0.246922,0.522828}%
\pgfsetfillcolor{currentfill}%
\pgfsetlinewidth{0.000000pt}%
\definecolor{currentstroke}{rgb}{0.000000,0.000000,0.000000}%
\pgfsetstrokecolor{currentstroke}%
\pgfsetdash{}{0pt}%
\pgfpathmoveto{\pgfqpoint{1.804000in}{0.902698in}}%
\pgfpathlineto{\pgfqpoint{1.806771in}{0.910049in}}%
\pgfpathlineto{\pgfqpoint{1.809556in}{0.917786in}}%
\pgfpathlineto{\pgfqpoint{1.812353in}{0.925915in}}%
\pgfpathlineto{\pgfqpoint{1.815165in}{0.934443in}}%
\pgfpathlineto{\pgfqpoint{1.800404in}{0.926799in}}%
\pgfpathlineto{\pgfqpoint{1.785144in}{0.919397in}}%
\pgfpathlineto{\pgfqpoint{1.769399in}{0.912247in}}%
\pgfpathlineto{\pgfqpoint{1.753185in}{0.905358in}}%
\pgfpathlineto{\pgfqpoint{1.750738in}{0.896969in}}%
\pgfpathlineto{\pgfqpoint{1.748304in}{0.888981in}}%
\pgfpathlineto{\pgfqpoint{1.745881in}{0.881386in}}%
\pgfpathlineto{\pgfqpoint{1.743470in}{0.874178in}}%
\pgfpathlineto{\pgfqpoint{1.759302in}{0.880933in}}%
\pgfpathlineto{\pgfqpoint{1.774678in}{0.887943in}}%
\pgfpathlineto{\pgfqpoint{1.789583in}{0.895201in}}%
\pgfpathlineto{\pgfqpoint{1.804000in}{0.902698in}}%
\pgfpathclose%
\pgfusepath{fill}%
\end{pgfscope}%
\begin{pgfscope}%
\pgfpathrectangle{\pgfqpoint{0.329460in}{0.284240in}}{\pgfqpoint{1.989680in}{1.989680in}}%
\pgfusepath{clip}%
\pgfsetbuttcap%
\pgfsetroundjoin%
\definecolor{currentfill}{rgb}{0.282327,0.094955,0.417331}%
\pgfsetfillcolor{currentfill}%
\pgfsetlinewidth{0.000000pt}%
\definecolor{currentstroke}{rgb}{0.000000,0.000000,0.000000}%
\pgfsetstrokecolor{currentstroke}%
\pgfsetdash{}{0pt}%
\pgfpathmoveto{\pgfqpoint{1.133771in}{0.771060in}}%
\pgfpathlineto{\pgfqpoint{1.132393in}{0.773849in}}%
\pgfpathlineto{\pgfqpoint{1.131010in}{0.776953in}}%
\pgfpathlineto{\pgfqpoint{1.129622in}{0.780379in}}%
\pgfpathlineto{\pgfqpoint{1.128228in}{0.784131in}}%
\pgfpathlineto{\pgfqpoint{1.110053in}{0.788102in}}%
\pgfpathlineto{\pgfqpoint{1.092153in}{0.792378in}}%
\pgfpathlineto{\pgfqpoint{1.074545in}{0.796955in}}%
\pgfpathlineto{\pgfqpoint{1.057251in}{0.801827in}}%
\pgfpathlineto{\pgfqpoint{1.059084in}{0.797974in}}%
\pgfpathlineto{\pgfqpoint{1.060909in}{0.794447in}}%
\pgfpathlineto{\pgfqpoint{1.062728in}{0.791241in}}%
\pgfpathlineto{\pgfqpoint{1.064540in}{0.788350in}}%
\pgfpathlineto{\pgfqpoint{1.081410in}{0.783590in}}%
\pgfpathlineto{\pgfqpoint{1.098585in}{0.779117in}}%
\pgfpathlineto{\pgfqpoint{1.116045in}{0.774939in}}%
\pgfpathlineto{\pgfqpoint{1.133771in}{0.771060in}}%
\pgfpathclose%
\pgfusepath{fill}%
\end{pgfscope}%
\begin{pgfscope}%
\pgfpathrectangle{\pgfqpoint{0.329460in}{0.284240in}}{\pgfqpoint{1.989680in}{1.989680in}}%
\pgfusepath{clip}%
\pgfsetbuttcap%
\pgfsetroundjoin%
\definecolor{currentfill}{rgb}{0.277941,0.056324,0.381191}%
\pgfsetfillcolor{currentfill}%
\pgfsetlinewidth{0.000000pt}%
\definecolor{currentstroke}{rgb}{0.000000,0.000000,0.000000}%
\pgfsetstrokecolor{currentstroke}%
\pgfsetdash{}{0pt}%
\pgfpathmoveto{\pgfqpoint{1.210566in}{0.750815in}}%
\pgfpathlineto{\pgfqpoint{1.209664in}{0.752321in}}%
\pgfpathlineto{\pgfqpoint{1.208758in}{0.754122in}}%
\pgfpathlineto{\pgfqpoint{1.207849in}{0.756222in}}%
\pgfpathlineto{\pgfqpoint{1.206936in}{0.758627in}}%
\pgfpathlineto{\pgfqpoint{1.188346in}{0.761264in}}%
\pgfpathlineto{\pgfqpoint{1.169942in}{0.764218in}}%
\pgfpathlineto{\pgfqpoint{1.151743in}{0.767485in}}%
\pgfpathlineto{\pgfqpoint{1.133771in}{0.771060in}}%
\pgfpathlineto{\pgfqpoint{1.135143in}{0.768581in}}%
\pgfpathlineto{\pgfqpoint{1.136511in}{0.766406in}}%
\pgfpathlineto{\pgfqpoint{1.137874in}{0.764531in}}%
\pgfpathlineto{\pgfqpoint{1.139232in}{0.762949in}}%
\pgfpathlineto{\pgfqpoint{1.156756in}{0.759460in}}%
\pgfpathlineto{\pgfqpoint{1.174499in}{0.756272in}}%
\pgfpathlineto{\pgfqpoint{1.192443in}{0.753389in}}%
\pgfpathlineto{\pgfqpoint{1.210566in}{0.750815in}}%
\pgfpathclose%
\pgfusepath{fill}%
\end{pgfscope}%
\begin{pgfscope}%
\pgfpathrectangle{\pgfqpoint{0.329460in}{0.284240in}}{\pgfqpoint{1.989680in}{1.989680in}}%
\pgfusepath{clip}%
\pgfsetbuttcap%
\pgfsetroundjoin%
\definecolor{currentfill}{rgb}{0.272594,0.025563,0.353093}%
\pgfsetfillcolor{currentfill}%
\pgfsetlinewidth{0.000000pt}%
\definecolor{currentstroke}{rgb}{0.000000,0.000000,0.000000}%
\pgfsetstrokecolor{currentstroke}%
\pgfsetdash{}{0pt}%
\pgfpathmoveto{\pgfqpoint{1.432372in}{0.741775in}}%
\pgfpathlineto{\pgfqpoint{1.432901in}{0.742120in}}%
\pgfpathlineto{\pgfqpoint{1.433431in}{0.742738in}}%
\pgfpathlineto{\pgfqpoint{1.433962in}{0.743635in}}%
\pgfpathlineto{\pgfqpoint{1.434496in}{0.744816in}}%
\pgfpathlineto{\pgfqpoint{1.415848in}{0.743542in}}%
\pgfpathlineto{\pgfqpoint{1.397127in}{0.742590in}}%
\pgfpathlineto{\pgfqpoint{1.378354in}{0.741960in}}%
\pgfpathlineto{\pgfqpoint{1.359549in}{0.741654in}}%
\pgfpathlineto{\pgfqpoint{1.359496in}{0.740493in}}%
\pgfpathlineto{\pgfqpoint{1.359442in}{0.739616in}}%
\pgfpathlineto{\pgfqpoint{1.359389in}{0.739018in}}%
\pgfpathlineto{\pgfqpoint{1.359336in}{0.738693in}}%
\pgfpathlineto{\pgfqpoint{1.377661in}{0.738991in}}%
\pgfpathlineto{\pgfqpoint{1.395956in}{0.739605in}}%
\pgfpathlineto{\pgfqpoint{1.414200in}{0.740533in}}%
\pgfpathlineto{\pgfqpoint{1.432372in}{0.741775in}}%
\pgfpathclose%
\pgfusepath{fill}%
\end{pgfscope}%
\begin{pgfscope}%
\pgfpathrectangle{\pgfqpoint{0.329460in}{0.284240in}}{\pgfqpoint{1.989680in}{1.989680in}}%
\pgfusepath{clip}%
\pgfsetbuttcap%
\pgfsetroundjoin%
\definecolor{currentfill}{rgb}{0.172719,0.448791,0.557885}%
\pgfsetfillcolor{currentfill}%
\pgfsetlinewidth{0.000000pt}%
\definecolor{currentstroke}{rgb}{0.000000,0.000000,0.000000}%
\pgfsetstrokecolor{currentstroke}%
\pgfsetdash{}{0pt}%
\pgfpathmoveto{\pgfqpoint{0.819590in}{1.044021in}}%
\pgfpathlineto{\pgfqpoint{0.816379in}{1.056569in}}%
\pgfpathlineto{\pgfqpoint{0.813150in}{1.069588in}}%
\pgfpathlineto{\pgfqpoint{0.809901in}{1.083086in}}%
\pgfpathlineto{\pgfqpoint{0.796529in}{1.092042in}}%
\pgfpathlineto{\pgfqpoint{0.783766in}{1.101199in}}%
\pgfpathlineto{\pgfqpoint{0.771624in}{1.110545in}}%
\pgfpathlineto{\pgfqpoint{0.760113in}{1.120070in}}%
\pgfpathlineto{\pgfqpoint{0.763641in}{1.106428in}}%
\pgfpathlineto{\pgfqpoint{0.767148in}{1.093263in}}%
\pgfpathlineto{\pgfqpoint{0.770634in}{1.080567in}}%
\pgfpathlineto{\pgfqpoint{0.781955in}{1.071154in}}%
\pgfpathlineto{\pgfqpoint{0.793895in}{1.061918in}}%
\pgfpathlineto{\pgfqpoint{0.806444in}{1.052870in}}%
\pgfpathlineto{\pgfqpoint{0.819590in}{1.044021in}}%
\pgfpathclose%
\pgfusepath{fill}%
\end{pgfscope}%
\begin{pgfscope}%
\pgfpathrectangle{\pgfqpoint{0.329460in}{0.284240in}}{\pgfqpoint{1.989680in}{1.989680in}}%
\pgfusepath{clip}%
\pgfsetbuttcap%
\pgfsetroundjoin%
\definecolor{currentfill}{rgb}{0.272594,0.025563,0.353093}%
\pgfsetfillcolor{currentfill}%
\pgfsetlinewidth{0.000000pt}%
\definecolor{currentstroke}{rgb}{0.000000,0.000000,0.000000}%
\pgfsetstrokecolor{currentstroke}%
\pgfsetdash{}{0pt}%
\pgfpathmoveto{\pgfqpoint{1.359336in}{0.738693in}}%
\pgfpathlineto{\pgfqpoint{1.359389in}{0.739018in}}%
\pgfpathlineto{\pgfqpoint{1.359442in}{0.739616in}}%
\pgfpathlineto{\pgfqpoint{1.359496in}{0.740493in}}%
\pgfpathlineto{\pgfqpoint{1.359549in}{0.741654in}}%
\pgfpathlineto{\pgfqpoint{1.340735in}{0.741672in}}%
\pgfpathlineto{\pgfqpoint{1.321933in}{0.742014in}}%
\pgfpathlineto{\pgfqpoint{1.303165in}{0.742679in}}%
\pgfpathlineto{\pgfqpoint{1.284451in}{0.743668in}}%
\pgfpathlineto{\pgfqpoint{1.284878in}{0.742494in}}%
\pgfpathlineto{\pgfqpoint{1.285304in}{0.741605in}}%
\pgfpathlineto{\pgfqpoint{1.285729in}{0.740993in}}%
\pgfpathlineto{\pgfqpoint{1.286152in}{0.740656in}}%
\pgfpathlineto{\pgfqpoint{1.304389in}{0.739693in}}%
\pgfpathlineto{\pgfqpoint{1.322679in}{0.739044in}}%
\pgfpathlineto{\pgfqpoint{1.341002in}{0.738710in}}%
\pgfpathlineto{\pgfqpoint{1.359336in}{0.738693in}}%
\pgfpathclose%
\pgfusepath{fill}%
\end{pgfscope}%
\begin{pgfscope}%
\pgfpathrectangle{\pgfqpoint{0.329460in}{0.284240in}}{\pgfqpoint{1.989680in}{1.989680in}}%
\pgfusepath{clip}%
\pgfsetbuttcap%
\pgfsetroundjoin%
\definecolor{currentfill}{rgb}{0.260571,0.246922,0.522828}%
\pgfsetfillcolor{currentfill}%
\pgfsetlinewidth{0.000000pt}%
\definecolor{currentstroke}{rgb}{0.000000,0.000000,0.000000}%
\pgfsetstrokecolor{currentstroke}%
\pgfsetdash{}{0pt}%
\pgfpathmoveto{\pgfqpoint{0.973347in}{0.868396in}}%
\pgfpathlineto{\pgfqpoint{0.971023in}{0.875575in}}%
\pgfpathlineto{\pgfqpoint{0.968687in}{0.883141in}}%
\pgfpathlineto{\pgfqpoint{0.966340in}{0.891100in}}%
\pgfpathlineto{\pgfqpoint{0.963981in}{0.899460in}}%
\pgfpathlineto{\pgfqpoint{0.947366in}{0.906110in}}%
\pgfpathlineto{\pgfqpoint{0.931204in}{0.913029in}}%
\pgfpathlineto{\pgfqpoint{0.915512in}{0.920207in}}%
\pgfpathlineto{\pgfqpoint{0.900306in}{0.927636in}}%
\pgfpathlineto{\pgfqpoint{0.903041in}{0.919141in}}%
\pgfpathlineto{\pgfqpoint{0.905762in}{0.911044in}}%
\pgfpathlineto{\pgfqpoint{0.908471in}{0.903340in}}%
\pgfpathlineto{\pgfqpoint{0.911166in}{0.896022in}}%
\pgfpathlineto{\pgfqpoint{0.926017in}{0.888738in}}%
\pgfpathlineto{\pgfqpoint{0.941342in}{0.881699in}}%
\pgfpathlineto{\pgfqpoint{0.957124in}{0.874916in}}%
\pgfpathlineto{\pgfqpoint{0.973347in}{0.868396in}}%
\pgfpathclose%
\pgfusepath{fill}%
\end{pgfscope}%
\begin{pgfscope}%
\pgfpathrectangle{\pgfqpoint{0.329460in}{0.284240in}}{\pgfqpoint{1.989680in}{1.989680in}}%
\pgfusepath{clip}%
\pgfsetbuttcap%
\pgfsetroundjoin%
\definecolor{currentfill}{rgb}{0.276194,0.190074,0.493001}%
\pgfsetfillcolor{currentfill}%
\pgfsetlinewidth{0.000000pt}%
\definecolor{currentstroke}{rgb}{0.000000,0.000000,0.000000}%
\pgfsetstrokecolor{currentstroke}%
\pgfsetdash{}{0pt}%
\pgfpathmoveto{\pgfqpoint{1.733937in}{0.849085in}}%
\pgfpathlineto{\pgfqpoint{1.736304in}{0.854810in}}%
\pgfpathlineto{\pgfqpoint{1.738682in}{0.860897in}}%
\pgfpathlineto{\pgfqpoint{1.741071in}{0.867351in}}%
\pgfpathlineto{\pgfqpoint{1.743470in}{0.874178in}}%
\pgfpathlineto{\pgfqpoint{1.727199in}{0.867688in}}%
\pgfpathlineto{\pgfqpoint{1.710505in}{0.861471in}}%
\pgfpathlineto{\pgfqpoint{1.693407in}{0.855534in}}%
\pgfpathlineto{\pgfqpoint{1.675922in}{0.849884in}}%
\pgfpathlineto{\pgfqpoint{1.673928in}{0.843181in}}%
\pgfpathlineto{\pgfqpoint{1.671944in}{0.836852in}}%
\pgfpathlineto{\pgfqpoint{1.669968in}{0.830892in}}%
\pgfpathlineto{\pgfqpoint{1.668002in}{0.825294in}}%
\pgfpathlineto{\pgfqpoint{1.685067in}{0.830826in}}%
\pgfpathlineto{\pgfqpoint{1.701757in}{0.836640in}}%
\pgfpathlineto{\pgfqpoint{1.718052in}{0.842729in}}%
\pgfpathlineto{\pgfqpoint{1.733937in}{0.849085in}}%
\pgfpathclose%
\pgfusepath{fill}%
\end{pgfscope}%
\begin{pgfscope}%
\pgfpathrectangle{\pgfqpoint{0.329460in}{0.284240in}}{\pgfqpoint{1.989680in}{1.989680in}}%
\pgfusepath{clip}%
\pgfsetbuttcap%
\pgfsetroundjoin%
\definecolor{currentfill}{rgb}{0.277941,0.056324,0.381191}%
\pgfsetfillcolor{currentfill}%
\pgfsetlinewidth{0.000000pt}%
\definecolor{currentstroke}{rgb}{0.000000,0.000000,0.000000}%
\pgfsetstrokecolor{currentstroke}%
\pgfsetdash{}{0pt}%
\pgfpathmoveto{\pgfqpoint{1.507927in}{0.753087in}}%
\pgfpathlineto{\pgfqpoint{1.508933in}{0.754608in}}%
\pgfpathlineto{\pgfqpoint{1.509942in}{0.756423in}}%
\pgfpathlineto{\pgfqpoint{1.510955in}{0.758537in}}%
\pgfpathlineto{\pgfqpoint{1.511972in}{0.760955in}}%
\pgfpathlineto{\pgfqpoint{1.493363in}{0.758353in}}%
\pgfpathlineto{\pgfqpoint{1.474592in}{0.756071in}}%
\pgfpathlineto{\pgfqpoint{1.455680in}{0.754113in}}%
\pgfpathlineto{\pgfqpoint{1.436648in}{0.752480in}}%
\pgfpathlineto{\pgfqpoint{1.436107in}{0.750112in}}%
\pgfpathlineto{\pgfqpoint{1.435568in}{0.748049in}}%
\pgfpathlineto{\pgfqpoint{1.435031in}{0.746285in}}%
\pgfpathlineto{\pgfqpoint{1.434496in}{0.744816in}}%
\pgfpathlineto{\pgfqpoint{1.453048in}{0.746409in}}%
\pgfpathlineto{\pgfqpoint{1.471485in}{0.748321in}}%
\pgfpathlineto{\pgfqpoint{1.489785in}{0.750548in}}%
\pgfpathlineto{\pgfqpoint{1.507927in}{0.753087in}}%
\pgfpathclose%
\pgfusepath{fill}%
\end{pgfscope}%
\begin{pgfscope}%
\pgfpathrectangle{\pgfqpoint{0.329460in}{0.284240in}}{\pgfqpoint{1.989680in}{1.989680in}}%
\pgfusepath{clip}%
\pgfsetbuttcap%
\pgfsetroundjoin%
\definecolor{currentfill}{rgb}{0.201239,0.383670,0.554294}%
\pgfsetfillcolor{currentfill}%
\pgfsetlinewidth{0.000000pt}%
\definecolor{currentstroke}{rgb}{0.000000,0.000000,0.000000}%
\pgfsetstrokecolor{currentstroke}%
\pgfsetdash{}{0pt}%
\pgfpathmoveto{\pgfqpoint{1.881576in}{1.006107in}}%
\pgfpathlineto{\pgfqpoint{1.884780in}{1.016882in}}%
\pgfpathlineto{\pgfqpoint{1.888002in}{1.028097in}}%
\pgfpathlineto{\pgfqpoint{1.891242in}{1.039759in}}%
\pgfpathlineto{\pgfqpoint{1.894500in}{1.051877in}}%
\pgfpathlineto{\pgfqpoint{1.881288in}{1.043050in}}%
\pgfpathlineto{\pgfqpoint{1.867492in}{1.034434in}}%
\pgfpathlineto{\pgfqpoint{1.853126in}{1.026039in}}%
\pgfpathlineto{\pgfqpoint{1.838202in}{1.017875in}}%
\pgfpathlineto{\pgfqpoint{1.835266in}{1.005899in}}%
\pgfpathlineto{\pgfqpoint{1.832347in}{0.994381in}}%
\pgfpathlineto{\pgfqpoint{1.829444in}{0.983311in}}%
\pgfpathlineto{\pgfqpoint{1.826558in}{0.972683in}}%
\pgfpathlineto{\pgfqpoint{1.841140in}{0.980708in}}%
\pgfpathlineto{\pgfqpoint{1.855179in}{0.988960in}}%
\pgfpathlineto{\pgfqpoint{1.868662in}{0.997430in}}%
\pgfpathlineto{\pgfqpoint{1.881576in}{1.006107in}}%
\pgfpathclose%
\pgfusepath{fill}%
\end{pgfscope}%
\begin{pgfscope}%
\pgfpathrectangle{\pgfqpoint{0.329460in}{0.284240in}}{\pgfqpoint{1.989680in}{1.989680in}}%
\pgfusepath{clip}%
\pgfsetbuttcap%
\pgfsetroundjoin%
\definecolor{currentfill}{rgb}{0.277941,0.056324,0.381191}%
\pgfsetfillcolor{currentfill}%
\pgfsetlinewidth{0.000000pt}%
\definecolor{currentstroke}{rgb}{0.000000,0.000000,0.000000}%
\pgfsetstrokecolor{currentstroke}%
\pgfsetdash{}{0pt}%
\pgfpathmoveto{\pgfqpoint{1.284451in}{0.743668in}}%
\pgfpathlineto{\pgfqpoint{1.284022in}{0.745130in}}%
\pgfpathlineto{\pgfqpoint{1.283592in}{0.746886in}}%
\pgfpathlineto{\pgfqpoint{1.283160in}{0.748942in}}%
\pgfpathlineto{\pgfqpoint{1.282726in}{0.751303in}}%
\pgfpathlineto{\pgfqpoint{1.263607in}{0.752645in}}%
\pgfpathlineto{\pgfqpoint{1.244588in}{0.754314in}}%
\pgfpathlineto{\pgfqpoint{1.225690in}{0.756309in}}%
\pgfpathlineto{\pgfqpoint{1.206936in}{0.758627in}}%
\pgfpathlineto{\pgfqpoint{1.207849in}{0.756222in}}%
\pgfpathlineto{\pgfqpoint{1.208758in}{0.754122in}}%
\pgfpathlineto{\pgfqpoint{1.209664in}{0.752321in}}%
\pgfpathlineto{\pgfqpoint{1.210566in}{0.750815in}}%
\pgfpathlineto{\pgfqpoint{1.228850in}{0.748553in}}%
\pgfpathlineto{\pgfqpoint{1.247272in}{0.746606in}}%
\pgfpathlineto{\pgfqpoint{1.265813in}{0.744977in}}%
\pgfpathlineto{\pgfqpoint{1.284451in}{0.743668in}}%
\pgfpathclose%
\pgfusepath{fill}%
\end{pgfscope}%
\begin{pgfscope}%
\pgfpathrectangle{\pgfqpoint{0.329460in}{0.284240in}}{\pgfqpoint{1.989680in}{1.989680in}}%
\pgfusepath{clip}%
\pgfsetbuttcap%
\pgfsetroundjoin%
\definecolor{currentfill}{rgb}{0.282884,0.135920,0.453427}%
\pgfsetfillcolor{currentfill}%
\pgfsetlinewidth{0.000000pt}%
\definecolor{currentstroke}{rgb}{0.000000,0.000000,0.000000}%
\pgfsetstrokecolor{currentstroke}%
\pgfsetdash{}{0pt}%
\pgfpathmoveto{\pgfqpoint{1.660219in}{0.806400in}}%
\pgfpathlineto{\pgfqpoint{1.662153in}{0.810611in}}%
\pgfpathlineto{\pgfqpoint{1.664094in}{0.815159in}}%
\pgfpathlineto{\pgfqpoint{1.666044in}{0.820052in}}%
\pgfpathlineto{\pgfqpoint{1.668002in}{0.825294in}}%
\pgfpathlineto{\pgfqpoint{1.650579in}{0.820051in}}%
\pgfpathlineto{\pgfqpoint{1.632817in}{0.815104in}}%
\pgfpathlineto{\pgfqpoint{1.614736in}{0.810458in}}%
\pgfpathlineto{\pgfqpoint{1.596355in}{0.806120in}}%
\pgfpathlineto{\pgfqpoint{1.594835in}{0.800982in}}%
\pgfpathlineto{\pgfqpoint{1.593322in}{0.796194in}}%
\pgfpathlineto{\pgfqpoint{1.591816in}{0.791750in}}%
\pgfpathlineto{\pgfqpoint{1.590316in}{0.787645in}}%
\pgfpathlineto{\pgfqpoint{1.608248in}{0.791888in}}%
\pgfpathlineto{\pgfqpoint{1.625888in}{0.796432in}}%
\pgfpathlineto{\pgfqpoint{1.643219in}{0.801271in}}%
\pgfpathlineto{\pgfqpoint{1.660219in}{0.806400in}}%
\pgfpathclose%
\pgfusepath{fill}%
\end{pgfscope}%
\begin{pgfscope}%
\pgfpathrectangle{\pgfqpoint{0.329460in}{0.284240in}}{\pgfqpoint{1.989680in}{1.989680in}}%
\pgfusepath{clip}%
\pgfsetbuttcap%
\pgfsetroundjoin%
\definecolor{currentfill}{rgb}{0.282327,0.094955,0.417331}%
\pgfsetfillcolor{currentfill}%
\pgfsetlinewidth{0.000000pt}%
\definecolor{currentstroke}{rgb}{0.000000,0.000000,0.000000}%
\pgfsetstrokecolor{currentstroke}%
\pgfsetdash{}{0pt}%
\pgfpathmoveto{\pgfqpoint{1.584373in}{0.774493in}}%
\pgfpathlineto{\pgfqpoint{1.585850in}{0.777302in}}%
\pgfpathlineto{\pgfqpoint{1.587333in}{0.780427in}}%
\pgfpathlineto{\pgfqpoint{1.588821in}{0.783873in}}%
\pgfpathlineto{\pgfqpoint{1.590316in}{0.787645in}}%
\pgfpathlineto{\pgfqpoint{1.572112in}{0.783709in}}%
\pgfpathlineto{\pgfqpoint{1.553658in}{0.780085in}}%
\pgfpathlineto{\pgfqpoint{1.534972in}{0.776777in}}%
\pgfpathlineto{\pgfqpoint{1.516078in}{0.773789in}}%
\pgfpathlineto{\pgfqpoint{1.515045in}{0.770095in}}%
\pgfpathlineto{\pgfqpoint{1.514017in}{0.766729in}}%
\pgfpathlineto{\pgfqpoint{1.512993in}{0.763684in}}%
\pgfpathlineto{\pgfqpoint{1.511972in}{0.760955in}}%
\pgfpathlineto{\pgfqpoint{1.530398in}{0.763874in}}%
\pgfpathlineto{\pgfqpoint{1.548621in}{0.767106in}}%
\pgfpathlineto{\pgfqpoint{1.566619in}{0.770647in}}%
\pgfpathlineto{\pgfqpoint{1.584373in}{0.774493in}}%
\pgfpathclose%
\pgfusepath{fill}%
\end{pgfscope}%
\begin{pgfscope}%
\pgfpathrectangle{\pgfqpoint{0.329460in}{0.284240in}}{\pgfqpoint{1.989680in}{1.989680in}}%
\pgfusepath{clip}%
\pgfsetbuttcap%
\pgfsetroundjoin%
\definecolor{currentfill}{rgb}{0.276194,0.190074,0.493001}%
\pgfsetfillcolor{currentfill}%
\pgfsetlinewidth{0.000000pt}%
\definecolor{currentstroke}{rgb}{0.000000,0.000000,0.000000}%
\pgfsetstrokecolor{currentstroke}%
\pgfsetdash{}{0pt}%
\pgfpathmoveto{\pgfqpoint{1.049843in}{0.820619in}}%
\pgfpathlineto{\pgfqpoint{1.047972in}{0.826192in}}%
\pgfpathlineto{\pgfqpoint{1.046091in}{0.832127in}}%
\pgfpathlineto{\pgfqpoint{1.044203in}{0.838431in}}%
\pgfpathlineto{\pgfqpoint{1.042305in}{0.845110in}}%
\pgfpathlineto{\pgfqpoint{1.024492in}{0.850498in}}%
\pgfpathlineto{\pgfqpoint{1.007049in}{0.856179in}}%
\pgfpathlineto{\pgfqpoint{0.989995in}{0.862148in}}%
\pgfpathlineto{\pgfqpoint{0.973347in}{0.868396in}}%
\pgfpathlineto{\pgfqpoint{0.975660in}{0.861598in}}%
\pgfpathlineto{\pgfqpoint{0.977963in}{0.855174in}}%
\pgfpathlineto{\pgfqpoint{0.980255in}{0.849117in}}%
\pgfpathlineto{\pgfqpoint{0.982537in}{0.843422in}}%
\pgfpathlineto{\pgfqpoint{0.998788in}{0.837303in}}%
\pgfpathlineto{\pgfqpoint{1.015435in}{0.831458in}}%
\pgfpathlineto{\pgfqpoint{1.032459in}{0.825895in}}%
\pgfpathlineto{\pgfqpoint{1.049843in}{0.820619in}}%
\pgfpathclose%
\pgfusepath{fill}%
\end{pgfscope}%
\begin{pgfscope}%
\pgfpathrectangle{\pgfqpoint{0.329460in}{0.284240in}}{\pgfqpoint{1.989680in}{1.989680in}}%
\pgfusepath{clip}%
\pgfsetbuttcap%
\pgfsetroundjoin%
\definecolor{currentfill}{rgb}{0.277941,0.056324,0.381191}%
\pgfsetfillcolor{currentfill}%
\pgfsetlinewidth{0.000000pt}%
\definecolor{currentstroke}{rgb}{0.000000,0.000000,0.000000}%
\pgfsetstrokecolor{currentstroke}%
\pgfsetdash{}{0pt}%
\pgfpathmoveto{\pgfqpoint{1.434496in}{0.744816in}}%
\pgfpathlineto{\pgfqpoint{1.435031in}{0.746285in}}%
\pgfpathlineto{\pgfqpoint{1.435568in}{0.748049in}}%
\pgfpathlineto{\pgfqpoint{1.436107in}{0.750112in}}%
\pgfpathlineto{\pgfqpoint{1.436648in}{0.752480in}}%
\pgfpathlineto{\pgfqpoint{1.417519in}{0.751175in}}%
\pgfpathlineto{\pgfqpoint{1.398315in}{0.750199in}}%
\pgfpathlineto{\pgfqpoint{1.379056in}{0.749554in}}%
\pgfpathlineto{\pgfqpoint{1.359766in}{0.749240in}}%
\pgfpathlineto{\pgfqpoint{1.359711in}{0.746892in}}%
\pgfpathlineto{\pgfqpoint{1.359657in}{0.744848in}}%
\pgfpathlineto{\pgfqpoint{1.359603in}{0.743104in}}%
\pgfpathlineto{\pgfqpoint{1.359549in}{0.741654in}}%
\pgfpathlineto{\pgfqpoint{1.378354in}{0.741960in}}%
\pgfpathlineto{\pgfqpoint{1.397127in}{0.742590in}}%
\pgfpathlineto{\pgfqpoint{1.415848in}{0.743542in}}%
\pgfpathlineto{\pgfqpoint{1.434496in}{0.744816in}}%
\pgfpathclose%
\pgfusepath{fill}%
\end{pgfscope}%
\begin{pgfscope}%
\pgfpathrectangle{\pgfqpoint{0.329460in}{0.284240in}}{\pgfqpoint{1.989680in}{1.989680in}}%
\pgfusepath{clip}%
\pgfsetbuttcap%
\pgfsetroundjoin%
\definecolor{currentfill}{rgb}{0.282327,0.094955,0.417331}%
\pgfsetfillcolor{currentfill}%
\pgfsetlinewidth{0.000000pt}%
\definecolor{currentstroke}{rgb}{0.000000,0.000000,0.000000}%
\pgfsetstrokecolor{currentstroke}%
\pgfsetdash{}{0pt}%
\pgfpathmoveto{\pgfqpoint{1.206936in}{0.758627in}}%
\pgfpathlineto{\pgfqpoint{1.206020in}{0.761342in}}%
\pgfpathlineto{\pgfqpoint{1.205101in}{0.764373in}}%
\pgfpathlineto{\pgfqpoint{1.204178in}{0.767725in}}%
\pgfpathlineto{\pgfqpoint{1.203251in}{0.771405in}}%
\pgfpathlineto{\pgfqpoint{1.184188in}{0.774105in}}%
\pgfpathlineto{\pgfqpoint{1.165316in}{0.777128in}}%
\pgfpathlineto{\pgfqpoint{1.146655in}{0.780472in}}%
\pgfpathlineto{\pgfqpoint{1.128228in}{0.784131in}}%
\pgfpathlineto{\pgfqpoint{1.129622in}{0.780379in}}%
\pgfpathlineto{\pgfqpoint{1.131010in}{0.776953in}}%
\pgfpathlineto{\pgfqpoint{1.132393in}{0.773849in}}%
\pgfpathlineto{\pgfqpoint{1.133771in}{0.771060in}}%
\pgfpathlineto{\pgfqpoint{1.151743in}{0.767485in}}%
\pgfpathlineto{\pgfqpoint{1.169942in}{0.764218in}}%
\pgfpathlineto{\pgfqpoint{1.188346in}{0.761264in}}%
\pgfpathlineto{\pgfqpoint{1.206936in}{0.758627in}}%
\pgfpathclose%
\pgfusepath{fill}%
\end{pgfscope}%
\begin{pgfscope}%
\pgfpathrectangle{\pgfqpoint{0.329460in}{0.284240in}}{\pgfqpoint{1.989680in}{1.989680in}}%
\pgfusepath{clip}%
\pgfsetbuttcap%
\pgfsetroundjoin%
\definecolor{currentfill}{rgb}{0.277941,0.056324,0.381191}%
\pgfsetfillcolor{currentfill}%
\pgfsetlinewidth{0.000000pt}%
\definecolor{currentstroke}{rgb}{0.000000,0.000000,0.000000}%
\pgfsetstrokecolor{currentstroke}%
\pgfsetdash{}{0pt}%
\pgfpathmoveto{\pgfqpoint{1.359549in}{0.741654in}}%
\pgfpathlineto{\pgfqpoint{1.359603in}{0.743104in}}%
\pgfpathlineto{\pgfqpoint{1.359657in}{0.744848in}}%
\pgfpathlineto{\pgfqpoint{1.359711in}{0.746892in}}%
\pgfpathlineto{\pgfqpoint{1.359766in}{0.749240in}}%
\pgfpathlineto{\pgfqpoint{1.340465in}{0.749259in}}%
\pgfpathlineto{\pgfqpoint{1.321177in}{0.749609in}}%
\pgfpathlineto{\pgfqpoint{1.301924in}{0.750291in}}%
\pgfpathlineto{\pgfqpoint{1.282726in}{0.751303in}}%
\pgfpathlineto{\pgfqpoint{1.283160in}{0.748942in}}%
\pgfpathlineto{\pgfqpoint{1.283592in}{0.746886in}}%
\pgfpathlineto{\pgfqpoint{1.284022in}{0.745130in}}%
\pgfpathlineto{\pgfqpoint{1.284451in}{0.743668in}}%
\pgfpathlineto{\pgfqpoint{1.303165in}{0.742679in}}%
\pgfpathlineto{\pgfqpoint{1.321933in}{0.742014in}}%
\pgfpathlineto{\pgfqpoint{1.340735in}{0.741672in}}%
\pgfpathlineto{\pgfqpoint{1.359549in}{0.741654in}}%
\pgfpathclose%
\pgfusepath{fill}%
\end{pgfscope}%
\begin{pgfscope}%
\pgfpathrectangle{\pgfqpoint{0.329460in}{0.284240in}}{\pgfqpoint{1.989680in}{1.989680in}}%
\pgfusepath{clip}%
\pgfsetbuttcap%
\pgfsetroundjoin%
\definecolor{currentfill}{rgb}{0.282884,0.135920,0.453427}%
\pgfsetfillcolor{currentfill}%
\pgfsetlinewidth{0.000000pt}%
\definecolor{currentstroke}{rgb}{0.000000,0.000000,0.000000}%
\pgfsetstrokecolor{currentstroke}%
\pgfsetdash{}{0pt}%
\pgfpathmoveto{\pgfqpoint{1.128228in}{0.784131in}}%
\pgfpathlineto{\pgfqpoint{1.126828in}{0.788217in}}%
\pgfpathlineto{\pgfqpoint{1.125423in}{0.792640in}}%
\pgfpathlineto{\pgfqpoint{1.124011in}{0.797409in}}%
\pgfpathlineto{\pgfqpoint{1.122594in}{0.802527in}}%
\pgfpathlineto{\pgfqpoint{1.103964in}{0.806587in}}%
\pgfpathlineto{\pgfqpoint{1.085615in}{0.810959in}}%
\pgfpathlineto{\pgfqpoint{1.067569in}{0.815638in}}%
\pgfpathlineto{\pgfqpoint{1.049843in}{0.820619in}}%
\pgfpathlineto{\pgfqpoint{1.051707in}{0.815402in}}%
\pgfpathlineto{\pgfqpoint{1.053563in}{0.810535in}}%
\pgfpathlineto{\pgfqpoint{1.055411in}{0.806012in}}%
\pgfpathlineto{\pgfqpoint{1.057251in}{0.801827in}}%
\pgfpathlineto{\pgfqpoint{1.074545in}{0.796955in}}%
\pgfpathlineto{\pgfqpoint{1.092153in}{0.792378in}}%
\pgfpathlineto{\pgfqpoint{1.110053in}{0.788102in}}%
\pgfpathlineto{\pgfqpoint{1.128228in}{0.784131in}}%
\pgfpathclose%
\pgfusepath{fill}%
\end{pgfscope}%
\begin{pgfscope}%
\pgfpathrectangle{\pgfqpoint{0.329460in}{0.284240in}}{\pgfqpoint{1.989680in}{1.989680in}}%
\pgfusepath{clip}%
\pgfsetbuttcap%
\pgfsetroundjoin%
\definecolor{currentfill}{rgb}{0.201239,0.383670,0.554294}%
\pgfsetfillcolor{currentfill}%
\pgfsetlinewidth{0.000000pt}%
\definecolor{currentstroke}{rgb}{0.000000,0.000000,0.000000}%
\pgfsetstrokecolor{currentstroke}%
\pgfsetdash{}{0pt}%
\pgfpathmoveto{\pgfqpoint{0.889223in}{0.965750in}}%
\pgfpathlineto{\pgfqpoint{0.886415in}{0.976346in}}%
\pgfpathlineto{\pgfqpoint{0.883591in}{0.987385in}}%
\pgfpathlineto{\pgfqpoint{0.880751in}{0.998874in}}%
\pgfpathlineto{\pgfqpoint{0.877895in}{1.010820in}}%
\pgfpathlineto{\pgfqpoint{0.862488in}{1.018770in}}%
\pgfpathlineto{\pgfqpoint{0.847625in}{1.026961in}}%
\pgfpathlineto{\pgfqpoint{0.833321in}{1.035381in}}%
\pgfpathlineto{\pgfqpoint{0.819590in}{1.044021in}}%
\pgfpathlineto{\pgfqpoint{0.822782in}{1.031936in}}%
\pgfpathlineto{\pgfqpoint{0.825955in}{1.020307in}}%
\pgfpathlineto{\pgfqpoint{0.829111in}{1.009125in}}%
\pgfpathlineto{\pgfqpoint{0.832250in}{0.998384in}}%
\pgfpathlineto{\pgfqpoint{0.845671in}{0.989891in}}%
\pgfpathlineto{\pgfqpoint{0.859649in}{0.981614in}}%
\pgfpathlineto{\pgfqpoint{0.874171in}{0.973564in}}%
\pgfpathlineto{\pgfqpoint{0.889223in}{0.965750in}}%
\pgfpathclose%
\pgfusepath{fill}%
\end{pgfscope}%
\begin{pgfscope}%
\pgfpathrectangle{\pgfqpoint{0.329460in}{0.284240in}}{\pgfqpoint{1.989680in}{1.989680in}}%
\pgfusepath{clip}%
\pgfsetbuttcap%
\pgfsetroundjoin%
\definecolor{currentfill}{rgb}{0.233603,0.313828,0.543914}%
\pgfsetfillcolor{currentfill}%
\pgfsetlinewidth{0.000000pt}%
\definecolor{currentstroke}{rgb}{0.000000,0.000000,0.000000}%
\pgfsetstrokecolor{currentstroke}%
\pgfsetdash{}{0pt}%
\pgfpathmoveto{\pgfqpoint{1.815165in}{0.934443in}}%
\pgfpathlineto{\pgfqpoint{1.817991in}{0.943377in}}%
\pgfpathlineto{\pgfqpoint{1.820832in}{0.952724in}}%
\pgfpathlineto{\pgfqpoint{1.823687in}{0.962490in}}%
\pgfpathlineto{\pgfqpoint{1.826558in}{0.972683in}}%
\pgfpathlineto{\pgfqpoint{1.811448in}{0.964896in}}%
\pgfpathlineto{\pgfqpoint{1.795824in}{0.957356in}}%
\pgfpathlineto{\pgfqpoint{1.779703in}{0.950072in}}%
\pgfpathlineto{\pgfqpoint{1.763101in}{0.943054in}}%
\pgfpathlineto{\pgfqpoint{1.760602in}{0.932994in}}%
\pgfpathlineto{\pgfqpoint{1.758117in}{0.923363in}}%
\pgfpathlineto{\pgfqpoint{1.755644in}{0.914154in}}%
\pgfpathlineto{\pgfqpoint{1.753185in}{0.905358in}}%
\pgfpathlineto{\pgfqpoint{1.769399in}{0.912247in}}%
\pgfpathlineto{\pgfqpoint{1.785144in}{0.919397in}}%
\pgfpathlineto{\pgfqpoint{1.800404in}{0.926799in}}%
\pgfpathlineto{\pgfqpoint{1.815165in}{0.934443in}}%
\pgfpathclose%
\pgfusepath{fill}%
\end{pgfscope}%
\begin{pgfscope}%
\pgfpathrectangle{\pgfqpoint{0.329460in}{0.284240in}}{\pgfqpoint{1.989680in}{1.989680in}}%
\pgfusepath{clip}%
\pgfsetbuttcap%
\pgfsetroundjoin%
\definecolor{currentfill}{rgb}{0.282327,0.094955,0.417331}%
\pgfsetfillcolor{currentfill}%
\pgfsetlinewidth{0.000000pt}%
\definecolor{currentstroke}{rgb}{0.000000,0.000000,0.000000}%
\pgfsetstrokecolor{currentstroke}%
\pgfsetdash{}{0pt}%
\pgfpathmoveto{\pgfqpoint{1.511972in}{0.760955in}}%
\pgfpathlineto{\pgfqpoint{1.512993in}{0.763684in}}%
\pgfpathlineto{\pgfqpoint{1.514017in}{0.766729in}}%
\pgfpathlineto{\pgfqpoint{1.515045in}{0.770095in}}%
\pgfpathlineto{\pgfqpoint{1.516078in}{0.773789in}}%
\pgfpathlineto{\pgfqpoint{1.496995in}{0.771125in}}%
\pgfpathlineto{\pgfqpoint{1.477745in}{0.768790in}}%
\pgfpathlineto{\pgfqpoint{1.458351in}{0.766785in}}%
\pgfpathlineto{\pgfqpoint{1.438833in}{0.765113in}}%
\pgfpathlineto{\pgfqpoint{1.438284in}{0.761469in}}%
\pgfpathlineto{\pgfqpoint{1.437737in}{0.758153in}}%
\pgfpathlineto{\pgfqpoint{1.437191in}{0.755158in}}%
\pgfpathlineto{\pgfqpoint{1.436648in}{0.752480in}}%
\pgfpathlineto{\pgfqpoint{1.455680in}{0.754113in}}%
\pgfpathlineto{\pgfqpoint{1.474592in}{0.756071in}}%
\pgfpathlineto{\pgfqpoint{1.493363in}{0.758353in}}%
\pgfpathlineto{\pgfqpoint{1.511972in}{0.760955in}}%
\pgfpathclose%
\pgfusepath{fill}%
\end{pgfscope}%
\begin{pgfscope}%
\pgfpathrectangle{\pgfqpoint{0.329460in}{0.284240in}}{\pgfqpoint{1.989680in}{1.989680in}}%
\pgfusepath{clip}%
\pgfsetbuttcap%
\pgfsetroundjoin%
\definecolor{currentfill}{rgb}{0.282327,0.094955,0.417331}%
\pgfsetfillcolor{currentfill}%
\pgfsetlinewidth{0.000000pt}%
\definecolor{currentstroke}{rgb}{0.000000,0.000000,0.000000}%
\pgfsetstrokecolor{currentstroke}%
\pgfsetdash{}{0pt}%
\pgfpathmoveto{\pgfqpoint{1.282726in}{0.751303in}}%
\pgfpathlineto{\pgfqpoint{1.282291in}{0.753975in}}%
\pgfpathlineto{\pgfqpoint{1.281854in}{0.756963in}}%
\pgfpathlineto{\pgfqpoint{1.281416in}{0.760272in}}%
\pgfpathlineto{\pgfqpoint{1.280975in}{0.763909in}}%
\pgfpathlineto{\pgfqpoint{1.261368in}{0.765282in}}%
\pgfpathlineto{\pgfqpoint{1.241863in}{0.766991in}}%
\pgfpathlineto{\pgfqpoint{1.222484in}{0.769033in}}%
\pgfpathlineto{\pgfqpoint{1.203251in}{0.771405in}}%
\pgfpathlineto{\pgfqpoint{1.204178in}{0.767725in}}%
\pgfpathlineto{\pgfqpoint{1.205101in}{0.764373in}}%
\pgfpathlineto{\pgfqpoint{1.206020in}{0.761342in}}%
\pgfpathlineto{\pgfqpoint{1.206936in}{0.758627in}}%
\pgfpathlineto{\pgfqpoint{1.225690in}{0.756309in}}%
\pgfpathlineto{\pgfqpoint{1.244588in}{0.754314in}}%
\pgfpathlineto{\pgfqpoint{1.263607in}{0.752645in}}%
\pgfpathlineto{\pgfqpoint{1.282726in}{0.751303in}}%
\pgfpathclose%
\pgfusepath{fill}%
\end{pgfscope}%
\begin{pgfscope}%
\pgfpathrectangle{\pgfqpoint{0.329460in}{0.284240in}}{\pgfqpoint{1.989680in}{1.989680in}}%
\pgfusepath{clip}%
\pgfsetbuttcap%
\pgfsetroundjoin%
\definecolor{currentfill}{rgb}{0.260571,0.246922,0.522828}%
\pgfsetfillcolor{currentfill}%
\pgfsetlinewidth{0.000000pt}%
\definecolor{currentstroke}{rgb}{0.000000,0.000000,0.000000}%
\pgfsetstrokecolor{currentstroke}%
\pgfsetdash{}{0pt}%
\pgfpathmoveto{\pgfqpoint{1.743470in}{0.874178in}}%
\pgfpathlineto{\pgfqpoint{1.745881in}{0.881386in}}%
\pgfpathlineto{\pgfqpoint{1.748304in}{0.888981in}}%
\pgfpathlineto{\pgfqpoint{1.750738in}{0.896969in}}%
\pgfpathlineto{\pgfqpoint{1.753185in}{0.905358in}}%
\pgfpathlineto{\pgfqpoint{1.736520in}{0.898738in}}%
\pgfpathlineto{\pgfqpoint{1.719421in}{0.892396in}}%
\pgfpathlineto{\pgfqpoint{1.701906in}{0.886340in}}%
\pgfpathlineto{\pgfqpoint{1.683994in}{0.880577in}}%
\pgfpathlineto{\pgfqpoint{1.681961in}{0.872308in}}%
\pgfpathlineto{\pgfqpoint{1.679938in}{0.864441in}}%
\pgfpathlineto{\pgfqpoint{1.677925in}{0.856969in}}%
\pgfpathlineto{\pgfqpoint{1.675922in}{0.849884in}}%
\pgfpathlineto{\pgfqpoint{1.693407in}{0.855534in}}%
\pgfpathlineto{\pgfqpoint{1.710505in}{0.861471in}}%
\pgfpathlineto{\pgfqpoint{1.727199in}{0.867688in}}%
\pgfpathlineto{\pgfqpoint{1.743470in}{0.874178in}}%
\pgfpathclose%
\pgfusepath{fill}%
\end{pgfscope}%
\begin{pgfscope}%
\pgfpathrectangle{\pgfqpoint{0.329460in}{0.284240in}}{\pgfqpoint{1.989680in}{1.989680in}}%
\pgfusepath{clip}%
\pgfsetbuttcap%
\pgfsetroundjoin%
\definecolor{currentfill}{rgb}{0.233603,0.313828,0.543914}%
\pgfsetfillcolor{currentfill}%
\pgfsetlinewidth{0.000000pt}%
\definecolor{currentstroke}{rgb}{0.000000,0.000000,0.000000}%
\pgfsetstrokecolor{currentstroke}%
\pgfsetdash{}{0pt}%
\pgfpathmoveto{\pgfqpoint{0.963981in}{0.899460in}}%
\pgfpathlineto{\pgfqpoint{0.961611in}{0.908228in}}%
\pgfpathlineto{\pgfqpoint{0.959227in}{0.917409in}}%
\pgfpathlineto{\pgfqpoint{0.956831in}{0.927013in}}%
\pgfpathlineto{\pgfqpoint{0.954422in}{0.937045in}}%
\pgfpathlineto{\pgfqpoint{0.937407in}{0.943820in}}%
\pgfpathlineto{\pgfqpoint{0.920857in}{0.950869in}}%
\pgfpathlineto{\pgfqpoint{0.904790in}{0.958182in}}%
\pgfpathlineto{\pgfqpoint{0.889223in}{0.965750in}}%
\pgfpathlineto{\pgfqpoint{0.892016in}{0.955587in}}%
\pgfpathlineto{\pgfqpoint{0.894794in}{0.945853in}}%
\pgfpathlineto{\pgfqpoint{0.897557in}{0.936538in}}%
\pgfpathlineto{\pgfqpoint{0.900306in}{0.927636in}}%
\pgfpathlineto{\pgfqpoint{0.915512in}{0.920207in}}%
\pgfpathlineto{\pgfqpoint{0.931204in}{0.913029in}}%
\pgfpathlineto{\pgfqpoint{0.947366in}{0.906110in}}%
\pgfpathlineto{\pgfqpoint{0.963981in}{0.899460in}}%
\pgfpathclose%
\pgfusepath{fill}%
\end{pgfscope}%
\begin{pgfscope}%
\pgfpathrectangle{\pgfqpoint{0.329460in}{0.284240in}}{\pgfqpoint{1.989680in}{1.989680in}}%
\pgfusepath{clip}%
\pgfsetbuttcap%
\pgfsetroundjoin%
\definecolor{currentfill}{rgb}{0.282884,0.135920,0.453427}%
\pgfsetfillcolor{currentfill}%
\pgfsetlinewidth{0.000000pt}%
\definecolor{currentstroke}{rgb}{0.000000,0.000000,0.000000}%
\pgfsetstrokecolor{currentstroke}%
\pgfsetdash{}{0pt}%
\pgfpathmoveto{\pgfqpoint{1.590316in}{0.787645in}}%
\pgfpathlineto{\pgfqpoint{1.591816in}{0.791750in}}%
\pgfpathlineto{\pgfqpoint{1.593322in}{0.796194in}}%
\pgfpathlineto{\pgfqpoint{1.594835in}{0.800982in}}%
\pgfpathlineto{\pgfqpoint{1.596355in}{0.806120in}}%
\pgfpathlineto{\pgfqpoint{1.577695in}{0.802096in}}%
\pgfpathlineto{\pgfqpoint{1.558777in}{0.798390in}}%
\pgfpathlineto{\pgfqpoint{1.539622in}{0.795007in}}%
\pgfpathlineto{\pgfqpoint{1.520251in}{0.791952in}}%
\pgfpathlineto{\pgfqpoint{1.519201in}{0.786891in}}%
\pgfpathlineto{\pgfqpoint{1.518155in}{0.782181in}}%
\pgfpathlineto{\pgfqpoint{1.517114in}{0.777815in}}%
\pgfpathlineto{\pgfqpoint{1.516078in}{0.773789in}}%
\pgfpathlineto{\pgfqpoint{1.534972in}{0.776777in}}%
\pgfpathlineto{\pgfqpoint{1.553658in}{0.780085in}}%
\pgfpathlineto{\pgfqpoint{1.572112in}{0.783709in}}%
\pgfpathlineto{\pgfqpoint{1.590316in}{0.787645in}}%
\pgfpathclose%
\pgfusepath{fill}%
\end{pgfscope}%
\begin{pgfscope}%
\pgfpathrectangle{\pgfqpoint{0.329460in}{0.284240in}}{\pgfqpoint{1.989680in}{1.989680in}}%
\pgfusepath{clip}%
\pgfsetbuttcap%
\pgfsetroundjoin%
\definecolor{currentfill}{rgb}{0.276194,0.190074,0.493001}%
\pgfsetfillcolor{currentfill}%
\pgfsetlinewidth{0.000000pt}%
\definecolor{currentstroke}{rgb}{0.000000,0.000000,0.000000}%
\pgfsetstrokecolor{currentstroke}%
\pgfsetdash{}{0pt}%
\pgfpathmoveto{\pgfqpoint{1.668002in}{0.825294in}}%
\pgfpathlineto{\pgfqpoint{1.669968in}{0.830892in}}%
\pgfpathlineto{\pgfqpoint{1.671944in}{0.836852in}}%
\pgfpathlineto{\pgfqpoint{1.673928in}{0.843181in}}%
\pgfpathlineto{\pgfqpoint{1.675922in}{0.849884in}}%
\pgfpathlineto{\pgfqpoint{1.658069in}{0.844530in}}%
\pgfpathlineto{\pgfqpoint{1.639868in}{0.839477in}}%
\pgfpathlineto{\pgfqpoint{1.621339in}{0.834732in}}%
\pgfpathlineto{\pgfqpoint{1.602502in}{0.830302in}}%
\pgfpathlineto{\pgfqpoint{1.600954in}{0.823700in}}%
\pgfpathlineto{\pgfqpoint{1.599414in}{0.817473in}}%
\pgfpathlineto{\pgfqpoint{1.597881in}{0.811615in}}%
\pgfpathlineto{\pgfqpoint{1.596355in}{0.806120in}}%
\pgfpathlineto{\pgfqpoint{1.614736in}{0.810458in}}%
\pgfpathlineto{\pgfqpoint{1.632817in}{0.815104in}}%
\pgfpathlineto{\pgfqpoint{1.650579in}{0.820051in}}%
\pgfpathlineto{\pgfqpoint{1.668002in}{0.825294in}}%
\pgfpathclose%
\pgfusepath{fill}%
\end{pgfscope}%
\begin{pgfscope}%
\pgfpathrectangle{\pgfqpoint{0.329460in}{0.284240in}}{\pgfqpoint{1.989680in}{1.989680in}}%
\pgfusepath{clip}%
\pgfsetbuttcap%
\pgfsetroundjoin%
\definecolor{currentfill}{rgb}{0.282327,0.094955,0.417331}%
\pgfsetfillcolor{currentfill}%
\pgfsetlinewidth{0.000000pt}%
\definecolor{currentstroke}{rgb}{0.000000,0.000000,0.000000}%
\pgfsetstrokecolor{currentstroke}%
\pgfsetdash{}{0pt}%
\pgfpathmoveto{\pgfqpoint{1.436648in}{0.752480in}}%
\pgfpathlineto{\pgfqpoint{1.437191in}{0.755158in}}%
\pgfpathlineto{\pgfqpoint{1.437737in}{0.758153in}}%
\pgfpathlineto{\pgfqpoint{1.438284in}{0.761469in}}%
\pgfpathlineto{\pgfqpoint{1.438833in}{0.765113in}}%
\pgfpathlineto{\pgfqpoint{1.419216in}{0.763777in}}%
\pgfpathlineto{\pgfqpoint{1.399520in}{0.762778in}}%
\pgfpathlineto{\pgfqpoint{1.379769in}{0.762118in}}%
\pgfpathlineto{\pgfqpoint{1.359985in}{0.761797in}}%
\pgfpathlineto{\pgfqpoint{1.359930in}{0.758172in}}%
\pgfpathlineto{\pgfqpoint{1.359875in}{0.754875in}}%
\pgfpathlineto{\pgfqpoint{1.359820in}{0.751900in}}%
\pgfpathlineto{\pgfqpoint{1.359766in}{0.749240in}}%
\pgfpathlineto{\pgfqpoint{1.379056in}{0.749554in}}%
\pgfpathlineto{\pgfqpoint{1.398315in}{0.750199in}}%
\pgfpathlineto{\pgfqpoint{1.417519in}{0.751175in}}%
\pgfpathlineto{\pgfqpoint{1.436648in}{0.752480in}}%
\pgfpathclose%
\pgfusepath{fill}%
\end{pgfscope}%
\begin{pgfscope}%
\pgfpathrectangle{\pgfqpoint{0.329460in}{0.284240in}}{\pgfqpoint{1.989680in}{1.989680in}}%
\pgfusepath{clip}%
\pgfsetbuttcap%
\pgfsetroundjoin%
\definecolor{currentfill}{rgb}{0.282327,0.094955,0.417331}%
\pgfsetfillcolor{currentfill}%
\pgfsetlinewidth{0.000000pt}%
\definecolor{currentstroke}{rgb}{0.000000,0.000000,0.000000}%
\pgfsetstrokecolor{currentstroke}%
\pgfsetdash{}{0pt}%
\pgfpathmoveto{\pgfqpoint{1.359766in}{0.749240in}}%
\pgfpathlineto{\pgfqpoint{1.359820in}{0.751900in}}%
\pgfpathlineto{\pgfqpoint{1.359875in}{0.754875in}}%
\pgfpathlineto{\pgfqpoint{1.359930in}{0.758172in}}%
\pgfpathlineto{\pgfqpoint{1.359985in}{0.761797in}}%
\pgfpathlineto{\pgfqpoint{1.340191in}{0.761816in}}%
\pgfpathlineto{\pgfqpoint{1.320410in}{0.762175in}}%
\pgfpathlineto{\pgfqpoint{1.300664in}{0.762873in}}%
\pgfpathlineto{\pgfqpoint{1.280975in}{0.763909in}}%
\pgfpathlineto{\pgfqpoint{1.281416in}{0.760272in}}%
\pgfpathlineto{\pgfqpoint{1.281854in}{0.756963in}}%
\pgfpathlineto{\pgfqpoint{1.282291in}{0.753975in}}%
\pgfpathlineto{\pgfqpoint{1.282726in}{0.751303in}}%
\pgfpathlineto{\pgfqpoint{1.301924in}{0.750291in}}%
\pgfpathlineto{\pgfqpoint{1.321177in}{0.749609in}}%
\pgfpathlineto{\pgfqpoint{1.340465in}{0.749259in}}%
\pgfpathlineto{\pgfqpoint{1.359766in}{0.749240in}}%
\pgfpathclose%
\pgfusepath{fill}%
\end{pgfscope}%
\begin{pgfscope}%
\pgfpathrectangle{\pgfqpoint{0.329460in}{0.284240in}}{\pgfqpoint{1.989680in}{1.989680in}}%
\pgfusepath{clip}%
\pgfsetbuttcap%
\pgfsetroundjoin%
\definecolor{currentfill}{rgb}{0.282884,0.135920,0.453427}%
\pgfsetfillcolor{currentfill}%
\pgfsetlinewidth{0.000000pt}%
\definecolor{currentstroke}{rgb}{0.000000,0.000000,0.000000}%
\pgfsetstrokecolor{currentstroke}%
\pgfsetdash{}{0pt}%
\pgfpathmoveto{\pgfqpoint{1.203251in}{0.771405in}}%
\pgfpathlineto{\pgfqpoint{1.202321in}{0.775418in}}%
\pgfpathlineto{\pgfqpoint{1.201387in}{0.779770in}}%
\pgfpathlineto{\pgfqpoint{1.200448in}{0.784467in}}%
\pgfpathlineto{\pgfqpoint{1.199506in}{0.789515in}}%
\pgfpathlineto{\pgfqpoint{1.179962in}{0.792275in}}%
\pgfpathlineto{\pgfqpoint{1.160614in}{0.795367in}}%
\pgfpathlineto{\pgfqpoint{1.141484in}{0.798786in}}%
\pgfpathlineto{\pgfqpoint{1.122594in}{0.802527in}}%
\pgfpathlineto{\pgfqpoint{1.124011in}{0.797409in}}%
\pgfpathlineto{\pgfqpoint{1.125423in}{0.792640in}}%
\pgfpathlineto{\pgfqpoint{1.126828in}{0.788217in}}%
\pgfpathlineto{\pgfqpoint{1.128228in}{0.784131in}}%
\pgfpathlineto{\pgfqpoint{1.146655in}{0.780472in}}%
\pgfpathlineto{\pgfqpoint{1.165316in}{0.777128in}}%
\pgfpathlineto{\pgfqpoint{1.184188in}{0.774105in}}%
\pgfpathlineto{\pgfqpoint{1.203251in}{0.771405in}}%
\pgfpathclose%
\pgfusepath{fill}%
\end{pgfscope}%
\begin{pgfscope}%
\pgfpathrectangle{\pgfqpoint{0.329460in}{0.284240in}}{\pgfqpoint{1.989680in}{1.989680in}}%
\pgfusepath{clip}%
\pgfsetbuttcap%
\pgfsetroundjoin%
\definecolor{currentfill}{rgb}{0.260571,0.246922,0.522828}%
\pgfsetfillcolor{currentfill}%
\pgfsetlinewidth{0.000000pt}%
\definecolor{currentstroke}{rgb}{0.000000,0.000000,0.000000}%
\pgfsetstrokecolor{currentstroke}%
\pgfsetdash{}{0pt}%
\pgfpathmoveto{\pgfqpoint{1.042305in}{0.845110in}}%
\pgfpathlineto{\pgfqpoint{1.040398in}{0.852170in}}%
\pgfpathlineto{\pgfqpoint{1.038482in}{0.859619in}}%
\pgfpathlineto{\pgfqpoint{1.036556in}{0.867462in}}%
\pgfpathlineto{\pgfqpoint{1.034621in}{0.875706in}}%
\pgfpathlineto{\pgfqpoint{1.016372in}{0.881202in}}%
\pgfpathlineto{\pgfqpoint{0.998503in}{0.886998in}}%
\pgfpathlineto{\pgfqpoint{0.981033in}{0.893087in}}%
\pgfpathlineto{\pgfqpoint{0.963981in}{0.899460in}}%
\pgfpathlineto{\pgfqpoint{0.966340in}{0.891100in}}%
\pgfpathlineto{\pgfqpoint{0.968687in}{0.883141in}}%
\pgfpathlineto{\pgfqpoint{0.971023in}{0.875575in}}%
\pgfpathlineto{\pgfqpoint{0.973347in}{0.868396in}}%
\pgfpathlineto{\pgfqpoint{0.989995in}{0.862148in}}%
\pgfpathlineto{\pgfqpoint{1.007049in}{0.856179in}}%
\pgfpathlineto{\pgfqpoint{1.024492in}{0.850498in}}%
\pgfpathlineto{\pgfqpoint{1.042305in}{0.845110in}}%
\pgfpathclose%
\pgfusepath{fill}%
\end{pgfscope}%
\begin{pgfscope}%
\pgfpathrectangle{\pgfqpoint{0.329460in}{0.284240in}}{\pgfqpoint{1.989680in}{1.989680in}}%
\pgfusepath{clip}%
\pgfsetbuttcap%
\pgfsetroundjoin%
\definecolor{currentfill}{rgb}{0.172719,0.448791,0.557885}%
\pgfsetfillcolor{currentfill}%
\pgfsetlinewidth{0.000000pt}%
\definecolor{currentstroke}{rgb}{0.000000,0.000000,0.000000}%
\pgfsetstrokecolor{currentstroke}%
\pgfsetdash{}{0pt}%
\pgfpathmoveto{\pgfqpoint{1.894500in}{1.051877in}}%
\pgfpathlineto{\pgfqpoint{1.897777in}{1.064457in}}%
\pgfpathlineto{\pgfqpoint{1.901074in}{1.077507in}}%
\pgfpathlineto{\pgfqpoint{1.904391in}{1.091037in}}%
\pgfpathlineto{\pgfqpoint{1.890952in}{1.082104in}}%
\pgfpathlineto{\pgfqpoint{1.876917in}{1.073384in}}%
\pgfpathlineto{\pgfqpoint{1.862300in}{1.064887in}}%
\pgfpathlineto{\pgfqpoint{1.847115in}{1.056623in}}%
\pgfpathlineto{\pgfqpoint{1.844126in}{1.043229in}}%
\pgfpathlineto{\pgfqpoint{1.841155in}{1.030315in}}%
\pgfpathlineto{\pgfqpoint{1.838202in}{1.017875in}}%
\pgfpathlineto{\pgfqpoint{1.853126in}{1.026039in}}%
\pgfpathlineto{\pgfqpoint{1.867492in}{1.034434in}}%
\pgfpathlineto{\pgfqpoint{1.881288in}{1.043050in}}%
\pgfpathlineto{\pgfqpoint{1.894500in}{1.051877in}}%
\pgfpathclose%
\pgfusepath{fill}%
\end{pgfscope}%
\begin{pgfscope}%
\pgfpathrectangle{\pgfqpoint{0.329460in}{0.284240in}}{\pgfqpoint{1.989680in}{1.989680in}}%
\pgfusepath{clip}%
\pgfsetbuttcap%
\pgfsetroundjoin%
\definecolor{currentfill}{rgb}{0.276194,0.190074,0.493001}%
\pgfsetfillcolor{currentfill}%
\pgfsetlinewidth{0.000000pt}%
\definecolor{currentstroke}{rgb}{0.000000,0.000000,0.000000}%
\pgfsetstrokecolor{currentstroke}%
\pgfsetdash{}{0pt}%
\pgfpathmoveto{\pgfqpoint{1.122594in}{0.802527in}}%
\pgfpathlineto{\pgfqpoint{1.121170in}{0.808003in}}%
\pgfpathlineto{\pgfqpoint{1.119740in}{0.813841in}}%
\pgfpathlineto{\pgfqpoint{1.118303in}{0.820049in}}%
\pgfpathlineto{\pgfqpoint{1.116859in}{0.826632in}}%
\pgfpathlineto{\pgfqpoint{1.097766in}{0.830779in}}%
\pgfpathlineto{\pgfqpoint{1.078962in}{0.835244in}}%
\pgfpathlineto{\pgfqpoint{1.060468in}{0.840023in}}%
\pgfpathlineto{\pgfqpoint{1.042305in}{0.845110in}}%
\pgfpathlineto{\pgfqpoint{1.044203in}{0.838431in}}%
\pgfpathlineto{\pgfqpoint{1.046091in}{0.832127in}}%
\pgfpathlineto{\pgfqpoint{1.047972in}{0.826192in}}%
\pgfpathlineto{\pgfqpoint{1.049843in}{0.820619in}}%
\pgfpathlineto{\pgfqpoint{1.067569in}{0.815638in}}%
\pgfpathlineto{\pgfqpoint{1.085615in}{0.810959in}}%
\pgfpathlineto{\pgfqpoint{1.103964in}{0.806587in}}%
\pgfpathlineto{\pgfqpoint{1.122594in}{0.802527in}}%
\pgfpathclose%
\pgfusepath{fill}%
\end{pgfscope}%
\begin{pgfscope}%
\pgfpathrectangle{\pgfqpoint{0.329460in}{0.284240in}}{\pgfqpoint{1.989680in}{1.989680in}}%
\pgfusepath{clip}%
\pgfsetbuttcap%
\pgfsetroundjoin%
\definecolor{currentfill}{rgb}{0.282884,0.135920,0.453427}%
\pgfsetfillcolor{currentfill}%
\pgfsetlinewidth{0.000000pt}%
\definecolor{currentstroke}{rgb}{0.000000,0.000000,0.000000}%
\pgfsetstrokecolor{currentstroke}%
\pgfsetdash{}{0pt}%
\pgfpathmoveto{\pgfqpoint{1.516078in}{0.773789in}}%
\pgfpathlineto{\pgfqpoint{1.517114in}{0.777815in}}%
\pgfpathlineto{\pgfqpoint{1.518155in}{0.782181in}}%
\pgfpathlineto{\pgfqpoint{1.519201in}{0.786891in}}%
\pgfpathlineto{\pgfqpoint{1.520251in}{0.791952in}}%
\pgfpathlineto{\pgfqpoint{1.500686in}{0.789229in}}%
\pgfpathlineto{\pgfqpoint{1.480950in}{0.786840in}}%
\pgfpathlineto{\pgfqpoint{1.461066in}{0.784790in}}%
\pgfpathlineto{\pgfqpoint{1.441054in}{0.783080in}}%
\pgfpathlineto{\pgfqpoint{1.440496in}{0.778068in}}%
\pgfpathlineto{\pgfqpoint{1.439939in}{0.773406in}}%
\pgfpathlineto{\pgfqpoint{1.439385in}{0.769090in}}%
\pgfpathlineto{\pgfqpoint{1.438833in}{0.765113in}}%
\pgfpathlineto{\pgfqpoint{1.458351in}{0.766785in}}%
\pgfpathlineto{\pgfqpoint{1.477745in}{0.768790in}}%
\pgfpathlineto{\pgfqpoint{1.496995in}{0.771125in}}%
\pgfpathlineto{\pgfqpoint{1.516078in}{0.773789in}}%
\pgfpathclose%
\pgfusepath{fill}%
\end{pgfscope}%
\begin{pgfscope}%
\pgfpathrectangle{\pgfqpoint{0.329460in}{0.284240in}}{\pgfqpoint{1.989680in}{1.989680in}}%
\pgfusepath{clip}%
\pgfsetbuttcap%
\pgfsetroundjoin%
\definecolor{currentfill}{rgb}{0.172719,0.448791,0.557885}%
\pgfsetfillcolor{currentfill}%
\pgfsetlinewidth{0.000000pt}%
\definecolor{currentstroke}{rgb}{0.000000,0.000000,0.000000}%
\pgfsetstrokecolor{currentstroke}%
\pgfsetdash{}{0pt}%
\pgfpathmoveto{\pgfqpoint{0.877895in}{1.010820in}}%
\pgfpathlineto{\pgfqpoint{0.875022in}{1.023231in}}%
\pgfpathlineto{\pgfqpoint{0.872131in}{1.036116in}}%
\pgfpathlineto{\pgfqpoint{0.869223in}{1.049482in}}%
\pgfpathlineto{\pgfqpoint{0.853545in}{1.057529in}}%
\pgfpathlineto{\pgfqpoint{0.838422in}{1.065819in}}%
\pgfpathlineto{\pgfqpoint{0.823870in}{1.074342in}}%
\pgfpathlineto{\pgfqpoint{0.809901in}{1.083086in}}%
\pgfpathlineto{\pgfqpoint{0.813150in}{1.069588in}}%
\pgfpathlineto{\pgfqpoint{0.816379in}{1.056569in}}%
\pgfpathlineto{\pgfqpoint{0.819590in}{1.044021in}}%
\pgfpathlineto{\pgfqpoint{0.833321in}{1.035381in}}%
\pgfpathlineto{\pgfqpoint{0.847625in}{1.026961in}}%
\pgfpathlineto{\pgfqpoint{0.862488in}{1.018770in}}%
\pgfpathlineto{\pgfqpoint{0.877895in}{1.010820in}}%
\pgfpathclose%
\pgfusepath{fill}%
\end{pgfscope}%
\begin{pgfscope}%
\pgfpathrectangle{\pgfqpoint{0.329460in}{0.284240in}}{\pgfqpoint{1.989680in}{1.989680in}}%
\pgfusepath{clip}%
\pgfsetbuttcap%
\pgfsetroundjoin%
\definecolor{currentfill}{rgb}{0.282884,0.135920,0.453427}%
\pgfsetfillcolor{currentfill}%
\pgfsetlinewidth{0.000000pt}%
\definecolor{currentstroke}{rgb}{0.000000,0.000000,0.000000}%
\pgfsetstrokecolor{currentstroke}%
\pgfsetdash{}{0pt}%
\pgfpathmoveto{\pgfqpoint{1.280975in}{0.763909in}}%
\pgfpathlineto{\pgfqpoint{1.280533in}{0.767879in}}%
\pgfpathlineto{\pgfqpoint{1.280089in}{0.772189in}}%
\pgfpathlineto{\pgfqpoint{1.279644in}{0.776843in}}%
\pgfpathlineto{\pgfqpoint{1.279196in}{0.781849in}}%
\pgfpathlineto{\pgfqpoint{1.259092in}{0.783253in}}%
\pgfpathlineto{\pgfqpoint{1.239093in}{0.785001in}}%
\pgfpathlineto{\pgfqpoint{1.219224in}{0.787089in}}%
\pgfpathlineto{\pgfqpoint{1.199506in}{0.789515in}}%
\pgfpathlineto{\pgfqpoint{1.200448in}{0.784467in}}%
\pgfpathlineto{\pgfqpoint{1.201387in}{0.779770in}}%
\pgfpathlineto{\pgfqpoint{1.202321in}{0.775418in}}%
\pgfpathlineto{\pgfqpoint{1.203251in}{0.771405in}}%
\pgfpathlineto{\pgfqpoint{1.222484in}{0.769033in}}%
\pgfpathlineto{\pgfqpoint{1.241863in}{0.766991in}}%
\pgfpathlineto{\pgfqpoint{1.261368in}{0.765282in}}%
\pgfpathlineto{\pgfqpoint{1.280975in}{0.763909in}}%
\pgfpathclose%
\pgfusepath{fill}%
\end{pgfscope}%
\begin{pgfscope}%
\pgfpathrectangle{\pgfqpoint{0.329460in}{0.284240in}}{\pgfqpoint{1.989680in}{1.989680in}}%
\pgfusepath{clip}%
\pgfsetbuttcap%
\pgfsetroundjoin%
\definecolor{currentfill}{rgb}{0.201239,0.383670,0.554294}%
\pgfsetfillcolor{currentfill}%
\pgfsetlinewidth{0.000000pt}%
\definecolor{currentstroke}{rgb}{0.000000,0.000000,0.000000}%
\pgfsetstrokecolor{currentstroke}%
\pgfsetdash{}{0pt}%
\pgfpathmoveto{\pgfqpoint{1.826558in}{0.972683in}}%
\pgfpathlineto{\pgfqpoint{1.829444in}{0.983311in}}%
\pgfpathlineto{\pgfqpoint{1.832347in}{0.994381in}}%
\pgfpathlineto{\pgfqpoint{1.835266in}{1.005899in}}%
\pgfpathlineto{\pgfqpoint{1.838202in}{1.017875in}}%
\pgfpathlineto{\pgfqpoint{1.822735in}{1.009952in}}%
\pgfpathlineto{\pgfqpoint{1.806741in}{1.002280in}}%
\pgfpathlineto{\pgfqpoint{1.790236in}{0.994868in}}%
\pgfpathlineto{\pgfqpoint{1.773237in}{0.987726in}}%
\pgfpathlineto{\pgfqpoint{1.770681in}{0.975878in}}%
\pgfpathlineto{\pgfqpoint{1.768140in}{0.964488in}}%
\pgfpathlineto{\pgfqpoint{1.765613in}{0.953549in}}%
\pgfpathlineto{\pgfqpoint{1.763101in}{0.943054in}}%
\pgfpathlineto{\pgfqpoint{1.779703in}{0.950072in}}%
\pgfpathlineto{\pgfqpoint{1.795824in}{0.957356in}}%
\pgfpathlineto{\pgfqpoint{1.811448in}{0.964896in}}%
\pgfpathlineto{\pgfqpoint{1.826558in}{0.972683in}}%
\pgfpathclose%
\pgfusepath{fill}%
\end{pgfscope}%
\begin{pgfscope}%
\pgfpathrectangle{\pgfqpoint{0.329460in}{0.284240in}}{\pgfqpoint{1.989680in}{1.989680in}}%
\pgfusepath{clip}%
\pgfsetbuttcap%
\pgfsetroundjoin%
\definecolor{currentfill}{rgb}{0.282884,0.135920,0.453427}%
\pgfsetfillcolor{currentfill}%
\pgfsetlinewidth{0.000000pt}%
\definecolor{currentstroke}{rgb}{0.000000,0.000000,0.000000}%
\pgfsetstrokecolor{currentstroke}%
\pgfsetdash{}{0pt}%
\pgfpathmoveto{\pgfqpoint{1.438833in}{0.765113in}}%
\pgfpathlineto{\pgfqpoint{1.439385in}{0.769090in}}%
\pgfpathlineto{\pgfqpoint{1.439939in}{0.773406in}}%
\pgfpathlineto{\pgfqpoint{1.440496in}{0.778068in}}%
\pgfpathlineto{\pgfqpoint{1.441054in}{0.783080in}}%
\pgfpathlineto{\pgfqpoint{1.420940in}{0.781714in}}%
\pgfpathlineto{\pgfqpoint{1.400745in}{0.780693in}}%
\pgfpathlineto{\pgfqpoint{1.380494in}{0.780017in}}%
\pgfpathlineto{\pgfqpoint{1.360208in}{0.779689in}}%
\pgfpathlineto{\pgfqpoint{1.360152in}{0.774695in}}%
\pgfpathlineto{\pgfqpoint{1.360096in}{0.770053in}}%
\pgfpathlineto{\pgfqpoint{1.360040in}{0.765755in}}%
\pgfpathlineto{\pgfqpoint{1.359985in}{0.761797in}}%
\pgfpathlineto{\pgfqpoint{1.379769in}{0.762118in}}%
\pgfpathlineto{\pgfqpoint{1.399520in}{0.762778in}}%
\pgfpathlineto{\pgfqpoint{1.419216in}{0.763777in}}%
\pgfpathlineto{\pgfqpoint{1.438833in}{0.765113in}}%
\pgfpathclose%
\pgfusepath{fill}%
\end{pgfscope}%
\begin{pgfscope}%
\pgfpathrectangle{\pgfqpoint{0.329460in}{0.284240in}}{\pgfqpoint{1.989680in}{1.989680in}}%
\pgfusepath{clip}%
\pgfsetbuttcap%
\pgfsetroundjoin%
\definecolor{currentfill}{rgb}{0.276194,0.190074,0.493001}%
\pgfsetfillcolor{currentfill}%
\pgfsetlinewidth{0.000000pt}%
\definecolor{currentstroke}{rgb}{0.000000,0.000000,0.000000}%
\pgfsetstrokecolor{currentstroke}%
\pgfsetdash{}{0pt}%
\pgfpathmoveto{\pgfqpoint{1.596355in}{0.806120in}}%
\pgfpathlineto{\pgfqpoint{1.597881in}{0.811615in}}%
\pgfpathlineto{\pgfqpoint{1.599414in}{0.817473in}}%
\pgfpathlineto{\pgfqpoint{1.600954in}{0.823700in}}%
\pgfpathlineto{\pgfqpoint{1.602502in}{0.830302in}}%
\pgfpathlineto{\pgfqpoint{1.583378in}{0.826191in}}%
\pgfpathlineto{\pgfqpoint{1.563988in}{0.822406in}}%
\pgfpathlineto{\pgfqpoint{1.544354in}{0.818951in}}%
\pgfpathlineto{\pgfqpoint{1.524499in}{0.815830in}}%
\pgfpathlineto{\pgfqpoint{1.523429in}{0.809303in}}%
\pgfpathlineto{\pgfqpoint{1.522365in}{0.803152in}}%
\pgfpathlineto{\pgfqpoint{1.521306in}{0.797370in}}%
\pgfpathlineto{\pgfqpoint{1.520251in}{0.791952in}}%
\pgfpathlineto{\pgfqpoint{1.539622in}{0.795007in}}%
\pgfpathlineto{\pgfqpoint{1.558777in}{0.798390in}}%
\pgfpathlineto{\pgfqpoint{1.577695in}{0.802096in}}%
\pgfpathlineto{\pgfqpoint{1.596355in}{0.806120in}}%
\pgfpathclose%
\pgfusepath{fill}%
\end{pgfscope}%
\begin{pgfscope}%
\pgfpathrectangle{\pgfqpoint{0.329460in}{0.284240in}}{\pgfqpoint{1.989680in}{1.989680in}}%
\pgfusepath{clip}%
\pgfsetbuttcap%
\pgfsetroundjoin%
\definecolor{currentfill}{rgb}{0.282884,0.135920,0.453427}%
\pgfsetfillcolor{currentfill}%
\pgfsetlinewidth{0.000000pt}%
\definecolor{currentstroke}{rgb}{0.000000,0.000000,0.000000}%
\pgfsetstrokecolor{currentstroke}%
\pgfsetdash{}{0pt}%
\pgfpathmoveto{\pgfqpoint{1.359985in}{0.761797in}}%
\pgfpathlineto{\pgfqpoint{1.360040in}{0.765755in}}%
\pgfpathlineto{\pgfqpoint{1.360096in}{0.770053in}}%
\pgfpathlineto{\pgfqpoint{1.360152in}{0.774695in}}%
\pgfpathlineto{\pgfqpoint{1.360208in}{0.779689in}}%
\pgfpathlineto{\pgfqpoint{1.339912in}{0.779709in}}%
\pgfpathlineto{\pgfqpoint{1.319629in}{0.780075in}}%
\pgfpathlineto{\pgfqpoint{1.299383in}{0.780789in}}%
\pgfpathlineto{\pgfqpoint{1.279196in}{0.781849in}}%
\pgfpathlineto{\pgfqpoint{1.279644in}{0.776843in}}%
\pgfpathlineto{\pgfqpoint{1.280089in}{0.772189in}}%
\pgfpathlineto{\pgfqpoint{1.280533in}{0.767879in}}%
\pgfpathlineto{\pgfqpoint{1.280975in}{0.763909in}}%
\pgfpathlineto{\pgfqpoint{1.300664in}{0.762873in}}%
\pgfpathlineto{\pgfqpoint{1.320410in}{0.762175in}}%
\pgfpathlineto{\pgfqpoint{1.340191in}{0.761816in}}%
\pgfpathlineto{\pgfqpoint{1.359985in}{0.761797in}}%
\pgfpathclose%
\pgfusepath{fill}%
\end{pgfscope}%
\begin{pgfscope}%
\pgfpathrectangle{\pgfqpoint{0.329460in}{0.284240in}}{\pgfqpoint{1.989680in}{1.989680in}}%
\pgfusepath{clip}%
\pgfsetbuttcap%
\pgfsetroundjoin%
\definecolor{currentfill}{rgb}{0.233603,0.313828,0.543914}%
\pgfsetfillcolor{currentfill}%
\pgfsetlinewidth{0.000000pt}%
\definecolor{currentstroke}{rgb}{0.000000,0.000000,0.000000}%
\pgfsetstrokecolor{currentstroke}%
\pgfsetdash{}{0pt}%
\pgfpathmoveto{\pgfqpoint{1.753185in}{0.905358in}}%
\pgfpathlineto{\pgfqpoint{1.755644in}{0.914154in}}%
\pgfpathlineto{\pgfqpoint{1.758117in}{0.923363in}}%
\pgfpathlineto{\pgfqpoint{1.760602in}{0.932994in}}%
\pgfpathlineto{\pgfqpoint{1.763101in}{0.943054in}}%
\pgfpathlineto{\pgfqpoint{1.746034in}{0.936310in}}%
\pgfpathlineto{\pgfqpoint{1.728522in}{0.929848in}}%
\pgfpathlineto{\pgfqpoint{1.710582in}{0.923677in}}%
\pgfpathlineto{\pgfqpoint{1.692234in}{0.917804in}}%
\pgfpathlineto{\pgfqpoint{1.690158in}{0.907860in}}%
\pgfpathlineto{\pgfqpoint{1.688092in}{0.898346in}}%
\pgfpathlineto{\pgfqpoint{1.686038in}{0.889254in}}%
\pgfpathlineto{\pgfqpoint{1.683994in}{0.880577in}}%
\pgfpathlineto{\pgfqpoint{1.701906in}{0.886340in}}%
\pgfpathlineto{\pgfqpoint{1.719421in}{0.892396in}}%
\pgfpathlineto{\pgfqpoint{1.736520in}{0.898738in}}%
\pgfpathlineto{\pgfqpoint{1.753185in}{0.905358in}}%
\pgfpathclose%
\pgfusepath{fill}%
\end{pgfscope}%
\begin{pgfscope}%
\pgfpathrectangle{\pgfqpoint{0.329460in}{0.284240in}}{\pgfqpoint{1.989680in}{1.989680in}}%
\pgfusepath{clip}%
\pgfsetbuttcap%
\pgfsetroundjoin%
\definecolor{currentfill}{rgb}{0.260571,0.246922,0.522828}%
\pgfsetfillcolor{currentfill}%
\pgfsetlinewidth{0.000000pt}%
\definecolor{currentstroke}{rgb}{0.000000,0.000000,0.000000}%
\pgfsetstrokecolor{currentstroke}%
\pgfsetdash{}{0pt}%
\pgfpathmoveto{\pgfqpoint{1.675922in}{0.849884in}}%
\pgfpathlineto{\pgfqpoint{1.677925in}{0.856969in}}%
\pgfpathlineto{\pgfqpoint{1.679938in}{0.864441in}}%
\pgfpathlineto{\pgfqpoint{1.681961in}{0.872308in}}%
\pgfpathlineto{\pgfqpoint{1.683994in}{0.880577in}}%
\pgfpathlineto{\pgfqpoint{1.665704in}{0.875114in}}%
\pgfpathlineto{\pgfqpoint{1.647056in}{0.869960in}}%
\pgfpathlineto{\pgfqpoint{1.628070in}{0.865119in}}%
\pgfpathlineto{\pgfqpoint{1.608768in}{0.860599in}}%
\pgfpathlineto{\pgfqpoint{1.607189in}{0.852428in}}%
\pgfpathlineto{\pgfqpoint{1.605619in}{0.844660in}}%
\pgfpathlineto{\pgfqpoint{1.604057in}{0.837286in}}%
\pgfpathlineto{\pgfqpoint{1.602502in}{0.830302in}}%
\pgfpathlineto{\pgfqpoint{1.621339in}{0.834732in}}%
\pgfpathlineto{\pgfqpoint{1.639868in}{0.839477in}}%
\pgfpathlineto{\pgfqpoint{1.658069in}{0.844530in}}%
\pgfpathlineto{\pgfqpoint{1.675922in}{0.849884in}}%
\pgfpathclose%
\pgfusepath{fill}%
\end{pgfscope}%
\begin{pgfscope}%
\pgfpathrectangle{\pgfqpoint{0.329460in}{0.284240in}}{\pgfqpoint{1.989680in}{1.989680in}}%
\pgfusepath{clip}%
\pgfsetbuttcap%
\pgfsetroundjoin%
\definecolor{currentfill}{rgb}{0.276194,0.190074,0.493001}%
\pgfsetfillcolor{currentfill}%
\pgfsetlinewidth{0.000000pt}%
\definecolor{currentstroke}{rgb}{0.000000,0.000000,0.000000}%
\pgfsetstrokecolor{currentstroke}%
\pgfsetdash{}{0pt}%
\pgfpathmoveto{\pgfqpoint{1.199506in}{0.789515in}}%
\pgfpathlineto{\pgfqpoint{1.198559in}{0.794920in}}%
\pgfpathlineto{\pgfqpoint{1.197609in}{0.800688in}}%
\pgfpathlineto{\pgfqpoint{1.196653in}{0.806826in}}%
\pgfpathlineto{\pgfqpoint{1.195693in}{0.813340in}}%
\pgfpathlineto{\pgfqpoint{1.175660in}{0.816160in}}%
\pgfpathlineto{\pgfqpoint{1.155828in}{0.819318in}}%
\pgfpathlineto{\pgfqpoint{1.136220in}{0.822811in}}%
\pgfpathlineto{\pgfqpoint{1.116859in}{0.826632in}}%
\pgfpathlineto{\pgfqpoint{1.118303in}{0.820049in}}%
\pgfpathlineto{\pgfqpoint{1.119740in}{0.813841in}}%
\pgfpathlineto{\pgfqpoint{1.121170in}{0.808003in}}%
\pgfpathlineto{\pgfqpoint{1.122594in}{0.802527in}}%
\pgfpathlineto{\pgfqpoint{1.141484in}{0.798786in}}%
\pgfpathlineto{\pgfqpoint{1.160614in}{0.795367in}}%
\pgfpathlineto{\pgfqpoint{1.179962in}{0.792275in}}%
\pgfpathlineto{\pgfqpoint{1.199506in}{0.789515in}}%
\pgfpathclose%
\pgfusepath{fill}%
\end{pgfscope}%
\begin{pgfscope}%
\pgfpathrectangle{\pgfqpoint{0.329460in}{0.284240in}}{\pgfqpoint{1.989680in}{1.989680in}}%
\pgfusepath{clip}%
\pgfsetbuttcap%
\pgfsetroundjoin%
\definecolor{currentfill}{rgb}{0.201239,0.383670,0.554294}%
\pgfsetfillcolor{currentfill}%
\pgfsetlinewidth{0.000000pt}%
\definecolor{currentstroke}{rgb}{0.000000,0.000000,0.000000}%
\pgfsetstrokecolor{currentstroke}%
\pgfsetdash{}{0pt}%
\pgfpathmoveto{\pgfqpoint{0.954422in}{0.937045in}}%
\pgfpathlineto{\pgfqpoint{0.952000in}{0.947513in}}%
\pgfpathlineto{\pgfqpoint{0.949564in}{0.958425in}}%
\pgfpathlineto{\pgfqpoint{0.947114in}{0.969789in}}%
\pgfpathlineto{\pgfqpoint{0.944649in}{0.981611in}}%
\pgfpathlineto{\pgfqpoint{0.927225in}{0.988506in}}%
\pgfpathlineto{\pgfqpoint{0.910280in}{0.995678in}}%
\pgfpathlineto{\pgfqpoint{0.893831in}{1.003120in}}%
\pgfpathlineto{\pgfqpoint{0.877895in}{1.010820in}}%
\pgfpathlineto{\pgfqpoint{0.880751in}{0.998874in}}%
\pgfpathlineto{\pgfqpoint{0.883591in}{0.987385in}}%
\pgfpathlineto{\pgfqpoint{0.886415in}{0.976346in}}%
\pgfpathlineto{\pgfqpoint{0.889223in}{0.965750in}}%
\pgfpathlineto{\pgfqpoint{0.904790in}{0.958182in}}%
\pgfpathlineto{\pgfqpoint{0.920857in}{0.950869in}}%
\pgfpathlineto{\pgfqpoint{0.937407in}{0.943820in}}%
\pgfpathlineto{\pgfqpoint{0.954422in}{0.937045in}}%
\pgfpathclose%
\pgfusepath{fill}%
\end{pgfscope}%
\begin{pgfscope}%
\pgfpathrectangle{\pgfqpoint{0.329460in}{0.284240in}}{\pgfqpoint{1.989680in}{1.989680in}}%
\pgfusepath{clip}%
\pgfsetbuttcap%
\pgfsetroundjoin%
\definecolor{currentfill}{rgb}{0.260571,0.246922,0.522828}%
\pgfsetfillcolor{currentfill}%
\pgfsetlinewidth{0.000000pt}%
\definecolor{currentstroke}{rgb}{0.000000,0.000000,0.000000}%
\pgfsetstrokecolor{currentstroke}%
\pgfsetdash{}{0pt}%
\pgfpathmoveto{\pgfqpoint{1.116859in}{0.826632in}}%
\pgfpathlineto{\pgfqpoint{1.115409in}{0.833598in}}%
\pgfpathlineto{\pgfqpoint{1.113951in}{0.840952in}}%
\pgfpathlineto{\pgfqpoint{1.112486in}{0.848702in}}%
\pgfpathlineto{\pgfqpoint{1.111013in}{0.856854in}}%
\pgfpathlineto{\pgfqpoint{1.091448in}{0.861085in}}%
\pgfpathlineto{\pgfqpoint{1.072180in}{0.865641in}}%
\pgfpathlineto{\pgfqpoint{1.053230in}{0.870517in}}%
\pgfpathlineto{\pgfqpoint{1.034621in}{0.875706in}}%
\pgfpathlineto{\pgfqpoint{1.036556in}{0.867462in}}%
\pgfpathlineto{\pgfqpoint{1.038482in}{0.859619in}}%
\pgfpathlineto{\pgfqpoint{1.040398in}{0.852170in}}%
\pgfpathlineto{\pgfqpoint{1.042305in}{0.845110in}}%
\pgfpathlineto{\pgfqpoint{1.060468in}{0.840023in}}%
\pgfpathlineto{\pgfqpoint{1.078962in}{0.835244in}}%
\pgfpathlineto{\pgfqpoint{1.097766in}{0.830779in}}%
\pgfpathlineto{\pgfqpoint{1.116859in}{0.826632in}}%
\pgfpathclose%
\pgfusepath{fill}%
\end{pgfscope}%
\begin{pgfscope}%
\pgfpathrectangle{\pgfqpoint{0.329460in}{0.284240in}}{\pgfqpoint{1.989680in}{1.989680in}}%
\pgfusepath{clip}%
\pgfsetbuttcap%
\pgfsetroundjoin%
\definecolor{currentfill}{rgb}{0.233603,0.313828,0.543914}%
\pgfsetfillcolor{currentfill}%
\pgfsetlinewidth{0.000000pt}%
\definecolor{currentstroke}{rgb}{0.000000,0.000000,0.000000}%
\pgfsetstrokecolor{currentstroke}%
\pgfsetdash{}{0pt}%
\pgfpathmoveto{\pgfqpoint{1.034621in}{0.875706in}}%
\pgfpathlineto{\pgfqpoint{1.032676in}{0.884359in}}%
\pgfpathlineto{\pgfqpoint{1.030720in}{0.893428in}}%
\pgfpathlineto{\pgfqpoint{1.028754in}{0.902920in}}%
\pgfpathlineto{\pgfqpoint{1.026777in}{0.912841in}}%
\pgfpathlineto{\pgfqpoint{1.008083in}{0.918442in}}%
\pgfpathlineto{\pgfqpoint{0.989779in}{0.924348in}}%
\pgfpathlineto{\pgfqpoint{0.971886in}{0.930551in}}%
\pgfpathlineto{\pgfqpoint{0.954422in}{0.937045in}}%
\pgfpathlineto{\pgfqpoint{0.956831in}{0.927013in}}%
\pgfpathlineto{\pgfqpoint{0.959227in}{0.917409in}}%
\pgfpathlineto{\pgfqpoint{0.961611in}{0.908228in}}%
\pgfpathlineto{\pgfqpoint{0.963981in}{0.899460in}}%
\pgfpathlineto{\pgfqpoint{0.981033in}{0.893087in}}%
\pgfpathlineto{\pgfqpoint{0.998503in}{0.886998in}}%
\pgfpathlineto{\pgfqpoint{1.016372in}{0.881202in}}%
\pgfpathlineto{\pgfqpoint{1.034621in}{0.875706in}}%
\pgfpathclose%
\pgfusepath{fill}%
\end{pgfscope}%
\begin{pgfscope}%
\pgfpathrectangle{\pgfqpoint{0.329460in}{0.284240in}}{\pgfqpoint{1.989680in}{1.989680in}}%
\pgfusepath{clip}%
\pgfsetbuttcap%
\pgfsetroundjoin%
\definecolor{currentfill}{rgb}{0.276194,0.190074,0.493001}%
\pgfsetfillcolor{currentfill}%
\pgfsetlinewidth{0.000000pt}%
\definecolor{currentstroke}{rgb}{0.000000,0.000000,0.000000}%
\pgfsetstrokecolor{currentstroke}%
\pgfsetdash{}{0pt}%
\pgfpathmoveto{\pgfqpoint{1.520251in}{0.791952in}}%
\pgfpathlineto{\pgfqpoint{1.521306in}{0.797370in}}%
\pgfpathlineto{\pgfqpoint{1.522365in}{0.803152in}}%
\pgfpathlineto{\pgfqpoint{1.523429in}{0.809303in}}%
\pgfpathlineto{\pgfqpoint{1.524499in}{0.815830in}}%
\pgfpathlineto{\pgfqpoint{1.504444in}{0.813048in}}%
\pgfpathlineto{\pgfqpoint{1.484214in}{0.810608in}}%
\pgfpathlineto{\pgfqpoint{1.463829in}{0.808514in}}%
\pgfpathlineto{\pgfqpoint{1.443316in}{0.806768in}}%
\pgfpathlineto{\pgfqpoint{1.442746in}{0.800288in}}%
\pgfpathlineto{\pgfqpoint{1.442180in}{0.794184in}}%
\pgfpathlineto{\pgfqpoint{1.441616in}{0.788450in}}%
\pgfpathlineto{\pgfqpoint{1.441054in}{0.783080in}}%
\pgfpathlineto{\pgfqpoint{1.461066in}{0.784790in}}%
\pgfpathlineto{\pgfqpoint{1.480950in}{0.786840in}}%
\pgfpathlineto{\pgfqpoint{1.500686in}{0.789229in}}%
\pgfpathlineto{\pgfqpoint{1.520251in}{0.791952in}}%
\pgfpathclose%
\pgfusepath{fill}%
\end{pgfscope}%
\begin{pgfscope}%
\pgfpathrectangle{\pgfqpoint{0.329460in}{0.284240in}}{\pgfqpoint{1.989680in}{1.989680in}}%
\pgfusepath{clip}%
\pgfsetbuttcap%
\pgfsetroundjoin%
\definecolor{currentfill}{rgb}{0.276194,0.190074,0.493001}%
\pgfsetfillcolor{currentfill}%
\pgfsetlinewidth{0.000000pt}%
\definecolor{currentstroke}{rgb}{0.000000,0.000000,0.000000}%
\pgfsetstrokecolor{currentstroke}%
\pgfsetdash{}{0pt}%
\pgfpathmoveto{\pgfqpoint{1.279196in}{0.781849in}}%
\pgfpathlineto{\pgfqpoint{1.278746in}{0.787212in}}%
\pgfpathlineto{\pgfqpoint{1.278294in}{0.792939in}}%
\pgfpathlineto{\pgfqpoint{1.277840in}{0.799036in}}%
\pgfpathlineto{\pgfqpoint{1.277384in}{0.805510in}}%
\pgfpathlineto{\pgfqpoint{1.256775in}{0.806944in}}%
\pgfpathlineto{\pgfqpoint{1.236274in}{0.808730in}}%
\pgfpathlineto{\pgfqpoint{1.215906in}{0.810862in}}%
\pgfpathlineto{\pgfqpoint{1.195693in}{0.813340in}}%
\pgfpathlineto{\pgfqpoint{1.196653in}{0.806826in}}%
\pgfpathlineto{\pgfqpoint{1.197609in}{0.800688in}}%
\pgfpathlineto{\pgfqpoint{1.198559in}{0.794920in}}%
\pgfpathlineto{\pgfqpoint{1.199506in}{0.789515in}}%
\pgfpathlineto{\pgfqpoint{1.219224in}{0.787089in}}%
\pgfpathlineto{\pgfqpoint{1.239093in}{0.785001in}}%
\pgfpathlineto{\pgfqpoint{1.259092in}{0.783253in}}%
\pgfpathlineto{\pgfqpoint{1.279196in}{0.781849in}}%
\pgfpathclose%
\pgfusepath{fill}%
\end{pgfscope}%
\begin{pgfscope}%
\pgfpathrectangle{\pgfqpoint{0.329460in}{0.284240in}}{\pgfqpoint{1.989680in}{1.989680in}}%
\pgfusepath{clip}%
\pgfsetbuttcap%
\pgfsetroundjoin%
\definecolor{currentfill}{rgb}{0.276194,0.190074,0.493001}%
\pgfsetfillcolor{currentfill}%
\pgfsetlinewidth{0.000000pt}%
\definecolor{currentstroke}{rgb}{0.000000,0.000000,0.000000}%
\pgfsetstrokecolor{currentstroke}%
\pgfsetdash{}{0pt}%
\pgfpathmoveto{\pgfqpoint{1.441054in}{0.783080in}}%
\pgfpathlineto{\pgfqpoint{1.441616in}{0.788450in}}%
\pgfpathlineto{\pgfqpoint{1.442180in}{0.794184in}}%
\pgfpathlineto{\pgfqpoint{1.442746in}{0.800288in}}%
\pgfpathlineto{\pgfqpoint{1.443316in}{0.806768in}}%
\pgfpathlineto{\pgfqpoint{1.422695in}{0.805372in}}%
\pgfpathlineto{\pgfqpoint{1.401993in}{0.804328in}}%
\pgfpathlineto{\pgfqpoint{1.381231in}{0.803639in}}%
\pgfpathlineto{\pgfqpoint{1.360435in}{0.803303in}}%
\pgfpathlineto{\pgfqpoint{1.360378in}{0.796841in}}%
\pgfpathlineto{\pgfqpoint{1.360321in}{0.790756in}}%
\pgfpathlineto{\pgfqpoint{1.360264in}{0.785041in}}%
\pgfpathlineto{\pgfqpoint{1.360208in}{0.779689in}}%
\pgfpathlineto{\pgfqpoint{1.380494in}{0.780017in}}%
\pgfpathlineto{\pgfqpoint{1.400745in}{0.780693in}}%
\pgfpathlineto{\pgfqpoint{1.420940in}{0.781714in}}%
\pgfpathlineto{\pgfqpoint{1.441054in}{0.783080in}}%
\pgfpathclose%
\pgfusepath{fill}%
\end{pgfscope}%
\begin{pgfscope}%
\pgfpathrectangle{\pgfqpoint{0.329460in}{0.284240in}}{\pgfqpoint{1.989680in}{1.989680in}}%
\pgfusepath{clip}%
\pgfsetbuttcap%
\pgfsetroundjoin%
\definecolor{currentfill}{rgb}{0.276194,0.190074,0.493001}%
\pgfsetfillcolor{currentfill}%
\pgfsetlinewidth{0.000000pt}%
\definecolor{currentstroke}{rgb}{0.000000,0.000000,0.000000}%
\pgfsetstrokecolor{currentstroke}%
\pgfsetdash{}{0pt}%
\pgfpathmoveto{\pgfqpoint{1.360208in}{0.779689in}}%
\pgfpathlineto{\pgfqpoint{1.360264in}{0.785041in}}%
\pgfpathlineto{\pgfqpoint{1.360321in}{0.790756in}}%
\pgfpathlineto{\pgfqpoint{1.360378in}{0.796841in}}%
\pgfpathlineto{\pgfqpoint{1.360435in}{0.803303in}}%
\pgfpathlineto{\pgfqpoint{1.339628in}{0.803323in}}%
\pgfpathlineto{\pgfqpoint{1.318835in}{0.803698in}}%
\pgfpathlineto{\pgfqpoint{1.298079in}{0.804427in}}%
\pgfpathlineto{\pgfqpoint{1.277384in}{0.805510in}}%
\pgfpathlineto{\pgfqpoint{1.277840in}{0.799036in}}%
\pgfpathlineto{\pgfqpoint{1.278294in}{0.792939in}}%
\pgfpathlineto{\pgfqpoint{1.278746in}{0.787212in}}%
\pgfpathlineto{\pgfqpoint{1.279196in}{0.781849in}}%
\pgfpathlineto{\pgfqpoint{1.299383in}{0.780789in}}%
\pgfpathlineto{\pgfqpoint{1.319629in}{0.780075in}}%
\pgfpathlineto{\pgfqpoint{1.339912in}{0.779709in}}%
\pgfpathlineto{\pgfqpoint{1.360208in}{0.779689in}}%
\pgfpathclose%
\pgfusepath{fill}%
\end{pgfscope}%
\begin{pgfscope}%
\pgfpathrectangle{\pgfqpoint{0.329460in}{0.284240in}}{\pgfqpoint{1.989680in}{1.989680in}}%
\pgfusepath{clip}%
\pgfsetbuttcap%
\pgfsetroundjoin%
\definecolor{currentfill}{rgb}{0.172719,0.448791,0.557885}%
\pgfsetfillcolor{currentfill}%
\pgfsetlinewidth{0.000000pt}%
\definecolor{currentstroke}{rgb}{0.000000,0.000000,0.000000}%
\pgfsetstrokecolor{currentstroke}%
\pgfsetdash{}{0pt}%
\pgfpathmoveto{\pgfqpoint{1.838202in}{1.017875in}}%
\pgfpathlineto{\pgfqpoint{1.841155in}{1.030315in}}%
\pgfpathlineto{\pgfqpoint{1.844126in}{1.043229in}}%
\pgfpathlineto{\pgfqpoint{1.847115in}{1.056623in}}%
\pgfpathlineto{\pgfqpoint{1.831376in}{1.048603in}}%
\pgfpathlineto{\pgfqpoint{1.815099in}{1.040836in}}%
\pgfpathlineto{\pgfqpoint{1.798301in}{1.033333in}}%
\pgfpathlineto{\pgfqpoint{1.780998in}{1.026103in}}%
\pgfpathlineto{\pgfqpoint{1.778395in}{1.012830in}}%
\pgfpathlineto{\pgfqpoint{1.775808in}{1.000041in}}%
\pgfpathlineto{\pgfqpoint{1.773237in}{0.987726in}}%
\pgfpathlineto{\pgfqpoint{1.790236in}{0.994868in}}%
\pgfpathlineto{\pgfqpoint{1.806741in}{1.002280in}}%
\pgfpathlineto{\pgfqpoint{1.822735in}{1.009952in}}%
\pgfpathlineto{\pgfqpoint{1.838202in}{1.017875in}}%
\pgfpathclose%
\pgfusepath{fill}%
\end{pgfscope}%
\begin{pgfscope}%
\pgfpathrectangle{\pgfqpoint{0.329460in}{0.284240in}}{\pgfqpoint{1.989680in}{1.989680in}}%
\pgfusepath{clip}%
\pgfsetbuttcap%
\pgfsetroundjoin%
\definecolor{currentfill}{rgb}{0.260571,0.246922,0.522828}%
\pgfsetfillcolor{currentfill}%
\pgfsetlinewidth{0.000000pt}%
\definecolor{currentstroke}{rgb}{0.000000,0.000000,0.000000}%
\pgfsetstrokecolor{currentstroke}%
\pgfsetdash{}{0pt}%
\pgfpathmoveto{\pgfqpoint{1.602502in}{0.830302in}}%
\pgfpathlineto{\pgfqpoint{1.604057in}{0.837286in}}%
\pgfpathlineto{\pgfqpoint{1.605619in}{0.844660in}}%
\pgfpathlineto{\pgfqpoint{1.607189in}{0.852428in}}%
\pgfpathlineto{\pgfqpoint{1.608768in}{0.860599in}}%
\pgfpathlineto{\pgfqpoint{1.589170in}{0.856405in}}%
\pgfpathlineto{\pgfqpoint{1.569300in}{0.852542in}}%
\pgfpathlineto{\pgfqpoint{1.549179in}{0.849017in}}%
\pgfpathlineto{\pgfqpoint{1.528830in}{0.845833in}}%
\pgfpathlineto{\pgfqpoint{1.527739in}{0.837734in}}%
\pgfpathlineto{\pgfqpoint{1.526653in}{0.830039in}}%
\pgfpathlineto{\pgfqpoint{1.525573in}{0.822740in}}%
\pgfpathlineto{\pgfqpoint{1.524499in}{0.815830in}}%
\pgfpathlineto{\pgfqpoint{1.544354in}{0.818951in}}%
\pgfpathlineto{\pgfqpoint{1.563988in}{0.822406in}}%
\pgfpathlineto{\pgfqpoint{1.583378in}{0.826191in}}%
\pgfpathlineto{\pgfqpoint{1.602502in}{0.830302in}}%
\pgfpathclose%
\pgfusepath{fill}%
\end{pgfscope}%
\begin{pgfscope}%
\pgfpathrectangle{\pgfqpoint{0.329460in}{0.284240in}}{\pgfqpoint{1.989680in}{1.989680in}}%
\pgfusepath{clip}%
\pgfsetbuttcap%
\pgfsetroundjoin%
\definecolor{currentfill}{rgb}{0.260571,0.246922,0.522828}%
\pgfsetfillcolor{currentfill}%
\pgfsetlinewidth{0.000000pt}%
\definecolor{currentstroke}{rgb}{0.000000,0.000000,0.000000}%
\pgfsetstrokecolor{currentstroke}%
\pgfsetdash{}{0pt}%
\pgfpathmoveto{\pgfqpoint{1.195693in}{0.813340in}}%
\pgfpathlineto{\pgfqpoint{1.194729in}{0.820237in}}%
\pgfpathlineto{\pgfqpoint{1.193760in}{0.827524in}}%
\pgfpathlineto{\pgfqpoint{1.192786in}{0.835206in}}%
\pgfpathlineto{\pgfqpoint{1.191806in}{0.843292in}}%
\pgfpathlineto{\pgfqpoint{1.171274in}{0.846169in}}%
\pgfpathlineto{\pgfqpoint{1.150949in}{0.849392in}}%
\pgfpathlineto{\pgfqpoint{1.130855in}{0.852955in}}%
\pgfpathlineto{\pgfqpoint{1.111013in}{0.856854in}}%
\pgfpathlineto{\pgfqpoint{1.112486in}{0.848702in}}%
\pgfpathlineto{\pgfqpoint{1.113951in}{0.840952in}}%
\pgfpathlineto{\pgfqpoint{1.115409in}{0.833598in}}%
\pgfpathlineto{\pgfqpoint{1.116859in}{0.826632in}}%
\pgfpathlineto{\pgfqpoint{1.136220in}{0.822811in}}%
\pgfpathlineto{\pgfqpoint{1.155828in}{0.819318in}}%
\pgfpathlineto{\pgfqpoint{1.175660in}{0.816160in}}%
\pgfpathlineto{\pgfqpoint{1.195693in}{0.813340in}}%
\pgfpathclose%
\pgfusepath{fill}%
\end{pgfscope}%
\begin{pgfscope}%
\pgfpathrectangle{\pgfqpoint{0.329460in}{0.284240in}}{\pgfqpoint{1.989680in}{1.989680in}}%
\pgfusepath{clip}%
\pgfsetbuttcap%
\pgfsetroundjoin%
\definecolor{currentfill}{rgb}{0.233603,0.313828,0.543914}%
\pgfsetfillcolor{currentfill}%
\pgfsetlinewidth{0.000000pt}%
\definecolor{currentstroke}{rgb}{0.000000,0.000000,0.000000}%
\pgfsetstrokecolor{currentstroke}%
\pgfsetdash{}{0pt}%
\pgfpathmoveto{\pgfqpoint{1.683994in}{0.880577in}}%
\pgfpathlineto{\pgfqpoint{1.686038in}{0.889254in}}%
\pgfpathlineto{\pgfqpoint{1.688092in}{0.898346in}}%
\pgfpathlineto{\pgfqpoint{1.690158in}{0.907860in}}%
\pgfpathlineto{\pgfqpoint{1.692234in}{0.917804in}}%
\pgfpathlineto{\pgfqpoint{1.673498in}{0.912238in}}%
\pgfpathlineto{\pgfqpoint{1.654394in}{0.906985in}}%
\pgfpathlineto{\pgfqpoint{1.634942in}{0.902052in}}%
\pgfpathlineto{\pgfqpoint{1.615165in}{0.897445in}}%
\pgfpathlineto{\pgfqpoint{1.613553in}{0.887594in}}%
\pgfpathlineto{\pgfqpoint{1.611949in}{0.878175in}}%
\pgfpathlineto{\pgfqpoint{1.610354in}{0.869179in}}%
\pgfpathlineto{\pgfqpoint{1.608768in}{0.860599in}}%
\pgfpathlineto{\pgfqpoint{1.628070in}{0.865119in}}%
\pgfpathlineto{\pgfqpoint{1.647056in}{0.869960in}}%
\pgfpathlineto{\pgfqpoint{1.665704in}{0.875114in}}%
\pgfpathlineto{\pgfqpoint{1.683994in}{0.880577in}}%
\pgfpathclose%
\pgfusepath{fill}%
\end{pgfscope}%
\begin{pgfscope}%
\pgfpathrectangle{\pgfqpoint{0.329460in}{0.284240in}}{\pgfqpoint{1.989680in}{1.989680in}}%
\pgfusepath{clip}%
\pgfsetbuttcap%
\pgfsetroundjoin%
\definecolor{currentfill}{rgb}{0.201239,0.383670,0.554294}%
\pgfsetfillcolor{currentfill}%
\pgfsetlinewidth{0.000000pt}%
\definecolor{currentstroke}{rgb}{0.000000,0.000000,0.000000}%
\pgfsetstrokecolor{currentstroke}%
\pgfsetdash{}{0pt}%
\pgfpathmoveto{\pgfqpoint{1.763101in}{0.943054in}}%
\pgfpathlineto{\pgfqpoint{1.765613in}{0.953549in}}%
\pgfpathlineto{\pgfqpoint{1.768140in}{0.964488in}}%
\pgfpathlineto{\pgfqpoint{1.770681in}{0.975878in}}%
\pgfpathlineto{\pgfqpoint{1.773237in}{0.987726in}}%
\pgfpathlineto{\pgfqpoint{1.755761in}{0.980862in}}%
\pgfpathlineto{\pgfqpoint{1.737827in}{0.974286in}}%
\pgfpathlineto{\pgfqpoint{1.719453in}{0.968005in}}%
\pgfpathlineto{\pgfqpoint{1.700660in}{0.962028in}}%
\pgfpathlineto{\pgfqpoint{1.698535in}{0.950290in}}%
\pgfpathlineto{\pgfqpoint{1.696423in}{0.939012in}}%
\pgfpathlineto{\pgfqpoint{1.694323in}{0.928186in}}%
\pgfpathlineto{\pgfqpoint{1.692234in}{0.917804in}}%
\pgfpathlineto{\pgfqpoint{1.710582in}{0.923677in}}%
\pgfpathlineto{\pgfqpoint{1.728522in}{0.929848in}}%
\pgfpathlineto{\pgfqpoint{1.746034in}{0.936310in}}%
\pgfpathlineto{\pgfqpoint{1.763101in}{0.943054in}}%
\pgfpathclose%
\pgfusepath{fill}%
\end{pgfscope}%
\begin{pgfscope}%
\pgfpathrectangle{\pgfqpoint{0.329460in}{0.284240in}}{\pgfqpoint{1.989680in}{1.989680in}}%
\pgfusepath{clip}%
\pgfsetbuttcap%
\pgfsetroundjoin%
\definecolor{currentfill}{rgb}{0.172719,0.448791,0.557885}%
\pgfsetfillcolor{currentfill}%
\pgfsetlinewidth{0.000000pt}%
\definecolor{currentstroke}{rgb}{0.000000,0.000000,0.000000}%
\pgfsetstrokecolor{currentstroke}%
\pgfsetdash{}{0pt}%
\pgfpathmoveto{\pgfqpoint{0.944649in}{0.981611in}}%
\pgfpathlineto{\pgfqpoint{0.942170in}{0.993900in}}%
\pgfpathlineto{\pgfqpoint{0.939676in}{1.006664in}}%
\pgfpathlineto{\pgfqpoint{0.937167in}{1.019912in}}%
\pgfpathlineto{\pgfqpoint{0.919430in}{1.026892in}}%
\pgfpathlineto{\pgfqpoint{0.902183in}{1.034154in}}%
\pgfpathlineto{\pgfqpoint{0.885441in}{1.041686in}}%
\pgfpathlineto{\pgfqpoint{0.869223in}{1.049482in}}%
\pgfpathlineto{\pgfqpoint{0.872131in}{1.036116in}}%
\pgfpathlineto{\pgfqpoint{0.875022in}{1.023231in}}%
\pgfpathlineto{\pgfqpoint{0.877895in}{1.010820in}}%
\pgfpathlineto{\pgfqpoint{0.893831in}{1.003120in}}%
\pgfpathlineto{\pgfqpoint{0.910280in}{0.995678in}}%
\pgfpathlineto{\pgfqpoint{0.927225in}{0.988506in}}%
\pgfpathlineto{\pgfqpoint{0.944649in}{0.981611in}}%
\pgfpathclose%
\pgfusepath{fill}%
\end{pgfscope}%
\begin{pgfscope}%
\pgfpathrectangle{\pgfqpoint{0.329460in}{0.284240in}}{\pgfqpoint{1.989680in}{1.989680in}}%
\pgfusepath{clip}%
\pgfsetbuttcap%
\pgfsetroundjoin%
\definecolor{currentfill}{rgb}{0.233603,0.313828,0.543914}%
\pgfsetfillcolor{currentfill}%
\pgfsetlinewidth{0.000000pt}%
\definecolor{currentstroke}{rgb}{0.000000,0.000000,0.000000}%
\pgfsetstrokecolor{currentstroke}%
\pgfsetdash{}{0pt}%
\pgfpathmoveto{\pgfqpoint{1.111013in}{0.856854in}}%
\pgfpathlineto{\pgfqpoint{1.109533in}{0.865416in}}%
\pgfpathlineto{\pgfqpoint{1.108045in}{0.874394in}}%
\pgfpathlineto{\pgfqpoint{1.106549in}{0.883796in}}%
\pgfpathlineto{\pgfqpoint{1.105045in}{0.893629in}}%
\pgfpathlineto{\pgfqpoint{1.084997in}{0.897940in}}%
\pgfpathlineto{\pgfqpoint{1.065255in}{0.902584in}}%
\pgfpathlineto{\pgfqpoint{1.045841in}{0.907553in}}%
\pgfpathlineto{\pgfqpoint{1.026777in}{0.912841in}}%
\pgfpathlineto{\pgfqpoint{1.028754in}{0.902920in}}%
\pgfpathlineto{\pgfqpoint{1.030720in}{0.893428in}}%
\pgfpathlineto{\pgfqpoint{1.032676in}{0.884359in}}%
\pgfpathlineto{\pgfqpoint{1.034621in}{0.875706in}}%
\pgfpathlineto{\pgfqpoint{1.053230in}{0.870517in}}%
\pgfpathlineto{\pgfqpoint{1.072180in}{0.865641in}}%
\pgfpathlineto{\pgfqpoint{1.091448in}{0.861085in}}%
\pgfpathlineto{\pgfqpoint{1.111013in}{0.856854in}}%
\pgfpathclose%
\pgfusepath{fill}%
\end{pgfscope}%
\begin{pgfscope}%
\pgfpathrectangle{\pgfqpoint{0.329460in}{0.284240in}}{\pgfqpoint{1.989680in}{1.989680in}}%
\pgfusepath{clip}%
\pgfsetbuttcap%
\pgfsetroundjoin%
\definecolor{currentfill}{rgb}{0.201239,0.383670,0.554294}%
\pgfsetfillcolor{currentfill}%
\pgfsetlinewidth{0.000000pt}%
\definecolor{currentstroke}{rgb}{0.000000,0.000000,0.000000}%
\pgfsetstrokecolor{currentstroke}%
\pgfsetdash{}{0pt}%
\pgfpathmoveto{\pgfqpoint{1.026777in}{0.912841in}}%
\pgfpathlineto{\pgfqpoint{1.024789in}{0.923200in}}%
\pgfpathlineto{\pgfqpoint{1.022790in}{0.934003in}}%
\pgfpathlineto{\pgfqpoint{1.020779in}{0.945260in}}%
\pgfpathlineto{\pgfqpoint{1.018756in}{0.956976in}}%
\pgfpathlineto{\pgfqpoint{0.999607in}{0.962677in}}%
\pgfpathlineto{\pgfqpoint{0.980859in}{0.968688in}}%
\pgfpathlineto{\pgfqpoint{0.962534in}{0.975002in}}%
\pgfpathlineto{\pgfqpoint{0.944649in}{0.981611in}}%
\pgfpathlineto{\pgfqpoint{0.947114in}{0.969789in}}%
\pgfpathlineto{\pgfqpoint{0.949564in}{0.958425in}}%
\pgfpathlineto{\pgfqpoint{0.952000in}{0.947513in}}%
\pgfpathlineto{\pgfqpoint{0.954422in}{0.937045in}}%
\pgfpathlineto{\pgfqpoint{0.971886in}{0.930551in}}%
\pgfpathlineto{\pgfqpoint{0.989779in}{0.924348in}}%
\pgfpathlineto{\pgfqpoint{1.008083in}{0.918442in}}%
\pgfpathlineto{\pgfqpoint{1.026777in}{0.912841in}}%
\pgfpathclose%
\pgfusepath{fill}%
\end{pgfscope}%
\begin{pgfscope}%
\pgfpathrectangle{\pgfqpoint{0.329460in}{0.284240in}}{\pgfqpoint{1.989680in}{1.989680in}}%
\pgfusepath{clip}%
\pgfsetbuttcap%
\pgfsetroundjoin%
\definecolor{currentfill}{rgb}{0.260571,0.246922,0.522828}%
\pgfsetfillcolor{currentfill}%
\pgfsetlinewidth{0.000000pt}%
\definecolor{currentstroke}{rgb}{0.000000,0.000000,0.000000}%
\pgfsetstrokecolor{currentstroke}%
\pgfsetdash{}{0pt}%
\pgfpathmoveto{\pgfqpoint{1.524499in}{0.815830in}}%
\pgfpathlineto{\pgfqpoint{1.525573in}{0.822740in}}%
\pgfpathlineto{\pgfqpoint{1.526653in}{0.830039in}}%
\pgfpathlineto{\pgfqpoint{1.527739in}{0.837734in}}%
\pgfpathlineto{\pgfqpoint{1.528830in}{0.845833in}}%
\pgfpathlineto{\pgfqpoint{1.508276in}{0.842994in}}%
\pgfpathlineto{\pgfqpoint{1.487540in}{0.840504in}}%
\pgfpathlineto{\pgfqpoint{1.466647in}{0.838367in}}%
\pgfpathlineto{\pgfqpoint{1.445621in}{0.836585in}}%
\pgfpathlineto{\pgfqpoint{1.445040in}{0.828532in}}%
\pgfpathlineto{\pgfqpoint{1.444462in}{0.820883in}}%
\pgfpathlineto{\pgfqpoint{1.443888in}{0.813631in}}%
\pgfpathlineto{\pgfqpoint{1.443316in}{0.806768in}}%
\pgfpathlineto{\pgfqpoint{1.463829in}{0.808514in}}%
\pgfpathlineto{\pgfqpoint{1.484214in}{0.810608in}}%
\pgfpathlineto{\pgfqpoint{1.504444in}{0.813048in}}%
\pgfpathlineto{\pgfqpoint{1.524499in}{0.815830in}}%
\pgfpathclose%
\pgfusepath{fill}%
\end{pgfscope}%
\begin{pgfscope}%
\pgfpathrectangle{\pgfqpoint{0.329460in}{0.284240in}}{\pgfqpoint{1.989680in}{1.989680in}}%
\pgfusepath{clip}%
\pgfsetbuttcap%
\pgfsetroundjoin%
\definecolor{currentfill}{rgb}{0.260571,0.246922,0.522828}%
\pgfsetfillcolor{currentfill}%
\pgfsetlinewidth{0.000000pt}%
\definecolor{currentstroke}{rgb}{0.000000,0.000000,0.000000}%
\pgfsetstrokecolor{currentstroke}%
\pgfsetdash{}{0pt}%
\pgfpathmoveto{\pgfqpoint{1.277384in}{0.805510in}}%
\pgfpathlineto{\pgfqpoint{1.276926in}{0.812366in}}%
\pgfpathlineto{\pgfqpoint{1.276465in}{0.819612in}}%
\pgfpathlineto{\pgfqpoint{1.276002in}{0.827255in}}%
\pgfpathlineto{\pgfqpoint{1.275537in}{0.835301in}}%
\pgfpathlineto{\pgfqpoint{1.254412in}{0.836765in}}%
\pgfpathlineto{\pgfqpoint{1.233399in}{0.838587in}}%
\pgfpathlineto{\pgfqpoint{1.212523in}{0.840763in}}%
\pgfpathlineto{\pgfqpoint{1.191806in}{0.843292in}}%
\pgfpathlineto{\pgfqpoint{1.192786in}{0.835206in}}%
\pgfpathlineto{\pgfqpoint{1.193760in}{0.827524in}}%
\pgfpathlineto{\pgfqpoint{1.194729in}{0.820237in}}%
\pgfpathlineto{\pgfqpoint{1.195693in}{0.813340in}}%
\pgfpathlineto{\pgfqpoint{1.215906in}{0.810862in}}%
\pgfpathlineto{\pgfqpoint{1.236274in}{0.808730in}}%
\pgfpathlineto{\pgfqpoint{1.256775in}{0.806944in}}%
\pgfpathlineto{\pgfqpoint{1.277384in}{0.805510in}}%
\pgfpathclose%
\pgfusepath{fill}%
\end{pgfscope}%
\begin{pgfscope}%
\pgfpathrectangle{\pgfqpoint{0.329460in}{0.284240in}}{\pgfqpoint{1.989680in}{1.989680in}}%
\pgfusepath{clip}%
\pgfsetbuttcap%
\pgfsetroundjoin%
\definecolor{currentfill}{rgb}{0.260571,0.246922,0.522828}%
\pgfsetfillcolor{currentfill}%
\pgfsetlinewidth{0.000000pt}%
\definecolor{currentstroke}{rgb}{0.000000,0.000000,0.000000}%
\pgfsetstrokecolor{currentstroke}%
\pgfsetdash{}{0pt}%
\pgfpathmoveto{\pgfqpoint{1.443316in}{0.806768in}}%
\pgfpathlineto{\pgfqpoint{1.443888in}{0.813631in}}%
\pgfpathlineto{\pgfqpoint{1.444462in}{0.820883in}}%
\pgfpathlineto{\pgfqpoint{1.445040in}{0.828532in}}%
\pgfpathlineto{\pgfqpoint{1.445621in}{0.836585in}}%
\pgfpathlineto{\pgfqpoint{1.424485in}{0.835160in}}%
\pgfpathlineto{\pgfqpoint{1.403264in}{0.834096in}}%
\pgfpathlineto{\pgfqpoint{1.381983in}{0.833392in}}%
\pgfpathlineto{\pgfqpoint{1.360667in}{0.833050in}}%
\pgfpathlineto{\pgfqpoint{1.360608in}{0.825015in}}%
\pgfpathlineto{\pgfqpoint{1.360550in}{0.817383in}}%
\pgfpathlineto{\pgfqpoint{1.360493in}{0.810148in}}%
\pgfpathlineto{\pgfqpoint{1.360435in}{0.803303in}}%
\pgfpathlineto{\pgfqpoint{1.381231in}{0.803639in}}%
\pgfpathlineto{\pgfqpoint{1.401993in}{0.804328in}}%
\pgfpathlineto{\pgfqpoint{1.422695in}{0.805372in}}%
\pgfpathlineto{\pgfqpoint{1.443316in}{0.806768in}}%
\pgfpathclose%
\pgfusepath{fill}%
\end{pgfscope}%
\begin{pgfscope}%
\pgfpathrectangle{\pgfqpoint{0.329460in}{0.284240in}}{\pgfqpoint{1.989680in}{1.989680in}}%
\pgfusepath{clip}%
\pgfsetbuttcap%
\pgfsetroundjoin%
\definecolor{currentfill}{rgb}{0.260571,0.246922,0.522828}%
\pgfsetfillcolor{currentfill}%
\pgfsetlinewidth{0.000000pt}%
\definecolor{currentstroke}{rgb}{0.000000,0.000000,0.000000}%
\pgfsetstrokecolor{currentstroke}%
\pgfsetdash{}{0pt}%
\pgfpathmoveto{\pgfqpoint{1.360435in}{0.803303in}}%
\pgfpathlineto{\pgfqpoint{1.360493in}{0.810148in}}%
\pgfpathlineto{\pgfqpoint{1.360550in}{0.817383in}}%
\pgfpathlineto{\pgfqpoint{1.360608in}{0.825015in}}%
\pgfpathlineto{\pgfqpoint{1.360667in}{0.833050in}}%
\pgfpathlineto{\pgfqpoint{1.339339in}{0.833070in}}%
\pgfpathlineto{\pgfqpoint{1.318025in}{0.833452in}}%
\pgfpathlineto{\pgfqpoint{1.296749in}{0.834196in}}%
\pgfpathlineto{\pgfqpoint{1.275537in}{0.835301in}}%
\pgfpathlineto{\pgfqpoint{1.276002in}{0.827255in}}%
\pgfpathlineto{\pgfqpoint{1.276465in}{0.819612in}}%
\pgfpathlineto{\pgfqpoint{1.276926in}{0.812366in}}%
\pgfpathlineto{\pgfqpoint{1.277384in}{0.805510in}}%
\pgfpathlineto{\pgfqpoint{1.298079in}{0.804427in}}%
\pgfpathlineto{\pgfqpoint{1.318835in}{0.803698in}}%
\pgfpathlineto{\pgfqpoint{1.339628in}{0.803323in}}%
\pgfpathlineto{\pgfqpoint{1.360435in}{0.803303in}}%
\pgfpathclose%
\pgfusepath{fill}%
\end{pgfscope}%
\begin{pgfscope}%
\pgfpathrectangle{\pgfqpoint{0.329460in}{0.284240in}}{\pgfqpoint{1.989680in}{1.989680in}}%
\pgfusepath{clip}%
\pgfsetbuttcap%
\pgfsetroundjoin%
\definecolor{currentfill}{rgb}{0.233603,0.313828,0.543914}%
\pgfsetfillcolor{currentfill}%
\pgfsetlinewidth{0.000000pt}%
\definecolor{currentstroke}{rgb}{0.000000,0.000000,0.000000}%
\pgfsetstrokecolor{currentstroke}%
\pgfsetdash{}{0pt}%
\pgfpathmoveto{\pgfqpoint{1.608768in}{0.860599in}}%
\pgfpathlineto{\pgfqpoint{1.610354in}{0.869179in}}%
\pgfpathlineto{\pgfqpoint{1.611949in}{0.878175in}}%
\pgfpathlineto{\pgfqpoint{1.613553in}{0.887594in}}%
\pgfpathlineto{\pgfqpoint{1.615165in}{0.897445in}}%
\pgfpathlineto{\pgfqpoint{1.595085in}{0.893170in}}%
\pgfpathlineto{\pgfqpoint{1.574724in}{0.889234in}}%
\pgfpathlineto{\pgfqpoint{1.554105in}{0.885640in}}%
\pgfpathlineto{\pgfqpoint{1.533252in}{0.882395in}}%
\pgfpathlineto{\pgfqpoint{1.532137in}{0.872614in}}%
\pgfpathlineto{\pgfqpoint{1.531029in}{0.863265in}}%
\pgfpathlineto{\pgfqpoint{1.529926in}{0.854340in}}%
\pgfpathlineto{\pgfqpoint{1.528830in}{0.845833in}}%
\pgfpathlineto{\pgfqpoint{1.549179in}{0.849017in}}%
\pgfpathlineto{\pgfqpoint{1.569300in}{0.852542in}}%
\pgfpathlineto{\pgfqpoint{1.589170in}{0.856405in}}%
\pgfpathlineto{\pgfqpoint{1.608768in}{0.860599in}}%
\pgfpathclose%
\pgfusepath{fill}%
\end{pgfscope}%
\begin{pgfscope}%
\pgfpathrectangle{\pgfqpoint{0.329460in}{0.284240in}}{\pgfqpoint{1.989680in}{1.989680in}}%
\pgfusepath{clip}%
\pgfsetbuttcap%
\pgfsetroundjoin%
\definecolor{currentfill}{rgb}{0.233603,0.313828,0.543914}%
\pgfsetfillcolor{currentfill}%
\pgfsetlinewidth{0.000000pt}%
\definecolor{currentstroke}{rgb}{0.000000,0.000000,0.000000}%
\pgfsetstrokecolor{currentstroke}%
\pgfsetdash{}{0pt}%
\pgfpathmoveto{\pgfqpoint{1.191806in}{0.843292in}}%
\pgfpathlineto{\pgfqpoint{1.190822in}{0.851787in}}%
\pgfpathlineto{\pgfqpoint{1.189833in}{0.860700in}}%
\pgfpathlineto{\pgfqpoint{1.188838in}{0.870037in}}%
\pgfpathlineto{\pgfqpoint{1.187837in}{0.879805in}}%
\pgfpathlineto{\pgfqpoint{1.166796in}{0.882738in}}%
\pgfpathlineto{\pgfqpoint{1.145967in}{0.886023in}}%
\pgfpathlineto{\pgfqpoint{1.125376in}{0.889654in}}%
\pgfpathlineto{\pgfqpoint{1.105045in}{0.893629in}}%
\pgfpathlineto{\pgfqpoint{1.106549in}{0.883796in}}%
\pgfpathlineto{\pgfqpoint{1.108045in}{0.874394in}}%
\pgfpathlineto{\pgfqpoint{1.109533in}{0.865416in}}%
\pgfpathlineto{\pgfqpoint{1.111013in}{0.856854in}}%
\pgfpathlineto{\pgfqpoint{1.130855in}{0.852955in}}%
\pgfpathlineto{\pgfqpoint{1.150949in}{0.849392in}}%
\pgfpathlineto{\pgfqpoint{1.171274in}{0.846169in}}%
\pgfpathlineto{\pgfqpoint{1.191806in}{0.843292in}}%
\pgfpathclose%
\pgfusepath{fill}%
\end{pgfscope}%
\begin{pgfscope}%
\pgfpathrectangle{\pgfqpoint{0.329460in}{0.284240in}}{\pgfqpoint{1.989680in}{1.989680in}}%
\pgfusepath{clip}%
\pgfsetbuttcap%
\pgfsetroundjoin%
\definecolor{currentfill}{rgb}{0.172719,0.448791,0.557885}%
\pgfsetfillcolor{currentfill}%
\pgfsetlinewidth{0.000000pt}%
\definecolor{currentstroke}{rgb}{0.000000,0.000000,0.000000}%
\pgfsetstrokecolor{currentstroke}%
\pgfsetdash{}{0pt}%
\pgfpathmoveto{\pgfqpoint{1.773237in}{0.987726in}}%
\pgfpathlineto{\pgfqpoint{1.775808in}{1.000041in}}%
\pgfpathlineto{\pgfqpoint{1.778395in}{1.012830in}}%
\pgfpathlineto{\pgfqpoint{1.780998in}{1.026103in}}%
\pgfpathlineto{\pgfqpoint{1.763209in}{1.019154in}}%
\pgfpathlineto{\pgfqpoint{1.744951in}{1.012495in}}%
\pgfpathlineto{\pgfqpoint{1.726246in}{1.006136in}}%
\pgfpathlineto{\pgfqpoint{1.707112in}{1.000084in}}%
\pgfpathlineto{\pgfqpoint{1.704948in}{0.986917in}}%
\pgfpathlineto{\pgfqpoint{1.702798in}{0.974235in}}%
\pgfpathlineto{\pgfqpoint{1.700660in}{0.962028in}}%
\pgfpathlineto{\pgfqpoint{1.719453in}{0.968005in}}%
\pgfpathlineto{\pgfqpoint{1.737827in}{0.974286in}}%
\pgfpathlineto{\pgfqpoint{1.755761in}{0.980862in}}%
\pgfpathlineto{\pgfqpoint{1.773237in}{0.987726in}}%
\pgfpathclose%
\pgfusepath{fill}%
\end{pgfscope}%
\begin{pgfscope}%
\pgfpathrectangle{\pgfqpoint{0.329460in}{0.284240in}}{\pgfqpoint{1.989680in}{1.989680in}}%
\pgfusepath{clip}%
\pgfsetbuttcap%
\pgfsetroundjoin%
\definecolor{currentfill}{rgb}{0.201239,0.383670,0.554294}%
\pgfsetfillcolor{currentfill}%
\pgfsetlinewidth{0.000000pt}%
\definecolor{currentstroke}{rgb}{0.000000,0.000000,0.000000}%
\pgfsetstrokecolor{currentstroke}%
\pgfsetdash{}{0pt}%
\pgfpathmoveto{\pgfqpoint{1.692234in}{0.917804in}}%
\pgfpathlineto{\pgfqpoint{1.694323in}{0.928186in}}%
\pgfpathlineto{\pgfqpoint{1.696423in}{0.939012in}}%
\pgfpathlineto{\pgfqpoint{1.698535in}{0.950290in}}%
\pgfpathlineto{\pgfqpoint{1.700660in}{0.962028in}}%
\pgfpathlineto{\pgfqpoint{1.681468in}{0.956362in}}%
\pgfpathlineto{\pgfqpoint{1.661897in}{0.951015in}}%
\pgfpathlineto{\pgfqpoint{1.641969in}{0.945994in}}%
\pgfpathlineto{\pgfqpoint{1.621707in}{0.941304in}}%
\pgfpathlineto{\pgfqpoint{1.620058in}{0.929655in}}%
\pgfpathlineto{\pgfqpoint{1.618417in}{0.918467in}}%
\pgfpathlineto{\pgfqpoint{1.616787in}{0.907733in}}%
\pgfpathlineto{\pgfqpoint{1.615165in}{0.897445in}}%
\pgfpathlineto{\pgfqpoint{1.634942in}{0.902052in}}%
\pgfpathlineto{\pgfqpoint{1.654394in}{0.906985in}}%
\pgfpathlineto{\pgfqpoint{1.673498in}{0.912238in}}%
\pgfpathlineto{\pgfqpoint{1.692234in}{0.917804in}}%
\pgfpathclose%
\pgfusepath{fill}%
\end{pgfscope}%
\begin{pgfscope}%
\pgfpathrectangle{\pgfqpoint{0.329460in}{0.284240in}}{\pgfqpoint{1.989680in}{1.989680in}}%
\pgfusepath{clip}%
\pgfsetbuttcap%
\pgfsetroundjoin%
\definecolor{currentfill}{rgb}{0.201239,0.383670,0.554294}%
\pgfsetfillcolor{currentfill}%
\pgfsetlinewidth{0.000000pt}%
\definecolor{currentstroke}{rgb}{0.000000,0.000000,0.000000}%
\pgfsetstrokecolor{currentstroke}%
\pgfsetdash{}{0pt}%
\pgfpathmoveto{\pgfqpoint{1.105045in}{0.893629in}}%
\pgfpathlineto{\pgfqpoint{1.103532in}{0.903900in}}%
\pgfpathlineto{\pgfqpoint{1.102011in}{0.914617in}}%
\pgfpathlineto{\pgfqpoint{1.100480in}{0.925787in}}%
\pgfpathlineto{\pgfqpoint{1.098941in}{0.937419in}}%
\pgfpathlineto{\pgfqpoint{1.078401in}{0.941809in}}%
\pgfpathlineto{\pgfqpoint{1.058175in}{0.946535in}}%
\pgfpathlineto{\pgfqpoint{1.038286in}{0.951593in}}%
\pgfpathlineto{\pgfqpoint{1.018756in}{0.956976in}}%
\pgfpathlineto{\pgfqpoint{1.020779in}{0.945260in}}%
\pgfpathlineto{\pgfqpoint{1.022790in}{0.934003in}}%
\pgfpathlineto{\pgfqpoint{1.024789in}{0.923200in}}%
\pgfpathlineto{\pgfqpoint{1.026777in}{0.912841in}}%
\pgfpathlineto{\pgfqpoint{1.045841in}{0.907553in}}%
\pgfpathlineto{\pgfqpoint{1.065255in}{0.902584in}}%
\pgfpathlineto{\pgfqpoint{1.084997in}{0.897940in}}%
\pgfpathlineto{\pgfqpoint{1.105045in}{0.893629in}}%
\pgfpathclose%
\pgfusepath{fill}%
\end{pgfscope}%
\begin{pgfscope}%
\pgfpathrectangle{\pgfqpoint{0.329460in}{0.284240in}}{\pgfqpoint{1.989680in}{1.989680in}}%
\pgfusepath{clip}%
\pgfsetbuttcap%
\pgfsetroundjoin%
\definecolor{currentfill}{rgb}{0.172719,0.448791,0.557885}%
\pgfsetfillcolor{currentfill}%
\pgfsetlinewidth{0.000000pt}%
\definecolor{currentstroke}{rgb}{0.000000,0.000000,0.000000}%
\pgfsetstrokecolor{currentstroke}%
\pgfsetdash{}{0pt}%
\pgfpathmoveto{\pgfqpoint{1.018756in}{0.956976in}}%
\pgfpathlineto{\pgfqpoint{1.016721in}{0.969161in}}%
\pgfpathlineto{\pgfqpoint{1.014674in}{0.981823in}}%
\pgfpathlineto{\pgfqpoint{1.012614in}{0.994969in}}%
\pgfpathlineto{\pgfqpoint{0.993117in}{1.000741in}}%
\pgfpathlineto{\pgfqpoint{0.974029in}{1.006828in}}%
\pgfpathlineto{\pgfqpoint{0.955373in}{1.013221in}}%
\pgfpathlineto{\pgfqpoint{0.937167in}{1.019912in}}%
\pgfpathlineto{\pgfqpoint{0.939676in}{1.006664in}}%
\pgfpathlineto{\pgfqpoint{0.942170in}{0.993900in}}%
\pgfpathlineto{\pgfqpoint{0.944649in}{0.981611in}}%
\pgfpathlineto{\pgfqpoint{0.962534in}{0.975002in}}%
\pgfpathlineto{\pgfqpoint{0.980859in}{0.968688in}}%
\pgfpathlineto{\pgfqpoint{0.999607in}{0.962677in}}%
\pgfpathlineto{\pgfqpoint{1.018756in}{0.956976in}}%
\pgfpathclose%
\pgfusepath{fill}%
\end{pgfscope}%
\begin{pgfscope}%
\pgfpathrectangle{\pgfqpoint{0.329460in}{0.284240in}}{\pgfqpoint{1.989680in}{1.989680in}}%
\pgfusepath{clip}%
\pgfsetbuttcap%
\pgfsetroundjoin%
\definecolor{currentfill}{rgb}{0.233603,0.313828,0.543914}%
\pgfsetfillcolor{currentfill}%
\pgfsetlinewidth{0.000000pt}%
\definecolor{currentstroke}{rgb}{0.000000,0.000000,0.000000}%
\pgfsetstrokecolor{currentstroke}%
\pgfsetdash{}{0pt}%
\pgfpathmoveto{\pgfqpoint{1.528830in}{0.845833in}}%
\pgfpathlineto{\pgfqpoint{1.529926in}{0.854340in}}%
\pgfpathlineto{\pgfqpoint{1.531029in}{0.863265in}}%
\pgfpathlineto{\pgfqpoint{1.532137in}{0.872614in}}%
\pgfpathlineto{\pgfqpoint{1.533252in}{0.882395in}}%
\pgfpathlineto{\pgfqpoint{1.512188in}{0.879501in}}%
\pgfpathlineto{\pgfqpoint{1.490937in}{0.876963in}}%
\pgfpathlineto{\pgfqpoint{1.469525in}{0.874785in}}%
\pgfpathlineto{\pgfqpoint{1.447975in}{0.872968in}}%
\pgfpathlineto{\pgfqpoint{1.447382in}{0.863232in}}%
\pgfpathlineto{\pgfqpoint{1.446792in}{0.853927in}}%
\pgfpathlineto{\pgfqpoint{1.446205in}{0.845047in}}%
\pgfpathlineto{\pgfqpoint{1.445621in}{0.836585in}}%
\pgfpathlineto{\pgfqpoint{1.466647in}{0.838367in}}%
\pgfpathlineto{\pgfqpoint{1.487540in}{0.840504in}}%
\pgfpathlineto{\pgfqpoint{1.508276in}{0.842994in}}%
\pgfpathlineto{\pgfqpoint{1.528830in}{0.845833in}}%
\pgfpathclose%
\pgfusepath{fill}%
\end{pgfscope}%
\begin{pgfscope}%
\pgfpathrectangle{\pgfqpoint{0.329460in}{0.284240in}}{\pgfqpoint{1.989680in}{1.989680in}}%
\pgfusepath{clip}%
\pgfsetbuttcap%
\pgfsetroundjoin%
\definecolor{currentfill}{rgb}{0.233603,0.313828,0.543914}%
\pgfsetfillcolor{currentfill}%
\pgfsetlinewidth{0.000000pt}%
\definecolor{currentstroke}{rgb}{0.000000,0.000000,0.000000}%
\pgfsetstrokecolor{currentstroke}%
\pgfsetdash{}{0pt}%
\pgfpathmoveto{\pgfqpoint{1.275537in}{0.835301in}}%
\pgfpathlineto{\pgfqpoint{1.275069in}{0.843757in}}%
\pgfpathlineto{\pgfqpoint{1.274599in}{0.852631in}}%
\pgfpathlineto{\pgfqpoint{1.274126in}{0.861929in}}%
\pgfpathlineto{\pgfqpoint{1.273651in}{0.871660in}}%
\pgfpathlineto{\pgfqpoint{1.252000in}{0.873152in}}%
\pgfpathlineto{\pgfqpoint{1.230464in}{0.875009in}}%
\pgfpathlineto{\pgfqpoint{1.209068in}{0.877228in}}%
\pgfpathlineto{\pgfqpoint{1.187837in}{0.879805in}}%
\pgfpathlineto{\pgfqpoint{1.188838in}{0.870037in}}%
\pgfpathlineto{\pgfqpoint{1.189833in}{0.860700in}}%
\pgfpathlineto{\pgfqpoint{1.190822in}{0.851787in}}%
\pgfpathlineto{\pgfqpoint{1.191806in}{0.843292in}}%
\pgfpathlineto{\pgfqpoint{1.212523in}{0.840763in}}%
\pgfpathlineto{\pgfqpoint{1.233399in}{0.838587in}}%
\pgfpathlineto{\pgfqpoint{1.254412in}{0.836765in}}%
\pgfpathlineto{\pgfqpoint{1.275537in}{0.835301in}}%
\pgfpathclose%
\pgfusepath{fill}%
\end{pgfscope}%
\begin{pgfscope}%
\pgfpathrectangle{\pgfqpoint{0.329460in}{0.284240in}}{\pgfqpoint{1.989680in}{1.989680in}}%
\pgfusepath{clip}%
\pgfsetbuttcap%
\pgfsetroundjoin%
\definecolor{currentfill}{rgb}{0.233603,0.313828,0.543914}%
\pgfsetfillcolor{currentfill}%
\pgfsetlinewidth{0.000000pt}%
\definecolor{currentstroke}{rgb}{0.000000,0.000000,0.000000}%
\pgfsetstrokecolor{currentstroke}%
\pgfsetdash{}{0pt}%
\pgfpathmoveto{\pgfqpoint{1.445621in}{0.836585in}}%
\pgfpathlineto{\pgfqpoint{1.446205in}{0.845047in}}%
\pgfpathlineto{\pgfqpoint{1.446792in}{0.853927in}}%
\pgfpathlineto{\pgfqpoint{1.447382in}{0.863232in}}%
\pgfpathlineto{\pgfqpoint{1.447975in}{0.872968in}}%
\pgfpathlineto{\pgfqpoint{1.426313in}{0.871517in}}%
\pgfpathlineto{\pgfqpoint{1.404563in}{0.870431in}}%
\pgfpathlineto{\pgfqpoint{1.382751in}{0.869714in}}%
\pgfpathlineto{\pgfqpoint{1.360903in}{0.869365in}}%
\pgfpathlineto{\pgfqpoint{1.360844in}{0.859645in}}%
\pgfpathlineto{\pgfqpoint{1.360784in}{0.850358in}}%
\pgfpathlineto{\pgfqpoint{1.360725in}{0.841495in}}%
\pgfpathlineto{\pgfqpoint{1.360667in}{0.833050in}}%
\pgfpathlineto{\pgfqpoint{1.381983in}{0.833392in}}%
\pgfpathlineto{\pgfqpoint{1.403264in}{0.834096in}}%
\pgfpathlineto{\pgfqpoint{1.424485in}{0.835160in}}%
\pgfpathlineto{\pgfqpoint{1.445621in}{0.836585in}}%
\pgfpathclose%
\pgfusepath{fill}%
\end{pgfscope}%
\begin{pgfscope}%
\pgfpathrectangle{\pgfqpoint{0.329460in}{0.284240in}}{\pgfqpoint{1.989680in}{1.989680in}}%
\pgfusepath{clip}%
\pgfsetbuttcap%
\pgfsetroundjoin%
\definecolor{currentfill}{rgb}{0.233603,0.313828,0.543914}%
\pgfsetfillcolor{currentfill}%
\pgfsetlinewidth{0.000000pt}%
\definecolor{currentstroke}{rgb}{0.000000,0.000000,0.000000}%
\pgfsetstrokecolor{currentstroke}%
\pgfsetdash{}{0pt}%
\pgfpathmoveto{\pgfqpoint{1.360667in}{0.833050in}}%
\pgfpathlineto{\pgfqpoint{1.360725in}{0.841495in}}%
\pgfpathlineto{\pgfqpoint{1.360784in}{0.850358in}}%
\pgfpathlineto{\pgfqpoint{1.360844in}{0.859645in}}%
\pgfpathlineto{\pgfqpoint{1.360903in}{0.869365in}}%
\pgfpathlineto{\pgfqpoint{1.339043in}{0.869385in}}%
\pgfpathlineto{\pgfqpoint{1.317198in}{0.869775in}}%
\pgfpathlineto{\pgfqpoint{1.295392in}{0.870534in}}%
\pgfpathlineto{\pgfqpoint{1.273651in}{0.871660in}}%
\pgfpathlineto{\pgfqpoint{1.274126in}{0.861929in}}%
\pgfpathlineto{\pgfqpoint{1.274599in}{0.852631in}}%
\pgfpathlineto{\pgfqpoint{1.275069in}{0.843757in}}%
\pgfpathlineto{\pgfqpoint{1.275537in}{0.835301in}}%
\pgfpathlineto{\pgfqpoint{1.296749in}{0.834196in}}%
\pgfpathlineto{\pgfqpoint{1.318025in}{0.833452in}}%
\pgfpathlineto{\pgfqpoint{1.339339in}{0.833070in}}%
\pgfpathlineto{\pgfqpoint{1.360667in}{0.833050in}}%
\pgfpathclose%
\pgfusepath{fill}%
\end{pgfscope}%
\begin{pgfscope}%
\pgfpathrectangle{\pgfqpoint{0.329460in}{0.284240in}}{\pgfqpoint{1.989680in}{1.989680in}}%
\pgfusepath{clip}%
\pgfsetbuttcap%
\pgfsetroundjoin%
\definecolor{currentfill}{rgb}{0.201239,0.383670,0.554294}%
\pgfsetfillcolor{currentfill}%
\pgfsetlinewidth{0.000000pt}%
\definecolor{currentstroke}{rgb}{0.000000,0.000000,0.000000}%
\pgfsetstrokecolor{currentstroke}%
\pgfsetdash{}{0pt}%
\pgfpathmoveto{\pgfqpoint{1.615165in}{0.897445in}}%
\pgfpathlineto{\pgfqpoint{1.616787in}{0.907733in}}%
\pgfpathlineto{\pgfqpoint{1.618417in}{0.918467in}}%
\pgfpathlineto{\pgfqpoint{1.620058in}{0.929655in}}%
\pgfpathlineto{\pgfqpoint{1.621707in}{0.941304in}}%
\pgfpathlineto{\pgfqpoint{1.601134in}{0.936953in}}%
\pgfpathlineto{\pgfqpoint{1.580271in}{0.932945in}}%
\pgfpathlineto{\pgfqpoint{1.559143in}{0.929287in}}%
\pgfpathlineto{\pgfqpoint{1.537774in}{0.925982in}}%
\pgfpathlineto{\pgfqpoint{1.536634in}{0.914400in}}%
\pgfpathlineto{\pgfqpoint{1.535500in}{0.903280in}}%
\pgfpathlineto{\pgfqpoint{1.534372in}{0.892614in}}%
\pgfpathlineto{\pgfqpoint{1.533252in}{0.882395in}}%
\pgfpathlineto{\pgfqpoint{1.554105in}{0.885640in}}%
\pgfpathlineto{\pgfqpoint{1.574724in}{0.889234in}}%
\pgfpathlineto{\pgfqpoint{1.595085in}{0.893170in}}%
\pgfpathlineto{\pgfqpoint{1.615165in}{0.897445in}}%
\pgfpathclose%
\pgfusepath{fill}%
\end{pgfscope}%
\begin{pgfscope}%
\pgfpathrectangle{\pgfqpoint{0.329460in}{0.284240in}}{\pgfqpoint{1.989680in}{1.989680in}}%
\pgfusepath{clip}%
\pgfsetbuttcap%
\pgfsetroundjoin%
\definecolor{currentfill}{rgb}{0.201239,0.383670,0.554294}%
\pgfsetfillcolor{currentfill}%
\pgfsetlinewidth{0.000000pt}%
\definecolor{currentstroke}{rgb}{0.000000,0.000000,0.000000}%
\pgfsetstrokecolor{currentstroke}%
\pgfsetdash{}{0pt}%
\pgfpathmoveto{\pgfqpoint{1.187837in}{0.879805in}}%
\pgfpathlineto{\pgfqpoint{1.186831in}{0.890013in}}%
\pgfpathlineto{\pgfqpoint{1.185819in}{0.900667in}}%
\pgfpathlineto{\pgfqpoint{1.184802in}{0.911775in}}%
\pgfpathlineto{\pgfqpoint{1.183778in}{0.923346in}}%
\pgfpathlineto{\pgfqpoint{1.162215in}{0.926332in}}%
\pgfpathlineto{\pgfqpoint{1.140872in}{0.929676in}}%
\pgfpathlineto{\pgfqpoint{1.119773in}{0.933373in}}%
\pgfpathlineto{\pgfqpoint{1.098941in}{0.937419in}}%
\pgfpathlineto{\pgfqpoint{1.100480in}{0.925787in}}%
\pgfpathlineto{\pgfqpoint{1.102011in}{0.914617in}}%
\pgfpathlineto{\pgfqpoint{1.103532in}{0.903900in}}%
\pgfpathlineto{\pgfqpoint{1.105045in}{0.893629in}}%
\pgfpathlineto{\pgfqpoint{1.125376in}{0.889654in}}%
\pgfpathlineto{\pgfqpoint{1.145967in}{0.886023in}}%
\pgfpathlineto{\pgfqpoint{1.166796in}{0.882738in}}%
\pgfpathlineto{\pgfqpoint{1.187837in}{0.879805in}}%
\pgfpathclose%
\pgfusepath{fill}%
\end{pgfscope}%
\begin{pgfscope}%
\pgfpathrectangle{\pgfqpoint{0.329460in}{0.284240in}}{\pgfqpoint{1.989680in}{1.989680in}}%
\pgfusepath{clip}%
\pgfsetbuttcap%
\pgfsetroundjoin%
\definecolor{currentfill}{rgb}{0.172719,0.448791,0.557885}%
\pgfsetfillcolor{currentfill}%
\pgfsetlinewidth{0.000000pt}%
\definecolor{currentstroke}{rgb}{0.000000,0.000000,0.000000}%
\pgfsetstrokecolor{currentstroke}%
\pgfsetdash{}{0pt}%
\pgfpathmoveto{\pgfqpoint{1.700660in}{0.962028in}}%
\pgfpathlineto{\pgfqpoint{1.702798in}{0.974235in}}%
\pgfpathlineto{\pgfqpoint{1.704948in}{0.986917in}}%
\pgfpathlineto{\pgfqpoint{1.707112in}{1.000084in}}%
\pgfpathlineto{\pgfqpoint{1.687571in}{0.994347in}}%
\pgfpathlineto{\pgfqpoint{1.667643in}{0.988933in}}%
\pgfpathlineto{\pgfqpoint{1.647351in}{0.983848in}}%
\pgfpathlineto{\pgfqpoint{1.626718in}{0.979099in}}%
\pgfpathlineto{\pgfqpoint{1.625037in}{0.966018in}}%
\pgfpathlineto{\pgfqpoint{1.623367in}{0.953422in}}%
\pgfpathlineto{\pgfqpoint{1.621707in}{0.941304in}}%
\pgfpathlineto{\pgfqpoint{1.641969in}{0.945994in}}%
\pgfpathlineto{\pgfqpoint{1.661897in}{0.951015in}}%
\pgfpathlineto{\pgfqpoint{1.681468in}{0.956362in}}%
\pgfpathlineto{\pgfqpoint{1.700660in}{0.962028in}}%
\pgfpathclose%
\pgfusepath{fill}%
\end{pgfscope}%
\begin{pgfscope}%
\pgfpathrectangle{\pgfqpoint{0.329460in}{0.284240in}}{\pgfqpoint{1.989680in}{1.989680in}}%
\pgfusepath{clip}%
\pgfsetbuttcap%
\pgfsetroundjoin%
\definecolor{currentfill}{rgb}{0.172719,0.448791,0.557885}%
\pgfsetfillcolor{currentfill}%
\pgfsetlinewidth{0.000000pt}%
\definecolor{currentstroke}{rgb}{0.000000,0.000000,0.000000}%
\pgfsetstrokecolor{currentstroke}%
\pgfsetdash{}{0pt}%
\pgfpathmoveto{\pgfqpoint{1.098941in}{0.937419in}}%
\pgfpathlineto{\pgfqpoint{1.097392in}{0.949521in}}%
\pgfpathlineto{\pgfqpoint{1.095834in}{0.962100in}}%
\pgfpathlineto{\pgfqpoint{1.094266in}{0.975165in}}%
\pgfpathlineto{\pgfqpoint{1.073349in}{0.979610in}}%
\pgfpathlineto{\pgfqpoint{1.052752in}{0.984396in}}%
\pgfpathlineto{\pgfqpoint{1.032500in}{0.989518in}}%
\pgfpathlineto{\pgfqpoint{1.012614in}{0.994969in}}%
\pgfpathlineto{\pgfqpoint{1.014674in}{0.981823in}}%
\pgfpathlineto{\pgfqpoint{1.016721in}{0.969161in}}%
\pgfpathlineto{\pgfqpoint{1.018756in}{0.956976in}}%
\pgfpathlineto{\pgfqpoint{1.038286in}{0.951593in}}%
\pgfpathlineto{\pgfqpoint{1.058175in}{0.946535in}}%
\pgfpathlineto{\pgfqpoint{1.078401in}{0.941809in}}%
\pgfpathlineto{\pgfqpoint{1.098941in}{0.937419in}}%
\pgfpathclose%
\pgfusepath{fill}%
\end{pgfscope}%
\begin{pgfscope}%
\pgfpathrectangle{\pgfqpoint{0.329460in}{0.284240in}}{\pgfqpoint{1.989680in}{1.989680in}}%
\pgfusepath{clip}%
\pgfsetbuttcap%
\pgfsetroundjoin%
\definecolor{currentfill}{rgb}{0.201239,0.383670,0.554294}%
\pgfsetfillcolor{currentfill}%
\pgfsetlinewidth{0.000000pt}%
\definecolor{currentstroke}{rgb}{0.000000,0.000000,0.000000}%
\pgfsetstrokecolor{currentstroke}%
\pgfsetdash{}{0pt}%
\pgfpathmoveto{\pgfqpoint{1.533252in}{0.882395in}}%
\pgfpathlineto{\pgfqpoint{1.534372in}{0.892614in}}%
\pgfpathlineto{\pgfqpoint{1.535500in}{0.903280in}}%
\pgfpathlineto{\pgfqpoint{1.536634in}{0.914400in}}%
\pgfpathlineto{\pgfqpoint{1.537774in}{0.925982in}}%
\pgfpathlineto{\pgfqpoint{1.516189in}{0.923036in}}%
\pgfpathlineto{\pgfqpoint{1.494412in}{0.920453in}}%
\pgfpathlineto{\pgfqpoint{1.472468in}{0.918235in}}%
\pgfpathlineto{\pgfqpoint{1.450383in}{0.916385in}}%
\pgfpathlineto{\pgfqpoint{1.449776in}{0.904845in}}%
\pgfpathlineto{\pgfqpoint{1.449172in}{0.893767in}}%
\pgfpathlineto{\pgfqpoint{1.448572in}{0.883144in}}%
\pgfpathlineto{\pgfqpoint{1.447975in}{0.872968in}}%
\pgfpathlineto{\pgfqpoint{1.469525in}{0.874785in}}%
\pgfpathlineto{\pgfqpoint{1.490937in}{0.876963in}}%
\pgfpathlineto{\pgfqpoint{1.512188in}{0.879501in}}%
\pgfpathlineto{\pgfqpoint{1.533252in}{0.882395in}}%
\pgfpathclose%
\pgfusepath{fill}%
\end{pgfscope}%
\begin{pgfscope}%
\pgfpathrectangle{\pgfqpoint{0.329460in}{0.284240in}}{\pgfqpoint{1.989680in}{1.989680in}}%
\pgfusepath{clip}%
\pgfsetbuttcap%
\pgfsetroundjoin%
\definecolor{currentfill}{rgb}{0.201239,0.383670,0.554294}%
\pgfsetfillcolor{currentfill}%
\pgfsetlinewidth{0.000000pt}%
\definecolor{currentstroke}{rgb}{0.000000,0.000000,0.000000}%
\pgfsetstrokecolor{currentstroke}%
\pgfsetdash{}{0pt}%
\pgfpathmoveto{\pgfqpoint{1.273651in}{0.871660in}}%
\pgfpathlineto{\pgfqpoint{1.273173in}{0.881830in}}%
\pgfpathlineto{\pgfqpoint{1.272692in}{0.892447in}}%
\pgfpathlineto{\pgfqpoint{1.272208in}{0.903518in}}%
\pgfpathlineto{\pgfqpoint{1.271721in}{0.915053in}}%
\pgfpathlineto{\pgfqpoint{1.249532in}{0.916572in}}%
\pgfpathlineto{\pgfqpoint{1.227462in}{0.918463in}}%
\pgfpathlineto{\pgfqpoint{1.205535in}{0.920722in}}%
\pgfpathlineto{\pgfqpoint{1.183778in}{0.923346in}}%
\pgfpathlineto{\pgfqpoint{1.184802in}{0.911775in}}%
\pgfpathlineto{\pgfqpoint{1.185819in}{0.900667in}}%
\pgfpathlineto{\pgfqpoint{1.186831in}{0.890013in}}%
\pgfpathlineto{\pgfqpoint{1.187837in}{0.879805in}}%
\pgfpathlineto{\pgfqpoint{1.209068in}{0.877228in}}%
\pgfpathlineto{\pgfqpoint{1.230464in}{0.875009in}}%
\pgfpathlineto{\pgfqpoint{1.252000in}{0.873152in}}%
\pgfpathlineto{\pgfqpoint{1.273651in}{0.871660in}}%
\pgfpathclose%
\pgfusepath{fill}%
\end{pgfscope}%
\begin{pgfscope}%
\pgfpathrectangle{\pgfqpoint{0.329460in}{0.284240in}}{\pgfqpoint{1.989680in}{1.989680in}}%
\pgfusepath{clip}%
\pgfsetbuttcap%
\pgfsetroundjoin%
\definecolor{currentfill}{rgb}{0.201239,0.383670,0.554294}%
\pgfsetfillcolor{currentfill}%
\pgfsetlinewidth{0.000000pt}%
\definecolor{currentstroke}{rgb}{0.000000,0.000000,0.000000}%
\pgfsetstrokecolor{currentstroke}%
\pgfsetdash{}{0pt}%
\pgfpathmoveto{\pgfqpoint{1.447975in}{0.872968in}}%
\pgfpathlineto{\pgfqpoint{1.448572in}{0.883144in}}%
\pgfpathlineto{\pgfqpoint{1.449172in}{0.893767in}}%
\pgfpathlineto{\pgfqpoint{1.449776in}{0.904845in}}%
\pgfpathlineto{\pgfqpoint{1.450383in}{0.916385in}}%
\pgfpathlineto{\pgfqpoint{1.428182in}{0.914907in}}%
\pgfpathlineto{\pgfqpoint{1.405891in}{0.913802in}}%
\pgfpathlineto{\pgfqpoint{1.383537in}{0.913071in}}%
\pgfpathlineto{\pgfqpoint{1.361145in}{0.912716in}}%
\pgfpathlineto{\pgfqpoint{1.361084in}{0.901192in}}%
\pgfpathlineto{\pgfqpoint{1.361023in}{0.890130in}}%
\pgfpathlineto{\pgfqpoint{1.360963in}{0.879524in}}%
\pgfpathlineto{\pgfqpoint{1.360903in}{0.869365in}}%
\pgfpathlineto{\pgfqpoint{1.382751in}{0.869714in}}%
\pgfpathlineto{\pgfqpoint{1.404563in}{0.870431in}}%
\pgfpathlineto{\pgfqpoint{1.426313in}{0.871517in}}%
\pgfpathlineto{\pgfqpoint{1.447975in}{0.872968in}}%
\pgfpathclose%
\pgfusepath{fill}%
\end{pgfscope}%
\begin{pgfscope}%
\pgfpathrectangle{\pgfqpoint{0.329460in}{0.284240in}}{\pgfqpoint{1.989680in}{1.989680in}}%
\pgfusepath{clip}%
\pgfsetbuttcap%
\pgfsetroundjoin%
\definecolor{currentfill}{rgb}{0.201239,0.383670,0.554294}%
\pgfsetfillcolor{currentfill}%
\pgfsetlinewidth{0.000000pt}%
\definecolor{currentstroke}{rgb}{0.000000,0.000000,0.000000}%
\pgfsetstrokecolor{currentstroke}%
\pgfsetdash{}{0pt}%
\pgfpathmoveto{\pgfqpoint{1.360903in}{0.869365in}}%
\pgfpathlineto{\pgfqpoint{1.360963in}{0.879524in}}%
\pgfpathlineto{\pgfqpoint{1.361023in}{0.890130in}}%
\pgfpathlineto{\pgfqpoint{1.361084in}{0.901192in}}%
\pgfpathlineto{\pgfqpoint{1.361145in}{0.912716in}}%
\pgfpathlineto{\pgfqpoint{1.338741in}{0.912737in}}%
\pgfpathlineto{\pgfqpoint{1.316352in}{0.913134in}}%
\pgfpathlineto{\pgfqpoint{1.294003in}{0.913906in}}%
\pgfpathlineto{\pgfqpoint{1.271721in}{0.915053in}}%
\pgfpathlineto{\pgfqpoint{1.272208in}{0.903518in}}%
\pgfpathlineto{\pgfqpoint{1.272692in}{0.892447in}}%
\pgfpathlineto{\pgfqpoint{1.273173in}{0.881830in}}%
\pgfpathlineto{\pgfqpoint{1.273651in}{0.871660in}}%
\pgfpathlineto{\pgfqpoint{1.295392in}{0.870534in}}%
\pgfpathlineto{\pgfqpoint{1.317198in}{0.869775in}}%
\pgfpathlineto{\pgfqpoint{1.339043in}{0.869385in}}%
\pgfpathlineto{\pgfqpoint{1.360903in}{0.869365in}}%
\pgfpathclose%
\pgfusepath{fill}%
\end{pgfscope}%
\begin{pgfscope}%
\pgfpathrectangle{\pgfqpoint{0.329460in}{0.284240in}}{\pgfqpoint{1.989680in}{1.989680in}}%
\pgfusepath{clip}%
\pgfsetbuttcap%
\pgfsetroundjoin%
\definecolor{currentfill}{rgb}{0.172719,0.448791,0.557885}%
\pgfsetfillcolor{currentfill}%
\pgfsetlinewidth{0.000000pt}%
\definecolor{currentstroke}{rgb}{0.000000,0.000000,0.000000}%
\pgfsetstrokecolor{currentstroke}%
\pgfsetdash{}{0pt}%
\pgfpathmoveto{\pgfqpoint{1.621707in}{0.941304in}}%
\pgfpathlineto{\pgfqpoint{1.623367in}{0.953422in}}%
\pgfpathlineto{\pgfqpoint{1.625037in}{0.966018in}}%
\pgfpathlineto{\pgfqpoint{1.626718in}{0.979099in}}%
\pgfpathlineto{\pgfqpoint{1.605766in}{0.974692in}}%
\pgfpathlineto{\pgfqpoint{1.584519in}{0.970634in}}%
\pgfpathlineto{\pgfqpoint{1.563002in}{0.966929in}}%
\pgfpathlineto{\pgfqpoint{1.541238in}{0.963583in}}%
\pgfpathlineto{\pgfqpoint{1.540076in}{0.950565in}}%
\pgfpathlineto{\pgfqpoint{1.538922in}{0.938035in}}%
\pgfpathlineto{\pgfqpoint{1.537774in}{0.925982in}}%
\pgfpathlineto{\pgfqpoint{1.559143in}{0.929287in}}%
\pgfpathlineto{\pgfqpoint{1.580271in}{0.932945in}}%
\pgfpathlineto{\pgfqpoint{1.601134in}{0.936953in}}%
\pgfpathlineto{\pgfqpoint{1.621707in}{0.941304in}}%
\pgfpathclose%
\pgfusepath{fill}%
\end{pgfscope}%
\begin{pgfscope}%
\pgfpathrectangle{\pgfqpoint{0.329460in}{0.284240in}}{\pgfqpoint{1.989680in}{1.989680in}}%
\pgfusepath{clip}%
\pgfsetbuttcap%
\pgfsetroundjoin%
\definecolor{currentfill}{rgb}{0.172719,0.448791,0.557885}%
\pgfsetfillcolor{currentfill}%
\pgfsetlinewidth{0.000000pt}%
\definecolor{currentstroke}{rgb}{0.000000,0.000000,0.000000}%
\pgfsetstrokecolor{currentstroke}%
\pgfsetdash{}{0pt}%
\pgfpathmoveto{\pgfqpoint{1.183778in}{0.923346in}}%
\pgfpathlineto{\pgfqpoint{1.182748in}{0.935387in}}%
\pgfpathlineto{\pgfqpoint{1.181712in}{0.947906in}}%
\pgfpathlineto{\pgfqpoint{1.180669in}{0.960912in}}%
\pgfpathlineto{\pgfqpoint{1.158707in}{0.963936in}}%
\pgfpathlineto{\pgfqpoint{1.136970in}{0.967323in}}%
\pgfpathlineto{\pgfqpoint{1.115481in}{0.971067in}}%
\pgfpathlineto{\pgfqpoint{1.094266in}{0.975165in}}%
\pgfpathlineto{\pgfqpoint{1.095834in}{0.962100in}}%
\pgfpathlineto{\pgfqpoint{1.097392in}{0.949521in}}%
\pgfpathlineto{\pgfqpoint{1.098941in}{0.937419in}}%
\pgfpathlineto{\pgfqpoint{1.119773in}{0.933373in}}%
\pgfpathlineto{\pgfqpoint{1.140872in}{0.929676in}}%
\pgfpathlineto{\pgfqpoint{1.162215in}{0.926332in}}%
\pgfpathlineto{\pgfqpoint{1.183778in}{0.923346in}}%
\pgfpathclose%
\pgfusepath{fill}%
\end{pgfscope}%
\begin{pgfscope}%
\pgfpathrectangle{\pgfqpoint{0.329460in}{0.284240in}}{\pgfqpoint{1.989680in}{1.989680in}}%
\pgfusepath{clip}%
\pgfsetbuttcap%
\pgfsetroundjoin%
\definecolor{currentfill}{rgb}{0.172719,0.448791,0.557885}%
\pgfsetfillcolor{currentfill}%
\pgfsetlinewidth{0.000000pt}%
\definecolor{currentstroke}{rgb}{0.000000,0.000000,0.000000}%
\pgfsetstrokecolor{currentstroke}%
\pgfsetdash{}{0pt}%
\pgfpathmoveto{\pgfqpoint{1.537774in}{0.925982in}}%
\pgfpathlineto{\pgfqpoint{1.538922in}{0.938035in}}%
\pgfpathlineto{\pgfqpoint{1.540076in}{0.950565in}}%
\pgfpathlineto{\pgfqpoint{1.541238in}{0.963583in}}%
\pgfpathlineto{\pgfqpoint{1.519254in}{0.960599in}}%
\pgfpathlineto{\pgfqpoint{1.497073in}{0.957982in}}%
\pgfpathlineto{\pgfqpoint{1.474722in}{0.955736in}}%
\pgfpathlineto{\pgfqpoint{1.452227in}{0.953863in}}%
\pgfpathlineto{\pgfqpoint{1.451609in}{0.940886in}}%
\pgfpathlineto{\pgfqpoint{1.450994in}{0.928396in}}%
\pgfpathlineto{\pgfqpoint{1.450383in}{0.916385in}}%
\pgfpathlineto{\pgfqpoint{1.472468in}{0.918235in}}%
\pgfpathlineto{\pgfqpoint{1.494412in}{0.920453in}}%
\pgfpathlineto{\pgfqpoint{1.516189in}{0.923036in}}%
\pgfpathlineto{\pgfqpoint{1.537774in}{0.925982in}}%
\pgfpathclose%
\pgfusepath{fill}%
\end{pgfscope}%
\begin{pgfscope}%
\pgfpathrectangle{\pgfqpoint{0.329460in}{0.284240in}}{\pgfqpoint{1.989680in}{1.989680in}}%
\pgfusepath{clip}%
\pgfsetbuttcap%
\pgfsetroundjoin%
\definecolor{currentfill}{rgb}{0.172719,0.448791,0.557885}%
\pgfsetfillcolor{currentfill}%
\pgfsetlinewidth{0.000000pt}%
\definecolor{currentstroke}{rgb}{0.000000,0.000000,0.000000}%
\pgfsetstrokecolor{currentstroke}%
\pgfsetdash{}{0pt}%
\pgfpathmoveto{\pgfqpoint{1.271721in}{0.915053in}}%
\pgfpathlineto{\pgfqpoint{1.271232in}{0.927058in}}%
\pgfpathlineto{\pgfqpoint{1.270739in}{0.939542in}}%
\pgfpathlineto{\pgfqpoint{1.270244in}{0.952513in}}%
\pgfpathlineto{\pgfqpoint{1.247642in}{0.954052in}}%
\pgfpathlineto{\pgfqpoint{1.225162in}{0.955967in}}%
\pgfpathlineto{\pgfqpoint{1.202829in}{0.958255in}}%
\pgfpathlineto{\pgfqpoint{1.180669in}{0.960912in}}%
\pgfpathlineto{\pgfqpoint{1.181712in}{0.947906in}}%
\pgfpathlineto{\pgfqpoint{1.182748in}{0.935387in}}%
\pgfpathlineto{\pgfqpoint{1.183778in}{0.923346in}}%
\pgfpathlineto{\pgfqpoint{1.205535in}{0.920722in}}%
\pgfpathlineto{\pgfqpoint{1.227462in}{0.918463in}}%
\pgfpathlineto{\pgfqpoint{1.249532in}{0.916572in}}%
\pgfpathlineto{\pgfqpoint{1.271721in}{0.915053in}}%
\pgfpathclose%
\pgfusepath{fill}%
\end{pgfscope}%
\begin{pgfscope}%
\pgfpathrectangle{\pgfqpoint{0.329460in}{0.284240in}}{\pgfqpoint{1.989680in}{1.989680in}}%
\pgfusepath{clip}%
\pgfsetbuttcap%
\pgfsetroundjoin%
\definecolor{currentfill}{rgb}{0.172719,0.448791,0.557885}%
\pgfsetfillcolor{currentfill}%
\pgfsetlinewidth{0.000000pt}%
\definecolor{currentstroke}{rgb}{0.000000,0.000000,0.000000}%
\pgfsetstrokecolor{currentstroke}%
\pgfsetdash{}{0pt}%
\pgfpathmoveto{\pgfqpoint{1.450383in}{0.916385in}}%
\pgfpathlineto{\pgfqpoint{1.450994in}{0.928396in}}%
\pgfpathlineto{\pgfqpoint{1.451609in}{0.940886in}}%
\pgfpathlineto{\pgfqpoint{1.452227in}{0.953863in}}%
\pgfpathlineto{\pgfqpoint{1.429614in}{0.952365in}}%
\pgfpathlineto{\pgfqpoint{1.406909in}{0.951246in}}%
\pgfpathlineto{\pgfqpoint{1.384139in}{0.950506in}}%
\pgfpathlineto{\pgfqpoint{1.361330in}{0.950146in}}%
\pgfpathlineto{\pgfqpoint{1.361268in}{0.937185in}}%
\pgfpathlineto{\pgfqpoint{1.361206in}{0.924711in}}%
\pgfpathlineto{\pgfqpoint{1.361145in}{0.912716in}}%
\pgfpathlineto{\pgfqpoint{1.383537in}{0.913071in}}%
\pgfpathlineto{\pgfqpoint{1.405891in}{0.913802in}}%
\pgfpathlineto{\pgfqpoint{1.428182in}{0.914907in}}%
\pgfpathlineto{\pgfqpoint{1.450383in}{0.916385in}}%
\pgfpathclose%
\pgfusepath{fill}%
\end{pgfscope}%
\begin{pgfscope}%
\pgfpathrectangle{\pgfqpoint{0.329460in}{0.284240in}}{\pgfqpoint{1.989680in}{1.989680in}}%
\pgfusepath{clip}%
\pgfsetbuttcap%
\pgfsetroundjoin%
\definecolor{currentfill}{rgb}{0.172719,0.448791,0.557885}%
\pgfsetfillcolor{currentfill}%
\pgfsetlinewidth{0.000000pt}%
\definecolor{currentstroke}{rgb}{0.000000,0.000000,0.000000}%
\pgfsetstrokecolor{currentstroke}%
\pgfsetdash{}{0pt}%
\pgfpathmoveto{\pgfqpoint{1.361145in}{0.912716in}}%
\pgfpathlineto{\pgfqpoint{1.361206in}{0.924711in}}%
\pgfpathlineto{\pgfqpoint{1.361268in}{0.937185in}}%
\pgfpathlineto{\pgfqpoint{1.361330in}{0.950146in}}%
\pgfpathlineto{\pgfqpoint{1.338510in}{0.950167in}}%
\pgfpathlineto{\pgfqpoint{1.315704in}{0.950569in}}%
\pgfpathlineto{\pgfqpoint{1.292940in}{0.951352in}}%
\pgfpathlineto{\pgfqpoint{1.270244in}{0.952513in}}%
\pgfpathlineto{\pgfqpoint{1.270739in}{0.939542in}}%
\pgfpathlineto{\pgfqpoint{1.271232in}{0.927058in}}%
\pgfpathlineto{\pgfqpoint{1.271721in}{0.915053in}}%
\pgfpathlineto{\pgfqpoint{1.294003in}{0.913906in}}%
\pgfpathlineto{\pgfqpoint{1.316352in}{0.913134in}}%
\pgfpathlineto{\pgfqpoint{1.338741in}{0.912737in}}%
\pgfpathlineto{\pgfqpoint{1.361145in}{0.912716in}}%
\pgfpathclose%
\pgfusepath{fill}%
\end{pgfscope}%
\begin{pgfscope}%
\pgfpathrectangle{\pgfqpoint{0.329460in}{0.284240in}}{\pgfqpoint{1.989680in}{1.989680in}}%
\pgfusepath{clip}%
\pgfsetbuttcap%
\pgfsetroundjoin%
\pgfsetlinewidth{1.505625pt}%
\definecolor{currentstroke}{rgb}{0.000000,0.000000,0.000000}%
\pgfsetstrokecolor{currentstroke}%
\pgfsetdash{}{0pt}%
\pgfpathmoveto{\pgfqpoint{0.946676in}{0.600382in}}%
\pgfpathlineto{\pgfqpoint{1.351188in}{1.168388in}}%
\pgfusepath{stroke}%
\end{pgfscope}%
\begin{pgfscope}%
\pgfpathrectangle{\pgfqpoint{0.329460in}{0.284240in}}{\pgfqpoint{1.989680in}{1.989680in}}%
\pgfusepath{clip}%
\pgfsetbuttcap%
\pgfsetroundjoin%
\pgfsetlinewidth{1.505625pt}%
\definecolor{currentstroke}{rgb}{0.000000,0.000000,0.000000}%
\pgfsetstrokecolor{currentstroke}%
\pgfsetdash{}{0pt}%
\pgfpathmoveto{\pgfqpoint{0.946676in}{0.600382in}}%
\pgfpathlineto{\pgfqpoint{0.969491in}{0.554150in}}%
\pgfusepath{stroke}%
\end{pgfscope}%
\begin{pgfscope}%
\pgfpathrectangle{\pgfqpoint{0.329460in}{0.284240in}}{\pgfqpoint{1.989680in}{1.989680in}}%
\pgfusepath{clip}%
\pgfsetbuttcap%
\pgfsetroundjoin%
\pgfsetlinewidth{1.505625pt}%
\definecolor{currentstroke}{rgb}{0.000000,0.000000,0.000000}%
\pgfsetstrokecolor{currentstroke}%
\pgfsetdash{}{0pt}%
\pgfpathmoveto{\pgfqpoint{0.946676in}{0.600382in}}%
\pgfpathlineto{\pgfqpoint{0.966233in}{0.706781in}}%
\pgfusepath{stroke}%
\end{pgfscope}%
\end{pgfpicture}%
\makeatother%
\endgroup%

	\end{subfigure}
	\caption{
		\textbf{Landau free energy and Mexican hat potential} (\subref{sfig:Landau free energy}) Landau free energy \(f_{\mathrm{L}}\) for a real-valued order parameter \(\Psi\) at different temperatures \(T\). (\subref{sfig:Ginzburg Landau free energy}) Landau free energy for a complex order parameter \(\Psi\).
	} 
	\label{fig:Landau free energy and Ginzburg-Landau free energy}
\end{figure}

\Cref{sfig:Landau free energy} shows the free energy as a function of a single-component, real order parameter \(\Psi\) and it illustrates the essence of Landau theory: there are two cases for the minima of the free energy \(f\)
\begin{equation}
	\Psi = \begin{cases}
		0 & T \geq T_C \\
		\pm \sqrt{\frac{a (T_C - T)}{u}} & T < T_C
	\end{cases} \;,
\end{equation}
so there is a for \(T < T_C\) there are two minima corresponding to ground states with broken symmetry.
When the order parameter can be calculated from some microscopic theory, the critical temperature \(T_C\) can be extracted from the behavior of the order parameter near \(T_C\) via a linear fit of
\begin{equation}
	\vert \Psi \vert^2 \propto T_C - T \;.
\end{equation}

Generalizing this from a one to an \(n\)-component order parameters is straightforward.
One example is the complex or two component order parameter that will become important for superconductivity
\begin{equation}
	\Psi = \Psi_1 + \iu \Psi_2 = \vert \Psi \vert e^{\iu \phi} \,.
\end{equation}
The Landau free energy then takes the form
\begin{equation}
	f_L [\Psi] = r \Psi^* \Psi + \frac{u}{2} (\Psi^* \Psi)^2 = r \vert \Psi \vert^2 + \frac{u}{2} \vert \Psi \vert^4
\end{equation}
with again
\begin{equation}
	r = a(T_C - T) \;.
\end{equation}
Instead of the two minima, the free energy here is rotational symmetry, because it is independent of the phase of the order parameter: 
\begin{equation}
	f_L [\Psi] = f_L [e^{\iu \phi} \Psi] \;.
\end{equation}
This gives the so called `Mexican hat' potential shown in \cref{sfig:Ginzburg Landau free energy}.
In this potential, the order parameter can be rotated continuously from one symmetry-broken state to another.

In 1950, Ginzburg and Landau published their theory of superconductivity, based on Landaus theory of phase transitions \cite{ginzburgTheorySuperconductivity1950}.
Where Landau theory as described above has an uniform order parameters, Ginzburg-Landau theory accounts for it being inhomogeneous, so an order parameter with spatially varying amplitude or direction.
This in turn leads to the order parameter developing a fixed phase, which is the underlying mechanism of the superflow in superconductors.

Ginzburg-Landau theory can be developed for a general \(n\)-component order parameter, but in superfluids and superconductors the order parameter is complex, i.e. two-component.
The Ginzburg-Landau free energy for a complex order parameter is
\begin{equation}
	f_{\mathrm{GL}} [\Psi, \adif{\Psi}] = \frac{\hbar^2}{2m^*} \vert \adif{\Psi} \vert^2 + r \vert \Psi \vert^2 + \frac{u}{2} \vert \Psi \vert^4 \;,
	\label{eq:free energy ginzburg-landau theory}
\end{equation}
where the gradient term \(\adif{\Psi}\) is added in comparison to the Landau free energy.
The prefactor \(\frac{\hbar^2}{2m^*}\) is chosen to illustrate the interpretation of the Ginzburg-Landau free energy as the energy of a condensate of bosons, where the gradient term \(\vert \adif{\Psi} \vert^2\) is the kinetic energy.
The free energy in \cref{eq:free energy ginzburg-landau theory} is sensitive to a twist of the phase of the order parameter.
Substituting the expression \(\Psi = \vert \Psi \vert e^{\iu \phi}\), the gradient term reads
\begin{equation}
	\Delta \Psi = (\adif{\vert \Psi \vert} + \iu \adif{\phi} \vert \Psi \vert) e^{\iu \phi} \;.
\end{equation}
With that, \cref{eq:free energy ginzburg-landau theory} becomes
\begin{equation}
	f_{GL}  = \frac{\hbar^2}{2m^*} \vert \Psi \vert^2 (\adif{\phi})^2 + \left[ \frac{\hbar^2}{2m^*} (\adif{\vert \Psi \vert})^2 + r \vert \Psi \vert^2 + \frac{u}{2} \vert \Psi \vert^4 \right] \;.
	\label{eq:free energy ginzburg-landau theory with phase}
\end{equation}
Now the contributions of phase and amplitude variations are split up: the first term describes energy cost of variations in the phase of the order parameter and the second term describes energy cost of variations in the magnitude of the order parameter.

The dominating fluctuation is determined by the ratio of the factors \(\frac{\hbar^2}{2m^*}\) and \(r\), which has the dimension \(\mathrm{Length}^2\), from which define the correlation length.
\begin{equation}
	\xi = \sqrt{\frac{\hbar^2}{2m^* \vert r \vert}} = \xi_0 \left(1 - \frac{T}{T_C}\right)^{-\frac{1}{2}}
	\label{eq:correlation length GL theory}
\end{equation}
where I define the zero temperature value as the coherence length \(\xi_0 = \xi(T=0) = \sqrt{\frac{\hbar^2}{2 m a T_C}}\).
On length scales above \(\xi\), the physics is entirely controlled by the phase degrees of freedom, i.e.
\begin{align}
	f_{\mathrm{GL}} &= \frac{\hbar^2}{2 m^*} \vert \Psi \vert^2 \left(\adif{\phi}\right)^2 + \mathrm{const.} \\
	&= \frac{\hbar^2}{4 m^*} n_{\mathrm{S}} \left(\adif{\phi}\right)^2 + \mathrm{const.} \\
	&= D_{\mathrm{S}} \left(\adif{\phi}\right)^2 + \mathrm{const.} \label{eq:GL energy above correlation length}
\end{align}
where \(\frac{n_{\mathrm{S}}}{2} = \vert \Psi \vert^2\) is the density of single electrons that form the Cooper pairs, also called the superfluid or superconducting density.
\Cref{eq:GL energy above correlation length} shows that twisting the phase of the condensate is associated with an energy cost.
This energy cost is characterized by the superfluid phase stiffness \(D_{\mathrm{S}}\).


Assuming frozen amplitude fluctuations \(\adif{\vert \Psi (\vb{r}) \vert} = 0\), the stationary point of \cref{eq:free energy ginzburg-landau theory with phase} is
\begin{equation}
	\vert \Psi \vert = \vert \Psi_0 \vert \sqrt{1 - \xi^2 \vert \adif{\phi} (\vb{r}) \vert^2} \;.
	\label{eq:breakdown of OP with phase fluctuations}
\end{equation}
This shows that the superconducting order gets suppressed and eventually destroyed by short-ranged (below \(\xi\)) phase fluctuations.
By introducing a particular form of phase fluctuations \(\phi = \vb{q} \cdot \vb{r}\) into a microscopic model, it is possible to probe this breakdown of superconductivity and thus gain insight into the nature of superconductivity, in particular this gives access to \(\xi\).

The discussion so far is valid for neutral superfluids, but superconductors are charged superfluids, so they couple to electromagnetic fields.
The Ginzburg-Landau free energy with minimal coupling to an electromagnetic field is
\begin{equation}
	f_{\mathrm{GL}} [\Psi, \vb{A}] = \frac{\hbar^2}{2 m^*} \left\vert \left( \Delta - \frac{\iu e^*}{\hbar} \vb{A} \right) \Psi \right\vert^2 + r \vert \Psi \vert^2 + \frac{u}{2} \vert \Psi \vert^4 + \frac{B^2}{2 \mu_0}
	\label{eq:GL energy with minimal coupling to EM field} \;.
\end{equation}
with an additional term to include the electromagnetic energy of the magnetic field \(\vb{B} = \nabla \times \vb{A}\).
It really describes two intertwined Ginzburg-Landau theories for \(\Psi\) and \(\vb{A}\).
This mean there are two length scales, the coherence length \(\xi\) governing amplitude fluctuations of \(\Psi\) and the London penetration depth \(\lambda_L\) which determines the distance magnetic fields penetrate into the superconductor.

The current density can be calculated from the stationary point condition of the free energy w.r.t. the vector potential \(\vb{A}\)
\begin{equation}
	\fdv{f_{\mathrm{GL}}}{\vb{A}} = -\vb{j} + \frac{1}{\mu_0} \nabla \times \vb{B} \overset{!}{=} 0
\end{equation}
defining the supercurrent density
\begin{equation}
	\vb{j} = -\iu \frac{e\hbar}{m^*} \left(\Psi^* \adif{\Psi} - \Psi \adif{\Psi^*}\right) - \frac{4 e^2}{m^*} \vert \Psi \vert^2 \vb{A} \;.
	\label{eq:supercurrent density GL theory with phase and vector potential}
\end{equation}
Introducing the order parameter with a fixed phase \(\Psi = \vert \Psi \vert e^{\iu \phi}\) gives
\begin{equation}
	\vb{j} = 2 e \vert \Psi \vert^2 \frac{\hbar}{m^*} \left(\nabla \phi - \frac{2 \pi}{\Phi_0} \vb{A} \right)
\end{equation}
with the magnetic flux quantum \(\Phi_0 = \frac{\pi \hbar}{e}\).
This shows that not only an applied field \(\vb{A}\) can induce a supercurrent, but also the phase twist \(\nabla \phi\) of the condensate ground state, which is the remarkable property of superconductors enabling the dissipationless current.
Where a conventional current is achieved by excitations above the ground state, the superflow is achieved through deformation of the ground-state phase.
The supercurrent can be gauge-transformed to
\begin{equation}
	\vb{j} = -\frac{4e^2 n_S}{m^*} \vb{A} = \tilde{D}_S \vb{A}
\end{equation}
which shows that the superfluid phase stiffness
\begin{equation}
	D_{\mathrm{S}} = \frac{\hbar^2}{(2e)^2} \tilde{D}_S
\end{equation}
also encodes the linear response of a system to a small applied vector field \(\vb{A}\).
\todo{Connection of SF weight and London penetration depth}


\subsection*{Superconducting Length Scales}

As already discussed in the last section, access to the breakdown of the order parameter with phase fluctuations can give information on the coherence length \(\xi\) and the London penetration depth \(\lambda_L\).
This is a method developed by \citeauthor{wittBypassingLatticeBCS2024} for characterizing superconductivity in alkali-doped fullerides, a material with strong electronic correlations.
The authors find that multiorbital effects enable a superconducting state with a short coherence length but robust stiffness and a domeless rise in critical temperatures with increasing pairing interaction \cite{wittBypassingLatticeBCS2024}.
Access to the superconducting length scales is especially important in the context of BCS-BEC crossover physics and characterizing new high \(T_C\) superconductors.
This is discussed in \cref{sec:BCS-BEC crossover}.

This section gives an introduction to the method in Ginzburg-Landau theory, the inclusion into microscopic theories like BCS-theory and DMFT will be discussed in the respective sections.

\todo{Work over paragraph}
As already discussed in the context of \cref{eq:breakdown of OP with phase fluctuations}, strong phase fluctuations destroy superconducting order.
A particular choice of phase fluctuations would be
\begin{equation}
	\phi (\vb{r}) = \vb{q} \cdot \vb{r}
\end{equation}
which corresponds to Cooper pairs with a finite momentum \(\vb{q}\).

In most materials: Cooper pairs do not carry finite center-of-mass momentum.
In presence of e.g.\ external fields or magnetism: SC states with FMP might arise \cite{chenFiniteMomentumCooper2018, wanOrbitalFuldeFerrell2023, yuanSupercurrentDiodeEffect2022}

Procedure in the paper: enforce FMP states via constraints on pair-center-of-mass momentum \(\vb{q}\), access characteristic lenght scales \(\xi_0, \lambda_L\) through analysis of the momentum and temperature-dependent OP\@.
FF-type pairing with Cooper pairs carrying finite momentum:
\begin{equation}
	\Psi_{\vb{q}} (\vb{r}) = \vert \Psi_{\vb{q}} \vert e^{\iu \vb{q} \vb{r}}
\end{equation}
Then the free energy density \cref{eq:free energy ginzburg-landau theory} is
\begin{equation}
	f_{GL} [\Psi_{\vb{q}}] = r \vert \Psi_{\vb{q}} \vert^2 + \frac{u}{2} \vert \Psi_{\vb{q}} \vert^4 + \frac{\hbar^2 q^2}{2 m^*} \vert \Psi_{\vb{q}} \vert^2
\end{equation}
Stationary point of the system:
\begin{equation}
	\fdv{f_{GL}}{\Psi_{\vb{q}}^*} = 2 \Psi_{\vb{q}} \left[r (1 - \xi^2 q^2) + u \vert \Psi_{\vb{q}} \vert^2\right] = 0
\end{equation}
which results in the \(\vb{q}\)-dependence of the OP
\begin{equation}
	\vert \Psi_{\vb{q}} \vert^2 = \vert \Psi_{0} \vert^2 \left(1 - \xi(T)^2 q^2\right)
\end{equation}
For some value, SC order breaks down, \(\psi_{\vb{q}_c} = 0\), because the kinetic energy from phase modulation exceeds the gain in energy from pairing.
In GL theory: \(q_c = \xi(T)^{-1}\).
The temperature dependence of the OP and extracted \(\xi(T)\) gives access to the coherence length via \cref{eq:correlation length GL theory}
\begin{equation}
	\xi(T) = \xi_0 \left(1 - \frac{T}{T_C}\right)^{-\frac{1}{2}}
\end{equation}
The momentum of the Cooper pairs entails a charge supercurrent \(\vb{j}_{\vb{q}}\).
The current coming from the momentum can be calculated from \cref{eq:supercurrent density GL theory with phase and vector potential} (using \(\phi(\vb{r}) = \vb{q} \cdot \vb{r}\) and \(\vb{A} = 0\)):
\begin{equation}
	\vb{j}_{\vb{q}} = \frac{2 \hbar e}{m^*} \vert \Psi_{\vb{q}} \vert^2 \vb{q}
\end{equation}
The current \(\vb{j}_{\vb{q}}\) is a non-monotonous function of \(\vb{q}\) with a maximum called the depairing current \(j_{dp}\) as can be seen in \cref{sfig:Ginzburg Landau current vs q} \todo{Depairing current connected with correlation length and penetration depth}.
\begin{figure}[t]
	\centering
	\begin{subfigure}[b]{0.5\textwidth}
		\caption{\hfill\null}\label{sfig:Ginzburg Landau OP vs q}
		\centering
		%% Creator: Matplotlib, PGF backend
%%
%% To include the figure in your LaTeX document, write
%%   \input{<filename>.pgf}
%%
%% Make sure the required packages are loaded in your preamble
%%   \usepackage{pgf}
%%
%% Also ensure that all the required font packages are loaded; for instance,
%% the lmodern package is sometimes necessary when using math font.
%%   \usepackage{lmodern}
%%
%% Figures using additional raster images can only be included by \input if
%% they are in the same directory as the main LaTeX file. For loading figures
%% from other directories you can use the `import` package
%%   \usepackage{import}
%%
%% and then include the figures with
%%   \import{<path to file>}{<filename>.pgf}
%%
%% Matplotlib used the following preamble
%%   \def\mathdefault#1{#1}
%%   \everymath=\expandafter{\the\everymath\displaystyle}
%%   \IfFileExists{scrextend.sty}{
%%     \usepackage[fontsize=11.000000pt]{scrextend}
%%   }{
%%     \renewcommand{\normalsize}{\fontsize{11.000000}{13.200000}\selectfont}
%%     \normalsize
%%   }
%%   \usepackage{fontspec}\usepackage{unicode-math}\setmathfont{texgyrepagella-math.otf}\setmainfont{texgyrepagella-math}\usepackage{nicefrac}
%%   \makeatletter\@ifpackageloaded{underscore}{}{\usepackage[strings]{underscore}}\makeatother
%%
\begingroup%
\makeatletter%
\begin{pgfpicture}%
\pgfpathrectangle{\pgfpointorigin}{\pgfqpoint{2.300000in}{1.150000in}}%
\pgfusepath{use as bounding box, clip}%
\begin{pgfscope}%
\pgfsetbuttcap%
\pgfsetmiterjoin%
\definecolor{currentfill}{rgb}{1.000000,1.000000,1.000000}%
\pgfsetfillcolor{currentfill}%
\pgfsetlinewidth{0.000000pt}%
\definecolor{currentstroke}{rgb}{1.000000,1.000000,1.000000}%
\pgfsetstrokecolor{currentstroke}%
\pgfsetdash{}{0pt}%
\pgfpathmoveto{\pgfqpoint{0.000000in}{0.000000in}}%
\pgfpathlineto{\pgfqpoint{2.300000in}{0.000000in}}%
\pgfpathlineto{\pgfqpoint{2.300000in}{1.150000in}}%
\pgfpathlineto{\pgfqpoint{0.000000in}{1.150000in}}%
\pgfpathlineto{\pgfqpoint{0.000000in}{0.000000in}}%
\pgfpathclose%
\pgfusepath{fill}%
\end{pgfscope}%
\begin{pgfscope}%
\pgfsetbuttcap%
\pgfsetmiterjoin%
\definecolor{currentfill}{rgb}{1.000000,1.000000,1.000000}%
\pgfsetfillcolor{currentfill}%
\pgfsetlinewidth{0.000000pt}%
\definecolor{currentstroke}{rgb}{0.000000,0.000000,0.000000}%
\pgfsetstrokecolor{currentstroke}%
\pgfsetstrokeopacity{0.000000}%
\pgfsetdash{}{0pt}%
\pgfpathmoveto{\pgfqpoint{0.388607in}{0.274559in}}%
\pgfpathlineto{\pgfqpoint{2.216663in}{0.274559in}}%
\pgfpathlineto{\pgfqpoint{2.216663in}{1.066663in}}%
\pgfpathlineto{\pgfqpoint{0.388607in}{1.066663in}}%
\pgfpathlineto{\pgfqpoint{0.388607in}{0.274559in}}%
\pgfpathclose%
\pgfusepath{fill}%
\end{pgfscope}%
\begin{pgfscope}%
\pgfsetbuttcap%
\pgfsetroundjoin%
\definecolor{currentfill}{rgb}{0.000000,0.000000,0.000000}%
\pgfsetfillcolor{currentfill}%
\pgfsetlinewidth{0.803000pt}%
\definecolor{currentstroke}{rgb}{0.000000,0.000000,0.000000}%
\pgfsetstrokecolor{currentstroke}%
\pgfsetdash{}{0pt}%
\pgfsys@defobject{currentmarker}{\pgfqpoint{0.000000in}{-0.048611in}}{\pgfqpoint{0.000000in}{0.000000in}}{%
\pgfpathmoveto{\pgfqpoint{0.000000in}{0.000000in}}%
\pgfpathlineto{\pgfqpoint{0.000000in}{-0.048611in}}%
\pgfusepath{stroke,fill}%
}%
\begin{pgfscope}%
\pgfsys@transformshift{2.050476in}{0.274559in}%
\pgfsys@useobject{currentmarker}{}%
\end{pgfscope}%
\end{pgfscope}%
\begin{pgfscope}%
\definecolor{textcolor}{rgb}{0.000000,0.000000,0.000000}%
\pgfsetstrokecolor{textcolor}%
\pgfsetfillcolor{textcolor}%
\pgftext[x=2.050476in,y=0.177337in,,top]{\color{textcolor}{\sffamily\fontsize{11.000000}{13.200000}\selectfont\catcode`\^=\active\def^{\ifmmode\sp\else\^{}\fi}\catcode`\%=\active\def%{\%}$q_c$}}%
\end{pgfscope}%
\begin{pgfscope}%
\definecolor{textcolor}{rgb}{0.000000,0.000000,0.000000}%
\pgfsetstrokecolor{textcolor}%
\pgfsetfillcolor{textcolor}%
\pgftext[x=2.253224in,y=0.472585in,,top]{\color{textcolor}{\sffamily\fontsize{11.000000}{13.200000}\selectfont\catcode`\^=\active\def^{\ifmmode\sp\else\^{}\fi}\catcode`\%=\active\def%{\%}$q$}}%
\end{pgfscope}%
\begin{pgfscope}%
\pgfsetbuttcap%
\pgfsetroundjoin%
\definecolor{currentfill}{rgb}{0.000000,0.000000,0.000000}%
\pgfsetfillcolor{currentfill}%
\pgfsetlinewidth{0.803000pt}%
\definecolor{currentstroke}{rgb}{0.000000,0.000000,0.000000}%
\pgfsetstrokecolor{currentstroke}%
\pgfsetdash{}{0pt}%
\pgfsys@defobject{currentmarker}{\pgfqpoint{-0.048611in}{0.000000in}}{\pgfqpoint{-0.000000in}{0.000000in}}{%
\pgfpathmoveto{\pgfqpoint{-0.000000in}{0.000000in}}%
\pgfpathlineto{\pgfqpoint{-0.048611in}{0.000000in}}%
\pgfusepath{stroke,fill}%
}%
\begin{pgfscope}%
\pgfsys@transformshift{0.388607in}{0.861303in}%
\pgfsys@useobject{currentmarker}{}%
\end{pgfscope}%
\end{pgfscope}%
\begin{pgfscope}%
\definecolor{textcolor}{rgb}{0.000000,0.000000,0.000000}%
\pgfsetstrokecolor{textcolor}%
\pgfsetfillcolor{textcolor}%
\pgftext[x=0.041670in, y=0.808442in, left, base]{\color{textcolor}{\sffamily\fontsize{11.000000}{13.200000}\selectfont\catcode`\^=\active\def^{\ifmmode\sp\else\^{}\fi}\catcode`\%=\active\def%{\%}$\vert \Psi_0 \vert$}}%
\end{pgfscope}%
\begin{pgfscope}%
\definecolor{textcolor}{rgb}{0.000000,0.000000,0.000000}%
\pgfsetstrokecolor{textcolor}%
\pgfsetfillcolor{textcolor}%
\pgftext[x=0.580553in,y=0.979532in,,bottom]{\color{textcolor}{\sffamily\fontsize{11.000000}{13.200000}\selectfont\catcode`\^=\active\def^{\ifmmode\sp\else\^{}\fi}\catcode`\%=\active\def%{\%}$\vert \Psi_q \vert$}}%
\end{pgfscope}%
\begin{pgfscope}%
\pgfpathrectangle{\pgfqpoint{0.388607in}{0.274559in}}{\pgfqpoint{1.828056in}{0.792104in}}%
\pgfusepath{clip}%
\pgfsetrectcap%
\pgfsetroundjoin%
\pgfsetlinewidth{1.505625pt}%
\definecolor{currentstroke}{rgb}{0.247059,0.564706,0.854902}%
\pgfsetstrokecolor{currentstroke}%
\pgfsetdash{}{0pt}%
\pgfpathmoveto{\pgfqpoint{0.388607in}{0.861303in}}%
\pgfpathlineto{\pgfqpoint{0.405394in}{0.861273in}}%
\pgfpathlineto{\pgfqpoint{0.422180in}{0.861183in}}%
\pgfpathlineto{\pgfqpoint{0.438967in}{0.861033in}}%
\pgfpathlineto{\pgfqpoint{0.455753in}{0.860824in}}%
\pgfpathlineto{\pgfqpoint{0.472540in}{0.860554in}}%
\pgfpathlineto{\pgfqpoint{0.489327in}{0.860224in}}%
\pgfpathlineto{\pgfqpoint{0.506113in}{0.859834in}}%
\pgfpathlineto{\pgfqpoint{0.522900in}{0.859384in}}%
\pgfpathlineto{\pgfqpoint{0.539686in}{0.858873in}}%
\pgfpathlineto{\pgfqpoint{0.556473in}{0.858302in}}%
\pgfpathlineto{\pgfqpoint{0.573259in}{0.857670in}}%
\pgfpathlineto{\pgfqpoint{0.590046in}{0.856977in}}%
\pgfpathlineto{\pgfqpoint{0.606832in}{0.856222in}}%
\pgfpathlineto{\pgfqpoint{0.623619in}{0.855406in}}%
\pgfpathlineto{\pgfqpoint{0.640406in}{0.854529in}}%
\pgfpathlineto{\pgfqpoint{0.657192in}{0.853589in}}%
\pgfpathlineto{\pgfqpoint{0.673979in}{0.852588in}}%
\pgfpathlineto{\pgfqpoint{0.690765in}{0.851523in}}%
\pgfpathlineto{\pgfqpoint{0.707552in}{0.850396in}}%
\pgfpathlineto{\pgfqpoint{0.724338in}{0.849205in}}%
\pgfpathlineto{\pgfqpoint{0.741125in}{0.847951in}}%
\pgfpathlineto{\pgfqpoint{0.757911in}{0.846632in}}%
\pgfpathlineto{\pgfqpoint{0.774698in}{0.845249in}}%
\pgfpathlineto{\pgfqpoint{0.791485in}{0.843801in}}%
\pgfpathlineto{\pgfqpoint{0.808271in}{0.842287in}}%
\pgfpathlineto{\pgfqpoint{0.825058in}{0.840707in}}%
\pgfpathlineto{\pgfqpoint{0.841844in}{0.839060in}}%
\pgfpathlineto{\pgfqpoint{0.858631in}{0.837346in}}%
\pgfpathlineto{\pgfqpoint{0.875417in}{0.835565in}}%
\pgfpathlineto{\pgfqpoint{0.892204in}{0.833715in}}%
\pgfpathlineto{\pgfqpoint{0.908991in}{0.831795in}}%
\pgfpathlineto{\pgfqpoint{0.925777in}{0.829806in}}%
\pgfpathlineto{\pgfqpoint{0.942564in}{0.827746in}}%
\pgfpathlineto{\pgfqpoint{0.959350in}{0.825615in}}%
\pgfpathlineto{\pgfqpoint{0.976137in}{0.823412in}}%
\pgfpathlineto{\pgfqpoint{0.992923in}{0.821135in}}%
\pgfpathlineto{\pgfqpoint{1.009710in}{0.818784in}}%
\pgfpathlineto{\pgfqpoint{1.026496in}{0.816358in}}%
\pgfpathlineto{\pgfqpoint{1.043283in}{0.813857in}}%
\pgfpathlineto{\pgfqpoint{1.060070in}{0.811278in}}%
\pgfpathlineto{\pgfqpoint{1.076856in}{0.808621in}}%
\pgfpathlineto{\pgfqpoint{1.093643in}{0.805884in}}%
\pgfpathlineto{\pgfqpoint{1.110429in}{0.803067in}}%
\pgfpathlineto{\pgfqpoint{1.127216in}{0.800168in}}%
\pgfpathlineto{\pgfqpoint{1.144002in}{0.797186in}}%
\pgfpathlineto{\pgfqpoint{1.160789in}{0.794119in}}%
\pgfpathlineto{\pgfqpoint{1.177575in}{0.790965in}}%
\pgfpathlineto{\pgfqpoint{1.194362in}{0.787724in}}%
\pgfpathlineto{\pgfqpoint{1.211149in}{0.784394in}}%
\pgfpathlineto{\pgfqpoint{1.227935in}{0.780972in}}%
\pgfpathlineto{\pgfqpoint{1.244722in}{0.777457in}}%
\pgfpathlineto{\pgfqpoint{1.261508in}{0.773847in}}%
\pgfpathlineto{\pgfqpoint{1.278295in}{0.770139in}}%
\pgfpathlineto{\pgfqpoint{1.295081in}{0.766333in}}%
\pgfpathlineto{\pgfqpoint{1.311868in}{0.762424in}}%
\pgfpathlineto{\pgfqpoint{1.328654in}{0.758412in}}%
\pgfpathlineto{\pgfqpoint{1.345441in}{0.754293in}}%
\pgfpathlineto{\pgfqpoint{1.362228in}{0.750064in}}%
\pgfpathlineto{\pgfqpoint{1.379014in}{0.745723in}}%
\pgfpathlineto{\pgfqpoint{1.395801in}{0.741266in}}%
\pgfpathlineto{\pgfqpoint{1.412587in}{0.736690in}}%
\pgfpathlineto{\pgfqpoint{1.429374in}{0.731992in}}%
\pgfpathlineto{\pgfqpoint{1.446160in}{0.727167in}}%
\pgfpathlineto{\pgfqpoint{1.462947in}{0.722212in}}%
\pgfpathlineto{\pgfqpoint{1.479733in}{0.717122in}}%
\pgfpathlineto{\pgfqpoint{1.496520in}{0.711892in}}%
\pgfpathlineto{\pgfqpoint{1.513307in}{0.706518in}}%
\pgfpathlineto{\pgfqpoint{1.530093in}{0.700994in}}%
\pgfpathlineto{\pgfqpoint{1.546880in}{0.695313in}}%
\pgfpathlineto{\pgfqpoint{1.563666in}{0.689471in}}%
\pgfpathlineto{\pgfqpoint{1.580453in}{0.683459in}}%
\pgfpathlineto{\pgfqpoint{1.597239in}{0.677270in}}%
\pgfpathlineto{\pgfqpoint{1.614026in}{0.670896in}}%
\pgfpathlineto{\pgfqpoint{1.630812in}{0.664327in}}%
\pgfpathlineto{\pgfqpoint{1.647599in}{0.657554in}}%
\pgfpathlineto{\pgfqpoint{1.664386in}{0.650566in}}%
\pgfpathlineto{\pgfqpoint{1.681172in}{0.643351in}}%
\pgfpathlineto{\pgfqpoint{1.697959in}{0.635894in}}%
\pgfpathlineto{\pgfqpoint{1.714745in}{0.628180in}}%
\pgfpathlineto{\pgfqpoint{1.731532in}{0.620193in}}%
\pgfpathlineto{\pgfqpoint{1.748318in}{0.611913in}}%
\pgfpathlineto{\pgfqpoint{1.765105in}{0.603318in}}%
\pgfpathlineto{\pgfqpoint{1.781891in}{0.594382in}}%
\pgfpathlineto{\pgfqpoint{1.798678in}{0.585075in}}%
\pgfpathlineto{\pgfqpoint{1.815465in}{0.575365in}}%
\pgfpathlineto{\pgfqpoint{1.832251in}{0.565209in}}%
\pgfpathlineto{\pgfqpoint{1.849038in}{0.554561in}}%
\pgfpathlineto{\pgfqpoint{1.865824in}{0.543360in}}%
\pgfpathlineto{\pgfqpoint{1.882611in}{0.531535in}}%
\pgfpathlineto{\pgfqpoint{1.899397in}{0.518995in}}%
\pgfpathlineto{\pgfqpoint{1.916184in}{0.505624in}}%
\pgfpathlineto{\pgfqpoint{1.932971in}{0.491269in}}%
\pgfpathlineto{\pgfqpoint{1.949757in}{0.475718in}}%
\pgfpathlineto{\pgfqpoint{1.966544in}{0.458669in}}%
\pgfpathlineto{\pgfqpoint{1.983330in}{0.439658in}}%
\pgfpathlineto{\pgfqpoint{2.000117in}{0.417907in}}%
\pgfpathlineto{\pgfqpoint{2.016903in}{0.391902in}}%
\pgfpathlineto{\pgfqpoint{2.033690in}{0.357744in}}%
\pgfpathlineto{\pgfqpoint{2.050476in}{0.274559in}}%
\pgfusepath{stroke}%
\end{pgfscope}%
\begin{pgfscope}%
\pgfsetbuttcap%
\pgfsetmiterjoin%
\definecolor{currentfill}{rgb}{0.000000,0.000000,0.000000}%
\pgfsetfillcolor{currentfill}%
\pgfsetlinewidth{1.003750pt}%
\definecolor{currentstroke}{rgb}{0.000000,0.000000,0.000000}%
\pgfsetstrokecolor{currentstroke}%
\pgfsetdash{}{0pt}%
\pgfsys@defobject{currentmarker}{\pgfqpoint{-0.041667in}{-0.041667in}}{\pgfqpoint{0.041667in}{0.041667in}}{%
\pgfpathmoveto{\pgfqpoint{0.041667in}{-0.000000in}}%
\pgfpathlineto{\pgfqpoint{-0.041667in}{0.041667in}}%
\pgfpathlineto{\pgfqpoint{-0.041667in}{-0.041667in}}%
\pgfpathlineto{\pgfqpoint{0.041667in}{-0.000000in}}%
\pgfpathclose%
\pgfusepath{stroke,fill}%
}%
\begin{pgfscope}%
\pgfsys@transformshift{2.216663in}{0.274559in}%
\pgfsys@useobject{currentmarker}{}%
\end{pgfscope}%
\end{pgfscope}%
\begin{pgfscope}%
\pgfsetbuttcap%
\pgfsetmiterjoin%
\definecolor{currentfill}{rgb}{0.000000,0.000000,0.000000}%
\pgfsetfillcolor{currentfill}%
\pgfsetlinewidth{1.003750pt}%
\definecolor{currentstroke}{rgb}{0.000000,0.000000,0.000000}%
\pgfsetstrokecolor{currentstroke}%
\pgfsetdash{}{0pt}%
\pgfsys@defobject{currentmarker}{\pgfqpoint{-0.041667in}{-0.041667in}}{\pgfqpoint{0.041667in}{0.041667in}}{%
\pgfpathmoveto{\pgfqpoint{0.000000in}{0.041667in}}%
\pgfpathlineto{\pgfqpoint{-0.041667in}{-0.041667in}}%
\pgfpathlineto{\pgfqpoint{0.041667in}{-0.041667in}}%
\pgfpathlineto{\pgfqpoint{0.000000in}{0.041667in}}%
\pgfpathclose%
\pgfusepath{stroke,fill}%
}%
\begin{pgfscope}%
\pgfsys@transformshift{0.388607in}{1.066663in}%
\pgfsys@useobject{currentmarker}{}%
\end{pgfscope}%
\end{pgfscope}%
\begin{pgfscope}%
\pgfsetrectcap%
\pgfsetmiterjoin%
\pgfsetlinewidth{0.803000pt}%
\definecolor{currentstroke}{rgb}{0.000000,0.000000,0.000000}%
\pgfsetstrokecolor{currentstroke}%
\pgfsetdash{}{0pt}%
\pgfpathmoveto{\pgfqpoint{0.388607in}{0.274559in}}%
\pgfpathlineto{\pgfqpoint{0.388607in}{1.066663in}}%
\pgfusepath{stroke}%
\end{pgfscope}%
\begin{pgfscope}%
\pgfsetrectcap%
\pgfsetmiterjoin%
\pgfsetlinewidth{0.803000pt}%
\definecolor{currentstroke}{rgb}{0.000000,0.000000,0.000000}%
\pgfsetstrokecolor{currentstroke}%
\pgfsetdash{}{0pt}%
\pgfpathmoveto{\pgfqpoint{0.388607in}{0.274559in}}%
\pgfpathlineto{\pgfqpoint{2.216663in}{0.274559in}}%
\pgfusepath{stroke}%
\end{pgfscope}%
\end{pgfpicture}%
\makeatother%
\endgroup%

	\end{subfigure}%
	\hfill
	\begin{subfigure}[b]{0.5\textwidth}
		\centering
		\caption{\hfill\null}\label{sfig:Ginzburg Landau current vs q}
		%% Creator: Matplotlib, PGF backend
%%
%% To include the figure in your LaTeX document, write
%%   \input{<filename>.pgf}
%%
%% Make sure the required packages are loaded in your preamble
%%   \usepackage{pgf}
%%
%% Also ensure that all the required font packages are loaded; for instance,
%% the lmodern package is sometimes necessary when using math font.
%%   \usepackage{lmodern}
%%
%% Figures using additional raster images can only be included by \input if
%% they are in the same directory as the main LaTeX file. For loading figures
%% from other directories you can use the `import` package
%%   \usepackage{import}
%%
%% and then include the figures with
%%   \import{<path to file>}{<filename>.pgf}
%%
%% Matplotlib used the following preamble
%%   \def\mathdefault#1{#1}
%%   \everymath=\expandafter{\the\everymath\displaystyle}
%%   \IfFileExists{scrextend.sty}{
%%     \usepackage[fontsize=11.000000pt]{scrextend}
%%   }{
%%     \renewcommand{\normalsize}{\fontsize{11.000000}{13.200000}\selectfont}
%%     \normalsize
%%   }
%%   \usepackage{fontspec}\usepackage{unicode-math}\setmathfont{texgyrepagella-math.otf}\setmainfont{texgyrepagella-math}\usepackage{nicefrac}
%%   \makeatletter\@ifpackageloaded{underscore}{}{\usepackage[strings]{underscore}}\makeatother
%%
\begingroup%
\makeatletter%
\begin{pgfpicture}%
\pgfpathrectangle{\pgfpointorigin}{\pgfqpoint{2.300000in}{1.150000in}}%
\pgfusepath{use as bounding box, clip}%
\begin{pgfscope}%
\pgfsetbuttcap%
\pgfsetmiterjoin%
\definecolor{currentfill}{rgb}{1.000000,1.000000,1.000000}%
\pgfsetfillcolor{currentfill}%
\pgfsetlinewidth{0.000000pt}%
\definecolor{currentstroke}{rgb}{1.000000,1.000000,1.000000}%
\pgfsetstrokecolor{currentstroke}%
\pgfsetdash{}{0pt}%
\pgfpathmoveto{\pgfqpoint{0.000000in}{0.000000in}}%
\pgfpathlineto{\pgfqpoint{2.300000in}{0.000000in}}%
\pgfpathlineto{\pgfqpoint{2.300000in}{1.150000in}}%
\pgfpathlineto{\pgfqpoint{0.000000in}{1.150000in}}%
\pgfpathlineto{\pgfqpoint{0.000000in}{0.000000in}}%
\pgfpathclose%
\pgfusepath{fill}%
\end{pgfscope}%
\begin{pgfscope}%
\pgfsetbuttcap%
\pgfsetmiterjoin%
\definecolor{currentfill}{rgb}{1.000000,1.000000,1.000000}%
\pgfsetfillcolor{currentfill}%
\pgfsetlinewidth{0.000000pt}%
\definecolor{currentstroke}{rgb}{0.000000,0.000000,0.000000}%
\pgfsetstrokecolor{currentstroke}%
\pgfsetstrokeopacity{0.000000}%
\pgfsetdash{}{0pt}%
\pgfpathmoveto{\pgfqpoint{0.317092in}{0.274559in}}%
\pgfpathlineto{\pgfqpoint{2.215339in}{0.274559in}}%
\pgfpathlineto{\pgfqpoint{2.215339in}{1.066663in}}%
\pgfpathlineto{\pgfqpoint{0.317092in}{1.066663in}}%
\pgfpathlineto{\pgfqpoint{0.317092in}{0.274559in}}%
\pgfpathclose%
\pgfusepath{fill}%
\end{pgfscope}%
\begin{pgfscope}%
\pgfsetbuttcap%
\pgfsetroundjoin%
\definecolor{currentfill}{rgb}{0.000000,0.000000,0.000000}%
\pgfsetfillcolor{currentfill}%
\pgfsetlinewidth{0.803000pt}%
\definecolor{currentstroke}{rgb}{0.000000,0.000000,0.000000}%
\pgfsetstrokecolor{currentstroke}%
\pgfsetdash{}{0pt}%
\pgfsys@defobject{currentmarker}{\pgfqpoint{0.000000in}{-0.048611in}}{\pgfqpoint{0.000000in}{0.000000in}}{%
\pgfpathmoveto{\pgfqpoint{0.000000in}{0.000000in}}%
\pgfpathlineto{\pgfqpoint{0.000000in}{-0.048611in}}%
\pgfusepath{stroke,fill}%
}%
\begin{pgfscope}%
\pgfsys@transformshift{1.360902in}{0.274559in}%
\pgfsys@useobject{currentmarker}{}%
\end{pgfscope}%
\end{pgfscope}%
\begin{pgfscope}%
\definecolor{textcolor}{rgb}{0.000000,0.000000,0.000000}%
\pgfsetstrokecolor{textcolor}%
\pgfsetfillcolor{textcolor}%
\pgftext[x=1.360902in,y=0.177337in,,top]{\color{textcolor}{\sffamily\fontsize{11.000000}{13.200000}\selectfont\catcode`\^=\active\def^{\ifmmode\sp\else\^{}\fi}\catcode`\%=\active\def%{\%}$q_{\mathrm{max}}$}}%
\end{pgfscope}%
\begin{pgfscope}%
\definecolor{textcolor}{rgb}{0.000000,0.000000,0.000000}%
\pgfsetstrokecolor{textcolor}%
\pgfsetfillcolor{textcolor}%
\pgftext[x=2.253304in,y=0.472585in,,top]{\color{textcolor}{\sffamily\fontsize{11.000000}{13.200000}\selectfont\catcode`\^=\active\def^{\ifmmode\sp\else\^{}\fi}\catcode`\%=\active\def%{\%}$q$}}%
\end{pgfscope}%
\begin{pgfscope}%
\pgfsetbuttcap%
\pgfsetroundjoin%
\definecolor{currentfill}{rgb}{0.000000,0.000000,0.000000}%
\pgfsetfillcolor{currentfill}%
\pgfsetlinewidth{0.803000pt}%
\definecolor{currentstroke}{rgb}{0.000000,0.000000,0.000000}%
\pgfsetstrokecolor{currentstroke}%
\pgfsetdash{}{0pt}%
\pgfsys@defobject{currentmarker}{\pgfqpoint{-0.048611in}{0.000000in}}{\pgfqpoint{-0.000000in}{0.000000in}}{%
\pgfpathmoveto{\pgfqpoint{-0.000000in}{0.000000in}}%
\pgfpathlineto{\pgfqpoint{-0.048611in}{0.000000in}}%
\pgfusepath{stroke,fill}%
}%
\begin{pgfscope}%
\pgfsys@transformshift{0.317092in}{0.934646in}%
\pgfsys@useobject{currentmarker}{}%
\end{pgfscope}%
\end{pgfscope}%
\begin{pgfscope}%
\definecolor{textcolor}{rgb}{0.000000,0.000000,0.000000}%
\pgfsetstrokecolor{textcolor}%
\pgfsetfillcolor{textcolor}%
\pgftext[x=0.041670in, y=0.883465in, left, base]{\color{textcolor}{\sffamily\fontsize{11.000000}{13.200000}\selectfont\catcode`\^=\active\def^{\ifmmode\sp\else\^{}\fi}\catcode`\%=\active\def%{\%}$j_{\mathrm{dp}}$}}%
\end{pgfscope}%
\begin{pgfscope}%
\definecolor{textcolor}{rgb}{0.000000,0.000000,0.000000}%
\pgfsetstrokecolor{textcolor}%
\pgfsetfillcolor{textcolor}%
\pgftext[x=0.468952in,y=0.979532in,,bottom]{\color{textcolor}{\sffamily\fontsize{11.000000}{13.200000}\selectfont\catcode`\^=\active\def^{\ifmmode\sp\else\^{}\fi}\catcode`\%=\active\def%{\%}$j_q$}}%
\end{pgfscope}%
\begin{pgfscope}%
\pgfpathrectangle{\pgfqpoint{0.317092in}{0.274559in}}{\pgfqpoint{1.898247in}{0.792104in}}%
\pgfusepath{clip}%
\pgfsetrectcap%
\pgfsetroundjoin%
\pgfsetlinewidth{1.505625pt}%
\definecolor{currentstroke}{rgb}{0.247059,0.564706,0.854902}%
\pgfsetstrokecolor{currentstroke}%
\pgfsetdash{}{0pt}%
\pgfpathmoveto{\pgfqpoint{0.317092in}{0.274559in}}%
\pgfpathlineto{\pgfqpoint{0.335353in}{0.291880in}}%
\pgfpathlineto{\pgfqpoint{0.353614in}{0.309190in}}%
\pgfpathlineto{\pgfqpoint{0.371875in}{0.326480in}}%
\pgfpathlineto{\pgfqpoint{0.390136in}{0.343737in}}%
\pgfpathlineto{\pgfqpoint{0.408398in}{0.360952in}}%
\pgfpathlineto{\pgfqpoint{0.426659in}{0.378114in}}%
\pgfpathlineto{\pgfqpoint{0.444920in}{0.395212in}}%
\pgfpathlineto{\pgfqpoint{0.463181in}{0.412236in}}%
\pgfpathlineto{\pgfqpoint{0.481442in}{0.429176in}}%
\pgfpathlineto{\pgfqpoint{0.499703in}{0.446019in}}%
\pgfpathlineto{\pgfqpoint{0.517964in}{0.462757in}}%
\pgfpathlineto{\pgfqpoint{0.536226in}{0.479378in}}%
\pgfpathlineto{\pgfqpoint{0.554487in}{0.495872in}}%
\pgfpathlineto{\pgfqpoint{0.572748in}{0.512228in}}%
\pgfpathlineto{\pgfqpoint{0.591009in}{0.528436in}}%
\pgfpathlineto{\pgfqpoint{0.609270in}{0.544484in}}%
\pgfpathlineto{\pgfqpoint{0.627531in}{0.560363in}}%
\pgfpathlineto{\pgfqpoint{0.645793in}{0.576061in}}%
\pgfpathlineto{\pgfqpoint{0.664054in}{0.591569in}}%
\pgfpathlineto{\pgfqpoint{0.682315in}{0.606875in}}%
\pgfpathlineto{\pgfqpoint{0.700576in}{0.621969in}}%
\pgfpathlineto{\pgfqpoint{0.718837in}{0.636840in}}%
\pgfpathlineto{\pgfqpoint{0.737098in}{0.651479in}}%
\pgfpathlineto{\pgfqpoint{0.755359in}{0.665873in}}%
\pgfpathlineto{\pgfqpoint{0.773621in}{0.680012in}}%
\pgfpathlineto{\pgfqpoint{0.791882in}{0.693887in}}%
\pgfpathlineto{\pgfqpoint{0.810143in}{0.707486in}}%
\pgfpathlineto{\pgfqpoint{0.828404in}{0.720798in}}%
\pgfpathlineto{\pgfqpoint{0.846665in}{0.733813in}}%
\pgfpathlineto{\pgfqpoint{0.864926in}{0.746521in}}%
\pgfpathlineto{\pgfqpoint{0.883188in}{0.758911in}}%
\pgfpathlineto{\pgfqpoint{0.901449in}{0.770972in}}%
\pgfpathlineto{\pgfqpoint{0.919710in}{0.782694in}}%
\pgfpathlineto{\pgfqpoint{0.937971in}{0.794066in}}%
\pgfpathlineto{\pgfqpoint{0.956232in}{0.805077in}}%
\pgfpathlineto{\pgfqpoint{0.974493in}{0.815717in}}%
\pgfpathlineto{\pgfqpoint{0.992754in}{0.825975in}}%
\pgfpathlineto{\pgfqpoint{1.011016in}{0.835841in}}%
\pgfpathlineto{\pgfqpoint{1.029277in}{0.845304in}}%
\pgfpathlineto{\pgfqpoint{1.047538in}{0.854354in}}%
\pgfpathlineto{\pgfqpoint{1.065799in}{0.862979in}}%
\pgfpathlineto{\pgfqpoint{1.084060in}{0.871169in}}%
\pgfpathlineto{\pgfqpoint{1.102321in}{0.878914in}}%
\pgfpathlineto{\pgfqpoint{1.120582in}{0.886203in}}%
\pgfpathlineto{\pgfqpoint{1.138844in}{0.893026in}}%
\pgfpathlineto{\pgfqpoint{1.157105in}{0.899371in}}%
\pgfpathlineto{\pgfqpoint{1.175366in}{0.905228in}}%
\pgfpathlineto{\pgfqpoint{1.193627in}{0.910587in}}%
\pgfpathlineto{\pgfqpoint{1.211888in}{0.915437in}}%
\pgfpathlineto{\pgfqpoint{1.230149in}{0.919767in}}%
\pgfpathlineto{\pgfqpoint{1.248411in}{0.923567in}}%
\pgfpathlineto{\pgfqpoint{1.266672in}{0.926826in}}%
\pgfpathlineto{\pgfqpoint{1.284933in}{0.929534in}}%
\pgfpathlineto{\pgfqpoint{1.303194in}{0.931680in}}%
\pgfpathlineto{\pgfqpoint{1.321455in}{0.933253in}}%
\pgfpathlineto{\pgfqpoint{1.339716in}{0.934242in}}%
\pgfpathlineto{\pgfqpoint{1.357977in}{0.934638in}}%
\pgfpathlineto{\pgfqpoint{1.376239in}{0.934430in}}%
\pgfpathlineto{\pgfqpoint{1.394500in}{0.933606in}}%
\pgfpathlineto{\pgfqpoint{1.412761in}{0.932157in}}%
\pgfpathlineto{\pgfqpoint{1.431022in}{0.930071in}}%
\pgfpathlineto{\pgfqpoint{1.449283in}{0.927339in}}%
\pgfpathlineto{\pgfqpoint{1.467544in}{0.923949in}}%
\pgfpathlineto{\pgfqpoint{1.485806in}{0.919891in}}%
\pgfpathlineto{\pgfqpoint{1.504067in}{0.915154in}}%
\pgfpathlineto{\pgfqpoint{1.522328in}{0.909728in}}%
\pgfpathlineto{\pgfqpoint{1.540589in}{0.903602in}}%
\pgfpathlineto{\pgfqpoint{1.558850in}{0.896765in}}%
\pgfpathlineto{\pgfqpoint{1.577111in}{0.889208in}}%
\pgfpathlineto{\pgfqpoint{1.595372in}{0.880918in}}%
\pgfpathlineto{\pgfqpoint{1.613634in}{0.871887in}}%
\pgfpathlineto{\pgfqpoint{1.631895in}{0.862102in}}%
\pgfpathlineto{\pgfqpoint{1.650156in}{0.851554in}}%
\pgfpathlineto{\pgfqpoint{1.668417in}{0.840232in}}%
\pgfpathlineto{\pgfqpoint{1.686678in}{0.828125in}}%
\pgfpathlineto{\pgfqpoint{1.704939in}{0.815222in}}%
\pgfpathlineto{\pgfqpoint{1.723201in}{0.801514in}}%
\pgfpathlineto{\pgfqpoint{1.741462in}{0.786989in}}%
\pgfpathlineto{\pgfqpoint{1.759723in}{0.771637in}}%
\pgfpathlineto{\pgfqpoint{1.777984in}{0.755447in}}%
\pgfpathlineto{\pgfqpoint{1.796245in}{0.738409in}}%
\pgfpathlineto{\pgfqpoint{1.814506in}{0.720512in}}%
\pgfpathlineto{\pgfqpoint{1.832767in}{0.701745in}}%
\pgfpathlineto{\pgfqpoint{1.851029in}{0.682098in}}%
\pgfpathlineto{\pgfqpoint{1.869290in}{0.661560in}}%
\pgfpathlineto{\pgfqpoint{1.887551in}{0.640121in}}%
\pgfpathlineto{\pgfqpoint{1.905812in}{0.617770in}}%
\pgfpathlineto{\pgfqpoint{1.924073in}{0.594496in}}%
\pgfpathlineto{\pgfqpoint{1.942334in}{0.570289in}}%
\pgfpathlineto{\pgfqpoint{1.960595in}{0.545138in}}%
\pgfpathlineto{\pgfqpoint{1.978857in}{0.519033in}}%
\pgfpathlineto{\pgfqpoint{1.997118in}{0.491962in}}%
\pgfpathlineto{\pgfqpoint{2.015379in}{0.463917in}}%
\pgfpathlineto{\pgfqpoint{2.033640in}{0.434884in}}%
\pgfpathlineto{\pgfqpoint{2.051901in}{0.404855in}}%
\pgfpathlineto{\pgfqpoint{2.070162in}{0.373819in}}%
\pgfpathlineto{\pgfqpoint{2.088424in}{0.341764in}}%
\pgfpathlineto{\pgfqpoint{2.106685in}{0.308681in}}%
\pgfpathlineto{\pgfqpoint{2.124946in}{0.274559in}}%
\pgfusepath{stroke}%
\end{pgfscope}%
\begin{pgfscope}%
\pgfpathrectangle{\pgfqpoint{0.317092in}{0.274559in}}{\pgfqpoint{1.898247in}{0.792104in}}%
\pgfusepath{clip}%
\pgfsetbuttcap%
\pgfsetroundjoin%
\pgfsetlinewidth{0.501875pt}%
\definecolor{currentstroke}{rgb}{0.501961,0.501961,0.501961}%
\pgfsetstrokecolor{currentstroke}%
\pgfsetdash{{1.850000pt}{0.800000pt}}{0.000000pt}%
\pgfpathmoveto{\pgfqpoint{0.317092in}{0.934646in}}%
\pgfpathlineto{\pgfqpoint{1.360857in}{0.934646in}}%
\pgfusepath{stroke}%
\end{pgfscope}%
\begin{pgfscope}%
\pgfpathrectangle{\pgfqpoint{0.317092in}{0.274559in}}{\pgfqpoint{1.898247in}{0.792104in}}%
\pgfusepath{clip}%
\pgfsetbuttcap%
\pgfsetroundjoin%
\pgfsetlinewidth{0.501875pt}%
\definecolor{currentstroke}{rgb}{0.501961,0.501961,0.501961}%
\pgfsetstrokecolor{currentstroke}%
\pgfsetdash{{1.850000pt}{0.800000pt}}{0.000000pt}%
\pgfpathmoveto{\pgfqpoint{1.360902in}{0.274559in}}%
\pgfpathlineto{\pgfqpoint{1.360902in}{0.934646in}}%
\pgfusepath{stroke}%
\end{pgfscope}%
\begin{pgfscope}%
\pgfsetbuttcap%
\pgfsetmiterjoin%
\definecolor{currentfill}{rgb}{0.000000,0.000000,0.000000}%
\pgfsetfillcolor{currentfill}%
\pgfsetlinewidth{1.003750pt}%
\definecolor{currentstroke}{rgb}{0.000000,0.000000,0.000000}%
\pgfsetstrokecolor{currentstroke}%
\pgfsetdash{}{0pt}%
\pgfsys@defobject{currentmarker}{\pgfqpoint{-0.041667in}{-0.041667in}}{\pgfqpoint{0.041667in}{0.041667in}}{%
\pgfpathmoveto{\pgfqpoint{0.041667in}{-0.000000in}}%
\pgfpathlineto{\pgfqpoint{-0.041667in}{0.041667in}}%
\pgfpathlineto{\pgfqpoint{-0.041667in}{-0.041667in}}%
\pgfpathlineto{\pgfqpoint{0.041667in}{-0.000000in}}%
\pgfpathclose%
\pgfusepath{stroke,fill}%
}%
\begin{pgfscope}%
\pgfsys@transformshift{2.215339in}{0.274559in}%
\pgfsys@useobject{currentmarker}{}%
\end{pgfscope}%
\end{pgfscope}%
\begin{pgfscope}%
\pgfsetbuttcap%
\pgfsetmiterjoin%
\definecolor{currentfill}{rgb}{0.000000,0.000000,0.000000}%
\pgfsetfillcolor{currentfill}%
\pgfsetlinewidth{1.003750pt}%
\definecolor{currentstroke}{rgb}{0.000000,0.000000,0.000000}%
\pgfsetstrokecolor{currentstroke}%
\pgfsetdash{}{0pt}%
\pgfsys@defobject{currentmarker}{\pgfqpoint{-0.041667in}{-0.041667in}}{\pgfqpoint{0.041667in}{0.041667in}}{%
\pgfpathmoveto{\pgfqpoint{0.000000in}{0.041667in}}%
\pgfpathlineto{\pgfqpoint{-0.041667in}{-0.041667in}}%
\pgfpathlineto{\pgfqpoint{0.041667in}{-0.041667in}}%
\pgfpathlineto{\pgfqpoint{0.000000in}{0.041667in}}%
\pgfpathclose%
\pgfusepath{stroke,fill}%
}%
\begin{pgfscope}%
\pgfsys@transformshift{0.317092in}{1.066663in}%
\pgfsys@useobject{currentmarker}{}%
\end{pgfscope}%
\end{pgfscope}%
\begin{pgfscope}%
\pgfsetrectcap%
\pgfsetmiterjoin%
\pgfsetlinewidth{0.803000pt}%
\definecolor{currentstroke}{rgb}{0.000000,0.000000,0.000000}%
\pgfsetstrokecolor{currentstroke}%
\pgfsetdash{}{0pt}%
\pgfpathmoveto{\pgfqpoint{0.317092in}{0.274559in}}%
\pgfpathlineto{\pgfqpoint{0.317092in}{1.066663in}}%
\pgfusepath{stroke}%
\end{pgfscope}%
\begin{pgfscope}%
\pgfsetrectcap%
\pgfsetmiterjoin%
\pgfsetlinewidth{0.803000pt}%
\definecolor{currentstroke}{rgb}{0.000000,0.000000,0.000000}%
\pgfsetstrokecolor{currentstroke}%
\pgfsetdash{}{0pt}%
\pgfpathmoveto{\pgfqpoint{0.317092in}{0.274559in}}%
\pgfpathlineto{\pgfqpoint{2.215339in}{0.274559in}}%
\pgfusepath{stroke}%
\end{pgfscope}%
\end{pgfpicture}%
\makeatother%
\endgroup%

	\end{subfigure}
	\caption{\textbf{Breakdown of the order parameter with larger \(q\) and superconducting current in Ginzburg-Landau theory}. \textbf{(\subref{sfig:Ginzburg Landau OP vs q})}. \textbf{(\subref{sfig:Ginzburg Landau current vs q})}}
	\label{fig:Ginzburg Landau OP and current vs q}
\end{figure}
The depairing current is an upper boundary for the maximal current that can flow through a material, also called the critical current \(\vb{j}_c\).
The value of \(\vb{j}_c\) is strongly dependent on the geometry of the sample \cite{bardeenCriticalFieldsCurrents1962, xuAchievingTheoreticalDepairing2010}, so \(\vb{j}_{dp}\) is not necessarily experimentally available, but it can be used to calculate the London penetration depth \cite{tinkhamIntroductionSuperconductivity1996}
\begin{equation}
	\lambda_L (T) = \sqrt{\frac{\Phi_0}{3 \sqrt{3} \pi \mu_0 \xi(T) j_{\mathrm{dp}} (T)}} = \lambda_{L,0} \left( 1 - \left( \frac{T}{T_{\mathrm{C}}}\right)^4 \right)^{-\frac{1}{2}} 
\end{equation}
\todo{Where does this formula come from? Second London equation}
The superfluid phase stiffness can then be calculated via
\begin{equation}
	D_S \propto \lambda_L^{-2}
\end{equation}
The finite-momentum method in the limit of \(\vb{q} \to 0\) is related to linear response techniques to calculate the superfluid weight \cite{peottaSuperfluidityTopologicallyNontrivial2015, liangBandGeometryBerry2017}.

\section{Bardeen-Coooper-Schrieffer Theory}\label{sec:bcs-theory}

The \glsxtrfull{bcs} description of superconductivity describes superconductivity as the condensation of electrons into pairs that form a macroscopic quantum state.
There exist many textbooks tackling BCS theory from different angles, such as refs. \cite{colemanIntroductionManyBodyPhysics2015, tinkhamIntroductionSuperconductivity1996}.
This section gives an introduction to the relevant physics of \glsxtrshort{bcs} theory as originally proposed, then derives (BCS) mean-field theory for the multiband Hubbard model.

\todo{Better introduction}

\subsection*{BCS Hamiltonian}

\todo{Work over paragraph}

BCS-Hamiltonian:
\begin{equation}\label{eq:BCS Hamiltonian}
	H_{\text{BCS}} = \sum_{\vb{k}\sigma} \epsilon_{\vb{k}\sigma} c_{\vb{k}\sigma}^{\dagger} c_{\vb{k}\sigma} + \sum_{\vb{k}, \vb{k}^{\prime}} V_{\vb{k}, \vb{k}^{\prime}} c_{\vb{k}\uparrow}^{\dagger} c_{-\vb{k}\downarrow}^{\dagger} c_{-\vb{k}^{\prime}\downarrow} c_{\vb{k}^{\prime}\uparrow}
\end{equation}
This Hamiltonian can be solved exactly using a mean field approach, because it involves an interaction at zero momentum and thus infinite range.
Order parameter in mean field BCS theory is the pairing amplitude
\begin{equation}
	\Delta = - \frac{U}{N_{\vb{k}}} \sum_{\vb{k}} \braket{c_{-\vb{k} \downarrow} c_{\vb{k} \uparrow}} = - U \braket{c_{-\vb{r}=0 \downarrow} c_{\vb{r=0} \uparrow}} \simeq U \Psi \;.
\end{equation}

A finite \(\Delta\) corresponds to the pairing introduced above: there is a finite expectation value for a coherent creation/annihilation of a pair of electrons with opposite momentum and spin.
A finite \(\Delta\) also introduces a band gap into the spectrum.
BCS theory brings multiple aspects together: concept of paired electrons with the pairing amplitude being the order parameter in SC, an explanation for the attractive interaction overcoming Coulomb repulsion and a model Hamiltonian that very elegantly captures the essential physics.
In particular, the model Hamiltonian can be expanded with other types of pairing interactions to give a picture of superconductivity in the cuprates, compare ch. 15 in \cite{colemanIntroductionManyBodyPhysics2015}.

BCS theory is very successful in two ways: on the one hand it could quantitatively predict effects in the SCs known at the time, for example the Hebel-Slichter peak that was measured in 1957 \cite{hebelNuclearRelaxationSuperconducting1957, hebelNuclearSpinRelaxation1959} and the band gap measured by Giaever in 1960 \cite{giaeverStudySuperconductorsElectron1961}.  
On the other hand, it established electronic pairing, i.e. the picture of a quantum-mechanical wave function with a defined phase as already described by Fritz London in 1937 \cite{londonNewConceptionSupraconductivity1937} as the microscopic mechanism behind SC.
This picture still holds today even for high \(T_C\)/unconventional superconductors, so SCs that cannot be described by BCS theory \cite{zhouHightemperatureSuperconductivity2021}.

\subsection*{Multiband BCS Theory}

\todo{Write chemical potential properly}
The Hubbard model is the simplest model for interacting electron systems.
It goes back to works by Hubbard \cite{hubbardElectronCorrelationsNarrow1963}, Kanamori \cite{kanamoriElectronCorrelationFerromagnetism1963} and Gutzweiler \cite{gutzwillerEffectCorrelationFerromagnetism1963} in the 1960s.
The Hamiltonian of the  single-band Hubbard model is
\begin{equation}
	H = H_0 + H_{\mathrm{int}} 
	= \sum_{\langle i j \rangle \sigma} \left(-t_{i, j} - \mu_{\sigma} \delta_{i, j}\right) c_{i, \sigma}^{\dagger} c_{j, \sigma} + \mathrm{h.c.} + U \sum_{i} c_{i, \uparrow}^{\dagger} c_{i, \downarrow}^{\dagger} c_{i, \downarrow} c_{i, \uparrow}
	\label{eq:Hubbard interaction repulsive}
\end{equation}
where \(U > 0\).
The interaction describes a repulsive interaction between electrons of different spin at the same lattice site.

The Hubbard model emphasizes the electronic correlations due to local interactions, but with the discovery of high \(T_C\) SC in the Cuprates, it was quickly realized that the 2D Hubbard model in the intermediate to strong-coupling regime could describe the \ce{CuO2} layers \cite{zhangEffectiveHamiltonianSuperconducting1988} well.
The Hubbard model has parameter regimes with i.e. \(d_{x^2 - y^2}\) superconductivity, strong antiferromagnetic correlations, stripes, pseudogaps, Fermi liquid, and bad metallic behavior, with the phase diagram lines and observables being similar as a function of doping and temperature. 
Besides the relevancy of the Hubbard model for the Cuprates, the character of the model as having few parameters and simultaneously a very rich phase diagram with a variety of many-body effects also made it a perfect playground for new numerical tools, among them diagonalization, diagrammatics, tensor network, Quantum Monte Carlo (QMC) methods and DMFT (see \cref{sec:Dynamical Mean-Field Theory}) \cite{qinHubbardModelComputational2022}.

The Hubbard model  in the form of \cref{eq:Hubbard interaction repulsive} can be extended in a multitude of ways to model a variety of physical system.
Here: extension to multiple orbitals (i.e. atoms in the unit cell for lattice systems) and an attractive interaction, i.e. a negative \(U\).
Physical motivation for taking a negative-U Hubbard model: electrons can experience a local attraction interaction, for example through electrons coupling with phononic degrees of freedom or with electronic excitations \cite{micnasSuperconductivityNarrowbandSystems1990}.
The form of the interaction term is then:
\begin{equation}
	H_{\mathrm{int}} = -\sum_{i, \alpha} U_{\alpha} c_{i, \alpha, \uparrow}^{\dagger} c_{i, \alpha, \downarrow}^{\dagger} c_{i, \alpha, \downarrow} c_{i, \alpha, \uparrow}
	\label{eq:Hubbard interaction multiband}
\end{equation}
where \(\alpha\) counts orbitals and the minus sign in front is taken so that \(U > 0\) now corresponds to an attractive interaction (this is purely convention).

There are a multitude of ways to derive a mean field description of a given interacting Hamiltonian.
Very rigorous in path integral formulations as saddle points, given for example in ref. \cite{colemanIntroductionManyBodyPhysics2015}.
The review follows ref. \cite{huhtinenSuperconductivityNormalState2023}.
A more intuitive way based on ref. \cite{bruusManyBodyQuantumTheory2004} discussed here looks at the operators and which one are small. 

Look at interaction term \cref{eq:Hubbard interaction multiband}.
Mean-field approximation: operators \(A\) do not deviate much from their average value \(\braket{A}\), i.e. the deviation \(\fdif{A} = A - \braket{A}\) are small.
Specifically for superconductivity i.e. pairing, the operators
\begin{align}
	d_{i, \alpha} = c_{i, \alpha, \uparrow}^{\dagger} c_{i, \alpha, \downarrow}^{\dagger} - \braket{c_{i, \alpha, \uparrow}^{\dagger} c_{i, \alpha, \downarrow}^{\dagger}} \\
	e_{i, \alpha} = c_{i, \alpha, \downarrow} c_{i, \alpha, \uparrow} - \braket{c_{i, \alpha, \downarrow} c_{i, \alpha, \uparrow}}
\end{align}
are small (dont contribute much to expectation values and correlation functions).
So that in the interaction part of the Hamiltonian
\begin{align}
	H_{\mathrm{int}} = &-\sum_{i, \alpha} U_{\alpha} c_{i, \alpha, \uparrow}^{\dagger} c_{i, \alpha, \downarrow}^{\dagger} c_{i, \alpha, \downarrow} c_{i, \alpha, \uparrow} \\
	= &-\sum_{i, \alpha} U_{\alpha} 
	\left( d_{i, \alpha}^{\dagger} + \braket{c_{i, \alpha, \uparrow}^{\dagger} c_{i, \alpha, \downarrow}^{\dagger}} \right)
	\left( e_{i, \alpha} + \braket{c_{i, \alpha, \downarrow} c_{i, \alpha, \uparrow}} \right) \\
	= &-\sum_{i, \alpha} U_{\alpha} (
		d_{i, \alpha} e_{i, \alpha}
		+ d_{i, \alpha} \braket{c_{i, \alpha, \downarrow} c_{i, \alpha, \uparrow}} \nonumber \\
		&+ e_{i, \alpha} \braket{c_{i, \alpha, \uparrow}^{\dagger} c_{i, \alpha, \downarrow}^{\dagger}}
		+ \braket{c_{i, \alpha, \uparrow}^{\dagger} c_{i, \alpha, \downarrow}^{\dagger}} \braket{c_{i, \alpha, \downarrow} c_{i, \alpha, \uparrow}})
\end{align}
the first term is quadratic in the deviation and can be neglected.
Thus arrive at the approximation
\begin{align}
	H_{\mathrm{int}} \approx &-\sum_{i, \alpha} U_{\alpha} \left(
	d_{i, \alpha} \braket{c_{i, \alpha, \downarrow} c_{i, \alpha, \uparrow}}
	+ e_{i, \alpha} \braket{c_{i, \alpha, \uparrow}^{\dagger} c_{i, \alpha, \downarrow}^{\dagger}} \right. \nonumber \\
	&\left.+ \braket{c_{i, \alpha, \uparrow}^{\dagger} c_{i, \alpha, \downarrow}^{\dagger}} \braket{c_{i, \alpha, \downarrow} c_{i, \alpha, \uparrow}}
	\right) \\
	= &-\sum_{i, \alpha} U_{\alpha} \left(
		c_{i, \alpha, \uparrow}^{\dagger} c_{i, \alpha, \downarrow}^{\dagger} \braket{c_{i, \alpha, \downarrow} c_{i, \alpha, \uparrow}}
		+ c_{i, \alpha, \downarrow} c_{i, \alpha, \uparrow} \braket{c_{i, \alpha, \uparrow}^{\dagger} c_{i, \alpha, \downarrow}^{\dagger}} \right. \nonumber \\
	&\left.- \braket{c_{i, \alpha, \uparrow}^{\dagger} c_{i, \alpha, \downarrow}^{\dagger}} \braket{c_{i, \alpha, \downarrow} c_{i, \alpha, \uparrow}} \right) \\
	= &\sum_{i, \alpha} \left(\Delta_{i, \alpha} c_{i, \alpha, \uparrow}^{\dagger} c_{i, \alpha, \downarrow}^{\dagger} + \Delta_{i, \alpha}^{*} c_{i, \alpha, \downarrow} c_{i, \alpha, \uparrow} - \frac{\vert \Delta_{i, \alpha} \vert^2}{U_{\alpha}}\right)
\end{align}
with the expectation value
\begin{align}
	\Delta_{i, \alpha} = -U_{\alpha} \braket{c_{i, \alpha, \downarrow} c_{i, \alpha, \uparrow}}
\end{align}
which is called the superconducting gap and is the order parameter introduced in Ginzburg-Landau theory in \cref{sec:Ginzburg-Landau theory of superconductivity}.
This results in the mean-field Hamiltonian \todo{MF Hamiltonian, with chemical potential}
\begin{equation}
	H_{\mathrm{MF}} = \sum_{\langle i j \rangle \sigma} \left(-t_{i, j} - \mu_{\sigma} \delta_{i, j}\right) c_{i, \sigma}^{\dagger} c_{j, \sigma} + \mathrm{h.c.} + U \sum_{i} c_{i, \uparrow}^{\dagger} c_{i, \downarrow}^{\dagger} c_{i, \downarrow} c_{i, \uparrow}
\end{equation}

To include finite momentum in BCS theory, take the ansatz of a Fulde-Ferrel (FF) type pairing \cite{kinnunenFuldeFerrellLarkin2018}:
\begin{equation}
	\Delta_{i, \alpha} = \Delta_{\alpha} e^{\iu \vb{q} \vb{r}_{i \alpha}}
\end{equation}
Using the Fourier transform (with position vectors \(\vb{r}_{i \alpha} = \vb{R}_i + \delta_{\alpha}\), position of the unit cell \(\vb{R}_i\) and position of the orbital inside the unit cell \(\vb{\delta}_{\alpha}\))
\begin{equation}
	c_{i \alpha \sigma} = \frac{1}{\sqrt{N}} \sum_{\vb{k}} e^{\iu \vb{k} \vb{r}_{i \alpha}} c_{\vb{k} \alpha \sigma}
\end{equation}
can write mean-field Hamiltonian as
\begin{align}
	H_{\mathrm{MF}} (\vb{q}) &= \sum_{\vb{k}} \vb{C}_{\vb{q}, \vb{k}}^{\dagger} H_{\mathrm{BdG}} (\vb{q}, \vb{k}) \vb{C}_{\vb{q}, \vb{k}} + K_{\vb{q}} \\
	C_{\vb{q}, \vb{k}} &= 
		\begin{pmatrix}
			c_{\vb{k} 1 \uparrow} & 
			c_{\vb{k} 2 \uparrow} &
			\ldots &
			c_{\vb{k} n_{\mathrm{orb}} \uparrow} &
			c_{\vb{q} -\vb{k} 1 \downarrow}^{\dagger} &
			c_{\vb{q} -\vb{k} 2 \downarrow}^{\dagger} &
			\ldots &
			c_{\vb{q} -\vb{k} n_{\mathrm{orb}} \downarrow}^{\dagger}
		\end{pmatrix}^{T} \\
		K_{\vb{q}} &= \sum_{\vb{k}} \Tr [H_{\vb{k}}^{\downarrow}] - n_{\mathrm{orb}} N \mu - N \sum_{\alpha} \frac{\vert \Delta_{\alpha} (\vb{q}) \vert^2}{U}
\end{align}
with the so-called \acrfull{bdg} matrix
\begin{equation}
	H_{\mathrm{BdG}} (\vb{k}) =
	\begin{pmatrix}
		H^{\uparrow}_{\vb{q} + \vb{k}} - \mu & \Delta (\vb{q}) \\
		\Delta^{\dagger} (\vb{q}) & - \left(H^{\downarrow}_{\vb{q} - \vb{k}}\right)^* + \mu
	\end{pmatrix}
\end{equation}
with \(H_{0, \sigma}\) being the F.T. of the kinetic term \todo{F.T. of kinetic term} and
\begin{equation}
	\Delta = \diag(\Delta_1 (\vb{q}), \Delta_2 (\vb{q}), \ldots, \Delta_{n_{\mathrm{orb}}} (\vb{q}))\;.
\end{equation}
For time-reversal symmetric systems, there exists a solution s.t. all \(\Delta_{\alpha}\) are real \cite{peottaSuperfluidityTopologicallyNontrivial2015}.
The introduction of the \(\vb{q}\) breaks time-reversal symmetry, s.t. in a multiband system, the order parameters in the orbital can develop different phases.

Problem is  now reduced to diagonalization of the \acrshort{bdg} matrix.
Write
\begin{equation}
	H_{\mathrm{BdG}} = U_{\vb{q}, \vb{k}} \epsilon_{\vb{q}, \vb{k}} U_{\vb{q}, \vb{k}}^{\dagger}
\end{equation}
and 
\begin{equation}
	H_{\mathrm{MF}} = \sum_{\vb{q}, \vb{k}} \gamma_{\vb{q}, \vb{k}} \epsilon_{\vb{q}, \vb{k}} \gamma_{\vb{q}, \vb{k}}^{\dagger}
\end{equation}
with quasi-particle operators
\begin{equation}
	\gamma_{\vb{q}, \vb{k}} = U_{\vb{q}, \vb{k}}^{\dagger} C_{\vb{q}, \vb{k}}
\end{equation}

\todo{Write indeces everywhere without comma}

Using the gap equation
\todo{gap equation}
\begin{align}
	\Delta_{\alpha} &= -U_{\alpha} \braket{c_{i, \alpha, \downarrow} c_{i, \alpha, \uparrow}} = -\frac{U}{N} \sum_{\vb{k}} \braket{c_{\vb{q}+\vb{k}}} \nonumber \\
	&= -\frac{U}{N} \\
	&= -\frac{U}{N} \nonumber
\end{align}
the order parameter can be determined self-consistently, i.e. starting from an initial value, the BdG matrix needs to be set up, diagonalized and then used to determine \(\Delta_{\alpha}\) again, until a converged value is found.

\todo{SC current in BCS}

\section{The BCS-BEC Crossover}\label{sec:BCS-BEC crossover}

One of the challenges in achieving high-temperature superconductivity is the fact that the two intrinsic energy scales of superconductors i.e. the pairing amplitude and the phase coherence often compete.
Can be seen in the phenomenon of BCS-BEC crossover physics \cite{chenWhenSuperconductivityCrosses2024}.
The picture of this crossover is the following: for a small attractive interaction, pairs of electrons are very loosely bound and mobile, while for a stronger interaction the pairs are bound together stronger and are not mobile, because hopping of a pair would involve a virtual hopping, thus breaking up the pair.
This is highly suppressed.
The crossover region between these two regimes is the BCS-BEC crossover.
The energy scales characterizing the crossover are the superconducting gap/order parameter describing how strong the order is and the superfluid weight describing how mobile the Cooper pairs are.
These two energy scales are equivalently defined via the coherence length \(\xi_0\) and the London penetration depth introduced in \cref{sub:Landau and Ginzburg-Landau Theory}.

\todo{Section about BCS-BEC crossover}



\section{Dynamical Mean-Field Theory}\label{sec:Dynamical Mean-Field Theory}

The foundational idea of \acrfull{dmft} is to map the full interacting problem to the problem of a single lattice site (or a small cluster of lattice sites) embedded in a mean field encompassing all non-local correlation effects, as seen in \cref{fig:DMFT AIM mapping}.

This 

\todo{What is the basic idea of DMFT?}

DMFT has

\todo{What has been achieved with DMFT}
	
This section describes the method of Green's function, which is the language DMFT is formulated in, the mapping of the lattice problem onto the impurity problem and the resulting self-consistency loop of DMFT.
Additionally, I will also briefly describe how to describe the superconducting state in terms of Green's function and the consequences for a DMFT implementation.
I will not fully derive the equations of DMFT here, for a more (pedagogical) introduction see refs. \cite{pavariniDynamicalMeanfieldTheory2022, georgesDynamicalMeanfieldTheory1996, colemanIntroductionManyBodyPhysics2015, bruusManyBodyQuantumTheory2004}.

\begin{figure}[t]
	\centering
	\import{images/}{AIM mapping.pdf_tex}
	\caption{\textbf{Mapping of the full lattice problem .} This also }
	\label{fig:DMFT AIM mapping}
\end{figure}

\subsection*{Green's Function Formalism}

Green's functions: method to encode influence of many-body effects on propagation of particles in a system.
Depending on the context, different kinds of Green's functions are employed.
For exampple, Matsubara Green's functions naturally includes finite temperatures.
This is done via a so-called Wick rotation of the time variable \(t\) into imaginary time
\begin{equation}
	t \to -\iu \tau
\end{equation}
where \(\tau\) is real and has the dimension time.
This enables the simultaneous expansion of exponential \(e^{-\beta H}\) coming from the thermodynamic average and \(e^{-\iu H t}\) coming from the time evolution of operators.

Matsubara Greens function are defined as
\begin{equation}
	G_{\alpha_1 \alpha_2} (\tau_1, \tau_2) = - \Braket{T_{\tau} (c_{\alpha_1}(\tau_1) c^{\dagger}_{\alpha_2} (\tau_2))}
\end{equation}
with:
\begin{itemize}
	\item \(c_{\alpha}, c^{\dagger}_{\alpha}\) fermionic creation/annihilation operators of quantum states \(\alpha\) in the Heisenberg time-evolution picture \(\ope{A} (\tau) = e^{\tau H} \ope{A} e^{-\tau H}\)
	\item \(\braket{\cdot} = \Tr(\ope{rho} \cdot)\) the thermal expectation value with statistical operator \(\ope{\rho} = \nicefrac{e^{-\beta H}}{Z}\) with partition function \(Z\) and Hamiltonian \(H\)
	\item Time-ordering operator
	\begin{align}
		T_{\tau} (A(\tau) B(\tau^{\prime})) &= \Theta(\tau - \tau^{\prime}) A(\tau) B(\tau^{\prime}) \pm \Theta(\tau^{\prime} - \tau) B(\tau^{\prime}) A(\tau) \\
		&= \begin{cases}
			A (\tau) B(\tau^{\prime}) & \text{if } \tau < \tau^{\prime} \\
			B(\tau^{\prime}) A (\tau) & \text{if } \tau^{\prime} < \tau
		\end{cases}
	\end{align}
\end{itemize}
Thermal equilibrium: Green's function only depends on time differences \(\tau_1 - \tau_2\), so can work with a single time \(\tau = \tau_1 - \tau_2\), shifting \(\tau_2 = 0\).
Fermionic Matsubara Green's functions are antiperiodic in time with periodicity \(\beta\). For \(-\beta < 0 < 0\), cyclic properties of trace tells:
\begin{equation}
	G (\tau) = - G (\tau + \beta)
\end{equation}
Restrict to interval \(0 < \tau < \beta\).
This means that there is a Fourier expansion with discrete frequencies \(\omega_n = \nicefrac{(2n + 1)\pi}{\beta}\):
\begin{equation}
	G (\iu \omega_n) = \int_{0}^{\beta} \odif{\tau} G (\tau) e^{\iu \omega_n \tau}
\end{equation}
and
\begin{equation}
	G (\tau) = \frac{1}{\beta} \sum_{n} G(\iu \omega_n) e^{-\iu \omega_n \tau}
\end{equation}

Spatial dependence \(\alpha = \vb{R}\).
In lattice context (translationally invariant, \(G\) only depends on \(\vb{R}_1 - \vb{R}_2\)), can transform between real-space and crystal momentum representation as 
\begin{equation}
	G (\vb{k}, \tau) = \sum_i G(\vb{R}_1) e^{\iu \vb{k} \vb{R}_j}
\end{equation}
and
\begin{equation}
	G(\vb{r}) = \frac{1}{N_{\vb{k}}} \sum_{\vb{k}} G(\vb{k}) e^{-\iu \vb{k} \vb{r}}
\end{equation}

\todo{Connection to experimental observables}
Can get the retarded GF \(G_{AB}^R (\omega)\) by analytic continuation:
\begin{equation}
	G_{AB}^R (\omega) = G_{AB} (\iu \omega_n \to \omega + \iu \eta)
\end{equation}
\todo{What is the eta there?}

\todo{Spectral representation of Matsubara and retarded GF}

\todo{How to get real frequency information from Matsubara GF?}
%So in particular the extrapolation of the Matsubara GF to zero is proportional to the density of states at the chemical potential.
%Gapped: density is zero (Matsubara GF goes to 0), metal: density is finite (Matsubara GF goes to finite value).

\subsection*{Dyson Equation}

\todo{Explain non-interacting GF}

\todo{Explain self-energy}

Dyson equation:
\begin{equation}
	\mathcal{G}_{\sigma} (\vb{k}, \iu \omega_n) = \frac{\mathcal{G}_{\sigma}^0 (\vb{k}, \iu \omega_n)}{1 - \mathcal{G}_{\sigma}^0 (\vb{k}, \iu \omega_n) \Sigma_{\sigma} (\vb{k}, \iu \omega_n)} = \frac{1}{\iu \omega_n - \xi_{\vb{k} - \Sigma_{\sigma} (\vb{k}, \iu \omega_n)}}
\end{equation}

\todo{Explanation of self-energy as tool of approximations}

\subsection*{Anderson Impurity Model}

\subsection*{DMFT}

\subsection*{Nambu-Gorkov Green's Functions}

To describe superconductivity, 


\todo{More general introduction into NG GFs, how they look like, what they describe etc.}

Order parameter can be chosen as the anomalous GF:
\begin{equation}
	\Psi = F^{\mathrm{loc}} (\tau = 0^-)
\end{equation}

\todo{DMFT with NG GFs}


\section{Quantum Metric}\label{sec:quantum-metric}

Topic in quantum materials: quantum geometry and its influence on a many (quantum) material properties \cite{yuQuantumGeometryQuantum2024}.
First (?) example: the Integer quantum Hall effect \cite{klitzingNewMethodHighAccuracy1980} that was explained by \citeauthor{thoulessQuantizedHallConductance1982} to be a consequence of the unique topology of the ground state of the electron \cite{thoulessQuantizedHallConductance1982}.

Concept of quantum geometry first formulated in 1980 by Provost and Vallee \cite{provostRiemannianStructureManifolds1980}.

Parameter dependent Hamiltonian \(\{H(\lambda)\}\), smooth dependence on parameter \(\lambda = (\lambda_1, \lambda_2, \ldots) \in \mathcal{M}\) (base manifold)

Hamiltonian acts on parametrized Hilbert space \(\mathcal{H} (\lambda)\)

Eigenenergies \(E_n (\lambda)\), eigenstates \(\ket{\phi_n (\lambda)}\)

System state \(\ket{\psi (\lambda)}\) is linear combination of \(\ket{\psi_n (\lambda)}\) at every point in \(\mathcal{M}\)

Infinitesimal variation of the parameter \(\odif{\lambda}\) \todo{Dont get it here}:
\begin{equation}
	\odif{s^2} = \vert\vert \psi (\lambda + \odif{\lambda}) - \si (\lambda) \vert \vert^2 = \braket{\fdif{\psi} | \fdif{\psi}} = \braket{\pdif{\mu} \psi | \pdif{\nu} \psi} \odif{\lambda^{\mu}} \odif{\lambda^{\nu}} = (\gamma_{\mu \nu} + \iu \sigma_{\mu \nu}) \odif{\lambda^{\mu}} \odif{\lambda^{\nu}}
\end{equation}
Last part is splitting up into real and imaginary part

Recently, the Quantum Geometric Tensor (and in turn the quantum metric) was measured \cite{kangMeasurementsQuantumGeometric2025}.

\subsection*{Quantum Metric and Superfluid Weight}

In the context of superconductivity: 

\cite{peottaSuperfluidityTopologicallyNontrivial2015, liangBandGeometryBerry2017, tormaSuperconductivitySuperfluidityQuantum2022}

\todo{Write up notes about quantum metric and superfluid weight}

\end{document}
