\documentclass[../main.tex]{subfiles}
\graphicspath{{\subfix{../images/}}, {\subfix{../}}}

\begin{document}

\chapter{Superconductivity}

This chapter gives an introduction to the phenomenology and theory of superconductivity.
Superconductivity describes the phenomenon of the electrical resistance of a metal suddenly dropping to zero below a critical temperature.

At the beginning of the 20th century, 

\todo{More details on history}

Discovered in mercury in 1911 by Heike Onnes \cite{onnes1911further}.

\todo{Meissner effect, leads to supercurrent}

\todo{Some words: why is it interesting?}

\section{BCS Theory}

Following \cite[ch. 14]{colemanIntroductionManyBodyPhysics2015}.

Theoretical description of SC: 1956 by Bardeen, his postdoc Leon Cooper and the graduate in the group, J. Robert Schrieffer.
Description is based on the fact, that the Fermi sea is unstable towards development of bound pairs under arbitrarily small attraction \cite{cooperBoundElectronPairs1956}.
These bound electrons show bosonic behaviour and 
\todo{Why supercurrent?}

\todo{BCS hamiltonian, pairing}
This model Hamiltonian

\todo{Mean-field}
It can be connected to 

\todo{Phonon interaction}
The final element in this description was the origin of the attractive interaction between electrons, which Bardeen, Cooper and Schrieffer identified as a retarded electron-phonon interaction \cite{bardeenTheorySuperconductivity1957}.
This so-called BCS-theory of superconductivity is very successful in explaining experimental results in many compounds, 
Surprisingly, it
\todo{Mean field level can already explain a lot}

BCS-theory gave a microscopic explanation to a phenomenological description of superconductivity pioneered by Fritz London in 1937 \cite{londonNewConceptionSupraconductivity1937}.
This descriptions is based on a one-particle wavefunction \(\phi (x)\)

Later, this one-particle wa
vefunction was identified as the order parameter in the developing GL-theory of phase transitions \cite{terhaar73THEORYSUPERCONDUCTIVITY1965}.
GL-theory is discussed in more detail in \cref{ch:Ginzburg-Landau theory of superconductivity}.
This explains the Meissner effect and in turn the supercurrent.

\todo{More recent developments: strongly correlated superconductors}

\end{document}
