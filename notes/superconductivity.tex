\documentclass[../notes.tex]{subfiles}
\graphicspath{{\subfix{../images/}}, {\subfix{../}}}

\begin{document}

\chapter{Superconductivity}

In this chapter: review theoretical concepts needed for describing SC.

Macroscopially, SC state can be described by a spontaneous breaking of a \(U(1)\) phase rotation symmetry, that is associated with an order parameter. 
Theory of this: GL theory \cref{sec:Ginzburg-Landau theory of superconductivity}.

One tool to describe superconductivity from a microscopic perspective: BCS theory \cref{sec:bcs-theory}.

Taking fluctuations beyond mean field into account: DMFT \cref{sec:Dynamical Mean-Field Theory}.

There are many textbooks covering these topics which can be referenced for a more detailed treatment, such as refs. \cite{colemanIntroductionManyBodyPhysics2015, tinkhamIntroductionSuperconductivity1996, bruusManyBodyQuantumTheory2004, larkinTheoryFluctuationsSuperconductors2005}.

\section{Ginzburg-Landau Theory of Superconductivity}\label{sec:Ginzburg-Landau theory of superconductivity}

For this review, follow chapter 11 in ref.~\cite{colemanIntroductionManyBodyPhysics2015}.

\todo{More extensive introduction}

\paragraph{Order parameter} 

\todo{Work over paragraph}

Similarly to a magnetically ordered state, the SC state is characterized by 

\todo{Introduce spontaneous symmetry breaking}


\cite{landauTheoryPhaseTransitions1965, landauTheorySuperconductivity1965}

Such a symmetry breaking (e.g.\ iron becomes magnetic, water freezes, superfluidity/superconductivity) is associated with the development of an order parameter \(\Psi\) when the temperature drops below the transition temperature \(T_C\).
\begin{equation}
	\vert \Psi \vert =
	\begin{cases}
		0\;,\; T > T_C \\
		\vert \Psi_0 \vert > 0 \;,\; T < T_C
	\end{cases}
\end{equation}

\todo{Intuitive understanding why that is?}

Ginzburg-Landau theory is concerned with the the properties of the 

It does not need microscopic expression for order parameter, it provides corse-grained description of the properties of matter.
The order parameter description is good at length scales above \(\xi_0\), the coherence length (e.g.\ size of Cooper pairs for SC).
On length scales above \(\xi_0\), the order parameter behaves as a smoothly varying function.

\paragraph{Landau theory} \todo{Work over paragraph}

Basic idea of Landau theory: write free energy as function \(F[\psi]\) of the order parameter.
Region of small \(\psi\), expand free energy of many-body system as simple polynomial:
\begin{equation}
	f_{L} = \frac{1}{V} F[\psi] = \frac{r}{2} \psi^2 + \frac{u}{4} \psi^4
\end{equation}
Provided \(r\) and \(u\) are greater that \(0\): minimum of \(f_L [\psi])\) lies at \(\psi = 0\).
Landau theory assumes: at phase transition temperature \(r\) changes sign, so:
\begin{equation}
	r = a(T - T_C)
\end{equation}
Minimum of free energy occurs for:
\begin{equation}
	\psi = \begin{cases}
		0 \\
		\pm \sqrt{\frac{a (T_C - T)}{u} }
	\end{cases}
\end{equation}

\todo{Make graphic for Landau free energy}

\todo{Make graphic for Landau OP and BCS OP}

Two minima for free energy function for \(T < T_C\).
With this, we can extract \(T_C\) from the knowledge of the dependence of \(\vert \psi \vert^2\) on \(T\) via a linear fit.
This is only valid for an area near \(T_C\) (where Landau theory holds), but can be used to get \(T_C\) from microscopic theories.

Going from a one to a \(n\)-component order parameters, OP acquires directions and magnitude.
Particularly important example: complex or two component order parameter in superfluids and superconductors:
\begin{equation}
	\psi = \psi_1 + \iu \psi_2 = \vert \psi \vert e^{\iu \phi}
\end{equation}
The Landau free energy takes the form:
\begin{equation}
	f[\psi] = r(\psi^* \psi) + \frac{u}{2} (\psi^* \psi)^2
\end{equation}
As before:
\begin{equation}
	r = a(T - T_C)
\end{equation}

Figure~\ref{fig:Landau free energy mexican hat potential} shows the Landau free energy as function of \(\psi\).

\begin{figure}[t]
	\centering
	\includegraphics[width=0.3\textwidth]{images/landau free energy mexican hat}
	\caption{Mexican hat potential}
	\label{fig:Landau free energy mexican hat potential}
\end{figure}

\todo{Make my own graphic for mexican hat potential}

Rotational symmetry, because free energy is independent of the global phase of the OP:
\begin{equation}
	f [\psi] = f [e^{\iu a } \psi]
\end{equation}
In this `Mexican hat' potential: order parameter can be rotated continuously from one broken-symmetry state to another.
If we want the phase to be rigid, we need to introduce an
There is a topological argument for the fact that the phase is rigid.
This leads to Ginzburg-Landau theory.
Will see later: well-defined phase is associated with persistent currents or superflow.

\paragraph{Ginzburg-Landau theory} \todo{Work over paragraph}

Landau theory: energy cost of a uniform order parameter, more general theory needs to account for inhomogenous order parameters, in which the amplitude varies or direction of order parameter is twisted -> GL theory.
First: one-component, `Ising' order parameter.
GL introduces additional energy \(\delta f \propto \vert \Delta \psi \vert^2\), \(f_{GL} [\psi, \Delta \psi] = \frac{s}{2} \vert \Delta \psi \vert^2 + f_L [\psi(s)]\), or in full:
\begin{equation}
	f_{GL} [\psi, \Delta \psi, h] = \frac{s}{2} (\Delta \psi)^2 + \frac{r}{2} \psi^2 + \frac{u}{4} \psi^4
\end{equation}
GL theory is only valid near critical point, where OP is small enough to permit leading-order expansion.
Dimensional analysis shows: \(\frac{s}{r} = L^2\) has dimension of length squared.
Length scale introduced by the gradient term: correlation length
\begin{equation}
	\xi (T) = \sqrt{\frac{s}{\vert r(T) \vert}} = \xi_0 \left\vert 1 - \frac{T}{T_C} \right\vert^{-\frac{1}{2}}
\end{equation}
sets characteristic length scale of order-parameter fluctuations, where
\begin{equation}
	\xi_0 = \xi (T = 0) = \sqrt{\frac{s}{\alpha T_C}}
\end{equation}
is a measure of the microscopic coherence length.
Near transition \(\xi (T)\) diverges, but far from transition it becomes comparable with the coherence length.

\paragraph{Complex order and superflow} \todo{Work over paragraph}

Now: GL theory of complex or two-component order parameters, so superfluids and superconductors.
Heart of discussion: emergence of a `macroscopic wavefunction', where the microscopic field operators \(\hat{\psi}(x)\) acquire an expectation value:
\begin{equation}
	\braket{\hat{\psi} (x)} = \psi (x) = \vert \psi (x) \vert e^{\iu \theta(x)}
\end{equation}
Reminder: Field operators are the real space representations of creation/annihilation operators.
They can be thought of the super position of all ways of creating a particle at position \(x\) via the basis coefficients.

Magnitude determines density of particles in the superfluid:
\begin{equation}
	\vert \psi(x) \vert^2 = n_s (x)
\end{equation}
Density operator is
\begin{equation}
	\hat{\rho} = \hat{\psi} (x) \hat{\psi^{\dagger}} (x)
\end{equation}
so expectation value of that is the formula above.

Twist/gradient of phase determines superfluid velocity:
\begin{equation}
	\vb{v}_s (x) = \frac{\hbar}{m} \Delta \phi (x)
\end{equation}
We will derive this later in the chapter.
Counterintuitive from quantum mechanics: GL suggested that \(\Phi(x)\) is a macroscopic manifestation of a macroscopic number of particles condensed into precisely the same quantum state.
Emergent phenomenon, collective properties of mater not a-priori evident from microscopic physics.

GL free energy density for superfluid (with one added term in comparison to Landau energy):
\begin{equation}
	f_{GL} [\psi, \Delta \psi] = s \vert \Delta \psi \vert^2 + r \vert \psi \vert^2 + \frac{u}{2} \vert \psi \vert^4
\end{equation}
Compare with the energy density of a bosonic field (with a quarctic interaction):
\begin{align}
	H = \int \odif[order=D]{x} \frac{\hbar^2}{2m} \vert \Delta \psi \vert^2 + r \vert \psi \vert^2 + \frac{u}{2} \vert \psi \vert^4
\end{align}
Interpret GL free energy as energy density of a condensate of bosons in which the field operator behaves as a complex order parameter.
Gives interpretation of gradient term as kinetic energy:
\begin{equation}
	s \vert \Delta \psi \vert^2 = \frac{\hbar^2}{2m} \braket{\Delta \hat{\psi}^{\dagger} \Delta \hat{\psi}} \implies s = \frac{\hbar^2}{2m}
\end{equation}
As in Ising order: correlation length/GL-coherence length governs characteristic range of amplitude fluctuations of the order parameter:
\begin{equation}
	\xi = \sqrt{\frac{s}{\vert r \vert}} = \sqrt{\frac{\hbar^2}{2m \vert r \vert}} = \xi_0 (1 - \frac{T}{T_C})^{-\frac{1}{2}}
\end{equation}
where \(\xi_0 = \xi(T=0) = \sqrt{\frac{\hbar^2}{2 m a T_C}}\) is the coherence length.
Beyond this length scale: only phase fluctuations survive.
\todo{I dont know why that is. Can I support that somehow better? -> See Niklas thesis}

Freeze out fluctuations in amplitude (no \(x\)-dependence in amplitude) \(\psi(x) = \sqrt{n_s} e^{\iu \phi(x)}\), then \(\Delta \psi = \iu \Delta \phi \psi\) and \(\vert \Delta \psi \vert^2 = n_s (\Delta \phi)^2\), dependency of kinetic energy on the phase twist is (bringing it into the form \(\frac{m}{2} v^2\)):
\begin{equation}
	\frac{\hbar^2 n_s}{2m} (\Delta \phi)^2 = \frac{m n_s}{2} (\frac{\hbar}{m} \Delta \phi)^2
\end{equation}
So twist of phase results in increase in kinetic energy, associated with a superfluid velocity:
\begin{equation}
	\vb{v}_s = \frac{\hbar}{m} \Delta \phi
\end{equation}
(this is explained in detail later).

For interpretation of superfluid states: coherent states.
These are eigenstates of the field operator
\begin{equation}
	\hat{\psi}(x) \ket{\psi} = \psi (x) \ket{\psi} 
\end{equation}
and don't have a definite particle number.
Importantly, this small uncertainty in particle number enables a high degree of precision in phase (which is the property of a condensate).


\paragraph{Phase rigidity and superflow}

In GL theory, energy is sensitive to a twist of the phase.
Substitute \(\psi = \vert \psi \vert e^{\iu \phi}\) into GL free energy, gradient term is:
\begin{equation}
	\Delta \psi = (\Delta \vert \psi \vert + \iu \Delta \phi \vert \psi \vert) e^{\iu \phi}
\end{equation}
So:
\begin{equation}
	f_{GL}  = \frac{\hbar}{2m} \vert \psi \vert^2 (\Delta \phi)^2 + \left[ \frac{\hbar}{2m} (\Delta \vert \psi \vert)^2 + r \vert \psi \vert^2 + \frac{u}{2} \vert \psi \vert^4 \right]
\end{equation}
The second term resembles GL functional for an Ising order parameter, describes energy cost of variations in the magnitude of the order parameter.

\todo{Phase rigidity and superflow}


\subsection{Superconducting length scales}

From~\cite{wittBypassingLatticeBCSBEC2024}.

\todo{Better introduction}

In most materials: Cooper pairs do not carry finite center-of-mass momentum.
In presence of e.g.\ external fields or magnetism: SC states with FMP might arise.

Theory/procedure in the paper: enforce FMP states via constraints on pair-center-of-mass momentum \(\vb{q}\), access characteristic lenght scales \(\xi_0, \lambda_L\) through analysis of the momentum and temperature-dependent OP\@.
FF-type pairing with Cooper pairs carrying finite momentum:
\begin{equation}
	\psi_{\vb{q}} (\vb{r}) = \vert \psi_{\vb{q}} \vert e^{\iu \vb{q} \vb{r}}
\end{equation}
Then the free energy density is
\begin{equation}
	f_{GL} [\psi_{\vb{q}}] = \alpha \vert \psi_{\vb{q}} \vert^2 + \frac{b}{2} \vert \psi_{\vb{q}} \vert^4 + \frac{\hbar^2 q^2}{2 m^*} \vert \psi_{\vb{q}} \vert^2
\end{equation}
Stationary point of the system:
\begin{equation}
	\fdv{f_{GL}}{\psi_{\vb{q}}^*} = 2 \psi_{\vb{q}} \left[\alpha (1 - \xi^2 q^2) + b \vert \psi_{\vb{q}} \vert^2\right] = 0
\end{equation}
which results in the \(\vb{q}\)-dependence of the OP
\begin{equation}
	\vert \psi_{\vb{q}} \vert^2 = \vert \psi_{0} \vert^2 (1 - \xi(T)^2 q^2)
\end{equation}
For some value, SC order breaks down, \(\psi_{\vb{q}_c} = 0\), because the kinetic energy from phase modulation exceeds the gain in energy from pairing.
In GL theory: \(q_c = \xi(T)^{-1}\).
The temperature dependence of the OP and extracted \(\xi(T)\) gives access to the coherence length via
\begin{equation}
	\xi(T) = \xi_0 (1 - \frac{T}{T_C})^{-\frac{1}{2}}
\end{equation}
Specifically: take
\begin{equation}
	\xi(T) = \frac{1}{\sqrt{2} \vert \vb{Q} \vert}
\end{equation}
with \(\vb{Q}\) such that
\begin{equation}
	\vert \frac{\psi_{\vb{Q}}(T)}{\psi_0 (T)} \vert = \frac{1}{\sqrt{2}}
\end{equation}
The Cooper pair \cite{yuanSupercurrentDiodeEffect2022, shimanoHiggsModeSuperconductors2020}

\todo{Depairing current from FMP}

\todo{Full formula for supercurrent, with sum over orbitals}

\todo{DS from FMP}

\todo{Write more about the connection between all the things here}

\section{Bardeen-Coooper-Schrieffer Theory}\label{sec:bcs-theory}

First phenomenological description of SC: Fritz London in 1937 \cite{londonNewConceptionSupraconductivity1937}.
He was motivated by the discovery of the Meissner effect in 1933 \cite{meissnerNeuerEffektBei1933}, where magnetic flux inside of the superconductor is always pushed out in contrast to a perfectly conducting material, which would hold a `memory' of the magnetic field at the time of the phase transition. 
This suggests that transition to the SC state is reversible and a SC is not just the limiting case of a conductor with infinite conductivity, in which according to the Maxwell equations, the magnetic flux would not change.
Londons first descriptions is based on a one-particle wave function \(\phi (x)\).
He proposed that persistent supercurrent is a property of the ground state associated with its rigidity against the application of a field.

In 1950 \cite{landauTheorySuperconductivity1965}: GL interpreted this wave function as a complex order parameter as explained in \cref{sec:Ginzburg-Landau theory of superconductivity}.

Following \cite[ch. 14]{colemanIntroductionManyBodyPhysics2015}.

\subsection{BCS Hamiltonian}

Microscopic description of SC: 1957 by John Bardeen, his postdoc Leon Cooper and the graduate in the group, J. Robert Schrieffer \cite{bardeenTheorySuperconductivity1957}.
Description is based on the fact that the Fermi sea is unstable towards development of bound pairs under arbitrarily small attraction \cite{cooperBoundElectronPairs1956}.
The final element in this description was the origin of the attractive interaction \(V_{{\vb{k}, \vb{k}^{\prime}}}\) between electrons, which Bardeen, Cooper and Schrieffer identified as a retarded electron-phonon interaction \cite{bardeenTheorySuperconductivity1957}.
BCS-Hamiltonian:
\begin{equation}\label{eq:BCS Hamiltonian}
	H_{\text{BCS}} = \sum_{\vb{k}\sigma} \epsilon_{\vb{k}\sigma} c_{\vb{k}\sigma}^{\dagger} c_{\vb{k}\sigma} + \sum_{\vb{k}, \vb{k}^{\prime}} V_{\vb{k}, \vb{k}^{\prime}} c_{\vb{k}\uparrow}^{\dagger} c_{-\vb{k}\downarrow}^{\dagger} c_{-\vb{k}^{\prime}\downarrow} c_{\vb{k}^{\prime}\uparrow}
\end{equation}
This Hamiltonian can be solved exactly using a mean field approach, because it involves an interaction at zero momentum and thus infinite range.
Order parameter in mean field BCS theory is the pairing amplitude
\begin{equation}
	\Delta = - \frac{U}{N_{\vb{k}}} \sum_{\vb{k}} \braket{c_{-\vb{k} \downarrow} c_{\vb{k} \uparrow}} = - U \braket{c_{-\vb{r}=0 \downarrow} c_{\vb{r=0} \uparrow}} \simeq U \Psi \;.
\end{equation}
A finite \(\Delta\) corresponds to the pairing introduced above: there is a finite expectation value for a coherent creation/annihilation of a pair of electrons with opposite momentum and spin.
BCS theory brings multiple aspects together: concept of paired electrons with the pairing amplitude being the order parameter in SC, an explanation for the attractive interaction overcoming Coulomb repulsion and a model Hamiltonian that very elegantly captures the essential physics.

\todo{Band gap!}

It is very successful in two ways: on the one hand \cite{giaeverStudySuperconductorsElectron1961}

\todo{What is explained by phononic pairing?}

 in establishing electronic pairing as the microscopic mechanism behind SC, which holds still today even for high \(T_C\)/unconventional superconductors

\todo{Sources for that}

\todo{Other pairing interactions can be taken, gives explanations for a lot of different SCs}

\subsection{Attractive Hubbard Model}

The Hubbard model is the simplest model for interacting electron systems.
It goes back to works by Hubbard \cite{hubbardElectronCorrelationsNarrow1963}, Kanamori \cite{kanamoriElectronCorrelationFerromagnetism1963} and Gutzweiler \cite{gutzwillerEffectCorrelationFerromagnetism1963}.
\begin{equation}
	H_{\mathrm{int}} = U \sum_{i} c_{i, \uparrow}^{\dagger} c_{i, \downarrow}^{\dagger} c_{i, \downarrow} c_{i, \uparrow}
\end{equation}
where \(U > 0\).

Besides 

\cite{qinHubbardModelComputational2022}

\todo{Some relevance of the repulsive Hubbard model}

This simple Hubbard model can be extended in a multitude of ways to model a variety of physical system.
In this work: extension to multiple orbitals (i.e. atoms in the unit cell for lattice systems) and an attractive interaction, i.e. a negative \(U\).
Physical motivation for taking a negative-U Hubbard model: electrons can experience a local attraction interaction, for example through electrons coupling with phononic degrees of freedom or with electronic excitations that can be described as bosons \cite{micnasSuperconductivityNarrowbandSystems1990}.
\todo{There are some more specific papers to the specific mechanisms (and also some more mechanism), could cite these here and say some more things}
The form of the interaction term is then: \todo{Order of operators? -> also in all other equations!}
\begin{equation}
	H_{\mathrm{int}} = -\sum_{i, \alpha} U_{\alpha} c_{i, \alpha, \uparrow}^{\dagger} c_{i, \alpha, \downarrow}^{\dagger} c_{i, \alpha, \downarrow} c_{i, \alpha, \uparrow}
	\label{eq:Hubbard interaction multiband}
\end{equation}
where \(\alpha\) counts orbitals and the minus sign in front is taken so that \(U > 0\) now corresponds to an attractive interaction (this is purely convention).

\paragraph{Multiband BCS Mean Field Theory}

There are a multitude of ways to derive a mean field description of a given interacting Hamiltonian.
Very rigorous in path integral formulations as saddle points, given for example in ref. \cite{colemanIntroductionManyBodyPhysics2015}.
A more intuitive way based on ref. \cite{bruusManyBodyQuantumTheory2004} discussed here looks at the operators and which one are small. 

Look at interaction term \cref{eq:Hubbard interaction multiband}.
Mean-field approximation (here specifically for superconductivity i.e. pairing): operators do not deviate much from their average value, i.e. the deviation operators \todo{there are other combinations, talk about that}
\begin{align}
	d_{i, \alpha} = c_{i, \alpha, \uparrow}^{\dagger} c_{i, \alpha, \downarrow}^{\dagger} - \braket{c_{i, \alpha, \uparrow}^{\dagger} c_{i, \alpha, \downarrow}^{\dagger}} \\
	e_{i, \alpha} = c_{i, \alpha, \downarrow} c_{i, \alpha, \uparrow} - \braket{c_{i, \alpha, \downarrow} c_{i, \alpha, \uparrow}}
\end{align}
are small (dont contribute much to expectation values and correlation functions), so that in the interaction part of the Hamiltonian
\begin{align}
	H_{\mathrm{int}} &= -\sum_{i, \alpha} U_{\alpha} c_{i, \alpha, \uparrow}^{\dagger} c_{i, \alpha, \downarrow}^{\dagger} c_{i, \alpha, \downarrow} c_{i, \alpha, \uparrow} \\
	&= -\sum_{i, \alpha} U_{\alpha} 
	\left( d_{i, \alpha}^{\dagger} + \braket{c_{i, \alpha, \uparrow}^{\dagger} c_{i, \alpha, \downarrow}^{\dagger}} \right)
	\left( e_{i, \alpha} + \braket{c_{i, \alpha, \downarrow} c_{i, \alpha, \uparrow}} \right) \\
	&= -\sum_{i, \alpha} U_{\alpha} (
		d_{i, \alpha} e_{i, \alpha}
		+ d_{i, \alpha} \braket{c_{i, \alpha, \downarrow} c_{i, \alpha, \uparrow}}
		+ e_{i, \alpha} \braket{c_{i, \alpha, \uparrow}^{\dagger} c_{i, \alpha, \downarrow}^{\dagger}} \\
		&+ \braket{c_{i, \alpha, \uparrow}^{\dagger} c_{i, \alpha, \downarrow}^{\dagger}} \braket{c_{i, \alpha, \downarrow} c_{i, \alpha, \uparrow}} )
\end{align}
the first term is quadratic in the deviation and can be neglected.
Thus arrive at the approximation
\begin{align}
	H_{\mathrm{int}} &\approx -\sum_{i, \alpha} U_{\alpha} \left(
	d_{i, \alpha} \braket{c_{i, \alpha, \downarrow} c_{i, \alpha, \uparrow}}
	+ e_{i, \alpha} \braket{c_{i, \alpha, \uparrow}^{\dagger} c_{i, \alpha, \downarrow}^{\dagger}}
	+ \braket{c_{i, \alpha, \uparrow}^{\dagger} c_{i, \alpha, \downarrow}^{\dagger}} \braket{c_{i, \alpha, \downarrow} c_{i, \alpha, \uparrow}}
	\right) \\
	&= -\sum_{i, \alpha} U_{\alpha} (
		c_{i, \alpha, \uparrow}^{\dagger} c_{i, \alpha, \downarrow}^{\dagger} \braket{c_{i, \alpha, \downarrow} c_{i, \alpha, \uparrow}}
		+ c_{i, \alpha, \downarrow} c_{i, \alpha, \uparrow} \braket{c_{i, \alpha, \uparrow}^{\dagger} c_{i, \alpha, \downarrow}^{\dagger}} \\	
	&- \braket{c_{i, \alpha, \uparrow}^{\dagger} c_{i, \alpha, \downarrow}^{\dagger}} \braket{c_{i, \alpha, \downarrow} c_{i, \alpha, \uparrow}} ) \\
	&= 
\end{align}
with the expectation values
\begin{equation}
	\Delta
\end{equation}

\todo{General multi-band mean field theory theory}

\todo{Mean field with finite momentum}


\begin{align}
	H_{\mathrm{int}} \approx \sum_{\alpha, \vb{k}} (\Delta_{\alpha} c_{\vb{k} \alpha \uparrow}^{\dagger} c_{-\vb{k} \alpha \downarrow}^{\dagger} + \Delta_{\alpha}^* c_{-\vb{k} \alpha \downarrow} c_{\vb{k} \alpha \uparrow})
\end{align}


Fourier transformation:
\begin{equation}
	H_{int} = - \frac{1}{N^2} \sum_{\alpha, \vb{k}_{1, 2, 3, 4}} U_{\alpha} e^{\iu (\vb{k}_1 + \vb{k}_4 - \vb{k}_1 - \vb{k}_3) r_{i \alpha}}  c_{\vb{k}_1 \alpha \uparrow}^{\dagger} c_{\vb{k}_3 \alpha \downarrow}^{\dagger} c_{\vb{k}_2 \alpha \downarrow} c_{\vb{k}_4 \alpha \uparrow}
\end{equation}
Impose zero-momentum pairing: \(\vb{k}_1 + \vb{k}_3 = 0\) and \(\vb{k}_2 + \vb{k}_4 = 0\):
\begin{align}
	H_{int} = - \sum_{\alpha, \vb{k}, \vb{k}^{\prime}} U_{\alpha} c_{\vb{k} \alpha \uparrow}^{\dagger} c_{-\vb{k} \alpha \downarrow}^{\dagger} c_{-\vb{k}^{\prime} \alpha \downarrow} c_{\vb{k}^{\prime} \alpha \uparrow}
\end{align}
Mean-field approximation:
\begin{align}
	H_{int} \approx \sum_{\alpha, \vb{k}} (\Delta_{\alpha} c_{\vb{k} \alpha \uparrow}^{\dagger} c_{-\vb{k} \alpha \downarrow}^{\dagger} + \Delta_{\alpha}^* c_{-\vb{k} \alpha \downarrow} c_{\vb{k} \alpha \uparrow})
\end{align}
with
\begin{align}
	\Delta_{\alpha} &= - U_{\alpha} \sum_{\vb{k}^{\prime}} \braket{c_{-\vb{k}^{\prime} \alpha \downarrow} c_{\vb{k}^{\prime} \alpha \uparrow}} \\
	\Delta_{\alpha}^* &= - U_{\alpha} \sum_{\vb{k}^{\prime}} \braket{c_{\vb{k}^{\prime} \alpha \uparrow}^{\dagger} c_{-\vb{k}^{\prime} \alpha \downarrow}^{\dagger}}
\end{align}
This gives the BCS mean field Hamiltonian:
\begin{align}
	H_{BCS} = \sum_{\vb{k} \alpha \beta \sigma} [H_{0, \sigma} (\vb{k})]_{\alpha \beta} c_{\vb{k} \alpha \sigma}^{\dagger} c_{\vb{k} \beta \sigma}
	-\mu \sum_{\vb{k} \alpha \sigma} n_{\vb{k} \alpha \sigma}
	+ \sum_{\alpha, \vb{k}} (\Delta_{\alpha} c_{\vb{k} \alpha \uparrow}^{\dagger} c_{-\vb{k} \alpha \downarrow}^{\dagger} + \Delta_{\alpha}^* c_{-\vb{k} \alpha \downarrow} c_{\vb{k} \alpha \uparrow})
\end{align}
with Nambu spinor

\todo{Nambu spinor}

\begin{equation}
	\Psi_{\vb{k}} =
	\begin{pmatrix}
		c_{1, \vb{k} \uparrow} \\
		c_{2, \vb{k} \uparrow} \\
		c_{3, \vb{k} \uparrow} \\
		c_{1, -\vb{k} \downarrow}^{\dagger} \\
		c_{2, -\vb{k} \downarrow}^{\dagger} \\
		c_{3, -\vb{k} \downarrow}^{\dagger} \\
	\end{pmatrix}
\end{equation}
we have:
\begin{equation}
	H_{MF} = \sum_{\vb{k}} \Psi_{\vb{k}}^{\dagger} \mathcal{H} (\vb{k}) \Psi_{\vb{k}}
\end{equation}
with
\begin{equation}
	\mathcal{H} (\vb{k}) =
	\begin{pmatrix}
		H_{0, \uparrow} (\vb{k}) - \mu & \Delta \\
		\Delta^{\dagger} & - H_{0, \downarrow}^* (-\vb{k}) + \mu
	\end{pmatrix}
\end{equation}
with \(H_{0, \sigma}\) being the F.T. of the kinetic term and \(\Delta = diag(\Delta_1, \Delta_2, \Delta_3)\).


\paragraph{Self-consistency}

Formula for OP using the Bogoliubov operators
\begin{equation}
	\Delta_{\alpha} = -U
\end{equation}

\todo{How to solve mean field theory self-consistently}

\paragraph{Finite momentum}

To include finite momentum, take the ansatz of a Fulde-Ferrel (FF) type pairing \cite{kinnunenFuldeFerrellLarkin2018}:
\begin{equation}
	\Delta
\end{equation}

\todo{How to include finite momentum}

\section{Dynamical Mean-Field Theory}\label{sec:Dynamical Mean-Field Theory}

\subsection{Green's Function Formalism}

Following~\cite{bruusManyBodyQuantumTheory2004}

\todo{Work over the paragraph}

Green's functions: method to encode influence of many-body effects on propagation of particles in a system.

Have different kinds of Green's functions, for example the retarded Green's function:
\begin{equation}
	G^R (\vb{r}\sigma t, \vb{r}^{\prime} \sigma^{\prime} t^{\prime}) = -\iu \Theta(t- t^{\prime}) \braket{ \{c_{\vb{r} \sigma} (t), c_{\vb{r} \sigma}^{\dagger} (t^{\prime})\}}
\end{equation}
They give the amplitude of a particle inserted at point \(\vb{r}^{\prime}\) at time \(t^{\prime}\) to propagate to position \(\vb{r}\) at time \(t\).
For time-independent Hamiltonians and systems in equilibrium, the GFs only depend on time differences:
\begin{equation}
	G^R (\vb{r}\sigma t, \vb{r}^{\prime} \sigma^{\prime} t^{\prime}) = G^R (\vb{r} \sigma, \vb{r}^{\prime} \sigma^{\prime}, t - t^{\prime})
\end{equation}
So we can take \(t^{\prime} = 0\) and consider \(t\) as the only free variable:
\begin{equation}
	G^R (\vb{r}\sigma, \vb{r}^{\prime} \sigma^{\prime}, t) = -\iu \Theta(t) \braket{ \{c_{\vb{r} \sigma} (t), c_{\vb{r} \sigma}^{\dagger} (0)\}}
\end{equation}
In a translation invariant system: can use \(\vb{k}\) as a natural basis set:
\begin{equation}
	G^R (\vb{k}, \sigma, \sigma^{\prime} t) = -\iu \Theta(t- t^{\prime}) \braket{ \{c_{\vb{k} \sigma} (t), c_{\vb{k} \sigma^{\prime}}^{\dagger} (0)\}}
\end{equation}
Define Fourier-transform:
\begin{equation}
	G^R (\vb{k}, \sigma, \sigma^{\prime}, \omega) = \int_{-\infty}^{\infty} \odif{t} G^R (\vb{k}, \sigma, \sigma^{\prime} t)
\end{equation}
Can define the spectral function from this:
\begin{equation}
	A(\vb{k} \sigma, \omega) = -2 \Im G^R (\vb{k} \sigma, \omega)
\end{equation}
Looking at the diagonal elements of \(G^R\) here.
The spectral function can be thought of as the energy resolution of a particle with energy \(\omega\).
This mean, for non-interacting systems, the spectral function is a delta-function around the single-particle energies:
\begin{equation}
	A_0 (\vb{k} \sigma, \omega) = 2\pi \delta (\omega - \epsilon_{\vb{k} \sigma})
\end{equation}
For interacting systems this is not true, but \(A\) can still be peaked.

\todo{Show GFs can be related to observables}

Mathematical technique to calculate retarded GFs involves defining GFs on imaginary times \gls{imaginary time}:
\begin{equation}
	t \to -\iu \tau
\end{equation}
where \(\tau\) is real and has the dimension time.
This enables the simultaneous expansion of exponential \(e^{-\beta H}\) coming from the thermodynamic average and \(e^{-\iu H t}\) coming from the time evolution of operators.

Define imaginary time/Matsubara GF \gls{matsubara correlation function}:
\begin{equation}
	\mathcal{C}_{A B} (\tau, 0) = - \Braket{T_{\tau} (A(\tau) B(0))}
\end{equation}
with time-ordering operator in imaginary time:
\begin{equation}
	T_{\tau} (A(\tau) B(\tau^{\prime})) = \Theta(\tau - \tau^{\prime}) A(\tau) B(\tau^{\prime}) \pm \Theta(\tau^{\prime} - \tau) B(\tau^{\prime}) A(\tau)
\end{equation}
so that operators with later `times' go to the left.

Can prove from properties of Matsubara GF, that they are only defined for
\begin{equation}
	-\beta < \tau < \beta
\end{equation}
Due to this, the Fourier transform of the Matsubara GF is defined on discrete values:
\begin{equation}
	\mathcal{C}_{A B} (\iu \omega_n) = \int_{0}^{\beta} \odif{\tau}
\end{equation}
with fermionic/bosonic Matsubara frequencies
\begin{equation}
	\omega_n =
	\begin{cases}
		\frac{2n \pi}{\beta} \, \text{for bosons} \\
		\frac{(2n + 1)\pi}{\beta} \, \text{for fermions}
	\end{cases}
\end{equation}

\todo{How to resolve ambiguity at borders of integral}

It turns out that Matsubara GFs and retarded GFs can be generated from a common function \(\mathcal{C}_{AB} (z)\) that is defined on the entire complex plane except for the real axis.
So we can get the retarded GF \(\mathcal{C}_{AB}^R (\omega)\) by analytic continuation:
\begin{equation}
	\mathcal{C}_{AB}^R (\omega) = \mathcal{C}_{AB} (\iu \omega_n \to \omega + \iu \eta)
\end{equation}
So in particular the extrapolation of the Matsubara GF to zero is proportional to the density of states at the chemical potential.
Gapped: density is zero (Matsubara GF goes to 0), metal: density is finite (Matsubara GF goes to finite value) ~\cite[8.3.4]{bruusManyBodyQuantumTheory2004}.

\todo{single-particle Matsubara GF}

\todo{equations of motion for Matsubara GF}

\subsection{Perturbation theory, Dyson equation}

\todo{Short introduction to diagrams}

\todo{Self energy}

\todo{Dyson equation}

Dyson equation:
\begin{equation}
	\mathcal{G}_{\sigma} (\vb{k}, \iu \omega_n) = \frac{\mathcal{G}_{\sigma}^0 (\vb{k}, \iu \omega_n)}{1 - \mathcal{G}_{\sigma}^0 (\vb{k}, \iu \omega_n) \Sigma_{\sigma} (\vb{k}, \iu \omega_n)} = \frac{1}{\iu \omega_n - \xi_{\vb{k} - \Sigma_{\sigma} (\vb{k}, \iu \omega_n)}}
\end{equation}


\subsection{Nambu-Gorkov GF}

Introduction following~\cite[ch. 14.7]{colemanIntroductionManyBodyPhysics2015}

\todo{More general introduction into NG GFs, how they look like, what they describe etc.}

Order parameter can be chosen as the anomalous GF:
\begin{equation}
	\Psi = F^{\mathrm{loc}} (\tau = 0^-)
\end{equation}
or the superconducting gap
\begin{equation}
	\Delta = Z \Sigma^{\mathrm{AN}}
\end{equation}
that can be calculated from the anomalous self-energy \(\Sigma^{\mathrm{AN}}\) and quasiparticle weight \(Z\)
\todo{Sources for these?}

\todo{How to get quasiparticle weight?}

\subsection{DMFT}

Following \cite{georgesDynamicalMeanfieldTheory1996}.

Most general non-interacting electronic Hamiltonian in second quantization:
\begin{equation}
	H_0 = \sum_{i, j, \sigma}
\end{equation}
with lattice coordinates \(i, j\) and spin \(\sigma\).

One particle Green's function (many-body object, coming from the Hubbard model):
\begin{equation}
	G(\vb{k}, \iu \omega_n) = \frac{1}{\iu \omega_n + \mu - \epsilon_{\vb{k}} - \Sigma(\vb{k}, \iu \omega_n)}
\end{equation}
with the self energy \(\Sigma(\iu \omega_n)\) coming from the solution of the effect on-site problem:

The Dyson equation
\begin{equation}
	G(\vb{k}, \iu \omega_n) = \left( G_0 (\vb{k}, \iu \omega_n) - \Sigma(\vb{k}, \iu \omega_n)\right)^{-1}
\end{equation}
relates the non-interacting Greens function \(G_0 (\vb{k}, \iu \omega_n)\) and the fully-interacting Greens function \(G (\vb{k}, \iu \omega_n)\) (inversion of a matrix!).

\end{document}
