\documentclass[../notes.tex]{subfiles}
\graphicspath{{\subfix{../images/}}, {\subfix{../}}}

\begin{document}
\raggedbottom

\chapter{Superconductivity}\label{ch:superconductivity}

Superconductivity is an example of an emergent phenomenon: the Schrödinger equation describing all interactions between electrons gives no indication that there exists parameters for which the electrons condense into phase coherent pairs.
In this chapter we review theoretical concepts needed for understanding superconductivity and introduce the tools used to study superconductivity in the later chapters.
There are many textbooks covering these topics which can be referenced for a more detailed treatment, such as refs. \cite{colemanIntroductionManyBodyPhysics2015, tinkhamIntroductionSuperconductivity1996, bruusManyBodyQuantumTheory2004, larkinTheoryFluctuationsSuperconductors2005, bennemannSuperconductivity2008}.

Macroscopially, the superconducting state can be described by a spontaneous breaking of a \(U(1)\) phase rotation symmetry that is associated with an order parameter.
Theory of spontaneous symmetry breaking and associated phase transitions is Ginzburg-Landau theory discussed in \cref{sec:Ginzburg-Landau theory of superconductivity}, following refs. \cite{beekmanIntroductionSpontaneousSymmetry2019, colemanIntroductionManyBodyPhysics2015}.
Ginzburg-Landau theory introduces two length scales: the coherence length \(\xi_0\) describing the length scale of amplitude variations of the order parameter and the London penetration depth \(\lambda_L\), which is connected to energy cost of phase variations of the order parameter.
They also 

The interplay of these length (energy) scales determine 
\todo{What is influenced by length/energy scales? See Niklas 2024 npj}
To this end, \cref{sec:Ginzburg-Landau theory of superconductivity} introduces a theoretical framework based on Cooper pairs with finite momentum \cite{wittBypassingLatticeBCS2024} that will be used in later chapters to calculate these length scales from microscopic theories.

Ginzburg Landau theory is a macroscopic theory, but it can be connected to microscopic theories: if a theory finds an expression for the order parameter describing the symmetry breakdown, it can be connected to quantities expressed by Ginzburg-Landau theory, such as the superconducting current.
One such theory to describe superconductivity from a microscopic perspective is \glsxtrfull{bcs} theory in \cref{sec:bcs-theory}.
A method to treat local interactions non-perturbatively is \glsxtrfull{dmft}. \Cref{sec:Dynamical Mean-Field Theory} briefly introduces the Greens function method to treat many-body problems and outlines the \glsxtrshort{dmft} self-consistency cycle.

Lastly, \todo{Introduce quantum metric}

\section{Ginzburg-Landau Theory of Superconductivity}\label{sec:Ginzburg-Landau theory of superconductivity}

\subsection{Spontaneous Symmetry Breaking and Order Parameter}

Symmetries are a powerful concept in physics.
Noethers theorem \cite{noetherInvarianteVariationsprobleme1918} connects the symmetries of physical theories to associated conservation laws.
An interesting facet of symmetries in physical theories is the fact, that a ground state of a system must not necessarily obey the same symmetries of its Hamiltonian, i.e. for a symmetry operation that is described by a unitary operator \(U\), the Hamiltonian commutes with \(U\) (which results in expectation values of the Hamiltonian being invariant under the symmetry operation) but the states \(\ket{\phi}\) and \(U \ket{\phi}\) are different.
This phenomenon is called \glsxtrfull{ssb} and the state \(\ket{\phi}\) is said to be symmetry-broken.

One consequence of this fact is that for a given symmetry-broken state \(\ket{\phi}\), there exists multiple states that can be reached by repeatedly applying \(U\) to \(\ket{\phi}\) and all have the same energy.
To differentiate the symmetry-broken states an operator can be defined that has all these equivalent states as eigenvectors with different eigenvalues and zero expectation value for symmetric states.
This is the microscopic notion of an order parameter.

The original notion of an order parameter was motivated from macroscopic observables that can then be related to the microscopic order parameter operator introduced above.
Macroscopically we characterize the symmetry breaking by an order parameter \(\Psi\) which generally can be a complex-valued vector that becomes non-zero below the transition temperature \(T_C\)
\begin{equation}
	\vert \Psi \vert =
	\begin{cases}
		0 & T > T_C \\
		\vert \Psi_0 \vert > 0 & T < T_C
	\end{cases} \;.
\end{equation}

In the example of a ferromagnet, a finite magnetization of a material is associated with a finite expectation value for the \(z\)-component of the spin operator, \(m_z = \braket{\ope{S_z}}\).
The order parameter describes the `degree of order` \cite{landauTheoryPhaseTransitions1937}.
Similarly to a magnetically ordered state, the SC state is characterized by an order parameter.
The theory of phase transitions in superconductors was developed by Ginzburg and Landau \cite{ginzburgTheorySuperconductivity1950}.
Landau theory and conversely Ginzburg-Landau theory is not concerned with the the microscopic properties of the order parameter, but describes the changes in thermodynamic properties of matter with the development of an order parameter.

\subsection{Landau and Ginzburg-Landau Theory}\label{sub:Landau and Ginzburg-Landau Theory}

The free energy is a thermodynamic quantity:
\begin{equation}
	F = E - T S
\end{equation}
with the energy of the system \(E\), temperature \(T\) and entropy \(S\).
A system in thermodynamic equilibrium has minimal free energy.
The fundamental idea underlying Landau theory is to write the free energy \(F[\Psi]\) as function of the order parameter \(\Psi\) and expand it as a polynomial:
\begin{equation}
	F_L[\Psi] = \int \odif[order={d}]{x} f_L [\Psi] \;,
\end{equation}
where
\begin{equation}
	f_L [\Psi] = \frac{r}{2} \Psi^2 + \frac{u}{4} \Psi^4
\end{equation}
is called the free energy density.
Provided the parameters \(r\) and \(u\) are greater than \(0\), there is a minimum of \(f_L [\Psi]\) that lies at \(\Psi = 0\).
Landau theory assumes that at the phase transition temperature \(T_C\) the parameter \(r\) changes sign, so it can be written in first order as
\begin{equation}
	r = a(T - T_C) \;.
\end{equation}

\begin{figure}[t]
	\centering
	\begin{subfigure}[b]{0.5\textwidth}
		\centering
		\caption{\hfill\null}\label{sfig:Landau free energy}
		%% Creator: Matplotlib, PGF backend
%%
%% To include the figure in your LaTeX document, write
%%   \input{<filename>.pgf}
%%
%% Make sure the required packages are loaded in your preamble
%%   \usepackage{pgf}
%%
%% Also ensure that all the required font packages are loaded; for instance,
%% the lmodern package is sometimes necessary when using math font.
%%   \usepackage{lmodern}
%%
%% Figures using additional raster images can only be included by \input if
%% they are in the same directory as the main LaTeX file. For loading figures
%% from other directories you can use the `import` package
%%   \usepackage{import}
%%
%% and then include the figures with
%%   \import{<path to file>}{<filename>.pgf}
%%
%% Matplotlib used the following preamble
%%   \def\mathdefault#1{#1}
%%   \everymath=\expandafter{\the\everymath\displaystyle}
%%   \IfFileExists{scrextend.sty}{
%%     \usepackage[fontsize=10.000000pt]{scrextend}
%%   }{
%%     \renewcommand{\normalsize}{\fontsize{10.000000}{12.000000}\selectfont}
%%     \normalsize
%%   }
%%   \usepackage{fontspec}\usepackage{unicode-math}\setmathfont{texgyrepagella-math.otf}\setmainfont{texgyrepagella-math}
%%   \makeatletter\@ifpackageloaded{underscore}{}{\usepackage[strings]{underscore}}\makeatother
%%
\begingroup%
\makeatletter%
\begin{pgfpicture}%
\pgfpathrectangle{\pgfpointorigin}{\pgfqpoint{2.252757in}{2.135967in}}%
\pgfusepath{use as bounding box, clip}%
\begin{pgfscope}%
\pgfsetbuttcap%
\pgfsetmiterjoin%
\definecolor{currentfill}{rgb}{1.000000,1.000000,1.000000}%
\pgfsetfillcolor{currentfill}%
\pgfsetlinewidth{0.000000pt}%
\definecolor{currentstroke}{rgb}{1.000000,1.000000,1.000000}%
\pgfsetstrokecolor{currentstroke}%
\pgfsetdash{}{0pt}%
\pgfpathmoveto{\pgfqpoint{0.000000in}{0.000000in}}%
\pgfpathlineto{\pgfqpoint{2.252757in}{0.000000in}}%
\pgfpathlineto{\pgfqpoint{2.252757in}{2.135967in}}%
\pgfpathlineto{\pgfqpoint{0.000000in}{2.135967in}}%
\pgfpathlineto{\pgfqpoint{0.000000in}{0.000000in}}%
\pgfpathclose%
\pgfusepath{fill}%
\end{pgfscope}%
\begin{pgfscope}%
\pgfsetbuttcap%
\pgfsetmiterjoin%
\definecolor{currentfill}{rgb}{1.000000,1.000000,1.000000}%
\pgfsetfillcolor{currentfill}%
\pgfsetlinewidth{0.000000pt}%
\definecolor{currentstroke}{rgb}{0.000000,0.000000,0.000000}%
\pgfsetstrokecolor{currentstroke}%
\pgfsetstrokeopacity{0.000000}%
\pgfsetdash{}{0pt}%
\pgfpathmoveto{\pgfqpoint{0.050000in}{0.050000in}}%
\pgfpathlineto{\pgfqpoint{2.057250in}{0.050000in}}%
\pgfpathlineto{\pgfqpoint{2.057250in}{2.044300in}}%
\pgfpathlineto{\pgfqpoint{0.050000in}{2.044300in}}%
\pgfpathlineto{\pgfqpoint{0.050000in}{0.050000in}}%
\pgfpathclose%
\pgfusepath{fill}%
\end{pgfscope}%
\begin{pgfscope}%
\definecolor{textcolor}{rgb}{0.000000,0.000000,0.000000}%
\pgfsetstrokecolor{textcolor}%
\pgfsetfillcolor{textcolor}%
\pgftext[x=2.197758in,y=0.887606in,,top]{\color{textcolor}{\rmfamily\fontsize{10.000000}{12.000000}\selectfont\catcode`\^=\active\def^{\ifmmode\sp\else\^{}\fi}\catcode`\%=\active\def%{\%}$\Psi$}}%
\end{pgfscope}%
\begin{pgfscope}%
\pgfpathrectangle{\pgfqpoint{0.050000in}{0.050000in}}{\pgfqpoint{2.007250in}{1.994300in}}%
\pgfusepath{clip}%
\pgfsetrectcap%
\pgfsetroundjoin%
\pgfsetlinewidth{1.003750pt}%
\definecolor{currentstroke}{rgb}{0.800000,0.400000,0.466667}%
\pgfsetstrokecolor{currentstroke}%
\pgfsetdash{}{0pt}%
\pgfpathmoveto{\pgfqpoint{0.192273in}{2.054300in}}%
\pgfpathlineto{\pgfqpoint{0.210719in}{1.899358in}}%
\pgfpathlineto{\pgfqpoint{0.232660in}{1.730734in}}%
\pgfpathlineto{\pgfqpoint{0.254601in}{1.578217in}}%
\pgfpathlineto{\pgfqpoint{0.276542in}{1.440873in}}%
\pgfpathlineto{\pgfqpoint{0.298484in}{1.317791in}}%
\pgfpathlineto{\pgfqpoint{0.320425in}{1.208089in}}%
\pgfpathlineto{\pgfqpoint{0.342366in}{1.110906in}}%
\pgfpathlineto{\pgfqpoint{0.360650in}{1.038882in}}%
\pgfpathlineto{\pgfqpoint{0.378934in}{0.974504in}}%
\pgfpathlineto{\pgfqpoint{0.397219in}{0.917315in}}%
\pgfpathlineto{\pgfqpoint{0.415503in}{0.866871in}}%
\pgfpathlineto{\pgfqpoint{0.433787in}{0.822739in}}%
\pgfpathlineto{\pgfqpoint{0.452072in}{0.784499in}}%
\pgfpathlineto{\pgfqpoint{0.470356in}{0.751744in}}%
\pgfpathlineto{\pgfqpoint{0.488640in}{0.724078in}}%
\pgfpathlineto{\pgfqpoint{0.506925in}{0.701116in}}%
\pgfpathlineto{\pgfqpoint{0.521552in}{0.685885in}}%
\pgfpathlineto{\pgfqpoint{0.536179in}{0.673243in}}%
\pgfpathlineto{\pgfqpoint{0.550807in}{0.663010in}}%
\pgfpathlineto{\pgfqpoint{0.565434in}{0.655012in}}%
\pgfpathlineto{\pgfqpoint{0.580062in}{0.649080in}}%
\pgfpathlineto{\pgfqpoint{0.598346in}{0.644320in}}%
\pgfpathlineto{\pgfqpoint{0.616630in}{0.642220in}}%
\pgfpathlineto{\pgfqpoint{0.634915in}{0.642484in}}%
\pgfpathlineto{\pgfqpoint{0.653199in}{0.644824in}}%
\pgfpathlineto{\pgfqpoint{0.675140in}{0.649989in}}%
\pgfpathlineto{\pgfqpoint{0.700738in}{0.658700in}}%
\pgfpathlineto{\pgfqpoint{0.729993in}{0.671380in}}%
\pgfpathlineto{\pgfqpoint{0.766562in}{0.689970in}}%
\pgfpathlineto{\pgfqpoint{0.898208in}{0.759831in}}%
\pgfpathlineto{\pgfqpoint{0.931120in}{0.773608in}}%
\pgfpathlineto{\pgfqpoint{0.960375in}{0.783559in}}%
\pgfpathlineto{\pgfqpoint{0.989630in}{0.791040in}}%
\pgfpathlineto{\pgfqpoint{1.015228in}{0.795389in}}%
\pgfpathlineto{\pgfqpoint{1.040826in}{0.797587in}}%
\pgfpathlineto{\pgfqpoint{1.066424in}{0.797587in}}%
\pgfpathlineto{\pgfqpoint{1.092022in}{0.795389in}}%
\pgfpathlineto{\pgfqpoint{1.117620in}{0.791040in}}%
\pgfpathlineto{\pgfqpoint{1.143218in}{0.784635in}}%
\pgfpathlineto{\pgfqpoint{1.172473in}{0.774978in}}%
\pgfpathlineto{\pgfqpoint{1.205385in}{0.761482in}}%
\pgfpathlineto{\pgfqpoint{1.241953in}{0.743846in}}%
\pgfpathlineto{\pgfqpoint{1.289492in}{0.718241in}}%
\pgfpathlineto{\pgfqpoint{1.366286in}{0.676702in}}%
\pgfpathlineto{\pgfqpoint{1.399198in}{0.661631in}}%
\pgfpathlineto{\pgfqpoint{1.424796in}{0.652210in}}%
\pgfpathlineto{\pgfqpoint{1.446737in}{0.646280in}}%
\pgfpathlineto{\pgfqpoint{1.468679in}{0.642794in}}%
\pgfpathlineto{\pgfqpoint{1.486963in}{0.642093in}}%
\pgfpathlineto{\pgfqpoint{1.505247in}{0.643697in}}%
\pgfpathlineto{\pgfqpoint{1.523531in}{0.647901in}}%
\pgfpathlineto{\pgfqpoint{1.538159in}{0.653342in}}%
\pgfpathlineto{\pgfqpoint{1.552786in}{0.660808in}}%
\pgfpathlineto{\pgfqpoint{1.567414in}{0.670466in}}%
\pgfpathlineto{\pgfqpoint{1.582041in}{0.682489in}}%
\pgfpathlineto{\pgfqpoint{1.596669in}{0.697056in}}%
\pgfpathlineto{\pgfqpoint{1.611296in}{0.714350in}}%
\pgfpathlineto{\pgfqpoint{1.625923in}{0.734558in}}%
\pgfpathlineto{\pgfqpoint{1.644208in}{0.764214in}}%
\pgfpathlineto{\pgfqpoint{1.662492in}{0.799115in}}%
\pgfpathlineto{\pgfqpoint{1.680776in}{0.839661in}}%
\pgfpathlineto{\pgfqpoint{1.699061in}{0.886267in}}%
\pgfpathlineto{\pgfqpoint{1.717345in}{0.939356in}}%
\pgfpathlineto{\pgfqpoint{1.735629in}{0.999367in}}%
\pgfpathlineto{\pgfqpoint{1.753914in}{1.066748in}}%
\pgfpathlineto{\pgfqpoint{1.772198in}{1.141961in}}%
\pgfpathlineto{\pgfqpoint{1.790482in}{1.225480in}}%
\pgfpathlineto{\pgfqpoint{1.812423in}{1.337352in}}%
\pgfpathlineto{\pgfqpoint{1.834364in}{1.462749in}}%
\pgfpathlineto{\pgfqpoint{1.856306in}{1.602558in}}%
\pgfpathlineto{\pgfqpoint{1.878247in}{1.757694in}}%
\pgfpathlineto{\pgfqpoint{1.900188in}{1.929095in}}%
\pgfpathlineto{\pgfqpoint{1.914977in}{2.054300in}}%
\pgfpathlineto{\pgfqpoint{1.914977in}{2.054300in}}%
\pgfusepath{stroke}%
\end{pgfscope}%
\begin{pgfscope}%
\pgfpathrectangle{\pgfqpoint{0.050000in}{0.050000in}}{\pgfqpoint{2.007250in}{1.994300in}}%
\pgfusepath{clip}%
\pgfsetrectcap%
\pgfsetroundjoin%
\pgfsetlinewidth{1.003750pt}%
\definecolor{currentstroke}{rgb}{0.200000,0.133333,0.533333}%
\pgfsetstrokecolor{currentstroke}%
\pgfsetdash{}{0pt}%
\pgfpathmoveto{\pgfqpoint{0.141239in}{1.148426in}}%
\pgfpathlineto{\pgfqpoint{0.163180in}{0.988971in}}%
\pgfpathlineto{\pgfqpoint{0.185121in}{0.847129in}}%
\pgfpathlineto{\pgfqpoint{0.207062in}{0.721885in}}%
\pgfpathlineto{\pgfqpoint{0.225346in}{0.629479in}}%
\pgfpathlineto{\pgfqpoint{0.243631in}{0.547353in}}%
\pgfpathlineto{\pgfqpoint{0.261915in}{0.474958in}}%
\pgfpathlineto{\pgfqpoint{0.280199in}{0.411761in}}%
\pgfpathlineto{\pgfqpoint{0.298484in}{0.357239in}}%
\pgfpathlineto{\pgfqpoint{0.316768in}{0.310880in}}%
\pgfpathlineto{\pgfqpoint{0.331395in}{0.279336in}}%
\pgfpathlineto{\pgfqpoint{0.346023in}{0.252449in}}%
\pgfpathlineto{\pgfqpoint{0.360650in}{0.229973in}}%
\pgfpathlineto{\pgfqpoint{0.375278in}{0.211670in}}%
\pgfpathlineto{\pgfqpoint{0.389905in}{0.197305in}}%
\pgfpathlineto{\pgfqpoint{0.404532in}{0.186648in}}%
\pgfpathlineto{\pgfqpoint{0.419160in}{0.179474in}}%
\pgfpathlineto{\pgfqpoint{0.430131in}{0.176249in}}%
\pgfpathlineto{\pgfqpoint{0.441101in}{0.174769in}}%
\pgfpathlineto{\pgfqpoint{0.455729in}{0.175356in}}%
\pgfpathlineto{\pgfqpoint{0.470356in}{0.178681in}}%
\pgfpathlineto{\pgfqpoint{0.484983in}{0.184542in}}%
\pgfpathlineto{\pgfqpoint{0.499611in}{0.192741in}}%
\pgfpathlineto{\pgfqpoint{0.517895in}{0.205988in}}%
\pgfpathlineto{\pgfqpoint{0.536179in}{0.222225in}}%
\pgfpathlineto{\pgfqpoint{0.558121in}{0.245162in}}%
\pgfpathlineto{\pgfqpoint{0.580062in}{0.271316in}}%
\pgfpathlineto{\pgfqpoint{0.605660in}{0.305149in}}%
\pgfpathlineto{\pgfqpoint{0.638571in}{0.352611in}}%
\pgfpathlineto{\pgfqpoint{0.682454in}{0.420143in}}%
\pgfpathlineto{\pgfqpoint{0.777532in}{0.567512in}}%
\pgfpathlineto{\pgfqpoint{0.814101in}{0.619565in}}%
\pgfpathlineto{\pgfqpoint{0.843356in}{0.657810in}}%
\pgfpathlineto{\pgfqpoint{0.872610in}{0.692363in}}%
\pgfpathlineto{\pgfqpoint{0.898208in}{0.719144in}}%
\pgfpathlineto{\pgfqpoint{0.920150in}{0.739287in}}%
\pgfpathlineto{\pgfqpoint{0.942091in}{0.756658in}}%
\pgfpathlineto{\pgfqpoint{0.964032in}{0.771114in}}%
\pgfpathlineto{\pgfqpoint{0.985973in}{0.782539in}}%
\pgfpathlineto{\pgfqpoint{1.004257in}{0.789679in}}%
\pgfpathlineto{\pgfqpoint{1.022542in}{0.794612in}}%
\pgfpathlineto{\pgfqpoint{1.040826in}{0.797311in}}%
\pgfpathlineto{\pgfqpoint{1.059110in}{0.797761in}}%
\pgfpathlineto{\pgfqpoint{1.077395in}{0.795961in}}%
\pgfpathlineto{\pgfqpoint{1.095679in}{0.791919in}}%
\pgfpathlineto{\pgfqpoint{1.113963in}{0.785658in}}%
\pgfpathlineto{\pgfqpoint{1.132247in}{0.777211in}}%
\pgfpathlineto{\pgfqpoint{1.150532in}{0.766626in}}%
\pgfpathlineto{\pgfqpoint{1.172473in}{0.751185in}}%
\pgfpathlineto{\pgfqpoint{1.194414in}{0.732873in}}%
\pgfpathlineto{\pgfqpoint{1.216355in}{0.711842in}}%
\pgfpathlineto{\pgfqpoint{1.241953in}{0.684101in}}%
\pgfpathlineto{\pgfqpoint{1.267551in}{0.653220in}}%
\pgfpathlineto{\pgfqpoint{1.296806in}{0.614555in}}%
\pgfpathlineto{\pgfqpoint{1.333375in}{0.562093in}}%
\pgfpathlineto{\pgfqpoint{1.377257in}{0.494952in}}%
\pgfpathlineto{\pgfqpoint{1.479649in}{0.336387in}}%
\pgfpathlineto{\pgfqpoint{1.512561in}{0.290262in}}%
\pgfpathlineto{\pgfqpoint{1.538159in}{0.257873in}}%
\pgfpathlineto{\pgfqpoint{1.560100in}{0.233256in}}%
\pgfpathlineto{\pgfqpoint{1.582041in}{0.212144in}}%
\pgfpathlineto{\pgfqpoint{1.600325in}{0.197658in}}%
\pgfpathlineto{\pgfqpoint{1.618610in}{0.186380in}}%
\pgfpathlineto{\pgfqpoint{1.633237in}{0.179916in}}%
\pgfpathlineto{\pgfqpoint{1.647865in}{0.175939in}}%
\pgfpathlineto{\pgfqpoint{1.662492in}{0.174648in}}%
\pgfpathlineto{\pgfqpoint{1.673463in}{0.175566in}}%
\pgfpathlineto{\pgfqpoint{1.684433in}{0.178200in}}%
\pgfpathlineto{\pgfqpoint{1.695404in}{0.182639in}}%
\pgfpathlineto{\pgfqpoint{1.706374in}{0.188976in}}%
\pgfpathlineto{\pgfqpoint{1.717345in}{0.197305in}}%
\pgfpathlineto{\pgfqpoint{1.731972in}{0.211670in}}%
\pgfpathlineto{\pgfqpoint{1.746600in}{0.229973in}}%
\pgfpathlineto{\pgfqpoint{1.761227in}{0.252449in}}%
\pgfpathlineto{\pgfqpoint{1.775855in}{0.279336in}}%
\pgfpathlineto{\pgfqpoint{1.790482in}{0.310880in}}%
\pgfpathlineto{\pgfqpoint{1.805110in}{0.347330in}}%
\pgfpathlineto{\pgfqpoint{1.819737in}{0.388940in}}%
\pgfpathlineto{\pgfqpoint{1.838021in}{0.448605in}}%
\pgfpathlineto{\pgfqpoint{1.856306in}{0.517257in}}%
\pgfpathlineto{\pgfqpoint{1.874590in}{0.595426in}}%
\pgfpathlineto{\pgfqpoint{1.892874in}{0.683653in}}%
\pgfpathlineto{\pgfqpoint{1.911158in}{0.782494in}}%
\pgfpathlineto{\pgfqpoint{1.929443in}{0.892516in}}%
\pgfpathlineto{\pgfqpoint{1.947727in}{1.014296in}}%
\pgfpathlineto{\pgfqpoint{1.966011in}{1.148426in}}%
\pgfpathlineto{\pgfqpoint{1.966011in}{1.148426in}}%
\pgfusepath{stroke}%
\end{pgfscope}%
\begin{pgfscope}%
\pgfpathrectangle{\pgfqpoint{0.050000in}{0.050000in}}{\pgfqpoint{2.007250in}{1.994300in}}%
\pgfusepath{clip}%
\pgfsetrectcap%
\pgfsetroundjoin%
\pgfsetlinewidth{1.003750pt}%
\definecolor{currentstroke}{rgb}{0.866667,0.800000,0.466667}%
\pgfsetstrokecolor{currentstroke}%
\pgfsetdash{}{0pt}%
\pgfpathmoveto{\pgfqpoint{0.710063in}{2.054300in}}%
\pgfpathlineto{\pgfqpoint{0.733650in}{1.880372in}}%
\pgfpathlineto{\pgfqpoint{0.759248in}{1.707893in}}%
\pgfpathlineto{\pgfqpoint{0.784846in}{1.551767in}}%
\pgfpathlineto{\pgfqpoint{0.806787in}{1.430566in}}%
\pgfpathlineto{\pgfqpoint{0.828728in}{1.320701in}}%
\pgfpathlineto{\pgfqpoint{0.850669in}{1.221900in}}%
\pgfpathlineto{\pgfqpoint{0.872610in}{1.133915in}}%
\pgfpathlineto{\pgfqpoint{0.890895in}{1.068697in}}%
\pgfpathlineto{\pgfqpoint{0.909179in}{1.010721in}}%
\pgfpathlineto{\pgfqpoint{0.927463in}{0.959885in}}%
\pgfpathlineto{\pgfqpoint{0.945748in}{0.916098in}}%
\pgfpathlineto{\pgfqpoint{0.960375in}{0.886092in}}%
\pgfpathlineto{\pgfqpoint{0.975003in}{0.860512in}}%
\pgfpathlineto{\pgfqpoint{0.989630in}{0.839330in}}%
\pgfpathlineto{\pgfqpoint{1.004257in}{0.822522in}}%
\pgfpathlineto{\pgfqpoint{1.015228in}{0.812773in}}%
\pgfpathlineto{\pgfqpoint{1.026199in}{0.805468in}}%
\pgfpathlineto{\pgfqpoint{1.037169in}{0.800600in}}%
\pgfpathlineto{\pgfqpoint{1.048140in}{0.798167in}}%
\pgfpathlineto{\pgfqpoint{1.059110in}{0.798167in}}%
\pgfpathlineto{\pgfqpoint{1.070081in}{0.800600in}}%
\pgfpathlineto{\pgfqpoint{1.081051in}{0.805468in}}%
\pgfpathlineto{\pgfqpoint{1.092022in}{0.812773in}}%
\pgfpathlineto{\pgfqpoint{1.102993in}{0.822522in}}%
\pgfpathlineto{\pgfqpoint{1.113963in}{0.834719in}}%
\pgfpathlineto{\pgfqpoint{1.128591in}{0.854805in}}%
\pgfpathlineto{\pgfqpoint{1.143218in}{0.879283in}}%
\pgfpathlineto{\pgfqpoint{1.157845in}{0.908180in}}%
\pgfpathlineto{\pgfqpoint{1.172473in}{0.941529in}}%
\pgfpathlineto{\pgfqpoint{1.190757in}{0.989535in}}%
\pgfpathlineto{\pgfqpoint{1.209042in}{1.044643in}}%
\pgfpathlineto{\pgfqpoint{1.227326in}{1.106951in}}%
\pgfpathlineto{\pgfqpoint{1.245610in}{1.176570in}}%
\pgfpathlineto{\pgfqpoint{1.267551in}{1.269933in}}%
\pgfpathlineto{\pgfqpoint{1.289492in}{1.374234in}}%
\pgfpathlineto{\pgfqpoint{1.311434in}{1.489731in}}%
\pgfpathlineto{\pgfqpoint{1.333375in}{1.616710in}}%
\pgfpathlineto{\pgfqpoint{1.355316in}{1.755481in}}%
\pgfpathlineto{\pgfqpoint{1.380914in}{1.932732in}}%
\pgfpathlineto{\pgfqpoint{1.397187in}{2.054300in}}%
\pgfpathlineto{\pgfqpoint{1.397187in}{2.054300in}}%
\pgfusepath{stroke}%
\end{pgfscope}%
\begin{pgfscope}%
\pgfpathrectangle{\pgfqpoint{0.050000in}{0.050000in}}{\pgfqpoint{2.007250in}{1.994300in}}%
\pgfusepath{clip}%
\pgfsetrectcap%
\pgfsetroundjoin%
\pgfsetlinewidth{1.003750pt}%
\definecolor{currentstroke}{rgb}{0.066667,0.466667,0.200000}%
\pgfsetstrokecolor{currentstroke}%
\pgfsetdash{}{0pt}%
\pgfpathmoveto{\pgfqpoint{0.328840in}{2.054300in}}%
\pgfpathlineto{\pgfqpoint{0.349680in}{1.915874in}}%
\pgfpathlineto{\pgfqpoint{0.371621in}{1.782868in}}%
\pgfpathlineto{\pgfqpoint{0.393562in}{1.662098in}}%
\pgfpathlineto{\pgfqpoint{0.415503in}{1.552790in}}%
\pgfpathlineto{\pgfqpoint{0.437444in}{1.454194in}}%
\pgfpathlineto{\pgfqpoint{0.459385in}{1.365586in}}%
\pgfpathlineto{\pgfqpoint{0.481327in}{1.286268in}}%
\pgfpathlineto{\pgfqpoint{0.503268in}{1.215567in}}%
\pgfpathlineto{\pgfqpoint{0.525209in}{1.152835in}}%
\pgfpathlineto{\pgfqpoint{0.547150in}{1.097449in}}%
\pgfpathlineto{\pgfqpoint{0.569091in}{1.048812in}}%
\pgfpathlineto{\pgfqpoint{0.591032in}{1.006353in}}%
\pgfpathlineto{\pgfqpoint{0.612973in}{0.969523in}}%
\pgfpathlineto{\pgfqpoint{0.631258in}{0.942757in}}%
\pgfpathlineto{\pgfqpoint{0.649542in}{0.919250in}}%
\pgfpathlineto{\pgfqpoint{0.671483in}{0.894956in}}%
\pgfpathlineto{\pgfqpoint{0.693424in}{0.874505in}}%
\pgfpathlineto{\pgfqpoint{0.715366in}{0.857469in}}%
\pgfpathlineto{\pgfqpoint{0.737307in}{0.843444in}}%
\pgfpathlineto{\pgfqpoint{0.759248in}{0.832053in}}%
\pgfpathlineto{\pgfqpoint{0.781189in}{0.822944in}}%
\pgfpathlineto{\pgfqpoint{0.806787in}{0.814765in}}%
\pgfpathlineto{\pgfqpoint{0.836042in}{0.808067in}}%
\pgfpathlineto{\pgfqpoint{0.868954in}{0.803158in}}%
\pgfpathlineto{\pgfqpoint{0.905522in}{0.800053in}}%
\pgfpathlineto{\pgfqpoint{0.953061in}{0.798328in}}%
\pgfpathlineto{\pgfqpoint{1.040826in}{0.797863in}}%
\pgfpathlineto{\pgfqpoint{1.176130in}{0.798888in}}%
\pgfpathlineto{\pgfqpoint{1.220012in}{0.801352in}}%
\pgfpathlineto{\pgfqpoint{1.256581in}{0.805587in}}%
\pgfpathlineto{\pgfqpoint{1.285836in}{0.811100in}}%
\pgfpathlineto{\pgfqpoint{1.311434in}{0.817976in}}%
\pgfpathlineto{\pgfqpoint{1.337032in}{0.827234in}}%
\pgfpathlineto{\pgfqpoint{1.358973in}{0.837442in}}%
\pgfpathlineto{\pgfqpoint{1.380914in}{0.850104in}}%
\pgfpathlineto{\pgfqpoint{1.402855in}{0.865586in}}%
\pgfpathlineto{\pgfqpoint{1.424796in}{0.884277in}}%
\pgfpathlineto{\pgfqpoint{1.443081in}{0.902605in}}%
\pgfpathlineto{\pgfqpoint{1.461365in}{0.923704in}}%
\pgfpathlineto{\pgfqpoint{1.479649in}{0.947840in}}%
\pgfpathlineto{\pgfqpoint{1.497933in}{0.975293in}}%
\pgfpathlineto{\pgfqpoint{1.516218in}{1.006353in}}%
\pgfpathlineto{\pgfqpoint{1.534502in}{1.041322in}}%
\pgfpathlineto{\pgfqpoint{1.552786in}{1.080516in}}%
\pgfpathlineto{\pgfqpoint{1.571071in}{1.124262in}}%
\pgfpathlineto{\pgfqpoint{1.593012in}{1.183245in}}%
\pgfpathlineto{\pgfqpoint{1.614953in}{1.249882in}}%
\pgfpathlineto{\pgfqpoint{1.636894in}{1.324808in}}%
\pgfpathlineto{\pgfqpoint{1.658835in}{1.408685in}}%
\pgfpathlineto{\pgfqpoint{1.680776in}{1.502198in}}%
\pgfpathlineto{\pgfqpoint{1.702718in}{1.606059in}}%
\pgfpathlineto{\pgfqpoint{1.724659in}{1.721003in}}%
\pgfpathlineto{\pgfqpoint{1.746600in}{1.847792in}}%
\pgfpathlineto{\pgfqpoint{1.768541in}{1.987214in}}%
\pgfpathlineto{\pgfqpoint{1.778410in}{2.054300in}}%
\pgfpathlineto{\pgfqpoint{1.778410in}{2.054300in}}%
\pgfusepath{stroke}%
\end{pgfscope}%
\begin{pgfscope}%
\pgfsetbuttcap%
\pgfsetmiterjoin%
\definecolor{currentfill}{rgb}{0.000000,0.000000,0.000000}%
\pgfsetfillcolor{currentfill}%
\pgfsetlinewidth{1.003750pt}%
\definecolor{currentstroke}{rgb}{0.000000,0.000000,0.000000}%
\pgfsetstrokecolor{currentstroke}%
\pgfsetdash{}{0pt}%
\pgfsys@defobject{currentmarker}{\pgfqpoint{-0.041667in}{-0.041667in}}{\pgfqpoint{0.041667in}{0.041667in}}{%
\pgfpathmoveto{\pgfqpoint{0.041667in}{-0.000000in}}%
\pgfpathlineto{\pgfqpoint{-0.041667in}{0.041667in}}%
\pgfpathlineto{\pgfqpoint{-0.041667in}{-0.041667in}}%
\pgfpathlineto{\pgfqpoint{0.041667in}{-0.000000in}}%
\pgfpathclose%
\pgfusepath{stroke,fill}%
}%
\begin{pgfscope}%
\pgfsys@transformshift{2.057250in}{0.797862in}%
\pgfsys@useobject{currentmarker}{}%
\end{pgfscope}%
\end{pgfscope}%
\begin{pgfscope}%
\pgfsetbuttcap%
\pgfsetmiterjoin%
\definecolor{currentfill}{rgb}{0.000000,0.000000,0.000000}%
\pgfsetfillcolor{currentfill}%
\pgfsetlinewidth{1.003750pt}%
\definecolor{currentstroke}{rgb}{0.000000,0.000000,0.000000}%
\pgfsetstrokecolor{currentstroke}%
\pgfsetdash{}{0pt}%
\pgfsys@defobject{currentmarker}{\pgfqpoint{-0.041667in}{-0.041667in}}{\pgfqpoint{0.041667in}{0.041667in}}{%
\pgfpathmoveto{\pgfqpoint{0.000000in}{0.041667in}}%
\pgfpathlineto{\pgfqpoint{-0.041667in}{-0.041667in}}%
\pgfpathlineto{\pgfqpoint{0.041667in}{-0.041667in}}%
\pgfpathlineto{\pgfqpoint{0.000000in}{0.041667in}}%
\pgfpathclose%
\pgfusepath{stroke,fill}%
}%
\begin{pgfscope}%
\pgfsys@transformshift{1.053625in}{2.044300in}%
\pgfsys@useobject{currentmarker}{}%
\end{pgfscope}%
\end{pgfscope}%
\begin{pgfscope}%
\pgfsetrectcap%
\pgfsetmiterjoin%
\pgfsetlinewidth{0.501875pt}%
\definecolor{currentstroke}{rgb}{0.000000,0.000000,0.000000}%
\pgfsetstrokecolor{currentstroke}%
\pgfsetdash{}{0pt}%
\pgfpathmoveto{\pgfqpoint{1.053625in}{0.050000in}}%
\pgfpathlineto{\pgfqpoint{1.053625in}{2.044300in}}%
\pgfusepath{stroke}%
\end{pgfscope}%
\begin{pgfscope}%
\pgfsetrectcap%
\pgfsetmiterjoin%
\pgfsetlinewidth{0.501875pt}%
\definecolor{currentstroke}{rgb}{0.000000,0.000000,0.000000}%
\pgfsetstrokecolor{currentstroke}%
\pgfsetdash{}{0pt}%
\pgfpathmoveto{\pgfqpoint{0.050000in}{0.797862in}}%
\pgfpathlineto{\pgfqpoint{2.057250in}{0.797862in}}%
\pgfusepath{stroke}%
\end{pgfscope}%
\begin{pgfscope}%
\definecolor{textcolor}{rgb}{0.800000,0.400000,0.466667}%
\pgfsetstrokecolor{textcolor}%
\pgfsetfillcolor{textcolor}%
\pgftext[x=1.102286in,y=1.296438in,left,base]{\color{textcolor}{\rmfamily\fontsize{10.000000}{12.000000}\selectfont\catcode`\^=\active\def^{\ifmmode\sp\else\^{}\fi}\catcode`\%=\active\def%{\%}$T < T_C$}}%
\end{pgfscope}%
\begin{pgfscope}%
\definecolor{textcolor}{rgb}{0.866667,0.800000,0.466667}%
\pgfsetstrokecolor{textcolor}%
\pgfsetfillcolor{textcolor}%
\pgftext[x=1.102286in,y=1.296438in,left,base]{\color{textcolor}{\rmfamily\fontsize{10.000000}{12.000000}\selectfont\catcode`\^=\active\def^{\ifmmode\sp\else\^{}\fi}\catcode`\%=\active\def%{\%}$T > T_C$}}%
\end{pgfscope}%
\begin{pgfscope}%
\definecolor{textcolor}{rgb}{0.066667,0.466667,0.200000}%
\pgfsetstrokecolor{textcolor}%
\pgfsetfillcolor{textcolor}%
\pgftext[x=1.102286in,y=1.795013in,left,base]{\color{textcolor}{\rmfamily\fontsize{10.000000}{12.000000}\selectfont\catcode`\^=\active\def^{\ifmmode\sp\else\^{}\fi}\catcode`\%=\active\def%{\%}$T = T_C$}}%
\end{pgfscope}%
\end{pgfpicture}%
\makeatother%
\endgroup%

	\end{subfigure}%
	\begin{subfigure}[b]{0.5\textwidth}
		\centering
		\caption{\hfill\null}\label{sfig:Ginzburg Landau free energy}
		%% Creator: Matplotlib, PGF backend
%%
%% To include the figure in your LaTeX document, write
%%   \input{<filename>.pgf}
%%
%% Make sure the required packages are loaded in your preamble
%%   \usepackage{pgf}
%%
%% Also ensure that all the required font packages are loaded; for instance,
%% the lmodern package is sometimes necessary when using math font.
%%   \usepackage{lmodern}
%%
%% Figures using additional raster images can only be included by \input if
%% they are in the same directory as the main LaTeX file. For loading figures
%% from other directories you can use the `import` package
%%   \usepackage{import}
%%
%% and then include the figures with
%%   \import{<path to file>}{<filename>.pgf}
%%
%% Matplotlib used the following preamble
%%   \def\mathdefault#1{#1}
%%   \everymath=\expandafter{\the\everymath\displaystyle}
%%   \IfFileExists{scrextend.sty}{
%%     \usepackage[fontsize=11.000000pt]{scrextend}
%%   }{
%%     \renewcommand{\normalsize}{\fontsize{11.000000}{13.200000}\selectfont}
%%     \normalsize
%%   }
%%   \usepackage{fontspec}\usepackage{unicode-math}\setmathfont{texgyrepagella-math.otf}\setmainfont{texgyrepagella-math}\usepackage{nicefrac}
%%   \makeatletter\@ifpackageloaded{underscore}{}{\usepackage[strings]{underscore}}\makeatother
%%
\begingroup%
\makeatletter%
\begin{pgfpicture}%
\pgfpathrectangle{\pgfpointorigin}{\pgfqpoint{2.300000in}{2.300000in}}%
\pgfusepath{use as bounding box, clip}%
\begin{pgfscope}%
\pgfsetbuttcap%
\pgfsetmiterjoin%
\definecolor{currentfill}{rgb}{1.000000,1.000000,1.000000}%
\pgfsetfillcolor{currentfill}%
\pgfsetlinewidth{0.000000pt}%
\definecolor{currentstroke}{rgb}{1.000000,1.000000,1.000000}%
\pgfsetstrokecolor{currentstroke}%
\pgfsetdash{}{0pt}%
\pgfpathmoveto{\pgfqpoint{0.000000in}{0.000000in}}%
\pgfpathlineto{\pgfqpoint{2.300000in}{0.000000in}}%
\pgfpathlineto{\pgfqpoint{2.300000in}{2.300000in}}%
\pgfpathlineto{\pgfqpoint{0.000000in}{2.300000in}}%
\pgfpathlineto{\pgfqpoint{0.000000in}{0.000000in}}%
\pgfpathclose%
\pgfusepath{fill}%
\end{pgfscope}%
\begin{pgfscope}%
\pgfsetbuttcap%
\pgfsetmiterjoin%
\definecolor{currentfill}{rgb}{1.000000,1.000000,1.000000}%
\pgfsetfillcolor{currentfill}%
\pgfsetlinewidth{0.000000pt}%
\definecolor{currentstroke}{rgb}{0.000000,0.000000,0.000000}%
\pgfsetstrokecolor{currentstroke}%
\pgfsetstrokeopacity{0.000000}%
\pgfsetdash{}{0pt}%
\pgfpathmoveto{\pgfqpoint{0.041670in}{0.041670in}}%
\pgfpathlineto{\pgfqpoint{2.258330in}{0.041670in}}%
\pgfpathlineto{\pgfqpoint{2.258330in}{2.258330in}}%
\pgfpathlineto{\pgfqpoint{0.041670in}{2.258330in}}%
\pgfpathlineto{\pgfqpoint{0.041670in}{0.041670in}}%
\pgfpathclose%
\pgfusepath{fill}%
\end{pgfscope}%
\begin{pgfscope}%
\pgfpathrectangle{\pgfqpoint{0.041670in}{0.041670in}}{\pgfqpoint{2.216660in}{2.216660in}}%
\pgfusepath{clip}%
\pgfsetbuttcap%
\pgfsetroundjoin%
\pgfsetlinewidth{1.505625pt}%
\definecolor{currentstroke}{rgb}{0.000000,0.000000,0.000000}%
\pgfsetstrokecolor{currentstroke}%
\pgfsetdash{}{0pt}%
\pgfpathmoveto{\pgfqpoint{0.076478in}{1.231946in}}%
\pgfpathlineto{\pgfqpoint{1.179955in}{1.026681in}}%
\pgfusepath{stroke}%
\end{pgfscope}%
\begin{pgfscope}%
\pgfpathrectangle{\pgfqpoint{0.041670in}{0.041670in}}{\pgfqpoint{2.216660in}{2.216660in}}%
\pgfusepath{clip}%
\pgfsetbuttcap%
\pgfsetroundjoin%
\pgfsetlinewidth{1.505625pt}%
\definecolor{currentstroke}{rgb}{0.000000,0.000000,0.000000}%
\pgfsetstrokecolor{currentstroke}%
\pgfsetdash{}{0pt}%
\pgfpathmoveto{\pgfqpoint{0.076478in}{1.231946in}}%
\pgfpathlineto{\pgfqpoint{0.132289in}{1.143372in}}%
\pgfusepath{stroke}%
\end{pgfscope}%
\begin{pgfscope}%
\pgfpathrectangle{\pgfqpoint{0.041670in}{0.041670in}}{\pgfqpoint{2.216660in}{2.216660in}}%
\pgfusepath{clip}%
\pgfsetbuttcap%
\pgfsetroundjoin%
\pgfsetlinewidth{1.505625pt}%
\definecolor{currentstroke}{rgb}{0.000000,0.000000,0.000000}%
\pgfsetstrokecolor{currentstroke}%
\pgfsetdash{}{0pt}%
\pgfpathmoveto{\pgfqpoint{0.076478in}{1.231946in}}%
\pgfpathlineto{\pgfqpoint{0.124275in}{1.301846in}}%
\pgfusepath{stroke}%
\end{pgfscope}%
\begin{pgfscope}%
\definecolor{textcolor}{rgb}{0.000000,0.000000,0.000000}%
\pgfsetstrokecolor{textcolor}%
\pgfsetfillcolor{textcolor}%
\pgftext[x=0.127305in,y=1.252121in,left,base]{\color{textcolor}{\rmfamily\fontsize{11.000000}{13.200000}\selectfont\catcode`\^=\active\def^{\ifmmode\sp\else\^{}\fi}\catcode`\%=\active\def%{\%}$\Re \Psi$}}%
\end{pgfscope}%
\begin{pgfscope}%
\definecolor{textcolor}{rgb}{0.000000,0.000000,0.000000}%
\pgfsetstrokecolor{textcolor}%
\pgfsetfillcolor{textcolor}%
\pgftext[x=0.802618in,y=0.366767in,left,base]{\color{textcolor}{\rmfamily\fontsize{11.000000}{13.200000}\selectfont\catcode`\^=\active\def^{\ifmmode\sp\else\^{}\fi}\catcode`\%=\active\def%{\%}$\Im \Psi$}}%
\end{pgfscope}%
\begin{pgfscope}%
\definecolor{textcolor}{rgb}{0.000000,0.000000,0.000000}%
\pgfsetstrokecolor{textcolor}%
\pgfsetfillcolor{textcolor}%
\pgftext[x=1.242222in,y=1.959229in,left,base]{\color{textcolor}{\rmfamily\fontsize{11.000000}{13.200000}\selectfont\catcode`\^=\active\def^{\ifmmode\sp\else\^{}\fi}\catcode`\%=\active\def%{\%}$f_{\mathrm{L}}$}}%
\end{pgfscope}%
\begin{pgfscope}%
\pgfpathrectangle{\pgfqpoint{0.041670in}{0.041670in}}{\pgfqpoint{2.216660in}{2.216660in}}%
\pgfusepath{clip}%
\pgfsetbuttcap%
\pgfsetroundjoin%
\pgfsetlinewidth{1.505625pt}%
\definecolor{currentstroke}{rgb}{0.000000,0.000000,0.000000}%
\pgfsetstrokecolor{currentstroke}%
\pgfsetdash{}{0pt}%
\pgfpathmoveto{\pgfqpoint{1.179955in}{1.981861in}}%
\pgfpathlineto{\pgfqpoint{1.179955in}{1.026681in}}%
\pgfusepath{stroke}%
\end{pgfscope}%
\begin{pgfscope}%
\pgfpathrectangle{\pgfqpoint{0.041670in}{0.041670in}}{\pgfqpoint{2.216660in}{2.216660in}}%
\pgfusepath{clip}%
\pgfsetbuttcap%
\pgfsetroundjoin%
\pgfsetlinewidth{1.505625pt}%
\definecolor{currentstroke}{rgb}{0.000000,0.000000,0.000000}%
\pgfsetstrokecolor{currentstroke}%
\pgfsetdash{}{0pt}%
\pgfpathmoveto{\pgfqpoint{1.179955in}{1.981861in}}%
\pgfpathlineto{\pgfqpoint{1.173245in}{1.839120in}}%
\pgfusepath{stroke}%
\end{pgfscope}%
\begin{pgfscope}%
\pgfpathrectangle{\pgfqpoint{0.041670in}{0.041670in}}{\pgfqpoint{2.216660in}{2.216660in}}%
\pgfusepath{clip}%
\pgfsetbuttcap%
\pgfsetroundjoin%
\pgfsetlinewidth{1.505625pt}%
\definecolor{currentstroke}{rgb}{0.000000,0.000000,0.000000}%
\pgfsetstrokecolor{currentstroke}%
\pgfsetdash{}{0pt}%
\pgfpathmoveto{\pgfqpoint{1.179955in}{1.981861in}}%
\pgfpathlineto{\pgfqpoint{1.186667in}{1.836910in}}%
\pgfusepath{stroke}%
\end{pgfscope}%
\begin{pgfscope}%
\pgfpathrectangle{\pgfqpoint{0.041670in}{0.041670in}}{\pgfqpoint{2.216660in}{2.216660in}}%
\pgfusepath{clip}%
\pgfsetbuttcap%
\pgfsetroundjoin%
\definecolor{currentfill}{rgb}{0.282327,0.094955,0.417331}%
\pgfsetfillcolor{currentfill}%
\pgfsetlinewidth{0.000000pt}%
\definecolor{currentstroke}{rgb}{0.000000,0.000000,0.000000}%
\pgfsetstrokecolor{currentstroke}%
\pgfsetdash{}{0pt}%
\pgfpathmoveto{\pgfqpoint{1.102912in}{1.232895in}}%
\pgfpathlineto{\pgfqpoint{1.102466in}{1.239487in}}%
\pgfpathlineto{\pgfqpoint{1.102019in}{1.246406in}}%
\pgfpathlineto{\pgfqpoint{1.101570in}{1.253660in}}%
\pgfpathlineto{\pgfqpoint{1.101121in}{1.261252in}}%
\pgfpathlineto{\pgfqpoint{1.121049in}{1.262326in}}%
\pgfpathlineto{\pgfqpoint{1.141027in}{1.263089in}}%
\pgfpathlineto{\pgfqpoint{1.161039in}{1.263540in}}%
\pgfpathlineto{\pgfqpoint{1.181068in}{1.263679in}}%
\pgfpathlineto{\pgfqpoint{1.181061in}{1.256076in}}%
\pgfpathlineto{\pgfqpoint{1.181055in}{1.248811in}}%
\pgfpathlineto{\pgfqpoint{1.181049in}{1.241880in}}%
\pgfpathlineto{\pgfqpoint{1.181042in}{1.235277in}}%
\pgfpathlineto{\pgfqpoint{1.161469in}{1.235140in}}%
\pgfpathlineto{\pgfqpoint{1.141911in}{1.234698in}}%
\pgfpathlineto{\pgfqpoint{1.122387in}{1.233949in}}%
\pgfpathlineto{\pgfqpoint{1.102912in}{1.232895in}}%
\pgfpathclose%
\pgfusepath{fill}%
\end{pgfscope}%
\begin{pgfscope}%
\pgfpathrectangle{\pgfqpoint{0.041670in}{0.041670in}}{\pgfqpoint{2.216660in}{2.216660in}}%
\pgfusepath{clip}%
\pgfsetbuttcap%
\pgfsetroundjoin%
\definecolor{currentfill}{rgb}{0.282327,0.094955,0.417331}%
\pgfsetfillcolor{currentfill}%
\pgfsetlinewidth{0.000000pt}%
\definecolor{currentstroke}{rgb}{0.000000,0.000000,0.000000}%
\pgfsetstrokecolor{currentstroke}%
\pgfsetdash{}{0pt}%
\pgfpathmoveto{\pgfqpoint{1.181042in}{1.235277in}}%
\pgfpathlineto{\pgfqpoint{1.181049in}{1.241880in}}%
\pgfpathlineto{\pgfqpoint{1.181055in}{1.248811in}}%
\pgfpathlineto{\pgfqpoint{1.181061in}{1.256076in}}%
\pgfpathlineto{\pgfqpoint{1.181068in}{1.263679in}}%
\pgfpathlineto{\pgfqpoint{1.201095in}{1.263506in}}%
\pgfpathlineto{\pgfqpoint{1.221104in}{1.263020in}}%
\pgfpathlineto{\pgfqpoint{1.241078in}{1.262222in}}%
\pgfpathlineto{\pgfqpoint{1.260999in}{1.261113in}}%
\pgfpathlineto{\pgfqpoint{1.260537in}{1.253522in}}%
\pgfpathlineto{\pgfqpoint{1.260076in}{1.246269in}}%
\pgfpathlineto{\pgfqpoint{1.259617in}{1.239350in}}%
\pgfpathlineto{\pgfqpoint{1.259158in}{1.232759in}}%
\pgfpathlineto{\pgfqpoint{1.239690in}{1.233847in}}%
\pgfpathlineto{\pgfqpoint{1.220170in}{1.234630in}}%
\pgfpathlineto{\pgfqpoint{1.200615in}{1.235106in}}%
\pgfpathlineto{\pgfqpoint{1.181042in}{1.235277in}}%
\pgfpathclose%
\pgfusepath{fill}%
\end{pgfscope}%
\begin{pgfscope}%
\pgfpathrectangle{\pgfqpoint{0.041670in}{0.041670in}}{\pgfqpoint{2.216660in}{2.216660in}}%
\pgfusepath{clip}%
\pgfsetbuttcap%
\pgfsetroundjoin%
\definecolor{currentfill}{rgb}{0.282884,0.135920,0.453427}%
\pgfsetfillcolor{currentfill}%
\pgfsetlinewidth{0.000000pt}%
\definecolor{currentstroke}{rgb}{0.000000,0.000000,0.000000}%
\pgfsetstrokecolor{currentstroke}%
\pgfsetdash{}{0pt}%
\pgfpathmoveto{\pgfqpoint{1.101121in}{1.261252in}}%
\pgfpathlineto{\pgfqpoint{1.100670in}{1.269189in}}%
\pgfpathlineto{\pgfqpoint{1.100218in}{1.277476in}}%
\pgfpathlineto{\pgfqpoint{1.099764in}{1.286119in}}%
\pgfpathlineto{\pgfqpoint{1.099309in}{1.295125in}}%
\pgfpathlineto{\pgfqpoint{1.119695in}{1.296219in}}%
\pgfpathlineto{\pgfqpoint{1.140133in}{1.296995in}}%
\pgfpathlineto{\pgfqpoint{1.160605in}{1.297454in}}%
\pgfpathlineto{\pgfqpoint{1.181093in}{1.297596in}}%
\pgfpathlineto{\pgfqpoint{1.181087in}{1.288580in}}%
\pgfpathlineto{\pgfqpoint{1.181080in}{1.279925in}}%
\pgfpathlineto{\pgfqpoint{1.181074in}{1.271627in}}%
\pgfpathlineto{\pgfqpoint{1.181068in}{1.263679in}}%
\pgfpathlineto{\pgfqpoint{1.161039in}{1.263540in}}%
\pgfpathlineto{\pgfqpoint{1.141027in}{1.263089in}}%
\pgfpathlineto{\pgfqpoint{1.121049in}{1.262326in}}%
\pgfpathlineto{\pgfqpoint{1.101121in}{1.261252in}}%
\pgfpathclose%
\pgfusepath{fill}%
\end{pgfscope}%
\begin{pgfscope}%
\pgfpathrectangle{\pgfqpoint{0.041670in}{0.041670in}}{\pgfqpoint{2.216660in}{2.216660in}}%
\pgfusepath{clip}%
\pgfsetbuttcap%
\pgfsetroundjoin%
\definecolor{currentfill}{rgb}{0.282884,0.135920,0.453427}%
\pgfsetfillcolor{currentfill}%
\pgfsetlinewidth{0.000000pt}%
\definecolor{currentstroke}{rgb}{0.000000,0.000000,0.000000}%
\pgfsetstrokecolor{currentstroke}%
\pgfsetdash{}{0pt}%
\pgfpathmoveto{\pgfqpoint{1.181068in}{1.263679in}}%
\pgfpathlineto{\pgfqpoint{1.181074in}{1.271627in}}%
\pgfpathlineto{\pgfqpoint{1.181080in}{1.279925in}}%
\pgfpathlineto{\pgfqpoint{1.181087in}{1.288580in}}%
\pgfpathlineto{\pgfqpoint{1.181093in}{1.297596in}}%
\pgfpathlineto{\pgfqpoint{1.201581in}{1.297419in}}%
\pgfpathlineto{\pgfqpoint{1.222050in}{1.296925in}}%
\pgfpathlineto{\pgfqpoint{1.242483in}{1.296113in}}%
\pgfpathlineto{\pgfqpoint{1.262862in}{1.294984in}}%
\pgfpathlineto{\pgfqpoint{1.262394in}{1.285979in}}%
\pgfpathlineto{\pgfqpoint{1.261928in}{1.277336in}}%
\pgfpathlineto{\pgfqpoint{1.261463in}{1.269049in}}%
\pgfpathlineto{\pgfqpoint{1.260999in}{1.261113in}}%
\pgfpathlineto{\pgfqpoint{1.241078in}{1.262222in}}%
\pgfpathlineto{\pgfqpoint{1.221104in}{1.263020in}}%
\pgfpathlineto{\pgfqpoint{1.201095in}{1.263506in}}%
\pgfpathlineto{\pgfqpoint{1.181068in}{1.263679in}}%
\pgfpathclose%
\pgfusepath{fill}%
\end{pgfscope}%
\begin{pgfscope}%
\pgfpathrectangle{\pgfqpoint{0.041670in}{0.041670in}}{\pgfqpoint{2.216660in}{2.216660in}}%
\pgfusepath{clip}%
\pgfsetbuttcap%
\pgfsetroundjoin%
\definecolor{currentfill}{rgb}{0.277941,0.056324,0.381191}%
\pgfsetfillcolor{currentfill}%
\pgfsetlinewidth{0.000000pt}%
\definecolor{currentstroke}{rgb}{0.000000,0.000000,0.000000}%
\pgfsetstrokecolor{currentstroke}%
\pgfsetdash{}{0pt}%
\pgfpathmoveto{\pgfqpoint{1.104684in}{1.209697in}}%
\pgfpathlineto{\pgfqpoint{1.104242in}{1.215031in}}%
\pgfpathlineto{\pgfqpoint{1.103800in}{1.220672in}}%
\pgfpathlineto{\pgfqpoint{1.103356in}{1.226625in}}%
\pgfpathlineto{\pgfqpoint{1.102912in}{1.232895in}}%
\pgfpathlineto{\pgfqpoint{1.122387in}{1.233949in}}%
\pgfpathlineto{\pgfqpoint{1.141911in}{1.234698in}}%
\pgfpathlineto{\pgfqpoint{1.161469in}{1.235140in}}%
\pgfpathlineto{\pgfqpoint{1.181042in}{1.235277in}}%
\pgfpathlineto{\pgfqpoint{1.181036in}{1.228996in}}%
\pgfpathlineto{\pgfqpoint{1.181030in}{1.223031in}}%
\pgfpathlineto{\pgfqpoint{1.181024in}{1.217378in}}%
\pgfpathlineto{\pgfqpoint{1.181017in}{1.212032in}}%
\pgfpathlineto{\pgfqpoint{1.161894in}{1.211899in}}%
\pgfpathlineto{\pgfqpoint{1.142786in}{1.211465in}}%
\pgfpathlineto{\pgfqpoint{1.123710in}{1.210730in}}%
\pgfpathlineto{\pgfqpoint{1.104684in}{1.209697in}}%
\pgfpathclose%
\pgfusepath{fill}%
\end{pgfscope}%
\begin{pgfscope}%
\pgfpathrectangle{\pgfqpoint{0.041670in}{0.041670in}}{\pgfqpoint{2.216660in}{2.216660in}}%
\pgfusepath{clip}%
\pgfsetbuttcap%
\pgfsetroundjoin%
\definecolor{currentfill}{rgb}{0.277941,0.056324,0.381191}%
\pgfsetfillcolor{currentfill}%
\pgfsetlinewidth{0.000000pt}%
\definecolor{currentstroke}{rgb}{0.000000,0.000000,0.000000}%
\pgfsetstrokecolor{currentstroke}%
\pgfsetdash{}{0pt}%
\pgfpathmoveto{\pgfqpoint{1.181017in}{1.212032in}}%
\pgfpathlineto{\pgfqpoint{1.181024in}{1.217378in}}%
\pgfpathlineto{\pgfqpoint{1.181030in}{1.223031in}}%
\pgfpathlineto{\pgfqpoint{1.181036in}{1.228996in}}%
\pgfpathlineto{\pgfqpoint{1.181042in}{1.235277in}}%
\pgfpathlineto{\pgfqpoint{1.200615in}{1.235106in}}%
\pgfpathlineto{\pgfqpoint{1.220170in}{1.234630in}}%
\pgfpathlineto{\pgfqpoint{1.239690in}{1.233847in}}%
\pgfpathlineto{\pgfqpoint{1.259158in}{1.232759in}}%
\pgfpathlineto{\pgfqpoint{1.258701in}{1.226490in}}%
\pgfpathlineto{\pgfqpoint{1.258245in}{1.220538in}}%
\pgfpathlineto{\pgfqpoint{1.257790in}{1.214897in}}%
\pgfpathlineto{\pgfqpoint{1.257337in}{1.209564in}}%
\pgfpathlineto{\pgfqpoint{1.238316in}{1.210630in}}%
\pgfpathlineto{\pgfqpoint{1.219245in}{1.211398in}}%
\pgfpathlineto{\pgfqpoint{1.200140in}{1.211865in}}%
\pgfpathlineto{\pgfqpoint{1.181017in}{1.212032in}}%
\pgfpathclose%
\pgfusepath{fill}%
\end{pgfscope}%
\begin{pgfscope}%
\pgfpathrectangle{\pgfqpoint{0.041670in}{0.041670in}}{\pgfqpoint{2.216660in}{2.216660in}}%
\pgfusepath{clip}%
\pgfsetbuttcap%
\pgfsetroundjoin%
\definecolor{currentfill}{rgb}{0.276194,0.190074,0.493001}%
\pgfsetfillcolor{currentfill}%
\pgfsetlinewidth{0.000000pt}%
\definecolor{currentstroke}{rgb}{0.000000,0.000000,0.000000}%
\pgfsetstrokecolor{currentstroke}%
\pgfsetdash{}{0pt}%
\pgfpathmoveto{\pgfqpoint{1.099309in}{1.295125in}}%
\pgfpathlineto{\pgfqpoint{1.098852in}{1.304498in}}%
\pgfpathlineto{\pgfqpoint{1.098394in}{1.314246in}}%
\pgfpathlineto{\pgfqpoint{1.097934in}{1.324374in}}%
\pgfpathlineto{\pgfqpoint{1.097473in}{1.334888in}}%
\pgfpathlineto{\pgfqpoint{1.118323in}{1.336000in}}%
\pgfpathlineto{\pgfqpoint{1.139226in}{1.336790in}}%
\pgfpathlineto{\pgfqpoint{1.160164in}{1.337257in}}%
\pgfpathlineto{\pgfqpoint{1.181119in}{1.337400in}}%
\pgfpathlineto{\pgfqpoint{1.181113in}{1.326876in}}%
\pgfpathlineto{\pgfqpoint{1.181106in}{1.316738in}}%
\pgfpathlineto{\pgfqpoint{1.181100in}{1.306980in}}%
\pgfpathlineto{\pgfqpoint{1.181093in}{1.297596in}}%
\pgfpathlineto{\pgfqpoint{1.160605in}{1.297454in}}%
\pgfpathlineto{\pgfqpoint{1.140133in}{1.296995in}}%
\pgfpathlineto{\pgfqpoint{1.119695in}{1.296219in}}%
\pgfpathlineto{\pgfqpoint{1.099309in}{1.295125in}}%
\pgfpathclose%
\pgfusepath{fill}%
\end{pgfscope}%
\begin{pgfscope}%
\pgfpathrectangle{\pgfqpoint{0.041670in}{0.041670in}}{\pgfqpoint{2.216660in}{2.216660in}}%
\pgfusepath{clip}%
\pgfsetbuttcap%
\pgfsetroundjoin%
\definecolor{currentfill}{rgb}{0.276194,0.190074,0.493001}%
\pgfsetfillcolor{currentfill}%
\pgfsetlinewidth{0.000000pt}%
\definecolor{currentstroke}{rgb}{0.000000,0.000000,0.000000}%
\pgfsetstrokecolor{currentstroke}%
\pgfsetdash{}{0pt}%
\pgfpathmoveto{\pgfqpoint{1.181093in}{1.297596in}}%
\pgfpathlineto{\pgfqpoint{1.181100in}{1.306980in}}%
\pgfpathlineto{\pgfqpoint{1.181106in}{1.316738in}}%
\pgfpathlineto{\pgfqpoint{1.181113in}{1.326876in}}%
\pgfpathlineto{\pgfqpoint{1.181119in}{1.337400in}}%
\pgfpathlineto{\pgfqpoint{1.202073in}{1.337221in}}%
\pgfpathlineto{\pgfqpoint{1.223008in}{1.336718in}}%
\pgfpathlineto{\pgfqpoint{1.243906in}{1.335892in}}%
\pgfpathlineto{\pgfqpoint{1.264750in}{1.334745in}}%
\pgfpathlineto{\pgfqpoint{1.264275in}{1.324231in}}%
\pgfpathlineto{\pgfqpoint{1.263803in}{1.314104in}}%
\pgfpathlineto{\pgfqpoint{1.263332in}{1.304357in}}%
\pgfpathlineto{\pgfqpoint{1.262862in}{1.294984in}}%
\pgfpathlineto{\pgfqpoint{1.242483in}{1.296113in}}%
\pgfpathlineto{\pgfqpoint{1.222050in}{1.296925in}}%
\pgfpathlineto{\pgfqpoint{1.201581in}{1.297419in}}%
\pgfpathlineto{\pgfqpoint{1.181093in}{1.297596in}}%
\pgfpathclose%
\pgfusepath{fill}%
\end{pgfscope}%
\begin{pgfscope}%
\pgfpathrectangle{\pgfqpoint{0.041670in}{0.041670in}}{\pgfqpoint{2.216660in}{2.216660in}}%
\pgfusepath{clip}%
\pgfsetbuttcap%
\pgfsetroundjoin%
\definecolor{currentfill}{rgb}{0.272594,0.025563,0.353093}%
\pgfsetfillcolor{currentfill}%
\pgfsetlinewidth{0.000000pt}%
\definecolor{currentstroke}{rgb}{0.000000,0.000000,0.000000}%
\pgfsetstrokecolor{currentstroke}%
\pgfsetdash{}{0pt}%
\pgfpathmoveto{\pgfqpoint{1.106439in}{1.191320in}}%
\pgfpathlineto{\pgfqpoint{1.106002in}{1.195480in}}%
\pgfpathlineto{\pgfqpoint{1.105563in}{1.199926in}}%
\pgfpathlineto{\pgfqpoint{1.105124in}{1.204663in}}%
\pgfpathlineto{\pgfqpoint{1.104684in}{1.209697in}}%
\pgfpathlineto{\pgfqpoint{1.123710in}{1.210730in}}%
\pgfpathlineto{\pgfqpoint{1.142786in}{1.211465in}}%
\pgfpathlineto{\pgfqpoint{1.161894in}{1.211899in}}%
\pgfpathlineto{\pgfqpoint{1.181017in}{1.212032in}}%
\pgfpathlineto{\pgfqpoint{1.181011in}{1.206987in}}%
\pgfpathlineto{\pgfqpoint{1.181005in}{1.202237in}}%
\pgfpathlineto{\pgfqpoint{1.180999in}{1.197779in}}%
\pgfpathlineto{\pgfqpoint{1.180993in}{1.193606in}}%
\pgfpathlineto{\pgfqpoint{1.162315in}{1.193476in}}%
\pgfpathlineto{\pgfqpoint{1.143653in}{1.193051in}}%
\pgfpathlineto{\pgfqpoint{1.125022in}{1.192332in}}%
\pgfpathlineto{\pgfqpoint{1.106439in}{1.191320in}}%
\pgfpathclose%
\pgfusepath{fill}%
\end{pgfscope}%
\begin{pgfscope}%
\pgfpathrectangle{\pgfqpoint{0.041670in}{0.041670in}}{\pgfqpoint{2.216660in}{2.216660in}}%
\pgfusepath{clip}%
\pgfsetbuttcap%
\pgfsetroundjoin%
\definecolor{currentfill}{rgb}{0.272594,0.025563,0.353093}%
\pgfsetfillcolor{currentfill}%
\pgfsetlinewidth{0.000000pt}%
\definecolor{currentstroke}{rgb}{0.000000,0.000000,0.000000}%
\pgfsetstrokecolor{currentstroke}%
\pgfsetdash{}{0pt}%
\pgfpathmoveto{\pgfqpoint{1.180993in}{1.193606in}}%
\pgfpathlineto{\pgfqpoint{1.180999in}{1.197779in}}%
\pgfpathlineto{\pgfqpoint{1.181005in}{1.202237in}}%
\pgfpathlineto{\pgfqpoint{1.181011in}{1.206987in}}%
\pgfpathlineto{\pgfqpoint{1.181017in}{1.212032in}}%
\pgfpathlineto{\pgfqpoint{1.200140in}{1.211865in}}%
\pgfpathlineto{\pgfqpoint{1.219245in}{1.211398in}}%
\pgfpathlineto{\pgfqpoint{1.238316in}{1.210630in}}%
\pgfpathlineto{\pgfqpoint{1.257337in}{1.209564in}}%
\pgfpathlineto{\pgfqpoint{1.256884in}{1.204531in}}%
\pgfpathlineto{\pgfqpoint{1.256432in}{1.199794in}}%
\pgfpathlineto{\pgfqpoint{1.255981in}{1.195349in}}%
\pgfpathlineto{\pgfqpoint{1.255531in}{1.191189in}}%
\pgfpathlineto{\pgfqpoint{1.236955in}{1.192234in}}%
\pgfpathlineto{\pgfqpoint{1.218329in}{1.192985in}}%
\pgfpathlineto{\pgfqpoint{1.199669in}{1.193443in}}%
\pgfpathlineto{\pgfqpoint{1.180993in}{1.193606in}}%
\pgfpathclose%
\pgfusepath{fill}%
\end{pgfscope}%
\begin{pgfscope}%
\pgfpathrectangle{\pgfqpoint{0.041670in}{0.041670in}}{\pgfqpoint{2.216660in}{2.216660in}}%
\pgfusepath{clip}%
\pgfsetbuttcap%
\pgfsetroundjoin%
\definecolor{currentfill}{rgb}{0.282327,0.094955,0.417331}%
\pgfsetfillcolor{currentfill}%
\pgfsetlinewidth{0.000000pt}%
\definecolor{currentstroke}{rgb}{0.000000,0.000000,0.000000}%
\pgfsetstrokecolor{currentstroke}%
\pgfsetdash{}{0pt}%
\pgfpathmoveto{\pgfqpoint{1.025845in}{1.225640in}}%
\pgfpathlineto{\pgfqpoint{1.024952in}{1.232197in}}%
\pgfpathlineto{\pgfqpoint{1.024057in}{1.239082in}}%
\pgfpathlineto{\pgfqpoint{1.023159in}{1.246301in}}%
\pgfpathlineto{\pgfqpoint{1.022259in}{1.253859in}}%
\pgfpathlineto{\pgfqpoint{1.041813in}{1.256169in}}%
\pgfpathlineto{\pgfqpoint{1.061486in}{1.258172in}}%
\pgfpathlineto{\pgfqpoint{1.081261in}{1.259867in}}%
\pgfpathlineto{\pgfqpoint{1.101121in}{1.261252in}}%
\pgfpathlineto{\pgfqpoint{1.101570in}{1.253660in}}%
\pgfpathlineto{\pgfqpoint{1.102019in}{1.246406in}}%
\pgfpathlineto{\pgfqpoint{1.102466in}{1.239487in}}%
\pgfpathlineto{\pgfqpoint{1.102912in}{1.232895in}}%
\pgfpathlineto{\pgfqpoint{1.083503in}{1.231535in}}%
\pgfpathlineto{\pgfqpoint{1.064178in}{1.229872in}}%
\pgfpathlineto{\pgfqpoint{1.044953in}{1.227907in}}%
\pgfpathlineto{\pgfqpoint{1.025845in}{1.225640in}}%
\pgfpathclose%
\pgfusepath{fill}%
\end{pgfscope}%
\begin{pgfscope}%
\pgfpathrectangle{\pgfqpoint{0.041670in}{0.041670in}}{\pgfqpoint{2.216660in}{2.216660in}}%
\pgfusepath{clip}%
\pgfsetbuttcap%
\pgfsetroundjoin%
\definecolor{currentfill}{rgb}{0.282327,0.094955,0.417331}%
\pgfsetfillcolor{currentfill}%
\pgfsetlinewidth{0.000000pt}%
\definecolor{currentstroke}{rgb}{0.000000,0.000000,0.000000}%
\pgfsetstrokecolor{currentstroke}%
\pgfsetdash{}{0pt}%
\pgfpathmoveto{\pgfqpoint{1.259158in}{1.232759in}}%
\pgfpathlineto{\pgfqpoint{1.259617in}{1.239350in}}%
\pgfpathlineto{\pgfqpoint{1.260076in}{1.246269in}}%
\pgfpathlineto{\pgfqpoint{1.260537in}{1.253522in}}%
\pgfpathlineto{\pgfqpoint{1.260999in}{1.261113in}}%
\pgfpathlineto{\pgfqpoint{1.280851in}{1.259694in}}%
\pgfpathlineto{\pgfqpoint{1.300615in}{1.257965in}}%
\pgfpathlineto{\pgfqpoint{1.320276in}{1.255927in}}%
\pgfpathlineto{\pgfqpoint{1.339815in}{1.253583in}}%
\pgfpathlineto{\pgfqpoint{1.338903in}{1.246027in}}%
\pgfpathlineto{\pgfqpoint{1.337993in}{1.238809in}}%
\pgfpathlineto{\pgfqpoint{1.337085in}{1.231925in}}%
\pgfpathlineto{\pgfqpoint{1.336180in}{1.225369in}}%
\pgfpathlineto{\pgfqpoint{1.317086in}{1.227670in}}%
\pgfpathlineto{\pgfqpoint{1.297873in}{1.229669in}}%
\pgfpathlineto{\pgfqpoint{1.278558in}{1.231365in}}%
\pgfpathlineto{\pgfqpoint{1.259158in}{1.232759in}}%
\pgfpathclose%
\pgfusepath{fill}%
\end{pgfscope}%
\begin{pgfscope}%
\pgfpathrectangle{\pgfqpoint{0.041670in}{0.041670in}}{\pgfqpoint{2.216660in}{2.216660in}}%
\pgfusepath{clip}%
\pgfsetbuttcap%
\pgfsetroundjoin%
\definecolor{currentfill}{rgb}{0.277941,0.056324,0.381191}%
\pgfsetfillcolor{currentfill}%
\pgfsetlinewidth{0.000000pt}%
\definecolor{currentstroke}{rgb}{0.000000,0.000000,0.000000}%
\pgfsetstrokecolor{currentstroke}%
\pgfsetdash{}{0pt}%
\pgfpathmoveto{\pgfqpoint{1.029393in}{1.202585in}}%
\pgfpathlineto{\pgfqpoint{1.028509in}{1.207883in}}%
\pgfpathlineto{\pgfqpoint{1.027623in}{1.213489in}}%
\pgfpathlineto{\pgfqpoint{1.026735in}{1.219405in}}%
\pgfpathlineto{\pgfqpoint{1.025845in}{1.225640in}}%
\pgfpathlineto{\pgfqpoint{1.044953in}{1.227907in}}%
\pgfpathlineto{\pgfqpoint{1.064178in}{1.229872in}}%
\pgfpathlineto{\pgfqpoint{1.083503in}{1.231535in}}%
\pgfpathlineto{\pgfqpoint{1.102912in}{1.232895in}}%
\pgfpathlineto{\pgfqpoint{1.103356in}{1.226625in}}%
\pgfpathlineto{\pgfqpoint{1.103800in}{1.220672in}}%
\pgfpathlineto{\pgfqpoint{1.104242in}{1.215031in}}%
\pgfpathlineto{\pgfqpoint{1.104684in}{1.209697in}}%
\pgfpathlineto{\pgfqpoint{1.085722in}{1.208364in}}%
\pgfpathlineto{\pgfqpoint{1.066842in}{1.206734in}}%
\pgfpathlineto{\pgfqpoint{1.048060in}{1.204807in}}%
\pgfpathlineto{\pgfqpoint{1.029393in}{1.202585in}}%
\pgfpathclose%
\pgfusepath{fill}%
\end{pgfscope}%
\begin{pgfscope}%
\pgfpathrectangle{\pgfqpoint{0.041670in}{0.041670in}}{\pgfqpoint{2.216660in}{2.216660in}}%
\pgfusepath{clip}%
\pgfsetbuttcap%
\pgfsetroundjoin%
\definecolor{currentfill}{rgb}{0.282884,0.135920,0.453427}%
\pgfsetfillcolor{currentfill}%
\pgfsetlinewidth{0.000000pt}%
\definecolor{currentstroke}{rgb}{0.000000,0.000000,0.000000}%
\pgfsetstrokecolor{currentstroke}%
\pgfsetdash{}{0pt}%
\pgfpathmoveto{\pgfqpoint{1.022259in}{1.253859in}}%
\pgfpathlineto{\pgfqpoint{1.021356in}{1.261762in}}%
\pgfpathlineto{\pgfqpoint{1.020450in}{1.270016in}}%
\pgfpathlineto{\pgfqpoint{1.019542in}{1.278626in}}%
\pgfpathlineto{\pgfqpoint{1.018630in}{1.287599in}}%
\pgfpathlineto{\pgfqpoint{1.038635in}{1.289951in}}%
\pgfpathlineto{\pgfqpoint{1.058761in}{1.291990in}}%
\pgfpathlineto{\pgfqpoint{1.078992in}{1.293715in}}%
\pgfpathlineto{\pgfqpoint{1.099309in}{1.295125in}}%
\pgfpathlineto{\pgfqpoint{1.099764in}{1.286119in}}%
\pgfpathlineto{\pgfqpoint{1.100218in}{1.277476in}}%
\pgfpathlineto{\pgfqpoint{1.100670in}{1.269189in}}%
\pgfpathlineto{\pgfqpoint{1.101121in}{1.261252in}}%
\pgfpathlineto{\pgfqpoint{1.081261in}{1.259867in}}%
\pgfpathlineto{\pgfqpoint{1.061486in}{1.258172in}}%
\pgfpathlineto{\pgfqpoint{1.041813in}{1.256169in}}%
\pgfpathlineto{\pgfqpoint{1.022259in}{1.253859in}}%
\pgfpathclose%
\pgfusepath{fill}%
\end{pgfscope}%
\begin{pgfscope}%
\pgfpathrectangle{\pgfqpoint{0.041670in}{0.041670in}}{\pgfqpoint{2.216660in}{2.216660in}}%
\pgfusepath{clip}%
\pgfsetbuttcap%
\pgfsetroundjoin%
\definecolor{currentfill}{rgb}{0.277941,0.056324,0.381191}%
\pgfsetfillcolor{currentfill}%
\pgfsetlinewidth{0.000000pt}%
\definecolor{currentstroke}{rgb}{0.000000,0.000000,0.000000}%
\pgfsetstrokecolor{currentstroke}%
\pgfsetdash{}{0pt}%
\pgfpathmoveto{\pgfqpoint{1.257337in}{1.209564in}}%
\pgfpathlineto{\pgfqpoint{1.257790in}{1.214897in}}%
\pgfpathlineto{\pgfqpoint{1.258245in}{1.220538in}}%
\pgfpathlineto{\pgfqpoint{1.258701in}{1.226490in}}%
\pgfpathlineto{\pgfqpoint{1.259158in}{1.232759in}}%
\pgfpathlineto{\pgfqpoint{1.278558in}{1.231365in}}%
\pgfpathlineto{\pgfqpoint{1.297873in}{1.229669in}}%
\pgfpathlineto{\pgfqpoint{1.317086in}{1.227670in}}%
\pgfpathlineto{\pgfqpoint{1.336180in}{1.225369in}}%
\pgfpathlineto{\pgfqpoint{1.335277in}{1.219136in}}%
\pgfpathlineto{\pgfqpoint{1.334377in}{1.213221in}}%
\pgfpathlineto{\pgfqpoint{1.333479in}{1.207617in}}%
\pgfpathlineto{\pgfqpoint{1.332583in}{1.202320in}}%
\pgfpathlineto{\pgfqpoint{1.313930in}{1.204575in}}%
\pgfpathlineto{\pgfqpoint{1.295160in}{1.206535in}}%
\pgfpathlineto{\pgfqpoint{1.276290in}{1.208198in}}%
\pgfpathlineto{\pgfqpoint{1.257337in}{1.209564in}}%
\pgfpathclose%
\pgfusepath{fill}%
\end{pgfscope}%
\begin{pgfscope}%
\pgfpathrectangle{\pgfqpoint{0.041670in}{0.041670in}}{\pgfqpoint{2.216660in}{2.216660in}}%
\pgfusepath{clip}%
\pgfsetbuttcap%
\pgfsetroundjoin%
\definecolor{currentfill}{rgb}{0.282884,0.135920,0.453427}%
\pgfsetfillcolor{currentfill}%
\pgfsetlinewidth{0.000000pt}%
\definecolor{currentstroke}{rgb}{0.000000,0.000000,0.000000}%
\pgfsetstrokecolor{currentstroke}%
\pgfsetdash{}{0pt}%
\pgfpathmoveto{\pgfqpoint{1.260999in}{1.261113in}}%
\pgfpathlineto{\pgfqpoint{1.261463in}{1.269049in}}%
\pgfpathlineto{\pgfqpoint{1.261928in}{1.277336in}}%
\pgfpathlineto{\pgfqpoint{1.262394in}{1.285979in}}%
\pgfpathlineto{\pgfqpoint{1.262862in}{1.294984in}}%
\pgfpathlineto{\pgfqpoint{1.283170in}{1.293539in}}%
\pgfpathlineto{\pgfqpoint{1.303390in}{1.291779in}}%
\pgfpathlineto{\pgfqpoint{1.323504in}{1.289705in}}%
\pgfpathlineto{\pgfqpoint{1.343494in}{1.287319in}}%
\pgfpathlineto{\pgfqpoint{1.342570in}{1.278347in}}%
\pgfpathlineto{\pgfqpoint{1.341649in}{1.269738in}}%
\pgfpathlineto{\pgfqpoint{1.340731in}{1.261485in}}%
\pgfpathlineto{\pgfqpoint{1.339815in}{1.253583in}}%
\pgfpathlineto{\pgfqpoint{1.320276in}{1.255927in}}%
\pgfpathlineto{\pgfqpoint{1.300615in}{1.257965in}}%
\pgfpathlineto{\pgfqpoint{1.280851in}{1.259694in}}%
\pgfpathlineto{\pgfqpoint{1.260999in}{1.261113in}}%
\pgfpathclose%
\pgfusepath{fill}%
\end{pgfscope}%
\begin{pgfscope}%
\pgfpathrectangle{\pgfqpoint{0.041670in}{0.041670in}}{\pgfqpoint{2.216660in}{2.216660in}}%
\pgfusepath{clip}%
\pgfsetbuttcap%
\pgfsetroundjoin%
\definecolor{currentfill}{rgb}{0.268510,0.009605,0.335427}%
\pgfsetfillcolor{currentfill}%
\pgfsetlinewidth{0.000000pt}%
\definecolor{currentstroke}{rgb}{0.000000,0.000000,0.000000}%
\pgfsetstrokecolor{currentstroke}%
\pgfsetdash{}{0pt}%
\pgfpathmoveto{\pgfqpoint{1.108181in}{1.177439in}}%
\pgfpathlineto{\pgfqpoint{1.107747in}{1.180505in}}%
\pgfpathlineto{\pgfqpoint{1.107312in}{1.183837in}}%
\pgfpathlineto{\pgfqpoint{1.106876in}{1.187440in}}%
\pgfpathlineto{\pgfqpoint{1.106439in}{1.191320in}}%
\pgfpathlineto{\pgfqpoint{1.125022in}{1.192332in}}%
\pgfpathlineto{\pgfqpoint{1.143653in}{1.193051in}}%
\pgfpathlineto{\pgfqpoint{1.162315in}{1.193476in}}%
\pgfpathlineto{\pgfqpoint{1.180993in}{1.193606in}}%
\pgfpathlineto{\pgfqpoint{1.180986in}{1.189715in}}%
\pgfpathlineto{\pgfqpoint{1.180980in}{1.186099in}}%
\pgfpathlineto{\pgfqpoint{1.180974in}{1.182754in}}%
\pgfpathlineto{\pgfqpoint{1.180968in}{1.179676in}}%
\pgfpathlineto{\pgfqpoint{1.162733in}{1.179548in}}%
\pgfpathlineto{\pgfqpoint{1.144513in}{1.179132in}}%
\pgfpathlineto{\pgfqpoint{1.126324in}{1.178429in}}%
\pgfpathlineto{\pgfqpoint{1.108181in}{1.177439in}}%
\pgfpathclose%
\pgfusepath{fill}%
\end{pgfscope}%
\begin{pgfscope}%
\pgfpathrectangle{\pgfqpoint{0.041670in}{0.041670in}}{\pgfqpoint{2.216660in}{2.216660in}}%
\pgfusepath{clip}%
\pgfsetbuttcap%
\pgfsetroundjoin%
\definecolor{currentfill}{rgb}{0.268510,0.009605,0.335427}%
\pgfsetfillcolor{currentfill}%
\pgfsetlinewidth{0.000000pt}%
\definecolor{currentstroke}{rgb}{0.000000,0.000000,0.000000}%
\pgfsetstrokecolor{currentstroke}%
\pgfsetdash{}{0pt}%
\pgfpathmoveto{\pgfqpoint{1.180968in}{1.179676in}}%
\pgfpathlineto{\pgfqpoint{1.180974in}{1.182754in}}%
\pgfpathlineto{\pgfqpoint{1.180980in}{1.186099in}}%
\pgfpathlineto{\pgfqpoint{1.180986in}{1.189715in}}%
\pgfpathlineto{\pgfqpoint{1.180993in}{1.193606in}}%
\pgfpathlineto{\pgfqpoint{1.199669in}{1.193443in}}%
\pgfpathlineto{\pgfqpoint{1.218329in}{1.192985in}}%
\pgfpathlineto{\pgfqpoint{1.236955in}{1.192234in}}%
\pgfpathlineto{\pgfqpoint{1.255531in}{1.191189in}}%
\pgfpathlineto{\pgfqpoint{1.255082in}{1.187310in}}%
\pgfpathlineto{\pgfqpoint{1.254634in}{1.183708in}}%
\pgfpathlineto{\pgfqpoint{1.254187in}{1.180376in}}%
\pgfpathlineto{\pgfqpoint{1.253741in}{1.177311in}}%
\pgfpathlineto{\pgfqpoint{1.235604in}{1.178333in}}%
\pgfpathlineto{\pgfqpoint{1.217420in}{1.179068in}}%
\pgfpathlineto{\pgfqpoint{1.199202in}{1.179516in}}%
\pgfpathlineto{\pgfqpoint{1.180968in}{1.179676in}}%
\pgfpathclose%
\pgfusepath{fill}%
\end{pgfscope}%
\begin{pgfscope}%
\pgfpathrectangle{\pgfqpoint{0.041670in}{0.041670in}}{\pgfqpoint{2.216660in}{2.216660in}}%
\pgfusepath{clip}%
\pgfsetbuttcap%
\pgfsetroundjoin%
\definecolor{currentfill}{rgb}{0.260571,0.246922,0.522828}%
\pgfsetfillcolor{currentfill}%
\pgfsetlinewidth{0.000000pt}%
\definecolor{currentstroke}{rgb}{0.000000,0.000000,0.000000}%
\pgfsetstrokecolor{currentstroke}%
\pgfsetdash{}{0pt}%
\pgfpathmoveto{\pgfqpoint{1.097473in}{1.334888in}}%
\pgfpathlineto{\pgfqpoint{1.097010in}{1.345795in}}%
\pgfpathlineto{\pgfqpoint{1.096545in}{1.357101in}}%
\pgfpathlineto{\pgfqpoint{1.096078in}{1.368813in}}%
\pgfpathlineto{\pgfqpoint{1.095610in}{1.380937in}}%
\pgfpathlineto{\pgfqpoint{1.116931in}{1.382066in}}%
\pgfpathlineto{\pgfqpoint{1.138307in}{1.382869in}}%
\pgfpathlineto{\pgfqpoint{1.159717in}{1.383343in}}%
\pgfpathlineto{\pgfqpoint{1.181145in}{1.383489in}}%
\pgfpathlineto{\pgfqpoint{1.181139in}{1.371355in}}%
\pgfpathlineto{\pgfqpoint{1.181132in}{1.359634in}}%
\pgfpathlineto{\pgfqpoint{1.181126in}{1.348318in}}%
\pgfpathlineto{\pgfqpoint{1.181119in}{1.337400in}}%
\pgfpathlineto{\pgfqpoint{1.160164in}{1.337257in}}%
\pgfpathlineto{\pgfqpoint{1.139226in}{1.336790in}}%
\pgfpathlineto{\pgfqpoint{1.118323in}{1.336000in}}%
\pgfpathlineto{\pgfqpoint{1.097473in}{1.334888in}}%
\pgfpathclose%
\pgfusepath{fill}%
\end{pgfscope}%
\begin{pgfscope}%
\pgfpathrectangle{\pgfqpoint{0.041670in}{0.041670in}}{\pgfqpoint{2.216660in}{2.216660in}}%
\pgfusepath{clip}%
\pgfsetbuttcap%
\pgfsetroundjoin%
\definecolor{currentfill}{rgb}{0.260571,0.246922,0.522828}%
\pgfsetfillcolor{currentfill}%
\pgfsetlinewidth{0.000000pt}%
\definecolor{currentstroke}{rgb}{0.000000,0.000000,0.000000}%
\pgfsetstrokecolor{currentstroke}%
\pgfsetdash{}{0pt}%
\pgfpathmoveto{\pgfqpoint{1.181119in}{1.337400in}}%
\pgfpathlineto{\pgfqpoint{1.181126in}{1.348318in}}%
\pgfpathlineto{\pgfqpoint{1.181132in}{1.359634in}}%
\pgfpathlineto{\pgfqpoint{1.181139in}{1.371355in}}%
\pgfpathlineto{\pgfqpoint{1.181145in}{1.383489in}}%
\pgfpathlineto{\pgfqpoint{1.202572in}{1.383306in}}%
\pgfpathlineto{\pgfqpoint{1.223980in}{1.382796in}}%
\pgfpathlineto{\pgfqpoint{1.245350in}{1.381957in}}%
\pgfpathlineto{\pgfqpoint{1.266665in}{1.380791in}}%
\pgfpathlineto{\pgfqpoint{1.266183in}{1.368668in}}%
\pgfpathlineto{\pgfqpoint{1.265704in}{1.356957in}}%
\pgfpathlineto{\pgfqpoint{1.265226in}{1.345651in}}%
\pgfpathlineto{\pgfqpoint{1.264750in}{1.334745in}}%
\pgfpathlineto{\pgfqpoint{1.243906in}{1.335892in}}%
\pgfpathlineto{\pgfqpoint{1.223008in}{1.336718in}}%
\pgfpathlineto{\pgfqpoint{1.202073in}{1.337221in}}%
\pgfpathlineto{\pgfqpoint{1.181119in}{1.337400in}}%
\pgfpathclose%
\pgfusepath{fill}%
\end{pgfscope}%
\begin{pgfscope}%
\pgfpathrectangle{\pgfqpoint{0.041670in}{0.041670in}}{\pgfqpoint{2.216660in}{2.216660in}}%
\pgfusepath{clip}%
\pgfsetbuttcap%
\pgfsetroundjoin%
\definecolor{currentfill}{rgb}{0.272594,0.025563,0.353093}%
\pgfsetfillcolor{currentfill}%
\pgfsetlinewidth{0.000000pt}%
\definecolor{currentstroke}{rgb}{0.000000,0.000000,0.000000}%
\pgfsetstrokecolor{currentstroke}%
\pgfsetdash{}{0pt}%
\pgfpathmoveto{\pgfqpoint{1.032909in}{1.184355in}}%
\pgfpathlineto{\pgfqpoint{1.032033in}{1.188479in}}%
\pgfpathlineto{\pgfqpoint{1.031155in}{1.192888in}}%
\pgfpathlineto{\pgfqpoint{1.030275in}{1.197588in}}%
\pgfpathlineto{\pgfqpoint{1.029393in}{1.202585in}}%
\pgfpathlineto{\pgfqpoint{1.048060in}{1.204807in}}%
\pgfpathlineto{\pgfqpoint{1.066842in}{1.206734in}}%
\pgfpathlineto{\pgfqpoint{1.085722in}{1.208364in}}%
\pgfpathlineto{\pgfqpoint{1.104684in}{1.209697in}}%
\pgfpathlineto{\pgfqpoint{1.105124in}{1.204663in}}%
\pgfpathlineto{\pgfqpoint{1.105563in}{1.199926in}}%
\pgfpathlineto{\pgfqpoint{1.106002in}{1.195480in}}%
\pgfpathlineto{\pgfqpoint{1.106439in}{1.191320in}}%
\pgfpathlineto{\pgfqpoint{1.087921in}{1.190015in}}%
\pgfpathlineto{\pgfqpoint{1.069482in}{1.188418in}}%
\pgfpathlineto{\pgfqpoint{1.051139in}{1.186531in}}%
\pgfpathlineto{\pgfqpoint{1.032909in}{1.184355in}}%
\pgfpathclose%
\pgfusepath{fill}%
\end{pgfscope}%
\begin{pgfscope}%
\pgfpathrectangle{\pgfqpoint{0.041670in}{0.041670in}}{\pgfqpoint{2.216660in}{2.216660in}}%
\pgfusepath{clip}%
\pgfsetbuttcap%
\pgfsetroundjoin%
\definecolor{currentfill}{rgb}{0.272594,0.025563,0.353093}%
\pgfsetfillcolor{currentfill}%
\pgfsetlinewidth{0.000000pt}%
\definecolor{currentstroke}{rgb}{0.000000,0.000000,0.000000}%
\pgfsetstrokecolor{currentstroke}%
\pgfsetdash{}{0pt}%
\pgfpathmoveto{\pgfqpoint{1.255531in}{1.191189in}}%
\pgfpathlineto{\pgfqpoint{1.255981in}{1.195349in}}%
\pgfpathlineto{\pgfqpoint{1.256432in}{1.199794in}}%
\pgfpathlineto{\pgfqpoint{1.256884in}{1.204531in}}%
\pgfpathlineto{\pgfqpoint{1.257337in}{1.209564in}}%
\pgfpathlineto{\pgfqpoint{1.276290in}{1.208198in}}%
\pgfpathlineto{\pgfqpoint{1.295160in}{1.206535in}}%
\pgfpathlineto{\pgfqpoint{1.313930in}{1.204575in}}%
\pgfpathlineto{\pgfqpoint{1.332583in}{1.202320in}}%
\pgfpathlineto{\pgfqpoint{1.331689in}{1.197325in}}%
\pgfpathlineto{\pgfqpoint{1.330797in}{1.192625in}}%
\pgfpathlineto{\pgfqpoint{1.329907in}{1.188218in}}%
\pgfpathlineto{\pgfqpoint{1.329019in}{1.184096in}}%
\pgfpathlineto{\pgfqpoint{1.310802in}{1.186304in}}%
\pgfpathlineto{\pgfqpoint{1.292471in}{1.188223in}}%
\pgfpathlineto{\pgfqpoint{1.274042in}{1.189852in}}%
\pgfpathlineto{\pgfqpoint{1.255531in}{1.191189in}}%
\pgfpathclose%
\pgfusepath{fill}%
\end{pgfscope}%
\begin{pgfscope}%
\pgfpathrectangle{\pgfqpoint{0.041670in}{0.041670in}}{\pgfqpoint{2.216660in}{2.216660in}}%
\pgfusepath{clip}%
\pgfsetbuttcap%
\pgfsetroundjoin%
\definecolor{currentfill}{rgb}{0.276194,0.190074,0.493001}%
\pgfsetfillcolor{currentfill}%
\pgfsetlinewidth{0.000000pt}%
\definecolor{currentstroke}{rgb}{0.000000,0.000000,0.000000}%
\pgfsetstrokecolor{currentstroke}%
\pgfsetdash{}{0pt}%
\pgfpathmoveto{\pgfqpoint{1.018630in}{1.287599in}}%
\pgfpathlineto{\pgfqpoint{1.017716in}{1.296941in}}%
\pgfpathlineto{\pgfqpoint{1.016798in}{1.306656in}}%
\pgfpathlineto{\pgfqpoint{1.015877in}{1.316753in}}%
\pgfpathlineto{\pgfqpoint{1.014953in}{1.327236in}}%
\pgfpathlineto{\pgfqpoint{1.035415in}{1.329627in}}%
\pgfpathlineto{\pgfqpoint{1.056001in}{1.331700in}}%
\pgfpathlineto{\pgfqpoint{1.076693in}{1.333455in}}%
\pgfpathlineto{\pgfqpoint{1.097473in}{1.334888in}}%
\pgfpathlineto{\pgfqpoint{1.097934in}{1.324374in}}%
\pgfpathlineto{\pgfqpoint{1.098394in}{1.314246in}}%
\pgfpathlineto{\pgfqpoint{1.098852in}{1.304498in}}%
\pgfpathlineto{\pgfqpoint{1.099309in}{1.295125in}}%
\pgfpathlineto{\pgfqpoint{1.078992in}{1.293715in}}%
\pgfpathlineto{\pgfqpoint{1.058761in}{1.291990in}}%
\pgfpathlineto{\pgfqpoint{1.038635in}{1.289951in}}%
\pgfpathlineto{\pgfqpoint{1.018630in}{1.287599in}}%
\pgfpathclose%
\pgfusepath{fill}%
\end{pgfscope}%
\begin{pgfscope}%
\pgfpathrectangle{\pgfqpoint{0.041670in}{0.041670in}}{\pgfqpoint{2.216660in}{2.216660in}}%
\pgfusepath{clip}%
\pgfsetbuttcap%
\pgfsetroundjoin%
\definecolor{currentfill}{rgb}{0.276194,0.190074,0.493001}%
\pgfsetfillcolor{currentfill}%
\pgfsetlinewidth{0.000000pt}%
\definecolor{currentstroke}{rgb}{0.000000,0.000000,0.000000}%
\pgfsetstrokecolor{currentstroke}%
\pgfsetdash{}{0pt}%
\pgfpathmoveto{\pgfqpoint{1.262862in}{1.294984in}}%
\pgfpathlineto{\pgfqpoint{1.263332in}{1.304357in}}%
\pgfpathlineto{\pgfqpoint{1.263803in}{1.314104in}}%
\pgfpathlineto{\pgfqpoint{1.264275in}{1.324231in}}%
\pgfpathlineto{\pgfqpoint{1.264750in}{1.334745in}}%
\pgfpathlineto{\pgfqpoint{1.285521in}{1.333275in}}%
\pgfpathlineto{\pgfqpoint{1.306202in}{1.331486in}}%
\pgfpathlineto{\pgfqpoint{1.326774in}{1.329377in}}%
\pgfpathlineto{\pgfqpoint{1.347221in}{1.326951in}}%
\pgfpathlineto{\pgfqpoint{1.346285in}{1.316469in}}%
\pgfpathlineto{\pgfqpoint{1.345351in}{1.306373in}}%
\pgfpathlineto{\pgfqpoint{1.344421in}{1.296659in}}%
\pgfpathlineto{\pgfqpoint{1.343494in}{1.287319in}}%
\pgfpathlineto{\pgfqpoint{1.323504in}{1.289705in}}%
\pgfpathlineto{\pgfqpoint{1.303390in}{1.291779in}}%
\pgfpathlineto{\pgfqpoint{1.283170in}{1.293539in}}%
\pgfpathlineto{\pgfqpoint{1.262862in}{1.294984in}}%
\pgfpathclose%
\pgfusepath{fill}%
\end{pgfscope}%
\begin{pgfscope}%
\pgfpathrectangle{\pgfqpoint{0.041670in}{0.041670in}}{\pgfqpoint{2.216660in}{2.216660in}}%
\pgfusepath{clip}%
\pgfsetbuttcap%
\pgfsetroundjoin%
\definecolor{currentfill}{rgb}{0.268510,0.009605,0.335427}%
\pgfsetfillcolor{currentfill}%
\pgfsetlinewidth{0.000000pt}%
\definecolor{currentstroke}{rgb}{0.000000,0.000000,0.000000}%
\pgfsetstrokecolor{currentstroke}%
\pgfsetdash{}{0pt}%
\pgfpathmoveto{\pgfqpoint{1.036397in}{1.170626in}}%
\pgfpathlineto{\pgfqpoint{1.035527in}{1.173654in}}%
\pgfpathlineto{\pgfqpoint{1.034656in}{1.176948in}}%
\pgfpathlineto{\pgfqpoint{1.033783in}{1.180514in}}%
\pgfpathlineto{\pgfqpoint{1.032909in}{1.184355in}}%
\pgfpathlineto{\pgfqpoint{1.051139in}{1.186531in}}%
\pgfpathlineto{\pgfqpoint{1.069482in}{1.188418in}}%
\pgfpathlineto{\pgfqpoint{1.087921in}{1.190015in}}%
\pgfpathlineto{\pgfqpoint{1.106439in}{1.191320in}}%
\pgfpathlineto{\pgfqpoint{1.106876in}{1.187440in}}%
\pgfpathlineto{\pgfqpoint{1.107312in}{1.183837in}}%
\pgfpathlineto{\pgfqpoint{1.107747in}{1.180505in}}%
\pgfpathlineto{\pgfqpoint{1.108181in}{1.177439in}}%
\pgfpathlineto{\pgfqpoint{1.090102in}{1.176162in}}%
\pgfpathlineto{\pgfqpoint{1.072100in}{1.174601in}}%
\pgfpathlineto{\pgfqpoint{1.054194in}{1.172755in}}%
\pgfpathlineto{\pgfqpoint{1.036397in}{1.170626in}}%
\pgfpathclose%
\pgfusepath{fill}%
\end{pgfscope}%
\begin{pgfscope}%
\pgfpathrectangle{\pgfqpoint{0.041670in}{0.041670in}}{\pgfqpoint{2.216660in}{2.216660in}}%
\pgfusepath{clip}%
\pgfsetbuttcap%
\pgfsetroundjoin%
\definecolor{currentfill}{rgb}{0.268510,0.009605,0.335427}%
\pgfsetfillcolor{currentfill}%
\pgfsetlinewidth{0.000000pt}%
\definecolor{currentstroke}{rgb}{0.000000,0.000000,0.000000}%
\pgfsetstrokecolor{currentstroke}%
\pgfsetdash{}{0pt}%
\pgfpathmoveto{\pgfqpoint{1.253741in}{1.177311in}}%
\pgfpathlineto{\pgfqpoint{1.254187in}{1.180376in}}%
\pgfpathlineto{\pgfqpoint{1.254634in}{1.183708in}}%
\pgfpathlineto{\pgfqpoint{1.255082in}{1.187310in}}%
\pgfpathlineto{\pgfqpoint{1.255531in}{1.191189in}}%
\pgfpathlineto{\pgfqpoint{1.274042in}{1.189852in}}%
\pgfpathlineto{\pgfqpoint{1.292471in}{1.188223in}}%
\pgfpathlineto{\pgfqpoint{1.310802in}{1.186304in}}%
\pgfpathlineto{\pgfqpoint{1.329019in}{1.184096in}}%
\pgfpathlineto{\pgfqpoint{1.328132in}{1.180255in}}%
\pgfpathlineto{\pgfqpoint{1.327248in}{1.176691in}}%
\pgfpathlineto{\pgfqpoint{1.326365in}{1.173398in}}%
\pgfpathlineto{\pgfqpoint{1.325483in}{1.170372in}}%
\pgfpathlineto{\pgfqpoint{1.307699in}{1.172532in}}%
\pgfpathlineto{\pgfqpoint{1.289804in}{1.174409in}}%
\pgfpathlineto{\pgfqpoint{1.271812in}{1.176003in}}%
\pgfpathlineto{\pgfqpoint{1.253741in}{1.177311in}}%
\pgfpathclose%
\pgfusepath{fill}%
\end{pgfscope}%
\begin{pgfscope}%
\pgfpathrectangle{\pgfqpoint{0.041670in}{0.041670in}}{\pgfqpoint{2.216660in}{2.216660in}}%
\pgfusepath{clip}%
\pgfsetbuttcap%
\pgfsetroundjoin%
\definecolor{currentfill}{rgb}{0.260571,0.246922,0.522828}%
\pgfsetfillcolor{currentfill}%
\pgfsetlinewidth{0.000000pt}%
\definecolor{currentstroke}{rgb}{0.000000,0.000000,0.000000}%
\pgfsetstrokecolor{currentstroke}%
\pgfsetdash{}{0pt}%
\pgfpathmoveto{\pgfqpoint{1.014953in}{1.327236in}}%
\pgfpathlineto{\pgfqpoint{1.014026in}{1.338112in}}%
\pgfpathlineto{\pgfqpoint{1.013095in}{1.349388in}}%
\pgfpathlineto{\pgfqpoint{1.012160in}{1.361071in}}%
\pgfpathlineto{\pgfqpoint{1.011222in}{1.373165in}}%
\pgfpathlineto{\pgfqpoint{1.032148in}{1.375594in}}%
\pgfpathlineto{\pgfqpoint{1.053200in}{1.377700in}}%
\pgfpathlineto{\pgfqpoint{1.074360in}{1.379481in}}%
\pgfpathlineto{\pgfqpoint{1.095610in}{1.380937in}}%
\pgfpathlineto{\pgfqpoint{1.096078in}{1.368813in}}%
\pgfpathlineto{\pgfqpoint{1.096545in}{1.357101in}}%
\pgfpathlineto{\pgfqpoint{1.097010in}{1.345795in}}%
\pgfpathlineto{\pgfqpoint{1.097473in}{1.334888in}}%
\pgfpathlineto{\pgfqpoint{1.076693in}{1.333455in}}%
\pgfpathlineto{\pgfqpoint{1.056001in}{1.331700in}}%
\pgfpathlineto{\pgfqpoint{1.035415in}{1.329627in}}%
\pgfpathlineto{\pgfqpoint{1.014953in}{1.327236in}}%
\pgfpathclose%
\pgfusepath{fill}%
\end{pgfscope}%
\begin{pgfscope}%
\pgfpathrectangle{\pgfqpoint{0.041670in}{0.041670in}}{\pgfqpoint{2.216660in}{2.216660in}}%
\pgfusepath{clip}%
\pgfsetbuttcap%
\pgfsetroundjoin%
\definecolor{currentfill}{rgb}{0.282327,0.094955,0.417331}%
\pgfsetfillcolor{currentfill}%
\pgfsetlinewidth{0.000000pt}%
\definecolor{currentstroke}{rgb}{0.000000,0.000000,0.000000}%
\pgfsetstrokecolor{currentstroke}%
\pgfsetdash{}{0pt}%
\pgfpathmoveto{\pgfqpoint{1.336180in}{1.225369in}}%
\pgfpathlineto{\pgfqpoint{1.337085in}{1.231925in}}%
\pgfpathlineto{\pgfqpoint{1.337993in}{1.238809in}}%
\pgfpathlineto{\pgfqpoint{1.338903in}{1.246027in}}%
\pgfpathlineto{\pgfqpoint{1.339815in}{1.253583in}}%
\pgfpathlineto{\pgfqpoint{1.359217in}{1.250934in}}%
\pgfpathlineto{\pgfqpoint{1.378463in}{1.247982in}}%
\pgfpathlineto{\pgfqpoint{1.397538in}{1.244729in}}%
\pgfpathlineto{\pgfqpoint{1.396294in}{1.237213in}}%
\pgfpathlineto{\pgfqpoint{1.395054in}{1.230037in}}%
\pgfpathlineto{\pgfqpoint{1.393817in}{1.223195in}}%
\pgfpathlineto{\pgfqpoint{1.392584in}{1.216681in}}%
\pgfpathlineto{\pgfqpoint{1.373946in}{1.219873in}}%
\pgfpathlineto{\pgfqpoint{1.355139in}{1.222770in}}%
\pgfpathlineto{\pgfqpoint{1.336180in}{1.225369in}}%
\pgfpathclose%
\pgfusepath{fill}%
\end{pgfscope}%
\begin{pgfscope}%
\pgfpathrectangle{\pgfqpoint{0.041670in}{0.041670in}}{\pgfqpoint{2.216660in}{2.216660in}}%
\pgfusepath{clip}%
\pgfsetbuttcap%
\pgfsetroundjoin%
\definecolor{currentfill}{rgb}{0.267004,0.004874,0.329415}%
\pgfsetfillcolor{currentfill}%
\pgfsetlinewidth{0.000000pt}%
\definecolor{currentstroke}{rgb}{0.000000,0.000000,0.000000}%
\pgfsetstrokecolor{currentstroke}%
\pgfsetdash{}{0pt}%
\pgfpathmoveto{\pgfqpoint{1.109912in}{1.167744in}}%
\pgfpathlineto{\pgfqpoint{1.109480in}{1.169792in}}%
\pgfpathlineto{\pgfqpoint{1.109048in}{1.172087in}}%
\pgfpathlineto{\pgfqpoint{1.108615in}{1.174635in}}%
\pgfpathlineto{\pgfqpoint{1.108181in}{1.177439in}}%
\pgfpathlineto{\pgfqpoint{1.126324in}{1.178429in}}%
\pgfpathlineto{\pgfqpoint{1.144513in}{1.179132in}}%
\pgfpathlineto{\pgfqpoint{1.162733in}{1.179548in}}%
\pgfpathlineto{\pgfqpoint{1.180968in}{1.179676in}}%
\pgfpathlineto{\pgfqpoint{1.180962in}{1.176859in}}%
\pgfpathlineto{\pgfqpoint{1.180956in}{1.174299in}}%
\pgfpathlineto{\pgfqpoint{1.180950in}{1.171991in}}%
\pgfpathlineto{\pgfqpoint{1.180944in}{1.169931in}}%
\pgfpathlineto{\pgfqpoint{1.163148in}{1.169806in}}%
\pgfpathlineto{\pgfqpoint{1.145367in}{1.169399in}}%
\pgfpathlineto{\pgfqpoint{1.127616in}{1.168712in}}%
\pgfpathlineto{\pgfqpoint{1.109912in}{1.167744in}}%
\pgfpathclose%
\pgfusepath{fill}%
\end{pgfscope}%
\begin{pgfscope}%
\pgfpathrectangle{\pgfqpoint{0.041670in}{0.041670in}}{\pgfqpoint{2.216660in}{2.216660in}}%
\pgfusepath{clip}%
\pgfsetbuttcap%
\pgfsetroundjoin%
\definecolor{currentfill}{rgb}{0.267004,0.004874,0.329415}%
\pgfsetfillcolor{currentfill}%
\pgfsetlinewidth{0.000000pt}%
\definecolor{currentstroke}{rgb}{0.000000,0.000000,0.000000}%
\pgfsetstrokecolor{currentstroke}%
\pgfsetdash{}{0pt}%
\pgfpathmoveto{\pgfqpoint{1.180944in}{1.169931in}}%
\pgfpathlineto{\pgfqpoint{1.180950in}{1.171991in}}%
\pgfpathlineto{\pgfqpoint{1.180956in}{1.174299in}}%
\pgfpathlineto{\pgfqpoint{1.180962in}{1.176859in}}%
\pgfpathlineto{\pgfqpoint{1.180968in}{1.179676in}}%
\pgfpathlineto{\pgfqpoint{1.199202in}{1.179516in}}%
\pgfpathlineto{\pgfqpoint{1.217420in}{1.179068in}}%
\pgfpathlineto{\pgfqpoint{1.235604in}{1.178333in}}%
\pgfpathlineto{\pgfqpoint{1.253741in}{1.177311in}}%
\pgfpathlineto{\pgfqpoint{1.253295in}{1.174508in}}%
\pgfpathlineto{\pgfqpoint{1.252850in}{1.171961in}}%
\pgfpathlineto{\pgfqpoint{1.252405in}{1.169666in}}%
\pgfpathlineto{\pgfqpoint{1.251962in}{1.167619in}}%
\pgfpathlineto{\pgfqpoint{1.234263in}{1.168618in}}%
\pgfpathlineto{\pgfqpoint{1.216517in}{1.169337in}}%
\pgfpathlineto{\pgfqpoint{1.198738in}{1.169774in}}%
\pgfpathlineto{\pgfqpoint{1.180944in}{1.169931in}}%
\pgfpathclose%
\pgfusepath{fill}%
\end{pgfscope}%
\begin{pgfscope}%
\pgfpathrectangle{\pgfqpoint{0.041670in}{0.041670in}}{\pgfqpoint{2.216660in}{2.216660in}}%
\pgfusepath{clip}%
\pgfsetbuttcap%
\pgfsetroundjoin%
\definecolor{currentfill}{rgb}{0.260571,0.246922,0.522828}%
\pgfsetfillcolor{currentfill}%
\pgfsetlinewidth{0.000000pt}%
\definecolor{currentstroke}{rgb}{0.000000,0.000000,0.000000}%
\pgfsetstrokecolor{currentstroke}%
\pgfsetdash{}{0pt}%
\pgfpathmoveto{\pgfqpoint{1.264750in}{1.334745in}}%
\pgfpathlineto{\pgfqpoint{1.265226in}{1.345651in}}%
\pgfpathlineto{\pgfqpoint{1.265704in}{1.356957in}}%
\pgfpathlineto{\pgfqpoint{1.266183in}{1.368668in}}%
\pgfpathlineto{\pgfqpoint{1.266665in}{1.380791in}}%
\pgfpathlineto{\pgfqpoint{1.287906in}{1.379299in}}%
\pgfpathlineto{\pgfqpoint{1.309055in}{1.377482in}}%
\pgfpathlineto{\pgfqpoint{1.330094in}{1.375340in}}%
\pgfpathlineto{\pgfqpoint{1.351004in}{1.372876in}}%
\pgfpathlineto{\pgfqpoint{1.350053in}{1.360782in}}%
\pgfpathlineto{\pgfqpoint{1.349106in}{1.349101in}}%
\pgfpathlineto{\pgfqpoint{1.348162in}{1.337826in}}%
\pgfpathlineto{\pgfqpoint{1.347221in}{1.326951in}}%
\pgfpathlineto{\pgfqpoint{1.326774in}{1.329377in}}%
\pgfpathlineto{\pgfqpoint{1.306202in}{1.331486in}}%
\pgfpathlineto{\pgfqpoint{1.285521in}{1.333275in}}%
\pgfpathlineto{\pgfqpoint{1.264750in}{1.334745in}}%
\pgfpathclose%
\pgfusepath{fill}%
\end{pgfscope}%
\begin{pgfscope}%
\pgfpathrectangle{\pgfqpoint{0.041670in}{0.041670in}}{\pgfqpoint{2.216660in}{2.216660in}}%
\pgfusepath{clip}%
\pgfsetbuttcap%
\pgfsetroundjoin%
\definecolor{currentfill}{rgb}{0.277941,0.056324,0.381191}%
\pgfsetfillcolor{currentfill}%
\pgfsetlinewidth{0.000000pt}%
\definecolor{currentstroke}{rgb}{0.000000,0.000000,0.000000}%
\pgfsetstrokecolor{currentstroke}%
\pgfsetdash{}{0pt}%
\pgfpathmoveto{\pgfqpoint{1.332583in}{1.202320in}}%
\pgfpathlineto{\pgfqpoint{1.333479in}{1.207617in}}%
\pgfpathlineto{\pgfqpoint{1.334377in}{1.213221in}}%
\pgfpathlineto{\pgfqpoint{1.335277in}{1.219136in}}%
\pgfpathlineto{\pgfqpoint{1.336180in}{1.225369in}}%
\pgfpathlineto{\pgfqpoint{1.355139in}{1.222770in}}%
\pgfpathlineto{\pgfqpoint{1.373946in}{1.219873in}}%
\pgfpathlineto{\pgfqpoint{1.392584in}{1.216681in}}%
\pgfpathlineto{\pgfqpoint{1.391354in}{1.210490in}}%
\pgfpathlineto{\pgfqpoint{1.390127in}{1.204618in}}%
\pgfpathlineto{\pgfqpoint{1.388903in}{1.199057in}}%
\pgfpathlineto{\pgfqpoint{1.387682in}{1.193804in}}%
\pgfpathlineto{\pgfqpoint{1.369475in}{1.196932in}}%
\pgfpathlineto{\pgfqpoint{1.351104in}{1.199772in}}%
\pgfpathlineto{\pgfqpoint{1.332583in}{1.202320in}}%
\pgfpathclose%
\pgfusepath{fill}%
\end{pgfscope}%
\begin{pgfscope}%
\pgfpathrectangle{\pgfqpoint{0.041670in}{0.041670in}}{\pgfqpoint{2.216660in}{2.216660in}}%
\pgfusepath{clip}%
\pgfsetbuttcap%
\pgfsetroundjoin%
\definecolor{currentfill}{rgb}{0.282884,0.135920,0.453427}%
\pgfsetfillcolor{currentfill}%
\pgfsetlinewidth{0.000000pt}%
\definecolor{currentstroke}{rgb}{0.000000,0.000000,0.000000}%
\pgfsetstrokecolor{currentstroke}%
\pgfsetdash{}{0pt}%
\pgfpathmoveto{\pgfqpoint{1.339815in}{1.253583in}}%
\pgfpathlineto{\pgfqpoint{1.340731in}{1.261485in}}%
\pgfpathlineto{\pgfqpoint{1.341649in}{1.269738in}}%
\pgfpathlineto{\pgfqpoint{1.342570in}{1.278347in}}%
\pgfpathlineto{\pgfqpoint{1.343494in}{1.287319in}}%
\pgfpathlineto{\pgfqpoint{1.363343in}{1.284622in}}%
\pgfpathlineto{\pgfqpoint{1.383035in}{1.281617in}}%
\pgfpathlineto{\pgfqpoint{1.402551in}{1.278305in}}%
\pgfpathlineto{\pgfqpoint{1.401292in}{1.269373in}}%
\pgfpathlineto{\pgfqpoint{1.400036in}{1.260803in}}%
\pgfpathlineto{\pgfqpoint{1.398785in}{1.252591in}}%
\pgfpathlineto{\pgfqpoint{1.397538in}{1.244729in}}%
\pgfpathlineto{\pgfqpoint{1.378463in}{1.247982in}}%
\pgfpathlineto{\pgfqpoint{1.359217in}{1.250934in}}%
\pgfpathlineto{\pgfqpoint{1.339815in}{1.253583in}}%
\pgfpathclose%
\pgfusepath{fill}%
\end{pgfscope}%
\begin{pgfscope}%
\pgfpathrectangle{\pgfqpoint{0.041670in}{0.041670in}}{\pgfqpoint{2.216660in}{2.216660in}}%
\pgfusepath{clip}%
\pgfsetbuttcap%
\pgfsetroundjoin%
\definecolor{currentfill}{rgb}{0.282327,0.094955,0.417331}%
\pgfsetfillcolor{currentfill}%
\pgfsetlinewidth{0.000000pt}%
\definecolor{currentstroke}{rgb}{0.000000,0.000000,0.000000}%
\pgfsetstrokecolor{currentstroke}%
\pgfsetdash{}{0pt}%
\pgfpathmoveto{\pgfqpoint{0.950914in}{1.213598in}}%
\pgfpathlineto{\pgfqpoint{0.949585in}{1.220096in}}%
\pgfpathlineto{\pgfqpoint{0.948252in}{1.226924in}}%
\pgfpathlineto{\pgfqpoint{0.946915in}{1.234085in}}%
\pgfpathlineto{\pgfqpoint{0.945575in}{1.241587in}}%
\pgfpathlineto{\pgfqpoint{0.964482in}{1.245105in}}%
\pgfpathlineto{\pgfqpoint{0.983577in}{1.248325in}}%
\pgfpathlineto{\pgfqpoint{1.002841in}{1.251244in}}%
\pgfpathlineto{\pgfqpoint{1.022259in}{1.253859in}}%
\pgfpathlineto{\pgfqpoint{1.023159in}{1.246301in}}%
\pgfpathlineto{\pgfqpoint{1.024057in}{1.239082in}}%
\pgfpathlineto{\pgfqpoint{1.024952in}{1.232197in}}%
\pgfpathlineto{\pgfqpoint{1.025845in}{1.225640in}}%
\pgfpathlineto{\pgfqpoint{1.006870in}{1.223073in}}%
\pgfpathlineto{\pgfqpoint{0.988046in}{1.220210in}}%
\pgfpathlineto{\pgfqpoint{0.969388in}{1.217050in}}%
\pgfpathlineto{\pgfqpoint{0.950914in}{1.213598in}}%
\pgfpathclose%
\pgfusepath{fill}%
\end{pgfscope}%
\begin{pgfscope}%
\pgfpathrectangle{\pgfqpoint{0.041670in}{0.041670in}}{\pgfqpoint{2.216660in}{2.216660in}}%
\pgfusepath{clip}%
\pgfsetbuttcap%
\pgfsetroundjoin%
\definecolor{currentfill}{rgb}{0.233603,0.313828,0.543914}%
\pgfsetfillcolor{currentfill}%
\pgfsetlinewidth{0.000000pt}%
\definecolor{currentstroke}{rgb}{0.000000,0.000000,0.000000}%
\pgfsetstrokecolor{currentstroke}%
\pgfsetdash{}{0pt}%
\pgfpathmoveto{\pgfqpoint{1.095610in}{1.380937in}}%
\pgfpathlineto{\pgfqpoint{1.095139in}{1.393480in}}%
\pgfpathlineto{\pgfqpoint{1.094667in}{1.406448in}}%
\pgfpathlineto{\pgfqpoint{1.094193in}{1.419849in}}%
\pgfpathlineto{\pgfqpoint{1.093716in}{1.433690in}}%
\pgfpathlineto{\pgfqpoint{1.115517in}{1.434836in}}%
\pgfpathlineto{\pgfqpoint{1.137372in}{1.435649in}}%
\pgfpathlineto{\pgfqpoint{1.159263in}{1.436130in}}%
\pgfpathlineto{\pgfqpoint{1.181172in}{1.436278in}}%
\pgfpathlineto{\pgfqpoint{1.181165in}{1.422429in}}%
\pgfpathlineto{\pgfqpoint{1.181159in}{1.409019in}}%
\pgfpathlineto{\pgfqpoint{1.181152in}{1.396041in}}%
\pgfpathlineto{\pgfqpoint{1.181145in}{1.383489in}}%
\pgfpathlineto{\pgfqpoint{1.159717in}{1.383343in}}%
\pgfpathlineto{\pgfqpoint{1.138307in}{1.382869in}}%
\pgfpathlineto{\pgfqpoint{1.116931in}{1.382066in}}%
\pgfpathlineto{\pgfqpoint{1.095610in}{1.380937in}}%
\pgfpathclose%
\pgfusepath{fill}%
\end{pgfscope}%
\begin{pgfscope}%
\pgfpathrectangle{\pgfqpoint{0.041670in}{0.041670in}}{\pgfqpoint{2.216660in}{2.216660in}}%
\pgfusepath{clip}%
\pgfsetbuttcap%
\pgfsetroundjoin%
\definecolor{currentfill}{rgb}{0.233603,0.313828,0.543914}%
\pgfsetfillcolor{currentfill}%
\pgfsetlinewidth{0.000000pt}%
\definecolor{currentstroke}{rgb}{0.000000,0.000000,0.000000}%
\pgfsetstrokecolor{currentstroke}%
\pgfsetdash{}{0pt}%
\pgfpathmoveto{\pgfqpoint{1.181145in}{1.383489in}}%
\pgfpathlineto{\pgfqpoint{1.181152in}{1.396041in}}%
\pgfpathlineto{\pgfqpoint{1.181159in}{1.409019in}}%
\pgfpathlineto{\pgfqpoint{1.181165in}{1.422429in}}%
\pgfpathlineto{\pgfqpoint{1.181172in}{1.436278in}}%
\pgfpathlineto{\pgfqpoint{1.203080in}{1.436093in}}%
\pgfpathlineto{\pgfqpoint{1.224968in}{1.435575in}}%
\pgfpathlineto{\pgfqpoint{1.246818in}{1.434725in}}%
\pgfpathlineto{\pgfqpoint{1.268612in}{1.433542in}}%
\pgfpathlineto{\pgfqpoint{1.268122in}{1.419702in}}%
\pgfpathlineto{\pgfqpoint{1.267634in}{1.406302in}}%
\pgfpathlineto{\pgfqpoint{1.267149in}{1.393334in}}%
\pgfpathlineto{\pgfqpoint{1.266665in}{1.380791in}}%
\pgfpathlineto{\pgfqpoint{1.245350in}{1.381957in}}%
\pgfpathlineto{\pgfqpoint{1.223980in}{1.382796in}}%
\pgfpathlineto{\pgfqpoint{1.202572in}{1.383306in}}%
\pgfpathlineto{\pgfqpoint{1.181145in}{1.383489in}}%
\pgfpathclose%
\pgfusepath{fill}%
\end{pgfscope}%
\begin{pgfscope}%
\pgfpathrectangle{\pgfqpoint{0.041670in}{0.041670in}}{\pgfqpoint{2.216660in}{2.216660in}}%
\pgfusepath{clip}%
\pgfsetbuttcap%
\pgfsetroundjoin%
\definecolor{currentfill}{rgb}{0.277941,0.056324,0.381191}%
\pgfsetfillcolor{currentfill}%
\pgfsetlinewidth{0.000000pt}%
\definecolor{currentstroke}{rgb}{0.000000,0.000000,0.000000}%
\pgfsetstrokecolor{currentstroke}%
\pgfsetdash{}{0pt}%
\pgfpathmoveto{\pgfqpoint{0.956196in}{1.190781in}}%
\pgfpathlineto{\pgfqpoint{0.954880in}{1.196019in}}%
\pgfpathlineto{\pgfqpoint{0.953562in}{1.201565in}}%
\pgfpathlineto{\pgfqpoint{0.952239in}{1.207422in}}%
\pgfpathlineto{\pgfqpoint{0.950914in}{1.213598in}}%
\pgfpathlineto{\pgfqpoint{0.969388in}{1.217050in}}%
\pgfpathlineto{\pgfqpoint{0.988046in}{1.220210in}}%
\pgfpathlineto{\pgfqpoint{1.006870in}{1.223073in}}%
\pgfpathlineto{\pgfqpoint{1.025845in}{1.225640in}}%
\pgfpathlineto{\pgfqpoint{1.026735in}{1.219405in}}%
\pgfpathlineto{\pgfqpoint{1.027623in}{1.213489in}}%
\pgfpathlineto{\pgfqpoint{1.028509in}{1.207883in}}%
\pgfpathlineto{\pgfqpoint{1.029393in}{1.202585in}}%
\pgfpathlineto{\pgfqpoint{1.010857in}{1.200070in}}%
\pgfpathlineto{\pgfqpoint{0.992468in}{1.197262in}}%
\pgfpathlineto{\pgfqpoint{0.974242in}{1.194165in}}%
\pgfpathlineto{\pgfqpoint{0.956196in}{1.190781in}}%
\pgfpathclose%
\pgfusepath{fill}%
\end{pgfscope}%
\begin{pgfscope}%
\pgfpathrectangle{\pgfqpoint{0.041670in}{0.041670in}}{\pgfqpoint{2.216660in}{2.216660in}}%
\pgfusepath{clip}%
\pgfsetbuttcap%
\pgfsetroundjoin%
\definecolor{currentfill}{rgb}{0.282884,0.135920,0.453427}%
\pgfsetfillcolor{currentfill}%
\pgfsetlinewidth{0.000000pt}%
\definecolor{currentstroke}{rgb}{0.000000,0.000000,0.000000}%
\pgfsetstrokecolor{currentstroke}%
\pgfsetdash{}{0pt}%
\pgfpathmoveto{\pgfqpoint{0.945575in}{1.241587in}}%
\pgfpathlineto{\pgfqpoint{0.944230in}{1.249434in}}%
\pgfpathlineto{\pgfqpoint{0.942882in}{1.257632in}}%
\pgfpathlineto{\pgfqpoint{0.941529in}{1.266188in}}%
\pgfpathlineto{\pgfqpoint{0.940172in}{1.275106in}}%
\pgfpathlineto{\pgfqpoint{0.959518in}{1.278688in}}%
\pgfpathlineto{\pgfqpoint{0.979054in}{1.281966in}}%
\pgfpathlineto{\pgfqpoint{0.998764in}{1.284937in}}%
\pgfpathlineto{\pgfqpoint{1.018630in}{1.287599in}}%
\pgfpathlineto{\pgfqpoint{1.019542in}{1.278626in}}%
\pgfpathlineto{\pgfqpoint{1.020450in}{1.270016in}}%
\pgfpathlineto{\pgfqpoint{1.021356in}{1.261762in}}%
\pgfpathlineto{\pgfqpoint{1.022259in}{1.253859in}}%
\pgfpathlineto{\pgfqpoint{1.002841in}{1.251244in}}%
\pgfpathlineto{\pgfqpoint{0.983577in}{1.248325in}}%
\pgfpathlineto{\pgfqpoint{0.964482in}{1.245105in}}%
\pgfpathlineto{\pgfqpoint{0.945575in}{1.241587in}}%
\pgfpathclose%
\pgfusepath{fill}%
\end{pgfscope}%
\begin{pgfscope}%
\pgfpathrectangle{\pgfqpoint{0.041670in}{0.041670in}}{\pgfqpoint{2.216660in}{2.216660in}}%
\pgfusepath{clip}%
\pgfsetbuttcap%
\pgfsetroundjoin%
\definecolor{currentfill}{rgb}{0.272594,0.025563,0.353093}%
\pgfsetfillcolor{currentfill}%
\pgfsetlinewidth{0.000000pt}%
\definecolor{currentstroke}{rgb}{0.000000,0.000000,0.000000}%
\pgfsetstrokecolor{currentstroke}%
\pgfsetdash{}{0pt}%
\pgfpathmoveto{\pgfqpoint{1.329019in}{1.184096in}}%
\pgfpathlineto{\pgfqpoint{1.329907in}{1.188218in}}%
\pgfpathlineto{\pgfqpoint{1.330797in}{1.192625in}}%
\pgfpathlineto{\pgfqpoint{1.331689in}{1.197325in}}%
\pgfpathlineto{\pgfqpoint{1.332583in}{1.202320in}}%
\pgfpathlineto{\pgfqpoint{1.351104in}{1.199772in}}%
\pgfpathlineto{\pgfqpoint{1.369475in}{1.196932in}}%
\pgfpathlineto{\pgfqpoint{1.387682in}{1.193804in}}%
\pgfpathlineto{\pgfqpoint{1.386464in}{1.188852in}}%
\pgfpathlineto{\pgfqpoint{1.385248in}{1.184197in}}%
\pgfpathlineto{\pgfqpoint{1.384035in}{1.179833in}}%
\pgfpathlineto{\pgfqpoint{1.382825in}{1.175756in}}%
\pgfpathlineto{\pgfqpoint{1.365046in}{1.178820in}}%
\pgfpathlineto{\pgfqpoint{1.347106in}{1.181601in}}%
\pgfpathlineto{\pgfqpoint{1.329019in}{1.184096in}}%
\pgfpathclose%
\pgfusepath{fill}%
\end{pgfscope}%
\begin{pgfscope}%
\pgfpathrectangle{\pgfqpoint{0.041670in}{0.041670in}}{\pgfqpoint{2.216660in}{2.216660in}}%
\pgfusepath{clip}%
\pgfsetbuttcap%
\pgfsetroundjoin%
\definecolor{currentfill}{rgb}{0.276194,0.190074,0.493001}%
\pgfsetfillcolor{currentfill}%
\pgfsetlinewidth{0.000000pt}%
\definecolor{currentstroke}{rgb}{0.000000,0.000000,0.000000}%
\pgfsetstrokecolor{currentstroke}%
\pgfsetdash{}{0pt}%
\pgfpathmoveto{\pgfqpoint{1.343494in}{1.287319in}}%
\pgfpathlineto{\pgfqpoint{1.344421in}{1.296659in}}%
\pgfpathlineto{\pgfqpoint{1.345351in}{1.306373in}}%
\pgfpathlineto{\pgfqpoint{1.346285in}{1.316469in}}%
\pgfpathlineto{\pgfqpoint{1.347221in}{1.326951in}}%
\pgfpathlineto{\pgfqpoint{1.367525in}{1.324209in}}%
\pgfpathlineto{\pgfqpoint{1.387668in}{1.321153in}}%
\pgfpathlineto{\pgfqpoint{1.407632in}{1.317785in}}%
\pgfpathlineto{\pgfqpoint{1.406355in}{1.307340in}}%
\pgfpathlineto{\pgfqpoint{1.405082in}{1.297283in}}%
\pgfpathlineto{\pgfqpoint{1.403815in}{1.287606in}}%
\pgfpathlineto{\pgfqpoint{1.402551in}{1.278305in}}%
\pgfpathlineto{\pgfqpoint{1.383035in}{1.281617in}}%
\pgfpathlineto{\pgfqpoint{1.363343in}{1.284622in}}%
\pgfpathlineto{\pgfqpoint{1.343494in}{1.287319in}}%
\pgfpathclose%
\pgfusepath{fill}%
\end{pgfscope}%
\begin{pgfscope}%
\pgfpathrectangle{\pgfqpoint{0.041670in}{0.041670in}}{\pgfqpoint{2.216660in}{2.216660in}}%
\pgfusepath{clip}%
\pgfsetbuttcap%
\pgfsetroundjoin%
\definecolor{currentfill}{rgb}{0.272594,0.025563,0.353093}%
\pgfsetfillcolor{currentfill}%
\pgfsetlinewidth{0.000000pt}%
\definecolor{currentstroke}{rgb}{0.000000,0.000000,0.000000}%
\pgfsetstrokecolor{currentstroke}%
\pgfsetdash{}{0pt}%
\pgfpathmoveto{\pgfqpoint{0.961430in}{1.172797in}}%
\pgfpathlineto{\pgfqpoint{0.960126in}{1.176858in}}%
\pgfpathlineto{\pgfqpoint{0.958819in}{1.181206in}}%
\pgfpathlineto{\pgfqpoint{0.957509in}{1.185845in}}%
\pgfpathlineto{\pgfqpoint{0.956196in}{1.190781in}}%
\pgfpathlineto{\pgfqpoint{0.974242in}{1.194165in}}%
\pgfpathlineto{\pgfqpoint{0.992468in}{1.197262in}}%
\pgfpathlineto{\pgfqpoint{1.010857in}{1.200070in}}%
\pgfpathlineto{\pgfqpoint{1.029393in}{1.202585in}}%
\pgfpathlineto{\pgfqpoint{1.030275in}{1.197588in}}%
\pgfpathlineto{\pgfqpoint{1.031155in}{1.192888in}}%
\pgfpathlineto{\pgfqpoint{1.032033in}{1.188479in}}%
\pgfpathlineto{\pgfqpoint{1.032909in}{1.184355in}}%
\pgfpathlineto{\pgfqpoint{1.014807in}{1.181892in}}%
\pgfpathlineto{\pgfqpoint{0.996849in}{1.179143in}}%
\pgfpathlineto{\pgfqpoint{0.979052in}{1.176111in}}%
\pgfpathlineto{\pgfqpoint{0.961430in}{1.172797in}}%
\pgfpathclose%
\pgfusepath{fill}%
\end{pgfscope}%
\begin{pgfscope}%
\pgfpathrectangle{\pgfqpoint{0.041670in}{0.041670in}}{\pgfqpoint{2.216660in}{2.216660in}}%
\pgfusepath{clip}%
\pgfsetbuttcap%
\pgfsetroundjoin%
\definecolor{currentfill}{rgb}{0.267004,0.004874,0.329415}%
\pgfsetfillcolor{currentfill}%
\pgfsetlinewidth{0.000000pt}%
\definecolor{currentstroke}{rgb}{0.000000,0.000000,0.000000}%
\pgfsetstrokecolor{currentstroke}%
\pgfsetdash{}{0pt}%
\pgfpathmoveto{\pgfqpoint{1.039861in}{1.161087in}}%
\pgfpathlineto{\pgfqpoint{1.038997in}{1.163095in}}%
\pgfpathlineto{\pgfqpoint{1.038131in}{1.165352in}}%
\pgfpathlineto{\pgfqpoint{1.037265in}{1.167860in}}%
\pgfpathlineto{\pgfqpoint{1.036397in}{1.170626in}}%
\pgfpathlineto{\pgfqpoint{1.054194in}{1.172755in}}%
\pgfpathlineto{\pgfqpoint{1.072100in}{1.174601in}}%
\pgfpathlineto{\pgfqpoint{1.090102in}{1.176162in}}%
\pgfpathlineto{\pgfqpoint{1.108181in}{1.177439in}}%
\pgfpathlineto{\pgfqpoint{1.108615in}{1.174635in}}%
\pgfpathlineto{\pgfqpoint{1.109048in}{1.172087in}}%
\pgfpathlineto{\pgfqpoint{1.109480in}{1.169792in}}%
\pgfpathlineto{\pgfqpoint{1.109912in}{1.167744in}}%
\pgfpathlineto{\pgfqpoint{1.092268in}{1.166497in}}%
\pgfpathlineto{\pgfqpoint{1.074702in}{1.164971in}}%
\pgfpathlineto{\pgfqpoint{1.057227in}{1.163167in}}%
\pgfpathlineto{\pgfqpoint{1.039861in}{1.161087in}}%
\pgfpathclose%
\pgfusepath{fill}%
\end{pgfscope}%
\begin{pgfscope}%
\pgfpathrectangle{\pgfqpoint{0.041670in}{0.041670in}}{\pgfqpoint{2.216660in}{2.216660in}}%
\pgfusepath{clip}%
\pgfsetbuttcap%
\pgfsetroundjoin%
\definecolor{currentfill}{rgb}{0.267004,0.004874,0.329415}%
\pgfsetfillcolor{currentfill}%
\pgfsetlinewidth{0.000000pt}%
\definecolor{currentstroke}{rgb}{0.000000,0.000000,0.000000}%
\pgfsetstrokecolor{currentstroke}%
\pgfsetdash{}{0pt}%
\pgfpathmoveto{\pgfqpoint{1.251962in}{1.167619in}}%
\pgfpathlineto{\pgfqpoint{1.252405in}{1.169666in}}%
\pgfpathlineto{\pgfqpoint{1.252850in}{1.171961in}}%
\pgfpathlineto{\pgfqpoint{1.253295in}{1.174508in}}%
\pgfpathlineto{\pgfqpoint{1.253741in}{1.177311in}}%
\pgfpathlineto{\pgfqpoint{1.271812in}{1.176003in}}%
\pgfpathlineto{\pgfqpoint{1.289804in}{1.174409in}}%
\pgfpathlineto{\pgfqpoint{1.307699in}{1.172532in}}%
\pgfpathlineto{\pgfqpoint{1.325483in}{1.170372in}}%
\pgfpathlineto{\pgfqpoint{1.324603in}{1.167608in}}%
\pgfpathlineto{\pgfqpoint{1.323724in}{1.165101in}}%
\pgfpathlineto{\pgfqpoint{1.322847in}{1.162846in}}%
\pgfpathlineto{\pgfqpoint{1.321971in}{1.160839in}}%
\pgfpathlineto{\pgfqpoint{1.304618in}{1.162949in}}%
\pgfpathlineto{\pgfqpoint{1.287155in}{1.164784in}}%
\pgfpathlineto{\pgfqpoint{1.269597in}{1.166341in}}%
\pgfpathlineto{\pgfqpoint{1.251962in}{1.167619in}}%
\pgfpathclose%
\pgfusepath{fill}%
\end{pgfscope}%
\begin{pgfscope}%
\pgfpathrectangle{\pgfqpoint{0.041670in}{0.041670in}}{\pgfqpoint{2.216660in}{2.216660in}}%
\pgfusepath{clip}%
\pgfsetbuttcap%
\pgfsetroundjoin%
\definecolor{currentfill}{rgb}{0.276194,0.190074,0.493001}%
\pgfsetfillcolor{currentfill}%
\pgfsetlinewidth{0.000000pt}%
\definecolor{currentstroke}{rgb}{0.000000,0.000000,0.000000}%
\pgfsetstrokecolor{currentstroke}%
\pgfsetdash{}{0pt}%
\pgfpathmoveto{\pgfqpoint{0.940172in}{1.275106in}}%
\pgfpathlineto{\pgfqpoint{0.938810in}{1.284394in}}%
\pgfpathlineto{\pgfqpoint{0.937444in}{1.294056in}}%
\pgfpathlineto{\pgfqpoint{0.936072in}{1.304100in}}%
\pgfpathlineto{\pgfqpoint{0.934696in}{1.314532in}}%
\pgfpathlineto{\pgfqpoint{0.954487in}{1.318175in}}%
\pgfpathlineto{\pgfqpoint{0.974472in}{1.321508in}}%
\pgfpathlineto{\pgfqpoint{0.994633in}{1.324529in}}%
\pgfpathlineto{\pgfqpoint{1.014953in}{1.327236in}}%
\pgfpathlineto{\pgfqpoint{1.015877in}{1.316753in}}%
\pgfpathlineto{\pgfqpoint{1.016798in}{1.306656in}}%
\pgfpathlineto{\pgfqpoint{1.017716in}{1.296941in}}%
\pgfpathlineto{\pgfqpoint{1.018630in}{1.287599in}}%
\pgfpathlineto{\pgfqpoint{0.998764in}{1.284937in}}%
\pgfpathlineto{\pgfqpoint{0.979054in}{1.281966in}}%
\pgfpathlineto{\pgfqpoint{0.959518in}{1.278688in}}%
\pgfpathlineto{\pgfqpoint{0.940172in}{1.275106in}}%
\pgfpathclose%
\pgfusepath{fill}%
\end{pgfscope}%
\begin{pgfscope}%
\pgfpathrectangle{\pgfqpoint{0.041670in}{0.041670in}}{\pgfqpoint{2.216660in}{2.216660in}}%
\pgfusepath{clip}%
\pgfsetbuttcap%
\pgfsetroundjoin%
\definecolor{currentfill}{rgb}{0.233603,0.313828,0.543914}%
\pgfsetfillcolor{currentfill}%
\pgfsetlinewidth{0.000000pt}%
\definecolor{currentstroke}{rgb}{0.000000,0.000000,0.000000}%
\pgfsetstrokecolor{currentstroke}%
\pgfsetdash{}{0pt}%
\pgfpathmoveto{\pgfqpoint{1.011222in}{1.373165in}}%
\pgfpathlineto{\pgfqpoint{1.010280in}{1.385679in}}%
\pgfpathlineto{\pgfqpoint{1.009334in}{1.398619in}}%
\pgfpathlineto{\pgfqpoint{1.008384in}{1.411993in}}%
\pgfpathlineto{\pgfqpoint{1.007430in}{1.425806in}}%
\pgfpathlineto{\pgfqpoint{1.028827in}{1.428270in}}%
\pgfpathlineto{\pgfqpoint{1.050353in}{1.430406in}}%
\pgfpathlineto{\pgfqpoint{1.071989in}{1.432213in}}%
\pgfpathlineto{\pgfqpoint{1.093716in}{1.433690in}}%
\pgfpathlineto{\pgfqpoint{1.094193in}{1.419849in}}%
\pgfpathlineto{\pgfqpoint{1.094667in}{1.406448in}}%
\pgfpathlineto{\pgfqpoint{1.095139in}{1.393480in}}%
\pgfpathlineto{\pgfqpoint{1.095610in}{1.380937in}}%
\pgfpathlineto{\pgfqpoint{1.074360in}{1.379481in}}%
\pgfpathlineto{\pgfqpoint{1.053200in}{1.377700in}}%
\pgfpathlineto{\pgfqpoint{1.032148in}{1.375594in}}%
\pgfpathlineto{\pgfqpoint{1.011222in}{1.373165in}}%
\pgfpathclose%
\pgfusepath{fill}%
\end{pgfscope}%
\begin{pgfscope}%
\pgfpathrectangle{\pgfqpoint{0.041670in}{0.041670in}}{\pgfqpoint{2.216660in}{2.216660in}}%
\pgfusepath{clip}%
\pgfsetbuttcap%
\pgfsetroundjoin%
\definecolor{currentfill}{rgb}{0.268510,0.009605,0.335427}%
\pgfsetfillcolor{currentfill}%
\pgfsetlinewidth{0.000000pt}%
\definecolor{currentstroke}{rgb}{0.000000,0.000000,0.000000}%
\pgfsetstrokecolor{currentstroke}%
\pgfsetdash{}{0pt}%
\pgfpathmoveto{\pgfqpoint{1.325483in}{1.170372in}}%
\pgfpathlineto{\pgfqpoint{1.326365in}{1.173398in}}%
\pgfpathlineto{\pgfqpoint{1.327248in}{1.176691in}}%
\pgfpathlineto{\pgfqpoint{1.328132in}{1.180255in}}%
\pgfpathlineto{\pgfqpoint{1.329019in}{1.184096in}}%
\pgfpathlineto{\pgfqpoint{1.347106in}{1.181601in}}%
\pgfpathlineto{\pgfqpoint{1.365046in}{1.178820in}}%
\pgfpathlineto{\pgfqpoint{1.382825in}{1.175756in}}%
\pgfpathlineto{\pgfqpoint{1.381617in}{1.171961in}}%
\pgfpathlineto{\pgfqpoint{1.380412in}{1.168442in}}%
\pgfpathlineto{\pgfqpoint{1.379209in}{1.165195in}}%
\pgfpathlineto{\pgfqpoint{1.378007in}{1.162215in}}%
\pgfpathlineto{\pgfqpoint{1.360653in}{1.165212in}}%
\pgfpathlineto{\pgfqpoint{1.343139in}{1.167932in}}%
\pgfpathlineto{\pgfqpoint{1.325483in}{1.170372in}}%
\pgfpathclose%
\pgfusepath{fill}%
\end{pgfscope}%
\begin{pgfscope}%
\pgfpathrectangle{\pgfqpoint{0.041670in}{0.041670in}}{\pgfqpoint{2.216660in}{2.216660in}}%
\pgfusepath{clip}%
\pgfsetbuttcap%
\pgfsetroundjoin%
\definecolor{currentfill}{rgb}{0.267004,0.004874,0.329415}%
\pgfsetfillcolor{currentfill}%
\pgfsetlinewidth{0.000000pt}%
\definecolor{currentstroke}{rgb}{0.000000,0.000000,0.000000}%
\pgfsetstrokecolor{currentstroke}%
\pgfsetdash{}{0pt}%
\pgfpathmoveto{\pgfqpoint{1.111632in}{1.161938in}}%
\pgfpathlineto{\pgfqpoint{1.111203in}{1.163041in}}%
\pgfpathlineto{\pgfqpoint{1.110773in}{1.164373in}}%
\pgfpathlineto{\pgfqpoint{1.110343in}{1.165940in}}%
\pgfpathlineto{\pgfqpoint{1.109912in}{1.167744in}}%
\pgfpathlineto{\pgfqpoint{1.127616in}{1.168712in}}%
\pgfpathlineto{\pgfqpoint{1.145367in}{1.169399in}}%
\pgfpathlineto{\pgfqpoint{1.163148in}{1.169806in}}%
\pgfpathlineto{\pgfqpoint{1.180944in}{1.169931in}}%
\pgfpathlineto{\pgfqpoint{1.180938in}{1.168113in}}%
\pgfpathlineto{\pgfqpoint{1.180931in}{1.166534in}}%
\pgfpathlineto{\pgfqpoint{1.180925in}{1.165189in}}%
\pgfpathlineto{\pgfqpoint{1.180919in}{1.164073in}}%
\pgfpathlineto{\pgfqpoint{1.163561in}{1.163951in}}%
\pgfpathlineto{\pgfqpoint{1.146217in}{1.163554in}}%
\pgfpathlineto{\pgfqpoint{1.128902in}{1.162883in}}%
\pgfpathlineto{\pgfqpoint{1.111632in}{1.161938in}}%
\pgfpathclose%
\pgfusepath{fill}%
\end{pgfscope}%
\begin{pgfscope}%
\pgfpathrectangle{\pgfqpoint{0.041670in}{0.041670in}}{\pgfqpoint{2.216660in}{2.216660in}}%
\pgfusepath{clip}%
\pgfsetbuttcap%
\pgfsetroundjoin%
\definecolor{currentfill}{rgb}{0.233603,0.313828,0.543914}%
\pgfsetfillcolor{currentfill}%
\pgfsetlinewidth{0.000000pt}%
\definecolor{currentstroke}{rgb}{0.000000,0.000000,0.000000}%
\pgfsetstrokecolor{currentstroke}%
\pgfsetdash{}{0pt}%
\pgfpathmoveto{\pgfqpoint{1.266665in}{1.380791in}}%
\pgfpathlineto{\pgfqpoint{1.267149in}{1.393334in}}%
\pgfpathlineto{\pgfqpoint{1.267634in}{1.406302in}}%
\pgfpathlineto{\pgfqpoint{1.268122in}{1.419702in}}%
\pgfpathlineto{\pgfqpoint{1.268612in}{1.433542in}}%
\pgfpathlineto{\pgfqpoint{1.290330in}{1.432029in}}%
\pgfpathlineto{\pgfqpoint{1.311954in}{1.430185in}}%
\pgfpathlineto{\pgfqpoint{1.333467in}{1.428012in}}%
\pgfpathlineto{\pgfqpoint{1.354849in}{1.425512in}}%
\pgfpathlineto{\pgfqpoint{1.353882in}{1.411700in}}%
\pgfpathlineto{\pgfqpoint{1.352918in}{1.398328in}}%
\pgfpathlineto{\pgfqpoint{1.351959in}{1.385389in}}%
\pgfpathlineto{\pgfqpoint{1.351004in}{1.372876in}}%
\pgfpathlineto{\pgfqpoint{1.330094in}{1.375340in}}%
\pgfpathlineto{\pgfqpoint{1.309055in}{1.377482in}}%
\pgfpathlineto{\pgfqpoint{1.287906in}{1.379299in}}%
\pgfpathlineto{\pgfqpoint{1.266665in}{1.380791in}}%
\pgfpathclose%
\pgfusepath{fill}%
\end{pgfscope}%
\begin{pgfscope}%
\pgfpathrectangle{\pgfqpoint{0.041670in}{0.041670in}}{\pgfqpoint{2.216660in}{2.216660in}}%
\pgfusepath{clip}%
\pgfsetbuttcap%
\pgfsetroundjoin%
\definecolor{currentfill}{rgb}{0.267004,0.004874,0.329415}%
\pgfsetfillcolor{currentfill}%
\pgfsetlinewidth{0.000000pt}%
\definecolor{currentstroke}{rgb}{0.000000,0.000000,0.000000}%
\pgfsetstrokecolor{currentstroke}%
\pgfsetdash{}{0pt}%
\pgfpathmoveto{\pgfqpoint{1.180919in}{1.164073in}}%
\pgfpathlineto{\pgfqpoint{1.180925in}{1.165189in}}%
\pgfpathlineto{\pgfqpoint{1.180931in}{1.166534in}}%
\pgfpathlineto{\pgfqpoint{1.180938in}{1.168113in}}%
\pgfpathlineto{\pgfqpoint{1.180944in}{1.169931in}}%
\pgfpathlineto{\pgfqpoint{1.198738in}{1.169774in}}%
\pgfpathlineto{\pgfqpoint{1.216517in}{1.169337in}}%
\pgfpathlineto{\pgfqpoint{1.234263in}{1.168618in}}%
\pgfpathlineto{\pgfqpoint{1.251962in}{1.167619in}}%
\pgfpathlineto{\pgfqpoint{1.251519in}{1.165816in}}%
\pgfpathlineto{\pgfqpoint{1.251076in}{1.164250in}}%
\pgfpathlineto{\pgfqpoint{1.250634in}{1.162919in}}%
\pgfpathlineto{\pgfqpoint{1.250193in}{1.161817in}}%
\pgfpathlineto{\pgfqpoint{1.232929in}{1.162792in}}%
\pgfpathlineto{\pgfqpoint{1.215619in}{1.163493in}}%
\pgfpathlineto{\pgfqpoint{1.198277in}{1.163920in}}%
\pgfpathlineto{\pgfqpoint{1.180919in}{1.164073in}}%
\pgfpathclose%
\pgfusepath{fill}%
\end{pgfscope}%
\begin{pgfscope}%
\pgfpathrectangle{\pgfqpoint{0.041670in}{0.041670in}}{\pgfqpoint{2.216660in}{2.216660in}}%
\pgfusepath{clip}%
\pgfsetbuttcap%
\pgfsetroundjoin%
\definecolor{currentfill}{rgb}{0.268510,0.009605,0.335427}%
\pgfsetfillcolor{currentfill}%
\pgfsetlinewidth{0.000000pt}%
\definecolor{currentstroke}{rgb}{0.000000,0.000000,0.000000}%
\pgfsetstrokecolor{currentstroke}%
\pgfsetdash{}{0pt}%
\pgfpathmoveto{\pgfqpoint{0.966622in}{1.159320in}}%
\pgfpathlineto{\pgfqpoint{0.965327in}{1.162284in}}%
\pgfpathlineto{\pgfqpoint{0.964031in}{1.165515in}}%
\pgfpathlineto{\pgfqpoint{0.962731in}{1.169018in}}%
\pgfpathlineto{\pgfqpoint{0.961430in}{1.172797in}}%
\pgfpathlineto{\pgfqpoint{0.979052in}{1.176111in}}%
\pgfpathlineto{\pgfqpoint{0.996849in}{1.179143in}}%
\pgfpathlineto{\pgfqpoint{1.014807in}{1.181892in}}%
\pgfpathlineto{\pgfqpoint{1.032909in}{1.184355in}}%
\pgfpathlineto{\pgfqpoint{1.033783in}{1.180514in}}%
\pgfpathlineto{\pgfqpoint{1.034656in}{1.176948in}}%
\pgfpathlineto{\pgfqpoint{1.035527in}{1.173654in}}%
\pgfpathlineto{\pgfqpoint{1.036397in}{1.170626in}}%
\pgfpathlineto{\pgfqpoint{1.018725in}{1.168217in}}%
\pgfpathlineto{\pgfqpoint{1.001195in}{1.165528in}}%
\pgfpathlineto{\pgfqpoint{0.983822in}{1.162561in}}%
\pgfpathlineto{\pgfqpoint{0.966622in}{1.159320in}}%
\pgfpathclose%
\pgfusepath{fill}%
\end{pgfscope}%
\begin{pgfscope}%
\pgfpathrectangle{\pgfqpoint{0.041670in}{0.041670in}}{\pgfqpoint{2.216660in}{2.216660in}}%
\pgfusepath{clip}%
\pgfsetbuttcap%
\pgfsetroundjoin%
\definecolor{currentfill}{rgb}{0.260571,0.246922,0.522828}%
\pgfsetfillcolor{currentfill}%
\pgfsetlinewidth{0.000000pt}%
\definecolor{currentstroke}{rgb}{0.000000,0.000000,0.000000}%
\pgfsetstrokecolor{currentstroke}%
\pgfsetdash{}{0pt}%
\pgfpathmoveto{\pgfqpoint{1.347221in}{1.326951in}}%
\pgfpathlineto{\pgfqpoint{1.348162in}{1.337826in}}%
\pgfpathlineto{\pgfqpoint{1.349106in}{1.349101in}}%
\pgfpathlineto{\pgfqpoint{1.350053in}{1.360782in}}%
\pgfpathlineto{\pgfqpoint{1.351004in}{1.372876in}}%
\pgfpathlineto{\pgfqpoint{1.371769in}{1.370090in}}%
\pgfpathlineto{\pgfqpoint{1.392369in}{1.366986in}}%
\pgfpathlineto{\pgfqpoint{1.412788in}{1.363566in}}%
\pgfpathlineto{\pgfqpoint{1.411491in}{1.351507in}}%
\pgfpathlineto{\pgfqpoint{1.410200in}{1.339862in}}%
\pgfpathlineto{\pgfqpoint{1.408913in}{1.328624in}}%
\pgfpathlineto{\pgfqpoint{1.407632in}{1.317785in}}%
\pgfpathlineto{\pgfqpoint{1.387668in}{1.321153in}}%
\pgfpathlineto{\pgfqpoint{1.367525in}{1.324209in}}%
\pgfpathlineto{\pgfqpoint{1.347221in}{1.326951in}}%
\pgfpathclose%
\pgfusepath{fill}%
\end{pgfscope}%
\begin{pgfscope}%
\pgfpathrectangle{\pgfqpoint{0.041670in}{0.041670in}}{\pgfqpoint{2.216660in}{2.216660in}}%
\pgfusepath{clip}%
\pgfsetbuttcap%
\pgfsetroundjoin%
\definecolor{currentfill}{rgb}{0.260571,0.246922,0.522828}%
\pgfsetfillcolor{currentfill}%
\pgfsetlinewidth{0.000000pt}%
\definecolor{currentstroke}{rgb}{0.000000,0.000000,0.000000}%
\pgfsetstrokecolor{currentstroke}%
\pgfsetdash{}{0pt}%
\pgfpathmoveto{\pgfqpoint{0.934696in}{1.314532in}}%
\pgfpathlineto{\pgfqpoint{0.933315in}{1.325357in}}%
\pgfpathlineto{\pgfqpoint{0.931929in}{1.336583in}}%
\pgfpathlineto{\pgfqpoint{0.930537in}{1.348215in}}%
\pgfpathlineto{\pgfqpoint{0.929139in}{1.360261in}}%
\pgfpathlineto{\pgfqpoint{0.949381in}{1.363961in}}%
\pgfpathlineto{\pgfqpoint{0.969821in}{1.367347in}}%
\pgfpathlineto{\pgfqpoint{0.990440in}{1.370416in}}%
\pgfpathlineto{\pgfqpoint{1.011222in}{1.373165in}}%
\pgfpathlineto{\pgfqpoint{1.012160in}{1.361071in}}%
\pgfpathlineto{\pgfqpoint{1.013095in}{1.349388in}}%
\pgfpathlineto{\pgfqpoint{1.014026in}{1.338112in}}%
\pgfpathlineto{\pgfqpoint{1.014953in}{1.327236in}}%
\pgfpathlineto{\pgfqpoint{0.994633in}{1.324529in}}%
\pgfpathlineto{\pgfqpoint{0.974472in}{1.321508in}}%
\pgfpathlineto{\pgfqpoint{0.954487in}{1.318175in}}%
\pgfpathlineto{\pgfqpoint{0.934696in}{1.314532in}}%
\pgfpathclose%
\pgfusepath{fill}%
\end{pgfscope}%
\begin{pgfscope}%
\pgfpathrectangle{\pgfqpoint{0.041670in}{0.041670in}}{\pgfqpoint{2.216660in}{2.216660in}}%
\pgfusepath{clip}%
\pgfsetbuttcap%
\pgfsetroundjoin%
\definecolor{currentfill}{rgb}{0.201239,0.383670,0.554294}%
\pgfsetfillcolor{currentfill}%
\pgfsetlinewidth{0.000000pt}%
\definecolor{currentstroke}{rgb}{0.000000,0.000000,0.000000}%
\pgfsetstrokecolor{currentstroke}%
\pgfsetdash{}{0pt}%
\pgfpathmoveto{\pgfqpoint{1.093716in}{1.433690in}}%
\pgfpathlineto{\pgfqpoint{1.093238in}{1.447977in}}%
\pgfpathlineto{\pgfqpoint{1.092757in}{1.462718in}}%
\pgfpathlineto{\pgfqpoint{1.092274in}{1.477920in}}%
\pgfpathlineto{\pgfqpoint{1.091789in}{1.493591in}}%
\pgfpathlineto{\pgfqpoint{1.114077in}{1.494752in}}%
\pgfpathlineto{\pgfqpoint{1.136421in}{1.495576in}}%
\pgfpathlineto{\pgfqpoint{1.158801in}{1.496064in}}%
\pgfpathlineto{\pgfqpoint{1.181199in}{1.496214in}}%
\pgfpathlineto{\pgfqpoint{1.181193in}{1.480535in}}%
\pgfpathlineto{\pgfqpoint{1.181186in}{1.465324in}}%
\pgfpathlineto{\pgfqpoint{1.181179in}{1.450574in}}%
\pgfpathlineto{\pgfqpoint{1.181172in}{1.436278in}}%
\pgfpathlineto{\pgfqpoint{1.159263in}{1.436130in}}%
\pgfpathlineto{\pgfqpoint{1.137372in}{1.435649in}}%
\pgfpathlineto{\pgfqpoint{1.115517in}{1.434836in}}%
\pgfpathlineto{\pgfqpoint{1.093716in}{1.433690in}}%
\pgfpathclose%
\pgfusepath{fill}%
\end{pgfscope}%
\begin{pgfscope}%
\pgfpathrectangle{\pgfqpoint{0.041670in}{0.041670in}}{\pgfqpoint{2.216660in}{2.216660in}}%
\pgfusepath{clip}%
\pgfsetbuttcap%
\pgfsetroundjoin%
\definecolor{currentfill}{rgb}{0.201239,0.383670,0.554294}%
\pgfsetfillcolor{currentfill}%
\pgfsetlinewidth{0.000000pt}%
\definecolor{currentstroke}{rgb}{0.000000,0.000000,0.000000}%
\pgfsetstrokecolor{currentstroke}%
\pgfsetdash{}{0pt}%
\pgfpathmoveto{\pgfqpoint{1.181172in}{1.436278in}}%
\pgfpathlineto{\pgfqpoint{1.181179in}{1.450574in}}%
\pgfpathlineto{\pgfqpoint{1.181186in}{1.465324in}}%
\pgfpathlineto{\pgfqpoint{1.181193in}{1.480535in}}%
\pgfpathlineto{\pgfqpoint{1.181199in}{1.496214in}}%
\pgfpathlineto{\pgfqpoint{1.203597in}{1.496026in}}%
\pgfpathlineto{\pgfqpoint{1.225974in}{1.495501in}}%
\pgfpathlineto{\pgfqpoint{1.248312in}{1.494640in}}%
\pgfpathlineto{\pgfqpoint{1.270593in}{1.493441in}}%
\pgfpathlineto{\pgfqpoint{1.270094in}{1.477771in}}%
\pgfpathlineto{\pgfqpoint{1.269598in}{1.462570in}}%
\pgfpathlineto{\pgfqpoint{1.269104in}{1.447829in}}%
\pgfpathlineto{\pgfqpoint{1.268612in}{1.433542in}}%
\pgfpathlineto{\pgfqpoint{1.246818in}{1.434725in}}%
\pgfpathlineto{\pgfqpoint{1.224968in}{1.435575in}}%
\pgfpathlineto{\pgfqpoint{1.203080in}{1.436093in}}%
\pgfpathlineto{\pgfqpoint{1.181172in}{1.436278in}}%
\pgfpathclose%
\pgfusepath{fill}%
\end{pgfscope}%
\begin{pgfscope}%
\pgfpathrectangle{\pgfqpoint{0.041670in}{0.041670in}}{\pgfqpoint{2.216660in}{2.216660in}}%
\pgfusepath{clip}%
\pgfsetbuttcap%
\pgfsetroundjoin%
\definecolor{currentfill}{rgb}{0.267004,0.004874,0.329415}%
\pgfsetfillcolor{currentfill}%
\pgfsetlinewidth{0.000000pt}%
\definecolor{currentstroke}{rgb}{0.000000,0.000000,0.000000}%
\pgfsetstrokecolor{currentstroke}%
\pgfsetdash{}{0pt}%
\pgfpathmoveto{\pgfqpoint{1.043305in}{1.155439in}}%
\pgfpathlineto{\pgfqpoint{1.042446in}{1.156502in}}%
\pgfpathlineto{\pgfqpoint{1.041585in}{1.157795in}}%
\pgfpathlineto{\pgfqpoint{1.040724in}{1.159321in}}%
\pgfpathlineto{\pgfqpoint{1.039861in}{1.161087in}}%
\pgfpathlineto{\pgfqpoint{1.057227in}{1.163167in}}%
\pgfpathlineto{\pgfqpoint{1.074702in}{1.164971in}}%
\pgfpathlineto{\pgfqpoint{1.092268in}{1.166497in}}%
\pgfpathlineto{\pgfqpoint{1.109912in}{1.167744in}}%
\pgfpathlineto{\pgfqpoint{1.110343in}{1.165940in}}%
\pgfpathlineto{\pgfqpoint{1.110773in}{1.164373in}}%
\pgfpathlineto{\pgfqpoint{1.111203in}{1.163041in}}%
\pgfpathlineto{\pgfqpoint{1.111632in}{1.161938in}}%
\pgfpathlineto{\pgfqpoint{1.094423in}{1.160721in}}%
\pgfpathlineto{\pgfqpoint{1.077288in}{1.159231in}}%
\pgfpathlineto{\pgfqpoint{1.060244in}{1.157470in}}%
\pgfpathlineto{\pgfqpoint{1.043305in}{1.155439in}}%
\pgfpathclose%
\pgfusepath{fill}%
\end{pgfscope}%
\begin{pgfscope}%
\pgfpathrectangle{\pgfqpoint{0.041670in}{0.041670in}}{\pgfqpoint{2.216660in}{2.216660in}}%
\pgfusepath{clip}%
\pgfsetbuttcap%
\pgfsetroundjoin%
\definecolor{currentfill}{rgb}{0.267004,0.004874,0.329415}%
\pgfsetfillcolor{currentfill}%
\pgfsetlinewidth{0.000000pt}%
\definecolor{currentstroke}{rgb}{0.000000,0.000000,0.000000}%
\pgfsetstrokecolor{currentstroke}%
\pgfsetdash{}{0pt}%
\pgfpathmoveto{\pgfqpoint{1.250193in}{1.161817in}}%
\pgfpathlineto{\pgfqpoint{1.250634in}{1.162919in}}%
\pgfpathlineto{\pgfqpoint{1.251076in}{1.164250in}}%
\pgfpathlineto{\pgfqpoint{1.251519in}{1.165816in}}%
\pgfpathlineto{\pgfqpoint{1.251962in}{1.167619in}}%
\pgfpathlineto{\pgfqpoint{1.269597in}{1.166341in}}%
\pgfpathlineto{\pgfqpoint{1.287155in}{1.164784in}}%
\pgfpathlineto{\pgfqpoint{1.304618in}{1.162949in}}%
\pgfpathlineto{\pgfqpoint{1.321971in}{1.160839in}}%
\pgfpathlineto{\pgfqpoint{1.321097in}{1.159075in}}%
\pgfpathlineto{\pgfqpoint{1.320223in}{1.157549in}}%
\pgfpathlineto{\pgfqpoint{1.319351in}{1.156258in}}%
\pgfpathlineto{\pgfqpoint{1.318479in}{1.155197in}}%
\pgfpathlineto{\pgfqpoint{1.301553in}{1.157257in}}%
\pgfpathlineto{\pgfqpoint{1.284520in}{1.159048in}}%
\pgfpathlineto{\pgfqpoint{1.267395in}{1.160568in}}%
\pgfpathlineto{\pgfqpoint{1.250193in}{1.161817in}}%
\pgfpathclose%
\pgfusepath{fill}%
\end{pgfscope}%
\begin{pgfscope}%
\pgfpathrectangle{\pgfqpoint{0.041670in}{0.041670in}}{\pgfqpoint{2.216660in}{2.216660in}}%
\pgfusepath{clip}%
\pgfsetbuttcap%
\pgfsetroundjoin%
\definecolor{currentfill}{rgb}{0.267004,0.004874,0.329415}%
\pgfsetfillcolor{currentfill}%
\pgfsetlinewidth{0.000000pt}%
\definecolor{currentstroke}{rgb}{0.000000,0.000000,0.000000}%
\pgfsetstrokecolor{currentstroke}%
\pgfsetdash{}{0pt}%
\pgfpathmoveto{\pgfqpoint{1.321971in}{1.160839in}}%
\pgfpathlineto{\pgfqpoint{1.322847in}{1.162846in}}%
\pgfpathlineto{\pgfqpoint{1.323724in}{1.165101in}}%
\pgfpathlineto{\pgfqpoint{1.324603in}{1.167608in}}%
\pgfpathlineto{\pgfqpoint{1.325483in}{1.170372in}}%
\pgfpathlineto{\pgfqpoint{1.343139in}{1.167932in}}%
\pgfpathlineto{\pgfqpoint{1.360653in}{1.165212in}}%
\pgfpathlineto{\pgfqpoint{1.378007in}{1.162215in}}%
\pgfpathlineto{\pgfqpoint{1.376808in}{1.159497in}}%
\pgfpathlineto{\pgfqpoint{1.375611in}{1.157036in}}%
\pgfpathlineto{\pgfqpoint{1.374416in}{1.154827in}}%
\pgfpathlineto{\pgfqpoint{1.373223in}{1.152867in}}%
\pgfpathlineto{\pgfqpoint{1.356289in}{1.155796in}}%
\pgfpathlineto{\pgfqpoint{1.339200in}{1.158454in}}%
\pgfpathlineto{\pgfqpoint{1.321971in}{1.160839in}}%
\pgfpathclose%
\pgfusepath{fill}%
\end{pgfscope}%
\begin{pgfscope}%
\pgfpathrectangle{\pgfqpoint{0.041670in}{0.041670in}}{\pgfqpoint{2.216660in}{2.216660in}}%
\pgfusepath{clip}%
\pgfsetbuttcap%
\pgfsetroundjoin%
\definecolor{currentfill}{rgb}{0.282327,0.094955,0.417331}%
\pgfsetfillcolor{currentfill}%
\pgfsetlinewidth{0.000000pt}%
\definecolor{currentstroke}{rgb}{0.000000,0.000000,0.000000}%
\pgfsetstrokecolor{currentstroke}%
\pgfsetdash{}{0pt}%
\pgfpathmoveto{\pgfqpoint{1.392584in}{1.216681in}}%
\pgfpathlineto{\pgfqpoint{1.393817in}{1.223195in}}%
\pgfpathlineto{\pgfqpoint{1.395054in}{1.230037in}}%
\pgfpathlineto{\pgfqpoint{1.396294in}{1.237213in}}%
\pgfpathlineto{\pgfqpoint{1.397538in}{1.244729in}}%
\pgfpathlineto{\pgfqpoint{1.416424in}{1.241177in}}%
\pgfpathlineto{\pgfqpoint{1.435104in}{1.237330in}}%
\pgfpathlineto{\pgfqpoint{1.453562in}{1.233189in}}%
\pgfpathlineto{\pgfqpoint{1.471782in}{1.228759in}}%
\pgfpathlineto{\pgfqpoint{1.470110in}{1.221317in}}%
\pgfpathlineto{\pgfqpoint{1.468443in}{1.214215in}}%
\pgfpathlineto{\pgfqpoint{1.466781in}{1.207449in}}%
\pgfpathlineto{\pgfqpoint{1.465123in}{1.201011in}}%
\pgfpathlineto{\pgfqpoint{1.447323in}{1.205358in}}%
\pgfpathlineto{\pgfqpoint{1.429288in}{1.209421in}}%
\pgfpathlineto{\pgfqpoint{1.411037in}{1.213196in}}%
\pgfpathlineto{\pgfqpoint{1.392584in}{1.216681in}}%
\pgfpathclose%
\pgfusepath{fill}%
\end{pgfscope}%
\begin{pgfscope}%
\pgfpathrectangle{\pgfqpoint{0.041670in}{0.041670in}}{\pgfqpoint{2.216660in}{2.216660in}}%
\pgfusepath{clip}%
\pgfsetbuttcap%
\pgfsetroundjoin%
\definecolor{currentfill}{rgb}{0.277941,0.056324,0.381191}%
\pgfsetfillcolor{currentfill}%
\pgfsetlinewidth{0.000000pt}%
\definecolor{currentstroke}{rgb}{0.000000,0.000000,0.000000}%
\pgfsetstrokecolor{currentstroke}%
\pgfsetdash{}{0pt}%
\pgfpathmoveto{\pgfqpoint{1.387682in}{1.193804in}}%
\pgfpathlineto{\pgfqpoint{1.388903in}{1.199057in}}%
\pgfpathlineto{\pgfqpoint{1.390127in}{1.204618in}}%
\pgfpathlineto{\pgfqpoint{1.391354in}{1.210490in}}%
\pgfpathlineto{\pgfqpoint{1.392584in}{1.216681in}}%
\pgfpathlineto{\pgfqpoint{1.411037in}{1.213196in}}%
\pgfpathlineto{\pgfqpoint{1.429288in}{1.209421in}}%
\pgfpathlineto{\pgfqpoint{1.447323in}{1.205358in}}%
\pgfpathlineto{\pgfqpoint{1.465123in}{1.201011in}}%
\pgfpathlineto{\pgfqpoint{1.463469in}{1.194897in}}%
\pgfpathlineto{\pgfqpoint{1.461820in}{1.189102in}}%
\pgfpathlineto{\pgfqpoint{1.460175in}{1.183620in}}%
\pgfpathlineto{\pgfqpoint{1.458534in}{1.178445in}}%
\pgfpathlineto{\pgfqpoint{1.441149in}{1.182705in}}%
\pgfpathlineto{\pgfqpoint{1.423534in}{1.186687in}}%
\pgfpathlineto{\pgfqpoint{1.405707in}{1.190388in}}%
\pgfpathlineto{\pgfqpoint{1.387682in}{1.193804in}}%
\pgfpathclose%
\pgfusepath{fill}%
\end{pgfscope}%
\begin{pgfscope}%
\pgfpathrectangle{\pgfqpoint{0.041670in}{0.041670in}}{\pgfqpoint{2.216660in}{2.216660in}}%
\pgfusepath{clip}%
\pgfsetbuttcap%
\pgfsetroundjoin%
\definecolor{currentfill}{rgb}{0.267004,0.004874,0.329415}%
\pgfsetfillcolor{currentfill}%
\pgfsetlinewidth{0.000000pt}%
\definecolor{currentstroke}{rgb}{0.000000,0.000000,0.000000}%
\pgfsetstrokecolor{currentstroke}%
\pgfsetdash{}{0pt}%
\pgfpathmoveto{\pgfqpoint{0.971778in}{1.150039in}}%
\pgfpathlineto{\pgfqpoint{0.970492in}{1.151982in}}%
\pgfpathlineto{\pgfqpoint{0.969204in}{1.154174in}}%
\pgfpathlineto{\pgfqpoint{0.967914in}{1.156618in}}%
\pgfpathlineto{\pgfqpoint{0.966622in}{1.159320in}}%
\pgfpathlineto{\pgfqpoint{0.983822in}{1.162561in}}%
\pgfpathlineto{\pgfqpoint{1.001195in}{1.165528in}}%
\pgfpathlineto{\pgfqpoint{1.018725in}{1.168217in}}%
\pgfpathlineto{\pgfqpoint{1.036397in}{1.170626in}}%
\pgfpathlineto{\pgfqpoint{1.037265in}{1.167860in}}%
\pgfpathlineto{\pgfqpoint{1.038131in}{1.165352in}}%
\pgfpathlineto{\pgfqpoint{1.038997in}{1.163095in}}%
\pgfpathlineto{\pgfqpoint{1.039861in}{1.161087in}}%
\pgfpathlineto{\pgfqpoint{1.022617in}{1.158732in}}%
\pgfpathlineto{\pgfqpoint{1.005512in}{1.156104in}}%
\pgfpathlineto{\pgfqpoint{0.988560in}{1.153206in}}%
\pgfpathlineto{\pgfqpoint{0.971778in}{1.150039in}}%
\pgfpathclose%
\pgfusepath{fill}%
\end{pgfscope}%
\begin{pgfscope}%
\pgfpathrectangle{\pgfqpoint{0.041670in}{0.041670in}}{\pgfqpoint{2.216660in}{2.216660in}}%
\pgfusepath{clip}%
\pgfsetbuttcap%
\pgfsetroundjoin%
\definecolor{currentfill}{rgb}{0.282884,0.135920,0.453427}%
\pgfsetfillcolor{currentfill}%
\pgfsetlinewidth{0.000000pt}%
\definecolor{currentstroke}{rgb}{0.000000,0.000000,0.000000}%
\pgfsetstrokecolor{currentstroke}%
\pgfsetdash{}{0pt}%
\pgfpathmoveto{\pgfqpoint{1.397538in}{1.244729in}}%
\pgfpathlineto{\pgfqpoint{1.398785in}{1.252591in}}%
\pgfpathlineto{\pgfqpoint{1.400036in}{1.260803in}}%
\pgfpathlineto{\pgfqpoint{1.401292in}{1.269373in}}%
\pgfpathlineto{\pgfqpoint{1.402551in}{1.278305in}}%
\pgfpathlineto{\pgfqpoint{1.421875in}{1.274689in}}%
\pgfpathlineto{\pgfqpoint{1.440989in}{1.270772in}}%
\pgfpathlineto{\pgfqpoint{1.459878in}{1.266557in}}%
\pgfpathlineto{\pgfqpoint{1.478523in}{1.262046in}}%
\pgfpathlineto{\pgfqpoint{1.476829in}{1.253184in}}%
\pgfpathlineto{\pgfqpoint{1.475142in}{1.244687in}}%
\pgfpathlineto{\pgfqpoint{1.473459in}{1.236547in}}%
\pgfpathlineto{\pgfqpoint{1.471782in}{1.228759in}}%
\pgfpathlineto{\pgfqpoint{1.453562in}{1.233189in}}%
\pgfpathlineto{\pgfqpoint{1.435104in}{1.237330in}}%
\pgfpathlineto{\pgfqpoint{1.416424in}{1.241177in}}%
\pgfpathlineto{\pgfqpoint{1.397538in}{1.244729in}}%
\pgfpathclose%
\pgfusepath{fill}%
\end{pgfscope}%
\begin{pgfscope}%
\pgfpathrectangle{\pgfqpoint{0.041670in}{0.041670in}}{\pgfqpoint{2.216660in}{2.216660in}}%
\pgfusepath{clip}%
\pgfsetbuttcap%
\pgfsetroundjoin%
\definecolor{currentfill}{rgb}{0.272594,0.025563,0.353093}%
\pgfsetfillcolor{currentfill}%
\pgfsetlinewidth{0.000000pt}%
\definecolor{currentstroke}{rgb}{0.000000,0.000000,0.000000}%
\pgfsetstrokecolor{currentstroke}%
\pgfsetdash{}{0pt}%
\pgfpathmoveto{\pgfqpoint{1.382825in}{1.175756in}}%
\pgfpathlineto{\pgfqpoint{1.384035in}{1.179833in}}%
\pgfpathlineto{\pgfqpoint{1.385248in}{1.184197in}}%
\pgfpathlineto{\pgfqpoint{1.386464in}{1.188852in}}%
\pgfpathlineto{\pgfqpoint{1.387682in}{1.193804in}}%
\pgfpathlineto{\pgfqpoint{1.405707in}{1.190388in}}%
\pgfpathlineto{\pgfqpoint{1.423534in}{1.186687in}}%
\pgfpathlineto{\pgfqpoint{1.441149in}{1.182705in}}%
\pgfpathlineto{\pgfqpoint{1.458534in}{1.178445in}}%
\pgfpathlineto{\pgfqpoint{1.456897in}{1.173572in}}%
\pgfpathlineto{\pgfqpoint{1.455263in}{1.168997in}}%
\pgfpathlineto{\pgfqpoint{1.453633in}{1.164714in}}%
\pgfpathlineto{\pgfqpoint{1.452007in}{1.160718in}}%
\pgfpathlineto{\pgfqpoint{1.435033in}{1.164890in}}%
\pgfpathlineto{\pgfqpoint{1.417834in}{1.168789in}}%
\pgfpathlineto{\pgfqpoint{1.400426in}{1.172412in}}%
\pgfpathlineto{\pgfqpoint{1.382825in}{1.175756in}}%
\pgfpathclose%
\pgfusepath{fill}%
\end{pgfscope}%
\begin{pgfscope}%
\pgfpathrectangle{\pgfqpoint{0.041670in}{0.041670in}}{\pgfqpoint{2.216660in}{2.216660in}}%
\pgfusepath{clip}%
\pgfsetbuttcap%
\pgfsetroundjoin%
\definecolor{currentfill}{rgb}{0.268510,0.009605,0.335427}%
\pgfsetfillcolor{currentfill}%
\pgfsetlinewidth{0.000000pt}%
\definecolor{currentstroke}{rgb}{0.000000,0.000000,0.000000}%
\pgfsetstrokecolor{currentstroke}%
\pgfsetdash{}{0pt}%
\pgfpathmoveto{\pgfqpoint{1.113345in}{1.159735in}}%
\pgfpathlineto{\pgfqpoint{1.112917in}{1.159963in}}%
\pgfpathlineto{\pgfqpoint{1.112489in}{1.160404in}}%
\pgfpathlineto{\pgfqpoint{1.112061in}{1.161061in}}%
\pgfpathlineto{\pgfqpoint{1.111632in}{1.161938in}}%
\pgfpathlineto{\pgfqpoint{1.128902in}{1.162883in}}%
\pgfpathlineto{\pgfqpoint{1.146217in}{1.163554in}}%
\pgfpathlineto{\pgfqpoint{1.163561in}{1.163951in}}%
\pgfpathlineto{\pgfqpoint{1.180919in}{1.164073in}}%
\pgfpathlineto{\pgfqpoint{1.180913in}{1.163182in}}%
\pgfpathlineto{\pgfqpoint{1.180907in}{1.162512in}}%
\pgfpathlineto{\pgfqpoint{1.180901in}{1.162058in}}%
\pgfpathlineto{\pgfqpoint{1.180895in}{1.161816in}}%
\pgfpathlineto{\pgfqpoint{1.163972in}{1.161697in}}%
\pgfpathlineto{\pgfqpoint{1.147062in}{1.161310in}}%
\pgfpathlineto{\pgfqpoint{1.130182in}{1.160656in}}%
\pgfpathlineto{\pgfqpoint{1.113345in}{1.159735in}}%
\pgfpathclose%
\pgfusepath{fill}%
\end{pgfscope}%
\begin{pgfscope}%
\pgfpathrectangle{\pgfqpoint{0.041670in}{0.041670in}}{\pgfqpoint{2.216660in}{2.216660in}}%
\pgfusepath{clip}%
\pgfsetbuttcap%
\pgfsetroundjoin%
\definecolor{currentfill}{rgb}{0.268510,0.009605,0.335427}%
\pgfsetfillcolor{currentfill}%
\pgfsetlinewidth{0.000000pt}%
\definecolor{currentstroke}{rgb}{0.000000,0.000000,0.000000}%
\pgfsetstrokecolor{currentstroke}%
\pgfsetdash{}{0pt}%
\pgfpathmoveto{\pgfqpoint{1.180895in}{1.161816in}}%
\pgfpathlineto{\pgfqpoint{1.180901in}{1.162058in}}%
\pgfpathlineto{\pgfqpoint{1.180907in}{1.162512in}}%
\pgfpathlineto{\pgfqpoint{1.180913in}{1.163182in}}%
\pgfpathlineto{\pgfqpoint{1.180919in}{1.164073in}}%
\pgfpathlineto{\pgfqpoint{1.198277in}{1.163920in}}%
\pgfpathlineto{\pgfqpoint{1.215619in}{1.163493in}}%
\pgfpathlineto{\pgfqpoint{1.232929in}{1.162792in}}%
\pgfpathlineto{\pgfqpoint{1.250193in}{1.161817in}}%
\pgfpathlineto{\pgfqpoint{1.249752in}{1.160940in}}%
\pgfpathlineto{\pgfqpoint{1.249312in}{1.160283in}}%
\pgfpathlineto{\pgfqpoint{1.248872in}{1.159844in}}%
\pgfpathlineto{\pgfqpoint{1.248432in}{1.159616in}}%
\pgfpathlineto{\pgfqpoint{1.231601in}{1.160567in}}%
\pgfpathlineto{\pgfqpoint{1.214725in}{1.161251in}}%
\pgfpathlineto{\pgfqpoint{1.197818in}{1.161667in}}%
\pgfpathlineto{\pgfqpoint{1.180895in}{1.161816in}}%
\pgfpathclose%
\pgfusepath{fill}%
\end{pgfscope}%
\begin{pgfscope}%
\pgfpathrectangle{\pgfqpoint{0.041670in}{0.041670in}}{\pgfqpoint{2.216660in}{2.216660in}}%
\pgfusepath{clip}%
\pgfsetbuttcap%
\pgfsetroundjoin%
\definecolor{currentfill}{rgb}{0.282327,0.094955,0.417331}%
\pgfsetfillcolor{currentfill}%
\pgfsetlinewidth{0.000000pt}%
\definecolor{currentstroke}{rgb}{0.000000,0.000000,0.000000}%
\pgfsetstrokecolor{currentstroke}%
\pgfsetdash{}{0pt}%
\pgfpathmoveto{\pgfqpoint{0.879174in}{1.196911in}}%
\pgfpathlineto{\pgfqpoint{0.877424in}{1.203328in}}%
\pgfpathlineto{\pgfqpoint{0.875670in}{1.210075in}}%
\pgfpathlineto{\pgfqpoint{0.873910in}{1.217157in}}%
\pgfpathlineto{\pgfqpoint{0.872145in}{1.224579in}}%
\pgfpathlineto{\pgfqpoint{0.890139in}{1.229265in}}%
\pgfpathlineto{\pgfqpoint{0.908387in}{1.233664in}}%
\pgfpathlineto{\pgfqpoint{0.926871in}{1.237772in}}%
\pgfpathlineto{\pgfqpoint{0.945575in}{1.241587in}}%
\pgfpathlineto{\pgfqpoint{0.946915in}{1.234085in}}%
\pgfpathlineto{\pgfqpoint{0.948252in}{1.226924in}}%
\pgfpathlineto{\pgfqpoint{0.949585in}{1.220096in}}%
\pgfpathlineto{\pgfqpoint{0.950914in}{1.213598in}}%
\pgfpathlineto{\pgfqpoint{0.932639in}{1.209855in}}%
\pgfpathlineto{\pgfqpoint{0.914580in}{1.205824in}}%
\pgfpathlineto{\pgfqpoint{0.896753in}{1.201508in}}%
\pgfpathlineto{\pgfqpoint{0.879174in}{1.196911in}}%
\pgfpathclose%
\pgfusepath{fill}%
\end{pgfscope}%
\begin{pgfscope}%
\pgfpathrectangle{\pgfqpoint{0.041670in}{0.041670in}}{\pgfqpoint{2.216660in}{2.216660in}}%
\pgfusepath{clip}%
\pgfsetbuttcap%
\pgfsetroundjoin%
\definecolor{currentfill}{rgb}{0.201239,0.383670,0.554294}%
\pgfsetfillcolor{currentfill}%
\pgfsetlinewidth{0.000000pt}%
\definecolor{currentstroke}{rgb}{0.000000,0.000000,0.000000}%
\pgfsetstrokecolor{currentstroke}%
\pgfsetdash{}{0pt}%
\pgfpathmoveto{\pgfqpoint{1.007430in}{1.425806in}}%
\pgfpathlineto{\pgfqpoint{1.006471in}{1.440066in}}%
\pgfpathlineto{\pgfqpoint{1.005508in}{1.454781in}}%
\pgfpathlineto{\pgfqpoint{1.004541in}{1.469958in}}%
\pgfpathlineto{\pgfqpoint{1.003569in}{1.485603in}}%
\pgfpathlineto{\pgfqpoint{1.025447in}{1.488099in}}%
\pgfpathlineto{\pgfqpoint{1.047455in}{1.490264in}}%
\pgfpathlineto{\pgfqpoint{1.069575in}{1.492095in}}%
\pgfpathlineto{\pgfqpoint{1.091789in}{1.493591in}}%
\pgfpathlineto{\pgfqpoint{1.092274in}{1.477920in}}%
\pgfpathlineto{\pgfqpoint{1.092757in}{1.462718in}}%
\pgfpathlineto{\pgfqpoint{1.093238in}{1.447977in}}%
\pgfpathlineto{\pgfqpoint{1.093716in}{1.433690in}}%
\pgfpathlineto{\pgfqpoint{1.071989in}{1.432213in}}%
\pgfpathlineto{\pgfqpoint{1.050353in}{1.430406in}}%
\pgfpathlineto{\pgfqpoint{1.028827in}{1.428270in}}%
\pgfpathlineto{\pgfqpoint{1.007430in}{1.425806in}}%
\pgfpathclose%
\pgfusepath{fill}%
\end{pgfscope}%
\begin{pgfscope}%
\pgfpathrectangle{\pgfqpoint{0.041670in}{0.041670in}}{\pgfqpoint{2.216660in}{2.216660in}}%
\pgfusepath{clip}%
\pgfsetbuttcap%
\pgfsetroundjoin%
\definecolor{currentfill}{rgb}{0.233603,0.313828,0.543914}%
\pgfsetfillcolor{currentfill}%
\pgfsetlinewidth{0.000000pt}%
\definecolor{currentstroke}{rgb}{0.000000,0.000000,0.000000}%
\pgfsetstrokecolor{currentstroke}%
\pgfsetdash{}{0pt}%
\pgfpathmoveto{\pgfqpoint{1.351004in}{1.372876in}}%
\pgfpathlineto{\pgfqpoint{1.351959in}{1.385389in}}%
\pgfpathlineto{\pgfqpoint{1.352918in}{1.398328in}}%
\pgfpathlineto{\pgfqpoint{1.353882in}{1.411700in}}%
\pgfpathlineto{\pgfqpoint{1.354849in}{1.425512in}}%
\pgfpathlineto{\pgfqpoint{1.376082in}{1.422687in}}%
\pgfpathlineto{\pgfqpoint{1.397148in}{1.419538in}}%
\pgfpathlineto{\pgfqpoint{1.418028in}{1.416067in}}%
\pgfpathlineto{\pgfqpoint{1.416710in}{1.402288in}}%
\pgfpathlineto{\pgfqpoint{1.415397in}{1.388949in}}%
\pgfpathlineto{\pgfqpoint{1.414090in}{1.376044in}}%
\pgfpathlineto{\pgfqpoint{1.412788in}{1.363566in}}%
\pgfpathlineto{\pgfqpoint{1.392369in}{1.366986in}}%
\pgfpathlineto{\pgfqpoint{1.371769in}{1.370090in}}%
\pgfpathlineto{\pgfqpoint{1.351004in}{1.372876in}}%
\pgfpathclose%
\pgfusepath{fill}%
\end{pgfscope}%
\begin{pgfscope}%
\pgfpathrectangle{\pgfqpoint{0.041670in}{0.041670in}}{\pgfqpoint{2.216660in}{2.216660in}}%
\pgfusepath{clip}%
\pgfsetbuttcap%
\pgfsetroundjoin%
\definecolor{currentfill}{rgb}{0.277941,0.056324,0.381191}%
\pgfsetfillcolor{currentfill}%
\pgfsetlinewidth{0.000000pt}%
\definecolor{currentstroke}{rgb}{0.000000,0.000000,0.000000}%
\pgfsetstrokecolor{currentstroke}%
\pgfsetdash{}{0pt}%
\pgfpathmoveto{\pgfqpoint{0.886127in}{1.174426in}}%
\pgfpathlineto{\pgfqpoint{0.884395in}{1.179580in}}%
\pgfpathlineto{\pgfqpoint{0.882659in}{1.185042in}}%
\pgfpathlineto{\pgfqpoint{0.880919in}{1.190817in}}%
\pgfpathlineto{\pgfqpoint{0.879174in}{1.196911in}}%
\pgfpathlineto{\pgfqpoint{0.896753in}{1.201508in}}%
\pgfpathlineto{\pgfqpoint{0.914580in}{1.205824in}}%
\pgfpathlineto{\pgfqpoint{0.932639in}{1.209855in}}%
\pgfpathlineto{\pgfqpoint{0.950914in}{1.213598in}}%
\pgfpathlineto{\pgfqpoint{0.952239in}{1.207422in}}%
\pgfpathlineto{\pgfqpoint{0.953562in}{1.201565in}}%
\pgfpathlineto{\pgfqpoint{0.954880in}{1.196019in}}%
\pgfpathlineto{\pgfqpoint{0.956196in}{1.190781in}}%
\pgfpathlineto{\pgfqpoint{0.938346in}{1.187112in}}%
\pgfpathlineto{\pgfqpoint{0.920707in}{1.183162in}}%
\pgfpathlineto{\pgfqpoint{0.903296in}{1.178932in}}%
\pgfpathlineto{\pgfqpoint{0.886127in}{1.174426in}}%
\pgfpathclose%
\pgfusepath{fill}%
\end{pgfscope}%
\begin{pgfscope}%
\pgfpathrectangle{\pgfqpoint{0.041670in}{0.041670in}}{\pgfqpoint{2.216660in}{2.216660in}}%
\pgfusepath{clip}%
\pgfsetbuttcap%
\pgfsetroundjoin%
\definecolor{currentfill}{rgb}{0.201239,0.383670,0.554294}%
\pgfsetfillcolor{currentfill}%
\pgfsetlinewidth{0.000000pt}%
\definecolor{currentstroke}{rgb}{0.000000,0.000000,0.000000}%
\pgfsetstrokecolor{currentstroke}%
\pgfsetdash{}{0pt}%
\pgfpathmoveto{\pgfqpoint{1.268612in}{1.433542in}}%
\pgfpathlineto{\pgfqpoint{1.269104in}{1.447829in}}%
\pgfpathlineto{\pgfqpoint{1.269598in}{1.462570in}}%
\pgfpathlineto{\pgfqpoint{1.270094in}{1.477771in}}%
\pgfpathlineto{\pgfqpoint{1.270593in}{1.493441in}}%
\pgfpathlineto{\pgfqpoint{1.292797in}{1.491908in}}%
\pgfpathlineto{\pgfqpoint{1.314906in}{1.490040in}}%
\pgfpathlineto{\pgfqpoint{1.336900in}{1.487838in}}%
\pgfpathlineto{\pgfqpoint{1.358762in}{1.485305in}}%
\pgfpathlineto{\pgfqpoint{1.357777in}{1.469661in}}%
\pgfpathlineto{\pgfqpoint{1.356796in}{1.454485in}}%
\pgfpathlineto{\pgfqpoint{1.355820in}{1.439771in}}%
\pgfpathlineto{\pgfqpoint{1.354849in}{1.425512in}}%
\pgfpathlineto{\pgfqpoint{1.333467in}{1.428012in}}%
\pgfpathlineto{\pgfqpoint{1.311954in}{1.430185in}}%
\pgfpathlineto{\pgfqpoint{1.290330in}{1.432029in}}%
\pgfpathlineto{\pgfqpoint{1.268612in}{1.433542in}}%
\pgfpathclose%
\pgfusepath{fill}%
\end{pgfscope}%
\begin{pgfscope}%
\pgfpathrectangle{\pgfqpoint{0.041670in}{0.041670in}}{\pgfqpoint{2.216660in}{2.216660in}}%
\pgfusepath{clip}%
\pgfsetbuttcap%
\pgfsetroundjoin%
\definecolor{currentfill}{rgb}{0.276194,0.190074,0.493001}%
\pgfsetfillcolor{currentfill}%
\pgfsetlinewidth{0.000000pt}%
\definecolor{currentstroke}{rgb}{0.000000,0.000000,0.000000}%
\pgfsetstrokecolor{currentstroke}%
\pgfsetdash{}{0pt}%
\pgfpathmoveto{\pgfqpoint{1.402551in}{1.278305in}}%
\pgfpathlineto{\pgfqpoint{1.403815in}{1.287606in}}%
\pgfpathlineto{\pgfqpoint{1.405082in}{1.297283in}}%
\pgfpathlineto{\pgfqpoint{1.406355in}{1.307340in}}%
\pgfpathlineto{\pgfqpoint{1.407632in}{1.317785in}}%
\pgfpathlineto{\pgfqpoint{1.427400in}{1.314108in}}%
\pgfpathlineto{\pgfqpoint{1.446954in}{1.310125in}}%
\pgfpathlineto{\pgfqpoint{1.466278in}{1.305838in}}%
\pgfpathlineto{\pgfqpoint{1.485354in}{1.301250in}}%
\pgfpathlineto{\pgfqpoint{1.483637in}{1.290873in}}%
\pgfpathlineto{\pgfqpoint{1.481926in}{1.280884in}}%
\pgfpathlineto{\pgfqpoint{1.480221in}{1.271277in}}%
\pgfpathlineto{\pgfqpoint{1.478523in}{1.262046in}}%
\pgfpathlineto{\pgfqpoint{1.459878in}{1.266557in}}%
\pgfpathlineto{\pgfqpoint{1.440989in}{1.270772in}}%
\pgfpathlineto{\pgfqpoint{1.421875in}{1.274689in}}%
\pgfpathlineto{\pgfqpoint{1.402551in}{1.278305in}}%
\pgfpathclose%
\pgfusepath{fill}%
\end{pgfscope}%
\begin{pgfscope}%
\pgfpathrectangle{\pgfqpoint{0.041670in}{0.041670in}}{\pgfqpoint{2.216660in}{2.216660in}}%
\pgfusepath{clip}%
\pgfsetbuttcap%
\pgfsetroundjoin%
\definecolor{currentfill}{rgb}{0.282884,0.135920,0.453427}%
\pgfsetfillcolor{currentfill}%
\pgfsetlinewidth{0.000000pt}%
\definecolor{currentstroke}{rgb}{0.000000,0.000000,0.000000}%
\pgfsetstrokecolor{currentstroke}%
\pgfsetdash{}{0pt}%
\pgfpathmoveto{\pgfqpoint{0.872145in}{1.224579in}}%
\pgfpathlineto{\pgfqpoint{0.870375in}{1.232348in}}%
\pgfpathlineto{\pgfqpoint{0.868600in}{1.240469in}}%
\pgfpathlineto{\pgfqpoint{0.866819in}{1.248948in}}%
\pgfpathlineto{\pgfqpoint{0.865032in}{1.257791in}}%
\pgfpathlineto{\pgfqpoint{0.883446in}{1.262561in}}%
\pgfpathlineto{\pgfqpoint{0.902119in}{1.267040in}}%
\pgfpathlineto{\pgfqpoint{0.921033in}{1.271222in}}%
\pgfpathlineto{\pgfqpoint{0.940172in}{1.275106in}}%
\pgfpathlineto{\pgfqpoint{0.941529in}{1.266188in}}%
\pgfpathlineto{\pgfqpoint{0.942882in}{1.257632in}}%
\pgfpathlineto{\pgfqpoint{0.944230in}{1.249434in}}%
\pgfpathlineto{\pgfqpoint{0.945575in}{1.241587in}}%
\pgfpathlineto{\pgfqpoint{0.926871in}{1.237772in}}%
\pgfpathlineto{\pgfqpoint{0.908387in}{1.233664in}}%
\pgfpathlineto{\pgfqpoint{0.890139in}{1.229265in}}%
\pgfpathlineto{\pgfqpoint{0.872145in}{1.224579in}}%
\pgfpathclose%
\pgfusepath{fill}%
\end{pgfscope}%
\begin{pgfscope}%
\pgfpathrectangle{\pgfqpoint{0.041670in}{0.041670in}}{\pgfqpoint{2.216660in}{2.216660in}}%
\pgfusepath{clip}%
\pgfsetbuttcap%
\pgfsetroundjoin%
\definecolor{currentfill}{rgb}{0.233603,0.313828,0.543914}%
\pgfsetfillcolor{currentfill}%
\pgfsetlinewidth{0.000000pt}%
\definecolor{currentstroke}{rgb}{0.000000,0.000000,0.000000}%
\pgfsetstrokecolor{currentstroke}%
\pgfsetdash{}{0pt}%
\pgfpathmoveto{\pgfqpoint{0.929139in}{1.360261in}}%
\pgfpathlineto{\pgfqpoint{0.927736in}{1.372727in}}%
\pgfpathlineto{\pgfqpoint{0.926327in}{1.385620in}}%
\pgfpathlineto{\pgfqpoint{0.924912in}{1.398947in}}%
\pgfpathlineto{\pgfqpoint{0.923491in}{1.412715in}}%
\pgfpathlineto{\pgfqpoint{0.944192in}{1.416469in}}%
\pgfpathlineto{\pgfqpoint{0.965094in}{1.419903in}}%
\pgfpathlineto{\pgfqpoint{0.986179in}{1.423017in}}%
\pgfpathlineto{\pgfqpoint{1.007430in}{1.425806in}}%
\pgfpathlineto{\pgfqpoint{1.008384in}{1.411993in}}%
\pgfpathlineto{\pgfqpoint{1.009334in}{1.398619in}}%
\pgfpathlineto{\pgfqpoint{1.010280in}{1.385679in}}%
\pgfpathlineto{\pgfqpoint{1.011222in}{1.373165in}}%
\pgfpathlineto{\pgfqpoint{0.990440in}{1.370416in}}%
\pgfpathlineto{\pgfqpoint{0.969821in}{1.367347in}}%
\pgfpathlineto{\pgfqpoint{0.949381in}{1.363961in}}%
\pgfpathlineto{\pgfqpoint{0.929139in}{1.360261in}}%
\pgfpathclose%
\pgfusepath{fill}%
\end{pgfscope}%
\begin{pgfscope}%
\pgfpathrectangle{\pgfqpoint{0.041670in}{0.041670in}}{\pgfqpoint{2.216660in}{2.216660in}}%
\pgfusepath{clip}%
\pgfsetbuttcap%
\pgfsetroundjoin%
\definecolor{currentfill}{rgb}{0.272594,0.025563,0.353093}%
\pgfsetfillcolor{currentfill}%
\pgfsetlinewidth{0.000000pt}%
\definecolor{currentstroke}{rgb}{0.000000,0.000000,0.000000}%
\pgfsetstrokecolor{currentstroke}%
\pgfsetdash{}{0pt}%
\pgfpathmoveto{\pgfqpoint{0.893016in}{1.156784in}}%
\pgfpathlineto{\pgfqpoint{0.891299in}{1.160759in}}%
\pgfpathlineto{\pgfqpoint{0.889579in}{1.165021in}}%
\pgfpathlineto{\pgfqpoint{0.887855in}{1.169575in}}%
\pgfpathlineto{\pgfqpoint{0.886127in}{1.174426in}}%
\pgfpathlineto{\pgfqpoint{0.903296in}{1.178932in}}%
\pgfpathlineto{\pgfqpoint{0.920707in}{1.183162in}}%
\pgfpathlineto{\pgfqpoint{0.938346in}{1.187112in}}%
\pgfpathlineto{\pgfqpoint{0.956196in}{1.190781in}}%
\pgfpathlineto{\pgfqpoint{0.957509in}{1.185845in}}%
\pgfpathlineto{\pgfqpoint{0.958819in}{1.181206in}}%
\pgfpathlineto{\pgfqpoint{0.960126in}{1.176858in}}%
\pgfpathlineto{\pgfqpoint{0.961430in}{1.172797in}}%
\pgfpathlineto{\pgfqpoint{0.944000in}{1.169205in}}%
\pgfpathlineto{\pgfqpoint{0.926777in}{1.165337in}}%
\pgfpathlineto{\pgfqpoint{0.909777in}{1.161195in}}%
\pgfpathlineto{\pgfqpoint{0.893016in}{1.156784in}}%
\pgfpathclose%
\pgfusepath{fill}%
\end{pgfscope}%
\begin{pgfscope}%
\pgfpathrectangle{\pgfqpoint{0.041670in}{0.041670in}}{\pgfqpoint{2.216660in}{2.216660in}}%
\pgfusepath{clip}%
\pgfsetbuttcap%
\pgfsetroundjoin%
\definecolor{currentfill}{rgb}{0.268510,0.009605,0.335427}%
\pgfsetfillcolor{currentfill}%
\pgfsetlinewidth{0.000000pt}%
\definecolor{currentstroke}{rgb}{0.000000,0.000000,0.000000}%
\pgfsetstrokecolor{currentstroke}%
\pgfsetdash{}{0pt}%
\pgfpathmoveto{\pgfqpoint{1.378007in}{1.162215in}}%
\pgfpathlineto{\pgfqpoint{1.379209in}{1.165195in}}%
\pgfpathlineto{\pgfqpoint{1.380412in}{1.168442in}}%
\pgfpathlineto{\pgfqpoint{1.381617in}{1.171961in}}%
\pgfpathlineto{\pgfqpoint{1.382825in}{1.175756in}}%
\pgfpathlineto{\pgfqpoint{1.400426in}{1.172412in}}%
\pgfpathlineto{\pgfqpoint{1.417834in}{1.168789in}}%
\pgfpathlineto{\pgfqpoint{1.435033in}{1.164890in}}%
\pgfpathlineto{\pgfqpoint{1.452007in}{1.160718in}}%
\pgfpathlineto{\pgfqpoint{1.450383in}{1.157004in}}%
\pgfpathlineto{\pgfqpoint{1.448763in}{1.153568in}}%
\pgfpathlineto{\pgfqpoint{1.447146in}{1.150403in}}%
\pgfpathlineto{\pgfqpoint{1.445532in}{1.147506in}}%
\pgfpathlineto{\pgfqpoint{1.428966in}{1.151586in}}%
\pgfpathlineto{\pgfqpoint{1.412179in}{1.155399in}}%
\pgfpathlineto{\pgfqpoint{1.395188in}{1.158943in}}%
\pgfpathlineto{\pgfqpoint{1.378007in}{1.162215in}}%
\pgfpathclose%
\pgfusepath{fill}%
\end{pgfscope}%
\begin{pgfscope}%
\pgfpathrectangle{\pgfqpoint{0.041670in}{0.041670in}}{\pgfqpoint{2.216660in}{2.216660in}}%
\pgfusepath{clip}%
\pgfsetbuttcap%
\pgfsetroundjoin%
\definecolor{currentfill}{rgb}{0.267004,0.004874,0.329415}%
\pgfsetfillcolor{currentfill}%
\pgfsetlinewidth{0.000000pt}%
\definecolor{currentstroke}{rgb}{0.000000,0.000000,0.000000}%
\pgfsetstrokecolor{currentstroke}%
\pgfsetdash{}{0pt}%
\pgfpathmoveto{\pgfqpoint{1.318479in}{1.155197in}}%
\pgfpathlineto{\pgfqpoint{1.319351in}{1.156258in}}%
\pgfpathlineto{\pgfqpoint{1.320223in}{1.157549in}}%
\pgfpathlineto{\pgfqpoint{1.321097in}{1.159075in}}%
\pgfpathlineto{\pgfqpoint{1.321971in}{1.160839in}}%
\pgfpathlineto{\pgfqpoint{1.339200in}{1.158454in}}%
\pgfpathlineto{\pgfqpoint{1.356289in}{1.155796in}}%
\pgfpathlineto{\pgfqpoint{1.373223in}{1.152867in}}%
\pgfpathlineto{\pgfqpoint{1.372031in}{1.151150in}}%
\pgfpathlineto{\pgfqpoint{1.370841in}{1.149673in}}%
\pgfpathlineto{\pgfqpoint{1.369652in}{1.148429in}}%
\pgfpathlineto{\pgfqpoint{1.368465in}{1.147415in}}%
\pgfpathlineto{\pgfqpoint{1.351950in}{1.150274in}}%
\pgfpathlineto{\pgfqpoint{1.335283in}{1.152869in}}%
\pgfpathlineto{\pgfqpoint{1.318479in}{1.155197in}}%
\pgfpathclose%
\pgfusepath{fill}%
\end{pgfscope}%
\begin{pgfscope}%
\pgfpathrectangle{\pgfqpoint{0.041670in}{0.041670in}}{\pgfqpoint{2.216660in}{2.216660in}}%
\pgfusepath{clip}%
\pgfsetbuttcap%
\pgfsetroundjoin%
\definecolor{currentfill}{rgb}{0.268510,0.009605,0.335427}%
\pgfsetfillcolor{currentfill}%
\pgfsetlinewidth{0.000000pt}%
\definecolor{currentstroke}{rgb}{0.000000,0.000000,0.000000}%
\pgfsetstrokecolor{currentstroke}%
\pgfsetdash{}{0pt}%
\pgfpathmoveto{\pgfqpoint{1.046734in}{1.153396in}}%
\pgfpathlineto{\pgfqpoint{1.045878in}{1.153584in}}%
\pgfpathlineto{\pgfqpoint{1.045021in}{1.153985in}}%
\pgfpathlineto{\pgfqpoint{1.044164in}{1.154601in}}%
\pgfpathlineto{\pgfqpoint{1.043305in}{1.155439in}}%
\pgfpathlineto{\pgfqpoint{1.060244in}{1.157470in}}%
\pgfpathlineto{\pgfqpoint{1.077288in}{1.159231in}}%
\pgfpathlineto{\pgfqpoint{1.094423in}{1.160721in}}%
\pgfpathlineto{\pgfqpoint{1.111632in}{1.161938in}}%
\pgfpathlineto{\pgfqpoint{1.112061in}{1.161061in}}%
\pgfpathlineto{\pgfqpoint{1.112489in}{1.160404in}}%
\pgfpathlineto{\pgfqpoint{1.112917in}{1.159963in}}%
\pgfpathlineto{\pgfqpoint{1.113345in}{1.159735in}}%
\pgfpathlineto{\pgfqpoint{1.096567in}{1.158547in}}%
\pgfpathlineto{\pgfqpoint{1.079863in}{1.157094in}}%
\pgfpathlineto{\pgfqpoint{1.063247in}{1.155377in}}%
\pgfpathlineto{\pgfqpoint{1.046734in}{1.153396in}}%
\pgfpathclose%
\pgfusepath{fill}%
\end{pgfscope}%
\begin{pgfscope}%
\pgfpathrectangle{\pgfqpoint{0.041670in}{0.041670in}}{\pgfqpoint{2.216660in}{2.216660in}}%
\pgfusepath{clip}%
\pgfsetbuttcap%
\pgfsetroundjoin%
\definecolor{currentfill}{rgb}{0.268510,0.009605,0.335427}%
\pgfsetfillcolor{currentfill}%
\pgfsetlinewidth{0.000000pt}%
\definecolor{currentstroke}{rgb}{0.000000,0.000000,0.000000}%
\pgfsetstrokecolor{currentstroke}%
\pgfsetdash{}{0pt}%
\pgfpathmoveto{\pgfqpoint{1.248432in}{1.159616in}}%
\pgfpathlineto{\pgfqpoint{1.248872in}{1.159844in}}%
\pgfpathlineto{\pgfqpoint{1.249312in}{1.160283in}}%
\pgfpathlineto{\pgfqpoint{1.249752in}{1.160940in}}%
\pgfpathlineto{\pgfqpoint{1.250193in}{1.161817in}}%
\pgfpathlineto{\pgfqpoint{1.267395in}{1.160568in}}%
\pgfpathlineto{\pgfqpoint{1.284520in}{1.159048in}}%
\pgfpathlineto{\pgfqpoint{1.301553in}{1.157257in}}%
\pgfpathlineto{\pgfqpoint{1.318479in}{1.155197in}}%
\pgfpathlineto{\pgfqpoint{1.317609in}{1.154361in}}%
\pgfpathlineto{\pgfqpoint{1.316740in}{1.153745in}}%
\pgfpathlineto{\pgfqpoint{1.315871in}{1.153347in}}%
\pgfpathlineto{\pgfqpoint{1.315003in}{1.153160in}}%
\pgfpathlineto{\pgfqpoint{1.298503in}{1.155170in}}%
\pgfpathlineto{\pgfqpoint{1.281898in}{1.156916in}}%
\pgfpathlineto{\pgfqpoint{1.265203in}{1.158399in}}%
\pgfpathlineto{\pgfqpoint{1.248432in}{1.159616in}}%
\pgfpathclose%
\pgfusepath{fill}%
\end{pgfscope}%
\begin{pgfscope}%
\pgfpathrectangle{\pgfqpoint{0.041670in}{0.041670in}}{\pgfqpoint{2.216660in}{2.216660in}}%
\pgfusepath{clip}%
\pgfsetbuttcap%
\pgfsetroundjoin%
\definecolor{currentfill}{rgb}{0.267004,0.004874,0.329415}%
\pgfsetfillcolor{currentfill}%
\pgfsetlinewidth{0.000000pt}%
\definecolor{currentstroke}{rgb}{0.000000,0.000000,0.000000}%
\pgfsetstrokecolor{currentstroke}%
\pgfsetdash{}{0pt}%
\pgfpathmoveto{\pgfqpoint{0.976904in}{1.144654in}}%
\pgfpathlineto{\pgfqpoint{0.975625in}{1.145651in}}%
\pgfpathlineto{\pgfqpoint{0.974344in}{1.146878in}}%
\pgfpathlineto{\pgfqpoint{0.973062in}{1.148339in}}%
\pgfpathlineto{\pgfqpoint{0.971778in}{1.150039in}}%
\pgfpathlineto{\pgfqpoint{0.988560in}{1.153206in}}%
\pgfpathlineto{\pgfqpoint{1.005512in}{1.156104in}}%
\pgfpathlineto{\pgfqpoint{1.022617in}{1.158732in}}%
\pgfpathlineto{\pgfqpoint{1.039861in}{1.161087in}}%
\pgfpathlineto{\pgfqpoint{1.040724in}{1.159321in}}%
\pgfpathlineto{\pgfqpoint{1.041585in}{1.157795in}}%
\pgfpathlineto{\pgfqpoint{1.042446in}{1.156502in}}%
\pgfpathlineto{\pgfqpoint{1.043305in}{1.155439in}}%
\pgfpathlineto{\pgfqpoint{1.026487in}{1.153140in}}%
\pgfpathlineto{\pgfqpoint{1.009804in}{1.150575in}}%
\pgfpathlineto{\pgfqpoint{0.993272in}{1.147746in}}%
\pgfpathlineto{\pgfqpoint{0.976904in}{1.144654in}}%
\pgfpathclose%
\pgfusepath{fill}%
\end{pgfscope}%
\begin{pgfscope}%
\pgfpathrectangle{\pgfqpoint{0.041670in}{0.041670in}}{\pgfqpoint{2.216660in}{2.216660in}}%
\pgfusepath{clip}%
\pgfsetbuttcap%
\pgfsetroundjoin%
\definecolor{currentfill}{rgb}{0.172719,0.448791,0.557885}%
\pgfsetfillcolor{currentfill}%
\pgfsetlinewidth{0.000000pt}%
\definecolor{currentstroke}{rgb}{0.000000,0.000000,0.000000}%
\pgfsetstrokecolor{currentstroke}%
\pgfsetdash{}{0pt}%
\pgfpathmoveto{\pgfqpoint{1.091789in}{1.493591in}}%
\pgfpathlineto{\pgfqpoint{1.091301in}{1.509738in}}%
\pgfpathlineto{\pgfqpoint{1.090812in}{1.526369in}}%
\pgfpathlineto{\pgfqpoint{1.090319in}{1.543491in}}%
\pgfpathlineto{\pgfqpoint{1.112979in}{1.544662in}}%
\pgfpathlineto{\pgfqpoint{1.135695in}{1.545494in}}%
\pgfpathlineto{\pgfqpoint{1.158449in}{1.545986in}}%
\pgfpathlineto{\pgfqpoint{1.181220in}{1.546137in}}%
\pgfpathlineto{\pgfqpoint{1.181213in}{1.529007in}}%
\pgfpathlineto{\pgfqpoint{1.181206in}{1.512368in}}%
\pgfpathlineto{\pgfqpoint{1.181199in}{1.496214in}}%
\pgfpathlineto{\pgfqpoint{1.158801in}{1.496064in}}%
\pgfpathlineto{\pgfqpoint{1.136421in}{1.495576in}}%
\pgfpathlineto{\pgfqpoint{1.114077in}{1.494752in}}%
\pgfpathlineto{\pgfqpoint{1.091789in}{1.493591in}}%
\pgfpathclose%
\pgfusepath{fill}%
\end{pgfscope}%
\begin{pgfscope}%
\pgfpathrectangle{\pgfqpoint{0.041670in}{0.041670in}}{\pgfqpoint{2.216660in}{2.216660in}}%
\pgfusepath{clip}%
\pgfsetbuttcap%
\pgfsetroundjoin%
\definecolor{currentfill}{rgb}{0.172719,0.448791,0.557885}%
\pgfsetfillcolor{currentfill}%
\pgfsetlinewidth{0.000000pt}%
\definecolor{currentstroke}{rgb}{0.000000,0.000000,0.000000}%
\pgfsetstrokecolor{currentstroke}%
\pgfsetdash{}{0pt}%
\pgfpathmoveto{\pgfqpoint{1.181199in}{1.496214in}}%
\pgfpathlineto{\pgfqpoint{1.181206in}{1.512368in}}%
\pgfpathlineto{\pgfqpoint{1.181213in}{1.529007in}}%
\pgfpathlineto{\pgfqpoint{1.181220in}{1.546137in}}%
\pgfpathlineto{\pgfqpoint{1.203991in}{1.545948in}}%
\pgfpathlineto{\pgfqpoint{1.226741in}{1.545418in}}%
\pgfpathlineto{\pgfqpoint{1.249452in}{1.544549in}}%
\pgfpathlineto{\pgfqpoint{1.272104in}{1.543340in}}%
\pgfpathlineto{\pgfqpoint{1.271598in}{1.526218in}}%
\pgfpathlineto{\pgfqpoint{1.271094in}{1.509588in}}%
\pgfpathlineto{\pgfqpoint{1.270593in}{1.493441in}}%
\pgfpathlineto{\pgfqpoint{1.248312in}{1.494640in}}%
\pgfpathlineto{\pgfqpoint{1.225974in}{1.495501in}}%
\pgfpathlineto{\pgfqpoint{1.203597in}{1.496026in}}%
\pgfpathlineto{\pgfqpoint{1.181199in}{1.496214in}}%
\pgfpathclose%
\pgfusepath{fill}%
\end{pgfscope}%
\begin{pgfscope}%
\pgfpathrectangle{\pgfqpoint{0.041670in}{0.041670in}}{\pgfqpoint{2.216660in}{2.216660in}}%
\pgfusepath{clip}%
\pgfsetbuttcap%
\pgfsetroundjoin%
\definecolor{currentfill}{rgb}{0.276194,0.190074,0.493001}%
\pgfsetfillcolor{currentfill}%
\pgfsetlinewidth{0.000000pt}%
\definecolor{currentstroke}{rgb}{0.000000,0.000000,0.000000}%
\pgfsetstrokecolor{currentstroke}%
\pgfsetdash{}{0pt}%
\pgfpathmoveto{\pgfqpoint{0.865032in}{1.257791in}}%
\pgfpathlineto{\pgfqpoint{0.863239in}{1.267004in}}%
\pgfpathlineto{\pgfqpoint{0.861439in}{1.276592in}}%
\pgfpathlineto{\pgfqpoint{0.859634in}{1.286563in}}%
\pgfpathlineto{\pgfqpoint{0.857822in}{1.296923in}}%
\pgfpathlineto{\pgfqpoint{0.876663in}{1.301774in}}%
\pgfpathlineto{\pgfqpoint{0.895767in}{1.306329in}}%
\pgfpathlineto{\pgfqpoint{0.915117in}{1.310582in}}%
\pgfpathlineto{\pgfqpoint{0.934696in}{1.314532in}}%
\pgfpathlineto{\pgfqpoint{0.936072in}{1.304100in}}%
\pgfpathlineto{\pgfqpoint{0.937444in}{1.294056in}}%
\pgfpathlineto{\pgfqpoint{0.938810in}{1.284394in}}%
\pgfpathlineto{\pgfqpoint{0.940172in}{1.275106in}}%
\pgfpathlineto{\pgfqpoint{0.921033in}{1.271222in}}%
\pgfpathlineto{\pgfqpoint{0.902119in}{1.267040in}}%
\pgfpathlineto{\pgfqpoint{0.883446in}{1.262561in}}%
\pgfpathlineto{\pgfqpoint{0.865032in}{1.257791in}}%
\pgfpathclose%
\pgfusepath{fill}%
\end{pgfscope}%
\begin{pgfscope}%
\pgfpathrectangle{\pgfqpoint{0.041670in}{0.041670in}}{\pgfqpoint{2.216660in}{2.216660in}}%
\pgfusepath{clip}%
\pgfsetbuttcap%
\pgfsetroundjoin%
\definecolor{currentfill}{rgb}{0.268510,0.009605,0.335427}%
\pgfsetfillcolor{currentfill}%
\pgfsetlinewidth{0.000000pt}%
\definecolor{currentstroke}{rgb}{0.000000,0.000000,0.000000}%
\pgfsetstrokecolor{currentstroke}%
\pgfsetdash{}{0pt}%
\pgfpathmoveto{\pgfqpoint{0.899848in}{1.143658in}}%
\pgfpathlineto{\pgfqpoint{0.898145in}{1.146533in}}%
\pgfpathlineto{\pgfqpoint{0.896438in}{1.149676in}}%
\pgfpathlineto{\pgfqpoint{0.894729in}{1.153091in}}%
\pgfpathlineto{\pgfqpoint{0.893016in}{1.156784in}}%
\pgfpathlineto{\pgfqpoint{0.909777in}{1.161195in}}%
\pgfpathlineto{\pgfqpoint{0.926777in}{1.165337in}}%
\pgfpathlineto{\pgfqpoint{0.944000in}{1.169205in}}%
\pgfpathlineto{\pgfqpoint{0.961430in}{1.172797in}}%
\pgfpathlineto{\pgfqpoint{0.962731in}{1.169018in}}%
\pgfpathlineto{\pgfqpoint{0.964031in}{1.165515in}}%
\pgfpathlineto{\pgfqpoint{0.965327in}{1.162284in}}%
\pgfpathlineto{\pgfqpoint{0.966622in}{1.159320in}}%
\pgfpathlineto{\pgfqpoint{0.949609in}{1.155806in}}%
\pgfpathlineto{\pgfqpoint{0.932799in}{1.152023in}}%
\pgfpathlineto{\pgfqpoint{0.916207in}{1.147972in}}%
\pgfpathlineto{\pgfqpoint{0.899848in}{1.143658in}}%
\pgfpathclose%
\pgfusepath{fill}%
\end{pgfscope}%
\begin{pgfscope}%
\pgfpathrectangle{\pgfqpoint{0.041670in}{0.041670in}}{\pgfqpoint{2.216660in}{2.216660in}}%
\pgfusepath{clip}%
\pgfsetbuttcap%
\pgfsetroundjoin%
\definecolor{currentfill}{rgb}{0.260571,0.246922,0.522828}%
\pgfsetfillcolor{currentfill}%
\pgfsetlinewidth{0.000000pt}%
\definecolor{currentstroke}{rgb}{0.000000,0.000000,0.000000}%
\pgfsetstrokecolor{currentstroke}%
\pgfsetdash{}{0pt}%
\pgfpathmoveto{\pgfqpoint{1.407632in}{1.317785in}}%
\pgfpathlineto{\pgfqpoint{1.408913in}{1.328624in}}%
\pgfpathlineto{\pgfqpoint{1.410200in}{1.339862in}}%
\pgfpathlineto{\pgfqpoint{1.411491in}{1.351507in}}%
\pgfpathlineto{\pgfqpoint{1.412788in}{1.363566in}}%
\pgfpathlineto{\pgfqpoint{1.433007in}{1.359831in}}%
\pgfpathlineto{\pgfqpoint{1.453008in}{1.355785in}}%
\pgfpathlineto{\pgfqpoint{1.472774in}{1.351430in}}%
\pgfpathlineto{\pgfqpoint{1.492287in}{1.346770in}}%
\pgfpathlineto{\pgfqpoint{1.490544in}{1.334775in}}%
\pgfpathlineto{\pgfqpoint{1.488807in}{1.323195in}}%
\pgfpathlineto{\pgfqpoint{1.487077in}{1.312022in}}%
\pgfpathlineto{\pgfqpoint{1.485354in}{1.301250in}}%
\pgfpathlineto{\pgfqpoint{1.466278in}{1.305838in}}%
\pgfpathlineto{\pgfqpoint{1.446954in}{1.310125in}}%
\pgfpathlineto{\pgfqpoint{1.427400in}{1.314108in}}%
\pgfpathlineto{\pgfqpoint{1.407632in}{1.317785in}}%
\pgfpathclose%
\pgfusepath{fill}%
\end{pgfscope}%
\begin{pgfscope}%
\pgfpathrectangle{\pgfqpoint{0.041670in}{0.041670in}}{\pgfqpoint{2.216660in}{2.216660in}}%
\pgfusepath{clip}%
\pgfsetbuttcap%
\pgfsetroundjoin%
\definecolor{currentfill}{rgb}{0.267004,0.004874,0.329415}%
\pgfsetfillcolor{currentfill}%
\pgfsetlinewidth{0.000000pt}%
\definecolor{currentstroke}{rgb}{0.000000,0.000000,0.000000}%
\pgfsetstrokecolor{currentstroke}%
\pgfsetdash{}{0pt}%
\pgfpathmoveto{\pgfqpoint{1.373223in}{1.152867in}}%
\pgfpathlineto{\pgfqpoint{1.374416in}{1.154827in}}%
\pgfpathlineto{\pgfqpoint{1.375611in}{1.157036in}}%
\pgfpathlineto{\pgfqpoint{1.376808in}{1.159497in}}%
\pgfpathlineto{\pgfqpoint{1.378007in}{1.162215in}}%
\pgfpathlineto{\pgfqpoint{1.395188in}{1.158943in}}%
\pgfpathlineto{\pgfqpoint{1.412179in}{1.155399in}}%
\pgfpathlineto{\pgfqpoint{1.428966in}{1.151586in}}%
\pgfpathlineto{\pgfqpoint{1.445532in}{1.147506in}}%
\pgfpathlineto{\pgfqpoint{1.443921in}{1.144871in}}%
\pgfpathlineto{\pgfqpoint{1.442313in}{1.142494in}}%
\pgfpathlineto{\pgfqpoint{1.440707in}{1.140370in}}%
\pgfpathlineto{\pgfqpoint{1.439103in}{1.138495in}}%
\pgfpathlineto{\pgfqpoint{1.422941in}{1.142482in}}%
\pgfpathlineto{\pgfqpoint{1.406564in}{1.146208in}}%
\pgfpathlineto{\pgfqpoint{1.389986in}{1.149670in}}%
\pgfpathlineto{\pgfqpoint{1.373223in}{1.152867in}}%
\pgfpathclose%
\pgfusepath{fill}%
\end{pgfscope}%
\begin{pgfscope}%
\pgfpathrectangle{\pgfqpoint{0.041670in}{0.041670in}}{\pgfqpoint{2.216660in}{2.216660in}}%
\pgfusepath{clip}%
\pgfsetbuttcap%
\pgfsetroundjoin%
\definecolor{currentfill}{rgb}{0.271305,0.019942,0.347269}%
\pgfsetfillcolor{currentfill}%
\pgfsetlinewidth{0.000000pt}%
\definecolor{currentstroke}{rgb}{0.000000,0.000000,0.000000}%
\pgfsetstrokecolor{currentstroke}%
\pgfsetdash{}{0pt}%
\pgfpathmoveto{\pgfqpoint{1.115052in}{1.160857in}}%
\pgfpathlineto{\pgfqpoint{1.114626in}{1.160279in}}%
\pgfpathlineto{\pgfqpoint{1.114199in}{1.159897in}}%
\pgfpathlineto{\pgfqpoint{1.113772in}{1.159714in}}%
\pgfpathlineto{\pgfqpoint{1.113345in}{1.159735in}}%
\pgfpathlineto{\pgfqpoint{1.130182in}{1.160656in}}%
\pgfpathlineto{\pgfqpoint{1.147062in}{1.161310in}}%
\pgfpathlineto{\pgfqpoint{1.163972in}{1.161697in}}%
\pgfpathlineto{\pgfqpoint{1.180895in}{1.161816in}}%
\pgfpathlineto{\pgfqpoint{1.180889in}{1.161782in}}%
\pgfpathlineto{\pgfqpoint{1.180883in}{1.161952in}}%
\pgfpathlineto{\pgfqpoint{1.180877in}{1.162321in}}%
\pgfpathlineto{\pgfqpoint{1.180871in}{1.162885in}}%
\pgfpathlineto{\pgfqpoint{1.164381in}{1.162769in}}%
\pgfpathlineto{\pgfqpoint{1.147905in}{1.162392in}}%
\pgfpathlineto{\pgfqpoint{1.131457in}{1.161755in}}%
\pgfpathlineto{\pgfqpoint{1.115052in}{1.160857in}}%
\pgfpathclose%
\pgfusepath{fill}%
\end{pgfscope}%
\begin{pgfscope}%
\pgfpathrectangle{\pgfqpoint{0.041670in}{0.041670in}}{\pgfqpoint{2.216660in}{2.216660in}}%
\pgfusepath{clip}%
\pgfsetbuttcap%
\pgfsetroundjoin%
\definecolor{currentfill}{rgb}{0.271305,0.019942,0.347269}%
\pgfsetfillcolor{currentfill}%
\pgfsetlinewidth{0.000000pt}%
\definecolor{currentstroke}{rgb}{0.000000,0.000000,0.000000}%
\pgfsetstrokecolor{currentstroke}%
\pgfsetdash{}{0pt}%
\pgfpathmoveto{\pgfqpoint{1.180871in}{1.162885in}}%
\pgfpathlineto{\pgfqpoint{1.180877in}{1.162321in}}%
\pgfpathlineto{\pgfqpoint{1.180883in}{1.161952in}}%
\pgfpathlineto{\pgfqpoint{1.180889in}{1.161782in}}%
\pgfpathlineto{\pgfqpoint{1.180895in}{1.161816in}}%
\pgfpathlineto{\pgfqpoint{1.197818in}{1.161667in}}%
\pgfpathlineto{\pgfqpoint{1.214725in}{1.161251in}}%
\pgfpathlineto{\pgfqpoint{1.231601in}{1.160567in}}%
\pgfpathlineto{\pgfqpoint{1.248432in}{1.159616in}}%
\pgfpathlineto{\pgfqpoint{1.247993in}{1.159596in}}%
\pgfpathlineto{\pgfqpoint{1.247554in}{1.159780in}}%
\pgfpathlineto{\pgfqpoint{1.247116in}{1.160163in}}%
\pgfpathlineto{\pgfqpoint{1.246677in}{1.160741in}}%
\pgfpathlineto{\pgfqpoint{1.230278in}{1.161668in}}%
\pgfpathlineto{\pgfqpoint{1.213834in}{1.162334in}}%
\pgfpathlineto{\pgfqpoint{1.197360in}{1.162740in}}%
\pgfpathlineto{\pgfqpoint{1.180871in}{1.162885in}}%
\pgfpathclose%
\pgfusepath{fill}%
\end{pgfscope}%
\begin{pgfscope}%
\pgfpathrectangle{\pgfqpoint{0.041670in}{0.041670in}}{\pgfqpoint{2.216660in}{2.216660in}}%
\pgfusepath{clip}%
\pgfsetbuttcap%
\pgfsetroundjoin%
\definecolor{currentfill}{rgb}{0.201239,0.383670,0.554294}%
\pgfsetfillcolor{currentfill}%
\pgfsetlinewidth{0.000000pt}%
\definecolor{currentstroke}{rgb}{0.000000,0.000000,0.000000}%
\pgfsetstrokecolor{currentstroke}%
\pgfsetdash{}{0pt}%
\pgfpathmoveto{\pgfqpoint{1.354849in}{1.425512in}}%
\pgfpathlineto{\pgfqpoint{1.355820in}{1.439771in}}%
\pgfpathlineto{\pgfqpoint{1.356796in}{1.454485in}}%
\pgfpathlineto{\pgfqpoint{1.357777in}{1.469661in}}%
\pgfpathlineto{\pgfqpoint{1.358762in}{1.485305in}}%
\pgfpathlineto{\pgfqpoint{1.380472in}{1.482443in}}%
\pgfpathlineto{\pgfqpoint{1.402012in}{1.479252in}}%
\pgfpathlineto{\pgfqpoint{1.423363in}{1.475735in}}%
\pgfpathlineto{\pgfqpoint{1.422020in}{1.460121in}}%
\pgfpathlineto{\pgfqpoint{1.420683in}{1.444977in}}%
\pgfpathlineto{\pgfqpoint{1.419353in}{1.430294in}}%
\pgfpathlineto{\pgfqpoint{1.418028in}{1.416067in}}%
\pgfpathlineto{\pgfqpoint{1.397148in}{1.419538in}}%
\pgfpathlineto{\pgfqpoint{1.376082in}{1.422687in}}%
\pgfpathlineto{\pgfqpoint{1.354849in}{1.425512in}}%
\pgfpathclose%
\pgfusepath{fill}%
\end{pgfscope}%
\begin{pgfscope}%
\pgfpathrectangle{\pgfqpoint{0.041670in}{0.041670in}}{\pgfqpoint{2.216660in}{2.216660in}}%
\pgfusepath{clip}%
\pgfsetbuttcap%
\pgfsetroundjoin%
\definecolor{currentfill}{rgb}{0.172719,0.448791,0.557885}%
\pgfsetfillcolor{currentfill}%
\pgfsetlinewidth{0.000000pt}%
\definecolor{currentstroke}{rgb}{0.000000,0.000000,0.000000}%
\pgfsetstrokecolor{currentstroke}%
\pgfsetdash{}{0pt}%
\pgfpathmoveto{\pgfqpoint{1.003569in}{1.485603in}}%
\pgfpathlineto{\pgfqpoint{1.002593in}{1.501725in}}%
\pgfpathlineto{\pgfqpoint{1.001612in}{1.518332in}}%
\pgfpathlineto{\pgfqpoint{1.000625in}{1.535431in}}%
\pgfpathlineto{\pgfqpoint{1.022869in}{1.537950in}}%
\pgfpathlineto{\pgfqpoint{1.045245in}{1.540134in}}%
\pgfpathlineto{\pgfqpoint{1.067735in}{1.541981in}}%
\pgfpathlineto{\pgfqpoint{1.090319in}{1.543491in}}%
\pgfpathlineto{\pgfqpoint{1.090812in}{1.526369in}}%
\pgfpathlineto{\pgfqpoint{1.091301in}{1.509738in}}%
\pgfpathlineto{\pgfqpoint{1.091789in}{1.493591in}}%
\pgfpathlineto{\pgfqpoint{1.069575in}{1.492095in}}%
\pgfpathlineto{\pgfqpoint{1.047455in}{1.490264in}}%
\pgfpathlineto{\pgfqpoint{1.025447in}{1.488099in}}%
\pgfpathlineto{\pgfqpoint{1.003569in}{1.485603in}}%
\pgfpathclose%
\pgfusepath{fill}%
\end{pgfscope}%
\begin{pgfscope}%
\pgfpathrectangle{\pgfqpoint{0.041670in}{0.041670in}}{\pgfqpoint{2.216660in}{2.216660in}}%
\pgfusepath{clip}%
\pgfsetbuttcap%
\pgfsetroundjoin%
\definecolor{currentfill}{rgb}{0.260571,0.246922,0.522828}%
\pgfsetfillcolor{currentfill}%
\pgfsetlinewidth{0.000000pt}%
\definecolor{currentstroke}{rgb}{0.000000,0.000000,0.000000}%
\pgfsetstrokecolor{currentstroke}%
\pgfsetdash{}{0pt}%
\pgfpathmoveto{\pgfqpoint{0.857822in}{1.296923in}}%
\pgfpathlineto{\pgfqpoint{0.856003in}{1.307677in}}%
\pgfpathlineto{\pgfqpoint{0.854177in}{1.318832in}}%
\pgfpathlineto{\pgfqpoint{0.852344in}{1.330396in}}%
\pgfpathlineto{\pgfqpoint{0.850503in}{1.342373in}}%
\pgfpathlineto{\pgfqpoint{0.869778in}{1.347302in}}%
\pgfpathlineto{\pgfqpoint{0.889320in}{1.351929in}}%
\pgfpathlineto{\pgfqpoint{0.909113in}{1.356250in}}%
\pgfpathlineto{\pgfqpoint{0.929139in}{1.360261in}}%
\pgfpathlineto{\pgfqpoint{0.930537in}{1.348215in}}%
\pgfpathlineto{\pgfqpoint{0.931929in}{1.336583in}}%
\pgfpathlineto{\pgfqpoint{0.933315in}{1.325357in}}%
\pgfpathlineto{\pgfqpoint{0.934696in}{1.314532in}}%
\pgfpathlineto{\pgfqpoint{0.915117in}{1.310582in}}%
\pgfpathlineto{\pgfqpoint{0.895767in}{1.306329in}}%
\pgfpathlineto{\pgfqpoint{0.876663in}{1.301774in}}%
\pgfpathlineto{\pgfqpoint{0.857822in}{1.296923in}}%
\pgfpathclose%
\pgfusepath{fill}%
\end{pgfscope}%
\begin{pgfscope}%
\pgfpathrectangle{\pgfqpoint{0.041670in}{0.041670in}}{\pgfqpoint{2.216660in}{2.216660in}}%
\pgfusepath{clip}%
\pgfsetbuttcap%
\pgfsetroundjoin%
\definecolor{currentfill}{rgb}{0.172719,0.448791,0.557885}%
\pgfsetfillcolor{currentfill}%
\pgfsetlinewidth{0.000000pt}%
\definecolor{currentstroke}{rgb}{0.000000,0.000000,0.000000}%
\pgfsetstrokecolor{currentstroke}%
\pgfsetdash{}{0pt}%
\pgfpathmoveto{\pgfqpoint{1.270593in}{1.493441in}}%
\pgfpathlineto{\pgfqpoint{1.271094in}{1.509588in}}%
\pgfpathlineto{\pgfqpoint{1.271598in}{1.526218in}}%
\pgfpathlineto{\pgfqpoint{1.272104in}{1.543340in}}%
\pgfpathlineto{\pgfqpoint{1.294679in}{1.541792in}}%
\pgfpathlineto{\pgfqpoint{1.317157in}{1.539907in}}%
\pgfpathlineto{\pgfqpoint{1.339519in}{1.537686in}}%
\pgfpathlineto{\pgfqpoint{1.361747in}{1.535130in}}%
\pgfpathlineto{\pgfqpoint{1.360747in}{1.518032in}}%
\pgfpathlineto{\pgfqpoint{1.359752in}{1.501427in}}%
\pgfpathlineto{\pgfqpoint{1.358762in}{1.485305in}}%
\pgfpathlineto{\pgfqpoint{1.336900in}{1.487838in}}%
\pgfpathlineto{\pgfqpoint{1.314906in}{1.490040in}}%
\pgfpathlineto{\pgfqpoint{1.292797in}{1.491908in}}%
\pgfpathlineto{\pgfqpoint{1.270593in}{1.493441in}}%
\pgfpathclose%
\pgfusepath{fill}%
\end{pgfscope}%
\begin{pgfscope}%
\pgfpathrectangle{\pgfqpoint{0.041670in}{0.041670in}}{\pgfqpoint{2.216660in}{2.216660in}}%
\pgfusepath{clip}%
\pgfsetbuttcap%
\pgfsetroundjoin%
\definecolor{currentfill}{rgb}{0.267004,0.004874,0.329415}%
\pgfsetfillcolor{currentfill}%
\pgfsetlinewidth{0.000000pt}%
\definecolor{currentstroke}{rgb}{0.000000,0.000000,0.000000}%
\pgfsetstrokecolor{currentstroke}%
\pgfsetdash{}{0pt}%
\pgfpathmoveto{\pgfqpoint{0.906633in}{1.134735in}}%
\pgfpathlineto{\pgfqpoint{0.904941in}{1.136588in}}%
\pgfpathlineto{\pgfqpoint{0.903246in}{1.138690in}}%
\pgfpathlineto{\pgfqpoint{0.901549in}{1.141045in}}%
\pgfpathlineto{\pgfqpoint{0.899848in}{1.143658in}}%
\pgfpathlineto{\pgfqpoint{0.916207in}{1.147972in}}%
\pgfpathlineto{\pgfqpoint{0.932799in}{1.152023in}}%
\pgfpathlineto{\pgfqpoint{0.949609in}{1.155806in}}%
\pgfpathlineto{\pgfqpoint{0.966622in}{1.159320in}}%
\pgfpathlineto{\pgfqpoint{0.967914in}{1.156618in}}%
\pgfpathlineto{\pgfqpoint{0.969204in}{1.154174in}}%
\pgfpathlineto{\pgfqpoint{0.970492in}{1.151982in}}%
\pgfpathlineto{\pgfqpoint{0.971778in}{1.150039in}}%
\pgfpathlineto{\pgfqpoint{0.955179in}{1.146605in}}%
\pgfpathlineto{\pgfqpoint{0.938778in}{1.142908in}}%
\pgfpathlineto{\pgfqpoint{0.922591in}{1.138951in}}%
\pgfpathlineto{\pgfqpoint{0.906633in}{1.134735in}}%
\pgfpathclose%
\pgfusepath{fill}%
\end{pgfscope}%
\begin{pgfscope}%
\pgfpathrectangle{\pgfqpoint{0.041670in}{0.041670in}}{\pgfqpoint{2.216660in}{2.216660in}}%
\pgfusepath{clip}%
\pgfsetbuttcap%
\pgfsetroundjoin%
\definecolor{currentfill}{rgb}{0.268510,0.009605,0.335427}%
\pgfsetfillcolor{currentfill}%
\pgfsetlinewidth{0.000000pt}%
\definecolor{currentstroke}{rgb}{0.000000,0.000000,0.000000}%
\pgfsetstrokecolor{currentstroke}%
\pgfsetdash{}{0pt}%
\pgfpathmoveto{\pgfqpoint{1.315003in}{1.153160in}}%
\pgfpathlineto{\pgfqpoint{1.315871in}{1.153347in}}%
\pgfpathlineto{\pgfqpoint{1.316740in}{1.153745in}}%
\pgfpathlineto{\pgfqpoint{1.317609in}{1.154361in}}%
\pgfpathlineto{\pgfqpoint{1.318479in}{1.155197in}}%
\pgfpathlineto{\pgfqpoint{1.335283in}{1.152869in}}%
\pgfpathlineto{\pgfqpoint{1.351950in}{1.150274in}}%
\pgfpathlineto{\pgfqpoint{1.368465in}{1.147415in}}%
\pgfpathlineto{\pgfqpoint{1.367279in}{1.146627in}}%
\pgfpathlineto{\pgfqpoint{1.366095in}{1.146060in}}%
\pgfpathlineto{\pgfqpoint{1.364912in}{1.145710in}}%
\pgfpathlineto{\pgfqpoint{1.363730in}{1.145572in}}%
\pgfpathlineto{\pgfqpoint{1.347631in}{1.148360in}}%
\pgfpathlineto{\pgfqpoint{1.331384in}{1.150890in}}%
\pgfpathlineto{\pgfqpoint{1.315003in}{1.153160in}}%
\pgfpathclose%
\pgfusepath{fill}%
\end{pgfscope}%
\begin{pgfscope}%
\pgfpathrectangle{\pgfqpoint{0.041670in}{0.041670in}}{\pgfqpoint{2.216660in}{2.216660in}}%
\pgfusepath{clip}%
\pgfsetbuttcap%
\pgfsetroundjoin%
\definecolor{currentfill}{rgb}{0.201239,0.383670,0.554294}%
\pgfsetfillcolor{currentfill}%
\pgfsetlinewidth{0.000000pt}%
\definecolor{currentstroke}{rgb}{0.000000,0.000000,0.000000}%
\pgfsetstrokecolor{currentstroke}%
\pgfsetdash{}{0pt}%
\pgfpathmoveto{\pgfqpoint{0.923491in}{1.412715in}}%
\pgfpathlineto{\pgfqpoint{0.922064in}{1.426930in}}%
\pgfpathlineto{\pgfqpoint{0.920630in}{1.441601in}}%
\pgfpathlineto{\pgfqpoint{0.919189in}{1.456735in}}%
\pgfpathlineto{\pgfqpoint{0.917741in}{1.472338in}}%
\pgfpathlineto{\pgfqpoint{0.938909in}{1.476142in}}%
\pgfpathlineto{\pgfqpoint{0.960282in}{1.479622in}}%
\pgfpathlineto{\pgfqpoint{0.981842in}{1.482777in}}%
\pgfpathlineto{\pgfqpoint{1.003569in}{1.485603in}}%
\pgfpathlineto{\pgfqpoint{1.004541in}{1.469958in}}%
\pgfpathlineto{\pgfqpoint{1.005508in}{1.454781in}}%
\pgfpathlineto{\pgfqpoint{1.006471in}{1.440066in}}%
\pgfpathlineto{\pgfqpoint{1.007430in}{1.425806in}}%
\pgfpathlineto{\pgfqpoint{0.986179in}{1.423017in}}%
\pgfpathlineto{\pgfqpoint{0.965094in}{1.419903in}}%
\pgfpathlineto{\pgfqpoint{0.944192in}{1.416469in}}%
\pgfpathlineto{\pgfqpoint{0.923491in}{1.412715in}}%
\pgfpathclose%
\pgfusepath{fill}%
\end{pgfscope}%
\begin{pgfscope}%
\pgfpathrectangle{\pgfqpoint{0.041670in}{0.041670in}}{\pgfqpoint{2.216660in}{2.216660in}}%
\pgfusepath{clip}%
\pgfsetbuttcap%
\pgfsetroundjoin%
\definecolor{currentfill}{rgb}{0.268510,0.009605,0.335427}%
\pgfsetfillcolor{currentfill}%
\pgfsetlinewidth{0.000000pt}%
\definecolor{currentstroke}{rgb}{0.000000,0.000000,0.000000}%
\pgfsetstrokecolor{currentstroke}%
\pgfsetdash{}{0pt}%
\pgfpathmoveto{\pgfqpoint{0.982007in}{1.142880in}}%
\pgfpathlineto{\pgfqpoint{0.980733in}{1.143000in}}%
\pgfpathlineto{\pgfqpoint{0.979458in}{1.143333in}}%
\pgfpathlineto{\pgfqpoint{0.978182in}{1.143883in}}%
\pgfpathlineto{\pgfqpoint{0.976904in}{1.144654in}}%
\pgfpathlineto{\pgfqpoint{0.993272in}{1.147746in}}%
\pgfpathlineto{\pgfqpoint{1.009804in}{1.150575in}}%
\pgfpathlineto{\pgfqpoint{1.026487in}{1.153140in}}%
\pgfpathlineto{\pgfqpoint{1.043305in}{1.155439in}}%
\pgfpathlineto{\pgfqpoint{1.044164in}{1.154601in}}%
\pgfpathlineto{\pgfqpoint{1.045021in}{1.153985in}}%
\pgfpathlineto{\pgfqpoint{1.045878in}{1.153584in}}%
\pgfpathlineto{\pgfqpoint{1.046734in}{1.153396in}}%
\pgfpathlineto{\pgfqpoint{1.030339in}{1.151155in}}%
\pgfpathlineto{\pgfqpoint{1.014076in}{1.148653in}}%
\pgfpathlineto{\pgfqpoint{0.997961in}{1.145894in}}%
\pgfpathlineto{\pgfqpoint{0.982007in}{1.142880in}}%
\pgfpathclose%
\pgfusepath{fill}%
\end{pgfscope}%
\begin{pgfscope}%
\pgfpathrectangle{\pgfqpoint{0.041670in}{0.041670in}}{\pgfqpoint{2.216660in}{2.216660in}}%
\pgfusepath{clip}%
\pgfsetbuttcap%
\pgfsetroundjoin%
\definecolor{currentfill}{rgb}{0.271305,0.019942,0.347269}%
\pgfsetfillcolor{currentfill}%
\pgfsetlinewidth{0.000000pt}%
\definecolor{currentstroke}{rgb}{0.000000,0.000000,0.000000}%
\pgfsetstrokecolor{currentstroke}%
\pgfsetdash{}{0pt}%
\pgfpathmoveto{\pgfqpoint{1.050151in}{1.154682in}}%
\pgfpathlineto{\pgfqpoint{1.049297in}{1.154064in}}%
\pgfpathlineto{\pgfqpoint{1.048444in}{1.153640in}}%
\pgfpathlineto{\pgfqpoint{1.047589in}{1.153416in}}%
\pgfpathlineto{\pgfqpoint{1.046734in}{1.153396in}}%
\pgfpathlineto{\pgfqpoint{1.063247in}{1.155377in}}%
\pgfpathlineto{\pgfqpoint{1.079863in}{1.157094in}}%
\pgfpathlineto{\pgfqpoint{1.096567in}{1.158547in}}%
\pgfpathlineto{\pgfqpoint{1.113345in}{1.159735in}}%
\pgfpathlineto{\pgfqpoint{1.113772in}{1.159714in}}%
\pgfpathlineto{\pgfqpoint{1.114199in}{1.159897in}}%
\pgfpathlineto{\pgfqpoint{1.114626in}{1.160279in}}%
\pgfpathlineto{\pgfqpoint{1.115052in}{1.160857in}}%
\pgfpathlineto{\pgfqpoint{1.098704in}{1.159700in}}%
\pgfpathlineto{\pgfqpoint{1.082428in}{1.158285in}}%
\pgfpathlineto{\pgfqpoint{1.066239in}{1.156612in}}%
\pgfpathlineto{\pgfqpoint{1.050151in}{1.154682in}}%
\pgfpathclose%
\pgfusepath{fill}%
\end{pgfscope}%
\begin{pgfscope}%
\pgfpathrectangle{\pgfqpoint{0.041670in}{0.041670in}}{\pgfqpoint{2.216660in}{2.216660in}}%
\pgfusepath{clip}%
\pgfsetbuttcap%
\pgfsetroundjoin%
\definecolor{currentfill}{rgb}{0.271305,0.019942,0.347269}%
\pgfsetfillcolor{currentfill}%
\pgfsetlinewidth{0.000000pt}%
\definecolor{currentstroke}{rgb}{0.000000,0.000000,0.000000}%
\pgfsetstrokecolor{currentstroke}%
\pgfsetdash{}{0pt}%
\pgfpathmoveto{\pgfqpoint{1.246677in}{1.160741in}}%
\pgfpathlineto{\pgfqpoint{1.247116in}{1.160163in}}%
\pgfpathlineto{\pgfqpoint{1.247554in}{1.159780in}}%
\pgfpathlineto{\pgfqpoint{1.247993in}{1.159596in}}%
\pgfpathlineto{\pgfqpoint{1.248432in}{1.159616in}}%
\pgfpathlineto{\pgfqpoint{1.265203in}{1.158399in}}%
\pgfpathlineto{\pgfqpoint{1.281898in}{1.156916in}}%
\pgfpathlineto{\pgfqpoint{1.298503in}{1.155170in}}%
\pgfpathlineto{\pgfqpoint{1.315003in}{1.153160in}}%
\pgfpathlineto{\pgfqpoint{1.314137in}{1.153182in}}%
\pgfpathlineto{\pgfqpoint{1.313270in}{1.153407in}}%
\pgfpathlineto{\pgfqpoint{1.312405in}{1.153832in}}%
\pgfpathlineto{\pgfqpoint{1.311540in}{1.154452in}}%
\pgfpathlineto{\pgfqpoint{1.295464in}{1.156410in}}%
\pgfpathlineto{\pgfqpoint{1.279285in}{1.158111in}}%
\pgfpathlineto{\pgfqpoint{1.263018in}{1.159556in}}%
\pgfpathlineto{\pgfqpoint{1.246677in}{1.160741in}}%
\pgfpathclose%
\pgfusepath{fill}%
\end{pgfscope}%
\begin{pgfscope}%
\pgfpathrectangle{\pgfqpoint{0.041670in}{0.041670in}}{\pgfqpoint{2.216660in}{2.216660in}}%
\pgfusepath{clip}%
\pgfsetbuttcap%
\pgfsetroundjoin%
\definecolor{currentfill}{rgb}{0.233603,0.313828,0.543914}%
\pgfsetfillcolor{currentfill}%
\pgfsetlinewidth{0.000000pt}%
\definecolor{currentstroke}{rgb}{0.000000,0.000000,0.000000}%
\pgfsetstrokecolor{currentstroke}%
\pgfsetdash{}{0pt}%
\pgfpathmoveto{\pgfqpoint{1.412788in}{1.363566in}}%
\pgfpathlineto{\pgfqpoint{1.414090in}{1.376044in}}%
\pgfpathlineto{\pgfqpoint{1.415397in}{1.388949in}}%
\pgfpathlineto{\pgfqpoint{1.416710in}{1.402288in}}%
\pgfpathlineto{\pgfqpoint{1.418028in}{1.416067in}}%
\pgfpathlineto{\pgfqpoint{1.438706in}{1.412278in}}%
\pgfpathlineto{\pgfqpoint{1.459161in}{1.408173in}}%
\pgfpathlineto{\pgfqpoint{1.479377in}{1.403754in}}%
\pgfpathlineto{\pgfqpoint{1.499335in}{1.399026in}}%
\pgfpathlineto{\pgfqpoint{1.497562in}{1.385306in}}%
\pgfpathlineto{\pgfqpoint{1.495796in}{1.372028in}}%
\pgfpathlineto{\pgfqpoint{1.494038in}{1.359185in}}%
\pgfpathlineto{\pgfqpoint{1.492287in}{1.346770in}}%
\pgfpathlineto{\pgfqpoint{1.472774in}{1.351430in}}%
\pgfpathlineto{\pgfqpoint{1.453008in}{1.355785in}}%
\pgfpathlineto{\pgfqpoint{1.433007in}{1.359831in}}%
\pgfpathlineto{\pgfqpoint{1.412788in}{1.363566in}}%
\pgfpathclose%
\pgfusepath{fill}%
\end{pgfscope}%
\begin{pgfscope}%
\pgfpathrectangle{\pgfqpoint{0.041670in}{0.041670in}}{\pgfqpoint{2.216660in}{2.216660in}}%
\pgfusepath{clip}%
\pgfsetbuttcap%
\pgfsetroundjoin%
\definecolor{currentfill}{rgb}{0.267004,0.004874,0.329415}%
\pgfsetfillcolor{currentfill}%
\pgfsetlinewidth{0.000000pt}%
\definecolor{currentstroke}{rgb}{0.000000,0.000000,0.000000}%
\pgfsetstrokecolor{currentstroke}%
\pgfsetdash{}{0pt}%
\pgfpathmoveto{\pgfqpoint{1.368465in}{1.147415in}}%
\pgfpathlineto{\pgfqpoint{1.369652in}{1.148429in}}%
\pgfpathlineto{\pgfqpoint{1.370841in}{1.149673in}}%
\pgfpathlineto{\pgfqpoint{1.372031in}{1.151150in}}%
\pgfpathlineto{\pgfqpoint{1.373223in}{1.152867in}}%
\pgfpathlineto{\pgfqpoint{1.389986in}{1.149670in}}%
\pgfpathlineto{\pgfqpoint{1.406564in}{1.146208in}}%
\pgfpathlineto{\pgfqpoint{1.422941in}{1.142482in}}%
\pgfpathlineto{\pgfqpoint{1.439103in}{1.138495in}}%
\pgfpathlineto{\pgfqpoint{1.437502in}{1.136864in}}%
\pgfpathlineto{\pgfqpoint{1.435903in}{1.135471in}}%
\pgfpathlineto{\pgfqpoint{1.434306in}{1.134314in}}%
\pgfpathlineto{\pgfqpoint{1.432711in}{1.133387in}}%
\pgfpathlineto{\pgfqpoint{1.416951in}{1.137278in}}%
\pgfpathlineto{\pgfqpoint{1.400980in}{1.140915in}}%
\pgfpathlineto{\pgfqpoint{1.384813in}{1.144295in}}%
\pgfpathlineto{\pgfqpoint{1.368465in}{1.147415in}}%
\pgfpathclose%
\pgfusepath{fill}%
\end{pgfscope}%
\begin{pgfscope}%
\pgfpathrectangle{\pgfqpoint{0.041670in}{0.041670in}}{\pgfqpoint{2.216660in}{2.216660in}}%
\pgfusepath{clip}%
\pgfsetbuttcap%
\pgfsetroundjoin%
\definecolor{currentfill}{rgb}{0.277941,0.056324,0.381191}%
\pgfsetfillcolor{currentfill}%
\pgfsetlinewidth{0.000000pt}%
\definecolor{currentstroke}{rgb}{0.000000,0.000000,0.000000}%
\pgfsetstrokecolor{currentstroke}%
\pgfsetdash{}{0pt}%
\pgfpathmoveto{\pgfqpoint{1.458534in}{1.178445in}}%
\pgfpathlineto{\pgfqpoint{1.460175in}{1.183620in}}%
\pgfpathlineto{\pgfqpoint{1.461820in}{1.189102in}}%
\pgfpathlineto{\pgfqpoint{1.463469in}{1.194897in}}%
\pgfpathlineto{\pgfqpoint{1.465123in}{1.201011in}}%
\pgfpathlineto{\pgfqpoint{1.482673in}{1.196383in}}%
\pgfpathlineto{\pgfqpoint{1.499958in}{1.191477in}}%
\pgfpathlineto{\pgfqpoint{1.516961in}{1.186297in}}%
\pgfpathlineto{\pgfqpoint{1.533666in}{1.180848in}}%
\pgfpathlineto{\pgfqpoint{1.531609in}{1.174833in}}%
\pgfpathlineto{\pgfqpoint{1.529558in}{1.169138in}}%
\pgfpathlineto{\pgfqpoint{1.527512in}{1.163757in}}%
\pgfpathlineto{\pgfqpoint{1.525470in}{1.158684in}}%
\pgfpathlineto{\pgfqpoint{1.509158in}{1.164025in}}%
\pgfpathlineto{\pgfqpoint{1.492554in}{1.169101in}}%
\pgfpathlineto{\pgfqpoint{1.475674in}{1.173909in}}%
\pgfpathlineto{\pgfqpoint{1.458534in}{1.178445in}}%
\pgfpathclose%
\pgfusepath{fill}%
\end{pgfscope}%
\begin{pgfscope}%
\pgfpathrectangle{\pgfqpoint{0.041670in}{0.041670in}}{\pgfqpoint{2.216660in}{2.216660in}}%
\pgfusepath{clip}%
\pgfsetbuttcap%
\pgfsetroundjoin%
\definecolor{currentfill}{rgb}{0.282327,0.094955,0.417331}%
\pgfsetfillcolor{currentfill}%
\pgfsetlinewidth{0.000000pt}%
\definecolor{currentstroke}{rgb}{0.000000,0.000000,0.000000}%
\pgfsetstrokecolor{currentstroke}%
\pgfsetdash{}{0pt}%
\pgfpathmoveto{\pgfqpoint{1.465123in}{1.201011in}}%
\pgfpathlineto{\pgfqpoint{1.466781in}{1.207449in}}%
\pgfpathlineto{\pgfqpoint{1.468443in}{1.214215in}}%
\pgfpathlineto{\pgfqpoint{1.470110in}{1.221317in}}%
\pgfpathlineto{\pgfqpoint{1.471782in}{1.228759in}}%
\pgfpathlineto{\pgfqpoint{1.489747in}{1.224041in}}%
\pgfpathlineto{\pgfqpoint{1.507441in}{1.219041in}}%
\pgfpathlineto{\pgfqpoint{1.524848in}{1.213761in}}%
\pgfpathlineto{\pgfqpoint{1.541951in}{1.208206in}}%
\pgfpathlineto{\pgfqpoint{1.539871in}{1.200860in}}%
\pgfpathlineto{\pgfqpoint{1.537797in}{1.193855in}}%
\pgfpathlineto{\pgfqpoint{1.535729in}{1.187186in}}%
\pgfpathlineto{\pgfqpoint{1.533666in}{1.180848in}}%
\pgfpathlineto{\pgfqpoint{1.516961in}{1.186297in}}%
\pgfpathlineto{\pgfqpoint{1.499958in}{1.191477in}}%
\pgfpathlineto{\pgfqpoint{1.482673in}{1.196383in}}%
\pgfpathlineto{\pgfqpoint{1.465123in}{1.201011in}}%
\pgfpathclose%
\pgfusepath{fill}%
\end{pgfscope}%
\begin{pgfscope}%
\pgfpathrectangle{\pgfqpoint{0.041670in}{0.041670in}}{\pgfqpoint{2.216660in}{2.216660in}}%
\pgfusepath{clip}%
\pgfsetbuttcap%
\pgfsetroundjoin%
\definecolor{currentfill}{rgb}{0.272594,0.025563,0.353093}%
\pgfsetfillcolor{currentfill}%
\pgfsetlinewidth{0.000000pt}%
\definecolor{currentstroke}{rgb}{0.000000,0.000000,0.000000}%
\pgfsetstrokecolor{currentstroke}%
\pgfsetdash{}{0pt}%
\pgfpathmoveto{\pgfqpoint{1.452007in}{1.160718in}}%
\pgfpathlineto{\pgfqpoint{1.453633in}{1.164714in}}%
\pgfpathlineto{\pgfqpoint{1.455263in}{1.168997in}}%
\pgfpathlineto{\pgfqpoint{1.456897in}{1.173572in}}%
\pgfpathlineto{\pgfqpoint{1.458534in}{1.178445in}}%
\pgfpathlineto{\pgfqpoint{1.475674in}{1.173909in}}%
\pgfpathlineto{\pgfqpoint{1.492554in}{1.169101in}}%
\pgfpathlineto{\pgfqpoint{1.509158in}{1.164025in}}%
\pgfpathlineto{\pgfqpoint{1.525470in}{1.158684in}}%
\pgfpathlineto{\pgfqpoint{1.523434in}{1.153914in}}%
\pgfpathlineto{\pgfqpoint{1.521402in}{1.149443in}}%
\pgfpathlineto{\pgfqpoint{1.519375in}{1.145264in}}%
\pgfpathlineto{\pgfqpoint{1.517352in}{1.141372in}}%
\pgfpathlineto{\pgfqpoint{1.501429in}{1.146601in}}%
\pgfpathlineto{\pgfqpoint{1.485220in}{1.151570in}}%
\pgfpathlineto{\pgfqpoint{1.468741in}{1.156277in}}%
\pgfpathlineto{\pgfqpoint{1.452007in}{1.160718in}}%
\pgfpathclose%
\pgfusepath{fill}%
\end{pgfscope}%
\begin{pgfscope}%
\pgfpathrectangle{\pgfqpoint{0.041670in}{0.041670in}}{\pgfqpoint{2.216660in}{2.216660in}}%
\pgfusepath{clip}%
\pgfsetbuttcap%
\pgfsetroundjoin%
\definecolor{currentfill}{rgb}{0.282884,0.135920,0.453427}%
\pgfsetfillcolor{currentfill}%
\pgfsetlinewidth{0.000000pt}%
\definecolor{currentstroke}{rgb}{0.000000,0.000000,0.000000}%
\pgfsetstrokecolor{currentstroke}%
\pgfsetdash{}{0pt}%
\pgfpathmoveto{\pgfqpoint{1.471782in}{1.228759in}}%
\pgfpathlineto{\pgfqpoint{1.473459in}{1.236547in}}%
\pgfpathlineto{\pgfqpoint{1.475142in}{1.244687in}}%
\pgfpathlineto{\pgfqpoint{1.476829in}{1.253184in}}%
\pgfpathlineto{\pgfqpoint{1.478523in}{1.262046in}}%
\pgfpathlineto{\pgfqpoint{1.496908in}{1.257243in}}%
\pgfpathlineto{\pgfqpoint{1.515016in}{1.252151in}}%
\pgfpathlineto{\pgfqpoint{1.532832in}{1.246775in}}%
\pgfpathlineto{\pgfqpoint{1.550338in}{1.241119in}}%
\pgfpathlineto{\pgfqpoint{1.548231in}{1.232349in}}%
\pgfpathlineto{\pgfqpoint{1.546131in}{1.223945in}}%
\pgfpathlineto{\pgfqpoint{1.544038in}{1.215899in}}%
\pgfpathlineto{\pgfqpoint{1.541951in}{1.208206in}}%
\pgfpathlineto{\pgfqpoint{1.524848in}{1.213761in}}%
\pgfpathlineto{\pgfqpoint{1.507441in}{1.219041in}}%
\pgfpathlineto{\pgfqpoint{1.489747in}{1.224041in}}%
\pgfpathlineto{\pgfqpoint{1.471782in}{1.228759in}}%
\pgfpathclose%
\pgfusepath{fill}%
\end{pgfscope}%
\begin{pgfscope}%
\pgfpathrectangle{\pgfqpoint{0.041670in}{0.041670in}}{\pgfqpoint{2.216660in}{2.216660in}}%
\pgfusepath{clip}%
\pgfsetbuttcap%
\pgfsetroundjoin%
\definecolor{currentfill}{rgb}{0.267004,0.004874,0.329415}%
\pgfsetfillcolor{currentfill}%
\pgfsetlinewidth{0.000000pt}%
\definecolor{currentstroke}{rgb}{0.000000,0.000000,0.000000}%
\pgfsetstrokecolor{currentstroke}%
\pgfsetdash{}{0pt}%
\pgfpathmoveto{\pgfqpoint{0.913378in}{1.129717in}}%
\pgfpathlineto{\pgfqpoint{0.911695in}{1.130622in}}%
\pgfpathlineto{\pgfqpoint{0.910010in}{1.131757in}}%
\pgfpathlineto{\pgfqpoint{0.908323in}{1.133126in}}%
\pgfpathlineto{\pgfqpoint{0.906633in}{1.134735in}}%
\pgfpathlineto{\pgfqpoint{0.922591in}{1.138951in}}%
\pgfpathlineto{\pgfqpoint{0.938778in}{1.142908in}}%
\pgfpathlineto{\pgfqpoint{0.955179in}{1.146605in}}%
\pgfpathlineto{\pgfqpoint{0.971778in}{1.150039in}}%
\pgfpathlineto{\pgfqpoint{0.973062in}{1.148339in}}%
\pgfpathlineto{\pgfqpoint{0.974344in}{1.146878in}}%
\pgfpathlineto{\pgfqpoint{0.975625in}{1.145651in}}%
\pgfpathlineto{\pgfqpoint{0.976904in}{1.144654in}}%
\pgfpathlineto{\pgfqpoint{0.960716in}{1.141303in}}%
\pgfpathlineto{\pgfqpoint{0.944723in}{1.137694in}}%
\pgfpathlineto{\pgfqpoint{0.928939in}{1.133831in}}%
\pgfpathlineto{\pgfqpoint{0.913378in}{1.129717in}}%
\pgfpathclose%
\pgfusepath{fill}%
\end{pgfscope}%
\begin{pgfscope}%
\pgfpathrectangle{\pgfqpoint{0.041670in}{0.041670in}}{\pgfqpoint{2.216660in}{2.216660in}}%
\pgfusepath{clip}%
\pgfsetbuttcap%
\pgfsetroundjoin%
\definecolor{currentfill}{rgb}{0.233603,0.313828,0.543914}%
\pgfsetfillcolor{currentfill}%
\pgfsetlinewidth{0.000000pt}%
\definecolor{currentstroke}{rgb}{0.000000,0.000000,0.000000}%
\pgfsetstrokecolor{currentstroke}%
\pgfsetdash{}{0pt}%
\pgfpathmoveto{\pgfqpoint{0.850503in}{1.342373in}}%
\pgfpathlineto{\pgfqpoint{0.848655in}{1.354772in}}%
\pgfpathlineto{\pgfqpoint{0.846800in}{1.367599in}}%
\pgfpathlineto{\pgfqpoint{0.844936in}{1.380861in}}%
\pgfpathlineto{\pgfqpoint{0.843064in}{1.394565in}}%
\pgfpathlineto{\pgfqpoint{0.862779in}{1.399567in}}%
\pgfpathlineto{\pgfqpoint{0.882767in}{1.404261in}}%
\pgfpathlineto{\pgfqpoint{0.903010in}{1.408645in}}%
\pgfpathlineto{\pgfqpoint{0.923491in}{1.412715in}}%
\pgfpathlineto{\pgfqpoint{0.924912in}{1.398947in}}%
\pgfpathlineto{\pgfqpoint{0.926327in}{1.385620in}}%
\pgfpathlineto{\pgfqpoint{0.927736in}{1.372727in}}%
\pgfpathlineto{\pgfqpoint{0.929139in}{1.360261in}}%
\pgfpathlineto{\pgfqpoint{0.909113in}{1.356250in}}%
\pgfpathlineto{\pgfqpoint{0.889320in}{1.351929in}}%
\pgfpathlineto{\pgfqpoint{0.869778in}{1.347302in}}%
\pgfpathlineto{\pgfqpoint{0.850503in}{1.342373in}}%
\pgfpathclose%
\pgfusepath{fill}%
\end{pgfscope}%
\begin{pgfscope}%
\pgfpathrectangle{\pgfqpoint{0.041670in}{0.041670in}}{\pgfqpoint{2.216660in}{2.216660in}}%
\pgfusepath{clip}%
\pgfsetbuttcap%
\pgfsetroundjoin%
\definecolor{currentfill}{rgb}{0.277941,0.056324,0.381191}%
\pgfsetfillcolor{currentfill}%
\pgfsetlinewidth{0.000000pt}%
\definecolor{currentstroke}{rgb}{0.000000,0.000000,0.000000}%
\pgfsetstrokecolor{currentstroke}%
\pgfsetdash{}{0pt}%
\pgfpathmoveto{\pgfqpoint{0.820196in}{1.153718in}}%
\pgfpathlineto{\pgfqpoint{0.818069in}{1.158765in}}%
\pgfpathlineto{\pgfqpoint{0.815937in}{1.164121in}}%
\pgfpathlineto{\pgfqpoint{0.813799in}{1.169791in}}%
\pgfpathlineto{\pgfqpoint{0.811656in}{1.175780in}}%
\pgfpathlineto{\pgfqpoint{0.828084in}{1.181466in}}%
\pgfpathlineto{\pgfqpoint{0.844824in}{1.186886in}}%
\pgfpathlineto{\pgfqpoint{0.861859in}{1.192036in}}%
\pgfpathlineto{\pgfqpoint{0.879174in}{1.196911in}}%
\pgfpathlineto{\pgfqpoint{0.880919in}{1.190817in}}%
\pgfpathlineto{\pgfqpoint{0.882659in}{1.185042in}}%
\pgfpathlineto{\pgfqpoint{0.884395in}{1.179580in}}%
\pgfpathlineto{\pgfqpoint{0.886127in}{1.174426in}}%
\pgfpathlineto{\pgfqpoint{0.869218in}{1.169648in}}%
\pgfpathlineto{\pgfqpoint{0.852583in}{1.164602in}}%
\pgfpathlineto{\pgfqpoint{0.836237in}{1.159290in}}%
\pgfpathlineto{\pgfqpoint{0.820196in}{1.153718in}}%
\pgfpathclose%
\pgfusepath{fill}%
\end{pgfscope}%
\begin{pgfscope}%
\pgfpathrectangle{\pgfqpoint{0.041670in}{0.041670in}}{\pgfqpoint{2.216660in}{2.216660in}}%
\pgfusepath{clip}%
\pgfsetbuttcap%
\pgfsetroundjoin%
\definecolor{currentfill}{rgb}{0.282327,0.094955,0.417331}%
\pgfsetfillcolor{currentfill}%
\pgfsetlinewidth{0.000000pt}%
\definecolor{currentstroke}{rgb}{0.000000,0.000000,0.000000}%
\pgfsetstrokecolor{currentstroke}%
\pgfsetdash{}{0pt}%
\pgfpathmoveto{\pgfqpoint{0.811656in}{1.175780in}}%
\pgfpathlineto{\pgfqpoint{0.809507in}{1.182094in}}%
\pgfpathlineto{\pgfqpoint{0.807352in}{1.188738in}}%
\pgfpathlineto{\pgfqpoint{0.805191in}{1.195718in}}%
\pgfpathlineto{\pgfqpoint{0.803023in}{1.203040in}}%
\pgfpathlineto{\pgfqpoint{0.819843in}{1.208837in}}%
\pgfpathlineto{\pgfqpoint{0.836981in}{1.214361in}}%
\pgfpathlineto{\pgfqpoint{0.854420in}{1.219610in}}%
\pgfpathlineto{\pgfqpoint{0.872145in}{1.224579in}}%
\pgfpathlineto{\pgfqpoint{0.873910in}{1.217157in}}%
\pgfpathlineto{\pgfqpoint{0.875670in}{1.210075in}}%
\pgfpathlineto{\pgfqpoint{0.877424in}{1.203328in}}%
\pgfpathlineto{\pgfqpoint{0.879174in}{1.196911in}}%
\pgfpathlineto{\pgfqpoint{0.861859in}{1.192036in}}%
\pgfpathlineto{\pgfqpoint{0.844824in}{1.186886in}}%
\pgfpathlineto{\pgfqpoint{0.828084in}{1.181466in}}%
\pgfpathlineto{\pgfqpoint{0.811656in}{1.175780in}}%
\pgfpathclose%
\pgfusepath{fill}%
\end{pgfscope}%
\begin{pgfscope}%
\pgfpathrectangle{\pgfqpoint{0.041670in}{0.041670in}}{\pgfqpoint{2.216660in}{2.216660in}}%
\pgfusepath{clip}%
\pgfsetbuttcap%
\pgfsetroundjoin%
\definecolor{currentfill}{rgb}{0.268510,0.009605,0.335427}%
\pgfsetfillcolor{currentfill}%
\pgfsetlinewidth{0.000000pt}%
\definecolor{currentstroke}{rgb}{0.000000,0.000000,0.000000}%
\pgfsetstrokecolor{currentstroke}%
\pgfsetdash{}{0pt}%
\pgfpathmoveto{\pgfqpoint{1.445532in}{1.147506in}}%
\pgfpathlineto{\pgfqpoint{1.447146in}{1.150403in}}%
\pgfpathlineto{\pgfqpoint{1.448763in}{1.153568in}}%
\pgfpathlineto{\pgfqpoint{1.450383in}{1.157004in}}%
\pgfpathlineto{\pgfqpoint{1.452007in}{1.160718in}}%
\pgfpathlineto{\pgfqpoint{1.468741in}{1.156277in}}%
\pgfpathlineto{\pgfqpoint{1.485220in}{1.151570in}}%
\pgfpathlineto{\pgfqpoint{1.501429in}{1.146601in}}%
\pgfpathlineto{\pgfqpoint{1.517352in}{1.141372in}}%
\pgfpathlineto{\pgfqpoint{1.515334in}{1.137764in}}%
\pgfpathlineto{\pgfqpoint{1.513319in}{1.134433in}}%
\pgfpathlineto{\pgfqpoint{1.511308in}{1.131375in}}%
\pgfpathlineto{\pgfqpoint{1.509301in}{1.128586in}}%
\pgfpathlineto{\pgfqpoint{1.493763in}{1.133699in}}%
\pgfpathlineto{\pgfqpoint{1.477946in}{1.138559in}}%
\pgfpathlineto{\pgfqpoint{1.461864in}{1.143162in}}%
\pgfpathlineto{\pgfqpoint{1.445532in}{1.147506in}}%
\pgfpathclose%
\pgfusepath{fill}%
\end{pgfscope}%
\begin{pgfscope}%
\pgfpathrectangle{\pgfqpoint{0.041670in}{0.041670in}}{\pgfqpoint{2.216660in}{2.216660in}}%
\pgfusepath{clip}%
\pgfsetbuttcap%
\pgfsetroundjoin%
\definecolor{currentfill}{rgb}{0.274952,0.037752,0.364543}%
\pgfsetfillcolor{currentfill}%
\pgfsetlinewidth{0.000000pt}%
\definecolor{currentstroke}{rgb}{0.000000,0.000000,0.000000}%
\pgfsetstrokecolor{currentstroke}%
\pgfsetdash{}{0pt}%
\pgfpathmoveto{\pgfqpoint{1.116754in}{1.165039in}}%
\pgfpathlineto{\pgfqpoint{1.116329in}{1.163721in}}%
\pgfpathlineto{\pgfqpoint{1.115903in}{1.162582in}}%
\pgfpathlineto{\pgfqpoint{1.115478in}{1.161626in}}%
\pgfpathlineto{\pgfqpoint{1.115052in}{1.160857in}}%
\pgfpathlineto{\pgfqpoint{1.131457in}{1.161755in}}%
\pgfpathlineto{\pgfqpoint{1.147905in}{1.162392in}}%
\pgfpathlineto{\pgfqpoint{1.164381in}{1.162769in}}%
\pgfpathlineto{\pgfqpoint{1.180871in}{1.162885in}}%
\pgfpathlineto{\pgfqpoint{1.180865in}{1.163640in}}%
\pgfpathlineto{\pgfqpoint{1.180859in}{1.164583in}}%
\pgfpathlineto{\pgfqpoint{1.180853in}{1.165708in}}%
\pgfpathlineto{\pgfqpoint{1.180847in}{1.167013in}}%
\pgfpathlineto{\pgfqpoint{1.164789in}{1.166900in}}%
\pgfpathlineto{\pgfqpoint{1.148745in}{1.166533in}}%
\pgfpathlineto{\pgfqpoint{1.132729in}{1.165913in}}%
\pgfpathlineto{\pgfqpoint{1.116754in}{1.165039in}}%
\pgfpathclose%
\pgfusepath{fill}%
\end{pgfscope}%
\begin{pgfscope}%
\pgfpathrectangle{\pgfqpoint{0.041670in}{0.041670in}}{\pgfqpoint{2.216660in}{2.216660in}}%
\pgfusepath{clip}%
\pgfsetbuttcap%
\pgfsetroundjoin%
\definecolor{currentfill}{rgb}{0.274952,0.037752,0.364543}%
\pgfsetfillcolor{currentfill}%
\pgfsetlinewidth{0.000000pt}%
\definecolor{currentstroke}{rgb}{0.000000,0.000000,0.000000}%
\pgfsetstrokecolor{currentstroke}%
\pgfsetdash{}{0pt}%
\pgfpathmoveto{\pgfqpoint{1.180847in}{1.167013in}}%
\pgfpathlineto{\pgfqpoint{1.180853in}{1.165708in}}%
\pgfpathlineto{\pgfqpoint{1.180859in}{1.164583in}}%
\pgfpathlineto{\pgfqpoint{1.180865in}{1.163640in}}%
\pgfpathlineto{\pgfqpoint{1.180871in}{1.162885in}}%
\pgfpathlineto{\pgfqpoint{1.197360in}{1.162740in}}%
\pgfpathlineto{\pgfqpoint{1.213834in}{1.162334in}}%
\pgfpathlineto{\pgfqpoint{1.230278in}{1.161668in}}%
\pgfpathlineto{\pgfqpoint{1.246677in}{1.160741in}}%
\pgfpathlineto{\pgfqpoint{1.246240in}{1.161511in}}%
\pgfpathlineto{\pgfqpoint{1.245802in}{1.162468in}}%
\pgfpathlineto{\pgfqpoint{1.245365in}{1.163608in}}%
\pgfpathlineto{\pgfqpoint{1.244927in}{1.164927in}}%
\pgfpathlineto{\pgfqpoint{1.228958in}{1.165828in}}%
\pgfpathlineto{\pgfqpoint{1.212946in}{1.166477in}}%
\pgfpathlineto{\pgfqpoint{1.196904in}{1.166872in}}%
\pgfpathlineto{\pgfqpoint{1.180847in}{1.167013in}}%
\pgfpathclose%
\pgfusepath{fill}%
\end{pgfscope}%
\begin{pgfscope}%
\pgfpathrectangle{\pgfqpoint{0.041670in}{0.041670in}}{\pgfqpoint{2.216660in}{2.216660in}}%
\pgfusepath{clip}%
\pgfsetbuttcap%
\pgfsetroundjoin%
\definecolor{currentfill}{rgb}{0.172719,0.448791,0.557885}%
\pgfsetfillcolor{currentfill}%
\pgfsetlinewidth{0.000000pt}%
\definecolor{currentstroke}{rgb}{0.000000,0.000000,0.000000}%
\pgfsetstrokecolor{currentstroke}%
\pgfsetdash{}{0pt}%
\pgfpathmoveto{\pgfqpoint{1.358762in}{1.485305in}}%
\pgfpathlineto{\pgfqpoint{1.359752in}{1.501427in}}%
\pgfpathlineto{\pgfqpoint{1.360747in}{1.518032in}}%
\pgfpathlineto{\pgfqpoint{1.361747in}{1.535130in}}%
\pgfpathlineto{\pgfqpoint{1.383821in}{1.532242in}}%
\pgfpathlineto{\pgfqpoint{1.405722in}{1.529022in}}%
\pgfpathlineto{\pgfqpoint{1.427432in}{1.525474in}}%
\pgfpathlineto{\pgfqpoint{1.426069in}{1.508404in}}%
\pgfpathlineto{\pgfqpoint{1.424713in}{1.491827in}}%
\pgfpathlineto{\pgfqpoint{1.423363in}{1.475735in}}%
\pgfpathlineto{\pgfqpoint{1.402012in}{1.479252in}}%
\pgfpathlineto{\pgfqpoint{1.380472in}{1.482443in}}%
\pgfpathlineto{\pgfqpoint{1.358762in}{1.485305in}}%
\pgfpathclose%
\pgfusepath{fill}%
\end{pgfscope}%
\begin{pgfscope}%
\pgfpathrectangle{\pgfqpoint{0.041670in}{0.041670in}}{\pgfqpoint{2.216660in}{2.216660in}}%
\pgfusepath{clip}%
\pgfsetbuttcap%
\pgfsetroundjoin%
\definecolor{currentfill}{rgb}{0.276194,0.190074,0.493001}%
\pgfsetfillcolor{currentfill}%
\pgfsetlinewidth{0.000000pt}%
\definecolor{currentstroke}{rgb}{0.000000,0.000000,0.000000}%
\pgfsetstrokecolor{currentstroke}%
\pgfsetdash{}{0pt}%
\pgfpathmoveto{\pgfqpoint{1.478523in}{1.262046in}}%
\pgfpathlineto{\pgfqpoint{1.480221in}{1.271277in}}%
\pgfpathlineto{\pgfqpoint{1.481926in}{1.280884in}}%
\pgfpathlineto{\pgfqpoint{1.483637in}{1.290873in}}%
\pgfpathlineto{\pgfqpoint{1.485354in}{1.301250in}}%
\pgfpathlineto{\pgfqpoint{1.504165in}{1.296365in}}%
\pgfpathlineto{\pgfqpoint{1.522694in}{1.291187in}}%
\pgfpathlineto{\pgfqpoint{1.540924in}{1.285719in}}%
\pgfpathlineto{\pgfqpoint{1.558839in}{1.279965in}}%
\pgfpathlineto{\pgfqpoint{1.556702in}{1.269676in}}%
\pgfpathlineto{\pgfqpoint{1.554573in}{1.259776in}}%
\pgfpathlineto{\pgfqpoint{1.552452in}{1.250259in}}%
\pgfpathlineto{\pgfqpoint{1.550338in}{1.241119in}}%
\pgfpathlineto{\pgfqpoint{1.532832in}{1.246775in}}%
\pgfpathlineto{\pgfqpoint{1.515016in}{1.252151in}}%
\pgfpathlineto{\pgfqpoint{1.496908in}{1.257243in}}%
\pgfpathlineto{\pgfqpoint{1.478523in}{1.262046in}}%
\pgfpathclose%
\pgfusepath{fill}%
\end{pgfscope}%
\begin{pgfscope}%
\pgfpathrectangle{\pgfqpoint{0.041670in}{0.041670in}}{\pgfqpoint{2.216660in}{2.216660in}}%
\pgfusepath{clip}%
\pgfsetbuttcap%
\pgfsetroundjoin%
\definecolor{currentfill}{rgb}{0.271305,0.019942,0.347269}%
\pgfsetfillcolor{currentfill}%
\pgfsetlinewidth{0.000000pt}%
\definecolor{currentstroke}{rgb}{0.000000,0.000000,0.000000}%
\pgfsetstrokecolor{currentstroke}%
\pgfsetdash{}{0pt}%
\pgfpathmoveto{\pgfqpoint{1.311540in}{1.154452in}}%
\pgfpathlineto{\pgfqpoint{1.312405in}{1.153832in}}%
\pgfpathlineto{\pgfqpoint{1.313270in}{1.153407in}}%
\pgfpathlineto{\pgfqpoint{1.314137in}{1.153182in}}%
\pgfpathlineto{\pgfqpoint{1.315003in}{1.153160in}}%
\pgfpathlineto{\pgfqpoint{1.331384in}{1.150890in}}%
\pgfpathlineto{\pgfqpoint{1.347631in}{1.148360in}}%
\pgfpathlineto{\pgfqpoint{1.363730in}{1.145572in}}%
\pgfpathlineto{\pgfqpoint{1.362549in}{1.145642in}}%
\pgfpathlineto{\pgfqpoint{1.361369in}{1.145917in}}%
\pgfpathlineto{\pgfqpoint{1.360190in}{1.146391in}}%
\pgfpathlineto{\pgfqpoint{1.359012in}{1.147060in}}%
\pgfpathlineto{\pgfqpoint{1.343328in}{1.149776in}}%
\pgfpathlineto{\pgfqpoint{1.327500in}{1.152240in}}%
\pgfpathlineto{\pgfqpoint{1.311540in}{1.154452in}}%
\pgfpathclose%
\pgfusepath{fill}%
\end{pgfscope}%
\begin{pgfscope}%
\pgfpathrectangle{\pgfqpoint{0.041670in}{0.041670in}}{\pgfqpoint{2.216660in}{2.216660in}}%
\pgfusepath{clip}%
\pgfsetbuttcap%
\pgfsetroundjoin%
\definecolor{currentfill}{rgb}{0.272594,0.025563,0.353093}%
\pgfsetfillcolor{currentfill}%
\pgfsetlinewidth{0.000000pt}%
\definecolor{currentstroke}{rgb}{0.000000,0.000000,0.000000}%
\pgfsetstrokecolor{currentstroke}%
\pgfsetdash{}{0pt}%
\pgfpathmoveto{\pgfqpoint{0.828655in}{1.136511in}}%
\pgfpathlineto{\pgfqpoint{0.826547in}{1.140376in}}%
\pgfpathlineto{\pgfqpoint{0.824435in}{1.144529in}}%
\pgfpathlineto{\pgfqpoint{0.822318in}{1.148974in}}%
\pgfpathlineto{\pgfqpoint{0.820196in}{1.153718in}}%
\pgfpathlineto{\pgfqpoint{0.836237in}{1.159290in}}%
\pgfpathlineto{\pgfqpoint{0.852583in}{1.164602in}}%
\pgfpathlineto{\pgfqpoint{0.869218in}{1.169648in}}%
\pgfpathlineto{\pgfqpoint{0.886127in}{1.174426in}}%
\pgfpathlineto{\pgfqpoint{0.887855in}{1.169575in}}%
\pgfpathlineto{\pgfqpoint{0.889579in}{1.165021in}}%
\pgfpathlineto{\pgfqpoint{0.891299in}{1.160759in}}%
\pgfpathlineto{\pgfqpoint{0.893016in}{1.156784in}}%
\pgfpathlineto{\pgfqpoint{0.876508in}{1.152106in}}%
\pgfpathlineto{\pgfqpoint{0.860268in}{1.147166in}}%
\pgfpathlineto{\pgfqpoint{0.844312in}{1.141966in}}%
\pgfpathlineto{\pgfqpoint{0.828655in}{1.136511in}}%
\pgfpathclose%
\pgfusepath{fill}%
\end{pgfscope}%
\begin{pgfscope}%
\pgfpathrectangle{\pgfqpoint{0.041670in}{0.041670in}}{\pgfqpoint{2.216660in}{2.216660in}}%
\pgfusepath{clip}%
\pgfsetbuttcap%
\pgfsetroundjoin%
\definecolor{currentfill}{rgb}{0.282884,0.135920,0.453427}%
\pgfsetfillcolor{currentfill}%
\pgfsetlinewidth{0.000000pt}%
\definecolor{currentstroke}{rgb}{0.000000,0.000000,0.000000}%
\pgfsetstrokecolor{currentstroke}%
\pgfsetdash{}{0pt}%
\pgfpathmoveto{\pgfqpoint{0.803023in}{1.203040in}}%
\pgfpathlineto{\pgfqpoint{0.800848in}{1.210709in}}%
\pgfpathlineto{\pgfqpoint{0.798667in}{1.218731in}}%
\pgfpathlineto{\pgfqpoint{0.796479in}{1.227112in}}%
\pgfpathlineto{\pgfqpoint{0.794284in}{1.235859in}}%
\pgfpathlineto{\pgfqpoint{0.811501in}{1.241761in}}%
\pgfpathlineto{\pgfqpoint{0.829043in}{1.247387in}}%
\pgfpathlineto{\pgfqpoint{0.846892in}{1.252731in}}%
\pgfpathlineto{\pgfqpoint{0.865032in}{1.257791in}}%
\pgfpathlineto{\pgfqpoint{0.866819in}{1.248948in}}%
\pgfpathlineto{\pgfqpoint{0.868600in}{1.240469in}}%
\pgfpathlineto{\pgfqpoint{0.870375in}{1.232348in}}%
\pgfpathlineto{\pgfqpoint{0.872145in}{1.224579in}}%
\pgfpathlineto{\pgfqpoint{0.854420in}{1.219610in}}%
\pgfpathlineto{\pgfqpoint{0.836981in}{1.214361in}}%
\pgfpathlineto{\pgfqpoint{0.819843in}{1.208837in}}%
\pgfpathlineto{\pgfqpoint{0.803023in}{1.203040in}}%
\pgfpathclose%
\pgfusepath{fill}%
\end{pgfscope}%
\begin{pgfscope}%
\pgfpathrectangle{\pgfqpoint{0.041670in}{0.041670in}}{\pgfqpoint{2.216660in}{2.216660in}}%
\pgfusepath{clip}%
\pgfsetbuttcap%
\pgfsetroundjoin%
\definecolor{currentfill}{rgb}{0.172719,0.448791,0.557885}%
\pgfsetfillcolor{currentfill}%
\pgfsetlinewidth{0.000000pt}%
\definecolor{currentstroke}{rgb}{0.000000,0.000000,0.000000}%
\pgfsetstrokecolor{currentstroke}%
\pgfsetdash{}{0pt}%
\pgfpathmoveto{\pgfqpoint{0.917741in}{1.472338in}}%
\pgfpathlineto{\pgfqpoint{0.916287in}{1.488420in}}%
\pgfpathlineto{\pgfqpoint{0.914825in}{1.504986in}}%
\pgfpathlineto{\pgfqpoint{0.913356in}{1.522046in}}%
\pgfpathlineto{\pgfqpoint{0.934880in}{1.525884in}}%
\pgfpathlineto{\pgfqpoint{0.956612in}{1.529396in}}%
\pgfpathlineto{\pgfqpoint{0.978533in}{1.532579in}}%
\pgfpathlineto{\pgfqpoint{1.000625in}{1.535431in}}%
\pgfpathlineto{\pgfqpoint{1.001612in}{1.518332in}}%
\pgfpathlineto{\pgfqpoint{1.002593in}{1.501725in}}%
\pgfpathlineto{\pgfqpoint{1.003569in}{1.485603in}}%
\pgfpathlineto{\pgfqpoint{0.981842in}{1.482777in}}%
\pgfpathlineto{\pgfqpoint{0.960282in}{1.479622in}}%
\pgfpathlineto{\pgfqpoint{0.938909in}{1.476142in}}%
\pgfpathlineto{\pgfqpoint{0.917741in}{1.472338in}}%
\pgfpathclose%
\pgfusepath{fill}%
\end{pgfscope}%
\begin{pgfscope}%
\pgfpathrectangle{\pgfqpoint{0.041670in}{0.041670in}}{\pgfqpoint{2.216660in}{2.216660in}}%
\pgfusepath{clip}%
\pgfsetbuttcap%
\pgfsetroundjoin%
\definecolor{currentfill}{rgb}{0.271305,0.019942,0.347269}%
\pgfsetfillcolor{currentfill}%
\pgfsetlinewidth{0.000000pt}%
\definecolor{currentstroke}{rgb}{0.000000,0.000000,0.000000}%
\pgfsetstrokecolor{currentstroke}%
\pgfsetdash{}{0pt}%
\pgfpathmoveto{\pgfqpoint{0.987091in}{1.144438in}}%
\pgfpathlineto{\pgfqpoint{0.985821in}{1.143751in}}%
\pgfpathlineto{\pgfqpoint{0.984551in}{1.143259in}}%
\pgfpathlineto{\pgfqpoint{0.983279in}{1.142967in}}%
\pgfpathlineto{\pgfqpoint{0.982007in}{1.142880in}}%
\pgfpathlineto{\pgfqpoint{0.997961in}{1.145894in}}%
\pgfpathlineto{\pgfqpoint{1.014076in}{1.148653in}}%
\pgfpathlineto{\pgfqpoint{1.030339in}{1.151155in}}%
\pgfpathlineto{\pgfqpoint{1.046734in}{1.153396in}}%
\pgfpathlineto{\pgfqpoint{1.047589in}{1.153416in}}%
\pgfpathlineto{\pgfqpoint{1.048444in}{1.153640in}}%
\pgfpathlineto{\pgfqpoint{1.049297in}{1.154064in}}%
\pgfpathlineto{\pgfqpoint{1.050151in}{1.154682in}}%
\pgfpathlineto{\pgfqpoint{1.034177in}{1.152499in}}%
\pgfpathlineto{\pgfqpoint{1.018333in}{1.150062in}}%
\pgfpathlineto{\pgfqpoint{1.002633in}{1.147374in}}%
\pgfpathlineto{\pgfqpoint{0.987091in}{1.144438in}}%
\pgfpathclose%
\pgfusepath{fill}%
\end{pgfscope}%
\begin{pgfscope}%
\pgfpathrectangle{\pgfqpoint{0.041670in}{0.041670in}}{\pgfqpoint{2.216660in}{2.216660in}}%
\pgfusepath{clip}%
\pgfsetbuttcap%
\pgfsetroundjoin%
\definecolor{currentfill}{rgb}{0.268510,0.009605,0.335427}%
\pgfsetfillcolor{currentfill}%
\pgfsetlinewidth{0.000000pt}%
\definecolor{currentstroke}{rgb}{0.000000,0.000000,0.000000}%
\pgfsetstrokecolor{currentstroke}%
\pgfsetdash{}{0pt}%
\pgfpathmoveto{\pgfqpoint{1.363730in}{1.145572in}}%
\pgfpathlineto{\pgfqpoint{1.364912in}{1.145710in}}%
\pgfpathlineto{\pgfqpoint{1.366095in}{1.146060in}}%
\pgfpathlineto{\pgfqpoint{1.367279in}{1.146627in}}%
\pgfpathlineto{\pgfqpoint{1.368465in}{1.147415in}}%
\pgfpathlineto{\pgfqpoint{1.384813in}{1.144295in}}%
\pgfpathlineto{\pgfqpoint{1.400980in}{1.140915in}}%
\pgfpathlineto{\pgfqpoint{1.416951in}{1.137278in}}%
\pgfpathlineto{\pgfqpoint{1.432711in}{1.133387in}}%
\pgfpathlineto{\pgfqpoint{1.431118in}{1.132685in}}%
\pgfpathlineto{\pgfqpoint{1.429527in}{1.132205in}}%
\pgfpathlineto{\pgfqpoint{1.427937in}{1.131943in}}%
\pgfpathlineto{\pgfqpoint{1.426349in}{1.131893in}}%
\pgfpathlineto{\pgfqpoint{1.410989in}{1.135687in}}%
\pgfpathlineto{\pgfqpoint{1.395423in}{1.139233in}}%
\pgfpathlineto{\pgfqpoint{1.379665in}{1.142529in}}%
\pgfpathlineto{\pgfqpoint{1.363730in}{1.145572in}}%
\pgfpathclose%
\pgfusepath{fill}%
\end{pgfscope}%
\begin{pgfscope}%
\pgfpathrectangle{\pgfqpoint{0.041670in}{0.041670in}}{\pgfqpoint{2.216660in}{2.216660in}}%
\pgfusepath{clip}%
\pgfsetbuttcap%
\pgfsetroundjoin%
\definecolor{currentfill}{rgb}{0.201239,0.383670,0.554294}%
\pgfsetfillcolor{currentfill}%
\pgfsetlinewidth{0.000000pt}%
\definecolor{currentstroke}{rgb}{0.000000,0.000000,0.000000}%
\pgfsetstrokecolor{currentstroke}%
\pgfsetdash{}{0pt}%
\pgfpathmoveto{\pgfqpoint{1.418028in}{1.416067in}}%
\pgfpathlineto{\pgfqpoint{1.419353in}{1.430294in}}%
\pgfpathlineto{\pgfqpoint{1.420683in}{1.444977in}}%
\pgfpathlineto{\pgfqpoint{1.422020in}{1.460121in}}%
\pgfpathlineto{\pgfqpoint{1.423363in}{1.475735in}}%
\pgfpathlineto{\pgfqpoint{1.444507in}{1.471896in}}%
\pgfpathlineto{\pgfqpoint{1.465425in}{1.467736in}}%
\pgfpathlineto{\pgfqpoint{1.486099in}{1.463259in}}%
\pgfpathlineto{\pgfqpoint{1.506511in}{1.458467in}}%
\pgfpathlineto{\pgfqpoint{1.504704in}{1.442908in}}%
\pgfpathlineto{\pgfqpoint{1.502906in}{1.427819in}}%
\pgfpathlineto{\pgfqpoint{1.501117in}{1.413195in}}%
\pgfpathlineto{\pgfqpoint{1.499335in}{1.399026in}}%
\pgfpathlineto{\pgfqpoint{1.479377in}{1.403754in}}%
\pgfpathlineto{\pgfqpoint{1.459161in}{1.408173in}}%
\pgfpathlineto{\pgfqpoint{1.438706in}{1.412278in}}%
\pgfpathlineto{\pgfqpoint{1.418028in}{1.416067in}}%
\pgfpathclose%
\pgfusepath{fill}%
\end{pgfscope}%
\begin{pgfscope}%
\pgfpathrectangle{\pgfqpoint{0.041670in}{0.041670in}}{\pgfqpoint{2.216660in}{2.216660in}}%
\pgfusepath{clip}%
\pgfsetbuttcap%
\pgfsetroundjoin%
\definecolor{currentfill}{rgb}{0.274952,0.037752,0.364543}%
\pgfsetfillcolor{currentfill}%
\pgfsetlinewidth{0.000000pt}%
\definecolor{currentstroke}{rgb}{0.000000,0.000000,0.000000}%
\pgfsetstrokecolor{currentstroke}%
\pgfsetdash{}{0pt}%
\pgfpathmoveto{\pgfqpoint{1.053558in}{1.159030in}}%
\pgfpathlineto{\pgfqpoint{1.052707in}{1.157671in}}%
\pgfpathlineto{\pgfqpoint{1.051855in}{1.156490in}}%
\pgfpathlineto{\pgfqpoint{1.051003in}{1.155493in}}%
\pgfpathlineto{\pgfqpoint{1.050151in}{1.154682in}}%
\pgfpathlineto{\pgfqpoint{1.066239in}{1.156612in}}%
\pgfpathlineto{\pgfqpoint{1.082428in}{1.158285in}}%
\pgfpathlineto{\pgfqpoint{1.098704in}{1.159700in}}%
\pgfpathlineto{\pgfqpoint{1.115052in}{1.160857in}}%
\pgfpathlineto{\pgfqpoint{1.115478in}{1.161626in}}%
\pgfpathlineto{\pgfqpoint{1.115903in}{1.162582in}}%
\pgfpathlineto{\pgfqpoint{1.116329in}{1.163721in}}%
\pgfpathlineto{\pgfqpoint{1.116754in}{1.165039in}}%
\pgfpathlineto{\pgfqpoint{1.100836in}{1.163913in}}%
\pgfpathlineto{\pgfqpoint{1.084987in}{1.162536in}}%
\pgfpathlineto{\pgfqpoint{1.069223in}{1.160908in}}%
\pgfpathlineto{\pgfqpoint{1.053558in}{1.159030in}}%
\pgfpathclose%
\pgfusepath{fill}%
\end{pgfscope}%
\begin{pgfscope}%
\pgfpathrectangle{\pgfqpoint{0.041670in}{0.041670in}}{\pgfqpoint{2.216660in}{2.216660in}}%
\pgfusepath{clip}%
\pgfsetbuttcap%
\pgfsetroundjoin%
\definecolor{currentfill}{rgb}{0.274952,0.037752,0.364543}%
\pgfsetfillcolor{currentfill}%
\pgfsetlinewidth{0.000000pt}%
\definecolor{currentstroke}{rgb}{0.000000,0.000000,0.000000}%
\pgfsetstrokecolor{currentstroke}%
\pgfsetdash{}{0pt}%
\pgfpathmoveto{\pgfqpoint{1.244927in}{1.164927in}}%
\pgfpathlineto{\pgfqpoint{1.245365in}{1.163608in}}%
\pgfpathlineto{\pgfqpoint{1.245802in}{1.162468in}}%
\pgfpathlineto{\pgfqpoint{1.246240in}{1.161511in}}%
\pgfpathlineto{\pgfqpoint{1.246677in}{1.160741in}}%
\pgfpathlineto{\pgfqpoint{1.263018in}{1.159556in}}%
\pgfpathlineto{\pgfqpoint{1.279285in}{1.158111in}}%
\pgfpathlineto{\pgfqpoint{1.295464in}{1.156410in}}%
\pgfpathlineto{\pgfqpoint{1.311540in}{1.154452in}}%
\pgfpathlineto{\pgfqpoint{1.310676in}{1.155264in}}%
\pgfpathlineto{\pgfqpoint{1.309812in}{1.156263in}}%
\pgfpathlineto{\pgfqpoint{1.308948in}{1.157445in}}%
\pgfpathlineto{\pgfqpoint{1.308086in}{1.158806in}}%
\pgfpathlineto{\pgfqpoint{1.292432in}{1.160711in}}%
\pgfpathlineto{\pgfqpoint{1.276678in}{1.162367in}}%
\pgfpathlineto{\pgfqpoint{1.260839in}{1.163773in}}%
\pgfpathlineto{\pgfqpoint{1.244927in}{1.164927in}}%
\pgfpathclose%
\pgfusepath{fill}%
\end{pgfscope}%
\begin{pgfscope}%
\pgfpathrectangle{\pgfqpoint{0.041670in}{0.041670in}}{\pgfqpoint{2.216660in}{2.216660in}}%
\pgfusepath{clip}%
\pgfsetbuttcap%
\pgfsetroundjoin%
\definecolor{currentfill}{rgb}{0.268510,0.009605,0.335427}%
\pgfsetfillcolor{currentfill}%
\pgfsetlinewidth{0.000000pt}%
\definecolor{currentstroke}{rgb}{0.000000,0.000000,0.000000}%
\pgfsetstrokecolor{currentstroke}%
\pgfsetdash{}{0pt}%
\pgfpathmoveto{\pgfqpoint{0.837043in}{1.123832in}}%
\pgfpathlineto{\pgfqpoint{0.834952in}{1.126595in}}%
\pgfpathlineto{\pgfqpoint{0.832857in}{1.129626in}}%
\pgfpathlineto{\pgfqpoint{0.830758in}{1.132930in}}%
\pgfpathlineto{\pgfqpoint{0.828655in}{1.136511in}}%
\pgfpathlineto{\pgfqpoint{0.844312in}{1.141966in}}%
\pgfpathlineto{\pgfqpoint{0.860268in}{1.147166in}}%
\pgfpathlineto{\pgfqpoint{0.876508in}{1.152106in}}%
\pgfpathlineto{\pgfqpoint{0.893016in}{1.156784in}}%
\pgfpathlineto{\pgfqpoint{0.894729in}{1.153091in}}%
\pgfpathlineto{\pgfqpoint{0.896438in}{1.149676in}}%
\pgfpathlineto{\pgfqpoint{0.898145in}{1.146533in}}%
\pgfpathlineto{\pgfqpoint{0.899848in}{1.143658in}}%
\pgfpathlineto{\pgfqpoint{0.883738in}{1.139083in}}%
\pgfpathlineto{\pgfqpoint{0.867891in}{1.134251in}}%
\pgfpathlineto{\pgfqpoint{0.852321in}{1.129166in}}%
\pgfpathlineto{\pgfqpoint{0.837043in}{1.123832in}}%
\pgfpathclose%
\pgfusepath{fill}%
\end{pgfscope}%
\begin{pgfscope}%
\pgfpathrectangle{\pgfqpoint{0.041670in}{0.041670in}}{\pgfqpoint{2.216660in}{2.216660in}}%
\pgfusepath{clip}%
\pgfsetbuttcap%
\pgfsetroundjoin%
\definecolor{currentfill}{rgb}{0.267004,0.004874,0.329415}%
\pgfsetfillcolor{currentfill}%
\pgfsetlinewidth{0.000000pt}%
\definecolor{currentstroke}{rgb}{0.000000,0.000000,0.000000}%
\pgfsetstrokecolor{currentstroke}%
\pgfsetdash{}{0pt}%
\pgfpathmoveto{\pgfqpoint{1.439103in}{1.138495in}}%
\pgfpathlineto{\pgfqpoint{1.440707in}{1.140370in}}%
\pgfpathlineto{\pgfqpoint{1.442313in}{1.142494in}}%
\pgfpathlineto{\pgfqpoint{1.443921in}{1.144871in}}%
\pgfpathlineto{\pgfqpoint{1.445532in}{1.147506in}}%
\pgfpathlineto{\pgfqpoint{1.461864in}{1.143162in}}%
\pgfpathlineto{\pgfqpoint{1.477946in}{1.138559in}}%
\pgfpathlineto{\pgfqpoint{1.493763in}{1.133699in}}%
\pgfpathlineto{\pgfqpoint{1.509301in}{1.128586in}}%
\pgfpathlineto{\pgfqpoint{1.507297in}{1.126059in}}%
\pgfpathlineto{\pgfqpoint{1.505297in}{1.123791in}}%
\pgfpathlineto{\pgfqpoint{1.503300in}{1.121776in}}%
\pgfpathlineto{\pgfqpoint{1.501307in}{1.120010in}}%
\pgfpathlineto{\pgfqpoint{1.486151in}{1.125005in}}%
\pgfpathlineto{\pgfqpoint{1.470723in}{1.129753in}}%
\pgfpathlineto{\pgfqpoint{1.455035in}{1.134251in}}%
\pgfpathlineto{\pgfqpoint{1.439103in}{1.138495in}}%
\pgfpathclose%
\pgfusepath{fill}%
\end{pgfscope}%
\begin{pgfscope}%
\pgfpathrectangle{\pgfqpoint{0.041670in}{0.041670in}}{\pgfqpoint{2.216660in}{2.216660in}}%
\pgfusepath{clip}%
\pgfsetbuttcap%
\pgfsetroundjoin%
\definecolor{currentfill}{rgb}{0.268510,0.009605,0.335427}%
\pgfsetfillcolor{currentfill}%
\pgfsetlinewidth{0.000000pt}%
\definecolor{currentstroke}{rgb}{0.000000,0.000000,0.000000}%
\pgfsetstrokecolor{currentstroke}%
\pgfsetdash{}{0pt}%
\pgfpathmoveto{\pgfqpoint{0.920091in}{1.128315in}}%
\pgfpathlineto{\pgfqpoint{0.918416in}{1.128342in}}%
\pgfpathlineto{\pgfqpoint{0.916738in}{1.128582in}}%
\pgfpathlineto{\pgfqpoint{0.915059in}{1.129039in}}%
\pgfpathlineto{\pgfqpoint{0.913378in}{1.129717in}}%
\pgfpathlineto{\pgfqpoint{0.928939in}{1.133831in}}%
\pgfpathlineto{\pgfqpoint{0.944723in}{1.137694in}}%
\pgfpathlineto{\pgfqpoint{0.960716in}{1.141303in}}%
\pgfpathlineto{\pgfqpoint{0.976904in}{1.144654in}}%
\pgfpathlineto{\pgfqpoint{0.978182in}{1.143883in}}%
\pgfpathlineto{\pgfqpoint{0.979458in}{1.143333in}}%
\pgfpathlineto{\pgfqpoint{0.980733in}{1.143000in}}%
\pgfpathlineto{\pgfqpoint{0.982007in}{1.142880in}}%
\pgfpathlineto{\pgfqpoint{0.966228in}{1.139612in}}%
\pgfpathlineto{\pgfqpoint{0.950640in}{1.136093in}}%
\pgfpathlineto{\pgfqpoint{0.935257in}{1.132326in}}%
\pgfpathlineto{\pgfqpoint{0.920091in}{1.128315in}}%
\pgfpathclose%
\pgfusepath{fill}%
\end{pgfscope}%
\begin{pgfscope}%
\pgfpathrectangle{\pgfqpoint{0.041670in}{0.041670in}}{\pgfqpoint{2.216660in}{2.216660in}}%
\pgfusepath{clip}%
\pgfsetbuttcap%
\pgfsetroundjoin%
\definecolor{currentfill}{rgb}{0.276194,0.190074,0.493001}%
\pgfsetfillcolor{currentfill}%
\pgfsetlinewidth{0.000000pt}%
\definecolor{currentstroke}{rgb}{0.000000,0.000000,0.000000}%
\pgfsetstrokecolor{currentstroke}%
\pgfsetdash{}{0pt}%
\pgfpathmoveto{\pgfqpoint{0.794284in}{1.235859in}}%
\pgfpathlineto{\pgfqpoint{0.792081in}{1.244976in}}%
\pgfpathlineto{\pgfqpoint{0.789870in}{1.254470in}}%
\pgfpathlineto{\pgfqpoint{0.787652in}{1.264348in}}%
\pgfpathlineto{\pgfqpoint{0.785425in}{1.274615in}}%
\pgfpathlineto{\pgfqpoint{0.803046in}{1.280619in}}%
\pgfpathlineto{\pgfqpoint{0.820996in}{1.286341in}}%
\pgfpathlineto{\pgfqpoint{0.839260in}{1.291777in}}%
\pgfpathlineto{\pgfqpoint{0.857822in}{1.296923in}}%
\pgfpathlineto{\pgfqpoint{0.859634in}{1.286563in}}%
\pgfpathlineto{\pgfqpoint{0.861439in}{1.276592in}}%
\pgfpathlineto{\pgfqpoint{0.863239in}{1.267004in}}%
\pgfpathlineto{\pgfqpoint{0.865032in}{1.257791in}}%
\pgfpathlineto{\pgfqpoint{0.846892in}{1.252731in}}%
\pgfpathlineto{\pgfqpoint{0.829043in}{1.247387in}}%
\pgfpathlineto{\pgfqpoint{0.811501in}{1.241761in}}%
\pgfpathlineto{\pgfqpoint{0.794284in}{1.235859in}}%
\pgfpathclose%
\pgfusepath{fill}%
\end{pgfscope}%
\begin{pgfscope}%
\pgfpathrectangle{\pgfqpoint{0.041670in}{0.041670in}}{\pgfqpoint{2.216660in}{2.216660in}}%
\pgfusepath{clip}%
\pgfsetbuttcap%
\pgfsetroundjoin%
\definecolor{currentfill}{rgb}{0.260571,0.246922,0.522828}%
\pgfsetfillcolor{currentfill}%
\pgfsetlinewidth{0.000000pt}%
\definecolor{currentstroke}{rgb}{0.000000,0.000000,0.000000}%
\pgfsetstrokecolor{currentstroke}%
\pgfsetdash{}{0pt}%
\pgfpathmoveto{\pgfqpoint{1.485354in}{1.301250in}}%
\pgfpathlineto{\pgfqpoint{1.487077in}{1.312022in}}%
\pgfpathlineto{\pgfqpoint{1.488807in}{1.323195in}}%
\pgfpathlineto{\pgfqpoint{1.490544in}{1.334775in}}%
\pgfpathlineto{\pgfqpoint{1.492287in}{1.346770in}}%
\pgfpathlineto{\pgfqpoint{1.511531in}{1.341807in}}%
\pgfpathlineto{\pgfqpoint{1.530487in}{1.336547in}}%
\pgfpathlineto{\pgfqpoint{1.549138in}{1.330991in}}%
\pgfpathlineto{\pgfqpoint{1.567468in}{1.325146in}}%
\pgfpathlineto{\pgfqpoint{1.565298in}{1.313234in}}%
\pgfpathlineto{\pgfqpoint{1.563136in}{1.301738in}}%
\pgfpathlineto{\pgfqpoint{1.560983in}{1.290651in}}%
\pgfpathlineto{\pgfqpoint{1.558839in}{1.279965in}}%
\pgfpathlineto{\pgfqpoint{1.540924in}{1.285719in}}%
\pgfpathlineto{\pgfqpoint{1.522694in}{1.291187in}}%
\pgfpathlineto{\pgfqpoint{1.504165in}{1.296365in}}%
\pgfpathlineto{\pgfqpoint{1.485354in}{1.301250in}}%
\pgfpathclose%
\pgfusepath{fill}%
\end{pgfscope}%
\begin{pgfscope}%
\pgfpathrectangle{\pgfqpoint{0.041670in}{0.041670in}}{\pgfqpoint{2.216660in}{2.216660in}}%
\pgfusepath{clip}%
\pgfsetbuttcap%
\pgfsetroundjoin%
\definecolor{currentfill}{rgb}{0.201239,0.383670,0.554294}%
\pgfsetfillcolor{currentfill}%
\pgfsetlinewidth{0.000000pt}%
\definecolor{currentstroke}{rgb}{0.000000,0.000000,0.000000}%
\pgfsetstrokecolor{currentstroke}%
\pgfsetdash{}{0pt}%
\pgfpathmoveto{\pgfqpoint{0.843064in}{1.394565in}}%
\pgfpathlineto{\pgfqpoint{0.841184in}{1.408719in}}%
\pgfpathlineto{\pgfqpoint{0.839295in}{1.423328in}}%
\pgfpathlineto{\pgfqpoint{0.837397in}{1.438402in}}%
\pgfpathlineto{\pgfqpoint{0.835490in}{1.453946in}}%
\pgfpathlineto{\pgfqpoint{0.855654in}{1.459015in}}%
\pgfpathlineto{\pgfqpoint{0.876095in}{1.463772in}}%
\pgfpathlineto{\pgfqpoint{0.896797in}{1.468214in}}%
\pgfpathlineto{\pgfqpoint{0.917741in}{1.472338in}}%
\pgfpathlineto{\pgfqpoint{0.919189in}{1.456735in}}%
\pgfpathlineto{\pgfqpoint{0.920630in}{1.441601in}}%
\pgfpathlineto{\pgfqpoint{0.922064in}{1.426930in}}%
\pgfpathlineto{\pgfqpoint{0.923491in}{1.412715in}}%
\pgfpathlineto{\pgfqpoint{0.903010in}{1.408645in}}%
\pgfpathlineto{\pgfqpoint{0.882767in}{1.404261in}}%
\pgfpathlineto{\pgfqpoint{0.862779in}{1.399567in}}%
\pgfpathlineto{\pgfqpoint{0.843064in}{1.394565in}}%
\pgfpathclose%
\pgfusepath{fill}%
\end{pgfscope}%
\begin{pgfscope}%
\pgfpathrectangle{\pgfqpoint{0.041670in}{0.041670in}}{\pgfqpoint{2.216660in}{2.216660in}}%
\pgfusepath{clip}%
\pgfsetbuttcap%
\pgfsetroundjoin%
\definecolor{currentfill}{rgb}{0.267004,0.004874,0.329415}%
\pgfsetfillcolor{currentfill}%
\pgfsetlinewidth{0.000000pt}%
\definecolor{currentstroke}{rgb}{0.000000,0.000000,0.000000}%
\pgfsetstrokecolor{currentstroke}%
\pgfsetdash{}{0pt}%
\pgfpathmoveto{\pgfqpoint{0.845373in}{1.115366in}}%
\pgfpathlineto{\pgfqpoint{0.843295in}{1.117104in}}%
\pgfpathlineto{\pgfqpoint{0.841215in}{1.119091in}}%
\pgfpathlineto{\pgfqpoint{0.839131in}{1.121332in}}%
\pgfpathlineto{\pgfqpoint{0.837043in}{1.123832in}}%
\pgfpathlineto{\pgfqpoint{0.852321in}{1.129166in}}%
\pgfpathlineto{\pgfqpoint{0.867891in}{1.134251in}}%
\pgfpathlineto{\pgfqpoint{0.883738in}{1.139083in}}%
\pgfpathlineto{\pgfqpoint{0.899848in}{1.143658in}}%
\pgfpathlineto{\pgfqpoint{0.901549in}{1.141045in}}%
\pgfpathlineto{\pgfqpoint{0.903246in}{1.138690in}}%
\pgfpathlineto{\pgfqpoint{0.904941in}{1.136588in}}%
\pgfpathlineto{\pgfqpoint{0.906633in}{1.134735in}}%
\pgfpathlineto{\pgfqpoint{0.890917in}{1.130266in}}%
\pgfpathlineto{\pgfqpoint{0.875459in}{1.125545in}}%
\pgfpathlineto{\pgfqpoint{0.860273in}{1.120577in}}%
\pgfpathlineto{\pgfqpoint{0.845373in}{1.115366in}}%
\pgfpathclose%
\pgfusepath{fill}%
\end{pgfscope}%
\begin{pgfscope}%
\pgfpathrectangle{\pgfqpoint{0.041670in}{0.041670in}}{\pgfqpoint{2.216660in}{2.216660in}}%
\pgfusepath{clip}%
\pgfsetbuttcap%
\pgfsetroundjoin%
\definecolor{currentfill}{rgb}{0.267004,0.004874,0.329415}%
\pgfsetfillcolor{currentfill}%
\pgfsetlinewidth{0.000000pt}%
\definecolor{currentstroke}{rgb}{0.000000,0.000000,0.000000}%
\pgfsetstrokecolor{currentstroke}%
\pgfsetdash{}{0pt}%
\pgfpathmoveto{\pgfqpoint{1.432711in}{1.133387in}}%
\pgfpathlineto{\pgfqpoint{1.434306in}{1.134314in}}%
\pgfpathlineto{\pgfqpoint{1.435903in}{1.135471in}}%
\pgfpathlineto{\pgfqpoint{1.437502in}{1.136864in}}%
\pgfpathlineto{\pgfqpoint{1.439103in}{1.138495in}}%
\pgfpathlineto{\pgfqpoint{1.455035in}{1.134251in}}%
\pgfpathlineto{\pgfqpoint{1.470723in}{1.129753in}}%
\pgfpathlineto{\pgfqpoint{1.486151in}{1.125005in}}%
\pgfpathlineto{\pgfqpoint{1.501307in}{1.120010in}}%
\pgfpathlineto{\pgfqpoint{1.499316in}{1.118489in}}%
\pgfpathlineto{\pgfqpoint{1.497328in}{1.117207in}}%
\pgfpathlineto{\pgfqpoint{1.495343in}{1.116161in}}%
\pgfpathlineto{\pgfqpoint{1.493360in}{1.115346in}}%
\pgfpathlineto{\pgfqpoint{1.478585in}{1.120221in}}%
\pgfpathlineto{\pgfqpoint{1.463542in}{1.124855in}}%
\pgfpathlineto{\pgfqpoint{1.448246in}{1.129245in}}%
\pgfpathlineto{\pgfqpoint{1.432711in}{1.133387in}}%
\pgfpathclose%
\pgfusepath{fill}%
\end{pgfscope}%
\begin{pgfscope}%
\pgfpathrectangle{\pgfqpoint{0.041670in}{0.041670in}}{\pgfqpoint{2.216660in}{2.216660in}}%
\pgfusepath{clip}%
\pgfsetbuttcap%
\pgfsetroundjoin%
\definecolor{currentfill}{rgb}{0.274952,0.037752,0.364543}%
\pgfsetfillcolor{currentfill}%
\pgfsetlinewidth{0.000000pt}%
\definecolor{currentstroke}{rgb}{0.000000,0.000000,0.000000}%
\pgfsetstrokecolor{currentstroke}%
\pgfsetdash{}{0pt}%
\pgfpathmoveto{\pgfqpoint{1.308086in}{1.158806in}}%
\pgfpathlineto{\pgfqpoint{1.308948in}{1.157445in}}%
\pgfpathlineto{\pgfqpoint{1.309812in}{1.156263in}}%
\pgfpathlineto{\pgfqpoint{1.310676in}{1.155264in}}%
\pgfpathlineto{\pgfqpoint{1.311540in}{1.154452in}}%
\pgfpathlineto{\pgfqpoint{1.327500in}{1.152240in}}%
\pgfpathlineto{\pgfqpoint{1.343328in}{1.149776in}}%
\pgfpathlineto{\pgfqpoint{1.359012in}{1.147060in}}%
\pgfpathlineto{\pgfqpoint{1.357834in}{1.147921in}}%
\pgfpathlineto{\pgfqpoint{1.356658in}{1.148970in}}%
\pgfpathlineto{\pgfqpoint{1.355482in}{1.150202in}}%
\pgfpathlineto{\pgfqpoint{1.354306in}{1.151613in}}%
\pgfpathlineto{\pgfqpoint{1.339036in}{1.154255in}}%
\pgfpathlineto{\pgfqpoint{1.323625in}{1.156654in}}%
\pgfpathlineto{\pgfqpoint{1.308086in}{1.158806in}}%
\pgfpathclose%
\pgfusepath{fill}%
\end{pgfscope}%
\begin{pgfscope}%
\pgfpathrectangle{\pgfqpoint{0.041670in}{0.041670in}}{\pgfqpoint{2.216660in}{2.216660in}}%
\pgfusepath{clip}%
\pgfsetbuttcap%
\pgfsetroundjoin%
\definecolor{currentfill}{rgb}{0.271305,0.019942,0.347269}%
\pgfsetfillcolor{currentfill}%
\pgfsetlinewidth{0.000000pt}%
\definecolor{currentstroke}{rgb}{0.000000,0.000000,0.000000}%
\pgfsetstrokecolor{currentstroke}%
\pgfsetdash{}{0pt}%
\pgfpathmoveto{\pgfqpoint{1.359012in}{1.147060in}}%
\pgfpathlineto{\pgfqpoint{1.360190in}{1.146391in}}%
\pgfpathlineto{\pgfqpoint{1.361369in}{1.145917in}}%
\pgfpathlineto{\pgfqpoint{1.362549in}{1.145642in}}%
\pgfpathlineto{\pgfqpoint{1.363730in}{1.145572in}}%
\pgfpathlineto{\pgfqpoint{1.379665in}{1.142529in}}%
\pgfpathlineto{\pgfqpoint{1.395423in}{1.139233in}}%
\pgfpathlineto{\pgfqpoint{1.410989in}{1.135687in}}%
\pgfpathlineto{\pgfqpoint{1.426349in}{1.131893in}}%
\pgfpathlineto{\pgfqpoint{1.424763in}{1.132051in}}%
\pgfpathlineto{\pgfqpoint{1.423178in}{1.132414in}}%
\pgfpathlineto{\pgfqpoint{1.421594in}{1.132977in}}%
\pgfpathlineto{\pgfqpoint{1.420011in}{1.133736in}}%
\pgfpathlineto{\pgfqpoint{1.405049in}{1.137431in}}%
\pgfpathlineto{\pgfqpoint{1.389886in}{1.140886in}}%
\pgfpathlineto{\pgfqpoint{1.374536in}{1.144096in}}%
\pgfpathlineto{\pgfqpoint{1.359012in}{1.147060in}}%
\pgfpathclose%
\pgfusepath{fill}%
\end{pgfscope}%
\begin{pgfscope}%
\pgfpathrectangle{\pgfqpoint{0.041670in}{0.041670in}}{\pgfqpoint{2.216660in}{2.216660in}}%
\pgfusepath{clip}%
\pgfsetbuttcap%
\pgfsetroundjoin%
\definecolor{currentfill}{rgb}{0.260571,0.246922,0.522828}%
\pgfsetfillcolor{currentfill}%
\pgfsetlinewidth{0.000000pt}%
\definecolor{currentstroke}{rgb}{0.000000,0.000000,0.000000}%
\pgfsetstrokecolor{currentstroke}%
\pgfsetdash{}{0pt}%
\pgfpathmoveto{\pgfqpoint{0.785425in}{1.274615in}}%
\pgfpathlineto{\pgfqpoint{0.783190in}{1.285278in}}%
\pgfpathlineto{\pgfqpoint{0.780946in}{1.296344in}}%
\pgfpathlineto{\pgfqpoint{0.778694in}{1.307819in}}%
\pgfpathlineto{\pgfqpoint{0.776432in}{1.319710in}}%
\pgfpathlineto{\pgfqpoint{0.794462in}{1.325810in}}%
\pgfpathlineto{\pgfqpoint{0.812828in}{1.331623in}}%
\pgfpathlineto{\pgfqpoint{0.831515in}{1.337146in}}%
\pgfpathlineto{\pgfqpoint{0.850503in}{1.342373in}}%
\pgfpathlineto{\pgfqpoint{0.852344in}{1.330396in}}%
\pgfpathlineto{\pgfqpoint{0.854177in}{1.318832in}}%
\pgfpathlineto{\pgfqpoint{0.856003in}{1.307677in}}%
\pgfpathlineto{\pgfqpoint{0.857822in}{1.296923in}}%
\pgfpathlineto{\pgfqpoint{0.839260in}{1.291777in}}%
\pgfpathlineto{\pgfqpoint{0.820996in}{1.286341in}}%
\pgfpathlineto{\pgfqpoint{0.803046in}{1.280619in}}%
\pgfpathlineto{\pgfqpoint{0.785425in}{1.274615in}}%
\pgfpathclose%
\pgfusepath{fill}%
\end{pgfscope}%
\begin{pgfscope}%
\pgfpathrectangle{\pgfqpoint{0.041670in}{0.041670in}}{\pgfqpoint{2.216660in}{2.216660in}}%
\pgfusepath{clip}%
\pgfsetbuttcap%
\pgfsetroundjoin%
\definecolor{currentfill}{rgb}{0.279566,0.067836,0.391917}%
\pgfsetfillcolor{currentfill}%
\pgfsetlinewidth{0.000000pt}%
\definecolor{currentstroke}{rgb}{0.000000,0.000000,0.000000}%
\pgfsetstrokecolor{currentstroke}%
\pgfsetdash{}{0pt}%
\pgfpathmoveto{\pgfqpoint{1.118454in}{1.172023in}}%
\pgfpathlineto{\pgfqpoint{1.118029in}{1.170028in}}%
\pgfpathlineto{\pgfqpoint{1.117604in}{1.168197in}}%
\pgfpathlineto{\pgfqpoint{1.117179in}{1.166532in}}%
\pgfpathlineto{\pgfqpoint{1.116754in}{1.165039in}}%
\pgfpathlineto{\pgfqpoint{1.132729in}{1.165913in}}%
\pgfpathlineto{\pgfqpoint{1.148745in}{1.166533in}}%
\pgfpathlineto{\pgfqpoint{1.164789in}{1.166900in}}%
\pgfpathlineto{\pgfqpoint{1.180847in}{1.167013in}}%
\pgfpathlineto{\pgfqpoint{1.180841in}{1.168492in}}%
\pgfpathlineto{\pgfqpoint{1.180835in}{1.170143in}}%
\pgfpathlineto{\pgfqpoint{1.180829in}{1.171961in}}%
\pgfpathlineto{\pgfqpoint{1.180823in}{1.173942in}}%
\pgfpathlineto{\pgfqpoint{1.165197in}{1.173832in}}%
\pgfpathlineto{\pgfqpoint{1.149584in}{1.173475in}}%
\pgfpathlineto{\pgfqpoint{1.133998in}{1.172872in}}%
\pgfpathlineto{\pgfqpoint{1.118454in}{1.172023in}}%
\pgfpathclose%
\pgfusepath{fill}%
\end{pgfscope}%
\begin{pgfscope}%
\pgfpathrectangle{\pgfqpoint{0.041670in}{0.041670in}}{\pgfqpoint{2.216660in}{2.216660in}}%
\pgfusepath{clip}%
\pgfsetbuttcap%
\pgfsetroundjoin%
\definecolor{currentfill}{rgb}{0.279566,0.067836,0.391917}%
\pgfsetfillcolor{currentfill}%
\pgfsetlinewidth{0.000000pt}%
\definecolor{currentstroke}{rgb}{0.000000,0.000000,0.000000}%
\pgfsetstrokecolor{currentstroke}%
\pgfsetdash{}{0pt}%
\pgfpathmoveto{\pgfqpoint{1.180823in}{1.173942in}}%
\pgfpathlineto{\pgfqpoint{1.180829in}{1.171961in}}%
\pgfpathlineto{\pgfqpoint{1.180835in}{1.170143in}}%
\pgfpathlineto{\pgfqpoint{1.180841in}{1.168492in}}%
\pgfpathlineto{\pgfqpoint{1.180847in}{1.167013in}}%
\pgfpathlineto{\pgfqpoint{1.196904in}{1.166872in}}%
\pgfpathlineto{\pgfqpoint{1.212946in}{1.166477in}}%
\pgfpathlineto{\pgfqpoint{1.228958in}{1.165828in}}%
\pgfpathlineto{\pgfqpoint{1.244927in}{1.164927in}}%
\pgfpathlineto{\pgfqpoint{1.244490in}{1.166420in}}%
\pgfpathlineto{\pgfqpoint{1.244054in}{1.168086in}}%
\pgfpathlineto{\pgfqpoint{1.243617in}{1.169918in}}%
\pgfpathlineto{\pgfqpoint{1.243180in}{1.171913in}}%
\pgfpathlineto{\pgfqpoint{1.227641in}{1.172790in}}%
\pgfpathlineto{\pgfqpoint{1.212059in}{1.173420in}}%
\pgfpathlineto{\pgfqpoint{1.196448in}{1.173804in}}%
\pgfpathlineto{\pgfqpoint{1.180823in}{1.173942in}}%
\pgfpathclose%
\pgfusepath{fill}%
\end{pgfscope}%
\begin{pgfscope}%
\pgfpathrectangle{\pgfqpoint{0.041670in}{0.041670in}}{\pgfqpoint{2.216660in}{2.216660in}}%
\pgfusepath{clip}%
\pgfsetbuttcap%
\pgfsetroundjoin%
\definecolor{currentfill}{rgb}{0.274952,0.037752,0.364543}%
\pgfsetfillcolor{currentfill}%
\pgfsetlinewidth{0.000000pt}%
\definecolor{currentstroke}{rgb}{0.000000,0.000000,0.000000}%
\pgfsetstrokecolor{currentstroke}%
\pgfsetdash{}{0pt}%
\pgfpathmoveto{\pgfqpoint{0.992161in}{1.149061in}}%
\pgfpathlineto{\pgfqpoint{0.990894in}{1.147632in}}%
\pgfpathlineto{\pgfqpoint{0.989627in}{1.146383in}}%
\pgfpathlineto{\pgfqpoint{0.988359in}{1.145316in}}%
\pgfpathlineto{\pgfqpoint{0.987091in}{1.144438in}}%
\pgfpathlineto{\pgfqpoint{1.002633in}{1.147374in}}%
\pgfpathlineto{\pgfqpoint{1.018333in}{1.150062in}}%
\pgfpathlineto{\pgfqpoint{1.034177in}{1.152499in}}%
\pgfpathlineto{\pgfqpoint{1.050151in}{1.154682in}}%
\pgfpathlineto{\pgfqpoint{1.051003in}{1.155493in}}%
\pgfpathlineto{\pgfqpoint{1.051855in}{1.156490in}}%
\pgfpathlineto{\pgfqpoint{1.052707in}{1.157671in}}%
\pgfpathlineto{\pgfqpoint{1.053558in}{1.159030in}}%
\pgfpathlineto{\pgfqpoint{1.038005in}{1.156905in}}%
\pgfpathlineto{\pgfqpoint{1.022579in}{1.154534in}}%
\pgfpathlineto{\pgfqpoint{1.007293in}{1.151919in}}%
\pgfpathlineto{\pgfqpoint{0.992161in}{1.149061in}}%
\pgfpathclose%
\pgfusepath{fill}%
\end{pgfscope}%
\begin{pgfscope}%
\pgfpathrectangle{\pgfqpoint{0.041670in}{0.041670in}}{\pgfqpoint{2.216660in}{2.216660in}}%
\pgfusepath{clip}%
\pgfsetbuttcap%
\pgfsetroundjoin%
\definecolor{currentfill}{rgb}{0.233603,0.313828,0.543914}%
\pgfsetfillcolor{currentfill}%
\pgfsetlinewidth{0.000000pt}%
\definecolor{currentstroke}{rgb}{0.000000,0.000000,0.000000}%
\pgfsetstrokecolor{currentstroke}%
\pgfsetdash{}{0pt}%
\pgfpathmoveto{\pgfqpoint{1.492287in}{1.346770in}}%
\pgfpathlineto{\pgfqpoint{1.494038in}{1.359185in}}%
\pgfpathlineto{\pgfqpoint{1.495796in}{1.372028in}}%
\pgfpathlineto{\pgfqpoint{1.497562in}{1.385306in}}%
\pgfpathlineto{\pgfqpoint{1.499335in}{1.399026in}}%
\pgfpathlineto{\pgfqpoint{1.519019in}{1.393991in}}%
\pgfpathlineto{\pgfqpoint{1.538409in}{1.388653in}}%
\pgfpathlineto{\pgfqpoint{1.557489in}{1.383016in}}%
\pgfpathlineto{\pgfqpoint{1.576241in}{1.377084in}}%
\pgfpathlineto{\pgfqpoint{1.574034in}{1.363442in}}%
\pgfpathlineto{\pgfqpoint{1.571836in}{1.350243in}}%
\pgfpathlineto{\pgfqpoint{1.569647in}{1.337480in}}%
\pgfpathlineto{\pgfqpoint{1.567468in}{1.325146in}}%
\pgfpathlineto{\pgfqpoint{1.549138in}{1.330991in}}%
\pgfpathlineto{\pgfqpoint{1.530487in}{1.336547in}}%
\pgfpathlineto{\pgfqpoint{1.511531in}{1.341807in}}%
\pgfpathlineto{\pgfqpoint{1.492287in}{1.346770in}}%
\pgfpathclose%
\pgfusepath{fill}%
\end{pgfscope}%
\begin{pgfscope}%
\pgfpathrectangle{\pgfqpoint{0.041670in}{0.041670in}}{\pgfqpoint{2.216660in}{2.216660in}}%
\pgfusepath{clip}%
\pgfsetbuttcap%
\pgfsetroundjoin%
\definecolor{currentfill}{rgb}{0.271305,0.019942,0.347269}%
\pgfsetfillcolor{currentfill}%
\pgfsetlinewidth{0.000000pt}%
\definecolor{currentstroke}{rgb}{0.000000,0.000000,0.000000}%
\pgfsetstrokecolor{currentstroke}%
\pgfsetdash{}{0pt}%
\pgfpathmoveto{\pgfqpoint{0.926779in}{1.130251in}}%
\pgfpathlineto{\pgfqpoint{0.925109in}{1.129469in}}%
\pgfpathlineto{\pgfqpoint{0.923438in}{1.128883in}}%
\pgfpathlineto{\pgfqpoint{0.921765in}{1.128496in}}%
\pgfpathlineto{\pgfqpoint{0.920091in}{1.128315in}}%
\pgfpathlineto{\pgfqpoint{0.935257in}{1.132326in}}%
\pgfpathlineto{\pgfqpoint{0.950640in}{1.136093in}}%
\pgfpathlineto{\pgfqpoint{0.966228in}{1.139612in}}%
\pgfpathlineto{\pgfqpoint{0.982007in}{1.142880in}}%
\pgfpathlineto{\pgfqpoint{0.983279in}{1.142967in}}%
\pgfpathlineto{\pgfqpoint{0.984551in}{1.143259in}}%
\pgfpathlineto{\pgfqpoint{0.985821in}{1.143751in}}%
\pgfpathlineto{\pgfqpoint{0.987091in}{1.144438in}}%
\pgfpathlineto{\pgfqpoint{0.971720in}{1.141254in}}%
\pgfpathlineto{\pgfqpoint{0.956535in}{1.137827in}}%
\pgfpathlineto{\pgfqpoint{0.941550in}{1.134158in}}%
\pgfpathlineto{\pgfqpoint{0.926779in}{1.130251in}}%
\pgfpathclose%
\pgfusepath{fill}%
\end{pgfscope}%
\begin{pgfscope}%
\pgfpathrectangle{\pgfqpoint{0.041670in}{0.041670in}}{\pgfqpoint{2.216660in}{2.216660in}}%
\pgfusepath{clip}%
\pgfsetbuttcap%
\pgfsetroundjoin%
\definecolor{currentfill}{rgb}{0.172719,0.448791,0.557885}%
\pgfsetfillcolor{currentfill}%
\pgfsetlinewidth{0.000000pt}%
\definecolor{currentstroke}{rgb}{0.000000,0.000000,0.000000}%
\pgfsetstrokecolor{currentstroke}%
\pgfsetdash{}{0pt}%
\pgfpathmoveto{\pgfqpoint{1.423363in}{1.475735in}}%
\pgfpathlineto{\pgfqpoint{1.424713in}{1.491827in}}%
\pgfpathlineto{\pgfqpoint{1.426069in}{1.508404in}}%
\pgfpathlineto{\pgfqpoint{1.427432in}{1.525474in}}%
\pgfpathlineto{\pgfqpoint{1.448932in}{1.521600in}}%
\pgfpathlineto{\pgfqpoint{1.470203in}{1.517402in}}%
\pgfpathlineto{\pgfqpoint{1.491226in}{1.512883in}}%
\pgfpathlineto{\pgfqpoint{1.511984in}{1.508048in}}%
\pgfpathlineto{\pgfqpoint{1.510150in}{1.491029in}}%
\pgfpathlineto{\pgfqpoint{1.508326in}{1.474505in}}%
\pgfpathlineto{\pgfqpoint{1.506511in}{1.458467in}}%
\pgfpathlineto{\pgfqpoint{1.486099in}{1.463259in}}%
\pgfpathlineto{\pgfqpoint{1.465425in}{1.467736in}}%
\pgfpathlineto{\pgfqpoint{1.444507in}{1.471896in}}%
\pgfpathlineto{\pgfqpoint{1.423363in}{1.475735in}}%
\pgfpathclose%
\pgfusepath{fill}%
\end{pgfscope}%
\begin{pgfscope}%
\pgfpathrectangle{\pgfqpoint{0.041670in}{0.041670in}}{\pgfqpoint{2.216660in}{2.216660in}}%
\pgfusepath{clip}%
\pgfsetbuttcap%
\pgfsetroundjoin%
\definecolor{currentfill}{rgb}{0.267004,0.004874,0.329415}%
\pgfsetfillcolor{currentfill}%
\pgfsetlinewidth{0.000000pt}%
\definecolor{currentstroke}{rgb}{0.000000,0.000000,0.000000}%
\pgfsetstrokecolor{currentstroke}%
\pgfsetdash{}{0pt}%
\pgfpathmoveto{\pgfqpoint{0.853652in}{1.110814in}}%
\pgfpathlineto{\pgfqpoint{0.851586in}{1.111601in}}%
\pgfpathlineto{\pgfqpoint{0.849518in}{1.112619in}}%
\pgfpathlineto{\pgfqpoint{0.847447in}{1.113873in}}%
\pgfpathlineto{\pgfqpoint{0.845373in}{1.115366in}}%
\pgfpathlineto{\pgfqpoint{0.860273in}{1.120577in}}%
\pgfpathlineto{\pgfqpoint{0.875459in}{1.125545in}}%
\pgfpathlineto{\pgfqpoint{0.890917in}{1.130266in}}%
\pgfpathlineto{\pgfqpoint{0.906633in}{1.134735in}}%
\pgfpathlineto{\pgfqpoint{0.908323in}{1.133126in}}%
\pgfpathlineto{\pgfqpoint{0.910010in}{1.131757in}}%
\pgfpathlineto{\pgfqpoint{0.911695in}{1.130622in}}%
\pgfpathlineto{\pgfqpoint{0.913378in}{1.129717in}}%
\pgfpathlineto{\pgfqpoint{0.898055in}{1.125355in}}%
\pgfpathlineto{\pgfqpoint{0.882984in}{1.120747in}}%
\pgfpathlineto{\pgfqpoint{0.868178in}{1.115899in}}%
\pgfpathlineto{\pgfqpoint{0.853652in}{1.110814in}}%
\pgfpathclose%
\pgfusepath{fill}%
\end{pgfscope}%
\begin{pgfscope}%
\pgfpathrectangle{\pgfqpoint{0.041670in}{0.041670in}}{\pgfqpoint{2.216660in}{2.216660in}}%
\pgfusepath{clip}%
\pgfsetbuttcap%
\pgfsetroundjoin%
\definecolor{currentfill}{rgb}{0.279566,0.067836,0.391917}%
\pgfsetfillcolor{currentfill}%
\pgfsetlinewidth{0.000000pt}%
\definecolor{currentstroke}{rgb}{0.000000,0.000000,0.000000}%
\pgfsetstrokecolor{currentstroke}%
\pgfsetdash{}{0pt}%
\pgfpathmoveto{\pgfqpoint{1.056960in}{1.166182in}}%
\pgfpathlineto{\pgfqpoint{1.056110in}{1.164145in}}%
\pgfpathlineto{\pgfqpoint{1.055260in}{1.162271in}}%
\pgfpathlineto{\pgfqpoint{1.054409in}{1.160565in}}%
\pgfpathlineto{\pgfqpoint{1.053558in}{1.159030in}}%
\pgfpathlineto{\pgfqpoint{1.069223in}{1.160908in}}%
\pgfpathlineto{\pgfqpoint{1.084987in}{1.162536in}}%
\pgfpathlineto{\pgfqpoint{1.100836in}{1.163913in}}%
\pgfpathlineto{\pgfqpoint{1.116754in}{1.165039in}}%
\pgfpathlineto{\pgfqpoint{1.117179in}{1.166532in}}%
\pgfpathlineto{\pgfqpoint{1.117604in}{1.168197in}}%
\pgfpathlineto{\pgfqpoint{1.118029in}{1.170028in}}%
\pgfpathlineto{\pgfqpoint{1.118454in}{1.172023in}}%
\pgfpathlineto{\pgfqpoint{1.102963in}{1.170928in}}%
\pgfpathlineto{\pgfqpoint{1.087542in}{1.169589in}}%
\pgfpathlineto{\pgfqpoint{1.072203in}{1.168007in}}%
\pgfpathlineto{\pgfqpoint{1.056960in}{1.166182in}}%
\pgfpathclose%
\pgfusepath{fill}%
\end{pgfscope}%
\begin{pgfscope}%
\pgfpathrectangle{\pgfqpoint{0.041670in}{0.041670in}}{\pgfqpoint{2.216660in}{2.216660in}}%
\pgfusepath{clip}%
\pgfsetbuttcap%
\pgfsetroundjoin%
\definecolor{currentfill}{rgb}{0.279566,0.067836,0.391917}%
\pgfsetfillcolor{currentfill}%
\pgfsetlinewidth{0.000000pt}%
\definecolor{currentstroke}{rgb}{0.000000,0.000000,0.000000}%
\pgfsetstrokecolor{currentstroke}%
\pgfsetdash{}{0pt}%
\pgfpathmoveto{\pgfqpoint{1.243180in}{1.171913in}}%
\pgfpathlineto{\pgfqpoint{1.243617in}{1.169918in}}%
\pgfpathlineto{\pgfqpoint{1.244054in}{1.168086in}}%
\pgfpathlineto{\pgfqpoint{1.244490in}{1.166420in}}%
\pgfpathlineto{\pgfqpoint{1.244927in}{1.164927in}}%
\pgfpathlineto{\pgfqpoint{1.260839in}{1.163773in}}%
\pgfpathlineto{\pgfqpoint{1.276678in}{1.162367in}}%
\pgfpathlineto{\pgfqpoint{1.292432in}{1.160711in}}%
\pgfpathlineto{\pgfqpoint{1.308086in}{1.158806in}}%
\pgfpathlineto{\pgfqpoint{1.307223in}{1.160343in}}%
\pgfpathlineto{\pgfqpoint{1.306361in}{1.162050in}}%
\pgfpathlineto{\pgfqpoint{1.305499in}{1.163926in}}%
\pgfpathlineto{\pgfqpoint{1.304637in}{1.165964in}}%
\pgfpathlineto{\pgfqpoint{1.289406in}{1.167816in}}%
\pgfpathlineto{\pgfqpoint{1.274076in}{1.169425in}}%
\pgfpathlineto{\pgfqpoint{1.258663in}{1.170792in}}%
\pgfpathlineto{\pgfqpoint{1.243180in}{1.171913in}}%
\pgfpathclose%
\pgfusepath{fill}%
\end{pgfscope}%
\begin{pgfscope}%
\pgfpathrectangle{\pgfqpoint{0.041670in}{0.041670in}}{\pgfqpoint{2.216660in}{2.216660in}}%
\pgfusepath{clip}%
\pgfsetbuttcap%
\pgfsetroundjoin%
\definecolor{currentfill}{rgb}{0.277941,0.056324,0.381191}%
\pgfsetfillcolor{currentfill}%
\pgfsetlinewidth{0.000000pt}%
\definecolor{currentstroke}{rgb}{0.000000,0.000000,0.000000}%
\pgfsetstrokecolor{currentstroke}%
\pgfsetdash{}{0pt}%
\pgfpathmoveto{\pgfqpoint{1.525470in}{1.158684in}}%
\pgfpathlineto{\pgfqpoint{1.527512in}{1.163757in}}%
\pgfpathlineto{\pgfqpoint{1.529558in}{1.169138in}}%
\pgfpathlineto{\pgfqpoint{1.531609in}{1.174833in}}%
\pgfpathlineto{\pgfqpoint{1.533666in}{1.180848in}}%
\pgfpathlineto{\pgfqpoint{1.550059in}{1.175132in}}%
\pgfpathlineto{\pgfqpoint{1.566124in}{1.169155in}}%
\pgfpathlineto{\pgfqpoint{1.581846in}{1.162922in}}%
\pgfpathlineto{\pgfqpoint{1.597209in}{1.156437in}}%
\pgfpathlineto{\pgfqpoint{1.594774in}{1.150544in}}%
\pgfpathlineto{\pgfqpoint{1.592346in}{1.144971in}}%
\pgfpathlineto{\pgfqpoint{1.589924in}{1.139713in}}%
\pgfpathlineto{\pgfqpoint{1.587508in}{1.134764in}}%
\pgfpathlineto{\pgfqpoint{1.572510in}{1.141119in}}%
\pgfpathlineto{\pgfqpoint{1.557161in}{1.147227in}}%
\pgfpathlineto{\pgfqpoint{1.541477in}{1.153083in}}%
\pgfpathlineto{\pgfqpoint{1.525470in}{1.158684in}}%
\pgfpathclose%
\pgfusepath{fill}%
\end{pgfscope}%
\begin{pgfscope}%
\pgfpathrectangle{\pgfqpoint{0.041670in}{0.041670in}}{\pgfqpoint{2.216660in}{2.216660in}}%
\pgfusepath{clip}%
\pgfsetbuttcap%
\pgfsetroundjoin%
\definecolor{currentfill}{rgb}{0.282327,0.094955,0.417331}%
\pgfsetfillcolor{currentfill}%
\pgfsetlinewidth{0.000000pt}%
\definecolor{currentstroke}{rgb}{0.000000,0.000000,0.000000}%
\pgfsetstrokecolor{currentstroke}%
\pgfsetdash{}{0pt}%
\pgfpathmoveto{\pgfqpoint{1.533666in}{1.180848in}}%
\pgfpathlineto{\pgfqpoint{1.535729in}{1.187186in}}%
\pgfpathlineto{\pgfqpoint{1.537797in}{1.193855in}}%
\pgfpathlineto{\pgfqpoint{1.539871in}{1.200860in}}%
\pgfpathlineto{\pgfqpoint{1.541951in}{1.208206in}}%
\pgfpathlineto{\pgfqpoint{1.558736in}{1.202380in}}%
\pgfpathlineto{\pgfqpoint{1.575185in}{1.196287in}}%
\pgfpathlineto{\pgfqpoint{1.591284in}{1.189932in}}%
\pgfpathlineto{\pgfqpoint{1.607018in}{1.183320in}}%
\pgfpathlineto{\pgfqpoint{1.604555in}{1.176091in}}%
\pgfpathlineto{\pgfqpoint{1.602099in}{1.169205in}}%
\pgfpathlineto{\pgfqpoint{1.599651in}{1.162655in}}%
\pgfpathlineto{\pgfqpoint{1.597209in}{1.156437in}}%
\pgfpathlineto{\pgfqpoint{1.581846in}{1.162922in}}%
\pgfpathlineto{\pgfqpoint{1.566124in}{1.169155in}}%
\pgfpathlineto{\pgfqpoint{1.550059in}{1.175132in}}%
\pgfpathlineto{\pgfqpoint{1.533666in}{1.180848in}}%
\pgfpathclose%
\pgfusepath{fill}%
\end{pgfscope}%
\begin{pgfscope}%
\pgfpathrectangle{\pgfqpoint{0.041670in}{0.041670in}}{\pgfqpoint{2.216660in}{2.216660in}}%
\pgfusepath{clip}%
\pgfsetbuttcap%
\pgfsetroundjoin%
\definecolor{currentfill}{rgb}{0.268510,0.009605,0.335427}%
\pgfsetfillcolor{currentfill}%
\pgfsetlinewidth{0.000000pt}%
\definecolor{currentstroke}{rgb}{0.000000,0.000000,0.000000}%
\pgfsetstrokecolor{currentstroke}%
\pgfsetdash{}{0pt}%
\pgfpathmoveto{\pgfqpoint{1.426349in}{1.131893in}}%
\pgfpathlineto{\pgfqpoint{1.427937in}{1.131943in}}%
\pgfpathlineto{\pgfqpoint{1.429527in}{1.132205in}}%
\pgfpathlineto{\pgfqpoint{1.431118in}{1.132685in}}%
\pgfpathlineto{\pgfqpoint{1.432711in}{1.133387in}}%
\pgfpathlineto{\pgfqpoint{1.448246in}{1.129245in}}%
\pgfpathlineto{\pgfqpoint{1.463542in}{1.124855in}}%
\pgfpathlineto{\pgfqpoint{1.478585in}{1.120221in}}%
\pgfpathlineto{\pgfqpoint{1.493360in}{1.115346in}}%
\pgfpathlineto{\pgfqpoint{1.491379in}{1.114757in}}%
\pgfpathlineto{\pgfqpoint{1.489401in}{1.114389in}}%
\pgfpathlineto{\pgfqpoint{1.487426in}{1.114240in}}%
\pgfpathlineto{\pgfqpoint{1.485452in}{1.114303in}}%
\pgfpathlineto{\pgfqpoint{1.471054in}{1.119056in}}%
\pgfpathlineto{\pgfqpoint{1.456396in}{1.123574in}}%
\pgfpathlineto{\pgfqpoint{1.441490in}{1.127854in}}%
\pgfpathlineto{\pgfqpoint{1.426349in}{1.131893in}}%
\pgfpathclose%
\pgfusepath{fill}%
\end{pgfscope}%
\begin{pgfscope}%
\pgfpathrectangle{\pgfqpoint{0.041670in}{0.041670in}}{\pgfqpoint{2.216660in}{2.216660in}}%
\pgfusepath{clip}%
\pgfsetbuttcap%
\pgfsetroundjoin%
\definecolor{currentfill}{rgb}{0.272594,0.025563,0.353093}%
\pgfsetfillcolor{currentfill}%
\pgfsetlinewidth{0.000000pt}%
\definecolor{currentstroke}{rgb}{0.000000,0.000000,0.000000}%
\pgfsetstrokecolor{currentstroke}%
\pgfsetdash{}{0pt}%
\pgfpathmoveto{\pgfqpoint{1.517352in}{1.141372in}}%
\pgfpathlineto{\pgfqpoint{1.519375in}{1.145264in}}%
\pgfpathlineto{\pgfqpoint{1.521402in}{1.149443in}}%
\pgfpathlineto{\pgfqpoint{1.523434in}{1.153914in}}%
\pgfpathlineto{\pgfqpoint{1.525470in}{1.158684in}}%
\pgfpathlineto{\pgfqpoint{1.541477in}{1.153083in}}%
\pgfpathlineto{\pgfqpoint{1.557161in}{1.147227in}}%
\pgfpathlineto{\pgfqpoint{1.572510in}{1.141119in}}%
\pgfpathlineto{\pgfqpoint{1.587508in}{1.134764in}}%
\pgfpathlineto{\pgfqpoint{1.585097in}{1.130120in}}%
\pgfpathlineto{\pgfqpoint{1.582693in}{1.125774in}}%
\pgfpathlineto{\pgfqpoint{1.580294in}{1.121722in}}%
\pgfpathlineto{\pgfqpoint{1.577900in}{1.117958in}}%
\pgfpathlineto{\pgfqpoint{1.563264in}{1.124178in}}%
\pgfpathlineto{\pgfqpoint{1.548284in}{1.130157in}}%
\pgfpathlineto{\pgfqpoint{1.532976in}{1.135890in}}%
\pgfpathlineto{\pgfqpoint{1.517352in}{1.141372in}}%
\pgfpathclose%
\pgfusepath{fill}%
\end{pgfscope}%
\begin{pgfscope}%
\pgfpathrectangle{\pgfqpoint{0.041670in}{0.041670in}}{\pgfqpoint{2.216660in}{2.216660in}}%
\pgfusepath{clip}%
\pgfsetbuttcap%
\pgfsetroundjoin%
\definecolor{currentfill}{rgb}{0.172719,0.448791,0.557885}%
\pgfsetfillcolor{currentfill}%
\pgfsetlinewidth{0.000000pt}%
\definecolor{currentstroke}{rgb}{0.000000,0.000000,0.000000}%
\pgfsetstrokecolor{currentstroke}%
\pgfsetdash{}{0pt}%
\pgfpathmoveto{\pgfqpoint{0.835490in}{1.453946in}}%
\pgfpathlineto{\pgfqpoint{0.833574in}{1.469970in}}%
\pgfpathlineto{\pgfqpoint{0.831648in}{1.486481in}}%
\pgfpathlineto{\pgfqpoint{0.829713in}{1.503486in}}%
\pgfpathlineto{\pgfqpoint{0.850219in}{1.508601in}}%
\pgfpathlineto{\pgfqpoint{0.871007in}{1.513401in}}%
\pgfpathlineto{\pgfqpoint{0.892058in}{1.517884in}}%
\pgfpathlineto{\pgfqpoint{0.913356in}{1.522046in}}%
\pgfpathlineto{\pgfqpoint{0.914825in}{1.504986in}}%
\pgfpathlineto{\pgfqpoint{0.916287in}{1.488420in}}%
\pgfpathlineto{\pgfqpoint{0.917741in}{1.472338in}}%
\pgfpathlineto{\pgfqpoint{0.896797in}{1.468214in}}%
\pgfpathlineto{\pgfqpoint{0.876095in}{1.463772in}}%
\pgfpathlineto{\pgfqpoint{0.855654in}{1.459015in}}%
\pgfpathlineto{\pgfqpoint{0.835490in}{1.453946in}}%
\pgfpathclose%
\pgfusepath{fill}%
\end{pgfscope}%
\begin{pgfscope}%
\pgfpathrectangle{\pgfqpoint{0.041670in}{0.041670in}}{\pgfqpoint{2.216660in}{2.216660in}}%
\pgfusepath{clip}%
\pgfsetbuttcap%
\pgfsetroundjoin%
\definecolor{currentfill}{rgb}{0.233603,0.313828,0.543914}%
\pgfsetfillcolor{currentfill}%
\pgfsetlinewidth{0.000000pt}%
\definecolor{currentstroke}{rgb}{0.000000,0.000000,0.000000}%
\pgfsetstrokecolor{currentstroke}%
\pgfsetdash{}{0pt}%
\pgfpathmoveto{\pgfqpoint{0.776432in}{1.319710in}}%
\pgfpathlineto{\pgfqpoint{0.774161in}{1.332023in}}%
\pgfpathlineto{\pgfqpoint{0.771880in}{1.344766in}}%
\pgfpathlineto{\pgfqpoint{0.769590in}{1.357945in}}%
\pgfpathlineto{\pgfqpoint{0.767289in}{1.371567in}}%
\pgfpathlineto{\pgfqpoint{0.785735in}{1.377758in}}%
\pgfpathlineto{\pgfqpoint{0.804525in}{1.383657in}}%
\pgfpathlineto{\pgfqpoint{0.823640in}{1.389261in}}%
\pgfpathlineto{\pgfqpoint{0.843064in}{1.394565in}}%
\pgfpathlineto{\pgfqpoint{0.844936in}{1.380861in}}%
\pgfpathlineto{\pgfqpoint{0.846800in}{1.367599in}}%
\pgfpathlineto{\pgfqpoint{0.848655in}{1.354772in}}%
\pgfpathlineto{\pgfqpoint{0.850503in}{1.342373in}}%
\pgfpathlineto{\pgfqpoint{0.831515in}{1.337146in}}%
\pgfpathlineto{\pgfqpoint{0.812828in}{1.331623in}}%
\pgfpathlineto{\pgfqpoint{0.794462in}{1.325810in}}%
\pgfpathlineto{\pgfqpoint{0.776432in}{1.319710in}}%
\pgfpathclose%
\pgfusepath{fill}%
\end{pgfscope}%
\begin{pgfscope}%
\pgfpathrectangle{\pgfqpoint{0.041670in}{0.041670in}}{\pgfqpoint{2.216660in}{2.216660in}}%
\pgfusepath{clip}%
\pgfsetbuttcap%
\pgfsetroundjoin%
\definecolor{currentfill}{rgb}{0.282884,0.135920,0.453427}%
\pgfsetfillcolor{currentfill}%
\pgfsetlinewidth{0.000000pt}%
\definecolor{currentstroke}{rgb}{0.000000,0.000000,0.000000}%
\pgfsetstrokecolor{currentstroke}%
\pgfsetdash{}{0pt}%
\pgfpathmoveto{\pgfqpoint{1.541951in}{1.208206in}}%
\pgfpathlineto{\pgfqpoint{1.544038in}{1.215899in}}%
\pgfpathlineto{\pgfqpoint{1.546131in}{1.223945in}}%
\pgfpathlineto{\pgfqpoint{1.548231in}{1.232349in}}%
\pgfpathlineto{\pgfqpoint{1.550338in}{1.241119in}}%
\pgfpathlineto{\pgfqpoint{1.567518in}{1.235186in}}%
\pgfpathlineto{\pgfqpoint{1.584358in}{1.228981in}}%
\pgfpathlineto{\pgfqpoint{1.600839in}{1.222509in}}%
\pgfpathlineto{\pgfqpoint{1.616948in}{1.215776in}}%
\pgfpathlineto{\pgfqpoint{1.614453in}{1.207119in}}%
\pgfpathlineto{\pgfqpoint{1.611967in}{1.198828in}}%
\pgfpathlineto{\pgfqpoint{1.609488in}{1.190897in}}%
\pgfpathlineto{\pgfqpoint{1.607018in}{1.183320in}}%
\pgfpathlineto{\pgfqpoint{1.591284in}{1.189932in}}%
\pgfpathlineto{\pgfqpoint{1.575185in}{1.196287in}}%
\pgfpathlineto{\pgfqpoint{1.558736in}{1.202380in}}%
\pgfpathlineto{\pgfqpoint{1.541951in}{1.208206in}}%
\pgfpathclose%
\pgfusepath{fill}%
\end{pgfscope}%
\begin{pgfscope}%
\pgfpathrectangle{\pgfqpoint{0.041670in}{0.041670in}}{\pgfqpoint{2.216660in}{2.216660in}}%
\pgfusepath{clip}%
\pgfsetbuttcap%
\pgfsetroundjoin%
\definecolor{currentfill}{rgb}{0.268510,0.009605,0.335427}%
\pgfsetfillcolor{currentfill}%
\pgfsetlinewidth{0.000000pt}%
\definecolor{currentstroke}{rgb}{0.000000,0.000000,0.000000}%
\pgfsetstrokecolor{currentstroke}%
\pgfsetdash{}{0pt}%
\pgfpathmoveto{\pgfqpoint{1.509301in}{1.128586in}}%
\pgfpathlineto{\pgfqpoint{1.511308in}{1.131375in}}%
\pgfpathlineto{\pgfqpoint{1.513319in}{1.134433in}}%
\pgfpathlineto{\pgfqpoint{1.515334in}{1.137764in}}%
\pgfpathlineto{\pgfqpoint{1.517352in}{1.141372in}}%
\pgfpathlineto{\pgfqpoint{1.532976in}{1.135890in}}%
\pgfpathlineto{\pgfqpoint{1.548284in}{1.130157in}}%
\pgfpathlineto{\pgfqpoint{1.563264in}{1.124178in}}%
\pgfpathlineto{\pgfqpoint{1.577900in}{1.117958in}}%
\pgfpathlineto{\pgfqpoint{1.575511in}{1.114478in}}%
\pgfpathlineto{\pgfqpoint{1.573127in}{1.111277in}}%
\pgfpathlineto{\pgfqpoint{1.570747in}{1.108349in}}%
\pgfpathlineto{\pgfqpoint{1.568372in}{1.105690in}}%
\pgfpathlineto{\pgfqpoint{1.554095in}{1.111772in}}%
\pgfpathlineto{\pgfqpoint{1.539481in}{1.117618in}}%
\pgfpathlineto{\pgfqpoint{1.524545in}{1.123224in}}%
\pgfpathlineto{\pgfqpoint{1.509301in}{1.128586in}}%
\pgfpathclose%
\pgfusepath{fill}%
\end{pgfscope}%
\begin{pgfscope}%
\pgfpathrectangle{\pgfqpoint{0.041670in}{0.041670in}}{\pgfqpoint{2.216660in}{2.216660in}}%
\pgfusepath{clip}%
\pgfsetbuttcap%
\pgfsetroundjoin%
\definecolor{currentfill}{rgb}{0.274952,0.037752,0.364543}%
\pgfsetfillcolor{currentfill}%
\pgfsetlinewidth{0.000000pt}%
\definecolor{currentstroke}{rgb}{0.000000,0.000000,0.000000}%
\pgfsetstrokecolor{currentstroke}%
\pgfsetdash{}{0pt}%
\pgfpathmoveto{\pgfqpoint{1.354306in}{1.151613in}}%
\pgfpathlineto{\pgfqpoint{1.355482in}{1.150202in}}%
\pgfpathlineto{\pgfqpoint{1.356658in}{1.148970in}}%
\pgfpathlineto{\pgfqpoint{1.357834in}{1.147921in}}%
\pgfpathlineto{\pgfqpoint{1.359012in}{1.147060in}}%
\pgfpathlineto{\pgfqpoint{1.374536in}{1.144096in}}%
\pgfpathlineto{\pgfqpoint{1.389886in}{1.140886in}}%
\pgfpathlineto{\pgfqpoint{1.405049in}{1.137431in}}%
\pgfpathlineto{\pgfqpoint{1.420011in}{1.133736in}}%
\pgfpathlineto{\pgfqpoint{1.418430in}{1.134686in}}%
\pgfpathlineto{\pgfqpoint{1.416849in}{1.135825in}}%
\pgfpathlineto{\pgfqpoint{1.415270in}{1.137146in}}%
\pgfpathlineto{\pgfqpoint{1.413691in}{1.138648in}}%
\pgfpathlineto{\pgfqpoint{1.399126in}{1.142243in}}%
\pgfpathlineto{\pgfqpoint{1.384364in}{1.145605in}}%
\pgfpathlineto{\pgfqpoint{1.369420in}{1.148729in}}%
\pgfpathlineto{\pgfqpoint{1.354306in}{1.151613in}}%
\pgfpathclose%
\pgfusepath{fill}%
\end{pgfscope}%
\begin{pgfscope}%
\pgfpathrectangle{\pgfqpoint{0.041670in}{0.041670in}}{\pgfqpoint{2.216660in}{2.216660in}}%
\pgfusepath{clip}%
\pgfsetbuttcap%
\pgfsetroundjoin%
\definecolor{currentfill}{rgb}{0.268510,0.009605,0.335427}%
\pgfsetfillcolor{currentfill}%
\pgfsetlinewidth{0.000000pt}%
\definecolor{currentstroke}{rgb}{0.000000,0.000000,0.000000}%
\pgfsetstrokecolor{currentstroke}%
\pgfsetdash{}{0pt}%
\pgfpathmoveto{\pgfqpoint{0.861891in}{1.109885in}}%
\pgfpathlineto{\pgfqpoint{0.859834in}{1.109793in}}%
\pgfpathlineto{\pgfqpoint{0.857776in}{1.109914in}}%
\pgfpathlineto{\pgfqpoint{0.855715in}{1.110253in}}%
\pgfpathlineto{\pgfqpoint{0.853652in}{1.110814in}}%
\pgfpathlineto{\pgfqpoint{0.868178in}{1.115899in}}%
\pgfpathlineto{\pgfqpoint{0.882984in}{1.120747in}}%
\pgfpathlineto{\pgfqpoint{0.898055in}{1.125355in}}%
\pgfpathlineto{\pgfqpoint{0.913378in}{1.129717in}}%
\pgfpathlineto{\pgfqpoint{0.915059in}{1.129039in}}%
\pgfpathlineto{\pgfqpoint{0.916738in}{1.128582in}}%
\pgfpathlineto{\pgfqpoint{0.918416in}{1.128342in}}%
\pgfpathlineto{\pgfqpoint{0.920091in}{1.128315in}}%
\pgfpathlineto{\pgfqpoint{0.905158in}{1.124061in}}%
\pgfpathlineto{\pgfqpoint{0.890471in}{1.119569in}}%
\pgfpathlineto{\pgfqpoint{0.876044in}{1.114843in}}%
\pgfpathlineto{\pgfqpoint{0.861891in}{1.109885in}}%
\pgfpathclose%
\pgfusepath{fill}%
\end{pgfscope}%
\begin{pgfscope}%
\pgfpathrectangle{\pgfqpoint{0.041670in}{0.041670in}}{\pgfqpoint{2.216660in}{2.216660in}}%
\pgfusepath{clip}%
\pgfsetbuttcap%
\pgfsetroundjoin%
\definecolor{currentfill}{rgb}{0.279566,0.067836,0.391917}%
\pgfsetfillcolor{currentfill}%
\pgfsetlinewidth{0.000000pt}%
\definecolor{currentstroke}{rgb}{0.000000,0.000000,0.000000}%
\pgfsetstrokecolor{currentstroke}%
\pgfsetdash{}{0pt}%
\pgfpathmoveto{\pgfqpoint{1.304637in}{1.165964in}}%
\pgfpathlineto{\pgfqpoint{1.305499in}{1.163926in}}%
\pgfpathlineto{\pgfqpoint{1.306361in}{1.162050in}}%
\pgfpathlineto{\pgfqpoint{1.307223in}{1.160343in}}%
\pgfpathlineto{\pgfqpoint{1.308086in}{1.158806in}}%
\pgfpathlineto{\pgfqpoint{1.323625in}{1.156654in}}%
\pgfpathlineto{\pgfqpoint{1.339036in}{1.154255in}}%
\pgfpathlineto{\pgfqpoint{1.354306in}{1.151613in}}%
\pgfpathlineto{\pgfqpoint{1.353131in}{1.153200in}}%
\pgfpathlineto{\pgfqpoint{1.351957in}{1.154957in}}%
\pgfpathlineto{\pgfqpoint{1.350783in}{1.156883in}}%
\pgfpathlineto{\pgfqpoint{1.349609in}{1.158972in}}%
\pgfpathlineto{\pgfqpoint{1.334752in}{1.161540in}}%
\pgfpathlineto{\pgfqpoint{1.319757in}{1.163872in}}%
\pgfpathlineto{\pgfqpoint{1.304637in}{1.165964in}}%
\pgfpathclose%
\pgfusepath{fill}%
\end{pgfscope}%
\begin{pgfscope}%
\pgfpathrectangle{\pgfqpoint{0.041670in}{0.041670in}}{\pgfqpoint{2.216660in}{2.216660in}}%
\pgfusepath{clip}%
\pgfsetbuttcap%
\pgfsetroundjoin%
\definecolor{currentfill}{rgb}{0.277941,0.056324,0.381191}%
\pgfsetfillcolor{currentfill}%
\pgfsetlinewidth{0.000000pt}%
\definecolor{currentstroke}{rgb}{0.000000,0.000000,0.000000}%
\pgfsetstrokecolor{currentstroke}%
\pgfsetdash{}{0pt}%
\pgfpathmoveto{\pgfqpoint{0.759377in}{1.128913in}}%
\pgfpathlineto{\pgfqpoint{0.756882in}{1.133831in}}%
\pgfpathlineto{\pgfqpoint{0.754380in}{1.139059in}}%
\pgfpathlineto{\pgfqpoint{0.751872in}{1.144601in}}%
\pgfpathlineto{\pgfqpoint{0.749357in}{1.150465in}}%
\pgfpathlineto{\pgfqpoint{0.764389in}{1.157169in}}%
\pgfpathlineto{\pgfqpoint{0.779793in}{1.163627in}}%
\pgfpathlineto{\pgfqpoint{0.795554in}{1.169832in}}%
\pgfpathlineto{\pgfqpoint{0.811656in}{1.175780in}}%
\pgfpathlineto{\pgfqpoint{0.813799in}{1.169791in}}%
\pgfpathlineto{\pgfqpoint{0.815937in}{1.164121in}}%
\pgfpathlineto{\pgfqpoint{0.818069in}{1.158765in}}%
\pgfpathlineto{\pgfqpoint{0.820196in}{1.153718in}}%
\pgfpathlineto{\pgfqpoint{0.804475in}{1.147890in}}%
\pgfpathlineto{\pgfqpoint{0.789088in}{1.141810in}}%
\pgfpathlineto{\pgfqpoint{0.774051in}{1.135482in}}%
\pgfpathlineto{\pgfqpoint{0.759377in}{1.128913in}}%
\pgfpathclose%
\pgfusepath{fill}%
\end{pgfscope}%
\begin{pgfscope}%
\pgfpathrectangle{\pgfqpoint{0.041670in}{0.041670in}}{\pgfqpoint{2.216660in}{2.216660in}}%
\pgfusepath{clip}%
\pgfsetbuttcap%
\pgfsetroundjoin%
\definecolor{currentfill}{rgb}{0.282327,0.094955,0.417331}%
\pgfsetfillcolor{currentfill}%
\pgfsetlinewidth{0.000000pt}%
\definecolor{currentstroke}{rgb}{0.000000,0.000000,0.000000}%
\pgfsetstrokecolor{currentstroke}%
\pgfsetdash{}{0pt}%
\pgfpathmoveto{\pgfqpoint{0.749357in}{1.150465in}}%
\pgfpathlineto{\pgfqpoint{0.746835in}{1.156653in}}%
\pgfpathlineto{\pgfqpoint{0.744306in}{1.163174in}}%
\pgfpathlineto{\pgfqpoint{0.741770in}{1.170031in}}%
\pgfpathlineto{\pgfqpoint{0.739226in}{1.177231in}}%
\pgfpathlineto{\pgfqpoint{0.754622in}{1.184067in}}%
\pgfpathlineto{\pgfqpoint{0.770397in}{1.190651in}}%
\pgfpathlineto{\pgfqpoint{0.786535in}{1.196977in}}%
\pgfpathlineto{\pgfqpoint{0.803023in}{1.203040in}}%
\pgfpathlineto{\pgfqpoint{0.805191in}{1.195718in}}%
\pgfpathlineto{\pgfqpoint{0.807352in}{1.188738in}}%
\pgfpathlineto{\pgfqpoint{0.809507in}{1.182094in}}%
\pgfpathlineto{\pgfqpoint{0.811656in}{1.175780in}}%
\pgfpathlineto{\pgfqpoint{0.795554in}{1.169832in}}%
\pgfpathlineto{\pgfqpoint{0.779793in}{1.163627in}}%
\pgfpathlineto{\pgfqpoint{0.764389in}{1.157169in}}%
\pgfpathlineto{\pgfqpoint{0.749357in}{1.150465in}}%
\pgfpathclose%
\pgfusepath{fill}%
\end{pgfscope}%
\begin{pgfscope}%
\pgfpathrectangle{\pgfqpoint{0.041670in}{0.041670in}}{\pgfqpoint{2.216660in}{2.216660in}}%
\pgfusepath{clip}%
\pgfsetbuttcap%
\pgfsetroundjoin%
\definecolor{currentfill}{rgb}{0.279566,0.067836,0.391917}%
\pgfsetfillcolor{currentfill}%
\pgfsetlinewidth{0.000000pt}%
\definecolor{currentstroke}{rgb}{0.000000,0.000000,0.000000}%
\pgfsetstrokecolor{currentstroke}%
\pgfsetdash{}{0pt}%
\pgfpathmoveto{\pgfqpoint{0.997222in}{1.156491in}}%
\pgfpathlineto{\pgfqpoint{0.995957in}{1.154384in}}%
\pgfpathlineto{\pgfqpoint{0.994692in}{1.152441in}}%
\pgfpathlineto{\pgfqpoint{0.993427in}{1.150665in}}%
\pgfpathlineto{\pgfqpoint{0.992161in}{1.149061in}}%
\pgfpathlineto{\pgfqpoint{1.007293in}{1.151919in}}%
\pgfpathlineto{\pgfqpoint{1.022579in}{1.154534in}}%
\pgfpathlineto{\pgfqpoint{1.038005in}{1.156905in}}%
\pgfpathlineto{\pgfqpoint{1.053558in}{1.159030in}}%
\pgfpathlineto{\pgfqpoint{1.054409in}{1.160565in}}%
\pgfpathlineto{\pgfqpoint{1.055260in}{1.162271in}}%
\pgfpathlineto{\pgfqpoint{1.056110in}{1.164145in}}%
\pgfpathlineto{\pgfqpoint{1.056960in}{1.166182in}}%
\pgfpathlineto{\pgfqpoint{1.041827in}{1.164116in}}%
\pgfpathlineto{\pgfqpoint{1.026817in}{1.161811in}}%
\pgfpathlineto{\pgfqpoint{1.011944in}{1.159269in}}%
\pgfpathlineto{\pgfqpoint{0.997222in}{1.156491in}}%
\pgfpathclose%
\pgfusepath{fill}%
\end{pgfscope}%
\begin{pgfscope}%
\pgfpathrectangle{\pgfqpoint{0.041670in}{0.041670in}}{\pgfqpoint{2.216660in}{2.216660in}}%
\pgfusepath{clip}%
\pgfsetbuttcap%
\pgfsetroundjoin%
\definecolor{currentfill}{rgb}{0.272594,0.025563,0.353093}%
\pgfsetfillcolor{currentfill}%
\pgfsetlinewidth{0.000000pt}%
\definecolor{currentstroke}{rgb}{0.000000,0.000000,0.000000}%
\pgfsetstrokecolor{currentstroke}%
\pgfsetdash{}{0pt}%
\pgfpathmoveto{\pgfqpoint{0.769300in}{1.112231in}}%
\pgfpathlineto{\pgfqpoint{0.766828in}{1.115963in}}%
\pgfpathlineto{\pgfqpoint{0.764350in}{1.119984in}}%
\pgfpathlineto{\pgfqpoint{0.761866in}{1.124299in}}%
\pgfpathlineto{\pgfqpoint{0.759377in}{1.128913in}}%
\pgfpathlineto{\pgfqpoint{0.774051in}{1.135482in}}%
\pgfpathlineto{\pgfqpoint{0.789088in}{1.141810in}}%
\pgfpathlineto{\pgfqpoint{0.804475in}{1.147890in}}%
\pgfpathlineto{\pgfqpoint{0.820196in}{1.153718in}}%
\pgfpathlineto{\pgfqpoint{0.822318in}{1.148974in}}%
\pgfpathlineto{\pgfqpoint{0.824435in}{1.144529in}}%
\pgfpathlineto{\pgfqpoint{0.826547in}{1.140376in}}%
\pgfpathlineto{\pgfqpoint{0.828655in}{1.136511in}}%
\pgfpathlineto{\pgfqpoint{0.813310in}{1.130806in}}%
\pgfpathlineto{\pgfqpoint{0.798294in}{1.124854in}}%
\pgfpathlineto{\pgfqpoint{0.783619in}{1.118661in}}%
\pgfpathlineto{\pgfqpoint{0.769300in}{1.112231in}}%
\pgfpathclose%
\pgfusepath{fill}%
\end{pgfscope}%
\begin{pgfscope}%
\pgfpathrectangle{\pgfqpoint{0.041670in}{0.041670in}}{\pgfqpoint{2.216660in}{2.216660in}}%
\pgfusepath{clip}%
\pgfsetbuttcap%
\pgfsetroundjoin%
\definecolor{currentfill}{rgb}{0.282327,0.094955,0.417331}%
\pgfsetfillcolor{currentfill}%
\pgfsetlinewidth{0.000000pt}%
\definecolor{currentstroke}{rgb}{0.000000,0.000000,0.000000}%
\pgfsetstrokecolor{currentstroke}%
\pgfsetdash{}{0pt}%
\pgfpathmoveto{\pgfqpoint{1.120152in}{1.181559in}}%
\pgfpathlineto{\pgfqpoint{1.119727in}{1.178949in}}%
\pgfpathlineto{\pgfqpoint{1.119303in}{1.176487in}}%
\pgfpathlineto{\pgfqpoint{1.118878in}{1.174177in}}%
\pgfpathlineto{\pgfqpoint{1.118454in}{1.172023in}}%
\pgfpathlineto{\pgfqpoint{1.133998in}{1.172872in}}%
\pgfpathlineto{\pgfqpoint{1.149584in}{1.173475in}}%
\pgfpathlineto{\pgfqpoint{1.165197in}{1.173832in}}%
\pgfpathlineto{\pgfqpoint{1.180823in}{1.173942in}}%
\pgfpathlineto{\pgfqpoint{1.180817in}{1.176082in}}%
\pgfpathlineto{\pgfqpoint{1.180811in}{1.178378in}}%
\pgfpathlineto{\pgfqpoint{1.180805in}{1.180826in}}%
\pgfpathlineto{\pgfqpoint{1.180799in}{1.183422in}}%
\pgfpathlineto{\pgfqpoint{1.165604in}{1.183315in}}%
\pgfpathlineto{\pgfqpoint{1.150423in}{1.182969in}}%
\pgfpathlineto{\pgfqpoint{1.135267in}{1.182383in}}%
\pgfpathlineto{\pgfqpoint{1.120152in}{1.181559in}}%
\pgfpathclose%
\pgfusepath{fill}%
\end{pgfscope}%
\begin{pgfscope}%
\pgfpathrectangle{\pgfqpoint{0.041670in}{0.041670in}}{\pgfqpoint{2.216660in}{2.216660in}}%
\pgfusepath{clip}%
\pgfsetbuttcap%
\pgfsetroundjoin%
\definecolor{currentfill}{rgb}{0.282327,0.094955,0.417331}%
\pgfsetfillcolor{currentfill}%
\pgfsetlinewidth{0.000000pt}%
\definecolor{currentstroke}{rgb}{0.000000,0.000000,0.000000}%
\pgfsetstrokecolor{currentstroke}%
\pgfsetdash{}{0pt}%
\pgfpathmoveto{\pgfqpoint{1.180799in}{1.183422in}}%
\pgfpathlineto{\pgfqpoint{1.180805in}{1.180826in}}%
\pgfpathlineto{\pgfqpoint{1.180811in}{1.178378in}}%
\pgfpathlineto{\pgfqpoint{1.180817in}{1.176082in}}%
\pgfpathlineto{\pgfqpoint{1.180823in}{1.173942in}}%
\pgfpathlineto{\pgfqpoint{1.196448in}{1.173804in}}%
\pgfpathlineto{\pgfqpoint{1.212059in}{1.173420in}}%
\pgfpathlineto{\pgfqpoint{1.227641in}{1.172790in}}%
\pgfpathlineto{\pgfqpoint{1.243180in}{1.171913in}}%
\pgfpathlineto{\pgfqpoint{1.242744in}{1.174068in}}%
\pgfpathlineto{\pgfqpoint{1.242307in}{1.176379in}}%
\pgfpathlineto{\pgfqpoint{1.241871in}{1.178842in}}%
\pgfpathlineto{\pgfqpoint{1.241434in}{1.181452in}}%
\pgfpathlineto{\pgfqpoint{1.226324in}{1.182303in}}%
\pgfpathlineto{\pgfqpoint{1.211173in}{1.182916in}}%
\pgfpathlineto{\pgfqpoint{1.195993in}{1.183288in}}%
\pgfpathlineto{\pgfqpoint{1.180799in}{1.183422in}}%
\pgfpathclose%
\pgfusepath{fill}%
\end{pgfscope}%
\begin{pgfscope}%
\pgfpathrectangle{\pgfqpoint{0.041670in}{0.041670in}}{\pgfqpoint{2.216660in}{2.216660in}}%
\pgfusepath{clip}%
\pgfsetbuttcap%
\pgfsetroundjoin%
\definecolor{currentfill}{rgb}{0.274952,0.037752,0.364543}%
\pgfsetfillcolor{currentfill}%
\pgfsetlinewidth{0.000000pt}%
\definecolor{currentstroke}{rgb}{0.000000,0.000000,0.000000}%
\pgfsetstrokecolor{currentstroke}%
\pgfsetdash{}{0pt}%
\pgfpathmoveto{\pgfqpoint{0.933448in}{1.135257in}}%
\pgfpathlineto{\pgfqpoint{0.931782in}{1.133732in}}%
\pgfpathlineto{\pgfqpoint{0.930115in}{1.132387in}}%
\pgfpathlineto{\pgfqpoint{0.928448in}{1.131225in}}%
\pgfpathlineto{\pgfqpoint{0.926779in}{1.130251in}}%
\pgfpathlineto{\pgfqpoint{0.941550in}{1.134158in}}%
\pgfpathlineto{\pgfqpoint{0.956535in}{1.137827in}}%
\pgfpathlineto{\pgfqpoint{0.971720in}{1.141254in}}%
\pgfpathlineto{\pgfqpoint{0.987091in}{1.144438in}}%
\pgfpathlineto{\pgfqpoint{0.988359in}{1.145316in}}%
\pgfpathlineto{\pgfqpoint{0.989627in}{1.146383in}}%
\pgfpathlineto{\pgfqpoint{0.990894in}{1.147632in}}%
\pgfpathlineto{\pgfqpoint{0.992161in}{1.149061in}}%
\pgfpathlineto{\pgfqpoint{0.977197in}{1.145964in}}%
\pgfpathlineto{\pgfqpoint{0.962414in}{1.142629in}}%
\pgfpathlineto{\pgfqpoint{0.947827in}{1.139059in}}%
\pgfpathlineto{\pgfqpoint{0.933448in}{1.135257in}}%
\pgfpathclose%
\pgfusepath{fill}%
\end{pgfscope}%
\begin{pgfscope}%
\pgfpathrectangle{\pgfqpoint{0.041670in}{0.041670in}}{\pgfqpoint{2.216660in}{2.216660in}}%
\pgfusepath{clip}%
\pgfsetbuttcap%
\pgfsetroundjoin%
\definecolor{currentfill}{rgb}{0.276194,0.190074,0.493001}%
\pgfsetfillcolor{currentfill}%
\pgfsetlinewidth{0.000000pt}%
\definecolor{currentstroke}{rgb}{0.000000,0.000000,0.000000}%
\pgfsetstrokecolor{currentstroke}%
\pgfsetdash{}{0pt}%
\pgfpathmoveto{\pgfqpoint{1.550338in}{1.241119in}}%
\pgfpathlineto{\pgfqpoint{1.552452in}{1.250259in}}%
\pgfpathlineto{\pgfqpoint{1.554573in}{1.259776in}}%
\pgfpathlineto{\pgfqpoint{1.556702in}{1.269676in}}%
\pgfpathlineto{\pgfqpoint{1.558839in}{1.279965in}}%
\pgfpathlineto{\pgfqpoint{1.576422in}{1.273931in}}%
\pgfpathlineto{\pgfqpoint{1.593656in}{1.267619in}}%
\pgfpathlineto{\pgfqpoint{1.610526in}{1.261036in}}%
\pgfpathlineto{\pgfqpoint{1.627016in}{1.254185in}}%
\pgfpathlineto{\pgfqpoint{1.624485in}{1.244003in}}%
\pgfpathlineto{\pgfqpoint{1.621964in}{1.234212in}}%
\pgfpathlineto{\pgfqpoint{1.619452in}{1.224805in}}%
\pgfpathlineto{\pgfqpoint{1.616948in}{1.215776in}}%
\pgfpathlineto{\pgfqpoint{1.600839in}{1.222509in}}%
\pgfpathlineto{\pgfqpoint{1.584358in}{1.228981in}}%
\pgfpathlineto{\pgfqpoint{1.567518in}{1.235186in}}%
\pgfpathlineto{\pgfqpoint{1.550338in}{1.241119in}}%
\pgfpathclose%
\pgfusepath{fill}%
\end{pgfscope}%
\begin{pgfscope}%
\pgfpathrectangle{\pgfqpoint{0.041670in}{0.041670in}}{\pgfqpoint{2.216660in}{2.216660in}}%
\pgfusepath{clip}%
\pgfsetbuttcap%
\pgfsetroundjoin%
\definecolor{currentfill}{rgb}{0.201239,0.383670,0.554294}%
\pgfsetfillcolor{currentfill}%
\pgfsetlinewidth{0.000000pt}%
\definecolor{currentstroke}{rgb}{0.000000,0.000000,0.000000}%
\pgfsetstrokecolor{currentstroke}%
\pgfsetdash{}{0pt}%
\pgfpathmoveto{\pgfqpoint{1.499335in}{1.399026in}}%
\pgfpathlineto{\pgfqpoint{1.501117in}{1.413195in}}%
\pgfpathlineto{\pgfqpoint{1.502906in}{1.427819in}}%
\pgfpathlineto{\pgfqpoint{1.504704in}{1.442908in}}%
\pgfpathlineto{\pgfqpoint{1.506511in}{1.458467in}}%
\pgfpathlineto{\pgfqpoint{1.526642in}{1.453364in}}%
\pgfpathlineto{\pgfqpoint{1.546475in}{1.447954in}}%
\pgfpathlineto{\pgfqpoint{1.565991in}{1.442241in}}%
\pgfpathlineto{\pgfqpoint{1.585174in}{1.436229in}}%
\pgfpathlineto{\pgfqpoint{1.582925in}{1.420742in}}%
\pgfpathlineto{\pgfqpoint{1.580687in}{1.405727in}}%
\pgfpathlineto{\pgfqpoint{1.578459in}{1.391176in}}%
\pgfpathlineto{\pgfqpoint{1.576241in}{1.377084in}}%
\pgfpathlineto{\pgfqpoint{1.557489in}{1.383016in}}%
\pgfpathlineto{\pgfqpoint{1.538409in}{1.388653in}}%
\pgfpathlineto{\pgfqpoint{1.519019in}{1.393991in}}%
\pgfpathlineto{\pgfqpoint{1.499335in}{1.399026in}}%
\pgfpathclose%
\pgfusepath{fill}%
\end{pgfscope}%
\begin{pgfscope}%
\pgfpathrectangle{\pgfqpoint{0.041670in}{0.041670in}}{\pgfqpoint{2.216660in}{2.216660in}}%
\pgfusepath{clip}%
\pgfsetbuttcap%
\pgfsetroundjoin%
\definecolor{currentfill}{rgb}{0.267004,0.004874,0.329415}%
\pgfsetfillcolor{currentfill}%
\pgfsetlinewidth{0.000000pt}%
\definecolor{currentstroke}{rgb}{0.000000,0.000000,0.000000}%
\pgfsetstrokecolor{currentstroke}%
\pgfsetdash{}{0pt}%
\pgfpathmoveto{\pgfqpoint{1.501307in}{1.120010in}}%
\pgfpathlineto{\pgfqpoint{1.503300in}{1.121776in}}%
\pgfpathlineto{\pgfqpoint{1.505297in}{1.123791in}}%
\pgfpathlineto{\pgfqpoint{1.507297in}{1.126059in}}%
\pgfpathlineto{\pgfqpoint{1.509301in}{1.128586in}}%
\pgfpathlineto{\pgfqpoint{1.524545in}{1.123224in}}%
\pgfpathlineto{\pgfqpoint{1.539481in}{1.117618in}}%
\pgfpathlineto{\pgfqpoint{1.554095in}{1.111772in}}%
\pgfpathlineto{\pgfqpoint{1.568372in}{1.105690in}}%
\pgfpathlineto{\pgfqpoint{1.566002in}{1.103295in}}%
\pgfpathlineto{\pgfqpoint{1.563635in}{1.101159in}}%
\pgfpathlineto{\pgfqpoint{1.561273in}{1.099277in}}%
\pgfpathlineto{\pgfqpoint{1.558914in}{1.097645in}}%
\pgfpathlineto{\pgfqpoint{1.544992in}{1.103585in}}%
\pgfpathlineto{\pgfqpoint{1.530741in}{1.109296in}}%
\pgfpathlineto{\pgfqpoint{1.516174in}{1.114772in}}%
\pgfpathlineto{\pgfqpoint{1.501307in}{1.120010in}}%
\pgfpathclose%
\pgfusepath{fill}%
\end{pgfscope}%
\begin{pgfscope}%
\pgfpathrectangle{\pgfqpoint{0.041670in}{0.041670in}}{\pgfqpoint{2.216660in}{2.216660in}}%
\pgfusepath{clip}%
\pgfsetbuttcap%
\pgfsetroundjoin%
\definecolor{currentfill}{rgb}{0.282884,0.135920,0.453427}%
\pgfsetfillcolor{currentfill}%
\pgfsetlinewidth{0.000000pt}%
\definecolor{currentstroke}{rgb}{0.000000,0.000000,0.000000}%
\pgfsetstrokecolor{currentstroke}%
\pgfsetdash{}{0pt}%
\pgfpathmoveto{\pgfqpoint{0.739226in}{1.177231in}}%
\pgfpathlineto{\pgfqpoint{0.736674in}{1.184779in}}%
\pgfpathlineto{\pgfqpoint{0.734114in}{1.192682in}}%
\pgfpathlineto{\pgfqpoint{0.731545in}{1.200945in}}%
\pgfpathlineto{\pgfqpoint{0.728968in}{1.209574in}}%
\pgfpathlineto{\pgfqpoint{0.744732in}{1.216537in}}%
\pgfpathlineto{\pgfqpoint{0.760883in}{1.223242in}}%
\pgfpathlineto{\pgfqpoint{0.777406in}{1.229684in}}%
\pgfpathlineto{\pgfqpoint{0.794284in}{1.235859in}}%
\pgfpathlineto{\pgfqpoint{0.796479in}{1.227112in}}%
\pgfpathlineto{\pgfqpoint{0.798667in}{1.218731in}}%
\pgfpathlineto{\pgfqpoint{0.800848in}{1.210709in}}%
\pgfpathlineto{\pgfqpoint{0.803023in}{1.203040in}}%
\pgfpathlineto{\pgfqpoint{0.786535in}{1.196977in}}%
\pgfpathlineto{\pgfqpoint{0.770397in}{1.190651in}}%
\pgfpathlineto{\pgfqpoint{0.754622in}{1.184067in}}%
\pgfpathlineto{\pgfqpoint{0.739226in}{1.177231in}}%
\pgfpathclose%
\pgfusepath{fill}%
\end{pgfscope}%
\begin{pgfscope}%
\pgfpathrectangle{\pgfqpoint{0.041670in}{0.041670in}}{\pgfqpoint{2.216660in}{2.216660in}}%
\pgfusepath{clip}%
\pgfsetbuttcap%
\pgfsetroundjoin%
\definecolor{currentfill}{rgb}{0.268510,0.009605,0.335427}%
\pgfsetfillcolor{currentfill}%
\pgfsetlinewidth{0.000000pt}%
\definecolor{currentstroke}{rgb}{0.000000,0.000000,0.000000}%
\pgfsetstrokecolor{currentstroke}%
\pgfsetdash{}{0pt}%
\pgfpathmoveto{\pgfqpoint{0.779140in}{1.100090in}}%
\pgfpathlineto{\pgfqpoint{0.776687in}{1.102717in}}%
\pgfpathlineto{\pgfqpoint{0.774230in}{1.105613in}}%
\pgfpathlineto{\pgfqpoint{0.771768in}{1.108783in}}%
\pgfpathlineto{\pgfqpoint{0.769300in}{1.112231in}}%
\pgfpathlineto{\pgfqpoint{0.783619in}{1.118661in}}%
\pgfpathlineto{\pgfqpoint{0.798294in}{1.124854in}}%
\pgfpathlineto{\pgfqpoint{0.813310in}{1.130806in}}%
\pgfpathlineto{\pgfqpoint{0.828655in}{1.136511in}}%
\pgfpathlineto{\pgfqpoint{0.830758in}{1.132930in}}%
\pgfpathlineto{\pgfqpoint{0.832857in}{1.129626in}}%
\pgfpathlineto{\pgfqpoint{0.834952in}{1.126595in}}%
\pgfpathlineto{\pgfqpoint{0.837043in}{1.123832in}}%
\pgfpathlineto{\pgfqpoint{0.822073in}{1.118253in}}%
\pgfpathlineto{\pgfqpoint{0.807422in}{1.112433in}}%
\pgfpathlineto{\pgfqpoint{0.793107in}{1.106377in}}%
\pgfpathlineto{\pgfqpoint{0.779140in}{1.100090in}}%
\pgfpathclose%
\pgfusepath{fill}%
\end{pgfscope}%
\begin{pgfscope}%
\pgfpathrectangle{\pgfqpoint{0.041670in}{0.041670in}}{\pgfqpoint{2.216660in}{2.216660in}}%
\pgfusepath{clip}%
\pgfsetbuttcap%
\pgfsetroundjoin%
\definecolor{currentfill}{rgb}{0.271305,0.019942,0.347269}%
\pgfsetfillcolor{currentfill}%
\pgfsetlinewidth{0.000000pt}%
\definecolor{currentstroke}{rgb}{0.000000,0.000000,0.000000}%
\pgfsetstrokecolor{currentstroke}%
\pgfsetdash{}{0pt}%
\pgfpathmoveto{\pgfqpoint{1.420011in}{1.133736in}}%
\pgfpathlineto{\pgfqpoint{1.421594in}{1.132977in}}%
\pgfpathlineto{\pgfqpoint{1.423178in}{1.132414in}}%
\pgfpathlineto{\pgfqpoint{1.424763in}{1.132051in}}%
\pgfpathlineto{\pgfqpoint{1.426349in}{1.131893in}}%
\pgfpathlineto{\pgfqpoint{1.441490in}{1.127854in}}%
\pgfpathlineto{\pgfqpoint{1.456396in}{1.123574in}}%
\pgfpathlineto{\pgfqpoint{1.471054in}{1.119056in}}%
\pgfpathlineto{\pgfqpoint{1.485452in}{1.114303in}}%
\pgfpathlineto{\pgfqpoint{1.483480in}{1.114576in}}%
\pgfpathlineto{\pgfqpoint{1.481510in}{1.115053in}}%
\pgfpathlineto{\pgfqpoint{1.479541in}{1.115730in}}%
\pgfpathlineto{\pgfqpoint{1.477574in}{1.116604in}}%
\pgfpathlineto{\pgfqpoint{1.463553in}{1.121233in}}%
\pgfpathlineto{\pgfqpoint{1.449277in}{1.125633in}}%
\pgfpathlineto{\pgfqpoint{1.434758in}{1.129802in}}%
\pgfpathlineto{\pgfqpoint{1.420011in}{1.133736in}}%
\pgfpathclose%
\pgfusepath{fill}%
\end{pgfscope}%
\begin{pgfscope}%
\pgfpathrectangle{\pgfqpoint{0.041670in}{0.041670in}}{\pgfqpoint{2.216660in}{2.216660in}}%
\pgfusepath{clip}%
\pgfsetbuttcap%
\pgfsetroundjoin%
\definecolor{currentfill}{rgb}{0.282327,0.094955,0.417331}%
\pgfsetfillcolor{currentfill}%
\pgfsetlinewidth{0.000000pt}%
\definecolor{currentstroke}{rgb}{0.000000,0.000000,0.000000}%
\pgfsetstrokecolor{currentstroke}%
\pgfsetdash{}{0pt}%
\pgfpathmoveto{\pgfqpoint{1.060359in}{1.175886in}}%
\pgfpathlineto{\pgfqpoint{1.059509in}{1.173234in}}%
\pgfpathlineto{\pgfqpoint{1.058659in}{1.170730in}}%
\pgfpathlineto{\pgfqpoint{1.057810in}{1.168378in}}%
\pgfpathlineto{\pgfqpoint{1.056960in}{1.166182in}}%
\pgfpathlineto{\pgfqpoint{1.072203in}{1.168007in}}%
\pgfpathlineto{\pgfqpoint{1.087542in}{1.169589in}}%
\pgfpathlineto{\pgfqpoint{1.102963in}{1.170928in}}%
\pgfpathlineto{\pgfqpoint{1.118454in}{1.172023in}}%
\pgfpathlineto{\pgfqpoint{1.118878in}{1.174177in}}%
\pgfpathlineto{\pgfqpoint{1.119303in}{1.176487in}}%
\pgfpathlineto{\pgfqpoint{1.119727in}{1.178949in}}%
\pgfpathlineto{\pgfqpoint{1.120152in}{1.181559in}}%
\pgfpathlineto{\pgfqpoint{1.105090in}{1.180496in}}%
\pgfpathlineto{\pgfqpoint{1.090094in}{1.179195in}}%
\pgfpathlineto{\pgfqpoint{1.075180in}{1.177658in}}%
\pgfpathlineto{\pgfqpoint{1.060359in}{1.175886in}}%
\pgfpathclose%
\pgfusepath{fill}%
\end{pgfscope}%
\begin{pgfscope}%
\pgfpathrectangle{\pgfqpoint{0.041670in}{0.041670in}}{\pgfqpoint{2.216660in}{2.216660in}}%
\pgfusepath{clip}%
\pgfsetbuttcap%
\pgfsetroundjoin%
\definecolor{currentfill}{rgb}{0.282327,0.094955,0.417331}%
\pgfsetfillcolor{currentfill}%
\pgfsetlinewidth{0.000000pt}%
\definecolor{currentstroke}{rgb}{0.000000,0.000000,0.000000}%
\pgfsetstrokecolor{currentstroke}%
\pgfsetdash{}{0pt}%
\pgfpathmoveto{\pgfqpoint{1.241434in}{1.181452in}}%
\pgfpathlineto{\pgfqpoint{1.241871in}{1.178842in}}%
\pgfpathlineto{\pgfqpoint{1.242307in}{1.176379in}}%
\pgfpathlineto{\pgfqpoint{1.242744in}{1.174068in}}%
\pgfpathlineto{\pgfqpoint{1.243180in}{1.171913in}}%
\pgfpathlineto{\pgfqpoint{1.258663in}{1.170792in}}%
\pgfpathlineto{\pgfqpoint{1.274076in}{1.169425in}}%
\pgfpathlineto{\pgfqpoint{1.289406in}{1.167816in}}%
\pgfpathlineto{\pgfqpoint{1.304637in}{1.165964in}}%
\pgfpathlineto{\pgfqpoint{1.303775in}{1.168162in}}%
\pgfpathlineto{\pgfqpoint{1.302914in}{1.170516in}}%
\pgfpathlineto{\pgfqpoint{1.302053in}{1.173021in}}%
\pgfpathlineto{\pgfqpoint{1.301191in}{1.175675in}}%
\pgfpathlineto{\pgfqpoint{1.286382in}{1.177473in}}%
\pgfpathlineto{\pgfqpoint{1.271477in}{1.179036in}}%
\pgfpathlineto{\pgfqpoint{1.256490in}{1.180363in}}%
\pgfpathlineto{\pgfqpoint{1.241434in}{1.181452in}}%
\pgfpathclose%
\pgfusepath{fill}%
\end{pgfscope}%
\begin{pgfscope}%
\pgfpathrectangle{\pgfqpoint{0.041670in}{0.041670in}}{\pgfqpoint{2.216660in}{2.216660in}}%
\pgfusepath{clip}%
\pgfsetbuttcap%
\pgfsetroundjoin%
\definecolor{currentfill}{rgb}{0.201239,0.383670,0.554294}%
\pgfsetfillcolor{currentfill}%
\pgfsetlinewidth{0.000000pt}%
\definecolor{currentstroke}{rgb}{0.000000,0.000000,0.000000}%
\pgfsetstrokecolor{currentstroke}%
\pgfsetdash{}{0pt}%
\pgfpathmoveto{\pgfqpoint{0.767289in}{1.371567in}}%
\pgfpathlineto{\pgfqpoint{0.764978in}{1.385640in}}%
\pgfpathlineto{\pgfqpoint{0.762656in}{1.400172in}}%
\pgfpathlineto{\pgfqpoint{0.760323in}{1.415168in}}%
\pgfpathlineto{\pgfqpoint{0.757979in}{1.430638in}}%
\pgfpathlineto{\pgfqpoint{0.776850in}{1.436912in}}%
\pgfpathlineto{\pgfqpoint{0.796071in}{1.442891in}}%
\pgfpathlineto{\pgfqpoint{0.815623in}{1.448571in}}%
\pgfpathlineto{\pgfqpoint{0.835490in}{1.453946in}}%
\pgfpathlineto{\pgfqpoint{0.837397in}{1.438402in}}%
\pgfpathlineto{\pgfqpoint{0.839295in}{1.423328in}}%
\pgfpathlineto{\pgfqpoint{0.841184in}{1.408719in}}%
\pgfpathlineto{\pgfqpoint{0.843064in}{1.394565in}}%
\pgfpathlineto{\pgfqpoint{0.823640in}{1.389261in}}%
\pgfpathlineto{\pgfqpoint{0.804525in}{1.383657in}}%
\pgfpathlineto{\pgfqpoint{0.785735in}{1.377758in}}%
\pgfpathlineto{\pgfqpoint{0.767289in}{1.371567in}}%
\pgfpathclose%
\pgfusepath{fill}%
\end{pgfscope}%
\begin{pgfscope}%
\pgfpathrectangle{\pgfqpoint{0.041670in}{0.041670in}}{\pgfqpoint{2.216660in}{2.216660in}}%
\pgfusepath{clip}%
\pgfsetbuttcap%
\pgfsetroundjoin%
\definecolor{currentfill}{rgb}{0.276194,0.190074,0.493001}%
\pgfsetfillcolor{currentfill}%
\pgfsetlinewidth{0.000000pt}%
\definecolor{currentstroke}{rgb}{0.000000,0.000000,0.000000}%
\pgfsetstrokecolor{currentstroke}%
\pgfsetdash{}{0pt}%
\pgfpathmoveto{\pgfqpoint{0.728968in}{1.209574in}}%
\pgfpathlineto{\pgfqpoint{0.726382in}{1.218576in}}%
\pgfpathlineto{\pgfqpoint{0.723787in}{1.227956in}}%
\pgfpathlineto{\pgfqpoint{0.721183in}{1.237721in}}%
\pgfpathlineto{\pgfqpoint{0.718569in}{1.247876in}}%
\pgfpathlineto{\pgfqpoint{0.734707in}{1.254960in}}%
\pgfpathlineto{\pgfqpoint{0.751239in}{1.261780in}}%
\pgfpathlineto{\pgfqpoint{0.768151in}{1.268334in}}%
\pgfpathlineto{\pgfqpoint{0.785425in}{1.274615in}}%
\pgfpathlineto{\pgfqpoint{0.787652in}{1.264348in}}%
\pgfpathlineto{\pgfqpoint{0.789870in}{1.254470in}}%
\pgfpathlineto{\pgfqpoint{0.792081in}{1.244976in}}%
\pgfpathlineto{\pgfqpoint{0.794284in}{1.235859in}}%
\pgfpathlineto{\pgfqpoint{0.777406in}{1.229684in}}%
\pgfpathlineto{\pgfqpoint{0.760883in}{1.223242in}}%
\pgfpathlineto{\pgfqpoint{0.744732in}{1.216537in}}%
\pgfpathlineto{\pgfqpoint{0.728968in}{1.209574in}}%
\pgfpathclose%
\pgfusepath{fill}%
\end{pgfscope}%
\begin{pgfscope}%
\pgfpathrectangle{\pgfqpoint{0.041670in}{0.041670in}}{\pgfqpoint{2.216660in}{2.216660in}}%
\pgfusepath{clip}%
\pgfsetbuttcap%
\pgfsetroundjoin%
\definecolor{currentfill}{rgb}{0.267004,0.004874,0.329415}%
\pgfsetfillcolor{currentfill}%
\pgfsetlinewidth{0.000000pt}%
\definecolor{currentstroke}{rgb}{0.000000,0.000000,0.000000}%
\pgfsetstrokecolor{currentstroke}%
\pgfsetdash{}{0pt}%
\pgfpathmoveto{\pgfqpoint{0.788908in}{1.092175in}}%
\pgfpathlineto{\pgfqpoint{0.786472in}{1.093775in}}%
\pgfpathlineto{\pgfqpoint{0.784032in}{1.095624in}}%
\pgfpathlineto{\pgfqpoint{0.781588in}{1.097727in}}%
\pgfpathlineto{\pgfqpoint{0.779140in}{1.100090in}}%
\pgfpathlineto{\pgfqpoint{0.793107in}{1.106377in}}%
\pgfpathlineto{\pgfqpoint{0.807422in}{1.112433in}}%
\pgfpathlineto{\pgfqpoint{0.822073in}{1.118253in}}%
\pgfpathlineto{\pgfqpoint{0.837043in}{1.123832in}}%
\pgfpathlineto{\pgfqpoint{0.839131in}{1.121332in}}%
\pgfpathlineto{\pgfqpoint{0.841215in}{1.119091in}}%
\pgfpathlineto{\pgfqpoint{0.843295in}{1.117104in}}%
\pgfpathlineto{\pgfqpoint{0.845373in}{1.115366in}}%
\pgfpathlineto{\pgfqpoint{0.830772in}{1.109916in}}%
\pgfpathlineto{\pgfqpoint{0.816485in}{1.104231in}}%
\pgfpathlineto{\pgfqpoint{0.802526in}{1.098316in}}%
\pgfpathlineto{\pgfqpoint{0.788908in}{1.092175in}}%
\pgfpathclose%
\pgfusepath{fill}%
\end{pgfscope}%
\begin{pgfscope}%
\pgfpathrectangle{\pgfqpoint{0.041670in}{0.041670in}}{\pgfqpoint{2.216660in}{2.216660in}}%
\pgfusepath{clip}%
\pgfsetbuttcap%
\pgfsetroundjoin%
\definecolor{currentfill}{rgb}{0.271305,0.019942,0.347269}%
\pgfsetfillcolor{currentfill}%
\pgfsetlinewidth{0.000000pt}%
\definecolor{currentstroke}{rgb}{0.000000,0.000000,0.000000}%
\pgfsetstrokecolor{currentstroke}%
\pgfsetdash{}{0pt}%
\pgfpathmoveto{\pgfqpoint{0.870098in}{1.112301in}}%
\pgfpathlineto{\pgfqpoint{0.868048in}{1.111399in}}%
\pgfpathlineto{\pgfqpoint{0.865998in}{1.110692in}}%
\pgfpathlineto{\pgfqpoint{0.863945in}{1.110186in}}%
\pgfpathlineto{\pgfqpoint{0.861891in}{1.109885in}}%
\pgfpathlineto{\pgfqpoint{0.876044in}{1.114843in}}%
\pgfpathlineto{\pgfqpoint{0.890471in}{1.119569in}}%
\pgfpathlineto{\pgfqpoint{0.905158in}{1.124061in}}%
\pgfpathlineto{\pgfqpoint{0.920091in}{1.128315in}}%
\pgfpathlineto{\pgfqpoint{0.921765in}{1.128496in}}%
\pgfpathlineto{\pgfqpoint{0.923438in}{1.128883in}}%
\pgfpathlineto{\pgfqpoint{0.925109in}{1.129469in}}%
\pgfpathlineto{\pgfqpoint{0.926779in}{1.130251in}}%
\pgfpathlineto{\pgfqpoint{0.912234in}{1.126108in}}%
\pgfpathlineto{\pgfqpoint{0.897931in}{1.121733in}}%
\pgfpathlineto{\pgfqpoint{0.883880in}{1.117130in}}%
\pgfpathlineto{\pgfqpoint{0.870098in}{1.112301in}}%
\pgfpathclose%
\pgfusepath{fill}%
\end{pgfscope}%
\begin{pgfscope}%
\pgfpathrectangle{\pgfqpoint{0.041670in}{0.041670in}}{\pgfqpoint{2.216660in}{2.216660in}}%
\pgfusepath{clip}%
\pgfsetbuttcap%
\pgfsetroundjoin%
\definecolor{currentfill}{rgb}{0.260571,0.246922,0.522828}%
\pgfsetfillcolor{currentfill}%
\pgfsetlinewidth{0.000000pt}%
\definecolor{currentstroke}{rgb}{0.000000,0.000000,0.000000}%
\pgfsetstrokecolor{currentstroke}%
\pgfsetdash{}{0pt}%
\pgfpathmoveto{\pgfqpoint{1.558839in}{1.279965in}}%
\pgfpathlineto{\pgfqpoint{1.560983in}{1.290651in}}%
\pgfpathlineto{\pgfqpoint{1.563136in}{1.301738in}}%
\pgfpathlineto{\pgfqpoint{1.565298in}{1.313234in}}%
\pgfpathlineto{\pgfqpoint{1.567468in}{1.325146in}}%
\pgfpathlineto{\pgfqpoint{1.585460in}{1.319014in}}%
\pgfpathlineto{\pgfqpoint{1.603096in}{1.312602in}}%
\pgfpathlineto{\pgfqpoint{1.620361in}{1.305912in}}%
\pgfpathlineto{\pgfqpoint{1.637238in}{1.298951in}}%
\pgfpathlineto{\pgfqpoint{1.634667in}{1.287141in}}%
\pgfpathlineto{\pgfqpoint{1.632106in}{1.275748in}}%
\pgfpathlineto{\pgfqpoint{1.629556in}{1.264765in}}%
\pgfpathlineto{\pgfqpoint{1.627016in}{1.254185in}}%
\pgfpathlineto{\pgfqpoint{1.610526in}{1.261036in}}%
\pgfpathlineto{\pgfqpoint{1.593656in}{1.267619in}}%
\pgfpathlineto{\pgfqpoint{1.576422in}{1.273931in}}%
\pgfpathlineto{\pgfqpoint{1.558839in}{1.279965in}}%
\pgfpathclose%
\pgfusepath{fill}%
\end{pgfscope}%
\begin{pgfscope}%
\pgfpathrectangle{\pgfqpoint{0.041670in}{0.041670in}}{\pgfqpoint{2.216660in}{2.216660in}}%
\pgfusepath{clip}%
\pgfsetbuttcap%
\pgfsetroundjoin%
\definecolor{currentfill}{rgb}{0.267004,0.004874,0.329415}%
\pgfsetfillcolor{currentfill}%
\pgfsetlinewidth{0.000000pt}%
\definecolor{currentstroke}{rgb}{0.000000,0.000000,0.000000}%
\pgfsetstrokecolor{currentstroke}%
\pgfsetdash{}{0pt}%
\pgfpathmoveto{\pgfqpoint{1.493360in}{1.115346in}}%
\pgfpathlineto{\pgfqpoint{1.495343in}{1.116161in}}%
\pgfpathlineto{\pgfqpoint{1.497328in}{1.117207in}}%
\pgfpathlineto{\pgfqpoint{1.499316in}{1.118489in}}%
\pgfpathlineto{\pgfqpoint{1.501307in}{1.120010in}}%
\pgfpathlineto{\pgfqpoint{1.516174in}{1.114772in}}%
\pgfpathlineto{\pgfqpoint{1.530741in}{1.109296in}}%
\pgfpathlineto{\pgfqpoint{1.544992in}{1.103585in}}%
\pgfpathlineto{\pgfqpoint{1.558914in}{1.097645in}}%
\pgfpathlineto{\pgfqpoint{1.556559in}{1.096257in}}%
\pgfpathlineto{\pgfqpoint{1.554207in}{1.095111in}}%
\pgfpathlineto{\pgfqpoint{1.551858in}{1.094200in}}%
\pgfpathlineto{\pgfqpoint{1.549513in}{1.093520in}}%
\pgfpathlineto{\pgfqpoint{1.535944in}{1.099317in}}%
\pgfpathlineto{\pgfqpoint{1.522053in}{1.104890in}}%
\pgfpathlineto{\pgfqpoint{1.507854in}{1.110234in}}%
\pgfpathlineto{\pgfqpoint{1.493360in}{1.115346in}}%
\pgfpathclose%
\pgfusepath{fill}%
\end{pgfscope}%
\begin{pgfscope}%
\pgfpathrectangle{\pgfqpoint{0.041670in}{0.041670in}}{\pgfqpoint{2.216660in}{2.216660in}}%
\pgfusepath{clip}%
\pgfsetbuttcap%
\pgfsetroundjoin%
\definecolor{currentfill}{rgb}{0.279566,0.067836,0.391917}%
\pgfsetfillcolor{currentfill}%
\pgfsetlinewidth{0.000000pt}%
\definecolor{currentstroke}{rgb}{0.000000,0.000000,0.000000}%
\pgfsetstrokecolor{currentstroke}%
\pgfsetdash{}{0pt}%
\pgfpathmoveto{\pgfqpoint{1.349609in}{1.158972in}}%
\pgfpathlineto{\pgfqpoint{1.350783in}{1.156883in}}%
\pgfpathlineto{\pgfqpoint{1.351957in}{1.154957in}}%
\pgfpathlineto{\pgfqpoint{1.353131in}{1.153200in}}%
\pgfpathlineto{\pgfqpoint{1.354306in}{1.151613in}}%
\pgfpathlineto{\pgfqpoint{1.369420in}{1.148729in}}%
\pgfpathlineto{\pgfqpoint{1.384364in}{1.145605in}}%
\pgfpathlineto{\pgfqpoint{1.399126in}{1.142243in}}%
\pgfpathlineto{\pgfqpoint{1.413691in}{1.138648in}}%
\pgfpathlineto{\pgfqpoint{1.412113in}{1.140325in}}%
\pgfpathlineto{\pgfqpoint{1.410535in}{1.142173in}}%
\pgfpathlineto{\pgfqpoint{1.408959in}{1.144190in}}%
\pgfpathlineto{\pgfqpoint{1.407382in}{1.146370in}}%
\pgfpathlineto{\pgfqpoint{1.393213in}{1.149865in}}%
\pgfpathlineto{\pgfqpoint{1.378853in}{1.153132in}}%
\pgfpathlineto{\pgfqpoint{1.364313in}{1.156168in}}%
\pgfpathlineto{\pgfqpoint{1.349609in}{1.158972in}}%
\pgfpathclose%
\pgfusepath{fill}%
\end{pgfscope}%
\begin{pgfscope}%
\pgfpathrectangle{\pgfqpoint{0.041670in}{0.041670in}}{\pgfqpoint{2.216660in}{2.216660in}}%
\pgfusepath{clip}%
\pgfsetbuttcap%
\pgfsetroundjoin%
\definecolor{currentfill}{rgb}{0.282327,0.094955,0.417331}%
\pgfsetfillcolor{currentfill}%
\pgfsetlinewidth{0.000000pt}%
\definecolor{currentstroke}{rgb}{0.000000,0.000000,0.000000}%
\pgfsetstrokecolor{currentstroke}%
\pgfsetdash{}{0pt}%
\pgfpathmoveto{\pgfqpoint{1.301191in}{1.175675in}}%
\pgfpathlineto{\pgfqpoint{1.302053in}{1.173021in}}%
\pgfpathlineto{\pgfqpoint{1.302914in}{1.170516in}}%
\pgfpathlineto{\pgfqpoint{1.303775in}{1.168162in}}%
\pgfpathlineto{\pgfqpoint{1.304637in}{1.165964in}}%
\pgfpathlineto{\pgfqpoint{1.319757in}{1.163872in}}%
\pgfpathlineto{\pgfqpoint{1.334752in}{1.161540in}}%
\pgfpathlineto{\pgfqpoint{1.349609in}{1.158972in}}%
\pgfpathlineto{\pgfqpoint{1.348435in}{1.161220in}}%
\pgfpathlineto{\pgfqpoint{1.347262in}{1.163625in}}%
\pgfpathlineto{\pgfqpoint{1.346089in}{1.166181in}}%
\pgfpathlineto{\pgfqpoint{1.344916in}{1.168886in}}%
\pgfpathlineto{\pgfqpoint{1.330472in}{1.171380in}}%
\pgfpathlineto{\pgfqpoint{1.315893in}{1.173643in}}%
\pgfpathlineto{\pgfqpoint{1.301191in}{1.175675in}}%
\pgfpathclose%
\pgfusepath{fill}%
\end{pgfscope}%
\begin{pgfscope}%
\pgfpathrectangle{\pgfqpoint{0.041670in}{0.041670in}}{\pgfqpoint{2.216660in}{2.216660in}}%
\pgfusepath{clip}%
\pgfsetbuttcap%
\pgfsetroundjoin%
\definecolor{currentfill}{rgb}{0.279566,0.067836,0.391917}%
\pgfsetfillcolor{currentfill}%
\pgfsetlinewidth{0.000000pt}%
\definecolor{currentstroke}{rgb}{0.000000,0.000000,0.000000}%
\pgfsetstrokecolor{currentstroke}%
\pgfsetdash{}{0pt}%
\pgfpathmoveto{\pgfqpoint{0.940104in}{1.143074in}}%
\pgfpathlineto{\pgfqpoint{0.938441in}{1.140870in}}%
\pgfpathlineto{\pgfqpoint{0.936777in}{1.138830in}}%
\pgfpathlineto{\pgfqpoint{0.935113in}{1.136958in}}%
\pgfpathlineto{\pgfqpoint{0.933448in}{1.135257in}}%
\pgfpathlineto{\pgfqpoint{0.947827in}{1.139059in}}%
\pgfpathlineto{\pgfqpoint{0.962414in}{1.142629in}}%
\pgfpathlineto{\pgfqpoint{0.977197in}{1.145964in}}%
\pgfpathlineto{\pgfqpoint{0.992161in}{1.149061in}}%
\pgfpathlineto{\pgfqpoint{0.993427in}{1.150665in}}%
\pgfpathlineto{\pgfqpoint{0.994692in}{1.152441in}}%
\pgfpathlineto{\pgfqpoint{0.995957in}{1.154384in}}%
\pgfpathlineto{\pgfqpoint{0.997222in}{1.156491in}}%
\pgfpathlineto{\pgfqpoint{0.982664in}{1.153480in}}%
\pgfpathlineto{\pgfqpoint{0.968283in}{1.150239in}}%
\pgfpathlineto{\pgfqpoint{0.954092in}{1.146769in}}%
\pgfpathlineto{\pgfqpoint{0.940104in}{1.143074in}}%
\pgfpathclose%
\pgfusepath{fill}%
\end{pgfscope}%
\begin{pgfscope}%
\pgfpathrectangle{\pgfqpoint{0.041670in}{0.041670in}}{\pgfqpoint{2.216660in}{2.216660in}}%
\pgfusepath{clip}%
\pgfsetbuttcap%
\pgfsetroundjoin%
\definecolor{currentfill}{rgb}{0.172719,0.448791,0.557885}%
\pgfsetfillcolor{currentfill}%
\pgfsetlinewidth{0.000000pt}%
\definecolor{currentstroke}{rgb}{0.000000,0.000000,0.000000}%
\pgfsetstrokecolor{currentstroke}%
\pgfsetdash{}{0pt}%
\pgfpathmoveto{\pgfqpoint{1.506511in}{1.458467in}}%
\pgfpathlineto{\pgfqpoint{1.508326in}{1.474505in}}%
\pgfpathlineto{\pgfqpoint{1.510150in}{1.491029in}}%
\pgfpathlineto{\pgfqpoint{1.511984in}{1.508048in}}%
\pgfpathlineto{\pgfqpoint{1.532457in}{1.502898in}}%
\pgfpathlineto{\pgfqpoint{1.552628in}{1.497439in}}%
\pgfpathlineto{\pgfqpoint{1.572478in}{1.491673in}}%
\pgfpathlineto{\pgfqpoint{1.591989in}{1.485605in}}%
\pgfpathlineto{\pgfqpoint{1.589706in}{1.468653in}}%
\pgfpathlineto{\pgfqpoint{1.587434in}{1.452197in}}%
\pgfpathlineto{\pgfqpoint{1.585174in}{1.436229in}}%
\pgfpathlineto{\pgfqpoint{1.565991in}{1.442241in}}%
\pgfpathlineto{\pgfqpoint{1.546475in}{1.447954in}}%
\pgfpathlineto{\pgfqpoint{1.526642in}{1.453364in}}%
\pgfpathlineto{\pgfqpoint{1.506511in}{1.458467in}}%
\pgfpathclose%
\pgfusepath{fill}%
\end{pgfscope}%
\begin{pgfscope}%
\pgfpathrectangle{\pgfqpoint{0.041670in}{0.041670in}}{\pgfqpoint{2.216660in}{2.216660in}}%
\pgfusepath{clip}%
\pgfsetbuttcap%
\pgfsetroundjoin%
\definecolor{currentfill}{rgb}{0.282327,0.094955,0.417331}%
\pgfsetfillcolor{currentfill}%
\pgfsetlinewidth{0.000000pt}%
\definecolor{currentstroke}{rgb}{0.000000,0.000000,0.000000}%
\pgfsetstrokecolor{currentstroke}%
\pgfsetdash{}{0pt}%
\pgfpathmoveto{\pgfqpoint{1.002279in}{1.166477in}}%
\pgfpathlineto{\pgfqpoint{1.001015in}{1.163755in}}%
\pgfpathlineto{\pgfqpoint{0.999751in}{1.161180in}}%
\pgfpathlineto{\pgfqpoint{0.998487in}{1.158758in}}%
\pgfpathlineto{\pgfqpoint{0.997222in}{1.156491in}}%
\pgfpathlineto{\pgfqpoint{1.011944in}{1.159269in}}%
\pgfpathlineto{\pgfqpoint{1.026817in}{1.161811in}}%
\pgfpathlineto{\pgfqpoint{1.041827in}{1.164116in}}%
\pgfpathlineto{\pgfqpoint{1.056960in}{1.166182in}}%
\pgfpathlineto{\pgfqpoint{1.057810in}{1.168378in}}%
\pgfpathlineto{\pgfqpoint{1.058659in}{1.170730in}}%
\pgfpathlineto{\pgfqpoint{1.059509in}{1.173234in}}%
\pgfpathlineto{\pgfqpoint{1.060359in}{1.175886in}}%
\pgfpathlineto{\pgfqpoint{1.045645in}{1.173881in}}%
\pgfpathlineto{\pgfqpoint{1.031051in}{1.171643in}}%
\pgfpathlineto{\pgfqpoint{1.016592in}{1.169174in}}%
\pgfpathlineto{\pgfqpoint{1.002279in}{1.166477in}}%
\pgfpathclose%
\pgfusepath{fill}%
\end{pgfscope}%
\begin{pgfscope}%
\pgfpathrectangle{\pgfqpoint{0.041670in}{0.041670in}}{\pgfqpoint{2.216660in}{2.216660in}}%
\pgfusepath{clip}%
\pgfsetbuttcap%
\pgfsetroundjoin%
\definecolor{currentfill}{rgb}{0.267004,0.004874,0.329415}%
\pgfsetfillcolor{currentfill}%
\pgfsetlinewidth{0.000000pt}%
\definecolor{currentstroke}{rgb}{0.000000,0.000000,0.000000}%
\pgfsetstrokecolor{currentstroke}%
\pgfsetdash{}{0pt}%
\pgfpathmoveto{\pgfqpoint{0.798616in}{1.088184in}}%
\pgfpathlineto{\pgfqpoint{0.796194in}{1.088830in}}%
\pgfpathlineto{\pgfqpoint{0.793769in}{1.089707in}}%
\pgfpathlineto{\pgfqpoint{0.791340in}{1.090821in}}%
\pgfpathlineto{\pgfqpoint{0.788908in}{1.092175in}}%
\pgfpathlineto{\pgfqpoint{0.802526in}{1.098316in}}%
\pgfpathlineto{\pgfqpoint{0.816485in}{1.104231in}}%
\pgfpathlineto{\pgfqpoint{0.830772in}{1.109916in}}%
\pgfpathlineto{\pgfqpoint{0.845373in}{1.115366in}}%
\pgfpathlineto{\pgfqpoint{0.847447in}{1.113873in}}%
\pgfpathlineto{\pgfqpoint{0.849518in}{1.112619in}}%
\pgfpathlineto{\pgfqpoint{0.851586in}{1.111601in}}%
\pgfpathlineto{\pgfqpoint{0.853652in}{1.110814in}}%
\pgfpathlineto{\pgfqpoint{0.839419in}{1.105495in}}%
\pgfpathlineto{\pgfqpoint{0.825494in}{1.099947in}}%
\pgfpathlineto{\pgfqpoint{0.811888in}{1.094175in}}%
\pgfpathlineto{\pgfqpoint{0.798616in}{1.088184in}}%
\pgfpathclose%
\pgfusepath{fill}%
\end{pgfscope}%
\begin{pgfscope}%
\pgfpathrectangle{\pgfqpoint{0.041670in}{0.041670in}}{\pgfqpoint{2.216660in}{2.216660in}}%
\pgfusepath{clip}%
\pgfsetbuttcap%
\pgfsetroundjoin%
\definecolor{currentfill}{rgb}{0.274952,0.037752,0.364543}%
\pgfsetfillcolor{currentfill}%
\pgfsetlinewidth{0.000000pt}%
\definecolor{currentstroke}{rgb}{0.000000,0.000000,0.000000}%
\pgfsetstrokecolor{currentstroke}%
\pgfsetdash{}{0pt}%
\pgfpathmoveto{\pgfqpoint{1.413691in}{1.138648in}}%
\pgfpathlineto{\pgfqpoint{1.415270in}{1.137146in}}%
\pgfpathlineto{\pgfqpoint{1.416849in}{1.135825in}}%
\pgfpathlineto{\pgfqpoint{1.418430in}{1.134686in}}%
\pgfpathlineto{\pgfqpoint{1.420011in}{1.133736in}}%
\pgfpathlineto{\pgfqpoint{1.434758in}{1.129802in}}%
\pgfpathlineto{\pgfqpoint{1.449277in}{1.125633in}}%
\pgfpathlineto{\pgfqpoint{1.463553in}{1.121233in}}%
\pgfpathlineto{\pgfqpoint{1.477574in}{1.116604in}}%
\pgfpathlineto{\pgfqpoint{1.475609in}{1.117670in}}%
\pgfpathlineto{\pgfqpoint{1.473644in}{1.118924in}}%
\pgfpathlineto{\pgfqpoint{1.471681in}{1.120362in}}%
\pgfpathlineto{\pgfqpoint{1.469719in}{1.121980in}}%
\pgfpathlineto{\pgfqpoint{1.456073in}{1.126483in}}%
\pgfpathlineto{\pgfqpoint{1.442178in}{1.130765in}}%
\pgfpathlineto{\pgfqpoint{1.428046in}{1.134820in}}%
\pgfpathlineto{\pgfqpoint{1.413691in}{1.138648in}}%
\pgfpathclose%
\pgfusepath{fill}%
\end{pgfscope}%
\begin{pgfscope}%
\pgfpathrectangle{\pgfqpoint{0.041670in}{0.041670in}}{\pgfqpoint{2.216660in}{2.216660in}}%
\pgfusepath{clip}%
\pgfsetbuttcap%
\pgfsetroundjoin%
\definecolor{currentfill}{rgb}{0.260571,0.246922,0.522828}%
\pgfsetfillcolor{currentfill}%
\pgfsetlinewidth{0.000000pt}%
\definecolor{currentstroke}{rgb}{0.000000,0.000000,0.000000}%
\pgfsetstrokecolor{currentstroke}%
\pgfsetdash{}{0pt}%
\pgfpathmoveto{\pgfqpoint{0.718569in}{1.247876in}}%
\pgfpathlineto{\pgfqpoint{0.715944in}{1.258430in}}%
\pgfpathlineto{\pgfqpoint{0.713310in}{1.269387in}}%
\pgfpathlineto{\pgfqpoint{0.710665in}{1.280755in}}%
\pgfpathlineto{\pgfqpoint{0.708009in}{1.292540in}}%
\pgfpathlineto{\pgfqpoint{0.724528in}{1.299738in}}%
\pgfpathlineto{\pgfqpoint{0.741448in}{1.306669in}}%
\pgfpathlineto{\pgfqpoint{0.758755in}{1.313328in}}%
\pgfpathlineto{\pgfqpoint{0.776432in}{1.319710in}}%
\pgfpathlineto{\pgfqpoint{0.778694in}{1.307819in}}%
\pgfpathlineto{\pgfqpoint{0.780946in}{1.296344in}}%
\pgfpathlineto{\pgfqpoint{0.783190in}{1.285278in}}%
\pgfpathlineto{\pgfqpoint{0.785425in}{1.274615in}}%
\pgfpathlineto{\pgfqpoint{0.768151in}{1.268334in}}%
\pgfpathlineto{\pgfqpoint{0.751239in}{1.261780in}}%
\pgfpathlineto{\pgfqpoint{0.734707in}{1.254960in}}%
\pgfpathlineto{\pgfqpoint{0.718569in}{1.247876in}}%
\pgfpathclose%
\pgfusepath{fill}%
\end{pgfscope}%
\begin{pgfscope}%
\pgfpathrectangle{\pgfqpoint{0.041670in}{0.041670in}}{\pgfqpoint{2.216660in}{2.216660in}}%
\pgfusepath{clip}%
\pgfsetbuttcap%
\pgfsetroundjoin%
\definecolor{currentfill}{rgb}{0.283072,0.130895,0.449241}%
\pgfsetfillcolor{currentfill}%
\pgfsetlinewidth{0.000000pt}%
\definecolor{currentstroke}{rgb}{0.000000,0.000000,0.000000}%
\pgfsetstrokecolor{currentstroke}%
\pgfsetdash{}{0pt}%
\pgfpathmoveto{\pgfqpoint{1.121850in}{1.193404in}}%
\pgfpathlineto{\pgfqpoint{1.121425in}{1.190239in}}%
\pgfpathlineto{\pgfqpoint{1.121001in}{1.187207in}}%
\pgfpathlineto{\pgfqpoint{1.120576in}{1.184313in}}%
\pgfpathlineto{\pgfqpoint{1.120152in}{1.181559in}}%
\pgfpathlineto{\pgfqpoint{1.135267in}{1.182383in}}%
\pgfpathlineto{\pgfqpoint{1.150423in}{1.182969in}}%
\pgfpathlineto{\pgfqpoint{1.165604in}{1.183315in}}%
\pgfpathlineto{\pgfqpoint{1.180799in}{1.183422in}}%
\pgfpathlineto{\pgfqpoint{1.180793in}{1.186162in}}%
\pgfpathlineto{\pgfqpoint{1.180787in}{1.189042in}}%
\pgfpathlineto{\pgfqpoint{1.180781in}{1.192060in}}%
\pgfpathlineto{\pgfqpoint{1.180775in}{1.195211in}}%
\pgfpathlineto{\pgfqpoint{1.166012in}{1.195107in}}%
\pgfpathlineto{\pgfqpoint{1.151261in}{1.194771in}}%
\pgfpathlineto{\pgfqpoint{1.136536in}{1.194203in}}%
\pgfpathlineto{\pgfqpoint{1.121850in}{1.193404in}}%
\pgfpathclose%
\pgfusepath{fill}%
\end{pgfscope}%
\begin{pgfscope}%
\pgfpathrectangle{\pgfqpoint{0.041670in}{0.041670in}}{\pgfqpoint{2.216660in}{2.216660in}}%
\pgfusepath{clip}%
\pgfsetbuttcap%
\pgfsetroundjoin%
\definecolor{currentfill}{rgb}{0.283072,0.130895,0.449241}%
\pgfsetfillcolor{currentfill}%
\pgfsetlinewidth{0.000000pt}%
\definecolor{currentstroke}{rgb}{0.000000,0.000000,0.000000}%
\pgfsetstrokecolor{currentstroke}%
\pgfsetdash{}{0pt}%
\pgfpathmoveto{\pgfqpoint{1.180775in}{1.195211in}}%
\pgfpathlineto{\pgfqpoint{1.180781in}{1.192060in}}%
\pgfpathlineto{\pgfqpoint{1.180787in}{1.189042in}}%
\pgfpathlineto{\pgfqpoint{1.180793in}{1.186162in}}%
\pgfpathlineto{\pgfqpoint{1.180799in}{1.183422in}}%
\pgfpathlineto{\pgfqpoint{1.195993in}{1.183288in}}%
\pgfpathlineto{\pgfqpoint{1.211173in}{1.182916in}}%
\pgfpathlineto{\pgfqpoint{1.226324in}{1.182303in}}%
\pgfpathlineto{\pgfqpoint{1.241434in}{1.181452in}}%
\pgfpathlineto{\pgfqpoint{1.240998in}{1.184207in}}%
\pgfpathlineto{\pgfqpoint{1.240562in}{1.187102in}}%
\pgfpathlineto{\pgfqpoint{1.240125in}{1.190135in}}%
\pgfpathlineto{\pgfqpoint{1.239689in}{1.193300in}}%
\pgfpathlineto{\pgfqpoint{1.225008in}{1.194126in}}%
\pgfpathlineto{\pgfqpoint{1.210286in}{1.194720in}}%
\pgfpathlineto{\pgfqpoint{1.195538in}{1.195081in}}%
\pgfpathlineto{\pgfqpoint{1.180775in}{1.195211in}}%
\pgfpathclose%
\pgfusepath{fill}%
\end{pgfscope}%
\begin{pgfscope}%
\pgfpathrectangle{\pgfqpoint{0.041670in}{0.041670in}}{\pgfqpoint{2.216660in}{2.216660in}}%
\pgfusepath{clip}%
\pgfsetbuttcap%
\pgfsetroundjoin%
\definecolor{currentfill}{rgb}{0.268510,0.009605,0.335427}%
\pgfsetfillcolor{currentfill}%
\pgfsetlinewidth{0.000000pt}%
\definecolor{currentstroke}{rgb}{0.000000,0.000000,0.000000}%
\pgfsetstrokecolor{currentstroke}%
\pgfsetdash{}{0pt}%
\pgfpathmoveto{\pgfqpoint{1.485452in}{1.114303in}}%
\pgfpathlineto{\pgfqpoint{1.487426in}{1.114240in}}%
\pgfpathlineto{\pgfqpoint{1.489401in}{1.114389in}}%
\pgfpathlineto{\pgfqpoint{1.491379in}{1.114757in}}%
\pgfpathlineto{\pgfqpoint{1.493360in}{1.115346in}}%
\pgfpathlineto{\pgfqpoint{1.507854in}{1.110234in}}%
\pgfpathlineto{\pgfqpoint{1.522053in}{1.104890in}}%
\pgfpathlineto{\pgfqpoint{1.535944in}{1.099317in}}%
\pgfpathlineto{\pgfqpoint{1.549513in}{1.093520in}}%
\pgfpathlineto{\pgfqpoint{1.547171in}{1.093068in}}%
\pgfpathlineto{\pgfqpoint{1.544831in}{1.092838in}}%
\pgfpathlineto{\pgfqpoint{1.542494in}{1.092825in}}%
\pgfpathlineto{\pgfqpoint{1.540160in}{1.093027in}}%
\pgfpathlineto{\pgfqpoint{1.526941in}{1.098678in}}%
\pgfpathlineto{\pgfqpoint{1.513408in}{1.104110in}}%
\pgfpathlineto{\pgfqpoint{1.499574in}{1.109320in}}%
\pgfpathlineto{\pgfqpoint{1.485452in}{1.114303in}}%
\pgfpathclose%
\pgfusepath{fill}%
\end{pgfscope}%
\begin{pgfscope}%
\pgfpathrectangle{\pgfqpoint{0.041670in}{0.041670in}}{\pgfqpoint{2.216660in}{2.216660in}}%
\pgfusepath{clip}%
\pgfsetbuttcap%
\pgfsetroundjoin%
\definecolor{currentfill}{rgb}{0.233603,0.313828,0.543914}%
\pgfsetfillcolor{currentfill}%
\pgfsetlinewidth{0.000000pt}%
\definecolor{currentstroke}{rgb}{0.000000,0.000000,0.000000}%
\pgfsetstrokecolor{currentstroke}%
\pgfsetdash{}{0pt}%
\pgfpathmoveto{\pgfqpoint{1.567468in}{1.325146in}}%
\pgfpathlineto{\pgfqpoint{1.569647in}{1.337480in}}%
\pgfpathlineto{\pgfqpoint{1.571836in}{1.350243in}}%
\pgfpathlineto{\pgfqpoint{1.574034in}{1.363442in}}%
\pgfpathlineto{\pgfqpoint{1.576241in}{1.377084in}}%
\pgfpathlineto{\pgfqpoint{1.594649in}{1.370862in}}%
\pgfpathlineto{\pgfqpoint{1.612694in}{1.364353in}}%
\pgfpathlineto{\pgfqpoint{1.630360in}{1.357564in}}%
\pgfpathlineto{\pgfqpoint{1.647631in}{1.350499in}}%
\pgfpathlineto{\pgfqpoint{1.645016in}{1.336952in}}%
\pgfpathlineto{\pgfqpoint{1.642412in}{1.323850in}}%
\pgfpathlineto{\pgfqpoint{1.639819in}{1.311185in}}%
\pgfpathlineto{\pgfqpoint{1.637238in}{1.298951in}}%
\pgfpathlineto{\pgfqpoint{1.620361in}{1.305912in}}%
\pgfpathlineto{\pgfqpoint{1.603096in}{1.312602in}}%
\pgfpathlineto{\pgfqpoint{1.585460in}{1.319014in}}%
\pgfpathlineto{\pgfqpoint{1.567468in}{1.325146in}}%
\pgfpathclose%
\pgfusepath{fill}%
\end{pgfscope}%
\begin{pgfscope}%
\pgfpathrectangle{\pgfqpoint{0.041670in}{0.041670in}}{\pgfqpoint{2.216660in}{2.216660in}}%
\pgfusepath{clip}%
\pgfsetbuttcap%
\pgfsetroundjoin%
\definecolor{currentfill}{rgb}{0.274952,0.037752,0.364543}%
\pgfsetfillcolor{currentfill}%
\pgfsetlinewidth{0.000000pt}%
\definecolor{currentstroke}{rgb}{0.000000,0.000000,0.000000}%
\pgfsetstrokecolor{currentstroke}%
\pgfsetdash{}{0pt}%
\pgfpathmoveto{\pgfqpoint{0.878280in}{1.117794in}}%
\pgfpathlineto{\pgfqpoint{0.876236in}{1.116147in}}%
\pgfpathlineto{\pgfqpoint{0.874191in}{1.114680in}}%
\pgfpathlineto{\pgfqpoint{0.872145in}{1.113397in}}%
\pgfpathlineto{\pgfqpoint{0.870098in}{1.112301in}}%
\pgfpathlineto{\pgfqpoint{0.883880in}{1.117130in}}%
\pgfpathlineto{\pgfqpoint{0.897931in}{1.121733in}}%
\pgfpathlineto{\pgfqpoint{0.912234in}{1.126108in}}%
\pgfpathlineto{\pgfqpoint{0.926779in}{1.130251in}}%
\pgfpathlineto{\pgfqpoint{0.928448in}{1.131225in}}%
\pgfpathlineto{\pgfqpoint{0.930115in}{1.132387in}}%
\pgfpathlineto{\pgfqpoint{0.931782in}{1.133732in}}%
\pgfpathlineto{\pgfqpoint{0.933448in}{1.135257in}}%
\pgfpathlineto{\pgfqpoint{0.919291in}{1.131226in}}%
\pgfpathlineto{\pgfqpoint{0.905368in}{1.126970in}}%
\pgfpathlineto{\pgfqpoint{0.891694in}{1.122491in}}%
\pgfpathlineto{\pgfqpoint{0.878280in}{1.117794in}}%
\pgfpathclose%
\pgfusepath{fill}%
\end{pgfscope}%
\begin{pgfscope}%
\pgfpathrectangle{\pgfqpoint{0.041670in}{0.041670in}}{\pgfqpoint{2.216660in}{2.216660in}}%
\pgfusepath{clip}%
\pgfsetbuttcap%
\pgfsetroundjoin%
\definecolor{currentfill}{rgb}{0.172719,0.448791,0.557885}%
\pgfsetfillcolor{currentfill}%
\pgfsetlinewidth{0.000000pt}%
\definecolor{currentstroke}{rgb}{0.000000,0.000000,0.000000}%
\pgfsetstrokecolor{currentstroke}%
\pgfsetdash{}{0pt}%
\pgfpathmoveto{\pgfqpoint{0.757979in}{1.430638in}}%
\pgfpathlineto{\pgfqpoint{0.755623in}{1.446588in}}%
\pgfpathlineto{\pgfqpoint{0.753256in}{1.463027in}}%
\pgfpathlineto{\pgfqpoint{0.750876in}{1.479962in}}%
\pgfpathlineto{\pgfqpoint{0.770071in}{1.486294in}}%
\pgfpathlineto{\pgfqpoint{0.789621in}{1.492329in}}%
\pgfpathlineto{\pgfqpoint{0.809508in}{1.498061in}}%
\pgfpathlineto{\pgfqpoint{0.829713in}{1.503486in}}%
\pgfpathlineto{\pgfqpoint{0.831648in}{1.486481in}}%
\pgfpathlineto{\pgfqpoint{0.833574in}{1.469970in}}%
\pgfpathlineto{\pgfqpoint{0.835490in}{1.453946in}}%
\pgfpathlineto{\pgfqpoint{0.815623in}{1.448571in}}%
\pgfpathlineto{\pgfqpoint{0.796071in}{1.442891in}}%
\pgfpathlineto{\pgfqpoint{0.776850in}{1.436912in}}%
\pgfpathlineto{\pgfqpoint{0.757979in}{1.430638in}}%
\pgfpathclose%
\pgfusepath{fill}%
\end{pgfscope}%
\begin{pgfscope}%
\pgfpathrectangle{\pgfqpoint{0.041670in}{0.041670in}}{\pgfqpoint{2.216660in}{2.216660in}}%
\pgfusepath{clip}%
\pgfsetbuttcap%
\pgfsetroundjoin%
\definecolor{currentfill}{rgb}{0.283072,0.130895,0.449241}%
\pgfsetfillcolor{currentfill}%
\pgfsetlinewidth{0.000000pt}%
\definecolor{currentstroke}{rgb}{0.000000,0.000000,0.000000}%
\pgfsetstrokecolor{currentstroke}%
\pgfsetdash{}{0pt}%
\pgfpathmoveto{\pgfqpoint{1.063757in}{1.187902in}}%
\pgfpathlineto{\pgfqpoint{1.062908in}{1.184695in}}%
\pgfpathlineto{\pgfqpoint{1.062058in}{1.181620in}}%
\pgfpathlineto{\pgfqpoint{1.061208in}{1.178683in}}%
\pgfpathlineto{\pgfqpoint{1.060359in}{1.175886in}}%
\pgfpathlineto{\pgfqpoint{1.075180in}{1.177658in}}%
\pgfpathlineto{\pgfqpoint{1.090094in}{1.179195in}}%
\pgfpathlineto{\pgfqpoint{1.105090in}{1.180496in}}%
\pgfpathlineto{\pgfqpoint{1.120152in}{1.181559in}}%
\pgfpathlineto{\pgfqpoint{1.120576in}{1.184313in}}%
\pgfpathlineto{\pgfqpoint{1.121001in}{1.187207in}}%
\pgfpathlineto{\pgfqpoint{1.121425in}{1.190239in}}%
\pgfpathlineto{\pgfqpoint{1.121850in}{1.193404in}}%
\pgfpathlineto{\pgfqpoint{1.107216in}{1.192373in}}%
\pgfpathlineto{\pgfqpoint{1.092647in}{1.191111in}}%
\pgfpathlineto{\pgfqpoint{1.078156in}{1.189621in}}%
\pgfpathlineto{\pgfqpoint{1.063757in}{1.187902in}}%
\pgfpathclose%
\pgfusepath{fill}%
\end{pgfscope}%
\begin{pgfscope}%
\pgfpathrectangle{\pgfqpoint{0.041670in}{0.041670in}}{\pgfqpoint{2.216660in}{2.216660in}}%
\pgfusepath{clip}%
\pgfsetbuttcap%
\pgfsetroundjoin%
\definecolor{currentfill}{rgb}{0.283072,0.130895,0.449241}%
\pgfsetfillcolor{currentfill}%
\pgfsetlinewidth{0.000000pt}%
\definecolor{currentstroke}{rgb}{0.000000,0.000000,0.000000}%
\pgfsetstrokecolor{currentstroke}%
\pgfsetdash{}{0pt}%
\pgfpathmoveto{\pgfqpoint{1.239689in}{1.193300in}}%
\pgfpathlineto{\pgfqpoint{1.240125in}{1.190135in}}%
\pgfpathlineto{\pgfqpoint{1.240562in}{1.187102in}}%
\pgfpathlineto{\pgfqpoint{1.240998in}{1.184207in}}%
\pgfpathlineto{\pgfqpoint{1.241434in}{1.181452in}}%
\pgfpathlineto{\pgfqpoint{1.256490in}{1.180363in}}%
\pgfpathlineto{\pgfqpoint{1.271477in}{1.179036in}}%
\pgfpathlineto{\pgfqpoint{1.286382in}{1.177473in}}%
\pgfpathlineto{\pgfqpoint{1.301191in}{1.175675in}}%
\pgfpathlineto{\pgfqpoint{1.300330in}{1.178473in}}%
\pgfpathlineto{\pgfqpoint{1.299469in}{1.181412in}}%
\pgfpathlineto{\pgfqpoint{1.298608in}{1.184488in}}%
\pgfpathlineto{\pgfqpoint{1.297746in}{1.187697in}}%
\pgfpathlineto{\pgfqpoint{1.283358in}{1.189441in}}%
\pgfpathlineto{\pgfqpoint{1.268877in}{1.190957in}}%
\pgfpathlineto{\pgfqpoint{1.254316in}{1.192244in}}%
\pgfpathlineto{\pgfqpoint{1.239689in}{1.193300in}}%
\pgfpathclose%
\pgfusepath{fill}%
\end{pgfscope}%
\begin{pgfscope}%
\pgfpathrectangle{\pgfqpoint{0.041670in}{0.041670in}}{\pgfqpoint{2.216660in}{2.216660in}}%
\pgfusepath{clip}%
\pgfsetbuttcap%
\pgfsetroundjoin%
\definecolor{currentfill}{rgb}{0.268510,0.009605,0.335427}%
\pgfsetfillcolor{currentfill}%
\pgfsetlinewidth{0.000000pt}%
\definecolor{currentstroke}{rgb}{0.000000,0.000000,0.000000}%
\pgfsetstrokecolor{currentstroke}%
\pgfsetdash{}{0pt}%
\pgfpathmoveto{\pgfqpoint{0.808274in}{1.087825in}}%
\pgfpathlineto{\pgfqpoint{0.805864in}{1.087590in}}%
\pgfpathlineto{\pgfqpoint{0.803451in}{1.087568in}}%
\pgfpathlineto{\pgfqpoint{0.801035in}{1.087764in}}%
\pgfpathlineto{\pgfqpoint{0.798616in}{1.088184in}}%
\pgfpathlineto{\pgfqpoint{0.811888in}{1.094175in}}%
\pgfpathlineto{\pgfqpoint{0.825494in}{1.099947in}}%
\pgfpathlineto{\pgfqpoint{0.839419in}{1.105495in}}%
\pgfpathlineto{\pgfqpoint{0.853652in}{1.110814in}}%
\pgfpathlineto{\pgfqpoint{0.855715in}{1.110253in}}%
\pgfpathlineto{\pgfqpoint{0.857776in}{1.109914in}}%
\pgfpathlineto{\pgfqpoint{0.859834in}{1.109793in}}%
\pgfpathlineto{\pgfqpoint{0.861891in}{1.109885in}}%
\pgfpathlineto{\pgfqpoint{0.848024in}{1.104700in}}%
\pgfpathlineto{\pgfqpoint{0.834457in}{1.099292in}}%
\pgfpathlineto{\pgfqpoint{0.821203in}{1.093666in}}%
\pgfpathlineto{\pgfqpoint{0.808274in}{1.087825in}}%
\pgfpathclose%
\pgfusepath{fill}%
\end{pgfscope}%
\begin{pgfscope}%
\pgfpathrectangle{\pgfqpoint{0.041670in}{0.041670in}}{\pgfqpoint{2.216660in}{2.216660in}}%
\pgfusepath{clip}%
\pgfsetbuttcap%
\pgfsetroundjoin%
\definecolor{currentfill}{rgb}{0.277941,0.056324,0.381191}%
\pgfsetfillcolor{currentfill}%
\pgfsetlinewidth{0.000000pt}%
\definecolor{currentstroke}{rgb}{0.000000,0.000000,0.000000}%
\pgfsetstrokecolor{currentstroke}%
\pgfsetdash{}{0pt}%
\pgfpathmoveto{\pgfqpoint{1.587508in}{1.134764in}}%
\pgfpathlineto{\pgfqpoint{1.589924in}{1.139713in}}%
\pgfpathlineto{\pgfqpoint{1.592346in}{1.144971in}}%
\pgfpathlineto{\pgfqpoint{1.594774in}{1.150544in}}%
\pgfpathlineto{\pgfqpoint{1.597209in}{1.156437in}}%
\pgfpathlineto{\pgfqpoint{1.612200in}{1.149704in}}%
\pgfpathlineto{\pgfqpoint{1.626803in}{1.142731in}}%
\pgfpathlineto{\pgfqpoint{1.641004in}{1.135522in}}%
\pgfpathlineto{\pgfqpoint{1.654789in}{1.128082in}}%
\pgfpathlineto{\pgfqpoint{1.652007in}{1.122332in}}%
\pgfpathlineto{\pgfqpoint{1.649232in}{1.116902in}}%
\pgfpathlineto{\pgfqpoint{1.646465in}{1.111789in}}%
\pgfpathlineto{\pgfqpoint{1.643704in}{1.106986in}}%
\pgfpathlineto{\pgfqpoint{1.630252in}{1.114273in}}%
\pgfpathlineto{\pgfqpoint{1.616393in}{1.121336in}}%
\pgfpathlineto{\pgfqpoint{1.602140in}{1.128168in}}%
\pgfpathlineto{\pgfqpoint{1.587508in}{1.134764in}}%
\pgfpathclose%
\pgfusepath{fill}%
\end{pgfscope}%
\begin{pgfscope}%
\pgfpathrectangle{\pgfqpoint{0.041670in}{0.041670in}}{\pgfqpoint{2.216660in}{2.216660in}}%
\pgfusepath{clip}%
\pgfsetbuttcap%
\pgfsetroundjoin%
\definecolor{currentfill}{rgb}{0.272594,0.025563,0.353093}%
\pgfsetfillcolor{currentfill}%
\pgfsetlinewidth{0.000000pt}%
\definecolor{currentstroke}{rgb}{0.000000,0.000000,0.000000}%
\pgfsetstrokecolor{currentstroke}%
\pgfsetdash{}{0pt}%
\pgfpathmoveto{\pgfqpoint{1.577900in}{1.117958in}}%
\pgfpathlineto{\pgfqpoint{1.580294in}{1.121722in}}%
\pgfpathlineto{\pgfqpoint{1.582693in}{1.125774in}}%
\pgfpathlineto{\pgfqpoint{1.585097in}{1.130120in}}%
\pgfpathlineto{\pgfqpoint{1.587508in}{1.134764in}}%
\pgfpathlineto{\pgfqpoint{1.602140in}{1.128168in}}%
\pgfpathlineto{\pgfqpoint{1.616393in}{1.121336in}}%
\pgfpathlineto{\pgfqpoint{1.630252in}{1.114273in}}%
\pgfpathlineto{\pgfqpoint{1.643704in}{1.106986in}}%
\pgfpathlineto{\pgfqpoint{1.640951in}{1.102487in}}%
\pgfpathlineto{\pgfqpoint{1.638204in}{1.098289in}}%
\pgfpathlineto{\pgfqpoint{1.635463in}{1.094386in}}%
\pgfpathlineto{\pgfqpoint{1.632729in}{1.090772in}}%
\pgfpathlineto{\pgfqpoint{1.619606in}{1.097904in}}%
\pgfpathlineto{\pgfqpoint{1.606085in}{1.104816in}}%
\pgfpathlineto{\pgfqpoint{1.592178in}{1.111502in}}%
\pgfpathlineto{\pgfqpoint{1.577900in}{1.117958in}}%
\pgfpathclose%
\pgfusepath{fill}%
\end{pgfscope}%
\begin{pgfscope}%
\pgfpathrectangle{\pgfqpoint{0.041670in}{0.041670in}}{\pgfqpoint{2.216660in}{2.216660in}}%
\pgfusepath{clip}%
\pgfsetbuttcap%
\pgfsetroundjoin%
\definecolor{currentfill}{rgb}{0.282327,0.094955,0.417331}%
\pgfsetfillcolor{currentfill}%
\pgfsetlinewidth{0.000000pt}%
\definecolor{currentstroke}{rgb}{0.000000,0.000000,0.000000}%
\pgfsetstrokecolor{currentstroke}%
\pgfsetdash{}{0pt}%
\pgfpathmoveto{\pgfqpoint{1.344916in}{1.168886in}}%
\pgfpathlineto{\pgfqpoint{1.346089in}{1.166181in}}%
\pgfpathlineto{\pgfqpoint{1.347262in}{1.163625in}}%
\pgfpathlineto{\pgfqpoint{1.348435in}{1.161220in}}%
\pgfpathlineto{\pgfqpoint{1.349609in}{1.158972in}}%
\pgfpathlineto{\pgfqpoint{1.364313in}{1.156168in}}%
\pgfpathlineto{\pgfqpoint{1.378853in}{1.153132in}}%
\pgfpathlineto{\pgfqpoint{1.393213in}{1.149865in}}%
\pgfpathlineto{\pgfqpoint{1.407382in}{1.146370in}}%
\pgfpathlineto{\pgfqpoint{1.405806in}{1.148710in}}%
\pgfpathlineto{\pgfqpoint{1.404231in}{1.151206in}}%
\pgfpathlineto{\pgfqpoint{1.402655in}{1.153854in}}%
\pgfpathlineto{\pgfqpoint{1.401080in}{1.156651in}}%
\pgfpathlineto{\pgfqpoint{1.387307in}{1.160044in}}%
\pgfpathlineto{\pgfqpoint{1.373346in}{1.163215in}}%
\pgfpathlineto{\pgfqpoint{1.359212in}{1.166164in}}%
\pgfpathlineto{\pgfqpoint{1.344916in}{1.168886in}}%
\pgfpathclose%
\pgfusepath{fill}%
\end{pgfscope}%
\begin{pgfscope}%
\pgfpathrectangle{\pgfqpoint{0.041670in}{0.041670in}}{\pgfqpoint{2.216660in}{2.216660in}}%
\pgfusepath{clip}%
\pgfsetbuttcap%
\pgfsetroundjoin%
\definecolor{currentfill}{rgb}{0.282327,0.094955,0.417331}%
\pgfsetfillcolor{currentfill}%
\pgfsetlinewidth{0.000000pt}%
\definecolor{currentstroke}{rgb}{0.000000,0.000000,0.000000}%
\pgfsetstrokecolor{currentstroke}%
\pgfsetdash{}{0pt}%
\pgfpathmoveto{\pgfqpoint{1.597209in}{1.156437in}}%
\pgfpathlineto{\pgfqpoint{1.599651in}{1.162655in}}%
\pgfpathlineto{\pgfqpoint{1.602099in}{1.169205in}}%
\pgfpathlineto{\pgfqpoint{1.604555in}{1.176091in}}%
\pgfpathlineto{\pgfqpoint{1.607018in}{1.183320in}}%
\pgfpathlineto{\pgfqpoint{1.622371in}{1.176456in}}%
\pgfpathlineto{\pgfqpoint{1.637328in}{1.169346in}}%
\pgfpathlineto{\pgfqpoint{1.651875in}{1.161995in}}%
\pgfpathlineto{\pgfqpoint{1.665998in}{1.154409in}}%
\pgfpathlineto{\pgfqpoint{1.663183in}{1.147317in}}%
\pgfpathlineto{\pgfqpoint{1.660377in}{1.140570in}}%
\pgfpathlineto{\pgfqpoint{1.657579in}{1.134160in}}%
\pgfpathlineto{\pgfqpoint{1.654789in}{1.128082in}}%
\pgfpathlineto{\pgfqpoint{1.641004in}{1.135522in}}%
\pgfpathlineto{\pgfqpoint{1.626803in}{1.142731in}}%
\pgfpathlineto{\pgfqpoint{1.612200in}{1.149704in}}%
\pgfpathlineto{\pgfqpoint{1.597209in}{1.156437in}}%
\pgfpathclose%
\pgfusepath{fill}%
\end{pgfscope}%
\begin{pgfscope}%
\pgfpathrectangle{\pgfqpoint{0.041670in}{0.041670in}}{\pgfqpoint{2.216660in}{2.216660in}}%
\pgfusepath{clip}%
\pgfsetbuttcap%
\pgfsetroundjoin%
\definecolor{currentfill}{rgb}{0.233603,0.313828,0.543914}%
\pgfsetfillcolor{currentfill}%
\pgfsetlinewidth{0.000000pt}%
\definecolor{currentstroke}{rgb}{0.000000,0.000000,0.000000}%
\pgfsetstrokecolor{currentstroke}%
\pgfsetdash{}{0pt}%
\pgfpathmoveto{\pgfqpoint{0.708009in}{1.292540in}}%
\pgfpathlineto{\pgfqpoint{0.705343in}{1.304750in}}%
\pgfpathlineto{\pgfqpoint{0.702664in}{1.317390in}}%
\pgfpathlineto{\pgfqpoint{0.699974in}{1.330468in}}%
\pgfpathlineto{\pgfqpoint{0.697272in}{1.343992in}}%
\pgfpathlineto{\pgfqpoint{0.714178in}{1.351298in}}%
\pgfpathlineto{\pgfqpoint{0.731493in}{1.358332in}}%
\pgfpathlineto{\pgfqpoint{0.749202in}{1.365091in}}%
\pgfpathlineto{\pgfqpoint{0.767289in}{1.371567in}}%
\pgfpathlineto{\pgfqpoint{0.769590in}{1.357945in}}%
\pgfpathlineto{\pgfqpoint{0.771880in}{1.344766in}}%
\pgfpathlineto{\pgfqpoint{0.774161in}{1.332023in}}%
\pgfpathlineto{\pgfqpoint{0.776432in}{1.319710in}}%
\pgfpathlineto{\pgfqpoint{0.758755in}{1.313328in}}%
\pgfpathlineto{\pgfqpoint{0.741448in}{1.306669in}}%
\pgfpathlineto{\pgfqpoint{0.724528in}{1.299738in}}%
\pgfpathlineto{\pgfqpoint{0.708009in}{1.292540in}}%
\pgfpathclose%
\pgfusepath{fill}%
\end{pgfscope}%
\begin{pgfscope}%
\pgfpathrectangle{\pgfqpoint{0.041670in}{0.041670in}}{\pgfqpoint{2.216660in}{2.216660in}}%
\pgfusepath{clip}%
\pgfsetbuttcap%
\pgfsetroundjoin%
\definecolor{currentfill}{rgb}{0.271305,0.019942,0.347269}%
\pgfsetfillcolor{currentfill}%
\pgfsetlinewidth{0.000000pt}%
\definecolor{currentstroke}{rgb}{0.000000,0.000000,0.000000}%
\pgfsetstrokecolor{currentstroke}%
\pgfsetdash{}{0pt}%
\pgfpathmoveto{\pgfqpoint{1.477574in}{1.116604in}}%
\pgfpathlineto{\pgfqpoint{1.479541in}{1.115730in}}%
\pgfpathlineto{\pgfqpoint{1.481510in}{1.115053in}}%
\pgfpathlineto{\pgfqpoint{1.483480in}{1.114576in}}%
\pgfpathlineto{\pgfqpoint{1.485452in}{1.114303in}}%
\pgfpathlineto{\pgfqpoint{1.499574in}{1.109320in}}%
\pgfpathlineto{\pgfqpoint{1.513408in}{1.104110in}}%
\pgfpathlineto{\pgfqpoint{1.526941in}{1.098678in}}%
\pgfpathlineto{\pgfqpoint{1.540160in}{1.093027in}}%
\pgfpathlineto{\pgfqpoint{1.537828in}{1.093438in}}%
\pgfpathlineto{\pgfqpoint{1.535498in}{1.094054in}}%
\pgfpathlineto{\pgfqpoint{1.533170in}{1.094871in}}%
\pgfpathlineto{\pgfqpoint{1.530844in}{1.095885in}}%
\pgfpathlineto{\pgfqpoint{1.517974in}{1.101387in}}%
\pgfpathlineto{\pgfqpoint{1.504798in}{1.106678in}}%
\pgfpathlineto{\pgfqpoint{1.491327in}{1.111751in}}%
\pgfpathlineto{\pgfqpoint{1.477574in}{1.116604in}}%
\pgfpathclose%
\pgfusepath{fill}%
\end{pgfscope}%
\begin{pgfscope}%
\pgfpathrectangle{\pgfqpoint{0.041670in}{0.041670in}}{\pgfqpoint{2.216660in}{2.216660in}}%
\pgfusepath{clip}%
\pgfsetbuttcap%
\pgfsetroundjoin%
\definecolor{currentfill}{rgb}{0.268510,0.009605,0.335427}%
\pgfsetfillcolor{currentfill}%
\pgfsetlinewidth{0.000000pt}%
\definecolor{currentstroke}{rgb}{0.000000,0.000000,0.000000}%
\pgfsetstrokecolor{currentstroke}%
\pgfsetdash{}{0pt}%
\pgfpathmoveto{\pgfqpoint{1.568372in}{1.105690in}}%
\pgfpathlineto{\pgfqpoint{1.570747in}{1.108349in}}%
\pgfpathlineto{\pgfqpoint{1.573127in}{1.111277in}}%
\pgfpathlineto{\pgfqpoint{1.575511in}{1.114478in}}%
\pgfpathlineto{\pgfqpoint{1.577900in}{1.117958in}}%
\pgfpathlineto{\pgfqpoint{1.592178in}{1.111502in}}%
\pgfpathlineto{\pgfqpoint{1.606085in}{1.104816in}}%
\pgfpathlineto{\pgfqpoint{1.619606in}{1.097904in}}%
\pgfpathlineto{\pgfqpoint{1.632729in}{1.090772in}}%
\pgfpathlineto{\pgfqpoint{1.630000in}{1.087442in}}%
\pgfpathlineto{\pgfqpoint{1.627278in}{1.084392in}}%
\pgfpathlineto{\pgfqpoint{1.624560in}{1.081616in}}%
\pgfpathlineto{\pgfqpoint{1.621848in}{1.079110in}}%
\pgfpathlineto{\pgfqpoint{1.609051in}{1.086082in}}%
\pgfpathlineto{\pgfqpoint{1.595864in}{1.092840in}}%
\pgfpathlineto{\pgfqpoint{1.582300in}{1.099378in}}%
\pgfpathlineto{\pgfqpoint{1.568372in}{1.105690in}}%
\pgfpathclose%
\pgfusepath{fill}%
\end{pgfscope}%
\begin{pgfscope}%
\pgfpathrectangle{\pgfqpoint{0.041670in}{0.041670in}}{\pgfqpoint{2.216660in}{2.216660in}}%
\pgfusepath{clip}%
\pgfsetbuttcap%
\pgfsetroundjoin%
\definecolor{currentfill}{rgb}{0.279566,0.067836,0.391917}%
\pgfsetfillcolor{currentfill}%
\pgfsetlinewidth{0.000000pt}%
\definecolor{currentstroke}{rgb}{0.000000,0.000000,0.000000}%
\pgfsetstrokecolor{currentstroke}%
\pgfsetdash{}{0pt}%
\pgfpathmoveto{\pgfqpoint{1.407382in}{1.146370in}}%
\pgfpathlineto{\pgfqpoint{1.408959in}{1.144190in}}%
\pgfpathlineto{\pgfqpoint{1.410535in}{1.142173in}}%
\pgfpathlineto{\pgfqpoint{1.412113in}{1.140325in}}%
\pgfpathlineto{\pgfqpoint{1.413691in}{1.138648in}}%
\pgfpathlineto{\pgfqpoint{1.428046in}{1.134820in}}%
\pgfpathlineto{\pgfqpoint{1.442178in}{1.130765in}}%
\pgfpathlineto{\pgfqpoint{1.456073in}{1.126483in}}%
\pgfpathlineto{\pgfqpoint{1.469719in}{1.121980in}}%
\pgfpathlineto{\pgfqpoint{1.467758in}{1.123774in}}%
\pgfpathlineto{\pgfqpoint{1.465798in}{1.125740in}}%
\pgfpathlineto{\pgfqpoint{1.463839in}{1.127873in}}%
\pgfpathlineto{\pgfqpoint{1.461880in}{1.130171in}}%
\pgfpathlineto{\pgfqpoint{1.448608in}{1.134547in}}%
\pgfpathlineto{\pgfqpoint{1.435092in}{1.138708in}}%
\pgfpathlineto{\pgfqpoint{1.421346in}{1.142650in}}%
\pgfpathlineto{\pgfqpoint{1.407382in}{1.146370in}}%
\pgfpathclose%
\pgfusepath{fill}%
\end{pgfscope}%
\begin{pgfscope}%
\pgfpathrectangle{\pgfqpoint{0.041670in}{0.041670in}}{\pgfqpoint{2.216660in}{2.216660in}}%
\pgfusepath{clip}%
\pgfsetbuttcap%
\pgfsetroundjoin%
\definecolor{currentfill}{rgb}{0.282327,0.094955,0.417331}%
\pgfsetfillcolor{currentfill}%
\pgfsetlinewidth{0.000000pt}%
\definecolor{currentstroke}{rgb}{0.000000,0.000000,0.000000}%
\pgfsetstrokecolor{currentstroke}%
\pgfsetdash{}{0pt}%
\pgfpathmoveto{\pgfqpoint{0.946754in}{1.153451in}}%
\pgfpathlineto{\pgfqpoint{0.945092in}{1.150630in}}%
\pgfpathlineto{\pgfqpoint{0.943430in}{1.147958in}}%
\pgfpathlineto{\pgfqpoint{0.941767in}{1.145438in}}%
\pgfpathlineto{\pgfqpoint{0.940104in}{1.143074in}}%
\pgfpathlineto{\pgfqpoint{0.954092in}{1.146769in}}%
\pgfpathlineto{\pgfqpoint{0.968283in}{1.150239in}}%
\pgfpathlineto{\pgfqpoint{0.982664in}{1.153480in}}%
\pgfpathlineto{\pgfqpoint{0.997222in}{1.156491in}}%
\pgfpathlineto{\pgfqpoint{0.998487in}{1.158758in}}%
\pgfpathlineto{\pgfqpoint{0.999751in}{1.161180in}}%
\pgfpathlineto{\pgfqpoint{1.001015in}{1.163755in}}%
\pgfpathlineto{\pgfqpoint{1.002279in}{1.166477in}}%
\pgfpathlineto{\pgfqpoint{0.988125in}{1.163554in}}%
\pgfpathlineto{\pgfqpoint{0.974145in}{1.160407in}}%
\pgfpathlineto{\pgfqpoint{0.960350in}{1.157038in}}%
\pgfpathlineto{\pgfqpoint{0.946754in}{1.153451in}}%
\pgfpathclose%
\pgfusepath{fill}%
\end{pgfscope}%
\begin{pgfscope}%
\pgfpathrectangle{\pgfqpoint{0.041670in}{0.041670in}}{\pgfqpoint{2.216660in}{2.216660in}}%
\pgfusepath{clip}%
\pgfsetbuttcap%
\pgfsetroundjoin%
\definecolor{currentfill}{rgb}{0.282884,0.135920,0.453427}%
\pgfsetfillcolor{currentfill}%
\pgfsetlinewidth{0.000000pt}%
\definecolor{currentstroke}{rgb}{0.000000,0.000000,0.000000}%
\pgfsetstrokecolor{currentstroke}%
\pgfsetdash{}{0pt}%
\pgfpathmoveto{\pgfqpoint{1.607018in}{1.183320in}}%
\pgfpathlineto{\pgfqpoint{1.609488in}{1.190897in}}%
\pgfpathlineto{\pgfqpoint{1.611967in}{1.198828in}}%
\pgfpathlineto{\pgfqpoint{1.614453in}{1.207119in}}%
\pgfpathlineto{\pgfqpoint{1.616948in}{1.215776in}}%
\pgfpathlineto{\pgfqpoint{1.632669in}{1.208785in}}%
\pgfpathlineto{\pgfqpoint{1.647985in}{1.201543in}}%
\pgfpathlineto{\pgfqpoint{1.662883in}{1.194056in}}%
\pgfpathlineto{\pgfqpoint{1.677348in}{1.186328in}}%
\pgfpathlineto{\pgfqpoint{1.674496in}{1.177804in}}%
\pgfpathlineto{\pgfqpoint{1.671654in}{1.169646in}}%
\pgfpathlineto{\pgfqpoint{1.668821in}{1.161850in}}%
\pgfpathlineto{\pgfqpoint{1.665998in}{1.154409in}}%
\pgfpathlineto{\pgfqpoint{1.651875in}{1.161995in}}%
\pgfpathlineto{\pgfqpoint{1.637328in}{1.169346in}}%
\pgfpathlineto{\pgfqpoint{1.622371in}{1.176456in}}%
\pgfpathlineto{\pgfqpoint{1.607018in}{1.183320in}}%
\pgfpathclose%
\pgfusepath{fill}%
\end{pgfscope}%
\begin{pgfscope}%
\pgfpathrectangle{\pgfqpoint{0.041670in}{0.041670in}}{\pgfqpoint{2.216660in}{2.216660in}}%
\pgfusepath{clip}%
\pgfsetbuttcap%
\pgfsetroundjoin%
\definecolor{currentfill}{rgb}{0.283072,0.130895,0.449241}%
\pgfsetfillcolor{currentfill}%
\pgfsetlinewidth{0.000000pt}%
\definecolor{currentstroke}{rgb}{0.000000,0.000000,0.000000}%
\pgfsetstrokecolor{currentstroke}%
\pgfsetdash{}{0pt}%
\pgfpathmoveto{\pgfqpoint{1.297746in}{1.187697in}}%
\pgfpathlineto{\pgfqpoint{1.298608in}{1.184488in}}%
\pgfpathlineto{\pgfqpoint{1.299469in}{1.181412in}}%
\pgfpathlineto{\pgfqpoint{1.300330in}{1.178473in}}%
\pgfpathlineto{\pgfqpoint{1.301191in}{1.175675in}}%
\pgfpathlineto{\pgfqpoint{1.315893in}{1.173643in}}%
\pgfpathlineto{\pgfqpoint{1.330472in}{1.171380in}}%
\pgfpathlineto{\pgfqpoint{1.344916in}{1.168886in}}%
\pgfpathlineto{\pgfqpoint{1.343743in}{1.171735in}}%
\pgfpathlineto{\pgfqpoint{1.342570in}{1.174725in}}%
\pgfpathlineto{\pgfqpoint{1.341397in}{1.177852in}}%
\pgfpathlineto{\pgfqpoint{1.340224in}{1.181112in}}%
\pgfpathlineto{\pgfqpoint{1.326192in}{1.183531in}}%
\pgfpathlineto{\pgfqpoint{1.312028in}{1.185727in}}%
\pgfpathlineto{\pgfqpoint{1.297746in}{1.187697in}}%
\pgfpathclose%
\pgfusepath{fill}%
\end{pgfscope}%
\begin{pgfscope}%
\pgfpathrectangle{\pgfqpoint{0.041670in}{0.041670in}}{\pgfqpoint{2.216660in}{2.216660in}}%
\pgfusepath{clip}%
\pgfsetbuttcap%
\pgfsetroundjoin%
\definecolor{currentfill}{rgb}{0.283072,0.130895,0.449241}%
\pgfsetfillcolor{currentfill}%
\pgfsetlinewidth{0.000000pt}%
\definecolor{currentstroke}{rgb}{0.000000,0.000000,0.000000}%
\pgfsetstrokecolor{currentstroke}%
\pgfsetdash{}{0pt}%
\pgfpathmoveto{\pgfqpoint{1.007334in}{1.178776in}}%
\pgfpathlineto{\pgfqpoint{1.006070in}{1.175498in}}%
\pgfpathlineto{\pgfqpoint{1.004806in}{1.172353in}}%
\pgfpathlineto{\pgfqpoint{1.003542in}{1.169345in}}%
\pgfpathlineto{\pgfqpoint{1.002279in}{1.166477in}}%
\pgfpathlineto{\pgfqpoint{1.016592in}{1.169174in}}%
\pgfpathlineto{\pgfqpoint{1.031051in}{1.171643in}}%
\pgfpathlineto{\pgfqpoint{1.045645in}{1.173881in}}%
\pgfpathlineto{\pgfqpoint{1.060359in}{1.175886in}}%
\pgfpathlineto{\pgfqpoint{1.061208in}{1.178683in}}%
\pgfpathlineto{\pgfqpoint{1.062058in}{1.181620in}}%
\pgfpathlineto{\pgfqpoint{1.062908in}{1.184695in}}%
\pgfpathlineto{\pgfqpoint{1.063757in}{1.187902in}}%
\pgfpathlineto{\pgfqpoint{1.049463in}{1.185957in}}%
\pgfpathlineto{\pgfqpoint{1.035285in}{1.183786in}}%
\pgfpathlineto{\pgfqpoint{1.021238in}{1.181392in}}%
\pgfpathlineto{\pgfqpoint{1.007334in}{1.178776in}}%
\pgfpathclose%
\pgfusepath{fill}%
\end{pgfscope}%
\begin{pgfscope}%
\pgfpathrectangle{\pgfqpoint{0.041670in}{0.041670in}}{\pgfqpoint{2.216660in}{2.216660in}}%
\pgfusepath{clip}%
\pgfsetbuttcap%
\pgfsetroundjoin%
\definecolor{currentfill}{rgb}{0.267004,0.004874,0.329415}%
\pgfsetfillcolor{currentfill}%
\pgfsetlinewidth{0.000000pt}%
\definecolor{currentstroke}{rgb}{0.000000,0.000000,0.000000}%
\pgfsetstrokecolor{currentstroke}%
\pgfsetdash{}{0pt}%
\pgfpathmoveto{\pgfqpoint{1.558914in}{1.097645in}}%
\pgfpathlineto{\pgfqpoint{1.561273in}{1.099277in}}%
\pgfpathlineto{\pgfqpoint{1.563635in}{1.101159in}}%
\pgfpathlineto{\pgfqpoint{1.566002in}{1.103295in}}%
\pgfpathlineto{\pgfqpoint{1.568372in}{1.105690in}}%
\pgfpathlineto{\pgfqpoint{1.582300in}{1.099378in}}%
\pgfpathlineto{\pgfqpoint{1.595864in}{1.092840in}}%
\pgfpathlineto{\pgfqpoint{1.609051in}{1.086082in}}%
\pgfpathlineto{\pgfqpoint{1.621848in}{1.079110in}}%
\pgfpathlineto{\pgfqpoint{1.619141in}{1.076869in}}%
\pgfpathlineto{\pgfqpoint{1.616438in}{1.074887in}}%
\pgfpathlineto{\pgfqpoint{1.613741in}{1.073160in}}%
\pgfpathlineto{\pgfqpoint{1.611047in}{1.071684in}}%
\pgfpathlineto{\pgfqpoint{1.598573in}{1.078494in}}%
\pgfpathlineto{\pgfqpoint{1.585718in}{1.085094in}}%
\pgfpathlineto{\pgfqpoint{1.572494in}{1.091479in}}%
\pgfpathlineto{\pgfqpoint{1.558914in}{1.097645in}}%
\pgfpathclose%
\pgfusepath{fill}%
\end{pgfscope}%
\begin{pgfscope}%
\pgfpathrectangle{\pgfqpoint{0.041670in}{0.041670in}}{\pgfqpoint{2.216660in}{2.216660in}}%
\pgfusepath{clip}%
\pgfsetbuttcap%
\pgfsetroundjoin%
\definecolor{currentfill}{rgb}{0.277941,0.056324,0.381191}%
\pgfsetfillcolor{currentfill}%
\pgfsetlinewidth{0.000000pt}%
\definecolor{currentstroke}{rgb}{0.000000,0.000000,0.000000}%
\pgfsetstrokecolor{currentstroke}%
\pgfsetdash{}{0pt}%
\pgfpathmoveto{\pgfqpoint{0.704601in}{1.100323in}}%
\pgfpathlineto{\pgfqpoint{0.701769in}{1.105091in}}%
\pgfpathlineto{\pgfqpoint{0.698929in}{1.110170in}}%
\pgfpathlineto{\pgfqpoint{0.696082in}{1.115565in}}%
\pgfpathlineto{\pgfqpoint{0.693228in}{1.121281in}}%
\pgfpathlineto{\pgfqpoint{0.706632in}{1.128920in}}%
\pgfpathlineto{\pgfqpoint{0.720464in}{1.136334in}}%
\pgfpathlineto{\pgfqpoint{0.734710in}{1.143518in}}%
\pgfpathlineto{\pgfqpoint{0.749357in}{1.150465in}}%
\pgfpathlineto{\pgfqpoint{0.751872in}{1.144601in}}%
\pgfpathlineto{\pgfqpoint{0.754380in}{1.139059in}}%
\pgfpathlineto{\pgfqpoint{0.756882in}{1.133831in}}%
\pgfpathlineto{\pgfqpoint{0.759377in}{1.128913in}}%
\pgfpathlineto{\pgfqpoint{0.745081in}{1.122107in}}%
\pgfpathlineto{\pgfqpoint{0.731177in}{1.115069in}}%
\pgfpathlineto{\pgfqpoint{0.717679in}{1.107806in}}%
\pgfpathlineto{\pgfqpoint{0.704601in}{1.100323in}}%
\pgfpathclose%
\pgfusepath{fill}%
\end{pgfscope}%
\begin{pgfscope}%
\pgfpathrectangle{\pgfqpoint{0.041670in}{0.041670in}}{\pgfqpoint{2.216660in}{2.216660in}}%
\pgfusepath{clip}%
\pgfsetbuttcap%
\pgfsetroundjoin%
\definecolor{currentfill}{rgb}{0.201239,0.383670,0.554294}%
\pgfsetfillcolor{currentfill}%
\pgfsetlinewidth{0.000000pt}%
\definecolor{currentstroke}{rgb}{0.000000,0.000000,0.000000}%
\pgfsetstrokecolor{currentstroke}%
\pgfsetdash{}{0pt}%
\pgfpathmoveto{\pgfqpoint{1.576241in}{1.377084in}}%
\pgfpathlineto{\pgfqpoint{1.578459in}{1.391176in}}%
\pgfpathlineto{\pgfqpoint{1.580687in}{1.405727in}}%
\pgfpathlineto{\pgfqpoint{1.582925in}{1.420742in}}%
\pgfpathlineto{\pgfqpoint{1.585174in}{1.436229in}}%
\pgfpathlineto{\pgfqpoint{1.604005in}{1.429923in}}%
\pgfpathlineto{\pgfqpoint{1.622467in}{1.423326in}}%
\pgfpathlineto{\pgfqpoint{1.640543in}{1.416444in}}%
\pgfpathlineto{\pgfqpoint{1.658216in}{1.409282in}}%
\pgfpathlineto{\pgfqpoint{1.655551in}{1.393882in}}%
\pgfpathlineto{\pgfqpoint{1.652898in}{1.378957in}}%
\pgfpathlineto{\pgfqpoint{1.650259in}{1.364499in}}%
\pgfpathlineto{\pgfqpoint{1.647631in}{1.350499in}}%
\pgfpathlineto{\pgfqpoint{1.630360in}{1.357564in}}%
\pgfpathlineto{\pgfqpoint{1.612694in}{1.364353in}}%
\pgfpathlineto{\pgfqpoint{1.594649in}{1.370862in}}%
\pgfpathlineto{\pgfqpoint{1.576241in}{1.377084in}}%
\pgfpathclose%
\pgfusepath{fill}%
\end{pgfscope}%
\begin{pgfscope}%
\pgfpathrectangle{\pgfqpoint{0.041670in}{0.041670in}}{\pgfqpoint{2.216660in}{2.216660in}}%
\pgfusepath{clip}%
\pgfsetbuttcap%
\pgfsetroundjoin%
\definecolor{currentfill}{rgb}{0.272594,0.025563,0.353093}%
\pgfsetfillcolor{currentfill}%
\pgfsetlinewidth{0.000000pt}%
\definecolor{currentstroke}{rgb}{0.000000,0.000000,0.000000}%
\pgfsetstrokecolor{currentstroke}%
\pgfsetdash{}{0pt}%
\pgfpathmoveto{\pgfqpoint{0.715861in}{1.084252in}}%
\pgfpathlineto{\pgfqpoint{0.713056in}{1.087830in}}%
\pgfpathlineto{\pgfqpoint{0.710244in}{1.091698in}}%
\pgfpathlineto{\pgfqpoint{0.707426in}{1.095861in}}%
\pgfpathlineto{\pgfqpoint{0.704601in}{1.100323in}}%
\pgfpathlineto{\pgfqpoint{0.717679in}{1.107806in}}%
\pgfpathlineto{\pgfqpoint{0.731177in}{1.115069in}}%
\pgfpathlineto{\pgfqpoint{0.745081in}{1.122107in}}%
\pgfpathlineto{\pgfqpoint{0.759377in}{1.128913in}}%
\pgfpathlineto{\pgfqpoint{0.761866in}{1.124299in}}%
\pgfpathlineto{\pgfqpoint{0.764350in}{1.119984in}}%
\pgfpathlineto{\pgfqpoint{0.766828in}{1.115963in}}%
\pgfpathlineto{\pgfqpoint{0.769300in}{1.112231in}}%
\pgfpathlineto{\pgfqpoint{0.755352in}{1.105570in}}%
\pgfpathlineto{\pgfqpoint{0.741787in}{1.098683in}}%
\pgfpathlineto{\pgfqpoint{0.728619in}{1.091575in}}%
\pgfpathlineto{\pgfqpoint{0.715861in}{1.084252in}}%
\pgfpathclose%
\pgfusepath{fill}%
\end{pgfscope}%
\begin{pgfscope}%
\pgfpathrectangle{\pgfqpoint{0.041670in}{0.041670in}}{\pgfqpoint{2.216660in}{2.216660in}}%
\pgfusepath{clip}%
\pgfsetbuttcap%
\pgfsetroundjoin%
\definecolor{currentfill}{rgb}{0.271305,0.019942,0.347269}%
\pgfsetfillcolor{currentfill}%
\pgfsetlinewidth{0.000000pt}%
\definecolor{currentstroke}{rgb}{0.000000,0.000000,0.000000}%
\pgfsetstrokecolor{currentstroke}%
\pgfsetdash{}{0pt}%
\pgfpathmoveto{\pgfqpoint{0.817894in}{1.090820in}}%
\pgfpathlineto{\pgfqpoint{0.815492in}{1.089772in}}%
\pgfpathlineto{\pgfqpoint{0.813089in}{1.088920in}}%
\pgfpathlineto{\pgfqpoint{0.810683in}{1.088270in}}%
\pgfpathlineto{\pgfqpoint{0.808274in}{1.087825in}}%
\pgfpathlineto{\pgfqpoint{0.821203in}{1.093666in}}%
\pgfpathlineto{\pgfqpoint{0.834457in}{1.099292in}}%
\pgfpathlineto{\pgfqpoint{0.848024in}{1.104700in}}%
\pgfpathlineto{\pgfqpoint{0.861891in}{1.109885in}}%
\pgfpathlineto{\pgfqpoint{0.863945in}{1.110186in}}%
\pgfpathlineto{\pgfqpoint{0.865998in}{1.110692in}}%
\pgfpathlineto{\pgfqpoint{0.868048in}{1.111399in}}%
\pgfpathlineto{\pgfqpoint{0.870098in}{1.112301in}}%
\pgfpathlineto{\pgfqpoint{0.856595in}{1.107252in}}%
\pgfpathlineto{\pgfqpoint{0.843385in}{1.101986in}}%
\pgfpathlineto{\pgfqpoint{0.830480in}{1.096507in}}%
\pgfpathlineto{\pgfqpoint{0.817894in}{1.090820in}}%
\pgfpathclose%
\pgfusepath{fill}%
\end{pgfscope}%
\begin{pgfscope}%
\pgfpathrectangle{\pgfqpoint{0.041670in}{0.041670in}}{\pgfqpoint{2.216660in}{2.216660in}}%
\pgfusepath{clip}%
\pgfsetbuttcap%
\pgfsetroundjoin%
\definecolor{currentfill}{rgb}{0.279566,0.067836,0.391917}%
\pgfsetfillcolor{currentfill}%
\pgfsetlinewidth{0.000000pt}%
\definecolor{currentstroke}{rgb}{0.000000,0.000000,0.000000}%
\pgfsetstrokecolor{currentstroke}%
\pgfsetdash{}{0pt}%
\pgfpathmoveto{\pgfqpoint{0.886446in}{1.126103in}}%
\pgfpathlineto{\pgfqpoint{0.884406in}{1.123776in}}%
\pgfpathlineto{\pgfqpoint{0.882365in}{1.121613in}}%
\pgfpathlineto{\pgfqpoint{0.880323in}{1.119617in}}%
\pgfpathlineto{\pgfqpoint{0.878280in}{1.117794in}}%
\pgfpathlineto{\pgfqpoint{0.891694in}{1.122491in}}%
\pgfpathlineto{\pgfqpoint{0.905368in}{1.126970in}}%
\pgfpathlineto{\pgfqpoint{0.919291in}{1.131226in}}%
\pgfpathlineto{\pgfqpoint{0.933448in}{1.135257in}}%
\pgfpathlineto{\pgfqpoint{0.935113in}{1.136958in}}%
\pgfpathlineto{\pgfqpoint{0.936777in}{1.138830in}}%
\pgfpathlineto{\pgfqpoint{0.938441in}{1.140870in}}%
\pgfpathlineto{\pgfqpoint{0.940104in}{1.143074in}}%
\pgfpathlineto{\pgfqpoint{0.926333in}{1.139157in}}%
\pgfpathlineto{\pgfqpoint{0.912792in}{1.135020in}}%
\pgfpathlineto{\pgfqpoint{0.899492in}{1.130668in}}%
\pgfpathlineto{\pgfqpoint{0.886446in}{1.126103in}}%
\pgfpathclose%
\pgfusepath{fill}%
\end{pgfscope}%
\begin{pgfscope}%
\pgfpathrectangle{\pgfqpoint{0.041670in}{0.041670in}}{\pgfqpoint{2.216660in}{2.216660in}}%
\pgfusepath{clip}%
\pgfsetbuttcap%
\pgfsetroundjoin%
\definecolor{currentfill}{rgb}{0.282327,0.094955,0.417331}%
\pgfsetfillcolor{currentfill}%
\pgfsetlinewidth{0.000000pt}%
\definecolor{currentstroke}{rgb}{0.000000,0.000000,0.000000}%
\pgfsetstrokecolor{currentstroke}%
\pgfsetdash{}{0pt}%
\pgfpathmoveto{\pgfqpoint{0.693228in}{1.121281in}}%
\pgfpathlineto{\pgfqpoint{0.690365in}{1.127325in}}%
\pgfpathlineto{\pgfqpoint{0.687494in}{1.133701in}}%
\pgfpathlineto{\pgfqpoint{0.684615in}{1.140415in}}%
\pgfpathlineto{\pgfqpoint{0.681727in}{1.147473in}}%
\pgfpathlineto{\pgfqpoint{0.695460in}{1.155263in}}%
\pgfpathlineto{\pgfqpoint{0.709630in}{1.162823in}}%
\pgfpathlineto{\pgfqpoint{0.724223in}{1.170148in}}%
\pgfpathlineto{\pgfqpoint{0.739226in}{1.177231in}}%
\pgfpathlineto{\pgfqpoint{0.741770in}{1.170031in}}%
\pgfpathlineto{\pgfqpoint{0.744306in}{1.163174in}}%
\pgfpathlineto{\pgfqpoint{0.746835in}{1.156653in}}%
\pgfpathlineto{\pgfqpoint{0.749357in}{1.150465in}}%
\pgfpathlineto{\pgfqpoint{0.734710in}{1.143518in}}%
\pgfpathlineto{\pgfqpoint{0.720464in}{1.136334in}}%
\pgfpathlineto{\pgfqpoint{0.706632in}{1.128920in}}%
\pgfpathlineto{\pgfqpoint{0.693228in}{1.121281in}}%
\pgfpathclose%
\pgfusepath{fill}%
\end{pgfscope}%
\begin{pgfscope}%
\pgfpathrectangle{\pgfqpoint{0.041670in}{0.041670in}}{\pgfqpoint{2.216660in}{2.216660in}}%
\pgfusepath{clip}%
\pgfsetbuttcap%
\pgfsetroundjoin%
\definecolor{currentfill}{rgb}{0.280255,0.165693,0.476498}%
\pgfsetfillcolor{currentfill}%
\pgfsetlinewidth{0.000000pt}%
\definecolor{currentstroke}{rgb}{0.000000,0.000000,0.000000}%
\pgfsetstrokecolor{currentstroke}%
\pgfsetdash{}{0pt}%
\pgfpathmoveto{\pgfqpoint{1.123549in}{1.207322in}}%
\pgfpathlineto{\pgfqpoint{1.123124in}{1.203661in}}%
\pgfpathlineto{\pgfqpoint{1.122699in}{1.200118in}}%
\pgfpathlineto{\pgfqpoint{1.122274in}{1.196698in}}%
\pgfpathlineto{\pgfqpoint{1.121850in}{1.193404in}}%
\pgfpathlineto{\pgfqpoint{1.136536in}{1.194203in}}%
\pgfpathlineto{\pgfqpoint{1.151261in}{1.194771in}}%
\pgfpathlineto{\pgfqpoint{1.166012in}{1.195107in}}%
\pgfpathlineto{\pgfqpoint{1.180775in}{1.195211in}}%
\pgfpathlineto{\pgfqpoint{1.180769in}{1.198491in}}%
\pgfpathlineto{\pgfqpoint{1.180763in}{1.201897in}}%
\pgfpathlineto{\pgfqpoint{1.180757in}{1.205426in}}%
\pgfpathlineto{\pgfqpoint{1.180751in}{1.209073in}}%
\pgfpathlineto{\pgfqpoint{1.166420in}{1.208973in}}%
\pgfpathlineto{\pgfqpoint{1.152100in}{1.208648in}}%
\pgfpathlineto{\pgfqpoint{1.137806in}{1.208097in}}%
\pgfpathlineto{\pgfqpoint{1.123549in}{1.207322in}}%
\pgfpathclose%
\pgfusepath{fill}%
\end{pgfscope}%
\begin{pgfscope}%
\pgfpathrectangle{\pgfqpoint{0.041670in}{0.041670in}}{\pgfqpoint{2.216660in}{2.216660in}}%
\pgfusepath{clip}%
\pgfsetbuttcap%
\pgfsetroundjoin%
\definecolor{currentfill}{rgb}{0.280255,0.165693,0.476498}%
\pgfsetfillcolor{currentfill}%
\pgfsetlinewidth{0.000000pt}%
\definecolor{currentstroke}{rgb}{0.000000,0.000000,0.000000}%
\pgfsetstrokecolor{currentstroke}%
\pgfsetdash{}{0pt}%
\pgfpathmoveto{\pgfqpoint{1.180751in}{1.209073in}}%
\pgfpathlineto{\pgfqpoint{1.180757in}{1.205426in}}%
\pgfpathlineto{\pgfqpoint{1.180763in}{1.201897in}}%
\pgfpathlineto{\pgfqpoint{1.180769in}{1.198491in}}%
\pgfpathlineto{\pgfqpoint{1.180775in}{1.195211in}}%
\pgfpathlineto{\pgfqpoint{1.195538in}{1.195081in}}%
\pgfpathlineto{\pgfqpoint{1.210286in}{1.194720in}}%
\pgfpathlineto{\pgfqpoint{1.225008in}{1.194126in}}%
\pgfpathlineto{\pgfqpoint{1.239689in}{1.193300in}}%
\pgfpathlineto{\pgfqpoint{1.239252in}{1.196596in}}%
\pgfpathlineto{\pgfqpoint{1.238816in}{1.200017in}}%
\pgfpathlineto{\pgfqpoint{1.238379in}{1.203560in}}%
\pgfpathlineto{\pgfqpoint{1.237942in}{1.207222in}}%
\pgfpathlineto{\pgfqpoint{1.223690in}{1.208022in}}%
\pgfpathlineto{\pgfqpoint{1.209399in}{1.208597in}}%
\pgfpathlineto{\pgfqpoint{1.195082in}{1.208948in}}%
\pgfpathlineto{\pgfqpoint{1.180751in}{1.209073in}}%
\pgfpathclose%
\pgfusepath{fill}%
\end{pgfscope}%
\begin{pgfscope}%
\pgfpathrectangle{\pgfqpoint{0.041670in}{0.041670in}}{\pgfqpoint{2.216660in}{2.216660in}}%
\pgfusepath{clip}%
\pgfsetbuttcap%
\pgfsetroundjoin%
\definecolor{currentfill}{rgb}{0.268510,0.009605,0.335427}%
\pgfsetfillcolor{currentfill}%
\pgfsetlinewidth{0.000000pt}%
\definecolor{currentstroke}{rgb}{0.000000,0.000000,0.000000}%
\pgfsetstrokecolor{currentstroke}%
\pgfsetdash{}{0pt}%
\pgfpathmoveto{\pgfqpoint{0.727025in}{1.072737in}}%
\pgfpathlineto{\pgfqpoint{0.724242in}{1.075206in}}%
\pgfpathlineto{\pgfqpoint{0.721454in}{1.077945in}}%
\pgfpathlineto{\pgfqpoint{0.718661in}{1.080959in}}%
\pgfpathlineto{\pgfqpoint{0.715861in}{1.084252in}}%
\pgfpathlineto{\pgfqpoint{0.728619in}{1.091575in}}%
\pgfpathlineto{\pgfqpoint{0.741787in}{1.098683in}}%
\pgfpathlineto{\pgfqpoint{0.755352in}{1.105570in}}%
\pgfpathlineto{\pgfqpoint{0.769300in}{1.112231in}}%
\pgfpathlineto{\pgfqpoint{0.771768in}{1.108783in}}%
\pgfpathlineto{\pgfqpoint{0.774230in}{1.105613in}}%
\pgfpathlineto{\pgfqpoint{0.776687in}{1.102717in}}%
\pgfpathlineto{\pgfqpoint{0.779140in}{1.100090in}}%
\pgfpathlineto{\pgfqpoint{0.765535in}{1.093578in}}%
\pgfpathlineto{\pgfqpoint{0.752305in}{1.086844in}}%
\pgfpathlineto{\pgfqpoint{0.739464in}{1.079895in}}%
\pgfpathlineto{\pgfqpoint{0.727025in}{1.072737in}}%
\pgfpathclose%
\pgfusepath{fill}%
\end{pgfscope}%
\begin{pgfscope}%
\pgfpathrectangle{\pgfqpoint{0.041670in}{0.041670in}}{\pgfqpoint{2.216660in}{2.216660in}}%
\pgfusepath{clip}%
\pgfsetbuttcap%
\pgfsetroundjoin%
\definecolor{currentfill}{rgb}{0.276194,0.190074,0.493001}%
\pgfsetfillcolor{currentfill}%
\pgfsetlinewidth{0.000000pt}%
\definecolor{currentstroke}{rgb}{0.000000,0.000000,0.000000}%
\pgfsetstrokecolor{currentstroke}%
\pgfsetdash{}{0pt}%
\pgfpathmoveto{\pgfqpoint{1.616948in}{1.215776in}}%
\pgfpathlineto{\pgfqpoint{1.619452in}{1.224805in}}%
\pgfpathlineto{\pgfqpoint{1.621964in}{1.234212in}}%
\pgfpathlineto{\pgfqpoint{1.624485in}{1.244003in}}%
\pgfpathlineto{\pgfqpoint{1.627016in}{1.254185in}}%
\pgfpathlineto{\pgfqpoint{1.643109in}{1.247074in}}%
\pgfpathlineto{\pgfqpoint{1.658791in}{1.239706in}}%
\pgfpathlineto{\pgfqpoint{1.674045in}{1.232087in}}%
\pgfpathlineto{\pgfqpoint{1.688857in}{1.224225in}}%
\pgfpathlineto{\pgfqpoint{1.685964in}{1.214169in}}%
\pgfpathlineto{\pgfqpoint{1.683082in}{1.204505in}}%
\pgfpathlineto{\pgfqpoint{1.680210in}{1.195227in}}%
\pgfpathlineto{\pgfqpoint{1.677348in}{1.186328in}}%
\pgfpathlineto{\pgfqpoint{1.662883in}{1.194056in}}%
\pgfpathlineto{\pgfqpoint{1.647985in}{1.201543in}}%
\pgfpathlineto{\pgfqpoint{1.632669in}{1.208785in}}%
\pgfpathlineto{\pgfqpoint{1.616948in}{1.215776in}}%
\pgfpathclose%
\pgfusepath{fill}%
\end{pgfscope}%
\begin{pgfscope}%
\pgfpathrectangle{\pgfqpoint{0.041670in}{0.041670in}}{\pgfqpoint{2.216660in}{2.216660in}}%
\pgfusepath{clip}%
\pgfsetbuttcap%
\pgfsetroundjoin%
\definecolor{currentfill}{rgb}{0.282884,0.135920,0.453427}%
\pgfsetfillcolor{currentfill}%
\pgfsetlinewidth{0.000000pt}%
\definecolor{currentstroke}{rgb}{0.000000,0.000000,0.000000}%
\pgfsetstrokecolor{currentstroke}%
\pgfsetdash{}{0pt}%
\pgfpathmoveto{\pgfqpoint{0.681727in}{1.147473in}}%
\pgfpathlineto{\pgfqpoint{0.678829in}{1.154881in}}%
\pgfpathlineto{\pgfqpoint{0.675923in}{1.162645in}}%
\pgfpathlineto{\pgfqpoint{0.673006in}{1.170771in}}%
\pgfpathlineto{\pgfqpoint{0.670080in}{1.179264in}}%
\pgfpathlineto{\pgfqpoint{0.684147in}{1.187199in}}%
\pgfpathlineto{\pgfqpoint{0.698661in}{1.194900in}}%
\pgfpathlineto{\pgfqpoint{0.713606in}{1.202360in}}%
\pgfpathlineto{\pgfqpoint{0.728968in}{1.209574in}}%
\pgfpathlineto{\pgfqpoint{0.731545in}{1.200945in}}%
\pgfpathlineto{\pgfqpoint{0.734114in}{1.192682in}}%
\pgfpathlineto{\pgfqpoint{0.736674in}{1.184779in}}%
\pgfpathlineto{\pgfqpoint{0.739226in}{1.177231in}}%
\pgfpathlineto{\pgfqpoint{0.724223in}{1.170148in}}%
\pgfpathlineto{\pgfqpoint{0.709630in}{1.162823in}}%
\pgfpathlineto{\pgfqpoint{0.695460in}{1.155263in}}%
\pgfpathlineto{\pgfqpoint{0.681727in}{1.147473in}}%
\pgfpathclose%
\pgfusepath{fill}%
\end{pgfscope}%
\begin{pgfscope}%
\pgfpathrectangle{\pgfqpoint{0.041670in}{0.041670in}}{\pgfqpoint{2.216660in}{2.216660in}}%
\pgfusepath{clip}%
\pgfsetbuttcap%
\pgfsetroundjoin%
\definecolor{currentfill}{rgb}{0.267004,0.004874,0.329415}%
\pgfsetfillcolor{currentfill}%
\pgfsetlinewidth{0.000000pt}%
\definecolor{currentstroke}{rgb}{0.000000,0.000000,0.000000}%
\pgfsetstrokecolor{currentstroke}%
\pgfsetdash{}{0pt}%
\pgfpathmoveto{\pgfqpoint{1.549513in}{1.093520in}}%
\pgfpathlineto{\pgfqpoint{1.551858in}{1.094200in}}%
\pgfpathlineto{\pgfqpoint{1.554207in}{1.095111in}}%
\pgfpathlineto{\pgfqpoint{1.556559in}{1.096257in}}%
\pgfpathlineto{\pgfqpoint{1.558914in}{1.097645in}}%
\pgfpathlineto{\pgfqpoint{1.572494in}{1.091479in}}%
\pgfpathlineto{\pgfqpoint{1.585718in}{1.085094in}}%
\pgfpathlineto{\pgfqpoint{1.598573in}{1.078494in}}%
\pgfpathlineto{\pgfqpoint{1.611047in}{1.071684in}}%
\pgfpathlineto{\pgfqpoint{1.608358in}{1.070454in}}%
\pgfpathlineto{\pgfqpoint{1.605673in}{1.069465in}}%
\pgfpathlineto{\pgfqpoint{1.602992in}{1.068712in}}%
\pgfpathlineto{\pgfqpoint{1.600315in}{1.068192in}}%
\pgfpathlineto{\pgfqpoint{1.588161in}{1.074835in}}%
\pgfpathlineto{\pgfqpoint{1.575634in}{1.081274in}}%
\pgfpathlineto{\pgfqpoint{1.562748in}{1.087504in}}%
\pgfpathlineto{\pgfqpoint{1.549513in}{1.093520in}}%
\pgfpathclose%
\pgfusepath{fill}%
\end{pgfscope}%
\begin{pgfscope}%
\pgfpathrectangle{\pgfqpoint{0.041670in}{0.041670in}}{\pgfqpoint{2.216660in}{2.216660in}}%
\pgfusepath{clip}%
\pgfsetbuttcap%
\pgfsetroundjoin%
\definecolor{currentfill}{rgb}{0.274952,0.037752,0.364543}%
\pgfsetfillcolor{currentfill}%
\pgfsetlinewidth{0.000000pt}%
\definecolor{currentstroke}{rgb}{0.000000,0.000000,0.000000}%
\pgfsetstrokecolor{currentstroke}%
\pgfsetdash{}{0pt}%
\pgfpathmoveto{\pgfqpoint{1.469719in}{1.121980in}}%
\pgfpathlineto{\pgfqpoint{1.471681in}{1.120362in}}%
\pgfpathlineto{\pgfqpoint{1.473644in}{1.118924in}}%
\pgfpathlineto{\pgfqpoint{1.475609in}{1.117670in}}%
\pgfpathlineto{\pgfqpoint{1.477574in}{1.116604in}}%
\pgfpathlineto{\pgfqpoint{1.491327in}{1.111751in}}%
\pgfpathlineto{\pgfqpoint{1.504798in}{1.106678in}}%
\pgfpathlineto{\pgfqpoint{1.517974in}{1.101387in}}%
\pgfpathlineto{\pgfqpoint{1.530844in}{1.095885in}}%
\pgfpathlineto{\pgfqpoint{1.528520in}{1.097092in}}%
\pgfpathlineto{\pgfqpoint{1.526197in}{1.098486in}}%
\pgfpathlineto{\pgfqpoint{1.523876in}{1.100066in}}%
\pgfpathlineto{\pgfqpoint{1.521556in}{1.101825in}}%
\pgfpathlineto{\pgfqpoint{1.509034in}{1.107177in}}%
\pgfpathlineto{\pgfqpoint{1.496213in}{1.112323in}}%
\pgfpathlineto{\pgfqpoint{1.483103in}{1.117259in}}%
\pgfpathlineto{\pgfqpoint{1.469719in}{1.121980in}}%
\pgfpathclose%
\pgfusepath{fill}%
\end{pgfscope}%
\begin{pgfscope}%
\pgfpathrectangle{\pgfqpoint{0.041670in}{0.041670in}}{\pgfqpoint{2.216660in}{2.216660in}}%
\pgfusepath{clip}%
\pgfsetbuttcap%
\pgfsetroundjoin%
\definecolor{currentfill}{rgb}{0.280255,0.165693,0.476498}%
\pgfsetfillcolor{currentfill}%
\pgfsetlinewidth{0.000000pt}%
\definecolor{currentstroke}{rgb}{0.000000,0.000000,0.000000}%
\pgfsetstrokecolor{currentstroke}%
\pgfsetdash{}{0pt}%
\pgfpathmoveto{\pgfqpoint{1.067158in}{1.201993in}}%
\pgfpathlineto{\pgfqpoint{1.066308in}{1.198288in}}%
\pgfpathlineto{\pgfqpoint{1.065457in}{1.194702in}}%
\pgfpathlineto{\pgfqpoint{1.064607in}{1.191239in}}%
\pgfpathlineto{\pgfqpoint{1.063757in}{1.187902in}}%
\pgfpathlineto{\pgfqpoint{1.078156in}{1.189621in}}%
\pgfpathlineto{\pgfqpoint{1.092647in}{1.191111in}}%
\pgfpathlineto{\pgfqpoint{1.107216in}{1.192373in}}%
\pgfpathlineto{\pgfqpoint{1.121850in}{1.193404in}}%
\pgfpathlineto{\pgfqpoint{1.122274in}{1.196698in}}%
\pgfpathlineto{\pgfqpoint{1.122699in}{1.200118in}}%
\pgfpathlineto{\pgfqpoint{1.123124in}{1.203661in}}%
\pgfpathlineto{\pgfqpoint{1.123549in}{1.207322in}}%
\pgfpathlineto{\pgfqpoint{1.109343in}{1.206324in}}%
\pgfpathlineto{\pgfqpoint{1.095201in}{1.205102in}}%
\pgfpathlineto{\pgfqpoint{1.081135in}{1.203658in}}%
\pgfpathlineto{\pgfqpoint{1.067158in}{1.201993in}}%
\pgfpathclose%
\pgfusepath{fill}%
\end{pgfscope}%
\begin{pgfscope}%
\pgfpathrectangle{\pgfqpoint{0.041670in}{0.041670in}}{\pgfqpoint{2.216660in}{2.216660in}}%
\pgfusepath{clip}%
\pgfsetbuttcap%
\pgfsetroundjoin%
\definecolor{currentfill}{rgb}{0.280255,0.165693,0.476498}%
\pgfsetfillcolor{currentfill}%
\pgfsetlinewidth{0.000000pt}%
\definecolor{currentstroke}{rgb}{0.000000,0.000000,0.000000}%
\pgfsetstrokecolor{currentstroke}%
\pgfsetdash{}{0pt}%
\pgfpathmoveto{\pgfqpoint{1.237942in}{1.207222in}}%
\pgfpathlineto{\pgfqpoint{1.238379in}{1.203560in}}%
\pgfpathlineto{\pgfqpoint{1.238816in}{1.200017in}}%
\pgfpathlineto{\pgfqpoint{1.239252in}{1.196596in}}%
\pgfpathlineto{\pgfqpoint{1.239689in}{1.193300in}}%
\pgfpathlineto{\pgfqpoint{1.254316in}{1.192244in}}%
\pgfpathlineto{\pgfqpoint{1.268877in}{1.190957in}}%
\pgfpathlineto{\pgfqpoint{1.283358in}{1.189441in}}%
\pgfpathlineto{\pgfqpoint{1.297746in}{1.187697in}}%
\pgfpathlineto{\pgfqpoint{1.296885in}{1.191036in}}%
\pgfpathlineto{\pgfqpoint{1.296023in}{1.194501in}}%
\pgfpathlineto{\pgfqpoint{1.295161in}{1.198088in}}%
\pgfpathlineto{\pgfqpoint{1.294299in}{1.201794in}}%
\pgfpathlineto{\pgfqpoint{1.280332in}{1.203483in}}%
\pgfpathlineto{\pgfqpoint{1.266276in}{1.204952in}}%
\pgfpathlineto{\pgfqpoint{1.252141in}{1.206199in}}%
\pgfpathlineto{\pgfqpoint{1.237942in}{1.207222in}}%
\pgfpathclose%
\pgfusepath{fill}%
\end{pgfscope}%
\begin{pgfscope}%
\pgfpathrectangle{\pgfqpoint{0.041670in}{0.041670in}}{\pgfqpoint{2.216660in}{2.216660in}}%
\pgfusepath{clip}%
\pgfsetbuttcap%
\pgfsetroundjoin%
\definecolor{currentfill}{rgb}{0.267004,0.004874,0.329415}%
\pgfsetfillcolor{currentfill}%
\pgfsetlinewidth{0.000000pt}%
\definecolor{currentstroke}{rgb}{0.000000,0.000000,0.000000}%
\pgfsetstrokecolor{currentstroke}%
\pgfsetdash{}{0pt}%
\pgfpathmoveto{\pgfqpoint{0.738105in}{1.065461in}}%
\pgfpathlineto{\pgfqpoint{0.735342in}{1.066899in}}%
\pgfpathlineto{\pgfqpoint{0.732574in}{1.068588in}}%
\pgfpathlineto{\pgfqpoint{0.729802in}{1.070532in}}%
\pgfpathlineto{\pgfqpoint{0.727025in}{1.072737in}}%
\pgfpathlineto{\pgfqpoint{0.739464in}{1.079895in}}%
\pgfpathlineto{\pgfqpoint{0.752305in}{1.086844in}}%
\pgfpathlineto{\pgfqpoint{0.765535in}{1.093578in}}%
\pgfpathlineto{\pgfqpoint{0.779140in}{1.100090in}}%
\pgfpathlineto{\pgfqpoint{0.781588in}{1.097727in}}%
\pgfpathlineto{\pgfqpoint{0.784032in}{1.095624in}}%
\pgfpathlineto{\pgfqpoint{0.786472in}{1.093775in}}%
\pgfpathlineto{\pgfqpoint{0.788908in}{1.092175in}}%
\pgfpathlineto{\pgfqpoint{0.775643in}{1.085814in}}%
\pgfpathlineto{\pgfqpoint{0.762746in}{1.079238in}}%
\pgfpathlineto{\pgfqpoint{0.750229in}{1.072451in}}%
\pgfpathlineto{\pgfqpoint{0.738105in}{1.065461in}}%
\pgfpathclose%
\pgfusepath{fill}%
\end{pgfscope}%
\begin{pgfscope}%
\pgfpathrectangle{\pgfqpoint{0.041670in}{0.041670in}}{\pgfqpoint{2.216660in}{2.216660in}}%
\pgfusepath{clip}%
\pgfsetbuttcap%
\pgfsetroundjoin%
\definecolor{currentfill}{rgb}{0.201239,0.383670,0.554294}%
\pgfsetfillcolor{currentfill}%
\pgfsetlinewidth{0.000000pt}%
\definecolor{currentstroke}{rgb}{0.000000,0.000000,0.000000}%
\pgfsetstrokecolor{currentstroke}%
\pgfsetdash{}{0pt}%
\pgfpathmoveto{\pgfqpoint{0.697272in}{1.343992in}}%
\pgfpathlineto{\pgfqpoint{0.694558in}{1.357968in}}%
\pgfpathlineto{\pgfqpoint{0.691831in}{1.372404in}}%
\pgfpathlineto{\pgfqpoint{0.689091in}{1.387307in}}%
\pgfpathlineto{\pgfqpoint{0.686337in}{1.402686in}}%
\pgfpathlineto{\pgfqpoint{0.703637in}{1.410092in}}%
\pgfpathlineto{\pgfqpoint{0.721355in}{1.417223in}}%
\pgfpathlineto{\pgfqpoint{0.739475in}{1.424073in}}%
\pgfpathlineto{\pgfqpoint{0.757979in}{1.430638in}}%
\pgfpathlineto{\pgfqpoint{0.760323in}{1.415168in}}%
\pgfpathlineto{\pgfqpoint{0.762656in}{1.400172in}}%
\pgfpathlineto{\pgfqpoint{0.764978in}{1.385640in}}%
\pgfpathlineto{\pgfqpoint{0.767289in}{1.371567in}}%
\pgfpathlineto{\pgfqpoint{0.749202in}{1.365091in}}%
\pgfpathlineto{\pgfqpoint{0.731493in}{1.358332in}}%
\pgfpathlineto{\pgfqpoint{0.714178in}{1.351298in}}%
\pgfpathlineto{\pgfqpoint{0.697272in}{1.343992in}}%
\pgfpathclose%
\pgfusepath{fill}%
\end{pgfscope}%
\begin{pgfscope}%
\pgfpathrectangle{\pgfqpoint{0.041670in}{0.041670in}}{\pgfqpoint{2.216660in}{2.216660in}}%
\pgfusepath{clip}%
\pgfsetbuttcap%
\pgfsetroundjoin%
\definecolor{currentfill}{rgb}{0.283072,0.130895,0.449241}%
\pgfsetfillcolor{currentfill}%
\pgfsetlinewidth{0.000000pt}%
\definecolor{currentstroke}{rgb}{0.000000,0.000000,0.000000}%
\pgfsetstrokecolor{currentstroke}%
\pgfsetdash{}{0pt}%
\pgfpathmoveto{\pgfqpoint{1.340224in}{1.181112in}}%
\pgfpathlineto{\pgfqpoint{1.341397in}{1.177852in}}%
\pgfpathlineto{\pgfqpoint{1.342570in}{1.174725in}}%
\pgfpathlineto{\pgfqpoint{1.343743in}{1.171735in}}%
\pgfpathlineto{\pgfqpoint{1.344916in}{1.168886in}}%
\pgfpathlineto{\pgfqpoint{1.359212in}{1.166164in}}%
\pgfpathlineto{\pgfqpoint{1.373346in}{1.163215in}}%
\pgfpathlineto{\pgfqpoint{1.387307in}{1.160044in}}%
\pgfpathlineto{\pgfqpoint{1.401080in}{1.156651in}}%
\pgfpathlineto{\pgfqpoint{1.399505in}{1.159592in}}%
\pgfpathlineto{\pgfqpoint{1.397930in}{1.162674in}}%
\pgfpathlineto{\pgfqpoint{1.396355in}{1.165894in}}%
\pgfpathlineto{\pgfqpoint{1.394779in}{1.169247in}}%
\pgfpathlineto{\pgfqpoint{1.381401in}{1.172537in}}%
\pgfpathlineto{\pgfqpoint{1.367841in}{1.175613in}}%
\pgfpathlineto{\pgfqpoint{1.354111in}{1.178472in}}%
\pgfpathlineto{\pgfqpoint{1.340224in}{1.181112in}}%
\pgfpathclose%
\pgfusepath{fill}%
\end{pgfscope}%
\begin{pgfscope}%
\pgfpathrectangle{\pgfqpoint{0.041670in}{0.041670in}}{\pgfqpoint{2.216660in}{2.216660in}}%
\pgfusepath{clip}%
\pgfsetbuttcap%
\pgfsetroundjoin%
\definecolor{currentfill}{rgb}{0.282327,0.094955,0.417331}%
\pgfsetfillcolor{currentfill}%
\pgfsetlinewidth{0.000000pt}%
\definecolor{currentstroke}{rgb}{0.000000,0.000000,0.000000}%
\pgfsetstrokecolor{currentstroke}%
\pgfsetdash{}{0pt}%
\pgfpathmoveto{\pgfqpoint{1.401080in}{1.156651in}}%
\pgfpathlineto{\pgfqpoint{1.402655in}{1.153854in}}%
\pgfpathlineto{\pgfqpoint{1.404231in}{1.151206in}}%
\pgfpathlineto{\pgfqpoint{1.405806in}{1.148710in}}%
\pgfpathlineto{\pgfqpoint{1.407382in}{1.146370in}}%
\pgfpathlineto{\pgfqpoint{1.421346in}{1.142650in}}%
\pgfpathlineto{\pgfqpoint{1.435092in}{1.138708in}}%
\pgfpathlineto{\pgfqpoint{1.448608in}{1.134547in}}%
\pgfpathlineto{\pgfqpoint{1.461880in}{1.130171in}}%
\pgfpathlineto{\pgfqpoint{1.459922in}{1.132629in}}%
\pgfpathlineto{\pgfqpoint{1.457965in}{1.135243in}}%
\pgfpathlineto{\pgfqpoint{1.456007in}{1.138010in}}%
\pgfpathlineto{\pgfqpoint{1.454050in}{1.140925in}}%
\pgfpathlineto{\pgfqpoint{1.441151in}{1.145173in}}%
\pgfpathlineto{\pgfqpoint{1.428015in}{1.149212in}}%
\pgfpathlineto{\pgfqpoint{1.414654in}{1.153039in}}%
\pgfpathlineto{\pgfqpoint{1.401080in}{1.156651in}}%
\pgfpathclose%
\pgfusepath{fill}%
\end{pgfscope}%
\begin{pgfscope}%
\pgfpathrectangle{\pgfqpoint{0.041670in}{0.041670in}}{\pgfqpoint{2.216660in}{2.216660in}}%
\pgfusepath{clip}%
\pgfsetbuttcap%
\pgfsetroundjoin%
\definecolor{currentfill}{rgb}{0.276194,0.190074,0.493001}%
\pgfsetfillcolor{currentfill}%
\pgfsetlinewidth{0.000000pt}%
\definecolor{currentstroke}{rgb}{0.000000,0.000000,0.000000}%
\pgfsetstrokecolor{currentstroke}%
\pgfsetdash{}{0pt}%
\pgfpathmoveto{\pgfqpoint{0.670080in}{1.179264in}}%
\pgfpathlineto{\pgfqpoint{0.667143in}{1.188130in}}%
\pgfpathlineto{\pgfqpoint{0.664196in}{1.197377in}}%
\pgfpathlineto{\pgfqpoint{0.661239in}{1.207010in}}%
\pgfpathlineto{\pgfqpoint{0.658269in}{1.217036in}}%
\pgfpathlineto{\pgfqpoint{0.672676in}{1.225110in}}%
\pgfpathlineto{\pgfqpoint{0.687538in}{1.232946in}}%
\pgfpathlineto{\pgfqpoint{0.702840in}{1.240537in}}%
\pgfpathlineto{\pgfqpoint{0.718569in}{1.247876in}}%
\pgfpathlineto{\pgfqpoint{0.721183in}{1.237721in}}%
\pgfpathlineto{\pgfqpoint{0.723787in}{1.227956in}}%
\pgfpathlineto{\pgfqpoint{0.726382in}{1.218576in}}%
\pgfpathlineto{\pgfqpoint{0.728968in}{1.209574in}}%
\pgfpathlineto{\pgfqpoint{0.713606in}{1.202360in}}%
\pgfpathlineto{\pgfqpoint{0.698661in}{1.194900in}}%
\pgfpathlineto{\pgfqpoint{0.684147in}{1.187199in}}%
\pgfpathlineto{\pgfqpoint{0.670080in}{1.179264in}}%
\pgfpathclose%
\pgfusepath{fill}%
\end{pgfscope}%
\begin{pgfscope}%
\pgfpathrectangle{\pgfqpoint{0.041670in}{0.041670in}}{\pgfqpoint{2.216660in}{2.216660in}}%
\pgfusepath{clip}%
\pgfsetbuttcap%
\pgfsetroundjoin%
\definecolor{currentfill}{rgb}{0.260571,0.246922,0.522828}%
\pgfsetfillcolor{currentfill}%
\pgfsetlinewidth{0.000000pt}%
\definecolor{currentstroke}{rgb}{0.000000,0.000000,0.000000}%
\pgfsetstrokecolor{currentstroke}%
\pgfsetdash{}{0pt}%
\pgfpathmoveto{\pgfqpoint{1.627016in}{1.254185in}}%
\pgfpathlineto{\pgfqpoint{1.629556in}{1.264765in}}%
\pgfpathlineto{\pgfqpoint{1.632106in}{1.275748in}}%
\pgfpathlineto{\pgfqpoint{1.634667in}{1.287141in}}%
\pgfpathlineto{\pgfqpoint{1.637238in}{1.298951in}}%
\pgfpathlineto{\pgfqpoint{1.653710in}{1.291724in}}%
\pgfpathlineto{\pgfqpoint{1.669763in}{1.284237in}}%
\pgfpathlineto{\pgfqpoint{1.685379in}{1.276494in}}%
\pgfpathlineto{\pgfqpoint{1.700545in}{1.268503in}}%
\pgfpathlineto{\pgfqpoint{1.697605in}{1.256812in}}%
\pgfpathlineto{\pgfqpoint{1.694678in}{1.245540in}}%
\pgfpathlineto{\pgfqpoint{1.691762in}{1.234680in}}%
\pgfpathlineto{\pgfqpoint{1.688857in}{1.224225in}}%
\pgfpathlineto{\pgfqpoint{1.674045in}{1.232087in}}%
\pgfpathlineto{\pgfqpoint{1.658791in}{1.239706in}}%
\pgfpathlineto{\pgfqpoint{1.643109in}{1.247074in}}%
\pgfpathlineto{\pgfqpoint{1.627016in}{1.254185in}}%
\pgfpathclose%
\pgfusepath{fill}%
\end{pgfscope}%
\begin{pgfscope}%
\pgfpathrectangle{\pgfqpoint{0.041670in}{0.041670in}}{\pgfqpoint{2.216660in}{2.216660in}}%
\pgfusepath{clip}%
\pgfsetbuttcap%
\pgfsetroundjoin%
\definecolor{currentfill}{rgb}{0.274952,0.037752,0.364543}%
\pgfsetfillcolor{currentfill}%
\pgfsetlinewidth{0.000000pt}%
\definecolor{currentstroke}{rgb}{0.000000,0.000000,0.000000}%
\pgfsetstrokecolor{currentstroke}%
\pgfsetdash{}{0pt}%
\pgfpathmoveto{\pgfqpoint{0.827484in}{1.096898in}}%
\pgfpathlineto{\pgfqpoint{0.825089in}{1.095104in}}%
\pgfpathlineto{\pgfqpoint{0.822692in}{1.093490in}}%
\pgfpathlineto{\pgfqpoint{0.820294in}{1.092061in}}%
\pgfpathlineto{\pgfqpoint{0.817894in}{1.090820in}}%
\pgfpathlineto{\pgfqpoint{0.830480in}{1.096507in}}%
\pgfpathlineto{\pgfqpoint{0.843385in}{1.101986in}}%
\pgfpathlineto{\pgfqpoint{0.856595in}{1.107252in}}%
\pgfpathlineto{\pgfqpoint{0.870098in}{1.112301in}}%
\pgfpathlineto{\pgfqpoint{0.872145in}{1.113397in}}%
\pgfpathlineto{\pgfqpoint{0.874191in}{1.114680in}}%
\pgfpathlineto{\pgfqpoint{0.876236in}{1.116147in}}%
\pgfpathlineto{\pgfqpoint{0.878280in}{1.117794in}}%
\pgfpathlineto{\pgfqpoint{0.865140in}{1.112882in}}%
\pgfpathlineto{\pgfqpoint{0.852285in}{1.107759in}}%
\pgfpathlineto{\pgfqpoint{0.839729in}{1.102430in}}%
\pgfpathlineto{\pgfqpoint{0.827484in}{1.096898in}}%
\pgfpathclose%
\pgfusepath{fill}%
\end{pgfscope}%
\begin{pgfscope}%
\pgfpathrectangle{\pgfqpoint{0.041670in}{0.041670in}}{\pgfqpoint{2.216660in}{2.216660in}}%
\pgfusepath{clip}%
\pgfsetbuttcap%
\pgfsetroundjoin%
\definecolor{currentfill}{rgb}{0.283072,0.130895,0.449241}%
\pgfsetfillcolor{currentfill}%
\pgfsetlinewidth{0.000000pt}%
\definecolor{currentstroke}{rgb}{0.000000,0.000000,0.000000}%
\pgfsetstrokecolor{currentstroke}%
\pgfsetdash{}{0pt}%
\pgfpathmoveto{\pgfqpoint{0.953402in}{1.166144in}}%
\pgfpathlineto{\pgfqpoint{0.951740in}{1.162767in}}%
\pgfpathlineto{\pgfqpoint{0.950078in}{1.159523in}}%
\pgfpathlineto{\pgfqpoint{0.948416in}{1.156417in}}%
\pgfpathlineto{\pgfqpoint{0.946754in}{1.153451in}}%
\pgfpathlineto{\pgfqpoint{0.960350in}{1.157038in}}%
\pgfpathlineto{\pgfqpoint{0.974145in}{1.160407in}}%
\pgfpathlineto{\pgfqpoint{0.988125in}{1.163554in}}%
\pgfpathlineto{\pgfqpoint{1.002279in}{1.166477in}}%
\pgfpathlineto{\pgfqpoint{1.003542in}{1.169345in}}%
\pgfpathlineto{\pgfqpoint{1.004806in}{1.172353in}}%
\pgfpathlineto{\pgfqpoint{1.006070in}{1.175498in}}%
\pgfpathlineto{\pgfqpoint{1.007334in}{1.178776in}}%
\pgfpathlineto{\pgfqpoint{0.993586in}{1.175942in}}%
\pgfpathlineto{\pgfqpoint{0.980006in}{1.172889in}}%
\pgfpathlineto{\pgfqpoint{0.966607in}{1.169623in}}%
\pgfpathlineto{\pgfqpoint{0.953402in}{1.166144in}}%
\pgfpathclose%
\pgfusepath{fill}%
\end{pgfscope}%
\begin{pgfscope}%
\pgfpathrectangle{\pgfqpoint{0.041670in}{0.041670in}}{\pgfqpoint{2.216660in}{2.216660in}}%
\pgfusepath{clip}%
\pgfsetbuttcap%
\pgfsetroundjoin%
\definecolor{currentfill}{rgb}{0.267004,0.004874,0.329415}%
\pgfsetfillcolor{currentfill}%
\pgfsetlinewidth{0.000000pt}%
\definecolor{currentstroke}{rgb}{0.000000,0.000000,0.000000}%
\pgfsetstrokecolor{currentstroke}%
\pgfsetdash{}{0pt}%
\pgfpathmoveto{\pgfqpoint{0.749115in}{1.062120in}}%
\pgfpathlineto{\pgfqpoint{0.746368in}{1.062602in}}%
\pgfpathlineto{\pgfqpoint{0.743618in}{1.063317in}}%
\pgfpathlineto{\pgfqpoint{0.740863in}{1.064268in}}%
\pgfpathlineto{\pgfqpoint{0.738105in}{1.065461in}}%
\pgfpathlineto{\pgfqpoint{0.750229in}{1.072451in}}%
\pgfpathlineto{\pgfqpoint{0.762746in}{1.079238in}}%
\pgfpathlineto{\pgfqpoint{0.775643in}{1.085814in}}%
\pgfpathlineto{\pgfqpoint{0.788908in}{1.092175in}}%
\pgfpathlineto{\pgfqpoint{0.791340in}{1.090821in}}%
\pgfpathlineto{\pgfqpoint{0.793769in}{1.089707in}}%
\pgfpathlineto{\pgfqpoint{0.796194in}{1.088830in}}%
\pgfpathlineto{\pgfqpoint{0.798616in}{1.088184in}}%
\pgfpathlineto{\pgfqpoint{0.785690in}{1.081977in}}%
\pgfpathlineto{\pgfqpoint{0.773122in}{1.075561in}}%
\pgfpathlineto{\pgfqpoint{0.760927in}{1.068940in}}%
\pgfpathlineto{\pgfqpoint{0.749115in}{1.062120in}}%
\pgfpathclose%
\pgfusepath{fill}%
\end{pgfscope}%
\begin{pgfscope}%
\pgfpathrectangle{\pgfqpoint{0.041670in}{0.041670in}}{\pgfqpoint{2.216660in}{2.216660in}}%
\pgfusepath{clip}%
\pgfsetbuttcap%
\pgfsetroundjoin%
\definecolor{currentfill}{rgb}{0.268510,0.009605,0.335427}%
\pgfsetfillcolor{currentfill}%
\pgfsetlinewidth{0.000000pt}%
\definecolor{currentstroke}{rgb}{0.000000,0.000000,0.000000}%
\pgfsetstrokecolor{currentstroke}%
\pgfsetdash{}{0pt}%
\pgfpathmoveto{\pgfqpoint{1.540160in}{1.093027in}}%
\pgfpathlineto{\pgfqpoint{1.542494in}{1.092825in}}%
\pgfpathlineto{\pgfqpoint{1.544831in}{1.092838in}}%
\pgfpathlineto{\pgfqpoint{1.547171in}{1.093068in}}%
\pgfpathlineto{\pgfqpoint{1.549513in}{1.093520in}}%
\pgfpathlineto{\pgfqpoint{1.562748in}{1.087504in}}%
\pgfpathlineto{\pgfqpoint{1.575634in}{1.081274in}}%
\pgfpathlineto{\pgfqpoint{1.588161in}{1.074835in}}%
\pgfpathlineto{\pgfqpoint{1.600315in}{1.068192in}}%
\pgfpathlineto{\pgfqpoint{1.597641in}{1.067898in}}%
\pgfpathlineto{\pgfqpoint{1.594970in}{1.067828in}}%
\pgfpathlineto{\pgfqpoint{1.592302in}{1.067977in}}%
\pgfpathlineto{\pgfqpoint{1.589638in}{1.068340in}}%
\pgfpathlineto{\pgfqpoint{1.577802in}{1.074814in}}%
\pgfpathlineto{\pgfqpoint{1.565603in}{1.081091in}}%
\pgfpathlineto{\pgfqpoint{1.553051in}{1.087163in}}%
\pgfpathlineto{\pgfqpoint{1.540160in}{1.093027in}}%
\pgfpathclose%
\pgfusepath{fill}%
\end{pgfscope}%
\begin{pgfscope}%
\pgfpathrectangle{\pgfqpoint{0.041670in}{0.041670in}}{\pgfqpoint{2.216660in}{2.216660in}}%
\pgfusepath{clip}%
\pgfsetbuttcap%
\pgfsetroundjoin%
\definecolor{currentfill}{rgb}{0.172719,0.448791,0.557885}%
\pgfsetfillcolor{currentfill}%
\pgfsetlinewidth{0.000000pt}%
\definecolor{currentstroke}{rgb}{0.000000,0.000000,0.000000}%
\pgfsetstrokecolor{currentstroke}%
\pgfsetdash{}{0pt}%
\pgfpathmoveto{\pgfqpoint{1.585174in}{1.436229in}}%
\pgfpathlineto{\pgfqpoint{1.587434in}{1.452197in}}%
\pgfpathlineto{\pgfqpoint{1.589706in}{1.468653in}}%
\pgfpathlineto{\pgfqpoint{1.591989in}{1.485605in}}%
\pgfpathlineto{\pgfqpoint{1.611144in}{1.479240in}}%
\pgfpathlineto{\pgfqpoint{1.629924in}{1.472582in}}%
\pgfpathlineto{\pgfqpoint{1.648313in}{1.465635in}}%
\pgfpathlineto{\pgfqpoint{1.666292in}{1.458406in}}%
\pgfpathlineto{\pgfqpoint{1.663586in}{1.441536in}}%
\pgfpathlineto{\pgfqpoint{1.660894in}{1.425164in}}%
\pgfpathlineto{\pgfqpoint{1.658216in}{1.409282in}}%
\pgfpathlineto{\pgfqpoint{1.640543in}{1.416444in}}%
\pgfpathlineto{\pgfqpoint{1.622467in}{1.423326in}}%
\pgfpathlineto{\pgfqpoint{1.604005in}{1.429923in}}%
\pgfpathlineto{\pgfqpoint{1.585174in}{1.436229in}}%
\pgfpathclose%
\pgfusepath{fill}%
\end{pgfscope}%
\begin{pgfscope}%
\pgfpathrectangle{\pgfqpoint{0.041670in}{0.041670in}}{\pgfqpoint{2.216660in}{2.216660in}}%
\pgfusepath{clip}%
\pgfsetbuttcap%
\pgfsetroundjoin%
\definecolor{currentfill}{rgb}{0.280255,0.165693,0.476498}%
\pgfsetfillcolor{currentfill}%
\pgfsetlinewidth{0.000000pt}%
\definecolor{currentstroke}{rgb}{0.000000,0.000000,0.000000}%
\pgfsetstrokecolor{currentstroke}%
\pgfsetdash{}{0pt}%
\pgfpathmoveto{\pgfqpoint{1.294299in}{1.201794in}}%
\pgfpathlineto{\pgfqpoint{1.295161in}{1.198088in}}%
\pgfpathlineto{\pgfqpoint{1.296023in}{1.194501in}}%
\pgfpathlineto{\pgfqpoint{1.296885in}{1.191036in}}%
\pgfpathlineto{\pgfqpoint{1.297746in}{1.187697in}}%
\pgfpathlineto{\pgfqpoint{1.312028in}{1.185727in}}%
\pgfpathlineto{\pgfqpoint{1.326192in}{1.183531in}}%
\pgfpathlineto{\pgfqpoint{1.340224in}{1.181112in}}%
\pgfpathlineto{\pgfqpoint{1.339050in}{1.184502in}}%
\pgfpathlineto{\pgfqpoint{1.337877in}{1.188019in}}%
\pgfpathlineto{\pgfqpoint{1.336703in}{1.191657in}}%
\pgfpathlineto{\pgfqpoint{1.335529in}{1.195415in}}%
\pgfpathlineto{\pgfqpoint{1.321909in}{1.197758in}}%
\pgfpathlineto{\pgfqpoint{1.308162in}{1.199885in}}%
\pgfpathlineto{\pgfqpoint{1.294299in}{1.201794in}}%
\pgfpathclose%
\pgfusepath{fill}%
\end{pgfscope}%
\begin{pgfscope}%
\pgfpathrectangle{\pgfqpoint{0.041670in}{0.041670in}}{\pgfqpoint{2.216660in}{2.216660in}}%
\pgfusepath{clip}%
\pgfsetbuttcap%
\pgfsetroundjoin%
\definecolor{currentfill}{rgb}{0.282327,0.094955,0.417331}%
\pgfsetfillcolor{currentfill}%
\pgfsetlinewidth{0.000000pt}%
\definecolor{currentstroke}{rgb}{0.000000,0.000000,0.000000}%
\pgfsetstrokecolor{currentstroke}%
\pgfsetdash{}{0pt}%
\pgfpathmoveto{\pgfqpoint{0.894602in}{1.136976in}}%
\pgfpathlineto{\pgfqpoint{0.892564in}{1.134031in}}%
\pgfpathlineto{\pgfqpoint{0.890525in}{1.131235in}}%
\pgfpathlineto{\pgfqpoint{0.888486in}{1.128591in}}%
\pgfpathlineto{\pgfqpoint{0.886446in}{1.126103in}}%
\pgfpathlineto{\pgfqpoint{0.899492in}{1.130668in}}%
\pgfpathlineto{\pgfqpoint{0.912792in}{1.135020in}}%
\pgfpathlineto{\pgfqpoint{0.926333in}{1.139157in}}%
\pgfpathlineto{\pgfqpoint{0.940104in}{1.143074in}}%
\pgfpathlineto{\pgfqpoint{0.941767in}{1.145438in}}%
\pgfpathlineto{\pgfqpoint{0.943430in}{1.147958in}}%
\pgfpathlineto{\pgfqpoint{0.945092in}{1.150630in}}%
\pgfpathlineto{\pgfqpoint{0.946754in}{1.153451in}}%
\pgfpathlineto{\pgfqpoint{0.933369in}{1.149648in}}%
\pgfpathlineto{\pgfqpoint{0.920207in}{1.145632in}}%
\pgfpathlineto{\pgfqpoint{0.907281in}{1.141407in}}%
\pgfpathlineto{\pgfqpoint{0.894602in}{1.136976in}}%
\pgfpathclose%
\pgfusepath{fill}%
\end{pgfscope}%
\begin{pgfscope}%
\pgfpathrectangle{\pgfqpoint{0.041670in}{0.041670in}}{\pgfqpoint{2.216660in}{2.216660in}}%
\pgfusepath{clip}%
\pgfsetbuttcap%
\pgfsetroundjoin%
\definecolor{currentfill}{rgb}{0.280255,0.165693,0.476498}%
\pgfsetfillcolor{currentfill}%
\pgfsetlinewidth{0.000000pt}%
\definecolor{currentstroke}{rgb}{0.000000,0.000000,0.000000}%
\pgfsetstrokecolor{currentstroke}%
\pgfsetdash{}{0pt}%
\pgfpathmoveto{\pgfqpoint{1.012393in}{1.193153in}}%
\pgfpathlineto{\pgfqpoint{1.011128in}{1.189376in}}%
\pgfpathlineto{\pgfqpoint{1.009863in}{1.185719in}}%
\pgfpathlineto{\pgfqpoint{1.008599in}{1.182185in}}%
\pgfpathlineto{\pgfqpoint{1.007334in}{1.178776in}}%
\pgfpathlineto{\pgfqpoint{1.021238in}{1.181392in}}%
\pgfpathlineto{\pgfqpoint{1.035285in}{1.183786in}}%
\pgfpathlineto{\pgfqpoint{1.049463in}{1.185957in}}%
\pgfpathlineto{\pgfqpoint{1.063757in}{1.187902in}}%
\pgfpathlineto{\pgfqpoint{1.064607in}{1.191239in}}%
\pgfpathlineto{\pgfqpoint{1.065457in}{1.194702in}}%
\pgfpathlineto{\pgfqpoint{1.066308in}{1.198288in}}%
\pgfpathlineto{\pgfqpoint{1.067158in}{1.201993in}}%
\pgfpathlineto{\pgfqpoint{1.053283in}{1.200108in}}%
\pgfpathlineto{\pgfqpoint{1.039522in}{1.198005in}}%
\pgfpathlineto{\pgfqpoint{1.025888in}{1.195686in}}%
\pgfpathlineto{\pgfqpoint{1.012393in}{1.193153in}}%
\pgfpathclose%
\pgfusepath{fill}%
\end{pgfscope}%
\begin{pgfscope}%
\pgfpathrectangle{\pgfqpoint{0.041670in}{0.041670in}}{\pgfqpoint{2.216660in}{2.216660in}}%
\pgfusepath{clip}%
\pgfsetbuttcap%
\pgfsetroundjoin%
\definecolor{currentfill}{rgb}{0.279566,0.067836,0.391917}%
\pgfsetfillcolor{currentfill}%
\pgfsetlinewidth{0.000000pt}%
\definecolor{currentstroke}{rgb}{0.000000,0.000000,0.000000}%
\pgfsetstrokecolor{currentstroke}%
\pgfsetdash{}{0pt}%
\pgfpathmoveto{\pgfqpoint{1.461880in}{1.130171in}}%
\pgfpathlineto{\pgfqpoint{1.463839in}{1.127873in}}%
\pgfpathlineto{\pgfqpoint{1.465798in}{1.125740in}}%
\pgfpathlineto{\pgfqpoint{1.467758in}{1.123774in}}%
\pgfpathlineto{\pgfqpoint{1.469719in}{1.121980in}}%
\pgfpathlineto{\pgfqpoint{1.483103in}{1.117259in}}%
\pgfpathlineto{\pgfqpoint{1.496213in}{1.112323in}}%
\pgfpathlineto{\pgfqpoint{1.509034in}{1.107177in}}%
\pgfpathlineto{\pgfqpoint{1.521556in}{1.101825in}}%
\pgfpathlineto{\pgfqpoint{1.519238in}{1.103761in}}%
\pgfpathlineto{\pgfqpoint{1.516921in}{1.105869in}}%
\pgfpathlineto{\pgfqpoint{1.514604in}{1.108145in}}%
\pgfpathlineto{\pgfqpoint{1.512289in}{1.110586in}}%
\pgfpathlineto{\pgfqpoint{1.500113in}{1.115786in}}%
\pgfpathlineto{\pgfqpoint{1.487645in}{1.120787in}}%
\pgfpathlineto{\pgfqpoint{1.474897in}{1.125583in}}%
\pgfpathlineto{\pgfqpoint{1.461880in}{1.130171in}}%
\pgfpathclose%
\pgfusepath{fill}%
\end{pgfscope}%
\begin{pgfscope}%
\pgfpathrectangle{\pgfqpoint{0.041670in}{0.041670in}}{\pgfqpoint{2.216660in}{2.216660in}}%
\pgfusepath{clip}%
\pgfsetbuttcap%
\pgfsetroundjoin%
\definecolor{currentfill}{rgb}{0.274128,0.199721,0.498911}%
\pgfsetfillcolor{currentfill}%
\pgfsetlinewidth{0.000000pt}%
\definecolor{currentstroke}{rgb}{0.000000,0.000000,0.000000}%
\pgfsetstrokecolor{currentstroke}%
\pgfsetdash{}{0pt}%
\pgfpathmoveto{\pgfqpoint{1.125251in}{1.223086in}}%
\pgfpathlineto{\pgfqpoint{1.124825in}{1.218984in}}%
\pgfpathlineto{\pgfqpoint{1.124399in}{1.214988in}}%
\pgfpathlineto{\pgfqpoint{1.123974in}{1.211099in}}%
\pgfpathlineto{\pgfqpoint{1.123549in}{1.207322in}}%
\pgfpathlineto{\pgfqpoint{1.137806in}{1.208097in}}%
\pgfpathlineto{\pgfqpoint{1.152100in}{1.208648in}}%
\pgfpathlineto{\pgfqpoint{1.166420in}{1.208973in}}%
\pgfpathlineto{\pgfqpoint{1.180751in}{1.209073in}}%
\pgfpathlineto{\pgfqpoint{1.180745in}{1.212836in}}%
\pgfpathlineto{\pgfqpoint{1.180739in}{1.216710in}}%
\pgfpathlineto{\pgfqpoint{1.180733in}{1.220693in}}%
\pgfpathlineto{\pgfqpoint{1.180727in}{1.224780in}}%
\pgfpathlineto{\pgfqpoint{1.166828in}{1.224683in}}%
\pgfpathlineto{\pgfqpoint{1.152940in}{1.224368in}}%
\pgfpathlineto{\pgfqpoint{1.139077in}{1.223836in}}%
\pgfpathlineto{\pgfqpoint{1.125251in}{1.223086in}}%
\pgfpathclose%
\pgfusepath{fill}%
\end{pgfscope}%
\begin{pgfscope}%
\pgfpathrectangle{\pgfqpoint{0.041670in}{0.041670in}}{\pgfqpoint{2.216660in}{2.216660in}}%
\pgfusepath{clip}%
\pgfsetbuttcap%
\pgfsetroundjoin%
\definecolor{currentfill}{rgb}{0.274128,0.199721,0.498911}%
\pgfsetfillcolor{currentfill}%
\pgfsetlinewidth{0.000000pt}%
\definecolor{currentstroke}{rgb}{0.000000,0.000000,0.000000}%
\pgfsetstrokecolor{currentstroke}%
\pgfsetdash{}{0pt}%
\pgfpathmoveto{\pgfqpoint{1.180727in}{1.224780in}}%
\pgfpathlineto{\pgfqpoint{1.180733in}{1.220693in}}%
\pgfpathlineto{\pgfqpoint{1.180739in}{1.216710in}}%
\pgfpathlineto{\pgfqpoint{1.180745in}{1.212836in}}%
\pgfpathlineto{\pgfqpoint{1.180751in}{1.209073in}}%
\pgfpathlineto{\pgfqpoint{1.195082in}{1.208948in}}%
\pgfpathlineto{\pgfqpoint{1.209399in}{1.208597in}}%
\pgfpathlineto{\pgfqpoint{1.223690in}{1.208022in}}%
\pgfpathlineto{\pgfqpoint{1.237942in}{1.207222in}}%
\pgfpathlineto{\pgfqpoint{1.237505in}{1.211000in}}%
\pgfpathlineto{\pgfqpoint{1.237068in}{1.214889in}}%
\pgfpathlineto{\pgfqpoint{1.236630in}{1.218887in}}%
\pgfpathlineto{\pgfqpoint{1.236193in}{1.222989in}}%
\pgfpathlineto{\pgfqpoint{1.222371in}{1.223763in}}%
\pgfpathlineto{\pgfqpoint{1.208511in}{1.224320in}}%
\pgfpathlineto{\pgfqpoint{1.194626in}{1.224659in}}%
\pgfpathlineto{\pgfqpoint{1.180727in}{1.224780in}}%
\pgfpathclose%
\pgfusepath{fill}%
\end{pgfscope}%
\begin{pgfscope}%
\pgfpathrectangle{\pgfqpoint{0.041670in}{0.041670in}}{\pgfqpoint{2.216660in}{2.216660in}}%
\pgfusepath{clip}%
\pgfsetbuttcap%
\pgfsetroundjoin%
\definecolor{currentfill}{rgb}{0.260571,0.246922,0.522828}%
\pgfsetfillcolor{currentfill}%
\pgfsetlinewidth{0.000000pt}%
\definecolor{currentstroke}{rgb}{0.000000,0.000000,0.000000}%
\pgfsetstrokecolor{currentstroke}%
\pgfsetdash{}{0pt}%
\pgfpathmoveto{\pgfqpoint{0.658269in}{1.217036in}}%
\pgfpathlineto{\pgfqpoint{0.655289in}{1.227461in}}%
\pgfpathlineto{\pgfqpoint{0.652297in}{1.238291in}}%
\pgfpathlineto{\pgfqpoint{0.649292in}{1.249534in}}%
\pgfpathlineto{\pgfqpoint{0.646275in}{1.261196in}}%
\pgfpathlineto{\pgfqpoint{0.661027in}{1.269403in}}%
\pgfpathlineto{\pgfqpoint{0.676243in}{1.277367in}}%
\pgfpathlineto{\pgfqpoint{0.691909in}{1.285081in}}%
\pgfpathlineto{\pgfqpoint{0.708009in}{1.292540in}}%
\pgfpathlineto{\pgfqpoint{0.710665in}{1.280755in}}%
\pgfpathlineto{\pgfqpoint{0.713310in}{1.269387in}}%
\pgfpathlineto{\pgfqpoint{0.715944in}{1.258430in}}%
\pgfpathlineto{\pgfqpoint{0.718569in}{1.247876in}}%
\pgfpathlineto{\pgfqpoint{0.702840in}{1.240537in}}%
\pgfpathlineto{\pgfqpoint{0.687538in}{1.232946in}}%
\pgfpathlineto{\pgfqpoint{0.672676in}{1.225110in}}%
\pgfpathlineto{\pgfqpoint{0.658269in}{1.217036in}}%
\pgfpathclose%
\pgfusepath{fill}%
\end{pgfscope}%
\begin{pgfscope}%
\pgfpathrectangle{\pgfqpoint{0.041670in}{0.041670in}}{\pgfqpoint{2.216660in}{2.216660in}}%
\pgfusepath{clip}%
\pgfsetbuttcap%
\pgfsetroundjoin%
\definecolor{currentfill}{rgb}{0.268510,0.009605,0.335427}%
\pgfsetfillcolor{currentfill}%
\pgfsetlinewidth{0.000000pt}%
\definecolor{currentstroke}{rgb}{0.000000,0.000000,0.000000}%
\pgfsetstrokecolor{currentstroke}%
\pgfsetdash{}{0pt}%
\pgfpathmoveto{\pgfqpoint{0.760067in}{1.062423in}}%
\pgfpathlineto{\pgfqpoint{0.757334in}{1.062021in}}%
\pgfpathlineto{\pgfqpoint{0.754597in}{1.061834in}}%
\pgfpathlineto{\pgfqpoint{0.751858in}{1.061865in}}%
\pgfpathlineto{\pgfqpoint{0.749115in}{1.062120in}}%
\pgfpathlineto{\pgfqpoint{0.760927in}{1.068940in}}%
\pgfpathlineto{\pgfqpoint{0.773122in}{1.075561in}}%
\pgfpathlineto{\pgfqpoint{0.785690in}{1.081977in}}%
\pgfpathlineto{\pgfqpoint{0.798616in}{1.088184in}}%
\pgfpathlineto{\pgfqpoint{0.801035in}{1.087764in}}%
\pgfpathlineto{\pgfqpoint{0.803451in}{1.087568in}}%
\pgfpathlineto{\pgfqpoint{0.805864in}{1.087590in}}%
\pgfpathlineto{\pgfqpoint{0.808274in}{1.087825in}}%
\pgfpathlineto{\pgfqpoint{0.795684in}{1.081775in}}%
\pgfpathlineto{\pgfqpoint{0.783445in}{1.075522in}}%
\pgfpathlineto{\pgfqpoint{0.771569in}{1.069069in}}%
\pgfpathlineto{\pgfqpoint{0.760067in}{1.062423in}}%
\pgfpathclose%
\pgfusepath{fill}%
\end{pgfscope}%
\begin{pgfscope}%
\pgfpathrectangle{\pgfqpoint{0.041670in}{0.041670in}}{\pgfqpoint{2.216660in}{2.216660in}}%
\pgfusepath{clip}%
\pgfsetbuttcap%
\pgfsetroundjoin%
\definecolor{currentfill}{rgb}{0.172719,0.448791,0.557885}%
\pgfsetfillcolor{currentfill}%
\pgfsetlinewidth{0.000000pt}%
\definecolor{currentstroke}{rgb}{0.000000,0.000000,0.000000}%
\pgfsetstrokecolor{currentstroke}%
\pgfsetdash{}{0pt}%
\pgfpathmoveto{\pgfqpoint{0.686337in}{1.402686in}}%
\pgfpathlineto{\pgfqpoint{0.683570in}{1.418546in}}%
\pgfpathlineto{\pgfqpoint{0.680789in}{1.434898in}}%
\pgfpathlineto{\pgfqpoint{0.677993in}{1.451747in}}%
\pgfpathlineto{\pgfqpoint{0.695594in}{1.459223in}}%
\pgfpathlineto{\pgfqpoint{0.713620in}{1.466421in}}%
\pgfpathlineto{\pgfqpoint{0.732053in}{1.473336in}}%
\pgfpathlineto{\pgfqpoint{0.750876in}{1.479962in}}%
\pgfpathlineto{\pgfqpoint{0.753256in}{1.463027in}}%
\pgfpathlineto{\pgfqpoint{0.755623in}{1.446588in}}%
\pgfpathlineto{\pgfqpoint{0.757979in}{1.430638in}}%
\pgfpathlineto{\pgfqpoint{0.739475in}{1.424073in}}%
\pgfpathlineto{\pgfqpoint{0.721355in}{1.417223in}}%
\pgfpathlineto{\pgfqpoint{0.703637in}{1.410092in}}%
\pgfpathlineto{\pgfqpoint{0.686337in}{1.402686in}}%
\pgfpathclose%
\pgfusepath{fill}%
\end{pgfscope}%
\begin{pgfscope}%
\pgfpathrectangle{\pgfqpoint{0.041670in}{0.041670in}}{\pgfqpoint{2.216660in}{2.216660in}}%
\pgfusepath{clip}%
\pgfsetbuttcap%
\pgfsetroundjoin%
\definecolor{currentfill}{rgb}{0.271305,0.019942,0.347269}%
\pgfsetfillcolor{currentfill}%
\pgfsetlinewidth{0.000000pt}%
\definecolor{currentstroke}{rgb}{0.000000,0.000000,0.000000}%
\pgfsetstrokecolor{currentstroke}%
\pgfsetdash{}{0pt}%
\pgfpathmoveto{\pgfqpoint{1.530844in}{1.095885in}}%
\pgfpathlineto{\pgfqpoint{1.533170in}{1.094871in}}%
\pgfpathlineto{\pgfqpoint{1.535498in}{1.094054in}}%
\pgfpathlineto{\pgfqpoint{1.537828in}{1.093438in}}%
\pgfpathlineto{\pgfqpoint{1.540160in}{1.093027in}}%
\pgfpathlineto{\pgfqpoint{1.553051in}{1.087163in}}%
\pgfpathlineto{\pgfqpoint{1.565603in}{1.081091in}}%
\pgfpathlineto{\pgfqpoint{1.577802in}{1.074814in}}%
\pgfpathlineto{\pgfqpoint{1.589638in}{1.068340in}}%
\pgfpathlineto{\pgfqpoint{1.586976in}{1.068913in}}%
\pgfpathlineto{\pgfqpoint{1.584317in}{1.069691in}}%
\pgfpathlineto{\pgfqpoint{1.581660in}{1.070671in}}%
\pgfpathlineto{\pgfqpoint{1.579006in}{1.071849in}}%
\pgfpathlineto{\pgfqpoint{1.567487in}{1.078152in}}%
\pgfpathlineto{\pgfqpoint{1.555612in}{1.084263in}}%
\pgfpathlineto{\pgfqpoint{1.543394in}{1.090175in}}%
\pgfpathlineto{\pgfqpoint{1.530844in}{1.095885in}}%
\pgfpathclose%
\pgfusepath{fill}%
\end{pgfscope}%
\begin{pgfscope}%
\pgfpathrectangle{\pgfqpoint{0.041670in}{0.041670in}}{\pgfqpoint{2.216660in}{2.216660in}}%
\pgfusepath{clip}%
\pgfsetbuttcap%
\pgfsetroundjoin%
\definecolor{currentfill}{rgb}{0.233603,0.313828,0.543914}%
\pgfsetfillcolor{currentfill}%
\pgfsetlinewidth{0.000000pt}%
\definecolor{currentstroke}{rgb}{0.000000,0.000000,0.000000}%
\pgfsetstrokecolor{currentstroke}%
\pgfsetdash{}{0pt}%
\pgfpathmoveto{\pgfqpoint{1.637238in}{1.298951in}}%
\pgfpathlineto{\pgfqpoint{1.639819in}{1.311185in}}%
\pgfpathlineto{\pgfqpoint{1.642412in}{1.323850in}}%
\pgfpathlineto{\pgfqpoint{1.645016in}{1.336952in}}%
\pgfpathlineto{\pgfqpoint{1.647631in}{1.350499in}}%
\pgfpathlineto{\pgfqpoint{1.664490in}{1.343164in}}%
\pgfpathlineto{\pgfqpoint{1.680920in}{1.335564in}}%
\pgfpathlineto{\pgfqpoint{1.696906in}{1.327704in}}%
\pgfpathlineto{\pgfqpoint{1.712432in}{1.319592in}}%
\pgfpathlineto{\pgfqpoint{1.709440in}{1.306157in}}%
\pgfpathlineto{\pgfqpoint{1.706462in}{1.293168in}}%
\pgfpathlineto{\pgfqpoint{1.703497in}{1.280620in}}%
\pgfpathlineto{\pgfqpoint{1.700545in}{1.268503in}}%
\pgfpathlineto{\pgfqpoint{1.685379in}{1.276494in}}%
\pgfpathlineto{\pgfqpoint{1.669763in}{1.284237in}}%
\pgfpathlineto{\pgfqpoint{1.653710in}{1.291724in}}%
\pgfpathlineto{\pgfqpoint{1.637238in}{1.298951in}}%
\pgfpathclose%
\pgfusepath{fill}%
\end{pgfscope}%
\begin{pgfscope}%
\pgfpathrectangle{\pgfqpoint{0.041670in}{0.041670in}}{\pgfqpoint{2.216660in}{2.216660in}}%
\pgfusepath{clip}%
\pgfsetbuttcap%
\pgfsetroundjoin%
\definecolor{currentfill}{rgb}{0.279566,0.067836,0.391917}%
\pgfsetfillcolor{currentfill}%
\pgfsetlinewidth{0.000000pt}%
\definecolor{currentstroke}{rgb}{0.000000,0.000000,0.000000}%
\pgfsetstrokecolor{currentstroke}%
\pgfsetdash{}{0pt}%
\pgfpathmoveto{\pgfqpoint{0.837053in}{1.105799in}}%
\pgfpathlineto{\pgfqpoint{0.834662in}{1.103323in}}%
\pgfpathlineto{\pgfqpoint{0.832270in}{1.101012in}}%
\pgfpathlineto{\pgfqpoint{0.829878in}{1.098869in}}%
\pgfpathlineto{\pgfqpoint{0.827484in}{1.096898in}}%
\pgfpathlineto{\pgfqpoint{0.839729in}{1.102430in}}%
\pgfpathlineto{\pgfqpoint{0.852285in}{1.107759in}}%
\pgfpathlineto{\pgfqpoint{0.865140in}{1.112882in}}%
\pgfpathlineto{\pgfqpoint{0.878280in}{1.117794in}}%
\pgfpathlineto{\pgfqpoint{0.880323in}{1.119617in}}%
\pgfpathlineto{\pgfqpoint{0.882365in}{1.121613in}}%
\pgfpathlineto{\pgfqpoint{0.884406in}{1.123776in}}%
\pgfpathlineto{\pgfqpoint{0.886446in}{1.126103in}}%
\pgfpathlineto{\pgfqpoint{0.873667in}{1.121330in}}%
\pgfpathlineto{\pgfqpoint{0.861168in}{1.116352in}}%
\pgfpathlineto{\pgfqpoint{0.848959in}{1.111173in}}%
\pgfpathlineto{\pgfqpoint{0.837053in}{1.105799in}}%
\pgfpathclose%
\pgfusepath{fill}%
\end{pgfscope}%
\begin{pgfscope}%
\pgfpathrectangle{\pgfqpoint{0.041670in}{0.041670in}}{\pgfqpoint{2.216660in}{2.216660in}}%
\pgfusepath{clip}%
\pgfsetbuttcap%
\pgfsetroundjoin%
\definecolor{currentfill}{rgb}{0.274128,0.199721,0.498911}%
\pgfsetfillcolor{currentfill}%
\pgfsetlinewidth{0.000000pt}%
\definecolor{currentstroke}{rgb}{0.000000,0.000000,0.000000}%
\pgfsetstrokecolor{currentstroke}%
\pgfsetdash{}{0pt}%
\pgfpathmoveto{\pgfqpoint{1.070563in}{1.217929in}}%
\pgfpathlineto{\pgfqpoint{1.069711in}{1.213784in}}%
\pgfpathlineto{\pgfqpoint{1.068860in}{1.209744in}}%
\pgfpathlineto{\pgfqpoint{1.068009in}{1.205812in}}%
\pgfpathlineto{\pgfqpoint{1.067158in}{1.201993in}}%
\pgfpathlineto{\pgfqpoint{1.081135in}{1.203658in}}%
\pgfpathlineto{\pgfqpoint{1.095201in}{1.205102in}}%
\pgfpathlineto{\pgfqpoint{1.109343in}{1.206324in}}%
\pgfpathlineto{\pgfqpoint{1.123549in}{1.207322in}}%
\pgfpathlineto{\pgfqpoint{1.123974in}{1.211099in}}%
\pgfpathlineto{\pgfqpoint{1.124399in}{1.214988in}}%
\pgfpathlineto{\pgfqpoint{1.124825in}{1.218984in}}%
\pgfpathlineto{\pgfqpoint{1.125251in}{1.223086in}}%
\pgfpathlineto{\pgfqpoint{1.111474in}{1.222119in}}%
\pgfpathlineto{\pgfqpoint{1.097759in}{1.220937in}}%
\pgfpathlineto{\pgfqpoint{1.084118in}{1.219540in}}%
\pgfpathlineto{\pgfqpoint{1.070563in}{1.217929in}}%
\pgfpathclose%
\pgfusepath{fill}%
\end{pgfscope}%
\begin{pgfscope}%
\pgfpathrectangle{\pgfqpoint{0.041670in}{0.041670in}}{\pgfqpoint{2.216660in}{2.216660in}}%
\pgfusepath{clip}%
\pgfsetbuttcap%
\pgfsetroundjoin%
\definecolor{currentfill}{rgb}{0.274128,0.199721,0.498911}%
\pgfsetfillcolor{currentfill}%
\pgfsetlinewidth{0.000000pt}%
\definecolor{currentstroke}{rgb}{0.000000,0.000000,0.000000}%
\pgfsetstrokecolor{currentstroke}%
\pgfsetdash{}{0pt}%
\pgfpathmoveto{\pgfqpoint{1.236193in}{1.222989in}}%
\pgfpathlineto{\pgfqpoint{1.236630in}{1.218887in}}%
\pgfpathlineto{\pgfqpoint{1.237068in}{1.214889in}}%
\pgfpathlineto{\pgfqpoint{1.237505in}{1.211000in}}%
\pgfpathlineto{\pgfqpoint{1.237942in}{1.207222in}}%
\pgfpathlineto{\pgfqpoint{1.252141in}{1.206199in}}%
\pgfpathlineto{\pgfqpoint{1.266276in}{1.204952in}}%
\pgfpathlineto{\pgfqpoint{1.280332in}{1.203483in}}%
\pgfpathlineto{\pgfqpoint{1.294299in}{1.201794in}}%
\pgfpathlineto{\pgfqpoint{1.293436in}{1.205615in}}%
\pgfpathlineto{\pgfqpoint{1.292573in}{1.209549in}}%
\pgfpathlineto{\pgfqpoint{1.291710in}{1.213590in}}%
\pgfpathlineto{\pgfqpoint{1.290847in}{1.217737in}}%
\pgfpathlineto{\pgfqpoint{1.277303in}{1.219371in}}%
\pgfpathlineto{\pgfqpoint{1.263671in}{1.220792in}}%
\pgfpathlineto{\pgfqpoint{1.249963in}{1.221999in}}%
\pgfpathlineto{\pgfqpoint{1.236193in}{1.222989in}}%
\pgfpathclose%
\pgfusepath{fill}%
\end{pgfscope}%
\begin{pgfscope}%
\pgfpathrectangle{\pgfqpoint{0.041670in}{0.041670in}}{\pgfqpoint{2.216660in}{2.216660in}}%
\pgfusepath{clip}%
\pgfsetbuttcap%
\pgfsetroundjoin%
\definecolor{currentfill}{rgb}{0.280255,0.165693,0.476498}%
\pgfsetfillcolor{currentfill}%
\pgfsetlinewidth{0.000000pt}%
\definecolor{currentstroke}{rgb}{0.000000,0.000000,0.000000}%
\pgfsetstrokecolor{currentstroke}%
\pgfsetdash{}{0pt}%
\pgfpathmoveto{\pgfqpoint{1.335529in}{1.195415in}}%
\pgfpathlineto{\pgfqpoint{1.336703in}{1.191657in}}%
\pgfpathlineto{\pgfqpoint{1.337877in}{1.188019in}}%
\pgfpathlineto{\pgfqpoint{1.339050in}{1.184502in}}%
\pgfpathlineto{\pgfqpoint{1.340224in}{1.181112in}}%
\pgfpathlineto{\pgfqpoint{1.354111in}{1.178472in}}%
\pgfpathlineto{\pgfqpoint{1.367841in}{1.175613in}}%
\pgfpathlineto{\pgfqpoint{1.381401in}{1.172537in}}%
\pgfpathlineto{\pgfqpoint{1.394779in}{1.169247in}}%
\pgfpathlineto{\pgfqpoint{1.393204in}{1.172730in}}%
\pgfpathlineto{\pgfqpoint{1.391628in}{1.176339in}}%
\pgfpathlineto{\pgfqpoint{1.390052in}{1.180071in}}%
\pgfpathlineto{\pgfqpoint{1.388476in}{1.183922in}}%
\pgfpathlineto{\pgfqpoint{1.375493in}{1.187109in}}%
\pgfpathlineto{\pgfqpoint{1.362333in}{1.190088in}}%
\pgfpathlineto{\pgfqpoint{1.349007in}{1.192858in}}%
\pgfpathlineto{\pgfqpoint{1.335529in}{1.195415in}}%
\pgfpathclose%
\pgfusepath{fill}%
\end{pgfscope}%
\begin{pgfscope}%
\pgfpathrectangle{\pgfqpoint{0.041670in}{0.041670in}}{\pgfqpoint{2.216660in}{2.216660in}}%
\pgfusepath{clip}%
\pgfsetbuttcap%
\pgfsetroundjoin%
\definecolor{currentfill}{rgb}{0.283072,0.130895,0.449241}%
\pgfsetfillcolor{currentfill}%
\pgfsetlinewidth{0.000000pt}%
\definecolor{currentstroke}{rgb}{0.000000,0.000000,0.000000}%
\pgfsetstrokecolor{currentstroke}%
\pgfsetdash{}{0pt}%
\pgfpathmoveto{\pgfqpoint{1.394779in}{1.169247in}}%
\pgfpathlineto{\pgfqpoint{1.396355in}{1.165894in}}%
\pgfpathlineto{\pgfqpoint{1.397930in}{1.162674in}}%
\pgfpathlineto{\pgfqpoint{1.399505in}{1.159592in}}%
\pgfpathlineto{\pgfqpoint{1.401080in}{1.156651in}}%
\pgfpathlineto{\pgfqpoint{1.414654in}{1.153039in}}%
\pgfpathlineto{\pgfqpoint{1.428015in}{1.149212in}}%
\pgfpathlineto{\pgfqpoint{1.441151in}{1.145173in}}%
\pgfpathlineto{\pgfqpoint{1.454050in}{1.140925in}}%
\pgfpathlineto{\pgfqpoint{1.452093in}{1.143985in}}%
\pgfpathlineto{\pgfqpoint{1.450137in}{1.147187in}}%
\pgfpathlineto{\pgfqpoint{1.448180in}{1.150525in}}%
\pgfpathlineto{\pgfqpoint{1.446223in}{1.153998in}}%
\pgfpathlineto{\pgfqpoint{1.433697in}{1.158117in}}%
\pgfpathlineto{\pgfqpoint{1.420939in}{1.162034in}}%
\pgfpathlineto{\pgfqpoint{1.407963in}{1.165745in}}%
\pgfpathlineto{\pgfqpoint{1.394779in}{1.169247in}}%
\pgfpathclose%
\pgfusepath{fill}%
\end{pgfscope}%
\begin{pgfscope}%
\pgfpathrectangle{\pgfqpoint{0.041670in}{0.041670in}}{\pgfqpoint{2.216660in}{2.216660in}}%
\pgfusepath{clip}%
\pgfsetbuttcap%
\pgfsetroundjoin%
\definecolor{currentfill}{rgb}{0.272594,0.025563,0.353093}%
\pgfsetfillcolor{currentfill}%
\pgfsetlinewidth{0.000000pt}%
\definecolor{currentstroke}{rgb}{0.000000,0.000000,0.000000}%
\pgfsetstrokecolor{currentstroke}%
\pgfsetdash{}{0pt}%
\pgfpathmoveto{\pgfqpoint{1.632729in}{1.090772in}}%
\pgfpathlineto{\pgfqpoint{1.635463in}{1.094386in}}%
\pgfpathlineto{\pgfqpoint{1.638204in}{1.098289in}}%
\pgfpathlineto{\pgfqpoint{1.640951in}{1.102487in}}%
\pgfpathlineto{\pgfqpoint{1.643704in}{1.106986in}}%
\pgfpathlineto{\pgfqpoint{1.656736in}{1.099479in}}%
\pgfpathlineto{\pgfqpoint{1.669333in}{1.091759in}}%
\pgfpathlineto{\pgfqpoint{1.681484in}{1.083832in}}%
\pgfpathlineto{\pgfqpoint{1.693174in}{1.075706in}}%
\pgfpathlineto{\pgfqpoint{1.690114in}{1.071374in}}%
\pgfpathlineto{\pgfqpoint{1.687060in}{1.067344in}}%
\pgfpathlineto{\pgfqpoint{1.684014in}{1.063609in}}%
\pgfpathlineto{\pgfqpoint{1.680975in}{1.060165in}}%
\pgfpathlineto{\pgfqpoint{1.669576in}{1.068116in}}%
\pgfpathlineto{\pgfqpoint{1.657727in}{1.075871in}}%
\pgfpathlineto{\pgfqpoint{1.645440in}{1.083426in}}%
\pgfpathlineto{\pgfqpoint{1.632729in}{1.090772in}}%
\pgfpathclose%
\pgfusepath{fill}%
\end{pgfscope}%
\begin{pgfscope}%
\pgfpathrectangle{\pgfqpoint{0.041670in}{0.041670in}}{\pgfqpoint{2.216660in}{2.216660in}}%
\pgfusepath{clip}%
\pgfsetbuttcap%
\pgfsetroundjoin%
\definecolor{currentfill}{rgb}{0.277941,0.056324,0.381191}%
\pgfsetfillcolor{currentfill}%
\pgfsetlinewidth{0.000000pt}%
\definecolor{currentstroke}{rgb}{0.000000,0.000000,0.000000}%
\pgfsetstrokecolor{currentstroke}%
\pgfsetdash{}{0pt}%
\pgfpathmoveto{\pgfqpoint{1.643704in}{1.106986in}}%
\pgfpathlineto{\pgfqpoint{1.646465in}{1.111789in}}%
\pgfpathlineto{\pgfqpoint{1.649232in}{1.116902in}}%
\pgfpathlineto{\pgfqpoint{1.652007in}{1.122332in}}%
\pgfpathlineto{\pgfqpoint{1.654789in}{1.128082in}}%
\pgfpathlineto{\pgfqpoint{1.668144in}{1.120419in}}%
\pgfpathlineto{\pgfqpoint{1.681056in}{1.112538in}}%
\pgfpathlineto{\pgfqpoint{1.693512in}{1.104445in}}%
\pgfpathlineto{\pgfqpoint{1.705497in}{1.096148in}}%
\pgfpathlineto{\pgfqpoint{1.702404in}{1.090559in}}%
\pgfpathlineto{\pgfqpoint{1.699319in}{1.085293in}}%
\pgfpathlineto{\pgfqpoint{1.696243in}{1.080344in}}%
\pgfpathlineto{\pgfqpoint{1.693174in}{1.075706in}}%
\pgfpathlineto{\pgfqpoint{1.681484in}{1.083832in}}%
\pgfpathlineto{\pgfqpoint{1.669333in}{1.091759in}}%
\pgfpathlineto{\pgfqpoint{1.656736in}{1.099479in}}%
\pgfpathlineto{\pgfqpoint{1.643704in}{1.106986in}}%
\pgfpathclose%
\pgfusepath{fill}%
\end{pgfscope}%
\begin{pgfscope}%
\pgfpathrectangle{\pgfqpoint{0.041670in}{0.041670in}}{\pgfqpoint{2.216660in}{2.216660in}}%
\pgfusepath{clip}%
\pgfsetbuttcap%
\pgfsetroundjoin%
\definecolor{currentfill}{rgb}{0.280255,0.165693,0.476498}%
\pgfsetfillcolor{currentfill}%
\pgfsetlinewidth{0.000000pt}%
\definecolor{currentstroke}{rgb}{0.000000,0.000000,0.000000}%
\pgfsetstrokecolor{currentstroke}%
\pgfsetdash{}{0pt}%
\pgfpathmoveto{\pgfqpoint{0.960053in}{1.180917in}}%
\pgfpathlineto{\pgfqpoint{0.958389in}{1.177041in}}%
\pgfpathlineto{\pgfqpoint{0.956726in}{1.173285in}}%
\pgfpathlineto{\pgfqpoint{0.955064in}{1.169651in}}%
\pgfpathlineto{\pgfqpoint{0.953402in}{1.166144in}}%
\pgfpathlineto{\pgfqpoint{0.966607in}{1.169623in}}%
\pgfpathlineto{\pgfqpoint{0.980006in}{1.172889in}}%
\pgfpathlineto{\pgfqpoint{0.993586in}{1.175942in}}%
\pgfpathlineto{\pgfqpoint{1.007334in}{1.178776in}}%
\pgfpathlineto{\pgfqpoint{1.008599in}{1.182185in}}%
\pgfpathlineto{\pgfqpoint{1.009863in}{1.185719in}}%
\pgfpathlineto{\pgfqpoint{1.011128in}{1.189376in}}%
\pgfpathlineto{\pgfqpoint{1.012393in}{1.193153in}}%
\pgfpathlineto{\pgfqpoint{0.999050in}{1.190406in}}%
\pgfpathlineto{\pgfqpoint{0.985871in}{1.187450in}}%
\pgfpathlineto{\pgfqpoint{0.972867in}{1.184286in}}%
\pgfpathlineto{\pgfqpoint{0.960053in}{1.180917in}}%
\pgfpathclose%
\pgfusepath{fill}%
\end{pgfscope}%
\begin{pgfscope}%
\pgfpathrectangle{\pgfqpoint{0.041670in}{0.041670in}}{\pgfqpoint{2.216660in}{2.216660in}}%
\pgfusepath{clip}%
\pgfsetbuttcap%
\pgfsetroundjoin%
\definecolor{currentfill}{rgb}{0.271305,0.019942,0.347269}%
\pgfsetfillcolor{currentfill}%
\pgfsetlinewidth{0.000000pt}%
\definecolor{currentstroke}{rgb}{0.000000,0.000000,0.000000}%
\pgfsetstrokecolor{currentstroke}%
\pgfsetdash{}{0pt}%
\pgfpathmoveto{\pgfqpoint{0.770972in}{1.066088in}}%
\pgfpathlineto{\pgfqpoint{0.768250in}{1.064871in}}%
\pgfpathlineto{\pgfqpoint{0.765525in}{1.063852in}}%
\pgfpathlineto{\pgfqpoint{0.762797in}{1.063034in}}%
\pgfpathlineto{\pgfqpoint{0.760067in}{1.062423in}}%
\pgfpathlineto{\pgfqpoint{0.771569in}{1.069069in}}%
\pgfpathlineto{\pgfqpoint{0.783445in}{1.075522in}}%
\pgfpathlineto{\pgfqpoint{0.795684in}{1.081775in}}%
\pgfpathlineto{\pgfqpoint{0.808274in}{1.087825in}}%
\pgfpathlineto{\pgfqpoint{0.810683in}{1.088270in}}%
\pgfpathlineto{\pgfqpoint{0.813089in}{1.088920in}}%
\pgfpathlineto{\pgfqpoint{0.815492in}{1.089772in}}%
\pgfpathlineto{\pgfqpoint{0.817894in}{1.090820in}}%
\pgfpathlineto{\pgfqpoint{0.805638in}{1.084930in}}%
\pgfpathlineto{\pgfqpoint{0.793725in}{1.078841in}}%
\pgfpathlineto{\pgfqpoint{0.782166in}{1.072558in}}%
\pgfpathlineto{\pgfqpoint{0.770972in}{1.066088in}}%
\pgfpathclose%
\pgfusepath{fill}%
\end{pgfscope}%
\begin{pgfscope}%
\pgfpathrectangle{\pgfqpoint{0.041670in}{0.041670in}}{\pgfqpoint{2.216660in}{2.216660in}}%
\pgfusepath{clip}%
\pgfsetbuttcap%
\pgfsetroundjoin%
\definecolor{currentfill}{rgb}{0.283072,0.130895,0.449241}%
\pgfsetfillcolor{currentfill}%
\pgfsetlinewidth{0.000000pt}%
\definecolor{currentstroke}{rgb}{0.000000,0.000000,0.000000}%
\pgfsetstrokecolor{currentstroke}%
\pgfsetdash{}{0pt}%
\pgfpathmoveto{\pgfqpoint{0.902756in}{1.150169in}}%
\pgfpathlineto{\pgfqpoint{0.900717in}{1.146667in}}%
\pgfpathlineto{\pgfqpoint{0.898679in}{1.143298in}}%
\pgfpathlineto{\pgfqpoint{0.896641in}{1.140066in}}%
\pgfpathlineto{\pgfqpoint{0.894602in}{1.136976in}}%
\pgfpathlineto{\pgfqpoint{0.907281in}{1.141407in}}%
\pgfpathlineto{\pgfqpoint{0.920207in}{1.145632in}}%
\pgfpathlineto{\pgfqpoint{0.933369in}{1.149648in}}%
\pgfpathlineto{\pgfqpoint{0.946754in}{1.153451in}}%
\pgfpathlineto{\pgfqpoint{0.948416in}{1.156417in}}%
\pgfpathlineto{\pgfqpoint{0.950078in}{1.159523in}}%
\pgfpathlineto{\pgfqpoint{0.951740in}{1.162767in}}%
\pgfpathlineto{\pgfqpoint{0.953402in}{1.166144in}}%
\pgfpathlineto{\pgfqpoint{0.940402in}{1.162456in}}%
\pgfpathlineto{\pgfqpoint{0.927619in}{1.158562in}}%
\pgfpathlineto{\pgfqpoint{0.915067in}{1.154466in}}%
\pgfpathlineto{\pgfqpoint{0.902756in}{1.150169in}}%
\pgfpathclose%
\pgfusepath{fill}%
\end{pgfscope}%
\begin{pgfscope}%
\pgfpathrectangle{\pgfqpoint{0.041670in}{0.041670in}}{\pgfqpoint{2.216660in}{2.216660in}}%
\pgfusepath{clip}%
\pgfsetbuttcap%
\pgfsetroundjoin%
\definecolor{currentfill}{rgb}{0.268510,0.009605,0.335427}%
\pgfsetfillcolor{currentfill}%
\pgfsetlinewidth{0.000000pt}%
\definecolor{currentstroke}{rgb}{0.000000,0.000000,0.000000}%
\pgfsetstrokecolor{currentstroke}%
\pgfsetdash{}{0pt}%
\pgfpathmoveto{\pgfqpoint{1.621848in}{1.079110in}}%
\pgfpathlineto{\pgfqpoint{1.624560in}{1.081616in}}%
\pgfpathlineto{\pgfqpoint{1.627278in}{1.084392in}}%
\pgfpathlineto{\pgfqpoint{1.630000in}{1.087442in}}%
\pgfpathlineto{\pgfqpoint{1.632729in}{1.090772in}}%
\pgfpathlineto{\pgfqpoint{1.645440in}{1.083426in}}%
\pgfpathlineto{\pgfqpoint{1.657727in}{1.075871in}}%
\pgfpathlineto{\pgfqpoint{1.669576in}{1.068116in}}%
\pgfpathlineto{\pgfqpoint{1.680975in}{1.060165in}}%
\pgfpathlineto{\pgfqpoint{1.677943in}{1.057006in}}%
\pgfpathlineto{\pgfqpoint{1.674917in}{1.054127in}}%
\pgfpathlineto{\pgfqpoint{1.671897in}{1.051524in}}%
\pgfpathlineto{\pgfqpoint{1.668883in}{1.049192in}}%
\pgfpathlineto{\pgfqpoint{1.657772in}{1.056963in}}%
\pgfpathlineto{\pgfqpoint{1.646221in}{1.064544in}}%
\pgfpathlineto{\pgfqpoint{1.634242in}{1.071929in}}%
\pgfpathlineto{\pgfqpoint{1.621848in}{1.079110in}}%
\pgfpathclose%
\pgfusepath{fill}%
\end{pgfscope}%
\begin{pgfscope}%
\pgfpathrectangle{\pgfqpoint{0.041670in}{0.041670in}}{\pgfqpoint{2.216660in}{2.216660in}}%
\pgfusepath{clip}%
\pgfsetbuttcap%
\pgfsetroundjoin%
\definecolor{currentfill}{rgb}{0.282327,0.094955,0.417331}%
\pgfsetfillcolor{currentfill}%
\pgfsetlinewidth{0.000000pt}%
\definecolor{currentstroke}{rgb}{0.000000,0.000000,0.000000}%
\pgfsetstrokecolor{currentstroke}%
\pgfsetdash{}{0pt}%
\pgfpathmoveto{\pgfqpoint{1.654789in}{1.128082in}}%
\pgfpathlineto{\pgfqpoint{1.657579in}{1.134160in}}%
\pgfpathlineto{\pgfqpoint{1.660377in}{1.140570in}}%
\pgfpathlineto{\pgfqpoint{1.663183in}{1.147317in}}%
\pgfpathlineto{\pgfqpoint{1.665998in}{1.154409in}}%
\pgfpathlineto{\pgfqpoint{1.679681in}{1.146594in}}%
\pgfpathlineto{\pgfqpoint{1.692912in}{1.138557in}}%
\pgfpathlineto{\pgfqpoint{1.705677in}{1.130303in}}%
\pgfpathlineto{\pgfqpoint{1.717962in}{1.121841in}}%
\pgfpathlineto{\pgfqpoint{1.714831in}{1.114906in}}%
\pgfpathlineto{\pgfqpoint{1.711711in}{1.108316in}}%
\pgfpathlineto{\pgfqpoint{1.708599in}{1.102065in}}%
\pgfpathlineto{\pgfqpoint{1.705497in}{1.096148in}}%
\pgfpathlineto{\pgfqpoint{1.693512in}{1.104445in}}%
\pgfpathlineto{\pgfqpoint{1.681056in}{1.112538in}}%
\pgfpathlineto{\pgfqpoint{1.668144in}{1.120419in}}%
\pgfpathlineto{\pgfqpoint{1.654789in}{1.128082in}}%
\pgfpathclose%
\pgfusepath{fill}%
\end{pgfscope}%
\begin{pgfscope}%
\pgfpathrectangle{\pgfqpoint{0.041670in}{0.041670in}}{\pgfqpoint{2.216660in}{2.216660in}}%
\pgfusepath{clip}%
\pgfsetbuttcap%
\pgfsetroundjoin%
\definecolor{currentfill}{rgb}{0.233603,0.313828,0.543914}%
\pgfsetfillcolor{currentfill}%
\pgfsetlinewidth{0.000000pt}%
\definecolor{currentstroke}{rgb}{0.000000,0.000000,0.000000}%
\pgfsetstrokecolor{currentstroke}%
\pgfsetdash{}{0pt}%
\pgfpathmoveto{\pgfqpoint{0.646275in}{1.261196in}}%
\pgfpathlineto{\pgfqpoint{0.643246in}{1.273284in}}%
\pgfpathlineto{\pgfqpoint{0.640203in}{1.285805in}}%
\pgfpathlineto{\pgfqpoint{0.637147in}{1.298766in}}%
\pgfpathlineto{\pgfqpoint{0.634076in}{1.312174in}}%
\pgfpathlineto{\pgfqpoint{0.649180in}{1.320506in}}%
\pgfpathlineto{\pgfqpoint{0.664757in}{1.328590in}}%
\pgfpathlineto{\pgfqpoint{0.680793in}{1.336421in}}%
\pgfpathlineto{\pgfqpoint{0.697272in}{1.343992in}}%
\pgfpathlineto{\pgfqpoint{0.699974in}{1.330468in}}%
\pgfpathlineto{\pgfqpoint{0.702664in}{1.317390in}}%
\pgfpathlineto{\pgfqpoint{0.705343in}{1.304750in}}%
\pgfpathlineto{\pgfqpoint{0.708009in}{1.292540in}}%
\pgfpathlineto{\pgfqpoint{0.691909in}{1.285081in}}%
\pgfpathlineto{\pgfqpoint{0.676243in}{1.277367in}}%
\pgfpathlineto{\pgfqpoint{0.661027in}{1.269403in}}%
\pgfpathlineto{\pgfqpoint{0.646275in}{1.261196in}}%
\pgfpathclose%
\pgfusepath{fill}%
\end{pgfscope}%
\begin{pgfscope}%
\pgfpathrectangle{\pgfqpoint{0.041670in}{0.041670in}}{\pgfqpoint{2.216660in}{2.216660in}}%
\pgfusepath{clip}%
\pgfsetbuttcap%
\pgfsetroundjoin%
\definecolor{currentfill}{rgb}{0.274128,0.199721,0.498911}%
\pgfsetfillcolor{currentfill}%
\pgfsetlinewidth{0.000000pt}%
\definecolor{currentstroke}{rgb}{0.000000,0.000000,0.000000}%
\pgfsetstrokecolor{currentstroke}%
\pgfsetdash{}{0pt}%
\pgfpathmoveto{\pgfqpoint{1.290847in}{1.217737in}}%
\pgfpathlineto{\pgfqpoint{1.291710in}{1.213590in}}%
\pgfpathlineto{\pgfqpoint{1.292573in}{1.209549in}}%
\pgfpathlineto{\pgfqpoint{1.293436in}{1.205615in}}%
\pgfpathlineto{\pgfqpoint{1.294299in}{1.201794in}}%
\pgfpathlineto{\pgfqpoint{1.308162in}{1.199885in}}%
\pgfpathlineto{\pgfqpoint{1.321909in}{1.197758in}}%
\pgfpathlineto{\pgfqpoint{1.335529in}{1.195415in}}%
\pgfpathlineto{\pgfqpoint{1.334354in}{1.199288in}}%
\pgfpathlineto{\pgfqpoint{1.333179in}{1.203273in}}%
\pgfpathlineto{\pgfqpoint{1.332003in}{1.207367in}}%
\pgfpathlineto{\pgfqpoint{1.330827in}{1.211565in}}%
\pgfpathlineto{\pgfqpoint{1.317621in}{1.213832in}}%
\pgfpathlineto{\pgfqpoint{1.304290in}{1.215890in}}%
\pgfpathlineto{\pgfqpoint{1.290847in}{1.217737in}}%
\pgfpathclose%
\pgfusepath{fill}%
\end{pgfscope}%
\begin{pgfscope}%
\pgfpathrectangle{\pgfqpoint{0.041670in}{0.041670in}}{\pgfqpoint{2.216660in}{2.216660in}}%
\pgfusepath{clip}%
\pgfsetbuttcap%
\pgfsetroundjoin%
\definecolor{currentfill}{rgb}{0.282327,0.094955,0.417331}%
\pgfsetfillcolor{currentfill}%
\pgfsetlinewidth{0.000000pt}%
\definecolor{currentstroke}{rgb}{0.000000,0.000000,0.000000}%
\pgfsetstrokecolor{currentstroke}%
\pgfsetdash{}{0pt}%
\pgfpathmoveto{\pgfqpoint{1.454050in}{1.140925in}}%
\pgfpathlineto{\pgfqpoint{1.456007in}{1.138010in}}%
\pgfpathlineto{\pgfqpoint{1.457965in}{1.135243in}}%
\pgfpathlineto{\pgfqpoint{1.459922in}{1.132629in}}%
\pgfpathlineto{\pgfqpoint{1.461880in}{1.130171in}}%
\pgfpathlineto{\pgfqpoint{1.474897in}{1.125583in}}%
\pgfpathlineto{\pgfqpoint{1.487645in}{1.120787in}}%
\pgfpathlineto{\pgfqpoint{1.500113in}{1.115786in}}%
\pgfpathlineto{\pgfqpoint{1.512289in}{1.110586in}}%
\pgfpathlineto{\pgfqpoint{1.509974in}{1.113187in}}%
\pgfpathlineto{\pgfqpoint{1.507660in}{1.115945in}}%
\pgfpathlineto{\pgfqpoint{1.505347in}{1.118855in}}%
\pgfpathlineto{\pgfqpoint{1.503034in}{1.121915in}}%
\pgfpathlineto{\pgfqpoint{1.491204in}{1.126962in}}%
\pgfpathlineto{\pgfqpoint{1.479088in}{1.131816in}}%
\pgfpathlineto{\pgfqpoint{1.466700in}{1.136471in}}%
\pgfpathlineto{\pgfqpoint{1.454050in}{1.140925in}}%
\pgfpathclose%
\pgfusepath{fill}%
\end{pgfscope}%
\begin{pgfscope}%
\pgfpathrectangle{\pgfqpoint{0.041670in}{0.041670in}}{\pgfqpoint{2.216660in}{2.216660in}}%
\pgfusepath{clip}%
\pgfsetbuttcap%
\pgfsetroundjoin%
\definecolor{currentfill}{rgb}{0.274128,0.199721,0.498911}%
\pgfsetfillcolor{currentfill}%
\pgfsetlinewidth{0.000000pt}%
\definecolor{currentstroke}{rgb}{0.000000,0.000000,0.000000}%
\pgfsetstrokecolor{currentstroke}%
\pgfsetdash{}{0pt}%
\pgfpathmoveto{\pgfqpoint{1.017458in}{1.209376in}}%
\pgfpathlineto{\pgfqpoint{1.016191in}{1.205159in}}%
\pgfpathlineto{\pgfqpoint{1.014924in}{1.201047in}}%
\pgfpathlineto{\pgfqpoint{1.013658in}{1.197044in}}%
\pgfpathlineto{\pgfqpoint{1.012393in}{1.193153in}}%
\pgfpathlineto{\pgfqpoint{1.025888in}{1.195686in}}%
\pgfpathlineto{\pgfqpoint{1.039522in}{1.198005in}}%
\pgfpathlineto{\pgfqpoint{1.053283in}{1.200108in}}%
\pgfpathlineto{\pgfqpoint{1.067158in}{1.201993in}}%
\pgfpathlineto{\pgfqpoint{1.068009in}{1.205812in}}%
\pgfpathlineto{\pgfqpoint{1.068860in}{1.209744in}}%
\pgfpathlineto{\pgfqpoint{1.069711in}{1.213784in}}%
\pgfpathlineto{\pgfqpoint{1.070563in}{1.217929in}}%
\pgfpathlineto{\pgfqpoint{1.057108in}{1.216105in}}%
\pgfpathlineto{\pgfqpoint{1.043764in}{1.214071in}}%
\pgfpathlineto{\pgfqpoint{1.030543in}{1.211827in}}%
\pgfpathlineto{\pgfqpoint{1.017458in}{1.209376in}}%
\pgfpathclose%
\pgfusepath{fill}%
\end{pgfscope}%
\begin{pgfscope}%
\pgfpathrectangle{\pgfqpoint{0.041670in}{0.041670in}}{\pgfqpoint{2.216660in}{2.216660in}}%
\pgfusepath{clip}%
\pgfsetbuttcap%
\pgfsetroundjoin%
\definecolor{currentfill}{rgb}{0.274952,0.037752,0.364543}%
\pgfsetfillcolor{currentfill}%
\pgfsetlinewidth{0.000000pt}%
\definecolor{currentstroke}{rgb}{0.000000,0.000000,0.000000}%
\pgfsetstrokecolor{currentstroke}%
\pgfsetdash{}{0pt}%
\pgfpathmoveto{\pgfqpoint{1.521556in}{1.101825in}}%
\pgfpathlineto{\pgfqpoint{1.523876in}{1.100066in}}%
\pgfpathlineto{\pgfqpoint{1.526197in}{1.098486in}}%
\pgfpathlineto{\pgfqpoint{1.528520in}{1.097092in}}%
\pgfpathlineto{\pgfqpoint{1.530844in}{1.095885in}}%
\pgfpathlineto{\pgfqpoint{1.543394in}{1.090175in}}%
\pgfpathlineto{\pgfqpoint{1.555612in}{1.084263in}}%
\pgfpathlineto{\pgfqpoint{1.567487in}{1.078152in}}%
\pgfpathlineto{\pgfqpoint{1.579006in}{1.071849in}}%
\pgfpathlineto{\pgfqpoint{1.576354in}{1.073219in}}%
\pgfpathlineto{\pgfqpoint{1.573704in}{1.074778in}}%
\pgfpathlineto{\pgfqpoint{1.571055in}{1.076522in}}%
\pgfpathlineto{\pgfqpoint{1.568409in}{1.078447in}}%
\pgfpathlineto{\pgfqpoint{1.557204in}{1.084577in}}%
\pgfpathlineto{\pgfqpoint{1.545653in}{1.090520in}}%
\pgfpathlineto{\pgfqpoint{1.533767in}{1.096271in}}%
\pgfpathlineto{\pgfqpoint{1.521556in}{1.101825in}}%
\pgfpathclose%
\pgfusepath{fill}%
\end{pgfscope}%
\begin{pgfscope}%
\pgfpathrectangle{\pgfqpoint{0.041670in}{0.041670in}}{\pgfqpoint{2.216660in}{2.216660in}}%
\pgfusepath{clip}%
\pgfsetbuttcap%
\pgfsetroundjoin%
\definecolor{currentfill}{rgb}{0.267004,0.004874,0.329415}%
\pgfsetfillcolor{currentfill}%
\pgfsetlinewidth{0.000000pt}%
\definecolor{currentstroke}{rgb}{0.000000,0.000000,0.000000}%
\pgfsetstrokecolor{currentstroke}%
\pgfsetdash{}{0pt}%
\pgfpathmoveto{\pgfqpoint{1.611047in}{1.071684in}}%
\pgfpathlineto{\pgfqpoint{1.613741in}{1.073160in}}%
\pgfpathlineto{\pgfqpoint{1.616438in}{1.074887in}}%
\pgfpathlineto{\pgfqpoint{1.619141in}{1.076869in}}%
\pgfpathlineto{\pgfqpoint{1.621848in}{1.079110in}}%
\pgfpathlineto{\pgfqpoint{1.634242in}{1.071929in}}%
\pgfpathlineto{\pgfqpoint{1.646221in}{1.064544in}}%
\pgfpathlineto{\pgfqpoint{1.657772in}{1.056963in}}%
\pgfpathlineto{\pgfqpoint{1.668883in}{1.049192in}}%
\pgfpathlineto{\pgfqpoint{1.665875in}{1.047125in}}%
\pgfpathlineto{\pgfqpoint{1.662873in}{1.045319in}}%
\pgfpathlineto{\pgfqpoint{1.659875in}{1.043769in}}%
\pgfpathlineto{\pgfqpoint{1.656883in}{1.042470in}}%
\pgfpathlineto{\pgfqpoint{1.646057in}{1.050058in}}%
\pgfpathlineto{\pgfqpoint{1.634802in}{1.057461in}}%
\pgfpathlineto{\pgfqpoint{1.623127in}{1.064671in}}%
\pgfpathlineto{\pgfqpoint{1.611047in}{1.071684in}}%
\pgfpathclose%
\pgfusepath{fill}%
\end{pgfscope}%
\begin{pgfscope}%
\pgfpathrectangle{\pgfqpoint{0.041670in}{0.041670in}}{\pgfqpoint{2.216660in}{2.216660in}}%
\pgfusepath{clip}%
\pgfsetbuttcap%
\pgfsetroundjoin%
\definecolor{currentfill}{rgb}{0.282884,0.135920,0.453427}%
\pgfsetfillcolor{currentfill}%
\pgfsetlinewidth{0.000000pt}%
\definecolor{currentstroke}{rgb}{0.000000,0.000000,0.000000}%
\pgfsetstrokecolor{currentstroke}%
\pgfsetdash{}{0pt}%
\pgfpathmoveto{\pgfqpoint{1.665998in}{1.154409in}}%
\pgfpathlineto{\pgfqpoint{1.668821in}{1.161850in}}%
\pgfpathlineto{\pgfqpoint{1.671654in}{1.169646in}}%
\pgfpathlineto{\pgfqpoint{1.674496in}{1.177804in}}%
\pgfpathlineto{\pgfqpoint{1.677348in}{1.186328in}}%
\pgfpathlineto{\pgfqpoint{1.691365in}{1.178368in}}%
\pgfpathlineto{\pgfqpoint{1.704919in}{1.170180in}}%
\pgfpathlineto{\pgfqpoint{1.717997in}{1.161771in}}%
\pgfpathlineto{\pgfqpoint{1.730585in}{1.153149in}}%
\pgfpathlineto{\pgfqpoint{1.727413in}{1.144775in}}%
\pgfpathlineto{\pgfqpoint{1.724252in}{1.136769in}}%
\pgfpathlineto{\pgfqpoint{1.721102in}{1.129127in}}%
\pgfpathlineto{\pgfqpoint{1.717962in}{1.121841in}}%
\pgfpathlineto{\pgfqpoint{1.705677in}{1.130303in}}%
\pgfpathlineto{\pgfqpoint{1.692912in}{1.138557in}}%
\pgfpathlineto{\pgfqpoint{1.679681in}{1.146594in}}%
\pgfpathlineto{\pgfqpoint{1.665998in}{1.154409in}}%
\pgfpathclose%
\pgfusepath{fill}%
\end{pgfscope}%
\begin{pgfscope}%
\pgfpathrectangle{\pgfqpoint{0.041670in}{0.041670in}}{\pgfqpoint{2.216660in}{2.216660in}}%
\pgfusepath{clip}%
\pgfsetbuttcap%
\pgfsetroundjoin%
\definecolor{currentfill}{rgb}{0.272594,0.025563,0.353093}%
\pgfsetfillcolor{currentfill}%
\pgfsetlinewidth{0.000000pt}%
\definecolor{currentstroke}{rgb}{0.000000,0.000000,0.000000}%
\pgfsetstrokecolor{currentstroke}%
\pgfsetdash{}{0pt}%
\pgfpathmoveto{\pgfqpoint{0.669189in}{1.052938in}}%
\pgfpathlineto{\pgfqpoint{0.666088in}{1.056342in}}%
\pgfpathlineto{\pgfqpoint{0.662979in}{1.060037in}}%
\pgfpathlineto{\pgfqpoint{0.659863in}{1.064028in}}%
\pgfpathlineto{\pgfqpoint{0.656740in}{1.068320in}}%
\pgfpathlineto{\pgfqpoint{0.668011in}{1.076618in}}%
\pgfpathlineto{\pgfqpoint{0.679754in}{1.084723in}}%
\pgfpathlineto{\pgfqpoint{0.691954in}{1.092627in}}%
\pgfpathlineto{\pgfqpoint{0.704601in}{1.100323in}}%
\pgfpathlineto{\pgfqpoint{0.707426in}{1.095861in}}%
\pgfpathlineto{\pgfqpoint{0.710244in}{1.091698in}}%
\pgfpathlineto{\pgfqpoint{0.713056in}{1.087830in}}%
\pgfpathlineto{\pgfqpoint{0.715861in}{1.084252in}}%
\pgfpathlineto{\pgfqpoint{0.703527in}{1.076721in}}%
\pgfpathlineto{\pgfqpoint{0.691629in}{1.068987in}}%
\pgfpathlineto{\pgfqpoint{0.680179in}{1.061057in}}%
\pgfpathlineto{\pgfqpoint{0.669189in}{1.052938in}}%
\pgfpathclose%
\pgfusepath{fill}%
\end{pgfscope}%
\begin{pgfscope}%
\pgfpathrectangle{\pgfqpoint{0.041670in}{0.041670in}}{\pgfqpoint{2.216660in}{2.216660in}}%
\pgfusepath{clip}%
\pgfsetbuttcap%
\pgfsetroundjoin%
\definecolor{currentfill}{rgb}{0.201239,0.383670,0.554294}%
\pgfsetfillcolor{currentfill}%
\pgfsetlinewidth{0.000000pt}%
\definecolor{currentstroke}{rgb}{0.000000,0.000000,0.000000}%
\pgfsetstrokecolor{currentstroke}%
\pgfsetdash{}{0pt}%
\pgfpathmoveto{\pgfqpoint{1.647631in}{1.350499in}}%
\pgfpathlineto{\pgfqpoint{1.650259in}{1.364499in}}%
\pgfpathlineto{\pgfqpoint{1.652898in}{1.378957in}}%
\pgfpathlineto{\pgfqpoint{1.655551in}{1.393882in}}%
\pgfpathlineto{\pgfqpoint{1.658216in}{1.409282in}}%
\pgfpathlineto{\pgfqpoint{1.675468in}{1.401846in}}%
\pgfpathlineto{\pgfqpoint{1.692284in}{1.394141in}}%
\pgfpathlineto{\pgfqpoint{1.708646in}{1.386173in}}%
\pgfpathlineto{\pgfqpoint{1.724540in}{1.377948in}}%
\pgfpathlineto{\pgfqpoint{1.721491in}{1.362652in}}%
\pgfpathlineto{\pgfqpoint{1.718457in}{1.347832in}}%
\pgfpathlineto{\pgfqpoint{1.715437in}{1.333481in}}%
\pgfpathlineto{\pgfqpoint{1.712432in}{1.319592in}}%
\pgfpathlineto{\pgfqpoint{1.696906in}{1.327704in}}%
\pgfpathlineto{\pgfqpoint{1.680920in}{1.335564in}}%
\pgfpathlineto{\pgfqpoint{1.664490in}{1.343164in}}%
\pgfpathlineto{\pgfqpoint{1.647631in}{1.350499in}}%
\pgfpathclose%
\pgfusepath{fill}%
\end{pgfscope}%
\begin{pgfscope}%
\pgfpathrectangle{\pgfqpoint{0.041670in}{0.041670in}}{\pgfqpoint{2.216660in}{2.216660in}}%
\pgfusepath{clip}%
\pgfsetbuttcap%
\pgfsetroundjoin%
\definecolor{currentfill}{rgb}{0.277941,0.056324,0.381191}%
\pgfsetfillcolor{currentfill}%
\pgfsetlinewidth{0.000000pt}%
\definecolor{currentstroke}{rgb}{0.000000,0.000000,0.000000}%
\pgfsetstrokecolor{currentstroke}%
\pgfsetdash{}{0pt}%
\pgfpathmoveto{\pgfqpoint{0.656740in}{1.068320in}}%
\pgfpathlineto{\pgfqpoint{0.653608in}{1.072918in}}%
\pgfpathlineto{\pgfqpoint{0.650468in}{1.077829in}}%
\pgfpathlineto{\pgfqpoint{0.647320in}{1.083056in}}%
\pgfpathlineto{\pgfqpoint{0.644163in}{1.088607in}}%
\pgfpathlineto{\pgfqpoint{0.655721in}{1.097080in}}%
\pgfpathlineto{\pgfqpoint{0.667759in}{1.105355in}}%
\pgfpathlineto{\pgfqpoint{0.680266in}{1.113424in}}%
\pgfpathlineto{\pgfqpoint{0.693228in}{1.121281in}}%
\pgfpathlineto{\pgfqpoint{0.696082in}{1.115565in}}%
\pgfpathlineto{\pgfqpoint{0.698929in}{1.110170in}}%
\pgfpathlineto{\pgfqpoint{0.701769in}{1.105091in}}%
\pgfpathlineto{\pgfqpoint{0.704601in}{1.100323in}}%
\pgfpathlineto{\pgfqpoint{0.691954in}{1.092627in}}%
\pgfpathlineto{\pgfqpoint{0.679754in}{1.084723in}}%
\pgfpathlineto{\pgfqpoint{0.668011in}{1.076618in}}%
\pgfpathlineto{\pgfqpoint{0.656740in}{1.068320in}}%
\pgfpathclose%
\pgfusepath{fill}%
\end{pgfscope}%
\begin{pgfscope}%
\pgfpathrectangle{\pgfqpoint{0.041670in}{0.041670in}}{\pgfqpoint{2.216660in}{2.216660in}}%
\pgfusepath{clip}%
\pgfsetbuttcap%
\pgfsetroundjoin%
\definecolor{currentfill}{rgb}{0.282327,0.094955,0.417331}%
\pgfsetfillcolor{currentfill}%
\pgfsetlinewidth{0.000000pt}%
\definecolor{currentstroke}{rgb}{0.000000,0.000000,0.000000}%
\pgfsetstrokecolor{currentstroke}%
\pgfsetdash{}{0pt}%
\pgfpathmoveto{\pgfqpoint{0.846609in}{1.117268in}}%
\pgfpathlineto{\pgfqpoint{0.844220in}{1.114174in}}%
\pgfpathlineto{\pgfqpoint{0.841832in}{1.111228in}}%
\pgfpathlineto{\pgfqpoint{0.839443in}{1.108435in}}%
\pgfpathlineto{\pgfqpoint{0.837053in}{1.105799in}}%
\pgfpathlineto{\pgfqpoint{0.848959in}{1.111173in}}%
\pgfpathlineto{\pgfqpoint{0.861168in}{1.116352in}}%
\pgfpathlineto{\pgfqpoint{0.873667in}{1.121330in}}%
\pgfpathlineto{\pgfqpoint{0.886446in}{1.126103in}}%
\pgfpathlineto{\pgfqpoint{0.888486in}{1.128591in}}%
\pgfpathlineto{\pgfqpoint{0.890525in}{1.131235in}}%
\pgfpathlineto{\pgfqpoint{0.892564in}{1.134031in}}%
\pgfpathlineto{\pgfqpoint{0.894602in}{1.136976in}}%
\pgfpathlineto{\pgfqpoint{0.882184in}{1.132343in}}%
\pgfpathlineto{\pgfqpoint{0.870038in}{1.127511in}}%
\pgfpathlineto{\pgfqpoint{0.858176in}{1.122485in}}%
\pgfpathlineto{\pgfqpoint{0.846609in}{1.117268in}}%
\pgfpathclose%
\pgfusepath{fill}%
\end{pgfscope}%
\begin{pgfscope}%
\pgfpathrectangle{\pgfqpoint{0.041670in}{0.041670in}}{\pgfqpoint{2.216660in}{2.216660in}}%
\pgfusepath{clip}%
\pgfsetbuttcap%
\pgfsetroundjoin%
\definecolor{currentfill}{rgb}{0.268510,0.009605,0.335427}%
\pgfsetfillcolor{currentfill}%
\pgfsetlinewidth{0.000000pt}%
\definecolor{currentstroke}{rgb}{0.000000,0.000000,0.000000}%
\pgfsetstrokecolor{currentstroke}%
\pgfsetdash{}{0pt}%
\pgfpathmoveto{\pgfqpoint{0.681529in}{1.042129in}}%
\pgfpathlineto{\pgfqpoint{0.678453in}{1.044421in}}%
\pgfpathlineto{\pgfqpoint{0.675372in}{1.046982in}}%
\pgfpathlineto{\pgfqpoint{0.672284in}{1.049820in}}%
\pgfpathlineto{\pgfqpoint{0.669189in}{1.052938in}}%
\pgfpathlineto{\pgfqpoint{0.680179in}{1.061057in}}%
\pgfpathlineto{\pgfqpoint{0.691629in}{1.068987in}}%
\pgfpathlineto{\pgfqpoint{0.703527in}{1.076721in}}%
\pgfpathlineto{\pgfqpoint{0.715861in}{1.084252in}}%
\pgfpathlineto{\pgfqpoint{0.718661in}{1.080959in}}%
\pgfpathlineto{\pgfqpoint{0.721454in}{1.077945in}}%
\pgfpathlineto{\pgfqpoint{0.724242in}{1.075206in}}%
\pgfpathlineto{\pgfqpoint{0.727025in}{1.072737in}}%
\pgfpathlineto{\pgfqpoint{0.714999in}{1.065375in}}%
\pgfpathlineto{\pgfqpoint{0.703400in}{1.057815in}}%
\pgfpathlineto{\pgfqpoint{0.692239in}{1.050065in}}%
\pgfpathlineto{\pgfqpoint{0.681529in}{1.042129in}}%
\pgfpathclose%
\pgfusepath{fill}%
\end{pgfscope}%
\begin{pgfscope}%
\pgfpathrectangle{\pgfqpoint{0.041670in}{0.041670in}}{\pgfqpoint{2.216660in}{2.216660in}}%
\pgfusepath{clip}%
\pgfsetbuttcap%
\pgfsetroundjoin%
\definecolor{currentfill}{rgb}{0.263663,0.237631,0.518762}%
\pgfsetfillcolor{currentfill}%
\pgfsetlinewidth{0.000000pt}%
\definecolor{currentstroke}{rgb}{0.000000,0.000000,0.000000}%
\pgfsetstrokecolor{currentstroke}%
\pgfsetdash{}{0pt}%
\pgfpathmoveto{\pgfqpoint{1.126956in}{1.240471in}}%
\pgfpathlineto{\pgfqpoint{1.126529in}{1.235985in}}%
\pgfpathlineto{\pgfqpoint{1.126103in}{1.231589in}}%
\pgfpathlineto{\pgfqpoint{1.125677in}{1.227289in}}%
\pgfpathlineto{\pgfqpoint{1.125251in}{1.223086in}}%
\pgfpathlineto{\pgfqpoint{1.139077in}{1.223836in}}%
\pgfpathlineto{\pgfqpoint{1.152940in}{1.224368in}}%
\pgfpathlineto{\pgfqpoint{1.166828in}{1.224683in}}%
\pgfpathlineto{\pgfqpoint{1.180727in}{1.224780in}}%
\pgfpathlineto{\pgfqpoint{1.180721in}{1.228969in}}%
\pgfpathlineto{\pgfqpoint{1.180715in}{1.233255in}}%
\pgfpathlineto{\pgfqpoint{1.180709in}{1.237636in}}%
\pgfpathlineto{\pgfqpoint{1.180703in}{1.242108in}}%
\pgfpathlineto{\pgfqpoint{1.167237in}{1.242015in}}%
\pgfpathlineto{\pgfqpoint{1.153782in}{1.241710in}}%
\pgfpathlineto{\pgfqpoint{1.140351in}{1.241196in}}%
\pgfpathlineto{\pgfqpoint{1.126956in}{1.240471in}}%
\pgfpathclose%
\pgfusepath{fill}%
\end{pgfscope}%
\begin{pgfscope}%
\pgfpathrectangle{\pgfqpoint{0.041670in}{0.041670in}}{\pgfqpoint{2.216660in}{2.216660in}}%
\pgfusepath{clip}%
\pgfsetbuttcap%
\pgfsetroundjoin%
\definecolor{currentfill}{rgb}{0.267004,0.004874,0.329415}%
\pgfsetfillcolor{currentfill}%
\pgfsetlinewidth{0.000000pt}%
\definecolor{currentstroke}{rgb}{0.000000,0.000000,0.000000}%
\pgfsetstrokecolor{currentstroke}%
\pgfsetdash{}{0pt}%
\pgfpathmoveto{\pgfqpoint{1.600315in}{1.068192in}}%
\pgfpathlineto{\pgfqpoint{1.602992in}{1.068712in}}%
\pgfpathlineto{\pgfqpoint{1.605673in}{1.069465in}}%
\pgfpathlineto{\pgfqpoint{1.608358in}{1.070454in}}%
\pgfpathlineto{\pgfqpoint{1.611047in}{1.071684in}}%
\pgfpathlineto{\pgfqpoint{1.623127in}{1.064671in}}%
\pgfpathlineto{\pgfqpoint{1.634802in}{1.057461in}}%
\pgfpathlineto{\pgfqpoint{1.646057in}{1.050058in}}%
\pgfpathlineto{\pgfqpoint{1.656883in}{1.042470in}}%
\pgfpathlineto{\pgfqpoint{1.653896in}{1.041417in}}%
\pgfpathlineto{\pgfqpoint{1.650913in}{1.040607in}}%
\pgfpathlineto{\pgfqpoint{1.647935in}{1.040034in}}%
\pgfpathlineto{\pgfqpoint{1.644961in}{1.039693in}}%
\pgfpathlineto{\pgfqpoint{1.634418in}{1.047095in}}%
\pgfpathlineto{\pgfqpoint{1.623455in}{1.054316in}}%
\pgfpathlineto{\pgfqpoint{1.612083in}{1.061350in}}%
\pgfpathlineto{\pgfqpoint{1.600315in}{1.068192in}}%
\pgfpathclose%
\pgfusepath{fill}%
\end{pgfscope}%
\begin{pgfscope}%
\pgfpathrectangle{\pgfqpoint{0.041670in}{0.041670in}}{\pgfqpoint{2.216660in}{2.216660in}}%
\pgfusepath{clip}%
\pgfsetbuttcap%
\pgfsetroundjoin%
\definecolor{currentfill}{rgb}{0.263663,0.237631,0.518762}%
\pgfsetfillcolor{currentfill}%
\pgfsetlinewidth{0.000000pt}%
\definecolor{currentstroke}{rgb}{0.000000,0.000000,0.000000}%
\pgfsetstrokecolor{currentstroke}%
\pgfsetdash{}{0pt}%
\pgfpathmoveto{\pgfqpoint{1.180703in}{1.242108in}}%
\pgfpathlineto{\pgfqpoint{1.180709in}{1.237636in}}%
\pgfpathlineto{\pgfqpoint{1.180715in}{1.233255in}}%
\pgfpathlineto{\pgfqpoint{1.180721in}{1.228969in}}%
\pgfpathlineto{\pgfqpoint{1.180727in}{1.224780in}}%
\pgfpathlineto{\pgfqpoint{1.194626in}{1.224659in}}%
\pgfpathlineto{\pgfqpoint{1.208511in}{1.224320in}}%
\pgfpathlineto{\pgfqpoint{1.222371in}{1.223763in}}%
\pgfpathlineto{\pgfqpoint{1.236193in}{1.222989in}}%
\pgfpathlineto{\pgfqpoint{1.235755in}{1.227193in}}%
\pgfpathlineto{\pgfqpoint{1.235317in}{1.231494in}}%
\pgfpathlineto{\pgfqpoint{1.234878in}{1.235891in}}%
\pgfpathlineto{\pgfqpoint{1.234440in}{1.240378in}}%
\pgfpathlineto{\pgfqpoint{1.221049in}{1.241126in}}%
\pgfpathlineto{\pgfqpoint{1.207621in}{1.241664in}}%
\pgfpathlineto{\pgfqpoint{1.194169in}{1.241991in}}%
\pgfpathlineto{\pgfqpoint{1.180703in}{1.242108in}}%
\pgfpathclose%
\pgfusepath{fill}%
\end{pgfscope}%
\begin{pgfscope}%
\pgfpathrectangle{\pgfqpoint{0.041670in}{0.041670in}}{\pgfqpoint{2.216660in}{2.216660in}}%
\pgfusepath{clip}%
\pgfsetbuttcap%
\pgfsetroundjoin%
\definecolor{currentfill}{rgb}{0.274952,0.037752,0.364543}%
\pgfsetfillcolor{currentfill}%
\pgfsetlinewidth{0.000000pt}%
\definecolor{currentstroke}{rgb}{0.000000,0.000000,0.000000}%
\pgfsetstrokecolor{currentstroke}%
\pgfsetdash{}{0pt}%
\pgfpathmoveto{\pgfqpoint{0.781842in}{1.072844in}}%
\pgfpathlineto{\pgfqpoint{0.779128in}{1.070880in}}%
\pgfpathlineto{\pgfqpoint{0.776411in}{1.069096in}}%
\pgfpathlineto{\pgfqpoint{0.773693in}{1.067497in}}%
\pgfpathlineto{\pgfqpoint{0.770972in}{1.066088in}}%
\pgfpathlineto{\pgfqpoint{0.782166in}{1.072558in}}%
\pgfpathlineto{\pgfqpoint{0.793725in}{1.078841in}}%
\pgfpathlineto{\pgfqpoint{0.805638in}{1.084930in}}%
\pgfpathlineto{\pgfqpoint{0.817894in}{1.090820in}}%
\pgfpathlineto{\pgfqpoint{0.820294in}{1.092061in}}%
\pgfpathlineto{\pgfqpoint{0.822692in}{1.093490in}}%
\pgfpathlineto{\pgfqpoint{0.825089in}{1.095104in}}%
\pgfpathlineto{\pgfqpoint{0.827484in}{1.096898in}}%
\pgfpathlineto{\pgfqpoint{0.815561in}{1.091169in}}%
\pgfpathlineto{\pgfqpoint{0.803972in}{1.085247in}}%
\pgfpathlineto{\pgfqpoint{0.792729in}{1.079137in}}%
\pgfpathlineto{\pgfqpoint{0.781842in}{1.072844in}}%
\pgfpathclose%
\pgfusepath{fill}%
\end{pgfscope}%
\begin{pgfscope}%
\pgfpathrectangle{\pgfqpoint{0.041670in}{0.041670in}}{\pgfqpoint{2.216660in}{2.216660in}}%
\pgfusepath{clip}%
\pgfsetbuttcap%
\pgfsetroundjoin%
\definecolor{currentfill}{rgb}{0.282327,0.094955,0.417331}%
\pgfsetfillcolor{currentfill}%
\pgfsetlinewidth{0.000000pt}%
\definecolor{currentstroke}{rgb}{0.000000,0.000000,0.000000}%
\pgfsetstrokecolor{currentstroke}%
\pgfsetdash{}{0pt}%
\pgfpathmoveto{\pgfqpoint{0.644163in}{1.088607in}}%
\pgfpathlineto{\pgfqpoint{0.640997in}{1.094486in}}%
\pgfpathlineto{\pgfqpoint{0.637822in}{1.100698in}}%
\pgfpathlineto{\pgfqpoint{0.634637in}{1.107251in}}%
\pgfpathlineto{\pgfqpoint{0.631442in}{1.114148in}}%
\pgfpathlineto{\pgfqpoint{0.643289in}{1.122791in}}%
\pgfpathlineto{\pgfqpoint{0.655628in}{1.131231in}}%
\pgfpathlineto{\pgfqpoint{0.668445in}{1.139460in}}%
\pgfpathlineto{\pgfqpoint{0.681727in}{1.147473in}}%
\pgfpathlineto{\pgfqpoint{0.684615in}{1.140415in}}%
\pgfpathlineto{\pgfqpoint{0.687494in}{1.133701in}}%
\pgfpathlineto{\pgfqpoint{0.690365in}{1.127325in}}%
\pgfpathlineto{\pgfqpoint{0.693228in}{1.121281in}}%
\pgfpathlineto{\pgfqpoint{0.680266in}{1.113424in}}%
\pgfpathlineto{\pgfqpoint{0.667759in}{1.105355in}}%
\pgfpathlineto{\pgfqpoint{0.655721in}{1.097080in}}%
\pgfpathlineto{\pgfqpoint{0.644163in}{1.088607in}}%
\pgfpathclose%
\pgfusepath{fill}%
\end{pgfscope}%
\begin{pgfscope}%
\pgfpathrectangle{\pgfqpoint{0.041670in}{0.041670in}}{\pgfqpoint{2.216660in}{2.216660in}}%
\pgfusepath{clip}%
\pgfsetbuttcap%
\pgfsetroundjoin%
\definecolor{currentfill}{rgb}{0.267004,0.004874,0.329415}%
\pgfsetfillcolor{currentfill}%
\pgfsetlinewidth{0.000000pt}%
\definecolor{currentstroke}{rgb}{0.000000,0.000000,0.000000}%
\pgfsetstrokecolor{currentstroke}%
\pgfsetdash{}{0pt}%
\pgfpathmoveto{\pgfqpoint{0.693774in}{1.035574in}}%
\pgfpathlineto{\pgfqpoint{0.690720in}{1.036831in}}%
\pgfpathlineto{\pgfqpoint{0.687662in}{1.038339in}}%
\pgfpathlineto{\pgfqpoint{0.684598in}{1.040104in}}%
\pgfpathlineto{\pgfqpoint{0.681529in}{1.042129in}}%
\pgfpathlineto{\pgfqpoint{0.692239in}{1.050065in}}%
\pgfpathlineto{\pgfqpoint{0.703400in}{1.057815in}}%
\pgfpathlineto{\pgfqpoint{0.714999in}{1.065375in}}%
\pgfpathlineto{\pgfqpoint{0.727025in}{1.072737in}}%
\pgfpathlineto{\pgfqpoint{0.729802in}{1.070532in}}%
\pgfpathlineto{\pgfqpoint{0.732574in}{1.068588in}}%
\pgfpathlineto{\pgfqpoint{0.735342in}{1.066899in}}%
\pgfpathlineto{\pgfqpoint{0.738105in}{1.065461in}}%
\pgfpathlineto{\pgfqpoint{0.726385in}{1.058271in}}%
\pgfpathlineto{\pgfqpoint{0.715082in}{1.050890in}}%
\pgfpathlineto{\pgfqpoint{0.704208in}{1.043322in}}%
\pgfpathlineto{\pgfqpoint{0.693774in}{1.035574in}}%
\pgfpathclose%
\pgfusepath{fill}%
\end{pgfscope}%
\begin{pgfscope}%
\pgfpathrectangle{\pgfqpoint{0.041670in}{0.041670in}}{\pgfqpoint{2.216660in}{2.216660in}}%
\pgfusepath{clip}%
\pgfsetbuttcap%
\pgfsetroundjoin%
\definecolor{currentfill}{rgb}{0.280255,0.165693,0.476498}%
\pgfsetfillcolor{currentfill}%
\pgfsetlinewidth{0.000000pt}%
\definecolor{currentstroke}{rgb}{0.000000,0.000000,0.000000}%
\pgfsetstrokecolor{currentstroke}%
\pgfsetdash{}{0pt}%
\pgfpathmoveto{\pgfqpoint{1.388476in}{1.183922in}}%
\pgfpathlineto{\pgfqpoint{1.390052in}{1.180071in}}%
\pgfpathlineto{\pgfqpoint{1.391628in}{1.176339in}}%
\pgfpathlineto{\pgfqpoint{1.393204in}{1.172730in}}%
\pgfpathlineto{\pgfqpoint{1.394779in}{1.169247in}}%
\pgfpathlineto{\pgfqpoint{1.407963in}{1.165745in}}%
\pgfpathlineto{\pgfqpoint{1.420939in}{1.162034in}}%
\pgfpathlineto{\pgfqpoint{1.433697in}{1.158117in}}%
\pgfpathlineto{\pgfqpoint{1.446223in}{1.153998in}}%
\pgfpathlineto{\pgfqpoint{1.444266in}{1.157601in}}%
\pgfpathlineto{\pgfqpoint{1.442309in}{1.161330in}}%
\pgfpathlineto{\pgfqpoint{1.440351in}{1.165182in}}%
\pgfpathlineto{\pgfqpoint{1.438393in}{1.169153in}}%
\pgfpathlineto{\pgfqpoint{1.426239in}{1.173142in}}%
\pgfpathlineto{\pgfqpoint{1.413861in}{1.176935in}}%
\pgfpathlineto{\pgfqpoint{1.401269in}{1.180530in}}%
\pgfpathlineto{\pgfqpoint{1.388476in}{1.183922in}}%
\pgfpathclose%
\pgfusepath{fill}%
\end{pgfscope}%
\begin{pgfscope}%
\pgfpathrectangle{\pgfqpoint{0.041670in}{0.041670in}}{\pgfqpoint{2.216660in}{2.216660in}}%
\pgfusepath{clip}%
\pgfsetbuttcap%
\pgfsetroundjoin%
\definecolor{currentfill}{rgb}{0.276194,0.190074,0.493001}%
\pgfsetfillcolor{currentfill}%
\pgfsetlinewidth{0.000000pt}%
\definecolor{currentstroke}{rgb}{0.000000,0.000000,0.000000}%
\pgfsetstrokecolor{currentstroke}%
\pgfsetdash{}{0pt}%
\pgfpathmoveto{\pgfqpoint{1.677348in}{1.186328in}}%
\pgfpathlineto{\pgfqpoint{1.680210in}{1.195227in}}%
\pgfpathlineto{\pgfqpoint{1.683082in}{1.204505in}}%
\pgfpathlineto{\pgfqpoint{1.685964in}{1.214169in}}%
\pgfpathlineto{\pgfqpoint{1.688857in}{1.224225in}}%
\pgfpathlineto{\pgfqpoint{1.703212in}{1.216124in}}%
\pgfpathlineto{\pgfqpoint{1.717095in}{1.207792in}}%
\pgfpathlineto{\pgfqpoint{1.730492in}{1.199235in}}%
\pgfpathlineto{\pgfqpoint{1.743389in}{1.190461in}}%
\pgfpathlineto{\pgfqpoint{1.740170in}{1.180548in}}%
\pgfpathlineto{\pgfqpoint{1.736964in}{1.171030in}}%
\pgfpathlineto{\pgfqpoint{1.733769in}{1.161899in}}%
\pgfpathlineto{\pgfqpoint{1.730585in}{1.153149in}}%
\pgfpathlineto{\pgfqpoint{1.717997in}{1.161771in}}%
\pgfpathlineto{\pgfqpoint{1.704919in}{1.170180in}}%
\pgfpathlineto{\pgfqpoint{1.691365in}{1.178368in}}%
\pgfpathlineto{\pgfqpoint{1.677348in}{1.186328in}}%
\pgfpathclose%
\pgfusepath{fill}%
\end{pgfscope}%
\begin{pgfscope}%
\pgfpathrectangle{\pgfqpoint{0.041670in}{0.041670in}}{\pgfqpoint{2.216660in}{2.216660in}}%
\pgfusepath{clip}%
\pgfsetbuttcap%
\pgfsetroundjoin%
\definecolor{currentfill}{rgb}{0.274128,0.199721,0.498911}%
\pgfsetfillcolor{currentfill}%
\pgfsetlinewidth{0.000000pt}%
\definecolor{currentstroke}{rgb}{0.000000,0.000000,0.000000}%
\pgfsetstrokecolor{currentstroke}%
\pgfsetdash{}{0pt}%
\pgfpathmoveto{\pgfqpoint{1.330827in}{1.211565in}}%
\pgfpathlineto{\pgfqpoint{1.332003in}{1.207367in}}%
\pgfpathlineto{\pgfqpoint{1.333179in}{1.203273in}}%
\pgfpathlineto{\pgfqpoint{1.334354in}{1.199288in}}%
\pgfpathlineto{\pgfqpoint{1.335529in}{1.195415in}}%
\pgfpathlineto{\pgfqpoint{1.349007in}{1.192858in}}%
\pgfpathlineto{\pgfqpoint{1.362333in}{1.190088in}}%
\pgfpathlineto{\pgfqpoint{1.375493in}{1.187109in}}%
\pgfpathlineto{\pgfqpoint{1.388476in}{1.183922in}}%
\pgfpathlineto{\pgfqpoint{1.386899in}{1.187888in}}%
\pgfpathlineto{\pgfqpoint{1.385321in}{1.191966in}}%
\pgfpathlineto{\pgfqpoint{1.383743in}{1.196153in}}%
\pgfpathlineto{\pgfqpoint{1.382164in}{1.200445in}}%
\pgfpathlineto{\pgfqpoint{1.369577in}{1.203529in}}%
\pgfpathlineto{\pgfqpoint{1.356817in}{1.206411in}}%
\pgfpathlineto{\pgfqpoint{1.343897in}{1.209091in}}%
\pgfpathlineto{\pgfqpoint{1.330827in}{1.211565in}}%
\pgfpathclose%
\pgfusepath{fill}%
\end{pgfscope}%
\begin{pgfscope}%
\pgfpathrectangle{\pgfqpoint{0.041670in}{0.041670in}}{\pgfqpoint{2.216660in}{2.216660in}}%
\pgfusepath{clip}%
\pgfsetbuttcap%
\pgfsetroundjoin%
\definecolor{currentfill}{rgb}{0.263663,0.237631,0.518762}%
\pgfsetfillcolor{currentfill}%
\pgfsetlinewidth{0.000000pt}%
\definecolor{currentstroke}{rgb}{0.000000,0.000000,0.000000}%
\pgfsetstrokecolor{currentstroke}%
\pgfsetdash{}{0pt}%
\pgfpathmoveto{\pgfqpoint{1.073975in}{1.235488in}}%
\pgfpathlineto{\pgfqpoint{1.073122in}{1.230958in}}%
\pgfpathlineto{\pgfqpoint{1.072268in}{1.226519in}}%
\pgfpathlineto{\pgfqpoint{1.071416in}{1.222175in}}%
\pgfpathlineto{\pgfqpoint{1.070563in}{1.217929in}}%
\pgfpathlineto{\pgfqpoint{1.084118in}{1.219540in}}%
\pgfpathlineto{\pgfqpoint{1.097759in}{1.220937in}}%
\pgfpathlineto{\pgfqpoint{1.111474in}{1.222119in}}%
\pgfpathlineto{\pgfqpoint{1.125251in}{1.223086in}}%
\pgfpathlineto{\pgfqpoint{1.125677in}{1.227289in}}%
\pgfpathlineto{\pgfqpoint{1.126103in}{1.231589in}}%
\pgfpathlineto{\pgfqpoint{1.126529in}{1.235985in}}%
\pgfpathlineto{\pgfqpoint{1.126956in}{1.240471in}}%
\pgfpathlineto{\pgfqpoint{1.113608in}{1.239537in}}%
\pgfpathlineto{\pgfqpoint{1.100321in}{1.238395in}}%
\pgfpathlineto{\pgfqpoint{1.087106in}{1.237045in}}%
\pgfpathlineto{\pgfqpoint{1.073975in}{1.235488in}}%
\pgfpathclose%
\pgfusepath{fill}%
\end{pgfscope}%
\begin{pgfscope}%
\pgfpathrectangle{\pgfqpoint{0.041670in}{0.041670in}}{\pgfqpoint{2.216660in}{2.216660in}}%
\pgfusepath{clip}%
\pgfsetbuttcap%
\pgfsetroundjoin%
\definecolor{currentfill}{rgb}{0.263663,0.237631,0.518762}%
\pgfsetfillcolor{currentfill}%
\pgfsetlinewidth{0.000000pt}%
\definecolor{currentstroke}{rgb}{0.000000,0.000000,0.000000}%
\pgfsetstrokecolor{currentstroke}%
\pgfsetdash{}{0pt}%
\pgfpathmoveto{\pgfqpoint{1.234440in}{1.240378in}}%
\pgfpathlineto{\pgfqpoint{1.234878in}{1.235891in}}%
\pgfpathlineto{\pgfqpoint{1.235317in}{1.231494in}}%
\pgfpathlineto{\pgfqpoint{1.235755in}{1.227193in}}%
\pgfpathlineto{\pgfqpoint{1.236193in}{1.222989in}}%
\pgfpathlineto{\pgfqpoint{1.249963in}{1.221999in}}%
\pgfpathlineto{\pgfqpoint{1.263671in}{1.220792in}}%
\pgfpathlineto{\pgfqpoint{1.277303in}{1.219371in}}%
\pgfpathlineto{\pgfqpoint{1.290847in}{1.217737in}}%
\pgfpathlineto{\pgfqpoint{1.289982in}{1.221985in}}%
\pgfpathlineto{\pgfqpoint{1.289118in}{1.226330in}}%
\pgfpathlineto{\pgfqpoint{1.288253in}{1.230771in}}%
\pgfpathlineto{\pgfqpoint{1.287388in}{1.235302in}}%
\pgfpathlineto{\pgfqpoint{1.274267in}{1.236882in}}%
\pgfpathlineto{\pgfqpoint{1.261061in}{1.238255in}}%
\pgfpathlineto{\pgfqpoint{1.247781in}{1.239421in}}%
\pgfpathlineto{\pgfqpoint{1.234440in}{1.240378in}}%
\pgfpathclose%
\pgfusepath{fill}%
\end{pgfscope}%
\begin{pgfscope}%
\pgfpathrectangle{\pgfqpoint{0.041670in}{0.041670in}}{\pgfqpoint{2.216660in}{2.216660in}}%
\pgfusepath{clip}%
\pgfsetbuttcap%
\pgfsetroundjoin%
\definecolor{currentfill}{rgb}{0.282884,0.135920,0.453427}%
\pgfsetfillcolor{currentfill}%
\pgfsetlinewidth{0.000000pt}%
\definecolor{currentstroke}{rgb}{0.000000,0.000000,0.000000}%
\pgfsetstrokecolor{currentstroke}%
\pgfsetdash{}{0pt}%
\pgfpathmoveto{\pgfqpoint{0.631442in}{1.114148in}}%
\pgfpathlineto{\pgfqpoint{0.628237in}{1.121398in}}%
\pgfpathlineto{\pgfqpoint{0.625021in}{1.129004in}}%
\pgfpathlineto{\pgfqpoint{0.621794in}{1.136973in}}%
\pgfpathlineto{\pgfqpoint{0.618557in}{1.145312in}}%
\pgfpathlineto{\pgfqpoint{0.630698in}{1.154118in}}%
\pgfpathlineto{\pgfqpoint{0.643342in}{1.162716in}}%
\pgfpathlineto{\pgfqpoint{0.656474in}{1.171101in}}%
\pgfpathlineto{\pgfqpoint{0.670080in}{1.179264in}}%
\pgfpathlineto{\pgfqpoint{0.673006in}{1.170771in}}%
\pgfpathlineto{\pgfqpoint{0.675923in}{1.162645in}}%
\pgfpathlineto{\pgfqpoint{0.678829in}{1.154881in}}%
\pgfpathlineto{\pgfqpoint{0.681727in}{1.147473in}}%
\pgfpathlineto{\pgfqpoint{0.668445in}{1.139460in}}%
\pgfpathlineto{\pgfqpoint{0.655628in}{1.131231in}}%
\pgfpathlineto{\pgfqpoint{0.643289in}{1.122791in}}%
\pgfpathlineto{\pgfqpoint{0.631442in}{1.114148in}}%
\pgfpathclose%
\pgfusepath{fill}%
\end{pgfscope}%
\begin{pgfscope}%
\pgfpathrectangle{\pgfqpoint{0.041670in}{0.041670in}}{\pgfqpoint{2.216660in}{2.216660in}}%
\pgfusepath{clip}%
\pgfsetbuttcap%
\pgfsetroundjoin%
\definecolor{currentfill}{rgb}{0.279566,0.067836,0.391917}%
\pgfsetfillcolor{currentfill}%
\pgfsetlinewidth{0.000000pt}%
\definecolor{currentstroke}{rgb}{0.000000,0.000000,0.000000}%
\pgfsetstrokecolor{currentstroke}%
\pgfsetdash{}{0pt}%
\pgfpathmoveto{\pgfqpoint{1.512289in}{1.110586in}}%
\pgfpathlineto{\pgfqpoint{1.514604in}{1.108145in}}%
\pgfpathlineto{\pgfqpoint{1.516921in}{1.105869in}}%
\pgfpathlineto{\pgfqpoint{1.519238in}{1.103761in}}%
\pgfpathlineto{\pgfqpoint{1.521556in}{1.101825in}}%
\pgfpathlineto{\pgfqpoint{1.533767in}{1.096271in}}%
\pgfpathlineto{\pgfqpoint{1.545653in}{1.090520in}}%
\pgfpathlineto{\pgfqpoint{1.557204in}{1.084577in}}%
\pgfpathlineto{\pgfqpoint{1.568409in}{1.078447in}}%
\pgfpathlineto{\pgfqpoint{1.565763in}{1.080548in}}%
\pgfpathlineto{\pgfqpoint{1.563120in}{1.082822in}}%
\pgfpathlineto{\pgfqpoint{1.560477in}{1.085265in}}%
\pgfpathlineto{\pgfqpoint{1.557836in}{1.087872in}}%
\pgfpathlineto{\pgfqpoint{1.546945in}{1.093828in}}%
\pgfpathlineto{\pgfqpoint{1.535717in}{1.099602in}}%
\pgfpathlineto{\pgfqpoint{1.524161in}{1.105190in}}%
\pgfpathlineto{\pgfqpoint{1.512289in}{1.110586in}}%
\pgfpathclose%
\pgfusepath{fill}%
\end{pgfscope}%
\begin{pgfscope}%
\pgfpathrectangle{\pgfqpoint{0.041670in}{0.041670in}}{\pgfqpoint{2.216660in}{2.216660in}}%
\pgfusepath{clip}%
\pgfsetbuttcap%
\pgfsetroundjoin%
\definecolor{currentfill}{rgb}{0.201239,0.383670,0.554294}%
\pgfsetfillcolor{currentfill}%
\pgfsetlinewidth{0.000000pt}%
\definecolor{currentstroke}{rgb}{0.000000,0.000000,0.000000}%
\pgfsetstrokecolor{currentstroke}%
\pgfsetdash{}{0pt}%
\pgfpathmoveto{\pgfqpoint{0.634076in}{1.312174in}}%
\pgfpathlineto{\pgfqpoint{0.630992in}{1.326037in}}%
\pgfpathlineto{\pgfqpoint{0.627893in}{1.340361in}}%
\pgfpathlineto{\pgfqpoint{0.624779in}{1.355155in}}%
\pgfpathlineto{\pgfqpoint{0.621650in}{1.370426in}}%
\pgfpathlineto{\pgfqpoint{0.637112in}{1.378874in}}%
\pgfpathlineto{\pgfqpoint{0.653058in}{1.387071in}}%
\pgfpathlineto{\pgfqpoint{0.669472in}{1.395010in}}%
\pgfpathlineto{\pgfqpoint{0.686337in}{1.402686in}}%
\pgfpathlineto{\pgfqpoint{0.689091in}{1.387307in}}%
\pgfpathlineto{\pgfqpoint{0.691831in}{1.372404in}}%
\pgfpathlineto{\pgfqpoint{0.694558in}{1.357968in}}%
\pgfpathlineto{\pgfqpoint{0.697272in}{1.343992in}}%
\pgfpathlineto{\pgfqpoint{0.680793in}{1.336421in}}%
\pgfpathlineto{\pgfqpoint{0.664757in}{1.328590in}}%
\pgfpathlineto{\pgfqpoint{0.649180in}{1.320506in}}%
\pgfpathlineto{\pgfqpoint{0.634076in}{1.312174in}}%
\pgfpathclose%
\pgfusepath{fill}%
\end{pgfscope}%
\begin{pgfscope}%
\pgfpathrectangle{\pgfqpoint{0.041670in}{0.041670in}}{\pgfqpoint{2.216660in}{2.216660in}}%
\pgfusepath{clip}%
\pgfsetbuttcap%
\pgfsetroundjoin%
\definecolor{currentfill}{rgb}{0.268510,0.009605,0.335427}%
\pgfsetfillcolor{currentfill}%
\pgfsetlinewidth{0.000000pt}%
\definecolor{currentstroke}{rgb}{0.000000,0.000000,0.000000}%
\pgfsetstrokecolor{currentstroke}%
\pgfsetdash{}{0pt}%
\pgfpathmoveto{\pgfqpoint{1.589638in}{1.068340in}}%
\pgfpathlineto{\pgfqpoint{1.592302in}{1.067977in}}%
\pgfpathlineto{\pgfqpoint{1.594970in}{1.067828in}}%
\pgfpathlineto{\pgfqpoint{1.597641in}{1.067898in}}%
\pgfpathlineto{\pgfqpoint{1.600315in}{1.068192in}}%
\pgfpathlineto{\pgfqpoint{1.612083in}{1.061350in}}%
\pgfpathlineto{\pgfqpoint{1.623455in}{1.054316in}}%
\pgfpathlineto{\pgfqpoint{1.634418in}{1.047095in}}%
\pgfpathlineto{\pgfqpoint{1.644961in}{1.039693in}}%
\pgfpathlineto{\pgfqpoint{1.641991in}{1.039581in}}%
\pgfpathlineto{\pgfqpoint{1.639025in}{1.039692in}}%
\pgfpathlineto{\pgfqpoint{1.636063in}{1.040023in}}%
\pgfpathlineto{\pgfqpoint{1.633104in}{1.040569in}}%
\pgfpathlineto{\pgfqpoint{1.622842in}{1.047781in}}%
\pgfpathlineto{\pgfqpoint{1.612169in}{1.054817in}}%
\pgfpathlineto{\pgfqpoint{1.601097in}{1.061672in}}%
\pgfpathlineto{\pgfqpoint{1.589638in}{1.068340in}}%
\pgfpathclose%
\pgfusepath{fill}%
\end{pgfscope}%
\begin{pgfscope}%
\pgfpathrectangle{\pgfqpoint{0.041670in}{0.041670in}}{\pgfqpoint{2.216660in}{2.216660in}}%
\pgfusepath{clip}%
\pgfsetbuttcap%
\pgfsetroundjoin%
\definecolor{currentfill}{rgb}{0.274128,0.199721,0.498911}%
\pgfsetfillcolor{currentfill}%
\pgfsetlinewidth{0.000000pt}%
\definecolor{currentstroke}{rgb}{0.000000,0.000000,0.000000}%
\pgfsetstrokecolor{currentstroke}%
\pgfsetdash{}{0pt}%
\pgfpathmoveto{\pgfqpoint{0.966711in}{1.197538in}}%
\pgfpathlineto{\pgfqpoint{0.965045in}{1.193222in}}%
\pgfpathlineto{\pgfqpoint{0.963381in}{1.189010in}}%
\pgfpathlineto{\pgfqpoint{0.961716in}{1.184907in}}%
\pgfpathlineto{\pgfqpoint{0.960053in}{1.180917in}}%
\pgfpathlineto{\pgfqpoint{0.972867in}{1.184286in}}%
\pgfpathlineto{\pgfqpoint{0.985871in}{1.187450in}}%
\pgfpathlineto{\pgfqpoint{0.999050in}{1.190406in}}%
\pgfpathlineto{\pgfqpoint{1.012393in}{1.193153in}}%
\pgfpathlineto{\pgfqpoint{1.013658in}{1.197044in}}%
\pgfpathlineto{\pgfqpoint{1.014924in}{1.201047in}}%
\pgfpathlineto{\pgfqpoint{1.016191in}{1.205159in}}%
\pgfpathlineto{\pgfqpoint{1.017458in}{1.209376in}}%
\pgfpathlineto{\pgfqpoint{1.004520in}{1.206719in}}%
\pgfpathlineto{\pgfqpoint{0.991742in}{1.203859in}}%
\pgfpathlineto{\pgfqpoint{0.979135in}{1.200798in}}%
\pgfpathlineto{\pgfqpoint{0.966711in}{1.197538in}}%
\pgfpathclose%
\pgfusepath{fill}%
\end{pgfscope}%
\begin{pgfscope}%
\pgfpathrectangle{\pgfqpoint{0.041670in}{0.041670in}}{\pgfqpoint{2.216660in}{2.216660in}}%
\pgfusepath{clip}%
\pgfsetbuttcap%
\pgfsetroundjoin%
\definecolor{currentfill}{rgb}{0.280255,0.165693,0.476498}%
\pgfsetfillcolor{currentfill}%
\pgfsetlinewidth{0.000000pt}%
\definecolor{currentstroke}{rgb}{0.000000,0.000000,0.000000}%
\pgfsetstrokecolor{currentstroke}%
\pgfsetdash{}{0pt}%
\pgfpathmoveto{\pgfqpoint{0.910911in}{1.165445in}}%
\pgfpathlineto{\pgfqpoint{0.908872in}{1.161443in}}%
\pgfpathlineto{\pgfqpoint{0.906833in}{1.157561in}}%
\pgfpathlineto{\pgfqpoint{0.904794in}{1.153802in}}%
\pgfpathlineto{\pgfqpoint{0.902756in}{1.150169in}}%
\pgfpathlineto{\pgfqpoint{0.915067in}{1.154466in}}%
\pgfpathlineto{\pgfqpoint{0.927619in}{1.158562in}}%
\pgfpathlineto{\pgfqpoint{0.940402in}{1.162456in}}%
\pgfpathlineto{\pgfqpoint{0.953402in}{1.166144in}}%
\pgfpathlineto{\pgfqpoint{0.955064in}{1.169651in}}%
\pgfpathlineto{\pgfqpoint{0.956726in}{1.173285in}}%
\pgfpathlineto{\pgfqpoint{0.958389in}{1.177041in}}%
\pgfpathlineto{\pgfqpoint{0.960053in}{1.180917in}}%
\pgfpathlineto{\pgfqpoint{0.947438in}{1.177345in}}%
\pgfpathlineto{\pgfqpoint{0.935035in}{1.173573in}}%
\pgfpathlineto{\pgfqpoint{0.922856in}{1.169606in}}%
\pgfpathlineto{\pgfqpoint{0.910911in}{1.165445in}}%
\pgfpathclose%
\pgfusepath{fill}%
\end{pgfscope}%
\begin{pgfscope}%
\pgfpathrectangle{\pgfqpoint{0.041670in}{0.041670in}}{\pgfqpoint{2.216660in}{2.216660in}}%
\pgfusepath{clip}%
\pgfsetbuttcap%
\pgfsetroundjoin%
\definecolor{currentfill}{rgb}{0.283072,0.130895,0.449241}%
\pgfsetfillcolor{currentfill}%
\pgfsetlinewidth{0.000000pt}%
\definecolor{currentstroke}{rgb}{0.000000,0.000000,0.000000}%
\pgfsetstrokecolor{currentstroke}%
\pgfsetdash{}{0pt}%
\pgfpathmoveto{\pgfqpoint{1.446223in}{1.153998in}}%
\pgfpathlineto{\pgfqpoint{1.448180in}{1.150525in}}%
\pgfpathlineto{\pgfqpoint{1.450137in}{1.147187in}}%
\pgfpathlineto{\pgfqpoint{1.452093in}{1.143985in}}%
\pgfpathlineto{\pgfqpoint{1.454050in}{1.140925in}}%
\pgfpathlineto{\pgfqpoint{1.466700in}{1.136471in}}%
\pgfpathlineto{\pgfqpoint{1.479088in}{1.131816in}}%
\pgfpathlineto{\pgfqpoint{1.491204in}{1.126962in}}%
\pgfpathlineto{\pgfqpoint{1.503034in}{1.121915in}}%
\pgfpathlineto{\pgfqpoint{1.500721in}{1.125119in}}%
\pgfpathlineto{\pgfqpoint{1.498408in}{1.128465in}}%
\pgfpathlineto{\pgfqpoint{1.496096in}{1.131949in}}%
\pgfpathlineto{\pgfqpoint{1.493783in}{1.135567in}}%
\pgfpathlineto{\pgfqpoint{1.482298in}{1.140460in}}%
\pgfpathlineto{\pgfqpoint{1.470535in}{1.145166in}}%
\pgfpathlineto{\pgfqpoint{1.458507in}{1.149680in}}%
\pgfpathlineto{\pgfqpoint{1.446223in}{1.153998in}}%
\pgfpathclose%
\pgfusepath{fill}%
\end{pgfscope}%
\begin{pgfscope}%
\pgfpathrectangle{\pgfqpoint{0.041670in}{0.041670in}}{\pgfqpoint{2.216660in}{2.216660in}}%
\pgfusepath{clip}%
\pgfsetbuttcap%
\pgfsetroundjoin%
\definecolor{currentfill}{rgb}{0.267004,0.004874,0.329415}%
\pgfsetfillcolor{currentfill}%
\pgfsetlinewidth{0.000000pt}%
\definecolor{currentstroke}{rgb}{0.000000,0.000000,0.000000}%
\pgfsetstrokecolor{currentstroke}%
\pgfsetdash{}{0pt}%
\pgfpathmoveto{\pgfqpoint{0.705938in}{1.032967in}}%
\pgfpathlineto{\pgfqpoint{0.702904in}{1.033265in}}%
\pgfpathlineto{\pgfqpoint{0.699865in}{1.033796in}}%
\pgfpathlineto{\pgfqpoint{0.696822in}{1.034564in}}%
\pgfpathlineto{\pgfqpoint{0.693774in}{1.035574in}}%
\pgfpathlineto{\pgfqpoint{0.704208in}{1.043322in}}%
\pgfpathlineto{\pgfqpoint{0.715082in}{1.050890in}}%
\pgfpathlineto{\pgfqpoint{0.726385in}{1.058271in}}%
\pgfpathlineto{\pgfqpoint{0.738105in}{1.065461in}}%
\pgfpathlineto{\pgfqpoint{0.740863in}{1.064268in}}%
\pgfpathlineto{\pgfqpoint{0.743618in}{1.063317in}}%
\pgfpathlineto{\pgfqpoint{0.746368in}{1.062602in}}%
\pgfpathlineto{\pgfqpoint{0.749115in}{1.062120in}}%
\pgfpathlineto{\pgfqpoint{0.737698in}{1.055107in}}%
\pgfpathlineto{\pgfqpoint{0.726689in}{1.047906in}}%
\pgfpathlineto{\pgfqpoint{0.716099in}{1.040524in}}%
\pgfpathlineto{\pgfqpoint{0.705938in}{1.032967in}}%
\pgfpathclose%
\pgfusepath{fill}%
\end{pgfscope}%
\begin{pgfscope}%
\pgfpathrectangle{\pgfqpoint{0.041670in}{0.041670in}}{\pgfqpoint{2.216660in}{2.216660in}}%
\pgfusepath{clip}%
\pgfsetbuttcap%
\pgfsetroundjoin%
\definecolor{currentfill}{rgb}{0.263663,0.237631,0.518762}%
\pgfsetfillcolor{currentfill}%
\pgfsetlinewidth{0.000000pt}%
\definecolor{currentstroke}{rgb}{0.000000,0.000000,0.000000}%
\pgfsetstrokecolor{currentstroke}%
\pgfsetdash{}{0pt}%
\pgfpathmoveto{\pgfqpoint{1.287388in}{1.235302in}}%
\pgfpathlineto{\pgfqpoint{1.288253in}{1.230771in}}%
\pgfpathlineto{\pgfqpoint{1.289118in}{1.226330in}}%
\pgfpathlineto{\pgfqpoint{1.289982in}{1.221985in}}%
\pgfpathlineto{\pgfqpoint{1.290847in}{1.217737in}}%
\pgfpathlineto{\pgfqpoint{1.304290in}{1.215890in}}%
\pgfpathlineto{\pgfqpoint{1.317621in}{1.213832in}}%
\pgfpathlineto{\pgfqpoint{1.330827in}{1.211565in}}%
\pgfpathlineto{\pgfqpoint{1.329651in}{1.215865in}}%
\pgfpathlineto{\pgfqpoint{1.328474in}{1.220262in}}%
\pgfpathlineto{\pgfqpoint{1.327296in}{1.224755in}}%
\pgfpathlineto{\pgfqpoint{1.326117in}{1.229338in}}%
\pgfpathlineto{\pgfqpoint{1.313325in}{1.231529in}}%
\pgfpathlineto{\pgfqpoint{1.300411in}{1.233517in}}%
\pgfpathlineto{\pgfqpoint{1.287388in}{1.235302in}}%
\pgfpathclose%
\pgfusepath{fill}%
\end{pgfscope}%
\begin{pgfscope}%
\pgfpathrectangle{\pgfqpoint{0.041670in}{0.041670in}}{\pgfqpoint{2.216660in}{2.216660in}}%
\pgfusepath{clip}%
\pgfsetbuttcap%
\pgfsetroundjoin%
\definecolor{currentfill}{rgb}{0.279566,0.067836,0.391917}%
\pgfsetfillcolor{currentfill}%
\pgfsetlinewidth{0.000000pt}%
\definecolor{currentstroke}{rgb}{0.000000,0.000000,0.000000}%
\pgfsetstrokecolor{currentstroke}%
\pgfsetdash{}{0pt}%
\pgfpathmoveto{\pgfqpoint{0.792686in}{1.082430in}}%
\pgfpathlineto{\pgfqpoint{0.789977in}{1.079782in}}%
\pgfpathlineto{\pgfqpoint{0.787267in}{1.077299in}}%
\pgfpathlineto{\pgfqpoint{0.784555in}{1.074985in}}%
\pgfpathlineto{\pgfqpoint{0.781842in}{1.072844in}}%
\pgfpathlineto{\pgfqpoint{0.792729in}{1.079137in}}%
\pgfpathlineto{\pgfqpoint{0.803972in}{1.085247in}}%
\pgfpathlineto{\pgfqpoint{0.815561in}{1.091169in}}%
\pgfpathlineto{\pgfqpoint{0.827484in}{1.096898in}}%
\pgfpathlineto{\pgfqpoint{0.829878in}{1.098869in}}%
\pgfpathlineto{\pgfqpoint{0.832270in}{1.101012in}}%
\pgfpathlineto{\pgfqpoint{0.834662in}{1.103323in}}%
\pgfpathlineto{\pgfqpoint{0.837053in}{1.105799in}}%
\pgfpathlineto{\pgfqpoint{0.825461in}{1.100232in}}%
\pgfpathlineto{\pgfqpoint{0.814195in}{1.094479in}}%
\pgfpathlineto{\pgfqpoint{0.803267in}{1.088543in}}%
\pgfpathlineto{\pgfqpoint{0.792686in}{1.082430in}}%
\pgfpathclose%
\pgfusepath{fill}%
\end{pgfscope}%
\begin{pgfscope}%
\pgfpathrectangle{\pgfqpoint{0.041670in}{0.041670in}}{\pgfqpoint{2.216660in}{2.216660in}}%
\pgfusepath{clip}%
\pgfsetbuttcap%
\pgfsetroundjoin%
\definecolor{currentfill}{rgb}{0.172719,0.448791,0.557885}%
\pgfsetfillcolor{currentfill}%
\pgfsetlinewidth{0.000000pt}%
\definecolor{currentstroke}{rgb}{0.000000,0.000000,0.000000}%
\pgfsetstrokecolor{currentstroke}%
\pgfsetdash{}{0pt}%
\pgfpathmoveto{\pgfqpoint{1.658216in}{1.409282in}}%
\pgfpathlineto{\pgfqpoint{1.660894in}{1.425164in}}%
\pgfpathlineto{\pgfqpoint{1.663586in}{1.441536in}}%
\pgfpathlineto{\pgfqpoint{1.666292in}{1.458406in}}%
\pgfpathlineto{\pgfqpoint{1.683846in}{1.450900in}}%
\pgfpathlineto{\pgfqpoint{1.700956in}{1.443121in}}%
\pgfpathlineto{\pgfqpoint{1.717606in}{1.435078in}}%
\pgfpathlineto{\pgfqpoint{1.733780in}{1.426774in}}%
\pgfpathlineto{\pgfqpoint{1.730684in}{1.410001in}}%
\pgfpathlineto{\pgfqpoint{1.727604in}{1.393728in}}%
\pgfpathlineto{\pgfqpoint{1.724540in}{1.377948in}}%
\pgfpathlineto{\pgfqpoint{1.708646in}{1.386173in}}%
\pgfpathlineto{\pgfqpoint{1.692284in}{1.394141in}}%
\pgfpathlineto{\pgfqpoint{1.675468in}{1.401846in}}%
\pgfpathlineto{\pgfqpoint{1.658216in}{1.409282in}}%
\pgfpathclose%
\pgfusepath{fill}%
\end{pgfscope}%
\begin{pgfscope}%
\pgfpathrectangle{\pgfqpoint{0.041670in}{0.041670in}}{\pgfqpoint{2.216660in}{2.216660in}}%
\pgfusepath{clip}%
\pgfsetbuttcap%
\pgfsetroundjoin%
\definecolor{currentfill}{rgb}{0.263663,0.237631,0.518762}%
\pgfsetfillcolor{currentfill}%
\pgfsetlinewidth{0.000000pt}%
\definecolor{currentstroke}{rgb}{0.000000,0.000000,0.000000}%
\pgfsetstrokecolor{currentstroke}%
\pgfsetdash{}{0pt}%
\pgfpathmoveto{\pgfqpoint{1.022532in}{1.227223in}}%
\pgfpathlineto{\pgfqpoint{1.021263in}{1.222621in}}%
\pgfpathlineto{\pgfqpoint{1.019994in}{1.218110in}}%
\pgfpathlineto{\pgfqpoint{1.018726in}{1.213694in}}%
\pgfpathlineto{\pgfqpoint{1.017458in}{1.209376in}}%
\pgfpathlineto{\pgfqpoint{1.030543in}{1.211827in}}%
\pgfpathlineto{\pgfqpoint{1.043764in}{1.214071in}}%
\pgfpathlineto{\pgfqpoint{1.057108in}{1.216105in}}%
\pgfpathlineto{\pgfqpoint{1.070563in}{1.217929in}}%
\pgfpathlineto{\pgfqpoint{1.071416in}{1.222175in}}%
\pgfpathlineto{\pgfqpoint{1.072268in}{1.226519in}}%
\pgfpathlineto{\pgfqpoint{1.073122in}{1.230958in}}%
\pgfpathlineto{\pgfqpoint{1.073975in}{1.235488in}}%
\pgfpathlineto{\pgfqpoint{1.060941in}{1.233726in}}%
\pgfpathlineto{\pgfqpoint{1.048014in}{1.231760in}}%
\pgfpathlineto{\pgfqpoint{1.035207in}{1.229592in}}%
\pgfpathlineto{\pgfqpoint{1.022532in}{1.227223in}}%
\pgfpathclose%
\pgfusepath{fill}%
\end{pgfscope}%
\begin{pgfscope}%
\pgfpathrectangle{\pgfqpoint{0.041670in}{0.041670in}}{\pgfqpoint{2.216660in}{2.216660in}}%
\pgfusepath{clip}%
\pgfsetbuttcap%
\pgfsetroundjoin%
\definecolor{currentfill}{rgb}{0.276194,0.190074,0.493001}%
\pgfsetfillcolor{currentfill}%
\pgfsetlinewidth{0.000000pt}%
\definecolor{currentstroke}{rgb}{0.000000,0.000000,0.000000}%
\pgfsetstrokecolor{currentstroke}%
\pgfsetdash{}{0pt}%
\pgfpathmoveto{\pgfqpoint{0.618557in}{1.145312in}}%
\pgfpathlineto{\pgfqpoint{0.615308in}{1.154026in}}%
\pgfpathlineto{\pgfqpoint{0.612046in}{1.163122in}}%
\pgfpathlineto{\pgfqpoint{0.608773in}{1.172605in}}%
\pgfpathlineto{\pgfqpoint{0.605487in}{1.182484in}}%
\pgfpathlineto{\pgfqpoint{0.617928in}{1.191446in}}%
\pgfpathlineto{\pgfqpoint{0.630881in}{1.200197in}}%
\pgfpathlineto{\pgfqpoint{0.644333in}{1.208729in}}%
\pgfpathlineto{\pgfqpoint{0.658269in}{1.217036in}}%
\pgfpathlineto{\pgfqpoint{0.661239in}{1.207010in}}%
\pgfpathlineto{\pgfqpoint{0.664196in}{1.197377in}}%
\pgfpathlineto{\pgfqpoint{0.667143in}{1.188130in}}%
\pgfpathlineto{\pgfqpoint{0.670080in}{1.179264in}}%
\pgfpathlineto{\pgfqpoint{0.656474in}{1.171101in}}%
\pgfpathlineto{\pgfqpoint{0.643342in}{1.162716in}}%
\pgfpathlineto{\pgfqpoint{0.630698in}{1.154118in}}%
\pgfpathlineto{\pgfqpoint{0.618557in}{1.145312in}}%
\pgfpathclose%
\pgfusepath{fill}%
\end{pgfscope}%
\begin{pgfscope}%
\pgfpathrectangle{\pgfqpoint{0.041670in}{0.041670in}}{\pgfqpoint{2.216660in}{2.216660in}}%
\pgfusepath{clip}%
\pgfsetbuttcap%
\pgfsetroundjoin%
\definecolor{currentfill}{rgb}{0.260571,0.246922,0.522828}%
\pgfsetfillcolor{currentfill}%
\pgfsetlinewidth{0.000000pt}%
\definecolor{currentstroke}{rgb}{0.000000,0.000000,0.000000}%
\pgfsetstrokecolor{currentstroke}%
\pgfsetdash{}{0pt}%
\pgfpathmoveto{\pgfqpoint{1.688857in}{1.224225in}}%
\pgfpathlineto{\pgfqpoint{1.691762in}{1.234680in}}%
\pgfpathlineto{\pgfqpoint{1.694678in}{1.245540in}}%
\pgfpathlineto{\pgfqpoint{1.697605in}{1.256812in}}%
\pgfpathlineto{\pgfqpoint{1.700545in}{1.268503in}}%
\pgfpathlineto{\pgfqpoint{1.715244in}{1.260270in}}%
\pgfpathlineto{\pgfqpoint{1.729462in}{1.251800in}}%
\pgfpathlineto{\pgfqpoint{1.743183in}{1.243102in}}%
\pgfpathlineto{\pgfqpoint{1.756394in}{1.234182in}}%
\pgfpathlineto{\pgfqpoint{1.753123in}{1.222627in}}%
\pgfpathlineto{\pgfqpoint{1.749865in}{1.211494in}}%
\pgfpathlineto{\pgfqpoint{1.746621in}{1.200773in}}%
\pgfpathlineto{\pgfqpoint{1.743389in}{1.190461in}}%
\pgfpathlineto{\pgfqpoint{1.730492in}{1.199235in}}%
\pgfpathlineto{\pgfqpoint{1.717095in}{1.207792in}}%
\pgfpathlineto{\pgfqpoint{1.703212in}{1.216124in}}%
\pgfpathlineto{\pgfqpoint{1.688857in}{1.224225in}}%
\pgfpathclose%
\pgfusepath{fill}%
\end{pgfscope}%
\begin{pgfscope}%
\pgfpathrectangle{\pgfqpoint{0.041670in}{0.041670in}}{\pgfqpoint{2.216660in}{2.216660in}}%
\pgfusepath{clip}%
\pgfsetbuttcap%
\pgfsetroundjoin%
\definecolor{currentfill}{rgb}{0.283072,0.130895,0.449241}%
\pgfsetfillcolor{currentfill}%
\pgfsetlinewidth{0.000000pt}%
\definecolor{currentstroke}{rgb}{0.000000,0.000000,0.000000}%
\pgfsetstrokecolor{currentstroke}%
\pgfsetdash{}{0pt}%
\pgfpathmoveto{\pgfqpoint{0.856159in}{1.131062in}}%
\pgfpathlineto{\pgfqpoint{0.853772in}{1.127409in}}%
\pgfpathlineto{\pgfqpoint{0.851384in}{1.123890in}}%
\pgfpathlineto{\pgfqpoint{0.848997in}{1.120508in}}%
\pgfpathlineto{\pgfqpoint{0.846609in}{1.117268in}}%
\pgfpathlineto{\pgfqpoint{0.858176in}{1.122485in}}%
\pgfpathlineto{\pgfqpoint{0.870038in}{1.127511in}}%
\pgfpathlineto{\pgfqpoint{0.882184in}{1.132343in}}%
\pgfpathlineto{\pgfqpoint{0.894602in}{1.136976in}}%
\pgfpathlineto{\pgfqpoint{0.896641in}{1.140066in}}%
\pgfpathlineto{\pgfqpoint{0.898679in}{1.143298in}}%
\pgfpathlineto{\pgfqpoint{0.900717in}{1.146667in}}%
\pgfpathlineto{\pgfqpoint{0.902756in}{1.150169in}}%
\pgfpathlineto{\pgfqpoint{0.890698in}{1.145677in}}%
\pgfpathlineto{\pgfqpoint{0.878905in}{1.140992in}}%
\pgfpathlineto{\pgfqpoint{0.867388in}{1.136119in}}%
\pgfpathlineto{\pgfqpoint{0.856159in}{1.131062in}}%
\pgfpathclose%
\pgfusepath{fill}%
\end{pgfscope}%
\begin{pgfscope}%
\pgfpathrectangle{\pgfqpoint{0.041670in}{0.041670in}}{\pgfqpoint{2.216660in}{2.216660in}}%
\pgfusepath{clip}%
\pgfsetbuttcap%
\pgfsetroundjoin%
\definecolor{currentfill}{rgb}{0.268510,0.009605,0.335427}%
\pgfsetfillcolor{currentfill}%
\pgfsetlinewidth{0.000000pt}%
\definecolor{currentstroke}{rgb}{0.000000,0.000000,0.000000}%
\pgfsetstrokecolor{currentstroke}%
\pgfsetdash{}{0pt}%
\pgfpathmoveto{\pgfqpoint{0.718037in}{1.034015in}}%
\pgfpathlineto{\pgfqpoint{0.715018in}{1.033426in}}%
\pgfpathlineto{\pgfqpoint{0.711995in}{1.033052in}}%
\pgfpathlineto{\pgfqpoint{0.708969in}{1.032898in}}%
\pgfpathlineto{\pgfqpoint{0.705938in}{1.032967in}}%
\pgfpathlineto{\pgfqpoint{0.716099in}{1.040524in}}%
\pgfpathlineto{\pgfqpoint{0.726689in}{1.047906in}}%
\pgfpathlineto{\pgfqpoint{0.737698in}{1.055107in}}%
\pgfpathlineto{\pgfqpoint{0.749115in}{1.062120in}}%
\pgfpathlineto{\pgfqpoint{0.751858in}{1.061865in}}%
\pgfpathlineto{\pgfqpoint{0.754597in}{1.061834in}}%
\pgfpathlineto{\pgfqpoint{0.757334in}{1.062021in}}%
\pgfpathlineto{\pgfqpoint{0.760067in}{1.062423in}}%
\pgfpathlineto{\pgfqpoint{0.748951in}{1.055588in}}%
\pgfpathlineto{\pgfqpoint{0.738234in}{1.048571in}}%
\pgfpathlineto{\pgfqpoint{0.727925in}{1.041378in}}%
\pgfpathlineto{\pgfqpoint{0.718037in}{1.034015in}}%
\pgfpathclose%
\pgfusepath{fill}%
\end{pgfscope}%
\begin{pgfscope}%
\pgfpathrectangle{\pgfqpoint{0.041670in}{0.041670in}}{\pgfqpoint{2.216660in}{2.216660in}}%
\pgfusepath{clip}%
\pgfsetbuttcap%
\pgfsetroundjoin%
\definecolor{currentfill}{rgb}{0.271305,0.019942,0.347269}%
\pgfsetfillcolor{currentfill}%
\pgfsetlinewidth{0.000000pt}%
\definecolor{currentstroke}{rgb}{0.000000,0.000000,0.000000}%
\pgfsetstrokecolor{currentstroke}%
\pgfsetdash{}{0pt}%
\pgfpathmoveto{\pgfqpoint{1.579006in}{1.071849in}}%
\pgfpathlineto{\pgfqpoint{1.581660in}{1.070671in}}%
\pgfpathlineto{\pgfqpoint{1.584317in}{1.069691in}}%
\pgfpathlineto{\pgfqpoint{1.586976in}{1.068913in}}%
\pgfpathlineto{\pgfqpoint{1.589638in}{1.068340in}}%
\pgfpathlineto{\pgfqpoint{1.601097in}{1.061672in}}%
\pgfpathlineto{\pgfqpoint{1.612169in}{1.054817in}}%
\pgfpathlineto{\pgfqpoint{1.622842in}{1.047781in}}%
\pgfpathlineto{\pgfqpoint{1.633104in}{1.040569in}}%
\pgfpathlineto{\pgfqpoint{1.630148in}{1.041325in}}%
\pgfpathlineto{\pgfqpoint{1.627196in}{1.042287in}}%
\pgfpathlineto{\pgfqpoint{1.624246in}{1.043452in}}%
\pgfpathlineto{\pgfqpoint{1.621299in}{1.044814in}}%
\pgfpathlineto{\pgfqpoint{1.611315in}{1.051835in}}%
\pgfpathlineto{\pgfqpoint{1.600931in}{1.058684in}}%
\pgfpathlineto{\pgfqpoint{1.590158in}{1.065357in}}%
\pgfpathlineto{\pgfqpoint{1.579006in}{1.071849in}}%
\pgfpathclose%
\pgfusepath{fill}%
\end{pgfscope}%
\begin{pgfscope}%
\pgfpathrectangle{\pgfqpoint{0.041670in}{0.041670in}}{\pgfqpoint{2.216660in}{2.216660in}}%
\pgfusepath{clip}%
\pgfsetbuttcap%
\pgfsetroundjoin%
\definecolor{currentfill}{rgb}{0.282327,0.094955,0.417331}%
\pgfsetfillcolor{currentfill}%
\pgfsetlinewidth{0.000000pt}%
\definecolor{currentstroke}{rgb}{0.000000,0.000000,0.000000}%
\pgfsetstrokecolor{currentstroke}%
\pgfsetdash{}{0pt}%
\pgfpathmoveto{\pgfqpoint{1.503034in}{1.121915in}}%
\pgfpathlineto{\pgfqpoint{1.505347in}{1.118855in}}%
\pgfpathlineto{\pgfqpoint{1.507660in}{1.115945in}}%
\pgfpathlineto{\pgfqpoint{1.509974in}{1.113187in}}%
\pgfpathlineto{\pgfqpoint{1.512289in}{1.110586in}}%
\pgfpathlineto{\pgfqpoint{1.524161in}{1.105190in}}%
\pgfpathlineto{\pgfqpoint{1.535717in}{1.099602in}}%
\pgfpathlineto{\pgfqpoint{1.546945in}{1.093828in}}%
\pgfpathlineto{\pgfqpoint{1.557836in}{1.087872in}}%
\pgfpathlineto{\pgfqpoint{1.555196in}{1.090640in}}%
\pgfpathlineto{\pgfqpoint{1.552556in}{1.093566in}}%
\pgfpathlineto{\pgfqpoint{1.549917in}{1.096644in}}%
\pgfpathlineto{\pgfqpoint{1.547279in}{1.099871in}}%
\pgfpathlineto{\pgfqpoint{1.536701in}{1.105651in}}%
\pgfpathlineto{\pgfqpoint{1.525794in}{1.111254in}}%
\pgfpathlineto{\pgfqpoint{1.514568in}{1.116677in}}%
\pgfpathlineto{\pgfqpoint{1.503034in}{1.121915in}}%
\pgfpathclose%
\pgfusepath{fill}%
\end{pgfscope}%
\begin{pgfscope}%
\pgfpathrectangle{\pgfqpoint{0.041670in}{0.041670in}}{\pgfqpoint{2.216660in}{2.216660in}}%
\pgfusepath{clip}%
\pgfsetbuttcap%
\pgfsetroundjoin%
\definecolor{currentfill}{rgb}{0.248629,0.278775,0.534556}%
\pgfsetfillcolor{currentfill}%
\pgfsetlinewidth{0.000000pt}%
\definecolor{currentstroke}{rgb}{0.000000,0.000000,0.000000}%
\pgfsetstrokecolor{currentstroke}%
\pgfsetdash{}{0pt}%
\pgfpathmoveto{\pgfqpoint{1.128665in}{1.259261in}}%
\pgfpathlineto{\pgfqpoint{1.128237in}{1.254444in}}%
\pgfpathlineto{\pgfqpoint{1.127810in}{1.249704in}}%
\pgfpathlineto{\pgfqpoint{1.127383in}{1.245045in}}%
\pgfpathlineto{\pgfqpoint{1.126956in}{1.240471in}}%
\pgfpathlineto{\pgfqpoint{1.140351in}{1.241196in}}%
\pgfpathlineto{\pgfqpoint{1.153782in}{1.241710in}}%
\pgfpathlineto{\pgfqpoint{1.167237in}{1.242015in}}%
\pgfpathlineto{\pgfqpoint{1.180703in}{1.242108in}}%
\pgfpathlineto{\pgfqpoint{1.180697in}{1.246668in}}%
\pgfpathlineto{\pgfqpoint{1.180691in}{1.251313in}}%
\pgfpathlineto{\pgfqpoint{1.180685in}{1.256038in}}%
\pgfpathlineto{\pgfqpoint{1.180679in}{1.260841in}}%
\pgfpathlineto{\pgfqpoint{1.167647in}{1.260751in}}%
\pgfpathlineto{\pgfqpoint{1.154626in}{1.260457in}}%
\pgfpathlineto{\pgfqpoint{1.141628in}{1.259961in}}%
\pgfpathlineto{\pgfqpoint{1.128665in}{1.259261in}}%
\pgfpathclose%
\pgfusepath{fill}%
\end{pgfscope}%
\begin{pgfscope}%
\pgfpathrectangle{\pgfqpoint{0.041670in}{0.041670in}}{\pgfqpoint{2.216660in}{2.216660in}}%
\pgfusepath{clip}%
\pgfsetbuttcap%
\pgfsetroundjoin%
\definecolor{currentfill}{rgb}{0.248629,0.278775,0.534556}%
\pgfsetfillcolor{currentfill}%
\pgfsetlinewidth{0.000000pt}%
\definecolor{currentstroke}{rgb}{0.000000,0.000000,0.000000}%
\pgfsetstrokecolor{currentstroke}%
\pgfsetdash{}{0pt}%
\pgfpathmoveto{\pgfqpoint{1.180679in}{1.260841in}}%
\pgfpathlineto{\pgfqpoint{1.180685in}{1.256038in}}%
\pgfpathlineto{\pgfqpoint{1.180691in}{1.251313in}}%
\pgfpathlineto{\pgfqpoint{1.180697in}{1.246668in}}%
\pgfpathlineto{\pgfqpoint{1.180703in}{1.242108in}}%
\pgfpathlineto{\pgfqpoint{1.194169in}{1.241991in}}%
\pgfpathlineto{\pgfqpoint{1.207621in}{1.241664in}}%
\pgfpathlineto{\pgfqpoint{1.221049in}{1.241126in}}%
\pgfpathlineto{\pgfqpoint{1.234440in}{1.240378in}}%
\pgfpathlineto{\pgfqpoint{1.234001in}{1.244953in}}%
\pgfpathlineto{\pgfqpoint{1.233562in}{1.249612in}}%
\pgfpathlineto{\pgfqpoint{1.233122in}{1.254353in}}%
\pgfpathlineto{\pgfqpoint{1.232682in}{1.259171in}}%
\pgfpathlineto{\pgfqpoint{1.219724in}{1.259893in}}%
\pgfpathlineto{\pgfqpoint{1.206729in}{1.260412in}}%
\pgfpathlineto{\pgfqpoint{1.193710in}{1.260728in}}%
\pgfpathlineto{\pgfqpoint{1.180679in}{1.260841in}}%
\pgfpathclose%
\pgfusepath{fill}%
\end{pgfscope}%
\begin{pgfscope}%
\pgfpathrectangle{\pgfqpoint{0.041670in}{0.041670in}}{\pgfqpoint{2.216660in}{2.216660in}}%
\pgfusepath{clip}%
\pgfsetbuttcap%
\pgfsetroundjoin%
\definecolor{currentfill}{rgb}{0.274128,0.199721,0.498911}%
\pgfsetfillcolor{currentfill}%
\pgfsetlinewidth{0.000000pt}%
\definecolor{currentstroke}{rgb}{0.000000,0.000000,0.000000}%
\pgfsetstrokecolor{currentstroke}%
\pgfsetdash{}{0pt}%
\pgfpathmoveto{\pgfqpoint{1.382164in}{1.200445in}}%
\pgfpathlineto{\pgfqpoint{1.383743in}{1.196153in}}%
\pgfpathlineto{\pgfqpoint{1.385321in}{1.191966in}}%
\pgfpathlineto{\pgfqpoint{1.386899in}{1.187888in}}%
\pgfpathlineto{\pgfqpoint{1.388476in}{1.183922in}}%
\pgfpathlineto{\pgfqpoint{1.401269in}{1.180530in}}%
\pgfpathlineto{\pgfqpoint{1.413861in}{1.176935in}}%
\pgfpathlineto{\pgfqpoint{1.426239in}{1.173142in}}%
\pgfpathlineto{\pgfqpoint{1.438393in}{1.169153in}}%
\pgfpathlineto{\pgfqpoint{1.436434in}{1.173239in}}%
\pgfpathlineto{\pgfqpoint{1.434475in}{1.177438in}}%
\pgfpathlineto{\pgfqpoint{1.432515in}{1.181746in}}%
\pgfpathlineto{\pgfqpoint{1.430555in}{1.186159in}}%
\pgfpathlineto{\pgfqpoint{1.418774in}{1.190018in}}%
\pgfpathlineto{\pgfqpoint{1.406774in}{1.193687in}}%
\pgfpathlineto{\pgfqpoint{1.394567in}{1.197164in}}%
\pgfpathlineto{\pgfqpoint{1.382164in}{1.200445in}}%
\pgfpathclose%
\pgfusepath{fill}%
\end{pgfscope}%
\begin{pgfscope}%
\pgfpathrectangle{\pgfqpoint{0.041670in}{0.041670in}}{\pgfqpoint{2.216660in}{2.216660in}}%
\pgfusepath{clip}%
\pgfsetbuttcap%
\pgfsetroundjoin%
\definecolor{currentfill}{rgb}{0.172719,0.448791,0.557885}%
\pgfsetfillcolor{currentfill}%
\pgfsetlinewidth{0.000000pt}%
\definecolor{currentstroke}{rgb}{0.000000,0.000000,0.000000}%
\pgfsetstrokecolor{currentstroke}%
\pgfsetdash{}{0pt}%
\pgfpathmoveto{\pgfqpoint{0.621650in}{1.370426in}}%
\pgfpathlineto{\pgfqpoint{0.618505in}{1.386182in}}%
\pgfpathlineto{\pgfqpoint{0.615344in}{1.402431in}}%
\pgfpathlineto{\pgfqpoint{0.612166in}{1.419180in}}%
\pgfpathlineto{\pgfqpoint{0.627903in}{1.427709in}}%
\pgfpathlineto{\pgfqpoint{0.644131in}{1.435984in}}%
\pgfpathlineto{\pgfqpoint{0.660833in}{1.443999in}}%
\pgfpathlineto{\pgfqpoint{0.677993in}{1.451747in}}%
\pgfpathlineto{\pgfqpoint{0.680789in}{1.434898in}}%
\pgfpathlineto{\pgfqpoint{0.683570in}{1.418546in}}%
\pgfpathlineto{\pgfqpoint{0.686337in}{1.402686in}}%
\pgfpathlineto{\pgfqpoint{0.669472in}{1.395010in}}%
\pgfpathlineto{\pgfqpoint{0.653058in}{1.387071in}}%
\pgfpathlineto{\pgfqpoint{0.637112in}{1.378874in}}%
\pgfpathlineto{\pgfqpoint{0.621650in}{1.370426in}}%
\pgfpathclose%
\pgfusepath{fill}%
\end{pgfscope}%
\begin{pgfscope}%
\pgfpathrectangle{\pgfqpoint{0.041670in}{0.041670in}}{\pgfqpoint{2.216660in}{2.216660in}}%
\pgfusepath{clip}%
\pgfsetbuttcap%
\pgfsetroundjoin%
\definecolor{currentfill}{rgb}{0.263663,0.237631,0.518762}%
\pgfsetfillcolor{currentfill}%
\pgfsetlinewidth{0.000000pt}%
\definecolor{currentstroke}{rgb}{0.000000,0.000000,0.000000}%
\pgfsetstrokecolor{currentstroke}%
\pgfsetdash{}{0pt}%
\pgfpathmoveto{\pgfqpoint{1.326117in}{1.229338in}}%
\pgfpathlineto{\pgfqpoint{1.327296in}{1.224755in}}%
\pgfpathlineto{\pgfqpoint{1.328474in}{1.220262in}}%
\pgfpathlineto{\pgfqpoint{1.329651in}{1.215865in}}%
\pgfpathlineto{\pgfqpoint{1.330827in}{1.211565in}}%
\pgfpathlineto{\pgfqpoint{1.343897in}{1.209091in}}%
\pgfpathlineto{\pgfqpoint{1.356817in}{1.206411in}}%
\pgfpathlineto{\pgfqpoint{1.369577in}{1.203529in}}%
\pgfpathlineto{\pgfqpoint{1.382164in}{1.200445in}}%
\pgfpathlineto{\pgfqpoint{1.380585in}{1.204839in}}%
\pgfpathlineto{\pgfqpoint{1.379005in}{1.209330in}}%
\pgfpathlineto{\pgfqpoint{1.377424in}{1.213917in}}%
\pgfpathlineto{\pgfqpoint{1.375842in}{1.218594in}}%
\pgfpathlineto{\pgfqpoint{1.363651in}{1.221573in}}%
\pgfpathlineto{\pgfqpoint{1.351292in}{1.224358in}}%
\pgfpathlineto{\pgfqpoint{1.338777in}{1.226947in}}%
\pgfpathlineto{\pgfqpoint{1.326117in}{1.229338in}}%
\pgfpathclose%
\pgfusepath{fill}%
\end{pgfscope}%
\begin{pgfscope}%
\pgfpathrectangle{\pgfqpoint{0.041670in}{0.041670in}}{\pgfqpoint{2.216660in}{2.216660in}}%
\pgfusepath{clip}%
\pgfsetbuttcap%
\pgfsetroundjoin%
\definecolor{currentfill}{rgb}{0.260571,0.246922,0.522828}%
\pgfsetfillcolor{currentfill}%
\pgfsetlinewidth{0.000000pt}%
\definecolor{currentstroke}{rgb}{0.000000,0.000000,0.000000}%
\pgfsetstrokecolor{currentstroke}%
\pgfsetdash{}{0pt}%
\pgfpathmoveto{\pgfqpoint{0.605487in}{1.182484in}}%
\pgfpathlineto{\pgfqpoint{0.602189in}{1.192763in}}%
\pgfpathlineto{\pgfqpoint{0.598877in}{1.203449in}}%
\pgfpathlineto{\pgfqpoint{0.595551in}{1.214550in}}%
\pgfpathlineto{\pgfqpoint{0.592212in}{1.226072in}}%
\pgfpathlineto{\pgfqpoint{0.604958in}{1.235184in}}%
\pgfpathlineto{\pgfqpoint{0.618226in}{1.244080in}}%
\pgfpathlineto{\pgfqpoint{0.632004in}{1.252753in}}%
\pgfpathlineto{\pgfqpoint{0.646275in}{1.261196in}}%
\pgfpathlineto{\pgfqpoint{0.649292in}{1.249534in}}%
\pgfpathlineto{\pgfqpoint{0.652297in}{1.238291in}}%
\pgfpathlineto{\pgfqpoint{0.655289in}{1.227461in}}%
\pgfpathlineto{\pgfqpoint{0.658269in}{1.217036in}}%
\pgfpathlineto{\pgfqpoint{0.644333in}{1.208729in}}%
\pgfpathlineto{\pgfqpoint{0.630881in}{1.200197in}}%
\pgfpathlineto{\pgfqpoint{0.617928in}{1.191446in}}%
\pgfpathlineto{\pgfqpoint{0.605487in}{1.182484in}}%
\pgfpathclose%
\pgfusepath{fill}%
\end{pgfscope}%
\begin{pgfscope}%
\pgfpathrectangle{\pgfqpoint{0.041670in}{0.041670in}}{\pgfqpoint{2.216660in}{2.216660in}}%
\pgfusepath{clip}%
\pgfsetbuttcap%
\pgfsetroundjoin%
\definecolor{currentfill}{rgb}{0.271305,0.019942,0.347269}%
\pgfsetfillcolor{currentfill}%
\pgfsetlinewidth{0.000000pt}%
\definecolor{currentstroke}{rgb}{0.000000,0.000000,0.000000}%
\pgfsetstrokecolor{currentstroke}%
\pgfsetdash{}{0pt}%
\pgfpathmoveto{\pgfqpoint{0.730081in}{1.038436in}}%
\pgfpathlineto{\pgfqpoint{0.727075in}{1.037030in}}%
\pgfpathlineto{\pgfqpoint{0.724065in}{1.035821in}}%
\pgfpathlineto{\pgfqpoint{0.721052in}{1.034815in}}%
\pgfpathlineto{\pgfqpoint{0.718037in}{1.034015in}}%
\pgfpathlineto{\pgfqpoint{0.727925in}{1.041378in}}%
\pgfpathlineto{\pgfqpoint{0.738234in}{1.048571in}}%
\pgfpathlineto{\pgfqpoint{0.748951in}{1.055588in}}%
\pgfpathlineto{\pgfqpoint{0.760067in}{1.062423in}}%
\pgfpathlineto{\pgfqpoint{0.762797in}{1.063034in}}%
\pgfpathlineto{\pgfqpoint{0.765525in}{1.063852in}}%
\pgfpathlineto{\pgfqpoint{0.768250in}{1.064871in}}%
\pgfpathlineto{\pgfqpoint{0.770972in}{1.066088in}}%
\pgfpathlineto{\pgfqpoint{0.760156in}{1.059435in}}%
\pgfpathlineto{\pgfqpoint{0.749729in}{1.052604in}}%
\pgfpathlineto{\pgfqpoint{0.739700in}{1.045603in}}%
\pgfpathlineto{\pgfqpoint{0.730081in}{1.038436in}}%
\pgfpathclose%
\pgfusepath{fill}%
\end{pgfscope}%
\begin{pgfscope}%
\pgfpathrectangle{\pgfqpoint{0.041670in}{0.041670in}}{\pgfqpoint{2.216660in}{2.216660in}}%
\pgfusepath{clip}%
\pgfsetbuttcap%
\pgfsetroundjoin%
\definecolor{currentfill}{rgb}{0.280255,0.165693,0.476498}%
\pgfsetfillcolor{currentfill}%
\pgfsetlinewidth{0.000000pt}%
\definecolor{currentstroke}{rgb}{0.000000,0.000000,0.000000}%
\pgfsetstrokecolor{currentstroke}%
\pgfsetdash{}{0pt}%
\pgfpathmoveto{\pgfqpoint{1.438393in}{1.169153in}}%
\pgfpathlineto{\pgfqpoint{1.440351in}{1.165182in}}%
\pgfpathlineto{\pgfqpoint{1.442309in}{1.161330in}}%
\pgfpathlineto{\pgfqpoint{1.444266in}{1.157601in}}%
\pgfpathlineto{\pgfqpoint{1.446223in}{1.153998in}}%
\pgfpathlineto{\pgfqpoint{1.458507in}{1.149680in}}%
\pgfpathlineto{\pgfqpoint{1.470535in}{1.145166in}}%
\pgfpathlineto{\pgfqpoint{1.482298in}{1.140460in}}%
\pgfpathlineto{\pgfqpoint{1.493783in}{1.135567in}}%
\pgfpathlineto{\pgfqpoint{1.491471in}{1.139314in}}%
\pgfpathlineto{\pgfqpoint{1.489158in}{1.143189in}}%
\pgfpathlineto{\pgfqpoint{1.486845in}{1.147187in}}%
\pgfpathlineto{\pgfqpoint{1.484531in}{1.151304in}}%
\pgfpathlineto{\pgfqpoint{1.473390in}{1.156042in}}%
\pgfpathlineto{\pgfqpoint{1.461980in}{1.160599in}}%
\pgfpathlineto{\pgfqpoint{1.450310in}{1.164971in}}%
\pgfpathlineto{\pgfqpoint{1.438393in}{1.169153in}}%
\pgfpathclose%
\pgfusepath{fill}%
\end{pgfscope}%
\begin{pgfscope}%
\pgfpathrectangle{\pgfqpoint{0.041670in}{0.041670in}}{\pgfqpoint{2.216660in}{2.216660in}}%
\pgfusepath{clip}%
\pgfsetbuttcap%
\pgfsetroundjoin%
\definecolor{currentfill}{rgb}{0.248629,0.278775,0.534556}%
\pgfsetfillcolor{currentfill}%
\pgfsetlinewidth{0.000000pt}%
\definecolor{currentstroke}{rgb}{0.000000,0.000000,0.000000}%
\pgfsetstrokecolor{currentstroke}%
\pgfsetdash{}{0pt}%
\pgfpathmoveto{\pgfqpoint{1.077396in}{1.254452in}}%
\pgfpathlineto{\pgfqpoint{1.076540in}{1.249591in}}%
\pgfpathlineto{\pgfqpoint{1.075684in}{1.244808in}}%
\pgfpathlineto{\pgfqpoint{1.074830in}{1.240105in}}%
\pgfpathlineto{\pgfqpoint{1.073975in}{1.235488in}}%
\pgfpathlineto{\pgfqpoint{1.087106in}{1.237045in}}%
\pgfpathlineto{\pgfqpoint{1.100321in}{1.238395in}}%
\pgfpathlineto{\pgfqpoint{1.113608in}{1.239537in}}%
\pgfpathlineto{\pgfqpoint{1.126956in}{1.240471in}}%
\pgfpathlineto{\pgfqpoint{1.127383in}{1.245045in}}%
\pgfpathlineto{\pgfqpoint{1.127810in}{1.249704in}}%
\pgfpathlineto{\pgfqpoint{1.128237in}{1.254444in}}%
\pgfpathlineto{\pgfqpoint{1.128665in}{1.259261in}}%
\pgfpathlineto{\pgfqpoint{1.115749in}{1.258360in}}%
\pgfpathlineto{\pgfqpoint{1.102890in}{1.257257in}}%
\pgfpathlineto{\pgfqpoint{1.090102in}{1.255954in}}%
\pgfpathlineto{\pgfqpoint{1.077396in}{1.254452in}}%
\pgfpathclose%
\pgfusepath{fill}%
\end{pgfscope}%
\begin{pgfscope}%
\pgfpathrectangle{\pgfqpoint{0.041670in}{0.041670in}}{\pgfqpoint{2.216660in}{2.216660in}}%
\pgfusepath{clip}%
\pgfsetbuttcap%
\pgfsetroundjoin%
\definecolor{currentfill}{rgb}{0.248629,0.278775,0.534556}%
\pgfsetfillcolor{currentfill}%
\pgfsetlinewidth{0.000000pt}%
\definecolor{currentstroke}{rgb}{0.000000,0.000000,0.000000}%
\pgfsetstrokecolor{currentstroke}%
\pgfsetdash{}{0pt}%
\pgfpathmoveto{\pgfqpoint{1.232682in}{1.259171in}}%
\pgfpathlineto{\pgfqpoint{1.233122in}{1.254353in}}%
\pgfpathlineto{\pgfqpoint{1.233562in}{1.249612in}}%
\pgfpathlineto{\pgfqpoint{1.234001in}{1.244953in}}%
\pgfpathlineto{\pgfqpoint{1.234440in}{1.240378in}}%
\pgfpathlineto{\pgfqpoint{1.247781in}{1.239421in}}%
\pgfpathlineto{\pgfqpoint{1.261061in}{1.238255in}}%
\pgfpathlineto{\pgfqpoint{1.274267in}{1.236882in}}%
\pgfpathlineto{\pgfqpoint{1.287388in}{1.235302in}}%
\pgfpathlineto{\pgfqpoint{1.286522in}{1.239921in}}%
\pgfpathlineto{\pgfqpoint{1.285655in}{1.244625in}}%
\pgfpathlineto{\pgfqpoint{1.284788in}{1.249410in}}%
\pgfpathlineto{\pgfqpoint{1.283920in}{1.254273in}}%
\pgfpathlineto{\pgfqpoint{1.271223in}{1.255797in}}%
\pgfpathlineto{\pgfqpoint{1.258444in}{1.257123in}}%
\pgfpathlineto{\pgfqpoint{1.245593in}{1.258247in}}%
\pgfpathlineto{\pgfqpoint{1.232682in}{1.259171in}}%
\pgfpathclose%
\pgfusepath{fill}%
\end{pgfscope}%
\begin{pgfscope}%
\pgfpathrectangle{\pgfqpoint{0.041670in}{0.041670in}}{\pgfqpoint{2.216660in}{2.216660in}}%
\pgfusepath{clip}%
\pgfsetbuttcap%
\pgfsetroundjoin%
\definecolor{currentfill}{rgb}{0.274128,0.199721,0.498911}%
\pgfsetfillcolor{currentfill}%
\pgfsetlinewidth{0.000000pt}%
\definecolor{currentstroke}{rgb}{0.000000,0.000000,0.000000}%
\pgfsetstrokecolor{currentstroke}%
\pgfsetdash{}{0pt}%
\pgfpathmoveto{\pgfqpoint{0.919076in}{1.182572in}}%
\pgfpathlineto{\pgfqpoint{0.917034in}{1.178129in}}%
\pgfpathlineto{\pgfqpoint{0.914992in}{1.173791in}}%
\pgfpathlineto{\pgfqpoint{0.912952in}{1.169562in}}%
\pgfpathlineto{\pgfqpoint{0.910911in}{1.165445in}}%
\pgfpathlineto{\pgfqpoint{0.922856in}{1.169606in}}%
\pgfpathlineto{\pgfqpoint{0.935035in}{1.173573in}}%
\pgfpathlineto{\pgfqpoint{0.947438in}{1.177345in}}%
\pgfpathlineto{\pgfqpoint{0.960053in}{1.180917in}}%
\pgfpathlineto{\pgfqpoint{0.961716in}{1.184907in}}%
\pgfpathlineto{\pgfqpoint{0.963381in}{1.189010in}}%
\pgfpathlineto{\pgfqpoint{0.965045in}{1.193222in}}%
\pgfpathlineto{\pgfqpoint{0.966711in}{1.197538in}}%
\pgfpathlineto{\pgfqpoint{0.954482in}{1.194083in}}%
\pgfpathlineto{\pgfqpoint{0.942459in}{1.190435in}}%
\pgfpathlineto{\pgfqpoint{0.930653in}{1.186597in}}%
\pgfpathlineto{\pgfqpoint{0.919076in}{1.182572in}}%
\pgfpathclose%
\pgfusepath{fill}%
\end{pgfscope}%
\begin{pgfscope}%
\pgfpathrectangle{\pgfqpoint{0.041670in}{0.041670in}}{\pgfqpoint{2.216660in}{2.216660in}}%
\pgfusepath{clip}%
\pgfsetbuttcap%
\pgfsetroundjoin%
\definecolor{currentfill}{rgb}{0.274952,0.037752,0.364543}%
\pgfsetfillcolor{currentfill}%
\pgfsetlinewidth{0.000000pt}%
\definecolor{currentstroke}{rgb}{0.000000,0.000000,0.000000}%
\pgfsetstrokecolor{currentstroke}%
\pgfsetdash{}{0pt}%
\pgfpathmoveto{\pgfqpoint{1.568409in}{1.078447in}}%
\pgfpathlineto{\pgfqpoint{1.571055in}{1.076522in}}%
\pgfpathlineto{\pgfqpoint{1.573704in}{1.074778in}}%
\pgfpathlineto{\pgfqpoint{1.576354in}{1.073219in}}%
\pgfpathlineto{\pgfqpoint{1.579006in}{1.071849in}}%
\pgfpathlineto{\pgfqpoint{1.590158in}{1.065357in}}%
\pgfpathlineto{\pgfqpoint{1.600931in}{1.058684in}}%
\pgfpathlineto{\pgfqpoint{1.611315in}{1.051835in}}%
\pgfpathlineto{\pgfqpoint{1.621299in}{1.044814in}}%
\pgfpathlineto{\pgfqpoint{1.618354in}{1.046370in}}%
\pgfpathlineto{\pgfqpoint{1.615412in}{1.048115in}}%
\pgfpathlineto{\pgfqpoint{1.612472in}{1.050046in}}%
\pgfpathlineto{\pgfqpoint{1.609534in}{1.052158in}}%
\pgfpathlineto{\pgfqpoint{1.599828in}{1.058984in}}%
\pgfpathlineto{\pgfqpoint{1.589731in}{1.065645in}}%
\pgfpathlineto{\pgfqpoint{1.579255in}{1.072134in}}%
\pgfpathlineto{\pgfqpoint{1.568409in}{1.078447in}}%
\pgfpathclose%
\pgfusepath{fill}%
\end{pgfscope}%
\begin{pgfscope}%
\pgfpathrectangle{\pgfqpoint{0.041670in}{0.041670in}}{\pgfqpoint{2.216660in}{2.216660in}}%
\pgfusepath{clip}%
\pgfsetbuttcap%
\pgfsetroundjoin%
\definecolor{currentfill}{rgb}{0.282327,0.094955,0.417331}%
\pgfsetfillcolor{currentfill}%
\pgfsetlinewidth{0.000000pt}%
\definecolor{currentstroke}{rgb}{0.000000,0.000000,0.000000}%
\pgfsetstrokecolor{currentstroke}%
\pgfsetdash{}{0pt}%
\pgfpathmoveto{\pgfqpoint{0.803514in}{1.094590in}}%
\pgfpathlineto{\pgfqpoint{0.800808in}{1.091322in}}%
\pgfpathlineto{\pgfqpoint{0.798101in}{1.088203in}}%
\pgfpathlineto{\pgfqpoint{0.795394in}{1.085238in}}%
\pgfpathlineto{\pgfqpoint{0.792686in}{1.082430in}}%
\pgfpathlineto{\pgfqpoint{0.803267in}{1.088543in}}%
\pgfpathlineto{\pgfqpoint{0.814195in}{1.094479in}}%
\pgfpathlineto{\pgfqpoint{0.825461in}{1.100232in}}%
\pgfpathlineto{\pgfqpoint{0.837053in}{1.105799in}}%
\pgfpathlineto{\pgfqpoint{0.839443in}{1.108435in}}%
\pgfpathlineto{\pgfqpoint{0.841832in}{1.111228in}}%
\pgfpathlineto{\pgfqpoint{0.844220in}{1.114174in}}%
\pgfpathlineto{\pgfqpoint{0.846609in}{1.117268in}}%
\pgfpathlineto{\pgfqpoint{0.835348in}{1.111866in}}%
\pgfpathlineto{\pgfqpoint{0.824404in}{1.106282in}}%
\pgfpathlineto{\pgfqpoint{0.813789in}{1.100522in}}%
\pgfpathlineto{\pgfqpoint{0.803514in}{1.094590in}}%
\pgfpathclose%
\pgfusepath{fill}%
\end{pgfscope}%
\begin{pgfscope}%
\pgfpathrectangle{\pgfqpoint{0.041670in}{0.041670in}}{\pgfqpoint{2.216660in}{2.216660in}}%
\pgfusepath{clip}%
\pgfsetbuttcap%
\pgfsetroundjoin%
\definecolor{currentfill}{rgb}{0.233603,0.313828,0.543914}%
\pgfsetfillcolor{currentfill}%
\pgfsetlinewidth{0.000000pt}%
\definecolor{currentstroke}{rgb}{0.000000,0.000000,0.000000}%
\pgfsetstrokecolor{currentstroke}%
\pgfsetdash{}{0pt}%
\pgfpathmoveto{\pgfqpoint{1.700545in}{1.268503in}}%
\pgfpathlineto{\pgfqpoint{1.703497in}{1.280620in}}%
\pgfpathlineto{\pgfqpoint{1.706462in}{1.293168in}}%
\pgfpathlineto{\pgfqpoint{1.709440in}{1.306157in}}%
\pgfpathlineto{\pgfqpoint{1.712432in}{1.319592in}}%
\pgfpathlineto{\pgfqpoint{1.727481in}{1.311234in}}%
\pgfpathlineto{\pgfqpoint{1.742040in}{1.302635in}}%
\pgfpathlineto{\pgfqpoint{1.756092in}{1.293803in}}%
\pgfpathlineto{\pgfqpoint{1.769624in}{1.284746in}}%
\pgfpathlineto{\pgfqpoint{1.766294in}{1.271439in}}%
\pgfpathlineto{\pgfqpoint{1.762980in}{1.258581in}}%
\pgfpathlineto{\pgfqpoint{1.759680in}{1.246164in}}%
\pgfpathlineto{\pgfqpoint{1.756394in}{1.234182in}}%
\pgfpathlineto{\pgfqpoint{1.743183in}{1.243102in}}%
\pgfpathlineto{\pgfqpoint{1.729462in}{1.251800in}}%
\pgfpathlineto{\pgfqpoint{1.715244in}{1.260270in}}%
\pgfpathlineto{\pgfqpoint{1.700545in}{1.268503in}}%
\pgfpathclose%
\pgfusepath{fill}%
\end{pgfscope}%
\begin{pgfscope}%
\pgfpathrectangle{\pgfqpoint{0.041670in}{0.041670in}}{\pgfqpoint{2.216660in}{2.216660in}}%
\pgfusepath{clip}%
\pgfsetbuttcap%
\pgfsetroundjoin%
\definecolor{currentfill}{rgb}{0.263663,0.237631,0.518762}%
\pgfsetfillcolor{currentfill}%
\pgfsetlinewidth{0.000000pt}%
\definecolor{currentstroke}{rgb}{0.000000,0.000000,0.000000}%
\pgfsetstrokecolor{currentstroke}%
\pgfsetdash{}{0pt}%
\pgfpathmoveto{\pgfqpoint{0.973381in}{1.215785in}}%
\pgfpathlineto{\pgfqpoint{0.971712in}{1.211083in}}%
\pgfpathlineto{\pgfqpoint{0.970044in}{1.206472in}}%
\pgfpathlineto{\pgfqpoint{0.968377in}{1.201956in}}%
\pgfpathlineto{\pgfqpoint{0.966711in}{1.197538in}}%
\pgfpathlineto{\pgfqpoint{0.979135in}{1.200798in}}%
\pgfpathlineto{\pgfqpoint{0.991742in}{1.203859in}}%
\pgfpathlineto{\pgfqpoint{1.004520in}{1.206719in}}%
\pgfpathlineto{\pgfqpoint{1.017458in}{1.209376in}}%
\pgfpathlineto{\pgfqpoint{1.018726in}{1.213694in}}%
\pgfpathlineto{\pgfqpoint{1.019994in}{1.218110in}}%
\pgfpathlineto{\pgfqpoint{1.021263in}{1.222621in}}%
\pgfpathlineto{\pgfqpoint{1.022532in}{1.227223in}}%
\pgfpathlineto{\pgfqpoint{1.010001in}{1.224656in}}%
\pgfpathlineto{\pgfqpoint{0.997624in}{1.221892in}}%
\pgfpathlineto{\pgfqpoint{0.985414in}{1.218935in}}%
\pgfpathlineto{\pgfqpoint{0.973381in}{1.215785in}}%
\pgfpathclose%
\pgfusepath{fill}%
\end{pgfscope}%
\begin{pgfscope}%
\pgfpathrectangle{\pgfqpoint{0.041670in}{0.041670in}}{\pgfqpoint{2.216660in}{2.216660in}}%
\pgfusepath{clip}%
\pgfsetbuttcap%
\pgfsetroundjoin%
\definecolor{currentfill}{rgb}{0.280255,0.165693,0.476498}%
\pgfsetfillcolor{currentfill}%
\pgfsetlinewidth{0.000000pt}%
\definecolor{currentstroke}{rgb}{0.000000,0.000000,0.000000}%
\pgfsetstrokecolor{currentstroke}%
\pgfsetdash{}{0pt}%
\pgfpathmoveto{\pgfqpoint{0.865712in}{1.146942in}}%
\pgfpathlineto{\pgfqpoint{0.863323in}{1.142789in}}%
\pgfpathlineto{\pgfqpoint{0.860935in}{1.138756in}}%
\pgfpathlineto{\pgfqpoint{0.858547in}{1.134846in}}%
\pgfpathlineto{\pgfqpoint{0.856159in}{1.131062in}}%
\pgfpathlineto{\pgfqpoint{0.867388in}{1.136119in}}%
\pgfpathlineto{\pgfqpoint{0.878905in}{1.140992in}}%
\pgfpathlineto{\pgfqpoint{0.890698in}{1.145677in}}%
\pgfpathlineto{\pgfqpoint{0.902756in}{1.150169in}}%
\pgfpathlineto{\pgfqpoint{0.904794in}{1.153802in}}%
\pgfpathlineto{\pgfqpoint{0.906833in}{1.157561in}}%
\pgfpathlineto{\pgfqpoint{0.908872in}{1.161443in}}%
\pgfpathlineto{\pgfqpoint{0.910911in}{1.165445in}}%
\pgfpathlineto{\pgfqpoint{0.899214in}{1.161094in}}%
\pgfpathlineto{\pgfqpoint{0.887774in}{1.156558in}}%
\pgfpathlineto{\pgfqpoint{0.876603in}{1.151839in}}%
\pgfpathlineto{\pgfqpoint{0.865712in}{1.146942in}}%
\pgfpathclose%
\pgfusepath{fill}%
\end{pgfscope}%
\begin{pgfscope}%
\pgfpathrectangle{\pgfqpoint{0.041670in}{0.041670in}}{\pgfqpoint{2.216660in}{2.216660in}}%
\pgfusepath{clip}%
\pgfsetbuttcap%
\pgfsetroundjoin%
\definecolor{currentfill}{rgb}{0.272594,0.025563,0.353093}%
\pgfsetfillcolor{currentfill}%
\pgfsetlinewidth{0.000000pt}%
\definecolor{currentstroke}{rgb}{0.000000,0.000000,0.000000}%
\pgfsetstrokecolor{currentstroke}%
\pgfsetdash{}{0pt}%
\pgfpathmoveto{\pgfqpoint{1.680975in}{1.060165in}}%
\pgfpathlineto{\pgfqpoint{1.684014in}{1.063609in}}%
\pgfpathlineto{\pgfqpoint{1.687060in}{1.067344in}}%
\pgfpathlineto{\pgfqpoint{1.690114in}{1.071374in}}%
\pgfpathlineto{\pgfqpoint{1.693174in}{1.075706in}}%
\pgfpathlineto{\pgfqpoint{1.704393in}{1.067386in}}%
\pgfpathlineto{\pgfqpoint{1.715128in}{1.058880in}}%
\pgfpathlineto{\pgfqpoint{1.725366in}{1.050195in}}%
\pgfpathlineto{\pgfqpoint{1.735098in}{1.041339in}}%
\pgfpathlineto{\pgfqpoint{1.731771in}{1.037192in}}%
\pgfpathlineto{\pgfqpoint{1.728452in}{1.033348in}}%
\pgfpathlineto{\pgfqpoint{1.725142in}{1.029800in}}%
\pgfpathlineto{\pgfqpoint{1.721839in}{1.026544in}}%
\pgfpathlineto{\pgfqpoint{1.712356in}{1.035208in}}%
\pgfpathlineto{\pgfqpoint{1.702377in}{1.043703in}}%
\pgfpathlineto{\pgfqpoint{1.691913in}{1.052025in}}%
\pgfpathlineto{\pgfqpoint{1.680975in}{1.060165in}}%
\pgfpathclose%
\pgfusepath{fill}%
\end{pgfscope}%
\begin{pgfscope}%
\pgfpathrectangle{\pgfqpoint{0.041670in}{0.041670in}}{\pgfqpoint{2.216660in}{2.216660in}}%
\pgfusepath{clip}%
\pgfsetbuttcap%
\pgfsetroundjoin%
\definecolor{currentfill}{rgb}{0.268510,0.009605,0.335427}%
\pgfsetfillcolor{currentfill}%
\pgfsetlinewidth{0.000000pt}%
\definecolor{currentstroke}{rgb}{0.000000,0.000000,0.000000}%
\pgfsetstrokecolor{currentstroke}%
\pgfsetdash{}{0pt}%
\pgfpathmoveto{\pgfqpoint{1.668883in}{1.049192in}}%
\pgfpathlineto{\pgfqpoint{1.671897in}{1.051524in}}%
\pgfpathlineto{\pgfqpoint{1.674917in}{1.054127in}}%
\pgfpathlineto{\pgfqpoint{1.677943in}{1.057006in}}%
\pgfpathlineto{\pgfqpoint{1.680975in}{1.060165in}}%
\pgfpathlineto{\pgfqpoint{1.691913in}{1.052025in}}%
\pgfpathlineto{\pgfqpoint{1.702377in}{1.043703in}}%
\pgfpathlineto{\pgfqpoint{1.712356in}{1.035208in}}%
\pgfpathlineto{\pgfqpoint{1.721839in}{1.026544in}}%
\pgfpathlineto{\pgfqpoint{1.718543in}{1.023575in}}%
\pgfpathlineto{\pgfqpoint{1.715255in}{1.020887in}}%
\pgfpathlineto{\pgfqpoint{1.711974in}{1.018476in}}%
\pgfpathlineto{\pgfqpoint{1.708700in}{1.016336in}}%
\pgfpathlineto{\pgfqpoint{1.699462in}{1.024801in}}%
\pgfpathlineto{\pgfqpoint{1.689739in}{1.033104in}}%
\pgfpathlineto{\pgfqpoint{1.679543in}{1.041236in}}%
\pgfpathlineto{\pgfqpoint{1.668883in}{1.049192in}}%
\pgfpathclose%
\pgfusepath{fill}%
\end{pgfscope}%
\begin{pgfscope}%
\pgfpathrectangle{\pgfqpoint{0.041670in}{0.041670in}}{\pgfqpoint{2.216660in}{2.216660in}}%
\pgfusepath{clip}%
\pgfsetbuttcap%
\pgfsetroundjoin%
\definecolor{currentfill}{rgb}{0.277941,0.056324,0.381191}%
\pgfsetfillcolor{currentfill}%
\pgfsetlinewidth{0.000000pt}%
\definecolor{currentstroke}{rgb}{0.000000,0.000000,0.000000}%
\pgfsetstrokecolor{currentstroke}%
\pgfsetdash{}{0pt}%
\pgfpathmoveto{\pgfqpoint{1.693174in}{1.075706in}}%
\pgfpathlineto{\pgfqpoint{1.696243in}{1.080344in}}%
\pgfpathlineto{\pgfqpoint{1.699319in}{1.085293in}}%
\pgfpathlineto{\pgfqpoint{1.702404in}{1.090559in}}%
\pgfpathlineto{\pgfqpoint{1.705497in}{1.096148in}}%
\pgfpathlineto{\pgfqpoint{1.717001in}{1.087653in}}%
\pgfpathlineto{\pgfqpoint{1.728009in}{1.078968in}}%
\pgfpathlineto{\pgfqpoint{1.738511in}{1.070099in}}%
\pgfpathlineto{\pgfqpoint{1.748494in}{1.061055in}}%
\pgfpathlineto{\pgfqpoint{1.745131in}{1.055645in}}%
\pgfpathlineto{\pgfqpoint{1.741778in}{1.050560in}}%
\pgfpathlineto{\pgfqpoint{1.738433in}{1.045793in}}%
\pgfpathlineto{\pgfqpoint{1.735098in}{1.041339in}}%
\pgfpathlineto{\pgfqpoint{1.725366in}{1.050195in}}%
\pgfpathlineto{\pgfqpoint{1.715128in}{1.058880in}}%
\pgfpathlineto{\pgfqpoint{1.704393in}{1.067386in}}%
\pgfpathlineto{\pgfqpoint{1.693174in}{1.075706in}}%
\pgfpathclose%
\pgfusepath{fill}%
\end{pgfscope}%
\begin{pgfscope}%
\pgfpathrectangle{\pgfqpoint{0.041670in}{0.041670in}}{\pgfqpoint{2.216660in}{2.216660in}}%
\pgfusepath{clip}%
\pgfsetbuttcap%
\pgfsetroundjoin%
\definecolor{currentfill}{rgb}{0.248629,0.278775,0.534556}%
\pgfsetfillcolor{currentfill}%
\pgfsetlinewidth{0.000000pt}%
\definecolor{currentstroke}{rgb}{0.000000,0.000000,0.000000}%
\pgfsetstrokecolor{currentstroke}%
\pgfsetdash{}{0pt}%
\pgfpathmoveto{\pgfqpoint{1.283920in}{1.254273in}}%
\pgfpathlineto{\pgfqpoint{1.284788in}{1.249410in}}%
\pgfpathlineto{\pgfqpoint{1.285655in}{1.244625in}}%
\pgfpathlineto{\pgfqpoint{1.286522in}{1.239921in}}%
\pgfpathlineto{\pgfqpoint{1.287388in}{1.235302in}}%
\pgfpathlineto{\pgfqpoint{1.300411in}{1.233517in}}%
\pgfpathlineto{\pgfqpoint{1.313325in}{1.231529in}}%
\pgfpathlineto{\pgfqpoint{1.326117in}{1.229338in}}%
\pgfpathlineto{\pgfqpoint{1.324938in}{1.234010in}}%
\pgfpathlineto{\pgfqpoint{1.323758in}{1.238766in}}%
\pgfpathlineto{\pgfqpoint{1.322577in}{1.243603in}}%
\pgfpathlineto{\pgfqpoint{1.321396in}{1.248518in}}%
\pgfpathlineto{\pgfqpoint{1.309018in}{1.250631in}}%
\pgfpathlineto{\pgfqpoint{1.296522in}{1.252550in}}%
\pgfpathlineto{\pgfqpoint{1.283920in}{1.254273in}}%
\pgfpathclose%
\pgfusepath{fill}%
\end{pgfscope}%
\begin{pgfscope}%
\pgfpathrectangle{\pgfqpoint{0.041670in}{0.041670in}}{\pgfqpoint{2.216660in}{2.216660in}}%
\pgfusepath{clip}%
\pgfsetbuttcap%
\pgfsetroundjoin%
\definecolor{currentfill}{rgb}{0.248629,0.278775,0.534556}%
\pgfsetfillcolor{currentfill}%
\pgfsetlinewidth{0.000000pt}%
\definecolor{currentstroke}{rgb}{0.000000,0.000000,0.000000}%
\pgfsetstrokecolor{currentstroke}%
\pgfsetdash{}{0pt}%
\pgfpathmoveto{\pgfqpoint{1.027619in}{1.246476in}}%
\pgfpathlineto{\pgfqpoint{1.026346in}{1.241543in}}%
\pgfpathlineto{\pgfqpoint{1.025074in}{1.236687in}}%
\pgfpathlineto{\pgfqpoint{1.023803in}{1.231913in}}%
\pgfpathlineto{\pgfqpoint{1.022532in}{1.227223in}}%
\pgfpathlineto{\pgfqpoint{1.035207in}{1.229592in}}%
\pgfpathlineto{\pgfqpoint{1.048014in}{1.231760in}}%
\pgfpathlineto{\pgfqpoint{1.060941in}{1.233726in}}%
\pgfpathlineto{\pgfqpoint{1.073975in}{1.235488in}}%
\pgfpathlineto{\pgfqpoint{1.074830in}{1.240105in}}%
\pgfpathlineto{\pgfqpoint{1.075684in}{1.244808in}}%
\pgfpathlineto{\pgfqpoint{1.076540in}{1.249591in}}%
\pgfpathlineto{\pgfqpoint{1.077396in}{1.254452in}}%
\pgfpathlineto{\pgfqpoint{1.064783in}{1.252752in}}%
\pgfpathlineto{\pgfqpoint{1.052275in}{1.250854in}}%
\pgfpathlineto{\pgfqpoint{1.039883in}{1.248762in}}%
\pgfpathlineto{\pgfqpoint{1.027619in}{1.246476in}}%
\pgfpathclose%
\pgfusepath{fill}%
\end{pgfscope}%
\begin{pgfscope}%
\pgfpathrectangle{\pgfqpoint{0.041670in}{0.041670in}}{\pgfqpoint{2.216660in}{2.216660in}}%
\pgfusepath{clip}%
\pgfsetbuttcap%
\pgfsetroundjoin%
\definecolor{currentfill}{rgb}{0.274952,0.037752,0.364543}%
\pgfsetfillcolor{currentfill}%
\pgfsetlinewidth{0.000000pt}%
\definecolor{currentstroke}{rgb}{0.000000,0.000000,0.000000}%
\pgfsetstrokecolor{currentstroke}%
\pgfsetdash{}{0pt}%
\pgfpathmoveto{\pgfqpoint{0.742084in}{1.045956in}}%
\pgfpathlineto{\pgfqpoint{0.739086in}{1.043800in}}%
\pgfpathlineto{\pgfqpoint{0.736087in}{1.041825in}}%
\pgfpathlineto{\pgfqpoint{0.733085in}{1.040036in}}%
\pgfpathlineto{\pgfqpoint{0.730081in}{1.038436in}}%
\pgfpathlineto{\pgfqpoint{0.739700in}{1.045603in}}%
\pgfpathlineto{\pgfqpoint{0.749729in}{1.052604in}}%
\pgfpathlineto{\pgfqpoint{0.760156in}{1.059435in}}%
\pgfpathlineto{\pgfqpoint{0.770972in}{1.066088in}}%
\pgfpathlineto{\pgfqpoint{0.773693in}{1.067497in}}%
\pgfpathlineto{\pgfqpoint{0.776411in}{1.069096in}}%
\pgfpathlineto{\pgfqpoint{0.779128in}{1.070880in}}%
\pgfpathlineto{\pgfqpoint{0.781842in}{1.072844in}}%
\pgfpathlineto{\pgfqpoint{0.771324in}{1.066374in}}%
\pgfpathlineto{\pgfqpoint{0.761184in}{1.059732in}}%
\pgfpathlineto{\pgfqpoint{0.751434in}{1.052924in}}%
\pgfpathlineto{\pgfqpoint{0.742084in}{1.045956in}}%
\pgfpathclose%
\pgfusepath{fill}%
\end{pgfscope}%
\begin{pgfscope}%
\pgfpathrectangle{\pgfqpoint{0.041670in}{0.041670in}}{\pgfqpoint{2.216660in}{2.216660in}}%
\pgfusepath{clip}%
\pgfsetbuttcap%
\pgfsetroundjoin%
\definecolor{currentfill}{rgb}{0.267004,0.004874,0.329415}%
\pgfsetfillcolor{currentfill}%
\pgfsetlinewidth{0.000000pt}%
\definecolor{currentstroke}{rgb}{0.000000,0.000000,0.000000}%
\pgfsetstrokecolor{currentstroke}%
\pgfsetdash{}{0pt}%
\pgfpathmoveto{\pgfqpoint{1.656883in}{1.042470in}}%
\pgfpathlineto{\pgfqpoint{1.659875in}{1.043769in}}%
\pgfpathlineto{\pgfqpoint{1.662873in}{1.045319in}}%
\pgfpathlineto{\pgfqpoint{1.665875in}{1.047125in}}%
\pgfpathlineto{\pgfqpoint{1.668883in}{1.049192in}}%
\pgfpathlineto{\pgfqpoint{1.679543in}{1.041236in}}%
\pgfpathlineto{\pgfqpoint{1.689739in}{1.033104in}}%
\pgfpathlineto{\pgfqpoint{1.699462in}{1.024801in}}%
\pgfpathlineto{\pgfqpoint{1.708700in}{1.016336in}}%
\pgfpathlineto{\pgfqpoint{1.705432in}{1.014462in}}%
\pgfpathlineto{\pgfqpoint{1.702170in}{1.012851in}}%
\pgfpathlineto{\pgfqpoint{1.698914in}{1.011496in}}%
\pgfpathlineto{\pgfqpoint{1.695664in}{1.010393in}}%
\pgfpathlineto{\pgfqpoint{1.686669in}{1.018657in}}%
\pgfpathlineto{\pgfqpoint{1.677200in}{1.026763in}}%
\pgfpathlineto{\pgfqpoint{1.667268in}{1.034702in}}%
\pgfpathlineto{\pgfqpoint{1.656883in}{1.042470in}}%
\pgfpathclose%
\pgfusepath{fill}%
\end{pgfscope}%
\begin{pgfscope}%
\pgfpathrectangle{\pgfqpoint{0.041670in}{0.041670in}}{\pgfqpoint{2.216660in}{2.216660in}}%
\pgfusepath{clip}%
\pgfsetbuttcap%
\pgfsetroundjoin%
\definecolor{currentfill}{rgb}{0.283072,0.130895,0.449241}%
\pgfsetfillcolor{currentfill}%
\pgfsetlinewidth{0.000000pt}%
\definecolor{currentstroke}{rgb}{0.000000,0.000000,0.000000}%
\pgfsetstrokecolor{currentstroke}%
\pgfsetdash{}{0pt}%
\pgfpathmoveto{\pgfqpoint{1.493783in}{1.135567in}}%
\pgfpathlineto{\pgfqpoint{1.496096in}{1.131949in}}%
\pgfpathlineto{\pgfqpoint{1.498408in}{1.128465in}}%
\pgfpathlineto{\pgfqpoint{1.500721in}{1.125119in}}%
\pgfpathlineto{\pgfqpoint{1.503034in}{1.121915in}}%
\pgfpathlineto{\pgfqpoint{1.514568in}{1.116677in}}%
\pgfpathlineto{\pgfqpoint{1.525794in}{1.111254in}}%
\pgfpathlineto{\pgfqpoint{1.536701in}{1.105651in}}%
\pgfpathlineto{\pgfqpoint{1.547279in}{1.099871in}}%
\pgfpathlineto{\pgfqpoint{1.544641in}{1.103244in}}%
\pgfpathlineto{\pgfqpoint{1.542004in}{1.106759in}}%
\pgfpathlineto{\pgfqpoint{1.539367in}{1.110411in}}%
\pgfpathlineto{\pgfqpoint{1.536729in}{1.114198in}}%
\pgfpathlineto{\pgfqpoint{1.526464in}{1.119800in}}%
\pgfpathlineto{\pgfqpoint{1.515877in}{1.125232in}}%
\pgfpathlineto{\pgfqpoint{1.504980in}{1.130489in}}%
\pgfpathlineto{\pgfqpoint{1.493783in}{1.135567in}}%
\pgfpathclose%
\pgfusepath{fill}%
\end{pgfscope}%
\begin{pgfscope}%
\pgfpathrectangle{\pgfqpoint{0.041670in}{0.041670in}}{\pgfqpoint{2.216660in}{2.216660in}}%
\pgfusepath{clip}%
\pgfsetbuttcap%
\pgfsetroundjoin%
\definecolor{currentfill}{rgb}{0.282327,0.094955,0.417331}%
\pgfsetfillcolor{currentfill}%
\pgfsetlinewidth{0.000000pt}%
\definecolor{currentstroke}{rgb}{0.000000,0.000000,0.000000}%
\pgfsetstrokecolor{currentstroke}%
\pgfsetdash{}{0pt}%
\pgfpathmoveto{\pgfqpoint{1.705497in}{1.096148in}}%
\pgfpathlineto{\pgfqpoint{1.708599in}{1.102065in}}%
\pgfpathlineto{\pgfqpoint{1.711711in}{1.108316in}}%
\pgfpathlineto{\pgfqpoint{1.714831in}{1.114906in}}%
\pgfpathlineto{\pgfqpoint{1.717962in}{1.121841in}}%
\pgfpathlineto{\pgfqpoint{1.729753in}{1.113176in}}%
\pgfpathlineto{\pgfqpoint{1.741040in}{1.104316in}}%
\pgfpathlineto{\pgfqpoint{1.751809in}{1.095269in}}%
\pgfpathlineto{\pgfqpoint{1.762047in}{1.086042in}}%
\pgfpathlineto{\pgfqpoint{1.758643in}{1.079281in}}%
\pgfpathlineto{\pgfqpoint{1.755250in}{1.072867in}}%
\pgfpathlineto{\pgfqpoint{1.751867in}{1.066793in}}%
\pgfpathlineto{\pgfqpoint{1.748494in}{1.061055in}}%
\pgfpathlineto{\pgfqpoint{1.738511in}{1.070099in}}%
\pgfpathlineto{\pgfqpoint{1.728009in}{1.078968in}}%
\pgfpathlineto{\pgfqpoint{1.717001in}{1.087653in}}%
\pgfpathlineto{\pgfqpoint{1.705497in}{1.096148in}}%
\pgfpathclose%
\pgfusepath{fill}%
\end{pgfscope}%
\begin{pgfscope}%
\pgfpathrectangle{\pgfqpoint{0.041670in}{0.041670in}}{\pgfqpoint{2.216660in}{2.216660in}}%
\pgfusepath{clip}%
\pgfsetbuttcap%
\pgfsetroundjoin%
\definecolor{currentfill}{rgb}{0.233603,0.313828,0.543914}%
\pgfsetfillcolor{currentfill}%
\pgfsetlinewidth{0.000000pt}%
\definecolor{currentstroke}{rgb}{0.000000,0.000000,0.000000}%
\pgfsetstrokecolor{currentstroke}%
\pgfsetdash{}{0pt}%
\pgfpathmoveto{\pgfqpoint{0.592212in}{1.226072in}}%
\pgfpathlineto{\pgfqpoint{0.588858in}{1.238022in}}%
\pgfpathlineto{\pgfqpoint{0.585490in}{1.250407in}}%
\pgfpathlineto{\pgfqpoint{0.582106in}{1.263235in}}%
\pgfpathlineto{\pgfqpoint{0.578707in}{1.276511in}}%
\pgfpathlineto{\pgfqpoint{0.591763in}{1.285763in}}%
\pgfpathlineto{\pgfqpoint{0.605353in}{1.294796in}}%
\pgfpathlineto{\pgfqpoint{0.619463in}{1.303602in}}%
\pgfpathlineto{\pgfqpoint{0.634076in}{1.312174in}}%
\pgfpathlineto{\pgfqpoint{0.637147in}{1.298766in}}%
\pgfpathlineto{\pgfqpoint{0.640203in}{1.285805in}}%
\pgfpathlineto{\pgfqpoint{0.643246in}{1.273284in}}%
\pgfpathlineto{\pgfqpoint{0.646275in}{1.261196in}}%
\pgfpathlineto{\pgfqpoint{0.632004in}{1.252753in}}%
\pgfpathlineto{\pgfqpoint{0.618226in}{1.244080in}}%
\pgfpathlineto{\pgfqpoint{0.604958in}{1.235184in}}%
\pgfpathlineto{\pgfqpoint{0.592212in}{1.226072in}}%
\pgfpathclose%
\pgfusepath{fill}%
\end{pgfscope}%
\begin{pgfscope}%
\pgfpathrectangle{\pgfqpoint{0.041670in}{0.041670in}}{\pgfqpoint{2.216660in}{2.216660in}}%
\pgfusepath{clip}%
\pgfsetbuttcap%
\pgfsetroundjoin%
\definecolor{currentfill}{rgb}{0.279566,0.067836,0.391917}%
\pgfsetfillcolor{currentfill}%
\pgfsetlinewidth{0.000000pt}%
\definecolor{currentstroke}{rgb}{0.000000,0.000000,0.000000}%
\pgfsetstrokecolor{currentstroke}%
\pgfsetdash{}{0pt}%
\pgfpathmoveto{\pgfqpoint{1.557836in}{1.087872in}}%
\pgfpathlineto{\pgfqpoint{1.560477in}{1.085265in}}%
\pgfpathlineto{\pgfqpoint{1.563120in}{1.082822in}}%
\pgfpathlineto{\pgfqpoint{1.565763in}{1.080548in}}%
\pgfpathlineto{\pgfqpoint{1.568409in}{1.078447in}}%
\pgfpathlineto{\pgfqpoint{1.579255in}{1.072134in}}%
\pgfpathlineto{\pgfqpoint{1.589731in}{1.065645in}}%
\pgfpathlineto{\pgfqpoint{1.599828in}{1.058984in}}%
\pgfpathlineto{\pgfqpoint{1.609534in}{1.052158in}}%
\pgfpathlineto{\pgfqpoint{1.606598in}{1.054447in}}%
\pgfpathlineto{\pgfqpoint{1.603664in}{1.056908in}}%
\pgfpathlineto{\pgfqpoint{1.600731in}{1.059539in}}%
\pgfpathlineto{\pgfqpoint{1.597800in}{1.062336in}}%
\pgfpathlineto{\pgfqpoint{1.588369in}{1.068966in}}%
\pgfpathlineto{\pgfqpoint{1.578558in}{1.075436in}}%
\pgfpathlineto{\pgfqpoint{1.568377in}{1.081740in}}%
\pgfpathlineto{\pgfqpoint{1.557836in}{1.087872in}}%
\pgfpathclose%
\pgfusepath{fill}%
\end{pgfscope}%
\begin{pgfscope}%
\pgfpathrectangle{\pgfqpoint{0.041670in}{0.041670in}}{\pgfqpoint{2.216660in}{2.216660in}}%
\pgfusepath{clip}%
\pgfsetbuttcap%
\pgfsetroundjoin%
\definecolor{currentfill}{rgb}{0.267004,0.004874,0.329415}%
\pgfsetfillcolor{currentfill}%
\pgfsetlinewidth{0.000000pt}%
\definecolor{currentstroke}{rgb}{0.000000,0.000000,0.000000}%
\pgfsetstrokecolor{currentstroke}%
\pgfsetdash{}{0pt}%
\pgfpathmoveto{\pgfqpoint{1.644961in}{1.039693in}}%
\pgfpathlineto{\pgfqpoint{1.647935in}{1.040034in}}%
\pgfpathlineto{\pgfqpoint{1.650913in}{1.040607in}}%
\pgfpathlineto{\pgfqpoint{1.653896in}{1.041417in}}%
\pgfpathlineto{\pgfqpoint{1.656883in}{1.042470in}}%
\pgfpathlineto{\pgfqpoint{1.667268in}{1.034702in}}%
\pgfpathlineto{\pgfqpoint{1.677200in}{1.026763in}}%
\pgfpathlineto{\pgfqpoint{1.686669in}{1.018657in}}%
\pgfpathlineto{\pgfqpoint{1.695664in}{1.010393in}}%
\pgfpathlineto{\pgfqpoint{1.692419in}{1.009538in}}%
\pgfpathlineto{\pgfqpoint{1.689179in}{1.008925in}}%
\pgfpathlineto{\pgfqpoint{1.685945in}{1.008550in}}%
\pgfpathlineto{\pgfqpoint{1.682715in}{1.008409in}}%
\pgfpathlineto{\pgfqpoint{1.673960in}{1.016468in}}%
\pgfpathlineto{\pgfqpoint{1.664743in}{1.024373in}}%
\pgfpathlineto{\pgfqpoint{1.655073in}{1.032117in}}%
\pgfpathlineto{\pgfqpoint{1.644961in}{1.039693in}}%
\pgfpathclose%
\pgfusepath{fill}%
\end{pgfscope}%
\begin{pgfscope}%
\pgfpathrectangle{\pgfqpoint{0.041670in}{0.041670in}}{\pgfqpoint{2.216660in}{2.216660in}}%
\pgfusepath{clip}%
\pgfsetbuttcap%
\pgfsetroundjoin%
\definecolor{currentfill}{rgb}{0.263663,0.237631,0.518762}%
\pgfsetfillcolor{currentfill}%
\pgfsetlinewidth{0.000000pt}%
\definecolor{currentstroke}{rgb}{0.000000,0.000000,0.000000}%
\pgfsetstrokecolor{currentstroke}%
\pgfsetdash{}{0pt}%
\pgfpathmoveto{\pgfqpoint{1.375842in}{1.218594in}}%
\pgfpathlineto{\pgfqpoint{1.377424in}{1.213917in}}%
\pgfpathlineto{\pgfqpoint{1.379005in}{1.209330in}}%
\pgfpathlineto{\pgfqpoint{1.380585in}{1.204839in}}%
\pgfpathlineto{\pgfqpoint{1.382164in}{1.200445in}}%
\pgfpathlineto{\pgfqpoint{1.394567in}{1.197164in}}%
\pgfpathlineto{\pgfqpoint{1.406774in}{1.193687in}}%
\pgfpathlineto{\pgfqpoint{1.418774in}{1.190018in}}%
\pgfpathlineto{\pgfqpoint{1.430555in}{1.186159in}}%
\pgfpathlineto{\pgfqpoint{1.428593in}{1.190673in}}%
\pgfpathlineto{\pgfqpoint{1.426631in}{1.195285in}}%
\pgfpathlineto{\pgfqpoint{1.424668in}{1.199993in}}%
\pgfpathlineto{\pgfqpoint{1.422703in}{1.204791in}}%
\pgfpathlineto{\pgfqpoint{1.411296in}{1.208519in}}%
\pgfpathlineto{\pgfqpoint{1.399675in}{1.212064in}}%
\pgfpathlineto{\pgfqpoint{1.387854in}{1.215424in}}%
\pgfpathlineto{\pgfqpoint{1.375842in}{1.218594in}}%
\pgfpathclose%
\pgfusepath{fill}%
\end{pgfscope}%
\begin{pgfscope}%
\pgfpathrectangle{\pgfqpoint{0.041670in}{0.041670in}}{\pgfqpoint{2.216660in}{2.216660in}}%
\pgfusepath{clip}%
\pgfsetbuttcap%
\pgfsetroundjoin%
\definecolor{currentfill}{rgb}{0.272594,0.025563,0.353093}%
\pgfsetfillcolor{currentfill}%
\pgfsetlinewidth{0.000000pt}%
\definecolor{currentstroke}{rgb}{0.000000,0.000000,0.000000}%
\pgfsetstrokecolor{currentstroke}%
\pgfsetdash{}{0pt}%
\pgfpathmoveto{\pgfqpoint{0.630067in}{1.018709in}}%
\pgfpathlineto{\pgfqpoint{0.626711in}{1.021921in}}%
\pgfpathlineto{\pgfqpoint{0.623348in}{1.025425in}}%
\pgfpathlineto{\pgfqpoint{0.619977in}{1.029226in}}%
\pgfpathlineto{\pgfqpoint{0.616597in}{1.033329in}}%
\pgfpathlineto{\pgfqpoint{0.625868in}{1.042331in}}%
\pgfpathlineto{\pgfqpoint{0.635656in}{1.051169in}}%
\pgfpathlineto{\pgfqpoint{0.645951in}{1.059834in}}%
\pgfpathlineto{\pgfqpoint{0.656740in}{1.068320in}}%
\pgfpathlineto{\pgfqpoint{0.659863in}{1.064028in}}%
\pgfpathlineto{\pgfqpoint{0.662979in}{1.060037in}}%
\pgfpathlineto{\pgfqpoint{0.666088in}{1.056342in}}%
\pgfpathlineto{\pgfqpoint{0.669189in}{1.052938in}}%
\pgfpathlineto{\pgfqpoint{0.658672in}{1.044637in}}%
\pgfpathlineto{\pgfqpoint{0.648638in}{1.036160in}}%
\pgfpathlineto{\pgfqpoint{0.639100in}{1.027515in}}%
\pgfpathlineto{\pgfqpoint{0.630067in}{1.018709in}}%
\pgfpathclose%
\pgfusepath{fill}%
\end{pgfscope}%
\begin{pgfscope}%
\pgfpathrectangle{\pgfqpoint{0.041670in}{0.041670in}}{\pgfqpoint{2.216660in}{2.216660in}}%
\pgfusepath{clip}%
\pgfsetbuttcap%
\pgfsetroundjoin%
\definecolor{currentfill}{rgb}{0.283072,0.130895,0.449241}%
\pgfsetfillcolor{currentfill}%
\pgfsetlinewidth{0.000000pt}%
\definecolor{currentstroke}{rgb}{0.000000,0.000000,0.000000}%
\pgfsetstrokecolor{currentstroke}%
\pgfsetdash{}{0pt}%
\pgfpathmoveto{\pgfqpoint{0.814333in}{1.109079in}}%
\pgfpathlineto{\pgfqpoint{0.811628in}{1.105252in}}%
\pgfpathlineto{\pgfqpoint{0.808924in}{1.101559in}}%
\pgfpathlineto{\pgfqpoint{0.806219in}{1.098003in}}%
\pgfpathlineto{\pgfqpoint{0.803514in}{1.094590in}}%
\pgfpathlineto{\pgfqpoint{0.813789in}{1.100522in}}%
\pgfpathlineto{\pgfqpoint{0.824404in}{1.106282in}}%
\pgfpathlineto{\pgfqpoint{0.835348in}{1.111866in}}%
\pgfpathlineto{\pgfqpoint{0.846609in}{1.117268in}}%
\pgfpathlineto{\pgfqpoint{0.848997in}{1.120508in}}%
\pgfpathlineto{\pgfqpoint{0.851384in}{1.123890in}}%
\pgfpathlineto{\pgfqpoint{0.853772in}{1.127409in}}%
\pgfpathlineto{\pgfqpoint{0.856159in}{1.131062in}}%
\pgfpathlineto{\pgfqpoint{0.845228in}{1.125825in}}%
\pgfpathlineto{\pgfqpoint{0.834607in}{1.120413in}}%
\pgfpathlineto{\pgfqpoint{0.824305in}{1.114829in}}%
\pgfpathlineto{\pgfqpoint{0.814333in}{1.109079in}}%
\pgfpathclose%
\pgfusepath{fill}%
\end{pgfscope}%
\begin{pgfscope}%
\pgfpathrectangle{\pgfqpoint{0.041670in}{0.041670in}}{\pgfqpoint{2.216660in}{2.216660in}}%
\pgfusepath{clip}%
\pgfsetbuttcap%
\pgfsetroundjoin%
\definecolor{currentfill}{rgb}{0.268510,0.009605,0.335427}%
\pgfsetfillcolor{currentfill}%
\pgfsetlinewidth{0.000000pt}%
\definecolor{currentstroke}{rgb}{0.000000,0.000000,0.000000}%
\pgfsetstrokecolor{currentstroke}%
\pgfsetdash{}{0pt}%
\pgfpathmoveto{\pgfqpoint{0.643414in}{1.008680in}}%
\pgfpathlineto{\pgfqpoint{0.640088in}{1.010775in}}%
\pgfpathlineto{\pgfqpoint{0.636755in}{1.013141in}}%
\pgfpathlineto{\pgfqpoint{0.633414in}{1.015784in}}%
\pgfpathlineto{\pgfqpoint{0.630067in}{1.018709in}}%
\pgfpathlineto{\pgfqpoint{0.639100in}{1.027515in}}%
\pgfpathlineto{\pgfqpoint{0.648638in}{1.036160in}}%
\pgfpathlineto{\pgfqpoint{0.658672in}{1.044637in}}%
\pgfpathlineto{\pgfqpoint{0.669189in}{1.052938in}}%
\pgfpathlineto{\pgfqpoint{0.672284in}{1.049820in}}%
\pgfpathlineto{\pgfqpoint{0.675372in}{1.046982in}}%
\pgfpathlineto{\pgfqpoint{0.678453in}{1.044421in}}%
\pgfpathlineto{\pgfqpoint{0.681529in}{1.042129in}}%
\pgfpathlineto{\pgfqpoint{0.671280in}{1.034016in}}%
\pgfpathlineto{\pgfqpoint{0.661504in}{1.025732in}}%
\pgfpathlineto{\pgfqpoint{0.652212in}{1.017284in}}%
\pgfpathlineto{\pgfqpoint{0.643414in}{1.008680in}}%
\pgfpathclose%
\pgfusepath{fill}%
\end{pgfscope}%
\begin{pgfscope}%
\pgfpathrectangle{\pgfqpoint{0.041670in}{0.041670in}}{\pgfqpoint{2.216660in}{2.216660in}}%
\pgfusepath{clip}%
\pgfsetbuttcap%
\pgfsetroundjoin%
\definecolor{currentfill}{rgb}{0.282884,0.135920,0.453427}%
\pgfsetfillcolor{currentfill}%
\pgfsetlinewidth{0.000000pt}%
\definecolor{currentstroke}{rgb}{0.000000,0.000000,0.000000}%
\pgfsetstrokecolor{currentstroke}%
\pgfsetdash{}{0pt}%
\pgfpathmoveto{\pgfqpoint{1.717962in}{1.121841in}}%
\pgfpathlineto{\pgfqpoint{1.721102in}{1.129127in}}%
\pgfpathlineto{\pgfqpoint{1.724252in}{1.136769in}}%
\pgfpathlineto{\pgfqpoint{1.727413in}{1.144775in}}%
\pgfpathlineto{\pgfqpoint{1.730585in}{1.153149in}}%
\pgfpathlineto{\pgfqpoint{1.742671in}{1.144321in}}%
\pgfpathlineto{\pgfqpoint{1.754239in}{1.135293in}}%
\pgfpathlineto{\pgfqpoint{1.765279in}{1.126074in}}%
\pgfpathlineto{\pgfqpoint{1.775777in}{1.116671in}}%
\pgfpathlineto{\pgfqpoint{1.772327in}{1.108464in}}%
\pgfpathlineto{\pgfqpoint{1.768889in}{1.100628in}}%
\pgfpathlineto{\pgfqpoint{1.765462in}{1.093156in}}%
\pgfpathlineto{\pgfqpoint{1.762047in}{1.086042in}}%
\pgfpathlineto{\pgfqpoint{1.751809in}{1.095269in}}%
\pgfpathlineto{\pgfqpoint{1.741040in}{1.104316in}}%
\pgfpathlineto{\pgfqpoint{1.729753in}{1.113176in}}%
\pgfpathlineto{\pgfqpoint{1.717962in}{1.121841in}}%
\pgfpathclose%
\pgfusepath{fill}%
\end{pgfscope}%
\begin{pgfscope}%
\pgfpathrectangle{\pgfqpoint{0.041670in}{0.041670in}}{\pgfqpoint{2.216660in}{2.216660in}}%
\pgfusepath{clip}%
\pgfsetbuttcap%
\pgfsetroundjoin%
\definecolor{currentfill}{rgb}{0.231674,0.318106,0.544834}%
\pgfsetfillcolor{currentfill}%
\pgfsetlinewidth{0.000000pt}%
\definecolor{currentstroke}{rgb}{0.000000,0.000000,0.000000}%
\pgfsetstrokecolor{currentstroke}%
\pgfsetdash{}{0pt}%
\pgfpathmoveto{\pgfqpoint{1.130380in}{1.279245in}}%
\pgfpathlineto{\pgfqpoint{1.129951in}{1.274149in}}%
\pgfpathlineto{\pgfqpoint{1.129522in}{1.269117in}}%
\pgfpathlineto{\pgfqpoint{1.129093in}{1.264154in}}%
\pgfpathlineto{\pgfqpoint{1.128665in}{1.259261in}}%
\pgfpathlineto{\pgfqpoint{1.141628in}{1.259961in}}%
\pgfpathlineto{\pgfqpoint{1.154626in}{1.260457in}}%
\pgfpathlineto{\pgfqpoint{1.167647in}{1.260751in}}%
\pgfpathlineto{\pgfqpoint{1.180679in}{1.260841in}}%
\pgfpathlineto{\pgfqpoint{1.180673in}{1.265719in}}%
\pgfpathlineto{\pgfqpoint{1.180667in}{1.270668in}}%
\pgfpathlineto{\pgfqpoint{1.180661in}{1.275686in}}%
\pgfpathlineto{\pgfqpoint{1.180655in}{1.280768in}}%
\pgfpathlineto{\pgfqpoint{1.168058in}{1.280681in}}%
\pgfpathlineto{\pgfqpoint{1.155473in}{1.280398in}}%
\pgfpathlineto{\pgfqpoint{1.142909in}{1.279919in}}%
\pgfpathlineto{\pgfqpoint{1.130380in}{1.279245in}}%
\pgfpathclose%
\pgfusepath{fill}%
\end{pgfscope}%
\begin{pgfscope}%
\pgfpathrectangle{\pgfqpoint{0.041670in}{0.041670in}}{\pgfqpoint{2.216660in}{2.216660in}}%
\pgfusepath{clip}%
\pgfsetbuttcap%
\pgfsetroundjoin%
\definecolor{currentfill}{rgb}{0.231674,0.318106,0.544834}%
\pgfsetfillcolor{currentfill}%
\pgfsetlinewidth{0.000000pt}%
\definecolor{currentstroke}{rgb}{0.000000,0.000000,0.000000}%
\pgfsetstrokecolor{currentstroke}%
\pgfsetdash{}{0pt}%
\pgfpathmoveto{\pgfqpoint{1.180655in}{1.280768in}}%
\pgfpathlineto{\pgfqpoint{1.180661in}{1.275686in}}%
\pgfpathlineto{\pgfqpoint{1.180667in}{1.270668in}}%
\pgfpathlineto{\pgfqpoint{1.180673in}{1.265719in}}%
\pgfpathlineto{\pgfqpoint{1.180679in}{1.260841in}}%
\pgfpathlineto{\pgfqpoint{1.193710in}{1.260728in}}%
\pgfpathlineto{\pgfqpoint{1.206729in}{1.260412in}}%
\pgfpathlineto{\pgfqpoint{1.219724in}{1.259893in}}%
\pgfpathlineto{\pgfqpoint{1.232682in}{1.259171in}}%
\pgfpathlineto{\pgfqpoint{1.232242in}{1.264064in}}%
\pgfpathlineto{\pgfqpoint{1.231802in}{1.269028in}}%
\pgfpathlineto{\pgfqpoint{1.231361in}{1.274061in}}%
\pgfpathlineto{\pgfqpoint{1.230920in}{1.279158in}}%
\pgfpathlineto{\pgfqpoint{1.218394in}{1.279854in}}%
\pgfpathlineto{\pgfqpoint{1.205834in}{1.280354in}}%
\pgfpathlineto{\pgfqpoint{1.193251in}{1.280659in}}%
\pgfpathlineto{\pgfqpoint{1.180655in}{1.280768in}}%
\pgfpathclose%
\pgfusepath{fill}%
\end{pgfscope}%
\begin{pgfscope}%
\pgfpathrectangle{\pgfqpoint{0.041670in}{0.041670in}}{\pgfqpoint{2.216660in}{2.216660in}}%
\pgfusepath{clip}%
\pgfsetbuttcap%
\pgfsetroundjoin%
\definecolor{currentfill}{rgb}{0.277941,0.056324,0.381191}%
\pgfsetfillcolor{currentfill}%
\pgfsetlinewidth{0.000000pt}%
\definecolor{currentstroke}{rgb}{0.000000,0.000000,0.000000}%
\pgfsetstrokecolor{currentstroke}%
\pgfsetdash{}{0pt}%
\pgfpathmoveto{\pgfqpoint{0.616597in}{1.033329in}}%
\pgfpathlineto{\pgfqpoint{0.613208in}{1.037740in}}%
\pgfpathlineto{\pgfqpoint{0.609810in}{1.042464in}}%
\pgfpathlineto{\pgfqpoint{0.606403in}{1.047507in}}%
\pgfpathlineto{\pgfqpoint{0.602986in}{1.052874in}}%
\pgfpathlineto{\pgfqpoint{0.612499in}{1.062068in}}%
\pgfpathlineto{\pgfqpoint{0.622540in}{1.071093in}}%
\pgfpathlineto{\pgfqpoint{0.633099in}{1.079942in}}%
\pgfpathlineto{\pgfqpoint{0.644163in}{1.088607in}}%
\pgfpathlineto{\pgfqpoint{0.647320in}{1.083056in}}%
\pgfpathlineto{\pgfqpoint{0.650468in}{1.077829in}}%
\pgfpathlineto{\pgfqpoint{0.653608in}{1.072918in}}%
\pgfpathlineto{\pgfqpoint{0.656740in}{1.068320in}}%
\pgfpathlineto{\pgfqpoint{0.645951in}{1.059834in}}%
\pgfpathlineto{\pgfqpoint{0.635656in}{1.051169in}}%
\pgfpathlineto{\pgfqpoint{0.625868in}{1.042331in}}%
\pgfpathlineto{\pgfqpoint{0.616597in}{1.033329in}}%
\pgfpathclose%
\pgfusepath{fill}%
\end{pgfscope}%
\begin{pgfscope}%
\pgfpathrectangle{\pgfqpoint{0.041670in}{0.041670in}}{\pgfqpoint{2.216660in}{2.216660in}}%
\pgfusepath{clip}%
\pgfsetbuttcap%
\pgfsetroundjoin%
\definecolor{currentfill}{rgb}{0.274128,0.199721,0.498911}%
\pgfsetfillcolor{currentfill}%
\pgfsetlinewidth{0.000000pt}%
\definecolor{currentstroke}{rgb}{0.000000,0.000000,0.000000}%
\pgfsetstrokecolor{currentstroke}%
\pgfsetdash{}{0pt}%
\pgfpathmoveto{\pgfqpoint{1.430555in}{1.186159in}}%
\pgfpathlineto{\pgfqpoint{1.432515in}{1.181746in}}%
\pgfpathlineto{\pgfqpoint{1.434475in}{1.177438in}}%
\pgfpathlineto{\pgfqpoint{1.436434in}{1.173239in}}%
\pgfpathlineto{\pgfqpoint{1.438393in}{1.169153in}}%
\pgfpathlineto{\pgfqpoint{1.450310in}{1.164971in}}%
\pgfpathlineto{\pgfqpoint{1.461980in}{1.160599in}}%
\pgfpathlineto{\pgfqpoint{1.473390in}{1.156042in}}%
\pgfpathlineto{\pgfqpoint{1.484531in}{1.151304in}}%
\pgfpathlineto{\pgfqpoint{1.482217in}{1.155537in}}%
\pgfpathlineto{\pgfqpoint{1.479902in}{1.159882in}}%
\pgfpathlineto{\pgfqpoint{1.477587in}{1.164336in}}%
\pgfpathlineto{\pgfqpoint{1.475270in}{1.168895in}}%
\pgfpathlineto{\pgfqpoint{1.464474in}{1.173478in}}%
\pgfpathlineto{\pgfqpoint{1.453416in}{1.177885in}}%
\pgfpathlineto{\pgfqpoint{1.442106in}{1.182113in}}%
\pgfpathlineto{\pgfqpoint{1.430555in}{1.186159in}}%
\pgfpathclose%
\pgfusepath{fill}%
\end{pgfscope}%
\begin{pgfscope}%
\pgfpathrectangle{\pgfqpoint{0.041670in}{0.041670in}}{\pgfqpoint{2.216660in}{2.216660in}}%
\pgfusepath{clip}%
\pgfsetbuttcap%
\pgfsetroundjoin%
\definecolor{currentfill}{rgb}{0.201239,0.383670,0.554294}%
\pgfsetfillcolor{currentfill}%
\pgfsetlinewidth{0.000000pt}%
\definecolor{currentstroke}{rgb}{0.000000,0.000000,0.000000}%
\pgfsetstrokecolor{currentstroke}%
\pgfsetdash{}{0pt}%
\pgfpathmoveto{\pgfqpoint{1.712432in}{1.319592in}}%
\pgfpathlineto{\pgfqpoint{1.715437in}{1.333481in}}%
\pgfpathlineto{\pgfqpoint{1.718457in}{1.347832in}}%
\pgfpathlineto{\pgfqpoint{1.721491in}{1.362652in}}%
\pgfpathlineto{\pgfqpoint{1.724540in}{1.377948in}}%
\pgfpathlineto{\pgfqpoint{1.739947in}{1.369472in}}%
\pgfpathlineto{\pgfqpoint{1.754854in}{1.360753in}}%
\pgfpathlineto{\pgfqpoint{1.769244in}{1.351797in}}%
\pgfpathlineto{\pgfqpoint{1.783103in}{1.342612in}}%
\pgfpathlineto{\pgfqpoint{1.779708in}{1.327435in}}%
\pgfpathlineto{\pgfqpoint{1.776331in}{1.312736in}}%
\pgfpathlineto{\pgfqpoint{1.772969in}{1.298509in}}%
\pgfpathlineto{\pgfqpoint{1.769624in}{1.284746in}}%
\pgfpathlineto{\pgfqpoint{1.756092in}{1.293803in}}%
\pgfpathlineto{\pgfqpoint{1.742040in}{1.302635in}}%
\pgfpathlineto{\pgfqpoint{1.727481in}{1.311234in}}%
\pgfpathlineto{\pgfqpoint{1.712432in}{1.319592in}}%
\pgfpathclose%
\pgfusepath{fill}%
\end{pgfscope}%
\begin{pgfscope}%
\pgfpathrectangle{\pgfqpoint{0.041670in}{0.041670in}}{\pgfqpoint{2.216660in}{2.216660in}}%
\pgfusepath{clip}%
\pgfsetbuttcap%
\pgfsetroundjoin%
\definecolor{currentfill}{rgb}{0.248629,0.278775,0.534556}%
\pgfsetfillcolor{currentfill}%
\pgfsetlinewidth{0.000000pt}%
\definecolor{currentstroke}{rgb}{0.000000,0.000000,0.000000}%
\pgfsetstrokecolor{currentstroke}%
\pgfsetdash{}{0pt}%
\pgfpathmoveto{\pgfqpoint{1.321396in}{1.248518in}}%
\pgfpathlineto{\pgfqpoint{1.322577in}{1.243603in}}%
\pgfpathlineto{\pgfqpoint{1.323758in}{1.238766in}}%
\pgfpathlineto{\pgfqpoint{1.324938in}{1.234010in}}%
\pgfpathlineto{\pgfqpoint{1.326117in}{1.229338in}}%
\pgfpathlineto{\pgfqpoint{1.338777in}{1.226947in}}%
\pgfpathlineto{\pgfqpoint{1.351292in}{1.224358in}}%
\pgfpathlineto{\pgfqpoint{1.363651in}{1.221573in}}%
\pgfpathlineto{\pgfqpoint{1.375842in}{1.218594in}}%
\pgfpathlineto{\pgfqpoint{1.374259in}{1.223360in}}%
\pgfpathlineto{\pgfqpoint{1.372675in}{1.228210in}}%
\pgfpathlineto{\pgfqpoint{1.371090in}{1.233141in}}%
\pgfpathlineto{\pgfqpoint{1.369504in}{1.238150in}}%
\pgfpathlineto{\pgfqpoint{1.357710in}{1.241025in}}%
\pgfpathlineto{\pgfqpoint{1.345753in}{1.243712in}}%
\pgfpathlineto{\pgfqpoint{1.333645in}{1.246211in}}%
\pgfpathlineto{\pgfqpoint{1.321396in}{1.248518in}}%
\pgfpathclose%
\pgfusepath{fill}%
\end{pgfscope}%
\begin{pgfscope}%
\pgfpathrectangle{\pgfqpoint{0.041670in}{0.041670in}}{\pgfqpoint{2.216660in}{2.216660in}}%
\pgfusepath{clip}%
\pgfsetbuttcap%
\pgfsetroundjoin%
\definecolor{currentfill}{rgb}{0.267004,0.004874,0.329415}%
\pgfsetfillcolor{currentfill}%
\pgfsetlinewidth{0.000000pt}%
\definecolor{currentstroke}{rgb}{0.000000,0.000000,0.000000}%
\pgfsetstrokecolor{currentstroke}%
\pgfsetdash{}{0pt}%
\pgfpathmoveto{\pgfqpoint{0.656656in}{1.002920in}}%
\pgfpathlineto{\pgfqpoint{0.653355in}{1.003977in}}%
\pgfpathlineto{\pgfqpoint{0.650048in}{1.005286in}}%
\pgfpathlineto{\pgfqpoint{0.646734in}{1.006852in}}%
\pgfpathlineto{\pgfqpoint{0.643414in}{1.008680in}}%
\pgfpathlineto{\pgfqpoint{0.652212in}{1.017284in}}%
\pgfpathlineto{\pgfqpoint{0.661504in}{1.025732in}}%
\pgfpathlineto{\pgfqpoint{0.671280in}{1.034016in}}%
\pgfpathlineto{\pgfqpoint{0.681529in}{1.042129in}}%
\pgfpathlineto{\pgfqpoint{0.684598in}{1.040104in}}%
\pgfpathlineto{\pgfqpoint{0.687662in}{1.038339in}}%
\pgfpathlineto{\pgfqpoint{0.690720in}{1.036831in}}%
\pgfpathlineto{\pgfqpoint{0.693774in}{1.035574in}}%
\pgfpathlineto{\pgfqpoint{0.683791in}{1.027653in}}%
\pgfpathlineto{\pgfqpoint{0.674270in}{1.019566in}}%
\pgfpathlineto{\pgfqpoint{0.665222in}{1.011319in}}%
\pgfpathlineto{\pgfqpoint{0.656656in}{1.002920in}}%
\pgfpathclose%
\pgfusepath{fill}%
\end{pgfscope}%
\begin{pgfscope}%
\pgfpathrectangle{\pgfqpoint{0.041670in}{0.041670in}}{\pgfqpoint{2.216660in}{2.216660in}}%
\pgfusepath{clip}%
\pgfsetbuttcap%
\pgfsetroundjoin%
\definecolor{currentfill}{rgb}{0.263663,0.237631,0.518762}%
\pgfsetfillcolor{currentfill}%
\pgfsetlinewidth{0.000000pt}%
\definecolor{currentstroke}{rgb}{0.000000,0.000000,0.000000}%
\pgfsetstrokecolor{currentstroke}%
\pgfsetdash{}{0pt}%
\pgfpathmoveto{\pgfqpoint{0.927253in}{1.201326in}}%
\pgfpathlineto{\pgfqpoint{0.925207in}{1.196497in}}%
\pgfpathlineto{\pgfqpoint{0.923162in}{1.191760in}}%
\pgfpathlineto{\pgfqpoint{0.921119in}{1.187117in}}%
\pgfpathlineto{\pgfqpoint{0.919076in}{1.182572in}}%
\pgfpathlineto{\pgfqpoint{0.930653in}{1.186597in}}%
\pgfpathlineto{\pgfqpoint{0.942459in}{1.190435in}}%
\pgfpathlineto{\pgfqpoint{0.954482in}{1.194083in}}%
\pgfpathlineto{\pgfqpoint{0.966711in}{1.197538in}}%
\pgfpathlineto{\pgfqpoint{0.968377in}{1.201956in}}%
\pgfpathlineto{\pgfqpoint{0.970044in}{1.206472in}}%
\pgfpathlineto{\pgfqpoint{0.971712in}{1.211083in}}%
\pgfpathlineto{\pgfqpoint{0.973381in}{1.215785in}}%
\pgfpathlineto{\pgfqpoint{0.961538in}{1.212447in}}%
\pgfpathlineto{\pgfqpoint{0.949895in}{1.208922in}}%
\pgfpathlineto{\pgfqpoint{0.938463in}{1.205215in}}%
\pgfpathlineto{\pgfqpoint{0.927253in}{1.201326in}}%
\pgfpathclose%
\pgfusepath{fill}%
\end{pgfscope}%
\begin{pgfscope}%
\pgfpathrectangle{\pgfqpoint{0.041670in}{0.041670in}}{\pgfqpoint{2.216660in}{2.216660in}}%
\pgfusepath{clip}%
\pgfsetbuttcap%
\pgfsetroundjoin%
\definecolor{currentfill}{rgb}{0.268510,0.009605,0.335427}%
\pgfsetfillcolor{currentfill}%
\pgfsetlinewidth{0.000000pt}%
\definecolor{currentstroke}{rgb}{0.000000,0.000000,0.000000}%
\pgfsetstrokecolor{currentstroke}%
\pgfsetdash{}{0pt}%
\pgfpathmoveto{\pgfqpoint{1.633104in}{1.040569in}}%
\pgfpathlineto{\pgfqpoint{1.636063in}{1.040023in}}%
\pgfpathlineto{\pgfqpoint{1.639025in}{1.039692in}}%
\pgfpathlineto{\pgfqpoint{1.641991in}{1.039581in}}%
\pgfpathlineto{\pgfqpoint{1.644961in}{1.039693in}}%
\pgfpathlineto{\pgfqpoint{1.655073in}{1.032117in}}%
\pgfpathlineto{\pgfqpoint{1.664743in}{1.024373in}}%
\pgfpathlineto{\pgfqpoint{1.673960in}{1.016468in}}%
\pgfpathlineto{\pgfqpoint{1.682715in}{1.008409in}}%
\pgfpathlineto{\pgfqpoint{1.679490in}{1.008497in}}%
\pgfpathlineto{\pgfqpoint{1.676269in}{1.008810in}}%
\pgfpathlineto{\pgfqpoint{1.673052in}{1.009342in}}%
\pgfpathlineto{\pgfqpoint{1.669840in}{1.010090in}}%
\pgfpathlineto{\pgfqpoint{1.661323in}{1.017941in}}%
\pgfpathlineto{\pgfqpoint{1.652355in}{1.025642in}}%
\pgfpathlineto{\pgfqpoint{1.642945in}{1.033187in}}%
\pgfpathlineto{\pgfqpoint{1.633104in}{1.040569in}}%
\pgfpathclose%
\pgfusepath{fill}%
\end{pgfscope}%
\begin{pgfscope}%
\pgfpathrectangle{\pgfqpoint{0.041670in}{0.041670in}}{\pgfqpoint{2.216660in}{2.216660in}}%
\pgfusepath{clip}%
\pgfsetbuttcap%
\pgfsetroundjoin%
\definecolor{currentfill}{rgb}{0.279566,0.067836,0.391917}%
\pgfsetfillcolor{currentfill}%
\pgfsetlinewidth{0.000000pt}%
\definecolor{currentstroke}{rgb}{0.000000,0.000000,0.000000}%
\pgfsetstrokecolor{currentstroke}%
\pgfsetdash{}{0pt}%
\pgfpathmoveto{\pgfqpoint{0.754055in}{1.056312in}}%
\pgfpathlineto{\pgfqpoint{0.751065in}{1.053471in}}%
\pgfpathlineto{\pgfqpoint{0.748073in}{1.050795in}}%
\pgfpathlineto{\pgfqpoint{0.745079in}{1.048289in}}%
\pgfpathlineto{\pgfqpoint{0.742084in}{1.045956in}}%
\pgfpathlineto{\pgfqpoint{0.751434in}{1.052924in}}%
\pgfpathlineto{\pgfqpoint{0.761184in}{1.059732in}}%
\pgfpathlineto{\pgfqpoint{0.771324in}{1.066374in}}%
\pgfpathlineto{\pgfqpoint{0.781842in}{1.072844in}}%
\pgfpathlineto{\pgfqpoint{0.784555in}{1.074985in}}%
\pgfpathlineto{\pgfqpoint{0.787267in}{1.077299in}}%
\pgfpathlineto{\pgfqpoint{0.789977in}{1.079782in}}%
\pgfpathlineto{\pgfqpoint{0.792686in}{1.082430in}}%
\pgfpathlineto{\pgfqpoint{0.782464in}{1.076145in}}%
\pgfpathlineto{\pgfqpoint{0.772612in}{1.069693in}}%
\pgfpathlineto{\pgfqpoint{0.763139in}{1.063080in}}%
\pgfpathlineto{\pgfqpoint{0.754055in}{1.056312in}}%
\pgfpathclose%
\pgfusepath{fill}%
\end{pgfscope}%
\begin{pgfscope}%
\pgfpathrectangle{\pgfqpoint{0.041670in}{0.041670in}}{\pgfqpoint{2.216660in}{2.216660in}}%
\pgfusepath{clip}%
\pgfsetbuttcap%
\pgfsetroundjoin%
\definecolor{currentfill}{rgb}{0.282327,0.094955,0.417331}%
\pgfsetfillcolor{currentfill}%
\pgfsetlinewidth{0.000000pt}%
\definecolor{currentstroke}{rgb}{0.000000,0.000000,0.000000}%
\pgfsetstrokecolor{currentstroke}%
\pgfsetdash{}{0pt}%
\pgfpathmoveto{\pgfqpoint{0.602986in}{1.052874in}}%
\pgfpathlineto{\pgfqpoint{0.599559in}{1.058571in}}%
\pgfpathlineto{\pgfqpoint{0.596122in}{1.064603in}}%
\pgfpathlineto{\pgfqpoint{0.592674in}{1.070976in}}%
\pgfpathlineto{\pgfqpoint{0.589216in}{1.077696in}}%
\pgfpathlineto{\pgfqpoint{0.598974in}{1.087076in}}%
\pgfpathlineto{\pgfqpoint{0.609272in}{1.096283in}}%
\pgfpathlineto{\pgfqpoint{0.620098in}{1.105310in}}%
\pgfpathlineto{\pgfqpoint{0.631442in}{1.114148in}}%
\pgfpathlineto{\pgfqpoint{0.634637in}{1.107251in}}%
\pgfpathlineto{\pgfqpoint{0.637822in}{1.100698in}}%
\pgfpathlineto{\pgfqpoint{0.640997in}{1.094486in}}%
\pgfpathlineto{\pgfqpoint{0.644163in}{1.088607in}}%
\pgfpathlineto{\pgfqpoint{0.633099in}{1.079942in}}%
\pgfpathlineto{\pgfqpoint{0.622540in}{1.071093in}}%
\pgfpathlineto{\pgfqpoint{0.612499in}{1.062068in}}%
\pgfpathlineto{\pgfqpoint{0.602986in}{1.052874in}}%
\pgfpathclose%
\pgfusepath{fill}%
\end{pgfscope}%
\begin{pgfscope}%
\pgfpathrectangle{\pgfqpoint{0.041670in}{0.041670in}}{\pgfqpoint{2.216660in}{2.216660in}}%
\pgfusepath{clip}%
\pgfsetbuttcap%
\pgfsetroundjoin%
\definecolor{currentfill}{rgb}{0.231674,0.318106,0.544834}%
\pgfsetfillcolor{currentfill}%
\pgfsetlinewidth{0.000000pt}%
\definecolor{currentstroke}{rgb}{0.000000,0.000000,0.000000}%
\pgfsetstrokecolor{currentstroke}%
\pgfsetdash{}{0pt}%
\pgfpathmoveto{\pgfqpoint{1.080827in}{1.274610in}}%
\pgfpathlineto{\pgfqpoint{1.079968in}{1.269470in}}%
\pgfpathlineto{\pgfqpoint{1.079110in}{1.264395in}}%
\pgfpathlineto{\pgfqpoint{1.078253in}{1.259388in}}%
\pgfpathlineto{\pgfqpoint{1.077396in}{1.254452in}}%
\pgfpathlineto{\pgfqpoint{1.090102in}{1.255954in}}%
\pgfpathlineto{\pgfqpoint{1.102890in}{1.257257in}}%
\pgfpathlineto{\pgfqpoint{1.115749in}{1.258360in}}%
\pgfpathlineto{\pgfqpoint{1.128665in}{1.259261in}}%
\pgfpathlineto{\pgfqpoint{1.129093in}{1.264154in}}%
\pgfpathlineto{\pgfqpoint{1.129522in}{1.269117in}}%
\pgfpathlineto{\pgfqpoint{1.129951in}{1.274149in}}%
\pgfpathlineto{\pgfqpoint{1.130380in}{1.279245in}}%
\pgfpathlineto{\pgfqpoint{1.117895in}{1.278376in}}%
\pgfpathlineto{\pgfqpoint{1.105468in}{1.277314in}}%
\pgfpathlineto{\pgfqpoint{1.093108in}{1.276058in}}%
\pgfpathlineto{\pgfqpoint{1.080827in}{1.274610in}}%
\pgfpathclose%
\pgfusepath{fill}%
\end{pgfscope}%
\begin{pgfscope}%
\pgfpathrectangle{\pgfqpoint{0.041670in}{0.041670in}}{\pgfqpoint{2.216660in}{2.216660in}}%
\pgfusepath{clip}%
\pgfsetbuttcap%
\pgfsetroundjoin%
\definecolor{currentfill}{rgb}{0.231674,0.318106,0.544834}%
\pgfsetfillcolor{currentfill}%
\pgfsetlinewidth{0.000000pt}%
\definecolor{currentstroke}{rgb}{0.000000,0.000000,0.000000}%
\pgfsetstrokecolor{currentstroke}%
\pgfsetdash{}{0pt}%
\pgfpathmoveto{\pgfqpoint{1.230920in}{1.279158in}}%
\pgfpathlineto{\pgfqpoint{1.231361in}{1.274061in}}%
\pgfpathlineto{\pgfqpoint{1.231802in}{1.269028in}}%
\pgfpathlineto{\pgfqpoint{1.232242in}{1.264064in}}%
\pgfpathlineto{\pgfqpoint{1.232682in}{1.259171in}}%
\pgfpathlineto{\pgfqpoint{1.245593in}{1.258247in}}%
\pgfpathlineto{\pgfqpoint{1.258444in}{1.257123in}}%
\pgfpathlineto{\pgfqpoint{1.271223in}{1.255797in}}%
\pgfpathlineto{\pgfqpoint{1.283920in}{1.254273in}}%
\pgfpathlineto{\pgfqpoint{1.283052in}{1.259210in}}%
\pgfpathlineto{\pgfqpoint{1.282182in}{1.264219in}}%
\pgfpathlineto{\pgfqpoint{1.281312in}{1.269296in}}%
\pgfpathlineto{\pgfqpoint{1.280442in}{1.274437in}}%
\pgfpathlineto{\pgfqpoint{1.268171in}{1.275906in}}%
\pgfpathlineto{\pgfqpoint{1.255819in}{1.277184in}}%
\pgfpathlineto{\pgfqpoint{1.243398in}{1.278268in}}%
\pgfpathlineto{\pgfqpoint{1.230920in}{1.279158in}}%
\pgfpathclose%
\pgfusepath{fill}%
\end{pgfscope}%
\begin{pgfscope}%
\pgfpathrectangle{\pgfqpoint{0.041670in}{0.041670in}}{\pgfqpoint{2.216660in}{2.216660in}}%
\pgfusepath{clip}%
\pgfsetbuttcap%
\pgfsetroundjoin%
\definecolor{currentfill}{rgb}{0.248629,0.278775,0.534556}%
\pgfsetfillcolor{currentfill}%
\pgfsetlinewidth{0.000000pt}%
\definecolor{currentstroke}{rgb}{0.000000,0.000000,0.000000}%
\pgfsetstrokecolor{currentstroke}%
\pgfsetdash{}{0pt}%
\pgfpathmoveto{\pgfqpoint{0.980067in}{1.235440in}}%
\pgfpathlineto{\pgfqpoint{0.978394in}{1.230406in}}%
\pgfpathlineto{\pgfqpoint{0.976722in}{1.225450in}}%
\pgfpathlineto{\pgfqpoint{0.975051in}{1.220575in}}%
\pgfpathlineto{\pgfqpoint{0.973381in}{1.215785in}}%
\pgfpathlineto{\pgfqpoint{0.985414in}{1.218935in}}%
\pgfpathlineto{\pgfqpoint{0.997624in}{1.221892in}}%
\pgfpathlineto{\pgfqpoint{1.010001in}{1.224656in}}%
\pgfpathlineto{\pgfqpoint{1.022532in}{1.227223in}}%
\pgfpathlineto{\pgfqpoint{1.023803in}{1.231913in}}%
\pgfpathlineto{\pgfqpoint{1.025074in}{1.236687in}}%
\pgfpathlineto{\pgfqpoint{1.026346in}{1.241543in}}%
\pgfpathlineto{\pgfqpoint{1.027619in}{1.246476in}}%
\pgfpathlineto{\pgfqpoint{1.015495in}{1.243999in}}%
\pgfpathlineto{\pgfqpoint{1.003520in}{1.241332in}}%
\pgfpathlineto{\pgfqpoint{0.991708in}{1.238479in}}%
\pgfpathlineto{\pgfqpoint{0.980067in}{1.235440in}}%
\pgfpathclose%
\pgfusepath{fill}%
\end{pgfscope}%
\begin{pgfscope}%
\pgfpathrectangle{\pgfqpoint{0.041670in}{0.041670in}}{\pgfqpoint{2.216660in}{2.216660in}}%
\pgfusepath{clip}%
\pgfsetbuttcap%
\pgfsetroundjoin%
\definecolor{currentfill}{rgb}{0.274128,0.199721,0.498911}%
\pgfsetfillcolor{currentfill}%
\pgfsetlinewidth{0.000000pt}%
\definecolor{currentstroke}{rgb}{0.000000,0.000000,0.000000}%
\pgfsetstrokecolor{currentstroke}%
\pgfsetdash{}{0pt}%
\pgfpathmoveto{\pgfqpoint{0.875272in}{1.164677in}}%
\pgfpathlineto{\pgfqpoint{0.872881in}{1.160082in}}%
\pgfpathlineto{\pgfqpoint{0.870490in}{1.155592in}}%
\pgfpathlineto{\pgfqpoint{0.868101in}{1.151211in}}%
\pgfpathlineto{\pgfqpoint{0.865712in}{1.146942in}}%
\pgfpathlineto{\pgfqpoint{0.876603in}{1.151839in}}%
\pgfpathlineto{\pgfqpoint{0.887774in}{1.156558in}}%
\pgfpathlineto{\pgfqpoint{0.899214in}{1.161094in}}%
\pgfpathlineto{\pgfqpoint{0.910911in}{1.165445in}}%
\pgfpathlineto{\pgfqpoint{0.912952in}{1.169562in}}%
\pgfpathlineto{\pgfqpoint{0.914992in}{1.173791in}}%
\pgfpathlineto{\pgfqpoint{0.917034in}{1.178129in}}%
\pgfpathlineto{\pgfqpoint{0.919076in}{1.182572in}}%
\pgfpathlineto{\pgfqpoint{0.907738in}{1.178364in}}%
\pgfpathlineto{\pgfqpoint{0.896651in}{1.173976in}}%
\pgfpathlineto{\pgfqpoint{0.885826in}{1.169412in}}%
\pgfpathlineto{\pgfqpoint{0.875272in}{1.164677in}}%
\pgfpathclose%
\pgfusepath{fill}%
\end{pgfscope}%
\begin{pgfscope}%
\pgfpathrectangle{\pgfqpoint{0.041670in}{0.041670in}}{\pgfqpoint{2.216660in}{2.216660in}}%
\pgfusepath{clip}%
\pgfsetbuttcap%
\pgfsetroundjoin%
\definecolor{currentfill}{rgb}{0.267004,0.004874,0.329415}%
\pgfsetfillcolor{currentfill}%
\pgfsetlinewidth{0.000000pt}%
\definecolor{currentstroke}{rgb}{0.000000,0.000000,0.000000}%
\pgfsetstrokecolor{currentstroke}%
\pgfsetdash{}{0pt}%
\pgfpathmoveto{\pgfqpoint{0.669809in}{1.001122in}}%
\pgfpathlineto{\pgfqpoint{0.666529in}{1.001216in}}%
\pgfpathlineto{\pgfqpoint{0.663243in}{1.001544in}}%
\pgfpathlineto{\pgfqpoint{0.659952in}{1.002111in}}%
\pgfpathlineto{\pgfqpoint{0.656656in}{1.002920in}}%
\pgfpathlineto{\pgfqpoint{0.665222in}{1.011319in}}%
\pgfpathlineto{\pgfqpoint{0.674270in}{1.019566in}}%
\pgfpathlineto{\pgfqpoint{0.683791in}{1.027653in}}%
\pgfpathlineto{\pgfqpoint{0.693774in}{1.035574in}}%
\pgfpathlineto{\pgfqpoint{0.696822in}{1.034564in}}%
\pgfpathlineto{\pgfqpoint{0.699865in}{1.033796in}}%
\pgfpathlineto{\pgfqpoint{0.702904in}{1.033265in}}%
\pgfpathlineto{\pgfqpoint{0.705938in}{1.032967in}}%
\pgfpathlineto{\pgfqpoint{0.696219in}{1.025242in}}%
\pgfpathlineto{\pgfqpoint{0.686951in}{1.017355in}}%
\pgfpathlineto{\pgfqpoint{0.678144in}{1.009312in}}%
\pgfpathlineto{\pgfqpoint{0.669809in}{1.001122in}}%
\pgfpathclose%
\pgfusepath{fill}%
\end{pgfscope}%
\begin{pgfscope}%
\pgfpathrectangle{\pgfqpoint{0.041670in}{0.041670in}}{\pgfqpoint{2.216660in}{2.216660in}}%
\pgfusepath{clip}%
\pgfsetbuttcap%
\pgfsetroundjoin%
\definecolor{currentfill}{rgb}{0.276194,0.190074,0.493001}%
\pgfsetfillcolor{currentfill}%
\pgfsetlinewidth{0.000000pt}%
\definecolor{currentstroke}{rgb}{0.000000,0.000000,0.000000}%
\pgfsetstrokecolor{currentstroke}%
\pgfsetdash{}{0pt}%
\pgfpathmoveto{\pgfqpoint{1.730585in}{1.153149in}}%
\pgfpathlineto{\pgfqpoint{1.733769in}{1.161899in}}%
\pgfpathlineto{\pgfqpoint{1.736964in}{1.171030in}}%
\pgfpathlineto{\pgfqpoint{1.740170in}{1.180548in}}%
\pgfpathlineto{\pgfqpoint{1.743389in}{1.190461in}}%
\pgfpathlineto{\pgfqpoint{1.755772in}{1.181475in}}%
\pgfpathlineto{\pgfqpoint{1.767628in}{1.172286in}}%
\pgfpathlineto{\pgfqpoint{1.778944in}{1.162902in}}%
\pgfpathlineto{\pgfqpoint{1.789707in}{1.153330in}}%
\pgfpathlineto{\pgfqpoint{1.786204in}{1.143578in}}%
\pgfpathlineto{\pgfqpoint{1.782716in}{1.134222in}}%
\pgfpathlineto{\pgfqpoint{1.779240in}{1.125255in}}%
\pgfpathlineto{\pgfqpoint{1.775777in}{1.116671in}}%
\pgfpathlineto{\pgfqpoint{1.765279in}{1.126074in}}%
\pgfpathlineto{\pgfqpoint{1.754239in}{1.135293in}}%
\pgfpathlineto{\pgfqpoint{1.742671in}{1.144321in}}%
\pgfpathlineto{\pgfqpoint{1.730585in}{1.153149in}}%
\pgfpathclose%
\pgfusepath{fill}%
\end{pgfscope}%
\begin{pgfscope}%
\pgfpathrectangle{\pgfqpoint{0.041670in}{0.041670in}}{\pgfqpoint{2.216660in}{2.216660in}}%
\pgfusepath{clip}%
\pgfsetbuttcap%
\pgfsetroundjoin%
\definecolor{currentfill}{rgb}{0.280255,0.165693,0.476498}%
\pgfsetfillcolor{currentfill}%
\pgfsetlinewidth{0.000000pt}%
\definecolor{currentstroke}{rgb}{0.000000,0.000000,0.000000}%
\pgfsetstrokecolor{currentstroke}%
\pgfsetdash{}{0pt}%
\pgfpathmoveto{\pgfqpoint{1.484531in}{1.151304in}}%
\pgfpathlineto{\pgfqpoint{1.486845in}{1.147187in}}%
\pgfpathlineto{\pgfqpoint{1.489158in}{1.143189in}}%
\pgfpathlineto{\pgfqpoint{1.491471in}{1.139314in}}%
\pgfpathlineto{\pgfqpoint{1.493783in}{1.135567in}}%
\pgfpathlineto{\pgfqpoint{1.504980in}{1.130489in}}%
\pgfpathlineto{\pgfqpoint{1.515877in}{1.125232in}}%
\pgfpathlineto{\pgfqpoint{1.526464in}{1.119800in}}%
\pgfpathlineto{\pgfqpoint{1.536729in}{1.114198in}}%
\pgfpathlineto{\pgfqpoint{1.534092in}{1.118115in}}%
\pgfpathlineto{\pgfqpoint{1.531455in}{1.122160in}}%
\pgfpathlineto{\pgfqpoint{1.528817in}{1.126327in}}%
\pgfpathlineto{\pgfqpoint{1.526179in}{1.130614in}}%
\pgfpathlineto{\pgfqpoint{1.516225in}{1.136038in}}%
\pgfpathlineto{\pgfqpoint{1.505959in}{1.141297in}}%
\pgfpathlineto{\pgfqpoint{1.495391in}{1.146387in}}%
\pgfpathlineto{\pgfqpoint{1.484531in}{1.151304in}}%
\pgfpathclose%
\pgfusepath{fill}%
\end{pgfscope}%
\begin{pgfscope}%
\pgfpathrectangle{\pgfqpoint{0.041670in}{0.041670in}}{\pgfqpoint{2.216660in}{2.216660in}}%
\pgfusepath{clip}%
\pgfsetbuttcap%
\pgfsetroundjoin%
\definecolor{currentfill}{rgb}{0.282327,0.094955,0.417331}%
\pgfsetfillcolor{currentfill}%
\pgfsetlinewidth{0.000000pt}%
\definecolor{currentstroke}{rgb}{0.000000,0.000000,0.000000}%
\pgfsetstrokecolor{currentstroke}%
\pgfsetdash{}{0pt}%
\pgfpathmoveto{\pgfqpoint{1.547279in}{1.099871in}}%
\pgfpathlineto{\pgfqpoint{1.549917in}{1.096644in}}%
\pgfpathlineto{\pgfqpoint{1.552556in}{1.093566in}}%
\pgfpathlineto{\pgfqpoint{1.555196in}{1.090640in}}%
\pgfpathlineto{\pgfqpoint{1.557836in}{1.087872in}}%
\pgfpathlineto{\pgfqpoint{1.568377in}{1.081740in}}%
\pgfpathlineto{\pgfqpoint{1.578558in}{1.075436in}}%
\pgfpathlineto{\pgfqpoint{1.588369in}{1.068966in}}%
\pgfpathlineto{\pgfqpoint{1.597800in}{1.062336in}}%
\pgfpathlineto{\pgfqpoint{1.594870in}{1.065293in}}%
\pgfpathlineto{\pgfqpoint{1.591940in}{1.068408in}}%
\pgfpathlineto{\pgfqpoint{1.589012in}{1.071676in}}%
\pgfpathlineto{\pgfqpoint{1.586085in}{1.075093in}}%
\pgfpathlineto{\pgfqpoint{1.576929in}{1.081526in}}%
\pgfpathlineto{\pgfqpoint{1.567403in}{1.087804in}}%
\pgfpathlineto{\pgfqpoint{1.557516in}{1.093921in}}%
\pgfpathlineto{\pgfqpoint{1.547279in}{1.099871in}}%
\pgfpathclose%
\pgfusepath{fill}%
\end{pgfscope}%
\begin{pgfscope}%
\pgfpathrectangle{\pgfqpoint{0.041670in}{0.041670in}}{\pgfqpoint{2.216660in}{2.216660in}}%
\pgfusepath{clip}%
\pgfsetbuttcap%
\pgfsetroundjoin%
\definecolor{currentfill}{rgb}{0.282884,0.135920,0.453427}%
\pgfsetfillcolor{currentfill}%
\pgfsetlinewidth{0.000000pt}%
\definecolor{currentstroke}{rgb}{0.000000,0.000000,0.000000}%
\pgfsetstrokecolor{currentstroke}%
\pgfsetdash{}{0pt}%
\pgfpathmoveto{\pgfqpoint{0.589216in}{1.077696in}}%
\pgfpathlineto{\pgfqpoint{0.585746in}{1.084769in}}%
\pgfpathlineto{\pgfqpoint{0.582264in}{1.092201in}}%
\pgfpathlineto{\pgfqpoint{0.578771in}{1.099997in}}%
\pgfpathlineto{\pgfqpoint{0.575265in}{1.108165in}}%
\pgfpathlineto{\pgfqpoint{0.585272in}{1.117725in}}%
\pgfpathlineto{\pgfqpoint{0.595831in}{1.127108in}}%
\pgfpathlineto{\pgfqpoint{0.606930in}{1.136306in}}%
\pgfpathlineto{\pgfqpoint{0.618557in}{1.145312in}}%
\pgfpathlineto{\pgfqpoint{0.621794in}{1.136973in}}%
\pgfpathlineto{\pgfqpoint{0.625021in}{1.129004in}}%
\pgfpathlineto{\pgfqpoint{0.628237in}{1.121398in}}%
\pgfpathlineto{\pgfqpoint{0.631442in}{1.114148in}}%
\pgfpathlineto{\pgfqpoint{0.620098in}{1.105310in}}%
\pgfpathlineto{\pgfqpoint{0.609272in}{1.096283in}}%
\pgfpathlineto{\pgfqpoint{0.598974in}{1.087076in}}%
\pgfpathlineto{\pgfqpoint{0.589216in}{1.077696in}}%
\pgfpathclose%
\pgfusepath{fill}%
\end{pgfscope}%
\begin{pgfscope}%
\pgfpathrectangle{\pgfqpoint{0.041670in}{0.041670in}}{\pgfqpoint{2.216660in}{2.216660in}}%
\pgfusepath{clip}%
\pgfsetbuttcap%
\pgfsetroundjoin%
\definecolor{currentfill}{rgb}{0.201239,0.383670,0.554294}%
\pgfsetfillcolor{currentfill}%
\pgfsetlinewidth{0.000000pt}%
\definecolor{currentstroke}{rgb}{0.000000,0.000000,0.000000}%
\pgfsetstrokecolor{currentstroke}%
\pgfsetdash{}{0pt}%
\pgfpathmoveto{\pgfqpoint{0.578707in}{1.276511in}}%
\pgfpathlineto{\pgfqpoint{0.575292in}{1.290244in}}%
\pgfpathlineto{\pgfqpoint{0.571860in}{1.304442in}}%
\pgfpathlineto{\pgfqpoint{0.568412in}{1.319111in}}%
\pgfpathlineto{\pgfqpoint{0.564946in}{1.334260in}}%
\pgfpathlineto{\pgfqpoint{0.578320in}{1.343644in}}%
\pgfpathlineto{\pgfqpoint{0.592239in}{1.352804in}}%
\pgfpathlineto{\pgfqpoint{0.606687in}{1.361734in}}%
\pgfpathlineto{\pgfqpoint{0.621650in}{1.370426in}}%
\pgfpathlineto{\pgfqpoint{0.624779in}{1.355155in}}%
\pgfpathlineto{\pgfqpoint{0.627893in}{1.340361in}}%
\pgfpathlineto{\pgfqpoint{0.630992in}{1.326037in}}%
\pgfpathlineto{\pgfqpoint{0.634076in}{1.312174in}}%
\pgfpathlineto{\pgfqpoint{0.619463in}{1.303602in}}%
\pgfpathlineto{\pgfqpoint{0.605353in}{1.294796in}}%
\pgfpathlineto{\pgfqpoint{0.591763in}{1.285763in}}%
\pgfpathlineto{\pgfqpoint{0.578707in}{1.276511in}}%
\pgfpathclose%
\pgfusepath{fill}%
\end{pgfscope}%
\begin{pgfscope}%
\pgfpathrectangle{\pgfqpoint{0.041670in}{0.041670in}}{\pgfqpoint{2.216660in}{2.216660in}}%
\pgfusepath{clip}%
\pgfsetbuttcap%
\pgfsetroundjoin%
\definecolor{currentfill}{rgb}{0.271305,0.019942,0.347269}%
\pgfsetfillcolor{currentfill}%
\pgfsetlinewidth{0.000000pt}%
\definecolor{currentstroke}{rgb}{0.000000,0.000000,0.000000}%
\pgfsetstrokecolor{currentstroke}%
\pgfsetdash{}{0pt}%
\pgfpathmoveto{\pgfqpoint{1.621299in}{1.044814in}}%
\pgfpathlineto{\pgfqpoint{1.624246in}{1.043452in}}%
\pgfpathlineto{\pgfqpoint{1.627196in}{1.042287in}}%
\pgfpathlineto{\pgfqpoint{1.630148in}{1.041325in}}%
\pgfpathlineto{\pgfqpoint{1.633104in}{1.040569in}}%
\pgfpathlineto{\pgfqpoint{1.642945in}{1.033187in}}%
\pgfpathlineto{\pgfqpoint{1.652355in}{1.025642in}}%
\pgfpathlineto{\pgfqpoint{1.661323in}{1.017941in}}%
\pgfpathlineto{\pgfqpoint{1.669840in}{1.010090in}}%
\pgfpathlineto{\pgfqpoint{1.666631in}{1.011049in}}%
\pgfpathlineto{\pgfqpoint{1.663425in}{1.012215in}}%
\pgfpathlineto{\pgfqpoint{1.660223in}{1.013584in}}%
\pgfpathlineto{\pgfqpoint{1.657024in}{1.015151in}}%
\pgfpathlineto{\pgfqpoint{1.648744in}{1.022791in}}%
\pgfpathlineto{\pgfqpoint{1.640023in}{1.030286in}}%
\pgfpathlineto{\pgfqpoint{1.630871in}{1.037629in}}%
\pgfpathlineto{\pgfqpoint{1.621299in}{1.044814in}}%
\pgfpathclose%
\pgfusepath{fill}%
\end{pgfscope}%
\begin{pgfscope}%
\pgfpathrectangle{\pgfqpoint{0.041670in}{0.041670in}}{\pgfqpoint{2.216660in}{2.216660in}}%
\pgfusepath{clip}%
\pgfsetbuttcap%
\pgfsetroundjoin%
\definecolor{currentfill}{rgb}{0.231674,0.318106,0.544834}%
\pgfsetfillcolor{currentfill}%
\pgfsetlinewidth{0.000000pt}%
\definecolor{currentstroke}{rgb}{0.000000,0.000000,0.000000}%
\pgfsetstrokecolor{currentstroke}%
\pgfsetdash{}{0pt}%
\pgfpathmoveto{\pgfqpoint{1.280442in}{1.274437in}}%
\pgfpathlineto{\pgfqpoint{1.281312in}{1.269296in}}%
\pgfpathlineto{\pgfqpoint{1.282182in}{1.264219in}}%
\pgfpathlineto{\pgfqpoint{1.283052in}{1.259210in}}%
\pgfpathlineto{\pgfqpoint{1.283920in}{1.254273in}}%
\pgfpathlineto{\pgfqpoint{1.296522in}{1.252550in}}%
\pgfpathlineto{\pgfqpoint{1.309018in}{1.250631in}}%
\pgfpathlineto{\pgfqpoint{1.321396in}{1.248518in}}%
\pgfpathlineto{\pgfqpoint{1.320213in}{1.253507in}}%
\pgfpathlineto{\pgfqpoint{1.319030in}{1.258568in}}%
\pgfpathlineto{\pgfqpoint{1.317845in}{1.263697in}}%
\pgfpathlineto{\pgfqpoint{1.316660in}{1.268891in}}%
\pgfpathlineto{\pgfqpoint{1.304698in}{1.270928in}}%
\pgfpathlineto{\pgfqpoint{1.292621in}{1.272778in}}%
\pgfpathlineto{\pgfqpoint{1.280442in}{1.274437in}}%
\pgfpathclose%
\pgfusepath{fill}%
\end{pgfscope}%
\begin{pgfscope}%
\pgfpathrectangle{\pgfqpoint{0.041670in}{0.041670in}}{\pgfqpoint{2.216660in}{2.216660in}}%
\pgfusepath{clip}%
\pgfsetbuttcap%
\pgfsetroundjoin%
\definecolor{currentfill}{rgb}{0.268510,0.009605,0.335427}%
\pgfsetfillcolor{currentfill}%
\pgfsetlinewidth{0.000000pt}%
\definecolor{currentstroke}{rgb}{0.000000,0.000000,0.000000}%
\pgfsetstrokecolor{currentstroke}%
\pgfsetdash{}{0pt}%
\pgfpathmoveto{\pgfqpoint{0.682887in}{1.002991in}}%
\pgfpathlineto{\pgfqpoint{0.679624in}{1.002196in}}%
\pgfpathlineto{\pgfqpoint{0.676356in}{1.001616in}}%
\pgfpathlineto{\pgfqpoint{0.673085in}{1.001257in}}%
\pgfpathlineto{\pgfqpoint{0.669809in}{1.001122in}}%
\pgfpathlineto{\pgfqpoint{0.678144in}{1.009312in}}%
\pgfpathlineto{\pgfqpoint{0.686951in}{1.017355in}}%
\pgfpathlineto{\pgfqpoint{0.696219in}{1.025242in}}%
\pgfpathlineto{\pgfqpoint{0.705938in}{1.032967in}}%
\pgfpathlineto{\pgfqpoint{0.708969in}{1.032898in}}%
\pgfpathlineto{\pgfqpoint{0.711995in}{1.033052in}}%
\pgfpathlineto{\pgfqpoint{0.715018in}{1.033426in}}%
\pgfpathlineto{\pgfqpoint{0.718037in}{1.034015in}}%
\pgfpathlineto{\pgfqpoint{0.708579in}{1.026488in}}%
\pgfpathlineto{\pgfqpoint{0.699561in}{1.018804in}}%
\pgfpathlineto{\pgfqpoint{0.690994in}{1.010969in}}%
\pgfpathlineto{\pgfqpoint{0.682887in}{1.002991in}}%
\pgfpathclose%
\pgfusepath{fill}%
\end{pgfscope}%
\begin{pgfscope}%
\pgfpathrectangle{\pgfqpoint{0.041670in}{0.041670in}}{\pgfqpoint{2.216660in}{2.216660in}}%
\pgfusepath{clip}%
\pgfsetbuttcap%
\pgfsetroundjoin%
\definecolor{currentfill}{rgb}{0.231674,0.318106,0.544834}%
\pgfsetfillcolor{currentfill}%
\pgfsetlinewidth{0.000000pt}%
\definecolor{currentstroke}{rgb}{0.000000,0.000000,0.000000}%
\pgfsetstrokecolor{currentstroke}%
\pgfsetdash{}{0pt}%
\pgfpathmoveto{\pgfqpoint{1.032721in}{1.266924in}}%
\pgfpathlineto{\pgfqpoint{1.031444in}{1.261712in}}%
\pgfpathlineto{\pgfqpoint{1.030168in}{1.256564in}}%
\pgfpathlineto{\pgfqpoint{1.028893in}{1.251484in}}%
\pgfpathlineto{\pgfqpoint{1.027619in}{1.246476in}}%
\pgfpathlineto{\pgfqpoint{1.039883in}{1.248762in}}%
\pgfpathlineto{\pgfqpoint{1.052275in}{1.250854in}}%
\pgfpathlineto{\pgfqpoint{1.064783in}{1.252752in}}%
\pgfpathlineto{\pgfqpoint{1.077396in}{1.254452in}}%
\pgfpathlineto{\pgfqpoint{1.078253in}{1.259388in}}%
\pgfpathlineto{\pgfqpoint{1.079110in}{1.264395in}}%
\pgfpathlineto{\pgfqpoint{1.079968in}{1.269470in}}%
\pgfpathlineto{\pgfqpoint{1.080827in}{1.274610in}}%
\pgfpathlineto{\pgfqpoint{1.068637in}{1.272971in}}%
\pgfpathlineto{\pgfqpoint{1.056549in}{1.271143in}}%
\pgfpathlineto{\pgfqpoint{1.044573in}{1.269127in}}%
\pgfpathlineto{\pgfqpoint{1.032721in}{1.266924in}}%
\pgfpathclose%
\pgfusepath{fill}%
\end{pgfscope}%
\begin{pgfscope}%
\pgfpathrectangle{\pgfqpoint{0.041670in}{0.041670in}}{\pgfqpoint{2.216660in}{2.216660in}}%
\pgfusepath{clip}%
\pgfsetbuttcap%
\pgfsetroundjoin%
\definecolor{currentfill}{rgb}{0.280255,0.165693,0.476498}%
\pgfsetfillcolor{currentfill}%
\pgfsetlinewidth{0.000000pt}%
\definecolor{currentstroke}{rgb}{0.000000,0.000000,0.000000}%
\pgfsetstrokecolor{currentstroke}%
\pgfsetdash{}{0pt}%
\pgfpathmoveto{\pgfqpoint{0.825152in}{1.125658in}}%
\pgfpathlineto{\pgfqpoint{0.822447in}{1.121330in}}%
\pgfpathlineto{\pgfqpoint{0.819742in}{1.117122in}}%
\pgfpathlineto{\pgfqpoint{0.817037in}{1.113037in}}%
\pgfpathlineto{\pgfqpoint{0.814333in}{1.109079in}}%
\pgfpathlineto{\pgfqpoint{0.824305in}{1.114829in}}%
\pgfpathlineto{\pgfqpoint{0.834607in}{1.120413in}}%
\pgfpathlineto{\pgfqpoint{0.845228in}{1.125825in}}%
\pgfpathlineto{\pgfqpoint{0.856159in}{1.131062in}}%
\pgfpathlineto{\pgfqpoint{0.858547in}{1.134846in}}%
\pgfpathlineto{\pgfqpoint{0.860935in}{1.138756in}}%
\pgfpathlineto{\pgfqpoint{0.863323in}{1.142789in}}%
\pgfpathlineto{\pgfqpoint{0.865712in}{1.146942in}}%
\pgfpathlineto{\pgfqpoint{0.855110in}{1.141871in}}%
\pgfpathlineto{\pgfqpoint{0.844810in}{1.136631in}}%
\pgfpathlineto{\pgfqpoint{0.834821in}{1.131225in}}%
\pgfpathlineto{\pgfqpoint{0.825152in}{1.125658in}}%
\pgfpathclose%
\pgfusepath{fill}%
\end{pgfscope}%
\begin{pgfscope}%
\pgfpathrectangle{\pgfqpoint{0.041670in}{0.041670in}}{\pgfqpoint{2.216660in}{2.216660in}}%
\pgfusepath{clip}%
\pgfsetbuttcap%
\pgfsetroundjoin%
\definecolor{currentfill}{rgb}{0.282327,0.094955,0.417331}%
\pgfsetfillcolor{currentfill}%
\pgfsetlinewidth{0.000000pt}%
\definecolor{currentstroke}{rgb}{0.000000,0.000000,0.000000}%
\pgfsetstrokecolor{currentstroke}%
\pgfsetdash{}{0pt}%
\pgfpathmoveto{\pgfqpoint{0.766006in}{1.069249in}}%
\pgfpathlineto{\pgfqpoint{0.763019in}{1.065787in}}%
\pgfpathlineto{\pgfqpoint{0.760032in}{1.062474in}}%
\pgfpathlineto{\pgfqpoint{0.757044in}{1.059314in}}%
\pgfpathlineto{\pgfqpoint{0.754055in}{1.056312in}}%
\pgfpathlineto{\pgfqpoint{0.763139in}{1.063080in}}%
\pgfpathlineto{\pgfqpoint{0.772612in}{1.069693in}}%
\pgfpathlineto{\pgfqpoint{0.782464in}{1.076145in}}%
\pgfpathlineto{\pgfqpoint{0.792686in}{1.082430in}}%
\pgfpathlineto{\pgfqpoint{0.795394in}{1.085238in}}%
\pgfpathlineto{\pgfqpoint{0.798101in}{1.088203in}}%
\pgfpathlineto{\pgfqpoint{0.800808in}{1.091322in}}%
\pgfpathlineto{\pgfqpoint{0.803514in}{1.094590in}}%
\pgfpathlineto{\pgfqpoint{0.793587in}{1.088491in}}%
\pgfpathlineto{\pgfqpoint{0.784021in}{1.082231in}}%
\pgfpathlineto{\pgfqpoint{0.774823in}{1.075815in}}%
\pgfpathlineto{\pgfqpoint{0.766006in}{1.069249in}}%
\pgfpathclose%
\pgfusepath{fill}%
\end{pgfscope}%
\begin{pgfscope}%
\pgfpathrectangle{\pgfqpoint{0.041670in}{0.041670in}}{\pgfqpoint{2.216660in}{2.216660in}}%
\pgfusepath{clip}%
\pgfsetbuttcap%
\pgfsetroundjoin%
\definecolor{currentfill}{rgb}{0.172719,0.448791,0.557885}%
\pgfsetfillcolor{currentfill}%
\pgfsetlinewidth{0.000000pt}%
\definecolor{currentstroke}{rgb}{0.000000,0.000000,0.000000}%
\pgfsetstrokecolor{currentstroke}%
\pgfsetdash{}{0pt}%
\pgfpathmoveto{\pgfqpoint{1.724540in}{1.377948in}}%
\pgfpathlineto{\pgfqpoint{1.727604in}{1.393728in}}%
\pgfpathlineto{\pgfqpoint{1.730684in}{1.410001in}}%
\pgfpathlineto{\pgfqpoint{1.733780in}{1.426774in}}%
\pgfpathlineto{\pgfqpoint{1.749461in}{1.418217in}}%
\pgfpathlineto{\pgfqpoint{1.764634in}{1.409414in}}%
\pgfpathlineto{\pgfqpoint{1.779282in}{1.400371in}}%
\pgfpathlineto{\pgfqpoint{1.793391in}{1.391096in}}%
\pgfpathlineto{\pgfqpoint{1.789944in}{1.374435in}}%
\pgfpathlineto{\pgfqpoint{1.786515in}{1.358276in}}%
\pgfpathlineto{\pgfqpoint{1.783103in}{1.342612in}}%
\pgfpathlineto{\pgfqpoint{1.769244in}{1.351797in}}%
\pgfpathlineto{\pgfqpoint{1.754854in}{1.360753in}}%
\pgfpathlineto{\pgfqpoint{1.739947in}{1.369472in}}%
\pgfpathlineto{\pgfqpoint{1.724540in}{1.377948in}}%
\pgfpathclose%
\pgfusepath{fill}%
\end{pgfscope}%
\begin{pgfscope}%
\pgfpathrectangle{\pgfqpoint{0.041670in}{0.041670in}}{\pgfqpoint{2.216660in}{2.216660in}}%
\pgfusepath{clip}%
\pgfsetbuttcap%
\pgfsetroundjoin%
\definecolor{currentfill}{rgb}{0.248629,0.278775,0.534556}%
\pgfsetfillcolor{currentfill}%
\pgfsetlinewidth{0.000000pt}%
\definecolor{currentstroke}{rgb}{0.000000,0.000000,0.000000}%
\pgfsetstrokecolor{currentstroke}%
\pgfsetdash{}{0pt}%
\pgfpathmoveto{\pgfqpoint{1.369504in}{1.238150in}}%
\pgfpathlineto{\pgfqpoint{1.371090in}{1.233141in}}%
\pgfpathlineto{\pgfqpoint{1.372675in}{1.228210in}}%
\pgfpathlineto{\pgfqpoint{1.374259in}{1.223360in}}%
\pgfpathlineto{\pgfqpoint{1.375842in}{1.218594in}}%
\pgfpathlineto{\pgfqpoint{1.387854in}{1.215424in}}%
\pgfpathlineto{\pgfqpoint{1.399675in}{1.212064in}}%
\pgfpathlineto{\pgfqpoint{1.411296in}{1.208519in}}%
\pgfpathlineto{\pgfqpoint{1.422703in}{1.204791in}}%
\pgfpathlineto{\pgfqpoint{1.420738in}{1.209678in}}%
\pgfpathlineto{\pgfqpoint{1.418771in}{1.214649in}}%
\pgfpathlineto{\pgfqpoint{1.416804in}{1.219702in}}%
\pgfpathlineto{\pgfqpoint{1.414835in}{1.224833in}}%
\pgfpathlineto{\pgfqpoint{1.403800in}{1.228430in}}%
\pgfpathlineto{\pgfqpoint{1.392560in}{1.231850in}}%
\pgfpathlineto{\pgfqpoint{1.381125in}{1.235091in}}%
\pgfpathlineto{\pgfqpoint{1.369504in}{1.238150in}}%
\pgfpathclose%
\pgfusepath{fill}%
\end{pgfscope}%
\begin{pgfscope}%
\pgfpathrectangle{\pgfqpoint{0.041670in}{0.041670in}}{\pgfqpoint{2.216660in}{2.216660in}}%
\pgfusepath{clip}%
\pgfsetbuttcap%
\pgfsetroundjoin%
\definecolor{currentfill}{rgb}{0.263663,0.237631,0.518762}%
\pgfsetfillcolor{currentfill}%
\pgfsetlinewidth{0.000000pt}%
\definecolor{currentstroke}{rgb}{0.000000,0.000000,0.000000}%
\pgfsetstrokecolor{currentstroke}%
\pgfsetdash{}{0pt}%
\pgfpathmoveto{\pgfqpoint{1.422703in}{1.204791in}}%
\pgfpathlineto{\pgfqpoint{1.424668in}{1.199993in}}%
\pgfpathlineto{\pgfqpoint{1.426631in}{1.195285in}}%
\pgfpathlineto{\pgfqpoint{1.428593in}{1.190673in}}%
\pgfpathlineto{\pgfqpoint{1.430555in}{1.186159in}}%
\pgfpathlineto{\pgfqpoint{1.442106in}{1.182113in}}%
\pgfpathlineto{\pgfqpoint{1.453416in}{1.177885in}}%
\pgfpathlineto{\pgfqpoint{1.464474in}{1.173478in}}%
\pgfpathlineto{\pgfqpoint{1.475270in}{1.168895in}}%
\pgfpathlineto{\pgfqpoint{1.472953in}{1.173556in}}%
\pgfpathlineto{\pgfqpoint{1.470635in}{1.178315in}}%
\pgfpathlineto{\pgfqpoint{1.468316in}{1.183169in}}%
\pgfpathlineto{\pgfqpoint{1.465996in}{1.188115in}}%
\pgfpathlineto{\pgfqpoint{1.455544in}{1.192542in}}%
\pgfpathlineto{\pgfqpoint{1.444838in}{1.196799in}}%
\pgfpathlineto{\pgfqpoint{1.433888in}{1.200883in}}%
\pgfpathlineto{\pgfqpoint{1.422703in}{1.204791in}}%
\pgfpathclose%
\pgfusepath{fill}%
\end{pgfscope}%
\begin{pgfscope}%
\pgfpathrectangle{\pgfqpoint{0.041670in}{0.041670in}}{\pgfqpoint{2.216660in}{2.216660in}}%
\pgfusepath{clip}%
\pgfsetbuttcap%
\pgfsetroundjoin%
\definecolor{currentfill}{rgb}{0.276194,0.190074,0.493001}%
\pgfsetfillcolor{currentfill}%
\pgfsetlinewidth{0.000000pt}%
\definecolor{currentstroke}{rgb}{0.000000,0.000000,0.000000}%
\pgfsetstrokecolor{currentstroke}%
\pgfsetdash{}{0pt}%
\pgfpathmoveto{\pgfqpoint{0.575265in}{1.108165in}}%
\pgfpathlineto{\pgfqpoint{0.571746in}{1.116710in}}%
\pgfpathlineto{\pgfqpoint{0.568214in}{1.125638in}}%
\pgfpathlineto{\pgfqpoint{0.564669in}{1.134956in}}%
\pgfpathlineto{\pgfqpoint{0.561111in}{1.144671in}}%
\pgfpathlineto{\pgfqpoint{0.571371in}{1.154403in}}%
\pgfpathlineto{\pgfqpoint{0.582196in}{1.163954in}}%
\pgfpathlineto{\pgfqpoint{0.593572in}{1.173317in}}%
\pgfpathlineto{\pgfqpoint{0.605487in}{1.182484in}}%
\pgfpathlineto{\pgfqpoint{0.608773in}{1.172605in}}%
\pgfpathlineto{\pgfqpoint{0.612046in}{1.163122in}}%
\pgfpathlineto{\pgfqpoint{0.615308in}{1.154026in}}%
\pgfpathlineto{\pgfqpoint{0.618557in}{1.145312in}}%
\pgfpathlineto{\pgfqpoint{0.606930in}{1.136306in}}%
\pgfpathlineto{\pgfqpoint{0.595831in}{1.127108in}}%
\pgfpathlineto{\pgfqpoint{0.585272in}{1.117725in}}%
\pgfpathlineto{\pgfqpoint{0.575265in}{1.108165in}}%
\pgfpathclose%
\pgfusepath{fill}%
\end{pgfscope}%
\begin{pgfscope}%
\pgfpathrectangle{\pgfqpoint{0.041670in}{0.041670in}}{\pgfqpoint{2.216660in}{2.216660in}}%
\pgfusepath{clip}%
\pgfsetbuttcap%
\pgfsetroundjoin%
\definecolor{currentfill}{rgb}{0.260571,0.246922,0.522828}%
\pgfsetfillcolor{currentfill}%
\pgfsetlinewidth{0.000000pt}%
\definecolor{currentstroke}{rgb}{0.000000,0.000000,0.000000}%
\pgfsetstrokecolor{currentstroke}%
\pgfsetdash{}{0pt}%
\pgfpathmoveto{\pgfqpoint{1.743389in}{1.190461in}}%
\pgfpathlineto{\pgfqpoint{1.746621in}{1.200773in}}%
\pgfpathlineto{\pgfqpoint{1.749865in}{1.211494in}}%
\pgfpathlineto{\pgfqpoint{1.753123in}{1.222627in}}%
\pgfpathlineto{\pgfqpoint{1.756394in}{1.234182in}}%
\pgfpathlineto{\pgfqpoint{1.769081in}{1.225047in}}%
\pgfpathlineto{\pgfqpoint{1.781229in}{1.215705in}}%
\pgfpathlineto{\pgfqpoint{1.792826in}{1.206163in}}%
\pgfpathlineto{\pgfqpoint{1.803858in}{1.196430in}}%
\pgfpathlineto{\pgfqpoint{1.800298in}{1.185028in}}%
\pgfpathlineto{\pgfqpoint{1.796753in}{1.174048in}}%
\pgfpathlineto{\pgfqpoint{1.793223in}{1.163485in}}%
\pgfpathlineto{\pgfqpoint{1.789707in}{1.153330in}}%
\pgfpathlineto{\pgfqpoint{1.778944in}{1.162902in}}%
\pgfpathlineto{\pgfqpoint{1.767628in}{1.172286in}}%
\pgfpathlineto{\pgfqpoint{1.755772in}{1.181475in}}%
\pgfpathlineto{\pgfqpoint{1.743389in}{1.190461in}}%
\pgfpathclose%
\pgfusepath{fill}%
\end{pgfscope}%
\begin{pgfscope}%
\pgfpathrectangle{\pgfqpoint{0.041670in}{0.041670in}}{\pgfqpoint{2.216660in}{2.216660in}}%
\pgfusepath{clip}%
\pgfsetbuttcap%
\pgfsetroundjoin%
\definecolor{currentfill}{rgb}{0.271305,0.019942,0.347269}%
\pgfsetfillcolor{currentfill}%
\pgfsetlinewidth{0.000000pt}%
\definecolor{currentstroke}{rgb}{0.000000,0.000000,0.000000}%
\pgfsetstrokecolor{currentstroke}%
\pgfsetdash{}{0pt}%
\pgfpathmoveto{\pgfqpoint{0.695903in}{1.008243in}}%
\pgfpathlineto{\pgfqpoint{0.692654in}{1.006628in}}%
\pgfpathlineto{\pgfqpoint{0.689402in}{1.005211in}}%
\pgfpathlineto{\pgfqpoint{0.686146in}{1.003998in}}%
\pgfpathlineto{\pgfqpoint{0.682887in}{1.002991in}}%
\pgfpathlineto{\pgfqpoint{0.690994in}{1.010969in}}%
\pgfpathlineto{\pgfqpoint{0.699561in}{1.018804in}}%
\pgfpathlineto{\pgfqpoint{0.708579in}{1.026488in}}%
\pgfpathlineto{\pgfqpoint{0.718037in}{1.034015in}}%
\pgfpathlineto{\pgfqpoint{0.721052in}{1.034815in}}%
\pgfpathlineto{\pgfqpoint{0.724065in}{1.035821in}}%
\pgfpathlineto{\pgfqpoint{0.727075in}{1.037030in}}%
\pgfpathlineto{\pgfqpoint{0.730081in}{1.038436in}}%
\pgfpathlineto{\pgfqpoint{0.720882in}{1.031110in}}%
\pgfpathlineto{\pgfqpoint{0.712113in}{1.023631in}}%
\pgfpathlineto{\pgfqpoint{0.703784in}{1.016007in}}%
\pgfpathlineto{\pgfqpoint{0.695903in}{1.008243in}}%
\pgfpathclose%
\pgfusepath{fill}%
\end{pgfscope}%
\begin{pgfscope}%
\pgfpathrectangle{\pgfqpoint{0.041670in}{0.041670in}}{\pgfqpoint{2.216660in}{2.216660in}}%
\pgfusepath{clip}%
\pgfsetbuttcap%
\pgfsetroundjoin%
\definecolor{currentfill}{rgb}{0.274952,0.037752,0.364543}%
\pgfsetfillcolor{currentfill}%
\pgfsetlinewidth{0.000000pt}%
\definecolor{currentstroke}{rgb}{0.000000,0.000000,0.000000}%
\pgfsetstrokecolor{currentstroke}%
\pgfsetdash{}{0pt}%
\pgfpathmoveto{\pgfqpoint{1.609534in}{1.052158in}}%
\pgfpathlineto{\pgfqpoint{1.612472in}{1.050046in}}%
\pgfpathlineto{\pgfqpoint{1.615412in}{1.048115in}}%
\pgfpathlineto{\pgfqpoint{1.618354in}{1.046370in}}%
\pgfpathlineto{\pgfqpoint{1.621299in}{1.044814in}}%
\pgfpathlineto{\pgfqpoint{1.630871in}{1.037629in}}%
\pgfpathlineto{\pgfqpoint{1.640023in}{1.030286in}}%
\pgfpathlineto{\pgfqpoint{1.648744in}{1.022791in}}%
\pgfpathlineto{\pgfqpoint{1.657024in}{1.015151in}}%
\pgfpathlineto{\pgfqpoint{1.653828in}{1.016912in}}%
\pgfpathlineto{\pgfqpoint{1.650634in}{1.018863in}}%
\pgfpathlineto{\pgfqpoint{1.647443in}{1.021000in}}%
\pgfpathlineto{\pgfqpoint{1.644255in}{1.023318in}}%
\pgfpathlineto{\pgfqpoint{1.636210in}{1.030746in}}%
\pgfpathlineto{\pgfqpoint{1.627735in}{1.038032in}}%
\pgfpathlineto{\pgfqpoint{1.618840in}{1.045172in}}%
\pgfpathlineto{\pgfqpoint{1.609534in}{1.052158in}}%
\pgfpathclose%
\pgfusepath{fill}%
\end{pgfscope}%
\begin{pgfscope}%
\pgfpathrectangle{\pgfqpoint{0.041670in}{0.041670in}}{\pgfqpoint{2.216660in}{2.216660in}}%
\pgfusepath{clip}%
\pgfsetbuttcap%
\pgfsetroundjoin%
\definecolor{currentfill}{rgb}{0.248629,0.278775,0.534556}%
\pgfsetfillcolor{currentfill}%
\pgfsetlinewidth{0.000000pt}%
\definecolor{currentstroke}{rgb}{0.000000,0.000000,0.000000}%
\pgfsetstrokecolor{currentstroke}%
\pgfsetdash{}{0pt}%
\pgfpathmoveto{\pgfqpoint{0.935449in}{1.221490in}}%
\pgfpathlineto{\pgfqpoint{0.933398in}{1.216329in}}%
\pgfpathlineto{\pgfqpoint{0.931348in}{1.211246in}}%
\pgfpathlineto{\pgfqpoint{0.929300in}{1.206244in}}%
\pgfpathlineto{\pgfqpoint{0.927253in}{1.201326in}}%
\pgfpathlineto{\pgfqpoint{0.938463in}{1.205215in}}%
\pgfpathlineto{\pgfqpoint{0.949895in}{1.208922in}}%
\pgfpathlineto{\pgfqpoint{0.961538in}{1.212447in}}%
\pgfpathlineto{\pgfqpoint{0.973381in}{1.215785in}}%
\pgfpathlineto{\pgfqpoint{0.975051in}{1.220575in}}%
\pgfpathlineto{\pgfqpoint{0.976722in}{1.225450in}}%
\pgfpathlineto{\pgfqpoint{0.978394in}{1.230406in}}%
\pgfpathlineto{\pgfqpoint{0.980067in}{1.235440in}}%
\pgfpathlineto{\pgfqpoint{0.968611in}{1.232219in}}%
\pgfpathlineto{\pgfqpoint{0.957348in}{1.228818in}}%
\pgfpathlineto{\pgfqpoint{0.946291in}{1.225241in}}%
\pgfpathlineto{\pgfqpoint{0.935449in}{1.221490in}}%
\pgfpathclose%
\pgfusepath{fill}%
\end{pgfscope}%
\begin{pgfscope}%
\pgfpathrectangle{\pgfqpoint{0.041670in}{0.041670in}}{\pgfqpoint{2.216660in}{2.216660in}}%
\pgfusepath{clip}%
\pgfsetbuttcap%
\pgfsetroundjoin%
\definecolor{currentfill}{rgb}{0.212395,0.359683,0.551710}%
\pgfsetfillcolor{currentfill}%
\pgfsetlinewidth{0.000000pt}%
\definecolor{currentstroke}{rgb}{0.000000,0.000000,0.000000}%
\pgfsetstrokecolor{currentstroke}%
\pgfsetdash{}{0pt}%
\pgfpathmoveto{\pgfqpoint{1.132101in}{1.300216in}}%
\pgfpathlineto{\pgfqpoint{1.131670in}{1.294892in}}%
\pgfpathlineto{\pgfqpoint{1.131239in}{1.289620in}}%
\pgfpathlineto{\pgfqpoint{1.130809in}{1.284403in}}%
\pgfpathlineto{\pgfqpoint{1.130380in}{1.279245in}}%
\pgfpathlineto{\pgfqpoint{1.142909in}{1.279919in}}%
\pgfpathlineto{\pgfqpoint{1.155473in}{1.280398in}}%
\pgfpathlineto{\pgfqpoint{1.168058in}{1.280681in}}%
\pgfpathlineto{\pgfqpoint{1.180655in}{1.280768in}}%
\pgfpathlineto{\pgfqpoint{1.180649in}{1.285911in}}%
\pgfpathlineto{\pgfqpoint{1.180643in}{1.291114in}}%
\pgfpathlineto{\pgfqpoint{1.180637in}{1.296371in}}%
\pgfpathlineto{\pgfqpoint{1.180630in}{1.301681in}}%
\pgfpathlineto{\pgfqpoint{1.168471in}{1.301598in}}%
\pgfpathlineto{\pgfqpoint{1.156322in}{1.301325in}}%
\pgfpathlineto{\pgfqpoint{1.144195in}{1.300865in}}%
\pgfpathlineto{\pgfqpoint{1.132101in}{1.300216in}}%
\pgfpathclose%
\pgfusepath{fill}%
\end{pgfscope}%
\begin{pgfscope}%
\pgfpathrectangle{\pgfqpoint{0.041670in}{0.041670in}}{\pgfqpoint{2.216660in}{2.216660in}}%
\pgfusepath{clip}%
\pgfsetbuttcap%
\pgfsetroundjoin%
\definecolor{currentfill}{rgb}{0.212395,0.359683,0.551710}%
\pgfsetfillcolor{currentfill}%
\pgfsetlinewidth{0.000000pt}%
\definecolor{currentstroke}{rgb}{0.000000,0.000000,0.000000}%
\pgfsetstrokecolor{currentstroke}%
\pgfsetdash{}{0pt}%
\pgfpathmoveto{\pgfqpoint{1.180630in}{1.301681in}}%
\pgfpathlineto{\pgfqpoint{1.180637in}{1.296371in}}%
\pgfpathlineto{\pgfqpoint{1.180643in}{1.291114in}}%
\pgfpathlineto{\pgfqpoint{1.180649in}{1.285911in}}%
\pgfpathlineto{\pgfqpoint{1.180655in}{1.280768in}}%
\pgfpathlineto{\pgfqpoint{1.193251in}{1.280659in}}%
\pgfpathlineto{\pgfqpoint{1.205834in}{1.280354in}}%
\pgfpathlineto{\pgfqpoint{1.218394in}{1.279854in}}%
\pgfpathlineto{\pgfqpoint{1.230920in}{1.279158in}}%
\pgfpathlineto{\pgfqpoint{1.230478in}{1.284317in}}%
\pgfpathlineto{\pgfqpoint{1.230036in}{1.289534in}}%
\pgfpathlineto{\pgfqpoint{1.229593in}{1.294807in}}%
\pgfpathlineto{\pgfqpoint{1.229150in}{1.300132in}}%
\pgfpathlineto{\pgfqpoint{1.217060in}{1.300802in}}%
\pgfpathlineto{\pgfqpoint{1.204936in}{1.301283in}}%
\pgfpathlineto{\pgfqpoint{1.192789in}{1.301577in}}%
\pgfpathlineto{\pgfqpoint{1.180630in}{1.301681in}}%
\pgfpathclose%
\pgfusepath{fill}%
\end{pgfscope}%
\begin{pgfscope}%
\pgfpathrectangle{\pgfqpoint{0.041670in}{0.041670in}}{\pgfqpoint{2.216660in}{2.216660in}}%
\pgfusepath{clip}%
\pgfsetbuttcap%
\pgfsetroundjoin%
\definecolor{currentfill}{rgb}{0.231674,0.318106,0.544834}%
\pgfsetfillcolor{currentfill}%
\pgfsetlinewidth{0.000000pt}%
\definecolor{currentstroke}{rgb}{0.000000,0.000000,0.000000}%
\pgfsetstrokecolor{currentstroke}%
\pgfsetdash{}{0pt}%
\pgfpathmoveto{\pgfqpoint{1.316660in}{1.268891in}}%
\pgfpathlineto{\pgfqpoint{1.317845in}{1.263697in}}%
\pgfpathlineto{\pgfqpoint{1.319030in}{1.258568in}}%
\pgfpathlineto{\pgfqpoint{1.320213in}{1.253507in}}%
\pgfpathlineto{\pgfqpoint{1.321396in}{1.248518in}}%
\pgfpathlineto{\pgfqpoint{1.333645in}{1.246211in}}%
\pgfpathlineto{\pgfqpoint{1.345753in}{1.243712in}}%
\pgfpathlineto{\pgfqpoint{1.357710in}{1.241025in}}%
\pgfpathlineto{\pgfqpoint{1.369504in}{1.238150in}}%
\pgfpathlineto{\pgfqpoint{1.367917in}{1.243234in}}%
\pgfpathlineto{\pgfqpoint{1.366329in}{1.248389in}}%
\pgfpathlineto{\pgfqpoint{1.364739in}{1.253613in}}%
\pgfpathlineto{\pgfqpoint{1.363148in}{1.258901in}}%
\pgfpathlineto{\pgfqpoint{1.351752in}{1.261671in}}%
\pgfpathlineto{\pgfqpoint{1.340198in}{1.264261in}}%
\pgfpathlineto{\pgfqpoint{1.328497in}{1.266668in}}%
\pgfpathlineto{\pgfqpoint{1.316660in}{1.268891in}}%
\pgfpathclose%
\pgfusepath{fill}%
\end{pgfscope}%
\begin{pgfscope}%
\pgfpathrectangle{\pgfqpoint{0.041670in}{0.041670in}}{\pgfqpoint{2.216660in}{2.216660in}}%
\pgfusepath{clip}%
\pgfsetbuttcap%
\pgfsetroundjoin%
\definecolor{currentfill}{rgb}{0.283072,0.130895,0.449241}%
\pgfsetfillcolor{currentfill}%
\pgfsetlinewidth{0.000000pt}%
\definecolor{currentstroke}{rgb}{0.000000,0.000000,0.000000}%
\pgfsetstrokecolor{currentstroke}%
\pgfsetdash{}{0pt}%
\pgfpathmoveto{\pgfqpoint{1.536729in}{1.114198in}}%
\pgfpathlineto{\pgfqpoint{1.539367in}{1.110411in}}%
\pgfpathlineto{\pgfqpoint{1.542004in}{1.106759in}}%
\pgfpathlineto{\pgfqpoint{1.544641in}{1.103244in}}%
\pgfpathlineto{\pgfqpoint{1.547279in}{1.099871in}}%
\pgfpathlineto{\pgfqpoint{1.557516in}{1.093921in}}%
\pgfpathlineto{\pgfqpoint{1.567403in}{1.087804in}}%
\pgfpathlineto{\pgfqpoint{1.576929in}{1.081526in}}%
\pgfpathlineto{\pgfqpoint{1.586085in}{1.075093in}}%
\pgfpathlineto{\pgfqpoint{1.583158in}{1.078657in}}%
\pgfpathlineto{\pgfqpoint{1.580232in}{1.082362in}}%
\pgfpathlineto{\pgfqpoint{1.577306in}{1.086205in}}%
\pgfpathlineto{\pgfqpoint{1.574381in}{1.090183in}}%
\pgfpathlineto{\pgfqpoint{1.565499in}{1.096418in}}%
\pgfpathlineto{\pgfqpoint{1.556257in}{1.102502in}}%
\pgfpathlineto{\pgfqpoint{1.546664in}{1.108430in}}%
\pgfpathlineto{\pgfqpoint{1.536729in}{1.114198in}}%
\pgfpathclose%
\pgfusepath{fill}%
\end{pgfscope}%
\begin{pgfscope}%
\pgfpathrectangle{\pgfqpoint{0.041670in}{0.041670in}}{\pgfqpoint{2.216660in}{2.216660in}}%
\pgfusepath{clip}%
\pgfsetbuttcap%
\pgfsetroundjoin%
\definecolor{currentfill}{rgb}{0.263663,0.237631,0.518762}%
\pgfsetfillcolor{currentfill}%
\pgfsetlinewidth{0.000000pt}%
\definecolor{currentstroke}{rgb}{0.000000,0.000000,0.000000}%
\pgfsetstrokecolor{currentstroke}%
\pgfsetdash{}{0pt}%
\pgfpathmoveto{\pgfqpoint{0.884846in}{1.184041in}}%
\pgfpathlineto{\pgfqpoint{0.882451in}{1.179059in}}%
\pgfpathlineto{\pgfqpoint{0.880057in}{1.174169in}}%
\pgfpathlineto{\pgfqpoint{0.877664in}{1.169373in}}%
\pgfpathlineto{\pgfqpoint{0.875272in}{1.164677in}}%
\pgfpathlineto{\pgfqpoint{0.885826in}{1.169412in}}%
\pgfpathlineto{\pgfqpoint{0.896651in}{1.173976in}}%
\pgfpathlineto{\pgfqpoint{0.907738in}{1.178364in}}%
\pgfpathlineto{\pgfqpoint{0.919076in}{1.182572in}}%
\pgfpathlineto{\pgfqpoint{0.921119in}{1.187117in}}%
\pgfpathlineto{\pgfqpoint{0.923162in}{1.191760in}}%
\pgfpathlineto{\pgfqpoint{0.925207in}{1.196497in}}%
\pgfpathlineto{\pgfqpoint{0.927253in}{1.201326in}}%
\pgfpathlineto{\pgfqpoint{0.916276in}{1.197262in}}%
\pgfpathlineto{\pgfqpoint{0.905542in}{1.193023in}}%
\pgfpathlineto{\pgfqpoint{0.895062in}{1.188615in}}%
\pgfpathlineto{\pgfqpoint{0.884846in}{1.184041in}}%
\pgfpathclose%
\pgfusepath{fill}%
\end{pgfscope}%
\begin{pgfscope}%
\pgfpathrectangle{\pgfqpoint{0.041670in}{0.041670in}}{\pgfqpoint{2.216660in}{2.216660in}}%
\pgfusepath{clip}%
\pgfsetbuttcap%
\pgfsetroundjoin%
\definecolor{currentfill}{rgb}{0.231674,0.318106,0.544834}%
\pgfsetfillcolor{currentfill}%
\pgfsetlinewidth{0.000000pt}%
\definecolor{currentstroke}{rgb}{0.000000,0.000000,0.000000}%
\pgfsetstrokecolor{currentstroke}%
\pgfsetdash{}{0pt}%
\pgfpathmoveto{\pgfqpoint{0.986772in}{1.256290in}}%
\pgfpathlineto{\pgfqpoint{0.985094in}{1.250977in}}%
\pgfpathlineto{\pgfqpoint{0.983417in}{1.245729in}}%
\pgfpathlineto{\pgfqpoint{0.981742in}{1.240548in}}%
\pgfpathlineto{\pgfqpoint{0.980067in}{1.235440in}}%
\pgfpathlineto{\pgfqpoint{0.991708in}{1.238479in}}%
\pgfpathlineto{\pgfqpoint{1.003520in}{1.241332in}}%
\pgfpathlineto{\pgfqpoint{1.015495in}{1.243999in}}%
\pgfpathlineto{\pgfqpoint{1.027619in}{1.246476in}}%
\pgfpathlineto{\pgfqpoint{1.028893in}{1.251484in}}%
\pgfpathlineto{\pgfqpoint{1.030168in}{1.256564in}}%
\pgfpathlineto{\pgfqpoint{1.031444in}{1.261712in}}%
\pgfpathlineto{\pgfqpoint{1.032721in}{1.266924in}}%
\pgfpathlineto{\pgfqpoint{1.021005in}{1.264537in}}%
\pgfpathlineto{\pgfqpoint{1.009434in}{1.261968in}}%
\pgfpathlineto{\pgfqpoint{0.998019in}{1.259218in}}%
\pgfpathlineto{\pgfqpoint{0.986772in}{1.256290in}}%
\pgfpathclose%
\pgfusepath{fill}%
\end{pgfscope}%
\begin{pgfscope}%
\pgfpathrectangle{\pgfqpoint{0.041670in}{0.041670in}}{\pgfqpoint{2.216660in}{2.216660in}}%
\pgfusepath{clip}%
\pgfsetbuttcap%
\pgfsetroundjoin%
\definecolor{currentfill}{rgb}{0.274128,0.199721,0.498911}%
\pgfsetfillcolor{currentfill}%
\pgfsetlinewidth{0.000000pt}%
\definecolor{currentstroke}{rgb}{0.000000,0.000000,0.000000}%
\pgfsetstrokecolor{currentstroke}%
\pgfsetdash{}{0pt}%
\pgfpathmoveto{\pgfqpoint{1.475270in}{1.168895in}}%
\pgfpathlineto{\pgfqpoint{1.477587in}{1.164336in}}%
\pgfpathlineto{\pgfqpoint{1.479902in}{1.159882in}}%
\pgfpathlineto{\pgfqpoint{1.482217in}{1.155537in}}%
\pgfpathlineto{\pgfqpoint{1.484531in}{1.151304in}}%
\pgfpathlineto{\pgfqpoint{1.495391in}{1.146387in}}%
\pgfpathlineto{\pgfqpoint{1.505959in}{1.141297in}}%
\pgfpathlineto{\pgfqpoint{1.516225in}{1.136038in}}%
\pgfpathlineto{\pgfqpoint{1.526179in}{1.130614in}}%
\pgfpathlineto{\pgfqpoint{1.523541in}{1.135017in}}%
\pgfpathlineto{\pgfqpoint{1.520902in}{1.139533in}}%
\pgfpathlineto{\pgfqpoint{1.518263in}{1.144157in}}%
\pgfpathlineto{\pgfqpoint{1.515622in}{1.148887in}}%
\pgfpathlineto{\pgfqpoint{1.505979in}{1.154132in}}%
\pgfpathlineto{\pgfqpoint{1.496033in}{1.159218in}}%
\pgfpathlineto{\pgfqpoint{1.485793in}{1.164140in}}%
\pgfpathlineto{\pgfqpoint{1.475270in}{1.168895in}}%
\pgfpathclose%
\pgfusepath{fill}%
\end{pgfscope}%
\begin{pgfscope}%
\pgfpathrectangle{\pgfqpoint{0.041670in}{0.041670in}}{\pgfqpoint{2.216660in}{2.216660in}}%
\pgfusepath{clip}%
\pgfsetbuttcap%
\pgfsetroundjoin%
\definecolor{currentfill}{rgb}{0.212395,0.359683,0.551710}%
\pgfsetfillcolor{currentfill}%
\pgfsetlinewidth{0.000000pt}%
\definecolor{currentstroke}{rgb}{0.000000,0.000000,0.000000}%
\pgfsetstrokecolor{currentstroke}%
\pgfsetdash{}{0pt}%
\pgfpathmoveto{\pgfqpoint{1.084270in}{1.295756in}}%
\pgfpathlineto{\pgfqpoint{1.083408in}{1.290388in}}%
\pgfpathlineto{\pgfqpoint{1.082547in}{1.285072in}}%
\pgfpathlineto{\pgfqpoint{1.081687in}{1.279812in}}%
\pgfpathlineto{\pgfqpoint{1.080827in}{1.274610in}}%
\pgfpathlineto{\pgfqpoint{1.093108in}{1.276058in}}%
\pgfpathlineto{\pgfqpoint{1.105468in}{1.277314in}}%
\pgfpathlineto{\pgfqpoint{1.117895in}{1.278376in}}%
\pgfpathlineto{\pgfqpoint{1.130380in}{1.279245in}}%
\pgfpathlineto{\pgfqpoint{1.130809in}{1.284403in}}%
\pgfpathlineto{\pgfqpoint{1.131239in}{1.289620in}}%
\pgfpathlineto{\pgfqpoint{1.131670in}{1.294892in}}%
\pgfpathlineto{\pgfqpoint{1.132101in}{1.300216in}}%
\pgfpathlineto{\pgfqpoint{1.120050in}{1.299380in}}%
\pgfpathlineto{\pgfqpoint{1.108054in}{1.298358in}}%
\pgfpathlineto{\pgfqpoint{1.096124in}{1.297149in}}%
\pgfpathlineto{\pgfqpoint{1.084270in}{1.295756in}}%
\pgfpathclose%
\pgfusepath{fill}%
\end{pgfscope}%
\begin{pgfscope}%
\pgfpathrectangle{\pgfqpoint{0.041670in}{0.041670in}}{\pgfqpoint{2.216660in}{2.216660in}}%
\pgfusepath{clip}%
\pgfsetbuttcap%
\pgfsetroundjoin%
\definecolor{currentfill}{rgb}{0.212395,0.359683,0.551710}%
\pgfsetfillcolor{currentfill}%
\pgfsetlinewidth{0.000000pt}%
\definecolor{currentstroke}{rgb}{0.000000,0.000000,0.000000}%
\pgfsetstrokecolor{currentstroke}%
\pgfsetdash{}{0pt}%
\pgfpathmoveto{\pgfqpoint{1.229150in}{1.300132in}}%
\pgfpathlineto{\pgfqpoint{1.229593in}{1.294807in}}%
\pgfpathlineto{\pgfqpoint{1.230036in}{1.289534in}}%
\pgfpathlineto{\pgfqpoint{1.230478in}{1.284317in}}%
\pgfpathlineto{\pgfqpoint{1.230920in}{1.279158in}}%
\pgfpathlineto{\pgfqpoint{1.243398in}{1.278268in}}%
\pgfpathlineto{\pgfqpoint{1.255819in}{1.277184in}}%
\pgfpathlineto{\pgfqpoint{1.268171in}{1.275906in}}%
\pgfpathlineto{\pgfqpoint{1.280442in}{1.274437in}}%
\pgfpathlineto{\pgfqpoint{1.279570in}{1.279641in}}%
\pgfpathlineto{\pgfqpoint{1.278698in}{1.284903in}}%
\pgfpathlineto{\pgfqpoint{1.277825in}{1.290220in}}%
\pgfpathlineto{\pgfqpoint{1.276951in}{1.295590in}}%
\pgfpathlineto{\pgfqpoint{1.265107in}{1.297003in}}%
\pgfpathlineto{\pgfqpoint{1.253185in}{1.298232in}}%
\pgfpathlineto{\pgfqpoint{1.241196in}{1.299276in}}%
\pgfpathlineto{\pgfqpoint{1.229150in}{1.300132in}}%
\pgfpathclose%
\pgfusepath{fill}%
\end{pgfscope}%
\begin{pgfscope}%
\pgfpathrectangle{\pgfqpoint{0.041670in}{0.041670in}}{\pgfqpoint{2.216660in}{2.216660in}}%
\pgfusepath{clip}%
\pgfsetbuttcap%
\pgfsetroundjoin%
\definecolor{currentfill}{rgb}{0.172719,0.448791,0.557885}%
\pgfsetfillcolor{currentfill}%
\pgfsetlinewidth{0.000000pt}%
\definecolor{currentstroke}{rgb}{0.000000,0.000000,0.000000}%
\pgfsetstrokecolor{currentstroke}%
\pgfsetdash{}{0pt}%
\pgfpathmoveto{\pgfqpoint{0.564946in}{1.334260in}}%
\pgfpathlineto{\pgfqpoint{0.561463in}{1.349896in}}%
\pgfpathlineto{\pgfqpoint{0.557962in}{1.366028in}}%
\pgfpathlineto{\pgfqpoint{0.554442in}{1.382663in}}%
\pgfpathlineto{\pgfqpoint{0.568059in}{1.392138in}}%
\pgfpathlineto{\pgfqpoint{0.582229in}{1.401388in}}%
\pgfpathlineto{\pgfqpoint{0.596936in}{1.410404in}}%
\pgfpathlineto{\pgfqpoint{0.612166in}{1.419180in}}%
\pgfpathlineto{\pgfqpoint{0.615344in}{1.402431in}}%
\pgfpathlineto{\pgfqpoint{0.618505in}{1.386182in}}%
\pgfpathlineto{\pgfqpoint{0.621650in}{1.370426in}}%
\pgfpathlineto{\pgfqpoint{0.606687in}{1.361734in}}%
\pgfpathlineto{\pgfqpoint{0.592239in}{1.352804in}}%
\pgfpathlineto{\pgfqpoint{0.578320in}{1.343644in}}%
\pgfpathlineto{\pgfqpoint{0.564946in}{1.334260in}}%
\pgfpathclose%
\pgfusepath{fill}%
\end{pgfscope}%
\begin{pgfscope}%
\pgfpathrectangle{\pgfqpoint{0.041670in}{0.041670in}}{\pgfqpoint{2.216660in}{2.216660in}}%
\pgfusepath{clip}%
\pgfsetbuttcap%
\pgfsetroundjoin%
\definecolor{currentfill}{rgb}{0.274952,0.037752,0.364543}%
\pgfsetfillcolor{currentfill}%
\pgfsetlinewidth{0.000000pt}%
\definecolor{currentstroke}{rgb}{0.000000,0.000000,0.000000}%
\pgfsetstrokecolor{currentstroke}%
\pgfsetdash{}{0pt}%
\pgfpathmoveto{\pgfqpoint{0.708871in}{1.016603in}}%
\pgfpathlineto{\pgfqpoint{0.705633in}{1.014236in}}%
\pgfpathlineto{\pgfqpoint{0.702392in}{1.012051in}}%
\pgfpathlineto{\pgfqpoint{0.699149in}{1.010052in}}%
\pgfpathlineto{\pgfqpoint{0.695903in}{1.008243in}}%
\pgfpathlineto{\pgfqpoint{0.703784in}{1.016007in}}%
\pgfpathlineto{\pgfqpoint{0.712113in}{1.023631in}}%
\pgfpathlineto{\pgfqpoint{0.720882in}{1.031110in}}%
\pgfpathlineto{\pgfqpoint{0.730081in}{1.038436in}}%
\pgfpathlineto{\pgfqpoint{0.733085in}{1.040036in}}%
\pgfpathlineto{\pgfqpoint{0.736087in}{1.041825in}}%
\pgfpathlineto{\pgfqpoint{0.739086in}{1.043800in}}%
\pgfpathlineto{\pgfqpoint{0.742084in}{1.045956in}}%
\pgfpathlineto{\pgfqpoint{0.733143in}{1.038833in}}%
\pgfpathlineto{\pgfqpoint{0.724621in}{1.031563in}}%
\pgfpathlineto{\pgfqpoint{0.716527in}{1.024150in}}%
\pgfpathlineto{\pgfqpoint{0.708871in}{1.016603in}}%
\pgfpathclose%
\pgfusepath{fill}%
\end{pgfscope}%
\begin{pgfscope}%
\pgfpathrectangle{\pgfqpoint{0.041670in}{0.041670in}}{\pgfqpoint{2.216660in}{2.216660in}}%
\pgfusepath{clip}%
\pgfsetbuttcap%
\pgfsetroundjoin%
\definecolor{currentfill}{rgb}{0.283072,0.130895,0.449241}%
\pgfsetfillcolor{currentfill}%
\pgfsetlinewidth{0.000000pt}%
\definecolor{currentstroke}{rgb}{0.000000,0.000000,0.000000}%
\pgfsetstrokecolor{currentstroke}%
\pgfsetdash{}{0pt}%
\pgfpathmoveto{\pgfqpoint{0.777945in}{1.084520in}}%
\pgfpathlineto{\pgfqpoint{0.774961in}{1.080497in}}%
\pgfpathlineto{\pgfqpoint{0.771976in}{1.076608in}}%
\pgfpathlineto{\pgfqpoint{0.768991in}{1.072858in}}%
\pgfpathlineto{\pgfqpoint{0.766006in}{1.069249in}}%
\pgfpathlineto{\pgfqpoint{0.774823in}{1.075815in}}%
\pgfpathlineto{\pgfqpoint{0.784021in}{1.082231in}}%
\pgfpathlineto{\pgfqpoint{0.793587in}{1.088491in}}%
\pgfpathlineto{\pgfqpoint{0.803514in}{1.094590in}}%
\pgfpathlineto{\pgfqpoint{0.806219in}{1.098003in}}%
\pgfpathlineto{\pgfqpoint{0.808924in}{1.101559in}}%
\pgfpathlineto{\pgfqpoint{0.811628in}{1.105252in}}%
\pgfpathlineto{\pgfqpoint{0.814333in}{1.109079in}}%
\pgfpathlineto{\pgfqpoint{0.804701in}{1.103168in}}%
\pgfpathlineto{\pgfqpoint{0.795420in}{1.097101in}}%
\pgfpathlineto{\pgfqpoint{0.786498in}{1.090883in}}%
\pgfpathlineto{\pgfqpoint{0.777945in}{1.084520in}}%
\pgfpathclose%
\pgfusepath{fill}%
\end{pgfscope}%
\begin{pgfscope}%
\pgfpathrectangle{\pgfqpoint{0.041670in}{0.041670in}}{\pgfqpoint{2.216660in}{2.216660in}}%
\pgfusepath{clip}%
\pgfsetbuttcap%
\pgfsetroundjoin%
\definecolor{currentfill}{rgb}{0.260571,0.246922,0.522828}%
\pgfsetfillcolor{currentfill}%
\pgfsetlinewidth{0.000000pt}%
\definecolor{currentstroke}{rgb}{0.000000,0.000000,0.000000}%
\pgfsetstrokecolor{currentstroke}%
\pgfsetdash{}{0pt}%
\pgfpathmoveto{\pgfqpoint{0.561111in}{1.144671in}}%
\pgfpathlineto{\pgfqpoint{0.557538in}{1.154788in}}%
\pgfpathlineto{\pgfqpoint{0.553950in}{1.165315in}}%
\pgfpathlineto{\pgfqpoint{0.550347in}{1.176258in}}%
\pgfpathlineto{\pgfqpoint{0.546730in}{1.187624in}}%
\pgfpathlineto{\pgfqpoint{0.557249in}{1.197521in}}%
\pgfpathlineto{\pgfqpoint{0.568345in}{1.207233in}}%
\pgfpathlineto{\pgfqpoint{0.580003in}{1.216753in}}%
\pgfpathlineto{\pgfqpoint{0.592212in}{1.226072in}}%
\pgfpathlineto{\pgfqpoint{0.595551in}{1.214550in}}%
\pgfpathlineto{\pgfqpoint{0.598877in}{1.203449in}}%
\pgfpathlineto{\pgfqpoint{0.602189in}{1.192763in}}%
\pgfpathlineto{\pgfqpoint{0.605487in}{1.182484in}}%
\pgfpathlineto{\pgfqpoint{0.593572in}{1.173317in}}%
\pgfpathlineto{\pgfqpoint{0.582196in}{1.163954in}}%
\pgfpathlineto{\pgfqpoint{0.571371in}{1.154403in}}%
\pgfpathlineto{\pgfqpoint{0.561111in}{1.144671in}}%
\pgfpathclose%
\pgfusepath{fill}%
\end{pgfscope}%
\begin{pgfscope}%
\pgfpathrectangle{\pgfqpoint{0.041670in}{0.041670in}}{\pgfqpoint{2.216660in}{2.216660in}}%
\pgfusepath{clip}%
\pgfsetbuttcap%
\pgfsetroundjoin%
\definecolor{currentfill}{rgb}{0.274128,0.199721,0.498911}%
\pgfsetfillcolor{currentfill}%
\pgfsetlinewidth{0.000000pt}%
\definecolor{currentstroke}{rgb}{0.000000,0.000000,0.000000}%
\pgfsetstrokecolor{currentstroke}%
\pgfsetdash{}{0pt}%
\pgfpathmoveto{\pgfqpoint{0.835979in}{1.144095in}}%
\pgfpathlineto{\pgfqpoint{0.833271in}{1.139324in}}%
\pgfpathlineto{\pgfqpoint{0.830564in}{1.134659in}}%
\pgfpathlineto{\pgfqpoint{0.827858in}{1.130102in}}%
\pgfpathlineto{\pgfqpoint{0.825152in}{1.125658in}}%
\pgfpathlineto{\pgfqpoint{0.834821in}{1.131225in}}%
\pgfpathlineto{\pgfqpoint{0.844810in}{1.136631in}}%
\pgfpathlineto{\pgfqpoint{0.855110in}{1.141871in}}%
\pgfpathlineto{\pgfqpoint{0.865712in}{1.146942in}}%
\pgfpathlineto{\pgfqpoint{0.868101in}{1.151211in}}%
\pgfpathlineto{\pgfqpoint{0.870490in}{1.155592in}}%
\pgfpathlineto{\pgfqpoint{0.872881in}{1.160082in}}%
\pgfpathlineto{\pgfqpoint{0.875272in}{1.164677in}}%
\pgfpathlineto{\pgfqpoint{0.865000in}{1.159773in}}%
\pgfpathlineto{\pgfqpoint{0.855021in}{1.154705in}}%
\pgfpathlineto{\pgfqpoint{0.845344in}{1.149477in}}%
\pgfpathlineto{\pgfqpoint{0.835979in}{1.144095in}}%
\pgfpathclose%
\pgfusepath{fill}%
\end{pgfscope}%
\begin{pgfscope}%
\pgfpathrectangle{\pgfqpoint{0.041670in}{0.041670in}}{\pgfqpoint{2.216660in}{2.216660in}}%
\pgfusepath{clip}%
\pgfsetbuttcap%
\pgfsetroundjoin%
\definecolor{currentfill}{rgb}{0.279566,0.067836,0.391917}%
\pgfsetfillcolor{currentfill}%
\pgfsetlinewidth{0.000000pt}%
\definecolor{currentstroke}{rgb}{0.000000,0.000000,0.000000}%
\pgfsetstrokecolor{currentstroke}%
\pgfsetdash{}{0pt}%
\pgfpathmoveto{\pgfqpoint{1.597800in}{1.062336in}}%
\pgfpathlineto{\pgfqpoint{1.600731in}{1.059539in}}%
\pgfpathlineto{\pgfqpoint{1.603664in}{1.056908in}}%
\pgfpathlineto{\pgfqpoint{1.606598in}{1.054447in}}%
\pgfpathlineto{\pgfqpoint{1.609534in}{1.052158in}}%
\pgfpathlineto{\pgfqpoint{1.618840in}{1.045172in}}%
\pgfpathlineto{\pgfqpoint{1.627735in}{1.038032in}}%
\pgfpathlineto{\pgfqpoint{1.636210in}{1.030746in}}%
\pgfpathlineto{\pgfqpoint{1.644255in}{1.023318in}}%
\pgfpathlineto{\pgfqpoint{1.641069in}{1.025814in}}%
\pgfpathlineto{\pgfqpoint{1.637885in}{1.028484in}}%
\pgfpathlineto{\pgfqpoint{1.634702in}{1.031323in}}%
\pgfpathlineto{\pgfqpoint{1.631521in}{1.034328in}}%
\pgfpathlineto{\pgfqpoint{1.623710in}{1.041540in}}%
\pgfpathlineto{\pgfqpoint{1.615479in}{1.048617in}}%
\pgfpathlineto{\pgfqpoint{1.606840in}{1.055551in}}%
\pgfpathlineto{\pgfqpoint{1.597800in}{1.062336in}}%
\pgfpathclose%
\pgfusepath{fill}%
\end{pgfscope}%
\begin{pgfscope}%
\pgfpathrectangle{\pgfqpoint{0.041670in}{0.041670in}}{\pgfqpoint{2.216660in}{2.216660in}}%
\pgfusepath{clip}%
\pgfsetbuttcap%
\pgfsetroundjoin%
\definecolor{currentfill}{rgb}{0.212395,0.359683,0.551710}%
\pgfsetfillcolor{currentfill}%
\pgfsetlinewidth{0.000000pt}%
\definecolor{currentstroke}{rgb}{0.000000,0.000000,0.000000}%
\pgfsetstrokecolor{currentstroke}%
\pgfsetdash{}{0pt}%
\pgfpathmoveto{\pgfqpoint{1.276951in}{1.295590in}}%
\pgfpathlineto{\pgfqpoint{1.277825in}{1.290220in}}%
\pgfpathlineto{\pgfqpoint{1.278698in}{1.284903in}}%
\pgfpathlineto{\pgfqpoint{1.279570in}{1.279641in}}%
\pgfpathlineto{\pgfqpoint{1.280442in}{1.274437in}}%
\pgfpathlineto{\pgfqpoint{1.292621in}{1.272778in}}%
\pgfpathlineto{\pgfqpoint{1.304698in}{1.270928in}}%
\pgfpathlineto{\pgfqpoint{1.316660in}{1.268891in}}%
\pgfpathlineto{\pgfqpoint{1.315474in}{1.274147in}}%
\pgfpathlineto{\pgfqpoint{1.314286in}{1.279461in}}%
\pgfpathlineto{\pgfqpoint{1.313098in}{1.284831in}}%
\pgfpathlineto{\pgfqpoint{1.311908in}{1.290253in}}%
\pgfpathlineto{\pgfqpoint{1.300363in}{1.292213in}}%
\pgfpathlineto{\pgfqpoint{1.288707in}{1.293992in}}%
\pgfpathlineto{\pgfqpoint{1.276951in}{1.295590in}}%
\pgfpathclose%
\pgfusepath{fill}%
\end{pgfscope}%
\begin{pgfscope}%
\pgfpathrectangle{\pgfqpoint{0.041670in}{0.041670in}}{\pgfqpoint{2.216660in}{2.216660in}}%
\pgfusepath{clip}%
\pgfsetbuttcap%
\pgfsetroundjoin%
\definecolor{currentfill}{rgb}{0.233603,0.313828,0.543914}%
\pgfsetfillcolor{currentfill}%
\pgfsetlinewidth{0.000000pt}%
\definecolor{currentstroke}{rgb}{0.000000,0.000000,0.000000}%
\pgfsetstrokecolor{currentstroke}%
\pgfsetdash{}{0pt}%
\pgfpathmoveto{\pgfqpoint{1.756394in}{1.234182in}}%
\pgfpathlineto{\pgfqpoint{1.759680in}{1.246164in}}%
\pgfpathlineto{\pgfqpoint{1.762980in}{1.258581in}}%
\pgfpathlineto{\pgfqpoint{1.766294in}{1.271439in}}%
\pgfpathlineto{\pgfqpoint{1.769624in}{1.284746in}}%
\pgfpathlineto{\pgfqpoint{1.782620in}{1.275470in}}%
\pgfpathlineto{\pgfqpoint{1.795067in}{1.265983in}}%
\pgfpathlineto{\pgfqpoint{1.806951in}{1.256293in}}%
\pgfpathlineto{\pgfqpoint{1.818258in}{1.246407in}}%
\pgfpathlineto{\pgfqpoint{1.814633in}{1.233243in}}%
\pgfpathlineto{\pgfqpoint{1.811025in}{1.220531in}}%
\pgfpathlineto{\pgfqpoint{1.807434in}{1.208262in}}%
\pgfpathlineto{\pgfqpoint{1.803858in}{1.196430in}}%
\pgfpathlineto{\pgfqpoint{1.792826in}{1.206163in}}%
\pgfpathlineto{\pgfqpoint{1.781229in}{1.215705in}}%
\pgfpathlineto{\pgfqpoint{1.769081in}{1.225047in}}%
\pgfpathlineto{\pgfqpoint{1.756394in}{1.234182in}}%
\pgfpathclose%
\pgfusepath{fill}%
\end{pgfscope}%
\begin{pgfscope}%
\pgfpathrectangle{\pgfqpoint{0.041670in}{0.041670in}}{\pgfqpoint{2.216660in}{2.216660in}}%
\pgfusepath{clip}%
\pgfsetbuttcap%
\pgfsetroundjoin%
\definecolor{currentfill}{rgb}{0.212395,0.359683,0.551710}%
\pgfsetfillcolor{currentfill}%
\pgfsetlinewidth{0.000000pt}%
\definecolor{currentstroke}{rgb}{0.000000,0.000000,0.000000}%
\pgfsetstrokecolor{currentstroke}%
\pgfsetdash{}{0pt}%
\pgfpathmoveto{\pgfqpoint{1.037841in}{1.288360in}}%
\pgfpathlineto{\pgfqpoint{1.036559in}{1.282920in}}%
\pgfpathlineto{\pgfqpoint{1.035279in}{1.277531in}}%
\pgfpathlineto{\pgfqpoint{1.033999in}{1.272199in}}%
\pgfpathlineto{\pgfqpoint{1.032721in}{1.266924in}}%
\pgfpathlineto{\pgfqpoint{1.044573in}{1.269127in}}%
\pgfpathlineto{\pgfqpoint{1.056549in}{1.271143in}}%
\pgfpathlineto{\pgfqpoint{1.068637in}{1.272971in}}%
\pgfpathlineto{\pgfqpoint{1.080827in}{1.274610in}}%
\pgfpathlineto{\pgfqpoint{1.081687in}{1.279812in}}%
\pgfpathlineto{\pgfqpoint{1.082547in}{1.285072in}}%
\pgfpathlineto{\pgfqpoint{1.083408in}{1.290388in}}%
\pgfpathlineto{\pgfqpoint{1.084270in}{1.295756in}}%
\pgfpathlineto{\pgfqpoint{1.072504in}{1.294179in}}%
\pgfpathlineto{\pgfqpoint{1.060837in}{1.292420in}}%
\pgfpathlineto{\pgfqpoint{1.049279in}{1.290480in}}%
\pgfpathlineto{\pgfqpoint{1.037841in}{1.288360in}}%
\pgfpathclose%
\pgfusepath{fill}%
\end{pgfscope}%
\begin{pgfscope}%
\pgfpathrectangle{\pgfqpoint{0.041670in}{0.041670in}}{\pgfqpoint{2.216660in}{2.216660in}}%
\pgfusepath{clip}%
\pgfsetbuttcap%
\pgfsetroundjoin%
\definecolor{currentfill}{rgb}{0.268510,0.009605,0.335427}%
\pgfsetfillcolor{currentfill}%
\pgfsetlinewidth{0.000000pt}%
\definecolor{currentstroke}{rgb}{0.000000,0.000000,0.000000}%
\pgfsetstrokecolor{currentstroke}%
\pgfsetdash{}{0pt}%
\pgfpathmoveto{\pgfqpoint{1.708700in}{1.016336in}}%
\pgfpathlineto{\pgfqpoint{1.711974in}{1.018476in}}%
\pgfpathlineto{\pgfqpoint{1.715255in}{1.020887in}}%
\pgfpathlineto{\pgfqpoint{1.718543in}{1.023575in}}%
\pgfpathlineto{\pgfqpoint{1.721839in}{1.026544in}}%
\pgfpathlineto{\pgfqpoint{1.730815in}{1.017721in}}%
\pgfpathlineto{\pgfqpoint{1.739274in}{1.008746in}}%
\pgfpathlineto{\pgfqpoint{1.747206in}{0.999627in}}%
\pgfpathlineto{\pgfqpoint{1.754602in}{0.990372in}}%
\pgfpathlineto{\pgfqpoint{1.751090in}{0.987608in}}%
\pgfpathlineto{\pgfqpoint{1.747585in}{0.985128in}}%
\pgfpathlineto{\pgfqpoint{1.744088in}{0.982924in}}%
\pgfpathlineto{\pgfqpoint{1.740599in}{0.980994in}}%
\pgfpathlineto{\pgfqpoint{1.733401in}{0.990036in}}%
\pgfpathlineto{\pgfqpoint{1.725679in}{0.998945in}}%
\pgfpathlineto{\pgfqpoint{1.717442in}{1.007714in}}%
\pgfpathlineto{\pgfqpoint{1.708700in}{1.016336in}}%
\pgfpathclose%
\pgfusepath{fill}%
\end{pgfscope}%
\begin{pgfscope}%
\pgfpathrectangle{\pgfqpoint{0.041670in}{0.041670in}}{\pgfqpoint{2.216660in}{2.216660in}}%
\pgfusepath{clip}%
\pgfsetbuttcap%
\pgfsetroundjoin%
\definecolor{currentfill}{rgb}{0.272594,0.025563,0.353093}%
\pgfsetfillcolor{currentfill}%
\pgfsetlinewidth{0.000000pt}%
\definecolor{currentstroke}{rgb}{0.000000,0.000000,0.000000}%
\pgfsetstrokecolor{currentstroke}%
\pgfsetdash{}{0pt}%
\pgfpathmoveto{\pgfqpoint{1.721839in}{1.026544in}}%
\pgfpathlineto{\pgfqpoint{1.725142in}{1.029800in}}%
\pgfpathlineto{\pgfqpoint{1.728452in}{1.033348in}}%
\pgfpathlineto{\pgfqpoint{1.731771in}{1.037192in}}%
\pgfpathlineto{\pgfqpoint{1.735098in}{1.041339in}}%
\pgfpathlineto{\pgfqpoint{1.744311in}{1.032319in}}%
\pgfpathlineto{\pgfqpoint{1.752995in}{1.023142in}}%
\pgfpathlineto{\pgfqpoint{1.761140in}{1.013818in}}%
\pgfpathlineto{\pgfqpoint{1.768736in}{1.004354in}}%
\pgfpathlineto{\pgfqpoint{1.765189in}{1.000409in}}%
\pgfpathlineto{\pgfqpoint{1.761651in}{0.996767in}}%
\pgfpathlineto{\pgfqpoint{1.758122in}{0.993423in}}%
\pgfpathlineto{\pgfqpoint{1.754602in}{0.990372in}}%
\pgfpathlineto{\pgfqpoint{1.747206in}{0.999627in}}%
\pgfpathlineto{\pgfqpoint{1.739274in}{1.008746in}}%
\pgfpathlineto{\pgfqpoint{1.730815in}{1.017721in}}%
\pgfpathlineto{\pgfqpoint{1.721839in}{1.026544in}}%
\pgfpathclose%
\pgfusepath{fill}%
\end{pgfscope}%
\begin{pgfscope}%
\pgfpathrectangle{\pgfqpoint{0.041670in}{0.041670in}}{\pgfqpoint{2.216660in}{2.216660in}}%
\pgfusepath{clip}%
\pgfsetbuttcap%
\pgfsetroundjoin%
\definecolor{currentfill}{rgb}{0.267004,0.004874,0.329415}%
\pgfsetfillcolor{currentfill}%
\pgfsetlinewidth{0.000000pt}%
\definecolor{currentstroke}{rgb}{0.000000,0.000000,0.000000}%
\pgfsetstrokecolor{currentstroke}%
\pgfsetdash{}{0pt}%
\pgfpathmoveto{\pgfqpoint{1.695664in}{1.010393in}}%
\pgfpathlineto{\pgfqpoint{1.698914in}{1.011496in}}%
\pgfpathlineto{\pgfqpoint{1.702170in}{1.012851in}}%
\pgfpathlineto{\pgfqpoint{1.705432in}{1.014462in}}%
\pgfpathlineto{\pgfqpoint{1.708700in}{1.016336in}}%
\pgfpathlineto{\pgfqpoint{1.717442in}{1.007714in}}%
\pgfpathlineto{\pgfqpoint{1.725679in}{0.998945in}}%
\pgfpathlineto{\pgfqpoint{1.733401in}{0.990036in}}%
\pgfpathlineto{\pgfqpoint{1.740599in}{0.980994in}}%
\pgfpathlineto{\pgfqpoint{1.737116in}{0.979331in}}%
\pgfpathlineto{\pgfqpoint{1.733641in}{0.977930in}}%
\pgfpathlineto{\pgfqpoint{1.730172in}{0.976787in}}%
\pgfpathlineto{\pgfqpoint{1.726709in}{0.975898in}}%
\pgfpathlineto{\pgfqpoint{1.719707in}{0.984722in}}%
\pgfpathlineto{\pgfqpoint{1.712192in}{0.993418in}}%
\pgfpathlineto{\pgfqpoint{1.704175in}{1.001978in}}%
\pgfpathlineto{\pgfqpoint{1.695664in}{1.010393in}}%
\pgfpathclose%
\pgfusepath{fill}%
\end{pgfscope}%
\begin{pgfscope}%
\pgfpathrectangle{\pgfqpoint{0.041670in}{0.041670in}}{\pgfqpoint{2.216660in}{2.216660in}}%
\pgfusepath{clip}%
\pgfsetbuttcap%
\pgfsetroundjoin%
\definecolor{currentfill}{rgb}{0.277941,0.056324,0.381191}%
\pgfsetfillcolor{currentfill}%
\pgfsetlinewidth{0.000000pt}%
\definecolor{currentstroke}{rgb}{0.000000,0.000000,0.000000}%
\pgfsetstrokecolor{currentstroke}%
\pgfsetdash{}{0pt}%
\pgfpathmoveto{\pgfqpoint{1.735098in}{1.041339in}}%
\pgfpathlineto{\pgfqpoint{1.738433in}{1.045793in}}%
\pgfpathlineto{\pgfqpoint{1.741778in}{1.050560in}}%
\pgfpathlineto{\pgfqpoint{1.745131in}{1.055645in}}%
\pgfpathlineto{\pgfqpoint{1.748494in}{1.061055in}}%
\pgfpathlineto{\pgfqpoint{1.757947in}{1.051842in}}%
\pgfpathlineto{\pgfqpoint{1.766860in}{1.042470in}}%
\pgfpathlineto{\pgfqpoint{1.775221in}{1.032946in}}%
\pgfpathlineto{\pgfqpoint{1.783021in}{1.023279in}}%
\pgfpathlineto{\pgfqpoint{1.779434in}{1.018065in}}%
\pgfpathlineto{\pgfqpoint{1.775858in}{1.013177in}}%
\pgfpathlineto{\pgfqpoint{1.772292in}{1.008609in}}%
\pgfpathlineto{\pgfqpoint{1.768736in}{1.004354in}}%
\pgfpathlineto{\pgfqpoint{1.761140in}{1.013818in}}%
\pgfpathlineto{\pgfqpoint{1.752995in}{1.023142in}}%
\pgfpathlineto{\pgfqpoint{1.744311in}{1.032319in}}%
\pgfpathlineto{\pgfqpoint{1.735098in}{1.041339in}}%
\pgfpathclose%
\pgfusepath{fill}%
\end{pgfscope}%
\begin{pgfscope}%
\pgfpathrectangle{\pgfqpoint{0.041670in}{0.041670in}}{\pgfqpoint{2.216660in}{2.216660in}}%
\pgfusepath{clip}%
\pgfsetbuttcap%
\pgfsetroundjoin%
\definecolor{currentfill}{rgb}{0.248629,0.278775,0.534556}%
\pgfsetfillcolor{currentfill}%
\pgfsetlinewidth{0.000000pt}%
\definecolor{currentstroke}{rgb}{0.000000,0.000000,0.000000}%
\pgfsetstrokecolor{currentstroke}%
\pgfsetdash{}{0pt}%
\pgfpathmoveto{\pgfqpoint{1.414835in}{1.224833in}}%
\pgfpathlineto{\pgfqpoint{1.416804in}{1.219702in}}%
\pgfpathlineto{\pgfqpoint{1.418771in}{1.214649in}}%
\pgfpathlineto{\pgfqpoint{1.420738in}{1.209678in}}%
\pgfpathlineto{\pgfqpoint{1.422703in}{1.204791in}}%
\pgfpathlineto{\pgfqpoint{1.433888in}{1.200883in}}%
\pgfpathlineto{\pgfqpoint{1.444838in}{1.196799in}}%
\pgfpathlineto{\pgfqpoint{1.455544in}{1.192542in}}%
\pgfpathlineto{\pgfqpoint{1.465996in}{1.188115in}}%
\pgfpathlineto{\pgfqpoint{1.463674in}{1.193149in}}%
\pgfpathlineto{\pgfqpoint{1.461351in}{1.198267in}}%
\pgfpathlineto{\pgfqpoint{1.459027in}{1.203467in}}%
\pgfpathlineto{\pgfqpoint{1.456702in}{1.208745in}}%
\pgfpathlineto{\pgfqpoint{1.446596in}{1.213015in}}%
\pgfpathlineto{\pgfqpoint{1.436242in}{1.217122in}}%
\pgfpathlineto{\pgfqpoint{1.425652in}{1.221063in}}%
\pgfpathlineto{\pgfqpoint{1.414835in}{1.224833in}}%
\pgfpathclose%
\pgfusepath{fill}%
\end{pgfscope}%
\begin{pgfscope}%
\pgfpathrectangle{\pgfqpoint{0.041670in}{0.041670in}}{\pgfqpoint{2.216660in}{2.216660in}}%
\pgfusepath{clip}%
\pgfsetbuttcap%
\pgfsetroundjoin%
\definecolor{currentfill}{rgb}{0.231674,0.318106,0.544834}%
\pgfsetfillcolor{currentfill}%
\pgfsetlinewidth{0.000000pt}%
\definecolor{currentstroke}{rgb}{0.000000,0.000000,0.000000}%
\pgfsetstrokecolor{currentstroke}%
\pgfsetdash{}{0pt}%
\pgfpathmoveto{\pgfqpoint{1.363148in}{1.258901in}}%
\pgfpathlineto{\pgfqpoint{1.364739in}{1.253613in}}%
\pgfpathlineto{\pgfqpoint{1.366329in}{1.248389in}}%
\pgfpathlineto{\pgfqpoint{1.367917in}{1.243234in}}%
\pgfpathlineto{\pgfqpoint{1.369504in}{1.238150in}}%
\pgfpathlineto{\pgfqpoint{1.381125in}{1.235091in}}%
\pgfpathlineto{\pgfqpoint{1.392560in}{1.231850in}}%
\pgfpathlineto{\pgfqpoint{1.403800in}{1.228430in}}%
\pgfpathlineto{\pgfqpoint{1.414835in}{1.224833in}}%
\pgfpathlineto{\pgfqpoint{1.412864in}{1.230038in}}%
\pgfpathlineto{\pgfqpoint{1.410892in}{1.235315in}}%
\pgfpathlineto{\pgfqpoint{1.408919in}{1.240660in}}%
\pgfpathlineto{\pgfqpoint{1.406944in}{1.246070in}}%
\pgfpathlineto{\pgfqpoint{1.396284in}{1.249535in}}%
\pgfpathlineto{\pgfqpoint{1.385425in}{1.252831in}}%
\pgfpathlineto{\pgfqpoint{1.374376in}{1.255954in}}%
\pgfpathlineto{\pgfqpoint{1.363148in}{1.258901in}}%
\pgfpathclose%
\pgfusepath{fill}%
\end{pgfscope}%
\begin{pgfscope}%
\pgfpathrectangle{\pgfqpoint{0.041670in}{0.041670in}}{\pgfqpoint{2.216660in}{2.216660in}}%
\pgfusepath{clip}%
\pgfsetbuttcap%
\pgfsetroundjoin%
\definecolor{currentfill}{rgb}{0.280255,0.165693,0.476498}%
\pgfsetfillcolor{currentfill}%
\pgfsetlinewidth{0.000000pt}%
\definecolor{currentstroke}{rgb}{0.000000,0.000000,0.000000}%
\pgfsetstrokecolor{currentstroke}%
\pgfsetdash{}{0pt}%
\pgfpathmoveto{\pgfqpoint{1.526179in}{1.130614in}}%
\pgfpathlineto{\pgfqpoint{1.528817in}{1.126327in}}%
\pgfpathlineto{\pgfqpoint{1.531455in}{1.122160in}}%
\pgfpathlineto{\pgfqpoint{1.534092in}{1.118115in}}%
\pgfpathlineto{\pgfqpoint{1.536729in}{1.114198in}}%
\pgfpathlineto{\pgfqpoint{1.546664in}{1.108430in}}%
\pgfpathlineto{\pgfqpoint{1.556257in}{1.102502in}}%
\pgfpathlineto{\pgfqpoint{1.565499in}{1.096418in}}%
\pgfpathlineto{\pgfqpoint{1.574381in}{1.090183in}}%
\pgfpathlineto{\pgfqpoint{1.571455in}{1.094292in}}%
\pgfpathlineto{\pgfqpoint{1.568530in}{1.098528in}}%
\pgfpathlineto{\pgfqpoint{1.565604in}{1.102887in}}%
\pgfpathlineto{\pgfqpoint{1.562679in}{1.107367in}}%
\pgfpathlineto{\pgfqpoint{1.554071in}{1.113401in}}%
\pgfpathlineto{\pgfqpoint{1.545112in}{1.119291in}}%
\pgfpathlineto{\pgfqpoint{1.535812in}{1.125030in}}%
\pgfpathlineto{\pgfqpoint{1.526179in}{1.130614in}}%
\pgfpathclose%
\pgfusepath{fill}%
\end{pgfscope}%
\begin{pgfscope}%
\pgfpathrectangle{\pgfqpoint{0.041670in}{0.041670in}}{\pgfqpoint{2.216660in}{2.216660in}}%
\pgfusepath{clip}%
\pgfsetbuttcap%
\pgfsetroundjoin%
\definecolor{currentfill}{rgb}{0.267004,0.004874,0.329415}%
\pgfsetfillcolor{currentfill}%
\pgfsetlinewidth{0.000000pt}%
\definecolor{currentstroke}{rgb}{0.000000,0.000000,0.000000}%
\pgfsetstrokecolor{currentstroke}%
\pgfsetdash{}{0pt}%
\pgfpathmoveto{\pgfqpoint{1.682715in}{1.008409in}}%
\pgfpathlineto{\pgfqpoint{1.685945in}{1.008550in}}%
\pgfpathlineto{\pgfqpoint{1.689179in}{1.008925in}}%
\pgfpathlineto{\pgfqpoint{1.692419in}{1.009538in}}%
\pgfpathlineto{\pgfqpoint{1.695664in}{1.010393in}}%
\pgfpathlineto{\pgfqpoint{1.704175in}{1.001978in}}%
\pgfpathlineto{\pgfqpoint{1.712192in}{0.993418in}}%
\pgfpathlineto{\pgfqpoint{1.719707in}{0.984722in}}%
\pgfpathlineto{\pgfqpoint{1.726709in}{0.975898in}}%
\pgfpathlineto{\pgfqpoint{1.723252in}{0.975256in}}%
\pgfpathlineto{\pgfqpoint{1.719801in}{0.974859in}}%
\pgfpathlineto{\pgfqpoint{1.716356in}{0.974700in}}%
\pgfpathlineto{\pgfqpoint{1.712916in}{0.974775in}}%
\pgfpathlineto{\pgfqpoint{1.706107in}{0.983378in}}%
\pgfpathlineto{\pgfqpoint{1.698797in}{0.991857in}}%
\pgfpathlineto{\pgfqpoint{1.690997in}{1.000203in}}%
\pgfpathlineto{\pgfqpoint{1.682715in}{1.008409in}}%
\pgfpathclose%
\pgfusepath{fill}%
\end{pgfscope}%
\begin{pgfscope}%
\pgfpathrectangle{\pgfqpoint{0.041670in}{0.041670in}}{\pgfqpoint{2.216660in}{2.216660in}}%
\pgfusepath{clip}%
\pgfsetbuttcap%
\pgfsetroundjoin%
\definecolor{currentfill}{rgb}{0.279566,0.067836,0.391917}%
\pgfsetfillcolor{currentfill}%
\pgfsetlinewidth{0.000000pt}%
\definecolor{currentstroke}{rgb}{0.000000,0.000000,0.000000}%
\pgfsetstrokecolor{currentstroke}%
\pgfsetdash{}{0pt}%
\pgfpathmoveto{\pgfqpoint{0.721803in}{1.027807in}}%
\pgfpathlineto{\pgfqpoint{0.718573in}{1.024753in}}%
\pgfpathlineto{\pgfqpoint{0.715341in}{1.021866in}}%
\pgfpathlineto{\pgfqpoint{0.712107in}{1.019148in}}%
\pgfpathlineto{\pgfqpoint{0.708871in}{1.016603in}}%
\pgfpathlineto{\pgfqpoint{0.716527in}{1.024150in}}%
\pgfpathlineto{\pgfqpoint{0.724621in}{1.031563in}}%
\pgfpathlineto{\pgfqpoint{0.733143in}{1.038833in}}%
\pgfpathlineto{\pgfqpoint{0.742084in}{1.045956in}}%
\pgfpathlineto{\pgfqpoint{0.745079in}{1.048289in}}%
\pgfpathlineto{\pgfqpoint{0.748073in}{1.050795in}}%
\pgfpathlineto{\pgfqpoint{0.751065in}{1.053471in}}%
\pgfpathlineto{\pgfqpoint{0.754055in}{1.056312in}}%
\pgfpathlineto{\pgfqpoint{0.745370in}{1.049394in}}%
\pgfpathlineto{\pgfqpoint{0.737094in}{1.042334in}}%
\pgfpathlineto{\pgfqpoint{0.729235in}{1.035136in}}%
\pgfpathlineto{\pgfqpoint{0.721803in}{1.027807in}}%
\pgfpathclose%
\pgfusepath{fill}%
\end{pgfscope}%
\begin{pgfscope}%
\pgfpathrectangle{\pgfqpoint{0.041670in}{0.041670in}}{\pgfqpoint{2.216660in}{2.216660in}}%
\pgfusepath{clip}%
\pgfsetbuttcap%
\pgfsetroundjoin%
\definecolor{currentfill}{rgb}{0.282327,0.094955,0.417331}%
\pgfsetfillcolor{currentfill}%
\pgfsetlinewidth{0.000000pt}%
\definecolor{currentstroke}{rgb}{0.000000,0.000000,0.000000}%
\pgfsetstrokecolor{currentstroke}%
\pgfsetdash{}{0pt}%
\pgfpathmoveto{\pgfqpoint{1.748494in}{1.061055in}}%
\pgfpathlineto{\pgfqpoint{1.751867in}{1.066793in}}%
\pgfpathlineto{\pgfqpoint{1.755250in}{1.072867in}}%
\pgfpathlineto{\pgfqpoint{1.758643in}{1.079281in}}%
\pgfpathlineto{\pgfqpoint{1.762047in}{1.086042in}}%
\pgfpathlineto{\pgfqpoint{1.771744in}{1.076644in}}%
\pgfpathlineto{\pgfqpoint{1.780889in}{1.067081in}}%
\pgfpathlineto{\pgfqpoint{1.789469in}{1.057363in}}%
\pgfpathlineto{\pgfqpoint{1.797476in}{1.047498in}}%
\pgfpathlineto{\pgfqpoint{1.793845in}{1.040927in}}%
\pgfpathlineto{\pgfqpoint{1.790226in}{1.034704in}}%
\pgfpathlineto{\pgfqpoint{1.786618in}{1.028823in}}%
\pgfpathlineto{\pgfqpoint{1.783021in}{1.023279in}}%
\pgfpathlineto{\pgfqpoint{1.775221in}{1.032946in}}%
\pgfpathlineto{\pgfqpoint{1.766860in}{1.042470in}}%
\pgfpathlineto{\pgfqpoint{1.757947in}{1.051842in}}%
\pgfpathlineto{\pgfqpoint{1.748494in}{1.061055in}}%
\pgfpathclose%
\pgfusepath{fill}%
\end{pgfscope}%
\begin{pgfscope}%
\pgfpathrectangle{\pgfqpoint{0.041670in}{0.041670in}}{\pgfqpoint{2.216660in}{2.216660in}}%
\pgfusepath{clip}%
\pgfsetbuttcap%
\pgfsetroundjoin%
\definecolor{currentfill}{rgb}{0.231674,0.318106,0.544834}%
\pgfsetfillcolor{currentfill}%
\pgfsetlinewidth{0.000000pt}%
\definecolor{currentstroke}{rgb}{0.000000,0.000000,0.000000}%
\pgfsetstrokecolor{currentstroke}%
\pgfsetdash{}{0pt}%
\pgfpathmoveto{\pgfqpoint{0.943667in}{1.242850in}}%
\pgfpathlineto{\pgfqpoint{0.941610in}{1.237409in}}%
\pgfpathlineto{\pgfqpoint{0.939555in}{1.232033in}}%
\pgfpathlineto{\pgfqpoint{0.937501in}{1.226726in}}%
\pgfpathlineto{\pgfqpoint{0.935449in}{1.221490in}}%
\pgfpathlineto{\pgfqpoint{0.946291in}{1.225241in}}%
\pgfpathlineto{\pgfqpoint{0.957348in}{1.228818in}}%
\pgfpathlineto{\pgfqpoint{0.968611in}{1.232219in}}%
\pgfpathlineto{\pgfqpoint{0.980067in}{1.235440in}}%
\pgfpathlineto{\pgfqpoint{0.981742in}{1.240548in}}%
\pgfpathlineto{\pgfqpoint{0.983417in}{1.245729in}}%
\pgfpathlineto{\pgfqpoint{0.985094in}{1.250977in}}%
\pgfpathlineto{\pgfqpoint{0.986772in}{1.256290in}}%
\pgfpathlineto{\pgfqpoint{0.975703in}{1.253186in}}%
\pgfpathlineto{\pgfqpoint{0.964822in}{1.249910in}}%
\pgfpathlineto{\pgfqpoint{0.954140in}{1.246463in}}%
\pgfpathlineto{\pgfqpoint{0.943667in}{1.242850in}}%
\pgfpathclose%
\pgfusepath{fill}%
\end{pgfscope}%
\begin{pgfscope}%
\pgfpathrectangle{\pgfqpoint{0.041670in}{0.041670in}}{\pgfqpoint{2.216660in}{2.216660in}}%
\pgfusepath{clip}%
\pgfsetbuttcap%
\pgfsetroundjoin%
\definecolor{currentfill}{rgb}{0.248629,0.278775,0.534556}%
\pgfsetfillcolor{currentfill}%
\pgfsetlinewidth{0.000000pt}%
\definecolor{currentstroke}{rgb}{0.000000,0.000000,0.000000}%
\pgfsetstrokecolor{currentstroke}%
\pgfsetdash{}{0pt}%
\pgfpathmoveto{\pgfqpoint{0.894441in}{1.204815in}}%
\pgfpathlineto{\pgfqpoint{0.892040in}{1.199501in}}%
\pgfpathlineto{\pgfqpoint{0.889641in}{1.194265in}}%
\pgfpathlineto{\pgfqpoint{0.887243in}{1.189110in}}%
\pgfpathlineto{\pgfqpoint{0.884846in}{1.184041in}}%
\pgfpathlineto{\pgfqpoint{0.895062in}{1.188615in}}%
\pgfpathlineto{\pgfqpoint{0.905542in}{1.193023in}}%
\pgfpathlineto{\pgfqpoint{0.916276in}{1.197262in}}%
\pgfpathlineto{\pgfqpoint{0.927253in}{1.201326in}}%
\pgfpathlineto{\pgfqpoint{0.929300in}{1.206244in}}%
\pgfpathlineto{\pgfqpoint{0.931348in}{1.211246in}}%
\pgfpathlineto{\pgfqpoint{0.933398in}{1.216329in}}%
\pgfpathlineto{\pgfqpoint{0.935449in}{1.221490in}}%
\pgfpathlineto{\pgfqpoint{0.924833in}{1.217568in}}%
\pgfpathlineto{\pgfqpoint{0.914452in}{1.213480in}}%
\pgfpathlineto{\pgfqpoint{0.904319in}{1.209227in}}%
\pgfpathlineto{\pgfqpoint{0.894441in}{1.204815in}}%
\pgfpathclose%
\pgfusepath{fill}%
\end{pgfscope}%
\begin{pgfscope}%
\pgfpathrectangle{\pgfqpoint{0.041670in}{0.041670in}}{\pgfqpoint{2.216660in}{2.216660in}}%
\pgfusepath{clip}%
\pgfsetbuttcap%
\pgfsetroundjoin%
\definecolor{currentfill}{rgb}{0.263663,0.237631,0.518762}%
\pgfsetfillcolor{currentfill}%
\pgfsetlinewidth{0.000000pt}%
\definecolor{currentstroke}{rgb}{0.000000,0.000000,0.000000}%
\pgfsetstrokecolor{currentstroke}%
\pgfsetdash{}{0pt}%
\pgfpathmoveto{\pgfqpoint{1.465996in}{1.188115in}}%
\pgfpathlineto{\pgfqpoint{1.468316in}{1.183169in}}%
\pgfpathlineto{\pgfqpoint{1.470635in}{1.178315in}}%
\pgfpathlineto{\pgfqpoint{1.472953in}{1.173556in}}%
\pgfpathlineto{\pgfqpoint{1.475270in}{1.168895in}}%
\pgfpathlineto{\pgfqpoint{1.485793in}{1.164140in}}%
\pgfpathlineto{\pgfqpoint{1.496033in}{1.159218in}}%
\pgfpathlineto{\pgfqpoint{1.505979in}{1.154132in}}%
\pgfpathlineto{\pgfqpoint{1.515622in}{1.148887in}}%
\pgfpathlineto{\pgfqpoint{1.512981in}{1.153719in}}%
\pgfpathlineto{\pgfqpoint{1.510338in}{1.158649in}}%
\pgfpathlineto{\pgfqpoint{1.507695in}{1.163674in}}%
\pgfpathlineto{\pgfqpoint{1.505051in}{1.168791in}}%
\pgfpathlineto{\pgfqpoint{1.495719in}{1.173856in}}%
\pgfpathlineto{\pgfqpoint{1.486093in}{1.178768in}}%
\pgfpathlineto{\pgfqpoint{1.476182in}{1.183522in}}%
\pgfpathlineto{\pgfqpoint{1.465996in}{1.188115in}}%
\pgfpathclose%
\pgfusepath{fill}%
\end{pgfscope}%
\begin{pgfscope}%
\pgfpathrectangle{\pgfqpoint{0.041670in}{0.041670in}}{\pgfqpoint{2.216660in}{2.216660in}}%
\pgfusepath{clip}%
\pgfsetbuttcap%
\pgfsetroundjoin%
\definecolor{currentfill}{rgb}{0.212395,0.359683,0.551710}%
\pgfsetfillcolor{currentfill}%
\pgfsetlinewidth{0.000000pt}%
\definecolor{currentstroke}{rgb}{0.000000,0.000000,0.000000}%
\pgfsetstrokecolor{currentstroke}%
\pgfsetdash{}{0pt}%
\pgfpathmoveto{\pgfqpoint{1.311908in}{1.290253in}}%
\pgfpathlineto{\pgfqpoint{1.313098in}{1.284831in}}%
\pgfpathlineto{\pgfqpoint{1.314286in}{1.279461in}}%
\pgfpathlineto{\pgfqpoint{1.315474in}{1.274147in}}%
\pgfpathlineto{\pgfqpoint{1.316660in}{1.268891in}}%
\pgfpathlineto{\pgfqpoint{1.328497in}{1.266668in}}%
\pgfpathlineto{\pgfqpoint{1.340198in}{1.264261in}}%
\pgfpathlineto{\pgfqpoint{1.351752in}{1.261671in}}%
\pgfpathlineto{\pgfqpoint{1.363148in}{1.258901in}}%
\pgfpathlineto{\pgfqpoint{1.361556in}{1.264251in}}%
\pgfpathlineto{\pgfqpoint{1.359963in}{1.269660in}}%
\pgfpathlineto{\pgfqpoint{1.358368in}{1.275124in}}%
\pgfpathlineto{\pgfqpoint{1.356771in}{1.280641in}}%
\pgfpathlineto{\pgfqpoint{1.345774in}{1.283306in}}%
\pgfpathlineto{\pgfqpoint{1.334624in}{1.285797in}}%
\pgfpathlineto{\pgfqpoint{1.323332in}{1.288114in}}%
\pgfpathlineto{\pgfqpoint{1.311908in}{1.290253in}}%
\pgfpathclose%
\pgfusepath{fill}%
\end{pgfscope}%
\begin{pgfscope}%
\pgfpathrectangle{\pgfqpoint{0.041670in}{0.041670in}}{\pgfqpoint{2.216660in}{2.216660in}}%
\pgfusepath{clip}%
\pgfsetbuttcap%
\pgfsetroundjoin%
\definecolor{currentfill}{rgb}{0.282327,0.094955,0.417331}%
\pgfsetfillcolor{currentfill}%
\pgfsetlinewidth{0.000000pt}%
\definecolor{currentstroke}{rgb}{0.000000,0.000000,0.000000}%
\pgfsetstrokecolor{currentstroke}%
\pgfsetdash{}{0pt}%
\pgfpathmoveto{\pgfqpoint{1.586085in}{1.075093in}}%
\pgfpathlineto{\pgfqpoint{1.589012in}{1.071676in}}%
\pgfpathlineto{\pgfqpoint{1.591940in}{1.068408in}}%
\pgfpathlineto{\pgfqpoint{1.594870in}{1.065293in}}%
\pgfpathlineto{\pgfqpoint{1.597800in}{1.062336in}}%
\pgfpathlineto{\pgfqpoint{1.606840in}{1.055551in}}%
\pgfpathlineto{\pgfqpoint{1.615479in}{1.048617in}}%
\pgfpathlineto{\pgfqpoint{1.623710in}{1.041540in}}%
\pgfpathlineto{\pgfqpoint{1.631521in}{1.034328in}}%
\pgfpathlineto{\pgfqpoint{1.628342in}{1.037494in}}%
\pgfpathlineto{\pgfqpoint{1.625164in}{1.040818in}}%
\pgfpathlineto{\pgfqpoint{1.621988in}{1.044295in}}%
\pgfpathlineto{\pgfqpoint{1.618812in}{1.047923in}}%
\pgfpathlineto{\pgfqpoint{1.611233in}{1.054919in}}%
\pgfpathlineto{\pgfqpoint{1.603246in}{1.061784in}}%
\pgfpathlineto{\pgfqpoint{1.594860in}{1.068511in}}%
\pgfpathlineto{\pgfqpoint{1.586085in}{1.075093in}}%
\pgfpathclose%
\pgfusepath{fill}%
\end{pgfscope}%
\begin{pgfscope}%
\pgfpathrectangle{\pgfqpoint{0.041670in}{0.041670in}}{\pgfqpoint{2.216660in}{2.216660in}}%
\pgfusepath{clip}%
\pgfsetbuttcap%
\pgfsetroundjoin%
\definecolor{currentfill}{rgb}{0.195860,0.395433,0.555276}%
\pgfsetfillcolor{currentfill}%
\pgfsetlinewidth{0.000000pt}%
\definecolor{currentstroke}{rgb}{0.000000,0.000000,0.000000}%
\pgfsetstrokecolor{currentstroke}%
\pgfsetdash{}{0pt}%
\pgfpathmoveto{\pgfqpoint{1.133828in}{1.321974in}}%
\pgfpathlineto{\pgfqpoint{1.133396in}{1.316471in}}%
\pgfpathlineto{\pgfqpoint{1.132963in}{1.311009in}}%
\pgfpathlineto{\pgfqpoint{1.132532in}{1.305589in}}%
\pgfpathlineto{\pgfqpoint{1.132101in}{1.300216in}}%
\pgfpathlineto{\pgfqpoint{1.144195in}{1.300865in}}%
\pgfpathlineto{\pgfqpoint{1.156322in}{1.301325in}}%
\pgfpathlineto{\pgfqpoint{1.168471in}{1.301598in}}%
\pgfpathlineto{\pgfqpoint{1.180630in}{1.301681in}}%
\pgfpathlineto{\pgfqpoint{1.180624in}{1.307040in}}%
\pgfpathlineto{\pgfqpoint{1.180618in}{1.312446in}}%
\pgfpathlineto{\pgfqpoint{1.180612in}{1.317894in}}%
\pgfpathlineto{\pgfqpoint{1.180606in}{1.323382in}}%
\pgfpathlineto{\pgfqpoint{1.168886in}{1.323301in}}%
\pgfpathlineto{\pgfqpoint{1.157175in}{1.323040in}}%
\pgfpathlineto{\pgfqpoint{1.145486in}{1.322597in}}%
\pgfpathlineto{\pgfqpoint{1.133828in}{1.321974in}}%
\pgfpathclose%
\pgfusepath{fill}%
\end{pgfscope}%
\begin{pgfscope}%
\pgfpathrectangle{\pgfqpoint{0.041670in}{0.041670in}}{\pgfqpoint{2.216660in}{2.216660in}}%
\pgfusepath{clip}%
\pgfsetbuttcap%
\pgfsetroundjoin%
\definecolor{currentfill}{rgb}{0.195860,0.395433,0.555276}%
\pgfsetfillcolor{currentfill}%
\pgfsetlinewidth{0.000000pt}%
\definecolor{currentstroke}{rgb}{0.000000,0.000000,0.000000}%
\pgfsetstrokecolor{currentstroke}%
\pgfsetdash{}{0pt}%
\pgfpathmoveto{\pgfqpoint{1.180606in}{1.323382in}}%
\pgfpathlineto{\pgfqpoint{1.180612in}{1.317894in}}%
\pgfpathlineto{\pgfqpoint{1.180618in}{1.312446in}}%
\pgfpathlineto{\pgfqpoint{1.180624in}{1.307040in}}%
\pgfpathlineto{\pgfqpoint{1.180630in}{1.301681in}}%
\pgfpathlineto{\pgfqpoint{1.192789in}{1.301577in}}%
\pgfpathlineto{\pgfqpoint{1.204936in}{1.301283in}}%
\pgfpathlineto{\pgfqpoint{1.217060in}{1.300802in}}%
\pgfpathlineto{\pgfqpoint{1.229150in}{1.300132in}}%
\pgfpathlineto{\pgfqpoint{1.228707in}{1.305507in}}%
\pgfpathlineto{\pgfqpoint{1.228263in}{1.310927in}}%
\pgfpathlineto{\pgfqpoint{1.227819in}{1.316390in}}%
\pgfpathlineto{\pgfqpoint{1.227374in}{1.321894in}}%
\pgfpathlineto{\pgfqpoint{1.215721in}{1.322537in}}%
\pgfpathlineto{\pgfqpoint{1.204034in}{1.322999in}}%
\pgfpathlineto{\pgfqpoint{1.192326in}{1.323281in}}%
\pgfpathlineto{\pgfqpoint{1.180606in}{1.323382in}}%
\pgfpathclose%
\pgfusepath{fill}%
\end{pgfscope}%
\begin{pgfscope}%
\pgfpathrectangle{\pgfqpoint{0.041670in}{0.041670in}}{\pgfqpoint{2.216660in}{2.216660in}}%
\pgfusepath{clip}%
\pgfsetbuttcap%
\pgfsetroundjoin%
\definecolor{currentfill}{rgb}{0.233603,0.313828,0.543914}%
\pgfsetfillcolor{currentfill}%
\pgfsetlinewidth{0.000000pt}%
\definecolor{currentstroke}{rgb}{0.000000,0.000000,0.000000}%
\pgfsetstrokecolor{currentstroke}%
\pgfsetdash{}{0pt}%
\pgfpathmoveto{\pgfqpoint{0.546730in}{1.187624in}}%
\pgfpathlineto{\pgfqpoint{0.543096in}{1.199421in}}%
\pgfpathlineto{\pgfqpoint{0.539446in}{1.211655in}}%
\pgfpathlineto{\pgfqpoint{0.535779in}{1.224333in}}%
\pgfpathlineto{\pgfqpoint{0.532096in}{1.237463in}}%
\pgfpathlineto{\pgfqpoint{0.542880in}{1.247515in}}%
\pgfpathlineto{\pgfqpoint{0.554251in}{1.257379in}}%
\pgfpathlineto{\pgfqpoint{0.566198in}{1.267047in}}%
\pgfpathlineto{\pgfqpoint{0.578707in}{1.276511in}}%
\pgfpathlineto{\pgfqpoint{0.582106in}{1.263235in}}%
\pgfpathlineto{\pgfqpoint{0.585490in}{1.250407in}}%
\pgfpathlineto{\pgfqpoint{0.588858in}{1.238022in}}%
\pgfpathlineto{\pgfqpoint{0.592212in}{1.226072in}}%
\pgfpathlineto{\pgfqpoint{0.580003in}{1.216753in}}%
\pgfpathlineto{\pgfqpoint{0.568345in}{1.207233in}}%
\pgfpathlineto{\pgfqpoint{0.557249in}{1.197521in}}%
\pgfpathlineto{\pgfqpoint{0.546730in}{1.187624in}}%
\pgfpathclose%
\pgfusepath{fill}%
\end{pgfscope}%
\begin{pgfscope}%
\pgfpathrectangle{\pgfqpoint{0.041670in}{0.041670in}}{\pgfqpoint{2.216660in}{2.216660in}}%
\pgfusepath{clip}%
\pgfsetbuttcap%
\pgfsetroundjoin%
\definecolor{currentfill}{rgb}{0.268510,0.009605,0.335427}%
\pgfsetfillcolor{currentfill}%
\pgfsetlinewidth{0.000000pt}%
\definecolor{currentstroke}{rgb}{0.000000,0.000000,0.000000}%
\pgfsetstrokecolor{currentstroke}%
\pgfsetdash{}{0pt}%
\pgfpathmoveto{\pgfqpoint{0.613360in}{0.972852in}}%
\pgfpathlineto{\pgfqpoint{0.609830in}{0.974734in}}%
\pgfpathlineto{\pgfqpoint{0.606292in}{0.976889in}}%
\pgfpathlineto{\pgfqpoint{0.602746in}{0.979322in}}%
\pgfpathlineto{\pgfqpoint{0.599192in}{0.982037in}}%
\pgfpathlineto{\pgfqpoint{0.606103in}{0.991407in}}%
\pgfpathlineto{\pgfqpoint{0.613558in}{1.000647in}}%
\pgfpathlineto{\pgfqpoint{0.621549in}{1.009751in}}%
\pgfpathlineto{\pgfqpoint{0.630067in}{1.018709in}}%
\pgfpathlineto{\pgfqpoint{0.633414in}{1.015784in}}%
\pgfpathlineto{\pgfqpoint{0.636755in}{1.013141in}}%
\pgfpathlineto{\pgfqpoint{0.640088in}{1.010775in}}%
\pgfpathlineto{\pgfqpoint{0.643414in}{1.008680in}}%
\pgfpathlineto{\pgfqpoint{0.635121in}{0.999927in}}%
\pgfpathlineto{\pgfqpoint{0.627341in}{0.991032in}}%
\pgfpathlineto{\pgfqpoint{0.620085in}{0.982005in}}%
\pgfpathlineto{\pgfqpoint{0.613360in}{0.972852in}}%
\pgfpathclose%
\pgfusepath{fill}%
\end{pgfscope}%
\begin{pgfscope}%
\pgfpathrectangle{\pgfqpoint{0.041670in}{0.041670in}}{\pgfqpoint{2.216660in}{2.216660in}}%
\pgfusepath{clip}%
\pgfsetbuttcap%
\pgfsetroundjoin%
\definecolor{currentfill}{rgb}{0.280255,0.165693,0.476498}%
\pgfsetfillcolor{currentfill}%
\pgfsetlinewidth{0.000000pt}%
\definecolor{currentstroke}{rgb}{0.000000,0.000000,0.000000}%
\pgfsetstrokecolor{currentstroke}%
\pgfsetdash{}{0pt}%
\pgfpathmoveto{\pgfqpoint{0.789881in}{1.101885in}}%
\pgfpathlineto{\pgfqpoint{0.786897in}{1.097360in}}%
\pgfpathlineto{\pgfqpoint{0.783913in}{1.092956in}}%
\pgfpathlineto{\pgfqpoint{0.780929in}{1.088674in}}%
\pgfpathlineto{\pgfqpoint{0.777945in}{1.084520in}}%
\pgfpathlineto{\pgfqpoint{0.786498in}{1.090883in}}%
\pgfpathlineto{\pgfqpoint{0.795420in}{1.097101in}}%
\pgfpathlineto{\pgfqpoint{0.804701in}{1.103168in}}%
\pgfpathlineto{\pgfqpoint{0.814333in}{1.109079in}}%
\pgfpathlineto{\pgfqpoint{0.817037in}{1.113037in}}%
\pgfpathlineto{\pgfqpoint{0.819742in}{1.117122in}}%
\pgfpathlineto{\pgfqpoint{0.822447in}{1.121330in}}%
\pgfpathlineto{\pgfqpoint{0.825152in}{1.125658in}}%
\pgfpathlineto{\pgfqpoint{0.815815in}{1.119936in}}%
\pgfpathlineto{\pgfqpoint{0.806817in}{1.114063in}}%
\pgfpathlineto{\pgfqpoint{0.798170in}{1.108044in}}%
\pgfpathlineto{\pgfqpoint{0.789881in}{1.101885in}}%
\pgfpathclose%
\pgfusepath{fill}%
\end{pgfscope}%
\begin{pgfscope}%
\pgfpathrectangle{\pgfqpoint{0.041670in}{0.041670in}}{\pgfqpoint{2.216660in}{2.216660in}}%
\pgfusepath{clip}%
\pgfsetbuttcap%
\pgfsetroundjoin%
\definecolor{currentfill}{rgb}{0.268510,0.009605,0.335427}%
\pgfsetfillcolor{currentfill}%
\pgfsetlinewidth{0.000000pt}%
\definecolor{currentstroke}{rgb}{0.000000,0.000000,0.000000}%
\pgfsetstrokecolor{currentstroke}%
\pgfsetdash{}{0pt}%
\pgfpathmoveto{\pgfqpoint{1.669840in}{1.010090in}}%
\pgfpathlineto{\pgfqpoint{1.673052in}{1.009342in}}%
\pgfpathlineto{\pgfqpoint{1.676269in}{1.008810in}}%
\pgfpathlineto{\pgfqpoint{1.679490in}{1.008497in}}%
\pgfpathlineto{\pgfqpoint{1.682715in}{1.008409in}}%
\pgfpathlineto{\pgfqpoint{1.690997in}{1.000203in}}%
\pgfpathlineto{\pgfqpoint{1.698797in}{0.991857in}}%
\pgfpathlineto{\pgfqpoint{1.706107in}{0.983378in}}%
\pgfpathlineto{\pgfqpoint{1.712916in}{0.974775in}}%
\pgfpathlineto{\pgfqpoint{1.709481in}{0.975080in}}%
\pgfpathlineto{\pgfqpoint{1.706051in}{0.975610in}}%
\pgfpathlineto{\pgfqpoint{1.702625in}{0.976361in}}%
\pgfpathlineto{\pgfqpoint{1.699204in}{0.977329in}}%
\pgfpathlineto{\pgfqpoint{1.692586in}{0.985708in}}%
\pgfpathlineto{\pgfqpoint{1.685480in}{0.993966in}}%
\pgfpathlineto{\pgfqpoint{1.677895in}{1.002096in}}%
\pgfpathlineto{\pgfqpoint{1.669840in}{1.010090in}}%
\pgfpathclose%
\pgfusepath{fill}%
\end{pgfscope}%
\begin{pgfscope}%
\pgfpathrectangle{\pgfqpoint{0.041670in}{0.041670in}}{\pgfqpoint{2.216660in}{2.216660in}}%
\pgfusepath{clip}%
\pgfsetbuttcap%
\pgfsetroundjoin%
\definecolor{currentfill}{rgb}{0.272594,0.025563,0.353093}%
\pgfsetfillcolor{currentfill}%
\pgfsetlinewidth{0.000000pt}%
\definecolor{currentstroke}{rgb}{0.000000,0.000000,0.000000}%
\pgfsetstrokecolor{currentstroke}%
\pgfsetdash{}{0pt}%
\pgfpathmoveto{\pgfqpoint{0.599192in}{0.982037in}}%
\pgfpathlineto{\pgfqpoint{0.595630in}{0.985041in}}%
\pgfpathlineto{\pgfqpoint{0.592059in}{0.988338in}}%
\pgfpathlineto{\pgfqpoint{0.588479in}{0.991933in}}%
\pgfpathlineto{\pgfqpoint{0.584890in}{0.995832in}}%
\pgfpathlineto{\pgfqpoint{0.591990in}{1.005413in}}%
\pgfpathlineto{\pgfqpoint{0.599648in}{1.014861in}}%
\pgfpathlineto{\pgfqpoint{0.607853in}{1.024170in}}%
\pgfpathlineto{\pgfqpoint{0.616597in}{1.033329in}}%
\pgfpathlineto{\pgfqpoint{0.619977in}{1.029226in}}%
\pgfpathlineto{\pgfqpoint{0.623348in}{1.025425in}}%
\pgfpathlineto{\pgfqpoint{0.626711in}{1.021921in}}%
\pgfpathlineto{\pgfqpoint{0.630067in}{1.018709in}}%
\pgfpathlineto{\pgfqpoint{0.621549in}{1.009751in}}%
\pgfpathlineto{\pgfqpoint{0.613558in}{1.000647in}}%
\pgfpathlineto{\pgfqpoint{0.606103in}{0.991407in}}%
\pgfpathlineto{\pgfqpoint{0.599192in}{0.982037in}}%
\pgfpathclose%
\pgfusepath{fill}%
\end{pgfscope}%
\begin{pgfscope}%
\pgfpathrectangle{\pgfqpoint{0.041670in}{0.041670in}}{\pgfqpoint{2.216660in}{2.216660in}}%
\pgfusepath{clip}%
\pgfsetbuttcap%
\pgfsetroundjoin%
\definecolor{currentfill}{rgb}{0.267004,0.004874,0.329415}%
\pgfsetfillcolor{currentfill}%
\pgfsetlinewidth{0.000000pt}%
\definecolor{currentstroke}{rgb}{0.000000,0.000000,0.000000}%
\pgfsetstrokecolor{currentstroke}%
\pgfsetdash{}{0pt}%
\pgfpathmoveto{\pgfqpoint{0.627413in}{0.967952in}}%
\pgfpathlineto{\pgfqpoint{0.623910in}{0.968793in}}%
\pgfpathlineto{\pgfqpoint{0.620400in}{0.969886in}}%
\pgfpathlineto{\pgfqpoint{0.616884in}{0.971238in}}%
\pgfpathlineto{\pgfqpoint{0.613360in}{0.972852in}}%
\pgfpathlineto{\pgfqpoint{0.620085in}{0.982005in}}%
\pgfpathlineto{\pgfqpoint{0.627341in}{0.991032in}}%
\pgfpathlineto{\pgfqpoint{0.635121in}{0.999927in}}%
\pgfpathlineto{\pgfqpoint{0.643414in}{1.008680in}}%
\pgfpathlineto{\pgfqpoint{0.646734in}{1.006852in}}%
\pgfpathlineto{\pgfqpoint{0.650048in}{1.005286in}}%
\pgfpathlineto{\pgfqpoint{0.653355in}{1.003977in}}%
\pgfpathlineto{\pgfqpoint{0.656656in}{1.002920in}}%
\pgfpathlineto{\pgfqpoint{0.648584in}{0.994376in}}%
\pgfpathlineto{\pgfqpoint{0.641013in}{0.985695in}}%
\pgfpathlineto{\pgfqpoint{0.633953in}{0.976884in}}%
\pgfpathlineto{\pgfqpoint{0.627413in}{0.967952in}}%
\pgfpathclose%
\pgfusepath{fill}%
\end{pgfscope}%
\begin{pgfscope}%
\pgfpathrectangle{\pgfqpoint{0.041670in}{0.041670in}}{\pgfqpoint{2.216660in}{2.216660in}}%
\pgfusepath{clip}%
\pgfsetbuttcap%
\pgfsetroundjoin%
\definecolor{currentfill}{rgb}{0.212395,0.359683,0.551710}%
\pgfsetfillcolor{currentfill}%
\pgfsetlinewidth{0.000000pt}%
\definecolor{currentstroke}{rgb}{0.000000,0.000000,0.000000}%
\pgfsetstrokecolor{currentstroke}%
\pgfsetdash{}{0pt}%
\pgfpathmoveto{\pgfqpoint{0.993500in}{1.278128in}}%
\pgfpathlineto{\pgfqpoint{0.991816in}{1.272587in}}%
\pgfpathlineto{\pgfqpoint{0.990133in}{1.267098in}}%
\pgfpathlineto{\pgfqpoint{0.988452in}{1.261665in}}%
\pgfpathlineto{\pgfqpoint{0.986772in}{1.256290in}}%
\pgfpathlineto{\pgfqpoint{0.998019in}{1.259218in}}%
\pgfpathlineto{\pgfqpoint{1.009434in}{1.261968in}}%
\pgfpathlineto{\pgfqpoint{1.021005in}{1.264537in}}%
\pgfpathlineto{\pgfqpoint{1.032721in}{1.266924in}}%
\pgfpathlineto{\pgfqpoint{1.033999in}{1.272199in}}%
\pgfpathlineto{\pgfqpoint{1.035279in}{1.277531in}}%
\pgfpathlineto{\pgfqpoint{1.036559in}{1.282920in}}%
\pgfpathlineto{\pgfqpoint{1.037841in}{1.288360in}}%
\pgfpathlineto{\pgfqpoint{1.026533in}{1.286063in}}%
\pgfpathlineto{\pgfqpoint{1.015367in}{1.283591in}}%
\pgfpathlineto{\pgfqpoint{1.004352in}{1.280945in}}%
\pgfpathlineto{\pgfqpoint{0.993500in}{1.278128in}}%
\pgfpathclose%
\pgfusepath{fill}%
\end{pgfscope}%
\begin{pgfscope}%
\pgfpathrectangle{\pgfqpoint{0.041670in}{0.041670in}}{\pgfqpoint{2.216660in}{2.216660in}}%
\pgfusepath{clip}%
\pgfsetbuttcap%
\pgfsetroundjoin%
\definecolor{currentfill}{rgb}{0.277941,0.056324,0.381191}%
\pgfsetfillcolor{currentfill}%
\pgfsetlinewidth{0.000000pt}%
\definecolor{currentstroke}{rgb}{0.000000,0.000000,0.000000}%
\pgfsetstrokecolor{currentstroke}%
\pgfsetdash{}{0pt}%
\pgfpathmoveto{\pgfqpoint{0.584890in}{0.995832in}}%
\pgfpathlineto{\pgfqpoint{0.581292in}{1.000039in}}%
\pgfpathlineto{\pgfqpoint{0.577683in}{1.004562in}}%
\pgfpathlineto{\pgfqpoint{0.574065in}{1.009404in}}%
\pgfpathlineto{\pgfqpoint{0.570435in}{1.014572in}}%
\pgfpathlineto{\pgfqpoint{0.577728in}{1.024360in}}%
\pgfpathlineto{\pgfqpoint{0.585590in}{1.034011in}}%
\pgfpathlineto{\pgfqpoint{0.594013in}{1.043519in}}%
\pgfpathlineto{\pgfqpoint{0.602986in}{1.052874in}}%
\pgfpathlineto{\pgfqpoint{0.606403in}{1.047507in}}%
\pgfpathlineto{\pgfqpoint{0.609810in}{1.042464in}}%
\pgfpathlineto{\pgfqpoint{0.613208in}{1.037740in}}%
\pgfpathlineto{\pgfqpoint{0.616597in}{1.033329in}}%
\pgfpathlineto{\pgfqpoint{0.607853in}{1.024170in}}%
\pgfpathlineto{\pgfqpoint{0.599648in}{1.014861in}}%
\pgfpathlineto{\pgfqpoint{0.591990in}{1.005413in}}%
\pgfpathlineto{\pgfqpoint{0.584890in}{0.995832in}}%
\pgfpathclose%
\pgfusepath{fill}%
\end{pgfscope}%
\begin{pgfscope}%
\pgfpathrectangle{\pgfqpoint{0.041670in}{0.041670in}}{\pgfqpoint{2.216660in}{2.216660in}}%
\pgfusepath{clip}%
\pgfsetbuttcap%
\pgfsetroundjoin%
\definecolor{currentfill}{rgb}{0.195860,0.395433,0.555276}%
\pgfsetfillcolor{currentfill}%
\pgfsetlinewidth{0.000000pt}%
\definecolor{currentstroke}{rgb}{0.000000,0.000000,0.000000}%
\pgfsetstrokecolor{currentstroke}%
\pgfsetdash{}{0pt}%
\pgfpathmoveto{\pgfqpoint{1.087727in}{1.317689in}}%
\pgfpathlineto{\pgfqpoint{1.086861in}{1.312142in}}%
\pgfpathlineto{\pgfqpoint{1.085997in}{1.306636in}}%
\pgfpathlineto{\pgfqpoint{1.085133in}{1.301173in}}%
\pgfpathlineto{\pgfqpoint{1.084270in}{1.295756in}}%
\pgfpathlineto{\pgfqpoint{1.096124in}{1.297149in}}%
\pgfpathlineto{\pgfqpoint{1.108054in}{1.298358in}}%
\pgfpathlineto{\pgfqpoint{1.120050in}{1.299380in}}%
\pgfpathlineto{\pgfqpoint{1.132101in}{1.300216in}}%
\pgfpathlineto{\pgfqpoint{1.132532in}{1.305589in}}%
\pgfpathlineto{\pgfqpoint{1.132963in}{1.311009in}}%
\pgfpathlineto{\pgfqpoint{1.133396in}{1.316471in}}%
\pgfpathlineto{\pgfqpoint{1.133828in}{1.321974in}}%
\pgfpathlineto{\pgfqpoint{1.122213in}{1.321171in}}%
\pgfpathlineto{\pgfqpoint{1.110650in}{1.320188in}}%
\pgfpathlineto{\pgfqpoint{1.099151in}{1.319027in}}%
\pgfpathlineto{\pgfqpoint{1.087727in}{1.317689in}}%
\pgfpathclose%
\pgfusepath{fill}%
\end{pgfscope}%
\begin{pgfscope}%
\pgfpathrectangle{\pgfqpoint{0.041670in}{0.041670in}}{\pgfqpoint{2.216660in}{2.216660in}}%
\pgfusepath{clip}%
\pgfsetbuttcap%
\pgfsetroundjoin%
\definecolor{currentfill}{rgb}{0.195860,0.395433,0.555276}%
\pgfsetfillcolor{currentfill}%
\pgfsetlinewidth{0.000000pt}%
\definecolor{currentstroke}{rgb}{0.000000,0.000000,0.000000}%
\pgfsetstrokecolor{currentstroke}%
\pgfsetdash{}{0pt}%
\pgfpathmoveto{\pgfqpoint{1.227374in}{1.321894in}}%
\pgfpathlineto{\pgfqpoint{1.227819in}{1.316390in}}%
\pgfpathlineto{\pgfqpoint{1.228263in}{1.310927in}}%
\pgfpathlineto{\pgfqpoint{1.228707in}{1.305507in}}%
\pgfpathlineto{\pgfqpoint{1.229150in}{1.300132in}}%
\pgfpathlineto{\pgfqpoint{1.241196in}{1.299276in}}%
\pgfpathlineto{\pgfqpoint{1.253185in}{1.298232in}}%
\pgfpathlineto{\pgfqpoint{1.265107in}{1.297003in}}%
\pgfpathlineto{\pgfqpoint{1.276951in}{1.295590in}}%
\pgfpathlineto{\pgfqpoint{1.276077in}{1.301008in}}%
\pgfpathlineto{\pgfqpoint{1.275201in}{1.306473in}}%
\pgfpathlineto{\pgfqpoint{1.274325in}{1.311981in}}%
\pgfpathlineto{\pgfqpoint{1.273447in}{1.317529in}}%
\pgfpathlineto{\pgfqpoint{1.262032in}{1.318887in}}%
\pgfpathlineto{\pgfqpoint{1.250541in}{1.320068in}}%
\pgfpathlineto{\pgfqpoint{1.238985in}{1.321070in}}%
\pgfpathlineto{\pgfqpoint{1.227374in}{1.321894in}}%
\pgfpathclose%
\pgfusepath{fill}%
\end{pgfscope}%
\begin{pgfscope}%
\pgfpathrectangle{\pgfqpoint{0.041670in}{0.041670in}}{\pgfqpoint{2.216660in}{2.216660in}}%
\pgfusepath{clip}%
\pgfsetbuttcap%
\pgfsetroundjoin%
\definecolor{currentfill}{rgb}{0.263663,0.237631,0.518762}%
\pgfsetfillcolor{currentfill}%
\pgfsetlinewidth{0.000000pt}%
\definecolor{currentstroke}{rgb}{0.000000,0.000000,0.000000}%
\pgfsetstrokecolor{currentstroke}%
\pgfsetdash{}{0pt}%
\pgfpathmoveto{\pgfqpoint{0.846819in}{1.164163in}}%
\pgfpathlineto{\pgfqpoint{0.844107in}{1.159005in}}%
\pgfpathlineto{\pgfqpoint{0.841397in}{1.153939in}}%
\pgfpathlineto{\pgfqpoint{0.838687in}{1.148968in}}%
\pgfpathlineto{\pgfqpoint{0.835979in}{1.144095in}}%
\pgfpathlineto{\pgfqpoint{0.845344in}{1.149477in}}%
\pgfpathlineto{\pgfqpoint{0.855021in}{1.154705in}}%
\pgfpathlineto{\pgfqpoint{0.865000in}{1.159773in}}%
\pgfpathlineto{\pgfqpoint{0.875272in}{1.164677in}}%
\pgfpathlineto{\pgfqpoint{0.877664in}{1.169373in}}%
\pgfpathlineto{\pgfqpoint{0.880057in}{1.174169in}}%
\pgfpathlineto{\pgfqpoint{0.882451in}{1.179059in}}%
\pgfpathlineto{\pgfqpoint{0.884846in}{1.184041in}}%
\pgfpathlineto{\pgfqpoint{0.874904in}{1.179304in}}%
\pgfpathlineto{\pgfqpoint{0.865246in}{1.174409in}}%
\pgfpathlineto{\pgfqpoint{0.855881in}{1.169361in}}%
\pgfpathlineto{\pgfqpoint{0.846819in}{1.164163in}}%
\pgfpathclose%
\pgfusepath{fill}%
\end{pgfscope}%
\begin{pgfscope}%
\pgfpathrectangle{\pgfqpoint{0.041670in}{0.041670in}}{\pgfqpoint{2.216660in}{2.216660in}}%
\pgfusepath{clip}%
\pgfsetbuttcap%
\pgfsetroundjoin%
\definecolor{currentfill}{rgb}{0.282884,0.135920,0.453427}%
\pgfsetfillcolor{currentfill}%
\pgfsetlinewidth{0.000000pt}%
\definecolor{currentstroke}{rgb}{0.000000,0.000000,0.000000}%
\pgfsetstrokecolor{currentstroke}%
\pgfsetdash{}{0pt}%
\pgfpathmoveto{\pgfqpoint{1.762047in}{1.086042in}}%
\pgfpathlineto{\pgfqpoint{1.765462in}{1.093156in}}%
\pgfpathlineto{\pgfqpoint{1.768889in}{1.100628in}}%
\pgfpathlineto{\pgfqpoint{1.772327in}{1.108464in}}%
\pgfpathlineto{\pgfqpoint{1.775777in}{1.116671in}}%
\pgfpathlineto{\pgfqpoint{1.785722in}{1.107092in}}%
\pgfpathlineto{\pgfqpoint{1.795102in}{1.097346in}}%
\pgfpathlineto{\pgfqpoint{1.803906in}{1.087441in}}%
\pgfpathlineto{\pgfqpoint{1.812124in}{1.077385in}}%
\pgfpathlineto{\pgfqpoint{1.808442in}{1.069361in}}%
\pgfpathlineto{\pgfqpoint{1.804774in}{1.061709in}}%
\pgfpathlineto{\pgfqpoint{1.801119in}{1.054424in}}%
\pgfpathlineto{\pgfqpoint{1.797476in}{1.047498in}}%
\pgfpathlineto{\pgfqpoint{1.789469in}{1.057363in}}%
\pgfpathlineto{\pgfqpoint{1.780889in}{1.067081in}}%
\pgfpathlineto{\pgfqpoint{1.771744in}{1.076644in}}%
\pgfpathlineto{\pgfqpoint{1.762047in}{1.086042in}}%
\pgfpathclose%
\pgfusepath{fill}%
\end{pgfscope}%
\begin{pgfscope}%
\pgfpathrectangle{\pgfqpoint{0.041670in}{0.041670in}}{\pgfqpoint{2.216660in}{2.216660in}}%
\pgfusepath{clip}%
\pgfsetbuttcap%
\pgfsetroundjoin%
\definecolor{currentfill}{rgb}{0.267004,0.004874,0.329415}%
\pgfsetfillcolor{currentfill}%
\pgfsetlinewidth{0.000000pt}%
\definecolor{currentstroke}{rgb}{0.000000,0.000000,0.000000}%
\pgfsetstrokecolor{currentstroke}%
\pgfsetdash{}{0pt}%
\pgfpathmoveto{\pgfqpoint{0.641367in}{0.967029in}}%
\pgfpathlineto{\pgfqpoint{0.637887in}{0.966903in}}%
\pgfpathlineto{\pgfqpoint{0.634402in}{0.967012in}}%
\pgfpathlineto{\pgfqpoint{0.630910in}{0.967360in}}%
\pgfpathlineto{\pgfqpoint{0.627413in}{0.967952in}}%
\pgfpathlineto{\pgfqpoint{0.633953in}{0.976884in}}%
\pgfpathlineto{\pgfqpoint{0.641013in}{0.985695in}}%
\pgfpathlineto{\pgfqpoint{0.648584in}{0.994376in}}%
\pgfpathlineto{\pgfqpoint{0.656656in}{1.002920in}}%
\pgfpathlineto{\pgfqpoint{0.659952in}{1.002111in}}%
\pgfpathlineto{\pgfqpoint{0.663243in}{1.001544in}}%
\pgfpathlineto{\pgfqpoint{0.666529in}{1.001216in}}%
\pgfpathlineto{\pgfqpoint{0.669809in}{1.001122in}}%
\pgfpathlineto{\pgfqpoint{0.661955in}{0.992791in}}%
\pgfpathlineto{\pgfqpoint{0.654591in}{0.984327in}}%
\pgfpathlineto{\pgfqpoint{0.647725in}{0.975737in}}%
\pgfpathlineto{\pgfqpoint{0.641367in}{0.967029in}}%
\pgfpathclose%
\pgfusepath{fill}%
\end{pgfscope}%
\begin{pgfscope}%
\pgfpathrectangle{\pgfqpoint{0.041670in}{0.041670in}}{\pgfqpoint{2.216660in}{2.216660in}}%
\pgfusepath{clip}%
\pgfsetbuttcap%
\pgfsetroundjoin%
\definecolor{currentfill}{rgb}{0.201239,0.383670,0.554294}%
\pgfsetfillcolor{currentfill}%
\pgfsetlinewidth{0.000000pt}%
\definecolor{currentstroke}{rgb}{0.000000,0.000000,0.000000}%
\pgfsetstrokecolor{currentstroke}%
\pgfsetdash{}{0pt}%
\pgfpathmoveto{\pgfqpoint{1.769624in}{1.284746in}}%
\pgfpathlineto{\pgfqpoint{1.772969in}{1.298509in}}%
\pgfpathlineto{\pgfqpoint{1.776331in}{1.312736in}}%
\pgfpathlineto{\pgfqpoint{1.779708in}{1.327435in}}%
\pgfpathlineto{\pgfqpoint{1.783103in}{1.342612in}}%
\pgfpathlineto{\pgfqpoint{1.796415in}{1.333204in}}%
\pgfpathlineto{\pgfqpoint{1.809167in}{1.323582in}}%
\pgfpathlineto{\pgfqpoint{1.821344in}{1.313753in}}%
\pgfpathlineto{\pgfqpoint{1.832932in}{1.303725in}}%
\pgfpathlineto{\pgfqpoint{1.829236in}{1.288681in}}%
\pgfpathlineto{\pgfqpoint{1.825559in}{1.274118in}}%
\pgfpathlineto{\pgfqpoint{1.821899in}{1.260030in}}%
\pgfpathlineto{\pgfqpoint{1.818258in}{1.246407in}}%
\pgfpathlineto{\pgfqpoint{1.806951in}{1.256293in}}%
\pgfpathlineto{\pgfqpoint{1.795067in}{1.265983in}}%
\pgfpathlineto{\pgfqpoint{1.782620in}{1.275470in}}%
\pgfpathlineto{\pgfqpoint{1.769624in}{1.284746in}}%
\pgfpathclose%
\pgfusepath{fill}%
\end{pgfscope}%
\begin{pgfscope}%
\pgfpathrectangle{\pgfqpoint{0.041670in}{0.041670in}}{\pgfqpoint{2.216660in}{2.216660in}}%
\pgfusepath{clip}%
\pgfsetbuttcap%
\pgfsetroundjoin%
\definecolor{currentfill}{rgb}{0.282327,0.094955,0.417331}%
\pgfsetfillcolor{currentfill}%
\pgfsetlinewidth{0.000000pt}%
\definecolor{currentstroke}{rgb}{0.000000,0.000000,0.000000}%
\pgfsetstrokecolor{currentstroke}%
\pgfsetdash{}{0pt}%
\pgfpathmoveto{\pgfqpoint{0.734709in}{1.041598in}}%
\pgfpathlineto{\pgfqpoint{0.731484in}{1.037921in}}%
\pgfpathlineto{\pgfqpoint{0.728258in}{1.034395in}}%
\pgfpathlineto{\pgfqpoint{0.725031in}{1.031022in}}%
\pgfpathlineto{\pgfqpoint{0.721803in}{1.027807in}}%
\pgfpathlineto{\pgfqpoint{0.729235in}{1.035136in}}%
\pgfpathlineto{\pgfqpoint{0.737094in}{1.042334in}}%
\pgfpathlineto{\pgfqpoint{0.745370in}{1.049394in}}%
\pgfpathlineto{\pgfqpoint{0.754055in}{1.056312in}}%
\pgfpathlineto{\pgfqpoint{0.757044in}{1.059314in}}%
\pgfpathlineto{\pgfqpoint{0.760032in}{1.062474in}}%
\pgfpathlineto{\pgfqpoint{0.763019in}{1.065787in}}%
\pgfpathlineto{\pgfqpoint{0.766006in}{1.069249in}}%
\pgfpathlineto{\pgfqpoint{0.757576in}{1.062538in}}%
\pgfpathlineto{\pgfqpoint{0.749545in}{1.055689in}}%
\pgfpathlineto{\pgfqpoint{0.741919in}{1.048707in}}%
\pgfpathlineto{\pgfqpoint{0.734709in}{1.041598in}}%
\pgfpathclose%
\pgfusepath{fill}%
\end{pgfscope}%
\begin{pgfscope}%
\pgfpathrectangle{\pgfqpoint{0.041670in}{0.041670in}}{\pgfqpoint{2.216660in}{2.216660in}}%
\pgfusepath{clip}%
\pgfsetbuttcap%
\pgfsetroundjoin%
\definecolor{currentfill}{rgb}{0.271305,0.019942,0.347269}%
\pgfsetfillcolor{currentfill}%
\pgfsetlinewidth{0.000000pt}%
\definecolor{currentstroke}{rgb}{0.000000,0.000000,0.000000}%
\pgfsetstrokecolor{currentstroke}%
\pgfsetdash{}{0pt}%
\pgfpathmoveto{\pgfqpoint{1.657024in}{1.015151in}}%
\pgfpathlineto{\pgfqpoint{1.660223in}{1.013584in}}%
\pgfpathlineto{\pgfqpoint{1.663425in}{1.012215in}}%
\pgfpathlineto{\pgfqpoint{1.666631in}{1.011049in}}%
\pgfpathlineto{\pgfqpoint{1.669840in}{1.010090in}}%
\pgfpathlineto{\pgfqpoint{1.677895in}{1.002096in}}%
\pgfpathlineto{\pgfqpoint{1.685480in}{0.993966in}}%
\pgfpathlineto{\pgfqpoint{1.692586in}{0.985708in}}%
\pgfpathlineto{\pgfqpoint{1.699204in}{0.977329in}}%
\pgfpathlineto{\pgfqpoint{1.695787in}{0.978508in}}%
\pgfpathlineto{\pgfqpoint{1.692374in}{0.979894in}}%
\pgfpathlineto{\pgfqpoint{1.688965in}{0.981484in}}%
\pgfpathlineto{\pgfqpoint{1.685559in}{0.983273in}}%
\pgfpathlineto{\pgfqpoint{1.679130in}{0.991426in}}%
\pgfpathlineto{\pgfqpoint{1.672226in}{0.999461in}}%
\pgfpathlineto{\pgfqpoint{1.664854in}{1.007372in}}%
\pgfpathlineto{\pgfqpoint{1.657024in}{1.015151in}}%
\pgfpathclose%
\pgfusepath{fill}%
\end{pgfscope}%
\begin{pgfscope}%
\pgfpathrectangle{\pgfqpoint{0.041670in}{0.041670in}}{\pgfqpoint{2.216660in}{2.216660in}}%
\pgfusepath{clip}%
\pgfsetbuttcap%
\pgfsetroundjoin%
\definecolor{currentfill}{rgb}{0.282327,0.094955,0.417331}%
\pgfsetfillcolor{currentfill}%
\pgfsetlinewidth{0.000000pt}%
\definecolor{currentstroke}{rgb}{0.000000,0.000000,0.000000}%
\pgfsetstrokecolor{currentstroke}%
\pgfsetdash{}{0pt}%
\pgfpathmoveto{\pgfqpoint{0.570435in}{1.014572in}}%
\pgfpathlineto{\pgfqpoint{0.566796in}{1.020071in}}%
\pgfpathlineto{\pgfqpoint{0.563144in}{1.025907in}}%
\pgfpathlineto{\pgfqpoint{0.559482in}{1.032086in}}%
\pgfpathlineto{\pgfqpoint{0.555807in}{1.038613in}}%
\pgfpathlineto{\pgfqpoint{0.563295in}{1.048601in}}%
\pgfpathlineto{\pgfqpoint{0.571366in}{1.058450in}}%
\pgfpathlineto{\pgfqpoint{0.580009in}{1.068151in}}%
\pgfpathlineto{\pgfqpoint{0.589216in}{1.077696in}}%
\pgfpathlineto{\pgfqpoint{0.592674in}{1.070976in}}%
\pgfpathlineto{\pgfqpoint{0.596122in}{1.064603in}}%
\pgfpathlineto{\pgfqpoint{0.599559in}{1.058571in}}%
\pgfpathlineto{\pgfqpoint{0.602986in}{1.052874in}}%
\pgfpathlineto{\pgfqpoint{0.594013in}{1.043519in}}%
\pgfpathlineto{\pgfqpoint{0.585590in}{1.034011in}}%
\pgfpathlineto{\pgfqpoint{0.577728in}{1.024360in}}%
\pgfpathlineto{\pgfqpoint{0.570435in}{1.014572in}}%
\pgfpathclose%
\pgfusepath{fill}%
\end{pgfscope}%
\begin{pgfscope}%
\pgfpathrectangle{\pgfqpoint{0.041670in}{0.041670in}}{\pgfqpoint{2.216660in}{2.216660in}}%
\pgfusepath{clip}%
\pgfsetbuttcap%
\pgfsetroundjoin%
\definecolor{currentfill}{rgb}{0.274128,0.199721,0.498911}%
\pgfsetfillcolor{currentfill}%
\pgfsetlinewidth{0.000000pt}%
\definecolor{currentstroke}{rgb}{0.000000,0.000000,0.000000}%
\pgfsetstrokecolor{currentstroke}%
\pgfsetdash{}{0pt}%
\pgfpathmoveto{\pgfqpoint{1.515622in}{1.148887in}}%
\pgfpathlineto{\pgfqpoint{1.518263in}{1.144157in}}%
\pgfpathlineto{\pgfqpoint{1.520902in}{1.139533in}}%
\pgfpathlineto{\pgfqpoint{1.523541in}{1.135017in}}%
\pgfpathlineto{\pgfqpoint{1.526179in}{1.130614in}}%
\pgfpathlineto{\pgfqpoint{1.535812in}{1.125030in}}%
\pgfpathlineto{\pgfqpoint{1.545112in}{1.119291in}}%
\pgfpathlineto{\pgfqpoint{1.554071in}{1.113401in}}%
\pgfpathlineto{\pgfqpoint{1.562679in}{1.107367in}}%
\pgfpathlineto{\pgfqpoint{1.559752in}{1.111962in}}%
\pgfpathlineto{\pgfqpoint{1.556826in}{1.116670in}}%
\pgfpathlineto{\pgfqpoint{1.553898in}{1.121487in}}%
\pgfpathlineto{\pgfqpoint{1.550970in}{1.126410in}}%
\pgfpathlineto{\pgfqpoint{1.542635in}{1.132244in}}%
\pgfpathlineto{\pgfqpoint{1.533960in}{1.137938in}}%
\pgfpathlineto{\pgfqpoint{1.524952in}{1.143487in}}%
\pgfpathlineto{\pgfqpoint{1.515622in}{1.148887in}}%
\pgfpathclose%
\pgfusepath{fill}%
\end{pgfscope}%
\begin{pgfscope}%
\pgfpathrectangle{\pgfqpoint{0.041670in}{0.041670in}}{\pgfqpoint{2.216660in}{2.216660in}}%
\pgfusepath{clip}%
\pgfsetbuttcap%
\pgfsetroundjoin%
\definecolor{currentfill}{rgb}{0.268510,0.009605,0.335427}%
\pgfsetfillcolor{currentfill}%
\pgfsetlinewidth{0.000000pt}%
\definecolor{currentstroke}{rgb}{0.000000,0.000000,0.000000}%
\pgfsetstrokecolor{currentstroke}%
\pgfsetdash{}{0pt}%
\pgfpathmoveto{\pgfqpoint{0.655239in}{0.969785in}}%
\pgfpathlineto{\pgfqpoint{0.651778in}{0.968767in}}%
\pgfpathlineto{\pgfqpoint{0.648313in}{0.967965in}}%
\pgfpathlineto{\pgfqpoint{0.644843in}{0.967384in}}%
\pgfpathlineto{\pgfqpoint{0.641367in}{0.967029in}}%
\pgfpathlineto{\pgfqpoint{0.647725in}{0.975737in}}%
\pgfpathlineto{\pgfqpoint{0.654591in}{0.984327in}}%
\pgfpathlineto{\pgfqpoint{0.661955in}{0.992791in}}%
\pgfpathlineto{\pgfqpoint{0.669809in}{1.001122in}}%
\pgfpathlineto{\pgfqpoint{0.673085in}{1.001257in}}%
\pgfpathlineto{\pgfqpoint{0.676356in}{1.001616in}}%
\pgfpathlineto{\pgfqpoint{0.679624in}{1.002196in}}%
\pgfpathlineto{\pgfqpoint{0.682887in}{1.002991in}}%
\pgfpathlineto{\pgfqpoint{0.675249in}{0.994876in}}%
\pgfpathlineto{\pgfqpoint{0.668090in}{0.986632in}}%
\pgfpathlineto{\pgfqpoint{0.661417in}{0.978265in}}%
\pgfpathlineto{\pgfqpoint{0.655239in}{0.969785in}}%
\pgfpathclose%
\pgfusepath{fill}%
\end{pgfscope}%
\begin{pgfscope}%
\pgfpathrectangle{\pgfqpoint{0.041670in}{0.041670in}}{\pgfqpoint{2.216660in}{2.216660in}}%
\pgfusepath{clip}%
\pgfsetbuttcap%
\pgfsetroundjoin%
\definecolor{currentfill}{rgb}{0.195860,0.395433,0.555276}%
\pgfsetfillcolor{currentfill}%
\pgfsetlinewidth{0.000000pt}%
\definecolor{currentstroke}{rgb}{0.000000,0.000000,0.000000}%
\pgfsetstrokecolor{currentstroke}%
\pgfsetdash{}{0pt}%
\pgfpathmoveto{\pgfqpoint{1.273447in}{1.317529in}}%
\pgfpathlineto{\pgfqpoint{1.274325in}{1.311981in}}%
\pgfpathlineto{\pgfqpoint{1.275201in}{1.306473in}}%
\pgfpathlineto{\pgfqpoint{1.276077in}{1.301008in}}%
\pgfpathlineto{\pgfqpoint{1.276951in}{1.295590in}}%
\pgfpathlineto{\pgfqpoint{1.288707in}{1.293992in}}%
\pgfpathlineto{\pgfqpoint{1.300363in}{1.292213in}}%
\pgfpathlineto{\pgfqpoint{1.311908in}{1.290253in}}%
\pgfpathlineto{\pgfqpoint{1.310717in}{1.295724in}}%
\pgfpathlineto{\pgfqpoint{1.309525in}{1.301241in}}%
\pgfpathlineto{\pgfqpoint{1.308332in}{1.306802in}}%
\pgfpathlineto{\pgfqpoint{1.307137in}{1.312402in}}%
\pgfpathlineto{\pgfqpoint{1.296011in}{1.314285in}}%
\pgfpathlineto{\pgfqpoint{1.284777in}{1.315994in}}%
\pgfpathlineto{\pgfqpoint{1.273447in}{1.317529in}}%
\pgfpathclose%
\pgfusepath{fill}%
\end{pgfscope}%
\begin{pgfscope}%
\pgfpathrectangle{\pgfqpoint{0.041670in}{0.041670in}}{\pgfqpoint{2.216660in}{2.216660in}}%
\pgfusepath{clip}%
\pgfsetbuttcap%
\pgfsetroundjoin%
\definecolor{currentfill}{rgb}{0.195860,0.395433,0.555276}%
\pgfsetfillcolor{currentfill}%
\pgfsetlinewidth{0.000000pt}%
\definecolor{currentstroke}{rgb}{0.000000,0.000000,0.000000}%
\pgfsetstrokecolor{currentstroke}%
\pgfsetdash{}{0pt}%
\pgfpathmoveto{\pgfqpoint{1.042980in}{1.310584in}}%
\pgfpathlineto{\pgfqpoint{1.041693in}{1.304965in}}%
\pgfpathlineto{\pgfqpoint{1.040408in}{1.299386in}}%
\pgfpathlineto{\pgfqpoint{1.039124in}{1.293850in}}%
\pgfpathlineto{\pgfqpoint{1.037841in}{1.288360in}}%
\pgfpathlineto{\pgfqpoint{1.049279in}{1.290480in}}%
\pgfpathlineto{\pgfqpoint{1.060837in}{1.292420in}}%
\pgfpathlineto{\pgfqpoint{1.072504in}{1.294179in}}%
\pgfpathlineto{\pgfqpoint{1.084270in}{1.295756in}}%
\pgfpathlineto{\pgfqpoint{1.085133in}{1.301173in}}%
\pgfpathlineto{\pgfqpoint{1.085997in}{1.306636in}}%
\pgfpathlineto{\pgfqpoint{1.086861in}{1.312142in}}%
\pgfpathlineto{\pgfqpoint{1.087727in}{1.317689in}}%
\pgfpathlineto{\pgfqpoint{1.076387in}{1.316174in}}%
\pgfpathlineto{\pgfqpoint{1.065142in}{1.314483in}}%
\pgfpathlineto{\pgfqpoint{1.054003in}{1.312620in}}%
\pgfpathlineto{\pgfqpoint{1.042980in}{1.310584in}}%
\pgfpathclose%
\pgfusepath{fill}%
\end{pgfscope}%
\begin{pgfscope}%
\pgfpathrectangle{\pgfqpoint{0.041670in}{0.041670in}}{\pgfqpoint{2.216660in}{2.216660in}}%
\pgfusepath{clip}%
\pgfsetbuttcap%
\pgfsetroundjoin%
\definecolor{currentfill}{rgb}{0.283072,0.130895,0.449241}%
\pgfsetfillcolor{currentfill}%
\pgfsetlinewidth{0.000000pt}%
\definecolor{currentstroke}{rgb}{0.000000,0.000000,0.000000}%
\pgfsetstrokecolor{currentstroke}%
\pgfsetdash{}{0pt}%
\pgfpathmoveto{\pgfqpoint{1.574381in}{1.090183in}}%
\pgfpathlineto{\pgfqpoint{1.577306in}{1.086205in}}%
\pgfpathlineto{\pgfqpoint{1.580232in}{1.082362in}}%
\pgfpathlineto{\pgfqpoint{1.583158in}{1.078657in}}%
\pgfpathlineto{\pgfqpoint{1.586085in}{1.075093in}}%
\pgfpathlineto{\pgfqpoint{1.594860in}{1.068511in}}%
\pgfpathlineto{\pgfqpoint{1.603246in}{1.061784in}}%
\pgfpathlineto{\pgfqpoint{1.611233in}{1.054919in}}%
\pgfpathlineto{\pgfqpoint{1.618812in}{1.047923in}}%
\pgfpathlineto{\pgfqpoint{1.615638in}{1.051696in}}%
\pgfpathlineto{\pgfqpoint{1.612464in}{1.055612in}}%
\pgfpathlineto{\pgfqpoint{1.609290in}{1.059667in}}%
\pgfpathlineto{\pgfqpoint{1.606117in}{1.063856in}}%
\pgfpathlineto{\pgfqpoint{1.598769in}{1.070635in}}%
\pgfpathlineto{\pgfqpoint{1.591025in}{1.077286in}}%
\pgfpathlineto{\pgfqpoint{1.582892in}{1.083804in}}%
\pgfpathlineto{\pgfqpoint{1.574381in}{1.090183in}}%
\pgfpathclose%
\pgfusepath{fill}%
\end{pgfscope}%
\begin{pgfscope}%
\pgfpathrectangle{\pgfqpoint{0.041670in}{0.041670in}}{\pgfqpoint{2.216660in}{2.216660in}}%
\pgfusepath{clip}%
\pgfsetbuttcap%
\pgfsetroundjoin%
\definecolor{currentfill}{rgb}{0.231674,0.318106,0.544834}%
\pgfsetfillcolor{currentfill}%
\pgfsetlinewidth{0.000000pt}%
\definecolor{currentstroke}{rgb}{0.000000,0.000000,0.000000}%
\pgfsetstrokecolor{currentstroke}%
\pgfsetdash{}{0pt}%
\pgfpathmoveto{\pgfqpoint{1.406944in}{1.246070in}}%
\pgfpathlineto{\pgfqpoint{1.408919in}{1.240660in}}%
\pgfpathlineto{\pgfqpoint{1.410892in}{1.235315in}}%
\pgfpathlineto{\pgfqpoint{1.412864in}{1.230038in}}%
\pgfpathlineto{\pgfqpoint{1.414835in}{1.224833in}}%
\pgfpathlineto{\pgfqpoint{1.425652in}{1.221063in}}%
\pgfpathlineto{\pgfqpoint{1.436242in}{1.217122in}}%
\pgfpathlineto{\pgfqpoint{1.446596in}{1.213015in}}%
\pgfpathlineto{\pgfqpoint{1.456702in}{1.208745in}}%
\pgfpathlineto{\pgfqpoint{1.454374in}{1.214098in}}%
\pgfpathlineto{\pgfqpoint{1.452046in}{1.219522in}}%
\pgfpathlineto{\pgfqpoint{1.449715in}{1.225014in}}%
\pgfpathlineto{\pgfqpoint{1.447383in}{1.230572in}}%
\pgfpathlineto{\pgfqpoint{1.437623in}{1.234686in}}%
\pgfpathlineto{\pgfqpoint{1.427623in}{1.238642in}}%
\pgfpathlineto{\pgfqpoint{1.417394in}{1.242438in}}%
\pgfpathlineto{\pgfqpoint{1.406944in}{1.246070in}}%
\pgfpathclose%
\pgfusepath{fill}%
\end{pgfscope}%
\begin{pgfscope}%
\pgfpathrectangle{\pgfqpoint{0.041670in}{0.041670in}}{\pgfqpoint{2.216660in}{2.216660in}}%
\pgfusepath{clip}%
\pgfsetbuttcap%
\pgfsetroundjoin%
\definecolor{currentfill}{rgb}{0.276194,0.190074,0.493001}%
\pgfsetfillcolor{currentfill}%
\pgfsetlinewidth{0.000000pt}%
\definecolor{currentstroke}{rgb}{0.000000,0.000000,0.000000}%
\pgfsetstrokecolor{currentstroke}%
\pgfsetdash{}{0pt}%
\pgfpathmoveto{\pgfqpoint{1.775777in}{1.116671in}}%
\pgfpathlineto{\pgfqpoint{1.779240in}{1.125255in}}%
\pgfpathlineto{\pgfqpoint{1.782716in}{1.134222in}}%
\pgfpathlineto{\pgfqpoint{1.786204in}{1.143578in}}%
\pgfpathlineto{\pgfqpoint{1.789707in}{1.153330in}}%
\pgfpathlineto{\pgfqpoint{1.799904in}{1.143578in}}%
\pgfpathlineto{\pgfqpoint{1.809524in}{1.133656in}}%
\pgfpathlineto{\pgfqpoint{1.818556in}{1.123570in}}%
\pgfpathlineto{\pgfqpoint{1.826988in}{1.113331in}}%
\pgfpathlineto{\pgfqpoint{1.823250in}{1.103754in}}%
\pgfpathlineto{\pgfqpoint{1.819527in}{1.094575in}}%
\pgfpathlineto{\pgfqpoint{1.815818in}{1.085788in}}%
\pgfpathlineto{\pgfqpoint{1.812124in}{1.077385in}}%
\pgfpathlineto{\pgfqpoint{1.803906in}{1.087441in}}%
\pgfpathlineto{\pgfqpoint{1.795102in}{1.097346in}}%
\pgfpathlineto{\pgfqpoint{1.785722in}{1.107092in}}%
\pgfpathlineto{\pgfqpoint{1.775777in}{1.116671in}}%
\pgfpathclose%
\pgfusepath{fill}%
\end{pgfscope}%
\begin{pgfscope}%
\pgfpathrectangle{\pgfqpoint{0.041670in}{0.041670in}}{\pgfqpoint{2.216660in}{2.216660in}}%
\pgfusepath{clip}%
\pgfsetbuttcap%
\pgfsetroundjoin%
\definecolor{currentfill}{rgb}{0.212395,0.359683,0.551710}%
\pgfsetfillcolor{currentfill}%
\pgfsetlinewidth{0.000000pt}%
\definecolor{currentstroke}{rgb}{0.000000,0.000000,0.000000}%
\pgfsetstrokecolor{currentstroke}%
\pgfsetdash{}{0pt}%
\pgfpathmoveto{\pgfqpoint{1.356771in}{1.280641in}}%
\pgfpathlineto{\pgfqpoint{1.358368in}{1.275124in}}%
\pgfpathlineto{\pgfqpoint{1.359963in}{1.269660in}}%
\pgfpathlineto{\pgfqpoint{1.361556in}{1.264251in}}%
\pgfpathlineto{\pgfqpoint{1.363148in}{1.258901in}}%
\pgfpathlineto{\pgfqpoint{1.374376in}{1.255954in}}%
\pgfpathlineto{\pgfqpoint{1.385425in}{1.252831in}}%
\pgfpathlineto{\pgfqpoint{1.396284in}{1.249535in}}%
\pgfpathlineto{\pgfqpoint{1.406944in}{1.246070in}}%
\pgfpathlineto{\pgfqpoint{1.404967in}{1.251542in}}%
\pgfpathlineto{\pgfqpoint{1.402989in}{1.257072in}}%
\pgfpathlineto{\pgfqpoint{1.401010in}{1.262658in}}%
\pgfpathlineto{\pgfqpoint{1.399028in}{1.268297in}}%
\pgfpathlineto{\pgfqpoint{1.388744in}{1.271630in}}%
\pgfpathlineto{\pgfqpoint{1.378266in}{1.274801in}}%
\pgfpathlineto{\pgfqpoint{1.367605in}{1.277805in}}%
\pgfpathlineto{\pgfqpoint{1.356771in}{1.280641in}}%
\pgfpathclose%
\pgfusepath{fill}%
\end{pgfscope}%
\begin{pgfscope}%
\pgfpathrectangle{\pgfqpoint{0.041670in}{0.041670in}}{\pgfqpoint{2.216660in}{2.216660in}}%
\pgfusepath{clip}%
\pgfsetbuttcap%
\pgfsetroundjoin%
\definecolor{currentfill}{rgb}{0.282884,0.135920,0.453427}%
\pgfsetfillcolor{currentfill}%
\pgfsetlinewidth{0.000000pt}%
\definecolor{currentstroke}{rgb}{0.000000,0.000000,0.000000}%
\pgfsetstrokecolor{currentstroke}%
\pgfsetdash{}{0pt}%
\pgfpathmoveto{\pgfqpoint{0.555807in}{1.038613in}}%
\pgfpathlineto{\pgfqpoint{0.552120in}{1.045495in}}%
\pgfpathlineto{\pgfqpoint{0.548421in}{1.052737in}}%
\pgfpathlineto{\pgfqpoint{0.544709in}{1.060346in}}%
\pgfpathlineto{\pgfqpoint{0.540983in}{1.068328in}}%
\pgfpathlineto{\pgfqpoint{0.548670in}{1.078509in}}%
\pgfpathlineto{\pgfqpoint{0.556953in}{1.088549in}}%
\pgfpathlineto{\pgfqpoint{0.565821in}{1.098437in}}%
\pgfpathlineto{\pgfqpoint{0.575265in}{1.108165in}}%
\pgfpathlineto{\pgfqpoint{0.578771in}{1.099997in}}%
\pgfpathlineto{\pgfqpoint{0.582264in}{1.092201in}}%
\pgfpathlineto{\pgfqpoint{0.585746in}{1.084769in}}%
\pgfpathlineto{\pgfqpoint{0.589216in}{1.077696in}}%
\pgfpathlineto{\pgfqpoint{0.580009in}{1.068151in}}%
\pgfpathlineto{\pgfqpoint{0.571366in}{1.058450in}}%
\pgfpathlineto{\pgfqpoint{0.563295in}{1.048601in}}%
\pgfpathlineto{\pgfqpoint{0.555807in}{1.038613in}}%
\pgfpathclose%
\pgfusepath{fill}%
\end{pgfscope}%
\begin{pgfscope}%
\pgfpathrectangle{\pgfqpoint{0.041670in}{0.041670in}}{\pgfqpoint{2.216660in}{2.216660in}}%
\pgfusepath{clip}%
\pgfsetbuttcap%
\pgfsetroundjoin%
\definecolor{currentfill}{rgb}{0.274952,0.037752,0.364543}%
\pgfsetfillcolor{currentfill}%
\pgfsetlinewidth{0.000000pt}%
\definecolor{currentstroke}{rgb}{0.000000,0.000000,0.000000}%
\pgfsetstrokecolor{currentstroke}%
\pgfsetdash{}{0pt}%
\pgfpathmoveto{\pgfqpoint{1.644255in}{1.023318in}}%
\pgfpathlineto{\pgfqpoint{1.647443in}{1.021000in}}%
\pgfpathlineto{\pgfqpoint{1.650634in}{1.018863in}}%
\pgfpathlineto{\pgfqpoint{1.653828in}{1.016912in}}%
\pgfpathlineto{\pgfqpoint{1.657024in}{1.015151in}}%
\pgfpathlineto{\pgfqpoint{1.664854in}{1.007372in}}%
\pgfpathlineto{\pgfqpoint{1.672226in}{0.999461in}}%
\pgfpathlineto{\pgfqpoint{1.679130in}{0.991426in}}%
\pgfpathlineto{\pgfqpoint{1.685559in}{0.983273in}}%
\pgfpathlineto{\pgfqpoint{1.682156in}{0.985257in}}%
\pgfpathlineto{\pgfqpoint{1.678757in}{0.987431in}}%
\pgfpathlineto{\pgfqpoint{1.675361in}{0.989791in}}%
\pgfpathlineto{\pgfqpoint{1.671967in}{0.992333in}}%
\pgfpathlineto{\pgfqpoint{1.665726in}{1.000257in}}%
\pgfpathlineto{\pgfqpoint{1.659022in}{1.008067in}}%
\pgfpathlineto{\pgfqpoint{1.651862in}{1.015756in}}%
\pgfpathlineto{\pgfqpoint{1.644255in}{1.023318in}}%
\pgfpathclose%
\pgfusepath{fill}%
\end{pgfscope}%
\begin{pgfscope}%
\pgfpathrectangle{\pgfqpoint{0.041670in}{0.041670in}}{\pgfqpoint{2.216660in}{2.216660in}}%
\pgfusepath{clip}%
\pgfsetbuttcap%
\pgfsetroundjoin%
\definecolor{currentfill}{rgb}{0.274128,0.199721,0.498911}%
\pgfsetfillcolor{currentfill}%
\pgfsetlinewidth{0.000000pt}%
\definecolor{currentstroke}{rgb}{0.000000,0.000000,0.000000}%
\pgfsetstrokecolor{currentstroke}%
\pgfsetdash{}{0pt}%
\pgfpathmoveto{\pgfqpoint{0.801823in}{1.121110in}}%
\pgfpathlineto{\pgfqpoint{0.798837in}{1.116142in}}%
\pgfpathlineto{\pgfqpoint{0.795851in}{1.111279in}}%
\pgfpathlineto{\pgfqpoint{0.792866in}{1.106526in}}%
\pgfpathlineto{\pgfqpoint{0.789881in}{1.101885in}}%
\pgfpathlineto{\pgfqpoint{0.798170in}{1.108044in}}%
\pgfpathlineto{\pgfqpoint{0.806817in}{1.114063in}}%
\pgfpathlineto{\pgfqpoint{0.815815in}{1.119936in}}%
\pgfpathlineto{\pgfqpoint{0.825152in}{1.125658in}}%
\pgfpathlineto{\pgfqpoint{0.827858in}{1.130102in}}%
\pgfpathlineto{\pgfqpoint{0.830564in}{1.134659in}}%
\pgfpathlineto{\pgfqpoint{0.833271in}{1.139324in}}%
\pgfpathlineto{\pgfqpoint{0.835979in}{1.144095in}}%
\pgfpathlineto{\pgfqpoint{0.826935in}{1.138562in}}%
\pgfpathlineto{\pgfqpoint{0.818222in}{1.132884in}}%
\pgfpathlineto{\pgfqpoint{0.809848in}{1.127065in}}%
\pgfpathlineto{\pgfqpoint{0.801823in}{1.121110in}}%
\pgfpathclose%
\pgfusepath{fill}%
\end{pgfscope}%
\begin{pgfscope}%
\pgfpathrectangle{\pgfqpoint{0.041670in}{0.041670in}}{\pgfqpoint{2.216660in}{2.216660in}}%
\pgfusepath{clip}%
\pgfsetbuttcap%
\pgfsetroundjoin%
\definecolor{currentfill}{rgb}{0.248629,0.278775,0.534556}%
\pgfsetfillcolor{currentfill}%
\pgfsetlinewidth{0.000000pt}%
\definecolor{currentstroke}{rgb}{0.000000,0.000000,0.000000}%
\pgfsetstrokecolor{currentstroke}%
\pgfsetdash{}{0pt}%
\pgfpathmoveto{\pgfqpoint{1.456702in}{1.208745in}}%
\pgfpathlineto{\pgfqpoint{1.459027in}{1.203467in}}%
\pgfpathlineto{\pgfqpoint{1.461351in}{1.198267in}}%
\pgfpathlineto{\pgfqpoint{1.463674in}{1.193149in}}%
\pgfpathlineto{\pgfqpoint{1.465996in}{1.188115in}}%
\pgfpathlineto{\pgfqpoint{1.476182in}{1.183522in}}%
\pgfpathlineto{\pgfqpoint{1.486093in}{1.178768in}}%
\pgfpathlineto{\pgfqpoint{1.495719in}{1.173856in}}%
\pgfpathlineto{\pgfqpoint{1.505051in}{1.168791in}}%
\pgfpathlineto{\pgfqpoint{1.502405in}{1.173996in}}%
\pgfpathlineto{\pgfqpoint{1.499758in}{1.179286in}}%
\pgfpathlineto{\pgfqpoint{1.497109in}{1.184657in}}%
\pgfpathlineto{\pgfqpoint{1.494459in}{1.190106in}}%
\pgfpathlineto{\pgfqpoint{1.485439in}{1.194992in}}%
\pgfpathlineto{\pgfqpoint{1.476133in}{1.199729in}}%
\pgfpathlineto{\pgfqpoint{1.466550in}{1.204315in}}%
\pgfpathlineto{\pgfqpoint{1.456702in}{1.208745in}}%
\pgfpathclose%
\pgfusepath{fill}%
\end{pgfscope}%
\begin{pgfscope}%
\pgfpathrectangle{\pgfqpoint{0.041670in}{0.041670in}}{\pgfqpoint{2.216660in}{2.216660in}}%
\pgfusepath{clip}%
\pgfsetbuttcap%
\pgfsetroundjoin%
\definecolor{currentfill}{rgb}{0.271305,0.019942,0.347269}%
\pgfsetfillcolor{currentfill}%
\pgfsetlinewidth{0.000000pt}%
\definecolor{currentstroke}{rgb}{0.000000,0.000000,0.000000}%
\pgfsetstrokecolor{currentstroke}%
\pgfsetdash{}{0pt}%
\pgfpathmoveto{\pgfqpoint{0.669042in}{0.975934in}}%
\pgfpathlineto{\pgfqpoint{0.665597in}{0.974094in}}%
\pgfpathlineto{\pgfqpoint{0.662148in}{0.972453in}}%
\pgfpathlineto{\pgfqpoint{0.658695in}{0.971015in}}%
\pgfpathlineto{\pgfqpoint{0.655239in}{0.969785in}}%
\pgfpathlineto{\pgfqpoint{0.661417in}{0.978265in}}%
\pgfpathlineto{\pgfqpoint{0.668090in}{0.986632in}}%
\pgfpathlineto{\pgfqpoint{0.675249in}{0.994876in}}%
\pgfpathlineto{\pgfqpoint{0.682887in}{1.002991in}}%
\pgfpathlineto{\pgfqpoint{0.686146in}{1.003998in}}%
\pgfpathlineto{\pgfqpoint{0.689402in}{1.005211in}}%
\pgfpathlineto{\pgfqpoint{0.692654in}{1.006628in}}%
\pgfpathlineto{\pgfqpoint{0.695903in}{1.008243in}}%
\pgfpathlineto{\pgfqpoint{0.688480in}{1.000346in}}%
\pgfpathlineto{\pgfqpoint{0.681524in}{0.992325in}}%
\pgfpathlineto{\pgfqpoint{0.675042in}{0.984185in}}%
\pgfpathlineto{\pgfqpoint{0.669042in}{0.975934in}}%
\pgfpathclose%
\pgfusepath{fill}%
\end{pgfscope}%
\begin{pgfscope}%
\pgfpathrectangle{\pgfqpoint{0.041670in}{0.041670in}}{\pgfqpoint{2.216660in}{2.216660in}}%
\pgfusepath{clip}%
\pgfsetbuttcap%
\pgfsetroundjoin%
\definecolor{currentfill}{rgb}{0.231674,0.318106,0.544834}%
\pgfsetfillcolor{currentfill}%
\pgfsetlinewidth{0.000000pt}%
\definecolor{currentstroke}{rgb}{0.000000,0.000000,0.000000}%
\pgfsetstrokecolor{currentstroke}%
\pgfsetdash{}{0pt}%
\pgfpathmoveto{\pgfqpoint{0.904059in}{1.226786in}}%
\pgfpathlineto{\pgfqpoint{0.901652in}{1.221193in}}%
\pgfpathlineto{\pgfqpoint{0.899247in}{1.215664in}}%
\pgfpathlineto{\pgfqpoint{0.896843in}{1.210204in}}%
\pgfpathlineto{\pgfqpoint{0.894441in}{1.204815in}}%
\pgfpathlineto{\pgfqpoint{0.904319in}{1.209227in}}%
\pgfpathlineto{\pgfqpoint{0.914452in}{1.213480in}}%
\pgfpathlineto{\pgfqpoint{0.924833in}{1.217568in}}%
\pgfpathlineto{\pgfqpoint{0.935449in}{1.221490in}}%
\pgfpathlineto{\pgfqpoint{0.937501in}{1.226726in}}%
\pgfpathlineto{\pgfqpoint{0.939555in}{1.232033in}}%
\pgfpathlineto{\pgfqpoint{0.941610in}{1.237409in}}%
\pgfpathlineto{\pgfqpoint{0.943667in}{1.242850in}}%
\pgfpathlineto{\pgfqpoint{0.933412in}{1.239072in}}%
\pgfpathlineto{\pgfqpoint{0.923386in}{1.235133in}}%
\pgfpathlineto{\pgfqpoint{0.913599in}{1.231037in}}%
\pgfpathlineto{\pgfqpoint{0.904059in}{1.226786in}}%
\pgfpathclose%
\pgfusepath{fill}%
\end{pgfscope}%
\begin{pgfscope}%
\pgfpathrectangle{\pgfqpoint{0.041670in}{0.041670in}}{\pgfqpoint{2.216660in}{2.216660in}}%
\pgfusepath{clip}%
\pgfsetbuttcap%
\pgfsetroundjoin%
\definecolor{currentfill}{rgb}{0.201239,0.383670,0.554294}%
\pgfsetfillcolor{currentfill}%
\pgfsetlinewidth{0.000000pt}%
\definecolor{currentstroke}{rgb}{0.000000,0.000000,0.000000}%
\pgfsetstrokecolor{currentstroke}%
\pgfsetdash{}{0pt}%
\pgfpathmoveto{\pgfqpoint{0.532096in}{1.237463in}}%
\pgfpathlineto{\pgfqpoint{0.528395in}{1.251053in}}%
\pgfpathlineto{\pgfqpoint{0.524676in}{1.265109in}}%
\pgfpathlineto{\pgfqpoint{0.520938in}{1.279639in}}%
\pgfpathlineto{\pgfqpoint{0.517182in}{1.294652in}}%
\pgfpathlineto{\pgfqpoint{0.528236in}{1.304849in}}%
\pgfpathlineto{\pgfqpoint{0.539890in}{1.314855in}}%
\pgfpathlineto{\pgfqpoint{0.552132in}{1.324661in}}%
\pgfpathlineto{\pgfqpoint{0.564946in}{1.334260in}}%
\pgfpathlineto{\pgfqpoint{0.568412in}{1.319111in}}%
\pgfpathlineto{\pgfqpoint{0.571860in}{1.304442in}}%
\pgfpathlineto{\pgfqpoint{0.575292in}{1.290244in}}%
\pgfpathlineto{\pgfqpoint{0.578707in}{1.276511in}}%
\pgfpathlineto{\pgfqpoint{0.566198in}{1.267047in}}%
\pgfpathlineto{\pgfqpoint{0.554251in}{1.257379in}}%
\pgfpathlineto{\pgfqpoint{0.542880in}{1.247515in}}%
\pgfpathlineto{\pgfqpoint{0.532096in}{1.237463in}}%
\pgfpathclose%
\pgfusepath{fill}%
\end{pgfscope}%
\begin{pgfscope}%
\pgfpathrectangle{\pgfqpoint{0.041670in}{0.041670in}}{\pgfqpoint{2.216660in}{2.216660in}}%
\pgfusepath{clip}%
\pgfsetbuttcap%
\pgfsetroundjoin%
\definecolor{currentfill}{rgb}{0.212395,0.359683,0.551710}%
\pgfsetfillcolor{currentfill}%
\pgfsetlinewidth{0.000000pt}%
\definecolor{currentstroke}{rgb}{0.000000,0.000000,0.000000}%
\pgfsetstrokecolor{currentstroke}%
\pgfsetdash{}{0pt}%
\pgfpathmoveto{\pgfqpoint{0.951911in}{1.265199in}}%
\pgfpathlineto{\pgfqpoint{0.949847in}{1.259530in}}%
\pgfpathlineto{\pgfqpoint{0.947785in}{1.253913in}}%
\pgfpathlineto{\pgfqpoint{0.945725in}{1.248352in}}%
\pgfpathlineto{\pgfqpoint{0.943667in}{1.242850in}}%
\pgfpathlineto{\pgfqpoint{0.954140in}{1.246463in}}%
\pgfpathlineto{\pgfqpoint{0.964822in}{1.249910in}}%
\pgfpathlineto{\pgfqpoint{0.975703in}{1.253186in}}%
\pgfpathlineto{\pgfqpoint{0.986772in}{1.256290in}}%
\pgfpathlineto{\pgfqpoint{0.988452in}{1.261665in}}%
\pgfpathlineto{\pgfqpoint{0.990133in}{1.267098in}}%
\pgfpathlineto{\pgfqpoint{0.991816in}{1.272587in}}%
\pgfpathlineto{\pgfqpoint{0.993500in}{1.278128in}}%
\pgfpathlineto{\pgfqpoint{0.982819in}{1.275143in}}%
\pgfpathlineto{\pgfqpoint{0.972321in}{1.271991in}}%
\pgfpathlineto{\pgfqpoint{0.962015in}{1.268675in}}%
\pgfpathlineto{\pgfqpoint{0.951911in}{1.265199in}}%
\pgfpathclose%
\pgfusepath{fill}%
\end{pgfscope}%
\begin{pgfscope}%
\pgfpathrectangle{\pgfqpoint{0.041670in}{0.041670in}}{\pgfqpoint{2.216660in}{2.216660in}}%
\pgfusepath{clip}%
\pgfsetbuttcap%
\pgfsetroundjoin%
\definecolor{currentfill}{rgb}{0.283072,0.130895,0.449241}%
\pgfsetfillcolor{currentfill}%
\pgfsetlinewidth{0.000000pt}%
\definecolor{currentstroke}{rgb}{0.000000,0.000000,0.000000}%
\pgfsetstrokecolor{currentstroke}%
\pgfsetdash{}{0pt}%
\pgfpathmoveto{\pgfqpoint{0.747600in}{1.057728in}}%
\pgfpathlineto{\pgfqpoint{0.744378in}{1.053489in}}%
\pgfpathlineto{\pgfqpoint{0.741156in}{1.049386in}}%
\pgfpathlineto{\pgfqpoint{0.737933in}{1.045421in}}%
\pgfpathlineto{\pgfqpoint{0.734709in}{1.041598in}}%
\pgfpathlineto{\pgfqpoint{0.741919in}{1.048707in}}%
\pgfpathlineto{\pgfqpoint{0.749545in}{1.055689in}}%
\pgfpathlineto{\pgfqpoint{0.757576in}{1.062538in}}%
\pgfpathlineto{\pgfqpoint{0.766006in}{1.069249in}}%
\pgfpathlineto{\pgfqpoint{0.768991in}{1.072858in}}%
\pgfpathlineto{\pgfqpoint{0.771976in}{1.076608in}}%
\pgfpathlineto{\pgfqpoint{0.774961in}{1.080497in}}%
\pgfpathlineto{\pgfqpoint{0.777945in}{1.084520in}}%
\pgfpathlineto{\pgfqpoint{0.769770in}{1.078017in}}%
\pgfpathlineto{\pgfqpoint{0.761982in}{1.071380in}}%
\pgfpathlineto{\pgfqpoint{0.754589in}{1.064615in}}%
\pgfpathlineto{\pgfqpoint{0.747600in}{1.057728in}}%
\pgfpathclose%
\pgfusepath{fill}%
\end{pgfscope}%
\begin{pgfscope}%
\pgfpathrectangle{\pgfqpoint{0.041670in}{0.041670in}}{\pgfqpoint{2.216660in}{2.216660in}}%
\pgfusepath{clip}%
\pgfsetbuttcap%
\pgfsetroundjoin%
\definecolor{currentfill}{rgb}{0.195860,0.395433,0.555276}%
\pgfsetfillcolor{currentfill}%
\pgfsetlinewidth{0.000000pt}%
\definecolor{currentstroke}{rgb}{0.000000,0.000000,0.000000}%
\pgfsetstrokecolor{currentstroke}%
\pgfsetdash{}{0pt}%
\pgfpathmoveto{\pgfqpoint{1.307137in}{1.312402in}}%
\pgfpathlineto{\pgfqpoint{1.308332in}{1.306802in}}%
\pgfpathlineto{\pgfqpoint{1.309525in}{1.301241in}}%
\pgfpathlineto{\pgfqpoint{1.310717in}{1.295724in}}%
\pgfpathlineto{\pgfqpoint{1.311908in}{1.290253in}}%
\pgfpathlineto{\pgfqpoint{1.323332in}{1.288114in}}%
\pgfpathlineto{\pgfqpoint{1.334624in}{1.285797in}}%
\pgfpathlineto{\pgfqpoint{1.345774in}{1.283306in}}%
\pgfpathlineto{\pgfqpoint{1.356771in}{1.280641in}}%
\pgfpathlineto{\pgfqpoint{1.355173in}{1.286206in}}%
\pgfpathlineto{\pgfqpoint{1.353574in}{1.291818in}}%
\pgfpathlineto{\pgfqpoint{1.351973in}{1.297473in}}%
\pgfpathlineto{\pgfqpoint{1.350370in}{1.303168in}}%
\pgfpathlineto{\pgfqpoint{1.339773in}{1.305728in}}%
\pgfpathlineto{\pgfqpoint{1.329029in}{1.308121in}}%
\pgfpathlineto{\pgfqpoint{1.318147in}{1.310347in}}%
\pgfpathlineto{\pgfqpoint{1.307137in}{1.312402in}}%
\pgfpathclose%
\pgfusepath{fill}%
\end{pgfscope}%
\begin{pgfscope}%
\pgfpathrectangle{\pgfqpoint{0.041670in}{0.041670in}}{\pgfqpoint{2.216660in}{2.216660in}}%
\pgfusepath{clip}%
\pgfsetbuttcap%
\pgfsetroundjoin%
\definecolor{currentfill}{rgb}{0.248629,0.278775,0.534556}%
\pgfsetfillcolor{currentfill}%
\pgfsetlinewidth{0.000000pt}%
\definecolor{currentstroke}{rgb}{0.000000,0.000000,0.000000}%
\pgfsetstrokecolor{currentstroke}%
\pgfsetdash{}{0pt}%
\pgfpathmoveto{\pgfqpoint{0.857680in}{1.185643in}}%
\pgfpathlineto{\pgfqpoint{0.854962in}{1.180153in}}%
\pgfpathlineto{\pgfqpoint{0.852246in}{1.174740in}}%
\pgfpathlineto{\pgfqpoint{0.849532in}{1.169409in}}%
\pgfpathlineto{\pgfqpoint{0.846819in}{1.164163in}}%
\pgfpathlineto{\pgfqpoint{0.855881in}{1.169361in}}%
\pgfpathlineto{\pgfqpoint{0.865246in}{1.174409in}}%
\pgfpathlineto{\pgfqpoint{0.874904in}{1.179304in}}%
\pgfpathlineto{\pgfqpoint{0.884846in}{1.184041in}}%
\pgfpathlineto{\pgfqpoint{0.887243in}{1.189110in}}%
\pgfpathlineto{\pgfqpoint{0.889641in}{1.194265in}}%
\pgfpathlineto{\pgfqpoint{0.892040in}{1.199501in}}%
\pgfpathlineto{\pgfqpoint{0.894441in}{1.204815in}}%
\pgfpathlineto{\pgfqpoint{0.884828in}{1.200246in}}%
\pgfpathlineto{\pgfqpoint{0.875491in}{1.195525in}}%
\pgfpathlineto{\pgfqpoint{0.866439in}{1.190656in}}%
\pgfpathlineto{\pgfqpoint{0.857680in}{1.185643in}}%
\pgfpathclose%
\pgfusepath{fill}%
\end{pgfscope}%
\begin{pgfscope}%
\pgfpathrectangle{\pgfqpoint{0.041670in}{0.041670in}}{\pgfqpoint{2.216660in}{2.216660in}}%
\pgfusepath{clip}%
\pgfsetbuttcap%
\pgfsetroundjoin%
\definecolor{currentfill}{rgb}{0.179019,0.433756,0.557430}%
\pgfsetfillcolor{currentfill}%
\pgfsetlinewidth{0.000000pt}%
\definecolor{currentstroke}{rgb}{0.000000,0.000000,0.000000}%
\pgfsetstrokecolor{currentstroke}%
\pgfsetdash{}{0pt}%
\pgfpathmoveto{\pgfqpoint{1.135563in}{1.344324in}}%
\pgfpathlineto{\pgfqpoint{1.135129in}{1.338691in}}%
\pgfpathlineto{\pgfqpoint{1.134695in}{1.333087in}}%
\pgfpathlineto{\pgfqpoint{1.134261in}{1.327513in}}%
\pgfpathlineto{\pgfqpoint{1.133828in}{1.321974in}}%
\pgfpathlineto{\pgfqpoint{1.145486in}{1.322597in}}%
\pgfpathlineto{\pgfqpoint{1.157175in}{1.323040in}}%
\pgfpathlineto{\pgfqpoint{1.168886in}{1.323301in}}%
\pgfpathlineto{\pgfqpoint{1.180606in}{1.323382in}}%
\pgfpathlineto{\pgfqpoint{1.180600in}{1.328907in}}%
\pgfpathlineto{\pgfqpoint{1.180594in}{1.334466in}}%
\pgfpathlineto{\pgfqpoint{1.180588in}{1.340056in}}%
\pgfpathlineto{\pgfqpoint{1.180582in}{1.345674in}}%
\pgfpathlineto{\pgfqpoint{1.169302in}{1.345597in}}%
\pgfpathlineto{\pgfqpoint{1.158032in}{1.345346in}}%
\pgfpathlineto{\pgfqpoint{1.146782in}{1.344921in}}%
\pgfpathlineto{\pgfqpoint{1.135563in}{1.344324in}}%
\pgfpathclose%
\pgfusepath{fill}%
\end{pgfscope}%
\begin{pgfscope}%
\pgfpathrectangle{\pgfqpoint{0.041670in}{0.041670in}}{\pgfqpoint{2.216660in}{2.216660in}}%
\pgfusepath{clip}%
\pgfsetbuttcap%
\pgfsetroundjoin%
\definecolor{currentfill}{rgb}{0.179019,0.433756,0.557430}%
\pgfsetfillcolor{currentfill}%
\pgfsetlinewidth{0.000000pt}%
\definecolor{currentstroke}{rgb}{0.000000,0.000000,0.000000}%
\pgfsetstrokecolor{currentstroke}%
\pgfsetdash{}{0pt}%
\pgfpathmoveto{\pgfqpoint{1.180582in}{1.345674in}}%
\pgfpathlineto{\pgfqpoint{1.180588in}{1.340056in}}%
\pgfpathlineto{\pgfqpoint{1.180594in}{1.334466in}}%
\pgfpathlineto{\pgfqpoint{1.180600in}{1.328907in}}%
\pgfpathlineto{\pgfqpoint{1.180606in}{1.323382in}}%
\pgfpathlineto{\pgfqpoint{1.192326in}{1.323281in}}%
\pgfpathlineto{\pgfqpoint{1.204034in}{1.322999in}}%
\pgfpathlineto{\pgfqpoint{1.215721in}{1.322537in}}%
\pgfpathlineto{\pgfqpoint{1.227374in}{1.321894in}}%
\pgfpathlineto{\pgfqpoint{1.226929in}{1.327434in}}%
\pgfpathlineto{\pgfqpoint{1.226484in}{1.333008in}}%
\pgfpathlineto{\pgfqpoint{1.226037in}{1.338613in}}%
\pgfpathlineto{\pgfqpoint{1.225591in}{1.344246in}}%
\pgfpathlineto{\pgfqpoint{1.214376in}{1.344863in}}%
\pgfpathlineto{\pgfqpoint{1.203129in}{1.345307in}}%
\pgfpathlineto{\pgfqpoint{1.191861in}{1.345578in}}%
\pgfpathlineto{\pgfqpoint{1.180582in}{1.345674in}}%
\pgfpathclose%
\pgfusepath{fill}%
\end{pgfscope}%
\begin{pgfscope}%
\pgfpathrectangle{\pgfqpoint{0.041670in}{0.041670in}}{\pgfqpoint{2.216660in}{2.216660in}}%
\pgfusepath{clip}%
\pgfsetbuttcap%
\pgfsetroundjoin%
\definecolor{currentfill}{rgb}{0.195860,0.395433,0.555276}%
\pgfsetfillcolor{currentfill}%
\pgfsetlinewidth{0.000000pt}%
\definecolor{currentstroke}{rgb}{0.000000,0.000000,0.000000}%
\pgfsetstrokecolor{currentstroke}%
\pgfsetdash{}{0pt}%
\pgfpathmoveto{\pgfqpoint{1.000252in}{1.300754in}}%
\pgfpathlineto{\pgfqpoint{0.998561in}{1.295035in}}%
\pgfpathlineto{\pgfqpoint{0.996873in}{1.289355in}}%
\pgfpathlineto{\pgfqpoint{0.995185in}{1.283719in}}%
\pgfpathlineto{\pgfqpoint{0.993500in}{1.278128in}}%
\pgfpathlineto{\pgfqpoint{1.004352in}{1.280945in}}%
\pgfpathlineto{\pgfqpoint{1.015367in}{1.283591in}}%
\pgfpathlineto{\pgfqpoint{1.026533in}{1.286063in}}%
\pgfpathlineto{\pgfqpoint{1.037841in}{1.288360in}}%
\pgfpathlineto{\pgfqpoint{1.039124in}{1.293850in}}%
\pgfpathlineto{\pgfqpoint{1.040408in}{1.299386in}}%
\pgfpathlineto{\pgfqpoint{1.041693in}{1.304965in}}%
\pgfpathlineto{\pgfqpoint{1.042980in}{1.310584in}}%
\pgfpathlineto{\pgfqpoint{1.032083in}{1.308377in}}%
\pgfpathlineto{\pgfqpoint{1.021323in}{1.306002in}}%
\pgfpathlineto{\pgfqpoint{1.010709in}{1.303460in}}%
\pgfpathlineto{\pgfqpoint{1.000252in}{1.300754in}}%
\pgfpathclose%
\pgfusepath{fill}%
\end{pgfscope}%
\begin{pgfscope}%
\pgfpathrectangle{\pgfqpoint{0.041670in}{0.041670in}}{\pgfqpoint{2.216660in}{2.216660in}}%
\pgfusepath{clip}%
\pgfsetbuttcap%
\pgfsetroundjoin%
\definecolor{currentfill}{rgb}{0.172719,0.448791,0.557885}%
\pgfsetfillcolor{currentfill}%
\pgfsetlinewidth{0.000000pt}%
\definecolor{currentstroke}{rgb}{0.000000,0.000000,0.000000}%
\pgfsetstrokecolor{currentstroke}%
\pgfsetdash{}{0pt}%
\pgfpathmoveto{\pgfqpoint{1.783103in}{1.342612in}}%
\pgfpathlineto{\pgfqpoint{1.786515in}{1.358276in}}%
\pgfpathlineto{\pgfqpoint{1.789944in}{1.374435in}}%
\pgfpathlineto{\pgfqpoint{1.793391in}{1.391096in}}%
\pgfpathlineto{\pgfqpoint{1.806946in}{1.381597in}}%
\pgfpathlineto{\pgfqpoint{1.819931in}{1.371880in}}%
\pgfpathlineto{\pgfqpoint{1.832332in}{1.361953in}}%
\pgfpathlineto{\pgfqpoint{1.844136in}{1.351826in}}%
\pgfpathlineto{\pgfqpoint{1.840382in}{1.335289in}}%
\pgfpathlineto{\pgfqpoint{1.836647in}{1.319258in}}%
\pgfpathlineto{\pgfqpoint{1.832932in}{1.303725in}}%
\pgfpathlineto{\pgfqpoint{1.821344in}{1.313753in}}%
\pgfpathlineto{\pgfqpoint{1.809167in}{1.323582in}}%
\pgfpathlineto{\pgfqpoint{1.796415in}{1.333204in}}%
\pgfpathlineto{\pgfqpoint{1.783103in}{1.342612in}}%
\pgfpathclose%
\pgfusepath{fill}%
\end{pgfscope}%
\begin{pgfscope}%
\pgfpathrectangle{\pgfqpoint{0.041670in}{0.041670in}}{\pgfqpoint{2.216660in}{2.216660in}}%
\pgfusepath{clip}%
\pgfsetbuttcap%
\pgfsetroundjoin%
\definecolor{currentfill}{rgb}{0.274952,0.037752,0.364543}%
\pgfsetfillcolor{currentfill}%
\pgfsetlinewidth{0.000000pt}%
\definecolor{currentstroke}{rgb}{0.000000,0.000000,0.000000}%
\pgfsetstrokecolor{currentstroke}%
\pgfsetdash{}{0pt}%
\pgfpathmoveto{\pgfqpoint{0.682790in}{0.985201in}}%
\pgfpathlineto{\pgfqpoint{0.679358in}{0.982606in}}%
\pgfpathlineto{\pgfqpoint{0.675922in}{0.980194in}}%
\pgfpathlineto{\pgfqpoint{0.672484in}{0.977969in}}%
\pgfpathlineto{\pgfqpoint{0.669042in}{0.975934in}}%
\pgfpathlineto{\pgfqpoint{0.675042in}{0.984185in}}%
\pgfpathlineto{\pgfqpoint{0.681524in}{0.992325in}}%
\pgfpathlineto{\pgfqpoint{0.688480in}{1.000346in}}%
\pgfpathlineto{\pgfqpoint{0.695903in}{1.008243in}}%
\pgfpathlineto{\pgfqpoint{0.699149in}{1.010052in}}%
\pgfpathlineto{\pgfqpoint{0.702392in}{1.012051in}}%
\pgfpathlineto{\pgfqpoint{0.705633in}{1.014236in}}%
\pgfpathlineto{\pgfqpoint{0.708871in}{1.016603in}}%
\pgfpathlineto{\pgfqpoint{0.701661in}{1.008927in}}%
\pgfpathlineto{\pgfqpoint{0.694906in}{1.001130in}}%
\pgfpathlineto{\pgfqpoint{0.688613in}{0.993219in}}%
\pgfpathlineto{\pgfqpoint{0.682790in}{0.985201in}}%
\pgfpathclose%
\pgfusepath{fill}%
\end{pgfscope}%
\begin{pgfscope}%
\pgfpathrectangle{\pgfqpoint{0.041670in}{0.041670in}}{\pgfqpoint{2.216660in}{2.216660in}}%
\pgfusepath{clip}%
\pgfsetbuttcap%
\pgfsetroundjoin%
\definecolor{currentfill}{rgb}{0.279566,0.067836,0.391917}%
\pgfsetfillcolor{currentfill}%
\pgfsetlinewidth{0.000000pt}%
\definecolor{currentstroke}{rgb}{0.000000,0.000000,0.000000}%
\pgfsetstrokecolor{currentstroke}%
\pgfsetdash{}{0pt}%
\pgfpathmoveto{\pgfqpoint{1.631521in}{1.034328in}}%
\pgfpathlineto{\pgfqpoint{1.634702in}{1.031323in}}%
\pgfpathlineto{\pgfqpoint{1.637885in}{1.028484in}}%
\pgfpathlineto{\pgfqpoint{1.641069in}{1.025814in}}%
\pgfpathlineto{\pgfqpoint{1.644255in}{1.023318in}}%
\pgfpathlineto{\pgfqpoint{1.651862in}{1.015756in}}%
\pgfpathlineto{\pgfqpoint{1.659022in}{1.008067in}}%
\pgfpathlineto{\pgfqpoint{1.665726in}{1.000257in}}%
\pgfpathlineto{\pgfqpoint{1.671967in}{0.992333in}}%
\pgfpathlineto{\pgfqpoint{1.668576in}{0.995054in}}%
\pgfpathlineto{\pgfqpoint{1.665187in}{0.997948in}}%
\pgfpathlineto{\pgfqpoint{1.661800in}{1.001013in}}%
\pgfpathlineto{\pgfqpoint{1.658416in}{1.004243in}}%
\pgfpathlineto{\pgfqpoint{1.652361in}{1.011936in}}%
\pgfpathlineto{\pgfqpoint{1.645855in}{1.019519in}}%
\pgfpathlineto{\pgfqpoint{1.638906in}{1.026985in}}%
\pgfpathlineto{\pgfqpoint{1.631521in}{1.034328in}}%
\pgfpathclose%
\pgfusepath{fill}%
\end{pgfscope}%
\begin{pgfscope}%
\pgfpathrectangle{\pgfqpoint{0.041670in}{0.041670in}}{\pgfqpoint{2.216660in}{2.216660in}}%
\pgfusepath{clip}%
\pgfsetbuttcap%
\pgfsetroundjoin%
\definecolor{currentfill}{rgb}{0.276194,0.190074,0.493001}%
\pgfsetfillcolor{currentfill}%
\pgfsetlinewidth{0.000000pt}%
\definecolor{currentstroke}{rgb}{0.000000,0.000000,0.000000}%
\pgfsetstrokecolor{currentstroke}%
\pgfsetdash{}{0pt}%
\pgfpathmoveto{\pgfqpoint{0.540983in}{1.068328in}}%
\pgfpathlineto{\pgfqpoint{0.537244in}{1.076688in}}%
\pgfpathlineto{\pgfqpoint{0.533490in}{1.085434in}}%
\pgfpathlineto{\pgfqpoint{0.529722in}{1.094572in}}%
\pgfpathlineto{\pgfqpoint{0.525939in}{1.104108in}}%
\pgfpathlineto{\pgfqpoint{0.533829in}{1.114476in}}%
\pgfpathlineto{\pgfqpoint{0.542328in}{1.124699in}}%
\pgfpathlineto{\pgfqpoint{0.551426in}{1.134766in}}%
\pgfpathlineto{\pgfqpoint{0.561111in}{1.144671in}}%
\pgfpathlineto{\pgfqpoint{0.564669in}{1.134956in}}%
\pgfpathlineto{\pgfqpoint{0.568214in}{1.125638in}}%
\pgfpathlineto{\pgfqpoint{0.571746in}{1.116710in}}%
\pgfpathlineto{\pgfqpoint{0.575265in}{1.108165in}}%
\pgfpathlineto{\pgfqpoint{0.565821in}{1.098437in}}%
\pgfpathlineto{\pgfqpoint{0.556953in}{1.088549in}}%
\pgfpathlineto{\pgfqpoint{0.548670in}{1.078509in}}%
\pgfpathlineto{\pgfqpoint{0.540983in}{1.068328in}}%
\pgfpathclose%
\pgfusepath{fill}%
\end{pgfscope}%
\begin{pgfscope}%
\pgfpathrectangle{\pgfqpoint{0.041670in}{0.041670in}}{\pgfqpoint{2.216660in}{2.216660in}}%
\pgfusepath{clip}%
\pgfsetbuttcap%
\pgfsetroundjoin%
\definecolor{currentfill}{rgb}{0.263663,0.237631,0.518762}%
\pgfsetfillcolor{currentfill}%
\pgfsetlinewidth{0.000000pt}%
\definecolor{currentstroke}{rgb}{0.000000,0.000000,0.000000}%
\pgfsetstrokecolor{currentstroke}%
\pgfsetdash{}{0pt}%
\pgfpathmoveto{\pgfqpoint{1.505051in}{1.168791in}}%
\pgfpathlineto{\pgfqpoint{1.507695in}{1.163674in}}%
\pgfpathlineto{\pgfqpoint{1.510338in}{1.158649in}}%
\pgfpathlineto{\pgfqpoint{1.512981in}{1.153719in}}%
\pgfpathlineto{\pgfqpoint{1.515622in}{1.148887in}}%
\pgfpathlineto{\pgfqpoint{1.524952in}{1.143487in}}%
\pgfpathlineto{\pgfqpoint{1.533960in}{1.137938in}}%
\pgfpathlineto{\pgfqpoint{1.542635in}{1.132244in}}%
\pgfpathlineto{\pgfqpoint{1.550970in}{1.126410in}}%
\pgfpathlineto{\pgfqpoint{1.548042in}{1.131434in}}%
\pgfpathlineto{\pgfqpoint{1.545112in}{1.136558in}}%
\pgfpathlineto{\pgfqpoint{1.542181in}{1.141776in}}%
\pgfpathlineto{\pgfqpoint{1.539249in}{1.147086in}}%
\pgfpathlineto{\pgfqpoint{1.531187in}{1.152719in}}%
\pgfpathlineto{\pgfqpoint{1.522794in}{1.158218in}}%
\pgfpathlineto{\pgfqpoint{1.514079in}{1.163576in}}%
\pgfpathlineto{\pgfqpoint{1.505051in}{1.168791in}}%
\pgfpathclose%
\pgfusepath{fill}%
\end{pgfscope}%
\begin{pgfscope}%
\pgfpathrectangle{\pgfqpoint{0.041670in}{0.041670in}}{\pgfqpoint{2.216660in}{2.216660in}}%
\pgfusepath{clip}%
\pgfsetbuttcap%
\pgfsetroundjoin%
\definecolor{currentfill}{rgb}{0.280255,0.165693,0.476498}%
\pgfsetfillcolor{currentfill}%
\pgfsetlinewidth{0.000000pt}%
\definecolor{currentstroke}{rgb}{0.000000,0.000000,0.000000}%
\pgfsetstrokecolor{currentstroke}%
\pgfsetdash{}{0pt}%
\pgfpathmoveto{\pgfqpoint{1.562679in}{1.107367in}}%
\pgfpathlineto{\pgfqpoint{1.565604in}{1.102887in}}%
\pgfpathlineto{\pgfqpoint{1.568530in}{1.098528in}}%
\pgfpathlineto{\pgfqpoint{1.571455in}{1.094292in}}%
\pgfpathlineto{\pgfqpoint{1.574381in}{1.090183in}}%
\pgfpathlineto{\pgfqpoint{1.582892in}{1.083804in}}%
\pgfpathlineto{\pgfqpoint{1.591025in}{1.077286in}}%
\pgfpathlineto{\pgfqpoint{1.598769in}{1.070635in}}%
\pgfpathlineto{\pgfqpoint{1.606117in}{1.063856in}}%
\pgfpathlineto{\pgfqpoint{1.602945in}{1.068176in}}%
\pgfpathlineto{\pgfqpoint{1.599772in}{1.072623in}}%
\pgfpathlineto{\pgfqpoint{1.596600in}{1.077194in}}%
\pgfpathlineto{\pgfqpoint{1.593428in}{1.081885in}}%
\pgfpathlineto{\pgfqpoint{1.586310in}{1.088446in}}%
\pgfpathlineto{\pgfqpoint{1.578807in}{1.094883in}}%
\pgfpathlineto{\pgfqpoint{1.570927in}{1.101192in}}%
\pgfpathlineto{\pgfqpoint{1.562679in}{1.107367in}}%
\pgfpathclose%
\pgfusepath{fill}%
\end{pgfscope}%
\begin{pgfscope}%
\pgfpathrectangle{\pgfqpoint{0.041670in}{0.041670in}}{\pgfqpoint{2.216660in}{2.216660in}}%
\pgfusepath{clip}%
\pgfsetbuttcap%
\pgfsetroundjoin%
\definecolor{currentfill}{rgb}{0.260571,0.246922,0.522828}%
\pgfsetfillcolor{currentfill}%
\pgfsetlinewidth{0.000000pt}%
\definecolor{currentstroke}{rgb}{0.000000,0.000000,0.000000}%
\pgfsetstrokecolor{currentstroke}%
\pgfsetdash{}{0pt}%
\pgfpathmoveto{\pgfqpoint{1.789707in}{1.153330in}}%
\pgfpathlineto{\pgfqpoint{1.793223in}{1.163485in}}%
\pgfpathlineto{\pgfqpoint{1.796753in}{1.174048in}}%
\pgfpathlineto{\pgfqpoint{1.800298in}{1.185028in}}%
\pgfpathlineto{\pgfqpoint{1.803858in}{1.196430in}}%
\pgfpathlineto{\pgfqpoint{1.814313in}{1.186514in}}%
\pgfpathlineto{\pgfqpoint{1.824178in}{1.176423in}}%
\pgfpathlineto{\pgfqpoint{1.833442in}{1.166166in}}%
\pgfpathlineto{\pgfqpoint{1.842093in}{1.155752in}}%
\pgfpathlineto{\pgfqpoint{1.838292in}{1.144516in}}%
\pgfpathlineto{\pgfqpoint{1.834508in}{1.133705in}}%
\pgfpathlineto{\pgfqpoint{1.830740in}{1.123313in}}%
\pgfpathlineto{\pgfqpoint{1.826988in}{1.113331in}}%
\pgfpathlineto{\pgfqpoint{1.818556in}{1.123570in}}%
\pgfpathlineto{\pgfqpoint{1.809524in}{1.133656in}}%
\pgfpathlineto{\pgfqpoint{1.799904in}{1.143578in}}%
\pgfpathlineto{\pgfqpoint{1.789707in}{1.153330in}}%
\pgfpathclose%
\pgfusepath{fill}%
\end{pgfscope}%
\begin{pgfscope}%
\pgfpathrectangle{\pgfqpoint{0.041670in}{0.041670in}}{\pgfqpoint{2.216660in}{2.216660in}}%
\pgfusepath{clip}%
\pgfsetbuttcap%
\pgfsetroundjoin%
\definecolor{currentfill}{rgb}{0.179019,0.433756,0.557430}%
\pgfsetfillcolor{currentfill}%
\pgfsetlinewidth{0.000000pt}%
\definecolor{currentstroke}{rgb}{0.000000,0.000000,0.000000}%
\pgfsetstrokecolor{currentstroke}%
\pgfsetdash{}{0pt}%
\pgfpathmoveto{\pgfqpoint{1.091198in}{1.340213in}}%
\pgfpathlineto{\pgfqpoint{1.090329in}{1.334537in}}%
\pgfpathlineto{\pgfqpoint{1.089460in}{1.328889in}}%
\pgfpathlineto{\pgfqpoint{1.088593in}{1.323272in}}%
\pgfpathlineto{\pgfqpoint{1.087727in}{1.317689in}}%
\pgfpathlineto{\pgfqpoint{1.099151in}{1.319027in}}%
\pgfpathlineto{\pgfqpoint{1.110650in}{1.320188in}}%
\pgfpathlineto{\pgfqpoint{1.122213in}{1.321171in}}%
\pgfpathlineto{\pgfqpoint{1.133828in}{1.321974in}}%
\pgfpathlineto{\pgfqpoint{1.134261in}{1.327513in}}%
\pgfpathlineto{\pgfqpoint{1.134695in}{1.333087in}}%
\pgfpathlineto{\pgfqpoint{1.135129in}{1.338691in}}%
\pgfpathlineto{\pgfqpoint{1.135563in}{1.344324in}}%
\pgfpathlineto{\pgfqpoint{1.124385in}{1.343553in}}%
\pgfpathlineto{\pgfqpoint{1.113258in}{1.342611in}}%
\pgfpathlineto{\pgfqpoint{1.102192in}{1.341497in}}%
\pgfpathlineto{\pgfqpoint{1.091198in}{1.340213in}}%
\pgfpathclose%
\pgfusepath{fill}%
\end{pgfscope}%
\begin{pgfscope}%
\pgfpathrectangle{\pgfqpoint{0.041670in}{0.041670in}}{\pgfqpoint{2.216660in}{2.216660in}}%
\pgfusepath{clip}%
\pgfsetbuttcap%
\pgfsetroundjoin%
\definecolor{currentfill}{rgb}{0.179019,0.433756,0.557430}%
\pgfsetfillcolor{currentfill}%
\pgfsetlinewidth{0.000000pt}%
\definecolor{currentstroke}{rgb}{0.000000,0.000000,0.000000}%
\pgfsetstrokecolor{currentstroke}%
\pgfsetdash{}{0pt}%
\pgfpathmoveto{\pgfqpoint{1.225591in}{1.344246in}}%
\pgfpathlineto{\pgfqpoint{1.226037in}{1.338613in}}%
\pgfpathlineto{\pgfqpoint{1.226484in}{1.333008in}}%
\pgfpathlineto{\pgfqpoint{1.226929in}{1.327434in}}%
\pgfpathlineto{\pgfqpoint{1.227374in}{1.321894in}}%
\pgfpathlineto{\pgfqpoint{1.238985in}{1.321070in}}%
\pgfpathlineto{\pgfqpoint{1.250541in}{1.320068in}}%
\pgfpathlineto{\pgfqpoint{1.262032in}{1.318887in}}%
\pgfpathlineto{\pgfqpoint{1.273447in}{1.317529in}}%
\pgfpathlineto{\pgfqpoint{1.272569in}{1.323114in}}%
\pgfpathlineto{\pgfqpoint{1.271690in}{1.328733in}}%
\pgfpathlineto{\pgfqpoint{1.270810in}{1.334382in}}%
\pgfpathlineto{\pgfqpoint{1.269928in}{1.340060in}}%
\pgfpathlineto{\pgfqpoint{1.258943in}{1.341363in}}%
\pgfpathlineto{\pgfqpoint{1.247885in}{1.342495in}}%
\pgfpathlineto{\pgfqpoint{1.236764in}{1.343457in}}%
\pgfpathlineto{\pgfqpoint{1.225591in}{1.344246in}}%
\pgfpathclose%
\pgfusepath{fill}%
\end{pgfscope}%
\begin{pgfscope}%
\pgfpathrectangle{\pgfqpoint{0.041670in}{0.041670in}}{\pgfqpoint{2.216660in}{2.216660in}}%
\pgfusepath{clip}%
\pgfsetbuttcap%
\pgfsetroundjoin%
\definecolor{currentfill}{rgb}{0.263663,0.237631,0.518762}%
\pgfsetfillcolor{currentfill}%
\pgfsetlinewidth{0.000000pt}%
\definecolor{currentstroke}{rgb}{0.000000,0.000000,0.000000}%
\pgfsetstrokecolor{currentstroke}%
\pgfsetdash{}{0pt}%
\pgfpathmoveto{\pgfqpoint{0.813778in}{1.141969in}}%
\pgfpathlineto{\pgfqpoint{0.810788in}{1.136614in}}%
\pgfpathlineto{\pgfqpoint{0.807799in}{1.131350in}}%
\pgfpathlineto{\pgfqpoint{0.804811in}{1.126181in}}%
\pgfpathlineto{\pgfqpoint{0.801823in}{1.121110in}}%
\pgfpathlineto{\pgfqpoint{0.809848in}{1.127065in}}%
\pgfpathlineto{\pgfqpoint{0.818222in}{1.132884in}}%
\pgfpathlineto{\pgfqpoint{0.826935in}{1.138562in}}%
\pgfpathlineto{\pgfqpoint{0.835979in}{1.144095in}}%
\pgfpathlineto{\pgfqpoint{0.838687in}{1.148968in}}%
\pgfpathlineto{\pgfqpoint{0.841397in}{1.153939in}}%
\pgfpathlineto{\pgfqpoint{0.844107in}{1.159005in}}%
\pgfpathlineto{\pgfqpoint{0.846819in}{1.164163in}}%
\pgfpathlineto{\pgfqpoint{0.838069in}{1.158820in}}%
\pgfpathlineto{\pgfqpoint{0.829640in}{1.153337in}}%
\pgfpathlineto{\pgfqpoint{0.821540in}{1.147718in}}%
\pgfpathlineto{\pgfqpoint{0.813778in}{1.141969in}}%
\pgfpathclose%
\pgfusepath{fill}%
\end{pgfscope}%
\begin{pgfscope}%
\pgfpathrectangle{\pgfqpoint{0.041670in}{0.041670in}}{\pgfqpoint{2.216660in}{2.216660in}}%
\pgfusepath{clip}%
\pgfsetbuttcap%
\pgfsetroundjoin%
\definecolor{currentfill}{rgb}{0.212395,0.359683,0.551710}%
\pgfsetfillcolor{currentfill}%
\pgfsetlinewidth{0.000000pt}%
\definecolor{currentstroke}{rgb}{0.000000,0.000000,0.000000}%
\pgfsetstrokecolor{currentstroke}%
\pgfsetdash{}{0pt}%
\pgfpathmoveto{\pgfqpoint{1.399028in}{1.268297in}}%
\pgfpathlineto{\pgfqpoint{1.401010in}{1.262658in}}%
\pgfpathlineto{\pgfqpoint{1.402989in}{1.257072in}}%
\pgfpathlineto{\pgfqpoint{1.404967in}{1.251542in}}%
\pgfpathlineto{\pgfqpoint{1.406944in}{1.246070in}}%
\pgfpathlineto{\pgfqpoint{1.417394in}{1.242438in}}%
\pgfpathlineto{\pgfqpoint{1.427623in}{1.238642in}}%
\pgfpathlineto{\pgfqpoint{1.437623in}{1.234686in}}%
\pgfpathlineto{\pgfqpoint{1.447383in}{1.230572in}}%
\pgfpathlineto{\pgfqpoint{1.445050in}{1.236191in}}%
\pgfpathlineto{\pgfqpoint{1.442714in}{1.241869in}}%
\pgfpathlineto{\pgfqpoint{1.440376in}{1.247603in}}%
\pgfpathlineto{\pgfqpoint{1.438037in}{1.253389in}}%
\pgfpathlineto{\pgfqpoint{1.428623in}{1.257345in}}%
\pgfpathlineto{\pgfqpoint{1.418977in}{1.261151in}}%
\pgfpathlineto{\pgfqpoint{1.409109in}{1.264803in}}%
\pgfpathlineto{\pgfqpoint{1.399028in}{1.268297in}}%
\pgfpathclose%
\pgfusepath{fill}%
\end{pgfscope}%
\begin{pgfscope}%
\pgfpathrectangle{\pgfqpoint{0.041670in}{0.041670in}}{\pgfqpoint{2.216660in}{2.216660in}}%
\pgfusepath{clip}%
\pgfsetbuttcap%
\pgfsetroundjoin%
\definecolor{currentfill}{rgb}{0.179019,0.433756,0.557430}%
\pgfsetfillcolor{currentfill}%
\pgfsetlinewidth{0.000000pt}%
\definecolor{currentstroke}{rgb}{0.000000,0.000000,0.000000}%
\pgfsetstrokecolor{currentstroke}%
\pgfsetdash{}{0pt}%
\pgfpathmoveto{\pgfqpoint{1.269928in}{1.340060in}}%
\pgfpathlineto{\pgfqpoint{1.270810in}{1.334382in}}%
\pgfpathlineto{\pgfqpoint{1.271690in}{1.328733in}}%
\pgfpathlineto{\pgfqpoint{1.272569in}{1.323114in}}%
\pgfpathlineto{\pgfqpoint{1.273447in}{1.317529in}}%
\pgfpathlineto{\pgfqpoint{1.284777in}{1.315994in}}%
\pgfpathlineto{\pgfqpoint{1.296011in}{1.314285in}}%
\pgfpathlineto{\pgfqpoint{1.307137in}{1.312402in}}%
\pgfpathlineto{\pgfqpoint{1.305942in}{1.318039in}}%
\pgfpathlineto{\pgfqpoint{1.304745in}{1.323710in}}%
\pgfpathlineto{\pgfqpoint{1.303546in}{1.329412in}}%
\pgfpathlineto{\pgfqpoint{1.302347in}{1.335142in}}%
\pgfpathlineto{\pgfqpoint{1.291641in}{1.336948in}}%
\pgfpathlineto{\pgfqpoint{1.280831in}{1.338588in}}%
\pgfpathlineto{\pgfqpoint{1.269928in}{1.340060in}}%
\pgfpathclose%
\pgfusepath{fill}%
\end{pgfscope}%
\begin{pgfscope}%
\pgfpathrectangle{\pgfqpoint{0.041670in}{0.041670in}}{\pgfqpoint{2.216660in}{2.216660in}}%
\pgfusepath{clip}%
\pgfsetbuttcap%
\pgfsetroundjoin%
\definecolor{currentfill}{rgb}{0.280255,0.165693,0.476498}%
\pgfsetfillcolor{currentfill}%
\pgfsetlinewidth{0.000000pt}%
\definecolor{currentstroke}{rgb}{0.000000,0.000000,0.000000}%
\pgfsetstrokecolor{currentstroke}%
\pgfsetdash{}{0pt}%
\pgfpathmoveto{\pgfqpoint{0.760485in}{1.075955in}}%
\pgfpathlineto{\pgfqpoint{0.757264in}{1.071215in}}%
\pgfpathlineto{\pgfqpoint{0.754042in}{1.066594in}}%
\pgfpathlineto{\pgfqpoint{0.750821in}{1.062097in}}%
\pgfpathlineto{\pgfqpoint{0.747600in}{1.057728in}}%
\pgfpathlineto{\pgfqpoint{0.754589in}{1.064615in}}%
\pgfpathlineto{\pgfqpoint{0.761982in}{1.071380in}}%
\pgfpathlineto{\pgfqpoint{0.769770in}{1.078017in}}%
\pgfpathlineto{\pgfqpoint{0.777945in}{1.084520in}}%
\pgfpathlineto{\pgfqpoint{0.780929in}{1.088674in}}%
\pgfpathlineto{\pgfqpoint{0.783913in}{1.092956in}}%
\pgfpathlineto{\pgfqpoint{0.786897in}{1.097360in}}%
\pgfpathlineto{\pgfqpoint{0.789881in}{1.101885in}}%
\pgfpathlineto{\pgfqpoint{0.781960in}{1.095591in}}%
\pgfpathlineto{\pgfqpoint{0.774415in}{1.089167in}}%
\pgfpathlineto{\pgfqpoint{0.767254in}{1.082620in}}%
\pgfpathlineto{\pgfqpoint{0.760485in}{1.075955in}}%
\pgfpathclose%
\pgfusepath{fill}%
\end{pgfscope}%
\begin{pgfscope}%
\pgfpathrectangle{\pgfqpoint{0.041670in}{0.041670in}}{\pgfqpoint{2.216660in}{2.216660in}}%
\pgfusepath{clip}%
\pgfsetbuttcap%
\pgfsetroundjoin%
\definecolor{currentfill}{rgb}{0.179019,0.433756,0.557430}%
\pgfsetfillcolor{currentfill}%
\pgfsetlinewidth{0.000000pt}%
\definecolor{currentstroke}{rgb}{0.000000,0.000000,0.000000}%
\pgfsetstrokecolor{currentstroke}%
\pgfsetdash{}{0pt}%
\pgfpathmoveto{\pgfqpoint{1.048141in}{1.333398in}}%
\pgfpathlineto{\pgfqpoint{1.046848in}{1.327650in}}%
\pgfpathlineto{\pgfqpoint{1.045558in}{1.321929in}}%
\pgfpathlineto{\pgfqpoint{1.044268in}{1.316239in}}%
\pgfpathlineto{\pgfqpoint{1.042980in}{1.310584in}}%
\pgfpathlineto{\pgfqpoint{1.054003in}{1.312620in}}%
\pgfpathlineto{\pgfqpoint{1.065142in}{1.314483in}}%
\pgfpathlineto{\pgfqpoint{1.076387in}{1.316174in}}%
\pgfpathlineto{\pgfqpoint{1.087727in}{1.317689in}}%
\pgfpathlineto{\pgfqpoint{1.088593in}{1.323272in}}%
\pgfpathlineto{\pgfqpoint{1.089460in}{1.328889in}}%
\pgfpathlineto{\pgfqpoint{1.090329in}{1.334537in}}%
\pgfpathlineto{\pgfqpoint{1.091198in}{1.340213in}}%
\pgfpathlineto{\pgfqpoint{1.080286in}{1.338760in}}%
\pgfpathlineto{\pgfqpoint{1.069465in}{1.337139in}}%
\pgfpathlineto{\pgfqpoint{1.058747in}{1.335351in}}%
\pgfpathlineto{\pgfqpoint{1.048141in}{1.333398in}}%
\pgfpathclose%
\pgfusepath{fill}%
\end{pgfscope}%
\begin{pgfscope}%
\pgfpathrectangle{\pgfqpoint{0.041670in}{0.041670in}}{\pgfqpoint{2.216660in}{2.216660in}}%
\pgfusepath{clip}%
\pgfsetbuttcap%
\pgfsetroundjoin%
\definecolor{currentfill}{rgb}{0.279566,0.067836,0.391917}%
\pgfsetfillcolor{currentfill}%
\pgfsetlinewidth{0.000000pt}%
\definecolor{currentstroke}{rgb}{0.000000,0.000000,0.000000}%
\pgfsetstrokecolor{currentstroke}%
\pgfsetdash{}{0pt}%
\pgfpathmoveto{\pgfqpoint{0.696497in}{0.997318in}}%
\pgfpathlineto{\pgfqpoint{0.693073in}{0.994036in}}%
\pgfpathlineto{\pgfqpoint{0.689648in}{0.990919in}}%
\pgfpathlineto{\pgfqpoint{0.686220in}{0.987973in}}%
\pgfpathlineto{\pgfqpoint{0.682790in}{0.985201in}}%
\pgfpathlineto{\pgfqpoint{0.688613in}{0.993219in}}%
\pgfpathlineto{\pgfqpoint{0.694906in}{1.001130in}}%
\pgfpathlineto{\pgfqpoint{0.701661in}{1.008927in}}%
\pgfpathlineto{\pgfqpoint{0.708871in}{1.016603in}}%
\pgfpathlineto{\pgfqpoint{0.712107in}{1.019148in}}%
\pgfpathlineto{\pgfqpoint{0.715341in}{1.021866in}}%
\pgfpathlineto{\pgfqpoint{0.718573in}{1.024753in}}%
\pgfpathlineto{\pgfqpoint{0.721803in}{1.027807in}}%
\pgfpathlineto{\pgfqpoint{0.714805in}{1.020354in}}%
\pgfpathlineto{\pgfqpoint{0.708249in}{1.012784in}}%
\pgfpathlineto{\pgfqpoint{0.702144in}{1.005103in}}%
\pgfpathlineto{\pgfqpoint{0.696497in}{0.997318in}}%
\pgfpathclose%
\pgfusepath{fill}%
\end{pgfscope}%
\begin{pgfscope}%
\pgfpathrectangle{\pgfqpoint{0.041670in}{0.041670in}}{\pgfqpoint{2.216660in}{2.216660in}}%
\pgfusepath{clip}%
\pgfsetbuttcap%
\pgfsetroundjoin%
\definecolor{currentfill}{rgb}{0.195860,0.395433,0.555276}%
\pgfsetfillcolor{currentfill}%
\pgfsetlinewidth{0.000000pt}%
\definecolor{currentstroke}{rgb}{0.000000,0.000000,0.000000}%
\pgfsetstrokecolor{currentstroke}%
\pgfsetdash{}{0pt}%
\pgfpathmoveto{\pgfqpoint{1.350370in}{1.303168in}}%
\pgfpathlineto{\pgfqpoint{1.351973in}{1.297473in}}%
\pgfpathlineto{\pgfqpoint{1.353574in}{1.291818in}}%
\pgfpathlineto{\pgfqpoint{1.355173in}{1.286206in}}%
\pgfpathlineto{\pgfqpoint{1.356771in}{1.280641in}}%
\pgfpathlineto{\pgfqpoint{1.367605in}{1.277805in}}%
\pgfpathlineto{\pgfqpoint{1.378266in}{1.274801in}}%
\pgfpathlineto{\pgfqpoint{1.388744in}{1.271630in}}%
\pgfpathlineto{\pgfqpoint{1.399028in}{1.268297in}}%
\pgfpathlineto{\pgfqpoint{1.397045in}{1.273984in}}%
\pgfpathlineto{\pgfqpoint{1.395059in}{1.279718in}}%
\pgfpathlineto{\pgfqpoint{1.393072in}{1.285494in}}%
\pgfpathlineto{\pgfqpoint{1.391083in}{1.291311in}}%
\pgfpathlineto{\pgfqpoint{1.381176in}{1.294513in}}%
\pgfpathlineto{\pgfqpoint{1.371081in}{1.297558in}}%
\pgfpathlineto{\pgfqpoint{1.360810in}{1.300444in}}%
\pgfpathlineto{\pgfqpoint{1.350370in}{1.303168in}}%
\pgfpathclose%
\pgfusepath{fill}%
\end{pgfscope}%
\begin{pgfscope}%
\pgfpathrectangle{\pgfqpoint{0.041670in}{0.041670in}}{\pgfqpoint{2.216660in}{2.216660in}}%
\pgfusepath{clip}%
\pgfsetbuttcap%
\pgfsetroundjoin%
\definecolor{currentfill}{rgb}{0.231674,0.318106,0.544834}%
\pgfsetfillcolor{currentfill}%
\pgfsetlinewidth{0.000000pt}%
\definecolor{currentstroke}{rgb}{0.000000,0.000000,0.000000}%
\pgfsetstrokecolor{currentstroke}%
\pgfsetdash{}{0pt}%
\pgfpathmoveto{\pgfqpoint{1.447383in}{1.230572in}}%
\pgfpathlineto{\pgfqpoint{1.449715in}{1.225014in}}%
\pgfpathlineto{\pgfqpoint{1.452046in}{1.219522in}}%
\pgfpathlineto{\pgfqpoint{1.454374in}{1.214098in}}%
\pgfpathlineto{\pgfqpoint{1.456702in}{1.208745in}}%
\pgfpathlineto{\pgfqpoint{1.466550in}{1.204315in}}%
\pgfpathlineto{\pgfqpoint{1.476133in}{1.199729in}}%
\pgfpathlineto{\pgfqpoint{1.485439in}{1.194992in}}%
\pgfpathlineto{\pgfqpoint{1.494459in}{1.190106in}}%
\pgfpathlineto{\pgfqpoint{1.491807in}{1.195631in}}%
\pgfpathlineto{\pgfqpoint{1.489154in}{1.201226in}}%
\pgfpathlineto{\pgfqpoint{1.486499in}{1.206890in}}%
\pgfpathlineto{\pgfqpoint{1.483842in}{1.212620in}}%
\pgfpathlineto{\pgfqpoint{1.475133in}{1.217325in}}%
\pgfpathlineto{\pgfqpoint{1.466148in}{1.221888in}}%
\pgfpathlineto{\pgfqpoint{1.456895in}{1.226305in}}%
\pgfpathlineto{\pgfqpoint{1.447383in}{1.230572in}}%
\pgfpathclose%
\pgfusepath{fill}%
\end{pgfscope}%
\begin{pgfscope}%
\pgfpathrectangle{\pgfqpoint{0.041670in}{0.041670in}}{\pgfqpoint{2.216660in}{2.216660in}}%
\pgfusepath{clip}%
\pgfsetbuttcap%
\pgfsetroundjoin%
\definecolor{currentfill}{rgb}{0.172719,0.448791,0.557885}%
\pgfsetfillcolor{currentfill}%
\pgfsetlinewidth{0.000000pt}%
\definecolor{currentstroke}{rgb}{0.000000,0.000000,0.000000}%
\pgfsetstrokecolor{currentstroke}%
\pgfsetdash{}{0pt}%
\pgfpathmoveto{\pgfqpoint{0.517182in}{1.294652in}}%
\pgfpathlineto{\pgfqpoint{0.513406in}{1.310154in}}%
\pgfpathlineto{\pgfqpoint{0.509610in}{1.326155in}}%
\pgfpathlineto{\pgfqpoint{0.505794in}{1.342662in}}%
\pgfpathlineto{\pgfqpoint{0.517055in}{1.352961in}}%
\pgfpathlineto{\pgfqpoint{0.528926in}{1.363066in}}%
\pgfpathlineto{\pgfqpoint{0.541393in}{1.372970in}}%
\pgfpathlineto{\pgfqpoint{0.554442in}{1.382663in}}%
\pgfpathlineto{\pgfqpoint{0.557962in}{1.366028in}}%
\pgfpathlineto{\pgfqpoint{0.561463in}{1.349896in}}%
\pgfpathlineto{\pgfqpoint{0.564946in}{1.334260in}}%
\pgfpathlineto{\pgfqpoint{0.552132in}{1.324661in}}%
\pgfpathlineto{\pgfqpoint{0.539890in}{1.314855in}}%
\pgfpathlineto{\pgfqpoint{0.528236in}{1.304849in}}%
\pgfpathlineto{\pgfqpoint{0.517182in}{1.294652in}}%
\pgfpathclose%
\pgfusepath{fill}%
\end{pgfscope}%
\begin{pgfscope}%
\pgfpathrectangle{\pgfqpoint{0.041670in}{0.041670in}}{\pgfqpoint{2.216660in}{2.216660in}}%
\pgfusepath{clip}%
\pgfsetbuttcap%
\pgfsetroundjoin%
\definecolor{currentfill}{rgb}{0.282327,0.094955,0.417331}%
\pgfsetfillcolor{currentfill}%
\pgfsetlinewidth{0.000000pt}%
\definecolor{currentstroke}{rgb}{0.000000,0.000000,0.000000}%
\pgfsetstrokecolor{currentstroke}%
\pgfsetdash{}{0pt}%
\pgfpathmoveto{\pgfqpoint{1.618812in}{1.047923in}}%
\pgfpathlineto{\pgfqpoint{1.621988in}{1.044295in}}%
\pgfpathlineto{\pgfqpoint{1.625164in}{1.040818in}}%
\pgfpathlineto{\pgfqpoint{1.628342in}{1.037494in}}%
\pgfpathlineto{\pgfqpoint{1.631521in}{1.034328in}}%
\pgfpathlineto{\pgfqpoint{1.638906in}{1.026985in}}%
\pgfpathlineto{\pgfqpoint{1.645855in}{1.019519in}}%
\pgfpathlineto{\pgfqpoint{1.652361in}{1.011936in}}%
\pgfpathlineto{\pgfqpoint{1.658416in}{1.004243in}}%
\pgfpathlineto{\pgfqpoint{1.655033in}{1.007635in}}%
\pgfpathlineto{\pgfqpoint{1.651652in}{1.011186in}}%
\pgfpathlineto{\pgfqpoint{1.648272in}{1.014890in}}%
\pgfpathlineto{\pgfqpoint{1.644894in}{1.018745in}}%
\pgfpathlineto{\pgfqpoint{1.639025in}{1.026205in}}%
\pgfpathlineto{\pgfqpoint{1.632716in}{1.033559in}}%
\pgfpathlineto{\pgfqpoint{1.625976in}{1.040801in}}%
\pgfpathlineto{\pgfqpoint{1.618812in}{1.047923in}}%
\pgfpathclose%
\pgfusepath{fill}%
\end{pgfscope}%
\begin{pgfscope}%
\pgfpathrectangle{\pgfqpoint{0.041670in}{0.041670in}}{\pgfqpoint{2.216660in}{2.216660in}}%
\pgfusepath{clip}%
\pgfsetbuttcap%
\pgfsetroundjoin%
\definecolor{currentfill}{rgb}{0.212395,0.359683,0.551710}%
\pgfsetfillcolor{currentfill}%
\pgfsetlinewidth{0.000000pt}%
\definecolor{currentstroke}{rgb}{0.000000,0.000000,0.000000}%
\pgfsetstrokecolor{currentstroke}%
\pgfsetdash{}{0pt}%
\pgfpathmoveto{\pgfqpoint{0.913707in}{1.249748in}}%
\pgfpathlineto{\pgfqpoint{0.911292in}{1.243926in}}%
\pgfpathlineto{\pgfqpoint{0.908879in}{1.238156in}}%
\pgfpathlineto{\pgfqpoint{0.906468in}{1.232442in}}%
\pgfpathlineto{\pgfqpoint{0.904059in}{1.226786in}}%
\pgfpathlineto{\pgfqpoint{0.913599in}{1.231037in}}%
\pgfpathlineto{\pgfqpoint{0.923386in}{1.235133in}}%
\pgfpathlineto{\pgfqpoint{0.933412in}{1.239072in}}%
\pgfpathlineto{\pgfqpoint{0.943667in}{1.242850in}}%
\pgfpathlineto{\pgfqpoint{0.945725in}{1.248352in}}%
\pgfpathlineto{\pgfqpoint{0.947785in}{1.253913in}}%
\pgfpathlineto{\pgfqpoint{0.949847in}{1.259530in}}%
\pgfpathlineto{\pgfqpoint{0.951911in}{1.265199in}}%
\pgfpathlineto{\pgfqpoint{0.942018in}{1.261564in}}%
\pgfpathlineto{\pgfqpoint{0.932347in}{1.257776in}}%
\pgfpathlineto{\pgfqpoint{0.922907in}{1.253836in}}%
\pgfpathlineto{\pgfqpoint{0.913707in}{1.249748in}}%
\pgfpathclose%
\pgfusepath{fill}%
\end{pgfscope}%
\begin{pgfscope}%
\pgfpathrectangle{\pgfqpoint{0.041670in}{0.041670in}}{\pgfqpoint{2.216660in}{2.216660in}}%
\pgfusepath{clip}%
\pgfsetbuttcap%
\pgfsetroundjoin%
\definecolor{currentfill}{rgb}{0.268510,0.009605,0.335427}%
\pgfsetfillcolor{currentfill}%
\pgfsetlinewidth{0.000000pt}%
\definecolor{currentstroke}{rgb}{0.000000,0.000000,0.000000}%
\pgfsetstrokecolor{currentstroke}%
\pgfsetdash{}{0pt}%
\pgfpathmoveto{\pgfqpoint{1.740599in}{0.980994in}}%
\pgfpathlineto{\pgfqpoint{1.744088in}{0.982924in}}%
\pgfpathlineto{\pgfqpoint{1.747585in}{0.985128in}}%
\pgfpathlineto{\pgfqpoint{1.751090in}{0.987608in}}%
\pgfpathlineto{\pgfqpoint{1.754602in}{0.990372in}}%
\pgfpathlineto{\pgfqpoint{1.761452in}{0.980989in}}%
\pgfpathlineto{\pgfqpoint{1.767747in}{0.971487in}}%
\pgfpathlineto{\pgfqpoint{1.773479in}{0.961874in}}%
\pgfpathlineto{\pgfqpoint{1.778641in}{0.952159in}}%
\pgfpathlineto{\pgfqpoint{1.774962in}{0.949616in}}%
\pgfpathlineto{\pgfqpoint{1.771292in}{0.947356in}}%
\pgfpathlineto{\pgfqpoint{1.767630in}{0.945375in}}%
\pgfpathlineto{\pgfqpoint{1.763976in}{0.943668in}}%
\pgfpathlineto{\pgfqpoint{1.758959in}{0.953156in}}%
\pgfpathlineto{\pgfqpoint{1.753386in}{0.962546in}}%
\pgfpathlineto{\pgfqpoint{1.747263in}{0.971828in}}%
\pgfpathlineto{\pgfqpoint{1.740599in}{0.980994in}}%
\pgfpathclose%
\pgfusepath{fill}%
\end{pgfscope}%
\begin{pgfscope}%
\pgfpathrectangle{\pgfqpoint{0.041670in}{0.041670in}}{\pgfqpoint{2.216660in}{2.216660in}}%
\pgfusepath{clip}%
\pgfsetbuttcap%
\pgfsetroundjoin%
\definecolor{currentfill}{rgb}{0.267004,0.004874,0.329415}%
\pgfsetfillcolor{currentfill}%
\pgfsetlinewidth{0.000000pt}%
\definecolor{currentstroke}{rgb}{0.000000,0.000000,0.000000}%
\pgfsetstrokecolor{currentstroke}%
\pgfsetdash{}{0pt}%
\pgfpathmoveto{\pgfqpoint{1.726709in}{0.975898in}}%
\pgfpathlineto{\pgfqpoint{1.730172in}{0.976787in}}%
\pgfpathlineto{\pgfqpoint{1.733641in}{0.977930in}}%
\pgfpathlineto{\pgfqpoint{1.737116in}{0.979331in}}%
\pgfpathlineto{\pgfqpoint{1.740599in}{0.980994in}}%
\pgfpathlineto{\pgfqpoint{1.747263in}{0.971828in}}%
\pgfpathlineto{\pgfqpoint{1.753386in}{0.962546in}}%
\pgfpathlineto{\pgfqpoint{1.758959in}{0.953156in}}%
\pgfpathlineto{\pgfqpoint{1.763976in}{0.943668in}}%
\pgfpathlineto{\pgfqpoint{1.760329in}{0.942229in}}%
\pgfpathlineto{\pgfqpoint{1.756690in}{0.941054in}}%
\pgfpathlineto{\pgfqpoint{1.753058in}{0.940138in}}%
\pgfpathlineto{\pgfqpoint{1.749433in}{0.939475in}}%
\pgfpathlineto{\pgfqpoint{1.744560in}{0.948733in}}%
\pgfpathlineto{\pgfqpoint{1.739144in}{0.957895in}}%
\pgfpathlineto{\pgfqpoint{1.733191in}{0.966953in}}%
\pgfpathlineto{\pgfqpoint{1.726709in}{0.975898in}}%
\pgfpathclose%
\pgfusepath{fill}%
\end{pgfscope}%
\begin{pgfscope}%
\pgfpathrectangle{\pgfqpoint{0.041670in}{0.041670in}}{\pgfqpoint{2.216660in}{2.216660in}}%
\pgfusepath{clip}%
\pgfsetbuttcap%
\pgfsetroundjoin%
\definecolor{currentfill}{rgb}{0.260571,0.246922,0.522828}%
\pgfsetfillcolor{currentfill}%
\pgfsetlinewidth{0.000000pt}%
\definecolor{currentstroke}{rgb}{0.000000,0.000000,0.000000}%
\pgfsetstrokecolor{currentstroke}%
\pgfsetdash{}{0pt}%
\pgfpathmoveto{\pgfqpoint{0.525939in}{1.104108in}}%
\pgfpathlineto{\pgfqpoint{0.522141in}{1.114049in}}%
\pgfpathlineto{\pgfqpoint{0.518327in}{1.124402in}}%
\pgfpathlineto{\pgfqpoint{0.514497in}{1.135173in}}%
\pgfpathlineto{\pgfqpoint{0.510650in}{1.146370in}}%
\pgfpathlineto{\pgfqpoint{0.518747in}{1.156916in}}%
\pgfpathlineto{\pgfqpoint{0.527467in}{1.167313in}}%
\pgfpathlineto{\pgfqpoint{0.536798in}{1.177552in}}%
\pgfpathlineto{\pgfqpoint{0.546730in}{1.187624in}}%
\pgfpathlineto{\pgfqpoint{0.550347in}{1.176258in}}%
\pgfpathlineto{\pgfqpoint{0.553950in}{1.165315in}}%
\pgfpathlineto{\pgfqpoint{0.557538in}{1.154788in}}%
\pgfpathlineto{\pgfqpoint{0.561111in}{1.144671in}}%
\pgfpathlineto{\pgfqpoint{0.551426in}{1.134766in}}%
\pgfpathlineto{\pgfqpoint{0.542328in}{1.124699in}}%
\pgfpathlineto{\pgfqpoint{0.533829in}{1.114476in}}%
\pgfpathlineto{\pgfqpoint{0.525939in}{1.104108in}}%
\pgfpathclose%
\pgfusepath{fill}%
\end{pgfscope}%
\begin{pgfscope}%
\pgfpathrectangle{\pgfqpoint{0.041670in}{0.041670in}}{\pgfqpoint{2.216660in}{2.216660in}}%
\pgfusepath{clip}%
\pgfsetbuttcap%
\pgfsetroundjoin%
\definecolor{currentfill}{rgb}{0.195860,0.395433,0.555276}%
\pgfsetfillcolor{currentfill}%
\pgfsetlinewidth{0.000000pt}%
\definecolor{currentstroke}{rgb}{0.000000,0.000000,0.000000}%
\pgfsetstrokecolor{currentstroke}%
\pgfsetdash{}{0pt}%
\pgfpathmoveto{\pgfqpoint{0.960184in}{1.288335in}}%
\pgfpathlineto{\pgfqpoint{0.958113in}{1.282488in}}%
\pgfpathlineto{\pgfqpoint{0.956043in}{1.276681in}}%
\pgfpathlineto{\pgfqpoint{0.953976in}{1.270917in}}%
\pgfpathlineto{\pgfqpoint{0.951911in}{1.265199in}}%
\pgfpathlineto{\pgfqpoint{0.962015in}{1.268675in}}%
\pgfpathlineto{\pgfqpoint{0.972321in}{1.271991in}}%
\pgfpathlineto{\pgfqpoint{0.982819in}{1.275143in}}%
\pgfpathlineto{\pgfqpoint{0.993500in}{1.278128in}}%
\pgfpathlineto{\pgfqpoint{0.995185in}{1.283719in}}%
\pgfpathlineto{\pgfqpoint{0.996873in}{1.289355in}}%
\pgfpathlineto{\pgfqpoint{0.998561in}{1.295035in}}%
\pgfpathlineto{\pgfqpoint{1.000252in}{1.300754in}}%
\pgfpathlineto{\pgfqpoint{0.989961in}{1.297886in}}%
\pgfpathlineto{\pgfqpoint{0.979847in}{1.294859in}}%
\pgfpathlineto{\pgfqpoint{0.969918in}{1.291674in}}%
\pgfpathlineto{\pgfqpoint{0.960184in}{1.288335in}}%
\pgfpathclose%
\pgfusepath{fill}%
\end{pgfscope}%
\begin{pgfscope}%
\pgfpathrectangle{\pgfqpoint{0.041670in}{0.041670in}}{\pgfqpoint{2.216660in}{2.216660in}}%
\pgfusepath{clip}%
\pgfsetbuttcap%
\pgfsetroundjoin%
\definecolor{currentfill}{rgb}{0.272594,0.025563,0.353093}%
\pgfsetfillcolor{currentfill}%
\pgfsetlinewidth{0.000000pt}%
\definecolor{currentstroke}{rgb}{0.000000,0.000000,0.000000}%
\pgfsetstrokecolor{currentstroke}%
\pgfsetdash{}{0pt}%
\pgfpathmoveto{\pgfqpoint{1.754602in}{0.990372in}}%
\pgfpathlineto{\pgfqpoint{1.758122in}{0.993423in}}%
\pgfpathlineto{\pgfqpoint{1.761651in}{0.996767in}}%
\pgfpathlineto{\pgfqpoint{1.765189in}{1.000409in}}%
\pgfpathlineto{\pgfqpoint{1.768736in}{1.004354in}}%
\pgfpathlineto{\pgfqpoint{1.775774in}{0.994759in}}%
\pgfpathlineto{\pgfqpoint{1.782244in}{0.985041in}}%
\pgfpathlineto{\pgfqpoint{1.788138in}{0.975210in}}%
\pgfpathlineto{\pgfqpoint{1.793448in}{0.965273in}}%
\pgfpathlineto{\pgfqpoint{1.789732in}{0.961543in}}%
\pgfpathlineto{\pgfqpoint{1.786025in}{0.958117in}}%
\pgfpathlineto{\pgfqpoint{1.782329in}{0.954991in}}%
\pgfpathlineto{\pgfqpoint{1.778641in}{0.952159in}}%
\pgfpathlineto{\pgfqpoint{1.773479in}{0.961874in}}%
\pgfpathlineto{\pgfqpoint{1.767747in}{0.971487in}}%
\pgfpathlineto{\pgfqpoint{1.761452in}{0.980989in}}%
\pgfpathlineto{\pgfqpoint{1.754602in}{0.990372in}}%
\pgfpathclose%
\pgfusepath{fill}%
\end{pgfscope}%
\begin{pgfscope}%
\pgfpathrectangle{\pgfqpoint{0.041670in}{0.041670in}}{\pgfqpoint{2.216660in}{2.216660in}}%
\pgfusepath{clip}%
\pgfsetbuttcap%
\pgfsetroundjoin%
\definecolor{currentfill}{rgb}{0.267004,0.004874,0.329415}%
\pgfsetfillcolor{currentfill}%
\pgfsetlinewidth{0.000000pt}%
\definecolor{currentstroke}{rgb}{0.000000,0.000000,0.000000}%
\pgfsetstrokecolor{currentstroke}%
\pgfsetdash{}{0pt}%
\pgfpathmoveto{\pgfqpoint{1.712916in}{0.974775in}}%
\pgfpathlineto{\pgfqpoint{1.716356in}{0.974700in}}%
\pgfpathlineto{\pgfqpoint{1.719801in}{0.974859in}}%
\pgfpathlineto{\pgfqpoint{1.723252in}{0.975256in}}%
\pgfpathlineto{\pgfqpoint{1.726709in}{0.975898in}}%
\pgfpathlineto{\pgfqpoint{1.733191in}{0.966953in}}%
\pgfpathlineto{\pgfqpoint{1.739144in}{0.957895in}}%
\pgfpathlineto{\pgfqpoint{1.744560in}{0.948733in}}%
\pgfpathlineto{\pgfqpoint{1.749433in}{0.939475in}}%
\pgfpathlineto{\pgfqpoint{1.745814in}{0.939062in}}%
\pgfpathlineto{\pgfqpoint{1.742201in}{0.938893in}}%
\pgfpathlineto{\pgfqpoint{1.738595in}{0.938964in}}%
\pgfpathlineto{\pgfqpoint{1.734995in}{0.939270in}}%
\pgfpathlineto{\pgfqpoint{1.730264in}{0.948294in}}%
\pgfpathlineto{\pgfqpoint{1.725002in}{0.957225in}}%
\pgfpathlineto{\pgfqpoint{1.719217in}{0.966054in}}%
\pgfpathlineto{\pgfqpoint{1.712916in}{0.974775in}}%
\pgfpathclose%
\pgfusepath{fill}%
\end{pgfscope}%
\begin{pgfscope}%
\pgfpathrectangle{\pgfqpoint{0.041670in}{0.041670in}}{\pgfqpoint{2.216660in}{2.216660in}}%
\pgfusepath{clip}%
\pgfsetbuttcap%
\pgfsetroundjoin%
\definecolor{currentfill}{rgb}{0.231674,0.318106,0.544834}%
\pgfsetfillcolor{currentfill}%
\pgfsetlinewidth{0.000000pt}%
\definecolor{currentstroke}{rgb}{0.000000,0.000000,0.000000}%
\pgfsetstrokecolor{currentstroke}%
\pgfsetdash{}{0pt}%
\pgfpathmoveto{\pgfqpoint{0.868566in}{1.208321in}}%
\pgfpathlineto{\pgfqpoint{0.865841in}{1.202551in}}%
\pgfpathlineto{\pgfqpoint{0.863119in}{1.196846in}}%
\pgfpathlineto{\pgfqpoint{0.860398in}{1.191209in}}%
\pgfpathlineto{\pgfqpoint{0.857680in}{1.185643in}}%
\pgfpathlineto{\pgfqpoint{0.866439in}{1.190656in}}%
\pgfpathlineto{\pgfqpoint{0.875491in}{1.195525in}}%
\pgfpathlineto{\pgfqpoint{0.884828in}{1.200246in}}%
\pgfpathlineto{\pgfqpoint{0.894441in}{1.204815in}}%
\pgfpathlineto{\pgfqpoint{0.896843in}{1.210204in}}%
\pgfpathlineto{\pgfqpoint{0.899247in}{1.215664in}}%
\pgfpathlineto{\pgfqpoint{0.901652in}{1.221193in}}%
\pgfpathlineto{\pgfqpoint{0.904059in}{1.226786in}}%
\pgfpathlineto{\pgfqpoint{0.894777in}{1.222386in}}%
\pgfpathlineto{\pgfqpoint{0.885762in}{1.217839in}}%
\pgfpathlineto{\pgfqpoint{0.877021in}{1.213149in}}%
\pgfpathlineto{\pgfqpoint{0.868566in}{1.208321in}}%
\pgfpathclose%
\pgfusepath{fill}%
\end{pgfscope}%
\begin{pgfscope}%
\pgfpathrectangle{\pgfqpoint{0.041670in}{0.041670in}}{\pgfqpoint{2.216660in}{2.216660in}}%
\pgfusepath{clip}%
\pgfsetbuttcap%
\pgfsetroundjoin%
\definecolor{currentfill}{rgb}{0.274128,0.199721,0.498911}%
\pgfsetfillcolor{currentfill}%
\pgfsetlinewidth{0.000000pt}%
\definecolor{currentstroke}{rgb}{0.000000,0.000000,0.000000}%
\pgfsetstrokecolor{currentstroke}%
\pgfsetdash{}{0pt}%
\pgfpathmoveto{\pgfqpoint{1.550970in}{1.126410in}}%
\pgfpathlineto{\pgfqpoint{1.553898in}{1.121487in}}%
\pgfpathlineto{\pgfqpoint{1.556826in}{1.116670in}}%
\pgfpathlineto{\pgfqpoint{1.559752in}{1.111962in}}%
\pgfpathlineto{\pgfqpoint{1.562679in}{1.107367in}}%
\pgfpathlineto{\pgfqpoint{1.570927in}{1.101192in}}%
\pgfpathlineto{\pgfqpoint{1.578807in}{1.094883in}}%
\pgfpathlineto{\pgfqpoint{1.586310in}{1.088446in}}%
\pgfpathlineto{\pgfqpoint{1.593428in}{1.081885in}}%
\pgfpathlineto{\pgfqpoint{1.590255in}{1.086693in}}%
\pgfpathlineto{\pgfqpoint{1.587082in}{1.091613in}}%
\pgfpathlineto{\pgfqpoint{1.583908in}{1.096643in}}%
\pgfpathlineto{\pgfqpoint{1.580734in}{1.101778in}}%
\pgfpathlineto{\pgfqpoint{1.573846in}{1.108119in}}%
\pgfpathlineto{\pgfqpoint{1.566584in}{1.114342in}}%
\pgfpathlineto{\pgfqpoint{1.558956in}{1.120441in}}%
\pgfpathlineto{\pgfqpoint{1.550970in}{1.126410in}}%
\pgfpathclose%
\pgfusepath{fill}%
\end{pgfscope}%
\begin{pgfscope}%
\pgfpathrectangle{\pgfqpoint{0.041670in}{0.041670in}}{\pgfqpoint{2.216660in}{2.216660in}}%
\pgfusepath{clip}%
\pgfsetbuttcap%
\pgfsetroundjoin%
\definecolor{currentfill}{rgb}{0.233603,0.313828,0.543914}%
\pgfsetfillcolor{currentfill}%
\pgfsetlinewidth{0.000000pt}%
\definecolor{currentstroke}{rgb}{0.000000,0.000000,0.000000}%
\pgfsetstrokecolor{currentstroke}%
\pgfsetdash{}{0pt}%
\pgfpathmoveto{\pgfqpoint{1.803858in}{1.196430in}}%
\pgfpathlineto{\pgfqpoint{1.807434in}{1.208262in}}%
\pgfpathlineto{\pgfqpoint{1.811025in}{1.220531in}}%
\pgfpathlineto{\pgfqpoint{1.814633in}{1.233243in}}%
\pgfpathlineto{\pgfqpoint{1.818258in}{1.246407in}}%
\pgfpathlineto{\pgfqpoint{1.828975in}{1.236335in}}%
\pgfpathlineto{\pgfqpoint{1.839091in}{1.226085in}}%
\pgfpathlineto{\pgfqpoint{1.848592in}{1.215666in}}%
\pgfpathlineto{\pgfqpoint{1.857467in}{1.205086in}}%
\pgfpathlineto{\pgfqpoint{1.853597in}{1.192079in}}%
\pgfpathlineto{\pgfqpoint{1.849745in}{1.179526in}}%
\pgfpathlineto{\pgfqpoint{1.845910in}{1.167419in}}%
\pgfpathlineto{\pgfqpoint{1.842093in}{1.155752in}}%
\pgfpathlineto{\pgfqpoint{1.833442in}{1.166166in}}%
\pgfpathlineto{\pgfqpoint{1.824178in}{1.176423in}}%
\pgfpathlineto{\pgfqpoint{1.814313in}{1.186514in}}%
\pgfpathlineto{\pgfqpoint{1.803858in}{1.196430in}}%
\pgfpathclose%
\pgfusepath{fill}%
\end{pgfscope}%
\begin{pgfscope}%
\pgfpathrectangle{\pgfqpoint{0.041670in}{0.041670in}}{\pgfqpoint{2.216660in}{2.216660in}}%
\pgfusepath{clip}%
\pgfsetbuttcap%
\pgfsetroundjoin%
\definecolor{currentfill}{rgb}{0.277941,0.056324,0.381191}%
\pgfsetfillcolor{currentfill}%
\pgfsetlinewidth{0.000000pt}%
\definecolor{currentstroke}{rgb}{0.000000,0.000000,0.000000}%
\pgfsetstrokecolor{currentstroke}%
\pgfsetdash{}{0pt}%
\pgfpathmoveto{\pgfqpoint{1.768736in}{1.004354in}}%
\pgfpathlineto{\pgfqpoint{1.772292in}{1.008609in}}%
\pgfpathlineto{\pgfqpoint{1.775858in}{1.013177in}}%
\pgfpathlineto{\pgfqpoint{1.779434in}{1.018065in}}%
\pgfpathlineto{\pgfqpoint{1.783021in}{1.023279in}}%
\pgfpathlineto{\pgfqpoint{1.790249in}{1.013476in}}%
\pgfpathlineto{\pgfqpoint{1.796897in}{1.003548in}}%
\pgfpathlineto{\pgfqpoint{1.802955in}{0.993504in}}%
\pgfpathlineto{\pgfqpoint{1.808416in}{0.983351in}}%
\pgfpathlineto{\pgfqpoint{1.804657in}{0.978347in}}%
\pgfpathlineto{\pgfqpoint{1.800910in}{0.973670in}}%
\pgfpathlineto{\pgfqpoint{1.797174in}{0.969314in}}%
\pgfpathlineto{\pgfqpoint{1.793448in}{0.965273in}}%
\pgfpathlineto{\pgfqpoint{1.788138in}{0.975210in}}%
\pgfpathlineto{\pgfqpoint{1.782244in}{0.985041in}}%
\pgfpathlineto{\pgfqpoint{1.775774in}{0.994759in}}%
\pgfpathlineto{\pgfqpoint{1.768736in}{1.004354in}}%
\pgfpathclose%
\pgfusepath{fill}%
\end{pgfscope}%
\begin{pgfscope}%
\pgfpathrectangle{\pgfqpoint{0.041670in}{0.041670in}}{\pgfqpoint{2.216660in}{2.216660in}}%
\pgfusepath{clip}%
\pgfsetbuttcap%
\pgfsetroundjoin%
\definecolor{currentfill}{rgb}{0.179019,0.433756,0.557430}%
\pgfsetfillcolor{currentfill}%
\pgfsetlinewidth{0.000000pt}%
\definecolor{currentstroke}{rgb}{0.000000,0.000000,0.000000}%
\pgfsetstrokecolor{currentstroke}%
\pgfsetdash{}{0pt}%
\pgfpathmoveto{\pgfqpoint{1.302347in}{1.335142in}}%
\pgfpathlineto{\pgfqpoint{1.303546in}{1.329412in}}%
\pgfpathlineto{\pgfqpoint{1.304745in}{1.323710in}}%
\pgfpathlineto{\pgfqpoint{1.305942in}{1.318039in}}%
\pgfpathlineto{\pgfqpoint{1.307137in}{1.312402in}}%
\pgfpathlineto{\pgfqpoint{1.318147in}{1.310347in}}%
\pgfpathlineto{\pgfqpoint{1.329029in}{1.308121in}}%
\pgfpathlineto{\pgfqpoint{1.339773in}{1.305728in}}%
\pgfpathlineto{\pgfqpoint{1.350370in}{1.303168in}}%
\pgfpathlineto{\pgfqpoint{1.348766in}{1.308899in}}%
\pgfpathlineto{\pgfqpoint{1.347160in}{1.314665in}}%
\pgfpathlineto{\pgfqpoint{1.345552in}{1.320462in}}%
\pgfpathlineto{\pgfqpoint{1.343943in}{1.326286in}}%
\pgfpathlineto{\pgfqpoint{1.333748in}{1.328741in}}%
\pgfpathlineto{\pgfqpoint{1.323410in}{1.331037in}}%
\pgfpathlineto{\pgfqpoint{1.312940in}{1.333171in}}%
\pgfpathlineto{\pgfqpoint{1.302347in}{1.335142in}}%
\pgfpathclose%
\pgfusepath{fill}%
\end{pgfscope}%
\begin{pgfscope}%
\pgfpathrectangle{\pgfqpoint{0.041670in}{0.041670in}}{\pgfqpoint{2.216660in}{2.216660in}}%
\pgfusepath{clip}%
\pgfsetbuttcap%
\pgfsetroundjoin%
\definecolor{currentfill}{rgb}{0.248629,0.278775,0.534556}%
\pgfsetfillcolor{currentfill}%
\pgfsetlinewidth{0.000000pt}%
\definecolor{currentstroke}{rgb}{0.000000,0.000000,0.000000}%
\pgfsetstrokecolor{currentstroke}%
\pgfsetdash{}{0pt}%
\pgfpathmoveto{\pgfqpoint{1.494459in}{1.190106in}}%
\pgfpathlineto{\pgfqpoint{1.497109in}{1.184657in}}%
\pgfpathlineto{\pgfqpoint{1.499758in}{1.179286in}}%
\pgfpathlineto{\pgfqpoint{1.502405in}{1.173996in}}%
\pgfpathlineto{\pgfqpoint{1.505051in}{1.168791in}}%
\pgfpathlineto{\pgfqpoint{1.514079in}{1.163576in}}%
\pgfpathlineto{\pgfqpoint{1.522794in}{1.158218in}}%
\pgfpathlineto{\pgfqpoint{1.531187in}{1.152719in}}%
\pgfpathlineto{\pgfqpoint{1.539249in}{1.147086in}}%
\pgfpathlineto{\pgfqpoint{1.536316in}{1.152484in}}%
\pgfpathlineto{\pgfqpoint{1.533381in}{1.157968in}}%
\pgfpathlineto{\pgfqpoint{1.530445in}{1.163532in}}%
\pgfpathlineto{\pgfqpoint{1.527508in}{1.169175in}}%
\pgfpathlineto{\pgfqpoint{1.519718in}{1.174607in}}%
\pgfpathlineto{\pgfqpoint{1.511607in}{1.179910in}}%
\pgfpathlineto{\pgfqpoint{1.503185in}{1.185077in}}%
\pgfpathlineto{\pgfqpoint{1.494459in}{1.190106in}}%
\pgfpathclose%
\pgfusepath{fill}%
\end{pgfscope}%
\begin{pgfscope}%
\pgfpathrectangle{\pgfqpoint{0.041670in}{0.041670in}}{\pgfqpoint{2.216660in}{2.216660in}}%
\pgfusepath{clip}%
\pgfsetbuttcap%
\pgfsetroundjoin%
\definecolor{currentfill}{rgb}{0.268510,0.009605,0.335427}%
\pgfsetfillcolor{currentfill}%
\pgfsetlinewidth{0.000000pt}%
\definecolor{currentstroke}{rgb}{0.000000,0.000000,0.000000}%
\pgfsetstrokecolor{currentstroke}%
\pgfsetdash{}{0pt}%
\pgfpathmoveto{\pgfqpoint{1.699204in}{0.977329in}}%
\pgfpathlineto{\pgfqpoint{1.702625in}{0.976361in}}%
\pgfpathlineto{\pgfqpoint{1.706051in}{0.975610in}}%
\pgfpathlineto{\pgfqpoint{1.709481in}{0.975080in}}%
\pgfpathlineto{\pgfqpoint{1.712916in}{0.974775in}}%
\pgfpathlineto{\pgfqpoint{1.719217in}{0.966054in}}%
\pgfpathlineto{\pgfqpoint{1.725002in}{0.957225in}}%
\pgfpathlineto{\pgfqpoint{1.730264in}{0.948294in}}%
\pgfpathlineto{\pgfqpoint{1.734995in}{0.939270in}}%
\pgfpathlineto{\pgfqpoint{1.731400in}{0.939807in}}%
\pgfpathlineto{\pgfqpoint{1.727810in}{0.940569in}}%
\pgfpathlineto{\pgfqpoint{1.724225in}{0.941553in}}%
\pgfpathlineto{\pgfqpoint{1.720645in}{0.942754in}}%
\pgfpathlineto{\pgfqpoint{1.716054in}{0.951540in}}%
\pgfpathlineto{\pgfqpoint{1.710946in}{0.960237in}}%
\pgfpathlineto{\pgfqpoint{1.705326in}{0.968836in}}%
\pgfpathlineto{\pgfqpoint{1.699204in}{0.977329in}}%
\pgfpathclose%
\pgfusepath{fill}%
\end{pgfscope}%
\begin{pgfscope}%
\pgfpathrectangle{\pgfqpoint{0.041670in}{0.041670in}}{\pgfqpoint{2.216660in}{2.216660in}}%
\pgfusepath{clip}%
\pgfsetbuttcap%
\pgfsetroundjoin%
\definecolor{currentfill}{rgb}{0.282327,0.094955,0.417331}%
\pgfsetfillcolor{currentfill}%
\pgfsetlinewidth{0.000000pt}%
\definecolor{currentstroke}{rgb}{0.000000,0.000000,0.000000}%
\pgfsetstrokecolor{currentstroke}%
\pgfsetdash{}{0pt}%
\pgfpathmoveto{\pgfqpoint{0.710173in}{1.012030in}}%
\pgfpathlineto{\pgfqpoint{0.706756in}{1.008123in}}%
\pgfpathlineto{\pgfqpoint{0.703338in}{1.004366in}}%
\pgfpathlineto{\pgfqpoint{0.699918in}{1.000763in}}%
\pgfpathlineto{\pgfqpoint{0.696497in}{0.997318in}}%
\pgfpathlineto{\pgfqpoint{0.702144in}{1.005103in}}%
\pgfpathlineto{\pgfqpoint{0.708249in}{1.012784in}}%
\pgfpathlineto{\pgfqpoint{0.714805in}{1.020354in}}%
\pgfpathlineto{\pgfqpoint{0.721803in}{1.027807in}}%
\pgfpathlineto{\pgfqpoint{0.725031in}{1.031022in}}%
\pgfpathlineto{\pgfqpoint{0.728258in}{1.034395in}}%
\pgfpathlineto{\pgfqpoint{0.731484in}{1.037921in}}%
\pgfpathlineto{\pgfqpoint{0.734709in}{1.041598in}}%
\pgfpathlineto{\pgfqpoint{0.727921in}{1.034369in}}%
\pgfpathlineto{\pgfqpoint{0.721565in}{1.027027in}}%
\pgfpathlineto{\pgfqpoint{0.715646in}{1.019579in}}%
\pgfpathlineto{\pgfqpoint{0.710173in}{1.012030in}}%
\pgfpathclose%
\pgfusepath{fill}%
\end{pgfscope}%
\begin{pgfscope}%
\pgfpathrectangle{\pgfqpoint{0.041670in}{0.041670in}}{\pgfqpoint{2.216660in}{2.216660in}}%
\pgfusepath{clip}%
\pgfsetbuttcap%
\pgfsetroundjoin%
\definecolor{currentfill}{rgb}{0.179019,0.433756,0.557430}%
\pgfsetfillcolor{currentfill}%
\pgfsetlinewidth{0.000000pt}%
\definecolor{currentstroke}{rgb}{0.000000,0.000000,0.000000}%
\pgfsetstrokecolor{currentstroke}%
\pgfsetdash{}{0pt}%
\pgfpathmoveto{\pgfqpoint{1.007032in}{1.323972in}}%
\pgfpathlineto{\pgfqpoint{1.005334in}{1.318123in}}%
\pgfpathlineto{\pgfqpoint{1.003638in}{1.312301in}}%
\pgfpathlineto{\pgfqpoint{1.001944in}{1.306511in}}%
\pgfpathlineto{\pgfqpoint{1.000252in}{1.300754in}}%
\pgfpathlineto{\pgfqpoint{1.010709in}{1.303460in}}%
\pgfpathlineto{\pgfqpoint{1.021323in}{1.306002in}}%
\pgfpathlineto{\pgfqpoint{1.032083in}{1.308377in}}%
\pgfpathlineto{\pgfqpoint{1.042980in}{1.310584in}}%
\pgfpathlineto{\pgfqpoint{1.044268in}{1.316239in}}%
\pgfpathlineto{\pgfqpoint{1.045558in}{1.321929in}}%
\pgfpathlineto{\pgfqpoint{1.046848in}{1.327650in}}%
\pgfpathlineto{\pgfqpoint{1.048141in}{1.333398in}}%
\pgfpathlineto{\pgfqpoint{1.037656in}{1.331282in}}%
\pgfpathlineto{\pgfqpoint{1.027304in}{1.329004in}}%
\pgfpathlineto{\pgfqpoint{1.017092in}{1.326567in}}%
\pgfpathlineto{\pgfqpoint{1.007032in}{1.323972in}}%
\pgfpathclose%
\pgfusepath{fill}%
\end{pgfscope}%
\begin{pgfscope}%
\pgfpathrectangle{\pgfqpoint{0.041670in}{0.041670in}}{\pgfqpoint{2.216660in}{2.216660in}}%
\pgfusepath{clip}%
\pgfsetbuttcap%
\pgfsetroundjoin%
\definecolor{currentfill}{rgb}{0.163625,0.471133,0.558148}%
\pgfsetfillcolor{currentfill}%
\pgfsetlinewidth{0.000000pt}%
\definecolor{currentstroke}{rgb}{0.000000,0.000000,0.000000}%
\pgfsetstrokecolor{currentstroke}%
\pgfsetdash{}{0pt}%
\pgfpathmoveto{\pgfqpoint{1.137306in}{1.367075in}}%
\pgfpathlineto{\pgfqpoint{1.136870in}{1.361360in}}%
\pgfpathlineto{\pgfqpoint{1.136434in}{1.355661in}}%
\pgfpathlineto{\pgfqpoint{1.135998in}{1.349981in}}%
\pgfpathlineto{\pgfqpoint{1.135563in}{1.344324in}}%
\pgfpathlineto{\pgfqpoint{1.146782in}{1.344921in}}%
\pgfpathlineto{\pgfqpoint{1.158032in}{1.345346in}}%
\pgfpathlineto{\pgfqpoint{1.169302in}{1.345597in}}%
\pgfpathlineto{\pgfqpoint{1.180582in}{1.345674in}}%
\pgfpathlineto{\pgfqpoint{1.180575in}{1.351317in}}%
\pgfpathlineto{\pgfqpoint{1.180569in}{1.356982in}}%
\pgfpathlineto{\pgfqpoint{1.180563in}{1.362667in}}%
\pgfpathlineto{\pgfqpoint{1.180557in}{1.368368in}}%
\pgfpathlineto{\pgfqpoint{1.169720in}{1.368294in}}%
\pgfpathlineto{\pgfqpoint{1.158893in}{1.368054in}}%
\pgfpathlineto{\pgfqpoint{1.148085in}{1.367647in}}%
\pgfpathlineto{\pgfqpoint{1.137306in}{1.367075in}}%
\pgfpathclose%
\pgfusepath{fill}%
\end{pgfscope}%
\begin{pgfscope}%
\pgfpathrectangle{\pgfqpoint{0.041670in}{0.041670in}}{\pgfqpoint{2.216660in}{2.216660in}}%
\pgfusepath{clip}%
\pgfsetbuttcap%
\pgfsetroundjoin%
\definecolor{currentfill}{rgb}{0.163625,0.471133,0.558148}%
\pgfsetfillcolor{currentfill}%
\pgfsetlinewidth{0.000000pt}%
\definecolor{currentstroke}{rgb}{0.000000,0.000000,0.000000}%
\pgfsetstrokecolor{currentstroke}%
\pgfsetdash{}{0pt}%
\pgfpathmoveto{\pgfqpoint{1.180557in}{1.368368in}}%
\pgfpathlineto{\pgfqpoint{1.180563in}{1.362667in}}%
\pgfpathlineto{\pgfqpoint{1.180569in}{1.356982in}}%
\pgfpathlineto{\pgfqpoint{1.180575in}{1.351317in}}%
\pgfpathlineto{\pgfqpoint{1.180582in}{1.345674in}}%
\pgfpathlineto{\pgfqpoint{1.191861in}{1.345578in}}%
\pgfpathlineto{\pgfqpoint{1.203129in}{1.345307in}}%
\pgfpathlineto{\pgfqpoint{1.214376in}{1.344863in}}%
\pgfpathlineto{\pgfqpoint{1.225591in}{1.344246in}}%
\pgfpathlineto{\pgfqpoint{1.225143in}{1.349905in}}%
\pgfpathlineto{\pgfqpoint{1.224696in}{1.355585in}}%
\pgfpathlineto{\pgfqpoint{1.224247in}{1.361285in}}%
\pgfpathlineto{\pgfqpoint{1.223799in}{1.367001in}}%
\pgfpathlineto{\pgfqpoint{1.213024in}{1.367592in}}%
\pgfpathlineto{\pgfqpoint{1.202219in}{1.368017in}}%
\pgfpathlineto{\pgfqpoint{1.191393in}{1.368275in}}%
\pgfpathlineto{\pgfqpoint{1.180557in}{1.368368in}}%
\pgfpathclose%
\pgfusepath{fill}%
\end{pgfscope}%
\begin{pgfscope}%
\pgfpathrectangle{\pgfqpoint{0.041670in}{0.041670in}}{\pgfqpoint{2.216660in}{2.216660in}}%
\pgfusepath{clip}%
\pgfsetbuttcap%
\pgfsetroundjoin%
\definecolor{currentfill}{rgb}{0.283072,0.130895,0.449241}%
\pgfsetfillcolor{currentfill}%
\pgfsetlinewidth{0.000000pt}%
\definecolor{currentstroke}{rgb}{0.000000,0.000000,0.000000}%
\pgfsetstrokecolor{currentstroke}%
\pgfsetdash{}{0pt}%
\pgfpathmoveto{\pgfqpoint{1.606117in}{1.063856in}}%
\pgfpathlineto{\pgfqpoint{1.609290in}{1.059667in}}%
\pgfpathlineto{\pgfqpoint{1.612464in}{1.055612in}}%
\pgfpathlineto{\pgfqpoint{1.615638in}{1.051696in}}%
\pgfpathlineto{\pgfqpoint{1.618812in}{1.047923in}}%
\pgfpathlineto{\pgfqpoint{1.625976in}{1.040801in}}%
\pgfpathlineto{\pgfqpoint{1.632716in}{1.033559in}}%
\pgfpathlineto{\pgfqpoint{1.639025in}{1.026205in}}%
\pgfpathlineto{\pgfqpoint{1.644894in}{1.018745in}}%
\pgfpathlineto{\pgfqpoint{1.641517in}{1.022746in}}%
\pgfpathlineto{\pgfqpoint{1.638141in}{1.026889in}}%
\pgfpathlineto{\pgfqpoint{1.634765in}{1.031172in}}%
\pgfpathlineto{\pgfqpoint{1.631391in}{1.035589in}}%
\pgfpathlineto{\pgfqpoint{1.625705in}{1.042816in}}%
\pgfpathlineto{\pgfqpoint{1.619593in}{1.049940in}}%
\pgfpathlineto{\pgfqpoint{1.613061in}{1.056955in}}%
\pgfpathlineto{\pgfqpoint{1.606117in}{1.063856in}}%
\pgfpathclose%
\pgfusepath{fill}%
\end{pgfscope}%
\begin{pgfscope}%
\pgfpathrectangle{\pgfqpoint{0.041670in}{0.041670in}}{\pgfqpoint{2.216660in}{2.216660in}}%
\pgfusepath{clip}%
\pgfsetbuttcap%
\pgfsetroundjoin%
\definecolor{currentfill}{rgb}{0.282327,0.094955,0.417331}%
\pgfsetfillcolor{currentfill}%
\pgfsetlinewidth{0.000000pt}%
\definecolor{currentstroke}{rgb}{0.000000,0.000000,0.000000}%
\pgfsetstrokecolor{currentstroke}%
\pgfsetdash{}{0pt}%
\pgfpathmoveto{\pgfqpoint{1.783021in}{1.023279in}}%
\pgfpathlineto{\pgfqpoint{1.786618in}{1.028823in}}%
\pgfpathlineto{\pgfqpoint{1.790226in}{1.034704in}}%
\pgfpathlineto{\pgfqpoint{1.793845in}{1.040927in}}%
\pgfpathlineto{\pgfqpoint{1.797476in}{1.047498in}}%
\pgfpathlineto{\pgfqpoint{1.804898in}{1.037495in}}%
\pgfpathlineto{\pgfqpoint{1.811727in}{1.027363in}}%
\pgfpathlineto{\pgfqpoint{1.817952in}{1.017111in}}%
\pgfpathlineto{\pgfqpoint{1.823567in}{1.006748in}}%
\pgfpathlineto{\pgfqpoint{1.819760in}{1.000380in}}%
\pgfpathlineto{\pgfqpoint{1.815967in}{0.994362in}}%
\pgfpathlineto{\pgfqpoint{1.812185in}{0.988687in}}%
\pgfpathlineto{\pgfqpoint{1.808416in}{0.983351in}}%
\pgfpathlineto{\pgfqpoint{1.802955in}{0.993504in}}%
\pgfpathlineto{\pgfqpoint{1.796897in}{1.003548in}}%
\pgfpathlineto{\pgfqpoint{1.790249in}{1.013476in}}%
\pgfpathlineto{\pgfqpoint{1.783021in}{1.023279in}}%
\pgfpathclose%
\pgfusepath{fill}%
\end{pgfscope}%
\begin{pgfscope}%
\pgfpathrectangle{\pgfqpoint{0.041670in}{0.041670in}}{\pgfqpoint{2.216660in}{2.216660in}}%
\pgfusepath{clip}%
\pgfsetbuttcap%
\pgfsetroundjoin%
\definecolor{currentfill}{rgb}{0.274128,0.199721,0.498911}%
\pgfsetfillcolor{currentfill}%
\pgfsetlinewidth{0.000000pt}%
\definecolor{currentstroke}{rgb}{0.000000,0.000000,0.000000}%
\pgfsetstrokecolor{currentstroke}%
\pgfsetdash{}{0pt}%
\pgfpathmoveto{\pgfqpoint{0.773373in}{1.096047in}}%
\pgfpathlineto{\pgfqpoint{0.770150in}{1.090862in}}%
\pgfpathlineto{\pgfqpoint{0.766928in}{1.085782in}}%
\pgfpathlineto{\pgfqpoint{0.763706in}{1.080813in}}%
\pgfpathlineto{\pgfqpoint{0.760485in}{1.075955in}}%
\pgfpathlineto{\pgfqpoint{0.767254in}{1.082620in}}%
\pgfpathlineto{\pgfqpoint{0.774415in}{1.089167in}}%
\pgfpathlineto{\pgfqpoint{0.781960in}{1.095591in}}%
\pgfpathlineto{\pgfqpoint{0.789881in}{1.101885in}}%
\pgfpathlineto{\pgfqpoint{0.792866in}{1.106526in}}%
\pgfpathlineto{\pgfqpoint{0.795851in}{1.111279in}}%
\pgfpathlineto{\pgfqpoint{0.798837in}{1.116142in}}%
\pgfpathlineto{\pgfqpoint{0.801823in}{1.121110in}}%
\pgfpathlineto{\pgfqpoint{0.794155in}{1.115026in}}%
\pgfpathlineto{\pgfqpoint{0.786852in}{1.108817in}}%
\pgfpathlineto{\pgfqpoint{0.779922in}{1.102488in}}%
\pgfpathlineto{\pgfqpoint{0.773373in}{1.096047in}}%
\pgfpathclose%
\pgfusepath{fill}%
\end{pgfscope}%
\begin{pgfscope}%
\pgfpathrectangle{\pgfqpoint{0.041670in}{0.041670in}}{\pgfqpoint{2.216660in}{2.216660in}}%
\pgfusepath{clip}%
\pgfsetbuttcap%
\pgfsetroundjoin%
\definecolor{currentfill}{rgb}{0.268510,0.009605,0.335427}%
\pgfsetfillcolor{currentfill}%
\pgfsetlinewidth{0.000000pt}%
\definecolor{currentstroke}{rgb}{0.000000,0.000000,0.000000}%
\pgfsetstrokecolor{currentstroke}%
\pgfsetdash{}{0pt}%
\pgfpathmoveto{\pgfqpoint{0.591949in}{0.935158in}}%
\pgfpathlineto{\pgfqpoint{0.588265in}{0.936814in}}%
\pgfpathlineto{\pgfqpoint{0.584574in}{0.938744in}}%
\pgfpathlineto{\pgfqpoint{0.580874in}{0.940953in}}%
\pgfpathlineto{\pgfqpoint{0.577165in}{0.943445in}}%
\pgfpathlineto{\pgfqpoint{0.581814in}{0.953243in}}%
\pgfpathlineto{\pgfqpoint{0.587039in}{0.962947in}}%
\pgfpathlineto{\pgfqpoint{0.592835in}{0.972548in}}%
\pgfpathlineto{\pgfqpoint{0.599192in}{0.982037in}}%
\pgfpathlineto{\pgfqpoint{0.602746in}{0.979322in}}%
\pgfpathlineto{\pgfqpoint{0.606292in}{0.976889in}}%
\pgfpathlineto{\pgfqpoint{0.609830in}{0.974734in}}%
\pgfpathlineto{\pgfqpoint{0.613360in}{0.972852in}}%
\pgfpathlineto{\pgfqpoint{0.607177in}{0.963582in}}%
\pgfpathlineto{\pgfqpoint{0.601542in}{0.954204in}}%
\pgfpathlineto{\pgfqpoint{0.596464in}{0.944727in}}%
\pgfpathlineto{\pgfqpoint{0.591949in}{0.935158in}}%
\pgfpathclose%
\pgfusepath{fill}%
\end{pgfscope}%
\begin{pgfscope}%
\pgfpathrectangle{\pgfqpoint{0.041670in}{0.041670in}}{\pgfqpoint{2.216660in}{2.216660in}}%
\pgfusepath{clip}%
\pgfsetbuttcap%
\pgfsetroundjoin%
\definecolor{currentfill}{rgb}{0.267004,0.004874,0.329415}%
\pgfsetfillcolor{currentfill}%
\pgfsetlinewidth{0.000000pt}%
\definecolor{currentstroke}{rgb}{0.000000,0.000000,0.000000}%
\pgfsetstrokecolor{currentstroke}%
\pgfsetdash{}{0pt}%
\pgfpathmoveto{\pgfqpoint{0.606608in}{0.931173in}}%
\pgfpathlineto{\pgfqpoint{0.602954in}{0.931783in}}%
\pgfpathlineto{\pgfqpoint{0.599293in}{0.932647in}}%
\pgfpathlineto{\pgfqpoint{0.595625in}{0.933771in}}%
\pgfpathlineto{\pgfqpoint{0.591949in}{0.935158in}}%
\pgfpathlineto{\pgfqpoint{0.596464in}{0.944727in}}%
\pgfpathlineto{\pgfqpoint{0.601542in}{0.954204in}}%
\pgfpathlineto{\pgfqpoint{0.607177in}{0.963582in}}%
\pgfpathlineto{\pgfqpoint{0.613360in}{0.972852in}}%
\pgfpathlineto{\pgfqpoint{0.616884in}{0.971238in}}%
\pgfpathlineto{\pgfqpoint{0.620400in}{0.969886in}}%
\pgfpathlineto{\pgfqpoint{0.623910in}{0.968793in}}%
\pgfpathlineto{\pgfqpoint{0.627413in}{0.967952in}}%
\pgfpathlineto{\pgfqpoint{0.621401in}{0.958907in}}%
\pgfpathlineto{\pgfqpoint{0.615925in}{0.949756in}}%
\pgfpathlineto{\pgfqpoint{0.610991in}{0.940508in}}%
\pgfpathlineto{\pgfqpoint{0.606608in}{0.931173in}}%
\pgfpathclose%
\pgfusepath{fill}%
\end{pgfscope}%
\begin{pgfscope}%
\pgfpathrectangle{\pgfqpoint{0.041670in}{0.041670in}}{\pgfqpoint{2.216660in}{2.216660in}}%
\pgfusepath{clip}%
\pgfsetbuttcap%
\pgfsetroundjoin%
\definecolor{currentfill}{rgb}{0.163625,0.471133,0.558148}%
\pgfsetfillcolor{currentfill}%
\pgfsetlinewidth{0.000000pt}%
\definecolor{currentstroke}{rgb}{0.000000,0.000000,0.000000}%
\pgfsetstrokecolor{currentstroke}%
\pgfsetdash{}{0pt}%
\pgfpathmoveto{\pgfqpoint{1.094685in}{1.363139in}}%
\pgfpathlineto{\pgfqpoint{1.093812in}{1.357380in}}%
\pgfpathlineto{\pgfqpoint{1.092940in}{1.351638in}}%
\pgfpathlineto{\pgfqpoint{1.092068in}{1.345914in}}%
\pgfpathlineto{\pgfqpoint{1.091198in}{1.340213in}}%
\pgfpathlineto{\pgfqpoint{1.102192in}{1.341497in}}%
\pgfpathlineto{\pgfqpoint{1.113258in}{1.342611in}}%
\pgfpathlineto{\pgfqpoint{1.124385in}{1.343553in}}%
\pgfpathlineto{\pgfqpoint{1.135563in}{1.344324in}}%
\pgfpathlineto{\pgfqpoint{1.135998in}{1.349981in}}%
\pgfpathlineto{\pgfqpoint{1.136434in}{1.355661in}}%
\pgfpathlineto{\pgfqpoint{1.136870in}{1.361360in}}%
\pgfpathlineto{\pgfqpoint{1.137306in}{1.367075in}}%
\pgfpathlineto{\pgfqpoint{1.126567in}{1.366337in}}%
\pgfpathlineto{\pgfqpoint{1.115877in}{1.365435in}}%
\pgfpathlineto{\pgfqpoint{1.105247in}{1.364369in}}%
\pgfpathlineto{\pgfqpoint{1.094685in}{1.363139in}}%
\pgfpathclose%
\pgfusepath{fill}%
\end{pgfscope}%
\begin{pgfscope}%
\pgfpathrectangle{\pgfqpoint{0.041670in}{0.041670in}}{\pgfqpoint{2.216660in}{2.216660in}}%
\pgfusepath{clip}%
\pgfsetbuttcap%
\pgfsetroundjoin%
\definecolor{currentfill}{rgb}{0.163625,0.471133,0.558148}%
\pgfsetfillcolor{currentfill}%
\pgfsetlinewidth{0.000000pt}%
\definecolor{currentstroke}{rgb}{0.000000,0.000000,0.000000}%
\pgfsetstrokecolor{currentstroke}%
\pgfsetdash{}{0pt}%
\pgfpathmoveto{\pgfqpoint{1.223799in}{1.367001in}}%
\pgfpathlineto{\pgfqpoint{1.224247in}{1.361285in}}%
\pgfpathlineto{\pgfqpoint{1.224696in}{1.355585in}}%
\pgfpathlineto{\pgfqpoint{1.225143in}{1.349905in}}%
\pgfpathlineto{\pgfqpoint{1.225591in}{1.344246in}}%
\pgfpathlineto{\pgfqpoint{1.236764in}{1.343457in}}%
\pgfpathlineto{\pgfqpoint{1.247885in}{1.342495in}}%
\pgfpathlineto{\pgfqpoint{1.258943in}{1.341363in}}%
\pgfpathlineto{\pgfqpoint{1.269928in}{1.340060in}}%
\pgfpathlineto{\pgfqpoint{1.269046in}{1.345763in}}%
\pgfpathlineto{\pgfqpoint{1.268163in}{1.351488in}}%
\pgfpathlineto{\pgfqpoint{1.267279in}{1.357232in}}%
\pgfpathlineto{\pgfqpoint{1.266393in}{1.362993in}}%
\pgfpathlineto{\pgfqpoint{1.255840in}{1.364240in}}%
\pgfpathlineto{\pgfqpoint{1.245217in}{1.365324in}}%
\pgfpathlineto{\pgfqpoint{1.234533in}{1.366245in}}%
\pgfpathlineto{\pgfqpoint{1.223799in}{1.367001in}}%
\pgfpathclose%
\pgfusepath{fill}%
\end{pgfscope}%
\begin{pgfscope}%
\pgfpathrectangle{\pgfqpoint{0.041670in}{0.041670in}}{\pgfqpoint{2.216660in}{2.216660in}}%
\pgfusepath{clip}%
\pgfsetbuttcap%
\pgfsetroundjoin%
\definecolor{currentfill}{rgb}{0.248629,0.278775,0.534556}%
\pgfsetfillcolor{currentfill}%
\pgfsetlinewidth{0.000000pt}%
\definecolor{currentstroke}{rgb}{0.000000,0.000000,0.000000}%
\pgfsetstrokecolor{currentstroke}%
\pgfsetdash{}{0pt}%
\pgfpathmoveto{\pgfqpoint{0.825753in}{1.164242in}}%
\pgfpathlineto{\pgfqpoint{0.822757in}{1.158553in}}%
\pgfpathlineto{\pgfqpoint{0.819763in}{1.152943in}}%
\pgfpathlineto{\pgfqpoint{0.816770in}{1.147413in}}%
\pgfpathlineto{\pgfqpoint{0.813778in}{1.141969in}}%
\pgfpathlineto{\pgfqpoint{0.821540in}{1.147718in}}%
\pgfpathlineto{\pgfqpoint{0.829640in}{1.153337in}}%
\pgfpathlineto{\pgfqpoint{0.838069in}{1.158820in}}%
\pgfpathlineto{\pgfqpoint{0.846819in}{1.164163in}}%
\pgfpathlineto{\pgfqpoint{0.849532in}{1.169409in}}%
\pgfpathlineto{\pgfqpoint{0.852246in}{1.174740in}}%
\pgfpathlineto{\pgfqpoint{0.854962in}{1.180153in}}%
\pgfpathlineto{\pgfqpoint{0.857680in}{1.185643in}}%
\pgfpathlineto{\pgfqpoint{0.849223in}{1.180491in}}%
\pgfpathlineto{\pgfqpoint{0.841077in}{1.175203in}}%
\pgfpathlineto{\pgfqpoint{0.833251in}{1.169785in}}%
\pgfpathlineto{\pgfqpoint{0.825753in}{1.164242in}}%
\pgfpathclose%
\pgfusepath{fill}%
\end{pgfscope}%
\begin{pgfscope}%
\pgfpathrectangle{\pgfqpoint{0.041670in}{0.041670in}}{\pgfqpoint{2.216660in}{2.216660in}}%
\pgfusepath{clip}%
\pgfsetbuttcap%
\pgfsetroundjoin%
\definecolor{currentfill}{rgb}{0.272594,0.025563,0.353093}%
\pgfsetfillcolor{currentfill}%
\pgfsetlinewidth{0.000000pt}%
\definecolor{currentstroke}{rgb}{0.000000,0.000000,0.000000}%
\pgfsetstrokecolor{currentstroke}%
\pgfsetdash{}{0pt}%
\pgfpathmoveto{\pgfqpoint{0.577165in}{0.943445in}}%
\pgfpathlineto{\pgfqpoint{0.573448in}{0.946227in}}%
\pgfpathlineto{\pgfqpoint{0.569721in}{0.949304in}}%
\pgfpathlineto{\pgfqpoint{0.565985in}{0.952680in}}%
\pgfpathlineto{\pgfqpoint{0.562239in}{0.956361in}}%
\pgfpathlineto{\pgfqpoint{0.567023in}{0.966382in}}%
\pgfpathlineto{\pgfqpoint{0.572398in}{0.976308in}}%
\pgfpathlineto{\pgfqpoint{0.578357in}{0.986127in}}%
\pgfpathlineto{\pgfqpoint{0.584890in}{0.995832in}}%
\pgfpathlineto{\pgfqpoint{0.588479in}{0.991933in}}%
\pgfpathlineto{\pgfqpoint{0.592059in}{0.988338in}}%
\pgfpathlineto{\pgfqpoint{0.595630in}{0.985041in}}%
\pgfpathlineto{\pgfqpoint{0.599192in}{0.982037in}}%
\pgfpathlineto{\pgfqpoint{0.592835in}{0.972548in}}%
\pgfpathlineto{\pgfqpoint{0.587039in}{0.962947in}}%
\pgfpathlineto{\pgfqpoint{0.581814in}{0.953243in}}%
\pgfpathlineto{\pgfqpoint{0.577165in}{0.943445in}}%
\pgfpathclose%
\pgfusepath{fill}%
\end{pgfscope}%
\begin{pgfscope}%
\pgfpathrectangle{\pgfqpoint{0.041670in}{0.041670in}}{\pgfqpoint{2.216660in}{2.216660in}}%
\pgfusepath{clip}%
\pgfsetbuttcap%
\pgfsetroundjoin%
\definecolor{currentfill}{rgb}{0.271305,0.019942,0.347269}%
\pgfsetfillcolor{currentfill}%
\pgfsetlinewidth{0.000000pt}%
\definecolor{currentstroke}{rgb}{0.000000,0.000000,0.000000}%
\pgfsetstrokecolor{currentstroke}%
\pgfsetdash{}{0pt}%
\pgfpathmoveto{\pgfqpoint{1.685559in}{0.983273in}}%
\pgfpathlineto{\pgfqpoint{1.688965in}{0.981484in}}%
\pgfpathlineto{\pgfqpoint{1.692374in}{0.979894in}}%
\pgfpathlineto{\pgfqpoint{1.695787in}{0.978508in}}%
\pgfpathlineto{\pgfqpoint{1.699204in}{0.977329in}}%
\pgfpathlineto{\pgfqpoint{1.705326in}{0.968836in}}%
\pgfpathlineto{\pgfqpoint{1.710946in}{0.960237in}}%
\pgfpathlineto{\pgfqpoint{1.716054in}{0.951540in}}%
\pgfpathlineto{\pgfqpoint{1.720645in}{0.942754in}}%
\pgfpathlineto{\pgfqpoint{1.717070in}{0.944168in}}%
\pgfpathlineto{\pgfqpoint{1.713499in}{0.945789in}}%
\pgfpathlineto{\pgfqpoint{1.709932in}{0.947615in}}%
\pgfpathlineto{\pgfqpoint{1.706370in}{0.949640in}}%
\pgfpathlineto{\pgfqpoint{1.701916in}{0.958186in}}%
\pgfpathlineto{\pgfqpoint{1.696959in}{0.966646in}}%
\pgfpathlineto{\pgfqpoint{1.691504in}{0.975011in}}%
\pgfpathlineto{\pgfqpoint{1.685559in}{0.983273in}}%
\pgfpathclose%
\pgfusepath{fill}%
\end{pgfscope}%
\begin{pgfscope}%
\pgfpathrectangle{\pgfqpoint{0.041670in}{0.041670in}}{\pgfqpoint{2.216660in}{2.216660in}}%
\pgfusepath{clip}%
\pgfsetbuttcap%
\pgfsetroundjoin%
\definecolor{currentfill}{rgb}{0.267004,0.004874,0.329415}%
\pgfsetfillcolor{currentfill}%
\pgfsetlinewidth{0.000000pt}%
\definecolor{currentstroke}{rgb}{0.000000,0.000000,0.000000}%
\pgfsetstrokecolor{currentstroke}%
\pgfsetdash{}{0pt}%
\pgfpathmoveto{\pgfqpoint{0.621161in}{0.931178in}}%
\pgfpathlineto{\pgfqpoint{0.617532in}{0.930819in}}%
\pgfpathlineto{\pgfqpoint{0.613897in}{0.930695in}}%
\pgfpathlineto{\pgfqpoint{0.610256in}{0.930812in}}%
\pgfpathlineto{\pgfqpoint{0.606608in}{0.931173in}}%
\pgfpathlineto{\pgfqpoint{0.610991in}{0.940508in}}%
\pgfpathlineto{\pgfqpoint{0.615925in}{0.949756in}}%
\pgfpathlineto{\pgfqpoint{0.621401in}{0.958907in}}%
\pgfpathlineto{\pgfqpoint{0.627413in}{0.967952in}}%
\pgfpathlineto{\pgfqpoint{0.630910in}{0.967360in}}%
\pgfpathlineto{\pgfqpoint{0.634402in}{0.967012in}}%
\pgfpathlineto{\pgfqpoint{0.637887in}{0.966903in}}%
\pgfpathlineto{\pgfqpoint{0.641367in}{0.967029in}}%
\pgfpathlineto{\pgfqpoint{0.635525in}{0.958211in}}%
\pgfpathlineto{\pgfqpoint{0.630205in}{0.949291in}}%
\pgfpathlineto{\pgfqpoint{0.625414in}{0.940277in}}%
\pgfpathlineto{\pgfqpoint{0.621161in}{0.931178in}}%
\pgfpathclose%
\pgfusepath{fill}%
\end{pgfscope}%
\begin{pgfscope}%
\pgfpathrectangle{\pgfqpoint{0.041670in}{0.041670in}}{\pgfqpoint{2.216660in}{2.216660in}}%
\pgfusepath{clip}%
\pgfsetbuttcap%
\pgfsetroundjoin%
\definecolor{currentfill}{rgb}{0.195860,0.395433,0.555276}%
\pgfsetfillcolor{currentfill}%
\pgfsetlinewidth{0.000000pt}%
\definecolor{currentstroke}{rgb}{0.000000,0.000000,0.000000}%
\pgfsetstrokecolor{currentstroke}%
\pgfsetdash{}{0pt}%
\pgfpathmoveto{\pgfqpoint{1.391083in}{1.291311in}}%
\pgfpathlineto{\pgfqpoint{1.393072in}{1.285494in}}%
\pgfpathlineto{\pgfqpoint{1.395059in}{1.279718in}}%
\pgfpathlineto{\pgfqpoint{1.397045in}{1.273984in}}%
\pgfpathlineto{\pgfqpoint{1.399028in}{1.268297in}}%
\pgfpathlineto{\pgfqpoint{1.409109in}{1.264803in}}%
\pgfpathlineto{\pgfqpoint{1.418977in}{1.261151in}}%
\pgfpathlineto{\pgfqpoint{1.428623in}{1.257345in}}%
\pgfpathlineto{\pgfqpoint{1.438037in}{1.253389in}}%
\pgfpathlineto{\pgfqpoint{1.435695in}{1.259224in}}%
\pgfpathlineto{\pgfqpoint{1.433352in}{1.265105in}}%
\pgfpathlineto{\pgfqpoint{1.431006in}{1.271029in}}%
\pgfpathlineto{\pgfqpoint{1.428658in}{1.276993in}}%
\pgfpathlineto{\pgfqpoint{1.419591in}{1.280793in}}%
\pgfpathlineto{\pgfqpoint{1.410300in}{1.284448in}}%
\pgfpathlineto{\pgfqpoint{1.400795in}{1.287955in}}%
\pgfpathlineto{\pgfqpoint{1.391083in}{1.291311in}}%
\pgfpathclose%
\pgfusepath{fill}%
\end{pgfscope}%
\begin{pgfscope}%
\pgfpathrectangle{\pgfqpoint{0.041670in}{0.041670in}}{\pgfqpoint{2.216660in}{2.216660in}}%
\pgfusepath{clip}%
\pgfsetbuttcap%
\pgfsetroundjoin%
\definecolor{currentfill}{rgb}{0.233603,0.313828,0.543914}%
\pgfsetfillcolor{currentfill}%
\pgfsetlinewidth{0.000000pt}%
\definecolor{currentstroke}{rgb}{0.000000,0.000000,0.000000}%
\pgfsetstrokecolor{currentstroke}%
\pgfsetdash{}{0pt}%
\pgfpathmoveto{\pgfqpoint{0.510650in}{1.146370in}}%
\pgfpathlineto{\pgfqpoint{0.506787in}{1.157999in}}%
\pgfpathlineto{\pgfqpoint{0.502905in}{1.170068in}}%
\pgfpathlineto{\pgfqpoint{0.499006in}{1.182584in}}%
\pgfpathlineto{\pgfqpoint{0.495088in}{1.195554in}}%
\pgfpathlineto{\pgfqpoint{0.503398in}{1.206269in}}%
\pgfpathlineto{\pgfqpoint{0.512343in}{1.216831in}}%
\pgfpathlineto{\pgfqpoint{0.521913in}{1.227233in}}%
\pgfpathlineto{\pgfqpoint{0.532096in}{1.237463in}}%
\pgfpathlineto{\pgfqpoint{0.535779in}{1.224333in}}%
\pgfpathlineto{\pgfqpoint{0.539446in}{1.211655in}}%
\pgfpathlineto{\pgfqpoint{0.543096in}{1.199421in}}%
\pgfpathlineto{\pgfqpoint{0.546730in}{1.187624in}}%
\pgfpathlineto{\pgfqpoint{0.536798in}{1.177552in}}%
\pgfpathlineto{\pgfqpoint{0.527467in}{1.167313in}}%
\pgfpathlineto{\pgfqpoint{0.518747in}{1.156916in}}%
\pgfpathlineto{\pgfqpoint{0.510650in}{1.146370in}}%
\pgfpathclose%
\pgfusepath{fill}%
\end{pgfscope}%
\begin{pgfscope}%
\pgfpathrectangle{\pgfqpoint{0.041670in}{0.041670in}}{\pgfqpoint{2.216660in}{2.216660in}}%
\pgfusepath{clip}%
\pgfsetbuttcap%
\pgfsetroundjoin%
\definecolor{currentfill}{rgb}{0.277941,0.056324,0.381191}%
\pgfsetfillcolor{currentfill}%
\pgfsetlinewidth{0.000000pt}%
\definecolor{currentstroke}{rgb}{0.000000,0.000000,0.000000}%
\pgfsetstrokecolor{currentstroke}%
\pgfsetdash{}{0pt}%
\pgfpathmoveto{\pgfqpoint{0.562239in}{0.956361in}}%
\pgfpathlineto{\pgfqpoint{0.558482in}{0.960352in}}%
\pgfpathlineto{\pgfqpoint{0.554715in}{0.964659in}}%
\pgfpathlineto{\pgfqpoint{0.550937in}{0.969288in}}%
\pgfpathlineto{\pgfqpoint{0.547148in}{0.974243in}}%
\pgfpathlineto{\pgfqpoint{0.552071in}{0.984484in}}%
\pgfpathlineto{\pgfqpoint{0.557599in}{0.994625in}}%
\pgfpathlineto{\pgfqpoint{0.563723in}{1.004658in}}%
\pgfpathlineto{\pgfqpoint{0.570435in}{1.014572in}}%
\pgfpathlineto{\pgfqpoint{0.574065in}{1.009404in}}%
\pgfpathlineto{\pgfqpoint{0.577683in}{1.004562in}}%
\pgfpathlineto{\pgfqpoint{0.581292in}{1.000039in}}%
\pgfpathlineto{\pgfqpoint{0.584890in}{0.995832in}}%
\pgfpathlineto{\pgfqpoint{0.578357in}{0.986127in}}%
\pgfpathlineto{\pgfqpoint{0.572398in}{0.976308in}}%
\pgfpathlineto{\pgfqpoint{0.567023in}{0.966382in}}%
\pgfpathlineto{\pgfqpoint{0.562239in}{0.956361in}}%
\pgfpathclose%
\pgfusepath{fill}%
\end{pgfscope}%
\begin{pgfscope}%
\pgfpathrectangle{\pgfqpoint{0.041670in}{0.041670in}}{\pgfqpoint{2.216660in}{2.216660in}}%
\pgfusepath{clip}%
\pgfsetbuttcap%
\pgfsetroundjoin%
\definecolor{currentfill}{rgb}{0.212395,0.359683,0.551710}%
\pgfsetfillcolor{currentfill}%
\pgfsetlinewidth{0.000000pt}%
\definecolor{currentstroke}{rgb}{0.000000,0.000000,0.000000}%
\pgfsetstrokecolor{currentstroke}%
\pgfsetdash{}{0pt}%
\pgfpathmoveto{\pgfqpoint{1.438037in}{1.253389in}}%
\pgfpathlineto{\pgfqpoint{1.440376in}{1.247603in}}%
\pgfpathlineto{\pgfqpoint{1.442714in}{1.241869in}}%
\pgfpathlineto{\pgfqpoint{1.445050in}{1.236191in}}%
\pgfpathlineto{\pgfqpoint{1.447383in}{1.230572in}}%
\pgfpathlineto{\pgfqpoint{1.456895in}{1.226305in}}%
\pgfpathlineto{\pgfqpoint{1.466148in}{1.221888in}}%
\pgfpathlineto{\pgfqpoint{1.475133in}{1.217325in}}%
\pgfpathlineto{\pgfqpoint{1.483842in}{1.212620in}}%
\pgfpathlineto{\pgfqpoint{1.481183in}{1.218411in}}%
\pgfpathlineto{\pgfqpoint{1.478522in}{1.224260in}}%
\pgfpathlineto{\pgfqpoint{1.475859in}{1.230165in}}%
\pgfpathlineto{\pgfqpoint{1.473194in}{1.236123in}}%
\pgfpathlineto{\pgfqpoint{1.464797in}{1.240648in}}%
\pgfpathlineto{\pgfqpoint{1.456133in}{1.245036in}}%
\pgfpathlineto{\pgfqpoint{1.447210in}{1.249284in}}%
\pgfpathlineto{\pgfqpoint{1.438037in}{1.253389in}}%
\pgfpathclose%
\pgfusepath{fill}%
\end{pgfscope}%
\begin{pgfscope}%
\pgfpathrectangle{\pgfqpoint{0.041670in}{0.041670in}}{\pgfqpoint{2.216660in}{2.216660in}}%
\pgfusepath{clip}%
\pgfsetbuttcap%
\pgfsetroundjoin%
\definecolor{currentfill}{rgb}{0.268510,0.009605,0.335427}%
\pgfsetfillcolor{currentfill}%
\pgfsetlinewidth{0.000000pt}%
\definecolor{currentstroke}{rgb}{0.000000,0.000000,0.000000}%
\pgfsetstrokecolor{currentstroke}%
\pgfsetdash{}{0pt}%
\pgfpathmoveto{\pgfqpoint{0.635623in}{0.934875in}}%
\pgfpathlineto{\pgfqpoint{0.632015in}{0.933621in}}%
\pgfpathlineto{\pgfqpoint{0.628402in}{0.932583in}}%
\pgfpathlineto{\pgfqpoint{0.624784in}{0.931768in}}%
\pgfpathlineto{\pgfqpoint{0.621161in}{0.931178in}}%
\pgfpathlineto{\pgfqpoint{0.625414in}{0.940277in}}%
\pgfpathlineto{\pgfqpoint{0.630205in}{0.949291in}}%
\pgfpathlineto{\pgfqpoint{0.635525in}{0.958211in}}%
\pgfpathlineto{\pgfqpoint{0.641367in}{0.967029in}}%
\pgfpathlineto{\pgfqpoint{0.644843in}{0.967384in}}%
\pgfpathlineto{\pgfqpoint{0.648313in}{0.967965in}}%
\pgfpathlineto{\pgfqpoint{0.651778in}{0.968767in}}%
\pgfpathlineto{\pgfqpoint{0.655239in}{0.969785in}}%
\pgfpathlineto{\pgfqpoint{0.649563in}{0.961197in}}%
\pgfpathlineto{\pgfqpoint{0.644398in}{0.952511in}}%
\pgfpathlineto{\pgfqpoint{0.639749in}{0.943735in}}%
\pgfpathlineto{\pgfqpoint{0.635623in}{0.934875in}}%
\pgfpathclose%
\pgfusepath{fill}%
\end{pgfscope}%
\begin{pgfscope}%
\pgfpathrectangle{\pgfqpoint{0.041670in}{0.041670in}}{\pgfqpoint{2.216660in}{2.216660in}}%
\pgfusepath{clip}%
\pgfsetbuttcap%
\pgfsetroundjoin%
\definecolor{currentfill}{rgb}{0.163625,0.471133,0.558148}%
\pgfsetfillcolor{currentfill}%
\pgfsetlinewidth{0.000000pt}%
\definecolor{currentstroke}{rgb}{0.000000,0.000000,0.000000}%
\pgfsetstrokecolor{currentstroke}%
\pgfsetdash{}{0pt}%
\pgfpathmoveto{\pgfqpoint{1.266393in}{1.362993in}}%
\pgfpathlineto{\pgfqpoint{1.267279in}{1.357232in}}%
\pgfpathlineto{\pgfqpoint{1.268163in}{1.351488in}}%
\pgfpathlineto{\pgfqpoint{1.269046in}{1.345763in}}%
\pgfpathlineto{\pgfqpoint{1.269928in}{1.340060in}}%
\pgfpathlineto{\pgfqpoint{1.280831in}{1.338588in}}%
\pgfpathlineto{\pgfqpoint{1.291641in}{1.336948in}}%
\pgfpathlineto{\pgfqpoint{1.302347in}{1.335142in}}%
\pgfpathlineto{\pgfqpoint{1.301146in}{1.340897in}}%
\pgfpathlineto{\pgfqpoint{1.299944in}{1.346675in}}%
\pgfpathlineto{\pgfqpoint{1.298740in}{1.352471in}}%
\pgfpathlineto{\pgfqpoint{1.297535in}{1.358284in}}%
\pgfpathlineto{\pgfqpoint{1.287250in}{1.360013in}}%
\pgfpathlineto{\pgfqpoint{1.276867in}{1.361583in}}%
\pgfpathlineto{\pgfqpoint{1.266393in}{1.362993in}}%
\pgfpathclose%
\pgfusepath{fill}%
\end{pgfscope}%
\begin{pgfscope}%
\pgfpathrectangle{\pgfqpoint{0.041670in}{0.041670in}}{\pgfqpoint{2.216660in}{2.216660in}}%
\pgfusepath{clip}%
\pgfsetbuttcap%
\pgfsetroundjoin%
\definecolor{currentfill}{rgb}{0.179019,0.433756,0.557430}%
\pgfsetfillcolor{currentfill}%
\pgfsetlinewidth{0.000000pt}%
\definecolor{currentstroke}{rgb}{0.000000,0.000000,0.000000}%
\pgfsetstrokecolor{currentstroke}%
\pgfsetdash{}{0pt}%
\pgfpathmoveto{\pgfqpoint{1.343943in}{1.326286in}}%
\pgfpathlineto{\pgfqpoint{1.345552in}{1.320462in}}%
\pgfpathlineto{\pgfqpoint{1.347160in}{1.314665in}}%
\pgfpathlineto{\pgfqpoint{1.348766in}{1.308899in}}%
\pgfpathlineto{\pgfqpoint{1.350370in}{1.303168in}}%
\pgfpathlineto{\pgfqpoint{1.360810in}{1.300444in}}%
\pgfpathlineto{\pgfqpoint{1.371081in}{1.297558in}}%
\pgfpathlineto{\pgfqpoint{1.381176in}{1.294513in}}%
\pgfpathlineto{\pgfqpoint{1.391083in}{1.291311in}}%
\pgfpathlineto{\pgfqpoint{1.389092in}{1.297164in}}%
\pgfpathlineto{\pgfqpoint{1.387099in}{1.303051in}}%
\pgfpathlineto{\pgfqpoint{1.385104in}{1.308970in}}%
\pgfpathlineto{\pgfqpoint{1.383107in}{1.314916in}}%
\pgfpathlineto{\pgfqpoint{1.373577in}{1.317986in}}%
\pgfpathlineto{\pgfqpoint{1.363867in}{1.320906in}}%
\pgfpathlineto{\pgfqpoint{1.353986in}{1.323674in}}%
\pgfpathlineto{\pgfqpoint{1.343943in}{1.326286in}}%
\pgfpathclose%
\pgfusepath{fill}%
\end{pgfscope}%
\begin{pgfscope}%
\pgfpathrectangle{\pgfqpoint{0.041670in}{0.041670in}}{\pgfqpoint{2.216660in}{2.216660in}}%
\pgfusepath{clip}%
\pgfsetbuttcap%
\pgfsetroundjoin%
\definecolor{currentfill}{rgb}{0.163625,0.471133,0.558148}%
\pgfsetfillcolor{currentfill}%
\pgfsetlinewidth{0.000000pt}%
\definecolor{currentstroke}{rgb}{0.000000,0.000000,0.000000}%
\pgfsetstrokecolor{currentstroke}%
\pgfsetdash{}{0pt}%
\pgfpathmoveto{\pgfqpoint{1.053325in}{1.356615in}}%
\pgfpathlineto{\pgfqpoint{1.052026in}{1.350783in}}%
\pgfpathlineto{\pgfqpoint{1.050730in}{1.344968in}}%
\pgfpathlineto{\pgfqpoint{1.049434in}{1.339172in}}%
\pgfpathlineto{\pgfqpoint{1.048141in}{1.333398in}}%
\pgfpathlineto{\pgfqpoint{1.058747in}{1.335351in}}%
\pgfpathlineto{\pgfqpoint{1.069465in}{1.337139in}}%
\pgfpathlineto{\pgfqpoint{1.080286in}{1.338760in}}%
\pgfpathlineto{\pgfqpoint{1.091198in}{1.340213in}}%
\pgfpathlineto{\pgfqpoint{1.092068in}{1.345914in}}%
\pgfpathlineto{\pgfqpoint{1.092940in}{1.351638in}}%
\pgfpathlineto{\pgfqpoint{1.093812in}{1.357380in}}%
\pgfpathlineto{\pgfqpoint{1.094685in}{1.363139in}}%
\pgfpathlineto{\pgfqpoint{1.084203in}{1.361748in}}%
\pgfpathlineto{\pgfqpoint{1.073808in}{1.360196in}}%
\pgfpathlineto{\pgfqpoint{1.063513in}{1.358484in}}%
\pgfpathlineto{\pgfqpoint{1.053325in}{1.356615in}}%
\pgfpathclose%
\pgfusepath{fill}%
\end{pgfscope}%
\begin{pgfscope}%
\pgfpathrectangle{\pgfqpoint{0.041670in}{0.041670in}}{\pgfqpoint{2.216660in}{2.216660in}}%
\pgfusepath{clip}%
\pgfsetbuttcap%
\pgfsetroundjoin%
\definecolor{currentfill}{rgb}{0.282884,0.135920,0.453427}%
\pgfsetfillcolor{currentfill}%
\pgfsetlinewidth{0.000000pt}%
\definecolor{currentstroke}{rgb}{0.000000,0.000000,0.000000}%
\pgfsetstrokecolor{currentstroke}%
\pgfsetdash{}{0pt}%
\pgfpathmoveto{\pgfqpoint{1.797476in}{1.047498in}}%
\pgfpathlineto{\pgfqpoint{1.801119in}{1.054424in}}%
\pgfpathlineto{\pgfqpoint{1.804774in}{1.061709in}}%
\pgfpathlineto{\pgfqpoint{1.808442in}{1.069361in}}%
\pgfpathlineto{\pgfqpoint{1.812124in}{1.077385in}}%
\pgfpathlineto{\pgfqpoint{1.819743in}{1.067188in}}%
\pgfpathlineto{\pgfqpoint{1.826756in}{1.056858in}}%
\pgfpathlineto{\pgfqpoint{1.833153in}{1.046406in}}%
\pgfpathlineto{\pgfqpoint{1.838924in}{1.035839in}}%
\pgfpathlineto{\pgfqpoint{1.835064in}{1.028011in}}%
\pgfpathlineto{\pgfqpoint{1.831218in}{1.020558in}}%
\pgfpathlineto{\pgfqpoint{1.827385in}{1.013472in}}%
\pgfpathlineto{\pgfqpoint{1.823567in}{1.006748in}}%
\pgfpathlineto{\pgfqpoint{1.817952in}{1.017111in}}%
\pgfpathlineto{\pgfqpoint{1.811727in}{1.027363in}}%
\pgfpathlineto{\pgfqpoint{1.804898in}{1.037495in}}%
\pgfpathlineto{\pgfqpoint{1.797476in}{1.047498in}}%
\pgfpathclose%
\pgfusepath{fill}%
\end{pgfscope}%
\begin{pgfscope}%
\pgfpathrectangle{\pgfqpoint{0.041670in}{0.041670in}}{\pgfqpoint{2.216660in}{2.216660in}}%
\pgfusepath{clip}%
\pgfsetbuttcap%
\pgfsetroundjoin%
\definecolor{currentfill}{rgb}{0.283072,0.130895,0.449241}%
\pgfsetfillcolor{currentfill}%
\pgfsetlinewidth{0.000000pt}%
\definecolor{currentstroke}{rgb}{0.000000,0.000000,0.000000}%
\pgfsetstrokecolor{currentstroke}%
\pgfsetdash{}{0pt}%
\pgfpathmoveto{\pgfqpoint{0.723829in}{1.029085in}}%
\pgfpathlineto{\pgfqpoint{0.720416in}{1.024615in}}%
\pgfpathlineto{\pgfqpoint{0.717003in}{1.020280in}}%
\pgfpathlineto{\pgfqpoint{0.713588in}{1.016083in}}%
\pgfpathlineto{\pgfqpoint{0.710173in}{1.012030in}}%
\pgfpathlineto{\pgfqpoint{0.715646in}{1.019579in}}%
\pgfpathlineto{\pgfqpoint{0.721565in}{1.027027in}}%
\pgfpathlineto{\pgfqpoint{0.727921in}{1.034369in}}%
\pgfpathlineto{\pgfqpoint{0.734709in}{1.041598in}}%
\pgfpathlineto{\pgfqpoint{0.737933in}{1.045421in}}%
\pgfpathlineto{\pgfqpoint{0.741156in}{1.049386in}}%
\pgfpathlineto{\pgfqpoint{0.744378in}{1.053489in}}%
\pgfpathlineto{\pgfqpoint{0.747600in}{1.057728in}}%
\pgfpathlineto{\pgfqpoint{0.741022in}{1.050725in}}%
\pgfpathlineto{\pgfqpoint{0.734863in}{1.043612in}}%
\pgfpathlineto{\pgfqpoint{0.729129in}{1.036397in}}%
\pgfpathlineto{\pgfqpoint{0.723829in}{1.029085in}}%
\pgfpathclose%
\pgfusepath{fill}%
\end{pgfscope}%
\begin{pgfscope}%
\pgfpathrectangle{\pgfqpoint{0.041670in}{0.041670in}}{\pgfqpoint{2.216660in}{2.216660in}}%
\pgfusepath{clip}%
\pgfsetbuttcap%
\pgfsetroundjoin%
\definecolor{currentfill}{rgb}{0.263663,0.237631,0.518762}%
\pgfsetfillcolor{currentfill}%
\pgfsetlinewidth{0.000000pt}%
\definecolor{currentstroke}{rgb}{0.000000,0.000000,0.000000}%
\pgfsetstrokecolor{currentstroke}%
\pgfsetdash{}{0pt}%
\pgfpathmoveto{\pgfqpoint{1.539249in}{1.147086in}}%
\pgfpathlineto{\pgfqpoint{1.542181in}{1.141776in}}%
\pgfpathlineto{\pgfqpoint{1.545112in}{1.136558in}}%
\pgfpathlineto{\pgfqpoint{1.548042in}{1.131434in}}%
\pgfpathlineto{\pgfqpoint{1.550970in}{1.126410in}}%
\pgfpathlineto{\pgfqpoint{1.558956in}{1.120441in}}%
\pgfpathlineto{\pgfqpoint{1.566584in}{1.114342in}}%
\pgfpathlineto{\pgfqpoint{1.573846in}{1.108119in}}%
\pgfpathlineto{\pgfqpoint{1.580734in}{1.101778in}}%
\pgfpathlineto{\pgfqpoint{1.577559in}{1.107015in}}%
\pgfpathlineto{\pgfqpoint{1.574384in}{1.112352in}}%
\pgfpathlineto{\pgfqpoint{1.571207in}{1.117783in}}%
\pgfpathlineto{\pgfqpoint{1.568029in}{1.123306in}}%
\pgfpathlineto{\pgfqpoint{1.561370in}{1.129427in}}%
\pgfpathlineto{\pgfqpoint{1.554349in}{1.135435in}}%
\pgfpathlineto{\pgfqpoint{1.546973in}{1.141323in}}%
\pgfpathlineto{\pgfqpoint{1.539249in}{1.147086in}}%
\pgfpathclose%
\pgfusepath{fill}%
\end{pgfscope}%
\begin{pgfscope}%
\pgfpathrectangle{\pgfqpoint{0.041670in}{0.041670in}}{\pgfqpoint{2.216660in}{2.216660in}}%
\pgfusepath{clip}%
\pgfsetbuttcap%
\pgfsetroundjoin%
\definecolor{currentfill}{rgb}{0.274952,0.037752,0.364543}%
\pgfsetfillcolor{currentfill}%
\pgfsetlinewidth{0.000000pt}%
\definecolor{currentstroke}{rgb}{0.000000,0.000000,0.000000}%
\pgfsetstrokecolor{currentstroke}%
\pgfsetdash{}{0pt}%
\pgfpathmoveto{\pgfqpoint{1.671967in}{0.992333in}}%
\pgfpathlineto{\pgfqpoint{1.675361in}{0.989791in}}%
\pgfpathlineto{\pgfqpoint{1.678757in}{0.987431in}}%
\pgfpathlineto{\pgfqpoint{1.682156in}{0.985257in}}%
\pgfpathlineto{\pgfqpoint{1.685559in}{0.983273in}}%
\pgfpathlineto{\pgfqpoint{1.691504in}{0.975011in}}%
\pgfpathlineto{\pgfqpoint{1.696959in}{0.966646in}}%
\pgfpathlineto{\pgfqpoint{1.701916in}{0.958186in}}%
\pgfpathlineto{\pgfqpoint{1.706370in}{0.949640in}}%
\pgfpathlineto{\pgfqpoint{1.702810in}{0.951860in}}%
\pgfpathlineto{\pgfqpoint{1.699255in}{0.954271in}}%
\pgfpathlineto{\pgfqpoint{1.695702in}{0.956869in}}%
\pgfpathlineto{\pgfqpoint{1.692153in}{0.959650in}}%
\pgfpathlineto{\pgfqpoint{1.687837in}{0.967954in}}%
\pgfpathlineto{\pgfqpoint{1.683029in}{0.976174in}}%
\pgfpathlineto{\pgfqpoint{1.677737in}{0.984303in}}%
\pgfpathlineto{\pgfqpoint{1.671967in}{0.992333in}}%
\pgfpathclose%
\pgfusepath{fill}%
\end{pgfscope}%
\begin{pgfscope}%
\pgfpathrectangle{\pgfqpoint{0.041670in}{0.041670in}}{\pgfqpoint{2.216660in}{2.216660in}}%
\pgfusepath{clip}%
\pgfsetbuttcap%
\pgfsetroundjoin%
\definecolor{currentfill}{rgb}{0.195860,0.395433,0.555276}%
\pgfsetfillcolor{currentfill}%
\pgfsetlinewidth{0.000000pt}%
\definecolor{currentstroke}{rgb}{0.000000,0.000000,0.000000}%
\pgfsetstrokecolor{currentstroke}%
\pgfsetdash{}{0pt}%
\pgfpathmoveto{\pgfqpoint{0.923388in}{1.273496in}}%
\pgfpathlineto{\pgfqpoint{0.920964in}{1.267496in}}%
\pgfpathlineto{\pgfqpoint{0.918543in}{1.261536in}}%
\pgfpathlineto{\pgfqpoint{0.916124in}{1.255619in}}%
\pgfpathlineto{\pgfqpoint{0.913707in}{1.249748in}}%
\pgfpathlineto{\pgfqpoint{0.922907in}{1.253836in}}%
\pgfpathlineto{\pgfqpoint{0.932347in}{1.257776in}}%
\pgfpathlineto{\pgfqpoint{0.942018in}{1.261564in}}%
\pgfpathlineto{\pgfqpoint{0.951911in}{1.265199in}}%
\pgfpathlineto{\pgfqpoint{0.953976in}{1.270917in}}%
\pgfpathlineto{\pgfqpoint{0.956043in}{1.276681in}}%
\pgfpathlineto{\pgfqpoint{0.958113in}{1.282488in}}%
\pgfpathlineto{\pgfqpoint{0.960184in}{1.288335in}}%
\pgfpathlineto{\pgfqpoint{0.950655in}{1.284845in}}%
\pgfpathlineto{\pgfqpoint{0.941340in}{1.281206in}}%
\pgfpathlineto{\pgfqpoint{0.932248in}{1.277422in}}%
\pgfpathlineto{\pgfqpoint{0.923388in}{1.273496in}}%
\pgfpathclose%
\pgfusepath{fill}%
\end{pgfscope}%
\begin{pgfscope}%
\pgfpathrectangle{\pgfqpoint{0.041670in}{0.041670in}}{\pgfqpoint{2.216660in}{2.216660in}}%
\pgfusepath{clip}%
\pgfsetbuttcap%
\pgfsetroundjoin%
\definecolor{currentfill}{rgb}{0.212395,0.359683,0.551710}%
\pgfsetfillcolor{currentfill}%
\pgfsetlinewidth{0.000000pt}%
\definecolor{currentstroke}{rgb}{0.000000,0.000000,0.000000}%
\pgfsetstrokecolor{currentstroke}%
\pgfsetdash{}{0pt}%
\pgfpathmoveto{\pgfqpoint{0.879483in}{1.231989in}}%
\pgfpathlineto{\pgfqpoint{0.876750in}{1.225991in}}%
\pgfpathlineto{\pgfqpoint{0.874020in}{1.220044in}}%
\pgfpathlineto{\pgfqpoint{0.871292in}{1.214154in}}%
\pgfpathlineto{\pgfqpoint{0.868566in}{1.208321in}}%
\pgfpathlineto{\pgfqpoint{0.877021in}{1.213149in}}%
\pgfpathlineto{\pgfqpoint{0.885762in}{1.217839in}}%
\pgfpathlineto{\pgfqpoint{0.894777in}{1.222386in}}%
\pgfpathlineto{\pgfqpoint{0.904059in}{1.226786in}}%
\pgfpathlineto{\pgfqpoint{0.906468in}{1.232442in}}%
\pgfpathlineto{\pgfqpoint{0.908879in}{1.238156in}}%
\pgfpathlineto{\pgfqpoint{0.911292in}{1.243926in}}%
\pgfpathlineto{\pgfqpoint{0.913707in}{1.249748in}}%
\pgfpathlineto{\pgfqpoint{0.904755in}{1.245515in}}%
\pgfpathlineto{\pgfqpoint{0.896062in}{1.241142in}}%
\pgfpathlineto{\pgfqpoint{0.887635in}{1.236632in}}%
\pgfpathlineto{\pgfqpoint{0.879483in}{1.231989in}}%
\pgfpathclose%
\pgfusepath{fill}%
\end{pgfscope}%
\begin{pgfscope}%
\pgfpathrectangle{\pgfqpoint{0.041670in}{0.041670in}}{\pgfqpoint{2.216660in}{2.216660in}}%
\pgfusepath{clip}%
\pgfsetbuttcap%
\pgfsetroundjoin%
\definecolor{currentfill}{rgb}{0.282327,0.094955,0.417331}%
\pgfsetfillcolor{currentfill}%
\pgfsetlinewidth{0.000000pt}%
\definecolor{currentstroke}{rgb}{0.000000,0.000000,0.000000}%
\pgfsetstrokecolor{currentstroke}%
\pgfsetdash{}{0pt}%
\pgfpathmoveto{\pgfqpoint{0.547148in}{0.974243in}}%
\pgfpathlineto{\pgfqpoint{0.543348in}{0.979532in}}%
\pgfpathlineto{\pgfqpoint{0.539535in}{0.985159in}}%
\pgfpathlineto{\pgfqpoint{0.535710in}{0.991130in}}%
\pgfpathlineto{\pgfqpoint{0.531873in}{0.997451in}}%
\pgfpathlineto{\pgfqpoint{0.536937in}{1.007905in}}%
\pgfpathlineto{\pgfqpoint{0.542619in}{1.018256in}}%
\pgfpathlineto{\pgfqpoint{0.548912in}{1.028495in}}%
\pgfpathlineto{\pgfqpoint{0.555807in}{1.038613in}}%
\pgfpathlineto{\pgfqpoint{0.559482in}{1.032086in}}%
\pgfpathlineto{\pgfqpoint{0.563144in}{1.025907in}}%
\pgfpathlineto{\pgfqpoint{0.566796in}{1.020071in}}%
\pgfpathlineto{\pgfqpoint{0.570435in}{1.014572in}}%
\pgfpathlineto{\pgfqpoint{0.563723in}{1.004658in}}%
\pgfpathlineto{\pgfqpoint{0.557599in}{0.994625in}}%
\pgfpathlineto{\pgfqpoint{0.552071in}{0.984484in}}%
\pgfpathlineto{\pgfqpoint{0.547148in}{0.974243in}}%
\pgfpathclose%
\pgfusepath{fill}%
\end{pgfscope}%
\begin{pgfscope}%
\pgfpathrectangle{\pgfqpoint{0.041670in}{0.041670in}}{\pgfqpoint{2.216660in}{2.216660in}}%
\pgfusepath{clip}%
\pgfsetbuttcap%
\pgfsetroundjoin%
\definecolor{currentfill}{rgb}{0.201239,0.383670,0.554294}%
\pgfsetfillcolor{currentfill}%
\pgfsetlinewidth{0.000000pt}%
\definecolor{currentstroke}{rgb}{0.000000,0.000000,0.000000}%
\pgfsetstrokecolor{currentstroke}%
\pgfsetdash{}{0pt}%
\pgfpathmoveto{\pgfqpoint{1.818258in}{1.246407in}}%
\pgfpathlineto{\pgfqpoint{1.821899in}{1.260030in}}%
\pgfpathlineto{\pgfqpoint{1.825559in}{1.274118in}}%
\pgfpathlineto{\pgfqpoint{1.829236in}{1.288681in}}%
\pgfpathlineto{\pgfqpoint{1.832932in}{1.303725in}}%
\pgfpathlineto{\pgfqpoint{1.843919in}{1.293507in}}%
\pgfpathlineto{\pgfqpoint{1.854290in}{1.283108in}}%
\pgfpathlineto{\pgfqpoint{1.864034in}{1.272536in}}%
\pgfpathlineto{\pgfqpoint{1.873139in}{1.261801in}}%
\pgfpathlineto{\pgfqpoint{1.869191in}{1.246904in}}%
\pgfpathlineto{\pgfqpoint{1.865264in}{1.232490in}}%
\pgfpathlineto{\pgfqpoint{1.861356in}{1.218554in}}%
\pgfpathlineto{\pgfqpoint{1.857467in}{1.205086in}}%
\pgfpathlineto{\pgfqpoint{1.848592in}{1.215666in}}%
\pgfpathlineto{\pgfqpoint{1.839091in}{1.226085in}}%
\pgfpathlineto{\pgfqpoint{1.828975in}{1.236335in}}%
\pgfpathlineto{\pgfqpoint{1.818258in}{1.246407in}}%
\pgfpathclose%
\pgfusepath{fill}%
\end{pgfscope}%
\begin{pgfscope}%
\pgfpathrectangle{\pgfqpoint{0.041670in}{0.041670in}}{\pgfqpoint{2.216660in}{2.216660in}}%
\pgfusepath{clip}%
\pgfsetbuttcap%
\pgfsetroundjoin%
\definecolor{currentfill}{rgb}{0.179019,0.433756,0.557430}%
\pgfsetfillcolor{currentfill}%
\pgfsetlinewidth{0.000000pt}%
\definecolor{currentstroke}{rgb}{0.000000,0.000000,0.000000}%
\pgfsetstrokecolor{currentstroke}%
\pgfsetdash{}{0pt}%
\pgfpathmoveto{\pgfqpoint{0.968490in}{1.312063in}}%
\pgfpathlineto{\pgfqpoint{0.966411in}{1.306086in}}%
\pgfpathlineto{\pgfqpoint{0.964333in}{1.300137in}}%
\pgfpathlineto{\pgfqpoint{0.962258in}{1.294219in}}%
\pgfpathlineto{\pgfqpoint{0.960184in}{1.288335in}}%
\pgfpathlineto{\pgfqpoint{0.969918in}{1.291674in}}%
\pgfpathlineto{\pgfqpoint{0.979847in}{1.294859in}}%
\pgfpathlineto{\pgfqpoint{0.989961in}{1.297886in}}%
\pgfpathlineto{\pgfqpoint{1.000252in}{1.300754in}}%
\pgfpathlineto{\pgfqpoint{1.001944in}{1.306511in}}%
\pgfpathlineto{\pgfqpoint{1.003638in}{1.312301in}}%
\pgfpathlineto{\pgfqpoint{1.005334in}{1.318123in}}%
\pgfpathlineto{\pgfqpoint{1.007032in}{1.323972in}}%
\pgfpathlineto{\pgfqpoint{0.997132in}{1.321222in}}%
\pgfpathlineto{\pgfqpoint{0.987403in}{1.318318in}}%
\pgfpathlineto{\pgfqpoint{0.977852in}{1.315264in}}%
\pgfpathlineto{\pgfqpoint{0.968490in}{1.312063in}}%
\pgfpathclose%
\pgfusepath{fill}%
\end{pgfscope}%
\begin{pgfscope}%
\pgfpathrectangle{\pgfqpoint{0.041670in}{0.041670in}}{\pgfqpoint{2.216660in}{2.216660in}}%
\pgfusepath{clip}%
\pgfsetbuttcap%
\pgfsetroundjoin%
\definecolor{currentfill}{rgb}{0.271305,0.019942,0.347269}%
\pgfsetfillcolor{currentfill}%
\pgfsetlinewidth{0.000000pt}%
\definecolor{currentstroke}{rgb}{0.000000,0.000000,0.000000}%
\pgfsetstrokecolor{currentstroke}%
\pgfsetdash{}{0pt}%
\pgfpathmoveto{\pgfqpoint{0.650010in}{0.941976in}}%
\pgfpathlineto{\pgfqpoint{0.646420in}{0.939897in}}%
\pgfpathlineto{\pgfqpoint{0.642825in}{0.938018in}}%
\pgfpathlineto{\pgfqpoint{0.639226in}{0.936342in}}%
\pgfpathlineto{\pgfqpoint{0.635623in}{0.934875in}}%
\pgfpathlineto{\pgfqpoint{0.639749in}{0.943735in}}%
\pgfpathlineto{\pgfqpoint{0.644398in}{0.952511in}}%
\pgfpathlineto{\pgfqpoint{0.649563in}{0.961197in}}%
\pgfpathlineto{\pgfqpoint{0.655239in}{0.969785in}}%
\pgfpathlineto{\pgfqpoint{0.658695in}{0.971015in}}%
\pgfpathlineto{\pgfqpoint{0.662148in}{0.972453in}}%
\pgfpathlineto{\pgfqpoint{0.665597in}{0.974094in}}%
\pgfpathlineto{\pgfqpoint{0.669042in}{0.975934in}}%
\pgfpathlineto{\pgfqpoint{0.663532in}{0.967580in}}%
\pgfpathlineto{\pgfqpoint{0.658519in}{0.959130in}}%
\pgfpathlineto{\pgfqpoint{0.654010in}{0.950593in}}%
\pgfpathlineto{\pgfqpoint{0.650010in}{0.941976in}}%
\pgfpathclose%
\pgfusepath{fill}%
\end{pgfscope}%
\begin{pgfscope}%
\pgfpathrectangle{\pgfqpoint{0.041670in}{0.041670in}}{\pgfqpoint{2.216660in}{2.216660in}}%
\pgfusepath{clip}%
\pgfsetbuttcap%
\pgfsetroundjoin%
\definecolor{currentfill}{rgb}{0.280255,0.165693,0.476498}%
\pgfsetfillcolor{currentfill}%
\pgfsetlinewidth{0.000000pt}%
\definecolor{currentstroke}{rgb}{0.000000,0.000000,0.000000}%
\pgfsetstrokecolor{currentstroke}%
\pgfsetdash{}{0pt}%
\pgfpathmoveto{\pgfqpoint{1.593428in}{1.081885in}}%
\pgfpathlineto{\pgfqpoint{1.596600in}{1.077194in}}%
\pgfpathlineto{\pgfqpoint{1.599772in}{1.072623in}}%
\pgfpathlineto{\pgfqpoint{1.602945in}{1.068176in}}%
\pgfpathlineto{\pgfqpoint{1.606117in}{1.063856in}}%
\pgfpathlineto{\pgfqpoint{1.613061in}{1.056955in}}%
\pgfpathlineto{\pgfqpoint{1.619593in}{1.049940in}}%
\pgfpathlineto{\pgfqpoint{1.625705in}{1.042816in}}%
\pgfpathlineto{\pgfqpoint{1.631391in}{1.035589in}}%
\pgfpathlineto{\pgfqpoint{1.628017in}{1.040138in}}%
\pgfpathlineto{\pgfqpoint{1.624643in}{1.044814in}}%
\pgfpathlineto{\pgfqpoint{1.621269in}{1.049614in}}%
\pgfpathlineto{\pgfqpoint{1.617896in}{1.054534in}}%
\pgfpathlineto{\pgfqpoint{1.612394in}{1.061526in}}%
\pgfpathlineto{\pgfqpoint{1.606477in}{1.068419in}}%
\pgfpathlineto{\pgfqpoint{1.600152in}{1.075208in}}%
\pgfpathlineto{\pgfqpoint{1.593428in}{1.081885in}}%
\pgfpathclose%
\pgfusepath{fill}%
\end{pgfscope}%
\begin{pgfscope}%
\pgfpathrectangle{\pgfqpoint{0.041670in}{0.041670in}}{\pgfqpoint{2.216660in}{2.216660in}}%
\pgfusepath{clip}%
\pgfsetbuttcap%
\pgfsetroundjoin%
\definecolor{currentfill}{rgb}{0.231674,0.318106,0.544834}%
\pgfsetfillcolor{currentfill}%
\pgfsetlinewidth{0.000000pt}%
\definecolor{currentstroke}{rgb}{0.000000,0.000000,0.000000}%
\pgfsetstrokecolor{currentstroke}%
\pgfsetdash{}{0pt}%
\pgfpathmoveto{\pgfqpoint{1.483842in}{1.212620in}}%
\pgfpathlineto{\pgfqpoint{1.486499in}{1.206890in}}%
\pgfpathlineto{\pgfqpoint{1.489154in}{1.201226in}}%
\pgfpathlineto{\pgfqpoint{1.491807in}{1.195631in}}%
\pgfpathlineto{\pgfqpoint{1.494459in}{1.190106in}}%
\pgfpathlineto{\pgfqpoint{1.503185in}{1.185077in}}%
\pgfpathlineto{\pgfqpoint{1.511607in}{1.179910in}}%
\pgfpathlineto{\pgfqpoint{1.519718in}{1.174607in}}%
\pgfpathlineto{\pgfqpoint{1.527508in}{1.169175in}}%
\pgfpathlineto{\pgfqpoint{1.524569in}{1.174893in}}%
\pgfpathlineto{\pgfqpoint{1.521628in}{1.180683in}}%
\pgfpathlineto{\pgfqpoint{1.518685in}{1.186541in}}%
\pgfpathlineto{\pgfqpoint{1.515741in}{1.192463in}}%
\pgfpathlineto{\pgfqpoint{1.508223in}{1.197694in}}%
\pgfpathlineto{\pgfqpoint{1.500395in}{1.202800in}}%
\pgfpathlineto{\pgfqpoint{1.492266in}{1.207777in}}%
\pgfpathlineto{\pgfqpoint{1.483842in}{1.212620in}}%
\pgfpathclose%
\pgfusepath{fill}%
\end{pgfscope}%
\begin{pgfscope}%
\pgfpathrectangle{\pgfqpoint{0.041670in}{0.041670in}}{\pgfqpoint{2.216660in}{2.216660in}}%
\pgfusepath{clip}%
\pgfsetbuttcap%
\pgfsetroundjoin%
\definecolor{currentfill}{rgb}{0.263663,0.237631,0.518762}%
\pgfsetfillcolor{currentfill}%
\pgfsetlinewidth{0.000000pt}%
\definecolor{currentstroke}{rgb}{0.000000,0.000000,0.000000}%
\pgfsetstrokecolor{currentstroke}%
\pgfsetdash{}{0pt}%
\pgfpathmoveto{\pgfqpoint{0.786272in}{1.117773in}}%
\pgfpathlineto{\pgfqpoint{0.783046in}{1.112201in}}%
\pgfpathlineto{\pgfqpoint{0.779821in}{1.106719in}}%
\pgfpathlineto{\pgfqpoint{0.776596in}{1.101334in}}%
\pgfpathlineto{\pgfqpoint{0.773373in}{1.096047in}}%
\pgfpathlineto{\pgfqpoint{0.779922in}{1.102488in}}%
\pgfpathlineto{\pgfqpoint{0.786852in}{1.108817in}}%
\pgfpathlineto{\pgfqpoint{0.794155in}{1.115026in}}%
\pgfpathlineto{\pgfqpoint{0.801823in}{1.121110in}}%
\pgfpathlineto{\pgfqpoint{0.804811in}{1.126181in}}%
\pgfpathlineto{\pgfqpoint{0.807799in}{1.131350in}}%
\pgfpathlineto{\pgfqpoint{0.810788in}{1.136614in}}%
\pgfpathlineto{\pgfqpoint{0.813778in}{1.141969in}}%
\pgfpathlineto{\pgfqpoint{0.806363in}{1.136095in}}%
\pgfpathlineto{\pgfqpoint{0.799302in}{1.130101in}}%
\pgfpathlineto{\pgfqpoint{0.792602in}{1.123992in}}%
\pgfpathlineto{\pgfqpoint{0.786272in}{1.117773in}}%
\pgfpathclose%
\pgfusepath{fill}%
\end{pgfscope}%
\begin{pgfscope}%
\pgfpathrectangle{\pgfqpoint{0.041670in}{0.041670in}}{\pgfqpoint{2.216660in}{2.216660in}}%
\pgfusepath{clip}%
\pgfsetbuttcap%
\pgfsetroundjoin%
\definecolor{currentfill}{rgb}{0.163625,0.471133,0.558148}%
\pgfsetfillcolor{currentfill}%
\pgfsetlinewidth{0.000000pt}%
\definecolor{currentstroke}{rgb}{0.000000,0.000000,0.000000}%
\pgfsetstrokecolor{currentstroke}%
\pgfsetdash{}{0pt}%
\pgfpathmoveto{\pgfqpoint{1.297535in}{1.358284in}}%
\pgfpathlineto{\pgfqpoint{1.298740in}{1.352471in}}%
\pgfpathlineto{\pgfqpoint{1.299944in}{1.346675in}}%
\pgfpathlineto{\pgfqpoint{1.301146in}{1.340897in}}%
\pgfpathlineto{\pgfqpoint{1.302347in}{1.335142in}}%
\pgfpathlineto{\pgfqpoint{1.312940in}{1.333171in}}%
\pgfpathlineto{\pgfqpoint{1.323410in}{1.331037in}}%
\pgfpathlineto{\pgfqpoint{1.333748in}{1.328741in}}%
\pgfpathlineto{\pgfqpoint{1.343943in}{1.326286in}}%
\pgfpathlineto{\pgfqpoint{1.342332in}{1.332136in}}%
\pgfpathlineto{\pgfqpoint{1.340719in}{1.338008in}}%
\pgfpathlineto{\pgfqpoint{1.339104in}{1.343899in}}%
\pgfpathlineto{\pgfqpoint{1.337487in}{1.349806in}}%
\pgfpathlineto{\pgfqpoint{1.327696in}{1.352156in}}%
\pgfpathlineto{\pgfqpoint{1.317767in}{1.354354in}}%
\pgfpathlineto{\pgfqpoint{1.307710in}{1.356397in}}%
\pgfpathlineto{\pgfqpoint{1.297535in}{1.358284in}}%
\pgfpathclose%
\pgfusepath{fill}%
\end{pgfscope}%
\begin{pgfscope}%
\pgfpathrectangle{\pgfqpoint{0.041670in}{0.041670in}}{\pgfqpoint{2.216660in}{2.216660in}}%
\pgfusepath{clip}%
\pgfsetbuttcap%
\pgfsetroundjoin%
\definecolor{currentfill}{rgb}{0.279566,0.067836,0.391917}%
\pgfsetfillcolor{currentfill}%
\pgfsetlinewidth{0.000000pt}%
\definecolor{currentstroke}{rgb}{0.000000,0.000000,0.000000}%
\pgfsetstrokecolor{currentstroke}%
\pgfsetdash{}{0pt}%
\pgfpathmoveto{\pgfqpoint{1.658416in}{1.004243in}}%
\pgfpathlineto{\pgfqpoint{1.661800in}{1.001013in}}%
\pgfpathlineto{\pgfqpoint{1.665187in}{0.997948in}}%
\pgfpathlineto{\pgfqpoint{1.668576in}{0.995054in}}%
\pgfpathlineto{\pgfqpoint{1.671967in}{0.992333in}}%
\pgfpathlineto{\pgfqpoint{1.677737in}{0.984303in}}%
\pgfpathlineto{\pgfqpoint{1.683029in}{0.976174in}}%
\pgfpathlineto{\pgfqpoint{1.687837in}{0.967954in}}%
\pgfpathlineto{\pgfqpoint{1.692153in}{0.959650in}}%
\pgfpathlineto{\pgfqpoint{1.688607in}{0.962609in}}%
\pgfpathlineto{\pgfqpoint{1.685063in}{0.965743in}}%
\pgfpathlineto{\pgfqpoint{1.681522in}{0.969047in}}%
\pgfpathlineto{\pgfqpoint{1.677983in}{0.972517in}}%
\pgfpathlineto{\pgfqpoint{1.673802in}{0.980577in}}%
\pgfpathlineto{\pgfqpoint{1.669143in}{0.988556in}}%
\pgfpathlineto{\pgfqpoint{1.664012in}{0.996447in}}%
\pgfpathlineto{\pgfqpoint{1.658416in}{1.004243in}}%
\pgfpathclose%
\pgfusepath{fill}%
\end{pgfscope}%
\begin{pgfscope}%
\pgfpathrectangle{\pgfqpoint{0.041670in}{0.041670in}}{\pgfqpoint{2.216660in}{2.216660in}}%
\pgfusepath{clip}%
\pgfsetbuttcap%
\pgfsetroundjoin%
\definecolor{currentfill}{rgb}{0.274952,0.037752,0.364543}%
\pgfsetfillcolor{currentfill}%
\pgfsetlinewidth{0.000000pt}%
\definecolor{currentstroke}{rgb}{0.000000,0.000000,0.000000}%
\pgfsetstrokecolor{currentstroke}%
\pgfsetdash{}{0pt}%
\pgfpathmoveto{\pgfqpoint{0.664337in}{0.952204in}}%
\pgfpathlineto{\pgfqpoint{0.660760in}{0.949369in}}%
\pgfpathlineto{\pgfqpoint{0.657180in}{0.946716in}}%
\pgfpathlineto{\pgfqpoint{0.653597in}{0.944251in}}%
\pgfpathlineto{\pgfqpoint{0.650010in}{0.941976in}}%
\pgfpathlineto{\pgfqpoint{0.654010in}{0.950593in}}%
\pgfpathlineto{\pgfqpoint{0.658519in}{0.959130in}}%
\pgfpathlineto{\pgfqpoint{0.663532in}{0.967580in}}%
\pgfpathlineto{\pgfqpoint{0.669042in}{0.975934in}}%
\pgfpathlineto{\pgfqpoint{0.672484in}{0.977969in}}%
\pgfpathlineto{\pgfqpoint{0.675922in}{0.980194in}}%
\pgfpathlineto{\pgfqpoint{0.679358in}{0.982606in}}%
\pgfpathlineto{\pgfqpoint{0.682790in}{0.985201in}}%
\pgfpathlineto{\pgfqpoint{0.677445in}{0.977082in}}%
\pgfpathlineto{\pgfqpoint{0.672583in}{0.968872in}}%
\pgfpathlineto{\pgfqpoint{0.668212in}{0.960576in}}%
\pgfpathlineto{\pgfqpoint{0.664337in}{0.952204in}}%
\pgfpathclose%
\pgfusepath{fill}%
\end{pgfscope}%
\begin{pgfscope}%
\pgfpathrectangle{\pgfqpoint{0.041670in}{0.041670in}}{\pgfqpoint{2.216660in}{2.216660in}}%
\pgfusepath{clip}%
\pgfsetbuttcap%
\pgfsetroundjoin%
\definecolor{currentfill}{rgb}{0.163625,0.471133,0.558148}%
\pgfsetfillcolor{currentfill}%
\pgfsetlinewidth{0.000000pt}%
\definecolor{currentstroke}{rgb}{0.000000,0.000000,0.000000}%
\pgfsetstrokecolor{currentstroke}%
\pgfsetdash{}{0pt}%
\pgfpathmoveto{\pgfqpoint{1.013842in}{1.347590in}}%
\pgfpathlineto{\pgfqpoint{1.012137in}{1.341658in}}%
\pgfpathlineto{\pgfqpoint{1.010433in}{1.335743in}}%
\pgfpathlineto{\pgfqpoint{1.008732in}{1.329846in}}%
\pgfpathlineto{\pgfqpoint{1.007032in}{1.323972in}}%
\pgfpathlineto{\pgfqpoint{1.017092in}{1.326567in}}%
\pgfpathlineto{\pgfqpoint{1.027304in}{1.329004in}}%
\pgfpathlineto{\pgfqpoint{1.037656in}{1.331282in}}%
\pgfpathlineto{\pgfqpoint{1.048141in}{1.333398in}}%
\pgfpathlineto{\pgfqpoint{1.049434in}{1.339172in}}%
\pgfpathlineto{\pgfqpoint{1.050730in}{1.344968in}}%
\pgfpathlineto{\pgfqpoint{1.052026in}{1.350783in}}%
\pgfpathlineto{\pgfqpoint{1.053325in}{1.356615in}}%
\pgfpathlineto{\pgfqpoint{1.043254in}{1.354589in}}%
\pgfpathlineto{\pgfqpoint{1.033311in}{1.352408in}}%
\pgfpathlineto{\pgfqpoint{1.023504in}{1.350075in}}%
\pgfpathlineto{\pgfqpoint{1.013842in}{1.347590in}}%
\pgfpathclose%
\pgfusepath{fill}%
\end{pgfscope}%
\begin{pgfscope}%
\pgfpathrectangle{\pgfqpoint{0.041670in}{0.041670in}}{\pgfqpoint{2.216660in}{2.216660in}}%
\pgfusepath{clip}%
\pgfsetbuttcap%
\pgfsetroundjoin%
\definecolor{currentfill}{rgb}{0.231674,0.318106,0.544834}%
\pgfsetfillcolor{currentfill}%
\pgfsetlinewidth{0.000000pt}%
\definecolor{currentstroke}{rgb}{0.000000,0.000000,0.000000}%
\pgfsetstrokecolor{currentstroke}%
\pgfsetdash{}{0pt}%
\pgfpathmoveto{\pgfqpoint{0.837753in}{1.187713in}}%
\pgfpathlineto{\pgfqpoint{0.834750in}{1.181744in}}%
\pgfpathlineto{\pgfqpoint{0.831750in}{1.175841in}}%
\pgfpathlineto{\pgfqpoint{0.828750in}{1.170006in}}%
\pgfpathlineto{\pgfqpoint{0.825753in}{1.164242in}}%
\pgfpathlineto{\pgfqpoint{0.833251in}{1.169785in}}%
\pgfpathlineto{\pgfqpoint{0.841077in}{1.175203in}}%
\pgfpathlineto{\pgfqpoint{0.849223in}{1.180491in}}%
\pgfpathlineto{\pgfqpoint{0.857680in}{1.185643in}}%
\pgfpathlineto{\pgfqpoint{0.860398in}{1.191209in}}%
\pgfpathlineto{\pgfqpoint{0.863119in}{1.196846in}}%
\pgfpathlineto{\pgfqpoint{0.865841in}{1.202551in}}%
\pgfpathlineto{\pgfqpoint{0.868566in}{1.208321in}}%
\pgfpathlineto{\pgfqpoint{0.860403in}{1.203359in}}%
\pgfpathlineto{\pgfqpoint{0.852541in}{1.198268in}}%
\pgfpathlineto{\pgfqpoint{0.844989in}{1.193051in}}%
\pgfpathlineto{\pgfqpoint{0.837753in}{1.187713in}}%
\pgfpathclose%
\pgfusepath{fill}%
\end{pgfscope}%
\begin{pgfscope}%
\pgfpathrectangle{\pgfqpoint{0.041670in}{0.041670in}}{\pgfqpoint{2.216660in}{2.216660in}}%
\pgfusepath{clip}%
\pgfsetbuttcap%
\pgfsetroundjoin%
\definecolor{currentfill}{rgb}{0.276194,0.190074,0.493001}%
\pgfsetfillcolor{currentfill}%
\pgfsetlinewidth{0.000000pt}%
\definecolor{currentstroke}{rgb}{0.000000,0.000000,0.000000}%
\pgfsetstrokecolor{currentstroke}%
\pgfsetdash{}{0pt}%
\pgfpathmoveto{\pgfqpoint{1.812124in}{1.077385in}}%
\pgfpathlineto{\pgfqpoint{1.815818in}{1.085788in}}%
\pgfpathlineto{\pgfqpoint{1.819527in}{1.094575in}}%
\pgfpathlineto{\pgfqpoint{1.823250in}{1.103754in}}%
\pgfpathlineto{\pgfqpoint{1.826988in}{1.113331in}}%
\pgfpathlineto{\pgfqpoint{1.834809in}{1.102947in}}%
\pgfpathlineto{\pgfqpoint{1.842010in}{1.092428in}}%
\pgfpathlineto{\pgfqpoint{1.848580in}{1.081783in}}%
\pgfpathlineto{\pgfqpoint{1.854511in}{1.071021in}}%
\pgfpathlineto{\pgfqpoint{1.850591in}{1.061632in}}%
\pgfpathlineto{\pgfqpoint{1.846687in}{1.052643in}}%
\pgfpathlineto{\pgfqpoint{1.842798in}{1.044048in}}%
\pgfpathlineto{\pgfqpoint{1.838924in}{1.035839in}}%
\pgfpathlineto{\pgfqpoint{1.833153in}{1.046406in}}%
\pgfpathlineto{\pgfqpoint{1.826756in}{1.056858in}}%
\pgfpathlineto{\pgfqpoint{1.819743in}{1.067188in}}%
\pgfpathlineto{\pgfqpoint{1.812124in}{1.077385in}}%
\pgfpathclose%
\pgfusepath{fill}%
\end{pgfscope}%
\begin{pgfscope}%
\pgfpathrectangle{\pgfqpoint{0.041670in}{0.041670in}}{\pgfqpoint{2.216660in}{2.216660in}}%
\pgfusepath{clip}%
\pgfsetbuttcap%
\pgfsetroundjoin%
\definecolor{currentfill}{rgb}{0.282884,0.135920,0.453427}%
\pgfsetfillcolor{currentfill}%
\pgfsetlinewidth{0.000000pt}%
\definecolor{currentstroke}{rgb}{0.000000,0.000000,0.000000}%
\pgfsetstrokecolor{currentstroke}%
\pgfsetdash{}{0pt}%
\pgfpathmoveto{\pgfqpoint{0.531873in}{0.997451in}}%
\pgfpathlineto{\pgfqpoint{0.528022in}{1.004129in}}%
\pgfpathlineto{\pgfqpoint{0.524158in}{1.011169in}}%
\pgfpathlineto{\pgfqpoint{0.520280in}{1.018577in}}%
\pgfpathlineto{\pgfqpoint{0.516388in}{1.026360in}}%
\pgfpathlineto{\pgfqpoint{0.521596in}{1.037019in}}%
\pgfpathlineto{\pgfqpoint{0.527437in}{1.047573in}}%
\pgfpathlineto{\pgfqpoint{0.533902in}{1.058012in}}%
\pgfpathlineto{\pgfqpoint{0.540983in}{1.068328in}}%
\pgfpathlineto{\pgfqpoint{0.544709in}{1.060346in}}%
\pgfpathlineto{\pgfqpoint{0.548421in}{1.052737in}}%
\pgfpathlineto{\pgfqpoint{0.552120in}{1.045495in}}%
\pgfpathlineto{\pgfqpoint{0.555807in}{1.038613in}}%
\pgfpathlineto{\pgfqpoint{0.548912in}{1.028495in}}%
\pgfpathlineto{\pgfqpoint{0.542619in}{1.018256in}}%
\pgfpathlineto{\pgfqpoint{0.536937in}{1.007905in}}%
\pgfpathlineto{\pgfqpoint{0.531873in}{0.997451in}}%
\pgfpathclose%
\pgfusepath{fill}%
\end{pgfscope}%
\begin{pgfscope}%
\pgfpathrectangle{\pgfqpoint{0.041670in}{0.041670in}}{\pgfqpoint{2.216660in}{2.216660in}}%
\pgfusepath{clip}%
\pgfsetbuttcap%
\pgfsetroundjoin%
\definecolor{currentfill}{rgb}{0.147607,0.511733,0.557049}%
\pgfsetfillcolor{currentfill}%
\pgfsetlinewidth{0.000000pt}%
\definecolor{currentstroke}{rgb}{0.000000,0.000000,0.000000}%
\pgfsetstrokecolor{currentstroke}%
\pgfsetdash{}{0pt}%
\pgfpathmoveto{\pgfqpoint{1.139058in}{1.390043in}}%
\pgfpathlineto{\pgfqpoint{1.138619in}{1.384291in}}%
\pgfpathlineto{\pgfqpoint{1.138181in}{1.378543in}}%
\pgfpathlineto{\pgfqpoint{1.137744in}{1.372804in}}%
\pgfpathlineto{\pgfqpoint{1.137306in}{1.367075in}}%
\pgfpathlineto{\pgfqpoint{1.148085in}{1.367647in}}%
\pgfpathlineto{\pgfqpoint{1.158893in}{1.368054in}}%
\pgfpathlineto{\pgfqpoint{1.169720in}{1.368294in}}%
\pgfpathlineto{\pgfqpoint{1.180557in}{1.368368in}}%
\pgfpathlineto{\pgfqpoint{1.180551in}{1.374082in}}%
\pgfpathlineto{\pgfqpoint{1.180545in}{1.379808in}}%
\pgfpathlineto{\pgfqpoint{1.180538in}{1.385541in}}%
\pgfpathlineto{\pgfqpoint{1.180532in}{1.391279in}}%
\pgfpathlineto{\pgfqpoint{1.170140in}{1.391208in}}%
\pgfpathlineto{\pgfqpoint{1.159758in}{1.390979in}}%
\pgfpathlineto{\pgfqpoint{1.149394in}{1.390590in}}%
\pgfpathlineto{\pgfqpoint{1.139058in}{1.390043in}}%
\pgfpathclose%
\pgfusepath{fill}%
\end{pgfscope}%
\begin{pgfscope}%
\pgfpathrectangle{\pgfqpoint{0.041670in}{0.041670in}}{\pgfqpoint{2.216660in}{2.216660in}}%
\pgfusepath{clip}%
\pgfsetbuttcap%
\pgfsetroundjoin%
\definecolor{currentfill}{rgb}{0.147607,0.511733,0.557049}%
\pgfsetfillcolor{currentfill}%
\pgfsetlinewidth{0.000000pt}%
\definecolor{currentstroke}{rgb}{0.000000,0.000000,0.000000}%
\pgfsetstrokecolor{currentstroke}%
\pgfsetdash{}{0pt}%
\pgfpathmoveto{\pgfqpoint{1.180532in}{1.391279in}}%
\pgfpathlineto{\pgfqpoint{1.180538in}{1.385541in}}%
\pgfpathlineto{\pgfqpoint{1.180545in}{1.379808in}}%
\pgfpathlineto{\pgfqpoint{1.180551in}{1.374082in}}%
\pgfpathlineto{\pgfqpoint{1.180557in}{1.368368in}}%
\pgfpathlineto{\pgfqpoint{1.191393in}{1.368275in}}%
\pgfpathlineto{\pgfqpoint{1.202219in}{1.368017in}}%
\pgfpathlineto{\pgfqpoint{1.213024in}{1.367592in}}%
\pgfpathlineto{\pgfqpoint{1.223799in}{1.367001in}}%
\pgfpathlineto{\pgfqpoint{1.223349in}{1.372731in}}%
\pgfpathlineto{\pgfqpoint{1.222899in}{1.378471in}}%
\pgfpathlineto{\pgfqpoint{1.222449in}{1.384219in}}%
\pgfpathlineto{\pgfqpoint{1.221998in}{1.389973in}}%
\pgfpathlineto{\pgfqpoint{1.211666in}{1.390537in}}%
\pgfpathlineto{\pgfqpoint{1.201305in}{1.390943in}}%
\pgfpathlineto{\pgfqpoint{1.190924in}{1.391191in}}%
\pgfpathlineto{\pgfqpoint{1.180532in}{1.391279in}}%
\pgfpathclose%
\pgfusepath{fill}%
\end{pgfscope}%
\begin{pgfscope}%
\pgfpathrectangle{\pgfqpoint{0.041670in}{0.041670in}}{\pgfqpoint{2.216660in}{2.216660in}}%
\pgfusepath{clip}%
\pgfsetbuttcap%
\pgfsetroundjoin%
\definecolor{currentfill}{rgb}{0.280255,0.165693,0.476498}%
\pgfsetfillcolor{currentfill}%
\pgfsetlinewidth{0.000000pt}%
\definecolor{currentstroke}{rgb}{0.000000,0.000000,0.000000}%
\pgfsetstrokecolor{currentstroke}%
\pgfsetdash{}{0pt}%
\pgfpathmoveto{\pgfqpoint{0.737476in}{1.048241in}}%
\pgfpathlineto{\pgfqpoint{0.734065in}{1.043268in}}%
\pgfpathlineto{\pgfqpoint{0.730653in}{1.038415in}}%
\pgfpathlineto{\pgfqpoint{0.727242in}{1.033686in}}%
\pgfpathlineto{\pgfqpoint{0.723829in}{1.029085in}}%
\pgfpathlineto{\pgfqpoint{0.729129in}{1.036397in}}%
\pgfpathlineto{\pgfqpoint{0.734863in}{1.043612in}}%
\pgfpathlineto{\pgfqpoint{0.741022in}{1.050725in}}%
\pgfpathlineto{\pgfqpoint{0.747600in}{1.057728in}}%
\pgfpathlineto{\pgfqpoint{0.750821in}{1.062097in}}%
\pgfpathlineto{\pgfqpoint{0.754042in}{1.066594in}}%
\pgfpathlineto{\pgfqpoint{0.757264in}{1.071215in}}%
\pgfpathlineto{\pgfqpoint{0.760485in}{1.075955in}}%
\pgfpathlineto{\pgfqpoint{0.754116in}{1.069179in}}%
\pgfpathlineto{\pgfqpoint{0.748153in}{1.062297in}}%
\pgfpathlineto{\pgfqpoint{0.742605in}{1.055316in}}%
\pgfpathlineto{\pgfqpoint{0.737476in}{1.048241in}}%
\pgfpathclose%
\pgfusepath{fill}%
\end{pgfscope}%
\begin{pgfscope}%
\pgfpathrectangle{\pgfqpoint{0.041670in}{0.041670in}}{\pgfqpoint{2.216660in}{2.216660in}}%
\pgfusepath{clip}%
\pgfsetbuttcap%
\pgfsetroundjoin%
\definecolor{currentfill}{rgb}{0.147607,0.511733,0.557049}%
\pgfsetfillcolor{currentfill}%
\pgfsetlinewidth{0.000000pt}%
\definecolor{currentstroke}{rgb}{0.000000,0.000000,0.000000}%
\pgfsetstrokecolor{currentstroke}%
\pgfsetdash{}{0pt}%
\pgfpathmoveto{\pgfqpoint{1.098189in}{1.386282in}}%
\pgfpathlineto{\pgfqpoint{1.097312in}{1.380486in}}%
\pgfpathlineto{\pgfqpoint{1.096435in}{1.374695in}}%
\pgfpathlineto{\pgfqpoint{1.095560in}{1.368912in}}%
\pgfpathlineto{\pgfqpoint{1.094685in}{1.363139in}}%
\pgfpathlineto{\pgfqpoint{1.105247in}{1.364369in}}%
\pgfpathlineto{\pgfqpoint{1.115877in}{1.365435in}}%
\pgfpathlineto{\pgfqpoint{1.126567in}{1.366337in}}%
\pgfpathlineto{\pgfqpoint{1.137306in}{1.367075in}}%
\pgfpathlineto{\pgfqpoint{1.137744in}{1.372804in}}%
\pgfpathlineto{\pgfqpoint{1.138181in}{1.378543in}}%
\pgfpathlineto{\pgfqpoint{1.138619in}{1.384291in}}%
\pgfpathlineto{\pgfqpoint{1.139058in}{1.390043in}}%
\pgfpathlineto{\pgfqpoint{1.128760in}{1.389338in}}%
\pgfpathlineto{\pgfqpoint{1.118510in}{1.388476in}}%
\pgfpathlineto{\pgfqpoint{1.108316in}{1.387457in}}%
\pgfpathlineto{\pgfqpoint{1.098189in}{1.386282in}}%
\pgfpathclose%
\pgfusepath{fill}%
\end{pgfscope}%
\begin{pgfscope}%
\pgfpathrectangle{\pgfqpoint{0.041670in}{0.041670in}}{\pgfqpoint{2.216660in}{2.216660in}}%
\pgfusepath{clip}%
\pgfsetbuttcap%
\pgfsetroundjoin%
\definecolor{currentfill}{rgb}{0.147607,0.511733,0.557049}%
\pgfsetfillcolor{currentfill}%
\pgfsetlinewidth{0.000000pt}%
\definecolor{currentstroke}{rgb}{0.000000,0.000000,0.000000}%
\pgfsetstrokecolor{currentstroke}%
\pgfsetdash{}{0pt}%
\pgfpathmoveto{\pgfqpoint{1.221998in}{1.389973in}}%
\pgfpathlineto{\pgfqpoint{1.222449in}{1.384219in}}%
\pgfpathlineto{\pgfqpoint{1.222899in}{1.378471in}}%
\pgfpathlineto{\pgfqpoint{1.223349in}{1.372731in}}%
\pgfpathlineto{\pgfqpoint{1.223799in}{1.367001in}}%
\pgfpathlineto{\pgfqpoint{1.234533in}{1.366245in}}%
\pgfpathlineto{\pgfqpoint{1.245217in}{1.365324in}}%
\pgfpathlineto{\pgfqpoint{1.255840in}{1.364240in}}%
\pgfpathlineto{\pgfqpoint{1.266393in}{1.362993in}}%
\pgfpathlineto{\pgfqpoint{1.265507in}{1.368767in}}%
\pgfpathlineto{\pgfqpoint{1.264619in}{1.374552in}}%
\pgfpathlineto{\pgfqpoint{1.263731in}{1.380344in}}%
\pgfpathlineto{\pgfqpoint{1.262841in}{1.386142in}}%
\pgfpathlineto{\pgfqpoint{1.252722in}{1.387334in}}%
\pgfpathlineto{\pgfqpoint{1.242535in}{1.388370in}}%
\pgfpathlineto{\pgfqpoint{1.232291in}{1.389250in}}%
\pgfpathlineto{\pgfqpoint{1.221998in}{1.389973in}}%
\pgfpathclose%
\pgfusepath{fill}%
\end{pgfscope}%
\begin{pgfscope}%
\pgfpathrectangle{\pgfqpoint{0.041670in}{0.041670in}}{\pgfqpoint{2.216660in}{2.216660in}}%
\pgfusepath{clip}%
\pgfsetbuttcap%
\pgfsetroundjoin%
\definecolor{currentfill}{rgb}{0.201239,0.383670,0.554294}%
\pgfsetfillcolor{currentfill}%
\pgfsetlinewidth{0.000000pt}%
\definecolor{currentstroke}{rgb}{0.000000,0.000000,0.000000}%
\pgfsetstrokecolor{currentstroke}%
\pgfsetdash{}{0pt}%
\pgfpathmoveto{\pgfqpoint{0.495088in}{1.195554in}}%
\pgfpathlineto{\pgfqpoint{0.491152in}{1.208986in}}%
\pgfpathlineto{\pgfqpoint{0.487196in}{1.222888in}}%
\pgfpathlineto{\pgfqpoint{0.483220in}{1.237266in}}%
\pgfpathlineto{\pgfqpoint{0.479224in}{1.252129in}}%
\pgfpathlineto{\pgfqpoint{0.487750in}{1.263002in}}%
\pgfpathlineto{\pgfqpoint{0.496927in}{1.273719in}}%
\pgfpathlineto{\pgfqpoint{0.506741in}{1.284272in}}%
\pgfpathlineto{\pgfqpoint{0.517182in}{1.294652in}}%
\pgfpathlineto{\pgfqpoint{0.520938in}{1.279639in}}%
\pgfpathlineto{\pgfqpoint{0.524676in}{1.265109in}}%
\pgfpathlineto{\pgfqpoint{0.528395in}{1.251053in}}%
\pgfpathlineto{\pgfqpoint{0.532096in}{1.237463in}}%
\pgfpathlineto{\pgfqpoint{0.521913in}{1.227233in}}%
\pgfpathlineto{\pgfqpoint{0.512343in}{1.216831in}}%
\pgfpathlineto{\pgfqpoint{0.503398in}{1.206269in}}%
\pgfpathlineto{\pgfqpoint{0.495088in}{1.195554in}}%
\pgfpathclose%
\pgfusepath{fill}%
\end{pgfscope}%
\begin{pgfscope}%
\pgfpathrectangle{\pgfqpoint{0.041670in}{0.041670in}}{\pgfqpoint{2.216660in}{2.216660in}}%
\pgfusepath{clip}%
\pgfsetbuttcap%
\pgfsetroundjoin%
\definecolor{currentfill}{rgb}{0.248629,0.278775,0.534556}%
\pgfsetfillcolor{currentfill}%
\pgfsetlinewidth{0.000000pt}%
\definecolor{currentstroke}{rgb}{0.000000,0.000000,0.000000}%
\pgfsetstrokecolor{currentstroke}%
\pgfsetdash{}{0pt}%
\pgfpathmoveto{\pgfqpoint{1.527508in}{1.169175in}}%
\pgfpathlineto{\pgfqpoint{1.530445in}{1.163532in}}%
\pgfpathlineto{\pgfqpoint{1.533381in}{1.157968in}}%
\pgfpathlineto{\pgfqpoint{1.536316in}{1.152484in}}%
\pgfpathlineto{\pgfqpoint{1.539249in}{1.147086in}}%
\pgfpathlineto{\pgfqpoint{1.546973in}{1.141323in}}%
\pgfpathlineto{\pgfqpoint{1.554349in}{1.135435in}}%
\pgfpathlineto{\pgfqpoint{1.561370in}{1.129427in}}%
\pgfpathlineto{\pgfqpoint{1.568029in}{1.123306in}}%
\pgfpathlineto{\pgfqpoint{1.564850in}{1.128917in}}%
\pgfpathlineto{\pgfqpoint{1.561670in}{1.134614in}}%
\pgfpathlineto{\pgfqpoint{1.558489in}{1.140392in}}%
\pgfpathlineto{\pgfqpoint{1.555305in}{1.146248in}}%
\pgfpathlineto{\pgfqpoint{1.548876in}{1.152150in}}%
\pgfpathlineto{\pgfqpoint{1.542095in}{1.157941in}}%
\pgfpathlineto{\pgfqpoint{1.534969in}{1.163618in}}%
\pgfpathlineto{\pgfqpoint{1.527508in}{1.169175in}}%
\pgfpathclose%
\pgfusepath{fill}%
\end{pgfscope}%
\begin{pgfscope}%
\pgfpathrectangle{\pgfqpoint{0.041670in}{0.041670in}}{\pgfqpoint{2.216660in}{2.216660in}}%
\pgfusepath{clip}%
\pgfsetbuttcap%
\pgfsetroundjoin%
\definecolor{currentfill}{rgb}{0.179019,0.433756,0.557430}%
\pgfsetfillcolor{currentfill}%
\pgfsetlinewidth{0.000000pt}%
\definecolor{currentstroke}{rgb}{0.000000,0.000000,0.000000}%
\pgfsetstrokecolor{currentstroke}%
\pgfsetdash{}{0pt}%
\pgfpathmoveto{\pgfqpoint{1.383107in}{1.314916in}}%
\pgfpathlineto{\pgfqpoint{1.385104in}{1.308970in}}%
\pgfpathlineto{\pgfqpoint{1.387099in}{1.303051in}}%
\pgfpathlineto{\pgfqpoint{1.389092in}{1.297164in}}%
\pgfpathlineto{\pgfqpoint{1.391083in}{1.291311in}}%
\pgfpathlineto{\pgfqpoint{1.400795in}{1.287955in}}%
\pgfpathlineto{\pgfqpoint{1.410300in}{1.284448in}}%
\pgfpathlineto{\pgfqpoint{1.419591in}{1.280793in}}%
\pgfpathlineto{\pgfqpoint{1.428658in}{1.276993in}}%
\pgfpathlineto{\pgfqpoint{1.426308in}{1.282994in}}%
\pgfpathlineto{\pgfqpoint{1.423955in}{1.289029in}}%
\pgfpathlineto{\pgfqpoint{1.421600in}{1.295094in}}%
\pgfpathlineto{\pgfqpoint{1.419243in}{1.301188in}}%
\pgfpathlineto{\pgfqpoint{1.410524in}{1.304831in}}%
\pgfpathlineto{\pgfqpoint{1.401589in}{1.308335in}}%
\pgfpathlineto{\pgfqpoint{1.392447in}{1.311698in}}%
\pgfpathlineto{\pgfqpoint{1.383107in}{1.314916in}}%
\pgfpathclose%
\pgfusepath{fill}%
\end{pgfscope}%
\begin{pgfscope}%
\pgfpathrectangle{\pgfqpoint{0.041670in}{0.041670in}}{\pgfqpoint{2.216660in}{2.216660in}}%
\pgfusepath{clip}%
\pgfsetbuttcap%
\pgfsetroundjoin%
\definecolor{currentfill}{rgb}{0.195860,0.395433,0.555276}%
\pgfsetfillcolor{currentfill}%
\pgfsetlinewidth{0.000000pt}%
\definecolor{currentstroke}{rgb}{0.000000,0.000000,0.000000}%
\pgfsetstrokecolor{currentstroke}%
\pgfsetdash{}{0pt}%
\pgfpathmoveto{\pgfqpoint{1.428658in}{1.276993in}}%
\pgfpathlineto{\pgfqpoint{1.431006in}{1.271029in}}%
\pgfpathlineto{\pgfqpoint{1.433352in}{1.265105in}}%
\pgfpathlineto{\pgfqpoint{1.435695in}{1.259224in}}%
\pgfpathlineto{\pgfqpoint{1.438037in}{1.253389in}}%
\pgfpathlineto{\pgfqpoint{1.447210in}{1.249284in}}%
\pgfpathlineto{\pgfqpoint{1.456133in}{1.245036in}}%
\pgfpathlineto{\pgfqpoint{1.464797in}{1.240648in}}%
\pgfpathlineto{\pgfqpoint{1.473194in}{1.236123in}}%
\pgfpathlineto{\pgfqpoint{1.470527in}{1.242129in}}%
\pgfpathlineto{\pgfqpoint{1.467858in}{1.248182in}}%
\pgfpathlineto{\pgfqpoint{1.465186in}{1.254278in}}%
\pgfpathlineto{\pgfqpoint{1.462512in}{1.260413in}}%
\pgfpathlineto{\pgfqpoint{1.454427in}{1.264758in}}%
\pgfpathlineto{\pgfqpoint{1.446085in}{1.268972in}}%
\pgfpathlineto{\pgfqpoint{1.437492in}{1.273051in}}%
\pgfpathlineto{\pgfqpoint{1.428658in}{1.276993in}}%
\pgfpathclose%
\pgfusepath{fill}%
\end{pgfscope}%
\begin{pgfscope}%
\pgfpathrectangle{\pgfqpoint{0.041670in}{0.041670in}}{\pgfqpoint{2.216660in}{2.216660in}}%
\pgfusepath{clip}%
\pgfsetbuttcap%
\pgfsetroundjoin%
\definecolor{currentfill}{rgb}{0.274128,0.199721,0.498911}%
\pgfsetfillcolor{currentfill}%
\pgfsetlinewidth{0.000000pt}%
\definecolor{currentstroke}{rgb}{0.000000,0.000000,0.000000}%
\pgfsetstrokecolor{currentstroke}%
\pgfsetdash{}{0pt}%
\pgfpathmoveto{\pgfqpoint{1.580734in}{1.101778in}}%
\pgfpathlineto{\pgfqpoint{1.583908in}{1.096643in}}%
\pgfpathlineto{\pgfqpoint{1.587082in}{1.091613in}}%
\pgfpathlineto{\pgfqpoint{1.590255in}{1.086693in}}%
\pgfpathlineto{\pgfqpoint{1.593428in}{1.081885in}}%
\pgfpathlineto{\pgfqpoint{1.600152in}{1.075208in}}%
\pgfpathlineto{\pgfqpoint{1.606477in}{1.068419in}}%
\pgfpathlineto{\pgfqpoint{1.612394in}{1.061526in}}%
\pgfpathlineto{\pgfqpoint{1.617896in}{1.054534in}}%
\pgfpathlineto{\pgfqpoint{1.614522in}{1.059571in}}%
\pgfpathlineto{\pgfqpoint{1.611149in}{1.064721in}}%
\pgfpathlineto{\pgfqpoint{1.607775in}{1.069980in}}%
\pgfpathlineto{\pgfqpoint{1.604400in}{1.075345in}}%
\pgfpathlineto{\pgfqpoint{1.599081in}{1.082101in}}%
\pgfpathlineto{\pgfqpoint{1.593359in}{1.088763in}}%
\pgfpathlineto{\pgfqpoint{1.587241in}{1.095324in}}%
\pgfpathlineto{\pgfqpoint{1.580734in}{1.101778in}}%
\pgfpathclose%
\pgfusepath{fill}%
\end{pgfscope}%
\begin{pgfscope}%
\pgfpathrectangle{\pgfqpoint{0.041670in}{0.041670in}}{\pgfqpoint{2.216660in}{2.216660in}}%
\pgfusepath{clip}%
\pgfsetbuttcap%
\pgfsetroundjoin%
\definecolor{currentfill}{rgb}{0.279566,0.067836,0.391917}%
\pgfsetfillcolor{currentfill}%
\pgfsetlinewidth{0.000000pt}%
\definecolor{currentstroke}{rgb}{0.000000,0.000000,0.000000}%
\pgfsetstrokecolor{currentstroke}%
\pgfsetdash{}{0pt}%
\pgfpathmoveto{\pgfqpoint{0.678616in}{0.965291in}}%
\pgfpathlineto{\pgfqpoint{0.675050in}{0.961765in}}%
\pgfpathlineto{\pgfqpoint{0.671482in}{0.958406in}}%
\pgfpathlineto{\pgfqpoint{0.667911in}{0.955218in}}%
\pgfpathlineto{\pgfqpoint{0.664337in}{0.952204in}}%
\pgfpathlineto{\pgfqpoint{0.668212in}{0.960576in}}%
\pgfpathlineto{\pgfqpoint{0.672583in}{0.968872in}}%
\pgfpathlineto{\pgfqpoint{0.677445in}{0.977082in}}%
\pgfpathlineto{\pgfqpoint{0.682790in}{0.985201in}}%
\pgfpathlineto{\pgfqpoint{0.686220in}{0.987973in}}%
\pgfpathlineto{\pgfqpoint{0.689648in}{0.990919in}}%
\pgfpathlineto{\pgfqpoint{0.693073in}{0.994036in}}%
\pgfpathlineto{\pgfqpoint{0.696497in}{0.997318in}}%
\pgfpathlineto{\pgfqpoint{0.691314in}{0.989438in}}%
\pgfpathlineto{\pgfqpoint{0.686602in}{0.981468in}}%
\pgfpathlineto{\pgfqpoint{0.682368in}{0.973416in}}%
\pgfpathlineto{\pgfqpoint{0.678616in}{0.965291in}}%
\pgfpathclose%
\pgfusepath{fill}%
\end{pgfscope}%
\begin{pgfscope}%
\pgfpathrectangle{\pgfqpoint{0.041670in}{0.041670in}}{\pgfqpoint{2.216660in}{2.216660in}}%
\pgfusepath{clip}%
\pgfsetbuttcap%
\pgfsetroundjoin%
\definecolor{currentfill}{rgb}{0.282327,0.094955,0.417331}%
\pgfsetfillcolor{currentfill}%
\pgfsetlinewidth{0.000000pt}%
\definecolor{currentstroke}{rgb}{0.000000,0.000000,0.000000}%
\pgfsetstrokecolor{currentstroke}%
\pgfsetdash{}{0pt}%
\pgfpathmoveto{\pgfqpoint{1.644894in}{1.018745in}}%
\pgfpathlineto{\pgfqpoint{1.648272in}{1.014890in}}%
\pgfpathlineto{\pgfqpoint{1.651652in}{1.011186in}}%
\pgfpathlineto{\pgfqpoint{1.655033in}{1.007635in}}%
\pgfpathlineto{\pgfqpoint{1.658416in}{1.004243in}}%
\pgfpathlineto{\pgfqpoint{1.664012in}{0.996447in}}%
\pgfpathlineto{\pgfqpoint{1.669143in}{0.988556in}}%
\pgfpathlineto{\pgfqpoint{1.673802in}{0.980577in}}%
\pgfpathlineto{\pgfqpoint{1.677983in}{0.972517in}}%
\pgfpathlineto{\pgfqpoint{1.674446in}{0.976150in}}%
\pgfpathlineto{\pgfqpoint{1.670911in}{0.979941in}}%
\pgfpathlineto{\pgfqpoint{1.667378in}{0.983886in}}%
\pgfpathlineto{\pgfqpoint{1.663847in}{0.987982in}}%
\pgfpathlineto{\pgfqpoint{1.659800in}{0.995797in}}%
\pgfpathlineto{\pgfqpoint{1.655288in}{1.003534in}}%
\pgfpathlineto{\pgfqpoint{1.650317in}{1.011185in}}%
\pgfpathlineto{\pgfqpoint{1.644894in}{1.018745in}}%
\pgfpathclose%
\pgfusepath{fill}%
\end{pgfscope}%
\begin{pgfscope}%
\pgfpathrectangle{\pgfqpoint{0.041670in}{0.041670in}}{\pgfqpoint{2.216660in}{2.216660in}}%
\pgfusepath{clip}%
\pgfsetbuttcap%
\pgfsetroundjoin%
\definecolor{currentfill}{rgb}{0.179019,0.433756,0.557430}%
\pgfsetfillcolor{currentfill}%
\pgfsetlinewidth{0.000000pt}%
\definecolor{currentstroke}{rgb}{0.000000,0.000000,0.000000}%
\pgfsetstrokecolor{currentstroke}%
\pgfsetdash{}{0pt}%
\pgfpathmoveto{\pgfqpoint{0.933105in}{1.297836in}}%
\pgfpathlineto{\pgfqpoint{0.930672in}{1.291706in}}%
\pgfpathlineto{\pgfqpoint{0.928241in}{1.285604in}}%
\pgfpathlineto{\pgfqpoint{0.925813in}{1.279533in}}%
\pgfpathlineto{\pgfqpoint{0.923388in}{1.273496in}}%
\pgfpathlineto{\pgfqpoint{0.932248in}{1.277422in}}%
\pgfpathlineto{\pgfqpoint{0.941340in}{1.281206in}}%
\pgfpathlineto{\pgfqpoint{0.950655in}{1.284845in}}%
\pgfpathlineto{\pgfqpoint{0.960184in}{1.288335in}}%
\pgfpathlineto{\pgfqpoint{0.962258in}{1.294219in}}%
\pgfpathlineto{\pgfqpoint{0.964333in}{1.300137in}}%
\pgfpathlineto{\pgfqpoint{0.966411in}{1.306086in}}%
\pgfpathlineto{\pgfqpoint{0.968490in}{1.312063in}}%
\pgfpathlineto{\pgfqpoint{0.959326in}{1.308716in}}%
\pgfpathlineto{\pgfqpoint{0.950368in}{1.305227in}}%
\pgfpathlineto{\pgfqpoint{0.941625in}{1.301599in}}%
\pgfpathlineto{\pgfqpoint{0.933105in}{1.297836in}}%
\pgfpathclose%
\pgfusepath{fill}%
\end{pgfscope}%
\begin{pgfscope}%
\pgfpathrectangle{\pgfqpoint{0.041670in}{0.041670in}}{\pgfqpoint{2.216660in}{2.216660in}}%
\pgfusepath{clip}%
\pgfsetbuttcap%
\pgfsetroundjoin%
\definecolor{currentfill}{rgb}{0.163625,0.471133,0.558148}%
\pgfsetfillcolor{currentfill}%
\pgfsetlinewidth{0.000000pt}%
\definecolor{currentstroke}{rgb}{0.000000,0.000000,0.000000}%
\pgfsetstrokecolor{currentstroke}%
\pgfsetdash{}{0pt}%
\pgfpathmoveto{\pgfqpoint{1.337487in}{1.349806in}}%
\pgfpathlineto{\pgfqpoint{1.339104in}{1.343899in}}%
\pgfpathlineto{\pgfqpoint{1.340719in}{1.338008in}}%
\pgfpathlineto{\pgfqpoint{1.342332in}{1.332136in}}%
\pgfpathlineto{\pgfqpoint{1.343943in}{1.326286in}}%
\pgfpathlineto{\pgfqpoint{1.353986in}{1.323674in}}%
\pgfpathlineto{\pgfqpoint{1.363867in}{1.320906in}}%
\pgfpathlineto{\pgfqpoint{1.373577in}{1.317986in}}%
\pgfpathlineto{\pgfqpoint{1.383107in}{1.314916in}}%
\pgfpathlineto{\pgfqpoint{1.381108in}{1.320887in}}%
\pgfpathlineto{\pgfqpoint{1.379106in}{1.326881in}}%
\pgfpathlineto{\pgfqpoint{1.377102in}{1.332893in}}%
\pgfpathlineto{\pgfqpoint{1.375096in}{1.338922in}}%
\pgfpathlineto{\pgfqpoint{1.365946in}{1.341861in}}%
\pgfpathlineto{\pgfqpoint{1.356621in}{1.344656in}}%
\pgfpathlineto{\pgfqpoint{1.347132in}{1.347305in}}%
\pgfpathlineto{\pgfqpoint{1.337487in}{1.349806in}}%
\pgfpathclose%
\pgfusepath{fill}%
\end{pgfscope}%
\begin{pgfscope}%
\pgfpathrectangle{\pgfqpoint{0.041670in}{0.041670in}}{\pgfqpoint{2.216660in}{2.216660in}}%
\pgfusepath{clip}%
\pgfsetbuttcap%
\pgfsetroundjoin%
\definecolor{currentfill}{rgb}{0.147607,0.511733,0.557049}%
\pgfsetfillcolor{currentfill}%
\pgfsetlinewidth{0.000000pt}%
\definecolor{currentstroke}{rgb}{0.000000,0.000000,0.000000}%
\pgfsetstrokecolor{currentstroke}%
\pgfsetdash{}{0pt}%
\pgfpathmoveto{\pgfqpoint{1.262841in}{1.386142in}}%
\pgfpathlineto{\pgfqpoint{1.263731in}{1.380344in}}%
\pgfpathlineto{\pgfqpoint{1.264619in}{1.374552in}}%
\pgfpathlineto{\pgfqpoint{1.265507in}{1.368767in}}%
\pgfpathlineto{\pgfqpoint{1.266393in}{1.362993in}}%
\pgfpathlineto{\pgfqpoint{1.276867in}{1.361583in}}%
\pgfpathlineto{\pgfqpoint{1.287250in}{1.360013in}}%
\pgfpathlineto{\pgfqpoint{1.297535in}{1.358284in}}%
\pgfpathlineto{\pgfqpoint{1.296328in}{1.364111in}}%
\pgfpathlineto{\pgfqpoint{1.295120in}{1.369948in}}%
\pgfpathlineto{\pgfqpoint{1.293910in}{1.375793in}}%
\pgfpathlineto{\pgfqpoint{1.292699in}{1.381643in}}%
\pgfpathlineto{\pgfqpoint{1.282839in}{1.383295in}}%
\pgfpathlineto{\pgfqpoint{1.272883in}{1.384795in}}%
\pgfpathlineto{\pgfqpoint{1.262841in}{1.386142in}}%
\pgfpathclose%
\pgfusepath{fill}%
\end{pgfscope}%
\begin{pgfscope}%
\pgfpathrectangle{\pgfqpoint{0.041670in}{0.041670in}}{\pgfqpoint{2.216660in}{2.216660in}}%
\pgfusepath{clip}%
\pgfsetbuttcap%
\pgfsetroundjoin%
\definecolor{currentfill}{rgb}{0.172719,0.448791,0.557885}%
\pgfsetfillcolor{currentfill}%
\pgfsetlinewidth{0.000000pt}%
\definecolor{currentstroke}{rgb}{0.000000,0.000000,0.000000}%
\pgfsetstrokecolor{currentstroke}%
\pgfsetdash{}{0pt}%
\pgfpathmoveto{\pgfqpoint{1.832932in}{1.303725in}}%
\pgfpathlineto{\pgfqpoint{1.836647in}{1.319258in}}%
\pgfpathlineto{\pgfqpoint{1.840382in}{1.335289in}}%
\pgfpathlineto{\pgfqpoint{1.844136in}{1.351826in}}%
\pgfpathlineto{\pgfqpoint{1.855328in}{1.341507in}}%
\pgfpathlineto{\pgfqpoint{1.865896in}{1.331003in}}%
\pgfpathlineto{\pgfqpoint{1.875827in}{1.320325in}}%
\pgfpathlineto{\pgfqpoint{1.885107in}{1.309481in}}%
\pgfpathlineto{\pgfqpoint{1.881096in}{1.293082in}}%
\pgfpathlineto{\pgfqpoint{1.877107in}{1.277191in}}%
\pgfpathlineto{\pgfqpoint{1.873139in}{1.261801in}}%
\pgfpathlineto{\pgfqpoint{1.864034in}{1.272536in}}%
\pgfpathlineto{\pgfqpoint{1.854290in}{1.283108in}}%
\pgfpathlineto{\pgfqpoint{1.843919in}{1.293507in}}%
\pgfpathlineto{\pgfqpoint{1.832932in}{1.303725in}}%
\pgfpathclose%
\pgfusepath{fill}%
\end{pgfscope}%
\begin{pgfscope}%
\pgfpathrectangle{\pgfqpoint{0.041670in}{0.041670in}}{\pgfqpoint{2.216660in}{2.216660in}}%
\pgfusepath{clip}%
\pgfsetbuttcap%
\pgfsetroundjoin%
\definecolor{currentfill}{rgb}{0.147607,0.511733,0.557049}%
\pgfsetfillcolor{currentfill}%
\pgfsetlinewidth{0.000000pt}%
\definecolor{currentstroke}{rgb}{0.000000,0.000000,0.000000}%
\pgfsetstrokecolor{currentstroke}%
\pgfsetdash{}{0pt}%
\pgfpathmoveto{\pgfqpoint{1.058533in}{1.380047in}}%
\pgfpathlineto{\pgfqpoint{1.057229in}{1.374179in}}%
\pgfpathlineto{\pgfqpoint{1.055926in}{1.368315in}}%
\pgfpathlineto{\pgfqpoint{1.054625in}{1.362460in}}%
\pgfpathlineto{\pgfqpoint{1.053325in}{1.356615in}}%
\pgfpathlineto{\pgfqpoint{1.063513in}{1.358484in}}%
\pgfpathlineto{\pgfqpoint{1.073808in}{1.360196in}}%
\pgfpathlineto{\pgfqpoint{1.084203in}{1.361748in}}%
\pgfpathlineto{\pgfqpoint{1.094685in}{1.363139in}}%
\pgfpathlineto{\pgfqpoint{1.095560in}{1.368912in}}%
\pgfpathlineto{\pgfqpoint{1.096435in}{1.374695in}}%
\pgfpathlineto{\pgfqpoint{1.097312in}{1.380486in}}%
\pgfpathlineto{\pgfqpoint{1.098189in}{1.386282in}}%
\pgfpathlineto{\pgfqpoint{1.088138in}{1.384953in}}%
\pgfpathlineto{\pgfqpoint{1.078172in}{1.383469in}}%
\pgfpathlineto{\pgfqpoint{1.068301in}{1.381834in}}%
\pgfpathlineto{\pgfqpoint{1.058533in}{1.380047in}}%
\pgfpathclose%
\pgfusepath{fill}%
\end{pgfscope}%
\begin{pgfscope}%
\pgfpathrectangle{\pgfqpoint{0.041670in}{0.041670in}}{\pgfqpoint{2.216660in}{2.216660in}}%
\pgfusepath{clip}%
\pgfsetbuttcap%
\pgfsetroundjoin%
\definecolor{currentfill}{rgb}{0.195860,0.395433,0.555276}%
\pgfsetfillcolor{currentfill}%
\pgfsetlinewidth{0.000000pt}%
\definecolor{currentstroke}{rgb}{0.000000,0.000000,0.000000}%
\pgfsetstrokecolor{currentstroke}%
\pgfsetdash{}{0pt}%
\pgfpathmoveto{\pgfqpoint{0.890435in}{1.256445in}}%
\pgfpathlineto{\pgfqpoint{0.887694in}{1.250268in}}%
\pgfpathlineto{\pgfqpoint{0.884954in}{1.244131in}}%
\pgfpathlineto{\pgfqpoint{0.882217in}{1.238037in}}%
\pgfpathlineto{\pgfqpoint{0.879483in}{1.231989in}}%
\pgfpathlineto{\pgfqpoint{0.887635in}{1.236632in}}%
\pgfpathlineto{\pgfqpoint{0.896062in}{1.241142in}}%
\pgfpathlineto{\pgfqpoint{0.904755in}{1.245515in}}%
\pgfpathlineto{\pgfqpoint{0.913707in}{1.249748in}}%
\pgfpathlineto{\pgfqpoint{0.916124in}{1.255619in}}%
\pgfpathlineto{\pgfqpoint{0.918543in}{1.261536in}}%
\pgfpathlineto{\pgfqpoint{0.920964in}{1.267496in}}%
\pgfpathlineto{\pgfqpoint{0.923388in}{1.273496in}}%
\pgfpathlineto{\pgfqpoint{0.914768in}{1.269432in}}%
\pgfpathlineto{\pgfqpoint{0.906397in}{1.265233in}}%
\pgfpathlineto{\pgfqpoint{0.898283in}{1.260902in}}%
\pgfpathlineto{\pgfqpoint{0.890435in}{1.256445in}}%
\pgfpathclose%
\pgfusepath{fill}%
\end{pgfscope}%
\begin{pgfscope}%
\pgfpathrectangle{\pgfqpoint{0.041670in}{0.041670in}}{\pgfqpoint{2.216660in}{2.216660in}}%
\pgfusepath{clip}%
\pgfsetbuttcap%
\pgfsetroundjoin%
\definecolor{currentfill}{rgb}{0.212395,0.359683,0.551710}%
\pgfsetfillcolor{currentfill}%
\pgfsetlinewidth{0.000000pt}%
\definecolor{currentstroke}{rgb}{0.000000,0.000000,0.000000}%
\pgfsetstrokecolor{currentstroke}%
\pgfsetdash{}{0pt}%
\pgfpathmoveto{\pgfqpoint{1.473194in}{1.236123in}}%
\pgfpathlineto{\pgfqpoint{1.475859in}{1.230165in}}%
\pgfpathlineto{\pgfqpoint{1.478522in}{1.224260in}}%
\pgfpathlineto{\pgfqpoint{1.481183in}{1.218411in}}%
\pgfpathlineto{\pgfqpoint{1.483842in}{1.212620in}}%
\pgfpathlineto{\pgfqpoint{1.492266in}{1.207777in}}%
\pgfpathlineto{\pgfqpoint{1.500395in}{1.202800in}}%
\pgfpathlineto{\pgfqpoint{1.508223in}{1.197694in}}%
\pgfpathlineto{\pgfqpoint{1.515741in}{1.192463in}}%
\pgfpathlineto{\pgfqpoint{1.512794in}{1.198448in}}%
\pgfpathlineto{\pgfqpoint{1.509846in}{1.204491in}}%
\pgfpathlineto{\pgfqpoint{1.506895in}{1.210590in}}%
\pgfpathlineto{\pgfqpoint{1.503942in}{1.216741in}}%
\pgfpathlineto{\pgfqpoint{1.496697in}{1.221770in}}%
\pgfpathlineto{\pgfqpoint{1.489152in}{1.226680in}}%
\pgfpathlineto{\pgfqpoint{1.481315in}{1.231466in}}%
\pgfpathlineto{\pgfqpoint{1.473194in}{1.236123in}}%
\pgfpathclose%
\pgfusepath{fill}%
\end{pgfscope}%
\begin{pgfscope}%
\pgfpathrectangle{\pgfqpoint{0.041670in}{0.041670in}}{\pgfqpoint{2.216660in}{2.216660in}}%
\pgfusepath{clip}%
\pgfsetbuttcap%
\pgfsetroundjoin%
\definecolor{currentfill}{rgb}{0.276194,0.190074,0.493001}%
\pgfsetfillcolor{currentfill}%
\pgfsetlinewidth{0.000000pt}%
\definecolor{currentstroke}{rgb}{0.000000,0.000000,0.000000}%
\pgfsetstrokecolor{currentstroke}%
\pgfsetdash{}{0pt}%
\pgfpathmoveto{\pgfqpoint{0.516388in}{1.026360in}}%
\pgfpathlineto{\pgfqpoint{0.512482in}{1.034523in}}%
\pgfpathlineto{\pgfqpoint{0.508560in}{1.043074in}}%
\pgfpathlineto{\pgfqpoint{0.504623in}{1.052019in}}%
\pgfpathlineto{\pgfqpoint{0.500670in}{1.061365in}}%
\pgfpathlineto{\pgfqpoint{0.506025in}{1.072222in}}%
\pgfpathlineto{\pgfqpoint{0.512028in}{1.082971in}}%
\pgfpathlineto{\pgfqpoint{0.518669in}{1.093603in}}%
\pgfpathlineto{\pgfqpoint{0.525939in}{1.104108in}}%
\pgfpathlineto{\pgfqpoint{0.529722in}{1.094572in}}%
\pgfpathlineto{\pgfqpoint{0.533490in}{1.085434in}}%
\pgfpathlineto{\pgfqpoint{0.537244in}{1.076688in}}%
\pgfpathlineto{\pgfqpoint{0.540983in}{1.068328in}}%
\pgfpathlineto{\pgfqpoint{0.533902in}{1.058012in}}%
\pgfpathlineto{\pgfqpoint{0.527437in}{1.047573in}}%
\pgfpathlineto{\pgfqpoint{0.521596in}{1.037019in}}%
\pgfpathlineto{\pgfqpoint{0.516388in}{1.026360in}}%
\pgfpathclose%
\pgfusepath{fill}%
\end{pgfscope}%
\begin{pgfscope}%
\pgfpathrectangle{\pgfqpoint{0.041670in}{0.041670in}}{\pgfqpoint{2.216660in}{2.216660in}}%
\pgfusepath{clip}%
\pgfsetbuttcap%
\pgfsetroundjoin%
\definecolor{currentfill}{rgb}{0.248629,0.278775,0.534556}%
\pgfsetfillcolor{currentfill}%
\pgfsetlinewidth{0.000000pt}%
\definecolor{currentstroke}{rgb}{0.000000,0.000000,0.000000}%
\pgfsetstrokecolor{currentstroke}%
\pgfsetdash{}{0pt}%
\pgfpathmoveto{\pgfqpoint{0.799190in}{1.140915in}}%
\pgfpathlineto{\pgfqpoint{0.795958in}{1.135009in}}%
\pgfpathlineto{\pgfqpoint{0.792728in}{1.129181in}}%
\pgfpathlineto{\pgfqpoint{0.789499in}{1.123435in}}%
\pgfpathlineto{\pgfqpoint{0.786272in}{1.117773in}}%
\pgfpathlineto{\pgfqpoint{0.792602in}{1.123992in}}%
\pgfpathlineto{\pgfqpoint{0.799302in}{1.130101in}}%
\pgfpathlineto{\pgfqpoint{0.806363in}{1.136095in}}%
\pgfpathlineto{\pgfqpoint{0.813778in}{1.141969in}}%
\pgfpathlineto{\pgfqpoint{0.816770in}{1.147413in}}%
\pgfpathlineto{\pgfqpoint{0.819763in}{1.152943in}}%
\pgfpathlineto{\pgfqpoint{0.822757in}{1.158553in}}%
\pgfpathlineto{\pgfqpoint{0.825753in}{1.164242in}}%
\pgfpathlineto{\pgfqpoint{0.818590in}{1.158578in}}%
\pgfpathlineto{\pgfqpoint{0.811770in}{1.152799in}}%
\pgfpathlineto{\pgfqpoint{0.805301in}{1.146909in}}%
\pgfpathlineto{\pgfqpoint{0.799190in}{1.140915in}}%
\pgfpathclose%
\pgfusepath{fill}%
\end{pgfscope}%
\begin{pgfscope}%
\pgfpathrectangle{\pgfqpoint{0.041670in}{0.041670in}}{\pgfqpoint{2.216660in}{2.216660in}}%
\pgfusepath{clip}%
\pgfsetbuttcap%
\pgfsetroundjoin%
\definecolor{currentfill}{rgb}{0.260571,0.246922,0.522828}%
\pgfsetfillcolor{currentfill}%
\pgfsetlinewidth{0.000000pt}%
\definecolor{currentstroke}{rgb}{0.000000,0.000000,0.000000}%
\pgfsetstrokecolor{currentstroke}%
\pgfsetdash{}{0pt}%
\pgfpathmoveto{\pgfqpoint{1.826988in}{1.113331in}}%
\pgfpathlineto{\pgfqpoint{1.830740in}{1.123313in}}%
\pgfpathlineto{\pgfqpoint{1.834508in}{1.133705in}}%
\pgfpathlineto{\pgfqpoint{1.838292in}{1.144516in}}%
\pgfpathlineto{\pgfqpoint{1.842093in}{1.155752in}}%
\pgfpathlineto{\pgfqpoint{1.850120in}{1.145190in}}%
\pgfpathlineto{\pgfqpoint{1.857513in}{1.134489in}}%
\pgfpathlineto{\pgfqpoint{1.864262in}{1.123659in}}%
\pgfpathlineto{\pgfqpoint{1.870357in}{1.112710in}}%
\pgfpathlineto{\pgfqpoint{1.866370in}{1.101654in}}%
\pgfpathlineto{\pgfqpoint{1.862400in}{1.091025in}}%
\pgfpathlineto{\pgfqpoint{1.858448in}{1.080816in}}%
\pgfpathlineto{\pgfqpoint{1.854511in}{1.071021in}}%
\pgfpathlineto{\pgfqpoint{1.848580in}{1.081783in}}%
\pgfpathlineto{\pgfqpoint{1.842010in}{1.092428in}}%
\pgfpathlineto{\pgfqpoint{1.834809in}{1.102947in}}%
\pgfpathlineto{\pgfqpoint{1.826988in}{1.113331in}}%
\pgfpathclose%
\pgfusepath{fill}%
\end{pgfscope}%
\begin{pgfscope}%
\pgfpathrectangle{\pgfqpoint{0.041670in}{0.041670in}}{\pgfqpoint{2.216660in}{2.216660in}}%
\pgfusepath{clip}%
\pgfsetbuttcap%
\pgfsetroundjoin%
\definecolor{currentfill}{rgb}{0.163625,0.471133,0.558148}%
\pgfsetfillcolor{currentfill}%
\pgfsetlinewidth{0.000000pt}%
\definecolor{currentstroke}{rgb}{0.000000,0.000000,0.000000}%
\pgfsetstrokecolor{currentstroke}%
\pgfsetdash{}{0pt}%
\pgfpathmoveto{\pgfqpoint{0.976832in}{1.336191in}}%
\pgfpathlineto{\pgfqpoint{0.974743in}{1.330132in}}%
\pgfpathlineto{\pgfqpoint{0.972657in}{1.324089in}}%
\pgfpathlineto{\pgfqpoint{0.970572in}{1.318065in}}%
\pgfpathlineto{\pgfqpoint{0.968490in}{1.312063in}}%
\pgfpathlineto{\pgfqpoint{0.977852in}{1.315264in}}%
\pgfpathlineto{\pgfqpoint{0.987403in}{1.318318in}}%
\pgfpathlineto{\pgfqpoint{0.997132in}{1.321222in}}%
\pgfpathlineto{\pgfqpoint{1.007032in}{1.323972in}}%
\pgfpathlineto{\pgfqpoint{1.008732in}{1.329846in}}%
\pgfpathlineto{\pgfqpoint{1.010433in}{1.335743in}}%
\pgfpathlineto{\pgfqpoint{1.012137in}{1.341658in}}%
\pgfpathlineto{\pgfqpoint{1.013842in}{1.347590in}}%
\pgfpathlineto{\pgfqpoint{1.004335in}{1.344958in}}%
\pgfpathlineto{\pgfqpoint{0.994992in}{1.342179in}}%
\pgfpathlineto{\pgfqpoint{0.985821in}{1.339255in}}%
\pgfpathlineto{\pgfqpoint{0.976832in}{1.336191in}}%
\pgfpathclose%
\pgfusepath{fill}%
\end{pgfscope}%
\begin{pgfscope}%
\pgfpathrectangle{\pgfqpoint{0.041670in}{0.041670in}}{\pgfqpoint{2.216660in}{2.216660in}}%
\pgfusepath{clip}%
\pgfsetbuttcap%
\pgfsetroundjoin%
\definecolor{currentfill}{rgb}{0.274128,0.199721,0.498911}%
\pgfsetfillcolor{currentfill}%
\pgfsetlinewidth{0.000000pt}%
\definecolor{currentstroke}{rgb}{0.000000,0.000000,0.000000}%
\pgfsetstrokecolor{currentstroke}%
\pgfsetdash{}{0pt}%
\pgfpathmoveto{\pgfqpoint{0.751123in}{1.069264in}}%
\pgfpathlineto{\pgfqpoint{0.747711in}{1.063846in}}%
\pgfpathlineto{\pgfqpoint{0.744299in}{1.058534in}}%
\pgfpathlineto{\pgfqpoint{0.740888in}{1.053331in}}%
\pgfpathlineto{\pgfqpoint{0.737476in}{1.048241in}}%
\pgfpathlineto{\pgfqpoint{0.742605in}{1.055316in}}%
\pgfpathlineto{\pgfqpoint{0.748153in}{1.062297in}}%
\pgfpathlineto{\pgfqpoint{0.754116in}{1.069179in}}%
\pgfpathlineto{\pgfqpoint{0.760485in}{1.075955in}}%
\pgfpathlineto{\pgfqpoint{0.763706in}{1.080813in}}%
\pgfpathlineto{\pgfqpoint{0.766928in}{1.085782in}}%
\pgfpathlineto{\pgfqpoint{0.770150in}{1.090862in}}%
\pgfpathlineto{\pgfqpoint{0.773373in}{1.096047in}}%
\pgfpathlineto{\pgfqpoint{0.767212in}{1.089497in}}%
\pgfpathlineto{\pgfqpoint{0.761445in}{1.082846in}}%
\pgfpathlineto{\pgfqpoint{0.756080in}{1.076100in}}%
\pgfpathlineto{\pgfqpoint{0.751123in}{1.069264in}}%
\pgfpathclose%
\pgfusepath{fill}%
\end{pgfscope}%
\begin{pgfscope}%
\pgfpathrectangle{\pgfqpoint{0.041670in}{0.041670in}}{\pgfqpoint{2.216660in}{2.216660in}}%
\pgfusepath{clip}%
\pgfsetbuttcap%
\pgfsetroundjoin%
\definecolor{currentfill}{rgb}{0.282327,0.094955,0.417331}%
\pgfsetfillcolor{currentfill}%
\pgfsetlinewidth{0.000000pt}%
\definecolor{currentstroke}{rgb}{0.000000,0.000000,0.000000}%
\pgfsetstrokecolor{currentstroke}%
\pgfsetdash{}{0pt}%
\pgfpathmoveto{\pgfqpoint{0.692860in}{0.980976in}}%
\pgfpathlineto{\pgfqpoint{0.689302in}{0.976825in}}%
\pgfpathlineto{\pgfqpoint{0.685742in}{0.972824in}}%
\pgfpathlineto{\pgfqpoint{0.682180in}{0.968978in}}%
\pgfpathlineto{\pgfqpoint{0.678616in}{0.965291in}}%
\pgfpathlineto{\pgfqpoint{0.682368in}{0.973416in}}%
\pgfpathlineto{\pgfqpoint{0.686602in}{0.981468in}}%
\pgfpathlineto{\pgfqpoint{0.691314in}{0.989438in}}%
\pgfpathlineto{\pgfqpoint{0.696497in}{0.997318in}}%
\pgfpathlineto{\pgfqpoint{0.699918in}{1.000763in}}%
\pgfpathlineto{\pgfqpoint{0.703338in}{1.004366in}}%
\pgfpathlineto{\pgfqpoint{0.706756in}{1.008123in}}%
\pgfpathlineto{\pgfqpoint{0.710173in}{1.012030in}}%
\pgfpathlineto{\pgfqpoint{0.705151in}{1.004388in}}%
\pgfpathlineto{\pgfqpoint{0.700588in}{0.996660in}}%
\pgfpathlineto{\pgfqpoint{0.696489in}{0.988854in}}%
\pgfpathlineto{\pgfqpoint{0.692860in}{0.980976in}}%
\pgfpathclose%
\pgfusepath{fill}%
\end{pgfscope}%
\begin{pgfscope}%
\pgfpathrectangle{\pgfqpoint{0.041670in}{0.041670in}}{\pgfqpoint{2.216660in}{2.216660in}}%
\pgfusepath{clip}%
\pgfsetbuttcap%
\pgfsetroundjoin%
\definecolor{currentfill}{rgb}{0.212395,0.359683,0.551710}%
\pgfsetfillcolor{currentfill}%
\pgfsetlinewidth{0.000000pt}%
\definecolor{currentstroke}{rgb}{0.000000,0.000000,0.000000}%
\pgfsetstrokecolor{currentstroke}%
\pgfsetdash{}{0pt}%
\pgfpathmoveto{\pgfqpoint{0.849785in}{1.212174in}}%
\pgfpathlineto{\pgfqpoint{0.846774in}{1.205977in}}%
\pgfpathlineto{\pgfqpoint{0.843765in}{1.199833in}}%
\pgfpathlineto{\pgfqpoint{0.840758in}{1.193743in}}%
\pgfpathlineto{\pgfqpoint{0.837753in}{1.187713in}}%
\pgfpathlineto{\pgfqpoint{0.844989in}{1.193051in}}%
\pgfpathlineto{\pgfqpoint{0.852541in}{1.198268in}}%
\pgfpathlineto{\pgfqpoint{0.860403in}{1.203359in}}%
\pgfpathlineto{\pgfqpoint{0.868566in}{1.208321in}}%
\pgfpathlineto{\pgfqpoint{0.871292in}{1.214154in}}%
\pgfpathlineto{\pgfqpoint{0.874020in}{1.220044in}}%
\pgfpathlineto{\pgfqpoint{0.876750in}{1.225991in}}%
\pgfpathlineto{\pgfqpoint{0.879483in}{1.231989in}}%
\pgfpathlineto{\pgfqpoint{0.871614in}{1.227218in}}%
\pgfpathlineto{\pgfqpoint{0.864036in}{1.222322in}}%
\pgfpathlineto{\pgfqpoint{0.856757in}{1.217306in}}%
\pgfpathlineto{\pgfqpoint{0.849785in}{1.212174in}}%
\pgfpathclose%
\pgfusepath{fill}%
\end{pgfscope}%
\begin{pgfscope}%
\pgfpathrectangle{\pgfqpoint{0.041670in}{0.041670in}}{\pgfqpoint{2.216660in}{2.216660in}}%
\pgfusepath{clip}%
\pgfsetbuttcap%
\pgfsetroundjoin%
\definecolor{currentfill}{rgb}{0.267004,0.004874,0.329415}%
\pgfsetfillcolor{currentfill}%
\pgfsetlinewidth{0.000000pt}%
\definecolor{currentstroke}{rgb}{0.000000,0.000000,0.000000}%
\pgfsetstrokecolor{currentstroke}%
\pgfsetdash{}{0pt}%
\pgfpathmoveto{\pgfqpoint{1.749433in}{0.939475in}}%
\pgfpathlineto{\pgfqpoint{1.753058in}{0.940138in}}%
\pgfpathlineto{\pgfqpoint{1.756690in}{0.941054in}}%
\pgfpathlineto{\pgfqpoint{1.760329in}{0.942229in}}%
\pgfpathlineto{\pgfqpoint{1.763976in}{0.943668in}}%
\pgfpathlineto{\pgfqpoint{1.768427in}{0.934089in}}%
\pgfpathlineto{\pgfqpoint{1.772308in}{0.924430in}}%
\pgfpathlineto{\pgfqpoint{1.775612in}{0.914699in}}%
\pgfpathlineto{\pgfqpoint{1.778333in}{0.904905in}}%
\pgfpathlineto{\pgfqpoint{1.774577in}{0.903702in}}%
\pgfpathlineto{\pgfqpoint{1.770829in}{0.902763in}}%
\pgfpathlineto{\pgfqpoint{1.767088in}{0.902084in}}%
\pgfpathlineto{\pgfqpoint{1.763354in}{0.901661in}}%
\pgfpathlineto{\pgfqpoint{1.760721in}{0.911214in}}%
\pgfpathlineto{\pgfqpoint{1.757519in}{0.920707in}}%
\pgfpathlineto{\pgfqpoint{1.753754in}{0.930130in}}%
\pgfpathlineto{\pgfqpoint{1.749433in}{0.939475in}}%
\pgfpathclose%
\pgfusepath{fill}%
\end{pgfscope}%
\begin{pgfscope}%
\pgfpathrectangle{\pgfqpoint{0.041670in}{0.041670in}}{\pgfqpoint{2.216660in}{2.216660in}}%
\pgfusepath{clip}%
\pgfsetbuttcap%
\pgfsetroundjoin%
\definecolor{currentfill}{rgb}{0.283072,0.130895,0.449241}%
\pgfsetfillcolor{currentfill}%
\pgfsetlinewidth{0.000000pt}%
\definecolor{currentstroke}{rgb}{0.000000,0.000000,0.000000}%
\pgfsetstrokecolor{currentstroke}%
\pgfsetdash{}{0pt}%
\pgfpathmoveto{\pgfqpoint{1.631391in}{1.035589in}}%
\pgfpathlineto{\pgfqpoint{1.634765in}{1.031172in}}%
\pgfpathlineto{\pgfqpoint{1.638141in}{1.026889in}}%
\pgfpathlineto{\pgfqpoint{1.641517in}{1.022746in}}%
\pgfpathlineto{\pgfqpoint{1.644894in}{1.018745in}}%
\pgfpathlineto{\pgfqpoint{1.650317in}{1.011185in}}%
\pgfpathlineto{\pgfqpoint{1.655288in}{1.003534in}}%
\pgfpathlineto{\pgfqpoint{1.659800in}{0.995797in}}%
\pgfpathlineto{\pgfqpoint{1.663847in}{0.987982in}}%
\pgfpathlineto{\pgfqpoint{1.660317in}{0.992225in}}%
\pgfpathlineto{\pgfqpoint{1.656788in}{0.996610in}}%
\pgfpathlineto{\pgfqpoint{1.653261in}{1.001135in}}%
\pgfpathlineto{\pgfqpoint{1.649734in}{1.005795in}}%
\pgfpathlineto{\pgfqpoint{1.645820in}{1.013363in}}%
\pgfpathlineto{\pgfqpoint{1.641454in}{1.020856in}}%
\pgfpathlineto{\pgfqpoint{1.636642in}{1.028267in}}%
\pgfpathlineto{\pgfqpoint{1.631391in}{1.035589in}}%
\pgfpathclose%
\pgfusepath{fill}%
\end{pgfscope}%
\begin{pgfscope}%
\pgfpathrectangle{\pgfqpoint{0.041670in}{0.041670in}}{\pgfqpoint{2.216660in}{2.216660in}}%
\pgfusepath{clip}%
\pgfsetbuttcap%
\pgfsetroundjoin%
\definecolor{currentfill}{rgb}{0.147607,0.511733,0.557049}%
\pgfsetfillcolor{currentfill}%
\pgfsetlinewidth{0.000000pt}%
\definecolor{currentstroke}{rgb}{0.000000,0.000000,0.000000}%
\pgfsetstrokecolor{currentstroke}%
\pgfsetdash{}{0pt}%
\pgfpathmoveto{\pgfqpoint{1.292699in}{1.381643in}}%
\pgfpathlineto{\pgfqpoint{1.293910in}{1.375793in}}%
\pgfpathlineto{\pgfqpoint{1.295120in}{1.369948in}}%
\pgfpathlineto{\pgfqpoint{1.296328in}{1.364111in}}%
\pgfpathlineto{\pgfqpoint{1.297535in}{1.358284in}}%
\pgfpathlineto{\pgfqpoint{1.307710in}{1.356397in}}%
\pgfpathlineto{\pgfqpoint{1.317767in}{1.354354in}}%
\pgfpathlineto{\pgfqpoint{1.327696in}{1.352156in}}%
\pgfpathlineto{\pgfqpoint{1.337487in}{1.349806in}}%
\pgfpathlineto{\pgfqpoint{1.335868in}{1.355727in}}%
\pgfpathlineto{\pgfqpoint{1.334248in}{1.361658in}}%
\pgfpathlineto{\pgfqpoint{1.332625in}{1.367598in}}%
\pgfpathlineto{\pgfqpoint{1.331001in}{1.373542in}}%
\pgfpathlineto{\pgfqpoint{1.321615in}{1.375787in}}%
\pgfpathlineto{\pgfqpoint{1.312096in}{1.377887in}}%
\pgfpathlineto{\pgfqpoint{1.302455in}{1.379840in}}%
\pgfpathlineto{\pgfqpoint{1.292699in}{1.381643in}}%
\pgfpathclose%
\pgfusepath{fill}%
\end{pgfscope}%
\begin{pgfscope}%
\pgfpathrectangle{\pgfqpoint{0.041670in}{0.041670in}}{\pgfqpoint{2.216660in}{2.216660in}}%
\pgfusepath{clip}%
\pgfsetbuttcap%
\pgfsetroundjoin%
\definecolor{currentfill}{rgb}{0.268510,0.009605,0.335427}%
\pgfsetfillcolor{currentfill}%
\pgfsetlinewidth{0.000000pt}%
\definecolor{currentstroke}{rgb}{0.000000,0.000000,0.000000}%
\pgfsetstrokecolor{currentstroke}%
\pgfsetdash{}{0pt}%
\pgfpathmoveto{\pgfqpoint{1.763976in}{0.943668in}}%
\pgfpathlineto{\pgfqpoint{1.767630in}{0.945375in}}%
\pgfpathlineto{\pgfqpoint{1.771292in}{0.947356in}}%
\pgfpathlineto{\pgfqpoint{1.774962in}{0.949616in}}%
\pgfpathlineto{\pgfqpoint{1.778641in}{0.952159in}}%
\pgfpathlineto{\pgfqpoint{1.783225in}{0.942351in}}%
\pgfpathlineto{\pgfqpoint{1.787224in}{0.932460in}}%
\pgfpathlineto{\pgfqpoint{1.790632in}{0.922495in}}%
\pgfpathlineto{\pgfqpoint{1.793443in}{0.912464in}}%
\pgfpathlineto{\pgfqpoint{1.789652in}{0.910152in}}%
\pgfpathlineto{\pgfqpoint{1.785871in}{0.908125in}}%
\pgfpathlineto{\pgfqpoint{1.782098in}{0.906378in}}%
\pgfpathlineto{\pgfqpoint{1.778333in}{0.904905in}}%
\pgfpathlineto{\pgfqpoint{1.775612in}{0.914699in}}%
\pgfpathlineto{\pgfqpoint{1.772308in}{0.924430in}}%
\pgfpathlineto{\pgfqpoint{1.768427in}{0.934089in}}%
\pgfpathlineto{\pgfqpoint{1.763976in}{0.943668in}}%
\pgfpathclose%
\pgfusepath{fill}%
\end{pgfscope}%
\begin{pgfscope}%
\pgfpathrectangle{\pgfqpoint{0.041670in}{0.041670in}}{\pgfqpoint{2.216660in}{2.216660in}}%
\pgfusepath{clip}%
\pgfsetbuttcap%
\pgfsetroundjoin%
\definecolor{currentfill}{rgb}{0.267004,0.004874,0.329415}%
\pgfsetfillcolor{currentfill}%
\pgfsetlinewidth{0.000000pt}%
\definecolor{currentstroke}{rgb}{0.000000,0.000000,0.000000}%
\pgfsetstrokecolor{currentstroke}%
\pgfsetdash{}{0pt}%
\pgfpathmoveto{\pgfqpoint{1.734995in}{0.939270in}}%
\pgfpathlineto{\pgfqpoint{1.738595in}{0.938964in}}%
\pgfpathlineto{\pgfqpoint{1.742201in}{0.938893in}}%
\pgfpathlineto{\pgfqpoint{1.745814in}{0.939062in}}%
\pgfpathlineto{\pgfqpoint{1.749433in}{0.939475in}}%
\pgfpathlineto{\pgfqpoint{1.753754in}{0.930130in}}%
\pgfpathlineto{\pgfqpoint{1.757519in}{0.920707in}}%
\pgfpathlineto{\pgfqpoint{1.760721in}{0.911214in}}%
\pgfpathlineto{\pgfqpoint{1.763354in}{0.901661in}}%
\pgfpathlineto{\pgfqpoint{1.759628in}{0.901487in}}%
\pgfpathlineto{\pgfqpoint{1.755908in}{0.901558in}}%
\pgfpathlineto{\pgfqpoint{1.752194in}{0.901870in}}%
\pgfpathlineto{\pgfqpoint{1.748487in}{0.902418in}}%
\pgfpathlineto{\pgfqpoint{1.745940in}{0.911727in}}%
\pgfpathlineto{\pgfqpoint{1.742839in}{0.920978in}}%
\pgfpathlineto{\pgfqpoint{1.739188in}{0.930162in}}%
\pgfpathlineto{\pgfqpoint{1.734995in}{0.939270in}}%
\pgfpathclose%
\pgfusepath{fill}%
\end{pgfscope}%
\begin{pgfscope}%
\pgfpathrectangle{\pgfqpoint{0.041670in}{0.041670in}}{\pgfqpoint{2.216660in}{2.216660in}}%
\pgfusepath{clip}%
\pgfsetbuttcap%
\pgfsetroundjoin%
\definecolor{currentfill}{rgb}{0.147607,0.511733,0.557049}%
\pgfsetfillcolor{currentfill}%
\pgfsetlinewidth{0.000000pt}%
\definecolor{currentstroke}{rgb}{0.000000,0.000000,0.000000}%
\pgfsetstrokecolor{currentstroke}%
\pgfsetdash{}{0pt}%
\pgfpathmoveto{\pgfqpoint{1.020683in}{1.371425in}}%
\pgfpathlineto{\pgfqpoint{1.018970in}{1.365456in}}%
\pgfpathlineto{\pgfqpoint{1.017259in}{1.359492in}}%
\pgfpathlineto{\pgfqpoint{1.015549in}{1.353536in}}%
\pgfpathlineto{\pgfqpoint{1.013842in}{1.347590in}}%
\pgfpathlineto{\pgfqpoint{1.023504in}{1.350075in}}%
\pgfpathlineto{\pgfqpoint{1.033311in}{1.352408in}}%
\pgfpathlineto{\pgfqpoint{1.043254in}{1.354589in}}%
\pgfpathlineto{\pgfqpoint{1.053325in}{1.356615in}}%
\pgfpathlineto{\pgfqpoint{1.054625in}{1.362460in}}%
\pgfpathlineto{\pgfqpoint{1.055926in}{1.368315in}}%
\pgfpathlineto{\pgfqpoint{1.057229in}{1.374179in}}%
\pgfpathlineto{\pgfqpoint{1.058533in}{1.380047in}}%
\pgfpathlineto{\pgfqpoint{1.048879in}{1.378112in}}%
\pgfpathlineto{\pgfqpoint{1.039346in}{1.376028in}}%
\pgfpathlineto{\pgfqpoint{1.029945in}{1.373799in}}%
\pgfpathlineto{\pgfqpoint{1.020683in}{1.371425in}}%
\pgfpathclose%
\pgfusepath{fill}%
\end{pgfscope}%
\begin{pgfscope}%
\pgfpathrectangle{\pgfqpoint{0.041670in}{0.041670in}}{\pgfqpoint{2.216660in}{2.216660in}}%
\pgfusepath{clip}%
\pgfsetbuttcap%
\pgfsetroundjoin%
\definecolor{currentfill}{rgb}{0.263663,0.237631,0.518762}%
\pgfsetfillcolor{currentfill}%
\pgfsetlinewidth{0.000000pt}%
\definecolor{currentstroke}{rgb}{0.000000,0.000000,0.000000}%
\pgfsetstrokecolor{currentstroke}%
\pgfsetdash{}{0pt}%
\pgfpathmoveto{\pgfqpoint{1.568029in}{1.123306in}}%
\pgfpathlineto{\pgfqpoint{1.571207in}{1.117783in}}%
\pgfpathlineto{\pgfqpoint{1.574384in}{1.112352in}}%
\pgfpathlineto{\pgfqpoint{1.577559in}{1.107015in}}%
\pgfpathlineto{\pgfqpoint{1.580734in}{1.101778in}}%
\pgfpathlineto{\pgfqpoint{1.587241in}{1.095324in}}%
\pgfpathlineto{\pgfqpoint{1.593359in}{1.088763in}}%
\pgfpathlineto{\pgfqpoint{1.599081in}{1.082101in}}%
\pgfpathlineto{\pgfqpoint{1.604400in}{1.075345in}}%
\pgfpathlineto{\pgfqpoint{1.601025in}{1.080812in}}%
\pgfpathlineto{\pgfqpoint{1.597650in}{1.086378in}}%
\pgfpathlineto{\pgfqpoint{1.594273in}{1.092039in}}%
\pgfpathlineto{\pgfqpoint{1.590896in}{1.097792in}}%
\pgfpathlineto{\pgfqpoint{1.585758in}{1.104313in}}%
\pgfpathlineto{\pgfqpoint{1.580230in}{1.110743in}}%
\pgfpathlineto{\pgfqpoint{1.574318in}{1.117076in}}%
\pgfpathlineto{\pgfqpoint{1.568029in}{1.123306in}}%
\pgfpathclose%
\pgfusepath{fill}%
\end{pgfscope}%
\begin{pgfscope}%
\pgfpathrectangle{\pgfqpoint{0.041670in}{0.041670in}}{\pgfqpoint{2.216660in}{2.216660in}}%
\pgfusepath{clip}%
\pgfsetbuttcap%
\pgfsetroundjoin%
\definecolor{currentfill}{rgb}{0.272594,0.025563,0.353093}%
\pgfsetfillcolor{currentfill}%
\pgfsetlinewidth{0.000000pt}%
\definecolor{currentstroke}{rgb}{0.000000,0.000000,0.000000}%
\pgfsetstrokecolor{currentstroke}%
\pgfsetdash{}{0pt}%
\pgfpathmoveto{\pgfqpoint{1.778641in}{0.952159in}}%
\pgfpathlineto{\pgfqpoint{1.782329in}{0.954991in}}%
\pgfpathlineto{\pgfqpoint{1.786025in}{0.958117in}}%
\pgfpathlineto{\pgfqpoint{1.789732in}{0.961543in}}%
\pgfpathlineto{\pgfqpoint{1.793448in}{0.965273in}}%
\pgfpathlineto{\pgfqpoint{1.798166in}{0.955241in}}%
\pgfpathlineto{\pgfqpoint{1.802285in}{0.945123in}}%
\pgfpathlineto{\pgfqpoint{1.805799in}{0.934928in}}%
\pgfpathlineto{\pgfqpoint{1.808702in}{0.924666in}}%
\pgfpathlineto{\pgfqpoint{1.804872in}{0.921162in}}%
\pgfpathlineto{\pgfqpoint{1.801052in}{0.917964in}}%
\pgfpathlineto{\pgfqpoint{1.797243in}{0.915067in}}%
\pgfpathlineto{\pgfqpoint{1.793443in}{0.912464in}}%
\pgfpathlineto{\pgfqpoint{1.790632in}{0.922495in}}%
\pgfpathlineto{\pgfqpoint{1.787224in}{0.932460in}}%
\pgfpathlineto{\pgfqpoint{1.783225in}{0.942351in}}%
\pgfpathlineto{\pgfqpoint{1.778641in}{0.952159in}}%
\pgfpathclose%
\pgfusepath{fill}%
\end{pgfscope}%
\begin{pgfscope}%
\pgfpathrectangle{\pgfqpoint{0.041670in}{0.041670in}}{\pgfqpoint{2.216660in}{2.216660in}}%
\pgfusepath{clip}%
\pgfsetbuttcap%
\pgfsetroundjoin%
\definecolor{currentfill}{rgb}{0.231674,0.318106,0.544834}%
\pgfsetfillcolor{currentfill}%
\pgfsetlinewidth{0.000000pt}%
\definecolor{currentstroke}{rgb}{0.000000,0.000000,0.000000}%
\pgfsetstrokecolor{currentstroke}%
\pgfsetdash{}{0pt}%
\pgfpathmoveto{\pgfqpoint{1.515741in}{1.192463in}}%
\pgfpathlineto{\pgfqpoint{1.518685in}{1.186541in}}%
\pgfpathlineto{\pgfqpoint{1.521628in}{1.180683in}}%
\pgfpathlineto{\pgfqpoint{1.524569in}{1.174893in}}%
\pgfpathlineto{\pgfqpoint{1.527508in}{1.169175in}}%
\pgfpathlineto{\pgfqpoint{1.534969in}{1.163618in}}%
\pgfpathlineto{\pgfqpoint{1.542095in}{1.157941in}}%
\pgfpathlineto{\pgfqpoint{1.548876in}{1.152150in}}%
\pgfpathlineto{\pgfqpoint{1.555305in}{1.146248in}}%
\pgfpathlineto{\pgfqpoint{1.552121in}{1.152179in}}%
\pgfpathlineto{\pgfqpoint{1.548935in}{1.158182in}}%
\pgfpathlineto{\pgfqpoint{1.545747in}{1.164253in}}%
\pgfpathlineto{\pgfqpoint{1.542557in}{1.170389in}}%
\pgfpathlineto{\pgfqpoint{1.536355in}{1.176071in}}%
\pgfpathlineto{\pgfqpoint{1.529814in}{1.181647in}}%
\pgfpathlineto{\pgfqpoint{1.522940in}{1.187113in}}%
\pgfpathlineto{\pgfqpoint{1.515741in}{1.192463in}}%
\pgfpathclose%
\pgfusepath{fill}%
\end{pgfscope}%
\begin{pgfscope}%
\pgfpathrectangle{\pgfqpoint{0.041670in}{0.041670in}}{\pgfqpoint{2.216660in}{2.216660in}}%
\pgfusepath{clip}%
\pgfsetbuttcap%
\pgfsetroundjoin%
\definecolor{currentfill}{rgb}{0.268510,0.009605,0.335427}%
\pgfsetfillcolor{currentfill}%
\pgfsetlinewidth{0.000000pt}%
\definecolor{currentstroke}{rgb}{0.000000,0.000000,0.000000}%
\pgfsetstrokecolor{currentstroke}%
\pgfsetdash{}{0pt}%
\pgfpathmoveto{\pgfqpoint{1.720645in}{0.942754in}}%
\pgfpathlineto{\pgfqpoint{1.724225in}{0.941553in}}%
\pgfpathlineto{\pgfqpoint{1.727810in}{0.940569in}}%
\pgfpathlineto{\pgfqpoint{1.731400in}{0.939807in}}%
\pgfpathlineto{\pgfqpoint{1.734995in}{0.939270in}}%
\pgfpathlineto{\pgfqpoint{1.739188in}{0.930162in}}%
\pgfpathlineto{\pgfqpoint{1.742839in}{0.920978in}}%
\pgfpathlineto{\pgfqpoint{1.745940in}{0.911727in}}%
\pgfpathlineto{\pgfqpoint{1.748487in}{0.902418in}}%
\pgfpathlineto{\pgfqpoint{1.744786in}{0.903197in}}%
\pgfpathlineto{\pgfqpoint{1.741090in}{0.904203in}}%
\pgfpathlineto{\pgfqpoint{1.737400in}{0.905431in}}%
\pgfpathlineto{\pgfqpoint{1.733715in}{0.906877in}}%
\pgfpathlineto{\pgfqpoint{1.731253in}{0.915939in}}%
\pgfpathlineto{\pgfqpoint{1.728250in}{0.924945in}}%
\pgfpathlineto{\pgfqpoint{1.724713in}{0.933886in}}%
\pgfpathlineto{\pgfqpoint{1.720645in}{0.942754in}}%
\pgfpathclose%
\pgfusepath{fill}%
\end{pgfscope}%
\begin{pgfscope}%
\pgfpathrectangle{\pgfqpoint{0.041670in}{0.041670in}}{\pgfqpoint{2.216660in}{2.216660in}}%
\pgfusepath{clip}%
\pgfsetbuttcap%
\pgfsetroundjoin%
\definecolor{currentfill}{rgb}{0.172719,0.448791,0.557885}%
\pgfsetfillcolor{currentfill}%
\pgfsetlinewidth{0.000000pt}%
\definecolor{currentstroke}{rgb}{0.000000,0.000000,0.000000}%
\pgfsetstrokecolor{currentstroke}%
\pgfsetdash{}{0pt}%
\pgfpathmoveto{\pgfqpoint{0.479224in}{1.252129in}}%
\pgfpathlineto{\pgfqpoint{0.475207in}{1.267486in}}%
\pgfpathlineto{\pgfqpoint{0.471168in}{1.283343in}}%
\pgfpathlineto{\pgfqpoint{0.467108in}{1.299710in}}%
\pgfpathlineto{\pgfqpoint{0.475801in}{1.310693in}}%
\pgfpathlineto{\pgfqpoint{0.485155in}{1.321520in}}%
\pgfpathlineto{\pgfqpoint{0.495157in}{1.332179in}}%
\pgfpathlineto{\pgfqpoint{0.505794in}{1.342662in}}%
\pgfpathlineto{\pgfqpoint{0.509610in}{1.326155in}}%
\pgfpathlineto{\pgfqpoint{0.513406in}{1.310154in}}%
\pgfpathlineto{\pgfqpoint{0.517182in}{1.294652in}}%
\pgfpathlineto{\pgfqpoint{0.506741in}{1.284272in}}%
\pgfpathlineto{\pgfqpoint{0.496927in}{1.273719in}}%
\pgfpathlineto{\pgfqpoint{0.487750in}{1.263002in}}%
\pgfpathlineto{\pgfqpoint{0.479224in}{1.252129in}}%
\pgfpathclose%
\pgfusepath{fill}%
\end{pgfscope}%
\begin{pgfscope}%
\pgfpathrectangle{\pgfqpoint{0.041670in}{0.041670in}}{\pgfqpoint{2.216660in}{2.216660in}}%
\pgfusepath{clip}%
\pgfsetbuttcap%
\pgfsetroundjoin%
\definecolor{currentfill}{rgb}{0.133743,0.548535,0.553541}%
\pgfsetfillcolor{currentfill}%
\pgfsetlinewidth{0.000000pt}%
\definecolor{currentstroke}{rgb}{0.000000,0.000000,0.000000}%
\pgfsetstrokecolor{currentstroke}%
\pgfsetdash{}{0pt}%
\pgfpathmoveto{\pgfqpoint{1.140819in}{1.413049in}}%
\pgfpathlineto{\pgfqpoint{1.140378in}{1.407304in}}%
\pgfpathlineto{\pgfqpoint{1.139937in}{1.401552in}}%
\pgfpathlineto{\pgfqpoint{1.139497in}{1.395798in}}%
\pgfpathlineto{\pgfqpoint{1.139058in}{1.390043in}}%
\pgfpathlineto{\pgfqpoint{1.149394in}{1.390590in}}%
\pgfpathlineto{\pgfqpoint{1.159758in}{1.390979in}}%
\pgfpathlineto{\pgfqpoint{1.170140in}{1.391208in}}%
\pgfpathlineto{\pgfqpoint{1.180532in}{1.391279in}}%
\pgfpathlineto{\pgfqpoint{1.180526in}{1.397019in}}%
\pgfpathlineto{\pgfqpoint{1.180520in}{1.402760in}}%
\pgfpathlineto{\pgfqpoint{1.180514in}{1.408497in}}%
\pgfpathlineto{\pgfqpoint{1.180507in}{1.414228in}}%
\pgfpathlineto{\pgfqpoint{1.170563in}{1.414160in}}%
\pgfpathlineto{\pgfqpoint{1.160627in}{1.413941in}}%
\pgfpathlineto{\pgfqpoint{1.150709in}{1.413571in}}%
\pgfpathlineto{\pgfqpoint{1.140819in}{1.413049in}}%
\pgfpathclose%
\pgfusepath{fill}%
\end{pgfscope}%
\begin{pgfscope}%
\pgfpathrectangle{\pgfqpoint{0.041670in}{0.041670in}}{\pgfqpoint{2.216660in}{2.216660in}}%
\pgfusepath{clip}%
\pgfsetbuttcap%
\pgfsetroundjoin%
\definecolor{currentfill}{rgb}{0.133743,0.548535,0.553541}%
\pgfsetfillcolor{currentfill}%
\pgfsetlinewidth{0.000000pt}%
\definecolor{currentstroke}{rgb}{0.000000,0.000000,0.000000}%
\pgfsetstrokecolor{currentstroke}%
\pgfsetdash{}{0pt}%
\pgfpathmoveto{\pgfqpoint{1.180507in}{1.414228in}}%
\pgfpathlineto{\pgfqpoint{1.180514in}{1.408497in}}%
\pgfpathlineto{\pgfqpoint{1.180520in}{1.402760in}}%
\pgfpathlineto{\pgfqpoint{1.180526in}{1.397019in}}%
\pgfpathlineto{\pgfqpoint{1.180532in}{1.391279in}}%
\pgfpathlineto{\pgfqpoint{1.190924in}{1.391191in}}%
\pgfpathlineto{\pgfqpoint{1.201305in}{1.390943in}}%
\pgfpathlineto{\pgfqpoint{1.211666in}{1.390537in}}%
\pgfpathlineto{\pgfqpoint{1.221998in}{1.389973in}}%
\pgfpathlineto{\pgfqpoint{1.221546in}{1.395728in}}%
\pgfpathlineto{\pgfqpoint{1.221094in}{1.401483in}}%
\pgfpathlineto{\pgfqpoint{1.220641in}{1.407236in}}%
\pgfpathlineto{\pgfqpoint{1.220188in}{1.412982in}}%
\pgfpathlineto{\pgfqpoint{1.210301in}{1.413520in}}%
\pgfpathlineto{\pgfqpoint{1.200386in}{1.413907in}}%
\pgfpathlineto{\pgfqpoint{1.190451in}{1.414143in}}%
\pgfpathlineto{\pgfqpoint{1.180507in}{1.414228in}}%
\pgfpathclose%
\pgfusepath{fill}%
\end{pgfscope}%
\begin{pgfscope}%
\pgfpathrectangle{\pgfqpoint{0.041670in}{0.041670in}}{\pgfqpoint{2.216660in}{2.216660in}}%
\pgfusepath{clip}%
\pgfsetbuttcap%
\pgfsetroundjoin%
\definecolor{currentfill}{rgb}{0.260571,0.246922,0.522828}%
\pgfsetfillcolor{currentfill}%
\pgfsetlinewidth{0.000000pt}%
\definecolor{currentstroke}{rgb}{0.000000,0.000000,0.000000}%
\pgfsetstrokecolor{currentstroke}%
\pgfsetdash{}{0pt}%
\pgfpathmoveto{\pgfqpoint{0.500670in}{1.061365in}}%
\pgfpathlineto{\pgfqpoint{0.496701in}{1.071117in}}%
\pgfpathlineto{\pgfqpoint{0.492715in}{1.081284in}}%
\pgfpathlineto{\pgfqpoint{0.488712in}{1.091871in}}%
\pgfpathlineto{\pgfqpoint{0.484691in}{1.102886in}}%
\pgfpathlineto{\pgfqpoint{0.490197in}{1.113932in}}%
\pgfpathlineto{\pgfqpoint{0.496365in}{1.124868in}}%
\pgfpathlineto{\pgfqpoint{0.503186in}{1.135684in}}%
\pgfpathlineto{\pgfqpoint{0.510650in}{1.146370in}}%
\pgfpathlineto{\pgfqpoint{0.514497in}{1.135173in}}%
\pgfpathlineto{\pgfqpoint{0.518327in}{1.124402in}}%
\pgfpathlineto{\pgfqpoint{0.522141in}{1.114049in}}%
\pgfpathlineto{\pgfqpoint{0.525939in}{1.104108in}}%
\pgfpathlineto{\pgfqpoint{0.518669in}{1.093603in}}%
\pgfpathlineto{\pgfqpoint{0.512028in}{1.082971in}}%
\pgfpathlineto{\pgfqpoint{0.506025in}{1.072222in}}%
\pgfpathlineto{\pgfqpoint{0.500670in}{1.061365in}}%
\pgfpathclose%
\pgfusepath{fill}%
\end{pgfscope}%
\begin{pgfscope}%
\pgfpathrectangle{\pgfqpoint{0.041670in}{0.041670in}}{\pgfqpoint{2.216660in}{2.216660in}}%
\pgfusepath{clip}%
\pgfsetbuttcap%
\pgfsetroundjoin%
\definecolor{currentfill}{rgb}{0.179019,0.433756,0.557430}%
\pgfsetfillcolor{currentfill}%
\pgfsetlinewidth{0.000000pt}%
\definecolor{currentstroke}{rgb}{0.000000,0.000000,0.000000}%
\pgfsetstrokecolor{currentstroke}%
\pgfsetdash{}{0pt}%
\pgfpathmoveto{\pgfqpoint{1.419243in}{1.301188in}}%
\pgfpathlineto{\pgfqpoint{1.421600in}{1.295094in}}%
\pgfpathlineto{\pgfqpoint{1.423955in}{1.289029in}}%
\pgfpathlineto{\pgfqpoint{1.426308in}{1.282994in}}%
\pgfpathlineto{\pgfqpoint{1.428658in}{1.276993in}}%
\pgfpathlineto{\pgfqpoint{1.437492in}{1.273051in}}%
\pgfpathlineto{\pgfqpoint{1.446085in}{1.268972in}}%
\pgfpathlineto{\pgfqpoint{1.454427in}{1.264758in}}%
\pgfpathlineto{\pgfqpoint{1.462512in}{1.260413in}}%
\pgfpathlineto{\pgfqpoint{1.459835in}{1.266586in}}%
\pgfpathlineto{\pgfqpoint{1.457156in}{1.272792in}}%
\pgfpathlineto{\pgfqpoint{1.454475in}{1.279029in}}%
\pgfpathlineto{\pgfqpoint{1.451790in}{1.285294in}}%
\pgfpathlineto{\pgfqpoint{1.444019in}{1.289459in}}%
\pgfpathlineto{\pgfqpoint{1.435999in}{1.293498in}}%
\pgfpathlineto{\pgfqpoint{1.427737in}{1.297409in}}%
\pgfpathlineto{\pgfqpoint{1.419243in}{1.301188in}}%
\pgfpathclose%
\pgfusepath{fill}%
\end{pgfscope}%
\begin{pgfscope}%
\pgfpathrectangle{\pgfqpoint{0.041670in}{0.041670in}}{\pgfqpoint{2.216660in}{2.216660in}}%
\pgfusepath{clip}%
\pgfsetbuttcap%
\pgfsetroundjoin%
\definecolor{currentfill}{rgb}{0.277941,0.056324,0.381191}%
\pgfsetfillcolor{currentfill}%
\pgfsetlinewidth{0.000000pt}%
\definecolor{currentstroke}{rgb}{0.000000,0.000000,0.000000}%
\pgfsetstrokecolor{currentstroke}%
\pgfsetdash{}{0pt}%
\pgfpathmoveto{\pgfqpoint{1.793448in}{0.965273in}}%
\pgfpathlineto{\pgfqpoint{1.797174in}{0.969314in}}%
\pgfpathlineto{\pgfqpoint{1.800910in}{0.973670in}}%
\pgfpathlineto{\pgfqpoint{1.804657in}{0.978347in}}%
\pgfpathlineto{\pgfqpoint{1.808416in}{0.983351in}}%
\pgfpathlineto{\pgfqpoint{1.813271in}{0.973100in}}%
\pgfpathlineto{\pgfqpoint{1.817513in}{0.962760in}}%
\pgfpathlineto{\pgfqpoint{1.821135in}{0.952341in}}%
\pgfpathlineto{\pgfqpoint{1.824132in}{0.941853in}}%
\pgfpathlineto{\pgfqpoint{1.820257in}{0.937069in}}%
\pgfpathlineto{\pgfqpoint{1.816394in}{0.932614in}}%
\pgfpathlineto{\pgfqpoint{1.812543in}{0.928482in}}%
\pgfpathlineto{\pgfqpoint{1.808702in}{0.924666in}}%
\pgfpathlineto{\pgfqpoint{1.805799in}{0.934928in}}%
\pgfpathlineto{\pgfqpoint{1.802285in}{0.945123in}}%
\pgfpathlineto{\pgfqpoint{1.798166in}{0.955241in}}%
\pgfpathlineto{\pgfqpoint{1.793448in}{0.965273in}}%
\pgfpathclose%
\pgfusepath{fill}%
\end{pgfscope}%
\begin{pgfscope}%
\pgfpathrectangle{\pgfqpoint{0.041670in}{0.041670in}}{\pgfqpoint{2.216660in}{2.216660in}}%
\pgfusepath{clip}%
\pgfsetbuttcap%
\pgfsetroundjoin%
\definecolor{currentfill}{rgb}{0.163625,0.471133,0.558148}%
\pgfsetfillcolor{currentfill}%
\pgfsetlinewidth{0.000000pt}%
\definecolor{currentstroke}{rgb}{0.000000,0.000000,0.000000}%
\pgfsetstrokecolor{currentstroke}%
\pgfsetdash{}{0pt}%
\pgfpathmoveto{\pgfqpoint{1.375096in}{1.338922in}}%
\pgfpathlineto{\pgfqpoint{1.377102in}{1.332893in}}%
\pgfpathlineto{\pgfqpoint{1.379106in}{1.326881in}}%
\pgfpathlineto{\pgfqpoint{1.381108in}{1.320887in}}%
\pgfpathlineto{\pgfqpoint{1.383107in}{1.314916in}}%
\pgfpathlineto{\pgfqpoint{1.392447in}{1.311698in}}%
\pgfpathlineto{\pgfqpoint{1.401589in}{1.308335in}}%
\pgfpathlineto{\pgfqpoint{1.410524in}{1.304831in}}%
\pgfpathlineto{\pgfqpoint{1.419243in}{1.301188in}}%
\pgfpathlineto{\pgfqpoint{1.416883in}{1.307307in}}%
\pgfpathlineto{\pgfqpoint{1.414521in}{1.313447in}}%
\pgfpathlineto{\pgfqpoint{1.412156in}{1.319607in}}%
\pgfpathlineto{\pgfqpoint{1.409789in}{1.325783in}}%
\pgfpathlineto{\pgfqpoint{1.401420in}{1.329270in}}%
\pgfpathlineto{\pgfqpoint{1.392842in}{1.332624in}}%
\pgfpathlineto{\pgfqpoint{1.384065in}{1.335842in}}%
\pgfpathlineto{\pgfqpoint{1.375096in}{1.338922in}}%
\pgfpathclose%
\pgfusepath{fill}%
\end{pgfscope}%
\begin{pgfscope}%
\pgfpathrectangle{\pgfqpoint{0.041670in}{0.041670in}}{\pgfqpoint{2.216660in}{2.216660in}}%
\pgfusepath{clip}%
\pgfsetbuttcap%
\pgfsetroundjoin%
\definecolor{currentfill}{rgb}{0.271305,0.019942,0.347269}%
\pgfsetfillcolor{currentfill}%
\pgfsetlinewidth{0.000000pt}%
\definecolor{currentstroke}{rgb}{0.000000,0.000000,0.000000}%
\pgfsetstrokecolor{currentstroke}%
\pgfsetdash{}{0pt}%
\pgfpathmoveto{\pgfqpoint{1.706370in}{0.949640in}}%
\pgfpathlineto{\pgfqpoint{1.709932in}{0.947615in}}%
\pgfpathlineto{\pgfqpoint{1.713499in}{0.945789in}}%
\pgfpathlineto{\pgfqpoint{1.717070in}{0.944168in}}%
\pgfpathlineto{\pgfqpoint{1.720645in}{0.942754in}}%
\pgfpathlineto{\pgfqpoint{1.724713in}{0.933886in}}%
\pgfpathlineto{\pgfqpoint{1.728250in}{0.924945in}}%
\pgfpathlineto{\pgfqpoint{1.731253in}{0.915939in}}%
\pgfpathlineto{\pgfqpoint{1.733715in}{0.906877in}}%
\pgfpathlineto{\pgfqpoint{1.730035in}{0.908536in}}%
\pgfpathlineto{\pgfqpoint{1.726360in}{0.910404in}}%
\pgfpathlineto{\pgfqpoint{1.722690in}{0.912476in}}%
\pgfpathlineto{\pgfqpoint{1.719023in}{0.914748in}}%
\pgfpathlineto{\pgfqpoint{1.716644in}{0.923560in}}%
\pgfpathlineto{\pgfqpoint{1.713739in}{0.932319in}}%
\pgfpathlineto{\pgfqpoint{1.710312in}{0.941014in}}%
\pgfpathlineto{\pgfqpoint{1.706370in}{0.949640in}}%
\pgfpathclose%
\pgfusepath{fill}%
\end{pgfscope}%
\begin{pgfscope}%
\pgfpathrectangle{\pgfqpoint{0.041670in}{0.041670in}}{\pgfqpoint{2.216660in}{2.216660in}}%
\pgfusepath{clip}%
\pgfsetbuttcap%
\pgfsetroundjoin%
\definecolor{currentfill}{rgb}{0.133743,0.548535,0.553541}%
\pgfsetfillcolor{currentfill}%
\pgfsetlinewidth{0.000000pt}%
\definecolor{currentstroke}{rgb}{0.000000,0.000000,0.000000}%
\pgfsetstrokecolor{currentstroke}%
\pgfsetdash{}{0pt}%
\pgfpathmoveto{\pgfqpoint{1.101711in}{1.409462in}}%
\pgfpathlineto{\pgfqpoint{1.100829in}{1.403673in}}%
\pgfpathlineto{\pgfqpoint{1.099948in}{1.397879in}}%
\pgfpathlineto{\pgfqpoint{1.099068in}{1.392081in}}%
\pgfpathlineto{\pgfqpoint{1.098189in}{1.386282in}}%
\pgfpathlineto{\pgfqpoint{1.108316in}{1.387457in}}%
\pgfpathlineto{\pgfqpoint{1.118510in}{1.388476in}}%
\pgfpathlineto{\pgfqpoint{1.128760in}{1.389338in}}%
\pgfpathlineto{\pgfqpoint{1.139058in}{1.390043in}}%
\pgfpathlineto{\pgfqpoint{1.139497in}{1.395798in}}%
\pgfpathlineto{\pgfqpoint{1.139937in}{1.401552in}}%
\pgfpathlineto{\pgfqpoint{1.140378in}{1.407304in}}%
\pgfpathlineto{\pgfqpoint{1.140819in}{1.413049in}}%
\pgfpathlineto{\pgfqpoint{1.130964in}{1.412377in}}%
\pgfpathlineto{\pgfqpoint{1.121156in}{1.411554in}}%
\pgfpathlineto{\pgfqpoint{1.111402in}{1.410583in}}%
\pgfpathlineto{\pgfqpoint{1.101711in}{1.409462in}}%
\pgfpathclose%
\pgfusepath{fill}%
\end{pgfscope}%
\begin{pgfscope}%
\pgfpathrectangle{\pgfqpoint{0.041670in}{0.041670in}}{\pgfqpoint{2.216660in}{2.216660in}}%
\pgfusepath{clip}%
\pgfsetbuttcap%
\pgfsetroundjoin%
\definecolor{currentfill}{rgb}{0.133743,0.548535,0.553541}%
\pgfsetfillcolor{currentfill}%
\pgfsetlinewidth{0.000000pt}%
\definecolor{currentstroke}{rgb}{0.000000,0.000000,0.000000}%
\pgfsetstrokecolor{currentstroke}%
\pgfsetdash{}{0pt}%
\pgfpathmoveto{\pgfqpoint{1.220188in}{1.412982in}}%
\pgfpathlineto{\pgfqpoint{1.220641in}{1.407236in}}%
\pgfpathlineto{\pgfqpoint{1.221094in}{1.401483in}}%
\pgfpathlineto{\pgfqpoint{1.221546in}{1.395728in}}%
\pgfpathlineto{\pgfqpoint{1.221998in}{1.389973in}}%
\pgfpathlineto{\pgfqpoint{1.232291in}{1.389250in}}%
\pgfpathlineto{\pgfqpoint{1.242535in}{1.388370in}}%
\pgfpathlineto{\pgfqpoint{1.252722in}{1.387334in}}%
\pgfpathlineto{\pgfqpoint{1.262841in}{1.386142in}}%
\pgfpathlineto{\pgfqpoint{1.261950in}{1.391942in}}%
\pgfpathlineto{\pgfqpoint{1.261058in}{1.397742in}}%
\pgfpathlineto{\pgfqpoint{1.260165in}{1.403538in}}%
\pgfpathlineto{\pgfqpoint{1.259271in}{1.409329in}}%
\pgfpathlineto{\pgfqpoint{1.249588in}{1.410466in}}%
\pgfpathlineto{\pgfqpoint{1.239841in}{1.411454in}}%
\pgfpathlineto{\pgfqpoint{1.230038in}{1.412293in}}%
\pgfpathlineto{\pgfqpoint{1.220188in}{1.412982in}}%
\pgfpathclose%
\pgfusepath{fill}%
\end{pgfscope}%
\begin{pgfscope}%
\pgfpathrectangle{\pgfqpoint{0.041670in}{0.041670in}}{\pgfqpoint{2.216660in}{2.216660in}}%
\pgfusepath{clip}%
\pgfsetbuttcap%
\pgfsetroundjoin%
\definecolor{currentfill}{rgb}{0.283072,0.130895,0.449241}%
\pgfsetfillcolor{currentfill}%
\pgfsetlinewidth{0.000000pt}%
\definecolor{currentstroke}{rgb}{0.000000,0.000000,0.000000}%
\pgfsetstrokecolor{currentstroke}%
\pgfsetdash{}{0pt}%
\pgfpathmoveto{\pgfqpoint{0.707080in}{0.999011in}}%
\pgfpathlineto{\pgfqpoint{0.703527in}{0.994295in}}%
\pgfpathlineto{\pgfqpoint{0.699972in}{0.989715in}}%
\pgfpathlineto{\pgfqpoint{0.696417in}{0.985274in}}%
\pgfpathlineto{\pgfqpoint{0.692860in}{0.980976in}}%
\pgfpathlineto{\pgfqpoint{0.696489in}{0.988854in}}%
\pgfpathlineto{\pgfqpoint{0.700588in}{0.996660in}}%
\pgfpathlineto{\pgfqpoint{0.705151in}{1.004388in}}%
\pgfpathlineto{\pgfqpoint{0.710173in}{1.012030in}}%
\pgfpathlineto{\pgfqpoint{0.713588in}{1.016083in}}%
\pgfpathlineto{\pgfqpoint{0.717003in}{1.020280in}}%
\pgfpathlineto{\pgfqpoint{0.720416in}{1.024615in}}%
\pgfpathlineto{\pgfqpoint{0.723829in}{1.029085in}}%
\pgfpathlineto{\pgfqpoint{0.718968in}{1.021683in}}%
\pgfpathlineto{\pgfqpoint{0.714553in}{1.014199in}}%
\pgfpathlineto{\pgfqpoint{0.710588in}{1.006639in}}%
\pgfpathlineto{\pgfqpoint{0.707080in}{0.999011in}}%
\pgfpathclose%
\pgfusepath{fill}%
\end{pgfscope}%
\begin{pgfscope}%
\pgfpathrectangle{\pgfqpoint{0.041670in}{0.041670in}}{\pgfqpoint{2.216660in}{2.216660in}}%
\pgfusepath{clip}%
\pgfsetbuttcap%
\pgfsetroundjoin%
\definecolor{currentfill}{rgb}{0.231674,0.318106,0.544834}%
\pgfsetfillcolor{currentfill}%
\pgfsetlinewidth{0.000000pt}%
\definecolor{currentstroke}{rgb}{0.000000,0.000000,0.000000}%
\pgfsetstrokecolor{currentstroke}%
\pgfsetdash{}{0pt}%
\pgfpathmoveto{\pgfqpoint{0.812132in}{1.165255in}}%
\pgfpathlineto{\pgfqpoint{0.808894in}{1.159069in}}%
\pgfpathlineto{\pgfqpoint{0.805657in}{1.152948in}}%
\pgfpathlineto{\pgfqpoint{0.802423in}{1.146896in}}%
\pgfpathlineto{\pgfqpoint{0.799190in}{1.140915in}}%
\pgfpathlineto{\pgfqpoint{0.805301in}{1.146909in}}%
\pgfpathlineto{\pgfqpoint{0.811770in}{1.152799in}}%
\pgfpathlineto{\pgfqpoint{0.818590in}{1.158578in}}%
\pgfpathlineto{\pgfqpoint{0.825753in}{1.164242in}}%
\pgfpathlineto{\pgfqpoint{0.828750in}{1.170006in}}%
\pgfpathlineto{\pgfqpoint{0.831750in}{1.175841in}}%
\pgfpathlineto{\pgfqpoint{0.834750in}{1.181744in}}%
\pgfpathlineto{\pgfqpoint{0.837753in}{1.187713in}}%
\pgfpathlineto{\pgfqpoint{0.830843in}{1.182260in}}%
\pgfpathlineto{\pgfqpoint{0.824264in}{1.176696in}}%
\pgfpathlineto{\pgfqpoint{0.818025in}{1.171026in}}%
\pgfpathlineto{\pgfqpoint{0.812132in}{1.165255in}}%
\pgfpathclose%
\pgfusepath{fill}%
\end{pgfscope}%
\begin{pgfscope}%
\pgfpathrectangle{\pgfqpoint{0.041670in}{0.041670in}}{\pgfqpoint{2.216660in}{2.216660in}}%
\pgfusepath{clip}%
\pgfsetbuttcap%
\pgfsetroundjoin%
\definecolor{currentfill}{rgb}{0.267004,0.004874,0.329415}%
\pgfsetfillcolor{currentfill}%
\pgfsetlinewidth{0.000000pt}%
\definecolor{currentstroke}{rgb}{0.000000,0.000000,0.000000}%
\pgfsetstrokecolor{currentstroke}%
\pgfsetdash{}{0pt}%
\pgfpathmoveto{\pgfqpoint{0.594696in}{0.893125in}}%
\pgfpathlineto{\pgfqpoint{0.590946in}{0.893495in}}%
\pgfpathlineto{\pgfqpoint{0.587188in}{0.894120in}}%
\pgfpathlineto{\pgfqpoint{0.583423in}{0.895005in}}%
\pgfpathlineto{\pgfqpoint{0.579651in}{0.896154in}}%
\pgfpathlineto{\pgfqpoint{0.581850in}{0.905996in}}%
\pgfpathlineto{\pgfqpoint{0.584636in}{0.915783in}}%
\pgfpathlineto{\pgfqpoint{0.588004in}{0.925507in}}%
\pgfpathlineto{\pgfqpoint{0.591949in}{0.935158in}}%
\pgfpathlineto{\pgfqpoint{0.595625in}{0.933771in}}%
\pgfpathlineto{\pgfqpoint{0.599293in}{0.932647in}}%
\pgfpathlineto{\pgfqpoint{0.602954in}{0.931783in}}%
\pgfpathlineto{\pgfqpoint{0.606608in}{0.931173in}}%
\pgfpathlineto{\pgfqpoint{0.602781in}{0.921757in}}%
\pgfpathlineto{\pgfqpoint{0.599517in}{0.912272in}}%
\pgfpathlineto{\pgfqpoint{0.596820in}{0.902725in}}%
\pgfpathlineto{\pgfqpoint{0.594696in}{0.893125in}}%
\pgfpathclose%
\pgfusepath{fill}%
\end{pgfscope}%
\begin{pgfscope}%
\pgfpathrectangle{\pgfqpoint{0.041670in}{0.041670in}}{\pgfqpoint{2.216660in}{2.216660in}}%
\pgfusepath{clip}%
\pgfsetbuttcap%
\pgfsetroundjoin%
\definecolor{currentfill}{rgb}{0.263663,0.237631,0.518762}%
\pgfsetfillcolor{currentfill}%
\pgfsetlinewidth{0.000000pt}%
\definecolor{currentstroke}{rgb}{0.000000,0.000000,0.000000}%
\pgfsetstrokecolor{currentstroke}%
\pgfsetdash{}{0pt}%
\pgfpathmoveto{\pgfqpoint{0.764779in}{1.091924in}}%
\pgfpathlineto{\pgfqpoint{0.761364in}{1.086118in}}%
\pgfpathlineto{\pgfqpoint{0.757950in}{1.080404in}}%
\pgfpathlineto{\pgfqpoint{0.754536in}{1.074784in}}%
\pgfpathlineto{\pgfqpoint{0.751123in}{1.069264in}}%
\pgfpathlineto{\pgfqpoint{0.756080in}{1.076100in}}%
\pgfpathlineto{\pgfqpoint{0.761445in}{1.082846in}}%
\pgfpathlineto{\pgfqpoint{0.767212in}{1.089497in}}%
\pgfpathlineto{\pgfqpoint{0.773373in}{1.096047in}}%
\pgfpathlineto{\pgfqpoint{0.776596in}{1.101334in}}%
\pgfpathlineto{\pgfqpoint{0.779821in}{1.106719in}}%
\pgfpathlineto{\pgfqpoint{0.783046in}{1.112201in}}%
\pgfpathlineto{\pgfqpoint{0.786272in}{1.117773in}}%
\pgfpathlineto{\pgfqpoint{0.780318in}{1.111452in}}%
\pgfpathlineto{\pgfqpoint{0.774747in}{1.105032in}}%
\pgfpathlineto{\pgfqpoint{0.769565in}{1.098521in}}%
\pgfpathlineto{\pgfqpoint{0.764779in}{1.091924in}}%
\pgfpathclose%
\pgfusepath{fill}%
\end{pgfscope}%
\begin{pgfscope}%
\pgfpathrectangle{\pgfqpoint{0.041670in}{0.041670in}}{\pgfqpoint{2.216660in}{2.216660in}}%
\pgfusepath{clip}%
\pgfsetbuttcap%
\pgfsetroundjoin%
\definecolor{currentfill}{rgb}{0.233603,0.313828,0.543914}%
\pgfsetfillcolor{currentfill}%
\pgfsetlinewidth{0.000000pt}%
\definecolor{currentstroke}{rgb}{0.000000,0.000000,0.000000}%
\pgfsetstrokecolor{currentstroke}%
\pgfsetdash{}{0pt}%
\pgfpathmoveto{\pgfqpoint{1.842093in}{1.155752in}}%
\pgfpathlineto{\pgfqpoint{1.845910in}{1.167419in}}%
\pgfpathlineto{\pgfqpoint{1.849745in}{1.179526in}}%
\pgfpathlineto{\pgfqpoint{1.853597in}{1.192079in}}%
\pgfpathlineto{\pgfqpoint{1.857467in}{1.205086in}}%
\pgfpathlineto{\pgfqpoint{1.865705in}{1.194355in}}%
\pgfpathlineto{\pgfqpoint{1.873295in}{1.183482in}}%
\pgfpathlineto{\pgfqpoint{1.880226in}{1.172478in}}%
\pgfpathlineto{\pgfqpoint{1.886489in}{1.161352in}}%
\pgfpathlineto{\pgfqpoint{1.882428in}{1.148514in}}%
\pgfpathlineto{\pgfqpoint{1.878385in}{1.136133in}}%
\pgfpathlineto{\pgfqpoint{1.874362in}{1.124201in}}%
\pgfpathlineto{\pgfqpoint{1.870357in}{1.112710in}}%
\pgfpathlineto{\pgfqpoint{1.864262in}{1.123659in}}%
\pgfpathlineto{\pgfqpoint{1.857513in}{1.134489in}}%
\pgfpathlineto{\pgfqpoint{1.850120in}{1.145190in}}%
\pgfpathlineto{\pgfqpoint{1.842093in}{1.155752in}}%
\pgfpathclose%
\pgfusepath{fill}%
\end{pgfscope}%
\begin{pgfscope}%
\pgfpathrectangle{\pgfqpoint{0.041670in}{0.041670in}}{\pgfqpoint{2.216660in}{2.216660in}}%
\pgfusepath{clip}%
\pgfsetbuttcap%
\pgfsetroundjoin%
\definecolor{currentfill}{rgb}{0.268510,0.009605,0.335427}%
\pgfsetfillcolor{currentfill}%
\pgfsetlinewidth{0.000000pt}%
\definecolor{currentstroke}{rgb}{0.000000,0.000000,0.000000}%
\pgfsetstrokecolor{currentstroke}%
\pgfsetdash{}{0pt}%
\pgfpathmoveto{\pgfqpoint{0.579651in}{0.896154in}}%
\pgfpathlineto{\pgfqpoint{0.575869in}{0.897574in}}%
\pgfpathlineto{\pgfqpoint{0.572080in}{0.899268in}}%
\pgfpathlineto{\pgfqpoint{0.568281in}{0.901242in}}%
\pgfpathlineto{\pgfqpoint{0.564474in}{0.903502in}}%
\pgfpathlineto{\pgfqpoint{0.566750in}{0.913582in}}%
\pgfpathlineto{\pgfqpoint{0.569627in}{0.923605in}}%
\pgfpathlineto{\pgfqpoint{0.573101in}{0.933563in}}%
\pgfpathlineto{\pgfqpoint{0.577165in}{0.943445in}}%
\pgfpathlineto{\pgfqpoint{0.580874in}{0.940953in}}%
\pgfpathlineto{\pgfqpoint{0.584574in}{0.938744in}}%
\pgfpathlineto{\pgfqpoint{0.588265in}{0.936814in}}%
\pgfpathlineto{\pgfqpoint{0.591949in}{0.935158in}}%
\pgfpathlineto{\pgfqpoint{0.588004in}{0.925507in}}%
\pgfpathlineto{\pgfqpoint{0.584636in}{0.915783in}}%
\pgfpathlineto{\pgfqpoint{0.581850in}{0.905996in}}%
\pgfpathlineto{\pgfqpoint{0.579651in}{0.896154in}}%
\pgfpathclose%
\pgfusepath{fill}%
\end{pgfscope}%
\begin{pgfscope}%
\pgfpathrectangle{\pgfqpoint{0.041670in}{0.041670in}}{\pgfqpoint{2.216660in}{2.216660in}}%
\pgfusepath{clip}%
\pgfsetbuttcap%
\pgfsetroundjoin%
\definecolor{currentfill}{rgb}{0.267004,0.004874,0.329415}%
\pgfsetfillcolor{currentfill}%
\pgfsetlinewidth{0.000000pt}%
\definecolor{currentstroke}{rgb}{0.000000,0.000000,0.000000}%
\pgfsetstrokecolor{currentstroke}%
\pgfsetdash{}{0pt}%
\pgfpathmoveto{\pgfqpoint{0.609627in}{0.894101in}}%
\pgfpathlineto{\pgfqpoint{0.605904in}{0.893499in}}%
\pgfpathlineto{\pgfqpoint{0.602175in}{0.893132in}}%
\pgfpathlineto{\pgfqpoint{0.598439in}{0.893006in}}%
\pgfpathlineto{\pgfqpoint{0.594696in}{0.893125in}}%
\pgfpathlineto{\pgfqpoint{0.596820in}{0.902725in}}%
\pgfpathlineto{\pgfqpoint{0.599517in}{0.912272in}}%
\pgfpathlineto{\pgfqpoint{0.602781in}{0.921757in}}%
\pgfpathlineto{\pgfqpoint{0.606608in}{0.931173in}}%
\pgfpathlineto{\pgfqpoint{0.610256in}{0.930812in}}%
\pgfpathlineto{\pgfqpoint{0.613897in}{0.930695in}}%
\pgfpathlineto{\pgfqpoint{0.617532in}{0.930819in}}%
\pgfpathlineto{\pgfqpoint{0.621161in}{0.931178in}}%
\pgfpathlineto{\pgfqpoint{0.617450in}{0.922002in}}%
\pgfpathlineto{\pgfqpoint{0.614287in}{0.912758in}}%
\pgfpathlineto{\pgfqpoint{0.611678in}{0.903455in}}%
\pgfpathlineto{\pgfqpoint{0.609627in}{0.894101in}}%
\pgfpathclose%
\pgfusepath{fill}%
\end{pgfscope}%
\begin{pgfscope}%
\pgfpathrectangle{\pgfqpoint{0.041670in}{0.041670in}}{\pgfqpoint{2.216660in}{2.216660in}}%
\pgfusepath{clip}%
\pgfsetbuttcap%
\pgfsetroundjoin%
\definecolor{currentfill}{rgb}{0.195860,0.395433,0.555276}%
\pgfsetfillcolor{currentfill}%
\pgfsetlinewidth{0.000000pt}%
\definecolor{currentstroke}{rgb}{0.000000,0.000000,0.000000}%
\pgfsetstrokecolor{currentstroke}%
\pgfsetdash{}{0pt}%
\pgfpathmoveto{\pgfqpoint{1.462512in}{1.260413in}}%
\pgfpathlineto{\pgfqpoint{1.465186in}{1.254278in}}%
\pgfpathlineto{\pgfqpoint{1.467858in}{1.248182in}}%
\pgfpathlineto{\pgfqpoint{1.470527in}{1.242129in}}%
\pgfpathlineto{\pgfqpoint{1.473194in}{1.236123in}}%
\pgfpathlineto{\pgfqpoint{1.481315in}{1.231466in}}%
\pgfpathlineto{\pgfqpoint{1.489152in}{1.226680in}}%
\pgfpathlineto{\pgfqpoint{1.496697in}{1.221770in}}%
\pgfpathlineto{\pgfqpoint{1.503942in}{1.216741in}}%
\pgfpathlineto{\pgfqpoint{1.500987in}{1.222941in}}%
\pgfpathlineto{\pgfqpoint{1.498030in}{1.229188in}}%
\pgfpathlineto{\pgfqpoint{1.495070in}{1.235477in}}%
\pgfpathlineto{\pgfqpoint{1.492107in}{1.241806in}}%
\pgfpathlineto{\pgfqpoint{1.485135in}{1.246634in}}%
\pgfpathlineto{\pgfqpoint{1.477874in}{1.251347in}}%
\pgfpathlineto{\pgfqpoint{1.470330in}{1.255942in}}%
\pgfpathlineto{\pgfqpoint{1.462512in}{1.260413in}}%
\pgfpathclose%
\pgfusepath{fill}%
\end{pgfscope}%
\begin{pgfscope}%
\pgfpathrectangle{\pgfqpoint{0.041670in}{0.041670in}}{\pgfqpoint{2.216660in}{2.216660in}}%
\pgfusepath{clip}%
\pgfsetbuttcap%
\pgfsetroundjoin%
\definecolor{currentfill}{rgb}{0.280255,0.165693,0.476498}%
\pgfsetfillcolor{currentfill}%
\pgfsetlinewidth{0.000000pt}%
\definecolor{currentstroke}{rgb}{0.000000,0.000000,0.000000}%
\pgfsetstrokecolor{currentstroke}%
\pgfsetdash{}{0pt}%
\pgfpathmoveto{\pgfqpoint{1.617896in}{1.054534in}}%
\pgfpathlineto{\pgfqpoint{1.621269in}{1.049614in}}%
\pgfpathlineto{\pgfqpoint{1.624643in}{1.044814in}}%
\pgfpathlineto{\pgfqpoint{1.628017in}{1.040138in}}%
\pgfpathlineto{\pgfqpoint{1.631391in}{1.035589in}}%
\pgfpathlineto{\pgfqpoint{1.636642in}{1.028267in}}%
\pgfpathlineto{\pgfqpoint{1.641454in}{1.020856in}}%
\pgfpathlineto{\pgfqpoint{1.645820in}{1.013363in}}%
\pgfpathlineto{\pgfqpoint{1.649734in}{1.005795in}}%
\pgfpathlineto{\pgfqpoint{1.646208in}{1.010586in}}%
\pgfpathlineto{\pgfqpoint{1.642683in}{1.015505in}}%
\pgfpathlineto{\pgfqpoint{1.639158in}{1.020548in}}%
\pgfpathlineto{\pgfqpoint{1.635633in}{1.025712in}}%
\pgfpathlineto{\pgfqpoint{1.631851in}{1.033032in}}%
\pgfpathlineto{\pgfqpoint{1.627631in}{1.040280in}}%
\pgfpathlineto{\pgfqpoint{1.622977in}{1.047450in}}%
\pgfpathlineto{\pgfqpoint{1.617896in}{1.054534in}}%
\pgfpathclose%
\pgfusepath{fill}%
\end{pgfscope}%
\begin{pgfscope}%
\pgfpathrectangle{\pgfqpoint{0.041670in}{0.041670in}}{\pgfqpoint{2.216660in}{2.216660in}}%
\pgfusepath{clip}%
\pgfsetbuttcap%
\pgfsetroundjoin%
\definecolor{currentfill}{rgb}{0.163625,0.471133,0.558148}%
\pgfsetfillcolor{currentfill}%
\pgfsetlinewidth{0.000000pt}%
\definecolor{currentstroke}{rgb}{0.000000,0.000000,0.000000}%
\pgfsetstrokecolor{currentstroke}%
\pgfsetdash{}{0pt}%
\pgfpathmoveto{\pgfqpoint{0.942862in}{1.322575in}}%
\pgfpathlineto{\pgfqpoint{0.940419in}{1.316363in}}%
\pgfpathlineto{\pgfqpoint{0.937978in}{1.310167in}}%
\pgfpathlineto{\pgfqpoint{0.935540in}{1.303990in}}%
\pgfpathlineto{\pgfqpoint{0.933105in}{1.297836in}}%
\pgfpathlineto{\pgfqpoint{0.941625in}{1.301599in}}%
\pgfpathlineto{\pgfqpoint{0.950368in}{1.305227in}}%
\pgfpathlineto{\pgfqpoint{0.959326in}{1.308716in}}%
\pgfpathlineto{\pgfqpoint{0.968490in}{1.312063in}}%
\pgfpathlineto{\pgfqpoint{0.970572in}{1.318065in}}%
\pgfpathlineto{\pgfqpoint{0.972657in}{1.324089in}}%
\pgfpathlineto{\pgfqpoint{0.974743in}{1.330132in}}%
\pgfpathlineto{\pgfqpoint{0.976832in}{1.336191in}}%
\pgfpathlineto{\pgfqpoint{0.968033in}{1.332988in}}%
\pgfpathlineto{\pgfqpoint{0.959433in}{1.329649in}}%
\pgfpathlineto{\pgfqpoint{0.951040in}{1.326177in}}%
\pgfpathlineto{\pgfqpoint{0.942862in}{1.322575in}}%
\pgfpathclose%
\pgfusepath{fill}%
\end{pgfscope}%
\begin{pgfscope}%
\pgfpathrectangle{\pgfqpoint{0.041670in}{0.041670in}}{\pgfqpoint{2.216660in}{2.216660in}}%
\pgfusepath{clip}%
\pgfsetbuttcap%
\pgfsetroundjoin%
\definecolor{currentfill}{rgb}{0.179019,0.433756,0.557430}%
\pgfsetfillcolor{currentfill}%
\pgfsetlinewidth{0.000000pt}%
\definecolor{currentstroke}{rgb}{0.000000,0.000000,0.000000}%
\pgfsetstrokecolor{currentstroke}%
\pgfsetdash{}{0pt}%
\pgfpathmoveto{\pgfqpoint{0.901427in}{1.281490in}}%
\pgfpathlineto{\pgfqpoint{0.898675in}{1.275184in}}%
\pgfpathlineto{\pgfqpoint{0.895926in}{1.268906in}}%
\pgfpathlineto{\pgfqpoint{0.893179in}{1.262658in}}%
\pgfpathlineto{\pgfqpoint{0.890435in}{1.256445in}}%
\pgfpathlineto{\pgfqpoint{0.898283in}{1.260902in}}%
\pgfpathlineto{\pgfqpoint{0.906397in}{1.265233in}}%
\pgfpathlineto{\pgfqpoint{0.914768in}{1.269432in}}%
\pgfpathlineto{\pgfqpoint{0.923388in}{1.273496in}}%
\pgfpathlineto{\pgfqpoint{0.925813in}{1.279533in}}%
\pgfpathlineto{\pgfqpoint{0.928241in}{1.285604in}}%
\pgfpathlineto{\pgfqpoint{0.930672in}{1.291706in}}%
\pgfpathlineto{\pgfqpoint{0.933105in}{1.297836in}}%
\pgfpathlineto{\pgfqpoint{0.924817in}{1.293939in}}%
\pgfpathlineto{\pgfqpoint{0.916770in}{1.289914in}}%
\pgfpathlineto{\pgfqpoint{0.908970in}{1.285763in}}%
\pgfpathlineto{\pgfqpoint{0.901427in}{1.281490in}}%
\pgfpathclose%
\pgfusepath{fill}%
\end{pgfscope}%
\begin{pgfscope}%
\pgfpathrectangle{\pgfqpoint{0.041670in}{0.041670in}}{\pgfqpoint{2.216660in}{2.216660in}}%
\pgfusepath{clip}%
\pgfsetbuttcap%
\pgfsetroundjoin%
\definecolor{currentfill}{rgb}{0.147607,0.511733,0.557049}%
\pgfsetfillcolor{currentfill}%
\pgfsetlinewidth{0.000000pt}%
\definecolor{currentstroke}{rgb}{0.000000,0.000000,0.000000}%
\pgfsetstrokecolor{currentstroke}%
\pgfsetdash{}{0pt}%
\pgfpathmoveto{\pgfqpoint{1.331001in}{1.373542in}}%
\pgfpathlineto{\pgfqpoint{1.332625in}{1.367598in}}%
\pgfpathlineto{\pgfqpoint{1.334248in}{1.361658in}}%
\pgfpathlineto{\pgfqpoint{1.335868in}{1.355727in}}%
\pgfpathlineto{\pgfqpoint{1.337487in}{1.349806in}}%
\pgfpathlineto{\pgfqpoint{1.347132in}{1.347305in}}%
\pgfpathlineto{\pgfqpoint{1.356621in}{1.344656in}}%
\pgfpathlineto{\pgfqpoint{1.365946in}{1.341861in}}%
\pgfpathlineto{\pgfqpoint{1.375096in}{1.338922in}}%
\pgfpathlineto{\pgfqpoint{1.373088in}{1.344964in}}%
\pgfpathlineto{\pgfqpoint{1.371078in}{1.351017in}}%
\pgfpathlineto{\pgfqpoint{1.369065in}{1.357078in}}%
\pgfpathlineto{\pgfqpoint{1.367049in}{1.363143in}}%
\pgfpathlineto{\pgfqpoint{1.358279in}{1.365951in}}%
\pgfpathlineto{\pgfqpoint{1.349342in}{1.368621in}}%
\pgfpathlineto{\pgfqpoint{1.340246in}{1.371152in}}%
\pgfpathlineto{\pgfqpoint{1.331001in}{1.373542in}}%
\pgfpathclose%
\pgfusepath{fill}%
\end{pgfscope}%
\begin{pgfscope}%
\pgfpathrectangle{\pgfqpoint{0.041670in}{0.041670in}}{\pgfqpoint{2.216660in}{2.216660in}}%
\pgfusepath{clip}%
\pgfsetbuttcap%
\pgfsetroundjoin%
\definecolor{currentfill}{rgb}{0.268510,0.009605,0.335427}%
\pgfsetfillcolor{currentfill}%
\pgfsetlinewidth{0.000000pt}%
\definecolor{currentstroke}{rgb}{0.000000,0.000000,0.000000}%
\pgfsetstrokecolor{currentstroke}%
\pgfsetdash{}{0pt}%
\pgfpathmoveto{\pgfqpoint{0.624463in}{0.898782in}}%
\pgfpathlineto{\pgfqpoint{0.620762in}{0.897280in}}%
\pgfpathlineto{\pgfqpoint{0.617056in}{0.895997in}}%
\pgfpathlineto{\pgfqpoint{0.613345in}{0.894936in}}%
\pgfpathlineto{\pgfqpoint{0.609627in}{0.894101in}}%
\pgfpathlineto{\pgfqpoint{0.611678in}{0.903455in}}%
\pgfpathlineto{\pgfqpoint{0.614287in}{0.912758in}}%
\pgfpathlineto{\pgfqpoint{0.617450in}{0.922002in}}%
\pgfpathlineto{\pgfqpoint{0.621161in}{0.931178in}}%
\pgfpathlineto{\pgfqpoint{0.624784in}{0.931768in}}%
\pgfpathlineto{\pgfqpoint{0.628402in}{0.932583in}}%
\pgfpathlineto{\pgfqpoint{0.632015in}{0.933621in}}%
\pgfpathlineto{\pgfqpoint{0.635623in}{0.934875in}}%
\pgfpathlineto{\pgfqpoint{0.632026in}{0.925942in}}%
\pgfpathlineto{\pgfqpoint{0.628964in}{0.916943in}}%
\pgfpathlineto{\pgfqpoint{0.626441in}{0.907886in}}%
\pgfpathlineto{\pgfqpoint{0.624463in}{0.898782in}}%
\pgfpathclose%
\pgfusepath{fill}%
\end{pgfscope}%
\begin{pgfscope}%
\pgfpathrectangle{\pgfqpoint{0.041670in}{0.041670in}}{\pgfqpoint{2.216660in}{2.216660in}}%
\pgfusepath{clip}%
\pgfsetbuttcap%
\pgfsetroundjoin%
\definecolor{currentfill}{rgb}{0.133743,0.548535,0.553541}%
\pgfsetfillcolor{currentfill}%
\pgfsetlinewidth{0.000000pt}%
\definecolor{currentstroke}{rgb}{0.000000,0.000000,0.000000}%
\pgfsetstrokecolor{currentstroke}%
\pgfsetdash{}{0pt}%
\pgfpathmoveto{\pgfqpoint{1.259271in}{1.409329in}}%
\pgfpathlineto{\pgfqpoint{1.260165in}{1.403538in}}%
\pgfpathlineto{\pgfqpoint{1.261058in}{1.397742in}}%
\pgfpathlineto{\pgfqpoint{1.261950in}{1.391942in}}%
\pgfpathlineto{\pgfqpoint{1.262841in}{1.386142in}}%
\pgfpathlineto{\pgfqpoint{1.272883in}{1.384795in}}%
\pgfpathlineto{\pgfqpoint{1.282839in}{1.383295in}}%
\pgfpathlineto{\pgfqpoint{1.292699in}{1.381643in}}%
\pgfpathlineto{\pgfqpoint{1.291487in}{1.387495in}}%
\pgfpathlineto{\pgfqpoint{1.290273in}{1.393347in}}%
\pgfpathlineto{\pgfqpoint{1.289057in}{1.399196in}}%
\pgfpathlineto{\pgfqpoint{1.287840in}{1.405038in}}%
\pgfpathlineto{\pgfqpoint{1.278405in}{1.406614in}}%
\pgfpathlineto{\pgfqpoint{1.268879in}{1.408045in}}%
\pgfpathlineto{\pgfqpoint{1.259271in}{1.409329in}}%
\pgfpathclose%
\pgfusepath{fill}%
\end{pgfscope}%
\begin{pgfscope}%
\pgfpathrectangle{\pgfqpoint{0.041670in}{0.041670in}}{\pgfqpoint{2.216660in}{2.216660in}}%
\pgfusepath{clip}%
\pgfsetbuttcap%
\pgfsetroundjoin%
\definecolor{currentfill}{rgb}{0.274952,0.037752,0.364543}%
\pgfsetfillcolor{currentfill}%
\pgfsetlinewidth{0.000000pt}%
\definecolor{currentstroke}{rgb}{0.000000,0.000000,0.000000}%
\pgfsetstrokecolor{currentstroke}%
\pgfsetdash{}{0pt}%
\pgfpathmoveto{\pgfqpoint{1.692153in}{0.959650in}}%
\pgfpathlineto{\pgfqpoint{1.695702in}{0.956869in}}%
\pgfpathlineto{\pgfqpoint{1.699255in}{0.954271in}}%
\pgfpathlineto{\pgfqpoint{1.702810in}{0.951860in}}%
\pgfpathlineto{\pgfqpoint{1.706370in}{0.949640in}}%
\pgfpathlineto{\pgfqpoint{1.710312in}{0.941014in}}%
\pgfpathlineto{\pgfqpoint{1.713739in}{0.932319in}}%
\pgfpathlineto{\pgfqpoint{1.716644in}{0.923560in}}%
\pgfpathlineto{\pgfqpoint{1.719023in}{0.914748in}}%
\pgfpathlineto{\pgfqpoint{1.715361in}{0.917216in}}%
\pgfpathlineto{\pgfqpoint{1.711702in}{0.919876in}}%
\pgfpathlineto{\pgfqpoint{1.708047in}{0.922722in}}%
\pgfpathlineto{\pgfqpoint{1.704396in}{0.925752in}}%
\pgfpathlineto{\pgfqpoint{1.702099in}{0.934313in}}%
\pgfpathlineto{\pgfqpoint{1.699289in}{0.942821in}}%
\pgfpathlineto{\pgfqpoint{1.695972in}{0.951270in}}%
\pgfpathlineto{\pgfqpoint{1.692153in}{0.959650in}}%
\pgfpathclose%
\pgfusepath{fill}%
\end{pgfscope}%
\begin{pgfscope}%
\pgfpathrectangle{\pgfqpoint{0.041670in}{0.041670in}}{\pgfqpoint{2.216660in}{2.216660in}}%
\pgfusepath{clip}%
\pgfsetbuttcap%
\pgfsetroundjoin%
\definecolor{currentfill}{rgb}{0.272594,0.025563,0.353093}%
\pgfsetfillcolor{currentfill}%
\pgfsetlinewidth{0.000000pt}%
\definecolor{currentstroke}{rgb}{0.000000,0.000000,0.000000}%
\pgfsetstrokecolor{currentstroke}%
\pgfsetdash{}{0pt}%
\pgfpathmoveto{\pgfqpoint{0.564474in}{0.903502in}}%
\pgfpathlineto{\pgfqpoint{0.560656in}{0.906052in}}%
\pgfpathlineto{\pgfqpoint{0.556830in}{0.908897in}}%
\pgfpathlineto{\pgfqpoint{0.552993in}{0.912044in}}%
\pgfpathlineto{\pgfqpoint{0.549145in}{0.915496in}}%
\pgfpathlineto{\pgfqpoint{0.551500in}{0.925810in}}%
\pgfpathlineto{\pgfqpoint{0.554471in}{0.936065in}}%
\pgfpathlineto{\pgfqpoint{0.558052in}{0.946251in}}%
\pgfpathlineto{\pgfqpoint{0.562239in}{0.956361in}}%
\pgfpathlineto{\pgfqpoint{0.565985in}{0.952680in}}%
\pgfpathlineto{\pgfqpoint{0.569721in}{0.949304in}}%
\pgfpathlineto{\pgfqpoint{0.573448in}{0.946227in}}%
\pgfpathlineto{\pgfqpoint{0.577165in}{0.943445in}}%
\pgfpathlineto{\pgfqpoint{0.573101in}{0.933563in}}%
\pgfpathlineto{\pgfqpoint{0.569627in}{0.923605in}}%
\pgfpathlineto{\pgfqpoint{0.566750in}{0.913582in}}%
\pgfpathlineto{\pgfqpoint{0.564474in}{0.903502in}}%
\pgfpathclose%
\pgfusepath{fill}%
\end{pgfscope}%
\begin{pgfscope}%
\pgfpathrectangle{\pgfqpoint{0.041670in}{0.041670in}}{\pgfqpoint{2.216660in}{2.216660in}}%
\pgfusepath{clip}%
\pgfsetbuttcap%
\pgfsetroundjoin%
\definecolor{currentfill}{rgb}{0.282327,0.094955,0.417331}%
\pgfsetfillcolor{currentfill}%
\pgfsetlinewidth{0.000000pt}%
\definecolor{currentstroke}{rgb}{0.000000,0.000000,0.000000}%
\pgfsetstrokecolor{currentstroke}%
\pgfsetdash{}{0pt}%
\pgfpathmoveto{\pgfqpoint{1.808416in}{0.983351in}}%
\pgfpathlineto{\pgfqpoint{1.812185in}{0.988687in}}%
\pgfpathlineto{\pgfqpoint{1.815967in}{0.994362in}}%
\pgfpathlineto{\pgfqpoint{1.819760in}{1.000380in}}%
\pgfpathlineto{\pgfqpoint{1.823567in}{1.006748in}}%
\pgfpathlineto{\pgfqpoint{1.828561in}{0.996284in}}%
\pgfpathlineto{\pgfqpoint{1.832929in}{0.985729in}}%
\pgfpathlineto{\pgfqpoint{1.836662in}{0.975092in}}%
\pgfpathlineto{\pgfqpoint{1.839754in}{0.964383in}}%
\pgfpathlineto{\pgfqpoint{1.835829in}{0.958229in}}%
\pgfpathlineto{\pgfqpoint{1.831917in}{0.952427in}}%
\pgfpathlineto{\pgfqpoint{1.828018in}{0.946970in}}%
\pgfpathlineto{\pgfqpoint{1.824132in}{0.941853in}}%
\pgfpathlineto{\pgfqpoint{1.821135in}{0.952341in}}%
\pgfpathlineto{\pgfqpoint{1.817513in}{0.962760in}}%
\pgfpathlineto{\pgfqpoint{1.813271in}{0.973100in}}%
\pgfpathlineto{\pgfqpoint{1.808416in}{0.983351in}}%
\pgfpathclose%
\pgfusepath{fill}%
\end{pgfscope}%
\begin{pgfscope}%
\pgfpathrectangle{\pgfqpoint{0.041670in}{0.041670in}}{\pgfqpoint{2.216660in}{2.216660in}}%
\pgfusepath{clip}%
\pgfsetbuttcap%
\pgfsetroundjoin%
\definecolor{currentfill}{rgb}{0.133743,0.548535,0.553541}%
\pgfsetfillcolor{currentfill}%
\pgfsetlinewidth{0.000000pt}%
\definecolor{currentstroke}{rgb}{0.000000,0.000000,0.000000}%
\pgfsetstrokecolor{currentstroke}%
\pgfsetdash{}{0pt}%
\pgfpathmoveto{\pgfqpoint{1.063768in}{1.403517in}}%
\pgfpathlineto{\pgfqpoint{1.062457in}{1.397656in}}%
\pgfpathlineto{\pgfqpoint{1.061148in}{1.391789in}}%
\pgfpathlineto{\pgfqpoint{1.059840in}{1.385918in}}%
\pgfpathlineto{\pgfqpoint{1.058533in}{1.380047in}}%
\pgfpathlineto{\pgfqpoint{1.068301in}{1.381834in}}%
\pgfpathlineto{\pgfqpoint{1.078172in}{1.383469in}}%
\pgfpathlineto{\pgfqpoint{1.088138in}{1.384953in}}%
\pgfpathlineto{\pgfqpoint{1.098189in}{1.386282in}}%
\pgfpathlineto{\pgfqpoint{1.099068in}{1.392081in}}%
\pgfpathlineto{\pgfqpoint{1.099948in}{1.397879in}}%
\pgfpathlineto{\pgfqpoint{1.100829in}{1.403673in}}%
\pgfpathlineto{\pgfqpoint{1.101711in}{1.409462in}}%
\pgfpathlineto{\pgfqpoint{1.092094in}{1.408195in}}%
\pgfpathlineto{\pgfqpoint{1.082558in}{1.406780in}}%
\pgfpathlineto{\pgfqpoint{1.073114in}{1.405221in}}%
\pgfpathlineto{\pgfqpoint{1.063768in}{1.403517in}}%
\pgfpathclose%
\pgfusepath{fill}%
\end{pgfscope}%
\begin{pgfscope}%
\pgfpathrectangle{\pgfqpoint{0.041670in}{0.041670in}}{\pgfqpoint{2.216660in}{2.216660in}}%
\pgfusepath{clip}%
\pgfsetbuttcap%
\pgfsetroundjoin%
\definecolor{currentfill}{rgb}{0.147607,0.511733,0.557049}%
\pgfsetfillcolor{currentfill}%
\pgfsetlinewidth{0.000000pt}%
\definecolor{currentstroke}{rgb}{0.000000,0.000000,0.000000}%
\pgfsetstrokecolor{currentstroke}%
\pgfsetdash{}{0pt}%
\pgfpathmoveto{\pgfqpoint{0.985212in}{1.360534in}}%
\pgfpathlineto{\pgfqpoint{0.983113in}{1.354438in}}%
\pgfpathlineto{\pgfqpoint{0.981017in}{1.348347in}}%
\pgfpathlineto{\pgfqpoint{0.978923in}{1.342264in}}%
\pgfpathlineto{\pgfqpoint{0.976832in}{1.336191in}}%
\pgfpathlineto{\pgfqpoint{0.985821in}{1.339255in}}%
\pgfpathlineto{\pgfqpoint{0.994992in}{1.342179in}}%
\pgfpathlineto{\pgfqpoint{1.004335in}{1.344958in}}%
\pgfpathlineto{\pgfqpoint{1.013842in}{1.347590in}}%
\pgfpathlineto{\pgfqpoint{1.015549in}{1.353536in}}%
\pgfpathlineto{\pgfqpoint{1.017259in}{1.359492in}}%
\pgfpathlineto{\pgfqpoint{1.018970in}{1.365456in}}%
\pgfpathlineto{\pgfqpoint{1.020683in}{1.371425in}}%
\pgfpathlineto{\pgfqpoint{1.011571in}{1.368910in}}%
\pgfpathlineto{\pgfqpoint{1.002616in}{1.366254in}}%
\pgfpathlineto{\pgfqpoint{0.993826in}{1.363462in}}%
\pgfpathlineto{\pgfqpoint{0.985212in}{1.360534in}}%
\pgfpathclose%
\pgfusepath{fill}%
\end{pgfscope}%
\begin{pgfscope}%
\pgfpathrectangle{\pgfqpoint{0.041670in}{0.041670in}}{\pgfqpoint{2.216660in}{2.216660in}}%
\pgfusepath{clip}%
\pgfsetbuttcap%
\pgfsetroundjoin%
\definecolor{currentfill}{rgb}{0.195860,0.395433,0.555276}%
\pgfsetfillcolor{currentfill}%
\pgfsetlinewidth{0.000000pt}%
\definecolor{currentstroke}{rgb}{0.000000,0.000000,0.000000}%
\pgfsetstrokecolor{currentstroke}%
\pgfsetdash{}{0pt}%
\pgfpathmoveto{\pgfqpoint{0.861854in}{1.237422in}}%
\pgfpathlineto{\pgfqpoint{0.858833in}{1.231047in}}%
\pgfpathlineto{\pgfqpoint{0.855815in}{1.224712in}}%
\pgfpathlineto{\pgfqpoint{0.852799in}{1.218420in}}%
\pgfpathlineto{\pgfqpoint{0.849785in}{1.212174in}}%
\pgfpathlineto{\pgfqpoint{0.856757in}{1.217306in}}%
\pgfpathlineto{\pgfqpoint{0.864036in}{1.222322in}}%
\pgfpathlineto{\pgfqpoint{0.871614in}{1.227218in}}%
\pgfpathlineto{\pgfqpoint{0.879483in}{1.231989in}}%
\pgfpathlineto{\pgfqpoint{0.882217in}{1.238037in}}%
\pgfpathlineto{\pgfqpoint{0.884954in}{1.244131in}}%
\pgfpathlineto{\pgfqpoint{0.887694in}{1.250268in}}%
\pgfpathlineto{\pgfqpoint{0.890435in}{1.256445in}}%
\pgfpathlineto{\pgfqpoint{0.882861in}{1.251864in}}%
\pgfpathlineto{\pgfqpoint{0.875567in}{1.247163in}}%
\pgfpathlineto{\pgfqpoint{0.868562in}{1.242348in}}%
\pgfpathlineto{\pgfqpoint{0.861854in}{1.237422in}}%
\pgfpathclose%
\pgfusepath{fill}%
\end{pgfscope}%
\begin{pgfscope}%
\pgfpathrectangle{\pgfqpoint{0.041670in}{0.041670in}}{\pgfqpoint{2.216660in}{2.216660in}}%
\pgfusepath{clip}%
\pgfsetbuttcap%
\pgfsetroundjoin%
\definecolor{currentfill}{rgb}{0.248629,0.278775,0.534556}%
\pgfsetfillcolor{currentfill}%
\pgfsetlinewidth{0.000000pt}%
\definecolor{currentstroke}{rgb}{0.000000,0.000000,0.000000}%
\pgfsetstrokecolor{currentstroke}%
\pgfsetdash{}{0pt}%
\pgfpathmoveto{\pgfqpoint{1.555305in}{1.146248in}}%
\pgfpathlineto{\pgfqpoint{1.558489in}{1.140392in}}%
\pgfpathlineto{\pgfqpoint{1.561670in}{1.134614in}}%
\pgfpathlineto{\pgfqpoint{1.564850in}{1.128917in}}%
\pgfpathlineto{\pgfqpoint{1.568029in}{1.123306in}}%
\pgfpathlineto{\pgfqpoint{1.574318in}{1.117076in}}%
\pgfpathlineto{\pgfqpoint{1.580230in}{1.110743in}}%
\pgfpathlineto{\pgfqpoint{1.585758in}{1.104313in}}%
\pgfpathlineto{\pgfqpoint{1.590896in}{1.097792in}}%
\pgfpathlineto{\pgfqpoint{1.587518in}{1.103634in}}%
\pgfpathlineto{\pgfqpoint{1.584138in}{1.109560in}}%
\pgfpathlineto{\pgfqpoint{1.580757in}{1.115569in}}%
\pgfpathlineto{\pgfqpoint{1.577375in}{1.121655in}}%
\pgfpathlineto{\pgfqpoint{1.572418in}{1.127940in}}%
\pgfpathlineto{\pgfqpoint{1.567083in}{1.134138in}}%
\pgfpathlineto{\pgfqpoint{1.561377in}{1.140242in}}%
\pgfpathlineto{\pgfqpoint{1.555305in}{1.146248in}}%
\pgfpathclose%
\pgfusepath{fill}%
\end{pgfscope}%
\begin{pgfscope}%
\pgfpathrectangle{\pgfqpoint{0.041670in}{0.041670in}}{\pgfqpoint{2.216660in}{2.216660in}}%
\pgfusepath{clip}%
\pgfsetbuttcap%
\pgfsetroundjoin%
\definecolor{currentfill}{rgb}{0.271305,0.019942,0.347269}%
\pgfsetfillcolor{currentfill}%
\pgfsetlinewidth{0.000000pt}%
\definecolor{currentstroke}{rgb}{0.000000,0.000000,0.000000}%
\pgfsetstrokecolor{currentstroke}%
\pgfsetdash{}{0pt}%
\pgfpathmoveto{\pgfqpoint{0.639217in}{0.906876in}}%
\pgfpathlineto{\pgfqpoint{0.635535in}{0.904548in}}%
\pgfpathlineto{\pgfqpoint{0.631849in}{0.902420in}}%
\pgfpathlineto{\pgfqpoint{0.628158in}{0.900496in}}%
\pgfpathlineto{\pgfqpoint{0.624463in}{0.898782in}}%
\pgfpathlineto{\pgfqpoint{0.626441in}{0.907886in}}%
\pgfpathlineto{\pgfqpoint{0.628964in}{0.916943in}}%
\pgfpathlineto{\pgfqpoint{0.632026in}{0.925942in}}%
\pgfpathlineto{\pgfqpoint{0.635623in}{0.934875in}}%
\pgfpathlineto{\pgfqpoint{0.639226in}{0.936342in}}%
\pgfpathlineto{\pgfqpoint{0.642825in}{0.938018in}}%
\pgfpathlineto{\pgfqpoint{0.646420in}{0.939897in}}%
\pgfpathlineto{\pgfqpoint{0.650010in}{0.941976in}}%
\pgfpathlineto{\pgfqpoint{0.646526in}{0.933288in}}%
\pgfpathlineto{\pgfqpoint{0.643563in}{0.924536in}}%
\pgfpathlineto{\pgfqpoint{0.641125in}{0.915730in}}%
\pgfpathlineto{\pgfqpoint{0.639217in}{0.906876in}}%
\pgfpathclose%
\pgfusepath{fill}%
\end{pgfscope}%
\begin{pgfscope}%
\pgfpathrectangle{\pgfqpoint{0.041670in}{0.041670in}}{\pgfqpoint{2.216660in}{2.216660in}}%
\pgfusepath{clip}%
\pgfsetbuttcap%
\pgfsetroundjoin%
\definecolor{currentfill}{rgb}{0.277941,0.056324,0.381191}%
\pgfsetfillcolor{currentfill}%
\pgfsetlinewidth{0.000000pt}%
\definecolor{currentstroke}{rgb}{0.000000,0.000000,0.000000}%
\pgfsetstrokecolor{currentstroke}%
\pgfsetdash{}{0pt}%
\pgfpathmoveto{\pgfqpoint{0.549145in}{0.915496in}}%
\pgfpathlineto{\pgfqpoint{0.545287in}{0.919261in}}%
\pgfpathlineto{\pgfqpoint{0.541418in}{0.923342in}}%
\pgfpathlineto{\pgfqpoint{0.537537in}{0.927747in}}%
\pgfpathlineto{\pgfqpoint{0.533645in}{0.932480in}}%
\pgfpathlineto{\pgfqpoint{0.536080in}{0.943021in}}%
\pgfpathlineto{\pgfqpoint{0.539146in}{0.953502in}}%
\pgfpathlineto{\pgfqpoint{0.542838in}{0.963913in}}%
\pgfpathlineto{\pgfqpoint{0.547148in}{0.974243in}}%
\pgfpathlineto{\pgfqpoint{0.550937in}{0.969288in}}%
\pgfpathlineto{\pgfqpoint{0.554715in}{0.964659in}}%
\pgfpathlineto{\pgfqpoint{0.558482in}{0.960352in}}%
\pgfpathlineto{\pgfqpoint{0.562239in}{0.956361in}}%
\pgfpathlineto{\pgfqpoint{0.558052in}{0.946251in}}%
\pgfpathlineto{\pgfqpoint{0.554471in}{0.936065in}}%
\pgfpathlineto{\pgfqpoint{0.551500in}{0.925810in}}%
\pgfpathlineto{\pgfqpoint{0.549145in}{0.915496in}}%
\pgfpathclose%
\pgfusepath{fill}%
\end{pgfscope}%
\begin{pgfscope}%
\pgfpathrectangle{\pgfqpoint{0.041670in}{0.041670in}}{\pgfqpoint{2.216660in}{2.216660in}}%
\pgfusepath{clip}%
\pgfsetbuttcap%
\pgfsetroundjoin%
\definecolor{currentfill}{rgb}{0.280255,0.165693,0.476498}%
\pgfsetfillcolor{currentfill}%
\pgfsetlinewidth{0.000000pt}%
\definecolor{currentstroke}{rgb}{0.000000,0.000000,0.000000}%
\pgfsetstrokecolor{currentstroke}%
\pgfsetdash{}{0pt}%
\pgfpathmoveto{\pgfqpoint{0.721287in}{1.019150in}}%
\pgfpathlineto{\pgfqpoint{0.717736in}{1.013931in}}%
\pgfpathlineto{\pgfqpoint{0.714185in}{1.008832in}}%
\pgfpathlineto{\pgfqpoint{0.710633in}{1.003857in}}%
\pgfpathlineto{\pgfqpoint{0.707080in}{0.999011in}}%
\pgfpathlineto{\pgfqpoint{0.710588in}{1.006639in}}%
\pgfpathlineto{\pgfqpoint{0.714553in}{1.014199in}}%
\pgfpathlineto{\pgfqpoint{0.718968in}{1.021683in}}%
\pgfpathlineto{\pgfqpoint{0.723829in}{1.029085in}}%
\pgfpathlineto{\pgfqpoint{0.727242in}{1.033686in}}%
\pgfpathlineto{\pgfqpoint{0.730653in}{1.038415in}}%
\pgfpathlineto{\pgfqpoint{0.734065in}{1.043268in}}%
\pgfpathlineto{\pgfqpoint{0.737476in}{1.048241in}}%
\pgfpathlineto{\pgfqpoint{0.732775in}{1.041081in}}%
\pgfpathlineto{\pgfqpoint{0.728506in}{1.033841in}}%
\pgfpathlineto{\pgfqpoint{0.724675in}{1.026528in}}%
\pgfpathlineto{\pgfqpoint{0.721287in}{1.019150in}}%
\pgfpathclose%
\pgfusepath{fill}%
\end{pgfscope}%
\begin{pgfscope}%
\pgfpathrectangle{\pgfqpoint{0.041670in}{0.041670in}}{\pgfqpoint{2.216660in}{2.216660in}}%
\pgfusepath{clip}%
\pgfsetbuttcap%
\pgfsetroundjoin%
\definecolor{currentfill}{rgb}{0.212395,0.359683,0.551710}%
\pgfsetfillcolor{currentfill}%
\pgfsetlinewidth{0.000000pt}%
\definecolor{currentstroke}{rgb}{0.000000,0.000000,0.000000}%
\pgfsetstrokecolor{currentstroke}%
\pgfsetdash{}{0pt}%
\pgfpathmoveto{\pgfqpoint{1.503942in}{1.216741in}}%
\pgfpathlineto{\pgfqpoint{1.506895in}{1.210590in}}%
\pgfpathlineto{\pgfqpoint{1.509846in}{1.204491in}}%
\pgfpathlineto{\pgfqpoint{1.512794in}{1.198448in}}%
\pgfpathlineto{\pgfqpoint{1.515741in}{1.192463in}}%
\pgfpathlineto{\pgfqpoint{1.522940in}{1.187113in}}%
\pgfpathlineto{\pgfqpoint{1.529814in}{1.181647in}}%
\pgfpathlineto{\pgfqpoint{1.536355in}{1.176071in}}%
\pgfpathlineto{\pgfqpoint{1.542557in}{1.170389in}}%
\pgfpathlineto{\pgfqpoint{1.539365in}{1.176587in}}%
\pgfpathlineto{\pgfqpoint{1.536171in}{1.182844in}}%
\pgfpathlineto{\pgfqpoint{1.532975in}{1.189156in}}%
\pgfpathlineto{\pgfqpoint{1.529777in}{1.195520in}}%
\pgfpathlineto{\pgfqpoint{1.523804in}{1.200981in}}%
\pgfpathlineto{\pgfqpoint{1.517503in}{1.206342in}}%
\pgfpathlineto{\pgfqpoint{1.510880in}{1.211597in}}%
\pgfpathlineto{\pgfqpoint{1.503942in}{1.216741in}}%
\pgfpathclose%
\pgfusepath{fill}%
\end{pgfscope}%
\begin{pgfscope}%
\pgfpathrectangle{\pgfqpoint{0.041670in}{0.041670in}}{\pgfqpoint{2.216660in}{2.216660in}}%
\pgfusepath{clip}%
\pgfsetbuttcap%
\pgfsetroundjoin%
\definecolor{currentfill}{rgb}{0.279566,0.067836,0.391917}%
\pgfsetfillcolor{currentfill}%
\pgfsetlinewidth{0.000000pt}%
\definecolor{currentstroke}{rgb}{0.000000,0.000000,0.000000}%
\pgfsetstrokecolor{currentstroke}%
\pgfsetdash{}{0pt}%
\pgfpathmoveto{\pgfqpoint{1.677983in}{0.972517in}}%
\pgfpathlineto{\pgfqpoint{1.681522in}{0.969047in}}%
\pgfpathlineto{\pgfqpoint{1.685063in}{0.965743in}}%
\pgfpathlineto{\pgfqpoint{1.688607in}{0.962609in}}%
\pgfpathlineto{\pgfqpoint{1.692153in}{0.959650in}}%
\pgfpathlineto{\pgfqpoint{1.695972in}{0.951270in}}%
\pgfpathlineto{\pgfqpoint{1.699289in}{0.942821in}}%
\pgfpathlineto{\pgfqpoint{1.702099in}{0.934313in}}%
\pgfpathlineto{\pgfqpoint{1.704396in}{0.925752in}}%
\pgfpathlineto{\pgfqpoint{1.700748in}{0.928961in}}%
\pgfpathlineto{\pgfqpoint{1.697102in}{0.932345in}}%
\pgfpathlineto{\pgfqpoint{1.693460in}{0.935900in}}%
\pgfpathlineto{\pgfqpoint{1.689820in}{0.939621in}}%
\pgfpathlineto{\pgfqpoint{1.687604in}{0.947928in}}%
\pgfpathlineto{\pgfqpoint{1.684889in}{0.956184in}}%
\pgfpathlineto{\pgfqpoint{1.681680in}{0.964383in}}%
\pgfpathlineto{\pgfqpoint{1.677983in}{0.972517in}}%
\pgfpathclose%
\pgfusepath{fill}%
\end{pgfscope}%
\begin{pgfscope}%
\pgfpathrectangle{\pgfqpoint{0.041670in}{0.041670in}}{\pgfqpoint{2.216660in}{2.216660in}}%
\pgfusepath{clip}%
\pgfsetbuttcap%
\pgfsetroundjoin%
\definecolor{currentfill}{rgb}{0.133743,0.548535,0.553541}%
\pgfsetfillcolor{currentfill}%
\pgfsetlinewidth{0.000000pt}%
\definecolor{currentstroke}{rgb}{0.000000,0.000000,0.000000}%
\pgfsetstrokecolor{currentstroke}%
\pgfsetdash{}{0pt}%
\pgfpathmoveto{\pgfqpoint{1.287840in}{1.405038in}}%
\pgfpathlineto{\pgfqpoint{1.289057in}{1.399196in}}%
\pgfpathlineto{\pgfqpoint{1.290273in}{1.393347in}}%
\pgfpathlineto{\pgfqpoint{1.291487in}{1.387495in}}%
\pgfpathlineto{\pgfqpoint{1.292699in}{1.381643in}}%
\pgfpathlineto{\pgfqpoint{1.302455in}{1.379840in}}%
\pgfpathlineto{\pgfqpoint{1.312096in}{1.377887in}}%
\pgfpathlineto{\pgfqpoint{1.321615in}{1.375787in}}%
\pgfpathlineto{\pgfqpoint{1.331001in}{1.373542in}}%
\pgfpathlineto{\pgfqpoint{1.329374in}{1.379488in}}%
\pgfpathlineto{\pgfqpoint{1.327746in}{1.385434in}}%
\pgfpathlineto{\pgfqpoint{1.326115in}{1.391377in}}%
\pgfpathlineto{\pgfqpoint{1.324483in}{1.397314in}}%
\pgfpathlineto{\pgfqpoint{1.315504in}{1.399455in}}%
\pgfpathlineto{\pgfqpoint{1.306397in}{1.401457in}}%
\pgfpathlineto{\pgfqpoint{1.297173in}{1.403319in}}%
\pgfpathlineto{\pgfqpoint{1.287840in}{1.405038in}}%
\pgfpathclose%
\pgfusepath{fill}%
\end{pgfscope}%
\begin{pgfscope}%
\pgfpathrectangle{\pgfqpoint{0.041670in}{0.041670in}}{\pgfqpoint{2.216660in}{2.216660in}}%
\pgfusepath{clip}%
\pgfsetbuttcap%
\pgfsetroundjoin%
\definecolor{currentfill}{rgb}{0.233603,0.313828,0.543914}%
\pgfsetfillcolor{currentfill}%
\pgfsetlinewidth{0.000000pt}%
\definecolor{currentstroke}{rgb}{0.000000,0.000000,0.000000}%
\pgfsetstrokecolor{currentstroke}%
\pgfsetdash{}{0pt}%
\pgfpathmoveto{\pgfqpoint{0.484691in}{1.102886in}}%
\pgfpathlineto{\pgfqpoint{0.480652in}{1.114336in}}%
\pgfpathlineto{\pgfqpoint{0.476594in}{1.126228in}}%
\pgfpathlineto{\pgfqpoint{0.472518in}{1.138569in}}%
\pgfpathlineto{\pgfqpoint{0.468422in}{1.151367in}}%
\pgfpathlineto{\pgfqpoint{0.474083in}{1.162594in}}%
\pgfpathlineto{\pgfqpoint{0.480421in}{1.173707in}}%
\pgfpathlineto{\pgfqpoint{0.487426in}{1.184697in}}%
\pgfpathlineto{\pgfqpoint{0.495088in}{1.195554in}}%
\pgfpathlineto{\pgfqpoint{0.499006in}{1.182584in}}%
\pgfpathlineto{\pgfqpoint{0.502905in}{1.170068in}}%
\pgfpathlineto{\pgfqpoint{0.506787in}{1.157999in}}%
\pgfpathlineto{\pgfqpoint{0.510650in}{1.146370in}}%
\pgfpathlineto{\pgfqpoint{0.503186in}{1.135684in}}%
\pgfpathlineto{\pgfqpoint{0.496365in}{1.124868in}}%
\pgfpathlineto{\pgfqpoint{0.490197in}{1.113932in}}%
\pgfpathlineto{\pgfqpoint{0.484691in}{1.102886in}}%
\pgfpathclose%
\pgfusepath{fill}%
\end{pgfscope}%
\begin{pgfscope}%
\pgfpathrectangle{\pgfqpoint{0.041670in}{0.041670in}}{\pgfqpoint{2.216660in}{2.216660in}}%
\pgfusepath{clip}%
\pgfsetbuttcap%
\pgfsetroundjoin%
\definecolor{currentfill}{rgb}{0.274952,0.037752,0.364543}%
\pgfsetfillcolor{currentfill}%
\pgfsetlinewidth{0.000000pt}%
\definecolor{currentstroke}{rgb}{0.000000,0.000000,0.000000}%
\pgfsetstrokecolor{currentstroke}%
\pgfsetdash{}{0pt}%
\pgfpathmoveto{\pgfqpoint{0.653905in}{0.918106in}}%
\pgfpathlineto{\pgfqpoint{0.650239in}{0.915020in}}%
\pgfpathlineto{\pgfqpoint{0.646568in}{0.912117in}}%
\pgfpathlineto{\pgfqpoint{0.642895in}{0.909401in}}%
\pgfpathlineto{\pgfqpoint{0.639217in}{0.906876in}}%
\pgfpathlineto{\pgfqpoint{0.641125in}{0.915730in}}%
\pgfpathlineto{\pgfqpoint{0.643563in}{0.924536in}}%
\pgfpathlineto{\pgfqpoint{0.646526in}{0.933288in}}%
\pgfpathlineto{\pgfqpoint{0.650010in}{0.941976in}}%
\pgfpathlineto{\pgfqpoint{0.653597in}{0.944251in}}%
\pgfpathlineto{\pgfqpoint{0.657180in}{0.946716in}}%
\pgfpathlineto{\pgfqpoint{0.660760in}{0.949369in}}%
\pgfpathlineto{\pgfqpoint{0.664337in}{0.952204in}}%
\pgfpathlineto{\pgfqpoint{0.660964in}{0.943763in}}%
\pgfpathlineto{\pgfqpoint{0.658098in}{0.935261in}}%
\pgfpathlineto{\pgfqpoint{0.655744in}{0.926706in}}%
\pgfpathlineto{\pgfqpoint{0.653905in}{0.918106in}}%
\pgfpathclose%
\pgfusepath{fill}%
\end{pgfscope}%
\begin{pgfscope}%
\pgfpathrectangle{\pgfqpoint{0.041670in}{0.041670in}}{\pgfqpoint{2.216660in}{2.216660in}}%
\pgfusepath{clip}%
\pgfsetbuttcap%
\pgfsetroundjoin%
\definecolor{currentfill}{rgb}{0.282884,0.135920,0.453427}%
\pgfsetfillcolor{currentfill}%
\pgfsetlinewidth{0.000000pt}%
\definecolor{currentstroke}{rgb}{0.000000,0.000000,0.000000}%
\pgfsetstrokecolor{currentstroke}%
\pgfsetdash{}{0pt}%
\pgfpathmoveto{\pgfqpoint{1.823567in}{1.006748in}}%
\pgfpathlineto{\pgfqpoint{1.827385in}{1.013472in}}%
\pgfpathlineto{\pgfqpoint{1.831218in}{1.020558in}}%
\pgfpathlineto{\pgfqpoint{1.835064in}{1.028011in}}%
\pgfpathlineto{\pgfqpoint{1.838924in}{1.035839in}}%
\pgfpathlineto{\pgfqpoint{1.844061in}{1.025169in}}%
\pgfpathlineto{\pgfqpoint{1.848556in}{1.014405in}}%
\pgfpathlineto{\pgfqpoint{1.852403in}{1.003557in}}%
\pgfpathlineto{\pgfqpoint{1.855594in}{0.992635in}}%
\pgfpathlineto{\pgfqpoint{1.851612in}{0.985015in}}%
\pgfpathlineto{\pgfqpoint{1.847645in}{0.977770in}}%
\pgfpathlineto{\pgfqpoint{1.843693in}{0.970895in}}%
\pgfpathlineto{\pgfqpoint{1.839754in}{0.964383in}}%
\pgfpathlineto{\pgfqpoint{1.836662in}{0.975092in}}%
\pgfpathlineto{\pgfqpoint{1.832929in}{0.985729in}}%
\pgfpathlineto{\pgfqpoint{1.828561in}{0.996284in}}%
\pgfpathlineto{\pgfqpoint{1.823567in}{1.006748in}}%
\pgfpathclose%
\pgfusepath{fill}%
\end{pgfscope}%
\begin{pgfscope}%
\pgfpathrectangle{\pgfqpoint{0.041670in}{0.041670in}}{\pgfqpoint{2.216660in}{2.216660in}}%
\pgfusepath{clip}%
\pgfsetbuttcap%
\pgfsetroundjoin%
\definecolor{currentfill}{rgb}{0.274128,0.199721,0.498911}%
\pgfsetfillcolor{currentfill}%
\pgfsetlinewidth{0.000000pt}%
\definecolor{currentstroke}{rgb}{0.000000,0.000000,0.000000}%
\pgfsetstrokecolor{currentstroke}%
\pgfsetdash{}{0pt}%
\pgfpathmoveto{\pgfqpoint{1.604400in}{1.075345in}}%
\pgfpathlineto{\pgfqpoint{1.607775in}{1.069980in}}%
\pgfpathlineto{\pgfqpoint{1.611149in}{1.064721in}}%
\pgfpathlineto{\pgfqpoint{1.614522in}{1.059571in}}%
\pgfpathlineto{\pgfqpoint{1.617896in}{1.054534in}}%
\pgfpathlineto{\pgfqpoint{1.622977in}{1.047450in}}%
\pgfpathlineto{\pgfqpoint{1.627631in}{1.040280in}}%
\pgfpathlineto{\pgfqpoint{1.631851in}{1.033032in}}%
\pgfpathlineto{\pgfqpoint{1.635633in}{1.025712in}}%
\pgfpathlineto{\pgfqpoint{1.632109in}{1.030992in}}%
\pgfpathlineto{\pgfqpoint{1.628585in}{1.036385in}}%
\pgfpathlineto{\pgfqpoint{1.625060in}{1.041888in}}%
\pgfpathlineto{\pgfqpoint{1.621536in}{1.047496in}}%
\pgfpathlineto{\pgfqpoint{1.617885in}{1.054569in}}%
\pgfpathlineto{\pgfqpoint{1.613808in}{1.061572in}}%
\pgfpathlineto{\pgfqpoint{1.609312in}{1.068499in}}%
\pgfpathlineto{\pgfqpoint{1.604400in}{1.075345in}}%
\pgfpathclose%
\pgfusepath{fill}%
\end{pgfscope}%
\begin{pgfscope}%
\pgfpathrectangle{\pgfqpoint{0.041670in}{0.041670in}}{\pgfqpoint{2.216660in}{2.216660in}}%
\pgfusepath{clip}%
\pgfsetbuttcap%
\pgfsetroundjoin%
\definecolor{currentfill}{rgb}{0.133743,0.548535,0.553541}%
\pgfsetfillcolor{currentfill}%
\pgfsetlinewidth{0.000000pt}%
\definecolor{currentstroke}{rgb}{0.000000,0.000000,0.000000}%
\pgfsetstrokecolor{currentstroke}%
\pgfsetdash{}{0pt}%
\pgfpathmoveto{\pgfqpoint{1.027559in}{1.395296in}}%
\pgfpathlineto{\pgfqpoint{1.025837in}{1.389334in}}%
\pgfpathlineto{\pgfqpoint{1.024117in}{1.383367in}}%
\pgfpathlineto{\pgfqpoint{1.022399in}{1.377396in}}%
\pgfpathlineto{\pgfqpoint{1.020683in}{1.371425in}}%
\pgfpathlineto{\pgfqpoint{1.029945in}{1.373799in}}%
\pgfpathlineto{\pgfqpoint{1.039346in}{1.376028in}}%
\pgfpathlineto{\pgfqpoint{1.048879in}{1.378112in}}%
\pgfpathlineto{\pgfqpoint{1.058533in}{1.380047in}}%
\pgfpathlineto{\pgfqpoint{1.059840in}{1.385918in}}%
\pgfpathlineto{\pgfqpoint{1.061148in}{1.391789in}}%
\pgfpathlineto{\pgfqpoint{1.062457in}{1.397656in}}%
\pgfpathlineto{\pgfqpoint{1.063768in}{1.403517in}}%
\pgfpathlineto{\pgfqpoint{1.054532in}{1.401671in}}%
\pgfpathlineto{\pgfqpoint{1.045412in}{1.399685in}}%
\pgfpathlineto{\pgfqpoint{1.036418in}{1.397559in}}%
\pgfpathlineto{\pgfqpoint{1.027559in}{1.395296in}}%
\pgfpathclose%
\pgfusepath{fill}%
\end{pgfscope}%
\begin{pgfscope}%
\pgfpathrectangle{\pgfqpoint{0.041670in}{0.041670in}}{\pgfqpoint{2.216660in}{2.216660in}}%
\pgfusepath{clip}%
\pgfsetbuttcap%
\pgfsetroundjoin%
\definecolor{currentfill}{rgb}{0.248629,0.278775,0.534556}%
\pgfsetfillcolor{currentfill}%
\pgfsetlinewidth{0.000000pt}%
\definecolor{currentstroke}{rgb}{0.000000,0.000000,0.000000}%
\pgfsetstrokecolor{currentstroke}%
\pgfsetdash{}{0pt}%
\pgfpathmoveto{\pgfqpoint{0.778450in}{1.116000in}}%
\pgfpathlineto{\pgfqpoint{0.775031in}{1.109860in}}%
\pgfpathlineto{\pgfqpoint{0.771612in}{1.103799in}}%
\pgfpathlineto{\pgfqpoint{0.768195in}{1.097819in}}%
\pgfpathlineto{\pgfqpoint{0.764779in}{1.091924in}}%
\pgfpathlineto{\pgfqpoint{0.769565in}{1.098521in}}%
\pgfpathlineto{\pgfqpoint{0.774747in}{1.105032in}}%
\pgfpathlineto{\pgfqpoint{0.780318in}{1.111452in}}%
\pgfpathlineto{\pgfqpoint{0.786272in}{1.117773in}}%
\pgfpathlineto{\pgfqpoint{0.789499in}{1.123435in}}%
\pgfpathlineto{\pgfqpoint{0.792728in}{1.129181in}}%
\pgfpathlineto{\pgfqpoint{0.795958in}{1.135009in}}%
\pgfpathlineto{\pgfqpoint{0.799190in}{1.140915in}}%
\pgfpathlineto{\pgfqpoint{0.793442in}{1.134821in}}%
\pgfpathlineto{\pgfqpoint{0.788066in}{1.128633in}}%
\pgfpathlineto{\pgfqpoint{0.783067in}{1.122358in}}%
\pgfpathlineto{\pgfqpoint{0.778450in}{1.116000in}}%
\pgfpathclose%
\pgfusepath{fill}%
\end{pgfscope}%
\begin{pgfscope}%
\pgfpathrectangle{\pgfqpoint{0.041670in}{0.041670in}}{\pgfqpoint{2.216660in}{2.216660in}}%
\pgfusepath{clip}%
\pgfsetbuttcap%
\pgfsetroundjoin%
\definecolor{currentfill}{rgb}{0.282327,0.094955,0.417331}%
\pgfsetfillcolor{currentfill}%
\pgfsetlinewidth{0.000000pt}%
\definecolor{currentstroke}{rgb}{0.000000,0.000000,0.000000}%
\pgfsetstrokecolor{currentstroke}%
\pgfsetdash{}{0pt}%
\pgfpathmoveto{\pgfqpoint{0.533645in}{0.932480in}}%
\pgfpathlineto{\pgfqpoint{0.529740in}{0.937547in}}%
\pgfpathlineto{\pgfqpoint{0.525823in}{0.942954in}}%
\pgfpathlineto{\pgfqpoint{0.521893in}{0.948708in}}%
\pgfpathlineto{\pgfqpoint{0.517949in}{0.954813in}}%
\pgfpathlineto{\pgfqpoint{0.520467in}{0.965576in}}%
\pgfpathlineto{\pgfqpoint{0.523631in}{0.976277in}}%
\pgfpathlineto{\pgfqpoint{0.527435in}{0.986906in}}%
\pgfpathlineto{\pgfqpoint{0.531873in}{0.997451in}}%
\pgfpathlineto{\pgfqpoint{0.535710in}{0.991130in}}%
\pgfpathlineto{\pgfqpoint{0.539535in}{0.985159in}}%
\pgfpathlineto{\pgfqpoint{0.543348in}{0.979532in}}%
\pgfpathlineto{\pgfqpoint{0.547148in}{0.974243in}}%
\pgfpathlineto{\pgfqpoint{0.542838in}{0.963913in}}%
\pgfpathlineto{\pgfqpoint{0.539146in}{0.953502in}}%
\pgfpathlineto{\pgfqpoint{0.536080in}{0.943021in}}%
\pgfpathlineto{\pgfqpoint{0.533645in}{0.932480in}}%
\pgfpathclose%
\pgfusepath{fill}%
\end{pgfscope}%
\begin{pgfscope}%
\pgfpathrectangle{\pgfqpoint{0.041670in}{0.041670in}}{\pgfqpoint{2.216660in}{2.216660in}}%
\pgfusepath{clip}%
\pgfsetbuttcap%
\pgfsetroundjoin%
\definecolor{currentfill}{rgb}{0.212395,0.359683,0.551710}%
\pgfsetfillcolor{currentfill}%
\pgfsetlinewidth{0.000000pt}%
\definecolor{currentstroke}{rgb}{0.000000,0.000000,0.000000}%
\pgfsetstrokecolor{currentstroke}%
\pgfsetdash{}{0pt}%
\pgfpathmoveto{\pgfqpoint{0.825105in}{1.190585in}}%
\pgfpathlineto{\pgfqpoint{0.821859in}{1.184171in}}%
\pgfpathlineto{\pgfqpoint{0.818614in}{1.177809in}}%
\pgfpathlineto{\pgfqpoint{0.815372in}{1.171503in}}%
\pgfpathlineto{\pgfqpoint{0.812132in}{1.165255in}}%
\pgfpathlineto{\pgfqpoint{0.818025in}{1.171026in}}%
\pgfpathlineto{\pgfqpoint{0.824264in}{1.176696in}}%
\pgfpathlineto{\pgfqpoint{0.830843in}{1.182260in}}%
\pgfpathlineto{\pgfqpoint{0.837753in}{1.187713in}}%
\pgfpathlineto{\pgfqpoint{0.840758in}{1.193743in}}%
\pgfpathlineto{\pgfqpoint{0.843765in}{1.199833in}}%
\pgfpathlineto{\pgfqpoint{0.846774in}{1.205977in}}%
\pgfpathlineto{\pgfqpoint{0.849785in}{1.212174in}}%
\pgfpathlineto{\pgfqpoint{0.843127in}{1.206931in}}%
\pgfpathlineto{\pgfqpoint{0.836790in}{1.201582in}}%
\pgfpathlineto{\pgfqpoint{0.830780in}{1.196132in}}%
\pgfpathlineto{\pgfqpoint{0.825105in}{1.190585in}}%
\pgfpathclose%
\pgfusepath{fill}%
\end{pgfscope}%
\begin{pgfscope}%
\pgfpathrectangle{\pgfqpoint{0.041670in}{0.041670in}}{\pgfqpoint{2.216660in}{2.216660in}}%
\pgfusepath{clip}%
\pgfsetbuttcap%
\pgfsetroundjoin%
\definecolor{currentfill}{rgb}{0.163625,0.471133,0.558148}%
\pgfsetfillcolor{currentfill}%
\pgfsetlinewidth{0.000000pt}%
\definecolor{currentstroke}{rgb}{0.000000,0.000000,0.000000}%
\pgfsetstrokecolor{currentstroke}%
\pgfsetdash{}{0pt}%
\pgfpathmoveto{\pgfqpoint{1.409789in}{1.325783in}}%
\pgfpathlineto{\pgfqpoint{1.412156in}{1.319607in}}%
\pgfpathlineto{\pgfqpoint{1.414521in}{1.313447in}}%
\pgfpathlineto{\pgfqpoint{1.416883in}{1.307307in}}%
\pgfpathlineto{\pgfqpoint{1.419243in}{1.301188in}}%
\pgfpathlineto{\pgfqpoint{1.427737in}{1.297409in}}%
\pgfpathlineto{\pgfqpoint{1.435999in}{1.293498in}}%
\pgfpathlineto{\pgfqpoint{1.444019in}{1.289459in}}%
\pgfpathlineto{\pgfqpoint{1.451790in}{1.285294in}}%
\pgfpathlineto{\pgfqpoint{1.449104in}{1.291584in}}%
\pgfpathlineto{\pgfqpoint{1.446414in}{1.297896in}}%
\pgfpathlineto{\pgfqpoint{1.443722in}{1.304227in}}%
\pgfpathlineto{\pgfqpoint{1.441027in}{1.310574in}}%
\pgfpathlineto{\pgfqpoint{1.433569in}{1.314559in}}%
\pgfpathlineto{\pgfqpoint{1.425872in}{1.318424in}}%
\pgfpathlineto{\pgfqpoint{1.417943in}{1.322167in}}%
\pgfpathlineto{\pgfqpoint{1.409789in}{1.325783in}}%
\pgfpathclose%
\pgfusepath{fill}%
\end{pgfscope}%
\begin{pgfscope}%
\pgfpathrectangle{\pgfqpoint{0.041670in}{0.041670in}}{\pgfqpoint{2.216660in}{2.216660in}}%
\pgfusepath{clip}%
\pgfsetbuttcap%
\pgfsetroundjoin%
\definecolor{currentfill}{rgb}{0.122606,0.585371,0.546557}%
\pgfsetfillcolor{currentfill}%
\pgfsetlinewidth{0.000000pt}%
\definecolor{currentstroke}{rgb}{0.000000,0.000000,0.000000}%
\pgfsetstrokecolor{currentstroke}%
\pgfsetdash{}{0pt}%
\pgfpathmoveto{\pgfqpoint{1.142589in}{1.435919in}}%
\pgfpathlineto{\pgfqpoint{1.142145in}{1.430224in}}%
\pgfpathlineto{\pgfqpoint{1.141703in}{1.424512in}}%
\pgfpathlineto{\pgfqpoint{1.141260in}{1.418786in}}%
\pgfpathlineto{\pgfqpoint{1.140819in}{1.413049in}}%
\pgfpathlineto{\pgfqpoint{1.150709in}{1.413571in}}%
\pgfpathlineto{\pgfqpoint{1.160627in}{1.413941in}}%
\pgfpathlineto{\pgfqpoint{1.170563in}{1.414160in}}%
\pgfpathlineto{\pgfqpoint{1.180507in}{1.414228in}}%
\pgfpathlineto{\pgfqpoint{1.180501in}{1.419950in}}%
\pgfpathlineto{\pgfqpoint{1.180495in}{1.425662in}}%
\pgfpathlineto{\pgfqpoint{1.180489in}{1.431359in}}%
\pgfpathlineto{\pgfqpoint{1.180482in}{1.437040in}}%
\pgfpathlineto{\pgfqpoint{1.170988in}{1.436976in}}%
\pgfpathlineto{\pgfqpoint{1.161501in}{1.436768in}}%
\pgfpathlineto{\pgfqpoint{1.152032in}{1.436415in}}%
\pgfpathlineto{\pgfqpoint{1.142589in}{1.435919in}}%
\pgfpathclose%
\pgfusepath{fill}%
\end{pgfscope}%
\begin{pgfscope}%
\pgfpathrectangle{\pgfqpoint{0.041670in}{0.041670in}}{\pgfqpoint{2.216660in}{2.216660in}}%
\pgfusepath{clip}%
\pgfsetbuttcap%
\pgfsetroundjoin%
\definecolor{currentfill}{rgb}{0.122606,0.585371,0.546557}%
\pgfsetfillcolor{currentfill}%
\pgfsetlinewidth{0.000000pt}%
\definecolor{currentstroke}{rgb}{0.000000,0.000000,0.000000}%
\pgfsetstrokecolor{currentstroke}%
\pgfsetdash{}{0pt}%
\pgfpathmoveto{\pgfqpoint{1.180482in}{1.437040in}}%
\pgfpathlineto{\pgfqpoint{1.180489in}{1.431359in}}%
\pgfpathlineto{\pgfqpoint{1.180495in}{1.425662in}}%
\pgfpathlineto{\pgfqpoint{1.180501in}{1.419950in}}%
\pgfpathlineto{\pgfqpoint{1.180507in}{1.414228in}}%
\pgfpathlineto{\pgfqpoint{1.190451in}{1.414143in}}%
\pgfpathlineto{\pgfqpoint{1.200386in}{1.413907in}}%
\pgfpathlineto{\pgfqpoint{1.210301in}{1.413520in}}%
\pgfpathlineto{\pgfqpoint{1.220188in}{1.412982in}}%
\pgfpathlineto{\pgfqpoint{1.219734in}{1.418720in}}%
\pgfpathlineto{\pgfqpoint{1.219279in}{1.424446in}}%
\pgfpathlineto{\pgfqpoint{1.218824in}{1.430159in}}%
\pgfpathlineto{\pgfqpoint{1.218368in}{1.435855in}}%
\pgfpathlineto{\pgfqpoint{1.208928in}{1.436367in}}%
\pgfpathlineto{\pgfqpoint{1.199462in}{1.436736in}}%
\pgfpathlineto{\pgfqpoint{1.189977in}{1.436960in}}%
\pgfpathlineto{\pgfqpoint{1.180482in}{1.437040in}}%
\pgfpathclose%
\pgfusepath{fill}%
\end{pgfscope}%
\begin{pgfscope}%
\pgfpathrectangle{\pgfqpoint{0.041670in}{0.041670in}}{\pgfqpoint{2.216660in}{2.216660in}}%
\pgfusepath{clip}%
\pgfsetbuttcap%
\pgfsetroundjoin%
\definecolor{currentfill}{rgb}{0.147607,0.511733,0.557049}%
\pgfsetfillcolor{currentfill}%
\pgfsetlinewidth{0.000000pt}%
\definecolor{currentstroke}{rgb}{0.000000,0.000000,0.000000}%
\pgfsetstrokecolor{currentstroke}%
\pgfsetdash{}{0pt}%
\pgfpathmoveto{\pgfqpoint{1.367049in}{1.363143in}}%
\pgfpathlineto{\pgfqpoint{1.369065in}{1.357078in}}%
\pgfpathlineto{\pgfqpoint{1.371078in}{1.351017in}}%
\pgfpathlineto{\pgfqpoint{1.373088in}{1.344964in}}%
\pgfpathlineto{\pgfqpoint{1.375096in}{1.338922in}}%
\pgfpathlineto{\pgfqpoint{1.384065in}{1.335842in}}%
\pgfpathlineto{\pgfqpoint{1.392842in}{1.332624in}}%
\pgfpathlineto{\pgfqpoint{1.401420in}{1.329270in}}%
\pgfpathlineto{\pgfqpoint{1.409789in}{1.325783in}}%
\pgfpathlineto{\pgfqpoint{1.407419in}{1.331972in}}%
\pgfpathlineto{\pgfqpoint{1.405047in}{1.338172in}}%
\pgfpathlineto{\pgfqpoint{1.402672in}{1.344380in}}%
\pgfpathlineto{\pgfqpoint{1.400294in}{1.350592in}}%
\pgfpathlineto{\pgfqpoint{1.392274in}{1.353923in}}%
\pgfpathlineto{\pgfqpoint{1.384055in}{1.357126in}}%
\pgfpathlineto{\pgfqpoint{1.375644in}{1.360201in}}%
\pgfpathlineto{\pgfqpoint{1.367049in}{1.363143in}}%
\pgfpathclose%
\pgfusepath{fill}%
\end{pgfscope}%
\begin{pgfscope}%
\pgfpathrectangle{\pgfqpoint{0.041670in}{0.041670in}}{\pgfqpoint{2.216660in}{2.216660in}}%
\pgfusepath{clip}%
\pgfsetbuttcap%
\pgfsetroundjoin%
\definecolor{currentfill}{rgb}{0.282327,0.094955,0.417331}%
\pgfsetfillcolor{currentfill}%
\pgfsetlinewidth{0.000000pt}%
\definecolor{currentstroke}{rgb}{0.000000,0.000000,0.000000}%
\pgfsetstrokecolor{currentstroke}%
\pgfsetdash{}{0pt}%
\pgfpathmoveto{\pgfqpoint{1.663847in}{0.987982in}}%
\pgfpathlineto{\pgfqpoint{1.667378in}{0.983886in}}%
\pgfpathlineto{\pgfqpoint{1.670911in}{0.979941in}}%
\pgfpathlineto{\pgfqpoint{1.674446in}{0.976150in}}%
\pgfpathlineto{\pgfqpoint{1.677983in}{0.972517in}}%
\pgfpathlineto{\pgfqpoint{1.681680in}{0.964383in}}%
\pgfpathlineto{\pgfqpoint{1.684889in}{0.956184in}}%
\pgfpathlineto{\pgfqpoint{1.687604in}{0.947928in}}%
\pgfpathlineto{\pgfqpoint{1.689820in}{0.939621in}}%
\pgfpathlineto{\pgfqpoint{1.686183in}{0.943505in}}%
\pgfpathlineto{\pgfqpoint{1.682548in}{0.947548in}}%
\pgfpathlineto{\pgfqpoint{1.678914in}{0.951745in}}%
\pgfpathlineto{\pgfqpoint{1.675283in}{0.956094in}}%
\pgfpathlineto{\pgfqpoint{1.673146in}{0.964145in}}%
\pgfpathlineto{\pgfqpoint{1.670525in}{0.972149in}}%
\pgfpathlineto{\pgfqpoint{1.667424in}{0.980097in}}%
\pgfpathlineto{\pgfqpoint{1.663847in}{0.987982in}}%
\pgfpathclose%
\pgfusepath{fill}%
\end{pgfscope}%
\begin{pgfscope}%
\pgfpathrectangle{\pgfqpoint{0.041670in}{0.041670in}}{\pgfqpoint{2.216660in}{2.216660in}}%
\pgfusepath{clip}%
\pgfsetbuttcap%
\pgfsetroundjoin%
\definecolor{currentfill}{rgb}{0.201239,0.383670,0.554294}%
\pgfsetfillcolor{currentfill}%
\pgfsetlinewidth{0.000000pt}%
\definecolor{currentstroke}{rgb}{0.000000,0.000000,0.000000}%
\pgfsetstrokecolor{currentstroke}%
\pgfsetdash{}{0pt}%
\pgfpathmoveto{\pgfqpoint{1.857467in}{1.205086in}}%
\pgfpathlineto{\pgfqpoint{1.861356in}{1.218554in}}%
\pgfpathlineto{\pgfqpoint{1.865264in}{1.232490in}}%
\pgfpathlineto{\pgfqpoint{1.869191in}{1.246904in}}%
\pgfpathlineto{\pgfqpoint{1.873139in}{1.261801in}}%
\pgfpathlineto{\pgfqpoint{1.881593in}{1.250912in}}%
\pgfpathlineto{\pgfqpoint{1.889384in}{1.239878in}}%
\pgfpathlineto{\pgfqpoint{1.896503in}{1.228710in}}%
\pgfpathlineto{\pgfqpoint{1.902939in}{1.217417in}}%
\pgfpathlineto{\pgfqpoint{1.898795in}{1.202678in}}%
\pgfpathlineto{\pgfqpoint{1.894672in}{1.188426in}}%
\pgfpathlineto{\pgfqpoint{1.890571in}{1.174653in}}%
\pgfpathlineto{\pgfqpoint{1.886489in}{1.161352in}}%
\pgfpathlineto{\pgfqpoint{1.880226in}{1.172478in}}%
\pgfpathlineto{\pgfqpoint{1.873295in}{1.183482in}}%
\pgfpathlineto{\pgfqpoint{1.865705in}{1.194355in}}%
\pgfpathlineto{\pgfqpoint{1.857467in}{1.205086in}}%
\pgfpathclose%
\pgfusepath{fill}%
\end{pgfscope}%
\begin{pgfscope}%
\pgfpathrectangle{\pgfqpoint{0.041670in}{0.041670in}}{\pgfqpoint{2.216660in}{2.216660in}}%
\pgfusepath{clip}%
\pgfsetbuttcap%
\pgfsetroundjoin%
\definecolor{currentfill}{rgb}{0.179019,0.433756,0.557430}%
\pgfsetfillcolor{currentfill}%
\pgfsetlinewidth{0.000000pt}%
\definecolor{currentstroke}{rgb}{0.000000,0.000000,0.000000}%
\pgfsetstrokecolor{currentstroke}%
\pgfsetdash{}{0pt}%
\pgfpathmoveto{\pgfqpoint{1.451790in}{1.285294in}}%
\pgfpathlineto{\pgfqpoint{1.454475in}{1.279029in}}%
\pgfpathlineto{\pgfqpoint{1.457156in}{1.272792in}}%
\pgfpathlineto{\pgfqpoint{1.459835in}{1.266586in}}%
\pgfpathlineto{\pgfqpoint{1.462512in}{1.260413in}}%
\pgfpathlineto{\pgfqpoint{1.470330in}{1.255942in}}%
\pgfpathlineto{\pgfqpoint{1.477874in}{1.251347in}}%
\pgfpathlineto{\pgfqpoint{1.485135in}{1.246634in}}%
\pgfpathlineto{\pgfqpoint{1.492107in}{1.241806in}}%
\pgfpathlineto{\pgfqpoint{1.489143in}{1.248172in}}%
\pgfpathlineto{\pgfqpoint{1.486175in}{1.254571in}}%
\pgfpathlineto{\pgfqpoint{1.483205in}{1.261002in}}%
\pgfpathlineto{\pgfqpoint{1.480232in}{1.267460in}}%
\pgfpathlineto{\pgfqpoint{1.473533in}{1.272087in}}%
\pgfpathlineto{\pgfqpoint{1.466555in}{1.276604in}}%
\pgfpathlineto{\pgfqpoint{1.459305in}{1.281008in}}%
\pgfpathlineto{\pgfqpoint{1.451790in}{1.285294in}}%
\pgfpathclose%
\pgfusepath{fill}%
\end{pgfscope}%
\begin{pgfscope}%
\pgfpathrectangle{\pgfqpoint{0.041670in}{0.041670in}}{\pgfqpoint{2.216660in}{2.216660in}}%
\pgfusepath{clip}%
\pgfsetbuttcap%
\pgfsetroundjoin%
\definecolor{currentfill}{rgb}{0.279566,0.067836,0.391917}%
\pgfsetfillcolor{currentfill}%
\pgfsetlinewidth{0.000000pt}%
\definecolor{currentstroke}{rgb}{0.000000,0.000000,0.000000}%
\pgfsetstrokecolor{currentstroke}%
\pgfsetdash{}{0pt}%
\pgfpathmoveto{\pgfqpoint{0.668541in}{0.932202in}}%
\pgfpathlineto{\pgfqpoint{0.664886in}{0.928424in}}%
\pgfpathlineto{\pgfqpoint{0.661229in}{0.924813in}}%
\pgfpathlineto{\pgfqpoint{0.657568in}{0.921372in}}%
\pgfpathlineto{\pgfqpoint{0.653905in}{0.918106in}}%
\pgfpathlineto{\pgfqpoint{0.655744in}{0.926706in}}%
\pgfpathlineto{\pgfqpoint{0.658098in}{0.935261in}}%
\pgfpathlineto{\pgfqpoint{0.660964in}{0.943763in}}%
\pgfpathlineto{\pgfqpoint{0.664337in}{0.952204in}}%
\pgfpathlineto{\pgfqpoint{0.667911in}{0.955218in}}%
\pgfpathlineto{\pgfqpoint{0.671482in}{0.958406in}}%
\pgfpathlineto{\pgfqpoint{0.675050in}{0.961765in}}%
\pgfpathlineto{\pgfqpoint{0.678616in}{0.965291in}}%
\pgfpathlineto{\pgfqpoint{0.675353in}{0.957098in}}%
\pgfpathlineto{\pgfqpoint{0.672583in}{0.948848in}}%
\pgfpathlineto{\pgfqpoint{0.670311in}{0.940546in}}%
\pgfpathlineto{\pgfqpoint{0.668541in}{0.932202in}}%
\pgfpathclose%
\pgfusepath{fill}%
\end{pgfscope}%
\begin{pgfscope}%
\pgfpathrectangle{\pgfqpoint{0.041670in}{0.041670in}}{\pgfqpoint{2.216660in}{2.216660in}}%
\pgfusepath{clip}%
\pgfsetbuttcap%
\pgfsetroundjoin%
\definecolor{currentfill}{rgb}{0.122606,0.585371,0.546557}%
\pgfsetfillcolor{currentfill}%
\pgfsetlinewidth{0.000000pt}%
\definecolor{currentstroke}{rgb}{0.000000,0.000000,0.000000}%
\pgfsetstrokecolor{currentstroke}%
\pgfsetdash{}{0pt}%
\pgfpathmoveto{\pgfqpoint{1.105252in}{1.432506in}}%
\pgfpathlineto{\pgfqpoint{1.104365in}{1.426768in}}%
\pgfpathlineto{\pgfqpoint{1.103479in}{1.421012in}}%
\pgfpathlineto{\pgfqpoint{1.102595in}{1.415243in}}%
\pgfpathlineto{\pgfqpoint{1.101711in}{1.409462in}}%
\pgfpathlineto{\pgfqpoint{1.111402in}{1.410583in}}%
\pgfpathlineto{\pgfqpoint{1.121156in}{1.411554in}}%
\pgfpathlineto{\pgfqpoint{1.130964in}{1.412377in}}%
\pgfpathlineto{\pgfqpoint{1.140819in}{1.413049in}}%
\pgfpathlineto{\pgfqpoint{1.141260in}{1.418786in}}%
\pgfpathlineto{\pgfqpoint{1.141703in}{1.424512in}}%
\pgfpathlineto{\pgfqpoint{1.142145in}{1.430224in}}%
\pgfpathlineto{\pgfqpoint{1.142589in}{1.435919in}}%
\pgfpathlineto{\pgfqpoint{1.133180in}{1.435279in}}%
\pgfpathlineto{\pgfqpoint{1.123816in}{1.434497in}}%
\pgfpathlineto{\pgfqpoint{1.114503in}{1.433572in}}%
\pgfpathlineto{\pgfqpoint{1.105252in}{1.432506in}}%
\pgfpathclose%
\pgfusepath{fill}%
\end{pgfscope}%
\begin{pgfscope}%
\pgfpathrectangle{\pgfqpoint{0.041670in}{0.041670in}}{\pgfqpoint{2.216660in}{2.216660in}}%
\pgfusepath{clip}%
\pgfsetbuttcap%
\pgfsetroundjoin%
\definecolor{currentfill}{rgb}{0.122606,0.585371,0.546557}%
\pgfsetfillcolor{currentfill}%
\pgfsetlinewidth{0.000000pt}%
\definecolor{currentstroke}{rgb}{0.000000,0.000000,0.000000}%
\pgfsetstrokecolor{currentstroke}%
\pgfsetdash{}{0pt}%
\pgfpathmoveto{\pgfqpoint{1.218368in}{1.435855in}}%
\pgfpathlineto{\pgfqpoint{1.218824in}{1.430159in}}%
\pgfpathlineto{\pgfqpoint{1.219279in}{1.424446in}}%
\pgfpathlineto{\pgfqpoint{1.219734in}{1.418720in}}%
\pgfpathlineto{\pgfqpoint{1.220188in}{1.412982in}}%
\pgfpathlineto{\pgfqpoint{1.230038in}{1.412293in}}%
\pgfpathlineto{\pgfqpoint{1.239841in}{1.411454in}}%
\pgfpathlineto{\pgfqpoint{1.249588in}{1.410466in}}%
\pgfpathlineto{\pgfqpoint{1.259271in}{1.409329in}}%
\pgfpathlineto{\pgfqpoint{1.258375in}{1.415111in}}%
\pgfpathlineto{\pgfqpoint{1.257478in}{1.420882in}}%
\pgfpathlineto{\pgfqpoint{1.256581in}{1.426639in}}%
\pgfpathlineto{\pgfqpoint{1.255682in}{1.432379in}}%
\pgfpathlineto{\pgfqpoint{1.246438in}{1.433461in}}%
\pgfpathlineto{\pgfqpoint{1.237132in}{1.434401in}}%
\pgfpathlineto{\pgfqpoint{1.227772in}{1.435200in}}%
\pgfpathlineto{\pgfqpoint{1.218368in}{1.435855in}}%
\pgfpathclose%
\pgfusepath{fill}%
\end{pgfscope}%
\begin{pgfscope}%
\pgfpathrectangle{\pgfqpoint{0.041670in}{0.041670in}}{\pgfqpoint{2.216660in}{2.216660in}}%
\pgfusepath{clip}%
\pgfsetbuttcap%
\pgfsetroundjoin%
\definecolor{currentfill}{rgb}{0.163625,0.471133,0.558148}%
\pgfsetfillcolor{currentfill}%
\pgfsetlinewidth{0.000000pt}%
\definecolor{currentstroke}{rgb}{0.000000,0.000000,0.000000}%
\pgfsetstrokecolor{currentstroke}%
\pgfsetdash{}{0pt}%
\pgfpathmoveto{\pgfqpoint{0.912462in}{1.306934in}}%
\pgfpathlineto{\pgfqpoint{0.909699in}{1.300546in}}%
\pgfpathlineto{\pgfqpoint{0.906939in}{1.294174in}}%
\pgfpathlineto{\pgfqpoint{0.904181in}{1.287821in}}%
\pgfpathlineto{\pgfqpoint{0.901427in}{1.281490in}}%
\pgfpathlineto{\pgfqpoint{0.908970in}{1.285763in}}%
\pgfpathlineto{\pgfqpoint{0.916770in}{1.289914in}}%
\pgfpathlineto{\pgfqpoint{0.924817in}{1.293939in}}%
\pgfpathlineto{\pgfqpoint{0.933105in}{1.297836in}}%
\pgfpathlineto{\pgfqpoint{0.935540in}{1.303990in}}%
\pgfpathlineto{\pgfqpoint{0.937978in}{1.310167in}}%
\pgfpathlineto{\pgfqpoint{0.940419in}{1.316363in}}%
\pgfpathlineto{\pgfqpoint{0.942862in}{1.322575in}}%
\pgfpathlineto{\pgfqpoint{0.934908in}{1.318846in}}%
\pgfpathlineto{\pgfqpoint{0.927184in}{1.314994in}}%
\pgfpathlineto{\pgfqpoint{0.919700in}{1.311022in}}%
\pgfpathlineto{\pgfqpoint{0.912462in}{1.306934in}}%
\pgfpathclose%
\pgfusepath{fill}%
\end{pgfscope}%
\begin{pgfscope}%
\pgfpathrectangle{\pgfqpoint{0.041670in}{0.041670in}}{\pgfqpoint{2.216660in}{2.216660in}}%
\pgfusepath{clip}%
\pgfsetbuttcap%
\pgfsetroundjoin%
\definecolor{currentfill}{rgb}{0.274128,0.199721,0.498911}%
\pgfsetfillcolor{currentfill}%
\pgfsetlinewidth{0.000000pt}%
\definecolor{currentstroke}{rgb}{0.000000,0.000000,0.000000}%
\pgfsetstrokecolor{currentstroke}%
\pgfsetdash{}{0pt}%
\pgfpathmoveto{\pgfqpoint{0.735490in}{1.041157in}}%
\pgfpathlineto{\pgfqpoint{0.731939in}{1.035493in}}%
\pgfpathlineto{\pgfqpoint{0.728388in}{1.029934in}}%
\pgfpathlineto{\pgfqpoint{0.724838in}{1.024485in}}%
\pgfpathlineto{\pgfqpoint{0.721287in}{1.019150in}}%
\pgfpathlineto{\pgfqpoint{0.724675in}{1.026528in}}%
\pgfpathlineto{\pgfqpoint{0.728506in}{1.033841in}}%
\pgfpathlineto{\pgfqpoint{0.732775in}{1.041081in}}%
\pgfpathlineto{\pgfqpoint{0.737476in}{1.048241in}}%
\pgfpathlineto{\pgfqpoint{0.740888in}{1.053331in}}%
\pgfpathlineto{\pgfqpoint{0.744299in}{1.058534in}}%
\pgfpathlineto{\pgfqpoint{0.747711in}{1.063846in}}%
\pgfpathlineto{\pgfqpoint{0.751123in}{1.069264in}}%
\pgfpathlineto{\pgfqpoint{0.746581in}{1.062345in}}%
\pgfpathlineto{\pgfqpoint{0.742457in}{1.055350in}}%
\pgfpathlineto{\pgfqpoint{0.738759in}{1.048285in}}%
\pgfpathlineto{\pgfqpoint{0.735490in}{1.041157in}}%
\pgfpathclose%
\pgfusepath{fill}%
\end{pgfscope}%
\begin{pgfscope}%
\pgfpathrectangle{\pgfqpoint{0.041670in}{0.041670in}}{\pgfqpoint{2.216660in}{2.216660in}}%
\pgfusepath{clip}%
\pgfsetbuttcap%
\pgfsetroundjoin%
\definecolor{currentfill}{rgb}{0.147607,0.511733,0.557049}%
\pgfsetfillcolor{currentfill}%
\pgfsetlinewidth{0.000000pt}%
\definecolor{currentstroke}{rgb}{0.000000,0.000000,0.000000}%
\pgfsetstrokecolor{currentstroke}%
\pgfsetdash{}{0pt}%
\pgfpathmoveto{\pgfqpoint{0.952662in}{1.347528in}}%
\pgfpathlineto{\pgfqpoint{0.950208in}{1.341280in}}%
\pgfpathlineto{\pgfqpoint{0.947756in}{1.335036in}}%
\pgfpathlineto{\pgfqpoint{0.945308in}{1.328800in}}%
\pgfpathlineto{\pgfqpoint{0.942862in}{1.322575in}}%
\pgfpathlineto{\pgfqpoint{0.951040in}{1.326177in}}%
\pgfpathlineto{\pgfqpoint{0.959433in}{1.329649in}}%
\pgfpathlineto{\pgfqpoint{0.968033in}{1.332988in}}%
\pgfpathlineto{\pgfqpoint{0.976832in}{1.336191in}}%
\pgfpathlineto{\pgfqpoint{0.978923in}{1.342264in}}%
\pgfpathlineto{\pgfqpoint{0.981017in}{1.348347in}}%
\pgfpathlineto{\pgfqpoint{0.983113in}{1.354438in}}%
\pgfpathlineto{\pgfqpoint{0.985212in}{1.360534in}}%
\pgfpathlineto{\pgfqpoint{0.976780in}{1.357475in}}%
\pgfpathlineto{\pgfqpoint{0.968539in}{1.354285in}}%
\pgfpathlineto{\pgfqpoint{0.960497in}{1.350968in}}%
\pgfpathlineto{\pgfqpoint{0.952662in}{1.347528in}}%
\pgfpathclose%
\pgfusepath{fill}%
\end{pgfscope}%
\begin{pgfscope}%
\pgfpathrectangle{\pgfqpoint{0.041670in}{0.041670in}}{\pgfqpoint{2.216660in}{2.216660in}}%
\pgfusepath{clip}%
\pgfsetbuttcap%
\pgfsetroundjoin%
\definecolor{currentfill}{rgb}{0.231674,0.318106,0.544834}%
\pgfsetfillcolor{currentfill}%
\pgfsetlinewidth{0.000000pt}%
\definecolor{currentstroke}{rgb}{0.000000,0.000000,0.000000}%
\pgfsetstrokecolor{currentstroke}%
\pgfsetdash{}{0pt}%
\pgfpathmoveto{\pgfqpoint{1.542557in}{1.170389in}}%
\pgfpathlineto{\pgfqpoint{1.545747in}{1.164253in}}%
\pgfpathlineto{\pgfqpoint{1.548935in}{1.158182in}}%
\pgfpathlineto{\pgfqpoint{1.552121in}{1.152179in}}%
\pgfpathlineto{\pgfqpoint{1.555305in}{1.146248in}}%
\pgfpathlineto{\pgfqpoint{1.561377in}{1.140242in}}%
\pgfpathlineto{\pgfqpoint{1.567083in}{1.134138in}}%
\pgfpathlineto{\pgfqpoint{1.572418in}{1.127940in}}%
\pgfpathlineto{\pgfqpoint{1.577375in}{1.121655in}}%
\pgfpathlineto{\pgfqpoint{1.573991in}{1.127816in}}%
\pgfpathlineto{\pgfqpoint{1.570606in}{1.134049in}}%
\pgfpathlineto{\pgfqpoint{1.567219in}{1.140351in}}%
\pgfpathlineto{\pgfqpoint{1.563830in}{1.146717in}}%
\pgfpathlineto{\pgfqpoint{1.559054in}{1.152766in}}%
\pgfpathlineto{\pgfqpoint{1.553912in}{1.158732in}}%
\pgfpathlineto{\pgfqpoint{1.548411in}{1.164608in}}%
\pgfpathlineto{\pgfqpoint{1.542557in}{1.170389in}}%
\pgfpathclose%
\pgfusepath{fill}%
\end{pgfscope}%
\begin{pgfscope}%
\pgfpathrectangle{\pgfqpoint{0.041670in}{0.041670in}}{\pgfqpoint{2.216660in}{2.216660in}}%
\pgfusepath{clip}%
\pgfsetbuttcap%
\pgfsetroundjoin%
\definecolor{currentfill}{rgb}{0.133743,0.548535,0.553541}%
\pgfsetfillcolor{currentfill}%
\pgfsetlinewidth{0.000000pt}%
\definecolor{currentstroke}{rgb}{0.000000,0.000000,0.000000}%
\pgfsetstrokecolor{currentstroke}%
\pgfsetdash{}{0pt}%
\pgfpathmoveto{\pgfqpoint{1.324483in}{1.397314in}}%
\pgfpathlineto{\pgfqpoint{1.326115in}{1.391377in}}%
\pgfpathlineto{\pgfqpoint{1.327746in}{1.385434in}}%
\pgfpathlineto{\pgfqpoint{1.329374in}{1.379488in}}%
\pgfpathlineto{\pgfqpoint{1.331001in}{1.373542in}}%
\pgfpathlineto{\pgfqpoint{1.340246in}{1.371152in}}%
\pgfpathlineto{\pgfqpoint{1.349342in}{1.368621in}}%
\pgfpathlineto{\pgfqpoint{1.358279in}{1.365951in}}%
\pgfpathlineto{\pgfqpoint{1.367049in}{1.363143in}}%
\pgfpathlineto{\pgfqpoint{1.365032in}{1.369211in}}%
\pgfpathlineto{\pgfqpoint{1.363011in}{1.375278in}}%
\pgfpathlineto{\pgfqpoint{1.360989in}{1.381342in}}%
\pgfpathlineto{\pgfqpoint{1.358964in}{1.387400in}}%
\pgfpathlineto{\pgfqpoint{1.350576in}{1.390077in}}%
\pgfpathlineto{\pgfqpoint{1.342027in}{1.392623in}}%
\pgfpathlineto{\pgfqpoint{1.333327in}{1.395036in}}%
\pgfpathlineto{\pgfqpoint{1.324483in}{1.397314in}}%
\pgfpathclose%
\pgfusepath{fill}%
\end{pgfscope}%
\begin{pgfscope}%
\pgfpathrectangle{\pgfqpoint{0.041670in}{0.041670in}}{\pgfqpoint{2.216660in}{2.216660in}}%
\pgfusepath{clip}%
\pgfsetbuttcap%
\pgfsetroundjoin%
\definecolor{currentfill}{rgb}{0.179019,0.433756,0.557430}%
\pgfsetfillcolor{currentfill}%
\pgfsetlinewidth{0.000000pt}%
\definecolor{currentstroke}{rgb}{0.000000,0.000000,0.000000}%
\pgfsetstrokecolor{currentstroke}%
\pgfsetdash{}{0pt}%
\pgfpathmoveto{\pgfqpoint{0.873963in}{1.263258in}}%
\pgfpathlineto{\pgfqpoint{0.870931in}{1.256754in}}%
\pgfpathlineto{\pgfqpoint{0.867903in}{1.250278in}}%
\pgfpathlineto{\pgfqpoint{0.864877in}{1.243833in}}%
\pgfpathlineto{\pgfqpoint{0.861854in}{1.237422in}}%
\pgfpathlineto{\pgfqpoint{0.868562in}{1.242348in}}%
\pgfpathlineto{\pgfqpoint{0.875567in}{1.247163in}}%
\pgfpathlineto{\pgfqpoint{0.882861in}{1.251864in}}%
\pgfpathlineto{\pgfqpoint{0.890435in}{1.256445in}}%
\pgfpathlineto{\pgfqpoint{0.893179in}{1.262658in}}%
\pgfpathlineto{\pgfqpoint{0.895926in}{1.268906in}}%
\pgfpathlineto{\pgfqpoint{0.898675in}{1.275184in}}%
\pgfpathlineto{\pgfqpoint{0.901427in}{1.281490in}}%
\pgfpathlineto{\pgfqpoint{0.894147in}{1.277099in}}%
\pgfpathlineto{\pgfqpoint{0.887138in}{1.272594in}}%
\pgfpathlineto{\pgfqpoint{0.880408in}{1.267979in}}%
\pgfpathlineto{\pgfqpoint{0.873963in}{1.263258in}}%
\pgfpathclose%
\pgfusepath{fill}%
\end{pgfscope}%
\begin{pgfscope}%
\pgfpathrectangle{\pgfqpoint{0.041670in}{0.041670in}}{\pgfqpoint{2.216660in}{2.216660in}}%
\pgfusepath{clip}%
\pgfsetbuttcap%
\pgfsetroundjoin%
\definecolor{currentfill}{rgb}{0.276194,0.190074,0.493001}%
\pgfsetfillcolor{currentfill}%
\pgfsetlinewidth{0.000000pt}%
\definecolor{currentstroke}{rgb}{0.000000,0.000000,0.000000}%
\pgfsetstrokecolor{currentstroke}%
\pgfsetdash{}{0pt}%
\pgfpathmoveto{\pgfqpoint{1.838924in}{1.035839in}}%
\pgfpathlineto{\pgfqpoint{1.842798in}{1.044048in}}%
\pgfpathlineto{\pgfqpoint{1.846687in}{1.052643in}}%
\pgfpathlineto{\pgfqpoint{1.850591in}{1.061632in}}%
\pgfpathlineto{\pgfqpoint{1.854511in}{1.071021in}}%
\pgfpathlineto{\pgfqpoint{1.859794in}{1.060152in}}%
\pgfpathlineto{\pgfqpoint{1.864421in}{1.049187in}}%
\pgfpathlineto{\pgfqpoint{1.868384in}{1.038136in}}%
\pgfpathlineto{\pgfqpoint{1.871676in}{1.027009in}}%
\pgfpathlineto{\pgfqpoint{1.867631in}{1.017819in}}%
\pgfpathlineto{\pgfqpoint{1.863603in}{1.009031in}}%
\pgfpathlineto{\pgfqpoint{1.859591in}{1.000639in}}%
\pgfpathlineto{\pgfqpoint{1.855594in}{0.992635in}}%
\pgfpathlineto{\pgfqpoint{1.852403in}{1.003557in}}%
\pgfpathlineto{\pgfqpoint{1.848556in}{1.014405in}}%
\pgfpathlineto{\pgfqpoint{1.844061in}{1.025169in}}%
\pgfpathlineto{\pgfqpoint{1.838924in}{1.035839in}}%
\pgfpathclose%
\pgfusepath{fill}%
\end{pgfscope}%
\begin{pgfscope}%
\pgfpathrectangle{\pgfqpoint{0.041670in}{0.041670in}}{\pgfqpoint{2.216660in}{2.216660in}}%
\pgfusepath{clip}%
\pgfsetbuttcap%
\pgfsetroundjoin%
\definecolor{currentfill}{rgb}{0.282884,0.135920,0.453427}%
\pgfsetfillcolor{currentfill}%
\pgfsetlinewidth{0.000000pt}%
\definecolor{currentstroke}{rgb}{0.000000,0.000000,0.000000}%
\pgfsetstrokecolor{currentstroke}%
\pgfsetdash{}{0pt}%
\pgfpathmoveto{\pgfqpoint{0.517949in}{0.954813in}}%
\pgfpathlineto{\pgfqpoint{0.513992in}{0.961276in}}%
\pgfpathlineto{\pgfqpoint{0.510021in}{0.968103in}}%
\pgfpathlineto{\pgfqpoint{0.506036in}{0.975300in}}%
\pgfpathlineto{\pgfqpoint{0.502035in}{0.982874in}}%
\pgfpathlineto{\pgfqpoint{0.504638in}{0.993852in}}%
\pgfpathlineto{\pgfqpoint{0.507902in}{1.004766in}}%
\pgfpathlineto{\pgfqpoint{0.511821in}{1.015606in}}%
\pgfpathlineto{\pgfqpoint{0.516388in}{1.026360in}}%
\pgfpathlineto{\pgfqpoint{0.520280in}{1.018577in}}%
\pgfpathlineto{\pgfqpoint{0.524158in}{1.011169in}}%
\pgfpathlineto{\pgfqpoint{0.528022in}{1.004129in}}%
\pgfpathlineto{\pgfqpoint{0.531873in}{0.997451in}}%
\pgfpathlineto{\pgfqpoint{0.527435in}{0.986906in}}%
\pgfpathlineto{\pgfqpoint{0.523631in}{0.976277in}}%
\pgfpathlineto{\pgfqpoint{0.520467in}{0.965576in}}%
\pgfpathlineto{\pgfqpoint{0.517949in}{0.954813in}}%
\pgfpathclose%
\pgfusepath{fill}%
\end{pgfscope}%
\begin{pgfscope}%
\pgfpathrectangle{\pgfqpoint{0.041670in}{0.041670in}}{\pgfqpoint{2.216660in}{2.216660in}}%
\pgfusepath{clip}%
\pgfsetbuttcap%
\pgfsetroundjoin%
\definecolor{currentfill}{rgb}{0.122606,0.585371,0.546557}%
\pgfsetfillcolor{currentfill}%
\pgfsetlinewidth{0.000000pt}%
\definecolor{currentstroke}{rgb}{0.000000,0.000000,0.000000}%
\pgfsetstrokecolor{currentstroke}%
\pgfsetdash{}{0pt}%
\pgfpathmoveto{\pgfqpoint{1.255682in}{1.432379in}}%
\pgfpathlineto{\pgfqpoint{1.256581in}{1.426639in}}%
\pgfpathlineto{\pgfqpoint{1.257478in}{1.420882in}}%
\pgfpathlineto{\pgfqpoint{1.258375in}{1.415111in}}%
\pgfpathlineto{\pgfqpoint{1.259271in}{1.409329in}}%
\pgfpathlineto{\pgfqpoint{1.268879in}{1.408045in}}%
\pgfpathlineto{\pgfqpoint{1.278405in}{1.406614in}}%
\pgfpathlineto{\pgfqpoint{1.287840in}{1.405038in}}%
\pgfpathlineto{\pgfqpoint{1.286621in}{1.410873in}}%
\pgfpathlineto{\pgfqpoint{1.285400in}{1.416696in}}%
\pgfpathlineto{\pgfqpoint{1.284178in}{1.422505in}}%
\pgfpathlineto{\pgfqpoint{1.282955in}{1.428297in}}%
\pgfpathlineto{\pgfqpoint{1.273949in}{1.429796in}}%
\pgfpathlineto{\pgfqpoint{1.264855in}{1.431157in}}%
\pgfpathlineto{\pgfqpoint{1.255682in}{1.432379in}}%
\pgfpathclose%
\pgfusepath{fill}%
\end{pgfscope}%
\begin{pgfscope}%
\pgfpathrectangle{\pgfqpoint{0.041670in}{0.041670in}}{\pgfqpoint{2.216660in}{2.216660in}}%
\pgfusepath{clip}%
\pgfsetbuttcap%
\pgfsetroundjoin%
\definecolor{currentfill}{rgb}{0.122606,0.585371,0.546557}%
\pgfsetfillcolor{currentfill}%
\pgfsetlinewidth{0.000000pt}%
\definecolor{currentstroke}{rgb}{0.000000,0.000000,0.000000}%
\pgfsetstrokecolor{currentstroke}%
\pgfsetdash{}{0pt}%
\pgfpathmoveto{\pgfqpoint{1.069030in}{1.426850in}}%
\pgfpathlineto{\pgfqpoint{1.067712in}{1.421039in}}%
\pgfpathlineto{\pgfqpoint{1.066396in}{1.415211in}}%
\pgfpathlineto{\pgfqpoint{1.065081in}{1.409370in}}%
\pgfpathlineto{\pgfqpoint{1.063768in}{1.403517in}}%
\pgfpathlineto{\pgfqpoint{1.073114in}{1.405221in}}%
\pgfpathlineto{\pgfqpoint{1.082558in}{1.406780in}}%
\pgfpathlineto{\pgfqpoint{1.092094in}{1.408195in}}%
\pgfpathlineto{\pgfqpoint{1.101711in}{1.409462in}}%
\pgfpathlineto{\pgfqpoint{1.102595in}{1.415243in}}%
\pgfpathlineto{\pgfqpoint{1.103479in}{1.421012in}}%
\pgfpathlineto{\pgfqpoint{1.104365in}{1.426768in}}%
\pgfpathlineto{\pgfqpoint{1.105252in}{1.432506in}}%
\pgfpathlineto{\pgfqpoint{1.096070in}{1.431300in}}%
\pgfpathlineto{\pgfqpoint{1.086967in}{1.429954in}}%
\pgfpathlineto{\pgfqpoint{1.077951in}{1.428471in}}%
\pgfpathlineto{\pgfqpoint{1.069030in}{1.426850in}}%
\pgfpathclose%
\pgfusepath{fill}%
\end{pgfscope}%
\begin{pgfscope}%
\pgfpathrectangle{\pgfqpoint{0.041670in}{0.041670in}}{\pgfqpoint{2.216660in}{2.216660in}}%
\pgfusepath{clip}%
\pgfsetbuttcap%
\pgfsetroundjoin%
\definecolor{currentfill}{rgb}{0.263663,0.237631,0.518762}%
\pgfsetfillcolor{currentfill}%
\pgfsetlinewidth{0.000000pt}%
\definecolor{currentstroke}{rgb}{0.000000,0.000000,0.000000}%
\pgfsetstrokecolor{currentstroke}%
\pgfsetdash{}{0pt}%
\pgfpathmoveto{\pgfqpoint{1.590896in}{1.097792in}}%
\pgfpathlineto{\pgfqpoint{1.594273in}{1.092039in}}%
\pgfpathlineto{\pgfqpoint{1.597650in}{1.086378in}}%
\pgfpathlineto{\pgfqpoint{1.601025in}{1.080812in}}%
\pgfpathlineto{\pgfqpoint{1.604400in}{1.075345in}}%
\pgfpathlineto{\pgfqpoint{1.609312in}{1.068499in}}%
\pgfpathlineto{\pgfqpoint{1.613808in}{1.061572in}}%
\pgfpathlineto{\pgfqpoint{1.617885in}{1.054569in}}%
\pgfpathlineto{\pgfqpoint{1.621536in}{1.047496in}}%
\pgfpathlineto{\pgfqpoint{1.618011in}{1.053207in}}%
\pgfpathlineto{\pgfqpoint{1.614485in}{1.059017in}}%
\pgfpathlineto{\pgfqpoint{1.610959in}{1.064923in}}%
\pgfpathlineto{\pgfqpoint{1.607432in}{1.070920in}}%
\pgfpathlineto{\pgfqpoint{1.603911in}{1.077744in}}%
\pgfpathlineto{\pgfqpoint{1.599978in}{1.084501in}}%
\pgfpathlineto{\pgfqpoint{1.595638in}{1.091186in}}%
\pgfpathlineto{\pgfqpoint{1.590896in}{1.097792in}}%
\pgfpathclose%
\pgfusepath{fill}%
\end{pgfscope}%
\begin{pgfscope}%
\pgfpathrectangle{\pgfqpoint{0.041670in}{0.041670in}}{\pgfqpoint{2.216660in}{2.216660in}}%
\pgfusepath{clip}%
\pgfsetbuttcap%
\pgfsetroundjoin%
\definecolor{currentfill}{rgb}{0.133743,0.548535,0.553541}%
\pgfsetfillcolor{currentfill}%
\pgfsetlinewidth{0.000000pt}%
\definecolor{currentstroke}{rgb}{0.000000,0.000000,0.000000}%
\pgfsetstrokecolor{currentstroke}%
\pgfsetdash{}{0pt}%
\pgfpathmoveto{\pgfqpoint{0.993631in}{1.384913in}}%
\pgfpathlineto{\pgfqpoint{0.991522in}{1.378825in}}%
\pgfpathlineto{\pgfqpoint{0.989416in}{1.372730in}}%
\pgfpathlineto{\pgfqpoint{0.987313in}{1.366633in}}%
\pgfpathlineto{\pgfqpoint{0.985212in}{1.360534in}}%
\pgfpathlineto{\pgfqpoint{0.993826in}{1.363462in}}%
\pgfpathlineto{\pgfqpoint{1.002616in}{1.366254in}}%
\pgfpathlineto{\pgfqpoint{1.011571in}{1.368910in}}%
\pgfpathlineto{\pgfqpoint{1.020683in}{1.371425in}}%
\pgfpathlineto{\pgfqpoint{1.022399in}{1.377396in}}%
\pgfpathlineto{\pgfqpoint{1.024117in}{1.383367in}}%
\pgfpathlineto{\pgfqpoint{1.025837in}{1.389334in}}%
\pgfpathlineto{\pgfqpoint{1.027559in}{1.395296in}}%
\pgfpathlineto{\pgfqpoint{1.018842in}{1.392898in}}%
\pgfpathlineto{\pgfqpoint{1.010276in}{1.390366in}}%
\pgfpathlineto{\pgfqpoint{1.001870in}{1.387704in}}%
\pgfpathlineto{\pgfqpoint{0.993631in}{1.384913in}}%
\pgfpathclose%
\pgfusepath{fill}%
\end{pgfscope}%
\begin{pgfscope}%
\pgfpathrectangle{\pgfqpoint{0.041670in}{0.041670in}}{\pgfqpoint{2.216660in}{2.216660in}}%
\pgfusepath{clip}%
\pgfsetbuttcap%
\pgfsetroundjoin%
\definecolor{currentfill}{rgb}{0.195860,0.395433,0.555276}%
\pgfsetfillcolor{currentfill}%
\pgfsetlinewidth{0.000000pt}%
\definecolor{currentstroke}{rgb}{0.000000,0.000000,0.000000}%
\pgfsetstrokecolor{currentstroke}%
\pgfsetdash{}{0pt}%
\pgfpathmoveto{\pgfqpoint{1.492107in}{1.241806in}}%
\pgfpathlineto{\pgfqpoint{1.495070in}{1.235477in}}%
\pgfpathlineto{\pgfqpoint{1.498030in}{1.229188in}}%
\pgfpathlineto{\pgfqpoint{1.500987in}{1.222941in}}%
\pgfpathlineto{\pgfqpoint{1.503942in}{1.216741in}}%
\pgfpathlineto{\pgfqpoint{1.510880in}{1.211597in}}%
\pgfpathlineto{\pgfqpoint{1.517503in}{1.206342in}}%
\pgfpathlineto{\pgfqpoint{1.523804in}{1.200981in}}%
\pgfpathlineto{\pgfqpoint{1.529777in}{1.195520in}}%
\pgfpathlineto{\pgfqpoint{1.526576in}{1.201933in}}%
\pgfpathlineto{\pgfqpoint{1.523373in}{1.208393in}}%
\pgfpathlineto{\pgfqpoint{1.520168in}{1.214895in}}%
\pgfpathlineto{\pgfqpoint{1.516960in}{1.221437in}}%
\pgfpathlineto{\pgfqpoint{1.511216in}{1.226678in}}%
\pgfpathlineto{\pgfqpoint{1.505155in}{1.231823in}}%
\pgfpathlineto{\pgfqpoint{1.498783in}{1.236868in}}%
\pgfpathlineto{\pgfqpoint{1.492107in}{1.241806in}}%
\pgfpathclose%
\pgfusepath{fill}%
\end{pgfscope}%
\begin{pgfscope}%
\pgfpathrectangle{\pgfqpoint{0.041670in}{0.041670in}}{\pgfqpoint{2.216660in}{2.216660in}}%
\pgfusepath{clip}%
\pgfsetbuttcap%
\pgfsetroundjoin%
\definecolor{currentfill}{rgb}{0.282327,0.094955,0.417331}%
\pgfsetfillcolor{currentfill}%
\pgfsetlinewidth{0.000000pt}%
\definecolor{currentstroke}{rgb}{0.000000,0.000000,0.000000}%
\pgfsetstrokecolor{currentstroke}%
\pgfsetdash{}{0pt}%
\pgfpathmoveto{\pgfqpoint{0.683137in}{0.948903in}}%
\pgfpathlineto{\pgfqpoint{0.679491in}{0.944497in}}%
\pgfpathlineto{\pgfqpoint{0.675843in}{0.940243in}}%
\pgfpathlineto{\pgfqpoint{0.672193in}{0.936143in}}%
\pgfpathlineto{\pgfqpoint{0.668541in}{0.932202in}}%
\pgfpathlineto{\pgfqpoint{0.670311in}{0.940546in}}%
\pgfpathlineto{\pgfqpoint{0.672583in}{0.948848in}}%
\pgfpathlineto{\pgfqpoint{0.675353in}{0.957098in}}%
\pgfpathlineto{\pgfqpoint{0.678616in}{0.965291in}}%
\pgfpathlineto{\pgfqpoint{0.682180in}{0.968978in}}%
\pgfpathlineto{\pgfqpoint{0.685742in}{0.972824in}}%
\pgfpathlineto{\pgfqpoint{0.689302in}{0.976825in}}%
\pgfpathlineto{\pgfqpoint{0.692860in}{0.980976in}}%
\pgfpathlineto{\pgfqpoint{0.689706in}{0.973035in}}%
\pgfpathlineto{\pgfqpoint{0.687031in}{0.965037in}}%
\pgfpathlineto{\pgfqpoint{0.684840in}{0.956990in}}%
\pgfpathlineto{\pgfqpoint{0.683137in}{0.948903in}}%
\pgfpathclose%
\pgfusepath{fill}%
\end{pgfscope}%
\begin{pgfscope}%
\pgfpathrectangle{\pgfqpoint{0.041670in}{0.041670in}}{\pgfqpoint{2.216660in}{2.216660in}}%
\pgfusepath{clip}%
\pgfsetbuttcap%
\pgfsetroundjoin%
\definecolor{currentfill}{rgb}{0.283072,0.130895,0.449241}%
\pgfsetfillcolor{currentfill}%
\pgfsetlinewidth{0.000000pt}%
\definecolor{currentstroke}{rgb}{0.000000,0.000000,0.000000}%
\pgfsetstrokecolor{currentstroke}%
\pgfsetdash{}{0pt}%
\pgfpathmoveto{\pgfqpoint{1.649734in}{1.005795in}}%
\pgfpathlineto{\pgfqpoint{1.653261in}{1.001135in}}%
\pgfpathlineto{\pgfqpoint{1.656788in}{0.996610in}}%
\pgfpathlineto{\pgfqpoint{1.660317in}{0.992225in}}%
\pgfpathlineto{\pgfqpoint{1.663847in}{0.987982in}}%
\pgfpathlineto{\pgfqpoint{1.667424in}{0.980097in}}%
\pgfpathlineto{\pgfqpoint{1.670525in}{0.972149in}}%
\pgfpathlineto{\pgfqpoint{1.673146in}{0.964145in}}%
\pgfpathlineto{\pgfqpoint{1.675283in}{0.956094in}}%
\pgfpathlineto{\pgfqpoint{1.671654in}{0.960589in}}%
\pgfpathlineto{\pgfqpoint{1.668026in}{0.965227in}}%
\pgfpathlineto{\pgfqpoint{1.664399in}{0.970005in}}%
\pgfpathlineto{\pgfqpoint{1.660774in}{0.974918in}}%
\pgfpathlineto{\pgfqpoint{1.658715in}{0.982713in}}%
\pgfpathlineto{\pgfqpoint{1.656186in}{0.990463in}}%
\pgfpathlineto{\pgfqpoint{1.653191in}{0.998159in}}%
\pgfpathlineto{\pgfqpoint{1.649734in}{1.005795in}}%
\pgfpathclose%
\pgfusepath{fill}%
\end{pgfscope}%
\begin{pgfscope}%
\pgfpathrectangle{\pgfqpoint{0.041670in}{0.041670in}}{\pgfqpoint{2.216660in}{2.216660in}}%
\pgfusepath{clip}%
\pgfsetbuttcap%
\pgfsetroundjoin%
\definecolor{currentfill}{rgb}{0.231674,0.318106,0.544834}%
\pgfsetfillcolor{currentfill}%
\pgfsetlinewidth{0.000000pt}%
\definecolor{currentstroke}{rgb}{0.000000,0.000000,0.000000}%
\pgfsetstrokecolor{currentstroke}%
\pgfsetdash{}{0pt}%
\pgfpathmoveto{\pgfqpoint{0.792145in}{1.141274in}}%
\pgfpathlineto{\pgfqpoint{0.788719in}{1.134855in}}%
\pgfpathlineto{\pgfqpoint{0.785294in}{1.128500in}}%
\pgfpathlineto{\pgfqpoint{0.781872in}{1.122214in}}%
\pgfpathlineto{\pgfqpoint{0.778450in}{1.116000in}}%
\pgfpathlineto{\pgfqpoint{0.783067in}{1.122358in}}%
\pgfpathlineto{\pgfqpoint{0.788066in}{1.128633in}}%
\pgfpathlineto{\pgfqpoint{0.793442in}{1.134821in}}%
\pgfpathlineto{\pgfqpoint{0.799190in}{1.140915in}}%
\pgfpathlineto{\pgfqpoint{0.802423in}{1.146896in}}%
\pgfpathlineto{\pgfqpoint{0.805657in}{1.152948in}}%
\pgfpathlineto{\pgfqpoint{0.808894in}{1.159069in}}%
\pgfpathlineto{\pgfqpoint{0.812132in}{1.165255in}}%
\pgfpathlineto{\pgfqpoint{0.806591in}{1.159389in}}%
\pgfpathlineto{\pgfqpoint{0.801409in}{1.153433in}}%
\pgfpathlineto{\pgfqpoint{0.796592in}{1.147393in}}%
\pgfpathlineto{\pgfqpoint{0.792145in}{1.141274in}}%
\pgfpathclose%
\pgfusepath{fill}%
\end{pgfscope}%
\begin{pgfscope}%
\pgfpathrectangle{\pgfqpoint{0.041670in}{0.041670in}}{\pgfqpoint{2.216660in}{2.216660in}}%
\pgfusepath{clip}%
\pgfsetbuttcap%
\pgfsetroundjoin%
\definecolor{currentfill}{rgb}{0.201239,0.383670,0.554294}%
\pgfsetfillcolor{currentfill}%
\pgfsetlinewidth{0.000000pt}%
\definecolor{currentstroke}{rgb}{0.000000,0.000000,0.000000}%
\pgfsetstrokecolor{currentstroke}%
\pgfsetdash{}{0pt}%
\pgfpathmoveto{\pgfqpoint{0.468422in}{1.151367in}}%
\pgfpathlineto{\pgfqpoint{0.464306in}{1.164630in}}%
\pgfpathlineto{\pgfqpoint{0.460169in}{1.178365in}}%
\pgfpathlineto{\pgfqpoint{0.456011in}{1.192580in}}%
\pgfpathlineto{\pgfqpoint{0.451832in}{1.207283in}}%
\pgfpathlineto{\pgfqpoint{0.457652in}{1.218678in}}%
\pgfpathlineto{\pgfqpoint{0.464164in}{1.229957in}}%
\pgfpathlineto{\pgfqpoint{0.471358in}{1.241111in}}%
\pgfpathlineto{\pgfqpoint{0.479224in}{1.252129in}}%
\pgfpathlineto{\pgfqpoint{0.483220in}{1.237266in}}%
\pgfpathlineto{\pgfqpoint{0.487196in}{1.222888in}}%
\pgfpathlineto{\pgfqpoint{0.491152in}{1.208986in}}%
\pgfpathlineto{\pgfqpoint{0.495088in}{1.195554in}}%
\pgfpathlineto{\pgfqpoint{0.487426in}{1.184697in}}%
\pgfpathlineto{\pgfqpoint{0.480421in}{1.173707in}}%
\pgfpathlineto{\pgfqpoint{0.474083in}{1.162594in}}%
\pgfpathlineto{\pgfqpoint{0.468422in}{1.151367in}}%
\pgfpathclose%
\pgfusepath{fill}%
\end{pgfscope}%
\begin{pgfscope}%
\pgfpathrectangle{\pgfqpoint{0.041670in}{0.041670in}}{\pgfqpoint{2.216660in}{2.216660in}}%
\pgfusepath{clip}%
\pgfsetbuttcap%
\pgfsetroundjoin%
\definecolor{currentfill}{rgb}{0.195860,0.395433,0.555276}%
\pgfsetfillcolor{currentfill}%
\pgfsetlinewidth{0.000000pt}%
\definecolor{currentstroke}{rgb}{0.000000,0.000000,0.000000}%
\pgfsetstrokecolor{currentstroke}%
\pgfsetdash{}{0pt}%
\pgfpathmoveto{\pgfqpoint{0.838115in}{1.216700in}}%
\pgfpathlineto{\pgfqpoint{0.834859in}{1.210109in}}%
\pgfpathlineto{\pgfqpoint{0.831605in}{1.203557in}}%
\pgfpathlineto{\pgfqpoint{0.828354in}{1.197048in}}%
\pgfpathlineto{\pgfqpoint{0.825105in}{1.190585in}}%
\pgfpathlineto{\pgfqpoint{0.830780in}{1.196132in}}%
\pgfpathlineto{\pgfqpoint{0.836790in}{1.201582in}}%
\pgfpathlineto{\pgfqpoint{0.843127in}{1.206931in}}%
\pgfpathlineto{\pgfqpoint{0.849785in}{1.212174in}}%
\pgfpathlineto{\pgfqpoint{0.852799in}{1.218420in}}%
\pgfpathlineto{\pgfqpoint{0.855815in}{1.224712in}}%
\pgfpathlineto{\pgfqpoint{0.858833in}{1.231047in}}%
\pgfpathlineto{\pgfqpoint{0.861854in}{1.237422in}}%
\pgfpathlineto{\pgfqpoint{0.855448in}{1.232389in}}%
\pgfpathlineto{\pgfqpoint{0.849352in}{1.227255in}}%
\pgfpathlineto{\pgfqpoint{0.843572in}{1.222024in}}%
\pgfpathlineto{\pgfqpoint{0.838115in}{1.216700in}}%
\pgfpathclose%
\pgfusepath{fill}%
\end{pgfscope}%
\begin{pgfscope}%
\pgfpathrectangle{\pgfqpoint{0.041670in}{0.041670in}}{\pgfqpoint{2.216660in}{2.216660in}}%
\pgfusepath{clip}%
\pgfsetbuttcap%
\pgfsetroundjoin%
\definecolor{currentfill}{rgb}{0.122606,0.585371,0.546557}%
\pgfsetfillcolor{currentfill}%
\pgfsetlinewidth{0.000000pt}%
\definecolor{currentstroke}{rgb}{0.000000,0.000000,0.000000}%
\pgfsetstrokecolor{currentstroke}%
\pgfsetdash{}{0pt}%
\pgfpathmoveto{\pgfqpoint{1.282955in}{1.428297in}}%
\pgfpathlineto{\pgfqpoint{1.284178in}{1.422505in}}%
\pgfpathlineto{\pgfqpoint{1.285400in}{1.416696in}}%
\pgfpathlineto{\pgfqpoint{1.286621in}{1.410873in}}%
\pgfpathlineto{\pgfqpoint{1.287840in}{1.405038in}}%
\pgfpathlineto{\pgfqpoint{1.297173in}{1.403319in}}%
\pgfpathlineto{\pgfqpoint{1.306397in}{1.401457in}}%
\pgfpathlineto{\pgfqpoint{1.315504in}{1.399455in}}%
\pgfpathlineto{\pgfqpoint{1.324483in}{1.397314in}}%
\pgfpathlineto{\pgfqpoint{1.322848in}{1.403242in}}%
\pgfpathlineto{\pgfqpoint{1.321211in}{1.409159in}}%
\pgfpathlineto{\pgfqpoint{1.319573in}{1.415062in}}%
\pgfpathlineto{\pgfqpoint{1.317932in}{1.420949in}}%
\pgfpathlineto{\pgfqpoint{1.309361in}{1.422986in}}%
\pgfpathlineto{\pgfqpoint{1.300669in}{1.424890in}}%
\pgfpathlineto{\pgfqpoint{1.291864in}{1.426662in}}%
\pgfpathlineto{\pgfqpoint{1.282955in}{1.428297in}}%
\pgfpathclose%
\pgfusepath{fill}%
\end{pgfscope}%
\begin{pgfscope}%
\pgfpathrectangle{\pgfqpoint{0.041670in}{0.041670in}}{\pgfqpoint{2.216660in}{2.216660in}}%
\pgfusepath{clip}%
\pgfsetbuttcap%
\pgfsetroundjoin%
\definecolor{currentfill}{rgb}{0.263663,0.237631,0.518762}%
\pgfsetfillcolor{currentfill}%
\pgfsetlinewidth{0.000000pt}%
\definecolor{currentstroke}{rgb}{0.000000,0.000000,0.000000}%
\pgfsetstrokecolor{currentstroke}%
\pgfsetdash{}{0pt}%
\pgfpathmoveto{\pgfqpoint{0.749698in}{1.064804in}}%
\pgfpathlineto{\pgfqpoint{0.746145in}{1.058751in}}%
\pgfpathlineto{\pgfqpoint{0.742593in}{1.052790in}}%
\pgfpathlineto{\pgfqpoint{0.739041in}{1.046924in}}%
\pgfpathlineto{\pgfqpoint{0.735490in}{1.041157in}}%
\pgfpathlineto{\pgfqpoint{0.738759in}{1.048285in}}%
\pgfpathlineto{\pgfqpoint{0.742457in}{1.055350in}}%
\pgfpathlineto{\pgfqpoint{0.746581in}{1.062345in}}%
\pgfpathlineto{\pgfqpoint{0.751123in}{1.069264in}}%
\pgfpathlineto{\pgfqpoint{0.754536in}{1.074784in}}%
\pgfpathlineto{\pgfqpoint{0.757950in}{1.080404in}}%
\pgfpathlineto{\pgfqpoint{0.761364in}{1.086118in}}%
\pgfpathlineto{\pgfqpoint{0.764779in}{1.091924in}}%
\pgfpathlineto{\pgfqpoint{0.760394in}{1.085248in}}%
\pgfpathlineto{\pgfqpoint{0.756415in}{1.078498in}}%
\pgfpathlineto{\pgfqpoint{0.752848in}{1.071681in}}%
\pgfpathlineto{\pgfqpoint{0.749698in}{1.064804in}}%
\pgfpathclose%
\pgfusepath{fill}%
\end{pgfscope}%
\begin{pgfscope}%
\pgfpathrectangle{\pgfqpoint{0.041670in}{0.041670in}}{\pgfqpoint{2.216660in}{2.216660in}}%
\pgfusepath{clip}%
\pgfsetbuttcap%
\pgfsetroundjoin%
\definecolor{currentfill}{rgb}{0.122606,0.585371,0.546557}%
\pgfsetfillcolor{currentfill}%
\pgfsetlinewidth{0.000000pt}%
\definecolor{currentstroke}{rgb}{0.000000,0.000000,0.000000}%
\pgfsetstrokecolor{currentstroke}%
\pgfsetdash{}{0pt}%
\pgfpathmoveto{\pgfqpoint{1.034468in}{1.419029in}}%
\pgfpathlineto{\pgfqpoint{1.032738in}{1.413118in}}%
\pgfpathlineto{\pgfqpoint{1.031009in}{1.407190in}}%
\pgfpathlineto{\pgfqpoint{1.029283in}{1.401249in}}%
\pgfpathlineto{\pgfqpoint{1.027559in}{1.395296in}}%
\pgfpathlineto{\pgfqpoint{1.036418in}{1.397559in}}%
\pgfpathlineto{\pgfqpoint{1.045412in}{1.399685in}}%
\pgfpathlineto{\pgfqpoint{1.054532in}{1.401671in}}%
\pgfpathlineto{\pgfqpoint{1.063768in}{1.403517in}}%
\pgfpathlineto{\pgfqpoint{1.065081in}{1.409370in}}%
\pgfpathlineto{\pgfqpoint{1.066396in}{1.415211in}}%
\pgfpathlineto{\pgfqpoint{1.067712in}{1.421039in}}%
\pgfpathlineto{\pgfqpoint{1.069030in}{1.426850in}}%
\pgfpathlineto{\pgfqpoint{1.060213in}{1.425094in}}%
\pgfpathlineto{\pgfqpoint{1.051508in}{1.423204in}}%
\pgfpathlineto{\pgfqpoint{1.042924in}{1.421181in}}%
\pgfpathlineto{\pgfqpoint{1.034468in}{1.419029in}}%
\pgfpathclose%
\pgfusepath{fill}%
\end{pgfscope}%
\begin{pgfscope}%
\pgfpathrectangle{\pgfqpoint{0.041670in}{0.041670in}}{\pgfqpoint{2.216660in}{2.216660in}}%
\pgfusepath{clip}%
\pgfsetbuttcap%
\pgfsetroundjoin%
\definecolor{currentfill}{rgb}{0.147607,0.511733,0.557049}%
\pgfsetfillcolor{currentfill}%
\pgfsetlinewidth{0.000000pt}%
\definecolor{currentstroke}{rgb}{0.000000,0.000000,0.000000}%
\pgfsetstrokecolor{currentstroke}%
\pgfsetdash{}{0pt}%
\pgfpathmoveto{\pgfqpoint{1.400294in}{1.350592in}}%
\pgfpathlineto{\pgfqpoint{1.402672in}{1.344380in}}%
\pgfpathlineto{\pgfqpoint{1.405047in}{1.338172in}}%
\pgfpathlineto{\pgfqpoint{1.407419in}{1.331972in}}%
\pgfpathlineto{\pgfqpoint{1.409789in}{1.325783in}}%
\pgfpathlineto{\pgfqpoint{1.417943in}{1.322167in}}%
\pgfpathlineto{\pgfqpoint{1.425872in}{1.318424in}}%
\pgfpathlineto{\pgfqpoint{1.433569in}{1.314559in}}%
\pgfpathlineto{\pgfqpoint{1.441027in}{1.310574in}}%
\pgfpathlineto{\pgfqpoint{1.438329in}{1.316934in}}%
\pgfpathlineto{\pgfqpoint{1.435628in}{1.323305in}}%
\pgfpathlineto{\pgfqpoint{1.432924in}{1.329683in}}%
\pgfpathlineto{\pgfqpoint{1.430217in}{1.336066in}}%
\pgfpathlineto{\pgfqpoint{1.423075in}{1.339872in}}%
\pgfpathlineto{\pgfqpoint{1.415701in}{1.343564in}}%
\pgfpathlineto{\pgfqpoint{1.408105in}{1.347138in}}%
\pgfpathlineto{\pgfqpoint{1.400294in}{1.350592in}}%
\pgfpathclose%
\pgfusepath{fill}%
\end{pgfscope}%
\begin{pgfscope}%
\pgfpathrectangle{\pgfqpoint{0.041670in}{0.041670in}}{\pgfqpoint{2.216660in}{2.216660in}}%
\pgfusepath{clip}%
\pgfsetbuttcap%
\pgfsetroundjoin%
\definecolor{currentfill}{rgb}{0.283072,0.130895,0.449241}%
\pgfsetfillcolor{currentfill}%
\pgfsetlinewidth{0.000000pt}%
\definecolor{currentstroke}{rgb}{0.000000,0.000000,0.000000}%
\pgfsetstrokecolor{currentstroke}%
\pgfsetdash{}{0pt}%
\pgfpathmoveto{\pgfqpoint{0.697704in}{0.967957in}}%
\pgfpathlineto{\pgfqpoint{0.694064in}{0.962986in}}%
\pgfpathlineto{\pgfqpoint{0.690423in}{0.958151in}}%
\pgfpathlineto{\pgfqpoint{0.686781in}{0.953455in}}%
\pgfpathlineto{\pgfqpoint{0.683137in}{0.948903in}}%
\pgfpathlineto{\pgfqpoint{0.684840in}{0.956990in}}%
\pgfpathlineto{\pgfqpoint{0.687031in}{0.965037in}}%
\pgfpathlineto{\pgfqpoint{0.689706in}{0.973035in}}%
\pgfpathlineto{\pgfqpoint{0.692860in}{0.980976in}}%
\pgfpathlineto{\pgfqpoint{0.696417in}{0.985274in}}%
\pgfpathlineto{\pgfqpoint{0.699972in}{0.989715in}}%
\pgfpathlineto{\pgfqpoint{0.703527in}{0.994295in}}%
\pgfpathlineto{\pgfqpoint{0.707080in}{0.999011in}}%
\pgfpathlineto{\pgfqpoint{0.704033in}{0.991321in}}%
\pgfpathlineto{\pgfqpoint{0.701453in}{0.983577in}}%
\pgfpathlineto{\pgfqpoint{0.699342in}{0.975786in}}%
\pgfpathlineto{\pgfqpoint{0.697704in}{0.967957in}}%
\pgfpathclose%
\pgfusepath{fill}%
\end{pgfscope}%
\begin{pgfscope}%
\pgfpathrectangle{\pgfqpoint{0.041670in}{0.041670in}}{\pgfqpoint{2.216660in}{2.216660in}}%
\pgfusepath{clip}%
\pgfsetbuttcap%
\pgfsetroundjoin%
\definecolor{currentfill}{rgb}{0.267004,0.004874,0.329415}%
\pgfsetfillcolor{currentfill}%
\pgfsetlinewidth{0.000000pt}%
\definecolor{currentstroke}{rgb}{0.000000,0.000000,0.000000}%
\pgfsetstrokecolor{currentstroke}%
\pgfsetdash{}{0pt}%
\pgfpathmoveto{\pgfqpoint{1.748487in}{0.902418in}}%
\pgfpathlineto{\pgfqpoint{1.752194in}{0.901870in}}%
\pgfpathlineto{\pgfqpoint{1.755908in}{0.901558in}}%
\pgfpathlineto{\pgfqpoint{1.759628in}{0.901487in}}%
\pgfpathlineto{\pgfqpoint{1.763354in}{0.901661in}}%
\pgfpathlineto{\pgfqpoint{1.765414in}{0.892056in}}%
\pgfpathlineto{\pgfqpoint{1.766897in}{0.882409in}}%
\pgfpathlineto{\pgfqpoint{1.767798in}{0.872730in}}%
\pgfpathlineto{\pgfqpoint{1.768115in}{0.863028in}}%
\pgfpathlineto{\pgfqpoint{1.764338in}{0.863101in}}%
\pgfpathlineto{\pgfqpoint{1.760568in}{0.863420in}}%
\pgfpathlineto{\pgfqpoint{1.756805in}{0.863981in}}%
\pgfpathlineto{\pgfqpoint{1.753048in}{0.864779in}}%
\pgfpathlineto{\pgfqpoint{1.752759in}{0.874231in}}%
\pgfpathlineto{\pgfqpoint{1.751901in}{0.883661in}}%
\pgfpathlineto{\pgfqpoint{1.750475in}{0.893059in}}%
\pgfpathlineto{\pgfqpoint{1.748487in}{0.902418in}}%
\pgfpathclose%
\pgfusepath{fill}%
\end{pgfscope}%
\begin{pgfscope}%
\pgfpathrectangle{\pgfqpoint{0.041670in}{0.041670in}}{\pgfqpoint{2.216660in}{2.216660in}}%
\pgfusepath{clip}%
\pgfsetbuttcap%
\pgfsetroundjoin%
\definecolor{currentfill}{rgb}{0.267004,0.004874,0.329415}%
\pgfsetfillcolor{currentfill}%
\pgfsetlinewidth{0.000000pt}%
\definecolor{currentstroke}{rgb}{0.000000,0.000000,0.000000}%
\pgfsetstrokecolor{currentstroke}%
\pgfsetdash{}{0pt}%
\pgfpathmoveto{\pgfqpoint{1.763354in}{0.901661in}}%
\pgfpathlineto{\pgfqpoint{1.767088in}{0.902084in}}%
\pgfpathlineto{\pgfqpoint{1.770829in}{0.902763in}}%
\pgfpathlineto{\pgfqpoint{1.774577in}{0.903702in}}%
\pgfpathlineto{\pgfqpoint{1.778333in}{0.904905in}}%
\pgfpathlineto{\pgfqpoint{1.780467in}{0.895058in}}%
\pgfpathlineto{\pgfqpoint{1.782009in}{0.885167in}}%
\pgfpathlineto{\pgfqpoint{1.782954in}{0.875242in}}%
\pgfpathlineto{\pgfqpoint{1.783300in}{0.865292in}}%
\pgfpathlineto{\pgfqpoint{1.779492in}{0.864333in}}%
\pgfpathlineto{\pgfqpoint{1.775692in}{0.863639in}}%
\pgfpathlineto{\pgfqpoint{1.771900in}{0.863205in}}%
\pgfpathlineto{\pgfqpoint{1.768115in}{0.863028in}}%
\pgfpathlineto{\pgfqpoint{1.767798in}{0.872730in}}%
\pgfpathlineto{\pgfqpoint{1.766897in}{0.882409in}}%
\pgfpathlineto{\pgfqpoint{1.765414in}{0.892056in}}%
\pgfpathlineto{\pgfqpoint{1.763354in}{0.901661in}}%
\pgfpathclose%
\pgfusepath{fill}%
\end{pgfscope}%
\begin{pgfscope}%
\pgfpathrectangle{\pgfqpoint{0.041670in}{0.041670in}}{\pgfqpoint{2.216660in}{2.216660in}}%
\pgfusepath{clip}%
\pgfsetbuttcap%
\pgfsetroundjoin%
\definecolor{currentfill}{rgb}{0.276194,0.190074,0.493001}%
\pgfsetfillcolor{currentfill}%
\pgfsetlinewidth{0.000000pt}%
\definecolor{currentstroke}{rgb}{0.000000,0.000000,0.000000}%
\pgfsetstrokecolor{currentstroke}%
\pgfsetdash{}{0pt}%
\pgfpathmoveto{\pgfqpoint{0.502035in}{0.982874in}}%
\pgfpathlineto{\pgfqpoint{0.498019in}{0.990831in}}%
\pgfpathlineto{\pgfqpoint{0.493988in}{0.999177in}}%
\pgfpathlineto{\pgfqpoint{0.489940in}{1.007918in}}%
\pgfpathlineto{\pgfqpoint{0.485876in}{1.017063in}}%
\pgfpathlineto{\pgfqpoint{0.488566in}{1.028249in}}%
\pgfpathlineto{\pgfqpoint{0.491933in}{1.039368in}}%
\pgfpathlineto{\pgfqpoint{0.495970in}{1.050410in}}%
\pgfpathlineto{\pgfqpoint{0.500670in}{1.061365in}}%
\pgfpathlineto{\pgfqpoint{0.504623in}{1.052019in}}%
\pgfpathlineto{\pgfqpoint{0.508560in}{1.043074in}}%
\pgfpathlineto{\pgfqpoint{0.512482in}{1.034523in}}%
\pgfpathlineto{\pgfqpoint{0.516388in}{1.026360in}}%
\pgfpathlineto{\pgfqpoint{0.511821in}{1.015606in}}%
\pgfpathlineto{\pgfqpoint{0.507902in}{1.004766in}}%
\pgfpathlineto{\pgfqpoint{0.504638in}{0.993852in}}%
\pgfpathlineto{\pgfqpoint{0.502035in}{0.982874in}}%
\pgfpathclose%
\pgfusepath{fill}%
\end{pgfscope}%
\begin{pgfscope}%
\pgfpathrectangle{\pgfqpoint{0.041670in}{0.041670in}}{\pgfqpoint{2.216660in}{2.216660in}}%
\pgfusepath{clip}%
\pgfsetbuttcap%
\pgfsetroundjoin%
\definecolor{currentfill}{rgb}{0.163625,0.471133,0.558148}%
\pgfsetfillcolor{currentfill}%
\pgfsetlinewidth{0.000000pt}%
\definecolor{currentstroke}{rgb}{0.000000,0.000000,0.000000}%
\pgfsetstrokecolor{currentstroke}%
\pgfsetdash{}{0pt}%
\pgfpathmoveto{\pgfqpoint{1.441027in}{1.310574in}}%
\pgfpathlineto{\pgfqpoint{1.443722in}{1.304227in}}%
\pgfpathlineto{\pgfqpoint{1.446414in}{1.297896in}}%
\pgfpathlineto{\pgfqpoint{1.449104in}{1.291584in}}%
\pgfpathlineto{\pgfqpoint{1.451790in}{1.285294in}}%
\pgfpathlineto{\pgfqpoint{1.459305in}{1.281008in}}%
\pgfpathlineto{\pgfqpoint{1.466555in}{1.276604in}}%
\pgfpathlineto{\pgfqpoint{1.473533in}{1.272087in}}%
\pgfpathlineto{\pgfqpoint{1.480232in}{1.267460in}}%
\pgfpathlineto{\pgfqpoint{1.477256in}{1.273943in}}%
\pgfpathlineto{\pgfqpoint{1.474278in}{1.280447in}}%
\pgfpathlineto{\pgfqpoint{1.471296in}{1.286971in}}%
\pgfpathlineto{\pgfqpoint{1.468312in}{1.293511in}}%
\pgfpathlineto{\pgfqpoint{1.461887in}{1.297937in}}%
\pgfpathlineto{\pgfqpoint{1.455192in}{1.302259in}}%
\pgfpathlineto{\pgfqpoint{1.448237in}{1.306473in}}%
\pgfpathlineto{\pgfqpoint{1.441027in}{1.310574in}}%
\pgfpathclose%
\pgfusepath{fill}%
\end{pgfscope}%
\begin{pgfscope}%
\pgfpathrectangle{\pgfqpoint{0.041670in}{0.041670in}}{\pgfqpoint{2.216660in}{2.216660in}}%
\pgfusepath{clip}%
\pgfsetbuttcap%
\pgfsetroundjoin%
\definecolor{currentfill}{rgb}{0.133743,0.548535,0.553541}%
\pgfsetfillcolor{currentfill}%
\pgfsetlinewidth{0.000000pt}%
\definecolor{currentstroke}{rgb}{0.000000,0.000000,0.000000}%
\pgfsetstrokecolor{currentstroke}%
\pgfsetdash{}{0pt}%
\pgfpathmoveto{\pgfqpoint{1.358964in}{1.387400in}}%
\pgfpathlineto{\pgfqpoint{1.360989in}{1.381342in}}%
\pgfpathlineto{\pgfqpoint{1.363011in}{1.375278in}}%
\pgfpathlineto{\pgfqpoint{1.365032in}{1.369211in}}%
\pgfpathlineto{\pgfqpoint{1.367049in}{1.363143in}}%
\pgfpathlineto{\pgfqpoint{1.375644in}{1.360201in}}%
\pgfpathlineto{\pgfqpoint{1.384055in}{1.357126in}}%
\pgfpathlineto{\pgfqpoint{1.392274in}{1.353923in}}%
\pgfpathlineto{\pgfqpoint{1.400294in}{1.350592in}}%
\pgfpathlineto{\pgfqpoint{1.397913in}{1.356807in}}%
\pgfpathlineto{\pgfqpoint{1.395530in}{1.363021in}}%
\pgfpathlineto{\pgfqpoint{1.393144in}{1.369231in}}%
\pgfpathlineto{\pgfqpoint{1.390755in}{1.375436in}}%
\pgfpathlineto{\pgfqpoint{1.383087in}{1.378610in}}%
\pgfpathlineto{\pgfqpoint{1.375227in}{1.381664in}}%
\pgfpathlineto{\pgfqpoint{1.367184in}{1.384595in}}%
\pgfpathlineto{\pgfqpoint{1.358964in}{1.387400in}}%
\pgfpathclose%
\pgfusepath{fill}%
\end{pgfscope}%
\begin{pgfscope}%
\pgfpathrectangle{\pgfqpoint{0.041670in}{0.041670in}}{\pgfqpoint{2.216660in}{2.216660in}}%
\pgfusepath{clip}%
\pgfsetbuttcap%
\pgfsetroundjoin%
\definecolor{currentfill}{rgb}{0.212395,0.359683,0.551710}%
\pgfsetfillcolor{currentfill}%
\pgfsetlinewidth{0.000000pt}%
\definecolor{currentstroke}{rgb}{0.000000,0.000000,0.000000}%
\pgfsetstrokecolor{currentstroke}%
\pgfsetdash{}{0pt}%
\pgfpathmoveto{\pgfqpoint{1.529777in}{1.195520in}}%
\pgfpathlineto{\pgfqpoint{1.532975in}{1.189156in}}%
\pgfpathlineto{\pgfqpoint{1.536171in}{1.182844in}}%
\pgfpathlineto{\pgfqpoint{1.539365in}{1.176587in}}%
\pgfpathlineto{\pgfqpoint{1.542557in}{1.170389in}}%
\pgfpathlineto{\pgfqpoint{1.548411in}{1.164608in}}%
\pgfpathlineto{\pgfqpoint{1.553912in}{1.158732in}}%
\pgfpathlineto{\pgfqpoint{1.559054in}{1.152766in}}%
\pgfpathlineto{\pgfqpoint{1.563830in}{1.146717in}}%
\pgfpathlineto{\pgfqpoint{1.560440in}{1.153145in}}%
\pgfpathlineto{\pgfqpoint{1.557047in}{1.159632in}}%
\pgfpathlineto{\pgfqpoint{1.553653in}{1.166173in}}%
\pgfpathlineto{\pgfqpoint{1.550256in}{1.172768in}}%
\pgfpathlineto{\pgfqpoint{1.545660in}{1.178581in}}%
\pgfpathlineto{\pgfqpoint{1.540711in}{1.184315in}}%
\pgfpathlineto{\pgfqpoint{1.535414in}{1.189963in}}%
\pgfpathlineto{\pgfqpoint{1.529777in}{1.195520in}}%
\pgfpathclose%
\pgfusepath{fill}%
\end{pgfscope}%
\begin{pgfscope}%
\pgfpathrectangle{\pgfqpoint{0.041670in}{0.041670in}}{\pgfqpoint{2.216660in}{2.216660in}}%
\pgfusepath{clip}%
\pgfsetbuttcap%
\pgfsetroundjoin%
\definecolor{currentfill}{rgb}{0.280255,0.165693,0.476498}%
\pgfsetfillcolor{currentfill}%
\pgfsetlinewidth{0.000000pt}%
\definecolor{currentstroke}{rgb}{0.000000,0.000000,0.000000}%
\pgfsetstrokecolor{currentstroke}%
\pgfsetdash{}{0pt}%
\pgfpathmoveto{\pgfqpoint{1.635633in}{1.025712in}}%
\pgfpathlineto{\pgfqpoint{1.639158in}{1.020548in}}%
\pgfpathlineto{\pgfqpoint{1.642683in}{1.015505in}}%
\pgfpathlineto{\pgfqpoint{1.646208in}{1.010586in}}%
\pgfpathlineto{\pgfqpoint{1.649734in}{1.005795in}}%
\pgfpathlineto{\pgfqpoint{1.653191in}{0.998159in}}%
\pgfpathlineto{\pgfqpoint{1.656186in}{0.990463in}}%
\pgfpathlineto{\pgfqpoint{1.658715in}{0.982713in}}%
\pgfpathlineto{\pgfqpoint{1.660774in}{0.974918in}}%
\pgfpathlineto{\pgfqpoint{1.657149in}{0.979963in}}%
\pgfpathlineto{\pgfqpoint{1.653526in}{0.985135in}}%
\pgfpathlineto{\pgfqpoint{1.649903in}{0.990432in}}%
\pgfpathlineto{\pgfqpoint{1.646281in}{0.995850in}}%
\pgfpathlineto{\pgfqpoint{1.644299in}{1.003388in}}%
\pgfpathlineto{\pgfqpoint{1.641862in}{1.010882in}}%
\pgfpathlineto{\pgfqpoint{1.638972in}{1.018326in}}%
\pgfpathlineto{\pgfqpoint{1.635633in}{1.025712in}}%
\pgfpathclose%
\pgfusepath{fill}%
\end{pgfscope}%
\begin{pgfscope}%
\pgfpathrectangle{\pgfqpoint{0.041670in}{0.041670in}}{\pgfqpoint{2.216660in}{2.216660in}}%
\pgfusepath{clip}%
\pgfsetbuttcap%
\pgfsetroundjoin%
\definecolor{currentfill}{rgb}{0.172719,0.448791,0.557885}%
\pgfsetfillcolor{currentfill}%
\pgfsetlinewidth{0.000000pt}%
\definecolor{currentstroke}{rgb}{0.000000,0.000000,0.000000}%
\pgfsetstrokecolor{currentstroke}%
\pgfsetdash{}{0pt}%
\pgfpathmoveto{\pgfqpoint{1.873139in}{1.261801in}}%
\pgfpathlineto{\pgfqpoint{1.877107in}{1.277191in}}%
\pgfpathlineto{\pgfqpoint{1.881096in}{1.293082in}}%
\pgfpathlineto{\pgfqpoint{1.885107in}{1.309481in}}%
\pgfpathlineto{\pgfqpoint{1.893727in}{1.298480in}}%
\pgfpathlineto{\pgfqpoint{1.901673in}{1.287334in}}%
\pgfpathlineto{\pgfqpoint{1.908936in}{1.276050in}}%
\pgfpathlineto{\pgfqpoint{1.915504in}{1.264640in}}%
\pgfpathlineto{\pgfqpoint{1.911293in}{1.248390in}}%
\pgfpathlineto{\pgfqpoint{1.907104in}{1.232652in}}%
\pgfpathlineto{\pgfqpoint{1.902939in}{1.217417in}}%
\pgfpathlineto{\pgfqpoint{1.896503in}{1.228710in}}%
\pgfpathlineto{\pgfqpoint{1.889384in}{1.239878in}}%
\pgfpathlineto{\pgfqpoint{1.881593in}{1.250912in}}%
\pgfpathlineto{\pgfqpoint{1.873139in}{1.261801in}}%
\pgfpathclose%
\pgfusepath{fill}%
\end{pgfscope}%
\begin{pgfscope}%
\pgfpathrectangle{\pgfqpoint{0.041670in}{0.041670in}}{\pgfqpoint{2.216660in}{2.216660in}}%
\pgfusepath{clip}%
\pgfsetbuttcap%
\pgfsetroundjoin%
\definecolor{currentfill}{rgb}{0.268510,0.009605,0.335427}%
\pgfsetfillcolor{currentfill}%
\pgfsetlinewidth{0.000000pt}%
\definecolor{currentstroke}{rgb}{0.000000,0.000000,0.000000}%
\pgfsetstrokecolor{currentstroke}%
\pgfsetdash{}{0pt}%
\pgfpathmoveto{\pgfqpoint{1.733715in}{0.906877in}}%
\pgfpathlineto{\pgfqpoint{1.737400in}{0.905431in}}%
\pgfpathlineto{\pgfqpoint{1.741090in}{0.904203in}}%
\pgfpathlineto{\pgfqpoint{1.744786in}{0.903197in}}%
\pgfpathlineto{\pgfqpoint{1.748487in}{0.902418in}}%
\pgfpathlineto{\pgfqpoint{1.750475in}{0.893059in}}%
\pgfpathlineto{\pgfqpoint{1.751901in}{0.883661in}}%
\pgfpathlineto{\pgfqpoint{1.752759in}{0.874231in}}%
\pgfpathlineto{\pgfqpoint{1.753048in}{0.864779in}}%
\pgfpathlineto{\pgfqpoint{1.749298in}{0.865808in}}%
\pgfpathlineto{\pgfqpoint{1.745554in}{0.867065in}}%
\pgfpathlineto{\pgfqpoint{1.741815in}{0.868545in}}%
\pgfpathlineto{\pgfqpoint{1.738082in}{0.870243in}}%
\pgfpathlineto{\pgfqpoint{1.737820in}{0.879442in}}%
\pgfpathlineto{\pgfqpoint{1.737003in}{0.888619in}}%
\pgfpathlineto{\pgfqpoint{1.735633in}{0.897768in}}%
\pgfpathlineto{\pgfqpoint{1.733715in}{0.906877in}}%
\pgfpathclose%
\pgfusepath{fill}%
\end{pgfscope}%
\begin{pgfscope}%
\pgfpathrectangle{\pgfqpoint{0.041670in}{0.041670in}}{\pgfqpoint{2.216660in}{2.216660in}}%
\pgfusepath{clip}%
\pgfsetbuttcap%
\pgfsetroundjoin%
\definecolor{currentfill}{rgb}{0.268510,0.009605,0.335427}%
\pgfsetfillcolor{currentfill}%
\pgfsetlinewidth{0.000000pt}%
\definecolor{currentstroke}{rgb}{0.000000,0.000000,0.000000}%
\pgfsetstrokecolor{currentstroke}%
\pgfsetdash{}{0pt}%
\pgfpathmoveto{\pgfqpoint{1.778333in}{0.904905in}}%
\pgfpathlineto{\pgfqpoint{1.782098in}{0.906378in}}%
\pgfpathlineto{\pgfqpoint{1.785871in}{0.908125in}}%
\pgfpathlineto{\pgfqpoint{1.789652in}{0.910152in}}%
\pgfpathlineto{\pgfqpoint{1.793443in}{0.912464in}}%
\pgfpathlineto{\pgfqpoint{1.795652in}{0.902379in}}%
\pgfpathlineto{\pgfqpoint{1.797254in}{0.892248in}}%
\pgfpathlineto{\pgfqpoint{1.798244in}{0.882081in}}%
\pgfpathlineto{\pgfqpoint{1.798621in}{0.871889in}}%
\pgfpathlineto{\pgfqpoint{1.794777in}{0.869816in}}%
\pgfpathlineto{\pgfqpoint{1.790942in}{0.868029in}}%
\pgfpathlineto{\pgfqpoint{1.787116in}{0.866523in}}%
\pgfpathlineto{\pgfqpoint{1.783300in}{0.865292in}}%
\pgfpathlineto{\pgfqpoint{1.782954in}{0.875242in}}%
\pgfpathlineto{\pgfqpoint{1.782009in}{0.885167in}}%
\pgfpathlineto{\pgfqpoint{1.780467in}{0.895058in}}%
\pgfpathlineto{\pgfqpoint{1.778333in}{0.904905in}}%
\pgfpathclose%
\pgfusepath{fill}%
\end{pgfscope}%
\begin{pgfscope}%
\pgfpathrectangle{\pgfqpoint{0.041670in}{0.041670in}}{\pgfqpoint{2.216660in}{2.216660in}}%
\pgfusepath{clip}%
\pgfsetbuttcap%
\pgfsetroundjoin%
\definecolor{currentfill}{rgb}{0.120081,0.622161,0.534946}%
\pgfsetfillcolor{currentfill}%
\pgfsetlinewidth{0.000000pt}%
\definecolor{currentstroke}{rgb}{0.000000,0.000000,0.000000}%
\pgfsetstrokecolor{currentstroke}%
\pgfsetdash{}{0pt}%
\pgfpathmoveto{\pgfqpoint{1.144368in}{1.458485in}}%
\pgfpathlineto{\pgfqpoint{1.143923in}{1.452881in}}%
\pgfpathlineto{\pgfqpoint{1.143477in}{1.447250in}}%
\pgfpathlineto{\pgfqpoint{1.143033in}{1.441595in}}%
\pgfpathlineto{\pgfqpoint{1.142589in}{1.435919in}}%
\pgfpathlineto{\pgfqpoint{1.152032in}{1.436415in}}%
\pgfpathlineto{\pgfqpoint{1.161501in}{1.436768in}}%
\pgfpathlineto{\pgfqpoint{1.170988in}{1.436976in}}%
\pgfpathlineto{\pgfqpoint{1.180482in}{1.437040in}}%
\pgfpathlineto{\pgfqpoint{1.180476in}{1.442703in}}%
\pgfpathlineto{\pgfqpoint{1.180470in}{1.448343in}}%
\pgfpathlineto{\pgfqpoint{1.180464in}{1.453959in}}%
\pgfpathlineto{\pgfqpoint{1.180457in}{1.459549in}}%
\pgfpathlineto{\pgfqpoint{1.171415in}{1.459488in}}%
\pgfpathlineto{\pgfqpoint{1.162380in}{1.459290in}}%
\pgfpathlineto{\pgfqpoint{1.153362in}{1.458956in}}%
\pgfpathlineto{\pgfqpoint{1.144368in}{1.458485in}}%
\pgfpathclose%
\pgfusepath{fill}%
\end{pgfscope}%
\begin{pgfscope}%
\pgfpathrectangle{\pgfqpoint{0.041670in}{0.041670in}}{\pgfqpoint{2.216660in}{2.216660in}}%
\pgfusepath{clip}%
\pgfsetbuttcap%
\pgfsetroundjoin%
\definecolor{currentfill}{rgb}{0.120081,0.622161,0.534946}%
\pgfsetfillcolor{currentfill}%
\pgfsetlinewidth{0.000000pt}%
\definecolor{currentstroke}{rgb}{0.000000,0.000000,0.000000}%
\pgfsetstrokecolor{currentstroke}%
\pgfsetdash{}{0pt}%
\pgfpathmoveto{\pgfqpoint{1.180457in}{1.459549in}}%
\pgfpathlineto{\pgfqpoint{1.180464in}{1.453959in}}%
\pgfpathlineto{\pgfqpoint{1.180470in}{1.448343in}}%
\pgfpathlineto{\pgfqpoint{1.180476in}{1.442703in}}%
\pgfpathlineto{\pgfqpoint{1.180482in}{1.437040in}}%
\pgfpathlineto{\pgfqpoint{1.189977in}{1.436960in}}%
\pgfpathlineto{\pgfqpoint{1.199462in}{1.436736in}}%
\pgfpathlineto{\pgfqpoint{1.208928in}{1.436367in}}%
\pgfpathlineto{\pgfqpoint{1.218368in}{1.435855in}}%
\pgfpathlineto{\pgfqpoint{1.217912in}{1.441532in}}%
\pgfpathlineto{\pgfqpoint{1.217455in}{1.447188in}}%
\pgfpathlineto{\pgfqpoint{1.216997in}{1.452819in}}%
\pgfpathlineto{\pgfqpoint{1.216539in}{1.458424in}}%
\pgfpathlineto{\pgfqpoint{1.207549in}{1.458910in}}%
\pgfpathlineto{\pgfqpoint{1.198533in}{1.459260in}}%
\pgfpathlineto{\pgfqpoint{1.189499in}{1.459473in}}%
\pgfpathlineto{\pgfqpoint{1.180457in}{1.459549in}}%
\pgfpathclose%
\pgfusepath{fill}%
\end{pgfscope}%
\begin{pgfscope}%
\pgfpathrectangle{\pgfqpoint{0.041670in}{0.041670in}}{\pgfqpoint{2.216660in}{2.216660in}}%
\pgfusepath{clip}%
\pgfsetbuttcap%
\pgfsetroundjoin%
\definecolor{currentfill}{rgb}{0.260571,0.246922,0.522828}%
\pgfsetfillcolor{currentfill}%
\pgfsetlinewidth{0.000000pt}%
\definecolor{currentstroke}{rgb}{0.000000,0.000000,0.000000}%
\pgfsetstrokecolor{currentstroke}%
\pgfsetdash{}{0pt}%
\pgfpathmoveto{\pgfqpoint{1.854511in}{1.071021in}}%
\pgfpathlineto{\pgfqpoint{1.858448in}{1.080816in}}%
\pgfpathlineto{\pgfqpoint{1.862400in}{1.091025in}}%
\pgfpathlineto{\pgfqpoint{1.866370in}{1.101654in}}%
\pgfpathlineto{\pgfqpoint{1.870357in}{1.112710in}}%
\pgfpathlineto{\pgfqpoint{1.875789in}{1.101652in}}%
\pgfpathlineto{\pgfqpoint{1.880551in}{1.090495in}}%
\pgfpathlineto{\pgfqpoint{1.884633in}{1.079249in}}%
\pgfpathlineto{\pgfqpoint{1.888029in}{1.067925in}}%
\pgfpathlineto{\pgfqpoint{1.883914in}{1.057059in}}%
\pgfpathlineto{\pgfqpoint{1.879817in}{1.046622in}}%
\pgfpathlineto{\pgfqpoint{1.875738in}{1.036607in}}%
\pgfpathlineto{\pgfqpoint{1.871676in}{1.027009in}}%
\pgfpathlineto{\pgfqpoint{1.868384in}{1.038136in}}%
\pgfpathlineto{\pgfqpoint{1.864421in}{1.049187in}}%
\pgfpathlineto{\pgfqpoint{1.859794in}{1.060152in}}%
\pgfpathlineto{\pgfqpoint{1.854511in}{1.071021in}}%
\pgfpathclose%
\pgfusepath{fill}%
\end{pgfscope}%
\begin{pgfscope}%
\pgfpathrectangle{\pgfqpoint{0.041670in}{0.041670in}}{\pgfqpoint{2.216660in}{2.216660in}}%
\pgfusepath{clip}%
\pgfsetbuttcap%
\pgfsetroundjoin%
\definecolor{currentfill}{rgb}{0.248629,0.278775,0.534556}%
\pgfsetfillcolor{currentfill}%
\pgfsetlinewidth{0.000000pt}%
\definecolor{currentstroke}{rgb}{0.000000,0.000000,0.000000}%
\pgfsetstrokecolor{currentstroke}%
\pgfsetdash{}{0pt}%
\pgfpathmoveto{\pgfqpoint{1.577375in}{1.121655in}}%
\pgfpathlineto{\pgfqpoint{1.580757in}{1.115569in}}%
\pgfpathlineto{\pgfqpoint{1.584138in}{1.109560in}}%
\pgfpathlineto{\pgfqpoint{1.587518in}{1.103634in}}%
\pgfpathlineto{\pgfqpoint{1.590896in}{1.097792in}}%
\pgfpathlineto{\pgfqpoint{1.595638in}{1.091186in}}%
\pgfpathlineto{\pgfqpoint{1.599978in}{1.084501in}}%
\pgfpathlineto{\pgfqpoint{1.603911in}{1.077744in}}%
\pgfpathlineto{\pgfqpoint{1.607432in}{1.070920in}}%
\pgfpathlineto{\pgfqpoint{1.603904in}{1.077005in}}%
\pgfpathlineto{\pgfqpoint{1.600376in}{1.083176in}}%
\pgfpathlineto{\pgfqpoint{1.596846in}{1.089428in}}%
\pgfpathlineto{\pgfqpoint{1.593315in}{1.095759in}}%
\pgfpathlineto{\pgfqpoint{1.589923in}{1.102334in}}%
\pgfpathlineto{\pgfqpoint{1.586133in}{1.108846in}}%
\pgfpathlineto{\pgfqpoint{1.581948in}{1.115288in}}%
\pgfpathlineto{\pgfqpoint{1.577375in}{1.121655in}}%
\pgfpathclose%
\pgfusepath{fill}%
\end{pgfscope}%
\begin{pgfscope}%
\pgfpathrectangle{\pgfqpoint{0.041670in}{0.041670in}}{\pgfqpoint{2.216660in}{2.216660in}}%
\pgfusepath{clip}%
\pgfsetbuttcap%
\pgfsetroundjoin%
\definecolor{currentfill}{rgb}{0.147607,0.511733,0.557049}%
\pgfsetfillcolor{currentfill}%
\pgfsetlinewidth{0.000000pt}%
\definecolor{currentstroke}{rgb}{0.000000,0.000000,0.000000}%
\pgfsetstrokecolor{currentstroke}%
\pgfsetdash{}{0pt}%
\pgfpathmoveto{\pgfqpoint{0.923542in}{1.332590in}}%
\pgfpathlineto{\pgfqpoint{0.920768in}{1.326166in}}%
\pgfpathlineto{\pgfqpoint{0.917996in}{1.319747in}}%
\pgfpathlineto{\pgfqpoint{0.915227in}{1.313335in}}%
\pgfpathlineto{\pgfqpoint{0.912462in}{1.306934in}}%
\pgfpathlineto{\pgfqpoint{0.919700in}{1.311022in}}%
\pgfpathlineto{\pgfqpoint{0.927184in}{1.314994in}}%
\pgfpathlineto{\pgfqpoint{0.934908in}{1.318846in}}%
\pgfpathlineto{\pgfqpoint{0.942862in}{1.322575in}}%
\pgfpathlineto{\pgfqpoint{0.945308in}{1.328800in}}%
\pgfpathlineto{\pgfqpoint{0.947756in}{1.335036in}}%
\pgfpathlineto{\pgfqpoint{0.950208in}{1.341280in}}%
\pgfpathlineto{\pgfqpoint{0.952662in}{1.347528in}}%
\pgfpathlineto{\pgfqpoint{0.945041in}{1.343967in}}%
\pgfpathlineto{\pgfqpoint{0.937643in}{1.340288in}}%
\pgfpathlineto{\pgfqpoint{0.930474in}{1.336495in}}%
\pgfpathlineto{\pgfqpoint{0.923542in}{1.332590in}}%
\pgfpathclose%
\pgfusepath{fill}%
\end{pgfscope}%
\begin{pgfscope}%
\pgfpathrectangle{\pgfqpoint{0.041670in}{0.041670in}}{\pgfqpoint{2.216660in}{2.216660in}}%
\pgfusepath{clip}%
\pgfsetbuttcap%
\pgfsetroundjoin%
\definecolor{currentfill}{rgb}{0.271305,0.019942,0.347269}%
\pgfsetfillcolor{currentfill}%
\pgfsetlinewidth{0.000000pt}%
\definecolor{currentstroke}{rgb}{0.000000,0.000000,0.000000}%
\pgfsetstrokecolor{currentstroke}%
\pgfsetdash{}{0pt}%
\pgfpathmoveto{\pgfqpoint{1.719023in}{0.914748in}}%
\pgfpathlineto{\pgfqpoint{1.722690in}{0.912476in}}%
\pgfpathlineto{\pgfqpoint{1.726360in}{0.910404in}}%
\pgfpathlineto{\pgfqpoint{1.730035in}{0.908536in}}%
\pgfpathlineto{\pgfqpoint{1.733715in}{0.906877in}}%
\pgfpathlineto{\pgfqpoint{1.735633in}{0.897768in}}%
\pgfpathlineto{\pgfqpoint{1.737003in}{0.888619in}}%
\pgfpathlineto{\pgfqpoint{1.737820in}{0.879442in}}%
\pgfpathlineto{\pgfqpoint{1.738082in}{0.870243in}}%
\pgfpathlineto{\pgfqpoint{1.734354in}{0.872155in}}%
\pgfpathlineto{\pgfqpoint{1.730632in}{0.874277in}}%
\pgfpathlineto{\pgfqpoint{1.726914in}{0.876603in}}%
\pgfpathlineto{\pgfqpoint{1.723201in}{0.879130in}}%
\pgfpathlineto{\pgfqpoint{1.722964in}{0.888073in}}%
\pgfpathlineto{\pgfqpoint{1.722187in}{0.896996in}}%
\pgfpathlineto{\pgfqpoint{1.720872in}{0.905890in}}%
\pgfpathlineto{\pgfqpoint{1.719023in}{0.914748in}}%
\pgfpathclose%
\pgfusepath{fill}%
\end{pgfscope}%
\begin{pgfscope}%
\pgfpathrectangle{\pgfqpoint{0.041670in}{0.041670in}}{\pgfqpoint{2.216660in}{2.216660in}}%
\pgfusepath{clip}%
\pgfsetbuttcap%
\pgfsetroundjoin%
\definecolor{currentfill}{rgb}{0.133743,0.548535,0.553541}%
\pgfsetfillcolor{currentfill}%
\pgfsetlinewidth{0.000000pt}%
\definecolor{currentstroke}{rgb}{0.000000,0.000000,0.000000}%
\pgfsetstrokecolor{currentstroke}%
\pgfsetdash{}{0pt}%
\pgfpathmoveto{\pgfqpoint{0.962506in}{1.372515in}}%
\pgfpathlineto{\pgfqpoint{0.960041in}{1.366275in}}%
\pgfpathlineto{\pgfqpoint{0.957578in}{1.360028in}}%
\pgfpathlineto{\pgfqpoint{0.955119in}{1.353779in}}%
\pgfpathlineto{\pgfqpoint{0.952662in}{1.347528in}}%
\pgfpathlineto{\pgfqpoint{0.960497in}{1.350968in}}%
\pgfpathlineto{\pgfqpoint{0.968539in}{1.354285in}}%
\pgfpathlineto{\pgfqpoint{0.976780in}{1.357475in}}%
\pgfpathlineto{\pgfqpoint{0.985212in}{1.360534in}}%
\pgfpathlineto{\pgfqpoint{0.987313in}{1.366633in}}%
\pgfpathlineto{\pgfqpoint{0.989416in}{1.372730in}}%
\pgfpathlineto{\pgfqpoint{0.991522in}{1.378825in}}%
\pgfpathlineto{\pgfqpoint{0.993631in}{1.384913in}}%
\pgfpathlineto{\pgfqpoint{0.985567in}{1.381996in}}%
\pgfpathlineto{\pgfqpoint{0.977687in}{1.378956in}}%
\pgfpathlineto{\pgfqpoint{0.969997in}{1.375794in}}%
\pgfpathlineto{\pgfqpoint{0.962506in}{1.372515in}}%
\pgfpathclose%
\pgfusepath{fill}%
\end{pgfscope}%
\begin{pgfscope}%
\pgfpathrectangle{\pgfqpoint{0.041670in}{0.041670in}}{\pgfqpoint{2.216660in}{2.216660in}}%
\pgfusepath{clip}%
\pgfsetbuttcap%
\pgfsetroundjoin%
\definecolor{currentfill}{rgb}{0.120081,0.622161,0.534946}%
\pgfsetfillcolor{currentfill}%
\pgfsetlinewidth{0.000000pt}%
\definecolor{currentstroke}{rgb}{0.000000,0.000000,0.000000}%
\pgfsetstrokecolor{currentstroke}%
\pgfsetdash{}{0pt}%
\pgfpathmoveto{\pgfqpoint{1.108812in}{1.455246in}}%
\pgfpathlineto{\pgfqpoint{1.107920in}{1.449598in}}%
\pgfpathlineto{\pgfqpoint{1.107029in}{1.443924in}}%
\pgfpathlineto{\pgfqpoint{1.106140in}{1.438226in}}%
\pgfpathlineto{\pgfqpoint{1.105252in}{1.432506in}}%
\pgfpathlineto{\pgfqpoint{1.114503in}{1.433572in}}%
\pgfpathlineto{\pgfqpoint{1.123816in}{1.434497in}}%
\pgfpathlineto{\pgfqpoint{1.133180in}{1.435279in}}%
\pgfpathlineto{\pgfqpoint{1.142589in}{1.435919in}}%
\pgfpathlineto{\pgfqpoint{1.143033in}{1.441595in}}%
\pgfpathlineto{\pgfqpoint{1.143477in}{1.447250in}}%
\pgfpathlineto{\pgfqpoint{1.143923in}{1.452881in}}%
\pgfpathlineto{\pgfqpoint{1.144368in}{1.458485in}}%
\pgfpathlineto{\pgfqpoint{1.135408in}{1.457878in}}%
\pgfpathlineto{\pgfqpoint{1.126490in}{1.457135in}}%
\pgfpathlineto{\pgfqpoint{1.117622in}{1.456257in}}%
\pgfpathlineto{\pgfqpoint{1.108812in}{1.455246in}}%
\pgfpathclose%
\pgfusepath{fill}%
\end{pgfscope}%
\begin{pgfscope}%
\pgfpathrectangle{\pgfqpoint{0.041670in}{0.041670in}}{\pgfqpoint{2.216660in}{2.216660in}}%
\pgfusepath{clip}%
\pgfsetbuttcap%
\pgfsetroundjoin%
\definecolor{currentfill}{rgb}{0.120081,0.622161,0.534946}%
\pgfsetfillcolor{currentfill}%
\pgfsetlinewidth{0.000000pt}%
\definecolor{currentstroke}{rgb}{0.000000,0.000000,0.000000}%
\pgfsetstrokecolor{currentstroke}%
\pgfsetdash{}{0pt}%
\pgfpathmoveto{\pgfqpoint{1.216539in}{1.458424in}}%
\pgfpathlineto{\pgfqpoint{1.216997in}{1.452819in}}%
\pgfpathlineto{\pgfqpoint{1.217455in}{1.447188in}}%
\pgfpathlineto{\pgfqpoint{1.217912in}{1.441532in}}%
\pgfpathlineto{\pgfqpoint{1.218368in}{1.435855in}}%
\pgfpathlineto{\pgfqpoint{1.227772in}{1.435200in}}%
\pgfpathlineto{\pgfqpoint{1.237132in}{1.434401in}}%
\pgfpathlineto{\pgfqpoint{1.246438in}{1.433461in}}%
\pgfpathlineto{\pgfqpoint{1.255682in}{1.432379in}}%
\pgfpathlineto{\pgfqpoint{1.254781in}{1.438101in}}%
\pgfpathlineto{\pgfqpoint{1.253880in}{1.443801in}}%
\pgfpathlineto{\pgfqpoint{1.252977in}{1.449476in}}%
\pgfpathlineto{\pgfqpoint{1.252073in}{1.455125in}}%
\pgfpathlineto{\pgfqpoint{1.243270in}{1.456152in}}%
\pgfpathlineto{\pgfqpoint{1.234408in}{1.457044in}}%
\pgfpathlineto{\pgfqpoint{1.225495in}{1.457802in}}%
\pgfpathlineto{\pgfqpoint{1.216539in}{1.458424in}}%
\pgfpathclose%
\pgfusepath{fill}%
\end{pgfscope}%
\begin{pgfscope}%
\pgfpathrectangle{\pgfqpoint{0.041670in}{0.041670in}}{\pgfqpoint{2.216660in}{2.216660in}}%
\pgfusepath{clip}%
\pgfsetbuttcap%
\pgfsetroundjoin%
\definecolor{currentfill}{rgb}{0.272594,0.025563,0.353093}%
\pgfsetfillcolor{currentfill}%
\pgfsetlinewidth{0.000000pt}%
\definecolor{currentstroke}{rgb}{0.000000,0.000000,0.000000}%
\pgfsetstrokecolor{currentstroke}%
\pgfsetdash{}{0pt}%
\pgfpathmoveto{\pgfqpoint{1.793443in}{0.912464in}}%
\pgfpathlineto{\pgfqpoint{1.797243in}{0.915067in}}%
\pgfpathlineto{\pgfqpoint{1.801052in}{0.917964in}}%
\pgfpathlineto{\pgfqpoint{1.804872in}{0.921162in}}%
\pgfpathlineto{\pgfqpoint{1.808702in}{0.924666in}}%
\pgfpathlineto{\pgfqpoint{1.810988in}{0.914347in}}%
\pgfpathlineto{\pgfqpoint{1.812652in}{0.903981in}}%
\pgfpathlineto{\pgfqpoint{1.813690in}{0.893577in}}%
\pgfpathlineto{\pgfqpoint{1.814098in}{0.883146in}}%
\pgfpathlineto{\pgfqpoint{1.810213in}{0.879876in}}%
\pgfpathlineto{\pgfqpoint{1.806338in}{0.876914in}}%
\pgfpathlineto{\pgfqpoint{1.802474in}{0.874253in}}%
\pgfpathlineto{\pgfqpoint{1.798621in}{0.871889in}}%
\pgfpathlineto{\pgfqpoint{1.798244in}{0.882081in}}%
\pgfpathlineto{\pgfqpoint{1.797254in}{0.892248in}}%
\pgfpathlineto{\pgfqpoint{1.795652in}{0.902379in}}%
\pgfpathlineto{\pgfqpoint{1.793443in}{0.912464in}}%
\pgfpathclose%
\pgfusepath{fill}%
\end{pgfscope}%
\begin{pgfscope}%
\pgfpathrectangle{\pgfqpoint{0.041670in}{0.041670in}}{\pgfqpoint{2.216660in}{2.216660in}}%
\pgfusepath{clip}%
\pgfsetbuttcap%
\pgfsetroundjoin%
\definecolor{currentfill}{rgb}{0.163625,0.471133,0.558148}%
\pgfsetfillcolor{currentfill}%
\pgfsetlinewidth{0.000000pt}%
\definecolor{currentstroke}{rgb}{0.000000,0.000000,0.000000}%
\pgfsetstrokecolor{currentstroke}%
\pgfsetdash{}{0pt}%
\pgfpathmoveto{\pgfqpoint{0.886117in}{1.289491in}}%
\pgfpathlineto{\pgfqpoint{0.883074in}{1.282906in}}%
\pgfpathlineto{\pgfqpoint{0.880034in}{1.276337in}}%
\pgfpathlineto{\pgfqpoint{0.876997in}{1.269787in}}%
\pgfpathlineto{\pgfqpoint{0.873963in}{1.263258in}}%
\pgfpathlineto{\pgfqpoint{0.880408in}{1.267979in}}%
\pgfpathlineto{\pgfqpoint{0.887138in}{1.272594in}}%
\pgfpathlineto{\pgfqpoint{0.894147in}{1.277099in}}%
\pgfpathlineto{\pgfqpoint{0.901427in}{1.281490in}}%
\pgfpathlineto{\pgfqpoint{0.904181in}{1.287821in}}%
\pgfpathlineto{\pgfqpoint{0.906939in}{1.294174in}}%
\pgfpathlineto{\pgfqpoint{0.909699in}{1.300546in}}%
\pgfpathlineto{\pgfqpoint{0.912462in}{1.306934in}}%
\pgfpathlineto{\pgfqpoint{0.905477in}{1.302733in}}%
\pgfpathlineto{\pgfqpoint{0.898754in}{1.298423in}}%
\pgfpathlineto{\pgfqpoint{0.892298in}{1.294008in}}%
\pgfpathlineto{\pgfqpoint{0.886117in}{1.289491in}}%
\pgfpathclose%
\pgfusepath{fill}%
\end{pgfscope}%
\begin{pgfscope}%
\pgfpathrectangle{\pgfqpoint{0.041670in}{0.041670in}}{\pgfqpoint{2.216660in}{2.216660in}}%
\pgfusepath{clip}%
\pgfsetbuttcap%
\pgfsetroundjoin%
\definecolor{currentfill}{rgb}{0.179019,0.433756,0.557430}%
\pgfsetfillcolor{currentfill}%
\pgfsetlinewidth{0.000000pt}%
\definecolor{currentstroke}{rgb}{0.000000,0.000000,0.000000}%
\pgfsetstrokecolor{currentstroke}%
\pgfsetdash{}{0pt}%
\pgfpathmoveto{\pgfqpoint{1.480232in}{1.267460in}}%
\pgfpathlineto{\pgfqpoint{1.483205in}{1.261002in}}%
\pgfpathlineto{\pgfqpoint{1.486175in}{1.254571in}}%
\pgfpathlineto{\pgfqpoint{1.489143in}{1.248172in}}%
\pgfpathlineto{\pgfqpoint{1.492107in}{1.241806in}}%
\pgfpathlineto{\pgfqpoint{1.498783in}{1.236868in}}%
\pgfpathlineto{\pgfqpoint{1.505155in}{1.231823in}}%
\pgfpathlineto{\pgfqpoint{1.511216in}{1.226678in}}%
\pgfpathlineto{\pgfqpoint{1.516960in}{1.221437in}}%
\pgfpathlineto{\pgfqpoint{1.513750in}{1.228015in}}%
\pgfpathlineto{\pgfqpoint{1.510537in}{1.234627in}}%
\pgfpathlineto{\pgfqpoint{1.507321in}{1.241270in}}%
\pgfpathlineto{\pgfqpoint{1.504103in}{1.247941in}}%
\pgfpathlineto{\pgfqpoint{1.498587in}{1.252963in}}%
\pgfpathlineto{\pgfqpoint{1.492766in}{1.257894in}}%
\pgfpathlineto{\pgfqpoint{1.486645in}{1.262727in}}%
\pgfpathlineto{\pgfqpoint{1.480232in}{1.267460in}}%
\pgfpathclose%
\pgfusepath{fill}%
\end{pgfscope}%
\begin{pgfscope}%
\pgfpathrectangle{\pgfqpoint{0.041670in}{0.041670in}}{\pgfqpoint{2.216660in}{2.216660in}}%
\pgfusepath{clip}%
\pgfsetbuttcap%
\pgfsetroundjoin%
\definecolor{currentfill}{rgb}{0.122606,0.585371,0.546557}%
\pgfsetfillcolor{currentfill}%
\pgfsetlinewidth{0.000000pt}%
\definecolor{currentstroke}{rgb}{0.000000,0.000000,0.000000}%
\pgfsetstrokecolor{currentstroke}%
\pgfsetdash{}{0pt}%
\pgfpathmoveto{\pgfqpoint{1.317932in}{1.420949in}}%
\pgfpathlineto{\pgfqpoint{1.319573in}{1.415062in}}%
\pgfpathlineto{\pgfqpoint{1.321211in}{1.409159in}}%
\pgfpathlineto{\pgfqpoint{1.322848in}{1.403242in}}%
\pgfpathlineto{\pgfqpoint{1.324483in}{1.397314in}}%
\pgfpathlineto{\pgfqpoint{1.333327in}{1.395036in}}%
\pgfpathlineto{\pgfqpoint{1.342027in}{1.392623in}}%
\pgfpathlineto{\pgfqpoint{1.350576in}{1.390077in}}%
\pgfpathlineto{\pgfqpoint{1.358964in}{1.387400in}}%
\pgfpathlineto{\pgfqpoint{1.356936in}{1.393449in}}%
\pgfpathlineto{\pgfqpoint{1.354906in}{1.399487in}}%
\pgfpathlineto{\pgfqpoint{1.352873in}{1.405511in}}%
\pgfpathlineto{\pgfqpoint{1.350838in}{1.411518in}}%
\pgfpathlineto{\pgfqpoint{1.342834in}{1.414064in}}%
\pgfpathlineto{\pgfqpoint{1.334676in}{1.416486in}}%
\pgfpathlineto{\pgfqpoint{1.326373in}{1.418782in}}%
\pgfpathlineto{\pgfqpoint{1.317932in}{1.420949in}}%
\pgfpathclose%
\pgfusepath{fill}%
\end{pgfscope}%
\begin{pgfscope}%
\pgfpathrectangle{\pgfqpoint{0.041670in}{0.041670in}}{\pgfqpoint{2.216660in}{2.216660in}}%
\pgfusepath{clip}%
\pgfsetbuttcap%
\pgfsetroundjoin%
\definecolor{currentfill}{rgb}{0.212395,0.359683,0.551710}%
\pgfsetfillcolor{currentfill}%
\pgfsetlinewidth{0.000000pt}%
\definecolor{currentstroke}{rgb}{0.000000,0.000000,0.000000}%
\pgfsetstrokecolor{currentstroke}%
\pgfsetdash{}{0pt}%
\pgfpathmoveto{\pgfqpoint{0.805869in}{1.167537in}}%
\pgfpathlineto{\pgfqpoint{0.802435in}{1.160890in}}%
\pgfpathlineto{\pgfqpoint{0.799003in}{1.154295in}}%
\pgfpathlineto{\pgfqpoint{0.795573in}{1.147755in}}%
\pgfpathlineto{\pgfqpoint{0.792145in}{1.141274in}}%
\pgfpathlineto{\pgfqpoint{0.796592in}{1.147393in}}%
\pgfpathlineto{\pgfqpoint{0.801409in}{1.153433in}}%
\pgfpathlineto{\pgfqpoint{0.806591in}{1.159389in}}%
\pgfpathlineto{\pgfqpoint{0.812132in}{1.165255in}}%
\pgfpathlineto{\pgfqpoint{0.815372in}{1.171503in}}%
\pgfpathlineto{\pgfqpoint{0.818614in}{1.177809in}}%
\pgfpathlineto{\pgfqpoint{0.821859in}{1.184171in}}%
\pgfpathlineto{\pgfqpoint{0.825105in}{1.190585in}}%
\pgfpathlineto{\pgfqpoint{0.819771in}{1.184947in}}%
\pgfpathlineto{\pgfqpoint{0.814783in}{1.179222in}}%
\pgfpathlineto{\pgfqpoint{0.810147in}{1.173417in}}%
\pgfpathlineto{\pgfqpoint{0.805869in}{1.167537in}}%
\pgfpathclose%
\pgfusepath{fill}%
\end{pgfscope}%
\begin{pgfscope}%
\pgfpathrectangle{\pgfqpoint{0.041670in}{0.041670in}}{\pgfqpoint{2.216660in}{2.216660in}}%
\pgfusepath{clip}%
\pgfsetbuttcap%
\pgfsetroundjoin%
\definecolor{currentfill}{rgb}{0.274952,0.037752,0.364543}%
\pgfsetfillcolor{currentfill}%
\pgfsetlinewidth{0.000000pt}%
\definecolor{currentstroke}{rgb}{0.000000,0.000000,0.000000}%
\pgfsetstrokecolor{currentstroke}%
\pgfsetdash{}{0pt}%
\pgfpathmoveto{\pgfqpoint{1.704396in}{0.925752in}}%
\pgfpathlineto{\pgfqpoint{1.708047in}{0.922722in}}%
\pgfpathlineto{\pgfqpoint{1.711702in}{0.919876in}}%
\pgfpathlineto{\pgfqpoint{1.715361in}{0.917216in}}%
\pgfpathlineto{\pgfqpoint{1.719023in}{0.914748in}}%
\pgfpathlineto{\pgfqpoint{1.720872in}{0.905890in}}%
\pgfpathlineto{\pgfqpoint{1.722187in}{0.896996in}}%
\pgfpathlineto{\pgfqpoint{1.722964in}{0.888073in}}%
\pgfpathlineto{\pgfqpoint{1.723201in}{0.879130in}}%
\pgfpathlineto{\pgfqpoint{1.719492in}{0.881854in}}%
\pgfpathlineto{\pgfqpoint{1.715787in}{0.884769in}}%
\pgfpathlineto{\pgfqpoint{1.712086in}{0.887872in}}%
\pgfpathlineto{\pgfqpoint{1.708389in}{0.891159in}}%
\pgfpathlineto{\pgfqpoint{1.708176in}{0.899843in}}%
\pgfpathlineto{\pgfqpoint{1.707438in}{0.908509in}}%
\pgfpathlineto{\pgfqpoint{1.706177in}{0.917149in}}%
\pgfpathlineto{\pgfqpoint{1.704396in}{0.925752in}}%
\pgfpathclose%
\pgfusepath{fill}%
\end{pgfscope}%
\begin{pgfscope}%
\pgfpathrectangle{\pgfqpoint{0.041670in}{0.041670in}}{\pgfqpoint{2.216660in}{2.216660in}}%
\pgfusepath{clip}%
\pgfsetbuttcap%
\pgfsetroundjoin%
\definecolor{currentfill}{rgb}{0.280255,0.165693,0.476498}%
\pgfsetfillcolor{currentfill}%
\pgfsetlinewidth{0.000000pt}%
\definecolor{currentstroke}{rgb}{0.000000,0.000000,0.000000}%
\pgfsetstrokecolor{currentstroke}%
\pgfsetdash{}{0pt}%
\pgfpathmoveto{\pgfqpoint{0.712254in}{0.989118in}}%
\pgfpathlineto{\pgfqpoint{0.708617in}{0.983643in}}%
\pgfpathlineto{\pgfqpoint{0.704980in}{0.978289in}}%
\pgfpathlineto{\pgfqpoint{0.701343in}{0.973059in}}%
\pgfpathlineto{\pgfqpoint{0.697704in}{0.967957in}}%
\pgfpathlineto{\pgfqpoint{0.699342in}{0.975786in}}%
\pgfpathlineto{\pgfqpoint{0.701453in}{0.983577in}}%
\pgfpathlineto{\pgfqpoint{0.704033in}{0.991321in}}%
\pgfpathlineto{\pgfqpoint{0.707080in}{0.999011in}}%
\pgfpathlineto{\pgfqpoint{0.710633in}{1.003857in}}%
\pgfpathlineto{\pgfqpoint{0.714185in}{1.008832in}}%
\pgfpathlineto{\pgfqpoint{0.717736in}{1.013931in}}%
\pgfpathlineto{\pgfqpoint{0.721287in}{1.019150in}}%
\pgfpathlineto{\pgfqpoint{0.718347in}{1.011712in}}%
\pgfpathlineto{\pgfqpoint{0.715859in}{1.004223in}}%
\pgfpathlineto{\pgfqpoint{0.713827in}{0.996689in}}%
\pgfpathlineto{\pgfqpoint{0.712254in}{0.989118in}}%
\pgfpathclose%
\pgfusepath{fill}%
\end{pgfscope}%
\begin{pgfscope}%
\pgfpathrectangle{\pgfqpoint{0.041670in}{0.041670in}}{\pgfqpoint{2.216660in}{2.216660in}}%
\pgfusepath{clip}%
\pgfsetbuttcap%
\pgfsetroundjoin%
\definecolor{currentfill}{rgb}{0.120081,0.622161,0.534946}%
\pgfsetfillcolor{currentfill}%
\pgfsetlinewidth{0.000000pt}%
\definecolor{currentstroke}{rgb}{0.000000,0.000000,0.000000}%
\pgfsetstrokecolor{currentstroke}%
\pgfsetdash{}{0pt}%
\pgfpathmoveto{\pgfqpoint{1.252073in}{1.455125in}}%
\pgfpathlineto{\pgfqpoint{1.252977in}{1.449476in}}%
\pgfpathlineto{\pgfqpoint{1.253880in}{1.443801in}}%
\pgfpathlineto{\pgfqpoint{1.254781in}{1.438101in}}%
\pgfpathlineto{\pgfqpoint{1.255682in}{1.432379in}}%
\pgfpathlineto{\pgfqpoint{1.264855in}{1.431157in}}%
\pgfpathlineto{\pgfqpoint{1.273949in}{1.429796in}}%
\pgfpathlineto{\pgfqpoint{1.282955in}{1.428297in}}%
\pgfpathlineto{\pgfqpoint{1.281730in}{1.434071in}}%
\pgfpathlineto{\pgfqpoint{1.280503in}{1.439823in}}%
\pgfpathlineto{\pgfqpoint{1.279274in}{1.445550in}}%
\pgfpathlineto{\pgfqpoint{1.278044in}{1.451251in}}%
\pgfpathlineto{\pgfqpoint{1.269468in}{1.452674in}}%
\pgfpathlineto{\pgfqpoint{1.260809in}{1.453965in}}%
\pgfpathlineto{\pgfqpoint{1.252073in}{1.455125in}}%
\pgfpathclose%
\pgfusepath{fill}%
\end{pgfscope}%
\begin{pgfscope}%
\pgfpathrectangle{\pgfqpoint{0.041670in}{0.041670in}}{\pgfqpoint{2.216660in}{2.216660in}}%
\pgfusepath{clip}%
\pgfsetbuttcap%
\pgfsetroundjoin%
\definecolor{currentfill}{rgb}{0.248629,0.278775,0.534556}%
\pgfsetfillcolor{currentfill}%
\pgfsetlinewidth{0.000000pt}%
\definecolor{currentstroke}{rgb}{0.000000,0.000000,0.000000}%
\pgfsetstrokecolor{currentstroke}%
\pgfsetdash{}{0pt}%
\pgfpathmoveto{\pgfqpoint{0.763919in}{1.089866in}}%
\pgfpathlineto{\pgfqpoint{0.760362in}{1.083479in}}%
\pgfpathlineto{\pgfqpoint{0.756806in}{1.077171in}}%
\pgfpathlineto{\pgfqpoint{0.753252in}{1.070945in}}%
\pgfpathlineto{\pgfqpoint{0.749698in}{1.064804in}}%
\pgfpathlineto{\pgfqpoint{0.752848in}{1.071681in}}%
\pgfpathlineto{\pgfqpoint{0.756415in}{1.078498in}}%
\pgfpathlineto{\pgfqpoint{0.760394in}{1.085248in}}%
\pgfpathlineto{\pgfqpoint{0.764779in}{1.091924in}}%
\pgfpathlineto{\pgfqpoint{0.768195in}{1.097819in}}%
\pgfpathlineto{\pgfqpoint{0.771612in}{1.103799in}}%
\pgfpathlineto{\pgfqpoint{0.775031in}{1.109860in}}%
\pgfpathlineto{\pgfqpoint{0.778450in}{1.116000in}}%
\pgfpathlineto{\pgfqpoint{0.774222in}{1.109565in}}%
\pgfpathlineto{\pgfqpoint{0.770388in}{1.103061in}}%
\pgfpathlineto{\pgfqpoint{0.766952in}{1.096492in}}%
\pgfpathlineto{\pgfqpoint{0.763919in}{1.089866in}}%
\pgfpathclose%
\pgfusepath{fill}%
\end{pgfscope}%
\begin{pgfscope}%
\pgfpathrectangle{\pgfqpoint{0.041670in}{0.041670in}}{\pgfqpoint{2.216660in}{2.216660in}}%
\pgfusepath{clip}%
\pgfsetbuttcap%
\pgfsetroundjoin%
\definecolor{currentfill}{rgb}{0.122606,0.585371,0.546557}%
\pgfsetfillcolor{currentfill}%
\pgfsetlinewidth{0.000000pt}%
\definecolor{currentstroke}{rgb}{0.000000,0.000000,0.000000}%
\pgfsetstrokecolor{currentstroke}%
\pgfsetdash{}{0pt}%
\pgfpathmoveto{\pgfqpoint{1.002092in}{1.409152in}}%
\pgfpathlineto{\pgfqpoint{0.999972in}{1.403115in}}%
\pgfpathlineto{\pgfqpoint{0.997856in}{1.397061in}}%
\pgfpathlineto{\pgfqpoint{0.995742in}{1.390993in}}%
\pgfpathlineto{\pgfqpoint{0.993631in}{1.384913in}}%
\pgfpathlineto{\pgfqpoint{1.001870in}{1.387704in}}%
\pgfpathlineto{\pgfqpoint{1.010276in}{1.390366in}}%
\pgfpathlineto{\pgfqpoint{1.018842in}{1.392898in}}%
\pgfpathlineto{\pgfqpoint{1.027559in}{1.395296in}}%
\pgfpathlineto{\pgfqpoint{1.029283in}{1.401249in}}%
\pgfpathlineto{\pgfqpoint{1.031009in}{1.407190in}}%
\pgfpathlineto{\pgfqpoint{1.032738in}{1.413118in}}%
\pgfpathlineto{\pgfqpoint{1.034468in}{1.419029in}}%
\pgfpathlineto{\pgfqpoint{1.026149in}{1.416747in}}%
\pgfpathlineto{\pgfqpoint{1.017975in}{1.414339in}}%
\pgfpathlineto{\pgfqpoint{1.009953in}{1.411807in}}%
\pgfpathlineto{\pgfqpoint{1.002092in}{1.409152in}}%
\pgfpathclose%
\pgfusepath{fill}%
\end{pgfscope}%
\begin{pgfscope}%
\pgfpathrectangle{\pgfqpoint{0.041670in}{0.041670in}}{\pgfqpoint{2.216660in}{2.216660in}}%
\pgfusepath{clip}%
\pgfsetbuttcap%
\pgfsetroundjoin%
\definecolor{currentfill}{rgb}{0.267004,0.004874,0.329415}%
\pgfsetfillcolor{currentfill}%
\pgfsetlinewidth{0.000000pt}%
\definecolor{currentstroke}{rgb}{0.000000,0.000000,0.000000}%
\pgfsetstrokecolor{currentstroke}%
\pgfsetdash{}{0pt}%
\pgfpathmoveto{\pgfqpoint{0.607085in}{0.856367in}}%
\pgfpathlineto{\pgfqpoint{0.603325in}{0.855513in}}%
\pgfpathlineto{\pgfqpoint{0.599559in}{0.854896in}}%
\pgfpathlineto{\pgfqpoint{0.595786in}{0.854521in}}%
\pgfpathlineto{\pgfqpoint{0.592006in}{0.854392in}}%
\pgfpathlineto{\pgfqpoint{0.591801in}{0.864106in}}%
\pgfpathlineto{\pgfqpoint{0.592183in}{0.873807in}}%
\pgfpathlineto{\pgfqpoint{0.593149in}{0.883483in}}%
\pgfpathlineto{\pgfqpoint{0.594696in}{0.893125in}}%
\pgfpathlineto{\pgfqpoint{0.598439in}{0.893006in}}%
\pgfpathlineto{\pgfqpoint{0.602175in}{0.893132in}}%
\pgfpathlineto{\pgfqpoint{0.605904in}{0.893499in}}%
\pgfpathlineto{\pgfqpoint{0.609627in}{0.894101in}}%
\pgfpathlineto{\pgfqpoint{0.608139in}{0.884707in}}%
\pgfpathlineto{\pgfqpoint{0.607218in}{0.875280in}}%
\pgfpathlineto{\pgfqpoint{0.606865in}{0.865830in}}%
\pgfpathlineto{\pgfqpoint{0.607085in}{0.856367in}}%
\pgfpathclose%
\pgfusepath{fill}%
\end{pgfscope}%
\begin{pgfscope}%
\pgfpathrectangle{\pgfqpoint{0.041670in}{0.041670in}}{\pgfqpoint{2.216660in}{2.216660in}}%
\pgfusepath{clip}%
\pgfsetbuttcap%
\pgfsetroundjoin%
\definecolor{currentfill}{rgb}{0.267004,0.004874,0.329415}%
\pgfsetfillcolor{currentfill}%
\pgfsetlinewidth{0.000000pt}%
\definecolor{currentstroke}{rgb}{0.000000,0.000000,0.000000}%
\pgfsetstrokecolor{currentstroke}%
\pgfsetdash{}{0pt}%
\pgfpathmoveto{\pgfqpoint{0.592006in}{0.854392in}}%
\pgfpathlineto{\pgfqpoint{0.588218in}{0.854514in}}%
\pgfpathlineto{\pgfqpoint{0.584423in}{0.854892in}}%
\pgfpathlineto{\pgfqpoint{0.580620in}{0.855531in}}%
\pgfpathlineto{\pgfqpoint{0.576808in}{0.856436in}}%
\pgfpathlineto{\pgfqpoint{0.576619in}{0.866399in}}%
\pgfpathlineto{\pgfqpoint{0.577031in}{0.876346in}}%
\pgfpathlineto{\pgfqpoint{0.578043in}{0.886268in}}%
\pgfpathlineto{\pgfqpoint{0.579651in}{0.896154in}}%
\pgfpathlineto{\pgfqpoint{0.583423in}{0.895005in}}%
\pgfpathlineto{\pgfqpoint{0.587188in}{0.894120in}}%
\pgfpathlineto{\pgfqpoint{0.590946in}{0.893495in}}%
\pgfpathlineto{\pgfqpoint{0.594696in}{0.893125in}}%
\pgfpathlineto{\pgfqpoint{0.593149in}{0.883483in}}%
\pgfpathlineto{\pgfqpoint{0.592183in}{0.873807in}}%
\pgfpathlineto{\pgfqpoint{0.591801in}{0.864106in}}%
\pgfpathlineto{\pgfqpoint{0.592006in}{0.854392in}}%
\pgfpathclose%
\pgfusepath{fill}%
\end{pgfscope}%
\begin{pgfscope}%
\pgfpathrectangle{\pgfqpoint{0.041670in}{0.041670in}}{\pgfqpoint{2.216660in}{2.216660in}}%
\pgfusepath{clip}%
\pgfsetbuttcap%
\pgfsetroundjoin%
\definecolor{currentfill}{rgb}{0.120081,0.622161,0.534946}%
\pgfsetfillcolor{currentfill}%
\pgfsetlinewidth{0.000000pt}%
\definecolor{currentstroke}{rgb}{0.000000,0.000000,0.000000}%
\pgfsetstrokecolor{currentstroke}%
\pgfsetdash{}{0pt}%
\pgfpathmoveto{\pgfqpoint{1.074320in}{1.449877in}}%
\pgfpathlineto{\pgfqpoint{1.072995in}{1.444158in}}%
\pgfpathlineto{\pgfqpoint{1.071672in}{1.438412in}}%
\pgfpathlineto{\pgfqpoint{1.070350in}{1.432642in}}%
\pgfpathlineto{\pgfqpoint{1.069030in}{1.426850in}}%
\pgfpathlineto{\pgfqpoint{1.077951in}{1.428471in}}%
\pgfpathlineto{\pgfqpoint{1.086967in}{1.429954in}}%
\pgfpathlineto{\pgfqpoint{1.096070in}{1.431300in}}%
\pgfpathlineto{\pgfqpoint{1.105252in}{1.432506in}}%
\pgfpathlineto{\pgfqpoint{1.106140in}{1.438226in}}%
\pgfpathlineto{\pgfqpoint{1.107029in}{1.443924in}}%
\pgfpathlineto{\pgfqpoint{1.107920in}{1.449598in}}%
\pgfpathlineto{\pgfqpoint{1.108812in}{1.455246in}}%
\pgfpathlineto{\pgfqpoint{1.100068in}{1.454101in}}%
\pgfpathlineto{\pgfqpoint{1.091400in}{1.452824in}}%
\pgfpathlineto{\pgfqpoint{1.082814in}{1.451416in}}%
\pgfpathlineto{\pgfqpoint{1.074320in}{1.449877in}}%
\pgfpathclose%
\pgfusepath{fill}%
\end{pgfscope}%
\begin{pgfscope}%
\pgfpathrectangle{\pgfqpoint{0.041670in}{0.041670in}}{\pgfqpoint{2.216660in}{2.216660in}}%
\pgfusepath{clip}%
\pgfsetbuttcap%
\pgfsetroundjoin%
\definecolor{currentfill}{rgb}{0.277941,0.056324,0.381191}%
\pgfsetfillcolor{currentfill}%
\pgfsetlinewidth{0.000000pt}%
\definecolor{currentstroke}{rgb}{0.000000,0.000000,0.000000}%
\pgfsetstrokecolor{currentstroke}%
\pgfsetdash{}{0pt}%
\pgfpathmoveto{\pgfqpoint{1.808702in}{0.924666in}}%
\pgfpathlineto{\pgfqpoint{1.812543in}{0.928482in}}%
\pgfpathlineto{\pgfqpoint{1.816394in}{0.932614in}}%
\pgfpathlineto{\pgfqpoint{1.820257in}{0.937069in}}%
\pgfpathlineto{\pgfqpoint{1.824132in}{0.941853in}}%
\pgfpathlineto{\pgfqpoint{1.826496in}{0.931305in}}%
\pgfpathlineto{\pgfqpoint{1.828224in}{0.920709in}}%
\pgfpathlineto{\pgfqpoint{1.829311in}{0.910073in}}%
\pgfpathlineto{\pgfqpoint{1.829752in}{0.899410in}}%
\pgfpathlineto{\pgfqpoint{1.825821in}{0.894855in}}%
\pgfpathlineto{\pgfqpoint{1.821901in}{0.890630in}}%
\pgfpathlineto{\pgfqpoint{1.817994in}{0.886729in}}%
\pgfpathlineto{\pgfqpoint{1.814098in}{0.883146in}}%
\pgfpathlineto{\pgfqpoint{1.813690in}{0.893577in}}%
\pgfpathlineto{\pgfqpoint{1.812652in}{0.903981in}}%
\pgfpathlineto{\pgfqpoint{1.810988in}{0.914347in}}%
\pgfpathlineto{\pgfqpoint{1.808702in}{0.924666in}}%
\pgfpathclose%
\pgfusepath{fill}%
\end{pgfscope}%
\begin{pgfscope}%
\pgfpathrectangle{\pgfqpoint{0.041670in}{0.041670in}}{\pgfqpoint{2.216660in}{2.216660in}}%
\pgfusepath{clip}%
\pgfsetbuttcap%
\pgfsetroundjoin%
\definecolor{currentfill}{rgb}{0.268510,0.009605,0.335427}%
\pgfsetfillcolor{currentfill}%
\pgfsetlinewidth{0.000000pt}%
\definecolor{currentstroke}{rgb}{0.000000,0.000000,0.000000}%
\pgfsetstrokecolor{currentstroke}%
\pgfsetdash{}{0pt}%
\pgfpathmoveto{\pgfqpoint{0.622062in}{0.862057in}}%
\pgfpathlineto{\pgfqpoint{0.618326in}{0.860302in}}%
\pgfpathlineto{\pgfqpoint{0.614585in}{0.858766in}}%
\pgfpathlineto{\pgfqpoint{0.610838in}{0.857452in}}%
\pgfpathlineto{\pgfqpoint{0.607085in}{0.856367in}}%
\pgfpathlineto{\pgfqpoint{0.606865in}{0.865830in}}%
\pgfpathlineto{\pgfqpoint{0.607218in}{0.875280in}}%
\pgfpathlineto{\pgfqpoint{0.608139in}{0.884707in}}%
\pgfpathlineto{\pgfqpoint{0.609627in}{0.894101in}}%
\pgfpathlineto{\pgfqpoint{0.613345in}{0.894936in}}%
\pgfpathlineto{\pgfqpoint{0.617056in}{0.895997in}}%
\pgfpathlineto{\pgfqpoint{0.620762in}{0.897280in}}%
\pgfpathlineto{\pgfqpoint{0.624463in}{0.898782in}}%
\pgfpathlineto{\pgfqpoint{0.623032in}{0.889638in}}%
\pgfpathlineto{\pgfqpoint{0.622153in}{0.880463in}}%
\pgfpathlineto{\pgfqpoint{0.621829in}{0.871266in}}%
\pgfpathlineto{\pgfqpoint{0.622062in}{0.862057in}}%
\pgfpathclose%
\pgfusepath{fill}%
\end{pgfscope}%
\begin{pgfscope}%
\pgfpathrectangle{\pgfqpoint{0.041670in}{0.041670in}}{\pgfqpoint{2.216660in}{2.216660in}}%
\pgfusepath{clip}%
\pgfsetbuttcap%
\pgfsetroundjoin%
\definecolor{currentfill}{rgb}{0.274128,0.199721,0.498911}%
\pgfsetfillcolor{currentfill}%
\pgfsetlinewidth{0.000000pt}%
\definecolor{currentstroke}{rgb}{0.000000,0.000000,0.000000}%
\pgfsetstrokecolor{currentstroke}%
\pgfsetdash{}{0pt}%
\pgfpathmoveto{\pgfqpoint{1.621536in}{1.047496in}}%
\pgfpathlineto{\pgfqpoint{1.625060in}{1.041888in}}%
\pgfpathlineto{\pgfqpoint{1.628585in}{1.036385in}}%
\pgfpathlineto{\pgfqpoint{1.632109in}{1.030992in}}%
\pgfpathlineto{\pgfqpoint{1.635633in}{1.025712in}}%
\pgfpathlineto{\pgfqpoint{1.638972in}{1.018326in}}%
\pgfpathlineto{\pgfqpoint{1.641862in}{1.010882in}}%
\pgfpathlineto{\pgfqpoint{1.644299in}{1.003388in}}%
\pgfpathlineto{\pgfqpoint{1.646281in}{0.995850in}}%
\pgfpathlineto{\pgfqpoint{1.642659in}{1.001384in}}%
\pgfpathlineto{\pgfqpoint{1.639037in}{1.007031in}}%
\pgfpathlineto{\pgfqpoint{1.635416in}{1.012788in}}%
\pgfpathlineto{\pgfqpoint{1.631795in}{1.018651in}}%
\pgfpathlineto{\pgfqpoint{1.629890in}{1.025932in}}%
\pgfpathlineto{\pgfqpoint{1.627542in}{1.033171in}}%
\pgfpathlineto{\pgfqpoint{1.624756in}{1.040362in}}%
\pgfpathlineto{\pgfqpoint{1.621536in}{1.047496in}}%
\pgfpathclose%
\pgfusepath{fill}%
\end{pgfscope}%
\begin{pgfscope}%
\pgfpathrectangle{\pgfqpoint{0.041670in}{0.041670in}}{\pgfqpoint{2.216660in}{2.216660in}}%
\pgfusepath{clip}%
\pgfsetbuttcap%
\pgfsetroundjoin%
\definecolor{currentfill}{rgb}{0.268510,0.009605,0.335427}%
\pgfsetfillcolor{currentfill}%
\pgfsetlinewidth{0.000000pt}%
\definecolor{currentstroke}{rgb}{0.000000,0.000000,0.000000}%
\pgfsetstrokecolor{currentstroke}%
\pgfsetdash{}{0pt}%
\pgfpathmoveto{\pgfqpoint{0.576808in}{0.856436in}}%
\pgfpathlineto{\pgfqpoint{0.572988in}{0.857612in}}%
\pgfpathlineto{\pgfqpoint{0.569159in}{0.859064in}}%
\pgfpathlineto{\pgfqpoint{0.565321in}{0.860797in}}%
\pgfpathlineto{\pgfqpoint{0.561473in}{0.862816in}}%
\pgfpathlineto{\pgfqpoint{0.561300in}{0.873022in}}%
\pgfpathlineto{\pgfqpoint{0.561745in}{0.883212in}}%
\pgfpathlineto{\pgfqpoint{0.562804in}{0.893375in}}%
\pgfpathlineto{\pgfqpoint{0.564474in}{0.903502in}}%
\pgfpathlineto{\pgfqpoint{0.568281in}{0.901242in}}%
\pgfpathlineto{\pgfqpoint{0.572080in}{0.899268in}}%
\pgfpathlineto{\pgfqpoint{0.575869in}{0.897574in}}%
\pgfpathlineto{\pgfqpoint{0.579651in}{0.896154in}}%
\pgfpathlineto{\pgfqpoint{0.578043in}{0.886268in}}%
\pgfpathlineto{\pgfqpoint{0.577031in}{0.876346in}}%
\pgfpathlineto{\pgfqpoint{0.576619in}{0.866399in}}%
\pgfpathlineto{\pgfqpoint{0.576808in}{0.856436in}}%
\pgfpathclose%
\pgfusepath{fill}%
\end{pgfscope}%
\begin{pgfscope}%
\pgfpathrectangle{\pgfqpoint{0.041670in}{0.041670in}}{\pgfqpoint{2.216660in}{2.216660in}}%
\pgfusepath{clip}%
\pgfsetbuttcap%
\pgfsetroundjoin%
\definecolor{currentfill}{rgb}{0.179019,0.433756,0.557430}%
\pgfsetfillcolor{currentfill}%
\pgfsetlinewidth{0.000000pt}%
\definecolor{currentstroke}{rgb}{0.000000,0.000000,0.000000}%
\pgfsetstrokecolor{currentstroke}%
\pgfsetdash{}{0pt}%
\pgfpathmoveto{\pgfqpoint{0.851165in}{1.243403in}}%
\pgfpathlineto{\pgfqpoint{0.847899in}{1.236683in}}%
\pgfpathlineto{\pgfqpoint{0.844635in}{1.229990in}}%
\pgfpathlineto{\pgfqpoint{0.841373in}{1.223328in}}%
\pgfpathlineto{\pgfqpoint{0.838115in}{1.216700in}}%
\pgfpathlineto{\pgfqpoint{0.843572in}{1.222024in}}%
\pgfpathlineto{\pgfqpoint{0.849352in}{1.227255in}}%
\pgfpathlineto{\pgfqpoint{0.855448in}{1.232389in}}%
\pgfpathlineto{\pgfqpoint{0.861854in}{1.237422in}}%
\pgfpathlineto{\pgfqpoint{0.864877in}{1.243833in}}%
\pgfpathlineto{\pgfqpoint{0.867903in}{1.250278in}}%
\pgfpathlineto{\pgfqpoint{0.870931in}{1.256754in}}%
\pgfpathlineto{\pgfqpoint{0.873963in}{1.263258in}}%
\pgfpathlineto{\pgfqpoint{0.867810in}{1.258435in}}%
\pgfpathlineto{\pgfqpoint{0.861955in}{1.253516in}}%
\pgfpathlineto{\pgfqpoint{0.856405in}{1.248503in}}%
\pgfpathlineto{\pgfqpoint{0.851165in}{1.243403in}}%
\pgfpathclose%
\pgfusepath{fill}%
\end{pgfscope}%
\begin{pgfscope}%
\pgfpathrectangle{\pgfqpoint{0.041670in}{0.041670in}}{\pgfqpoint{2.216660in}{2.216660in}}%
\pgfusepath{clip}%
\pgfsetbuttcap%
\pgfsetroundjoin%
\definecolor{currentfill}{rgb}{0.172719,0.448791,0.557885}%
\pgfsetfillcolor{currentfill}%
\pgfsetlinewidth{0.000000pt}%
\definecolor{currentstroke}{rgb}{0.000000,0.000000,0.000000}%
\pgfsetstrokecolor{currentstroke}%
\pgfsetdash{}{0pt}%
\pgfpathmoveto{\pgfqpoint{0.451832in}{1.207283in}}%
\pgfpathlineto{\pgfqpoint{0.447630in}{1.222481in}}%
\pgfpathlineto{\pgfqpoint{0.443406in}{1.238184in}}%
\pgfpathlineto{\pgfqpoint{0.439158in}{1.254400in}}%
\pgfpathlineto{\pgfqpoint{0.445101in}{1.265914in}}%
\pgfpathlineto{\pgfqpoint{0.451747in}{1.277310in}}%
\pgfpathlineto{\pgfqpoint{0.459086in}{1.288579in}}%
\pgfpathlineto{\pgfqpoint{0.467108in}{1.299710in}}%
\pgfpathlineto{\pgfqpoint{0.471168in}{1.283343in}}%
\pgfpathlineto{\pgfqpoint{0.475207in}{1.267486in}}%
\pgfpathlineto{\pgfqpoint{0.479224in}{1.252129in}}%
\pgfpathlineto{\pgfqpoint{0.471358in}{1.241111in}}%
\pgfpathlineto{\pgfqpoint{0.464164in}{1.229957in}}%
\pgfpathlineto{\pgfqpoint{0.457652in}{1.218678in}}%
\pgfpathlineto{\pgfqpoint{0.451832in}{1.207283in}}%
\pgfpathclose%
\pgfusepath{fill}%
\end{pgfscope}%
\begin{pgfscope}%
\pgfpathrectangle{\pgfqpoint{0.041670in}{0.041670in}}{\pgfqpoint{2.216660in}{2.216660in}}%
\pgfusepath{clip}%
\pgfsetbuttcap%
\pgfsetroundjoin%
\definecolor{currentfill}{rgb}{0.271305,0.019942,0.347269}%
\pgfsetfillcolor{currentfill}%
\pgfsetlinewidth{0.000000pt}%
\definecolor{currentstroke}{rgb}{0.000000,0.000000,0.000000}%
\pgfsetstrokecolor{currentstroke}%
\pgfsetdash{}{0pt}%
\pgfpathmoveto{\pgfqpoint{0.636954in}{0.871172in}}%
\pgfpathlineto{\pgfqpoint{0.633238in}{0.868588in}}%
\pgfpathlineto{\pgfqpoint{0.629518in}{0.866205in}}%
\pgfpathlineto{\pgfqpoint{0.625792in}{0.864026in}}%
\pgfpathlineto{\pgfqpoint{0.622062in}{0.862057in}}%
\pgfpathlineto{\pgfqpoint{0.621829in}{0.871266in}}%
\pgfpathlineto{\pgfqpoint{0.622153in}{0.880463in}}%
\pgfpathlineto{\pgfqpoint{0.623032in}{0.889638in}}%
\pgfpathlineto{\pgfqpoint{0.624463in}{0.898782in}}%
\pgfpathlineto{\pgfqpoint{0.628158in}{0.900496in}}%
\pgfpathlineto{\pgfqpoint{0.631849in}{0.902420in}}%
\pgfpathlineto{\pgfqpoint{0.635535in}{0.904548in}}%
\pgfpathlineto{\pgfqpoint{0.639217in}{0.906876in}}%
\pgfpathlineto{\pgfqpoint{0.637843in}{0.897985in}}%
\pgfpathlineto{\pgfqpoint{0.637006in}{0.889065in}}%
\pgfpathlineto{\pgfqpoint{0.636709in}{0.880125in}}%
\pgfpathlineto{\pgfqpoint{0.636954in}{0.871172in}}%
\pgfpathclose%
\pgfusepath{fill}%
\end{pgfscope}%
\begin{pgfscope}%
\pgfpathrectangle{\pgfqpoint{0.041670in}{0.041670in}}{\pgfqpoint{2.216660in}{2.216660in}}%
\pgfusepath{clip}%
\pgfsetbuttcap%
\pgfsetroundjoin%
\definecolor{currentfill}{rgb}{0.260571,0.246922,0.522828}%
\pgfsetfillcolor{currentfill}%
\pgfsetlinewidth{0.000000pt}%
\definecolor{currentstroke}{rgb}{0.000000,0.000000,0.000000}%
\pgfsetstrokecolor{currentstroke}%
\pgfsetdash{}{0pt}%
\pgfpathmoveto{\pgfqpoint{0.485876in}{1.017063in}}%
\pgfpathlineto{\pgfqpoint{0.481794in}{1.026617in}}%
\pgfpathlineto{\pgfqpoint{0.477695in}{1.036586in}}%
\pgfpathlineto{\pgfqpoint{0.473579in}{1.046979in}}%
\pgfpathlineto{\pgfqpoint{0.469443in}{1.057803in}}%
\pgfpathlineto{\pgfqpoint{0.472224in}{1.069187in}}%
\pgfpathlineto{\pgfqpoint{0.475696in}{1.080503in}}%
\pgfpathlineto{\pgfqpoint{0.479855in}{1.091739in}}%
\pgfpathlineto{\pgfqpoint{0.484691in}{1.102886in}}%
\pgfpathlineto{\pgfqpoint{0.488712in}{1.091871in}}%
\pgfpathlineto{\pgfqpoint{0.492715in}{1.081284in}}%
\pgfpathlineto{\pgfqpoint{0.496701in}{1.071117in}}%
\pgfpathlineto{\pgfqpoint{0.500670in}{1.061365in}}%
\pgfpathlineto{\pgfqpoint{0.495970in}{1.050410in}}%
\pgfpathlineto{\pgfqpoint{0.491933in}{1.039368in}}%
\pgfpathlineto{\pgfqpoint{0.488566in}{1.028249in}}%
\pgfpathlineto{\pgfqpoint{0.485876in}{1.017063in}}%
\pgfpathclose%
\pgfusepath{fill}%
\end{pgfscope}%
\begin{pgfscope}%
\pgfpathrectangle{\pgfqpoint{0.041670in}{0.041670in}}{\pgfqpoint{2.216660in}{2.216660in}}%
\pgfusepath{clip}%
\pgfsetbuttcap%
\pgfsetroundjoin%
\definecolor{currentfill}{rgb}{0.279566,0.067836,0.391917}%
\pgfsetfillcolor{currentfill}%
\pgfsetlinewidth{0.000000pt}%
\definecolor{currentstroke}{rgb}{0.000000,0.000000,0.000000}%
\pgfsetstrokecolor{currentstroke}%
\pgfsetdash{}{0pt}%
\pgfpathmoveto{\pgfqpoint{1.689820in}{0.939621in}}%
\pgfpathlineto{\pgfqpoint{1.693460in}{0.935900in}}%
\pgfpathlineto{\pgfqpoint{1.697102in}{0.932345in}}%
\pgfpathlineto{\pgfqpoint{1.700748in}{0.928961in}}%
\pgfpathlineto{\pgfqpoint{1.704396in}{0.925752in}}%
\pgfpathlineto{\pgfqpoint{1.706177in}{0.917149in}}%
\pgfpathlineto{\pgfqpoint{1.707438in}{0.908509in}}%
\pgfpathlineto{\pgfqpoint{1.708176in}{0.899843in}}%
\pgfpathlineto{\pgfqpoint{1.708389in}{0.891159in}}%
\pgfpathlineto{\pgfqpoint{1.704695in}{0.894625in}}%
\pgfpathlineto{\pgfqpoint{1.701005in}{0.898266in}}%
\pgfpathlineto{\pgfqpoint{1.697318in}{0.902079in}}%
\pgfpathlineto{\pgfqpoint{1.693633in}{0.906058in}}%
\pgfpathlineto{\pgfqpoint{1.693444in}{0.914483in}}%
\pgfpathlineto{\pgfqpoint{1.692743in}{0.922891in}}%
\pgfpathlineto{\pgfqpoint{1.691535in}{0.931273in}}%
\pgfpathlineto{\pgfqpoint{1.689820in}{0.939621in}}%
\pgfpathclose%
\pgfusepath{fill}%
\end{pgfscope}%
\begin{pgfscope}%
\pgfpathrectangle{\pgfqpoint{0.041670in}{0.041670in}}{\pgfqpoint{2.216660in}{2.216660in}}%
\pgfusepath{clip}%
\pgfsetbuttcap%
\pgfsetroundjoin%
\definecolor{currentfill}{rgb}{0.272594,0.025563,0.353093}%
\pgfsetfillcolor{currentfill}%
\pgfsetlinewidth{0.000000pt}%
\definecolor{currentstroke}{rgb}{0.000000,0.000000,0.000000}%
\pgfsetstrokecolor{currentstroke}%
\pgfsetdash{}{0pt}%
\pgfpathmoveto{\pgfqpoint{0.561473in}{0.862816in}}%
\pgfpathlineto{\pgfqpoint{0.557616in}{0.865126in}}%
\pgfpathlineto{\pgfqpoint{0.553748in}{0.867734in}}%
\pgfpathlineto{\pgfqpoint{0.549870in}{0.870643in}}%
\pgfpathlineto{\pgfqpoint{0.545981in}{0.873860in}}%
\pgfpathlineto{\pgfqpoint{0.545826in}{0.884306in}}%
\pgfpathlineto{\pgfqpoint{0.546304in}{0.894734in}}%
\pgfpathlineto{\pgfqpoint{0.547412in}{0.905134in}}%
\pgfpathlineto{\pgfqpoint{0.549145in}{0.915496in}}%
\pgfpathlineto{\pgfqpoint{0.552993in}{0.912044in}}%
\pgfpathlineto{\pgfqpoint{0.556830in}{0.908897in}}%
\pgfpathlineto{\pgfqpoint{0.560656in}{0.906052in}}%
\pgfpathlineto{\pgfqpoint{0.564474in}{0.903502in}}%
\pgfpathlineto{\pgfqpoint{0.562804in}{0.893375in}}%
\pgfpathlineto{\pgfqpoint{0.561745in}{0.883212in}}%
\pgfpathlineto{\pgfqpoint{0.561300in}{0.873022in}}%
\pgfpathlineto{\pgfqpoint{0.561473in}{0.862816in}}%
\pgfpathclose%
\pgfusepath{fill}%
\end{pgfscope}%
\begin{pgfscope}%
\pgfpathrectangle{\pgfqpoint{0.041670in}{0.041670in}}{\pgfqpoint{2.216660in}{2.216660in}}%
\pgfusepath{clip}%
\pgfsetbuttcap%
\pgfsetroundjoin%
\definecolor{currentfill}{rgb}{0.231674,0.318106,0.544834}%
\pgfsetfillcolor{currentfill}%
\pgfsetlinewidth{0.000000pt}%
\definecolor{currentstroke}{rgb}{0.000000,0.000000,0.000000}%
\pgfsetstrokecolor{currentstroke}%
\pgfsetdash{}{0pt}%
\pgfpathmoveto{\pgfqpoint{1.563830in}{1.146717in}}%
\pgfpathlineto{\pgfqpoint{1.567219in}{1.140351in}}%
\pgfpathlineto{\pgfqpoint{1.570606in}{1.134049in}}%
\pgfpathlineto{\pgfqpoint{1.573991in}{1.127816in}}%
\pgfpathlineto{\pgfqpoint{1.577375in}{1.121655in}}%
\pgfpathlineto{\pgfqpoint{1.581948in}{1.115288in}}%
\pgfpathlineto{\pgfqpoint{1.586133in}{1.108846in}}%
\pgfpathlineto{\pgfqpoint{1.589923in}{1.102334in}}%
\pgfpathlineto{\pgfqpoint{1.593315in}{1.095759in}}%
\pgfpathlineto{\pgfqpoint{1.589782in}{1.102164in}}%
\pgfpathlineto{\pgfqpoint{1.586249in}{1.108641in}}%
\pgfpathlineto{\pgfqpoint{1.582713in}{1.115186in}}%
\pgfpathlineto{\pgfqpoint{1.579176in}{1.121796in}}%
\pgfpathlineto{\pgfqpoint{1.575914in}{1.128123in}}%
\pgfpathlineto{\pgfqpoint{1.572265in}{1.134390in}}%
\pgfpathlineto{\pgfqpoint{1.568236in}{1.140589in}}%
\pgfpathlineto{\pgfqpoint{1.563830in}{1.146717in}}%
\pgfpathclose%
\pgfusepath{fill}%
\end{pgfscope}%
\begin{pgfscope}%
\pgfpathrectangle{\pgfqpoint{0.041670in}{0.041670in}}{\pgfqpoint{2.216660in}{2.216660in}}%
\pgfusepath{clip}%
\pgfsetbuttcap%
\pgfsetroundjoin%
\definecolor{currentfill}{rgb}{0.195860,0.395433,0.555276}%
\pgfsetfillcolor{currentfill}%
\pgfsetlinewidth{0.000000pt}%
\definecolor{currentstroke}{rgb}{0.000000,0.000000,0.000000}%
\pgfsetstrokecolor{currentstroke}%
\pgfsetdash{}{0pt}%
\pgfpathmoveto{\pgfqpoint{1.516960in}{1.221437in}}%
\pgfpathlineto{\pgfqpoint{1.520168in}{1.214895in}}%
\pgfpathlineto{\pgfqpoint{1.523373in}{1.208393in}}%
\pgfpathlineto{\pgfqpoint{1.526576in}{1.201933in}}%
\pgfpathlineto{\pgfqpoint{1.529777in}{1.195520in}}%
\pgfpathlineto{\pgfqpoint{1.535414in}{1.189963in}}%
\pgfpathlineto{\pgfqpoint{1.540711in}{1.184315in}}%
\pgfpathlineto{\pgfqpoint{1.545660in}{1.178581in}}%
\pgfpathlineto{\pgfqpoint{1.550256in}{1.172768in}}%
\pgfpathlineto{\pgfqpoint{1.546857in}{1.179411in}}%
\pgfpathlineto{\pgfqpoint{1.543456in}{1.186100in}}%
\pgfpathlineto{\pgfqpoint{1.540052in}{1.192831in}}%
\pgfpathlineto{\pgfqpoint{1.536646in}{1.199603in}}%
\pgfpathlineto{\pgfqpoint{1.532230in}{1.205181in}}%
\pgfpathlineto{\pgfqpoint{1.527473in}{1.210683in}}%
\pgfpathlineto{\pgfqpoint{1.522381in}{1.216103in}}%
\pgfpathlineto{\pgfqpoint{1.516960in}{1.221437in}}%
\pgfpathclose%
\pgfusepath{fill}%
\end{pgfscope}%
\begin{pgfscope}%
\pgfpathrectangle{\pgfqpoint{0.041670in}{0.041670in}}{\pgfqpoint{2.216660in}{2.216660in}}%
\pgfusepath{clip}%
\pgfsetbuttcap%
\pgfsetroundjoin%
\definecolor{currentfill}{rgb}{0.120081,0.622161,0.534946}%
\pgfsetfillcolor{currentfill}%
\pgfsetlinewidth{0.000000pt}%
\definecolor{currentstroke}{rgb}{0.000000,0.000000,0.000000}%
\pgfsetstrokecolor{currentstroke}%
\pgfsetdash{}{0pt}%
\pgfpathmoveto{\pgfqpoint{1.278044in}{1.451251in}}%
\pgfpathlineto{\pgfqpoint{1.279274in}{1.445550in}}%
\pgfpathlineto{\pgfqpoint{1.280503in}{1.439823in}}%
\pgfpathlineto{\pgfqpoint{1.281730in}{1.434071in}}%
\pgfpathlineto{\pgfqpoint{1.282955in}{1.428297in}}%
\pgfpathlineto{\pgfqpoint{1.291864in}{1.426662in}}%
\pgfpathlineto{\pgfqpoint{1.300669in}{1.424890in}}%
\pgfpathlineto{\pgfqpoint{1.309361in}{1.422986in}}%
\pgfpathlineto{\pgfqpoint{1.317932in}{1.420949in}}%
\pgfpathlineto{\pgfqpoint{1.316289in}{1.426816in}}%
\pgfpathlineto{\pgfqpoint{1.314643in}{1.432662in}}%
\pgfpathlineto{\pgfqpoint{1.312996in}{1.438483in}}%
\pgfpathlineto{\pgfqpoint{1.311346in}{1.444277in}}%
\pgfpathlineto{\pgfqpoint{1.303187in}{1.446210in}}%
\pgfpathlineto{\pgfqpoint{1.294911in}{1.448018in}}%
\pgfpathlineto{\pgfqpoint{1.286528in}{1.449699in}}%
\pgfpathlineto{\pgfqpoint{1.278044in}{1.451251in}}%
\pgfpathclose%
\pgfusepath{fill}%
\end{pgfscope}%
\begin{pgfscope}%
\pgfpathrectangle{\pgfqpoint{0.041670in}{0.041670in}}{\pgfqpoint{2.216660in}{2.216660in}}%
\pgfusepath{clip}%
\pgfsetbuttcap%
\pgfsetroundjoin%
\definecolor{currentfill}{rgb}{0.133743,0.548535,0.553541}%
\pgfsetfillcolor{currentfill}%
\pgfsetlinewidth{0.000000pt}%
\definecolor{currentstroke}{rgb}{0.000000,0.000000,0.000000}%
\pgfsetstrokecolor{currentstroke}%
\pgfsetdash{}{0pt}%
\pgfpathmoveto{\pgfqpoint{1.390755in}{1.375436in}}%
\pgfpathlineto{\pgfqpoint{1.393144in}{1.369231in}}%
\pgfpathlineto{\pgfqpoint{1.395530in}{1.363021in}}%
\pgfpathlineto{\pgfqpoint{1.397913in}{1.356807in}}%
\pgfpathlineto{\pgfqpoint{1.400294in}{1.350592in}}%
\pgfpathlineto{\pgfqpoint{1.408105in}{1.347138in}}%
\pgfpathlineto{\pgfqpoint{1.415701in}{1.343564in}}%
\pgfpathlineto{\pgfqpoint{1.423075in}{1.339872in}}%
\pgfpathlineto{\pgfqpoint{1.430217in}{1.336066in}}%
\pgfpathlineto{\pgfqpoint{1.427508in}{1.342451in}}%
\pgfpathlineto{\pgfqpoint{1.424795in}{1.348836in}}%
\pgfpathlineto{\pgfqpoint{1.422079in}{1.355216in}}%
\pgfpathlineto{\pgfqpoint{1.419360in}{1.361591in}}%
\pgfpathlineto{\pgfqpoint{1.412533in}{1.365218in}}%
\pgfpathlineto{\pgfqpoint{1.405485in}{1.368737in}}%
\pgfpathlineto{\pgfqpoint{1.398223in}{1.372144in}}%
\pgfpathlineto{\pgfqpoint{1.390755in}{1.375436in}}%
\pgfpathclose%
\pgfusepath{fill}%
\end{pgfscope}%
\begin{pgfscope}%
\pgfpathrectangle{\pgfqpoint{0.041670in}{0.041670in}}{\pgfqpoint{2.216660in}{2.216660in}}%
\pgfusepath{clip}%
\pgfsetbuttcap%
\pgfsetroundjoin%
\definecolor{currentfill}{rgb}{0.274952,0.037752,0.364543}%
\pgfsetfillcolor{currentfill}%
\pgfsetlinewidth{0.000000pt}%
\definecolor{currentstroke}{rgb}{0.000000,0.000000,0.000000}%
\pgfsetstrokecolor{currentstroke}%
\pgfsetdash{}{0pt}%
\pgfpathmoveto{\pgfqpoint{0.651775in}{0.883431in}}%
\pgfpathlineto{\pgfqpoint{0.648075in}{0.880087in}}%
\pgfpathlineto{\pgfqpoint{0.644372in}{0.876926in}}%
\pgfpathlineto{\pgfqpoint{0.640665in}{0.873953in}}%
\pgfpathlineto{\pgfqpoint{0.636954in}{0.871172in}}%
\pgfpathlineto{\pgfqpoint{0.636709in}{0.880125in}}%
\pgfpathlineto{\pgfqpoint{0.637006in}{0.889065in}}%
\pgfpathlineto{\pgfqpoint{0.637843in}{0.897985in}}%
\pgfpathlineto{\pgfqpoint{0.639217in}{0.906876in}}%
\pgfpathlineto{\pgfqpoint{0.642895in}{0.909401in}}%
\pgfpathlineto{\pgfqpoint{0.646568in}{0.912117in}}%
\pgfpathlineto{\pgfqpoint{0.650239in}{0.915020in}}%
\pgfpathlineto{\pgfqpoint{0.653905in}{0.918106in}}%
\pgfpathlineto{\pgfqpoint{0.652586in}{0.909471in}}%
\pgfpathlineto{\pgfqpoint{0.651790in}{0.900807in}}%
\pgfpathlineto{\pgfqpoint{0.651518in}{0.892124in}}%
\pgfpathlineto{\pgfqpoint{0.651775in}{0.883431in}}%
\pgfpathclose%
\pgfusepath{fill}%
\end{pgfscope}%
\begin{pgfscope}%
\pgfpathrectangle{\pgfqpoint{0.041670in}{0.041670in}}{\pgfqpoint{2.216660in}{2.216660in}}%
\pgfusepath{clip}%
\pgfsetbuttcap%
\pgfsetroundjoin%
\definecolor{currentfill}{rgb}{0.282327,0.094955,0.417331}%
\pgfsetfillcolor{currentfill}%
\pgfsetlinewidth{0.000000pt}%
\definecolor{currentstroke}{rgb}{0.000000,0.000000,0.000000}%
\pgfsetstrokecolor{currentstroke}%
\pgfsetdash{}{0pt}%
\pgfpathmoveto{\pgfqpoint{1.824132in}{0.941853in}}%
\pgfpathlineto{\pgfqpoint{1.828018in}{0.946970in}}%
\pgfpathlineto{\pgfqpoint{1.831917in}{0.952427in}}%
\pgfpathlineto{\pgfqpoint{1.835829in}{0.958229in}}%
\pgfpathlineto{\pgfqpoint{1.839754in}{0.964383in}}%
\pgfpathlineto{\pgfqpoint{1.842200in}{0.953613in}}%
\pgfpathlineto{\pgfqpoint{1.843994in}{0.942793in}}%
\pgfpathlineto{\pgfqpoint{1.845131in}{0.931932in}}%
\pgfpathlineto{\pgfqpoint{1.845607in}{0.921041in}}%
\pgfpathlineto{\pgfqpoint{1.841623in}{0.915110in}}%
\pgfpathlineto{\pgfqpoint{1.837653in}{0.909531in}}%
\pgfpathlineto{\pgfqpoint{1.833696in}{0.904300in}}%
\pgfpathlineto{\pgfqpoint{1.829752in}{0.899410in}}%
\pgfpathlineto{\pgfqpoint{1.829311in}{0.910073in}}%
\pgfpathlineto{\pgfqpoint{1.828224in}{0.920709in}}%
\pgfpathlineto{\pgfqpoint{1.826496in}{0.931305in}}%
\pgfpathlineto{\pgfqpoint{1.824132in}{0.941853in}}%
\pgfpathclose%
\pgfusepath{fill}%
\end{pgfscope}%
\begin{pgfscope}%
\pgfpathrectangle{\pgfqpoint{0.041670in}{0.041670in}}{\pgfqpoint{2.216660in}{2.216660in}}%
\pgfusepath{clip}%
\pgfsetbuttcap%
\pgfsetroundjoin%
\definecolor{currentfill}{rgb}{0.147607,0.511733,0.557049}%
\pgfsetfillcolor{currentfill}%
\pgfsetlinewidth{0.000000pt}%
\definecolor{currentstroke}{rgb}{0.000000,0.000000,0.000000}%
\pgfsetstrokecolor{currentstroke}%
\pgfsetdash{}{0pt}%
\pgfpathmoveto{\pgfqpoint{1.430217in}{1.336066in}}%
\pgfpathlineto{\pgfqpoint{1.432924in}{1.329683in}}%
\pgfpathlineto{\pgfqpoint{1.435628in}{1.323305in}}%
\pgfpathlineto{\pgfqpoint{1.438329in}{1.316934in}}%
\pgfpathlineto{\pgfqpoint{1.441027in}{1.310574in}}%
\pgfpathlineto{\pgfqpoint{1.448237in}{1.306473in}}%
\pgfpathlineto{\pgfqpoint{1.455192in}{1.302259in}}%
\pgfpathlineto{\pgfqpoint{1.461887in}{1.297937in}}%
\pgfpathlineto{\pgfqpoint{1.468312in}{1.293511in}}%
\pgfpathlineto{\pgfqpoint{1.465325in}{1.300064in}}%
\pgfpathlineto{\pgfqpoint{1.462334in}{1.306627in}}%
\pgfpathlineto{\pgfqpoint{1.459341in}{1.313197in}}%
\pgfpathlineto{\pgfqpoint{1.456345in}{1.319773in}}%
\pgfpathlineto{\pgfqpoint{1.450193in}{1.323999in}}%
\pgfpathlineto{\pgfqpoint{1.443783in}{1.328126in}}%
\pgfpathlineto{\pgfqpoint{1.437123in}{1.332150in}}%
\pgfpathlineto{\pgfqpoint{1.430217in}{1.336066in}}%
\pgfpathclose%
\pgfusepath{fill}%
\end{pgfscope}%
\begin{pgfscope}%
\pgfpathrectangle{\pgfqpoint{0.041670in}{0.041670in}}{\pgfqpoint{2.216660in}{2.216660in}}%
\pgfusepath{clip}%
\pgfsetbuttcap%
\pgfsetroundjoin%
\definecolor{currentfill}{rgb}{0.233603,0.313828,0.543914}%
\pgfsetfillcolor{currentfill}%
\pgfsetlinewidth{0.000000pt}%
\definecolor{currentstroke}{rgb}{0.000000,0.000000,0.000000}%
\pgfsetstrokecolor{currentstroke}%
\pgfsetdash{}{0pt}%
\pgfpathmoveto{\pgfqpoint{1.870357in}{1.112710in}}%
\pgfpathlineto{\pgfqpoint{1.874362in}{1.124201in}}%
\pgfpathlineto{\pgfqpoint{1.878385in}{1.136133in}}%
\pgfpathlineto{\pgfqpoint{1.882428in}{1.148514in}}%
\pgfpathlineto{\pgfqpoint{1.886489in}{1.161352in}}%
\pgfpathlineto{\pgfqpoint{1.892075in}{1.150113in}}%
\pgfpathlineto{\pgfqpoint{1.896974in}{1.138774in}}%
\pgfpathlineto{\pgfqpoint{1.901179in}{1.127343in}}%
\pgfpathlineto{\pgfqpoint{1.904683in}{1.115832in}}%
\pgfpathlineto{\pgfqpoint{1.900490in}{1.103175in}}%
\pgfpathlineto{\pgfqpoint{1.896316in}{1.090976in}}%
\pgfpathlineto{\pgfqpoint{1.892163in}{1.079228in}}%
\pgfpathlineto{\pgfqpoint{1.888029in}{1.067925in}}%
\pgfpathlineto{\pgfqpoint{1.884633in}{1.079249in}}%
\pgfpathlineto{\pgfqpoint{1.880551in}{1.090495in}}%
\pgfpathlineto{\pgfqpoint{1.875789in}{1.101652in}}%
\pgfpathlineto{\pgfqpoint{1.870357in}{1.112710in}}%
\pgfpathclose%
\pgfusepath{fill}%
\end{pgfscope}%
\begin{pgfscope}%
\pgfpathrectangle{\pgfqpoint{0.041670in}{0.041670in}}{\pgfqpoint{2.216660in}{2.216660in}}%
\pgfusepath{clip}%
\pgfsetbuttcap%
\pgfsetroundjoin%
\definecolor{currentfill}{rgb}{0.120081,0.622161,0.534946}%
\pgfsetfillcolor{currentfill}%
\pgfsetlinewidth{0.000000pt}%
\definecolor{currentstroke}{rgb}{0.000000,0.000000,0.000000}%
\pgfsetstrokecolor{currentstroke}%
\pgfsetdash{}{0pt}%
\pgfpathmoveto{\pgfqpoint{1.041414in}{1.442455in}}%
\pgfpathlineto{\pgfqpoint{1.039674in}{1.436636in}}%
\pgfpathlineto{\pgfqpoint{1.037937in}{1.430791in}}%
\pgfpathlineto{\pgfqpoint{1.036201in}{1.424921in}}%
\pgfpathlineto{\pgfqpoint{1.034468in}{1.419029in}}%
\pgfpathlineto{\pgfqpoint{1.042924in}{1.421181in}}%
\pgfpathlineto{\pgfqpoint{1.051508in}{1.423204in}}%
\pgfpathlineto{\pgfqpoint{1.060213in}{1.425094in}}%
\pgfpathlineto{\pgfqpoint{1.069030in}{1.426850in}}%
\pgfpathlineto{\pgfqpoint{1.070350in}{1.432642in}}%
\pgfpathlineto{\pgfqpoint{1.071672in}{1.438412in}}%
\pgfpathlineto{\pgfqpoint{1.072995in}{1.444158in}}%
\pgfpathlineto{\pgfqpoint{1.074320in}{1.449877in}}%
\pgfpathlineto{\pgfqpoint{1.065925in}{1.448211in}}%
\pgfpathlineto{\pgfqpoint{1.057637in}{1.446417in}}%
\pgfpathlineto{\pgfqpoint{1.049464in}{1.444498in}}%
\pgfpathlineto{\pgfqpoint{1.041414in}{1.442455in}}%
\pgfpathclose%
\pgfusepath{fill}%
\end{pgfscope}%
\begin{pgfscope}%
\pgfpathrectangle{\pgfqpoint{0.041670in}{0.041670in}}{\pgfqpoint{2.216660in}{2.216660in}}%
\pgfusepath{clip}%
\pgfsetbuttcap%
\pgfsetroundjoin%
\definecolor{currentfill}{rgb}{0.274128,0.199721,0.498911}%
\pgfsetfillcolor{currentfill}%
\pgfsetlinewidth{0.000000pt}%
\definecolor{currentstroke}{rgb}{0.000000,0.000000,0.000000}%
\pgfsetstrokecolor{currentstroke}%
\pgfsetdash{}{0pt}%
\pgfpathmoveto{\pgfqpoint{0.726796in}{1.012150in}}%
\pgfpathlineto{\pgfqpoint{0.723161in}{1.006229in}}%
\pgfpathlineto{\pgfqpoint{0.719525in}{1.000415in}}%
\pgfpathlineto{\pgfqpoint{0.715890in}{0.994710in}}%
\pgfpathlineto{\pgfqpoint{0.712254in}{0.989118in}}%
\pgfpathlineto{\pgfqpoint{0.713827in}{0.996689in}}%
\pgfpathlineto{\pgfqpoint{0.715859in}{1.004223in}}%
\pgfpathlineto{\pgfqpoint{0.718347in}{1.011712in}}%
\pgfpathlineto{\pgfqpoint{0.721287in}{1.019150in}}%
\pgfpathlineto{\pgfqpoint{0.724838in}{1.024485in}}%
\pgfpathlineto{\pgfqpoint{0.728388in}{1.029934in}}%
\pgfpathlineto{\pgfqpoint{0.731939in}{1.035493in}}%
\pgfpathlineto{\pgfqpoint{0.735490in}{1.041157in}}%
\pgfpathlineto{\pgfqpoint{0.732655in}{1.033973in}}%
\pgfpathlineto{\pgfqpoint{0.730259in}{1.026739in}}%
\pgfpathlineto{\pgfqpoint{0.728305in}{1.019462in}}%
\pgfpathlineto{\pgfqpoint{0.726796in}{1.012150in}}%
\pgfpathclose%
\pgfusepath{fill}%
\end{pgfscope}%
\begin{pgfscope}%
\pgfpathrectangle{\pgfqpoint{0.041670in}{0.041670in}}{\pgfqpoint{2.216660in}{2.216660in}}%
\pgfusepath{clip}%
\pgfsetbuttcap%
\pgfsetroundjoin%
\definecolor{currentfill}{rgb}{0.122606,0.585371,0.546557}%
\pgfsetfillcolor{currentfill}%
\pgfsetlinewidth{0.000000pt}%
\definecolor{currentstroke}{rgb}{0.000000,0.000000,0.000000}%
\pgfsetstrokecolor{currentstroke}%
\pgfsetdash{}{0pt}%
\pgfpathmoveto{\pgfqpoint{1.350838in}{1.411518in}}%
\pgfpathlineto{\pgfqpoint{1.352873in}{1.405511in}}%
\pgfpathlineto{\pgfqpoint{1.354906in}{1.399487in}}%
\pgfpathlineto{\pgfqpoint{1.356936in}{1.393449in}}%
\pgfpathlineto{\pgfqpoint{1.358964in}{1.387400in}}%
\pgfpathlineto{\pgfqpoint{1.367184in}{1.384595in}}%
\pgfpathlineto{\pgfqpoint{1.375227in}{1.381664in}}%
\pgfpathlineto{\pgfqpoint{1.383087in}{1.378610in}}%
\pgfpathlineto{\pgfqpoint{1.390755in}{1.375436in}}%
\pgfpathlineto{\pgfqpoint{1.388363in}{1.381631in}}%
\pgfpathlineto{\pgfqpoint{1.385968in}{1.387815in}}%
\pgfpathlineto{\pgfqpoint{1.383570in}{1.393985in}}%
\pgfpathlineto{\pgfqpoint{1.381170in}{1.400139in}}%
\pgfpathlineto{\pgfqpoint{1.373855in}{1.403158in}}%
\pgfpathlineto{\pgfqpoint{1.366356in}{1.406063in}}%
\pgfpathlineto{\pgfqpoint{1.358681in}{1.408850in}}%
\pgfpathlineto{\pgfqpoint{1.350838in}{1.411518in}}%
\pgfpathclose%
\pgfusepath{fill}%
\end{pgfscope}%
\begin{pgfscope}%
\pgfpathrectangle{\pgfqpoint{0.041670in}{0.041670in}}{\pgfqpoint{2.216660in}{2.216660in}}%
\pgfusepath{clip}%
\pgfsetbuttcap%
\pgfsetroundjoin%
\definecolor{currentfill}{rgb}{0.282327,0.094955,0.417331}%
\pgfsetfillcolor{currentfill}%
\pgfsetlinewidth{0.000000pt}%
\definecolor{currentstroke}{rgb}{0.000000,0.000000,0.000000}%
\pgfsetstrokecolor{currentstroke}%
\pgfsetdash{}{0pt}%
\pgfpathmoveto{\pgfqpoint{1.675283in}{0.956094in}}%
\pgfpathlineto{\pgfqpoint{1.678914in}{0.951745in}}%
\pgfpathlineto{\pgfqpoint{1.682548in}{0.947548in}}%
\pgfpathlineto{\pgfqpoint{1.686183in}{0.943505in}}%
\pgfpathlineto{\pgfqpoint{1.689820in}{0.939621in}}%
\pgfpathlineto{\pgfqpoint{1.691535in}{0.931273in}}%
\pgfpathlineto{\pgfqpoint{1.692743in}{0.922891in}}%
\pgfpathlineto{\pgfqpoint{1.693444in}{0.914483in}}%
\pgfpathlineto{\pgfqpoint{1.693633in}{0.906058in}}%
\pgfpathlineto{\pgfqpoint{1.689952in}{0.910201in}}%
\pgfpathlineto{\pgfqpoint{1.686273in}{0.914503in}}%
\pgfpathlineto{\pgfqpoint{1.682596in}{0.918959in}}%
\pgfpathlineto{\pgfqpoint{1.678921in}{0.923567in}}%
\pgfpathlineto{\pgfqpoint{1.678754in}{0.931731in}}%
\pgfpathlineto{\pgfqpoint{1.678090in}{0.939879in}}%
\pgfpathlineto{\pgfqpoint{1.676932in}{0.948002in}}%
\pgfpathlineto{\pgfqpoint{1.675283in}{0.956094in}}%
\pgfpathclose%
\pgfusepath{fill}%
\end{pgfscope}%
\begin{pgfscope}%
\pgfpathrectangle{\pgfqpoint{0.041670in}{0.041670in}}{\pgfqpoint{2.216660in}{2.216660in}}%
\pgfusepath{clip}%
\pgfsetbuttcap%
\pgfsetroundjoin%
\definecolor{currentfill}{rgb}{0.133743,0.548535,0.553541}%
\pgfsetfillcolor{currentfill}%
\pgfsetlinewidth{0.000000pt}%
\definecolor{currentstroke}{rgb}{0.000000,0.000000,0.000000}%
\pgfsetstrokecolor{currentstroke}%
\pgfsetdash{}{0pt}%
\pgfpathmoveto{\pgfqpoint{0.934672in}{1.358278in}}%
\pgfpathlineto{\pgfqpoint{0.931885in}{1.351863in}}%
\pgfpathlineto{\pgfqpoint{0.929101in}{1.345441in}}%
\pgfpathlineto{\pgfqpoint{0.926320in}{1.339016in}}%
\pgfpathlineto{\pgfqpoint{0.923542in}{1.332590in}}%
\pgfpathlineto{\pgfqpoint{0.930474in}{1.336495in}}%
\pgfpathlineto{\pgfqpoint{0.937643in}{1.340288in}}%
\pgfpathlineto{\pgfqpoint{0.945041in}{1.343967in}}%
\pgfpathlineto{\pgfqpoint{0.952662in}{1.347528in}}%
\pgfpathlineto{\pgfqpoint{0.955119in}{1.353779in}}%
\pgfpathlineto{\pgfqpoint{0.957578in}{1.360028in}}%
\pgfpathlineto{\pgfqpoint{0.960041in}{1.366275in}}%
\pgfpathlineto{\pgfqpoint{0.962506in}{1.372515in}}%
\pgfpathlineto{\pgfqpoint{0.955221in}{1.369121in}}%
\pgfpathlineto{\pgfqpoint{0.948149in}{1.365614in}}%
\pgfpathlineto{\pgfqpoint{0.941297in}{1.361999in}}%
\pgfpathlineto{\pgfqpoint{0.934672in}{1.358278in}}%
\pgfpathclose%
\pgfusepath{fill}%
\end{pgfscope}%
\begin{pgfscope}%
\pgfpathrectangle{\pgfqpoint{0.041670in}{0.041670in}}{\pgfqpoint{2.216660in}{2.216660in}}%
\pgfusepath{clip}%
\pgfsetbuttcap%
\pgfsetroundjoin%
\definecolor{currentfill}{rgb}{0.277941,0.056324,0.381191}%
\pgfsetfillcolor{currentfill}%
\pgfsetlinewidth{0.000000pt}%
\definecolor{currentstroke}{rgb}{0.000000,0.000000,0.000000}%
\pgfsetstrokecolor{currentstroke}%
\pgfsetdash{}{0pt}%
\pgfpathmoveto{\pgfqpoint{0.545981in}{0.873860in}}%
\pgfpathlineto{\pgfqpoint{0.542081in}{0.877390in}}%
\pgfpathlineto{\pgfqpoint{0.538169in}{0.881239in}}%
\pgfpathlineto{\pgfqpoint{0.534246in}{0.885413in}}%
\pgfpathlineto{\pgfqpoint{0.530310in}{0.889916in}}%
\pgfpathlineto{\pgfqpoint{0.530175in}{0.900596in}}%
\pgfpathlineto{\pgfqpoint{0.530688in}{0.911257in}}%
\pgfpathlineto{\pgfqpoint{0.531846in}{0.921888in}}%
\pgfpathlineto{\pgfqpoint{0.533645in}{0.932480in}}%
\pgfpathlineto{\pgfqpoint{0.537537in}{0.927747in}}%
\pgfpathlineto{\pgfqpoint{0.541418in}{0.923342in}}%
\pgfpathlineto{\pgfqpoint{0.545287in}{0.919261in}}%
\pgfpathlineto{\pgfqpoint{0.549145in}{0.915496in}}%
\pgfpathlineto{\pgfqpoint{0.547412in}{0.905134in}}%
\pgfpathlineto{\pgfqpoint{0.546304in}{0.894734in}}%
\pgfpathlineto{\pgfqpoint{0.545826in}{0.884306in}}%
\pgfpathlineto{\pgfqpoint{0.545981in}{0.873860in}}%
\pgfpathclose%
\pgfusepath{fill}%
\end{pgfscope}%
\begin{pgfscope}%
\pgfpathrectangle{\pgfqpoint{0.041670in}{0.041670in}}{\pgfqpoint{2.216660in}{2.216660in}}%
\pgfusepath{clip}%
\pgfsetbuttcap%
\pgfsetroundjoin%
\definecolor{currentfill}{rgb}{0.195860,0.395433,0.555276}%
\pgfsetfillcolor{currentfill}%
\pgfsetlinewidth{0.000000pt}%
\definecolor{currentstroke}{rgb}{0.000000,0.000000,0.000000}%
\pgfsetstrokecolor{currentstroke}%
\pgfsetdash{}{0pt}%
\pgfpathmoveto{\pgfqpoint{0.819628in}{1.194584in}}%
\pgfpathlineto{\pgfqpoint{0.816184in}{1.187760in}}%
\pgfpathlineto{\pgfqpoint{0.812744in}{1.180975in}}%
\pgfpathlineto{\pgfqpoint{0.809305in}{1.174233in}}%
\pgfpathlineto{\pgfqpoint{0.805869in}{1.167537in}}%
\pgfpathlineto{\pgfqpoint{0.810147in}{1.173417in}}%
\pgfpathlineto{\pgfqpoint{0.814783in}{1.179222in}}%
\pgfpathlineto{\pgfqpoint{0.819771in}{1.184947in}}%
\pgfpathlineto{\pgfqpoint{0.825105in}{1.190585in}}%
\pgfpathlineto{\pgfqpoint{0.828354in}{1.197048in}}%
\pgfpathlineto{\pgfqpoint{0.831605in}{1.203557in}}%
\pgfpathlineto{\pgfqpoint{0.834859in}{1.210109in}}%
\pgfpathlineto{\pgfqpoint{0.838115in}{1.216700in}}%
\pgfpathlineto{\pgfqpoint{0.832986in}{1.211289in}}%
\pgfpathlineto{\pgfqpoint{0.828192in}{1.205796in}}%
\pgfpathlineto{\pgfqpoint{0.823737in}{1.200226in}}%
\pgfpathlineto{\pgfqpoint{0.819628in}{1.194584in}}%
\pgfpathclose%
\pgfusepath{fill}%
\end{pgfscope}%
\begin{pgfscope}%
\pgfpathrectangle{\pgfqpoint{0.041670in}{0.041670in}}{\pgfqpoint{2.216660in}{2.216660in}}%
\pgfusepath{clip}%
\pgfsetbuttcap%
\pgfsetroundjoin%
\definecolor{currentfill}{rgb}{0.134692,0.658636,0.517649}%
\pgfsetfillcolor{currentfill}%
\pgfsetlinewidth{0.000000pt}%
\definecolor{currentstroke}{rgb}{0.000000,0.000000,0.000000}%
\pgfsetstrokecolor{currentstroke}%
\pgfsetdash{}{0pt}%
\pgfpathmoveto{\pgfqpoint{1.146158in}{1.480584in}}%
\pgfpathlineto{\pgfqpoint{1.145710in}{1.475112in}}%
\pgfpathlineto{\pgfqpoint{1.145262in}{1.469603in}}%
\pgfpathlineto{\pgfqpoint{1.144815in}{1.464060in}}%
\pgfpathlineto{\pgfqpoint{1.144368in}{1.458485in}}%
\pgfpathlineto{\pgfqpoint{1.153362in}{1.458956in}}%
\pgfpathlineto{\pgfqpoint{1.162380in}{1.459290in}}%
\pgfpathlineto{\pgfqpoint{1.171415in}{1.459488in}}%
\pgfpathlineto{\pgfqpoint{1.180457in}{1.459549in}}%
\pgfpathlineto{\pgfqpoint{1.180451in}{1.465110in}}%
\pgfpathlineto{\pgfqpoint{1.180445in}{1.470639in}}%
\pgfpathlineto{\pgfqpoint{1.180438in}{1.476133in}}%
\pgfpathlineto{\pgfqpoint{1.180432in}{1.481591in}}%
\pgfpathlineto{\pgfqpoint{1.171844in}{1.481534in}}%
\pgfpathlineto{\pgfqpoint{1.163264in}{1.481346in}}%
\pgfpathlineto{\pgfqpoint{1.154699in}{1.481030in}}%
\pgfpathlineto{\pgfqpoint{1.146158in}{1.480584in}}%
\pgfpathclose%
\pgfusepath{fill}%
\end{pgfscope}%
\begin{pgfscope}%
\pgfpathrectangle{\pgfqpoint{0.041670in}{0.041670in}}{\pgfqpoint{2.216660in}{2.216660in}}%
\pgfusepath{clip}%
\pgfsetbuttcap%
\pgfsetroundjoin%
\definecolor{currentfill}{rgb}{0.134692,0.658636,0.517649}%
\pgfsetfillcolor{currentfill}%
\pgfsetlinewidth{0.000000pt}%
\definecolor{currentstroke}{rgb}{0.000000,0.000000,0.000000}%
\pgfsetstrokecolor{currentstroke}%
\pgfsetdash{}{0pt}%
\pgfpathmoveto{\pgfqpoint{1.180432in}{1.481591in}}%
\pgfpathlineto{\pgfqpoint{1.180438in}{1.476133in}}%
\pgfpathlineto{\pgfqpoint{1.180445in}{1.470639in}}%
\pgfpathlineto{\pgfqpoint{1.180451in}{1.465110in}}%
\pgfpathlineto{\pgfqpoint{1.180457in}{1.459549in}}%
\pgfpathlineto{\pgfqpoint{1.189499in}{1.459473in}}%
\pgfpathlineto{\pgfqpoint{1.198533in}{1.459260in}}%
\pgfpathlineto{\pgfqpoint{1.207549in}{1.458910in}}%
\pgfpathlineto{\pgfqpoint{1.216539in}{1.458424in}}%
\pgfpathlineto{\pgfqpoint{1.216080in}{1.464000in}}%
\pgfpathlineto{\pgfqpoint{1.215620in}{1.469543in}}%
\pgfpathlineto{\pgfqpoint{1.215160in}{1.475053in}}%
\pgfpathlineto{\pgfqpoint{1.214699in}{1.480526in}}%
\pgfpathlineto{\pgfqpoint{1.206161in}{1.480987in}}%
\pgfpathlineto{\pgfqpoint{1.197599in}{1.481318in}}%
\pgfpathlineto{\pgfqpoint{1.189020in}{1.481519in}}%
\pgfpathlineto{\pgfqpoint{1.180432in}{1.481591in}}%
\pgfpathclose%
\pgfusepath{fill}%
\end{pgfscope}%
\begin{pgfscope}%
\pgfpathrectangle{\pgfqpoint{0.041670in}{0.041670in}}{\pgfqpoint{2.216660in}{2.216660in}}%
\pgfusepath{clip}%
\pgfsetbuttcap%
\pgfsetroundjoin%
\definecolor{currentfill}{rgb}{0.231674,0.318106,0.544834}%
\pgfsetfillcolor{currentfill}%
\pgfsetlinewidth{0.000000pt}%
\definecolor{currentstroke}{rgb}{0.000000,0.000000,0.000000}%
\pgfsetstrokecolor{currentstroke}%
\pgfsetdash{}{0pt}%
\pgfpathmoveto{\pgfqpoint{0.778160in}{1.116126in}}%
\pgfpathlineto{\pgfqpoint{0.774598in}{1.109460in}}%
\pgfpathlineto{\pgfqpoint{0.771037in}{1.102859in}}%
\pgfpathlineto{\pgfqpoint{0.767477in}{1.096327in}}%
\pgfpathlineto{\pgfqpoint{0.763919in}{1.089866in}}%
\pgfpathlineto{\pgfqpoint{0.766952in}{1.096492in}}%
\pgfpathlineto{\pgfqpoint{0.770388in}{1.103061in}}%
\pgfpathlineto{\pgfqpoint{0.774222in}{1.109565in}}%
\pgfpathlineto{\pgfqpoint{0.778450in}{1.116000in}}%
\pgfpathlineto{\pgfqpoint{0.781872in}{1.122214in}}%
\pgfpathlineto{\pgfqpoint{0.785294in}{1.128500in}}%
\pgfpathlineto{\pgfqpoint{0.788719in}{1.134855in}}%
\pgfpathlineto{\pgfqpoint{0.792145in}{1.141274in}}%
\pgfpathlineto{\pgfqpoint{0.788073in}{1.135082in}}%
\pgfpathlineto{\pgfqpoint{0.784382in}{1.128822in}}%
\pgfpathlineto{\pgfqpoint{0.781077in}{1.122502in}}%
\pgfpathlineto{\pgfqpoint{0.778160in}{1.116126in}}%
\pgfpathclose%
\pgfusepath{fill}%
\end{pgfscope}%
\begin{pgfscope}%
\pgfpathrectangle{\pgfqpoint{0.041670in}{0.041670in}}{\pgfqpoint{2.216660in}{2.216660in}}%
\pgfusepath{clip}%
\pgfsetbuttcap%
\pgfsetroundjoin%
\definecolor{currentfill}{rgb}{0.263663,0.237631,0.518762}%
\pgfsetfillcolor{currentfill}%
\pgfsetlinewidth{0.000000pt}%
\definecolor{currentstroke}{rgb}{0.000000,0.000000,0.000000}%
\pgfsetstrokecolor{currentstroke}%
\pgfsetdash{}{0pt}%
\pgfpathmoveto{\pgfqpoint{1.607432in}{1.070920in}}%
\pgfpathlineto{\pgfqpoint{1.610959in}{1.064923in}}%
\pgfpathlineto{\pgfqpoint{1.614485in}{1.059017in}}%
\pgfpathlineto{\pgfqpoint{1.618011in}{1.053207in}}%
\pgfpathlineto{\pgfqpoint{1.621536in}{1.047496in}}%
\pgfpathlineto{\pgfqpoint{1.624756in}{1.040362in}}%
\pgfpathlineto{\pgfqpoint{1.627542in}{1.033171in}}%
\pgfpathlineto{\pgfqpoint{1.629890in}{1.025932in}}%
\pgfpathlineto{\pgfqpoint{1.631795in}{1.018651in}}%
\pgfpathlineto{\pgfqpoint{1.628173in}{1.024617in}}%
\pgfpathlineto{\pgfqpoint{1.624551in}{1.030681in}}%
\pgfpathlineto{\pgfqpoint{1.620929in}{1.036841in}}%
\pgfpathlineto{\pgfqpoint{1.617306in}{1.043092in}}%
\pgfpathlineto{\pgfqpoint{1.615477in}{1.050116in}}%
\pgfpathlineto{\pgfqpoint{1.613219in}{1.057099in}}%
\pgfpathlineto{\pgfqpoint{1.610536in}{1.064036in}}%
\pgfpathlineto{\pgfqpoint{1.607432in}{1.070920in}}%
\pgfpathclose%
\pgfusepath{fill}%
\end{pgfscope}%
\begin{pgfscope}%
\pgfpathrectangle{\pgfqpoint{0.041670in}{0.041670in}}{\pgfqpoint{2.216660in}{2.216660in}}%
\pgfusepath{clip}%
\pgfsetbuttcap%
\pgfsetroundjoin%
\definecolor{currentfill}{rgb}{0.147607,0.511733,0.557049}%
\pgfsetfillcolor{currentfill}%
\pgfsetlinewidth{0.000000pt}%
\definecolor{currentstroke}{rgb}{0.000000,0.000000,0.000000}%
\pgfsetstrokecolor{currentstroke}%
\pgfsetdash{}{0pt}%
\pgfpathmoveto{\pgfqpoint{0.898319in}{1.315935in}}%
\pgfpathlineto{\pgfqpoint{0.895264in}{1.309315in}}%
\pgfpathlineto{\pgfqpoint{0.892212in}{1.302698in}}%
\pgfpathlineto{\pgfqpoint{0.889163in}{1.296090in}}%
\pgfpathlineto{\pgfqpoint{0.886117in}{1.289491in}}%
\pgfpathlineto{\pgfqpoint{0.892298in}{1.294008in}}%
\pgfpathlineto{\pgfqpoint{0.898754in}{1.298423in}}%
\pgfpathlineto{\pgfqpoint{0.905477in}{1.302733in}}%
\pgfpathlineto{\pgfqpoint{0.912462in}{1.306934in}}%
\pgfpathlineto{\pgfqpoint{0.915227in}{1.313335in}}%
\pgfpathlineto{\pgfqpoint{0.917996in}{1.319747in}}%
\pgfpathlineto{\pgfqpoint{0.920768in}{1.326166in}}%
\pgfpathlineto{\pgfqpoint{0.923542in}{1.332590in}}%
\pgfpathlineto{\pgfqpoint{0.916854in}{1.328578in}}%
\pgfpathlineto{\pgfqpoint{0.910416in}{1.324463in}}%
\pgfpathlineto{\pgfqpoint{0.904236in}{1.320247in}}%
\pgfpathlineto{\pgfqpoint{0.898319in}{1.315935in}}%
\pgfpathclose%
\pgfusepath{fill}%
\end{pgfscope}%
\begin{pgfscope}%
\pgfpathrectangle{\pgfqpoint{0.041670in}{0.041670in}}{\pgfqpoint{2.216660in}{2.216660in}}%
\pgfusepath{clip}%
\pgfsetbuttcap%
\pgfsetroundjoin%
\definecolor{currentfill}{rgb}{0.279566,0.067836,0.391917}%
\pgfsetfillcolor{currentfill}%
\pgfsetlinewidth{0.000000pt}%
\definecolor{currentstroke}{rgb}{0.000000,0.000000,0.000000}%
\pgfsetstrokecolor{currentstroke}%
\pgfsetdash{}{0pt}%
\pgfpathmoveto{\pgfqpoint{0.666539in}{0.898562in}}%
\pgfpathlineto{\pgfqpoint{0.662852in}{0.894525in}}%
\pgfpathlineto{\pgfqpoint{0.659163in}{0.890654in}}%
\pgfpathlineto{\pgfqpoint{0.655471in}{0.886955in}}%
\pgfpathlineto{\pgfqpoint{0.651775in}{0.883431in}}%
\pgfpathlineto{\pgfqpoint{0.651518in}{0.892124in}}%
\pgfpathlineto{\pgfqpoint{0.651790in}{0.900807in}}%
\pgfpathlineto{\pgfqpoint{0.652586in}{0.909471in}}%
\pgfpathlineto{\pgfqpoint{0.653905in}{0.918106in}}%
\pgfpathlineto{\pgfqpoint{0.657568in}{0.921372in}}%
\pgfpathlineto{\pgfqpoint{0.661229in}{0.924813in}}%
\pgfpathlineto{\pgfqpoint{0.664886in}{0.928424in}}%
\pgfpathlineto{\pgfqpoint{0.668541in}{0.932202in}}%
\pgfpathlineto{\pgfqpoint{0.667276in}{0.923823in}}%
\pgfpathlineto{\pgfqpoint{0.666518in}{0.915418in}}%
\pgfpathlineto{\pgfqpoint{0.666272in}{0.906995in}}%
\pgfpathlineto{\pgfqpoint{0.666539in}{0.898562in}}%
\pgfpathclose%
\pgfusepath{fill}%
\end{pgfscope}%
\begin{pgfscope}%
\pgfpathrectangle{\pgfqpoint{0.041670in}{0.041670in}}{\pgfqpoint{2.216660in}{2.216660in}}%
\pgfusepath{clip}%
\pgfsetbuttcap%
\pgfsetroundjoin%
\definecolor{currentfill}{rgb}{0.163625,0.471133,0.558148}%
\pgfsetfillcolor{currentfill}%
\pgfsetlinewidth{0.000000pt}%
\definecolor{currentstroke}{rgb}{0.000000,0.000000,0.000000}%
\pgfsetstrokecolor{currentstroke}%
\pgfsetdash{}{0pt}%
\pgfpathmoveto{\pgfqpoint{1.468312in}{1.293511in}}%
\pgfpathlineto{\pgfqpoint{1.471296in}{1.286971in}}%
\pgfpathlineto{\pgfqpoint{1.474278in}{1.280447in}}%
\pgfpathlineto{\pgfqpoint{1.477256in}{1.273943in}}%
\pgfpathlineto{\pgfqpoint{1.480232in}{1.267460in}}%
\pgfpathlineto{\pgfqpoint{1.486645in}{1.262727in}}%
\pgfpathlineto{\pgfqpoint{1.492766in}{1.257894in}}%
\pgfpathlineto{\pgfqpoint{1.498587in}{1.252963in}}%
\pgfpathlineto{\pgfqpoint{1.504103in}{1.247941in}}%
\pgfpathlineto{\pgfqpoint{1.500881in}{1.254636in}}%
\pgfpathlineto{\pgfqpoint{1.497657in}{1.261353in}}%
\pgfpathlineto{\pgfqpoint{1.494430in}{1.268089in}}%
\pgfpathlineto{\pgfqpoint{1.491200in}{1.274840in}}%
\pgfpathlineto{\pgfqpoint{1.485912in}{1.279644in}}%
\pgfpathlineto{\pgfqpoint{1.480331in}{1.284360in}}%
\pgfpathlineto{\pgfqpoint{1.474463in}{1.288984in}}%
\pgfpathlineto{\pgfqpoint{1.468312in}{1.293511in}}%
\pgfpathclose%
\pgfusepath{fill}%
\end{pgfscope}%
\begin{pgfscope}%
\pgfpathrectangle{\pgfqpoint{0.041670in}{0.041670in}}{\pgfqpoint{2.216660in}{2.216660in}}%
\pgfusepath{clip}%
\pgfsetbuttcap%
\pgfsetroundjoin%
\definecolor{currentfill}{rgb}{0.122606,0.585371,0.546557}%
\pgfsetfillcolor{currentfill}%
\pgfsetlinewidth{0.000000pt}%
\definecolor{currentstroke}{rgb}{0.000000,0.000000,0.000000}%
\pgfsetstrokecolor{currentstroke}%
\pgfsetdash{}{0pt}%
\pgfpathmoveto{\pgfqpoint{0.972398in}{1.397361in}}%
\pgfpathlineto{\pgfqpoint{0.969920in}{1.391172in}}%
\pgfpathlineto{\pgfqpoint{0.967446in}{1.384966in}}%
\pgfpathlineto{\pgfqpoint{0.964975in}{1.378746in}}%
\pgfpathlineto{\pgfqpoint{0.962506in}{1.372515in}}%
\pgfpathlineto{\pgfqpoint{0.969997in}{1.375794in}}%
\pgfpathlineto{\pgfqpoint{0.977687in}{1.378956in}}%
\pgfpathlineto{\pgfqpoint{0.985567in}{1.381996in}}%
\pgfpathlineto{\pgfqpoint{0.993631in}{1.384913in}}%
\pgfpathlineto{\pgfqpoint{0.995742in}{1.390993in}}%
\pgfpathlineto{\pgfqpoint{0.997856in}{1.397061in}}%
\pgfpathlineto{\pgfqpoint{0.999972in}{1.403115in}}%
\pgfpathlineto{\pgfqpoint{1.002092in}{1.409152in}}%
\pgfpathlineto{\pgfqpoint{0.994398in}{1.406378in}}%
\pgfpathlineto{\pgfqpoint{0.986879in}{1.403486in}}%
\pgfpathlineto{\pgfqpoint{0.979544in}{1.400480in}}%
\pgfpathlineto{\pgfqpoint{0.972398in}{1.397361in}}%
\pgfpathclose%
\pgfusepath{fill}%
\end{pgfscope}%
\begin{pgfscope}%
\pgfpathrectangle{\pgfqpoint{0.041670in}{0.041670in}}{\pgfqpoint{2.216660in}{2.216660in}}%
\pgfusepath{clip}%
\pgfsetbuttcap%
\pgfsetroundjoin%
\definecolor{currentfill}{rgb}{0.134692,0.658636,0.517649}%
\pgfsetfillcolor{currentfill}%
\pgfsetlinewidth{0.000000pt}%
\definecolor{currentstroke}{rgb}{0.000000,0.000000,0.000000}%
\pgfsetstrokecolor{currentstroke}%
\pgfsetdash{}{0pt}%
\pgfpathmoveto{\pgfqpoint{1.112391in}{1.477518in}}%
\pgfpathlineto{\pgfqpoint{1.111494in}{1.472002in}}%
\pgfpathlineto{\pgfqpoint{1.110599in}{1.466450in}}%
\pgfpathlineto{\pgfqpoint{1.109704in}{1.460864in}}%
\pgfpathlineto{\pgfqpoint{1.108812in}{1.455246in}}%
\pgfpathlineto{\pgfqpoint{1.117622in}{1.456257in}}%
\pgfpathlineto{\pgfqpoint{1.126490in}{1.457135in}}%
\pgfpathlineto{\pgfqpoint{1.135408in}{1.457878in}}%
\pgfpathlineto{\pgfqpoint{1.144368in}{1.458485in}}%
\pgfpathlineto{\pgfqpoint{1.144815in}{1.464060in}}%
\pgfpathlineto{\pgfqpoint{1.145262in}{1.469603in}}%
\pgfpathlineto{\pgfqpoint{1.145710in}{1.475112in}}%
\pgfpathlineto{\pgfqpoint{1.146158in}{1.480584in}}%
\pgfpathlineto{\pgfqpoint{1.137648in}{1.480009in}}%
\pgfpathlineto{\pgfqpoint{1.129179in}{1.479306in}}%
\pgfpathlineto{\pgfqpoint{1.120757in}{1.478476in}}%
\pgfpathlineto{\pgfqpoint{1.112391in}{1.477518in}}%
\pgfpathclose%
\pgfusepath{fill}%
\end{pgfscope}%
\begin{pgfscope}%
\pgfpathrectangle{\pgfqpoint{0.041670in}{0.041670in}}{\pgfqpoint{2.216660in}{2.216660in}}%
\pgfusepath{clip}%
\pgfsetbuttcap%
\pgfsetroundjoin%
\definecolor{currentfill}{rgb}{0.134692,0.658636,0.517649}%
\pgfsetfillcolor{currentfill}%
\pgfsetlinewidth{0.000000pt}%
\definecolor{currentstroke}{rgb}{0.000000,0.000000,0.000000}%
\pgfsetstrokecolor{currentstroke}%
\pgfsetdash{}{0pt}%
\pgfpathmoveto{\pgfqpoint{1.214699in}{1.480526in}}%
\pgfpathlineto{\pgfqpoint{1.215160in}{1.475053in}}%
\pgfpathlineto{\pgfqpoint{1.215620in}{1.469543in}}%
\pgfpathlineto{\pgfqpoint{1.216080in}{1.464000in}}%
\pgfpathlineto{\pgfqpoint{1.216539in}{1.458424in}}%
\pgfpathlineto{\pgfqpoint{1.225495in}{1.457802in}}%
\pgfpathlineto{\pgfqpoint{1.234408in}{1.457044in}}%
\pgfpathlineto{\pgfqpoint{1.243270in}{1.456152in}}%
\pgfpathlineto{\pgfqpoint{1.252073in}{1.455125in}}%
\pgfpathlineto{\pgfqpoint{1.251168in}{1.460745in}}%
\pgfpathlineto{\pgfqpoint{1.250262in}{1.466333in}}%
\pgfpathlineto{\pgfqpoint{1.249354in}{1.471887in}}%
\pgfpathlineto{\pgfqpoint{1.248445in}{1.477404in}}%
\pgfpathlineto{\pgfqpoint{1.240085in}{1.478375in}}%
\pgfpathlineto{\pgfqpoint{1.231669in}{1.479220in}}%
\pgfpathlineto{\pgfqpoint{1.223204in}{1.479937in}}%
\pgfpathlineto{\pgfqpoint{1.214699in}{1.480526in}}%
\pgfpathclose%
\pgfusepath{fill}%
\end{pgfscope}%
\begin{pgfscope}%
\pgfpathrectangle{\pgfqpoint{0.041670in}{0.041670in}}{\pgfqpoint{2.216660in}{2.216660in}}%
\pgfusepath{clip}%
\pgfsetbuttcap%
\pgfsetroundjoin%
\definecolor{currentfill}{rgb}{0.120081,0.622161,0.534946}%
\pgfsetfillcolor{currentfill}%
\pgfsetlinewidth{0.000000pt}%
\definecolor{currentstroke}{rgb}{0.000000,0.000000,0.000000}%
\pgfsetstrokecolor{currentstroke}%
\pgfsetdash{}{0pt}%
\pgfpathmoveto{\pgfqpoint{1.311346in}{1.444277in}}%
\pgfpathlineto{\pgfqpoint{1.312996in}{1.438483in}}%
\pgfpathlineto{\pgfqpoint{1.314643in}{1.432662in}}%
\pgfpathlineto{\pgfqpoint{1.316289in}{1.426816in}}%
\pgfpathlineto{\pgfqpoint{1.317932in}{1.420949in}}%
\pgfpathlineto{\pgfqpoint{1.326373in}{1.418782in}}%
\pgfpathlineto{\pgfqpoint{1.334676in}{1.416486in}}%
\pgfpathlineto{\pgfqpoint{1.342834in}{1.414064in}}%
\pgfpathlineto{\pgfqpoint{1.350838in}{1.411518in}}%
\pgfpathlineto{\pgfqpoint{1.348800in}{1.417506in}}%
\pgfpathlineto{\pgfqpoint{1.346760in}{1.423472in}}%
\pgfpathlineto{\pgfqpoint{1.344717in}{1.429414in}}%
\pgfpathlineto{\pgfqpoint{1.342671in}{1.435329in}}%
\pgfpathlineto{\pgfqpoint{1.335052in}{1.437744in}}%
\pgfpathlineto{\pgfqpoint{1.327287in}{1.440042in}}%
\pgfpathlineto{\pgfqpoint{1.319382in}{1.442221in}}%
\pgfpathlineto{\pgfqpoint{1.311346in}{1.444277in}}%
\pgfpathclose%
\pgfusepath{fill}%
\end{pgfscope}%
\begin{pgfscope}%
\pgfpathrectangle{\pgfqpoint{0.041670in}{0.041670in}}{\pgfqpoint{2.216660in}{2.216660in}}%
\pgfusepath{clip}%
\pgfsetbuttcap%
\pgfsetroundjoin%
\definecolor{currentfill}{rgb}{0.282884,0.135920,0.453427}%
\pgfsetfillcolor{currentfill}%
\pgfsetlinewidth{0.000000pt}%
\definecolor{currentstroke}{rgb}{0.000000,0.000000,0.000000}%
\pgfsetstrokecolor{currentstroke}%
\pgfsetdash{}{0pt}%
\pgfpathmoveto{\pgfqpoint{1.839754in}{0.964383in}}%
\pgfpathlineto{\pgfqpoint{1.843693in}{0.970895in}}%
\pgfpathlineto{\pgfqpoint{1.847645in}{0.977770in}}%
\pgfpathlineto{\pgfqpoint{1.851612in}{0.985015in}}%
\pgfpathlineto{\pgfqpoint{1.855594in}{0.992635in}}%
\pgfpathlineto{\pgfqpoint{1.858123in}{0.981651in}}%
\pgfpathlineto{\pgfqpoint{1.859984in}{0.970613in}}%
\pgfpathlineto{\pgfqpoint{1.861174in}{0.959533in}}%
\pgfpathlineto{\pgfqpoint{1.861686in}{0.948423in}}%
\pgfpathlineto{\pgfqpoint{1.857644in}{0.941017in}}%
\pgfpathlineto{\pgfqpoint{1.853617in}{0.933989in}}%
\pgfpathlineto{\pgfqpoint{1.849604in}{0.927332in}}%
\pgfpathlineto{\pgfqpoint{1.845607in}{0.921041in}}%
\pgfpathlineto{\pgfqpoint{1.845131in}{0.931932in}}%
\pgfpathlineto{\pgfqpoint{1.843994in}{0.942793in}}%
\pgfpathlineto{\pgfqpoint{1.842200in}{0.953613in}}%
\pgfpathlineto{\pgfqpoint{1.839754in}{0.964383in}}%
\pgfpathclose%
\pgfusepath{fill}%
\end{pgfscope}%
\begin{pgfscope}%
\pgfpathrectangle{\pgfqpoint{0.041670in}{0.041670in}}{\pgfqpoint{2.216660in}{2.216660in}}%
\pgfusepath{clip}%
\pgfsetbuttcap%
\pgfsetroundjoin%
\definecolor{currentfill}{rgb}{0.163625,0.471133,0.558148}%
\pgfsetfillcolor{currentfill}%
\pgfsetlinewidth{0.000000pt}%
\definecolor{currentstroke}{rgb}{0.000000,0.000000,0.000000}%
\pgfsetstrokecolor{currentstroke}%
\pgfsetdash{}{0pt}%
\pgfpathmoveto{\pgfqpoint{0.864261in}{1.270500in}}%
\pgfpathlineto{\pgfqpoint{0.860983in}{1.263699in}}%
\pgfpathlineto{\pgfqpoint{0.857708in}{1.256914in}}%
\pgfpathlineto{\pgfqpoint{0.854435in}{1.250148in}}%
\pgfpathlineto{\pgfqpoint{0.851165in}{1.243403in}}%
\pgfpathlineto{\pgfqpoint{0.856405in}{1.248503in}}%
\pgfpathlineto{\pgfqpoint{0.861955in}{1.253516in}}%
\pgfpathlineto{\pgfqpoint{0.867810in}{1.258435in}}%
\pgfpathlineto{\pgfqpoint{0.873963in}{1.263258in}}%
\pgfpathlineto{\pgfqpoint{0.876997in}{1.269787in}}%
\pgfpathlineto{\pgfqpoint{0.880034in}{1.276337in}}%
\pgfpathlineto{\pgfqpoint{0.883074in}{1.282906in}}%
\pgfpathlineto{\pgfqpoint{0.886117in}{1.289491in}}%
\pgfpathlineto{\pgfqpoint{0.880216in}{1.284878in}}%
\pgfpathlineto{\pgfqpoint{0.874603in}{1.280173in}}%
\pgfpathlineto{\pgfqpoint{0.869283in}{1.275378in}}%
\pgfpathlineto{\pgfqpoint{0.864261in}{1.270500in}}%
\pgfpathclose%
\pgfusepath{fill}%
\end{pgfscope}%
\begin{pgfscope}%
\pgfpathrectangle{\pgfqpoint{0.041670in}{0.041670in}}{\pgfqpoint{2.216660in}{2.216660in}}%
\pgfusepath{clip}%
\pgfsetbuttcap%
\pgfsetroundjoin%
\definecolor{currentfill}{rgb}{0.283072,0.130895,0.449241}%
\pgfsetfillcolor{currentfill}%
\pgfsetlinewidth{0.000000pt}%
\definecolor{currentstroke}{rgb}{0.000000,0.000000,0.000000}%
\pgfsetstrokecolor{currentstroke}%
\pgfsetdash{}{0pt}%
\pgfpathmoveto{\pgfqpoint{1.660774in}{0.974918in}}%
\pgfpathlineto{\pgfqpoint{1.664399in}{0.970005in}}%
\pgfpathlineto{\pgfqpoint{1.668026in}{0.965227in}}%
\pgfpathlineto{\pgfqpoint{1.671654in}{0.960589in}}%
\pgfpathlineto{\pgfqpoint{1.675283in}{0.956094in}}%
\pgfpathlineto{\pgfqpoint{1.676932in}{0.948002in}}%
\pgfpathlineto{\pgfqpoint{1.678090in}{0.939879in}}%
\pgfpathlineto{\pgfqpoint{1.678754in}{0.931731in}}%
\pgfpathlineto{\pgfqpoint{1.678921in}{0.923567in}}%
\pgfpathlineto{\pgfqpoint{1.675248in}{0.928322in}}%
\pgfpathlineto{\pgfqpoint{1.671577in}{0.933221in}}%
\pgfpathlineto{\pgfqpoint{1.667908in}{0.938258in}}%
\pgfpathlineto{\pgfqpoint{1.664240in}{0.943432in}}%
\pgfpathlineto{\pgfqpoint{1.664094in}{0.951334in}}%
\pgfpathlineto{\pgfqpoint{1.663466in}{0.959220in}}%
\pgfpathlineto{\pgfqpoint{1.662358in}{0.967084in}}%
\pgfpathlineto{\pgfqpoint{1.660774in}{0.974918in}}%
\pgfpathclose%
\pgfusepath{fill}%
\end{pgfscope}%
\begin{pgfscope}%
\pgfpathrectangle{\pgfqpoint{0.041670in}{0.041670in}}{\pgfqpoint{2.216660in}{2.216660in}}%
\pgfusepath{clip}%
\pgfsetbuttcap%
\pgfsetroundjoin%
\definecolor{currentfill}{rgb}{0.120081,0.622161,0.534946}%
\pgfsetfillcolor{currentfill}%
\pgfsetlinewidth{0.000000pt}%
\definecolor{currentstroke}{rgb}{0.000000,0.000000,0.000000}%
\pgfsetstrokecolor{currentstroke}%
\pgfsetdash{}{0pt}%
\pgfpathmoveto{\pgfqpoint{1.010595in}{1.433084in}}%
\pgfpathlineto{\pgfqpoint{1.008465in}{1.427139in}}%
\pgfpathlineto{\pgfqpoint{1.006338in}{1.421167in}}%
\pgfpathlineto{\pgfqpoint{1.004213in}{1.415171in}}%
\pgfpathlineto{\pgfqpoint{1.002092in}{1.409152in}}%
\pgfpathlineto{\pgfqpoint{1.009953in}{1.411807in}}%
\pgfpathlineto{\pgfqpoint{1.017975in}{1.414339in}}%
\pgfpathlineto{\pgfqpoint{1.026149in}{1.416747in}}%
\pgfpathlineto{\pgfqpoint{1.034468in}{1.419029in}}%
\pgfpathlineto{\pgfqpoint{1.036201in}{1.424921in}}%
\pgfpathlineto{\pgfqpoint{1.037937in}{1.430791in}}%
\pgfpathlineto{\pgfqpoint{1.039674in}{1.436636in}}%
\pgfpathlineto{\pgfqpoint{1.041414in}{1.442455in}}%
\pgfpathlineto{\pgfqpoint{1.033495in}{1.440290in}}%
\pgfpathlineto{\pgfqpoint{1.025713in}{1.438006in}}%
\pgfpathlineto{\pgfqpoint{1.018078in}{1.435603in}}%
\pgfpathlineto{\pgfqpoint{1.010595in}{1.433084in}}%
\pgfpathclose%
\pgfusepath{fill}%
\end{pgfscope}%
\begin{pgfscope}%
\pgfpathrectangle{\pgfqpoint{0.041670in}{0.041670in}}{\pgfqpoint{2.216660in}{2.216660in}}%
\pgfusepath{clip}%
\pgfsetbuttcap%
\pgfsetroundjoin%
\definecolor{currentfill}{rgb}{0.282327,0.094955,0.417331}%
\pgfsetfillcolor{currentfill}%
\pgfsetlinewidth{0.000000pt}%
\definecolor{currentstroke}{rgb}{0.000000,0.000000,0.000000}%
\pgfsetstrokecolor{currentstroke}%
\pgfsetdash{}{0pt}%
\pgfpathmoveto{\pgfqpoint{0.530310in}{0.889916in}}%
\pgfpathlineto{\pgfqpoint{0.526362in}{0.894755in}}%
\pgfpathlineto{\pgfqpoint{0.522401in}{0.899935in}}%
\pgfpathlineto{\pgfqpoint{0.518426in}{0.905463in}}%
\pgfpathlineto{\pgfqpoint{0.514438in}{0.911345in}}%
\pgfpathlineto{\pgfqpoint{0.514323in}{0.922253in}}%
\pgfpathlineto{\pgfqpoint{0.514873in}{0.933140in}}%
\pgfpathlineto{\pgfqpoint{0.516083in}{0.943997in}}%
\pgfpathlineto{\pgfqpoint{0.517949in}{0.954813in}}%
\pgfpathlineto{\pgfqpoint{0.521893in}{0.948708in}}%
\pgfpathlineto{\pgfqpoint{0.525823in}{0.942954in}}%
\pgfpathlineto{\pgfqpoint{0.529740in}{0.937547in}}%
\pgfpathlineto{\pgfqpoint{0.533645in}{0.932480in}}%
\pgfpathlineto{\pgfqpoint{0.531846in}{0.921888in}}%
\pgfpathlineto{\pgfqpoint{0.530688in}{0.911257in}}%
\pgfpathlineto{\pgfqpoint{0.530175in}{0.900596in}}%
\pgfpathlineto{\pgfqpoint{0.530310in}{0.889916in}}%
\pgfpathclose%
\pgfusepath{fill}%
\end{pgfscope}%
\begin{pgfscope}%
\pgfpathrectangle{\pgfqpoint{0.041670in}{0.041670in}}{\pgfqpoint{2.216660in}{2.216660in}}%
\pgfusepath{clip}%
\pgfsetbuttcap%
\pgfsetroundjoin%
\definecolor{currentfill}{rgb}{0.282327,0.094955,0.417331}%
\pgfsetfillcolor{currentfill}%
\pgfsetlinewidth{0.000000pt}%
\definecolor{currentstroke}{rgb}{0.000000,0.000000,0.000000}%
\pgfsetstrokecolor{currentstroke}%
\pgfsetdash{}{0pt}%
\pgfpathmoveto{\pgfqpoint{0.681258in}{0.916304in}}%
\pgfpathlineto{\pgfqpoint{0.677582in}{0.911638in}}%
\pgfpathlineto{\pgfqpoint{0.673903in}{0.907123in}}%
\pgfpathlineto{\pgfqpoint{0.670222in}{0.902763in}}%
\pgfpathlineto{\pgfqpoint{0.666539in}{0.898562in}}%
\pgfpathlineto{\pgfqpoint{0.666272in}{0.906995in}}%
\pgfpathlineto{\pgfqpoint{0.666518in}{0.915418in}}%
\pgfpathlineto{\pgfqpoint{0.667276in}{0.923823in}}%
\pgfpathlineto{\pgfqpoint{0.668541in}{0.932202in}}%
\pgfpathlineto{\pgfqpoint{0.672193in}{0.936143in}}%
\pgfpathlineto{\pgfqpoint{0.675843in}{0.940243in}}%
\pgfpathlineto{\pgfqpoint{0.679491in}{0.944497in}}%
\pgfpathlineto{\pgfqpoint{0.683137in}{0.948903in}}%
\pgfpathlineto{\pgfqpoint{0.681924in}{0.940783in}}%
\pgfpathlineto{\pgfqpoint{0.681205in}{0.932637in}}%
\pgfpathlineto{\pgfqpoint{0.680983in}{0.924475in}}%
\pgfpathlineto{\pgfqpoint{0.681258in}{0.916304in}}%
\pgfpathclose%
\pgfusepath{fill}%
\end{pgfscope}%
\begin{pgfscope}%
\pgfpathrectangle{\pgfqpoint{0.041670in}{0.041670in}}{\pgfqpoint{2.216660in}{2.216660in}}%
\pgfusepath{clip}%
\pgfsetbuttcap%
\pgfsetroundjoin%
\definecolor{currentfill}{rgb}{0.212395,0.359683,0.551710}%
\pgfsetfillcolor{currentfill}%
\pgfsetlinewidth{0.000000pt}%
\definecolor{currentstroke}{rgb}{0.000000,0.000000,0.000000}%
\pgfsetstrokecolor{currentstroke}%
\pgfsetdash{}{0pt}%
\pgfpathmoveto{\pgfqpoint{1.550256in}{1.172768in}}%
\pgfpathlineto{\pgfqpoint{1.553653in}{1.166173in}}%
\pgfpathlineto{\pgfqpoint{1.557047in}{1.159632in}}%
\pgfpathlineto{\pgfqpoint{1.560440in}{1.153145in}}%
\pgfpathlineto{\pgfqpoint{1.563830in}{1.146717in}}%
\pgfpathlineto{\pgfqpoint{1.568236in}{1.140589in}}%
\pgfpathlineto{\pgfqpoint{1.572265in}{1.134390in}}%
\pgfpathlineto{\pgfqpoint{1.575914in}{1.128123in}}%
\pgfpathlineto{\pgfqpoint{1.579176in}{1.121796in}}%
\pgfpathlineto{\pgfqpoint{1.575638in}{1.128468in}}%
\pgfpathlineto{\pgfqpoint{1.572097in}{1.135198in}}%
\pgfpathlineto{\pgfqpoint{1.568555in}{1.141983in}}%
\pgfpathlineto{\pgfqpoint{1.565010in}{1.148821in}}%
\pgfpathlineto{\pgfqpoint{1.561876in}{1.154900in}}%
\pgfpathlineto{\pgfqpoint{1.558369in}{1.160921in}}%
\pgfpathlineto{\pgfqpoint{1.554494in}{1.166879in}}%
\pgfpathlineto{\pgfqpoint{1.550256in}{1.172768in}}%
\pgfpathclose%
\pgfusepath{fill}%
\end{pgfscope}%
\begin{pgfscope}%
\pgfpathrectangle{\pgfqpoint{0.041670in}{0.041670in}}{\pgfqpoint{2.216660in}{2.216660in}}%
\pgfusepath{clip}%
\pgfsetbuttcap%
\pgfsetroundjoin%
\definecolor{currentfill}{rgb}{0.134692,0.658636,0.517649}%
\pgfsetfillcolor{currentfill}%
\pgfsetlinewidth{0.000000pt}%
\definecolor{currentstroke}{rgb}{0.000000,0.000000,0.000000}%
\pgfsetstrokecolor{currentstroke}%
\pgfsetdash{}{0pt}%
\pgfpathmoveto{\pgfqpoint{1.248445in}{1.477404in}}%
\pgfpathlineto{\pgfqpoint{1.249354in}{1.471887in}}%
\pgfpathlineto{\pgfqpoint{1.250262in}{1.466333in}}%
\pgfpathlineto{\pgfqpoint{1.251168in}{1.460745in}}%
\pgfpathlineto{\pgfqpoint{1.252073in}{1.455125in}}%
\pgfpathlineto{\pgfqpoint{1.260809in}{1.453965in}}%
\pgfpathlineto{\pgfqpoint{1.269468in}{1.452674in}}%
\pgfpathlineto{\pgfqpoint{1.278044in}{1.451251in}}%
\pgfpathlineto{\pgfqpoint{1.276812in}{1.456923in}}%
\pgfpathlineto{\pgfqpoint{1.275579in}{1.462562in}}%
\pgfpathlineto{\pgfqpoint{1.274343in}{1.468168in}}%
\pgfpathlineto{\pgfqpoint{1.273106in}{1.473737in}}%
\pgfpathlineto{\pgfqpoint{1.264963in}{1.475084in}}%
\pgfpathlineto{\pgfqpoint{1.256740in}{1.476306in}}%
\pgfpathlineto{\pgfqpoint{1.248445in}{1.477404in}}%
\pgfpathclose%
\pgfusepath{fill}%
\end{pgfscope}%
\begin{pgfscope}%
\pgfpathrectangle{\pgfqpoint{0.041670in}{0.041670in}}{\pgfqpoint{2.216660in}{2.216660in}}%
\pgfusepath{clip}%
\pgfsetbuttcap%
\pgfsetroundjoin%
\definecolor{currentfill}{rgb}{0.263663,0.237631,0.518762}%
\pgfsetfillcolor{currentfill}%
\pgfsetlinewidth{0.000000pt}%
\definecolor{currentstroke}{rgb}{0.000000,0.000000,0.000000}%
\pgfsetstrokecolor{currentstroke}%
\pgfsetdash{}{0pt}%
\pgfpathmoveto{\pgfqpoint{0.741340in}{1.036821in}}%
\pgfpathlineto{\pgfqpoint{0.737703in}{1.030512in}}%
\pgfpathlineto{\pgfqpoint{0.734067in}{1.024295in}}%
\pgfpathlineto{\pgfqpoint{0.730431in}{1.018173in}}%
\pgfpathlineto{\pgfqpoint{0.726796in}{1.012150in}}%
\pgfpathlineto{\pgfqpoint{0.728305in}{1.019462in}}%
\pgfpathlineto{\pgfqpoint{0.730259in}{1.026739in}}%
\pgfpathlineto{\pgfqpoint{0.732655in}{1.033973in}}%
\pgfpathlineto{\pgfqpoint{0.735490in}{1.041157in}}%
\pgfpathlineto{\pgfqpoint{0.739041in}{1.046924in}}%
\pgfpathlineto{\pgfqpoint{0.742593in}{1.052790in}}%
\pgfpathlineto{\pgfqpoint{0.746145in}{1.058751in}}%
\pgfpathlineto{\pgfqpoint{0.749698in}{1.064804in}}%
\pgfpathlineto{\pgfqpoint{0.746968in}{1.057872in}}%
\pgfpathlineto{\pgfqpoint{0.744663in}{1.050894in}}%
\pgfpathlineto{\pgfqpoint{0.742786in}{1.043874in}}%
\pgfpathlineto{\pgfqpoint{0.741340in}{1.036821in}}%
\pgfpathclose%
\pgfusepath{fill}%
\end{pgfscope}%
\begin{pgfscope}%
\pgfpathrectangle{\pgfqpoint{0.041670in}{0.041670in}}{\pgfqpoint{2.216660in}{2.216660in}}%
\pgfusepath{clip}%
\pgfsetbuttcap%
\pgfsetroundjoin%
\definecolor{currentfill}{rgb}{0.233603,0.313828,0.543914}%
\pgfsetfillcolor{currentfill}%
\pgfsetlinewidth{0.000000pt}%
\definecolor{currentstroke}{rgb}{0.000000,0.000000,0.000000}%
\pgfsetstrokecolor{currentstroke}%
\pgfsetdash{}{0pt}%
\pgfpathmoveto{\pgfqpoint{0.469443in}{1.057803in}}%
\pgfpathlineto{\pgfqpoint{0.465289in}{1.069063in}}%
\pgfpathlineto{\pgfqpoint{0.461115in}{1.080768in}}%
\pgfpathlineto{\pgfqpoint{0.456922in}{1.092925in}}%
\pgfpathlineto{\pgfqpoint{0.452708in}{1.105542in}}%
\pgfpathlineto{\pgfqpoint{0.455581in}{1.117115in}}%
\pgfpathlineto{\pgfqpoint{0.459163in}{1.128617in}}%
\pgfpathlineto{\pgfqpoint{0.463446in}{1.140038in}}%
\pgfpathlineto{\pgfqpoint{0.468422in}{1.151367in}}%
\pgfpathlineto{\pgfqpoint{0.472518in}{1.138569in}}%
\pgfpathlineto{\pgfqpoint{0.476594in}{1.126228in}}%
\pgfpathlineto{\pgfqpoint{0.480652in}{1.114336in}}%
\pgfpathlineto{\pgfqpoint{0.484691in}{1.102886in}}%
\pgfpathlineto{\pgfqpoint{0.479855in}{1.091739in}}%
\pgfpathlineto{\pgfqpoint{0.475696in}{1.080503in}}%
\pgfpathlineto{\pgfqpoint{0.472224in}{1.069187in}}%
\pgfpathlineto{\pgfqpoint{0.469443in}{1.057803in}}%
\pgfpathclose%
\pgfusepath{fill}%
\end{pgfscope}%
\begin{pgfscope}%
\pgfpathrectangle{\pgfqpoint{0.041670in}{0.041670in}}{\pgfqpoint{2.216660in}{2.216660in}}%
\pgfusepath{clip}%
\pgfsetbuttcap%
\pgfsetroundjoin%
\definecolor{currentfill}{rgb}{0.134692,0.658636,0.517649}%
\pgfsetfillcolor{currentfill}%
\pgfsetlinewidth{0.000000pt}%
\definecolor{currentstroke}{rgb}{0.000000,0.000000,0.000000}%
\pgfsetstrokecolor{currentstroke}%
\pgfsetdash{}{0pt}%
\pgfpathmoveto{\pgfqpoint{1.079638in}{1.472437in}}%
\pgfpathlineto{\pgfqpoint{1.078306in}{1.466850in}}%
\pgfpathlineto{\pgfqpoint{1.076976in}{1.461226in}}%
\pgfpathlineto{\pgfqpoint{1.075647in}{1.455567in}}%
\pgfpathlineto{\pgfqpoint{1.074320in}{1.449877in}}%
\pgfpathlineto{\pgfqpoint{1.082814in}{1.451416in}}%
\pgfpathlineto{\pgfqpoint{1.091400in}{1.452824in}}%
\pgfpathlineto{\pgfqpoint{1.100068in}{1.454101in}}%
\pgfpathlineto{\pgfqpoint{1.108812in}{1.455246in}}%
\pgfpathlineto{\pgfqpoint{1.109704in}{1.460864in}}%
\pgfpathlineto{\pgfqpoint{1.110599in}{1.466450in}}%
\pgfpathlineto{\pgfqpoint{1.111494in}{1.472002in}}%
\pgfpathlineto{\pgfqpoint{1.112391in}{1.477518in}}%
\pgfpathlineto{\pgfqpoint{1.104088in}{1.476434in}}%
\pgfpathlineto{\pgfqpoint{1.095856in}{1.475226in}}%
\pgfpathlineto{\pgfqpoint{1.087704in}{1.473893in}}%
\pgfpathlineto{\pgfqpoint{1.079638in}{1.472437in}}%
\pgfpathclose%
\pgfusepath{fill}%
\end{pgfscope}%
\begin{pgfscope}%
\pgfpathrectangle{\pgfqpoint{0.041670in}{0.041670in}}{\pgfqpoint{2.216660in}{2.216660in}}%
\pgfusepath{clip}%
\pgfsetbuttcap%
\pgfsetroundjoin%
\definecolor{currentfill}{rgb}{0.179019,0.433756,0.557430}%
\pgfsetfillcolor{currentfill}%
\pgfsetlinewidth{0.000000pt}%
\definecolor{currentstroke}{rgb}{0.000000,0.000000,0.000000}%
\pgfsetstrokecolor{currentstroke}%
\pgfsetdash{}{0pt}%
\pgfpathmoveto{\pgfqpoint{1.504103in}{1.247941in}}%
\pgfpathlineto{\pgfqpoint{1.507321in}{1.241270in}}%
\pgfpathlineto{\pgfqpoint{1.510537in}{1.234627in}}%
\pgfpathlineto{\pgfqpoint{1.513750in}{1.228015in}}%
\pgfpathlineto{\pgfqpoint{1.516960in}{1.221437in}}%
\pgfpathlineto{\pgfqpoint{1.522381in}{1.216103in}}%
\pgfpathlineto{\pgfqpoint{1.527473in}{1.210683in}}%
\pgfpathlineto{\pgfqpoint{1.532230in}{1.205181in}}%
\pgfpathlineto{\pgfqpoint{1.536646in}{1.199603in}}%
\pgfpathlineto{\pgfqpoint{1.533238in}{1.206411in}}%
\pgfpathlineto{\pgfqpoint{1.529826in}{1.213252in}}%
\pgfpathlineto{\pgfqpoint{1.526412in}{1.220124in}}%
\pgfpathlineto{\pgfqpoint{1.522996in}{1.227023in}}%
\pgfpathlineto{\pgfqpoint{1.518759in}{1.232367in}}%
\pgfpathlineto{\pgfqpoint{1.514194in}{1.237638in}}%
\pgfpathlineto{\pgfqpoint{1.509307in}{1.242831in}}%
\pgfpathlineto{\pgfqpoint{1.504103in}{1.247941in}}%
\pgfpathclose%
\pgfusepath{fill}%
\end{pgfscope}%
\begin{pgfscope}%
\pgfpathrectangle{\pgfqpoint{0.041670in}{0.041670in}}{\pgfqpoint{2.216660in}{2.216660in}}%
\pgfusepath{clip}%
\pgfsetbuttcap%
\pgfsetroundjoin%
\definecolor{currentfill}{rgb}{0.248629,0.278775,0.534556}%
\pgfsetfillcolor{currentfill}%
\pgfsetlinewidth{0.000000pt}%
\definecolor{currentstroke}{rgb}{0.000000,0.000000,0.000000}%
\pgfsetstrokecolor{currentstroke}%
\pgfsetdash{}{0pt}%
\pgfpathmoveto{\pgfqpoint{1.593315in}{1.095759in}}%
\pgfpathlineto{\pgfqpoint{1.596846in}{1.089428in}}%
\pgfpathlineto{\pgfqpoint{1.600376in}{1.083176in}}%
\pgfpathlineto{\pgfqpoint{1.603904in}{1.077005in}}%
\pgfpathlineto{\pgfqpoint{1.607432in}{1.070920in}}%
\pgfpathlineto{\pgfqpoint{1.610536in}{1.064036in}}%
\pgfpathlineto{\pgfqpoint{1.613219in}{1.057099in}}%
\pgfpathlineto{\pgfqpoint{1.615477in}{1.050116in}}%
\pgfpathlineto{\pgfqpoint{1.617306in}{1.043092in}}%
\pgfpathlineto{\pgfqpoint{1.613683in}{1.049432in}}%
\pgfpathlineto{\pgfqpoint{1.610059in}{1.055857in}}%
\pgfpathlineto{\pgfqpoint{1.606434in}{1.062364in}}%
\pgfpathlineto{\pgfqpoint{1.602807in}{1.068949in}}%
\pgfpathlineto{\pgfqpoint{1.601053in}{1.075714in}}%
\pgfpathlineto{\pgfqpoint{1.598883in}{1.082442in}}%
\pgfpathlineto{\pgfqpoint{1.596303in}{1.089126in}}%
\pgfpathlineto{\pgfqpoint{1.593315in}{1.095759in}}%
\pgfpathclose%
\pgfusepath{fill}%
\end{pgfscope}%
\begin{pgfscope}%
\pgfpathrectangle{\pgfqpoint{0.041670in}{0.041670in}}{\pgfqpoint{2.216660in}{2.216660in}}%
\pgfusepath{clip}%
\pgfsetbuttcap%
\pgfsetroundjoin%
\definecolor{currentfill}{rgb}{0.122606,0.585371,0.546557}%
\pgfsetfillcolor{currentfill}%
\pgfsetlinewidth{0.000000pt}%
\definecolor{currentstroke}{rgb}{0.000000,0.000000,0.000000}%
\pgfsetstrokecolor{currentstroke}%
\pgfsetdash{}{0pt}%
\pgfpathmoveto{\pgfqpoint{1.381170in}{1.400139in}}%
\pgfpathlineto{\pgfqpoint{1.383570in}{1.393985in}}%
\pgfpathlineto{\pgfqpoint{1.385968in}{1.387815in}}%
\pgfpathlineto{\pgfqpoint{1.388363in}{1.381631in}}%
\pgfpathlineto{\pgfqpoint{1.390755in}{1.375436in}}%
\pgfpathlineto{\pgfqpoint{1.398223in}{1.372144in}}%
\pgfpathlineto{\pgfqpoint{1.405485in}{1.368737in}}%
\pgfpathlineto{\pgfqpoint{1.412533in}{1.365218in}}%
\pgfpathlineto{\pgfqpoint{1.419360in}{1.361591in}}%
\pgfpathlineto{\pgfqpoint{1.416638in}{1.367957in}}%
\pgfpathlineto{\pgfqpoint{1.413913in}{1.374311in}}%
\pgfpathlineto{\pgfqpoint{1.411185in}{1.380650in}}%
\pgfpathlineto{\pgfqpoint{1.408453in}{1.386973in}}%
\pgfpathlineto{\pgfqpoint{1.401943in}{1.390422in}}%
\pgfpathlineto{\pgfqpoint{1.395220in}{1.393768in}}%
\pgfpathlineto{\pgfqpoint{1.388294in}{1.397008in}}%
\pgfpathlineto{\pgfqpoint{1.381170in}{1.400139in}}%
\pgfpathclose%
\pgfusepath{fill}%
\end{pgfscope}%
\begin{pgfscope}%
\pgfpathrectangle{\pgfqpoint{0.041670in}{0.041670in}}{\pgfqpoint{2.216660in}{2.216660in}}%
\pgfusepath{clip}%
\pgfsetbuttcap%
\pgfsetroundjoin%
\definecolor{currentfill}{rgb}{0.133743,0.548535,0.553541}%
\pgfsetfillcolor{currentfill}%
\pgfsetlinewidth{0.000000pt}%
\definecolor{currentstroke}{rgb}{0.000000,0.000000,0.000000}%
\pgfsetstrokecolor{currentstroke}%
\pgfsetdash{}{0pt}%
\pgfpathmoveto{\pgfqpoint{1.419360in}{1.361591in}}%
\pgfpathlineto{\pgfqpoint{1.422079in}{1.355216in}}%
\pgfpathlineto{\pgfqpoint{1.424795in}{1.348836in}}%
\pgfpathlineto{\pgfqpoint{1.427508in}{1.342451in}}%
\pgfpathlineto{\pgfqpoint{1.430217in}{1.336066in}}%
\pgfpathlineto{\pgfqpoint{1.437123in}{1.332150in}}%
\pgfpathlineto{\pgfqpoint{1.443783in}{1.328126in}}%
\pgfpathlineto{\pgfqpoint{1.450193in}{1.323999in}}%
\pgfpathlineto{\pgfqpoint{1.456345in}{1.319773in}}%
\pgfpathlineto{\pgfqpoint{1.453345in}{1.326350in}}%
\pgfpathlineto{\pgfqpoint{1.450342in}{1.332926in}}%
\pgfpathlineto{\pgfqpoint{1.447336in}{1.339499in}}%
\pgfpathlineto{\pgfqpoint{1.444326in}{1.346065in}}%
\pgfpathlineto{\pgfqpoint{1.438449in}{1.350092in}}%
\pgfpathlineto{\pgfqpoint{1.432325in}{1.354025in}}%
\pgfpathlineto{\pgfqpoint{1.425960in}{1.357859in}}%
\pgfpathlineto{\pgfqpoint{1.419360in}{1.361591in}}%
\pgfpathclose%
\pgfusepath{fill}%
\end{pgfscope}%
\begin{pgfscope}%
\pgfpathrectangle{\pgfqpoint{0.041670in}{0.041670in}}{\pgfqpoint{2.216660in}{2.216660in}}%
\pgfusepath{clip}%
\pgfsetbuttcap%
\pgfsetroundjoin%
\definecolor{currentfill}{rgb}{0.212395,0.359683,0.551710}%
\pgfsetfillcolor{currentfill}%
\pgfsetlinewidth{0.000000pt}%
\definecolor{currentstroke}{rgb}{0.000000,0.000000,0.000000}%
\pgfsetstrokecolor{currentstroke}%
\pgfsetdash{}{0pt}%
\pgfpathmoveto{\pgfqpoint{0.792429in}{1.143373in}}%
\pgfpathlineto{\pgfqpoint{0.788859in}{1.136480in}}%
\pgfpathlineto{\pgfqpoint{0.785291in}{1.129639in}}%
\pgfpathlineto{\pgfqpoint{0.781725in}{1.122853in}}%
\pgfpathlineto{\pgfqpoint{0.778160in}{1.116126in}}%
\pgfpathlineto{\pgfqpoint{0.781077in}{1.122502in}}%
\pgfpathlineto{\pgfqpoint{0.784382in}{1.128822in}}%
\pgfpathlineto{\pgfqpoint{0.788073in}{1.135082in}}%
\pgfpathlineto{\pgfqpoint{0.792145in}{1.141274in}}%
\pgfpathlineto{\pgfqpoint{0.795573in}{1.147755in}}%
\pgfpathlineto{\pgfqpoint{0.799003in}{1.154295in}}%
\pgfpathlineto{\pgfqpoint{0.802435in}{1.160890in}}%
\pgfpathlineto{\pgfqpoint{0.805869in}{1.167537in}}%
\pgfpathlineto{\pgfqpoint{0.801953in}{1.161587in}}%
\pgfpathlineto{\pgfqpoint{0.798405in}{1.155572in}}%
\pgfpathlineto{\pgfqpoint{0.795229in}{1.149499in}}%
\pgfpathlineto{\pgfqpoint{0.792429in}{1.143373in}}%
\pgfpathclose%
\pgfusepath{fill}%
\end{pgfscope}%
\begin{pgfscope}%
\pgfpathrectangle{\pgfqpoint{0.041670in}{0.041670in}}{\pgfqpoint{2.216660in}{2.216660in}}%
\pgfusepath{clip}%
\pgfsetbuttcap%
\pgfsetroundjoin%
\definecolor{currentfill}{rgb}{0.134692,0.658636,0.517649}%
\pgfsetfillcolor{currentfill}%
\pgfsetlinewidth{0.000000pt}%
\definecolor{currentstroke}{rgb}{0.000000,0.000000,0.000000}%
\pgfsetstrokecolor{currentstroke}%
\pgfsetdash{}{0pt}%
\pgfpathmoveto{\pgfqpoint{1.273106in}{1.473737in}}%
\pgfpathlineto{\pgfqpoint{1.274343in}{1.468168in}}%
\pgfpathlineto{\pgfqpoint{1.275579in}{1.462562in}}%
\pgfpathlineto{\pgfqpoint{1.276812in}{1.456923in}}%
\pgfpathlineto{\pgfqpoint{1.278044in}{1.451251in}}%
\pgfpathlineto{\pgfqpoint{1.286528in}{1.449699in}}%
\pgfpathlineto{\pgfqpoint{1.294911in}{1.448018in}}%
\pgfpathlineto{\pgfqpoint{1.303187in}{1.446210in}}%
\pgfpathlineto{\pgfqpoint{1.311346in}{1.444277in}}%
\pgfpathlineto{\pgfqpoint{1.309694in}{1.450042in}}%
\pgfpathlineto{\pgfqpoint{1.308040in}{1.455776in}}%
\pgfpathlineto{\pgfqpoint{1.306384in}{1.461475in}}%
\pgfpathlineto{\pgfqpoint{1.304726in}{1.467137in}}%
\pgfpathlineto{\pgfqpoint{1.296979in}{1.468966in}}%
\pgfpathlineto{\pgfqpoint{1.289122in}{1.470677in}}%
\pgfpathlineto{\pgfqpoint{1.281162in}{1.472268in}}%
\pgfpathlineto{\pgfqpoint{1.273106in}{1.473737in}}%
\pgfpathclose%
\pgfusepath{fill}%
\end{pgfscope}%
\begin{pgfscope}%
\pgfpathrectangle{\pgfqpoint{0.041670in}{0.041670in}}{\pgfqpoint{2.216660in}{2.216660in}}%
\pgfusepath{clip}%
\pgfsetbuttcap%
\pgfsetroundjoin%
\definecolor{currentfill}{rgb}{0.283072,0.130895,0.449241}%
\pgfsetfillcolor{currentfill}%
\pgfsetlinewidth{0.000000pt}%
\definecolor{currentstroke}{rgb}{0.000000,0.000000,0.000000}%
\pgfsetstrokecolor{currentstroke}%
\pgfsetdash{}{0pt}%
\pgfpathmoveto{\pgfqpoint{0.695946in}{0.936402in}}%
\pgfpathlineto{\pgfqpoint{0.692276in}{0.931170in}}%
\pgfpathlineto{\pgfqpoint{0.688605in}{0.926074in}}%
\pgfpathlineto{\pgfqpoint{0.684933in}{0.921117in}}%
\pgfpathlineto{\pgfqpoint{0.681258in}{0.916304in}}%
\pgfpathlineto{\pgfqpoint{0.680983in}{0.924475in}}%
\pgfpathlineto{\pgfqpoint{0.681205in}{0.932637in}}%
\pgfpathlineto{\pgfqpoint{0.681924in}{0.940783in}}%
\pgfpathlineto{\pgfqpoint{0.683137in}{0.948903in}}%
\pgfpathlineto{\pgfqpoint{0.686781in}{0.953455in}}%
\pgfpathlineto{\pgfqpoint{0.690423in}{0.958151in}}%
\pgfpathlineto{\pgfqpoint{0.694064in}{0.962986in}}%
\pgfpathlineto{\pgfqpoint{0.697704in}{0.967957in}}%
\pgfpathlineto{\pgfqpoint{0.696543in}{0.960096in}}%
\pgfpathlineto{\pgfqpoint{0.695862in}{0.952211in}}%
\pgfpathlineto{\pgfqpoint{0.695662in}{0.944310in}}%
\pgfpathlineto{\pgfqpoint{0.695946in}{0.936402in}}%
\pgfpathclose%
\pgfusepath{fill}%
\end{pgfscope}%
\begin{pgfscope}%
\pgfpathrectangle{\pgfqpoint{0.041670in}{0.041670in}}{\pgfqpoint{2.216660in}{2.216660in}}%
\pgfusepath{clip}%
\pgfsetbuttcap%
\pgfsetroundjoin%
\definecolor{currentfill}{rgb}{0.280255,0.165693,0.476498}%
\pgfsetfillcolor{currentfill}%
\pgfsetlinewidth{0.000000pt}%
\definecolor{currentstroke}{rgb}{0.000000,0.000000,0.000000}%
\pgfsetstrokecolor{currentstroke}%
\pgfsetdash{}{0pt}%
\pgfpathmoveto{\pgfqpoint{1.646281in}{0.995850in}}%
\pgfpathlineto{\pgfqpoint{1.649903in}{0.990432in}}%
\pgfpathlineto{\pgfqpoint{1.653526in}{0.985135in}}%
\pgfpathlineto{\pgfqpoint{1.657149in}{0.979963in}}%
\pgfpathlineto{\pgfqpoint{1.660774in}{0.974918in}}%
\pgfpathlineto{\pgfqpoint{1.662358in}{0.967084in}}%
\pgfpathlineto{\pgfqpoint{1.663466in}{0.959220in}}%
\pgfpathlineto{\pgfqpoint{1.664094in}{0.951334in}}%
\pgfpathlineto{\pgfqpoint{1.664240in}{0.943432in}}%
\pgfpathlineto{\pgfqpoint{1.660574in}{0.948737in}}%
\pgfpathlineto{\pgfqpoint{1.656908in}{0.954170in}}%
\pgfpathlineto{\pgfqpoint{1.653244in}{0.959728in}}%
\pgfpathlineto{\pgfqpoint{1.649580in}{0.965406in}}%
\pgfpathlineto{\pgfqpoint{1.649454in}{0.973046in}}%
\pgfpathlineto{\pgfqpoint{1.648861in}{0.980671in}}%
\pgfpathlineto{\pgfqpoint{1.647802in}{0.988275in}}%
\pgfpathlineto{\pgfqpoint{1.646281in}{0.995850in}}%
\pgfpathclose%
\pgfusepath{fill}%
\end{pgfscope}%
\begin{pgfscope}%
\pgfpathrectangle{\pgfqpoint{0.041670in}{0.041670in}}{\pgfqpoint{2.216660in}{2.216660in}}%
\pgfusepath{clip}%
\pgfsetbuttcap%
\pgfsetroundjoin%
\definecolor{currentfill}{rgb}{0.179019,0.433756,0.557430}%
\pgfsetfillcolor{currentfill}%
\pgfsetlinewidth{0.000000pt}%
\definecolor{currentstroke}{rgb}{0.000000,0.000000,0.000000}%
\pgfsetstrokecolor{currentstroke}%
\pgfsetdash{}{0pt}%
\pgfpathmoveto{\pgfqpoint{0.833427in}{1.222216in}}%
\pgfpathlineto{\pgfqpoint{0.829973in}{1.215264in}}%
\pgfpathlineto{\pgfqpoint{0.826522in}{1.208339in}}%
\pgfpathlineto{\pgfqpoint{0.823074in}{1.201445in}}%
\pgfpathlineto{\pgfqpoint{0.819628in}{1.194584in}}%
\pgfpathlineto{\pgfqpoint{0.823737in}{1.200226in}}%
\pgfpathlineto{\pgfqpoint{0.828192in}{1.205796in}}%
\pgfpathlineto{\pgfqpoint{0.832986in}{1.211289in}}%
\pgfpathlineto{\pgfqpoint{0.838115in}{1.216700in}}%
\pgfpathlineto{\pgfqpoint{0.841373in}{1.223328in}}%
\pgfpathlineto{\pgfqpoint{0.844635in}{1.229990in}}%
\pgfpathlineto{\pgfqpoint{0.847899in}{1.236683in}}%
\pgfpathlineto{\pgfqpoint{0.851165in}{1.243403in}}%
\pgfpathlineto{\pgfqpoint{0.846243in}{1.238219in}}%
\pgfpathlineto{\pgfqpoint{0.841642in}{1.232957in}}%
\pgfpathlineto{\pgfqpoint{0.837368in}{1.227621in}}%
\pgfpathlineto{\pgfqpoint{0.833427in}{1.222216in}}%
\pgfpathclose%
\pgfusepath{fill}%
\end{pgfscope}%
\begin{pgfscope}%
\pgfpathrectangle{\pgfqpoint{0.041670in}{0.041670in}}{\pgfqpoint{2.216660in}{2.216660in}}%
\pgfusepath{clip}%
\pgfsetbuttcap%
\pgfsetroundjoin%
\definecolor{currentfill}{rgb}{0.201239,0.383670,0.554294}%
\pgfsetfillcolor{currentfill}%
\pgfsetlinewidth{0.000000pt}%
\definecolor{currentstroke}{rgb}{0.000000,0.000000,0.000000}%
\pgfsetstrokecolor{currentstroke}%
\pgfsetdash{}{0pt}%
\pgfpathmoveto{\pgfqpoint{1.886489in}{1.161352in}}%
\pgfpathlineto{\pgfqpoint{1.890571in}{1.174653in}}%
\pgfpathlineto{\pgfqpoint{1.894672in}{1.188426in}}%
\pgfpathlineto{\pgfqpoint{1.898795in}{1.202678in}}%
\pgfpathlineto{\pgfqpoint{1.902939in}{1.217417in}}%
\pgfpathlineto{\pgfqpoint{1.908682in}{1.206010in}}%
\pgfpathlineto{\pgfqpoint{1.913723in}{1.194499in}}%
\pgfpathlineto{\pgfqpoint{1.918055in}{1.182894in}}%
\pgfpathlineto{\pgfqpoint{1.921669in}{1.171208in}}%
\pgfpathlineto{\pgfqpoint{1.917389in}{1.156636in}}%
\pgfpathlineto{\pgfqpoint{1.913132in}{1.142555in}}%
\pgfpathlineto{\pgfqpoint{1.908897in}{1.128957in}}%
\pgfpathlineto{\pgfqpoint{1.904683in}{1.115832in}}%
\pgfpathlineto{\pgfqpoint{1.901179in}{1.127343in}}%
\pgfpathlineto{\pgfqpoint{1.896974in}{1.138774in}}%
\pgfpathlineto{\pgfqpoint{1.892075in}{1.150113in}}%
\pgfpathlineto{\pgfqpoint{1.886489in}{1.161352in}}%
\pgfpathclose%
\pgfusepath{fill}%
\end{pgfscope}%
\begin{pgfscope}%
\pgfpathrectangle{\pgfqpoint{0.041670in}{0.041670in}}{\pgfqpoint{2.216660in}{2.216660in}}%
\pgfusepath{clip}%
\pgfsetbuttcap%
\pgfsetroundjoin%
\definecolor{currentfill}{rgb}{0.134692,0.658636,0.517649}%
\pgfsetfillcolor{currentfill}%
\pgfsetlinewidth{0.000000pt}%
\definecolor{currentstroke}{rgb}{0.000000,0.000000,0.000000}%
\pgfsetstrokecolor{currentstroke}%
\pgfsetdash{}{0pt}%
\pgfpathmoveto{\pgfqpoint{1.048397in}{1.465413in}}%
\pgfpathlineto{\pgfqpoint{1.046648in}{1.459726in}}%
\pgfpathlineto{\pgfqpoint{1.044901in}{1.454003in}}%
\pgfpathlineto{\pgfqpoint{1.043156in}{1.448245in}}%
\pgfpathlineto{\pgfqpoint{1.041414in}{1.442455in}}%
\pgfpathlineto{\pgfqpoint{1.049464in}{1.444498in}}%
\pgfpathlineto{\pgfqpoint{1.057637in}{1.446417in}}%
\pgfpathlineto{\pgfqpoint{1.065925in}{1.448211in}}%
\pgfpathlineto{\pgfqpoint{1.074320in}{1.449877in}}%
\pgfpathlineto{\pgfqpoint{1.075647in}{1.455567in}}%
\pgfpathlineto{\pgfqpoint{1.076976in}{1.461226in}}%
\pgfpathlineto{\pgfqpoint{1.078306in}{1.466850in}}%
\pgfpathlineto{\pgfqpoint{1.079638in}{1.472437in}}%
\pgfpathlineto{\pgfqpoint{1.071667in}{1.470860in}}%
\pgfpathlineto{\pgfqpoint{1.063798in}{1.469162in}}%
\pgfpathlineto{\pgfqpoint{1.056039in}{1.467346in}}%
\pgfpathlineto{\pgfqpoint{1.048397in}{1.465413in}}%
\pgfpathclose%
\pgfusepath{fill}%
\end{pgfscope}%
\begin{pgfscope}%
\pgfpathrectangle{\pgfqpoint{0.041670in}{0.041670in}}{\pgfqpoint{2.216660in}{2.216660in}}%
\pgfusepath{clip}%
\pgfsetbuttcap%
\pgfsetroundjoin%
\definecolor{currentfill}{rgb}{0.120081,0.622161,0.534946}%
\pgfsetfillcolor{currentfill}%
\pgfsetlinewidth{0.000000pt}%
\definecolor{currentstroke}{rgb}{0.000000,0.000000,0.000000}%
\pgfsetstrokecolor{currentstroke}%
\pgfsetdash{}{0pt}%
\pgfpathmoveto{\pgfqpoint{1.342671in}{1.435329in}}%
\pgfpathlineto{\pgfqpoint{1.344717in}{1.429414in}}%
\pgfpathlineto{\pgfqpoint{1.346760in}{1.423472in}}%
\pgfpathlineto{\pgfqpoint{1.348800in}{1.417506in}}%
\pgfpathlineto{\pgfqpoint{1.350838in}{1.411518in}}%
\pgfpathlineto{\pgfqpoint{1.358681in}{1.408850in}}%
\pgfpathlineto{\pgfqpoint{1.366356in}{1.406063in}}%
\pgfpathlineto{\pgfqpoint{1.373855in}{1.403158in}}%
\pgfpathlineto{\pgfqpoint{1.381170in}{1.400139in}}%
\pgfpathlineto{\pgfqpoint{1.378766in}{1.406273in}}%
\pgfpathlineto{\pgfqpoint{1.376359in}{1.412385in}}%
\pgfpathlineto{\pgfqpoint{1.373950in}{1.418472in}}%
\pgfpathlineto{\pgfqpoint{1.371537in}{1.424533in}}%
\pgfpathlineto{\pgfqpoint{1.364577in}{1.427397in}}%
\pgfpathlineto{\pgfqpoint{1.357441in}{1.430153in}}%
\pgfpathlineto{\pgfqpoint{1.350136in}{1.432797in}}%
\pgfpathlineto{\pgfqpoint{1.342671in}{1.435329in}}%
\pgfpathclose%
\pgfusepath{fill}%
\end{pgfscope}%
\begin{pgfscope}%
\pgfpathrectangle{\pgfqpoint{0.041670in}{0.041670in}}{\pgfqpoint{2.216660in}{2.216660in}}%
\pgfusepath{clip}%
\pgfsetbuttcap%
\pgfsetroundjoin%
\definecolor{currentfill}{rgb}{0.282884,0.135920,0.453427}%
\pgfsetfillcolor{currentfill}%
\pgfsetlinewidth{0.000000pt}%
\definecolor{currentstroke}{rgb}{0.000000,0.000000,0.000000}%
\pgfsetstrokecolor{currentstroke}%
\pgfsetdash{}{0pt}%
\pgfpathmoveto{\pgfqpoint{0.514438in}{0.911345in}}%
\pgfpathlineto{\pgfqpoint{0.510435in}{0.917586in}}%
\pgfpathlineto{\pgfqpoint{0.506418in}{0.924193in}}%
\pgfpathlineto{\pgfqpoint{0.502386in}{0.931172in}}%
\pgfpathlineto{\pgfqpoint{0.498339in}{0.938529in}}%
\pgfpathlineto{\pgfqpoint{0.498247in}{0.949658in}}%
\pgfpathlineto{\pgfqpoint{0.498835in}{0.960766in}}%
\pgfpathlineto{\pgfqpoint{0.500099in}{0.971842in}}%
\pgfpathlineto{\pgfqpoint{0.502035in}{0.982874in}}%
\pgfpathlineto{\pgfqpoint{0.506036in}{0.975300in}}%
\pgfpathlineto{\pgfqpoint{0.510021in}{0.968103in}}%
\pgfpathlineto{\pgfqpoint{0.513992in}{0.961276in}}%
\pgfpathlineto{\pgfqpoint{0.517949in}{0.954813in}}%
\pgfpathlineto{\pgfqpoint{0.516083in}{0.943997in}}%
\pgfpathlineto{\pgfqpoint{0.514873in}{0.933140in}}%
\pgfpathlineto{\pgfqpoint{0.514323in}{0.922253in}}%
\pgfpathlineto{\pgfqpoint{0.514438in}{0.911345in}}%
\pgfpathclose%
\pgfusepath{fill}%
\end{pgfscope}%
\begin{pgfscope}%
\pgfpathrectangle{\pgfqpoint{0.041670in}{0.041670in}}{\pgfqpoint{2.216660in}{2.216660in}}%
\pgfusepath{clip}%
\pgfsetbuttcap%
\pgfsetroundjoin%
\definecolor{currentfill}{rgb}{0.122606,0.585371,0.546557}%
\pgfsetfillcolor{currentfill}%
\pgfsetlinewidth{0.000000pt}%
\definecolor{currentstroke}{rgb}{0.000000,0.000000,0.000000}%
\pgfsetstrokecolor{currentstroke}%
\pgfsetdash{}{0pt}%
\pgfpathmoveto{\pgfqpoint{0.945852in}{1.383824in}}%
\pgfpathlineto{\pgfqpoint{0.943052in}{1.377460in}}%
\pgfpathlineto{\pgfqpoint{0.940255in}{1.371080in}}%
\pgfpathlineto{\pgfqpoint{0.937462in}{1.364685in}}%
\pgfpathlineto{\pgfqpoint{0.934672in}{1.358278in}}%
\pgfpathlineto{\pgfqpoint{0.941297in}{1.361999in}}%
\pgfpathlineto{\pgfqpoint{0.948149in}{1.365614in}}%
\pgfpathlineto{\pgfqpoint{0.955221in}{1.369121in}}%
\pgfpathlineto{\pgfqpoint{0.962506in}{1.372515in}}%
\pgfpathlineto{\pgfqpoint{0.964975in}{1.378746in}}%
\pgfpathlineto{\pgfqpoint{0.967446in}{1.384966in}}%
\pgfpathlineto{\pgfqpoint{0.969920in}{1.391172in}}%
\pgfpathlineto{\pgfqpoint{0.972398in}{1.397361in}}%
\pgfpathlineto{\pgfqpoint{0.965449in}{1.394133in}}%
\pgfpathlineto{\pgfqpoint{0.958704in}{1.390799in}}%
\pgfpathlineto{\pgfqpoint{0.952169in}{1.387362in}}%
\pgfpathlineto{\pgfqpoint{0.945852in}{1.383824in}}%
\pgfpathclose%
\pgfusepath{fill}%
\end{pgfscope}%
\begin{pgfscope}%
\pgfpathrectangle{\pgfqpoint{0.041670in}{0.041670in}}{\pgfqpoint{2.216660in}{2.216660in}}%
\pgfusepath{clip}%
\pgfsetbuttcap%
\pgfsetroundjoin%
\definecolor{currentfill}{rgb}{0.147607,0.511733,0.557049}%
\pgfsetfillcolor{currentfill}%
\pgfsetlinewidth{0.000000pt}%
\definecolor{currentstroke}{rgb}{0.000000,0.000000,0.000000}%
\pgfsetstrokecolor{currentstroke}%
\pgfsetdash{}{0pt}%
\pgfpathmoveto{\pgfqpoint{1.456345in}{1.319773in}}%
\pgfpathlineto{\pgfqpoint{1.459341in}{1.313197in}}%
\pgfpathlineto{\pgfqpoint{1.462334in}{1.306627in}}%
\pgfpathlineto{\pgfqpoint{1.465325in}{1.300064in}}%
\pgfpathlineto{\pgfqpoint{1.468312in}{1.293511in}}%
\pgfpathlineto{\pgfqpoint{1.474463in}{1.288984in}}%
\pgfpathlineto{\pgfqpoint{1.480331in}{1.284360in}}%
\pgfpathlineto{\pgfqpoint{1.485912in}{1.279644in}}%
\pgfpathlineto{\pgfqpoint{1.491200in}{1.274840in}}%
\pgfpathlineto{\pgfqpoint{1.487967in}{1.281605in}}%
\pgfpathlineto{\pgfqpoint{1.484730in}{1.288380in}}%
\pgfpathlineto{\pgfqpoint{1.481491in}{1.295162in}}%
\pgfpathlineto{\pgfqpoint{1.478248in}{1.301948in}}%
\pgfpathlineto{\pgfqpoint{1.473190in}{1.306534in}}%
\pgfpathlineto{\pgfqpoint{1.467849in}{1.311036in}}%
\pgfpathlineto{\pgfqpoint{1.462232in}{1.315450in}}%
\pgfpathlineto{\pgfqpoint{1.456345in}{1.319773in}}%
\pgfpathclose%
\pgfusepath{fill}%
\end{pgfscope}%
\begin{pgfscope}%
\pgfpathrectangle{\pgfqpoint{0.041670in}{0.041670in}}{\pgfqpoint{2.216660in}{2.216660in}}%
\pgfusepath{clip}%
\pgfsetbuttcap%
\pgfsetroundjoin%
\definecolor{currentfill}{rgb}{0.133743,0.548535,0.553541}%
\pgfsetfillcolor{currentfill}%
\pgfsetlinewidth{0.000000pt}%
\definecolor{currentstroke}{rgb}{0.000000,0.000000,0.000000}%
\pgfsetstrokecolor{currentstroke}%
\pgfsetdash{}{0pt}%
\pgfpathmoveto{\pgfqpoint{0.910572in}{1.342409in}}%
\pgfpathlineto{\pgfqpoint{0.907504in}{1.335797in}}%
\pgfpathlineto{\pgfqpoint{0.904439in}{1.329179in}}%
\pgfpathlineto{\pgfqpoint{0.901377in}{1.322558in}}%
\pgfpathlineto{\pgfqpoint{0.898319in}{1.315935in}}%
\pgfpathlineto{\pgfqpoint{0.904236in}{1.320247in}}%
\pgfpathlineto{\pgfqpoint{0.910416in}{1.324463in}}%
\pgfpathlineto{\pgfqpoint{0.916854in}{1.328578in}}%
\pgfpathlineto{\pgfqpoint{0.923542in}{1.332590in}}%
\pgfpathlineto{\pgfqpoint{0.926320in}{1.339016in}}%
\pgfpathlineto{\pgfqpoint{0.929101in}{1.345441in}}%
\pgfpathlineto{\pgfqpoint{0.931885in}{1.351863in}}%
\pgfpathlineto{\pgfqpoint{0.934672in}{1.358278in}}%
\pgfpathlineto{\pgfqpoint{0.928280in}{1.354455in}}%
\pgfpathlineto{\pgfqpoint{0.922129in}{1.350534in}}%
\pgfpathlineto{\pgfqpoint{0.916224in}{1.346517in}}%
\pgfpathlineto{\pgfqpoint{0.910572in}{1.342409in}}%
\pgfpathclose%
\pgfusepath{fill}%
\end{pgfscope}%
\begin{pgfscope}%
\pgfpathrectangle{\pgfqpoint{0.041670in}{0.041670in}}{\pgfqpoint{2.216660in}{2.216660in}}%
\pgfusepath{clip}%
\pgfsetbuttcap%
\pgfsetroundjoin%
\definecolor{currentfill}{rgb}{0.276194,0.190074,0.493001}%
\pgfsetfillcolor{currentfill}%
\pgfsetlinewidth{0.000000pt}%
\definecolor{currentstroke}{rgb}{0.000000,0.000000,0.000000}%
\pgfsetstrokecolor{currentstroke}%
\pgfsetdash{}{0pt}%
\pgfpathmoveto{\pgfqpoint{1.855594in}{0.992635in}}%
\pgfpathlineto{\pgfqpoint{1.859591in}{1.000639in}}%
\pgfpathlineto{\pgfqpoint{1.863603in}{1.009031in}}%
\pgfpathlineto{\pgfqpoint{1.867631in}{1.017819in}}%
\pgfpathlineto{\pgfqpoint{1.871676in}{1.027009in}}%
\pgfpathlineto{\pgfqpoint{1.874291in}{1.015816in}}%
\pgfpathlineto{\pgfqpoint{1.876223in}{1.004569in}}%
\pgfpathlineto{\pgfqpoint{1.877466in}{0.993279in}}%
\pgfpathlineto{\pgfqpoint{1.878017in}{0.981956in}}%
\pgfpathlineto{\pgfqpoint{1.873909in}{0.972973in}}%
\pgfpathlineto{\pgfqpoint{1.869818in}{0.964394in}}%
\pgfpathlineto{\pgfqpoint{1.865744in}{0.956213in}}%
\pgfpathlineto{\pgfqpoint{1.861686in}{0.948423in}}%
\pgfpathlineto{\pgfqpoint{1.861174in}{0.959533in}}%
\pgfpathlineto{\pgfqpoint{1.859984in}{0.970613in}}%
\pgfpathlineto{\pgfqpoint{1.858123in}{0.981651in}}%
\pgfpathlineto{\pgfqpoint{1.855594in}{0.992635in}}%
\pgfpathclose%
\pgfusepath{fill}%
\end{pgfscope}%
\begin{pgfscope}%
\pgfpathrectangle{\pgfqpoint{0.041670in}{0.041670in}}{\pgfqpoint{2.216660in}{2.216660in}}%
\pgfusepath{clip}%
\pgfsetbuttcap%
\pgfsetroundjoin%
\definecolor{currentfill}{rgb}{0.248629,0.278775,0.534556}%
\pgfsetfillcolor{currentfill}%
\pgfsetlinewidth{0.000000pt}%
\definecolor{currentstroke}{rgb}{0.000000,0.000000,0.000000}%
\pgfsetstrokecolor{currentstroke}%
\pgfsetdash{}{0pt}%
\pgfpathmoveto{\pgfqpoint{0.755893in}{1.062908in}}%
\pgfpathlineto{\pgfqpoint{0.752253in}{1.056266in}}%
\pgfpathlineto{\pgfqpoint{0.748614in}{1.049701in}}%
\pgfpathlineto{\pgfqpoint{0.744977in}{1.043219in}}%
\pgfpathlineto{\pgfqpoint{0.741340in}{1.036821in}}%
\pgfpathlineto{\pgfqpoint{0.742786in}{1.043874in}}%
\pgfpathlineto{\pgfqpoint{0.744663in}{1.050894in}}%
\pgfpathlineto{\pgfqpoint{0.746968in}{1.057872in}}%
\pgfpathlineto{\pgfqpoint{0.749698in}{1.064804in}}%
\pgfpathlineto{\pgfqpoint{0.753252in}{1.070945in}}%
\pgfpathlineto{\pgfqpoint{0.756806in}{1.077171in}}%
\pgfpathlineto{\pgfqpoint{0.760362in}{1.083479in}}%
\pgfpathlineto{\pgfqpoint{0.763919in}{1.089866in}}%
\pgfpathlineto{\pgfqpoint{0.761293in}{1.083187in}}%
\pgfpathlineto{\pgfqpoint{0.759078in}{1.076464in}}%
\pgfpathlineto{\pgfqpoint{0.757277in}{1.069702in}}%
\pgfpathlineto{\pgfqpoint{0.755893in}{1.062908in}}%
\pgfpathclose%
\pgfusepath{fill}%
\end{pgfscope}%
\begin{pgfscope}%
\pgfpathrectangle{\pgfqpoint{0.041670in}{0.041670in}}{\pgfqpoint{2.216660in}{2.216660in}}%
\pgfusepath{clip}%
\pgfsetbuttcap%
\pgfsetroundjoin%
\definecolor{currentfill}{rgb}{0.120081,0.622161,0.534946}%
\pgfsetfillcolor{currentfill}%
\pgfsetlinewidth{0.000000pt}%
\definecolor{currentstroke}{rgb}{0.000000,0.000000,0.000000}%
\pgfsetstrokecolor{currentstroke}%
\pgfsetdash{}{0pt}%
\pgfpathmoveto{\pgfqpoint{0.982338in}{1.421898in}}%
\pgfpathlineto{\pgfqpoint{0.979848in}{1.415802in}}%
\pgfpathlineto{\pgfqpoint{0.977362in}{1.409679in}}%
\pgfpathlineto{\pgfqpoint{0.974878in}{1.403531in}}%
\pgfpathlineto{\pgfqpoint{0.972398in}{1.397361in}}%
\pgfpathlineto{\pgfqpoint{0.979544in}{1.400480in}}%
\pgfpathlineto{\pgfqpoint{0.986879in}{1.403486in}}%
\pgfpathlineto{\pgfqpoint{0.994398in}{1.406378in}}%
\pgfpathlineto{\pgfqpoint{1.002092in}{1.409152in}}%
\pgfpathlineto{\pgfqpoint{1.004213in}{1.415171in}}%
\pgfpathlineto{\pgfqpoint{1.006338in}{1.421167in}}%
\pgfpathlineto{\pgfqpoint{1.008465in}{1.427139in}}%
\pgfpathlineto{\pgfqpoint{1.010595in}{1.433084in}}%
\pgfpathlineto{\pgfqpoint{1.003273in}{1.430452in}}%
\pgfpathlineto{\pgfqpoint{0.996118in}{1.427708in}}%
\pgfpathlineto{\pgfqpoint{0.989137in}{1.424856in}}%
\pgfpathlineto{\pgfqpoint{0.982338in}{1.421898in}}%
\pgfpathclose%
\pgfusepath{fill}%
\end{pgfscope}%
\begin{pgfscope}%
\pgfpathrectangle{\pgfqpoint{0.041670in}{0.041670in}}{\pgfqpoint{2.216660in}{2.216660in}}%
\pgfusepath{clip}%
\pgfsetbuttcap%
\pgfsetroundjoin%
\definecolor{currentfill}{rgb}{0.166383,0.690856,0.496502}%
\pgfsetfillcolor{currentfill}%
\pgfsetlinewidth{0.000000pt}%
\definecolor{currentstroke}{rgb}{0.000000,0.000000,0.000000}%
\pgfsetstrokecolor{currentstroke}%
\pgfsetdash{}{0pt}%
\pgfpathmoveto{\pgfqpoint{1.147957in}{1.502060in}}%
\pgfpathlineto{\pgfqpoint{1.147506in}{1.496758in}}%
\pgfpathlineto{\pgfqpoint{1.147056in}{1.491409in}}%
\pgfpathlineto{\pgfqpoint{1.146607in}{1.486017in}}%
\pgfpathlineto{\pgfqpoint{1.146158in}{1.480584in}}%
\pgfpathlineto{\pgfqpoint{1.154699in}{1.481030in}}%
\pgfpathlineto{\pgfqpoint{1.163264in}{1.481346in}}%
\pgfpathlineto{\pgfqpoint{1.171844in}{1.481534in}}%
\pgfpathlineto{\pgfqpoint{1.180432in}{1.481591in}}%
\pgfpathlineto{\pgfqpoint{1.180426in}{1.487011in}}%
\pgfpathlineto{\pgfqpoint{1.180419in}{1.492388in}}%
\pgfpathlineto{\pgfqpoint{1.180413in}{1.497723in}}%
\pgfpathlineto{\pgfqpoint{1.180407in}{1.503011in}}%
\pgfpathlineto{\pgfqpoint{1.172276in}{1.502956in}}%
\pgfpathlineto{\pgfqpoint{1.164152in}{1.502779in}}%
\pgfpathlineto{\pgfqpoint{1.156044in}{1.502481in}}%
\pgfpathlineto{\pgfqpoint{1.147957in}{1.502060in}}%
\pgfpathclose%
\pgfusepath{fill}%
\end{pgfscope}%
\begin{pgfscope}%
\pgfpathrectangle{\pgfqpoint{0.041670in}{0.041670in}}{\pgfqpoint{2.216660in}{2.216660in}}%
\pgfusepath{clip}%
\pgfsetbuttcap%
\pgfsetroundjoin%
\definecolor{currentfill}{rgb}{0.166383,0.690856,0.496502}%
\pgfsetfillcolor{currentfill}%
\pgfsetlinewidth{0.000000pt}%
\definecolor{currentstroke}{rgb}{0.000000,0.000000,0.000000}%
\pgfsetstrokecolor{currentstroke}%
\pgfsetdash{}{0pt}%
\pgfpathmoveto{\pgfqpoint{1.180407in}{1.503011in}}%
\pgfpathlineto{\pgfqpoint{1.180413in}{1.497723in}}%
\pgfpathlineto{\pgfqpoint{1.180419in}{1.492388in}}%
\pgfpathlineto{\pgfqpoint{1.180426in}{1.487011in}}%
\pgfpathlineto{\pgfqpoint{1.180432in}{1.481591in}}%
\pgfpathlineto{\pgfqpoint{1.189020in}{1.481519in}}%
\pgfpathlineto{\pgfqpoint{1.197599in}{1.481318in}}%
\pgfpathlineto{\pgfqpoint{1.206161in}{1.480987in}}%
\pgfpathlineto{\pgfqpoint{1.214699in}{1.480526in}}%
\pgfpathlineto{\pgfqpoint{1.214238in}{1.485961in}}%
\pgfpathlineto{\pgfqpoint{1.213775in}{1.491353in}}%
\pgfpathlineto{\pgfqpoint{1.213313in}{1.496703in}}%
\pgfpathlineto{\pgfqpoint{1.212849in}{1.502006in}}%
\pgfpathlineto{\pgfqpoint{1.204766in}{1.502440in}}%
\pgfpathlineto{\pgfqpoint{1.196659in}{1.502752in}}%
\pgfpathlineto{\pgfqpoint{1.188537in}{1.502943in}}%
\pgfpathlineto{\pgfqpoint{1.180407in}{1.503011in}}%
\pgfpathclose%
\pgfusepath{fill}%
\end{pgfscope}%
\begin{pgfscope}%
\pgfpathrectangle{\pgfqpoint{0.041670in}{0.041670in}}{\pgfqpoint{2.216660in}{2.216660in}}%
\pgfusepath{clip}%
\pgfsetbuttcap%
\pgfsetroundjoin%
\definecolor{currentfill}{rgb}{0.195860,0.395433,0.555276}%
\pgfsetfillcolor{currentfill}%
\pgfsetlinewidth{0.000000pt}%
\definecolor{currentstroke}{rgb}{0.000000,0.000000,0.000000}%
\pgfsetstrokecolor{currentstroke}%
\pgfsetdash{}{0pt}%
\pgfpathmoveto{\pgfqpoint{1.536646in}{1.199603in}}%
\pgfpathlineto{\pgfqpoint{1.540052in}{1.192831in}}%
\pgfpathlineto{\pgfqpoint{1.543456in}{1.186100in}}%
\pgfpathlineto{\pgfqpoint{1.546857in}{1.179411in}}%
\pgfpathlineto{\pgfqpoint{1.550256in}{1.172768in}}%
\pgfpathlineto{\pgfqpoint{1.554494in}{1.166879in}}%
\pgfpathlineto{\pgfqpoint{1.558369in}{1.160921in}}%
\pgfpathlineto{\pgfqpoint{1.561876in}{1.154900in}}%
\pgfpathlineto{\pgfqpoint{1.565010in}{1.148821in}}%
\pgfpathlineto{\pgfqpoint{1.561464in}{1.155707in}}%
\pgfpathlineto{\pgfqpoint{1.557915in}{1.162640in}}%
\pgfpathlineto{\pgfqpoint{1.554364in}{1.169614in}}%
\pgfpathlineto{\pgfqpoint{1.550811in}{1.176628in}}%
\pgfpathlineto{\pgfqpoint{1.547804in}{1.182460in}}%
\pgfpathlineto{\pgfqpoint{1.544438in}{1.188237in}}%
\pgfpathlineto{\pgfqpoint{1.540717in}{1.193953in}}%
\pgfpathlineto{\pgfqpoint{1.536646in}{1.199603in}}%
\pgfpathclose%
\pgfusepath{fill}%
\end{pgfscope}%
\begin{pgfscope}%
\pgfpathrectangle{\pgfqpoint{0.041670in}{0.041670in}}{\pgfqpoint{2.216660in}{2.216660in}}%
\pgfusepath{clip}%
\pgfsetbuttcap%
\pgfsetroundjoin%
\definecolor{currentfill}{rgb}{0.267004,0.004874,0.329415}%
\pgfsetfillcolor{currentfill}%
\pgfsetlinewidth{0.000000pt}%
\definecolor{currentstroke}{rgb}{0.000000,0.000000,0.000000}%
\pgfsetstrokecolor{currentstroke}%
\pgfsetdash{}{0pt}%
\pgfpathmoveto{\pgfqpoint{1.753048in}{0.864779in}}%
\pgfpathlineto{\pgfqpoint{1.756805in}{0.863981in}}%
\pgfpathlineto{\pgfqpoint{1.760568in}{0.863420in}}%
\pgfpathlineto{\pgfqpoint{1.764338in}{0.863101in}}%
\pgfpathlineto{\pgfqpoint{1.768115in}{0.863028in}}%
\pgfpathlineto{\pgfqpoint{1.767845in}{0.853312in}}%
\pgfpathlineto{\pgfqpoint{1.766985in}{0.843593in}}%
\pgfpathlineto{\pgfqpoint{1.765534in}{0.833879in}}%
\pgfpathlineto{\pgfqpoint{1.763492in}{0.824182in}}%
\pgfpathlineto{\pgfqpoint{1.759724in}{0.824506in}}%
\pgfpathlineto{\pgfqpoint{1.755964in}{0.825078in}}%
\pgfpathlineto{\pgfqpoint{1.752211in}{0.825892in}}%
\pgfpathlineto{\pgfqpoint{1.748465in}{0.826943in}}%
\pgfpathlineto{\pgfqpoint{1.750474in}{0.836387in}}%
\pgfpathlineto{\pgfqpoint{1.751908in}{0.845847in}}%
\pgfpathlineto{\pgfqpoint{1.752765in}{0.855315in}}%
\pgfpathlineto{\pgfqpoint{1.753048in}{0.864779in}}%
\pgfpathclose%
\pgfusepath{fill}%
\end{pgfscope}%
\begin{pgfscope}%
\pgfpathrectangle{\pgfqpoint{0.041670in}{0.041670in}}{\pgfqpoint{2.216660in}{2.216660in}}%
\pgfusepath{clip}%
\pgfsetbuttcap%
\pgfsetroundjoin%
\definecolor{currentfill}{rgb}{0.268510,0.009605,0.335427}%
\pgfsetfillcolor{currentfill}%
\pgfsetlinewidth{0.000000pt}%
\definecolor{currentstroke}{rgb}{0.000000,0.000000,0.000000}%
\pgfsetstrokecolor{currentstroke}%
\pgfsetdash{}{0pt}%
\pgfpathmoveto{\pgfqpoint{1.738082in}{0.870243in}}%
\pgfpathlineto{\pgfqpoint{1.741815in}{0.868545in}}%
\pgfpathlineto{\pgfqpoint{1.745554in}{0.867065in}}%
\pgfpathlineto{\pgfqpoint{1.749298in}{0.865808in}}%
\pgfpathlineto{\pgfqpoint{1.753048in}{0.864779in}}%
\pgfpathlineto{\pgfqpoint{1.752765in}{0.855315in}}%
\pgfpathlineto{\pgfqpoint{1.751908in}{0.845847in}}%
\pgfpathlineto{\pgfqpoint{1.750474in}{0.836387in}}%
\pgfpathlineto{\pgfqpoint{1.748465in}{0.826943in}}%
\pgfpathlineto{\pgfqpoint{1.744725in}{0.828226in}}%
\pgfpathlineto{\pgfqpoint{1.740992in}{0.829738in}}%
\pgfpathlineto{\pgfqpoint{1.737265in}{0.831474in}}%
\pgfpathlineto{\pgfqpoint{1.733543in}{0.833428in}}%
\pgfpathlineto{\pgfqpoint{1.735519in}{0.842617in}}%
\pgfpathlineto{\pgfqpoint{1.736933in}{0.851822in}}%
\pgfpathlineto{\pgfqpoint{1.737787in}{0.861034in}}%
\pgfpathlineto{\pgfqpoint{1.738082in}{0.870243in}}%
\pgfpathclose%
\pgfusepath{fill}%
\end{pgfscope}%
\begin{pgfscope}%
\pgfpathrectangle{\pgfqpoint{0.041670in}{0.041670in}}{\pgfqpoint{2.216660in}{2.216660in}}%
\pgfusepath{clip}%
\pgfsetbuttcap%
\pgfsetroundjoin%
\definecolor{currentfill}{rgb}{0.147607,0.511733,0.557049}%
\pgfsetfillcolor{currentfill}%
\pgfsetlinewidth{0.000000pt}%
\definecolor{currentstroke}{rgb}{0.000000,0.000000,0.000000}%
\pgfsetstrokecolor{currentstroke}%
\pgfsetdash{}{0pt}%
\pgfpathmoveto{\pgfqpoint{0.877406in}{1.297805in}}%
\pgfpathlineto{\pgfqpoint{0.874115in}{1.290970in}}%
\pgfpathlineto{\pgfqpoint{0.870827in}{1.284138in}}%
\pgfpathlineto{\pgfqpoint{0.867543in}{1.277314in}}%
\pgfpathlineto{\pgfqpoint{0.864261in}{1.270500in}}%
\pgfpathlineto{\pgfqpoint{0.869283in}{1.275378in}}%
\pgfpathlineto{\pgfqpoint{0.874603in}{1.280173in}}%
\pgfpathlineto{\pgfqpoint{0.880216in}{1.284878in}}%
\pgfpathlineto{\pgfqpoint{0.886117in}{1.289491in}}%
\pgfpathlineto{\pgfqpoint{0.889163in}{1.296090in}}%
\pgfpathlineto{\pgfqpoint{0.892212in}{1.302698in}}%
\pgfpathlineto{\pgfqpoint{0.895264in}{1.309315in}}%
\pgfpathlineto{\pgfqpoint{0.898319in}{1.315935in}}%
\pgfpathlineto{\pgfqpoint{0.892671in}{1.311531in}}%
\pgfpathlineto{\pgfqpoint{0.887300in}{1.307038in}}%
\pgfpathlineto{\pgfqpoint{0.882209in}{1.302462in}}%
\pgfpathlineto{\pgfqpoint{0.877406in}{1.297805in}}%
\pgfpathclose%
\pgfusepath{fill}%
\end{pgfscope}%
\begin{pgfscope}%
\pgfpathrectangle{\pgfqpoint{0.041670in}{0.041670in}}{\pgfqpoint{2.216660in}{2.216660in}}%
\pgfusepath{clip}%
\pgfsetbuttcap%
\pgfsetroundjoin%
\definecolor{currentfill}{rgb}{0.267004,0.004874,0.329415}%
\pgfsetfillcolor{currentfill}%
\pgfsetlinewidth{0.000000pt}%
\definecolor{currentstroke}{rgb}{0.000000,0.000000,0.000000}%
\pgfsetstrokecolor{currentstroke}%
\pgfsetdash{}{0pt}%
\pgfpathmoveto{\pgfqpoint{1.768115in}{0.863028in}}%
\pgfpathlineto{\pgfqpoint{1.771900in}{0.863205in}}%
\pgfpathlineto{\pgfqpoint{1.775692in}{0.863639in}}%
\pgfpathlineto{\pgfqpoint{1.779492in}{0.864333in}}%
\pgfpathlineto{\pgfqpoint{1.783300in}{0.865292in}}%
\pgfpathlineto{\pgfqpoint{1.783043in}{0.855329in}}%
\pgfpathlineto{\pgfqpoint{1.782182in}{0.845360in}}%
\pgfpathlineto{\pgfqpoint{1.780715in}{0.835398in}}%
\pgfpathlineto{\pgfqpoint{1.778640in}{0.825451in}}%
\pgfpathlineto{\pgfqpoint{1.774840in}{0.824739in}}%
\pgfpathlineto{\pgfqpoint{1.771049in}{0.824293in}}%
\pgfpathlineto{\pgfqpoint{1.767267in}{0.824109in}}%
\pgfpathlineto{\pgfqpoint{1.763492in}{0.824182in}}%
\pgfpathlineto{\pgfqpoint{1.765534in}{0.833879in}}%
\pgfpathlineto{\pgfqpoint{1.766985in}{0.843593in}}%
\pgfpathlineto{\pgfqpoint{1.767845in}{0.853312in}}%
\pgfpathlineto{\pgfqpoint{1.768115in}{0.863028in}}%
\pgfpathclose%
\pgfusepath{fill}%
\end{pgfscope}%
\begin{pgfscope}%
\pgfpathrectangle{\pgfqpoint{0.041670in}{0.041670in}}{\pgfqpoint{2.216660in}{2.216660in}}%
\pgfusepath{clip}%
\pgfsetbuttcap%
\pgfsetroundjoin%
\definecolor{currentfill}{rgb}{0.280255,0.165693,0.476498}%
\pgfsetfillcolor{currentfill}%
\pgfsetlinewidth{0.000000pt}%
\definecolor{currentstroke}{rgb}{0.000000,0.000000,0.000000}%
\pgfsetstrokecolor{currentstroke}%
\pgfsetdash{}{0pt}%
\pgfpathmoveto{\pgfqpoint{0.710611in}{0.958610in}}%
\pgfpathlineto{\pgfqpoint{0.706946in}{0.952873in}}%
\pgfpathlineto{\pgfqpoint{0.703280in}{0.947257in}}%
\pgfpathlineto{\pgfqpoint{0.699614in}{0.941765in}}%
\pgfpathlineto{\pgfqpoint{0.695946in}{0.936402in}}%
\pgfpathlineto{\pgfqpoint{0.695662in}{0.944310in}}%
\pgfpathlineto{\pgfqpoint{0.695862in}{0.952211in}}%
\pgfpathlineto{\pgfqpoint{0.696543in}{0.960096in}}%
\pgfpathlineto{\pgfqpoint{0.697704in}{0.967957in}}%
\pgfpathlineto{\pgfqpoint{0.701343in}{0.973059in}}%
\pgfpathlineto{\pgfqpoint{0.704980in}{0.978289in}}%
\pgfpathlineto{\pgfqpoint{0.708617in}{0.983643in}}%
\pgfpathlineto{\pgfqpoint{0.712254in}{0.989118in}}%
\pgfpathlineto{\pgfqpoint{0.711144in}{0.981517in}}%
\pgfpathlineto{\pgfqpoint{0.710498in}{0.973894in}}%
\pgfpathlineto{\pgfqpoint{0.710320in}{0.966256in}}%
\pgfpathlineto{\pgfqpoint{0.710611in}{0.958610in}}%
\pgfpathclose%
\pgfusepath{fill}%
\end{pgfscope}%
\begin{pgfscope}%
\pgfpathrectangle{\pgfqpoint{0.041670in}{0.041670in}}{\pgfqpoint{2.216660in}{2.216660in}}%
\pgfusepath{clip}%
\pgfsetbuttcap%
\pgfsetroundjoin%
\definecolor{currentfill}{rgb}{0.166383,0.690856,0.496502}%
\pgfsetfillcolor{currentfill}%
\pgfsetlinewidth{0.000000pt}%
\definecolor{currentstroke}{rgb}{0.000000,0.000000,0.000000}%
\pgfsetstrokecolor{currentstroke}%
\pgfsetdash{}{0pt}%
\pgfpathmoveto{\pgfqpoint{1.115990in}{1.499167in}}%
\pgfpathlineto{\pgfqpoint{1.115088in}{1.493821in}}%
\pgfpathlineto{\pgfqpoint{1.114188in}{1.488430in}}%
\pgfpathlineto{\pgfqpoint{1.113288in}{1.482995in}}%
\pgfpathlineto{\pgfqpoint{1.112391in}{1.477518in}}%
\pgfpathlineto{\pgfqpoint{1.120757in}{1.478476in}}%
\pgfpathlineto{\pgfqpoint{1.129179in}{1.479306in}}%
\pgfpathlineto{\pgfqpoint{1.137648in}{1.480009in}}%
\pgfpathlineto{\pgfqpoint{1.146158in}{1.480584in}}%
\pgfpathlineto{\pgfqpoint{1.146607in}{1.486017in}}%
\pgfpathlineto{\pgfqpoint{1.147056in}{1.491409in}}%
\pgfpathlineto{\pgfqpoint{1.147506in}{1.496758in}}%
\pgfpathlineto{\pgfqpoint{1.147957in}{1.502060in}}%
\pgfpathlineto{\pgfqpoint{1.139901in}{1.501518in}}%
\pgfpathlineto{\pgfqpoint{1.131883in}{1.500854in}}%
\pgfpathlineto{\pgfqpoint{1.123910in}{1.500070in}}%
\pgfpathlineto{\pgfqpoint{1.115990in}{1.499167in}}%
\pgfpathclose%
\pgfusepath{fill}%
\end{pgfscope}%
\begin{pgfscope}%
\pgfpathrectangle{\pgfqpoint{0.041670in}{0.041670in}}{\pgfqpoint{2.216660in}{2.216660in}}%
\pgfusepath{clip}%
\pgfsetbuttcap%
\pgfsetroundjoin%
\definecolor{currentfill}{rgb}{0.166383,0.690856,0.496502}%
\pgfsetfillcolor{currentfill}%
\pgfsetlinewidth{0.000000pt}%
\definecolor{currentstroke}{rgb}{0.000000,0.000000,0.000000}%
\pgfsetstrokecolor{currentstroke}%
\pgfsetdash{}{0pt}%
\pgfpathmoveto{\pgfqpoint{1.212849in}{1.502006in}}%
\pgfpathlineto{\pgfqpoint{1.213313in}{1.496703in}}%
\pgfpathlineto{\pgfqpoint{1.213775in}{1.491353in}}%
\pgfpathlineto{\pgfqpoint{1.214238in}{1.485961in}}%
\pgfpathlineto{\pgfqpoint{1.214699in}{1.480526in}}%
\pgfpathlineto{\pgfqpoint{1.223204in}{1.479937in}}%
\pgfpathlineto{\pgfqpoint{1.231669in}{1.479220in}}%
\pgfpathlineto{\pgfqpoint{1.240085in}{1.478375in}}%
\pgfpathlineto{\pgfqpoint{1.248445in}{1.477404in}}%
\pgfpathlineto{\pgfqpoint{1.247535in}{1.482882in}}%
\pgfpathlineto{\pgfqpoint{1.246623in}{1.488319in}}%
\pgfpathlineto{\pgfqpoint{1.245711in}{1.493712in}}%
\pgfpathlineto{\pgfqpoint{1.244797in}{1.499059in}}%
\pgfpathlineto{\pgfqpoint{1.236883in}{1.499976in}}%
\pgfpathlineto{\pgfqpoint{1.228915in}{1.500773in}}%
\pgfpathlineto{\pgfqpoint{1.220901in}{1.501450in}}%
\pgfpathlineto{\pgfqpoint{1.212849in}{1.502006in}}%
\pgfpathclose%
\pgfusepath{fill}%
\end{pgfscope}%
\begin{pgfscope}%
\pgfpathrectangle{\pgfqpoint{0.041670in}{0.041670in}}{\pgfqpoint{2.216660in}{2.216660in}}%
\pgfusepath{clip}%
\pgfsetbuttcap%
\pgfsetroundjoin%
\definecolor{currentfill}{rgb}{0.134692,0.658636,0.517649}%
\pgfsetfillcolor{currentfill}%
\pgfsetlinewidth{0.000000pt}%
\definecolor{currentstroke}{rgb}{0.000000,0.000000,0.000000}%
\pgfsetstrokecolor{currentstroke}%
\pgfsetdash{}{0pt}%
\pgfpathmoveto{\pgfqpoint{1.304726in}{1.467137in}}%
\pgfpathlineto{\pgfqpoint{1.306384in}{1.461475in}}%
\pgfpathlineto{\pgfqpoint{1.308040in}{1.455776in}}%
\pgfpathlineto{\pgfqpoint{1.309694in}{1.450042in}}%
\pgfpathlineto{\pgfqpoint{1.311346in}{1.444277in}}%
\pgfpathlineto{\pgfqpoint{1.319382in}{1.442221in}}%
\pgfpathlineto{\pgfqpoint{1.327287in}{1.440042in}}%
\pgfpathlineto{\pgfqpoint{1.335052in}{1.437744in}}%
\pgfpathlineto{\pgfqpoint{1.342671in}{1.435329in}}%
\pgfpathlineto{\pgfqpoint{1.340623in}{1.441214in}}%
\pgfpathlineto{\pgfqpoint{1.338572in}{1.447068in}}%
\pgfpathlineto{\pgfqpoint{1.336518in}{1.452887in}}%
\pgfpathlineto{\pgfqpoint{1.334462in}{1.458669in}}%
\pgfpathlineto{\pgfqpoint{1.327230in}{1.460955in}}%
\pgfpathlineto{\pgfqpoint{1.319858in}{1.463130in}}%
\pgfpathlineto{\pgfqpoint{1.312355in}{1.465191in}}%
\pgfpathlineto{\pgfqpoint{1.304726in}{1.467137in}}%
\pgfpathclose%
\pgfusepath{fill}%
\end{pgfscope}%
\begin{pgfscope}%
\pgfpathrectangle{\pgfqpoint{0.041670in}{0.041670in}}{\pgfqpoint{2.216660in}{2.216660in}}%
\pgfusepath{clip}%
\pgfsetbuttcap%
\pgfsetroundjoin%
\definecolor{currentfill}{rgb}{0.271305,0.019942,0.347269}%
\pgfsetfillcolor{currentfill}%
\pgfsetlinewidth{0.000000pt}%
\definecolor{currentstroke}{rgb}{0.000000,0.000000,0.000000}%
\pgfsetstrokecolor{currentstroke}%
\pgfsetdash{}{0pt}%
\pgfpathmoveto{\pgfqpoint{1.723201in}{0.879130in}}%
\pgfpathlineto{\pgfqpoint{1.726914in}{0.876603in}}%
\pgfpathlineto{\pgfqpoint{1.730632in}{0.874277in}}%
\pgfpathlineto{\pgfqpoint{1.734354in}{0.872155in}}%
\pgfpathlineto{\pgfqpoint{1.738082in}{0.870243in}}%
\pgfpathlineto{\pgfqpoint{1.737787in}{0.861034in}}%
\pgfpathlineto{\pgfqpoint{1.736933in}{0.851822in}}%
\pgfpathlineto{\pgfqpoint{1.735519in}{0.842617in}}%
\pgfpathlineto{\pgfqpoint{1.733543in}{0.833428in}}%
\pgfpathlineto{\pgfqpoint{1.729827in}{0.835597in}}%
\pgfpathlineto{\pgfqpoint{1.726116in}{0.837976in}}%
\pgfpathlineto{\pgfqpoint{1.722410in}{0.840560in}}%
\pgfpathlineto{\pgfqpoint{1.718709in}{0.843345in}}%
\pgfpathlineto{\pgfqpoint{1.720650in}{0.852276in}}%
\pgfpathlineto{\pgfqpoint{1.722045in}{0.861223in}}%
\pgfpathlineto{\pgfqpoint{1.722895in}{0.870177in}}%
\pgfpathlineto{\pgfqpoint{1.723201in}{0.879130in}}%
\pgfpathclose%
\pgfusepath{fill}%
\end{pgfscope}%
\begin{pgfscope}%
\pgfpathrectangle{\pgfqpoint{0.041670in}{0.041670in}}{\pgfqpoint{2.216660in}{2.216660in}}%
\pgfusepath{clip}%
\pgfsetbuttcap%
\pgfsetroundjoin%
\definecolor{currentfill}{rgb}{0.274128,0.199721,0.498911}%
\pgfsetfillcolor{currentfill}%
\pgfsetlinewidth{0.000000pt}%
\definecolor{currentstroke}{rgb}{0.000000,0.000000,0.000000}%
\pgfsetstrokecolor{currentstroke}%
\pgfsetdash{}{0pt}%
\pgfpathmoveto{\pgfqpoint{1.631795in}{1.018651in}}%
\pgfpathlineto{\pgfqpoint{1.635416in}{1.012788in}}%
\pgfpathlineto{\pgfqpoint{1.639037in}{1.007031in}}%
\pgfpathlineto{\pgfqpoint{1.642659in}{1.001384in}}%
\pgfpathlineto{\pgfqpoint{1.646281in}{0.995850in}}%
\pgfpathlineto{\pgfqpoint{1.647802in}{0.988275in}}%
\pgfpathlineto{\pgfqpoint{1.648861in}{0.980671in}}%
\pgfpathlineto{\pgfqpoint{1.649454in}{0.973046in}}%
\pgfpathlineto{\pgfqpoint{1.649580in}{0.965406in}}%
\pgfpathlineto{\pgfqpoint{1.645917in}{0.971201in}}%
\pgfpathlineto{\pgfqpoint{1.642255in}{0.977110in}}%
\pgfpathlineto{\pgfqpoint{1.638593in}{0.983128in}}%
\pgfpathlineto{\pgfqpoint{1.634931in}{0.989252in}}%
\pgfpathlineto{\pgfqpoint{1.634824in}{0.996629in}}%
\pgfpathlineto{\pgfqpoint{1.634264in}{1.003992in}}%
\pgfpathlineto{\pgfqpoint{1.633254in}{1.011335in}}%
\pgfpathlineto{\pgfqpoint{1.631795in}{1.018651in}}%
\pgfpathclose%
\pgfusepath{fill}%
\end{pgfscope}%
\begin{pgfscope}%
\pgfpathrectangle{\pgfqpoint{0.041670in}{0.041670in}}{\pgfqpoint{2.216660in}{2.216660in}}%
\pgfusepath{clip}%
\pgfsetbuttcap%
\pgfsetroundjoin%
\definecolor{currentfill}{rgb}{0.231674,0.318106,0.544834}%
\pgfsetfillcolor{currentfill}%
\pgfsetlinewidth{0.000000pt}%
\definecolor{currentstroke}{rgb}{0.000000,0.000000,0.000000}%
\pgfsetstrokecolor{currentstroke}%
\pgfsetdash{}{0pt}%
\pgfpathmoveto{\pgfqpoint{1.579176in}{1.121796in}}%
\pgfpathlineto{\pgfqpoint{1.582713in}{1.115186in}}%
\pgfpathlineto{\pgfqpoint{1.586249in}{1.108641in}}%
\pgfpathlineto{\pgfqpoint{1.589782in}{1.102164in}}%
\pgfpathlineto{\pgfqpoint{1.593315in}{1.095759in}}%
\pgfpathlineto{\pgfqpoint{1.596303in}{1.089126in}}%
\pgfpathlineto{\pgfqpoint{1.598883in}{1.082442in}}%
\pgfpathlineto{\pgfqpoint{1.601053in}{1.075714in}}%
\pgfpathlineto{\pgfqpoint{1.602807in}{1.068949in}}%
\pgfpathlineto{\pgfqpoint{1.599180in}{1.075608in}}%
\pgfpathlineto{\pgfqpoint{1.595552in}{1.082339in}}%
\pgfpathlineto{\pgfqpoint{1.591922in}{1.089139in}}%
\pgfpathlineto{\pgfqpoint{1.588291in}{1.096003in}}%
\pgfpathlineto{\pgfqpoint{1.586610in}{1.102511in}}%
\pgfpathlineto{\pgfqpoint{1.584528in}{1.108984in}}%
\pgfpathlineto{\pgfqpoint{1.582049in}{1.115414in}}%
\pgfpathlineto{\pgfqpoint{1.579176in}{1.121796in}}%
\pgfpathclose%
\pgfusepath{fill}%
\end{pgfscope}%
\begin{pgfscope}%
\pgfpathrectangle{\pgfqpoint{0.041670in}{0.041670in}}{\pgfqpoint{2.216660in}{2.216660in}}%
\pgfusepath{clip}%
\pgfsetbuttcap%
\pgfsetroundjoin%
\definecolor{currentfill}{rgb}{0.163625,0.471133,0.558148}%
\pgfsetfillcolor{currentfill}%
\pgfsetlinewidth{0.000000pt}%
\definecolor{currentstroke}{rgb}{0.000000,0.000000,0.000000}%
\pgfsetstrokecolor{currentstroke}%
\pgfsetdash{}{0pt}%
\pgfpathmoveto{\pgfqpoint{1.491200in}{1.274840in}}%
\pgfpathlineto{\pgfqpoint{1.494430in}{1.268089in}}%
\pgfpathlineto{\pgfqpoint{1.497657in}{1.261353in}}%
\pgfpathlineto{\pgfqpoint{1.500881in}{1.254636in}}%
\pgfpathlineto{\pgfqpoint{1.504103in}{1.247941in}}%
\pgfpathlineto{\pgfqpoint{1.509307in}{1.242831in}}%
\pgfpathlineto{\pgfqpoint{1.514194in}{1.237638in}}%
\pgfpathlineto{\pgfqpoint{1.518759in}{1.232367in}}%
\pgfpathlineto{\pgfqpoint{1.522996in}{1.227023in}}%
\pgfpathlineto{\pgfqpoint{1.519577in}{1.233947in}}%
\pgfpathlineto{\pgfqpoint{1.516154in}{1.240893in}}%
\pgfpathlineto{\pgfqpoint{1.512729in}{1.247857in}}%
\pgfpathlineto{\pgfqpoint{1.509301in}{1.254837in}}%
\pgfpathlineto{\pgfqpoint{1.505243in}{1.259947in}}%
\pgfpathlineto{\pgfqpoint{1.500871in}{1.264987in}}%
\pgfpathlineto{\pgfqpoint{1.496188in}{1.269953in}}%
\pgfpathlineto{\pgfqpoint{1.491200in}{1.274840in}}%
\pgfpathclose%
\pgfusepath{fill}%
\end{pgfscope}%
\begin{pgfscope}%
\pgfpathrectangle{\pgfqpoint{0.041670in}{0.041670in}}{\pgfqpoint{2.216660in}{2.216660in}}%
\pgfusepath{clip}%
\pgfsetbuttcap%
\pgfsetroundjoin%
\definecolor{currentfill}{rgb}{0.268510,0.009605,0.335427}%
\pgfsetfillcolor{currentfill}%
\pgfsetlinewidth{0.000000pt}%
\definecolor{currentstroke}{rgb}{0.000000,0.000000,0.000000}%
\pgfsetstrokecolor{currentstroke}%
\pgfsetdash{}{0pt}%
\pgfpathmoveto{\pgfqpoint{1.783300in}{0.865292in}}%
\pgfpathlineto{\pgfqpoint{1.787116in}{0.866523in}}%
\pgfpathlineto{\pgfqpoint{1.790942in}{0.868029in}}%
\pgfpathlineto{\pgfqpoint{1.794777in}{0.869816in}}%
\pgfpathlineto{\pgfqpoint{1.798621in}{0.871889in}}%
\pgfpathlineto{\pgfqpoint{1.798379in}{0.861681in}}%
\pgfpathlineto{\pgfqpoint{1.797518in}{0.851468in}}%
\pgfpathlineto{\pgfqpoint{1.796034in}{0.841260in}}%
\pgfpathlineto{\pgfqpoint{1.793928in}{0.831067in}}%
\pgfpathlineto{\pgfqpoint{1.790092in}{0.829238in}}%
\pgfpathlineto{\pgfqpoint{1.786265in}{0.827695in}}%
\pgfpathlineto{\pgfqpoint{1.782448in}{0.826435in}}%
\pgfpathlineto{\pgfqpoint{1.778640in}{0.825451in}}%
\pgfpathlineto{\pgfqpoint{1.780715in}{0.835398in}}%
\pgfpathlineto{\pgfqpoint{1.782182in}{0.845360in}}%
\pgfpathlineto{\pgfqpoint{1.783043in}{0.855329in}}%
\pgfpathlineto{\pgfqpoint{1.783300in}{0.865292in}}%
\pgfpathclose%
\pgfusepath{fill}%
\end{pgfscope}%
\begin{pgfscope}%
\pgfpathrectangle{\pgfqpoint{0.041670in}{0.041670in}}{\pgfqpoint{2.216660in}{2.216660in}}%
\pgfusepath{clip}%
\pgfsetbuttcap%
\pgfsetroundjoin%
\definecolor{currentfill}{rgb}{0.134692,0.658636,0.517649}%
\pgfsetfillcolor{currentfill}%
\pgfsetlinewidth{0.000000pt}%
\definecolor{currentstroke}{rgb}{0.000000,0.000000,0.000000}%
\pgfsetstrokecolor{currentstroke}%
\pgfsetdash{}{0pt}%
\pgfpathmoveto{\pgfqpoint{1.019142in}{1.456545in}}%
\pgfpathlineto{\pgfqpoint{1.017001in}{1.450733in}}%
\pgfpathlineto{\pgfqpoint{1.014863in}{1.444884in}}%
\pgfpathlineto{\pgfqpoint{1.012728in}{1.439000in}}%
\pgfpathlineto{\pgfqpoint{1.010595in}{1.433084in}}%
\pgfpathlineto{\pgfqpoint{1.018078in}{1.435603in}}%
\pgfpathlineto{\pgfqpoint{1.025713in}{1.438006in}}%
\pgfpathlineto{\pgfqpoint{1.033495in}{1.440290in}}%
\pgfpathlineto{\pgfqpoint{1.041414in}{1.442455in}}%
\pgfpathlineto{\pgfqpoint{1.043156in}{1.448245in}}%
\pgfpathlineto{\pgfqpoint{1.044901in}{1.454003in}}%
\pgfpathlineto{\pgfqpoint{1.046648in}{1.459726in}}%
\pgfpathlineto{\pgfqpoint{1.048397in}{1.465413in}}%
\pgfpathlineto{\pgfqpoint{1.040879in}{1.463364in}}%
\pgfpathlineto{\pgfqpoint{1.033492in}{1.461202in}}%
\pgfpathlineto{\pgfqpoint{1.026244in}{1.458929in}}%
\pgfpathlineto{\pgfqpoint{1.019142in}{1.456545in}}%
\pgfpathclose%
\pgfusepath{fill}%
\end{pgfscope}%
\begin{pgfscope}%
\pgfpathrectangle{\pgfqpoint{0.041670in}{0.041670in}}{\pgfqpoint{2.216660in}{2.216660in}}%
\pgfusepath{clip}%
\pgfsetbuttcap%
\pgfsetroundjoin%
\definecolor{currentfill}{rgb}{0.274952,0.037752,0.364543}%
\pgfsetfillcolor{currentfill}%
\pgfsetlinewidth{0.000000pt}%
\definecolor{currentstroke}{rgb}{0.000000,0.000000,0.000000}%
\pgfsetstrokecolor{currentstroke}%
\pgfsetdash{}{0pt}%
\pgfpathmoveto{\pgfqpoint{1.708389in}{0.891159in}}%
\pgfpathlineto{\pgfqpoint{1.712086in}{0.887872in}}%
\pgfpathlineto{\pgfqpoint{1.715787in}{0.884769in}}%
\pgfpathlineto{\pgfqpoint{1.719492in}{0.881854in}}%
\pgfpathlineto{\pgfqpoint{1.723201in}{0.879130in}}%
\pgfpathlineto{\pgfqpoint{1.722895in}{0.870177in}}%
\pgfpathlineto{\pgfqpoint{1.722045in}{0.861223in}}%
\pgfpathlineto{\pgfqpoint{1.720650in}{0.852276in}}%
\pgfpathlineto{\pgfqpoint{1.718709in}{0.843345in}}%
\pgfpathlineto{\pgfqpoint{1.715013in}{0.846328in}}%
\pgfpathlineto{\pgfqpoint{1.711321in}{0.849503in}}%
\pgfpathlineto{\pgfqpoint{1.707633in}{0.852866in}}%
\pgfpathlineto{\pgfqpoint{1.703949in}{0.856412in}}%
\pgfpathlineto{\pgfqpoint{1.705855in}{0.865083in}}%
\pgfpathlineto{\pgfqpoint{1.707230in}{0.873770in}}%
\pgfpathlineto{\pgfqpoint{1.708074in}{0.882465in}}%
\pgfpathlineto{\pgfqpoint{1.708389in}{0.891159in}}%
\pgfpathclose%
\pgfusepath{fill}%
\end{pgfscope}%
\begin{pgfscope}%
\pgfpathrectangle{\pgfqpoint{0.041670in}{0.041670in}}{\pgfqpoint{2.216660in}{2.216660in}}%
\pgfusepath{clip}%
\pgfsetbuttcap%
\pgfsetroundjoin%
\definecolor{currentfill}{rgb}{0.201239,0.383670,0.554294}%
\pgfsetfillcolor{currentfill}%
\pgfsetlinewidth{0.000000pt}%
\definecolor{currentstroke}{rgb}{0.000000,0.000000,0.000000}%
\pgfsetstrokecolor{currentstroke}%
\pgfsetdash{}{0pt}%
\pgfpathmoveto{\pgfqpoint{0.452708in}{1.105542in}}%
\pgfpathlineto{\pgfqpoint{0.448473in}{1.118626in}}%
\pgfpathlineto{\pgfqpoint{0.444216in}{1.132185in}}%
\pgfpathlineto{\pgfqpoint{0.439938in}{1.146227in}}%
\pgfpathlineto{\pgfqpoint{0.435637in}{1.160759in}}%
\pgfpathlineto{\pgfqpoint{0.438606in}{1.172510in}}%
\pgfpathlineto{\pgfqpoint{0.442301in}{1.184188in}}%
\pgfpathlineto{\pgfqpoint{0.446712in}{1.195782in}}%
\pgfpathlineto{\pgfqpoint{0.451832in}{1.207283in}}%
\pgfpathlineto{\pgfqpoint{0.456011in}{1.192580in}}%
\pgfpathlineto{\pgfqpoint{0.460169in}{1.178365in}}%
\pgfpathlineto{\pgfqpoint{0.464306in}{1.164630in}}%
\pgfpathlineto{\pgfqpoint{0.468422in}{1.151367in}}%
\pgfpathlineto{\pgfqpoint{0.463446in}{1.140038in}}%
\pgfpathlineto{\pgfqpoint{0.459163in}{1.128617in}}%
\pgfpathlineto{\pgfqpoint{0.455581in}{1.117115in}}%
\pgfpathlineto{\pgfqpoint{0.452708in}{1.105542in}}%
\pgfpathclose%
\pgfusepath{fill}%
\end{pgfscope}%
\begin{pgfscope}%
\pgfpathrectangle{\pgfqpoint{0.041670in}{0.041670in}}{\pgfqpoint{2.216660in}{2.216660in}}%
\pgfusepath{clip}%
\pgfsetbuttcap%
\pgfsetroundjoin%
\definecolor{currentfill}{rgb}{0.166383,0.690856,0.496502}%
\pgfsetfillcolor{currentfill}%
\pgfsetlinewidth{0.000000pt}%
\definecolor{currentstroke}{rgb}{0.000000,0.000000,0.000000}%
\pgfsetstrokecolor{currentstroke}%
\pgfsetdash{}{0pt}%
\pgfpathmoveto{\pgfqpoint{1.244797in}{1.499059in}}%
\pgfpathlineto{\pgfqpoint{1.245711in}{1.493712in}}%
\pgfpathlineto{\pgfqpoint{1.246623in}{1.488319in}}%
\pgfpathlineto{\pgfqpoint{1.247535in}{1.482882in}}%
\pgfpathlineto{\pgfqpoint{1.248445in}{1.477404in}}%
\pgfpathlineto{\pgfqpoint{1.256740in}{1.476306in}}%
\pgfpathlineto{\pgfqpoint{1.264963in}{1.475084in}}%
\pgfpathlineto{\pgfqpoint{1.273106in}{1.473737in}}%
\pgfpathlineto{\pgfqpoint{1.271868in}{1.479267in}}%
\pgfpathlineto{\pgfqpoint{1.270628in}{1.484756in}}%
\pgfpathlineto{\pgfqpoint{1.269386in}{1.490200in}}%
\pgfpathlineto{\pgfqpoint{1.268142in}{1.495599in}}%
\pgfpathlineto{\pgfqpoint{1.260434in}{1.496870in}}%
\pgfpathlineto{\pgfqpoint{1.252649in}{1.498023in}}%
\pgfpathlineto{\pgfqpoint{1.244797in}{1.499059in}}%
\pgfpathclose%
\pgfusepath{fill}%
\end{pgfscope}%
\begin{pgfscope}%
\pgfpathrectangle{\pgfqpoint{0.041670in}{0.041670in}}{\pgfqpoint{2.216660in}{2.216660in}}%
\pgfusepath{clip}%
\pgfsetbuttcap%
\pgfsetroundjoin%
\definecolor{currentfill}{rgb}{0.195860,0.395433,0.555276}%
\pgfsetfillcolor{currentfill}%
\pgfsetlinewidth{0.000000pt}%
\definecolor{currentstroke}{rgb}{0.000000,0.000000,0.000000}%
\pgfsetstrokecolor{currentstroke}%
\pgfsetdash{}{0pt}%
\pgfpathmoveto{\pgfqpoint{0.806730in}{1.171403in}}%
\pgfpathlineto{\pgfqpoint{0.803151in}{1.164333in}}%
\pgfpathlineto{\pgfqpoint{0.799575in}{1.157303in}}%
\pgfpathlineto{\pgfqpoint{0.796001in}{1.150315in}}%
\pgfpathlineto{\pgfqpoint{0.792429in}{1.143373in}}%
\pgfpathlineto{\pgfqpoint{0.795229in}{1.149499in}}%
\pgfpathlineto{\pgfqpoint{0.798405in}{1.155572in}}%
\pgfpathlineto{\pgfqpoint{0.801953in}{1.161587in}}%
\pgfpathlineto{\pgfqpoint{0.805869in}{1.167537in}}%
\pgfpathlineto{\pgfqpoint{0.809305in}{1.174233in}}%
\pgfpathlineto{\pgfqpoint{0.812744in}{1.180975in}}%
\pgfpathlineto{\pgfqpoint{0.816184in}{1.187760in}}%
\pgfpathlineto{\pgfqpoint{0.819628in}{1.194584in}}%
\pgfpathlineto{\pgfqpoint{0.815868in}{1.188875in}}%
\pgfpathlineto{\pgfqpoint{0.812462in}{1.183105in}}%
\pgfpathlineto{\pgfqpoint{0.809415in}{1.177279in}}%
\pgfpathlineto{\pgfqpoint{0.806730in}{1.171403in}}%
\pgfpathclose%
\pgfusepath{fill}%
\end{pgfscope}%
\begin{pgfscope}%
\pgfpathrectangle{\pgfqpoint{0.041670in}{0.041670in}}{\pgfqpoint{2.216660in}{2.216660in}}%
\pgfusepath{clip}%
\pgfsetbuttcap%
\pgfsetroundjoin%
\definecolor{currentfill}{rgb}{0.166383,0.690856,0.496502}%
\pgfsetfillcolor{currentfill}%
\pgfsetlinewidth{0.000000pt}%
\definecolor{currentstroke}{rgb}{0.000000,0.000000,0.000000}%
\pgfsetstrokecolor{currentstroke}%
\pgfsetdash{}{0pt}%
\pgfpathmoveto{\pgfqpoint{1.084986in}{1.494372in}}%
\pgfpathlineto{\pgfqpoint{1.083646in}{1.488955in}}%
\pgfpathlineto{\pgfqpoint{1.082308in}{1.483492in}}%
\pgfpathlineto{\pgfqpoint{1.080972in}{1.477985in}}%
\pgfpathlineto{\pgfqpoint{1.079638in}{1.472437in}}%
\pgfpathlineto{\pgfqpoint{1.087704in}{1.473893in}}%
\pgfpathlineto{\pgfqpoint{1.095856in}{1.475226in}}%
\pgfpathlineto{\pgfqpoint{1.104088in}{1.476434in}}%
\pgfpathlineto{\pgfqpoint{1.112391in}{1.477518in}}%
\pgfpathlineto{\pgfqpoint{1.113288in}{1.482995in}}%
\pgfpathlineto{\pgfqpoint{1.114188in}{1.488430in}}%
\pgfpathlineto{\pgfqpoint{1.115088in}{1.493821in}}%
\pgfpathlineto{\pgfqpoint{1.115990in}{1.499167in}}%
\pgfpathlineto{\pgfqpoint{1.108130in}{1.498144in}}%
\pgfpathlineto{\pgfqpoint{1.100337in}{1.497004in}}%
\pgfpathlineto{\pgfqpoint{1.092620in}{1.495746in}}%
\pgfpathlineto{\pgfqpoint{1.084986in}{1.494372in}}%
\pgfpathclose%
\pgfusepath{fill}%
\end{pgfscope}%
\begin{pgfscope}%
\pgfpathrectangle{\pgfqpoint{0.041670in}{0.041670in}}{\pgfqpoint{2.216660in}{2.216660in}}%
\pgfusepath{clip}%
\pgfsetbuttcap%
\pgfsetroundjoin%
\definecolor{currentfill}{rgb}{0.276194,0.190074,0.493001}%
\pgfsetfillcolor{currentfill}%
\pgfsetlinewidth{0.000000pt}%
\definecolor{currentstroke}{rgb}{0.000000,0.000000,0.000000}%
\pgfsetstrokecolor{currentstroke}%
\pgfsetdash{}{0pt}%
\pgfpathmoveto{\pgfqpoint{0.498339in}{0.938529in}}%
\pgfpathlineto{\pgfqpoint{0.494276in}{0.946271in}}%
\pgfpathlineto{\pgfqpoint{0.490197in}{0.954405in}}%
\pgfpathlineto{\pgfqpoint{0.486101in}{0.962936in}}%
\pgfpathlineto{\pgfqpoint{0.481988in}{0.971872in}}%
\pgfpathlineto{\pgfqpoint{0.481920in}{0.983215in}}%
\pgfpathlineto{\pgfqpoint{0.482548in}{0.994535in}}%
\pgfpathlineto{\pgfqpoint{0.483868in}{1.005821in}}%
\pgfpathlineto{\pgfqpoint{0.485876in}{1.017063in}}%
\pgfpathlineto{\pgfqpoint{0.489940in}{1.007918in}}%
\pgfpathlineto{\pgfqpoint{0.493988in}{0.999177in}}%
\pgfpathlineto{\pgfqpoint{0.498019in}{0.990831in}}%
\pgfpathlineto{\pgfqpoint{0.502035in}{0.982874in}}%
\pgfpathlineto{\pgfqpoint{0.500099in}{0.971842in}}%
\pgfpathlineto{\pgfqpoint{0.498835in}{0.960766in}}%
\pgfpathlineto{\pgfqpoint{0.498247in}{0.949658in}}%
\pgfpathlineto{\pgfqpoint{0.498339in}{0.938529in}}%
\pgfpathclose%
\pgfusepath{fill}%
\end{pgfscope}%
\begin{pgfscope}%
\pgfpathrectangle{\pgfqpoint{0.041670in}{0.041670in}}{\pgfqpoint{2.216660in}{2.216660in}}%
\pgfusepath{clip}%
\pgfsetbuttcap%
\pgfsetroundjoin%
\definecolor{currentfill}{rgb}{0.272594,0.025563,0.353093}%
\pgfsetfillcolor{currentfill}%
\pgfsetlinewidth{0.000000pt}%
\definecolor{currentstroke}{rgb}{0.000000,0.000000,0.000000}%
\pgfsetstrokecolor{currentstroke}%
\pgfsetdash{}{0pt}%
\pgfpathmoveto{\pgfqpoint{1.798621in}{0.871889in}}%
\pgfpathlineto{\pgfqpoint{1.802474in}{0.874253in}}%
\pgfpathlineto{\pgfqpoint{1.806338in}{0.876914in}}%
\pgfpathlineto{\pgfqpoint{1.810213in}{0.879876in}}%
\pgfpathlineto{\pgfqpoint{1.814098in}{0.883146in}}%
\pgfpathlineto{\pgfqpoint{1.813873in}{0.872699in}}%
\pgfpathlineto{\pgfqpoint{1.813012in}{0.862246in}}%
\pgfpathlineto{\pgfqpoint{1.811514in}{0.851797in}}%
\pgfpathlineto{\pgfqpoint{1.809377in}{0.841362in}}%
\pgfpathlineto{\pgfqpoint{1.805498in}{0.838331in}}%
\pgfpathlineto{\pgfqpoint{1.801631in}{0.835609in}}%
\pgfpathlineto{\pgfqpoint{1.797775in}{0.833189in}}%
\pgfpathlineto{\pgfqpoint{1.793928in}{0.831067in}}%
\pgfpathlineto{\pgfqpoint{1.796034in}{0.841260in}}%
\pgfpathlineto{\pgfqpoint{1.797518in}{0.851468in}}%
\pgfpathlineto{\pgfqpoint{1.798379in}{0.861681in}}%
\pgfpathlineto{\pgfqpoint{1.798621in}{0.871889in}}%
\pgfpathclose%
\pgfusepath{fill}%
\end{pgfscope}%
\begin{pgfscope}%
\pgfpathrectangle{\pgfqpoint{0.041670in}{0.041670in}}{\pgfqpoint{2.216660in}{2.216660in}}%
\pgfusepath{clip}%
\pgfsetbuttcap%
\pgfsetroundjoin%
\definecolor{currentfill}{rgb}{0.163625,0.471133,0.558148}%
\pgfsetfillcolor{currentfill}%
\pgfsetlinewidth{0.000000pt}%
\definecolor{currentstroke}{rgb}{0.000000,0.000000,0.000000}%
\pgfsetstrokecolor{currentstroke}%
\pgfsetdash{}{0pt}%
\pgfpathmoveto{\pgfqpoint{0.847270in}{1.250240in}}%
\pgfpathlineto{\pgfqpoint{0.843805in}{1.243208in}}%
\pgfpathlineto{\pgfqpoint{0.840343in}{1.236191in}}%
\pgfpathlineto{\pgfqpoint{0.836883in}{1.229193in}}%
\pgfpathlineto{\pgfqpoint{0.833427in}{1.222216in}}%
\pgfpathlineto{\pgfqpoint{0.837368in}{1.227621in}}%
\pgfpathlineto{\pgfqpoint{0.841642in}{1.232957in}}%
\pgfpathlineto{\pgfqpoint{0.846243in}{1.238219in}}%
\pgfpathlineto{\pgfqpoint{0.851165in}{1.243403in}}%
\pgfpathlineto{\pgfqpoint{0.854435in}{1.250148in}}%
\pgfpathlineto{\pgfqpoint{0.857708in}{1.256914in}}%
\pgfpathlineto{\pgfqpoint{0.860983in}{1.263699in}}%
\pgfpathlineto{\pgfqpoint{0.864261in}{1.270500in}}%
\pgfpathlineto{\pgfqpoint{0.859544in}{1.265543in}}%
\pgfpathlineto{\pgfqpoint{0.855137in}{1.260510in}}%
\pgfpathlineto{\pgfqpoint{0.851044in}{1.255408in}}%
\pgfpathlineto{\pgfqpoint{0.847270in}{1.250240in}}%
\pgfpathclose%
\pgfusepath{fill}%
\end{pgfscope}%
\begin{pgfscope}%
\pgfpathrectangle{\pgfqpoint{0.041670in}{0.041670in}}{\pgfqpoint{2.216660in}{2.216660in}}%
\pgfusepath{clip}%
\pgfsetbuttcap%
\pgfsetroundjoin%
\definecolor{currentfill}{rgb}{0.122606,0.585371,0.546557}%
\pgfsetfillcolor{currentfill}%
\pgfsetlinewidth{0.000000pt}%
\definecolor{currentstroke}{rgb}{0.000000,0.000000,0.000000}%
\pgfsetstrokecolor{currentstroke}%
\pgfsetdash{}{0pt}%
\pgfpathmoveto{\pgfqpoint{1.408453in}{1.386973in}}%
\pgfpathlineto{\pgfqpoint{1.411185in}{1.380650in}}%
\pgfpathlineto{\pgfqpoint{1.413913in}{1.374311in}}%
\pgfpathlineto{\pgfqpoint{1.416638in}{1.367957in}}%
\pgfpathlineto{\pgfqpoint{1.419360in}{1.361591in}}%
\pgfpathlineto{\pgfqpoint{1.425960in}{1.357859in}}%
\pgfpathlineto{\pgfqpoint{1.432325in}{1.354025in}}%
\pgfpathlineto{\pgfqpoint{1.438449in}{1.350092in}}%
\pgfpathlineto{\pgfqpoint{1.444326in}{1.346065in}}%
\pgfpathlineto{\pgfqpoint{1.441314in}{1.352622in}}%
\pgfpathlineto{\pgfqpoint{1.438298in}{1.359167in}}%
\pgfpathlineto{\pgfqpoint{1.435278in}{1.365698in}}%
\pgfpathlineto{\pgfqpoint{1.432255in}{1.372212in}}%
\pgfpathlineto{\pgfqpoint{1.426653in}{1.376041in}}%
\pgfpathlineto{\pgfqpoint{1.420815in}{1.379779in}}%
\pgfpathlineto{\pgfqpoint{1.414746in}{1.383425in}}%
\pgfpathlineto{\pgfqpoint{1.408453in}{1.386973in}}%
\pgfpathclose%
\pgfusepath{fill}%
\end{pgfscope}%
\begin{pgfscope}%
\pgfpathrectangle{\pgfqpoint{0.041670in}{0.041670in}}{\pgfqpoint{2.216660in}{2.216660in}}%
\pgfusepath{clip}%
\pgfsetbuttcap%
\pgfsetroundjoin%
\definecolor{currentfill}{rgb}{0.267004,0.004874,0.329415}%
\pgfsetfillcolor{currentfill}%
\pgfsetlinewidth{0.000000pt}%
\definecolor{currentstroke}{rgb}{0.000000,0.000000,0.000000}%
\pgfsetstrokecolor{currentstroke}%
\pgfsetdash{}{0pt}%
\pgfpathmoveto{\pgfqpoint{0.613715in}{0.818569in}}%
\pgfpathlineto{\pgfqpoint{0.609979in}{0.817462in}}%
\pgfpathlineto{\pgfqpoint{0.606237in}{0.816592in}}%
\pgfpathlineto{\pgfqpoint{0.602488in}{0.815964in}}%
\pgfpathlineto{\pgfqpoint{0.598731in}{0.815584in}}%
\pgfpathlineto{\pgfqpoint{0.596162in}{0.825258in}}%
\pgfpathlineto{\pgfqpoint{0.594185in}{0.834958in}}%
\pgfpathlineto{\pgfqpoint{0.592800in}{0.844672in}}%
\pgfpathlineto{\pgfqpoint{0.592006in}{0.854392in}}%
\pgfpathlineto{\pgfqpoint{0.595786in}{0.854521in}}%
\pgfpathlineto{\pgfqpoint{0.599559in}{0.854896in}}%
\pgfpathlineto{\pgfqpoint{0.603325in}{0.855513in}}%
\pgfpathlineto{\pgfqpoint{0.607085in}{0.856367in}}%
\pgfpathlineto{\pgfqpoint{0.607878in}{0.846899in}}%
\pgfpathlineto{\pgfqpoint{0.609248in}{0.837437in}}%
\pgfpathlineto{\pgfqpoint{0.611193in}{0.827991in}}%
\pgfpathlineto{\pgfqpoint{0.613715in}{0.818569in}}%
\pgfpathclose%
\pgfusepath{fill}%
\end{pgfscope}%
\begin{pgfscope}%
\pgfpathrectangle{\pgfqpoint{0.041670in}{0.041670in}}{\pgfqpoint{2.216660in}{2.216660in}}%
\pgfusepath{clip}%
\pgfsetbuttcap%
\pgfsetroundjoin%
\definecolor{currentfill}{rgb}{0.231674,0.318106,0.544834}%
\pgfsetfillcolor{currentfill}%
\pgfsetlinewidth{0.000000pt}%
\definecolor{currentstroke}{rgb}{0.000000,0.000000,0.000000}%
\pgfsetstrokecolor{currentstroke}%
\pgfsetdash{}{0pt}%
\pgfpathmoveto{\pgfqpoint{0.770463in}{1.090192in}}%
\pgfpathlineto{\pgfqpoint{0.766818in}{1.083270in}}%
\pgfpathlineto{\pgfqpoint{0.763175in}{1.076414in}}%
\pgfpathlineto{\pgfqpoint{0.759533in}{1.069625in}}%
\pgfpathlineto{\pgfqpoint{0.755893in}{1.062908in}}%
\pgfpathlineto{\pgfqpoint{0.757277in}{1.069702in}}%
\pgfpathlineto{\pgfqpoint{0.759078in}{1.076464in}}%
\pgfpathlineto{\pgfqpoint{0.761293in}{1.083187in}}%
\pgfpathlineto{\pgfqpoint{0.763919in}{1.089866in}}%
\pgfpathlineto{\pgfqpoint{0.767477in}{1.096327in}}%
\pgfpathlineto{\pgfqpoint{0.771037in}{1.102859in}}%
\pgfpathlineto{\pgfqpoint{0.774598in}{1.109460in}}%
\pgfpathlineto{\pgfqpoint{0.778160in}{1.116126in}}%
\pgfpathlineto{\pgfqpoint{0.775638in}{1.109701in}}%
\pgfpathlineto{\pgfqpoint{0.773512in}{1.103232in}}%
\pgfpathlineto{\pgfqpoint{0.771786in}{1.096727in}}%
\pgfpathlineto{\pgfqpoint{0.770463in}{1.090192in}}%
\pgfpathclose%
\pgfusepath{fill}%
\end{pgfscope}%
\begin{pgfscope}%
\pgfpathrectangle{\pgfqpoint{0.041670in}{0.041670in}}{\pgfqpoint{2.216660in}{2.216660in}}%
\pgfusepath{clip}%
\pgfsetbuttcap%
\pgfsetroundjoin%
\definecolor{currentfill}{rgb}{0.279566,0.067836,0.391917}%
\pgfsetfillcolor{currentfill}%
\pgfsetlinewidth{0.000000pt}%
\definecolor{currentstroke}{rgb}{0.000000,0.000000,0.000000}%
\pgfsetstrokecolor{currentstroke}%
\pgfsetdash{}{0pt}%
\pgfpathmoveto{\pgfqpoint{1.693633in}{0.906058in}}%
\pgfpathlineto{\pgfqpoint{1.697318in}{0.902079in}}%
\pgfpathlineto{\pgfqpoint{1.701005in}{0.898266in}}%
\pgfpathlineto{\pgfqpoint{1.704695in}{0.894625in}}%
\pgfpathlineto{\pgfqpoint{1.708389in}{0.891159in}}%
\pgfpathlineto{\pgfqpoint{1.708074in}{0.882465in}}%
\pgfpathlineto{\pgfqpoint{1.707230in}{0.873770in}}%
\pgfpathlineto{\pgfqpoint{1.705855in}{0.865083in}}%
\pgfpathlineto{\pgfqpoint{1.703949in}{0.856412in}}%
\pgfpathlineto{\pgfqpoint{1.700269in}{0.860139in}}%
\pgfpathlineto{\pgfqpoint{1.696592in}{0.864042in}}%
\pgfpathlineto{\pgfqpoint{1.692919in}{0.868115in}}%
\pgfpathlineto{\pgfqpoint{1.689249in}{0.872357in}}%
\pgfpathlineto{\pgfqpoint{1.691118in}{0.880766in}}%
\pgfpathlineto{\pgfqpoint{1.692472in}{0.889191in}}%
\pgfpathlineto{\pgfqpoint{1.693310in}{0.897625in}}%
\pgfpathlineto{\pgfqpoint{1.693633in}{0.906058in}}%
\pgfpathclose%
\pgfusepath{fill}%
\end{pgfscope}%
\begin{pgfscope}%
\pgfpathrectangle{\pgfqpoint{0.041670in}{0.041670in}}{\pgfqpoint{2.216660in}{2.216660in}}%
\pgfusepath{clip}%
\pgfsetbuttcap%
\pgfsetroundjoin%
\definecolor{currentfill}{rgb}{0.268510,0.009605,0.335427}%
\pgfsetfillcolor{currentfill}%
\pgfsetlinewidth{0.000000pt}%
\definecolor{currentstroke}{rgb}{0.000000,0.000000,0.000000}%
\pgfsetstrokecolor{currentstroke}%
\pgfsetdash{}{0pt}%
\pgfpathmoveto{\pgfqpoint{0.628594in}{0.825282in}}%
\pgfpathlineto{\pgfqpoint{0.624883in}{0.823270in}}%
\pgfpathlineto{\pgfqpoint{0.621167in}{0.821478in}}%
\pgfpathlineto{\pgfqpoint{0.617444in}{0.819909in}}%
\pgfpathlineto{\pgfqpoint{0.613715in}{0.818569in}}%
\pgfpathlineto{\pgfqpoint{0.611193in}{0.827991in}}%
\pgfpathlineto{\pgfqpoint{0.609248in}{0.837437in}}%
\pgfpathlineto{\pgfqpoint{0.607878in}{0.846899in}}%
\pgfpathlineto{\pgfqpoint{0.607085in}{0.856367in}}%
\pgfpathlineto{\pgfqpoint{0.610838in}{0.857452in}}%
\pgfpathlineto{\pgfqpoint{0.614585in}{0.858766in}}%
\pgfpathlineto{\pgfqpoint{0.618326in}{0.860302in}}%
\pgfpathlineto{\pgfqpoint{0.622062in}{0.862057in}}%
\pgfpathlineto{\pgfqpoint{0.622854in}{0.852845in}}%
\pgfpathlineto{\pgfqpoint{0.624206in}{0.843639in}}%
\pgfpathlineto{\pgfqpoint{0.626120in}{0.834448in}}%
\pgfpathlineto{\pgfqpoint{0.628594in}{0.825282in}}%
\pgfpathclose%
\pgfusepath{fill}%
\end{pgfscope}%
\begin{pgfscope}%
\pgfpathrectangle{\pgfqpoint{0.041670in}{0.041670in}}{\pgfqpoint{2.216660in}{2.216660in}}%
\pgfusepath{clip}%
\pgfsetbuttcap%
\pgfsetroundjoin%
\definecolor{currentfill}{rgb}{0.120081,0.622161,0.534946}%
\pgfsetfillcolor{currentfill}%
\pgfsetlinewidth{0.000000pt}%
\definecolor{currentstroke}{rgb}{0.000000,0.000000,0.000000}%
\pgfsetstrokecolor{currentstroke}%
\pgfsetdash{}{0pt}%
\pgfpathmoveto{\pgfqpoint{1.371537in}{1.424533in}}%
\pgfpathlineto{\pgfqpoint{1.373950in}{1.418472in}}%
\pgfpathlineto{\pgfqpoint{1.376359in}{1.412385in}}%
\pgfpathlineto{\pgfqpoint{1.378766in}{1.406273in}}%
\pgfpathlineto{\pgfqpoint{1.381170in}{1.400139in}}%
\pgfpathlineto{\pgfqpoint{1.388294in}{1.397008in}}%
\pgfpathlineto{\pgfqpoint{1.395220in}{1.393768in}}%
\pgfpathlineto{\pgfqpoint{1.401943in}{1.390422in}}%
\pgfpathlineto{\pgfqpoint{1.408453in}{1.386973in}}%
\pgfpathlineto{\pgfqpoint{1.405718in}{1.393277in}}%
\pgfpathlineto{\pgfqpoint{1.402980in}{1.399558in}}%
\pgfpathlineto{\pgfqpoint{1.400239in}{1.405815in}}%
\pgfpathlineto{\pgfqpoint{1.397495in}{1.412044in}}%
\pgfpathlineto{\pgfqpoint{1.391301in}{1.415316in}}%
\pgfpathlineto{\pgfqpoint{1.384906in}{1.418489in}}%
\pgfpathlineto{\pgfqpoint{1.378316in}{1.421563in}}%
\pgfpathlineto{\pgfqpoint{1.371537in}{1.424533in}}%
\pgfpathclose%
\pgfusepath{fill}%
\end{pgfscope}%
\begin{pgfscope}%
\pgfpathrectangle{\pgfqpoint{0.041670in}{0.041670in}}{\pgfqpoint{2.216660in}{2.216660in}}%
\pgfusepath{clip}%
\pgfsetbuttcap%
\pgfsetroundjoin%
\definecolor{currentfill}{rgb}{0.274128,0.199721,0.498911}%
\pgfsetfillcolor{currentfill}%
\pgfsetlinewidth{0.000000pt}%
\definecolor{currentstroke}{rgb}{0.000000,0.000000,0.000000}%
\pgfsetstrokecolor{currentstroke}%
\pgfsetdash{}{0pt}%
\pgfpathmoveto{\pgfqpoint{0.725265in}{0.982690in}}%
\pgfpathlineto{\pgfqpoint{0.721602in}{0.976507in}}%
\pgfpathlineto{\pgfqpoint{0.717939in}{0.970431in}}%
\pgfpathlineto{\pgfqpoint{0.714275in}{0.964464in}}%
\pgfpathlineto{\pgfqpoint{0.710611in}{0.958610in}}%
\pgfpathlineto{\pgfqpoint{0.710320in}{0.966256in}}%
\pgfpathlineto{\pgfqpoint{0.710498in}{0.973894in}}%
\pgfpathlineto{\pgfqpoint{0.711144in}{0.981517in}}%
\pgfpathlineto{\pgfqpoint{0.712254in}{0.989118in}}%
\pgfpathlineto{\pgfqpoint{0.715890in}{0.994710in}}%
\pgfpathlineto{\pgfqpoint{0.719525in}{1.000415in}}%
\pgfpathlineto{\pgfqpoint{0.723161in}{1.006229in}}%
\pgfpathlineto{\pgfqpoint{0.726796in}{1.012150in}}%
\pgfpathlineto{\pgfqpoint{0.725735in}{1.004809in}}%
\pgfpathlineto{\pgfqpoint{0.725125in}{0.997448in}}%
\pgfpathlineto{\pgfqpoint{0.724968in}{0.990072in}}%
\pgfpathlineto{\pgfqpoint{0.725265in}{0.982690in}}%
\pgfpathclose%
\pgfusepath{fill}%
\end{pgfscope}%
\begin{pgfscope}%
\pgfpathrectangle{\pgfqpoint{0.041670in}{0.041670in}}{\pgfqpoint{2.216660in}{2.216660in}}%
\pgfusepath{clip}%
\pgfsetbuttcap%
\pgfsetroundjoin%
\definecolor{currentfill}{rgb}{0.267004,0.004874,0.329415}%
\pgfsetfillcolor{currentfill}%
\pgfsetlinewidth{0.000000pt}%
\definecolor{currentstroke}{rgb}{0.000000,0.000000,0.000000}%
\pgfsetstrokecolor{currentstroke}%
\pgfsetdash{}{0pt}%
\pgfpathmoveto{\pgfqpoint{0.598731in}{0.815584in}}%
\pgfpathlineto{\pgfqpoint{0.594966in}{0.815455in}}%
\pgfpathlineto{\pgfqpoint{0.591194in}{0.815583in}}%
\pgfpathlineto{\pgfqpoint{0.587413in}{0.815974in}}%
\pgfpathlineto{\pgfqpoint{0.583624in}{0.816631in}}%
\pgfpathlineto{\pgfqpoint{0.581009in}{0.826555in}}%
\pgfpathlineto{\pgfqpoint{0.579002in}{0.836504in}}%
\pgfpathlineto{\pgfqpoint{0.577602in}{0.846468in}}%
\pgfpathlineto{\pgfqpoint{0.576808in}{0.856436in}}%
\pgfpathlineto{\pgfqpoint{0.580620in}{0.855531in}}%
\pgfpathlineto{\pgfqpoint{0.584423in}{0.854892in}}%
\pgfpathlineto{\pgfqpoint{0.588218in}{0.854514in}}%
\pgfpathlineto{\pgfqpoint{0.592006in}{0.854392in}}%
\pgfpathlineto{\pgfqpoint{0.592800in}{0.844672in}}%
\pgfpathlineto{\pgfqpoint{0.594185in}{0.834958in}}%
\pgfpathlineto{\pgfqpoint{0.596162in}{0.825258in}}%
\pgfpathlineto{\pgfqpoint{0.598731in}{0.815584in}}%
\pgfpathclose%
\pgfusepath{fill}%
\end{pgfscope}%
\begin{pgfscope}%
\pgfpathrectangle{\pgfqpoint{0.041670in}{0.041670in}}{\pgfqpoint{2.216660in}{2.216660in}}%
\pgfusepath{clip}%
\pgfsetbuttcap%
\pgfsetroundjoin%
\definecolor{currentfill}{rgb}{0.260571,0.246922,0.522828}%
\pgfsetfillcolor{currentfill}%
\pgfsetlinewidth{0.000000pt}%
\definecolor{currentstroke}{rgb}{0.000000,0.000000,0.000000}%
\pgfsetstrokecolor{currentstroke}%
\pgfsetdash{}{0pt}%
\pgfpathmoveto{\pgfqpoint{1.871676in}{1.027009in}}%
\pgfpathlineto{\pgfqpoint{1.875738in}{1.036607in}}%
\pgfpathlineto{\pgfqpoint{1.879817in}{1.046622in}}%
\pgfpathlineto{\pgfqpoint{1.883914in}{1.057059in}}%
\pgfpathlineto{\pgfqpoint{1.888029in}{1.067925in}}%
\pgfpathlineto{\pgfqpoint{1.890732in}{1.056534in}}%
\pgfpathlineto{\pgfqpoint{1.892737in}{1.045086in}}%
\pgfpathlineto{\pgfqpoint{1.894037in}{1.033594in}}%
\pgfpathlineto{\pgfqpoint{1.894628in}{1.022067in}}%
\pgfpathlineto{\pgfqpoint{1.890447in}{1.011398in}}%
\pgfpathlineto{\pgfqpoint{1.886285in}{1.001161in}}%
\pgfpathlineto{\pgfqpoint{1.882142in}{0.991350in}}%
\pgfpathlineto{\pgfqpoint{1.878017in}{0.981956in}}%
\pgfpathlineto{\pgfqpoint{1.877466in}{0.993279in}}%
\pgfpathlineto{\pgfqpoint{1.876223in}{1.004569in}}%
\pgfpathlineto{\pgfqpoint{1.874291in}{1.015816in}}%
\pgfpathlineto{\pgfqpoint{1.871676in}{1.027009in}}%
\pgfpathclose%
\pgfusepath{fill}%
\end{pgfscope}%
\begin{pgfscope}%
\pgfpathrectangle{\pgfqpoint{0.041670in}{0.041670in}}{\pgfqpoint{2.216660in}{2.216660in}}%
\pgfusepath{clip}%
\pgfsetbuttcap%
\pgfsetroundjoin%
\definecolor{currentfill}{rgb}{0.271305,0.019942,0.347269}%
\pgfsetfillcolor{currentfill}%
\pgfsetlinewidth{0.000000pt}%
\definecolor{currentstroke}{rgb}{0.000000,0.000000,0.000000}%
\pgfsetstrokecolor{currentstroke}%
\pgfsetdash{}{0pt}%
\pgfpathmoveto{\pgfqpoint{0.643384in}{0.835429in}}%
\pgfpathlineto{\pgfqpoint{0.639694in}{0.832586in}}%
\pgfpathlineto{\pgfqpoint{0.635999in}{0.829944in}}%
\pgfpathlineto{\pgfqpoint{0.632299in}{0.827508in}}%
\pgfpathlineto{\pgfqpoint{0.628594in}{0.825282in}}%
\pgfpathlineto{\pgfqpoint{0.626120in}{0.834448in}}%
\pgfpathlineto{\pgfqpoint{0.624206in}{0.843639in}}%
\pgfpathlineto{\pgfqpoint{0.622854in}{0.852845in}}%
\pgfpathlineto{\pgfqpoint{0.622062in}{0.862057in}}%
\pgfpathlineto{\pgfqpoint{0.625792in}{0.864026in}}%
\pgfpathlineto{\pgfqpoint{0.629518in}{0.866205in}}%
\pgfpathlineto{\pgfqpoint{0.633238in}{0.868588in}}%
\pgfpathlineto{\pgfqpoint{0.636954in}{0.871172in}}%
\pgfpathlineto{\pgfqpoint{0.637743in}{0.862218in}}%
\pgfpathlineto{\pgfqpoint{0.639077in}{0.853269in}}%
\pgfpathlineto{\pgfqpoint{0.640958in}{0.844337in}}%
\pgfpathlineto{\pgfqpoint{0.643384in}{0.835429in}}%
\pgfpathclose%
\pgfusepath{fill}%
\end{pgfscope}%
\begin{pgfscope}%
\pgfpathrectangle{\pgfqpoint{0.041670in}{0.041670in}}{\pgfqpoint{2.216660in}{2.216660in}}%
\pgfusepath{clip}%
\pgfsetbuttcap%
\pgfsetroundjoin%
\definecolor{currentfill}{rgb}{0.166383,0.690856,0.496502}%
\pgfsetfillcolor{currentfill}%
\pgfsetlinewidth{0.000000pt}%
\definecolor{currentstroke}{rgb}{0.000000,0.000000,0.000000}%
\pgfsetstrokecolor{currentstroke}%
\pgfsetdash{}{0pt}%
\pgfpathmoveto{\pgfqpoint{1.268142in}{1.495599in}}%
\pgfpathlineto{\pgfqpoint{1.269386in}{1.490200in}}%
\pgfpathlineto{\pgfqpoint{1.270628in}{1.484756in}}%
\pgfpathlineto{\pgfqpoint{1.271868in}{1.479267in}}%
\pgfpathlineto{\pgfqpoint{1.273106in}{1.473737in}}%
\pgfpathlineto{\pgfqpoint{1.281162in}{1.472268in}}%
\pgfpathlineto{\pgfqpoint{1.289122in}{1.470677in}}%
\pgfpathlineto{\pgfqpoint{1.296979in}{1.468966in}}%
\pgfpathlineto{\pgfqpoint{1.304726in}{1.467137in}}%
\pgfpathlineto{\pgfqpoint{1.303065in}{1.472760in}}%
\pgfpathlineto{\pgfqpoint{1.301402in}{1.478342in}}%
\pgfpathlineto{\pgfqpoint{1.299737in}{1.483880in}}%
\pgfpathlineto{\pgfqpoint{1.298070in}{1.489372in}}%
\pgfpathlineto{\pgfqpoint{1.290738in}{1.491098in}}%
\pgfpathlineto{\pgfqpoint{1.283301in}{1.492712in}}%
\pgfpathlineto{\pgfqpoint{1.275767in}{1.494213in}}%
\pgfpathlineto{\pgfqpoint{1.268142in}{1.495599in}}%
\pgfpathclose%
\pgfusepath{fill}%
\end{pgfscope}%
\begin{pgfscope}%
\pgfpathrectangle{\pgfqpoint{0.041670in}{0.041670in}}{\pgfqpoint{2.216660in}{2.216660in}}%
\pgfusepath{clip}%
\pgfsetbuttcap%
\pgfsetroundjoin%
\definecolor{currentfill}{rgb}{0.263663,0.237631,0.518762}%
\pgfsetfillcolor{currentfill}%
\pgfsetlinewidth{0.000000pt}%
\definecolor{currentstroke}{rgb}{0.000000,0.000000,0.000000}%
\pgfsetstrokecolor{currentstroke}%
\pgfsetdash{}{0pt}%
\pgfpathmoveto{\pgfqpoint{1.617306in}{1.043092in}}%
\pgfpathlineto{\pgfqpoint{1.620929in}{1.036841in}}%
\pgfpathlineto{\pgfqpoint{1.624551in}{1.030681in}}%
\pgfpathlineto{\pgfqpoint{1.628173in}{1.024617in}}%
\pgfpathlineto{\pgfqpoint{1.631795in}{1.018651in}}%
\pgfpathlineto{\pgfqpoint{1.633254in}{1.011335in}}%
\pgfpathlineto{\pgfqpoint{1.634264in}{1.003992in}}%
\pgfpathlineto{\pgfqpoint{1.634824in}{0.996629in}}%
\pgfpathlineto{\pgfqpoint{1.634931in}{0.989252in}}%
\pgfpathlineto{\pgfqpoint{1.631269in}{0.995479in}}%
\pgfpathlineto{\pgfqpoint{1.627608in}{1.001804in}}%
\pgfpathlineto{\pgfqpoint{1.623946in}{1.008225in}}%
\pgfpathlineto{\pgfqpoint{1.620283in}{1.014738in}}%
\pgfpathlineto{\pgfqpoint{1.620195in}{1.021852in}}%
\pgfpathlineto{\pgfqpoint{1.619668in}{1.028953in}}%
\pgfpathlineto{\pgfqpoint{1.618704in}{1.036036in}}%
\pgfpathlineto{\pgfqpoint{1.617306in}{1.043092in}}%
\pgfpathclose%
\pgfusepath{fill}%
\end{pgfscope}%
\begin{pgfscope}%
\pgfpathrectangle{\pgfqpoint{0.041670in}{0.041670in}}{\pgfqpoint{2.216660in}{2.216660in}}%
\pgfusepath{clip}%
\pgfsetbuttcap%
\pgfsetroundjoin%
\definecolor{currentfill}{rgb}{0.133743,0.548535,0.553541}%
\pgfsetfillcolor{currentfill}%
\pgfsetlinewidth{0.000000pt}%
\definecolor{currentstroke}{rgb}{0.000000,0.000000,0.000000}%
\pgfsetstrokecolor{currentstroke}%
\pgfsetdash{}{0pt}%
\pgfpathmoveto{\pgfqpoint{1.444326in}{1.346065in}}%
\pgfpathlineto{\pgfqpoint{1.447336in}{1.339499in}}%
\pgfpathlineto{\pgfqpoint{1.450342in}{1.332926in}}%
\pgfpathlineto{\pgfqpoint{1.453345in}{1.326350in}}%
\pgfpathlineto{\pgfqpoint{1.456345in}{1.319773in}}%
\pgfpathlineto{\pgfqpoint{1.462232in}{1.315450in}}%
\pgfpathlineto{\pgfqpoint{1.467849in}{1.311036in}}%
\pgfpathlineto{\pgfqpoint{1.473190in}{1.306534in}}%
\pgfpathlineto{\pgfqpoint{1.478248in}{1.301948in}}%
\pgfpathlineto{\pgfqpoint{1.475002in}{1.308737in}}%
\pgfpathlineto{\pgfqpoint{1.471753in}{1.315524in}}%
\pgfpathlineto{\pgfqpoint{1.468501in}{1.322307in}}%
\pgfpathlineto{\pgfqpoint{1.465245in}{1.329084in}}%
\pgfpathlineto{\pgfqpoint{1.460415in}{1.333452in}}%
\pgfpathlineto{\pgfqpoint{1.455315in}{1.337741in}}%
\pgfpathlineto{\pgfqpoint{1.449950in}{1.341947in}}%
\pgfpathlineto{\pgfqpoint{1.444326in}{1.346065in}}%
\pgfpathclose%
\pgfusepath{fill}%
\end{pgfscope}%
\begin{pgfscope}%
\pgfpathrectangle{\pgfqpoint{0.041670in}{0.041670in}}{\pgfqpoint{2.216660in}{2.216660in}}%
\pgfusepath{clip}%
\pgfsetbuttcap%
\pgfsetroundjoin%
\definecolor{currentfill}{rgb}{0.172719,0.448791,0.557885}%
\pgfsetfillcolor{currentfill}%
\pgfsetlinewidth{0.000000pt}%
\definecolor{currentstroke}{rgb}{0.000000,0.000000,0.000000}%
\pgfsetstrokecolor{currentstroke}%
\pgfsetdash{}{0pt}%
\pgfpathmoveto{\pgfqpoint{1.902939in}{1.217417in}}%
\pgfpathlineto{\pgfqpoint{1.907104in}{1.232652in}}%
\pgfpathlineto{\pgfqpoint{1.911293in}{1.248390in}}%
\pgfpathlineto{\pgfqpoint{1.915504in}{1.264640in}}%
\pgfpathlineto{\pgfqpoint{1.921368in}{1.253114in}}%
\pgfpathlineto{\pgfqpoint{1.926519in}{1.241482in}}%
\pgfpathlineto{\pgfqpoint{1.930949in}{1.229755in}}%
\pgfpathlineto{\pgfqpoint{1.934648in}{1.217945in}}%
\pgfpathlineto{\pgfqpoint{1.930298in}{1.201853in}}%
\pgfpathlineto{\pgfqpoint{1.925972in}{1.186277in}}%
\pgfpathlineto{\pgfqpoint{1.921669in}{1.171208in}}%
\pgfpathlineto{\pgfqpoint{1.918055in}{1.182894in}}%
\pgfpathlineto{\pgfqpoint{1.913723in}{1.194499in}}%
\pgfpathlineto{\pgfqpoint{1.908682in}{1.206010in}}%
\pgfpathlineto{\pgfqpoint{1.902939in}{1.217417in}}%
\pgfpathclose%
\pgfusepath{fill}%
\end{pgfscope}%
\begin{pgfscope}%
\pgfpathrectangle{\pgfqpoint{0.041670in}{0.041670in}}{\pgfqpoint{2.216660in}{2.216660in}}%
\pgfusepath{clip}%
\pgfsetbuttcap%
\pgfsetroundjoin%
\definecolor{currentfill}{rgb}{0.120081,0.622161,0.534946}%
\pgfsetfillcolor{currentfill}%
\pgfsetlinewidth{0.000000pt}%
\definecolor{currentstroke}{rgb}{0.000000,0.000000,0.000000}%
\pgfsetstrokecolor{currentstroke}%
\pgfsetdash{}{0pt}%
\pgfpathmoveto{\pgfqpoint{0.957084in}{1.409057in}}%
\pgfpathlineto{\pgfqpoint{0.954271in}{1.402787in}}%
\pgfpathlineto{\pgfqpoint{0.951461in}{1.396490in}}%
\pgfpathlineto{\pgfqpoint{0.948655in}{1.390168in}}%
\pgfpathlineto{\pgfqpoint{0.945852in}{1.383824in}}%
\pgfpathlineto{\pgfqpoint{0.952169in}{1.387362in}}%
\pgfpathlineto{\pgfqpoint{0.958704in}{1.390799in}}%
\pgfpathlineto{\pgfqpoint{0.965449in}{1.394133in}}%
\pgfpathlineto{\pgfqpoint{0.972398in}{1.397361in}}%
\pgfpathlineto{\pgfqpoint{0.974878in}{1.403531in}}%
\pgfpathlineto{\pgfqpoint{0.977362in}{1.409679in}}%
\pgfpathlineto{\pgfqpoint{0.979848in}{1.415802in}}%
\pgfpathlineto{\pgfqpoint{0.982338in}{1.421898in}}%
\pgfpathlineto{\pgfqpoint{0.975726in}{1.418836in}}%
\pgfpathlineto{\pgfqpoint{0.969309in}{1.415673in}}%
\pgfpathlineto{\pgfqpoint{0.963093in}{1.412413in}}%
\pgfpathlineto{\pgfqpoint{0.957084in}{1.409057in}}%
\pgfpathclose%
\pgfusepath{fill}%
\end{pgfscope}%
\begin{pgfscope}%
\pgfpathrectangle{\pgfqpoint{0.041670in}{0.041670in}}{\pgfqpoint{2.216660in}{2.216660in}}%
\pgfusepath{clip}%
\pgfsetbuttcap%
\pgfsetroundjoin%
\definecolor{currentfill}{rgb}{0.122606,0.585371,0.546557}%
\pgfsetfillcolor{currentfill}%
\pgfsetlinewidth{0.000000pt}%
\definecolor{currentstroke}{rgb}{0.000000,0.000000,0.000000}%
\pgfsetstrokecolor{currentstroke}%
\pgfsetdash{}{0pt}%
\pgfpathmoveto{\pgfqpoint{0.922878in}{1.368736in}}%
\pgfpathlineto{\pgfqpoint{0.919796in}{1.362177in}}%
\pgfpathlineto{\pgfqpoint{0.916718in}{1.355601in}}%
\pgfpathlineto{\pgfqpoint{0.913643in}{1.349011in}}%
\pgfpathlineto{\pgfqpoint{0.910572in}{1.342409in}}%
\pgfpathlineto{\pgfqpoint{0.916224in}{1.346517in}}%
\pgfpathlineto{\pgfqpoint{0.922129in}{1.350534in}}%
\pgfpathlineto{\pgfqpoint{0.928280in}{1.354455in}}%
\pgfpathlineto{\pgfqpoint{0.934672in}{1.358278in}}%
\pgfpathlineto{\pgfqpoint{0.937462in}{1.364685in}}%
\pgfpathlineto{\pgfqpoint{0.940255in}{1.371080in}}%
\pgfpathlineto{\pgfqpoint{0.943052in}{1.377460in}}%
\pgfpathlineto{\pgfqpoint{0.945852in}{1.383824in}}%
\pgfpathlineto{\pgfqpoint{0.939758in}{1.380189in}}%
\pgfpathlineto{\pgfqpoint{0.933894in}{1.376460in}}%
\pgfpathlineto{\pgfqpoint{0.928265in}{1.372642in}}%
\pgfpathlineto{\pgfqpoint{0.922878in}{1.368736in}}%
\pgfpathclose%
\pgfusepath{fill}%
\end{pgfscope}%
\begin{pgfscope}%
\pgfpathrectangle{\pgfqpoint{0.041670in}{0.041670in}}{\pgfqpoint{2.216660in}{2.216660in}}%
\pgfusepath{clip}%
\pgfsetbuttcap%
\pgfsetroundjoin%
\definecolor{currentfill}{rgb}{0.166383,0.690856,0.496502}%
\pgfsetfillcolor{currentfill}%
\pgfsetlinewidth{0.000000pt}%
\definecolor{currentstroke}{rgb}{0.000000,0.000000,0.000000}%
\pgfsetstrokecolor{currentstroke}%
\pgfsetdash{}{0pt}%
\pgfpathmoveto{\pgfqpoint{1.055416in}{1.487745in}}%
\pgfpathlineto{\pgfqpoint{1.053658in}{1.482229in}}%
\pgfpathlineto{\pgfqpoint{1.051902in}{1.476667in}}%
\pgfpathlineto{\pgfqpoint{1.050148in}{1.471061in}}%
\pgfpathlineto{\pgfqpoint{1.048397in}{1.465413in}}%
\pgfpathlineto{\pgfqpoint{1.056039in}{1.467346in}}%
\pgfpathlineto{\pgfqpoint{1.063798in}{1.469162in}}%
\pgfpathlineto{\pgfqpoint{1.071667in}{1.470860in}}%
\pgfpathlineto{\pgfqpoint{1.079638in}{1.472437in}}%
\pgfpathlineto{\pgfqpoint{1.080972in}{1.477985in}}%
\pgfpathlineto{\pgfqpoint{1.082308in}{1.483492in}}%
\pgfpathlineto{\pgfqpoint{1.083646in}{1.488955in}}%
\pgfpathlineto{\pgfqpoint{1.084986in}{1.494372in}}%
\pgfpathlineto{\pgfqpoint{1.077441in}{1.492884in}}%
\pgfpathlineto{\pgfqpoint{1.069993in}{1.491282in}}%
\pgfpathlineto{\pgfqpoint{1.062649in}{1.489569in}}%
\pgfpathlineto{\pgfqpoint{1.055416in}{1.487745in}}%
\pgfpathclose%
\pgfusepath{fill}%
\end{pgfscope}%
\begin{pgfscope}%
\pgfpathrectangle{\pgfqpoint{0.041670in}{0.041670in}}{\pgfqpoint{2.216660in}{2.216660in}}%
\pgfusepath{clip}%
\pgfsetbuttcap%
\pgfsetroundjoin%
\definecolor{currentfill}{rgb}{0.134692,0.658636,0.517649}%
\pgfsetfillcolor{currentfill}%
\pgfsetlinewidth{0.000000pt}%
\definecolor{currentstroke}{rgb}{0.000000,0.000000,0.000000}%
\pgfsetstrokecolor{currentstroke}%
\pgfsetdash{}{0pt}%
\pgfpathmoveto{\pgfqpoint{1.334462in}{1.458669in}}%
\pgfpathlineto{\pgfqpoint{1.336518in}{1.452887in}}%
\pgfpathlineto{\pgfqpoint{1.338572in}{1.447068in}}%
\pgfpathlineto{\pgfqpoint{1.340623in}{1.441214in}}%
\pgfpathlineto{\pgfqpoint{1.342671in}{1.435329in}}%
\pgfpathlineto{\pgfqpoint{1.350136in}{1.432797in}}%
\pgfpathlineto{\pgfqpoint{1.357441in}{1.430153in}}%
\pgfpathlineto{\pgfqpoint{1.364577in}{1.427397in}}%
\pgfpathlineto{\pgfqpoint{1.371537in}{1.424533in}}%
\pgfpathlineto{\pgfqpoint{1.369122in}{1.430564in}}%
\pgfpathlineto{\pgfqpoint{1.366703in}{1.436563in}}%
\pgfpathlineto{\pgfqpoint{1.364282in}{1.442527in}}%
\pgfpathlineto{\pgfqpoint{1.361857in}{1.448455in}}%
\pgfpathlineto{\pgfqpoint{1.355252in}{1.451164in}}%
\pgfpathlineto{\pgfqpoint{1.348480in}{1.453772in}}%
\pgfpathlineto{\pgfqpoint{1.341547in}{1.456274in}}%
\pgfpathlineto{\pgfqpoint{1.334462in}{1.458669in}}%
\pgfpathclose%
\pgfusepath{fill}%
\end{pgfscope}%
\begin{pgfscope}%
\pgfpathrectangle{\pgfqpoint{0.041670in}{0.041670in}}{\pgfqpoint{2.216660in}{2.216660in}}%
\pgfusepath{clip}%
\pgfsetbuttcap%
\pgfsetroundjoin%
\definecolor{currentfill}{rgb}{0.179019,0.433756,0.557430}%
\pgfsetfillcolor{currentfill}%
\pgfsetlinewidth{0.000000pt}%
\definecolor{currentstroke}{rgb}{0.000000,0.000000,0.000000}%
\pgfsetstrokecolor{currentstroke}%
\pgfsetdash{}{0pt}%
\pgfpathmoveto{\pgfqpoint{1.522996in}{1.227023in}}%
\pgfpathlineto{\pgfqpoint{1.526412in}{1.220124in}}%
\pgfpathlineto{\pgfqpoint{1.529826in}{1.213252in}}%
\pgfpathlineto{\pgfqpoint{1.533238in}{1.206411in}}%
\pgfpathlineto{\pgfqpoint{1.536646in}{1.199603in}}%
\pgfpathlineto{\pgfqpoint{1.540717in}{1.193953in}}%
\pgfpathlineto{\pgfqpoint{1.544438in}{1.188237in}}%
\pgfpathlineto{\pgfqpoint{1.547804in}{1.182460in}}%
\pgfpathlineto{\pgfqpoint{1.550811in}{1.176628in}}%
\pgfpathlineto{\pgfqpoint{1.547256in}{1.183679in}}%
\pgfpathlineto{\pgfqpoint{1.543697in}{1.190763in}}%
\pgfpathlineto{\pgfqpoint{1.540137in}{1.197877in}}%
\pgfpathlineto{\pgfqpoint{1.536573in}{1.205019in}}%
\pgfpathlineto{\pgfqpoint{1.533693in}{1.210604in}}%
\pgfpathlineto{\pgfqpoint{1.530467in}{1.216137in}}%
\pgfpathlineto{\pgfqpoint{1.526900in}{1.221612in}}%
\pgfpathlineto{\pgfqpoint{1.522996in}{1.227023in}}%
\pgfpathclose%
\pgfusepath{fill}%
\end{pgfscope}%
\begin{pgfscope}%
\pgfpathrectangle{\pgfqpoint{0.041670in}{0.041670in}}{\pgfqpoint{2.216660in}{2.216660in}}%
\pgfusepath{clip}%
\pgfsetbuttcap%
\pgfsetroundjoin%
\definecolor{currentfill}{rgb}{0.268510,0.009605,0.335427}%
\pgfsetfillcolor{currentfill}%
\pgfsetlinewidth{0.000000pt}%
\definecolor{currentstroke}{rgb}{0.000000,0.000000,0.000000}%
\pgfsetstrokecolor{currentstroke}%
\pgfsetdash{}{0pt}%
\pgfpathmoveto{\pgfqpoint{0.583624in}{0.816631in}}%
\pgfpathlineto{\pgfqpoint{0.579826in}{0.817559in}}%
\pgfpathlineto{\pgfqpoint{0.576019in}{0.818765in}}%
\pgfpathlineto{\pgfqpoint{0.572203in}{0.820253in}}%
\pgfpathlineto{\pgfqpoint{0.568377in}{0.822028in}}%
\pgfpathlineto{\pgfqpoint{0.565717in}{0.832198in}}%
\pgfpathlineto{\pgfqpoint{0.563680in}{0.842393in}}%
\pgfpathlineto{\pgfqpoint{0.562266in}{0.852603in}}%
\pgfpathlineto{\pgfqpoint{0.561473in}{0.862816in}}%
\pgfpathlineto{\pgfqpoint{0.565321in}{0.860797in}}%
\pgfpathlineto{\pgfqpoint{0.569159in}{0.859064in}}%
\pgfpathlineto{\pgfqpoint{0.572988in}{0.857612in}}%
\pgfpathlineto{\pgfqpoint{0.576808in}{0.856436in}}%
\pgfpathlineto{\pgfqpoint{0.577602in}{0.846468in}}%
\pgfpathlineto{\pgfqpoint{0.579002in}{0.836504in}}%
\pgfpathlineto{\pgfqpoint{0.581009in}{0.826555in}}%
\pgfpathlineto{\pgfqpoint{0.583624in}{0.816631in}}%
\pgfpathclose%
\pgfusepath{fill}%
\end{pgfscope}%
\begin{pgfscope}%
\pgfpathrectangle{\pgfqpoint{0.041670in}{0.041670in}}{\pgfqpoint{2.216660in}{2.216660in}}%
\pgfusepath{clip}%
\pgfsetbuttcap%
\pgfsetroundjoin%
\definecolor{currentfill}{rgb}{0.274952,0.037752,0.364543}%
\pgfsetfillcolor{currentfill}%
\pgfsetlinewidth{0.000000pt}%
\definecolor{currentstroke}{rgb}{0.000000,0.000000,0.000000}%
\pgfsetstrokecolor{currentstroke}%
\pgfsetdash{}{0pt}%
\pgfpathmoveto{\pgfqpoint{0.658100in}{0.848727in}}%
\pgfpathlineto{\pgfqpoint{0.654427in}{0.845122in}}%
\pgfpathlineto{\pgfqpoint{0.650751in}{0.841701in}}%
\pgfpathlineto{\pgfqpoint{0.647070in}{0.838468in}}%
\pgfpathlineto{\pgfqpoint{0.643384in}{0.835429in}}%
\pgfpathlineto{\pgfqpoint{0.640958in}{0.844337in}}%
\pgfpathlineto{\pgfqpoint{0.639077in}{0.853269in}}%
\pgfpathlineto{\pgfqpoint{0.637743in}{0.862218in}}%
\pgfpathlineto{\pgfqpoint{0.636954in}{0.871172in}}%
\pgfpathlineto{\pgfqpoint{0.640665in}{0.873953in}}%
\pgfpathlineto{\pgfqpoint{0.644372in}{0.876926in}}%
\pgfpathlineto{\pgfqpoint{0.648075in}{0.880087in}}%
\pgfpathlineto{\pgfqpoint{0.651775in}{0.883431in}}%
\pgfpathlineto{\pgfqpoint{0.652560in}{0.874736in}}%
\pgfpathlineto{\pgfqpoint{0.653876in}{0.866047in}}%
\pgfpathlineto{\pgfqpoint{0.655723in}{0.857375in}}%
\pgfpathlineto{\pgfqpoint{0.658100in}{0.848727in}}%
\pgfpathclose%
\pgfusepath{fill}%
\end{pgfscope}%
\begin{pgfscope}%
\pgfpathrectangle{\pgfqpoint{0.041670in}{0.041670in}}{\pgfqpoint{2.216660in}{2.216660in}}%
\pgfusepath{clip}%
\pgfsetbuttcap%
\pgfsetroundjoin%
\definecolor{currentfill}{rgb}{0.282327,0.094955,0.417331}%
\pgfsetfillcolor{currentfill}%
\pgfsetlinewidth{0.000000pt}%
\definecolor{currentstroke}{rgb}{0.000000,0.000000,0.000000}%
\pgfsetstrokecolor{currentstroke}%
\pgfsetdash{}{0pt}%
\pgfpathmoveto{\pgfqpoint{1.678921in}{0.923567in}}%
\pgfpathlineto{\pgfqpoint{1.682596in}{0.918959in}}%
\pgfpathlineto{\pgfqpoint{1.686273in}{0.914503in}}%
\pgfpathlineto{\pgfqpoint{1.689952in}{0.910201in}}%
\pgfpathlineto{\pgfqpoint{1.693633in}{0.906058in}}%
\pgfpathlineto{\pgfqpoint{1.693310in}{0.897625in}}%
\pgfpathlineto{\pgfqpoint{1.692472in}{0.889191in}}%
\pgfpathlineto{\pgfqpoint{1.691118in}{0.880766in}}%
\pgfpathlineto{\pgfqpoint{1.689249in}{0.872357in}}%
\pgfpathlineto{\pgfqpoint{1.685582in}{0.876761in}}%
\pgfpathlineto{\pgfqpoint{1.681917in}{0.881325in}}%
\pgfpathlineto{\pgfqpoint{1.678255in}{0.886044in}}%
\pgfpathlineto{\pgfqpoint{1.674596in}{0.890915in}}%
\pgfpathlineto{\pgfqpoint{1.676428in}{0.899061in}}%
\pgfpathlineto{\pgfqpoint{1.677759in}{0.907224in}}%
\pgfpathlineto{\pgfqpoint{1.678590in}{0.915396in}}%
\pgfpathlineto{\pgfqpoint{1.678921in}{0.923567in}}%
\pgfpathclose%
\pgfusepath{fill}%
\end{pgfscope}%
\begin{pgfscope}%
\pgfpathrectangle{\pgfqpoint{0.041670in}{0.041670in}}{\pgfqpoint{2.216660in}{2.216660in}}%
\pgfusepath{clip}%
\pgfsetbuttcap%
\pgfsetroundjoin%
\definecolor{currentfill}{rgb}{0.277941,0.056324,0.381191}%
\pgfsetfillcolor{currentfill}%
\pgfsetlinewidth{0.000000pt}%
\definecolor{currentstroke}{rgb}{0.000000,0.000000,0.000000}%
\pgfsetstrokecolor{currentstroke}%
\pgfsetdash{}{0pt}%
\pgfpathmoveto{\pgfqpoint{1.814098in}{0.883146in}}%
\pgfpathlineto{\pgfqpoint{1.817994in}{0.886729in}}%
\pgfpathlineto{\pgfqpoint{1.821901in}{0.890630in}}%
\pgfpathlineto{\pgfqpoint{1.825821in}{0.894855in}}%
\pgfpathlineto{\pgfqpoint{1.829752in}{0.899410in}}%
\pgfpathlineto{\pgfqpoint{1.829545in}{0.888728in}}%
\pgfpathlineto{\pgfqpoint{1.828686in}{0.878040in}}%
\pgfpathlineto{\pgfqpoint{1.827174in}{0.867355in}}%
\pgfpathlineto{\pgfqpoint{1.825007in}{0.856685in}}%
\pgfpathlineto{\pgfqpoint{1.821081in}{0.852363in}}%
\pgfpathlineto{\pgfqpoint{1.817168in}{0.848373in}}%
\pgfpathlineto{\pgfqpoint{1.813266in}{0.844708in}}%
\pgfpathlineto{\pgfqpoint{1.809377in}{0.841362in}}%
\pgfpathlineto{\pgfqpoint{1.811514in}{0.851797in}}%
\pgfpathlineto{\pgfqpoint{1.813012in}{0.862246in}}%
\pgfpathlineto{\pgfqpoint{1.813873in}{0.872699in}}%
\pgfpathlineto{\pgfqpoint{1.814098in}{0.883146in}}%
\pgfpathclose%
\pgfusepath{fill}%
\end{pgfscope}%
\begin{pgfscope}%
\pgfpathrectangle{\pgfqpoint{0.041670in}{0.041670in}}{\pgfqpoint{2.216660in}{2.216660in}}%
\pgfusepath{clip}%
\pgfsetbuttcap%
\pgfsetroundjoin%
\definecolor{currentfill}{rgb}{0.212395,0.359683,0.551710}%
\pgfsetfillcolor{currentfill}%
\pgfsetlinewidth{0.000000pt}%
\definecolor{currentstroke}{rgb}{0.000000,0.000000,0.000000}%
\pgfsetstrokecolor{currentstroke}%
\pgfsetdash{}{0pt}%
\pgfpathmoveto{\pgfqpoint{1.565010in}{1.148821in}}%
\pgfpathlineto{\pgfqpoint{1.568555in}{1.141983in}}%
\pgfpathlineto{\pgfqpoint{1.572097in}{1.135198in}}%
\pgfpathlineto{\pgfqpoint{1.575638in}{1.128468in}}%
\pgfpathlineto{\pgfqpoint{1.579176in}{1.121796in}}%
\pgfpathlineto{\pgfqpoint{1.582049in}{1.115414in}}%
\pgfpathlineto{\pgfqpoint{1.584528in}{1.108984in}}%
\pgfpathlineto{\pgfqpoint{1.586610in}{1.102511in}}%
\pgfpathlineto{\pgfqpoint{1.588291in}{1.096003in}}%
\pgfpathlineto{\pgfqpoint{1.584658in}{1.102928in}}%
\pgfpathlineto{\pgfqpoint{1.581024in}{1.109912in}}%
\pgfpathlineto{\pgfqpoint{1.577388in}{1.116952in}}%
\pgfpathlineto{\pgfqpoint{1.573750in}{1.124043in}}%
\pgfpathlineto{\pgfqpoint{1.572142in}{1.130295in}}%
\pgfpathlineto{\pgfqpoint{1.570147in}{1.136512in}}%
\pgfpathlineto{\pgfqpoint{1.567769in}{1.142690in}}%
\pgfpathlineto{\pgfqpoint{1.565010in}{1.148821in}}%
\pgfpathclose%
\pgfusepath{fill}%
\end{pgfscope}%
\begin{pgfscope}%
\pgfpathrectangle{\pgfqpoint{0.041670in}{0.041670in}}{\pgfqpoint{2.216660in}{2.216660in}}%
\pgfusepath{clip}%
\pgfsetbuttcap%
\pgfsetroundjoin%
\definecolor{currentfill}{rgb}{0.134692,0.658636,0.517649}%
\pgfsetfillcolor{currentfill}%
\pgfsetlinewidth{0.000000pt}%
\definecolor{currentstroke}{rgb}{0.000000,0.000000,0.000000}%
\pgfsetstrokecolor{currentstroke}%
\pgfsetdash{}{0pt}%
\pgfpathmoveto{\pgfqpoint{0.992327in}{1.445962in}}%
\pgfpathlineto{\pgfqpoint{0.989825in}{1.439999in}}%
\pgfpathlineto{\pgfqpoint{0.987326in}{1.433999in}}%
\pgfpathlineto{\pgfqpoint{0.984830in}{1.427964in}}%
\pgfpathlineto{\pgfqpoint{0.982338in}{1.421898in}}%
\pgfpathlineto{\pgfqpoint{0.989137in}{1.424856in}}%
\pgfpathlineto{\pgfqpoint{0.996118in}{1.427708in}}%
\pgfpathlineto{\pgfqpoint{1.003273in}{1.430452in}}%
\pgfpathlineto{\pgfqpoint{1.010595in}{1.433084in}}%
\pgfpathlineto{\pgfqpoint{1.012728in}{1.439000in}}%
\pgfpathlineto{\pgfqpoint{1.014863in}{1.444884in}}%
\pgfpathlineto{\pgfqpoint{1.017001in}{1.450733in}}%
\pgfpathlineto{\pgfqpoint{1.019142in}{1.456545in}}%
\pgfpathlineto{\pgfqpoint{1.012193in}{1.454055in}}%
\pgfpathlineto{\pgfqpoint{1.005402in}{1.451459in}}%
\pgfpathlineto{\pgfqpoint{0.998778in}{1.448761in}}%
\pgfpathlineto{\pgfqpoint{0.992327in}{1.445962in}}%
\pgfpathclose%
\pgfusepath{fill}%
\end{pgfscope}%
\begin{pgfscope}%
\pgfpathrectangle{\pgfqpoint{0.041670in}{0.041670in}}{\pgfqpoint{2.216660in}{2.216660in}}%
\pgfusepath{clip}%
\pgfsetbuttcap%
\pgfsetroundjoin%
\definecolor{currentfill}{rgb}{0.133743,0.548535,0.553541}%
\pgfsetfillcolor{currentfill}%
\pgfsetlinewidth{0.000000pt}%
\definecolor{currentstroke}{rgb}{0.000000,0.000000,0.000000}%
\pgfsetstrokecolor{currentstroke}%
\pgfsetdash{}{0pt}%
\pgfpathmoveto{\pgfqpoint{0.890602in}{1.325137in}}%
\pgfpathlineto{\pgfqpoint{0.887298in}{1.318311in}}%
\pgfpathlineto{\pgfqpoint{0.883998in}{1.311479in}}%
\pgfpathlineto{\pgfqpoint{0.880700in}{1.304643in}}%
\pgfpathlineto{\pgfqpoint{0.877406in}{1.297805in}}%
\pgfpathlineto{\pgfqpoint{0.882209in}{1.302462in}}%
\pgfpathlineto{\pgfqpoint{0.887300in}{1.307038in}}%
\pgfpathlineto{\pgfqpoint{0.892671in}{1.311531in}}%
\pgfpathlineto{\pgfqpoint{0.898319in}{1.315935in}}%
\pgfpathlineto{\pgfqpoint{0.901377in}{1.322558in}}%
\pgfpathlineto{\pgfqpoint{0.904439in}{1.329179in}}%
\pgfpathlineto{\pgfqpoint{0.907504in}{1.335797in}}%
\pgfpathlineto{\pgfqpoint{0.910572in}{1.342409in}}%
\pgfpathlineto{\pgfqpoint{0.905178in}{1.338212in}}%
\pgfpathlineto{\pgfqpoint{0.900048in}{1.333933in}}%
\pgfpathlineto{\pgfqpoint{0.895188in}{1.329573in}}%
\pgfpathlineto{\pgfqpoint{0.890602in}{1.325137in}}%
\pgfpathclose%
\pgfusepath{fill}%
\end{pgfscope}%
\begin{pgfscope}%
\pgfpathrectangle{\pgfqpoint{0.041670in}{0.041670in}}{\pgfqpoint{2.216660in}{2.216660in}}%
\pgfusepath{clip}%
\pgfsetbuttcap%
\pgfsetroundjoin%
\definecolor{currentfill}{rgb}{0.279566,0.067836,0.391917}%
\pgfsetfillcolor{currentfill}%
\pgfsetlinewidth{0.000000pt}%
\definecolor{currentstroke}{rgb}{0.000000,0.000000,0.000000}%
\pgfsetstrokecolor{currentstroke}%
\pgfsetdash{}{0pt}%
\pgfpathmoveto{\pgfqpoint{0.672755in}{0.864903in}}%
\pgfpathlineto{\pgfqpoint{0.669097in}{0.860604in}}%
\pgfpathlineto{\pgfqpoint{0.665435in}{0.856472in}}%
\pgfpathlineto{\pgfqpoint{0.661769in}{0.852511in}}%
\pgfpathlineto{\pgfqpoint{0.658100in}{0.848727in}}%
\pgfpathlineto{\pgfqpoint{0.655723in}{0.857375in}}%
\pgfpathlineto{\pgfqpoint{0.653876in}{0.866047in}}%
\pgfpathlineto{\pgfqpoint{0.652560in}{0.874736in}}%
\pgfpathlineto{\pgfqpoint{0.651775in}{0.883431in}}%
\pgfpathlineto{\pgfqpoint{0.655471in}{0.886955in}}%
\pgfpathlineto{\pgfqpoint{0.659163in}{0.890654in}}%
\pgfpathlineto{\pgfqpoint{0.662852in}{0.894525in}}%
\pgfpathlineto{\pgfqpoint{0.666539in}{0.898562in}}%
\pgfpathlineto{\pgfqpoint{0.667319in}{0.890128in}}%
\pgfpathlineto{\pgfqpoint{0.668616in}{0.881701in}}%
\pgfpathlineto{\pgfqpoint{0.670428in}{0.873290in}}%
\pgfpathlineto{\pgfqpoint{0.672755in}{0.864903in}}%
\pgfpathclose%
\pgfusepath{fill}%
\end{pgfscope}%
\begin{pgfscope}%
\pgfpathrectangle{\pgfqpoint{0.041670in}{0.041670in}}{\pgfqpoint{2.216660in}{2.216660in}}%
\pgfusepath{clip}%
\pgfsetbuttcap%
\pgfsetroundjoin%
\definecolor{currentfill}{rgb}{0.147607,0.511733,0.557049}%
\pgfsetfillcolor{currentfill}%
\pgfsetlinewidth{0.000000pt}%
\definecolor{currentstroke}{rgb}{0.000000,0.000000,0.000000}%
\pgfsetstrokecolor{currentstroke}%
\pgfsetdash{}{0pt}%
\pgfpathmoveto{\pgfqpoint{1.478248in}{1.301948in}}%
\pgfpathlineto{\pgfqpoint{1.481491in}{1.295162in}}%
\pgfpathlineto{\pgfqpoint{1.484730in}{1.288380in}}%
\pgfpathlineto{\pgfqpoint{1.487967in}{1.281605in}}%
\pgfpathlineto{\pgfqpoint{1.491200in}{1.274840in}}%
\pgfpathlineto{\pgfqpoint{1.496188in}{1.269953in}}%
\pgfpathlineto{\pgfqpoint{1.500871in}{1.264987in}}%
\pgfpathlineto{\pgfqpoint{1.505243in}{1.259947in}}%
\pgfpathlineto{\pgfqpoint{1.509301in}{1.254837in}}%
\pgfpathlineto{\pgfqpoint{1.505870in}{1.261830in}}%
\pgfpathlineto{\pgfqpoint{1.502436in}{1.268832in}}%
\pgfpathlineto{\pgfqpoint{1.498998in}{1.275842in}}%
\pgfpathlineto{\pgfqpoint{1.495558in}{1.282856in}}%
\pgfpathlineto{\pgfqpoint{1.491679in}{1.287733in}}%
\pgfpathlineto{\pgfqpoint{1.487498in}{1.292543in}}%
\pgfpathlineto{\pgfqpoint{1.483020in}{1.297283in}}%
\pgfpathlineto{\pgfqpoint{1.478248in}{1.301948in}}%
\pgfpathclose%
\pgfusepath{fill}%
\end{pgfscope}%
\begin{pgfscope}%
\pgfpathrectangle{\pgfqpoint{0.041670in}{0.041670in}}{\pgfqpoint{2.216660in}{2.216660in}}%
\pgfusepath{clip}%
\pgfsetbuttcap%
\pgfsetroundjoin%
\definecolor{currentfill}{rgb}{0.272594,0.025563,0.353093}%
\pgfsetfillcolor{currentfill}%
\pgfsetlinewidth{0.000000pt}%
\definecolor{currentstroke}{rgb}{0.000000,0.000000,0.000000}%
\pgfsetstrokecolor{currentstroke}%
\pgfsetdash{}{0pt}%
\pgfpathmoveto{\pgfqpoint{0.568377in}{0.822028in}}%
\pgfpathlineto{\pgfqpoint{0.564541in}{0.824096in}}%
\pgfpathlineto{\pgfqpoint{0.560694in}{0.826462in}}%
\pgfpathlineto{\pgfqpoint{0.556837in}{0.829131in}}%
\pgfpathlineto{\pgfqpoint{0.552969in}{0.832109in}}%
\pgfpathlineto{\pgfqpoint{0.550264in}{0.842521in}}%
\pgfpathlineto{\pgfqpoint{0.548198in}{0.852957in}}%
\pgfpathlineto{\pgfqpoint{0.546771in}{0.863407in}}%
\pgfpathlineto{\pgfqpoint{0.545981in}{0.873860in}}%
\pgfpathlineto{\pgfqpoint{0.549870in}{0.870643in}}%
\pgfpathlineto{\pgfqpoint{0.553748in}{0.867734in}}%
\pgfpathlineto{\pgfqpoint{0.557616in}{0.865126in}}%
\pgfpathlineto{\pgfqpoint{0.561473in}{0.862816in}}%
\pgfpathlineto{\pgfqpoint{0.562266in}{0.852603in}}%
\pgfpathlineto{\pgfqpoint{0.563680in}{0.842393in}}%
\pgfpathlineto{\pgfqpoint{0.565717in}{0.832198in}}%
\pgfpathlineto{\pgfqpoint{0.568377in}{0.822028in}}%
\pgfpathclose%
\pgfusepath{fill}%
\end{pgfscope}%
\begin{pgfscope}%
\pgfpathrectangle{\pgfqpoint{0.041670in}{0.041670in}}{\pgfqpoint{2.216660in}{2.216660in}}%
\pgfusepath{clip}%
\pgfsetbuttcap%
\pgfsetroundjoin%
\definecolor{currentfill}{rgb}{0.220124,0.725509,0.466226}%
\pgfsetfillcolor{currentfill}%
\pgfsetlinewidth{0.000000pt}%
\definecolor{currentstroke}{rgb}{0.000000,0.000000,0.000000}%
\pgfsetstrokecolor{currentstroke}%
\pgfsetdash{}{0pt}%
\pgfpathmoveto{\pgfqpoint{1.149767in}{1.522762in}}%
\pgfpathlineto{\pgfqpoint{1.149314in}{1.517667in}}%
\pgfpathlineto{\pgfqpoint{1.148861in}{1.512517in}}%
\pgfpathlineto{\pgfqpoint{1.148409in}{1.507314in}}%
\pgfpathlineto{\pgfqpoint{1.147957in}{1.502060in}}%
\pgfpathlineto{\pgfqpoint{1.156044in}{1.502481in}}%
\pgfpathlineto{\pgfqpoint{1.164152in}{1.502779in}}%
\pgfpathlineto{\pgfqpoint{1.172276in}{1.502956in}}%
\pgfpathlineto{\pgfqpoint{1.180407in}{1.503011in}}%
\pgfpathlineto{\pgfqpoint{1.180400in}{1.508250in}}%
\pgfpathlineto{\pgfqpoint{1.180394in}{1.513439in}}%
\pgfpathlineto{\pgfqpoint{1.180387in}{1.518576in}}%
\pgfpathlineto{\pgfqpoint{1.180381in}{1.523657in}}%
\pgfpathlineto{\pgfqpoint{1.172710in}{1.523605in}}%
\pgfpathlineto{\pgfqpoint{1.165046in}{1.523439in}}%
\pgfpathlineto{\pgfqpoint{1.157396in}{1.523158in}}%
\pgfpathlineto{\pgfqpoint{1.149767in}{1.522762in}}%
\pgfpathclose%
\pgfusepath{fill}%
\end{pgfscope}%
\begin{pgfscope}%
\pgfpathrectangle{\pgfqpoint{0.041670in}{0.041670in}}{\pgfqpoint{2.216660in}{2.216660in}}%
\pgfusepath{clip}%
\pgfsetbuttcap%
\pgfsetroundjoin%
\definecolor{currentfill}{rgb}{0.220124,0.725509,0.466226}%
\pgfsetfillcolor{currentfill}%
\pgfsetlinewidth{0.000000pt}%
\definecolor{currentstroke}{rgb}{0.000000,0.000000,0.000000}%
\pgfsetstrokecolor{currentstroke}%
\pgfsetdash{}{0pt}%
\pgfpathmoveto{\pgfqpoint{1.180381in}{1.523657in}}%
\pgfpathlineto{\pgfqpoint{1.180387in}{1.518576in}}%
\pgfpathlineto{\pgfqpoint{1.180394in}{1.513439in}}%
\pgfpathlineto{\pgfqpoint{1.180400in}{1.508250in}}%
\pgfpathlineto{\pgfqpoint{1.180407in}{1.503011in}}%
\pgfpathlineto{\pgfqpoint{1.188537in}{1.502943in}}%
\pgfpathlineto{\pgfqpoint{1.196659in}{1.502752in}}%
\pgfpathlineto{\pgfqpoint{1.204766in}{1.502440in}}%
\pgfpathlineto{\pgfqpoint{1.212849in}{1.502006in}}%
\pgfpathlineto{\pgfqpoint{1.212385in}{1.507260in}}%
\pgfpathlineto{\pgfqpoint{1.211920in}{1.512464in}}%
\pgfpathlineto{\pgfqpoint{1.211455in}{1.517615in}}%
\pgfpathlineto{\pgfqpoint{1.210989in}{1.522711in}}%
\pgfpathlineto{\pgfqpoint{1.203363in}{1.523120in}}%
\pgfpathlineto{\pgfqpoint{1.195715in}{1.523414in}}%
\pgfpathlineto{\pgfqpoint{1.188052in}{1.523593in}}%
\pgfpathlineto{\pgfqpoint{1.180381in}{1.523657in}}%
\pgfpathclose%
\pgfusepath{fill}%
\end{pgfscope}%
\begin{pgfscope}%
\pgfpathrectangle{\pgfqpoint{0.041670in}{0.041670in}}{\pgfqpoint{2.216660in}{2.216660in}}%
\pgfusepath{clip}%
\pgfsetbuttcap%
\pgfsetroundjoin%
\definecolor{currentfill}{rgb}{0.263663,0.237631,0.518762}%
\pgfsetfillcolor{currentfill}%
\pgfsetlinewidth{0.000000pt}%
\definecolor{currentstroke}{rgb}{0.000000,0.000000,0.000000}%
\pgfsetstrokecolor{currentstroke}%
\pgfsetdash{}{0pt}%
\pgfpathmoveto{\pgfqpoint{0.739917in}{1.008410in}}%
\pgfpathlineto{\pgfqpoint{0.736253in}{1.001839in}}%
\pgfpathlineto{\pgfqpoint{0.732591in}{0.995359in}}%
\pgfpathlineto{\pgfqpoint{0.728928in}{0.988975in}}%
\pgfpathlineto{\pgfqpoint{0.725265in}{0.982690in}}%
\pgfpathlineto{\pgfqpoint{0.724968in}{0.990072in}}%
\pgfpathlineto{\pgfqpoint{0.725125in}{0.997448in}}%
\pgfpathlineto{\pgfqpoint{0.725735in}{1.004809in}}%
\pgfpathlineto{\pgfqpoint{0.726796in}{1.012150in}}%
\pgfpathlineto{\pgfqpoint{0.730431in}{1.018173in}}%
\pgfpathlineto{\pgfqpoint{0.734067in}{1.024295in}}%
\pgfpathlineto{\pgfqpoint{0.737703in}{1.030512in}}%
\pgfpathlineto{\pgfqpoint{0.741340in}{1.036821in}}%
\pgfpathlineto{\pgfqpoint{0.740328in}{1.029741in}}%
\pgfpathlineto{\pgfqpoint{0.739752in}{1.022641in}}%
\pgfpathlineto{\pgfqpoint{0.739614in}{1.015529in}}%
\pgfpathlineto{\pgfqpoint{0.739917in}{1.008410in}}%
\pgfpathclose%
\pgfusepath{fill}%
\end{pgfscope}%
\begin{pgfscope}%
\pgfpathrectangle{\pgfqpoint{0.041670in}{0.041670in}}{\pgfqpoint{2.216660in}{2.216660in}}%
\pgfusepath{clip}%
\pgfsetbuttcap%
\pgfsetroundjoin%
\definecolor{currentfill}{rgb}{0.179019,0.433756,0.557430}%
\pgfsetfillcolor{currentfill}%
\pgfsetlinewidth{0.000000pt}%
\definecolor{currentstroke}{rgb}{0.000000,0.000000,0.000000}%
\pgfsetstrokecolor{currentstroke}%
\pgfsetdash{}{0pt}%
\pgfpathmoveto{\pgfqpoint{0.821069in}{1.200015in}}%
\pgfpathlineto{\pgfqpoint{0.817480in}{1.192818in}}%
\pgfpathlineto{\pgfqpoint{0.813894in}{1.185648in}}%
\pgfpathlineto{\pgfqpoint{0.810311in}{1.178509in}}%
\pgfpathlineto{\pgfqpoint{0.806730in}{1.171403in}}%
\pgfpathlineto{\pgfqpoint{0.809415in}{1.177279in}}%
\pgfpathlineto{\pgfqpoint{0.812462in}{1.183105in}}%
\pgfpathlineto{\pgfqpoint{0.815868in}{1.188875in}}%
\pgfpathlineto{\pgfqpoint{0.819628in}{1.194584in}}%
\pgfpathlineto{\pgfqpoint{0.823074in}{1.201445in}}%
\pgfpathlineto{\pgfqpoint{0.826522in}{1.208339in}}%
\pgfpathlineto{\pgfqpoint{0.829973in}{1.215264in}}%
\pgfpathlineto{\pgfqpoint{0.833427in}{1.222216in}}%
\pgfpathlineto{\pgfqpoint{0.829822in}{1.216748in}}%
\pgfpathlineto{\pgfqpoint{0.826558in}{1.211222in}}%
\pgfpathlineto{\pgfqpoint{0.823639in}{1.205642in}}%
\pgfpathlineto{\pgfqpoint{0.821069in}{1.200015in}}%
\pgfpathclose%
\pgfusepath{fill}%
\end{pgfscope}%
\begin{pgfscope}%
\pgfpathrectangle{\pgfqpoint{0.041670in}{0.041670in}}{\pgfqpoint{2.216660in}{2.216660in}}%
\pgfusepath{clip}%
\pgfsetbuttcap%
\pgfsetroundjoin%
\definecolor{currentfill}{rgb}{0.166383,0.690856,0.496502}%
\pgfsetfillcolor{currentfill}%
\pgfsetlinewidth{0.000000pt}%
\definecolor{currentstroke}{rgb}{0.000000,0.000000,0.000000}%
\pgfsetstrokecolor{currentstroke}%
\pgfsetdash{}{0pt}%
\pgfpathmoveto{\pgfqpoint{1.298070in}{1.489372in}}%
\pgfpathlineto{\pgfqpoint{1.299737in}{1.483880in}}%
\pgfpathlineto{\pgfqpoint{1.301402in}{1.478342in}}%
\pgfpathlineto{\pgfqpoint{1.303065in}{1.472760in}}%
\pgfpathlineto{\pgfqpoint{1.304726in}{1.467137in}}%
\pgfpathlineto{\pgfqpoint{1.312355in}{1.465191in}}%
\pgfpathlineto{\pgfqpoint{1.319858in}{1.463130in}}%
\pgfpathlineto{\pgfqpoint{1.327230in}{1.460955in}}%
\pgfpathlineto{\pgfqpoint{1.334462in}{1.458669in}}%
\pgfpathlineto{\pgfqpoint{1.332403in}{1.464412in}}%
\pgfpathlineto{\pgfqpoint{1.330341in}{1.470114in}}%
\pgfpathlineto{\pgfqpoint{1.328277in}{1.475772in}}%
\pgfpathlineto{\pgfqpoint{1.326210in}{1.481383in}}%
\pgfpathlineto{\pgfqpoint{1.319366in}{1.483539in}}%
\pgfpathlineto{\pgfqpoint{1.312391in}{1.485591in}}%
\pgfpathlineto{\pgfqpoint{1.305290in}{1.487536in}}%
\pgfpathlineto{\pgfqpoint{1.298070in}{1.489372in}}%
\pgfpathclose%
\pgfusepath{fill}%
\end{pgfscope}%
\begin{pgfscope}%
\pgfpathrectangle{\pgfqpoint{0.041670in}{0.041670in}}{\pgfqpoint{2.216660in}{2.216660in}}%
\pgfusepath{clip}%
\pgfsetbuttcap%
\pgfsetroundjoin%
\definecolor{currentfill}{rgb}{0.220124,0.725509,0.466226}%
\pgfsetfillcolor{currentfill}%
\pgfsetlinewidth{0.000000pt}%
\definecolor{currentstroke}{rgb}{0.000000,0.000000,0.000000}%
\pgfsetstrokecolor{currentstroke}%
\pgfsetdash{}{0pt}%
\pgfpathmoveto{\pgfqpoint{1.119608in}{1.520042in}}%
\pgfpathlineto{\pgfqpoint{1.118702in}{1.514903in}}%
\pgfpathlineto{\pgfqpoint{1.117796in}{1.509710in}}%
\pgfpathlineto{\pgfqpoint{1.116892in}{1.504464in}}%
\pgfpathlineto{\pgfqpoint{1.115990in}{1.499167in}}%
\pgfpathlineto{\pgfqpoint{1.123910in}{1.500070in}}%
\pgfpathlineto{\pgfqpoint{1.131883in}{1.500854in}}%
\pgfpathlineto{\pgfqpoint{1.139901in}{1.501518in}}%
\pgfpathlineto{\pgfqpoint{1.147957in}{1.502060in}}%
\pgfpathlineto{\pgfqpoint{1.148409in}{1.507314in}}%
\pgfpathlineto{\pgfqpoint{1.148861in}{1.512517in}}%
\pgfpathlineto{\pgfqpoint{1.149314in}{1.517667in}}%
\pgfpathlineto{\pgfqpoint{1.149767in}{1.522762in}}%
\pgfpathlineto{\pgfqpoint{1.142167in}{1.522252in}}%
\pgfpathlineto{\pgfqpoint{1.134602in}{1.521628in}}%
\pgfpathlineto{\pgfqpoint{1.127080in}{1.520891in}}%
\pgfpathlineto{\pgfqpoint{1.119608in}{1.520042in}}%
\pgfpathclose%
\pgfusepath{fill}%
\end{pgfscope}%
\begin{pgfscope}%
\pgfpathrectangle{\pgfqpoint{0.041670in}{0.041670in}}{\pgfqpoint{2.216660in}{2.216660in}}%
\pgfusepath{clip}%
\pgfsetbuttcap%
\pgfsetroundjoin%
\definecolor{currentfill}{rgb}{0.283072,0.130895,0.449241}%
\pgfsetfillcolor{currentfill}%
\pgfsetlinewidth{0.000000pt}%
\definecolor{currentstroke}{rgb}{0.000000,0.000000,0.000000}%
\pgfsetstrokecolor{currentstroke}%
\pgfsetdash{}{0pt}%
\pgfpathmoveto{\pgfqpoint{1.664240in}{0.943432in}}%
\pgfpathlineto{\pgfqpoint{1.667908in}{0.938258in}}%
\pgfpathlineto{\pgfqpoint{1.671577in}{0.933221in}}%
\pgfpathlineto{\pgfqpoint{1.675248in}{0.928322in}}%
\pgfpathlineto{\pgfqpoint{1.678921in}{0.923567in}}%
\pgfpathlineto{\pgfqpoint{1.678590in}{0.915396in}}%
\pgfpathlineto{\pgfqpoint{1.677759in}{0.907224in}}%
\pgfpathlineto{\pgfqpoint{1.676428in}{0.899061in}}%
\pgfpathlineto{\pgfqpoint{1.674596in}{0.890915in}}%
\pgfpathlineto{\pgfqpoint{1.670939in}{0.895933in}}%
\pgfpathlineto{\pgfqpoint{1.667283in}{0.901094in}}%
\pgfpathlineto{\pgfqpoint{1.663630in}{0.906395in}}%
\pgfpathlineto{\pgfqpoint{1.659978in}{0.911832in}}%
\pgfpathlineto{\pgfqpoint{1.661772in}{0.919715in}}%
\pgfpathlineto{\pgfqpoint{1.663080in}{0.927614in}}%
\pgfpathlineto{\pgfqpoint{1.663903in}{0.935523in}}%
\pgfpathlineto{\pgfqpoint{1.664240in}{0.943432in}}%
\pgfpathclose%
\pgfusepath{fill}%
\end{pgfscope}%
\begin{pgfscope}%
\pgfpathrectangle{\pgfqpoint{0.041670in}{0.041670in}}{\pgfqpoint{2.216660in}{2.216660in}}%
\pgfusepath{clip}%
\pgfsetbuttcap%
\pgfsetroundjoin%
\definecolor{currentfill}{rgb}{0.220124,0.725509,0.466226}%
\pgfsetfillcolor{currentfill}%
\pgfsetlinewidth{0.000000pt}%
\definecolor{currentstroke}{rgb}{0.000000,0.000000,0.000000}%
\pgfsetstrokecolor{currentstroke}%
\pgfsetdash{}{0pt}%
\pgfpathmoveto{\pgfqpoint{1.210989in}{1.522711in}}%
\pgfpathlineto{\pgfqpoint{1.211455in}{1.517615in}}%
\pgfpathlineto{\pgfqpoint{1.211920in}{1.512464in}}%
\pgfpathlineto{\pgfqpoint{1.212385in}{1.507260in}}%
\pgfpathlineto{\pgfqpoint{1.212849in}{1.502006in}}%
\pgfpathlineto{\pgfqpoint{1.220901in}{1.501450in}}%
\pgfpathlineto{\pgfqpoint{1.228915in}{1.500773in}}%
\pgfpathlineto{\pgfqpoint{1.236883in}{1.499976in}}%
\pgfpathlineto{\pgfqpoint{1.244797in}{1.499059in}}%
\pgfpathlineto{\pgfqpoint{1.243881in}{1.504358in}}%
\pgfpathlineto{\pgfqpoint{1.242965in}{1.509605in}}%
\pgfpathlineto{\pgfqpoint{1.242047in}{1.514800in}}%
\pgfpathlineto{\pgfqpoint{1.241128in}{1.519940in}}%
\pgfpathlineto{\pgfqpoint{1.233662in}{1.520802in}}%
\pgfpathlineto{\pgfqpoint{1.226146in}{1.521552in}}%
\pgfpathlineto{\pgfqpoint{1.218586in}{1.522189in}}%
\pgfpathlineto{\pgfqpoint{1.210989in}{1.522711in}}%
\pgfpathclose%
\pgfusepath{fill}%
\end{pgfscope}%
\begin{pgfscope}%
\pgfpathrectangle{\pgfqpoint{0.041670in}{0.041670in}}{\pgfqpoint{2.216660in}{2.216660in}}%
\pgfusepath{clip}%
\pgfsetbuttcap%
\pgfsetroundjoin%
\definecolor{currentfill}{rgb}{0.212395,0.359683,0.551710}%
\pgfsetfillcolor{currentfill}%
\pgfsetlinewidth{0.000000pt}%
\definecolor{currentstroke}{rgb}{0.000000,0.000000,0.000000}%
\pgfsetstrokecolor{currentstroke}%
\pgfsetdash{}{0pt}%
\pgfpathmoveto{\pgfqpoint{0.785057in}{1.118461in}}%
\pgfpathlineto{\pgfqpoint{0.781406in}{1.111313in}}%
\pgfpathlineto{\pgfqpoint{0.777756in}{1.104217in}}%
\pgfpathlineto{\pgfqpoint{0.774109in}{1.097175in}}%
\pgfpathlineto{\pgfqpoint{0.770463in}{1.090192in}}%
\pgfpathlineto{\pgfqpoint{0.771786in}{1.096727in}}%
\pgfpathlineto{\pgfqpoint{0.773512in}{1.103232in}}%
\pgfpathlineto{\pgfqpoint{0.775638in}{1.109701in}}%
\pgfpathlineto{\pgfqpoint{0.778160in}{1.116126in}}%
\pgfpathlineto{\pgfqpoint{0.781725in}{1.122853in}}%
\pgfpathlineto{\pgfqpoint{0.785291in}{1.129639in}}%
\pgfpathlineto{\pgfqpoint{0.788859in}{1.136480in}}%
\pgfpathlineto{\pgfqpoint{0.792429in}{1.143373in}}%
\pgfpathlineto{\pgfqpoint{0.790008in}{1.137201in}}%
\pgfpathlineto{\pgfqpoint{0.787971in}{1.130987in}}%
\pgfpathlineto{\pgfqpoint{0.786319in}{1.124739in}}%
\pgfpathlineto{\pgfqpoint{0.785057in}{1.118461in}}%
\pgfpathclose%
\pgfusepath{fill}%
\end{pgfscope}%
\begin{pgfscope}%
\pgfpathrectangle{\pgfqpoint{0.041670in}{0.041670in}}{\pgfqpoint{2.216660in}{2.216660in}}%
\pgfusepath{clip}%
\pgfsetbuttcap%
\pgfsetroundjoin%
\definecolor{currentfill}{rgb}{0.248629,0.278775,0.534556}%
\pgfsetfillcolor{currentfill}%
\pgfsetlinewidth{0.000000pt}%
\definecolor{currentstroke}{rgb}{0.000000,0.000000,0.000000}%
\pgfsetstrokecolor{currentstroke}%
\pgfsetdash{}{0pt}%
\pgfpathmoveto{\pgfqpoint{1.602807in}{1.068949in}}%
\pgfpathlineto{\pgfqpoint{1.606434in}{1.062364in}}%
\pgfpathlineto{\pgfqpoint{1.610059in}{1.055857in}}%
\pgfpathlineto{\pgfqpoint{1.613683in}{1.049432in}}%
\pgfpathlineto{\pgfqpoint{1.617306in}{1.043092in}}%
\pgfpathlineto{\pgfqpoint{1.618704in}{1.036036in}}%
\pgfpathlineto{\pgfqpoint{1.619668in}{1.028953in}}%
\pgfpathlineto{\pgfqpoint{1.620195in}{1.021852in}}%
\pgfpathlineto{\pgfqpoint{1.620283in}{1.014738in}}%
\pgfpathlineto{\pgfqpoint{1.616621in}{1.021339in}}%
\pgfpathlineto{\pgfqpoint{1.612958in}{1.028025in}}%
\pgfpathlineto{\pgfqpoint{1.609294in}{1.034793in}}%
\pgfpathlineto{\pgfqpoint{1.605629in}{1.041638in}}%
\pgfpathlineto{\pgfqpoint{1.605558in}{1.048490in}}%
\pgfpathlineto{\pgfqpoint{1.605062in}{1.055329in}}%
\pgfpathlineto{\pgfqpoint{1.604145in}{1.062151in}}%
\pgfpathlineto{\pgfqpoint{1.602807in}{1.068949in}}%
\pgfpathclose%
\pgfusepath{fill}%
\end{pgfscope}%
\begin{pgfscope}%
\pgfpathrectangle{\pgfqpoint{0.041670in}{0.041670in}}{\pgfqpoint{2.216660in}{2.216660in}}%
\pgfusepath{clip}%
\pgfsetbuttcap%
\pgfsetroundjoin%
\definecolor{currentfill}{rgb}{0.260571,0.246922,0.522828}%
\pgfsetfillcolor{currentfill}%
\pgfsetlinewidth{0.000000pt}%
\definecolor{currentstroke}{rgb}{0.000000,0.000000,0.000000}%
\pgfsetstrokecolor{currentstroke}%
\pgfsetdash{}{0pt}%
\pgfpathmoveto{\pgfqpoint{0.481988in}{0.971872in}}%
\pgfpathlineto{\pgfqpoint{0.477857in}{0.981220in}}%
\pgfpathlineto{\pgfqpoint{0.473709in}{0.990986in}}%
\pgfpathlineto{\pgfqpoint{0.469542in}{1.001178in}}%
\pgfpathlineto{\pgfqpoint{0.465356in}{1.011802in}}%
\pgfpathlineto{\pgfqpoint{0.465313in}{1.023349in}}%
\pgfpathlineto{\pgfqpoint{0.465983in}{1.034873in}}%
\pgfpathlineto{\pgfqpoint{0.467361in}{1.046361in}}%
\pgfpathlineto{\pgfqpoint{0.469443in}{1.057803in}}%
\pgfpathlineto{\pgfqpoint{0.473579in}{1.046979in}}%
\pgfpathlineto{\pgfqpoint{0.477695in}{1.036586in}}%
\pgfpathlineto{\pgfqpoint{0.481794in}{1.026617in}}%
\pgfpathlineto{\pgfqpoint{0.485876in}{1.017063in}}%
\pgfpathlineto{\pgfqpoint{0.483868in}{1.005821in}}%
\pgfpathlineto{\pgfqpoint{0.482548in}{0.994535in}}%
\pgfpathlineto{\pgfqpoint{0.481920in}{0.983215in}}%
\pgfpathlineto{\pgfqpoint{0.481988in}{0.971872in}}%
\pgfpathclose%
\pgfusepath{fill}%
\end{pgfscope}%
\begin{pgfscope}%
\pgfpathrectangle{\pgfqpoint{0.041670in}{0.041670in}}{\pgfqpoint{2.216660in}{2.216660in}}%
\pgfusepath{clip}%
\pgfsetbuttcap%
\pgfsetroundjoin%
\definecolor{currentfill}{rgb}{0.166383,0.690856,0.496502}%
\pgfsetfillcolor{currentfill}%
\pgfsetlinewidth{0.000000pt}%
\definecolor{currentstroke}{rgb}{0.000000,0.000000,0.000000}%
\pgfsetstrokecolor{currentstroke}%
\pgfsetdash{}{0pt}%
\pgfpathmoveto{\pgfqpoint{1.027734in}{1.479380in}}%
\pgfpathlineto{\pgfqpoint{1.025582in}{1.473738in}}%
\pgfpathlineto{\pgfqpoint{1.023432in}{1.468050in}}%
\pgfpathlineto{\pgfqpoint{1.021286in}{1.462319in}}%
\pgfpathlineto{\pgfqpoint{1.019142in}{1.456545in}}%
\pgfpathlineto{\pgfqpoint{1.026244in}{1.458929in}}%
\pgfpathlineto{\pgfqpoint{1.033492in}{1.461202in}}%
\pgfpathlineto{\pgfqpoint{1.040879in}{1.463364in}}%
\pgfpathlineto{\pgfqpoint{1.048397in}{1.465413in}}%
\pgfpathlineto{\pgfqpoint{1.050148in}{1.471061in}}%
\pgfpathlineto{\pgfqpoint{1.051902in}{1.476667in}}%
\pgfpathlineto{\pgfqpoint{1.053658in}{1.482229in}}%
\pgfpathlineto{\pgfqpoint{1.055416in}{1.487745in}}%
\pgfpathlineto{\pgfqpoint{1.048302in}{1.485812in}}%
\pgfpathlineto{\pgfqpoint{1.041312in}{1.483773in}}%
\pgfpathlineto{\pgfqpoint{1.034454in}{1.481628in}}%
\pgfpathlineto{\pgfqpoint{1.027734in}{1.479380in}}%
\pgfpathclose%
\pgfusepath{fill}%
\end{pgfscope}%
\begin{pgfscope}%
\pgfpathrectangle{\pgfqpoint{0.041670in}{0.041670in}}{\pgfqpoint{2.216660in}{2.216660in}}%
\pgfusepath{clip}%
\pgfsetbuttcap%
\pgfsetroundjoin%
\definecolor{currentfill}{rgb}{0.282327,0.094955,0.417331}%
\pgfsetfillcolor{currentfill}%
\pgfsetlinewidth{0.000000pt}%
\definecolor{currentstroke}{rgb}{0.000000,0.000000,0.000000}%
\pgfsetstrokecolor{currentstroke}%
\pgfsetdash{}{0pt}%
\pgfpathmoveto{\pgfqpoint{0.687363in}{0.883695in}}%
\pgfpathlineto{\pgfqpoint{0.683715in}{0.878766in}}%
\pgfpathlineto{\pgfqpoint{0.680064in}{0.873988in}}%
\pgfpathlineto{\pgfqpoint{0.676411in}{0.869366in}}%
\pgfpathlineto{\pgfqpoint{0.672755in}{0.864903in}}%
\pgfpathlineto{\pgfqpoint{0.670428in}{0.873290in}}%
\pgfpathlineto{\pgfqpoint{0.668616in}{0.881701in}}%
\pgfpathlineto{\pgfqpoint{0.667319in}{0.890128in}}%
\pgfpathlineto{\pgfqpoint{0.666539in}{0.898562in}}%
\pgfpathlineto{\pgfqpoint{0.670222in}{0.902763in}}%
\pgfpathlineto{\pgfqpoint{0.673903in}{0.907123in}}%
\pgfpathlineto{\pgfqpoint{0.677582in}{0.911638in}}%
\pgfpathlineto{\pgfqpoint{0.681258in}{0.916304in}}%
\pgfpathlineto{\pgfqpoint{0.682033in}{0.908132in}}%
\pgfpathlineto{\pgfqpoint{0.683309in}{0.899967in}}%
\pgfpathlineto{\pgfqpoint{0.685086in}{0.891819in}}%
\pgfpathlineto{\pgfqpoint{0.687363in}{0.883695in}}%
\pgfpathclose%
\pgfusepath{fill}%
\end{pgfscope}%
\begin{pgfscope}%
\pgfpathrectangle{\pgfqpoint{0.041670in}{0.041670in}}{\pgfqpoint{2.216660in}{2.216660in}}%
\pgfusepath{clip}%
\pgfsetbuttcap%
\pgfsetroundjoin%
\definecolor{currentfill}{rgb}{0.147607,0.511733,0.557049}%
\pgfsetfillcolor{currentfill}%
\pgfsetlinewidth{0.000000pt}%
\definecolor{currentstroke}{rgb}{0.000000,0.000000,0.000000}%
\pgfsetstrokecolor{currentstroke}%
\pgfsetdash{}{0pt}%
\pgfpathmoveto{\pgfqpoint{0.861162in}{1.278469in}}%
\pgfpathlineto{\pgfqpoint{0.857684in}{1.271403in}}%
\pgfpathlineto{\pgfqpoint{0.854210in}{1.264341in}}%
\pgfpathlineto{\pgfqpoint{0.850738in}{1.257286in}}%
\pgfpathlineto{\pgfqpoint{0.847270in}{1.250240in}}%
\pgfpathlineto{\pgfqpoint{0.851044in}{1.255408in}}%
\pgfpathlineto{\pgfqpoint{0.855137in}{1.260510in}}%
\pgfpathlineto{\pgfqpoint{0.859544in}{1.265543in}}%
\pgfpathlineto{\pgfqpoint{0.864261in}{1.270500in}}%
\pgfpathlineto{\pgfqpoint{0.867543in}{1.277314in}}%
\pgfpathlineto{\pgfqpoint{0.870827in}{1.284138in}}%
\pgfpathlineto{\pgfqpoint{0.874115in}{1.290970in}}%
\pgfpathlineto{\pgfqpoint{0.877406in}{1.297805in}}%
\pgfpathlineto{\pgfqpoint{0.872895in}{1.293073in}}%
\pgfpathlineto{\pgfqpoint{0.868680in}{1.288270in}}%
\pgfpathlineto{\pgfqpoint{0.864768in}{1.283401in}}%
\pgfpathlineto{\pgfqpoint{0.861162in}{1.278469in}}%
\pgfpathclose%
\pgfusepath{fill}%
\end{pgfscope}%
\begin{pgfscope}%
\pgfpathrectangle{\pgfqpoint{0.041670in}{0.041670in}}{\pgfqpoint{2.216660in}{2.216660in}}%
\pgfusepath{clip}%
\pgfsetbuttcap%
\pgfsetroundjoin%
\definecolor{currentfill}{rgb}{0.282327,0.094955,0.417331}%
\pgfsetfillcolor{currentfill}%
\pgfsetlinewidth{0.000000pt}%
\definecolor{currentstroke}{rgb}{0.000000,0.000000,0.000000}%
\pgfsetstrokecolor{currentstroke}%
\pgfsetdash{}{0pt}%
\pgfpathmoveto{\pgfqpoint{1.829752in}{0.899410in}}%
\pgfpathlineto{\pgfqpoint{1.833696in}{0.904300in}}%
\pgfpathlineto{\pgfqpoint{1.837653in}{0.909531in}}%
\pgfpathlineto{\pgfqpoint{1.841623in}{0.915110in}}%
\pgfpathlineto{\pgfqpoint{1.845607in}{0.921041in}}%
\pgfpathlineto{\pgfqpoint{1.845418in}{0.910132in}}%
\pgfpathlineto{\pgfqpoint{1.844563in}{0.899214in}}%
\pgfpathlineto{\pgfqpoint{1.843037in}{0.888300in}}%
\pgfpathlineto{\pgfqpoint{1.840841in}{0.877399in}}%
\pgfpathlineto{\pgfqpoint{1.836862in}{0.871694in}}%
\pgfpathlineto{\pgfqpoint{1.832897in}{0.866345in}}%
\pgfpathlineto{\pgfqpoint{1.828945in}{0.861343in}}%
\pgfpathlineto{\pgfqpoint{1.825007in}{0.856685in}}%
\pgfpathlineto{\pgfqpoint{1.827174in}{0.867355in}}%
\pgfpathlineto{\pgfqpoint{1.828686in}{0.878040in}}%
\pgfpathlineto{\pgfqpoint{1.829545in}{0.888728in}}%
\pgfpathlineto{\pgfqpoint{1.829752in}{0.899410in}}%
\pgfpathclose%
\pgfusepath{fill}%
\end{pgfscope}%
\begin{pgfscope}%
\pgfpathrectangle{\pgfqpoint{0.041670in}{0.041670in}}{\pgfqpoint{2.216660in}{2.216660in}}%
\pgfusepath{clip}%
\pgfsetbuttcap%
\pgfsetroundjoin%
\definecolor{currentfill}{rgb}{0.172719,0.448791,0.557885}%
\pgfsetfillcolor{currentfill}%
\pgfsetlinewidth{0.000000pt}%
\definecolor{currentstroke}{rgb}{0.000000,0.000000,0.000000}%
\pgfsetstrokecolor{currentstroke}%
\pgfsetdash{}{0pt}%
\pgfpathmoveto{\pgfqpoint{0.435637in}{1.160759in}}%
\pgfpathlineto{\pgfqpoint{0.431312in}{1.175791in}}%
\pgfpathlineto{\pgfqpoint{0.426964in}{1.191330in}}%
\pgfpathlineto{\pgfqpoint{0.422592in}{1.207385in}}%
\pgfpathlineto{\pgfqpoint{0.425636in}{1.219261in}}%
\pgfpathlineto{\pgfqpoint{0.429417in}{1.231062in}}%
\pgfpathlineto{\pgfqpoint{0.433927in}{1.242779in}}%
\pgfpathlineto{\pgfqpoint{0.439158in}{1.254400in}}%
\pgfpathlineto{\pgfqpoint{0.443406in}{1.238184in}}%
\pgfpathlineto{\pgfqpoint{0.447630in}{1.222481in}}%
\pgfpathlineto{\pgfqpoint{0.451832in}{1.207283in}}%
\pgfpathlineto{\pgfqpoint{0.446712in}{1.195782in}}%
\pgfpathlineto{\pgfqpoint{0.442301in}{1.184188in}}%
\pgfpathlineto{\pgfqpoint{0.438606in}{1.172510in}}%
\pgfpathlineto{\pgfqpoint{0.435637in}{1.160759in}}%
\pgfpathclose%
\pgfusepath{fill}%
\end{pgfscope}%
\begin{pgfscope}%
\pgfpathrectangle{\pgfqpoint{0.041670in}{0.041670in}}{\pgfqpoint{2.216660in}{2.216660in}}%
\pgfusepath{clip}%
\pgfsetbuttcap%
\pgfsetroundjoin%
\definecolor{currentfill}{rgb}{0.220124,0.725509,0.466226}%
\pgfsetfillcolor{currentfill}%
\pgfsetlinewidth{0.000000pt}%
\definecolor{currentstroke}{rgb}{0.000000,0.000000,0.000000}%
\pgfsetstrokecolor{currentstroke}%
\pgfsetdash{}{0pt}%
\pgfpathmoveto{\pgfqpoint{1.241128in}{1.519940in}}%
\pgfpathlineto{\pgfqpoint{1.242047in}{1.514800in}}%
\pgfpathlineto{\pgfqpoint{1.242965in}{1.509605in}}%
\pgfpathlineto{\pgfqpoint{1.243881in}{1.504358in}}%
\pgfpathlineto{\pgfqpoint{1.244797in}{1.499059in}}%
\pgfpathlineto{\pgfqpoint{1.252649in}{1.498023in}}%
\pgfpathlineto{\pgfqpoint{1.260434in}{1.496870in}}%
\pgfpathlineto{\pgfqpoint{1.268142in}{1.495599in}}%
\pgfpathlineto{\pgfqpoint{1.266896in}{1.500949in}}%
\pgfpathlineto{\pgfqpoint{1.265649in}{1.506249in}}%
\pgfpathlineto{\pgfqpoint{1.264401in}{1.511495in}}%
\pgfpathlineto{\pgfqpoint{1.263150in}{1.516686in}}%
\pgfpathlineto{\pgfqpoint{1.255879in}{1.517881in}}%
\pgfpathlineto{\pgfqpoint{1.248536in}{1.518966in}}%
\pgfpathlineto{\pgfqpoint{1.241128in}{1.519940in}}%
\pgfpathclose%
\pgfusepath{fill}%
\end{pgfscope}%
\begin{pgfscope}%
\pgfpathrectangle{\pgfqpoint{0.041670in}{0.041670in}}{\pgfqpoint{2.216660in}{2.216660in}}%
\pgfusepath{clip}%
\pgfsetbuttcap%
\pgfsetroundjoin%
\definecolor{currentfill}{rgb}{0.277941,0.056324,0.381191}%
\pgfsetfillcolor{currentfill}%
\pgfsetlinewidth{0.000000pt}%
\definecolor{currentstroke}{rgb}{0.000000,0.000000,0.000000}%
\pgfsetstrokecolor{currentstroke}%
\pgfsetdash{}{0pt}%
\pgfpathmoveto{\pgfqpoint{0.552969in}{0.832109in}}%
\pgfpathlineto{\pgfqpoint{0.549090in}{0.835401in}}%
\pgfpathlineto{\pgfqpoint{0.545198in}{0.839014in}}%
\pgfpathlineto{\pgfqpoint{0.541295in}{0.842952in}}%
\pgfpathlineto{\pgfqpoint{0.537379in}{0.847221in}}%
\pgfpathlineto{\pgfqpoint{0.534630in}{0.857869in}}%
\pgfpathlineto{\pgfqpoint{0.532536in}{0.868542in}}%
\pgfpathlineto{\pgfqpoint{0.531096in}{0.879227in}}%
\pgfpathlineto{\pgfqpoint{0.530310in}{0.889916in}}%
\pgfpathlineto{\pgfqpoint{0.534246in}{0.885413in}}%
\pgfpathlineto{\pgfqpoint{0.538169in}{0.881239in}}%
\pgfpathlineto{\pgfqpoint{0.542081in}{0.877390in}}%
\pgfpathlineto{\pgfqpoint{0.545981in}{0.873860in}}%
\pgfpathlineto{\pgfqpoint{0.546771in}{0.863407in}}%
\pgfpathlineto{\pgfqpoint{0.548198in}{0.852957in}}%
\pgfpathlineto{\pgfqpoint{0.550264in}{0.842521in}}%
\pgfpathlineto{\pgfqpoint{0.552969in}{0.832109in}}%
\pgfpathclose%
\pgfusepath{fill}%
\end{pgfscope}%
\begin{pgfscope}%
\pgfpathrectangle{\pgfqpoint{0.041670in}{0.041670in}}{\pgfqpoint{2.216660in}{2.216660in}}%
\pgfusepath{clip}%
\pgfsetbuttcap%
\pgfsetroundjoin%
\definecolor{currentfill}{rgb}{0.220124,0.725509,0.466226}%
\pgfsetfillcolor{currentfill}%
\pgfsetlinewidth{0.000000pt}%
\definecolor{currentstroke}{rgb}{0.000000,0.000000,0.000000}%
\pgfsetstrokecolor{currentstroke}%
\pgfsetdash{}{0pt}%
\pgfpathmoveto{\pgfqpoint{1.090362in}{1.515533in}}%
\pgfpathlineto{\pgfqpoint{1.089016in}{1.510323in}}%
\pgfpathlineto{\pgfqpoint{1.087670in}{1.505059in}}%
\pgfpathlineto{\pgfqpoint{1.086327in}{1.499741in}}%
\pgfpathlineto{\pgfqpoint{1.084986in}{1.494372in}}%
\pgfpathlineto{\pgfqpoint{1.092620in}{1.495746in}}%
\pgfpathlineto{\pgfqpoint{1.100337in}{1.497004in}}%
\pgfpathlineto{\pgfqpoint{1.108130in}{1.498144in}}%
\pgfpathlineto{\pgfqpoint{1.115990in}{1.499167in}}%
\pgfpathlineto{\pgfqpoint{1.116892in}{1.504464in}}%
\pgfpathlineto{\pgfqpoint{1.117796in}{1.509710in}}%
\pgfpathlineto{\pgfqpoint{1.118702in}{1.514903in}}%
\pgfpathlineto{\pgfqpoint{1.119608in}{1.520042in}}%
\pgfpathlineto{\pgfqpoint{1.112194in}{1.519080in}}%
\pgfpathlineto{\pgfqpoint{1.104843in}{1.518007in}}%
\pgfpathlineto{\pgfqpoint{1.097564in}{1.516825in}}%
\pgfpathlineto{\pgfqpoint{1.090362in}{1.515533in}}%
\pgfpathclose%
\pgfusepath{fill}%
\end{pgfscope}%
\begin{pgfscope}%
\pgfpathrectangle{\pgfqpoint{0.041670in}{0.041670in}}{\pgfqpoint{2.216660in}{2.216660in}}%
\pgfusepath{clip}%
\pgfsetbuttcap%
\pgfsetroundjoin%
\definecolor{currentfill}{rgb}{0.120081,0.622161,0.534946}%
\pgfsetfillcolor{currentfill}%
\pgfsetlinewidth{0.000000pt}%
\definecolor{currentstroke}{rgb}{0.000000,0.000000,0.000000}%
\pgfsetstrokecolor{currentstroke}%
\pgfsetdash{}{0pt}%
\pgfpathmoveto{\pgfqpoint{1.397495in}{1.412044in}}%
\pgfpathlineto{\pgfqpoint{1.400239in}{1.405815in}}%
\pgfpathlineto{\pgfqpoint{1.402980in}{1.399558in}}%
\pgfpathlineto{\pgfqpoint{1.405718in}{1.393277in}}%
\pgfpathlineto{\pgfqpoint{1.408453in}{1.386973in}}%
\pgfpathlineto{\pgfqpoint{1.414746in}{1.383425in}}%
\pgfpathlineto{\pgfqpoint{1.420815in}{1.379779in}}%
\pgfpathlineto{\pgfqpoint{1.426653in}{1.376041in}}%
\pgfpathlineto{\pgfqpoint{1.432255in}{1.372212in}}%
\pgfpathlineto{\pgfqpoint{1.429229in}{1.378706in}}%
\pgfpathlineto{\pgfqpoint{1.426199in}{1.385178in}}%
\pgfpathlineto{\pgfqpoint{1.423166in}{1.391625in}}%
\pgfpathlineto{\pgfqpoint{1.420130in}{1.398045in}}%
\pgfpathlineto{\pgfqpoint{1.414803in}{1.401676in}}%
\pgfpathlineto{\pgfqpoint{1.409252in}{1.405221in}}%
\pgfpathlineto{\pgfqpoint{1.403480in}{1.408679in}}%
\pgfpathlineto{\pgfqpoint{1.397495in}{1.412044in}}%
\pgfpathclose%
\pgfusepath{fill}%
\end{pgfscope}%
\begin{pgfscope}%
\pgfpathrectangle{\pgfqpoint{0.041670in}{0.041670in}}{\pgfqpoint{2.216660in}{2.216660in}}%
\pgfusepath{clip}%
\pgfsetbuttcap%
\pgfsetroundjoin%
\definecolor{currentfill}{rgb}{0.134692,0.658636,0.517649}%
\pgfsetfillcolor{currentfill}%
\pgfsetlinewidth{0.000000pt}%
\definecolor{currentstroke}{rgb}{0.000000,0.000000,0.000000}%
\pgfsetstrokecolor{currentstroke}%
\pgfsetdash{}{0pt}%
\pgfpathmoveto{\pgfqpoint{1.361857in}{1.448455in}}%
\pgfpathlineto{\pgfqpoint{1.364282in}{1.442527in}}%
\pgfpathlineto{\pgfqpoint{1.366703in}{1.436563in}}%
\pgfpathlineto{\pgfqpoint{1.369122in}{1.430564in}}%
\pgfpathlineto{\pgfqpoint{1.371537in}{1.424533in}}%
\pgfpathlineto{\pgfqpoint{1.378316in}{1.421563in}}%
\pgfpathlineto{\pgfqpoint{1.384906in}{1.418489in}}%
\pgfpathlineto{\pgfqpoint{1.391301in}{1.415316in}}%
\pgfpathlineto{\pgfqpoint{1.397495in}{1.412044in}}%
\pgfpathlineto{\pgfqpoint{1.394747in}{1.418244in}}%
\pgfpathlineto{\pgfqpoint{1.391995in}{1.424412in}}%
\pgfpathlineto{\pgfqpoint{1.389241in}{1.430545in}}%
\pgfpathlineto{\pgfqpoint{1.386483in}{1.436641in}}%
\pgfpathlineto{\pgfqpoint{1.380608in}{1.439735in}}%
\pgfpathlineto{\pgfqpoint{1.374541in}{1.442737in}}%
\pgfpathlineto{\pgfqpoint{1.368289in}{1.445645in}}%
\pgfpathlineto{\pgfqpoint{1.361857in}{1.448455in}}%
\pgfpathclose%
\pgfusepath{fill}%
\end{pgfscope}%
\begin{pgfscope}%
\pgfpathrectangle{\pgfqpoint{0.041670in}{0.041670in}}{\pgfqpoint{2.216660in}{2.216660in}}%
\pgfusepath{clip}%
\pgfsetbuttcap%
\pgfsetroundjoin%
\definecolor{currentfill}{rgb}{0.195860,0.395433,0.555276}%
\pgfsetfillcolor{currentfill}%
\pgfsetlinewidth{0.000000pt}%
\definecolor{currentstroke}{rgb}{0.000000,0.000000,0.000000}%
\pgfsetstrokecolor{currentstroke}%
\pgfsetdash{}{0pt}%
\pgfpathmoveto{\pgfqpoint{1.550811in}{1.176628in}}%
\pgfpathlineto{\pgfqpoint{1.554364in}{1.169614in}}%
\pgfpathlineto{\pgfqpoint{1.557915in}{1.162640in}}%
\pgfpathlineto{\pgfqpoint{1.561464in}{1.155707in}}%
\pgfpathlineto{\pgfqpoint{1.565010in}{1.148821in}}%
\pgfpathlineto{\pgfqpoint{1.567769in}{1.142690in}}%
\pgfpathlineto{\pgfqpoint{1.570147in}{1.136512in}}%
\pgfpathlineto{\pgfqpoint{1.572142in}{1.130295in}}%
\pgfpathlineto{\pgfqpoint{1.573750in}{1.124043in}}%
\pgfpathlineto{\pgfqpoint{1.570110in}{1.131182in}}%
\pgfpathlineto{\pgfqpoint{1.566468in}{1.138368in}}%
\pgfpathlineto{\pgfqpoint{1.562825in}{1.145595in}}%
\pgfpathlineto{\pgfqpoint{1.559179in}{1.152862in}}%
\pgfpathlineto{\pgfqpoint{1.557642in}{1.158858in}}%
\pgfpathlineto{\pgfqpoint{1.555734in}{1.164822in}}%
\pgfpathlineto{\pgfqpoint{1.553456in}{1.170747in}}%
\pgfpathlineto{\pgfqpoint{1.550811in}{1.176628in}}%
\pgfpathclose%
\pgfusepath{fill}%
\end{pgfscope}%
\begin{pgfscope}%
\pgfpathrectangle{\pgfqpoint{0.041670in}{0.041670in}}{\pgfqpoint{2.216660in}{2.216660in}}%
\pgfusepath{clip}%
\pgfsetbuttcap%
\pgfsetroundjoin%
\definecolor{currentfill}{rgb}{0.163625,0.471133,0.558148}%
\pgfsetfillcolor{currentfill}%
\pgfsetlinewidth{0.000000pt}%
\definecolor{currentstroke}{rgb}{0.000000,0.000000,0.000000}%
\pgfsetstrokecolor{currentstroke}%
\pgfsetdash{}{0pt}%
\pgfpathmoveto{\pgfqpoint{1.509301in}{1.254837in}}%
\pgfpathlineto{\pgfqpoint{1.512729in}{1.247857in}}%
\pgfpathlineto{\pgfqpoint{1.516154in}{1.240893in}}%
\pgfpathlineto{\pgfqpoint{1.519577in}{1.233947in}}%
\pgfpathlineto{\pgfqpoint{1.522996in}{1.227023in}}%
\pgfpathlineto{\pgfqpoint{1.526900in}{1.221612in}}%
\pgfpathlineto{\pgfqpoint{1.530467in}{1.216137in}}%
\pgfpathlineto{\pgfqpoint{1.533693in}{1.210604in}}%
\pgfpathlineto{\pgfqpoint{1.536573in}{1.205019in}}%
\pgfpathlineto{\pgfqpoint{1.533007in}{1.212185in}}%
\pgfpathlineto{\pgfqpoint{1.529439in}{1.219372in}}%
\pgfpathlineto{\pgfqpoint{1.525867in}{1.226578in}}%
\pgfpathlineto{\pgfqpoint{1.522293in}{1.233799in}}%
\pgfpathlineto{\pgfqpoint{1.519539in}{1.239138in}}%
\pgfpathlineto{\pgfqpoint{1.516453in}{1.244428in}}%
\pgfpathlineto{\pgfqpoint{1.513039in}{1.249662in}}%
\pgfpathlineto{\pgfqpoint{1.509301in}{1.254837in}}%
\pgfpathclose%
\pgfusepath{fill}%
\end{pgfscope}%
\begin{pgfscope}%
\pgfpathrectangle{\pgfqpoint{0.041670in}{0.041670in}}{\pgfqpoint{2.216660in}{2.216660in}}%
\pgfusepath{clip}%
\pgfsetbuttcap%
\pgfsetroundjoin%
\definecolor{currentfill}{rgb}{0.233603,0.313828,0.543914}%
\pgfsetfillcolor{currentfill}%
\pgfsetlinewidth{0.000000pt}%
\definecolor{currentstroke}{rgb}{0.000000,0.000000,0.000000}%
\pgfsetstrokecolor{currentstroke}%
\pgfsetdash{}{0pt}%
\pgfpathmoveto{\pgfqpoint{1.888029in}{1.067925in}}%
\pgfpathlineto{\pgfqpoint{1.892163in}{1.079228in}}%
\pgfpathlineto{\pgfqpoint{1.896316in}{1.090976in}}%
\pgfpathlineto{\pgfqpoint{1.900490in}{1.103175in}}%
\pgfpathlineto{\pgfqpoint{1.904683in}{1.115832in}}%
\pgfpathlineto{\pgfqpoint{1.907477in}{1.104252in}}%
\pgfpathlineto{\pgfqpoint{1.909557in}{1.092614in}}%
\pgfpathlineto{\pgfqpoint{1.910915in}{1.080929in}}%
\pgfpathlineto{\pgfqpoint{1.911549in}{1.069208in}}%
\pgfpathlineto{\pgfqpoint{1.907287in}{1.056738in}}%
\pgfpathlineto{\pgfqpoint{1.903047in}{1.044729in}}%
\pgfpathlineto{\pgfqpoint{1.898827in}{1.033175in}}%
\pgfpathlineto{\pgfqpoint{1.894628in}{1.022067in}}%
\pgfpathlineto{\pgfqpoint{1.894037in}{1.033594in}}%
\pgfpathlineto{\pgfqpoint{1.892737in}{1.045086in}}%
\pgfpathlineto{\pgfqpoint{1.890732in}{1.056534in}}%
\pgfpathlineto{\pgfqpoint{1.888029in}{1.067925in}}%
\pgfpathclose%
\pgfusepath{fill}%
\end{pgfscope}%
\begin{pgfscope}%
\pgfpathrectangle{\pgfqpoint{0.041670in}{0.041670in}}{\pgfqpoint{2.216660in}{2.216660in}}%
\pgfusepath{clip}%
\pgfsetbuttcap%
\pgfsetroundjoin%
\definecolor{currentfill}{rgb}{0.122606,0.585371,0.546557}%
\pgfsetfillcolor{currentfill}%
\pgfsetlinewidth{0.000000pt}%
\definecolor{currentstroke}{rgb}{0.000000,0.000000,0.000000}%
\pgfsetstrokecolor{currentstroke}%
\pgfsetdash{}{0pt}%
\pgfpathmoveto{\pgfqpoint{1.432255in}{1.372212in}}%
\pgfpathlineto{\pgfqpoint{1.435278in}{1.365698in}}%
\pgfpathlineto{\pgfqpoint{1.438298in}{1.359167in}}%
\pgfpathlineto{\pgfqpoint{1.441314in}{1.352622in}}%
\pgfpathlineto{\pgfqpoint{1.444326in}{1.346065in}}%
\pgfpathlineto{\pgfqpoint{1.449950in}{1.341947in}}%
\pgfpathlineto{\pgfqpoint{1.455315in}{1.337741in}}%
\pgfpathlineto{\pgfqpoint{1.460415in}{1.333452in}}%
\pgfpathlineto{\pgfqpoint{1.465245in}{1.329084in}}%
\pgfpathlineto{\pgfqpoint{1.461986in}{1.335851in}}%
\pgfpathlineto{\pgfqpoint{1.458723in}{1.342606in}}%
\pgfpathlineto{\pgfqpoint{1.455457in}{1.349347in}}%
\pgfpathlineto{\pgfqpoint{1.452188in}{1.356071in}}%
\pgfpathlineto{\pgfqpoint{1.447587in}{1.360223in}}%
\pgfpathlineto{\pgfqpoint{1.442728in}{1.364299in}}%
\pgfpathlineto{\pgfqpoint{1.437615in}{1.368297in}}%
\pgfpathlineto{\pgfqpoint{1.432255in}{1.372212in}}%
\pgfpathclose%
\pgfusepath{fill}%
\end{pgfscope}%
\begin{pgfscope}%
\pgfpathrectangle{\pgfqpoint{0.041670in}{0.041670in}}{\pgfqpoint{2.216660in}{2.216660in}}%
\pgfusepath{clip}%
\pgfsetbuttcap%
\pgfsetroundjoin%
\definecolor{currentfill}{rgb}{0.280255,0.165693,0.476498}%
\pgfsetfillcolor{currentfill}%
\pgfsetlinewidth{0.000000pt}%
\definecolor{currentstroke}{rgb}{0.000000,0.000000,0.000000}%
\pgfsetstrokecolor{currentstroke}%
\pgfsetdash{}{0pt}%
\pgfpathmoveto{\pgfqpoint{1.649580in}{0.965406in}}%
\pgfpathlineto{\pgfqpoint{1.653244in}{0.959728in}}%
\pgfpathlineto{\pgfqpoint{1.656908in}{0.954170in}}%
\pgfpathlineto{\pgfqpoint{1.660574in}{0.948737in}}%
\pgfpathlineto{\pgfqpoint{1.664240in}{0.943432in}}%
\pgfpathlineto{\pgfqpoint{1.663903in}{0.935523in}}%
\pgfpathlineto{\pgfqpoint{1.663080in}{0.927614in}}%
\pgfpathlineto{\pgfqpoint{1.661772in}{0.919715in}}%
\pgfpathlineto{\pgfqpoint{1.659978in}{0.911832in}}%
\pgfpathlineto{\pgfqpoint{1.656328in}{0.917401in}}%
\pgfpathlineto{\pgfqpoint{1.652679in}{0.923098in}}%
\pgfpathlineto{\pgfqpoint{1.649031in}{0.928919in}}%
\pgfpathlineto{\pgfqpoint{1.645385in}{0.934861in}}%
\pgfpathlineto{\pgfqpoint{1.647140in}{0.942480in}}%
\pgfpathlineto{\pgfqpoint{1.648424in}{0.950116in}}%
\pgfpathlineto{\pgfqpoint{1.649237in}{0.957760in}}%
\pgfpathlineto{\pgfqpoint{1.649580in}{0.965406in}}%
\pgfpathclose%
\pgfusepath{fill}%
\end{pgfscope}%
\begin{pgfscope}%
\pgfpathrectangle{\pgfqpoint{0.041670in}{0.041670in}}{\pgfqpoint{2.216660in}{2.216660in}}%
\pgfusepath{clip}%
\pgfsetbuttcap%
\pgfsetroundjoin%
\definecolor{currentfill}{rgb}{0.283072,0.130895,0.449241}%
\pgfsetfillcolor{currentfill}%
\pgfsetlinewidth{0.000000pt}%
\definecolor{currentstroke}{rgb}{0.000000,0.000000,0.000000}%
\pgfsetstrokecolor{currentstroke}%
\pgfsetdash{}{0pt}%
\pgfpathmoveto{\pgfqpoint{0.701934in}{0.904846in}}%
\pgfpathlineto{\pgfqpoint{0.698294in}{0.899350in}}%
\pgfpathlineto{\pgfqpoint{0.694652in}{0.893991in}}%
\pgfpathlineto{\pgfqpoint{0.691009in}{0.888771in}}%
\pgfpathlineto{\pgfqpoint{0.687363in}{0.883695in}}%
\pgfpathlineto{\pgfqpoint{0.685086in}{0.891819in}}%
\pgfpathlineto{\pgfqpoint{0.683309in}{0.899967in}}%
\pgfpathlineto{\pgfqpoint{0.682033in}{0.908132in}}%
\pgfpathlineto{\pgfqpoint{0.681258in}{0.916304in}}%
\pgfpathlineto{\pgfqpoint{0.684933in}{0.921117in}}%
\pgfpathlineto{\pgfqpoint{0.688605in}{0.926074in}}%
\pgfpathlineto{\pgfqpoint{0.692276in}{0.931170in}}%
\pgfpathlineto{\pgfqpoint{0.695946in}{0.936402in}}%
\pgfpathlineto{\pgfqpoint{0.696714in}{0.928493in}}%
\pgfpathlineto{\pgfqpoint{0.697968in}{0.920592in}}%
\pgfpathlineto{\pgfqpoint{0.699708in}{0.912707in}}%
\pgfpathlineto{\pgfqpoint{0.701934in}{0.904846in}}%
\pgfpathclose%
\pgfusepath{fill}%
\end{pgfscope}%
\begin{pgfscope}%
\pgfpathrectangle{\pgfqpoint{0.041670in}{0.041670in}}{\pgfqpoint{2.216660in}{2.216660in}}%
\pgfusepath{clip}%
\pgfsetbuttcap%
\pgfsetroundjoin%
\definecolor{currentfill}{rgb}{0.248629,0.278775,0.534556}%
\pgfsetfillcolor{currentfill}%
\pgfsetlinewidth{0.000000pt}%
\definecolor{currentstroke}{rgb}{0.000000,0.000000,0.000000}%
\pgfsetstrokecolor{currentstroke}%
\pgfsetdash{}{0pt}%
\pgfpathmoveto{\pgfqpoint{0.754574in}{1.035545in}}%
\pgfpathlineto{\pgfqpoint{0.750908in}{1.028641in}}%
\pgfpathlineto{\pgfqpoint{0.747244in}{1.021814in}}%
\pgfpathlineto{\pgfqpoint{0.743580in}{1.015070in}}%
\pgfpathlineto{\pgfqpoint{0.739917in}{1.008410in}}%
\pgfpathlineto{\pgfqpoint{0.739614in}{1.015529in}}%
\pgfpathlineto{\pgfqpoint{0.739752in}{1.022641in}}%
\pgfpathlineto{\pgfqpoint{0.740328in}{1.029741in}}%
\pgfpathlineto{\pgfqpoint{0.741340in}{1.036821in}}%
\pgfpathlineto{\pgfqpoint{0.744977in}{1.043219in}}%
\pgfpathlineto{\pgfqpoint{0.748614in}{1.049701in}}%
\pgfpathlineto{\pgfqpoint{0.752253in}{1.056266in}}%
\pgfpathlineto{\pgfqpoint{0.755893in}{1.062908in}}%
\pgfpathlineto{\pgfqpoint{0.754928in}{1.056088in}}%
\pgfpathlineto{\pgfqpoint{0.754386in}{1.049250in}}%
\pgfpathlineto{\pgfqpoint{0.754267in}{1.042400in}}%
\pgfpathlineto{\pgfqpoint{0.754574in}{1.035545in}}%
\pgfpathclose%
\pgfusepath{fill}%
\end{pgfscope}%
\begin{pgfscope}%
\pgfpathrectangle{\pgfqpoint{0.041670in}{0.041670in}}{\pgfqpoint{2.216660in}{2.216660in}}%
\pgfusepath{clip}%
\pgfsetbuttcap%
\pgfsetroundjoin%
\definecolor{currentfill}{rgb}{0.120081,0.622161,0.534946}%
\pgfsetfillcolor{currentfill}%
\pgfsetlinewidth{0.000000pt}%
\definecolor{currentstroke}{rgb}{0.000000,0.000000,0.000000}%
\pgfsetstrokecolor{currentstroke}%
\pgfsetdash{}{0pt}%
\pgfpathmoveto{\pgfqpoint{0.935239in}{1.394749in}}%
\pgfpathlineto{\pgfqpoint{0.932144in}{1.388284in}}%
\pgfpathlineto{\pgfqpoint{0.929051in}{1.381792in}}%
\pgfpathlineto{\pgfqpoint{0.925963in}{1.375275in}}%
\pgfpathlineto{\pgfqpoint{0.922878in}{1.368736in}}%
\pgfpathlineto{\pgfqpoint{0.928265in}{1.372642in}}%
\pgfpathlineto{\pgfqpoint{0.933894in}{1.376460in}}%
\pgfpathlineto{\pgfqpoint{0.939758in}{1.380189in}}%
\pgfpathlineto{\pgfqpoint{0.945852in}{1.383824in}}%
\pgfpathlineto{\pgfqpoint{0.948655in}{1.390168in}}%
\pgfpathlineto{\pgfqpoint{0.951461in}{1.396490in}}%
\pgfpathlineto{\pgfqpoint{0.954271in}{1.402787in}}%
\pgfpathlineto{\pgfqpoint{0.957084in}{1.409057in}}%
\pgfpathlineto{\pgfqpoint{0.951289in}{1.405610in}}%
\pgfpathlineto{\pgfqpoint{0.945712in}{1.402074in}}%
\pgfpathlineto{\pgfqpoint{0.940360in}{1.398452in}}%
\pgfpathlineto{\pgfqpoint{0.935239in}{1.394749in}}%
\pgfpathclose%
\pgfusepath{fill}%
\end{pgfscope}%
\begin{pgfscope}%
\pgfpathrectangle{\pgfqpoint{0.041670in}{0.041670in}}{\pgfqpoint{2.216660in}{2.216660in}}%
\pgfusepath{clip}%
\pgfsetbuttcap%
\pgfsetroundjoin%
\definecolor{currentfill}{rgb}{0.134692,0.658636,0.517649}%
\pgfsetfillcolor{currentfill}%
\pgfsetlinewidth{0.000000pt}%
\definecolor{currentstroke}{rgb}{0.000000,0.000000,0.000000}%
\pgfsetstrokecolor{currentstroke}%
\pgfsetdash{}{0pt}%
\pgfpathmoveto{\pgfqpoint{0.968371in}{1.433815in}}%
\pgfpathlineto{\pgfqpoint{0.965544in}{1.427679in}}%
\pgfpathlineto{\pgfqpoint{0.962721in}{1.421506in}}%
\pgfpathlineto{\pgfqpoint{0.959901in}{1.415297in}}%
\pgfpathlineto{\pgfqpoint{0.957084in}{1.409057in}}%
\pgfpathlineto{\pgfqpoint{0.963093in}{1.412413in}}%
\pgfpathlineto{\pgfqpoint{0.969309in}{1.415673in}}%
\pgfpathlineto{\pgfqpoint{0.975726in}{1.418836in}}%
\pgfpathlineto{\pgfqpoint{0.982338in}{1.421898in}}%
\pgfpathlineto{\pgfqpoint{0.984830in}{1.427964in}}%
\pgfpathlineto{\pgfqpoint{0.987326in}{1.433999in}}%
\pgfpathlineto{\pgfqpoint{0.989825in}{1.439999in}}%
\pgfpathlineto{\pgfqpoint{0.992327in}{1.445962in}}%
\pgfpathlineto{\pgfqpoint{0.986054in}{1.443065in}}%
\pgfpathlineto{\pgfqpoint{0.979966in}{1.440073in}}%
\pgfpathlineto{\pgfqpoint{0.974070in}{1.436989in}}%
\pgfpathlineto{\pgfqpoint{0.968371in}{1.433815in}}%
\pgfpathclose%
\pgfusepath{fill}%
\end{pgfscope}%
\begin{pgfscope}%
\pgfpathrectangle{\pgfqpoint{0.041670in}{0.041670in}}{\pgfqpoint{2.216660in}{2.216660in}}%
\pgfusepath{clip}%
\pgfsetbuttcap%
\pgfsetroundjoin%
\definecolor{currentfill}{rgb}{0.220124,0.725509,0.466226}%
\pgfsetfillcolor{currentfill}%
\pgfsetlinewidth{0.000000pt}%
\definecolor{currentstroke}{rgb}{0.000000,0.000000,0.000000}%
\pgfsetstrokecolor{currentstroke}%
\pgfsetdash{}{0pt}%
\pgfpathmoveto{\pgfqpoint{1.263150in}{1.516686in}}%
\pgfpathlineto{\pgfqpoint{1.264401in}{1.511495in}}%
\pgfpathlineto{\pgfqpoint{1.265649in}{1.506249in}}%
\pgfpathlineto{\pgfqpoint{1.266896in}{1.500949in}}%
\pgfpathlineto{\pgfqpoint{1.268142in}{1.495599in}}%
\pgfpathlineto{\pgfqpoint{1.275767in}{1.494213in}}%
\pgfpathlineto{\pgfqpoint{1.283301in}{1.492712in}}%
\pgfpathlineto{\pgfqpoint{1.290738in}{1.491098in}}%
\pgfpathlineto{\pgfqpoint{1.298070in}{1.489372in}}%
\pgfpathlineto{\pgfqpoint{1.296400in}{1.494815in}}%
\pgfpathlineto{\pgfqpoint{1.294728in}{1.500207in}}%
\pgfpathlineto{\pgfqpoint{1.293054in}{1.505547in}}%
\pgfpathlineto{\pgfqpoint{1.291378in}{1.510831in}}%
\pgfpathlineto{\pgfqpoint{1.284463in}{1.512453in}}%
\pgfpathlineto{\pgfqpoint{1.277449in}{1.513971in}}%
\pgfpathlineto{\pgfqpoint{1.270342in}{1.515383in}}%
\pgfpathlineto{\pgfqpoint{1.263150in}{1.516686in}}%
\pgfpathclose%
\pgfusepath{fill}%
\end{pgfscope}%
\begin{pgfscope}%
\pgfpathrectangle{\pgfqpoint{0.041670in}{0.041670in}}{\pgfqpoint{2.216660in}{2.216660in}}%
\pgfusepath{clip}%
\pgfsetbuttcap%
\pgfsetroundjoin%
\definecolor{currentfill}{rgb}{0.166383,0.690856,0.496502}%
\pgfsetfillcolor{currentfill}%
\pgfsetlinewidth{0.000000pt}%
\definecolor{currentstroke}{rgb}{0.000000,0.000000,0.000000}%
\pgfsetstrokecolor{currentstroke}%
\pgfsetdash{}{0pt}%
\pgfpathmoveto{\pgfqpoint{1.326210in}{1.481383in}}%
\pgfpathlineto{\pgfqpoint{1.328277in}{1.475772in}}%
\pgfpathlineto{\pgfqpoint{1.330341in}{1.470114in}}%
\pgfpathlineto{\pgfqpoint{1.332403in}{1.464412in}}%
\pgfpathlineto{\pgfqpoint{1.334462in}{1.458669in}}%
\pgfpathlineto{\pgfqpoint{1.341547in}{1.456274in}}%
\pgfpathlineto{\pgfqpoint{1.348480in}{1.453772in}}%
\pgfpathlineto{\pgfqpoint{1.355252in}{1.451164in}}%
\pgfpathlineto{\pgfqpoint{1.361857in}{1.448455in}}%
\pgfpathlineto{\pgfqpoint{1.359429in}{1.454343in}}%
\pgfpathlineto{\pgfqpoint{1.356998in}{1.460189in}}%
\pgfpathlineto{\pgfqpoint{1.354564in}{1.465992in}}%
\pgfpathlineto{\pgfqpoint{1.352127in}{1.471748in}}%
\pgfpathlineto{\pgfqpoint{1.345879in}{1.474304in}}%
\pgfpathlineto{\pgfqpoint{1.339472in}{1.476763in}}%
\pgfpathlineto{\pgfqpoint{1.332914in}{1.479123in}}%
\pgfpathlineto{\pgfqpoint{1.326210in}{1.481383in}}%
\pgfpathclose%
\pgfusepath{fill}%
\end{pgfscope}%
\begin{pgfscope}%
\pgfpathrectangle{\pgfqpoint{0.041670in}{0.041670in}}{\pgfqpoint{2.216660in}{2.216660in}}%
\pgfusepath{clip}%
\pgfsetbuttcap%
\pgfsetroundjoin%
\definecolor{currentfill}{rgb}{0.220124,0.725509,0.466226}%
\pgfsetfillcolor{currentfill}%
\pgfsetlinewidth{0.000000pt}%
\definecolor{currentstroke}{rgb}{0.000000,0.000000,0.000000}%
\pgfsetstrokecolor{currentstroke}%
\pgfsetdash{}{0pt}%
\pgfpathmoveto{\pgfqpoint{1.062474in}{1.509301in}}%
\pgfpathlineto{\pgfqpoint{1.060706in}{1.503993in}}%
\pgfpathlineto{\pgfqpoint{1.058940in}{1.498629in}}%
\pgfpathlineto{\pgfqpoint{1.057177in}{1.493212in}}%
\pgfpathlineto{\pgfqpoint{1.055416in}{1.487745in}}%
\pgfpathlineto{\pgfqpoint{1.062649in}{1.489569in}}%
\pgfpathlineto{\pgfqpoint{1.069993in}{1.491282in}}%
\pgfpathlineto{\pgfqpoint{1.077441in}{1.492884in}}%
\pgfpathlineto{\pgfqpoint{1.084986in}{1.494372in}}%
\pgfpathlineto{\pgfqpoint{1.086327in}{1.499741in}}%
\pgfpathlineto{\pgfqpoint{1.087670in}{1.505059in}}%
\pgfpathlineto{\pgfqpoint{1.089016in}{1.510323in}}%
\pgfpathlineto{\pgfqpoint{1.090362in}{1.515533in}}%
\pgfpathlineto{\pgfqpoint{1.083246in}{1.514133in}}%
\pgfpathlineto{\pgfqpoint{1.076221in}{1.512627in}}%
\pgfpathlineto{\pgfqpoint{1.069295in}{1.511016in}}%
\pgfpathlineto{\pgfqpoint{1.062474in}{1.509301in}}%
\pgfpathclose%
\pgfusepath{fill}%
\end{pgfscope}%
\begin{pgfscope}%
\pgfpathrectangle{\pgfqpoint{0.041670in}{0.041670in}}{\pgfqpoint{2.216660in}{2.216660in}}%
\pgfusepath{clip}%
\pgfsetbuttcap%
\pgfsetroundjoin%
\definecolor{currentfill}{rgb}{0.122606,0.585371,0.546557}%
\pgfsetfillcolor{currentfill}%
\pgfsetlinewidth{0.000000pt}%
\definecolor{currentstroke}{rgb}{0.000000,0.000000,0.000000}%
\pgfsetstrokecolor{currentstroke}%
\pgfsetdash{}{0pt}%
\pgfpathmoveto{\pgfqpoint{0.903853in}{1.352320in}}%
\pgfpathlineto{\pgfqpoint{0.900535in}{1.345548in}}%
\pgfpathlineto{\pgfqpoint{0.897221in}{1.338758in}}%
\pgfpathlineto{\pgfqpoint{0.893910in}{1.331954in}}%
\pgfpathlineto{\pgfqpoint{0.890602in}{1.325137in}}%
\pgfpathlineto{\pgfqpoint{0.895188in}{1.329573in}}%
\pgfpathlineto{\pgfqpoint{0.900048in}{1.333933in}}%
\pgfpathlineto{\pgfqpoint{0.905178in}{1.338212in}}%
\pgfpathlineto{\pgfqpoint{0.910572in}{1.342409in}}%
\pgfpathlineto{\pgfqpoint{0.913643in}{1.349011in}}%
\pgfpathlineto{\pgfqpoint{0.916718in}{1.355601in}}%
\pgfpathlineto{\pgfqpoint{0.919796in}{1.362177in}}%
\pgfpathlineto{\pgfqpoint{0.922878in}{1.368736in}}%
\pgfpathlineto{\pgfqpoint{0.917738in}{1.364748in}}%
\pgfpathlineto{\pgfqpoint{0.912850in}{1.360679in}}%
\pgfpathlineto{\pgfqpoint{0.908220in}{1.356536in}}%
\pgfpathlineto{\pgfqpoint{0.903853in}{1.352320in}}%
\pgfpathclose%
\pgfusepath{fill}%
\end{pgfscope}%
\begin{pgfscope}%
\pgfpathrectangle{\pgfqpoint{0.041670in}{0.041670in}}{\pgfqpoint{2.216660in}{2.216660in}}%
\pgfusepath{clip}%
\pgfsetbuttcap%
\pgfsetroundjoin%
\definecolor{currentfill}{rgb}{0.231674,0.318106,0.544834}%
\pgfsetfillcolor{currentfill}%
\pgfsetlinewidth{0.000000pt}%
\definecolor{currentstroke}{rgb}{0.000000,0.000000,0.000000}%
\pgfsetstrokecolor{currentstroke}%
\pgfsetdash{}{0pt}%
\pgfpathmoveto{\pgfqpoint{1.588291in}{1.096003in}}%
\pgfpathlineto{\pgfqpoint{1.591922in}{1.089139in}}%
\pgfpathlineto{\pgfqpoint{1.595552in}{1.082339in}}%
\pgfpathlineto{\pgfqpoint{1.599180in}{1.075608in}}%
\pgfpathlineto{\pgfqpoint{1.602807in}{1.068949in}}%
\pgfpathlineto{\pgfqpoint{1.604145in}{1.062151in}}%
\pgfpathlineto{\pgfqpoint{1.605062in}{1.055329in}}%
\pgfpathlineto{\pgfqpoint{1.605558in}{1.048490in}}%
\pgfpathlineto{\pgfqpoint{1.605629in}{1.041638in}}%
\pgfpathlineto{\pgfqpoint{1.601964in}{1.048559in}}%
\pgfpathlineto{\pgfqpoint{1.598297in}{1.055551in}}%
\pgfpathlineto{\pgfqpoint{1.594630in}{1.062611in}}%
\pgfpathlineto{\pgfqpoint{1.590961in}{1.069735in}}%
\pgfpathlineto{\pgfqpoint{1.590906in}{1.076324in}}%
\pgfpathlineto{\pgfqpoint{1.590441in}{1.082903in}}%
\pgfpathlineto{\pgfqpoint{1.589569in}{1.089464in}}%
\pgfpathlineto{\pgfqpoint{1.588291in}{1.096003in}}%
\pgfpathclose%
\pgfusepath{fill}%
\end{pgfscope}%
\begin{pgfscope}%
\pgfpathrectangle{\pgfqpoint{0.041670in}{0.041670in}}{\pgfqpoint{2.216660in}{2.216660in}}%
\pgfusepath{clip}%
\pgfsetbuttcap%
\pgfsetroundjoin%
\definecolor{currentfill}{rgb}{0.163625,0.471133,0.558148}%
\pgfsetfillcolor{currentfill}%
\pgfsetlinewidth{0.000000pt}%
\definecolor{currentstroke}{rgb}{0.000000,0.000000,0.000000}%
\pgfsetstrokecolor{currentstroke}%
\pgfsetdash{}{0pt}%
\pgfpathmoveto{\pgfqpoint{0.835451in}{1.229016in}}%
\pgfpathlineto{\pgfqpoint{0.831851in}{1.221739in}}%
\pgfpathlineto{\pgfqpoint{0.828254in}{1.214479in}}%
\pgfpathlineto{\pgfqpoint{0.824660in}{1.207236in}}%
\pgfpathlineto{\pgfqpoint{0.821069in}{1.200015in}}%
\pgfpathlineto{\pgfqpoint{0.823639in}{1.205642in}}%
\pgfpathlineto{\pgfqpoint{0.826558in}{1.211222in}}%
\pgfpathlineto{\pgfqpoint{0.829822in}{1.216748in}}%
\pgfpathlineto{\pgfqpoint{0.833427in}{1.222216in}}%
\pgfpathlineto{\pgfqpoint{0.836883in}{1.229193in}}%
\pgfpathlineto{\pgfqpoint{0.840343in}{1.236191in}}%
\pgfpathlineto{\pgfqpoint{0.843805in}{1.243208in}}%
\pgfpathlineto{\pgfqpoint{0.847270in}{1.250240in}}%
\pgfpathlineto{\pgfqpoint{0.843820in}{1.245012in}}%
\pgfpathlineto{\pgfqpoint{0.840697in}{1.239729in}}%
\pgfpathlineto{\pgfqpoint{0.837906in}{1.234395in}}%
\pgfpathlineto{\pgfqpoint{0.835451in}{1.229016in}}%
\pgfpathclose%
\pgfusepath{fill}%
\end{pgfscope}%
\begin{pgfscope}%
\pgfpathrectangle{\pgfqpoint{0.041670in}{0.041670in}}{\pgfqpoint{2.216660in}{2.216660in}}%
\pgfusepath{clip}%
\pgfsetbuttcap%
\pgfsetroundjoin%
\definecolor{currentfill}{rgb}{0.133743,0.548535,0.553541}%
\pgfsetfillcolor{currentfill}%
\pgfsetlinewidth{0.000000pt}%
\definecolor{currentstroke}{rgb}{0.000000,0.000000,0.000000}%
\pgfsetstrokecolor{currentstroke}%
\pgfsetdash{}{0pt}%
\pgfpathmoveto{\pgfqpoint{1.465245in}{1.329084in}}%
\pgfpathlineto{\pgfqpoint{1.468501in}{1.322307in}}%
\pgfpathlineto{\pgfqpoint{1.471753in}{1.315524in}}%
\pgfpathlineto{\pgfqpoint{1.475002in}{1.308737in}}%
\pgfpathlineto{\pgfqpoint{1.478248in}{1.301948in}}%
\pgfpathlineto{\pgfqpoint{1.483020in}{1.297283in}}%
\pgfpathlineto{\pgfqpoint{1.487498in}{1.292543in}}%
\pgfpathlineto{\pgfqpoint{1.491679in}{1.287733in}}%
\pgfpathlineto{\pgfqpoint{1.495558in}{1.282856in}}%
\pgfpathlineto{\pgfqpoint{1.492114in}{1.289871in}}%
\pgfpathlineto{\pgfqpoint{1.488667in}{1.296885in}}%
\pgfpathlineto{\pgfqpoint{1.485217in}{1.303895in}}%
\pgfpathlineto{\pgfqpoint{1.481763in}{1.310898in}}%
\pgfpathlineto{\pgfqpoint{1.478063in}{1.315543in}}%
\pgfpathlineto{\pgfqpoint{1.474074in}{1.320125in}}%
\pgfpathlineto{\pgfqpoint{1.469800in}{1.324640in}}%
\pgfpathlineto{\pgfqpoint{1.465245in}{1.329084in}}%
\pgfpathclose%
\pgfusepath{fill}%
\end{pgfscope}%
\begin{pgfscope}%
\pgfpathrectangle{\pgfqpoint{0.041670in}{0.041670in}}{\pgfqpoint{2.216660in}{2.216660in}}%
\pgfusepath{clip}%
\pgfsetbuttcap%
\pgfsetroundjoin%
\definecolor{currentfill}{rgb}{0.195860,0.395433,0.555276}%
\pgfsetfillcolor{currentfill}%
\pgfsetlinewidth{0.000000pt}%
\definecolor{currentstroke}{rgb}{0.000000,0.000000,0.000000}%
\pgfsetstrokecolor{currentstroke}%
\pgfsetdash{}{0pt}%
\pgfpathmoveto{\pgfqpoint{0.799680in}{1.147510in}}%
\pgfpathlineto{\pgfqpoint{0.796021in}{1.140186in}}%
\pgfpathlineto{\pgfqpoint{0.792364in}{1.132901in}}%
\pgfpathlineto{\pgfqpoint{0.788710in}{1.125658in}}%
\pgfpathlineto{\pgfqpoint{0.785057in}{1.118461in}}%
\pgfpathlineto{\pgfqpoint{0.786319in}{1.124739in}}%
\pgfpathlineto{\pgfqpoint{0.787971in}{1.130987in}}%
\pgfpathlineto{\pgfqpoint{0.790008in}{1.137201in}}%
\pgfpathlineto{\pgfqpoint{0.792429in}{1.143373in}}%
\pgfpathlineto{\pgfqpoint{0.796001in}{1.150315in}}%
\pgfpathlineto{\pgfqpoint{0.799575in}{1.157303in}}%
\pgfpathlineto{\pgfqpoint{0.803151in}{1.164333in}}%
\pgfpathlineto{\pgfqpoint{0.806730in}{1.171403in}}%
\pgfpathlineto{\pgfqpoint{0.804411in}{1.165482in}}%
\pgfpathlineto{\pgfqpoint{0.802461in}{1.159523in}}%
\pgfpathlineto{\pgfqpoint{0.800883in}{1.153530in}}%
\pgfpathlineto{\pgfqpoint{0.799680in}{1.147510in}}%
\pgfpathclose%
\pgfusepath{fill}%
\end{pgfscope}%
\begin{pgfscope}%
\pgfpathrectangle{\pgfqpoint{0.041670in}{0.041670in}}{\pgfqpoint{2.216660in}{2.216660in}}%
\pgfusepath{clip}%
\pgfsetbuttcap%
\pgfsetroundjoin%
\definecolor{currentfill}{rgb}{0.282884,0.135920,0.453427}%
\pgfsetfillcolor{currentfill}%
\pgfsetlinewidth{0.000000pt}%
\definecolor{currentstroke}{rgb}{0.000000,0.000000,0.000000}%
\pgfsetstrokecolor{currentstroke}%
\pgfsetdash{}{0pt}%
\pgfpathmoveto{\pgfqpoint{1.845607in}{0.921041in}}%
\pgfpathlineto{\pgfqpoint{1.849604in}{0.927332in}}%
\pgfpathlineto{\pgfqpoint{1.853617in}{0.933989in}}%
\pgfpathlineto{\pgfqpoint{1.857644in}{0.941017in}}%
\pgfpathlineto{\pgfqpoint{1.861686in}{0.948423in}}%
\pgfpathlineto{\pgfqpoint{1.861518in}{0.937292in}}%
\pgfpathlineto{\pgfqpoint{1.860667in}{0.926152in}}%
\pgfpathlineto{\pgfqpoint{1.859130in}{0.915014in}}%
\pgfpathlineto{\pgfqpoint{1.856905in}{0.903890in}}%
\pgfpathlineto{\pgfqpoint{1.852866in}{0.896704in}}%
\pgfpathlineto{\pgfqpoint{1.848843in}{0.889898in}}%
\pgfpathlineto{\pgfqpoint{1.844835in}{0.883465in}}%
\pgfpathlineto{\pgfqpoint{1.840841in}{0.877399in}}%
\pgfpathlineto{\pgfqpoint{1.843037in}{0.888300in}}%
\pgfpathlineto{\pgfqpoint{1.844563in}{0.899214in}}%
\pgfpathlineto{\pgfqpoint{1.845418in}{0.910132in}}%
\pgfpathlineto{\pgfqpoint{1.845607in}{0.921041in}}%
\pgfpathclose%
\pgfusepath{fill}%
\end{pgfscope}%
\begin{pgfscope}%
\pgfpathrectangle{\pgfqpoint{0.041670in}{0.041670in}}{\pgfqpoint{2.216660in}{2.216660in}}%
\pgfusepath{clip}%
\pgfsetbuttcap%
\pgfsetroundjoin%
\definecolor{currentfill}{rgb}{0.282327,0.094955,0.417331}%
\pgfsetfillcolor{currentfill}%
\pgfsetlinewidth{0.000000pt}%
\definecolor{currentstroke}{rgb}{0.000000,0.000000,0.000000}%
\pgfsetstrokecolor{currentstroke}%
\pgfsetdash{}{0pt}%
\pgfpathmoveto{\pgfqpoint{0.537379in}{0.847221in}}%
\pgfpathlineto{\pgfqpoint{0.533451in}{0.851828in}}%
\pgfpathlineto{\pgfqpoint{0.529509in}{0.856778in}}%
\pgfpathlineto{\pgfqpoint{0.525554in}{0.862077in}}%
\pgfpathlineto{\pgfqpoint{0.521584in}{0.867731in}}%
\pgfpathlineto{\pgfqpoint{0.518791in}{0.878609in}}%
\pgfpathlineto{\pgfqpoint{0.516670in}{0.889512in}}%
\pgfpathlineto{\pgfqpoint{0.515219in}{0.900427in}}%
\pgfpathlineto{\pgfqpoint{0.514438in}{0.911345in}}%
\pgfpathlineto{\pgfqpoint{0.518426in}{0.905463in}}%
\pgfpathlineto{\pgfqpoint{0.522401in}{0.899935in}}%
\pgfpathlineto{\pgfqpoint{0.526362in}{0.894755in}}%
\pgfpathlineto{\pgfqpoint{0.530310in}{0.889916in}}%
\pgfpathlineto{\pgfqpoint{0.531096in}{0.879227in}}%
\pgfpathlineto{\pgfqpoint{0.532536in}{0.868542in}}%
\pgfpathlineto{\pgfqpoint{0.534630in}{0.857869in}}%
\pgfpathlineto{\pgfqpoint{0.537379in}{0.847221in}}%
\pgfpathclose%
\pgfusepath{fill}%
\end{pgfscope}%
\begin{pgfscope}%
\pgfpathrectangle{\pgfqpoint{0.041670in}{0.041670in}}{\pgfqpoint{2.216660in}{2.216660in}}%
\pgfusepath{clip}%
\pgfsetbuttcap%
\pgfsetroundjoin%
\definecolor{currentfill}{rgb}{0.166383,0.690856,0.496502}%
\pgfsetfillcolor{currentfill}%
\pgfsetlinewidth{0.000000pt}%
\definecolor{currentstroke}{rgb}{0.000000,0.000000,0.000000}%
\pgfsetstrokecolor{currentstroke}%
\pgfsetdash{}{0pt}%
\pgfpathmoveto{\pgfqpoint{1.002366in}{1.469397in}}%
\pgfpathlineto{\pgfqpoint{0.999851in}{1.463605in}}%
\pgfpathlineto{\pgfqpoint{0.997340in}{1.457768in}}%
\pgfpathlineto{\pgfqpoint{0.994832in}{1.451886in}}%
\pgfpathlineto{\pgfqpoint{0.992327in}{1.445962in}}%
\pgfpathlineto{\pgfqpoint{0.998778in}{1.448761in}}%
\pgfpathlineto{\pgfqpoint{1.005402in}{1.451459in}}%
\pgfpathlineto{\pgfqpoint{1.012193in}{1.454055in}}%
\pgfpathlineto{\pgfqpoint{1.019142in}{1.456545in}}%
\pgfpathlineto{\pgfqpoint{1.021286in}{1.462319in}}%
\pgfpathlineto{\pgfqpoint{1.023432in}{1.468050in}}%
\pgfpathlineto{\pgfqpoint{1.025582in}{1.473738in}}%
\pgfpathlineto{\pgfqpoint{1.027734in}{1.479380in}}%
\pgfpathlineto{\pgfqpoint{1.021159in}{1.477030in}}%
\pgfpathlineto{\pgfqpoint{1.014735in}{1.474582in}}%
\pgfpathlineto{\pgfqpoint{1.008469in}{1.472037in}}%
\pgfpathlineto{\pgfqpoint{1.002366in}{1.469397in}}%
\pgfpathclose%
\pgfusepath{fill}%
\end{pgfscope}%
\begin{pgfscope}%
\pgfpathrectangle{\pgfqpoint{0.041670in}{0.041670in}}{\pgfqpoint{2.216660in}{2.216660in}}%
\pgfusepath{clip}%
\pgfsetbuttcap%
\pgfsetroundjoin%
\definecolor{currentfill}{rgb}{0.280255,0.165693,0.476498}%
\pgfsetfillcolor{currentfill}%
\pgfsetlinewidth{0.000000pt}%
\definecolor{currentstroke}{rgb}{0.000000,0.000000,0.000000}%
\pgfsetstrokecolor{currentstroke}%
\pgfsetdash{}{0pt}%
\pgfpathmoveto{\pgfqpoint{0.716480in}{0.928109in}}%
\pgfpathlineto{\pgfqpoint{0.712846in}{0.922108in}}%
\pgfpathlineto{\pgfqpoint{0.709210in}{0.916229in}}%
\pgfpathlineto{\pgfqpoint{0.705573in}{0.910473in}}%
\pgfpathlineto{\pgfqpoint{0.701934in}{0.904846in}}%
\pgfpathlineto{\pgfqpoint{0.699708in}{0.912707in}}%
\pgfpathlineto{\pgfqpoint{0.697968in}{0.920592in}}%
\pgfpathlineto{\pgfqpoint{0.696714in}{0.928493in}}%
\pgfpathlineto{\pgfqpoint{0.695946in}{0.936402in}}%
\pgfpathlineto{\pgfqpoint{0.699614in}{0.941765in}}%
\pgfpathlineto{\pgfqpoint{0.703280in}{0.947257in}}%
\pgfpathlineto{\pgfqpoint{0.706946in}{0.952873in}}%
\pgfpathlineto{\pgfqpoint{0.710611in}{0.958610in}}%
\pgfpathlineto{\pgfqpoint{0.711372in}{0.950965in}}%
\pgfpathlineto{\pgfqpoint{0.712604in}{0.943328in}}%
\pgfpathlineto{\pgfqpoint{0.714307in}{0.935707in}}%
\pgfpathlineto{\pgfqpoint{0.716480in}{0.928109in}}%
\pgfpathclose%
\pgfusepath{fill}%
\end{pgfscope}%
\begin{pgfscope}%
\pgfpathrectangle{\pgfqpoint{0.041670in}{0.041670in}}{\pgfqpoint{2.216660in}{2.216660in}}%
\pgfusepath{clip}%
\pgfsetbuttcap%
\pgfsetroundjoin%
\definecolor{currentfill}{rgb}{0.274128,0.199721,0.498911}%
\pgfsetfillcolor{currentfill}%
\pgfsetlinewidth{0.000000pt}%
\definecolor{currentstroke}{rgb}{0.000000,0.000000,0.000000}%
\pgfsetstrokecolor{currentstroke}%
\pgfsetdash{}{0pt}%
\pgfpathmoveto{\pgfqpoint{1.634931in}{0.989252in}}%
\pgfpathlineto{\pgfqpoint{1.638593in}{0.983128in}}%
\pgfpathlineto{\pgfqpoint{1.642255in}{0.977110in}}%
\pgfpathlineto{\pgfqpoint{1.645917in}{0.971201in}}%
\pgfpathlineto{\pgfqpoint{1.649580in}{0.965406in}}%
\pgfpathlineto{\pgfqpoint{1.649237in}{0.957760in}}%
\pgfpathlineto{\pgfqpoint{1.648424in}{0.950116in}}%
\pgfpathlineto{\pgfqpoint{1.647140in}{0.942480in}}%
\pgfpathlineto{\pgfqpoint{1.645385in}{0.934861in}}%
\pgfpathlineto{\pgfqpoint{1.641739in}{0.940920in}}%
\pgfpathlineto{\pgfqpoint{1.638094in}{0.947093in}}%
\pgfpathlineto{\pgfqpoint{1.634450in}{0.953375in}}%
\pgfpathlineto{\pgfqpoint{1.630806in}{0.959763in}}%
\pgfpathlineto{\pgfqpoint{1.632522in}{0.967118in}}%
\pgfpathlineto{\pgfqpoint{1.633781in}{0.974489in}}%
\pgfpathlineto{\pgfqpoint{1.634584in}{0.981870in}}%
\pgfpathlineto{\pgfqpoint{1.634931in}{0.989252in}}%
\pgfpathclose%
\pgfusepath{fill}%
\end{pgfscope}%
\begin{pgfscope}%
\pgfpathrectangle{\pgfqpoint{0.041670in}{0.041670in}}{\pgfqpoint{2.216660in}{2.216660in}}%
\pgfusepath{clip}%
\pgfsetbuttcap%
\pgfsetroundjoin%
\definecolor{currentfill}{rgb}{0.281477,0.755203,0.432552}%
\pgfsetfillcolor{currentfill}%
\pgfsetlinewidth{0.000000pt}%
\definecolor{currentstroke}{rgb}{0.000000,0.000000,0.000000}%
\pgfsetstrokecolor{currentstroke}%
\pgfsetdash{}{0pt}%
\pgfpathmoveto{\pgfqpoint{1.151587in}{1.542548in}}%
\pgfpathlineto{\pgfqpoint{1.151131in}{1.537695in}}%
\pgfpathlineto{\pgfqpoint{1.150675in}{1.532778in}}%
\pgfpathlineto{\pgfqpoint{1.150221in}{1.527800in}}%
\pgfpathlineto{\pgfqpoint{1.149767in}{1.522762in}}%
\pgfpathlineto{\pgfqpoint{1.157396in}{1.523158in}}%
\pgfpathlineto{\pgfqpoint{1.165046in}{1.523439in}}%
\pgfpathlineto{\pgfqpoint{1.172710in}{1.523605in}}%
\pgfpathlineto{\pgfqpoint{1.180381in}{1.523657in}}%
\pgfpathlineto{\pgfqpoint{1.180375in}{1.528680in}}%
\pgfpathlineto{\pgfqpoint{1.180368in}{1.533644in}}%
\pgfpathlineto{\pgfqpoint{1.180362in}{1.538547in}}%
\pgfpathlineto{\pgfqpoint{1.180355in}{1.543385in}}%
\pgfpathlineto{\pgfqpoint{1.173147in}{1.543337in}}%
\pgfpathlineto{\pgfqpoint{1.165944in}{1.543182in}}%
\pgfpathlineto{\pgfqpoint{1.158755in}{1.542918in}}%
\pgfpathlineto{\pgfqpoint{1.151587in}{1.542548in}}%
\pgfpathclose%
\pgfusepath{fill}%
\end{pgfscope}%
\begin{pgfscope}%
\pgfpathrectangle{\pgfqpoint{0.041670in}{0.041670in}}{\pgfqpoint{2.216660in}{2.216660in}}%
\pgfusepath{clip}%
\pgfsetbuttcap%
\pgfsetroundjoin%
\definecolor{currentfill}{rgb}{0.281477,0.755203,0.432552}%
\pgfsetfillcolor{currentfill}%
\pgfsetlinewidth{0.000000pt}%
\definecolor{currentstroke}{rgb}{0.000000,0.000000,0.000000}%
\pgfsetstrokecolor{currentstroke}%
\pgfsetdash{}{0pt}%
\pgfpathmoveto{\pgfqpoint{1.180355in}{1.543385in}}%
\pgfpathlineto{\pgfqpoint{1.180362in}{1.538547in}}%
\pgfpathlineto{\pgfqpoint{1.180368in}{1.533644in}}%
\pgfpathlineto{\pgfqpoint{1.180375in}{1.528680in}}%
\pgfpathlineto{\pgfqpoint{1.180381in}{1.523657in}}%
\pgfpathlineto{\pgfqpoint{1.188052in}{1.523593in}}%
\pgfpathlineto{\pgfqpoint{1.195715in}{1.523414in}}%
\pgfpathlineto{\pgfqpoint{1.203363in}{1.523120in}}%
\pgfpathlineto{\pgfqpoint{1.210989in}{1.522711in}}%
\pgfpathlineto{\pgfqpoint{1.210522in}{1.527750in}}%
\pgfpathlineto{\pgfqpoint{1.210055in}{1.532729in}}%
\pgfpathlineto{\pgfqpoint{1.209587in}{1.537646in}}%
\pgfpathlineto{\pgfqpoint{1.209118in}{1.542500in}}%
\pgfpathlineto{\pgfqpoint{1.201952in}{1.542882in}}%
\pgfpathlineto{\pgfqpoint{1.194765in}{1.543158in}}%
\pgfpathlineto{\pgfqpoint{1.187564in}{1.543325in}}%
\pgfpathlineto{\pgfqpoint{1.180355in}{1.543385in}}%
\pgfpathclose%
\pgfusepath{fill}%
\end{pgfscope}%
\begin{pgfscope}%
\pgfpathrectangle{\pgfqpoint{0.041670in}{0.041670in}}{\pgfqpoint{2.216660in}{2.216660in}}%
\pgfusepath{clip}%
\pgfsetbuttcap%
\pgfsetroundjoin%
\definecolor{currentfill}{rgb}{0.133743,0.548535,0.553541}%
\pgfsetfillcolor{currentfill}%
\pgfsetlinewidth{0.000000pt}%
\definecolor{currentstroke}{rgb}{0.000000,0.000000,0.000000}%
\pgfsetstrokecolor{currentstroke}%
\pgfsetdash{}{0pt}%
\pgfpathmoveto{\pgfqpoint{0.875104in}{1.306721in}}%
\pgfpathlineto{\pgfqpoint{0.871614in}{1.299666in}}%
\pgfpathlineto{\pgfqpoint{0.868126in}{1.292603in}}%
\pgfpathlineto{\pgfqpoint{0.864642in}{1.285537in}}%
\pgfpathlineto{\pgfqpoint{0.861162in}{1.278469in}}%
\pgfpathlineto{\pgfqpoint{0.864768in}{1.283401in}}%
\pgfpathlineto{\pgfqpoint{0.868680in}{1.288270in}}%
\pgfpathlineto{\pgfqpoint{0.872895in}{1.293073in}}%
\pgfpathlineto{\pgfqpoint{0.877406in}{1.297805in}}%
\pgfpathlineto{\pgfqpoint{0.880700in}{1.304643in}}%
\pgfpathlineto{\pgfqpoint{0.883998in}{1.311479in}}%
\pgfpathlineto{\pgfqpoint{0.887298in}{1.318311in}}%
\pgfpathlineto{\pgfqpoint{0.890602in}{1.325137in}}%
\pgfpathlineto{\pgfqpoint{0.886297in}{1.320630in}}%
\pgfpathlineto{\pgfqpoint{0.882276in}{1.316055in}}%
\pgfpathlineto{\pgfqpoint{0.878543in}{1.311417in}}%
\pgfpathlineto{\pgfqpoint{0.875104in}{1.306721in}}%
\pgfpathclose%
\pgfusepath{fill}%
\end{pgfscope}%
\begin{pgfscope}%
\pgfpathrectangle{\pgfqpoint{0.041670in}{0.041670in}}{\pgfqpoint{2.216660in}{2.216660in}}%
\pgfusepath{clip}%
\pgfsetbuttcap%
\pgfsetroundjoin%
\definecolor{currentfill}{rgb}{0.220124,0.725509,0.466226}%
\pgfsetfillcolor{currentfill}%
\pgfsetlinewidth{0.000000pt}%
\definecolor{currentstroke}{rgb}{0.000000,0.000000,0.000000}%
\pgfsetstrokecolor{currentstroke}%
\pgfsetdash{}{0pt}%
\pgfpathmoveto{\pgfqpoint{1.291378in}{1.510831in}}%
\pgfpathlineto{\pgfqpoint{1.293054in}{1.505547in}}%
\pgfpathlineto{\pgfqpoint{1.294728in}{1.500207in}}%
\pgfpathlineto{\pgfqpoint{1.296400in}{1.494815in}}%
\pgfpathlineto{\pgfqpoint{1.298070in}{1.489372in}}%
\pgfpathlineto{\pgfqpoint{1.305290in}{1.487536in}}%
\pgfpathlineto{\pgfqpoint{1.312391in}{1.485591in}}%
\pgfpathlineto{\pgfqpoint{1.319366in}{1.483539in}}%
\pgfpathlineto{\pgfqpoint{1.326210in}{1.481383in}}%
\pgfpathlineto{\pgfqpoint{1.324140in}{1.486946in}}%
\pgfpathlineto{\pgfqpoint{1.322067in}{1.492458in}}%
\pgfpathlineto{\pgfqpoint{1.319992in}{1.497916in}}%
\pgfpathlineto{\pgfqpoint{1.317914in}{1.503319in}}%
\pgfpathlineto{\pgfqpoint{1.311461in}{1.505347in}}%
\pgfpathlineto{\pgfqpoint{1.304884in}{1.507276in}}%
\pgfpathlineto{\pgfqpoint{1.298187in}{1.509104in}}%
\pgfpathlineto{\pgfqpoint{1.291378in}{1.510831in}}%
\pgfpathclose%
\pgfusepath{fill}%
\end{pgfscope}%
\begin{pgfscope}%
\pgfpathrectangle{\pgfqpoint{0.041670in}{0.041670in}}{\pgfqpoint{2.216660in}{2.216660in}}%
\pgfusepath{clip}%
\pgfsetbuttcap%
\pgfsetroundjoin%
\definecolor{currentfill}{rgb}{0.233603,0.313828,0.543914}%
\pgfsetfillcolor{currentfill}%
\pgfsetlinewidth{0.000000pt}%
\definecolor{currentstroke}{rgb}{0.000000,0.000000,0.000000}%
\pgfsetstrokecolor{currentstroke}%
\pgfsetdash{}{0pt}%
\pgfpathmoveto{\pgfqpoint{0.465356in}{1.011802in}}%
\pgfpathlineto{\pgfqpoint{0.461150in}{1.022865in}}%
\pgfpathlineto{\pgfqpoint{0.456924in}{1.034376in}}%
\pgfpathlineto{\pgfqpoint{0.452678in}{1.046342in}}%
\pgfpathlineto{\pgfqpoint{0.448411in}{1.058770in}}%
\pgfpathlineto{\pgfqpoint{0.448396in}{1.070512in}}%
\pgfpathlineto{\pgfqpoint{0.449109in}{1.082229in}}%
\pgfpathlineto{\pgfqpoint{0.450549in}{1.093910in}}%
\pgfpathlineto{\pgfqpoint{0.452708in}{1.105542in}}%
\pgfpathlineto{\pgfqpoint{0.456922in}{1.092925in}}%
\pgfpathlineto{\pgfqpoint{0.461115in}{1.080768in}}%
\pgfpathlineto{\pgfqpoint{0.465289in}{1.069063in}}%
\pgfpathlineto{\pgfqpoint{0.469443in}{1.057803in}}%
\pgfpathlineto{\pgfqpoint{0.467361in}{1.046361in}}%
\pgfpathlineto{\pgfqpoint{0.465983in}{1.034873in}}%
\pgfpathlineto{\pgfqpoint{0.465313in}{1.023349in}}%
\pgfpathlineto{\pgfqpoint{0.465356in}{1.011802in}}%
\pgfpathclose%
\pgfusepath{fill}%
\end{pgfscope}%
\begin{pgfscope}%
\pgfpathrectangle{\pgfqpoint{0.041670in}{0.041670in}}{\pgfqpoint{2.216660in}{2.216660in}}%
\pgfusepath{clip}%
\pgfsetbuttcap%
\pgfsetroundjoin%
\definecolor{currentfill}{rgb}{0.231674,0.318106,0.544834}%
\pgfsetfillcolor{currentfill}%
\pgfsetlinewidth{0.000000pt}%
\definecolor{currentstroke}{rgb}{0.000000,0.000000,0.000000}%
\pgfsetstrokecolor{currentstroke}%
\pgfsetdash{}{0pt}%
\pgfpathmoveto{\pgfqpoint{0.769244in}{1.063875in}}%
\pgfpathlineto{\pgfqpoint{0.765575in}{1.056692in}}%
\pgfpathlineto{\pgfqpoint{0.761907in}{1.049574in}}%
\pgfpathlineto{\pgfqpoint{0.758240in}{1.042524in}}%
\pgfpathlineto{\pgfqpoint{0.754574in}{1.035545in}}%
\pgfpathlineto{\pgfqpoint{0.754267in}{1.042400in}}%
\pgfpathlineto{\pgfqpoint{0.754386in}{1.049250in}}%
\pgfpathlineto{\pgfqpoint{0.754928in}{1.056088in}}%
\pgfpathlineto{\pgfqpoint{0.755893in}{1.062908in}}%
\pgfpathlineto{\pgfqpoint{0.759533in}{1.069625in}}%
\pgfpathlineto{\pgfqpoint{0.763175in}{1.076414in}}%
\pgfpathlineto{\pgfqpoint{0.766818in}{1.083270in}}%
\pgfpathlineto{\pgfqpoint{0.770463in}{1.090192in}}%
\pgfpathlineto{\pgfqpoint{0.769545in}{1.083633in}}%
\pgfpathlineto{\pgfqpoint{0.769035in}{1.077056in}}%
\pgfpathlineto{\pgfqpoint{0.768935in}{1.070468in}}%
\pgfpathlineto{\pgfqpoint{0.769244in}{1.063875in}}%
\pgfpathclose%
\pgfusepath{fill}%
\end{pgfscope}%
\begin{pgfscope}%
\pgfpathrectangle{\pgfqpoint{0.041670in}{0.041670in}}{\pgfqpoint{2.216660in}{2.216660in}}%
\pgfusepath{clip}%
\pgfsetbuttcap%
\pgfsetroundjoin%
\definecolor{currentfill}{rgb}{0.281477,0.755203,0.432552}%
\pgfsetfillcolor{currentfill}%
\pgfsetlinewidth{0.000000pt}%
\definecolor{currentstroke}{rgb}{0.000000,0.000000,0.000000}%
\pgfsetstrokecolor{currentstroke}%
\pgfsetdash{}{0pt}%
\pgfpathmoveto{\pgfqpoint{1.123247in}{1.539998in}}%
\pgfpathlineto{\pgfqpoint{1.122336in}{1.535103in}}%
\pgfpathlineto{\pgfqpoint{1.121425in}{1.530143in}}%
\pgfpathlineto{\pgfqpoint{1.120516in}{1.525122in}}%
\pgfpathlineto{\pgfqpoint{1.119608in}{1.520042in}}%
\pgfpathlineto{\pgfqpoint{1.127080in}{1.520891in}}%
\pgfpathlineto{\pgfqpoint{1.134602in}{1.521628in}}%
\pgfpathlineto{\pgfqpoint{1.142167in}{1.522252in}}%
\pgfpathlineto{\pgfqpoint{1.149767in}{1.522762in}}%
\pgfpathlineto{\pgfqpoint{1.150221in}{1.527800in}}%
\pgfpathlineto{\pgfqpoint{1.150675in}{1.532778in}}%
\pgfpathlineto{\pgfqpoint{1.151131in}{1.537695in}}%
\pgfpathlineto{\pgfqpoint{1.151587in}{1.542548in}}%
\pgfpathlineto{\pgfqpoint{1.144445in}{1.542070in}}%
\pgfpathlineto{\pgfqpoint{1.137336in}{1.541485in}}%
\pgfpathlineto{\pgfqpoint{1.130268in}{1.540795in}}%
\pgfpathlineto{\pgfqpoint{1.123247in}{1.539998in}}%
\pgfpathclose%
\pgfusepath{fill}%
\end{pgfscope}%
\begin{pgfscope}%
\pgfpathrectangle{\pgfqpoint{0.041670in}{0.041670in}}{\pgfqpoint{2.216660in}{2.216660in}}%
\pgfusepath{clip}%
\pgfsetbuttcap%
\pgfsetroundjoin%
\definecolor{currentfill}{rgb}{0.281477,0.755203,0.432552}%
\pgfsetfillcolor{currentfill}%
\pgfsetlinewidth{0.000000pt}%
\definecolor{currentstroke}{rgb}{0.000000,0.000000,0.000000}%
\pgfsetstrokecolor{currentstroke}%
\pgfsetdash{}{0pt}%
\pgfpathmoveto{\pgfqpoint{1.209118in}{1.542500in}}%
\pgfpathlineto{\pgfqpoint{1.209587in}{1.537646in}}%
\pgfpathlineto{\pgfqpoint{1.210055in}{1.532729in}}%
\pgfpathlineto{\pgfqpoint{1.210522in}{1.527750in}}%
\pgfpathlineto{\pgfqpoint{1.210989in}{1.522711in}}%
\pgfpathlineto{\pgfqpoint{1.218586in}{1.522189in}}%
\pgfpathlineto{\pgfqpoint{1.226146in}{1.521552in}}%
\pgfpathlineto{\pgfqpoint{1.233662in}{1.520802in}}%
\pgfpathlineto{\pgfqpoint{1.241128in}{1.519940in}}%
\pgfpathlineto{\pgfqpoint{1.240208in}{1.525023in}}%
\pgfpathlineto{\pgfqpoint{1.239286in}{1.530045in}}%
\pgfpathlineto{\pgfqpoint{1.238363in}{1.535006in}}%
\pgfpathlineto{\pgfqpoint{1.237439in}{1.539904in}}%
\pgfpathlineto{\pgfqpoint{1.230424in}{1.540711in}}%
\pgfpathlineto{\pgfqpoint{1.223361in}{1.541414in}}%
\pgfpathlineto{\pgfqpoint{1.216257in}{1.542010in}}%
\pgfpathlineto{\pgfqpoint{1.209118in}{1.542500in}}%
\pgfpathclose%
\pgfusepath{fill}%
\end{pgfscope}%
\begin{pgfscope}%
\pgfpathrectangle{\pgfqpoint{0.041670in}{0.041670in}}{\pgfqpoint{2.216660in}{2.216660in}}%
\pgfusepath{clip}%
\pgfsetbuttcap%
\pgfsetroundjoin%
\definecolor{currentfill}{rgb}{0.179019,0.433756,0.557430}%
\pgfsetfillcolor{currentfill}%
\pgfsetlinewidth{0.000000pt}%
\definecolor{currentstroke}{rgb}{0.000000,0.000000,0.000000}%
\pgfsetstrokecolor{currentstroke}%
\pgfsetdash{}{0pt}%
\pgfpathmoveto{\pgfqpoint{1.536573in}{1.205019in}}%
\pgfpathlineto{\pgfqpoint{1.540137in}{1.197877in}}%
\pgfpathlineto{\pgfqpoint{1.543697in}{1.190763in}}%
\pgfpathlineto{\pgfqpoint{1.547256in}{1.183679in}}%
\pgfpathlineto{\pgfqpoint{1.550811in}{1.176628in}}%
\pgfpathlineto{\pgfqpoint{1.553456in}{1.170747in}}%
\pgfpathlineto{\pgfqpoint{1.555734in}{1.164822in}}%
\pgfpathlineto{\pgfqpoint{1.557642in}{1.158858in}}%
\pgfpathlineto{\pgfqpoint{1.559179in}{1.152862in}}%
\pgfpathlineto{\pgfqpoint{1.555530in}{1.160166in}}%
\pgfpathlineto{\pgfqpoint{1.551880in}{1.167502in}}%
\pgfpathlineto{\pgfqpoint{1.548227in}{1.174869in}}%
\pgfpathlineto{\pgfqpoint{1.544572in}{1.182262in}}%
\pgfpathlineto{\pgfqpoint{1.543106in}{1.188003in}}%
\pgfpathlineto{\pgfqpoint{1.541283in}{1.193713in}}%
\pgfpathlineto{\pgfqpoint{1.539105in}{1.199387in}}%
\pgfpathlineto{\pgfqpoint{1.536573in}{1.205019in}}%
\pgfpathclose%
\pgfusepath{fill}%
\end{pgfscope}%
\begin{pgfscope}%
\pgfpathrectangle{\pgfqpoint{0.041670in}{0.041670in}}{\pgfqpoint{2.216660in}{2.216660in}}%
\pgfusepath{clip}%
\pgfsetbuttcap%
\pgfsetroundjoin%
\definecolor{currentfill}{rgb}{0.220124,0.725509,0.466226}%
\pgfsetfillcolor{currentfill}%
\pgfsetlinewidth{0.000000pt}%
\definecolor{currentstroke}{rgb}{0.000000,0.000000,0.000000}%
\pgfsetstrokecolor{currentstroke}%
\pgfsetdash{}{0pt}%
\pgfpathmoveto{\pgfqpoint{1.036370in}{1.501436in}}%
\pgfpathlineto{\pgfqpoint{1.034207in}{1.496003in}}%
\pgfpathlineto{\pgfqpoint{1.032046in}{1.490514in}}%
\pgfpathlineto{\pgfqpoint{1.029889in}{1.484972in}}%
\pgfpathlineto{\pgfqpoint{1.027734in}{1.479380in}}%
\pgfpathlineto{\pgfqpoint{1.034454in}{1.481628in}}%
\pgfpathlineto{\pgfqpoint{1.041312in}{1.483773in}}%
\pgfpathlineto{\pgfqpoint{1.048302in}{1.485812in}}%
\pgfpathlineto{\pgfqpoint{1.055416in}{1.487745in}}%
\pgfpathlineto{\pgfqpoint{1.057177in}{1.493212in}}%
\pgfpathlineto{\pgfqpoint{1.058940in}{1.498629in}}%
\pgfpathlineto{\pgfqpoint{1.060706in}{1.503993in}}%
\pgfpathlineto{\pgfqpoint{1.062474in}{1.509301in}}%
\pgfpathlineto{\pgfqpoint{1.055764in}{1.507484in}}%
\pgfpathlineto{\pgfqpoint{1.049173in}{1.505566in}}%
\pgfpathlineto{\pgfqpoint{1.042706in}{1.503550in}}%
\pgfpathlineto{\pgfqpoint{1.036370in}{1.501436in}}%
\pgfpathclose%
\pgfusepath{fill}%
\end{pgfscope}%
\begin{pgfscope}%
\pgfpathrectangle{\pgfqpoint{0.041670in}{0.041670in}}{\pgfqpoint{2.216660in}{2.216660in}}%
\pgfusepath{clip}%
\pgfsetbuttcap%
\pgfsetroundjoin%
\definecolor{currentfill}{rgb}{0.268510,0.009605,0.335427}%
\pgfsetfillcolor{currentfill}%
\pgfsetlinewidth{0.000000pt}%
\definecolor{currentstroke}{rgb}{0.000000,0.000000,0.000000}%
\pgfsetstrokecolor{currentstroke}%
\pgfsetdash{}{0pt}%
\pgfpathmoveto{\pgfqpoint{1.733543in}{0.833428in}}%
\pgfpathlineto{\pgfqpoint{1.737265in}{0.831474in}}%
\pgfpathlineto{\pgfqpoint{1.740992in}{0.829738in}}%
\pgfpathlineto{\pgfqpoint{1.744725in}{0.828226in}}%
\pgfpathlineto{\pgfqpoint{1.748465in}{0.826943in}}%
\pgfpathlineto{\pgfqpoint{1.745879in}{0.817524in}}%
\pgfpathlineto{\pgfqpoint{1.742717in}{0.808141in}}%
\pgfpathlineto{\pgfqpoint{1.738979in}{0.798804in}}%
\pgfpathlineto{\pgfqpoint{1.734668in}{0.789522in}}%
\pgfpathlineto{\pgfqpoint{1.731001in}{0.791059in}}%
\pgfpathlineto{\pgfqpoint{1.727339in}{0.792826in}}%
\pgfpathlineto{\pgfqpoint{1.723684in}{0.794817in}}%
\pgfpathlineto{\pgfqpoint{1.720035in}{0.797027in}}%
\pgfpathlineto{\pgfqpoint{1.724251in}{0.806055in}}%
\pgfpathlineto{\pgfqpoint{1.727908in}{0.815138in}}%
\pgfpathlineto{\pgfqpoint{1.731006in}{0.824265in}}%
\pgfpathlineto{\pgfqpoint{1.733543in}{0.833428in}}%
\pgfpathclose%
\pgfusepath{fill}%
\end{pgfscope}%
\begin{pgfscope}%
\pgfpathrectangle{\pgfqpoint{0.041670in}{0.041670in}}{\pgfqpoint{2.216660in}{2.216660in}}%
\pgfusepath{clip}%
\pgfsetbuttcap%
\pgfsetroundjoin%
\definecolor{currentfill}{rgb}{0.147607,0.511733,0.557049}%
\pgfsetfillcolor{currentfill}%
\pgfsetlinewidth{0.000000pt}%
\definecolor{currentstroke}{rgb}{0.000000,0.000000,0.000000}%
\pgfsetstrokecolor{currentstroke}%
\pgfsetdash{}{0pt}%
\pgfpathmoveto{\pgfqpoint{1.495558in}{1.282856in}}%
\pgfpathlineto{\pgfqpoint{1.498998in}{1.275842in}}%
\pgfpathlineto{\pgfqpoint{1.502436in}{1.268832in}}%
\pgfpathlineto{\pgfqpoint{1.505870in}{1.261830in}}%
\pgfpathlineto{\pgfqpoint{1.509301in}{1.254837in}}%
\pgfpathlineto{\pgfqpoint{1.513039in}{1.249662in}}%
\pgfpathlineto{\pgfqpoint{1.516453in}{1.244428in}}%
\pgfpathlineto{\pgfqpoint{1.519539in}{1.239138in}}%
\pgfpathlineto{\pgfqpoint{1.522293in}{1.233799in}}%
\pgfpathlineto{\pgfqpoint{1.518715in}{1.241033in}}%
\pgfpathlineto{\pgfqpoint{1.515135in}{1.248277in}}%
\pgfpathlineto{\pgfqpoint{1.511552in}{1.255527in}}%
\pgfpathlineto{\pgfqpoint{1.507965in}{1.262781in}}%
\pgfpathlineto{\pgfqpoint{1.505337in}{1.267875in}}%
\pgfpathlineto{\pgfqpoint{1.502391in}{1.272923in}}%
\pgfpathlineto{\pgfqpoint{1.499130in}{1.277918in}}%
\pgfpathlineto{\pgfqpoint{1.495558in}{1.282856in}}%
\pgfpathclose%
\pgfusepath{fill}%
\end{pgfscope}%
\begin{pgfscope}%
\pgfpathrectangle{\pgfqpoint{0.041670in}{0.041670in}}{\pgfqpoint{2.216660in}{2.216660in}}%
\pgfusepath{clip}%
\pgfsetbuttcap%
\pgfsetroundjoin%
\definecolor{currentfill}{rgb}{0.271305,0.019942,0.347269}%
\pgfsetfillcolor{currentfill}%
\pgfsetlinewidth{0.000000pt}%
\definecolor{currentstroke}{rgb}{0.000000,0.000000,0.000000}%
\pgfsetstrokecolor{currentstroke}%
\pgfsetdash{}{0pt}%
\pgfpathmoveto{\pgfqpoint{1.718709in}{0.843345in}}%
\pgfpathlineto{\pgfqpoint{1.722410in}{0.840560in}}%
\pgfpathlineto{\pgfqpoint{1.726116in}{0.837976in}}%
\pgfpathlineto{\pgfqpoint{1.729827in}{0.835597in}}%
\pgfpathlineto{\pgfqpoint{1.733543in}{0.833428in}}%
\pgfpathlineto{\pgfqpoint{1.731006in}{0.824265in}}%
\pgfpathlineto{\pgfqpoint{1.727908in}{0.815138in}}%
\pgfpathlineto{\pgfqpoint{1.724251in}{0.806055in}}%
\pgfpathlineto{\pgfqpoint{1.720035in}{0.797027in}}%
\pgfpathlineto{\pgfqpoint{1.716392in}{0.799452in}}%
\pgfpathlineto{\pgfqpoint{1.712754in}{0.802088in}}%
\pgfpathlineto{\pgfqpoint{1.709121in}{0.804929in}}%
\pgfpathlineto{\pgfqpoint{1.705493in}{0.807973in}}%
\pgfpathlineto{\pgfqpoint{1.709613in}{0.816745in}}%
\pgfpathlineto{\pgfqpoint{1.713190in}{0.825571in}}%
\pgfpathlineto{\pgfqpoint{1.716222in}{0.834441in}}%
\pgfpathlineto{\pgfqpoint{1.718709in}{0.843345in}}%
\pgfpathclose%
\pgfusepath{fill}%
\end{pgfscope}%
\begin{pgfscope}%
\pgfpathrectangle{\pgfqpoint{0.041670in}{0.041670in}}{\pgfqpoint{2.216660in}{2.216660in}}%
\pgfusepath{clip}%
\pgfsetbuttcap%
\pgfsetroundjoin%
\definecolor{currentfill}{rgb}{0.267004,0.004874,0.329415}%
\pgfsetfillcolor{currentfill}%
\pgfsetlinewidth{0.000000pt}%
\definecolor{currentstroke}{rgb}{0.000000,0.000000,0.000000}%
\pgfsetstrokecolor{currentstroke}%
\pgfsetdash{}{0pt}%
\pgfpathmoveto{\pgfqpoint{1.748465in}{0.826943in}}%
\pgfpathlineto{\pgfqpoint{1.752211in}{0.825892in}}%
\pgfpathlineto{\pgfqpoint{1.755964in}{0.825078in}}%
\pgfpathlineto{\pgfqpoint{1.759724in}{0.824506in}}%
\pgfpathlineto{\pgfqpoint{1.763492in}{0.824182in}}%
\pgfpathlineto{\pgfqpoint{1.760857in}{0.814511in}}%
\pgfpathlineto{\pgfqpoint{1.757631in}{0.804875in}}%
\pgfpathlineto{\pgfqpoint{1.753814in}{0.795286in}}%
\pgfpathlineto{\pgfqpoint{1.749408in}{0.785752in}}%
\pgfpathlineto{\pgfqpoint{1.745712in}{0.786328in}}%
\pgfpathlineto{\pgfqpoint{1.742024in}{0.787151in}}%
\pgfpathlineto{\pgfqpoint{1.738342in}{0.788217in}}%
\pgfpathlineto{\pgfqpoint{1.734668in}{0.789522in}}%
\pgfpathlineto{\pgfqpoint{1.738979in}{0.798804in}}%
\pgfpathlineto{\pgfqpoint{1.742717in}{0.808141in}}%
\pgfpathlineto{\pgfqpoint{1.745879in}{0.817524in}}%
\pgfpathlineto{\pgfqpoint{1.748465in}{0.826943in}}%
\pgfpathclose%
\pgfusepath{fill}%
\end{pgfscope}%
\begin{pgfscope}%
\pgfpathrectangle{\pgfqpoint{0.041670in}{0.041670in}}{\pgfqpoint{2.216660in}{2.216660in}}%
\pgfusepath{clip}%
\pgfsetbuttcap%
\pgfsetroundjoin%
\definecolor{currentfill}{rgb}{0.134692,0.658636,0.517649}%
\pgfsetfillcolor{currentfill}%
\pgfsetlinewidth{0.000000pt}%
\definecolor{currentstroke}{rgb}{0.000000,0.000000,0.000000}%
\pgfsetstrokecolor{currentstroke}%
\pgfsetdash{}{0pt}%
\pgfpathmoveto{\pgfqpoint{1.386483in}{1.436641in}}%
\pgfpathlineto{\pgfqpoint{1.389241in}{1.430545in}}%
\pgfpathlineto{\pgfqpoint{1.391995in}{1.424412in}}%
\pgfpathlineto{\pgfqpoint{1.394747in}{1.418244in}}%
\pgfpathlineto{\pgfqpoint{1.397495in}{1.412044in}}%
\pgfpathlineto{\pgfqpoint{1.403480in}{1.408679in}}%
\pgfpathlineto{\pgfqpoint{1.409252in}{1.405221in}}%
\pgfpathlineto{\pgfqpoint{1.414803in}{1.401676in}}%
\pgfpathlineto{\pgfqpoint{1.420130in}{1.398045in}}%
\pgfpathlineto{\pgfqpoint{1.417090in}{1.404435in}}%
\pgfpathlineto{\pgfqpoint{1.414046in}{1.410792in}}%
\pgfpathlineto{\pgfqpoint{1.410999in}{1.417115in}}%
\pgfpathlineto{\pgfqpoint{1.407948in}{1.423400in}}%
\pgfpathlineto{\pgfqpoint{1.402898in}{1.426834in}}%
\pgfpathlineto{\pgfqpoint{1.397634in}{1.430187in}}%
\pgfpathlineto{\pgfqpoint{1.392160in}{1.433457in}}%
\pgfpathlineto{\pgfqpoint{1.386483in}{1.436641in}}%
\pgfpathclose%
\pgfusepath{fill}%
\end{pgfscope}%
\begin{pgfscope}%
\pgfpathrectangle{\pgfqpoint{0.041670in}{0.041670in}}{\pgfqpoint{2.216660in}{2.216660in}}%
\pgfusepath{clip}%
\pgfsetbuttcap%
\pgfsetroundjoin%
\definecolor{currentfill}{rgb}{0.274128,0.199721,0.498911}%
\pgfsetfillcolor{currentfill}%
\pgfsetlinewidth{0.000000pt}%
\definecolor{currentstroke}{rgb}{0.000000,0.000000,0.000000}%
\pgfsetstrokecolor{currentstroke}%
\pgfsetdash{}{0pt}%
\pgfpathmoveto{\pgfqpoint{0.731011in}{0.953245in}}%
\pgfpathlineto{\pgfqpoint{0.727379in}{0.946798in}}%
\pgfpathlineto{\pgfqpoint{0.723747in}{0.940458in}}%
\pgfpathlineto{\pgfqpoint{0.720114in}{0.934227in}}%
\pgfpathlineto{\pgfqpoint{0.716480in}{0.928109in}}%
\pgfpathlineto{\pgfqpoint{0.714307in}{0.935707in}}%
\pgfpathlineto{\pgfqpoint{0.712604in}{0.943328in}}%
\pgfpathlineto{\pgfqpoint{0.711372in}{0.950965in}}%
\pgfpathlineto{\pgfqpoint{0.710611in}{0.958610in}}%
\pgfpathlineto{\pgfqpoint{0.714275in}{0.964464in}}%
\pgfpathlineto{\pgfqpoint{0.717939in}{0.970431in}}%
\pgfpathlineto{\pgfqpoint{0.721602in}{0.976507in}}%
\pgfpathlineto{\pgfqpoint{0.725265in}{0.982690in}}%
\pgfpathlineto{\pgfqpoint{0.726017in}{0.975309in}}%
\pgfpathlineto{\pgfqpoint{0.727226in}{0.967936in}}%
\pgfpathlineto{\pgfqpoint{0.728890in}{0.960579in}}%
\pgfpathlineto{\pgfqpoint{0.731011in}{0.953245in}}%
\pgfpathclose%
\pgfusepath{fill}%
\end{pgfscope}%
\begin{pgfscope}%
\pgfpathrectangle{\pgfqpoint{0.041670in}{0.041670in}}{\pgfqpoint{2.216660in}{2.216660in}}%
\pgfusepath{clip}%
\pgfsetbuttcap%
\pgfsetroundjoin%
\definecolor{currentfill}{rgb}{0.281477,0.755203,0.432552}%
\pgfsetfillcolor{currentfill}%
\pgfsetlinewidth{0.000000pt}%
\definecolor{currentstroke}{rgb}{0.000000,0.000000,0.000000}%
\pgfsetstrokecolor{currentstroke}%
\pgfsetdash{}{0pt}%
\pgfpathmoveto{\pgfqpoint{1.237439in}{1.539904in}}%
\pgfpathlineto{\pgfqpoint{1.238363in}{1.535006in}}%
\pgfpathlineto{\pgfqpoint{1.239286in}{1.530045in}}%
\pgfpathlineto{\pgfqpoint{1.240208in}{1.525023in}}%
\pgfpathlineto{\pgfqpoint{1.241128in}{1.519940in}}%
\pgfpathlineto{\pgfqpoint{1.248536in}{1.518966in}}%
\pgfpathlineto{\pgfqpoint{1.255879in}{1.517881in}}%
\pgfpathlineto{\pgfqpoint{1.263150in}{1.516686in}}%
\pgfpathlineto{\pgfqpoint{1.261898in}{1.521820in}}%
\pgfpathlineto{\pgfqpoint{1.260644in}{1.526894in}}%
\pgfpathlineto{\pgfqpoint{1.259388in}{1.531907in}}%
\pgfpathlineto{\pgfqpoint{1.258131in}{1.536855in}}%
\pgfpathlineto{\pgfqpoint{1.251299in}{1.537975in}}%
\pgfpathlineto{\pgfqpoint{1.244400in}{1.538991in}}%
\pgfpathlineto{\pgfqpoint{1.237439in}{1.539904in}}%
\pgfpathclose%
\pgfusepath{fill}%
\end{pgfscope}%
\begin{pgfscope}%
\pgfpathrectangle{\pgfqpoint{0.041670in}{0.041670in}}{\pgfqpoint{2.216660in}{2.216660in}}%
\pgfusepath{clip}%
\pgfsetbuttcap%
\pgfsetroundjoin%
\definecolor{currentfill}{rgb}{0.274952,0.037752,0.364543}%
\pgfsetfillcolor{currentfill}%
\pgfsetlinewidth{0.000000pt}%
\definecolor{currentstroke}{rgb}{0.000000,0.000000,0.000000}%
\pgfsetstrokecolor{currentstroke}%
\pgfsetdash{}{0pt}%
\pgfpathmoveto{\pgfqpoint{1.703949in}{0.856412in}}%
\pgfpathlineto{\pgfqpoint{1.707633in}{0.852866in}}%
\pgfpathlineto{\pgfqpoint{1.711321in}{0.849503in}}%
\pgfpathlineto{\pgfqpoint{1.715013in}{0.846328in}}%
\pgfpathlineto{\pgfqpoint{1.718709in}{0.843345in}}%
\pgfpathlineto{\pgfqpoint{1.716222in}{0.834441in}}%
\pgfpathlineto{\pgfqpoint{1.713190in}{0.825571in}}%
\pgfpathlineto{\pgfqpoint{1.709613in}{0.816745in}}%
\pgfpathlineto{\pgfqpoint{1.705493in}{0.807973in}}%
\pgfpathlineto{\pgfqpoint{1.701870in}{0.811213in}}%
\pgfpathlineto{\pgfqpoint{1.698251in}{0.814647in}}%
\pgfpathlineto{\pgfqpoint{1.694637in}{0.818269in}}%
\pgfpathlineto{\pgfqpoint{1.691027in}{0.822075in}}%
\pgfpathlineto{\pgfqpoint{1.695050in}{0.830590in}}%
\pgfpathlineto{\pgfqpoint{1.698546in}{0.839157in}}%
\pgfpathlineto{\pgfqpoint{1.701513in}{0.847768in}}%
\pgfpathlineto{\pgfqpoint{1.703949in}{0.856412in}}%
\pgfpathclose%
\pgfusepath{fill}%
\end{pgfscope}%
\begin{pgfscope}%
\pgfpathrectangle{\pgfqpoint{0.041670in}{0.041670in}}{\pgfqpoint{2.216660in}{2.216660in}}%
\pgfusepath{clip}%
\pgfsetbuttcap%
\pgfsetroundjoin%
\definecolor{currentfill}{rgb}{0.120081,0.622161,0.534946}%
\pgfsetfillcolor{currentfill}%
\pgfsetlinewidth{0.000000pt}%
\definecolor{currentstroke}{rgb}{0.000000,0.000000,0.000000}%
\pgfsetstrokecolor{currentstroke}%
\pgfsetdash{}{0pt}%
\pgfpathmoveto{\pgfqpoint{1.420130in}{1.398045in}}%
\pgfpathlineto{\pgfqpoint{1.423166in}{1.391625in}}%
\pgfpathlineto{\pgfqpoint{1.426199in}{1.385178in}}%
\pgfpathlineto{\pgfqpoint{1.429229in}{1.378706in}}%
\pgfpathlineto{\pgfqpoint{1.432255in}{1.372212in}}%
\pgfpathlineto{\pgfqpoint{1.437615in}{1.368297in}}%
\pgfpathlineto{\pgfqpoint{1.442728in}{1.364299in}}%
\pgfpathlineto{\pgfqpoint{1.447587in}{1.360223in}}%
\pgfpathlineto{\pgfqpoint{1.452188in}{1.356071in}}%
\pgfpathlineto{\pgfqpoint{1.448915in}{1.362774in}}%
\pgfpathlineto{\pgfqpoint{1.445638in}{1.369455in}}%
\pgfpathlineto{\pgfqpoint{1.442358in}{1.376112in}}%
\pgfpathlineto{\pgfqpoint{1.439074in}{1.382740in}}%
\pgfpathlineto{\pgfqpoint{1.434702in}{1.386677in}}%
\pgfpathlineto{\pgfqpoint{1.430084in}{1.390542in}}%
\pgfpathlineto{\pgfqpoint{1.425225in}{1.394333in}}%
\pgfpathlineto{\pgfqpoint{1.420130in}{1.398045in}}%
\pgfpathclose%
\pgfusepath{fill}%
\end{pgfscope}%
\begin{pgfscope}%
\pgfpathrectangle{\pgfqpoint{0.041670in}{0.041670in}}{\pgfqpoint{2.216660in}{2.216660in}}%
\pgfusepath{clip}%
\pgfsetbuttcap%
\pgfsetroundjoin%
\definecolor{currentfill}{rgb}{0.212395,0.359683,0.551710}%
\pgfsetfillcolor{currentfill}%
\pgfsetlinewidth{0.000000pt}%
\definecolor{currentstroke}{rgb}{0.000000,0.000000,0.000000}%
\pgfsetstrokecolor{currentstroke}%
\pgfsetdash{}{0pt}%
\pgfpathmoveto{\pgfqpoint{1.573750in}{1.124043in}}%
\pgfpathlineto{\pgfqpoint{1.577388in}{1.116952in}}%
\pgfpathlineto{\pgfqpoint{1.581024in}{1.109912in}}%
\pgfpathlineto{\pgfqpoint{1.584658in}{1.102928in}}%
\pgfpathlineto{\pgfqpoint{1.588291in}{1.096003in}}%
\pgfpathlineto{\pgfqpoint{1.589569in}{1.089464in}}%
\pgfpathlineto{\pgfqpoint{1.590441in}{1.082903in}}%
\pgfpathlineto{\pgfqpoint{1.590906in}{1.076324in}}%
\pgfpathlineto{\pgfqpoint{1.590961in}{1.069735in}}%
\pgfpathlineto{\pgfqpoint{1.587291in}{1.076921in}}%
\pgfpathlineto{\pgfqpoint{1.583620in}{1.084165in}}%
\pgfpathlineto{\pgfqpoint{1.579947in}{1.091464in}}%
\pgfpathlineto{\pgfqpoint{1.576272in}{1.098815in}}%
\pgfpathlineto{\pgfqpoint{1.576232in}{1.105143in}}%
\pgfpathlineto{\pgfqpoint{1.575797in}{1.111460in}}%
\pgfpathlineto{\pgfqpoint{1.574969in}{1.117762in}}%
\pgfpathlineto{\pgfqpoint{1.573750in}{1.124043in}}%
\pgfpathclose%
\pgfusepath{fill}%
\end{pgfscope}%
\begin{pgfscope}%
\pgfpathrectangle{\pgfqpoint{0.041670in}{0.041670in}}{\pgfqpoint{2.216660in}{2.216660in}}%
\pgfusepath{clip}%
\pgfsetbuttcap%
\pgfsetroundjoin%
\definecolor{currentfill}{rgb}{0.267004,0.004874,0.329415}%
\pgfsetfillcolor{currentfill}%
\pgfsetlinewidth{0.000000pt}%
\definecolor{currentstroke}{rgb}{0.000000,0.000000,0.000000}%
\pgfsetstrokecolor{currentstroke}%
\pgfsetdash{}{0pt}%
\pgfpathmoveto{\pgfqpoint{1.763492in}{0.824182in}}%
\pgfpathlineto{\pgfqpoint{1.767267in}{0.824109in}}%
\pgfpathlineto{\pgfqpoint{1.771049in}{0.824293in}}%
\pgfpathlineto{\pgfqpoint{1.774840in}{0.824739in}}%
\pgfpathlineto{\pgfqpoint{1.778640in}{0.825451in}}%
\pgfpathlineto{\pgfqpoint{1.775957in}{0.815530in}}%
\pgfpathlineto{\pgfqpoint{1.772668in}{0.805645in}}%
\pgfpathlineto{\pgfqpoint{1.768771in}{0.795806in}}%
\pgfpathlineto{\pgfqpoint{1.764270in}{0.786025in}}%
\pgfpathlineto{\pgfqpoint{1.760542in}{0.785560in}}%
\pgfpathlineto{\pgfqpoint{1.756822in}{0.785363in}}%
\pgfpathlineto{\pgfqpoint{1.753111in}{0.785429in}}%
\pgfpathlineto{\pgfqpoint{1.749408in}{0.785752in}}%
\pgfpathlineto{\pgfqpoint{1.753814in}{0.795286in}}%
\pgfpathlineto{\pgfqpoint{1.757631in}{0.804875in}}%
\pgfpathlineto{\pgfqpoint{1.760857in}{0.814511in}}%
\pgfpathlineto{\pgfqpoint{1.763492in}{0.824182in}}%
\pgfpathclose%
\pgfusepath{fill}%
\end{pgfscope}%
\begin{pgfscope}%
\pgfpathrectangle{\pgfqpoint{0.041670in}{0.041670in}}{\pgfqpoint{2.216660in}{2.216660in}}%
\pgfusepath{clip}%
\pgfsetbuttcap%
\pgfsetroundjoin%
\definecolor{currentfill}{rgb}{0.281477,0.755203,0.432552}%
\pgfsetfillcolor{currentfill}%
\pgfsetlinewidth{0.000000pt}%
\definecolor{currentstroke}{rgb}{0.000000,0.000000,0.000000}%
\pgfsetstrokecolor{currentstroke}%
\pgfsetdash{}{0pt}%
\pgfpathmoveto{\pgfqpoint{1.095769in}{1.535775in}}%
\pgfpathlineto{\pgfqpoint{1.094414in}{1.530808in}}%
\pgfpathlineto{\pgfqpoint{1.093062in}{1.525777in}}%
\pgfpathlineto{\pgfqpoint{1.091711in}{1.520685in}}%
\pgfpathlineto{\pgfqpoint{1.090362in}{1.515533in}}%
\pgfpathlineto{\pgfqpoint{1.097564in}{1.516825in}}%
\pgfpathlineto{\pgfqpoint{1.104843in}{1.518007in}}%
\pgfpathlineto{\pgfqpoint{1.112194in}{1.519080in}}%
\pgfpathlineto{\pgfqpoint{1.119608in}{1.520042in}}%
\pgfpathlineto{\pgfqpoint{1.120516in}{1.525122in}}%
\pgfpathlineto{\pgfqpoint{1.121425in}{1.530143in}}%
\pgfpathlineto{\pgfqpoint{1.122336in}{1.535103in}}%
\pgfpathlineto{\pgfqpoint{1.123247in}{1.539998in}}%
\pgfpathlineto{\pgfqpoint{1.116280in}{1.539098in}}%
\pgfpathlineto{\pgfqpoint{1.109374in}{1.538093in}}%
\pgfpathlineto{\pgfqpoint{1.102534in}{1.536985in}}%
\pgfpathlineto{\pgfqpoint{1.095769in}{1.535775in}}%
\pgfpathclose%
\pgfusepath{fill}%
\end{pgfscope}%
\begin{pgfscope}%
\pgfpathrectangle{\pgfqpoint{0.041670in}{0.041670in}}{\pgfqpoint{2.216660in}{2.216660in}}%
\pgfusepath{clip}%
\pgfsetbuttcap%
\pgfsetroundjoin%
\definecolor{currentfill}{rgb}{0.166383,0.690856,0.496502}%
\pgfsetfillcolor{currentfill}%
\pgfsetlinewidth{0.000000pt}%
\definecolor{currentstroke}{rgb}{0.000000,0.000000,0.000000}%
\pgfsetstrokecolor{currentstroke}%
\pgfsetdash{}{0pt}%
\pgfpathmoveto{\pgfqpoint{1.352127in}{1.471748in}}%
\pgfpathlineto{\pgfqpoint{1.354564in}{1.465992in}}%
\pgfpathlineto{\pgfqpoint{1.356998in}{1.460189in}}%
\pgfpathlineto{\pgfqpoint{1.359429in}{1.454343in}}%
\pgfpathlineto{\pgfqpoint{1.361857in}{1.448455in}}%
\pgfpathlineto{\pgfqpoint{1.368289in}{1.445645in}}%
\pgfpathlineto{\pgfqpoint{1.374541in}{1.442737in}}%
\pgfpathlineto{\pgfqpoint{1.380608in}{1.439735in}}%
\pgfpathlineto{\pgfqpoint{1.386483in}{1.436641in}}%
\pgfpathlineto{\pgfqpoint{1.383722in}{1.442697in}}%
\pgfpathlineto{\pgfqpoint{1.380957in}{1.448712in}}%
\pgfpathlineto{\pgfqpoint{1.378189in}{1.454682in}}%
\pgfpathlineto{\pgfqpoint{1.375418in}{1.460606in}}%
\pgfpathlineto{\pgfqpoint{1.369862in}{1.463524in}}%
\pgfpathlineto{\pgfqpoint{1.364125in}{1.466356in}}%
\pgfpathlineto{\pgfqpoint{1.358211in}{1.469098in}}%
\pgfpathlineto{\pgfqpoint{1.352127in}{1.471748in}}%
\pgfpathclose%
\pgfusepath{fill}%
\end{pgfscope}%
\begin{pgfscope}%
\pgfpathrectangle{\pgfqpoint{0.041670in}{0.041670in}}{\pgfqpoint{2.216660in}{2.216660in}}%
\pgfusepath{clip}%
\pgfsetbuttcap%
\pgfsetroundjoin%
\definecolor{currentfill}{rgb}{0.263663,0.237631,0.518762}%
\pgfsetfillcolor{currentfill}%
\pgfsetlinewidth{0.000000pt}%
\definecolor{currentstroke}{rgb}{0.000000,0.000000,0.000000}%
\pgfsetstrokecolor{currentstroke}%
\pgfsetdash{}{0pt}%
\pgfpathmoveto{\pgfqpoint{1.620283in}{1.014738in}}%
\pgfpathlineto{\pgfqpoint{1.623946in}{1.008225in}}%
\pgfpathlineto{\pgfqpoint{1.627608in}{1.001804in}}%
\pgfpathlineto{\pgfqpoint{1.631269in}{0.995479in}}%
\pgfpathlineto{\pgfqpoint{1.634931in}{0.989252in}}%
\pgfpathlineto{\pgfqpoint{1.634584in}{0.981870in}}%
\pgfpathlineto{\pgfqpoint{1.633781in}{0.974489in}}%
\pgfpathlineto{\pgfqpoint{1.632522in}{0.967118in}}%
\pgfpathlineto{\pgfqpoint{1.630806in}{0.959763in}}%
\pgfpathlineto{\pgfqpoint{1.627163in}{0.966253in}}%
\pgfpathlineto{\pgfqpoint{1.623520in}{0.972843in}}%
\pgfpathlineto{\pgfqpoint{1.619876in}{0.979528in}}%
\pgfpathlineto{\pgfqpoint{1.616233in}{0.986304in}}%
\pgfpathlineto{\pgfqpoint{1.617908in}{0.993395in}}%
\pgfpathlineto{\pgfqpoint{1.619141in}{1.000502in}}%
\pgfpathlineto{\pgfqpoint{1.619932in}{1.007619in}}%
\pgfpathlineto{\pgfqpoint{1.620283in}{1.014738in}}%
\pgfpathclose%
\pgfusepath{fill}%
\end{pgfscope}%
\begin{pgfscope}%
\pgfpathrectangle{\pgfqpoint{0.041670in}{0.041670in}}{\pgfqpoint{2.216660in}{2.216660in}}%
\pgfusepath{clip}%
\pgfsetbuttcap%
\pgfsetroundjoin%
\definecolor{currentfill}{rgb}{0.282884,0.135920,0.453427}%
\pgfsetfillcolor{currentfill}%
\pgfsetlinewidth{0.000000pt}%
\definecolor{currentstroke}{rgb}{0.000000,0.000000,0.000000}%
\pgfsetstrokecolor{currentstroke}%
\pgfsetdash{}{0pt}%
\pgfpathmoveto{\pgfqpoint{0.521584in}{0.867731in}}%
\pgfpathlineto{\pgfqpoint{0.517601in}{0.873746in}}%
\pgfpathlineto{\pgfqpoint{0.513602in}{0.880129in}}%
\pgfpathlineto{\pgfqpoint{0.509589in}{0.886886in}}%
\pgfpathlineto{\pgfqpoint{0.505560in}{0.894023in}}%
\pgfpathlineto{\pgfqpoint{0.502723in}{0.905125in}}%
\pgfpathlineto{\pgfqpoint{0.500575in}{0.916251in}}%
\pgfpathlineto{\pgfqpoint{0.499114in}{0.927390in}}%
\pgfpathlineto{\pgfqpoint{0.498339in}{0.938529in}}%
\pgfpathlineto{\pgfqpoint{0.502386in}{0.931172in}}%
\pgfpathlineto{\pgfqpoint{0.506418in}{0.924193in}}%
\pgfpathlineto{\pgfqpoint{0.510435in}{0.917586in}}%
\pgfpathlineto{\pgfqpoint{0.514438in}{0.911345in}}%
\pgfpathlineto{\pgfqpoint{0.515219in}{0.900427in}}%
\pgfpathlineto{\pgfqpoint{0.516670in}{0.889512in}}%
\pgfpathlineto{\pgfqpoint{0.518791in}{0.878609in}}%
\pgfpathlineto{\pgfqpoint{0.521584in}{0.867731in}}%
\pgfpathclose%
\pgfusepath{fill}%
\end{pgfscope}%
\begin{pgfscope}%
\pgfpathrectangle{\pgfqpoint{0.041670in}{0.041670in}}{\pgfqpoint{2.216660in}{2.216660in}}%
\pgfusepath{clip}%
\pgfsetbuttcap%
\pgfsetroundjoin%
\definecolor{currentfill}{rgb}{0.179019,0.433756,0.557430}%
\pgfsetfillcolor{currentfill}%
\pgfsetlinewidth{0.000000pt}%
\definecolor{currentstroke}{rgb}{0.000000,0.000000,0.000000}%
\pgfsetstrokecolor{currentstroke}%
\pgfsetdash{}{0pt}%
\pgfpathmoveto{\pgfqpoint{0.814339in}{1.177138in}}%
\pgfpathlineto{\pgfqpoint{0.810670in}{1.169688in}}%
\pgfpathlineto{\pgfqpoint{0.807005in}{1.162264in}}%
\pgfpathlineto{\pgfqpoint{0.803341in}{1.154871in}}%
\pgfpathlineto{\pgfqpoint{0.799680in}{1.147510in}}%
\pgfpathlineto{\pgfqpoint{0.800883in}{1.153530in}}%
\pgfpathlineto{\pgfqpoint{0.802461in}{1.159523in}}%
\pgfpathlineto{\pgfqpoint{0.804411in}{1.165482in}}%
\pgfpathlineto{\pgfqpoint{0.806730in}{1.171403in}}%
\pgfpathlineto{\pgfqpoint{0.810311in}{1.178509in}}%
\pgfpathlineto{\pgfqpoint{0.813894in}{1.185648in}}%
\pgfpathlineto{\pgfqpoint{0.817480in}{1.192818in}}%
\pgfpathlineto{\pgfqpoint{0.821069in}{1.200015in}}%
\pgfpathlineto{\pgfqpoint{0.818851in}{1.194345in}}%
\pgfpathlineto{\pgfqpoint{0.816988in}{1.188639in}}%
\pgfpathlineto{\pgfqpoint{0.815483in}{1.182901in}}%
\pgfpathlineto{\pgfqpoint{0.814339in}{1.177138in}}%
\pgfpathclose%
\pgfusepath{fill}%
\end{pgfscope}%
\begin{pgfscope}%
\pgfpathrectangle{\pgfqpoint{0.041670in}{0.041670in}}{\pgfqpoint{2.216660in}{2.216660in}}%
\pgfusepath{clip}%
\pgfsetbuttcap%
\pgfsetroundjoin%
\definecolor{currentfill}{rgb}{0.134692,0.658636,0.517649}%
\pgfsetfillcolor{currentfill}%
\pgfsetlinewidth{0.000000pt}%
\definecolor{currentstroke}{rgb}{0.000000,0.000000,0.000000}%
\pgfsetstrokecolor{currentstroke}%
\pgfsetdash{}{0pt}%
\pgfpathmoveto{\pgfqpoint{0.947657in}{1.420284in}}%
\pgfpathlineto{\pgfqpoint{0.944547in}{1.413953in}}%
\pgfpathlineto{\pgfqpoint{0.941441in}{1.407586in}}%
\pgfpathlineto{\pgfqpoint{0.938338in}{1.401184in}}%
\pgfpathlineto{\pgfqpoint{0.935239in}{1.394749in}}%
\pgfpathlineto{\pgfqpoint{0.940360in}{1.398452in}}%
\pgfpathlineto{\pgfqpoint{0.945712in}{1.402074in}}%
\pgfpathlineto{\pgfqpoint{0.951289in}{1.405610in}}%
\pgfpathlineto{\pgfqpoint{0.957084in}{1.409057in}}%
\pgfpathlineto{\pgfqpoint{0.959901in}{1.415297in}}%
\pgfpathlineto{\pgfqpoint{0.962721in}{1.421506in}}%
\pgfpathlineto{\pgfqpoint{0.965544in}{1.427679in}}%
\pgfpathlineto{\pgfqpoint{0.968371in}{1.433815in}}%
\pgfpathlineto{\pgfqpoint{0.962874in}{1.430555in}}%
\pgfpathlineto{\pgfqpoint{0.957586in}{1.427211in}}%
\pgfpathlineto{\pgfqpoint{0.952512in}{1.423786in}}%
\pgfpathlineto{\pgfqpoint{0.947657in}{1.420284in}}%
\pgfpathclose%
\pgfusepath{fill}%
\end{pgfscope}%
\begin{pgfscope}%
\pgfpathrectangle{\pgfqpoint{0.041670in}{0.041670in}}{\pgfqpoint{2.216660in}{2.216660in}}%
\pgfusepath{clip}%
\pgfsetbuttcap%
\pgfsetroundjoin%
\definecolor{currentfill}{rgb}{0.276194,0.190074,0.493001}%
\pgfsetfillcolor{currentfill}%
\pgfsetlinewidth{0.000000pt}%
\definecolor{currentstroke}{rgb}{0.000000,0.000000,0.000000}%
\pgfsetstrokecolor{currentstroke}%
\pgfsetdash{}{0pt}%
\pgfpathmoveto{\pgfqpoint{1.861686in}{0.948423in}}%
\pgfpathlineto{\pgfqpoint{1.865744in}{0.956213in}}%
\pgfpathlineto{\pgfqpoint{1.869818in}{0.964394in}}%
\pgfpathlineto{\pgfqpoint{1.873909in}{0.972973in}}%
\pgfpathlineto{\pgfqpoint{1.878017in}{0.981956in}}%
\pgfpathlineto{\pgfqpoint{1.877871in}{0.970611in}}%
\pgfpathlineto{\pgfqpoint{1.877025in}{0.959257in}}%
\pgfpathlineto{\pgfqpoint{1.875477in}{0.947904in}}%
\pgfpathlineto{\pgfqpoint{1.873225in}{0.936564in}}%
\pgfpathlineto{\pgfqpoint{1.869120in}{0.927793in}}%
\pgfpathlineto{\pgfqpoint{1.865031in}{0.919428in}}%
\pgfpathlineto{\pgfqpoint{1.860960in}{0.911463in}}%
\pgfpathlineto{\pgfqpoint{1.856905in}{0.903890in}}%
\pgfpathlineto{\pgfqpoint{1.859130in}{0.915014in}}%
\pgfpathlineto{\pgfqpoint{1.860667in}{0.926152in}}%
\pgfpathlineto{\pgfqpoint{1.861518in}{0.937292in}}%
\pgfpathlineto{\pgfqpoint{1.861686in}{0.948423in}}%
\pgfpathclose%
\pgfusepath{fill}%
\end{pgfscope}%
\begin{pgfscope}%
\pgfpathrectangle{\pgfqpoint{0.041670in}{0.041670in}}{\pgfqpoint{2.216660in}{2.216660in}}%
\pgfusepath{clip}%
\pgfsetbuttcap%
\pgfsetroundjoin%
\definecolor{currentfill}{rgb}{0.147607,0.511733,0.557049}%
\pgfsetfillcolor{currentfill}%
\pgfsetlinewidth{0.000000pt}%
\definecolor{currentstroke}{rgb}{0.000000,0.000000,0.000000}%
\pgfsetstrokecolor{currentstroke}%
\pgfsetdash{}{0pt}%
\pgfpathmoveto{\pgfqpoint{0.849878in}{1.258217in}}%
\pgfpathlineto{\pgfqpoint{0.846267in}{1.250908in}}%
\pgfpathlineto{\pgfqpoint{0.842658in}{1.243603in}}%
\pgfpathlineto{\pgfqpoint{0.839053in}{1.236304in}}%
\pgfpathlineto{\pgfqpoint{0.835451in}{1.229016in}}%
\pgfpathlineto{\pgfqpoint{0.837906in}{1.234395in}}%
\pgfpathlineto{\pgfqpoint{0.840697in}{1.239729in}}%
\pgfpathlineto{\pgfqpoint{0.843820in}{1.245012in}}%
\pgfpathlineto{\pgfqpoint{0.847270in}{1.250240in}}%
\pgfpathlineto{\pgfqpoint{0.850738in}{1.257286in}}%
\pgfpathlineto{\pgfqpoint{0.854210in}{1.264341in}}%
\pgfpathlineto{\pgfqpoint{0.857684in}{1.271403in}}%
\pgfpathlineto{\pgfqpoint{0.861162in}{1.278469in}}%
\pgfpathlineto{\pgfqpoint{0.857866in}{1.273480in}}%
\pgfpathlineto{\pgfqpoint{0.854884in}{1.268439in}}%
\pgfpathlineto{\pgfqpoint{0.852221in}{1.263349in}}%
\pgfpathlineto{\pgfqpoint{0.849878in}{1.258217in}}%
\pgfpathclose%
\pgfusepath{fill}%
\end{pgfscope}%
\begin{pgfscope}%
\pgfpathrectangle{\pgfqpoint{0.041670in}{0.041670in}}{\pgfqpoint{2.216660in}{2.216660in}}%
\pgfusepath{clip}%
\pgfsetbuttcap%
\pgfsetroundjoin%
\definecolor{currentfill}{rgb}{0.279566,0.067836,0.391917}%
\pgfsetfillcolor{currentfill}%
\pgfsetlinewidth{0.000000pt}%
\definecolor{currentstroke}{rgb}{0.000000,0.000000,0.000000}%
\pgfsetstrokecolor{currentstroke}%
\pgfsetdash{}{0pt}%
\pgfpathmoveto{\pgfqpoint{1.689249in}{0.872357in}}%
\pgfpathlineto{\pgfqpoint{1.692919in}{0.868115in}}%
\pgfpathlineto{\pgfqpoint{1.696592in}{0.864042in}}%
\pgfpathlineto{\pgfqpoint{1.700269in}{0.860139in}}%
\pgfpathlineto{\pgfqpoint{1.703949in}{0.856412in}}%
\pgfpathlineto{\pgfqpoint{1.701513in}{0.847768in}}%
\pgfpathlineto{\pgfqpoint{1.698546in}{0.839157in}}%
\pgfpathlineto{\pgfqpoint{1.695050in}{0.830590in}}%
\pgfpathlineto{\pgfqpoint{1.691027in}{0.822075in}}%
\pgfpathlineto{\pgfqpoint{1.687420in}{0.826062in}}%
\pgfpathlineto{\pgfqpoint{1.683818in}{0.830224in}}%
\pgfpathlineto{\pgfqpoint{1.680219in}{0.834559in}}%
\pgfpathlineto{\pgfqpoint{1.676623in}{0.839061in}}%
\pgfpathlineto{\pgfqpoint{1.680549in}{0.847316in}}%
\pgfpathlineto{\pgfqpoint{1.683964in}{0.855623in}}%
\pgfpathlineto{\pgfqpoint{1.686864in}{0.863973in}}%
\pgfpathlineto{\pgfqpoint{1.689249in}{0.872357in}}%
\pgfpathclose%
\pgfusepath{fill}%
\end{pgfscope}%
\begin{pgfscope}%
\pgfpathrectangle{\pgfqpoint{0.041670in}{0.041670in}}{\pgfqpoint{2.216660in}{2.216660in}}%
\pgfusepath{clip}%
\pgfsetbuttcap%
\pgfsetroundjoin%
\definecolor{currentfill}{rgb}{0.201239,0.383670,0.554294}%
\pgfsetfillcolor{currentfill}%
\pgfsetlinewidth{0.000000pt}%
\definecolor{currentstroke}{rgb}{0.000000,0.000000,0.000000}%
\pgfsetstrokecolor{currentstroke}%
\pgfsetdash{}{0pt}%
\pgfpathmoveto{\pgfqpoint{1.904683in}{1.115832in}}%
\pgfpathlineto{\pgfqpoint{1.908897in}{1.128957in}}%
\pgfpathlineto{\pgfqpoint{1.913132in}{1.142555in}}%
\pgfpathlineto{\pgfqpoint{1.917389in}{1.156636in}}%
\pgfpathlineto{\pgfqpoint{1.921669in}{1.171208in}}%
\pgfpathlineto{\pgfqpoint{1.924558in}{1.159450in}}%
\pgfpathlineto{\pgfqpoint{1.926716in}{1.147632in}}%
\pgfpathlineto{\pgfqpoint{1.928136in}{1.135765in}}%
\pgfpathlineto{\pgfqpoint{1.928813in}{1.123862in}}%
\pgfpathlineto{\pgfqpoint{1.924463in}{1.109467in}}%
\pgfpathlineto{\pgfqpoint{1.920136in}{1.095565in}}%
\pgfpathlineto{\pgfqpoint{1.915831in}{1.082148in}}%
\pgfpathlineto{\pgfqpoint{1.911549in}{1.069208in}}%
\pgfpathlineto{\pgfqpoint{1.910915in}{1.080929in}}%
\pgfpathlineto{\pgfqpoint{1.909557in}{1.092614in}}%
\pgfpathlineto{\pgfqpoint{1.907477in}{1.104252in}}%
\pgfpathlineto{\pgfqpoint{1.904683in}{1.115832in}}%
\pgfpathclose%
\pgfusepath{fill}%
\end{pgfscope}%
\begin{pgfscope}%
\pgfpathrectangle{\pgfqpoint{0.041670in}{0.041670in}}{\pgfqpoint{2.216660in}{2.216660in}}%
\pgfusepath{clip}%
\pgfsetbuttcap%
\pgfsetroundjoin%
\definecolor{currentfill}{rgb}{0.166383,0.690856,0.496502}%
\pgfsetfillcolor{currentfill}%
\pgfsetlinewidth{0.000000pt}%
\definecolor{currentstroke}{rgb}{0.000000,0.000000,0.000000}%
\pgfsetstrokecolor{currentstroke}%
\pgfsetdash{}{0pt}%
\pgfpathmoveto{\pgfqpoint{0.979711in}{1.457942in}}%
\pgfpathlineto{\pgfqpoint{0.976871in}{1.451978in}}%
\pgfpathlineto{\pgfqpoint{0.974034in}{1.445967in}}%
\pgfpathlineto{\pgfqpoint{0.971201in}{1.439912in}}%
\pgfpathlineto{\pgfqpoint{0.968371in}{1.433815in}}%
\pgfpathlineto{\pgfqpoint{0.974070in}{1.436989in}}%
\pgfpathlineto{\pgfqpoint{0.979966in}{1.440073in}}%
\pgfpathlineto{\pgfqpoint{0.986054in}{1.443065in}}%
\pgfpathlineto{\pgfqpoint{0.992327in}{1.445962in}}%
\pgfpathlineto{\pgfqpoint{0.994832in}{1.451886in}}%
\pgfpathlineto{\pgfqpoint{0.997340in}{1.457768in}}%
\pgfpathlineto{\pgfqpoint{0.999851in}{1.463605in}}%
\pgfpathlineto{\pgfqpoint{1.002366in}{1.469397in}}%
\pgfpathlineto{\pgfqpoint{0.996433in}{1.466665in}}%
\pgfpathlineto{\pgfqpoint{0.990676in}{1.463843in}}%
\pgfpathlineto{\pgfqpoint{0.985100in}{1.460935in}}%
\pgfpathlineto{\pgfqpoint{0.979711in}{1.457942in}}%
\pgfpathclose%
\pgfusepath{fill}%
\end{pgfscope}%
\begin{pgfscope}%
\pgfpathrectangle{\pgfqpoint{0.041670in}{0.041670in}}{\pgfqpoint{2.216660in}{2.216660in}}%
\pgfusepath{clip}%
\pgfsetbuttcap%
\pgfsetroundjoin%
\definecolor{currentfill}{rgb}{0.268510,0.009605,0.335427}%
\pgfsetfillcolor{currentfill}%
\pgfsetlinewidth{0.000000pt}%
\definecolor{currentstroke}{rgb}{0.000000,0.000000,0.000000}%
\pgfsetstrokecolor{currentstroke}%
\pgfsetdash{}{0pt}%
\pgfpathmoveto{\pgfqpoint{1.778640in}{0.825451in}}%
\pgfpathlineto{\pgfqpoint{1.782448in}{0.826435in}}%
\pgfpathlineto{\pgfqpoint{1.786265in}{0.827695in}}%
\pgfpathlineto{\pgfqpoint{1.790092in}{0.829238in}}%
\pgfpathlineto{\pgfqpoint{1.793928in}{0.831067in}}%
\pgfpathlineto{\pgfqpoint{1.791199in}{0.820900in}}%
\pgfpathlineto{\pgfqpoint{1.787846in}{0.810770in}}%
\pgfpathlineto{\pgfqpoint{1.783871in}{0.800686in}}%
\pgfpathlineto{\pgfqpoint{1.779275in}{0.790660in}}%
\pgfpathlineto{\pgfqpoint{1.775509in}{0.789074in}}%
\pgfpathlineto{\pgfqpoint{1.771754in}{0.787776in}}%
\pgfpathlineto{\pgfqpoint{1.768007in}{0.786762in}}%
\pgfpathlineto{\pgfqpoint{1.764270in}{0.786025in}}%
\pgfpathlineto{\pgfqpoint{1.768771in}{0.795806in}}%
\pgfpathlineto{\pgfqpoint{1.772668in}{0.805645in}}%
\pgfpathlineto{\pgfqpoint{1.775957in}{0.815530in}}%
\pgfpathlineto{\pgfqpoint{1.778640in}{0.825451in}}%
\pgfpathclose%
\pgfusepath{fill}%
\end{pgfscope}%
\begin{pgfscope}%
\pgfpathrectangle{\pgfqpoint{0.041670in}{0.041670in}}{\pgfqpoint{2.216660in}{2.216660in}}%
\pgfusepath{clip}%
\pgfsetbuttcap%
\pgfsetroundjoin%
\definecolor{currentfill}{rgb}{0.120081,0.622161,0.534946}%
\pgfsetfillcolor{currentfill}%
\pgfsetlinewidth{0.000000pt}%
\definecolor{currentstroke}{rgb}{0.000000,0.000000,0.000000}%
\pgfsetstrokecolor{currentstroke}%
\pgfsetdash{}{0pt}%
\pgfpathmoveto{\pgfqpoint{0.917160in}{1.379184in}}%
\pgfpathlineto{\pgfqpoint{0.913828in}{1.372507in}}%
\pgfpathlineto{\pgfqpoint{0.910499in}{1.365802in}}%
\pgfpathlineto{\pgfqpoint{0.907174in}{1.359072in}}%
\pgfpathlineto{\pgfqpoint{0.903853in}{1.352320in}}%
\pgfpathlineto{\pgfqpoint{0.908220in}{1.356536in}}%
\pgfpathlineto{\pgfqpoint{0.912850in}{1.360679in}}%
\pgfpathlineto{\pgfqpoint{0.917738in}{1.364748in}}%
\pgfpathlineto{\pgfqpoint{0.922878in}{1.368736in}}%
\pgfpathlineto{\pgfqpoint{0.925963in}{1.375275in}}%
\pgfpathlineto{\pgfqpoint{0.929051in}{1.381792in}}%
\pgfpathlineto{\pgfqpoint{0.932144in}{1.388284in}}%
\pgfpathlineto{\pgfqpoint{0.935239in}{1.394749in}}%
\pgfpathlineto{\pgfqpoint{0.930353in}{1.390967in}}%
\pgfpathlineto{\pgfqpoint{0.925708in}{1.387110in}}%
\pgfpathlineto{\pgfqpoint{0.921309in}{1.383181in}}%
\pgfpathlineto{\pgfqpoint{0.917160in}{1.379184in}}%
\pgfpathclose%
\pgfusepath{fill}%
\end{pgfscope}%
\begin{pgfscope}%
\pgfpathrectangle{\pgfqpoint{0.041670in}{0.041670in}}{\pgfqpoint{2.216660in}{2.216660in}}%
\pgfusepath{clip}%
\pgfsetbuttcap%
\pgfsetroundjoin%
\definecolor{currentfill}{rgb}{0.122606,0.585371,0.546557}%
\pgfsetfillcolor{currentfill}%
\pgfsetlinewidth{0.000000pt}%
\definecolor{currentstroke}{rgb}{0.000000,0.000000,0.000000}%
\pgfsetstrokecolor{currentstroke}%
\pgfsetdash{}{0pt}%
\pgfpathmoveto{\pgfqpoint{1.452188in}{1.356071in}}%
\pgfpathlineto{\pgfqpoint{1.455457in}{1.349347in}}%
\pgfpathlineto{\pgfqpoint{1.458723in}{1.342606in}}%
\pgfpathlineto{\pgfqpoint{1.461986in}{1.335851in}}%
\pgfpathlineto{\pgfqpoint{1.465245in}{1.329084in}}%
\pgfpathlineto{\pgfqpoint{1.469800in}{1.324640in}}%
\pgfpathlineto{\pgfqpoint{1.474074in}{1.320125in}}%
\pgfpathlineto{\pgfqpoint{1.478063in}{1.315543in}}%
\pgfpathlineto{\pgfqpoint{1.481763in}{1.310898in}}%
\pgfpathlineto{\pgfqpoint{1.478306in}{1.317892in}}%
\pgfpathlineto{\pgfqpoint{1.474846in}{1.324874in}}%
\pgfpathlineto{\pgfqpoint{1.471382in}{1.331840in}}%
\pgfpathlineto{\pgfqpoint{1.467914in}{1.338789in}}%
\pgfpathlineto{\pgfqpoint{1.464393in}{1.343203in}}%
\pgfpathlineto{\pgfqpoint{1.460595in}{1.347557in}}%
\pgfpathlineto{\pgfqpoint{1.456525in}{1.351847in}}%
\pgfpathlineto{\pgfqpoint{1.452188in}{1.356071in}}%
\pgfpathclose%
\pgfusepath{fill}%
\end{pgfscope}%
\begin{pgfscope}%
\pgfpathrectangle{\pgfqpoint{0.041670in}{0.041670in}}{\pgfqpoint{2.216660in}{2.216660in}}%
\pgfusepath{clip}%
\pgfsetbuttcap%
\pgfsetroundjoin%
\definecolor{currentfill}{rgb}{0.281477,0.755203,0.432552}%
\pgfsetfillcolor{currentfill}%
\pgfsetlinewidth{0.000000pt}%
\definecolor{currentstroke}{rgb}{0.000000,0.000000,0.000000}%
\pgfsetstrokecolor{currentstroke}%
\pgfsetdash{}{0pt}%
\pgfpathmoveto{\pgfqpoint{1.258131in}{1.536855in}}%
\pgfpathlineto{\pgfqpoint{1.259388in}{1.531907in}}%
\pgfpathlineto{\pgfqpoint{1.260644in}{1.526894in}}%
\pgfpathlineto{\pgfqpoint{1.261898in}{1.521820in}}%
\pgfpathlineto{\pgfqpoint{1.263150in}{1.516686in}}%
\pgfpathlineto{\pgfqpoint{1.270342in}{1.515383in}}%
\pgfpathlineto{\pgfqpoint{1.277449in}{1.513971in}}%
\pgfpathlineto{\pgfqpoint{1.284463in}{1.512453in}}%
\pgfpathlineto{\pgfqpoint{1.291378in}{1.510831in}}%
\pgfpathlineto{\pgfqpoint{1.289699in}{1.516057in}}%
\pgfpathlineto{\pgfqpoint{1.288019in}{1.521224in}}%
\pgfpathlineto{\pgfqpoint{1.286335in}{1.526329in}}%
\pgfpathlineto{\pgfqpoint{1.284650in}{1.531370in}}%
\pgfpathlineto{\pgfqpoint{1.278154in}{1.532890in}}%
\pgfpathlineto{\pgfqpoint{1.271565in}{1.534312in}}%
\pgfpathlineto{\pgfqpoint{1.264888in}{1.535634in}}%
\pgfpathlineto{\pgfqpoint{1.258131in}{1.536855in}}%
\pgfpathclose%
\pgfusepath{fill}%
\end{pgfscope}%
\begin{pgfscope}%
\pgfpathrectangle{\pgfqpoint{0.041670in}{0.041670in}}{\pgfqpoint{2.216660in}{2.216660in}}%
\pgfusepath{clip}%
\pgfsetbuttcap%
\pgfsetroundjoin%
\definecolor{currentfill}{rgb}{0.220124,0.725509,0.466226}%
\pgfsetfillcolor{currentfill}%
\pgfsetlinewidth{0.000000pt}%
\definecolor{currentstroke}{rgb}{0.000000,0.000000,0.000000}%
\pgfsetstrokecolor{currentstroke}%
\pgfsetdash{}{0pt}%
\pgfpathmoveto{\pgfqpoint{1.317914in}{1.503319in}}%
\pgfpathlineto{\pgfqpoint{1.319992in}{1.497916in}}%
\pgfpathlineto{\pgfqpoint{1.322067in}{1.492458in}}%
\pgfpathlineto{\pgfqpoint{1.324140in}{1.486946in}}%
\pgfpathlineto{\pgfqpoint{1.326210in}{1.481383in}}%
\pgfpathlineto{\pgfqpoint{1.332914in}{1.479123in}}%
\pgfpathlineto{\pgfqpoint{1.339472in}{1.476763in}}%
\pgfpathlineto{\pgfqpoint{1.345879in}{1.474304in}}%
\pgfpathlineto{\pgfqpoint{1.352127in}{1.471748in}}%
\pgfpathlineto{\pgfqpoint{1.349687in}{1.477455in}}%
\pgfpathlineto{\pgfqpoint{1.347244in}{1.483111in}}%
\pgfpathlineto{\pgfqpoint{1.344798in}{1.488714in}}%
\pgfpathlineto{\pgfqpoint{1.342348in}{1.494262in}}%
\pgfpathlineto{\pgfqpoint{1.336458in}{1.496664in}}%
\pgfpathlineto{\pgfqpoint{1.330419in}{1.498976in}}%
\pgfpathlineto{\pgfqpoint{1.324235in}{1.501195in}}%
\pgfpathlineto{\pgfqpoint{1.317914in}{1.503319in}}%
\pgfpathclose%
\pgfusepath{fill}%
\end{pgfscope}%
\begin{pgfscope}%
\pgfpathrectangle{\pgfqpoint{0.041670in}{0.041670in}}{\pgfqpoint{2.216660in}{2.216660in}}%
\pgfusepath{clip}%
\pgfsetbuttcap%
\pgfsetroundjoin%
\definecolor{currentfill}{rgb}{0.268510,0.009605,0.335427}%
\pgfsetfillcolor{currentfill}%
\pgfsetlinewidth{0.000000pt}%
\definecolor{currentstroke}{rgb}{0.000000,0.000000,0.000000}%
\pgfsetstrokecolor{currentstroke}%
\pgfsetdash{}{0pt}%
\pgfpathmoveto{\pgfqpoint{0.644088in}{0.789055in}}%
\pgfpathlineto{\pgfqpoint{0.640464in}{0.786788in}}%
\pgfpathlineto{\pgfqpoint{0.636833in}{0.784741in}}%
\pgfpathlineto{\pgfqpoint{0.633196in}{0.782919in}}%
\pgfpathlineto{\pgfqpoint{0.629553in}{0.781325in}}%
\pgfpathlineto{\pgfqpoint{0.624734in}{0.790550in}}%
\pgfpathlineto{\pgfqpoint{0.620487in}{0.799839in}}%
\pgfpathlineto{\pgfqpoint{0.616813in}{0.809182in}}%
\pgfpathlineto{\pgfqpoint{0.613715in}{0.818569in}}%
\pgfpathlineto{\pgfqpoint{0.617444in}{0.819909in}}%
\pgfpathlineto{\pgfqpoint{0.621167in}{0.821478in}}%
\pgfpathlineto{\pgfqpoint{0.624883in}{0.823270in}}%
\pgfpathlineto{\pgfqpoint{0.628594in}{0.825282in}}%
\pgfpathlineto{\pgfqpoint{0.631630in}{0.816150in}}%
\pgfpathlineto{\pgfqpoint{0.635225in}{0.807062in}}%
\pgfpathlineto{\pgfqpoint{0.639379in}{0.798027in}}%
\pgfpathlineto{\pgfqpoint{0.644088in}{0.789055in}}%
\pgfpathclose%
\pgfusepath{fill}%
\end{pgfscope}%
\begin{pgfscope}%
\pgfpathrectangle{\pgfqpoint{0.041670in}{0.041670in}}{\pgfqpoint{2.216660in}{2.216660in}}%
\pgfusepath{clip}%
\pgfsetbuttcap%
\pgfsetroundjoin%
\definecolor{currentfill}{rgb}{0.281477,0.755203,0.432552}%
\pgfsetfillcolor{currentfill}%
\pgfsetlinewidth{0.000000pt}%
\definecolor{currentstroke}{rgb}{0.000000,0.000000,0.000000}%
\pgfsetstrokecolor{currentstroke}%
\pgfsetdash{}{0pt}%
\pgfpathmoveto{\pgfqpoint{1.069569in}{1.529937in}}%
\pgfpathlineto{\pgfqpoint{1.067792in}{1.524872in}}%
\pgfpathlineto{\pgfqpoint{1.066017in}{1.519743in}}%
\pgfpathlineto{\pgfqpoint{1.064244in}{1.514552in}}%
\pgfpathlineto{\pgfqpoint{1.062474in}{1.509301in}}%
\pgfpathlineto{\pgfqpoint{1.069295in}{1.511016in}}%
\pgfpathlineto{\pgfqpoint{1.076221in}{1.512627in}}%
\pgfpathlineto{\pgfqpoint{1.083246in}{1.514133in}}%
\pgfpathlineto{\pgfqpoint{1.090362in}{1.515533in}}%
\pgfpathlineto{\pgfqpoint{1.091711in}{1.520685in}}%
\pgfpathlineto{\pgfqpoint{1.093062in}{1.525777in}}%
\pgfpathlineto{\pgfqpoint{1.094414in}{1.530808in}}%
\pgfpathlineto{\pgfqpoint{1.095769in}{1.535775in}}%
\pgfpathlineto{\pgfqpoint{1.089083in}{1.534464in}}%
\pgfpathlineto{\pgfqpoint{1.082483in}{1.533053in}}%
\pgfpathlineto{\pgfqpoint{1.075977in}{1.531544in}}%
\pgfpathlineto{\pgfqpoint{1.069569in}{1.529937in}}%
\pgfpathclose%
\pgfusepath{fill}%
\end{pgfscope}%
\begin{pgfscope}%
\pgfpathrectangle{\pgfqpoint{0.041670in}{0.041670in}}{\pgfqpoint{2.216660in}{2.216660in}}%
\pgfusepath{clip}%
\pgfsetbuttcap%
\pgfsetroundjoin%
\definecolor{currentfill}{rgb}{0.212395,0.359683,0.551710}%
\pgfsetfillcolor{currentfill}%
\pgfsetlinewidth{0.000000pt}%
\definecolor{currentstroke}{rgb}{0.000000,0.000000,0.000000}%
\pgfsetstrokecolor{currentstroke}%
\pgfsetdash{}{0pt}%
\pgfpathmoveto{\pgfqpoint{0.783934in}{1.093188in}}%
\pgfpathlineto{\pgfqpoint{0.780259in}{1.085779in}}%
\pgfpathlineto{\pgfqpoint{0.776586in}{1.078422in}}%
\pgfpathlineto{\pgfqpoint{0.772914in}{1.071119in}}%
\pgfpathlineto{\pgfqpoint{0.769244in}{1.063875in}}%
\pgfpathlineto{\pgfqpoint{0.768935in}{1.070468in}}%
\pgfpathlineto{\pgfqpoint{0.769035in}{1.077056in}}%
\pgfpathlineto{\pgfqpoint{0.769545in}{1.083633in}}%
\pgfpathlineto{\pgfqpoint{0.770463in}{1.090192in}}%
\pgfpathlineto{\pgfqpoint{0.774109in}{1.097175in}}%
\pgfpathlineto{\pgfqpoint{0.777756in}{1.104217in}}%
\pgfpathlineto{\pgfqpoint{0.781406in}{1.111313in}}%
\pgfpathlineto{\pgfqpoint{0.785057in}{1.118461in}}%
\pgfpathlineto{\pgfqpoint{0.784185in}{1.112162in}}%
\pgfpathlineto{\pgfqpoint{0.783707in}{1.105845in}}%
\pgfpathlineto{\pgfqpoint{0.783623in}{1.099519in}}%
\pgfpathlineto{\pgfqpoint{0.783934in}{1.093188in}}%
\pgfpathclose%
\pgfusepath{fill}%
\end{pgfscope}%
\begin{pgfscope}%
\pgfpathrectangle{\pgfqpoint{0.041670in}{0.041670in}}{\pgfqpoint{2.216660in}{2.216660in}}%
\pgfusepath{clip}%
\pgfsetbuttcap%
\pgfsetroundjoin%
\definecolor{currentfill}{rgb}{0.271305,0.019942,0.347269}%
\pgfsetfillcolor{currentfill}%
\pgfsetlinewidth{0.000000pt}%
\definecolor{currentstroke}{rgb}{0.000000,0.000000,0.000000}%
\pgfsetstrokecolor{currentstroke}%
\pgfsetdash{}{0pt}%
\pgfpathmoveto{\pgfqpoint{0.658532in}{0.800227in}}%
\pgfpathlineto{\pgfqpoint{0.654929in}{0.797127in}}%
\pgfpathlineto{\pgfqpoint{0.651321in}{0.794229in}}%
\pgfpathlineto{\pgfqpoint{0.647707in}{0.791536in}}%
\pgfpathlineto{\pgfqpoint{0.644088in}{0.789055in}}%
\pgfpathlineto{\pgfqpoint{0.639379in}{0.798027in}}%
\pgfpathlineto{\pgfqpoint{0.635225in}{0.807062in}}%
\pgfpathlineto{\pgfqpoint{0.631630in}{0.816150in}}%
\pgfpathlineto{\pgfqpoint{0.628594in}{0.825282in}}%
\pgfpathlineto{\pgfqpoint{0.632299in}{0.827508in}}%
\pgfpathlineto{\pgfqpoint{0.635999in}{0.829944in}}%
\pgfpathlineto{\pgfqpoint{0.639694in}{0.832586in}}%
\pgfpathlineto{\pgfqpoint{0.643384in}{0.835429in}}%
\pgfpathlineto{\pgfqpoint{0.646356in}{0.826554in}}%
\pgfpathlineto{\pgfqpoint{0.649873in}{0.817723in}}%
\pgfpathlineto{\pgfqpoint{0.653932in}{0.808944in}}%
\pgfpathlineto{\pgfqpoint{0.658532in}{0.800227in}}%
\pgfpathclose%
\pgfusepath{fill}%
\end{pgfscope}%
\begin{pgfscope}%
\pgfpathrectangle{\pgfqpoint{0.041670in}{0.041670in}}{\pgfqpoint{2.216660in}{2.216660in}}%
\pgfusepath{clip}%
\pgfsetbuttcap%
\pgfsetroundjoin%
\definecolor{currentfill}{rgb}{0.267004,0.004874,0.329415}%
\pgfsetfillcolor{currentfill}%
\pgfsetlinewidth{0.000000pt}%
\definecolor{currentstroke}{rgb}{0.000000,0.000000,0.000000}%
\pgfsetstrokecolor{currentstroke}%
\pgfsetdash{}{0pt}%
\pgfpathmoveto{\pgfqpoint{0.629553in}{0.781325in}}%
\pgfpathlineto{\pgfqpoint{0.625904in}{0.779965in}}%
\pgfpathlineto{\pgfqpoint{0.622247in}{0.778843in}}%
\pgfpathlineto{\pgfqpoint{0.618583in}{0.777964in}}%
\pgfpathlineto{\pgfqpoint{0.614912in}{0.777333in}}%
\pgfpathlineto{\pgfqpoint{0.609984in}{0.786808in}}%
\pgfpathlineto{\pgfqpoint{0.605643in}{0.796348in}}%
\pgfpathlineto{\pgfqpoint{0.601891in}{0.805944in}}%
\pgfpathlineto{\pgfqpoint{0.598731in}{0.815584in}}%
\pgfpathlineto{\pgfqpoint{0.602488in}{0.815964in}}%
\pgfpathlineto{\pgfqpoint{0.606237in}{0.816592in}}%
\pgfpathlineto{\pgfqpoint{0.609979in}{0.817462in}}%
\pgfpathlineto{\pgfqpoint{0.613715in}{0.818569in}}%
\pgfpathlineto{\pgfqpoint{0.616813in}{0.809182in}}%
\pgfpathlineto{\pgfqpoint{0.620487in}{0.799839in}}%
\pgfpathlineto{\pgfqpoint{0.624734in}{0.790550in}}%
\pgfpathlineto{\pgfqpoint{0.629553in}{0.781325in}}%
\pgfpathclose%
\pgfusepath{fill}%
\end{pgfscope}%
\begin{pgfscope}%
\pgfpathrectangle{\pgfqpoint{0.041670in}{0.041670in}}{\pgfqpoint{2.216660in}{2.216660in}}%
\pgfusepath{clip}%
\pgfsetbuttcap%
\pgfsetroundjoin%
\definecolor{currentfill}{rgb}{0.282327,0.094955,0.417331}%
\pgfsetfillcolor{currentfill}%
\pgfsetlinewidth{0.000000pt}%
\definecolor{currentstroke}{rgb}{0.000000,0.000000,0.000000}%
\pgfsetstrokecolor{currentstroke}%
\pgfsetdash{}{0pt}%
\pgfpathmoveto{\pgfqpoint{1.674596in}{0.890915in}}%
\pgfpathlineto{\pgfqpoint{1.678255in}{0.886044in}}%
\pgfpathlineto{\pgfqpoint{1.681917in}{0.881325in}}%
\pgfpathlineto{\pgfqpoint{1.685582in}{0.876761in}}%
\pgfpathlineto{\pgfqpoint{1.689249in}{0.872357in}}%
\pgfpathlineto{\pgfqpoint{1.686864in}{0.863973in}}%
\pgfpathlineto{\pgfqpoint{1.683964in}{0.855623in}}%
\pgfpathlineto{\pgfqpoint{1.680549in}{0.847316in}}%
\pgfpathlineto{\pgfqpoint{1.676623in}{0.839061in}}%
\pgfpathlineto{\pgfqpoint{1.673030in}{0.843726in}}%
\pgfpathlineto{\pgfqpoint{1.669441in}{0.848551in}}%
\pgfpathlineto{\pgfqpoint{1.665854in}{0.853532in}}%
\pgfpathlineto{\pgfqpoint{1.662269in}{0.858664in}}%
\pgfpathlineto{\pgfqpoint{1.666098in}{0.866660in}}%
\pgfpathlineto{\pgfqpoint{1.669430in}{0.874706in}}%
\pgfpathlineto{\pgfqpoint{1.672263in}{0.882794in}}%
\pgfpathlineto{\pgfqpoint{1.674596in}{0.890915in}}%
\pgfpathclose%
\pgfusepath{fill}%
\end{pgfscope}%
\begin{pgfscope}%
\pgfpathrectangle{\pgfqpoint{0.041670in}{0.041670in}}{\pgfqpoint{2.216660in}{2.216660in}}%
\pgfusepath{clip}%
\pgfsetbuttcap%
\pgfsetroundjoin%
\definecolor{currentfill}{rgb}{0.220124,0.725509,0.466226}%
\pgfsetfillcolor{currentfill}%
\pgfsetlinewidth{0.000000pt}%
\definecolor{currentstroke}{rgb}{0.000000,0.000000,0.000000}%
\pgfsetstrokecolor{currentstroke}%
\pgfsetdash{}{0pt}%
\pgfpathmoveto{\pgfqpoint{1.012456in}{1.492052in}}%
\pgfpathlineto{\pgfqpoint{1.009929in}{1.486469in}}%
\pgfpathlineto{\pgfqpoint{1.007405in}{1.480831in}}%
\pgfpathlineto{\pgfqpoint{1.004884in}{1.475139in}}%
\pgfpathlineto{\pgfqpoint{1.002366in}{1.469397in}}%
\pgfpathlineto{\pgfqpoint{1.008469in}{1.472037in}}%
\pgfpathlineto{\pgfqpoint{1.014735in}{1.474582in}}%
\pgfpathlineto{\pgfqpoint{1.021159in}{1.477030in}}%
\pgfpathlineto{\pgfqpoint{1.027734in}{1.479380in}}%
\pgfpathlineto{\pgfqpoint{1.029889in}{1.484972in}}%
\pgfpathlineto{\pgfqpoint{1.032046in}{1.490514in}}%
\pgfpathlineto{\pgfqpoint{1.034207in}{1.496003in}}%
\pgfpathlineto{\pgfqpoint{1.036370in}{1.501436in}}%
\pgfpathlineto{\pgfqpoint{1.030171in}{1.499227in}}%
\pgfpathlineto{\pgfqpoint{1.024115in}{1.496926in}}%
\pgfpathlineto{\pgfqpoint{1.018208in}{1.494533in}}%
\pgfpathlineto{\pgfqpoint{1.012456in}{1.492052in}}%
\pgfpathclose%
\pgfusepath{fill}%
\end{pgfscope}%
\begin{pgfscope}%
\pgfpathrectangle{\pgfqpoint{0.041670in}{0.041670in}}{\pgfqpoint{2.216660in}{2.216660in}}%
\pgfusepath{clip}%
\pgfsetbuttcap%
\pgfsetroundjoin%
\definecolor{currentfill}{rgb}{0.263663,0.237631,0.518762}%
\pgfsetfillcolor{currentfill}%
\pgfsetlinewidth{0.000000pt}%
\definecolor{currentstroke}{rgb}{0.000000,0.000000,0.000000}%
\pgfsetstrokecolor{currentstroke}%
\pgfsetdash{}{0pt}%
\pgfpathmoveto{\pgfqpoint{0.745535in}{0.980021in}}%
\pgfpathlineto{\pgfqpoint{0.741904in}{0.973186in}}%
\pgfpathlineto{\pgfqpoint{0.738273in}{0.966442in}}%
\pgfpathlineto{\pgfqpoint{0.734642in}{0.959794in}}%
\pgfpathlineto{\pgfqpoint{0.731011in}{0.953245in}}%
\pgfpathlineto{\pgfqpoint{0.728890in}{0.960579in}}%
\pgfpathlineto{\pgfqpoint{0.727226in}{0.967936in}}%
\pgfpathlineto{\pgfqpoint{0.726017in}{0.975309in}}%
\pgfpathlineto{\pgfqpoint{0.725265in}{0.982690in}}%
\pgfpathlineto{\pgfqpoint{0.728928in}{0.988975in}}%
\pgfpathlineto{\pgfqpoint{0.732591in}{0.995359in}}%
\pgfpathlineto{\pgfqpoint{0.736253in}{1.001839in}}%
\pgfpathlineto{\pgfqpoint{0.739917in}{1.008410in}}%
\pgfpathlineto{\pgfqpoint{0.740659in}{1.001293in}}%
\pgfpathlineto{\pgfqpoint{0.741843in}{0.994184in}}%
\pgfpathlineto{\pgfqpoint{0.743469in}{0.987091in}}%
\pgfpathlineto{\pgfqpoint{0.745535in}{0.980021in}}%
\pgfpathclose%
\pgfusepath{fill}%
\end{pgfscope}%
\begin{pgfscope}%
\pgfpathrectangle{\pgfqpoint{0.041670in}{0.041670in}}{\pgfqpoint{2.216660in}{2.216660in}}%
\pgfusepath{clip}%
\pgfsetbuttcap%
\pgfsetroundjoin%
\definecolor{currentfill}{rgb}{0.274952,0.037752,0.364543}%
\pgfsetfillcolor{currentfill}%
\pgfsetlinewidth{0.000000pt}%
\definecolor{currentstroke}{rgb}{0.000000,0.000000,0.000000}%
\pgfsetstrokecolor{currentstroke}%
\pgfsetdash{}{0pt}%
\pgfpathmoveto{\pgfqpoint{0.672900in}{0.814558in}}%
\pgfpathlineto{\pgfqpoint{0.669315in}{0.810694in}}%
\pgfpathlineto{\pgfqpoint{0.665725in}{0.807015in}}%
\pgfpathlineto{\pgfqpoint{0.662131in}{0.803525in}}%
\pgfpathlineto{\pgfqpoint{0.658532in}{0.800227in}}%
\pgfpathlineto{\pgfqpoint{0.653932in}{0.808944in}}%
\pgfpathlineto{\pgfqpoint{0.649873in}{0.817723in}}%
\pgfpathlineto{\pgfqpoint{0.646356in}{0.826554in}}%
\pgfpathlineto{\pgfqpoint{0.643384in}{0.835429in}}%
\pgfpathlineto{\pgfqpoint{0.647070in}{0.838468in}}%
\pgfpathlineto{\pgfqpoint{0.650751in}{0.841701in}}%
\pgfpathlineto{\pgfqpoint{0.654427in}{0.845122in}}%
\pgfpathlineto{\pgfqpoint{0.658100in}{0.848727in}}%
\pgfpathlineto{\pgfqpoint{0.661008in}{0.840112in}}%
\pgfpathlineto{\pgfqpoint{0.664445in}{0.831540in}}%
\pgfpathlineto{\pgfqpoint{0.668410in}{0.823018in}}%
\pgfpathlineto{\pgfqpoint{0.672900in}{0.814558in}}%
\pgfpathclose%
\pgfusepath{fill}%
\end{pgfscope}%
\begin{pgfscope}%
\pgfpathrectangle{\pgfqpoint{0.041670in}{0.041670in}}{\pgfqpoint{2.216660in}{2.216660in}}%
\pgfusepath{clip}%
\pgfsetbuttcap%
\pgfsetroundjoin%
\definecolor{currentfill}{rgb}{0.122606,0.585371,0.546557}%
\pgfsetfillcolor{currentfill}%
\pgfsetlinewidth{0.000000pt}%
\definecolor{currentstroke}{rgb}{0.000000,0.000000,0.000000}%
\pgfsetstrokecolor{currentstroke}%
\pgfsetdash{}{0pt}%
\pgfpathmoveto{\pgfqpoint{0.889101in}{1.334820in}}%
\pgfpathlineto{\pgfqpoint{0.885597in}{1.327819in}}%
\pgfpathlineto{\pgfqpoint{0.882096in}{1.320800in}}%
\pgfpathlineto{\pgfqpoint{0.878598in}{1.313767in}}%
\pgfpathlineto{\pgfqpoint{0.875104in}{1.306721in}}%
\pgfpathlineto{\pgfqpoint{0.878543in}{1.311417in}}%
\pgfpathlineto{\pgfqpoint{0.882276in}{1.316055in}}%
\pgfpathlineto{\pgfqpoint{0.886297in}{1.320630in}}%
\pgfpathlineto{\pgfqpoint{0.890602in}{1.325137in}}%
\pgfpathlineto{\pgfqpoint{0.893910in}{1.331954in}}%
\pgfpathlineto{\pgfqpoint{0.897221in}{1.338758in}}%
\pgfpathlineto{\pgfqpoint{0.900535in}{1.345548in}}%
\pgfpathlineto{\pgfqpoint{0.903853in}{1.352320in}}%
\pgfpathlineto{\pgfqpoint{0.899753in}{1.348037in}}%
\pgfpathlineto{\pgfqpoint{0.895925in}{1.343689in}}%
\pgfpathlineto{\pgfqpoint{0.892373in}{1.339282in}}%
\pgfpathlineto{\pgfqpoint{0.889101in}{1.334820in}}%
\pgfpathclose%
\pgfusepath{fill}%
\end{pgfscope}%
\begin{pgfscope}%
\pgfpathrectangle{\pgfqpoint{0.041670in}{0.041670in}}{\pgfqpoint{2.216660in}{2.216660in}}%
\pgfusepath{clip}%
\pgfsetbuttcap%
\pgfsetroundjoin%
\definecolor{currentfill}{rgb}{0.163625,0.471133,0.558148}%
\pgfsetfillcolor{currentfill}%
\pgfsetlinewidth{0.000000pt}%
\definecolor{currentstroke}{rgb}{0.000000,0.000000,0.000000}%
\pgfsetstrokecolor{currentstroke}%
\pgfsetdash{}{0pt}%
\pgfpathmoveto{\pgfqpoint{1.522293in}{1.233799in}}%
\pgfpathlineto{\pgfqpoint{1.525867in}{1.226578in}}%
\pgfpathlineto{\pgfqpoint{1.529439in}{1.219372in}}%
\pgfpathlineto{\pgfqpoint{1.533007in}{1.212185in}}%
\pgfpathlineto{\pgfqpoint{1.536573in}{1.205019in}}%
\pgfpathlineto{\pgfqpoint{1.539105in}{1.199387in}}%
\pgfpathlineto{\pgfqpoint{1.541283in}{1.193713in}}%
\pgfpathlineto{\pgfqpoint{1.543106in}{1.188003in}}%
\pgfpathlineto{\pgfqpoint{1.544572in}{1.182262in}}%
\pgfpathlineto{\pgfqpoint{1.540914in}{1.189680in}}%
\pgfpathlineto{\pgfqpoint{1.537253in}{1.197119in}}%
\pgfpathlineto{\pgfqpoint{1.533590in}{1.204576in}}%
\pgfpathlineto{\pgfqpoint{1.529924in}{1.212048in}}%
\pgfpathlineto{\pgfqpoint{1.528530in}{1.217534in}}%
\pgfpathlineto{\pgfqpoint{1.526791in}{1.222992in}}%
\pgfpathlineto{\pgfqpoint{1.524711in}{1.228415in}}%
\pgfpathlineto{\pgfqpoint{1.522293in}{1.233799in}}%
\pgfpathclose%
\pgfusepath{fill}%
\end{pgfscope}%
\begin{pgfscope}%
\pgfpathrectangle{\pgfqpoint{0.041670in}{0.041670in}}{\pgfqpoint{2.216660in}{2.216660in}}%
\pgfusepath{clip}%
\pgfsetbuttcap%
\pgfsetroundjoin%
\definecolor{currentfill}{rgb}{0.267004,0.004874,0.329415}%
\pgfsetfillcolor{currentfill}%
\pgfsetlinewidth{0.000000pt}%
\definecolor{currentstroke}{rgb}{0.000000,0.000000,0.000000}%
\pgfsetstrokecolor{currentstroke}%
\pgfsetdash{}{0pt}%
\pgfpathmoveto{\pgfqpoint{0.614912in}{0.777333in}}%
\pgfpathlineto{\pgfqpoint{0.611233in}{0.776955in}}%
\pgfpathlineto{\pgfqpoint{0.607546in}{0.776834in}}%
\pgfpathlineto{\pgfqpoint{0.603850in}{0.776976in}}%
\pgfpathlineto{\pgfqpoint{0.600147in}{0.777386in}}%
\pgfpathlineto{\pgfqpoint{0.595110in}{0.787108in}}%
\pgfpathlineto{\pgfqpoint{0.590675in}{0.796897in}}%
\pgfpathlineto{\pgfqpoint{0.586847in}{0.806741in}}%
\pgfpathlineto{\pgfqpoint{0.583624in}{0.816631in}}%
\pgfpathlineto{\pgfqpoint{0.587413in}{0.815974in}}%
\pgfpathlineto{\pgfqpoint{0.591194in}{0.815583in}}%
\pgfpathlineto{\pgfqpoint{0.594966in}{0.815455in}}%
\pgfpathlineto{\pgfqpoint{0.598731in}{0.815584in}}%
\pgfpathlineto{\pgfqpoint{0.601891in}{0.805944in}}%
\pgfpathlineto{\pgfqpoint{0.605643in}{0.796348in}}%
\pgfpathlineto{\pgfqpoint{0.609984in}{0.786808in}}%
\pgfpathlineto{\pgfqpoint{0.614912in}{0.777333in}}%
\pgfpathclose%
\pgfusepath{fill}%
\end{pgfscope}%
\begin{pgfscope}%
\pgfpathrectangle{\pgfqpoint{0.041670in}{0.041670in}}{\pgfqpoint{2.216660in}{2.216660in}}%
\pgfusepath{clip}%
\pgfsetbuttcap%
\pgfsetroundjoin%
\definecolor{currentfill}{rgb}{0.272594,0.025563,0.353093}%
\pgfsetfillcolor{currentfill}%
\pgfsetlinewidth{0.000000pt}%
\definecolor{currentstroke}{rgb}{0.000000,0.000000,0.000000}%
\pgfsetstrokecolor{currentstroke}%
\pgfsetdash{}{0pt}%
\pgfpathmoveto{\pgfqpoint{1.793928in}{0.831067in}}%
\pgfpathlineto{\pgfqpoint{1.797775in}{0.833189in}}%
\pgfpathlineto{\pgfqpoint{1.801631in}{0.835609in}}%
\pgfpathlineto{\pgfqpoint{1.805498in}{0.838331in}}%
\pgfpathlineto{\pgfqpoint{1.809377in}{0.841362in}}%
\pgfpathlineto{\pgfqpoint{1.806601in}{0.830954in}}%
\pgfpathlineto{\pgfqpoint{1.803185in}{0.820582in}}%
\pgfpathlineto{\pgfqpoint{1.799131in}{0.810258in}}%
\pgfpathlineto{\pgfqpoint{1.794441in}{0.799991in}}%
\pgfpathlineto{\pgfqpoint{1.790633in}{0.797199in}}%
\pgfpathlineto{\pgfqpoint{1.786836in}{0.794717in}}%
\pgfpathlineto{\pgfqpoint{1.783050in}{0.792539in}}%
\pgfpathlineto{\pgfqpoint{1.779275in}{0.790660in}}%
\pgfpathlineto{\pgfqpoint{1.783871in}{0.800686in}}%
\pgfpathlineto{\pgfqpoint{1.787846in}{0.810770in}}%
\pgfpathlineto{\pgfqpoint{1.791199in}{0.820900in}}%
\pgfpathlineto{\pgfqpoint{1.793928in}{0.831067in}}%
\pgfpathclose%
\pgfusepath{fill}%
\end{pgfscope}%
\begin{pgfscope}%
\pgfpathrectangle{\pgfqpoint{0.041670in}{0.041670in}}{\pgfqpoint{2.216660in}{2.216660in}}%
\pgfusepath{clip}%
\pgfsetbuttcap%
\pgfsetroundjoin%
\definecolor{currentfill}{rgb}{0.248629,0.278775,0.534556}%
\pgfsetfillcolor{currentfill}%
\pgfsetlinewidth{0.000000pt}%
\definecolor{currentstroke}{rgb}{0.000000,0.000000,0.000000}%
\pgfsetstrokecolor{currentstroke}%
\pgfsetdash{}{0pt}%
\pgfpathmoveto{\pgfqpoint{1.605629in}{1.041638in}}%
\pgfpathlineto{\pgfqpoint{1.609294in}{1.034793in}}%
\pgfpathlineto{\pgfqpoint{1.612958in}{1.028025in}}%
\pgfpathlineto{\pgfqpoint{1.616621in}{1.021339in}}%
\pgfpathlineto{\pgfqpoint{1.620283in}{1.014738in}}%
\pgfpathlineto{\pgfqpoint{1.619932in}{1.007619in}}%
\pgfpathlineto{\pgfqpoint{1.619141in}{1.000502in}}%
\pgfpathlineto{\pgfqpoint{1.617908in}{0.993395in}}%
\pgfpathlineto{\pgfqpoint{1.616233in}{0.986304in}}%
\pgfpathlineto{\pgfqpoint{1.612590in}{0.993169in}}%
\pgfpathlineto{\pgfqpoint{1.608946in}{1.000119in}}%
\pgfpathlineto{\pgfqpoint{1.605302in}{1.007150in}}%
\pgfpathlineto{\pgfqpoint{1.601658in}{1.014259in}}%
\pgfpathlineto{\pgfqpoint{1.603290in}{1.021086in}}%
\pgfpathlineto{\pgfqpoint{1.604496in}{1.027930in}}%
\pgfpathlineto{\pgfqpoint{1.605276in}{1.034783in}}%
\pgfpathlineto{\pgfqpoint{1.605629in}{1.041638in}}%
\pgfpathclose%
\pgfusepath{fill}%
\end{pgfscope}%
\begin{pgfscope}%
\pgfpathrectangle{\pgfqpoint{0.041670in}{0.041670in}}{\pgfqpoint{2.216660in}{2.216660in}}%
\pgfusepath{clip}%
\pgfsetbuttcap%
\pgfsetroundjoin%
\definecolor{currentfill}{rgb}{0.279566,0.067836,0.391917}%
\pgfsetfillcolor{currentfill}%
\pgfsetlinewidth{0.000000pt}%
\definecolor{currentstroke}{rgb}{0.000000,0.000000,0.000000}%
\pgfsetstrokecolor{currentstroke}%
\pgfsetdash{}{0pt}%
\pgfpathmoveto{\pgfqpoint{0.687205in}{0.831772in}}%
\pgfpathlineto{\pgfqpoint{0.683634in}{0.827213in}}%
\pgfpathlineto{\pgfqpoint{0.680060in}{0.822821in}}%
\pgfpathlineto{\pgfqpoint{0.676482in}{0.818602in}}%
\pgfpathlineto{\pgfqpoint{0.672900in}{0.814558in}}%
\pgfpathlineto{\pgfqpoint{0.668410in}{0.823018in}}%
\pgfpathlineto{\pgfqpoint{0.664445in}{0.831540in}}%
\pgfpathlineto{\pgfqpoint{0.661008in}{0.840112in}}%
\pgfpathlineto{\pgfqpoint{0.658100in}{0.848727in}}%
\pgfpathlineto{\pgfqpoint{0.661769in}{0.852511in}}%
\pgfpathlineto{\pgfqpoint{0.665435in}{0.856472in}}%
\pgfpathlineto{\pgfqpoint{0.669097in}{0.860604in}}%
\pgfpathlineto{\pgfqpoint{0.672755in}{0.864903in}}%
\pgfpathlineto{\pgfqpoint{0.675598in}{0.856549in}}%
\pgfpathlineto{\pgfqpoint{0.678956in}{0.848237in}}%
\pgfpathlineto{\pgfqpoint{0.682825in}{0.839975in}}%
\pgfpathlineto{\pgfqpoint{0.687205in}{0.831772in}}%
\pgfpathclose%
\pgfusepath{fill}%
\end{pgfscope}%
\begin{pgfscope}%
\pgfpathrectangle{\pgfqpoint{0.041670in}{0.041670in}}{\pgfqpoint{2.216660in}{2.216660in}}%
\pgfusepath{clip}%
\pgfsetbuttcap%
\pgfsetroundjoin%
\definecolor{currentfill}{rgb}{0.195860,0.395433,0.555276}%
\pgfsetfillcolor{currentfill}%
\pgfsetlinewidth{0.000000pt}%
\definecolor{currentstroke}{rgb}{0.000000,0.000000,0.000000}%
\pgfsetstrokecolor{currentstroke}%
\pgfsetdash{}{0pt}%
\pgfpathmoveto{\pgfqpoint{1.559179in}{1.152862in}}%
\pgfpathlineto{\pgfqpoint{1.562825in}{1.145595in}}%
\pgfpathlineto{\pgfqpoint{1.566468in}{1.138368in}}%
\pgfpathlineto{\pgfqpoint{1.570110in}{1.131182in}}%
\pgfpathlineto{\pgfqpoint{1.573750in}{1.124043in}}%
\pgfpathlineto{\pgfqpoint{1.574969in}{1.117762in}}%
\pgfpathlineto{\pgfqpoint{1.575797in}{1.111460in}}%
\pgfpathlineto{\pgfqpoint{1.576232in}{1.105143in}}%
\pgfpathlineto{\pgfqpoint{1.576272in}{1.098815in}}%
\pgfpathlineto{\pgfqpoint{1.572596in}{1.106215in}}%
\pgfpathlineto{\pgfqpoint{1.568918in}{1.113659in}}%
\pgfpathlineto{\pgfqpoint{1.565238in}{1.121146in}}%
\pgfpathlineto{\pgfqpoint{1.561556in}{1.128672in}}%
\pgfpathlineto{\pgfqpoint{1.561531in}{1.134739in}}%
\pgfpathlineto{\pgfqpoint{1.561125in}{1.140797in}}%
\pgfpathlineto{\pgfqpoint{1.560340in}{1.146840in}}%
\pgfpathlineto{\pgfqpoint{1.559179in}{1.152862in}}%
\pgfpathclose%
\pgfusepath{fill}%
\end{pgfscope}%
\begin{pgfscope}%
\pgfpathrectangle{\pgfqpoint{0.041670in}{0.041670in}}{\pgfqpoint{2.216660in}{2.216660in}}%
\pgfusepath{clip}%
\pgfsetbuttcap%
\pgfsetroundjoin%
\definecolor{currentfill}{rgb}{0.133743,0.548535,0.553541}%
\pgfsetfillcolor{currentfill}%
\pgfsetlinewidth{0.000000pt}%
\definecolor{currentstroke}{rgb}{0.000000,0.000000,0.000000}%
\pgfsetstrokecolor{currentstroke}%
\pgfsetdash{}{0pt}%
\pgfpathmoveto{\pgfqpoint{1.481763in}{1.310898in}}%
\pgfpathlineto{\pgfqpoint{1.485217in}{1.303895in}}%
\pgfpathlineto{\pgfqpoint{1.488667in}{1.296885in}}%
\pgfpathlineto{\pgfqpoint{1.492114in}{1.289871in}}%
\pgfpathlineto{\pgfqpoint{1.495558in}{1.282856in}}%
\pgfpathlineto{\pgfqpoint{1.499130in}{1.277918in}}%
\pgfpathlineto{\pgfqpoint{1.502391in}{1.272923in}}%
\pgfpathlineto{\pgfqpoint{1.505337in}{1.267875in}}%
\pgfpathlineto{\pgfqpoint{1.507965in}{1.262781in}}%
\pgfpathlineto{\pgfqpoint{1.504376in}{1.270037in}}%
\pgfpathlineto{\pgfqpoint{1.500783in}{1.277291in}}%
\pgfpathlineto{\pgfqpoint{1.497187in}{1.284540in}}%
\pgfpathlineto{\pgfqpoint{1.493588in}{1.291783in}}%
\pgfpathlineto{\pgfqpoint{1.491085in}{1.296633in}}%
\pgfpathlineto{\pgfqpoint{1.488278in}{1.301439in}}%
\pgfpathlineto{\pgfqpoint{1.485169in}{1.306196in}}%
\pgfpathlineto{\pgfqpoint{1.481763in}{1.310898in}}%
\pgfpathclose%
\pgfusepath{fill}%
\end{pgfscope}%
\begin{pgfscope}%
\pgfpathrectangle{\pgfqpoint{0.041670in}{0.041670in}}{\pgfqpoint{2.216660in}{2.216660in}}%
\pgfusepath{clip}%
\pgfsetbuttcap%
\pgfsetroundjoin%
\definecolor{currentfill}{rgb}{0.283072,0.130895,0.449241}%
\pgfsetfillcolor{currentfill}%
\pgfsetlinewidth{0.000000pt}%
\definecolor{currentstroke}{rgb}{0.000000,0.000000,0.000000}%
\pgfsetstrokecolor{currentstroke}%
\pgfsetdash{}{0pt}%
\pgfpathmoveto{\pgfqpoint{1.659978in}{0.911832in}}%
\pgfpathlineto{\pgfqpoint{1.663630in}{0.906395in}}%
\pgfpathlineto{\pgfqpoint{1.667283in}{0.901094in}}%
\pgfpathlineto{\pgfqpoint{1.670939in}{0.895933in}}%
\pgfpathlineto{\pgfqpoint{1.674596in}{0.890915in}}%
\pgfpathlineto{\pgfqpoint{1.672263in}{0.882794in}}%
\pgfpathlineto{\pgfqpoint{1.669430in}{0.874706in}}%
\pgfpathlineto{\pgfqpoint{1.666098in}{0.866660in}}%
\pgfpathlineto{\pgfqpoint{1.662269in}{0.858664in}}%
\pgfpathlineto{\pgfqpoint{1.658687in}{0.863944in}}%
\pgfpathlineto{\pgfqpoint{1.655107in}{0.869367in}}%
\pgfpathlineto{\pgfqpoint{1.651530in}{0.874930in}}%
\pgfpathlineto{\pgfqpoint{1.647954in}{0.880629in}}%
\pgfpathlineto{\pgfqpoint{1.651685in}{0.888365in}}%
\pgfpathlineto{\pgfqpoint{1.654933in}{0.896149in}}%
\pgfpathlineto{\pgfqpoint{1.657698in}{0.903974in}}%
\pgfpathlineto{\pgfqpoint{1.659978in}{0.911832in}}%
\pgfpathclose%
\pgfusepath{fill}%
\end{pgfscope}%
\begin{pgfscope}%
\pgfpathrectangle{\pgfqpoint{0.041670in}{0.041670in}}{\pgfqpoint{2.216660in}{2.216660in}}%
\pgfusepath{clip}%
\pgfsetbuttcap%
\pgfsetroundjoin%
\definecolor{currentfill}{rgb}{0.344074,0.780029,0.397381}%
\pgfsetfillcolor{currentfill}%
\pgfsetlinewidth{0.000000pt}%
\definecolor{currentstroke}{rgb}{0.000000,0.000000,0.000000}%
\pgfsetstrokecolor{currentstroke}%
\pgfsetdash{}{0pt}%
\pgfpathmoveto{\pgfqpoint{1.153416in}{1.561278in}}%
\pgfpathlineto{\pgfqpoint{1.152958in}{1.556702in}}%
\pgfpathlineto{\pgfqpoint{1.152500in}{1.552053in}}%
\pgfpathlineto{\pgfqpoint{1.152043in}{1.547334in}}%
\pgfpathlineto{\pgfqpoint{1.151587in}{1.542548in}}%
\pgfpathlineto{\pgfqpoint{1.158755in}{1.542918in}}%
\pgfpathlineto{\pgfqpoint{1.165944in}{1.543182in}}%
\pgfpathlineto{\pgfqpoint{1.173147in}{1.543337in}}%
\pgfpathlineto{\pgfqpoint{1.180355in}{1.543385in}}%
\pgfpathlineto{\pgfqpoint{1.180349in}{1.548158in}}%
\pgfpathlineto{\pgfqpoint{1.180343in}{1.552863in}}%
\pgfpathlineto{\pgfqpoint{1.180336in}{1.557497in}}%
\pgfpathlineto{\pgfqpoint{1.180330in}{1.562060in}}%
\pgfpathlineto{\pgfqpoint{1.173586in}{1.562015in}}%
\pgfpathlineto{\pgfqpoint{1.166848in}{1.561870in}}%
\pgfpathlineto{\pgfqpoint{1.160123in}{1.561624in}}%
\pgfpathlineto{\pgfqpoint{1.153416in}{1.561278in}}%
\pgfpathclose%
\pgfusepath{fill}%
\end{pgfscope}%
\begin{pgfscope}%
\pgfpathrectangle{\pgfqpoint{0.041670in}{0.041670in}}{\pgfqpoint{2.216660in}{2.216660in}}%
\pgfusepath{clip}%
\pgfsetbuttcap%
\pgfsetroundjoin%
\definecolor{currentfill}{rgb}{0.344074,0.780029,0.397381}%
\pgfsetfillcolor{currentfill}%
\pgfsetlinewidth{0.000000pt}%
\definecolor{currentstroke}{rgb}{0.000000,0.000000,0.000000}%
\pgfsetstrokecolor{currentstroke}%
\pgfsetdash{}{0pt}%
\pgfpathmoveto{\pgfqpoint{1.180330in}{1.562060in}}%
\pgfpathlineto{\pgfqpoint{1.180336in}{1.557497in}}%
\pgfpathlineto{\pgfqpoint{1.180343in}{1.552863in}}%
\pgfpathlineto{\pgfqpoint{1.180349in}{1.548158in}}%
\pgfpathlineto{\pgfqpoint{1.180355in}{1.543385in}}%
\pgfpathlineto{\pgfqpoint{1.187564in}{1.543325in}}%
\pgfpathlineto{\pgfqpoint{1.194765in}{1.543158in}}%
\pgfpathlineto{\pgfqpoint{1.201952in}{1.542882in}}%
\pgfpathlineto{\pgfqpoint{1.209118in}{1.542500in}}%
\pgfpathlineto{\pgfqpoint{1.208649in}{1.547287in}}%
\pgfpathlineto{\pgfqpoint{1.208179in}{1.552007in}}%
\pgfpathlineto{\pgfqpoint{1.207708in}{1.556656in}}%
\pgfpathlineto{\pgfqpoint{1.207237in}{1.561233in}}%
\pgfpathlineto{\pgfqpoint{1.200533in}{1.561590in}}%
\pgfpathlineto{\pgfqpoint{1.193810in}{1.561847in}}%
\pgfpathlineto{\pgfqpoint{1.187073in}{1.562004in}}%
\pgfpathlineto{\pgfqpoint{1.180330in}{1.562060in}}%
\pgfpathclose%
\pgfusepath{fill}%
\end{pgfscope}%
\begin{pgfscope}%
\pgfpathrectangle{\pgfqpoint{0.041670in}{0.041670in}}{\pgfqpoint{2.216660in}{2.216660in}}%
\pgfusepath{clip}%
\pgfsetbuttcap%
\pgfsetroundjoin%
\definecolor{currentfill}{rgb}{0.281477,0.755203,0.432552}%
\pgfsetfillcolor{currentfill}%
\pgfsetlinewidth{0.000000pt}%
\definecolor{currentstroke}{rgb}{0.000000,0.000000,0.000000}%
\pgfsetstrokecolor{currentstroke}%
\pgfsetdash{}{0pt}%
\pgfpathmoveto{\pgfqpoint{1.284650in}{1.531370in}}%
\pgfpathlineto{\pgfqpoint{1.286335in}{1.526329in}}%
\pgfpathlineto{\pgfqpoint{1.288019in}{1.521224in}}%
\pgfpathlineto{\pgfqpoint{1.289699in}{1.516057in}}%
\pgfpathlineto{\pgfqpoint{1.291378in}{1.510831in}}%
\pgfpathlineto{\pgfqpoint{1.298187in}{1.509104in}}%
\pgfpathlineto{\pgfqpoint{1.304884in}{1.507276in}}%
\pgfpathlineto{\pgfqpoint{1.311461in}{1.505347in}}%
\pgfpathlineto{\pgfqpoint{1.317914in}{1.503319in}}%
\pgfpathlineto{\pgfqpoint{1.315833in}{1.508665in}}%
\pgfpathlineto{\pgfqpoint{1.313750in}{1.513951in}}%
\pgfpathlineto{\pgfqpoint{1.311664in}{1.519175in}}%
\pgfpathlineto{\pgfqpoint{1.309575in}{1.524335in}}%
\pgfpathlineto{\pgfqpoint{1.303515in}{1.526234in}}%
\pgfpathlineto{\pgfqpoint{1.297337in}{1.528040in}}%
\pgfpathlineto{\pgfqpoint{1.291046in}{1.529753in}}%
\pgfpathlineto{\pgfqpoint{1.284650in}{1.531370in}}%
\pgfpathclose%
\pgfusepath{fill}%
\end{pgfscope}%
\begin{pgfscope}%
\pgfpathrectangle{\pgfqpoint{0.041670in}{0.041670in}}{\pgfqpoint{2.216660in}{2.216660in}}%
\pgfusepath{clip}%
\pgfsetbuttcap%
\pgfsetroundjoin%
\definecolor{currentfill}{rgb}{0.268510,0.009605,0.335427}%
\pgfsetfillcolor{currentfill}%
\pgfsetlinewidth{0.000000pt}%
\definecolor{currentstroke}{rgb}{0.000000,0.000000,0.000000}%
\pgfsetstrokecolor{currentstroke}%
\pgfsetdash{}{0pt}%
\pgfpathmoveto{\pgfqpoint{0.600147in}{0.777386in}}%
\pgfpathlineto{\pgfqpoint{0.596434in}{0.778069in}}%
\pgfpathlineto{\pgfqpoint{0.592712in}{0.779029in}}%
\pgfpathlineto{\pgfqpoint{0.588981in}{0.780273in}}%
\pgfpathlineto{\pgfqpoint{0.585240in}{0.781805in}}%
\pgfpathlineto{\pgfqpoint{0.580094in}{0.791771in}}%
\pgfpathlineto{\pgfqpoint{0.575567in}{0.801804in}}%
\pgfpathlineto{\pgfqpoint{0.571661in}{0.811893in}}%
\pgfpathlineto{\pgfqpoint{0.568377in}{0.822028in}}%
\pgfpathlineto{\pgfqpoint{0.572203in}{0.820253in}}%
\pgfpathlineto{\pgfqpoint{0.576019in}{0.818765in}}%
\pgfpathlineto{\pgfqpoint{0.579826in}{0.817559in}}%
\pgfpathlineto{\pgfqpoint{0.583624in}{0.816631in}}%
\pgfpathlineto{\pgfqpoint{0.586847in}{0.806741in}}%
\pgfpathlineto{\pgfqpoint{0.590675in}{0.796897in}}%
\pgfpathlineto{\pgfqpoint{0.595110in}{0.787108in}}%
\pgfpathlineto{\pgfqpoint{0.600147in}{0.777386in}}%
\pgfpathclose%
\pgfusepath{fill}%
\end{pgfscope}%
\begin{pgfscope}%
\pgfpathrectangle{\pgfqpoint{0.041670in}{0.041670in}}{\pgfqpoint{2.216660in}{2.216660in}}%
\pgfusepath{clip}%
\pgfsetbuttcap%
\pgfsetroundjoin%
\definecolor{currentfill}{rgb}{0.276194,0.190074,0.493001}%
\pgfsetfillcolor{currentfill}%
\pgfsetlinewidth{0.000000pt}%
\definecolor{currentstroke}{rgb}{0.000000,0.000000,0.000000}%
\pgfsetstrokecolor{currentstroke}%
\pgfsetdash{}{0pt}%
\pgfpathmoveto{\pgfqpoint{0.505560in}{0.894023in}}%
\pgfpathlineto{\pgfqpoint{0.501515in}{0.901547in}}%
\pgfpathlineto{\pgfqpoint{0.497453in}{0.909463in}}%
\pgfpathlineto{\pgfqpoint{0.493374in}{0.917780in}}%
\pgfpathlineto{\pgfqpoint{0.489279in}{0.926504in}}%
\pgfpathlineto{\pgfqpoint{0.486400in}{0.937823in}}%
\pgfpathlineto{\pgfqpoint{0.484226in}{0.949165in}}%
\pgfpathlineto{\pgfqpoint{0.482756in}{0.960519in}}%
\pgfpathlineto{\pgfqpoint{0.481988in}{0.971872in}}%
\pgfpathlineto{\pgfqpoint{0.486101in}{0.962936in}}%
\pgfpathlineto{\pgfqpoint{0.490197in}{0.954405in}}%
\pgfpathlineto{\pgfqpoint{0.494276in}{0.946271in}}%
\pgfpathlineto{\pgfqpoint{0.498339in}{0.938529in}}%
\pgfpathlineto{\pgfqpoint{0.499114in}{0.927390in}}%
\pgfpathlineto{\pgfqpoint{0.500575in}{0.916251in}}%
\pgfpathlineto{\pgfqpoint{0.502723in}{0.905125in}}%
\pgfpathlineto{\pgfqpoint{0.505560in}{0.894023in}}%
\pgfpathclose%
\pgfusepath{fill}%
\end{pgfscope}%
\begin{pgfscope}%
\pgfpathrectangle{\pgfqpoint{0.041670in}{0.041670in}}{\pgfqpoint{2.216660in}{2.216660in}}%
\pgfusepath{clip}%
\pgfsetbuttcap%
\pgfsetroundjoin%
\definecolor{currentfill}{rgb}{0.344074,0.780029,0.397381}%
\pgfsetfillcolor{currentfill}%
\pgfsetlinewidth{0.000000pt}%
\definecolor{currentstroke}{rgb}{0.000000,0.000000,0.000000}%
\pgfsetstrokecolor{currentstroke}%
\pgfsetdash{}{0pt}%
\pgfpathmoveto{\pgfqpoint{1.126906in}{1.558900in}}%
\pgfpathlineto{\pgfqpoint{1.125990in}{1.554281in}}%
\pgfpathlineto{\pgfqpoint{1.125074in}{1.549590in}}%
\pgfpathlineto{\pgfqpoint{1.124160in}{1.544828in}}%
\pgfpathlineto{\pgfqpoint{1.123247in}{1.539998in}}%
\pgfpathlineto{\pgfqpoint{1.130268in}{1.540795in}}%
\pgfpathlineto{\pgfqpoint{1.137336in}{1.541485in}}%
\pgfpathlineto{\pgfqpoint{1.144445in}{1.542070in}}%
\pgfpathlineto{\pgfqpoint{1.151587in}{1.542548in}}%
\pgfpathlineto{\pgfqpoint{1.152043in}{1.547334in}}%
\pgfpathlineto{\pgfqpoint{1.152500in}{1.552053in}}%
\pgfpathlineto{\pgfqpoint{1.152958in}{1.556702in}}%
\pgfpathlineto{\pgfqpoint{1.153416in}{1.561278in}}%
\pgfpathlineto{\pgfqpoint{1.146735in}{1.560832in}}%
\pgfpathlineto{\pgfqpoint{1.140085in}{1.560287in}}%
\pgfpathlineto{\pgfqpoint{1.133474in}{1.559643in}}%
\pgfpathlineto{\pgfqpoint{1.126906in}{1.558900in}}%
\pgfpathclose%
\pgfusepath{fill}%
\end{pgfscope}%
\begin{pgfscope}%
\pgfpathrectangle{\pgfqpoint{0.041670in}{0.041670in}}{\pgfqpoint{2.216660in}{2.216660in}}%
\pgfusepath{clip}%
\pgfsetbuttcap%
\pgfsetroundjoin%
\definecolor{currentfill}{rgb}{0.344074,0.780029,0.397381}%
\pgfsetfillcolor{currentfill}%
\pgfsetlinewidth{0.000000pt}%
\definecolor{currentstroke}{rgb}{0.000000,0.000000,0.000000}%
\pgfsetstrokecolor{currentstroke}%
\pgfsetdash{}{0pt}%
\pgfpathmoveto{\pgfqpoint{1.207237in}{1.561233in}}%
\pgfpathlineto{\pgfqpoint{1.207708in}{1.556656in}}%
\pgfpathlineto{\pgfqpoint{1.208179in}{1.552007in}}%
\pgfpathlineto{\pgfqpoint{1.208649in}{1.547287in}}%
\pgfpathlineto{\pgfqpoint{1.209118in}{1.542500in}}%
\pgfpathlineto{\pgfqpoint{1.216257in}{1.542010in}}%
\pgfpathlineto{\pgfqpoint{1.223361in}{1.541414in}}%
\pgfpathlineto{\pgfqpoint{1.230424in}{1.540711in}}%
\pgfpathlineto{\pgfqpoint{1.237439in}{1.539904in}}%
\pgfpathlineto{\pgfqpoint{1.236514in}{1.544735in}}%
\pgfpathlineto{\pgfqpoint{1.235587in}{1.549498in}}%
\pgfpathlineto{\pgfqpoint{1.234659in}{1.554191in}}%
\pgfpathlineto{\pgfqpoint{1.233730in}{1.558812in}}%
\pgfpathlineto{\pgfqpoint{1.227168in}{1.559565in}}%
\pgfpathlineto{\pgfqpoint{1.220561in}{1.560220in}}%
\pgfpathlineto{\pgfqpoint{1.213915in}{1.560777in}}%
\pgfpathlineto{\pgfqpoint{1.207237in}{1.561233in}}%
\pgfpathclose%
\pgfusepath{fill}%
\end{pgfscope}%
\begin{pgfscope}%
\pgfpathrectangle{\pgfqpoint{0.041670in}{0.041670in}}{\pgfqpoint{2.216660in}{2.216660in}}%
\pgfusepath{clip}%
\pgfsetbuttcap%
\pgfsetroundjoin%
\definecolor{currentfill}{rgb}{0.163625,0.471133,0.558148}%
\pgfsetfillcolor{currentfill}%
\pgfsetlinewidth{0.000000pt}%
\definecolor{currentstroke}{rgb}{0.000000,0.000000,0.000000}%
\pgfsetstrokecolor{currentstroke}%
\pgfsetdash{}{0pt}%
\pgfpathmoveto{\pgfqpoint{0.829036in}{1.207151in}}%
\pgfpathlineto{\pgfqpoint{0.825358in}{1.199622in}}%
\pgfpathlineto{\pgfqpoint{0.821682in}{1.192108in}}%
\pgfpathlineto{\pgfqpoint{0.818009in}{1.184613in}}%
\pgfpathlineto{\pgfqpoint{0.814339in}{1.177138in}}%
\pgfpathlineto{\pgfqpoint{0.815483in}{1.182901in}}%
\pgfpathlineto{\pgfqpoint{0.816988in}{1.188639in}}%
\pgfpathlineto{\pgfqpoint{0.818851in}{1.194345in}}%
\pgfpathlineto{\pgfqpoint{0.821069in}{1.200015in}}%
\pgfpathlineto{\pgfqpoint{0.824660in}{1.207236in}}%
\pgfpathlineto{\pgfqpoint{0.828254in}{1.214479in}}%
\pgfpathlineto{\pgfqpoint{0.831851in}{1.221739in}}%
\pgfpathlineto{\pgfqpoint{0.835451in}{1.229016in}}%
\pgfpathlineto{\pgfqpoint{0.833333in}{1.223596in}}%
\pgfpathlineto{\pgfqpoint{0.831556in}{1.218142in}}%
\pgfpathlineto{\pgfqpoint{0.830123in}{1.212659in}}%
\pgfpathlineto{\pgfqpoint{0.829036in}{1.207151in}}%
\pgfpathclose%
\pgfusepath{fill}%
\end{pgfscope}%
\begin{pgfscope}%
\pgfpathrectangle{\pgfqpoint{0.041670in}{0.041670in}}{\pgfqpoint{2.216660in}{2.216660in}}%
\pgfusepath{clip}%
\pgfsetbuttcap%
\pgfsetroundjoin%
\definecolor{currentfill}{rgb}{0.281477,0.755203,0.432552}%
\pgfsetfillcolor{currentfill}%
\pgfsetlinewidth{0.000000pt}%
\definecolor{currentstroke}{rgb}{0.000000,0.000000,0.000000}%
\pgfsetstrokecolor{currentstroke}%
\pgfsetdash{}{0pt}%
\pgfpathmoveto{\pgfqpoint{1.045052in}{1.522571in}}%
\pgfpathlineto{\pgfqpoint{1.042877in}{1.517381in}}%
\pgfpathlineto{\pgfqpoint{1.040705in}{1.512127in}}%
\pgfpathlineto{\pgfqpoint{1.038536in}{1.506812in}}%
\pgfpathlineto{\pgfqpoint{1.036370in}{1.501436in}}%
\pgfpathlineto{\pgfqpoint{1.042706in}{1.503550in}}%
\pgfpathlineto{\pgfqpoint{1.049173in}{1.505566in}}%
\pgfpathlineto{\pgfqpoint{1.055764in}{1.507484in}}%
\pgfpathlineto{\pgfqpoint{1.062474in}{1.509301in}}%
\pgfpathlineto{\pgfqpoint{1.064244in}{1.514552in}}%
\pgfpathlineto{\pgfqpoint{1.066017in}{1.519743in}}%
\pgfpathlineto{\pgfqpoint{1.067792in}{1.524872in}}%
\pgfpathlineto{\pgfqpoint{1.069569in}{1.529937in}}%
\pgfpathlineto{\pgfqpoint{1.063267in}{1.528235in}}%
\pgfpathlineto{\pgfqpoint{1.057076in}{1.526439in}}%
\pgfpathlineto{\pgfqpoint{1.051002in}{1.524550in}}%
\pgfpathlineto{\pgfqpoint{1.045052in}{1.522571in}}%
\pgfpathclose%
\pgfusepath{fill}%
\end{pgfscope}%
\begin{pgfscope}%
\pgfpathrectangle{\pgfqpoint{0.041670in}{0.041670in}}{\pgfqpoint{2.216660in}{2.216660in}}%
\pgfusepath{clip}%
\pgfsetbuttcap%
\pgfsetroundjoin%
\definecolor{currentfill}{rgb}{0.166383,0.690856,0.496502}%
\pgfsetfillcolor{currentfill}%
\pgfsetlinewidth{0.000000pt}%
\definecolor{currentstroke}{rgb}{0.000000,0.000000,0.000000}%
\pgfsetstrokecolor{currentstroke}%
\pgfsetdash{}{0pt}%
\pgfpathmoveto{\pgfqpoint{1.375418in}{1.460606in}}%
\pgfpathlineto{\pgfqpoint{1.378189in}{1.454682in}}%
\pgfpathlineto{\pgfqpoint{1.380957in}{1.448712in}}%
\pgfpathlineto{\pgfqpoint{1.383722in}{1.442697in}}%
\pgfpathlineto{\pgfqpoint{1.386483in}{1.436641in}}%
\pgfpathlineto{\pgfqpoint{1.392160in}{1.433457in}}%
\pgfpathlineto{\pgfqpoint{1.397634in}{1.430187in}}%
\pgfpathlineto{\pgfqpoint{1.402898in}{1.426834in}}%
\pgfpathlineto{\pgfqpoint{1.407948in}{1.423400in}}%
\pgfpathlineto{\pgfqpoint{1.404894in}{1.429646in}}%
\pgfpathlineto{\pgfqpoint{1.401836in}{1.435850in}}%
\pgfpathlineto{\pgfqpoint{1.398775in}{1.442009in}}%
\pgfpathlineto{\pgfqpoint{1.395710in}{1.448122in}}%
\pgfpathlineto{\pgfqpoint{1.390937in}{1.451359in}}%
\pgfpathlineto{\pgfqpoint{1.385960in}{1.454521in}}%
\pgfpathlineto{\pgfqpoint{1.380785in}{1.457604in}}%
\pgfpathlineto{\pgfqpoint{1.375418in}{1.460606in}}%
\pgfpathclose%
\pgfusepath{fill}%
\end{pgfscope}%
\begin{pgfscope}%
\pgfpathrectangle{\pgfqpoint{0.041670in}{0.041670in}}{\pgfqpoint{2.216660in}{2.216660in}}%
\pgfusepath{clip}%
\pgfsetbuttcap%
\pgfsetroundjoin%
\definecolor{currentfill}{rgb}{0.201239,0.383670,0.554294}%
\pgfsetfillcolor{currentfill}%
\pgfsetlinewidth{0.000000pt}%
\definecolor{currentstroke}{rgb}{0.000000,0.000000,0.000000}%
\pgfsetstrokecolor{currentstroke}%
\pgfsetdash{}{0pt}%
\pgfpathmoveto{\pgfqpoint{0.448411in}{1.058770in}}%
\pgfpathlineto{\pgfqpoint{0.444123in}{1.071667in}}%
\pgfpathlineto{\pgfqpoint{0.439812in}{1.085043in}}%
\pgfpathlineto{\pgfqpoint{0.435478in}{1.098904in}}%
\pgfpathlineto{\pgfqpoint{0.431122in}{1.113259in}}%
\pgfpathlineto{\pgfqpoint{0.431135in}{1.125186in}}%
\pgfpathlineto{\pgfqpoint{0.431895in}{1.137086in}}%
\pgfpathlineto{\pgfqpoint{0.433398in}{1.148947in}}%
\pgfpathlineto{\pgfqpoint{0.435637in}{1.160759in}}%
\pgfpathlineto{\pgfqpoint{0.439938in}{1.146227in}}%
\pgfpathlineto{\pgfqpoint{0.444216in}{1.132185in}}%
\pgfpathlineto{\pgfqpoint{0.448473in}{1.118626in}}%
\pgfpathlineto{\pgfqpoint{0.452708in}{1.105542in}}%
\pgfpathlineto{\pgfqpoint{0.450549in}{1.093910in}}%
\pgfpathlineto{\pgfqpoint{0.449109in}{1.082229in}}%
\pgfpathlineto{\pgfqpoint{0.448396in}{1.070512in}}%
\pgfpathlineto{\pgfqpoint{0.448411in}{1.058770in}}%
\pgfpathclose%
\pgfusepath{fill}%
\end{pgfscope}%
\begin{pgfscope}%
\pgfpathrectangle{\pgfqpoint{0.041670in}{0.041670in}}{\pgfqpoint{2.216660in}{2.216660in}}%
\pgfusepath{clip}%
\pgfsetbuttcap%
\pgfsetroundjoin%
\definecolor{currentfill}{rgb}{0.282327,0.094955,0.417331}%
\pgfsetfillcolor{currentfill}%
\pgfsetlinewidth{0.000000pt}%
\definecolor{currentstroke}{rgb}{0.000000,0.000000,0.000000}%
\pgfsetstrokecolor{currentstroke}%
\pgfsetdash{}{0pt}%
\pgfpathmoveto{\pgfqpoint{0.701460in}{0.851606in}}%
\pgfpathlineto{\pgfqpoint{0.697900in}{0.846416in}}%
\pgfpathlineto{\pgfqpoint{0.694338in}{0.841378in}}%
\pgfpathlineto{\pgfqpoint{0.690773in}{0.836495in}}%
\pgfpathlineto{\pgfqpoint{0.687205in}{0.831772in}}%
\pgfpathlineto{\pgfqpoint{0.682825in}{0.839975in}}%
\pgfpathlineto{\pgfqpoint{0.678956in}{0.848237in}}%
\pgfpathlineto{\pgfqpoint{0.675598in}{0.856549in}}%
\pgfpathlineto{\pgfqpoint{0.672755in}{0.864903in}}%
\pgfpathlineto{\pgfqpoint{0.676411in}{0.869366in}}%
\pgfpathlineto{\pgfqpoint{0.680064in}{0.873988in}}%
\pgfpathlineto{\pgfqpoint{0.683715in}{0.878766in}}%
\pgfpathlineto{\pgfqpoint{0.687363in}{0.883695in}}%
\pgfpathlineto{\pgfqpoint{0.690140in}{0.875603in}}%
\pgfpathlineto{\pgfqpoint{0.693417in}{0.867552in}}%
\pgfpathlineto{\pgfqpoint{0.697191in}{0.859550in}}%
\pgfpathlineto{\pgfqpoint{0.701460in}{0.851606in}}%
\pgfpathclose%
\pgfusepath{fill}%
\end{pgfscope}%
\begin{pgfscope}%
\pgfpathrectangle{\pgfqpoint{0.041670in}{0.041670in}}{\pgfqpoint{2.216660in}{2.216660in}}%
\pgfusepath{clip}%
\pgfsetbuttcap%
\pgfsetroundjoin%
\definecolor{currentfill}{rgb}{0.134692,0.658636,0.517649}%
\pgfsetfillcolor{currentfill}%
\pgfsetlinewidth{0.000000pt}%
\definecolor{currentstroke}{rgb}{0.000000,0.000000,0.000000}%
\pgfsetstrokecolor{currentstroke}%
\pgfsetdash{}{0pt}%
\pgfpathmoveto{\pgfqpoint{1.407948in}{1.423400in}}%
\pgfpathlineto{\pgfqpoint{1.410999in}{1.417115in}}%
\pgfpathlineto{\pgfqpoint{1.414046in}{1.410792in}}%
\pgfpathlineto{\pgfqpoint{1.417090in}{1.404435in}}%
\pgfpathlineto{\pgfqpoint{1.420130in}{1.398045in}}%
\pgfpathlineto{\pgfqpoint{1.425225in}{1.394333in}}%
\pgfpathlineto{\pgfqpoint{1.430084in}{1.390542in}}%
\pgfpathlineto{\pgfqpoint{1.434702in}{1.386677in}}%
\pgfpathlineto{\pgfqpoint{1.439074in}{1.382740in}}%
\pgfpathlineto{\pgfqpoint{1.435787in}{1.389339in}}%
\pgfpathlineto{\pgfqpoint{1.432496in}{1.395904in}}%
\pgfpathlineto{\pgfqpoint{1.429201in}{1.402435in}}%
\pgfpathlineto{\pgfqpoint{1.425903in}{1.408929in}}%
\pgfpathlineto{\pgfqpoint{1.421761in}{1.412650in}}%
\pgfpathlineto{\pgfqpoint{1.417385in}{1.416305in}}%
\pgfpathlineto{\pgfqpoint{1.412779in}{1.419890in}}%
\pgfpathlineto{\pgfqpoint{1.407948in}{1.423400in}}%
\pgfpathclose%
\pgfusepath{fill}%
\end{pgfscope}%
\begin{pgfscope}%
\pgfpathrectangle{\pgfqpoint{0.041670in}{0.041670in}}{\pgfqpoint{2.216660in}{2.216660in}}%
\pgfusepath{clip}%
\pgfsetbuttcap%
\pgfsetroundjoin%
\definecolor{currentfill}{rgb}{0.248629,0.278775,0.534556}%
\pgfsetfillcolor{currentfill}%
\pgfsetlinewidth{0.000000pt}%
\definecolor{currentstroke}{rgb}{0.000000,0.000000,0.000000}%
\pgfsetstrokecolor{currentstroke}%
\pgfsetdash{}{0pt}%
\pgfpathmoveto{\pgfqpoint{0.760061in}{1.008210in}}%
\pgfpathlineto{\pgfqpoint{0.756429in}{1.001042in}}%
\pgfpathlineto{\pgfqpoint{0.752797in}{0.993952in}}%
\pgfpathlineto{\pgfqpoint{0.749166in}{0.986944in}}%
\pgfpathlineto{\pgfqpoint{0.745535in}{0.980021in}}%
\pgfpathlineto{\pgfqpoint{0.743469in}{0.987091in}}%
\pgfpathlineto{\pgfqpoint{0.741843in}{0.994184in}}%
\pgfpathlineto{\pgfqpoint{0.740659in}{1.001293in}}%
\pgfpathlineto{\pgfqpoint{0.739917in}{1.008410in}}%
\pgfpathlineto{\pgfqpoint{0.743580in}{1.015070in}}%
\pgfpathlineto{\pgfqpoint{0.747244in}{1.021814in}}%
\pgfpathlineto{\pgfqpoint{0.750908in}{1.028641in}}%
\pgfpathlineto{\pgfqpoint{0.754574in}{1.035545in}}%
\pgfpathlineto{\pgfqpoint{0.755306in}{1.028691in}}%
\pgfpathlineto{\pgfqpoint{0.756465in}{1.021846in}}%
\pgfpathlineto{\pgfqpoint{0.758050in}{1.015017in}}%
\pgfpathlineto{\pgfqpoint{0.760061in}{1.008210in}}%
\pgfpathclose%
\pgfusepath{fill}%
\end{pgfscope}%
\begin{pgfscope}%
\pgfpathrectangle{\pgfqpoint{0.041670in}{0.041670in}}{\pgfqpoint{2.216660in}{2.216660in}}%
\pgfusepath{clip}%
\pgfsetbuttcap%
\pgfsetroundjoin%
\definecolor{currentfill}{rgb}{0.133743,0.548535,0.553541}%
\pgfsetfillcolor{currentfill}%
\pgfsetlinewidth{0.000000pt}%
\definecolor{currentstroke}{rgb}{0.000000,0.000000,0.000000}%
\pgfsetstrokecolor{currentstroke}%
\pgfsetdash{}{0pt}%
\pgfpathmoveto{\pgfqpoint{0.864356in}{1.287437in}}%
\pgfpathlineto{\pgfqpoint{0.860731in}{1.280140in}}%
\pgfpathlineto{\pgfqpoint{0.857111in}{1.272836in}}%
\pgfpathlineto{\pgfqpoint{0.853493in}{1.265527in}}%
\pgfpathlineto{\pgfqpoint{0.849878in}{1.258217in}}%
\pgfpathlineto{\pgfqpoint{0.852221in}{1.263349in}}%
\pgfpathlineto{\pgfqpoint{0.854884in}{1.268439in}}%
\pgfpathlineto{\pgfqpoint{0.857866in}{1.273480in}}%
\pgfpathlineto{\pgfqpoint{0.861162in}{1.278469in}}%
\pgfpathlineto{\pgfqpoint{0.864642in}{1.285537in}}%
\pgfpathlineto{\pgfqpoint{0.868126in}{1.292603in}}%
\pgfpathlineto{\pgfqpoint{0.871614in}{1.299666in}}%
\pgfpathlineto{\pgfqpoint{0.875104in}{1.306721in}}%
\pgfpathlineto{\pgfqpoint{0.871963in}{1.301970in}}%
\pgfpathlineto{\pgfqpoint{0.869122in}{1.297169in}}%
\pgfpathlineto{\pgfqpoint{0.866585in}{1.292324in}}%
\pgfpathlineto{\pgfqpoint{0.864356in}{1.287437in}}%
\pgfpathclose%
\pgfusepath{fill}%
\end{pgfscope}%
\begin{pgfscope}%
\pgfpathrectangle{\pgfqpoint{0.041670in}{0.041670in}}{\pgfqpoint{2.216660in}{2.216660in}}%
\pgfusepath{clip}%
\pgfsetbuttcap%
\pgfsetroundjoin%
\definecolor{currentfill}{rgb}{0.277941,0.056324,0.381191}%
\pgfsetfillcolor{currentfill}%
\pgfsetlinewidth{0.000000pt}%
\definecolor{currentstroke}{rgb}{0.000000,0.000000,0.000000}%
\pgfsetstrokecolor{currentstroke}%
\pgfsetdash{}{0pt}%
\pgfpathmoveto{\pgfqpoint{1.809377in}{0.841362in}}%
\pgfpathlineto{\pgfqpoint{1.813266in}{0.844708in}}%
\pgfpathlineto{\pgfqpoint{1.817168in}{0.848373in}}%
\pgfpathlineto{\pgfqpoint{1.821081in}{0.852363in}}%
\pgfpathlineto{\pgfqpoint{1.825007in}{0.856685in}}%
\pgfpathlineto{\pgfqpoint{1.822185in}{0.846040in}}%
\pgfpathlineto{\pgfqpoint{1.818707in}{0.835432in}}%
\pgfpathlineto{\pgfqpoint{1.814575in}{0.824872in}}%
\pgfpathlineto{\pgfqpoint{1.809789in}{0.814370in}}%
\pgfpathlineto{\pgfqpoint{1.805933in}{0.810282in}}%
\pgfpathlineto{\pgfqpoint{1.802091in}{0.806527in}}%
\pgfpathlineto{\pgfqpoint{1.798260in}{0.803099in}}%
\pgfpathlineto{\pgfqpoint{1.794441in}{0.799991in}}%
\pgfpathlineto{\pgfqpoint{1.799131in}{0.810258in}}%
\pgfpathlineto{\pgfqpoint{1.803185in}{0.820582in}}%
\pgfpathlineto{\pgfqpoint{1.806601in}{0.830954in}}%
\pgfpathlineto{\pgfqpoint{1.809377in}{0.841362in}}%
\pgfpathclose%
\pgfusepath{fill}%
\end{pgfscope}%
\begin{pgfscope}%
\pgfpathrectangle{\pgfqpoint{0.041670in}{0.041670in}}{\pgfqpoint{2.216660in}{2.216660in}}%
\pgfusepath{clip}%
\pgfsetbuttcap%
\pgfsetroundjoin%
\definecolor{currentfill}{rgb}{0.220124,0.725509,0.466226}%
\pgfsetfillcolor{currentfill}%
\pgfsetlinewidth{0.000000pt}%
\definecolor{currentstroke}{rgb}{0.000000,0.000000,0.000000}%
\pgfsetstrokecolor{currentstroke}%
\pgfsetdash{}{0pt}%
\pgfpathmoveto{\pgfqpoint{1.342348in}{1.494262in}}%
\pgfpathlineto{\pgfqpoint{1.344798in}{1.488714in}}%
\pgfpathlineto{\pgfqpoint{1.347244in}{1.483111in}}%
\pgfpathlineto{\pgfqpoint{1.349687in}{1.477455in}}%
\pgfpathlineto{\pgfqpoint{1.352127in}{1.471748in}}%
\pgfpathlineto{\pgfqpoint{1.358211in}{1.469098in}}%
\pgfpathlineto{\pgfqpoint{1.364125in}{1.466356in}}%
\pgfpathlineto{\pgfqpoint{1.369862in}{1.463524in}}%
\pgfpathlineto{\pgfqpoint{1.375418in}{1.460606in}}%
\pgfpathlineto{\pgfqpoint{1.372643in}{1.466481in}}%
\pgfpathlineto{\pgfqpoint{1.369865in}{1.472305in}}%
\pgfpathlineto{\pgfqpoint{1.367083in}{1.478075in}}%
\pgfpathlineto{\pgfqpoint{1.364298in}{1.483790in}}%
\pgfpathlineto{\pgfqpoint{1.359063in}{1.486532in}}%
\pgfpathlineto{\pgfqpoint{1.353656in}{1.489193in}}%
\pgfpathlineto{\pgfqpoint{1.348083in}{1.491771in}}%
\pgfpathlineto{\pgfqpoint{1.342348in}{1.494262in}}%
\pgfpathclose%
\pgfusepath{fill}%
\end{pgfscope}%
\begin{pgfscope}%
\pgfpathrectangle{\pgfqpoint{0.041670in}{0.041670in}}{\pgfqpoint{2.216660in}{2.216660in}}%
\pgfusepath{clip}%
\pgfsetbuttcap%
\pgfsetroundjoin%
\definecolor{currentfill}{rgb}{0.195860,0.395433,0.555276}%
\pgfsetfillcolor{currentfill}%
\pgfsetlinewidth{0.000000pt}%
\definecolor{currentstroke}{rgb}{0.000000,0.000000,0.000000}%
\pgfsetstrokecolor{currentstroke}%
\pgfsetdash{}{0pt}%
\pgfpathmoveto{\pgfqpoint{0.798650in}{1.123277in}}%
\pgfpathlineto{\pgfqpoint{0.794969in}{1.115693in}}%
\pgfpathlineto{\pgfqpoint{0.791289in}{1.108148in}}%
\pgfpathlineto{\pgfqpoint{0.787611in}{1.100645in}}%
\pgfpathlineto{\pgfqpoint{0.783934in}{1.093188in}}%
\pgfpathlineto{\pgfqpoint{0.783623in}{1.099519in}}%
\pgfpathlineto{\pgfqpoint{0.783707in}{1.105845in}}%
\pgfpathlineto{\pgfqpoint{0.784185in}{1.112162in}}%
\pgfpathlineto{\pgfqpoint{0.785057in}{1.118461in}}%
\pgfpathlineto{\pgfqpoint{0.788710in}{1.125658in}}%
\pgfpathlineto{\pgfqpoint{0.792364in}{1.132901in}}%
\pgfpathlineto{\pgfqpoint{0.796021in}{1.140186in}}%
\pgfpathlineto{\pgfqpoint{0.799680in}{1.147510in}}%
\pgfpathlineto{\pgfqpoint{0.798854in}{1.141469in}}%
\pgfpathlineto{\pgfqpoint{0.798406in}{1.135412in}}%
\pgfpathlineto{\pgfqpoint{0.798337in}{1.129346in}}%
\pgfpathlineto{\pgfqpoint{0.798650in}{1.123277in}}%
\pgfpathclose%
\pgfusepath{fill}%
\end{pgfscope}%
\begin{pgfscope}%
\pgfpathrectangle{\pgfqpoint{0.041670in}{0.041670in}}{\pgfqpoint{2.216660in}{2.216660in}}%
\pgfusepath{clip}%
\pgfsetbuttcap%
\pgfsetroundjoin%
\definecolor{currentfill}{rgb}{0.344074,0.780029,0.397381}%
\pgfsetfillcolor{currentfill}%
\pgfsetlinewidth{0.000000pt}%
\definecolor{currentstroke}{rgb}{0.000000,0.000000,0.000000}%
\pgfsetstrokecolor{currentstroke}%
\pgfsetdash{}{0pt}%
\pgfpathmoveto{\pgfqpoint{1.233730in}{1.558812in}}%
\pgfpathlineto{\pgfqpoint{1.234659in}{1.554191in}}%
\pgfpathlineto{\pgfqpoint{1.235587in}{1.549498in}}%
\pgfpathlineto{\pgfqpoint{1.236514in}{1.544735in}}%
\pgfpathlineto{\pgfqpoint{1.237439in}{1.539904in}}%
\pgfpathlineto{\pgfqpoint{1.244400in}{1.538991in}}%
\pgfpathlineto{\pgfqpoint{1.251299in}{1.537975in}}%
\pgfpathlineto{\pgfqpoint{1.258131in}{1.536855in}}%
\pgfpathlineto{\pgfqpoint{1.256872in}{1.541738in}}%
\pgfpathlineto{\pgfqpoint{1.255611in}{1.546552in}}%
\pgfpathlineto{\pgfqpoint{1.254349in}{1.551296in}}%
\pgfpathlineto{\pgfqpoint{1.253084in}{1.555969in}}%
\pgfpathlineto{\pgfqpoint{1.246694in}{1.557013in}}%
\pgfpathlineto{\pgfqpoint{1.240241in}{1.557961in}}%
\pgfpathlineto{\pgfqpoint{1.233730in}{1.558812in}}%
\pgfpathclose%
\pgfusepath{fill}%
\end{pgfscope}%
\begin{pgfscope}%
\pgfpathrectangle{\pgfqpoint{0.041670in}{0.041670in}}{\pgfqpoint{2.216660in}{2.216660in}}%
\pgfusepath{clip}%
\pgfsetbuttcap%
\pgfsetroundjoin%
\definecolor{currentfill}{rgb}{0.260571,0.246922,0.522828}%
\pgfsetfillcolor{currentfill}%
\pgfsetlinewidth{0.000000pt}%
\definecolor{currentstroke}{rgb}{0.000000,0.000000,0.000000}%
\pgfsetstrokecolor{currentstroke}%
\pgfsetdash{}{0pt}%
\pgfpathmoveto{\pgfqpoint{1.878017in}{0.981956in}}%
\pgfpathlineto{\pgfqpoint{1.882142in}{0.991350in}}%
\pgfpathlineto{\pgfqpoint{1.886285in}{1.001161in}}%
\pgfpathlineto{\pgfqpoint{1.890447in}{1.011398in}}%
\pgfpathlineto{\pgfqpoint{1.894628in}{1.022067in}}%
\pgfpathlineto{\pgfqpoint{1.894505in}{1.010518in}}%
\pgfpathlineto{\pgfqpoint{1.893666in}{0.998957in}}%
\pgfpathlineto{\pgfqpoint{1.892108in}{0.987397in}}%
\pgfpathlineto{\pgfqpoint{1.889829in}{0.975850in}}%
\pgfpathlineto{\pgfqpoint{1.885650in}{0.965384in}}%
\pgfpathlineto{\pgfqpoint{1.881490in}{0.955352in}}%
\pgfpathlineto{\pgfqpoint{1.877348in}{0.945748in}}%
\pgfpathlineto{\pgfqpoint{1.873225in}{0.936564in}}%
\pgfpathlineto{\pgfqpoint{1.875477in}{0.947904in}}%
\pgfpathlineto{\pgfqpoint{1.877025in}{0.959257in}}%
\pgfpathlineto{\pgfqpoint{1.877871in}{0.970611in}}%
\pgfpathlineto{\pgfqpoint{1.878017in}{0.981956in}}%
\pgfpathclose%
\pgfusepath{fill}%
\end{pgfscope}%
\begin{pgfscope}%
\pgfpathrectangle{\pgfqpoint{0.041670in}{0.041670in}}{\pgfqpoint{2.216660in}{2.216660in}}%
\pgfusepath{clip}%
\pgfsetbuttcap%
\pgfsetroundjoin%
\definecolor{currentfill}{rgb}{0.166383,0.690856,0.496502}%
\pgfsetfillcolor{currentfill}%
\pgfsetlinewidth{0.000000pt}%
\definecolor{currentstroke}{rgb}{0.000000,0.000000,0.000000}%
\pgfsetstrokecolor{currentstroke}%
\pgfsetdash{}{0pt}%
\pgfpathmoveto{\pgfqpoint{0.960132in}{1.445183in}}%
\pgfpathlineto{\pgfqpoint{0.957008in}{1.439026in}}%
\pgfpathlineto{\pgfqpoint{0.953887in}{1.432822in}}%
\pgfpathlineto{\pgfqpoint{0.950770in}{1.426574in}}%
\pgfpathlineto{\pgfqpoint{0.947657in}{1.420284in}}%
\pgfpathlineto{\pgfqpoint{0.952512in}{1.423786in}}%
\pgfpathlineto{\pgfqpoint{0.957586in}{1.427211in}}%
\pgfpathlineto{\pgfqpoint{0.962874in}{1.430555in}}%
\pgfpathlineto{\pgfqpoint{0.968371in}{1.433815in}}%
\pgfpathlineto{\pgfqpoint{0.971201in}{1.439912in}}%
\pgfpathlineto{\pgfqpoint{0.974034in}{1.445967in}}%
\pgfpathlineto{\pgfqpoint{0.976871in}{1.451978in}}%
\pgfpathlineto{\pgfqpoint{0.979711in}{1.457942in}}%
\pgfpathlineto{\pgfqpoint{0.974515in}{1.454867in}}%
\pgfpathlineto{\pgfqpoint{0.969516in}{1.451714in}}%
\pgfpathlineto{\pgfqpoint{0.964720in}{1.448485in}}%
\pgfpathlineto{\pgfqpoint{0.960132in}{1.445183in}}%
\pgfpathclose%
\pgfusepath{fill}%
\end{pgfscope}%
\begin{pgfscope}%
\pgfpathrectangle{\pgfqpoint{0.041670in}{0.041670in}}{\pgfqpoint{2.216660in}{2.216660in}}%
\pgfusepath{clip}%
\pgfsetbuttcap%
\pgfsetroundjoin%
\definecolor{currentfill}{rgb}{0.280255,0.165693,0.476498}%
\pgfsetfillcolor{currentfill}%
\pgfsetlinewidth{0.000000pt}%
\definecolor{currentstroke}{rgb}{0.000000,0.000000,0.000000}%
\pgfsetstrokecolor{currentstroke}%
\pgfsetdash{}{0pt}%
\pgfpathmoveto{\pgfqpoint{1.645385in}{0.934861in}}%
\pgfpathlineto{\pgfqpoint{1.649031in}{0.928919in}}%
\pgfpathlineto{\pgfqpoint{1.652679in}{0.923098in}}%
\pgfpathlineto{\pgfqpoint{1.656328in}{0.917401in}}%
\pgfpathlineto{\pgfqpoint{1.659978in}{0.911832in}}%
\pgfpathlineto{\pgfqpoint{1.657698in}{0.903974in}}%
\pgfpathlineto{\pgfqpoint{1.654933in}{0.896149in}}%
\pgfpathlineto{\pgfqpoint{1.651685in}{0.888365in}}%
\pgfpathlineto{\pgfqpoint{1.647954in}{0.880629in}}%
\pgfpathlineto{\pgfqpoint{1.644380in}{0.886460in}}%
\pgfpathlineto{\pgfqpoint{1.640807in}{0.892420in}}%
\pgfpathlineto{\pgfqpoint{1.637236in}{0.898504in}}%
\pgfpathlineto{\pgfqpoint{1.633666in}{0.904708in}}%
\pgfpathlineto{\pgfqpoint{1.637298in}{0.912182in}}%
\pgfpathlineto{\pgfqpoint{1.640463in}{0.919705in}}%
\pgfpathlineto{\pgfqpoint{1.643159in}{0.927267in}}%
\pgfpathlineto{\pgfqpoint{1.645385in}{0.934861in}}%
\pgfpathclose%
\pgfusepath{fill}%
\end{pgfscope}%
\begin{pgfscope}%
\pgfpathrectangle{\pgfqpoint{0.041670in}{0.041670in}}{\pgfqpoint{2.216660in}{2.216660in}}%
\pgfusepath{clip}%
\pgfsetbuttcap%
\pgfsetroundjoin%
\definecolor{currentfill}{rgb}{0.344074,0.780029,0.397381}%
\pgfsetfillcolor{currentfill}%
\pgfsetlinewidth{0.000000pt}%
\definecolor{currentstroke}{rgb}{0.000000,0.000000,0.000000}%
\pgfsetstrokecolor{currentstroke}%
\pgfsetdash{}{0pt}%
\pgfpathmoveto{\pgfqpoint{1.101204in}{1.554961in}}%
\pgfpathlineto{\pgfqpoint{1.099842in}{1.550270in}}%
\pgfpathlineto{\pgfqpoint{1.098483in}{1.545508in}}%
\pgfpathlineto{\pgfqpoint{1.097125in}{1.540675in}}%
\pgfpathlineto{\pgfqpoint{1.095769in}{1.535775in}}%
\pgfpathlineto{\pgfqpoint{1.102534in}{1.536985in}}%
\pgfpathlineto{\pgfqpoint{1.109374in}{1.538093in}}%
\pgfpathlineto{\pgfqpoint{1.116280in}{1.539098in}}%
\pgfpathlineto{\pgfqpoint{1.123247in}{1.539998in}}%
\pgfpathlineto{\pgfqpoint{1.124160in}{1.544828in}}%
\pgfpathlineto{\pgfqpoint{1.125074in}{1.549590in}}%
\pgfpathlineto{\pgfqpoint{1.125990in}{1.554281in}}%
\pgfpathlineto{\pgfqpoint{1.126906in}{1.558900in}}%
\pgfpathlineto{\pgfqpoint{1.120389in}{1.558060in}}%
\pgfpathlineto{\pgfqpoint{1.113929in}{1.557123in}}%
\pgfpathlineto{\pgfqpoint{1.107532in}{1.556089in}}%
\pgfpathlineto{\pgfqpoint{1.101204in}{1.554961in}}%
\pgfpathclose%
\pgfusepath{fill}%
\end{pgfscope}%
\begin{pgfscope}%
\pgfpathrectangle{\pgfqpoint{0.041670in}{0.041670in}}{\pgfqpoint{2.216660in}{2.216660in}}%
\pgfusepath{clip}%
\pgfsetbuttcap%
\pgfsetroundjoin%
\definecolor{currentfill}{rgb}{0.120081,0.622161,0.534946}%
\pgfsetfillcolor{currentfill}%
\pgfsetlinewidth{0.000000pt}%
\definecolor{currentstroke}{rgb}{0.000000,0.000000,0.000000}%
\pgfsetstrokecolor{currentstroke}%
\pgfsetdash{}{0pt}%
\pgfpathmoveto{\pgfqpoint{1.439074in}{1.382740in}}%
\pgfpathlineto{\pgfqpoint{1.442358in}{1.376112in}}%
\pgfpathlineto{\pgfqpoint{1.445638in}{1.369455in}}%
\pgfpathlineto{\pgfqpoint{1.448915in}{1.362774in}}%
\pgfpathlineto{\pgfqpoint{1.452188in}{1.356071in}}%
\pgfpathlineto{\pgfqpoint{1.456525in}{1.351847in}}%
\pgfpathlineto{\pgfqpoint{1.460595in}{1.347557in}}%
\pgfpathlineto{\pgfqpoint{1.464393in}{1.343203in}}%
\pgfpathlineto{\pgfqpoint{1.467914in}{1.338789in}}%
\pgfpathlineto{\pgfqpoint{1.464443in}{1.345718in}}%
\pgfpathlineto{\pgfqpoint{1.460969in}{1.352624in}}%
\pgfpathlineto{\pgfqpoint{1.457491in}{1.359505in}}%
\pgfpathlineto{\pgfqpoint{1.454009in}{1.366358in}}%
\pgfpathlineto{\pgfqpoint{1.450667in}{1.370541in}}%
\pgfpathlineto{\pgfqpoint{1.447061in}{1.374669in}}%
\pgfpathlineto{\pgfqpoint{1.443195in}{1.378736in}}%
\pgfpathlineto{\pgfqpoint{1.439074in}{1.382740in}}%
\pgfpathclose%
\pgfusepath{fill}%
\end{pgfscope}%
\begin{pgfscope}%
\pgfpathrectangle{\pgfqpoint{0.041670in}{0.041670in}}{\pgfqpoint{2.216660in}{2.216660in}}%
\pgfusepath{clip}%
\pgfsetbuttcap%
\pgfsetroundjoin%
\definecolor{currentfill}{rgb}{0.231674,0.318106,0.544834}%
\pgfsetfillcolor{currentfill}%
\pgfsetlinewidth{0.000000pt}%
\definecolor{currentstroke}{rgb}{0.000000,0.000000,0.000000}%
\pgfsetstrokecolor{currentstroke}%
\pgfsetdash{}{0pt}%
\pgfpathmoveto{\pgfqpoint{1.590961in}{1.069735in}}%
\pgfpathlineto{\pgfqpoint{1.594630in}{1.062611in}}%
\pgfpathlineto{\pgfqpoint{1.598297in}{1.055551in}}%
\pgfpathlineto{\pgfqpoint{1.601964in}{1.048559in}}%
\pgfpathlineto{\pgfqpoint{1.605629in}{1.041638in}}%
\pgfpathlineto{\pgfqpoint{1.605276in}{1.034783in}}%
\pgfpathlineto{\pgfqpoint{1.604496in}{1.027930in}}%
\pgfpathlineto{\pgfqpoint{1.603290in}{1.021086in}}%
\pgfpathlineto{\pgfqpoint{1.601658in}{1.014259in}}%
\pgfpathlineto{\pgfqpoint{1.598012in}{1.021443in}}%
\pgfpathlineto{\pgfqpoint{1.594366in}{1.028698in}}%
\pgfpathlineto{\pgfqpoint{1.590720in}{1.036021in}}%
\pgfpathlineto{\pgfqpoint{1.587072in}{1.043408in}}%
\pgfpathlineto{\pgfqpoint{1.588661in}{1.049972in}}%
\pgfpathlineto{\pgfqpoint{1.589840in}{1.056553in}}%
\pgfpathlineto{\pgfqpoint{1.590606in}{1.063142in}}%
\pgfpathlineto{\pgfqpoint{1.590961in}{1.069735in}}%
\pgfpathclose%
\pgfusepath{fill}%
\end{pgfscope}%
\begin{pgfscope}%
\pgfpathrectangle{\pgfqpoint{0.041670in}{0.041670in}}{\pgfqpoint{2.216660in}{2.216660in}}%
\pgfusepath{clip}%
\pgfsetbuttcap%
\pgfsetroundjoin%
\definecolor{currentfill}{rgb}{0.272594,0.025563,0.353093}%
\pgfsetfillcolor{currentfill}%
\pgfsetlinewidth{0.000000pt}%
\definecolor{currentstroke}{rgb}{0.000000,0.000000,0.000000}%
\pgfsetstrokecolor{currentstroke}%
\pgfsetdash{}{0pt}%
\pgfpathmoveto{\pgfqpoint{0.585240in}{0.781805in}}%
\pgfpathlineto{\pgfqpoint{0.581489in}{0.783630in}}%
\pgfpathlineto{\pgfqpoint{0.577727in}{0.785755in}}%
\pgfpathlineto{\pgfqpoint{0.573955in}{0.788184in}}%
\pgfpathlineto{\pgfqpoint{0.570172in}{0.790923in}}%
\pgfpathlineto{\pgfqpoint{0.564917in}{0.801129in}}%
\pgfpathlineto{\pgfqpoint{0.560296in}{0.811402in}}%
\pgfpathlineto{\pgfqpoint{0.556313in}{0.821733in}}%
\pgfpathlineto{\pgfqpoint{0.552969in}{0.832109in}}%
\pgfpathlineto{\pgfqpoint{0.556837in}{0.829131in}}%
\pgfpathlineto{\pgfqpoint{0.560694in}{0.826462in}}%
\pgfpathlineto{\pgfqpoint{0.564541in}{0.824096in}}%
\pgfpathlineto{\pgfqpoint{0.568377in}{0.822028in}}%
\pgfpathlineto{\pgfqpoint{0.571661in}{0.811893in}}%
\pgfpathlineto{\pgfqpoint{0.575567in}{0.801804in}}%
\pgfpathlineto{\pgfqpoint{0.580094in}{0.791771in}}%
\pgfpathlineto{\pgfqpoint{0.585240in}{0.781805in}}%
\pgfpathclose%
\pgfusepath{fill}%
\end{pgfscope}%
\begin{pgfscope}%
\pgfpathrectangle{\pgfqpoint{0.041670in}{0.041670in}}{\pgfqpoint{2.216660in}{2.216660in}}%
\pgfusepath{clip}%
\pgfsetbuttcap%
\pgfsetroundjoin%
\definecolor{currentfill}{rgb}{0.134692,0.658636,0.517649}%
\pgfsetfillcolor{currentfill}%
\pgfsetlinewidth{0.000000pt}%
\definecolor{currentstroke}{rgb}{0.000000,0.000000,0.000000}%
\pgfsetstrokecolor{currentstroke}%
\pgfsetdash{}{0pt}%
\pgfpathmoveto{\pgfqpoint{0.930524in}{1.405567in}}%
\pgfpathlineto{\pgfqpoint{0.927178in}{1.399025in}}%
\pgfpathlineto{\pgfqpoint{0.923835in}{1.392446in}}%
\pgfpathlineto{\pgfqpoint{0.920496in}{1.385831in}}%
\pgfpathlineto{\pgfqpoint{0.917160in}{1.379184in}}%
\pgfpathlineto{\pgfqpoint{0.921309in}{1.383181in}}%
\pgfpathlineto{\pgfqpoint{0.925708in}{1.387110in}}%
\pgfpathlineto{\pgfqpoint{0.930353in}{1.390967in}}%
\pgfpathlineto{\pgfqpoint{0.935239in}{1.394749in}}%
\pgfpathlineto{\pgfqpoint{0.938338in}{1.401184in}}%
\pgfpathlineto{\pgfqpoint{0.941441in}{1.407586in}}%
\pgfpathlineto{\pgfqpoint{0.944547in}{1.413953in}}%
\pgfpathlineto{\pgfqpoint{0.947657in}{1.420284in}}%
\pgfpathlineto{\pgfqpoint{0.943026in}{1.416707in}}%
\pgfpathlineto{\pgfqpoint{0.938624in}{1.413060in}}%
\pgfpathlineto{\pgfqpoint{0.934455in}{1.409345in}}%
\pgfpathlineto{\pgfqpoint{0.930524in}{1.405567in}}%
\pgfpathclose%
\pgfusepath{fill}%
\end{pgfscope}%
\begin{pgfscope}%
\pgfpathrectangle{\pgfqpoint{0.041670in}{0.041670in}}{\pgfqpoint{2.216660in}{2.216660in}}%
\pgfusepath{clip}%
\pgfsetbuttcap%
\pgfsetroundjoin%
\definecolor{currentfill}{rgb}{0.283072,0.130895,0.449241}%
\pgfsetfillcolor{currentfill}%
\pgfsetlinewidth{0.000000pt}%
\definecolor{currentstroke}{rgb}{0.000000,0.000000,0.000000}%
\pgfsetstrokecolor{currentstroke}%
\pgfsetdash{}{0pt}%
\pgfpathmoveto{\pgfqpoint{0.715675in}{0.873801in}}%
\pgfpathlineto{\pgfqpoint{0.712124in}{0.868045in}}%
\pgfpathlineto{\pgfqpoint{0.708572in}{0.862424in}}%
\pgfpathlineto{\pgfqpoint{0.705017in}{0.856943in}}%
\pgfpathlineto{\pgfqpoint{0.701460in}{0.851606in}}%
\pgfpathlineto{\pgfqpoint{0.697191in}{0.859550in}}%
\pgfpathlineto{\pgfqpoint{0.693417in}{0.867552in}}%
\pgfpathlineto{\pgfqpoint{0.690140in}{0.875603in}}%
\pgfpathlineto{\pgfqpoint{0.687363in}{0.883695in}}%
\pgfpathlineto{\pgfqpoint{0.691009in}{0.888771in}}%
\pgfpathlineto{\pgfqpoint{0.694652in}{0.893991in}}%
\pgfpathlineto{\pgfqpoint{0.698294in}{0.899350in}}%
\pgfpathlineto{\pgfqpoint{0.701934in}{0.904846in}}%
\pgfpathlineto{\pgfqpoint{0.704645in}{0.897016in}}%
\pgfpathlineto{\pgfqpoint{0.707840in}{0.889227in}}%
\pgfpathlineto{\pgfqpoint{0.711518in}{0.881486in}}%
\pgfpathlineto{\pgfqpoint{0.715675in}{0.873801in}}%
\pgfpathclose%
\pgfusepath{fill}%
\end{pgfscope}%
\begin{pgfscope}%
\pgfpathrectangle{\pgfqpoint{0.041670in}{0.041670in}}{\pgfqpoint{2.216660in}{2.216660in}}%
\pgfusepath{clip}%
\pgfsetbuttcap%
\pgfsetroundjoin%
\definecolor{currentfill}{rgb}{0.220124,0.725509,0.466226}%
\pgfsetfillcolor{currentfill}%
\pgfsetlinewidth{0.000000pt}%
\definecolor{currentstroke}{rgb}{0.000000,0.000000,0.000000}%
\pgfsetstrokecolor{currentstroke}%
\pgfsetdash{}{0pt}%
\pgfpathmoveto{\pgfqpoint{0.991107in}{1.481286in}}%
\pgfpathlineto{\pgfqpoint{0.988253in}{1.475531in}}%
\pgfpathlineto{\pgfqpoint{0.985402in}{1.469721in}}%
\pgfpathlineto{\pgfqpoint{0.982555in}{1.463857in}}%
\pgfpathlineto{\pgfqpoint{0.979711in}{1.457942in}}%
\pgfpathlineto{\pgfqpoint{0.985100in}{1.460935in}}%
\pgfpathlineto{\pgfqpoint{0.990676in}{1.463843in}}%
\pgfpathlineto{\pgfqpoint{0.996433in}{1.466665in}}%
\pgfpathlineto{\pgfqpoint{1.002366in}{1.469397in}}%
\pgfpathlineto{\pgfqpoint{1.004884in}{1.475139in}}%
\pgfpathlineto{\pgfqpoint{1.007405in}{1.480831in}}%
\pgfpathlineto{\pgfqpoint{1.009929in}{1.486469in}}%
\pgfpathlineto{\pgfqpoint{1.012456in}{1.492052in}}%
\pgfpathlineto{\pgfqpoint{1.006864in}{1.489484in}}%
\pgfpathlineto{\pgfqpoint{1.001439in}{1.486832in}}%
\pgfpathlineto{\pgfqpoint{0.996185in}{1.484098in}}%
\pgfpathlineto{\pgfqpoint{0.991107in}{1.481286in}}%
\pgfpathclose%
\pgfusepath{fill}%
\end{pgfscope}%
\begin{pgfscope}%
\pgfpathrectangle{\pgfqpoint{0.041670in}{0.041670in}}{\pgfqpoint{2.216660in}{2.216660in}}%
\pgfusepath{clip}%
\pgfsetbuttcap%
\pgfsetroundjoin%
\definecolor{currentfill}{rgb}{0.147607,0.511733,0.557049}%
\pgfsetfillcolor{currentfill}%
\pgfsetlinewidth{0.000000pt}%
\definecolor{currentstroke}{rgb}{0.000000,0.000000,0.000000}%
\pgfsetstrokecolor{currentstroke}%
\pgfsetdash{}{0pt}%
\pgfpathmoveto{\pgfqpoint{1.507965in}{1.262781in}}%
\pgfpathlineto{\pgfqpoint{1.511552in}{1.255527in}}%
\pgfpathlineto{\pgfqpoint{1.515135in}{1.248277in}}%
\pgfpathlineto{\pgfqpoint{1.518715in}{1.241033in}}%
\pgfpathlineto{\pgfqpoint{1.522293in}{1.233799in}}%
\pgfpathlineto{\pgfqpoint{1.524711in}{1.228415in}}%
\pgfpathlineto{\pgfqpoint{1.526791in}{1.222992in}}%
\pgfpathlineto{\pgfqpoint{1.528530in}{1.217534in}}%
\pgfpathlineto{\pgfqpoint{1.529924in}{1.212048in}}%
\pgfpathlineto{\pgfqpoint{1.526256in}{1.219532in}}%
\pgfpathlineto{\pgfqpoint{1.522584in}{1.227026in}}%
\pgfpathlineto{\pgfqpoint{1.518910in}{1.234527in}}%
\pgfpathlineto{\pgfqpoint{1.515233in}{1.242031in}}%
\pgfpathlineto{\pgfqpoint{1.513908in}{1.247264in}}%
\pgfpathlineto{\pgfqpoint{1.512254in}{1.252471in}}%
\pgfpathlineto{\pgfqpoint{1.510272in}{1.257645in}}%
\pgfpathlineto{\pgfqpoint{1.507965in}{1.262781in}}%
\pgfpathclose%
\pgfusepath{fill}%
\end{pgfscope}%
\begin{pgfscope}%
\pgfpathrectangle{\pgfqpoint{0.041670in}{0.041670in}}{\pgfqpoint{2.216660in}{2.216660in}}%
\pgfusepath{clip}%
\pgfsetbuttcap%
\pgfsetroundjoin%
\definecolor{currentfill}{rgb}{0.281477,0.755203,0.432552}%
\pgfsetfillcolor{currentfill}%
\pgfsetlinewidth{0.000000pt}%
\definecolor{currentstroke}{rgb}{0.000000,0.000000,0.000000}%
\pgfsetstrokecolor{currentstroke}%
\pgfsetdash{}{0pt}%
\pgfpathmoveto{\pgfqpoint{1.309575in}{1.524335in}}%
\pgfpathlineto{\pgfqpoint{1.311664in}{1.519175in}}%
\pgfpathlineto{\pgfqpoint{1.313750in}{1.513951in}}%
\pgfpathlineto{\pgfqpoint{1.315833in}{1.508665in}}%
\pgfpathlineto{\pgfqpoint{1.317914in}{1.503319in}}%
\pgfpathlineto{\pgfqpoint{1.324235in}{1.501195in}}%
\pgfpathlineto{\pgfqpoint{1.330419in}{1.498976in}}%
\pgfpathlineto{\pgfqpoint{1.336458in}{1.496664in}}%
\pgfpathlineto{\pgfqpoint{1.342348in}{1.494262in}}%
\pgfpathlineto{\pgfqpoint{1.339896in}{1.499751in}}%
\pgfpathlineto{\pgfqpoint{1.337440in}{1.505181in}}%
\pgfpathlineto{\pgfqpoint{1.334981in}{1.510549in}}%
\pgfpathlineto{\pgfqpoint{1.332520in}{1.515852in}}%
\pgfpathlineto{\pgfqpoint{1.326989in}{1.518102in}}%
\pgfpathlineto{\pgfqpoint{1.321318in}{1.520267in}}%
\pgfpathlineto{\pgfqpoint{1.315511in}{1.522345in}}%
\pgfpathlineto{\pgfqpoint{1.309575in}{1.524335in}}%
\pgfpathclose%
\pgfusepath{fill}%
\end{pgfscope}%
\begin{pgfscope}%
\pgfpathrectangle{\pgfqpoint{0.041670in}{0.041670in}}{\pgfqpoint{2.216660in}{2.216660in}}%
\pgfusepath{clip}%
\pgfsetbuttcap%
\pgfsetroundjoin%
\definecolor{currentfill}{rgb}{0.172719,0.448791,0.557885}%
\pgfsetfillcolor{currentfill}%
\pgfsetlinewidth{0.000000pt}%
\definecolor{currentstroke}{rgb}{0.000000,0.000000,0.000000}%
\pgfsetstrokecolor{currentstroke}%
\pgfsetdash{}{0pt}%
\pgfpathmoveto{\pgfqpoint{1.921669in}{1.171208in}}%
\pgfpathlineto{\pgfqpoint{1.925972in}{1.186277in}}%
\pgfpathlineto{\pgfqpoint{1.930298in}{1.201853in}}%
\pgfpathlineto{\pgfqpoint{1.934648in}{1.217945in}}%
\pgfpathlineto{\pgfqpoint{1.937610in}{1.206061in}}%
\pgfpathlineto{\pgfqpoint{1.939828in}{1.194117in}}%
\pgfpathlineto{\pgfqpoint{1.941296in}{1.182122in}}%
\pgfpathlineto{\pgfqpoint{1.942008in}{1.170090in}}%
\pgfpathlineto{\pgfqpoint{1.937585in}{1.154165in}}%
\pgfpathlineto{\pgfqpoint{1.933187in}{1.138758in}}%
\pgfpathlineto{\pgfqpoint{1.928813in}{1.123862in}}%
\pgfpathlineto{\pgfqpoint{1.928136in}{1.135765in}}%
\pgfpathlineto{\pgfqpoint{1.926716in}{1.147632in}}%
\pgfpathlineto{\pgfqpoint{1.924558in}{1.159450in}}%
\pgfpathlineto{\pgfqpoint{1.921669in}{1.171208in}}%
\pgfpathclose%
\pgfusepath{fill}%
\end{pgfscope}%
\begin{pgfscope}%
\pgfpathrectangle{\pgfqpoint{0.041670in}{0.041670in}}{\pgfqpoint{2.216660in}{2.216660in}}%
\pgfusepath{clip}%
\pgfsetbuttcap%
\pgfsetroundjoin%
\definecolor{currentfill}{rgb}{0.344074,0.780029,0.397381}%
\pgfsetfillcolor{currentfill}%
\pgfsetlinewidth{0.000000pt}%
\definecolor{currentstroke}{rgb}{0.000000,0.000000,0.000000}%
\pgfsetstrokecolor{currentstroke}%
\pgfsetdash{}{0pt}%
\pgfpathmoveto{\pgfqpoint{1.253084in}{1.555969in}}%
\pgfpathlineto{\pgfqpoint{1.254349in}{1.551296in}}%
\pgfpathlineto{\pgfqpoint{1.255611in}{1.546552in}}%
\pgfpathlineto{\pgfqpoint{1.256872in}{1.541738in}}%
\pgfpathlineto{\pgfqpoint{1.258131in}{1.536855in}}%
\pgfpathlineto{\pgfqpoint{1.264888in}{1.535634in}}%
\pgfpathlineto{\pgfqpoint{1.271565in}{1.534312in}}%
\pgfpathlineto{\pgfqpoint{1.278154in}{1.532890in}}%
\pgfpathlineto{\pgfqpoint{1.284650in}{1.531370in}}%
\pgfpathlineto{\pgfqpoint{1.282963in}{1.536345in}}%
\pgfpathlineto{\pgfqpoint{1.281273in}{1.541252in}}%
\pgfpathlineto{\pgfqpoint{1.279581in}{1.546088in}}%
\pgfpathlineto{\pgfqpoint{1.277886in}{1.550853in}}%
\pgfpathlineto{\pgfqpoint{1.271812in}{1.552270in}}%
\pgfpathlineto{\pgfqpoint{1.265649in}{1.553596in}}%
\pgfpathlineto{\pgfqpoint{1.259405in}{1.554829in}}%
\pgfpathlineto{\pgfqpoint{1.253084in}{1.555969in}}%
\pgfpathclose%
\pgfusepath{fill}%
\end{pgfscope}%
\begin{pgfscope}%
\pgfpathrectangle{\pgfqpoint{0.041670in}{0.041670in}}{\pgfqpoint{2.216660in}{2.216660in}}%
\pgfusepath{clip}%
\pgfsetbuttcap%
\pgfsetroundjoin%
\definecolor{currentfill}{rgb}{0.179019,0.433756,0.557430}%
\pgfsetfillcolor{currentfill}%
\pgfsetlinewidth{0.000000pt}%
\definecolor{currentstroke}{rgb}{0.000000,0.000000,0.000000}%
\pgfsetstrokecolor{currentstroke}%
\pgfsetdash{}{0pt}%
\pgfpathmoveto{\pgfqpoint{1.544572in}{1.182262in}}%
\pgfpathlineto{\pgfqpoint{1.548227in}{1.174869in}}%
\pgfpathlineto{\pgfqpoint{1.551880in}{1.167502in}}%
\pgfpathlineto{\pgfqpoint{1.555530in}{1.160166in}}%
\pgfpathlineto{\pgfqpoint{1.559179in}{1.152862in}}%
\pgfpathlineto{\pgfqpoint{1.560340in}{1.146840in}}%
\pgfpathlineto{\pgfqpoint{1.561125in}{1.140797in}}%
\pgfpathlineto{\pgfqpoint{1.561531in}{1.134739in}}%
\pgfpathlineto{\pgfqpoint{1.561556in}{1.128672in}}%
\pgfpathlineto{\pgfqpoint{1.557873in}{1.136234in}}%
\pgfpathlineto{\pgfqpoint{1.554187in}{1.143829in}}%
\pgfpathlineto{\pgfqpoint{1.550499in}{1.151454in}}%
\pgfpathlineto{\pgfqpoint{1.546808in}{1.159105in}}%
\pgfpathlineto{\pgfqpoint{1.546796in}{1.164912in}}%
\pgfpathlineto{\pgfqpoint{1.546418in}{1.170711in}}%
\pgfpathlineto{\pgfqpoint{1.545676in}{1.176496in}}%
\pgfpathlineto{\pgfqpoint{1.544572in}{1.182262in}}%
\pgfpathclose%
\pgfusepath{fill}%
\end{pgfscope}%
\begin{pgfscope}%
\pgfpathrectangle{\pgfqpoint{0.041670in}{0.041670in}}{\pgfqpoint{2.216660in}{2.216660in}}%
\pgfusepath{clip}%
\pgfsetbuttcap%
\pgfsetroundjoin%
\definecolor{currentfill}{rgb}{0.120081,0.622161,0.534946}%
\pgfsetfillcolor{currentfill}%
\pgfsetlinewidth{0.000000pt}%
\definecolor{currentstroke}{rgb}{0.000000,0.000000,0.000000}%
\pgfsetstrokecolor{currentstroke}%
\pgfsetdash{}{0pt}%
\pgfpathmoveto{\pgfqpoint{0.903153in}{1.362596in}}%
\pgfpathlineto{\pgfqpoint{0.899635in}{1.355691in}}%
\pgfpathlineto{\pgfqpoint{0.896120in}{1.348759in}}%
\pgfpathlineto{\pgfqpoint{0.892609in}{1.341801in}}%
\pgfpathlineto{\pgfqpoint{0.889101in}{1.334820in}}%
\pgfpathlineto{\pgfqpoint{0.892373in}{1.339282in}}%
\pgfpathlineto{\pgfqpoint{0.895925in}{1.343689in}}%
\pgfpathlineto{\pgfqpoint{0.899753in}{1.348037in}}%
\pgfpathlineto{\pgfqpoint{0.903853in}{1.352320in}}%
\pgfpathlineto{\pgfqpoint{0.907174in}{1.359072in}}%
\pgfpathlineto{\pgfqpoint{0.910499in}{1.365802in}}%
\pgfpathlineto{\pgfqpoint{0.913828in}{1.372507in}}%
\pgfpathlineto{\pgfqpoint{0.917160in}{1.379184in}}%
\pgfpathlineto{\pgfqpoint{0.913266in}{1.375124in}}%
\pgfpathlineto{\pgfqpoint{0.909631in}{1.371003in}}%
\pgfpathlineto{\pgfqpoint{0.906259in}{1.366826in}}%
\pgfpathlineto{\pgfqpoint{0.903153in}{1.362596in}}%
\pgfpathclose%
\pgfusepath{fill}%
\end{pgfscope}%
\begin{pgfscope}%
\pgfpathrectangle{\pgfqpoint{0.041670in}{0.041670in}}{\pgfqpoint{2.216660in}{2.216660in}}%
\pgfusepath{clip}%
\pgfsetbuttcap%
\pgfsetroundjoin%
\definecolor{currentfill}{rgb}{0.344074,0.780029,0.397381}%
\pgfsetfillcolor{currentfill}%
\pgfsetlinewidth{0.000000pt}%
\definecolor{currentstroke}{rgb}{0.000000,0.000000,0.000000}%
\pgfsetstrokecolor{currentstroke}%
\pgfsetdash{}{0pt}%
\pgfpathmoveto{\pgfqpoint{1.076702in}{1.549516in}}%
\pgfpathlineto{\pgfqpoint{1.074915in}{1.544728in}}%
\pgfpathlineto{\pgfqpoint{1.073131in}{1.539867in}}%
\pgfpathlineto{\pgfqpoint{1.071349in}{1.534936in}}%
\pgfpathlineto{\pgfqpoint{1.069569in}{1.529937in}}%
\pgfpathlineto{\pgfqpoint{1.075977in}{1.531544in}}%
\pgfpathlineto{\pgfqpoint{1.082483in}{1.533053in}}%
\pgfpathlineto{\pgfqpoint{1.089083in}{1.534464in}}%
\pgfpathlineto{\pgfqpoint{1.095769in}{1.535775in}}%
\pgfpathlineto{\pgfqpoint{1.097125in}{1.540675in}}%
\pgfpathlineto{\pgfqpoint{1.098483in}{1.545508in}}%
\pgfpathlineto{\pgfqpoint{1.099842in}{1.550270in}}%
\pgfpathlineto{\pgfqpoint{1.101204in}{1.554961in}}%
\pgfpathlineto{\pgfqpoint{1.094951in}{1.553738in}}%
\pgfpathlineto{\pgfqpoint{1.088779in}{1.552422in}}%
\pgfpathlineto{\pgfqpoint{1.082694in}{1.551015in}}%
\pgfpathlineto{\pgfqpoint{1.076702in}{1.549516in}}%
\pgfpathclose%
\pgfusepath{fill}%
\end{pgfscope}%
\begin{pgfscope}%
\pgfpathrectangle{\pgfqpoint{0.041670in}{0.041670in}}{\pgfqpoint{2.216660in}{2.216660in}}%
\pgfusepath{clip}%
\pgfsetbuttcap%
\pgfsetroundjoin%
\definecolor{currentfill}{rgb}{0.281477,0.755203,0.432552}%
\pgfsetfillcolor{currentfill}%
\pgfsetlinewidth{0.000000pt}%
\definecolor{currentstroke}{rgb}{0.000000,0.000000,0.000000}%
\pgfsetstrokecolor{currentstroke}%
\pgfsetdash{}{0pt}%
\pgfpathmoveto{\pgfqpoint{1.022597in}{1.513783in}}%
\pgfpathlineto{\pgfqpoint{1.020057in}{1.508444in}}%
\pgfpathlineto{\pgfqpoint{1.017520in}{1.503042in}}%
\pgfpathlineto{\pgfqpoint{1.014987in}{1.497577in}}%
\pgfpathlineto{\pgfqpoint{1.012456in}{1.492052in}}%
\pgfpathlineto{\pgfqpoint{1.018208in}{1.494533in}}%
\pgfpathlineto{\pgfqpoint{1.024115in}{1.496926in}}%
\pgfpathlineto{\pgfqpoint{1.030171in}{1.499227in}}%
\pgfpathlineto{\pgfqpoint{1.036370in}{1.501436in}}%
\pgfpathlineto{\pgfqpoint{1.038536in}{1.506812in}}%
\pgfpathlineto{\pgfqpoint{1.040705in}{1.512127in}}%
\pgfpathlineto{\pgfqpoint{1.042877in}{1.517381in}}%
\pgfpathlineto{\pgfqpoint{1.045052in}{1.522571in}}%
\pgfpathlineto{\pgfqpoint{1.039230in}{1.520503in}}%
\pgfpathlineto{\pgfqpoint{1.033544in}{1.518347in}}%
\pgfpathlineto{\pgfqpoint{1.027997in}{1.516107in}}%
\pgfpathlineto{\pgfqpoint{1.022597in}{1.513783in}}%
\pgfpathclose%
\pgfusepath{fill}%
\end{pgfscope}%
\begin{pgfscope}%
\pgfpathrectangle{\pgfqpoint{0.041670in}{0.041670in}}{\pgfqpoint{2.216660in}{2.216660in}}%
\pgfusepath{clip}%
\pgfsetbuttcap%
\pgfsetroundjoin%
\definecolor{currentfill}{rgb}{0.274128,0.199721,0.498911}%
\pgfsetfillcolor{currentfill}%
\pgfsetlinewidth{0.000000pt}%
\definecolor{currentstroke}{rgb}{0.000000,0.000000,0.000000}%
\pgfsetstrokecolor{currentstroke}%
\pgfsetdash{}{0pt}%
\pgfpathmoveto{\pgfqpoint{1.630806in}{0.959763in}}%
\pgfpathlineto{\pgfqpoint{1.634450in}{0.953375in}}%
\pgfpathlineto{\pgfqpoint{1.638094in}{0.947093in}}%
\pgfpathlineto{\pgfqpoint{1.641739in}{0.940920in}}%
\pgfpathlineto{\pgfqpoint{1.645385in}{0.934861in}}%
\pgfpathlineto{\pgfqpoint{1.643159in}{0.927267in}}%
\pgfpathlineto{\pgfqpoint{1.640463in}{0.919705in}}%
\pgfpathlineto{\pgfqpoint{1.637298in}{0.912182in}}%
\pgfpathlineto{\pgfqpoint{1.633666in}{0.904708in}}%
\pgfpathlineto{\pgfqpoint{1.630097in}{0.911030in}}%
\pgfpathlineto{\pgfqpoint{1.626530in}{0.917465in}}%
\pgfpathlineto{\pgfqpoint{1.622963in}{0.924009in}}%
\pgfpathlineto{\pgfqpoint{1.619397in}{0.930660in}}%
\pgfpathlineto{\pgfqpoint{1.622929in}{0.937873in}}%
\pgfpathlineto{\pgfqpoint{1.626009in}{0.945133in}}%
\pgfpathlineto{\pgfqpoint{1.628635in}{0.952432in}}%
\pgfpathlineto{\pgfqpoint{1.630806in}{0.959763in}}%
\pgfpathclose%
\pgfusepath{fill}%
\end{pgfscope}%
\begin{pgfscope}%
\pgfpathrectangle{\pgfqpoint{0.041670in}{0.041670in}}{\pgfqpoint{2.216660in}{2.216660in}}%
\pgfusepath{clip}%
\pgfsetbuttcap%
\pgfsetroundjoin%
\definecolor{currentfill}{rgb}{0.122606,0.585371,0.546557}%
\pgfsetfillcolor{currentfill}%
\pgfsetlinewidth{0.000000pt}%
\definecolor{currentstroke}{rgb}{0.000000,0.000000,0.000000}%
\pgfsetstrokecolor{currentstroke}%
\pgfsetdash{}{0pt}%
\pgfpathmoveto{\pgfqpoint{1.467914in}{1.338789in}}%
\pgfpathlineto{\pgfqpoint{1.471382in}{1.331840in}}%
\pgfpathlineto{\pgfqpoint{1.474846in}{1.324874in}}%
\pgfpathlineto{\pgfqpoint{1.478306in}{1.317892in}}%
\pgfpathlineto{\pgfqpoint{1.481763in}{1.310898in}}%
\pgfpathlineto{\pgfqpoint{1.485169in}{1.306196in}}%
\pgfpathlineto{\pgfqpoint{1.488278in}{1.301439in}}%
\pgfpathlineto{\pgfqpoint{1.491085in}{1.296633in}}%
\pgfpathlineto{\pgfqpoint{1.493588in}{1.291783in}}%
\pgfpathlineto{\pgfqpoint{1.489985in}{1.299015in}}%
\pgfpathlineto{\pgfqpoint{1.486380in}{1.306236in}}%
\pgfpathlineto{\pgfqpoint{1.482770in}{1.313441in}}%
\pgfpathlineto{\pgfqpoint{1.479158in}{1.320628in}}%
\pgfpathlineto{\pgfqpoint{1.476780in}{1.325236in}}%
\pgfpathlineto{\pgfqpoint{1.474111in}{1.329801in}}%
\pgfpathlineto{\pgfqpoint{1.471155in}{1.334321in}}%
\pgfpathlineto{\pgfqpoint{1.467914in}{1.338789in}}%
\pgfpathclose%
\pgfusepath{fill}%
\end{pgfscope}%
\begin{pgfscope}%
\pgfpathrectangle{\pgfqpoint{0.041670in}{0.041670in}}{\pgfqpoint{2.216660in}{2.216660in}}%
\pgfusepath{clip}%
\pgfsetbuttcap%
\pgfsetroundjoin%
\definecolor{currentfill}{rgb}{0.280255,0.165693,0.476498}%
\pgfsetfillcolor{currentfill}%
\pgfsetlinewidth{0.000000pt}%
\definecolor{currentstroke}{rgb}{0.000000,0.000000,0.000000}%
\pgfsetstrokecolor{currentstroke}%
\pgfsetdash{}{0pt}%
\pgfpathmoveto{\pgfqpoint{0.729862in}{0.898111in}}%
\pgfpathlineto{\pgfqpoint{0.726317in}{0.891849in}}%
\pgfpathlineto{\pgfqpoint{0.722772in}{0.885707in}}%
\pgfpathlineto{\pgfqpoint{0.719224in}{0.879690in}}%
\pgfpathlineto{\pgfqpoint{0.715675in}{0.873801in}}%
\pgfpathlineto{\pgfqpoint{0.711518in}{0.881486in}}%
\pgfpathlineto{\pgfqpoint{0.707840in}{0.889227in}}%
\pgfpathlineto{\pgfqpoint{0.704645in}{0.897016in}}%
\pgfpathlineto{\pgfqpoint{0.701934in}{0.904846in}}%
\pgfpathlineto{\pgfqpoint{0.705573in}{0.910473in}}%
\pgfpathlineto{\pgfqpoint{0.709210in}{0.916229in}}%
\pgfpathlineto{\pgfqpoint{0.712846in}{0.922108in}}%
\pgfpathlineto{\pgfqpoint{0.716480in}{0.928109in}}%
\pgfpathlineto{\pgfqpoint{0.719124in}{0.920543in}}%
\pgfpathlineto{\pgfqpoint{0.722237in}{0.913016in}}%
\pgfpathlineto{\pgfqpoint{0.725817in}{0.905536in}}%
\pgfpathlineto{\pgfqpoint{0.729862in}{0.898111in}}%
\pgfpathclose%
\pgfusepath{fill}%
\end{pgfscope}%
\begin{pgfscope}%
\pgfpathrectangle{\pgfqpoint{0.041670in}{0.041670in}}{\pgfqpoint{2.216660in}{2.216660in}}%
\pgfusepath{clip}%
\pgfsetbuttcap%
\pgfsetroundjoin%
\definecolor{currentfill}{rgb}{0.231674,0.318106,0.544834}%
\pgfsetfillcolor{currentfill}%
\pgfsetlinewidth{0.000000pt}%
\definecolor{currentstroke}{rgb}{0.000000,0.000000,0.000000}%
\pgfsetstrokecolor{currentstroke}%
\pgfsetdash{}{0pt}%
\pgfpathmoveto{\pgfqpoint{0.774596in}{1.037592in}}%
\pgfpathlineto{\pgfqpoint{0.770961in}{1.030146in}}%
\pgfpathlineto{\pgfqpoint{0.767327in}{1.022765in}}%
\pgfpathlineto{\pgfqpoint{0.763694in}{1.015452in}}%
\pgfpathlineto{\pgfqpoint{0.760061in}{1.008210in}}%
\pgfpathlineto{\pgfqpoint{0.758050in}{1.015017in}}%
\pgfpathlineto{\pgfqpoint{0.756465in}{1.021846in}}%
\pgfpathlineto{\pgfqpoint{0.755306in}{1.028691in}}%
\pgfpathlineto{\pgfqpoint{0.754574in}{1.035545in}}%
\pgfpathlineto{\pgfqpoint{0.758240in}{1.042524in}}%
\pgfpathlineto{\pgfqpoint{0.761907in}{1.049574in}}%
\pgfpathlineto{\pgfqpoint{0.765575in}{1.056692in}}%
\pgfpathlineto{\pgfqpoint{0.769244in}{1.063875in}}%
\pgfpathlineto{\pgfqpoint{0.769965in}{1.057284in}}%
\pgfpathlineto{\pgfqpoint{0.771097in}{1.050703in}}%
\pgfpathlineto{\pgfqpoint{0.772641in}{1.044136in}}%
\pgfpathlineto{\pgfqpoint{0.774596in}{1.037592in}}%
\pgfpathclose%
\pgfusepath{fill}%
\end{pgfscope}%
\begin{pgfscope}%
\pgfpathrectangle{\pgfqpoint{0.041670in}{0.041670in}}{\pgfqpoint{2.216660in}{2.216660in}}%
\pgfusepath{clip}%
\pgfsetbuttcap%
\pgfsetroundjoin%
\definecolor{currentfill}{rgb}{0.282327,0.094955,0.417331}%
\pgfsetfillcolor{currentfill}%
\pgfsetlinewidth{0.000000pt}%
\definecolor{currentstroke}{rgb}{0.000000,0.000000,0.000000}%
\pgfsetstrokecolor{currentstroke}%
\pgfsetdash{}{0pt}%
\pgfpathmoveto{\pgfqpoint{1.825007in}{0.856685in}}%
\pgfpathlineto{\pgfqpoint{1.828945in}{0.861343in}}%
\pgfpathlineto{\pgfqpoint{1.832897in}{0.866345in}}%
\pgfpathlineto{\pgfqpoint{1.836862in}{0.871694in}}%
\pgfpathlineto{\pgfqpoint{1.840841in}{0.877399in}}%
\pgfpathlineto{\pgfqpoint{1.837974in}{0.866524in}}%
\pgfpathlineto{\pgfqpoint{1.834434in}{0.855686in}}%
\pgfpathlineto{\pgfqpoint{1.830223in}{0.844895in}}%
\pgfpathlineto{\pgfqpoint{1.825342in}{0.834164in}}%
\pgfpathlineto{\pgfqpoint{1.821433in}{0.828687in}}%
\pgfpathlineto{\pgfqpoint{1.817539in}{0.823566in}}%
\pgfpathlineto{\pgfqpoint{1.813657in}{0.818796in}}%
\pgfpathlineto{\pgfqpoint{1.809789in}{0.814370in}}%
\pgfpathlineto{\pgfqpoint{1.814575in}{0.824872in}}%
\pgfpathlineto{\pgfqpoint{1.818707in}{0.835432in}}%
\pgfpathlineto{\pgfqpoint{1.822185in}{0.846040in}}%
\pgfpathlineto{\pgfqpoint{1.825007in}{0.856685in}}%
\pgfpathclose%
\pgfusepath{fill}%
\end{pgfscope}%
\begin{pgfscope}%
\pgfpathrectangle{\pgfqpoint{0.041670in}{0.041670in}}{\pgfqpoint{2.216660in}{2.216660in}}%
\pgfusepath{clip}%
\pgfsetbuttcap%
\pgfsetroundjoin%
\definecolor{currentfill}{rgb}{0.277941,0.056324,0.381191}%
\pgfsetfillcolor{currentfill}%
\pgfsetlinewidth{0.000000pt}%
\definecolor{currentstroke}{rgb}{0.000000,0.000000,0.000000}%
\pgfsetstrokecolor{currentstroke}%
\pgfsetdash{}{0pt}%
\pgfpathmoveto{\pgfqpoint{0.570172in}{0.790923in}}%
\pgfpathlineto{\pgfqpoint{0.566377in}{0.793978in}}%
\pgfpathlineto{\pgfqpoint{0.562571in}{0.797354in}}%
\pgfpathlineto{\pgfqpoint{0.558752in}{0.801058in}}%
\pgfpathlineto{\pgfqpoint{0.554922in}{0.805094in}}%
\pgfpathlineto{\pgfqpoint{0.549557in}{0.815534in}}%
\pgfpathlineto{\pgfqpoint{0.544844in}{0.826042in}}%
\pgfpathlineto{\pgfqpoint{0.540784in}{0.836609in}}%
\pgfpathlineto{\pgfqpoint{0.537379in}{0.847221in}}%
\pgfpathlineto{\pgfqpoint{0.541295in}{0.842952in}}%
\pgfpathlineto{\pgfqpoint{0.545198in}{0.839014in}}%
\pgfpathlineto{\pgfqpoint{0.549090in}{0.835401in}}%
\pgfpathlineto{\pgfqpoint{0.552969in}{0.832109in}}%
\pgfpathlineto{\pgfqpoint{0.556313in}{0.821733in}}%
\pgfpathlineto{\pgfqpoint{0.560296in}{0.811402in}}%
\pgfpathlineto{\pgfqpoint{0.564917in}{0.801129in}}%
\pgfpathlineto{\pgfqpoint{0.570172in}{0.790923in}}%
\pgfpathclose%
\pgfusepath{fill}%
\end{pgfscope}%
\begin{pgfscope}%
\pgfpathrectangle{\pgfqpoint{0.041670in}{0.041670in}}{\pgfqpoint{2.216660in}{2.216660in}}%
\pgfusepath{clip}%
\pgfsetbuttcap%
\pgfsetroundjoin%
\definecolor{currentfill}{rgb}{0.147607,0.511733,0.557049}%
\pgfsetfillcolor{currentfill}%
\pgfsetlinewidth{0.000000pt}%
\definecolor{currentstroke}{rgb}{0.000000,0.000000,0.000000}%
\pgfsetstrokecolor{currentstroke}%
\pgfsetdash{}{0pt}%
\pgfpathmoveto{\pgfqpoint{0.843778in}{1.237360in}}%
\pgfpathlineto{\pgfqpoint{0.840088in}{1.229799in}}%
\pgfpathlineto{\pgfqpoint{0.836401in}{1.222242in}}%
\pgfpathlineto{\pgfqpoint{0.832717in}{1.214692in}}%
\pgfpathlineto{\pgfqpoint{0.829036in}{1.207151in}}%
\pgfpathlineto{\pgfqpoint{0.830123in}{1.212659in}}%
\pgfpathlineto{\pgfqpoint{0.831556in}{1.218142in}}%
\pgfpathlineto{\pgfqpoint{0.833333in}{1.223596in}}%
\pgfpathlineto{\pgfqpoint{0.835451in}{1.229016in}}%
\pgfpathlineto{\pgfqpoint{0.839053in}{1.236304in}}%
\pgfpathlineto{\pgfqpoint{0.842658in}{1.243603in}}%
\pgfpathlineto{\pgfqpoint{0.846267in}{1.250908in}}%
\pgfpathlineto{\pgfqpoint{0.849878in}{1.258217in}}%
\pgfpathlineto{\pgfqpoint{0.847860in}{1.253047in}}%
\pgfpathlineto{\pgfqpoint{0.846169in}{1.247844in}}%
\pgfpathlineto{\pgfqpoint{0.844808in}{1.242613in}}%
\pgfpathlineto{\pgfqpoint{0.843778in}{1.237360in}}%
\pgfpathclose%
\pgfusepath{fill}%
\end{pgfscope}%
\begin{pgfscope}%
\pgfpathrectangle{\pgfqpoint{0.041670in}{0.041670in}}{\pgfqpoint{2.216660in}{2.216660in}}%
\pgfusepath{clip}%
\pgfsetbuttcap%
\pgfsetroundjoin%
\definecolor{currentfill}{rgb}{0.179019,0.433756,0.557430}%
\pgfsetfillcolor{currentfill}%
\pgfsetlinewidth{0.000000pt}%
\definecolor{currentstroke}{rgb}{0.000000,0.000000,0.000000}%
\pgfsetstrokecolor{currentstroke}%
\pgfsetdash{}{0pt}%
\pgfpathmoveto{\pgfqpoint{0.813398in}{1.153942in}}%
\pgfpathlineto{\pgfqpoint{0.809708in}{1.146232in}}%
\pgfpathlineto{\pgfqpoint{0.806020in}{1.138550in}}%
\pgfpathlineto{\pgfqpoint{0.802334in}{1.130897in}}%
\pgfpathlineto{\pgfqpoint{0.798650in}{1.123277in}}%
\pgfpathlineto{\pgfqpoint{0.798337in}{1.129346in}}%
\pgfpathlineto{\pgfqpoint{0.798406in}{1.135412in}}%
\pgfpathlineto{\pgfqpoint{0.798854in}{1.141469in}}%
\pgfpathlineto{\pgfqpoint{0.799680in}{1.147510in}}%
\pgfpathlineto{\pgfqpoint{0.803341in}{1.154871in}}%
\pgfpathlineto{\pgfqpoint{0.807005in}{1.162264in}}%
\pgfpathlineto{\pgfqpoint{0.810670in}{1.169688in}}%
\pgfpathlineto{\pgfqpoint{0.814339in}{1.177138in}}%
\pgfpathlineto{\pgfqpoint{0.813556in}{1.171355in}}%
\pgfpathlineto{\pgfqpoint{0.813138in}{1.165557in}}%
\pgfpathlineto{\pgfqpoint{0.813085in}{1.159751in}}%
\pgfpathlineto{\pgfqpoint{0.813398in}{1.153942in}}%
\pgfpathclose%
\pgfusepath{fill}%
\end{pgfscope}%
\begin{pgfscope}%
\pgfpathrectangle{\pgfqpoint{0.041670in}{0.041670in}}{\pgfqpoint{2.216660in}{2.216660in}}%
\pgfusepath{clip}%
\pgfsetbuttcap%
\pgfsetroundjoin%
\definecolor{currentfill}{rgb}{0.212395,0.359683,0.551710}%
\pgfsetfillcolor{currentfill}%
\pgfsetlinewidth{0.000000pt}%
\definecolor{currentstroke}{rgb}{0.000000,0.000000,0.000000}%
\pgfsetstrokecolor{currentstroke}%
\pgfsetdash{}{0pt}%
\pgfpathmoveto{\pgfqpoint{1.576272in}{1.098815in}}%
\pgfpathlineto{\pgfqpoint{1.579947in}{1.091464in}}%
\pgfpathlineto{\pgfqpoint{1.583620in}{1.084165in}}%
\pgfpathlineto{\pgfqpoint{1.587291in}{1.076921in}}%
\pgfpathlineto{\pgfqpoint{1.590961in}{1.069735in}}%
\pgfpathlineto{\pgfqpoint{1.590606in}{1.063142in}}%
\pgfpathlineto{\pgfqpoint{1.589840in}{1.056553in}}%
\pgfpathlineto{\pgfqpoint{1.588661in}{1.049972in}}%
\pgfpathlineto{\pgfqpoint{1.587072in}{1.043408in}}%
\pgfpathlineto{\pgfqpoint{1.583423in}{1.050857in}}%
\pgfpathlineto{\pgfqpoint{1.579773in}{1.058363in}}%
\pgfpathlineto{\pgfqpoint{1.576122in}{1.065924in}}%
\pgfpathlineto{\pgfqpoint{1.572469in}{1.073537in}}%
\pgfpathlineto{\pgfqpoint{1.574015in}{1.079839in}}%
\pgfpathlineto{\pgfqpoint{1.575164in}{1.086157in}}%
\pgfpathlineto{\pgfqpoint{1.575916in}{1.092485in}}%
\pgfpathlineto{\pgfqpoint{1.576272in}{1.098815in}}%
\pgfpathclose%
\pgfusepath{fill}%
\end{pgfscope}%
\begin{pgfscope}%
\pgfpathrectangle{\pgfqpoint{0.041670in}{0.041670in}}{\pgfqpoint{2.216660in}{2.216660in}}%
\pgfusepath{clip}%
\pgfsetbuttcap%
\pgfsetroundjoin%
\definecolor{currentfill}{rgb}{0.260571,0.246922,0.522828}%
\pgfsetfillcolor{currentfill}%
\pgfsetlinewidth{0.000000pt}%
\definecolor{currentstroke}{rgb}{0.000000,0.000000,0.000000}%
\pgfsetstrokecolor{currentstroke}%
\pgfsetdash{}{0pt}%
\pgfpathmoveto{\pgfqpoint{0.489279in}{0.926504in}}%
\pgfpathlineto{\pgfqpoint{0.485165in}{0.935641in}}%
\pgfpathlineto{\pgfqpoint{0.481033in}{0.945199in}}%
\pgfpathlineto{\pgfqpoint{0.476883in}{0.955185in}}%
\pgfpathlineto{\pgfqpoint{0.472713in}{0.965605in}}%
\pgfpathlineto{\pgfqpoint{0.469792in}{0.977132in}}%
\pgfpathlineto{\pgfqpoint{0.467593in}{0.988681in}}%
\pgfpathlineto{\pgfqpoint{0.466115in}{1.000242in}}%
\pgfpathlineto{\pgfqpoint{0.465356in}{1.011802in}}%
\pgfpathlineto{\pgfqpoint{0.469542in}{1.001178in}}%
\pgfpathlineto{\pgfqpoint{0.473709in}{0.990986in}}%
\pgfpathlineto{\pgfqpoint{0.477857in}{0.981220in}}%
\pgfpathlineto{\pgfqpoint{0.481988in}{0.971872in}}%
\pgfpathlineto{\pgfqpoint{0.482756in}{0.960519in}}%
\pgfpathlineto{\pgfqpoint{0.484226in}{0.949165in}}%
\pgfpathlineto{\pgfqpoint{0.486400in}{0.937823in}}%
\pgfpathlineto{\pgfqpoint{0.489279in}{0.926504in}}%
\pgfpathclose%
\pgfusepath{fill}%
\end{pgfscope}%
\begin{pgfscope}%
\pgfpathrectangle{\pgfqpoint{0.041670in}{0.041670in}}{\pgfqpoint{2.216660in}{2.216660in}}%
\pgfusepath{clip}%
\pgfsetbuttcap%
\pgfsetroundjoin%
\definecolor{currentfill}{rgb}{0.344074,0.780029,0.397381}%
\pgfsetfillcolor{currentfill}%
\pgfsetlinewidth{0.000000pt}%
\definecolor{currentstroke}{rgb}{0.000000,0.000000,0.000000}%
\pgfsetstrokecolor{currentstroke}%
\pgfsetdash{}{0pt}%
\pgfpathmoveto{\pgfqpoint{1.277886in}{1.550853in}}%
\pgfpathlineto{\pgfqpoint{1.279581in}{1.546088in}}%
\pgfpathlineto{\pgfqpoint{1.281273in}{1.541252in}}%
\pgfpathlineto{\pgfqpoint{1.282963in}{1.536345in}}%
\pgfpathlineto{\pgfqpoint{1.284650in}{1.531370in}}%
\pgfpathlineto{\pgfqpoint{1.291046in}{1.529753in}}%
\pgfpathlineto{\pgfqpoint{1.297337in}{1.528040in}}%
\pgfpathlineto{\pgfqpoint{1.303515in}{1.526234in}}%
\pgfpathlineto{\pgfqpoint{1.309575in}{1.524335in}}%
\pgfpathlineto{\pgfqpoint{1.307484in}{1.529429in}}%
\pgfpathlineto{\pgfqpoint{1.305389in}{1.534454in}}%
\pgfpathlineto{\pgfqpoint{1.303292in}{1.539409in}}%
\pgfpathlineto{\pgfqpoint{1.301193in}{1.544292in}}%
\pgfpathlineto{\pgfqpoint{1.295527in}{1.546063in}}%
\pgfpathlineto{\pgfqpoint{1.289750in}{1.547747in}}%
\pgfpathlineto{\pgfqpoint{1.283868in}{1.549344in}}%
\pgfpathlineto{\pgfqpoint{1.277886in}{1.550853in}}%
\pgfpathclose%
\pgfusepath{fill}%
\end{pgfscope}%
\begin{pgfscope}%
\pgfpathrectangle{\pgfqpoint{0.041670in}{0.041670in}}{\pgfqpoint{2.216660in}{2.216660in}}%
\pgfusepath{clip}%
\pgfsetbuttcap%
\pgfsetroundjoin%
\definecolor{currentfill}{rgb}{0.122606,0.585371,0.546557}%
\pgfsetfillcolor{currentfill}%
\pgfsetlinewidth{0.000000pt}%
\definecolor{currentstroke}{rgb}{0.000000,0.000000,0.000000}%
\pgfsetstrokecolor{currentstroke}%
\pgfsetdash{}{0pt}%
\pgfpathmoveto{\pgfqpoint{0.878885in}{1.316500in}}%
\pgfpathlineto{\pgfqpoint{0.875248in}{1.309259in}}%
\pgfpathlineto{\pgfqpoint{0.871614in}{1.301999in}}%
\pgfpathlineto{\pgfqpoint{0.867983in}{1.294725in}}%
\pgfpathlineto{\pgfqpoint{0.864356in}{1.287437in}}%
\pgfpathlineto{\pgfqpoint{0.866585in}{1.292324in}}%
\pgfpathlineto{\pgfqpoint{0.869122in}{1.297169in}}%
\pgfpathlineto{\pgfqpoint{0.871963in}{1.301970in}}%
\pgfpathlineto{\pgfqpoint{0.875104in}{1.306721in}}%
\pgfpathlineto{\pgfqpoint{0.878598in}{1.313767in}}%
\pgfpathlineto{\pgfqpoint{0.882096in}{1.320800in}}%
\pgfpathlineto{\pgfqpoint{0.885597in}{1.327819in}}%
\pgfpathlineto{\pgfqpoint{0.889101in}{1.334820in}}%
\pgfpathlineto{\pgfqpoint{0.886113in}{1.330306in}}%
\pgfpathlineto{\pgfqpoint{0.883412in}{1.325745in}}%
\pgfpathlineto{\pgfqpoint{0.881002in}{1.321142in}}%
\pgfpathlineto{\pgfqpoint{0.878885in}{1.316500in}}%
\pgfpathclose%
\pgfusepath{fill}%
\end{pgfscope}%
\begin{pgfscope}%
\pgfpathrectangle{\pgfqpoint{0.041670in}{0.041670in}}{\pgfqpoint{2.216660in}{2.216660in}}%
\pgfusepath{clip}%
\pgfsetbuttcap%
\pgfsetroundjoin%
\definecolor{currentfill}{rgb}{0.412913,0.803041,0.357269}%
\pgfsetfillcolor{currentfill}%
\pgfsetlinewidth{0.000000pt}%
\definecolor{currentstroke}{rgb}{0.000000,0.000000,0.000000}%
\pgfsetstrokecolor{currentstroke}%
\pgfsetdash{}{0pt}%
\pgfpathmoveto{\pgfqpoint{1.155256in}{1.578824in}}%
\pgfpathlineto{\pgfqpoint{1.154795in}{1.574555in}}%
\pgfpathlineto{\pgfqpoint{1.154335in}{1.570207in}}%
\pgfpathlineto{\pgfqpoint{1.153875in}{1.565780in}}%
\pgfpathlineto{\pgfqpoint{1.153416in}{1.561278in}}%
\pgfpathlineto{\pgfqpoint{1.160123in}{1.561624in}}%
\pgfpathlineto{\pgfqpoint{1.166848in}{1.561870in}}%
\pgfpathlineto{\pgfqpoint{1.173586in}{1.562015in}}%
\pgfpathlineto{\pgfqpoint{1.180330in}{1.562060in}}%
\pgfpathlineto{\pgfqpoint{1.180323in}{1.566548in}}%
\pgfpathlineto{\pgfqpoint{1.180317in}{1.570960in}}%
\pgfpathlineto{\pgfqpoint{1.180310in}{1.575295in}}%
\pgfpathlineto{\pgfqpoint{1.180304in}{1.579549in}}%
\pgfpathlineto{\pgfqpoint{1.174027in}{1.579507in}}%
\pgfpathlineto{\pgfqpoint{1.167757in}{1.579373in}}%
\pgfpathlineto{\pgfqpoint{1.161497in}{1.579145in}}%
\pgfpathlineto{\pgfqpoint{1.155256in}{1.578824in}}%
\pgfpathclose%
\pgfusepath{fill}%
\end{pgfscope}%
\begin{pgfscope}%
\pgfpathrectangle{\pgfqpoint{0.041670in}{0.041670in}}{\pgfqpoint{2.216660in}{2.216660in}}%
\pgfusepath{clip}%
\pgfsetbuttcap%
\pgfsetroundjoin%
\definecolor{currentfill}{rgb}{0.412913,0.803041,0.357269}%
\pgfsetfillcolor{currentfill}%
\pgfsetlinewidth{0.000000pt}%
\definecolor{currentstroke}{rgb}{0.000000,0.000000,0.000000}%
\pgfsetstrokecolor{currentstroke}%
\pgfsetdash{}{0pt}%
\pgfpathmoveto{\pgfqpoint{1.180304in}{1.579549in}}%
\pgfpathlineto{\pgfqpoint{1.180310in}{1.575295in}}%
\pgfpathlineto{\pgfqpoint{1.180317in}{1.570960in}}%
\pgfpathlineto{\pgfqpoint{1.180323in}{1.566548in}}%
\pgfpathlineto{\pgfqpoint{1.180330in}{1.562060in}}%
\pgfpathlineto{\pgfqpoint{1.187073in}{1.562004in}}%
\pgfpathlineto{\pgfqpoint{1.193810in}{1.561847in}}%
\pgfpathlineto{\pgfqpoint{1.200533in}{1.561590in}}%
\pgfpathlineto{\pgfqpoint{1.207237in}{1.561233in}}%
\pgfpathlineto{\pgfqpoint{1.206765in}{1.565737in}}%
\pgfpathlineto{\pgfqpoint{1.206293in}{1.570164in}}%
\pgfpathlineto{\pgfqpoint{1.205820in}{1.574513in}}%
\pgfpathlineto{\pgfqpoint{1.205346in}{1.578782in}}%
\pgfpathlineto{\pgfqpoint{1.199107in}{1.579113in}}%
\pgfpathlineto{\pgfqpoint{1.192849in}{1.579352in}}%
\pgfpathlineto{\pgfqpoint{1.186580in}{1.579497in}}%
\pgfpathlineto{\pgfqpoint{1.180304in}{1.579549in}}%
\pgfpathclose%
\pgfusepath{fill}%
\end{pgfscope}%
\begin{pgfscope}%
\pgfpathrectangle{\pgfqpoint{0.041670in}{0.041670in}}{\pgfqpoint{2.216660in}{2.216660in}}%
\pgfusepath{clip}%
\pgfsetbuttcap%
\pgfsetroundjoin%
\definecolor{currentfill}{rgb}{0.220124,0.725509,0.466226}%
\pgfsetfillcolor{currentfill}%
\pgfsetlinewidth{0.000000pt}%
\definecolor{currentstroke}{rgb}{0.000000,0.000000,0.000000}%
\pgfsetstrokecolor{currentstroke}%
\pgfsetdash{}{0pt}%
\pgfpathmoveto{\pgfqpoint{1.364298in}{1.483790in}}%
\pgfpathlineto{\pgfqpoint{1.367083in}{1.478075in}}%
\pgfpathlineto{\pgfqpoint{1.369865in}{1.472305in}}%
\pgfpathlineto{\pgfqpoint{1.372643in}{1.466481in}}%
\pgfpathlineto{\pgfqpoint{1.375418in}{1.460606in}}%
\pgfpathlineto{\pgfqpoint{1.380785in}{1.457604in}}%
\pgfpathlineto{\pgfqpoint{1.385960in}{1.454521in}}%
\pgfpathlineto{\pgfqpoint{1.390937in}{1.451359in}}%
\pgfpathlineto{\pgfqpoint{1.395710in}{1.448122in}}%
\pgfpathlineto{\pgfqpoint{1.392641in}{1.454185in}}%
\pgfpathlineto{\pgfqpoint{1.389569in}{1.460197in}}%
\pgfpathlineto{\pgfqpoint{1.386493in}{1.466155in}}%
\pgfpathlineto{\pgfqpoint{1.383414in}{1.472058in}}%
\pgfpathlineto{\pgfqpoint{1.378918in}{1.475100in}}%
\pgfpathlineto{\pgfqpoint{1.374231in}{1.478071in}}%
\pgfpathlineto{\pgfqpoint{1.369356in}{1.480968in}}%
\pgfpathlineto{\pgfqpoint{1.364298in}{1.483790in}}%
\pgfpathclose%
\pgfusepath{fill}%
\end{pgfscope}%
\begin{pgfscope}%
\pgfpathrectangle{\pgfqpoint{0.041670in}{0.041670in}}{\pgfqpoint{2.216660in}{2.216660in}}%
\pgfusepath{clip}%
\pgfsetbuttcap%
\pgfsetroundjoin%
\definecolor{currentfill}{rgb}{0.166383,0.690856,0.496502}%
\pgfsetfillcolor{currentfill}%
\pgfsetlinewidth{0.000000pt}%
\definecolor{currentstroke}{rgb}{0.000000,0.000000,0.000000}%
\pgfsetstrokecolor{currentstroke}%
\pgfsetdash{}{0pt}%
\pgfpathmoveto{\pgfqpoint{1.395710in}{1.448122in}}%
\pgfpathlineto{\pgfqpoint{1.398775in}{1.442009in}}%
\pgfpathlineto{\pgfqpoint{1.401836in}{1.435850in}}%
\pgfpathlineto{\pgfqpoint{1.404894in}{1.429646in}}%
\pgfpathlineto{\pgfqpoint{1.407948in}{1.423400in}}%
\pgfpathlineto{\pgfqpoint{1.412779in}{1.419890in}}%
\pgfpathlineto{\pgfqpoint{1.417385in}{1.416305in}}%
\pgfpathlineto{\pgfqpoint{1.421761in}{1.412650in}}%
\pgfpathlineto{\pgfqpoint{1.425903in}{1.408929in}}%
\pgfpathlineto{\pgfqpoint{1.422601in}{1.415382in}}%
\pgfpathlineto{\pgfqpoint{1.419296in}{1.421793in}}%
\pgfpathlineto{\pgfqpoint{1.415987in}{1.428159in}}%
\pgfpathlineto{\pgfqpoint{1.412674in}{1.434479in}}%
\pgfpathlineto{\pgfqpoint{1.408761in}{1.437987in}}%
\pgfpathlineto{\pgfqpoint{1.404627in}{1.441433in}}%
\pgfpathlineto{\pgfqpoint{1.400275in}{1.444812in}}%
\pgfpathlineto{\pgfqpoint{1.395710in}{1.448122in}}%
\pgfpathclose%
\pgfusepath{fill}%
\end{pgfscope}%
\begin{pgfscope}%
\pgfpathrectangle{\pgfqpoint{0.041670in}{0.041670in}}{\pgfqpoint{2.216660in}{2.216660in}}%
\pgfusepath{clip}%
\pgfsetbuttcap%
\pgfsetroundjoin%
\definecolor{currentfill}{rgb}{0.344074,0.780029,0.397381}%
\pgfsetfillcolor{currentfill}%
\pgfsetlinewidth{0.000000pt}%
\definecolor{currentstroke}{rgb}{0.000000,0.000000,0.000000}%
\pgfsetstrokecolor{currentstroke}%
\pgfsetdash{}{0pt}%
\pgfpathmoveto{\pgfqpoint{1.053778in}{1.542647in}}%
\pgfpathlineto{\pgfqpoint{1.051592in}{1.537735in}}%
\pgfpathlineto{\pgfqpoint{1.049409in}{1.532750in}}%
\pgfpathlineto{\pgfqpoint{1.047229in}{1.527694in}}%
\pgfpathlineto{\pgfqpoint{1.045052in}{1.522571in}}%
\pgfpathlineto{\pgfqpoint{1.051002in}{1.524550in}}%
\pgfpathlineto{\pgfqpoint{1.057076in}{1.526439in}}%
\pgfpathlineto{\pgfqpoint{1.063267in}{1.528235in}}%
\pgfpathlineto{\pgfqpoint{1.069569in}{1.529937in}}%
\pgfpathlineto{\pgfqpoint{1.071349in}{1.534936in}}%
\pgfpathlineto{\pgfqpoint{1.073131in}{1.539867in}}%
\pgfpathlineto{\pgfqpoint{1.074915in}{1.544728in}}%
\pgfpathlineto{\pgfqpoint{1.076702in}{1.549516in}}%
\pgfpathlineto{\pgfqpoint{1.070808in}{1.547929in}}%
\pgfpathlineto{\pgfqpoint{1.065020in}{1.546254in}}%
\pgfpathlineto{\pgfqpoint{1.059341in}{1.544493in}}%
\pgfpathlineto{\pgfqpoint{1.053778in}{1.542647in}}%
\pgfpathclose%
\pgfusepath{fill}%
\end{pgfscope}%
\begin{pgfscope}%
\pgfpathrectangle{\pgfqpoint{0.041670in}{0.041670in}}{\pgfqpoint{2.216660in}{2.216660in}}%
\pgfusepath{clip}%
\pgfsetbuttcap%
\pgfsetroundjoin%
\definecolor{currentfill}{rgb}{0.274128,0.199721,0.498911}%
\pgfsetfillcolor{currentfill}%
\pgfsetlinewidth{0.000000pt}%
\definecolor{currentstroke}{rgb}{0.000000,0.000000,0.000000}%
\pgfsetstrokecolor{currentstroke}%
\pgfsetdash{}{0pt}%
\pgfpathmoveto{\pgfqpoint{0.744030in}{0.924293in}}%
\pgfpathlineto{\pgfqpoint{0.740490in}{0.917585in}}%
\pgfpathlineto{\pgfqpoint{0.736948in}{0.910983in}}%
\pgfpathlineto{\pgfqpoint{0.733406in}{0.904490in}}%
\pgfpathlineto{\pgfqpoint{0.729862in}{0.898111in}}%
\pgfpathlineto{\pgfqpoint{0.725817in}{0.905536in}}%
\pgfpathlineto{\pgfqpoint{0.722237in}{0.913016in}}%
\pgfpathlineto{\pgfqpoint{0.719124in}{0.920543in}}%
\pgfpathlineto{\pgfqpoint{0.716480in}{0.928109in}}%
\pgfpathlineto{\pgfqpoint{0.720114in}{0.934227in}}%
\pgfpathlineto{\pgfqpoint{0.723747in}{0.940458in}}%
\pgfpathlineto{\pgfqpoint{0.727379in}{0.946798in}}%
\pgfpathlineto{\pgfqpoint{0.731011in}{0.953245in}}%
\pgfpathlineto{\pgfqpoint{0.733587in}{0.945942in}}%
\pgfpathlineto{\pgfqpoint{0.736616in}{0.938677in}}%
\pgfpathlineto{\pgfqpoint{0.740098in}{0.931459in}}%
\pgfpathlineto{\pgfqpoint{0.744030in}{0.924293in}}%
\pgfpathclose%
\pgfusepath{fill}%
\end{pgfscope}%
\begin{pgfscope}%
\pgfpathrectangle{\pgfqpoint{0.041670in}{0.041670in}}{\pgfqpoint{2.216660in}{2.216660in}}%
\pgfusepath{clip}%
\pgfsetbuttcap%
\pgfsetroundjoin%
\definecolor{currentfill}{rgb}{0.412913,0.803041,0.357269}%
\pgfsetfillcolor{currentfill}%
\pgfsetlinewidth{0.000000pt}%
\definecolor{currentstroke}{rgb}{0.000000,0.000000,0.000000}%
\pgfsetstrokecolor{currentstroke}%
\pgfsetdash{}{0pt}%
\pgfpathmoveto{\pgfqpoint{1.130585in}{1.576617in}}%
\pgfpathlineto{\pgfqpoint{1.129664in}{1.572306in}}%
\pgfpathlineto{\pgfqpoint{1.128743in}{1.567915in}}%
\pgfpathlineto{\pgfqpoint{1.127824in}{1.563445in}}%
\pgfpathlineto{\pgfqpoint{1.126906in}{1.558900in}}%
\pgfpathlineto{\pgfqpoint{1.133474in}{1.559643in}}%
\pgfpathlineto{\pgfqpoint{1.140085in}{1.560287in}}%
\pgfpathlineto{\pgfqpoint{1.146735in}{1.560832in}}%
\pgfpathlineto{\pgfqpoint{1.153416in}{1.561278in}}%
\pgfpathlineto{\pgfqpoint{1.153875in}{1.565780in}}%
\pgfpathlineto{\pgfqpoint{1.154335in}{1.570207in}}%
\pgfpathlineto{\pgfqpoint{1.154795in}{1.574555in}}%
\pgfpathlineto{\pgfqpoint{1.155256in}{1.578824in}}%
\pgfpathlineto{\pgfqpoint{1.149038in}{1.578410in}}%
\pgfpathlineto{\pgfqpoint{1.142850in}{1.577904in}}%
\pgfpathlineto{\pgfqpoint{1.136697in}{1.577306in}}%
\pgfpathlineto{\pgfqpoint{1.130585in}{1.576617in}}%
\pgfpathclose%
\pgfusepath{fill}%
\end{pgfscope}%
\begin{pgfscope}%
\pgfpathrectangle{\pgfqpoint{0.041670in}{0.041670in}}{\pgfqpoint{2.216660in}{2.216660in}}%
\pgfusepath{clip}%
\pgfsetbuttcap%
\pgfsetroundjoin%
\definecolor{currentfill}{rgb}{0.412913,0.803041,0.357269}%
\pgfsetfillcolor{currentfill}%
\pgfsetlinewidth{0.000000pt}%
\definecolor{currentstroke}{rgb}{0.000000,0.000000,0.000000}%
\pgfsetstrokecolor{currentstroke}%
\pgfsetdash{}{0pt}%
\pgfpathmoveto{\pgfqpoint{1.205346in}{1.578782in}}%
\pgfpathlineto{\pgfqpoint{1.205820in}{1.574513in}}%
\pgfpathlineto{\pgfqpoint{1.206293in}{1.570164in}}%
\pgfpathlineto{\pgfqpoint{1.206765in}{1.565737in}}%
\pgfpathlineto{\pgfqpoint{1.207237in}{1.561233in}}%
\pgfpathlineto{\pgfqpoint{1.213915in}{1.560777in}}%
\pgfpathlineto{\pgfqpoint{1.220561in}{1.560220in}}%
\pgfpathlineto{\pgfqpoint{1.227168in}{1.559565in}}%
\pgfpathlineto{\pgfqpoint{1.233730in}{1.558812in}}%
\pgfpathlineto{\pgfqpoint{1.232800in}{1.563358in}}%
\pgfpathlineto{\pgfqpoint{1.231868in}{1.567829in}}%
\pgfpathlineto{\pgfqpoint{1.230935in}{1.572222in}}%
\pgfpathlineto{\pgfqpoint{1.230001in}{1.576534in}}%
\pgfpathlineto{\pgfqpoint{1.223894in}{1.577234in}}%
\pgfpathlineto{\pgfqpoint{1.217745in}{1.577842in}}%
\pgfpathlineto{\pgfqpoint{1.211561in}{1.578358in}}%
\pgfpathlineto{\pgfqpoint{1.205346in}{1.578782in}}%
\pgfpathclose%
\pgfusepath{fill}%
\end{pgfscope}%
\begin{pgfscope}%
\pgfpathrectangle{\pgfqpoint{0.041670in}{0.041670in}}{\pgfqpoint{2.216660in}{2.216660in}}%
\pgfusepath{clip}%
\pgfsetbuttcap%
\pgfsetroundjoin%
\definecolor{currentfill}{rgb}{0.263663,0.237631,0.518762}%
\pgfsetfillcolor{currentfill}%
\pgfsetlinewidth{0.000000pt}%
\definecolor{currentstroke}{rgb}{0.000000,0.000000,0.000000}%
\pgfsetstrokecolor{currentstroke}%
\pgfsetdash{}{0pt}%
\pgfpathmoveto{\pgfqpoint{1.616233in}{0.986304in}}%
\pgfpathlineto{\pgfqpoint{1.619876in}{0.979528in}}%
\pgfpathlineto{\pgfqpoint{1.623520in}{0.972843in}}%
\pgfpathlineto{\pgfqpoint{1.627163in}{0.966253in}}%
\pgfpathlineto{\pgfqpoint{1.630806in}{0.959763in}}%
\pgfpathlineto{\pgfqpoint{1.628635in}{0.952432in}}%
\pgfpathlineto{\pgfqpoint{1.626009in}{0.945133in}}%
\pgfpathlineto{\pgfqpoint{1.622929in}{0.937873in}}%
\pgfpathlineto{\pgfqpoint{1.619397in}{0.930660in}}%
\pgfpathlineto{\pgfqpoint{1.615831in}{0.937413in}}%
\pgfpathlineto{\pgfqpoint{1.612266in}{0.944264in}}%
\pgfpathlineto{\pgfqpoint{1.608701in}{0.951211in}}%
\pgfpathlineto{\pgfqpoint{1.605136in}{0.958250in}}%
\pgfpathlineto{\pgfqpoint{1.608568in}{0.965203in}}%
\pgfpathlineto{\pgfqpoint{1.611562in}{0.972201in}}%
\pgfpathlineto{\pgfqpoint{1.614118in}{0.979237in}}%
\pgfpathlineto{\pgfqpoint{1.616233in}{0.986304in}}%
\pgfpathclose%
\pgfusepath{fill}%
\end{pgfscope}%
\begin{pgfscope}%
\pgfpathrectangle{\pgfqpoint{0.041670in}{0.041670in}}{\pgfqpoint{2.216660in}{2.216660in}}%
\pgfusepath{clip}%
\pgfsetbuttcap%
\pgfsetroundjoin%
\definecolor{currentfill}{rgb}{0.281477,0.755203,0.432552}%
\pgfsetfillcolor{currentfill}%
\pgfsetlinewidth{0.000000pt}%
\definecolor{currentstroke}{rgb}{0.000000,0.000000,0.000000}%
\pgfsetstrokecolor{currentstroke}%
\pgfsetdash{}{0pt}%
\pgfpathmoveto{\pgfqpoint{1.332520in}{1.515852in}}%
\pgfpathlineto{\pgfqpoint{1.334981in}{1.510549in}}%
\pgfpathlineto{\pgfqpoint{1.337440in}{1.505181in}}%
\pgfpathlineto{\pgfqpoint{1.339896in}{1.499751in}}%
\pgfpathlineto{\pgfqpoint{1.342348in}{1.494262in}}%
\pgfpathlineto{\pgfqpoint{1.348083in}{1.491771in}}%
\pgfpathlineto{\pgfqpoint{1.353656in}{1.489193in}}%
\pgfpathlineto{\pgfqpoint{1.359063in}{1.486532in}}%
\pgfpathlineto{\pgfqpoint{1.364298in}{1.483790in}}%
\pgfpathlineto{\pgfqpoint{1.361510in}{1.489446in}}%
\pgfpathlineto{\pgfqpoint{1.358718in}{1.495043in}}%
\pgfpathlineto{\pgfqpoint{1.355923in}{1.500577in}}%
\pgfpathlineto{\pgfqpoint{1.353124in}{1.506047in}}%
\pgfpathlineto{\pgfqpoint{1.348211in}{1.508615in}}%
\pgfpathlineto{\pgfqpoint{1.343136in}{1.511107in}}%
\pgfpathlineto{\pgfqpoint{1.337904in}{1.513520in}}%
\pgfpathlineto{\pgfqpoint{1.332520in}{1.515852in}}%
\pgfpathclose%
\pgfusepath{fill}%
\end{pgfscope}%
\begin{pgfscope}%
\pgfpathrectangle{\pgfqpoint{0.041670in}{0.041670in}}{\pgfqpoint{2.216660in}{2.216660in}}%
\pgfusepath{clip}%
\pgfsetbuttcap%
\pgfsetroundjoin%
\definecolor{currentfill}{rgb}{0.134692,0.658636,0.517649}%
\pgfsetfillcolor{currentfill}%
\pgfsetlinewidth{0.000000pt}%
\definecolor{currentstroke}{rgb}{0.000000,0.000000,0.000000}%
\pgfsetstrokecolor{currentstroke}%
\pgfsetdash{}{0pt}%
\pgfpathmoveto{\pgfqpoint{1.425903in}{1.408929in}}%
\pgfpathlineto{\pgfqpoint{1.429201in}{1.402435in}}%
\pgfpathlineto{\pgfqpoint{1.432496in}{1.395904in}}%
\pgfpathlineto{\pgfqpoint{1.435787in}{1.389339in}}%
\pgfpathlineto{\pgfqpoint{1.439074in}{1.382740in}}%
\pgfpathlineto{\pgfqpoint{1.443195in}{1.378736in}}%
\pgfpathlineto{\pgfqpoint{1.447061in}{1.374669in}}%
\pgfpathlineto{\pgfqpoint{1.450667in}{1.370541in}}%
\pgfpathlineto{\pgfqpoint{1.454009in}{1.366358in}}%
\pgfpathlineto{\pgfqpoint{1.450524in}{1.373181in}}%
\pgfpathlineto{\pgfqpoint{1.447036in}{1.379971in}}%
\pgfpathlineto{\pgfqpoint{1.443543in}{1.386725in}}%
\pgfpathlineto{\pgfqpoint{1.440047in}{1.393442in}}%
\pgfpathlineto{\pgfqpoint{1.436883in}{1.397396in}}%
\pgfpathlineto{\pgfqpoint{1.433468in}{1.401298in}}%
\pgfpathlineto{\pgfqpoint{1.429807in}{1.405143in}}%
\pgfpathlineto{\pgfqpoint{1.425903in}{1.408929in}}%
\pgfpathclose%
\pgfusepath{fill}%
\end{pgfscope}%
\begin{pgfscope}%
\pgfpathrectangle{\pgfqpoint{0.041670in}{0.041670in}}{\pgfqpoint{2.216660in}{2.216660in}}%
\pgfusepath{clip}%
\pgfsetbuttcap%
\pgfsetroundjoin%
\definecolor{currentfill}{rgb}{0.220124,0.725509,0.466226}%
\pgfsetfillcolor{currentfill}%
\pgfsetlinewidth{0.000000pt}%
\definecolor{currentstroke}{rgb}{0.000000,0.000000,0.000000}%
\pgfsetstrokecolor{currentstroke}%
\pgfsetdash{}{0pt}%
\pgfpathmoveto{\pgfqpoint{0.972664in}{1.469297in}}%
\pgfpathlineto{\pgfqpoint{0.969526in}{1.463350in}}%
\pgfpathlineto{\pgfqpoint{0.966391in}{1.457347in}}%
\pgfpathlineto{\pgfqpoint{0.963259in}{1.451291in}}%
\pgfpathlineto{\pgfqpoint{0.960132in}{1.445183in}}%
\pgfpathlineto{\pgfqpoint{0.964720in}{1.448485in}}%
\pgfpathlineto{\pgfqpoint{0.969516in}{1.451714in}}%
\pgfpathlineto{\pgfqpoint{0.974515in}{1.454867in}}%
\pgfpathlineto{\pgfqpoint{0.979711in}{1.457942in}}%
\pgfpathlineto{\pgfqpoint{0.982555in}{1.463857in}}%
\pgfpathlineto{\pgfqpoint{0.985402in}{1.469721in}}%
\pgfpathlineto{\pgfqpoint{0.988253in}{1.475531in}}%
\pgfpathlineto{\pgfqpoint{0.991107in}{1.481286in}}%
\pgfpathlineto{\pgfqpoint{0.986212in}{1.478396in}}%
\pgfpathlineto{\pgfqpoint{0.981503in}{1.475433in}}%
\pgfpathlineto{\pgfqpoint{0.976985in}{1.472399in}}%
\pgfpathlineto{\pgfqpoint{0.972664in}{1.469297in}}%
\pgfpathclose%
\pgfusepath{fill}%
\end{pgfscope}%
\begin{pgfscope}%
\pgfpathrectangle{\pgfqpoint{0.041670in}{0.041670in}}{\pgfqpoint{2.216660in}{2.216660in}}%
\pgfusepath{clip}%
\pgfsetbuttcap%
\pgfsetroundjoin%
\definecolor{currentfill}{rgb}{0.172719,0.448791,0.557885}%
\pgfsetfillcolor{currentfill}%
\pgfsetlinewidth{0.000000pt}%
\definecolor{currentstroke}{rgb}{0.000000,0.000000,0.000000}%
\pgfsetstrokecolor{currentstroke}%
\pgfsetdash{}{0pt}%
\pgfpathmoveto{\pgfqpoint{0.431122in}{1.113259in}}%
\pgfpathlineto{\pgfqpoint{0.426741in}{1.128117in}}%
\pgfpathlineto{\pgfqpoint{0.422336in}{1.143485in}}%
\pgfpathlineto{\pgfqpoint{0.417907in}{1.159372in}}%
\pgfpathlineto{\pgfqpoint{0.417943in}{1.171428in}}%
\pgfpathlineto{\pgfqpoint{0.418740in}{1.183457in}}%
\pgfpathlineto{\pgfqpoint{0.420291in}{1.195446in}}%
\pgfpathlineto{\pgfqpoint{0.422592in}{1.207385in}}%
\pgfpathlineto{\pgfqpoint{0.426964in}{1.191330in}}%
\pgfpathlineto{\pgfqpoint{0.431312in}{1.175791in}}%
\pgfpathlineto{\pgfqpoint{0.435637in}{1.160759in}}%
\pgfpathlineto{\pgfqpoint{0.433398in}{1.148947in}}%
\pgfpathlineto{\pgfqpoint{0.431895in}{1.137086in}}%
\pgfpathlineto{\pgfqpoint{0.431135in}{1.125186in}}%
\pgfpathlineto{\pgfqpoint{0.431122in}{1.113259in}}%
\pgfpathclose%
\pgfusepath{fill}%
\end{pgfscope}%
\begin{pgfscope}%
\pgfpathrectangle{\pgfqpoint{0.041670in}{0.041670in}}{\pgfqpoint{2.216660in}{2.216660in}}%
\pgfusepath{clip}%
\pgfsetbuttcap%
\pgfsetroundjoin%
\definecolor{currentfill}{rgb}{0.166383,0.690856,0.496502}%
\pgfsetfillcolor{currentfill}%
\pgfsetlinewidth{0.000000pt}%
\definecolor{currentstroke}{rgb}{0.000000,0.000000,0.000000}%
\pgfsetstrokecolor{currentstroke}%
\pgfsetdash{}{0pt}%
\pgfpathmoveto{\pgfqpoint{0.943947in}{1.431310in}}%
\pgfpathlineto{\pgfqpoint{0.940586in}{1.424943in}}%
\pgfpathlineto{\pgfqpoint{0.937229in}{1.418528in}}%
\pgfpathlineto{\pgfqpoint{0.933875in}{1.412069in}}%
\pgfpathlineto{\pgfqpoint{0.930524in}{1.405567in}}%
\pgfpathlineto{\pgfqpoint{0.934455in}{1.409345in}}%
\pgfpathlineto{\pgfqpoint{0.938624in}{1.413060in}}%
\pgfpathlineto{\pgfqpoint{0.943026in}{1.416707in}}%
\pgfpathlineto{\pgfqpoint{0.947657in}{1.420284in}}%
\pgfpathlineto{\pgfqpoint{0.950770in}{1.426574in}}%
\pgfpathlineto{\pgfqpoint{0.953887in}{1.432822in}}%
\pgfpathlineto{\pgfqpoint{0.957008in}{1.439026in}}%
\pgfpathlineto{\pgfqpoint{0.960132in}{1.445183in}}%
\pgfpathlineto{\pgfqpoint{0.955756in}{1.441811in}}%
\pgfpathlineto{\pgfqpoint{0.951597in}{1.438373in}}%
\pgfpathlineto{\pgfqpoint{0.947659in}{1.434872in}}%
\pgfpathlineto{\pgfqpoint{0.943947in}{1.431310in}}%
\pgfpathclose%
\pgfusepath{fill}%
\end{pgfscope}%
\begin{pgfscope}%
\pgfpathrectangle{\pgfqpoint{0.041670in}{0.041670in}}{\pgfqpoint{2.216660in}{2.216660in}}%
\pgfusepath{clip}%
\pgfsetbuttcap%
\pgfsetroundjoin%
\definecolor{currentfill}{rgb}{0.163625,0.471133,0.558148}%
\pgfsetfillcolor{currentfill}%
\pgfsetlinewidth{0.000000pt}%
\definecolor{currentstroke}{rgb}{0.000000,0.000000,0.000000}%
\pgfsetstrokecolor{currentstroke}%
\pgfsetdash{}{0pt}%
\pgfpathmoveto{\pgfqpoint{1.529924in}{1.212048in}}%
\pgfpathlineto{\pgfqpoint{1.533590in}{1.204576in}}%
\pgfpathlineto{\pgfqpoint{1.537253in}{1.197119in}}%
\pgfpathlineto{\pgfqpoint{1.540914in}{1.189680in}}%
\pgfpathlineto{\pgfqpoint{1.544572in}{1.182262in}}%
\pgfpathlineto{\pgfqpoint{1.545676in}{1.176496in}}%
\pgfpathlineto{\pgfqpoint{1.546418in}{1.170711in}}%
\pgfpathlineto{\pgfqpoint{1.546796in}{1.164912in}}%
\pgfpathlineto{\pgfqpoint{1.546808in}{1.159105in}}%
\pgfpathlineto{\pgfqpoint{1.543116in}{1.166781in}}%
\pgfpathlineto{\pgfqpoint{1.539421in}{1.174477in}}%
\pgfpathlineto{\pgfqpoint{1.535724in}{1.182191in}}%
\pgfpathlineto{\pgfqpoint{1.532024in}{1.189920in}}%
\pgfpathlineto{\pgfqpoint{1.532024in}{1.195468in}}%
\pgfpathlineto{\pgfqpoint{1.531673in}{1.201009in}}%
\pgfpathlineto{\pgfqpoint{1.530973in}{1.206538in}}%
\pgfpathlineto{\pgfqpoint{1.529924in}{1.212048in}}%
\pgfpathclose%
\pgfusepath{fill}%
\end{pgfscope}%
\begin{pgfscope}%
\pgfpathrectangle{\pgfqpoint{0.041670in}{0.041670in}}{\pgfqpoint{2.216660in}{2.216660in}}%
\pgfusepath{clip}%
\pgfsetbuttcap%
\pgfsetroundjoin%
\definecolor{currentfill}{rgb}{0.133743,0.548535,0.553541}%
\pgfsetfillcolor{currentfill}%
\pgfsetlinewidth{0.000000pt}%
\definecolor{currentstroke}{rgb}{0.000000,0.000000,0.000000}%
\pgfsetstrokecolor{currentstroke}%
\pgfsetdash{}{0pt}%
\pgfpathmoveto{\pgfqpoint{1.493588in}{1.291783in}}%
\pgfpathlineto{\pgfqpoint{1.497187in}{1.284540in}}%
\pgfpathlineto{\pgfqpoint{1.500783in}{1.277291in}}%
\pgfpathlineto{\pgfqpoint{1.504376in}{1.270037in}}%
\pgfpathlineto{\pgfqpoint{1.507965in}{1.262781in}}%
\pgfpathlineto{\pgfqpoint{1.510272in}{1.257645in}}%
\pgfpathlineto{\pgfqpoint{1.512254in}{1.252471in}}%
\pgfpathlineto{\pgfqpoint{1.513908in}{1.247264in}}%
\pgfpathlineto{\pgfqpoint{1.515233in}{1.242031in}}%
\pgfpathlineto{\pgfqpoint{1.511553in}{1.249536in}}%
\pgfpathlineto{\pgfqpoint{1.507870in}{1.257039in}}%
\pgfpathlineto{\pgfqpoint{1.504184in}{1.264537in}}%
\pgfpathlineto{\pgfqpoint{1.500494in}{1.272029in}}%
\pgfpathlineto{\pgfqpoint{1.499238in}{1.277010in}}%
\pgfpathlineto{\pgfqpoint{1.497667in}{1.281967in}}%
\pgfpathlineto{\pgfqpoint{1.495783in}{1.286892in}}%
\pgfpathlineto{\pgfqpoint{1.493588in}{1.291783in}}%
\pgfpathclose%
\pgfusepath{fill}%
\end{pgfscope}%
\begin{pgfscope}%
\pgfpathrectangle{\pgfqpoint{0.041670in}{0.041670in}}{\pgfqpoint{2.216660in}{2.216660in}}%
\pgfusepath{clip}%
\pgfsetbuttcap%
\pgfsetroundjoin%
\definecolor{currentfill}{rgb}{0.212395,0.359683,0.551710}%
\pgfsetfillcolor{currentfill}%
\pgfsetlinewidth{0.000000pt}%
\definecolor{currentstroke}{rgb}{0.000000,0.000000,0.000000}%
\pgfsetstrokecolor{currentstroke}%
\pgfsetdash{}{0pt}%
\pgfpathmoveto{\pgfqpoint{0.789147in}{1.067954in}}%
\pgfpathlineto{\pgfqpoint{0.785508in}{1.060283in}}%
\pgfpathlineto{\pgfqpoint{0.781869in}{1.052664in}}%
\pgfpathlineto{\pgfqpoint{0.778232in}{1.045099in}}%
\pgfpathlineto{\pgfqpoint{0.774596in}{1.037592in}}%
\pgfpathlineto{\pgfqpoint{0.772641in}{1.044136in}}%
\pgfpathlineto{\pgfqpoint{0.771097in}{1.050703in}}%
\pgfpathlineto{\pgfqpoint{0.769965in}{1.057284in}}%
\pgfpathlineto{\pgfqpoint{0.769244in}{1.063875in}}%
\pgfpathlineto{\pgfqpoint{0.772914in}{1.071119in}}%
\pgfpathlineto{\pgfqpoint{0.776586in}{1.078422in}}%
\pgfpathlineto{\pgfqpoint{0.780259in}{1.085779in}}%
\pgfpathlineto{\pgfqpoint{0.783934in}{1.093188in}}%
\pgfpathlineto{\pgfqpoint{0.784642in}{1.086860in}}%
\pgfpathlineto{\pgfqpoint{0.785747in}{1.080541in}}%
\pgfpathlineto{\pgfqpoint{0.787249in}{1.074237in}}%
\pgfpathlineto{\pgfqpoint{0.789147in}{1.067954in}}%
\pgfpathclose%
\pgfusepath{fill}%
\end{pgfscope}%
\begin{pgfscope}%
\pgfpathrectangle{\pgfqpoint{0.041670in}{0.041670in}}{\pgfqpoint{2.216660in}{2.216660in}}%
\pgfusepath{clip}%
\pgfsetbuttcap%
\pgfsetroundjoin%
\definecolor{currentfill}{rgb}{0.412913,0.803041,0.357269}%
\pgfsetfillcolor{currentfill}%
\pgfsetlinewidth{0.000000pt}%
\definecolor{currentstroke}{rgb}{0.000000,0.000000,0.000000}%
\pgfsetstrokecolor{currentstroke}%
\pgfsetdash{}{0pt}%
\pgfpathmoveto{\pgfqpoint{1.230001in}{1.576534in}}%
\pgfpathlineto{\pgfqpoint{1.230935in}{1.572222in}}%
\pgfpathlineto{\pgfqpoint{1.231868in}{1.567829in}}%
\pgfpathlineto{\pgfqpoint{1.232800in}{1.563358in}}%
\pgfpathlineto{\pgfqpoint{1.233730in}{1.558812in}}%
\pgfpathlineto{\pgfqpoint{1.240241in}{1.557961in}}%
\pgfpathlineto{\pgfqpoint{1.246694in}{1.557013in}}%
\pgfpathlineto{\pgfqpoint{1.253084in}{1.555969in}}%
\pgfpathlineto{\pgfqpoint{1.251819in}{1.560566in}}%
\pgfpathlineto{\pgfqpoint{1.250551in}{1.565088in}}%
\pgfpathlineto{\pgfqpoint{1.249282in}{1.569532in}}%
\pgfpathlineto{\pgfqpoint{1.248011in}{1.573896in}}%
\pgfpathlineto{\pgfqpoint{1.242065in}{1.574865in}}%
\pgfpathlineto{\pgfqpoint{1.236060in}{1.575744in}}%
\pgfpathlineto{\pgfqpoint{1.230001in}{1.576534in}}%
\pgfpathclose%
\pgfusepath{fill}%
\end{pgfscope}%
\begin{pgfscope}%
\pgfpathrectangle{\pgfqpoint{0.041670in}{0.041670in}}{\pgfqpoint{2.216660in}{2.216660in}}%
\pgfusepath{clip}%
\pgfsetbuttcap%
\pgfsetroundjoin%
\definecolor{currentfill}{rgb}{0.233603,0.313828,0.543914}%
\pgfsetfillcolor{currentfill}%
\pgfsetlinewidth{0.000000pt}%
\definecolor{currentstroke}{rgb}{0.000000,0.000000,0.000000}%
\pgfsetstrokecolor{currentstroke}%
\pgfsetdash{}{0pt}%
\pgfpathmoveto{\pgfqpoint{1.894628in}{1.022067in}}%
\pgfpathlineto{\pgfqpoint{1.898827in}{1.033175in}}%
\pgfpathlineto{\pgfqpoint{1.903047in}{1.044729in}}%
\pgfpathlineto{\pgfqpoint{1.907287in}{1.056738in}}%
\pgfpathlineto{\pgfqpoint{1.911549in}{1.069208in}}%
\pgfpathlineto{\pgfqpoint{1.911452in}{1.057464in}}%
\pgfpathlineto{\pgfqpoint{1.910621in}{1.045707in}}%
\pgfpathlineto{\pgfqpoint{1.909054in}{1.033950in}}%
\pgfpathlineto{\pgfqpoint{1.906749in}{1.022205in}}%
\pgfpathlineto{\pgfqpoint{1.902488in}{1.009927in}}%
\pgfpathlineto{\pgfqpoint{1.898248in}{0.998114in}}%
\pgfpathlineto{\pgfqpoint{1.894028in}{0.986757in}}%
\pgfpathlineto{\pgfqpoint{1.889829in}{0.975850in}}%
\pgfpathlineto{\pgfqpoint{1.892108in}{0.987397in}}%
\pgfpathlineto{\pgfqpoint{1.893666in}{0.998957in}}%
\pgfpathlineto{\pgfqpoint{1.894505in}{1.010518in}}%
\pgfpathlineto{\pgfqpoint{1.894628in}{1.022067in}}%
\pgfpathclose%
\pgfusepath{fill}%
\end{pgfscope}%
\begin{pgfscope}%
\pgfpathrectangle{\pgfqpoint{0.041670in}{0.041670in}}{\pgfqpoint{2.216660in}{2.216660in}}%
\pgfusepath{clip}%
\pgfsetbuttcap%
\pgfsetroundjoin%
\definecolor{currentfill}{rgb}{0.412913,0.803041,0.357269}%
\pgfsetfillcolor{currentfill}%
\pgfsetlinewidth{0.000000pt}%
\definecolor{currentstroke}{rgb}{0.000000,0.000000,0.000000}%
\pgfsetstrokecolor{currentstroke}%
\pgfsetdash{}{0pt}%
\pgfpathmoveto{\pgfqpoint{1.106668in}{1.572960in}}%
\pgfpathlineto{\pgfqpoint{1.105299in}{1.568578in}}%
\pgfpathlineto{\pgfqpoint{1.103932in}{1.564117in}}%
\pgfpathlineto{\pgfqpoint{1.102567in}{1.559577in}}%
\pgfpathlineto{\pgfqpoint{1.101204in}{1.554961in}}%
\pgfpathlineto{\pgfqpoint{1.107532in}{1.556089in}}%
\pgfpathlineto{\pgfqpoint{1.113929in}{1.557123in}}%
\pgfpathlineto{\pgfqpoint{1.120389in}{1.558060in}}%
\pgfpathlineto{\pgfqpoint{1.126906in}{1.558900in}}%
\pgfpathlineto{\pgfqpoint{1.127824in}{1.563445in}}%
\pgfpathlineto{\pgfqpoint{1.128743in}{1.567915in}}%
\pgfpathlineto{\pgfqpoint{1.129664in}{1.572306in}}%
\pgfpathlineto{\pgfqpoint{1.130585in}{1.576617in}}%
\pgfpathlineto{\pgfqpoint{1.124521in}{1.575837in}}%
\pgfpathlineto{\pgfqpoint{1.118509in}{1.574967in}}%
\pgfpathlineto{\pgfqpoint{1.112556in}{1.574008in}}%
\pgfpathlineto{\pgfqpoint{1.106668in}{1.572960in}}%
\pgfpathclose%
\pgfusepath{fill}%
\end{pgfscope}%
\begin{pgfscope}%
\pgfpathrectangle{\pgfqpoint{0.041670in}{0.041670in}}{\pgfqpoint{2.216660in}{2.216660in}}%
\pgfusepath{clip}%
\pgfsetbuttcap%
\pgfsetroundjoin%
\definecolor{currentfill}{rgb}{0.271305,0.019942,0.347269}%
\pgfsetfillcolor{currentfill}%
\pgfsetlinewidth{0.000000pt}%
\definecolor{currentstroke}{rgb}{0.000000,0.000000,0.000000}%
\pgfsetstrokecolor{currentstroke}%
\pgfsetdash{}{0pt}%
\pgfpathmoveto{\pgfqpoint{1.705493in}{0.807973in}}%
\pgfpathlineto{\pgfqpoint{1.709121in}{0.804929in}}%
\pgfpathlineto{\pgfqpoint{1.712754in}{0.802088in}}%
\pgfpathlineto{\pgfqpoint{1.716392in}{0.799452in}}%
\pgfpathlineto{\pgfqpoint{1.720035in}{0.797027in}}%
\pgfpathlineto{\pgfqpoint{1.715264in}{0.788062in}}%
\pgfpathlineto{\pgfqpoint{1.709940in}{0.779171in}}%
\pgfpathlineto{\pgfqpoint{1.704066in}{0.770362in}}%
\pgfpathlineto{\pgfqpoint{1.697647in}{0.761646in}}%
\pgfpathlineto{\pgfqpoint{1.694138in}{0.764323in}}%
\pgfpathlineto{\pgfqpoint{1.690634in}{0.767210in}}%
\pgfpathlineto{\pgfqpoint{1.687136in}{0.770304in}}%
\pgfpathlineto{\pgfqpoint{1.683642in}{0.773601in}}%
\pgfpathlineto{\pgfqpoint{1.689904in}{0.782067in}}%
\pgfpathlineto{\pgfqpoint{1.695635in}{0.790625in}}%
\pgfpathlineto{\pgfqpoint{1.700833in}{0.799263in}}%
\pgfpathlineto{\pgfqpoint{1.705493in}{0.807973in}}%
\pgfpathclose%
\pgfusepath{fill}%
\end{pgfscope}%
\begin{pgfscope}%
\pgfpathrectangle{\pgfqpoint{0.041670in}{0.041670in}}{\pgfqpoint{2.216660in}{2.216660in}}%
\pgfusepath{clip}%
\pgfsetbuttcap%
\pgfsetroundjoin%
\definecolor{currentfill}{rgb}{0.281477,0.755203,0.432552}%
\pgfsetfillcolor{currentfill}%
\pgfsetlinewidth{0.000000pt}%
\definecolor{currentstroke}{rgb}{0.000000,0.000000,0.000000}%
\pgfsetstrokecolor{currentstroke}%
\pgfsetdash{}{0pt}%
\pgfpathmoveto{\pgfqpoint{1.002558in}{1.503703in}}%
\pgfpathlineto{\pgfqpoint{0.999690in}{1.498193in}}%
\pgfpathlineto{\pgfqpoint{0.996826in}{1.492619in}}%
\pgfpathlineto{\pgfqpoint{0.993965in}{1.486982in}}%
\pgfpathlineto{\pgfqpoint{0.991107in}{1.481286in}}%
\pgfpathlineto{\pgfqpoint{0.996185in}{1.484098in}}%
\pgfpathlineto{\pgfqpoint{1.001439in}{1.486832in}}%
\pgfpathlineto{\pgfqpoint{1.006864in}{1.489484in}}%
\pgfpathlineto{\pgfqpoint{1.012456in}{1.492052in}}%
\pgfpathlineto{\pgfqpoint{1.014987in}{1.497577in}}%
\pgfpathlineto{\pgfqpoint{1.017520in}{1.503042in}}%
\pgfpathlineto{\pgfqpoint{1.020057in}{1.508444in}}%
\pgfpathlineto{\pgfqpoint{1.022597in}{1.513783in}}%
\pgfpathlineto{\pgfqpoint{1.017348in}{1.511379in}}%
\pgfpathlineto{\pgfqpoint{1.012255in}{1.508896in}}%
\pgfpathlineto{\pgfqpoint{1.007323in}{1.506336in}}%
\pgfpathlineto{\pgfqpoint{1.002558in}{1.503703in}}%
\pgfpathclose%
\pgfusepath{fill}%
\end{pgfscope}%
\begin{pgfscope}%
\pgfpathrectangle{\pgfqpoint{0.041670in}{0.041670in}}{\pgfqpoint{2.216660in}{2.216660in}}%
\pgfusepath{clip}%
\pgfsetbuttcap%
\pgfsetroundjoin%
\definecolor{currentfill}{rgb}{0.268510,0.009605,0.335427}%
\pgfsetfillcolor{currentfill}%
\pgfsetlinewidth{0.000000pt}%
\definecolor{currentstroke}{rgb}{0.000000,0.000000,0.000000}%
\pgfsetstrokecolor{currentstroke}%
\pgfsetdash{}{0pt}%
\pgfpathmoveto{\pgfqpoint{1.720035in}{0.797027in}}%
\pgfpathlineto{\pgfqpoint{1.723684in}{0.794817in}}%
\pgfpathlineto{\pgfqpoint{1.727339in}{0.792826in}}%
\pgfpathlineto{\pgfqpoint{1.731001in}{0.791059in}}%
\pgfpathlineto{\pgfqpoint{1.734668in}{0.789522in}}%
\pgfpathlineto{\pgfqpoint{1.729786in}{0.780304in}}%
\pgfpathlineto{\pgfqpoint{1.724335in}{0.771162in}}%
\pgfpathlineto{\pgfqpoint{1.718320in}{0.762104in}}%
\pgfpathlineto{\pgfqpoint{1.711743in}{0.753140in}}%
\pgfpathlineto{\pgfqpoint{1.708210in}{0.754927in}}%
\pgfpathlineto{\pgfqpoint{1.704683in}{0.756944in}}%
\pgfpathlineto{\pgfqpoint{1.701162in}{0.759185in}}%
\pgfpathlineto{\pgfqpoint{1.697647in}{0.761646in}}%
\pgfpathlineto{\pgfqpoint{1.704066in}{0.770362in}}%
\pgfpathlineto{\pgfqpoint{1.709940in}{0.779171in}}%
\pgfpathlineto{\pgfqpoint{1.715264in}{0.788062in}}%
\pgfpathlineto{\pgfqpoint{1.720035in}{0.797027in}}%
\pgfpathclose%
\pgfusepath{fill}%
\end{pgfscope}%
\begin{pgfscope}%
\pgfpathrectangle{\pgfqpoint{0.041670in}{0.041670in}}{\pgfqpoint{2.216660in}{2.216660in}}%
\pgfusepath{clip}%
\pgfsetbuttcap%
\pgfsetroundjoin%
\definecolor{currentfill}{rgb}{0.282327,0.094955,0.417331}%
\pgfsetfillcolor{currentfill}%
\pgfsetlinewidth{0.000000pt}%
\definecolor{currentstroke}{rgb}{0.000000,0.000000,0.000000}%
\pgfsetstrokecolor{currentstroke}%
\pgfsetdash{}{0pt}%
\pgfpathmoveto{\pgfqpoint{0.554922in}{0.805094in}}%
\pgfpathlineto{\pgfqpoint{0.551078in}{0.809468in}}%
\pgfpathlineto{\pgfqpoint{0.547221in}{0.814187in}}%
\pgfpathlineto{\pgfqpoint{0.543351in}{0.819257in}}%
\pgfpathlineto{\pgfqpoint{0.539466in}{0.824683in}}%
\pgfpathlineto{\pgfqpoint{0.533992in}{0.835353in}}%
\pgfpathlineto{\pgfqpoint{0.529186in}{0.846091in}}%
\pgfpathlineto{\pgfqpoint{0.525049in}{0.856888in}}%
\pgfpathlineto{\pgfqpoint{0.521584in}{0.867731in}}%
\pgfpathlineto{\pgfqpoint{0.525554in}{0.862077in}}%
\pgfpathlineto{\pgfqpoint{0.529509in}{0.856778in}}%
\pgfpathlineto{\pgfqpoint{0.533451in}{0.851828in}}%
\pgfpathlineto{\pgfqpoint{0.537379in}{0.847221in}}%
\pgfpathlineto{\pgfqpoint{0.540784in}{0.836609in}}%
\pgfpathlineto{\pgfqpoint{0.544844in}{0.826042in}}%
\pgfpathlineto{\pgfqpoint{0.549557in}{0.815534in}}%
\pgfpathlineto{\pgfqpoint{0.554922in}{0.805094in}}%
\pgfpathclose%
\pgfusepath{fill}%
\end{pgfscope}%
\begin{pgfscope}%
\pgfpathrectangle{\pgfqpoint{0.041670in}{0.041670in}}{\pgfqpoint{2.216660in}{2.216660in}}%
\pgfusepath{clip}%
\pgfsetbuttcap%
\pgfsetroundjoin%
\definecolor{currentfill}{rgb}{0.274952,0.037752,0.364543}%
\pgfsetfillcolor{currentfill}%
\pgfsetlinewidth{0.000000pt}%
\definecolor{currentstroke}{rgb}{0.000000,0.000000,0.000000}%
\pgfsetstrokecolor{currentstroke}%
\pgfsetdash{}{0pt}%
\pgfpathmoveto{\pgfqpoint{1.691027in}{0.822075in}}%
\pgfpathlineto{\pgfqpoint{1.694637in}{0.818269in}}%
\pgfpathlineto{\pgfqpoint{1.698251in}{0.814647in}}%
\pgfpathlineto{\pgfqpoint{1.701870in}{0.811213in}}%
\pgfpathlineto{\pgfqpoint{1.705493in}{0.807973in}}%
\pgfpathlineto{\pgfqpoint{1.700833in}{0.799263in}}%
\pgfpathlineto{\pgfqpoint{1.695635in}{0.790625in}}%
\pgfpathlineto{\pgfqpoint{1.689904in}{0.782067in}}%
\pgfpathlineto{\pgfqpoint{1.683642in}{0.773601in}}%
\pgfpathlineto{\pgfqpoint{1.680153in}{0.777094in}}%
\pgfpathlineto{\pgfqpoint{1.676669in}{0.780781in}}%
\pgfpathlineto{\pgfqpoint{1.673189in}{0.784658in}}%
\pgfpathlineto{\pgfqpoint{1.669714in}{0.788718in}}%
\pgfpathlineto{\pgfqpoint{1.675818in}{0.796934in}}%
\pgfpathlineto{\pgfqpoint{1.681407in}{0.805238in}}%
\pgfpathlineto{\pgfqpoint{1.686478in}{0.813622in}}%
\pgfpathlineto{\pgfqpoint{1.691027in}{0.822075in}}%
\pgfpathclose%
\pgfusepath{fill}%
\end{pgfscope}%
\begin{pgfscope}%
\pgfpathrectangle{\pgfqpoint{0.041670in}{0.041670in}}{\pgfqpoint{2.216660in}{2.216660in}}%
\pgfusepath{clip}%
\pgfsetbuttcap%
\pgfsetroundjoin%
\definecolor{currentfill}{rgb}{0.134692,0.658636,0.517649}%
\pgfsetfillcolor{currentfill}%
\pgfsetlinewidth{0.000000pt}%
\definecolor{currentstroke}{rgb}{0.000000,0.000000,0.000000}%
\pgfsetstrokecolor{currentstroke}%
\pgfsetdash{}{0pt}%
\pgfpathmoveto{\pgfqpoint{0.917263in}{1.389886in}}%
\pgfpathlineto{\pgfqpoint{0.913730in}{1.383118in}}%
\pgfpathlineto{\pgfqpoint{0.910201in}{1.376312in}}%
\pgfpathlineto{\pgfqpoint{0.906675in}{1.369470in}}%
\pgfpathlineto{\pgfqpoint{0.903153in}{1.362596in}}%
\pgfpathlineto{\pgfqpoint{0.906259in}{1.366826in}}%
\pgfpathlineto{\pgfqpoint{0.909631in}{1.371003in}}%
\pgfpathlineto{\pgfqpoint{0.913266in}{1.375124in}}%
\pgfpathlineto{\pgfqpoint{0.917160in}{1.379184in}}%
\pgfpathlineto{\pgfqpoint{0.920496in}{1.385831in}}%
\pgfpathlineto{\pgfqpoint{0.923835in}{1.392446in}}%
\pgfpathlineto{\pgfqpoint{0.927178in}{1.399025in}}%
\pgfpathlineto{\pgfqpoint{0.930524in}{1.405567in}}%
\pgfpathlineto{\pgfqpoint{0.926836in}{1.401728in}}%
\pgfpathlineto{\pgfqpoint{0.923394in}{1.397832in}}%
\pgfpathlineto{\pgfqpoint{0.920202in}{1.393884in}}%
\pgfpathlineto{\pgfqpoint{0.917263in}{1.389886in}}%
\pgfpathclose%
\pgfusepath{fill}%
\end{pgfscope}%
\begin{pgfscope}%
\pgfpathrectangle{\pgfqpoint{0.041670in}{0.041670in}}{\pgfqpoint{2.216660in}{2.216660in}}%
\pgfusepath{clip}%
\pgfsetbuttcap%
\pgfsetroundjoin%
\definecolor{currentfill}{rgb}{0.282884,0.135920,0.453427}%
\pgfsetfillcolor{currentfill}%
\pgfsetlinewidth{0.000000pt}%
\definecolor{currentstroke}{rgb}{0.000000,0.000000,0.000000}%
\pgfsetstrokecolor{currentstroke}%
\pgfsetdash{}{0pt}%
\pgfpathmoveto{\pgfqpoint{1.840841in}{0.877399in}}%
\pgfpathlineto{\pgfqpoint{1.844835in}{0.883465in}}%
\pgfpathlineto{\pgfqpoint{1.848843in}{0.889898in}}%
\pgfpathlineto{\pgfqpoint{1.852866in}{0.896704in}}%
\pgfpathlineto{\pgfqpoint{1.856905in}{0.903890in}}%
\pgfpathlineto{\pgfqpoint{1.853992in}{0.892791in}}%
\pgfpathlineto{\pgfqpoint{1.850391in}{0.881729in}}%
\pgfpathlineto{\pgfqpoint{1.846102in}{0.870714in}}%
\pgfpathlineto{\pgfqpoint{1.841126in}{0.859760in}}%
\pgfpathlineto{\pgfqpoint{1.837157in}{0.852795in}}%
\pgfpathlineto{\pgfqpoint{1.833204in}{0.846211in}}%
\pgfpathlineto{\pgfqpoint{1.829266in}{0.840003in}}%
\pgfpathlineto{\pgfqpoint{1.825342in}{0.834164in}}%
\pgfpathlineto{\pgfqpoint{1.830223in}{0.844895in}}%
\pgfpathlineto{\pgfqpoint{1.834434in}{0.855686in}}%
\pgfpathlineto{\pgfqpoint{1.837974in}{0.866524in}}%
\pgfpathlineto{\pgfqpoint{1.840841in}{0.877399in}}%
\pgfpathclose%
\pgfusepath{fill}%
\end{pgfscope}%
\begin{pgfscope}%
\pgfpathrectangle{\pgfqpoint{0.041670in}{0.041670in}}{\pgfqpoint{2.216660in}{2.216660in}}%
\pgfusepath{clip}%
\pgfsetbuttcap%
\pgfsetroundjoin%
\definecolor{currentfill}{rgb}{0.267004,0.004874,0.329415}%
\pgfsetfillcolor{currentfill}%
\pgfsetlinewidth{0.000000pt}%
\definecolor{currentstroke}{rgb}{0.000000,0.000000,0.000000}%
\pgfsetstrokecolor{currentstroke}%
\pgfsetdash{}{0pt}%
\pgfpathmoveto{\pgfqpoint{1.734668in}{0.789522in}}%
\pgfpathlineto{\pgfqpoint{1.738342in}{0.788217in}}%
\pgfpathlineto{\pgfqpoint{1.742024in}{0.787151in}}%
\pgfpathlineto{\pgfqpoint{1.745712in}{0.786328in}}%
\pgfpathlineto{\pgfqpoint{1.749408in}{0.785752in}}%
\pgfpathlineto{\pgfqpoint{1.744414in}{0.776285in}}%
\pgfpathlineto{\pgfqpoint{1.738837in}{0.766893in}}%
\pgfpathlineto{\pgfqpoint{1.732680in}{0.757588in}}%
\pgfpathlineto{\pgfqpoint{1.725946in}{0.748379in}}%
\pgfpathlineto{\pgfqpoint{1.722384in}{0.749202in}}%
\pgfpathlineto{\pgfqpoint{1.718830in}{0.750273in}}%
\pgfpathlineto{\pgfqpoint{1.715283in}{0.751587in}}%
\pgfpathlineto{\pgfqpoint{1.711743in}{0.753140in}}%
\pgfpathlineto{\pgfqpoint{1.718320in}{0.762104in}}%
\pgfpathlineto{\pgfqpoint{1.724335in}{0.771162in}}%
\pgfpathlineto{\pgfqpoint{1.729786in}{0.780304in}}%
\pgfpathlineto{\pgfqpoint{1.734668in}{0.789522in}}%
\pgfpathclose%
\pgfusepath{fill}%
\end{pgfscope}%
\begin{pgfscope}%
\pgfpathrectangle{\pgfqpoint{0.041670in}{0.041670in}}{\pgfqpoint{2.216660in}{2.216660in}}%
\pgfusepath{clip}%
\pgfsetbuttcap%
\pgfsetroundjoin%
\definecolor{currentfill}{rgb}{0.344074,0.780029,0.397381}%
\pgfsetfillcolor{currentfill}%
\pgfsetlinewidth{0.000000pt}%
\definecolor{currentstroke}{rgb}{0.000000,0.000000,0.000000}%
\pgfsetstrokecolor{currentstroke}%
\pgfsetdash{}{0pt}%
\pgfpathmoveto{\pgfqpoint{1.301193in}{1.544292in}}%
\pgfpathlineto{\pgfqpoint{1.303292in}{1.539409in}}%
\pgfpathlineto{\pgfqpoint{1.305389in}{1.534454in}}%
\pgfpathlineto{\pgfqpoint{1.307484in}{1.529429in}}%
\pgfpathlineto{\pgfqpoint{1.309575in}{1.524335in}}%
\pgfpathlineto{\pgfqpoint{1.315511in}{1.522345in}}%
\pgfpathlineto{\pgfqpoint{1.321318in}{1.520267in}}%
\pgfpathlineto{\pgfqpoint{1.326989in}{1.518102in}}%
\pgfpathlineto{\pgfqpoint{1.332520in}{1.515852in}}%
\pgfpathlineto{\pgfqpoint{1.330055in}{1.521090in}}%
\pgfpathlineto{\pgfqpoint{1.327587in}{1.526258in}}%
\pgfpathlineto{\pgfqpoint{1.325116in}{1.531357in}}%
\pgfpathlineto{\pgfqpoint{1.322642in}{1.536383in}}%
\pgfpathlineto{\pgfqpoint{1.317472in}{1.538480in}}%
\pgfpathlineto{\pgfqpoint{1.312171in}{1.540499in}}%
\pgfpathlineto{\pgfqpoint{1.306743in}{1.542437in}}%
\pgfpathlineto{\pgfqpoint{1.301193in}{1.544292in}}%
\pgfpathclose%
\pgfusepath{fill}%
\end{pgfscope}%
\begin{pgfscope}%
\pgfpathrectangle{\pgfqpoint{0.041670in}{0.041670in}}{\pgfqpoint{2.216660in}{2.216660in}}%
\pgfusepath{clip}%
\pgfsetbuttcap%
\pgfsetroundjoin%
\definecolor{currentfill}{rgb}{0.279566,0.067836,0.391917}%
\pgfsetfillcolor{currentfill}%
\pgfsetlinewidth{0.000000pt}%
\definecolor{currentstroke}{rgb}{0.000000,0.000000,0.000000}%
\pgfsetstrokecolor{currentstroke}%
\pgfsetdash{}{0pt}%
\pgfpathmoveto{\pgfqpoint{1.676623in}{0.839061in}}%
\pgfpathlineto{\pgfqpoint{1.680219in}{0.834559in}}%
\pgfpathlineto{\pgfqpoint{1.683818in}{0.830224in}}%
\pgfpathlineto{\pgfqpoint{1.687420in}{0.826062in}}%
\pgfpathlineto{\pgfqpoint{1.691027in}{0.822075in}}%
\pgfpathlineto{\pgfqpoint{1.686478in}{0.813622in}}%
\pgfpathlineto{\pgfqpoint{1.681407in}{0.805238in}}%
\pgfpathlineto{\pgfqpoint{1.675818in}{0.796934in}}%
\pgfpathlineto{\pgfqpoint{1.669714in}{0.788718in}}%
\pgfpathlineto{\pgfqpoint{1.666242in}{0.792959in}}%
\pgfpathlineto{\pgfqpoint{1.662774in}{0.797377in}}%
\pgfpathlineto{\pgfqpoint{1.659310in}{0.801966in}}%
\pgfpathlineto{\pgfqpoint{1.655849in}{0.806724in}}%
\pgfpathlineto{\pgfqpoint{1.661796in}{0.814688in}}%
\pgfpathlineto{\pgfqpoint{1.667243in}{0.822738in}}%
\pgfpathlineto{\pgfqpoint{1.672187in}{0.830865in}}%
\pgfpathlineto{\pgfqpoint{1.676623in}{0.839061in}}%
\pgfpathclose%
\pgfusepath{fill}%
\end{pgfscope}%
\begin{pgfscope}%
\pgfpathrectangle{\pgfqpoint{0.041670in}{0.041670in}}{\pgfqpoint{2.216660in}{2.216660in}}%
\pgfusepath{clip}%
\pgfsetbuttcap%
\pgfsetroundjoin%
\definecolor{currentfill}{rgb}{0.120081,0.622161,0.534946}%
\pgfsetfillcolor{currentfill}%
\pgfsetlinewidth{0.000000pt}%
\definecolor{currentstroke}{rgb}{0.000000,0.000000,0.000000}%
\pgfsetstrokecolor{currentstroke}%
\pgfsetdash{}{0pt}%
\pgfpathmoveto{\pgfqpoint{1.454009in}{1.366358in}}%
\pgfpathlineto{\pgfqpoint{1.457491in}{1.359505in}}%
\pgfpathlineto{\pgfqpoint{1.460969in}{1.352624in}}%
\pgfpathlineto{\pgfqpoint{1.464443in}{1.345718in}}%
\pgfpathlineto{\pgfqpoint{1.467914in}{1.338789in}}%
\pgfpathlineto{\pgfqpoint{1.471155in}{1.334321in}}%
\pgfpathlineto{\pgfqpoint{1.474111in}{1.329801in}}%
\pgfpathlineto{\pgfqpoint{1.476780in}{1.325236in}}%
\pgfpathlineto{\pgfqpoint{1.479158in}{1.320628in}}%
\pgfpathlineto{\pgfqpoint{1.475542in}{1.327795in}}%
\pgfpathlineto{\pgfqpoint{1.471922in}{1.334938in}}%
\pgfpathlineto{\pgfqpoint{1.468299in}{1.342057in}}%
\pgfpathlineto{\pgfqpoint{1.464673in}{1.349147in}}%
\pgfpathlineto{\pgfqpoint{1.462420in}{1.353513in}}%
\pgfpathlineto{\pgfqpoint{1.459889in}{1.357840in}}%
\pgfpathlineto{\pgfqpoint{1.457085in}{1.362123in}}%
\pgfpathlineto{\pgfqpoint{1.454009in}{1.366358in}}%
\pgfpathclose%
\pgfusepath{fill}%
\end{pgfscope}%
\begin{pgfscope}%
\pgfpathrectangle{\pgfqpoint{0.041670in}{0.041670in}}{\pgfqpoint{2.216660in}{2.216660in}}%
\pgfusepath{clip}%
\pgfsetbuttcap%
\pgfsetroundjoin%
\definecolor{currentfill}{rgb}{0.195860,0.395433,0.555276}%
\pgfsetfillcolor{currentfill}%
\pgfsetlinewidth{0.000000pt}%
\definecolor{currentstroke}{rgb}{0.000000,0.000000,0.000000}%
\pgfsetstrokecolor{currentstroke}%
\pgfsetdash{}{0pt}%
\pgfpathmoveto{\pgfqpoint{1.561556in}{1.128672in}}%
\pgfpathlineto{\pgfqpoint{1.565238in}{1.121146in}}%
\pgfpathlineto{\pgfqpoint{1.568918in}{1.113659in}}%
\pgfpathlineto{\pgfqpoint{1.572596in}{1.106215in}}%
\pgfpathlineto{\pgfqpoint{1.576272in}{1.098815in}}%
\pgfpathlineto{\pgfqpoint{1.575916in}{1.092485in}}%
\pgfpathlineto{\pgfqpoint{1.575164in}{1.086157in}}%
\pgfpathlineto{\pgfqpoint{1.574015in}{1.079839in}}%
\pgfpathlineto{\pgfqpoint{1.572469in}{1.073537in}}%
\pgfpathlineto{\pgfqpoint{1.568815in}{1.081198in}}%
\pgfpathlineto{\pgfqpoint{1.565160in}{1.088905in}}%
\pgfpathlineto{\pgfqpoint{1.561503in}{1.096653in}}%
\pgfpathlineto{\pgfqpoint{1.557844in}{1.104440in}}%
\pgfpathlineto{\pgfqpoint{1.559345in}{1.110481in}}%
\pgfpathlineto{\pgfqpoint{1.560464in}{1.116537in}}%
\pgfpathlineto{\pgfqpoint{1.561201in}{1.122603in}}%
\pgfpathlineto{\pgfqpoint{1.561556in}{1.128672in}}%
\pgfpathclose%
\pgfusepath{fill}%
\end{pgfscope}%
\begin{pgfscope}%
\pgfpathrectangle{\pgfqpoint{0.041670in}{0.041670in}}{\pgfqpoint{2.216660in}{2.216660in}}%
\pgfusepath{clip}%
\pgfsetbuttcap%
\pgfsetroundjoin%
\definecolor{currentfill}{rgb}{0.263663,0.237631,0.518762}%
\pgfsetfillcolor{currentfill}%
\pgfsetlinewidth{0.000000pt}%
\definecolor{currentstroke}{rgb}{0.000000,0.000000,0.000000}%
\pgfsetstrokecolor{currentstroke}%
\pgfsetdash{}{0pt}%
\pgfpathmoveto{\pgfqpoint{0.758189in}{0.952114in}}%
\pgfpathlineto{\pgfqpoint{0.754650in}{0.945018in}}%
\pgfpathlineto{\pgfqpoint{0.751110in}{0.938013in}}%
\pgfpathlineto{\pgfqpoint{0.747571in}{0.931104in}}%
\pgfpathlineto{\pgfqpoint{0.744030in}{0.924293in}}%
\pgfpathlineto{\pgfqpoint{0.740098in}{0.931459in}}%
\pgfpathlineto{\pgfqpoint{0.736616in}{0.938677in}}%
\pgfpathlineto{\pgfqpoint{0.733587in}{0.945942in}}%
\pgfpathlineto{\pgfqpoint{0.731011in}{0.953245in}}%
\pgfpathlineto{\pgfqpoint{0.734642in}{0.959794in}}%
\pgfpathlineto{\pgfqpoint{0.738273in}{0.966442in}}%
\pgfpathlineto{\pgfqpoint{0.741904in}{0.973186in}}%
\pgfpathlineto{\pgfqpoint{0.745535in}{0.980021in}}%
\pgfpathlineto{\pgfqpoint{0.748042in}{0.972981in}}%
\pgfpathlineto{\pgfqpoint{0.750987in}{0.965978in}}%
\pgfpathlineto{\pgfqpoint{0.754370in}{0.959020in}}%
\pgfpathlineto{\pgfqpoint{0.758189in}{0.952114in}}%
\pgfpathclose%
\pgfusepath{fill}%
\end{pgfscope}%
\begin{pgfscope}%
\pgfpathrectangle{\pgfqpoint{0.041670in}{0.041670in}}{\pgfqpoint{2.216660in}{2.216660in}}%
\pgfusepath{clip}%
\pgfsetbuttcap%
\pgfsetroundjoin%
\definecolor{currentfill}{rgb}{0.133743,0.548535,0.553541}%
\pgfsetfillcolor{currentfill}%
\pgfsetlinewidth{0.000000pt}%
\definecolor{currentstroke}{rgb}{0.000000,0.000000,0.000000}%
\pgfsetstrokecolor{currentstroke}%
\pgfsetdash{}{0pt}%
\pgfpathmoveto{\pgfqpoint{0.858565in}{1.267583in}}%
\pgfpathlineto{\pgfqpoint{0.854864in}{1.260036in}}%
\pgfpathlineto{\pgfqpoint{0.851165in}{1.252481in}}%
\pgfpathlineto{\pgfqpoint{0.847470in}{1.244921in}}%
\pgfpathlineto{\pgfqpoint{0.843778in}{1.237360in}}%
\pgfpathlineto{\pgfqpoint{0.844808in}{1.242613in}}%
\pgfpathlineto{\pgfqpoint{0.846169in}{1.247844in}}%
\pgfpathlineto{\pgfqpoint{0.847860in}{1.253047in}}%
\pgfpathlineto{\pgfqpoint{0.849878in}{1.258217in}}%
\pgfpathlineto{\pgfqpoint{0.853493in}{1.265527in}}%
\pgfpathlineto{\pgfqpoint{0.857111in}{1.272836in}}%
\pgfpathlineto{\pgfqpoint{0.860731in}{1.280140in}}%
\pgfpathlineto{\pgfqpoint{0.864356in}{1.287437in}}%
\pgfpathlineto{\pgfqpoint{0.862437in}{1.282515in}}%
\pgfpathlineto{\pgfqpoint{0.860830in}{1.277562in}}%
\pgfpathlineto{\pgfqpoint{0.859539in}{1.272583in}}%
\pgfpathlineto{\pgfqpoint{0.858565in}{1.267583in}}%
\pgfpathclose%
\pgfusepath{fill}%
\end{pgfscope}%
\begin{pgfscope}%
\pgfpathrectangle{\pgfqpoint{0.041670in}{0.041670in}}{\pgfqpoint{2.216660in}{2.216660in}}%
\pgfusepath{clip}%
\pgfsetbuttcap%
\pgfsetroundjoin%
\definecolor{currentfill}{rgb}{0.412913,0.803041,0.357269}%
\pgfsetfillcolor{currentfill}%
\pgfsetlinewidth{0.000000pt}%
\definecolor{currentstroke}{rgb}{0.000000,0.000000,0.000000}%
\pgfsetstrokecolor{currentstroke}%
\pgfsetdash{}{0pt}%
\pgfpathmoveto{\pgfqpoint{1.248011in}{1.573896in}}%
\pgfpathlineto{\pgfqpoint{1.249282in}{1.569532in}}%
\pgfpathlineto{\pgfqpoint{1.250551in}{1.565088in}}%
\pgfpathlineto{\pgfqpoint{1.251819in}{1.560566in}}%
\pgfpathlineto{\pgfqpoint{1.253084in}{1.555969in}}%
\pgfpathlineto{\pgfqpoint{1.259405in}{1.554829in}}%
\pgfpathlineto{\pgfqpoint{1.265649in}{1.553596in}}%
\pgfpathlineto{\pgfqpoint{1.271812in}{1.552270in}}%
\pgfpathlineto{\pgfqpoint{1.277886in}{1.550853in}}%
\pgfpathlineto{\pgfqpoint{1.276190in}{1.555543in}}%
\pgfpathlineto{\pgfqpoint{1.274491in}{1.560156in}}%
\pgfpathlineto{\pgfqpoint{1.272790in}{1.564692in}}%
\pgfpathlineto{\pgfqpoint{1.271087in}{1.569148in}}%
\pgfpathlineto{\pgfqpoint{1.265436in}{1.570463in}}%
\pgfpathlineto{\pgfqpoint{1.259702in}{1.571694in}}%
\pgfpathlineto{\pgfqpoint{1.253892in}{1.572838in}}%
\pgfpathlineto{\pgfqpoint{1.248011in}{1.573896in}}%
\pgfpathclose%
\pgfusepath{fill}%
\end{pgfscope}%
\begin{pgfscope}%
\pgfpathrectangle{\pgfqpoint{0.041670in}{0.041670in}}{\pgfqpoint{2.216660in}{2.216660in}}%
\pgfusepath{clip}%
\pgfsetbuttcap%
\pgfsetroundjoin%
\definecolor{currentfill}{rgb}{0.163625,0.471133,0.558148}%
\pgfsetfillcolor{currentfill}%
\pgfsetlinewidth{0.000000pt}%
\definecolor{currentstroke}{rgb}{0.000000,0.000000,0.000000}%
\pgfsetstrokecolor{currentstroke}%
\pgfsetdash{}{0pt}%
\pgfpathmoveto{\pgfqpoint{0.828181in}{1.184986in}}%
\pgfpathlineto{\pgfqpoint{0.824481in}{1.177200in}}%
\pgfpathlineto{\pgfqpoint{0.820785in}{1.169428in}}%
\pgfpathlineto{\pgfqpoint{0.817090in}{1.161675in}}%
\pgfpathlineto{\pgfqpoint{0.813398in}{1.153942in}}%
\pgfpathlineto{\pgfqpoint{0.813085in}{1.159751in}}%
\pgfpathlineto{\pgfqpoint{0.813138in}{1.165557in}}%
\pgfpathlineto{\pgfqpoint{0.813556in}{1.171355in}}%
\pgfpathlineto{\pgfqpoint{0.814339in}{1.177138in}}%
\pgfpathlineto{\pgfqpoint{0.818009in}{1.184613in}}%
\pgfpathlineto{\pgfqpoint{0.821682in}{1.192108in}}%
\pgfpathlineto{\pgfqpoint{0.825358in}{1.199622in}}%
\pgfpathlineto{\pgfqpoint{0.829036in}{1.207151in}}%
\pgfpathlineto{\pgfqpoint{0.828297in}{1.201624in}}%
\pgfpathlineto{\pgfqpoint{0.827907in}{1.196084in}}%
\pgfpathlineto{\pgfqpoint{0.827868in}{1.190536in}}%
\pgfpathlineto{\pgfqpoint{0.828181in}{1.184986in}}%
\pgfpathclose%
\pgfusepath{fill}%
\end{pgfscope}%
\begin{pgfscope}%
\pgfpathrectangle{\pgfqpoint{0.041670in}{0.041670in}}{\pgfqpoint{2.216660in}{2.216660in}}%
\pgfusepath{clip}%
\pgfsetbuttcap%
\pgfsetroundjoin%
\definecolor{currentfill}{rgb}{0.248629,0.278775,0.534556}%
\pgfsetfillcolor{currentfill}%
\pgfsetlinewidth{0.000000pt}%
\definecolor{currentstroke}{rgb}{0.000000,0.000000,0.000000}%
\pgfsetstrokecolor{currentstroke}%
\pgfsetdash{}{0pt}%
\pgfpathmoveto{\pgfqpoint{1.601658in}{1.014259in}}%
\pgfpathlineto{\pgfqpoint{1.605302in}{1.007150in}}%
\pgfpathlineto{\pgfqpoint{1.608946in}{1.000119in}}%
\pgfpathlineto{\pgfqpoint{1.612590in}{0.993169in}}%
\pgfpathlineto{\pgfqpoint{1.616233in}{0.986304in}}%
\pgfpathlineto{\pgfqpoint{1.614118in}{0.979237in}}%
\pgfpathlineto{\pgfqpoint{1.611562in}{0.972201in}}%
\pgfpathlineto{\pgfqpoint{1.608568in}{0.965203in}}%
\pgfpathlineto{\pgfqpoint{1.605136in}{0.958250in}}%
\pgfpathlineto{\pgfqpoint{1.601572in}{0.965377in}}%
\pgfpathlineto{\pgfqpoint{1.598007in}{0.972589in}}%
\pgfpathlineto{\pgfqpoint{1.594442in}{0.979882in}}%
\pgfpathlineto{\pgfqpoint{1.590877in}{0.987252in}}%
\pgfpathlineto{\pgfqpoint{1.594207in}{0.993945in}}%
\pgfpathlineto{\pgfqpoint{1.597115in}{1.000681in}}%
\pgfpathlineto{\pgfqpoint{1.599599in}{1.007455in}}%
\pgfpathlineto{\pgfqpoint{1.601658in}{1.014259in}}%
\pgfpathclose%
\pgfusepath{fill}%
\end{pgfscope}%
\begin{pgfscope}%
\pgfpathrectangle{\pgfqpoint{0.041670in}{0.041670in}}{\pgfqpoint{2.216660in}{2.216660in}}%
\pgfusepath{clip}%
\pgfsetbuttcap%
\pgfsetroundjoin%
\definecolor{currentfill}{rgb}{0.344074,0.780029,0.397381}%
\pgfsetfillcolor{currentfill}%
\pgfsetlinewidth{0.000000pt}%
\definecolor{currentstroke}{rgb}{0.000000,0.000000,0.000000}%
\pgfsetstrokecolor{currentstroke}%
\pgfsetdash{}{0pt}%
\pgfpathmoveto{\pgfqpoint{1.032788in}{1.534453in}}%
\pgfpathlineto{\pgfqpoint{1.030236in}{1.529393in}}%
\pgfpathlineto{\pgfqpoint{1.027686in}{1.524259in}}%
\pgfpathlineto{\pgfqpoint{1.025140in}{1.519055in}}%
\pgfpathlineto{\pgfqpoint{1.022597in}{1.513783in}}%
\pgfpathlineto{\pgfqpoint{1.027997in}{1.516107in}}%
\pgfpathlineto{\pgfqpoint{1.033544in}{1.518347in}}%
\pgfpathlineto{\pgfqpoint{1.039230in}{1.520503in}}%
\pgfpathlineto{\pgfqpoint{1.045052in}{1.522571in}}%
\pgfpathlineto{\pgfqpoint{1.047229in}{1.527694in}}%
\pgfpathlineto{\pgfqpoint{1.049409in}{1.532750in}}%
\pgfpathlineto{\pgfqpoint{1.051592in}{1.537735in}}%
\pgfpathlineto{\pgfqpoint{1.053778in}{1.542647in}}%
\pgfpathlineto{\pgfqpoint{1.048336in}{1.540718in}}%
\pgfpathlineto{\pgfqpoint{1.043020in}{1.538709in}}%
\pgfpathlineto{\pgfqpoint{1.037836in}{1.536620in}}%
\pgfpathlineto{\pgfqpoint{1.032788in}{1.534453in}}%
\pgfpathclose%
\pgfusepath{fill}%
\end{pgfscope}%
\begin{pgfscope}%
\pgfpathrectangle{\pgfqpoint{0.041670in}{0.041670in}}{\pgfqpoint{2.216660in}{2.216660in}}%
\pgfusepath{clip}%
\pgfsetbuttcap%
\pgfsetroundjoin%
\definecolor{currentfill}{rgb}{0.412913,0.803041,0.357269}%
\pgfsetfillcolor{currentfill}%
\pgfsetlinewidth{0.000000pt}%
\definecolor{currentstroke}{rgb}{0.000000,0.000000,0.000000}%
\pgfsetstrokecolor{currentstroke}%
\pgfsetdash{}{0pt}%
\pgfpathmoveto{\pgfqpoint{1.083872in}{1.567908in}}%
\pgfpathlineto{\pgfqpoint{1.082076in}{1.563428in}}%
\pgfpathlineto{\pgfqpoint{1.080282in}{1.558868in}}%
\pgfpathlineto{\pgfqpoint{1.078491in}{1.554231in}}%
\pgfpathlineto{\pgfqpoint{1.076702in}{1.549516in}}%
\pgfpathlineto{\pgfqpoint{1.082694in}{1.551015in}}%
\pgfpathlineto{\pgfqpoint{1.088779in}{1.552422in}}%
\pgfpathlineto{\pgfqpoint{1.094951in}{1.553738in}}%
\pgfpathlineto{\pgfqpoint{1.101204in}{1.554961in}}%
\pgfpathlineto{\pgfqpoint{1.102567in}{1.559577in}}%
\pgfpathlineto{\pgfqpoint{1.103932in}{1.564117in}}%
\pgfpathlineto{\pgfqpoint{1.105299in}{1.568578in}}%
\pgfpathlineto{\pgfqpoint{1.106668in}{1.572960in}}%
\pgfpathlineto{\pgfqpoint{1.100850in}{1.571825in}}%
\pgfpathlineto{\pgfqpoint{1.095107in}{1.570604in}}%
\pgfpathlineto{\pgfqpoint{1.089446in}{1.569298in}}%
\pgfpathlineto{\pgfqpoint{1.083872in}{1.567908in}}%
\pgfpathclose%
\pgfusepath{fill}%
\end{pgfscope}%
\begin{pgfscope}%
\pgfpathrectangle{\pgfqpoint{0.041670in}{0.041670in}}{\pgfqpoint{2.216660in}{2.216660in}}%
\pgfusepath{clip}%
\pgfsetbuttcap%
\pgfsetroundjoin%
\definecolor{currentfill}{rgb}{0.267004,0.004874,0.329415}%
\pgfsetfillcolor{currentfill}%
\pgfsetlinewidth{0.000000pt}%
\definecolor{currentstroke}{rgb}{0.000000,0.000000,0.000000}%
\pgfsetstrokecolor{currentstroke}%
\pgfsetdash{}{0pt}%
\pgfpathmoveto{\pgfqpoint{1.749408in}{0.785752in}}%
\pgfpathlineto{\pgfqpoint{1.753111in}{0.785429in}}%
\pgfpathlineto{\pgfqpoint{1.756822in}{0.785363in}}%
\pgfpathlineto{\pgfqpoint{1.760542in}{0.785560in}}%
\pgfpathlineto{\pgfqpoint{1.764270in}{0.786025in}}%
\pgfpathlineto{\pgfqpoint{1.759166in}{0.776310in}}%
\pgfpathlineto{\pgfqpoint{1.753463in}{0.766673in}}%
\pgfpathlineto{\pgfqpoint{1.747163in}{0.757124in}}%
\pgfpathlineto{\pgfqpoint{1.740271in}{0.747673in}}%
\pgfpathlineto{\pgfqpoint{1.736677in}{0.747452in}}%
\pgfpathlineto{\pgfqpoint{1.733092in}{0.747500in}}%
\pgfpathlineto{\pgfqpoint{1.729515in}{0.747810in}}%
\pgfpathlineto{\pgfqpoint{1.725946in}{0.748379in}}%
\pgfpathlineto{\pgfqpoint{1.732680in}{0.757588in}}%
\pgfpathlineto{\pgfqpoint{1.738837in}{0.766893in}}%
\pgfpathlineto{\pgfqpoint{1.744414in}{0.776285in}}%
\pgfpathlineto{\pgfqpoint{1.749408in}{0.785752in}}%
\pgfpathclose%
\pgfusepath{fill}%
\end{pgfscope}%
\begin{pgfscope}%
\pgfpathrectangle{\pgfqpoint{0.041670in}{0.041670in}}{\pgfqpoint{2.216660in}{2.216660in}}%
\pgfusepath{clip}%
\pgfsetbuttcap%
\pgfsetroundjoin%
\definecolor{currentfill}{rgb}{0.282327,0.094955,0.417331}%
\pgfsetfillcolor{currentfill}%
\pgfsetlinewidth{0.000000pt}%
\definecolor{currentstroke}{rgb}{0.000000,0.000000,0.000000}%
\pgfsetstrokecolor{currentstroke}%
\pgfsetdash{}{0pt}%
\pgfpathmoveto{\pgfqpoint{1.662269in}{0.858664in}}%
\pgfpathlineto{\pgfqpoint{1.665854in}{0.853532in}}%
\pgfpathlineto{\pgfqpoint{1.669441in}{0.848551in}}%
\pgfpathlineto{\pgfqpoint{1.673030in}{0.843726in}}%
\pgfpathlineto{\pgfqpoint{1.676623in}{0.839061in}}%
\pgfpathlineto{\pgfqpoint{1.672187in}{0.830865in}}%
\pgfpathlineto{\pgfqpoint{1.667243in}{0.822738in}}%
\pgfpathlineto{\pgfqpoint{1.661796in}{0.814688in}}%
\pgfpathlineto{\pgfqpoint{1.655849in}{0.806724in}}%
\pgfpathlineto{\pgfqpoint{1.652391in}{0.811645in}}%
\pgfpathlineto{\pgfqpoint{1.648937in}{0.816726in}}%
\pgfpathlineto{\pgfqpoint{1.645485in}{0.821962in}}%
\pgfpathlineto{\pgfqpoint{1.642036in}{0.827351in}}%
\pgfpathlineto{\pgfqpoint{1.647825in}{0.835062in}}%
\pgfpathlineto{\pgfqpoint{1.653130in}{0.842857in}}%
\pgfpathlineto{\pgfqpoint{1.657945in}{0.850727in}}%
\pgfpathlineto{\pgfqpoint{1.662269in}{0.858664in}}%
\pgfpathclose%
\pgfusepath{fill}%
\end{pgfscope}%
\begin{pgfscope}%
\pgfpathrectangle{\pgfqpoint{0.041670in}{0.041670in}}{\pgfqpoint{2.216660in}{2.216660in}}%
\pgfusepath{clip}%
\pgfsetbuttcap%
\pgfsetroundjoin%
\definecolor{currentfill}{rgb}{0.120081,0.622161,0.534946}%
\pgfsetfillcolor{currentfill}%
\pgfsetlinewidth{0.000000pt}%
\definecolor{currentstroke}{rgb}{0.000000,0.000000,0.000000}%
\pgfsetstrokecolor{currentstroke}%
\pgfsetdash{}{0pt}%
\pgfpathmoveto{\pgfqpoint{0.893468in}{1.345235in}}%
\pgfpathlineto{\pgfqpoint{0.889817in}{1.338091in}}%
\pgfpathlineto{\pgfqpoint{0.886170in}{1.330919in}}%
\pgfpathlineto{\pgfqpoint{0.882526in}{1.323721in}}%
\pgfpathlineto{\pgfqpoint{0.878885in}{1.316500in}}%
\pgfpathlineto{\pgfqpoint{0.881002in}{1.321142in}}%
\pgfpathlineto{\pgfqpoint{0.883412in}{1.325745in}}%
\pgfpathlineto{\pgfqpoint{0.886113in}{1.330306in}}%
\pgfpathlineto{\pgfqpoint{0.889101in}{1.334820in}}%
\pgfpathlineto{\pgfqpoint{0.892609in}{1.341801in}}%
\pgfpathlineto{\pgfqpoint{0.896120in}{1.348759in}}%
\pgfpathlineto{\pgfqpoint{0.899635in}{1.355691in}}%
\pgfpathlineto{\pgfqpoint{0.903153in}{1.362596in}}%
\pgfpathlineto{\pgfqpoint{0.900319in}{1.358318in}}%
\pgfpathlineto{\pgfqpoint{0.897758in}{1.353996in}}%
\pgfpathlineto{\pgfqpoint{0.895473in}{1.349634in}}%
\pgfpathlineto{\pgfqpoint{0.893468in}{1.345235in}}%
\pgfpathclose%
\pgfusepath{fill}%
\end{pgfscope}%
\begin{pgfscope}%
\pgfpathrectangle{\pgfqpoint{0.041670in}{0.041670in}}{\pgfqpoint{2.216660in}{2.216660in}}%
\pgfusepath{clip}%
\pgfsetbuttcap%
\pgfsetroundjoin%
\definecolor{currentfill}{rgb}{0.271305,0.019942,0.347269}%
\pgfsetfillcolor{currentfill}%
\pgfsetlinewidth{0.000000pt}%
\definecolor{currentstroke}{rgb}{0.000000,0.000000,0.000000}%
\pgfsetstrokecolor{currentstroke}%
\pgfsetdash{}{0pt}%
\pgfpathmoveto{\pgfqpoint{0.682275in}{0.766158in}}%
\pgfpathlineto{\pgfqpoint{0.678819in}{0.762806in}}%
\pgfpathlineto{\pgfqpoint{0.675359in}{0.759657in}}%
\pgfpathlineto{\pgfqpoint{0.671893in}{0.756715in}}%
\pgfpathlineto{\pgfqpoint{0.668422in}{0.753983in}}%
\pgfpathlineto{\pgfqpoint{0.661523in}{0.762609in}}%
\pgfpathlineto{\pgfqpoint{0.655164in}{0.771336in}}%
\pgfpathlineto{\pgfqpoint{0.649351in}{0.780155in}}%
\pgfpathlineto{\pgfqpoint{0.644088in}{0.789055in}}%
\pgfpathlineto{\pgfqpoint{0.647707in}{0.791536in}}%
\pgfpathlineto{\pgfqpoint{0.651321in}{0.794229in}}%
\pgfpathlineto{\pgfqpoint{0.654929in}{0.797127in}}%
\pgfpathlineto{\pgfqpoint{0.658532in}{0.800227in}}%
\pgfpathlineto{\pgfqpoint{0.663671in}{0.791581in}}%
\pgfpathlineto{\pgfqpoint{0.669343in}{0.783014in}}%
\pgfpathlineto{\pgfqpoint{0.675546in}{0.774537in}}%
\pgfpathlineto{\pgfqpoint{0.682275in}{0.766158in}}%
\pgfpathclose%
\pgfusepath{fill}%
\end{pgfscope}%
\begin{pgfscope}%
\pgfpathrectangle{\pgfqpoint{0.041670in}{0.041670in}}{\pgfqpoint{2.216660in}{2.216660in}}%
\pgfusepath{clip}%
\pgfsetbuttcap%
\pgfsetroundjoin%
\definecolor{currentfill}{rgb}{0.268510,0.009605,0.335427}%
\pgfsetfillcolor{currentfill}%
\pgfsetlinewidth{0.000000pt}%
\definecolor{currentstroke}{rgb}{0.000000,0.000000,0.000000}%
\pgfsetstrokecolor{currentstroke}%
\pgfsetdash{}{0pt}%
\pgfpathmoveto{\pgfqpoint{0.668422in}{0.753983in}}%
\pgfpathlineto{\pgfqpoint{0.664945in}{0.751467in}}%
\pgfpathlineto{\pgfqpoint{0.661463in}{0.749172in}}%
\pgfpathlineto{\pgfqpoint{0.657974in}{0.747101in}}%
\pgfpathlineto{\pgfqpoint{0.654479in}{0.745259in}}%
\pgfpathlineto{\pgfqpoint{0.647408in}{0.754131in}}%
\pgfpathlineto{\pgfqpoint{0.640894in}{0.763106in}}%
\pgfpathlineto{\pgfqpoint{0.634941in}{0.772173in}}%
\pgfpathlineto{\pgfqpoint{0.629553in}{0.781325in}}%
\pgfpathlineto{\pgfqpoint{0.633196in}{0.782919in}}%
\pgfpathlineto{\pgfqpoint{0.636833in}{0.784741in}}%
\pgfpathlineto{\pgfqpoint{0.640464in}{0.786788in}}%
\pgfpathlineto{\pgfqpoint{0.644088in}{0.789055in}}%
\pgfpathlineto{\pgfqpoint{0.649351in}{0.780155in}}%
\pgfpathlineto{\pgfqpoint{0.655164in}{0.771336in}}%
\pgfpathlineto{\pgfqpoint{0.661523in}{0.762609in}}%
\pgfpathlineto{\pgfqpoint{0.668422in}{0.753983in}}%
\pgfpathclose%
\pgfusepath{fill}%
\end{pgfscope}%
\begin{pgfscope}%
\pgfpathrectangle{\pgfqpoint{0.041670in}{0.041670in}}{\pgfqpoint{2.216660in}{2.216660in}}%
\pgfusepath{clip}%
\pgfsetbuttcap%
\pgfsetroundjoin%
\definecolor{currentfill}{rgb}{0.274952,0.037752,0.364543}%
\pgfsetfillcolor{currentfill}%
\pgfsetlinewidth{0.000000pt}%
\definecolor{currentstroke}{rgb}{0.000000,0.000000,0.000000}%
\pgfsetstrokecolor{currentstroke}%
\pgfsetdash{}{0pt}%
\pgfpathmoveto{\pgfqpoint{0.696051in}{0.781496in}}%
\pgfpathlineto{\pgfqpoint{0.692613in}{0.777380in}}%
\pgfpathlineto{\pgfqpoint{0.689171in}{0.773449in}}%
\pgfpathlineto{\pgfqpoint{0.685725in}{0.769706in}}%
\pgfpathlineto{\pgfqpoint{0.682275in}{0.766158in}}%
\pgfpathlineto{\pgfqpoint{0.675546in}{0.774537in}}%
\pgfpathlineto{\pgfqpoint{0.669343in}{0.783014in}}%
\pgfpathlineto{\pgfqpoint{0.663671in}{0.791581in}}%
\pgfpathlineto{\pgfqpoint{0.658532in}{0.800227in}}%
\pgfpathlineto{\pgfqpoint{0.662131in}{0.803525in}}%
\pgfpathlineto{\pgfqpoint{0.665725in}{0.807015in}}%
\pgfpathlineto{\pgfqpoint{0.669315in}{0.810694in}}%
\pgfpathlineto{\pgfqpoint{0.672900in}{0.814558in}}%
\pgfpathlineto{\pgfqpoint{0.677913in}{0.806166in}}%
\pgfpathlineto{\pgfqpoint{0.683445in}{0.797853in}}%
\pgfpathlineto{\pgfqpoint{0.689493in}{0.789627in}}%
\pgfpathlineto{\pgfqpoint{0.696051in}{0.781496in}}%
\pgfpathclose%
\pgfusepath{fill}%
\end{pgfscope}%
\begin{pgfscope}%
\pgfpathrectangle{\pgfqpoint{0.041670in}{0.041670in}}{\pgfqpoint{2.216660in}{2.216660in}}%
\pgfusepath{clip}%
\pgfsetbuttcap%
\pgfsetroundjoin%
\definecolor{currentfill}{rgb}{0.195860,0.395433,0.555276}%
\pgfsetfillcolor{currentfill}%
\pgfsetlinewidth{0.000000pt}%
\definecolor{currentstroke}{rgb}{0.000000,0.000000,0.000000}%
\pgfsetstrokecolor{currentstroke}%
\pgfsetdash{}{0pt}%
\pgfpathmoveto{\pgfqpoint{0.803720in}{1.099089in}}%
\pgfpathlineto{\pgfqpoint{0.800075in}{1.091244in}}%
\pgfpathlineto{\pgfqpoint{0.796431in}{1.083438in}}%
\pgfpathlineto{\pgfqpoint{0.792788in}{1.075673in}}%
\pgfpathlineto{\pgfqpoint{0.789147in}{1.067954in}}%
\pgfpathlineto{\pgfqpoint{0.787249in}{1.074237in}}%
\pgfpathlineto{\pgfqpoint{0.785747in}{1.080541in}}%
\pgfpathlineto{\pgfqpoint{0.784642in}{1.086860in}}%
\pgfpathlineto{\pgfqpoint{0.783934in}{1.093188in}}%
\pgfpathlineto{\pgfqpoint{0.787611in}{1.100645in}}%
\pgfpathlineto{\pgfqpoint{0.791289in}{1.108148in}}%
\pgfpathlineto{\pgfqpoint{0.794969in}{1.115693in}}%
\pgfpathlineto{\pgfqpoint{0.798650in}{1.123277in}}%
\pgfpathlineto{\pgfqpoint{0.799345in}{1.117211in}}%
\pgfpathlineto{\pgfqpoint{0.800422in}{1.111153in}}%
\pgfpathlineto{\pgfqpoint{0.801880in}{1.105110in}}%
\pgfpathlineto{\pgfqpoint{0.803720in}{1.099089in}}%
\pgfpathclose%
\pgfusepath{fill}%
\end{pgfscope}%
\begin{pgfscope}%
\pgfpathrectangle{\pgfqpoint{0.041670in}{0.041670in}}{\pgfqpoint{2.216660in}{2.216660in}}%
\pgfusepath{clip}%
\pgfsetbuttcap%
\pgfsetroundjoin%
\definecolor{currentfill}{rgb}{0.220124,0.725509,0.466226}%
\pgfsetfillcolor{currentfill}%
\pgfsetlinewidth{0.000000pt}%
\definecolor{currentstroke}{rgb}{0.000000,0.000000,0.000000}%
\pgfsetstrokecolor{currentstroke}%
\pgfsetdash{}{0pt}%
\pgfpathmoveto{\pgfqpoint{1.383414in}{1.472058in}}%
\pgfpathlineto{\pgfqpoint{1.386493in}{1.466155in}}%
\pgfpathlineto{\pgfqpoint{1.389569in}{1.460197in}}%
\pgfpathlineto{\pgfqpoint{1.392641in}{1.454185in}}%
\pgfpathlineto{\pgfqpoint{1.395710in}{1.448122in}}%
\pgfpathlineto{\pgfqpoint{1.400275in}{1.444812in}}%
\pgfpathlineto{\pgfqpoint{1.404627in}{1.441433in}}%
\pgfpathlineto{\pgfqpoint{1.408761in}{1.437987in}}%
\pgfpathlineto{\pgfqpoint{1.412674in}{1.434479in}}%
\pgfpathlineto{\pgfqpoint{1.409358in}{1.440749in}}%
\pgfpathlineto{\pgfqpoint{1.406037in}{1.446967in}}%
\pgfpathlineto{\pgfqpoint{1.402713in}{1.453132in}}%
\pgfpathlineto{\pgfqpoint{1.399386in}{1.459240in}}%
\pgfpathlineto{\pgfqpoint{1.395703in}{1.462536in}}%
\pgfpathlineto{\pgfqpoint{1.391811in}{1.465773in}}%
\pgfpathlineto{\pgfqpoint{1.387713in}{1.468948in}}%
\pgfpathlineto{\pgfqpoint{1.383414in}{1.472058in}}%
\pgfpathclose%
\pgfusepath{fill}%
\end{pgfscope}%
\begin{pgfscope}%
\pgfpathrectangle{\pgfqpoint{0.041670in}{0.041670in}}{\pgfqpoint{2.216660in}{2.216660in}}%
\pgfusepath{clip}%
\pgfsetbuttcap%
\pgfsetroundjoin%
\definecolor{currentfill}{rgb}{0.147607,0.511733,0.557049}%
\pgfsetfillcolor{currentfill}%
\pgfsetlinewidth{0.000000pt}%
\definecolor{currentstroke}{rgb}{0.000000,0.000000,0.000000}%
\pgfsetstrokecolor{currentstroke}%
\pgfsetdash{}{0pt}%
\pgfpathmoveto{\pgfqpoint{1.515233in}{1.242031in}}%
\pgfpathlineto{\pgfqpoint{1.518910in}{1.234527in}}%
\pgfpathlineto{\pgfqpoint{1.522584in}{1.227026in}}%
\pgfpathlineto{\pgfqpoint{1.526256in}{1.219532in}}%
\pgfpathlineto{\pgfqpoint{1.529924in}{1.212048in}}%
\pgfpathlineto{\pgfqpoint{1.530973in}{1.206538in}}%
\pgfpathlineto{\pgfqpoint{1.531673in}{1.201009in}}%
\pgfpathlineto{\pgfqpoint{1.532024in}{1.195468in}}%
\pgfpathlineto{\pgfqpoint{1.532024in}{1.189920in}}%
\pgfpathlineto{\pgfqpoint{1.528322in}{1.197661in}}%
\pgfpathlineto{\pgfqpoint{1.524617in}{1.205411in}}%
\pgfpathlineto{\pgfqpoint{1.520909in}{1.213167in}}%
\pgfpathlineto{\pgfqpoint{1.517199in}{1.220927in}}%
\pgfpathlineto{\pgfqpoint{1.517210in}{1.226218in}}%
\pgfpathlineto{\pgfqpoint{1.516886in}{1.231502in}}%
\pgfpathlineto{\pgfqpoint{1.516226in}{1.236775in}}%
\pgfpathlineto{\pgfqpoint{1.515233in}{1.242031in}}%
\pgfpathclose%
\pgfusepath{fill}%
\end{pgfscope}%
\begin{pgfscope}%
\pgfpathrectangle{\pgfqpoint{0.041670in}{0.041670in}}{\pgfqpoint{2.216660in}{2.216660in}}%
\pgfusepath{clip}%
\pgfsetbuttcap%
\pgfsetroundjoin%
\definecolor{currentfill}{rgb}{0.281477,0.755203,0.432552}%
\pgfsetfillcolor{currentfill}%
\pgfsetlinewidth{0.000000pt}%
\definecolor{currentstroke}{rgb}{0.000000,0.000000,0.000000}%
\pgfsetstrokecolor{currentstroke}%
\pgfsetdash{}{0pt}%
\pgfpathmoveto{\pgfqpoint{1.353124in}{1.506047in}}%
\pgfpathlineto{\pgfqpoint{1.355923in}{1.500577in}}%
\pgfpathlineto{\pgfqpoint{1.358718in}{1.495043in}}%
\pgfpathlineto{\pgfqpoint{1.361510in}{1.489446in}}%
\pgfpathlineto{\pgfqpoint{1.364298in}{1.483790in}}%
\pgfpathlineto{\pgfqpoint{1.369356in}{1.480968in}}%
\pgfpathlineto{\pgfqpoint{1.374231in}{1.478071in}}%
\pgfpathlineto{\pgfqpoint{1.378918in}{1.475100in}}%
\pgfpathlineto{\pgfqpoint{1.383414in}{1.472058in}}%
\pgfpathlineto{\pgfqpoint{1.380331in}{1.477902in}}%
\pgfpathlineto{\pgfqpoint{1.377245in}{1.483686in}}%
\pgfpathlineto{\pgfqpoint{1.374155in}{1.489408in}}%
\pgfpathlineto{\pgfqpoint{1.371061in}{1.495065in}}%
\pgfpathlineto{\pgfqpoint{1.366844in}{1.497912in}}%
\pgfpathlineto{\pgfqpoint{1.362445in}{1.500693in}}%
\pgfpathlineto{\pgfqpoint{1.357871in}{1.503406in}}%
\pgfpathlineto{\pgfqpoint{1.353124in}{1.506047in}}%
\pgfpathclose%
\pgfusepath{fill}%
\end{pgfscope}%
\begin{pgfscope}%
\pgfpathrectangle{\pgfqpoint{0.041670in}{0.041670in}}{\pgfqpoint{2.216660in}{2.216660in}}%
\pgfusepath{clip}%
\pgfsetbuttcap%
\pgfsetroundjoin%
\definecolor{currentfill}{rgb}{0.283072,0.130895,0.449241}%
\pgfsetfillcolor{currentfill}%
\pgfsetlinewidth{0.000000pt}%
\definecolor{currentstroke}{rgb}{0.000000,0.000000,0.000000}%
\pgfsetstrokecolor{currentstroke}%
\pgfsetdash{}{0pt}%
\pgfpathmoveto{\pgfqpoint{1.647954in}{0.880629in}}%
\pgfpathlineto{\pgfqpoint{1.651530in}{0.874930in}}%
\pgfpathlineto{\pgfqpoint{1.655107in}{0.869367in}}%
\pgfpathlineto{\pgfqpoint{1.658687in}{0.863944in}}%
\pgfpathlineto{\pgfqpoint{1.662269in}{0.858664in}}%
\pgfpathlineto{\pgfqpoint{1.657945in}{0.850727in}}%
\pgfpathlineto{\pgfqpoint{1.653130in}{0.842857in}}%
\pgfpathlineto{\pgfqpoint{1.647825in}{0.835062in}}%
\pgfpathlineto{\pgfqpoint{1.642036in}{0.827351in}}%
\pgfpathlineto{\pgfqpoint{1.638590in}{0.832887in}}%
\pgfpathlineto{\pgfqpoint{1.635146in}{0.838566in}}%
\pgfpathlineto{\pgfqpoint{1.631704in}{0.844386in}}%
\pgfpathlineto{\pgfqpoint{1.628264in}{0.850341in}}%
\pgfpathlineto{\pgfqpoint{1.633895in}{0.857799in}}%
\pgfpathlineto{\pgfqpoint{1.639056in}{0.865339in}}%
\pgfpathlineto{\pgfqpoint{1.643743in}{0.872951in}}%
\pgfpathlineto{\pgfqpoint{1.647954in}{0.880629in}}%
\pgfpathclose%
\pgfusepath{fill}%
\end{pgfscope}%
\begin{pgfscope}%
\pgfpathrectangle{\pgfqpoint{0.041670in}{0.041670in}}{\pgfqpoint{2.216660in}{2.216660in}}%
\pgfusepath{clip}%
\pgfsetbuttcap%
\pgfsetroundjoin%
\definecolor{currentfill}{rgb}{0.268510,0.009605,0.335427}%
\pgfsetfillcolor{currentfill}%
\pgfsetlinewidth{0.000000pt}%
\definecolor{currentstroke}{rgb}{0.000000,0.000000,0.000000}%
\pgfsetstrokecolor{currentstroke}%
\pgfsetdash{}{0pt}%
\pgfpathmoveto{\pgfqpoint{1.764270in}{0.786025in}}%
\pgfpathlineto{\pgfqpoint{1.768007in}{0.786762in}}%
\pgfpathlineto{\pgfqpoint{1.771754in}{0.787776in}}%
\pgfpathlineto{\pgfqpoint{1.775509in}{0.789074in}}%
\pgfpathlineto{\pgfqpoint{1.779275in}{0.790660in}}%
\pgfpathlineto{\pgfqpoint{1.774060in}{0.780702in}}%
\pgfpathlineto{\pgfqpoint{1.768230in}{0.770823in}}%
\pgfpathlineto{\pgfqpoint{1.761787in}{0.761033in}}%
\pgfpathlineto{\pgfqpoint{1.754737in}{0.751343in}}%
\pgfpathlineto{\pgfqpoint{1.751106in}{0.749997in}}%
\pgfpathlineto{\pgfqpoint{1.747485in}{0.748940in}}%
\pgfpathlineto{\pgfqpoint{1.743873in}{0.748167in}}%
\pgfpathlineto{\pgfqpoint{1.740271in}{0.747673in}}%
\pgfpathlineto{\pgfqpoint{1.747163in}{0.757124in}}%
\pgfpathlineto{\pgfqpoint{1.753463in}{0.766673in}}%
\pgfpathlineto{\pgfqpoint{1.759166in}{0.776310in}}%
\pgfpathlineto{\pgfqpoint{1.764270in}{0.786025in}}%
\pgfpathclose%
\pgfusepath{fill}%
\end{pgfscope}%
\begin{pgfscope}%
\pgfpathrectangle{\pgfqpoint{0.041670in}{0.041670in}}{\pgfqpoint{2.216660in}{2.216660in}}%
\pgfusepath{clip}%
\pgfsetbuttcap%
\pgfsetroundjoin%
\definecolor{currentfill}{rgb}{0.412913,0.803041,0.357269}%
\pgfsetfillcolor{currentfill}%
\pgfsetlinewidth{0.000000pt}%
\definecolor{currentstroke}{rgb}{0.000000,0.000000,0.000000}%
\pgfsetstrokecolor{currentstroke}%
\pgfsetdash{}{0pt}%
\pgfpathmoveto{\pgfqpoint{1.271087in}{1.569148in}}%
\pgfpathlineto{\pgfqpoint{1.272790in}{1.564692in}}%
\pgfpathlineto{\pgfqpoint{1.274491in}{1.560156in}}%
\pgfpathlineto{\pgfqpoint{1.276190in}{1.555543in}}%
\pgfpathlineto{\pgfqpoint{1.277886in}{1.550853in}}%
\pgfpathlineto{\pgfqpoint{1.283868in}{1.549344in}}%
\pgfpathlineto{\pgfqpoint{1.289750in}{1.547747in}}%
\pgfpathlineto{\pgfqpoint{1.295527in}{1.546063in}}%
\pgfpathlineto{\pgfqpoint{1.301193in}{1.544292in}}%
\pgfpathlineto{\pgfqpoint{1.299091in}{1.549100in}}%
\pgfpathlineto{\pgfqpoint{1.296986in}{1.553832in}}%
\pgfpathlineto{\pgfqpoint{1.294878in}{1.558486in}}%
\pgfpathlineto{\pgfqpoint{1.292768in}{1.563060in}}%
\pgfpathlineto{\pgfqpoint{1.287497in}{1.564703in}}%
\pgfpathlineto{\pgfqpoint{1.282124in}{1.566266in}}%
\pgfpathlineto{\pgfqpoint{1.276652in}{1.567748in}}%
\pgfpathlineto{\pgfqpoint{1.271087in}{1.569148in}}%
\pgfpathclose%
\pgfusepath{fill}%
\end{pgfscope}%
\begin{pgfscope}%
\pgfpathrectangle{\pgfqpoint{0.041670in}{0.041670in}}{\pgfqpoint{2.216660in}{2.216660in}}%
\pgfusepath{clip}%
\pgfsetbuttcap%
\pgfsetroundjoin%
\definecolor{currentfill}{rgb}{0.279566,0.067836,0.391917}%
\pgfsetfillcolor{currentfill}%
\pgfsetlinewidth{0.000000pt}%
\definecolor{currentstroke}{rgb}{0.000000,0.000000,0.000000}%
\pgfsetstrokecolor{currentstroke}%
\pgfsetdash{}{0pt}%
\pgfpathmoveto{\pgfqpoint{0.709763in}{0.799724in}}%
\pgfpathlineto{\pgfqpoint{0.706340in}{0.794911in}}%
\pgfpathlineto{\pgfqpoint{0.702914in}{0.790266in}}%
\pgfpathlineto{\pgfqpoint{0.699484in}{0.785793in}}%
\pgfpathlineto{\pgfqpoint{0.696051in}{0.781496in}}%
\pgfpathlineto{\pgfqpoint{0.689493in}{0.789627in}}%
\pgfpathlineto{\pgfqpoint{0.683445in}{0.797853in}}%
\pgfpathlineto{\pgfqpoint{0.677913in}{0.806166in}}%
\pgfpathlineto{\pgfqpoint{0.672900in}{0.814558in}}%
\pgfpathlineto{\pgfqpoint{0.676482in}{0.818602in}}%
\pgfpathlineto{\pgfqpoint{0.680060in}{0.822821in}}%
\pgfpathlineto{\pgfqpoint{0.683634in}{0.827213in}}%
\pgfpathlineto{\pgfqpoint{0.687205in}{0.831772in}}%
\pgfpathlineto{\pgfqpoint{0.692092in}{0.823637in}}%
\pgfpathlineto{\pgfqpoint{0.697484in}{0.815578in}}%
\pgfpathlineto{\pgfqpoint{0.703375in}{0.807604in}}%
\pgfpathlineto{\pgfqpoint{0.709763in}{0.799724in}}%
\pgfpathclose%
\pgfusepath{fill}%
\end{pgfscope}%
\begin{pgfscope}%
\pgfpathrectangle{\pgfqpoint{0.041670in}{0.041670in}}{\pgfqpoint{2.216660in}{2.216660in}}%
\pgfusepath{clip}%
\pgfsetbuttcap%
\pgfsetroundjoin%
\definecolor{currentfill}{rgb}{0.267004,0.004874,0.329415}%
\pgfsetfillcolor{currentfill}%
\pgfsetlinewidth{0.000000pt}%
\definecolor{currentstroke}{rgb}{0.000000,0.000000,0.000000}%
\pgfsetstrokecolor{currentstroke}%
\pgfsetdash{}{0pt}%
\pgfpathmoveto{\pgfqpoint{0.654479in}{0.745259in}}%
\pgfpathlineto{\pgfqpoint{0.650977in}{0.743652in}}%
\pgfpathlineto{\pgfqpoint{0.647468in}{0.742284in}}%
\pgfpathlineto{\pgfqpoint{0.643952in}{0.741159in}}%
\pgfpathlineto{\pgfqpoint{0.640429in}{0.740283in}}%
\pgfpathlineto{\pgfqpoint{0.633187in}{0.749398in}}%
\pgfpathlineto{\pgfqpoint{0.626517in}{0.758618in}}%
\pgfpathlineto{\pgfqpoint{0.620424in}{0.767933in}}%
\pgfpathlineto{\pgfqpoint{0.614912in}{0.777333in}}%
\pgfpathlineto{\pgfqpoint{0.618583in}{0.777964in}}%
\pgfpathlineto{\pgfqpoint{0.622247in}{0.778843in}}%
\pgfpathlineto{\pgfqpoint{0.625904in}{0.779965in}}%
\pgfpathlineto{\pgfqpoint{0.629553in}{0.781325in}}%
\pgfpathlineto{\pgfqpoint{0.634941in}{0.772173in}}%
\pgfpathlineto{\pgfqpoint{0.640894in}{0.763106in}}%
\pgfpathlineto{\pgfqpoint{0.647408in}{0.754131in}}%
\pgfpathlineto{\pgfqpoint{0.654479in}{0.745259in}}%
\pgfpathclose%
\pgfusepath{fill}%
\end{pgfscope}%
\begin{pgfscope}%
\pgfpathrectangle{\pgfqpoint{0.041670in}{0.041670in}}{\pgfqpoint{2.216660in}{2.216660in}}%
\pgfusepath{clip}%
\pgfsetbuttcap%
\pgfsetroundjoin%
\definecolor{currentfill}{rgb}{0.248629,0.278775,0.534556}%
\pgfsetfillcolor{currentfill}%
\pgfsetlinewidth{0.000000pt}%
\definecolor{currentstroke}{rgb}{0.000000,0.000000,0.000000}%
\pgfsetstrokecolor{currentstroke}%
\pgfsetdash{}{0pt}%
\pgfpathmoveto{\pgfqpoint{0.772345in}{0.981346in}}%
\pgfpathlineto{\pgfqpoint{0.768806in}{0.973918in}}%
\pgfpathlineto{\pgfqpoint{0.765267in}{0.966568in}}%
\pgfpathlineto{\pgfqpoint{0.761728in}{0.959298in}}%
\pgfpathlineto{\pgfqpoint{0.758189in}{0.952114in}}%
\pgfpathlineto{\pgfqpoint{0.754370in}{0.959020in}}%
\pgfpathlineto{\pgfqpoint{0.750987in}{0.965978in}}%
\pgfpathlineto{\pgfqpoint{0.748042in}{0.972981in}}%
\pgfpathlineto{\pgfqpoint{0.745535in}{0.980021in}}%
\pgfpathlineto{\pgfqpoint{0.749166in}{0.986944in}}%
\pgfpathlineto{\pgfqpoint{0.752797in}{0.993952in}}%
\pgfpathlineto{\pgfqpoint{0.756429in}{1.001042in}}%
\pgfpathlineto{\pgfqpoint{0.760061in}{1.008210in}}%
\pgfpathlineto{\pgfqpoint{0.762498in}{1.001432in}}%
\pgfpathlineto{\pgfqpoint{0.765359in}{0.994691in}}%
\pgfpathlineto{\pgfqpoint{0.768642in}{0.987994in}}%
\pgfpathlineto{\pgfqpoint{0.772345in}{0.981346in}}%
\pgfpathclose%
\pgfusepath{fill}%
\end{pgfscope}%
\begin{pgfscope}%
\pgfpathrectangle{\pgfqpoint{0.041670in}{0.041670in}}{\pgfqpoint{2.216660in}{2.216660in}}%
\pgfusepath{clip}%
\pgfsetbuttcap%
\pgfsetroundjoin%
\definecolor{currentfill}{rgb}{0.233603,0.313828,0.543914}%
\pgfsetfillcolor{currentfill}%
\pgfsetlinewidth{0.000000pt}%
\definecolor{currentstroke}{rgb}{0.000000,0.000000,0.000000}%
\pgfsetstrokecolor{currentstroke}%
\pgfsetdash{}{0pt}%
\pgfpathmoveto{\pgfqpoint{0.472713in}{0.965605in}}%
\pgfpathlineto{\pgfqpoint{0.468523in}{0.976468in}}%
\pgfpathlineto{\pgfqpoint{0.464313in}{0.987780in}}%
\pgfpathlineto{\pgfqpoint{0.460083in}{0.999550in}}%
\pgfpathlineto{\pgfqpoint{0.455831in}{1.011784in}}%
\pgfpathlineto{\pgfqpoint{0.452868in}{1.023509in}}%
\pgfpathlineto{\pgfqpoint{0.450645in}{1.035256in}}%
\pgfpathlineto{\pgfqpoint{0.449160in}{1.047014in}}%
\pgfpathlineto{\pgfqpoint{0.448411in}{1.058770in}}%
\pgfpathlineto{\pgfqpoint{0.452678in}{1.046342in}}%
\pgfpathlineto{\pgfqpoint{0.456924in}{1.034376in}}%
\pgfpathlineto{\pgfqpoint{0.461150in}{1.022865in}}%
\pgfpathlineto{\pgfqpoint{0.465356in}{1.011802in}}%
\pgfpathlineto{\pgfqpoint{0.466115in}{1.000242in}}%
\pgfpathlineto{\pgfqpoint{0.467593in}{0.988681in}}%
\pgfpathlineto{\pgfqpoint{0.469792in}{0.977132in}}%
\pgfpathlineto{\pgfqpoint{0.472713in}{0.965605in}}%
\pgfpathclose%
\pgfusepath{fill}%
\end{pgfscope}%
\begin{pgfscope}%
\pgfpathrectangle{\pgfqpoint{0.041670in}{0.041670in}}{\pgfqpoint{2.216660in}{2.216660in}}%
\pgfusepath{clip}%
\pgfsetbuttcap%
\pgfsetroundjoin%
\definecolor{currentfill}{rgb}{0.122606,0.585371,0.546557}%
\pgfsetfillcolor{currentfill}%
\pgfsetlinewidth{0.000000pt}%
\definecolor{currentstroke}{rgb}{0.000000,0.000000,0.000000}%
\pgfsetstrokecolor{currentstroke}%
\pgfsetdash{}{0pt}%
\pgfpathmoveto{\pgfqpoint{1.479158in}{1.320628in}}%
\pgfpathlineto{\pgfqpoint{1.482770in}{1.313441in}}%
\pgfpathlineto{\pgfqpoint{1.486380in}{1.306236in}}%
\pgfpathlineto{\pgfqpoint{1.489985in}{1.299015in}}%
\pgfpathlineto{\pgfqpoint{1.493588in}{1.291783in}}%
\pgfpathlineto{\pgfqpoint{1.495783in}{1.286892in}}%
\pgfpathlineto{\pgfqpoint{1.497667in}{1.281967in}}%
\pgfpathlineto{\pgfqpoint{1.499238in}{1.277010in}}%
\pgfpathlineto{\pgfqpoint{1.500494in}{1.272029in}}%
\pgfpathlineto{\pgfqpoint{1.496802in}{1.279510in}}%
\pgfpathlineto{\pgfqpoint{1.493106in}{1.286978in}}%
\pgfpathlineto{\pgfqpoint{1.489408in}{1.294430in}}%
\pgfpathlineto{\pgfqpoint{1.485706in}{1.301865in}}%
\pgfpathlineto{\pgfqpoint{1.484518in}{1.306596in}}%
\pgfpathlineto{\pgfqpoint{1.483029in}{1.311304in}}%
\pgfpathlineto{\pgfqpoint{1.481242in}{1.315982in}}%
\pgfpathlineto{\pgfqpoint{1.479158in}{1.320628in}}%
\pgfpathclose%
\pgfusepath{fill}%
\end{pgfscope}%
\begin{pgfscope}%
\pgfpathrectangle{\pgfqpoint{0.041670in}{0.041670in}}{\pgfqpoint{2.216660in}{2.216660in}}%
\pgfusepath{clip}%
\pgfsetbuttcap%
\pgfsetroundjoin%
\definecolor{currentfill}{rgb}{0.166383,0.690856,0.496502}%
\pgfsetfillcolor{currentfill}%
\pgfsetlinewidth{0.000000pt}%
\definecolor{currentstroke}{rgb}{0.000000,0.000000,0.000000}%
\pgfsetstrokecolor{currentstroke}%
\pgfsetdash{}{0pt}%
\pgfpathmoveto{\pgfqpoint{1.412674in}{1.434479in}}%
\pgfpathlineto{\pgfqpoint{1.415987in}{1.428159in}}%
\pgfpathlineto{\pgfqpoint{1.419296in}{1.421793in}}%
\pgfpathlineto{\pgfqpoint{1.422601in}{1.415382in}}%
\pgfpathlineto{\pgfqpoint{1.425903in}{1.408929in}}%
\pgfpathlineto{\pgfqpoint{1.429807in}{1.405143in}}%
\pgfpathlineto{\pgfqpoint{1.433468in}{1.401298in}}%
\pgfpathlineto{\pgfqpoint{1.436883in}{1.397396in}}%
\pgfpathlineto{\pgfqpoint{1.440047in}{1.393442in}}%
\pgfpathlineto{\pgfqpoint{1.436547in}{1.400118in}}%
\pgfpathlineto{\pgfqpoint{1.433044in}{1.406752in}}%
\pgfpathlineto{\pgfqpoint{1.429537in}{1.413341in}}%
\pgfpathlineto{\pgfqpoint{1.426026in}{1.419883in}}%
\pgfpathlineto{\pgfqpoint{1.423041in}{1.423609in}}%
\pgfpathlineto{\pgfqpoint{1.419817in}{1.427287in}}%
\pgfpathlineto{\pgfqpoint{1.416361in}{1.430911in}}%
\pgfpathlineto{\pgfqpoint{1.412674in}{1.434479in}}%
\pgfpathclose%
\pgfusepath{fill}%
\end{pgfscope}%
\begin{pgfscope}%
\pgfpathrectangle{\pgfqpoint{0.041670in}{0.041670in}}{\pgfqpoint{2.216660in}{2.216660in}}%
\pgfusepath{clip}%
\pgfsetbuttcap%
\pgfsetroundjoin%
\definecolor{currentfill}{rgb}{0.487026,0.823929,0.312321}%
\pgfsetfillcolor{currentfill}%
\pgfsetlinewidth{0.000000pt}%
\definecolor{currentstroke}{rgb}{0.000000,0.000000,0.000000}%
\pgfsetstrokecolor{currentstroke}%
\pgfsetdash{}{0pt}%
\pgfpathmoveto{\pgfqpoint{1.157106in}{1.595061in}}%
\pgfpathlineto{\pgfqpoint{1.156642in}{1.591130in}}%
\pgfpathlineto{\pgfqpoint{1.156180in}{1.587113in}}%
\pgfpathlineto{\pgfqpoint{1.155718in}{1.583010in}}%
\pgfpathlineto{\pgfqpoint{1.155256in}{1.578824in}}%
\pgfpathlineto{\pgfqpoint{1.161497in}{1.579145in}}%
\pgfpathlineto{\pgfqpoint{1.167757in}{1.579373in}}%
\pgfpathlineto{\pgfqpoint{1.174027in}{1.579507in}}%
\pgfpathlineto{\pgfqpoint{1.180304in}{1.579549in}}%
\pgfpathlineto{\pgfqpoint{1.180297in}{1.583721in}}%
\pgfpathlineto{\pgfqpoint{1.180291in}{1.587810in}}%
\pgfpathlineto{\pgfqpoint{1.180284in}{1.591814in}}%
\pgfpathlineto{\pgfqpoint{1.180277in}{1.595730in}}%
\pgfpathlineto{\pgfqpoint{1.174471in}{1.595692in}}%
\pgfpathlineto{\pgfqpoint{1.168670in}{1.595567in}}%
\pgfpathlineto{\pgfqpoint{1.162880in}{1.595357in}}%
\pgfpathlineto{\pgfqpoint{1.157106in}{1.595061in}}%
\pgfpathclose%
\pgfusepath{fill}%
\end{pgfscope}%
\begin{pgfscope}%
\pgfpathrectangle{\pgfqpoint{0.041670in}{0.041670in}}{\pgfqpoint{2.216660in}{2.216660in}}%
\pgfusepath{clip}%
\pgfsetbuttcap%
\pgfsetroundjoin%
\definecolor{currentfill}{rgb}{0.487026,0.823929,0.312321}%
\pgfsetfillcolor{currentfill}%
\pgfsetlinewidth{0.000000pt}%
\definecolor{currentstroke}{rgb}{0.000000,0.000000,0.000000}%
\pgfsetstrokecolor{currentstroke}%
\pgfsetdash{}{0pt}%
\pgfpathmoveto{\pgfqpoint{1.180277in}{1.595730in}}%
\pgfpathlineto{\pgfqpoint{1.180284in}{1.591814in}}%
\pgfpathlineto{\pgfqpoint{1.180291in}{1.587810in}}%
\pgfpathlineto{\pgfqpoint{1.180297in}{1.583721in}}%
\pgfpathlineto{\pgfqpoint{1.180304in}{1.579549in}}%
\pgfpathlineto{\pgfqpoint{1.186580in}{1.579497in}}%
\pgfpathlineto{\pgfqpoint{1.192849in}{1.579352in}}%
\pgfpathlineto{\pgfqpoint{1.199107in}{1.579113in}}%
\pgfpathlineto{\pgfqpoint{1.205346in}{1.578782in}}%
\pgfpathlineto{\pgfqpoint{1.204871in}{1.582970in}}%
\pgfpathlineto{\pgfqpoint{1.204396in}{1.587073in}}%
\pgfpathlineto{\pgfqpoint{1.203921in}{1.591091in}}%
\pgfpathlineto{\pgfqpoint{1.203444in}{1.595022in}}%
\pgfpathlineto{\pgfqpoint{1.197673in}{1.595328in}}%
\pgfpathlineto{\pgfqpoint{1.191884in}{1.595548in}}%
\pgfpathlineto{\pgfqpoint{1.186084in}{1.595682in}}%
\pgfpathlineto{\pgfqpoint{1.180277in}{1.595730in}}%
\pgfpathclose%
\pgfusepath{fill}%
\end{pgfscope}%
\begin{pgfscope}%
\pgfpathrectangle{\pgfqpoint{0.041670in}{0.041670in}}{\pgfqpoint{2.216660in}{2.216660in}}%
\pgfusepath{clip}%
\pgfsetbuttcap%
\pgfsetroundjoin%
\definecolor{currentfill}{rgb}{0.412913,0.803041,0.357269}%
\pgfsetfillcolor{currentfill}%
\pgfsetlinewidth{0.000000pt}%
\definecolor{currentstroke}{rgb}{0.000000,0.000000,0.000000}%
\pgfsetstrokecolor{currentstroke}%
\pgfsetdash{}{0pt}%
\pgfpathmoveto{\pgfqpoint{1.062548in}{1.561534in}}%
\pgfpathlineto{\pgfqpoint{1.060351in}{1.556931in}}%
\pgfpathlineto{\pgfqpoint{1.058157in}{1.552247in}}%
\pgfpathlineto{\pgfqpoint{1.055966in}{1.547485in}}%
\pgfpathlineto{\pgfqpoint{1.053778in}{1.542647in}}%
\pgfpathlineto{\pgfqpoint{1.059341in}{1.544493in}}%
\pgfpathlineto{\pgfqpoint{1.065020in}{1.546254in}}%
\pgfpathlineto{\pgfqpoint{1.070808in}{1.547929in}}%
\pgfpathlineto{\pgfqpoint{1.076702in}{1.549516in}}%
\pgfpathlineto{\pgfqpoint{1.078491in}{1.554231in}}%
\pgfpathlineto{\pgfqpoint{1.080282in}{1.558868in}}%
\pgfpathlineto{\pgfqpoint{1.082076in}{1.563428in}}%
\pgfpathlineto{\pgfqpoint{1.083872in}{1.567908in}}%
\pgfpathlineto{\pgfqpoint{1.078389in}{1.566435in}}%
\pgfpathlineto{\pgfqpoint{1.073004in}{1.564881in}}%
\pgfpathlineto{\pgfqpoint{1.067722in}{1.563247in}}%
\pgfpathlineto{\pgfqpoint{1.062548in}{1.561534in}}%
\pgfpathclose%
\pgfusepath{fill}%
\end{pgfscope}%
\begin{pgfscope}%
\pgfpathrectangle{\pgfqpoint{0.041670in}{0.041670in}}{\pgfqpoint{2.216660in}{2.216660in}}%
\pgfusepath{clip}%
\pgfsetbuttcap%
\pgfsetroundjoin%
\definecolor{currentfill}{rgb}{0.282884,0.135920,0.453427}%
\pgfsetfillcolor{currentfill}%
\pgfsetlinewidth{0.000000pt}%
\definecolor{currentstroke}{rgb}{0.000000,0.000000,0.000000}%
\pgfsetstrokecolor{currentstroke}%
\pgfsetdash{}{0pt}%
\pgfpathmoveto{\pgfqpoint{0.539466in}{0.824683in}}%
\pgfpathlineto{\pgfqpoint{0.535568in}{0.830473in}}%
\pgfpathlineto{\pgfqpoint{0.531654in}{0.836631in}}%
\pgfpathlineto{\pgfqpoint{0.527726in}{0.843166in}}%
\pgfpathlineto{\pgfqpoint{0.523782in}{0.850082in}}%
\pgfpathlineto{\pgfqpoint{0.518197in}{0.860974in}}%
\pgfpathlineto{\pgfqpoint{0.513297in}{0.871935in}}%
\pgfpathlineto{\pgfqpoint{0.509084in}{0.882956in}}%
\pgfpathlineto{\pgfqpoint{0.505560in}{0.894023in}}%
\pgfpathlineto{\pgfqpoint{0.509589in}{0.886886in}}%
\pgfpathlineto{\pgfqpoint{0.513602in}{0.880129in}}%
\pgfpathlineto{\pgfqpoint{0.517601in}{0.873746in}}%
\pgfpathlineto{\pgfqpoint{0.521584in}{0.867731in}}%
\pgfpathlineto{\pgfqpoint{0.525049in}{0.856888in}}%
\pgfpathlineto{\pgfqpoint{0.529186in}{0.846091in}}%
\pgfpathlineto{\pgfqpoint{0.533992in}{0.835353in}}%
\pgfpathlineto{\pgfqpoint{0.539466in}{0.824683in}}%
\pgfpathclose%
\pgfusepath{fill}%
\end{pgfscope}%
\begin{pgfscope}%
\pgfpathrectangle{\pgfqpoint{0.041670in}{0.041670in}}{\pgfqpoint{2.216660in}{2.216660in}}%
\pgfusepath{clip}%
\pgfsetbuttcap%
\pgfsetroundjoin%
\definecolor{currentfill}{rgb}{0.231674,0.318106,0.544834}%
\pgfsetfillcolor{currentfill}%
\pgfsetlinewidth{0.000000pt}%
\definecolor{currentstroke}{rgb}{0.000000,0.000000,0.000000}%
\pgfsetstrokecolor{currentstroke}%
\pgfsetdash{}{0pt}%
\pgfpathmoveto{\pgfqpoint{1.587072in}{1.043408in}}%
\pgfpathlineto{\pgfqpoint{1.590720in}{1.036021in}}%
\pgfpathlineto{\pgfqpoint{1.594366in}{1.028698in}}%
\pgfpathlineto{\pgfqpoint{1.598012in}{1.021443in}}%
\pgfpathlineto{\pgfqpoint{1.601658in}{1.014259in}}%
\pgfpathlineto{\pgfqpoint{1.599599in}{1.007455in}}%
\pgfpathlineto{\pgfqpoint{1.597115in}{1.000681in}}%
\pgfpathlineto{\pgfqpoint{1.594207in}{0.993945in}}%
\pgfpathlineto{\pgfqpoint{1.590877in}{0.987252in}}%
\pgfpathlineto{\pgfqpoint{1.587312in}{0.994698in}}%
\pgfpathlineto{\pgfqpoint{1.583746in}{1.002214in}}%
\pgfpathlineto{\pgfqpoint{1.580179in}{1.009798in}}%
\pgfpathlineto{\pgfqpoint{1.576612in}{1.017446in}}%
\pgfpathlineto{\pgfqpoint{1.579839in}{1.023879in}}%
\pgfpathlineto{\pgfqpoint{1.582660in}{1.030355in}}%
\pgfpathlineto{\pgfqpoint{1.585071in}{1.036866in}}%
\pgfpathlineto{\pgfqpoint{1.587072in}{1.043408in}}%
\pgfpathclose%
\pgfusepath{fill}%
\end{pgfscope}%
\begin{pgfscope}%
\pgfpathrectangle{\pgfqpoint{0.041670in}{0.041670in}}{\pgfqpoint{2.216660in}{2.216660in}}%
\pgfusepath{clip}%
\pgfsetbuttcap%
\pgfsetroundjoin%
\definecolor{currentfill}{rgb}{0.220124,0.725509,0.466226}%
\pgfsetfillcolor{currentfill}%
\pgfsetlinewidth{0.000000pt}%
\definecolor{currentstroke}{rgb}{0.000000,0.000000,0.000000}%
\pgfsetstrokecolor{currentstroke}%
\pgfsetdash{}{0pt}%
\pgfpathmoveto{\pgfqpoint{0.957429in}{1.456264in}}%
\pgfpathlineto{\pgfqpoint{0.954053in}{1.450107in}}%
\pgfpathlineto{\pgfqpoint{0.950681in}{1.443895in}}%
\pgfpathlineto{\pgfqpoint{0.947312in}{1.437628in}}%
\pgfpathlineto{\pgfqpoint{0.943947in}{1.431310in}}%
\pgfpathlineto{\pgfqpoint{0.947659in}{1.434872in}}%
\pgfpathlineto{\pgfqpoint{0.951597in}{1.438373in}}%
\pgfpathlineto{\pgfqpoint{0.955756in}{1.441811in}}%
\pgfpathlineto{\pgfqpoint{0.960132in}{1.445183in}}%
\pgfpathlineto{\pgfqpoint{0.963259in}{1.451291in}}%
\pgfpathlineto{\pgfqpoint{0.966391in}{1.457347in}}%
\pgfpathlineto{\pgfqpoint{0.969526in}{1.463350in}}%
\pgfpathlineto{\pgfqpoint{0.972664in}{1.469297in}}%
\pgfpathlineto{\pgfqpoint{0.968544in}{1.466129in}}%
\pgfpathlineto{\pgfqpoint{0.964629in}{1.462899in}}%
\pgfpathlineto{\pgfqpoint{0.960923in}{1.459609in}}%
\pgfpathlineto{\pgfqpoint{0.957429in}{1.456264in}}%
\pgfpathclose%
\pgfusepath{fill}%
\end{pgfscope}%
\begin{pgfscope}%
\pgfpathrectangle{\pgfqpoint{0.041670in}{0.041670in}}{\pgfqpoint{2.216660in}{2.216660in}}%
\pgfusepath{clip}%
\pgfsetbuttcap%
\pgfsetroundjoin%
\definecolor{currentfill}{rgb}{0.281477,0.755203,0.432552}%
\pgfsetfillcolor{currentfill}%
\pgfsetlinewidth{0.000000pt}%
\definecolor{currentstroke}{rgb}{0.000000,0.000000,0.000000}%
\pgfsetstrokecolor{currentstroke}%
\pgfsetdash{}{0pt}%
\pgfpathmoveto{\pgfqpoint{0.985255in}{1.492480in}}%
\pgfpathlineto{\pgfqpoint{0.982102in}{1.486779in}}%
\pgfpathlineto{\pgfqpoint{0.978953in}{1.481013in}}%
\pgfpathlineto{\pgfqpoint{0.975807in}{1.475185in}}%
\pgfpathlineto{\pgfqpoint{0.972664in}{1.469297in}}%
\pgfpathlineto{\pgfqpoint{0.976985in}{1.472399in}}%
\pgfpathlineto{\pgfqpoint{0.981503in}{1.475433in}}%
\pgfpathlineto{\pgfqpoint{0.986212in}{1.478396in}}%
\pgfpathlineto{\pgfqpoint{0.991107in}{1.481286in}}%
\pgfpathlineto{\pgfqpoint{0.993965in}{1.486982in}}%
\pgfpathlineto{\pgfqpoint{0.996826in}{1.492619in}}%
\pgfpathlineto{\pgfqpoint{0.999690in}{1.498193in}}%
\pgfpathlineto{\pgfqpoint{1.002558in}{1.503703in}}%
\pgfpathlineto{\pgfqpoint{0.997964in}{1.500998in}}%
\pgfpathlineto{\pgfqpoint{0.993546in}{1.498224in}}%
\pgfpathlineto{\pgfqpoint{0.989308in}{1.495384in}}%
\pgfpathlineto{\pgfqpoint{0.985255in}{1.492480in}}%
\pgfpathclose%
\pgfusepath{fill}%
\end{pgfscope}%
\begin{pgfscope}%
\pgfpathrectangle{\pgfqpoint{0.041670in}{0.041670in}}{\pgfqpoint{2.216660in}{2.216660in}}%
\pgfusepath{clip}%
\pgfsetbuttcap%
\pgfsetroundjoin%
\definecolor{currentfill}{rgb}{0.344074,0.780029,0.397381}%
\pgfsetfillcolor{currentfill}%
\pgfsetlinewidth{0.000000pt}%
\definecolor{currentstroke}{rgb}{0.000000,0.000000,0.000000}%
\pgfsetstrokecolor{currentstroke}%
\pgfsetdash{}{0pt}%
\pgfpathmoveto{\pgfqpoint{1.322642in}{1.536383in}}%
\pgfpathlineto{\pgfqpoint{1.325116in}{1.531357in}}%
\pgfpathlineto{\pgfqpoint{1.327587in}{1.526258in}}%
\pgfpathlineto{\pgfqpoint{1.330055in}{1.521090in}}%
\pgfpathlineto{\pgfqpoint{1.332520in}{1.515852in}}%
\pgfpathlineto{\pgfqpoint{1.337904in}{1.513520in}}%
\pgfpathlineto{\pgfqpoint{1.343136in}{1.511107in}}%
\pgfpathlineto{\pgfqpoint{1.348211in}{1.508615in}}%
\pgfpathlineto{\pgfqpoint{1.353124in}{1.506047in}}%
\pgfpathlineto{\pgfqpoint{1.350322in}{1.511451in}}%
\pgfpathlineto{\pgfqpoint{1.347517in}{1.516786in}}%
\pgfpathlineto{\pgfqpoint{1.344709in}{1.522050in}}%
\pgfpathlineto{\pgfqpoint{1.341897in}{1.527242in}}%
\pgfpathlineto{\pgfqpoint{1.337306in}{1.529635in}}%
\pgfpathlineto{\pgfqpoint{1.332563in}{1.531958in}}%
\pgfpathlineto{\pgfqpoint{1.327674in}{1.534208in}}%
\pgfpathlineto{\pgfqpoint{1.322642in}{1.536383in}}%
\pgfpathclose%
\pgfusepath{fill}%
\end{pgfscope}%
\begin{pgfscope}%
\pgfpathrectangle{\pgfqpoint{0.041670in}{0.041670in}}{\pgfqpoint{2.216660in}{2.216660in}}%
\pgfusepath{clip}%
\pgfsetbuttcap%
\pgfsetroundjoin%
\definecolor{currentfill}{rgb}{0.179019,0.433756,0.557430}%
\pgfsetfillcolor{currentfill}%
\pgfsetlinewidth{0.000000pt}%
\definecolor{currentstroke}{rgb}{0.000000,0.000000,0.000000}%
\pgfsetstrokecolor{currentstroke}%
\pgfsetdash{}{0pt}%
\pgfpathmoveto{\pgfqpoint{1.546808in}{1.159105in}}%
\pgfpathlineto{\pgfqpoint{1.550499in}{1.151454in}}%
\pgfpathlineto{\pgfqpoint{1.554187in}{1.143829in}}%
\pgfpathlineto{\pgfqpoint{1.557873in}{1.136234in}}%
\pgfpathlineto{\pgfqpoint{1.561556in}{1.128672in}}%
\pgfpathlineto{\pgfqpoint{1.561201in}{1.122603in}}%
\pgfpathlineto{\pgfqpoint{1.560464in}{1.116537in}}%
\pgfpathlineto{\pgfqpoint{1.559345in}{1.110481in}}%
\pgfpathlineto{\pgfqpoint{1.557844in}{1.104440in}}%
\pgfpathlineto{\pgfqpoint{1.554183in}{1.112263in}}%
\pgfpathlineto{\pgfqpoint{1.550521in}{1.120118in}}%
\pgfpathlineto{\pgfqpoint{1.546857in}{1.128003in}}%
\pgfpathlineto{\pgfqpoint{1.543191in}{1.135914in}}%
\pgfpathlineto{\pgfqpoint{1.544646in}{1.141695in}}%
\pgfpathlineto{\pgfqpoint{1.545734in}{1.147491in}}%
\pgfpathlineto{\pgfqpoint{1.546455in}{1.153296in}}%
\pgfpathlineto{\pgfqpoint{1.546808in}{1.159105in}}%
\pgfpathclose%
\pgfusepath{fill}%
\end{pgfscope}%
\begin{pgfscope}%
\pgfpathrectangle{\pgfqpoint{0.041670in}{0.041670in}}{\pgfqpoint{2.216660in}{2.216660in}}%
\pgfusepath{clip}%
\pgfsetbuttcap%
\pgfsetroundjoin%
\definecolor{currentfill}{rgb}{0.487026,0.823929,0.312321}%
\pgfsetfillcolor{currentfill}%
\pgfsetlinewidth{0.000000pt}%
\definecolor{currentstroke}{rgb}{0.000000,0.000000,0.000000}%
\pgfsetstrokecolor{currentstroke}%
\pgfsetdash{}{0pt}%
\pgfpathmoveto{\pgfqpoint{1.134283in}{1.593024in}}%
\pgfpathlineto{\pgfqpoint{1.133357in}{1.589051in}}%
\pgfpathlineto{\pgfqpoint{1.132432in}{1.584991in}}%
\pgfpathlineto{\pgfqpoint{1.131508in}{1.580846in}}%
\pgfpathlineto{\pgfqpoint{1.130585in}{1.576617in}}%
\pgfpathlineto{\pgfqpoint{1.136697in}{1.577306in}}%
\pgfpathlineto{\pgfqpoint{1.142850in}{1.577904in}}%
\pgfpathlineto{\pgfqpoint{1.149038in}{1.578410in}}%
\pgfpathlineto{\pgfqpoint{1.155256in}{1.578824in}}%
\pgfpathlineto{\pgfqpoint{1.155718in}{1.583010in}}%
\pgfpathlineto{\pgfqpoint{1.156180in}{1.587113in}}%
\pgfpathlineto{\pgfqpoint{1.156642in}{1.591130in}}%
\pgfpathlineto{\pgfqpoint{1.157106in}{1.595061in}}%
\pgfpathlineto{\pgfqpoint{1.151354in}{1.594679in}}%
\pgfpathlineto{\pgfqpoint{1.145629in}{1.594212in}}%
\pgfpathlineto{\pgfqpoint{1.139937in}{1.593660in}}%
\pgfpathlineto{\pgfqpoint{1.134283in}{1.593024in}}%
\pgfpathclose%
\pgfusepath{fill}%
\end{pgfscope}%
\begin{pgfscope}%
\pgfpathrectangle{\pgfqpoint{0.041670in}{0.041670in}}{\pgfqpoint{2.216660in}{2.216660in}}%
\pgfusepath{clip}%
\pgfsetbuttcap%
\pgfsetroundjoin%
\definecolor{currentfill}{rgb}{0.487026,0.823929,0.312321}%
\pgfsetfillcolor{currentfill}%
\pgfsetlinewidth{0.000000pt}%
\definecolor{currentstroke}{rgb}{0.000000,0.000000,0.000000}%
\pgfsetstrokecolor{currentstroke}%
\pgfsetdash{}{0pt}%
\pgfpathmoveto{\pgfqpoint{1.203444in}{1.595022in}}%
\pgfpathlineto{\pgfqpoint{1.203921in}{1.591091in}}%
\pgfpathlineto{\pgfqpoint{1.204396in}{1.587073in}}%
\pgfpathlineto{\pgfqpoint{1.204871in}{1.582970in}}%
\pgfpathlineto{\pgfqpoint{1.205346in}{1.578782in}}%
\pgfpathlineto{\pgfqpoint{1.211561in}{1.578358in}}%
\pgfpathlineto{\pgfqpoint{1.217745in}{1.577842in}}%
\pgfpathlineto{\pgfqpoint{1.223894in}{1.577234in}}%
\pgfpathlineto{\pgfqpoint{1.230001in}{1.576534in}}%
\pgfpathlineto{\pgfqpoint{1.229065in}{1.580765in}}%
\pgfpathlineto{\pgfqpoint{1.228129in}{1.584912in}}%
\pgfpathlineto{\pgfqpoint{1.227191in}{1.588974in}}%
\pgfpathlineto{\pgfqpoint{1.226252in}{1.592948in}}%
\pgfpathlineto{\pgfqpoint{1.220603in}{1.593593in}}%
\pgfpathlineto{\pgfqpoint{1.214915in}{1.594155in}}%
\pgfpathlineto{\pgfqpoint{1.209194in}{1.594631in}}%
\pgfpathlineto{\pgfqpoint{1.203444in}{1.595022in}}%
\pgfpathclose%
\pgfusepath{fill}%
\end{pgfscope}%
\begin{pgfscope}%
\pgfpathrectangle{\pgfqpoint{0.041670in}{0.041670in}}{\pgfqpoint{2.216660in}{2.216660in}}%
\pgfusepath{clip}%
\pgfsetbuttcap%
\pgfsetroundjoin%
\definecolor{currentfill}{rgb}{0.282327,0.094955,0.417331}%
\pgfsetfillcolor{currentfill}%
\pgfsetlinewidth{0.000000pt}%
\definecolor{currentstroke}{rgb}{0.000000,0.000000,0.000000}%
\pgfsetstrokecolor{currentstroke}%
\pgfsetdash{}{0pt}%
\pgfpathmoveto{\pgfqpoint{0.723423in}{0.820573in}}%
\pgfpathlineto{\pgfqpoint{0.720012in}{0.815129in}}%
\pgfpathlineto{\pgfqpoint{0.716598in}{0.809837in}}%
\pgfpathlineto{\pgfqpoint{0.713182in}{0.804700in}}%
\pgfpathlineto{\pgfqpoint{0.709763in}{0.799724in}}%
\pgfpathlineto{\pgfqpoint{0.703375in}{0.807604in}}%
\pgfpathlineto{\pgfqpoint{0.697484in}{0.815578in}}%
\pgfpathlineto{\pgfqpoint{0.692092in}{0.823637in}}%
\pgfpathlineto{\pgfqpoint{0.687205in}{0.831772in}}%
\pgfpathlineto{\pgfqpoint{0.690773in}{0.836495in}}%
\pgfpathlineto{\pgfqpoint{0.694338in}{0.841378in}}%
\pgfpathlineto{\pgfqpoint{0.697900in}{0.846416in}}%
\pgfpathlineto{\pgfqpoint{0.701460in}{0.851606in}}%
\pgfpathlineto{\pgfqpoint{0.706221in}{0.843728in}}%
\pgfpathlineto{\pgfqpoint{0.711471in}{0.835924in}}%
\pgfpathlineto{\pgfqpoint{0.717207in}{0.828203in}}%
\pgfpathlineto{\pgfqpoint{0.723423in}{0.820573in}}%
\pgfpathclose%
\pgfusepath{fill}%
\end{pgfscope}%
\begin{pgfscope}%
\pgfpathrectangle{\pgfqpoint{0.041670in}{0.041670in}}{\pgfqpoint{2.216660in}{2.216660in}}%
\pgfusepath{clip}%
\pgfsetbuttcap%
\pgfsetroundjoin%
\definecolor{currentfill}{rgb}{0.267004,0.004874,0.329415}%
\pgfsetfillcolor{currentfill}%
\pgfsetlinewidth{0.000000pt}%
\definecolor{currentstroke}{rgb}{0.000000,0.000000,0.000000}%
\pgfsetstrokecolor{currentstroke}%
\pgfsetdash{}{0pt}%
\pgfpathmoveto{\pgfqpoint{0.640429in}{0.740283in}}%
\pgfpathlineto{\pgfqpoint{0.636899in}{0.739660in}}%
\pgfpathlineto{\pgfqpoint{0.633360in}{0.739296in}}%
\pgfpathlineto{\pgfqpoint{0.629813in}{0.739195in}}%
\pgfpathlineto{\pgfqpoint{0.626258in}{0.739363in}}%
\pgfpathlineto{\pgfqpoint{0.618844in}{0.748718in}}%
\pgfpathlineto{\pgfqpoint{0.612018in}{0.758180in}}%
\pgfpathlineto{\pgfqpoint{0.605784in}{0.767740in}}%
\pgfpathlineto{\pgfqpoint{0.600147in}{0.777386in}}%
\pgfpathlineto{\pgfqpoint{0.603850in}{0.776976in}}%
\pgfpathlineto{\pgfqpoint{0.607546in}{0.776834in}}%
\pgfpathlineto{\pgfqpoint{0.611233in}{0.776955in}}%
\pgfpathlineto{\pgfqpoint{0.614912in}{0.777333in}}%
\pgfpathlineto{\pgfqpoint{0.620424in}{0.767933in}}%
\pgfpathlineto{\pgfqpoint{0.626517in}{0.758618in}}%
\pgfpathlineto{\pgfqpoint{0.633187in}{0.749398in}}%
\pgfpathlineto{\pgfqpoint{0.640429in}{0.740283in}}%
\pgfpathclose%
\pgfusepath{fill}%
\end{pgfscope}%
\begin{pgfscope}%
\pgfpathrectangle{\pgfqpoint{0.041670in}{0.041670in}}{\pgfqpoint{2.216660in}{2.216660in}}%
\pgfusepath{clip}%
\pgfsetbuttcap%
\pgfsetroundjoin%
\definecolor{currentfill}{rgb}{0.280255,0.165693,0.476498}%
\pgfsetfillcolor{currentfill}%
\pgfsetlinewidth{0.000000pt}%
\definecolor{currentstroke}{rgb}{0.000000,0.000000,0.000000}%
\pgfsetstrokecolor{currentstroke}%
\pgfsetdash{}{0pt}%
\pgfpathmoveto{\pgfqpoint{1.633666in}{0.904708in}}%
\pgfpathlineto{\pgfqpoint{1.637236in}{0.898504in}}%
\pgfpathlineto{\pgfqpoint{1.640807in}{0.892420in}}%
\pgfpathlineto{\pgfqpoint{1.644380in}{0.886460in}}%
\pgfpathlineto{\pgfqpoint{1.647954in}{0.880629in}}%
\pgfpathlineto{\pgfqpoint{1.643743in}{0.872951in}}%
\pgfpathlineto{\pgfqpoint{1.639056in}{0.865339in}}%
\pgfpathlineto{\pgfqpoint{1.633895in}{0.857799in}}%
\pgfpathlineto{\pgfqpoint{1.628264in}{0.850341in}}%
\pgfpathlineto{\pgfqpoint{1.624826in}{0.856429in}}%
\pgfpathlineto{\pgfqpoint{1.621389in}{0.862645in}}%
\pgfpathlineto{\pgfqpoint{1.617955in}{0.868985in}}%
\pgfpathlineto{\pgfqpoint{1.614522in}{0.875446in}}%
\pgfpathlineto{\pgfqpoint{1.619994in}{0.882651in}}%
\pgfpathlineto{\pgfqpoint{1.625011in}{0.889935in}}%
\pgfpathlineto{\pgfqpoint{1.629570in}{0.897290in}}%
\pgfpathlineto{\pgfqpoint{1.633666in}{0.904708in}}%
\pgfpathclose%
\pgfusepath{fill}%
\end{pgfscope}%
\begin{pgfscope}%
\pgfpathrectangle{\pgfqpoint{0.041670in}{0.041670in}}{\pgfqpoint{2.216660in}{2.216660in}}%
\pgfusepath{clip}%
\pgfsetbuttcap%
\pgfsetroundjoin%
\definecolor{currentfill}{rgb}{0.147607,0.511733,0.557049}%
\pgfsetfillcolor{currentfill}%
\pgfsetlinewidth{0.000000pt}%
\definecolor{currentstroke}{rgb}{0.000000,0.000000,0.000000}%
\pgfsetstrokecolor{currentstroke}%
\pgfsetdash{}{0pt}%
\pgfpathmoveto{\pgfqpoint{0.843003in}{1.216222in}}%
\pgfpathlineto{\pgfqpoint{0.839294in}{1.208406in}}%
\pgfpathlineto{\pgfqpoint{0.835587in}{1.200592in}}%
\pgfpathlineto{\pgfqpoint{0.831883in}{1.192785in}}%
\pgfpathlineto{\pgfqpoint{0.828181in}{1.184986in}}%
\pgfpathlineto{\pgfqpoint{0.827868in}{1.190536in}}%
\pgfpathlineto{\pgfqpoint{0.827907in}{1.196084in}}%
\pgfpathlineto{\pgfqpoint{0.828297in}{1.201624in}}%
\pgfpathlineto{\pgfqpoint{0.829036in}{1.207151in}}%
\pgfpathlineto{\pgfqpoint{0.832717in}{1.214692in}}%
\pgfpathlineto{\pgfqpoint{0.836401in}{1.222242in}}%
\pgfpathlineto{\pgfqpoint{0.840088in}{1.229799in}}%
\pgfpathlineto{\pgfqpoint{0.843778in}{1.237360in}}%
\pgfpathlineto{\pgfqpoint{0.843081in}{1.232089in}}%
\pgfpathlineto{\pgfqpoint{0.842719in}{1.226805in}}%
\pgfpathlineto{\pgfqpoint{0.842693in}{1.221515in}}%
\pgfpathlineto{\pgfqpoint{0.843003in}{1.216222in}}%
\pgfpathclose%
\pgfusepath{fill}%
\end{pgfscope}%
\begin{pgfscope}%
\pgfpathrectangle{\pgfqpoint{0.041670in}{0.041670in}}{\pgfqpoint{2.216660in}{2.216660in}}%
\pgfusepath{clip}%
\pgfsetbuttcap%
\pgfsetroundjoin%
\definecolor{currentfill}{rgb}{0.166383,0.690856,0.496502}%
\pgfsetfillcolor{currentfill}%
\pgfsetlinewidth{0.000000pt}%
\definecolor{currentstroke}{rgb}{0.000000,0.000000,0.000000}%
\pgfsetstrokecolor{currentstroke}%
\pgfsetdash{}{0pt}%
\pgfpathmoveto{\pgfqpoint{0.931431in}{1.416532in}}%
\pgfpathlineto{\pgfqpoint{0.927884in}{1.409939in}}%
\pgfpathlineto{\pgfqpoint{0.924340in}{1.403299in}}%
\pgfpathlineto{\pgfqpoint{0.920800in}{1.396614in}}%
\pgfpathlineto{\pgfqpoint{0.917263in}{1.389886in}}%
\pgfpathlineto{\pgfqpoint{0.920202in}{1.393884in}}%
\pgfpathlineto{\pgfqpoint{0.923394in}{1.397832in}}%
\pgfpathlineto{\pgfqpoint{0.926836in}{1.401728in}}%
\pgfpathlineto{\pgfqpoint{0.930524in}{1.405567in}}%
\pgfpathlineto{\pgfqpoint{0.933875in}{1.412069in}}%
\pgfpathlineto{\pgfqpoint{0.937229in}{1.418528in}}%
\pgfpathlineto{\pgfqpoint{0.940586in}{1.424943in}}%
\pgfpathlineto{\pgfqpoint{0.943947in}{1.431310in}}%
\pgfpathlineto{\pgfqpoint{0.940465in}{1.427692in}}%
\pgfpathlineto{\pgfqpoint{0.937216in}{1.424020in}}%
\pgfpathlineto{\pgfqpoint{0.934203in}{1.420299in}}%
\pgfpathlineto{\pgfqpoint{0.931431in}{1.416532in}}%
\pgfpathclose%
\pgfusepath{fill}%
\end{pgfscope}%
\begin{pgfscope}%
\pgfpathrectangle{\pgfqpoint{0.041670in}{0.041670in}}{\pgfqpoint{2.216660in}{2.216660in}}%
\pgfusepath{clip}%
\pgfsetbuttcap%
\pgfsetroundjoin%
\definecolor{currentfill}{rgb}{0.276194,0.190074,0.493001}%
\pgfsetfillcolor{currentfill}%
\pgfsetlinewidth{0.000000pt}%
\definecolor{currentstroke}{rgb}{0.000000,0.000000,0.000000}%
\pgfsetstrokecolor{currentstroke}%
\pgfsetdash{}{0pt}%
\pgfpathmoveto{\pgfqpoint{1.856905in}{0.903890in}}%
\pgfpathlineto{\pgfqpoint{1.860960in}{0.911463in}}%
\pgfpathlineto{\pgfqpoint{1.865031in}{0.919428in}}%
\pgfpathlineto{\pgfqpoint{1.869120in}{0.927793in}}%
\pgfpathlineto{\pgfqpoint{1.873225in}{0.936564in}}%
\pgfpathlineto{\pgfqpoint{1.870268in}{0.925248in}}%
\pgfpathlineto{\pgfqpoint{1.866605in}{0.913969in}}%
\pgfpathlineto{\pgfqpoint{1.862237in}{0.902739in}}%
\pgfpathlineto{\pgfqpoint{1.857165in}{0.891568in}}%
\pgfpathlineto{\pgfqpoint{1.853130in}{0.883010in}}%
\pgfpathlineto{\pgfqpoint{1.849112in}{0.874861in}}%
\pgfpathlineto{\pgfqpoint{1.845111in}{0.867113in}}%
\pgfpathlineto{\pgfqpoint{1.841126in}{0.859760in}}%
\pgfpathlineto{\pgfqpoint{1.846102in}{0.870714in}}%
\pgfpathlineto{\pgfqpoint{1.850391in}{0.881729in}}%
\pgfpathlineto{\pgfqpoint{1.853992in}{0.892791in}}%
\pgfpathlineto{\pgfqpoint{1.856905in}{0.903890in}}%
\pgfpathclose%
\pgfusepath{fill}%
\end{pgfscope}%
\begin{pgfscope}%
\pgfpathrectangle{\pgfqpoint{0.041670in}{0.041670in}}{\pgfqpoint{2.216660in}{2.216660in}}%
\pgfusepath{clip}%
\pgfsetbuttcap%
\pgfsetroundjoin%
\definecolor{currentfill}{rgb}{0.344074,0.780029,0.397381}%
\pgfsetfillcolor{currentfill}%
\pgfsetlinewidth{0.000000pt}%
\definecolor{currentstroke}{rgb}{0.000000,0.000000,0.000000}%
\pgfsetstrokecolor{currentstroke}%
\pgfsetdash{}{0pt}%
\pgfpathmoveto{\pgfqpoint{1.014064in}{1.525057in}}%
\pgfpathlineto{\pgfqpoint{1.011182in}{1.519825in}}%
\pgfpathlineto{\pgfqpoint{1.008304in}{1.514521in}}%
\pgfpathlineto{\pgfqpoint{1.005429in}{1.509146in}}%
\pgfpathlineto{\pgfqpoint{1.002558in}{1.503703in}}%
\pgfpathlineto{\pgfqpoint{1.007323in}{1.506336in}}%
\pgfpathlineto{\pgfqpoint{1.012255in}{1.508896in}}%
\pgfpathlineto{\pgfqpoint{1.017348in}{1.511379in}}%
\pgfpathlineto{\pgfqpoint{1.022597in}{1.513783in}}%
\pgfpathlineto{\pgfqpoint{1.025140in}{1.519055in}}%
\pgfpathlineto{\pgfqpoint{1.027686in}{1.524259in}}%
\pgfpathlineto{\pgfqpoint{1.030236in}{1.529393in}}%
\pgfpathlineto{\pgfqpoint{1.032788in}{1.534453in}}%
\pgfpathlineto{\pgfqpoint{1.027882in}{1.532212in}}%
\pgfpathlineto{\pgfqpoint{1.023123in}{1.529897in}}%
\pgfpathlineto{\pgfqpoint{1.018515in}{1.527511in}}%
\pgfpathlineto{\pgfqpoint{1.014064in}{1.525057in}}%
\pgfpathclose%
\pgfusepath{fill}%
\end{pgfscope}%
\begin{pgfscope}%
\pgfpathrectangle{\pgfqpoint{0.041670in}{0.041670in}}{\pgfqpoint{2.216660in}{2.216660in}}%
\pgfusepath{clip}%
\pgfsetbuttcap%
\pgfsetroundjoin%
\definecolor{currentfill}{rgb}{0.122606,0.585371,0.546557}%
\pgfsetfillcolor{currentfill}%
\pgfsetlinewidth{0.000000pt}%
\definecolor{currentstroke}{rgb}{0.000000,0.000000,0.000000}%
\pgfsetstrokecolor{currentstroke}%
\pgfsetdash{}{0pt}%
\pgfpathmoveto{\pgfqpoint{0.873402in}{1.297643in}}%
\pgfpathlineto{\pgfqpoint{0.869688in}{1.290153in}}%
\pgfpathlineto{\pgfqpoint{0.865978in}{1.282644in}}%
\pgfpathlineto{\pgfqpoint{0.862270in}{1.275120in}}%
\pgfpathlineto{\pgfqpoint{0.858565in}{1.267583in}}%
\pgfpathlineto{\pgfqpoint{0.859539in}{1.272583in}}%
\pgfpathlineto{\pgfqpoint{0.860830in}{1.277562in}}%
\pgfpathlineto{\pgfqpoint{0.862437in}{1.282515in}}%
\pgfpathlineto{\pgfqpoint{0.864356in}{1.287437in}}%
\pgfpathlineto{\pgfqpoint{0.867983in}{1.294725in}}%
\pgfpathlineto{\pgfqpoint{0.871614in}{1.301999in}}%
\pgfpathlineto{\pgfqpoint{0.875248in}{1.309259in}}%
\pgfpathlineto{\pgfqpoint{0.878885in}{1.316500in}}%
\pgfpathlineto{\pgfqpoint{0.877064in}{1.311825in}}%
\pgfpathlineto{\pgfqpoint{0.875542in}{1.307121in}}%
\pgfpathlineto{\pgfqpoint{0.874321in}{1.302392in}}%
\pgfpathlineto{\pgfqpoint{0.873402in}{1.297643in}}%
\pgfpathclose%
\pgfusepath{fill}%
\end{pgfscope}%
\begin{pgfscope}%
\pgfpathrectangle{\pgfqpoint{0.041670in}{0.041670in}}{\pgfqpoint{2.216660in}{2.216660in}}%
\pgfusepath{clip}%
\pgfsetbuttcap%
\pgfsetroundjoin%
\definecolor{currentfill}{rgb}{0.134692,0.658636,0.517649}%
\pgfsetfillcolor{currentfill}%
\pgfsetlinewidth{0.000000pt}%
\definecolor{currentstroke}{rgb}{0.000000,0.000000,0.000000}%
\pgfsetstrokecolor{currentstroke}%
\pgfsetdash{}{0pt}%
\pgfpathmoveto{\pgfqpoint{1.440047in}{1.393442in}}%
\pgfpathlineto{\pgfqpoint{1.443543in}{1.386725in}}%
\pgfpathlineto{\pgfqpoint{1.447036in}{1.379971in}}%
\pgfpathlineto{\pgfqpoint{1.450524in}{1.373181in}}%
\pgfpathlineto{\pgfqpoint{1.454009in}{1.366358in}}%
\pgfpathlineto{\pgfqpoint{1.457085in}{1.362123in}}%
\pgfpathlineto{\pgfqpoint{1.459889in}{1.357840in}}%
\pgfpathlineto{\pgfqpoint{1.462420in}{1.353513in}}%
\pgfpathlineto{\pgfqpoint{1.464673in}{1.349147in}}%
\pgfpathlineto{\pgfqpoint{1.461043in}{1.356206in}}%
\pgfpathlineto{\pgfqpoint{1.457410in}{1.363232in}}%
\pgfpathlineto{\pgfqpoint{1.453773in}{1.370223in}}%
\pgfpathlineto{\pgfqpoint{1.450132in}{1.377175in}}%
\pgfpathlineto{\pgfqpoint{1.448003in}{1.381301in}}%
\pgfpathlineto{\pgfqpoint{1.445610in}{1.385391in}}%
\pgfpathlineto{\pgfqpoint{1.442957in}{1.389439in}}%
\pgfpathlineto{\pgfqpoint{1.440047in}{1.393442in}}%
\pgfpathclose%
\pgfusepath{fill}%
\end{pgfscope}%
\begin{pgfscope}%
\pgfpathrectangle{\pgfqpoint{0.041670in}{0.041670in}}{\pgfqpoint{2.216660in}{2.216660in}}%
\pgfusepath{clip}%
\pgfsetbuttcap%
\pgfsetroundjoin%
\definecolor{currentfill}{rgb}{0.487026,0.823929,0.312321}%
\pgfsetfillcolor{currentfill}%
\pgfsetlinewidth{0.000000pt}%
\definecolor{currentstroke}{rgb}{0.000000,0.000000,0.000000}%
\pgfsetstrokecolor{currentstroke}%
\pgfsetdash{}{0pt}%
\pgfpathmoveto{\pgfqpoint{1.226252in}{1.592948in}}%
\pgfpathlineto{\pgfqpoint{1.227191in}{1.588974in}}%
\pgfpathlineto{\pgfqpoint{1.228129in}{1.584912in}}%
\pgfpathlineto{\pgfqpoint{1.229065in}{1.580765in}}%
\pgfpathlineto{\pgfqpoint{1.230001in}{1.576534in}}%
\pgfpathlineto{\pgfqpoint{1.236060in}{1.575744in}}%
\pgfpathlineto{\pgfqpoint{1.242065in}{1.574865in}}%
\pgfpathlineto{\pgfqpoint{1.248011in}{1.573896in}}%
\pgfpathlineto{\pgfqpoint{1.246738in}{1.578177in}}%
\pgfpathlineto{\pgfqpoint{1.245464in}{1.582376in}}%
\pgfpathlineto{\pgfqpoint{1.244188in}{1.586488in}}%
\pgfpathlineto{\pgfqpoint{1.242911in}{1.590513in}}%
\pgfpathlineto{\pgfqpoint{1.237411in}{1.591407in}}%
\pgfpathlineto{\pgfqpoint{1.231856in}{1.592219in}}%
\pgfpathlineto{\pgfqpoint{1.226252in}{1.592948in}}%
\pgfpathclose%
\pgfusepath{fill}%
\end{pgfscope}%
\begin{pgfscope}%
\pgfpathrectangle{\pgfqpoint{0.041670in}{0.041670in}}{\pgfqpoint{2.216660in}{2.216660in}}%
\pgfusepath{clip}%
\pgfsetbuttcap%
\pgfsetroundjoin%
\definecolor{currentfill}{rgb}{0.272594,0.025563,0.353093}%
\pgfsetfillcolor{currentfill}%
\pgfsetlinewidth{0.000000pt}%
\definecolor{currentstroke}{rgb}{0.000000,0.000000,0.000000}%
\pgfsetstrokecolor{currentstroke}%
\pgfsetdash{}{0pt}%
\pgfpathmoveto{\pgfqpoint{1.779275in}{0.790660in}}%
\pgfpathlineto{\pgfqpoint{1.783050in}{0.792539in}}%
\pgfpathlineto{\pgfqpoint{1.786836in}{0.794717in}}%
\pgfpathlineto{\pgfqpoint{1.790633in}{0.797199in}}%
\pgfpathlineto{\pgfqpoint{1.794441in}{0.799991in}}%
\pgfpathlineto{\pgfqpoint{1.789115in}{0.789794in}}%
\pgfpathlineto{\pgfqpoint{1.783157in}{0.779677in}}%
\pgfpathlineto{\pgfqpoint{1.776572in}{0.769650in}}%
\pgfpathlineto{\pgfqpoint{1.769362in}{0.759726in}}%
\pgfpathlineto{\pgfqpoint{1.765689in}{0.757169in}}%
\pgfpathlineto{\pgfqpoint{1.762028in}{0.754924in}}%
\pgfpathlineto{\pgfqpoint{1.758377in}{0.752983in}}%
\pgfpathlineto{\pgfqpoint{1.754737in}{0.751343in}}%
\pgfpathlineto{\pgfqpoint{1.761787in}{0.761033in}}%
\pgfpathlineto{\pgfqpoint{1.768230in}{0.770823in}}%
\pgfpathlineto{\pgfqpoint{1.774060in}{0.780702in}}%
\pgfpathlineto{\pgfqpoint{1.779275in}{0.790660in}}%
\pgfpathclose%
\pgfusepath{fill}%
\end{pgfscope}%
\begin{pgfscope}%
\pgfpathrectangle{\pgfqpoint{0.041670in}{0.041670in}}{\pgfqpoint{2.216660in}{2.216660in}}%
\pgfusepath{clip}%
\pgfsetbuttcap%
\pgfsetroundjoin%
\definecolor{currentfill}{rgb}{0.487026,0.823929,0.312321}%
\pgfsetfillcolor{currentfill}%
\pgfsetlinewidth{0.000000pt}%
\definecolor{currentstroke}{rgb}{0.000000,0.000000,0.000000}%
\pgfsetstrokecolor{currentstroke}%
\pgfsetdash{}{0pt}%
\pgfpathmoveto{\pgfqpoint{1.112161in}{1.589650in}}%
\pgfpathlineto{\pgfqpoint{1.110785in}{1.585607in}}%
\pgfpathlineto{\pgfqpoint{1.109411in}{1.581476in}}%
\pgfpathlineto{\pgfqpoint{1.108039in}{1.577260in}}%
\pgfpathlineto{\pgfqpoint{1.106668in}{1.572960in}}%
\pgfpathlineto{\pgfqpoint{1.112556in}{1.574008in}}%
\pgfpathlineto{\pgfqpoint{1.118509in}{1.574967in}}%
\pgfpathlineto{\pgfqpoint{1.124521in}{1.575837in}}%
\pgfpathlineto{\pgfqpoint{1.130585in}{1.576617in}}%
\pgfpathlineto{\pgfqpoint{1.131508in}{1.580846in}}%
\pgfpathlineto{\pgfqpoint{1.132432in}{1.584991in}}%
\pgfpathlineto{\pgfqpoint{1.133357in}{1.589051in}}%
\pgfpathlineto{\pgfqpoint{1.134283in}{1.593024in}}%
\pgfpathlineto{\pgfqpoint{1.128674in}{1.592304in}}%
\pgfpathlineto{\pgfqpoint{1.123113in}{1.591502in}}%
\pgfpathlineto{\pgfqpoint{1.117607in}{1.590617in}}%
\pgfpathlineto{\pgfqpoint{1.112161in}{1.589650in}}%
\pgfpathclose%
\pgfusepath{fill}%
\end{pgfscope}%
\begin{pgfscope}%
\pgfpathrectangle{\pgfqpoint{0.041670in}{0.041670in}}{\pgfqpoint{2.216660in}{2.216660in}}%
\pgfusepath{clip}%
\pgfsetbuttcap%
\pgfsetroundjoin%
\definecolor{currentfill}{rgb}{0.283072,0.130895,0.449241}%
\pgfsetfillcolor{currentfill}%
\pgfsetlinewidth{0.000000pt}%
\definecolor{currentstroke}{rgb}{0.000000,0.000000,0.000000}%
\pgfsetstrokecolor{currentstroke}%
\pgfsetdash{}{0pt}%
\pgfpathmoveto{\pgfqpoint{0.737042in}{0.843787in}}%
\pgfpathlineto{\pgfqpoint{0.733640in}{0.837775in}}%
\pgfpathlineto{\pgfqpoint{0.730237in}{0.831900in}}%
\pgfpathlineto{\pgfqpoint{0.726831in}{0.826165in}}%
\pgfpathlineto{\pgfqpoint{0.723423in}{0.820573in}}%
\pgfpathlineto{\pgfqpoint{0.717207in}{0.828203in}}%
\pgfpathlineto{\pgfqpoint{0.711471in}{0.835924in}}%
\pgfpathlineto{\pgfqpoint{0.706221in}{0.843728in}}%
\pgfpathlineto{\pgfqpoint{0.701460in}{0.851606in}}%
\pgfpathlineto{\pgfqpoint{0.705017in}{0.856943in}}%
\pgfpathlineto{\pgfqpoint{0.708572in}{0.862424in}}%
\pgfpathlineto{\pgfqpoint{0.712124in}{0.868045in}}%
\pgfpathlineto{\pgfqpoint{0.715675in}{0.873801in}}%
\pgfpathlineto{\pgfqpoint{0.720310in}{0.866181in}}%
\pgfpathlineto{\pgfqpoint{0.725418in}{0.858633in}}%
\pgfpathlineto{\pgfqpoint{0.730997in}{0.851166in}}%
\pgfpathlineto{\pgfqpoint{0.737042in}{0.843787in}}%
\pgfpathclose%
\pgfusepath{fill}%
\end{pgfscope}%
\begin{pgfscope}%
\pgfpathrectangle{\pgfqpoint{0.041670in}{0.041670in}}{\pgfqpoint{2.216660in}{2.216660in}}%
\pgfusepath{clip}%
\pgfsetbuttcap%
\pgfsetroundjoin%
\definecolor{currentfill}{rgb}{0.412913,0.803041,0.357269}%
\pgfsetfillcolor{currentfill}%
\pgfsetlinewidth{0.000000pt}%
\definecolor{currentstroke}{rgb}{0.000000,0.000000,0.000000}%
\pgfsetstrokecolor{currentstroke}%
\pgfsetdash{}{0pt}%
\pgfpathmoveto{\pgfqpoint{1.292768in}{1.563060in}}%
\pgfpathlineto{\pgfqpoint{1.294878in}{1.558486in}}%
\pgfpathlineto{\pgfqpoint{1.296986in}{1.553832in}}%
\pgfpathlineto{\pgfqpoint{1.299091in}{1.549100in}}%
\pgfpathlineto{\pgfqpoint{1.301193in}{1.544292in}}%
\pgfpathlineto{\pgfqpoint{1.306743in}{1.542437in}}%
\pgfpathlineto{\pgfqpoint{1.312171in}{1.540499in}}%
\pgfpathlineto{\pgfqpoint{1.317472in}{1.538480in}}%
\pgfpathlineto{\pgfqpoint{1.322642in}{1.536383in}}%
\pgfpathlineto{\pgfqpoint{1.320165in}{1.541334in}}%
\pgfpathlineto{\pgfqpoint{1.317684in}{1.546209in}}%
\pgfpathlineto{\pgfqpoint{1.315201in}{1.551006in}}%
\pgfpathlineto{\pgfqpoint{1.312715in}{1.555722in}}%
\pgfpathlineto{\pgfqpoint{1.307908in}{1.557668in}}%
\pgfpathlineto{\pgfqpoint{1.302978in}{1.559541in}}%
\pgfpathlineto{\pgfqpoint{1.297930in}{1.561339in}}%
\pgfpathlineto{\pgfqpoint{1.292768in}{1.563060in}}%
\pgfpathclose%
\pgfusepath{fill}%
\end{pgfscope}%
\begin{pgfscope}%
\pgfpathrectangle{\pgfqpoint{0.041670in}{0.041670in}}{\pgfqpoint{2.216660in}{2.216660in}}%
\pgfusepath{clip}%
\pgfsetbuttcap%
\pgfsetroundjoin%
\definecolor{currentfill}{rgb}{0.231674,0.318106,0.544834}%
\pgfsetfillcolor{currentfill}%
\pgfsetlinewidth{0.000000pt}%
\definecolor{currentstroke}{rgb}{0.000000,0.000000,0.000000}%
\pgfsetstrokecolor{currentstroke}%
\pgfsetdash{}{0pt}%
\pgfpathmoveto{\pgfqpoint{0.786507in}{1.011769in}}%
\pgfpathlineto{\pgfqpoint{0.782965in}{1.004064in}}%
\pgfpathlineto{\pgfqpoint{0.779425in}{0.996423in}}%
\pgfpathlineto{\pgfqpoint{0.775885in}{0.988849in}}%
\pgfpathlineto{\pgfqpoint{0.772345in}{0.981346in}}%
\pgfpathlineto{\pgfqpoint{0.768642in}{0.987994in}}%
\pgfpathlineto{\pgfqpoint{0.765359in}{0.994691in}}%
\pgfpathlineto{\pgfqpoint{0.762498in}{1.001432in}}%
\pgfpathlineto{\pgfqpoint{0.760061in}{1.008210in}}%
\pgfpathlineto{\pgfqpoint{0.763694in}{1.015452in}}%
\pgfpathlineto{\pgfqpoint{0.767327in}{1.022765in}}%
\pgfpathlineto{\pgfqpoint{0.770961in}{1.030146in}}%
\pgfpathlineto{\pgfqpoint{0.774596in}{1.037592in}}%
\pgfpathlineto{\pgfqpoint{0.776962in}{1.031076in}}%
\pgfpathlineto{\pgfqpoint{0.779737in}{1.024596in}}%
\pgfpathlineto{\pgfqpoint{0.782919in}{1.018158in}}%
\pgfpathlineto{\pgfqpoint{0.786507in}{1.011769in}}%
\pgfpathclose%
\pgfusepath{fill}%
\end{pgfscope}%
\begin{pgfscope}%
\pgfpathrectangle{\pgfqpoint{0.041670in}{0.041670in}}{\pgfqpoint{2.216660in}{2.216660in}}%
\pgfusepath{clip}%
\pgfsetbuttcap%
\pgfsetroundjoin%
\definecolor{currentfill}{rgb}{0.179019,0.433756,0.557430}%
\pgfsetfillcolor{currentfill}%
\pgfsetlinewidth{0.000000pt}%
\definecolor{currentstroke}{rgb}{0.000000,0.000000,0.000000}%
\pgfsetstrokecolor{currentstroke}%
\pgfsetdash{}{0pt}%
\pgfpathmoveto{\pgfqpoint{0.818320in}{1.130794in}}%
\pgfpathlineto{\pgfqpoint{0.814667in}{1.122825in}}%
\pgfpathlineto{\pgfqpoint{0.811016in}{1.114882in}}%
\pgfpathlineto{\pgfqpoint{0.807367in}{1.106969in}}%
\pgfpathlineto{\pgfqpoint{0.803720in}{1.099089in}}%
\pgfpathlineto{\pgfqpoint{0.801880in}{1.105110in}}%
\pgfpathlineto{\pgfqpoint{0.800422in}{1.111153in}}%
\pgfpathlineto{\pgfqpoint{0.799345in}{1.117211in}}%
\pgfpathlineto{\pgfqpoint{0.798650in}{1.123277in}}%
\pgfpathlineto{\pgfqpoint{0.802334in}{1.130897in}}%
\pgfpathlineto{\pgfqpoint{0.806020in}{1.138550in}}%
\pgfpathlineto{\pgfqpoint{0.809708in}{1.146232in}}%
\pgfpathlineto{\pgfqpoint{0.813398in}{1.153942in}}%
\pgfpathlineto{\pgfqpoint{0.814078in}{1.148136in}}%
\pgfpathlineto{\pgfqpoint{0.815125in}{1.142338in}}%
\pgfpathlineto{\pgfqpoint{0.816539in}{1.136556in}}%
\pgfpathlineto{\pgfqpoint{0.818320in}{1.130794in}}%
\pgfpathclose%
\pgfusepath{fill}%
\end{pgfscope}%
\begin{pgfscope}%
\pgfpathrectangle{\pgfqpoint{0.041670in}{0.041670in}}{\pgfqpoint{2.216660in}{2.216660in}}%
\pgfusepath{clip}%
\pgfsetbuttcap%
\pgfsetroundjoin%
\definecolor{currentfill}{rgb}{0.268510,0.009605,0.335427}%
\pgfsetfillcolor{currentfill}%
\pgfsetlinewidth{0.000000pt}%
\definecolor{currentstroke}{rgb}{0.000000,0.000000,0.000000}%
\pgfsetstrokecolor{currentstroke}%
\pgfsetdash{}{0pt}%
\pgfpathmoveto{\pgfqpoint{0.626258in}{0.739363in}}%
\pgfpathlineto{\pgfqpoint{0.622694in}{0.739804in}}%
\pgfpathlineto{\pgfqpoint{0.619121in}{0.740525in}}%
\pgfpathlineto{\pgfqpoint{0.615538in}{0.741529in}}%
\pgfpathlineto{\pgfqpoint{0.611947in}{0.742822in}}%
\pgfpathlineto{\pgfqpoint{0.604360in}{0.752414in}}%
\pgfpathlineto{\pgfqpoint{0.597377in}{0.762116in}}%
\pgfpathlineto{\pgfqpoint{0.591002in}{0.771916in}}%
\pgfpathlineto{\pgfqpoint{0.585240in}{0.781805in}}%
\pgfpathlineto{\pgfqpoint{0.588981in}{0.780273in}}%
\pgfpathlineto{\pgfqpoint{0.592712in}{0.779029in}}%
\pgfpathlineto{\pgfqpoint{0.596434in}{0.778069in}}%
\pgfpathlineto{\pgfqpoint{0.600147in}{0.777386in}}%
\pgfpathlineto{\pgfqpoint{0.605784in}{0.767740in}}%
\pgfpathlineto{\pgfqpoint{0.612018in}{0.758180in}}%
\pgfpathlineto{\pgfqpoint{0.618844in}{0.748718in}}%
\pgfpathlineto{\pgfqpoint{0.626258in}{0.739363in}}%
\pgfpathclose%
\pgfusepath{fill}%
\end{pgfscope}%
\begin{pgfscope}%
\pgfpathrectangle{\pgfqpoint{0.041670in}{0.041670in}}{\pgfqpoint{2.216660in}{2.216660in}}%
\pgfusepath{clip}%
\pgfsetbuttcap%
\pgfsetroundjoin%
\definecolor{currentfill}{rgb}{0.201239,0.383670,0.554294}%
\pgfsetfillcolor{currentfill}%
\pgfsetlinewidth{0.000000pt}%
\definecolor{currentstroke}{rgb}{0.000000,0.000000,0.000000}%
\pgfsetstrokecolor{currentstroke}%
\pgfsetdash{}{0pt}%
\pgfpathmoveto{\pgfqpoint{1.911549in}{1.069208in}}%
\pgfpathlineto{\pgfqpoint{1.915831in}{1.082148in}}%
\pgfpathlineto{\pgfqpoint{1.920136in}{1.095565in}}%
\pgfpathlineto{\pgfqpoint{1.924463in}{1.109467in}}%
\pgfpathlineto{\pgfqpoint{1.928813in}{1.123862in}}%
\pgfpathlineto{\pgfqpoint{1.928743in}{1.111933in}}%
\pgfpathlineto{\pgfqpoint{1.927923in}{1.099991in}}%
\pgfpathlineto{\pgfqpoint{1.926348in}{1.088048in}}%
\pgfpathlineto{\pgfqpoint{1.924018in}{1.076116in}}%
\pgfpathlineto{\pgfqpoint{1.919666in}{1.061902in}}%
\pgfpathlineto{\pgfqpoint{1.915338in}{1.048184in}}%
\pgfpathlineto{\pgfqpoint{1.911032in}{1.034955in}}%
\pgfpathlineto{\pgfqpoint{1.906749in}{1.022205in}}%
\pgfpathlineto{\pgfqpoint{1.909054in}{1.033950in}}%
\pgfpathlineto{\pgfqpoint{1.910621in}{1.045707in}}%
\pgfpathlineto{\pgfqpoint{1.911452in}{1.057464in}}%
\pgfpathlineto{\pgfqpoint{1.911549in}{1.069208in}}%
\pgfpathclose%
\pgfusepath{fill}%
\end{pgfscope}%
\begin{pgfscope}%
\pgfpathrectangle{\pgfqpoint{0.041670in}{0.041670in}}{\pgfqpoint{2.216660in}{2.216660in}}%
\pgfusepath{clip}%
\pgfsetbuttcap%
\pgfsetroundjoin%
\definecolor{currentfill}{rgb}{0.487026,0.823929,0.312321}%
\pgfsetfillcolor{currentfill}%
\pgfsetlinewidth{0.000000pt}%
\definecolor{currentstroke}{rgb}{0.000000,0.000000,0.000000}%
\pgfsetstrokecolor{currentstroke}%
\pgfsetdash{}{0pt}%
\pgfpathmoveto{\pgfqpoint{1.242911in}{1.590513in}}%
\pgfpathlineto{\pgfqpoint{1.244188in}{1.586488in}}%
\pgfpathlineto{\pgfqpoint{1.245464in}{1.582376in}}%
\pgfpathlineto{\pgfqpoint{1.246738in}{1.578177in}}%
\pgfpathlineto{\pgfqpoint{1.248011in}{1.573896in}}%
\pgfpathlineto{\pgfqpoint{1.253892in}{1.572838in}}%
\pgfpathlineto{\pgfqpoint{1.259702in}{1.571694in}}%
\pgfpathlineto{\pgfqpoint{1.265436in}{1.570463in}}%
\pgfpathlineto{\pgfqpoint{1.271087in}{1.569148in}}%
\pgfpathlineto{\pgfqpoint{1.269382in}{1.573522in}}%
\pgfpathlineto{\pgfqpoint{1.267675in}{1.577811in}}%
\pgfpathlineto{\pgfqpoint{1.265965in}{1.582016in}}%
\pgfpathlineto{\pgfqpoint{1.264253in}{1.586133in}}%
\pgfpathlineto{\pgfqpoint{1.259027in}{1.587347in}}%
\pgfpathlineto{\pgfqpoint{1.253724in}{1.588482in}}%
\pgfpathlineto{\pgfqpoint{1.248350in}{1.589538in}}%
\pgfpathlineto{\pgfqpoint{1.242911in}{1.590513in}}%
\pgfpathclose%
\pgfusepath{fill}%
\end{pgfscope}%
\begin{pgfscope}%
\pgfpathrectangle{\pgfqpoint{0.041670in}{0.041670in}}{\pgfqpoint{2.216660in}{2.216660in}}%
\pgfusepath{clip}%
\pgfsetbuttcap%
\pgfsetroundjoin%
\definecolor{currentfill}{rgb}{0.274128,0.199721,0.498911}%
\pgfsetfillcolor{currentfill}%
\pgfsetlinewidth{0.000000pt}%
\definecolor{currentstroke}{rgb}{0.000000,0.000000,0.000000}%
\pgfsetstrokecolor{currentstroke}%
\pgfsetdash{}{0pt}%
\pgfpathmoveto{\pgfqpoint{1.619397in}{0.930660in}}%
\pgfpathlineto{\pgfqpoint{1.622963in}{0.924009in}}%
\pgfpathlineto{\pgfqpoint{1.626530in}{0.917465in}}%
\pgfpathlineto{\pgfqpoint{1.630097in}{0.911030in}}%
\pgfpathlineto{\pgfqpoint{1.633666in}{0.904708in}}%
\pgfpathlineto{\pgfqpoint{1.629570in}{0.897290in}}%
\pgfpathlineto{\pgfqpoint{1.625011in}{0.889935in}}%
\pgfpathlineto{\pgfqpoint{1.619994in}{0.882651in}}%
\pgfpathlineto{\pgfqpoint{1.614522in}{0.875446in}}%
\pgfpathlineto{\pgfqpoint{1.611090in}{0.882025in}}%
\pgfpathlineto{\pgfqpoint{1.607659in}{0.888716in}}%
\pgfpathlineto{\pgfqpoint{1.604229in}{0.895517in}}%
\pgfpathlineto{\pgfqpoint{1.600800in}{0.902424in}}%
\pgfpathlineto{\pgfqpoint{1.606113in}{0.909375in}}%
\pgfpathlineto{\pgfqpoint{1.610986in}{0.916403in}}%
\pgfpathlineto{\pgfqpoint{1.615415in}{0.923501in}}%
\pgfpathlineto{\pgfqpoint{1.619397in}{0.930660in}}%
\pgfpathclose%
\pgfusepath{fill}%
\end{pgfscope}%
\begin{pgfscope}%
\pgfpathrectangle{\pgfqpoint{0.041670in}{0.041670in}}{\pgfqpoint{2.216660in}{2.216660in}}%
\pgfusepath{clip}%
\pgfsetbuttcap%
\pgfsetroundjoin%
\definecolor{currentfill}{rgb}{0.134692,0.658636,0.517649}%
\pgfsetfillcolor{currentfill}%
\pgfsetlinewidth{0.000000pt}%
\definecolor{currentstroke}{rgb}{0.000000,0.000000,0.000000}%
\pgfsetstrokecolor{currentstroke}%
\pgfsetdash{}{0pt}%
\pgfpathmoveto{\pgfqpoint{0.908108in}{1.373479in}}%
\pgfpathlineto{\pgfqpoint{0.904442in}{1.366473in}}%
\pgfpathlineto{\pgfqpoint{0.900781in}{1.359429in}}%
\pgfpathlineto{\pgfqpoint{0.897123in}{1.352349in}}%
\pgfpathlineto{\pgfqpoint{0.893468in}{1.345235in}}%
\pgfpathlineto{\pgfqpoint{0.895473in}{1.349634in}}%
\pgfpathlineto{\pgfqpoint{0.897758in}{1.353996in}}%
\pgfpathlineto{\pgfqpoint{0.900319in}{1.358318in}}%
\pgfpathlineto{\pgfqpoint{0.903153in}{1.362596in}}%
\pgfpathlineto{\pgfqpoint{0.906675in}{1.369470in}}%
\pgfpathlineto{\pgfqpoint{0.910201in}{1.376312in}}%
\pgfpathlineto{\pgfqpoint{0.913730in}{1.383118in}}%
\pgfpathlineto{\pgfqpoint{0.917263in}{1.389886in}}%
\pgfpathlineto{\pgfqpoint{0.914581in}{1.385842in}}%
\pgfpathlineto{\pgfqpoint{0.912160in}{1.381758in}}%
\pgfpathlineto{\pgfqpoint{0.910001in}{1.377635in}}%
\pgfpathlineto{\pgfqpoint{0.908108in}{1.373479in}}%
\pgfpathclose%
\pgfusepath{fill}%
\end{pgfscope}%
\begin{pgfscope}%
\pgfpathrectangle{\pgfqpoint{0.041670in}{0.041670in}}{\pgfqpoint{2.216660in}{2.216660in}}%
\pgfusepath{clip}%
\pgfsetbuttcap%
\pgfsetroundjoin%
\definecolor{currentfill}{rgb}{0.412913,0.803041,0.357269}%
\pgfsetfillcolor{currentfill}%
\pgfsetlinewidth{0.000000pt}%
\definecolor{currentstroke}{rgb}{0.000000,0.000000,0.000000}%
\pgfsetstrokecolor{currentstroke}%
\pgfsetdash{}{0pt}%
\pgfpathmoveto{\pgfqpoint{1.043029in}{1.553933in}}%
\pgfpathlineto{\pgfqpoint{1.040464in}{1.549181in}}%
\pgfpathlineto{\pgfqpoint{1.037903in}{1.544350in}}%
\pgfpathlineto{\pgfqpoint{1.035344in}{1.539440in}}%
\pgfpathlineto{\pgfqpoint{1.032788in}{1.534453in}}%
\pgfpathlineto{\pgfqpoint{1.037836in}{1.536620in}}%
\pgfpathlineto{\pgfqpoint{1.043020in}{1.538709in}}%
\pgfpathlineto{\pgfqpoint{1.048336in}{1.540718in}}%
\pgfpathlineto{\pgfqpoint{1.053778in}{1.542647in}}%
\pgfpathlineto{\pgfqpoint{1.055966in}{1.547485in}}%
\pgfpathlineto{\pgfqpoint{1.058157in}{1.552247in}}%
\pgfpathlineto{\pgfqpoint{1.060351in}{1.556931in}}%
\pgfpathlineto{\pgfqpoint{1.062548in}{1.561534in}}%
\pgfpathlineto{\pgfqpoint{1.057487in}{1.559745in}}%
\pgfpathlineto{\pgfqpoint{1.052543in}{1.557880in}}%
\pgfpathlineto{\pgfqpoint{1.047723in}{1.555942in}}%
\pgfpathlineto{\pgfqpoint{1.043029in}{1.553933in}}%
\pgfpathclose%
\pgfusepath{fill}%
\end{pgfscope}%
\begin{pgfscope}%
\pgfpathrectangle{\pgfqpoint{0.041670in}{0.041670in}}{\pgfqpoint{2.216660in}{2.216660in}}%
\pgfusepath{clip}%
\pgfsetbuttcap%
\pgfsetroundjoin%
\definecolor{currentfill}{rgb}{0.212395,0.359683,0.551710}%
\pgfsetfillcolor{currentfill}%
\pgfsetlinewidth{0.000000pt}%
\definecolor{currentstroke}{rgb}{0.000000,0.000000,0.000000}%
\pgfsetstrokecolor{currentstroke}%
\pgfsetdash{}{0pt}%
\pgfpathmoveto{\pgfqpoint{1.572469in}{1.073537in}}%
\pgfpathlineto{\pgfqpoint{1.576122in}{1.065924in}}%
\pgfpathlineto{\pgfqpoint{1.579773in}{1.058363in}}%
\pgfpathlineto{\pgfqpoint{1.583423in}{1.050857in}}%
\pgfpathlineto{\pgfqpoint{1.587072in}{1.043408in}}%
\pgfpathlineto{\pgfqpoint{1.585071in}{1.036866in}}%
\pgfpathlineto{\pgfqpoint{1.582660in}{1.030355in}}%
\pgfpathlineto{\pgfqpoint{1.579839in}{1.023879in}}%
\pgfpathlineto{\pgfqpoint{1.576612in}{1.017446in}}%
\pgfpathlineto{\pgfqpoint{1.573044in}{1.025155in}}%
\pgfpathlineto{\pgfqpoint{1.569475in}{1.032922in}}%
\pgfpathlineto{\pgfqpoint{1.565905in}{1.040744in}}%
\pgfpathlineto{\pgfqpoint{1.562334in}{1.048617in}}%
\pgfpathlineto{\pgfqpoint{1.565458in}{1.054791in}}%
\pgfpathlineto{\pgfqpoint{1.568190in}{1.061007in}}%
\pgfpathlineto{\pgfqpoint{1.570527in}{1.067258in}}%
\pgfpathlineto{\pgfqpoint{1.572469in}{1.073537in}}%
\pgfpathclose%
\pgfusepath{fill}%
\end{pgfscope}%
\begin{pgfscope}%
\pgfpathrectangle{\pgfqpoint{0.041670in}{0.041670in}}{\pgfqpoint{2.216660in}{2.216660in}}%
\pgfusepath{clip}%
\pgfsetbuttcap%
\pgfsetroundjoin%
\definecolor{currentfill}{rgb}{0.133743,0.548535,0.553541}%
\pgfsetfillcolor{currentfill}%
\pgfsetlinewidth{0.000000pt}%
\definecolor{currentstroke}{rgb}{0.000000,0.000000,0.000000}%
\pgfsetstrokecolor{currentstroke}%
\pgfsetdash{}{0pt}%
\pgfpathmoveto{\pgfqpoint{1.500494in}{1.272029in}}%
\pgfpathlineto{\pgfqpoint{1.504184in}{1.264537in}}%
\pgfpathlineto{\pgfqpoint{1.507870in}{1.257039in}}%
\pgfpathlineto{\pgfqpoint{1.511553in}{1.249536in}}%
\pgfpathlineto{\pgfqpoint{1.515233in}{1.242031in}}%
\pgfpathlineto{\pgfqpoint{1.516226in}{1.236775in}}%
\pgfpathlineto{\pgfqpoint{1.516886in}{1.231502in}}%
\pgfpathlineto{\pgfqpoint{1.517210in}{1.226218in}}%
\pgfpathlineto{\pgfqpoint{1.517199in}{1.220927in}}%
\pgfpathlineto{\pgfqpoint{1.513486in}{1.228687in}}%
\pgfpathlineto{\pgfqpoint{1.509770in}{1.236445in}}%
\pgfpathlineto{\pgfqpoint{1.506052in}{1.244198in}}%
\pgfpathlineto{\pgfqpoint{1.502330in}{1.251943in}}%
\pgfpathlineto{\pgfqpoint{1.502352in}{1.256979in}}%
\pgfpathlineto{\pgfqpoint{1.502053in}{1.262008in}}%
\pgfpathlineto{\pgfqpoint{1.501433in}{1.267026in}}%
\pgfpathlineto{\pgfqpoint{1.500494in}{1.272029in}}%
\pgfpathclose%
\pgfusepath{fill}%
\end{pgfscope}%
\begin{pgfscope}%
\pgfpathrectangle{\pgfqpoint{0.041670in}{0.041670in}}{\pgfqpoint{2.216660in}{2.216660in}}%
\pgfusepath{clip}%
\pgfsetbuttcap%
\pgfsetroundjoin%
\definecolor{currentfill}{rgb}{0.487026,0.823929,0.312321}%
\pgfsetfillcolor{currentfill}%
\pgfsetlinewidth{0.000000pt}%
\definecolor{currentstroke}{rgb}{0.000000,0.000000,0.000000}%
\pgfsetstrokecolor{currentstroke}%
\pgfsetdash{}{0pt}%
\pgfpathmoveto{\pgfqpoint{1.091078in}{1.584989in}}%
\pgfpathlineto{\pgfqpoint{1.089273in}{1.580848in}}%
\pgfpathlineto{\pgfqpoint{1.087470in}{1.576620in}}%
\pgfpathlineto{\pgfqpoint{1.085670in}{1.572306in}}%
\pgfpathlineto{\pgfqpoint{1.083872in}{1.567908in}}%
\pgfpathlineto{\pgfqpoint{1.089446in}{1.569298in}}%
\pgfpathlineto{\pgfqpoint{1.095107in}{1.570604in}}%
\pgfpathlineto{\pgfqpoint{1.100850in}{1.571825in}}%
\pgfpathlineto{\pgfqpoint{1.106668in}{1.572960in}}%
\pgfpathlineto{\pgfqpoint{1.108039in}{1.577260in}}%
\pgfpathlineto{\pgfqpoint{1.109411in}{1.581476in}}%
\pgfpathlineto{\pgfqpoint{1.110785in}{1.585607in}}%
\pgfpathlineto{\pgfqpoint{1.112161in}{1.589650in}}%
\pgfpathlineto{\pgfqpoint{1.106780in}{1.588603in}}%
\pgfpathlineto{\pgfqpoint{1.101469in}{1.587477in}}%
\pgfpathlineto{\pgfqpoint{1.096233in}{1.586271in}}%
\pgfpathlineto{\pgfqpoint{1.091078in}{1.584989in}}%
\pgfpathclose%
\pgfusepath{fill}%
\end{pgfscope}%
\begin{pgfscope}%
\pgfpathrectangle{\pgfqpoint{0.041670in}{0.041670in}}{\pgfqpoint{2.216660in}{2.216660in}}%
\pgfusepath{clip}%
\pgfsetbuttcap%
\pgfsetroundjoin%
\definecolor{currentfill}{rgb}{0.280255,0.165693,0.476498}%
\pgfsetfillcolor{currentfill}%
\pgfsetlinewidth{0.000000pt}%
\definecolor{currentstroke}{rgb}{0.000000,0.000000,0.000000}%
\pgfsetstrokecolor{currentstroke}%
\pgfsetdash{}{0pt}%
\pgfpathmoveto{\pgfqpoint{0.750631in}{0.869115in}}%
\pgfpathlineto{\pgfqpoint{0.747236in}{0.862598in}}%
\pgfpathlineto{\pgfqpoint{0.743840in}{0.856202in}}%
\pgfpathlineto{\pgfqpoint{0.740442in}{0.849930in}}%
\pgfpathlineto{\pgfqpoint{0.737042in}{0.843787in}}%
\pgfpathlineto{\pgfqpoint{0.730997in}{0.851166in}}%
\pgfpathlineto{\pgfqpoint{0.725418in}{0.858633in}}%
\pgfpathlineto{\pgfqpoint{0.720310in}{0.866181in}}%
\pgfpathlineto{\pgfqpoint{0.715675in}{0.873801in}}%
\pgfpathlineto{\pgfqpoint{0.719224in}{0.879690in}}%
\pgfpathlineto{\pgfqpoint{0.722772in}{0.885707in}}%
\pgfpathlineto{\pgfqpoint{0.726317in}{0.891849in}}%
\pgfpathlineto{\pgfqpoint{0.729862in}{0.898111in}}%
\pgfpathlineto{\pgfqpoint{0.734369in}{0.890749in}}%
\pgfpathlineto{\pgfqpoint{0.739336in}{0.883457in}}%
\pgfpathlineto{\pgfqpoint{0.744758in}{0.876243in}}%
\pgfpathlineto{\pgfqpoint{0.750631in}{0.869115in}}%
\pgfpathclose%
\pgfusepath{fill}%
\end{pgfscope}%
\begin{pgfscope}%
\pgfpathrectangle{\pgfqpoint{0.041670in}{0.041670in}}{\pgfqpoint{2.216660in}{2.216660in}}%
\pgfusepath{clip}%
\pgfsetbuttcap%
\pgfsetroundjoin%
\definecolor{currentfill}{rgb}{0.120081,0.622161,0.534946}%
\pgfsetfillcolor{currentfill}%
\pgfsetlinewidth{0.000000pt}%
\definecolor{currentstroke}{rgb}{0.000000,0.000000,0.000000}%
\pgfsetstrokecolor{currentstroke}%
\pgfsetdash{}{0pt}%
\pgfpathmoveto{\pgfqpoint{1.464673in}{1.349147in}}%
\pgfpathlineto{\pgfqpoint{1.468299in}{1.342057in}}%
\pgfpathlineto{\pgfqpoint{1.471922in}{1.334938in}}%
\pgfpathlineto{\pgfqpoint{1.475542in}{1.327795in}}%
\pgfpathlineto{\pgfqpoint{1.479158in}{1.320628in}}%
\pgfpathlineto{\pgfqpoint{1.481242in}{1.315982in}}%
\pgfpathlineto{\pgfqpoint{1.483029in}{1.311304in}}%
\pgfpathlineto{\pgfqpoint{1.484518in}{1.306596in}}%
\pgfpathlineto{\pgfqpoint{1.485706in}{1.301865in}}%
\pgfpathlineto{\pgfqpoint{1.482000in}{1.309279in}}%
\pgfpathlineto{\pgfqpoint{1.478292in}{1.316669in}}%
\pgfpathlineto{\pgfqpoint{1.474580in}{1.324034in}}%
\pgfpathlineto{\pgfqpoint{1.470865in}{1.331370in}}%
\pgfpathlineto{\pgfqpoint{1.469745in}{1.335852in}}%
\pgfpathlineto{\pgfqpoint{1.468338in}{1.340312in}}%
\pgfpathlineto{\pgfqpoint{1.466647in}{1.344745in}}%
\pgfpathlineto{\pgfqpoint{1.464673in}{1.349147in}}%
\pgfpathclose%
\pgfusepath{fill}%
\end{pgfscope}%
\begin{pgfscope}%
\pgfpathrectangle{\pgfqpoint{0.041670in}{0.041670in}}{\pgfqpoint{2.216660in}{2.216660in}}%
\pgfusepath{clip}%
\pgfsetbuttcap%
\pgfsetroundjoin%
\definecolor{currentfill}{rgb}{0.163625,0.471133,0.558148}%
\pgfsetfillcolor{currentfill}%
\pgfsetlinewidth{0.000000pt}%
\definecolor{currentstroke}{rgb}{0.000000,0.000000,0.000000}%
\pgfsetstrokecolor{currentstroke}%
\pgfsetdash{}{0pt}%
\pgfpathmoveto{\pgfqpoint{1.532024in}{1.189920in}}%
\pgfpathlineto{\pgfqpoint{1.535724in}{1.182191in}}%
\pgfpathlineto{\pgfqpoint{1.539421in}{1.174477in}}%
\pgfpathlineto{\pgfqpoint{1.543116in}{1.166781in}}%
\pgfpathlineto{\pgfqpoint{1.546808in}{1.159105in}}%
\pgfpathlineto{\pgfqpoint{1.546455in}{1.153296in}}%
\pgfpathlineto{\pgfqpoint{1.545734in}{1.147491in}}%
\pgfpathlineto{\pgfqpoint{1.544646in}{1.141695in}}%
\pgfpathlineto{\pgfqpoint{1.543191in}{1.135914in}}%
\pgfpathlineto{\pgfqpoint{1.539523in}{1.143849in}}%
\pgfpathlineto{\pgfqpoint{1.535852in}{1.151805in}}%
\pgfpathlineto{\pgfqpoint{1.532180in}{1.159778in}}%
\pgfpathlineto{\pgfqpoint{1.528505in}{1.167765in}}%
\pgfpathlineto{\pgfqpoint{1.529913in}{1.173287in}}%
\pgfpathlineto{\pgfqpoint{1.530969in}{1.178824in}}%
\pgfpathlineto{\pgfqpoint{1.531672in}{1.184370in}}%
\pgfpathlineto{\pgfqpoint{1.532024in}{1.189920in}}%
\pgfpathclose%
\pgfusepath{fill}%
\end{pgfscope}%
\begin{pgfscope}%
\pgfpathrectangle{\pgfqpoint{0.041670in}{0.041670in}}{\pgfqpoint{2.216660in}{2.216660in}}%
\pgfusepath{clip}%
\pgfsetbuttcap%
\pgfsetroundjoin%
\definecolor{currentfill}{rgb}{0.281477,0.755203,0.432552}%
\pgfsetfillcolor{currentfill}%
\pgfsetlinewidth{0.000000pt}%
\definecolor{currentstroke}{rgb}{0.000000,0.000000,0.000000}%
\pgfsetstrokecolor{currentstroke}%
\pgfsetdash{}{0pt}%
\pgfpathmoveto{\pgfqpoint{1.371061in}{1.495065in}}%
\pgfpathlineto{\pgfqpoint{1.374155in}{1.489408in}}%
\pgfpathlineto{\pgfqpoint{1.377245in}{1.483686in}}%
\pgfpathlineto{\pgfqpoint{1.380331in}{1.477902in}}%
\pgfpathlineto{\pgfqpoint{1.383414in}{1.472058in}}%
\pgfpathlineto{\pgfqpoint{1.387713in}{1.468948in}}%
\pgfpathlineto{\pgfqpoint{1.391811in}{1.465773in}}%
\pgfpathlineto{\pgfqpoint{1.395703in}{1.462536in}}%
\pgfpathlineto{\pgfqpoint{1.399386in}{1.459240in}}%
\pgfpathlineto{\pgfqpoint{1.396055in}{1.465290in}}%
\pgfpathlineto{\pgfqpoint{1.392720in}{1.471280in}}%
\pgfpathlineto{\pgfqpoint{1.389381in}{1.477207in}}%
\pgfpathlineto{\pgfqpoint{1.386039in}{1.483069in}}%
\pgfpathlineto{\pgfqpoint{1.382586in}{1.486153in}}%
\pgfpathlineto{\pgfqpoint{1.378937in}{1.489182in}}%
\pgfpathlineto{\pgfqpoint{1.375094in}{1.492154in}}%
\pgfpathlineto{\pgfqpoint{1.371061in}{1.495065in}}%
\pgfpathclose%
\pgfusepath{fill}%
\end{pgfscope}%
\begin{pgfscope}%
\pgfpathrectangle{\pgfqpoint{0.041670in}{0.041670in}}{\pgfqpoint{2.216660in}{2.216660in}}%
\pgfusepath{clip}%
\pgfsetbuttcap%
\pgfsetroundjoin%
\definecolor{currentfill}{rgb}{0.344074,0.780029,0.397381}%
\pgfsetfillcolor{currentfill}%
\pgfsetlinewidth{0.000000pt}%
\definecolor{currentstroke}{rgb}{0.000000,0.000000,0.000000}%
\pgfsetstrokecolor{currentstroke}%
\pgfsetdash{}{0pt}%
\pgfpathmoveto{\pgfqpoint{1.341897in}{1.527242in}}%
\pgfpathlineto{\pgfqpoint{1.344709in}{1.522050in}}%
\pgfpathlineto{\pgfqpoint{1.347517in}{1.516786in}}%
\pgfpathlineto{\pgfqpoint{1.350322in}{1.511451in}}%
\pgfpathlineto{\pgfqpoint{1.353124in}{1.506047in}}%
\pgfpathlineto{\pgfqpoint{1.357871in}{1.503406in}}%
\pgfpathlineto{\pgfqpoint{1.362445in}{1.500693in}}%
\pgfpathlineto{\pgfqpoint{1.366844in}{1.497912in}}%
\pgfpathlineto{\pgfqpoint{1.371061in}{1.495065in}}%
\pgfpathlineto{\pgfqpoint{1.367964in}{1.500655in}}%
\pgfpathlineto{\pgfqpoint{1.364863in}{1.506177in}}%
\pgfpathlineto{\pgfqpoint{1.361759in}{1.511627in}}%
\pgfpathlineto{\pgfqpoint{1.358651in}{1.517005in}}%
\pgfpathlineto{\pgfqpoint{1.354713in}{1.519659in}}%
\pgfpathlineto{\pgfqpoint{1.350604in}{1.522251in}}%
\pgfpathlineto{\pgfqpoint{1.346331in}{1.524780in}}%
\pgfpathlineto{\pgfqpoint{1.341897in}{1.527242in}}%
\pgfpathclose%
\pgfusepath{fill}%
\end{pgfscope}%
\begin{pgfscope}%
\pgfpathrectangle{\pgfqpoint{0.041670in}{0.041670in}}{\pgfqpoint{2.216660in}{2.216660in}}%
\pgfusepath{clip}%
\pgfsetbuttcap%
\pgfsetroundjoin%
\definecolor{currentfill}{rgb}{0.277941,0.056324,0.381191}%
\pgfsetfillcolor{currentfill}%
\pgfsetlinewidth{0.000000pt}%
\definecolor{currentstroke}{rgb}{0.000000,0.000000,0.000000}%
\pgfsetstrokecolor{currentstroke}%
\pgfsetdash{}{0pt}%
\pgfpathmoveto{\pgfqpoint{1.794441in}{0.799991in}}%
\pgfpathlineto{\pgfqpoint{1.798260in}{0.803099in}}%
\pgfpathlineto{\pgfqpoint{1.802091in}{0.806527in}}%
\pgfpathlineto{\pgfqpoint{1.805933in}{0.810282in}}%
\pgfpathlineto{\pgfqpoint{1.809789in}{0.814370in}}%
\pgfpathlineto{\pgfqpoint{1.804352in}{0.803938in}}%
\pgfpathlineto{\pgfqpoint{1.798267in}{0.793588in}}%
\pgfpathlineto{\pgfqpoint{1.791537in}{0.783329in}}%
\pgfpathlineto{\pgfqpoint{1.784167in}{0.773174in}}%
\pgfpathlineto{\pgfqpoint{1.780448in}{0.769317in}}%
\pgfpathlineto{\pgfqpoint{1.776740in}{0.765794in}}%
\pgfpathlineto{\pgfqpoint{1.773045in}{0.762599in}}%
\pgfpathlineto{\pgfqpoint{1.769362in}{0.759726in}}%
\pgfpathlineto{\pgfqpoint{1.776572in}{0.769650in}}%
\pgfpathlineto{\pgfqpoint{1.783157in}{0.779677in}}%
\pgfpathlineto{\pgfqpoint{1.789115in}{0.789794in}}%
\pgfpathlineto{\pgfqpoint{1.794441in}{0.799991in}}%
\pgfpathclose%
\pgfusepath{fill}%
\end{pgfscope}%
\begin{pgfscope}%
\pgfpathrectangle{\pgfqpoint{0.041670in}{0.041670in}}{\pgfqpoint{2.216660in}{2.216660in}}%
\pgfusepath{clip}%
\pgfsetbuttcap%
\pgfsetroundjoin%
\definecolor{currentfill}{rgb}{0.220124,0.725509,0.466226}%
\pgfsetfillcolor{currentfill}%
\pgfsetlinewidth{0.000000pt}%
\definecolor{currentstroke}{rgb}{0.000000,0.000000,0.000000}%
\pgfsetstrokecolor{currentstroke}%
\pgfsetdash{}{0pt}%
\pgfpathmoveto{\pgfqpoint{1.399386in}{1.459240in}}%
\pgfpathlineto{\pgfqpoint{1.402713in}{1.453132in}}%
\pgfpathlineto{\pgfqpoint{1.406037in}{1.446967in}}%
\pgfpathlineto{\pgfqpoint{1.409358in}{1.440749in}}%
\pgfpathlineto{\pgfqpoint{1.412674in}{1.434479in}}%
\pgfpathlineto{\pgfqpoint{1.416361in}{1.430911in}}%
\pgfpathlineto{\pgfqpoint{1.419817in}{1.427287in}}%
\pgfpathlineto{\pgfqpoint{1.423041in}{1.423609in}}%
\pgfpathlineto{\pgfqpoint{1.426026in}{1.419883in}}%
\pgfpathlineto{\pgfqpoint{1.422512in}{1.426375in}}%
\pgfpathlineto{\pgfqpoint{1.418994in}{1.432815in}}%
\pgfpathlineto{\pgfqpoint{1.415472in}{1.439201in}}%
\pgfpathlineto{\pgfqpoint{1.411947in}{1.445531in}}%
\pgfpathlineto{\pgfqpoint{1.409139in}{1.449030in}}%
\pgfpathlineto{\pgfqpoint{1.406108in}{1.452484in}}%
\pgfpathlineto{\pgfqpoint{1.402855in}{1.455889in}}%
\pgfpathlineto{\pgfqpoint{1.399386in}{1.459240in}}%
\pgfpathclose%
\pgfusepath{fill}%
\end{pgfscope}%
\begin{pgfscope}%
\pgfpathrectangle{\pgfqpoint{0.041670in}{0.041670in}}{\pgfqpoint{2.216660in}{2.216660in}}%
\pgfusepath{clip}%
\pgfsetbuttcap%
\pgfsetroundjoin%
\definecolor{currentfill}{rgb}{0.272594,0.025563,0.353093}%
\pgfsetfillcolor{currentfill}%
\pgfsetlinewidth{0.000000pt}%
\definecolor{currentstroke}{rgb}{0.000000,0.000000,0.000000}%
\pgfsetstrokecolor{currentstroke}%
\pgfsetdash{}{0pt}%
\pgfpathmoveto{\pgfqpoint{0.611947in}{0.742822in}}%
\pgfpathlineto{\pgfqpoint{0.608345in}{0.744411in}}%
\pgfpathlineto{\pgfqpoint{0.604733in}{0.746299in}}%
\pgfpathlineto{\pgfqpoint{0.601110in}{0.748493in}}%
\pgfpathlineto{\pgfqpoint{0.597477in}{0.750998in}}%
\pgfpathlineto{\pgfqpoint{0.589716in}{0.760823in}}%
\pgfpathlineto{\pgfqpoint{0.582576in}{0.770760in}}%
\pgfpathlineto{\pgfqpoint{0.576059in}{0.780797in}}%
\pgfpathlineto{\pgfqpoint{0.570172in}{0.790923in}}%
\pgfpathlineto{\pgfqpoint{0.573955in}{0.788184in}}%
\pgfpathlineto{\pgfqpoint{0.577727in}{0.785755in}}%
\pgfpathlineto{\pgfqpoint{0.581489in}{0.783630in}}%
\pgfpathlineto{\pgfqpoint{0.585240in}{0.781805in}}%
\pgfpathlineto{\pgfqpoint{0.591002in}{0.771916in}}%
\pgfpathlineto{\pgfqpoint{0.597377in}{0.762116in}}%
\pgfpathlineto{\pgfqpoint{0.604360in}{0.752414in}}%
\pgfpathlineto{\pgfqpoint{0.611947in}{0.742822in}}%
\pgfpathclose%
\pgfusepath{fill}%
\end{pgfscope}%
\begin{pgfscope}%
\pgfpathrectangle{\pgfqpoint{0.041670in}{0.041670in}}{\pgfqpoint{2.216660in}{2.216660in}}%
\pgfusepath{clip}%
\pgfsetbuttcap%
\pgfsetroundjoin%
\definecolor{currentfill}{rgb}{0.276194,0.190074,0.493001}%
\pgfsetfillcolor{currentfill}%
\pgfsetlinewidth{0.000000pt}%
\definecolor{currentstroke}{rgb}{0.000000,0.000000,0.000000}%
\pgfsetstrokecolor{currentstroke}%
\pgfsetdash{}{0pt}%
\pgfpathmoveto{\pgfqpoint{0.523782in}{0.850082in}}%
\pgfpathlineto{\pgfqpoint{0.519822in}{0.857386in}}%
\pgfpathlineto{\pgfqpoint{0.515846in}{0.865086in}}%
\pgfpathlineto{\pgfqpoint{0.511852in}{0.873188in}}%
\pgfpathlineto{\pgfqpoint{0.507842in}{0.881698in}}%
\pgfpathlineto{\pgfqpoint{0.502146in}{0.892806in}}%
\pgfpathlineto{\pgfqpoint{0.497153in}{0.903984in}}%
\pgfpathlineto{\pgfqpoint{0.492863in}{0.915220in}}%
\pgfpathlineto{\pgfqpoint{0.489279in}{0.926504in}}%
\pgfpathlineto{\pgfqpoint{0.493374in}{0.917780in}}%
\pgfpathlineto{\pgfqpoint{0.497453in}{0.909463in}}%
\pgfpathlineto{\pgfqpoint{0.501515in}{0.901547in}}%
\pgfpathlineto{\pgfqpoint{0.505560in}{0.894023in}}%
\pgfpathlineto{\pgfqpoint{0.509084in}{0.882956in}}%
\pgfpathlineto{\pgfqpoint{0.513297in}{0.871935in}}%
\pgfpathlineto{\pgfqpoint{0.518197in}{0.860974in}}%
\pgfpathlineto{\pgfqpoint{0.523782in}{0.850082in}}%
\pgfpathclose%
\pgfusepath{fill}%
\end{pgfscope}%
\begin{pgfscope}%
\pgfpathrectangle{\pgfqpoint{0.041670in}{0.041670in}}{\pgfqpoint{2.216660in}{2.216660in}}%
\pgfusepath{clip}%
\pgfsetbuttcap%
\pgfsetroundjoin%
\definecolor{currentfill}{rgb}{0.487026,0.823929,0.312321}%
\pgfsetfillcolor{currentfill}%
\pgfsetlinewidth{0.000000pt}%
\definecolor{currentstroke}{rgb}{0.000000,0.000000,0.000000}%
\pgfsetstrokecolor{currentstroke}%
\pgfsetdash{}{0pt}%
\pgfpathmoveto{\pgfqpoint{1.264253in}{1.586133in}}%
\pgfpathlineto{\pgfqpoint{1.265965in}{1.582016in}}%
\pgfpathlineto{\pgfqpoint{1.267675in}{1.577811in}}%
\pgfpathlineto{\pgfqpoint{1.269382in}{1.573522in}}%
\pgfpathlineto{\pgfqpoint{1.271087in}{1.569148in}}%
\pgfpathlineto{\pgfqpoint{1.276652in}{1.567748in}}%
\pgfpathlineto{\pgfqpoint{1.282124in}{1.566266in}}%
\pgfpathlineto{\pgfqpoint{1.287497in}{1.564703in}}%
\pgfpathlineto{\pgfqpoint{1.292768in}{1.563060in}}%
\pgfpathlineto{\pgfqpoint{1.290655in}{1.567552in}}%
\pgfpathlineto{\pgfqpoint{1.288539in}{1.571960in}}%
\pgfpathlineto{\pgfqpoint{1.286421in}{1.576282in}}%
\pgfpathlineto{\pgfqpoint{1.284301in}{1.580517in}}%
\pgfpathlineto{\pgfqpoint{1.279428in}{1.582032in}}%
\pgfpathlineto{\pgfqpoint{1.274459in}{1.583474in}}%
\pgfpathlineto{\pgfqpoint{1.269399in}{1.584842in}}%
\pgfpathlineto{\pgfqpoint{1.264253in}{1.586133in}}%
\pgfpathclose%
\pgfusepath{fill}%
\end{pgfscope}%
\begin{pgfscope}%
\pgfpathrectangle{\pgfqpoint{0.041670in}{0.041670in}}{\pgfqpoint{2.216660in}{2.216660in}}%
\pgfusepath{clip}%
\pgfsetbuttcap%
\pgfsetroundjoin%
\definecolor{currentfill}{rgb}{0.133743,0.548535,0.553541}%
\pgfsetfillcolor{currentfill}%
\pgfsetlinewidth{0.000000pt}%
\definecolor{currentstroke}{rgb}{0.000000,0.000000,0.000000}%
\pgfsetstrokecolor{currentstroke}%
\pgfsetdash{}{0pt}%
\pgfpathmoveto{\pgfqpoint{0.857869in}{1.247467in}}%
\pgfpathlineto{\pgfqpoint{0.854148in}{1.239665in}}%
\pgfpathlineto{\pgfqpoint{0.850431in}{1.231855in}}%
\pgfpathlineto{\pgfqpoint{0.846716in}{1.224040in}}%
\pgfpathlineto{\pgfqpoint{0.843003in}{1.216222in}}%
\pgfpathlineto{\pgfqpoint{0.842693in}{1.221515in}}%
\pgfpathlineto{\pgfqpoint{0.842719in}{1.226805in}}%
\pgfpathlineto{\pgfqpoint{0.843081in}{1.232089in}}%
\pgfpathlineto{\pgfqpoint{0.843778in}{1.237360in}}%
\pgfpathlineto{\pgfqpoint{0.847470in}{1.244921in}}%
\pgfpathlineto{\pgfqpoint{0.851165in}{1.252481in}}%
\pgfpathlineto{\pgfqpoint{0.854864in}{1.260036in}}%
\pgfpathlineto{\pgfqpoint{0.858565in}{1.267583in}}%
\pgfpathlineto{\pgfqpoint{0.857910in}{1.262566in}}%
\pgfpathlineto{\pgfqpoint{0.857575in}{1.257538in}}%
\pgfpathlineto{\pgfqpoint{0.857561in}{1.252503in}}%
\pgfpathlineto{\pgfqpoint{0.857869in}{1.247467in}}%
\pgfpathclose%
\pgfusepath{fill}%
\end{pgfscope}%
\begin{pgfscope}%
\pgfpathrectangle{\pgfqpoint{0.041670in}{0.041670in}}{\pgfqpoint{2.216660in}{2.216660in}}%
\pgfusepath{clip}%
\pgfsetbuttcap%
\pgfsetroundjoin%
\definecolor{currentfill}{rgb}{0.263663,0.237631,0.518762}%
\pgfsetfillcolor{currentfill}%
\pgfsetlinewidth{0.000000pt}%
\definecolor{currentstroke}{rgb}{0.000000,0.000000,0.000000}%
\pgfsetstrokecolor{currentstroke}%
\pgfsetdash{}{0pt}%
\pgfpathmoveto{\pgfqpoint{1.605136in}{0.958250in}}%
\pgfpathlineto{\pgfqpoint{1.608701in}{0.951211in}}%
\pgfpathlineto{\pgfqpoint{1.612266in}{0.944264in}}%
\pgfpathlineto{\pgfqpoint{1.615831in}{0.937413in}}%
\pgfpathlineto{\pgfqpoint{1.619397in}{0.930660in}}%
\pgfpathlineto{\pgfqpoint{1.615415in}{0.923501in}}%
\pgfpathlineto{\pgfqpoint{1.610986in}{0.916403in}}%
\pgfpathlineto{\pgfqpoint{1.606113in}{0.909375in}}%
\pgfpathlineto{\pgfqpoint{1.600800in}{0.902424in}}%
\pgfpathlineto{\pgfqpoint{1.597372in}{0.909433in}}%
\pgfpathlineto{\pgfqpoint{1.593945in}{0.916541in}}%
\pgfpathlineto{\pgfqpoint{1.590518in}{0.923745in}}%
\pgfpathlineto{\pgfqpoint{1.587092in}{0.931039in}}%
\pgfpathlineto{\pgfqpoint{1.592244in}{0.937738in}}%
\pgfpathlineto{\pgfqpoint{1.596972in}{0.944510in}}%
\pgfpathlineto{\pgfqpoint{1.601270in}{0.951350in}}%
\pgfpathlineto{\pgfqpoint{1.605136in}{0.958250in}}%
\pgfpathclose%
\pgfusepath{fill}%
\end{pgfscope}%
\begin{pgfscope}%
\pgfpathrectangle{\pgfqpoint{0.041670in}{0.041670in}}{\pgfqpoint{2.216660in}{2.216660in}}%
\pgfusepath{clip}%
\pgfsetbuttcap%
\pgfsetroundjoin%
\definecolor{currentfill}{rgb}{0.281477,0.755203,0.432552}%
\pgfsetfillcolor{currentfill}%
\pgfsetlinewidth{0.000000pt}%
\definecolor{currentstroke}{rgb}{0.000000,0.000000,0.000000}%
\pgfsetstrokecolor{currentstroke}%
\pgfsetdash{}{0pt}%
\pgfpathmoveto{\pgfqpoint{0.970970in}{1.480283in}}%
\pgfpathlineto{\pgfqpoint{0.967579in}{1.474373in}}%
\pgfpathlineto{\pgfqpoint{0.964192in}{1.468399in}}%
\pgfpathlineto{\pgfqpoint{0.960809in}{1.462362in}}%
\pgfpathlineto{\pgfqpoint{0.957429in}{1.456264in}}%
\pgfpathlineto{\pgfqpoint{0.960923in}{1.459609in}}%
\pgfpathlineto{\pgfqpoint{0.964629in}{1.462899in}}%
\pgfpathlineto{\pgfqpoint{0.968544in}{1.466129in}}%
\pgfpathlineto{\pgfqpoint{0.972664in}{1.469297in}}%
\pgfpathlineto{\pgfqpoint{0.975807in}{1.475185in}}%
\pgfpathlineto{\pgfqpoint{0.978953in}{1.481013in}}%
\pgfpathlineto{\pgfqpoint{0.982102in}{1.486779in}}%
\pgfpathlineto{\pgfqpoint{0.985255in}{1.492480in}}%
\pgfpathlineto{\pgfqpoint{0.981391in}{1.489515in}}%
\pgfpathlineto{\pgfqpoint{0.977719in}{1.486492in}}%
\pgfpathlineto{\pgfqpoint{0.974244in}{1.483414in}}%
\pgfpathlineto{\pgfqpoint{0.970970in}{1.480283in}}%
\pgfpathclose%
\pgfusepath{fill}%
\end{pgfscope}%
\begin{pgfscope}%
\pgfpathrectangle{\pgfqpoint{0.041670in}{0.041670in}}{\pgfqpoint{2.216660in}{2.216660in}}%
\pgfusepath{clip}%
\pgfsetbuttcap%
\pgfsetroundjoin%
\definecolor{currentfill}{rgb}{0.212395,0.359683,0.551710}%
\pgfsetfillcolor{currentfill}%
\pgfsetlinewidth{0.000000pt}%
\definecolor{currentstroke}{rgb}{0.000000,0.000000,0.000000}%
\pgfsetstrokecolor{currentstroke}%
\pgfsetdash{}{0pt}%
\pgfpathmoveto{\pgfqpoint{0.800680in}{1.043169in}}%
\pgfpathlineto{\pgfqpoint{0.797135in}{1.035239in}}%
\pgfpathlineto{\pgfqpoint{0.793591in}{1.027360in}}%
\pgfpathlineto{\pgfqpoint{0.790049in}{1.019536in}}%
\pgfpathlineto{\pgfqpoint{0.786507in}{1.011769in}}%
\pgfpathlineto{\pgfqpoint{0.782919in}{1.018158in}}%
\pgfpathlineto{\pgfqpoint{0.779737in}{1.024596in}}%
\pgfpathlineto{\pgfqpoint{0.776962in}{1.031076in}}%
\pgfpathlineto{\pgfqpoint{0.774596in}{1.037592in}}%
\pgfpathlineto{\pgfqpoint{0.778232in}{1.045099in}}%
\pgfpathlineto{\pgfqpoint{0.781869in}{1.052664in}}%
\pgfpathlineto{\pgfqpoint{0.785508in}{1.060283in}}%
\pgfpathlineto{\pgfqpoint{0.789147in}{1.067954in}}%
\pgfpathlineto{\pgfqpoint{0.791441in}{1.061700in}}%
\pgfpathlineto{\pgfqpoint{0.794129in}{1.055480in}}%
\pgfpathlineto{\pgfqpoint{0.797209in}{1.049301in}}%
\pgfpathlineto{\pgfqpoint{0.800680in}{1.043169in}}%
\pgfpathclose%
\pgfusepath{fill}%
\end{pgfscope}%
\begin{pgfscope}%
\pgfpathrectangle{\pgfqpoint{0.041670in}{0.041670in}}{\pgfqpoint{2.216660in}{2.216660in}}%
\pgfusepath{clip}%
\pgfsetbuttcap%
\pgfsetroundjoin%
\definecolor{currentfill}{rgb}{0.274128,0.199721,0.498911}%
\pgfsetfillcolor{currentfill}%
\pgfsetlinewidth{0.000000pt}%
\definecolor{currentstroke}{rgb}{0.000000,0.000000,0.000000}%
\pgfsetstrokecolor{currentstroke}%
\pgfsetdash{}{0pt}%
\pgfpathmoveto{\pgfqpoint{0.764198in}{0.896316in}}%
\pgfpathlineto{\pgfqpoint{0.760808in}{0.889353in}}%
\pgfpathlineto{\pgfqpoint{0.757416in}{0.882496in}}%
\pgfpathlineto{\pgfqpoint{0.754024in}{0.875749in}}%
\pgfpathlineto{\pgfqpoint{0.750631in}{0.869115in}}%
\pgfpathlineto{\pgfqpoint{0.744758in}{0.876243in}}%
\pgfpathlineto{\pgfqpoint{0.739336in}{0.883457in}}%
\pgfpathlineto{\pgfqpoint{0.734369in}{0.890749in}}%
\pgfpathlineto{\pgfqpoint{0.729862in}{0.898111in}}%
\pgfpathlineto{\pgfqpoint{0.733406in}{0.904490in}}%
\pgfpathlineto{\pgfqpoint{0.736948in}{0.910983in}}%
\pgfpathlineto{\pgfqpoint{0.740490in}{0.917585in}}%
\pgfpathlineto{\pgfqpoint{0.744030in}{0.924293in}}%
\pgfpathlineto{\pgfqpoint{0.748410in}{0.917189in}}%
\pgfpathlineto{\pgfqpoint{0.753233in}{0.910153in}}%
\pgfpathlineto{\pgfqpoint{0.758497in}{0.903192in}}%
\pgfpathlineto{\pgfqpoint{0.764198in}{0.896316in}}%
\pgfpathclose%
\pgfusepath{fill}%
\end{pgfscope}%
\begin{pgfscope}%
\pgfpathrectangle{\pgfqpoint{0.041670in}{0.041670in}}{\pgfqpoint{2.216660in}{2.216660in}}%
\pgfusepath{clip}%
\pgfsetbuttcap%
\pgfsetroundjoin%
\definecolor{currentfill}{rgb}{0.344074,0.780029,0.397381}%
\pgfsetfillcolor{currentfill}%
\pgfsetlinewidth{0.000000pt}%
\definecolor{currentstroke}{rgb}{0.000000,0.000000,0.000000}%
\pgfsetstrokecolor{currentstroke}%
\pgfsetdash{}{0pt}%
\pgfpathmoveto{\pgfqpoint{0.997903in}{1.514597in}}%
\pgfpathlineto{\pgfqpoint{0.994736in}{1.509175in}}%
\pgfpathlineto{\pgfqpoint{0.991572in}{1.503680in}}%
\pgfpathlineto{\pgfqpoint{0.988412in}{1.498115in}}%
\pgfpathlineto{\pgfqpoint{0.985255in}{1.492480in}}%
\pgfpathlineto{\pgfqpoint{0.989308in}{1.495384in}}%
\pgfpathlineto{\pgfqpoint{0.993546in}{1.498224in}}%
\pgfpathlineto{\pgfqpoint{0.997964in}{1.500998in}}%
\pgfpathlineto{\pgfqpoint{1.002558in}{1.503703in}}%
\pgfpathlineto{\pgfqpoint{1.005429in}{1.509146in}}%
\pgfpathlineto{\pgfqpoint{1.008304in}{1.514521in}}%
\pgfpathlineto{\pgfqpoint{1.011182in}{1.519825in}}%
\pgfpathlineto{\pgfqpoint{1.014064in}{1.525057in}}%
\pgfpathlineto{\pgfqpoint{1.009772in}{1.522536in}}%
\pgfpathlineto{\pgfqpoint{1.005645in}{1.519950in}}%
\pgfpathlineto{\pgfqpoint{1.001688in}{1.517303in}}%
\pgfpathlineto{\pgfqpoint{0.997903in}{1.514597in}}%
\pgfpathclose%
\pgfusepath{fill}%
\end{pgfscope}%
\begin{pgfscope}%
\pgfpathrectangle{\pgfqpoint{0.041670in}{0.041670in}}{\pgfqpoint{2.216660in}{2.216660in}}%
\pgfusepath{clip}%
\pgfsetbuttcap%
\pgfsetroundjoin%
\definecolor{currentfill}{rgb}{0.120081,0.622161,0.534946}%
\pgfsetfillcolor{currentfill}%
\pgfsetlinewidth{0.000000pt}%
\definecolor{currentstroke}{rgb}{0.000000,0.000000,0.000000}%
\pgfsetstrokecolor{currentstroke}%
\pgfsetdash{}{0pt}%
\pgfpathmoveto{\pgfqpoint{0.888291in}{1.327370in}}%
\pgfpathlineto{\pgfqpoint{0.884564in}{1.319979in}}%
\pgfpathlineto{\pgfqpoint{0.880840in}{1.312559in}}%
\pgfpathlineto{\pgfqpoint{0.877120in}{1.305112in}}%
\pgfpathlineto{\pgfqpoint{0.873402in}{1.297643in}}%
\pgfpathlineto{\pgfqpoint{0.874321in}{1.302392in}}%
\pgfpathlineto{\pgfqpoint{0.875542in}{1.307121in}}%
\pgfpathlineto{\pgfqpoint{0.877064in}{1.311825in}}%
\pgfpathlineto{\pgfqpoint{0.878885in}{1.316500in}}%
\pgfpathlineto{\pgfqpoint{0.882526in}{1.323721in}}%
\pgfpathlineto{\pgfqpoint{0.886170in}{1.330919in}}%
\pgfpathlineto{\pgfqpoint{0.889817in}{1.338091in}}%
\pgfpathlineto{\pgfqpoint{0.893468in}{1.345235in}}%
\pgfpathlineto{\pgfqpoint{0.891746in}{1.340806in}}%
\pgfpathlineto{\pgfqpoint{0.890307in}{1.336349in}}%
\pgfpathlineto{\pgfqpoint{0.889155in}{1.331869in}}%
\pgfpathlineto{\pgfqpoint{0.888291in}{1.327370in}}%
\pgfpathclose%
\pgfusepath{fill}%
\end{pgfscope}%
\begin{pgfscope}%
\pgfpathrectangle{\pgfqpoint{0.041670in}{0.041670in}}{\pgfqpoint{2.216660in}{2.216660in}}%
\pgfusepath{clip}%
\pgfsetbuttcap%
\pgfsetroundjoin%
\definecolor{currentfill}{rgb}{0.487026,0.823929,0.312321}%
\pgfsetfillcolor{currentfill}%
\pgfsetlinewidth{0.000000pt}%
\definecolor{currentstroke}{rgb}{0.000000,0.000000,0.000000}%
\pgfsetstrokecolor{currentstroke}%
\pgfsetdash{}{0pt}%
\pgfpathmoveto{\pgfqpoint{1.071362in}{1.579109in}}%
\pgfpathlineto{\pgfqpoint{1.069154in}{1.574845in}}%
\pgfpathlineto{\pgfqpoint{1.066950in}{1.570493in}}%
\pgfpathlineto{\pgfqpoint{1.064747in}{1.566055in}}%
\pgfpathlineto{\pgfqpoint{1.062548in}{1.561534in}}%
\pgfpathlineto{\pgfqpoint{1.067722in}{1.563247in}}%
\pgfpathlineto{\pgfqpoint{1.073004in}{1.564881in}}%
\pgfpathlineto{\pgfqpoint{1.078389in}{1.566435in}}%
\pgfpathlineto{\pgfqpoint{1.083872in}{1.567908in}}%
\pgfpathlineto{\pgfqpoint{1.085670in}{1.572306in}}%
\pgfpathlineto{\pgfqpoint{1.087470in}{1.576620in}}%
\pgfpathlineto{\pgfqpoint{1.089273in}{1.580848in}}%
\pgfpathlineto{\pgfqpoint{1.091078in}{1.584989in}}%
\pgfpathlineto{\pgfqpoint{1.086009in}{1.583630in}}%
\pgfpathlineto{\pgfqpoint{1.081030in}{1.582196in}}%
\pgfpathlineto{\pgfqpoint{1.076146in}{1.580689in}}%
\pgfpathlineto{\pgfqpoint{1.071362in}{1.579109in}}%
\pgfpathclose%
\pgfusepath{fill}%
\end{pgfscope}%
\begin{pgfscope}%
\pgfpathrectangle{\pgfqpoint{0.041670in}{0.041670in}}{\pgfqpoint{2.216660in}{2.216660in}}%
\pgfusepath{clip}%
\pgfsetbuttcap%
\pgfsetroundjoin%
\definecolor{currentfill}{rgb}{0.412913,0.803041,0.357269}%
\pgfsetfillcolor{currentfill}%
\pgfsetlinewidth{0.000000pt}%
\definecolor{currentstroke}{rgb}{0.000000,0.000000,0.000000}%
\pgfsetstrokecolor{currentstroke}%
\pgfsetdash{}{0pt}%
\pgfpathmoveto{\pgfqpoint{1.312715in}{1.555722in}}%
\pgfpathlineto{\pgfqpoint{1.315201in}{1.551006in}}%
\pgfpathlineto{\pgfqpoint{1.317684in}{1.546209in}}%
\pgfpathlineto{\pgfqpoint{1.320165in}{1.541334in}}%
\pgfpathlineto{\pgfqpoint{1.322642in}{1.536383in}}%
\pgfpathlineto{\pgfqpoint{1.327674in}{1.534208in}}%
\pgfpathlineto{\pgfqpoint{1.332563in}{1.531958in}}%
\pgfpathlineto{\pgfqpoint{1.337306in}{1.529635in}}%
\pgfpathlineto{\pgfqpoint{1.341897in}{1.527242in}}%
\pgfpathlineto{\pgfqpoint{1.339081in}{1.532359in}}%
\pgfpathlineto{\pgfqpoint{1.336263in}{1.537400in}}%
\pgfpathlineto{\pgfqpoint{1.333441in}{1.542362in}}%
\pgfpathlineto{\pgfqpoint{1.330616in}{1.547243in}}%
\pgfpathlineto{\pgfqpoint{1.326349in}{1.549463in}}%
\pgfpathlineto{\pgfqpoint{1.321940in}{1.551618in}}%
\pgfpathlineto{\pgfqpoint{1.317394in}{1.553705in}}%
\pgfpathlineto{\pgfqpoint{1.312715in}{1.555722in}}%
\pgfpathclose%
\pgfusepath{fill}%
\end{pgfscope}%
\begin{pgfscope}%
\pgfpathrectangle{\pgfqpoint{0.041670in}{0.041670in}}{\pgfqpoint{2.216660in}{2.216660in}}%
\pgfusepath{clip}%
\pgfsetbuttcap%
\pgfsetroundjoin%
\definecolor{currentfill}{rgb}{0.565498,0.842430,0.262877}%
\pgfsetfillcolor{currentfill}%
\pgfsetlinewidth{0.000000pt}%
\definecolor{currentstroke}{rgb}{0.000000,0.000000,0.000000}%
\pgfsetstrokecolor{currentstroke}%
\pgfsetdash{}{0pt}%
\pgfpathmoveto{\pgfqpoint{1.158965in}{1.609873in}}%
\pgfpathlineto{\pgfqpoint{1.158499in}{1.606310in}}%
\pgfpathlineto{\pgfqpoint{1.158034in}{1.602652in}}%
\pgfpathlineto{\pgfqpoint{1.157570in}{1.598902in}}%
\pgfpathlineto{\pgfqpoint{1.157106in}{1.595061in}}%
\pgfpathlineto{\pgfqpoint{1.162880in}{1.595357in}}%
\pgfpathlineto{\pgfqpoint{1.168670in}{1.595567in}}%
\pgfpathlineto{\pgfqpoint{1.174471in}{1.595692in}}%
\pgfpathlineto{\pgfqpoint{1.180277in}{1.595730in}}%
\pgfpathlineto{\pgfqpoint{1.180271in}{1.599557in}}%
\pgfpathlineto{\pgfqpoint{1.180264in}{1.603293in}}%
\pgfpathlineto{\pgfqpoint{1.180258in}{1.606937in}}%
\pgfpathlineto{\pgfqpoint{1.180251in}{1.610487in}}%
\pgfpathlineto{\pgfqpoint{1.174917in}{1.610452in}}%
\pgfpathlineto{\pgfqpoint{1.169588in}{1.610338in}}%
\pgfpathlineto{\pgfqpoint{1.164269in}{1.610145in}}%
\pgfpathlineto{\pgfqpoint{1.158965in}{1.609873in}}%
\pgfpathclose%
\pgfusepath{fill}%
\end{pgfscope}%
\begin{pgfscope}%
\pgfpathrectangle{\pgfqpoint{0.041670in}{0.041670in}}{\pgfqpoint{2.216660in}{2.216660in}}%
\pgfusepath{clip}%
\pgfsetbuttcap%
\pgfsetroundjoin%
\definecolor{currentfill}{rgb}{0.565498,0.842430,0.262877}%
\pgfsetfillcolor{currentfill}%
\pgfsetlinewidth{0.000000pt}%
\definecolor{currentstroke}{rgb}{0.000000,0.000000,0.000000}%
\pgfsetstrokecolor{currentstroke}%
\pgfsetdash{}{0pt}%
\pgfpathmoveto{\pgfqpoint{1.180251in}{1.610487in}}%
\pgfpathlineto{\pgfqpoint{1.180258in}{1.606937in}}%
\pgfpathlineto{\pgfqpoint{1.180264in}{1.603293in}}%
\pgfpathlineto{\pgfqpoint{1.180271in}{1.599557in}}%
\pgfpathlineto{\pgfqpoint{1.180277in}{1.595730in}}%
\pgfpathlineto{\pgfqpoint{1.186084in}{1.595682in}}%
\pgfpathlineto{\pgfqpoint{1.191884in}{1.595548in}}%
\pgfpathlineto{\pgfqpoint{1.197673in}{1.595328in}}%
\pgfpathlineto{\pgfqpoint{1.203444in}{1.595022in}}%
\pgfpathlineto{\pgfqpoint{1.202967in}{1.598864in}}%
\pgfpathlineto{\pgfqpoint{1.202490in}{1.602615in}}%
\pgfpathlineto{\pgfqpoint{1.202012in}{1.606274in}}%
\pgfpathlineto{\pgfqpoint{1.201533in}{1.609838in}}%
\pgfpathlineto{\pgfqpoint{1.196231in}{1.610118in}}%
\pgfpathlineto{\pgfqpoint{1.190913in}{1.610320in}}%
\pgfpathlineto{\pgfqpoint{1.185585in}{1.610443in}}%
\pgfpathlineto{\pgfqpoint{1.180251in}{1.610487in}}%
\pgfpathclose%
\pgfusepath{fill}%
\end{pgfscope}%
\begin{pgfscope}%
\pgfpathrectangle{\pgfqpoint{0.041670in}{0.041670in}}{\pgfqpoint{2.216660in}{2.216660in}}%
\pgfusepath{clip}%
\pgfsetbuttcap%
\pgfsetroundjoin%
\definecolor{currentfill}{rgb}{0.220124,0.725509,0.466226}%
\pgfsetfillcolor{currentfill}%
\pgfsetlinewidth{0.000000pt}%
\definecolor{currentstroke}{rgb}{0.000000,0.000000,0.000000}%
\pgfsetstrokecolor{currentstroke}%
\pgfsetdash{}{0pt}%
\pgfpathmoveto{\pgfqpoint{0.945658in}{1.442384in}}%
\pgfpathlineto{\pgfqpoint{0.942096in}{1.436003in}}%
\pgfpathlineto{\pgfqpoint{0.938537in}{1.429566in}}%
\pgfpathlineto{\pgfqpoint{0.934982in}{1.423075in}}%
\pgfpathlineto{\pgfqpoint{0.931431in}{1.416532in}}%
\pgfpathlineto{\pgfqpoint{0.934203in}{1.420299in}}%
\pgfpathlineto{\pgfqpoint{0.937216in}{1.424020in}}%
\pgfpathlineto{\pgfqpoint{0.940465in}{1.427692in}}%
\pgfpathlineto{\pgfqpoint{0.943947in}{1.431310in}}%
\pgfpathlineto{\pgfqpoint{0.947312in}{1.437628in}}%
\pgfpathlineto{\pgfqpoint{0.950681in}{1.443895in}}%
\pgfpathlineto{\pgfqpoint{0.954053in}{1.450107in}}%
\pgfpathlineto{\pgfqpoint{0.957429in}{1.456264in}}%
\pgfpathlineto{\pgfqpoint{0.954153in}{1.452865in}}%
\pgfpathlineto{\pgfqpoint{0.951096in}{1.449417in}}%
\pgfpathlineto{\pgfqpoint{0.948264in}{1.445922in}}%
\pgfpathlineto{\pgfqpoint{0.945658in}{1.442384in}}%
\pgfpathclose%
\pgfusepath{fill}%
\end{pgfscope}%
\begin{pgfscope}%
\pgfpathrectangle{\pgfqpoint{0.041670in}{0.041670in}}{\pgfqpoint{2.216660in}{2.216660in}}%
\pgfusepath{clip}%
\pgfsetbuttcap%
\pgfsetroundjoin%
\definecolor{currentfill}{rgb}{0.163625,0.471133,0.558148}%
\pgfsetfillcolor{currentfill}%
\pgfsetlinewidth{0.000000pt}%
\definecolor{currentstroke}{rgb}{0.000000,0.000000,0.000000}%
\pgfsetstrokecolor{currentstroke}%
\pgfsetdash{}{0pt}%
\pgfpathmoveto{\pgfqpoint{0.832951in}{1.162874in}}%
\pgfpathlineto{\pgfqpoint{0.829290in}{1.154830in}}%
\pgfpathlineto{\pgfqpoint{0.825631in}{1.146799in}}%
\pgfpathlineto{\pgfqpoint{0.821975in}{1.138786in}}%
\pgfpathlineto{\pgfqpoint{0.818320in}{1.130794in}}%
\pgfpathlineto{\pgfqpoint{0.816539in}{1.136556in}}%
\pgfpathlineto{\pgfqpoint{0.815125in}{1.142338in}}%
\pgfpathlineto{\pgfqpoint{0.814078in}{1.148136in}}%
\pgfpathlineto{\pgfqpoint{0.813398in}{1.153942in}}%
\pgfpathlineto{\pgfqpoint{0.817090in}{1.161675in}}%
\pgfpathlineto{\pgfqpoint{0.820785in}{1.169428in}}%
\pgfpathlineto{\pgfqpoint{0.824481in}{1.177200in}}%
\pgfpathlineto{\pgfqpoint{0.828181in}{1.184986in}}%
\pgfpathlineto{\pgfqpoint{0.828845in}{1.179440in}}%
\pgfpathlineto{\pgfqpoint{0.829862in}{1.173902in}}%
\pgfpathlineto{\pgfqpoint{0.831231in}{1.168378in}}%
\pgfpathlineto{\pgfqpoint{0.832951in}{1.162874in}}%
\pgfpathclose%
\pgfusepath{fill}%
\end{pgfscope}%
\begin{pgfscope}%
\pgfpathrectangle{\pgfqpoint{0.041670in}{0.041670in}}{\pgfqpoint{2.216660in}{2.216660in}}%
\pgfusepath{clip}%
\pgfsetbuttcap%
\pgfsetroundjoin%
\definecolor{currentfill}{rgb}{0.166383,0.690856,0.496502}%
\pgfsetfillcolor{currentfill}%
\pgfsetlinewidth{0.000000pt}%
\definecolor{currentstroke}{rgb}{0.000000,0.000000,0.000000}%
\pgfsetstrokecolor{currentstroke}%
\pgfsetdash{}{0pt}%
\pgfpathmoveto{\pgfqpoint{1.426026in}{1.419883in}}%
\pgfpathlineto{\pgfqpoint{1.429537in}{1.413341in}}%
\pgfpathlineto{\pgfqpoint{1.433044in}{1.406752in}}%
\pgfpathlineto{\pgfqpoint{1.436547in}{1.400118in}}%
\pgfpathlineto{\pgfqpoint{1.440047in}{1.393442in}}%
\pgfpathlineto{\pgfqpoint{1.442957in}{1.389439in}}%
\pgfpathlineto{\pgfqpoint{1.445610in}{1.385391in}}%
\pgfpathlineto{\pgfqpoint{1.448003in}{1.381301in}}%
\pgfpathlineto{\pgfqpoint{1.450132in}{1.377175in}}%
\pgfpathlineto{\pgfqpoint{1.446488in}{1.384087in}}%
\pgfpathlineto{\pgfqpoint{1.442841in}{1.390956in}}%
\pgfpathlineto{\pgfqpoint{1.439189in}{1.397780in}}%
\pgfpathlineto{\pgfqpoint{1.435535in}{1.404556in}}%
\pgfpathlineto{\pgfqpoint{1.433529in}{1.408443in}}%
\pgfpathlineto{\pgfqpoint{1.431273in}{1.412296in}}%
\pgfpathlineto{\pgfqpoint{1.428772in}{1.416111in}}%
\pgfpathlineto{\pgfqpoint{1.426026in}{1.419883in}}%
\pgfpathclose%
\pgfusepath{fill}%
\end{pgfscope}%
\begin{pgfscope}%
\pgfpathrectangle{\pgfqpoint{0.041670in}{0.041670in}}{\pgfqpoint{2.216660in}{2.216660in}}%
\pgfusepath{clip}%
\pgfsetbuttcap%
\pgfsetroundjoin%
\definecolor{currentfill}{rgb}{0.565498,0.842430,0.262877}%
\pgfsetfillcolor{currentfill}%
\pgfsetlinewidth{0.000000pt}%
\definecolor{currentstroke}{rgb}{0.000000,0.000000,0.000000}%
\pgfsetstrokecolor{currentstroke}%
\pgfsetdash{}{0pt}%
\pgfpathmoveto{\pgfqpoint{1.138001in}{1.608007in}}%
\pgfpathlineto{\pgfqpoint{1.137070in}{1.604401in}}%
\pgfpathlineto{\pgfqpoint{1.136140in}{1.600700in}}%
\pgfpathlineto{\pgfqpoint{1.135211in}{1.596908in}}%
\pgfpathlineto{\pgfqpoint{1.134283in}{1.593024in}}%
\pgfpathlineto{\pgfqpoint{1.139937in}{1.593660in}}%
\pgfpathlineto{\pgfqpoint{1.145629in}{1.594212in}}%
\pgfpathlineto{\pgfqpoint{1.151354in}{1.594679in}}%
\pgfpathlineto{\pgfqpoint{1.157106in}{1.595061in}}%
\pgfpathlineto{\pgfqpoint{1.157570in}{1.598902in}}%
\pgfpathlineto{\pgfqpoint{1.158034in}{1.602652in}}%
\pgfpathlineto{\pgfqpoint{1.158499in}{1.606310in}}%
\pgfpathlineto{\pgfqpoint{1.158965in}{1.609873in}}%
\pgfpathlineto{\pgfqpoint{1.153681in}{1.609523in}}%
\pgfpathlineto{\pgfqpoint{1.148422in}{1.609095in}}%
\pgfpathlineto{\pgfqpoint{1.143194in}{1.608590in}}%
\pgfpathlineto{\pgfqpoint{1.138001in}{1.608007in}}%
\pgfpathclose%
\pgfusepath{fill}%
\end{pgfscope}%
\begin{pgfscope}%
\pgfpathrectangle{\pgfqpoint{0.041670in}{0.041670in}}{\pgfqpoint{2.216660in}{2.216660in}}%
\pgfusepath{clip}%
\pgfsetbuttcap%
\pgfsetroundjoin%
\definecolor{currentfill}{rgb}{0.565498,0.842430,0.262877}%
\pgfsetfillcolor{currentfill}%
\pgfsetlinewidth{0.000000pt}%
\definecolor{currentstroke}{rgb}{0.000000,0.000000,0.000000}%
\pgfsetstrokecolor{currentstroke}%
\pgfsetdash{}{0pt}%
\pgfpathmoveto{\pgfqpoint{1.201533in}{1.609838in}}%
\pgfpathlineto{\pgfqpoint{1.202012in}{1.606274in}}%
\pgfpathlineto{\pgfqpoint{1.202490in}{1.602615in}}%
\pgfpathlineto{\pgfqpoint{1.202967in}{1.598864in}}%
\pgfpathlineto{\pgfqpoint{1.203444in}{1.595022in}}%
\pgfpathlineto{\pgfqpoint{1.209194in}{1.594631in}}%
\pgfpathlineto{\pgfqpoint{1.214915in}{1.594155in}}%
\pgfpathlineto{\pgfqpoint{1.220603in}{1.593593in}}%
\pgfpathlineto{\pgfqpoint{1.226252in}{1.592948in}}%
\pgfpathlineto{\pgfqpoint{1.225311in}{1.596833in}}%
\pgfpathlineto{\pgfqpoint{1.224370in}{1.600628in}}%
\pgfpathlineto{\pgfqpoint{1.223427in}{1.604330in}}%
\pgfpathlineto{\pgfqpoint{1.222483in}{1.607937in}}%
\pgfpathlineto{\pgfqpoint{1.217295in}{1.608529in}}%
\pgfpathlineto{\pgfqpoint{1.212070in}{1.609043in}}%
\pgfpathlineto{\pgfqpoint{1.206814in}{1.609480in}}%
\pgfpathlineto{\pgfqpoint{1.201533in}{1.609838in}}%
\pgfpathclose%
\pgfusepath{fill}%
\end{pgfscope}%
\begin{pgfscope}%
\pgfpathrectangle{\pgfqpoint{0.041670in}{0.041670in}}{\pgfqpoint{2.216660in}{2.216660in}}%
\pgfusepath{clip}%
\pgfsetbuttcap%
\pgfsetroundjoin%
\definecolor{currentfill}{rgb}{0.195860,0.395433,0.555276}%
\pgfsetfillcolor{currentfill}%
\pgfsetlinewidth{0.000000pt}%
\definecolor{currentstroke}{rgb}{0.000000,0.000000,0.000000}%
\pgfsetstrokecolor{currentstroke}%
\pgfsetdash{}{0pt}%
\pgfpathmoveto{\pgfqpoint{1.557844in}{1.104440in}}%
\pgfpathlineto{\pgfqpoint{1.561503in}{1.096653in}}%
\pgfpathlineto{\pgfqpoint{1.565160in}{1.088905in}}%
\pgfpathlineto{\pgfqpoint{1.568815in}{1.081198in}}%
\pgfpathlineto{\pgfqpoint{1.572469in}{1.073537in}}%
\pgfpathlineto{\pgfqpoint{1.570527in}{1.067258in}}%
\pgfpathlineto{\pgfqpoint{1.568190in}{1.061007in}}%
\pgfpathlineto{\pgfqpoint{1.565458in}{1.054791in}}%
\pgfpathlineto{\pgfqpoint{1.562334in}{1.048617in}}%
\pgfpathlineto{\pgfqpoint{1.558762in}{1.056538in}}%
\pgfpathlineto{\pgfqpoint{1.555189in}{1.064503in}}%
\pgfpathlineto{\pgfqpoint{1.551614in}{1.072511in}}%
\pgfpathlineto{\pgfqpoint{1.548038in}{1.080557in}}%
\pgfpathlineto{\pgfqpoint{1.551057in}{1.086473in}}%
\pgfpathlineto{\pgfqpoint{1.553699in}{1.092430in}}%
\pgfpathlineto{\pgfqpoint{1.555962in}{1.098421in}}%
\pgfpathlineto{\pgfqpoint{1.557844in}{1.104440in}}%
\pgfpathclose%
\pgfusepath{fill}%
\end{pgfscope}%
\begin{pgfscope}%
\pgfpathrectangle{\pgfqpoint{0.041670in}{0.041670in}}{\pgfqpoint{2.216660in}{2.216660in}}%
\pgfusepath{clip}%
\pgfsetbuttcap%
\pgfsetroundjoin%
\definecolor{currentfill}{rgb}{0.412913,0.803041,0.357269}%
\pgfsetfillcolor{currentfill}%
\pgfsetlinewidth{0.000000pt}%
\definecolor{currentstroke}{rgb}{0.000000,0.000000,0.000000}%
\pgfsetstrokecolor{currentstroke}%
\pgfsetdash{}{0pt}%
\pgfpathmoveto{\pgfqpoint{1.025623in}{1.545217in}}%
\pgfpathlineto{\pgfqpoint{1.022728in}{1.540295in}}%
\pgfpathlineto{\pgfqpoint{1.019837in}{1.535294in}}%
\pgfpathlineto{\pgfqpoint{1.016948in}{1.530214in}}%
\pgfpathlineto{\pgfqpoint{1.014064in}{1.525057in}}%
\pgfpathlineto{\pgfqpoint{1.018515in}{1.527511in}}%
\pgfpathlineto{\pgfqpoint{1.023123in}{1.529897in}}%
\pgfpathlineto{\pgfqpoint{1.027882in}{1.532212in}}%
\pgfpathlineto{\pgfqpoint{1.032788in}{1.534453in}}%
\pgfpathlineto{\pgfqpoint{1.035344in}{1.539440in}}%
\pgfpathlineto{\pgfqpoint{1.037903in}{1.544350in}}%
\pgfpathlineto{\pgfqpoint{1.040464in}{1.549181in}}%
\pgfpathlineto{\pgfqpoint{1.043029in}{1.553933in}}%
\pgfpathlineto{\pgfqpoint{1.038468in}{1.551853in}}%
\pgfpathlineto{\pgfqpoint{1.034044in}{1.549706in}}%
\pgfpathlineto{\pgfqpoint{1.029761in}{1.547493in}}%
\pgfpathlineto{\pgfqpoint{1.025623in}{1.545217in}}%
\pgfpathclose%
\pgfusepath{fill}%
\end{pgfscope}%
\begin{pgfscope}%
\pgfpathrectangle{\pgfqpoint{0.041670in}{0.041670in}}{\pgfqpoint{2.216660in}{2.216660in}}%
\pgfusepath{clip}%
\pgfsetbuttcap%
\pgfsetroundjoin%
\definecolor{currentfill}{rgb}{0.201239,0.383670,0.554294}%
\pgfsetfillcolor{currentfill}%
\pgfsetlinewidth{0.000000pt}%
\definecolor{currentstroke}{rgb}{0.000000,0.000000,0.000000}%
\pgfsetstrokecolor{currentstroke}%
\pgfsetdash{}{0pt}%
\pgfpathmoveto{\pgfqpoint{0.455831in}{1.011784in}}%
\pgfpathlineto{\pgfqpoint{0.451557in}{1.024491in}}%
\pgfpathlineto{\pgfqpoint{0.447261in}{1.037679in}}%
\pgfpathlineto{\pgfqpoint{0.442942in}{1.051355in}}%
\pgfpathlineto{\pgfqpoint{0.438600in}{1.065528in}}%
\pgfpathlineto{\pgfqpoint{0.435596in}{1.077441in}}%
\pgfpathlineto{\pgfqpoint{0.433349in}{1.089375in}}%
\pgfpathlineto{\pgfqpoint{0.431859in}{1.101318in}}%
\pgfpathlineto{\pgfqpoint{0.431122in}{1.113259in}}%
\pgfpathlineto{\pgfqpoint{0.435478in}{1.098904in}}%
\pgfpathlineto{\pgfqpoint{0.439812in}{1.085043in}}%
\pgfpathlineto{\pgfqpoint{0.444123in}{1.071667in}}%
\pgfpathlineto{\pgfqpoint{0.448411in}{1.058770in}}%
\pgfpathlineto{\pgfqpoint{0.449160in}{1.047014in}}%
\pgfpathlineto{\pgfqpoint{0.450645in}{1.035256in}}%
\pgfpathlineto{\pgfqpoint{0.452868in}{1.023509in}}%
\pgfpathlineto{\pgfqpoint{0.455831in}{1.011784in}}%
\pgfpathclose%
\pgfusepath{fill}%
\end{pgfscope}%
\begin{pgfscope}%
\pgfpathrectangle{\pgfqpoint{0.041670in}{0.041670in}}{\pgfqpoint{2.216660in}{2.216660in}}%
\pgfusepath{clip}%
\pgfsetbuttcap%
\pgfsetroundjoin%
\definecolor{currentfill}{rgb}{0.260571,0.246922,0.522828}%
\pgfsetfillcolor{currentfill}%
\pgfsetlinewidth{0.000000pt}%
\definecolor{currentstroke}{rgb}{0.000000,0.000000,0.000000}%
\pgfsetstrokecolor{currentstroke}%
\pgfsetdash{}{0pt}%
\pgfpathmoveto{\pgfqpoint{1.873225in}{0.936564in}}%
\pgfpathlineto{\pgfqpoint{1.877348in}{0.945748in}}%
\pgfpathlineto{\pgfqpoint{1.881490in}{0.955352in}}%
\pgfpathlineto{\pgfqpoint{1.885650in}{0.965384in}}%
\pgfpathlineto{\pgfqpoint{1.889829in}{0.975850in}}%
\pgfpathlineto{\pgfqpoint{1.886828in}{0.964326in}}%
\pgfpathlineto{\pgfqpoint{1.883104in}{0.952839in}}%
\pgfpathlineto{\pgfqpoint{1.878657in}{0.941401in}}%
\pgfpathlineto{\pgfqpoint{1.873489in}{0.930022in}}%
\pgfpathlineto{\pgfqpoint{1.869380in}{0.919761in}}%
\pgfpathlineto{\pgfqpoint{1.865290in}{0.909936in}}%
\pgfpathlineto{\pgfqpoint{1.861218in}{0.900541in}}%
\pgfpathlineto{\pgfqpoint{1.857165in}{0.891568in}}%
\pgfpathlineto{\pgfqpoint{1.862237in}{0.902739in}}%
\pgfpathlineto{\pgfqpoint{1.866605in}{0.913969in}}%
\pgfpathlineto{\pgfqpoint{1.870268in}{0.925248in}}%
\pgfpathlineto{\pgfqpoint{1.873225in}{0.936564in}}%
\pgfpathclose%
\pgfusepath{fill}%
\end{pgfscope}%
\begin{pgfscope}%
\pgfpathrectangle{\pgfqpoint{0.041670in}{0.041670in}}{\pgfqpoint{2.216660in}{2.216660in}}%
\pgfusepath{clip}%
\pgfsetbuttcap%
\pgfsetroundjoin%
\definecolor{currentfill}{rgb}{0.565498,0.842430,0.262877}%
\pgfsetfillcolor{currentfill}%
\pgfsetlinewidth{0.000000pt}%
\definecolor{currentstroke}{rgb}{0.000000,0.000000,0.000000}%
\pgfsetstrokecolor{currentstroke}%
\pgfsetdash{}{0pt}%
\pgfpathmoveto{\pgfqpoint{1.222483in}{1.607937in}}%
\pgfpathlineto{\pgfqpoint{1.223427in}{1.604330in}}%
\pgfpathlineto{\pgfqpoint{1.224370in}{1.600628in}}%
\pgfpathlineto{\pgfqpoint{1.225311in}{1.596833in}}%
\pgfpathlineto{\pgfqpoint{1.226252in}{1.592948in}}%
\pgfpathlineto{\pgfqpoint{1.231856in}{1.592219in}}%
\pgfpathlineto{\pgfqpoint{1.237411in}{1.591407in}}%
\pgfpathlineto{\pgfqpoint{1.242911in}{1.590513in}}%
\pgfpathlineto{\pgfqpoint{1.241632in}{1.594449in}}%
\pgfpathlineto{\pgfqpoint{1.240351in}{1.598295in}}%
\pgfpathlineto{\pgfqpoint{1.239069in}{1.602047in}}%
\pgfpathlineto{\pgfqpoint{1.237785in}{1.605706in}}%
\pgfpathlineto{\pgfqpoint{1.232733in}{1.606525in}}%
\pgfpathlineto{\pgfqpoint{1.227631in}{1.607269in}}%
\pgfpathlineto{\pgfqpoint{1.222483in}{1.607937in}}%
\pgfpathclose%
\pgfusepath{fill}%
\end{pgfscope}%
\begin{pgfscope}%
\pgfpathrectangle{\pgfqpoint{0.041670in}{0.041670in}}{\pgfqpoint{2.216660in}{2.216660in}}%
\pgfusepath{clip}%
\pgfsetbuttcap%
\pgfsetroundjoin%
\definecolor{currentfill}{rgb}{0.122606,0.585371,0.546557}%
\pgfsetfillcolor{currentfill}%
\pgfsetlinewidth{0.000000pt}%
\definecolor{currentstroke}{rgb}{0.000000,0.000000,0.000000}%
\pgfsetstrokecolor{currentstroke}%
\pgfsetdash{}{0pt}%
\pgfpathmoveto{\pgfqpoint{1.485706in}{1.301865in}}%
\pgfpathlineto{\pgfqpoint{1.489408in}{1.294430in}}%
\pgfpathlineto{\pgfqpoint{1.493106in}{1.286978in}}%
\pgfpathlineto{\pgfqpoint{1.496802in}{1.279510in}}%
\pgfpathlineto{\pgfqpoint{1.500494in}{1.272029in}}%
\pgfpathlineto{\pgfqpoint{1.501433in}{1.267026in}}%
\pgfpathlineto{\pgfqpoint{1.502053in}{1.262008in}}%
\pgfpathlineto{\pgfqpoint{1.502352in}{1.256979in}}%
\pgfpathlineto{\pgfqpoint{1.502330in}{1.251943in}}%
\pgfpathlineto{\pgfqpoint{1.498606in}{1.259678in}}%
\pgfpathlineto{\pgfqpoint{1.494879in}{1.267400in}}%
\pgfpathlineto{\pgfqpoint{1.491148in}{1.275106in}}%
\pgfpathlineto{\pgfqpoint{1.487415in}{1.282793in}}%
\pgfpathlineto{\pgfqpoint{1.487446in}{1.287573in}}%
\pgfpathlineto{\pgfqpoint{1.487171in}{1.292349in}}%
\pgfpathlineto{\pgfqpoint{1.486591in}{1.297114in}}%
\pgfpathlineto{\pgfqpoint{1.485706in}{1.301865in}}%
\pgfpathclose%
\pgfusepath{fill}%
\end{pgfscope}%
\begin{pgfscope}%
\pgfpathrectangle{\pgfqpoint{0.041670in}{0.041670in}}{\pgfqpoint{2.216660in}{2.216660in}}%
\pgfusepath{clip}%
\pgfsetbuttcap%
\pgfsetroundjoin%
\definecolor{currentfill}{rgb}{0.565498,0.842430,0.262877}%
\pgfsetfillcolor{currentfill}%
\pgfsetlinewidth{0.000000pt}%
\definecolor{currentstroke}{rgb}{0.000000,0.000000,0.000000}%
\pgfsetstrokecolor{currentstroke}%
\pgfsetdash{}{0pt}%
\pgfpathmoveto{\pgfqpoint{1.117682in}{1.604915in}}%
\pgfpathlineto{\pgfqpoint{1.116299in}{1.601238in}}%
\pgfpathlineto{\pgfqpoint{1.114918in}{1.597468in}}%
\pgfpathlineto{\pgfqpoint{1.113538in}{1.593604in}}%
\pgfpathlineto{\pgfqpoint{1.112161in}{1.589650in}}%
\pgfpathlineto{\pgfqpoint{1.117607in}{1.590617in}}%
\pgfpathlineto{\pgfqpoint{1.123113in}{1.591502in}}%
\pgfpathlineto{\pgfqpoint{1.128674in}{1.592304in}}%
\pgfpathlineto{\pgfqpoint{1.134283in}{1.593024in}}%
\pgfpathlineto{\pgfqpoint{1.135211in}{1.596908in}}%
\pgfpathlineto{\pgfqpoint{1.136140in}{1.600700in}}%
\pgfpathlineto{\pgfqpoint{1.137070in}{1.604401in}}%
\pgfpathlineto{\pgfqpoint{1.138001in}{1.608007in}}%
\pgfpathlineto{\pgfqpoint{1.132848in}{1.607347in}}%
\pgfpathlineto{\pgfqpoint{1.127741in}{1.606611in}}%
\pgfpathlineto{\pgfqpoint{1.122684in}{1.605800in}}%
\pgfpathlineto{\pgfqpoint{1.117682in}{1.604915in}}%
\pgfpathclose%
\pgfusepath{fill}%
\end{pgfscope}%
\begin{pgfscope}%
\pgfpathrectangle{\pgfqpoint{0.041670in}{0.041670in}}{\pgfqpoint{2.216660in}{2.216660in}}%
\pgfusepath{clip}%
\pgfsetbuttcap%
\pgfsetroundjoin%
\definecolor{currentfill}{rgb}{0.263663,0.237631,0.518762}%
\pgfsetfillcolor{currentfill}%
\pgfsetlinewidth{0.000000pt}%
\definecolor{currentstroke}{rgb}{0.000000,0.000000,0.000000}%
\pgfsetstrokecolor{currentstroke}%
\pgfsetdash{}{0pt}%
\pgfpathmoveto{\pgfqpoint{0.777752in}{0.925154in}}%
\pgfpathlineto{\pgfqpoint{0.774364in}{0.917803in}}%
\pgfpathlineto{\pgfqpoint{0.770976in}{0.910544in}}%
\pgfpathlineto{\pgfqpoint{0.767587in}{0.903381in}}%
\pgfpathlineto{\pgfqpoint{0.764198in}{0.896316in}}%
\pgfpathlineto{\pgfqpoint{0.758497in}{0.903192in}}%
\pgfpathlineto{\pgfqpoint{0.753233in}{0.910153in}}%
\pgfpathlineto{\pgfqpoint{0.748410in}{0.917189in}}%
\pgfpathlineto{\pgfqpoint{0.744030in}{0.924293in}}%
\pgfpathlineto{\pgfqpoint{0.747571in}{0.931104in}}%
\pgfpathlineto{\pgfqpoint{0.751110in}{0.938013in}}%
\pgfpathlineto{\pgfqpoint{0.754650in}{0.945018in}}%
\pgfpathlineto{\pgfqpoint{0.758189in}{0.952114in}}%
\pgfpathlineto{\pgfqpoint{0.762439in}{0.945267in}}%
\pgfpathlineto{\pgfqpoint{0.767119in}{0.938487in}}%
\pgfpathlineto{\pgfqpoint{0.772225in}{0.931780in}}%
\pgfpathlineto{\pgfqpoint{0.777752in}{0.925154in}}%
\pgfpathclose%
\pgfusepath{fill}%
\end{pgfscope}%
\begin{pgfscope}%
\pgfpathrectangle{\pgfqpoint{0.041670in}{0.041670in}}{\pgfqpoint{2.216660in}{2.216660in}}%
\pgfusepath{clip}%
\pgfsetbuttcap%
\pgfsetroundjoin%
\definecolor{currentfill}{rgb}{0.166383,0.690856,0.496502}%
\pgfsetfillcolor{currentfill}%
\pgfsetlinewidth{0.000000pt}%
\definecolor{currentstroke}{rgb}{0.000000,0.000000,0.000000}%
\pgfsetstrokecolor{currentstroke}%
\pgfsetdash{}{0pt}%
\pgfpathmoveto{\pgfqpoint{0.922803in}{1.401074in}}%
\pgfpathlineto{\pgfqpoint{0.919124in}{1.394244in}}%
\pgfpathlineto{\pgfqpoint{0.915448in}{1.387367in}}%
\pgfpathlineto{\pgfqpoint{0.911776in}{1.380445in}}%
\pgfpathlineto{\pgfqpoint{0.908108in}{1.373479in}}%
\pgfpathlineto{\pgfqpoint{0.910001in}{1.377635in}}%
\pgfpathlineto{\pgfqpoint{0.912160in}{1.381758in}}%
\pgfpathlineto{\pgfqpoint{0.914581in}{1.385842in}}%
\pgfpathlineto{\pgfqpoint{0.917263in}{1.389886in}}%
\pgfpathlineto{\pgfqpoint{0.920800in}{1.396614in}}%
\pgfpathlineto{\pgfqpoint{0.924340in}{1.403299in}}%
\pgfpathlineto{\pgfqpoint{0.927884in}{1.409939in}}%
\pgfpathlineto{\pgfqpoint{0.931431in}{1.416532in}}%
\pgfpathlineto{\pgfqpoint{0.928902in}{1.412722in}}%
\pgfpathlineto{\pgfqpoint{0.926620in}{1.408873in}}%
\pgfpathlineto{\pgfqpoint{0.924586in}{1.404989in}}%
\pgfpathlineto{\pgfqpoint{0.922803in}{1.401074in}}%
\pgfpathclose%
\pgfusepath{fill}%
\end{pgfscope}%
\begin{pgfscope}%
\pgfpathrectangle{\pgfqpoint{0.041670in}{0.041670in}}{\pgfqpoint{2.216660in}{2.216660in}}%
\pgfusepath{clip}%
\pgfsetbuttcap%
\pgfsetroundjoin%
\definecolor{currentfill}{rgb}{0.248629,0.278775,0.534556}%
\pgfsetfillcolor{currentfill}%
\pgfsetlinewidth{0.000000pt}%
\definecolor{currentstroke}{rgb}{0.000000,0.000000,0.000000}%
\pgfsetstrokecolor{currentstroke}%
\pgfsetdash{}{0pt}%
\pgfpathmoveto{\pgfqpoint{1.590877in}{0.987252in}}%
\pgfpathlineto{\pgfqpoint{1.594442in}{0.979882in}}%
\pgfpathlineto{\pgfqpoint{1.598007in}{0.972589in}}%
\pgfpathlineto{\pgfqpoint{1.601572in}{0.965377in}}%
\pgfpathlineto{\pgfqpoint{1.605136in}{0.958250in}}%
\pgfpathlineto{\pgfqpoint{1.601270in}{0.951350in}}%
\pgfpathlineto{\pgfqpoint{1.596972in}{0.944510in}}%
\pgfpathlineto{\pgfqpoint{1.592244in}{0.937738in}}%
\pgfpathlineto{\pgfqpoint{1.587092in}{0.931039in}}%
\pgfpathlineto{\pgfqpoint{1.583665in}{0.938422in}}%
\pgfpathlineto{\pgfqpoint{1.580239in}{0.945890in}}%
\pgfpathlineto{\pgfqpoint{1.576813in}{0.953438in}}%
\pgfpathlineto{\pgfqpoint{1.573387in}{0.961064in}}%
\pgfpathlineto{\pgfqpoint{1.578379in}{0.967510in}}%
\pgfpathlineto{\pgfqpoint{1.582960in}{0.974028in}}%
\pgfpathlineto{\pgfqpoint{1.587127in}{0.980611in}}%
\pgfpathlineto{\pgfqpoint{1.590877in}{0.987252in}}%
\pgfpathclose%
\pgfusepath{fill}%
\end{pgfscope}%
\begin{pgfscope}%
\pgfpathrectangle{\pgfqpoint{0.041670in}{0.041670in}}{\pgfqpoint{2.216660in}{2.216660in}}%
\pgfusepath{clip}%
\pgfsetbuttcap%
\pgfsetroundjoin%
\definecolor{currentfill}{rgb}{0.147607,0.511733,0.557049}%
\pgfsetfillcolor{currentfill}%
\pgfsetlinewidth{0.000000pt}%
\definecolor{currentstroke}{rgb}{0.000000,0.000000,0.000000}%
\pgfsetstrokecolor{currentstroke}%
\pgfsetdash{}{0pt}%
\pgfpathmoveto{\pgfqpoint{1.517199in}{1.220927in}}%
\pgfpathlineto{\pgfqpoint{1.520909in}{1.213167in}}%
\pgfpathlineto{\pgfqpoint{1.524617in}{1.205411in}}%
\pgfpathlineto{\pgfqpoint{1.528322in}{1.197661in}}%
\pgfpathlineto{\pgfqpoint{1.532024in}{1.189920in}}%
\pgfpathlineto{\pgfqpoint{1.531672in}{1.184370in}}%
\pgfpathlineto{\pgfqpoint{1.530969in}{1.178824in}}%
\pgfpathlineto{\pgfqpoint{1.529913in}{1.173287in}}%
\pgfpathlineto{\pgfqpoint{1.528505in}{1.167765in}}%
\pgfpathlineto{\pgfqpoint{1.524828in}{1.175765in}}%
\pgfpathlineto{\pgfqpoint{1.521149in}{1.183773in}}%
\pgfpathlineto{\pgfqpoint{1.517467in}{1.191786in}}%
\pgfpathlineto{\pgfqpoint{1.513783in}{1.199803in}}%
\pgfpathlineto{\pgfqpoint{1.515143in}{1.205068in}}%
\pgfpathlineto{\pgfqpoint{1.516166in}{1.210346in}}%
\pgfpathlineto{\pgfqpoint{1.516851in}{1.215635in}}%
\pgfpathlineto{\pgfqpoint{1.517199in}{1.220927in}}%
\pgfpathclose%
\pgfusepath{fill}%
\end{pgfscope}%
\begin{pgfscope}%
\pgfpathrectangle{\pgfqpoint{0.041670in}{0.041670in}}{\pgfqpoint{2.216660in}{2.216660in}}%
\pgfusepath{clip}%
\pgfsetbuttcap%
\pgfsetroundjoin%
\definecolor{currentfill}{rgb}{0.487026,0.823929,0.312321}%
\pgfsetfillcolor{currentfill}%
\pgfsetlinewidth{0.000000pt}%
\definecolor{currentstroke}{rgb}{0.000000,0.000000,0.000000}%
\pgfsetstrokecolor{currentstroke}%
\pgfsetdash{}{0pt}%
\pgfpathmoveto{\pgfqpoint{1.284301in}{1.580517in}}%
\pgfpathlineto{\pgfqpoint{1.286421in}{1.576282in}}%
\pgfpathlineto{\pgfqpoint{1.288539in}{1.571960in}}%
\pgfpathlineto{\pgfqpoint{1.290655in}{1.567552in}}%
\pgfpathlineto{\pgfqpoint{1.292768in}{1.563060in}}%
\pgfpathlineto{\pgfqpoint{1.297930in}{1.561339in}}%
\pgfpathlineto{\pgfqpoint{1.302978in}{1.559541in}}%
\pgfpathlineto{\pgfqpoint{1.307908in}{1.557668in}}%
\pgfpathlineto{\pgfqpoint{1.312715in}{1.555722in}}%
\pgfpathlineto{\pgfqpoint{1.310226in}{1.560357in}}%
\pgfpathlineto{\pgfqpoint{1.307734in}{1.564907in}}%
\pgfpathlineto{\pgfqpoint{1.305239in}{1.569371in}}%
\pgfpathlineto{\pgfqpoint{1.302740in}{1.573748in}}%
\pgfpathlineto{\pgfqpoint{1.298297in}{1.575543in}}%
\pgfpathlineto{\pgfqpoint{1.293740in}{1.577271in}}%
\pgfpathlineto{\pgfqpoint{1.289073in}{1.578929in}}%
\pgfpathlineto{\pgfqpoint{1.284301in}{1.580517in}}%
\pgfpathclose%
\pgfusepath{fill}%
\end{pgfscope}%
\begin{pgfscope}%
\pgfpathrectangle{\pgfqpoint{0.041670in}{0.041670in}}{\pgfqpoint{2.216660in}{2.216660in}}%
\pgfusepath{clip}%
\pgfsetbuttcap%
\pgfsetroundjoin%
\definecolor{currentfill}{rgb}{0.277941,0.056324,0.381191}%
\pgfsetfillcolor{currentfill}%
\pgfsetlinewidth{0.000000pt}%
\definecolor{currentstroke}{rgb}{0.000000,0.000000,0.000000}%
\pgfsetstrokecolor{currentstroke}%
\pgfsetdash{}{0pt}%
\pgfpathmoveto{\pgfqpoint{0.597477in}{0.750998in}}%
\pgfpathlineto{\pgfqpoint{0.593832in}{0.753820in}}%
\pgfpathlineto{\pgfqpoint{0.590176in}{0.756964in}}%
\pgfpathlineto{\pgfqpoint{0.586508in}{0.760437in}}%
\pgfpathlineto{\pgfqpoint{0.582828in}{0.764243in}}%
\pgfpathlineto{\pgfqpoint{0.574892in}{0.774297in}}%
\pgfpathlineto{\pgfqpoint{0.567593in}{0.784464in}}%
\pgfpathlineto{\pgfqpoint{0.560935in}{0.794733in}}%
\pgfpathlineto{\pgfqpoint{0.554922in}{0.805094in}}%
\pgfpathlineto{\pgfqpoint{0.558752in}{0.801058in}}%
\pgfpathlineto{\pgfqpoint{0.562571in}{0.797354in}}%
\pgfpathlineto{\pgfqpoint{0.566377in}{0.793978in}}%
\pgfpathlineto{\pgfqpoint{0.570172in}{0.790923in}}%
\pgfpathlineto{\pgfqpoint{0.576059in}{0.780797in}}%
\pgfpathlineto{\pgfqpoint{0.582576in}{0.770760in}}%
\pgfpathlineto{\pgfqpoint{0.589716in}{0.760823in}}%
\pgfpathlineto{\pgfqpoint{0.597477in}{0.750998in}}%
\pgfpathclose%
\pgfusepath{fill}%
\end{pgfscope}%
\begin{pgfscope}%
\pgfpathrectangle{\pgfqpoint{0.041670in}{0.041670in}}{\pgfqpoint{2.216660in}{2.216660in}}%
\pgfusepath{clip}%
\pgfsetbuttcap%
\pgfsetroundjoin%
\definecolor{currentfill}{rgb}{0.282327,0.094955,0.417331}%
\pgfsetfillcolor{currentfill}%
\pgfsetlinewidth{0.000000pt}%
\definecolor{currentstroke}{rgb}{0.000000,0.000000,0.000000}%
\pgfsetstrokecolor{currentstroke}%
\pgfsetdash{}{0pt}%
\pgfpathmoveto{\pgfqpoint{1.809789in}{0.814370in}}%
\pgfpathlineto{\pgfqpoint{1.813657in}{0.818796in}}%
\pgfpathlineto{\pgfqpoint{1.817539in}{0.823566in}}%
\pgfpathlineto{\pgfqpoint{1.821433in}{0.828687in}}%
\pgfpathlineto{\pgfqpoint{1.825342in}{0.834164in}}%
\pgfpathlineto{\pgfqpoint{1.819794in}{0.823503in}}%
\pgfpathlineto{\pgfqpoint{1.813581in}{0.812924in}}%
\pgfpathlineto{\pgfqpoint{1.806706in}{0.802439in}}%
\pgfpathlineto{\pgfqpoint{1.799174in}{0.792059in}}%
\pgfpathlineto{\pgfqpoint{1.795402in}{0.786807in}}%
\pgfpathlineto{\pgfqpoint{1.791644in}{0.781913in}}%
\pgfpathlineto{\pgfqpoint{1.787899in}{0.777371in}}%
\pgfpathlineto{\pgfqpoint{1.784167in}{0.773174in}}%
\pgfpathlineto{\pgfqpoint{1.791537in}{0.783329in}}%
\pgfpathlineto{\pgfqpoint{1.798267in}{0.793588in}}%
\pgfpathlineto{\pgfqpoint{1.804352in}{0.803938in}}%
\pgfpathlineto{\pgfqpoint{1.809789in}{0.814370in}}%
\pgfpathclose%
\pgfusepath{fill}%
\end{pgfscope}%
\begin{pgfscope}%
\pgfpathrectangle{\pgfqpoint{0.041670in}{0.041670in}}{\pgfqpoint{2.216660in}{2.216660in}}%
\pgfusepath{clip}%
\pgfsetbuttcap%
\pgfsetroundjoin%
\definecolor{currentfill}{rgb}{0.134692,0.658636,0.517649}%
\pgfsetfillcolor{currentfill}%
\pgfsetlinewidth{0.000000pt}%
\definecolor{currentstroke}{rgb}{0.000000,0.000000,0.000000}%
\pgfsetstrokecolor{currentstroke}%
\pgfsetdash{}{0pt}%
\pgfpathmoveto{\pgfqpoint{1.450132in}{1.377175in}}%
\pgfpathlineto{\pgfqpoint{1.453773in}{1.370223in}}%
\pgfpathlineto{\pgfqpoint{1.457410in}{1.363232in}}%
\pgfpathlineto{\pgfqpoint{1.461043in}{1.356206in}}%
\pgfpathlineto{\pgfqpoint{1.464673in}{1.349147in}}%
\pgfpathlineto{\pgfqpoint{1.466647in}{1.344745in}}%
\pgfpathlineto{\pgfqpoint{1.468338in}{1.340312in}}%
\pgfpathlineto{\pgfqpoint{1.469745in}{1.335852in}}%
\pgfpathlineto{\pgfqpoint{1.470865in}{1.331370in}}%
\pgfpathlineto{\pgfqpoint{1.467147in}{1.338674in}}%
\pgfpathlineto{\pgfqpoint{1.463425in}{1.345946in}}%
\pgfpathlineto{\pgfqpoint{1.459699in}{1.353181in}}%
\pgfpathlineto{\pgfqpoint{1.455971in}{1.360378in}}%
\pgfpathlineto{\pgfqpoint{1.454917in}{1.364613in}}%
\pgfpathlineto{\pgfqpoint{1.453592in}{1.368827in}}%
\pgfpathlineto{\pgfqpoint{1.451996in}{1.373015in}}%
\pgfpathlineto{\pgfqpoint{1.450132in}{1.377175in}}%
\pgfpathclose%
\pgfusepath{fill}%
\end{pgfscope}%
\begin{pgfscope}%
\pgfpathrectangle{\pgfqpoint{0.041670in}{0.041670in}}{\pgfqpoint{2.216660in}{2.216660in}}%
\pgfusepath{clip}%
\pgfsetbuttcap%
\pgfsetroundjoin%
\definecolor{currentfill}{rgb}{0.274952,0.037752,0.364543}%
\pgfsetfillcolor{currentfill}%
\pgfsetlinewidth{0.000000pt}%
\definecolor{currentstroke}{rgb}{0.000000,0.000000,0.000000}%
\pgfsetstrokecolor{currentstroke}%
\pgfsetdash{}{0pt}%
\pgfpathmoveto{\pgfqpoint{1.669714in}{0.788718in}}%
\pgfpathlineto{\pgfqpoint{1.673189in}{0.784658in}}%
\pgfpathlineto{\pgfqpoint{1.676669in}{0.780781in}}%
\pgfpathlineto{\pgfqpoint{1.680153in}{0.777094in}}%
\pgfpathlineto{\pgfqpoint{1.683642in}{0.773601in}}%
\pgfpathlineto{\pgfqpoint{1.676855in}{0.765233in}}%
\pgfpathlineto{\pgfqpoint{1.669548in}{0.756974in}}%
\pgfpathlineto{\pgfqpoint{1.661728in}{0.748832in}}%
\pgfpathlineto{\pgfqpoint{1.653399in}{0.740818in}}%
\pgfpathlineto{\pgfqpoint{1.650106in}{0.744555in}}%
\pgfpathlineto{\pgfqpoint{1.646816in}{0.748486in}}%
\pgfpathlineto{\pgfqpoint{1.643531in}{0.752606in}}%
\pgfpathlineto{\pgfqpoint{1.640250in}{0.756912in}}%
\pgfpathlineto{\pgfqpoint{1.648361in}{0.764687in}}%
\pgfpathlineto{\pgfqpoint{1.655980in}{0.772586in}}%
\pgfpathlineto{\pgfqpoint{1.663099in}{0.780599in}}%
\pgfpathlineto{\pgfqpoint{1.669714in}{0.788718in}}%
\pgfpathclose%
\pgfusepath{fill}%
\end{pgfscope}%
\begin{pgfscope}%
\pgfpathrectangle{\pgfqpoint{0.041670in}{0.041670in}}{\pgfqpoint{2.216660in}{2.216660in}}%
\pgfusepath{clip}%
\pgfsetbuttcap%
\pgfsetroundjoin%
\definecolor{currentfill}{rgb}{0.271305,0.019942,0.347269}%
\pgfsetfillcolor{currentfill}%
\pgfsetlinewidth{0.000000pt}%
\definecolor{currentstroke}{rgb}{0.000000,0.000000,0.000000}%
\pgfsetstrokecolor{currentstroke}%
\pgfsetdash{}{0pt}%
\pgfpathmoveto{\pgfqpoint{1.683642in}{0.773601in}}%
\pgfpathlineto{\pgfqpoint{1.687136in}{0.770304in}}%
\pgfpathlineto{\pgfqpoint{1.690634in}{0.767210in}}%
\pgfpathlineto{\pgfqpoint{1.694138in}{0.764323in}}%
\pgfpathlineto{\pgfqpoint{1.697647in}{0.761646in}}%
\pgfpathlineto{\pgfqpoint{1.690688in}{0.753031in}}%
\pgfpathlineto{\pgfqpoint{1.683193in}{0.744528in}}%
\pgfpathlineto{\pgfqpoint{1.675170in}{0.736145in}}%
\pgfpathlineto{\pgfqpoint{1.666624in}{0.727892in}}%
\pgfpathlineto{\pgfqpoint{1.663310in}{0.730811in}}%
\pgfpathlineto{\pgfqpoint{1.660002in}{0.733941in}}%
\pgfpathlineto{\pgfqpoint{1.656698in}{0.737278in}}%
\pgfpathlineto{\pgfqpoint{1.653399in}{0.740818in}}%
\pgfpathlineto{\pgfqpoint{1.661728in}{0.748832in}}%
\pgfpathlineto{\pgfqpoint{1.669548in}{0.756974in}}%
\pgfpathlineto{\pgfqpoint{1.676855in}{0.765233in}}%
\pgfpathlineto{\pgfqpoint{1.683642in}{0.773601in}}%
\pgfpathclose%
\pgfusepath{fill}%
\end{pgfscope}%
\begin{pgfscope}%
\pgfpathrectangle{\pgfqpoint{0.041670in}{0.041670in}}{\pgfqpoint{2.216660in}{2.216660in}}%
\pgfusepath{clip}%
\pgfsetbuttcap%
\pgfsetroundjoin%
\definecolor{currentfill}{rgb}{0.195860,0.395433,0.555276}%
\pgfsetfillcolor{currentfill}%
\pgfsetlinewidth{0.000000pt}%
\definecolor{currentstroke}{rgb}{0.000000,0.000000,0.000000}%
\pgfsetstrokecolor{currentstroke}%
\pgfsetdash{}{0pt}%
\pgfpathmoveto{\pgfqpoint{0.814871in}{1.075336in}}%
\pgfpathlineto{\pgfqpoint{0.811321in}{1.067234in}}%
\pgfpathlineto{\pgfqpoint{0.807773in}{1.059169in}}%
\pgfpathlineto{\pgfqpoint{0.804226in}{1.051147in}}%
\pgfpathlineto{\pgfqpoint{0.800680in}{1.043169in}}%
\pgfpathlineto{\pgfqpoint{0.797209in}{1.049301in}}%
\pgfpathlineto{\pgfqpoint{0.794129in}{1.055480in}}%
\pgfpathlineto{\pgfqpoint{0.791441in}{1.061700in}}%
\pgfpathlineto{\pgfqpoint{0.789147in}{1.067954in}}%
\pgfpathlineto{\pgfqpoint{0.792788in}{1.075673in}}%
\pgfpathlineto{\pgfqpoint{0.796431in}{1.083438in}}%
\pgfpathlineto{\pgfqpoint{0.800075in}{1.091244in}}%
\pgfpathlineto{\pgfqpoint{0.803720in}{1.099089in}}%
\pgfpathlineto{\pgfqpoint{0.805941in}{1.093094in}}%
\pgfpathlineto{\pgfqpoint{0.808540in}{1.087133in}}%
\pgfpathlineto{\pgfqpoint{0.811518in}{1.081212in}}%
\pgfpathlineto{\pgfqpoint{0.814871in}{1.075336in}}%
\pgfpathclose%
\pgfusepath{fill}%
\end{pgfscope}%
\begin{pgfscope}%
\pgfpathrectangle{\pgfqpoint{0.041670in}{0.041670in}}{\pgfqpoint{2.216660in}{2.216660in}}%
\pgfusepath{clip}%
\pgfsetbuttcap%
\pgfsetroundjoin%
\definecolor{currentfill}{rgb}{0.487026,0.823929,0.312321}%
\pgfsetfillcolor{currentfill}%
\pgfsetlinewidth{0.000000pt}%
\definecolor{currentstroke}{rgb}{0.000000,0.000000,0.000000}%
\pgfsetstrokecolor{currentstroke}%
\pgfsetdash{}{0pt}%
\pgfpathmoveto{\pgfqpoint{1.053320in}{1.572097in}}%
\pgfpathlineto{\pgfqpoint{1.050742in}{1.567686in}}%
\pgfpathlineto{\pgfqpoint{1.048168in}{1.563187in}}%
\pgfpathlineto{\pgfqpoint{1.045597in}{1.558602in}}%
\pgfpathlineto{\pgfqpoint{1.043029in}{1.553933in}}%
\pgfpathlineto{\pgfqpoint{1.047723in}{1.555942in}}%
\pgfpathlineto{\pgfqpoint{1.052543in}{1.557880in}}%
\pgfpathlineto{\pgfqpoint{1.057487in}{1.559745in}}%
\pgfpathlineto{\pgfqpoint{1.062548in}{1.561534in}}%
\pgfpathlineto{\pgfqpoint{1.064747in}{1.566055in}}%
\pgfpathlineto{\pgfqpoint{1.066950in}{1.570493in}}%
\pgfpathlineto{\pgfqpoint{1.069154in}{1.574845in}}%
\pgfpathlineto{\pgfqpoint{1.071362in}{1.579109in}}%
\pgfpathlineto{\pgfqpoint{1.066683in}{1.577458in}}%
\pgfpathlineto{\pgfqpoint{1.062113in}{1.575738in}}%
\pgfpathlineto{\pgfqpoint{1.057657in}{1.573951in}}%
\pgfpathlineto{\pgfqpoint{1.053320in}{1.572097in}}%
\pgfpathclose%
\pgfusepath{fill}%
\end{pgfscope}%
\begin{pgfscope}%
\pgfpathrectangle{\pgfqpoint{0.041670in}{0.041670in}}{\pgfqpoint{2.216660in}{2.216660in}}%
\pgfusepath{clip}%
\pgfsetbuttcap%
\pgfsetroundjoin%
\definecolor{currentfill}{rgb}{0.565498,0.842430,0.262877}%
\pgfsetfillcolor{currentfill}%
\pgfsetlinewidth{0.000000pt}%
\definecolor{currentstroke}{rgb}{0.000000,0.000000,0.000000}%
\pgfsetstrokecolor{currentstroke}%
\pgfsetdash{}{0pt}%
\pgfpathmoveto{\pgfqpoint{1.237785in}{1.605706in}}%
\pgfpathlineto{\pgfqpoint{1.239069in}{1.602047in}}%
\pgfpathlineto{\pgfqpoint{1.240351in}{1.598295in}}%
\pgfpathlineto{\pgfqpoint{1.241632in}{1.594449in}}%
\pgfpathlineto{\pgfqpoint{1.242911in}{1.590513in}}%
\pgfpathlineto{\pgfqpoint{1.248350in}{1.589538in}}%
\pgfpathlineto{\pgfqpoint{1.253724in}{1.588482in}}%
\pgfpathlineto{\pgfqpoint{1.259027in}{1.587347in}}%
\pgfpathlineto{\pgfqpoint{1.264253in}{1.586133in}}%
\pgfpathlineto{\pgfqpoint{1.262539in}{1.590160in}}%
\pgfpathlineto{\pgfqpoint{1.260823in}{1.594097in}}%
\pgfpathlineto{\pgfqpoint{1.259105in}{1.597941in}}%
\pgfpathlineto{\pgfqpoint{1.257385in}{1.601691in}}%
\pgfpathlineto{\pgfqpoint{1.252585in}{1.602804in}}%
\pgfpathlineto{\pgfqpoint{1.247715in}{1.603844in}}%
\pgfpathlineto{\pgfqpoint{1.242780in}{1.604812in}}%
\pgfpathlineto{\pgfqpoint{1.237785in}{1.605706in}}%
\pgfpathclose%
\pgfusepath{fill}%
\end{pgfscope}%
\begin{pgfscope}%
\pgfpathrectangle{\pgfqpoint{0.041670in}{0.041670in}}{\pgfqpoint{2.216660in}{2.216660in}}%
\pgfusepath{clip}%
\pgfsetbuttcap%
\pgfsetroundjoin%
\definecolor{currentfill}{rgb}{0.279566,0.067836,0.391917}%
\pgfsetfillcolor{currentfill}%
\pgfsetlinewidth{0.000000pt}%
\definecolor{currentstroke}{rgb}{0.000000,0.000000,0.000000}%
\pgfsetstrokecolor{currentstroke}%
\pgfsetdash{}{0pt}%
\pgfpathmoveto{\pgfqpoint{1.655849in}{0.806724in}}%
\pgfpathlineto{\pgfqpoint{1.659310in}{0.801966in}}%
\pgfpathlineto{\pgfqpoint{1.662774in}{0.797377in}}%
\pgfpathlineto{\pgfqpoint{1.666242in}{0.792959in}}%
\pgfpathlineto{\pgfqpoint{1.669714in}{0.788718in}}%
\pgfpathlineto{\pgfqpoint{1.663099in}{0.780599in}}%
\pgfpathlineto{\pgfqpoint{1.655980in}{0.772586in}}%
\pgfpathlineto{\pgfqpoint{1.648361in}{0.764687in}}%
\pgfpathlineto{\pgfqpoint{1.640250in}{0.756912in}}%
\pgfpathlineto{\pgfqpoint{1.636973in}{0.761398in}}%
\pgfpathlineto{\pgfqpoint{1.633700in}{0.766060in}}%
\pgfpathlineto{\pgfqpoint{1.630431in}{0.770895in}}%
\pgfpathlineto{\pgfqpoint{1.627165in}{0.775897in}}%
\pgfpathlineto{\pgfqpoint{1.635059in}{0.783433in}}%
\pgfpathlineto{\pgfqpoint{1.642475in}{0.791088in}}%
\pgfpathlineto{\pgfqpoint{1.649407in}{0.798854in}}%
\pgfpathlineto{\pgfqpoint{1.655849in}{0.806724in}}%
\pgfpathclose%
\pgfusepath{fill}%
\end{pgfscope}%
\begin{pgfscope}%
\pgfpathrectangle{\pgfqpoint{0.041670in}{0.041670in}}{\pgfqpoint{2.216660in}{2.216660in}}%
\pgfusepath{clip}%
\pgfsetbuttcap%
\pgfsetroundjoin%
\definecolor{currentfill}{rgb}{0.268510,0.009605,0.335427}%
\pgfsetfillcolor{currentfill}%
\pgfsetlinewidth{0.000000pt}%
\definecolor{currentstroke}{rgb}{0.000000,0.000000,0.000000}%
\pgfsetstrokecolor{currentstroke}%
\pgfsetdash{}{0pt}%
\pgfpathmoveto{\pgfqpoint{1.697647in}{0.761646in}}%
\pgfpathlineto{\pgfqpoint{1.701162in}{0.759185in}}%
\pgfpathlineto{\pgfqpoint{1.704683in}{0.756944in}}%
\pgfpathlineto{\pgfqpoint{1.708210in}{0.754927in}}%
\pgfpathlineto{\pgfqpoint{1.711743in}{0.753140in}}%
\pgfpathlineto{\pgfqpoint{1.704611in}{0.744280in}}%
\pgfpathlineto{\pgfqpoint{1.696929in}{0.735535in}}%
\pgfpathlineto{\pgfqpoint{1.688702in}{0.726912in}}%
\pgfpathlineto{\pgfqpoint{1.679938in}{0.718423in}}%
\pgfpathlineto{\pgfqpoint{1.676601in}{0.720450in}}%
\pgfpathlineto{\pgfqpoint{1.673269in}{0.722707in}}%
\pgfpathlineto{\pgfqpoint{1.669944in}{0.725189in}}%
\pgfpathlineto{\pgfqpoint{1.666624in}{0.727892in}}%
\pgfpathlineto{\pgfqpoint{1.675170in}{0.736145in}}%
\pgfpathlineto{\pgfqpoint{1.683193in}{0.744528in}}%
\pgfpathlineto{\pgfqpoint{1.690688in}{0.753031in}}%
\pgfpathlineto{\pgfqpoint{1.697647in}{0.761646in}}%
\pgfpathclose%
\pgfusepath{fill}%
\end{pgfscope}%
\begin{pgfscope}%
\pgfpathrectangle{\pgfqpoint{0.041670in}{0.041670in}}{\pgfqpoint{2.216660in}{2.216660in}}%
\pgfusepath{clip}%
\pgfsetbuttcap%
\pgfsetroundjoin%
\definecolor{currentfill}{rgb}{0.565498,0.842430,0.262877}%
\pgfsetfillcolor{currentfill}%
\pgfsetlinewidth{0.000000pt}%
\definecolor{currentstroke}{rgb}{0.000000,0.000000,0.000000}%
\pgfsetstrokecolor{currentstroke}%
\pgfsetdash{}{0pt}%
\pgfpathmoveto{\pgfqpoint{1.098321in}{1.600643in}}%
\pgfpathlineto{\pgfqpoint{1.096507in}{1.596869in}}%
\pgfpathlineto{\pgfqpoint{1.094695in}{1.593001in}}%
\pgfpathlineto{\pgfqpoint{1.092885in}{1.589041in}}%
\pgfpathlineto{\pgfqpoint{1.091078in}{1.584989in}}%
\pgfpathlineto{\pgfqpoint{1.096233in}{1.586271in}}%
\pgfpathlineto{\pgfqpoint{1.101469in}{1.587477in}}%
\pgfpathlineto{\pgfqpoint{1.106780in}{1.588603in}}%
\pgfpathlineto{\pgfqpoint{1.112161in}{1.589650in}}%
\pgfpathlineto{\pgfqpoint{1.113538in}{1.593604in}}%
\pgfpathlineto{\pgfqpoint{1.114918in}{1.597468in}}%
\pgfpathlineto{\pgfqpoint{1.116299in}{1.601238in}}%
\pgfpathlineto{\pgfqpoint{1.117682in}{1.604915in}}%
\pgfpathlineto{\pgfqpoint{1.112739in}{1.603955in}}%
\pgfpathlineto{\pgfqpoint{1.107862in}{1.602923in}}%
\pgfpathlineto{\pgfqpoint{1.103054in}{1.601819in}}%
\pgfpathlineto{\pgfqpoint{1.098321in}{1.600643in}}%
\pgfpathclose%
\pgfusepath{fill}%
\end{pgfscope}%
\begin{pgfscope}%
\pgfpathrectangle{\pgfqpoint{0.041670in}{0.041670in}}{\pgfqpoint{2.216660in}{2.216660in}}%
\pgfusepath{clip}%
\pgfsetbuttcap%
\pgfsetroundjoin%
\definecolor{currentfill}{rgb}{0.344074,0.780029,0.397381}%
\pgfsetfillcolor{currentfill}%
\pgfsetlinewidth{0.000000pt}%
\definecolor{currentstroke}{rgb}{0.000000,0.000000,0.000000}%
\pgfsetstrokecolor{currentstroke}%
\pgfsetdash{}{0pt}%
\pgfpathmoveto{\pgfqpoint{1.358651in}{1.517005in}}%
\pgfpathlineto{\pgfqpoint{1.361759in}{1.511627in}}%
\pgfpathlineto{\pgfqpoint{1.364863in}{1.506177in}}%
\pgfpathlineto{\pgfqpoint{1.367964in}{1.500655in}}%
\pgfpathlineto{\pgfqpoint{1.371061in}{1.495065in}}%
\pgfpathlineto{\pgfqpoint{1.375094in}{1.492154in}}%
\pgfpathlineto{\pgfqpoint{1.378937in}{1.489182in}}%
\pgfpathlineto{\pgfqpoint{1.382586in}{1.486153in}}%
\pgfpathlineto{\pgfqpoint{1.386039in}{1.483069in}}%
\pgfpathlineto{\pgfqpoint{1.382693in}{1.488864in}}%
\pgfpathlineto{\pgfqpoint{1.379344in}{1.494590in}}%
\pgfpathlineto{\pgfqpoint{1.375991in}{1.500245in}}%
\pgfpathlineto{\pgfqpoint{1.372634in}{1.505827in}}%
\pgfpathlineto{\pgfqpoint{1.369411in}{1.508701in}}%
\pgfpathlineto{\pgfqpoint{1.366005in}{1.511524in}}%
\pgfpathlineto{\pgfqpoint{1.362417in}{1.514293in}}%
\pgfpathlineto{\pgfqpoint{1.358651in}{1.517005in}}%
\pgfpathclose%
\pgfusepath{fill}%
\end{pgfscope}%
\begin{pgfscope}%
\pgfpathrectangle{\pgfqpoint{0.041670in}{0.041670in}}{\pgfqpoint{2.216660in}{2.216660in}}%
\pgfusepath{clip}%
\pgfsetbuttcap%
\pgfsetroundjoin%
\definecolor{currentfill}{rgb}{0.122606,0.585371,0.546557}%
\pgfsetfillcolor{currentfill}%
\pgfsetlinewidth{0.000000pt}%
\definecolor{currentstroke}{rgb}{0.000000,0.000000,0.000000}%
\pgfsetstrokecolor{currentstroke}%
\pgfsetdash{}{0pt}%
\pgfpathmoveto{\pgfqpoint{0.872780in}{1.278543in}}%
\pgfpathlineto{\pgfqpoint{0.869048in}{1.270799in}}%
\pgfpathlineto{\pgfqpoint{0.865319in}{1.263037in}}%
\pgfpathlineto{\pgfqpoint{0.861592in}{1.255258in}}%
\pgfpathlineto{\pgfqpoint{0.857869in}{1.247467in}}%
\pgfpathlineto{\pgfqpoint{0.857561in}{1.252503in}}%
\pgfpathlineto{\pgfqpoint{0.857575in}{1.257538in}}%
\pgfpathlineto{\pgfqpoint{0.857910in}{1.262566in}}%
\pgfpathlineto{\pgfqpoint{0.858565in}{1.267583in}}%
\pgfpathlineto{\pgfqpoint{0.862270in}{1.275120in}}%
\pgfpathlineto{\pgfqpoint{0.865978in}{1.282644in}}%
\pgfpathlineto{\pgfqpoint{0.869688in}{1.290153in}}%
\pgfpathlineto{\pgfqpoint{0.873402in}{1.297643in}}%
\pgfpathlineto{\pgfqpoint{0.872788in}{1.292879in}}%
\pgfpathlineto{\pgfqpoint{0.872479in}{1.288104in}}%
\pgfpathlineto{\pgfqpoint{0.872476in}{1.283324in}}%
\pgfpathlineto{\pgfqpoint{0.872780in}{1.278543in}}%
\pgfpathclose%
\pgfusepath{fill}%
\end{pgfscope}%
\begin{pgfscope}%
\pgfpathrectangle{\pgfqpoint{0.041670in}{0.041670in}}{\pgfqpoint{2.216660in}{2.216660in}}%
\pgfusepath{clip}%
\pgfsetbuttcap%
\pgfsetroundjoin%
\definecolor{currentfill}{rgb}{0.281477,0.755203,0.432552}%
\pgfsetfillcolor{currentfill}%
\pgfsetlinewidth{0.000000pt}%
\definecolor{currentstroke}{rgb}{0.000000,0.000000,0.000000}%
\pgfsetstrokecolor{currentstroke}%
\pgfsetdash{}{0pt}%
\pgfpathmoveto{\pgfqpoint{1.386039in}{1.483069in}}%
\pgfpathlineto{\pgfqpoint{1.389381in}{1.477207in}}%
\pgfpathlineto{\pgfqpoint{1.392720in}{1.471280in}}%
\pgfpathlineto{\pgfqpoint{1.396055in}{1.465290in}}%
\pgfpathlineto{\pgfqpoint{1.399386in}{1.459240in}}%
\pgfpathlineto{\pgfqpoint{1.402855in}{1.455889in}}%
\pgfpathlineto{\pgfqpoint{1.406108in}{1.452484in}}%
\pgfpathlineto{\pgfqpoint{1.409139in}{1.449030in}}%
\pgfpathlineto{\pgfqpoint{1.411947in}{1.445531in}}%
\pgfpathlineto{\pgfqpoint{1.408418in}{1.451801in}}%
\pgfpathlineto{\pgfqpoint{1.404885in}{1.458012in}}%
\pgfpathlineto{\pgfqpoint{1.401349in}{1.464159in}}%
\pgfpathlineto{\pgfqpoint{1.397809in}{1.470241in}}%
\pgfpathlineto{\pgfqpoint{1.395179in}{1.473515in}}%
\pgfpathlineto{\pgfqpoint{1.392339in}{1.476747in}}%
\pgfpathlineto{\pgfqpoint{1.389291in}{1.479932in}}%
\pgfpathlineto{\pgfqpoint{1.386039in}{1.483069in}}%
\pgfpathclose%
\pgfusepath{fill}%
\end{pgfscope}%
\begin{pgfscope}%
\pgfpathrectangle{\pgfqpoint{0.041670in}{0.041670in}}{\pgfqpoint{2.216660in}{2.216660in}}%
\pgfusepath{clip}%
\pgfsetbuttcap%
\pgfsetroundjoin%
\definecolor{currentfill}{rgb}{0.282327,0.094955,0.417331}%
\pgfsetfillcolor{currentfill}%
\pgfsetlinewidth{0.000000pt}%
\definecolor{currentstroke}{rgb}{0.000000,0.000000,0.000000}%
\pgfsetstrokecolor{currentstroke}%
\pgfsetdash{}{0pt}%
\pgfpathmoveto{\pgfqpoint{1.642036in}{0.827351in}}%
\pgfpathlineto{\pgfqpoint{1.645485in}{0.821962in}}%
\pgfpathlineto{\pgfqpoint{1.648937in}{0.816726in}}%
\pgfpathlineto{\pgfqpoint{1.652391in}{0.811645in}}%
\pgfpathlineto{\pgfqpoint{1.655849in}{0.806724in}}%
\pgfpathlineto{\pgfqpoint{1.649407in}{0.798854in}}%
\pgfpathlineto{\pgfqpoint{1.642475in}{0.791088in}}%
\pgfpathlineto{\pgfqpoint{1.635059in}{0.783433in}}%
\pgfpathlineto{\pgfqpoint{1.627165in}{0.775897in}}%
\pgfpathlineto{\pgfqpoint{1.623902in}{0.781064in}}%
\pgfpathlineto{\pgfqpoint{1.620642in}{0.786391in}}%
\pgfpathlineto{\pgfqpoint{1.617385in}{0.791873in}}%
\pgfpathlineto{\pgfqpoint{1.614131in}{0.797507in}}%
\pgfpathlineto{\pgfqpoint{1.621809in}{0.804801in}}%
\pgfpathlineto{\pgfqpoint{1.629022in}{0.812212in}}%
\pgfpathlineto{\pgfqpoint{1.635767in}{0.819731in}}%
\pgfpathlineto{\pgfqpoint{1.642036in}{0.827351in}}%
\pgfpathclose%
\pgfusepath{fill}%
\end{pgfscope}%
\begin{pgfscope}%
\pgfpathrectangle{\pgfqpoint{0.041670in}{0.041670in}}{\pgfqpoint{2.216660in}{2.216660in}}%
\pgfusepath{clip}%
\pgfsetbuttcap%
\pgfsetroundjoin%
\definecolor{currentfill}{rgb}{0.179019,0.433756,0.557430}%
\pgfsetfillcolor{currentfill}%
\pgfsetlinewidth{0.000000pt}%
\definecolor{currentstroke}{rgb}{0.000000,0.000000,0.000000}%
\pgfsetstrokecolor{currentstroke}%
\pgfsetdash{}{0pt}%
\pgfpathmoveto{\pgfqpoint{1.543191in}{1.135914in}}%
\pgfpathlineto{\pgfqpoint{1.546857in}{1.128003in}}%
\pgfpathlineto{\pgfqpoint{1.550521in}{1.120118in}}%
\pgfpathlineto{\pgfqpoint{1.554183in}{1.112263in}}%
\pgfpathlineto{\pgfqpoint{1.557844in}{1.104440in}}%
\pgfpathlineto{\pgfqpoint{1.555962in}{1.098421in}}%
\pgfpathlineto{\pgfqpoint{1.553699in}{1.092430in}}%
\pgfpathlineto{\pgfqpoint{1.551057in}{1.086473in}}%
\pgfpathlineto{\pgfqpoint{1.548038in}{1.080557in}}%
\pgfpathlineto{\pgfqpoint{1.544460in}{1.088638in}}%
\pgfpathlineto{\pgfqpoint{1.540881in}{1.096752in}}%
\pgfpathlineto{\pgfqpoint{1.537301in}{1.104895in}}%
\pgfpathlineto{\pgfqpoint{1.533718in}{1.113064in}}%
\pgfpathlineto{\pgfqpoint{1.536632in}{1.118724in}}%
\pgfpathlineto{\pgfqpoint{1.539183in}{1.124423in}}%
\pgfpathlineto{\pgfqpoint{1.541369in}{1.130155in}}%
\pgfpathlineto{\pgfqpoint{1.543191in}{1.135914in}}%
\pgfpathclose%
\pgfusepath{fill}%
\end{pgfscope}%
\begin{pgfscope}%
\pgfpathrectangle{\pgfqpoint{0.041670in}{0.041670in}}{\pgfqpoint{2.216660in}{2.216660in}}%
\pgfusepath{clip}%
\pgfsetbuttcap%
\pgfsetroundjoin%
\definecolor{currentfill}{rgb}{0.412913,0.803041,0.357269}%
\pgfsetfillcolor{currentfill}%
\pgfsetlinewidth{0.000000pt}%
\definecolor{currentstroke}{rgb}{0.000000,0.000000,0.000000}%
\pgfsetstrokecolor{currentstroke}%
\pgfsetdash{}{0pt}%
\pgfpathmoveto{\pgfqpoint{1.330616in}{1.547243in}}%
\pgfpathlineto{\pgfqpoint{1.333441in}{1.542362in}}%
\pgfpathlineto{\pgfqpoint{1.336263in}{1.537400in}}%
\pgfpathlineto{\pgfqpoint{1.339081in}{1.532359in}}%
\pgfpathlineto{\pgfqpoint{1.341897in}{1.527242in}}%
\pgfpathlineto{\pgfqpoint{1.346331in}{1.524780in}}%
\pgfpathlineto{\pgfqpoint{1.350604in}{1.522251in}}%
\pgfpathlineto{\pgfqpoint{1.354713in}{1.519659in}}%
\pgfpathlineto{\pgfqpoint{1.358651in}{1.517005in}}%
\pgfpathlineto{\pgfqpoint{1.355540in}{1.522309in}}%
\pgfpathlineto{\pgfqpoint{1.352425in}{1.527535in}}%
\pgfpathlineto{\pgfqpoint{1.349307in}{1.532683in}}%
\pgfpathlineto{\pgfqpoint{1.346185in}{1.537750in}}%
\pgfpathlineto{\pgfqpoint{1.342526in}{1.540211in}}%
\pgfpathlineto{\pgfqpoint{1.338709in}{1.542615in}}%
\pgfpathlineto{\pgfqpoint{1.334737in}{1.544960in}}%
\pgfpathlineto{\pgfqpoint{1.330616in}{1.547243in}}%
\pgfpathclose%
\pgfusepath{fill}%
\end{pgfscope}%
\begin{pgfscope}%
\pgfpathrectangle{\pgfqpoint{0.041670in}{0.041670in}}{\pgfqpoint{2.216660in}{2.216660in}}%
\pgfusepath{clip}%
\pgfsetbuttcap%
\pgfsetroundjoin%
\definecolor{currentfill}{rgb}{0.172719,0.448791,0.557885}%
\pgfsetfillcolor{currentfill}%
\pgfsetlinewidth{0.000000pt}%
\definecolor{currentstroke}{rgb}{0.000000,0.000000,0.000000}%
\pgfsetstrokecolor{currentstroke}%
\pgfsetdash{}{0pt}%
\pgfpathmoveto{\pgfqpoint{1.928813in}{1.123862in}}%
\pgfpathlineto{\pgfqpoint{1.933187in}{1.138758in}}%
\pgfpathlineto{\pgfqpoint{1.937585in}{1.154165in}}%
\pgfpathlineto{\pgfqpoint{1.942008in}{1.170090in}}%
\pgfpathlineto{\pgfqpoint{1.941960in}{1.158031in}}%
\pgfpathlineto{\pgfqpoint{1.941148in}{1.145958in}}%
\pgfpathlineto{\pgfqpoint{1.939569in}{1.133883in}}%
\pgfpathlineto{\pgfqpoint{1.937220in}{1.121819in}}%
\pgfpathlineto{\pgfqpoint{1.932794in}{1.106065in}}%
\pgfpathlineto{\pgfqpoint{1.928394in}{1.090834in}}%
\pgfpathlineto{\pgfqpoint{1.924018in}{1.076116in}}%
\pgfpathlineto{\pgfqpoint{1.926348in}{1.088048in}}%
\pgfpathlineto{\pgfqpoint{1.927923in}{1.099991in}}%
\pgfpathlineto{\pgfqpoint{1.928743in}{1.111933in}}%
\pgfpathlineto{\pgfqpoint{1.928813in}{1.123862in}}%
\pgfpathclose%
\pgfusepath{fill}%
\end{pgfscope}%
\begin{pgfscope}%
\pgfpathrectangle{\pgfqpoint{0.041670in}{0.041670in}}{\pgfqpoint{2.216660in}{2.216660in}}%
\pgfusepath{clip}%
\pgfsetbuttcap%
\pgfsetroundjoin%
\definecolor{currentfill}{rgb}{0.147607,0.511733,0.557049}%
\pgfsetfillcolor{currentfill}%
\pgfsetlinewidth{0.000000pt}%
\definecolor{currentstroke}{rgb}{0.000000,0.000000,0.000000}%
\pgfsetstrokecolor{currentstroke}%
\pgfsetdash{}{0pt}%
\pgfpathmoveto{\pgfqpoint{0.847618in}{1.195141in}}%
\pgfpathlineto{\pgfqpoint{0.843947in}{1.187067in}}%
\pgfpathlineto{\pgfqpoint{0.840280in}{1.178996in}}%
\pgfpathlineto{\pgfqpoint{0.836614in}{1.170931in}}%
\pgfpathlineto{\pgfqpoint{0.832951in}{1.162874in}}%
\pgfpathlineto{\pgfqpoint{0.831231in}{1.168378in}}%
\pgfpathlineto{\pgfqpoint{0.829862in}{1.173902in}}%
\pgfpathlineto{\pgfqpoint{0.828845in}{1.179440in}}%
\pgfpathlineto{\pgfqpoint{0.828181in}{1.184986in}}%
\pgfpathlineto{\pgfqpoint{0.831883in}{1.192785in}}%
\pgfpathlineto{\pgfqpoint{0.835587in}{1.200592in}}%
\pgfpathlineto{\pgfqpoint{0.839294in}{1.208406in}}%
\pgfpathlineto{\pgfqpoint{0.843003in}{1.216222in}}%
\pgfpathlineto{\pgfqpoint{0.843651in}{1.210934in}}%
\pgfpathlineto{\pgfqpoint{0.844636in}{1.205653in}}%
\pgfpathlineto{\pgfqpoint{0.845959in}{1.200387in}}%
\pgfpathlineto{\pgfqpoint{0.847618in}{1.195141in}}%
\pgfpathclose%
\pgfusepath{fill}%
\end{pgfscope}%
\begin{pgfscope}%
\pgfpathrectangle{\pgfqpoint{0.041670in}{0.041670in}}{\pgfqpoint{2.216660in}{2.216660in}}%
\pgfusepath{clip}%
\pgfsetbuttcap%
\pgfsetroundjoin%
\definecolor{currentfill}{rgb}{0.248629,0.278775,0.534556}%
\pgfsetfillcolor{currentfill}%
\pgfsetlinewidth{0.000000pt}%
\definecolor{currentstroke}{rgb}{0.000000,0.000000,0.000000}%
\pgfsetstrokecolor{currentstroke}%
\pgfsetdash{}{0pt}%
\pgfpathmoveto{\pgfqpoint{0.791301in}{0.955401in}}%
\pgfpathlineto{\pgfqpoint{0.787914in}{0.947719in}}%
\pgfpathlineto{\pgfqpoint{0.784527in}{0.940115in}}%
\pgfpathlineto{\pgfqpoint{0.781139in}{0.932592in}}%
\pgfpathlineto{\pgfqpoint{0.777752in}{0.925154in}}%
\pgfpathlineto{\pgfqpoint{0.772225in}{0.931780in}}%
\pgfpathlineto{\pgfqpoint{0.767119in}{0.938487in}}%
\pgfpathlineto{\pgfqpoint{0.762439in}{0.945267in}}%
\pgfpathlineto{\pgfqpoint{0.758189in}{0.952114in}}%
\pgfpathlineto{\pgfqpoint{0.761728in}{0.959298in}}%
\pgfpathlineto{\pgfqpoint{0.765267in}{0.966568in}}%
\pgfpathlineto{\pgfqpoint{0.768806in}{0.973918in}}%
\pgfpathlineto{\pgfqpoint{0.772345in}{0.981346in}}%
\pgfpathlineto{\pgfqpoint{0.776466in}{0.974757in}}%
\pgfpathlineto{\pgfqpoint{0.781001in}{0.968231in}}%
\pgfpathlineto{\pgfqpoint{0.785948in}{0.961777in}}%
\pgfpathlineto{\pgfqpoint{0.791301in}{0.955401in}}%
\pgfpathclose%
\pgfusepath{fill}%
\end{pgfscope}%
\begin{pgfscope}%
\pgfpathrectangle{\pgfqpoint{0.041670in}{0.041670in}}{\pgfqpoint{2.216660in}{2.216660in}}%
\pgfusepath{clip}%
\pgfsetbuttcap%
\pgfsetroundjoin%
\definecolor{currentfill}{rgb}{0.267004,0.004874,0.329415}%
\pgfsetfillcolor{currentfill}%
\pgfsetlinewidth{0.000000pt}%
\definecolor{currentstroke}{rgb}{0.000000,0.000000,0.000000}%
\pgfsetstrokecolor{currentstroke}%
\pgfsetdash{}{0pt}%
\pgfpathmoveto{\pgfqpoint{1.711743in}{0.753140in}}%
\pgfpathlineto{\pgfqpoint{1.715283in}{0.751587in}}%
\pgfpathlineto{\pgfqpoint{1.718830in}{0.750273in}}%
\pgfpathlineto{\pgfqpoint{1.722384in}{0.749202in}}%
\pgfpathlineto{\pgfqpoint{1.725946in}{0.748379in}}%
\pgfpathlineto{\pgfqpoint{1.718641in}{0.739277in}}%
\pgfpathlineto{\pgfqpoint{1.710770in}{0.730291in}}%
\pgfpathlineto{\pgfqpoint{1.702339in}{0.721431in}}%
\pgfpathlineto{\pgfqpoint{1.693356in}{0.712707in}}%
\pgfpathlineto{\pgfqpoint{1.689991in}{0.713767in}}%
\pgfpathlineto{\pgfqpoint{1.686633in}{0.715077in}}%
\pgfpathlineto{\pgfqpoint{1.683282in}{0.716630in}}%
\pgfpathlineto{\pgfqpoint{1.679938in}{0.718423in}}%
\pgfpathlineto{\pgfqpoint{1.688702in}{0.726912in}}%
\pgfpathlineto{\pgfqpoint{1.696929in}{0.735535in}}%
\pgfpathlineto{\pgfqpoint{1.704611in}{0.744280in}}%
\pgfpathlineto{\pgfqpoint{1.711743in}{0.753140in}}%
\pgfpathclose%
\pgfusepath{fill}%
\end{pgfscope}%
\begin{pgfscope}%
\pgfpathrectangle{\pgfqpoint{0.041670in}{0.041670in}}{\pgfqpoint{2.216660in}{2.216660in}}%
\pgfusepath{clip}%
\pgfsetbuttcap%
\pgfsetroundjoin%
\definecolor{currentfill}{rgb}{0.134692,0.658636,0.517649}%
\pgfsetfillcolor{currentfill}%
\pgfsetlinewidth{0.000000pt}%
\definecolor{currentstroke}{rgb}{0.000000,0.000000,0.000000}%
\pgfsetstrokecolor{currentstroke}%
\pgfsetdash{}{0pt}%
\pgfpathmoveto{\pgfqpoint{0.903232in}{1.356600in}}%
\pgfpathlineto{\pgfqpoint{0.899492in}{1.349348in}}%
\pgfpathlineto{\pgfqpoint{0.895755in}{1.342057in}}%
\pgfpathlineto{\pgfqpoint{0.892021in}{1.334730in}}%
\pgfpathlineto{\pgfqpoint{0.888291in}{1.327370in}}%
\pgfpathlineto{\pgfqpoint{0.889155in}{1.331869in}}%
\pgfpathlineto{\pgfqpoint{0.890307in}{1.336349in}}%
\pgfpathlineto{\pgfqpoint{0.891746in}{1.340806in}}%
\pgfpathlineto{\pgfqpoint{0.893468in}{1.345235in}}%
\pgfpathlineto{\pgfqpoint{0.897123in}{1.352349in}}%
\pgfpathlineto{\pgfqpoint{0.900781in}{1.359429in}}%
\pgfpathlineto{\pgfqpoint{0.904442in}{1.366473in}}%
\pgfpathlineto{\pgfqpoint{0.908108in}{1.373479in}}%
\pgfpathlineto{\pgfqpoint{0.906482in}{1.369293in}}%
\pgfpathlineto{\pgfqpoint{0.905126in}{1.365082in}}%
\pgfpathlineto{\pgfqpoint{0.904042in}{1.360850in}}%
\pgfpathlineto{\pgfqpoint{0.903232in}{1.356600in}}%
\pgfpathclose%
\pgfusepath{fill}%
\end{pgfscope}%
\begin{pgfscope}%
\pgfpathrectangle{\pgfqpoint{0.041670in}{0.041670in}}{\pgfqpoint{2.216660in}{2.216660in}}%
\pgfusepath{clip}%
\pgfsetbuttcap%
\pgfsetroundjoin%
\definecolor{currentfill}{rgb}{0.231674,0.318106,0.544834}%
\pgfsetfillcolor{currentfill}%
\pgfsetlinewidth{0.000000pt}%
\definecolor{currentstroke}{rgb}{0.000000,0.000000,0.000000}%
\pgfsetstrokecolor{currentstroke}%
\pgfsetdash{}{0pt}%
\pgfpathmoveto{\pgfqpoint{1.576612in}{1.017446in}}%
\pgfpathlineto{\pgfqpoint{1.580179in}{1.009798in}}%
\pgfpathlineto{\pgfqpoint{1.583746in}{1.002214in}}%
\pgfpathlineto{\pgfqpoint{1.587312in}{0.994698in}}%
\pgfpathlineto{\pgfqpoint{1.590877in}{0.987252in}}%
\pgfpathlineto{\pgfqpoint{1.587127in}{0.980611in}}%
\pgfpathlineto{\pgfqpoint{1.582960in}{0.974028in}}%
\pgfpathlineto{\pgfqpoint{1.578379in}{0.967510in}}%
\pgfpathlineto{\pgfqpoint{1.573387in}{0.961064in}}%
\pgfpathlineto{\pgfqpoint{1.569961in}{0.968765in}}%
\pgfpathlineto{\pgfqpoint{1.566535in}{0.976536in}}%
\pgfpathlineto{\pgfqpoint{1.563108in}{0.984375in}}%
\pgfpathlineto{\pgfqpoint{1.559681in}{0.992278in}}%
\pgfpathlineto{\pgfqpoint{1.564511in}{0.998472in}}%
\pgfpathlineto{\pgfqpoint{1.568945in}{1.004736in}}%
\pgfpathlineto{\pgfqpoint{1.572980in}{1.011063in}}%
\pgfpathlineto{\pgfqpoint{1.576612in}{1.017446in}}%
\pgfpathclose%
\pgfusepath{fill}%
\end{pgfscope}%
\begin{pgfscope}%
\pgfpathrectangle{\pgfqpoint{0.041670in}{0.041670in}}{\pgfqpoint{2.216660in}{2.216660in}}%
\pgfusepath{clip}%
\pgfsetbuttcap%
\pgfsetroundjoin%
\definecolor{currentfill}{rgb}{0.344074,0.780029,0.397381}%
\pgfsetfillcolor{currentfill}%
\pgfsetlinewidth{0.000000pt}%
\definecolor{currentstroke}{rgb}{0.000000,0.000000,0.000000}%
\pgfsetstrokecolor{currentstroke}%
\pgfsetdash{}{0pt}%
\pgfpathmoveto{\pgfqpoint{0.984569in}{1.503232in}}%
\pgfpathlineto{\pgfqpoint{0.981164in}{1.497602in}}%
\pgfpathlineto{\pgfqpoint{0.977762in}{1.491900in}}%
\pgfpathlineto{\pgfqpoint{0.974364in}{1.486126in}}%
\pgfpathlineto{\pgfqpoint{0.970970in}{1.480283in}}%
\pgfpathlineto{\pgfqpoint{0.974244in}{1.483414in}}%
\pgfpathlineto{\pgfqpoint{0.977719in}{1.486492in}}%
\pgfpathlineto{\pgfqpoint{0.981391in}{1.489515in}}%
\pgfpathlineto{\pgfqpoint{0.985255in}{1.492480in}}%
\pgfpathlineto{\pgfqpoint{0.988412in}{1.498115in}}%
\pgfpathlineto{\pgfqpoint{0.991572in}{1.503680in}}%
\pgfpathlineto{\pgfqpoint{0.994736in}{1.509175in}}%
\pgfpathlineto{\pgfqpoint{0.997903in}{1.514597in}}%
\pgfpathlineto{\pgfqpoint{0.994295in}{1.511834in}}%
\pgfpathlineto{\pgfqpoint{0.990868in}{1.509017in}}%
\pgfpathlineto{\pgfqpoint{0.987625in}{1.506149in}}%
\pgfpathlineto{\pgfqpoint{0.984569in}{1.503232in}}%
\pgfpathclose%
\pgfusepath{fill}%
\end{pgfscope}%
\begin{pgfscope}%
\pgfpathrectangle{\pgfqpoint{0.041670in}{0.041670in}}{\pgfqpoint{2.216660in}{2.216660in}}%
\pgfusepath{clip}%
\pgfsetbuttcap%
\pgfsetroundjoin%
\definecolor{currentfill}{rgb}{0.412913,0.803041,0.357269}%
\pgfsetfillcolor{currentfill}%
\pgfsetlinewidth{0.000000pt}%
\definecolor{currentstroke}{rgb}{0.000000,0.000000,0.000000}%
\pgfsetstrokecolor{currentstroke}%
\pgfsetdash{}{0pt}%
\pgfpathmoveto{\pgfqpoint{1.010607in}{1.535517in}}%
\pgfpathlineto{\pgfqpoint{1.007426in}{1.530406in}}%
\pgfpathlineto{\pgfqpoint{1.004248in}{1.525214in}}%
\pgfpathlineto{\pgfqpoint{1.001074in}{1.519944in}}%
\pgfpathlineto{\pgfqpoint{0.997903in}{1.514597in}}%
\pgfpathlineto{\pgfqpoint{1.001688in}{1.517303in}}%
\pgfpathlineto{\pgfqpoint{1.005645in}{1.519950in}}%
\pgfpathlineto{\pgfqpoint{1.009772in}{1.522536in}}%
\pgfpathlineto{\pgfqpoint{1.014064in}{1.525057in}}%
\pgfpathlineto{\pgfqpoint{1.016948in}{1.530214in}}%
\pgfpathlineto{\pgfqpoint{1.019837in}{1.535294in}}%
\pgfpathlineto{\pgfqpoint{1.022728in}{1.540295in}}%
\pgfpathlineto{\pgfqpoint{1.025623in}{1.545217in}}%
\pgfpathlineto{\pgfqpoint{1.021635in}{1.542878in}}%
\pgfpathlineto{\pgfqpoint{1.017800in}{1.540481in}}%
\pgfpathlineto{\pgfqpoint{1.014123in}{1.538026in}}%
\pgfpathlineto{\pgfqpoint{1.010607in}{1.535517in}}%
\pgfpathclose%
\pgfusepath{fill}%
\end{pgfscope}%
\begin{pgfscope}%
\pgfpathrectangle{\pgfqpoint{0.041670in}{0.041670in}}{\pgfqpoint{2.216660in}{2.216660in}}%
\pgfusepath{clip}%
\pgfsetbuttcap%
\pgfsetroundjoin%
\definecolor{currentfill}{rgb}{0.283072,0.130895,0.449241}%
\pgfsetfillcolor{currentfill}%
\pgfsetlinewidth{0.000000pt}%
\definecolor{currentstroke}{rgb}{0.000000,0.000000,0.000000}%
\pgfsetstrokecolor{currentstroke}%
\pgfsetdash{}{0pt}%
\pgfpathmoveto{\pgfqpoint{1.628264in}{0.850341in}}%
\pgfpathlineto{\pgfqpoint{1.631704in}{0.844386in}}%
\pgfpathlineto{\pgfqpoint{1.635146in}{0.838566in}}%
\pgfpathlineto{\pgfqpoint{1.638590in}{0.832887in}}%
\pgfpathlineto{\pgfqpoint{1.642036in}{0.827351in}}%
\pgfpathlineto{\pgfqpoint{1.635767in}{0.819731in}}%
\pgfpathlineto{\pgfqpoint{1.629022in}{0.812212in}}%
\pgfpathlineto{\pgfqpoint{1.621809in}{0.804801in}}%
\pgfpathlineto{\pgfqpoint{1.614131in}{0.797507in}}%
\pgfpathlineto{\pgfqpoint{1.610880in}{0.803289in}}%
\pgfpathlineto{\pgfqpoint{1.607630in}{0.809215in}}%
\pgfpathlineto{\pgfqpoint{1.604384in}{0.815281in}}%
\pgfpathlineto{\pgfqpoint{1.601139in}{0.821483in}}%
\pgfpathlineto{\pgfqpoint{1.608599in}{0.828535in}}%
\pgfpathlineto{\pgfqpoint{1.615611in}{0.835701in}}%
\pgfpathlineto{\pgfqpoint{1.622168in}{0.842973in}}%
\pgfpathlineto{\pgfqpoint{1.628264in}{0.850341in}}%
\pgfpathclose%
\pgfusepath{fill}%
\end{pgfscope}%
\begin{pgfscope}%
\pgfpathrectangle{\pgfqpoint{0.041670in}{0.041670in}}{\pgfqpoint{2.216660in}{2.216660in}}%
\pgfusepath{clip}%
\pgfsetbuttcap%
\pgfsetroundjoin%
\definecolor{currentfill}{rgb}{0.220124,0.725509,0.466226}%
\pgfsetfillcolor{currentfill}%
\pgfsetlinewidth{0.000000pt}%
\definecolor{currentstroke}{rgb}{0.000000,0.000000,0.000000}%
\pgfsetstrokecolor{currentstroke}%
\pgfsetdash{}{0pt}%
\pgfpathmoveto{\pgfqpoint{1.411947in}{1.445531in}}%
\pgfpathlineto{\pgfqpoint{1.415472in}{1.439201in}}%
\pgfpathlineto{\pgfqpoint{1.418994in}{1.432815in}}%
\pgfpathlineto{\pgfqpoint{1.422512in}{1.426375in}}%
\pgfpathlineto{\pgfqpoint{1.426026in}{1.419883in}}%
\pgfpathlineto{\pgfqpoint{1.428772in}{1.416111in}}%
\pgfpathlineto{\pgfqpoint{1.431273in}{1.412296in}}%
\pgfpathlineto{\pgfqpoint{1.433529in}{1.408443in}}%
\pgfpathlineto{\pgfqpoint{1.435535in}{1.404556in}}%
\pgfpathlineto{\pgfqpoint{1.431876in}{1.411282in}}%
\pgfpathlineto{\pgfqpoint{1.428214in}{1.417956in}}%
\pgfpathlineto{\pgfqpoint{1.424549in}{1.424575in}}%
\pgfpathlineto{\pgfqpoint{1.420879in}{1.431138in}}%
\pgfpathlineto{\pgfqpoint{1.418996in}{1.434788in}}%
\pgfpathlineto{\pgfqpoint{1.416878in}{1.438406in}}%
\pgfpathlineto{\pgfqpoint{1.414528in}{1.441988in}}%
\pgfpathlineto{\pgfqpoint{1.411947in}{1.445531in}}%
\pgfpathclose%
\pgfusepath{fill}%
\end{pgfscope}%
\begin{pgfscope}%
\pgfpathrectangle{\pgfqpoint{0.041670in}{0.041670in}}{\pgfqpoint{2.216660in}{2.216660in}}%
\pgfusepath{clip}%
\pgfsetbuttcap%
\pgfsetroundjoin%
\definecolor{currentfill}{rgb}{0.565498,0.842430,0.262877}%
\pgfsetfillcolor{currentfill}%
\pgfsetlinewidth{0.000000pt}%
\definecolor{currentstroke}{rgb}{0.000000,0.000000,0.000000}%
\pgfsetstrokecolor{currentstroke}%
\pgfsetdash{}{0pt}%
\pgfpathmoveto{\pgfqpoint{1.257385in}{1.601691in}}%
\pgfpathlineto{\pgfqpoint{1.259105in}{1.597941in}}%
\pgfpathlineto{\pgfqpoint{1.260823in}{1.594097in}}%
\pgfpathlineto{\pgfqpoint{1.262539in}{1.590160in}}%
\pgfpathlineto{\pgfqpoint{1.264253in}{1.586133in}}%
\pgfpathlineto{\pgfqpoint{1.269399in}{1.584842in}}%
\pgfpathlineto{\pgfqpoint{1.274459in}{1.583474in}}%
\pgfpathlineto{\pgfqpoint{1.279428in}{1.582032in}}%
\pgfpathlineto{\pgfqpoint{1.284301in}{1.580517in}}%
\pgfpathlineto{\pgfqpoint{1.282177in}{1.584662in}}%
\pgfpathlineto{\pgfqpoint{1.280052in}{1.588716in}}%
\pgfpathlineto{\pgfqpoint{1.277923in}{1.592678in}}%
\pgfpathlineto{\pgfqpoint{1.275792in}{1.596546in}}%
\pgfpathlineto{\pgfqpoint{1.271318in}{1.597934in}}%
\pgfpathlineto{\pgfqpoint{1.266756in}{1.599256in}}%
\pgfpathlineto{\pgfqpoint{1.262110in}{1.600508in}}%
\pgfpathlineto{\pgfqpoint{1.257385in}{1.601691in}}%
\pgfpathclose%
\pgfusepath{fill}%
\end{pgfscope}%
\begin{pgfscope}%
\pgfpathrectangle{\pgfqpoint{0.041670in}{0.041670in}}{\pgfqpoint{2.216660in}{2.216660in}}%
\pgfusepath{clip}%
\pgfsetbuttcap%
\pgfsetroundjoin%
\definecolor{currentfill}{rgb}{0.281477,0.755203,0.432552}%
\pgfsetfillcolor{currentfill}%
\pgfsetlinewidth{0.000000pt}%
\definecolor{currentstroke}{rgb}{0.000000,0.000000,0.000000}%
\pgfsetstrokecolor{currentstroke}%
\pgfsetdash{}{0pt}%
\pgfpathmoveto{\pgfqpoint{0.959943in}{1.467297in}}%
\pgfpathlineto{\pgfqpoint{0.956366in}{1.461164in}}%
\pgfpathlineto{\pgfqpoint{0.952793in}{1.454966in}}%
\pgfpathlineto{\pgfqpoint{0.949223in}{1.448705in}}%
\pgfpathlineto{\pgfqpoint{0.945658in}{1.442384in}}%
\pgfpathlineto{\pgfqpoint{0.948264in}{1.445922in}}%
\pgfpathlineto{\pgfqpoint{0.951096in}{1.449417in}}%
\pgfpathlineto{\pgfqpoint{0.954153in}{1.452865in}}%
\pgfpathlineto{\pgfqpoint{0.957429in}{1.456264in}}%
\pgfpathlineto{\pgfqpoint{0.960809in}{1.462362in}}%
\pgfpathlineto{\pgfqpoint{0.964192in}{1.468399in}}%
\pgfpathlineto{\pgfqpoint{0.967579in}{1.474373in}}%
\pgfpathlineto{\pgfqpoint{0.970970in}{1.480283in}}%
\pgfpathlineto{\pgfqpoint{0.967899in}{1.477103in}}%
\pgfpathlineto{\pgfqpoint{0.965036in}{1.473876in}}%
\pgfpathlineto{\pgfqpoint{0.962383in}{1.470607in}}%
\pgfpathlineto{\pgfqpoint{0.959943in}{1.467297in}}%
\pgfpathclose%
\pgfusepath{fill}%
\end{pgfscope}%
\begin{pgfscope}%
\pgfpathrectangle{\pgfqpoint{0.041670in}{0.041670in}}{\pgfqpoint{2.216660in}{2.216660in}}%
\pgfusepath{clip}%
\pgfsetbuttcap%
\pgfsetroundjoin%
\definecolor{currentfill}{rgb}{0.260571,0.246922,0.522828}%
\pgfsetfillcolor{currentfill}%
\pgfsetlinewidth{0.000000pt}%
\definecolor{currentstroke}{rgb}{0.000000,0.000000,0.000000}%
\pgfsetstrokecolor{currentstroke}%
\pgfsetdash{}{0pt}%
\pgfpathmoveto{\pgfqpoint{0.507842in}{0.881698in}}%
\pgfpathlineto{\pgfqpoint{0.503814in}{0.890625in}}%
\pgfpathlineto{\pgfqpoint{0.499767in}{0.899973in}}%
\pgfpathlineto{\pgfqpoint{0.495702in}{0.909752in}}%
\pgfpathlineto{\pgfqpoint{0.491618in}{0.919968in}}%
\pgfpathlineto{\pgfqpoint{0.485811in}{0.931283in}}%
\pgfpathlineto{\pgfqpoint{0.480723in}{0.942669in}}%
\pgfpathlineto{\pgfqpoint{0.476357in}{0.954114in}}%
\pgfpathlineto{\pgfqpoint{0.472713in}{0.965605in}}%
\pgfpathlineto{\pgfqpoint{0.476883in}{0.955185in}}%
\pgfpathlineto{\pgfqpoint{0.481033in}{0.945199in}}%
\pgfpathlineto{\pgfqpoint{0.485165in}{0.935641in}}%
\pgfpathlineto{\pgfqpoint{0.489279in}{0.926504in}}%
\pgfpathlineto{\pgfqpoint{0.492863in}{0.915220in}}%
\pgfpathlineto{\pgfqpoint{0.497153in}{0.903984in}}%
\pgfpathlineto{\pgfqpoint{0.502146in}{0.892806in}}%
\pgfpathlineto{\pgfqpoint{0.507842in}{0.881698in}}%
\pgfpathclose%
\pgfusepath{fill}%
\end{pgfscope}%
\begin{pgfscope}%
\pgfpathrectangle{\pgfqpoint{0.041670in}{0.041670in}}{\pgfqpoint{2.216660in}{2.216660in}}%
\pgfusepath{clip}%
\pgfsetbuttcap%
\pgfsetroundjoin%
\definecolor{currentfill}{rgb}{0.487026,0.823929,0.312321}%
\pgfsetfillcolor{currentfill}%
\pgfsetlinewidth{0.000000pt}%
\definecolor{currentstroke}{rgb}{0.000000,0.000000,0.000000}%
\pgfsetstrokecolor{currentstroke}%
\pgfsetdash{}{0pt}%
\pgfpathmoveto{\pgfqpoint{1.302740in}{1.573748in}}%
\pgfpathlineto{\pgfqpoint{1.305239in}{1.569371in}}%
\pgfpathlineto{\pgfqpoint{1.307734in}{1.564907in}}%
\pgfpathlineto{\pgfqpoint{1.310226in}{1.560357in}}%
\pgfpathlineto{\pgfqpoint{1.312715in}{1.555722in}}%
\pgfpathlineto{\pgfqpoint{1.317394in}{1.553705in}}%
\pgfpathlineto{\pgfqpoint{1.321940in}{1.551618in}}%
\pgfpathlineto{\pgfqpoint{1.326349in}{1.549463in}}%
\pgfpathlineto{\pgfqpoint{1.330616in}{1.547243in}}%
\pgfpathlineto{\pgfqpoint{1.327787in}{1.552043in}}%
\pgfpathlineto{\pgfqpoint{1.324956in}{1.556758in}}%
\pgfpathlineto{\pgfqpoint{1.322121in}{1.561387in}}%
\pgfpathlineto{\pgfqpoint{1.319283in}{1.565929in}}%
\pgfpathlineto{\pgfqpoint{1.315340in}{1.567976in}}%
\pgfpathlineto{\pgfqpoint{1.311266in}{1.569963in}}%
\pgfpathlineto{\pgfqpoint{1.307065in}{1.571888in}}%
\pgfpathlineto{\pgfqpoint{1.302740in}{1.573748in}}%
\pgfpathclose%
\pgfusepath{fill}%
\end{pgfscope}%
\begin{pgfscope}%
\pgfpathrectangle{\pgfqpoint{0.041670in}{0.041670in}}{\pgfqpoint{2.216660in}{2.216660in}}%
\pgfusepath{clip}%
\pgfsetbuttcap%
\pgfsetroundjoin%
\definecolor{currentfill}{rgb}{0.274952,0.037752,0.364543}%
\pgfsetfillcolor{currentfill}%
\pgfsetlinewidth{0.000000pt}%
\definecolor{currentstroke}{rgb}{0.000000,0.000000,0.000000}%
\pgfsetstrokecolor{currentstroke}%
\pgfsetdash{}{0pt}%
\pgfpathmoveto{\pgfqpoint{0.727277in}{0.750110in}}%
\pgfpathlineto{\pgfqpoint{0.724048in}{0.745753in}}%
\pgfpathlineto{\pgfqpoint{0.720814in}{0.741580in}}%
\pgfpathlineto{\pgfqpoint{0.717576in}{0.737596in}}%
\pgfpathlineto{\pgfqpoint{0.714333in}{0.733807in}}%
\pgfpathlineto{\pgfqpoint{0.705560in}{0.741702in}}%
\pgfpathlineto{\pgfqpoint{0.697288in}{0.749731in}}%
\pgfpathlineto{\pgfqpoint{0.689524in}{0.757886in}}%
\pgfpathlineto{\pgfqpoint{0.682275in}{0.766158in}}%
\pgfpathlineto{\pgfqpoint{0.685725in}{0.769706in}}%
\pgfpathlineto{\pgfqpoint{0.689171in}{0.773449in}}%
\pgfpathlineto{\pgfqpoint{0.692613in}{0.777380in}}%
\pgfpathlineto{\pgfqpoint{0.696051in}{0.781496in}}%
\pgfpathlineto{\pgfqpoint{0.703114in}{0.773471in}}%
\pgfpathlineto{\pgfqpoint{0.710677in}{0.765559in}}%
\pgfpathlineto{\pgfqpoint{0.718734in}{0.757769in}}%
\pgfpathlineto{\pgfqpoint{0.727277in}{0.750110in}}%
\pgfpathclose%
\pgfusepath{fill}%
\end{pgfscope}%
\begin{pgfscope}%
\pgfpathrectangle{\pgfqpoint{0.041670in}{0.041670in}}{\pgfqpoint{2.216660in}{2.216660in}}%
\pgfusepath{clip}%
\pgfsetbuttcap%
\pgfsetroundjoin%
\definecolor{currentfill}{rgb}{0.271305,0.019942,0.347269}%
\pgfsetfillcolor{currentfill}%
\pgfsetlinewidth{0.000000pt}%
\definecolor{currentstroke}{rgb}{0.000000,0.000000,0.000000}%
\pgfsetstrokecolor{currentstroke}%
\pgfsetdash{}{0pt}%
\pgfpathmoveto{\pgfqpoint{0.714333in}{0.733807in}}%
\pgfpathlineto{\pgfqpoint{0.711086in}{0.730215in}}%
\pgfpathlineto{\pgfqpoint{0.707834in}{0.726826in}}%
\pgfpathlineto{\pgfqpoint{0.704577in}{0.723643in}}%
\pgfpathlineto{\pgfqpoint{0.701314in}{0.720673in}}%
\pgfpathlineto{\pgfqpoint{0.692310in}{0.728802in}}%
\pgfpathlineto{\pgfqpoint{0.683822in}{0.737070in}}%
\pgfpathlineto{\pgfqpoint{0.675857in}{0.745467in}}%
\pgfpathlineto{\pgfqpoint{0.668422in}{0.753983in}}%
\pgfpathlineto{\pgfqpoint{0.671893in}{0.756715in}}%
\pgfpathlineto{\pgfqpoint{0.675359in}{0.759657in}}%
\pgfpathlineto{\pgfqpoint{0.678819in}{0.762806in}}%
\pgfpathlineto{\pgfqpoint{0.682275in}{0.766158in}}%
\pgfpathlineto{\pgfqpoint{0.689524in}{0.757886in}}%
\pgfpathlineto{\pgfqpoint{0.697288in}{0.749731in}}%
\pgfpathlineto{\pgfqpoint{0.705560in}{0.741702in}}%
\pgfpathlineto{\pgfqpoint{0.714333in}{0.733807in}}%
\pgfpathclose%
\pgfusepath{fill}%
\end{pgfscope}%
\begin{pgfscope}%
\pgfpathrectangle{\pgfqpoint{0.041670in}{0.041670in}}{\pgfqpoint{2.216660in}{2.216660in}}%
\pgfusepath{clip}%
\pgfsetbuttcap%
\pgfsetroundjoin%
\definecolor{currentfill}{rgb}{0.565498,0.842430,0.262877}%
\pgfsetfillcolor{currentfill}%
\pgfsetlinewidth{0.000000pt}%
\definecolor{currentstroke}{rgb}{0.000000,0.000000,0.000000}%
\pgfsetstrokecolor{currentstroke}%
\pgfsetdash{}{0pt}%
\pgfpathmoveto{\pgfqpoint{1.080218in}{1.595256in}}%
\pgfpathlineto{\pgfqpoint{1.078000in}{1.591359in}}%
\pgfpathlineto{\pgfqpoint{1.075785in}{1.587368in}}%
\pgfpathlineto{\pgfqpoint{1.073572in}{1.583284in}}%
\pgfpathlineto{\pgfqpoint{1.071362in}{1.579109in}}%
\pgfpathlineto{\pgfqpoint{1.076146in}{1.580689in}}%
\pgfpathlineto{\pgfqpoint{1.081030in}{1.582196in}}%
\pgfpathlineto{\pgfqpoint{1.086009in}{1.583630in}}%
\pgfpathlineto{\pgfqpoint{1.091078in}{1.584989in}}%
\pgfpathlineto{\pgfqpoint{1.092885in}{1.589041in}}%
\pgfpathlineto{\pgfqpoint{1.094695in}{1.593001in}}%
\pgfpathlineto{\pgfqpoint{1.096507in}{1.596869in}}%
\pgfpathlineto{\pgfqpoint{1.098321in}{1.600643in}}%
\pgfpathlineto{\pgfqpoint{1.093666in}{1.599398in}}%
\pgfpathlineto{\pgfqpoint{1.089094in}{1.598084in}}%
\pgfpathlineto{\pgfqpoint{1.084610in}{1.596703in}}%
\pgfpathlineto{\pgfqpoint{1.080218in}{1.595256in}}%
\pgfpathclose%
\pgfusepath{fill}%
\end{pgfscope}%
\begin{pgfscope}%
\pgfpathrectangle{\pgfqpoint{0.041670in}{0.041670in}}{\pgfqpoint{2.216660in}{2.216660in}}%
\pgfusepath{clip}%
\pgfsetbuttcap%
\pgfsetroundjoin%
\definecolor{currentfill}{rgb}{0.267004,0.004874,0.329415}%
\pgfsetfillcolor{currentfill}%
\pgfsetlinewidth{0.000000pt}%
\definecolor{currentstroke}{rgb}{0.000000,0.000000,0.000000}%
\pgfsetstrokecolor{currentstroke}%
\pgfsetdash{}{0pt}%
\pgfpathmoveto{\pgfqpoint{1.725946in}{0.748379in}}%
\pgfpathlineto{\pgfqpoint{1.729515in}{0.747810in}}%
\pgfpathlineto{\pgfqpoint{1.733092in}{0.747500in}}%
\pgfpathlineto{\pgfqpoint{1.736677in}{0.747452in}}%
\pgfpathlineto{\pgfqpoint{1.740271in}{0.747673in}}%
\pgfpathlineto{\pgfqpoint{1.732792in}{0.738331in}}%
\pgfpathlineto{\pgfqpoint{1.724732in}{0.729107in}}%
\pgfpathlineto{\pgfqpoint{1.716097in}{0.720012in}}%
\pgfpathlineto{\pgfqpoint{1.706893in}{0.711057in}}%
\pgfpathlineto{\pgfqpoint{1.703497in}{0.711071in}}%
\pgfpathlineto{\pgfqpoint{1.700109in}{0.711354in}}%
\pgfpathlineto{\pgfqpoint{1.696729in}{0.711901in}}%
\pgfpathlineto{\pgfqpoint{1.693356in}{0.712707in}}%
\pgfpathlineto{\pgfqpoint{1.702339in}{0.721431in}}%
\pgfpathlineto{\pgfqpoint{1.710770in}{0.730291in}}%
\pgfpathlineto{\pgfqpoint{1.718641in}{0.739277in}}%
\pgfpathlineto{\pgfqpoint{1.725946in}{0.748379in}}%
\pgfpathclose%
\pgfusepath{fill}%
\end{pgfscope}%
\begin{pgfscope}%
\pgfpathrectangle{\pgfqpoint{0.041670in}{0.041670in}}{\pgfqpoint{2.216660in}{2.216660in}}%
\pgfusepath{clip}%
\pgfsetbuttcap%
\pgfsetroundjoin%
\definecolor{currentfill}{rgb}{0.279566,0.067836,0.391917}%
\pgfsetfillcolor{currentfill}%
\pgfsetlinewidth{0.000000pt}%
\definecolor{currentstroke}{rgb}{0.000000,0.000000,0.000000}%
\pgfsetstrokecolor{currentstroke}%
\pgfsetdash{}{0pt}%
\pgfpathmoveto{\pgfqpoint{0.740158in}{0.769307in}}%
\pgfpathlineto{\pgfqpoint{0.736943in}{0.764251in}}%
\pgfpathlineto{\pgfqpoint{0.733725in}{0.759364in}}%
\pgfpathlineto{\pgfqpoint{0.730503in}{0.754649in}}%
\pgfpathlineto{\pgfqpoint{0.727277in}{0.750110in}}%
\pgfpathlineto{\pgfqpoint{0.718734in}{0.757769in}}%
\pgfpathlineto{\pgfqpoint{0.710677in}{0.765559in}}%
\pgfpathlineto{\pgfqpoint{0.703114in}{0.773471in}}%
\pgfpathlineto{\pgfqpoint{0.696051in}{0.781496in}}%
\pgfpathlineto{\pgfqpoint{0.699484in}{0.785793in}}%
\pgfpathlineto{\pgfqpoint{0.702914in}{0.790266in}}%
\pgfpathlineto{\pgfqpoint{0.706340in}{0.794911in}}%
\pgfpathlineto{\pgfqpoint{0.709763in}{0.799724in}}%
\pgfpathlineto{\pgfqpoint{0.716640in}{0.791945in}}%
\pgfpathlineto{\pgfqpoint{0.724003in}{0.784277in}}%
\pgfpathlineto{\pgfqpoint{0.731844in}{0.776728in}}%
\pgfpathlineto{\pgfqpoint{0.740158in}{0.769307in}}%
\pgfpathclose%
\pgfusepath{fill}%
\end{pgfscope}%
\begin{pgfscope}%
\pgfpathrectangle{\pgfqpoint{0.041670in}{0.041670in}}{\pgfqpoint{2.216660in}{2.216660in}}%
\pgfusepath{clip}%
\pgfsetbuttcap%
\pgfsetroundjoin%
\definecolor{currentfill}{rgb}{0.120081,0.622161,0.534946}%
\pgfsetfillcolor{currentfill}%
\pgfsetlinewidth{0.000000pt}%
\definecolor{currentstroke}{rgb}{0.000000,0.000000,0.000000}%
\pgfsetstrokecolor{currentstroke}%
\pgfsetdash{}{0pt}%
\pgfpathmoveto{\pgfqpoint{1.470865in}{1.331370in}}%
\pgfpathlineto{\pgfqpoint{1.474580in}{1.324034in}}%
\pgfpathlineto{\pgfqpoint{1.478292in}{1.316669in}}%
\pgfpathlineto{\pgfqpoint{1.482000in}{1.309279in}}%
\pgfpathlineto{\pgfqpoint{1.485706in}{1.301865in}}%
\pgfpathlineto{\pgfqpoint{1.486591in}{1.297114in}}%
\pgfpathlineto{\pgfqpoint{1.487171in}{1.292349in}}%
\pgfpathlineto{\pgfqpoint{1.487446in}{1.287573in}}%
\pgfpathlineto{\pgfqpoint{1.487415in}{1.282793in}}%
\pgfpathlineto{\pgfqpoint{1.483679in}{1.290459in}}%
\pgfpathlineto{\pgfqpoint{1.479940in}{1.298101in}}%
\pgfpathlineto{\pgfqpoint{1.476197in}{1.305717in}}%
\pgfpathlineto{\pgfqpoint{1.472452in}{1.313304in}}%
\pgfpathlineto{\pgfqpoint{1.472491in}{1.317832in}}%
\pgfpathlineto{\pgfqpoint{1.472240in}{1.322355in}}%
\pgfpathlineto{\pgfqpoint{1.471697in}{1.326869in}}%
\pgfpathlineto{\pgfqpoint{1.470865in}{1.331370in}}%
\pgfpathclose%
\pgfusepath{fill}%
\end{pgfscope}%
\begin{pgfscope}%
\pgfpathrectangle{\pgfqpoint{0.041670in}{0.041670in}}{\pgfqpoint{2.216660in}{2.216660in}}%
\pgfusepath{clip}%
\pgfsetbuttcap%
\pgfsetroundjoin%
\definecolor{currentfill}{rgb}{0.133743,0.548535,0.553541}%
\pgfsetfillcolor{currentfill}%
\pgfsetlinewidth{0.000000pt}%
\definecolor{currentstroke}{rgb}{0.000000,0.000000,0.000000}%
\pgfsetstrokecolor{currentstroke}%
\pgfsetdash{}{0pt}%
\pgfpathmoveto{\pgfqpoint{1.502330in}{1.251943in}}%
\pgfpathlineto{\pgfqpoint{1.506052in}{1.244198in}}%
\pgfpathlineto{\pgfqpoint{1.509770in}{1.236445in}}%
\pgfpathlineto{\pgfqpoint{1.513486in}{1.228687in}}%
\pgfpathlineto{\pgfqpoint{1.517199in}{1.220927in}}%
\pgfpathlineto{\pgfqpoint{1.516851in}{1.215635in}}%
\pgfpathlineto{\pgfqpoint{1.516166in}{1.210346in}}%
\pgfpathlineto{\pgfqpoint{1.515143in}{1.205068in}}%
\pgfpathlineto{\pgfqpoint{1.513783in}{1.199803in}}%
\pgfpathlineto{\pgfqpoint{1.510097in}{1.207820in}}%
\pgfpathlineto{\pgfqpoint{1.506408in}{1.215835in}}%
\pgfpathlineto{\pgfqpoint{1.502716in}{1.223844in}}%
\pgfpathlineto{\pgfqpoint{1.499022in}{1.231845in}}%
\pgfpathlineto{\pgfqpoint{1.500332in}{1.236853in}}%
\pgfpathlineto{\pgfqpoint{1.501321in}{1.241876in}}%
\pgfpathlineto{\pgfqpoint{1.501987in}{1.246907in}}%
\pgfpathlineto{\pgfqpoint{1.502330in}{1.251943in}}%
\pgfpathclose%
\pgfusepath{fill}%
\end{pgfscope}%
\begin{pgfscope}%
\pgfpathrectangle{\pgfqpoint{0.041670in}{0.041670in}}{\pgfqpoint{2.216660in}{2.216660in}}%
\pgfusepath{clip}%
\pgfsetbuttcap%
\pgfsetroundjoin%
\definecolor{currentfill}{rgb}{0.636902,0.856542,0.216620}%
\pgfsetfillcolor{currentfill}%
\pgfsetlinewidth{0.000000pt}%
\definecolor{currentstroke}{rgb}{0.000000,0.000000,0.000000}%
\pgfsetstrokecolor{currentstroke}%
\pgfsetdash{}{0pt}%
\pgfpathmoveto{\pgfqpoint{1.160834in}{1.623153in}}%
\pgfpathlineto{\pgfqpoint{1.160366in}{1.619982in}}%
\pgfpathlineto{\pgfqpoint{1.159898in}{1.616711in}}%
\pgfpathlineto{\pgfqpoint{1.159431in}{1.613341in}}%
\pgfpathlineto{\pgfqpoint{1.158965in}{1.609873in}}%
\pgfpathlineto{\pgfqpoint{1.164269in}{1.610145in}}%
\pgfpathlineto{\pgfqpoint{1.169588in}{1.610338in}}%
\pgfpathlineto{\pgfqpoint{1.174917in}{1.610452in}}%
\pgfpathlineto{\pgfqpoint{1.180251in}{1.610487in}}%
\pgfpathlineto{\pgfqpoint{1.180245in}{1.613941in}}%
\pgfpathlineto{\pgfqpoint{1.180238in}{1.617297in}}%
\pgfpathlineto{\pgfqpoint{1.180231in}{1.620554in}}%
\pgfpathlineto{\pgfqpoint{1.180225in}{1.623711in}}%
\pgfpathlineto{\pgfqpoint{1.175366in}{1.623679in}}%
\pgfpathlineto{\pgfqpoint{1.170511in}{1.623575in}}%
\pgfpathlineto{\pgfqpoint{1.165665in}{1.623400in}}%
\pgfpathlineto{\pgfqpoint{1.160834in}{1.623153in}}%
\pgfpathclose%
\pgfusepath{fill}%
\end{pgfscope}%
\begin{pgfscope}%
\pgfpathrectangle{\pgfqpoint{0.041670in}{0.041670in}}{\pgfqpoint{2.216660in}{2.216660in}}%
\pgfusepath{clip}%
\pgfsetbuttcap%
\pgfsetroundjoin%
\definecolor{currentfill}{rgb}{0.636902,0.856542,0.216620}%
\pgfsetfillcolor{currentfill}%
\pgfsetlinewidth{0.000000pt}%
\definecolor{currentstroke}{rgb}{0.000000,0.000000,0.000000}%
\pgfsetstrokecolor{currentstroke}%
\pgfsetdash{}{0pt}%
\pgfpathmoveto{\pgfqpoint{1.180225in}{1.623711in}}%
\pgfpathlineto{\pgfqpoint{1.180231in}{1.620554in}}%
\pgfpathlineto{\pgfqpoint{1.180238in}{1.617297in}}%
\pgfpathlineto{\pgfqpoint{1.180245in}{1.613941in}}%
\pgfpathlineto{\pgfqpoint{1.180251in}{1.610487in}}%
\pgfpathlineto{\pgfqpoint{1.185585in}{1.610443in}}%
\pgfpathlineto{\pgfqpoint{1.190913in}{1.610320in}}%
\pgfpathlineto{\pgfqpoint{1.196231in}{1.610118in}}%
\pgfpathlineto{\pgfqpoint{1.201533in}{1.609838in}}%
\pgfpathlineto{\pgfqpoint{1.201054in}{1.613307in}}%
\pgfpathlineto{\pgfqpoint{1.200574in}{1.616678in}}%
\pgfpathlineto{\pgfqpoint{1.200093in}{1.619950in}}%
\pgfpathlineto{\pgfqpoint{1.199612in}{1.623121in}}%
\pgfpathlineto{\pgfqpoint{1.194782in}{1.623376in}}%
\pgfpathlineto{\pgfqpoint{1.189938in}{1.623559in}}%
\pgfpathlineto{\pgfqpoint{1.185084in}{1.623671in}}%
\pgfpathlineto{\pgfqpoint{1.180225in}{1.623711in}}%
\pgfpathclose%
\pgfusepath{fill}%
\end{pgfscope}%
\begin{pgfscope}%
\pgfpathrectangle{\pgfqpoint{0.041670in}{0.041670in}}{\pgfqpoint{2.216660in}{2.216660in}}%
\pgfusepath{clip}%
\pgfsetbuttcap%
\pgfsetroundjoin%
\definecolor{currentfill}{rgb}{0.179019,0.433756,0.557430}%
\pgfsetfillcolor{currentfill}%
\pgfsetlinewidth{0.000000pt}%
\definecolor{currentstroke}{rgb}{0.000000,0.000000,0.000000}%
\pgfsetstrokecolor{currentstroke}%
\pgfsetdash{}{0pt}%
\pgfpathmoveto{\pgfqpoint{0.829084in}{1.108070in}}%
\pgfpathlineto{\pgfqpoint{0.825528in}{1.099844in}}%
\pgfpathlineto{\pgfqpoint{0.821974in}{1.091645in}}%
\pgfpathlineto{\pgfqpoint{0.818422in}{1.083474in}}%
\pgfpathlineto{\pgfqpoint{0.814871in}{1.075336in}}%
\pgfpathlineto{\pgfqpoint{0.811518in}{1.081212in}}%
\pgfpathlineto{\pgfqpoint{0.808540in}{1.087133in}}%
\pgfpathlineto{\pgfqpoint{0.805941in}{1.093094in}}%
\pgfpathlineto{\pgfqpoint{0.803720in}{1.099089in}}%
\pgfpathlineto{\pgfqpoint{0.807367in}{1.106969in}}%
\pgfpathlineto{\pgfqpoint{0.811016in}{1.114882in}}%
\pgfpathlineto{\pgfqpoint{0.814667in}{1.122825in}}%
\pgfpathlineto{\pgfqpoint{0.818320in}{1.130794in}}%
\pgfpathlineto{\pgfqpoint{0.820466in}{1.125058in}}%
\pgfpathlineto{\pgfqpoint{0.822977in}{1.119355in}}%
\pgfpathlineto{\pgfqpoint{0.825850in}{1.113690in}}%
\pgfpathlineto{\pgfqpoint{0.829084in}{1.108070in}}%
\pgfpathclose%
\pgfusepath{fill}%
\end{pgfscope}%
\begin{pgfscope}%
\pgfpathrectangle{\pgfqpoint{0.041670in}{0.041670in}}{\pgfqpoint{2.216660in}{2.216660in}}%
\pgfusepath{clip}%
\pgfsetbuttcap%
\pgfsetroundjoin%
\definecolor{currentfill}{rgb}{0.282327,0.094955,0.417331}%
\pgfsetfillcolor{currentfill}%
\pgfsetlinewidth{0.000000pt}%
\definecolor{currentstroke}{rgb}{0.000000,0.000000,0.000000}%
\pgfsetstrokecolor{currentstroke}%
\pgfsetdash{}{0pt}%
\pgfpathmoveto{\pgfqpoint{0.582828in}{0.764243in}}%
\pgfpathlineto{\pgfqpoint{0.579135in}{0.768390in}}%
\pgfpathlineto{\pgfqpoint{0.575429in}{0.772883in}}%
\pgfpathlineto{\pgfqpoint{0.571710in}{0.777727in}}%
\pgfpathlineto{\pgfqpoint{0.567978in}{0.782930in}}%
\pgfpathlineto{\pgfqpoint{0.559866in}{0.793207in}}%
\pgfpathlineto{\pgfqpoint{0.552407in}{0.803599in}}%
\pgfpathlineto{\pgfqpoint{0.545606in}{0.814095in}}%
\pgfpathlineto{\pgfqpoint{0.539466in}{0.824683in}}%
\pgfpathlineto{\pgfqpoint{0.543351in}{0.819257in}}%
\pgfpathlineto{\pgfqpoint{0.547221in}{0.814187in}}%
\pgfpathlineto{\pgfqpoint{0.551078in}{0.809468in}}%
\pgfpathlineto{\pgfqpoint{0.554922in}{0.805094in}}%
\pgfpathlineto{\pgfqpoint{0.560935in}{0.794733in}}%
\pgfpathlineto{\pgfqpoint{0.567593in}{0.784464in}}%
\pgfpathlineto{\pgfqpoint{0.574892in}{0.774297in}}%
\pgfpathlineto{\pgfqpoint{0.582828in}{0.764243in}}%
\pgfpathclose%
\pgfusepath{fill}%
\end{pgfscope}%
\begin{pgfscope}%
\pgfpathrectangle{\pgfqpoint{0.041670in}{0.041670in}}{\pgfqpoint{2.216660in}{2.216660in}}%
\pgfusepath{clip}%
\pgfsetbuttcap%
\pgfsetroundjoin%
\definecolor{currentfill}{rgb}{0.268510,0.009605,0.335427}%
\pgfsetfillcolor{currentfill}%
\pgfsetlinewidth{0.000000pt}%
\definecolor{currentstroke}{rgb}{0.000000,0.000000,0.000000}%
\pgfsetstrokecolor{currentstroke}%
\pgfsetdash{}{0pt}%
\pgfpathmoveto{\pgfqpoint{0.701314in}{0.720673in}}%
\pgfpathlineto{\pgfqpoint{0.698046in}{0.717918in}}%
\pgfpathlineto{\pgfqpoint{0.694772in}{0.715384in}}%
\pgfpathlineto{\pgfqpoint{0.691493in}{0.713075in}}%
\pgfpathlineto{\pgfqpoint{0.688207in}{0.710996in}}%
\pgfpathlineto{\pgfqpoint{0.678971in}{0.719359in}}%
\pgfpathlineto{\pgfqpoint{0.670267in}{0.727864in}}%
\pgfpathlineto{\pgfqpoint{0.662100in}{0.736500in}}%
\pgfpathlineto{\pgfqpoint{0.654479in}{0.745259in}}%
\pgfpathlineto{\pgfqpoint{0.657974in}{0.747101in}}%
\pgfpathlineto{\pgfqpoint{0.661463in}{0.749172in}}%
\pgfpathlineto{\pgfqpoint{0.664945in}{0.751467in}}%
\pgfpathlineto{\pgfqpoint{0.668422in}{0.753983in}}%
\pgfpathlineto{\pgfqpoint{0.675857in}{0.745467in}}%
\pgfpathlineto{\pgfqpoint{0.683822in}{0.737070in}}%
\pgfpathlineto{\pgfqpoint{0.692310in}{0.728802in}}%
\pgfpathlineto{\pgfqpoint{0.701314in}{0.720673in}}%
\pgfpathclose%
\pgfusepath{fill}%
\end{pgfscope}%
\begin{pgfscope}%
\pgfpathrectangle{\pgfqpoint{0.041670in}{0.041670in}}{\pgfqpoint{2.216660in}{2.216660in}}%
\pgfusepath{clip}%
\pgfsetbuttcap%
\pgfsetroundjoin%
\definecolor{currentfill}{rgb}{0.487026,0.823929,0.312321}%
\pgfsetfillcolor{currentfill}%
\pgfsetlinewidth{0.000000pt}%
\definecolor{currentstroke}{rgb}{0.000000,0.000000,0.000000}%
\pgfsetstrokecolor{currentstroke}%
\pgfsetdash{}{0pt}%
\pgfpathmoveto{\pgfqpoint{1.037236in}{1.564060in}}%
\pgfpathlineto{\pgfqpoint{1.034328in}{1.559479in}}%
\pgfpathlineto{\pgfqpoint{1.031423in}{1.554810in}}%
\pgfpathlineto{\pgfqpoint{1.028521in}{1.550055in}}%
\pgfpathlineto{\pgfqpoint{1.025623in}{1.545217in}}%
\pgfpathlineto{\pgfqpoint{1.029761in}{1.547493in}}%
\pgfpathlineto{\pgfqpoint{1.034044in}{1.549706in}}%
\pgfpathlineto{\pgfqpoint{1.038468in}{1.551853in}}%
\pgfpathlineto{\pgfqpoint{1.043029in}{1.553933in}}%
\pgfpathlineto{\pgfqpoint{1.045597in}{1.558602in}}%
\pgfpathlineto{\pgfqpoint{1.048168in}{1.563187in}}%
\pgfpathlineto{\pgfqpoint{1.050742in}{1.567686in}}%
\pgfpathlineto{\pgfqpoint{1.053320in}{1.572097in}}%
\pgfpathlineto{\pgfqpoint{1.049104in}{1.570180in}}%
\pgfpathlineto{\pgfqpoint{1.045016in}{1.568200in}}%
\pgfpathlineto{\pgfqpoint{1.041058in}{1.566159in}}%
\pgfpathlineto{\pgfqpoint{1.037236in}{1.564060in}}%
\pgfpathclose%
\pgfusepath{fill}%
\end{pgfscope}%
\begin{pgfscope}%
\pgfpathrectangle{\pgfqpoint{0.041670in}{0.041670in}}{\pgfqpoint{2.216660in}{2.216660in}}%
\pgfusepath{clip}%
\pgfsetbuttcap%
\pgfsetroundjoin%
\definecolor{currentfill}{rgb}{0.280255,0.165693,0.476498}%
\pgfsetfillcolor{currentfill}%
\pgfsetlinewidth{0.000000pt}%
\definecolor{currentstroke}{rgb}{0.000000,0.000000,0.000000}%
\pgfsetstrokecolor{currentstroke}%
\pgfsetdash{}{0pt}%
\pgfpathmoveto{\pgfqpoint{1.614522in}{0.875446in}}%
\pgfpathlineto{\pgfqpoint{1.617955in}{0.868985in}}%
\pgfpathlineto{\pgfqpoint{1.621389in}{0.862645in}}%
\pgfpathlineto{\pgfqpoint{1.624826in}{0.856429in}}%
\pgfpathlineto{\pgfqpoint{1.628264in}{0.850341in}}%
\pgfpathlineto{\pgfqpoint{1.622168in}{0.842973in}}%
\pgfpathlineto{\pgfqpoint{1.615611in}{0.835701in}}%
\pgfpathlineto{\pgfqpoint{1.608599in}{0.828535in}}%
\pgfpathlineto{\pgfqpoint{1.601139in}{0.821483in}}%
\pgfpathlineto{\pgfqpoint{1.597896in}{0.827816in}}%
\pgfpathlineto{\pgfqpoint{1.594655in}{0.834279in}}%
\pgfpathlineto{\pgfqpoint{1.591416in}{0.840865in}}%
\pgfpathlineto{\pgfqpoint{1.588178in}{0.847573in}}%
\pgfpathlineto{\pgfqpoint{1.595422in}{0.854384in}}%
\pgfpathlineto{\pgfqpoint{1.602230in}{0.861305in}}%
\pgfpathlineto{\pgfqpoint{1.608599in}{0.868329in}}%
\pgfpathlineto{\pgfqpoint{1.614522in}{0.875446in}}%
\pgfpathclose%
\pgfusepath{fill}%
\end{pgfscope}%
\begin{pgfscope}%
\pgfpathrectangle{\pgfqpoint{0.041670in}{0.041670in}}{\pgfqpoint{2.216660in}{2.216660in}}%
\pgfusepath{clip}%
\pgfsetbuttcap%
\pgfsetroundjoin%
\definecolor{currentfill}{rgb}{0.220124,0.725509,0.466226}%
\pgfsetfillcolor{currentfill}%
\pgfsetlinewidth{0.000000pt}%
\definecolor{currentstroke}{rgb}{0.000000,0.000000,0.000000}%
\pgfsetstrokecolor{currentstroke}%
\pgfsetdash{}{0pt}%
\pgfpathmoveto{\pgfqpoint{0.937556in}{1.427869in}}%
\pgfpathlineto{\pgfqpoint{0.933862in}{1.421253in}}%
\pgfpathlineto{\pgfqpoint{0.930172in}{1.414580in}}%
\pgfpathlineto{\pgfqpoint{0.926486in}{1.407853in}}%
\pgfpathlineto{\pgfqpoint{0.922803in}{1.401074in}}%
\pgfpathlineto{\pgfqpoint{0.924586in}{1.404989in}}%
\pgfpathlineto{\pgfqpoint{0.926620in}{1.408873in}}%
\pgfpathlineto{\pgfqpoint{0.928902in}{1.412722in}}%
\pgfpathlineto{\pgfqpoint{0.931431in}{1.416532in}}%
\pgfpathlineto{\pgfqpoint{0.934982in}{1.423075in}}%
\pgfpathlineto{\pgfqpoint{0.938537in}{1.429566in}}%
\pgfpathlineto{\pgfqpoint{0.942096in}{1.436003in}}%
\pgfpathlineto{\pgfqpoint{0.945658in}{1.442384in}}%
\pgfpathlineto{\pgfqpoint{0.943281in}{1.438806in}}%
\pgfpathlineto{\pgfqpoint{0.941137in}{1.435192in}}%
\pgfpathlineto{\pgfqpoint{0.939228in}{1.431545in}}%
\pgfpathlineto{\pgfqpoint{0.937556in}{1.427869in}}%
\pgfpathclose%
\pgfusepath{fill}%
\end{pgfscope}%
\begin{pgfscope}%
\pgfpathrectangle{\pgfqpoint{0.041670in}{0.041670in}}{\pgfqpoint{2.216660in}{2.216660in}}%
\pgfusepath{clip}%
\pgfsetbuttcap%
\pgfsetroundjoin%
\definecolor{currentfill}{rgb}{0.282327,0.094955,0.417331}%
\pgfsetfillcolor{currentfill}%
\pgfsetlinewidth{0.000000pt}%
\definecolor{currentstroke}{rgb}{0.000000,0.000000,0.000000}%
\pgfsetstrokecolor{currentstroke}%
\pgfsetdash{}{0pt}%
\pgfpathmoveto{\pgfqpoint{0.752987in}{0.791128in}}%
\pgfpathlineto{\pgfqpoint{0.749784in}{0.785441in}}%
\pgfpathlineto{\pgfqpoint{0.746578in}{0.779906in}}%
\pgfpathlineto{\pgfqpoint{0.743369in}{0.774526in}}%
\pgfpathlineto{\pgfqpoint{0.740158in}{0.769307in}}%
\pgfpathlineto{\pgfqpoint{0.731844in}{0.776728in}}%
\pgfpathlineto{\pgfqpoint{0.724003in}{0.784277in}}%
\pgfpathlineto{\pgfqpoint{0.716640in}{0.791945in}}%
\pgfpathlineto{\pgfqpoint{0.709763in}{0.799724in}}%
\pgfpathlineto{\pgfqpoint{0.713182in}{0.804700in}}%
\pgfpathlineto{\pgfqpoint{0.716598in}{0.809837in}}%
\pgfpathlineto{\pgfqpoint{0.720012in}{0.815129in}}%
\pgfpathlineto{\pgfqpoint{0.723423in}{0.820573in}}%
\pgfpathlineto{\pgfqpoint{0.730115in}{0.813042in}}%
\pgfpathlineto{\pgfqpoint{0.737277in}{0.805619in}}%
\pgfpathlineto{\pgfqpoint{0.744903in}{0.798312in}}%
\pgfpathlineto{\pgfqpoint{0.752987in}{0.791128in}}%
\pgfpathclose%
\pgfusepath{fill}%
\end{pgfscope}%
\begin{pgfscope}%
\pgfpathrectangle{\pgfqpoint{0.041670in}{0.041670in}}{\pgfqpoint{2.216660in}{2.216660in}}%
\pgfusepath{clip}%
\pgfsetbuttcap%
\pgfsetroundjoin%
\definecolor{currentfill}{rgb}{0.636902,0.856542,0.216620}%
\pgfsetfillcolor{currentfill}%
\pgfsetlinewidth{0.000000pt}%
\definecolor{currentstroke}{rgb}{0.000000,0.000000,0.000000}%
\pgfsetstrokecolor{currentstroke}%
\pgfsetdash{}{0pt}%
\pgfpathmoveto{\pgfqpoint{1.141737in}{1.621456in}}%
\pgfpathlineto{\pgfqpoint{1.140801in}{1.618243in}}%
\pgfpathlineto{\pgfqpoint{1.139867in}{1.614930in}}%
\pgfpathlineto{\pgfqpoint{1.138933in}{1.611517in}}%
\pgfpathlineto{\pgfqpoint{1.138001in}{1.608007in}}%
\pgfpathlineto{\pgfqpoint{1.143194in}{1.608590in}}%
\pgfpathlineto{\pgfqpoint{1.148422in}{1.609095in}}%
\pgfpathlineto{\pgfqpoint{1.153681in}{1.609523in}}%
\pgfpathlineto{\pgfqpoint{1.158965in}{1.609873in}}%
\pgfpathlineto{\pgfqpoint{1.159431in}{1.613341in}}%
\pgfpathlineto{\pgfqpoint{1.159898in}{1.616711in}}%
\pgfpathlineto{\pgfqpoint{1.160366in}{1.619982in}}%
\pgfpathlineto{\pgfqpoint{1.160834in}{1.623153in}}%
\pgfpathlineto{\pgfqpoint{1.156020in}{1.622835in}}%
\pgfpathlineto{\pgfqpoint{1.151230in}{1.622446in}}%
\pgfpathlineto{\pgfqpoint{1.146467in}{1.621986in}}%
\pgfpathlineto{\pgfqpoint{1.141737in}{1.621456in}}%
\pgfpathclose%
\pgfusepath{fill}%
\end{pgfscope}%
\begin{pgfscope}%
\pgfpathrectangle{\pgfqpoint{0.041670in}{0.041670in}}{\pgfqpoint{2.216660in}{2.216660in}}%
\pgfusepath{clip}%
\pgfsetbuttcap%
\pgfsetroundjoin%
\definecolor{currentfill}{rgb}{0.636902,0.856542,0.216620}%
\pgfsetfillcolor{currentfill}%
\pgfsetlinewidth{0.000000pt}%
\definecolor{currentstroke}{rgb}{0.000000,0.000000,0.000000}%
\pgfsetstrokecolor{currentstroke}%
\pgfsetdash{}{0pt}%
\pgfpathmoveto{\pgfqpoint{1.199612in}{1.623121in}}%
\pgfpathlineto{\pgfqpoint{1.200093in}{1.619950in}}%
\pgfpathlineto{\pgfqpoint{1.200574in}{1.616678in}}%
\pgfpathlineto{\pgfqpoint{1.201054in}{1.613307in}}%
\pgfpathlineto{\pgfqpoint{1.201533in}{1.609838in}}%
\pgfpathlineto{\pgfqpoint{1.206814in}{1.609480in}}%
\pgfpathlineto{\pgfqpoint{1.212070in}{1.609043in}}%
\pgfpathlineto{\pgfqpoint{1.217295in}{1.608529in}}%
\pgfpathlineto{\pgfqpoint{1.222483in}{1.607937in}}%
\pgfpathlineto{\pgfqpoint{1.221538in}{1.611449in}}%
\pgfpathlineto{\pgfqpoint{1.220592in}{1.614863in}}%
\pgfpathlineto{\pgfqpoint{1.219645in}{1.618178in}}%
\pgfpathlineto{\pgfqpoint{1.218696in}{1.621393in}}%
\pgfpathlineto{\pgfqpoint{1.213970in}{1.621931in}}%
\pgfpathlineto{\pgfqpoint{1.209210in}{1.622398in}}%
\pgfpathlineto{\pgfqpoint{1.204423in}{1.622795in}}%
\pgfpathlineto{\pgfqpoint{1.199612in}{1.623121in}}%
\pgfpathclose%
\pgfusepath{fill}%
\end{pgfscope}%
\begin{pgfscope}%
\pgfpathrectangle{\pgfqpoint{0.041670in}{0.041670in}}{\pgfqpoint{2.216660in}{2.216660in}}%
\pgfusepath{clip}%
\pgfsetbuttcap%
\pgfsetroundjoin%
\definecolor{currentfill}{rgb}{0.231674,0.318106,0.544834}%
\pgfsetfillcolor{currentfill}%
\pgfsetlinewidth{0.000000pt}%
\definecolor{currentstroke}{rgb}{0.000000,0.000000,0.000000}%
\pgfsetstrokecolor{currentstroke}%
\pgfsetdash{}{0pt}%
\pgfpathmoveto{\pgfqpoint{0.804851in}{0.986835in}}%
\pgfpathlineto{\pgfqpoint{0.801463in}{0.978877in}}%
\pgfpathlineto{\pgfqpoint{0.798076in}{0.970983in}}%
\pgfpathlineto{\pgfqpoint{0.794688in}{0.963157in}}%
\pgfpathlineto{\pgfqpoint{0.791301in}{0.955401in}}%
\pgfpathlineto{\pgfqpoint{0.785948in}{0.961777in}}%
\pgfpathlineto{\pgfqpoint{0.781001in}{0.968231in}}%
\pgfpathlineto{\pgfqpoint{0.776466in}{0.974757in}}%
\pgfpathlineto{\pgfqpoint{0.772345in}{0.981346in}}%
\pgfpathlineto{\pgfqpoint{0.775885in}{0.988849in}}%
\pgfpathlineto{\pgfqpoint{0.779425in}{0.996423in}}%
\pgfpathlineto{\pgfqpoint{0.782965in}{1.004064in}}%
\pgfpathlineto{\pgfqpoint{0.786507in}{1.011769in}}%
\pgfpathlineto{\pgfqpoint{0.790497in}{1.005436in}}%
\pgfpathlineto{\pgfqpoint{0.794887in}{0.999165in}}%
\pgfpathlineto{\pgfqpoint{0.799673in}{0.992962in}}%
\pgfpathlineto{\pgfqpoint{0.804851in}{0.986835in}}%
\pgfpathclose%
\pgfusepath{fill}%
\end{pgfscope}%
\begin{pgfscope}%
\pgfpathrectangle{\pgfqpoint{0.041670in}{0.041670in}}{\pgfqpoint{2.216660in}{2.216660in}}%
\pgfusepath{clip}%
\pgfsetbuttcap%
\pgfsetroundjoin%
\definecolor{currentfill}{rgb}{0.282884,0.135920,0.453427}%
\pgfsetfillcolor{currentfill}%
\pgfsetlinewidth{0.000000pt}%
\definecolor{currentstroke}{rgb}{0.000000,0.000000,0.000000}%
\pgfsetstrokecolor{currentstroke}%
\pgfsetdash{}{0pt}%
\pgfpathmoveto{\pgfqpoint{1.825342in}{0.834164in}}%
\pgfpathlineto{\pgfqpoint{1.829266in}{0.840003in}}%
\pgfpathlineto{\pgfqpoint{1.833204in}{0.846211in}}%
\pgfpathlineto{\pgfqpoint{1.837157in}{0.852795in}}%
\pgfpathlineto{\pgfqpoint{1.841126in}{0.859760in}}%
\pgfpathlineto{\pgfqpoint{1.835465in}{0.848876in}}%
\pgfpathlineto{\pgfqpoint{1.829123in}{0.838076in}}%
\pgfpathlineto{\pgfqpoint{1.822102in}{0.827371in}}%
\pgfpathlineto{\pgfqpoint{1.814408in}{0.816772in}}%
\pgfpathlineto{\pgfqpoint{1.810577in}{0.810025in}}%
\pgfpathlineto{\pgfqpoint{1.806761in}{0.803662in}}%
\pgfpathlineto{\pgfqpoint{1.802960in}{0.797675in}}%
\pgfpathlineto{\pgfqpoint{1.799174in}{0.792059in}}%
\pgfpathlineto{\pgfqpoint{1.806706in}{0.802439in}}%
\pgfpathlineto{\pgfqpoint{1.813581in}{0.812924in}}%
\pgfpathlineto{\pgfqpoint{1.819794in}{0.823503in}}%
\pgfpathlineto{\pgfqpoint{1.825342in}{0.834164in}}%
\pgfpathclose%
\pgfusepath{fill}%
\end{pgfscope}%
\begin{pgfscope}%
\pgfpathrectangle{\pgfqpoint{0.041670in}{0.041670in}}{\pgfqpoint{2.216660in}{2.216660in}}%
\pgfusepath{clip}%
\pgfsetbuttcap%
\pgfsetroundjoin%
\definecolor{currentfill}{rgb}{0.166383,0.690856,0.496502}%
\pgfsetfillcolor{currentfill}%
\pgfsetlinewidth{0.000000pt}%
\definecolor{currentstroke}{rgb}{0.000000,0.000000,0.000000}%
\pgfsetstrokecolor{currentstroke}%
\pgfsetdash{}{0pt}%
\pgfpathmoveto{\pgfqpoint{1.435535in}{1.404556in}}%
\pgfpathlineto{\pgfqpoint{1.439189in}{1.397780in}}%
\pgfpathlineto{\pgfqpoint{1.442841in}{1.390956in}}%
\pgfpathlineto{\pgfqpoint{1.446488in}{1.384087in}}%
\pgfpathlineto{\pgfqpoint{1.450132in}{1.377175in}}%
\pgfpathlineto{\pgfqpoint{1.451996in}{1.373015in}}%
\pgfpathlineto{\pgfqpoint{1.453592in}{1.368827in}}%
\pgfpathlineto{\pgfqpoint{1.454917in}{1.364613in}}%
\pgfpathlineto{\pgfqpoint{1.455971in}{1.360378in}}%
\pgfpathlineto{\pgfqpoint{1.452239in}{1.367534in}}%
\pgfpathlineto{\pgfqpoint{1.448503in}{1.374647in}}%
\pgfpathlineto{\pgfqpoint{1.444765in}{1.381714in}}%
\pgfpathlineto{\pgfqpoint{1.441022in}{1.388733in}}%
\pgfpathlineto{\pgfqpoint{1.440035in}{1.392722in}}%
\pgfpathlineto{\pgfqpoint{1.438790in}{1.396691in}}%
\pgfpathlineto{\pgfqpoint{1.437289in}{1.400637in}}%
\pgfpathlineto{\pgfqpoint{1.435535in}{1.404556in}}%
\pgfpathclose%
\pgfusepath{fill}%
\end{pgfscope}%
\begin{pgfscope}%
\pgfpathrectangle{\pgfqpoint{0.041670in}{0.041670in}}{\pgfqpoint{2.216660in}{2.216660in}}%
\pgfusepath{clip}%
\pgfsetbuttcap%
\pgfsetroundjoin%
\definecolor{currentfill}{rgb}{0.267004,0.004874,0.329415}%
\pgfsetfillcolor{currentfill}%
\pgfsetlinewidth{0.000000pt}%
\definecolor{currentstroke}{rgb}{0.000000,0.000000,0.000000}%
\pgfsetstrokecolor{currentstroke}%
\pgfsetdash{}{0pt}%
\pgfpathmoveto{\pgfqpoint{0.688207in}{0.710996in}}%
\pgfpathlineto{\pgfqpoint{0.684914in}{0.709152in}}%
\pgfpathlineto{\pgfqpoint{0.681615in}{0.707547in}}%
\pgfpathlineto{\pgfqpoint{0.678309in}{0.706186in}}%
\pgfpathlineto{\pgfqpoint{0.674996in}{0.705075in}}%
\pgfpathlineto{\pgfqpoint{0.665528in}{0.713670in}}%
\pgfpathlineto{\pgfqpoint{0.656606in}{0.722409in}}%
\pgfpathlineto{\pgfqpoint{0.648238in}{0.731283in}}%
\pgfpathlineto{\pgfqpoint{0.640429in}{0.740283in}}%
\pgfpathlineto{\pgfqpoint{0.643952in}{0.741159in}}%
\pgfpathlineto{\pgfqpoint{0.647468in}{0.742284in}}%
\pgfpathlineto{\pgfqpoint{0.650977in}{0.743652in}}%
\pgfpathlineto{\pgfqpoint{0.654479in}{0.745259in}}%
\pgfpathlineto{\pgfqpoint{0.662100in}{0.736500in}}%
\pgfpathlineto{\pgfqpoint{0.670267in}{0.727864in}}%
\pgfpathlineto{\pgfqpoint{0.678971in}{0.719359in}}%
\pgfpathlineto{\pgfqpoint{0.688207in}{0.710996in}}%
\pgfpathclose%
\pgfusepath{fill}%
\end{pgfscope}%
\begin{pgfscope}%
\pgfpathrectangle{\pgfqpoint{0.041670in}{0.041670in}}{\pgfqpoint{2.216660in}{2.216660in}}%
\pgfusepath{clip}%
\pgfsetbuttcap%
\pgfsetroundjoin%
\definecolor{currentfill}{rgb}{0.212395,0.359683,0.551710}%
\pgfsetfillcolor{currentfill}%
\pgfsetlinewidth{0.000000pt}%
\definecolor{currentstroke}{rgb}{0.000000,0.000000,0.000000}%
\pgfsetstrokecolor{currentstroke}%
\pgfsetdash{}{0pt}%
\pgfpathmoveto{\pgfqpoint{1.562334in}{1.048617in}}%
\pgfpathlineto{\pgfqpoint{1.565905in}{1.040744in}}%
\pgfpathlineto{\pgfqpoint{1.569475in}{1.032922in}}%
\pgfpathlineto{\pgfqpoint{1.573044in}{1.025155in}}%
\pgfpathlineto{\pgfqpoint{1.576612in}{1.017446in}}%
\pgfpathlineto{\pgfqpoint{1.572980in}{1.011063in}}%
\pgfpathlineto{\pgfqpoint{1.568945in}{1.004736in}}%
\pgfpathlineto{\pgfqpoint{1.564511in}{0.998472in}}%
\pgfpathlineto{\pgfqpoint{1.559681in}{0.992278in}}%
\pgfpathlineto{\pgfqpoint{1.556253in}{1.000241in}}%
\pgfpathlineto{\pgfqpoint{1.552825in}{1.008263in}}%
\pgfpathlineto{\pgfqpoint{1.549395in}{1.016338in}}%
\pgfpathlineto{\pgfqpoint{1.545965in}{1.024464in}}%
\pgfpathlineto{\pgfqpoint{1.550633in}{1.030408in}}%
\pgfpathlineto{\pgfqpoint{1.554919in}{1.036419in}}%
\pgfpathlineto{\pgfqpoint{1.558820in}{1.042491in}}%
\pgfpathlineto{\pgfqpoint{1.562334in}{1.048617in}}%
\pgfpathclose%
\pgfusepath{fill}%
\end{pgfscope}%
\begin{pgfscope}%
\pgfpathrectangle{\pgfqpoint{0.041670in}{0.041670in}}{\pgfqpoint{2.216660in}{2.216660in}}%
\pgfusepath{clip}%
\pgfsetbuttcap%
\pgfsetroundjoin%
\definecolor{currentfill}{rgb}{0.636902,0.856542,0.216620}%
\pgfsetfillcolor{currentfill}%
\pgfsetlinewidth{0.000000pt}%
\definecolor{currentstroke}{rgb}{0.000000,0.000000,0.000000}%
\pgfsetstrokecolor{currentstroke}%
\pgfsetdash{}{0pt}%
\pgfpathmoveto{\pgfqpoint{1.218696in}{1.621393in}}%
\pgfpathlineto{\pgfqpoint{1.219645in}{1.618178in}}%
\pgfpathlineto{\pgfqpoint{1.220592in}{1.614863in}}%
\pgfpathlineto{\pgfqpoint{1.221538in}{1.611449in}}%
\pgfpathlineto{\pgfqpoint{1.222483in}{1.607937in}}%
\pgfpathlineto{\pgfqpoint{1.227631in}{1.607269in}}%
\pgfpathlineto{\pgfqpoint{1.232733in}{1.606525in}}%
\pgfpathlineto{\pgfqpoint{1.237785in}{1.605706in}}%
\pgfpathlineto{\pgfqpoint{1.236499in}{1.609268in}}%
\pgfpathlineto{\pgfqpoint{1.235212in}{1.612733in}}%
\pgfpathlineto{\pgfqpoint{1.233923in}{1.616099in}}%
\pgfpathlineto{\pgfqpoint{1.232633in}{1.619365in}}%
\pgfpathlineto{\pgfqpoint{1.228032in}{1.620109in}}%
\pgfpathlineto{\pgfqpoint{1.223385in}{1.620786in}}%
\pgfpathlineto{\pgfqpoint{1.218696in}{1.621393in}}%
\pgfpathclose%
\pgfusepath{fill}%
\end{pgfscope}%
\begin{pgfscope}%
\pgfpathrectangle{\pgfqpoint{0.041670in}{0.041670in}}{\pgfqpoint{2.216660in}{2.216660in}}%
\pgfusepath{clip}%
\pgfsetbuttcap%
\pgfsetroundjoin%
\definecolor{currentfill}{rgb}{0.163625,0.471133,0.558148}%
\pgfsetfillcolor{currentfill}%
\pgfsetlinewidth{0.000000pt}%
\definecolor{currentstroke}{rgb}{0.000000,0.000000,0.000000}%
\pgfsetstrokecolor{currentstroke}%
\pgfsetdash{}{0pt}%
\pgfpathmoveto{\pgfqpoint{1.528505in}{1.167765in}}%
\pgfpathlineto{\pgfqpoint{1.532180in}{1.159778in}}%
\pgfpathlineto{\pgfqpoint{1.535852in}{1.151805in}}%
\pgfpathlineto{\pgfqpoint{1.539523in}{1.143849in}}%
\pgfpathlineto{\pgfqpoint{1.543191in}{1.135914in}}%
\pgfpathlineto{\pgfqpoint{1.541369in}{1.130155in}}%
\pgfpathlineto{\pgfqpoint{1.539183in}{1.124423in}}%
\pgfpathlineto{\pgfqpoint{1.536632in}{1.118724in}}%
\pgfpathlineto{\pgfqpoint{1.533718in}{1.113064in}}%
\pgfpathlineto{\pgfqpoint{1.530134in}{1.121256in}}%
\pgfpathlineto{\pgfqpoint{1.526548in}{1.129468in}}%
\pgfpathlineto{\pgfqpoint{1.522960in}{1.137698in}}%
\pgfpathlineto{\pgfqpoint{1.519370in}{1.145942in}}%
\pgfpathlineto{\pgfqpoint{1.522177in}{1.151347in}}%
\pgfpathlineto{\pgfqpoint{1.524636in}{1.156790in}}%
\pgfpathlineto{\pgfqpoint{1.526746in}{1.162264in}}%
\pgfpathlineto{\pgfqpoint{1.528505in}{1.167765in}}%
\pgfpathclose%
\pgfusepath{fill}%
\end{pgfscope}%
\begin{pgfscope}%
\pgfpathrectangle{\pgfqpoint{0.041670in}{0.041670in}}{\pgfqpoint{2.216660in}{2.216660in}}%
\pgfusepath{clip}%
\pgfsetbuttcap%
\pgfsetroundjoin%
\definecolor{currentfill}{rgb}{0.283072,0.130895,0.449241}%
\pgfsetfillcolor{currentfill}%
\pgfsetlinewidth{0.000000pt}%
\definecolor{currentstroke}{rgb}{0.000000,0.000000,0.000000}%
\pgfsetstrokecolor{currentstroke}%
\pgfsetdash{}{0pt}%
\pgfpathmoveto{\pgfqpoint{0.765774in}{0.815315in}}%
\pgfpathlineto{\pgfqpoint{0.762580in}{0.809060in}}%
\pgfpathlineto{\pgfqpoint{0.759385in}{0.802941in}}%
\pgfpathlineto{\pgfqpoint{0.756187in}{0.796963in}}%
\pgfpathlineto{\pgfqpoint{0.752987in}{0.791128in}}%
\pgfpathlineto{\pgfqpoint{0.744903in}{0.798312in}}%
\pgfpathlineto{\pgfqpoint{0.737277in}{0.805619in}}%
\pgfpathlineto{\pgfqpoint{0.730115in}{0.813042in}}%
\pgfpathlineto{\pgfqpoint{0.723423in}{0.820573in}}%
\pgfpathlineto{\pgfqpoint{0.726831in}{0.826165in}}%
\pgfpathlineto{\pgfqpoint{0.730237in}{0.831900in}}%
\pgfpathlineto{\pgfqpoint{0.733640in}{0.837775in}}%
\pgfpathlineto{\pgfqpoint{0.737042in}{0.843787in}}%
\pgfpathlineto{\pgfqpoint{0.743548in}{0.836504in}}%
\pgfpathlineto{\pgfqpoint{0.750509in}{0.829326in}}%
\pgfpathlineto{\pgfqpoint{0.757920in}{0.822260in}}%
\pgfpathlineto{\pgfqpoint{0.765774in}{0.815315in}}%
\pgfpathclose%
\pgfusepath{fill}%
\end{pgfscope}%
\begin{pgfscope}%
\pgfpathrectangle{\pgfqpoint{0.041670in}{0.041670in}}{\pgfqpoint{2.216660in}{2.216660in}}%
\pgfusepath{clip}%
\pgfsetbuttcap%
\pgfsetroundjoin%
\definecolor{currentfill}{rgb}{0.636902,0.856542,0.216620}%
\pgfsetfillcolor{currentfill}%
\pgfsetlinewidth{0.000000pt}%
\definecolor{currentstroke}{rgb}{0.000000,0.000000,0.000000}%
\pgfsetstrokecolor{currentstroke}%
\pgfsetdash{}{0pt}%
\pgfpathmoveto{\pgfqpoint{1.123230in}{1.618646in}}%
\pgfpathlineto{\pgfqpoint{1.121840in}{1.615362in}}%
\pgfpathlineto{\pgfqpoint{1.120452in}{1.611978in}}%
\pgfpathlineto{\pgfqpoint{1.119066in}{1.608495in}}%
\pgfpathlineto{\pgfqpoint{1.117682in}{1.604915in}}%
\pgfpathlineto{\pgfqpoint{1.122684in}{1.605800in}}%
\pgfpathlineto{\pgfqpoint{1.127741in}{1.606611in}}%
\pgfpathlineto{\pgfqpoint{1.132848in}{1.607347in}}%
\pgfpathlineto{\pgfqpoint{1.138001in}{1.608007in}}%
\pgfpathlineto{\pgfqpoint{1.138933in}{1.611517in}}%
\pgfpathlineto{\pgfqpoint{1.139867in}{1.614930in}}%
\pgfpathlineto{\pgfqpoint{1.140801in}{1.618243in}}%
\pgfpathlineto{\pgfqpoint{1.141737in}{1.621456in}}%
\pgfpathlineto{\pgfqpoint{1.137044in}{1.620857in}}%
\pgfpathlineto{\pgfqpoint{1.132391in}{1.620188in}}%
\pgfpathlineto{\pgfqpoint{1.127785in}{1.619451in}}%
\pgfpathlineto{\pgfqpoint{1.123230in}{1.618646in}}%
\pgfpathclose%
\pgfusepath{fill}%
\end{pgfscope}%
\begin{pgfscope}%
\pgfpathrectangle{\pgfqpoint{0.041670in}{0.041670in}}{\pgfqpoint{2.216660in}{2.216660in}}%
\pgfusepath{clip}%
\pgfsetbuttcap%
\pgfsetroundjoin%
\definecolor{currentfill}{rgb}{0.120081,0.622161,0.534946}%
\pgfsetfillcolor{currentfill}%
\pgfsetlinewidth{0.000000pt}%
\definecolor{currentstroke}{rgb}{0.000000,0.000000,0.000000}%
\pgfsetstrokecolor{currentstroke}%
\pgfsetdash{}{0pt}%
\pgfpathmoveto{\pgfqpoint{0.887738in}{1.309279in}}%
\pgfpathlineto{\pgfqpoint{0.883994in}{1.301636in}}%
\pgfpathlineto{\pgfqpoint{0.880253in}{1.293964in}}%
\pgfpathlineto{\pgfqpoint{0.876515in}{1.286265in}}%
\pgfpathlineto{\pgfqpoint{0.872780in}{1.278543in}}%
\pgfpathlineto{\pgfqpoint{0.872476in}{1.283324in}}%
\pgfpathlineto{\pgfqpoint{0.872479in}{1.288104in}}%
\pgfpathlineto{\pgfqpoint{0.872788in}{1.292879in}}%
\pgfpathlineto{\pgfqpoint{0.873402in}{1.297643in}}%
\pgfpathlineto{\pgfqpoint{0.877120in}{1.305112in}}%
\pgfpathlineto{\pgfqpoint{0.880840in}{1.312559in}}%
\pgfpathlineto{\pgfqpoint{0.884564in}{1.319979in}}%
\pgfpathlineto{\pgfqpoint{0.888291in}{1.327370in}}%
\pgfpathlineto{\pgfqpoint{0.887716in}{1.322857in}}%
\pgfpathlineto{\pgfqpoint{0.887432in}{1.318335in}}%
\pgfpathlineto{\pgfqpoint{0.887439in}{1.313808in}}%
\pgfpathlineto{\pgfqpoint{0.887738in}{1.309279in}}%
\pgfpathclose%
\pgfusepath{fill}%
\end{pgfscope}%
\begin{pgfscope}%
\pgfpathrectangle{\pgfqpoint{0.041670in}{0.041670in}}{\pgfqpoint{2.216660in}{2.216660in}}%
\pgfusepath{clip}%
\pgfsetbuttcap%
\pgfsetroundjoin%
\definecolor{currentfill}{rgb}{0.565498,0.842430,0.262877}%
\pgfsetfillcolor{currentfill}%
\pgfsetlinewidth{0.000000pt}%
\definecolor{currentstroke}{rgb}{0.000000,0.000000,0.000000}%
\pgfsetstrokecolor{currentstroke}%
\pgfsetdash{}{0pt}%
\pgfpathmoveto{\pgfqpoint{1.275792in}{1.596546in}}%
\pgfpathlineto{\pgfqpoint{1.277923in}{1.592678in}}%
\pgfpathlineto{\pgfqpoint{1.280052in}{1.588716in}}%
\pgfpathlineto{\pgfqpoint{1.282177in}{1.584662in}}%
\pgfpathlineto{\pgfqpoint{1.284301in}{1.580517in}}%
\pgfpathlineto{\pgfqpoint{1.289073in}{1.578929in}}%
\pgfpathlineto{\pgfqpoint{1.293740in}{1.577271in}}%
\pgfpathlineto{\pgfqpoint{1.298297in}{1.575543in}}%
\pgfpathlineto{\pgfqpoint{1.302740in}{1.573748in}}%
\pgfpathlineto{\pgfqpoint{1.300239in}{1.578036in}}%
\pgfpathlineto{\pgfqpoint{1.297736in}{1.582232in}}%
\pgfpathlineto{\pgfqpoint{1.295229in}{1.586336in}}%
\pgfpathlineto{\pgfqpoint{1.292719in}{1.590345in}}%
\pgfpathlineto{\pgfqpoint{1.288641in}{1.591989in}}%
\pgfpathlineto{\pgfqpoint{1.284458in}{1.593572in}}%
\pgfpathlineto{\pgfqpoint{1.280174in}{1.595091in}}%
\pgfpathlineto{\pgfqpoint{1.275792in}{1.596546in}}%
\pgfpathclose%
\pgfusepath{fill}%
\end{pgfscope}%
\begin{pgfscope}%
\pgfpathrectangle{\pgfqpoint{0.041670in}{0.041670in}}{\pgfqpoint{2.216660in}{2.216660in}}%
\pgfusepath{clip}%
\pgfsetbuttcap%
\pgfsetroundjoin%
\definecolor{currentfill}{rgb}{0.133743,0.548535,0.553541}%
\pgfsetfillcolor{currentfill}%
\pgfsetlinewidth{0.000000pt}%
\definecolor{currentstroke}{rgb}{0.000000,0.000000,0.000000}%
\pgfsetstrokecolor{currentstroke}%
\pgfsetdash{}{0pt}%
\pgfpathmoveto{\pgfqpoint{0.862322in}{1.227410in}}%
\pgfpathlineto{\pgfqpoint{0.858642in}{1.219352in}}%
\pgfpathlineto{\pgfqpoint{0.854965in}{1.211286in}}%
\pgfpathlineto{\pgfqpoint{0.851290in}{1.203215in}}%
\pgfpathlineto{\pgfqpoint{0.847618in}{1.195141in}}%
\pgfpathlineto{\pgfqpoint{0.845959in}{1.200387in}}%
\pgfpathlineto{\pgfqpoint{0.844636in}{1.205653in}}%
\pgfpathlineto{\pgfqpoint{0.843651in}{1.210934in}}%
\pgfpathlineto{\pgfqpoint{0.843003in}{1.216222in}}%
\pgfpathlineto{\pgfqpoint{0.846716in}{1.224040in}}%
\pgfpathlineto{\pgfqpoint{0.850431in}{1.231855in}}%
\pgfpathlineto{\pgfqpoint{0.854148in}{1.239665in}}%
\pgfpathlineto{\pgfqpoint{0.857869in}{1.247467in}}%
\pgfpathlineto{\pgfqpoint{0.858499in}{1.242434in}}%
\pgfpathlineto{\pgfqpoint{0.859452in}{1.237411in}}%
\pgfpathlineto{\pgfqpoint{0.860726in}{1.232401in}}%
\pgfpathlineto{\pgfqpoint{0.862322in}{1.227410in}}%
\pgfpathclose%
\pgfusepath{fill}%
\end{pgfscope}%
\begin{pgfscope}%
\pgfpathrectangle{\pgfqpoint{0.041670in}{0.041670in}}{\pgfqpoint{2.216660in}{2.216660in}}%
\pgfusepath{clip}%
\pgfsetbuttcap%
\pgfsetroundjoin%
\definecolor{currentfill}{rgb}{0.268510,0.009605,0.335427}%
\pgfsetfillcolor{currentfill}%
\pgfsetlinewidth{0.000000pt}%
\definecolor{currentstroke}{rgb}{0.000000,0.000000,0.000000}%
\pgfsetstrokecolor{currentstroke}%
\pgfsetdash{}{0pt}%
\pgfpathmoveto{\pgfqpoint{1.740271in}{0.747673in}}%
\pgfpathlineto{\pgfqpoint{1.743873in}{0.748167in}}%
\pgfpathlineto{\pgfqpoint{1.747485in}{0.748940in}}%
\pgfpathlineto{\pgfqpoint{1.751106in}{0.749997in}}%
\pgfpathlineto{\pgfqpoint{1.754737in}{0.751343in}}%
\pgfpathlineto{\pgfqpoint{1.747083in}{0.741764in}}%
\pgfpathlineto{\pgfqpoint{1.738833in}{0.732306in}}%
\pgfpathlineto{\pgfqpoint{1.729992in}{0.722980in}}%
\pgfpathlineto{\pgfqpoint{1.720567in}{0.713795in}}%
\pgfpathlineto{\pgfqpoint{1.717135in}{0.712681in}}%
\pgfpathlineto{\pgfqpoint{1.713712in}{0.711857in}}%
\pgfpathlineto{\pgfqpoint{1.710298in}{0.711317in}}%
\pgfpathlineto{\pgfqpoint{1.706893in}{0.711057in}}%
\pgfpathlineto{\pgfqpoint{1.716097in}{0.720012in}}%
\pgfpathlineto{\pgfqpoint{1.724732in}{0.729107in}}%
\pgfpathlineto{\pgfqpoint{1.732792in}{0.738331in}}%
\pgfpathlineto{\pgfqpoint{1.740271in}{0.747673in}}%
\pgfpathclose%
\pgfusepath{fill}%
\end{pgfscope}%
\begin{pgfscope}%
\pgfpathrectangle{\pgfqpoint{0.041670in}{0.041670in}}{\pgfqpoint{2.216660in}{2.216660in}}%
\pgfusepath{clip}%
\pgfsetbuttcap%
\pgfsetroundjoin%
\definecolor{currentfill}{rgb}{0.233603,0.313828,0.543914}%
\pgfsetfillcolor{currentfill}%
\pgfsetlinewidth{0.000000pt}%
\definecolor{currentstroke}{rgb}{0.000000,0.000000,0.000000}%
\pgfsetstrokecolor{currentstroke}%
\pgfsetdash{}{0pt}%
\pgfpathmoveto{\pgfqpoint{1.889829in}{0.975850in}}%
\pgfpathlineto{\pgfqpoint{1.894028in}{0.986757in}}%
\pgfpathlineto{\pgfqpoint{1.898248in}{0.998114in}}%
\pgfpathlineto{\pgfqpoint{1.902488in}{1.009927in}}%
\pgfpathlineto{\pgfqpoint{1.906749in}{1.022205in}}%
\pgfpathlineto{\pgfqpoint{1.903704in}{1.010483in}}%
\pgfpathlineto{\pgfqpoint{1.899919in}{0.998798in}}%
\pgfpathlineto{\pgfqpoint{1.895393in}{0.987160in}}%
\pgfpathlineto{\pgfqpoint{1.890129in}{0.975583in}}%
\pgfpathlineto{\pgfqpoint{1.885937in}{0.963500in}}%
\pgfpathlineto{\pgfqpoint{1.881768in}{0.951884in}}%
\pgfpathlineto{\pgfqpoint{1.877618in}{0.940727in}}%
\pgfpathlineto{\pgfqpoint{1.873489in}{0.930022in}}%
\pgfpathlineto{\pgfqpoint{1.878657in}{0.941401in}}%
\pgfpathlineto{\pgfqpoint{1.883104in}{0.952839in}}%
\pgfpathlineto{\pgfqpoint{1.886828in}{0.964326in}}%
\pgfpathlineto{\pgfqpoint{1.889829in}{0.975850in}}%
\pgfpathclose%
\pgfusepath{fill}%
\end{pgfscope}%
\begin{pgfscope}%
\pgfpathrectangle{\pgfqpoint{0.041670in}{0.041670in}}{\pgfqpoint{2.216660in}{2.216660in}}%
\pgfusepath{clip}%
\pgfsetbuttcap%
\pgfsetroundjoin%
\definecolor{currentfill}{rgb}{0.274128,0.199721,0.498911}%
\pgfsetfillcolor{currentfill}%
\pgfsetlinewidth{0.000000pt}%
\definecolor{currentstroke}{rgb}{0.000000,0.000000,0.000000}%
\pgfsetstrokecolor{currentstroke}%
\pgfsetdash{}{0pt}%
\pgfpathmoveto{\pgfqpoint{1.600800in}{0.902424in}}%
\pgfpathlineto{\pgfqpoint{1.604229in}{0.895517in}}%
\pgfpathlineto{\pgfqpoint{1.607659in}{0.888716in}}%
\pgfpathlineto{\pgfqpoint{1.611090in}{0.882025in}}%
\pgfpathlineto{\pgfqpoint{1.614522in}{0.875446in}}%
\pgfpathlineto{\pgfqpoint{1.608599in}{0.868329in}}%
\pgfpathlineto{\pgfqpoint{1.602230in}{0.861305in}}%
\pgfpathlineto{\pgfqpoint{1.595422in}{0.854384in}}%
\pgfpathlineto{\pgfqpoint{1.588178in}{0.847573in}}%
\pgfpathlineto{\pgfqpoint{1.584942in}{0.854397in}}%
\pgfpathlineto{\pgfqpoint{1.581707in}{0.861335in}}%
\pgfpathlineto{\pgfqpoint{1.578474in}{0.868382in}}%
\pgfpathlineto{\pgfqpoint{1.575241in}{0.875534in}}%
\pgfpathlineto{\pgfqpoint{1.582267in}{0.882105in}}%
\pgfpathlineto{\pgfqpoint{1.588872in}{0.888782in}}%
\pgfpathlineto{\pgfqpoint{1.595052in}{0.895557in}}%
\pgfpathlineto{\pgfqpoint{1.600800in}{0.902424in}}%
\pgfpathclose%
\pgfusepath{fill}%
\end{pgfscope}%
\begin{pgfscope}%
\pgfpathrectangle{\pgfqpoint{0.041670in}{0.041670in}}{\pgfqpoint{2.216660in}{2.216660in}}%
\pgfusepath{clip}%
\pgfsetbuttcap%
\pgfsetroundjoin%
\definecolor{currentfill}{rgb}{0.565498,0.842430,0.262877}%
\pgfsetfillcolor{currentfill}%
\pgfsetlinewidth{0.000000pt}%
\definecolor{currentstroke}{rgb}{0.000000,0.000000,0.000000}%
\pgfsetstrokecolor{currentstroke}%
\pgfsetdash{}{0pt}%
\pgfpathmoveto{\pgfqpoint{1.063657in}{1.588833in}}%
\pgfpathlineto{\pgfqpoint{1.061069in}{1.584789in}}%
\pgfpathlineto{\pgfqpoint{1.058483in}{1.580651in}}%
\pgfpathlineto{\pgfqpoint{1.055900in}{1.576420in}}%
\pgfpathlineto{\pgfqpoint{1.053320in}{1.572097in}}%
\pgfpathlineto{\pgfqpoint{1.057657in}{1.573951in}}%
\pgfpathlineto{\pgfqpoint{1.062113in}{1.575738in}}%
\pgfpathlineto{\pgfqpoint{1.066683in}{1.577458in}}%
\pgfpathlineto{\pgfqpoint{1.071362in}{1.579109in}}%
\pgfpathlineto{\pgfqpoint{1.073572in}{1.583284in}}%
\pgfpathlineto{\pgfqpoint{1.075785in}{1.587368in}}%
\pgfpathlineto{\pgfqpoint{1.078000in}{1.591359in}}%
\pgfpathlineto{\pgfqpoint{1.080218in}{1.595256in}}%
\pgfpathlineto{\pgfqpoint{1.075923in}{1.593744in}}%
\pgfpathlineto{\pgfqpoint{1.071728in}{1.592168in}}%
\pgfpathlineto{\pgfqpoint{1.067638in}{1.590531in}}%
\pgfpathlineto{\pgfqpoint{1.063657in}{1.588833in}}%
\pgfpathclose%
\pgfusepath{fill}%
\end{pgfscope}%
\begin{pgfscope}%
\pgfpathrectangle{\pgfqpoint{0.041670in}{0.041670in}}{\pgfqpoint{2.216660in}{2.216660in}}%
\pgfusepath{clip}%
\pgfsetbuttcap%
\pgfsetroundjoin%
\definecolor{currentfill}{rgb}{0.172719,0.448791,0.557885}%
\pgfsetfillcolor{currentfill}%
\pgfsetlinewidth{0.000000pt}%
\definecolor{currentstroke}{rgb}{0.000000,0.000000,0.000000}%
\pgfsetstrokecolor{currentstroke}%
\pgfsetdash{}{0pt}%
\pgfpathmoveto{\pgfqpoint{0.438600in}{1.065528in}}%
\pgfpathlineto{\pgfqpoint{0.434233in}{1.080207in}}%
\pgfpathlineto{\pgfqpoint{0.429842in}{1.095399in}}%
\pgfpathlineto{\pgfqpoint{0.425426in}{1.111113in}}%
\pgfpathlineto{\pgfqpoint{0.422391in}{1.123158in}}%
\pgfpathlineto{\pgfqpoint{0.420128in}{1.135224in}}%
\pgfpathlineto{\pgfqpoint{0.418634in}{1.147300in}}%
\pgfpathlineto{\pgfqpoint{0.417907in}{1.159372in}}%
\pgfpathlineto{\pgfqpoint{0.422336in}{1.143485in}}%
\pgfpathlineto{\pgfqpoint{0.426741in}{1.128117in}}%
\pgfpathlineto{\pgfqpoint{0.431122in}{1.113259in}}%
\pgfpathlineto{\pgfqpoint{0.431859in}{1.101318in}}%
\pgfpathlineto{\pgfqpoint{0.433349in}{1.089375in}}%
\pgfpathlineto{\pgfqpoint{0.435596in}{1.077441in}}%
\pgfpathlineto{\pgfqpoint{0.438600in}{1.065528in}}%
\pgfpathclose%
\pgfusepath{fill}%
\end{pgfscope}%
\begin{pgfscope}%
\pgfpathrectangle{\pgfqpoint{0.041670in}{0.041670in}}{\pgfqpoint{2.216660in}{2.216660in}}%
\pgfusepath{clip}%
\pgfsetbuttcap%
\pgfsetroundjoin%
\definecolor{currentfill}{rgb}{0.412913,0.803041,0.357269}%
\pgfsetfillcolor{currentfill}%
\pgfsetlinewidth{0.000000pt}%
\definecolor{currentstroke}{rgb}{0.000000,0.000000,0.000000}%
\pgfsetstrokecolor{currentstroke}%
\pgfsetdash{}{0pt}%
\pgfpathmoveto{\pgfqpoint{1.346185in}{1.537750in}}%
\pgfpathlineto{\pgfqpoint{1.349307in}{1.532683in}}%
\pgfpathlineto{\pgfqpoint{1.352425in}{1.527535in}}%
\pgfpathlineto{\pgfqpoint{1.355540in}{1.522309in}}%
\pgfpathlineto{\pgfqpoint{1.358651in}{1.517005in}}%
\pgfpathlineto{\pgfqpoint{1.362417in}{1.514293in}}%
\pgfpathlineto{\pgfqpoint{1.366005in}{1.511524in}}%
\pgfpathlineto{\pgfqpoint{1.369411in}{1.508701in}}%
\pgfpathlineto{\pgfqpoint{1.372634in}{1.505827in}}%
\pgfpathlineto{\pgfqpoint{1.369273in}{1.511334in}}%
\pgfpathlineto{\pgfqpoint{1.365910in}{1.516764in}}%
\pgfpathlineto{\pgfqpoint{1.362542in}{1.522115in}}%
\pgfpathlineto{\pgfqpoint{1.359171in}{1.527385in}}%
\pgfpathlineto{\pgfqpoint{1.356179in}{1.530050in}}%
\pgfpathlineto{\pgfqpoint{1.353015in}{1.532667in}}%
\pgfpathlineto{\pgfqpoint{1.349683in}{1.535235in}}%
\pgfpathlineto{\pgfqpoint{1.346185in}{1.537750in}}%
\pgfpathclose%
\pgfusepath{fill}%
\end{pgfscope}%
\begin{pgfscope}%
\pgfpathrectangle{\pgfqpoint{0.041670in}{0.041670in}}{\pgfqpoint{2.216660in}{2.216660in}}%
\pgfusepath{clip}%
\pgfsetbuttcap%
\pgfsetroundjoin%
\definecolor{currentfill}{rgb}{0.344074,0.780029,0.397381}%
\pgfsetfillcolor{currentfill}%
\pgfsetlinewidth{0.000000pt}%
\definecolor{currentstroke}{rgb}{0.000000,0.000000,0.000000}%
\pgfsetstrokecolor{currentstroke}%
\pgfsetdash{}{0pt}%
\pgfpathmoveto{\pgfqpoint{1.372634in}{1.505827in}}%
\pgfpathlineto{\pgfqpoint{1.375991in}{1.500245in}}%
\pgfpathlineto{\pgfqpoint{1.379344in}{1.494590in}}%
\pgfpathlineto{\pgfqpoint{1.382693in}{1.488864in}}%
\pgfpathlineto{\pgfqpoint{1.386039in}{1.483069in}}%
\pgfpathlineto{\pgfqpoint{1.389291in}{1.479932in}}%
\pgfpathlineto{\pgfqpoint{1.392339in}{1.476747in}}%
\pgfpathlineto{\pgfqpoint{1.395179in}{1.473515in}}%
\pgfpathlineto{\pgfqpoint{1.397809in}{1.470241in}}%
\pgfpathlineto{\pgfqpoint{1.394265in}{1.476256in}}%
\pgfpathlineto{\pgfqpoint{1.390718in}{1.482201in}}%
\pgfpathlineto{\pgfqpoint{1.387167in}{1.488075in}}%
\pgfpathlineto{\pgfqpoint{1.383612in}{1.493876in}}%
\pgfpathlineto{\pgfqpoint{1.381160in}{1.496927in}}%
\pgfpathlineto{\pgfqpoint{1.378511in}{1.499937in}}%
\pgfpathlineto{\pgfqpoint{1.375668in}{1.502905in}}%
\pgfpathlineto{\pgfqpoint{1.372634in}{1.505827in}}%
\pgfpathclose%
\pgfusepath{fill}%
\end{pgfscope}%
\begin{pgfscope}%
\pgfpathrectangle{\pgfqpoint{0.041670in}{0.041670in}}{\pgfqpoint{2.216660in}{2.216660in}}%
\pgfusepath{clip}%
\pgfsetbuttcap%
\pgfsetroundjoin%
\definecolor{currentfill}{rgb}{0.166383,0.690856,0.496502}%
\pgfsetfillcolor{currentfill}%
\pgfsetlinewidth{0.000000pt}%
\definecolor{currentstroke}{rgb}{0.000000,0.000000,0.000000}%
\pgfsetstrokecolor{currentstroke}%
\pgfsetdash{}{0pt}%
\pgfpathmoveto{\pgfqpoint{0.918227in}{1.385175in}}%
\pgfpathlineto{\pgfqpoint{0.914473in}{1.378101in}}%
\pgfpathlineto{\pgfqpoint{0.910723in}{1.370979in}}%
\pgfpathlineto{\pgfqpoint{0.906976in}{1.363811in}}%
\pgfpathlineto{\pgfqpoint{0.903232in}{1.356600in}}%
\pgfpathlineto{\pgfqpoint{0.904042in}{1.360850in}}%
\pgfpathlineto{\pgfqpoint{0.905126in}{1.365082in}}%
\pgfpathlineto{\pgfqpoint{0.906482in}{1.369293in}}%
\pgfpathlineto{\pgfqpoint{0.908108in}{1.373479in}}%
\pgfpathlineto{\pgfqpoint{0.911776in}{1.380445in}}%
\pgfpathlineto{\pgfqpoint{0.915448in}{1.387367in}}%
\pgfpathlineto{\pgfqpoint{0.919124in}{1.394244in}}%
\pgfpathlineto{\pgfqpoint{0.922803in}{1.401074in}}%
\pgfpathlineto{\pgfqpoint{0.921274in}{1.397131in}}%
\pgfpathlineto{\pgfqpoint{0.920001in}{1.393164in}}%
\pgfpathlineto{\pgfqpoint{0.918984in}{1.389177in}}%
\pgfpathlineto{\pgfqpoint{0.918227in}{1.385175in}}%
\pgfpathclose%
\pgfusepath{fill}%
\end{pgfscope}%
\begin{pgfscope}%
\pgfpathrectangle{\pgfqpoint{0.041670in}{0.041670in}}{\pgfqpoint{2.216660in}{2.216660in}}%
\pgfusepath{clip}%
\pgfsetbuttcap%
\pgfsetroundjoin%
\definecolor{currentfill}{rgb}{0.636902,0.856542,0.216620}%
\pgfsetfillcolor{currentfill}%
\pgfsetlinewidth{0.000000pt}%
\definecolor{currentstroke}{rgb}{0.000000,0.000000,0.000000}%
\pgfsetstrokecolor{currentstroke}%
\pgfsetdash{}{0pt}%
\pgfpathmoveto{\pgfqpoint{1.232633in}{1.619365in}}%
\pgfpathlineto{\pgfqpoint{1.233923in}{1.616099in}}%
\pgfpathlineto{\pgfqpoint{1.235212in}{1.612733in}}%
\pgfpathlineto{\pgfqpoint{1.236499in}{1.609268in}}%
\pgfpathlineto{\pgfqpoint{1.237785in}{1.605706in}}%
\pgfpathlineto{\pgfqpoint{1.242780in}{1.604812in}}%
\pgfpathlineto{\pgfqpoint{1.247715in}{1.603844in}}%
\pgfpathlineto{\pgfqpoint{1.252585in}{1.602804in}}%
\pgfpathlineto{\pgfqpoint{1.257385in}{1.601691in}}%
\pgfpathlineto{\pgfqpoint{1.255663in}{1.605345in}}%
\pgfpathlineto{\pgfqpoint{1.253939in}{1.608902in}}%
\pgfpathlineto{\pgfqpoint{1.252212in}{1.612359in}}%
\pgfpathlineto{\pgfqpoint{1.250484in}{1.615716in}}%
\pgfpathlineto{\pgfqpoint{1.246113in}{1.616727in}}%
\pgfpathlineto{\pgfqpoint{1.241678in}{1.617673in}}%
\pgfpathlineto{\pgfqpoint{1.237183in}{1.618552in}}%
\pgfpathlineto{\pgfqpoint{1.232633in}{1.619365in}}%
\pgfpathclose%
\pgfusepath{fill}%
\end{pgfscope}%
\begin{pgfscope}%
\pgfpathrectangle{\pgfqpoint{0.041670in}{0.041670in}}{\pgfqpoint{2.216660in}{2.216660in}}%
\pgfusepath{clip}%
\pgfsetbuttcap%
\pgfsetroundjoin%
\definecolor{currentfill}{rgb}{0.267004,0.004874,0.329415}%
\pgfsetfillcolor{currentfill}%
\pgfsetlinewidth{0.000000pt}%
\definecolor{currentstroke}{rgb}{0.000000,0.000000,0.000000}%
\pgfsetstrokecolor{currentstroke}%
\pgfsetdash{}{0pt}%
\pgfpathmoveto{\pgfqpoint{0.674996in}{0.705075in}}%
\pgfpathlineto{\pgfqpoint{0.671676in}{0.704218in}}%
\pgfpathlineto{\pgfqpoint{0.668348in}{0.703620in}}%
\pgfpathlineto{\pgfqpoint{0.665012in}{0.703286in}}%
\pgfpathlineto{\pgfqpoint{0.661668in}{0.703221in}}%
\pgfpathlineto{\pgfqpoint{0.651966in}{0.712045in}}%
\pgfpathlineto{\pgfqpoint{0.642825in}{0.721016in}}%
\pgfpathlineto{\pgfqpoint{0.634254in}{0.730126in}}%
\pgfpathlineto{\pgfqpoint{0.626258in}{0.739363in}}%
\pgfpathlineto{\pgfqpoint{0.629813in}{0.739195in}}%
\pgfpathlineto{\pgfqpoint{0.633360in}{0.739296in}}%
\pgfpathlineto{\pgfqpoint{0.636899in}{0.739660in}}%
\pgfpathlineto{\pgfqpoint{0.640429in}{0.740283in}}%
\pgfpathlineto{\pgfqpoint{0.648238in}{0.731283in}}%
\pgfpathlineto{\pgfqpoint{0.656606in}{0.722409in}}%
\pgfpathlineto{\pgfqpoint{0.665528in}{0.713670in}}%
\pgfpathlineto{\pgfqpoint{0.674996in}{0.705075in}}%
\pgfpathclose%
\pgfusepath{fill}%
\end{pgfscope}%
\begin{pgfscope}%
\pgfpathrectangle{\pgfqpoint{0.041670in}{0.041670in}}{\pgfqpoint{2.216660in}{2.216660in}}%
\pgfusepath{clip}%
\pgfsetbuttcap%
\pgfsetroundjoin%
\definecolor{currentfill}{rgb}{0.280255,0.165693,0.476498}%
\pgfsetfillcolor{currentfill}%
\pgfsetlinewidth{0.000000pt}%
\definecolor{currentstroke}{rgb}{0.000000,0.000000,0.000000}%
\pgfsetstrokecolor{currentstroke}%
\pgfsetdash{}{0pt}%
\pgfpathmoveto{\pgfqpoint{0.778529in}{0.841616in}}%
\pgfpathlineto{\pgfqpoint{0.775343in}{0.834856in}}%
\pgfpathlineto{\pgfqpoint{0.772155in}{0.828216in}}%
\pgfpathlineto{\pgfqpoint{0.768965in}{0.821701in}}%
\pgfpathlineto{\pgfqpoint{0.765774in}{0.815315in}}%
\pgfpathlineto{\pgfqpoint{0.757920in}{0.822260in}}%
\pgfpathlineto{\pgfqpoint{0.750509in}{0.829326in}}%
\pgfpathlineto{\pgfqpoint{0.743548in}{0.836504in}}%
\pgfpathlineto{\pgfqpoint{0.737042in}{0.843787in}}%
\pgfpathlineto{\pgfqpoint{0.740442in}{0.849930in}}%
\pgfpathlineto{\pgfqpoint{0.743840in}{0.856202in}}%
\pgfpathlineto{\pgfqpoint{0.747236in}{0.862598in}}%
\pgfpathlineto{\pgfqpoint{0.750631in}{0.869115in}}%
\pgfpathlineto{\pgfqpoint{0.756950in}{0.862081in}}%
\pgfpathlineto{\pgfqpoint{0.763710in}{0.855148in}}%
\pgfpathlineto{\pgfqpoint{0.770905in}{0.848324in}}%
\pgfpathlineto{\pgfqpoint{0.778529in}{0.841616in}}%
\pgfpathclose%
\pgfusepath{fill}%
\end{pgfscope}%
\begin{pgfscope}%
\pgfpathrectangle{\pgfqpoint{0.041670in}{0.041670in}}{\pgfqpoint{2.216660in}{2.216660in}}%
\pgfusepath{clip}%
\pgfsetbuttcap%
\pgfsetroundjoin%
\definecolor{currentfill}{rgb}{0.636902,0.856542,0.216620}%
\pgfsetfillcolor{currentfill}%
\pgfsetlinewidth{0.000000pt}%
\definecolor{currentstroke}{rgb}{0.000000,0.000000,0.000000}%
\pgfsetstrokecolor{currentstroke}%
\pgfsetdash{}{0pt}%
\pgfpathmoveto{\pgfqpoint{1.105598in}{1.614763in}}%
\pgfpathlineto{\pgfqpoint{1.103775in}{1.611383in}}%
\pgfpathlineto{\pgfqpoint{1.101955in}{1.607901in}}%
\pgfpathlineto{\pgfqpoint{1.100137in}{1.604321in}}%
\pgfpathlineto{\pgfqpoint{1.098321in}{1.600643in}}%
\pgfpathlineto{\pgfqpoint{1.103054in}{1.601819in}}%
\pgfpathlineto{\pgfqpoint{1.107862in}{1.602923in}}%
\pgfpathlineto{\pgfqpoint{1.112739in}{1.603955in}}%
\pgfpathlineto{\pgfqpoint{1.117682in}{1.604915in}}%
\pgfpathlineto{\pgfqpoint{1.119066in}{1.608495in}}%
\pgfpathlineto{\pgfqpoint{1.120452in}{1.611978in}}%
\pgfpathlineto{\pgfqpoint{1.121840in}{1.615362in}}%
\pgfpathlineto{\pgfqpoint{1.123230in}{1.618646in}}%
\pgfpathlineto{\pgfqpoint{1.118729in}{1.617774in}}%
\pgfpathlineto{\pgfqpoint{1.114287in}{1.616835in}}%
\pgfpathlineto{\pgfqpoint{1.109908in}{1.615831in}}%
\pgfpathlineto{\pgfqpoint{1.105598in}{1.614763in}}%
\pgfpathclose%
\pgfusepath{fill}%
\end{pgfscope}%
\begin{pgfscope}%
\pgfpathrectangle{\pgfqpoint{0.041670in}{0.041670in}}{\pgfqpoint{2.216660in}{2.216660in}}%
\pgfusepath{clip}%
\pgfsetbuttcap%
\pgfsetroundjoin%
\definecolor{currentfill}{rgb}{0.487026,0.823929,0.312321}%
\pgfsetfillcolor{currentfill}%
\pgfsetlinewidth{0.000000pt}%
\definecolor{currentstroke}{rgb}{0.000000,0.000000,0.000000}%
\pgfsetstrokecolor{currentstroke}%
\pgfsetdash{}{0pt}%
\pgfpathmoveto{\pgfqpoint{1.319283in}{1.565929in}}%
\pgfpathlineto{\pgfqpoint{1.322121in}{1.561387in}}%
\pgfpathlineto{\pgfqpoint{1.324956in}{1.556758in}}%
\pgfpathlineto{\pgfqpoint{1.327787in}{1.552043in}}%
\pgfpathlineto{\pgfqpoint{1.330616in}{1.547243in}}%
\pgfpathlineto{\pgfqpoint{1.334737in}{1.544960in}}%
\pgfpathlineto{\pgfqpoint{1.338709in}{1.542615in}}%
\pgfpathlineto{\pgfqpoint{1.342526in}{1.540211in}}%
\pgfpathlineto{\pgfqpoint{1.346185in}{1.537750in}}%
\pgfpathlineto{\pgfqpoint{1.343060in}{1.542735in}}%
\pgfpathlineto{\pgfqpoint{1.339932in}{1.547635in}}%
\pgfpathlineto{\pgfqpoint{1.336800in}{1.552449in}}%
\pgfpathlineto{\pgfqpoint{1.333664in}{1.557175in}}%
\pgfpathlineto{\pgfqpoint{1.330285in}{1.559444in}}%
\pgfpathlineto{\pgfqpoint{1.326759in}{1.561661in}}%
\pgfpathlineto{\pgfqpoint{1.323090in}{1.563823in}}%
\pgfpathlineto{\pgfqpoint{1.319283in}{1.565929in}}%
\pgfpathclose%
\pgfusepath{fill}%
\end{pgfscope}%
\begin{pgfscope}%
\pgfpathrectangle{\pgfqpoint{0.041670in}{0.041670in}}{\pgfqpoint{2.216660in}{2.216660in}}%
\pgfusepath{clip}%
\pgfsetbuttcap%
\pgfsetroundjoin%
\definecolor{currentfill}{rgb}{0.163625,0.471133,0.558148}%
\pgfsetfillcolor{currentfill}%
\pgfsetlinewidth{0.000000pt}%
\definecolor{currentstroke}{rgb}{0.000000,0.000000,0.000000}%
\pgfsetstrokecolor{currentstroke}%
\pgfsetdash{}{0pt}%
\pgfpathmoveto{\pgfqpoint{0.843324in}{1.141173in}}%
\pgfpathlineto{\pgfqpoint{0.839761in}{1.132873in}}%
\pgfpathlineto{\pgfqpoint{0.836200in}{1.124587in}}%
\pgfpathlineto{\pgfqpoint{0.832641in}{1.116318in}}%
\pgfpathlineto{\pgfqpoint{0.829084in}{1.108070in}}%
\pgfpathlineto{\pgfqpoint{0.825850in}{1.113690in}}%
\pgfpathlineto{\pgfqpoint{0.822977in}{1.119355in}}%
\pgfpathlineto{\pgfqpoint{0.820466in}{1.125058in}}%
\pgfpathlineto{\pgfqpoint{0.818320in}{1.130794in}}%
\pgfpathlineto{\pgfqpoint{0.821975in}{1.138786in}}%
\pgfpathlineto{\pgfqpoint{0.825631in}{1.146799in}}%
\pgfpathlineto{\pgfqpoint{0.829290in}{1.154830in}}%
\pgfpathlineto{\pgfqpoint{0.832951in}{1.162874in}}%
\pgfpathlineto{\pgfqpoint{0.835022in}{1.157397in}}%
\pgfpathlineto{\pgfqpoint{0.837442in}{1.151950in}}%
\pgfpathlineto{\pgfqpoint{0.840210in}{1.146540in}}%
\pgfpathlineto{\pgfqpoint{0.843324in}{1.141173in}}%
\pgfpathclose%
\pgfusepath{fill}%
\end{pgfscope}%
\begin{pgfscope}%
\pgfpathrectangle{\pgfqpoint{0.041670in}{0.041670in}}{\pgfqpoint{2.216660in}{2.216660in}}%
\pgfusepath{clip}%
\pgfsetbuttcap%
\pgfsetroundjoin%
\definecolor{currentfill}{rgb}{0.412913,0.803041,0.357269}%
\pgfsetfillcolor{currentfill}%
\pgfsetlinewidth{0.000000pt}%
\definecolor{currentstroke}{rgb}{0.000000,0.000000,0.000000}%
\pgfsetstrokecolor{currentstroke}%
\pgfsetdash{}{0pt}%
\pgfpathmoveto{\pgfqpoint{0.998226in}{1.524979in}}%
\pgfpathlineto{\pgfqpoint{0.994807in}{1.519662in}}%
\pgfpathlineto{\pgfqpoint{0.991391in}{1.514264in}}%
\pgfpathlineto{\pgfqpoint{0.987978in}{1.508786in}}%
\pgfpathlineto{\pgfqpoint{0.984569in}{1.503232in}}%
\pgfpathlineto{\pgfqpoint{0.987625in}{1.506149in}}%
\pgfpathlineto{\pgfqpoint{0.990868in}{1.509017in}}%
\pgfpathlineto{\pgfqpoint{0.994295in}{1.511834in}}%
\pgfpathlineto{\pgfqpoint{0.997903in}{1.514597in}}%
\pgfpathlineto{\pgfqpoint{1.001074in}{1.519944in}}%
\pgfpathlineto{\pgfqpoint{1.004248in}{1.525214in}}%
\pgfpathlineto{\pgfqpoint{1.007426in}{1.530406in}}%
\pgfpathlineto{\pgfqpoint{1.010607in}{1.535517in}}%
\pgfpathlineto{\pgfqpoint{1.007256in}{1.532955in}}%
\pgfpathlineto{\pgfqpoint{1.004074in}{1.530343in}}%
\pgfpathlineto{\pgfqpoint{1.001062in}{1.527684in}}%
\pgfpathlineto{\pgfqpoint{0.998226in}{1.524979in}}%
\pgfpathclose%
\pgfusepath{fill}%
\end{pgfscope}%
\begin{pgfscope}%
\pgfpathrectangle{\pgfqpoint{0.041670in}{0.041670in}}{\pgfqpoint{2.216660in}{2.216660in}}%
\pgfusepath{clip}%
\pgfsetbuttcap%
\pgfsetroundjoin%
\definecolor{currentfill}{rgb}{0.212395,0.359683,0.551710}%
\pgfsetfillcolor{currentfill}%
\pgfsetlinewidth{0.000000pt}%
\definecolor{currentstroke}{rgb}{0.000000,0.000000,0.000000}%
\pgfsetstrokecolor{currentstroke}%
\pgfsetdash{}{0pt}%
\pgfpathmoveto{\pgfqpoint{0.818410in}{1.019243in}}%
\pgfpathlineto{\pgfqpoint{0.815019in}{1.011061in}}%
\pgfpathlineto{\pgfqpoint{0.811629in}{1.002931in}}%
\pgfpathlineto{\pgfqpoint{0.808240in}{0.994854in}}%
\pgfpathlineto{\pgfqpoint{0.804851in}{0.986835in}}%
\pgfpathlineto{\pgfqpoint{0.799673in}{0.992962in}}%
\pgfpathlineto{\pgfqpoint{0.794887in}{0.999165in}}%
\pgfpathlineto{\pgfqpoint{0.790497in}{1.005436in}}%
\pgfpathlineto{\pgfqpoint{0.786507in}{1.011769in}}%
\pgfpathlineto{\pgfqpoint{0.790049in}{1.019536in}}%
\pgfpathlineto{\pgfqpoint{0.793591in}{1.027360in}}%
\pgfpathlineto{\pgfqpoint{0.797135in}{1.035239in}}%
\pgfpathlineto{\pgfqpoint{0.800680in}{1.043169in}}%
\pgfpathlineto{\pgfqpoint{0.804538in}{1.037091in}}%
\pgfpathlineto{\pgfqpoint{0.808782in}{1.031073in}}%
\pgfpathlineto{\pgfqpoint{0.813407in}{1.025121in}}%
\pgfpathlineto{\pgfqpoint{0.818410in}{1.019243in}}%
\pgfpathclose%
\pgfusepath{fill}%
\end{pgfscope}%
\begin{pgfscope}%
\pgfpathrectangle{\pgfqpoint{0.041670in}{0.041670in}}{\pgfqpoint{2.216660in}{2.216660in}}%
\pgfusepath{clip}%
\pgfsetbuttcap%
\pgfsetroundjoin%
\definecolor{currentfill}{rgb}{0.281477,0.755203,0.432552}%
\pgfsetfillcolor{currentfill}%
\pgfsetlinewidth{0.000000pt}%
\definecolor{currentstroke}{rgb}{0.000000,0.000000,0.000000}%
\pgfsetstrokecolor{currentstroke}%
\pgfsetdash{}{0pt}%
\pgfpathmoveto{\pgfqpoint{1.397809in}{1.470241in}}%
\pgfpathlineto{\pgfqpoint{1.401349in}{1.464159in}}%
\pgfpathlineto{\pgfqpoint{1.404885in}{1.458012in}}%
\pgfpathlineto{\pgfqpoint{1.408418in}{1.451801in}}%
\pgfpathlineto{\pgfqpoint{1.411947in}{1.445531in}}%
\pgfpathlineto{\pgfqpoint{1.414528in}{1.441988in}}%
\pgfpathlineto{\pgfqpoint{1.416878in}{1.438406in}}%
\pgfpathlineto{\pgfqpoint{1.418996in}{1.434788in}}%
\pgfpathlineto{\pgfqpoint{1.420879in}{1.431138in}}%
\pgfpathlineto{\pgfqpoint{1.417207in}{1.437641in}}%
\pgfpathlineto{\pgfqpoint{1.413530in}{1.444084in}}%
\pgfpathlineto{\pgfqpoint{1.409850in}{1.450463in}}%
\pgfpathlineto{\pgfqpoint{1.406167in}{1.456777in}}%
\pgfpathlineto{\pgfqpoint{1.404406in}{1.460191in}}%
\pgfpathlineto{\pgfqpoint{1.402425in}{1.463576in}}%
\pgfpathlineto{\pgfqpoint{1.400225in}{1.466926in}}%
\pgfpathlineto{\pgfqpoint{1.397809in}{1.470241in}}%
\pgfpathclose%
\pgfusepath{fill}%
\end{pgfscope}%
\begin{pgfscope}%
\pgfpathrectangle{\pgfqpoint{0.041670in}{0.041670in}}{\pgfqpoint{2.216660in}{2.216660in}}%
\pgfusepath{clip}%
\pgfsetbuttcap%
\pgfsetroundjoin%
\definecolor{currentfill}{rgb}{0.344074,0.780029,0.397381}%
\pgfsetfillcolor{currentfill}%
\pgfsetlinewidth{0.000000pt}%
\definecolor{currentstroke}{rgb}{0.000000,0.000000,0.000000}%
\pgfsetstrokecolor{currentstroke}%
\pgfsetdash{}{0pt}%
\pgfpathmoveto{\pgfqpoint{0.974286in}{1.491134in}}%
\pgfpathlineto{\pgfqpoint{0.970695in}{1.485283in}}%
\pgfpathlineto{\pgfqpoint{0.967107in}{1.479358in}}%
\pgfpathlineto{\pgfqpoint{0.963523in}{1.473362in}}%
\pgfpathlineto{\pgfqpoint{0.959943in}{1.467297in}}%
\pgfpathlineto{\pgfqpoint{0.962383in}{1.470607in}}%
\pgfpathlineto{\pgfqpoint{0.965036in}{1.473876in}}%
\pgfpathlineto{\pgfqpoint{0.967899in}{1.477103in}}%
\pgfpathlineto{\pgfqpoint{0.970970in}{1.480283in}}%
\pgfpathlineto{\pgfqpoint{0.974364in}{1.486126in}}%
\pgfpathlineto{\pgfqpoint{0.977762in}{1.491900in}}%
\pgfpathlineto{\pgfqpoint{0.981164in}{1.497602in}}%
\pgfpathlineto{\pgfqpoint{0.984569in}{1.503232in}}%
\pgfpathlineto{\pgfqpoint{0.981705in}{1.500269in}}%
\pgfpathlineto{\pgfqpoint{0.979034in}{1.497263in}}%
\pgfpathlineto{\pgfqpoint{0.976560in}{1.494217in}}%
\pgfpathlineto{\pgfqpoint{0.974286in}{1.491134in}}%
\pgfpathclose%
\pgfusepath{fill}%
\end{pgfscope}%
\begin{pgfscope}%
\pgfpathrectangle{\pgfqpoint{0.041670in}{0.041670in}}{\pgfqpoint{2.216660in}{2.216660in}}%
\pgfusepath{clip}%
\pgfsetbuttcap%
\pgfsetroundjoin%
\definecolor{currentfill}{rgb}{0.122606,0.585371,0.546557}%
\pgfsetfillcolor{currentfill}%
\pgfsetlinewidth{0.000000pt}%
\definecolor{currentstroke}{rgb}{0.000000,0.000000,0.000000}%
\pgfsetstrokecolor{currentstroke}%
\pgfsetdash{}{0pt}%
\pgfpathmoveto{\pgfqpoint{1.487415in}{1.282793in}}%
\pgfpathlineto{\pgfqpoint{1.491148in}{1.275106in}}%
\pgfpathlineto{\pgfqpoint{1.494879in}{1.267400in}}%
\pgfpathlineto{\pgfqpoint{1.498606in}{1.259678in}}%
\pgfpathlineto{\pgfqpoint{1.502330in}{1.251943in}}%
\pgfpathlineto{\pgfqpoint{1.501987in}{1.246907in}}%
\pgfpathlineto{\pgfqpoint{1.501321in}{1.241876in}}%
\pgfpathlineto{\pgfqpoint{1.500332in}{1.236853in}}%
\pgfpathlineto{\pgfqpoint{1.499022in}{1.231845in}}%
\pgfpathlineto{\pgfqpoint{1.495325in}{1.239835in}}%
\pgfpathlineto{\pgfqpoint{1.491626in}{1.247812in}}%
\pgfpathlineto{\pgfqpoint{1.487923in}{1.255772in}}%
\pgfpathlineto{\pgfqpoint{1.484219in}{1.263713in}}%
\pgfpathlineto{\pgfqpoint{1.485479in}{1.268467in}}%
\pgfpathlineto{\pgfqpoint{1.486431in}{1.273235in}}%
\pgfpathlineto{\pgfqpoint{1.487077in}{1.278012in}}%
\pgfpathlineto{\pgfqpoint{1.487415in}{1.282793in}}%
\pgfpathclose%
\pgfusepath{fill}%
\end{pgfscope}%
\begin{pgfscope}%
\pgfpathrectangle{\pgfqpoint{0.041670in}{0.041670in}}{\pgfqpoint{2.216660in}{2.216660in}}%
\pgfusepath{clip}%
\pgfsetbuttcap%
\pgfsetroundjoin%
\definecolor{currentfill}{rgb}{0.134692,0.658636,0.517649}%
\pgfsetfillcolor{currentfill}%
\pgfsetlinewidth{0.000000pt}%
\definecolor{currentstroke}{rgb}{0.000000,0.000000,0.000000}%
\pgfsetstrokecolor{currentstroke}%
\pgfsetdash{}{0pt}%
\pgfpathmoveto{\pgfqpoint{1.455971in}{1.360378in}}%
\pgfpathlineto{\pgfqpoint{1.459699in}{1.353181in}}%
\pgfpathlineto{\pgfqpoint{1.463425in}{1.345946in}}%
\pgfpathlineto{\pgfqpoint{1.467147in}{1.338674in}}%
\pgfpathlineto{\pgfqpoint{1.470865in}{1.331370in}}%
\pgfpathlineto{\pgfqpoint{1.471697in}{1.326869in}}%
\pgfpathlineto{\pgfqpoint{1.472240in}{1.322355in}}%
\pgfpathlineto{\pgfqpoint{1.472491in}{1.317832in}}%
\pgfpathlineto{\pgfqpoint{1.472452in}{1.313304in}}%
\pgfpathlineto{\pgfqpoint{1.468703in}{1.320860in}}%
\pgfpathlineto{\pgfqpoint{1.464951in}{1.328382in}}%
\pgfpathlineto{\pgfqpoint{1.461196in}{1.335867in}}%
\pgfpathlineto{\pgfqpoint{1.457438in}{1.343314in}}%
\pgfpathlineto{\pgfqpoint{1.457486in}{1.347590in}}%
\pgfpathlineto{\pgfqpoint{1.457256in}{1.351863in}}%
\pgfpathlineto{\pgfqpoint{1.456751in}{1.356127in}}%
\pgfpathlineto{\pgfqpoint{1.455971in}{1.360378in}}%
\pgfpathclose%
\pgfusepath{fill}%
\end{pgfscope}%
\begin{pgfscope}%
\pgfpathrectangle{\pgfqpoint{0.041670in}{0.041670in}}{\pgfqpoint{2.216660in}{2.216660in}}%
\pgfusepath{clip}%
\pgfsetbuttcap%
\pgfsetroundjoin%
\definecolor{currentfill}{rgb}{0.487026,0.823929,0.312321}%
\pgfsetfillcolor{currentfill}%
\pgfsetlinewidth{0.000000pt}%
\definecolor{currentstroke}{rgb}{0.000000,0.000000,0.000000}%
\pgfsetstrokecolor{currentstroke}%
\pgfsetdash{}{0pt}%
\pgfpathmoveto{\pgfqpoint{1.023367in}{1.555116in}}%
\pgfpathlineto{\pgfqpoint{1.020172in}{1.550347in}}%
\pgfpathlineto{\pgfqpoint{1.016980in}{1.545489in}}%
\pgfpathlineto{\pgfqpoint{1.013792in}{1.540545in}}%
\pgfpathlineto{\pgfqpoint{1.010607in}{1.535517in}}%
\pgfpathlineto{\pgfqpoint{1.014123in}{1.538026in}}%
\pgfpathlineto{\pgfqpoint{1.017800in}{1.540481in}}%
\pgfpathlineto{\pgfqpoint{1.021635in}{1.542878in}}%
\pgfpathlineto{\pgfqpoint{1.025623in}{1.545217in}}%
\pgfpathlineto{\pgfqpoint{1.028521in}{1.550055in}}%
\pgfpathlineto{\pgfqpoint{1.031423in}{1.554810in}}%
\pgfpathlineto{\pgfqpoint{1.034328in}{1.559479in}}%
\pgfpathlineto{\pgfqpoint{1.037236in}{1.564060in}}%
\pgfpathlineto{\pgfqpoint{1.033551in}{1.561904in}}%
\pgfpathlineto{\pgfqpoint{1.030010in}{1.559693in}}%
\pgfpathlineto{\pgfqpoint{1.026614in}{1.557430in}}%
\pgfpathlineto{\pgfqpoint{1.023367in}{1.555116in}}%
\pgfpathclose%
\pgfusepath{fill}%
\end{pgfscope}%
\begin{pgfscope}%
\pgfpathrectangle{\pgfqpoint{0.041670in}{0.041670in}}{\pgfqpoint{2.216660in}{2.216660in}}%
\pgfusepath{clip}%
\pgfsetbuttcap%
\pgfsetroundjoin%
\definecolor{currentfill}{rgb}{0.263663,0.237631,0.518762}%
\pgfsetfillcolor{currentfill}%
\pgfsetlinewidth{0.000000pt}%
\definecolor{currentstroke}{rgb}{0.000000,0.000000,0.000000}%
\pgfsetstrokecolor{currentstroke}%
\pgfsetdash{}{0pt}%
\pgfpathmoveto{\pgfqpoint{1.587092in}{0.931039in}}%
\pgfpathlineto{\pgfqpoint{1.590518in}{0.923745in}}%
\pgfpathlineto{\pgfqpoint{1.593945in}{0.916541in}}%
\pgfpathlineto{\pgfqpoint{1.597372in}{0.909433in}}%
\pgfpathlineto{\pgfqpoint{1.600800in}{0.902424in}}%
\pgfpathlineto{\pgfqpoint{1.595052in}{0.895557in}}%
\pgfpathlineto{\pgfqpoint{1.588872in}{0.888782in}}%
\pgfpathlineto{\pgfqpoint{1.582267in}{0.882105in}}%
\pgfpathlineto{\pgfqpoint{1.575241in}{0.875534in}}%
\pgfpathlineto{\pgfqpoint{1.572009in}{0.882790in}}%
\pgfpathlineto{\pgfqpoint{1.568778in}{0.890143in}}%
\pgfpathlineto{\pgfqpoint{1.565548in}{0.897592in}}%
\pgfpathlineto{\pgfqpoint{1.562318in}{0.905132in}}%
\pgfpathlineto{\pgfqpoint{1.569126in}{0.911462in}}%
\pgfpathlineto{\pgfqpoint{1.575527in}{0.917895in}}%
\pgfpathlineto{\pgfqpoint{1.581518in}{0.924423in}}%
\pgfpathlineto{\pgfqpoint{1.587092in}{0.931039in}}%
\pgfpathclose%
\pgfusepath{fill}%
\end{pgfscope}%
\begin{pgfscope}%
\pgfpathrectangle{\pgfqpoint{0.041670in}{0.041670in}}{\pgfqpoint{2.216660in}{2.216660in}}%
\pgfusepath{clip}%
\pgfsetbuttcap%
\pgfsetroundjoin%
\definecolor{currentfill}{rgb}{0.195860,0.395433,0.555276}%
\pgfsetfillcolor{currentfill}%
\pgfsetlinewidth{0.000000pt}%
\definecolor{currentstroke}{rgb}{0.000000,0.000000,0.000000}%
\pgfsetstrokecolor{currentstroke}%
\pgfsetdash{}{0pt}%
\pgfpathmoveto{\pgfqpoint{1.548038in}{1.080557in}}%
\pgfpathlineto{\pgfqpoint{1.551614in}{1.072511in}}%
\pgfpathlineto{\pgfqpoint{1.555189in}{1.064503in}}%
\pgfpathlineto{\pgfqpoint{1.558762in}{1.056538in}}%
\pgfpathlineto{\pgfqpoint{1.562334in}{1.048617in}}%
\pgfpathlineto{\pgfqpoint{1.558820in}{1.042491in}}%
\pgfpathlineto{\pgfqpoint{1.554919in}{1.036419in}}%
\pgfpathlineto{\pgfqpoint{1.550633in}{1.030408in}}%
\pgfpathlineto{\pgfqpoint{1.545965in}{1.024464in}}%
\pgfpathlineto{\pgfqpoint{1.542535in}{1.032639in}}%
\pgfpathlineto{\pgfqpoint{1.539103in}{1.040857in}}%
\pgfpathlineto{\pgfqpoint{1.535670in}{1.049117in}}%
\pgfpathlineto{\pgfqpoint{1.532236in}{1.057416in}}%
\pgfpathlineto{\pgfqpoint{1.536739in}{1.063110in}}%
\pgfpathlineto{\pgfqpoint{1.540876in}{1.068869in}}%
\pgfpathlineto{\pgfqpoint{1.544643in}{1.074687in}}%
\pgfpathlineto{\pgfqpoint{1.548038in}{1.080557in}}%
\pgfpathclose%
\pgfusepath{fill}%
\end{pgfscope}%
\begin{pgfscope}%
\pgfpathrectangle{\pgfqpoint{0.041670in}{0.041670in}}{\pgfqpoint{2.216660in}{2.216660in}}%
\pgfusepath{clip}%
\pgfsetbuttcap%
\pgfsetroundjoin%
\definecolor{currentfill}{rgb}{0.274128,0.199721,0.498911}%
\pgfsetfillcolor{currentfill}%
\pgfsetlinewidth{0.000000pt}%
\definecolor{currentstroke}{rgb}{0.000000,0.000000,0.000000}%
\pgfsetstrokecolor{currentstroke}%
\pgfsetdash{}{0pt}%
\pgfpathmoveto{\pgfqpoint{0.791262in}{0.869789in}}%
\pgfpathlineto{\pgfqpoint{0.788080in}{0.862583in}}%
\pgfpathlineto{\pgfqpoint{0.784898in}{0.855484in}}%
\pgfpathlineto{\pgfqpoint{0.781714in}{0.848493in}}%
\pgfpathlineto{\pgfqpoint{0.778529in}{0.841616in}}%
\pgfpathlineto{\pgfqpoint{0.770905in}{0.848324in}}%
\pgfpathlineto{\pgfqpoint{0.763710in}{0.855148in}}%
\pgfpathlineto{\pgfqpoint{0.756950in}{0.862081in}}%
\pgfpathlineto{\pgfqpoint{0.750631in}{0.869115in}}%
\pgfpathlineto{\pgfqpoint{0.754024in}{0.875749in}}%
\pgfpathlineto{\pgfqpoint{0.757416in}{0.882496in}}%
\pgfpathlineto{\pgfqpoint{0.760808in}{0.889353in}}%
\pgfpathlineto{\pgfqpoint{0.764198in}{0.896316in}}%
\pgfpathlineto{\pgfqpoint{0.770330in}{0.889530in}}%
\pgfpathlineto{\pgfqpoint{0.776888in}{0.882842in}}%
\pgfpathlineto{\pgfqpoint{0.783868in}{0.876259in}}%
\pgfpathlineto{\pgfqpoint{0.791262in}{0.869789in}}%
\pgfpathclose%
\pgfusepath{fill}%
\end{pgfscope}%
\begin{pgfscope}%
\pgfpathrectangle{\pgfqpoint{0.041670in}{0.041670in}}{\pgfqpoint{2.216660in}{2.216660in}}%
\pgfusepath{clip}%
\pgfsetbuttcap%
\pgfsetroundjoin%
\definecolor{currentfill}{rgb}{0.636902,0.856542,0.216620}%
\pgfsetfillcolor{currentfill}%
\pgfsetlinewidth{0.000000pt}%
\definecolor{currentstroke}{rgb}{0.000000,0.000000,0.000000}%
\pgfsetstrokecolor{currentstroke}%
\pgfsetdash{}{0pt}%
\pgfpathmoveto{\pgfqpoint{1.250484in}{1.615716in}}%
\pgfpathlineto{\pgfqpoint{1.252212in}{1.612359in}}%
\pgfpathlineto{\pgfqpoint{1.253939in}{1.608902in}}%
\pgfpathlineto{\pgfqpoint{1.255663in}{1.605345in}}%
\pgfpathlineto{\pgfqpoint{1.257385in}{1.601691in}}%
\pgfpathlineto{\pgfqpoint{1.262110in}{1.600508in}}%
\pgfpathlineto{\pgfqpoint{1.266756in}{1.599256in}}%
\pgfpathlineto{\pgfqpoint{1.271318in}{1.597934in}}%
\pgfpathlineto{\pgfqpoint{1.275792in}{1.596546in}}%
\pgfpathlineto{\pgfqpoint{1.273659in}{1.600317in}}%
\pgfpathlineto{\pgfqpoint{1.271523in}{1.603991in}}%
\pgfpathlineto{\pgfqpoint{1.269385in}{1.607565in}}%
\pgfpathlineto{\pgfqpoint{1.267244in}{1.611039in}}%
\pgfpathlineto{\pgfqpoint{1.263171in}{1.612301in}}%
\pgfpathlineto{\pgfqpoint{1.259017in}{1.613502in}}%
\pgfpathlineto{\pgfqpoint{1.254787in}{1.614641in}}%
\pgfpathlineto{\pgfqpoint{1.250484in}{1.615716in}}%
\pgfpathclose%
\pgfusepath{fill}%
\end{pgfscope}%
\begin{pgfscope}%
\pgfpathrectangle{\pgfqpoint{0.041670in}{0.041670in}}{\pgfqpoint{2.216660in}{2.216660in}}%
\pgfusepath{clip}%
\pgfsetbuttcap%
\pgfsetroundjoin%
\definecolor{currentfill}{rgb}{0.272594,0.025563,0.353093}%
\pgfsetfillcolor{currentfill}%
\pgfsetlinewidth{0.000000pt}%
\definecolor{currentstroke}{rgb}{0.000000,0.000000,0.000000}%
\pgfsetstrokecolor{currentstroke}%
\pgfsetdash{}{0pt}%
\pgfpathmoveto{\pgfqpoint{1.754737in}{0.751343in}}%
\pgfpathlineto{\pgfqpoint{1.758377in}{0.752983in}}%
\pgfpathlineto{\pgfqpoint{1.762028in}{0.754924in}}%
\pgfpathlineto{\pgfqpoint{1.765689in}{0.757169in}}%
\pgfpathlineto{\pgfqpoint{1.769362in}{0.759726in}}%
\pgfpathlineto{\pgfqpoint{1.761533in}{0.749914in}}%
\pgfpathlineto{\pgfqpoint{1.753091in}{0.740225in}}%
\pgfpathlineto{\pgfqpoint{1.744042in}{0.730671in}}%
\pgfpathlineto{\pgfqpoint{1.734395in}{0.721262in}}%
\pgfpathlineto{\pgfqpoint{1.730922in}{0.718933in}}%
\pgfpathlineto{\pgfqpoint{1.727460in}{0.716916in}}%
\pgfpathlineto{\pgfqpoint{1.724009in}{0.715205in}}%
\pgfpathlineto{\pgfqpoint{1.720567in}{0.713795in}}%
\pgfpathlineto{\pgfqpoint{1.729992in}{0.722980in}}%
\pgfpathlineto{\pgfqpoint{1.738833in}{0.732306in}}%
\pgfpathlineto{\pgfqpoint{1.747083in}{0.741764in}}%
\pgfpathlineto{\pgfqpoint{1.754737in}{0.751343in}}%
\pgfpathclose%
\pgfusepath{fill}%
\end{pgfscope}%
\begin{pgfscope}%
\pgfpathrectangle{\pgfqpoint{0.041670in}{0.041670in}}{\pgfqpoint{2.216660in}{2.216660in}}%
\pgfusepath{clip}%
\pgfsetbuttcap%
\pgfsetroundjoin%
\definecolor{currentfill}{rgb}{0.565498,0.842430,0.262877}%
\pgfsetfillcolor{currentfill}%
\pgfsetlinewidth{0.000000pt}%
\definecolor{currentstroke}{rgb}{0.000000,0.000000,0.000000}%
\pgfsetstrokecolor{currentstroke}%
\pgfsetdash{}{0pt}%
\pgfpathmoveto{\pgfqpoint{1.292719in}{1.590345in}}%
\pgfpathlineto{\pgfqpoint{1.295229in}{1.586336in}}%
\pgfpathlineto{\pgfqpoint{1.297736in}{1.582232in}}%
\pgfpathlineto{\pgfqpoint{1.300239in}{1.578036in}}%
\pgfpathlineto{\pgfqpoint{1.302740in}{1.573748in}}%
\pgfpathlineto{\pgfqpoint{1.307065in}{1.571888in}}%
\pgfpathlineto{\pgfqpoint{1.311266in}{1.569963in}}%
\pgfpathlineto{\pgfqpoint{1.315340in}{1.567976in}}%
\pgfpathlineto{\pgfqpoint{1.319283in}{1.565929in}}%
\pgfpathlineto{\pgfqpoint{1.316441in}{1.570381in}}%
\pgfpathlineto{\pgfqpoint{1.313597in}{1.574741in}}%
\pgfpathlineto{\pgfqpoint{1.310749in}{1.579009in}}%
\pgfpathlineto{\pgfqpoint{1.307899in}{1.583182in}}%
\pgfpathlineto{\pgfqpoint{1.304281in}{1.585058in}}%
\pgfpathlineto{\pgfqpoint{1.300543in}{1.586877in}}%
\pgfpathlineto{\pgfqpoint{1.296688in}{1.588640in}}%
\pgfpathlineto{\pgfqpoint{1.292719in}{1.590345in}}%
\pgfpathclose%
\pgfusepath{fill}%
\end{pgfscope}%
\begin{pgfscope}%
\pgfpathrectangle{\pgfqpoint{0.041670in}{0.041670in}}{\pgfqpoint{2.216660in}{2.216660in}}%
\pgfusepath{clip}%
\pgfsetbuttcap%
\pgfsetroundjoin%
\definecolor{currentfill}{rgb}{0.282884,0.135920,0.453427}%
\pgfsetfillcolor{currentfill}%
\pgfsetlinewidth{0.000000pt}%
\definecolor{currentstroke}{rgb}{0.000000,0.000000,0.000000}%
\pgfsetstrokecolor{currentstroke}%
\pgfsetdash{}{0pt}%
\pgfpathmoveto{\pgfqpoint{0.567978in}{0.782930in}}%
\pgfpathlineto{\pgfqpoint{0.564231in}{0.788497in}}%
\pgfpathlineto{\pgfqpoint{0.560470in}{0.794435in}}%
\pgfpathlineto{\pgfqpoint{0.556694in}{0.800750in}}%
\pgfpathlineto{\pgfqpoint{0.552903in}{0.807449in}}%
\pgfpathlineto{\pgfqpoint{0.544614in}{0.817944in}}%
\pgfpathlineto{\pgfqpoint{0.536994in}{0.828555in}}%
\pgfpathlineto{\pgfqpoint{0.530049in}{0.839272in}}%
\pgfpathlineto{\pgfqpoint{0.523782in}{0.850082in}}%
\pgfpathlineto{\pgfqpoint{0.527726in}{0.843166in}}%
\pgfpathlineto{\pgfqpoint{0.531654in}{0.836631in}}%
\pgfpathlineto{\pgfqpoint{0.535568in}{0.830473in}}%
\pgfpathlineto{\pgfqpoint{0.539466in}{0.824683in}}%
\pgfpathlineto{\pgfqpoint{0.545606in}{0.814095in}}%
\pgfpathlineto{\pgfqpoint{0.552407in}{0.803599in}}%
\pgfpathlineto{\pgfqpoint{0.559866in}{0.793207in}}%
\pgfpathlineto{\pgfqpoint{0.567978in}{0.782930in}}%
\pgfpathclose%
\pgfusepath{fill}%
\end{pgfscope}%
\begin{pgfscope}%
\pgfpathrectangle{\pgfqpoint{0.041670in}{0.041670in}}{\pgfqpoint{2.216660in}{2.216660in}}%
\pgfusepath{clip}%
\pgfsetbuttcap%
\pgfsetroundjoin%
\definecolor{currentfill}{rgb}{0.147607,0.511733,0.557049}%
\pgfsetfillcolor{currentfill}%
\pgfsetlinewidth{0.000000pt}%
\definecolor{currentstroke}{rgb}{0.000000,0.000000,0.000000}%
\pgfsetstrokecolor{currentstroke}%
\pgfsetdash{}{0pt}%
\pgfpathmoveto{\pgfqpoint{1.513783in}{1.199803in}}%
\pgfpathlineto{\pgfqpoint{1.517467in}{1.191786in}}%
\pgfpathlineto{\pgfqpoint{1.521149in}{1.183773in}}%
\pgfpathlineto{\pgfqpoint{1.524828in}{1.175765in}}%
\pgfpathlineto{\pgfqpoint{1.528505in}{1.167765in}}%
\pgfpathlineto{\pgfqpoint{1.526746in}{1.162264in}}%
\pgfpathlineto{\pgfqpoint{1.524636in}{1.156790in}}%
\pgfpathlineto{\pgfqpoint{1.522177in}{1.151347in}}%
\pgfpathlineto{\pgfqpoint{1.519370in}{1.145942in}}%
\pgfpathlineto{\pgfqpoint{1.515779in}{1.154197in}}%
\pgfpathlineto{\pgfqpoint{1.512185in}{1.162460in}}%
\pgfpathlineto{\pgfqpoint{1.508589in}{1.170729in}}%
\pgfpathlineto{\pgfqpoint{1.504991in}{1.179001in}}%
\pgfpathlineto{\pgfqpoint{1.507690in}{1.184153in}}%
\pgfpathlineto{\pgfqpoint{1.510055in}{1.189341in}}%
\pgfpathlineto{\pgfqpoint{1.512087in}{1.194559in}}%
\pgfpathlineto{\pgfqpoint{1.513783in}{1.199803in}}%
\pgfpathclose%
\pgfusepath{fill}%
\end{pgfscope}%
\begin{pgfscope}%
\pgfpathrectangle{\pgfqpoint{0.041670in}{0.041670in}}{\pgfqpoint{2.216660in}{2.216660in}}%
\pgfusepath{clip}%
\pgfsetbuttcap%
\pgfsetroundjoin%
\definecolor{currentfill}{rgb}{0.281477,0.755203,0.432552}%
\pgfsetfillcolor{currentfill}%
\pgfsetlinewidth{0.000000pt}%
\definecolor{currentstroke}{rgb}{0.000000,0.000000,0.000000}%
\pgfsetstrokecolor{currentstroke}%
\pgfsetdash{}{0pt}%
\pgfpathmoveto{\pgfqpoint{0.952365in}{1.453719in}}%
\pgfpathlineto{\pgfqpoint{0.948657in}{1.447353in}}%
\pgfpathlineto{\pgfqpoint{0.944953in}{1.440921in}}%
\pgfpathlineto{\pgfqpoint{0.941253in}{1.434425in}}%
\pgfpathlineto{\pgfqpoint{0.937556in}{1.427869in}}%
\pgfpathlineto{\pgfqpoint{0.939228in}{1.431545in}}%
\pgfpathlineto{\pgfqpoint{0.941137in}{1.435192in}}%
\pgfpathlineto{\pgfqpoint{0.943281in}{1.438806in}}%
\pgfpathlineto{\pgfqpoint{0.945658in}{1.442384in}}%
\pgfpathlineto{\pgfqpoint{0.949223in}{1.448705in}}%
\pgfpathlineto{\pgfqpoint{0.952793in}{1.454966in}}%
\pgfpathlineto{\pgfqpoint{0.956366in}{1.461164in}}%
\pgfpathlineto{\pgfqpoint{0.959943in}{1.467297in}}%
\pgfpathlineto{\pgfqpoint{0.957719in}{1.463950in}}%
\pgfpathlineto{\pgfqpoint{0.955713in}{1.460569in}}%
\pgfpathlineto{\pgfqpoint{0.953928in}{1.457158in}}%
\pgfpathlineto{\pgfqpoint{0.952365in}{1.453719in}}%
\pgfpathclose%
\pgfusepath{fill}%
\end{pgfscope}%
\begin{pgfscope}%
\pgfpathrectangle{\pgfqpoint{0.041670in}{0.041670in}}{\pgfqpoint{2.216660in}{2.216660in}}%
\pgfusepath{clip}%
\pgfsetbuttcap%
\pgfsetroundjoin%
\definecolor{currentfill}{rgb}{0.268510,0.009605,0.335427}%
\pgfsetfillcolor{currentfill}%
\pgfsetlinewidth{0.000000pt}%
\definecolor{currentstroke}{rgb}{0.000000,0.000000,0.000000}%
\pgfsetstrokecolor{currentstroke}%
\pgfsetdash{}{0pt}%
\pgfpathmoveto{\pgfqpoint{0.661668in}{0.703221in}}%
\pgfpathlineto{\pgfqpoint{0.658315in}{0.703431in}}%
\pgfpathlineto{\pgfqpoint{0.654954in}{0.703921in}}%
\pgfpathlineto{\pgfqpoint{0.651583in}{0.704695in}}%
\pgfpathlineto{\pgfqpoint{0.648204in}{0.705759in}}%
\pgfpathlineto{\pgfqpoint{0.638267in}{0.714808in}}%
\pgfpathlineto{\pgfqpoint{0.628907in}{0.724009in}}%
\pgfpathlineto{\pgfqpoint{0.620131in}{0.733350in}}%
\pgfpathlineto{\pgfqpoint{0.611947in}{0.742822in}}%
\pgfpathlineto{\pgfqpoint{0.615538in}{0.741529in}}%
\pgfpathlineto{\pgfqpoint{0.619121in}{0.740525in}}%
\pgfpathlineto{\pgfqpoint{0.622694in}{0.739804in}}%
\pgfpathlineto{\pgfqpoint{0.626258in}{0.739363in}}%
\pgfpathlineto{\pgfqpoint{0.634254in}{0.730126in}}%
\pgfpathlineto{\pgfqpoint{0.642825in}{0.721016in}}%
\pgfpathlineto{\pgfqpoint{0.651966in}{0.712045in}}%
\pgfpathlineto{\pgfqpoint{0.661668in}{0.703221in}}%
\pgfpathclose%
\pgfusepath{fill}%
\end{pgfscope}%
\begin{pgfscope}%
\pgfpathrectangle{\pgfqpoint{0.041670in}{0.041670in}}{\pgfqpoint{2.216660in}{2.216660in}}%
\pgfusepath{clip}%
\pgfsetbuttcap%
\pgfsetroundjoin%
\definecolor{currentfill}{rgb}{0.636902,0.856542,0.216620}%
\pgfsetfillcolor{currentfill}%
\pgfsetlinewidth{0.000000pt}%
\definecolor{currentstroke}{rgb}{0.000000,0.000000,0.000000}%
\pgfsetstrokecolor{currentstroke}%
\pgfsetdash{}{0pt}%
\pgfpathmoveto{\pgfqpoint{1.089116in}{1.609867in}}%
\pgfpathlineto{\pgfqpoint{1.086888in}{1.606364in}}%
\pgfpathlineto{\pgfqpoint{1.084662in}{1.602760in}}%
\pgfpathlineto{\pgfqpoint{1.082439in}{1.599057in}}%
\pgfpathlineto{\pgfqpoint{1.080218in}{1.595256in}}%
\pgfpathlineto{\pgfqpoint{1.084610in}{1.596703in}}%
\pgfpathlineto{\pgfqpoint{1.089094in}{1.598084in}}%
\pgfpathlineto{\pgfqpoint{1.093666in}{1.599398in}}%
\pgfpathlineto{\pgfqpoint{1.098321in}{1.600643in}}%
\pgfpathlineto{\pgfqpoint{1.100137in}{1.604321in}}%
\pgfpathlineto{\pgfqpoint{1.101955in}{1.607901in}}%
\pgfpathlineto{\pgfqpoint{1.103775in}{1.611383in}}%
\pgfpathlineto{\pgfqpoint{1.105598in}{1.614763in}}%
\pgfpathlineto{\pgfqpoint{1.101359in}{1.613632in}}%
\pgfpathlineto{\pgfqpoint{1.097196in}{1.612438in}}%
\pgfpathlineto{\pgfqpoint{1.093114in}{1.611182in}}%
\pgfpathlineto{\pgfqpoint{1.089116in}{1.609867in}}%
\pgfpathclose%
\pgfusepath{fill}%
\end{pgfscope}%
\begin{pgfscope}%
\pgfpathrectangle{\pgfqpoint{0.041670in}{0.041670in}}{\pgfqpoint{2.216660in}{2.216660in}}%
\pgfusepath{clip}%
\pgfsetbuttcap%
\pgfsetroundjoin%
\definecolor{currentfill}{rgb}{0.565498,0.842430,0.262877}%
\pgfsetfillcolor{currentfill}%
\pgfsetlinewidth{0.000000pt}%
\definecolor{currentstroke}{rgb}{0.000000,0.000000,0.000000}%
\pgfsetstrokecolor{currentstroke}%
\pgfsetdash{}{0pt}%
\pgfpathmoveto{\pgfqpoint{1.048900in}{1.581471in}}%
\pgfpathlineto{\pgfqpoint{1.045979in}{1.577258in}}%
\pgfpathlineto{\pgfqpoint{1.043061in}{1.572951in}}%
\pgfpathlineto{\pgfqpoint{1.040147in}{1.568551in}}%
\pgfpathlineto{\pgfqpoint{1.037236in}{1.564060in}}%
\pgfpathlineto{\pgfqpoint{1.041058in}{1.566159in}}%
\pgfpathlineto{\pgfqpoint{1.045016in}{1.568200in}}%
\pgfpathlineto{\pgfqpoint{1.049104in}{1.570180in}}%
\pgfpathlineto{\pgfqpoint{1.053320in}{1.572097in}}%
\pgfpathlineto{\pgfqpoint{1.055900in}{1.576420in}}%
\pgfpathlineto{\pgfqpoint{1.058483in}{1.580651in}}%
\pgfpathlineto{\pgfqpoint{1.061069in}{1.584789in}}%
\pgfpathlineto{\pgfqpoint{1.063657in}{1.588833in}}%
\pgfpathlineto{\pgfqpoint{1.059789in}{1.587076in}}%
\pgfpathlineto{\pgfqpoint{1.056038in}{1.585263in}}%
\pgfpathlineto{\pgfqpoint{1.052407in}{1.583393in}}%
\pgfpathlineto{\pgfqpoint{1.048900in}{1.581471in}}%
\pgfpathclose%
\pgfusepath{fill}%
\end{pgfscope}%
\begin{pgfscope}%
\pgfpathrectangle{\pgfqpoint{0.041670in}{0.041670in}}{\pgfqpoint{2.216660in}{2.216660in}}%
\pgfusepath{clip}%
\pgfsetbuttcap%
\pgfsetroundjoin%
\definecolor{currentfill}{rgb}{0.220124,0.725509,0.466226}%
\pgfsetfillcolor{currentfill}%
\pgfsetlinewidth{0.000000pt}%
\definecolor{currentstroke}{rgb}{0.000000,0.000000,0.000000}%
\pgfsetstrokecolor{currentstroke}%
\pgfsetdash{}{0pt}%
\pgfpathmoveto{\pgfqpoint{1.420879in}{1.431138in}}%
\pgfpathlineto{\pgfqpoint{1.424549in}{1.424575in}}%
\pgfpathlineto{\pgfqpoint{1.428214in}{1.417956in}}%
\pgfpathlineto{\pgfqpoint{1.431876in}{1.411282in}}%
\pgfpathlineto{\pgfqpoint{1.435535in}{1.404556in}}%
\pgfpathlineto{\pgfqpoint{1.437289in}{1.400637in}}%
\pgfpathlineto{\pgfqpoint{1.438790in}{1.396691in}}%
\pgfpathlineto{\pgfqpoint{1.440035in}{1.392722in}}%
\pgfpathlineto{\pgfqpoint{1.441022in}{1.388733in}}%
\pgfpathlineto{\pgfqpoint{1.437277in}{1.395702in}}%
\pgfpathlineto{\pgfqpoint{1.433528in}{1.402618in}}%
\pgfpathlineto{\pgfqpoint{1.429775in}{1.409480in}}%
\pgfpathlineto{\pgfqpoint{1.426019in}{1.416284in}}%
\pgfpathlineto{\pgfqpoint{1.425097in}{1.420028in}}%
\pgfpathlineto{\pgfqpoint{1.423931in}{1.423754in}}%
\pgfpathlineto{\pgfqpoint{1.422525in}{1.427458in}}%
\pgfpathlineto{\pgfqpoint{1.420879in}{1.431138in}}%
\pgfpathclose%
\pgfusepath{fill}%
\end{pgfscope}%
\begin{pgfscope}%
\pgfpathrectangle{\pgfqpoint{0.041670in}{0.041670in}}{\pgfqpoint{2.216660in}{2.216660in}}%
\pgfusepath{clip}%
\pgfsetbuttcap%
\pgfsetroundjoin%
\definecolor{currentfill}{rgb}{0.122606,0.585371,0.546557}%
\pgfsetfillcolor{currentfill}%
\pgfsetlinewidth{0.000000pt}%
\definecolor{currentstroke}{rgb}{0.000000,0.000000,0.000000}%
\pgfsetstrokecolor{currentstroke}%
\pgfsetdash{}{0pt}%
\pgfpathmoveto{\pgfqpoint{0.877068in}{1.259503in}}%
\pgfpathlineto{\pgfqpoint{0.873378in}{1.251506in}}%
\pgfpathlineto{\pgfqpoint{0.869690in}{1.243489in}}%
\pgfpathlineto{\pgfqpoint{0.866005in}{1.235456in}}%
\pgfpathlineto{\pgfqpoint{0.862322in}{1.227410in}}%
\pgfpathlineto{\pgfqpoint{0.860726in}{1.232401in}}%
\pgfpathlineto{\pgfqpoint{0.859452in}{1.237411in}}%
\pgfpathlineto{\pgfqpoint{0.858499in}{1.242434in}}%
\pgfpathlineto{\pgfqpoint{0.857869in}{1.247467in}}%
\pgfpathlineto{\pgfqpoint{0.861592in}{1.255258in}}%
\pgfpathlineto{\pgfqpoint{0.865319in}{1.263037in}}%
\pgfpathlineto{\pgfqpoint{0.869048in}{1.270799in}}%
\pgfpathlineto{\pgfqpoint{0.872780in}{1.278543in}}%
\pgfpathlineto{\pgfqpoint{0.873391in}{1.273765in}}%
\pgfpathlineto{\pgfqpoint{0.874310in}{1.268996in}}%
\pgfpathlineto{\pgfqpoint{0.875536in}{1.264241in}}%
\pgfpathlineto{\pgfqpoint{0.877068in}{1.259503in}}%
\pgfpathclose%
\pgfusepath{fill}%
\end{pgfscope}%
\begin{pgfscope}%
\pgfpathrectangle{\pgfqpoint{0.041670in}{0.041670in}}{\pgfqpoint{2.216660in}{2.216660in}}%
\pgfusepath{clip}%
\pgfsetbuttcap%
\pgfsetroundjoin%
\definecolor{currentfill}{rgb}{0.134692,0.658636,0.517649}%
\pgfsetfillcolor{currentfill}%
\pgfsetlinewidth{0.000000pt}%
\definecolor{currentstroke}{rgb}{0.000000,0.000000,0.000000}%
\pgfsetstrokecolor{currentstroke}%
\pgfsetdash{}{0pt}%
\pgfpathmoveto{\pgfqpoint{0.902746in}{1.339512in}}%
\pgfpathlineto{\pgfqpoint{0.898989in}{1.332010in}}%
\pgfpathlineto{\pgfqpoint{0.895236in}{1.324469in}}%
\pgfpathlineto{\pgfqpoint{0.891486in}{1.316891in}}%
\pgfpathlineto{\pgfqpoint{0.887738in}{1.309279in}}%
\pgfpathlineto{\pgfqpoint{0.887439in}{1.313808in}}%
\pgfpathlineto{\pgfqpoint{0.887432in}{1.318335in}}%
\pgfpathlineto{\pgfqpoint{0.887716in}{1.322857in}}%
\pgfpathlineto{\pgfqpoint{0.888291in}{1.327370in}}%
\pgfpathlineto{\pgfqpoint{0.892021in}{1.334730in}}%
\pgfpathlineto{\pgfqpoint{0.895755in}{1.342057in}}%
\pgfpathlineto{\pgfqpoint{0.899492in}{1.349348in}}%
\pgfpathlineto{\pgfqpoint{0.903232in}{1.356600in}}%
\pgfpathlineto{\pgfqpoint{0.902696in}{1.352337in}}%
\pgfpathlineto{\pgfqpoint{0.902436in}{1.348065in}}%
\pgfpathlineto{\pgfqpoint{0.902452in}{1.343789in}}%
\pgfpathlineto{\pgfqpoint{0.902746in}{1.339512in}}%
\pgfpathclose%
\pgfusepath{fill}%
\end{pgfscope}%
\begin{pgfscope}%
\pgfpathrectangle{\pgfqpoint{0.041670in}{0.041670in}}{\pgfqpoint{2.216660in}{2.216660in}}%
\pgfusepath{clip}%
\pgfsetbuttcap%
\pgfsetroundjoin%
\definecolor{currentfill}{rgb}{0.699415,0.867117,0.175971}%
\pgfsetfillcolor{currentfill}%
\pgfsetlinewidth{0.000000pt}%
\definecolor{currentstroke}{rgb}{0.000000,0.000000,0.000000}%
\pgfsetstrokecolor{currentstroke}%
\pgfsetdash{}{0pt}%
\pgfpathmoveto{\pgfqpoint{1.162711in}{1.634800in}}%
\pgfpathlineto{\pgfqpoint{1.162241in}{1.632047in}}%
\pgfpathlineto{\pgfqpoint{1.161771in}{1.629187in}}%
\pgfpathlineto{\pgfqpoint{1.161302in}{1.626222in}}%
\pgfpathlineto{\pgfqpoint{1.160834in}{1.623153in}}%
\pgfpathlineto{\pgfqpoint{1.165665in}{1.623400in}}%
\pgfpathlineto{\pgfqpoint{1.170511in}{1.623575in}}%
\pgfpathlineto{\pgfqpoint{1.175366in}{1.623679in}}%
\pgfpathlineto{\pgfqpoint{1.180225in}{1.623711in}}%
\pgfpathlineto{\pgfqpoint{1.180218in}{1.626766in}}%
\pgfpathlineto{\pgfqpoint{1.180212in}{1.629717in}}%
\pgfpathlineto{\pgfqpoint{1.180205in}{1.632563in}}%
\pgfpathlineto{\pgfqpoint{1.180198in}{1.635302in}}%
\pgfpathlineto{\pgfqpoint{1.175816in}{1.635273in}}%
\pgfpathlineto{\pgfqpoint{1.171438in}{1.635180in}}%
\pgfpathlineto{\pgfqpoint{1.167069in}{1.635022in}}%
\pgfpathlineto{\pgfqpoint{1.162711in}{1.634800in}}%
\pgfpathclose%
\pgfusepath{fill}%
\end{pgfscope}%
\begin{pgfscope}%
\pgfpathrectangle{\pgfqpoint{0.041670in}{0.041670in}}{\pgfqpoint{2.216660in}{2.216660in}}%
\pgfusepath{clip}%
\pgfsetbuttcap%
\pgfsetroundjoin%
\definecolor{currentfill}{rgb}{0.699415,0.867117,0.175971}%
\pgfsetfillcolor{currentfill}%
\pgfsetlinewidth{0.000000pt}%
\definecolor{currentstroke}{rgb}{0.000000,0.000000,0.000000}%
\pgfsetstrokecolor{currentstroke}%
\pgfsetdash{}{0pt}%
\pgfpathmoveto{\pgfqpoint{1.180198in}{1.635302in}}%
\pgfpathlineto{\pgfqpoint{1.180205in}{1.632563in}}%
\pgfpathlineto{\pgfqpoint{1.180212in}{1.629717in}}%
\pgfpathlineto{\pgfqpoint{1.180218in}{1.626766in}}%
\pgfpathlineto{\pgfqpoint{1.180225in}{1.623711in}}%
\pgfpathlineto{\pgfqpoint{1.185084in}{1.623671in}}%
\pgfpathlineto{\pgfqpoint{1.189938in}{1.623559in}}%
\pgfpathlineto{\pgfqpoint{1.194782in}{1.623376in}}%
\pgfpathlineto{\pgfqpoint{1.199612in}{1.623121in}}%
\pgfpathlineto{\pgfqpoint{1.199130in}{1.626191in}}%
\pgfpathlineto{\pgfqpoint{1.198648in}{1.629156in}}%
\pgfpathlineto{\pgfqpoint{1.198165in}{1.632017in}}%
\pgfpathlineto{\pgfqpoint{1.197681in}{1.634771in}}%
\pgfpathlineto{\pgfqpoint{1.193326in}{1.635001in}}%
\pgfpathlineto{\pgfqpoint{1.188957in}{1.635166in}}%
\pgfpathlineto{\pgfqpoint{1.184580in}{1.635266in}}%
\pgfpathlineto{\pgfqpoint{1.180198in}{1.635302in}}%
\pgfpathclose%
\pgfusepath{fill}%
\end{pgfscope}%
\begin{pgfscope}%
\pgfpathrectangle{\pgfqpoint{0.041670in}{0.041670in}}{\pgfqpoint{2.216660in}{2.216660in}}%
\pgfusepath{clip}%
\pgfsetbuttcap%
\pgfsetroundjoin%
\definecolor{currentfill}{rgb}{0.233603,0.313828,0.543914}%
\pgfsetfillcolor{currentfill}%
\pgfsetlinewidth{0.000000pt}%
\definecolor{currentstroke}{rgb}{0.000000,0.000000,0.000000}%
\pgfsetstrokecolor{currentstroke}%
\pgfsetdash{}{0pt}%
\pgfpathmoveto{\pgfqpoint{0.491618in}{0.919968in}}%
\pgfpathlineto{\pgfqpoint{0.487514in}{0.930629in}}%
\pgfpathlineto{\pgfqpoint{0.483390in}{0.941741in}}%
\pgfpathlineto{\pgfqpoint{0.479246in}{0.953314in}}%
\pgfpathlineto{\pgfqpoint{0.475080in}{0.965353in}}%
\pgfpathlineto{\pgfqpoint{0.469160in}{0.976866in}}%
\pgfpathlineto{\pgfqpoint{0.463977in}{0.988450in}}%
\pgfpathlineto{\pgfqpoint{0.459534in}{1.000094in}}%
\pgfpathlineto{\pgfqpoint{0.455831in}{1.011784in}}%
\pgfpathlineto{\pgfqpoint{0.460083in}{0.999550in}}%
\pgfpathlineto{\pgfqpoint{0.464313in}{0.987780in}}%
\pgfpathlineto{\pgfqpoint{0.468523in}{0.976468in}}%
\pgfpathlineto{\pgfqpoint{0.472713in}{0.965605in}}%
\pgfpathlineto{\pgfqpoint{0.476357in}{0.954114in}}%
\pgfpathlineto{\pgfqpoint{0.480723in}{0.942669in}}%
\pgfpathlineto{\pgfqpoint{0.485811in}{0.931283in}}%
\pgfpathlineto{\pgfqpoint{0.491618in}{0.919968in}}%
\pgfpathclose%
\pgfusepath{fill}%
\end{pgfscope}%
\begin{pgfscope}%
\pgfpathrectangle{\pgfqpoint{0.041670in}{0.041670in}}{\pgfqpoint{2.216660in}{2.216660in}}%
\pgfusepath{clip}%
\pgfsetbuttcap%
\pgfsetroundjoin%
\definecolor{currentfill}{rgb}{0.248629,0.278775,0.534556}%
\pgfsetfillcolor{currentfill}%
\pgfsetlinewidth{0.000000pt}%
\definecolor{currentstroke}{rgb}{0.000000,0.000000,0.000000}%
\pgfsetstrokecolor{currentstroke}%
\pgfsetdash{}{0pt}%
\pgfpathmoveto{\pgfqpoint{1.573387in}{0.961064in}}%
\pgfpathlineto{\pgfqpoint{1.576813in}{0.953438in}}%
\pgfpathlineto{\pgfqpoint{1.580239in}{0.945890in}}%
\pgfpathlineto{\pgfqpoint{1.583665in}{0.938422in}}%
\pgfpathlineto{\pgfqpoint{1.587092in}{0.931039in}}%
\pgfpathlineto{\pgfqpoint{1.581518in}{0.924423in}}%
\pgfpathlineto{\pgfqpoint{1.575527in}{0.917895in}}%
\pgfpathlineto{\pgfqpoint{1.569126in}{0.911462in}}%
\pgfpathlineto{\pgfqpoint{1.562318in}{0.905132in}}%
\pgfpathlineto{\pgfqpoint{1.559089in}{0.912760in}}%
\pgfpathlineto{\pgfqpoint{1.555860in}{0.920473in}}%
\pgfpathlineto{\pgfqpoint{1.552631in}{0.928266in}}%
\pgfpathlineto{\pgfqpoint{1.549403in}{0.936137in}}%
\pgfpathlineto{\pgfqpoint{1.555992in}{0.942227in}}%
\pgfpathlineto{\pgfqpoint{1.562189in}{0.948416in}}%
\pgfpathlineto{\pgfqpoint{1.567989in}{0.954697in}}%
\pgfpathlineto{\pgfqpoint{1.573387in}{0.961064in}}%
\pgfpathclose%
\pgfusepath{fill}%
\end{pgfscope}%
\begin{pgfscope}%
\pgfpathrectangle{\pgfqpoint{0.041670in}{0.041670in}}{\pgfqpoint{2.216660in}{2.216660in}}%
\pgfusepath{clip}%
\pgfsetbuttcap%
\pgfsetroundjoin%
\definecolor{currentfill}{rgb}{0.699415,0.867117,0.175971}%
\pgfsetfillcolor{currentfill}%
\pgfsetlinewidth{0.000000pt}%
\definecolor{currentstroke}{rgb}{0.000000,0.000000,0.000000}%
\pgfsetstrokecolor{currentstroke}%
\pgfsetdash{}{0pt}%
\pgfpathmoveto{\pgfqpoint{1.145491in}{1.633273in}}%
\pgfpathlineto{\pgfqpoint{1.144551in}{1.630477in}}%
\pgfpathlineto{\pgfqpoint{1.143612in}{1.627575in}}%
\pgfpathlineto{\pgfqpoint{1.142674in}{1.624567in}}%
\pgfpathlineto{\pgfqpoint{1.141737in}{1.621456in}}%
\pgfpathlineto{\pgfqpoint{1.146467in}{1.621986in}}%
\pgfpathlineto{\pgfqpoint{1.151230in}{1.622446in}}%
\pgfpathlineto{\pgfqpoint{1.156020in}{1.622835in}}%
\pgfpathlineto{\pgfqpoint{1.160834in}{1.623153in}}%
\pgfpathlineto{\pgfqpoint{1.161302in}{1.626222in}}%
\pgfpathlineto{\pgfqpoint{1.161771in}{1.629187in}}%
\pgfpathlineto{\pgfqpoint{1.162241in}{1.632047in}}%
\pgfpathlineto{\pgfqpoint{1.162711in}{1.634800in}}%
\pgfpathlineto{\pgfqpoint{1.158371in}{1.634514in}}%
\pgfpathlineto{\pgfqpoint{1.154051in}{1.634163in}}%
\pgfpathlineto{\pgfqpoint{1.149757in}{1.633750in}}%
\pgfpathlineto{\pgfqpoint{1.145491in}{1.633273in}}%
\pgfpathclose%
\pgfusepath{fill}%
\end{pgfscope}%
\begin{pgfscope}%
\pgfpathrectangle{\pgfqpoint{0.041670in}{0.041670in}}{\pgfqpoint{2.216660in}{2.216660in}}%
\pgfusepath{clip}%
\pgfsetbuttcap%
\pgfsetroundjoin%
\definecolor{currentfill}{rgb}{0.699415,0.867117,0.175971}%
\pgfsetfillcolor{currentfill}%
\pgfsetlinewidth{0.000000pt}%
\definecolor{currentstroke}{rgb}{0.000000,0.000000,0.000000}%
\pgfsetstrokecolor{currentstroke}%
\pgfsetdash{}{0pt}%
\pgfpathmoveto{\pgfqpoint{1.197681in}{1.634771in}}%
\pgfpathlineto{\pgfqpoint{1.198165in}{1.632017in}}%
\pgfpathlineto{\pgfqpoint{1.198648in}{1.629156in}}%
\pgfpathlineto{\pgfqpoint{1.199130in}{1.626191in}}%
\pgfpathlineto{\pgfqpoint{1.199612in}{1.623121in}}%
\pgfpathlineto{\pgfqpoint{1.204423in}{1.622795in}}%
\pgfpathlineto{\pgfqpoint{1.209210in}{1.622398in}}%
\pgfpathlineto{\pgfqpoint{1.213970in}{1.621931in}}%
\pgfpathlineto{\pgfqpoint{1.218696in}{1.621393in}}%
\pgfpathlineto{\pgfqpoint{1.217746in}{1.624506in}}%
\pgfpathlineto{\pgfqpoint{1.216796in}{1.627515in}}%
\pgfpathlineto{\pgfqpoint{1.215844in}{1.630418in}}%
\pgfpathlineto{\pgfqpoint{1.214891in}{1.633216in}}%
\pgfpathlineto{\pgfqpoint{1.210629in}{1.633700in}}%
\pgfpathlineto{\pgfqpoint{1.206337in}{1.634121in}}%
\pgfpathlineto{\pgfqpoint{1.202020in}{1.634478in}}%
\pgfpathlineto{\pgfqpoint{1.197681in}{1.634771in}}%
\pgfpathclose%
\pgfusepath{fill}%
\end{pgfscope}%
\begin{pgfscope}%
\pgfpathrectangle{\pgfqpoint{0.041670in}{0.041670in}}{\pgfqpoint{2.216660in}{2.216660in}}%
\pgfusepath{clip}%
\pgfsetbuttcap%
\pgfsetroundjoin%
\definecolor{currentfill}{rgb}{0.263663,0.237631,0.518762}%
\pgfsetfillcolor{currentfill}%
\pgfsetlinewidth{0.000000pt}%
\definecolor{currentstroke}{rgb}{0.000000,0.000000,0.000000}%
\pgfsetstrokecolor{currentstroke}%
\pgfsetdash{}{0pt}%
\pgfpathmoveto{\pgfqpoint{0.803979in}{0.899598in}}%
\pgfpathlineto{\pgfqpoint{0.800800in}{0.892005in}}%
\pgfpathlineto{\pgfqpoint{0.797622in}{0.884503in}}%
\pgfpathlineto{\pgfqpoint{0.794442in}{0.877097in}}%
\pgfpathlineto{\pgfqpoint{0.791262in}{0.869789in}}%
\pgfpathlineto{\pgfqpoint{0.783868in}{0.876259in}}%
\pgfpathlineto{\pgfqpoint{0.776888in}{0.882842in}}%
\pgfpathlineto{\pgfqpoint{0.770330in}{0.889530in}}%
\pgfpathlineto{\pgfqpoint{0.764198in}{0.896316in}}%
\pgfpathlineto{\pgfqpoint{0.767587in}{0.903381in}}%
\pgfpathlineto{\pgfqpoint{0.770976in}{0.910544in}}%
\pgfpathlineto{\pgfqpoint{0.774364in}{0.917803in}}%
\pgfpathlineto{\pgfqpoint{0.777752in}{0.925154in}}%
\pgfpathlineto{\pgfqpoint{0.783696in}{0.918615in}}%
\pgfpathlineto{\pgfqpoint{0.790053in}{0.912172in}}%
\pgfpathlineto{\pgfqpoint{0.796815in}{0.905830in}}%
\pgfpathlineto{\pgfqpoint{0.803979in}{0.899598in}}%
\pgfpathclose%
\pgfusepath{fill}%
\end{pgfscope}%
\begin{pgfscope}%
\pgfpathrectangle{\pgfqpoint{0.041670in}{0.041670in}}{\pgfqpoint{2.216660in}{2.216660in}}%
\pgfusepath{clip}%
\pgfsetbuttcap%
\pgfsetroundjoin%
\definecolor{currentfill}{rgb}{0.195860,0.395433,0.555276}%
\pgfsetfillcolor{currentfill}%
\pgfsetlinewidth{0.000000pt}%
\definecolor{currentstroke}{rgb}{0.000000,0.000000,0.000000}%
\pgfsetstrokecolor{currentstroke}%
\pgfsetdash{}{0pt}%
\pgfpathmoveto{\pgfqpoint{0.831981in}{1.052413in}}%
\pgfpathlineto{\pgfqpoint{0.828587in}{1.044060in}}%
\pgfpathlineto{\pgfqpoint{0.825194in}{1.035745in}}%
\pgfpathlineto{\pgfqpoint{0.821801in}{1.027472in}}%
\pgfpathlineto{\pgfqpoint{0.818410in}{1.019243in}}%
\pgfpathlineto{\pgfqpoint{0.813407in}{1.025121in}}%
\pgfpathlineto{\pgfqpoint{0.808782in}{1.031073in}}%
\pgfpathlineto{\pgfqpoint{0.804538in}{1.037091in}}%
\pgfpathlineto{\pgfqpoint{0.800680in}{1.043169in}}%
\pgfpathlineto{\pgfqpoint{0.804226in}{1.051147in}}%
\pgfpathlineto{\pgfqpoint{0.807773in}{1.059169in}}%
\pgfpathlineto{\pgfqpoint{0.811321in}{1.067234in}}%
\pgfpathlineto{\pgfqpoint{0.814871in}{1.075336in}}%
\pgfpathlineto{\pgfqpoint{0.818597in}{1.069513in}}%
\pgfpathlineto{\pgfqpoint{0.822693in}{1.063747in}}%
\pgfpathlineto{\pgfqpoint{0.827155in}{1.058045in}}%
\pgfpathlineto{\pgfqpoint{0.831981in}{1.052413in}}%
\pgfpathclose%
\pgfusepath{fill}%
\end{pgfscope}%
\begin{pgfscope}%
\pgfpathrectangle{\pgfqpoint{0.041670in}{0.041670in}}{\pgfqpoint{2.216660in}{2.216660in}}%
\pgfusepath{clip}%
\pgfsetbuttcap%
\pgfsetroundjoin%
\definecolor{currentfill}{rgb}{0.276194,0.190074,0.493001}%
\pgfsetfillcolor{currentfill}%
\pgfsetlinewidth{0.000000pt}%
\definecolor{currentstroke}{rgb}{0.000000,0.000000,0.000000}%
\pgfsetstrokecolor{currentstroke}%
\pgfsetdash{}{0pt}%
\pgfpathmoveto{\pgfqpoint{1.841126in}{0.859760in}}%
\pgfpathlineto{\pgfqpoint{1.845111in}{0.867113in}}%
\pgfpathlineto{\pgfqpoint{1.849112in}{0.874861in}}%
\pgfpathlineto{\pgfqpoint{1.853130in}{0.883010in}}%
\pgfpathlineto{\pgfqpoint{1.857165in}{0.891568in}}%
\pgfpathlineto{\pgfqpoint{1.851392in}{0.880469in}}%
\pgfpathlineto{\pgfqpoint{1.844919in}{0.869454in}}%
\pgfpathlineto{\pgfqpoint{1.837751in}{0.858535in}}%
\pgfpathlineto{\pgfqpoint{1.829892in}{0.847724in}}%
\pgfpathlineto{\pgfqpoint{1.825996in}{0.839377in}}%
\pgfpathlineto{\pgfqpoint{1.822117in}{0.831441in}}%
\pgfpathlineto{\pgfqpoint{1.818254in}{0.823908in}}%
\pgfpathlineto{\pgfqpoint{1.814408in}{0.816772in}}%
\pgfpathlineto{\pgfqpoint{1.822102in}{0.827371in}}%
\pgfpathlineto{\pgfqpoint{1.829123in}{0.838076in}}%
\pgfpathlineto{\pgfqpoint{1.835465in}{0.848876in}}%
\pgfpathlineto{\pgfqpoint{1.841126in}{0.859760in}}%
\pgfpathclose%
\pgfusepath{fill}%
\end{pgfscope}%
\begin{pgfscope}%
\pgfpathrectangle{\pgfqpoint{0.041670in}{0.041670in}}{\pgfqpoint{2.216660in}{2.216660in}}%
\pgfusepath{clip}%
\pgfsetbuttcap%
\pgfsetroundjoin%
\definecolor{currentfill}{rgb}{0.147607,0.511733,0.557049}%
\pgfsetfillcolor{currentfill}%
\pgfsetlinewidth{0.000000pt}%
\definecolor{currentstroke}{rgb}{0.000000,0.000000,0.000000}%
\pgfsetstrokecolor{currentstroke}%
\pgfsetdash{}{0pt}%
\pgfpathmoveto{\pgfqpoint{0.857595in}{1.174456in}}%
\pgfpathlineto{\pgfqpoint{0.854024in}{1.166129in}}%
\pgfpathlineto{\pgfqpoint{0.850455in}{1.157804in}}%
\pgfpathlineto{\pgfqpoint{0.846889in}{1.149484in}}%
\pgfpathlineto{\pgfqpoint{0.843324in}{1.141173in}}%
\pgfpathlineto{\pgfqpoint{0.840210in}{1.146540in}}%
\pgfpathlineto{\pgfqpoint{0.837442in}{1.151950in}}%
\pgfpathlineto{\pgfqpoint{0.835022in}{1.157397in}}%
\pgfpathlineto{\pgfqpoint{0.832951in}{1.162874in}}%
\pgfpathlineto{\pgfqpoint{0.836614in}{1.170931in}}%
\pgfpathlineto{\pgfqpoint{0.840280in}{1.178996in}}%
\pgfpathlineto{\pgfqpoint{0.843947in}{1.187067in}}%
\pgfpathlineto{\pgfqpoint{0.847618in}{1.195141in}}%
\pgfpathlineto{\pgfqpoint{0.849612in}{1.189919in}}%
\pgfpathlineto{\pgfqpoint{0.851941in}{1.184727in}}%
\pgfpathlineto{\pgfqpoint{0.854602in}{1.179571in}}%
\pgfpathlineto{\pgfqpoint{0.857595in}{1.174456in}}%
\pgfpathclose%
\pgfusepath{fill}%
\end{pgfscope}%
\begin{pgfscope}%
\pgfpathrectangle{\pgfqpoint{0.041670in}{0.041670in}}{\pgfqpoint{2.216660in}{2.216660in}}%
\pgfusepath{clip}%
\pgfsetbuttcap%
\pgfsetroundjoin%
\definecolor{currentfill}{rgb}{0.220124,0.725509,0.466226}%
\pgfsetfillcolor{currentfill}%
\pgfsetlinewidth{0.000000pt}%
\definecolor{currentstroke}{rgb}{0.000000,0.000000,0.000000}%
\pgfsetstrokecolor{currentstroke}%
\pgfsetdash{}{0pt}%
\pgfpathmoveto{\pgfqpoint{0.933275in}{1.412944in}}%
\pgfpathlineto{\pgfqpoint{0.929508in}{1.406085in}}%
\pgfpathlineto{\pgfqpoint{0.925744in}{1.399169in}}%
\pgfpathlineto{\pgfqpoint{0.921984in}{1.392198in}}%
\pgfpathlineto{\pgfqpoint{0.918227in}{1.385175in}}%
\pgfpathlineto{\pgfqpoint{0.918984in}{1.389177in}}%
\pgfpathlineto{\pgfqpoint{0.920001in}{1.393164in}}%
\pgfpathlineto{\pgfqpoint{0.921274in}{1.397131in}}%
\pgfpathlineto{\pgfqpoint{0.922803in}{1.401074in}}%
\pgfpathlineto{\pgfqpoint{0.926486in}{1.407853in}}%
\pgfpathlineto{\pgfqpoint{0.930172in}{1.414580in}}%
\pgfpathlineto{\pgfqpoint{0.933862in}{1.421253in}}%
\pgfpathlineto{\pgfqpoint{0.937556in}{1.427869in}}%
\pgfpathlineto{\pgfqpoint{0.936123in}{1.424167in}}%
\pgfpathlineto{\pgfqpoint{0.934931in}{1.420443in}}%
\pgfpathlineto{\pgfqpoint{0.933981in}{1.416701in}}%
\pgfpathlineto{\pgfqpoint{0.933275in}{1.412944in}}%
\pgfpathclose%
\pgfusepath{fill}%
\end{pgfscope}%
\begin{pgfscope}%
\pgfpathrectangle{\pgfqpoint{0.041670in}{0.041670in}}{\pgfqpoint{2.216660in}{2.216660in}}%
\pgfusepath{clip}%
\pgfsetbuttcap%
\pgfsetroundjoin%
\definecolor{currentfill}{rgb}{0.636902,0.856542,0.216620}%
\pgfsetfillcolor{currentfill}%
\pgfsetlinewidth{0.000000pt}%
\definecolor{currentstroke}{rgb}{0.000000,0.000000,0.000000}%
\pgfsetstrokecolor{currentstroke}%
\pgfsetdash{}{0pt}%
\pgfpathmoveto{\pgfqpoint{1.267244in}{1.611039in}}%
\pgfpathlineto{\pgfqpoint{1.269385in}{1.607565in}}%
\pgfpathlineto{\pgfqpoint{1.271523in}{1.603991in}}%
\pgfpathlineto{\pgfqpoint{1.273659in}{1.600317in}}%
\pgfpathlineto{\pgfqpoint{1.275792in}{1.596546in}}%
\pgfpathlineto{\pgfqpoint{1.280174in}{1.595091in}}%
\pgfpathlineto{\pgfqpoint{1.284458in}{1.593572in}}%
\pgfpathlineto{\pgfqpoint{1.288641in}{1.591989in}}%
\pgfpathlineto{\pgfqpoint{1.292719in}{1.590345in}}%
\pgfpathlineto{\pgfqpoint{1.290207in}{1.594258in}}%
\pgfpathlineto{\pgfqpoint{1.287691in}{1.598073in}}%
\pgfpathlineto{\pgfqpoint{1.285173in}{1.601789in}}%
\pgfpathlineto{\pgfqpoint{1.282652in}{1.605405in}}%
\pgfpathlineto{\pgfqpoint{1.278941in}{1.606899in}}%
\pgfpathlineto{\pgfqpoint{1.275133in}{1.608337in}}%
\pgfpathlineto{\pgfqpoint{1.271233in}{1.609717in}}%
\pgfpathlineto{\pgfqpoint{1.267244in}{1.611039in}}%
\pgfpathclose%
\pgfusepath{fill}%
\end{pgfscope}%
\begin{pgfscope}%
\pgfpathrectangle{\pgfqpoint{0.041670in}{0.041670in}}{\pgfqpoint{2.216660in}{2.216660in}}%
\pgfusepath{clip}%
\pgfsetbuttcap%
\pgfsetroundjoin%
\definecolor{currentfill}{rgb}{0.699415,0.867117,0.175971}%
\pgfsetfillcolor{currentfill}%
\pgfsetlinewidth{0.000000pt}%
\definecolor{currentstroke}{rgb}{0.000000,0.000000,0.000000}%
\pgfsetstrokecolor{currentstroke}%
\pgfsetdash{}{0pt}%
\pgfpathmoveto{\pgfqpoint{1.214891in}{1.633216in}}%
\pgfpathlineto{\pgfqpoint{1.215844in}{1.630418in}}%
\pgfpathlineto{\pgfqpoint{1.216796in}{1.627515in}}%
\pgfpathlineto{\pgfqpoint{1.217746in}{1.624506in}}%
\pgfpathlineto{\pgfqpoint{1.218696in}{1.621393in}}%
\pgfpathlineto{\pgfqpoint{1.223385in}{1.620786in}}%
\pgfpathlineto{\pgfqpoint{1.228032in}{1.620109in}}%
\pgfpathlineto{\pgfqpoint{1.232633in}{1.619365in}}%
\pgfpathlineto{\pgfqpoint{1.231341in}{1.622528in}}%
\pgfpathlineto{\pgfqpoint{1.230048in}{1.625587in}}%
\pgfpathlineto{\pgfqpoint{1.228753in}{1.628542in}}%
\pgfpathlineto{\pgfqpoint{1.227457in}{1.631390in}}%
\pgfpathlineto{\pgfqpoint{1.223309in}{1.632060in}}%
\pgfpathlineto{\pgfqpoint{1.219119in}{1.632669in}}%
\pgfpathlineto{\pgfqpoint{1.214891in}{1.633216in}}%
\pgfpathclose%
\pgfusepath{fill}%
\end{pgfscope}%
\begin{pgfscope}%
\pgfpathrectangle{\pgfqpoint{0.041670in}{0.041670in}}{\pgfqpoint{2.216660in}{2.216660in}}%
\pgfusepath{clip}%
\pgfsetbuttcap%
\pgfsetroundjoin%
\definecolor{currentfill}{rgb}{0.699415,0.867117,0.175971}%
\pgfsetfillcolor{currentfill}%
\pgfsetlinewidth{0.000000pt}%
\definecolor{currentstroke}{rgb}{0.000000,0.000000,0.000000}%
\pgfsetstrokecolor{currentstroke}%
\pgfsetdash{}{0pt}%
\pgfpathmoveto{\pgfqpoint{1.128804in}{1.630743in}}%
\pgfpathlineto{\pgfqpoint{1.127408in}{1.627877in}}%
\pgfpathlineto{\pgfqpoint{1.126013in}{1.624904in}}%
\pgfpathlineto{\pgfqpoint{1.124621in}{1.621827in}}%
\pgfpathlineto{\pgfqpoint{1.123230in}{1.618646in}}%
\pgfpathlineto{\pgfqpoint{1.127785in}{1.619451in}}%
\pgfpathlineto{\pgfqpoint{1.132391in}{1.620188in}}%
\pgfpathlineto{\pgfqpoint{1.137044in}{1.620857in}}%
\pgfpathlineto{\pgfqpoint{1.141737in}{1.621456in}}%
\pgfpathlineto{\pgfqpoint{1.142674in}{1.624567in}}%
\pgfpathlineto{\pgfqpoint{1.143612in}{1.627575in}}%
\pgfpathlineto{\pgfqpoint{1.144551in}{1.630477in}}%
\pgfpathlineto{\pgfqpoint{1.145491in}{1.633273in}}%
\pgfpathlineto{\pgfqpoint{1.141259in}{1.632733in}}%
\pgfpathlineto{\pgfqpoint{1.137064in}{1.632131in}}%
\pgfpathlineto{\pgfqpoint{1.132911in}{1.631467in}}%
\pgfpathlineto{\pgfqpoint{1.128804in}{1.630743in}}%
\pgfpathclose%
\pgfusepath{fill}%
\end{pgfscope}%
\begin{pgfscope}%
\pgfpathrectangle{\pgfqpoint{0.041670in}{0.041670in}}{\pgfqpoint{2.216660in}{2.216660in}}%
\pgfusepath{clip}%
\pgfsetbuttcap%
\pgfsetroundjoin%
\definecolor{currentfill}{rgb}{0.412913,0.803041,0.357269}%
\pgfsetfillcolor{currentfill}%
\pgfsetlinewidth{0.000000pt}%
\definecolor{currentstroke}{rgb}{0.000000,0.000000,0.000000}%
\pgfsetstrokecolor{currentstroke}%
\pgfsetdash{}{0pt}%
\pgfpathmoveto{\pgfqpoint{1.359171in}{1.527385in}}%
\pgfpathlineto{\pgfqpoint{1.362542in}{1.522115in}}%
\pgfpathlineto{\pgfqpoint{1.365910in}{1.516764in}}%
\pgfpathlineto{\pgfqpoint{1.369273in}{1.511334in}}%
\pgfpathlineto{\pgfqpoint{1.372634in}{1.505827in}}%
\pgfpathlineto{\pgfqpoint{1.375668in}{1.502905in}}%
\pgfpathlineto{\pgfqpoint{1.378511in}{1.499937in}}%
\pgfpathlineto{\pgfqpoint{1.381160in}{1.496927in}}%
\pgfpathlineto{\pgfqpoint{1.383612in}{1.493876in}}%
\pgfpathlineto{\pgfqpoint{1.380054in}{1.499602in}}%
\pgfpathlineto{\pgfqpoint{1.376492in}{1.505250in}}%
\pgfpathlineto{\pgfqpoint{1.372927in}{1.510819in}}%
\pgfpathlineto{\pgfqpoint{1.369358in}{1.516307in}}%
\pgfpathlineto{\pgfqpoint{1.367084in}{1.519135in}}%
\pgfpathlineto{\pgfqpoint{1.364626in}{1.521925in}}%
\pgfpathlineto{\pgfqpoint{1.361988in}{1.524676in}}%
\pgfpathlineto{\pgfqpoint{1.359171in}{1.527385in}}%
\pgfpathclose%
\pgfusepath{fill}%
\end{pgfscope}%
\begin{pgfscope}%
\pgfpathrectangle{\pgfqpoint{0.041670in}{0.041670in}}{\pgfqpoint{2.216660in}{2.216660in}}%
\pgfusepath{clip}%
\pgfsetbuttcap%
\pgfsetroundjoin%
\definecolor{currentfill}{rgb}{0.487026,0.823929,0.312321}%
\pgfsetfillcolor{currentfill}%
\pgfsetlinewidth{0.000000pt}%
\definecolor{currentstroke}{rgb}{0.000000,0.000000,0.000000}%
\pgfsetstrokecolor{currentstroke}%
\pgfsetdash{}{0pt}%
\pgfpathmoveto{\pgfqpoint{1.333664in}{1.557175in}}%
\pgfpathlineto{\pgfqpoint{1.336800in}{1.552449in}}%
\pgfpathlineto{\pgfqpoint{1.339932in}{1.547635in}}%
\pgfpathlineto{\pgfqpoint{1.343060in}{1.542735in}}%
\pgfpathlineto{\pgfqpoint{1.346185in}{1.537750in}}%
\pgfpathlineto{\pgfqpoint{1.349683in}{1.535235in}}%
\pgfpathlineto{\pgfqpoint{1.353015in}{1.532667in}}%
\pgfpathlineto{\pgfqpoint{1.356179in}{1.530050in}}%
\pgfpathlineto{\pgfqpoint{1.359171in}{1.527385in}}%
\pgfpathlineto{\pgfqpoint{1.355797in}{1.532573in}}%
\pgfpathlineto{\pgfqpoint{1.352419in}{1.537676in}}%
\pgfpathlineto{\pgfqpoint{1.349037in}{1.542693in}}%
\pgfpathlineto{\pgfqpoint{1.345652in}{1.547621in}}%
\pgfpathlineto{\pgfqpoint{1.342891in}{1.550077in}}%
\pgfpathlineto{\pgfqpoint{1.339971in}{1.552489in}}%
\pgfpathlineto{\pgfqpoint{1.336894in}{1.554856in}}%
\pgfpathlineto{\pgfqpoint{1.333664in}{1.557175in}}%
\pgfpathclose%
\pgfusepath{fill}%
\end{pgfscope}%
\begin{pgfscope}%
\pgfpathrectangle{\pgfqpoint{0.041670in}{0.041670in}}{\pgfqpoint{2.216660in}{2.216660in}}%
\pgfusepath{clip}%
\pgfsetbuttcap%
\pgfsetroundjoin%
\definecolor{currentfill}{rgb}{0.179019,0.433756,0.557430}%
\pgfsetfillcolor{currentfill}%
\pgfsetlinewidth{0.000000pt}%
\definecolor{currentstroke}{rgb}{0.000000,0.000000,0.000000}%
\pgfsetstrokecolor{currentstroke}%
\pgfsetdash{}{0pt}%
\pgfpathmoveto{\pgfqpoint{1.533718in}{1.113064in}}%
\pgfpathlineto{\pgfqpoint{1.537301in}{1.104895in}}%
\pgfpathlineto{\pgfqpoint{1.540881in}{1.096752in}}%
\pgfpathlineto{\pgfqpoint{1.544460in}{1.088638in}}%
\pgfpathlineto{\pgfqpoint{1.548038in}{1.080557in}}%
\pgfpathlineto{\pgfqpoint{1.544643in}{1.074687in}}%
\pgfpathlineto{\pgfqpoint{1.540876in}{1.068869in}}%
\pgfpathlineto{\pgfqpoint{1.536739in}{1.063110in}}%
\pgfpathlineto{\pgfqpoint{1.532236in}{1.057416in}}%
\pgfpathlineto{\pgfqpoint{1.528801in}{1.065749in}}%
\pgfpathlineto{\pgfqpoint{1.525364in}{1.074114in}}%
\pgfpathlineto{\pgfqpoint{1.521927in}{1.082509in}}%
\pgfpathlineto{\pgfqpoint{1.518488in}{1.090929in}}%
\pgfpathlineto{\pgfqpoint{1.522826in}{1.096375in}}%
\pgfpathlineto{\pgfqpoint{1.526813in}{1.101883in}}%
\pgfpathlineto{\pgfqpoint{1.530444in}{1.107448in}}%
\pgfpathlineto{\pgfqpoint{1.533718in}{1.113064in}}%
\pgfpathclose%
\pgfusepath{fill}%
\end{pgfscope}%
\begin{pgfscope}%
\pgfpathrectangle{\pgfqpoint{0.041670in}{0.041670in}}{\pgfqpoint{2.216660in}{2.216660in}}%
\pgfusepath{clip}%
\pgfsetbuttcap%
\pgfsetroundjoin%
\definecolor{currentfill}{rgb}{0.636902,0.856542,0.216620}%
\pgfsetfillcolor{currentfill}%
\pgfsetlinewidth{0.000000pt}%
\definecolor{currentstroke}{rgb}{0.000000,0.000000,0.000000}%
\pgfsetstrokecolor{currentstroke}%
\pgfsetdash{}{0pt}%
\pgfpathmoveto{\pgfqpoint{1.074042in}{1.604031in}}%
\pgfpathlineto{\pgfqpoint{1.071441in}{1.600381in}}%
\pgfpathlineto{\pgfqpoint{1.068844in}{1.596630in}}%
\pgfpathlineto{\pgfqpoint{1.066249in}{1.592780in}}%
\pgfpathlineto{\pgfqpoint{1.063657in}{1.588833in}}%
\pgfpathlineto{\pgfqpoint{1.067638in}{1.590531in}}%
\pgfpathlineto{\pgfqpoint{1.071728in}{1.592168in}}%
\pgfpathlineto{\pgfqpoint{1.075923in}{1.593744in}}%
\pgfpathlineto{\pgfqpoint{1.080218in}{1.595256in}}%
\pgfpathlineto{\pgfqpoint{1.082439in}{1.599057in}}%
\pgfpathlineto{\pgfqpoint{1.084662in}{1.602760in}}%
\pgfpathlineto{\pgfqpoint{1.086888in}{1.606364in}}%
\pgfpathlineto{\pgfqpoint{1.089116in}{1.609867in}}%
\pgfpathlineto{\pgfqpoint{1.085206in}{1.608493in}}%
\pgfpathlineto{\pgfqpoint{1.081387in}{1.607061in}}%
\pgfpathlineto{\pgfqpoint{1.077665in}{1.605573in}}%
\pgfpathlineto{\pgfqpoint{1.074042in}{1.604031in}}%
\pgfpathclose%
\pgfusepath{fill}%
\end{pgfscope}%
\begin{pgfscope}%
\pgfpathrectangle{\pgfqpoint{0.041670in}{0.041670in}}{\pgfqpoint{2.216660in}{2.216660in}}%
\pgfusepath{clip}%
\pgfsetbuttcap%
\pgfsetroundjoin%
\definecolor{currentfill}{rgb}{0.120081,0.622161,0.534946}%
\pgfsetfillcolor{currentfill}%
\pgfsetlinewidth{0.000000pt}%
\definecolor{currentstroke}{rgb}{0.000000,0.000000,0.000000}%
\pgfsetstrokecolor{currentstroke}%
\pgfsetdash{}{0pt}%
\pgfpathmoveto{\pgfqpoint{1.472452in}{1.313304in}}%
\pgfpathlineto{\pgfqpoint{1.476197in}{1.305717in}}%
\pgfpathlineto{\pgfqpoint{1.479940in}{1.298101in}}%
\pgfpathlineto{\pgfqpoint{1.483679in}{1.290459in}}%
\pgfpathlineto{\pgfqpoint{1.487415in}{1.282793in}}%
\pgfpathlineto{\pgfqpoint{1.487077in}{1.278012in}}%
\pgfpathlineto{\pgfqpoint{1.486431in}{1.273235in}}%
\pgfpathlineto{\pgfqpoint{1.485479in}{1.268467in}}%
\pgfpathlineto{\pgfqpoint{1.484219in}{1.263713in}}%
\pgfpathlineto{\pgfqpoint{1.480511in}{1.271633in}}%
\pgfpathlineto{\pgfqpoint{1.476800in}{1.279529in}}%
\pgfpathlineto{\pgfqpoint{1.473087in}{1.287398in}}%
\pgfpathlineto{\pgfqpoint{1.469371in}{1.295237in}}%
\pgfpathlineto{\pgfqpoint{1.470580in}{1.299738in}}%
\pgfpathlineto{\pgfqpoint{1.471496in}{1.304253in}}%
\pgfpathlineto{\pgfqpoint{1.472120in}{1.308776in}}%
\pgfpathlineto{\pgfqpoint{1.472452in}{1.313304in}}%
\pgfpathclose%
\pgfusepath{fill}%
\end{pgfscope}%
\begin{pgfscope}%
\pgfpathrectangle{\pgfqpoint{0.041670in}{0.041670in}}{\pgfqpoint{2.216660in}{2.216660in}}%
\pgfusepath{clip}%
\pgfsetbuttcap%
\pgfsetroundjoin%
\definecolor{currentfill}{rgb}{0.272594,0.025563,0.353093}%
\pgfsetfillcolor{currentfill}%
\pgfsetlinewidth{0.000000pt}%
\definecolor{currentstroke}{rgb}{0.000000,0.000000,0.000000}%
\pgfsetstrokecolor{currentstroke}%
\pgfsetdash{}{0pt}%
\pgfpathmoveto{\pgfqpoint{0.648204in}{0.705759in}}%
\pgfpathlineto{\pgfqpoint{0.644815in}{0.707119in}}%
\pgfpathlineto{\pgfqpoint{0.641416in}{0.708780in}}%
\pgfpathlineto{\pgfqpoint{0.638007in}{0.710748in}}%
\pgfpathlineto{\pgfqpoint{0.634588in}{0.713028in}}%
\pgfpathlineto{\pgfqpoint{0.624414in}{0.722300in}}%
\pgfpathlineto{\pgfqpoint{0.614832in}{0.731726in}}%
\pgfpathlineto{\pgfqpoint{0.605851in}{0.741295in}}%
\pgfpathlineto{\pgfqpoint{0.597477in}{0.750998in}}%
\pgfpathlineto{\pgfqpoint{0.601110in}{0.748493in}}%
\pgfpathlineto{\pgfqpoint{0.604733in}{0.746299in}}%
\pgfpathlineto{\pgfqpoint{0.608345in}{0.744411in}}%
\pgfpathlineto{\pgfqpoint{0.611947in}{0.742822in}}%
\pgfpathlineto{\pgfqpoint{0.620131in}{0.733350in}}%
\pgfpathlineto{\pgfqpoint{0.628907in}{0.724009in}}%
\pgfpathlineto{\pgfqpoint{0.638267in}{0.714808in}}%
\pgfpathlineto{\pgfqpoint{0.648204in}{0.705759in}}%
\pgfpathclose%
\pgfusepath{fill}%
\end{pgfscope}%
\begin{pgfscope}%
\pgfpathrectangle{\pgfqpoint{0.041670in}{0.041670in}}{\pgfqpoint{2.216660in}{2.216660in}}%
\pgfusepath{clip}%
\pgfsetbuttcap%
\pgfsetroundjoin%
\definecolor{currentfill}{rgb}{0.166383,0.690856,0.496502}%
\pgfsetfillcolor{currentfill}%
\pgfsetlinewidth{0.000000pt}%
\definecolor{currentstroke}{rgb}{0.000000,0.000000,0.000000}%
\pgfsetstrokecolor{currentstroke}%
\pgfsetdash{}{0pt}%
\pgfpathmoveto{\pgfqpoint{1.441022in}{1.388733in}}%
\pgfpathlineto{\pgfqpoint{1.444765in}{1.381714in}}%
\pgfpathlineto{\pgfqpoint{1.448503in}{1.374647in}}%
\pgfpathlineto{\pgfqpoint{1.452239in}{1.367534in}}%
\pgfpathlineto{\pgfqpoint{1.455971in}{1.360378in}}%
\pgfpathlineto{\pgfqpoint{1.456751in}{1.356127in}}%
\pgfpathlineto{\pgfqpoint{1.457256in}{1.351863in}}%
\pgfpathlineto{\pgfqpoint{1.457486in}{1.347590in}}%
\pgfpathlineto{\pgfqpoint{1.457438in}{1.343314in}}%
\pgfpathlineto{\pgfqpoint{1.453677in}{1.350719in}}%
\pgfpathlineto{\pgfqpoint{1.449913in}{1.358080in}}%
\pgfpathlineto{\pgfqpoint{1.446145in}{1.365396in}}%
\pgfpathlineto{\pgfqpoint{1.442374in}{1.372663in}}%
\pgfpathlineto{\pgfqpoint{1.442428in}{1.376690in}}%
\pgfpathlineto{\pgfqpoint{1.442220in}{1.380713in}}%
\pgfpathlineto{\pgfqpoint{1.441751in}{1.384729in}}%
\pgfpathlineto{\pgfqpoint{1.441022in}{1.388733in}}%
\pgfpathclose%
\pgfusepath{fill}%
\end{pgfscope}%
\begin{pgfscope}%
\pgfpathrectangle{\pgfqpoint{0.041670in}{0.041670in}}{\pgfqpoint{2.216660in}{2.216660in}}%
\pgfusepath{clip}%
\pgfsetbuttcap%
\pgfsetroundjoin%
\definecolor{currentfill}{rgb}{0.344074,0.780029,0.397381}%
\pgfsetfillcolor{currentfill}%
\pgfsetlinewidth{0.000000pt}%
\definecolor{currentstroke}{rgb}{0.000000,0.000000,0.000000}%
\pgfsetstrokecolor{currentstroke}%
\pgfsetdash{}{0pt}%
\pgfpathmoveto{\pgfqpoint{1.383612in}{1.493876in}}%
\pgfpathlineto{\pgfqpoint{1.387167in}{1.488075in}}%
\pgfpathlineto{\pgfqpoint{1.390718in}{1.482201in}}%
\pgfpathlineto{\pgfqpoint{1.394265in}{1.476256in}}%
\pgfpathlineto{\pgfqpoint{1.397809in}{1.470241in}}%
\pgfpathlineto{\pgfqpoint{1.400225in}{1.466926in}}%
\pgfpathlineto{\pgfqpoint{1.402425in}{1.463576in}}%
\pgfpathlineto{\pgfqpoint{1.404406in}{1.460191in}}%
\pgfpathlineto{\pgfqpoint{1.406167in}{1.456777in}}%
\pgfpathlineto{\pgfqpoint{1.402480in}{1.463023in}}%
\pgfpathlineto{\pgfqpoint{1.398789in}{1.469200in}}%
\pgfpathlineto{\pgfqpoint{1.395095in}{1.475305in}}%
\pgfpathlineto{\pgfqpoint{1.391398in}{1.481336in}}%
\pgfpathlineto{\pgfqpoint{1.389759in}{1.484516in}}%
\pgfpathlineto{\pgfqpoint{1.387914in}{1.487668in}}%
\pgfpathlineto{\pgfqpoint{1.385864in}{1.490789in}}%
\pgfpathlineto{\pgfqpoint{1.383612in}{1.493876in}}%
\pgfpathclose%
\pgfusepath{fill}%
\end{pgfscope}%
\begin{pgfscope}%
\pgfpathrectangle{\pgfqpoint{0.041670in}{0.041670in}}{\pgfqpoint{2.216660in}{2.216660in}}%
\pgfusepath{clip}%
\pgfsetbuttcap%
\pgfsetroundjoin%
\definecolor{currentfill}{rgb}{0.277941,0.056324,0.381191}%
\pgfsetfillcolor{currentfill}%
\pgfsetlinewidth{0.000000pt}%
\definecolor{currentstroke}{rgb}{0.000000,0.000000,0.000000}%
\pgfsetstrokecolor{currentstroke}%
\pgfsetdash{}{0pt}%
\pgfpathmoveto{\pgfqpoint{1.769362in}{0.759726in}}%
\pgfpathlineto{\pgfqpoint{1.773045in}{0.762599in}}%
\pgfpathlineto{\pgfqpoint{1.776740in}{0.765794in}}%
\pgfpathlineto{\pgfqpoint{1.780448in}{0.769317in}}%
\pgfpathlineto{\pgfqpoint{1.784167in}{0.773174in}}%
\pgfpathlineto{\pgfqpoint{1.776161in}{0.763134in}}%
\pgfpathlineto{\pgfqpoint{1.767527in}{0.753219in}}%
\pgfpathlineto{\pgfqpoint{1.758269in}{0.743441in}}%
\pgfpathlineto{\pgfqpoint{1.748396in}{0.733811in}}%
\pgfpathlineto{\pgfqpoint{1.744878in}{0.730177in}}%
\pgfpathlineto{\pgfqpoint{1.741372in}{0.726878in}}%
\pgfpathlineto{\pgfqpoint{1.737878in}{0.723908in}}%
\pgfpathlineto{\pgfqpoint{1.734395in}{0.721262in}}%
\pgfpathlineto{\pgfqpoint{1.744042in}{0.730671in}}%
\pgfpathlineto{\pgfqpoint{1.753091in}{0.740225in}}%
\pgfpathlineto{\pgfqpoint{1.761533in}{0.749914in}}%
\pgfpathlineto{\pgfqpoint{1.769362in}{0.759726in}}%
\pgfpathclose%
\pgfusepath{fill}%
\end{pgfscope}%
\begin{pgfscope}%
\pgfpathrectangle{\pgfqpoint{0.041670in}{0.041670in}}{\pgfqpoint{2.216660in}{2.216660in}}%
\pgfusepath{clip}%
\pgfsetbuttcap%
\pgfsetroundjoin%
\definecolor{currentfill}{rgb}{0.565498,0.842430,0.262877}%
\pgfsetfillcolor{currentfill}%
\pgfsetlinewidth{0.000000pt}%
\definecolor{currentstroke}{rgb}{0.000000,0.000000,0.000000}%
\pgfsetstrokecolor{currentstroke}%
\pgfsetdash{}{0pt}%
\pgfpathmoveto{\pgfqpoint{1.307899in}{1.583182in}}%
\pgfpathlineto{\pgfqpoint{1.310749in}{1.579009in}}%
\pgfpathlineto{\pgfqpoint{1.313597in}{1.574741in}}%
\pgfpathlineto{\pgfqpoint{1.316441in}{1.570381in}}%
\pgfpathlineto{\pgfqpoint{1.319283in}{1.565929in}}%
\pgfpathlineto{\pgfqpoint{1.323090in}{1.563823in}}%
\pgfpathlineto{\pgfqpoint{1.326759in}{1.561661in}}%
\pgfpathlineto{\pgfqpoint{1.330285in}{1.559444in}}%
\pgfpathlineto{\pgfqpoint{1.333664in}{1.557175in}}%
\pgfpathlineto{\pgfqpoint{1.330526in}{1.561812in}}%
\pgfpathlineto{\pgfqpoint{1.327384in}{1.566357in}}%
\pgfpathlineto{\pgfqpoint{1.324238in}{1.570809in}}%
\pgfpathlineto{\pgfqpoint{1.321090in}{1.575167in}}%
\pgfpathlineto{\pgfqpoint{1.317991in}{1.577244in}}%
\pgfpathlineto{\pgfqpoint{1.314757in}{1.579274in}}%
\pgfpathlineto{\pgfqpoint{1.311392in}{1.581254in}}%
\pgfpathlineto{\pgfqpoint{1.307899in}{1.583182in}}%
\pgfpathclose%
\pgfusepath{fill}%
\end{pgfscope}%
\begin{pgfscope}%
\pgfpathrectangle{\pgfqpoint{0.041670in}{0.041670in}}{\pgfqpoint{2.216660in}{2.216660in}}%
\pgfusepath{clip}%
\pgfsetbuttcap%
\pgfsetroundjoin%
\definecolor{currentfill}{rgb}{0.699415,0.867117,0.175971}%
\pgfsetfillcolor{currentfill}%
\pgfsetlinewidth{0.000000pt}%
\definecolor{currentstroke}{rgb}{0.000000,0.000000,0.000000}%
\pgfsetstrokecolor{currentstroke}%
\pgfsetdash{}{0pt}%
\pgfpathmoveto{\pgfqpoint{1.227457in}{1.631390in}}%
\pgfpathlineto{\pgfqpoint{1.228753in}{1.628542in}}%
\pgfpathlineto{\pgfqpoint{1.230048in}{1.625587in}}%
\pgfpathlineto{\pgfqpoint{1.231341in}{1.622528in}}%
\pgfpathlineto{\pgfqpoint{1.232633in}{1.619365in}}%
\pgfpathlineto{\pgfqpoint{1.237183in}{1.618552in}}%
\pgfpathlineto{\pgfqpoint{1.241678in}{1.617673in}}%
\pgfpathlineto{\pgfqpoint{1.246113in}{1.616727in}}%
\pgfpathlineto{\pgfqpoint{1.250484in}{1.615716in}}%
\pgfpathlineto{\pgfqpoint{1.248753in}{1.618970in}}%
\pgfpathlineto{\pgfqpoint{1.247021in}{1.622121in}}%
\pgfpathlineto{\pgfqpoint{1.245287in}{1.625167in}}%
\pgfpathlineto{\pgfqpoint{1.243550in}{1.628106in}}%
\pgfpathlineto{\pgfqpoint{1.239610in}{1.629016in}}%
\pgfpathlineto{\pgfqpoint{1.235612in}{1.629867in}}%
\pgfpathlineto{\pgfqpoint{1.231559in}{1.630659in}}%
\pgfpathlineto{\pgfqpoint{1.227457in}{1.631390in}}%
\pgfpathclose%
\pgfusepath{fill}%
\end{pgfscope}%
\begin{pgfscope}%
\pgfpathrectangle{\pgfqpoint{0.041670in}{0.041670in}}{\pgfqpoint{2.216660in}{2.216660in}}%
\pgfusepath{clip}%
\pgfsetbuttcap%
\pgfsetroundjoin%
\definecolor{currentfill}{rgb}{0.487026,0.823929,0.312321}%
\pgfsetfillcolor{currentfill}%
\pgfsetlinewidth{0.000000pt}%
\definecolor{currentstroke}{rgb}{0.000000,0.000000,0.000000}%
\pgfsetstrokecolor{currentstroke}%
\pgfsetdash{}{0pt}%
\pgfpathmoveto{\pgfqpoint{1.011939in}{1.545403in}}%
\pgfpathlineto{\pgfqpoint{1.008506in}{1.540428in}}%
\pgfpathlineto{\pgfqpoint{1.005076in}{1.535364in}}%
\pgfpathlineto{\pgfqpoint{1.001649in}{1.530214in}}%
\pgfpathlineto{\pgfqpoint{0.998226in}{1.524979in}}%
\pgfpathlineto{\pgfqpoint{1.001062in}{1.527684in}}%
\pgfpathlineto{\pgfqpoint{1.004074in}{1.530343in}}%
\pgfpathlineto{\pgfqpoint{1.007256in}{1.532955in}}%
\pgfpathlineto{\pgfqpoint{1.010607in}{1.535517in}}%
\pgfpathlineto{\pgfqpoint{1.013792in}{1.540545in}}%
\pgfpathlineto{\pgfqpoint{1.016980in}{1.545489in}}%
\pgfpathlineto{\pgfqpoint{1.020172in}{1.550347in}}%
\pgfpathlineto{\pgfqpoint{1.023367in}{1.555116in}}%
\pgfpathlineto{\pgfqpoint{1.020273in}{1.552755in}}%
\pgfpathlineto{\pgfqpoint{1.017335in}{1.550347in}}%
\pgfpathlineto{\pgfqpoint{1.014556in}{1.547896in}}%
\pgfpathlineto{\pgfqpoint{1.011939in}{1.545403in}}%
\pgfpathclose%
\pgfusepath{fill}%
\end{pgfscope}%
\begin{pgfscope}%
\pgfpathrectangle{\pgfqpoint{0.041670in}{0.041670in}}{\pgfqpoint{2.216660in}{2.216660in}}%
\pgfusepath{clip}%
\pgfsetbuttcap%
\pgfsetroundjoin%
\definecolor{currentfill}{rgb}{0.412913,0.803041,0.357269}%
\pgfsetfillcolor{currentfill}%
\pgfsetlinewidth{0.000000pt}%
\definecolor{currentstroke}{rgb}{0.000000,0.000000,0.000000}%
\pgfsetstrokecolor{currentstroke}%
\pgfsetdash{}{0pt}%
\pgfpathmoveto{\pgfqpoint{0.988686in}{1.513766in}}%
\pgfpathlineto{\pgfqpoint{0.985081in}{1.508228in}}%
\pgfpathlineto{\pgfqpoint{0.981479in}{1.502608in}}%
\pgfpathlineto{\pgfqpoint{0.977881in}{1.496910in}}%
\pgfpathlineto{\pgfqpoint{0.974286in}{1.491134in}}%
\pgfpathlineto{\pgfqpoint{0.976560in}{1.494217in}}%
\pgfpathlineto{\pgfqpoint{0.979034in}{1.497263in}}%
\pgfpathlineto{\pgfqpoint{0.981705in}{1.500269in}}%
\pgfpathlineto{\pgfqpoint{0.984569in}{1.503232in}}%
\pgfpathlineto{\pgfqpoint{0.987978in}{1.508786in}}%
\pgfpathlineto{\pgfqpoint{0.991391in}{1.514264in}}%
\pgfpathlineto{\pgfqpoint{0.994807in}{1.519662in}}%
\pgfpathlineto{\pgfqpoint{0.998226in}{1.524979in}}%
\pgfpathlineto{\pgfqpoint{0.995568in}{1.522233in}}%
\pgfpathlineto{\pgfqpoint{0.993090in}{1.519447in}}%
\pgfpathlineto{\pgfqpoint{0.990795in}{1.516623in}}%
\pgfpathlineto{\pgfqpoint{0.988686in}{1.513766in}}%
\pgfpathclose%
\pgfusepath{fill}%
\end{pgfscope}%
\begin{pgfscope}%
\pgfpathrectangle{\pgfqpoint{0.041670in}{0.041670in}}{\pgfqpoint{2.216660in}{2.216660in}}%
\pgfusepath{clip}%
\pgfsetbuttcap%
\pgfsetroundjoin%
\definecolor{currentfill}{rgb}{0.133743,0.548535,0.553541}%
\pgfsetfillcolor{currentfill}%
\pgfsetlinewidth{0.000000pt}%
\definecolor{currentstroke}{rgb}{0.000000,0.000000,0.000000}%
\pgfsetstrokecolor{currentstroke}%
\pgfsetdash{}{0pt}%
\pgfpathmoveto{\pgfqpoint{1.499022in}{1.231845in}}%
\pgfpathlineto{\pgfqpoint{1.502716in}{1.223844in}}%
\pgfpathlineto{\pgfqpoint{1.506408in}{1.215835in}}%
\pgfpathlineto{\pgfqpoint{1.510097in}{1.207820in}}%
\pgfpathlineto{\pgfqpoint{1.513783in}{1.199803in}}%
\pgfpathlineto{\pgfqpoint{1.512087in}{1.194559in}}%
\pgfpathlineto{\pgfqpoint{1.510055in}{1.189341in}}%
\pgfpathlineto{\pgfqpoint{1.507690in}{1.184153in}}%
\pgfpathlineto{\pgfqpoint{1.504991in}{1.179001in}}%
\pgfpathlineto{\pgfqpoint{1.501391in}{1.187272in}}%
\pgfpathlineto{\pgfqpoint{1.497789in}{1.195541in}}%
\pgfpathlineto{\pgfqpoint{1.494184in}{1.203803in}}%
\pgfpathlineto{\pgfqpoint{1.490577in}{1.212057in}}%
\pgfpathlineto{\pgfqpoint{1.493166in}{1.216957in}}%
\pgfpathlineto{\pgfqpoint{1.495438in}{1.221892in}}%
\pgfpathlineto{\pgfqpoint{1.497390in}{1.226856in}}%
\pgfpathlineto{\pgfqpoint{1.499022in}{1.231845in}}%
\pgfpathclose%
\pgfusepath{fill}%
\end{pgfscope}%
\begin{pgfscope}%
\pgfpathrectangle{\pgfqpoint{0.041670in}{0.041670in}}{\pgfqpoint{2.216660in}{2.216660in}}%
\pgfusepath{clip}%
\pgfsetbuttcap%
\pgfsetroundjoin%
\definecolor{currentfill}{rgb}{0.248629,0.278775,0.534556}%
\pgfsetfillcolor{currentfill}%
\pgfsetlinewidth{0.000000pt}%
\definecolor{currentstroke}{rgb}{0.000000,0.000000,0.000000}%
\pgfsetstrokecolor{currentstroke}%
\pgfsetdash{}{0pt}%
\pgfpathmoveto{\pgfqpoint{0.816688in}{0.930813in}}%
\pgfpathlineto{\pgfqpoint{0.813511in}{0.922889in}}%
\pgfpathlineto{\pgfqpoint{0.810334in}{0.915043in}}%
\pgfpathlineto{\pgfqpoint{0.807156in}{0.907278in}}%
\pgfpathlineto{\pgfqpoint{0.803979in}{0.899598in}}%
\pgfpathlineto{\pgfqpoint{0.796815in}{0.905830in}}%
\pgfpathlineto{\pgfqpoint{0.790053in}{0.912172in}}%
\pgfpathlineto{\pgfqpoint{0.783696in}{0.918615in}}%
\pgfpathlineto{\pgfqpoint{0.777752in}{0.925154in}}%
\pgfpathlineto{\pgfqpoint{0.781139in}{0.932592in}}%
\pgfpathlineto{\pgfqpoint{0.784527in}{0.940115in}}%
\pgfpathlineto{\pgfqpoint{0.787914in}{0.947719in}}%
\pgfpathlineto{\pgfqpoint{0.791301in}{0.955401in}}%
\pgfpathlineto{\pgfqpoint{0.797057in}{0.949109in}}%
\pgfpathlineto{\pgfqpoint{0.803210in}{0.942910in}}%
\pgfpathlineto{\pgfqpoint{0.809756in}{0.936809in}}%
\pgfpathlineto{\pgfqpoint{0.816688in}{0.930813in}}%
\pgfpathclose%
\pgfusepath{fill}%
\end{pgfscope}%
\begin{pgfscope}%
\pgfpathrectangle{\pgfqpoint{0.041670in}{0.041670in}}{\pgfqpoint{2.216660in}{2.216660in}}%
\pgfusepath{clip}%
\pgfsetbuttcap%
\pgfsetroundjoin%
\definecolor{currentfill}{rgb}{0.231674,0.318106,0.544834}%
\pgfsetfillcolor{currentfill}%
\pgfsetlinewidth{0.000000pt}%
\definecolor{currentstroke}{rgb}{0.000000,0.000000,0.000000}%
\pgfsetstrokecolor{currentstroke}%
\pgfsetdash{}{0pt}%
\pgfpathmoveto{\pgfqpoint{1.559681in}{0.992278in}}%
\pgfpathlineto{\pgfqpoint{1.563108in}{0.984375in}}%
\pgfpathlineto{\pgfqpoint{1.566535in}{0.976536in}}%
\pgfpathlineto{\pgfqpoint{1.569961in}{0.968765in}}%
\pgfpathlineto{\pgfqpoint{1.573387in}{0.961064in}}%
\pgfpathlineto{\pgfqpoint{1.567989in}{0.954697in}}%
\pgfpathlineto{\pgfqpoint{1.562189in}{0.948416in}}%
\pgfpathlineto{\pgfqpoint{1.555992in}{0.942227in}}%
\pgfpathlineto{\pgfqpoint{1.549403in}{0.936137in}}%
\pgfpathlineto{\pgfqpoint{1.546174in}{0.944082in}}%
\pgfpathlineto{\pgfqpoint{1.542946in}{0.952098in}}%
\pgfpathlineto{\pgfqpoint{1.539717in}{0.960181in}}%
\pgfpathlineto{\pgfqpoint{1.536488in}{0.968327in}}%
\pgfpathlineto{\pgfqpoint{1.542858in}{0.974178in}}%
\pgfpathlineto{\pgfqpoint{1.548850in}{0.980124in}}%
\pgfpathlineto{\pgfqpoint{1.554459in}{0.986160in}}%
\pgfpathlineto{\pgfqpoint{1.559681in}{0.992278in}}%
\pgfpathclose%
\pgfusepath{fill}%
\end{pgfscope}%
\begin{pgfscope}%
\pgfpathrectangle{\pgfqpoint{0.041670in}{0.041670in}}{\pgfqpoint{2.216660in}{2.216660in}}%
\pgfusepath{clip}%
\pgfsetbuttcap%
\pgfsetroundjoin%
\definecolor{currentfill}{rgb}{0.699415,0.867117,0.175971}%
\pgfsetfillcolor{currentfill}%
\pgfsetlinewidth{0.000000pt}%
\definecolor{currentstroke}{rgb}{0.000000,0.000000,0.000000}%
\pgfsetstrokecolor{currentstroke}%
\pgfsetdash{}{0pt}%
\pgfpathmoveto{\pgfqpoint{1.112909in}{1.627249in}}%
\pgfpathlineto{\pgfqpoint{1.111078in}{1.624286in}}%
\pgfpathlineto{\pgfqpoint{1.109249in}{1.621216in}}%
\pgfpathlineto{\pgfqpoint{1.107422in}{1.618042in}}%
\pgfpathlineto{\pgfqpoint{1.105598in}{1.614763in}}%
\pgfpathlineto{\pgfqpoint{1.109908in}{1.615831in}}%
\pgfpathlineto{\pgfqpoint{1.114287in}{1.616835in}}%
\pgfpathlineto{\pgfqpoint{1.118729in}{1.617774in}}%
\pgfpathlineto{\pgfqpoint{1.123230in}{1.618646in}}%
\pgfpathlineto{\pgfqpoint{1.124621in}{1.621827in}}%
\pgfpathlineto{\pgfqpoint{1.126013in}{1.624904in}}%
\pgfpathlineto{\pgfqpoint{1.127408in}{1.627877in}}%
\pgfpathlineto{\pgfqpoint{1.128804in}{1.630743in}}%
\pgfpathlineto{\pgfqpoint{1.124746in}{1.629958in}}%
\pgfpathlineto{\pgfqpoint{1.120741in}{1.629113in}}%
\pgfpathlineto{\pgfqpoint{1.116794in}{1.628210in}}%
\pgfpathlineto{\pgfqpoint{1.112909in}{1.627249in}}%
\pgfpathclose%
\pgfusepath{fill}%
\end{pgfscope}%
\begin{pgfscope}%
\pgfpathrectangle{\pgfqpoint{0.041670in}{0.041670in}}{\pgfqpoint{2.216660in}{2.216660in}}%
\pgfusepath{clip}%
\pgfsetbuttcap%
\pgfsetroundjoin%
\definecolor{currentfill}{rgb}{0.274952,0.037752,0.364543}%
\pgfsetfillcolor{currentfill}%
\pgfsetlinewidth{0.000000pt}%
\definecolor{currentstroke}{rgb}{0.000000,0.000000,0.000000}%
\pgfsetstrokecolor{currentstroke}%
\pgfsetdash{}{0pt}%
\pgfpathmoveto{\pgfqpoint{1.640250in}{0.756912in}}%
\pgfpathlineto{\pgfqpoint{1.643531in}{0.752606in}}%
\pgfpathlineto{\pgfqpoint{1.646816in}{0.748486in}}%
\pgfpathlineto{\pgfqpoint{1.650106in}{0.744555in}}%
\pgfpathlineto{\pgfqpoint{1.653399in}{0.740818in}}%
\pgfpathlineto{\pgfqpoint{1.644571in}{0.732938in}}%
\pgfpathlineto{\pgfqpoint{1.635250in}{0.725203in}}%
\pgfpathlineto{\pgfqpoint{1.625445in}{0.717621in}}%
\pgfpathlineto{\pgfqpoint{1.615164in}{0.710200in}}%
\pgfpathlineto{\pgfqpoint{1.612124in}{0.714167in}}%
\pgfpathlineto{\pgfqpoint{1.609088in}{0.718327in}}%
\pgfpathlineto{\pgfqpoint{1.606056in}{0.722677in}}%
\pgfpathlineto{\pgfqpoint{1.603028in}{0.727213in}}%
\pgfpathlineto{\pgfqpoint{1.613034in}{0.734410in}}%
\pgfpathlineto{\pgfqpoint{1.622579in}{0.741765in}}%
\pgfpathlineto{\pgfqpoint{1.631654in}{0.749268in}}%
\pgfpathlineto{\pgfqpoint{1.640250in}{0.756912in}}%
\pgfpathclose%
\pgfusepath{fill}%
\end{pgfscope}%
\begin{pgfscope}%
\pgfpathrectangle{\pgfqpoint{0.041670in}{0.041670in}}{\pgfqpoint{2.216660in}{2.216660in}}%
\pgfusepath{clip}%
\pgfsetbuttcap%
\pgfsetroundjoin%
\definecolor{currentfill}{rgb}{0.279566,0.067836,0.391917}%
\pgfsetfillcolor{currentfill}%
\pgfsetlinewidth{0.000000pt}%
\definecolor{currentstroke}{rgb}{0.000000,0.000000,0.000000}%
\pgfsetstrokecolor{currentstroke}%
\pgfsetdash{}{0pt}%
\pgfpathmoveto{\pgfqpoint{1.627165in}{0.775897in}}%
\pgfpathlineto{\pgfqpoint{1.630431in}{0.770895in}}%
\pgfpathlineto{\pgfqpoint{1.633700in}{0.766060in}}%
\pgfpathlineto{\pgfqpoint{1.636973in}{0.761398in}}%
\pgfpathlineto{\pgfqpoint{1.640250in}{0.756912in}}%
\pgfpathlineto{\pgfqpoint{1.631654in}{0.749268in}}%
\pgfpathlineto{\pgfqpoint{1.622579in}{0.741765in}}%
\pgfpathlineto{\pgfqpoint{1.613034in}{0.734410in}}%
\pgfpathlineto{\pgfqpoint{1.603028in}{0.727213in}}%
\pgfpathlineto{\pgfqpoint{1.600004in}{0.731929in}}%
\pgfpathlineto{\pgfqpoint{1.596984in}{0.736822in}}%
\pgfpathlineto{\pgfqpoint{1.593967in}{0.741887in}}%
\pgfpathlineto{\pgfqpoint{1.590954in}{0.747121in}}%
\pgfpathlineto{\pgfqpoint{1.600686in}{0.754094in}}%
\pgfpathlineto{\pgfqpoint{1.609970in}{0.761220in}}%
\pgfpathlineto{\pgfqpoint{1.618799in}{0.768490in}}%
\pgfpathlineto{\pgfqpoint{1.627165in}{0.775897in}}%
\pgfpathclose%
\pgfusepath{fill}%
\end{pgfscope}%
\begin{pgfscope}%
\pgfpathrectangle{\pgfqpoint{0.041670in}{0.041670in}}{\pgfqpoint{2.216660in}{2.216660in}}%
\pgfusepath{clip}%
\pgfsetbuttcap%
\pgfsetroundjoin%
\definecolor{currentfill}{rgb}{0.565498,0.842430,0.262877}%
\pgfsetfillcolor{currentfill}%
\pgfsetlinewidth{0.000000pt}%
\definecolor{currentstroke}{rgb}{0.000000,0.000000,0.000000}%
\pgfsetstrokecolor{currentstroke}%
\pgfsetdash{}{0pt}%
\pgfpathmoveto{\pgfqpoint{1.036181in}{1.573281in}}%
\pgfpathlineto{\pgfqpoint{1.032972in}{1.568881in}}%
\pgfpathlineto{\pgfqpoint{1.029767in}{1.564385in}}%
\pgfpathlineto{\pgfqpoint{1.026566in}{1.559797in}}%
\pgfpathlineto{\pgfqpoint{1.023367in}{1.555116in}}%
\pgfpathlineto{\pgfqpoint{1.026614in}{1.557430in}}%
\pgfpathlineto{\pgfqpoint{1.030010in}{1.559693in}}%
\pgfpathlineto{\pgfqpoint{1.033551in}{1.561904in}}%
\pgfpathlineto{\pgfqpoint{1.037236in}{1.564060in}}%
\pgfpathlineto{\pgfqpoint{1.040147in}{1.568551in}}%
\pgfpathlineto{\pgfqpoint{1.043061in}{1.572951in}}%
\pgfpathlineto{\pgfqpoint{1.045979in}{1.577258in}}%
\pgfpathlineto{\pgfqpoint{1.048900in}{1.581471in}}%
\pgfpathlineto{\pgfqpoint{1.045520in}{1.579496in}}%
\pgfpathlineto{\pgfqpoint{1.042272in}{1.577472in}}%
\pgfpathlineto{\pgfqpoint{1.039158in}{1.575400in}}%
\pgfpathlineto{\pgfqpoint{1.036181in}{1.573281in}}%
\pgfpathclose%
\pgfusepath{fill}%
\end{pgfscope}%
\begin{pgfscope}%
\pgfpathrectangle{\pgfqpoint{0.041670in}{0.041670in}}{\pgfqpoint{2.216660in}{2.216660in}}%
\pgfusepath{clip}%
\pgfsetbuttcap%
\pgfsetroundjoin%
\definecolor{currentfill}{rgb}{0.271305,0.019942,0.347269}%
\pgfsetfillcolor{currentfill}%
\pgfsetlinewidth{0.000000pt}%
\definecolor{currentstroke}{rgb}{0.000000,0.000000,0.000000}%
\pgfsetstrokecolor{currentstroke}%
\pgfsetdash{}{0pt}%
\pgfpathmoveto{\pgfqpoint{1.653399in}{0.740818in}}%
\pgfpathlineto{\pgfqpoint{1.656698in}{0.737278in}}%
\pgfpathlineto{\pgfqpoint{1.660002in}{0.733941in}}%
\pgfpathlineto{\pgfqpoint{1.663310in}{0.730811in}}%
\pgfpathlineto{\pgfqpoint{1.666624in}{0.727892in}}%
\pgfpathlineto{\pgfqpoint{1.657563in}{0.719778in}}%
\pgfpathlineto{\pgfqpoint{1.647995in}{0.711812in}}%
\pgfpathlineto{\pgfqpoint{1.637929in}{0.704003in}}%
\pgfpathlineto{\pgfqpoint{1.627373in}{0.696360in}}%
\pgfpathlineto{\pgfqpoint{1.624314in}{0.699507in}}%
\pgfpathlineto{\pgfqpoint{1.621259in}{0.702866in}}%
\pgfpathlineto{\pgfqpoint{1.618209in}{0.706431in}}%
\pgfpathlineto{\pgfqpoint{1.615164in}{0.710200in}}%
\pgfpathlineto{\pgfqpoint{1.625445in}{0.717621in}}%
\pgfpathlineto{\pgfqpoint{1.635250in}{0.725203in}}%
\pgfpathlineto{\pgfqpoint{1.644571in}{0.732938in}}%
\pgfpathlineto{\pgfqpoint{1.653399in}{0.740818in}}%
\pgfpathclose%
\pgfusepath{fill}%
\end{pgfscope}%
\begin{pgfscope}%
\pgfpathrectangle{\pgfqpoint{0.041670in}{0.041670in}}{\pgfqpoint{2.216660in}{2.216660in}}%
\pgfusepath{clip}%
\pgfsetbuttcap%
\pgfsetroundjoin%
\definecolor{currentfill}{rgb}{0.282327,0.094955,0.417331}%
\pgfsetfillcolor{currentfill}%
\pgfsetlinewidth{0.000000pt}%
\definecolor{currentstroke}{rgb}{0.000000,0.000000,0.000000}%
\pgfsetstrokecolor{currentstroke}%
\pgfsetdash{}{0pt}%
\pgfpathmoveto{\pgfqpoint{1.614131in}{0.797507in}}%
\pgfpathlineto{\pgfqpoint{1.617385in}{0.791873in}}%
\pgfpathlineto{\pgfqpoint{1.620642in}{0.786391in}}%
\pgfpathlineto{\pgfqpoint{1.623902in}{0.781064in}}%
\pgfpathlineto{\pgfqpoint{1.627165in}{0.775897in}}%
\pgfpathlineto{\pgfqpoint{1.618799in}{0.768490in}}%
\pgfpathlineto{\pgfqpoint{1.609970in}{0.761220in}}%
\pgfpathlineto{\pgfqpoint{1.600686in}{0.754094in}}%
\pgfpathlineto{\pgfqpoint{1.590954in}{0.747121in}}%
\pgfpathlineto{\pgfqpoint{1.587943in}{0.752518in}}%
\pgfpathlineto{\pgfqpoint{1.584936in}{0.758076in}}%
\pgfpathlineto{\pgfqpoint{1.581931in}{0.763790in}}%
\pgfpathlineto{\pgfqpoint{1.578930in}{0.769655in}}%
\pgfpathlineto{\pgfqpoint{1.588388in}{0.776404in}}%
\pgfpathlineto{\pgfqpoint{1.597413in}{0.783301in}}%
\pgfpathlineto{\pgfqpoint{1.605997in}{0.790338in}}%
\pgfpathlineto{\pgfqpoint{1.614131in}{0.797507in}}%
\pgfpathclose%
\pgfusepath{fill}%
\end{pgfscope}%
\begin{pgfscope}%
\pgfpathrectangle{\pgfqpoint{0.041670in}{0.041670in}}{\pgfqpoint{2.216660in}{2.216660in}}%
\pgfusepath{clip}%
\pgfsetbuttcap%
\pgfsetroundjoin%
\definecolor{currentfill}{rgb}{0.344074,0.780029,0.397381}%
\pgfsetfillcolor{currentfill}%
\pgfsetlinewidth{0.000000pt}%
\definecolor{currentstroke}{rgb}{0.000000,0.000000,0.000000}%
\pgfsetstrokecolor{currentstroke}%
\pgfsetdash{}{0pt}%
\pgfpathmoveto{\pgfqpoint{0.967231in}{1.478489in}}%
\pgfpathlineto{\pgfqpoint{0.963509in}{1.472405in}}%
\pgfpathlineto{\pgfqpoint{0.959791in}{1.466248in}}%
\pgfpathlineto{\pgfqpoint{0.956076in}{1.460018in}}%
\pgfpathlineto{\pgfqpoint{0.952365in}{1.453719in}}%
\pgfpathlineto{\pgfqpoint{0.953928in}{1.457158in}}%
\pgfpathlineto{\pgfqpoint{0.955713in}{1.460569in}}%
\pgfpathlineto{\pgfqpoint{0.957719in}{1.463950in}}%
\pgfpathlineto{\pgfqpoint{0.959943in}{1.467297in}}%
\pgfpathlineto{\pgfqpoint{0.963523in}{1.473362in}}%
\pgfpathlineto{\pgfqpoint{0.967107in}{1.479358in}}%
\pgfpathlineto{\pgfqpoint{0.970695in}{1.485283in}}%
\pgfpathlineto{\pgfqpoint{0.974286in}{1.491134in}}%
\pgfpathlineto{\pgfqpoint{0.972213in}{1.488016in}}%
\pgfpathlineto{\pgfqpoint{0.970345in}{1.484868in}}%
\pgfpathlineto{\pgfqpoint{0.968684in}{1.481691in}}%
\pgfpathlineto{\pgfqpoint{0.967231in}{1.478489in}}%
\pgfpathclose%
\pgfusepath{fill}%
\end{pgfscope}%
\begin{pgfscope}%
\pgfpathrectangle{\pgfqpoint{0.041670in}{0.041670in}}{\pgfqpoint{2.216660in}{2.216660in}}%
\pgfusepath{clip}%
\pgfsetbuttcap%
\pgfsetroundjoin%
\definecolor{currentfill}{rgb}{0.179019,0.433756,0.557430}%
\pgfsetfillcolor{currentfill}%
\pgfsetlinewidth{0.000000pt}%
\definecolor{currentstroke}{rgb}{0.000000,0.000000,0.000000}%
\pgfsetstrokecolor{currentstroke}%
\pgfsetdash{}{0pt}%
\pgfpathmoveto{\pgfqpoint{0.845571in}{1.086145in}}%
\pgfpathlineto{\pgfqpoint{0.842172in}{1.077670in}}%
\pgfpathlineto{\pgfqpoint{0.838774in}{1.069221in}}%
\pgfpathlineto{\pgfqpoint{0.835377in}{1.060801in}}%
\pgfpathlineto{\pgfqpoint{0.831981in}{1.052413in}}%
\pgfpathlineto{\pgfqpoint{0.827155in}{1.058045in}}%
\pgfpathlineto{\pgfqpoint{0.822693in}{1.063747in}}%
\pgfpathlineto{\pgfqpoint{0.818597in}{1.069513in}}%
\pgfpathlineto{\pgfqpoint{0.814871in}{1.075336in}}%
\pgfpathlineto{\pgfqpoint{0.818422in}{1.083474in}}%
\pgfpathlineto{\pgfqpoint{0.821974in}{1.091645in}}%
\pgfpathlineto{\pgfqpoint{0.825528in}{1.099844in}}%
\pgfpathlineto{\pgfqpoint{0.829084in}{1.108070in}}%
\pgfpathlineto{\pgfqpoint{0.832676in}{1.102499in}}%
\pgfpathlineto{\pgfqpoint{0.836623in}{1.096984in}}%
\pgfpathlineto{\pgfqpoint{0.840923in}{1.091531in}}%
\pgfpathlineto{\pgfqpoint{0.845571in}{1.086145in}}%
\pgfpathclose%
\pgfusepath{fill}%
\end{pgfscope}%
\begin{pgfscope}%
\pgfpathrectangle{\pgfqpoint{0.041670in}{0.041670in}}{\pgfqpoint{2.216660in}{2.216660in}}%
\pgfusepath{clip}%
\pgfsetbuttcap%
\pgfsetroundjoin%
\definecolor{currentfill}{rgb}{0.120081,0.622161,0.534946}%
\pgfsetfillcolor{currentfill}%
\pgfsetlinewidth{0.000000pt}%
\definecolor{currentstroke}{rgb}{0.000000,0.000000,0.000000}%
\pgfsetstrokecolor{currentstroke}%
\pgfsetdash{}{0pt}%
\pgfpathmoveto{\pgfqpoint{0.891857in}{1.291251in}}%
\pgfpathlineto{\pgfqpoint{0.888156in}{1.283356in}}%
\pgfpathlineto{\pgfqpoint{0.884457in}{1.275431in}}%
\pgfpathlineto{\pgfqpoint{0.880761in}{1.267479in}}%
\pgfpathlineto{\pgfqpoint{0.877068in}{1.259503in}}%
\pgfpathlineto{\pgfqpoint{0.875536in}{1.264241in}}%
\pgfpathlineto{\pgfqpoint{0.874310in}{1.268996in}}%
\pgfpathlineto{\pgfqpoint{0.873391in}{1.273765in}}%
\pgfpathlineto{\pgfqpoint{0.872780in}{1.278543in}}%
\pgfpathlineto{\pgfqpoint{0.876515in}{1.286265in}}%
\pgfpathlineto{\pgfqpoint{0.880253in}{1.293964in}}%
\pgfpathlineto{\pgfqpoint{0.883994in}{1.301636in}}%
\pgfpathlineto{\pgfqpoint{0.887738in}{1.309279in}}%
\pgfpathlineto{\pgfqpoint{0.888330in}{1.304755in}}%
\pgfpathlineto{\pgfqpoint{0.889214in}{1.300239in}}%
\pgfpathlineto{\pgfqpoint{0.890390in}{1.295736in}}%
\pgfpathlineto{\pgfqpoint{0.891857in}{1.291251in}}%
\pgfpathclose%
\pgfusepath{fill}%
\end{pgfscope}%
\begin{pgfscope}%
\pgfpathrectangle{\pgfqpoint{0.041670in}{0.041670in}}{\pgfqpoint{2.216660in}{2.216660in}}%
\pgfusepath{clip}%
\pgfsetbuttcap%
\pgfsetroundjoin%
\definecolor{currentfill}{rgb}{0.166383,0.690856,0.496502}%
\pgfsetfillcolor{currentfill}%
\pgfsetlinewidth{0.000000pt}%
\definecolor{currentstroke}{rgb}{0.000000,0.000000,0.000000}%
\pgfsetstrokecolor{currentstroke}%
\pgfsetdash{}{0pt}%
\pgfpathmoveto{\pgfqpoint{0.917803in}{1.369084in}}%
\pgfpathlineto{\pgfqpoint{0.914034in}{1.361761in}}%
\pgfpathlineto{\pgfqpoint{0.910268in}{1.354390in}}%
\pgfpathlineto{\pgfqpoint{0.906505in}{1.346973in}}%
\pgfpathlineto{\pgfqpoint{0.902746in}{1.339512in}}%
\pgfpathlineto{\pgfqpoint{0.902452in}{1.343789in}}%
\pgfpathlineto{\pgfqpoint{0.902436in}{1.348065in}}%
\pgfpathlineto{\pgfqpoint{0.902696in}{1.352337in}}%
\pgfpathlineto{\pgfqpoint{0.903232in}{1.356600in}}%
\pgfpathlineto{\pgfqpoint{0.906976in}{1.363811in}}%
\pgfpathlineto{\pgfqpoint{0.910723in}{1.370979in}}%
\pgfpathlineto{\pgfqpoint{0.914473in}{1.378101in}}%
\pgfpathlineto{\pgfqpoint{0.918227in}{1.385175in}}%
\pgfpathlineto{\pgfqpoint{0.917729in}{1.381160in}}%
\pgfpathlineto{\pgfqpoint{0.917492in}{1.377137in}}%
\pgfpathlineto{\pgfqpoint{0.917516in}{1.373111in}}%
\pgfpathlineto{\pgfqpoint{0.917803in}{1.369084in}}%
\pgfpathclose%
\pgfusepath{fill}%
\end{pgfscope}%
\begin{pgfscope}%
\pgfpathrectangle{\pgfqpoint{0.041670in}{0.041670in}}{\pgfqpoint{2.216660in}{2.216660in}}%
\pgfusepath{clip}%
\pgfsetbuttcap%
\pgfsetroundjoin%
\definecolor{currentfill}{rgb}{0.281477,0.755203,0.432552}%
\pgfsetfillcolor{currentfill}%
\pgfsetlinewidth{0.000000pt}%
\definecolor{currentstroke}{rgb}{0.000000,0.000000,0.000000}%
\pgfsetstrokecolor{currentstroke}%
\pgfsetdash{}{0pt}%
\pgfpathmoveto{\pgfqpoint{1.406167in}{1.456777in}}%
\pgfpathlineto{\pgfqpoint{1.409850in}{1.450463in}}%
\pgfpathlineto{\pgfqpoint{1.413530in}{1.444084in}}%
\pgfpathlineto{\pgfqpoint{1.417207in}{1.437641in}}%
\pgfpathlineto{\pgfqpoint{1.420879in}{1.431138in}}%
\pgfpathlineto{\pgfqpoint{1.422525in}{1.427458in}}%
\pgfpathlineto{\pgfqpoint{1.423931in}{1.423754in}}%
\pgfpathlineto{\pgfqpoint{1.425097in}{1.420028in}}%
\pgfpathlineto{\pgfqpoint{1.426019in}{1.416284in}}%
\pgfpathlineto{\pgfqpoint{1.422260in}{1.423029in}}%
\pgfpathlineto{\pgfqpoint{1.418497in}{1.429712in}}%
\pgfpathlineto{\pgfqpoint{1.414731in}{1.436332in}}%
\pgfpathlineto{\pgfqpoint{1.410961in}{1.442886in}}%
\pgfpathlineto{\pgfqpoint{1.410103in}{1.446387in}}%
\pgfpathlineto{\pgfqpoint{1.409017in}{1.449871in}}%
\pgfpathlineto{\pgfqpoint{1.407704in}{1.453336in}}%
\pgfpathlineto{\pgfqpoint{1.406167in}{1.456777in}}%
\pgfpathclose%
\pgfusepath{fill}%
\end{pgfscope}%
\begin{pgfscope}%
\pgfpathrectangle{\pgfqpoint{0.041670in}{0.041670in}}{\pgfqpoint{2.216660in}{2.216660in}}%
\pgfusepath{clip}%
\pgfsetbuttcap%
\pgfsetroundjoin%
\definecolor{currentfill}{rgb}{0.636902,0.856542,0.216620}%
\pgfsetfillcolor{currentfill}%
\pgfsetlinewidth{0.000000pt}%
\definecolor{currentstroke}{rgb}{0.000000,0.000000,0.000000}%
\pgfsetstrokecolor{currentstroke}%
\pgfsetdash{}{0pt}%
\pgfpathmoveto{\pgfqpoint{1.282652in}{1.605405in}}%
\pgfpathlineto{\pgfqpoint{1.285173in}{1.601789in}}%
\pgfpathlineto{\pgfqpoint{1.287691in}{1.598073in}}%
\pgfpathlineto{\pgfqpoint{1.290207in}{1.594258in}}%
\pgfpathlineto{\pgfqpoint{1.292719in}{1.590345in}}%
\pgfpathlineto{\pgfqpoint{1.296688in}{1.588640in}}%
\pgfpathlineto{\pgfqpoint{1.300543in}{1.586877in}}%
\pgfpathlineto{\pgfqpoint{1.304281in}{1.585058in}}%
\pgfpathlineto{\pgfqpoint{1.307899in}{1.583182in}}%
\pgfpathlineto{\pgfqpoint{1.305045in}{1.587259in}}%
\pgfpathlineto{\pgfqpoint{1.302188in}{1.591239in}}%
\pgfpathlineto{\pgfqpoint{1.299328in}{1.595118in}}%
\pgfpathlineto{\pgfqpoint{1.296465in}{1.598897in}}%
\pgfpathlineto{\pgfqpoint{1.293174in}{1.600601in}}%
\pgfpathlineto{\pgfqpoint{1.289773in}{1.602254in}}%
\pgfpathlineto{\pgfqpoint{1.286264in}{1.603856in}}%
\pgfpathlineto{\pgfqpoint{1.282652in}{1.605405in}}%
\pgfpathclose%
\pgfusepath{fill}%
\end{pgfscope}%
\begin{pgfscope}%
\pgfpathrectangle{\pgfqpoint{0.041670in}{0.041670in}}{\pgfqpoint{2.216660in}{2.216660in}}%
\pgfusepath{clip}%
\pgfsetbuttcap%
\pgfsetroundjoin%
\definecolor{currentfill}{rgb}{0.201239,0.383670,0.554294}%
\pgfsetfillcolor{currentfill}%
\pgfsetlinewidth{0.000000pt}%
\definecolor{currentstroke}{rgb}{0.000000,0.000000,0.000000}%
\pgfsetstrokecolor{currentstroke}%
\pgfsetdash{}{0pt}%
\pgfpathmoveto{\pgfqpoint{1.906749in}{1.022205in}}%
\pgfpathlineto{\pgfqpoint{1.911032in}{1.034955in}}%
\pgfpathlineto{\pgfqpoint{1.915338in}{1.048184in}}%
\pgfpathlineto{\pgfqpoint{1.919666in}{1.061902in}}%
\pgfpathlineto{\pgfqpoint{1.924018in}{1.076116in}}%
\pgfpathlineto{\pgfqpoint{1.920929in}{1.064207in}}%
\pgfpathlineto{\pgfqpoint{1.917083in}{1.052333in}}%
\pgfpathlineto{\pgfqpoint{1.912478in}{1.040507in}}%
\pgfpathlineto{\pgfqpoint{1.907116in}{1.028742in}}%
\pgfpathlineto{\pgfqpoint{1.902835in}{1.014712in}}%
\pgfpathlineto{\pgfqpoint{1.898577in}{1.001181in}}%
\pgfpathlineto{\pgfqpoint{1.894341in}{0.988141in}}%
\pgfpathlineto{\pgfqpoint{1.890129in}{0.975583in}}%
\pgfpathlineto{\pgfqpoint{1.895393in}{0.987160in}}%
\pgfpathlineto{\pgfqpoint{1.899919in}{0.998798in}}%
\pgfpathlineto{\pgfqpoint{1.903704in}{1.010483in}}%
\pgfpathlineto{\pgfqpoint{1.906749in}{1.022205in}}%
\pgfpathclose%
\pgfusepath{fill}%
\end{pgfscope}%
\begin{pgfscope}%
\pgfpathrectangle{\pgfqpoint{0.041670in}{0.041670in}}{\pgfqpoint{2.216660in}{2.216660in}}%
\pgfusepath{clip}%
\pgfsetbuttcap%
\pgfsetroundjoin%
\definecolor{currentfill}{rgb}{0.699415,0.867117,0.175971}%
\pgfsetfillcolor{currentfill}%
\pgfsetlinewidth{0.000000pt}%
\definecolor{currentstroke}{rgb}{0.000000,0.000000,0.000000}%
\pgfsetstrokecolor{currentstroke}%
\pgfsetdash{}{0pt}%
\pgfpathmoveto{\pgfqpoint{1.243550in}{1.628106in}}%
\pgfpathlineto{\pgfqpoint{1.245287in}{1.625167in}}%
\pgfpathlineto{\pgfqpoint{1.247021in}{1.622121in}}%
\pgfpathlineto{\pgfqpoint{1.248753in}{1.618970in}}%
\pgfpathlineto{\pgfqpoint{1.250484in}{1.615716in}}%
\pgfpathlineto{\pgfqpoint{1.254787in}{1.614641in}}%
\pgfpathlineto{\pgfqpoint{1.259017in}{1.613502in}}%
\pgfpathlineto{\pgfqpoint{1.263171in}{1.612301in}}%
\pgfpathlineto{\pgfqpoint{1.267244in}{1.611039in}}%
\pgfpathlineto{\pgfqpoint{1.265101in}{1.614411in}}%
\pgfpathlineto{\pgfqpoint{1.262956in}{1.617679in}}%
\pgfpathlineto{\pgfqpoint{1.260808in}{1.620841in}}%
\pgfpathlineto{\pgfqpoint{1.258657in}{1.623897in}}%
\pgfpathlineto{\pgfqpoint{1.254986in}{1.625033in}}%
\pgfpathlineto{\pgfqpoint{1.251242in}{1.626114in}}%
\pgfpathlineto{\pgfqpoint{1.247429in}{1.627138in}}%
\pgfpathlineto{\pgfqpoint{1.243550in}{1.628106in}}%
\pgfpathclose%
\pgfusepath{fill}%
\end{pgfscope}%
\begin{pgfscope}%
\pgfpathrectangle{\pgfqpoint{0.041670in}{0.041670in}}{\pgfqpoint{2.216660in}{2.216660in}}%
\pgfusepath{clip}%
\pgfsetbuttcap%
\pgfsetroundjoin%
\definecolor{currentfill}{rgb}{0.268510,0.009605,0.335427}%
\pgfsetfillcolor{currentfill}%
\pgfsetlinewidth{0.000000pt}%
\definecolor{currentstroke}{rgb}{0.000000,0.000000,0.000000}%
\pgfsetstrokecolor{currentstroke}%
\pgfsetdash{}{0pt}%
\pgfpathmoveto{\pgfqpoint{1.666624in}{0.727892in}}%
\pgfpathlineto{\pgfqpoint{1.669944in}{0.725189in}}%
\pgfpathlineto{\pgfqpoint{1.673269in}{0.722707in}}%
\pgfpathlineto{\pgfqpoint{1.676601in}{0.720450in}}%
\pgfpathlineto{\pgfqpoint{1.679938in}{0.718423in}}%
\pgfpathlineto{\pgfqpoint{1.670644in}{0.710076in}}%
\pgfpathlineto{\pgfqpoint{1.660829in}{0.701880in}}%
\pgfpathlineto{\pgfqpoint{1.650500in}{0.693846in}}%
\pgfpathlineto{\pgfqpoint{1.639667in}{0.685983in}}%
\pgfpathlineto{\pgfqpoint{1.636585in}{0.688236in}}%
\pgfpathlineto{\pgfqpoint{1.633509in}{0.690720in}}%
\pgfpathlineto{\pgfqpoint{1.630438in}{0.693430in}}%
\pgfpathlineto{\pgfqpoint{1.627373in}{0.696360in}}%
\pgfpathlineto{\pgfqpoint{1.637929in}{0.704003in}}%
\pgfpathlineto{\pgfqpoint{1.647995in}{0.711812in}}%
\pgfpathlineto{\pgfqpoint{1.657563in}{0.719778in}}%
\pgfpathlineto{\pgfqpoint{1.666624in}{0.727892in}}%
\pgfpathclose%
\pgfusepath{fill}%
\end{pgfscope}%
\begin{pgfscope}%
\pgfpathrectangle{\pgfqpoint{0.041670in}{0.041670in}}{\pgfqpoint{2.216660in}{2.216660in}}%
\pgfusepath{clip}%
\pgfsetbuttcap%
\pgfsetroundjoin%
\definecolor{currentfill}{rgb}{0.283072,0.130895,0.449241}%
\pgfsetfillcolor{currentfill}%
\pgfsetlinewidth{0.000000pt}%
\definecolor{currentstroke}{rgb}{0.000000,0.000000,0.000000}%
\pgfsetstrokecolor{currentstroke}%
\pgfsetdash{}{0pt}%
\pgfpathmoveto{\pgfqpoint{1.601139in}{0.821483in}}%
\pgfpathlineto{\pgfqpoint{1.604384in}{0.815281in}}%
\pgfpathlineto{\pgfqpoint{1.607630in}{0.809215in}}%
\pgfpathlineto{\pgfqpoint{1.610880in}{0.803289in}}%
\pgfpathlineto{\pgfqpoint{1.614131in}{0.797507in}}%
\pgfpathlineto{\pgfqpoint{1.605997in}{0.790338in}}%
\pgfpathlineto{\pgfqpoint{1.597413in}{0.783301in}}%
\pgfpathlineto{\pgfqpoint{1.588388in}{0.776404in}}%
\pgfpathlineto{\pgfqpoint{1.578930in}{0.769655in}}%
\pgfpathlineto{\pgfqpoint{1.575930in}{0.775668in}}%
\pgfpathlineto{\pgfqpoint{1.572934in}{0.781825in}}%
\pgfpathlineto{\pgfqpoint{1.569939in}{0.788123in}}%
\pgfpathlineto{\pgfqpoint{1.566947in}{0.794556in}}%
\pgfpathlineto{\pgfqpoint{1.576132in}{0.801080in}}%
\pgfpathlineto{\pgfqpoint{1.584898in}{0.807747in}}%
\pgfpathlineto{\pgfqpoint{1.593236in}{0.814551in}}%
\pgfpathlineto{\pgfqpoint{1.601139in}{0.821483in}}%
\pgfpathclose%
\pgfusepath{fill}%
\end{pgfscope}%
\begin{pgfscope}%
\pgfpathrectangle{\pgfqpoint{0.041670in}{0.041670in}}{\pgfqpoint{2.216660in}{2.216660in}}%
\pgfusepath{clip}%
\pgfsetbuttcap%
\pgfsetroundjoin%
\definecolor{currentfill}{rgb}{0.133743,0.548535,0.553541}%
\pgfsetfillcolor{currentfill}%
\pgfsetlinewidth{0.000000pt}%
\definecolor{currentstroke}{rgb}{0.000000,0.000000,0.000000}%
\pgfsetstrokecolor{currentstroke}%
\pgfsetdash{}{0pt}%
\pgfpathmoveto{\pgfqpoint{0.871899in}{1.207735in}}%
\pgfpathlineto{\pgfqpoint{0.868320in}{1.199425in}}%
\pgfpathlineto{\pgfqpoint{0.864742in}{1.191107in}}%
\pgfpathlineto{\pgfqpoint{0.861167in}{1.182783in}}%
\pgfpathlineto{\pgfqpoint{0.857595in}{1.174456in}}%
\pgfpathlineto{\pgfqpoint{0.854602in}{1.179571in}}%
\pgfpathlineto{\pgfqpoint{0.851941in}{1.184727in}}%
\pgfpathlineto{\pgfqpoint{0.849612in}{1.189919in}}%
\pgfpathlineto{\pgfqpoint{0.847618in}{1.195141in}}%
\pgfpathlineto{\pgfqpoint{0.851290in}{1.203215in}}%
\pgfpathlineto{\pgfqpoint{0.854965in}{1.211286in}}%
\pgfpathlineto{\pgfqpoint{0.858642in}{1.219352in}}%
\pgfpathlineto{\pgfqpoint{0.862322in}{1.227410in}}%
\pgfpathlineto{\pgfqpoint{0.864239in}{1.222442in}}%
\pgfpathlineto{\pgfqpoint{0.866475in}{1.217504in}}%
\pgfpathlineto{\pgfqpoint{0.869029in}{1.212600in}}%
\pgfpathlineto{\pgfqpoint{0.871899in}{1.207735in}}%
\pgfpathclose%
\pgfusepath{fill}%
\end{pgfscope}%
\begin{pgfscope}%
\pgfpathrectangle{\pgfqpoint{0.041670in}{0.041670in}}{\pgfqpoint{2.216660in}{2.216660in}}%
\pgfusepath{clip}%
\pgfsetbuttcap%
\pgfsetroundjoin%
\definecolor{currentfill}{rgb}{0.276194,0.190074,0.493001}%
\pgfsetfillcolor{currentfill}%
\pgfsetlinewidth{0.000000pt}%
\definecolor{currentstroke}{rgb}{0.000000,0.000000,0.000000}%
\pgfsetstrokecolor{currentstroke}%
\pgfsetdash{}{0pt}%
\pgfpathmoveto{\pgfqpoint{0.552903in}{0.807449in}}%
\pgfpathlineto{\pgfqpoint{0.549097in}{0.814538in}}%
\pgfpathlineto{\pgfqpoint{0.545274in}{0.822024in}}%
\pgfpathlineto{\pgfqpoint{0.541435in}{0.829914in}}%
\pgfpathlineto{\pgfqpoint{0.537579in}{0.838214in}}%
\pgfpathlineto{\pgfqpoint{0.529110in}{0.848920in}}%
\pgfpathlineto{\pgfqpoint{0.521328in}{0.859743in}}%
\pgfpathlineto{\pgfqpoint{0.514237in}{0.870674in}}%
\pgfpathlineto{\pgfqpoint{0.507842in}{0.881698in}}%
\pgfpathlineto{\pgfqpoint{0.511852in}{0.873188in}}%
\pgfpathlineto{\pgfqpoint{0.515846in}{0.865086in}}%
\pgfpathlineto{\pgfqpoint{0.519822in}{0.857386in}}%
\pgfpathlineto{\pgfqpoint{0.523782in}{0.850082in}}%
\pgfpathlineto{\pgfqpoint{0.530049in}{0.839272in}}%
\pgfpathlineto{\pgfqpoint{0.536994in}{0.828555in}}%
\pgfpathlineto{\pgfqpoint{0.544614in}{0.817944in}}%
\pgfpathlineto{\pgfqpoint{0.552903in}{0.807449in}}%
\pgfpathclose%
\pgfusepath{fill}%
\end{pgfscope}%
\begin{pgfscope}%
\pgfpathrectangle{\pgfqpoint{0.041670in}{0.041670in}}{\pgfqpoint{2.216660in}{2.216660in}}%
\pgfusepath{clip}%
\pgfsetbuttcap%
\pgfsetroundjoin%
\definecolor{currentfill}{rgb}{0.699415,0.867117,0.175971}%
\pgfsetfillcolor{currentfill}%
\pgfsetlinewidth{0.000000pt}%
\definecolor{currentstroke}{rgb}{0.000000,0.000000,0.000000}%
\pgfsetstrokecolor{currentstroke}%
\pgfsetdash{}{0pt}%
\pgfpathmoveto{\pgfqpoint{1.098053in}{1.622843in}}%
\pgfpathlineto{\pgfqpoint{1.095815in}{1.619757in}}%
\pgfpathlineto{\pgfqpoint{1.093580in}{1.616565in}}%
\pgfpathlineto{\pgfqpoint{1.091346in}{1.613268in}}%
\pgfpathlineto{\pgfqpoint{1.089116in}{1.609867in}}%
\pgfpathlineto{\pgfqpoint{1.093114in}{1.611182in}}%
\pgfpathlineto{\pgfqpoint{1.097196in}{1.612438in}}%
\pgfpathlineto{\pgfqpoint{1.101359in}{1.613632in}}%
\pgfpathlineto{\pgfqpoint{1.105598in}{1.614763in}}%
\pgfpathlineto{\pgfqpoint{1.107422in}{1.618042in}}%
\pgfpathlineto{\pgfqpoint{1.109249in}{1.621216in}}%
\pgfpathlineto{\pgfqpoint{1.111078in}{1.624286in}}%
\pgfpathlineto{\pgfqpoint{1.112909in}{1.627249in}}%
\pgfpathlineto{\pgfqpoint{1.109088in}{1.626230in}}%
\pgfpathlineto{\pgfqpoint{1.105336in}{1.625156in}}%
\pgfpathlineto{\pgfqpoint{1.101657in}{1.624026in}}%
\pgfpathlineto{\pgfqpoint{1.098053in}{1.622843in}}%
\pgfpathclose%
\pgfusepath{fill}%
\end{pgfscope}%
\begin{pgfscope}%
\pgfpathrectangle{\pgfqpoint{0.041670in}{0.041670in}}{\pgfqpoint{2.216660in}{2.216660in}}%
\pgfusepath{clip}%
\pgfsetbuttcap%
\pgfsetroundjoin%
\definecolor{currentfill}{rgb}{0.163625,0.471133,0.558148}%
\pgfsetfillcolor{currentfill}%
\pgfsetlinewidth{0.000000pt}%
\definecolor{currentstroke}{rgb}{0.000000,0.000000,0.000000}%
\pgfsetstrokecolor{currentstroke}%
\pgfsetdash{}{0pt}%
\pgfpathmoveto{\pgfqpoint{1.519370in}{1.145942in}}%
\pgfpathlineto{\pgfqpoint{1.522960in}{1.137698in}}%
\pgfpathlineto{\pgfqpoint{1.526548in}{1.129468in}}%
\pgfpathlineto{\pgfqpoint{1.530134in}{1.121256in}}%
\pgfpathlineto{\pgfqpoint{1.533718in}{1.113064in}}%
\pgfpathlineto{\pgfqpoint{1.530444in}{1.107448in}}%
\pgfpathlineto{\pgfqpoint{1.526813in}{1.101883in}}%
\pgfpathlineto{\pgfqpoint{1.522826in}{1.096375in}}%
\pgfpathlineto{\pgfqpoint{1.518488in}{1.090929in}}%
\pgfpathlineto{\pgfqpoint{1.515047in}{1.099372in}}%
\pgfpathlineto{\pgfqpoint{1.511605in}{1.107834in}}%
\pgfpathlineto{\pgfqpoint{1.508161in}{1.116314in}}%
\pgfpathlineto{\pgfqpoint{1.504716in}{1.124807in}}%
\pgfpathlineto{\pgfqpoint{1.508888in}{1.130007in}}%
\pgfpathlineto{\pgfqpoint{1.512723in}{1.135266in}}%
\pgfpathlineto{\pgfqpoint{1.516219in}{1.140579in}}%
\pgfpathlineto{\pgfqpoint{1.519370in}{1.145942in}}%
\pgfpathclose%
\pgfusepath{fill}%
\end{pgfscope}%
\begin{pgfscope}%
\pgfpathrectangle{\pgfqpoint{0.041670in}{0.041670in}}{\pgfqpoint{2.216660in}{2.216660in}}%
\pgfusepath{clip}%
\pgfsetbuttcap%
\pgfsetroundjoin%
\definecolor{currentfill}{rgb}{0.636902,0.856542,0.216620}%
\pgfsetfillcolor{currentfill}%
\pgfsetlinewidth{0.000000pt}%
\definecolor{currentstroke}{rgb}{0.000000,0.000000,0.000000}%
\pgfsetstrokecolor{currentstroke}%
\pgfsetdash{}{0pt}%
\pgfpathmoveto{\pgfqpoint{1.060614in}{1.597342in}}%
\pgfpathlineto{\pgfqpoint{1.057681in}{1.593524in}}%
\pgfpathlineto{\pgfqpoint{1.054751in}{1.589605in}}%
\pgfpathlineto{\pgfqpoint{1.051824in}{1.585587in}}%
\pgfpathlineto{\pgfqpoint{1.048900in}{1.581471in}}%
\pgfpathlineto{\pgfqpoint{1.052407in}{1.583393in}}%
\pgfpathlineto{\pgfqpoint{1.056038in}{1.585263in}}%
\pgfpathlineto{\pgfqpoint{1.059789in}{1.587076in}}%
\pgfpathlineto{\pgfqpoint{1.063657in}{1.588833in}}%
\pgfpathlineto{\pgfqpoint{1.066249in}{1.592780in}}%
\pgfpathlineto{\pgfqpoint{1.068844in}{1.596630in}}%
\pgfpathlineto{\pgfqpoint{1.071441in}{1.600381in}}%
\pgfpathlineto{\pgfqpoint{1.074042in}{1.604031in}}%
\pgfpathlineto{\pgfqpoint{1.070522in}{1.602435in}}%
\pgfpathlineto{\pgfqpoint{1.067108in}{1.600787in}}%
\pgfpathlineto{\pgfqpoint{1.063804in}{1.599089in}}%
\pgfpathlineto{\pgfqpoint{1.060614in}{1.597342in}}%
\pgfpathclose%
\pgfusepath{fill}%
\end{pgfscope}%
\begin{pgfscope}%
\pgfpathrectangle{\pgfqpoint{0.041670in}{0.041670in}}{\pgfqpoint{2.216660in}{2.216660in}}%
\pgfusepath{clip}%
\pgfsetbuttcap%
\pgfsetroundjoin%
\definecolor{currentfill}{rgb}{0.231674,0.318106,0.544834}%
\pgfsetfillcolor{currentfill}%
\pgfsetlinewidth{0.000000pt}%
\definecolor{currentstroke}{rgb}{0.000000,0.000000,0.000000}%
\pgfsetstrokecolor{currentstroke}%
\pgfsetdash{}{0pt}%
\pgfpathmoveto{\pgfqpoint{0.829395in}{0.963212in}}%
\pgfpathlineto{\pgfqpoint{0.826218in}{0.955013in}}%
\pgfpathlineto{\pgfqpoint{0.823041in}{0.946878in}}%
\pgfpathlineto{\pgfqpoint{0.819865in}{0.938810in}}%
\pgfpathlineto{\pgfqpoint{0.816688in}{0.930813in}}%
\pgfpathlineto{\pgfqpoint{0.809756in}{0.936809in}}%
\pgfpathlineto{\pgfqpoint{0.803210in}{0.942910in}}%
\pgfpathlineto{\pgfqpoint{0.797057in}{0.949109in}}%
\pgfpathlineto{\pgfqpoint{0.791301in}{0.955401in}}%
\pgfpathlineto{\pgfqpoint{0.794688in}{0.963157in}}%
\pgfpathlineto{\pgfqpoint{0.798076in}{0.970983in}}%
\pgfpathlineto{\pgfqpoint{0.801463in}{0.978877in}}%
\pgfpathlineto{\pgfqpoint{0.804851in}{0.986835in}}%
\pgfpathlineto{\pgfqpoint{0.810418in}{0.980790in}}%
\pgfpathlineto{\pgfqpoint{0.816368in}{0.974834in}}%
\pgfpathlineto{\pgfqpoint{0.822695in}{0.968972in}}%
\pgfpathlineto{\pgfqpoint{0.829395in}{0.963212in}}%
\pgfpathclose%
\pgfusepath{fill}%
\end{pgfscope}%
\begin{pgfscope}%
\pgfpathrectangle{\pgfqpoint{0.041670in}{0.041670in}}{\pgfqpoint{2.216660in}{2.216660in}}%
\pgfusepath{clip}%
\pgfsetbuttcap%
\pgfsetroundjoin%
\definecolor{currentfill}{rgb}{0.212395,0.359683,0.551710}%
\pgfsetfillcolor{currentfill}%
\pgfsetlinewidth{0.000000pt}%
\definecolor{currentstroke}{rgb}{0.000000,0.000000,0.000000}%
\pgfsetstrokecolor{currentstroke}%
\pgfsetdash{}{0pt}%
\pgfpathmoveto{\pgfqpoint{1.545965in}{1.024464in}}%
\pgfpathlineto{\pgfqpoint{1.549395in}{1.016338in}}%
\pgfpathlineto{\pgfqpoint{1.552825in}{1.008263in}}%
\pgfpathlineto{\pgfqpoint{1.556253in}{1.000241in}}%
\pgfpathlineto{\pgfqpoint{1.559681in}{0.992278in}}%
\pgfpathlineto{\pgfqpoint{1.554459in}{0.986160in}}%
\pgfpathlineto{\pgfqpoint{1.548850in}{0.980124in}}%
\pgfpathlineto{\pgfqpoint{1.542858in}{0.974178in}}%
\pgfpathlineto{\pgfqpoint{1.536488in}{0.968327in}}%
\pgfpathlineto{\pgfqpoint{1.533259in}{0.976534in}}%
\pgfpathlineto{\pgfqpoint{1.530030in}{0.984799in}}%
\pgfpathlineto{\pgfqpoint{1.526799in}{0.993117in}}%
\pgfpathlineto{\pgfqpoint{1.523569in}{1.001486in}}%
\pgfpathlineto{\pgfqpoint{1.529718in}{1.007099in}}%
\pgfpathlineto{\pgfqpoint{1.535504in}{1.012804in}}%
\pgfpathlineto{\pgfqpoint{1.540921in}{1.018594in}}%
\pgfpathlineto{\pgfqpoint{1.545965in}{1.024464in}}%
\pgfpathclose%
\pgfusepath{fill}%
\end{pgfscope}%
\begin{pgfscope}%
\pgfpathrectangle{\pgfqpoint{0.041670in}{0.041670in}}{\pgfqpoint{2.216660in}{2.216660in}}%
\pgfusepath{clip}%
\pgfsetbuttcap%
\pgfsetroundjoin%
\definecolor{currentfill}{rgb}{0.281477,0.755203,0.432552}%
\pgfsetfillcolor{currentfill}%
\pgfsetlinewidth{0.000000pt}%
\definecolor{currentstroke}{rgb}{0.000000,0.000000,0.000000}%
\pgfsetstrokecolor{currentstroke}%
\pgfsetdash{}{0pt}%
\pgfpathmoveto{\pgfqpoint{0.948378in}{1.439762in}}%
\pgfpathlineto{\pgfqpoint{0.944597in}{1.433154in}}%
\pgfpathlineto{\pgfqpoint{0.940820in}{1.426480in}}%
\pgfpathlineto{\pgfqpoint{0.937046in}{1.419743in}}%
\pgfpathlineto{\pgfqpoint{0.933275in}{1.412944in}}%
\pgfpathlineto{\pgfqpoint{0.933981in}{1.416701in}}%
\pgfpathlineto{\pgfqpoint{0.934931in}{1.420443in}}%
\pgfpathlineto{\pgfqpoint{0.936123in}{1.424167in}}%
\pgfpathlineto{\pgfqpoint{0.937556in}{1.427869in}}%
\pgfpathlineto{\pgfqpoint{0.941253in}{1.434425in}}%
\pgfpathlineto{\pgfqpoint{0.944953in}{1.440921in}}%
\pgfpathlineto{\pgfqpoint{0.948657in}{1.447353in}}%
\pgfpathlineto{\pgfqpoint{0.952365in}{1.453719in}}%
\pgfpathlineto{\pgfqpoint{0.951027in}{1.450257in}}%
\pgfpathlineto{\pgfqpoint{0.949916in}{1.446775in}}%
\pgfpathlineto{\pgfqpoint{0.949032in}{1.443275in}}%
\pgfpathlineto{\pgfqpoint{0.948378in}{1.439762in}}%
\pgfpathclose%
\pgfusepath{fill}%
\end{pgfscope}%
\begin{pgfscope}%
\pgfpathrectangle{\pgfqpoint{0.041670in}{0.041670in}}{\pgfqpoint{2.216660in}{2.216660in}}%
\pgfusepath{clip}%
\pgfsetbuttcap%
\pgfsetroundjoin%
\definecolor{currentfill}{rgb}{0.280255,0.165693,0.476498}%
\pgfsetfillcolor{currentfill}%
\pgfsetlinewidth{0.000000pt}%
\definecolor{currentstroke}{rgb}{0.000000,0.000000,0.000000}%
\pgfsetstrokecolor{currentstroke}%
\pgfsetdash{}{0pt}%
\pgfpathmoveto{\pgfqpoint{1.588178in}{0.847573in}}%
\pgfpathlineto{\pgfqpoint{1.591416in}{0.840865in}}%
\pgfpathlineto{\pgfqpoint{1.594655in}{0.834279in}}%
\pgfpathlineto{\pgfqpoint{1.597896in}{0.827816in}}%
\pgfpathlineto{\pgfqpoint{1.601139in}{0.821483in}}%
\pgfpathlineto{\pgfqpoint{1.593236in}{0.814551in}}%
\pgfpathlineto{\pgfqpoint{1.584898in}{0.807747in}}%
\pgfpathlineto{\pgfqpoint{1.576132in}{0.801080in}}%
\pgfpathlineto{\pgfqpoint{1.566947in}{0.794556in}}%
\pgfpathlineto{\pgfqpoint{1.563956in}{0.801121in}}%
\pgfpathlineto{\pgfqpoint{1.560968in}{0.807814in}}%
\pgfpathlineto{\pgfqpoint{1.557981in}{0.814632in}}%
\pgfpathlineto{\pgfqpoint{1.554996in}{0.821571in}}%
\pgfpathlineto{\pgfqpoint{1.563908in}{0.827870in}}%
\pgfpathlineto{\pgfqpoint{1.572414in}{0.834308in}}%
\pgfpathlineto{\pgfqpoint{1.580507in}{0.840878in}}%
\pgfpathlineto{\pgfqpoint{1.588178in}{0.847573in}}%
\pgfpathclose%
\pgfusepath{fill}%
\end{pgfscope}%
\begin{pgfscope}%
\pgfpathrectangle{\pgfqpoint{0.041670in}{0.041670in}}{\pgfqpoint{2.216660in}{2.216660in}}%
\pgfusepath{clip}%
\pgfsetbuttcap%
\pgfsetroundjoin%
\definecolor{currentfill}{rgb}{0.279566,0.067836,0.391917}%
\pgfsetfillcolor{currentfill}%
\pgfsetlinewidth{0.000000pt}%
\definecolor{currentstroke}{rgb}{0.000000,0.000000,0.000000}%
\pgfsetstrokecolor{currentstroke}%
\pgfsetdash{}{0pt}%
\pgfpathmoveto{\pgfqpoint{0.777975in}{0.741057in}}%
\pgfpathlineto{\pgfqpoint{0.775025in}{0.735774in}}%
\pgfpathlineto{\pgfqpoint{0.772072in}{0.730660in}}%
\pgfpathlineto{\pgfqpoint{0.769115in}{0.725719in}}%
\pgfpathlineto{\pgfqpoint{0.766155in}{0.720954in}}%
\pgfpathlineto{\pgfqpoint{0.755747in}{0.728004in}}%
\pgfpathlineto{\pgfqpoint{0.745792in}{0.735220in}}%
\pgfpathlineto{\pgfqpoint{0.736299in}{0.742591in}}%
\pgfpathlineto{\pgfqpoint{0.727277in}{0.750110in}}%
\pgfpathlineto{\pgfqpoint{0.730503in}{0.754649in}}%
\pgfpathlineto{\pgfqpoint{0.733725in}{0.759364in}}%
\pgfpathlineto{\pgfqpoint{0.736943in}{0.764251in}}%
\pgfpathlineto{\pgfqpoint{0.740158in}{0.769307in}}%
\pgfpathlineto{\pgfqpoint{0.748936in}{0.762021in}}%
\pgfpathlineto{\pgfqpoint{0.758170in}{0.754878in}}%
\pgfpathlineto{\pgfqpoint{0.767853in}{0.747888in}}%
\pgfpathlineto{\pgfqpoint{0.777975in}{0.741057in}}%
\pgfpathclose%
\pgfusepath{fill}%
\end{pgfscope}%
\begin{pgfscope}%
\pgfpathrectangle{\pgfqpoint{0.041670in}{0.041670in}}{\pgfqpoint{2.216660in}{2.216660in}}%
\pgfusepath{clip}%
\pgfsetbuttcap%
\pgfsetroundjoin%
\definecolor{currentfill}{rgb}{0.762373,0.876424,0.137064}%
\pgfsetfillcolor{currentfill}%
\pgfsetlinewidth{0.000000pt}%
\definecolor{currentstroke}{rgb}{0.000000,0.000000,0.000000}%
\pgfsetstrokecolor{currentstroke}%
\pgfsetdash{}{0pt}%
\pgfpathmoveto{\pgfqpoint{1.164598in}{1.644721in}}%
\pgfpathlineto{\pgfqpoint{1.164126in}{1.642408in}}%
\pgfpathlineto{\pgfqpoint{1.163654in}{1.639982in}}%
\pgfpathlineto{\pgfqpoint{1.163182in}{1.637446in}}%
\pgfpathlineto{\pgfqpoint{1.162711in}{1.634800in}}%
\pgfpathlineto{\pgfqpoint{1.167069in}{1.635022in}}%
\pgfpathlineto{\pgfqpoint{1.171438in}{1.635180in}}%
\pgfpathlineto{\pgfqpoint{1.175816in}{1.635273in}}%
\pgfpathlineto{\pgfqpoint{1.180198in}{1.635302in}}%
\pgfpathlineto{\pgfqpoint{1.180192in}{1.637934in}}%
\pgfpathlineto{\pgfqpoint{1.180185in}{1.640456in}}%
\pgfpathlineto{\pgfqpoint{1.180178in}{1.642868in}}%
\pgfpathlineto{\pgfqpoint{1.180172in}{1.645168in}}%
\pgfpathlineto{\pgfqpoint{1.176269in}{1.645142in}}%
\pgfpathlineto{\pgfqpoint{1.172370in}{1.645059in}}%
\pgfpathlineto{\pgfqpoint{1.168479in}{1.644919in}}%
\pgfpathlineto{\pgfqpoint{1.164598in}{1.644721in}}%
\pgfpathclose%
\pgfusepath{fill}%
\end{pgfscope}%
\begin{pgfscope}%
\pgfpathrectangle{\pgfqpoint{0.041670in}{0.041670in}}{\pgfqpoint{2.216660in}{2.216660in}}%
\pgfusepath{clip}%
\pgfsetbuttcap%
\pgfsetroundjoin%
\definecolor{currentfill}{rgb}{0.762373,0.876424,0.137064}%
\pgfsetfillcolor{currentfill}%
\pgfsetlinewidth{0.000000pt}%
\definecolor{currentstroke}{rgb}{0.000000,0.000000,0.000000}%
\pgfsetstrokecolor{currentstroke}%
\pgfsetdash{}{0pt}%
\pgfpathmoveto{\pgfqpoint{1.180172in}{1.645168in}}%
\pgfpathlineto{\pgfqpoint{1.180178in}{1.642868in}}%
\pgfpathlineto{\pgfqpoint{1.180185in}{1.640456in}}%
\pgfpathlineto{\pgfqpoint{1.180192in}{1.637934in}}%
\pgfpathlineto{\pgfqpoint{1.180198in}{1.635302in}}%
\pgfpathlineto{\pgfqpoint{1.184580in}{1.635266in}}%
\pgfpathlineto{\pgfqpoint{1.188957in}{1.635166in}}%
\pgfpathlineto{\pgfqpoint{1.193326in}{1.635001in}}%
\pgfpathlineto{\pgfqpoint{1.197681in}{1.634771in}}%
\pgfpathlineto{\pgfqpoint{1.197197in}{1.637418in}}%
\pgfpathlineto{\pgfqpoint{1.196713in}{1.639955in}}%
\pgfpathlineto{\pgfqpoint{1.196228in}{1.642381in}}%
\pgfpathlineto{\pgfqpoint{1.195742in}{1.644696in}}%
\pgfpathlineto{\pgfqpoint{1.191863in}{1.644900in}}%
\pgfpathlineto{\pgfqpoint{1.187972in}{1.645047in}}%
\pgfpathlineto{\pgfqpoint{1.184074in}{1.645136in}}%
\pgfpathlineto{\pgfqpoint{1.180172in}{1.645168in}}%
\pgfpathclose%
\pgfusepath{fill}%
\end{pgfscope}%
\begin{pgfscope}%
\pgfpathrectangle{\pgfqpoint{0.041670in}{0.041670in}}{\pgfqpoint{2.216660in}{2.216660in}}%
\pgfusepath{clip}%
\pgfsetbuttcap%
\pgfsetroundjoin%
\definecolor{currentfill}{rgb}{0.267004,0.004874,0.329415}%
\pgfsetfillcolor{currentfill}%
\pgfsetlinewidth{0.000000pt}%
\definecolor{currentstroke}{rgb}{0.000000,0.000000,0.000000}%
\pgfsetstrokecolor{currentstroke}%
\pgfsetdash{}{0pt}%
\pgfpathmoveto{\pgfqpoint{1.679938in}{0.718423in}}%
\pgfpathlineto{\pgfqpoint{1.683282in}{0.716630in}}%
\pgfpathlineto{\pgfqpoint{1.686633in}{0.715077in}}%
\pgfpathlineto{\pgfqpoint{1.689991in}{0.713767in}}%
\pgfpathlineto{\pgfqpoint{1.693356in}{0.712707in}}%
\pgfpathlineto{\pgfqpoint{1.683828in}{0.704129in}}%
\pgfpathlineto{\pgfqpoint{1.673763in}{0.695707in}}%
\pgfpathlineto{\pgfqpoint{1.663171in}{0.687450in}}%
\pgfpathlineto{\pgfqpoint{1.652060in}{0.679367in}}%
\pgfpathlineto{\pgfqpoint{1.648952in}{0.680651in}}%
\pgfpathlineto{\pgfqpoint{1.645851in}{0.682185in}}%
\pgfpathlineto{\pgfqpoint{1.642756in}{0.683964in}}%
\pgfpathlineto{\pgfqpoint{1.639667in}{0.685983in}}%
\pgfpathlineto{\pgfqpoint{1.650500in}{0.693846in}}%
\pgfpathlineto{\pgfqpoint{1.660829in}{0.701880in}}%
\pgfpathlineto{\pgfqpoint{1.670644in}{0.710076in}}%
\pgfpathlineto{\pgfqpoint{1.679938in}{0.718423in}}%
\pgfpathclose%
\pgfusepath{fill}%
\end{pgfscope}%
\begin{pgfscope}%
\pgfpathrectangle{\pgfqpoint{0.041670in}{0.041670in}}{\pgfqpoint{2.216660in}{2.216660in}}%
\pgfusepath{clip}%
\pgfsetbuttcap%
\pgfsetroundjoin%
\definecolor{currentfill}{rgb}{0.274952,0.037752,0.364543}%
\pgfsetfillcolor{currentfill}%
\pgfsetlinewidth{0.000000pt}%
\definecolor{currentstroke}{rgb}{0.000000,0.000000,0.000000}%
\pgfsetstrokecolor{currentstroke}%
\pgfsetdash{}{0pt}%
\pgfpathmoveto{\pgfqpoint{0.766155in}{0.720954in}}%
\pgfpathlineto{\pgfqpoint{0.763191in}{0.716369in}}%
\pgfpathlineto{\pgfqpoint{0.760223in}{0.711971in}}%
\pgfpathlineto{\pgfqpoint{0.757251in}{0.707761in}}%
\pgfpathlineto{\pgfqpoint{0.754275in}{0.703746in}}%
\pgfpathlineto{\pgfqpoint{0.743580in}{0.711016in}}%
\pgfpathlineto{\pgfqpoint{0.733352in}{0.718455in}}%
\pgfpathlineto{\pgfqpoint{0.723600in}{0.726055in}}%
\pgfpathlineto{\pgfqpoint{0.714333in}{0.733807in}}%
\pgfpathlineto{\pgfqpoint{0.717576in}{0.737596in}}%
\pgfpathlineto{\pgfqpoint{0.720814in}{0.741580in}}%
\pgfpathlineto{\pgfqpoint{0.724048in}{0.745753in}}%
\pgfpathlineto{\pgfqpoint{0.727277in}{0.750110in}}%
\pgfpathlineto{\pgfqpoint{0.736299in}{0.742591in}}%
\pgfpathlineto{\pgfqpoint{0.745792in}{0.735220in}}%
\pgfpathlineto{\pgfqpoint{0.755747in}{0.728004in}}%
\pgfpathlineto{\pgfqpoint{0.766155in}{0.720954in}}%
\pgfpathclose%
\pgfusepath{fill}%
\end{pgfscope}%
\begin{pgfscope}%
\pgfpathrectangle{\pgfqpoint{0.041670in}{0.041670in}}{\pgfqpoint{2.216660in}{2.216660in}}%
\pgfusepath{clip}%
\pgfsetbuttcap%
\pgfsetroundjoin%
\definecolor{currentfill}{rgb}{0.134692,0.658636,0.517649}%
\pgfsetfillcolor{currentfill}%
\pgfsetlinewidth{0.000000pt}%
\definecolor{currentstroke}{rgb}{0.000000,0.000000,0.000000}%
\pgfsetstrokecolor{currentstroke}%
\pgfsetdash{}{0pt}%
\pgfpathmoveto{\pgfqpoint{1.457438in}{1.343314in}}%
\pgfpathlineto{\pgfqpoint{1.461196in}{1.335867in}}%
\pgfpathlineto{\pgfqpoint{1.464951in}{1.328382in}}%
\pgfpathlineto{\pgfqpoint{1.468703in}{1.320860in}}%
\pgfpathlineto{\pgfqpoint{1.472452in}{1.313304in}}%
\pgfpathlineto{\pgfqpoint{1.472120in}{1.308776in}}%
\pgfpathlineto{\pgfqpoint{1.471496in}{1.304253in}}%
\pgfpathlineto{\pgfqpoint{1.470580in}{1.299738in}}%
\pgfpathlineto{\pgfqpoint{1.469371in}{1.295237in}}%
\pgfpathlineto{\pgfqpoint{1.465652in}{1.303045in}}%
\pgfpathlineto{\pgfqpoint{1.461931in}{1.310818in}}%
\pgfpathlineto{\pgfqpoint{1.458206in}{1.318555in}}%
\pgfpathlineto{\pgfqpoint{1.454479in}{1.326252in}}%
\pgfpathlineto{\pgfqpoint{1.455634in}{1.330502in}}%
\pgfpathlineto{\pgfqpoint{1.456513in}{1.334765in}}%
\pgfpathlineto{\pgfqpoint{1.457114in}{1.339037in}}%
\pgfpathlineto{\pgfqpoint{1.457438in}{1.343314in}}%
\pgfpathclose%
\pgfusepath{fill}%
\end{pgfscope}%
\begin{pgfscope}%
\pgfpathrectangle{\pgfqpoint{0.041670in}{0.041670in}}{\pgfqpoint{2.216660in}{2.216660in}}%
\pgfusepath{clip}%
\pgfsetbuttcap%
\pgfsetroundjoin%
\definecolor{currentfill}{rgb}{0.277941,0.056324,0.381191}%
\pgfsetfillcolor{currentfill}%
\pgfsetlinewidth{0.000000pt}%
\definecolor{currentstroke}{rgb}{0.000000,0.000000,0.000000}%
\pgfsetstrokecolor{currentstroke}%
\pgfsetdash{}{0pt}%
\pgfpathmoveto{\pgfqpoint{0.634588in}{0.713028in}}%
\pgfpathlineto{\pgfqpoint{0.631157in}{0.715626in}}%
\pgfpathlineto{\pgfqpoint{0.627716in}{0.718547in}}%
\pgfpathlineto{\pgfqpoint{0.624264in}{0.721798in}}%
\pgfpathlineto{\pgfqpoint{0.620799in}{0.725384in}}%
\pgfpathlineto{\pgfqpoint{0.610386in}{0.734873in}}%
\pgfpathlineto{\pgfqpoint{0.600581in}{0.744521in}}%
\pgfpathlineto{\pgfqpoint{0.591393in}{0.754314in}}%
\pgfpathlineto{\pgfqpoint{0.582828in}{0.764243in}}%
\pgfpathlineto{\pgfqpoint{0.586508in}{0.760437in}}%
\pgfpathlineto{\pgfqpoint{0.590176in}{0.756964in}}%
\pgfpathlineto{\pgfqpoint{0.593832in}{0.753820in}}%
\pgfpathlineto{\pgfqpoint{0.597477in}{0.750998in}}%
\pgfpathlineto{\pgfqpoint{0.605851in}{0.741295in}}%
\pgfpathlineto{\pgfqpoint{0.614832in}{0.731726in}}%
\pgfpathlineto{\pgfqpoint{0.624414in}{0.722300in}}%
\pgfpathlineto{\pgfqpoint{0.634588in}{0.713028in}}%
\pgfpathclose%
\pgfusepath{fill}%
\end{pgfscope}%
\begin{pgfscope}%
\pgfpathrectangle{\pgfqpoint{0.041670in}{0.041670in}}{\pgfqpoint{2.216660in}{2.216660in}}%
\pgfusepath{clip}%
\pgfsetbuttcap%
\pgfsetroundjoin%
\definecolor{currentfill}{rgb}{0.282327,0.094955,0.417331}%
\pgfsetfillcolor{currentfill}%
\pgfsetlinewidth{0.000000pt}%
\definecolor{currentstroke}{rgb}{0.000000,0.000000,0.000000}%
\pgfsetstrokecolor{currentstroke}%
\pgfsetdash{}{0pt}%
\pgfpathmoveto{\pgfqpoint{0.789744in}{0.763787in}}%
\pgfpathlineto{\pgfqpoint{0.786806in}{0.757872in}}%
\pgfpathlineto{\pgfqpoint{0.783865in}{0.752110in}}%
\pgfpathlineto{\pgfqpoint{0.780921in}{0.746503in}}%
\pgfpathlineto{\pgfqpoint{0.777975in}{0.741057in}}%
\pgfpathlineto{\pgfqpoint{0.767853in}{0.747888in}}%
\pgfpathlineto{\pgfqpoint{0.758170in}{0.754878in}}%
\pgfpathlineto{\pgfqpoint{0.748936in}{0.762021in}}%
\pgfpathlineto{\pgfqpoint{0.740158in}{0.769307in}}%
\pgfpathlineto{\pgfqpoint{0.743369in}{0.774526in}}%
\pgfpathlineto{\pgfqpoint{0.746578in}{0.779906in}}%
\pgfpathlineto{\pgfqpoint{0.749784in}{0.785441in}}%
\pgfpathlineto{\pgfqpoint{0.752987in}{0.791128in}}%
\pgfpathlineto{\pgfqpoint{0.761521in}{0.784076in}}%
\pgfpathlineto{\pgfqpoint{0.770497in}{0.777163in}}%
\pgfpathlineto{\pgfqpoint{0.779908in}{0.770397in}}%
\pgfpathlineto{\pgfqpoint{0.789744in}{0.763787in}}%
\pgfpathclose%
\pgfusepath{fill}%
\end{pgfscope}%
\begin{pgfscope}%
\pgfpathrectangle{\pgfqpoint{0.041670in}{0.041670in}}{\pgfqpoint{2.216660in}{2.216660in}}%
\pgfusepath{clip}%
\pgfsetbuttcap%
\pgfsetroundjoin%
\definecolor{currentfill}{rgb}{0.762373,0.876424,0.137064}%
\pgfsetfillcolor{currentfill}%
\pgfsetlinewidth{0.000000pt}%
\definecolor{currentstroke}{rgb}{0.000000,0.000000,0.000000}%
\pgfsetstrokecolor{currentstroke}%
\pgfsetdash{}{0pt}%
\pgfpathmoveto{\pgfqpoint{1.149263in}{1.643363in}}%
\pgfpathlineto{\pgfqpoint{1.148318in}{1.641007in}}%
\pgfpathlineto{\pgfqpoint{1.147375in}{1.638539in}}%
\pgfpathlineto{\pgfqpoint{1.146432in}{1.635960in}}%
\pgfpathlineto{\pgfqpoint{1.145491in}{1.633273in}}%
\pgfpathlineto{\pgfqpoint{1.149757in}{1.633750in}}%
\pgfpathlineto{\pgfqpoint{1.154051in}{1.634163in}}%
\pgfpathlineto{\pgfqpoint{1.158371in}{1.634514in}}%
\pgfpathlineto{\pgfqpoint{1.162711in}{1.634800in}}%
\pgfpathlineto{\pgfqpoint{1.163182in}{1.637446in}}%
\pgfpathlineto{\pgfqpoint{1.163654in}{1.639982in}}%
\pgfpathlineto{\pgfqpoint{1.164126in}{1.642408in}}%
\pgfpathlineto{\pgfqpoint{1.164598in}{1.644721in}}%
\pgfpathlineto{\pgfqpoint{1.160732in}{1.644467in}}%
\pgfpathlineto{\pgfqpoint{1.156886in}{1.644155in}}%
\pgfpathlineto{\pgfqpoint{1.153061in}{1.643787in}}%
\pgfpathlineto{\pgfqpoint{1.149263in}{1.643363in}}%
\pgfpathclose%
\pgfusepath{fill}%
\end{pgfscope}%
\begin{pgfscope}%
\pgfpathrectangle{\pgfqpoint{0.041670in}{0.041670in}}{\pgfqpoint{2.216660in}{2.216660in}}%
\pgfusepath{clip}%
\pgfsetbuttcap%
\pgfsetroundjoin%
\definecolor{currentfill}{rgb}{0.762373,0.876424,0.137064}%
\pgfsetfillcolor{currentfill}%
\pgfsetlinewidth{0.000000pt}%
\definecolor{currentstroke}{rgb}{0.000000,0.000000,0.000000}%
\pgfsetstrokecolor{currentstroke}%
\pgfsetdash{}{0pt}%
\pgfpathmoveto{\pgfqpoint{1.195742in}{1.644696in}}%
\pgfpathlineto{\pgfqpoint{1.196228in}{1.642381in}}%
\pgfpathlineto{\pgfqpoint{1.196713in}{1.639955in}}%
\pgfpathlineto{\pgfqpoint{1.197197in}{1.637418in}}%
\pgfpathlineto{\pgfqpoint{1.197681in}{1.634771in}}%
\pgfpathlineto{\pgfqpoint{1.202020in}{1.634478in}}%
\pgfpathlineto{\pgfqpoint{1.206337in}{1.634121in}}%
\pgfpathlineto{\pgfqpoint{1.210629in}{1.633700in}}%
\pgfpathlineto{\pgfqpoint{1.214891in}{1.633216in}}%
\pgfpathlineto{\pgfqpoint{1.213936in}{1.635905in}}%
\pgfpathlineto{\pgfqpoint{1.212981in}{1.638485in}}%
\pgfpathlineto{\pgfqpoint{1.212025in}{1.640955in}}%
\pgfpathlineto{\pgfqpoint{1.211067in}{1.643313in}}%
\pgfpathlineto{\pgfqpoint{1.207272in}{1.643743in}}%
\pgfpathlineto{\pgfqpoint{1.203450in}{1.644117in}}%
\pgfpathlineto{\pgfqpoint{1.199606in}{1.644435in}}%
\pgfpathlineto{\pgfqpoint{1.195742in}{1.644696in}}%
\pgfpathclose%
\pgfusepath{fill}%
\end{pgfscope}%
\begin{pgfscope}%
\pgfpathrectangle{\pgfqpoint{0.041670in}{0.041670in}}{\pgfqpoint{2.216660in}{2.216660in}}%
\pgfusepath{clip}%
\pgfsetbuttcap%
\pgfsetroundjoin%
\definecolor{currentfill}{rgb}{0.487026,0.823929,0.312321}%
\pgfsetfillcolor{currentfill}%
\pgfsetlinewidth{0.000000pt}%
\definecolor{currentstroke}{rgb}{0.000000,0.000000,0.000000}%
\pgfsetstrokecolor{currentstroke}%
\pgfsetdash{}{0pt}%
\pgfpathmoveto{\pgfqpoint{1.345652in}{1.547621in}}%
\pgfpathlineto{\pgfqpoint{1.349037in}{1.542693in}}%
\pgfpathlineto{\pgfqpoint{1.352419in}{1.537676in}}%
\pgfpathlineto{\pgfqpoint{1.355797in}{1.532573in}}%
\pgfpathlineto{\pgfqpoint{1.359171in}{1.527385in}}%
\pgfpathlineto{\pgfqpoint{1.361988in}{1.524676in}}%
\pgfpathlineto{\pgfqpoint{1.364626in}{1.521925in}}%
\pgfpathlineto{\pgfqpoint{1.367084in}{1.519135in}}%
\pgfpathlineto{\pgfqpoint{1.369358in}{1.516307in}}%
\pgfpathlineto{\pgfqpoint{1.365786in}{1.521712in}}%
\pgfpathlineto{\pgfqpoint{1.362210in}{1.527033in}}%
\pgfpathlineto{\pgfqpoint{1.358631in}{1.532266in}}%
\pgfpathlineto{\pgfqpoint{1.355049in}{1.537411in}}%
\pgfpathlineto{\pgfqpoint{1.352952in}{1.540017in}}%
\pgfpathlineto{\pgfqpoint{1.350685in}{1.542589in}}%
\pgfpathlineto{\pgfqpoint{1.348251in}{1.545124in}}%
\pgfpathlineto{\pgfqpoint{1.345652in}{1.547621in}}%
\pgfpathclose%
\pgfusepath{fill}%
\end{pgfscope}%
\begin{pgfscope}%
\pgfpathrectangle{\pgfqpoint{0.041670in}{0.041670in}}{\pgfqpoint{2.216660in}{2.216660in}}%
\pgfusepath{clip}%
\pgfsetbuttcap%
\pgfsetroundjoin%
\definecolor{currentfill}{rgb}{0.271305,0.019942,0.347269}%
\pgfsetfillcolor{currentfill}%
\pgfsetlinewidth{0.000000pt}%
\definecolor{currentstroke}{rgb}{0.000000,0.000000,0.000000}%
\pgfsetstrokecolor{currentstroke}%
\pgfsetdash{}{0pt}%
\pgfpathmoveto{\pgfqpoint{0.754275in}{0.703746in}}%
\pgfpathlineto{\pgfqpoint{0.751294in}{0.699929in}}%
\pgfpathlineto{\pgfqpoint{0.748308in}{0.696315in}}%
\pgfpathlineto{\pgfqpoint{0.745318in}{0.692908in}}%
\pgfpathlineto{\pgfqpoint{0.742322in}{0.689713in}}%
\pgfpathlineto{\pgfqpoint{0.731340in}{0.697201in}}%
\pgfpathlineto{\pgfqpoint{0.720838in}{0.704863in}}%
\pgfpathlineto{\pgfqpoint{0.710826in}{0.712690in}}%
\pgfpathlineto{\pgfqpoint{0.701314in}{0.720673in}}%
\pgfpathlineto{\pgfqpoint{0.704577in}{0.723643in}}%
\pgfpathlineto{\pgfqpoint{0.707834in}{0.726826in}}%
\pgfpathlineto{\pgfqpoint{0.711086in}{0.730215in}}%
\pgfpathlineto{\pgfqpoint{0.714333in}{0.733807in}}%
\pgfpathlineto{\pgfqpoint{0.723600in}{0.726055in}}%
\pgfpathlineto{\pgfqpoint{0.733352in}{0.718455in}}%
\pgfpathlineto{\pgfqpoint{0.743580in}{0.711016in}}%
\pgfpathlineto{\pgfqpoint{0.754275in}{0.703746in}}%
\pgfpathclose%
\pgfusepath{fill}%
\end{pgfscope}%
\begin{pgfscope}%
\pgfpathrectangle{\pgfqpoint{0.041670in}{0.041670in}}{\pgfqpoint{2.216660in}{2.216660in}}%
\pgfusepath{clip}%
\pgfsetbuttcap%
\pgfsetroundjoin%
\definecolor{currentfill}{rgb}{0.122606,0.585371,0.546557}%
\pgfsetfillcolor{currentfill}%
\pgfsetlinewidth{0.000000pt}%
\definecolor{currentstroke}{rgb}{0.000000,0.000000,0.000000}%
\pgfsetstrokecolor{currentstroke}%
\pgfsetdash{}{0pt}%
\pgfpathmoveto{\pgfqpoint{1.484219in}{1.263713in}}%
\pgfpathlineto{\pgfqpoint{1.487923in}{1.255772in}}%
\pgfpathlineto{\pgfqpoint{1.491626in}{1.247812in}}%
\pgfpathlineto{\pgfqpoint{1.495325in}{1.239835in}}%
\pgfpathlineto{\pgfqpoint{1.499022in}{1.231845in}}%
\pgfpathlineto{\pgfqpoint{1.497390in}{1.226856in}}%
\pgfpathlineto{\pgfqpoint{1.495438in}{1.221892in}}%
\pgfpathlineto{\pgfqpoint{1.493166in}{1.216957in}}%
\pgfpathlineto{\pgfqpoint{1.490577in}{1.212057in}}%
\pgfpathlineto{\pgfqpoint{1.486968in}{1.220300in}}%
\pgfpathlineto{\pgfqpoint{1.483356in}{1.228529in}}%
\pgfpathlineto{\pgfqpoint{1.479742in}{1.236741in}}%
\pgfpathlineto{\pgfqpoint{1.476126in}{1.244934in}}%
\pgfpathlineto{\pgfqpoint{1.478605in}{1.249584in}}%
\pgfpathlineto{\pgfqpoint{1.480781in}{1.254267in}}%
\pgfpathlineto{\pgfqpoint{1.482652in}{1.258978in}}%
\pgfpathlineto{\pgfqpoint{1.484219in}{1.263713in}}%
\pgfpathclose%
\pgfusepath{fill}%
\end{pgfscope}%
\begin{pgfscope}%
\pgfpathrectangle{\pgfqpoint{0.041670in}{0.041670in}}{\pgfqpoint{2.216660in}{2.216660in}}%
\pgfusepath{clip}%
\pgfsetbuttcap%
\pgfsetroundjoin%
\definecolor{currentfill}{rgb}{0.565498,0.842430,0.262877}%
\pgfsetfillcolor{currentfill}%
\pgfsetlinewidth{0.000000pt}%
\definecolor{currentstroke}{rgb}{0.000000,0.000000,0.000000}%
\pgfsetstrokecolor{currentstroke}%
\pgfsetdash{}{0pt}%
\pgfpathmoveto{\pgfqpoint{1.321090in}{1.575167in}}%
\pgfpathlineto{\pgfqpoint{1.324238in}{1.570809in}}%
\pgfpathlineto{\pgfqpoint{1.327384in}{1.566357in}}%
\pgfpathlineto{\pgfqpoint{1.330526in}{1.561812in}}%
\pgfpathlineto{\pgfqpoint{1.333664in}{1.557175in}}%
\pgfpathlineto{\pgfqpoint{1.336894in}{1.554856in}}%
\pgfpathlineto{\pgfqpoint{1.339971in}{1.552489in}}%
\pgfpathlineto{\pgfqpoint{1.342891in}{1.550077in}}%
\pgfpathlineto{\pgfqpoint{1.345652in}{1.547621in}}%
\pgfpathlineto{\pgfqpoint{1.342264in}{1.552460in}}%
\pgfpathlineto{\pgfqpoint{1.338872in}{1.557207in}}%
\pgfpathlineto{\pgfqpoint{1.335477in}{1.561860in}}%
\pgfpathlineto{\pgfqpoint{1.332079in}{1.566419in}}%
\pgfpathlineto{\pgfqpoint{1.329548in}{1.568667in}}%
\pgfpathlineto{\pgfqpoint{1.326872in}{1.570876in}}%
\pgfpathlineto{\pgfqpoint{1.324051in}{1.573043in}}%
\pgfpathlineto{\pgfqpoint{1.321090in}{1.575167in}}%
\pgfpathclose%
\pgfusepath{fill}%
\end{pgfscope}%
\begin{pgfscope}%
\pgfpathrectangle{\pgfqpoint{0.041670in}{0.041670in}}{\pgfqpoint{2.216660in}{2.216660in}}%
\pgfusepath{clip}%
\pgfsetbuttcap%
\pgfsetroundjoin%
\definecolor{currentfill}{rgb}{0.220124,0.725509,0.466226}%
\pgfsetfillcolor{currentfill}%
\pgfsetlinewidth{0.000000pt}%
\definecolor{currentstroke}{rgb}{0.000000,0.000000,0.000000}%
\pgfsetstrokecolor{currentstroke}%
\pgfsetdash{}{0pt}%
\pgfpathmoveto{\pgfqpoint{1.426019in}{1.416284in}}%
\pgfpathlineto{\pgfqpoint{1.429775in}{1.409480in}}%
\pgfpathlineto{\pgfqpoint{1.433528in}{1.402618in}}%
\pgfpathlineto{\pgfqpoint{1.437277in}{1.395702in}}%
\pgfpathlineto{\pgfqpoint{1.441022in}{1.388733in}}%
\pgfpathlineto{\pgfqpoint{1.441751in}{1.384729in}}%
\pgfpathlineto{\pgfqpoint{1.442220in}{1.380713in}}%
\pgfpathlineto{\pgfqpoint{1.442428in}{1.376690in}}%
\pgfpathlineto{\pgfqpoint{1.442374in}{1.372663in}}%
\pgfpathlineto{\pgfqpoint{1.438600in}{1.379879in}}%
\pgfpathlineto{\pgfqpoint{1.434823in}{1.387043in}}%
\pgfpathlineto{\pgfqpoint{1.431043in}{1.394151in}}%
\pgfpathlineto{\pgfqpoint{1.427259in}{1.401201in}}%
\pgfpathlineto{\pgfqpoint{1.427319in}{1.404980in}}%
\pgfpathlineto{\pgfqpoint{1.427131in}{1.408756in}}%
\pgfpathlineto{\pgfqpoint{1.426698in}{1.412525in}}%
\pgfpathlineto{\pgfqpoint{1.426019in}{1.416284in}}%
\pgfpathclose%
\pgfusepath{fill}%
\end{pgfscope}%
\begin{pgfscope}%
\pgfpathrectangle{\pgfqpoint{0.041670in}{0.041670in}}{\pgfqpoint{2.216660in}{2.216660in}}%
\pgfusepath{clip}%
\pgfsetbuttcap%
\pgfsetroundjoin%
\definecolor{currentfill}{rgb}{0.699415,0.867117,0.175971}%
\pgfsetfillcolor{currentfill}%
\pgfsetlinewidth{0.000000pt}%
\definecolor{currentstroke}{rgb}{0.000000,0.000000,0.000000}%
\pgfsetstrokecolor{currentstroke}%
\pgfsetdash{}{0pt}%
\pgfpathmoveto{\pgfqpoint{1.258657in}{1.623897in}}%
\pgfpathlineto{\pgfqpoint{1.260808in}{1.620841in}}%
\pgfpathlineto{\pgfqpoint{1.262956in}{1.617679in}}%
\pgfpathlineto{\pgfqpoint{1.265101in}{1.614411in}}%
\pgfpathlineto{\pgfqpoint{1.267244in}{1.611039in}}%
\pgfpathlineto{\pgfqpoint{1.271233in}{1.609717in}}%
\pgfpathlineto{\pgfqpoint{1.275133in}{1.608337in}}%
\pgfpathlineto{\pgfqpoint{1.278941in}{1.606899in}}%
\pgfpathlineto{\pgfqpoint{1.282652in}{1.605405in}}%
\pgfpathlineto{\pgfqpoint{1.280129in}{1.608917in}}%
\pgfpathlineto{\pgfqpoint{1.277602in}{1.612327in}}%
\pgfpathlineto{\pgfqpoint{1.275074in}{1.615630in}}%
\pgfpathlineto{\pgfqpoint{1.272542in}{1.618828in}}%
\pgfpathlineto{\pgfqpoint{1.269198in}{1.620172in}}%
\pgfpathlineto{\pgfqpoint{1.265767in}{1.621466in}}%
\pgfpathlineto{\pgfqpoint{1.262252in}{1.622708in}}%
\pgfpathlineto{\pgfqpoint{1.258657in}{1.623897in}}%
\pgfpathclose%
\pgfusepath{fill}%
\end{pgfscope}%
\begin{pgfscope}%
\pgfpathrectangle{\pgfqpoint{0.041670in}{0.041670in}}{\pgfqpoint{2.216660in}{2.216660in}}%
\pgfusepath{clip}%
\pgfsetbuttcap%
\pgfsetroundjoin%
\definecolor{currentfill}{rgb}{0.282327,0.094955,0.417331}%
\pgfsetfillcolor{currentfill}%
\pgfsetlinewidth{0.000000pt}%
\definecolor{currentstroke}{rgb}{0.000000,0.000000,0.000000}%
\pgfsetstrokecolor{currentstroke}%
\pgfsetdash{}{0pt}%
\pgfpathmoveto{\pgfqpoint{1.784167in}{0.773174in}}%
\pgfpathlineto{\pgfqpoint{1.787899in}{0.777371in}}%
\pgfpathlineto{\pgfqpoint{1.791644in}{0.781913in}}%
\pgfpathlineto{\pgfqpoint{1.795402in}{0.786807in}}%
\pgfpathlineto{\pgfqpoint{1.799174in}{0.792059in}}%
\pgfpathlineto{\pgfqpoint{1.790991in}{0.781796in}}%
\pgfpathlineto{\pgfqpoint{1.782161in}{0.771660in}}%
\pgfpathlineto{\pgfqpoint{1.772693in}{0.761663in}}%
\pgfpathlineto{\pgfqpoint{1.762593in}{0.751817in}}%
\pgfpathlineto{\pgfqpoint{1.759024in}{0.746783in}}%
\pgfpathlineto{\pgfqpoint{1.755469in}{0.742108in}}%
\pgfpathlineto{\pgfqpoint{1.751926in}{0.737786in}}%
\pgfpathlineto{\pgfqpoint{1.748396in}{0.733811in}}%
\pgfpathlineto{\pgfqpoint{1.758269in}{0.743441in}}%
\pgfpathlineto{\pgfqpoint{1.767527in}{0.753219in}}%
\pgfpathlineto{\pgfqpoint{1.776161in}{0.763134in}}%
\pgfpathlineto{\pgfqpoint{1.784167in}{0.773174in}}%
\pgfpathclose%
\pgfusepath{fill}%
\end{pgfscope}%
\begin{pgfscope}%
\pgfpathrectangle{\pgfqpoint{0.041670in}{0.041670in}}{\pgfqpoint{2.216660in}{2.216660in}}%
\pgfusepath{clip}%
\pgfsetbuttcap%
\pgfsetroundjoin%
\definecolor{currentfill}{rgb}{0.163625,0.471133,0.558148}%
\pgfsetfillcolor{currentfill}%
\pgfsetlinewidth{0.000000pt}%
\definecolor{currentstroke}{rgb}{0.000000,0.000000,0.000000}%
\pgfsetstrokecolor{currentstroke}%
\pgfsetdash{}{0pt}%
\pgfpathmoveto{\pgfqpoint{0.859183in}{1.120240in}}%
\pgfpathlineto{\pgfqpoint{0.855778in}{1.111692in}}%
\pgfpathlineto{\pgfqpoint{0.852374in}{1.103159in}}%
\pgfpathlineto{\pgfqpoint{0.848972in}{1.094642in}}%
\pgfpathlineto{\pgfqpoint{0.845571in}{1.086145in}}%
\pgfpathlineto{\pgfqpoint{0.840923in}{1.091531in}}%
\pgfpathlineto{\pgfqpoint{0.836623in}{1.096984in}}%
\pgfpathlineto{\pgfqpoint{0.832676in}{1.102499in}}%
\pgfpathlineto{\pgfqpoint{0.829084in}{1.108070in}}%
\pgfpathlineto{\pgfqpoint{0.832641in}{1.116318in}}%
\pgfpathlineto{\pgfqpoint{0.836200in}{1.124587in}}%
\pgfpathlineto{\pgfqpoint{0.839761in}{1.132873in}}%
\pgfpathlineto{\pgfqpoint{0.843324in}{1.141173in}}%
\pgfpathlineto{\pgfqpoint{0.846781in}{1.135854in}}%
\pgfpathlineto{\pgfqpoint{0.850579in}{1.130588in}}%
\pgfpathlineto{\pgfqpoint{0.854714in}{1.125382in}}%
\pgfpathlineto{\pgfqpoint{0.859183in}{1.120240in}}%
\pgfpathclose%
\pgfusepath{fill}%
\end{pgfscope}%
\begin{pgfscope}%
\pgfpathrectangle{\pgfqpoint{0.041670in}{0.041670in}}{\pgfqpoint{2.216660in}{2.216660in}}%
\pgfusepath{clip}%
\pgfsetbuttcap%
\pgfsetroundjoin%
\definecolor{currentfill}{rgb}{0.412913,0.803041,0.357269}%
\pgfsetfillcolor{currentfill}%
\pgfsetlinewidth{0.000000pt}%
\definecolor{currentstroke}{rgb}{0.000000,0.000000,0.000000}%
\pgfsetstrokecolor{currentstroke}%
\pgfsetdash{}{0pt}%
\pgfpathmoveto{\pgfqpoint{1.369358in}{1.516307in}}%
\pgfpathlineto{\pgfqpoint{1.372927in}{1.510819in}}%
\pgfpathlineto{\pgfqpoint{1.376492in}{1.505250in}}%
\pgfpathlineto{\pgfqpoint{1.380054in}{1.499602in}}%
\pgfpathlineto{\pgfqpoint{1.383612in}{1.493876in}}%
\pgfpathlineto{\pgfqpoint{1.385864in}{1.490789in}}%
\pgfpathlineto{\pgfqpoint{1.387914in}{1.487668in}}%
\pgfpathlineto{\pgfqpoint{1.389759in}{1.484516in}}%
\pgfpathlineto{\pgfqpoint{1.391398in}{1.481336in}}%
\pgfpathlineto{\pgfqpoint{1.387697in}{1.487292in}}%
\pgfpathlineto{\pgfqpoint{1.383992in}{1.493170in}}%
\pgfpathlineto{\pgfqpoint{1.380284in}{1.498969in}}%
\pgfpathlineto{\pgfqpoint{1.376573in}{1.504686in}}%
\pgfpathlineto{\pgfqpoint{1.375056in}{1.507633in}}%
\pgfpathlineto{\pgfqpoint{1.373346in}{1.510554in}}%
\pgfpathlineto{\pgfqpoint{1.371447in}{1.513446in}}%
\pgfpathlineto{\pgfqpoint{1.369358in}{1.516307in}}%
\pgfpathclose%
\pgfusepath{fill}%
\end{pgfscope}%
\begin{pgfscope}%
\pgfpathrectangle{\pgfqpoint{0.041670in}{0.041670in}}{\pgfqpoint{2.216660in}{2.216660in}}%
\pgfusepath{clip}%
\pgfsetbuttcap%
\pgfsetroundjoin%
\definecolor{currentfill}{rgb}{0.283072,0.130895,0.449241}%
\pgfsetfillcolor{currentfill}%
\pgfsetlinewidth{0.000000pt}%
\definecolor{currentstroke}{rgb}{0.000000,0.000000,0.000000}%
\pgfsetstrokecolor{currentstroke}%
\pgfsetdash{}{0pt}%
\pgfpathmoveto{\pgfqpoint{0.801473in}{0.788883in}}%
\pgfpathlineto{\pgfqpoint{0.798544in}{0.782401in}}%
\pgfpathlineto{\pgfqpoint{0.795613in}{0.776055in}}%
\pgfpathlineto{\pgfqpoint{0.792680in}{0.769849in}}%
\pgfpathlineto{\pgfqpoint{0.789744in}{0.763787in}}%
\pgfpathlineto{\pgfqpoint{0.779908in}{0.770397in}}%
\pgfpathlineto{\pgfqpoint{0.770497in}{0.777163in}}%
\pgfpathlineto{\pgfqpoint{0.761521in}{0.784076in}}%
\pgfpathlineto{\pgfqpoint{0.752987in}{0.791128in}}%
\pgfpathlineto{\pgfqpoint{0.756187in}{0.796963in}}%
\pgfpathlineto{\pgfqpoint{0.759385in}{0.802941in}}%
\pgfpathlineto{\pgfqpoint{0.762580in}{0.809060in}}%
\pgfpathlineto{\pgfqpoint{0.765774in}{0.815315in}}%
\pgfpathlineto{\pgfqpoint{0.774064in}{0.808496in}}%
\pgfpathlineto{\pgfqpoint{0.782783in}{0.801813in}}%
\pgfpathlineto{\pgfqpoint{0.791922in}{0.795273in}}%
\pgfpathlineto{\pgfqpoint{0.801473in}{0.788883in}}%
\pgfpathclose%
\pgfusepath{fill}%
\end{pgfscope}%
\begin{pgfscope}%
\pgfpathrectangle{\pgfqpoint{0.041670in}{0.041670in}}{\pgfqpoint{2.216660in}{2.216660in}}%
\pgfusepath{clip}%
\pgfsetbuttcap%
\pgfsetroundjoin%
\definecolor{currentfill}{rgb}{0.762373,0.876424,0.137064}%
\pgfsetfillcolor{currentfill}%
\pgfsetlinewidth{0.000000pt}%
\definecolor{currentstroke}{rgb}{0.000000,0.000000,0.000000}%
\pgfsetstrokecolor{currentstroke}%
\pgfsetdash{}{0pt}%
\pgfpathmoveto{\pgfqpoint{1.211067in}{1.643313in}}%
\pgfpathlineto{\pgfqpoint{1.212025in}{1.640955in}}%
\pgfpathlineto{\pgfqpoint{1.212981in}{1.638485in}}%
\pgfpathlineto{\pgfqpoint{1.213936in}{1.635905in}}%
\pgfpathlineto{\pgfqpoint{1.214891in}{1.633216in}}%
\pgfpathlineto{\pgfqpoint{1.219119in}{1.632669in}}%
\pgfpathlineto{\pgfqpoint{1.223309in}{1.632060in}}%
\pgfpathlineto{\pgfqpoint{1.227457in}{1.631390in}}%
\pgfpathlineto{\pgfqpoint{1.226159in}{1.634130in}}%
\pgfpathlineto{\pgfqpoint{1.224860in}{1.636761in}}%
\pgfpathlineto{\pgfqpoint{1.223560in}{1.639281in}}%
\pgfpathlineto{\pgfqpoint{1.222257in}{1.641689in}}%
\pgfpathlineto{\pgfqpoint{1.218564in}{1.642285in}}%
\pgfpathlineto{\pgfqpoint{1.214833in}{1.642827in}}%
\pgfpathlineto{\pgfqpoint{1.211067in}{1.643313in}}%
\pgfpathclose%
\pgfusepath{fill}%
\end{pgfscope}%
\begin{pgfscope}%
\pgfpathrectangle{\pgfqpoint{0.041670in}{0.041670in}}{\pgfqpoint{2.216660in}{2.216660in}}%
\pgfusepath{clip}%
\pgfsetbuttcap%
\pgfsetroundjoin%
\definecolor{currentfill}{rgb}{0.274128,0.199721,0.498911}%
\pgfsetfillcolor{currentfill}%
\pgfsetlinewidth{0.000000pt}%
\definecolor{currentstroke}{rgb}{0.000000,0.000000,0.000000}%
\pgfsetstrokecolor{currentstroke}%
\pgfsetdash{}{0pt}%
\pgfpathmoveto{\pgfqpoint{1.575241in}{0.875534in}}%
\pgfpathlineto{\pgfqpoint{1.578474in}{0.868382in}}%
\pgfpathlineto{\pgfqpoint{1.581707in}{0.861335in}}%
\pgfpathlineto{\pgfqpoint{1.584942in}{0.854397in}}%
\pgfpathlineto{\pgfqpoint{1.588178in}{0.847573in}}%
\pgfpathlineto{\pgfqpoint{1.580507in}{0.840878in}}%
\pgfpathlineto{\pgfqpoint{1.572414in}{0.834308in}}%
\pgfpathlineto{\pgfqpoint{1.563908in}{0.827870in}}%
\pgfpathlineto{\pgfqpoint{1.554996in}{0.821571in}}%
\pgfpathlineto{\pgfqpoint{1.552012in}{0.828626in}}%
\pgfpathlineto{\pgfqpoint{1.549029in}{0.835795in}}%
\pgfpathlineto{\pgfqpoint{1.546048in}{0.843073in}}%
\pgfpathlineto{\pgfqpoint{1.543068in}{0.850457in}}%
\pgfpathlineto{\pgfqpoint{1.551708in}{0.856532in}}%
\pgfpathlineto{\pgfqpoint{1.559954in}{0.862741in}}%
\pgfpathlineto{\pgfqpoint{1.567801in}{0.869077in}}%
\pgfpathlineto{\pgfqpoint{1.575241in}{0.875534in}}%
\pgfpathclose%
\pgfusepath{fill}%
\end{pgfscope}%
\begin{pgfscope}%
\pgfpathrectangle{\pgfqpoint{0.041670in}{0.041670in}}{\pgfqpoint{2.216660in}{2.216660in}}%
\pgfusepath{clip}%
\pgfsetbuttcap%
\pgfsetroundjoin%
\definecolor{currentfill}{rgb}{0.762373,0.876424,0.137064}%
\pgfsetfillcolor{currentfill}%
\pgfsetlinewidth{0.000000pt}%
\definecolor{currentstroke}{rgb}{0.000000,0.000000,0.000000}%
\pgfsetstrokecolor{currentstroke}%
\pgfsetdash{}{0pt}%
\pgfpathmoveto{\pgfqpoint{1.134403in}{1.641114in}}%
\pgfpathlineto{\pgfqpoint{1.133001in}{1.638688in}}%
\pgfpathlineto{\pgfqpoint{1.131600in}{1.636150in}}%
\pgfpathlineto{\pgfqpoint{1.130201in}{1.633501in}}%
\pgfpathlineto{\pgfqpoint{1.128804in}{1.630743in}}%
\pgfpathlineto{\pgfqpoint{1.132911in}{1.631467in}}%
\pgfpathlineto{\pgfqpoint{1.137064in}{1.632131in}}%
\pgfpathlineto{\pgfqpoint{1.141259in}{1.632733in}}%
\pgfpathlineto{\pgfqpoint{1.145491in}{1.633273in}}%
\pgfpathlineto{\pgfqpoint{1.146432in}{1.635960in}}%
\pgfpathlineto{\pgfqpoint{1.147375in}{1.638539in}}%
\pgfpathlineto{\pgfqpoint{1.148318in}{1.641007in}}%
\pgfpathlineto{\pgfqpoint{1.149263in}{1.643363in}}%
\pgfpathlineto{\pgfqpoint{1.145494in}{1.642883in}}%
\pgfpathlineto{\pgfqpoint{1.141759in}{1.642348in}}%
\pgfpathlineto{\pgfqpoint{1.138061in}{1.641758in}}%
\pgfpathlineto{\pgfqpoint{1.134403in}{1.641114in}}%
\pgfpathclose%
\pgfusepath{fill}%
\end{pgfscope}%
\begin{pgfscope}%
\pgfpathrectangle{\pgfqpoint{0.041670in}{0.041670in}}{\pgfqpoint{2.216660in}{2.216660in}}%
\pgfusepath{clip}%
\pgfsetbuttcap%
\pgfsetroundjoin%
\definecolor{currentfill}{rgb}{0.699415,0.867117,0.175971}%
\pgfsetfillcolor{currentfill}%
\pgfsetlinewidth{0.000000pt}%
\definecolor{currentstroke}{rgb}{0.000000,0.000000,0.000000}%
\pgfsetstrokecolor{currentstroke}%
\pgfsetdash{}{0pt}%
\pgfpathmoveto{\pgfqpoint{1.084471in}{1.617591in}}%
\pgfpathlineto{\pgfqpoint{1.081859in}{1.614360in}}%
\pgfpathlineto{\pgfqpoint{1.079251in}{1.611022in}}%
\pgfpathlineto{\pgfqpoint{1.076645in}{1.607578in}}%
\pgfpathlineto{\pgfqpoint{1.074042in}{1.604031in}}%
\pgfpathlineto{\pgfqpoint{1.077665in}{1.605573in}}%
\pgfpathlineto{\pgfqpoint{1.081387in}{1.607061in}}%
\pgfpathlineto{\pgfqpoint{1.085206in}{1.608493in}}%
\pgfpathlineto{\pgfqpoint{1.089116in}{1.609867in}}%
\pgfpathlineto{\pgfqpoint{1.091346in}{1.613268in}}%
\pgfpathlineto{\pgfqpoint{1.093580in}{1.616565in}}%
\pgfpathlineto{\pgfqpoint{1.095815in}{1.619757in}}%
\pgfpathlineto{\pgfqpoint{1.098053in}{1.622843in}}%
\pgfpathlineto{\pgfqpoint{1.094530in}{1.621606in}}%
\pgfpathlineto{\pgfqpoint{1.091089in}{1.620318in}}%
\pgfpathlineto{\pgfqpoint{1.087735in}{1.618979in}}%
\pgfpathlineto{\pgfqpoint{1.084471in}{1.617591in}}%
\pgfpathclose%
\pgfusepath{fill}%
\end{pgfscope}%
\begin{pgfscope}%
\pgfpathrectangle{\pgfqpoint{0.041670in}{0.041670in}}{\pgfqpoint{2.216660in}{2.216660in}}%
\pgfusepath{clip}%
\pgfsetbuttcap%
\pgfsetroundjoin%
\definecolor{currentfill}{rgb}{0.268510,0.009605,0.335427}%
\pgfsetfillcolor{currentfill}%
\pgfsetlinewidth{0.000000pt}%
\definecolor{currentstroke}{rgb}{0.000000,0.000000,0.000000}%
\pgfsetstrokecolor{currentstroke}%
\pgfsetdash{}{0pt}%
\pgfpathmoveto{\pgfqpoint{0.742322in}{0.689713in}}%
\pgfpathlineto{\pgfqpoint{0.739322in}{0.686734in}}%
\pgfpathlineto{\pgfqpoint{0.736315in}{0.683976in}}%
\pgfpathlineto{\pgfqpoint{0.733304in}{0.681444in}}%
\pgfpathlineto{\pgfqpoint{0.730286in}{0.679143in}}%
\pgfpathlineto{\pgfqpoint{0.719014in}{0.686848in}}%
\pgfpathlineto{\pgfqpoint{0.708237in}{0.694731in}}%
\pgfpathlineto{\pgfqpoint{0.697965in}{0.702783in}}%
\pgfpathlineto{\pgfqpoint{0.688207in}{0.710996in}}%
\pgfpathlineto{\pgfqpoint{0.691493in}{0.713075in}}%
\pgfpathlineto{\pgfqpoint{0.694772in}{0.715384in}}%
\pgfpathlineto{\pgfqpoint{0.698046in}{0.717918in}}%
\pgfpathlineto{\pgfqpoint{0.701314in}{0.720673in}}%
\pgfpathlineto{\pgfqpoint{0.710826in}{0.712690in}}%
\pgfpathlineto{\pgfqpoint{0.720838in}{0.704863in}}%
\pgfpathlineto{\pgfqpoint{0.731340in}{0.697201in}}%
\pgfpathlineto{\pgfqpoint{0.742322in}{0.689713in}}%
\pgfpathclose%
\pgfusepath{fill}%
\end{pgfscope}%
\begin{pgfscope}%
\pgfpathrectangle{\pgfqpoint{0.041670in}{0.041670in}}{\pgfqpoint{2.216660in}{2.216660in}}%
\pgfusepath{clip}%
\pgfsetbuttcap%
\pgfsetroundjoin%
\definecolor{currentfill}{rgb}{0.487026,0.823929,0.312321}%
\pgfsetfillcolor{currentfill}%
\pgfsetlinewidth{0.000000pt}%
\definecolor{currentstroke}{rgb}{0.000000,0.000000,0.000000}%
\pgfsetstrokecolor{currentstroke}%
\pgfsetdash{}{0pt}%
\pgfpathmoveto{\pgfqpoint{1.003142in}{1.535069in}}%
\pgfpathlineto{\pgfqpoint{0.999523in}{1.529874in}}%
\pgfpathlineto{\pgfqpoint{0.995907in}{1.524591in}}%
\pgfpathlineto{\pgfqpoint{0.992295in}{1.519221in}}%
\pgfpathlineto{\pgfqpoint{0.988686in}{1.513766in}}%
\pgfpathlineto{\pgfqpoint{0.990795in}{1.516623in}}%
\pgfpathlineto{\pgfqpoint{0.993090in}{1.519447in}}%
\pgfpathlineto{\pgfqpoint{0.995568in}{1.522233in}}%
\pgfpathlineto{\pgfqpoint{0.998226in}{1.524979in}}%
\pgfpathlineto{\pgfqpoint{1.001649in}{1.530214in}}%
\pgfpathlineto{\pgfqpoint{1.005076in}{1.535364in}}%
\pgfpathlineto{\pgfqpoint{1.008506in}{1.540428in}}%
\pgfpathlineto{\pgfqpoint{1.011939in}{1.545403in}}%
\pgfpathlineto{\pgfqpoint{1.009487in}{1.542872in}}%
\pgfpathlineto{\pgfqpoint{1.007201in}{1.540304in}}%
\pgfpathlineto{\pgfqpoint{1.005086in}{1.537703in}}%
\pgfpathlineto{\pgfqpoint{1.003142in}{1.535069in}}%
\pgfpathclose%
\pgfusepath{fill}%
\end{pgfscope}%
\begin{pgfscope}%
\pgfpathrectangle{\pgfqpoint{0.041670in}{0.041670in}}{\pgfqpoint{2.216660in}{2.216660in}}%
\pgfusepath{clip}%
\pgfsetbuttcap%
\pgfsetroundjoin%
\definecolor{currentfill}{rgb}{0.260571,0.246922,0.522828}%
\pgfsetfillcolor{currentfill}%
\pgfsetlinewidth{0.000000pt}%
\definecolor{currentstroke}{rgb}{0.000000,0.000000,0.000000}%
\pgfsetstrokecolor{currentstroke}%
\pgfsetdash{}{0pt}%
\pgfpathmoveto{\pgfqpoint{1.857165in}{0.891568in}}%
\pgfpathlineto{\pgfqpoint{1.861218in}{0.900541in}}%
\pgfpathlineto{\pgfqpoint{1.865290in}{0.909936in}}%
\pgfpathlineto{\pgfqpoint{1.869380in}{0.919761in}}%
\pgfpathlineto{\pgfqpoint{1.873489in}{0.930022in}}%
\pgfpathlineto{\pgfqpoint{1.867602in}{0.918716in}}%
\pgfpathlineto{\pgfqpoint{1.860999in}{0.907495in}}%
\pgfpathlineto{\pgfqpoint{1.853682in}{0.896370in}}%
\pgfpathlineto{\pgfqpoint{1.845657in}{0.885354in}}%
\pgfpathlineto{\pgfqpoint{1.841688in}{0.875296in}}%
\pgfpathlineto{\pgfqpoint{1.837738in}{0.865676in}}%
\pgfpathlineto{\pgfqpoint{1.833806in}{0.856488in}}%
\pgfpathlineto{\pgfqpoint{1.829892in}{0.847724in}}%
\pgfpathlineto{\pgfqpoint{1.837751in}{0.858535in}}%
\pgfpathlineto{\pgfqpoint{1.844919in}{0.869454in}}%
\pgfpathlineto{\pgfqpoint{1.851392in}{0.880469in}}%
\pgfpathlineto{\pgfqpoint{1.857165in}{0.891568in}}%
\pgfpathclose%
\pgfusepath{fill}%
\end{pgfscope}%
\begin{pgfscope}%
\pgfpathrectangle{\pgfqpoint{0.041670in}{0.041670in}}{\pgfqpoint{2.216660in}{2.216660in}}%
\pgfusepath{clip}%
\pgfsetbuttcap%
\pgfsetroundjoin%
\definecolor{currentfill}{rgb}{0.565498,0.842430,0.262877}%
\pgfsetfillcolor{currentfill}%
\pgfsetlinewidth{0.000000pt}%
\definecolor{currentstroke}{rgb}{0.000000,0.000000,0.000000}%
\pgfsetstrokecolor{currentstroke}%
\pgfsetdash{}{0pt}%
\pgfpathmoveto{\pgfqpoint{1.025707in}{1.564389in}}%
\pgfpathlineto{\pgfqpoint{1.022260in}{1.559783in}}%
\pgfpathlineto{\pgfqpoint{1.018816in}{1.555083in}}%
\pgfpathlineto{\pgfqpoint{1.015376in}{1.550289in}}%
\pgfpathlineto{\pgfqpoint{1.011939in}{1.545403in}}%
\pgfpathlineto{\pgfqpoint{1.014556in}{1.547896in}}%
\pgfpathlineto{\pgfqpoint{1.017335in}{1.550347in}}%
\pgfpathlineto{\pgfqpoint{1.020273in}{1.552755in}}%
\pgfpathlineto{\pgfqpoint{1.023367in}{1.555116in}}%
\pgfpathlineto{\pgfqpoint{1.026566in}{1.559797in}}%
\pgfpathlineto{\pgfqpoint{1.029767in}{1.564385in}}%
\pgfpathlineto{\pgfqpoint{1.032972in}{1.568881in}}%
\pgfpathlineto{\pgfqpoint{1.036181in}{1.573281in}}%
\pgfpathlineto{\pgfqpoint{1.033344in}{1.571119in}}%
\pgfpathlineto{\pgfqpoint{1.030652in}{1.568915in}}%
\pgfpathlineto{\pgfqpoint{1.028105in}{1.566671in}}%
\pgfpathlineto{\pgfqpoint{1.025707in}{1.564389in}}%
\pgfpathclose%
\pgfusepath{fill}%
\end{pgfscope}%
\begin{pgfscope}%
\pgfpathrectangle{\pgfqpoint{0.041670in}{0.041670in}}{\pgfqpoint{2.216660in}{2.216660in}}%
\pgfusepath{clip}%
\pgfsetbuttcap%
\pgfsetroundjoin%
\definecolor{currentfill}{rgb}{0.636902,0.856542,0.216620}%
\pgfsetfillcolor{currentfill}%
\pgfsetlinewidth{0.000000pt}%
\definecolor{currentstroke}{rgb}{0.000000,0.000000,0.000000}%
\pgfsetstrokecolor{currentstroke}%
\pgfsetdash{}{0pt}%
\pgfpathmoveto{\pgfqpoint{1.296465in}{1.598897in}}%
\pgfpathlineto{\pgfqpoint{1.299328in}{1.595118in}}%
\pgfpathlineto{\pgfqpoint{1.302188in}{1.591239in}}%
\pgfpathlineto{\pgfqpoint{1.305045in}{1.587259in}}%
\pgfpathlineto{\pgfqpoint{1.307899in}{1.583182in}}%
\pgfpathlineto{\pgfqpoint{1.311392in}{1.581254in}}%
\pgfpathlineto{\pgfqpoint{1.314757in}{1.579274in}}%
\pgfpathlineto{\pgfqpoint{1.317991in}{1.577244in}}%
\pgfpathlineto{\pgfqpoint{1.321090in}{1.575167in}}%
\pgfpathlineto{\pgfqpoint{1.317938in}{1.579427in}}%
\pgfpathlineto{\pgfqpoint{1.314783in}{1.583590in}}%
\pgfpathlineto{\pgfqpoint{1.311625in}{1.587654in}}%
\pgfpathlineto{\pgfqpoint{1.308464in}{1.591616in}}%
\pgfpathlineto{\pgfqpoint{1.305645in}{1.593503in}}%
\pgfpathlineto{\pgfqpoint{1.302704in}{1.595347in}}%
\pgfpathlineto{\pgfqpoint{1.299643in}{1.597145in}}%
\pgfpathlineto{\pgfqpoint{1.296465in}{1.598897in}}%
\pgfpathclose%
\pgfusepath{fill}%
\end{pgfscope}%
\begin{pgfscope}%
\pgfpathrectangle{\pgfqpoint{0.041670in}{0.041670in}}{\pgfqpoint{2.216660in}{2.216660in}}%
\pgfusepath{clip}%
\pgfsetbuttcap%
\pgfsetroundjoin%
\definecolor{currentfill}{rgb}{0.134692,0.658636,0.517649}%
\pgfsetfillcolor{currentfill}%
\pgfsetlinewidth{0.000000pt}%
\definecolor{currentstroke}{rgb}{0.000000,0.000000,0.000000}%
\pgfsetstrokecolor{currentstroke}%
\pgfsetdash{}{0pt}%
\pgfpathmoveto{\pgfqpoint{0.906690in}{1.322488in}}%
\pgfpathlineto{\pgfqpoint{0.902978in}{1.314736in}}%
\pgfpathlineto{\pgfqpoint{0.899268in}{1.306944in}}%
\pgfpathlineto{\pgfqpoint{0.895561in}{1.299115in}}%
\pgfpathlineto{\pgfqpoint{0.891857in}{1.291251in}}%
\pgfpathlineto{\pgfqpoint{0.890390in}{1.295736in}}%
\pgfpathlineto{\pgfqpoint{0.889214in}{1.300239in}}%
\pgfpathlineto{\pgfqpoint{0.888330in}{1.304755in}}%
\pgfpathlineto{\pgfqpoint{0.887738in}{1.309279in}}%
\pgfpathlineto{\pgfqpoint{0.891486in}{1.316891in}}%
\pgfpathlineto{\pgfqpoint{0.895236in}{1.324469in}}%
\pgfpathlineto{\pgfqpoint{0.898989in}{1.332010in}}%
\pgfpathlineto{\pgfqpoint{0.902746in}{1.339512in}}%
\pgfpathlineto{\pgfqpoint{0.903316in}{1.335240in}}%
\pgfpathlineto{\pgfqpoint{0.904164in}{1.330975in}}%
\pgfpathlineto{\pgfqpoint{0.905289in}{1.326723in}}%
\pgfpathlineto{\pgfqpoint{0.906690in}{1.322488in}}%
\pgfpathclose%
\pgfusepath{fill}%
\end{pgfscope}%
\begin{pgfscope}%
\pgfpathrectangle{\pgfqpoint{0.041670in}{0.041670in}}{\pgfqpoint{2.216660in}{2.216660in}}%
\pgfusepath{clip}%
\pgfsetbuttcap%
\pgfsetroundjoin%
\definecolor{currentfill}{rgb}{0.212395,0.359683,0.551710}%
\pgfsetfillcolor{currentfill}%
\pgfsetlinewidth{0.000000pt}%
\definecolor{currentstroke}{rgb}{0.000000,0.000000,0.000000}%
\pgfsetstrokecolor{currentstroke}%
\pgfsetdash{}{0pt}%
\pgfpathmoveto{\pgfqpoint{0.842107in}{0.996580in}}%
\pgfpathlineto{\pgfqpoint{0.838928in}{0.988158in}}%
\pgfpathlineto{\pgfqpoint{0.835750in}{0.979788in}}%
\pgfpathlineto{\pgfqpoint{0.832573in}{0.971471in}}%
\pgfpathlineto{\pgfqpoint{0.829395in}{0.963212in}}%
\pgfpathlineto{\pgfqpoint{0.822695in}{0.968972in}}%
\pgfpathlineto{\pgfqpoint{0.816368in}{0.974834in}}%
\pgfpathlineto{\pgfqpoint{0.810418in}{0.980790in}}%
\pgfpathlineto{\pgfqpoint{0.804851in}{0.986835in}}%
\pgfpathlineto{\pgfqpoint{0.808240in}{0.994854in}}%
\pgfpathlineto{\pgfqpoint{0.811629in}{1.002931in}}%
\pgfpathlineto{\pgfqpoint{0.815019in}{1.011061in}}%
\pgfpathlineto{\pgfqpoint{0.818410in}{1.019243in}}%
\pgfpathlineto{\pgfqpoint{0.823786in}{1.013443in}}%
\pgfpathlineto{\pgfqpoint{0.829531in}{1.007728in}}%
\pgfpathlineto{\pgfqpoint{0.835640in}{1.002105in}}%
\pgfpathlineto{\pgfqpoint{0.842107in}{0.996580in}}%
\pgfpathclose%
\pgfusepath{fill}%
\end{pgfscope}%
\begin{pgfscope}%
\pgfpathrectangle{\pgfqpoint{0.041670in}{0.041670in}}{\pgfqpoint{2.216660in}{2.216660in}}%
\pgfusepath{clip}%
\pgfsetbuttcap%
\pgfsetroundjoin%
\definecolor{currentfill}{rgb}{0.267004,0.004874,0.329415}%
\pgfsetfillcolor{currentfill}%
\pgfsetlinewidth{0.000000pt}%
\definecolor{currentstroke}{rgb}{0.000000,0.000000,0.000000}%
\pgfsetstrokecolor{currentstroke}%
\pgfsetdash{}{0pt}%
\pgfpathmoveto{\pgfqpoint{1.693356in}{0.712707in}}%
\pgfpathlineto{\pgfqpoint{1.696729in}{0.711901in}}%
\pgfpathlineto{\pgfqpoint{1.700109in}{0.711354in}}%
\pgfpathlineto{\pgfqpoint{1.703497in}{0.711071in}}%
\pgfpathlineto{\pgfqpoint{1.706893in}{0.711057in}}%
\pgfpathlineto{\pgfqpoint{1.697130in}{0.702251in}}%
\pgfpathlineto{\pgfqpoint{1.686815in}{0.693603in}}%
\pgfpathlineto{\pgfqpoint{1.675957in}{0.685125in}}%
\pgfpathlineto{\pgfqpoint{1.664566in}{0.676826in}}%
\pgfpathlineto{\pgfqpoint{1.661428in}{0.677062in}}%
\pgfpathlineto{\pgfqpoint{1.658298in}{0.677567in}}%
\pgfpathlineto{\pgfqpoint{1.655175in}{0.678337in}}%
\pgfpathlineto{\pgfqpoint{1.652060in}{0.679367in}}%
\pgfpathlineto{\pgfqpoint{1.663171in}{0.687450in}}%
\pgfpathlineto{\pgfqpoint{1.673763in}{0.695707in}}%
\pgfpathlineto{\pgfqpoint{1.683828in}{0.704129in}}%
\pgfpathlineto{\pgfqpoint{1.693356in}{0.712707in}}%
\pgfpathclose%
\pgfusepath{fill}%
\end{pgfscope}%
\begin{pgfscope}%
\pgfpathrectangle{\pgfqpoint{0.041670in}{0.041670in}}{\pgfqpoint{2.216660in}{2.216660in}}%
\pgfusepath{clip}%
\pgfsetbuttcap%
\pgfsetroundjoin%
\definecolor{currentfill}{rgb}{0.280255,0.165693,0.476498}%
\pgfsetfillcolor{currentfill}%
\pgfsetlinewidth{0.000000pt}%
\definecolor{currentstroke}{rgb}{0.000000,0.000000,0.000000}%
\pgfsetstrokecolor{currentstroke}%
\pgfsetdash{}{0pt}%
\pgfpathmoveto{\pgfqpoint{0.813170in}{0.816094in}}%
\pgfpathlineto{\pgfqpoint{0.810248in}{0.809106in}}%
\pgfpathlineto{\pgfqpoint{0.807325in}{0.802239in}}%
\pgfpathlineto{\pgfqpoint{0.804400in}{0.795497in}}%
\pgfpathlineto{\pgfqpoint{0.801473in}{0.788883in}}%
\pgfpathlineto{\pgfqpoint{0.791922in}{0.795273in}}%
\pgfpathlineto{\pgfqpoint{0.782783in}{0.801813in}}%
\pgfpathlineto{\pgfqpoint{0.774064in}{0.808496in}}%
\pgfpathlineto{\pgfqpoint{0.765774in}{0.815315in}}%
\pgfpathlineto{\pgfqpoint{0.768965in}{0.821701in}}%
\pgfpathlineto{\pgfqpoint{0.772155in}{0.828216in}}%
\pgfpathlineto{\pgfqpoint{0.775343in}{0.834856in}}%
\pgfpathlineto{\pgfqpoint{0.778529in}{0.841616in}}%
\pgfpathlineto{\pgfqpoint{0.786576in}{0.835032in}}%
\pgfpathlineto{\pgfqpoint{0.795036in}{0.828579in}}%
\pgfpathlineto{\pgfqpoint{0.803904in}{0.822264in}}%
\pgfpathlineto{\pgfqpoint{0.813170in}{0.816094in}}%
\pgfpathclose%
\pgfusepath{fill}%
\end{pgfscope}%
\begin{pgfscope}%
\pgfpathrectangle{\pgfqpoint{0.041670in}{0.041670in}}{\pgfqpoint{2.216660in}{2.216660in}}%
\pgfusepath{clip}%
\pgfsetbuttcap%
\pgfsetroundjoin%
\definecolor{currentfill}{rgb}{0.220124,0.725509,0.466226}%
\pgfsetfillcolor{currentfill}%
\pgfsetlinewidth{0.000000pt}%
\definecolor{currentstroke}{rgb}{0.000000,0.000000,0.000000}%
\pgfsetstrokecolor{currentstroke}%
\pgfsetdash{}{0pt}%
\pgfpathmoveto{\pgfqpoint{0.932910in}{1.397843in}}%
\pgfpathlineto{\pgfqpoint{0.929129in}{1.390737in}}%
\pgfpathlineto{\pgfqpoint{0.925350in}{1.383574in}}%
\pgfpathlineto{\pgfqpoint{0.921575in}{1.376355in}}%
\pgfpathlineto{\pgfqpoint{0.917803in}{1.369084in}}%
\pgfpathlineto{\pgfqpoint{0.917516in}{1.373111in}}%
\pgfpathlineto{\pgfqpoint{0.917492in}{1.377137in}}%
\pgfpathlineto{\pgfqpoint{0.917729in}{1.381160in}}%
\pgfpathlineto{\pgfqpoint{0.918227in}{1.385175in}}%
\pgfpathlineto{\pgfqpoint{0.921984in}{1.392198in}}%
\pgfpathlineto{\pgfqpoint{0.925744in}{1.399169in}}%
\pgfpathlineto{\pgfqpoint{0.929508in}{1.406085in}}%
\pgfpathlineto{\pgfqpoint{0.933275in}{1.412944in}}%
\pgfpathlineto{\pgfqpoint{0.932815in}{1.409176in}}%
\pgfpathlineto{\pgfqpoint{0.932600in}{1.405400in}}%
\pgfpathlineto{\pgfqpoint{0.932632in}{1.401621in}}%
\pgfpathlineto{\pgfqpoint{0.932910in}{1.397843in}}%
\pgfpathclose%
\pgfusepath{fill}%
\end{pgfscope}%
\begin{pgfscope}%
\pgfpathrectangle{\pgfqpoint{0.041670in}{0.041670in}}{\pgfqpoint{2.216660in}{2.216660in}}%
\pgfusepath{clip}%
\pgfsetbuttcap%
\pgfsetroundjoin%
\definecolor{currentfill}{rgb}{0.147607,0.511733,0.557049}%
\pgfsetfillcolor{currentfill}%
\pgfsetlinewidth{0.000000pt}%
\definecolor{currentstroke}{rgb}{0.000000,0.000000,0.000000}%
\pgfsetstrokecolor{currentstroke}%
\pgfsetdash{}{0pt}%
\pgfpathmoveto{\pgfqpoint{1.504991in}{1.179001in}}%
\pgfpathlineto{\pgfqpoint{1.508589in}{1.170729in}}%
\pgfpathlineto{\pgfqpoint{1.512185in}{1.162460in}}%
\pgfpathlineto{\pgfqpoint{1.515779in}{1.154197in}}%
\pgfpathlineto{\pgfqpoint{1.519370in}{1.145942in}}%
\pgfpathlineto{\pgfqpoint{1.516219in}{1.140579in}}%
\pgfpathlineto{\pgfqpoint{1.512723in}{1.135266in}}%
\pgfpathlineto{\pgfqpoint{1.508888in}{1.130007in}}%
\pgfpathlineto{\pgfqpoint{1.504716in}{1.124807in}}%
\pgfpathlineto{\pgfqpoint{1.501269in}{1.133311in}}%
\pgfpathlineto{\pgfqpoint{1.497820in}{1.141824in}}%
\pgfpathlineto{\pgfqpoint{1.494370in}{1.150341in}}%
\pgfpathlineto{\pgfqpoint{1.490917in}{1.158861in}}%
\pgfpathlineto{\pgfqpoint{1.494922in}{1.163815in}}%
\pgfpathlineto{\pgfqpoint{1.498605in}{1.168827in}}%
\pgfpathlineto{\pgfqpoint{1.501962in}{1.173890in}}%
\pgfpathlineto{\pgfqpoint{1.504991in}{1.179001in}}%
\pgfpathclose%
\pgfusepath{fill}%
\end{pgfscope}%
\begin{pgfscope}%
\pgfpathrectangle{\pgfqpoint{0.041670in}{0.041670in}}{\pgfqpoint{2.216660in}{2.216660in}}%
\pgfusepath{clip}%
\pgfsetbuttcap%
\pgfsetroundjoin%
\definecolor{currentfill}{rgb}{0.762373,0.876424,0.137064}%
\pgfsetfillcolor{currentfill}%
\pgfsetlinewidth{0.000000pt}%
\definecolor{currentstroke}{rgb}{0.000000,0.000000,0.000000}%
\pgfsetstrokecolor{currentstroke}%
\pgfsetdash{}{0pt}%
\pgfpathmoveto{\pgfqpoint{1.222257in}{1.641689in}}%
\pgfpathlineto{\pgfqpoint{1.223560in}{1.639281in}}%
\pgfpathlineto{\pgfqpoint{1.224860in}{1.636761in}}%
\pgfpathlineto{\pgfqpoint{1.226159in}{1.634130in}}%
\pgfpathlineto{\pgfqpoint{1.227457in}{1.631390in}}%
\pgfpathlineto{\pgfqpoint{1.231559in}{1.630659in}}%
\pgfpathlineto{\pgfqpoint{1.235612in}{1.629867in}}%
\pgfpathlineto{\pgfqpoint{1.239610in}{1.629016in}}%
\pgfpathlineto{\pgfqpoint{1.243550in}{1.628106in}}%
\pgfpathlineto{\pgfqpoint{1.241812in}{1.630937in}}%
\pgfpathlineto{\pgfqpoint{1.240072in}{1.633659in}}%
\pgfpathlineto{\pgfqpoint{1.238330in}{1.636270in}}%
\pgfpathlineto{\pgfqpoint{1.236586in}{1.638770in}}%
\pgfpathlineto{\pgfqpoint{1.233078in}{1.639579in}}%
\pgfpathlineto{\pgfqpoint{1.229518in}{1.640335in}}%
\pgfpathlineto{\pgfqpoint{1.225910in}{1.641039in}}%
\pgfpathlineto{\pgfqpoint{1.222257in}{1.641689in}}%
\pgfpathclose%
\pgfusepath{fill}%
\end{pgfscope}%
\begin{pgfscope}%
\pgfpathrectangle{\pgfqpoint{0.041670in}{0.041670in}}{\pgfqpoint{2.216660in}{2.216660in}}%
\pgfusepath{clip}%
\pgfsetbuttcap%
\pgfsetroundjoin%
\definecolor{currentfill}{rgb}{0.412913,0.803041,0.357269}%
\pgfsetfillcolor{currentfill}%
\pgfsetlinewidth{0.000000pt}%
\definecolor{currentstroke}{rgb}{0.000000,0.000000,0.000000}%
\pgfsetstrokecolor{currentstroke}%
\pgfsetdash{}{0pt}%
\pgfpathmoveto{\pgfqpoint{0.982151in}{1.502048in}}%
\pgfpathlineto{\pgfqpoint{0.978416in}{1.496279in}}%
\pgfpathlineto{\pgfqpoint{0.974684in}{1.490428in}}%
\pgfpathlineto{\pgfqpoint{0.970956in}{1.484497in}}%
\pgfpathlineto{\pgfqpoint{0.967231in}{1.478489in}}%
\pgfpathlineto{\pgfqpoint{0.968684in}{1.481691in}}%
\pgfpathlineto{\pgfqpoint{0.970345in}{1.484868in}}%
\pgfpathlineto{\pgfqpoint{0.972213in}{1.488016in}}%
\pgfpathlineto{\pgfqpoint{0.974286in}{1.491134in}}%
\pgfpathlineto{\pgfqpoint{0.977881in}{1.496910in}}%
\pgfpathlineto{\pgfqpoint{0.981479in}{1.502608in}}%
\pgfpathlineto{\pgfqpoint{0.985081in}{1.508228in}}%
\pgfpathlineto{\pgfqpoint{0.988686in}{1.513766in}}%
\pgfpathlineto{\pgfqpoint{0.986765in}{1.510876in}}%
\pgfpathlineto{\pgfqpoint{0.985034in}{1.507959in}}%
\pgfpathlineto{\pgfqpoint{0.983496in}{1.505015in}}%
\pgfpathlineto{\pgfqpoint{0.982151in}{1.502048in}}%
\pgfpathclose%
\pgfusepath{fill}%
\end{pgfscope}%
\begin{pgfscope}%
\pgfpathrectangle{\pgfqpoint{0.041670in}{0.041670in}}{\pgfqpoint{2.216660in}{2.216660in}}%
\pgfusepath{clip}%
\pgfsetbuttcap%
\pgfsetroundjoin%
\definecolor{currentfill}{rgb}{0.195860,0.395433,0.555276}%
\pgfsetfillcolor{currentfill}%
\pgfsetlinewidth{0.000000pt}%
\definecolor{currentstroke}{rgb}{0.000000,0.000000,0.000000}%
\pgfsetstrokecolor{currentstroke}%
\pgfsetdash{}{0pt}%
\pgfpathmoveto{\pgfqpoint{1.532236in}{1.057416in}}%
\pgfpathlineto{\pgfqpoint{1.535670in}{1.049117in}}%
\pgfpathlineto{\pgfqpoint{1.539103in}{1.040857in}}%
\pgfpathlineto{\pgfqpoint{1.542535in}{1.032639in}}%
\pgfpathlineto{\pgfqpoint{1.545965in}{1.024464in}}%
\pgfpathlineto{\pgfqpoint{1.540921in}{1.018594in}}%
\pgfpathlineto{\pgfqpoint{1.535504in}{1.012804in}}%
\pgfpathlineto{\pgfqpoint{1.529718in}{1.007099in}}%
\pgfpathlineto{\pgfqpoint{1.523569in}{1.001486in}}%
\pgfpathlineto{\pgfqpoint{1.520338in}{1.009903in}}%
\pgfpathlineto{\pgfqpoint{1.517106in}{1.018364in}}%
\pgfpathlineto{\pgfqpoint{1.513873in}{1.026865in}}%
\pgfpathlineto{\pgfqpoint{1.510639in}{1.035405in}}%
\pgfpathlineto{\pgfqpoint{1.516567in}{1.040781in}}%
\pgfpathlineto{\pgfqpoint{1.522145in}{1.046245in}}%
\pgfpathlineto{\pgfqpoint{1.527370in}{1.051792in}}%
\pgfpathlineto{\pgfqpoint{1.532236in}{1.057416in}}%
\pgfpathclose%
\pgfusepath{fill}%
\end{pgfscope}%
\begin{pgfscope}%
\pgfpathrectangle{\pgfqpoint{0.041670in}{0.041670in}}{\pgfqpoint{2.216660in}{2.216660in}}%
\pgfusepath{clip}%
\pgfsetbuttcap%
\pgfsetroundjoin%
\definecolor{currentfill}{rgb}{0.122606,0.585371,0.546557}%
\pgfsetfillcolor{currentfill}%
\pgfsetlinewidth{0.000000pt}%
\definecolor{currentstroke}{rgb}{0.000000,0.000000,0.000000}%
\pgfsetstrokecolor{currentstroke}%
\pgfsetdash{}{0pt}%
\pgfpathmoveto{\pgfqpoint{0.886239in}{1.240832in}}%
\pgfpathlineto{\pgfqpoint{0.882651in}{1.232585in}}%
\pgfpathlineto{\pgfqpoint{0.879064in}{1.224317in}}%
\pgfpathlineto{\pgfqpoint{0.875481in}{1.216033in}}%
\pgfpathlineto{\pgfqpoint{0.871899in}{1.207735in}}%
\pgfpathlineto{\pgfqpoint{0.869029in}{1.212600in}}%
\pgfpathlineto{\pgfqpoint{0.866475in}{1.217504in}}%
\pgfpathlineto{\pgfqpoint{0.864239in}{1.222442in}}%
\pgfpathlineto{\pgfqpoint{0.862322in}{1.227410in}}%
\pgfpathlineto{\pgfqpoint{0.866005in}{1.235456in}}%
\pgfpathlineto{\pgfqpoint{0.869690in}{1.243489in}}%
\pgfpathlineto{\pgfqpoint{0.873378in}{1.251506in}}%
\pgfpathlineto{\pgfqpoint{0.877068in}{1.259503in}}%
\pgfpathlineto{\pgfqpoint{0.878906in}{1.254789in}}%
\pgfpathlineto{\pgfqpoint{0.881048in}{1.250103in}}%
\pgfpathlineto{\pgfqpoint{0.883493in}{1.245449in}}%
\pgfpathlineto{\pgfqpoint{0.886239in}{1.240832in}}%
\pgfpathclose%
\pgfusepath{fill}%
\end{pgfscope}%
\begin{pgfscope}%
\pgfpathrectangle{\pgfqpoint{0.041670in}{0.041670in}}{\pgfqpoint{2.216660in}{2.216660in}}%
\pgfusepath{clip}%
\pgfsetbuttcap%
\pgfsetroundjoin%
\definecolor{currentfill}{rgb}{0.636902,0.856542,0.216620}%
\pgfsetfillcolor{currentfill}%
\pgfsetlinewidth{0.000000pt}%
\definecolor{currentstroke}{rgb}{0.000000,0.000000,0.000000}%
\pgfsetstrokecolor{currentstroke}%
\pgfsetdash{}{0pt}%
\pgfpathmoveto{\pgfqpoint{1.049046in}{1.589904in}}%
\pgfpathlineto{\pgfqpoint{1.045825in}{1.585898in}}%
\pgfpathlineto{\pgfqpoint{1.042607in}{1.581792in}}%
\pgfpathlineto{\pgfqpoint{1.039392in}{1.577586in}}%
\pgfpathlineto{\pgfqpoint{1.036181in}{1.573281in}}%
\pgfpathlineto{\pgfqpoint{1.039158in}{1.575400in}}%
\pgfpathlineto{\pgfqpoint{1.042272in}{1.577472in}}%
\pgfpathlineto{\pgfqpoint{1.045520in}{1.579496in}}%
\pgfpathlineto{\pgfqpoint{1.048900in}{1.581471in}}%
\pgfpathlineto{\pgfqpoint{1.051824in}{1.585587in}}%
\pgfpathlineto{\pgfqpoint{1.054751in}{1.589605in}}%
\pgfpathlineto{\pgfqpoint{1.057681in}{1.593524in}}%
\pgfpathlineto{\pgfqpoint{1.060614in}{1.597342in}}%
\pgfpathlineto{\pgfqpoint{1.057540in}{1.595549in}}%
\pgfpathlineto{\pgfqpoint{1.054585in}{1.593710in}}%
\pgfpathlineto{\pgfqpoint{1.051753in}{1.591828in}}%
\pgfpathlineto{\pgfqpoint{1.049046in}{1.589904in}}%
\pgfpathclose%
\pgfusepath{fill}%
\end{pgfscope}%
\begin{pgfscope}%
\pgfpathrectangle{\pgfqpoint{0.041670in}{0.041670in}}{\pgfqpoint{2.216660in}{2.216660in}}%
\pgfusepath{clip}%
\pgfsetbuttcap%
\pgfsetroundjoin%
\definecolor{currentfill}{rgb}{0.344074,0.780029,0.397381}%
\pgfsetfillcolor{currentfill}%
\pgfsetlinewidth{0.000000pt}%
\definecolor{currentstroke}{rgb}{0.000000,0.000000,0.000000}%
\pgfsetstrokecolor{currentstroke}%
\pgfsetdash{}{0pt}%
\pgfpathmoveto{\pgfqpoint{1.391398in}{1.481336in}}%
\pgfpathlineto{\pgfqpoint{1.395095in}{1.475305in}}%
\pgfpathlineto{\pgfqpoint{1.398789in}{1.469200in}}%
\pgfpathlineto{\pgfqpoint{1.402480in}{1.463023in}}%
\pgfpathlineto{\pgfqpoint{1.406167in}{1.456777in}}%
\pgfpathlineto{\pgfqpoint{1.407704in}{1.453336in}}%
\pgfpathlineto{\pgfqpoint{1.409017in}{1.449871in}}%
\pgfpathlineto{\pgfqpoint{1.410103in}{1.446387in}}%
\pgfpathlineto{\pgfqpoint{1.410961in}{1.442886in}}%
\pgfpathlineto{\pgfqpoint{1.407188in}{1.449372in}}%
\pgfpathlineto{\pgfqpoint{1.403412in}{1.455788in}}%
\pgfpathlineto{\pgfqpoint{1.399633in}{1.462132in}}%
\pgfpathlineto{\pgfqpoint{1.395850in}{1.468402in}}%
\pgfpathlineto{\pgfqpoint{1.395055in}{1.471662in}}%
\pgfpathlineto{\pgfqpoint{1.394047in}{1.474906in}}%
\pgfpathlineto{\pgfqpoint{1.392828in}{1.478132in}}%
\pgfpathlineto{\pgfqpoint{1.391398in}{1.481336in}}%
\pgfpathclose%
\pgfusepath{fill}%
\end{pgfscope}%
\begin{pgfscope}%
\pgfpathrectangle{\pgfqpoint{0.041670in}{0.041670in}}{\pgfqpoint{2.216660in}{2.216660in}}%
\pgfusepath{clip}%
\pgfsetbuttcap%
\pgfsetroundjoin%
\definecolor{currentfill}{rgb}{0.762373,0.876424,0.137064}%
\pgfsetfillcolor{currentfill}%
\pgfsetlinewidth{0.000000pt}%
\definecolor{currentstroke}{rgb}{0.000000,0.000000,0.000000}%
\pgfsetstrokecolor{currentstroke}%
\pgfsetdash{}{0pt}%
\pgfpathmoveto{\pgfqpoint{1.120252in}{1.638007in}}%
\pgfpathlineto{\pgfqpoint{1.118413in}{1.635484in}}%
\pgfpathlineto{\pgfqpoint{1.116576in}{1.632849in}}%
\pgfpathlineto{\pgfqpoint{1.114741in}{1.630104in}}%
\pgfpathlineto{\pgfqpoint{1.112909in}{1.627249in}}%
\pgfpathlineto{\pgfqpoint{1.116794in}{1.628210in}}%
\pgfpathlineto{\pgfqpoint{1.120741in}{1.629113in}}%
\pgfpathlineto{\pgfqpoint{1.124746in}{1.629958in}}%
\pgfpathlineto{\pgfqpoint{1.128804in}{1.630743in}}%
\pgfpathlineto{\pgfqpoint{1.130201in}{1.633501in}}%
\pgfpathlineto{\pgfqpoint{1.131600in}{1.636150in}}%
\pgfpathlineto{\pgfqpoint{1.133001in}{1.638688in}}%
\pgfpathlineto{\pgfqpoint{1.134403in}{1.641114in}}%
\pgfpathlineto{\pgfqpoint{1.130790in}{1.640416in}}%
\pgfpathlineto{\pgfqpoint{1.127225in}{1.639665in}}%
\pgfpathlineto{\pgfqpoint{1.123711in}{1.638862in}}%
\pgfpathlineto{\pgfqpoint{1.120252in}{1.638007in}}%
\pgfpathclose%
\pgfusepath{fill}%
\end{pgfscope}%
\begin{pgfscope}%
\pgfpathrectangle{\pgfqpoint{0.041670in}{0.041670in}}{\pgfqpoint{2.216660in}{2.216660in}}%
\pgfusepath{clip}%
\pgfsetbuttcap%
\pgfsetroundjoin%
\definecolor{currentfill}{rgb}{0.201239,0.383670,0.554294}%
\pgfsetfillcolor{currentfill}%
\pgfsetlinewidth{0.000000pt}%
\definecolor{currentstroke}{rgb}{0.000000,0.000000,0.000000}%
\pgfsetstrokecolor{currentstroke}%
\pgfsetdash{}{0pt}%
\pgfpathmoveto{\pgfqpoint{0.475080in}{0.965353in}}%
\pgfpathlineto{\pgfqpoint{0.470892in}{0.977868in}}%
\pgfpathlineto{\pgfqpoint{0.466682in}{0.990866in}}%
\pgfpathlineto{\pgfqpoint{0.462450in}{1.004356in}}%
\pgfpathlineto{\pgfqpoint{0.458194in}{1.018346in}}%
\pgfpathlineto{\pgfqpoint{0.452160in}{1.030046in}}%
\pgfpathlineto{\pgfqpoint{0.446883in}{1.041819in}}%
\pgfpathlineto{\pgfqpoint{0.442362in}{1.053650in}}%
\pgfpathlineto{\pgfqpoint{0.438600in}{1.065528in}}%
\pgfpathlineto{\pgfqpoint{0.442942in}{1.051355in}}%
\pgfpathlineto{\pgfqpoint{0.447261in}{1.037679in}}%
\pgfpathlineto{\pgfqpoint{0.451557in}{1.024491in}}%
\pgfpathlineto{\pgfqpoint{0.455831in}{1.011784in}}%
\pgfpathlineto{\pgfqpoint{0.459534in}{1.000094in}}%
\pgfpathlineto{\pgfqpoint{0.463977in}{0.988450in}}%
\pgfpathlineto{\pgfqpoint{0.469160in}{0.976866in}}%
\pgfpathlineto{\pgfqpoint{0.475080in}{0.965353in}}%
\pgfpathclose%
\pgfusepath{fill}%
\end{pgfscope}%
\begin{pgfscope}%
\pgfpathrectangle{\pgfqpoint{0.041670in}{0.041670in}}{\pgfqpoint{2.216660in}{2.216660in}}%
\pgfusepath{clip}%
\pgfsetbuttcap%
\pgfsetroundjoin%
\definecolor{currentfill}{rgb}{0.267004,0.004874,0.329415}%
\pgfsetfillcolor{currentfill}%
\pgfsetlinewidth{0.000000pt}%
\definecolor{currentstroke}{rgb}{0.000000,0.000000,0.000000}%
\pgfsetstrokecolor{currentstroke}%
\pgfsetdash{}{0pt}%
\pgfpathmoveto{\pgfqpoint{0.730286in}{0.679143in}}%
\pgfpathlineto{\pgfqpoint{0.727262in}{0.677076in}}%
\pgfpathlineto{\pgfqpoint{0.724232in}{0.675250in}}%
\pgfpathlineto{\pgfqpoint{0.721196in}{0.673668in}}%
\pgfpathlineto{\pgfqpoint{0.718152in}{0.672336in}}%
\pgfpathlineto{\pgfqpoint{0.706590in}{0.680256in}}%
\pgfpathlineto{\pgfqpoint{0.695536in}{0.688359in}}%
\pgfpathlineto{\pgfqpoint{0.685002in}{0.696635in}}%
\pgfpathlineto{\pgfqpoint{0.674996in}{0.705075in}}%
\pgfpathlineto{\pgfqpoint{0.678309in}{0.706186in}}%
\pgfpathlineto{\pgfqpoint{0.681615in}{0.707547in}}%
\pgfpathlineto{\pgfqpoint{0.684914in}{0.709152in}}%
\pgfpathlineto{\pgfqpoint{0.688207in}{0.710996in}}%
\pgfpathlineto{\pgfqpoint{0.697965in}{0.702783in}}%
\pgfpathlineto{\pgfqpoint{0.708237in}{0.694731in}}%
\pgfpathlineto{\pgfqpoint{0.719014in}{0.686848in}}%
\pgfpathlineto{\pgfqpoint{0.730286in}{0.679143in}}%
\pgfpathclose%
\pgfusepath{fill}%
\end{pgfscope}%
\begin{pgfscope}%
\pgfpathrectangle{\pgfqpoint{0.041670in}{0.041670in}}{\pgfqpoint{2.216660in}{2.216660in}}%
\pgfusepath{clip}%
\pgfsetbuttcap%
\pgfsetroundjoin%
\definecolor{currentfill}{rgb}{0.263663,0.237631,0.518762}%
\pgfsetfillcolor{currentfill}%
\pgfsetlinewidth{0.000000pt}%
\definecolor{currentstroke}{rgb}{0.000000,0.000000,0.000000}%
\pgfsetstrokecolor{currentstroke}%
\pgfsetdash{}{0pt}%
\pgfpathmoveto{\pgfqpoint{1.562318in}{0.905132in}}%
\pgfpathlineto{\pgfqpoint{1.565548in}{0.897592in}}%
\pgfpathlineto{\pgfqpoint{1.568778in}{0.890143in}}%
\pgfpathlineto{\pgfqpoint{1.572009in}{0.882790in}}%
\pgfpathlineto{\pgfqpoint{1.575241in}{0.875534in}}%
\pgfpathlineto{\pgfqpoint{1.567801in}{0.869077in}}%
\pgfpathlineto{\pgfqpoint{1.559954in}{0.862741in}}%
\pgfpathlineto{\pgfqpoint{1.551708in}{0.856532in}}%
\pgfpathlineto{\pgfqpoint{1.543068in}{0.850457in}}%
\pgfpathlineto{\pgfqpoint{1.540089in}{0.857942in}}%
\pgfpathlineto{\pgfqpoint{1.537111in}{0.865527in}}%
\pgfpathlineto{\pgfqpoint{1.534134in}{0.873206in}}%
\pgfpathlineto{\pgfqpoint{1.531157in}{0.880976in}}%
\pgfpathlineto{\pgfqpoint{1.539523in}{0.886828in}}%
\pgfpathlineto{\pgfqpoint{1.547510in}{0.892808in}}%
\pgfpathlineto{\pgfqpoint{1.555111in}{0.898912in}}%
\pgfpathlineto{\pgfqpoint{1.562318in}{0.905132in}}%
\pgfpathclose%
\pgfusepath{fill}%
\end{pgfscope}%
\begin{pgfscope}%
\pgfpathrectangle{\pgfqpoint{0.041670in}{0.041670in}}{\pgfqpoint{2.216660in}{2.216660in}}%
\pgfusepath{clip}%
\pgfsetbuttcap%
\pgfsetroundjoin%
\definecolor{currentfill}{rgb}{0.699415,0.867117,0.175971}%
\pgfsetfillcolor{currentfill}%
\pgfsetlinewidth{0.000000pt}%
\definecolor{currentstroke}{rgb}{0.000000,0.000000,0.000000}%
\pgfsetstrokecolor{currentstroke}%
\pgfsetdash{}{0pt}%
\pgfpathmoveto{\pgfqpoint{1.272542in}{1.618828in}}%
\pgfpathlineto{\pgfqpoint{1.275074in}{1.615630in}}%
\pgfpathlineto{\pgfqpoint{1.277602in}{1.612327in}}%
\pgfpathlineto{\pgfqpoint{1.280129in}{1.608917in}}%
\pgfpathlineto{\pgfqpoint{1.282652in}{1.605405in}}%
\pgfpathlineto{\pgfqpoint{1.286264in}{1.603856in}}%
\pgfpathlineto{\pgfqpoint{1.289773in}{1.602254in}}%
\pgfpathlineto{\pgfqpoint{1.293174in}{1.600601in}}%
\pgfpathlineto{\pgfqpoint{1.296465in}{1.598897in}}%
\pgfpathlineto{\pgfqpoint{1.293600in}{1.602574in}}%
\pgfpathlineto{\pgfqpoint{1.290731in}{1.606146in}}%
\pgfpathlineto{\pgfqpoint{1.287859in}{1.609613in}}%
\pgfpathlineto{\pgfqpoint{1.284985in}{1.612974in}}%
\pgfpathlineto{\pgfqpoint{1.282020in}{1.614506in}}%
\pgfpathlineto{\pgfqpoint{1.278956in}{1.615993in}}%
\pgfpathlineto{\pgfqpoint{1.275796in}{1.617434in}}%
\pgfpathlineto{\pgfqpoint{1.272542in}{1.618828in}}%
\pgfpathclose%
\pgfusepath{fill}%
\end{pgfscope}%
\begin{pgfscope}%
\pgfpathrectangle{\pgfqpoint{0.041670in}{0.041670in}}{\pgfqpoint{2.216660in}{2.216660in}}%
\pgfusepath{clip}%
\pgfsetbuttcap%
\pgfsetroundjoin%
\definecolor{currentfill}{rgb}{0.274128,0.199721,0.498911}%
\pgfsetfillcolor{currentfill}%
\pgfsetlinewidth{0.000000pt}%
\definecolor{currentstroke}{rgb}{0.000000,0.000000,0.000000}%
\pgfsetstrokecolor{currentstroke}%
\pgfsetdash{}{0pt}%
\pgfpathmoveto{\pgfqpoint{0.824843in}{0.845175in}}%
\pgfpathlineto{\pgfqpoint{0.821927in}{0.837742in}}%
\pgfpathlineto{\pgfqpoint{0.819009in}{0.830416in}}%
\pgfpathlineto{\pgfqpoint{0.816090in}{0.823198in}}%
\pgfpathlineto{\pgfqpoint{0.813170in}{0.816094in}}%
\pgfpathlineto{\pgfqpoint{0.803904in}{0.822264in}}%
\pgfpathlineto{\pgfqpoint{0.795036in}{0.828579in}}%
\pgfpathlineto{\pgfqpoint{0.786576in}{0.835032in}}%
\pgfpathlineto{\pgfqpoint{0.778529in}{0.841616in}}%
\pgfpathlineto{\pgfqpoint{0.781714in}{0.848493in}}%
\pgfpathlineto{\pgfqpoint{0.784898in}{0.855484in}}%
\pgfpathlineto{\pgfqpoint{0.788080in}{0.862583in}}%
\pgfpathlineto{\pgfqpoint{0.791262in}{0.869789in}}%
\pgfpathlineto{\pgfqpoint{0.799064in}{0.863439in}}%
\pgfpathlineto{\pgfqpoint{0.807266in}{0.857215in}}%
\pgfpathlineto{\pgfqpoint{0.815862in}{0.851125in}}%
\pgfpathlineto{\pgfqpoint{0.824843in}{0.845175in}}%
\pgfpathclose%
\pgfusepath{fill}%
\end{pgfscope}%
\begin{pgfscope}%
\pgfpathrectangle{\pgfqpoint{0.041670in}{0.041670in}}{\pgfqpoint{2.216660in}{2.216660in}}%
\pgfusepath{clip}%
\pgfsetbuttcap%
\pgfsetroundjoin%
\definecolor{currentfill}{rgb}{0.344074,0.780029,0.397381}%
\pgfsetfillcolor{currentfill}%
\pgfsetlinewidth{0.000000pt}%
\definecolor{currentstroke}{rgb}{0.000000,0.000000,0.000000}%
\pgfsetstrokecolor{currentstroke}%
\pgfsetdash{}{0pt}%
\pgfpathmoveto{\pgfqpoint{0.963534in}{1.465494in}}%
\pgfpathlineto{\pgfqpoint{0.959740in}{1.459170in}}%
\pgfpathlineto{\pgfqpoint{0.955949in}{1.452773in}}%
\pgfpathlineto{\pgfqpoint{0.952162in}{1.446302in}}%
\pgfpathlineto{\pgfqpoint{0.948378in}{1.439762in}}%
\pgfpathlineto{\pgfqpoint{0.949032in}{1.443275in}}%
\pgfpathlineto{\pgfqpoint{0.949916in}{1.446775in}}%
\pgfpathlineto{\pgfqpoint{0.951027in}{1.450257in}}%
\pgfpathlineto{\pgfqpoint{0.952365in}{1.453719in}}%
\pgfpathlineto{\pgfqpoint{0.956076in}{1.460018in}}%
\pgfpathlineto{\pgfqpoint{0.959791in}{1.466248in}}%
\pgfpathlineto{\pgfqpoint{0.963509in}{1.472405in}}%
\pgfpathlineto{\pgfqpoint{0.967231in}{1.478489in}}%
\pgfpathlineto{\pgfqpoint{0.965988in}{1.475265in}}%
\pgfpathlineto{\pgfqpoint{0.964956in}{1.472023in}}%
\pgfpathlineto{\pgfqpoint{0.964138in}{1.468765in}}%
\pgfpathlineto{\pgfqpoint{0.963534in}{1.465494in}}%
\pgfpathclose%
\pgfusepath{fill}%
\end{pgfscope}%
\begin{pgfscope}%
\pgfpathrectangle{\pgfqpoint{0.041670in}{0.041670in}}{\pgfqpoint{2.216660in}{2.216660in}}%
\pgfusepath{clip}%
\pgfsetbuttcap%
\pgfsetroundjoin%
\definecolor{currentfill}{rgb}{0.166383,0.690856,0.496502}%
\pgfsetfillcolor{currentfill}%
\pgfsetlinewidth{0.000000pt}%
\definecolor{currentstroke}{rgb}{0.000000,0.000000,0.000000}%
\pgfsetstrokecolor{currentstroke}%
\pgfsetdash{}{0pt}%
\pgfpathmoveto{\pgfqpoint{1.442374in}{1.372663in}}%
\pgfpathlineto{\pgfqpoint{1.446145in}{1.365396in}}%
\pgfpathlineto{\pgfqpoint{1.449913in}{1.358080in}}%
\pgfpathlineto{\pgfqpoint{1.453677in}{1.350719in}}%
\pgfpathlineto{\pgfqpoint{1.457438in}{1.343314in}}%
\pgfpathlineto{\pgfqpoint{1.457114in}{1.339037in}}%
\pgfpathlineto{\pgfqpoint{1.456513in}{1.334765in}}%
\pgfpathlineto{\pgfqpoint{1.455634in}{1.330502in}}%
\pgfpathlineto{\pgfqpoint{1.454479in}{1.326252in}}%
\pgfpathlineto{\pgfqpoint{1.450748in}{1.333908in}}%
\pgfpathlineto{\pgfqpoint{1.447015in}{1.341519in}}%
\pgfpathlineto{\pgfqpoint{1.443279in}{1.349084in}}%
\pgfpathlineto{\pgfqpoint{1.439540in}{1.356600in}}%
\pgfpathlineto{\pgfqpoint{1.440641in}{1.360601in}}%
\pgfpathlineto{\pgfqpoint{1.441481in}{1.364615in}}%
\pgfpathlineto{\pgfqpoint{1.442059in}{1.368637in}}%
\pgfpathlineto{\pgfqpoint{1.442374in}{1.372663in}}%
\pgfpathclose%
\pgfusepath{fill}%
\end{pgfscope}%
\begin{pgfscope}%
\pgfpathrectangle{\pgfqpoint{0.041670in}{0.041670in}}{\pgfqpoint{2.216660in}{2.216660in}}%
\pgfusepath{clip}%
\pgfsetbuttcap%
\pgfsetroundjoin%
\definecolor{currentfill}{rgb}{0.762373,0.876424,0.137064}%
\pgfsetfillcolor{currentfill}%
\pgfsetlinewidth{0.000000pt}%
\definecolor{currentstroke}{rgb}{0.000000,0.000000,0.000000}%
\pgfsetstrokecolor{currentstroke}%
\pgfsetdash{}{0pt}%
\pgfpathmoveto{\pgfqpoint{1.236586in}{1.638770in}}%
\pgfpathlineto{\pgfqpoint{1.238330in}{1.636270in}}%
\pgfpathlineto{\pgfqpoint{1.240072in}{1.633659in}}%
\pgfpathlineto{\pgfqpoint{1.241812in}{1.630937in}}%
\pgfpathlineto{\pgfqpoint{1.243550in}{1.628106in}}%
\pgfpathlineto{\pgfqpoint{1.247429in}{1.627138in}}%
\pgfpathlineto{\pgfqpoint{1.251242in}{1.626114in}}%
\pgfpathlineto{\pgfqpoint{1.254986in}{1.625033in}}%
\pgfpathlineto{\pgfqpoint{1.258657in}{1.623897in}}%
\pgfpathlineto{\pgfqpoint{1.256505in}{1.626845in}}%
\pgfpathlineto{\pgfqpoint{1.254350in}{1.629684in}}%
\pgfpathlineto{\pgfqpoint{1.252193in}{1.632412in}}%
\pgfpathlineto{\pgfqpoint{1.250034in}{1.635028in}}%
\pgfpathlineto{\pgfqpoint{1.246766in}{1.636038in}}%
\pgfpathlineto{\pgfqpoint{1.243433in}{1.636998in}}%
\pgfpathlineto{\pgfqpoint{1.240039in}{1.637909in}}%
\pgfpathlineto{\pgfqpoint{1.236586in}{1.638770in}}%
\pgfpathclose%
\pgfusepath{fill}%
\end{pgfscope}%
\begin{pgfscope}%
\pgfpathrectangle{\pgfqpoint{0.041670in}{0.041670in}}{\pgfqpoint{2.216660in}{2.216660in}}%
\pgfusepath{clip}%
\pgfsetbuttcap%
\pgfsetroundjoin%
\definecolor{currentfill}{rgb}{0.699415,0.867117,0.175971}%
\pgfsetfillcolor{currentfill}%
\pgfsetlinewidth{0.000000pt}%
\definecolor{currentstroke}{rgb}{0.000000,0.000000,0.000000}%
\pgfsetstrokecolor{currentstroke}%
\pgfsetdash{}{0pt}%
\pgfpathmoveto{\pgfqpoint{1.072376in}{1.611575in}}%
\pgfpathlineto{\pgfqpoint{1.069431in}{1.608176in}}%
\pgfpathlineto{\pgfqpoint{1.066489in}{1.604669in}}%
\pgfpathlineto{\pgfqpoint{1.063550in}{1.601058in}}%
\pgfpathlineto{\pgfqpoint{1.060614in}{1.597342in}}%
\pgfpathlineto{\pgfqpoint{1.063804in}{1.599089in}}%
\pgfpathlineto{\pgfqpoint{1.067108in}{1.600787in}}%
\pgfpathlineto{\pgfqpoint{1.070522in}{1.602435in}}%
\pgfpathlineto{\pgfqpoint{1.074042in}{1.604031in}}%
\pgfpathlineto{\pgfqpoint{1.076645in}{1.607578in}}%
\pgfpathlineto{\pgfqpoint{1.079251in}{1.611022in}}%
\pgfpathlineto{\pgfqpoint{1.081859in}{1.614360in}}%
\pgfpathlineto{\pgfqpoint{1.084471in}{1.617591in}}%
\pgfpathlineto{\pgfqpoint{1.081300in}{1.616156in}}%
\pgfpathlineto{\pgfqpoint{1.078225in}{1.614673in}}%
\pgfpathlineto{\pgfqpoint{1.075249in}{1.613146in}}%
\pgfpathlineto{\pgfqpoint{1.072376in}{1.611575in}}%
\pgfpathclose%
\pgfusepath{fill}%
\end{pgfscope}%
\begin{pgfscope}%
\pgfpathrectangle{\pgfqpoint{0.041670in}{0.041670in}}{\pgfqpoint{2.216660in}{2.216660in}}%
\pgfusepath{clip}%
\pgfsetbuttcap%
\pgfsetroundjoin%
\definecolor{currentfill}{rgb}{0.120081,0.622161,0.534946}%
\pgfsetfillcolor{currentfill}%
\pgfsetlinewidth{0.000000pt}%
\definecolor{currentstroke}{rgb}{0.000000,0.000000,0.000000}%
\pgfsetstrokecolor{currentstroke}%
\pgfsetdash{}{0pt}%
\pgfpathmoveto{\pgfqpoint{1.469371in}{1.295237in}}%
\pgfpathlineto{\pgfqpoint{1.473087in}{1.287398in}}%
\pgfpathlineto{\pgfqpoint{1.476800in}{1.279529in}}%
\pgfpathlineto{\pgfqpoint{1.480511in}{1.271633in}}%
\pgfpathlineto{\pgfqpoint{1.484219in}{1.263713in}}%
\pgfpathlineto{\pgfqpoint{1.482652in}{1.258978in}}%
\pgfpathlineto{\pgfqpoint{1.480781in}{1.254267in}}%
\pgfpathlineto{\pgfqpoint{1.478605in}{1.249584in}}%
\pgfpathlineto{\pgfqpoint{1.476126in}{1.244934in}}%
\pgfpathlineto{\pgfqpoint{1.472507in}{1.253105in}}%
\pgfpathlineto{\pgfqpoint{1.468886in}{1.261251in}}%
\pgfpathlineto{\pgfqpoint{1.465262in}{1.269370in}}%
\pgfpathlineto{\pgfqpoint{1.461636in}{1.277459in}}%
\pgfpathlineto{\pgfqpoint{1.464003in}{1.281860in}}%
\pgfpathlineto{\pgfqpoint{1.466082in}{1.286294in}}%
\pgfpathlineto{\pgfqpoint{1.467872in}{1.290754in}}%
\pgfpathlineto{\pgfqpoint{1.469371in}{1.295237in}}%
\pgfpathclose%
\pgfusepath{fill}%
\end{pgfscope}%
\begin{pgfscope}%
\pgfpathrectangle{\pgfqpoint{0.041670in}{0.041670in}}{\pgfqpoint{2.216660in}{2.216660in}}%
\pgfusepath{clip}%
\pgfsetbuttcap%
\pgfsetroundjoin%
\definecolor{currentfill}{rgb}{0.147607,0.511733,0.557049}%
\pgfsetfillcolor{currentfill}%
\pgfsetlinewidth{0.000000pt}%
\definecolor{currentstroke}{rgb}{0.000000,0.000000,0.000000}%
\pgfsetstrokecolor{currentstroke}%
\pgfsetdash{}{0pt}%
\pgfpathmoveto{\pgfqpoint{0.872820in}{1.154509in}}%
\pgfpathlineto{\pgfqpoint{0.869408in}{1.145936in}}%
\pgfpathlineto{\pgfqpoint{0.865998in}{1.137364in}}%
\pgfpathlineto{\pgfqpoint{0.862590in}{1.128798in}}%
\pgfpathlineto{\pgfqpoint{0.859183in}{1.120240in}}%
\pgfpathlineto{\pgfqpoint{0.854714in}{1.125382in}}%
\pgfpathlineto{\pgfqpoint{0.850579in}{1.130588in}}%
\pgfpathlineto{\pgfqpoint{0.846781in}{1.135854in}}%
\pgfpathlineto{\pgfqpoint{0.843324in}{1.141173in}}%
\pgfpathlineto{\pgfqpoint{0.846889in}{1.149484in}}%
\pgfpathlineto{\pgfqpoint{0.850455in}{1.157804in}}%
\pgfpathlineto{\pgfqpoint{0.854024in}{1.166129in}}%
\pgfpathlineto{\pgfqpoint{0.857595in}{1.174456in}}%
\pgfpathlineto{\pgfqpoint{0.860916in}{1.169387in}}%
\pgfpathlineto{\pgfqpoint{0.864562in}{1.164369in}}%
\pgfpathlineto{\pgfqpoint{0.868532in}{1.159408in}}%
\pgfpathlineto{\pgfqpoint{0.872820in}{1.154509in}}%
\pgfpathclose%
\pgfusepath{fill}%
\end{pgfscope}%
\begin{pgfscope}%
\pgfpathrectangle{\pgfqpoint{0.041670in}{0.041670in}}{\pgfqpoint{2.216660in}{2.216660in}}%
\pgfusepath{clip}%
\pgfsetbuttcap%
\pgfsetroundjoin%
\definecolor{currentfill}{rgb}{0.762373,0.876424,0.137064}%
\pgfsetfillcolor{currentfill}%
\pgfsetlinewidth{0.000000pt}%
\definecolor{currentstroke}{rgb}{0.000000,0.000000,0.000000}%
\pgfsetstrokecolor{currentstroke}%
\pgfsetdash{}{0pt}%
\pgfpathmoveto{\pgfqpoint{1.107029in}{1.634091in}}%
\pgfpathlineto{\pgfqpoint{1.104781in}{1.631445in}}%
\pgfpathlineto{\pgfqpoint{1.102536in}{1.628688in}}%
\pgfpathlineto{\pgfqpoint{1.100294in}{1.625820in}}%
\pgfpathlineto{\pgfqpoint{1.098053in}{1.622843in}}%
\pgfpathlineto{\pgfqpoint{1.101657in}{1.624026in}}%
\pgfpathlineto{\pgfqpoint{1.105336in}{1.625156in}}%
\pgfpathlineto{\pgfqpoint{1.109088in}{1.626230in}}%
\pgfpathlineto{\pgfqpoint{1.112909in}{1.627249in}}%
\pgfpathlineto{\pgfqpoint{1.114741in}{1.630104in}}%
\pgfpathlineto{\pgfqpoint{1.116576in}{1.632849in}}%
\pgfpathlineto{\pgfqpoint{1.118413in}{1.635484in}}%
\pgfpathlineto{\pgfqpoint{1.120252in}{1.638007in}}%
\pgfpathlineto{\pgfqpoint{1.116851in}{1.637102in}}%
\pgfpathlineto{\pgfqpoint{1.113511in}{1.636147in}}%
\pgfpathlineto{\pgfqpoint{1.110236in}{1.635143in}}%
\pgfpathlineto{\pgfqpoint{1.107029in}{1.634091in}}%
\pgfpathclose%
\pgfusepath{fill}%
\end{pgfscope}%
\begin{pgfscope}%
\pgfpathrectangle{\pgfqpoint{0.041670in}{0.041670in}}{\pgfqpoint{2.216660in}{2.216660in}}%
\pgfusepath{clip}%
\pgfsetbuttcap%
\pgfsetroundjoin%
\definecolor{currentfill}{rgb}{0.195860,0.395433,0.555276}%
\pgfsetfillcolor{currentfill}%
\pgfsetlinewidth{0.000000pt}%
\definecolor{currentstroke}{rgb}{0.000000,0.000000,0.000000}%
\pgfsetstrokecolor{currentstroke}%
\pgfsetdash{}{0pt}%
\pgfpathmoveto{\pgfqpoint{0.854828in}{1.030706in}}%
\pgfpathlineto{\pgfqpoint{0.851647in}{1.022114in}}%
\pgfpathlineto{\pgfqpoint{0.848466in}{1.013561in}}%
\pgfpathlineto{\pgfqpoint{0.845286in}{1.005048in}}%
\pgfpathlineto{\pgfqpoint{0.842107in}{0.996580in}}%
\pgfpathlineto{\pgfqpoint{0.835640in}{1.002105in}}%
\pgfpathlineto{\pgfqpoint{0.829531in}{1.007728in}}%
\pgfpathlineto{\pgfqpoint{0.823786in}{1.013443in}}%
\pgfpathlineto{\pgfqpoint{0.818410in}{1.019243in}}%
\pgfpathlineto{\pgfqpoint{0.821801in}{1.027472in}}%
\pgfpathlineto{\pgfqpoint{0.825194in}{1.035745in}}%
\pgfpathlineto{\pgfqpoint{0.828587in}{1.044060in}}%
\pgfpathlineto{\pgfqpoint{0.831981in}{1.052413in}}%
\pgfpathlineto{\pgfqpoint{0.837166in}{1.046858in}}%
\pgfpathlineto{\pgfqpoint{0.842706in}{1.041384in}}%
\pgfpathlineto{\pgfqpoint{0.848595in}{1.035998in}}%
\pgfpathlineto{\pgfqpoint{0.854828in}{1.030706in}}%
\pgfpathclose%
\pgfusepath{fill}%
\end{pgfscope}%
\begin{pgfscope}%
\pgfpathrectangle{\pgfqpoint{0.041670in}{0.041670in}}{\pgfqpoint{2.216660in}{2.216660in}}%
\pgfusepath{clip}%
\pgfsetbuttcap%
\pgfsetroundjoin%
\definecolor{currentfill}{rgb}{0.268510,0.009605,0.335427}%
\pgfsetfillcolor{currentfill}%
\pgfsetlinewidth{0.000000pt}%
\definecolor{currentstroke}{rgb}{0.000000,0.000000,0.000000}%
\pgfsetstrokecolor{currentstroke}%
\pgfsetdash{}{0pt}%
\pgfpathmoveto{\pgfqpoint{1.706893in}{0.711057in}}%
\pgfpathlineto{\pgfqpoint{1.710298in}{0.711317in}}%
\pgfpathlineto{\pgfqpoint{1.713712in}{0.711857in}}%
\pgfpathlineto{\pgfqpoint{1.717135in}{0.712681in}}%
\pgfpathlineto{\pgfqpoint{1.720567in}{0.713795in}}%
\pgfpathlineto{\pgfqpoint{1.710566in}{0.704764in}}%
\pgfpathlineto{\pgfqpoint{1.699999in}{0.695895in}}%
\pgfpathlineto{\pgfqpoint{1.688874in}{0.687198in}}%
\pgfpathlineto{\pgfqpoint{1.677201in}{0.678685in}}%
\pgfpathlineto{\pgfqpoint{1.674029in}{0.677789in}}%
\pgfpathlineto{\pgfqpoint{1.670866in}{0.677184in}}%
\pgfpathlineto{\pgfqpoint{1.667712in}{0.676865in}}%
\pgfpathlineto{\pgfqpoint{1.664566in}{0.676826in}}%
\pgfpathlineto{\pgfqpoint{1.675957in}{0.685125in}}%
\pgfpathlineto{\pgfqpoint{1.686815in}{0.693603in}}%
\pgfpathlineto{\pgfqpoint{1.697130in}{0.702251in}}%
\pgfpathlineto{\pgfqpoint{1.706893in}{0.711057in}}%
\pgfpathclose%
\pgfusepath{fill}%
\end{pgfscope}%
\begin{pgfscope}%
\pgfpathrectangle{\pgfqpoint{0.041670in}{0.041670in}}{\pgfqpoint{2.216660in}{2.216660in}}%
\pgfusepath{clip}%
\pgfsetbuttcap%
\pgfsetroundjoin%
\definecolor{currentfill}{rgb}{0.282327,0.094955,0.417331}%
\pgfsetfillcolor{currentfill}%
\pgfsetlinewidth{0.000000pt}%
\definecolor{currentstroke}{rgb}{0.000000,0.000000,0.000000}%
\pgfsetstrokecolor{currentstroke}%
\pgfsetdash{}{0pt}%
\pgfpathmoveto{\pgfqpoint{0.620799in}{0.725384in}}%
\pgfpathlineto{\pgfqpoint{0.617323in}{0.729311in}}%
\pgfpathlineto{\pgfqpoint{0.613834in}{0.733585in}}%
\pgfpathlineto{\pgfqpoint{0.610333in}{0.738213in}}%
\pgfpathlineto{\pgfqpoint{0.606818in}{0.743200in}}%
\pgfpathlineto{\pgfqpoint{0.596164in}{0.752903in}}%
\pgfpathlineto{\pgfqpoint{0.586133in}{0.762767in}}%
\pgfpathlineto{\pgfqpoint{0.576736in}{0.772779in}}%
\pgfpathlineto{\pgfqpoint{0.567978in}{0.782930in}}%
\pgfpathlineto{\pgfqpoint{0.571710in}{0.777727in}}%
\pgfpathlineto{\pgfqpoint{0.575429in}{0.772883in}}%
\pgfpathlineto{\pgfqpoint{0.579135in}{0.768390in}}%
\pgfpathlineto{\pgfqpoint{0.582828in}{0.764243in}}%
\pgfpathlineto{\pgfqpoint{0.591393in}{0.754314in}}%
\pgfpathlineto{\pgfqpoint{0.600581in}{0.744521in}}%
\pgfpathlineto{\pgfqpoint{0.610386in}{0.734873in}}%
\pgfpathlineto{\pgfqpoint{0.620799in}{0.725384in}}%
\pgfpathclose%
\pgfusepath{fill}%
\end{pgfscope}%
\begin{pgfscope}%
\pgfpathrectangle{\pgfqpoint{0.041670in}{0.041670in}}{\pgfqpoint{2.216660in}{2.216660in}}%
\pgfusepath{clip}%
\pgfsetbuttcap%
\pgfsetroundjoin%
\definecolor{currentfill}{rgb}{0.281477,0.755203,0.432552}%
\pgfsetfillcolor{currentfill}%
\pgfsetlinewidth{0.000000pt}%
\definecolor{currentstroke}{rgb}{0.000000,0.000000,0.000000}%
\pgfsetstrokecolor{currentstroke}%
\pgfsetdash{}{0pt}%
\pgfpathmoveto{\pgfqpoint{1.410961in}{1.442886in}}%
\pgfpathlineto{\pgfqpoint{1.414731in}{1.436332in}}%
\pgfpathlineto{\pgfqpoint{1.418497in}{1.429712in}}%
\pgfpathlineto{\pgfqpoint{1.422260in}{1.423029in}}%
\pgfpathlineto{\pgfqpoint{1.426019in}{1.416284in}}%
\pgfpathlineto{\pgfqpoint{1.426698in}{1.412525in}}%
\pgfpathlineto{\pgfqpoint{1.427131in}{1.408756in}}%
\pgfpathlineto{\pgfqpoint{1.427319in}{1.404980in}}%
\pgfpathlineto{\pgfqpoint{1.427259in}{1.401201in}}%
\pgfpathlineto{\pgfqpoint{1.423473in}{1.408192in}}%
\pgfpathlineto{\pgfqpoint{1.419683in}{1.415121in}}%
\pgfpathlineto{\pgfqpoint{1.415890in}{1.421986in}}%
\pgfpathlineto{\pgfqpoint{1.412094in}{1.428785in}}%
\pgfpathlineto{\pgfqpoint{1.412157in}{1.432317in}}%
\pgfpathlineto{\pgfqpoint{1.411989in}{1.435847in}}%
\pgfpathlineto{\pgfqpoint{1.411590in}{1.439371in}}%
\pgfpathlineto{\pgfqpoint{1.410961in}{1.442886in}}%
\pgfpathclose%
\pgfusepath{fill}%
\end{pgfscope}%
\begin{pgfscope}%
\pgfpathrectangle{\pgfqpoint{0.041670in}{0.041670in}}{\pgfqpoint{2.216660in}{2.216660in}}%
\pgfusepath{clip}%
\pgfsetbuttcap%
\pgfsetroundjoin%
\definecolor{currentfill}{rgb}{0.248629,0.278775,0.534556}%
\pgfsetfillcolor{currentfill}%
\pgfsetlinewidth{0.000000pt}%
\definecolor{currentstroke}{rgb}{0.000000,0.000000,0.000000}%
\pgfsetstrokecolor{currentstroke}%
\pgfsetdash{}{0pt}%
\pgfpathmoveto{\pgfqpoint{1.549403in}{0.936137in}}%
\pgfpathlineto{\pgfqpoint{1.552631in}{0.928266in}}%
\pgfpathlineto{\pgfqpoint{1.555860in}{0.920473in}}%
\pgfpathlineto{\pgfqpoint{1.559089in}{0.912760in}}%
\pgfpathlineto{\pgfqpoint{1.562318in}{0.905132in}}%
\pgfpathlineto{\pgfqpoint{1.555111in}{0.898912in}}%
\pgfpathlineto{\pgfqpoint{1.547510in}{0.892808in}}%
\pgfpathlineto{\pgfqpoint{1.539523in}{0.886828in}}%
\pgfpathlineto{\pgfqpoint{1.531157in}{0.880976in}}%
\pgfpathlineto{\pgfqpoint{1.528182in}{0.888835in}}%
\pgfpathlineto{\pgfqpoint{1.525206in}{0.896777in}}%
\pgfpathlineto{\pgfqpoint{1.522231in}{0.904800in}}%
\pgfpathlineto{\pgfqpoint{1.519256in}{0.912901in}}%
\pgfpathlineto{\pgfqpoint{1.527348in}{0.918529in}}%
\pgfpathlineto{\pgfqpoint{1.535074in}{0.924282in}}%
\pgfpathlineto{\pgfqpoint{1.542428in}{0.930153in}}%
\pgfpathlineto{\pgfqpoint{1.549403in}{0.936137in}}%
\pgfpathclose%
\pgfusepath{fill}%
\end{pgfscope}%
\begin{pgfscope}%
\pgfpathrectangle{\pgfqpoint{0.041670in}{0.041670in}}{\pgfqpoint{2.216660in}{2.216660in}}%
\pgfusepath{clip}%
\pgfsetbuttcap%
\pgfsetroundjoin%
\definecolor{currentfill}{rgb}{0.267004,0.004874,0.329415}%
\pgfsetfillcolor{currentfill}%
\pgfsetlinewidth{0.000000pt}%
\definecolor{currentstroke}{rgb}{0.000000,0.000000,0.000000}%
\pgfsetstrokecolor{currentstroke}%
\pgfsetdash{}{0pt}%
\pgfpathmoveto{\pgfqpoint{0.718152in}{0.672336in}}%
\pgfpathlineto{\pgfqpoint{0.715102in}{0.671259in}}%
\pgfpathlineto{\pgfqpoint{0.712045in}{0.670442in}}%
\pgfpathlineto{\pgfqpoint{0.708980in}{0.669889in}}%
\pgfpathlineto{\pgfqpoint{0.705907in}{0.669606in}}%
\pgfpathlineto{\pgfqpoint{0.694052in}{0.677739in}}%
\pgfpathlineto{\pgfqpoint{0.682720in}{0.686059in}}%
\pgfpathlineto{\pgfqpoint{0.671922in}{0.694556in}}%
\pgfpathlineto{\pgfqpoint{0.661668in}{0.703221in}}%
\pgfpathlineto{\pgfqpoint{0.665012in}{0.703286in}}%
\pgfpathlineto{\pgfqpoint{0.668348in}{0.703620in}}%
\pgfpathlineto{\pgfqpoint{0.671676in}{0.704218in}}%
\pgfpathlineto{\pgfqpoint{0.674996in}{0.705075in}}%
\pgfpathlineto{\pgfqpoint{0.685002in}{0.696635in}}%
\pgfpathlineto{\pgfqpoint{0.695536in}{0.688359in}}%
\pgfpathlineto{\pgfqpoint{0.706590in}{0.680256in}}%
\pgfpathlineto{\pgfqpoint{0.718152in}{0.672336in}}%
\pgfpathclose%
\pgfusepath{fill}%
\end{pgfscope}%
\begin{pgfscope}%
\pgfpathrectangle{\pgfqpoint{0.041670in}{0.041670in}}{\pgfqpoint{2.216660in}{2.216660in}}%
\pgfusepath{clip}%
\pgfsetbuttcap%
\pgfsetroundjoin%
\definecolor{currentfill}{rgb}{0.565498,0.842430,0.262877}%
\pgfsetfillcolor{currentfill}%
\pgfsetlinewidth{0.000000pt}%
\definecolor{currentstroke}{rgb}{0.000000,0.000000,0.000000}%
\pgfsetstrokecolor{currentstroke}%
\pgfsetdash{}{0pt}%
\pgfpathmoveto{\pgfqpoint{1.332079in}{1.566419in}}%
\pgfpathlineto{\pgfqpoint{1.335477in}{1.561860in}}%
\pgfpathlineto{\pgfqpoint{1.338872in}{1.557207in}}%
\pgfpathlineto{\pgfqpoint{1.342264in}{1.552460in}}%
\pgfpathlineto{\pgfqpoint{1.345652in}{1.547621in}}%
\pgfpathlineto{\pgfqpoint{1.348251in}{1.545124in}}%
\pgfpathlineto{\pgfqpoint{1.350685in}{1.542589in}}%
\pgfpathlineto{\pgfqpoint{1.352952in}{1.540017in}}%
\pgfpathlineto{\pgfqpoint{1.355049in}{1.537411in}}%
\pgfpathlineto{\pgfqpoint{1.351463in}{1.542467in}}%
\pgfpathlineto{\pgfqpoint{1.347874in}{1.547430in}}%
\pgfpathlineto{\pgfqpoint{1.344281in}{1.552299in}}%
\pgfpathlineto{\pgfqpoint{1.340685in}{1.557074in}}%
\pgfpathlineto{\pgfqpoint{1.338766in}{1.559458in}}%
\pgfpathlineto{\pgfqpoint{1.336690in}{1.561812in}}%
\pgfpathlineto{\pgfqpoint{1.334460in}{1.564133in}}%
\pgfpathlineto{\pgfqpoint{1.332079in}{1.566419in}}%
\pgfpathclose%
\pgfusepath{fill}%
\end{pgfscope}%
\begin{pgfscope}%
\pgfpathrectangle{\pgfqpoint{0.041670in}{0.041670in}}{\pgfqpoint{2.216660in}{2.216660in}}%
\pgfusepath{clip}%
\pgfsetbuttcap%
\pgfsetroundjoin%
\definecolor{currentfill}{rgb}{0.263663,0.237631,0.518762}%
\pgfsetfillcolor{currentfill}%
\pgfsetlinewidth{0.000000pt}%
\definecolor{currentstroke}{rgb}{0.000000,0.000000,0.000000}%
\pgfsetstrokecolor{currentstroke}%
\pgfsetdash{}{0pt}%
\pgfpathmoveto{\pgfqpoint{0.836500in}{0.875890in}}%
\pgfpathlineto{\pgfqpoint{0.833587in}{0.868071in}}%
\pgfpathlineto{\pgfqpoint{0.830673in}{0.860343in}}%
\pgfpathlineto{\pgfqpoint{0.827759in}{0.852710in}}%
\pgfpathlineto{\pgfqpoint{0.824843in}{0.845175in}}%
\pgfpathlineto{\pgfqpoint{0.815862in}{0.851125in}}%
\pgfpathlineto{\pgfqpoint{0.807266in}{0.857215in}}%
\pgfpathlineto{\pgfqpoint{0.799064in}{0.863439in}}%
\pgfpathlineto{\pgfqpoint{0.791262in}{0.869789in}}%
\pgfpathlineto{\pgfqpoint{0.794442in}{0.877097in}}%
\pgfpathlineto{\pgfqpoint{0.797622in}{0.884503in}}%
\pgfpathlineto{\pgfqpoint{0.800800in}{0.892005in}}%
\pgfpathlineto{\pgfqpoint{0.803979in}{0.899598in}}%
\pgfpathlineto{\pgfqpoint{0.811536in}{0.893481in}}%
\pgfpathlineto{\pgfqpoint{0.819480in}{0.887486in}}%
\pgfpathlineto{\pgfqpoint{0.827804in}{0.881620in}}%
\pgfpathlineto{\pgfqpoint{0.836500in}{0.875890in}}%
\pgfpathclose%
\pgfusepath{fill}%
\end{pgfscope}%
\begin{pgfscope}%
\pgfpathrectangle{\pgfqpoint{0.041670in}{0.041670in}}{\pgfqpoint{2.216660in}{2.216660in}}%
\pgfusepath{clip}%
\pgfsetbuttcap%
\pgfsetroundjoin%
\definecolor{currentfill}{rgb}{0.179019,0.433756,0.557430}%
\pgfsetfillcolor{currentfill}%
\pgfsetlinewidth{0.000000pt}%
\definecolor{currentstroke}{rgb}{0.000000,0.000000,0.000000}%
\pgfsetstrokecolor{currentstroke}%
\pgfsetdash{}{0pt}%
\pgfpathmoveto{\pgfqpoint{1.518488in}{1.090929in}}%
\pgfpathlineto{\pgfqpoint{1.521927in}{1.082509in}}%
\pgfpathlineto{\pgfqpoint{1.525364in}{1.074114in}}%
\pgfpathlineto{\pgfqpoint{1.528801in}{1.065749in}}%
\pgfpathlineto{\pgfqpoint{1.532236in}{1.057416in}}%
\pgfpathlineto{\pgfqpoint{1.527370in}{1.051792in}}%
\pgfpathlineto{\pgfqpoint{1.522145in}{1.046245in}}%
\pgfpathlineto{\pgfqpoint{1.516567in}{1.040781in}}%
\pgfpathlineto{\pgfqpoint{1.510639in}{1.035405in}}%
\pgfpathlineto{\pgfqpoint{1.507405in}{1.043980in}}%
\pgfpathlineto{\pgfqpoint{1.504169in}{1.052586in}}%
\pgfpathlineto{\pgfqpoint{1.500933in}{1.061220in}}%
\pgfpathlineto{\pgfqpoint{1.497695in}{1.069881in}}%
\pgfpathlineto{\pgfqpoint{1.503400in}{1.075021in}}%
\pgfpathlineto{\pgfqpoint{1.508770in}{1.080246in}}%
\pgfpathlineto{\pgfqpoint{1.513801in}{1.085551in}}%
\pgfpathlineto{\pgfqpoint{1.518488in}{1.090929in}}%
\pgfpathclose%
\pgfusepath{fill}%
\end{pgfscope}%
\begin{pgfscope}%
\pgfpathrectangle{\pgfqpoint{0.041670in}{0.041670in}}{\pgfqpoint{2.216660in}{2.216660in}}%
\pgfusepath{clip}%
\pgfsetbuttcap%
\pgfsetroundjoin%
\definecolor{currentfill}{rgb}{0.636902,0.856542,0.216620}%
\pgfsetfillcolor{currentfill}%
\pgfsetlinewidth{0.000000pt}%
\definecolor{currentstroke}{rgb}{0.000000,0.000000,0.000000}%
\pgfsetstrokecolor{currentstroke}%
\pgfsetdash{}{0pt}%
\pgfpathmoveto{\pgfqpoint{1.308464in}{1.591616in}}%
\pgfpathlineto{\pgfqpoint{1.311625in}{1.587654in}}%
\pgfpathlineto{\pgfqpoint{1.314783in}{1.583590in}}%
\pgfpathlineto{\pgfqpoint{1.317938in}{1.579427in}}%
\pgfpathlineto{\pgfqpoint{1.321090in}{1.575167in}}%
\pgfpathlineto{\pgfqpoint{1.324051in}{1.573043in}}%
\pgfpathlineto{\pgfqpoint{1.326872in}{1.570876in}}%
\pgfpathlineto{\pgfqpoint{1.329548in}{1.568667in}}%
\pgfpathlineto{\pgfqpoint{1.332079in}{1.566419in}}%
\pgfpathlineto{\pgfqpoint{1.328677in}{1.570881in}}%
\pgfpathlineto{\pgfqpoint{1.325272in}{1.575245in}}%
\pgfpathlineto{\pgfqpoint{1.321864in}{1.579509in}}%
\pgfpathlineto{\pgfqpoint{1.318453in}{1.583672in}}%
\pgfpathlineto{\pgfqpoint{1.316153in}{1.585714in}}%
\pgfpathlineto{\pgfqpoint{1.313720in}{1.587720in}}%
\pgfpathlineto{\pgfqpoint{1.311156in}{1.589688in}}%
\pgfpathlineto{\pgfqpoint{1.308464in}{1.591616in}}%
\pgfpathclose%
\pgfusepath{fill}%
\end{pgfscope}%
\begin{pgfscope}%
\pgfpathrectangle{\pgfqpoint{0.041670in}{0.041670in}}{\pgfqpoint{2.216660in}{2.216660in}}%
\pgfusepath{clip}%
\pgfsetbuttcap%
\pgfsetroundjoin%
\definecolor{currentfill}{rgb}{0.487026,0.823929,0.312321}%
\pgfsetfillcolor{currentfill}%
\pgfsetlinewidth{0.000000pt}%
\definecolor{currentstroke}{rgb}{0.000000,0.000000,0.000000}%
\pgfsetstrokecolor{currentstroke}%
\pgfsetdash{}{0pt}%
\pgfpathmoveto{\pgfqpoint{1.355049in}{1.537411in}}%
\pgfpathlineto{\pgfqpoint{1.358631in}{1.532266in}}%
\pgfpathlineto{\pgfqpoint{1.362210in}{1.527033in}}%
\pgfpathlineto{\pgfqpoint{1.365786in}{1.521712in}}%
\pgfpathlineto{\pgfqpoint{1.369358in}{1.516307in}}%
\pgfpathlineto{\pgfqpoint{1.371447in}{1.513446in}}%
\pgfpathlineto{\pgfqpoint{1.373346in}{1.510554in}}%
\pgfpathlineto{\pgfqpoint{1.375056in}{1.507633in}}%
\pgfpathlineto{\pgfqpoint{1.376573in}{1.504686in}}%
\pgfpathlineto{\pgfqpoint{1.372858in}{1.510320in}}%
\pgfpathlineto{\pgfqpoint{1.369140in}{1.515869in}}%
\pgfpathlineto{\pgfqpoint{1.365419in}{1.521331in}}%
\pgfpathlineto{\pgfqpoint{1.361694in}{1.526704in}}%
\pgfpathlineto{\pgfqpoint{1.360298in}{1.529419in}}%
\pgfpathlineto{\pgfqpoint{1.358724in}{1.532110in}}%
\pgfpathlineto{\pgfqpoint{1.356973in}{1.534775in}}%
\pgfpathlineto{\pgfqpoint{1.355049in}{1.537411in}}%
\pgfpathclose%
\pgfusepath{fill}%
\end{pgfscope}%
\begin{pgfscope}%
\pgfpathrectangle{\pgfqpoint{0.041670in}{0.041670in}}{\pgfqpoint{2.216660in}{2.216660in}}%
\pgfusepath{clip}%
\pgfsetbuttcap%
\pgfsetroundjoin%
\definecolor{currentfill}{rgb}{0.166383,0.690856,0.496502}%
\pgfsetfillcolor{currentfill}%
\pgfsetlinewidth{0.000000pt}%
\definecolor{currentstroke}{rgb}{0.000000,0.000000,0.000000}%
\pgfsetstrokecolor{currentstroke}%
\pgfsetdash{}{0pt}%
\pgfpathmoveto{\pgfqpoint{0.921568in}{1.353057in}}%
\pgfpathlineto{\pgfqpoint{0.917845in}{1.345486in}}%
\pgfpathlineto{\pgfqpoint{0.914124in}{1.337866in}}%
\pgfpathlineto{\pgfqpoint{0.910406in}{1.330199in}}%
\pgfpathlineto{\pgfqpoint{0.906690in}{1.322488in}}%
\pgfpathlineto{\pgfqpoint{0.905289in}{1.326723in}}%
\pgfpathlineto{\pgfqpoint{0.904164in}{1.330975in}}%
\pgfpathlineto{\pgfqpoint{0.903316in}{1.335240in}}%
\pgfpathlineto{\pgfqpoint{0.902746in}{1.339512in}}%
\pgfpathlineto{\pgfqpoint{0.906505in}{1.346973in}}%
\pgfpathlineto{\pgfqpoint{0.910268in}{1.354390in}}%
\pgfpathlineto{\pgfqpoint{0.914034in}{1.361761in}}%
\pgfpathlineto{\pgfqpoint{0.917803in}{1.369084in}}%
\pgfpathlineto{\pgfqpoint{0.918352in}{1.365061in}}%
\pgfpathlineto{\pgfqpoint{0.919162in}{1.361046in}}%
\pgfpathlineto{\pgfqpoint{0.920235in}{1.357044in}}%
\pgfpathlineto{\pgfqpoint{0.921568in}{1.353057in}}%
\pgfpathclose%
\pgfusepath{fill}%
\end{pgfscope}%
\begin{pgfscope}%
\pgfpathrectangle{\pgfqpoint{0.041670in}{0.041670in}}{\pgfqpoint{2.216660in}{2.216660in}}%
\pgfusepath{clip}%
\pgfsetbuttcap%
\pgfsetroundjoin%
\definecolor{currentfill}{rgb}{0.133743,0.548535,0.553541}%
\pgfsetfillcolor{currentfill}%
\pgfsetlinewidth{0.000000pt}%
\definecolor{currentstroke}{rgb}{0.000000,0.000000,0.000000}%
\pgfsetstrokecolor{currentstroke}%
\pgfsetdash{}{0pt}%
\pgfpathmoveto{\pgfqpoint{1.490577in}{1.212057in}}%
\pgfpathlineto{\pgfqpoint{1.494184in}{1.203803in}}%
\pgfpathlineto{\pgfqpoint{1.497789in}{1.195541in}}%
\pgfpathlineto{\pgfqpoint{1.501391in}{1.187272in}}%
\pgfpathlineto{\pgfqpoint{1.504991in}{1.179001in}}%
\pgfpathlineto{\pgfqpoint{1.501962in}{1.173890in}}%
\pgfpathlineto{\pgfqpoint{1.498605in}{1.168827in}}%
\pgfpathlineto{\pgfqpoint{1.494922in}{1.163815in}}%
\pgfpathlineto{\pgfqpoint{1.490917in}{1.158861in}}%
\pgfpathlineto{\pgfqpoint{1.487463in}{1.167379in}}%
\pgfpathlineto{\pgfqpoint{1.484007in}{1.175895in}}%
\pgfpathlineto{\pgfqpoint{1.480549in}{1.184404in}}%
\pgfpathlineto{\pgfqpoint{1.477089in}{1.192905in}}%
\pgfpathlineto{\pgfqpoint{1.480926in}{1.197616in}}%
\pgfpathlineto{\pgfqpoint{1.484454in}{1.202382in}}%
\pgfpathlineto{\pgfqpoint{1.487672in}{1.207197in}}%
\pgfpathlineto{\pgfqpoint{1.490577in}{1.212057in}}%
\pgfpathclose%
\pgfusepath{fill}%
\end{pgfscope}%
\begin{pgfscope}%
\pgfpathrectangle{\pgfqpoint{0.041670in}{0.041670in}}{\pgfqpoint{2.216660in}{2.216660in}}%
\pgfusepath{clip}%
\pgfsetbuttcap%
\pgfsetroundjoin%
\definecolor{currentfill}{rgb}{0.814576,0.883393,0.110347}%
\pgfsetfillcolor{currentfill}%
\pgfsetlinewidth{0.000000pt}%
\definecolor{currentstroke}{rgb}{0.000000,0.000000,0.000000}%
\pgfsetstrokecolor{currentstroke}%
\pgfsetdash{}{0pt}%
\pgfpathmoveto{\pgfqpoint{1.166493in}{1.652834in}}%
\pgfpathlineto{\pgfqpoint{1.166019in}{1.650980in}}%
\pgfpathlineto{\pgfqpoint{1.165545in}{1.649009in}}%
\pgfpathlineto{\pgfqpoint{1.165071in}{1.646922in}}%
\pgfpathlineto{\pgfqpoint{1.164598in}{1.644721in}}%
\pgfpathlineto{\pgfqpoint{1.168479in}{1.644919in}}%
\pgfpathlineto{\pgfqpoint{1.172370in}{1.645059in}}%
\pgfpathlineto{\pgfqpoint{1.176269in}{1.645142in}}%
\pgfpathlineto{\pgfqpoint{1.180172in}{1.645168in}}%
\pgfpathlineto{\pgfqpoint{1.180165in}{1.647355in}}%
\pgfpathlineto{\pgfqpoint{1.180158in}{1.649427in}}%
\pgfpathlineto{\pgfqpoint{1.180152in}{1.651384in}}%
\pgfpathlineto{\pgfqpoint{1.180145in}{1.653225in}}%
\pgfpathlineto{\pgfqpoint{1.176724in}{1.653202in}}%
\pgfpathlineto{\pgfqpoint{1.173306in}{1.653130in}}%
\pgfpathlineto{\pgfqpoint{1.169895in}{1.653007in}}%
\pgfpathlineto{\pgfqpoint{1.166493in}{1.652834in}}%
\pgfpathclose%
\pgfusepath{fill}%
\end{pgfscope}%
\begin{pgfscope}%
\pgfpathrectangle{\pgfqpoint{0.041670in}{0.041670in}}{\pgfqpoint{2.216660in}{2.216660in}}%
\pgfusepath{clip}%
\pgfsetbuttcap%
\pgfsetroundjoin%
\definecolor{currentfill}{rgb}{0.814576,0.883393,0.110347}%
\pgfsetfillcolor{currentfill}%
\pgfsetlinewidth{0.000000pt}%
\definecolor{currentstroke}{rgb}{0.000000,0.000000,0.000000}%
\pgfsetstrokecolor{currentstroke}%
\pgfsetdash{}{0pt}%
\pgfpathmoveto{\pgfqpoint{1.180145in}{1.653225in}}%
\pgfpathlineto{\pgfqpoint{1.180152in}{1.651384in}}%
\pgfpathlineto{\pgfqpoint{1.180158in}{1.649427in}}%
\pgfpathlineto{\pgfqpoint{1.180165in}{1.647355in}}%
\pgfpathlineto{\pgfqpoint{1.180172in}{1.645168in}}%
\pgfpathlineto{\pgfqpoint{1.184074in}{1.645136in}}%
\pgfpathlineto{\pgfqpoint{1.187972in}{1.645047in}}%
\pgfpathlineto{\pgfqpoint{1.191863in}{1.644900in}}%
\pgfpathlineto{\pgfqpoint{1.195742in}{1.644696in}}%
\pgfpathlineto{\pgfqpoint{1.195256in}{1.646898in}}%
\pgfpathlineto{\pgfqpoint{1.194769in}{1.648985in}}%
\pgfpathlineto{\pgfqpoint{1.194282in}{1.650956in}}%
\pgfpathlineto{\pgfqpoint{1.193794in}{1.652811in}}%
\pgfpathlineto{\pgfqpoint{1.190394in}{1.652990in}}%
\pgfpathlineto{\pgfqpoint{1.186983in}{1.653118in}}%
\pgfpathlineto{\pgfqpoint{1.183566in}{1.653197in}}%
\pgfpathlineto{\pgfqpoint{1.180145in}{1.653225in}}%
\pgfpathclose%
\pgfusepath{fill}%
\end{pgfscope}%
\begin{pgfscope}%
\pgfpathrectangle{\pgfqpoint{0.041670in}{0.041670in}}{\pgfqpoint{2.216660in}{2.216660in}}%
\pgfusepath{clip}%
\pgfsetbuttcap%
\pgfsetroundjoin%
\definecolor{currentfill}{rgb}{0.260571,0.246922,0.522828}%
\pgfsetfillcolor{currentfill}%
\pgfsetlinewidth{0.000000pt}%
\definecolor{currentstroke}{rgb}{0.000000,0.000000,0.000000}%
\pgfsetstrokecolor{currentstroke}%
\pgfsetdash{}{0pt}%
\pgfpathmoveto{\pgfqpoint{0.537579in}{0.838214in}}%
\pgfpathlineto{\pgfqpoint{0.533706in}{0.846932in}}%
\pgfpathlineto{\pgfqpoint{0.529814in}{0.856075in}}%
\pgfpathlineto{\pgfqpoint{0.525905in}{0.865650in}}%
\pgfpathlineto{\pgfqpoint{0.521977in}{0.875664in}}%
\pgfpathlineto{\pgfqpoint{0.513326in}{0.886573in}}%
\pgfpathlineto{\pgfqpoint{0.505380in}{0.897601in}}%
\pgfpathlineto{\pgfqpoint{0.498142in}{0.908737in}}%
\pgfpathlineto{\pgfqpoint{0.491618in}{0.919968in}}%
\pgfpathlineto{\pgfqpoint{0.495702in}{0.909752in}}%
\pgfpathlineto{\pgfqpoint{0.499767in}{0.899973in}}%
\pgfpathlineto{\pgfqpoint{0.503814in}{0.890625in}}%
\pgfpathlineto{\pgfqpoint{0.507842in}{0.881698in}}%
\pgfpathlineto{\pgfqpoint{0.514237in}{0.870674in}}%
\pgfpathlineto{\pgfqpoint{0.521328in}{0.859743in}}%
\pgfpathlineto{\pgfqpoint{0.529110in}{0.848920in}}%
\pgfpathlineto{\pgfqpoint{0.537579in}{0.838214in}}%
\pgfpathclose%
\pgfusepath{fill}%
\end{pgfscope}%
\begin{pgfscope}%
\pgfpathrectangle{\pgfqpoint{0.041670in}{0.041670in}}{\pgfqpoint{2.216660in}{2.216660in}}%
\pgfusepath{clip}%
\pgfsetbuttcap%
\pgfsetroundjoin%
\definecolor{currentfill}{rgb}{0.120081,0.622161,0.534946}%
\pgfsetfillcolor{currentfill}%
\pgfsetlinewidth{0.000000pt}%
\definecolor{currentstroke}{rgb}{0.000000,0.000000,0.000000}%
\pgfsetstrokecolor{currentstroke}%
\pgfsetdash{}{0pt}%
\pgfpathmoveto{\pgfqpoint{0.900618in}{1.273577in}}%
\pgfpathlineto{\pgfqpoint{0.897019in}{1.265433in}}%
\pgfpathlineto{\pgfqpoint{0.893424in}{1.257259in}}%
\pgfpathlineto{\pgfqpoint{0.889830in}{1.249058in}}%
\pgfpathlineto{\pgfqpoint{0.886239in}{1.240832in}}%
\pgfpathlineto{\pgfqpoint{0.883493in}{1.245449in}}%
\pgfpathlineto{\pgfqpoint{0.881048in}{1.250103in}}%
\pgfpathlineto{\pgfqpoint{0.878906in}{1.254789in}}%
\pgfpathlineto{\pgfqpoint{0.877068in}{1.259503in}}%
\pgfpathlineto{\pgfqpoint{0.880761in}{1.267479in}}%
\pgfpathlineto{\pgfqpoint{0.884457in}{1.275431in}}%
\pgfpathlineto{\pgfqpoint{0.888156in}{1.283356in}}%
\pgfpathlineto{\pgfqpoint{0.891857in}{1.291251in}}%
\pgfpathlineto{\pgfqpoint{0.893615in}{1.286788in}}%
\pgfpathlineto{\pgfqpoint{0.895662in}{1.282351in}}%
\pgfpathlineto{\pgfqpoint{0.897997in}{1.277946in}}%
\pgfpathlineto{\pgfqpoint{0.900618in}{1.273577in}}%
\pgfpathclose%
\pgfusepath{fill}%
\end{pgfscope}%
\begin{pgfscope}%
\pgfpathrectangle{\pgfqpoint{0.041670in}{0.041670in}}{\pgfqpoint{2.216660in}{2.216660in}}%
\pgfusepath{clip}%
\pgfsetbuttcap%
\pgfsetroundjoin%
\definecolor{currentfill}{rgb}{0.762373,0.876424,0.137064}%
\pgfsetfillcolor{currentfill}%
\pgfsetlinewidth{0.000000pt}%
\definecolor{currentstroke}{rgb}{0.000000,0.000000,0.000000}%
\pgfsetstrokecolor{currentstroke}%
\pgfsetdash{}{0pt}%
\pgfpathmoveto{\pgfqpoint{1.250034in}{1.635028in}}%
\pgfpathlineto{\pgfqpoint{1.252193in}{1.632412in}}%
\pgfpathlineto{\pgfqpoint{1.254350in}{1.629684in}}%
\pgfpathlineto{\pgfqpoint{1.256505in}{1.626845in}}%
\pgfpathlineto{\pgfqpoint{1.258657in}{1.623897in}}%
\pgfpathlineto{\pgfqpoint{1.262252in}{1.622708in}}%
\pgfpathlineto{\pgfqpoint{1.265767in}{1.621466in}}%
\pgfpathlineto{\pgfqpoint{1.269198in}{1.620172in}}%
\pgfpathlineto{\pgfqpoint{1.272542in}{1.618828in}}%
\pgfpathlineto{\pgfqpoint{1.270008in}{1.621917in}}%
\pgfpathlineto{\pgfqpoint{1.267471in}{1.624896in}}%
\pgfpathlineto{\pgfqpoint{1.264932in}{1.627765in}}%
\pgfpathlineto{\pgfqpoint{1.262390in}{1.630522in}}%
\pgfpathlineto{\pgfqpoint{1.259414in}{1.631717in}}%
\pgfpathlineto{\pgfqpoint{1.256361in}{1.632867in}}%
\pgfpathlineto{\pgfqpoint{1.253233in}{1.633971in}}%
\pgfpathlineto{\pgfqpoint{1.250034in}{1.635028in}}%
\pgfpathclose%
\pgfusepath{fill}%
\end{pgfscope}%
\begin{pgfscope}%
\pgfpathrectangle{\pgfqpoint{0.041670in}{0.041670in}}{\pgfqpoint{2.216660in}{2.216660in}}%
\pgfusepath{clip}%
\pgfsetbuttcap%
\pgfsetroundjoin%
\definecolor{currentfill}{rgb}{0.814576,0.883393,0.110347}%
\pgfsetfillcolor{currentfill}%
\pgfsetlinewidth{0.000000pt}%
\definecolor{currentstroke}{rgb}{0.000000,0.000000,0.000000}%
\pgfsetstrokecolor{currentstroke}%
\pgfsetdash{}{0pt}%
\pgfpathmoveto{\pgfqpoint{1.153051in}{1.651645in}}%
\pgfpathlineto{\pgfqpoint{1.152102in}{1.649748in}}%
\pgfpathlineto{\pgfqpoint{1.151155in}{1.647735in}}%
\pgfpathlineto{\pgfqpoint{1.150208in}{1.645606in}}%
\pgfpathlineto{\pgfqpoint{1.149263in}{1.643363in}}%
\pgfpathlineto{\pgfqpoint{1.153061in}{1.643787in}}%
\pgfpathlineto{\pgfqpoint{1.156886in}{1.644155in}}%
\pgfpathlineto{\pgfqpoint{1.160732in}{1.644467in}}%
\pgfpathlineto{\pgfqpoint{1.164598in}{1.644721in}}%
\pgfpathlineto{\pgfqpoint{1.165071in}{1.646922in}}%
\pgfpathlineto{\pgfqpoint{1.165545in}{1.649009in}}%
\pgfpathlineto{\pgfqpoint{1.166019in}{1.650980in}}%
\pgfpathlineto{\pgfqpoint{1.166493in}{1.652834in}}%
\pgfpathlineto{\pgfqpoint{1.163105in}{1.652611in}}%
\pgfpathlineto{\pgfqpoint{1.159732in}{1.652338in}}%
\pgfpathlineto{\pgfqpoint{1.156380in}{1.652016in}}%
\pgfpathlineto{\pgfqpoint{1.153051in}{1.651645in}}%
\pgfpathclose%
\pgfusepath{fill}%
\end{pgfscope}%
\begin{pgfscope}%
\pgfpathrectangle{\pgfqpoint{0.041670in}{0.041670in}}{\pgfqpoint{2.216660in}{2.216660in}}%
\pgfusepath{clip}%
\pgfsetbuttcap%
\pgfsetroundjoin%
\definecolor{currentfill}{rgb}{0.565498,0.842430,0.262877}%
\pgfsetfillcolor{currentfill}%
\pgfsetlinewidth{0.000000pt}%
\definecolor{currentstroke}{rgb}{0.000000,0.000000,0.000000}%
\pgfsetstrokecolor{currentstroke}%
\pgfsetdash{}{0pt}%
\pgfpathmoveto{\pgfqpoint{1.017651in}{1.554930in}}%
\pgfpathlineto{\pgfqpoint{1.014019in}{1.550106in}}%
\pgfpathlineto{\pgfqpoint{1.010390in}{1.545187in}}%
\pgfpathlineto{\pgfqpoint{1.006764in}{1.540174in}}%
\pgfpathlineto{\pgfqpoint{1.003142in}{1.535069in}}%
\pgfpathlineto{\pgfqpoint{1.005086in}{1.537703in}}%
\pgfpathlineto{\pgfqpoint{1.007201in}{1.540304in}}%
\pgfpathlineto{\pgfqpoint{1.009487in}{1.542872in}}%
\pgfpathlineto{\pgfqpoint{1.011939in}{1.545403in}}%
\pgfpathlineto{\pgfqpoint{1.015376in}{1.550289in}}%
\pgfpathlineto{\pgfqpoint{1.018816in}{1.555083in}}%
\pgfpathlineto{\pgfqpoint{1.022260in}{1.559783in}}%
\pgfpathlineto{\pgfqpoint{1.025707in}{1.564389in}}%
\pgfpathlineto{\pgfqpoint{1.023460in}{1.562072in}}%
\pgfpathlineto{\pgfqpoint{1.021367in}{1.559722in}}%
\pgfpathlineto{\pgfqpoint{1.019430in}{1.557340in}}%
\pgfpathlineto{\pgfqpoint{1.017651in}{1.554930in}}%
\pgfpathclose%
\pgfusepath{fill}%
\end{pgfscope}%
\begin{pgfscope}%
\pgfpathrectangle{\pgfqpoint{0.041670in}{0.041670in}}{\pgfqpoint{2.216660in}{2.216660in}}%
\pgfusepath{clip}%
\pgfsetbuttcap%
\pgfsetroundjoin%
\definecolor{currentfill}{rgb}{0.814576,0.883393,0.110347}%
\pgfsetfillcolor{currentfill}%
\pgfsetlinewidth{0.000000pt}%
\definecolor{currentstroke}{rgb}{0.000000,0.000000,0.000000}%
\pgfsetstrokecolor{currentstroke}%
\pgfsetdash{}{0pt}%
\pgfpathmoveto{\pgfqpoint{1.193794in}{1.652811in}}%
\pgfpathlineto{\pgfqpoint{1.194282in}{1.650956in}}%
\pgfpathlineto{\pgfqpoint{1.194769in}{1.648985in}}%
\pgfpathlineto{\pgfqpoint{1.195256in}{1.646898in}}%
\pgfpathlineto{\pgfqpoint{1.195742in}{1.644696in}}%
\pgfpathlineto{\pgfqpoint{1.199606in}{1.644435in}}%
\pgfpathlineto{\pgfqpoint{1.203450in}{1.644117in}}%
\pgfpathlineto{\pgfqpoint{1.207272in}{1.643743in}}%
\pgfpathlineto{\pgfqpoint{1.211067in}{1.643313in}}%
\pgfpathlineto{\pgfqpoint{1.210109in}{1.645557in}}%
\pgfpathlineto{\pgfqpoint{1.209150in}{1.647688in}}%
\pgfpathlineto{\pgfqpoint{1.208189in}{1.649702in}}%
\pgfpathlineto{\pgfqpoint{1.207227in}{1.651600in}}%
\pgfpathlineto{\pgfqpoint{1.203901in}{1.651977in}}%
\pgfpathlineto{\pgfqpoint{1.200551in}{1.652305in}}%
\pgfpathlineto{\pgfqpoint{1.197181in}{1.652583in}}%
\pgfpathlineto{\pgfqpoint{1.193794in}{1.652811in}}%
\pgfpathclose%
\pgfusepath{fill}%
\end{pgfscope}%
\begin{pgfscope}%
\pgfpathrectangle{\pgfqpoint{0.041670in}{0.041670in}}{\pgfqpoint{2.216660in}{2.216660in}}%
\pgfusepath{clip}%
\pgfsetbuttcap%
\pgfsetroundjoin%
\definecolor{currentfill}{rgb}{0.281477,0.755203,0.432552}%
\pgfsetfillcolor{currentfill}%
\pgfsetlinewidth{0.000000pt}%
\definecolor{currentstroke}{rgb}{0.000000,0.000000,0.000000}%
\pgfsetstrokecolor{currentstroke}%
\pgfsetdash{}{0pt}%
\pgfpathmoveto{\pgfqpoint{0.948067in}{1.425645in}}%
\pgfpathlineto{\pgfqpoint{0.944273in}{1.418792in}}%
\pgfpathlineto{\pgfqpoint{0.940482in}{1.411872in}}%
\pgfpathlineto{\pgfqpoint{0.936695in}{1.404888in}}%
\pgfpathlineto{\pgfqpoint{0.932910in}{1.397843in}}%
\pgfpathlineto{\pgfqpoint{0.932632in}{1.401621in}}%
\pgfpathlineto{\pgfqpoint{0.932600in}{1.405400in}}%
\pgfpathlineto{\pgfqpoint{0.932815in}{1.409176in}}%
\pgfpathlineto{\pgfqpoint{0.933275in}{1.412944in}}%
\pgfpathlineto{\pgfqpoint{0.937046in}{1.419743in}}%
\pgfpathlineto{\pgfqpoint{0.940820in}{1.426480in}}%
\pgfpathlineto{\pgfqpoint{0.944597in}{1.433154in}}%
\pgfpathlineto{\pgfqpoint{0.948378in}{1.439762in}}%
\pgfpathlineto{\pgfqpoint{0.947953in}{1.436239in}}%
\pgfpathlineto{\pgfqpoint{0.947760in}{1.432710in}}%
\pgfpathlineto{\pgfqpoint{0.947798in}{1.429177in}}%
\pgfpathlineto{\pgfqpoint{0.948067in}{1.425645in}}%
\pgfpathclose%
\pgfusepath{fill}%
\end{pgfscope}%
\begin{pgfscope}%
\pgfpathrectangle{\pgfqpoint{0.041670in}{0.041670in}}{\pgfqpoint{2.216660in}{2.216660in}}%
\pgfusepath{clip}%
\pgfsetbuttcap%
\pgfsetroundjoin%
\definecolor{currentfill}{rgb}{0.636902,0.856542,0.216620}%
\pgfsetfillcolor{currentfill}%
\pgfsetlinewidth{0.000000pt}%
\definecolor{currentstroke}{rgb}{0.000000,0.000000,0.000000}%
\pgfsetstrokecolor{currentstroke}%
\pgfsetdash{}{0pt}%
\pgfpathmoveto{\pgfqpoint{1.039527in}{1.581829in}}%
\pgfpathlineto{\pgfqpoint{1.036067in}{1.577619in}}%
\pgfpathlineto{\pgfqpoint{1.032610in}{1.573308in}}%
\pgfpathlineto{\pgfqpoint{1.029157in}{1.568898in}}%
\pgfpathlineto{\pgfqpoint{1.025707in}{1.564389in}}%
\pgfpathlineto{\pgfqpoint{1.028105in}{1.566671in}}%
\pgfpathlineto{\pgfqpoint{1.030652in}{1.568915in}}%
\pgfpathlineto{\pgfqpoint{1.033344in}{1.571119in}}%
\pgfpathlineto{\pgfqpoint{1.036181in}{1.573281in}}%
\pgfpathlineto{\pgfqpoint{1.039392in}{1.577586in}}%
\pgfpathlineto{\pgfqpoint{1.042607in}{1.581792in}}%
\pgfpathlineto{\pgfqpoint{1.045825in}{1.585898in}}%
\pgfpathlineto{\pgfqpoint{1.049046in}{1.589904in}}%
\pgfpathlineto{\pgfqpoint{1.046468in}{1.587940in}}%
\pgfpathlineto{\pgfqpoint{1.044020in}{1.585938in}}%
\pgfpathlineto{\pgfqpoint{1.041706in}{1.583901in}}%
\pgfpathlineto{\pgfqpoint{1.039527in}{1.581829in}}%
\pgfpathclose%
\pgfusepath{fill}%
\end{pgfscope}%
\begin{pgfscope}%
\pgfpathrectangle{\pgfqpoint{0.041670in}{0.041670in}}{\pgfqpoint{2.216660in}{2.216660in}}%
\pgfusepath{clip}%
\pgfsetbuttcap%
\pgfsetroundjoin%
\definecolor{currentfill}{rgb}{0.282884,0.135920,0.453427}%
\pgfsetfillcolor{currentfill}%
\pgfsetlinewidth{0.000000pt}%
\definecolor{currentstroke}{rgb}{0.000000,0.000000,0.000000}%
\pgfsetstrokecolor{currentstroke}%
\pgfsetdash{}{0pt}%
\pgfpathmoveto{\pgfqpoint{1.799174in}{0.792059in}}%
\pgfpathlineto{\pgfqpoint{1.802960in}{0.797675in}}%
\pgfpathlineto{\pgfqpoint{1.806761in}{0.803662in}}%
\pgfpathlineto{\pgfqpoint{1.810577in}{0.810025in}}%
\pgfpathlineto{\pgfqpoint{1.814408in}{0.816772in}}%
\pgfpathlineto{\pgfqpoint{1.806044in}{0.806291in}}%
\pgfpathlineto{\pgfqpoint{1.797019in}{0.795940in}}%
\pgfpathlineto{\pgfqpoint{1.787337in}{0.785730in}}%
\pgfpathlineto{\pgfqpoint{1.777007in}{0.775673in}}%
\pgfpathlineto{\pgfqpoint{1.773382in}{0.769138in}}%
\pgfpathlineto{\pgfqpoint{1.769771in}{0.762988in}}%
\pgfpathlineto{\pgfqpoint{1.766175in}{0.757217in}}%
\pgfpathlineto{\pgfqpoint{1.762593in}{0.751817in}}%
\pgfpathlineto{\pgfqpoint{1.772693in}{0.761663in}}%
\pgfpathlineto{\pgfqpoint{1.782161in}{0.771660in}}%
\pgfpathlineto{\pgfqpoint{1.790991in}{0.781796in}}%
\pgfpathlineto{\pgfqpoint{1.799174in}{0.792059in}}%
\pgfpathclose%
\pgfusepath{fill}%
\end{pgfscope}%
\begin{pgfscope}%
\pgfpathrectangle{\pgfqpoint{0.041670in}{0.041670in}}{\pgfqpoint{2.216660in}{2.216660in}}%
\pgfusepath{clip}%
\pgfsetbuttcap%
\pgfsetroundjoin%
\definecolor{currentfill}{rgb}{0.762373,0.876424,0.137064}%
\pgfsetfillcolor{currentfill}%
\pgfsetlinewidth{0.000000pt}%
\definecolor{currentstroke}{rgb}{0.000000,0.000000,0.000000}%
\pgfsetstrokecolor{currentstroke}%
\pgfsetdash{}{0pt}%
\pgfpathmoveto{\pgfqpoint{1.094942in}{1.629424in}}%
\pgfpathlineto{\pgfqpoint{1.092321in}{1.626632in}}%
\pgfpathlineto{\pgfqpoint{1.089701in}{1.623729in}}%
\pgfpathlineto{\pgfqpoint{1.087085in}{1.620715in}}%
\pgfpathlineto{\pgfqpoint{1.084471in}{1.617591in}}%
\pgfpathlineto{\pgfqpoint{1.087735in}{1.618979in}}%
\pgfpathlineto{\pgfqpoint{1.091089in}{1.620318in}}%
\pgfpathlineto{\pgfqpoint{1.094530in}{1.621606in}}%
\pgfpathlineto{\pgfqpoint{1.098053in}{1.622843in}}%
\pgfpathlineto{\pgfqpoint{1.100294in}{1.625820in}}%
\pgfpathlineto{\pgfqpoint{1.102536in}{1.628688in}}%
\pgfpathlineto{\pgfqpoint{1.104781in}{1.631445in}}%
\pgfpathlineto{\pgfqpoint{1.107029in}{1.634091in}}%
\pgfpathlineto{\pgfqpoint{1.103893in}{1.632992in}}%
\pgfpathlineto{\pgfqpoint{1.100831in}{1.631847in}}%
\pgfpathlineto{\pgfqpoint{1.097847in}{1.630657in}}%
\pgfpathlineto{\pgfqpoint{1.094942in}{1.629424in}}%
\pgfpathclose%
\pgfusepath{fill}%
\end{pgfscope}%
\begin{pgfscope}%
\pgfpathrectangle{\pgfqpoint{0.041670in}{0.041670in}}{\pgfqpoint{2.216660in}{2.216660in}}%
\pgfusepath{clip}%
\pgfsetbuttcap%
\pgfsetroundjoin%
\definecolor{currentfill}{rgb}{0.172719,0.448791,0.557885}%
\pgfsetfillcolor{currentfill}%
\pgfsetlinewidth{0.000000pt}%
\definecolor{currentstroke}{rgb}{0.000000,0.000000,0.000000}%
\pgfsetstrokecolor{currentstroke}%
\pgfsetdash{}{0pt}%
\pgfpathmoveto{\pgfqpoint{1.924018in}{1.076116in}}%
\pgfpathlineto{\pgfqpoint{1.928394in}{1.090834in}}%
\pgfpathlineto{\pgfqpoint{1.932794in}{1.106065in}}%
\pgfpathlineto{\pgfqpoint{1.937220in}{1.121819in}}%
\pgfpathlineto{\pgfqpoint{1.934099in}{1.109777in}}%
\pgfpathlineto{\pgfqpoint{1.930207in}{1.097770in}}%
\pgfpathlineto{\pgfqpoint{1.925542in}{1.085811in}}%
\pgfpathlineto{\pgfqpoint{1.920107in}{1.073913in}}%
\pgfpathlineto{\pgfqpoint{1.915752in}{1.058334in}}%
\pgfpathlineto{\pgfqpoint{1.911422in}{1.043280in}}%
\pgfpathlineto{\pgfqpoint{1.907116in}{1.028742in}}%
\pgfpathlineto{\pgfqpoint{1.912478in}{1.040507in}}%
\pgfpathlineto{\pgfqpoint{1.917083in}{1.052333in}}%
\pgfpathlineto{\pgfqpoint{1.920929in}{1.064207in}}%
\pgfpathlineto{\pgfqpoint{1.924018in}{1.076116in}}%
\pgfpathclose%
\pgfusepath{fill}%
\end{pgfscope}%
\begin{pgfscope}%
\pgfpathrectangle{\pgfqpoint{0.041670in}{0.041670in}}{\pgfqpoint{2.216660in}{2.216660in}}%
\pgfusepath{clip}%
\pgfsetbuttcap%
\pgfsetroundjoin%
\definecolor{currentfill}{rgb}{0.699415,0.867117,0.175971}%
\pgfsetfillcolor{currentfill}%
\pgfsetlinewidth{0.000000pt}%
\definecolor{currentstroke}{rgb}{0.000000,0.000000,0.000000}%
\pgfsetstrokecolor{currentstroke}%
\pgfsetdash{}{0pt}%
\pgfpathmoveto{\pgfqpoint{1.284985in}{1.612974in}}%
\pgfpathlineto{\pgfqpoint{1.287859in}{1.609613in}}%
\pgfpathlineto{\pgfqpoint{1.290731in}{1.606146in}}%
\pgfpathlineto{\pgfqpoint{1.293600in}{1.602574in}}%
\pgfpathlineto{\pgfqpoint{1.296465in}{1.598897in}}%
\pgfpathlineto{\pgfqpoint{1.299643in}{1.597145in}}%
\pgfpathlineto{\pgfqpoint{1.302704in}{1.595347in}}%
\pgfpathlineto{\pgfqpoint{1.305645in}{1.593503in}}%
\pgfpathlineto{\pgfqpoint{1.308464in}{1.591616in}}%
\pgfpathlineto{\pgfqpoint{1.305299in}{1.595476in}}%
\pgfpathlineto{\pgfqpoint{1.302132in}{1.599231in}}%
\pgfpathlineto{\pgfqpoint{1.298961in}{1.602882in}}%
\pgfpathlineto{\pgfqpoint{1.295788in}{1.606425in}}%
\pgfpathlineto{\pgfqpoint{1.293251in}{1.608122in}}%
\pgfpathlineto{\pgfqpoint{1.290603in}{1.609780in}}%
\pgfpathlineto{\pgfqpoint{1.287847in}{1.611398in}}%
\pgfpathlineto{\pgfqpoint{1.284985in}{1.612974in}}%
\pgfpathclose%
\pgfusepath{fill}%
\end{pgfscope}%
\begin{pgfscope}%
\pgfpathrectangle{\pgfqpoint{0.041670in}{0.041670in}}{\pgfqpoint{2.216660in}{2.216660in}}%
\pgfusepath{clip}%
\pgfsetbuttcap%
\pgfsetroundjoin%
\definecolor{currentfill}{rgb}{0.487026,0.823929,0.312321}%
\pgfsetfillcolor{currentfill}%
\pgfsetlinewidth{0.000000pt}%
\definecolor{currentstroke}{rgb}{0.000000,0.000000,0.000000}%
\pgfsetstrokecolor{currentstroke}%
\pgfsetdash{}{0pt}%
\pgfpathmoveto{\pgfqpoint{0.997125in}{1.524274in}}%
\pgfpathlineto{\pgfqpoint{0.993377in}{1.518849in}}%
\pgfpathlineto{\pgfqpoint{0.989632in}{1.513335in}}%
\pgfpathlineto{\pgfqpoint{0.985890in}{1.507734in}}%
\pgfpathlineto{\pgfqpoint{0.982151in}{1.502048in}}%
\pgfpathlineto{\pgfqpoint{0.983496in}{1.505015in}}%
\pgfpathlineto{\pgfqpoint{0.985034in}{1.507959in}}%
\pgfpathlineto{\pgfqpoint{0.986765in}{1.510876in}}%
\pgfpathlineto{\pgfqpoint{0.988686in}{1.513766in}}%
\pgfpathlineto{\pgfqpoint{0.992295in}{1.519221in}}%
\pgfpathlineto{\pgfqpoint{0.995907in}{1.524591in}}%
\pgfpathlineto{\pgfqpoint{0.999523in}{1.529874in}}%
\pgfpathlineto{\pgfqpoint{1.003142in}{1.535069in}}%
\pgfpathlineto{\pgfqpoint{1.001372in}{1.532407in}}%
\pgfpathlineto{\pgfqpoint{0.999778in}{1.529719in}}%
\pgfpathlineto{\pgfqpoint{0.998362in}{1.527007in}}%
\pgfpathlineto{\pgfqpoint{0.997125in}{1.524274in}}%
\pgfpathclose%
\pgfusepath{fill}%
\end{pgfscope}%
\begin{pgfscope}%
\pgfpathrectangle{\pgfqpoint{0.041670in}{0.041670in}}{\pgfqpoint{2.216660in}{2.216660in}}%
\pgfusepath{clip}%
\pgfsetbuttcap%
\pgfsetroundjoin%
\definecolor{currentfill}{rgb}{0.412913,0.803041,0.357269}%
\pgfsetfillcolor{currentfill}%
\pgfsetlinewidth{0.000000pt}%
\definecolor{currentstroke}{rgb}{0.000000,0.000000,0.000000}%
\pgfsetstrokecolor{currentstroke}%
\pgfsetdash{}{0pt}%
\pgfpathmoveto{\pgfqpoint{1.376573in}{1.504686in}}%
\pgfpathlineto{\pgfqpoint{1.380284in}{1.498969in}}%
\pgfpathlineto{\pgfqpoint{1.383992in}{1.493170in}}%
\pgfpathlineto{\pgfqpoint{1.387697in}{1.487292in}}%
\pgfpathlineto{\pgfqpoint{1.391398in}{1.481336in}}%
\pgfpathlineto{\pgfqpoint{1.392828in}{1.478132in}}%
\pgfpathlineto{\pgfqpoint{1.394047in}{1.474906in}}%
\pgfpathlineto{\pgfqpoint{1.395055in}{1.471662in}}%
\pgfpathlineto{\pgfqpoint{1.395850in}{1.468402in}}%
\pgfpathlineto{\pgfqpoint{1.392063in}{1.474596in}}%
\pgfpathlineto{\pgfqpoint{1.388274in}{1.480712in}}%
\pgfpathlineto{\pgfqpoint{1.384481in}{1.486748in}}%
\pgfpathlineto{\pgfqpoint{1.380685in}{1.492703in}}%
\pgfpathlineto{\pgfqpoint{1.379953in}{1.495722in}}%
\pgfpathlineto{\pgfqpoint{1.379023in}{1.498728in}}%
\pgfpathlineto{\pgfqpoint{1.377896in}{1.501717in}}%
\pgfpathlineto{\pgfqpoint{1.376573in}{1.504686in}}%
\pgfpathclose%
\pgfusepath{fill}%
\end{pgfscope}%
\begin{pgfscope}%
\pgfpathrectangle{\pgfqpoint{0.041670in}{0.041670in}}{\pgfqpoint{2.216660in}{2.216660in}}%
\pgfusepath{clip}%
\pgfsetbuttcap%
\pgfsetroundjoin%
\definecolor{currentfill}{rgb}{0.814576,0.883393,0.110347}%
\pgfsetfillcolor{currentfill}%
\pgfsetlinewidth{0.000000pt}%
\definecolor{currentstroke}{rgb}{0.000000,0.000000,0.000000}%
\pgfsetstrokecolor{currentstroke}%
\pgfsetdash{}{0pt}%
\pgfpathmoveto{\pgfqpoint{1.207227in}{1.651600in}}%
\pgfpathlineto{\pgfqpoint{1.208189in}{1.649702in}}%
\pgfpathlineto{\pgfqpoint{1.209150in}{1.647688in}}%
\pgfpathlineto{\pgfqpoint{1.210109in}{1.645557in}}%
\pgfpathlineto{\pgfqpoint{1.211067in}{1.643313in}}%
\pgfpathlineto{\pgfqpoint{1.214833in}{1.642827in}}%
\pgfpathlineto{\pgfqpoint{1.218564in}{1.642285in}}%
\pgfpathlineto{\pgfqpoint{1.222257in}{1.641689in}}%
\pgfpathlineto{\pgfqpoint{1.220954in}{1.643985in}}%
\pgfpathlineto{\pgfqpoint{1.219649in}{1.646165in}}%
\pgfpathlineto{\pgfqpoint{1.218343in}{1.648231in}}%
\pgfpathlineto{\pgfqpoint{1.217035in}{1.650179in}}%
\pgfpathlineto{\pgfqpoint{1.213798in}{1.650701in}}%
\pgfpathlineto{\pgfqpoint{1.210528in}{1.651175in}}%
\pgfpathlineto{\pgfqpoint{1.207227in}{1.651600in}}%
\pgfpathclose%
\pgfusepath{fill}%
\end{pgfscope}%
\begin{pgfscope}%
\pgfpathrectangle{\pgfqpoint{0.041670in}{0.041670in}}{\pgfqpoint{2.216660in}{2.216660in}}%
\pgfusepath{clip}%
\pgfsetbuttcap%
\pgfsetroundjoin%
\definecolor{currentfill}{rgb}{0.814576,0.883393,0.110347}%
\pgfsetfillcolor{currentfill}%
\pgfsetlinewidth{0.000000pt}%
\definecolor{currentstroke}{rgb}{0.000000,0.000000,0.000000}%
\pgfsetstrokecolor{currentstroke}%
\pgfsetdash{}{0pt}%
\pgfpathmoveto{\pgfqpoint{1.140027in}{1.649676in}}%
\pgfpathlineto{\pgfqpoint{1.138619in}{1.647709in}}%
\pgfpathlineto{\pgfqpoint{1.137212in}{1.645626in}}%
\pgfpathlineto{\pgfqpoint{1.135807in}{1.643427in}}%
\pgfpathlineto{\pgfqpoint{1.134403in}{1.641114in}}%
\pgfpathlineto{\pgfqpoint{1.138061in}{1.641758in}}%
\pgfpathlineto{\pgfqpoint{1.141759in}{1.642348in}}%
\pgfpathlineto{\pgfqpoint{1.145494in}{1.642883in}}%
\pgfpathlineto{\pgfqpoint{1.149263in}{1.643363in}}%
\pgfpathlineto{\pgfqpoint{1.150208in}{1.645606in}}%
\pgfpathlineto{\pgfqpoint{1.151155in}{1.647735in}}%
\pgfpathlineto{\pgfqpoint{1.152102in}{1.649748in}}%
\pgfpathlineto{\pgfqpoint{1.153051in}{1.651645in}}%
\pgfpathlineto{\pgfqpoint{1.149747in}{1.651225in}}%
\pgfpathlineto{\pgfqpoint{1.146474in}{1.650756in}}%
\pgfpathlineto{\pgfqpoint{1.143232in}{1.650240in}}%
\pgfpathlineto{\pgfqpoint{1.140027in}{1.649676in}}%
\pgfpathclose%
\pgfusepath{fill}%
\end{pgfscope}%
\begin{pgfscope}%
\pgfpathrectangle{\pgfqpoint{0.041670in}{0.041670in}}{\pgfqpoint{2.216660in}{2.216660in}}%
\pgfusepath{clip}%
\pgfsetbuttcap%
\pgfsetroundjoin%
\definecolor{currentfill}{rgb}{0.231674,0.318106,0.544834}%
\pgfsetfillcolor{currentfill}%
\pgfsetlinewidth{0.000000pt}%
\definecolor{currentstroke}{rgb}{0.000000,0.000000,0.000000}%
\pgfsetstrokecolor{currentstroke}%
\pgfsetdash{}{0pt}%
\pgfpathmoveto{\pgfqpoint{1.536488in}{0.968327in}}%
\pgfpathlineto{\pgfqpoint{1.539717in}{0.960181in}}%
\pgfpathlineto{\pgfqpoint{1.542946in}{0.952098in}}%
\pgfpathlineto{\pgfqpoint{1.546174in}{0.944082in}}%
\pgfpathlineto{\pgfqpoint{1.549403in}{0.936137in}}%
\pgfpathlineto{\pgfqpoint{1.542428in}{0.930153in}}%
\pgfpathlineto{\pgfqpoint{1.535074in}{0.924282in}}%
\pgfpathlineto{\pgfqpoint{1.527348in}{0.918529in}}%
\pgfpathlineto{\pgfqpoint{1.519256in}{0.912901in}}%
\pgfpathlineto{\pgfqpoint{1.516282in}{0.921075in}}%
\pgfpathlineto{\pgfqpoint{1.513307in}{0.929319in}}%
\pgfpathlineto{\pgfqpoint{1.510333in}{0.937631in}}%
\pgfpathlineto{\pgfqpoint{1.507358in}{0.946006in}}%
\pgfpathlineto{\pgfqpoint{1.515176in}{0.951412in}}%
\pgfpathlineto{\pgfqpoint{1.522641in}{0.956938in}}%
\pgfpathlineto{\pgfqpoint{1.529747in}{0.962579in}}%
\pgfpathlineto{\pgfqpoint{1.536488in}{0.968327in}}%
\pgfpathclose%
\pgfusepath{fill}%
\end{pgfscope}%
\begin{pgfscope}%
\pgfpathrectangle{\pgfqpoint{0.041670in}{0.041670in}}{\pgfqpoint{2.216660in}{2.216660in}}%
\pgfusepath{clip}%
\pgfsetbuttcap%
\pgfsetroundjoin%
\definecolor{currentfill}{rgb}{0.248629,0.278775,0.534556}%
\pgfsetfillcolor{currentfill}%
\pgfsetlinewidth{0.000000pt}%
\definecolor{currentstroke}{rgb}{0.000000,0.000000,0.000000}%
\pgfsetstrokecolor{currentstroke}%
\pgfsetdash{}{0pt}%
\pgfpathmoveto{\pgfqpoint{0.848147in}{0.908008in}}%
\pgfpathlineto{\pgfqpoint{0.845236in}{0.899859in}}%
\pgfpathlineto{\pgfqpoint{0.842324in}{0.891788in}}%
\pgfpathlineto{\pgfqpoint{0.839412in}{0.883797in}}%
\pgfpathlineto{\pgfqpoint{0.836500in}{0.875890in}}%
\pgfpathlineto{\pgfqpoint{0.827804in}{0.881620in}}%
\pgfpathlineto{\pgfqpoint{0.819480in}{0.887486in}}%
\pgfpathlineto{\pgfqpoint{0.811536in}{0.893481in}}%
\pgfpathlineto{\pgfqpoint{0.803979in}{0.899598in}}%
\pgfpathlineto{\pgfqpoint{0.807156in}{0.907278in}}%
\pgfpathlineto{\pgfqpoint{0.810334in}{0.915043in}}%
\pgfpathlineto{\pgfqpoint{0.813511in}{0.922889in}}%
\pgfpathlineto{\pgfqpoint{0.816688in}{0.930813in}}%
\pgfpathlineto{\pgfqpoint{0.824000in}{0.924928in}}%
\pgfpathlineto{\pgfqpoint{0.831685in}{0.919162in}}%
\pgfpathlineto{\pgfqpoint{0.839737in}{0.913520in}}%
\pgfpathlineto{\pgfqpoint{0.848147in}{0.908008in}}%
\pgfpathclose%
\pgfusepath{fill}%
\end{pgfscope}%
\begin{pgfscope}%
\pgfpathrectangle{\pgfqpoint{0.041670in}{0.041670in}}{\pgfqpoint{2.216660in}{2.216660in}}%
\pgfusepath{clip}%
\pgfsetbuttcap%
\pgfsetroundjoin%
\definecolor{currentfill}{rgb}{0.699415,0.867117,0.175971}%
\pgfsetfillcolor{currentfill}%
\pgfsetlinewidth{0.000000pt}%
\definecolor{currentstroke}{rgb}{0.000000,0.000000,0.000000}%
\pgfsetstrokecolor{currentstroke}%
\pgfsetdash{}{0pt}%
\pgfpathmoveto{\pgfqpoint{1.061962in}{1.604885in}}%
\pgfpathlineto{\pgfqpoint{1.058729in}{1.601299in}}%
\pgfpathlineto{\pgfqpoint{1.055498in}{1.597606in}}%
\pgfpathlineto{\pgfqpoint{1.052271in}{1.593807in}}%
\pgfpathlineto{\pgfqpoint{1.049046in}{1.589904in}}%
\pgfpathlineto{\pgfqpoint{1.051753in}{1.591828in}}%
\pgfpathlineto{\pgfqpoint{1.054585in}{1.593710in}}%
\pgfpathlineto{\pgfqpoint{1.057540in}{1.595549in}}%
\pgfpathlineto{\pgfqpoint{1.060614in}{1.597342in}}%
\pgfpathlineto{\pgfqpoint{1.063550in}{1.601058in}}%
\pgfpathlineto{\pgfqpoint{1.066489in}{1.604669in}}%
\pgfpathlineto{\pgfqpoint{1.069431in}{1.608176in}}%
\pgfpathlineto{\pgfqpoint{1.072376in}{1.611575in}}%
\pgfpathlineto{\pgfqpoint{1.069608in}{1.609962in}}%
\pgfpathlineto{\pgfqpoint{1.066948in}{1.608308in}}%
\pgfpathlineto{\pgfqpoint{1.064398in}{1.606615in}}%
\pgfpathlineto{\pgfqpoint{1.061962in}{1.604885in}}%
\pgfpathclose%
\pgfusepath{fill}%
\end{pgfscope}%
\begin{pgfscope}%
\pgfpathrectangle{\pgfqpoint{0.041670in}{0.041670in}}{\pgfqpoint{2.216660in}{2.216660in}}%
\pgfusepath{clip}%
\pgfsetbuttcap%
\pgfsetroundjoin%
\definecolor{currentfill}{rgb}{0.179019,0.433756,0.557430}%
\pgfsetfillcolor{currentfill}%
\pgfsetlinewidth{0.000000pt}%
\definecolor{currentstroke}{rgb}{0.000000,0.000000,0.000000}%
\pgfsetstrokecolor{currentstroke}%
\pgfsetdash{}{0pt}%
\pgfpathmoveto{\pgfqpoint{0.867563in}{1.065387in}}%
\pgfpathlineto{\pgfqpoint{0.864378in}{1.056676in}}%
\pgfpathlineto{\pgfqpoint{0.861194in}{1.047990in}}%
\pgfpathlineto{\pgfqpoint{0.858010in}{1.039332in}}%
\pgfpathlineto{\pgfqpoint{0.854828in}{1.030706in}}%
\pgfpathlineto{\pgfqpoint{0.848595in}{1.035998in}}%
\pgfpathlineto{\pgfqpoint{0.842706in}{1.041384in}}%
\pgfpathlineto{\pgfqpoint{0.837166in}{1.046858in}}%
\pgfpathlineto{\pgfqpoint{0.831981in}{1.052413in}}%
\pgfpathlineto{\pgfqpoint{0.835377in}{1.060801in}}%
\pgfpathlineto{\pgfqpoint{0.838774in}{1.069221in}}%
\pgfpathlineto{\pgfqpoint{0.842172in}{1.077670in}}%
\pgfpathlineto{\pgfqpoint{0.845571in}{1.086145in}}%
\pgfpathlineto{\pgfqpoint{0.850564in}{1.080832in}}%
\pgfpathlineto{\pgfqpoint{0.855896in}{1.075597in}}%
\pgfpathlineto{\pgfqpoint{0.861565in}{1.070447in}}%
\pgfpathlineto{\pgfqpoint{0.867563in}{1.065387in}}%
\pgfpathclose%
\pgfusepath{fill}%
\end{pgfscope}%
\begin{pgfscope}%
\pgfpathrectangle{\pgfqpoint{0.041670in}{0.041670in}}{\pgfqpoint{2.216660in}{2.216660in}}%
\pgfusepath{clip}%
\pgfsetbuttcap%
\pgfsetroundjoin%
\definecolor{currentfill}{rgb}{0.133743,0.548535,0.553541}%
\pgfsetfillcolor{currentfill}%
\pgfsetlinewidth{0.000000pt}%
\definecolor{currentstroke}{rgb}{0.000000,0.000000,0.000000}%
\pgfsetstrokecolor{currentstroke}%
\pgfsetdash{}{0pt}%
\pgfpathmoveto{\pgfqpoint{0.886487in}{1.188767in}}%
\pgfpathlineto{\pgfqpoint{0.883067in}{1.180213in}}%
\pgfpathlineto{\pgfqpoint{0.879650in}{1.171651in}}%
\pgfpathlineto{\pgfqpoint{0.876234in}{1.163081in}}%
\pgfpathlineto{\pgfqpoint{0.872820in}{1.154509in}}%
\pgfpathlineto{\pgfqpoint{0.868532in}{1.159408in}}%
\pgfpathlineto{\pgfqpoint{0.864562in}{1.164369in}}%
\pgfpathlineto{\pgfqpoint{0.860916in}{1.169387in}}%
\pgfpathlineto{\pgfqpoint{0.857595in}{1.174456in}}%
\pgfpathlineto{\pgfqpoint{0.861167in}{1.182783in}}%
\pgfpathlineto{\pgfqpoint{0.864742in}{1.191107in}}%
\pgfpathlineto{\pgfqpoint{0.868320in}{1.199425in}}%
\pgfpathlineto{\pgfqpoint{0.871899in}{1.207735in}}%
\pgfpathlineto{\pgfqpoint{0.875082in}{1.202914in}}%
\pgfpathlineto{\pgfqpoint{0.878577in}{1.198143in}}%
\pgfpathlineto{\pgfqpoint{0.882379in}{1.193425in}}%
\pgfpathlineto{\pgfqpoint{0.886487in}{1.188767in}}%
\pgfpathclose%
\pgfusepath{fill}%
\end{pgfscope}%
\begin{pgfscope}%
\pgfpathrectangle{\pgfqpoint{0.041670in}{0.041670in}}{\pgfqpoint{2.216660in}{2.216660in}}%
\pgfusepath{clip}%
\pgfsetbuttcap%
\pgfsetroundjoin%
\definecolor{currentfill}{rgb}{0.814576,0.883393,0.110347}%
\pgfsetfillcolor{currentfill}%
\pgfsetlinewidth{0.000000pt}%
\definecolor{currentstroke}{rgb}{0.000000,0.000000,0.000000}%
\pgfsetstrokecolor{currentstroke}%
\pgfsetdash{}{0pt}%
\pgfpathmoveto{\pgfqpoint{1.217035in}{1.650179in}}%
\pgfpathlineto{\pgfqpoint{1.218343in}{1.648231in}}%
\pgfpathlineto{\pgfqpoint{1.219649in}{1.646165in}}%
\pgfpathlineto{\pgfqpoint{1.220954in}{1.643985in}}%
\pgfpathlineto{\pgfqpoint{1.222257in}{1.641689in}}%
\pgfpathlineto{\pgfqpoint{1.225910in}{1.641039in}}%
\pgfpathlineto{\pgfqpoint{1.229518in}{1.640335in}}%
\pgfpathlineto{\pgfqpoint{1.233078in}{1.639579in}}%
\pgfpathlineto{\pgfqpoint{1.236586in}{1.638770in}}%
\pgfpathlineto{\pgfqpoint{1.234840in}{1.641156in}}%
\pgfpathlineto{\pgfqpoint{1.233093in}{1.643428in}}%
\pgfpathlineto{\pgfqpoint{1.231343in}{1.645584in}}%
\pgfpathlineto{\pgfqpoint{1.229592in}{1.647623in}}%
\pgfpathlineto{\pgfqpoint{1.226518in}{1.648332in}}%
\pgfpathlineto{\pgfqpoint{1.223398in}{1.648994in}}%
\pgfpathlineto{\pgfqpoint{1.220236in}{1.649610in}}%
\pgfpathlineto{\pgfqpoint{1.217035in}{1.650179in}}%
\pgfpathclose%
\pgfusepath{fill}%
\end{pgfscope}%
\begin{pgfscope}%
\pgfpathrectangle{\pgfqpoint{0.041670in}{0.041670in}}{\pgfqpoint{2.216660in}{2.216660in}}%
\pgfusepath{clip}%
\pgfsetbuttcap%
\pgfsetroundjoin%
\definecolor{currentfill}{rgb}{0.220124,0.725509,0.466226}%
\pgfsetfillcolor{currentfill}%
\pgfsetlinewidth{0.000000pt}%
\definecolor{currentstroke}{rgb}{0.000000,0.000000,0.000000}%
\pgfsetstrokecolor{currentstroke}%
\pgfsetdash{}{0pt}%
\pgfpathmoveto{\pgfqpoint{1.427259in}{1.401201in}}%
\pgfpathlineto{\pgfqpoint{1.431043in}{1.394151in}}%
\pgfpathlineto{\pgfqpoint{1.434823in}{1.387043in}}%
\pgfpathlineto{\pgfqpoint{1.438600in}{1.379879in}}%
\pgfpathlineto{\pgfqpoint{1.442374in}{1.372663in}}%
\pgfpathlineto{\pgfqpoint{1.442059in}{1.368637in}}%
\pgfpathlineto{\pgfqpoint{1.441481in}{1.364615in}}%
\pgfpathlineto{\pgfqpoint{1.440641in}{1.360601in}}%
\pgfpathlineto{\pgfqpoint{1.439540in}{1.356600in}}%
\pgfpathlineto{\pgfqpoint{1.435798in}{1.364065in}}%
\pgfpathlineto{\pgfqpoint{1.432053in}{1.371476in}}%
\pgfpathlineto{\pgfqpoint{1.428305in}{1.378832in}}%
\pgfpathlineto{\pgfqpoint{1.424554in}{1.386130in}}%
\pgfpathlineto{\pgfqpoint{1.425601in}{1.389883in}}%
\pgfpathlineto{\pgfqpoint{1.426401in}{1.393649in}}%
\pgfpathlineto{\pgfqpoint{1.426954in}{1.397423in}}%
\pgfpathlineto{\pgfqpoint{1.427259in}{1.401201in}}%
\pgfpathclose%
\pgfusepath{fill}%
\end{pgfscope}%
\begin{pgfscope}%
\pgfpathrectangle{\pgfqpoint{0.041670in}{0.041670in}}{\pgfqpoint{2.216660in}{2.216660in}}%
\pgfusepath{clip}%
\pgfsetbuttcap%
\pgfsetroundjoin%
\definecolor{currentfill}{rgb}{0.268510,0.009605,0.335427}%
\pgfsetfillcolor{currentfill}%
\pgfsetlinewidth{0.000000pt}%
\definecolor{currentstroke}{rgb}{0.000000,0.000000,0.000000}%
\pgfsetstrokecolor{currentstroke}%
\pgfsetdash{}{0pt}%
\pgfpathmoveto{\pgfqpoint{0.705907in}{0.669606in}}%
\pgfpathlineto{\pgfqpoint{0.702827in}{0.669598in}}%
\pgfpathlineto{\pgfqpoint{0.699738in}{0.669871in}}%
\pgfpathlineto{\pgfqpoint{0.696641in}{0.670429in}}%
\pgfpathlineto{\pgfqpoint{0.693536in}{0.671279in}}%
\pgfpathlineto{\pgfqpoint{0.681385in}{0.679621in}}%
\pgfpathlineto{\pgfqpoint{0.669772in}{0.688156in}}%
\pgfpathlineto{\pgfqpoint{0.658709in}{0.696872in}}%
\pgfpathlineto{\pgfqpoint{0.648204in}{0.705759in}}%
\pgfpathlineto{\pgfqpoint{0.651583in}{0.704695in}}%
\pgfpathlineto{\pgfqpoint{0.654954in}{0.703921in}}%
\pgfpathlineto{\pgfqpoint{0.658315in}{0.703431in}}%
\pgfpathlineto{\pgfqpoint{0.661668in}{0.703221in}}%
\pgfpathlineto{\pgfqpoint{0.671922in}{0.694556in}}%
\pgfpathlineto{\pgfqpoint{0.682720in}{0.686059in}}%
\pgfpathlineto{\pgfqpoint{0.694052in}{0.677739in}}%
\pgfpathlineto{\pgfqpoint{0.705907in}{0.669606in}}%
\pgfpathclose%
\pgfusepath{fill}%
\end{pgfscope}%
\begin{pgfscope}%
\pgfpathrectangle{\pgfqpoint{0.041670in}{0.041670in}}{\pgfqpoint{2.216660in}{2.216660in}}%
\pgfusepath{clip}%
\pgfsetbuttcap%
\pgfsetroundjoin%
\definecolor{currentfill}{rgb}{0.134692,0.658636,0.517649}%
\pgfsetfillcolor{currentfill}%
\pgfsetlinewidth{0.000000pt}%
\definecolor{currentstroke}{rgb}{0.000000,0.000000,0.000000}%
\pgfsetstrokecolor{currentstroke}%
\pgfsetdash{}{0pt}%
\pgfpathmoveto{\pgfqpoint{1.454479in}{1.326252in}}%
\pgfpathlineto{\pgfqpoint{1.458206in}{1.318555in}}%
\pgfpathlineto{\pgfqpoint{1.461931in}{1.310818in}}%
\pgfpathlineto{\pgfqpoint{1.465652in}{1.303045in}}%
\pgfpathlineto{\pgfqpoint{1.469371in}{1.295237in}}%
\pgfpathlineto{\pgfqpoint{1.467872in}{1.290754in}}%
\pgfpathlineto{\pgfqpoint{1.466082in}{1.286294in}}%
\pgfpathlineto{\pgfqpoint{1.464003in}{1.281860in}}%
\pgfpathlineto{\pgfqpoint{1.461636in}{1.277459in}}%
\pgfpathlineto{\pgfqpoint{1.458007in}{1.285516in}}%
\pgfpathlineto{\pgfqpoint{1.454376in}{1.293538in}}%
\pgfpathlineto{\pgfqpoint{1.450742in}{1.301523in}}%
\pgfpathlineto{\pgfqpoint{1.447105in}{1.309468in}}%
\pgfpathlineto{\pgfqpoint{1.449359in}{1.313623in}}%
\pgfpathlineto{\pgfqpoint{1.451340in}{1.317808in}}%
\pgfpathlineto{\pgfqpoint{1.453047in}{1.322019in}}%
\pgfpathlineto{\pgfqpoint{1.454479in}{1.326252in}}%
\pgfpathclose%
\pgfusepath{fill}%
\end{pgfscope}%
\begin{pgfscope}%
\pgfpathrectangle{\pgfqpoint{0.041670in}{0.041670in}}{\pgfqpoint{2.216660in}{2.216660in}}%
\pgfusepath{clip}%
\pgfsetbuttcap%
\pgfsetroundjoin%
\definecolor{currentfill}{rgb}{0.412913,0.803041,0.357269}%
\pgfsetfillcolor{currentfill}%
\pgfsetlinewidth{0.000000pt}%
\definecolor{currentstroke}{rgb}{0.000000,0.000000,0.000000}%
\pgfsetstrokecolor{currentstroke}%
\pgfsetdash{}{0pt}%
\pgfpathmoveto{\pgfqpoint{0.978741in}{1.490009in}}%
\pgfpathlineto{\pgfqpoint{0.974934in}{1.484001in}}%
\pgfpathlineto{\pgfqpoint{0.971131in}{1.477912in}}%
\pgfpathlineto{\pgfqpoint{0.967331in}{1.471742in}}%
\pgfpathlineto{\pgfqpoint{0.963534in}{1.465494in}}%
\pgfpathlineto{\pgfqpoint{0.964138in}{1.468765in}}%
\pgfpathlineto{\pgfqpoint{0.964956in}{1.472023in}}%
\pgfpathlineto{\pgfqpoint{0.965988in}{1.475265in}}%
\pgfpathlineto{\pgfqpoint{0.967231in}{1.478489in}}%
\pgfpathlineto{\pgfqpoint{0.970956in}{1.484497in}}%
\pgfpathlineto{\pgfqpoint{0.974684in}{1.490428in}}%
\pgfpathlineto{\pgfqpoint{0.978416in}{1.496279in}}%
\pgfpathlineto{\pgfqpoint{0.982151in}{1.502048in}}%
\pgfpathlineto{\pgfqpoint{0.981002in}{1.499061in}}%
\pgfpathlineto{\pgfqpoint{0.980050in}{1.496057in}}%
\pgfpathlineto{\pgfqpoint{0.979296in}{1.493039in}}%
\pgfpathlineto{\pgfqpoint{0.978741in}{1.490009in}}%
\pgfpathclose%
\pgfusepath{fill}%
\end{pgfscope}%
\begin{pgfscope}%
\pgfpathrectangle{\pgfqpoint{0.041670in}{0.041670in}}{\pgfqpoint{2.216660in}{2.216660in}}%
\pgfusepath{clip}%
\pgfsetbuttcap%
\pgfsetroundjoin%
\definecolor{currentfill}{rgb}{0.272594,0.025563,0.353093}%
\pgfsetfillcolor{currentfill}%
\pgfsetlinewidth{0.000000pt}%
\definecolor{currentstroke}{rgb}{0.000000,0.000000,0.000000}%
\pgfsetstrokecolor{currentstroke}%
\pgfsetdash{}{0pt}%
\pgfpathmoveto{\pgfqpoint{1.720567in}{0.713795in}}%
\pgfpathlineto{\pgfqpoint{1.724009in}{0.715205in}}%
\pgfpathlineto{\pgfqpoint{1.727460in}{0.716916in}}%
\pgfpathlineto{\pgfqpoint{1.730922in}{0.718933in}}%
\pgfpathlineto{\pgfqpoint{1.734395in}{0.721262in}}%
\pgfpathlineto{\pgfqpoint{1.724155in}{0.712008in}}%
\pgfpathlineto{\pgfqpoint{1.713334in}{0.702920in}}%
\pgfpathlineto{\pgfqpoint{1.701939in}{0.694009in}}%
\pgfpathlineto{\pgfqpoint{1.689982in}{0.685285in}}%
\pgfpathlineto{\pgfqpoint{1.686772in}{0.683171in}}%
\pgfpathlineto{\pgfqpoint{1.683572in}{0.681370in}}%
\pgfpathlineto{\pgfqpoint{1.680382in}{0.679877in}}%
\pgfpathlineto{\pgfqpoint{1.677201in}{0.678685in}}%
\pgfpathlineto{\pgfqpoint{1.688874in}{0.687198in}}%
\pgfpathlineto{\pgfqpoint{1.699999in}{0.695895in}}%
\pgfpathlineto{\pgfqpoint{1.710566in}{0.704764in}}%
\pgfpathlineto{\pgfqpoint{1.720567in}{0.713795in}}%
\pgfpathclose%
\pgfusepath{fill}%
\end{pgfscope}%
\begin{pgfscope}%
\pgfpathrectangle{\pgfqpoint{0.041670in}{0.041670in}}{\pgfqpoint{2.216660in}{2.216660in}}%
\pgfusepath{clip}%
\pgfsetbuttcap%
\pgfsetroundjoin%
\definecolor{currentfill}{rgb}{0.814576,0.883393,0.110347}%
\pgfsetfillcolor{currentfill}%
\pgfsetlinewidth{0.000000pt}%
\definecolor{currentstroke}{rgb}{0.000000,0.000000,0.000000}%
\pgfsetstrokecolor{currentstroke}%
\pgfsetdash{}{0pt}%
\pgfpathmoveto{\pgfqpoint{1.127626in}{1.646956in}}%
\pgfpathlineto{\pgfqpoint{1.125779in}{1.644893in}}%
\pgfpathlineto{\pgfqpoint{1.123935in}{1.642713in}}%
\pgfpathlineto{\pgfqpoint{1.122092in}{1.640417in}}%
\pgfpathlineto{\pgfqpoint{1.120252in}{1.638007in}}%
\pgfpathlineto{\pgfqpoint{1.123711in}{1.638862in}}%
\pgfpathlineto{\pgfqpoint{1.127225in}{1.639665in}}%
\pgfpathlineto{\pgfqpoint{1.130790in}{1.640416in}}%
\pgfpathlineto{\pgfqpoint{1.134403in}{1.641114in}}%
\pgfpathlineto{\pgfqpoint{1.135807in}{1.643427in}}%
\pgfpathlineto{\pgfqpoint{1.137212in}{1.645626in}}%
\pgfpathlineto{\pgfqpoint{1.138619in}{1.647709in}}%
\pgfpathlineto{\pgfqpoint{1.140027in}{1.649676in}}%
\pgfpathlineto{\pgfqpoint{1.136861in}{1.649065in}}%
\pgfpathlineto{\pgfqpoint{1.133736in}{1.648407in}}%
\pgfpathlineto{\pgfqpoint{1.130657in}{1.647704in}}%
\pgfpathlineto{\pgfqpoint{1.127626in}{1.646956in}}%
\pgfpathclose%
\pgfusepath{fill}%
\end{pgfscope}%
\begin{pgfscope}%
\pgfpathrectangle{\pgfqpoint{0.041670in}{0.041670in}}{\pgfqpoint{2.216660in}{2.216660in}}%
\pgfusepath{clip}%
\pgfsetbuttcap%
\pgfsetroundjoin%
\definecolor{currentfill}{rgb}{0.163625,0.471133,0.558148}%
\pgfsetfillcolor{currentfill}%
\pgfsetlinewidth{0.000000pt}%
\definecolor{currentstroke}{rgb}{0.000000,0.000000,0.000000}%
\pgfsetstrokecolor{currentstroke}%
\pgfsetdash{}{0pt}%
\pgfpathmoveto{\pgfqpoint{1.504716in}{1.124807in}}%
\pgfpathlineto{\pgfqpoint{1.508161in}{1.116314in}}%
\pgfpathlineto{\pgfqpoint{1.511605in}{1.107834in}}%
\pgfpathlineto{\pgfqpoint{1.515047in}{1.099372in}}%
\pgfpathlineto{\pgfqpoint{1.518488in}{1.090929in}}%
\pgfpathlineto{\pgfqpoint{1.513801in}{1.085551in}}%
\pgfpathlineto{\pgfqpoint{1.508770in}{1.080246in}}%
\pgfpathlineto{\pgfqpoint{1.503400in}{1.075021in}}%
\pgfpathlineto{\pgfqpoint{1.497695in}{1.069881in}}%
\pgfpathlineto{\pgfqpoint{1.494456in}{1.078563in}}%
\pgfpathlineto{\pgfqpoint{1.491216in}{1.087265in}}%
\pgfpathlineto{\pgfqpoint{1.487975in}{1.095983in}}%
\pgfpathlineto{\pgfqpoint{1.484732in}{1.104715in}}%
\pgfpathlineto{\pgfqpoint{1.490213in}{1.109622in}}%
\pgfpathlineto{\pgfqpoint{1.495374in}{1.114609in}}%
\pgfpathlineto{\pgfqpoint{1.500210in}{1.119673in}}%
\pgfpathlineto{\pgfqpoint{1.504716in}{1.124807in}}%
\pgfpathclose%
\pgfusepath{fill}%
\end{pgfscope}%
\begin{pgfscope}%
\pgfpathrectangle{\pgfqpoint{0.041670in}{0.041670in}}{\pgfqpoint{2.216660in}{2.216660in}}%
\pgfusepath{clip}%
\pgfsetbuttcap%
\pgfsetroundjoin%
\definecolor{currentfill}{rgb}{0.762373,0.876424,0.137064}%
\pgfsetfillcolor{currentfill}%
\pgfsetlinewidth{0.000000pt}%
\definecolor{currentstroke}{rgb}{0.000000,0.000000,0.000000}%
\pgfsetstrokecolor{currentstroke}%
\pgfsetdash{}{0pt}%
\pgfpathmoveto{\pgfqpoint{1.262390in}{1.630522in}}%
\pgfpathlineto{\pgfqpoint{1.264932in}{1.627765in}}%
\pgfpathlineto{\pgfqpoint{1.267471in}{1.624896in}}%
\pgfpathlineto{\pgfqpoint{1.270008in}{1.621917in}}%
\pgfpathlineto{\pgfqpoint{1.272542in}{1.618828in}}%
\pgfpathlineto{\pgfqpoint{1.275796in}{1.617434in}}%
\pgfpathlineto{\pgfqpoint{1.278956in}{1.615993in}}%
\pgfpathlineto{\pgfqpoint{1.282020in}{1.614506in}}%
\pgfpathlineto{\pgfqpoint{1.284985in}{1.612974in}}%
\pgfpathlineto{\pgfqpoint{1.282107in}{1.616226in}}%
\pgfpathlineto{\pgfqpoint{1.279227in}{1.619368in}}%
\pgfpathlineto{\pgfqpoint{1.276345in}{1.622400in}}%
\pgfpathlineto{\pgfqpoint{1.273459in}{1.625320in}}%
\pgfpathlineto{\pgfqpoint{1.270822in}{1.626682in}}%
\pgfpathlineto{\pgfqpoint{1.268097in}{1.628003in}}%
\pgfpathlineto{\pgfqpoint{1.265285in}{1.629284in}}%
\pgfpathlineto{\pgfqpoint{1.262390in}{1.630522in}}%
\pgfpathclose%
\pgfusepath{fill}%
\end{pgfscope}%
\begin{pgfscope}%
\pgfpathrectangle{\pgfqpoint{0.041670in}{0.041670in}}{\pgfqpoint{2.216660in}{2.216660in}}%
\pgfusepath{clip}%
\pgfsetbuttcap%
\pgfsetroundjoin%
\definecolor{currentfill}{rgb}{0.344074,0.780029,0.397381}%
\pgfsetfillcolor{currentfill}%
\pgfsetlinewidth{0.000000pt}%
\definecolor{currentstroke}{rgb}{0.000000,0.000000,0.000000}%
\pgfsetstrokecolor{currentstroke}%
\pgfsetdash{}{0pt}%
\pgfpathmoveto{\pgfqpoint{1.395850in}{1.468402in}}%
\pgfpathlineto{\pgfqpoint{1.399633in}{1.462132in}}%
\pgfpathlineto{\pgfqpoint{1.403412in}{1.455788in}}%
\pgfpathlineto{\pgfqpoint{1.407188in}{1.449372in}}%
\pgfpathlineto{\pgfqpoint{1.410961in}{1.442886in}}%
\pgfpathlineto{\pgfqpoint{1.411590in}{1.439371in}}%
\pgfpathlineto{\pgfqpoint{1.411989in}{1.435847in}}%
\pgfpathlineto{\pgfqpoint{1.412157in}{1.432317in}}%
\pgfpathlineto{\pgfqpoint{1.412094in}{1.428785in}}%
\pgfpathlineto{\pgfqpoint{1.408294in}{1.435515in}}%
\pgfpathlineto{\pgfqpoint{1.404492in}{1.442175in}}%
\pgfpathlineto{\pgfqpoint{1.400686in}{1.448762in}}%
\pgfpathlineto{\pgfqpoint{1.396878in}{1.455276in}}%
\pgfpathlineto{\pgfqpoint{1.396945in}{1.458564in}}%
\pgfpathlineto{\pgfqpoint{1.396795in}{1.461850in}}%
\pgfpathlineto{\pgfqpoint{1.396430in}{1.465130in}}%
\pgfpathlineto{\pgfqpoint{1.395850in}{1.468402in}}%
\pgfpathclose%
\pgfusepath{fill}%
\end{pgfscope}%
\begin{pgfscope}%
\pgfpathrectangle{\pgfqpoint{0.041670in}{0.041670in}}{\pgfqpoint{2.216660in}{2.216660in}}%
\pgfusepath{clip}%
\pgfsetbuttcap%
\pgfsetroundjoin%
\definecolor{currentfill}{rgb}{0.279566,0.067836,0.391917}%
\pgfsetfillcolor{currentfill}%
\pgfsetlinewidth{0.000000pt}%
\definecolor{currentstroke}{rgb}{0.000000,0.000000,0.000000}%
\pgfsetstrokecolor{currentstroke}%
\pgfsetdash{}{0pt}%
\pgfpathmoveto{\pgfqpoint{1.590954in}{0.747121in}}%
\pgfpathlineto{\pgfqpoint{1.593967in}{0.741887in}}%
\pgfpathlineto{\pgfqpoint{1.596984in}{0.736822in}}%
\pgfpathlineto{\pgfqpoint{1.600004in}{0.731929in}}%
\pgfpathlineto{\pgfqpoint{1.603028in}{0.727213in}}%
\pgfpathlineto{\pgfqpoint{1.592571in}{0.720181in}}%
\pgfpathlineto{\pgfqpoint{1.581671in}{0.713322in}}%
\pgfpathlineto{\pgfqpoint{1.570341in}{0.706644in}}%
\pgfpathlineto{\pgfqpoint{1.558590in}{0.700155in}}%
\pgfpathlineto{\pgfqpoint{1.555874in}{0.705082in}}%
\pgfpathlineto{\pgfqpoint{1.553161in}{0.710187in}}%
\pgfpathlineto{\pgfqpoint{1.550451in}{0.715464in}}%
\pgfpathlineto{\pgfqpoint{1.547745in}{0.720909in}}%
\pgfpathlineto{\pgfqpoint{1.559169in}{0.727195in}}%
\pgfpathlineto{\pgfqpoint{1.570185in}{0.733663in}}%
\pgfpathlineto{\pgfqpoint{1.580784in}{0.740308in}}%
\pgfpathlineto{\pgfqpoint{1.590954in}{0.747121in}}%
\pgfpathclose%
\pgfusepath{fill}%
\end{pgfscope}%
\begin{pgfscope}%
\pgfpathrectangle{\pgfqpoint{0.041670in}{0.041670in}}{\pgfqpoint{2.216660in}{2.216660in}}%
\pgfusepath{clip}%
\pgfsetbuttcap%
\pgfsetroundjoin%
\definecolor{currentfill}{rgb}{0.122606,0.585371,0.546557}%
\pgfsetfillcolor{currentfill}%
\pgfsetlinewidth{0.000000pt}%
\definecolor{currentstroke}{rgb}{0.000000,0.000000,0.000000}%
\pgfsetstrokecolor{currentstroke}%
\pgfsetdash{}{0pt}%
\pgfpathmoveto{\pgfqpoint{1.476126in}{1.244934in}}%
\pgfpathlineto{\pgfqpoint{1.479742in}{1.236741in}}%
\pgfpathlineto{\pgfqpoint{1.483356in}{1.228529in}}%
\pgfpathlineto{\pgfqpoint{1.486968in}{1.220300in}}%
\pgfpathlineto{\pgfqpoint{1.490577in}{1.212057in}}%
\pgfpathlineto{\pgfqpoint{1.487672in}{1.207197in}}%
\pgfpathlineto{\pgfqpoint{1.484454in}{1.202382in}}%
\pgfpathlineto{\pgfqpoint{1.480926in}{1.197616in}}%
\pgfpathlineto{\pgfqpoint{1.477089in}{1.192905in}}%
\pgfpathlineto{\pgfqpoint{1.473627in}{1.201393in}}%
\pgfpathlineto{\pgfqpoint{1.470163in}{1.209867in}}%
\pgfpathlineto{\pgfqpoint{1.466697in}{1.218324in}}%
\pgfpathlineto{\pgfqpoint{1.463229in}{1.226762in}}%
\pgfpathlineto{\pgfqpoint{1.466896in}{1.231231in}}%
\pgfpathlineto{\pgfqpoint{1.470269in}{1.235753in}}%
\pgfpathlineto{\pgfqpoint{1.473347in}{1.240322in}}%
\pgfpathlineto{\pgfqpoint{1.476126in}{1.244934in}}%
\pgfpathclose%
\pgfusepath{fill}%
\end{pgfscope}%
\begin{pgfscope}%
\pgfpathrectangle{\pgfqpoint{0.041670in}{0.041670in}}{\pgfqpoint{2.216660in}{2.216660in}}%
\pgfusepath{clip}%
\pgfsetbuttcap%
\pgfsetroundjoin%
\definecolor{currentfill}{rgb}{0.282327,0.094955,0.417331}%
\pgfsetfillcolor{currentfill}%
\pgfsetlinewidth{0.000000pt}%
\definecolor{currentstroke}{rgb}{0.000000,0.000000,0.000000}%
\pgfsetstrokecolor{currentstroke}%
\pgfsetdash{}{0pt}%
\pgfpathmoveto{\pgfqpoint{1.578930in}{0.769655in}}%
\pgfpathlineto{\pgfqpoint{1.581931in}{0.763790in}}%
\pgfpathlineto{\pgfqpoint{1.584936in}{0.758076in}}%
\pgfpathlineto{\pgfqpoint{1.587943in}{0.752518in}}%
\pgfpathlineto{\pgfqpoint{1.590954in}{0.747121in}}%
\pgfpathlineto{\pgfqpoint{1.580784in}{0.740308in}}%
\pgfpathlineto{\pgfqpoint{1.570185in}{0.733663in}}%
\pgfpathlineto{\pgfqpoint{1.559169in}{0.727195in}}%
\pgfpathlineto{\pgfqpoint{1.547745in}{0.720909in}}%
\pgfpathlineto{\pgfqpoint{1.545041in}{0.726518in}}%
\pgfpathlineto{\pgfqpoint{1.542341in}{0.732288in}}%
\pgfpathlineto{\pgfqpoint{1.539643in}{0.738213in}}%
\pgfpathlineto{\pgfqpoint{1.536948in}{0.744291in}}%
\pgfpathlineto{\pgfqpoint{1.548046in}{0.750373in}}%
\pgfpathlineto{\pgfqpoint{1.558749in}{0.756632in}}%
\pgfpathlineto{\pgfqpoint{1.569047in}{0.763062in}}%
\pgfpathlineto{\pgfqpoint{1.578930in}{0.769655in}}%
\pgfpathclose%
\pgfusepath{fill}%
\end{pgfscope}%
\begin{pgfscope}%
\pgfpathrectangle{\pgfqpoint{0.041670in}{0.041670in}}{\pgfqpoint{2.216660in}{2.216660in}}%
\pgfusepath{clip}%
\pgfsetbuttcap%
\pgfsetroundjoin%
\definecolor{currentfill}{rgb}{0.762373,0.876424,0.137064}%
\pgfsetfillcolor{currentfill}%
\pgfsetlinewidth{0.000000pt}%
\definecolor{currentstroke}{rgb}{0.000000,0.000000,0.000000}%
\pgfsetstrokecolor{currentstroke}%
\pgfsetdash{}{0pt}%
\pgfpathmoveto{\pgfqpoint{1.084184in}{1.624077in}}%
\pgfpathlineto{\pgfqpoint{1.081228in}{1.621119in}}%
\pgfpathlineto{\pgfqpoint{1.078274in}{1.618048in}}%
\pgfpathlineto{\pgfqpoint{1.075324in}{1.614866in}}%
\pgfpathlineto{\pgfqpoint{1.072376in}{1.611575in}}%
\pgfpathlineto{\pgfqpoint{1.075249in}{1.613146in}}%
\pgfpathlineto{\pgfqpoint{1.078225in}{1.614673in}}%
\pgfpathlineto{\pgfqpoint{1.081300in}{1.616156in}}%
\pgfpathlineto{\pgfqpoint{1.084471in}{1.617591in}}%
\pgfpathlineto{\pgfqpoint{1.087085in}{1.620715in}}%
\pgfpathlineto{\pgfqpoint{1.089701in}{1.623729in}}%
\pgfpathlineto{\pgfqpoint{1.092321in}{1.626632in}}%
\pgfpathlineto{\pgfqpoint{1.094942in}{1.629424in}}%
\pgfpathlineto{\pgfqpoint{1.092121in}{1.628148in}}%
\pgfpathlineto{\pgfqpoint{1.089386in}{1.626831in}}%
\pgfpathlineto{\pgfqpoint{1.086739in}{1.625473in}}%
\pgfpathlineto{\pgfqpoint{1.084184in}{1.624077in}}%
\pgfpathclose%
\pgfusepath{fill}%
\end{pgfscope}%
\begin{pgfscope}%
\pgfpathrectangle{\pgfqpoint{0.041670in}{0.041670in}}{\pgfqpoint{2.216660in}{2.216660in}}%
\pgfusepath{clip}%
\pgfsetbuttcap%
\pgfsetroundjoin%
\definecolor{currentfill}{rgb}{0.274952,0.037752,0.364543}%
\pgfsetfillcolor{currentfill}%
\pgfsetlinewidth{0.000000pt}%
\definecolor{currentstroke}{rgb}{0.000000,0.000000,0.000000}%
\pgfsetstrokecolor{currentstroke}%
\pgfsetdash{}{0pt}%
\pgfpathmoveto{\pgfqpoint{1.603028in}{0.727213in}}%
\pgfpathlineto{\pgfqpoint{1.606056in}{0.722677in}}%
\pgfpathlineto{\pgfqpoint{1.609088in}{0.718327in}}%
\pgfpathlineto{\pgfqpoint{1.612124in}{0.714167in}}%
\pgfpathlineto{\pgfqpoint{1.615164in}{0.710200in}}%
\pgfpathlineto{\pgfqpoint{1.604418in}{0.702949in}}%
\pgfpathlineto{\pgfqpoint{1.593217in}{0.695877in}}%
\pgfpathlineto{\pgfqpoint{1.581571in}{0.688990in}}%
\pgfpathlineto{\pgfqpoint{1.569492in}{0.682299in}}%
\pgfpathlineto{\pgfqpoint{1.566761in}{0.686476in}}%
\pgfpathlineto{\pgfqpoint{1.564033in}{0.690847in}}%
\pgfpathlineto{\pgfqpoint{1.561310in}{0.695408in}}%
\pgfpathlineto{\pgfqpoint{1.558590in}{0.700155in}}%
\pgfpathlineto{\pgfqpoint{1.570341in}{0.706644in}}%
\pgfpathlineto{\pgfqpoint{1.581671in}{0.713322in}}%
\pgfpathlineto{\pgfqpoint{1.592571in}{0.720181in}}%
\pgfpathlineto{\pgfqpoint{1.603028in}{0.727213in}}%
\pgfpathclose%
\pgfusepath{fill}%
\end{pgfscope}%
\begin{pgfscope}%
\pgfpathrectangle{\pgfqpoint{0.041670in}{0.041670in}}{\pgfqpoint{2.216660in}{2.216660in}}%
\pgfusepath{clip}%
\pgfsetbuttcap%
\pgfsetroundjoin%
\definecolor{currentfill}{rgb}{0.231674,0.318106,0.544834}%
\pgfsetfillcolor{currentfill}%
\pgfsetlinewidth{0.000000pt}%
\definecolor{currentstroke}{rgb}{0.000000,0.000000,0.000000}%
\pgfsetstrokecolor{currentstroke}%
\pgfsetdash{}{0pt}%
\pgfpathmoveto{\pgfqpoint{0.859789in}{0.941307in}}%
\pgfpathlineto{\pgfqpoint{0.856879in}{0.932884in}}%
\pgfpathlineto{\pgfqpoint{0.853968in}{0.924524in}}%
\pgfpathlineto{\pgfqpoint{0.851057in}{0.916231in}}%
\pgfpathlineto{\pgfqpoint{0.848147in}{0.908008in}}%
\pgfpathlineto{\pgfqpoint{0.839737in}{0.913520in}}%
\pgfpathlineto{\pgfqpoint{0.831685in}{0.919162in}}%
\pgfpathlineto{\pgfqpoint{0.824000in}{0.924928in}}%
\pgfpathlineto{\pgfqpoint{0.816688in}{0.930813in}}%
\pgfpathlineto{\pgfqpoint{0.819865in}{0.938810in}}%
\pgfpathlineto{\pgfqpoint{0.823041in}{0.946878in}}%
\pgfpathlineto{\pgfqpoint{0.826218in}{0.955013in}}%
\pgfpathlineto{\pgfqpoint{0.829395in}{0.963212in}}%
\pgfpathlineto{\pgfqpoint{0.836461in}{0.957559in}}%
\pgfpathlineto{\pgfqpoint{0.843887in}{0.952020in}}%
\pgfpathlineto{\pgfqpoint{0.851666in}{0.946601in}}%
\pgfpathlineto{\pgfqpoint{0.859789in}{0.941307in}}%
\pgfpathclose%
\pgfusepath{fill}%
\end{pgfscope}%
\begin{pgfscope}%
\pgfpathrectangle{\pgfqpoint{0.041670in}{0.041670in}}{\pgfqpoint{2.216660in}{2.216660in}}%
\pgfusepath{clip}%
\pgfsetbuttcap%
\pgfsetroundjoin%
\definecolor{currentfill}{rgb}{0.814576,0.883393,0.110347}%
\pgfsetfillcolor{currentfill}%
\pgfsetlinewidth{0.000000pt}%
\definecolor{currentstroke}{rgb}{0.000000,0.000000,0.000000}%
\pgfsetstrokecolor{currentstroke}%
\pgfsetdash{}{0pt}%
\pgfpathmoveto{\pgfqpoint{1.229592in}{1.647623in}}%
\pgfpathlineto{\pgfqpoint{1.231343in}{1.645584in}}%
\pgfpathlineto{\pgfqpoint{1.233093in}{1.643428in}}%
\pgfpathlineto{\pgfqpoint{1.234840in}{1.641156in}}%
\pgfpathlineto{\pgfqpoint{1.236586in}{1.638770in}}%
\pgfpathlineto{\pgfqpoint{1.240039in}{1.637909in}}%
\pgfpathlineto{\pgfqpoint{1.243433in}{1.636998in}}%
\pgfpathlineto{\pgfqpoint{1.246766in}{1.636038in}}%
\pgfpathlineto{\pgfqpoint{1.250034in}{1.635028in}}%
\pgfpathlineto{\pgfqpoint{1.247872in}{1.637531in}}%
\pgfpathlineto{\pgfqpoint{1.245708in}{1.639920in}}%
\pgfpathlineto{\pgfqpoint{1.243543in}{1.642193in}}%
\pgfpathlineto{\pgfqpoint{1.241375in}{1.644349in}}%
\pgfpathlineto{\pgfqpoint{1.238512in}{1.645232in}}%
\pgfpathlineto{\pgfqpoint{1.235592in}{1.646073in}}%
\pgfpathlineto{\pgfqpoint{1.232618in}{1.646870in}}%
\pgfpathlineto{\pgfqpoint{1.229592in}{1.647623in}}%
\pgfpathclose%
\pgfusepath{fill}%
\end{pgfscope}%
\begin{pgfscope}%
\pgfpathrectangle{\pgfqpoint{0.041670in}{0.041670in}}{\pgfqpoint{2.216660in}{2.216660in}}%
\pgfusepath{clip}%
\pgfsetbuttcap%
\pgfsetroundjoin%
\definecolor{currentfill}{rgb}{0.220124,0.725509,0.466226}%
\pgfsetfillcolor{currentfill}%
\pgfsetlinewidth{0.000000pt}%
\definecolor{currentstroke}{rgb}{0.000000,0.000000,0.000000}%
\pgfsetstrokecolor{currentstroke}%
\pgfsetdash{}{0pt}%
\pgfpathmoveto{\pgfqpoint{0.936492in}{1.382806in}}%
\pgfpathlineto{\pgfqpoint{0.932757in}{1.375454in}}%
\pgfpathlineto{\pgfqpoint{0.929025in}{1.368043in}}%
\pgfpathlineto{\pgfqpoint{0.925295in}{1.360577in}}%
\pgfpathlineto{\pgfqpoint{0.921568in}{1.353057in}}%
\pgfpathlineto{\pgfqpoint{0.920235in}{1.357044in}}%
\pgfpathlineto{\pgfqpoint{0.919162in}{1.361046in}}%
\pgfpathlineto{\pgfqpoint{0.918352in}{1.365061in}}%
\pgfpathlineto{\pgfqpoint{0.917803in}{1.369084in}}%
\pgfpathlineto{\pgfqpoint{0.921575in}{1.376355in}}%
\pgfpathlineto{\pgfqpoint{0.925350in}{1.383574in}}%
\pgfpathlineto{\pgfqpoint{0.929129in}{1.390737in}}%
\pgfpathlineto{\pgfqpoint{0.932910in}{1.397843in}}%
\pgfpathlineto{\pgfqpoint{0.933436in}{1.394068in}}%
\pgfpathlineto{\pgfqpoint{0.934208in}{1.390301in}}%
\pgfpathlineto{\pgfqpoint{0.935227in}{1.386546in}}%
\pgfpathlineto{\pgfqpoint{0.936492in}{1.382806in}}%
\pgfpathclose%
\pgfusepath{fill}%
\end{pgfscope}%
\begin{pgfscope}%
\pgfpathrectangle{\pgfqpoint{0.041670in}{0.041670in}}{\pgfqpoint{2.216660in}{2.216660in}}%
\pgfusepath{clip}%
\pgfsetbuttcap%
\pgfsetroundjoin%
\definecolor{currentfill}{rgb}{0.212395,0.359683,0.551710}%
\pgfsetfillcolor{currentfill}%
\pgfsetlinewidth{0.000000pt}%
\definecolor{currentstroke}{rgb}{0.000000,0.000000,0.000000}%
\pgfsetstrokecolor{currentstroke}%
\pgfsetdash{}{0pt}%
\pgfpathmoveto{\pgfqpoint{1.523569in}{1.001486in}}%
\pgfpathlineto{\pgfqpoint{1.526799in}{0.993117in}}%
\pgfpathlineto{\pgfqpoint{1.530030in}{0.984799in}}%
\pgfpathlineto{\pgfqpoint{1.533259in}{0.976534in}}%
\pgfpathlineto{\pgfqpoint{1.536488in}{0.968327in}}%
\pgfpathlineto{\pgfqpoint{1.529747in}{0.962579in}}%
\pgfpathlineto{\pgfqpoint{1.522641in}{0.956938in}}%
\pgfpathlineto{\pgfqpoint{1.515176in}{0.951412in}}%
\pgfpathlineto{\pgfqpoint{1.507358in}{0.946006in}}%
\pgfpathlineto{\pgfqpoint{1.504384in}{0.954442in}}%
\pgfpathlineto{\pgfqpoint{1.501409in}{0.962934in}}%
\pgfpathlineto{\pgfqpoint{1.498434in}{0.971480in}}%
\pgfpathlineto{\pgfqpoint{1.495459in}{0.980077in}}%
\pgfpathlineto{\pgfqpoint{1.503001in}{0.985261in}}%
\pgfpathlineto{\pgfqpoint{1.510204in}{0.990562in}}%
\pgfpathlineto{\pgfqpoint{1.517062in}{0.995972in}}%
\pgfpathlineto{\pgfqpoint{1.523569in}{1.001486in}}%
\pgfpathclose%
\pgfusepath{fill}%
\end{pgfscope}%
\begin{pgfscope}%
\pgfpathrectangle{\pgfqpoint{0.041670in}{0.041670in}}{\pgfqpoint{2.216660in}{2.216660in}}%
\pgfusepath{clip}%
\pgfsetbuttcap%
\pgfsetroundjoin%
\definecolor{currentfill}{rgb}{0.233603,0.313828,0.543914}%
\pgfsetfillcolor{currentfill}%
\pgfsetlinewidth{0.000000pt}%
\definecolor{currentstroke}{rgb}{0.000000,0.000000,0.000000}%
\pgfsetstrokecolor{currentstroke}%
\pgfsetdash{}{0pt}%
\pgfpathmoveto{\pgfqpoint{1.873489in}{0.930022in}}%
\pgfpathlineto{\pgfqpoint{1.877618in}{0.940727in}}%
\pgfpathlineto{\pgfqpoint{1.881768in}{0.951884in}}%
\pgfpathlineto{\pgfqpoint{1.885937in}{0.963500in}}%
\pgfpathlineto{\pgfqpoint{1.890129in}{0.975583in}}%
\pgfpathlineto{\pgfqpoint{1.884127in}{0.964079in}}%
\pgfpathlineto{\pgfqpoint{1.877391in}{0.952660in}}%
\pgfpathlineto{\pgfqpoint{1.869924in}{0.941339in}}%
\pgfpathlineto{\pgfqpoint{1.861730in}{0.930128in}}%
\pgfpathlineto{\pgfqpoint{1.857681in}{0.918238in}}%
\pgfpathlineto{\pgfqpoint{1.853653in}{0.906818in}}%
\pgfpathlineto{\pgfqpoint{1.849645in}{0.895859in}}%
\pgfpathlineto{\pgfqpoint{1.845657in}{0.885354in}}%
\pgfpathlineto{\pgfqpoint{1.853682in}{0.896370in}}%
\pgfpathlineto{\pgfqpoint{1.860999in}{0.907495in}}%
\pgfpathlineto{\pgfqpoint{1.867602in}{0.918716in}}%
\pgfpathlineto{\pgfqpoint{1.873489in}{0.930022in}}%
\pgfpathclose%
\pgfusepath{fill}%
\end{pgfscope}%
\begin{pgfscope}%
\pgfpathrectangle{\pgfqpoint{0.041670in}{0.041670in}}{\pgfqpoint{2.216660in}{2.216660in}}%
\pgfusepath{clip}%
\pgfsetbuttcap%
\pgfsetroundjoin%
\definecolor{currentfill}{rgb}{0.636902,0.856542,0.216620}%
\pgfsetfillcolor{currentfill}%
\pgfsetlinewidth{0.000000pt}%
\definecolor{currentstroke}{rgb}{0.000000,0.000000,0.000000}%
\pgfsetstrokecolor{currentstroke}%
\pgfsetdash{}{0pt}%
\pgfpathmoveto{\pgfqpoint{1.318453in}{1.583672in}}%
\pgfpathlineto{\pgfqpoint{1.321864in}{1.579509in}}%
\pgfpathlineto{\pgfqpoint{1.325272in}{1.575245in}}%
\pgfpathlineto{\pgfqpoint{1.328677in}{1.570881in}}%
\pgfpathlineto{\pgfqpoint{1.332079in}{1.566419in}}%
\pgfpathlineto{\pgfqpoint{1.334460in}{1.564133in}}%
\pgfpathlineto{\pgfqpoint{1.336690in}{1.561812in}}%
\pgfpathlineto{\pgfqpoint{1.338766in}{1.559458in}}%
\pgfpathlineto{\pgfqpoint{1.340685in}{1.557074in}}%
\pgfpathlineto{\pgfqpoint{1.337086in}{1.561751in}}%
\pgfpathlineto{\pgfqpoint{1.333484in}{1.566331in}}%
\pgfpathlineto{\pgfqpoint{1.329878in}{1.570810in}}%
\pgfpathlineto{\pgfqpoint{1.326270in}{1.575188in}}%
\pgfpathlineto{\pgfqpoint{1.324527in}{1.577352in}}%
\pgfpathlineto{\pgfqpoint{1.322642in}{1.579489in}}%
\pgfpathlineto{\pgfqpoint{1.320616in}{1.581597in}}%
\pgfpathlineto{\pgfqpoint{1.318453in}{1.583672in}}%
\pgfpathclose%
\pgfusepath{fill}%
\end{pgfscope}%
\begin{pgfscope}%
\pgfpathrectangle{\pgfqpoint{0.041670in}{0.041670in}}{\pgfqpoint{2.216660in}{2.216660in}}%
\pgfusepath{clip}%
\pgfsetbuttcap%
\pgfsetroundjoin%
\definecolor{currentfill}{rgb}{0.283072,0.130895,0.449241}%
\pgfsetfillcolor{currentfill}%
\pgfsetlinewidth{0.000000pt}%
\definecolor{currentstroke}{rgb}{0.000000,0.000000,0.000000}%
\pgfsetstrokecolor{currentstroke}%
\pgfsetdash{}{0pt}%
\pgfpathmoveto{\pgfqpoint{1.566947in}{0.794556in}}%
\pgfpathlineto{\pgfqpoint{1.569939in}{0.788123in}}%
\pgfpathlineto{\pgfqpoint{1.572934in}{0.781825in}}%
\pgfpathlineto{\pgfqpoint{1.575930in}{0.775668in}}%
\pgfpathlineto{\pgfqpoint{1.578930in}{0.769655in}}%
\pgfpathlineto{\pgfqpoint{1.569047in}{0.763062in}}%
\pgfpathlineto{\pgfqpoint{1.558749in}{0.756632in}}%
\pgfpathlineto{\pgfqpoint{1.548046in}{0.750373in}}%
\pgfpathlineto{\pgfqpoint{1.536948in}{0.744291in}}%
\pgfpathlineto{\pgfqpoint{1.534255in}{0.750516in}}%
\pgfpathlineto{\pgfqpoint{1.531564in}{0.756885in}}%
\pgfpathlineto{\pgfqpoint{1.528876in}{0.763394in}}%
\pgfpathlineto{\pgfqpoint{1.526190in}{0.770039in}}%
\pgfpathlineto{\pgfqpoint{1.536962in}{0.775917in}}%
\pgfpathlineto{\pgfqpoint{1.547352in}{0.781967in}}%
\pgfpathlineto{\pgfqpoint{1.557350in}{0.788182in}}%
\pgfpathlineto{\pgfqpoint{1.566947in}{0.794556in}}%
\pgfpathclose%
\pgfusepath{fill}%
\end{pgfscope}%
\begin{pgfscope}%
\pgfpathrectangle{\pgfqpoint{0.041670in}{0.041670in}}{\pgfqpoint{2.216660in}{2.216660in}}%
\pgfusepath{clip}%
\pgfsetbuttcap%
\pgfsetroundjoin%
\definecolor{currentfill}{rgb}{0.134692,0.658636,0.517649}%
\pgfsetfillcolor{currentfill}%
\pgfsetlinewidth{0.000000pt}%
\definecolor{currentstroke}{rgb}{0.000000,0.000000,0.000000}%
\pgfsetstrokecolor{currentstroke}%
\pgfsetdash{}{0pt}%
\pgfpathmoveto{\pgfqpoint{0.915035in}{1.305803in}}%
\pgfpathlineto{\pgfqpoint{0.911427in}{1.297804in}}%
\pgfpathlineto{\pgfqpoint{0.907821in}{1.289765in}}%
\pgfpathlineto{\pgfqpoint{0.904218in}{1.281688in}}%
\pgfpathlineto{\pgfqpoint{0.900618in}{1.273577in}}%
\pgfpathlineto{\pgfqpoint{0.897997in}{1.277946in}}%
\pgfpathlineto{\pgfqpoint{0.895662in}{1.282351in}}%
\pgfpathlineto{\pgfqpoint{0.893615in}{1.286788in}}%
\pgfpathlineto{\pgfqpoint{0.891857in}{1.291251in}}%
\pgfpathlineto{\pgfqpoint{0.895561in}{1.299115in}}%
\pgfpathlineto{\pgfqpoint{0.899268in}{1.306944in}}%
\pgfpathlineto{\pgfqpoint{0.902978in}{1.314736in}}%
\pgfpathlineto{\pgfqpoint{0.906690in}{1.322488in}}%
\pgfpathlineto{\pgfqpoint{0.908367in}{1.318275in}}%
\pgfpathlineto{\pgfqpoint{0.910317in}{1.314086in}}%
\pgfpathlineto{\pgfqpoint{0.912541in}{1.309928in}}%
\pgfpathlineto{\pgfqpoint{0.915035in}{1.305803in}}%
\pgfpathclose%
\pgfusepath{fill}%
\end{pgfscope}%
\begin{pgfscope}%
\pgfpathrectangle{\pgfqpoint{0.041670in}{0.041670in}}{\pgfqpoint{2.216660in}{2.216660in}}%
\pgfusepath{clip}%
\pgfsetbuttcap%
\pgfsetroundjoin%
\definecolor{currentfill}{rgb}{0.814576,0.883393,0.110347}%
\pgfsetfillcolor{currentfill}%
\pgfsetlinewidth{0.000000pt}%
\definecolor{currentstroke}{rgb}{0.000000,0.000000,0.000000}%
\pgfsetstrokecolor{currentstroke}%
\pgfsetdash{}{0pt}%
\pgfpathmoveto{\pgfqpoint{1.116041in}{1.643528in}}%
\pgfpathlineto{\pgfqpoint{1.113784in}{1.641343in}}%
\pgfpathlineto{\pgfqpoint{1.111530in}{1.639041in}}%
\pgfpathlineto{\pgfqpoint{1.109278in}{1.636623in}}%
\pgfpathlineto{\pgfqpoint{1.107029in}{1.634091in}}%
\pgfpathlineto{\pgfqpoint{1.110236in}{1.635143in}}%
\pgfpathlineto{\pgfqpoint{1.113511in}{1.636147in}}%
\pgfpathlineto{\pgfqpoint{1.116851in}{1.637102in}}%
\pgfpathlineto{\pgfqpoint{1.120252in}{1.638007in}}%
\pgfpathlineto{\pgfqpoint{1.122092in}{1.640417in}}%
\pgfpathlineto{\pgfqpoint{1.123935in}{1.642713in}}%
\pgfpathlineto{\pgfqpoint{1.125779in}{1.644893in}}%
\pgfpathlineto{\pgfqpoint{1.127626in}{1.646956in}}%
\pgfpathlineto{\pgfqpoint{1.124646in}{1.646164in}}%
\pgfpathlineto{\pgfqpoint{1.121719in}{1.645328in}}%
\pgfpathlineto{\pgfqpoint{1.118850in}{1.644449in}}%
\pgfpathlineto{\pgfqpoint{1.116041in}{1.643528in}}%
\pgfpathclose%
\pgfusepath{fill}%
\end{pgfscope}%
\begin{pgfscope}%
\pgfpathrectangle{\pgfqpoint{0.041670in}{0.041670in}}{\pgfqpoint{2.216660in}{2.216660in}}%
\pgfusepath{clip}%
\pgfsetbuttcap%
\pgfsetroundjoin%
\definecolor{currentfill}{rgb}{0.565498,0.842430,0.262877}%
\pgfsetfillcolor{currentfill}%
\pgfsetlinewidth{0.000000pt}%
\definecolor{currentstroke}{rgb}{0.000000,0.000000,0.000000}%
\pgfsetstrokecolor{currentstroke}%
\pgfsetdash{}{0pt}%
\pgfpathmoveto{\pgfqpoint{1.340685in}{1.557074in}}%
\pgfpathlineto{\pgfqpoint{1.344281in}{1.552299in}}%
\pgfpathlineto{\pgfqpoint{1.347874in}{1.547430in}}%
\pgfpathlineto{\pgfqpoint{1.351463in}{1.542467in}}%
\pgfpathlineto{\pgfqpoint{1.355049in}{1.537411in}}%
\pgfpathlineto{\pgfqpoint{1.356973in}{1.534775in}}%
\pgfpathlineto{\pgfqpoint{1.358724in}{1.532110in}}%
\pgfpathlineto{\pgfqpoint{1.360298in}{1.529419in}}%
\pgfpathlineto{\pgfqpoint{1.361694in}{1.526704in}}%
\pgfpathlineto{\pgfqpoint{1.357966in}{1.531987in}}%
\pgfpathlineto{\pgfqpoint{1.354235in}{1.537178in}}%
\pgfpathlineto{\pgfqpoint{1.350500in}{1.542274in}}%
\pgfpathlineto{\pgfqpoint{1.346763in}{1.547276in}}%
\pgfpathlineto{\pgfqpoint{1.345487in}{1.549760in}}%
\pgfpathlineto{\pgfqpoint{1.344048in}{1.552222in}}%
\pgfpathlineto{\pgfqpoint{1.342446in}{1.554661in}}%
\pgfpathlineto{\pgfqpoint{1.340685in}{1.557074in}}%
\pgfpathclose%
\pgfusepath{fill}%
\end{pgfscope}%
\begin{pgfscope}%
\pgfpathrectangle{\pgfqpoint{0.041670in}{0.041670in}}{\pgfqpoint{2.216660in}{2.216660in}}%
\pgfusepath{clip}%
\pgfsetbuttcap%
\pgfsetroundjoin%
\definecolor{currentfill}{rgb}{0.271305,0.019942,0.347269}%
\pgfsetfillcolor{currentfill}%
\pgfsetlinewidth{0.000000pt}%
\definecolor{currentstroke}{rgb}{0.000000,0.000000,0.000000}%
\pgfsetstrokecolor{currentstroke}%
\pgfsetdash{}{0pt}%
\pgfpathmoveto{\pgfqpoint{1.615164in}{0.710200in}}%
\pgfpathlineto{\pgfqpoint{1.618209in}{0.706431in}}%
\pgfpathlineto{\pgfqpoint{1.621259in}{0.702866in}}%
\pgfpathlineto{\pgfqpoint{1.624314in}{0.699507in}}%
\pgfpathlineto{\pgfqpoint{1.627373in}{0.696360in}}%
\pgfpathlineto{\pgfqpoint{1.616338in}{0.688892in}}%
\pgfpathlineto{\pgfqpoint{1.604833in}{0.681607in}}%
\pgfpathlineto{\pgfqpoint{1.592871in}{0.674513in}}%
\pgfpathlineto{\pgfqpoint{1.580462in}{0.667620in}}%
\pgfpathlineto{\pgfqpoint{1.577713in}{0.670976in}}%
\pgfpathlineto{\pgfqpoint{1.574968in}{0.674544in}}%
\pgfpathlineto{\pgfqpoint{1.572228in}{0.678320in}}%
\pgfpathlineto{\pgfqpoint{1.569492in}{0.682299in}}%
\pgfpathlineto{\pgfqpoint{1.581571in}{0.688990in}}%
\pgfpathlineto{\pgfqpoint{1.593217in}{0.695877in}}%
\pgfpathlineto{\pgfqpoint{1.604418in}{0.702949in}}%
\pgfpathlineto{\pgfqpoint{1.615164in}{0.710200in}}%
\pgfpathclose%
\pgfusepath{fill}%
\end{pgfscope}%
\begin{pgfscope}%
\pgfpathrectangle{\pgfqpoint{0.041670in}{0.041670in}}{\pgfqpoint{2.216660in}{2.216660in}}%
\pgfusepath{clip}%
\pgfsetbuttcap%
\pgfsetroundjoin%
\definecolor{currentfill}{rgb}{0.699415,0.867117,0.175971}%
\pgfsetfillcolor{currentfill}%
\pgfsetlinewidth{0.000000pt}%
\definecolor{currentstroke}{rgb}{0.000000,0.000000,0.000000}%
\pgfsetstrokecolor{currentstroke}%
\pgfsetdash{}{0pt}%
\pgfpathmoveto{\pgfqpoint{1.295788in}{1.606425in}}%
\pgfpathlineto{\pgfqpoint{1.298961in}{1.602882in}}%
\pgfpathlineto{\pgfqpoint{1.302132in}{1.599231in}}%
\pgfpathlineto{\pgfqpoint{1.305299in}{1.595476in}}%
\pgfpathlineto{\pgfqpoint{1.308464in}{1.591616in}}%
\pgfpathlineto{\pgfqpoint{1.311156in}{1.589688in}}%
\pgfpathlineto{\pgfqpoint{1.313720in}{1.587720in}}%
\pgfpathlineto{\pgfqpoint{1.316153in}{1.585714in}}%
\pgfpathlineto{\pgfqpoint{1.318453in}{1.583672in}}%
\pgfpathlineto{\pgfqpoint{1.315039in}{1.587732in}}%
\pgfpathlineto{\pgfqpoint{1.311621in}{1.591688in}}%
\pgfpathlineto{\pgfqpoint{1.308200in}{1.595538in}}%
\pgfpathlineto{\pgfqpoint{1.304777in}{1.599282in}}%
\pgfpathlineto{\pgfqpoint{1.302708in}{1.601117in}}%
\pgfpathlineto{\pgfqpoint{1.300519in}{1.602921in}}%
\pgfpathlineto{\pgfqpoint{1.298211in}{1.604691in}}%
\pgfpathlineto{\pgfqpoint{1.295788in}{1.606425in}}%
\pgfpathclose%
\pgfusepath{fill}%
\end{pgfscope}%
\begin{pgfscope}%
\pgfpathrectangle{\pgfqpoint{0.041670in}{0.041670in}}{\pgfqpoint{2.216660in}{2.216660in}}%
\pgfusepath{clip}%
\pgfsetbuttcap%
\pgfsetroundjoin%
\definecolor{currentfill}{rgb}{0.282884,0.135920,0.453427}%
\pgfsetfillcolor{currentfill}%
\pgfsetlinewidth{0.000000pt}%
\definecolor{currentstroke}{rgb}{0.000000,0.000000,0.000000}%
\pgfsetstrokecolor{currentstroke}%
\pgfsetdash{}{0pt}%
\pgfpathmoveto{\pgfqpoint{0.606818in}{0.743200in}}%
\pgfpathlineto{\pgfqpoint{0.603290in}{0.748553in}}%
\pgfpathlineto{\pgfqpoint{0.599748in}{0.754278in}}%
\pgfpathlineto{\pgfqpoint{0.596192in}{0.760382in}}%
\pgfpathlineto{\pgfqpoint{0.592622in}{0.766872in}}%
\pgfpathlineto{\pgfqpoint{0.581723in}{0.776783in}}%
\pgfpathlineto{\pgfqpoint{0.571465in}{0.786857in}}%
\pgfpathlineto{\pgfqpoint{0.561856in}{0.797083in}}%
\pgfpathlineto{\pgfqpoint{0.552903in}{0.807449in}}%
\pgfpathlineto{\pgfqpoint{0.556694in}{0.800750in}}%
\pgfpathlineto{\pgfqpoint{0.560470in}{0.794435in}}%
\pgfpathlineto{\pgfqpoint{0.564231in}{0.788497in}}%
\pgfpathlineto{\pgfqpoint{0.567978in}{0.782930in}}%
\pgfpathlineto{\pgfqpoint{0.576736in}{0.772779in}}%
\pgfpathlineto{\pgfqpoint{0.586133in}{0.762767in}}%
\pgfpathlineto{\pgfqpoint{0.596164in}{0.752903in}}%
\pgfpathlineto{\pgfqpoint{0.606818in}{0.743200in}}%
\pgfpathclose%
\pgfusepath{fill}%
\end{pgfscope}%
\begin{pgfscope}%
\pgfpathrectangle{\pgfqpoint{0.041670in}{0.041670in}}{\pgfqpoint{2.216660in}{2.216660in}}%
\pgfusepath{clip}%
\pgfsetbuttcap%
\pgfsetroundjoin%
\definecolor{currentfill}{rgb}{0.344074,0.780029,0.397381}%
\pgfsetfillcolor{currentfill}%
\pgfsetlinewidth{0.000000pt}%
\definecolor{currentstroke}{rgb}{0.000000,0.000000,0.000000}%
\pgfsetstrokecolor{currentstroke}%
\pgfsetdash{}{0pt}%
\pgfpathmoveto{\pgfqpoint{0.963273in}{1.452354in}}%
\pgfpathlineto{\pgfqpoint{0.959467in}{1.445786in}}%
\pgfpathlineto{\pgfqpoint{0.955664in}{1.439144in}}%
\pgfpathlineto{\pgfqpoint{0.951864in}{1.432430in}}%
\pgfpathlineto{\pgfqpoint{0.948067in}{1.425645in}}%
\pgfpathlineto{\pgfqpoint{0.947798in}{1.429177in}}%
\pgfpathlineto{\pgfqpoint{0.947760in}{1.432710in}}%
\pgfpathlineto{\pgfqpoint{0.947953in}{1.436239in}}%
\pgfpathlineto{\pgfqpoint{0.948378in}{1.439762in}}%
\pgfpathlineto{\pgfqpoint{0.952162in}{1.446302in}}%
\pgfpathlineto{\pgfqpoint{0.955949in}{1.452773in}}%
\pgfpathlineto{\pgfqpoint{0.959740in}{1.459170in}}%
\pgfpathlineto{\pgfqpoint{0.963534in}{1.465494in}}%
\pgfpathlineto{\pgfqpoint{0.963144in}{1.462215in}}%
\pgfpathlineto{\pgfqpoint{0.962971in}{1.458929in}}%
\pgfpathlineto{\pgfqpoint{0.963014in}{1.455641in}}%
\pgfpathlineto{\pgfqpoint{0.963273in}{1.452354in}}%
\pgfpathclose%
\pgfusepath{fill}%
\end{pgfscope}%
\begin{pgfscope}%
\pgfpathrectangle{\pgfqpoint{0.041670in}{0.041670in}}{\pgfqpoint{2.216660in}{2.216660in}}%
\pgfusepath{clip}%
\pgfsetbuttcap%
\pgfsetroundjoin%
\definecolor{currentfill}{rgb}{0.163625,0.471133,0.558148}%
\pgfsetfillcolor{currentfill}%
\pgfsetlinewidth{0.000000pt}%
\definecolor{currentstroke}{rgb}{0.000000,0.000000,0.000000}%
\pgfsetstrokecolor{currentstroke}%
\pgfsetdash{}{0pt}%
\pgfpathmoveto{\pgfqpoint{0.880316in}{1.100427in}}%
\pgfpathlineto{\pgfqpoint{0.877126in}{1.091644in}}%
\pgfpathlineto{\pgfqpoint{0.873937in}{1.082874in}}%
\pgfpathlineto{\pgfqpoint{0.870749in}{1.074121in}}%
\pgfpathlineto{\pgfqpoint{0.867563in}{1.065387in}}%
\pgfpathlineto{\pgfqpoint{0.861565in}{1.070447in}}%
\pgfpathlineto{\pgfqpoint{0.855896in}{1.075597in}}%
\pgfpathlineto{\pgfqpoint{0.850564in}{1.080832in}}%
\pgfpathlineto{\pgfqpoint{0.845571in}{1.086145in}}%
\pgfpathlineto{\pgfqpoint{0.848972in}{1.094642in}}%
\pgfpathlineto{\pgfqpoint{0.852374in}{1.103159in}}%
\pgfpathlineto{\pgfqpoint{0.855778in}{1.111692in}}%
\pgfpathlineto{\pgfqpoint{0.859183in}{1.120240in}}%
\pgfpathlineto{\pgfqpoint{0.863982in}{1.115168in}}%
\pgfpathlineto{\pgfqpoint{0.869107in}{1.110172in}}%
\pgfpathlineto{\pgfqpoint{0.874553in}{1.105256in}}%
\pgfpathlineto{\pgfqpoint{0.880316in}{1.100427in}}%
\pgfpathclose%
\pgfusepath{fill}%
\end{pgfscope}%
\begin{pgfscope}%
\pgfpathrectangle{\pgfqpoint{0.041670in}{0.041670in}}{\pgfqpoint{2.216660in}{2.216660in}}%
\pgfusepath{clip}%
\pgfsetbuttcap%
\pgfsetroundjoin%
\definecolor{currentfill}{rgb}{0.280255,0.165693,0.476498}%
\pgfsetfillcolor{currentfill}%
\pgfsetlinewidth{0.000000pt}%
\definecolor{currentstroke}{rgb}{0.000000,0.000000,0.000000}%
\pgfsetstrokecolor{currentstroke}%
\pgfsetdash{}{0pt}%
\pgfpathmoveto{\pgfqpoint{1.554996in}{0.821571in}}%
\pgfpathlineto{\pgfqpoint{1.557981in}{0.814632in}}%
\pgfpathlineto{\pgfqpoint{1.560968in}{0.807814in}}%
\pgfpathlineto{\pgfqpoint{1.563956in}{0.801121in}}%
\pgfpathlineto{\pgfqpoint{1.566947in}{0.794556in}}%
\pgfpathlineto{\pgfqpoint{1.557350in}{0.788182in}}%
\pgfpathlineto{\pgfqpoint{1.547352in}{0.781967in}}%
\pgfpathlineto{\pgfqpoint{1.536962in}{0.775917in}}%
\pgfpathlineto{\pgfqpoint{1.526190in}{0.770039in}}%
\pgfpathlineto{\pgfqpoint{1.523505in}{0.776816in}}%
\pgfpathlineto{\pgfqpoint{1.520823in}{0.783722in}}%
\pgfpathlineto{\pgfqpoint{1.518142in}{0.790752in}}%
\pgfpathlineto{\pgfqpoint{1.515462in}{0.797902in}}%
\pgfpathlineto{\pgfqpoint{1.525910in}{0.803576in}}%
\pgfpathlineto{\pgfqpoint{1.535987in}{0.809417in}}%
\pgfpathlineto{\pgfqpoint{1.545686in}{0.815417in}}%
\pgfpathlineto{\pgfqpoint{1.554996in}{0.821571in}}%
\pgfpathclose%
\pgfusepath{fill}%
\end{pgfscope}%
\begin{pgfscope}%
\pgfpathrectangle{\pgfqpoint{0.041670in}{0.041670in}}{\pgfqpoint{2.216660in}{2.216660in}}%
\pgfusepath{clip}%
\pgfsetbuttcap%
\pgfsetroundjoin%
\definecolor{currentfill}{rgb}{0.636902,0.856542,0.216620}%
\pgfsetfillcolor{currentfill}%
\pgfsetlinewidth{0.000000pt}%
\definecolor{currentstroke}{rgb}{0.000000,0.000000,0.000000}%
\pgfsetstrokecolor{currentstroke}%
\pgfsetdash{}{0pt}%
\pgfpathmoveto{\pgfqpoint{1.032212in}{1.573242in}}%
\pgfpathlineto{\pgfqpoint{1.028567in}{1.568815in}}%
\pgfpathlineto{\pgfqpoint{1.024925in}{1.564286in}}%
\pgfpathlineto{\pgfqpoint{1.021287in}{1.559658in}}%
\pgfpathlineto{\pgfqpoint{1.017651in}{1.554930in}}%
\pgfpathlineto{\pgfqpoint{1.019430in}{1.557340in}}%
\pgfpathlineto{\pgfqpoint{1.021367in}{1.559722in}}%
\pgfpathlineto{\pgfqpoint{1.023460in}{1.562072in}}%
\pgfpathlineto{\pgfqpoint{1.025707in}{1.564389in}}%
\pgfpathlineto{\pgfqpoint{1.029157in}{1.568898in}}%
\pgfpathlineto{\pgfqpoint{1.032610in}{1.573308in}}%
\pgfpathlineto{\pgfqpoint{1.036067in}{1.577619in}}%
\pgfpathlineto{\pgfqpoint{1.039527in}{1.581829in}}%
\pgfpathlineto{\pgfqpoint{1.037486in}{1.579725in}}%
\pgfpathlineto{\pgfqpoint{1.035585in}{1.577591in}}%
\pgfpathlineto{\pgfqpoint{1.033827in}{1.575429in}}%
\pgfpathlineto{\pgfqpoint{1.032212in}{1.573242in}}%
\pgfpathclose%
\pgfusepath{fill}%
\end{pgfscope}%
\begin{pgfscope}%
\pgfpathrectangle{\pgfqpoint{0.041670in}{0.041670in}}{\pgfqpoint{2.216660in}{2.216660in}}%
\pgfusepath{clip}%
\pgfsetbuttcap%
\pgfsetroundjoin%
\definecolor{currentfill}{rgb}{0.487026,0.823929,0.312321}%
\pgfsetfillcolor{currentfill}%
\pgfsetlinewidth{0.000000pt}%
\definecolor{currentstroke}{rgb}{0.000000,0.000000,0.000000}%
\pgfsetstrokecolor{currentstroke}%
\pgfsetdash{}{0pt}%
\pgfpathmoveto{\pgfqpoint{1.361694in}{1.526704in}}%
\pgfpathlineto{\pgfqpoint{1.365419in}{1.521331in}}%
\pgfpathlineto{\pgfqpoint{1.369140in}{1.515869in}}%
\pgfpathlineto{\pgfqpoint{1.372858in}{1.510320in}}%
\pgfpathlineto{\pgfqpoint{1.376573in}{1.504686in}}%
\pgfpathlineto{\pgfqpoint{1.377896in}{1.501717in}}%
\pgfpathlineto{\pgfqpoint{1.379023in}{1.498728in}}%
\pgfpathlineto{\pgfqpoint{1.379953in}{1.495722in}}%
\pgfpathlineto{\pgfqpoint{1.380685in}{1.492703in}}%
\pgfpathlineto{\pgfqpoint{1.376886in}{1.498574in}}%
\pgfpathlineto{\pgfqpoint{1.373084in}{1.504359in}}%
\pgfpathlineto{\pgfqpoint{1.369278in}{1.510057in}}%
\pgfpathlineto{\pgfqpoint{1.365470in}{1.515666in}}%
\pgfpathlineto{\pgfqpoint{1.364800in}{1.518447in}}%
\pgfpathlineto{\pgfqpoint{1.363946in}{1.521216in}}%
\pgfpathlineto{\pgfqpoint{1.362911in}{1.523969in}}%
\pgfpathlineto{\pgfqpoint{1.361694in}{1.526704in}}%
\pgfpathclose%
\pgfusepath{fill}%
\end{pgfscope}%
\begin{pgfscope}%
\pgfpathrectangle{\pgfqpoint{0.041670in}{0.041670in}}{\pgfqpoint{2.216660in}{2.216660in}}%
\pgfusepath{clip}%
\pgfsetbuttcap%
\pgfsetroundjoin%
\definecolor{currentfill}{rgb}{0.699415,0.867117,0.175971}%
\pgfsetfillcolor{currentfill}%
\pgfsetlinewidth{0.000000pt}%
\definecolor{currentstroke}{rgb}{0.000000,0.000000,0.000000}%
\pgfsetstrokecolor{currentstroke}%
\pgfsetdash{}{0pt}%
\pgfpathmoveto{\pgfqpoint{1.053397in}{1.597624in}}%
\pgfpathlineto{\pgfqpoint{1.049925in}{1.593835in}}%
\pgfpathlineto{\pgfqpoint{1.046456in}{1.589938in}}%
\pgfpathlineto{\pgfqpoint{1.042990in}{1.585936in}}%
\pgfpathlineto{\pgfqpoint{1.039527in}{1.581829in}}%
\pgfpathlineto{\pgfqpoint{1.041706in}{1.583901in}}%
\pgfpathlineto{\pgfqpoint{1.044020in}{1.585938in}}%
\pgfpathlineto{\pgfqpoint{1.046468in}{1.587940in}}%
\pgfpathlineto{\pgfqpoint{1.049046in}{1.589904in}}%
\pgfpathlineto{\pgfqpoint{1.052271in}{1.593807in}}%
\pgfpathlineto{\pgfqpoint{1.055498in}{1.597606in}}%
\pgfpathlineto{\pgfqpoint{1.058729in}{1.601299in}}%
\pgfpathlineto{\pgfqpoint{1.061962in}{1.604885in}}%
\pgfpathlineto{\pgfqpoint{1.059641in}{1.603119in}}%
\pgfpathlineto{\pgfqpoint{1.057439in}{1.601319in}}%
\pgfpathlineto{\pgfqpoint{1.055357in}{1.599487in}}%
\pgfpathlineto{\pgfqpoint{1.053397in}{1.597624in}}%
\pgfpathclose%
\pgfusepath{fill}%
\end{pgfscope}%
\begin{pgfscope}%
\pgfpathrectangle{\pgfqpoint{0.041670in}{0.041670in}}{\pgfqpoint{2.216660in}{2.216660in}}%
\pgfusepath{clip}%
\pgfsetbuttcap%
\pgfsetroundjoin%
\definecolor{currentfill}{rgb}{0.565498,0.842430,0.262877}%
\pgfsetfillcolor{currentfill}%
\pgfsetlinewidth{0.000000pt}%
\definecolor{currentstroke}{rgb}{0.000000,0.000000,0.000000}%
\pgfsetstrokecolor{currentstroke}%
\pgfsetdash{}{0pt}%
\pgfpathmoveto{\pgfqpoint{1.012151in}{1.545052in}}%
\pgfpathlineto{\pgfqpoint{1.008390in}{1.539999in}}%
\pgfpathlineto{\pgfqpoint{1.004632in}{1.534851in}}%
\pgfpathlineto{\pgfqpoint{1.000877in}{1.529609in}}%
\pgfpathlineto{\pgfqpoint{0.997125in}{1.524274in}}%
\pgfpathlineto{\pgfqpoint{0.998362in}{1.527007in}}%
\pgfpathlineto{\pgfqpoint{0.999778in}{1.529719in}}%
\pgfpathlineto{\pgfqpoint{1.001372in}{1.532407in}}%
\pgfpathlineto{\pgfqpoint{1.003142in}{1.535069in}}%
\pgfpathlineto{\pgfqpoint{1.006764in}{1.540174in}}%
\pgfpathlineto{\pgfqpoint{1.010390in}{1.545187in}}%
\pgfpathlineto{\pgfqpoint{1.014019in}{1.550106in}}%
\pgfpathlineto{\pgfqpoint{1.017651in}{1.554930in}}%
\pgfpathlineto{\pgfqpoint{1.016032in}{1.552494in}}%
\pgfpathlineto{\pgfqpoint{1.014575in}{1.550034in}}%
\pgfpathlineto{\pgfqpoint{1.013281in}{1.547553in}}%
\pgfpathlineto{\pgfqpoint{1.012151in}{1.545052in}}%
\pgfpathclose%
\pgfusepath{fill}%
\end{pgfscope}%
\begin{pgfscope}%
\pgfpathrectangle{\pgfqpoint{0.041670in}{0.041670in}}{\pgfqpoint{2.216660in}{2.216660in}}%
\pgfusepath{clip}%
\pgfsetbuttcap%
\pgfsetroundjoin%
\definecolor{currentfill}{rgb}{0.122606,0.585371,0.546557}%
\pgfsetfillcolor{currentfill}%
\pgfsetlinewidth{0.000000pt}%
\definecolor{currentstroke}{rgb}{0.000000,0.000000,0.000000}%
\pgfsetstrokecolor{currentstroke}%
\pgfsetdash{}{0pt}%
\pgfpathmoveto{\pgfqpoint{0.900184in}{1.222836in}}%
\pgfpathlineto{\pgfqpoint{0.896756in}{1.214346in}}%
\pgfpathlineto{\pgfqpoint{0.893331in}{1.205836in}}%
\pgfpathlineto{\pgfqpoint{0.889908in}{1.197309in}}%
\pgfpathlineto{\pgfqpoint{0.886487in}{1.188767in}}%
\pgfpathlineto{\pgfqpoint{0.882379in}{1.193425in}}%
\pgfpathlineto{\pgfqpoint{0.878577in}{1.198143in}}%
\pgfpathlineto{\pgfqpoint{0.875082in}{1.202914in}}%
\pgfpathlineto{\pgfqpoint{0.871899in}{1.207735in}}%
\pgfpathlineto{\pgfqpoint{0.875481in}{1.216033in}}%
\pgfpathlineto{\pgfqpoint{0.879064in}{1.224317in}}%
\pgfpathlineto{\pgfqpoint{0.882651in}{1.232585in}}%
\pgfpathlineto{\pgfqpoint{0.886239in}{1.240832in}}%
\pgfpathlineto{\pgfqpoint{0.889284in}{1.236258in}}%
\pgfpathlineto{\pgfqpoint{0.892625in}{1.231731in}}%
\pgfpathlineto{\pgfqpoint{0.896259in}{1.227256in}}%
\pgfpathlineto{\pgfqpoint{0.900184in}{1.222836in}}%
\pgfpathclose%
\pgfusepath{fill}%
\end{pgfscope}%
\begin{pgfscope}%
\pgfpathrectangle{\pgfqpoint{0.041670in}{0.041670in}}{\pgfqpoint{2.216660in}{2.216660in}}%
\pgfusepath{clip}%
\pgfsetbuttcap%
\pgfsetroundjoin%
\definecolor{currentfill}{rgb}{0.814576,0.883393,0.110347}%
\pgfsetfillcolor{currentfill}%
\pgfsetlinewidth{0.000000pt}%
\definecolor{currentstroke}{rgb}{0.000000,0.000000,0.000000}%
\pgfsetstrokecolor{currentstroke}%
\pgfsetdash{}{0pt}%
\pgfpathmoveto{\pgfqpoint{1.241375in}{1.644349in}}%
\pgfpathlineto{\pgfqpoint{1.243543in}{1.642193in}}%
\pgfpathlineto{\pgfqpoint{1.245708in}{1.639920in}}%
\pgfpathlineto{\pgfqpoint{1.247872in}{1.637531in}}%
\pgfpathlineto{\pgfqpoint{1.250034in}{1.635028in}}%
\pgfpathlineto{\pgfqpoint{1.253233in}{1.633971in}}%
\pgfpathlineto{\pgfqpoint{1.256361in}{1.632867in}}%
\pgfpathlineto{\pgfqpoint{1.259414in}{1.631717in}}%
\pgfpathlineto{\pgfqpoint{1.262390in}{1.630522in}}%
\pgfpathlineto{\pgfqpoint{1.259845in}{1.633166in}}%
\pgfpathlineto{\pgfqpoint{1.257299in}{1.635695in}}%
\pgfpathlineto{\pgfqpoint{1.254749in}{1.638109in}}%
\pgfpathlineto{\pgfqpoint{1.252198in}{1.640406in}}%
\pgfpathlineto{\pgfqpoint{1.249592in}{1.641451in}}%
\pgfpathlineto{\pgfqpoint{1.246917in}{1.642457in}}%
\pgfpathlineto{\pgfqpoint{1.244177in}{1.643424in}}%
\pgfpathlineto{\pgfqpoint{1.241375in}{1.644349in}}%
\pgfpathclose%
\pgfusepath{fill}%
\end{pgfscope}%
\begin{pgfscope}%
\pgfpathrectangle{\pgfqpoint{0.041670in}{0.041670in}}{\pgfqpoint{2.216660in}{2.216660in}}%
\pgfusepath{clip}%
\pgfsetbuttcap%
\pgfsetroundjoin%
\definecolor{currentfill}{rgb}{0.268510,0.009605,0.335427}%
\pgfsetfillcolor{currentfill}%
\pgfsetlinewidth{0.000000pt}%
\definecolor{currentstroke}{rgb}{0.000000,0.000000,0.000000}%
\pgfsetstrokecolor{currentstroke}%
\pgfsetdash{}{0pt}%
\pgfpathmoveto{\pgfqpoint{1.627373in}{0.696360in}}%
\pgfpathlineto{\pgfqpoint{1.630438in}{0.693430in}}%
\pgfpathlineto{\pgfqpoint{1.633509in}{0.690720in}}%
\pgfpathlineto{\pgfqpoint{1.636585in}{0.688236in}}%
\pgfpathlineto{\pgfqpoint{1.639667in}{0.685983in}}%
\pgfpathlineto{\pgfqpoint{1.628341in}{0.678298in}}%
\pgfpathlineto{\pgfqpoint{1.616532in}{0.670802in}}%
\pgfpathlineto{\pgfqpoint{1.604251in}{0.663502in}}%
\pgfpathlineto{\pgfqpoint{1.591511in}{0.656408in}}%
\pgfpathlineto{\pgfqpoint{1.588741in}{0.658869in}}%
\pgfpathlineto{\pgfqpoint{1.585976in}{0.661562in}}%
\pgfpathlineto{\pgfqpoint{1.583217in}{0.664480in}}%
\pgfpathlineto{\pgfqpoint{1.580462in}{0.667620in}}%
\pgfpathlineto{\pgfqpoint{1.592871in}{0.674513in}}%
\pgfpathlineto{\pgfqpoint{1.604833in}{0.681607in}}%
\pgfpathlineto{\pgfqpoint{1.616338in}{0.688892in}}%
\pgfpathlineto{\pgfqpoint{1.627373in}{0.696360in}}%
\pgfpathclose%
\pgfusepath{fill}%
\end{pgfscope}%
\begin{pgfscope}%
\pgfpathrectangle{\pgfqpoint{0.041670in}{0.041670in}}{\pgfqpoint{2.216660in}{2.216660in}}%
\pgfusepath{clip}%
\pgfsetbuttcap%
\pgfsetroundjoin%
\definecolor{currentfill}{rgb}{0.855810,0.888601,0.097452}%
\pgfsetfillcolor{currentfill}%
\pgfsetlinewidth{0.000000pt}%
\definecolor{currentstroke}{rgb}{0.000000,0.000000,0.000000}%
\pgfsetstrokecolor{currentstroke}%
\pgfsetdash{}{0pt}%
\pgfpathmoveto{\pgfqpoint{1.168396in}{1.659061in}}%
\pgfpathlineto{\pgfqpoint{1.167920in}{1.657685in}}%
\pgfpathlineto{\pgfqpoint{1.167444in}{1.656188in}}%
\pgfpathlineto{\pgfqpoint{1.166968in}{1.654570in}}%
\pgfpathlineto{\pgfqpoint{1.166493in}{1.652834in}}%
\pgfpathlineto{\pgfqpoint{1.169895in}{1.653007in}}%
\pgfpathlineto{\pgfqpoint{1.173306in}{1.653130in}}%
\pgfpathlineto{\pgfqpoint{1.176724in}{1.653202in}}%
\pgfpathlineto{\pgfqpoint{1.180145in}{1.653225in}}%
\pgfpathlineto{\pgfqpoint{1.180138in}{1.654947in}}%
\pgfpathlineto{\pgfqpoint{1.180132in}{1.656551in}}%
\pgfpathlineto{\pgfqpoint{1.180125in}{1.658034in}}%
\pgfpathlineto{\pgfqpoint{1.180118in}{1.659397in}}%
\pgfpathlineto{\pgfqpoint{1.177181in}{1.659378in}}%
\pgfpathlineto{\pgfqpoint{1.174246in}{1.659315in}}%
\pgfpathlineto{\pgfqpoint{1.171317in}{1.659210in}}%
\pgfpathlineto{\pgfqpoint{1.168396in}{1.659061in}}%
\pgfpathclose%
\pgfusepath{fill}%
\end{pgfscope}%
\begin{pgfscope}%
\pgfpathrectangle{\pgfqpoint{0.041670in}{0.041670in}}{\pgfqpoint{2.216660in}{2.216660in}}%
\pgfusepath{clip}%
\pgfsetbuttcap%
\pgfsetroundjoin%
\definecolor{currentfill}{rgb}{0.855810,0.888601,0.097452}%
\pgfsetfillcolor{currentfill}%
\pgfsetlinewidth{0.000000pt}%
\definecolor{currentstroke}{rgb}{0.000000,0.000000,0.000000}%
\pgfsetstrokecolor{currentstroke}%
\pgfsetdash{}{0pt}%
\pgfpathmoveto{\pgfqpoint{1.180118in}{1.659397in}}%
\pgfpathlineto{\pgfqpoint{1.180125in}{1.658034in}}%
\pgfpathlineto{\pgfqpoint{1.180132in}{1.656551in}}%
\pgfpathlineto{\pgfqpoint{1.180138in}{1.654947in}}%
\pgfpathlineto{\pgfqpoint{1.180145in}{1.653225in}}%
\pgfpathlineto{\pgfqpoint{1.183566in}{1.653197in}}%
\pgfpathlineto{\pgfqpoint{1.186983in}{1.653118in}}%
\pgfpathlineto{\pgfqpoint{1.190394in}{1.652990in}}%
\pgfpathlineto{\pgfqpoint{1.193794in}{1.652811in}}%
\pgfpathlineto{\pgfqpoint{1.193306in}{1.654549in}}%
\pgfpathlineto{\pgfqpoint{1.192817in}{1.656167in}}%
\pgfpathlineto{\pgfqpoint{1.192327in}{1.657665in}}%
\pgfpathlineto{\pgfqpoint{1.191838in}{1.659042in}}%
\pgfpathlineto{\pgfqpoint{1.188918in}{1.659195in}}%
\pgfpathlineto{\pgfqpoint{1.185990in}{1.659306in}}%
\pgfpathlineto{\pgfqpoint{1.183055in}{1.659373in}}%
\pgfpathlineto{\pgfqpoint{1.180118in}{1.659397in}}%
\pgfpathclose%
\pgfusepath{fill}%
\end{pgfscope}%
\begin{pgfscope}%
\pgfpathrectangle{\pgfqpoint{0.041670in}{0.041670in}}{\pgfqpoint{2.216660in}{2.216660in}}%
\pgfusepath{clip}%
\pgfsetbuttcap%
\pgfsetroundjoin%
\definecolor{currentfill}{rgb}{0.147607,0.511733,0.557049}%
\pgfsetfillcolor{currentfill}%
\pgfsetlinewidth{0.000000pt}%
\definecolor{currentstroke}{rgb}{0.000000,0.000000,0.000000}%
\pgfsetstrokecolor{currentstroke}%
\pgfsetdash{}{0pt}%
\pgfpathmoveto{\pgfqpoint{1.490917in}{1.158861in}}%
\pgfpathlineto{\pgfqpoint{1.494370in}{1.150341in}}%
\pgfpathlineto{\pgfqpoint{1.497820in}{1.141824in}}%
\pgfpathlineto{\pgfqpoint{1.501269in}{1.133311in}}%
\pgfpathlineto{\pgfqpoint{1.504716in}{1.124807in}}%
\pgfpathlineto{\pgfqpoint{1.500210in}{1.119673in}}%
\pgfpathlineto{\pgfqpoint{1.495374in}{1.114609in}}%
\pgfpathlineto{\pgfqpoint{1.490213in}{1.109622in}}%
\pgfpathlineto{\pgfqpoint{1.484732in}{1.104715in}}%
\pgfpathlineto{\pgfqpoint{1.481488in}{1.113458in}}%
\pgfpathlineto{\pgfqpoint{1.478242in}{1.122208in}}%
\pgfpathlineto{\pgfqpoint{1.474995in}{1.130962in}}%
\pgfpathlineto{\pgfqpoint{1.471747in}{1.139719in}}%
\pgfpathlineto{\pgfqpoint{1.477004in}{1.144392in}}%
\pgfpathlineto{\pgfqpoint{1.481954in}{1.149144in}}%
\pgfpathlineto{\pgfqpoint{1.486593in}{1.153969in}}%
\pgfpathlineto{\pgfqpoint{1.490917in}{1.158861in}}%
\pgfpathclose%
\pgfusepath{fill}%
\end{pgfscope}%
\begin{pgfscope}%
\pgfpathrectangle{\pgfqpoint{0.041670in}{0.041670in}}{\pgfqpoint{2.216660in}{2.216660in}}%
\pgfusepath{clip}%
\pgfsetbuttcap%
\pgfsetroundjoin%
\definecolor{currentfill}{rgb}{0.172719,0.448791,0.557885}%
\pgfsetfillcolor{currentfill}%
\pgfsetlinewidth{0.000000pt}%
\definecolor{currentstroke}{rgb}{0.000000,0.000000,0.000000}%
\pgfsetstrokecolor{currentstroke}%
\pgfsetdash{}{0pt}%
\pgfpathmoveto{\pgfqpoint{0.458194in}{1.018346in}}%
\pgfpathlineto{\pgfqpoint{0.453914in}{1.032843in}}%
\pgfpathlineto{\pgfqpoint{0.449610in}{1.047858in}}%
\pgfpathlineto{\pgfqpoint{0.445281in}{1.063398in}}%
\pgfpathlineto{\pgfqpoint{0.439161in}{1.075231in}}%
\pgfpathlineto{\pgfqpoint{0.433811in}{1.087137in}}%
\pgfpathlineto{\pgfqpoint{0.429232in}{1.099102in}}%
\pgfpathlineto{\pgfqpoint{0.425426in}{1.111113in}}%
\pgfpathlineto{\pgfqpoint{0.429842in}{1.095399in}}%
\pgfpathlineto{\pgfqpoint{0.434233in}{1.080207in}}%
\pgfpathlineto{\pgfqpoint{0.438600in}{1.065528in}}%
\pgfpathlineto{\pgfqpoint{0.442362in}{1.053650in}}%
\pgfpathlineto{\pgfqpoint{0.446883in}{1.041819in}}%
\pgfpathlineto{\pgfqpoint{0.452160in}{1.030046in}}%
\pgfpathlineto{\pgfqpoint{0.458194in}{1.018346in}}%
\pgfpathclose%
\pgfusepath{fill}%
\end{pgfscope}%
\begin{pgfscope}%
\pgfpathrectangle{\pgfqpoint{0.041670in}{0.041670in}}{\pgfqpoint{2.216660in}{2.216660in}}%
\pgfusepath{clip}%
\pgfsetbuttcap%
\pgfsetroundjoin%
\definecolor{currentfill}{rgb}{0.762373,0.876424,0.137064}%
\pgfsetfillcolor{currentfill}%
\pgfsetlinewidth{0.000000pt}%
\definecolor{currentstroke}{rgb}{0.000000,0.000000,0.000000}%
\pgfsetstrokecolor{currentstroke}%
\pgfsetdash{}{0pt}%
\pgfpathmoveto{\pgfqpoint{1.273459in}{1.625320in}}%
\pgfpathlineto{\pgfqpoint{1.276345in}{1.622400in}}%
\pgfpathlineto{\pgfqpoint{1.279227in}{1.619368in}}%
\pgfpathlineto{\pgfqpoint{1.282107in}{1.616226in}}%
\pgfpathlineto{\pgfqpoint{1.284985in}{1.612974in}}%
\pgfpathlineto{\pgfqpoint{1.287847in}{1.611398in}}%
\pgfpathlineto{\pgfqpoint{1.290603in}{1.609780in}}%
\pgfpathlineto{\pgfqpoint{1.293251in}{1.608122in}}%
\pgfpathlineto{\pgfqpoint{1.295788in}{1.606425in}}%
\pgfpathlineto{\pgfqpoint{1.292611in}{1.609860in}}%
\pgfpathlineto{\pgfqpoint{1.289432in}{1.613185in}}%
\pgfpathlineto{\pgfqpoint{1.286250in}{1.616399in}}%
\pgfpathlineto{\pgfqpoint{1.283065in}{1.619502in}}%
\pgfpathlineto{\pgfqpoint{1.280810in}{1.621009in}}%
\pgfpathlineto{\pgfqpoint{1.278455in}{1.622483in}}%
\pgfpathlineto{\pgfqpoint{1.276004in}{1.623920in}}%
\pgfpathlineto{\pgfqpoint{1.273459in}{1.625320in}}%
\pgfpathclose%
\pgfusepath{fill}%
\end{pgfscope}%
\begin{pgfscope}%
\pgfpathrectangle{\pgfqpoint{0.041670in}{0.041670in}}{\pgfqpoint{2.216660in}{2.216660in}}%
\pgfusepath{clip}%
\pgfsetbuttcap%
\pgfsetroundjoin%
\definecolor{currentfill}{rgb}{0.272594,0.025563,0.353093}%
\pgfsetfillcolor{currentfill}%
\pgfsetlinewidth{0.000000pt}%
\definecolor{currentstroke}{rgb}{0.000000,0.000000,0.000000}%
\pgfsetstrokecolor{currentstroke}%
\pgfsetdash{}{0pt}%
\pgfpathmoveto{\pgfqpoint{0.693536in}{0.671279in}}%
\pgfpathlineto{\pgfqpoint{0.690421in}{0.672424in}}%
\pgfpathlineto{\pgfqpoint{0.687297in}{0.673872in}}%
\pgfpathlineto{\pgfqpoint{0.684164in}{0.675627in}}%
\pgfpathlineto{\pgfqpoint{0.681021in}{0.677695in}}%
\pgfpathlineto{\pgfqpoint{0.668572in}{0.686245in}}%
\pgfpathlineto{\pgfqpoint{0.656677in}{0.694990in}}%
\pgfpathlineto{\pgfqpoint{0.645345in}{0.703922in}}%
\pgfpathlineto{\pgfqpoint{0.634588in}{0.713028in}}%
\pgfpathlineto{\pgfqpoint{0.638007in}{0.710748in}}%
\pgfpathlineto{\pgfqpoint{0.641416in}{0.708780in}}%
\pgfpathlineto{\pgfqpoint{0.644815in}{0.707119in}}%
\pgfpathlineto{\pgfqpoint{0.648204in}{0.705759in}}%
\pgfpathlineto{\pgfqpoint{0.658709in}{0.696872in}}%
\pgfpathlineto{\pgfqpoint{0.669772in}{0.688156in}}%
\pgfpathlineto{\pgfqpoint{0.681385in}{0.679621in}}%
\pgfpathlineto{\pgfqpoint{0.693536in}{0.671279in}}%
\pgfpathclose%
\pgfusepath{fill}%
\end{pgfscope}%
\begin{pgfscope}%
\pgfpathrectangle{\pgfqpoint{0.041670in}{0.041670in}}{\pgfqpoint{2.216660in}{2.216660in}}%
\pgfusepath{clip}%
\pgfsetbuttcap%
\pgfsetroundjoin%
\definecolor{currentfill}{rgb}{0.279566,0.067836,0.391917}%
\pgfsetfillcolor{currentfill}%
\pgfsetlinewidth{0.000000pt}%
\definecolor{currentstroke}{rgb}{0.000000,0.000000,0.000000}%
\pgfsetstrokecolor{currentstroke}%
\pgfsetdash{}{0pt}%
\pgfpathmoveto{\pgfqpoint{0.822652in}{0.715481in}}%
\pgfpathlineto{\pgfqpoint{0.820021in}{0.709992in}}%
\pgfpathlineto{\pgfqpoint{0.817387in}{0.704671in}}%
\pgfpathlineto{\pgfqpoint{0.814750in}{0.699523in}}%
\pgfpathlineto{\pgfqpoint{0.812109in}{0.694552in}}%
\pgfpathlineto{\pgfqpoint{0.799994in}{0.700866in}}%
\pgfpathlineto{\pgfqpoint{0.788289in}{0.707377in}}%
\pgfpathlineto{\pgfqpoint{0.777006in}{0.714075in}}%
\pgfpathlineto{\pgfqpoint{0.766155in}{0.720954in}}%
\pgfpathlineto{\pgfqpoint{0.769115in}{0.725719in}}%
\pgfpathlineto{\pgfqpoint{0.772072in}{0.730660in}}%
\pgfpathlineto{\pgfqpoint{0.775025in}{0.735774in}}%
\pgfpathlineto{\pgfqpoint{0.777975in}{0.741057in}}%
\pgfpathlineto{\pgfqpoint{0.788526in}{0.734393in}}%
\pgfpathlineto{\pgfqpoint{0.799497in}{0.727904in}}%
\pgfpathlineto{\pgfqpoint{0.810876in}{0.721598in}}%
\pgfpathlineto{\pgfqpoint{0.822652in}{0.715481in}}%
\pgfpathclose%
\pgfusepath{fill}%
\end{pgfscope}%
\begin{pgfscope}%
\pgfpathrectangle{\pgfqpoint{0.041670in}{0.041670in}}{\pgfqpoint{2.216660in}{2.216660in}}%
\pgfusepath{clip}%
\pgfsetbuttcap%
\pgfsetroundjoin%
\definecolor{currentfill}{rgb}{0.282327,0.094955,0.417331}%
\pgfsetfillcolor{currentfill}%
\pgfsetlinewidth{0.000000pt}%
\definecolor{currentstroke}{rgb}{0.000000,0.000000,0.000000}%
\pgfsetstrokecolor{currentstroke}%
\pgfsetdash{}{0pt}%
\pgfpathmoveto{\pgfqpoint{0.833149in}{0.739039in}}%
\pgfpathlineto{\pgfqpoint{0.830529in}{0.732918in}}%
\pgfpathlineto{\pgfqpoint{0.827906in}{0.726948in}}%
\pgfpathlineto{\pgfqpoint{0.825281in}{0.721135in}}%
\pgfpathlineto{\pgfqpoint{0.822652in}{0.715481in}}%
\pgfpathlineto{\pgfqpoint{0.810876in}{0.721598in}}%
\pgfpathlineto{\pgfqpoint{0.799497in}{0.727904in}}%
\pgfpathlineto{\pgfqpoint{0.788526in}{0.734393in}}%
\pgfpathlineto{\pgfqpoint{0.777975in}{0.741057in}}%
\pgfpathlineto{\pgfqpoint{0.780921in}{0.746503in}}%
\pgfpathlineto{\pgfqpoint{0.783865in}{0.752110in}}%
\pgfpathlineto{\pgfqpoint{0.786806in}{0.757872in}}%
\pgfpathlineto{\pgfqpoint{0.789744in}{0.763787in}}%
\pgfpathlineto{\pgfqpoint{0.799997in}{0.757338in}}%
\pgfpathlineto{\pgfqpoint{0.810655in}{0.751059in}}%
\pgfpathlineto{\pgfqpoint{0.821710in}{0.744957in}}%
\pgfpathlineto{\pgfqpoint{0.833149in}{0.739039in}}%
\pgfpathclose%
\pgfusepath{fill}%
\end{pgfscope}%
\begin{pgfscope}%
\pgfpathrectangle{\pgfqpoint{0.041670in}{0.041670in}}{\pgfqpoint{2.216660in}{2.216660in}}%
\pgfusepath{clip}%
\pgfsetbuttcap%
\pgfsetroundjoin%
\definecolor{currentfill}{rgb}{0.166383,0.690856,0.496502}%
\pgfsetfillcolor{currentfill}%
\pgfsetlinewidth{0.000000pt}%
\definecolor{currentstroke}{rgb}{0.000000,0.000000,0.000000}%
\pgfsetstrokecolor{currentstroke}%
\pgfsetdash{}{0pt}%
\pgfpathmoveto{\pgfqpoint{1.439540in}{1.356600in}}%
\pgfpathlineto{\pgfqpoint{1.443279in}{1.349084in}}%
\pgfpathlineto{\pgfqpoint{1.447015in}{1.341519in}}%
\pgfpathlineto{\pgfqpoint{1.450748in}{1.333908in}}%
\pgfpathlineto{\pgfqpoint{1.454479in}{1.326252in}}%
\pgfpathlineto{\pgfqpoint{1.453047in}{1.322019in}}%
\pgfpathlineto{\pgfqpoint{1.451340in}{1.317808in}}%
\pgfpathlineto{\pgfqpoint{1.449359in}{1.313623in}}%
\pgfpathlineto{\pgfqpoint{1.447105in}{1.309468in}}%
\pgfpathlineto{\pgfqpoint{1.443466in}{1.317371in}}%
\pgfpathlineto{\pgfqpoint{1.439825in}{1.325229in}}%
\pgfpathlineto{\pgfqpoint{1.436180in}{1.333040in}}%
\pgfpathlineto{\pgfqpoint{1.432533in}{1.340802in}}%
\pgfpathlineto{\pgfqpoint{1.434673in}{1.344713in}}%
\pgfpathlineto{\pgfqpoint{1.436554in}{1.348652in}}%
\pgfpathlineto{\pgfqpoint{1.438177in}{1.352616in}}%
\pgfpathlineto{\pgfqpoint{1.439540in}{1.356600in}}%
\pgfpathclose%
\pgfusepath{fill}%
\end{pgfscope}%
\begin{pgfscope}%
\pgfpathrectangle{\pgfqpoint{0.041670in}{0.041670in}}{\pgfqpoint{2.216660in}{2.216660in}}%
\pgfusepath{clip}%
\pgfsetbuttcap%
\pgfsetroundjoin%
\definecolor{currentfill}{rgb}{0.212395,0.359683,0.551710}%
\pgfsetfillcolor{currentfill}%
\pgfsetlinewidth{0.000000pt}%
\definecolor{currentstroke}{rgb}{0.000000,0.000000,0.000000}%
\pgfsetstrokecolor{currentstroke}%
\pgfsetdash{}{0pt}%
\pgfpathmoveto{\pgfqpoint{0.871433in}{0.975570in}}%
\pgfpathlineto{\pgfqpoint{0.868522in}{0.966925in}}%
\pgfpathlineto{\pgfqpoint{0.865611in}{0.958331in}}%
\pgfpathlineto{\pgfqpoint{0.862700in}{0.949791in}}%
\pgfpathlineto{\pgfqpoint{0.859789in}{0.941307in}}%
\pgfpathlineto{\pgfqpoint{0.851666in}{0.946601in}}%
\pgfpathlineto{\pgfqpoint{0.843887in}{0.952020in}}%
\pgfpathlineto{\pgfqpoint{0.836461in}{0.957559in}}%
\pgfpathlineto{\pgfqpoint{0.829395in}{0.963212in}}%
\pgfpathlineto{\pgfqpoint{0.832573in}{0.971471in}}%
\pgfpathlineto{\pgfqpoint{0.835750in}{0.979788in}}%
\pgfpathlineto{\pgfqpoint{0.838928in}{0.988158in}}%
\pgfpathlineto{\pgfqpoint{0.842107in}{0.996580in}}%
\pgfpathlineto{\pgfqpoint{0.848926in}{0.991157in}}%
\pgfpathlineto{\pgfqpoint{0.856092in}{0.985844in}}%
\pgfpathlineto{\pgfqpoint{0.863596in}{0.980647in}}%
\pgfpathlineto{\pgfqpoint{0.871433in}{0.975570in}}%
\pgfpathclose%
\pgfusepath{fill}%
\end{pgfscope}%
\begin{pgfscope}%
\pgfpathrectangle{\pgfqpoint{0.041670in}{0.041670in}}{\pgfqpoint{2.216660in}{2.216660in}}%
\pgfusepath{clip}%
\pgfsetbuttcap%
\pgfsetroundjoin%
\definecolor{currentfill}{rgb}{0.281477,0.755203,0.432552}%
\pgfsetfillcolor{currentfill}%
\pgfsetlinewidth{0.000000pt}%
\definecolor{currentstroke}{rgb}{0.000000,0.000000,0.000000}%
\pgfsetstrokecolor{currentstroke}%
\pgfsetdash{}{0pt}%
\pgfpathmoveto{\pgfqpoint{1.412094in}{1.428785in}}%
\pgfpathlineto{\pgfqpoint{1.415890in}{1.421986in}}%
\pgfpathlineto{\pgfqpoint{1.419683in}{1.415121in}}%
\pgfpathlineto{\pgfqpoint{1.423473in}{1.408192in}}%
\pgfpathlineto{\pgfqpoint{1.427259in}{1.401201in}}%
\pgfpathlineto{\pgfqpoint{1.426954in}{1.397423in}}%
\pgfpathlineto{\pgfqpoint{1.426401in}{1.393649in}}%
\pgfpathlineto{\pgfqpoint{1.425601in}{1.389883in}}%
\pgfpathlineto{\pgfqpoint{1.424554in}{1.386130in}}%
\pgfpathlineto{\pgfqpoint{1.420801in}{1.393367in}}%
\pgfpathlineto{\pgfqpoint{1.417044in}{1.400543in}}%
\pgfpathlineto{\pgfqpoint{1.413285in}{1.407653in}}%
\pgfpathlineto{\pgfqpoint{1.409523in}{1.414697in}}%
\pgfpathlineto{\pgfqpoint{1.410513in}{1.418206in}}%
\pgfpathlineto{\pgfqpoint{1.411271in}{1.421725in}}%
\pgfpathlineto{\pgfqpoint{1.411798in}{1.425253in}}%
\pgfpathlineto{\pgfqpoint{1.412094in}{1.428785in}}%
\pgfpathclose%
\pgfusepath{fill}%
\end{pgfscope}%
\begin{pgfscope}%
\pgfpathrectangle{\pgfqpoint{0.041670in}{0.041670in}}{\pgfqpoint{2.216660in}{2.216660in}}%
\pgfusepath{clip}%
\pgfsetbuttcap%
\pgfsetroundjoin%
\definecolor{currentfill}{rgb}{0.855810,0.888601,0.097452}%
\pgfsetfillcolor{currentfill}%
\pgfsetlinewidth{0.000000pt}%
\definecolor{currentstroke}{rgb}{0.000000,0.000000,0.000000}%
\pgfsetstrokecolor{currentstroke}%
\pgfsetdash{}{0pt}%
\pgfpathmoveto{\pgfqpoint{1.156855in}{1.658042in}}%
\pgfpathlineto{\pgfqpoint{1.155902in}{1.656623in}}%
\pgfpathlineto{\pgfqpoint{1.154951in}{1.655083in}}%
\pgfpathlineto{\pgfqpoint{1.154000in}{1.653423in}}%
\pgfpathlineto{\pgfqpoint{1.153051in}{1.651645in}}%
\pgfpathlineto{\pgfqpoint{1.156380in}{1.652016in}}%
\pgfpathlineto{\pgfqpoint{1.159732in}{1.652338in}}%
\pgfpathlineto{\pgfqpoint{1.163105in}{1.652611in}}%
\pgfpathlineto{\pgfqpoint{1.166493in}{1.652834in}}%
\pgfpathlineto{\pgfqpoint{1.166968in}{1.654570in}}%
\pgfpathlineto{\pgfqpoint{1.167444in}{1.656188in}}%
\pgfpathlineto{\pgfqpoint{1.167920in}{1.657685in}}%
\pgfpathlineto{\pgfqpoint{1.168396in}{1.659061in}}%
\pgfpathlineto{\pgfqpoint{1.165487in}{1.658870in}}%
\pgfpathlineto{\pgfqpoint{1.162591in}{1.658636in}}%
\pgfpathlineto{\pgfqpoint{1.159713in}{1.658360in}}%
\pgfpathlineto{\pgfqpoint{1.156855in}{1.658042in}}%
\pgfpathclose%
\pgfusepath{fill}%
\end{pgfscope}%
\begin{pgfscope}%
\pgfpathrectangle{\pgfqpoint{0.041670in}{0.041670in}}{\pgfqpoint{2.216660in}{2.216660in}}%
\pgfusepath{clip}%
\pgfsetbuttcap%
\pgfsetroundjoin%
\definecolor{currentfill}{rgb}{0.274128,0.199721,0.498911}%
\pgfsetfillcolor{currentfill}%
\pgfsetlinewidth{0.000000pt}%
\definecolor{currentstroke}{rgb}{0.000000,0.000000,0.000000}%
\pgfsetstrokecolor{currentstroke}%
\pgfsetdash{}{0pt}%
\pgfpathmoveto{\pgfqpoint{1.543068in}{0.850457in}}%
\pgfpathlineto{\pgfqpoint{1.546048in}{0.843073in}}%
\pgfpathlineto{\pgfqpoint{1.549029in}{0.835795in}}%
\pgfpathlineto{\pgfqpoint{1.552012in}{0.828626in}}%
\pgfpathlineto{\pgfqpoint{1.554996in}{0.821571in}}%
\pgfpathlineto{\pgfqpoint{1.545686in}{0.815417in}}%
\pgfpathlineto{\pgfqpoint{1.535987in}{0.809417in}}%
\pgfpathlineto{\pgfqpoint{1.525910in}{0.803576in}}%
\pgfpathlineto{\pgfqpoint{1.515462in}{0.797902in}}%
\pgfpathlineto{\pgfqpoint{1.512785in}{0.805169in}}%
\pgfpathlineto{\pgfqpoint{1.510108in}{0.812549in}}%
\pgfpathlineto{\pgfqpoint{1.507433in}{0.820039in}}%
\pgfpathlineto{\pgfqpoint{1.504759in}{0.827634in}}%
\pgfpathlineto{\pgfqpoint{1.514881in}{0.833105in}}%
\pgfpathlineto{\pgfqpoint{1.524646in}{0.838737in}}%
\pgfpathlineto{\pgfqpoint{1.534045in}{0.844523in}}%
\pgfpathlineto{\pgfqpoint{1.543068in}{0.850457in}}%
\pgfpathclose%
\pgfusepath{fill}%
\end{pgfscope}%
\begin{pgfscope}%
\pgfpathrectangle{\pgfqpoint{0.041670in}{0.041670in}}{\pgfqpoint{2.216660in}{2.216660in}}%
\pgfusepath{clip}%
\pgfsetbuttcap%
\pgfsetroundjoin%
\definecolor{currentfill}{rgb}{0.855810,0.888601,0.097452}%
\pgfsetfillcolor{currentfill}%
\pgfsetlinewidth{0.000000pt}%
\definecolor{currentstroke}{rgb}{0.000000,0.000000,0.000000}%
\pgfsetstrokecolor{currentstroke}%
\pgfsetdash{}{0pt}%
\pgfpathmoveto{\pgfqpoint{1.191838in}{1.659042in}}%
\pgfpathlineto{\pgfqpoint{1.192327in}{1.657665in}}%
\pgfpathlineto{\pgfqpoint{1.192817in}{1.656167in}}%
\pgfpathlineto{\pgfqpoint{1.193306in}{1.654549in}}%
\pgfpathlineto{\pgfqpoint{1.193794in}{1.652811in}}%
\pgfpathlineto{\pgfqpoint{1.197181in}{1.652583in}}%
\pgfpathlineto{\pgfqpoint{1.200551in}{1.652305in}}%
\pgfpathlineto{\pgfqpoint{1.203901in}{1.651977in}}%
\pgfpathlineto{\pgfqpoint{1.207227in}{1.651600in}}%
\pgfpathlineto{\pgfqpoint{1.206265in}{1.653381in}}%
\pgfpathlineto{\pgfqpoint{1.205301in}{1.655042in}}%
\pgfpathlineto{\pgfqpoint{1.204337in}{1.656583in}}%
\pgfpathlineto{\pgfqpoint{1.203371in}{1.658004in}}%
\pgfpathlineto{\pgfqpoint{1.200515in}{1.658327in}}%
\pgfpathlineto{\pgfqpoint{1.197639in}{1.658608in}}%
\pgfpathlineto{\pgfqpoint{1.194746in}{1.658846in}}%
\pgfpathlineto{\pgfqpoint{1.191838in}{1.659042in}}%
\pgfpathclose%
\pgfusepath{fill}%
\end{pgfscope}%
\begin{pgfscope}%
\pgfpathrectangle{\pgfqpoint{0.041670in}{0.041670in}}{\pgfqpoint{2.216660in}{2.216660in}}%
\pgfusepath{clip}%
\pgfsetbuttcap%
\pgfsetroundjoin%
\definecolor{currentfill}{rgb}{0.814576,0.883393,0.110347}%
\pgfsetfillcolor{currentfill}%
\pgfsetlinewidth{0.000000pt}%
\definecolor{currentstroke}{rgb}{0.000000,0.000000,0.000000}%
\pgfsetstrokecolor{currentstroke}%
\pgfsetdash{}{0pt}%
\pgfpathmoveto{\pgfqpoint{1.105455in}{1.639444in}}%
\pgfpathlineto{\pgfqpoint{1.102823in}{1.637113in}}%
\pgfpathlineto{\pgfqpoint{1.100194in}{1.634665in}}%
\pgfpathlineto{\pgfqpoint{1.097567in}{1.632102in}}%
\pgfpathlineto{\pgfqpoint{1.094942in}{1.629424in}}%
\pgfpathlineto{\pgfqpoint{1.097847in}{1.630657in}}%
\pgfpathlineto{\pgfqpoint{1.100831in}{1.631847in}}%
\pgfpathlineto{\pgfqpoint{1.103893in}{1.632992in}}%
\pgfpathlineto{\pgfqpoint{1.107029in}{1.634091in}}%
\pgfpathlineto{\pgfqpoint{1.109278in}{1.636623in}}%
\pgfpathlineto{\pgfqpoint{1.111530in}{1.639041in}}%
\pgfpathlineto{\pgfqpoint{1.113784in}{1.641343in}}%
\pgfpathlineto{\pgfqpoint{1.116041in}{1.643528in}}%
\pgfpathlineto{\pgfqpoint{1.113294in}{1.642567in}}%
\pgfpathlineto{\pgfqpoint{1.110612in}{1.641565in}}%
\pgfpathlineto{\pgfqpoint{1.107998in}{1.640524in}}%
\pgfpathlineto{\pgfqpoint{1.105455in}{1.639444in}}%
\pgfpathclose%
\pgfusepath{fill}%
\end{pgfscope}%
\begin{pgfscope}%
\pgfpathrectangle{\pgfqpoint{0.041670in}{0.041670in}}{\pgfqpoint{2.216660in}{2.216660in}}%
\pgfusepath{clip}%
\pgfsetbuttcap%
\pgfsetroundjoin%
\definecolor{currentfill}{rgb}{0.274952,0.037752,0.364543}%
\pgfsetfillcolor{currentfill}%
\pgfsetlinewidth{0.000000pt}%
\definecolor{currentstroke}{rgb}{0.000000,0.000000,0.000000}%
\pgfsetstrokecolor{currentstroke}%
\pgfsetdash{}{0pt}%
\pgfpathmoveto{\pgfqpoint{0.812109in}{0.694552in}}%
\pgfpathlineto{\pgfqpoint{0.809464in}{0.689761in}}%
\pgfpathlineto{\pgfqpoint{0.806816in}{0.685156in}}%
\pgfpathlineto{\pgfqpoint{0.804165in}{0.680741in}}%
\pgfpathlineto{\pgfqpoint{0.801509in}{0.676520in}}%
\pgfpathlineto{\pgfqpoint{0.789054in}{0.683032in}}%
\pgfpathlineto{\pgfqpoint{0.777023in}{0.689746in}}%
\pgfpathlineto{\pgfqpoint{0.765426in}{0.696653in}}%
\pgfpathlineto{\pgfqpoint{0.754275in}{0.703746in}}%
\pgfpathlineto{\pgfqpoint{0.757251in}{0.707761in}}%
\pgfpathlineto{\pgfqpoint{0.760223in}{0.711971in}}%
\pgfpathlineto{\pgfqpoint{0.763191in}{0.716369in}}%
\pgfpathlineto{\pgfqpoint{0.766155in}{0.720954in}}%
\pgfpathlineto{\pgfqpoint{0.777006in}{0.714075in}}%
\pgfpathlineto{\pgfqpoint{0.788289in}{0.707377in}}%
\pgfpathlineto{\pgfqpoint{0.799994in}{0.700866in}}%
\pgfpathlineto{\pgfqpoint{0.812109in}{0.694552in}}%
\pgfpathclose%
\pgfusepath{fill}%
\end{pgfscope}%
\begin{pgfscope}%
\pgfpathrectangle{\pgfqpoint{0.041670in}{0.041670in}}{\pgfqpoint{2.216660in}{2.216660in}}%
\pgfusepath{clip}%
\pgfsetbuttcap%
\pgfsetroundjoin%
\definecolor{currentfill}{rgb}{0.195860,0.395433,0.555276}%
\pgfsetfillcolor{currentfill}%
\pgfsetlinewidth{0.000000pt}%
\definecolor{currentstroke}{rgb}{0.000000,0.000000,0.000000}%
\pgfsetstrokecolor{currentstroke}%
\pgfsetdash{}{0pt}%
\pgfpathmoveto{\pgfqpoint{1.510639in}{1.035405in}}%
\pgfpathlineto{\pgfqpoint{1.513873in}{1.026865in}}%
\pgfpathlineto{\pgfqpoint{1.517106in}{1.018364in}}%
\pgfpathlineto{\pgfqpoint{1.520338in}{1.009903in}}%
\pgfpathlineto{\pgfqpoint{1.523569in}{1.001486in}}%
\pgfpathlineto{\pgfqpoint{1.517062in}{0.995972in}}%
\pgfpathlineto{\pgfqpoint{1.510204in}{0.990562in}}%
\pgfpathlineto{\pgfqpoint{1.503001in}{0.985261in}}%
\pgfpathlineto{\pgfqpoint{1.495459in}{0.980077in}}%
\pgfpathlineto{\pgfqpoint{1.492483in}{0.988720in}}%
\pgfpathlineto{\pgfqpoint{1.489507in}{0.997408in}}%
\pgfpathlineto{\pgfqpoint{1.486530in}{1.006136in}}%
\pgfpathlineto{\pgfqpoint{1.483552in}{1.014902in}}%
\pgfpathlineto{\pgfqpoint{1.490818in}{1.019866in}}%
\pgfpathlineto{\pgfqpoint{1.497759in}{1.024942in}}%
\pgfpathlineto{\pgfqpoint{1.504368in}{1.030124in}}%
\pgfpathlineto{\pgfqpoint{1.510639in}{1.035405in}}%
\pgfpathclose%
\pgfusepath{fill}%
\end{pgfscope}%
\begin{pgfscope}%
\pgfpathrectangle{\pgfqpoint{0.041670in}{0.041670in}}{\pgfqpoint{2.216660in}{2.216660in}}%
\pgfusepath{clip}%
\pgfsetbuttcap%
\pgfsetroundjoin%
\definecolor{currentfill}{rgb}{0.283072,0.130895,0.449241}%
\pgfsetfillcolor{currentfill}%
\pgfsetlinewidth{0.000000pt}%
\definecolor{currentstroke}{rgb}{0.000000,0.000000,0.000000}%
\pgfsetstrokecolor{currentstroke}%
\pgfsetdash{}{0pt}%
\pgfpathmoveto{\pgfqpoint{0.843608in}{0.764964in}}%
\pgfpathlineto{\pgfqpoint{0.840996in}{0.758275in}}%
\pgfpathlineto{\pgfqpoint{0.838383in}{0.751722in}}%
\pgfpathlineto{\pgfqpoint{0.835767in}{0.745309in}}%
\pgfpathlineto{\pgfqpoint{0.833149in}{0.739039in}}%
\pgfpathlineto{\pgfqpoint{0.821710in}{0.744957in}}%
\pgfpathlineto{\pgfqpoint{0.810655in}{0.751059in}}%
\pgfpathlineto{\pgfqpoint{0.799997in}{0.757338in}}%
\pgfpathlineto{\pgfqpoint{0.789744in}{0.763787in}}%
\pgfpathlineto{\pgfqpoint{0.792680in}{0.769849in}}%
\pgfpathlineto{\pgfqpoint{0.795613in}{0.776055in}}%
\pgfpathlineto{\pgfqpoint{0.798544in}{0.782401in}}%
\pgfpathlineto{\pgfqpoint{0.801473in}{0.788883in}}%
\pgfpathlineto{\pgfqpoint{0.811427in}{0.782650in}}%
\pgfpathlineto{\pgfqpoint{0.821774in}{0.776581in}}%
\pgfpathlineto{\pgfqpoint{0.832505in}{0.770684in}}%
\pgfpathlineto{\pgfqpoint{0.843608in}{0.764964in}}%
\pgfpathclose%
\pgfusepath{fill}%
\end{pgfscope}%
\begin{pgfscope}%
\pgfpathrectangle{\pgfqpoint{0.041670in}{0.041670in}}{\pgfqpoint{2.216660in}{2.216660in}}%
\pgfusepath{clip}%
\pgfsetbuttcap%
\pgfsetroundjoin%
\definecolor{currentfill}{rgb}{0.762373,0.876424,0.137064}%
\pgfsetfillcolor{currentfill}%
\pgfsetlinewidth{0.000000pt}%
\definecolor{currentstroke}{rgb}{0.000000,0.000000,0.000000}%
\pgfsetstrokecolor{currentstroke}%
\pgfsetdash{}{0pt}%
\pgfpathmoveto{\pgfqpoint{1.074925in}{1.618134in}}%
\pgfpathlineto{\pgfqpoint{1.071680in}{1.614989in}}%
\pgfpathlineto{\pgfqpoint{1.068438in}{1.611731in}}%
\pgfpathlineto{\pgfqpoint{1.065198in}{1.608363in}}%
\pgfpathlineto{\pgfqpoint{1.061962in}{1.604885in}}%
\pgfpathlineto{\pgfqpoint{1.064398in}{1.606615in}}%
\pgfpathlineto{\pgfqpoint{1.066948in}{1.608308in}}%
\pgfpathlineto{\pgfqpoint{1.069608in}{1.609962in}}%
\pgfpathlineto{\pgfqpoint{1.072376in}{1.611575in}}%
\pgfpathlineto{\pgfqpoint{1.075324in}{1.614866in}}%
\pgfpathlineto{\pgfqpoint{1.078274in}{1.618048in}}%
\pgfpathlineto{\pgfqpoint{1.081228in}{1.621119in}}%
\pgfpathlineto{\pgfqpoint{1.084184in}{1.624077in}}%
\pgfpathlineto{\pgfqpoint{1.081722in}{1.622644in}}%
\pgfpathlineto{\pgfqpoint{1.079357in}{1.621175in}}%
\pgfpathlineto{\pgfqpoint{1.077090in}{1.619671in}}%
\pgfpathlineto{\pgfqpoint{1.074925in}{1.618134in}}%
\pgfpathclose%
\pgfusepath{fill}%
\end{pgfscope}%
\begin{pgfscope}%
\pgfpathrectangle{\pgfqpoint{0.041670in}{0.041670in}}{\pgfqpoint{2.216660in}{2.216660in}}%
\pgfusepath{clip}%
\pgfsetbuttcap%
\pgfsetroundjoin%
\definecolor{currentfill}{rgb}{0.487026,0.823929,0.312321}%
\pgfsetfillcolor{currentfill}%
\pgfsetlinewidth{0.000000pt}%
\definecolor{currentstroke}{rgb}{0.000000,0.000000,0.000000}%
\pgfsetstrokecolor{currentstroke}%
\pgfsetdash{}{0pt}%
\pgfpathmoveto{\pgfqpoint{0.993999in}{1.513186in}}%
\pgfpathlineto{\pgfqpoint{0.990180in}{1.507523in}}%
\pgfpathlineto{\pgfqpoint{0.986364in}{1.501772in}}%
\pgfpathlineto{\pgfqpoint{0.982551in}{1.495934in}}%
\pgfpathlineto{\pgfqpoint{0.978741in}{1.490009in}}%
\pgfpathlineto{\pgfqpoint{0.979296in}{1.493039in}}%
\pgfpathlineto{\pgfqpoint{0.980050in}{1.496057in}}%
\pgfpathlineto{\pgfqpoint{0.981002in}{1.499061in}}%
\pgfpathlineto{\pgfqpoint{0.982151in}{1.502048in}}%
\pgfpathlineto{\pgfqpoint{0.985890in}{1.507734in}}%
\pgfpathlineto{\pgfqpoint{0.989632in}{1.513335in}}%
\pgfpathlineto{\pgfqpoint{0.993377in}{1.518849in}}%
\pgfpathlineto{\pgfqpoint{0.997125in}{1.524274in}}%
\pgfpathlineto{\pgfqpoint{0.996070in}{1.521523in}}%
\pgfpathlineto{\pgfqpoint{0.995196in}{1.518756in}}%
\pgfpathlineto{\pgfqpoint{0.994506in}{1.515976in}}%
\pgfpathlineto{\pgfqpoint{0.993999in}{1.513186in}}%
\pgfpathclose%
\pgfusepath{fill}%
\end{pgfscope}%
\begin{pgfscope}%
\pgfpathrectangle{\pgfqpoint{0.041670in}{0.041670in}}{\pgfqpoint{2.216660in}{2.216660in}}%
\pgfusepath{clip}%
\pgfsetbuttcap%
\pgfsetroundjoin%
\definecolor{currentfill}{rgb}{0.855810,0.888601,0.097452}%
\pgfsetfillcolor{currentfill}%
\pgfsetlinewidth{0.000000pt}%
\definecolor{currentstroke}{rgb}{0.000000,0.000000,0.000000}%
\pgfsetstrokecolor{currentstroke}%
\pgfsetdash{}{0pt}%
\pgfpathmoveto{\pgfqpoint{1.203371in}{1.658004in}}%
\pgfpathlineto{\pgfqpoint{1.204337in}{1.656583in}}%
\pgfpathlineto{\pgfqpoint{1.205301in}{1.655042in}}%
\pgfpathlineto{\pgfqpoint{1.206265in}{1.653381in}}%
\pgfpathlineto{\pgfqpoint{1.207227in}{1.651600in}}%
\pgfpathlineto{\pgfqpoint{1.210528in}{1.651175in}}%
\pgfpathlineto{\pgfqpoint{1.213798in}{1.650701in}}%
\pgfpathlineto{\pgfqpoint{1.217035in}{1.650179in}}%
\pgfpathlineto{\pgfqpoint{1.215726in}{1.652010in}}%
\pgfpathlineto{\pgfqpoint{1.214416in}{1.653722in}}%
\pgfpathlineto{\pgfqpoint{1.213104in}{1.655314in}}%
\pgfpathlineto{\pgfqpoint{1.211792in}{1.656785in}}%
\pgfpathlineto{\pgfqpoint{1.209012in}{1.657232in}}%
\pgfpathlineto{\pgfqpoint{1.206205in}{1.657639in}}%
\pgfpathlineto{\pgfqpoint{1.203371in}{1.658004in}}%
\pgfpathclose%
\pgfusepath{fill}%
\end{pgfscope}%
\begin{pgfscope}%
\pgfpathrectangle{\pgfqpoint{0.041670in}{0.041670in}}{\pgfqpoint{2.216660in}{2.216660in}}%
\pgfusepath{clip}%
\pgfsetbuttcap%
\pgfsetroundjoin%
\definecolor{currentfill}{rgb}{0.277941,0.056324,0.381191}%
\pgfsetfillcolor{currentfill}%
\pgfsetlinewidth{0.000000pt}%
\definecolor{currentstroke}{rgb}{0.000000,0.000000,0.000000}%
\pgfsetstrokecolor{currentstroke}%
\pgfsetdash{}{0pt}%
\pgfpathmoveto{\pgfqpoint{1.734395in}{0.721262in}}%
\pgfpathlineto{\pgfqpoint{1.737878in}{0.723908in}}%
\pgfpathlineto{\pgfqpoint{1.741372in}{0.726878in}}%
\pgfpathlineto{\pgfqpoint{1.744878in}{0.730177in}}%
\pgfpathlineto{\pgfqpoint{1.748396in}{0.733811in}}%
\pgfpathlineto{\pgfqpoint{1.737916in}{0.724339in}}%
\pgfpathlineto{\pgfqpoint{1.726838in}{0.715037in}}%
\pgfpathlineto{\pgfqpoint{1.715171in}{0.705915in}}%
\pgfpathlineto{\pgfqpoint{1.702926in}{0.696984in}}%
\pgfpathlineto{\pgfqpoint{1.699674in}{0.693561in}}%
\pgfpathlineto{\pgfqpoint{1.696432in}{0.690474in}}%
\pgfpathlineto{\pgfqpoint{1.693202in}{0.687717in}}%
\pgfpathlineto{\pgfqpoint{1.689982in}{0.685285in}}%
\pgfpathlineto{\pgfqpoint{1.701939in}{0.694009in}}%
\pgfpathlineto{\pgfqpoint{1.713334in}{0.702920in}}%
\pgfpathlineto{\pgfqpoint{1.724155in}{0.712008in}}%
\pgfpathlineto{\pgfqpoint{1.734395in}{0.721262in}}%
\pgfpathclose%
\pgfusepath{fill}%
\end{pgfscope}%
\begin{pgfscope}%
\pgfpathrectangle{\pgfqpoint{0.041670in}{0.041670in}}{\pgfqpoint{2.216660in}{2.216660in}}%
\pgfusepath{clip}%
\pgfsetbuttcap%
\pgfsetroundjoin%
\definecolor{currentfill}{rgb}{0.855810,0.888601,0.097452}%
\pgfsetfillcolor{currentfill}%
\pgfsetlinewidth{0.000000pt}%
\definecolor{currentstroke}{rgb}{0.000000,0.000000,0.000000}%
\pgfsetstrokecolor{currentstroke}%
\pgfsetdash{}{0pt}%
\pgfpathmoveto{\pgfqpoint{1.145674in}{1.656353in}}%
\pgfpathlineto{\pgfqpoint{1.144260in}{1.654864in}}%
\pgfpathlineto{\pgfqpoint{1.142848in}{1.653254in}}%
\pgfpathlineto{\pgfqpoint{1.141437in}{1.651524in}}%
\pgfpathlineto{\pgfqpoint{1.140027in}{1.649676in}}%
\pgfpathlineto{\pgfqpoint{1.143232in}{1.650240in}}%
\pgfpathlineto{\pgfqpoint{1.146474in}{1.650756in}}%
\pgfpathlineto{\pgfqpoint{1.149747in}{1.651225in}}%
\pgfpathlineto{\pgfqpoint{1.153051in}{1.651645in}}%
\pgfpathlineto{\pgfqpoint{1.154000in}{1.653423in}}%
\pgfpathlineto{\pgfqpoint{1.154951in}{1.655083in}}%
\pgfpathlineto{\pgfqpoint{1.155902in}{1.656623in}}%
\pgfpathlineto{\pgfqpoint{1.156855in}{1.658042in}}%
\pgfpathlineto{\pgfqpoint{1.154019in}{1.657681in}}%
\pgfpathlineto{\pgfqpoint{1.151208in}{1.657279in}}%
\pgfpathlineto{\pgfqpoint{1.148425in}{1.656836in}}%
\pgfpathlineto{\pgfqpoint{1.145674in}{1.656353in}}%
\pgfpathclose%
\pgfusepath{fill}%
\end{pgfscope}%
\begin{pgfscope}%
\pgfpathrectangle{\pgfqpoint{0.041670in}{0.041670in}}{\pgfqpoint{2.216660in}{2.216660in}}%
\pgfusepath{clip}%
\pgfsetbuttcap%
\pgfsetroundjoin%
\definecolor{currentfill}{rgb}{0.120081,0.622161,0.534946}%
\pgfsetfillcolor{currentfill}%
\pgfsetlinewidth{0.000000pt}%
\definecolor{currentstroke}{rgb}{0.000000,0.000000,0.000000}%
\pgfsetstrokecolor{currentstroke}%
\pgfsetdash{}{0pt}%
\pgfpathmoveto{\pgfqpoint{1.461636in}{1.277459in}}%
\pgfpathlineto{\pgfqpoint{1.465262in}{1.269370in}}%
\pgfpathlineto{\pgfqpoint{1.468886in}{1.261251in}}%
\pgfpathlineto{\pgfqpoint{1.472507in}{1.253105in}}%
\pgfpathlineto{\pgfqpoint{1.476126in}{1.244934in}}%
\pgfpathlineto{\pgfqpoint{1.473347in}{1.240322in}}%
\pgfpathlineto{\pgfqpoint{1.470269in}{1.235753in}}%
\pgfpathlineto{\pgfqpoint{1.466896in}{1.231231in}}%
\pgfpathlineto{\pgfqpoint{1.463229in}{1.226762in}}%
\pgfpathlineto{\pgfqpoint{1.459759in}{1.235176in}}%
\pgfpathlineto{\pgfqpoint{1.456286in}{1.243566in}}%
\pgfpathlineto{\pgfqpoint{1.452811in}{1.251928in}}%
\pgfpathlineto{\pgfqpoint{1.449335in}{1.260260in}}%
\pgfpathlineto{\pgfqpoint{1.452830in}{1.264490in}}%
\pgfpathlineto{\pgfqpoint{1.456047in}{1.268769in}}%
\pgfpathlineto{\pgfqpoint{1.458983in}{1.273094in}}%
\pgfpathlineto{\pgfqpoint{1.461636in}{1.277459in}}%
\pgfpathclose%
\pgfusepath{fill}%
\end{pgfscope}%
\begin{pgfscope}%
\pgfpathrectangle{\pgfqpoint{0.041670in}{0.041670in}}{\pgfqpoint{2.216660in}{2.216660in}}%
\pgfusepath{clip}%
\pgfsetbuttcap%
\pgfsetroundjoin%
\definecolor{currentfill}{rgb}{0.271305,0.019942,0.347269}%
\pgfsetfillcolor{currentfill}%
\pgfsetlinewidth{0.000000pt}%
\definecolor{currentstroke}{rgb}{0.000000,0.000000,0.000000}%
\pgfsetstrokecolor{currentstroke}%
\pgfsetdash{}{0pt}%
\pgfpathmoveto{\pgfqpoint{0.801509in}{0.676520in}}%
\pgfpathlineto{\pgfqpoint{0.798849in}{0.672497in}}%
\pgfpathlineto{\pgfqpoint{0.796185in}{0.668678in}}%
\pgfpathlineto{\pgfqpoint{0.793516in}{0.665066in}}%
\pgfpathlineto{\pgfqpoint{0.790843in}{0.661666in}}%
\pgfpathlineto{\pgfqpoint{0.778047in}{0.668375in}}%
\pgfpathlineto{\pgfqpoint{0.765687in}{0.675292in}}%
\pgfpathlineto{\pgfqpoint{0.753775in}{0.682407in}}%
\pgfpathlineto{\pgfqpoint{0.742322in}{0.689713in}}%
\pgfpathlineto{\pgfqpoint{0.745318in}{0.692908in}}%
\pgfpathlineto{\pgfqpoint{0.748308in}{0.696315in}}%
\pgfpathlineto{\pgfqpoint{0.751294in}{0.699929in}}%
\pgfpathlineto{\pgfqpoint{0.754275in}{0.703746in}}%
\pgfpathlineto{\pgfqpoint{0.765426in}{0.696653in}}%
\pgfpathlineto{\pgfqpoint{0.777023in}{0.689746in}}%
\pgfpathlineto{\pgfqpoint{0.789054in}{0.683032in}}%
\pgfpathlineto{\pgfqpoint{0.801509in}{0.676520in}}%
\pgfpathclose%
\pgfusepath{fill}%
\end{pgfscope}%
\begin{pgfscope}%
\pgfpathrectangle{\pgfqpoint{0.041670in}{0.041670in}}{\pgfqpoint{2.216660in}{2.216660in}}%
\pgfusepath{clip}%
\pgfsetbuttcap%
\pgfsetroundjoin%
\definecolor{currentfill}{rgb}{0.412913,0.803041,0.357269}%
\pgfsetfillcolor{currentfill}%
\pgfsetlinewidth{0.000000pt}%
\definecolor{currentstroke}{rgb}{0.000000,0.000000,0.000000}%
\pgfsetstrokecolor{currentstroke}%
\pgfsetdash{}{0pt}%
\pgfpathmoveto{\pgfqpoint{1.380685in}{1.492703in}}%
\pgfpathlineto{\pgfqpoint{1.384481in}{1.486748in}}%
\pgfpathlineto{\pgfqpoint{1.388274in}{1.480712in}}%
\pgfpathlineto{\pgfqpoint{1.392063in}{1.474596in}}%
\pgfpathlineto{\pgfqpoint{1.395850in}{1.468402in}}%
\pgfpathlineto{\pgfqpoint{1.396430in}{1.465130in}}%
\pgfpathlineto{\pgfqpoint{1.396795in}{1.461850in}}%
\pgfpathlineto{\pgfqpoint{1.396945in}{1.458564in}}%
\pgfpathlineto{\pgfqpoint{1.396878in}{1.455276in}}%
\pgfpathlineto{\pgfqpoint{1.393066in}{1.461712in}}%
\pgfpathlineto{\pgfqpoint{1.389252in}{1.468071in}}%
\pgfpathlineto{\pgfqpoint{1.385434in}{1.474349in}}%
\pgfpathlineto{\pgfqpoint{1.381613in}{1.480545in}}%
\pgfpathlineto{\pgfqpoint{1.381682in}{1.483590in}}%
\pgfpathlineto{\pgfqpoint{1.381550in}{1.486634in}}%
\pgfpathlineto{\pgfqpoint{1.381218in}{1.489672in}}%
\pgfpathlineto{\pgfqpoint{1.380685in}{1.492703in}}%
\pgfpathclose%
\pgfusepath{fill}%
\end{pgfscope}%
\begin{pgfscope}%
\pgfpathrectangle{\pgfqpoint{0.041670in}{0.041670in}}{\pgfqpoint{2.216660in}{2.216660in}}%
\pgfusepath{clip}%
\pgfsetbuttcap%
\pgfsetroundjoin%
\definecolor{currentfill}{rgb}{0.280255,0.165693,0.476498}%
\pgfsetfillcolor{currentfill}%
\pgfsetlinewidth{0.000000pt}%
\definecolor{currentstroke}{rgb}{0.000000,0.000000,0.000000}%
\pgfsetstrokecolor{currentstroke}%
\pgfsetdash{}{0pt}%
\pgfpathmoveto{\pgfqpoint{0.854035in}{0.793003in}}%
\pgfpathlineto{\pgfqpoint{0.851431in}{0.785808in}}%
\pgfpathlineto{\pgfqpoint{0.848825in}{0.778735in}}%
\pgfpathlineto{\pgfqpoint{0.846217in}{0.771785in}}%
\pgfpathlineto{\pgfqpoint{0.843608in}{0.764964in}}%
\pgfpathlineto{\pgfqpoint{0.832505in}{0.770684in}}%
\pgfpathlineto{\pgfqpoint{0.821774in}{0.776581in}}%
\pgfpathlineto{\pgfqpoint{0.811427in}{0.782650in}}%
\pgfpathlineto{\pgfqpoint{0.801473in}{0.788883in}}%
\pgfpathlineto{\pgfqpoint{0.804400in}{0.795497in}}%
\pgfpathlineto{\pgfqpoint{0.807325in}{0.802239in}}%
\pgfpathlineto{\pgfqpoint{0.810248in}{0.809106in}}%
\pgfpathlineto{\pgfqpoint{0.813170in}{0.816094in}}%
\pgfpathlineto{\pgfqpoint{0.822826in}{0.810076in}}%
\pgfpathlineto{\pgfqpoint{0.832862in}{0.804217in}}%
\pgfpathlineto{\pgfqpoint{0.843269in}{0.798524in}}%
\pgfpathlineto{\pgfqpoint{0.854035in}{0.793003in}}%
\pgfpathclose%
\pgfusepath{fill}%
\end{pgfscope}%
\begin{pgfscope}%
\pgfpathrectangle{\pgfqpoint{0.041670in}{0.041670in}}{\pgfqpoint{2.216660in}{2.216660in}}%
\pgfusepath{clip}%
\pgfsetbuttcap%
\pgfsetroundjoin%
\definecolor{currentfill}{rgb}{0.276194,0.190074,0.493001}%
\pgfsetfillcolor{currentfill}%
\pgfsetlinewidth{0.000000pt}%
\definecolor{currentstroke}{rgb}{0.000000,0.000000,0.000000}%
\pgfsetstrokecolor{currentstroke}%
\pgfsetdash{}{0pt}%
\pgfpathmoveto{\pgfqpoint{1.814408in}{0.816772in}}%
\pgfpathlineto{\pgfqpoint{1.818254in}{0.823908in}}%
\pgfpathlineto{\pgfqpoint{1.822117in}{0.831441in}}%
\pgfpathlineto{\pgfqpoint{1.825996in}{0.839377in}}%
\pgfpathlineto{\pgfqpoint{1.829892in}{0.847724in}}%
\pgfpathlineto{\pgfqpoint{1.821348in}{0.837033in}}%
\pgfpathlineto{\pgfqpoint{1.812123in}{0.826473in}}%
\pgfpathlineto{\pgfqpoint{1.802226in}{0.816057in}}%
\pgfpathlineto{\pgfqpoint{1.791664in}{0.805796in}}%
\pgfpathlineto{\pgfqpoint{1.787975in}{0.797654in}}%
\pgfpathlineto{\pgfqpoint{1.784304in}{0.789924in}}%
\pgfpathlineto{\pgfqpoint{1.780648in}{0.782600in}}%
\pgfpathlineto{\pgfqpoint{1.777007in}{0.775673in}}%
\pgfpathlineto{\pgfqpoint{1.787337in}{0.785730in}}%
\pgfpathlineto{\pgfqpoint{1.797019in}{0.795940in}}%
\pgfpathlineto{\pgfqpoint{1.806044in}{0.806291in}}%
\pgfpathlineto{\pgfqpoint{1.814408in}{0.816772in}}%
\pgfpathclose%
\pgfusepath{fill}%
\end{pgfscope}%
\begin{pgfscope}%
\pgfpathrectangle{\pgfqpoint{0.041670in}{0.041670in}}{\pgfqpoint{2.216660in}{2.216660in}}%
\pgfusepath{clip}%
\pgfsetbuttcap%
\pgfsetroundjoin%
\definecolor{currentfill}{rgb}{0.263663,0.237631,0.518762}%
\pgfsetfillcolor{currentfill}%
\pgfsetlinewidth{0.000000pt}%
\definecolor{currentstroke}{rgb}{0.000000,0.000000,0.000000}%
\pgfsetstrokecolor{currentstroke}%
\pgfsetdash{}{0pt}%
\pgfpathmoveto{\pgfqpoint{1.531157in}{0.880976in}}%
\pgfpathlineto{\pgfqpoint{1.534134in}{0.873206in}}%
\pgfpathlineto{\pgfqpoint{1.537111in}{0.865527in}}%
\pgfpathlineto{\pgfqpoint{1.540089in}{0.857942in}}%
\pgfpathlineto{\pgfqpoint{1.543068in}{0.850457in}}%
\pgfpathlineto{\pgfqpoint{1.534045in}{0.844523in}}%
\pgfpathlineto{\pgfqpoint{1.524646in}{0.838737in}}%
\pgfpathlineto{\pgfqpoint{1.514881in}{0.833105in}}%
\pgfpathlineto{\pgfqpoint{1.504759in}{0.827634in}}%
\pgfpathlineto{\pgfqpoint{1.502086in}{0.835331in}}%
\pgfpathlineto{\pgfqpoint{1.499414in}{0.843126in}}%
\pgfpathlineto{\pgfqpoint{1.496743in}{0.851016in}}%
\pgfpathlineto{\pgfqpoint{1.494072in}{0.858997in}}%
\pgfpathlineto{\pgfqpoint{1.503869in}{0.864266in}}%
\pgfpathlineto{\pgfqpoint{1.513322in}{0.869689in}}%
\pgfpathlineto{\pgfqpoint{1.522421in}{0.875262in}}%
\pgfpathlineto{\pgfqpoint{1.531157in}{0.880976in}}%
\pgfpathclose%
\pgfusepath{fill}%
\end{pgfscope}%
\begin{pgfscope}%
\pgfpathrectangle{\pgfqpoint{0.041670in}{0.041670in}}{\pgfqpoint{2.216660in}{2.216660in}}%
\pgfusepath{clip}%
\pgfsetbuttcap%
\pgfsetroundjoin%
\definecolor{currentfill}{rgb}{0.267004,0.004874,0.329415}%
\pgfsetfillcolor{currentfill}%
\pgfsetlinewidth{0.000000pt}%
\definecolor{currentstroke}{rgb}{0.000000,0.000000,0.000000}%
\pgfsetstrokecolor{currentstroke}%
\pgfsetdash{}{0pt}%
\pgfpathmoveto{\pgfqpoint{1.639667in}{0.685983in}}%
\pgfpathlineto{\pgfqpoint{1.642756in}{0.683964in}}%
\pgfpathlineto{\pgfqpoint{1.645851in}{0.682185in}}%
\pgfpathlineto{\pgfqpoint{1.648952in}{0.680651in}}%
\pgfpathlineto{\pgfqpoint{1.652060in}{0.679367in}}%
\pgfpathlineto{\pgfqpoint{1.640442in}{0.671468in}}%
\pgfpathlineto{\pgfqpoint{1.628326in}{0.663762in}}%
\pgfpathlineto{\pgfqpoint{1.615725in}{0.656258in}}%
\pgfpathlineto{\pgfqpoint{1.602651in}{0.648964in}}%
\pgfpathlineto{\pgfqpoint{1.599857in}{0.650455in}}%
\pgfpathlineto{\pgfqpoint{1.597069in}{0.652196in}}%
\pgfpathlineto{\pgfqpoint{1.594287in}{0.654181in}}%
\pgfpathlineto{\pgfqpoint{1.591511in}{0.656408in}}%
\pgfpathlineto{\pgfqpoint{1.604251in}{0.663502in}}%
\pgfpathlineto{\pgfqpoint{1.616532in}{0.670802in}}%
\pgfpathlineto{\pgfqpoint{1.628341in}{0.678298in}}%
\pgfpathlineto{\pgfqpoint{1.639667in}{0.685983in}}%
\pgfpathclose%
\pgfusepath{fill}%
\end{pgfscope}%
\begin{pgfscope}%
\pgfpathrectangle{\pgfqpoint{0.041670in}{0.041670in}}{\pgfqpoint{2.216660in}{2.216660in}}%
\pgfusepath{clip}%
\pgfsetbuttcap%
\pgfsetroundjoin%
\definecolor{currentfill}{rgb}{0.281477,0.755203,0.432552}%
\pgfsetfillcolor{currentfill}%
\pgfsetlinewidth{0.000000pt}%
\definecolor{currentstroke}{rgb}{0.000000,0.000000,0.000000}%
\pgfsetstrokecolor{currentstroke}%
\pgfsetdash{}{0pt}%
\pgfpathmoveto{\pgfqpoint{0.951461in}{1.411592in}}%
\pgfpathlineto{\pgfqpoint{0.947714in}{1.404493in}}%
\pgfpathlineto{\pgfqpoint{0.943971in}{1.397328in}}%
\pgfpathlineto{\pgfqpoint{0.940230in}{1.390099in}}%
\pgfpathlineto{\pgfqpoint{0.936492in}{1.382806in}}%
\pgfpathlineto{\pgfqpoint{0.935227in}{1.386546in}}%
\pgfpathlineto{\pgfqpoint{0.934208in}{1.390301in}}%
\pgfpathlineto{\pgfqpoint{0.933436in}{1.394068in}}%
\pgfpathlineto{\pgfqpoint{0.932910in}{1.397843in}}%
\pgfpathlineto{\pgfqpoint{0.936695in}{1.404888in}}%
\pgfpathlineto{\pgfqpoint{0.940482in}{1.411872in}}%
\pgfpathlineto{\pgfqpoint{0.944273in}{1.418792in}}%
\pgfpathlineto{\pgfqpoint{0.948067in}{1.425645in}}%
\pgfpathlineto{\pgfqpoint{0.948568in}{1.422117in}}%
\pgfpathlineto{\pgfqpoint{0.949301in}{1.418596in}}%
\pgfpathlineto{\pgfqpoint{0.950266in}{1.415087in}}%
\pgfpathlineto{\pgfqpoint{0.951461in}{1.411592in}}%
\pgfpathclose%
\pgfusepath{fill}%
\end{pgfscope}%
\begin{pgfscope}%
\pgfpathrectangle{\pgfqpoint{0.041670in}{0.041670in}}{\pgfqpoint{2.216660in}{2.216660in}}%
\pgfusepath{clip}%
\pgfsetbuttcap%
\pgfsetroundjoin%
\definecolor{currentfill}{rgb}{0.855810,0.888601,0.097452}%
\pgfsetfillcolor{currentfill}%
\pgfsetlinewidth{0.000000pt}%
\definecolor{currentstroke}{rgb}{0.000000,0.000000,0.000000}%
\pgfsetstrokecolor{currentstroke}%
\pgfsetdash{}{0pt}%
\pgfpathmoveto{\pgfqpoint{1.211792in}{1.656785in}}%
\pgfpathlineto{\pgfqpoint{1.213104in}{1.655314in}}%
\pgfpathlineto{\pgfqpoint{1.214416in}{1.653722in}}%
\pgfpathlineto{\pgfqpoint{1.215726in}{1.652010in}}%
\pgfpathlineto{\pgfqpoint{1.217035in}{1.650179in}}%
\pgfpathlineto{\pgfqpoint{1.220236in}{1.649610in}}%
\pgfpathlineto{\pgfqpoint{1.223398in}{1.648994in}}%
\pgfpathlineto{\pgfqpoint{1.226518in}{1.648332in}}%
\pgfpathlineto{\pgfqpoint{1.229592in}{1.647623in}}%
\pgfpathlineto{\pgfqpoint{1.227839in}{1.649545in}}%
\pgfpathlineto{\pgfqpoint{1.226085in}{1.651348in}}%
\pgfpathlineto{\pgfqpoint{1.224328in}{1.653031in}}%
\pgfpathlineto{\pgfqpoint{1.222570in}{1.654593in}}%
\pgfpathlineto{\pgfqpoint{1.219932in}{1.655200in}}%
\pgfpathlineto{\pgfqpoint{1.217254in}{1.655768in}}%
\pgfpathlineto{\pgfqpoint{1.214540in}{1.656296in}}%
\pgfpathlineto{\pgfqpoint{1.211792in}{1.656785in}}%
\pgfpathclose%
\pgfusepath{fill}%
\end{pgfscope}%
\begin{pgfscope}%
\pgfpathrectangle{\pgfqpoint{0.041670in}{0.041670in}}{\pgfqpoint{2.216660in}{2.216660in}}%
\pgfusepath{clip}%
\pgfsetbuttcap%
\pgfsetroundjoin%
\definecolor{currentfill}{rgb}{0.166383,0.690856,0.496502}%
\pgfsetfillcolor{currentfill}%
\pgfsetlinewidth{0.000000pt}%
\definecolor{currentstroke}{rgb}{0.000000,0.000000,0.000000}%
\pgfsetstrokecolor{currentstroke}%
\pgfsetdash{}{0pt}%
\pgfpathmoveto{\pgfqpoint{0.929492in}{1.337354in}}%
\pgfpathlineto{\pgfqpoint{0.925874in}{1.329538in}}%
\pgfpathlineto{\pgfqpoint{0.922259in}{1.321672in}}%
\pgfpathlineto{\pgfqpoint{0.918646in}{1.313760in}}%
\pgfpathlineto{\pgfqpoint{0.915035in}{1.305803in}}%
\pgfpathlineto{\pgfqpoint{0.912541in}{1.309928in}}%
\pgfpathlineto{\pgfqpoint{0.910317in}{1.314086in}}%
\pgfpathlineto{\pgfqpoint{0.908367in}{1.318275in}}%
\pgfpathlineto{\pgfqpoint{0.906690in}{1.322488in}}%
\pgfpathlineto{\pgfqpoint{0.910406in}{1.330199in}}%
\pgfpathlineto{\pgfqpoint{0.914124in}{1.337866in}}%
\pgfpathlineto{\pgfqpoint{0.917845in}{1.345486in}}%
\pgfpathlineto{\pgfqpoint{0.921568in}{1.353057in}}%
\pgfpathlineto{\pgfqpoint{0.923162in}{1.349091in}}%
\pgfpathlineto{\pgfqpoint{0.925015in}{1.345149in}}%
\pgfpathlineto{\pgfqpoint{0.927126in}{1.341235in}}%
\pgfpathlineto{\pgfqpoint{0.929492in}{1.337354in}}%
\pgfpathclose%
\pgfusepath{fill}%
\end{pgfscope}%
\begin{pgfscope}%
\pgfpathrectangle{\pgfqpoint{0.041670in}{0.041670in}}{\pgfqpoint{2.216660in}{2.216660in}}%
\pgfusepath{clip}%
\pgfsetbuttcap%
\pgfsetroundjoin%
\definecolor{currentfill}{rgb}{0.147607,0.511733,0.557049}%
\pgfsetfillcolor{currentfill}%
\pgfsetlinewidth{0.000000pt}%
\definecolor{currentstroke}{rgb}{0.000000,0.000000,0.000000}%
\pgfsetstrokecolor{currentstroke}%
\pgfsetdash{}{0pt}%
\pgfpathmoveto{\pgfqpoint{0.893089in}{1.135633in}}%
\pgfpathlineto{\pgfqpoint{0.889894in}{1.126826in}}%
\pgfpathlineto{\pgfqpoint{0.886700in}{1.118021in}}%
\pgfpathlineto{\pgfqpoint{0.883507in}{1.109220in}}%
\pgfpathlineto{\pgfqpoint{0.880316in}{1.100427in}}%
\pgfpathlineto{\pgfqpoint{0.874553in}{1.105256in}}%
\pgfpathlineto{\pgfqpoint{0.869107in}{1.110172in}}%
\pgfpathlineto{\pgfqpoint{0.863982in}{1.115168in}}%
\pgfpathlineto{\pgfqpoint{0.859183in}{1.120240in}}%
\pgfpathlineto{\pgfqpoint{0.862590in}{1.128798in}}%
\pgfpathlineto{\pgfqpoint{0.865998in}{1.137364in}}%
\pgfpathlineto{\pgfqpoint{0.869408in}{1.145936in}}%
\pgfpathlineto{\pgfqpoint{0.872820in}{1.154509in}}%
\pgfpathlineto{\pgfqpoint{0.877425in}{1.149677in}}%
\pgfpathlineto{\pgfqpoint{0.882341in}{1.144917in}}%
\pgfpathlineto{\pgfqpoint{0.887564in}{1.140234in}}%
\pgfpathlineto{\pgfqpoint{0.893089in}{1.135633in}}%
\pgfpathclose%
\pgfusepath{fill}%
\end{pgfscope}%
\begin{pgfscope}%
\pgfpathrectangle{\pgfqpoint{0.041670in}{0.041670in}}{\pgfqpoint{2.216660in}{2.216660in}}%
\pgfusepath{clip}%
\pgfsetbuttcap%
\pgfsetroundjoin%
\definecolor{currentfill}{rgb}{0.855810,0.888601,0.097452}%
\pgfsetfillcolor{currentfill}%
\pgfsetlinewidth{0.000000pt}%
\definecolor{currentstroke}{rgb}{0.000000,0.000000,0.000000}%
\pgfsetstrokecolor{currentstroke}%
\pgfsetdash{}{0pt}%
\pgfpathmoveto{\pgfqpoint{1.135029in}{1.654020in}}%
\pgfpathlineto{\pgfqpoint{1.133175in}{1.652435in}}%
\pgfpathlineto{\pgfqpoint{1.131324in}{1.650728in}}%
\pgfpathlineto{\pgfqpoint{1.129474in}{1.648902in}}%
\pgfpathlineto{\pgfqpoint{1.127626in}{1.646956in}}%
\pgfpathlineto{\pgfqpoint{1.130657in}{1.647704in}}%
\pgfpathlineto{\pgfqpoint{1.133736in}{1.648407in}}%
\pgfpathlineto{\pgfqpoint{1.136861in}{1.649065in}}%
\pgfpathlineto{\pgfqpoint{1.140027in}{1.649676in}}%
\pgfpathlineto{\pgfqpoint{1.141437in}{1.651524in}}%
\pgfpathlineto{\pgfqpoint{1.142848in}{1.653254in}}%
\pgfpathlineto{\pgfqpoint{1.144260in}{1.654864in}}%
\pgfpathlineto{\pgfqpoint{1.145674in}{1.656353in}}%
\pgfpathlineto{\pgfqpoint{1.142956in}{1.655829in}}%
\pgfpathlineto{\pgfqpoint{1.140274in}{1.655265in}}%
\pgfpathlineto{\pgfqpoint{1.137631in}{1.654662in}}%
\pgfpathlineto{\pgfqpoint{1.135029in}{1.654020in}}%
\pgfpathclose%
\pgfusepath{fill}%
\end{pgfscope}%
\begin{pgfscope}%
\pgfpathrectangle{\pgfqpoint{0.041670in}{0.041670in}}{\pgfqpoint{2.216660in}{2.216660in}}%
\pgfusepath{clip}%
\pgfsetbuttcap%
\pgfsetroundjoin%
\definecolor{currentfill}{rgb}{0.814576,0.883393,0.110347}%
\pgfsetfillcolor{currentfill}%
\pgfsetlinewidth{0.000000pt}%
\definecolor{currentstroke}{rgb}{0.000000,0.000000,0.000000}%
\pgfsetstrokecolor{currentstroke}%
\pgfsetdash{}{0pt}%
\pgfpathmoveto{\pgfqpoint{1.252198in}{1.640406in}}%
\pgfpathlineto{\pgfqpoint{1.254749in}{1.638109in}}%
\pgfpathlineto{\pgfqpoint{1.257299in}{1.635695in}}%
\pgfpathlineto{\pgfqpoint{1.259845in}{1.633166in}}%
\pgfpathlineto{\pgfqpoint{1.262390in}{1.630522in}}%
\pgfpathlineto{\pgfqpoint{1.265285in}{1.629284in}}%
\pgfpathlineto{\pgfqpoint{1.268097in}{1.628003in}}%
\pgfpathlineto{\pgfqpoint{1.270822in}{1.626682in}}%
\pgfpathlineto{\pgfqpoint{1.273459in}{1.625320in}}%
\pgfpathlineto{\pgfqpoint{1.270571in}{1.628127in}}%
\pgfpathlineto{\pgfqpoint{1.267680in}{1.630819in}}%
\pgfpathlineto{\pgfqpoint{1.264786in}{1.633395in}}%
\pgfpathlineto{\pgfqpoint{1.261890in}{1.635854in}}%
\pgfpathlineto{\pgfqpoint{1.259582in}{1.637045in}}%
\pgfpathlineto{\pgfqpoint{1.257195in}{1.638202in}}%
\pgfpathlineto{\pgfqpoint{1.254733in}{1.639322in}}%
\pgfpathlineto{\pgfqpoint{1.252198in}{1.640406in}}%
\pgfpathclose%
\pgfusepath{fill}%
\end{pgfscope}%
\begin{pgfscope}%
\pgfpathrectangle{\pgfqpoint{0.041670in}{0.041670in}}{\pgfqpoint{2.216660in}{2.216660in}}%
\pgfusepath{clip}%
\pgfsetbuttcap%
\pgfsetroundjoin%
\definecolor{currentfill}{rgb}{0.274128,0.199721,0.498911}%
\pgfsetfillcolor{currentfill}%
\pgfsetlinewidth{0.000000pt}%
\definecolor{currentstroke}{rgb}{0.000000,0.000000,0.000000}%
\pgfsetstrokecolor{currentstroke}%
\pgfsetdash{}{0pt}%
\pgfpathmoveto{\pgfqpoint{0.864440in}{0.822910in}}%
\pgfpathlineto{\pgfqpoint{0.861841in}{0.815271in}}%
\pgfpathlineto{\pgfqpoint{0.859240in}{0.807738in}}%
\pgfpathlineto{\pgfqpoint{0.856638in}{0.800314in}}%
\pgfpathlineto{\pgfqpoint{0.854035in}{0.793003in}}%
\pgfpathlineto{\pgfqpoint{0.843269in}{0.798524in}}%
\pgfpathlineto{\pgfqpoint{0.832862in}{0.804217in}}%
\pgfpathlineto{\pgfqpoint{0.822826in}{0.810076in}}%
\pgfpathlineto{\pgfqpoint{0.813170in}{0.816094in}}%
\pgfpathlineto{\pgfqpoint{0.816090in}{0.823198in}}%
\pgfpathlineto{\pgfqpoint{0.819009in}{0.830416in}}%
\pgfpathlineto{\pgfqpoint{0.821927in}{0.837742in}}%
\pgfpathlineto{\pgfqpoint{0.824843in}{0.845175in}}%
\pgfpathlineto{\pgfqpoint{0.834201in}{0.839372in}}%
\pgfpathlineto{\pgfqpoint{0.843926in}{0.833723in}}%
\pgfpathlineto{\pgfqpoint{0.854009in}{0.828234in}}%
\pgfpathlineto{\pgfqpoint{0.864440in}{0.822910in}}%
\pgfpathclose%
\pgfusepath{fill}%
\end{pgfscope}%
\begin{pgfscope}%
\pgfpathrectangle{\pgfqpoint{0.041670in}{0.041670in}}{\pgfqpoint{2.216660in}{2.216660in}}%
\pgfusepath{clip}%
\pgfsetbuttcap%
\pgfsetroundjoin%
\definecolor{currentfill}{rgb}{0.233603,0.313828,0.543914}%
\pgfsetfillcolor{currentfill}%
\pgfsetlinewidth{0.000000pt}%
\definecolor{currentstroke}{rgb}{0.000000,0.000000,0.000000}%
\pgfsetstrokecolor{currentstroke}%
\pgfsetdash{}{0pt}%
\pgfpathmoveto{\pgfqpoint{0.521977in}{0.875664in}}%
\pgfpathlineto{\pgfqpoint{0.518030in}{0.886125in}}%
\pgfpathlineto{\pgfqpoint{0.514063in}{0.897041in}}%
\pgfpathlineto{\pgfqpoint{0.510076in}{0.908418in}}%
\pgfpathlineto{\pgfqpoint{0.506068in}{0.920265in}}%
\pgfpathlineto{\pgfqpoint{0.497233in}{0.931368in}}%
\pgfpathlineto{\pgfqpoint{0.489120in}{0.942592in}}%
\pgfpathlineto{\pgfqpoint{0.481734in}{0.953924in}}%
\pgfpathlineto{\pgfqpoint{0.475080in}{0.965353in}}%
\pgfpathlineto{\pgfqpoint{0.479246in}{0.953314in}}%
\pgfpathlineto{\pgfqpoint{0.483390in}{0.941741in}}%
\pgfpathlineto{\pgfqpoint{0.487514in}{0.930629in}}%
\pgfpathlineto{\pgfqpoint{0.491618in}{0.919968in}}%
\pgfpathlineto{\pgfqpoint{0.498142in}{0.908737in}}%
\pgfpathlineto{\pgfqpoint{0.505380in}{0.897601in}}%
\pgfpathlineto{\pgfqpoint{0.513326in}{0.886573in}}%
\pgfpathlineto{\pgfqpoint{0.521977in}{0.875664in}}%
\pgfpathclose%
\pgfusepath{fill}%
\end{pgfscope}%
\begin{pgfscope}%
\pgfpathrectangle{\pgfqpoint{0.041670in}{0.041670in}}{\pgfqpoint{2.216660in}{2.216660in}}%
\pgfusepath{clip}%
\pgfsetbuttcap%
\pgfsetroundjoin%
\definecolor{currentfill}{rgb}{0.699415,0.867117,0.175971}%
\pgfsetfillcolor{currentfill}%
\pgfsetlinewidth{0.000000pt}%
\definecolor{currentstroke}{rgb}{0.000000,0.000000,0.000000}%
\pgfsetstrokecolor{currentstroke}%
\pgfsetdash{}{0pt}%
\pgfpathmoveto{\pgfqpoint{1.304777in}{1.599282in}}%
\pgfpathlineto{\pgfqpoint{1.308200in}{1.595538in}}%
\pgfpathlineto{\pgfqpoint{1.311621in}{1.591688in}}%
\pgfpathlineto{\pgfqpoint{1.315039in}{1.587732in}}%
\pgfpathlineto{\pgfqpoint{1.318453in}{1.583672in}}%
\pgfpathlineto{\pgfqpoint{1.320616in}{1.581597in}}%
\pgfpathlineto{\pgfqpoint{1.322642in}{1.579489in}}%
\pgfpathlineto{\pgfqpoint{1.324527in}{1.577352in}}%
\pgfpathlineto{\pgfqpoint{1.326270in}{1.575188in}}%
\pgfpathlineto{\pgfqpoint{1.322658in}{1.579462in}}%
\pgfpathlineto{\pgfqpoint{1.319043in}{1.583633in}}%
\pgfpathlineto{\pgfqpoint{1.315425in}{1.587697in}}%
\pgfpathlineto{\pgfqpoint{1.311805in}{1.591654in}}%
\pgfpathlineto{\pgfqpoint{1.310239in}{1.593600in}}%
\pgfpathlineto{\pgfqpoint{1.308544in}{1.595521in}}%
\pgfpathlineto{\pgfqpoint{1.306723in}{1.597416in}}%
\pgfpathlineto{\pgfqpoint{1.304777in}{1.599282in}}%
\pgfpathclose%
\pgfusepath{fill}%
\end{pgfscope}%
\begin{pgfscope}%
\pgfpathrectangle{\pgfqpoint{0.041670in}{0.041670in}}{\pgfqpoint{2.216660in}{2.216660in}}%
\pgfusepath{clip}%
\pgfsetbuttcap%
\pgfsetroundjoin%
\definecolor{currentfill}{rgb}{0.636902,0.856542,0.216620}%
\pgfsetfillcolor{currentfill}%
\pgfsetlinewidth{0.000000pt}%
\definecolor{currentstroke}{rgb}{0.000000,0.000000,0.000000}%
\pgfsetstrokecolor{currentstroke}%
\pgfsetdash{}{0pt}%
\pgfpathmoveto{\pgfqpoint{1.326270in}{1.575188in}}%
\pgfpathlineto{\pgfqpoint{1.329878in}{1.570810in}}%
\pgfpathlineto{\pgfqpoint{1.333484in}{1.566331in}}%
\pgfpathlineto{\pgfqpoint{1.337086in}{1.561751in}}%
\pgfpathlineto{\pgfqpoint{1.340685in}{1.557074in}}%
\pgfpathlineto{\pgfqpoint{1.342446in}{1.554661in}}%
\pgfpathlineto{\pgfqpoint{1.344048in}{1.552222in}}%
\pgfpathlineto{\pgfqpoint{1.345487in}{1.549760in}}%
\pgfpathlineto{\pgfqpoint{1.346763in}{1.547276in}}%
\pgfpathlineto{\pgfqpoint{1.343022in}{1.552180in}}%
\pgfpathlineto{\pgfqpoint{1.339278in}{1.556985in}}%
\pgfpathlineto{\pgfqpoint{1.335532in}{1.561691in}}%
\pgfpathlineto{\pgfqpoint{1.331782in}{1.566294in}}%
\pgfpathlineto{\pgfqpoint{1.330626in}{1.568548in}}%
\pgfpathlineto{\pgfqpoint{1.329321in}{1.570784in}}%
\pgfpathlineto{\pgfqpoint{1.327868in}{1.572997in}}%
\pgfpathlineto{\pgfqpoint{1.326270in}{1.575188in}}%
\pgfpathclose%
\pgfusepath{fill}%
\end{pgfscope}%
\begin{pgfscope}%
\pgfpathrectangle{\pgfqpoint{0.041670in}{0.041670in}}{\pgfqpoint{2.216660in}{2.216660in}}%
\pgfusepath{clip}%
\pgfsetbuttcap%
\pgfsetroundjoin%
\definecolor{currentfill}{rgb}{0.814576,0.883393,0.110347}%
\pgfsetfillcolor{currentfill}%
\pgfsetlinewidth{0.000000pt}%
\definecolor{currentstroke}{rgb}{0.000000,0.000000,0.000000}%
\pgfsetstrokecolor{currentstroke}%
\pgfsetdash{}{0pt}%
\pgfpathmoveto{\pgfqpoint{1.096035in}{1.634767in}}%
\pgfpathlineto{\pgfqpoint{1.093068in}{1.632269in}}%
\pgfpathlineto{\pgfqpoint{1.090104in}{1.629654in}}%
\pgfpathlineto{\pgfqpoint{1.087143in}{1.626923in}}%
\pgfpathlineto{\pgfqpoint{1.084184in}{1.624077in}}%
\pgfpathlineto{\pgfqpoint{1.086739in}{1.625473in}}%
\pgfpathlineto{\pgfqpoint{1.089386in}{1.626831in}}%
\pgfpathlineto{\pgfqpoint{1.092121in}{1.628148in}}%
\pgfpathlineto{\pgfqpoint{1.094942in}{1.629424in}}%
\pgfpathlineto{\pgfqpoint{1.097567in}{1.632102in}}%
\pgfpathlineto{\pgfqpoint{1.100194in}{1.634665in}}%
\pgfpathlineto{\pgfqpoint{1.102823in}{1.637113in}}%
\pgfpathlineto{\pgfqpoint{1.105455in}{1.639444in}}%
\pgfpathlineto{\pgfqpoint{1.102984in}{1.638328in}}%
\pgfpathlineto{\pgfqpoint{1.100589in}{1.637176in}}%
\pgfpathlineto{\pgfqpoint{1.098272in}{1.635988in}}%
\pgfpathlineto{\pgfqpoint{1.096035in}{1.634767in}}%
\pgfpathclose%
\pgfusepath{fill}%
\end{pgfscope}%
\begin{pgfscope}%
\pgfpathrectangle{\pgfqpoint{0.041670in}{0.041670in}}{\pgfqpoint{2.216660in}{2.216660in}}%
\pgfusepath{clip}%
\pgfsetbuttcap%
\pgfsetroundjoin%
\definecolor{currentfill}{rgb}{0.195860,0.395433,0.555276}%
\pgfsetfillcolor{currentfill}%
\pgfsetlinewidth{0.000000pt}%
\definecolor{currentstroke}{rgb}{0.000000,0.000000,0.000000}%
\pgfsetstrokecolor{currentstroke}%
\pgfsetdash{}{0pt}%
\pgfpathmoveto{\pgfqpoint{0.883083in}{1.010587in}}%
\pgfpathlineto{\pgfqpoint{0.880170in}{1.001773in}}%
\pgfpathlineto{\pgfqpoint{0.877257in}{0.992997in}}%
\pgfpathlineto{\pgfqpoint{0.874345in}{0.984261in}}%
\pgfpathlineto{\pgfqpoint{0.871433in}{0.975570in}}%
\pgfpathlineto{\pgfqpoint{0.863596in}{0.980647in}}%
\pgfpathlineto{\pgfqpoint{0.856092in}{0.985844in}}%
\pgfpathlineto{\pgfqpoint{0.848926in}{0.991157in}}%
\pgfpathlineto{\pgfqpoint{0.842107in}{0.996580in}}%
\pgfpathlineto{\pgfqpoint{0.845286in}{1.005048in}}%
\pgfpathlineto{\pgfqpoint{0.848466in}{1.013561in}}%
\pgfpathlineto{\pgfqpoint{0.851647in}{1.022114in}}%
\pgfpathlineto{\pgfqpoint{0.854828in}{1.030706in}}%
\pgfpathlineto{\pgfqpoint{0.861400in}{1.025513in}}%
\pgfpathlineto{\pgfqpoint{0.868304in}{1.020425in}}%
\pgfpathlineto{\pgfqpoint{0.875534in}{1.015448in}}%
\pgfpathlineto{\pgfqpoint{0.883083in}{1.010587in}}%
\pgfpathclose%
\pgfusepath{fill}%
\end{pgfscope}%
\begin{pgfscope}%
\pgfpathrectangle{\pgfqpoint{0.041670in}{0.041670in}}{\pgfqpoint{2.216660in}{2.216660in}}%
\pgfusepath{clip}%
\pgfsetbuttcap%
\pgfsetroundjoin%
\definecolor{currentfill}{rgb}{0.268510,0.009605,0.335427}%
\pgfsetfillcolor{currentfill}%
\pgfsetlinewidth{0.000000pt}%
\definecolor{currentstroke}{rgb}{0.000000,0.000000,0.000000}%
\pgfsetstrokecolor{currentstroke}%
\pgfsetdash{}{0pt}%
\pgfpathmoveto{\pgfqpoint{0.790843in}{0.661666in}}%
\pgfpathlineto{\pgfqpoint{0.788164in}{0.658483in}}%
\pgfpathlineto{\pgfqpoint{0.785481in}{0.655521in}}%
\pgfpathlineto{\pgfqpoint{0.782793in}{0.652786in}}%
\pgfpathlineto{\pgfqpoint{0.780099in}{0.650281in}}%
\pgfpathlineto{\pgfqpoint{0.766961in}{0.657186in}}%
\pgfpathlineto{\pgfqpoint{0.754271in}{0.664303in}}%
\pgfpathlineto{\pgfqpoint{0.742042in}{0.671625in}}%
\pgfpathlineto{\pgfqpoint{0.730286in}{0.679143in}}%
\pgfpathlineto{\pgfqpoint{0.733304in}{0.681444in}}%
\pgfpathlineto{\pgfqpoint{0.736315in}{0.683976in}}%
\pgfpathlineto{\pgfqpoint{0.739322in}{0.686734in}}%
\pgfpathlineto{\pgfqpoint{0.742322in}{0.689713in}}%
\pgfpathlineto{\pgfqpoint{0.753775in}{0.682407in}}%
\pgfpathlineto{\pgfqpoint{0.765687in}{0.675292in}}%
\pgfpathlineto{\pgfqpoint{0.778047in}{0.668375in}}%
\pgfpathlineto{\pgfqpoint{0.790843in}{0.661666in}}%
\pgfpathclose%
\pgfusepath{fill}%
\end{pgfscope}%
\begin{pgfscope}%
\pgfpathrectangle{\pgfqpoint{0.041670in}{0.041670in}}{\pgfqpoint{2.216660in}{2.216660in}}%
\pgfusepath{clip}%
\pgfsetbuttcap%
\pgfsetroundjoin%
\definecolor{currentfill}{rgb}{0.412913,0.803041,0.357269}%
\pgfsetfillcolor{currentfill}%
\pgfsetlinewidth{0.000000pt}%
\definecolor{currentstroke}{rgb}{0.000000,0.000000,0.000000}%
\pgfsetstrokecolor{currentstroke}%
\pgfsetdash{}{0pt}%
\pgfpathmoveto{\pgfqpoint{0.978527in}{1.477839in}}%
\pgfpathlineto{\pgfqpoint{0.974709in}{1.471589in}}%
\pgfpathlineto{\pgfqpoint{0.970894in}{1.465257in}}%
\pgfpathlineto{\pgfqpoint{0.967082in}{1.458844in}}%
\pgfpathlineto{\pgfqpoint{0.963273in}{1.452354in}}%
\pgfpathlineto{\pgfqpoint{0.963014in}{1.455641in}}%
\pgfpathlineto{\pgfqpoint{0.962971in}{1.458929in}}%
\pgfpathlineto{\pgfqpoint{0.963144in}{1.462215in}}%
\pgfpathlineto{\pgfqpoint{0.963534in}{1.465494in}}%
\pgfpathlineto{\pgfqpoint{0.967331in}{1.471742in}}%
\pgfpathlineto{\pgfqpoint{0.971131in}{1.477912in}}%
\pgfpathlineto{\pgfqpoint{0.974934in}{1.484001in}}%
\pgfpathlineto{\pgfqpoint{0.978741in}{1.490009in}}%
\pgfpathlineto{\pgfqpoint{0.978386in}{1.486972in}}%
\pgfpathlineto{\pgfqpoint{0.978232in}{1.483929in}}%
\pgfpathlineto{\pgfqpoint{0.978279in}{1.480883in}}%
\pgfpathlineto{\pgfqpoint{0.978527in}{1.477839in}}%
\pgfpathclose%
\pgfusepath{fill}%
\end{pgfscope}%
\begin{pgfscope}%
\pgfpathrectangle{\pgfqpoint{0.041670in}{0.041670in}}{\pgfqpoint{2.216660in}{2.216660in}}%
\pgfusepath{clip}%
\pgfsetbuttcap%
\pgfsetroundjoin%
\definecolor{currentfill}{rgb}{0.120081,0.622161,0.534946}%
\pgfsetfillcolor{currentfill}%
\pgfsetlinewidth{0.000000pt}%
\definecolor{currentstroke}{rgb}{0.000000,0.000000,0.000000}%
\pgfsetstrokecolor{currentstroke}%
\pgfsetdash{}{0pt}%
\pgfpathmoveto{\pgfqpoint{0.913913in}{1.256546in}}%
\pgfpathlineto{\pgfqpoint{0.910478in}{1.248161in}}%
\pgfpathlineto{\pgfqpoint{0.907044in}{1.239747in}}%
\pgfpathlineto{\pgfqpoint{0.903613in}{1.231304in}}%
\pgfpathlineto{\pgfqpoint{0.900184in}{1.222836in}}%
\pgfpathlineto{\pgfqpoint{0.896259in}{1.227256in}}%
\pgfpathlineto{\pgfqpoint{0.892625in}{1.231731in}}%
\pgfpathlineto{\pgfqpoint{0.889284in}{1.236258in}}%
\pgfpathlineto{\pgfqpoint{0.886239in}{1.240832in}}%
\pgfpathlineto{\pgfqpoint{0.889830in}{1.249058in}}%
\pgfpathlineto{\pgfqpoint{0.893424in}{1.257259in}}%
\pgfpathlineto{\pgfqpoint{0.897019in}{1.265433in}}%
\pgfpathlineto{\pgfqpoint{0.900618in}{1.273577in}}%
\pgfpathlineto{\pgfqpoint{0.903522in}{1.269247in}}%
\pgfpathlineto{\pgfqpoint{0.906708in}{1.264963in}}%
\pgfpathlineto{\pgfqpoint{0.910173in}{1.260727in}}%
\pgfpathlineto{\pgfqpoint{0.913913in}{1.256546in}}%
\pgfpathclose%
\pgfusepath{fill}%
\end{pgfscope}%
\begin{pgfscope}%
\pgfpathrectangle{\pgfqpoint{0.041670in}{0.041670in}}{\pgfqpoint{2.216660in}{2.216660in}}%
\pgfusepath{clip}%
\pgfsetbuttcap%
\pgfsetroundjoin%
\definecolor{currentfill}{rgb}{0.133743,0.548535,0.553541}%
\pgfsetfillcolor{currentfill}%
\pgfsetlinewidth{0.000000pt}%
\definecolor{currentstroke}{rgb}{0.000000,0.000000,0.000000}%
\pgfsetstrokecolor{currentstroke}%
\pgfsetdash{}{0pt}%
\pgfpathmoveto{\pgfqpoint{1.477089in}{1.192905in}}%
\pgfpathlineto{\pgfqpoint{1.480549in}{1.184404in}}%
\pgfpathlineto{\pgfqpoint{1.484007in}{1.175895in}}%
\pgfpathlineto{\pgfqpoint{1.487463in}{1.167379in}}%
\pgfpathlineto{\pgfqpoint{1.490917in}{1.158861in}}%
\pgfpathlineto{\pgfqpoint{1.486593in}{1.153969in}}%
\pgfpathlineto{\pgfqpoint{1.481954in}{1.149144in}}%
\pgfpathlineto{\pgfqpoint{1.477004in}{1.144392in}}%
\pgfpathlineto{\pgfqpoint{1.471747in}{1.139719in}}%
\pgfpathlineto{\pgfqpoint{1.468497in}{1.148474in}}%
\pgfpathlineto{\pgfqpoint{1.465245in}{1.157226in}}%
\pgfpathlineto{\pgfqpoint{1.461992in}{1.165970in}}%
\pgfpathlineto{\pgfqpoint{1.458737in}{1.174706in}}%
\pgfpathlineto{\pgfqpoint{1.463768in}{1.179149in}}%
\pgfpathlineto{\pgfqpoint{1.468506in}{1.183666in}}%
\pgfpathlineto{\pgfqpoint{1.472948in}{1.188253in}}%
\pgfpathlineto{\pgfqpoint{1.477089in}{1.192905in}}%
\pgfpathclose%
\pgfusepath{fill}%
\end{pgfscope}%
\begin{pgfscope}%
\pgfpathrectangle{\pgfqpoint{0.041670in}{0.041670in}}{\pgfqpoint{2.216660in}{2.216660in}}%
\pgfusepath{clip}%
\pgfsetbuttcap%
\pgfsetroundjoin%
\definecolor{currentfill}{rgb}{0.855810,0.888601,0.097452}%
\pgfsetfillcolor{currentfill}%
\pgfsetlinewidth{0.000000pt}%
\definecolor{currentstroke}{rgb}{0.000000,0.000000,0.000000}%
\pgfsetstrokecolor{currentstroke}%
\pgfsetdash{}{0pt}%
\pgfpathmoveto{\pgfqpoint{1.222570in}{1.654593in}}%
\pgfpathlineto{\pgfqpoint{1.224328in}{1.653031in}}%
\pgfpathlineto{\pgfqpoint{1.226085in}{1.651348in}}%
\pgfpathlineto{\pgfqpoint{1.227839in}{1.649545in}}%
\pgfpathlineto{\pgfqpoint{1.229592in}{1.647623in}}%
\pgfpathlineto{\pgfqpoint{1.232618in}{1.646870in}}%
\pgfpathlineto{\pgfqpoint{1.235592in}{1.646073in}}%
\pgfpathlineto{\pgfqpoint{1.238512in}{1.645232in}}%
\pgfpathlineto{\pgfqpoint{1.241375in}{1.644349in}}%
\pgfpathlineto{\pgfqpoint{1.239205in}{1.646387in}}%
\pgfpathlineto{\pgfqpoint{1.237033in}{1.648307in}}%
\pgfpathlineto{\pgfqpoint{1.234858in}{1.650106in}}%
\pgfpathlineto{\pgfqpoint{1.232682in}{1.651784in}}%
\pgfpathlineto{\pgfqpoint{1.230226in}{1.652542in}}%
\pgfpathlineto{\pgfqpoint{1.227720in}{1.653263in}}%
\pgfpathlineto{\pgfqpoint{1.225167in}{1.653947in}}%
\pgfpathlineto{\pgfqpoint{1.222570in}{1.654593in}}%
\pgfpathclose%
\pgfusepath{fill}%
\end{pgfscope}%
\begin{pgfscope}%
\pgfpathrectangle{\pgfqpoint{0.041670in}{0.041670in}}{\pgfqpoint{2.216660in}{2.216660in}}%
\pgfusepath{clip}%
\pgfsetbuttcap%
\pgfsetroundjoin%
\definecolor{currentfill}{rgb}{0.179019,0.433756,0.557430}%
\pgfsetfillcolor{currentfill}%
\pgfsetlinewidth{0.000000pt}%
\definecolor{currentstroke}{rgb}{0.000000,0.000000,0.000000}%
\pgfsetstrokecolor{currentstroke}%
\pgfsetdash{}{0pt}%
\pgfpathmoveto{\pgfqpoint{1.497695in}{1.069881in}}%
\pgfpathlineto{\pgfqpoint{1.500933in}{1.061220in}}%
\pgfpathlineto{\pgfqpoint{1.504169in}{1.052586in}}%
\pgfpathlineto{\pgfqpoint{1.507405in}{1.043980in}}%
\pgfpathlineto{\pgfqpoint{1.510639in}{1.035405in}}%
\pgfpathlineto{\pgfqpoint{1.504368in}{1.030124in}}%
\pgfpathlineto{\pgfqpoint{1.497759in}{1.024942in}}%
\pgfpathlineto{\pgfqpoint{1.490818in}{1.019866in}}%
\pgfpathlineto{\pgfqpoint{1.483552in}{1.014902in}}%
\pgfpathlineto{\pgfqpoint{1.480574in}{1.023702in}}%
\pgfpathlineto{\pgfqpoint{1.477595in}{1.032534in}}%
\pgfpathlineto{\pgfqpoint{1.474616in}{1.041393in}}%
\pgfpathlineto{\pgfqpoint{1.471635in}{1.050278in}}%
\pgfpathlineto{\pgfqpoint{1.478624in}{1.055024in}}%
\pgfpathlineto{\pgfqpoint{1.485301in}{1.059877in}}%
\pgfpathlineto{\pgfqpoint{1.491660in}{1.064831in}}%
\pgfpathlineto{\pgfqpoint{1.497695in}{1.069881in}}%
\pgfpathclose%
\pgfusepath{fill}%
\end{pgfscope}%
\begin{pgfscope}%
\pgfpathrectangle{\pgfqpoint{0.041670in}{0.041670in}}{\pgfqpoint{2.216660in}{2.216660in}}%
\pgfusepath{clip}%
\pgfsetbuttcap%
\pgfsetroundjoin%
\definecolor{currentfill}{rgb}{0.762373,0.876424,0.137064}%
\pgfsetfillcolor{currentfill}%
\pgfsetlinewidth{0.000000pt}%
\definecolor{currentstroke}{rgb}{0.000000,0.000000,0.000000}%
\pgfsetstrokecolor{currentstroke}%
\pgfsetdash{}{0pt}%
\pgfpathmoveto{\pgfqpoint{1.283065in}{1.619502in}}%
\pgfpathlineto{\pgfqpoint{1.286250in}{1.616399in}}%
\pgfpathlineto{\pgfqpoint{1.289432in}{1.613185in}}%
\pgfpathlineto{\pgfqpoint{1.292611in}{1.609860in}}%
\pgfpathlineto{\pgfqpoint{1.295788in}{1.606425in}}%
\pgfpathlineto{\pgfqpoint{1.298211in}{1.604691in}}%
\pgfpathlineto{\pgfqpoint{1.300519in}{1.602921in}}%
\pgfpathlineto{\pgfqpoint{1.302708in}{1.601117in}}%
\pgfpathlineto{\pgfqpoint{1.304777in}{1.599282in}}%
\pgfpathlineto{\pgfqpoint{1.301350in}{1.602916in}}%
\pgfpathlineto{\pgfqpoint{1.297921in}{1.606441in}}%
\pgfpathlineto{\pgfqpoint{1.294488in}{1.609855in}}%
\pgfpathlineto{\pgfqpoint{1.291053in}{1.613157in}}%
\pgfpathlineto{\pgfqpoint{1.289215in}{1.614787in}}%
\pgfpathlineto{\pgfqpoint{1.287270in}{1.616389in}}%
\pgfpathlineto{\pgfqpoint{1.285219in}{1.617961in}}%
\pgfpathlineto{\pgfqpoint{1.283065in}{1.619502in}}%
\pgfpathclose%
\pgfusepath{fill}%
\end{pgfscope}%
\begin{pgfscope}%
\pgfpathrectangle{\pgfqpoint{0.041670in}{0.041670in}}{\pgfqpoint{2.216660in}{2.216660in}}%
\pgfusepath{clip}%
\pgfsetbuttcap%
\pgfsetroundjoin%
\definecolor{currentfill}{rgb}{0.699415,0.867117,0.175971}%
\pgfsetfillcolor{currentfill}%
\pgfsetlinewidth{0.000000pt}%
\definecolor{currentstroke}{rgb}{0.000000,0.000000,0.000000}%
\pgfsetstrokecolor{currentstroke}%
\pgfsetdash{}{0pt}%
\pgfpathmoveto{\pgfqpoint{1.046822in}{1.589905in}}%
\pgfpathlineto{\pgfqpoint{1.043165in}{1.585899in}}%
\pgfpathlineto{\pgfqpoint{1.039511in}{1.581785in}}%
\pgfpathlineto{\pgfqpoint{1.035860in}{1.577566in}}%
\pgfpathlineto{\pgfqpoint{1.032212in}{1.573242in}}%
\pgfpathlineto{\pgfqpoint{1.033827in}{1.575429in}}%
\pgfpathlineto{\pgfqpoint{1.035585in}{1.577591in}}%
\pgfpathlineto{\pgfqpoint{1.037486in}{1.579725in}}%
\pgfpathlineto{\pgfqpoint{1.039527in}{1.581829in}}%
\pgfpathlineto{\pgfqpoint{1.042990in}{1.585936in}}%
\pgfpathlineto{\pgfqpoint{1.046456in}{1.589938in}}%
\pgfpathlineto{\pgfqpoint{1.049925in}{1.593835in}}%
\pgfpathlineto{\pgfqpoint{1.053397in}{1.597624in}}%
\pgfpathlineto{\pgfqpoint{1.051562in}{1.595733in}}%
\pgfpathlineto{\pgfqpoint{1.049853in}{1.593815in}}%
\pgfpathlineto{\pgfqpoint{1.048273in}{1.591872in}}%
\pgfpathlineto{\pgfqpoint{1.046822in}{1.589905in}}%
\pgfpathclose%
\pgfusepath{fill}%
\end{pgfscope}%
\begin{pgfscope}%
\pgfpathrectangle{\pgfqpoint{0.041670in}{0.041670in}}{\pgfqpoint{2.216660in}{2.216660in}}%
\pgfusepath{clip}%
\pgfsetbuttcap%
\pgfsetroundjoin%
\definecolor{currentfill}{rgb}{0.248629,0.278775,0.534556}%
\pgfsetfillcolor{currentfill}%
\pgfsetlinewidth{0.000000pt}%
\definecolor{currentstroke}{rgb}{0.000000,0.000000,0.000000}%
\pgfsetstrokecolor{currentstroke}%
\pgfsetdash{}{0pt}%
\pgfpathmoveto{\pgfqpoint{1.519256in}{0.912901in}}%
\pgfpathlineto{\pgfqpoint{1.522231in}{0.904800in}}%
\pgfpathlineto{\pgfqpoint{1.525206in}{0.896777in}}%
\pgfpathlineto{\pgfqpoint{1.528182in}{0.888835in}}%
\pgfpathlineto{\pgfqpoint{1.531157in}{0.880976in}}%
\pgfpathlineto{\pgfqpoint{1.522421in}{0.875262in}}%
\pgfpathlineto{\pgfqpoint{1.513322in}{0.869689in}}%
\pgfpathlineto{\pgfqpoint{1.503869in}{0.864266in}}%
\pgfpathlineto{\pgfqpoint{1.494072in}{0.858997in}}%
\pgfpathlineto{\pgfqpoint{1.491402in}{0.867066in}}%
\pgfpathlineto{\pgfqpoint{1.488733in}{0.875219in}}%
\pgfpathlineto{\pgfqpoint{1.486065in}{0.883452in}}%
\pgfpathlineto{\pgfqpoint{1.483396in}{0.891763in}}%
\pgfpathlineto{\pgfqpoint{1.492868in}{0.896829in}}%
\pgfpathlineto{\pgfqpoint{1.502008in}{0.902045in}}%
\pgfpathlineto{\pgfqpoint{1.510807in}{0.907404in}}%
\pgfpathlineto{\pgfqpoint{1.519256in}{0.912901in}}%
\pgfpathclose%
\pgfusepath{fill}%
\end{pgfscope}%
\begin{pgfscope}%
\pgfpathrectangle{\pgfqpoint{0.041670in}{0.041670in}}{\pgfqpoint{2.216660in}{2.216660in}}%
\pgfusepath{clip}%
\pgfsetbuttcap%
\pgfsetroundjoin%
\definecolor{currentfill}{rgb}{0.855810,0.888601,0.097452}%
\pgfsetfillcolor{currentfill}%
\pgfsetlinewidth{0.000000pt}%
\definecolor{currentstroke}{rgb}{0.000000,0.000000,0.000000}%
\pgfsetstrokecolor{currentstroke}%
\pgfsetdash{}{0pt}%
\pgfpathmoveto{\pgfqpoint{1.125087in}{1.651081in}}%
\pgfpathlineto{\pgfqpoint{1.122822in}{1.649373in}}%
\pgfpathlineto{\pgfqpoint{1.120560in}{1.647545in}}%
\pgfpathlineto{\pgfqpoint{1.118299in}{1.645596in}}%
\pgfpathlineto{\pgfqpoint{1.116041in}{1.643528in}}%
\pgfpathlineto{\pgfqpoint{1.118850in}{1.644449in}}%
\pgfpathlineto{\pgfqpoint{1.121719in}{1.645328in}}%
\pgfpathlineto{\pgfqpoint{1.124646in}{1.646164in}}%
\pgfpathlineto{\pgfqpoint{1.127626in}{1.646956in}}%
\pgfpathlineto{\pgfqpoint{1.129474in}{1.648902in}}%
\pgfpathlineto{\pgfqpoint{1.131324in}{1.650728in}}%
\pgfpathlineto{\pgfqpoint{1.133175in}{1.652435in}}%
\pgfpathlineto{\pgfqpoint{1.135029in}{1.654020in}}%
\pgfpathlineto{\pgfqpoint{1.132471in}{1.653341in}}%
\pgfpathlineto{\pgfqpoint{1.129960in}{1.652624in}}%
\pgfpathlineto{\pgfqpoint{1.127498in}{1.651870in}}%
\pgfpathlineto{\pgfqpoint{1.125087in}{1.651081in}}%
\pgfpathclose%
\pgfusepath{fill}%
\end{pgfscope}%
\begin{pgfscope}%
\pgfpathrectangle{\pgfqpoint{0.041670in}{0.041670in}}{\pgfqpoint{2.216660in}{2.216660in}}%
\pgfusepath{clip}%
\pgfsetbuttcap%
\pgfsetroundjoin%
\definecolor{currentfill}{rgb}{0.565498,0.842430,0.262877}%
\pgfsetfillcolor{currentfill}%
\pgfsetlinewidth{0.000000pt}%
\definecolor{currentstroke}{rgb}{0.000000,0.000000,0.000000}%
\pgfsetstrokecolor{currentstroke}%
\pgfsetdash{}{0pt}%
\pgfpathmoveto{\pgfqpoint{1.346763in}{1.547276in}}%
\pgfpathlineto{\pgfqpoint{1.350500in}{1.542274in}}%
\pgfpathlineto{\pgfqpoint{1.354235in}{1.537178in}}%
\pgfpathlineto{\pgfqpoint{1.357966in}{1.531987in}}%
\pgfpathlineto{\pgfqpoint{1.361694in}{1.526704in}}%
\pgfpathlineto{\pgfqpoint{1.362911in}{1.523969in}}%
\pgfpathlineto{\pgfqpoint{1.363946in}{1.521216in}}%
\pgfpathlineto{\pgfqpoint{1.364800in}{1.518447in}}%
\pgfpathlineto{\pgfqpoint{1.365470in}{1.515666in}}%
\pgfpathlineto{\pgfqpoint{1.361658in}{1.521184in}}%
\pgfpathlineto{\pgfqpoint{1.357843in}{1.526610in}}%
\pgfpathlineto{\pgfqpoint{1.354025in}{1.531942in}}%
\pgfpathlineto{\pgfqpoint{1.350205in}{1.537178in}}%
\pgfpathlineto{\pgfqpoint{1.349596in}{1.539722in}}%
\pgfpathlineto{\pgfqpoint{1.348818in}{1.542254in}}%
\pgfpathlineto{\pgfqpoint{1.347874in}{1.544773in}}%
\pgfpathlineto{\pgfqpoint{1.346763in}{1.547276in}}%
\pgfpathclose%
\pgfusepath{fill}%
\end{pgfscope}%
\begin{pgfscope}%
\pgfpathrectangle{\pgfqpoint{0.041670in}{0.041670in}}{\pgfqpoint{2.216660in}{2.216660in}}%
\pgfusepath{clip}%
\pgfsetbuttcap%
\pgfsetroundjoin%
\definecolor{currentfill}{rgb}{0.636902,0.856542,0.216620}%
\pgfsetfillcolor{currentfill}%
\pgfsetlinewidth{0.000000pt}%
\definecolor{currentstroke}{rgb}{0.000000,0.000000,0.000000}%
\pgfsetstrokecolor{currentstroke}%
\pgfsetdash{}{0pt}%
\pgfpathmoveto{\pgfqpoint{1.027226in}{1.564276in}}%
\pgfpathlineto{\pgfqpoint{1.023453in}{1.559621in}}%
\pgfpathlineto{\pgfqpoint{1.019683in}{1.554865in}}%
\pgfpathlineto{\pgfqpoint{1.015915in}{1.550008in}}%
\pgfpathlineto{\pgfqpoint{1.012151in}{1.545052in}}%
\pgfpathlineto{\pgfqpoint{1.013281in}{1.547553in}}%
\pgfpathlineto{\pgfqpoint{1.014575in}{1.550034in}}%
\pgfpathlineto{\pgfqpoint{1.016032in}{1.552494in}}%
\pgfpathlineto{\pgfqpoint{1.017651in}{1.554930in}}%
\pgfpathlineto{\pgfqpoint{1.021287in}{1.559658in}}%
\pgfpathlineto{\pgfqpoint{1.024925in}{1.564286in}}%
\pgfpathlineto{\pgfqpoint{1.028567in}{1.568815in}}%
\pgfpathlineto{\pgfqpoint{1.032212in}{1.573242in}}%
\pgfpathlineto{\pgfqpoint{1.030743in}{1.571031in}}%
\pgfpathlineto{\pgfqpoint{1.029422in}{1.568798in}}%
\pgfpathlineto{\pgfqpoint{1.028249in}{1.566546in}}%
\pgfpathlineto{\pgfqpoint{1.027226in}{1.564276in}}%
\pgfpathclose%
\pgfusepath{fill}%
\end{pgfscope}%
\begin{pgfscope}%
\pgfpathrectangle{\pgfqpoint{0.041670in}{0.041670in}}{\pgfqpoint{2.216660in}{2.216660in}}%
\pgfusepath{clip}%
\pgfsetbuttcap%
\pgfsetroundjoin%
\definecolor{currentfill}{rgb}{0.220124,0.725509,0.466226}%
\pgfsetfillcolor{currentfill}%
\pgfsetlinewidth{0.000000pt}%
\definecolor{currentstroke}{rgb}{0.000000,0.000000,0.000000}%
\pgfsetstrokecolor{currentstroke}%
\pgfsetdash{}{0pt}%
\pgfpathmoveto{\pgfqpoint{1.424554in}{1.386130in}}%
\pgfpathlineto{\pgfqpoint{1.428305in}{1.378832in}}%
\pgfpathlineto{\pgfqpoint{1.432053in}{1.371476in}}%
\pgfpathlineto{\pgfqpoint{1.435798in}{1.364065in}}%
\pgfpathlineto{\pgfqpoint{1.439540in}{1.356600in}}%
\pgfpathlineto{\pgfqpoint{1.438177in}{1.352616in}}%
\pgfpathlineto{\pgfqpoint{1.436554in}{1.348652in}}%
\pgfpathlineto{\pgfqpoint{1.434673in}{1.344713in}}%
\pgfpathlineto{\pgfqpoint{1.432533in}{1.340802in}}%
\pgfpathlineto{\pgfqpoint{1.428884in}{1.348513in}}%
\pgfpathlineto{\pgfqpoint{1.425232in}{1.356169in}}%
\pgfpathlineto{\pgfqpoint{1.421578in}{1.363769in}}%
\pgfpathlineto{\pgfqpoint{1.417920in}{1.371311in}}%
\pgfpathlineto{\pgfqpoint{1.419944in}{1.374979in}}%
\pgfpathlineto{\pgfqpoint{1.421725in}{1.378674in}}%
\pgfpathlineto{\pgfqpoint{1.423262in}{1.382392in}}%
\pgfpathlineto{\pgfqpoint{1.424554in}{1.386130in}}%
\pgfpathclose%
\pgfusepath{fill}%
\end{pgfscope}%
\begin{pgfscope}%
\pgfpathrectangle{\pgfqpoint{0.041670in}{0.041670in}}{\pgfqpoint{2.216660in}{2.216660in}}%
\pgfusepath{clip}%
\pgfsetbuttcap%
\pgfsetroundjoin%
\definecolor{currentfill}{rgb}{0.263663,0.237631,0.518762}%
\pgfsetfillcolor{currentfill}%
\pgfsetlinewidth{0.000000pt}%
\definecolor{currentstroke}{rgb}{0.000000,0.000000,0.000000}%
\pgfsetstrokecolor{currentstroke}%
\pgfsetdash{}{0pt}%
\pgfpathmoveto{\pgfqpoint{0.874827in}{0.854449in}}%
\pgfpathlineto{\pgfqpoint{0.872232in}{0.846424in}}%
\pgfpathlineto{\pgfqpoint{0.869635in}{0.838490in}}%
\pgfpathlineto{\pgfqpoint{0.867038in}{0.830651in}}%
\pgfpathlineto{\pgfqpoint{0.864440in}{0.822910in}}%
\pgfpathlineto{\pgfqpoint{0.854009in}{0.828234in}}%
\pgfpathlineto{\pgfqpoint{0.843926in}{0.833723in}}%
\pgfpathlineto{\pgfqpoint{0.834201in}{0.839372in}}%
\pgfpathlineto{\pgfqpoint{0.824843in}{0.845175in}}%
\pgfpathlineto{\pgfqpoint{0.827759in}{0.852710in}}%
\pgfpathlineto{\pgfqpoint{0.830673in}{0.860343in}}%
\pgfpathlineto{\pgfqpoint{0.833587in}{0.868071in}}%
\pgfpathlineto{\pgfqpoint{0.836500in}{0.875890in}}%
\pgfpathlineto{\pgfqpoint{0.845559in}{0.870301in}}%
\pgfpathlineto{\pgfqpoint{0.854973in}{0.864861in}}%
\pgfpathlineto{\pgfqpoint{0.864732in}{0.859575in}}%
\pgfpathlineto{\pgfqpoint{0.874827in}{0.854449in}}%
\pgfpathclose%
\pgfusepath{fill}%
\end{pgfscope}%
\begin{pgfscope}%
\pgfpathrectangle{\pgfqpoint{0.041670in}{0.041670in}}{\pgfqpoint{2.216660in}{2.216660in}}%
\pgfusepath{clip}%
\pgfsetbuttcap%
\pgfsetroundjoin%
\definecolor{currentfill}{rgb}{0.344074,0.780029,0.397381}%
\pgfsetfillcolor{currentfill}%
\pgfsetlinewidth{0.000000pt}%
\definecolor{currentstroke}{rgb}{0.000000,0.000000,0.000000}%
\pgfsetstrokecolor{currentstroke}%
\pgfsetdash{}{0pt}%
\pgfpathmoveto{\pgfqpoint{1.396878in}{1.455276in}}%
\pgfpathlineto{\pgfqpoint{1.400686in}{1.448762in}}%
\pgfpathlineto{\pgfqpoint{1.404492in}{1.442175in}}%
\pgfpathlineto{\pgfqpoint{1.408294in}{1.435515in}}%
\pgfpathlineto{\pgfqpoint{1.412094in}{1.428785in}}%
\pgfpathlineto{\pgfqpoint{1.411798in}{1.425253in}}%
\pgfpathlineto{\pgfqpoint{1.411271in}{1.421725in}}%
\pgfpathlineto{\pgfqpoint{1.410513in}{1.418206in}}%
\pgfpathlineto{\pgfqpoint{1.409523in}{1.414697in}}%
\pgfpathlineto{\pgfqpoint{1.405758in}{1.421673in}}%
\pgfpathlineto{\pgfqpoint{1.401990in}{1.428577in}}%
\pgfpathlineto{\pgfqpoint{1.398219in}{1.435409in}}%
\pgfpathlineto{\pgfqpoint{1.394446in}{1.442166in}}%
\pgfpathlineto{\pgfqpoint{1.395378in}{1.445430in}}%
\pgfpathlineto{\pgfqpoint{1.396095in}{1.448706in}}%
\pgfpathlineto{\pgfqpoint{1.396595in}{1.451988in}}%
\pgfpathlineto{\pgfqpoint{1.396878in}{1.455276in}}%
\pgfpathclose%
\pgfusepath{fill}%
\end{pgfscope}%
\begin{pgfscope}%
\pgfpathrectangle{\pgfqpoint{0.041670in}{0.041670in}}{\pgfqpoint{2.216660in}{2.216660in}}%
\pgfusepath{clip}%
\pgfsetbuttcap%
\pgfsetroundjoin%
\definecolor{currentfill}{rgb}{0.762373,0.876424,0.137064}%
\pgfsetfillcolor{currentfill}%
\pgfsetlinewidth{0.000000pt}%
\definecolor{currentstroke}{rgb}{0.000000,0.000000,0.000000}%
\pgfsetstrokecolor{currentstroke}%
\pgfsetdash{}{0pt}%
\pgfpathmoveto{\pgfqpoint{1.067315in}{1.611685in}}%
\pgfpathlineto{\pgfqpoint{1.063831in}{1.608337in}}%
\pgfpathlineto{\pgfqpoint{1.060350in}{1.604877in}}%
\pgfpathlineto{\pgfqpoint{1.056872in}{1.601305in}}%
\pgfpathlineto{\pgfqpoint{1.053397in}{1.597624in}}%
\pgfpathlineto{\pgfqpoint{1.055357in}{1.599487in}}%
\pgfpathlineto{\pgfqpoint{1.057439in}{1.601319in}}%
\pgfpathlineto{\pgfqpoint{1.059641in}{1.603119in}}%
\pgfpathlineto{\pgfqpoint{1.061962in}{1.604885in}}%
\pgfpathlineto{\pgfqpoint{1.065198in}{1.608363in}}%
\pgfpathlineto{\pgfqpoint{1.068438in}{1.611731in}}%
\pgfpathlineto{\pgfqpoint{1.071680in}{1.614989in}}%
\pgfpathlineto{\pgfqpoint{1.074925in}{1.618134in}}%
\pgfpathlineto{\pgfqpoint{1.072862in}{1.616565in}}%
\pgfpathlineto{\pgfqpoint{1.070905in}{1.614966in}}%
\pgfpathlineto{\pgfqpoint{1.069055in}{1.613339in}}%
\pgfpathlineto{\pgfqpoint{1.067315in}{1.611685in}}%
\pgfpathclose%
\pgfusepath{fill}%
\end{pgfscope}%
\begin{pgfscope}%
\pgfpathrectangle{\pgfqpoint{0.041670in}{0.041670in}}{\pgfqpoint{2.216660in}{2.216660in}}%
\pgfusepath{clip}%
\pgfsetbuttcap%
\pgfsetroundjoin%
\definecolor{currentfill}{rgb}{0.267004,0.004874,0.329415}%
\pgfsetfillcolor{currentfill}%
\pgfsetlinewidth{0.000000pt}%
\definecolor{currentstroke}{rgb}{0.000000,0.000000,0.000000}%
\pgfsetstrokecolor{currentstroke}%
\pgfsetdash{}{0pt}%
\pgfpathmoveto{\pgfqpoint{1.652060in}{0.679367in}}%
\pgfpathlineto{\pgfqpoint{1.655175in}{0.678337in}}%
\pgfpathlineto{\pgfqpoint{1.658298in}{0.677567in}}%
\pgfpathlineto{\pgfqpoint{1.661428in}{0.677062in}}%
\pgfpathlineto{\pgfqpoint{1.664566in}{0.676826in}}%
\pgfpathlineto{\pgfqpoint{1.652653in}{0.668715in}}%
\pgfpathlineto{\pgfqpoint{1.640229in}{0.660801in}}%
\pgfpathlineto{\pgfqpoint{1.627306in}{0.653094in}}%
\pgfpathlineto{\pgfqpoint{1.613896in}{0.645604in}}%
\pgfpathlineto{\pgfqpoint{1.611074in}{0.646043in}}%
\pgfpathlineto{\pgfqpoint{1.608259in}{0.646754in}}%
\pgfpathlineto{\pgfqpoint{1.605452in}{0.647729in}}%
\pgfpathlineto{\pgfqpoint{1.602651in}{0.648964in}}%
\pgfpathlineto{\pgfqpoint{1.615725in}{0.656258in}}%
\pgfpathlineto{\pgfqpoint{1.628326in}{0.663762in}}%
\pgfpathlineto{\pgfqpoint{1.640442in}{0.671468in}}%
\pgfpathlineto{\pgfqpoint{1.652060in}{0.679367in}}%
\pgfpathclose%
\pgfusepath{fill}%
\end{pgfscope}%
\begin{pgfscope}%
\pgfpathrectangle{\pgfqpoint{0.041670in}{0.041670in}}{\pgfqpoint{2.216660in}{2.216660in}}%
\pgfusepath{clip}%
\pgfsetbuttcap%
\pgfsetroundjoin%
\definecolor{currentfill}{rgb}{0.277941,0.056324,0.381191}%
\pgfsetfillcolor{currentfill}%
\pgfsetlinewidth{0.000000pt}%
\definecolor{currentstroke}{rgb}{0.000000,0.000000,0.000000}%
\pgfsetstrokecolor{currentstroke}%
\pgfsetdash{}{0pt}%
\pgfpathmoveto{\pgfqpoint{0.681021in}{0.677695in}}%
\pgfpathlineto{\pgfqpoint{0.677867in}{0.680082in}}%
\pgfpathlineto{\pgfqpoint{0.674704in}{0.682793in}}%
\pgfpathlineto{\pgfqpoint{0.671529in}{0.685835in}}%
\pgfpathlineto{\pgfqpoint{0.668344in}{0.689214in}}%
\pgfpathlineto{\pgfqpoint{0.655595in}{0.697967in}}%
\pgfpathlineto{\pgfqpoint{0.643414in}{0.706920in}}%
\pgfpathlineto{\pgfqpoint{0.631812in}{0.716062in}}%
\pgfpathlineto{\pgfqpoint{0.620799in}{0.725384in}}%
\pgfpathlineto{\pgfqpoint{0.624264in}{0.721798in}}%
\pgfpathlineto{\pgfqpoint{0.627716in}{0.718547in}}%
\pgfpathlineto{\pgfqpoint{0.631157in}{0.715626in}}%
\pgfpathlineto{\pgfqpoint{0.634588in}{0.713028in}}%
\pgfpathlineto{\pgfqpoint{0.645345in}{0.703922in}}%
\pgfpathlineto{\pgfqpoint{0.656677in}{0.694990in}}%
\pgfpathlineto{\pgfqpoint{0.668572in}{0.686245in}}%
\pgfpathlineto{\pgfqpoint{0.681021in}{0.677695in}}%
\pgfpathclose%
\pgfusepath{fill}%
\end{pgfscope}%
\begin{pgfscope}%
\pgfpathrectangle{\pgfqpoint{0.041670in}{0.041670in}}{\pgfqpoint{2.216660in}{2.216660in}}%
\pgfusepath{clip}%
\pgfsetbuttcap%
\pgfsetroundjoin%
\definecolor{currentfill}{rgb}{0.134692,0.658636,0.517649}%
\pgfsetfillcolor{currentfill}%
\pgfsetlinewidth{0.000000pt}%
\definecolor{currentstroke}{rgb}{0.000000,0.000000,0.000000}%
\pgfsetstrokecolor{currentstroke}%
\pgfsetdash{}{0pt}%
\pgfpathmoveto{\pgfqpoint{1.447105in}{1.309468in}}%
\pgfpathlineto{\pgfqpoint{1.450742in}{1.301523in}}%
\pgfpathlineto{\pgfqpoint{1.454376in}{1.293538in}}%
\pgfpathlineto{\pgfqpoint{1.458007in}{1.285516in}}%
\pgfpathlineto{\pgfqpoint{1.461636in}{1.277459in}}%
\pgfpathlineto{\pgfqpoint{1.458983in}{1.273094in}}%
\pgfpathlineto{\pgfqpoint{1.456047in}{1.268769in}}%
\pgfpathlineto{\pgfqpoint{1.452830in}{1.264490in}}%
\pgfpathlineto{\pgfqpoint{1.449335in}{1.260260in}}%
\pgfpathlineto{\pgfqpoint{1.445856in}{1.268559in}}%
\pgfpathlineto{\pgfqpoint{1.442374in}{1.276823in}}%
\pgfpathlineto{\pgfqpoint{1.438891in}{1.285049in}}%
\pgfpathlineto{\pgfqpoint{1.435405in}{1.293235in}}%
\pgfpathlineto{\pgfqpoint{1.438728in}{1.297226in}}%
\pgfpathlineto{\pgfqpoint{1.441788in}{1.301265in}}%
\pgfpathlineto{\pgfqpoint{1.444581in}{1.305347in}}%
\pgfpathlineto{\pgfqpoint{1.447105in}{1.309468in}}%
\pgfpathclose%
\pgfusepath{fill}%
\end{pgfscope}%
\begin{pgfscope}%
\pgfpathrectangle{\pgfqpoint{0.041670in}{0.041670in}}{\pgfqpoint{2.216660in}{2.216660in}}%
\pgfusepath{clip}%
\pgfsetbuttcap%
\pgfsetroundjoin%
\definecolor{currentfill}{rgb}{0.267004,0.004874,0.329415}%
\pgfsetfillcolor{currentfill}%
\pgfsetlinewidth{0.000000pt}%
\definecolor{currentstroke}{rgb}{0.000000,0.000000,0.000000}%
\pgfsetstrokecolor{currentstroke}%
\pgfsetdash{}{0pt}%
\pgfpathmoveto{\pgfqpoint{0.780099in}{0.650281in}}%
\pgfpathlineto{\pgfqpoint{0.777400in}{0.648011in}}%
\pgfpathlineto{\pgfqpoint{0.774695in}{0.645982in}}%
\pgfpathlineto{\pgfqpoint{0.771984in}{0.644199in}}%
\pgfpathlineto{\pgfqpoint{0.769267in}{0.642665in}}%
\pgfpathlineto{\pgfqpoint{0.755783in}{0.649764in}}%
\pgfpathlineto{\pgfqpoint{0.742761in}{0.657081in}}%
\pgfpathlineto{\pgfqpoint{0.730213in}{0.664608in}}%
\pgfpathlineto{\pgfqpoint{0.718152in}{0.672336in}}%
\pgfpathlineto{\pgfqpoint{0.721196in}{0.673668in}}%
\pgfpathlineto{\pgfqpoint{0.724232in}{0.675250in}}%
\pgfpathlineto{\pgfqpoint{0.727262in}{0.677076in}}%
\pgfpathlineto{\pgfqpoint{0.730286in}{0.679143in}}%
\pgfpathlineto{\pgfqpoint{0.742042in}{0.671625in}}%
\pgfpathlineto{\pgfqpoint{0.754271in}{0.664303in}}%
\pgfpathlineto{\pgfqpoint{0.766961in}{0.657186in}}%
\pgfpathlineto{\pgfqpoint{0.780099in}{0.650281in}}%
\pgfpathclose%
\pgfusepath{fill}%
\end{pgfscope}%
\begin{pgfscope}%
\pgfpathrectangle{\pgfqpoint{0.041670in}{0.041670in}}{\pgfqpoint{2.216660in}{2.216660in}}%
\pgfusepath{clip}%
\pgfsetbuttcap%
\pgfsetroundjoin%
\definecolor{currentfill}{rgb}{0.814576,0.883393,0.110347}%
\pgfsetfillcolor{currentfill}%
\pgfsetlinewidth{0.000000pt}%
\definecolor{currentstroke}{rgb}{0.000000,0.000000,0.000000}%
\pgfsetstrokecolor{currentstroke}%
\pgfsetdash{}{0pt}%
\pgfpathmoveto{\pgfqpoint{1.261890in}{1.635854in}}%
\pgfpathlineto{\pgfqpoint{1.264786in}{1.633395in}}%
\pgfpathlineto{\pgfqpoint{1.267680in}{1.630819in}}%
\pgfpathlineto{\pgfqpoint{1.270571in}{1.628127in}}%
\pgfpathlineto{\pgfqpoint{1.273459in}{1.625320in}}%
\pgfpathlineto{\pgfqpoint{1.276004in}{1.623920in}}%
\pgfpathlineto{\pgfqpoint{1.278455in}{1.622483in}}%
\pgfpathlineto{\pgfqpoint{1.280810in}{1.621009in}}%
\pgfpathlineto{\pgfqpoint{1.283065in}{1.619502in}}%
\pgfpathlineto{\pgfqpoint{1.279877in}{1.622491in}}%
\pgfpathlineto{\pgfqpoint{1.276687in}{1.625365in}}%
\pgfpathlineto{\pgfqpoint{1.273494in}{1.628123in}}%
\pgfpathlineto{\pgfqpoint{1.270298in}{1.630764in}}%
\pgfpathlineto{\pgfqpoint{1.268324in}{1.632083in}}%
\pgfpathlineto{\pgfqpoint{1.266264in}{1.633372in}}%
\pgfpathlineto{\pgfqpoint{1.264118in}{1.634629in}}%
\pgfpathlineto{\pgfqpoint{1.261890in}{1.635854in}}%
\pgfpathclose%
\pgfusepath{fill}%
\end{pgfscope}%
\begin{pgfscope}%
\pgfpathrectangle{\pgfqpoint{0.041670in}{0.041670in}}{\pgfqpoint{2.216660in}{2.216660in}}%
\pgfusepath{clip}%
\pgfsetbuttcap%
\pgfsetroundjoin%
\definecolor{currentfill}{rgb}{0.565498,0.842430,0.262877}%
\pgfsetfillcolor{currentfill}%
\pgfsetlinewidth{0.000000pt}%
\definecolor{currentstroke}{rgb}{0.000000,0.000000,0.000000}%
\pgfsetstrokecolor{currentstroke}%
\pgfsetdash{}{0pt}%
\pgfpathmoveto{\pgfqpoint{1.009306in}{1.534909in}}%
\pgfpathlineto{\pgfqpoint{1.005475in}{1.529620in}}%
\pgfpathlineto{\pgfqpoint{1.001646in}{1.524236in}}%
\pgfpathlineto{\pgfqpoint{0.997821in}{1.518757in}}%
\pgfpathlineto{\pgfqpoint{0.993999in}{1.513186in}}%
\pgfpathlineto{\pgfqpoint{0.994506in}{1.515976in}}%
\pgfpathlineto{\pgfqpoint{0.995196in}{1.518756in}}%
\pgfpathlineto{\pgfqpoint{0.996070in}{1.521523in}}%
\pgfpathlineto{\pgfqpoint{0.997125in}{1.524274in}}%
\pgfpathlineto{\pgfqpoint{1.000877in}{1.529609in}}%
\pgfpathlineto{\pgfqpoint{1.004632in}{1.534851in}}%
\pgfpathlineto{\pgfqpoint{1.008390in}{1.539999in}}%
\pgfpathlineto{\pgfqpoint{1.012151in}{1.545052in}}%
\pgfpathlineto{\pgfqpoint{1.011188in}{1.542535in}}%
\pgfpathlineto{\pgfqpoint{1.010392in}{1.540004in}}%
\pgfpathlineto{\pgfqpoint{1.009764in}{1.537461in}}%
\pgfpathlineto{\pgfqpoint{1.009306in}{1.534909in}}%
\pgfpathclose%
\pgfusepath{fill}%
\end{pgfscope}%
\begin{pgfscope}%
\pgfpathrectangle{\pgfqpoint{0.041670in}{0.041670in}}{\pgfqpoint{2.216660in}{2.216660in}}%
\pgfusepath{clip}%
\pgfsetbuttcap%
\pgfsetroundjoin%
\definecolor{currentfill}{rgb}{0.855810,0.888601,0.097452}%
\pgfsetfillcolor{currentfill}%
\pgfsetlinewidth{0.000000pt}%
\definecolor{currentstroke}{rgb}{0.000000,0.000000,0.000000}%
\pgfsetstrokecolor{currentstroke}%
\pgfsetdash{}{0pt}%
\pgfpathmoveto{\pgfqpoint{1.232682in}{1.651784in}}%
\pgfpathlineto{\pgfqpoint{1.234858in}{1.650106in}}%
\pgfpathlineto{\pgfqpoint{1.237033in}{1.648307in}}%
\pgfpathlineto{\pgfqpoint{1.239205in}{1.646387in}}%
\pgfpathlineto{\pgfqpoint{1.241375in}{1.644349in}}%
\pgfpathlineto{\pgfqpoint{1.244177in}{1.643424in}}%
\pgfpathlineto{\pgfqpoint{1.246917in}{1.642457in}}%
\pgfpathlineto{\pgfqpoint{1.249592in}{1.641451in}}%
\pgfpathlineto{\pgfqpoint{1.252198in}{1.640406in}}%
\pgfpathlineto{\pgfqpoint{1.249644in}{1.642585in}}%
\pgfpathlineto{\pgfqpoint{1.247088in}{1.644645in}}%
\pgfpathlineto{\pgfqpoint{1.244529in}{1.646585in}}%
\pgfpathlineto{\pgfqpoint{1.241968in}{1.648403in}}%
\pgfpathlineto{\pgfqpoint{1.239733in}{1.649300in}}%
\pgfpathlineto{\pgfqpoint{1.237438in}{1.650163in}}%
\pgfpathlineto{\pgfqpoint{1.235087in}{1.650991in}}%
\pgfpathlineto{\pgfqpoint{1.232682in}{1.651784in}}%
\pgfpathclose%
\pgfusepath{fill}%
\end{pgfscope}%
\begin{pgfscope}%
\pgfpathrectangle{\pgfqpoint{0.041670in}{0.041670in}}{\pgfqpoint{2.216660in}{2.216660in}}%
\pgfusepath{clip}%
\pgfsetbuttcap%
\pgfsetroundjoin%
\definecolor{currentfill}{rgb}{0.133743,0.548535,0.553541}%
\pgfsetfillcolor{currentfill}%
\pgfsetlinewidth{0.000000pt}%
\definecolor{currentstroke}{rgb}{0.000000,0.000000,0.000000}%
\pgfsetstrokecolor{currentstroke}%
\pgfsetdash{}{0pt}%
\pgfpathmoveto{\pgfqpoint{0.905887in}{1.170823in}}%
\pgfpathlineto{\pgfqpoint{0.902685in}{1.162037in}}%
\pgfpathlineto{\pgfqpoint{0.899485in}{1.153242in}}%
\pgfpathlineto{\pgfqpoint{0.896286in}{1.144439in}}%
\pgfpathlineto{\pgfqpoint{0.893089in}{1.135633in}}%
\pgfpathlineto{\pgfqpoint{0.887564in}{1.140234in}}%
\pgfpathlineto{\pgfqpoint{0.882341in}{1.144917in}}%
\pgfpathlineto{\pgfqpoint{0.877425in}{1.149677in}}%
\pgfpathlineto{\pgfqpoint{0.872820in}{1.154509in}}%
\pgfpathlineto{\pgfqpoint{0.876234in}{1.163081in}}%
\pgfpathlineto{\pgfqpoint{0.879650in}{1.171651in}}%
\pgfpathlineto{\pgfqpoint{0.883067in}{1.180213in}}%
\pgfpathlineto{\pgfqpoint{0.886487in}{1.188767in}}%
\pgfpathlineto{\pgfqpoint{0.890895in}{1.184173in}}%
\pgfpathlineto{\pgfqpoint{0.895601in}{1.179647in}}%
\pgfpathlineto{\pgfqpoint{0.900600in}{1.175196in}}%
\pgfpathlineto{\pgfqpoint{0.905887in}{1.170823in}}%
\pgfpathclose%
\pgfusepath{fill}%
\end{pgfscope}%
\begin{pgfscope}%
\pgfpathrectangle{\pgfqpoint{0.041670in}{0.041670in}}{\pgfqpoint{2.216660in}{2.216660in}}%
\pgfusepath{clip}%
\pgfsetbuttcap%
\pgfsetroundjoin%
\definecolor{currentfill}{rgb}{0.487026,0.823929,0.312321}%
\pgfsetfillcolor{currentfill}%
\pgfsetlinewidth{0.000000pt}%
\definecolor{currentstroke}{rgb}{0.000000,0.000000,0.000000}%
\pgfsetstrokecolor{currentstroke}%
\pgfsetdash{}{0pt}%
\pgfpathmoveto{\pgfqpoint{1.365470in}{1.515666in}}%
\pgfpathlineto{\pgfqpoint{1.369278in}{1.510057in}}%
\pgfpathlineto{\pgfqpoint{1.373084in}{1.504359in}}%
\pgfpathlineto{\pgfqpoint{1.376886in}{1.498574in}}%
\pgfpathlineto{\pgfqpoint{1.380685in}{1.492703in}}%
\pgfpathlineto{\pgfqpoint{1.381218in}{1.489672in}}%
\pgfpathlineto{\pgfqpoint{1.381550in}{1.486634in}}%
\pgfpathlineto{\pgfqpoint{1.381682in}{1.483590in}}%
\pgfpathlineto{\pgfqpoint{1.381613in}{1.480545in}}%
\pgfpathlineto{\pgfqpoint{1.377790in}{1.486657in}}%
\pgfpathlineto{\pgfqpoint{1.373963in}{1.492683in}}%
\pgfpathlineto{\pgfqpoint{1.370134in}{1.498621in}}%
\pgfpathlineto{\pgfqpoint{1.366301in}{1.504470in}}%
\pgfpathlineto{\pgfqpoint{1.366372in}{1.507274in}}%
\pgfpathlineto{\pgfqpoint{1.366256in}{1.510077in}}%
\pgfpathlineto{\pgfqpoint{1.365955in}{1.512875in}}%
\pgfpathlineto{\pgfqpoint{1.365470in}{1.515666in}}%
\pgfpathclose%
\pgfusepath{fill}%
\end{pgfscope}%
\begin{pgfscope}%
\pgfpathrectangle{\pgfqpoint{0.041670in}{0.041670in}}{\pgfqpoint{2.216660in}{2.216660in}}%
\pgfusepath{clip}%
\pgfsetbuttcap%
\pgfsetroundjoin%
\definecolor{currentfill}{rgb}{0.179019,0.433756,0.557430}%
\pgfsetfillcolor{currentfill}%
\pgfsetlinewidth{0.000000pt}%
\definecolor{currentstroke}{rgb}{0.000000,0.000000,0.000000}%
\pgfsetstrokecolor{currentstroke}%
\pgfsetdash{}{0pt}%
\pgfpathmoveto{\pgfqpoint{0.894743in}{1.046153in}}%
\pgfpathlineto{\pgfqpoint{0.891827in}{1.037221in}}%
\pgfpathlineto{\pgfqpoint{0.888912in}{1.028314in}}%
\pgfpathlineto{\pgfqpoint{0.885997in}{1.019434in}}%
\pgfpathlineto{\pgfqpoint{0.883083in}{1.010587in}}%
\pgfpathlineto{\pgfqpoint{0.875534in}{1.015448in}}%
\pgfpathlineto{\pgfqpoint{0.868304in}{1.020425in}}%
\pgfpathlineto{\pgfqpoint{0.861400in}{1.025513in}}%
\pgfpathlineto{\pgfqpoint{0.854828in}{1.030706in}}%
\pgfpathlineto{\pgfqpoint{0.858010in}{1.039332in}}%
\pgfpathlineto{\pgfqpoint{0.861194in}{1.047990in}}%
\pgfpathlineto{\pgfqpoint{0.864378in}{1.056676in}}%
\pgfpathlineto{\pgfqpoint{0.867563in}{1.065387in}}%
\pgfpathlineto{\pgfqpoint{0.873886in}{1.060422in}}%
\pgfpathlineto{\pgfqpoint{0.880528in}{1.055558in}}%
\pgfpathlineto{\pgfqpoint{0.887483in}{1.050800in}}%
\pgfpathlineto{\pgfqpoint{0.894743in}{1.046153in}}%
\pgfpathclose%
\pgfusepath{fill}%
\end{pgfscope}%
\begin{pgfscope}%
\pgfpathrectangle{\pgfqpoint{0.041670in}{0.041670in}}{\pgfqpoint{2.216660in}{2.216660in}}%
\pgfusepath{clip}%
\pgfsetbuttcap%
\pgfsetroundjoin%
\definecolor{currentfill}{rgb}{0.855810,0.888601,0.097452}%
\pgfsetfillcolor{currentfill}%
\pgfsetlinewidth{0.000000pt}%
\definecolor{currentstroke}{rgb}{0.000000,0.000000,0.000000}%
\pgfsetstrokecolor{currentstroke}%
\pgfsetdash{}{0pt}%
\pgfpathmoveto{\pgfqpoint{1.116005in}{1.647579in}}%
\pgfpathlineto{\pgfqpoint{1.113364in}{1.645726in}}%
\pgfpathlineto{\pgfqpoint{1.110725in}{1.643752in}}%
\pgfpathlineto{\pgfqpoint{1.108089in}{1.641658in}}%
\pgfpathlineto{\pgfqpoint{1.105455in}{1.639444in}}%
\pgfpathlineto{\pgfqpoint{1.107998in}{1.640524in}}%
\pgfpathlineto{\pgfqpoint{1.110612in}{1.641565in}}%
\pgfpathlineto{\pgfqpoint{1.113294in}{1.642567in}}%
\pgfpathlineto{\pgfqpoint{1.116041in}{1.643528in}}%
\pgfpathlineto{\pgfqpoint{1.118299in}{1.645596in}}%
\pgfpathlineto{\pgfqpoint{1.120560in}{1.647545in}}%
\pgfpathlineto{\pgfqpoint{1.122822in}{1.649373in}}%
\pgfpathlineto{\pgfqpoint{1.125087in}{1.651081in}}%
\pgfpathlineto{\pgfqpoint{1.122730in}{1.650256in}}%
\pgfpathlineto{\pgfqpoint{1.120429in}{1.649397in}}%
\pgfpathlineto{\pgfqpoint{1.118187in}{1.648505in}}%
\pgfpathlineto{\pgfqpoint{1.116005in}{1.647579in}}%
\pgfpathclose%
\pgfusepath{fill}%
\end{pgfscope}%
\begin{pgfscope}%
\pgfpathrectangle{\pgfqpoint{0.041670in}{0.041670in}}{\pgfqpoint{2.216660in}{2.216660in}}%
\pgfusepath{clip}%
\pgfsetbuttcap%
\pgfsetroundjoin%
\definecolor{currentfill}{rgb}{0.814576,0.883393,0.110347}%
\pgfsetfillcolor{currentfill}%
\pgfsetlinewidth{0.000000pt}%
\definecolor{currentstroke}{rgb}{0.000000,0.000000,0.000000}%
\pgfsetstrokecolor{currentstroke}%
\pgfsetdash{}{0pt}%
\pgfpathmoveto{\pgfqpoint{1.087932in}{1.629568in}}%
\pgfpathlineto{\pgfqpoint{1.084676in}{1.626884in}}%
\pgfpathlineto{\pgfqpoint{1.081423in}{1.624083in}}%
\pgfpathlineto{\pgfqpoint{1.078173in}{1.621166in}}%
\pgfpathlineto{\pgfqpoint{1.074925in}{1.618134in}}%
\pgfpathlineto{\pgfqpoint{1.077090in}{1.619671in}}%
\pgfpathlineto{\pgfqpoint{1.079357in}{1.621175in}}%
\pgfpathlineto{\pgfqpoint{1.081722in}{1.622644in}}%
\pgfpathlineto{\pgfqpoint{1.084184in}{1.624077in}}%
\pgfpathlineto{\pgfqpoint{1.087143in}{1.626923in}}%
\pgfpathlineto{\pgfqpoint{1.090104in}{1.629654in}}%
\pgfpathlineto{\pgfqpoint{1.093068in}{1.632269in}}%
\pgfpathlineto{\pgfqpoint{1.096035in}{1.634767in}}%
\pgfpathlineto{\pgfqpoint{1.093880in}{1.633513in}}%
\pgfpathlineto{\pgfqpoint{1.091810in}{1.632228in}}%
\pgfpathlineto{\pgfqpoint{1.089827in}{1.630912in}}%
\pgfpathlineto{\pgfqpoint{1.087932in}{1.629568in}}%
\pgfpathclose%
\pgfusepath{fill}%
\end{pgfscope}%
\begin{pgfscope}%
\pgfpathrectangle{\pgfqpoint{0.041670in}{0.041670in}}{\pgfqpoint{2.216660in}{2.216660in}}%
\pgfusepath{clip}%
\pgfsetbuttcap%
\pgfsetroundjoin%
\definecolor{currentfill}{rgb}{0.231674,0.318106,0.544834}%
\pgfsetfillcolor{currentfill}%
\pgfsetlinewidth{0.000000pt}%
\definecolor{currentstroke}{rgb}{0.000000,0.000000,0.000000}%
\pgfsetstrokecolor{currentstroke}%
\pgfsetdash{}{0pt}%
\pgfpathmoveto{\pgfqpoint{1.507358in}{0.946006in}}%
\pgfpathlineto{\pgfqpoint{1.510333in}{0.937631in}}%
\pgfpathlineto{\pgfqpoint{1.513307in}{0.929319in}}%
\pgfpathlineto{\pgfqpoint{1.516282in}{0.921075in}}%
\pgfpathlineto{\pgfqpoint{1.519256in}{0.912901in}}%
\pgfpathlineto{\pgfqpoint{1.510807in}{0.907404in}}%
\pgfpathlineto{\pgfqpoint{1.502008in}{0.902045in}}%
\pgfpathlineto{\pgfqpoint{1.492868in}{0.896829in}}%
\pgfpathlineto{\pgfqpoint{1.483396in}{0.891763in}}%
\pgfpathlineto{\pgfqpoint{1.480728in}{0.900146in}}%
\pgfpathlineto{\pgfqpoint{1.478061in}{0.908600in}}%
\pgfpathlineto{\pgfqpoint{1.475393in}{0.917121in}}%
\pgfpathlineto{\pgfqpoint{1.472726in}{0.925705in}}%
\pgfpathlineto{\pgfqpoint{1.481872in}{0.930571in}}%
\pgfpathlineto{\pgfqpoint{1.490699in}{0.935580in}}%
\pgfpathlineto{\pgfqpoint{1.499197in}{0.940727in}}%
\pgfpathlineto{\pgfqpoint{1.507358in}{0.946006in}}%
\pgfpathclose%
\pgfusepath{fill}%
\end{pgfscope}%
\begin{pgfscope}%
\pgfpathrectangle{\pgfqpoint{0.041670in}{0.041670in}}{\pgfqpoint{2.216660in}{2.216660in}}%
\pgfusepath{clip}%
\pgfsetbuttcap%
\pgfsetroundjoin%
\definecolor{currentfill}{rgb}{0.220124,0.725509,0.466226}%
\pgfsetfillcolor{currentfill}%
\pgfsetlinewidth{0.000000pt}%
\definecolor{currentstroke}{rgb}{0.000000,0.000000,0.000000}%
\pgfsetstrokecolor{currentstroke}%
\pgfsetdash{}{0pt}%
\pgfpathmoveto{\pgfqpoint{0.943990in}{1.368077in}}%
\pgfpathlineto{\pgfqpoint{0.940362in}{1.360482in}}%
\pgfpathlineto{\pgfqpoint{0.936736in}{1.352828in}}%
\pgfpathlineto{\pgfqpoint{0.933113in}{1.345118in}}%
\pgfpathlineto{\pgfqpoint{0.929492in}{1.337354in}}%
\pgfpathlineto{\pgfqpoint{0.927126in}{1.341235in}}%
\pgfpathlineto{\pgfqpoint{0.925015in}{1.345149in}}%
\pgfpathlineto{\pgfqpoint{0.923162in}{1.349091in}}%
\pgfpathlineto{\pgfqpoint{0.921568in}{1.353057in}}%
\pgfpathlineto{\pgfqpoint{0.925295in}{1.360577in}}%
\pgfpathlineto{\pgfqpoint{0.929025in}{1.368043in}}%
\pgfpathlineto{\pgfqpoint{0.932757in}{1.375454in}}%
\pgfpathlineto{\pgfqpoint{0.936492in}{1.382806in}}%
\pgfpathlineto{\pgfqpoint{0.938002in}{1.379086in}}%
\pgfpathlineto{\pgfqpoint{0.939756in}{1.375388in}}%
\pgfpathlineto{\pgfqpoint{0.941753in}{1.371717in}}%
\pgfpathlineto{\pgfqpoint{0.943990in}{1.368077in}}%
\pgfpathclose%
\pgfusepath{fill}%
\end{pgfscope}%
\begin{pgfscope}%
\pgfpathrectangle{\pgfqpoint{0.041670in}{0.041670in}}{\pgfqpoint{2.216660in}{2.216660in}}%
\pgfusepath{clip}%
\pgfsetbuttcap%
\pgfsetroundjoin%
\definecolor{currentfill}{rgb}{0.896320,0.893616,0.096335}%
\pgfsetfillcolor{currentfill}%
\pgfsetlinewidth{0.000000pt}%
\definecolor{currentstroke}{rgb}{0.000000,0.000000,0.000000}%
\pgfsetstrokecolor{currentstroke}%
\pgfsetdash{}{0pt}%
\pgfpathmoveto{\pgfqpoint{1.170307in}{1.663338in}}%
\pgfpathlineto{\pgfqpoint{1.169828in}{1.662455in}}%
\pgfpathlineto{\pgfqpoint{1.169350in}{1.661448in}}%
\pgfpathlineto{\pgfqpoint{1.168873in}{1.660316in}}%
\pgfpathlineto{\pgfqpoint{1.168396in}{1.659061in}}%
\pgfpathlineto{\pgfqpoint{1.171317in}{1.659210in}}%
\pgfpathlineto{\pgfqpoint{1.174246in}{1.659315in}}%
\pgfpathlineto{\pgfqpoint{1.177181in}{1.659378in}}%
\pgfpathlineto{\pgfqpoint{1.180118in}{1.659397in}}%
\pgfpathlineto{\pgfqpoint{1.180111in}{1.660637in}}%
\pgfpathlineto{\pgfqpoint{1.180105in}{1.661755in}}%
\pgfpathlineto{\pgfqpoint{1.180098in}{1.662749in}}%
\pgfpathlineto{\pgfqpoint{1.180091in}{1.663618in}}%
\pgfpathlineto{\pgfqpoint{1.177639in}{1.663602in}}%
\pgfpathlineto{\pgfqpoint{1.175190in}{1.663550in}}%
\pgfpathlineto{\pgfqpoint{1.172745in}{1.663462in}}%
\pgfpathlineto{\pgfqpoint{1.170307in}{1.663338in}}%
\pgfpathclose%
\pgfusepath{fill}%
\end{pgfscope}%
\begin{pgfscope}%
\pgfpathrectangle{\pgfqpoint{0.041670in}{0.041670in}}{\pgfqpoint{2.216660in}{2.216660in}}%
\pgfusepath{clip}%
\pgfsetbuttcap%
\pgfsetroundjoin%
\definecolor{currentfill}{rgb}{0.896320,0.893616,0.096335}%
\pgfsetfillcolor{currentfill}%
\pgfsetlinewidth{0.000000pt}%
\definecolor{currentstroke}{rgb}{0.000000,0.000000,0.000000}%
\pgfsetstrokecolor{currentstroke}%
\pgfsetdash{}{0pt}%
\pgfpathmoveto{\pgfqpoint{1.180091in}{1.663618in}}%
\pgfpathlineto{\pgfqpoint{1.180098in}{1.662749in}}%
\pgfpathlineto{\pgfqpoint{1.180105in}{1.661755in}}%
\pgfpathlineto{\pgfqpoint{1.180111in}{1.660637in}}%
\pgfpathlineto{\pgfqpoint{1.180118in}{1.659397in}}%
\pgfpathlineto{\pgfqpoint{1.183055in}{1.659373in}}%
\pgfpathlineto{\pgfqpoint{1.185990in}{1.659306in}}%
\pgfpathlineto{\pgfqpoint{1.188918in}{1.659195in}}%
\pgfpathlineto{\pgfqpoint{1.191838in}{1.659042in}}%
\pgfpathlineto{\pgfqpoint{1.191347in}{1.660298in}}%
\pgfpathlineto{\pgfqpoint{1.190856in}{1.661430in}}%
\pgfpathlineto{\pgfqpoint{1.190365in}{1.662438in}}%
\pgfpathlineto{\pgfqpoint{1.189873in}{1.663322in}}%
\pgfpathlineto{\pgfqpoint{1.187436in}{1.663450in}}%
\pgfpathlineto{\pgfqpoint{1.184992in}{1.663542in}}%
\pgfpathlineto{\pgfqpoint{1.182543in}{1.663598in}}%
\pgfpathlineto{\pgfqpoint{1.180091in}{1.663618in}}%
\pgfpathclose%
\pgfusepath{fill}%
\end{pgfscope}%
\begin{pgfscope}%
\pgfpathrectangle{\pgfqpoint{0.041670in}{0.041670in}}{\pgfqpoint{2.216660in}{2.216660in}}%
\pgfusepath{clip}%
\pgfsetbuttcap%
\pgfsetroundjoin%
\definecolor{currentfill}{rgb}{0.344074,0.780029,0.397381}%
\pgfsetfillcolor{currentfill}%
\pgfsetlinewidth{0.000000pt}%
\definecolor{currentstroke}{rgb}{0.000000,0.000000,0.000000}%
\pgfsetstrokecolor{currentstroke}%
\pgfsetdash{}{0pt}%
\pgfpathmoveto{\pgfqpoint{0.966474in}{1.439276in}}%
\pgfpathlineto{\pgfqpoint{0.962716in}{1.432465in}}%
\pgfpathlineto{\pgfqpoint{0.958962in}{1.425580in}}%
\pgfpathlineto{\pgfqpoint{0.955210in}{1.418621in}}%
\pgfpathlineto{\pgfqpoint{0.951461in}{1.411592in}}%
\pgfpathlineto{\pgfqpoint{0.950266in}{1.415087in}}%
\pgfpathlineto{\pgfqpoint{0.949301in}{1.418596in}}%
\pgfpathlineto{\pgfqpoint{0.948568in}{1.422117in}}%
\pgfpathlineto{\pgfqpoint{0.948067in}{1.425645in}}%
\pgfpathlineto{\pgfqpoint{0.951864in}{1.432430in}}%
\pgfpathlineto{\pgfqpoint{0.955664in}{1.439144in}}%
\pgfpathlineto{\pgfqpoint{0.959467in}{1.445786in}}%
\pgfpathlineto{\pgfqpoint{0.963273in}{1.452354in}}%
\pgfpathlineto{\pgfqpoint{0.963749in}{1.449070in}}%
\pgfpathlineto{\pgfqpoint{0.964441in}{1.445794in}}%
\pgfpathlineto{\pgfqpoint{0.965350in}{1.442528in}}%
\pgfpathlineto{\pgfqpoint{0.966474in}{1.439276in}}%
\pgfpathclose%
\pgfusepath{fill}%
\end{pgfscope}%
\begin{pgfscope}%
\pgfpathrectangle{\pgfqpoint{0.041670in}{0.041670in}}{\pgfqpoint{2.216660in}{2.216660in}}%
\pgfusepath{clip}%
\pgfsetbuttcap%
\pgfsetroundjoin%
\definecolor{currentfill}{rgb}{0.276194,0.190074,0.493001}%
\pgfsetfillcolor{currentfill}%
\pgfsetlinewidth{0.000000pt}%
\definecolor{currentstroke}{rgb}{0.000000,0.000000,0.000000}%
\pgfsetstrokecolor{currentstroke}%
\pgfsetdash{}{0pt}%
\pgfpathmoveto{\pgfqpoint{0.592622in}{0.766872in}}%
\pgfpathlineto{\pgfqpoint{0.589036in}{0.773753in}}%
\pgfpathlineto{\pgfqpoint{0.585436in}{0.781032in}}%
\pgfpathlineto{\pgfqpoint{0.581819in}{0.788717in}}%
\pgfpathlineto{\pgfqpoint{0.578186in}{0.796815in}}%
\pgfpathlineto{\pgfqpoint{0.567040in}{0.806928in}}%
\pgfpathlineto{\pgfqpoint{0.556551in}{0.817207in}}%
\pgfpathlineto{\pgfqpoint{0.546728in}{0.827639in}}%
\pgfpathlineto{\pgfqpoint{0.537579in}{0.838214in}}%
\pgfpathlineto{\pgfqpoint{0.541435in}{0.829914in}}%
\pgfpathlineto{\pgfqpoint{0.545274in}{0.822024in}}%
\pgfpathlineto{\pgfqpoint{0.549097in}{0.814538in}}%
\pgfpathlineto{\pgfqpoint{0.552903in}{0.807449in}}%
\pgfpathlineto{\pgfqpoint{0.561856in}{0.797083in}}%
\pgfpathlineto{\pgfqpoint{0.571465in}{0.786857in}}%
\pgfpathlineto{\pgfqpoint{0.581723in}{0.776783in}}%
\pgfpathlineto{\pgfqpoint{0.592622in}{0.766872in}}%
\pgfpathclose%
\pgfusepath{fill}%
\end{pgfscope}%
\begin{pgfscope}%
\pgfpathrectangle{\pgfqpoint{0.041670in}{0.041670in}}{\pgfqpoint{2.216660in}{2.216660in}}%
\pgfusepath{clip}%
\pgfsetbuttcap%
\pgfsetroundjoin%
\definecolor{currentfill}{rgb}{0.248629,0.278775,0.534556}%
\pgfsetfillcolor{currentfill}%
\pgfsetlinewidth{0.000000pt}%
\definecolor{currentstroke}{rgb}{0.000000,0.000000,0.000000}%
\pgfsetstrokecolor{currentstroke}%
\pgfsetdash{}{0pt}%
\pgfpathmoveto{\pgfqpoint{0.885204in}{0.887389in}}%
\pgfpathlineto{\pgfqpoint{0.882611in}{0.879035in}}%
\pgfpathlineto{\pgfqpoint{0.880017in}{0.870758in}}%
\pgfpathlineto{\pgfqpoint{0.877422in}{0.862562in}}%
\pgfpathlineto{\pgfqpoint{0.874827in}{0.854449in}}%
\pgfpathlineto{\pgfqpoint{0.864732in}{0.859575in}}%
\pgfpathlineto{\pgfqpoint{0.854973in}{0.864861in}}%
\pgfpathlineto{\pgfqpoint{0.845559in}{0.870301in}}%
\pgfpathlineto{\pgfqpoint{0.836500in}{0.875890in}}%
\pgfpathlineto{\pgfqpoint{0.839412in}{0.883797in}}%
\pgfpathlineto{\pgfqpoint{0.842324in}{0.891788in}}%
\pgfpathlineto{\pgfqpoint{0.845236in}{0.899859in}}%
\pgfpathlineto{\pgfqpoint{0.848147in}{0.908008in}}%
\pgfpathlineto{\pgfqpoint{0.856907in}{0.902633in}}%
\pgfpathlineto{\pgfqpoint{0.866009in}{0.897401in}}%
\pgfpathlineto{\pgfqpoint{0.875445in}{0.892318in}}%
\pgfpathlineto{\pgfqpoint{0.885204in}{0.887389in}}%
\pgfpathclose%
\pgfusepath{fill}%
\end{pgfscope}%
\begin{pgfscope}%
\pgfpathrectangle{\pgfqpoint{0.041670in}{0.041670in}}{\pgfqpoint{2.216660in}{2.216660in}}%
\pgfusepath{clip}%
\pgfsetbuttcap%
\pgfsetroundjoin%
\definecolor{currentfill}{rgb}{0.896320,0.893616,0.096335}%
\pgfsetfillcolor{currentfill}%
\pgfsetlinewidth{0.000000pt}%
\definecolor{currentstroke}{rgb}{0.000000,0.000000,0.000000}%
\pgfsetstrokecolor{currentstroke}%
\pgfsetdash{}{0pt}%
\pgfpathmoveto{\pgfqpoint{1.160673in}{1.662487in}}%
\pgfpathlineto{\pgfqpoint{1.159717in}{1.661562in}}%
\pgfpathlineto{\pgfqpoint{1.158762in}{1.660512in}}%
\pgfpathlineto{\pgfqpoint{1.157808in}{1.659338in}}%
\pgfpathlineto{\pgfqpoint{1.156855in}{1.658042in}}%
\pgfpathlineto{\pgfqpoint{1.159713in}{1.658360in}}%
\pgfpathlineto{\pgfqpoint{1.162591in}{1.658636in}}%
\pgfpathlineto{\pgfqpoint{1.165487in}{1.658870in}}%
\pgfpathlineto{\pgfqpoint{1.168396in}{1.659061in}}%
\pgfpathlineto{\pgfqpoint{1.168873in}{1.660316in}}%
\pgfpathlineto{\pgfqpoint{1.169350in}{1.661448in}}%
\pgfpathlineto{\pgfqpoint{1.169828in}{1.662455in}}%
\pgfpathlineto{\pgfqpoint{1.170307in}{1.663338in}}%
\pgfpathlineto{\pgfqpoint{1.167878in}{1.663179in}}%
\pgfpathlineto{\pgfqpoint{1.165462in}{1.662984in}}%
\pgfpathlineto{\pgfqpoint{1.163059in}{1.662753in}}%
\pgfpathlineto{\pgfqpoint{1.160673in}{1.662487in}}%
\pgfpathclose%
\pgfusepath{fill}%
\end{pgfscope}%
\begin{pgfscope}%
\pgfpathrectangle{\pgfqpoint{0.041670in}{0.041670in}}{\pgfqpoint{2.216660in}{2.216660in}}%
\pgfusepath{clip}%
\pgfsetbuttcap%
\pgfsetroundjoin%
\definecolor{currentfill}{rgb}{0.896320,0.893616,0.096335}%
\pgfsetfillcolor{currentfill}%
\pgfsetlinewidth{0.000000pt}%
\definecolor{currentstroke}{rgb}{0.000000,0.000000,0.000000}%
\pgfsetstrokecolor{currentstroke}%
\pgfsetdash{}{0pt}%
\pgfpathmoveto{\pgfqpoint{1.189873in}{1.663322in}}%
\pgfpathlineto{\pgfqpoint{1.190365in}{1.662438in}}%
\pgfpathlineto{\pgfqpoint{1.190856in}{1.661430in}}%
\pgfpathlineto{\pgfqpoint{1.191347in}{1.660298in}}%
\pgfpathlineto{\pgfqpoint{1.191838in}{1.659042in}}%
\pgfpathlineto{\pgfqpoint{1.194746in}{1.658846in}}%
\pgfpathlineto{\pgfqpoint{1.197639in}{1.658608in}}%
\pgfpathlineto{\pgfqpoint{1.200515in}{1.658327in}}%
\pgfpathlineto{\pgfqpoint{1.203371in}{1.658004in}}%
\pgfpathlineto{\pgfqpoint{1.202405in}{1.659302in}}%
\pgfpathlineto{\pgfqpoint{1.201438in}{1.660477in}}%
\pgfpathlineto{\pgfqpoint{1.200470in}{1.661529in}}%
\pgfpathlineto{\pgfqpoint{1.199500in}{1.662456in}}%
\pgfpathlineto{\pgfqpoint{1.197117in}{1.662725in}}%
\pgfpathlineto{\pgfqpoint{1.194716in}{1.662960in}}%
\pgfpathlineto{\pgfqpoint{1.192301in}{1.663159in}}%
\pgfpathlineto{\pgfqpoint{1.189873in}{1.663322in}}%
\pgfpathclose%
\pgfusepath{fill}%
\end{pgfscope}%
\begin{pgfscope}%
\pgfpathrectangle{\pgfqpoint{0.041670in}{0.041670in}}{\pgfqpoint{2.216660in}{2.216660in}}%
\pgfusepath{clip}%
\pgfsetbuttcap%
\pgfsetroundjoin%
\definecolor{currentfill}{rgb}{0.163625,0.471133,0.558148}%
\pgfsetfillcolor{currentfill}%
\pgfsetlinewidth{0.000000pt}%
\definecolor{currentstroke}{rgb}{0.000000,0.000000,0.000000}%
\pgfsetstrokecolor{currentstroke}%
\pgfsetdash{}{0pt}%
\pgfpathmoveto{\pgfqpoint{1.484732in}{1.104715in}}%
\pgfpathlineto{\pgfqpoint{1.487975in}{1.095983in}}%
\pgfpathlineto{\pgfqpoint{1.491216in}{1.087265in}}%
\pgfpathlineto{\pgfqpoint{1.494456in}{1.078563in}}%
\pgfpathlineto{\pgfqpoint{1.497695in}{1.069881in}}%
\pgfpathlineto{\pgfqpoint{1.491660in}{1.064831in}}%
\pgfpathlineto{\pgfqpoint{1.485301in}{1.059877in}}%
\pgfpathlineto{\pgfqpoint{1.478624in}{1.055024in}}%
\pgfpathlineto{\pgfqpoint{1.471635in}{1.050278in}}%
\pgfpathlineto{\pgfqpoint{1.468654in}{1.059185in}}%
\pgfpathlineto{\pgfqpoint{1.465671in}{1.068111in}}%
\pgfpathlineto{\pgfqpoint{1.462688in}{1.077053in}}%
\pgfpathlineto{\pgfqpoint{1.459704in}{1.086008in}}%
\pgfpathlineto{\pgfqpoint{1.466414in}{1.090537in}}%
\pgfpathlineto{\pgfqpoint{1.472827in}{1.095168in}}%
\pgfpathlineto{\pgfqpoint{1.478934in}{1.099896in}}%
\pgfpathlineto{\pgfqpoint{1.484732in}{1.104715in}}%
\pgfpathclose%
\pgfusepath{fill}%
\end{pgfscope}%
\begin{pgfscope}%
\pgfpathrectangle{\pgfqpoint{0.041670in}{0.041670in}}{\pgfqpoint{2.216660in}{2.216660in}}%
\pgfusepath{clip}%
\pgfsetbuttcap%
\pgfsetroundjoin%
\definecolor{currentfill}{rgb}{0.282327,0.094955,0.417331}%
\pgfsetfillcolor{currentfill}%
\pgfsetlinewidth{0.000000pt}%
\definecolor{currentstroke}{rgb}{0.000000,0.000000,0.000000}%
\pgfsetstrokecolor{currentstroke}%
\pgfsetdash{}{0pt}%
\pgfpathmoveto{\pgfqpoint{1.748396in}{0.733811in}}%
\pgfpathlineto{\pgfqpoint{1.751926in}{0.737786in}}%
\pgfpathlineto{\pgfqpoint{1.755469in}{0.742108in}}%
\pgfpathlineto{\pgfqpoint{1.759024in}{0.746783in}}%
\pgfpathlineto{\pgfqpoint{1.762593in}{0.751817in}}%
\pgfpathlineto{\pgfqpoint{1.751870in}{0.742132in}}%
\pgfpathlineto{\pgfqpoint{1.740532in}{0.732620in}}%
\pgfpathlineto{\pgfqpoint{1.728590in}{0.723292in}}%
\pgfpathlineto{\pgfqpoint{1.716054in}{0.714158in}}%
\pgfpathlineto{\pgfqpoint{1.712754in}{0.709330in}}%
\pgfpathlineto{\pgfqpoint{1.709466in}{0.704862in}}%
\pgfpathlineto{\pgfqpoint{1.706190in}{0.700749in}}%
\pgfpathlineto{\pgfqpoint{1.702926in}{0.696984in}}%
\pgfpathlineto{\pgfqpoint{1.715171in}{0.705915in}}%
\pgfpathlineto{\pgfqpoint{1.726838in}{0.715037in}}%
\pgfpathlineto{\pgfqpoint{1.737916in}{0.724339in}}%
\pgfpathlineto{\pgfqpoint{1.748396in}{0.733811in}}%
\pgfpathclose%
\pgfusepath{fill}%
\end{pgfscope}%
\begin{pgfscope}%
\pgfpathrectangle{\pgfqpoint{0.041670in}{0.041670in}}{\pgfqpoint{2.216660in}{2.216660in}}%
\pgfusepath{clip}%
\pgfsetbuttcap%
\pgfsetroundjoin%
\definecolor{currentfill}{rgb}{0.122606,0.585371,0.546557}%
\pgfsetfillcolor{currentfill}%
\pgfsetlinewidth{0.000000pt}%
\definecolor{currentstroke}{rgb}{0.000000,0.000000,0.000000}%
\pgfsetstrokecolor{currentstroke}%
\pgfsetdash{}{0pt}%
\pgfpathmoveto{\pgfqpoint{1.463229in}{1.226762in}}%
\pgfpathlineto{\pgfqpoint{1.466697in}{1.218324in}}%
\pgfpathlineto{\pgfqpoint{1.470163in}{1.209867in}}%
\pgfpathlineto{\pgfqpoint{1.473627in}{1.201393in}}%
\pgfpathlineto{\pgfqpoint{1.477089in}{1.192905in}}%
\pgfpathlineto{\pgfqpoint{1.472948in}{1.188253in}}%
\pgfpathlineto{\pgfqpoint{1.468506in}{1.183666in}}%
\pgfpathlineto{\pgfqpoint{1.463768in}{1.179149in}}%
\pgfpathlineto{\pgfqpoint{1.458737in}{1.174706in}}%
\pgfpathlineto{\pgfqpoint{1.455480in}{1.183429in}}%
\pgfpathlineto{\pgfqpoint{1.452222in}{1.192138in}}%
\pgfpathlineto{\pgfqpoint{1.448962in}{1.200829in}}%
\pgfpathlineto{\pgfqpoint{1.445700in}{1.209499in}}%
\pgfpathlineto{\pgfqpoint{1.450504in}{1.213713in}}%
\pgfpathlineto{\pgfqpoint{1.455029in}{1.217998in}}%
\pgfpathlineto{\pgfqpoint{1.459272in}{1.222349in}}%
\pgfpathlineto{\pgfqpoint{1.463229in}{1.226762in}}%
\pgfpathclose%
\pgfusepath{fill}%
\end{pgfscope}%
\begin{pgfscope}%
\pgfpathrectangle{\pgfqpoint{0.041670in}{0.041670in}}{\pgfqpoint{2.216660in}{2.216660in}}%
\pgfusepath{clip}%
\pgfsetbuttcap%
\pgfsetroundjoin%
\definecolor{currentfill}{rgb}{0.896320,0.893616,0.096335}%
\pgfsetfillcolor{currentfill}%
\pgfsetlinewidth{0.000000pt}%
\definecolor{currentstroke}{rgb}{0.000000,0.000000,0.000000}%
\pgfsetstrokecolor{currentstroke}%
\pgfsetdash{}{0pt}%
\pgfpathmoveto{\pgfqpoint{1.199500in}{1.662456in}}%
\pgfpathlineto{\pgfqpoint{1.200470in}{1.661529in}}%
\pgfpathlineto{\pgfqpoint{1.201438in}{1.660477in}}%
\pgfpathlineto{\pgfqpoint{1.202405in}{1.659302in}}%
\pgfpathlineto{\pgfqpoint{1.203371in}{1.658004in}}%
\pgfpathlineto{\pgfqpoint{1.206205in}{1.657639in}}%
\pgfpathlineto{\pgfqpoint{1.209012in}{1.657232in}}%
\pgfpathlineto{\pgfqpoint{1.211792in}{1.656785in}}%
\pgfpathlineto{\pgfqpoint{1.210477in}{1.658133in}}%
\pgfpathlineto{\pgfqpoint{1.209162in}{1.659359in}}%
\pgfpathlineto{\pgfqpoint{1.207846in}{1.660462in}}%
\pgfpathlineto{\pgfqpoint{1.206528in}{1.661439in}}%
\pgfpathlineto{\pgfqpoint{1.204208in}{1.661812in}}%
\pgfpathlineto{\pgfqpoint{1.201865in}{1.662151in}}%
\pgfpathlineto{\pgfqpoint{1.199500in}{1.662456in}}%
\pgfpathclose%
\pgfusepath{fill}%
\end{pgfscope}%
\begin{pgfscope}%
\pgfpathrectangle{\pgfqpoint{0.041670in}{0.041670in}}{\pgfqpoint{2.216660in}{2.216660in}}%
\pgfusepath{clip}%
\pgfsetbuttcap%
\pgfsetroundjoin%
\definecolor{currentfill}{rgb}{0.134692,0.658636,0.517649}%
\pgfsetfillcolor{currentfill}%
\pgfsetlinewidth{0.000000pt}%
\definecolor{currentstroke}{rgb}{0.000000,0.000000,0.000000}%
\pgfsetstrokecolor{currentstroke}%
\pgfsetdash{}{0pt}%
\pgfpathmoveto{\pgfqpoint{0.927677in}{1.289729in}}%
\pgfpathlineto{\pgfqpoint{0.924233in}{1.281492in}}%
\pgfpathlineto{\pgfqpoint{0.920791in}{1.273213in}}%
\pgfpathlineto{\pgfqpoint{0.917351in}{1.264897in}}%
\pgfpathlineto{\pgfqpoint{0.913913in}{1.256546in}}%
\pgfpathlineto{\pgfqpoint{0.910173in}{1.260727in}}%
\pgfpathlineto{\pgfqpoint{0.906708in}{1.264963in}}%
\pgfpathlineto{\pgfqpoint{0.903522in}{1.269247in}}%
\pgfpathlineto{\pgfqpoint{0.900618in}{1.273577in}}%
\pgfpathlineto{\pgfqpoint{0.904218in}{1.281688in}}%
\pgfpathlineto{\pgfqpoint{0.907821in}{1.289765in}}%
\pgfpathlineto{\pgfqpoint{0.911427in}{1.297804in}}%
\pgfpathlineto{\pgfqpoint{0.915035in}{1.305803in}}%
\pgfpathlineto{\pgfqpoint{0.917798in}{1.301717in}}%
\pgfpathlineto{\pgfqpoint{0.920828in}{1.297673in}}%
\pgfpathlineto{\pgfqpoint{0.924122in}{1.293676in}}%
\pgfpathlineto{\pgfqpoint{0.927677in}{1.289729in}}%
\pgfpathclose%
\pgfusepath{fill}%
\end{pgfscope}%
\begin{pgfscope}%
\pgfpathrectangle{\pgfqpoint{0.041670in}{0.041670in}}{\pgfqpoint{2.216660in}{2.216660in}}%
\pgfusepath{clip}%
\pgfsetbuttcap%
\pgfsetroundjoin%
\definecolor{currentfill}{rgb}{0.896320,0.893616,0.096335}%
\pgfsetfillcolor{currentfill}%
\pgfsetlinewidth{0.000000pt}%
\definecolor{currentstroke}{rgb}{0.000000,0.000000,0.000000}%
\pgfsetstrokecolor{currentstroke}%
\pgfsetdash{}{0pt}%
\pgfpathmoveto{\pgfqpoint{1.151342in}{1.661079in}}%
\pgfpathlineto{\pgfqpoint{1.149923in}{1.660083in}}%
\pgfpathlineto{\pgfqpoint{1.148505in}{1.658963in}}%
\pgfpathlineto{\pgfqpoint{1.147089in}{1.657719in}}%
\pgfpathlineto{\pgfqpoint{1.145674in}{1.656353in}}%
\pgfpathlineto{\pgfqpoint{1.148425in}{1.656836in}}%
\pgfpathlineto{\pgfqpoint{1.151208in}{1.657279in}}%
\pgfpathlineto{\pgfqpoint{1.154019in}{1.657681in}}%
\pgfpathlineto{\pgfqpoint{1.156855in}{1.658042in}}%
\pgfpathlineto{\pgfqpoint{1.157808in}{1.659338in}}%
\pgfpathlineto{\pgfqpoint{1.158762in}{1.660512in}}%
\pgfpathlineto{\pgfqpoint{1.159717in}{1.661562in}}%
\pgfpathlineto{\pgfqpoint{1.160673in}{1.662487in}}%
\pgfpathlineto{\pgfqpoint{1.158306in}{1.662187in}}%
\pgfpathlineto{\pgfqpoint{1.155961in}{1.661852in}}%
\pgfpathlineto{\pgfqpoint{1.153638in}{1.661482in}}%
\pgfpathlineto{\pgfqpoint{1.151342in}{1.661079in}}%
\pgfpathclose%
\pgfusepath{fill}%
\end{pgfscope}%
\begin{pgfscope}%
\pgfpathrectangle{\pgfqpoint{0.041670in}{0.041670in}}{\pgfqpoint{2.216660in}{2.216660in}}%
\pgfusepath{clip}%
\pgfsetbuttcap%
\pgfsetroundjoin%
\definecolor{currentfill}{rgb}{0.487026,0.823929,0.312321}%
\pgfsetfillcolor{currentfill}%
\pgfsetlinewidth{0.000000pt}%
\definecolor{currentstroke}{rgb}{0.000000,0.000000,0.000000}%
\pgfsetstrokecolor{currentstroke}%
\pgfsetdash{}{0pt}%
\pgfpathmoveto{\pgfqpoint{0.993827in}{1.501979in}}%
\pgfpathlineto{\pgfqpoint{0.989998in}{1.496076in}}%
\pgfpathlineto{\pgfqpoint{0.986171in}{1.490084in}}%
\pgfpathlineto{\pgfqpoint{0.982348in}{1.484005in}}%
\pgfpathlineto{\pgfqpoint{0.978527in}{1.477839in}}%
\pgfpathlineto{\pgfqpoint{0.978279in}{1.480883in}}%
\pgfpathlineto{\pgfqpoint{0.978232in}{1.483929in}}%
\pgfpathlineto{\pgfqpoint{0.978386in}{1.486972in}}%
\pgfpathlineto{\pgfqpoint{0.978741in}{1.490009in}}%
\pgfpathlineto{\pgfqpoint{0.982551in}{1.495934in}}%
\pgfpathlineto{\pgfqpoint{0.986364in}{1.501772in}}%
\pgfpathlineto{\pgfqpoint{0.990180in}{1.507523in}}%
\pgfpathlineto{\pgfqpoint{0.993999in}{1.513186in}}%
\pgfpathlineto{\pgfqpoint{0.993678in}{1.510388in}}%
\pgfpathlineto{\pgfqpoint{0.993542in}{1.507586in}}%
\pgfpathlineto{\pgfqpoint{0.993591in}{1.504782in}}%
\pgfpathlineto{\pgfqpoint{0.993827in}{1.501979in}}%
\pgfpathclose%
\pgfusepath{fill}%
\end{pgfscope}%
\begin{pgfscope}%
\pgfpathrectangle{\pgfqpoint{0.041670in}{0.041670in}}{\pgfqpoint{2.216660in}{2.216660in}}%
\pgfusepath{clip}%
\pgfsetbuttcap%
\pgfsetroundjoin%
\definecolor{currentfill}{rgb}{0.699415,0.867117,0.175971}%
\pgfsetfillcolor{currentfill}%
\pgfsetlinewidth{0.000000pt}%
\definecolor{currentstroke}{rgb}{0.000000,0.000000,0.000000}%
\pgfsetstrokecolor{currentstroke}%
\pgfsetdash{}{0pt}%
\pgfpathmoveto{\pgfqpoint{1.311805in}{1.591654in}}%
\pgfpathlineto{\pgfqpoint{1.315425in}{1.587697in}}%
\pgfpathlineto{\pgfqpoint{1.319043in}{1.583633in}}%
\pgfpathlineto{\pgfqpoint{1.322658in}{1.579462in}}%
\pgfpathlineto{\pgfqpoint{1.326270in}{1.575188in}}%
\pgfpathlineto{\pgfqpoint{1.327868in}{1.572997in}}%
\pgfpathlineto{\pgfqpoint{1.329321in}{1.570784in}}%
\pgfpathlineto{\pgfqpoint{1.330626in}{1.568548in}}%
\pgfpathlineto{\pgfqpoint{1.331782in}{1.566294in}}%
\pgfpathlineto{\pgfqpoint{1.328029in}{1.570794in}}%
\pgfpathlineto{\pgfqpoint{1.324273in}{1.575190in}}%
\pgfpathlineto{\pgfqpoint{1.320515in}{1.579479in}}%
\pgfpathlineto{\pgfqpoint{1.316754in}{1.583661in}}%
\pgfpathlineto{\pgfqpoint{1.315716in}{1.585687in}}%
\pgfpathlineto{\pgfqpoint{1.314545in}{1.587696in}}%
\pgfpathlineto{\pgfqpoint{1.313241in}{1.589685in}}%
\pgfpathlineto{\pgfqpoint{1.311805in}{1.591654in}}%
\pgfpathclose%
\pgfusepath{fill}%
\end{pgfscope}%
\begin{pgfscope}%
\pgfpathrectangle{\pgfqpoint{0.041670in}{0.041670in}}{\pgfqpoint{2.216660in}{2.216660in}}%
\pgfusepath{clip}%
\pgfsetbuttcap%
\pgfsetroundjoin%
\definecolor{currentfill}{rgb}{0.762373,0.876424,0.137064}%
\pgfsetfillcolor{currentfill}%
\pgfsetlinewidth{0.000000pt}%
\definecolor{currentstroke}{rgb}{0.000000,0.000000,0.000000}%
\pgfsetstrokecolor{currentstroke}%
\pgfsetdash{}{0pt}%
\pgfpathmoveto{\pgfqpoint{1.291053in}{1.613157in}}%
\pgfpathlineto{\pgfqpoint{1.294488in}{1.609855in}}%
\pgfpathlineto{\pgfqpoint{1.297921in}{1.606441in}}%
\pgfpathlineto{\pgfqpoint{1.301350in}{1.602916in}}%
\pgfpathlineto{\pgfqpoint{1.304777in}{1.599282in}}%
\pgfpathlineto{\pgfqpoint{1.306723in}{1.597416in}}%
\pgfpathlineto{\pgfqpoint{1.308544in}{1.595521in}}%
\pgfpathlineto{\pgfqpoint{1.310239in}{1.593600in}}%
\pgfpathlineto{\pgfqpoint{1.311805in}{1.591654in}}%
\pgfpathlineto{\pgfqpoint{1.308181in}{1.595503in}}%
\pgfpathlineto{\pgfqpoint{1.304555in}{1.599241in}}%
\pgfpathlineto{\pgfqpoint{1.300925in}{1.602868in}}%
\pgfpathlineto{\pgfqpoint{1.297293in}{1.606383in}}%
\pgfpathlineto{\pgfqpoint{1.295903in}{1.608111in}}%
\pgfpathlineto{\pgfqpoint{1.294399in}{1.609817in}}%
\pgfpathlineto{\pgfqpoint{1.292782in}{1.611499in}}%
\pgfpathlineto{\pgfqpoint{1.291053in}{1.613157in}}%
\pgfpathclose%
\pgfusepath{fill}%
\end{pgfscope}%
\begin{pgfscope}%
\pgfpathrectangle{\pgfqpoint{0.041670in}{0.041670in}}{\pgfqpoint{2.216660in}{2.216660in}}%
\pgfusepath{clip}%
\pgfsetbuttcap%
\pgfsetroundjoin%
\definecolor{currentfill}{rgb}{0.855810,0.888601,0.097452}%
\pgfsetfillcolor{currentfill}%
\pgfsetlinewidth{0.000000pt}%
\definecolor{currentstroke}{rgb}{0.000000,0.000000,0.000000}%
\pgfsetstrokecolor{currentstroke}%
\pgfsetdash{}{0pt}%
\pgfpathmoveto{\pgfqpoint{1.241968in}{1.648403in}}%
\pgfpathlineto{\pgfqpoint{1.244529in}{1.646585in}}%
\pgfpathlineto{\pgfqpoint{1.247088in}{1.644645in}}%
\pgfpathlineto{\pgfqpoint{1.249644in}{1.642585in}}%
\pgfpathlineto{\pgfqpoint{1.252198in}{1.640406in}}%
\pgfpathlineto{\pgfqpoint{1.254733in}{1.639322in}}%
\pgfpathlineto{\pgfqpoint{1.257195in}{1.638202in}}%
\pgfpathlineto{\pgfqpoint{1.259582in}{1.637045in}}%
\pgfpathlineto{\pgfqpoint{1.261890in}{1.635854in}}%
\pgfpathlineto{\pgfqpoint{1.258992in}{1.638196in}}%
\pgfpathlineto{\pgfqpoint{1.256091in}{1.640418in}}%
\pgfpathlineto{\pgfqpoint{1.253187in}{1.642520in}}%
\pgfpathlineto{\pgfqpoint{1.250281in}{1.644502in}}%
\pgfpathlineto{\pgfqpoint{1.248302in}{1.645523in}}%
\pgfpathlineto{\pgfqpoint{1.246255in}{1.646514in}}%
\pgfpathlineto{\pgfqpoint{1.244143in}{1.647475in}}%
\pgfpathlineto{\pgfqpoint{1.241968in}{1.648403in}}%
\pgfpathclose%
\pgfusepath{fill}%
\end{pgfscope}%
\begin{pgfscope}%
\pgfpathrectangle{\pgfqpoint{0.041670in}{0.041670in}}{\pgfqpoint{2.216660in}{2.216660in}}%
\pgfusepath{clip}%
\pgfsetbuttcap%
\pgfsetroundjoin%
\definecolor{currentfill}{rgb}{0.896320,0.893616,0.096335}%
\pgfsetfillcolor{currentfill}%
\pgfsetlinewidth{0.000000pt}%
\definecolor{currentstroke}{rgb}{0.000000,0.000000,0.000000}%
\pgfsetstrokecolor{currentstroke}%
\pgfsetdash{}{0pt}%
\pgfpathmoveto{\pgfqpoint{1.206528in}{1.661439in}}%
\pgfpathlineto{\pgfqpoint{1.207846in}{1.660462in}}%
\pgfpathlineto{\pgfqpoint{1.209162in}{1.659359in}}%
\pgfpathlineto{\pgfqpoint{1.210477in}{1.658133in}}%
\pgfpathlineto{\pgfqpoint{1.211792in}{1.656785in}}%
\pgfpathlineto{\pgfqpoint{1.214540in}{1.656296in}}%
\pgfpathlineto{\pgfqpoint{1.217254in}{1.655768in}}%
\pgfpathlineto{\pgfqpoint{1.219932in}{1.655200in}}%
\pgfpathlineto{\pgfqpoint{1.222570in}{1.654593in}}%
\pgfpathlineto{\pgfqpoint{1.220811in}{1.656032in}}%
\pgfpathlineto{\pgfqpoint{1.219050in}{1.657349in}}%
\pgfpathlineto{\pgfqpoint{1.217287in}{1.658542in}}%
\pgfpathlineto{\pgfqpoint{1.215522in}{1.659611in}}%
\pgfpathlineto{\pgfqpoint{1.213321in}{1.660117in}}%
\pgfpathlineto{\pgfqpoint{1.211086in}{1.660591in}}%
\pgfpathlineto{\pgfqpoint{1.208821in}{1.661032in}}%
\pgfpathlineto{\pgfqpoint{1.206528in}{1.661439in}}%
\pgfpathclose%
\pgfusepath{fill}%
\end{pgfscope}%
\begin{pgfscope}%
\pgfpathrectangle{\pgfqpoint{0.041670in}{0.041670in}}{\pgfqpoint{2.216660in}{2.216660in}}%
\pgfusepath{clip}%
\pgfsetbuttcap%
\pgfsetroundjoin%
\definecolor{currentfill}{rgb}{0.281477,0.755203,0.432552}%
\pgfsetfillcolor{currentfill}%
\pgfsetlinewidth{0.000000pt}%
\definecolor{currentstroke}{rgb}{0.000000,0.000000,0.000000}%
\pgfsetstrokecolor{currentstroke}%
\pgfsetdash{}{0pt}%
\pgfpathmoveto{\pgfqpoint{1.409523in}{1.414697in}}%
\pgfpathlineto{\pgfqpoint{1.413285in}{1.407653in}}%
\pgfpathlineto{\pgfqpoint{1.417044in}{1.400543in}}%
\pgfpathlineto{\pgfqpoint{1.420801in}{1.393367in}}%
\pgfpathlineto{\pgfqpoint{1.424554in}{1.386130in}}%
\pgfpathlineto{\pgfqpoint{1.423262in}{1.382392in}}%
\pgfpathlineto{\pgfqpoint{1.421725in}{1.378674in}}%
\pgfpathlineto{\pgfqpoint{1.419944in}{1.374979in}}%
\pgfpathlineto{\pgfqpoint{1.417920in}{1.371311in}}%
\pgfpathlineto{\pgfqpoint{1.414261in}{1.378792in}}%
\pgfpathlineto{\pgfqpoint{1.410598in}{1.386211in}}%
\pgfpathlineto{\pgfqpoint{1.406934in}{1.393564in}}%
\pgfpathlineto{\pgfqpoint{1.403266in}{1.400851in}}%
\pgfpathlineto{\pgfqpoint{1.405173in}{1.404277in}}%
\pgfpathlineto{\pgfqpoint{1.406852in}{1.407730in}}%
\pgfpathlineto{\pgfqpoint{1.408302in}{1.411204in}}%
\pgfpathlineto{\pgfqpoint{1.409523in}{1.414697in}}%
\pgfpathclose%
\pgfusepath{fill}%
\end{pgfscope}%
\begin{pgfscope}%
\pgfpathrectangle{\pgfqpoint{0.041670in}{0.041670in}}{\pgfqpoint{2.216660in}{2.216660in}}%
\pgfusepath{clip}%
\pgfsetbuttcap%
\pgfsetroundjoin%
\definecolor{currentfill}{rgb}{0.267004,0.004874,0.329415}%
\pgfsetfillcolor{currentfill}%
\pgfsetlinewidth{0.000000pt}%
\definecolor{currentstroke}{rgb}{0.000000,0.000000,0.000000}%
\pgfsetstrokecolor{currentstroke}%
\pgfsetdash{}{0pt}%
\pgfpathmoveto{\pgfqpoint{0.769267in}{0.642665in}}%
\pgfpathlineto{\pgfqpoint{0.766543in}{0.641387in}}%
\pgfpathlineto{\pgfqpoint{0.763813in}{0.640369in}}%
\pgfpathlineto{\pgfqpoint{0.761076in}{0.639616in}}%
\pgfpathlineto{\pgfqpoint{0.758332in}{0.639134in}}%
\pgfpathlineto{\pgfqpoint{0.744501in}{0.646425in}}%
\pgfpathlineto{\pgfqpoint{0.731144in}{0.653940in}}%
\pgfpathlineto{\pgfqpoint{0.718275in}{0.661670in}}%
\pgfpathlineto{\pgfqpoint{0.705907in}{0.669606in}}%
\pgfpathlineto{\pgfqpoint{0.708980in}{0.669889in}}%
\pgfpathlineto{\pgfqpoint{0.712045in}{0.670442in}}%
\pgfpathlineto{\pgfqpoint{0.715102in}{0.671259in}}%
\pgfpathlineto{\pgfqpoint{0.718152in}{0.672336in}}%
\pgfpathlineto{\pgfqpoint{0.730213in}{0.664608in}}%
\pgfpathlineto{\pgfqpoint{0.742761in}{0.657081in}}%
\pgfpathlineto{\pgfqpoint{0.755783in}{0.649764in}}%
\pgfpathlineto{\pgfqpoint{0.769267in}{0.642665in}}%
\pgfpathclose%
\pgfusepath{fill}%
\end{pgfscope}%
\begin{pgfscope}%
\pgfpathrectangle{\pgfqpoint{0.041670in}{0.041670in}}{\pgfqpoint{2.216660in}{2.216660in}}%
\pgfusepath{clip}%
\pgfsetbuttcap%
\pgfsetroundjoin%
\definecolor{currentfill}{rgb}{0.896320,0.893616,0.096335}%
\pgfsetfillcolor{currentfill}%
\pgfsetlinewidth{0.000000pt}%
\definecolor{currentstroke}{rgb}{0.000000,0.000000,0.000000}%
\pgfsetstrokecolor{currentstroke}%
\pgfsetdash{}{0pt}%
\pgfpathmoveto{\pgfqpoint{1.142460in}{1.659134in}}%
\pgfpathlineto{\pgfqpoint{1.140599in}{1.658042in}}%
\pgfpathlineto{\pgfqpoint{1.138741in}{1.656825in}}%
\pgfpathlineto{\pgfqpoint{1.136884in}{1.655484in}}%
\pgfpathlineto{\pgfqpoint{1.135029in}{1.654020in}}%
\pgfpathlineto{\pgfqpoint{1.137631in}{1.654662in}}%
\pgfpathlineto{\pgfqpoint{1.140274in}{1.655265in}}%
\pgfpathlineto{\pgfqpoint{1.142956in}{1.655829in}}%
\pgfpathlineto{\pgfqpoint{1.145674in}{1.656353in}}%
\pgfpathlineto{\pgfqpoint{1.147089in}{1.657719in}}%
\pgfpathlineto{\pgfqpoint{1.148505in}{1.658963in}}%
\pgfpathlineto{\pgfqpoint{1.149923in}{1.660083in}}%
\pgfpathlineto{\pgfqpoint{1.151342in}{1.661079in}}%
\pgfpathlineto{\pgfqpoint{1.149074in}{1.660642in}}%
\pgfpathlineto{\pgfqpoint{1.146836in}{1.660172in}}%
\pgfpathlineto{\pgfqpoint{1.144630in}{1.659669in}}%
\pgfpathlineto{\pgfqpoint{1.142460in}{1.659134in}}%
\pgfpathclose%
\pgfusepath{fill}%
\end{pgfscope}%
\begin{pgfscope}%
\pgfpathrectangle{\pgfqpoint{0.041670in}{0.041670in}}{\pgfqpoint{2.216660in}{2.216660in}}%
\pgfusepath{clip}%
\pgfsetbuttcap%
\pgfsetroundjoin%
\definecolor{currentfill}{rgb}{0.855810,0.888601,0.097452}%
\pgfsetfillcolor{currentfill}%
\pgfsetlinewidth{0.000000pt}%
\definecolor{currentstroke}{rgb}{0.000000,0.000000,0.000000}%
\pgfsetstrokecolor{currentstroke}%
\pgfsetdash{}{0pt}%
\pgfpathmoveto{\pgfqpoint{1.107927in}{1.643570in}}%
\pgfpathlineto{\pgfqpoint{1.104950in}{1.641550in}}%
\pgfpathlineto{\pgfqpoint{1.101976in}{1.639408in}}%
\pgfpathlineto{\pgfqpoint{1.099004in}{1.637147in}}%
\pgfpathlineto{\pgfqpoint{1.096035in}{1.634767in}}%
\pgfpathlineto{\pgfqpoint{1.098272in}{1.635988in}}%
\pgfpathlineto{\pgfqpoint{1.100589in}{1.637176in}}%
\pgfpathlineto{\pgfqpoint{1.102984in}{1.638328in}}%
\pgfpathlineto{\pgfqpoint{1.105455in}{1.639444in}}%
\pgfpathlineto{\pgfqpoint{1.108089in}{1.641658in}}%
\pgfpathlineto{\pgfqpoint{1.110725in}{1.643752in}}%
\pgfpathlineto{\pgfqpoint{1.113364in}{1.645726in}}%
\pgfpathlineto{\pgfqpoint{1.116005in}{1.647579in}}%
\pgfpathlineto{\pgfqpoint{1.113886in}{1.646622in}}%
\pgfpathlineto{\pgfqpoint{1.111832in}{1.645634in}}%
\pgfpathlineto{\pgfqpoint{1.109845in}{1.644617in}}%
\pgfpathlineto{\pgfqpoint{1.107927in}{1.643570in}}%
\pgfpathclose%
\pgfusepath{fill}%
\end{pgfscope}%
\begin{pgfscope}%
\pgfpathrectangle{\pgfqpoint{0.041670in}{0.041670in}}{\pgfqpoint{2.216660in}{2.216660in}}%
\pgfusepath{clip}%
\pgfsetbuttcap%
\pgfsetroundjoin%
\definecolor{currentfill}{rgb}{0.636902,0.856542,0.216620}%
\pgfsetfillcolor{currentfill}%
\pgfsetlinewidth{0.000000pt}%
\definecolor{currentstroke}{rgb}{0.000000,0.000000,0.000000}%
\pgfsetstrokecolor{currentstroke}%
\pgfsetdash{}{0pt}%
\pgfpathmoveto{\pgfqpoint{1.331782in}{1.566294in}}%
\pgfpathlineto{\pgfqpoint{1.335532in}{1.561691in}}%
\pgfpathlineto{\pgfqpoint{1.339278in}{1.556985in}}%
\pgfpathlineto{\pgfqpoint{1.343022in}{1.552180in}}%
\pgfpathlineto{\pgfqpoint{1.346763in}{1.547276in}}%
\pgfpathlineto{\pgfqpoint{1.347874in}{1.544773in}}%
\pgfpathlineto{\pgfqpoint{1.348818in}{1.542254in}}%
\pgfpathlineto{\pgfqpoint{1.349596in}{1.539722in}}%
\pgfpathlineto{\pgfqpoint{1.350205in}{1.537178in}}%
\pgfpathlineto{\pgfqpoint{1.346381in}{1.542316in}}%
\pgfpathlineto{\pgfqpoint{1.342555in}{1.547355in}}%
\pgfpathlineto{\pgfqpoint{1.338725in}{1.552294in}}%
\pgfpathlineto{\pgfqpoint{1.334893in}{1.557131in}}%
\pgfpathlineto{\pgfqpoint{1.334344in}{1.559439in}}%
\pgfpathlineto{\pgfqpoint{1.333642in}{1.561737in}}%
\pgfpathlineto{\pgfqpoint{1.332788in}{1.564023in}}%
\pgfpathlineto{\pgfqpoint{1.331782in}{1.566294in}}%
\pgfpathclose%
\pgfusepath{fill}%
\end{pgfscope}%
\begin{pgfscope}%
\pgfpathrectangle{\pgfqpoint{0.041670in}{0.041670in}}{\pgfqpoint{2.216660in}{2.216660in}}%
\pgfusepath{clip}%
\pgfsetbuttcap%
\pgfsetroundjoin%
\definecolor{currentfill}{rgb}{0.231674,0.318106,0.544834}%
\pgfsetfillcolor{currentfill}%
\pgfsetlinewidth{0.000000pt}%
\definecolor{currentstroke}{rgb}{0.000000,0.000000,0.000000}%
\pgfsetstrokecolor{currentstroke}%
\pgfsetdash{}{0pt}%
\pgfpathmoveto{\pgfqpoint{0.895575in}{0.921505in}}%
\pgfpathlineto{\pgfqpoint{0.892983in}{0.912878in}}%
\pgfpathlineto{\pgfqpoint{0.890390in}{0.904314in}}%
\pgfpathlineto{\pgfqpoint{0.887797in}{0.895816in}}%
\pgfpathlineto{\pgfqpoint{0.885204in}{0.887389in}}%
\pgfpathlineto{\pgfqpoint{0.875445in}{0.892318in}}%
\pgfpathlineto{\pgfqpoint{0.866009in}{0.897401in}}%
\pgfpathlineto{\pgfqpoint{0.856907in}{0.902633in}}%
\pgfpathlineto{\pgfqpoint{0.848147in}{0.908008in}}%
\pgfpathlineto{\pgfqpoint{0.851057in}{0.916231in}}%
\pgfpathlineto{\pgfqpoint{0.853968in}{0.924524in}}%
\pgfpathlineto{\pgfqpoint{0.856879in}{0.932884in}}%
\pgfpathlineto{\pgfqpoint{0.859789in}{0.941307in}}%
\pgfpathlineto{\pgfqpoint{0.868251in}{0.936145in}}%
\pgfpathlineto{\pgfqpoint{0.877041in}{0.931120in}}%
\pgfpathlineto{\pgfqpoint{0.886152in}{0.926239in}}%
\pgfpathlineto{\pgfqpoint{0.895575in}{0.921505in}}%
\pgfpathclose%
\pgfusepath{fill}%
\end{pgfscope}%
\begin{pgfscope}%
\pgfpathrectangle{\pgfqpoint{0.041670in}{0.041670in}}{\pgfqpoint{2.216660in}{2.216660in}}%
\pgfusepath{clip}%
\pgfsetbuttcap%
\pgfsetroundjoin%
\definecolor{currentfill}{rgb}{0.212395,0.359683,0.551710}%
\pgfsetfillcolor{currentfill}%
\pgfsetlinewidth{0.000000pt}%
\definecolor{currentstroke}{rgb}{0.000000,0.000000,0.000000}%
\pgfsetstrokecolor{currentstroke}%
\pgfsetdash{}{0pt}%
\pgfpathmoveto{\pgfqpoint{1.495459in}{0.980077in}}%
\pgfpathlineto{\pgfqpoint{1.498434in}{0.971480in}}%
\pgfpathlineto{\pgfqpoint{1.501409in}{0.962934in}}%
\pgfpathlineto{\pgfqpoint{1.504384in}{0.954442in}}%
\pgfpathlineto{\pgfqpoint{1.507358in}{0.946006in}}%
\pgfpathlineto{\pgfqpoint{1.499197in}{0.940727in}}%
\pgfpathlineto{\pgfqpoint{1.490699in}{0.935580in}}%
\pgfpathlineto{\pgfqpoint{1.481872in}{0.930571in}}%
\pgfpathlineto{\pgfqpoint{1.472726in}{0.925705in}}%
\pgfpathlineto{\pgfqpoint{1.470058in}{0.934349in}}%
\pgfpathlineto{\pgfqpoint{1.467391in}{0.943050in}}%
\pgfpathlineto{\pgfqpoint{1.464723in}{0.951805in}}%
\pgfpathlineto{\pgfqpoint{1.462056in}{0.960609in}}%
\pgfpathlineto{\pgfqpoint{1.470876in}{0.965274in}}%
\pgfpathlineto{\pgfqpoint{1.479389in}{0.970078in}}%
\pgfpathlineto{\pgfqpoint{1.487586in}{0.975014in}}%
\pgfpathlineto{\pgfqpoint{1.495459in}{0.980077in}}%
\pgfpathclose%
\pgfusepath{fill}%
\end{pgfscope}%
\begin{pgfscope}%
\pgfpathrectangle{\pgfqpoint{0.041670in}{0.041670in}}{\pgfqpoint{2.216660in}{2.216660in}}%
\pgfusepath{clip}%
\pgfsetbuttcap%
\pgfsetroundjoin%
\definecolor{currentfill}{rgb}{0.201239,0.383670,0.554294}%
\pgfsetfillcolor{currentfill}%
\pgfsetlinewidth{0.000000pt}%
\definecolor{currentstroke}{rgb}{0.000000,0.000000,0.000000}%
\pgfsetstrokecolor{currentstroke}%
\pgfsetdash{}{0pt}%
\pgfpathmoveto{\pgfqpoint{1.890129in}{0.975583in}}%
\pgfpathlineto{\pgfqpoint{1.894341in}{0.988141in}}%
\pgfpathlineto{\pgfqpoint{1.898577in}{1.001181in}}%
\pgfpathlineto{\pgfqpoint{1.902835in}{1.014712in}}%
\pgfpathlineto{\pgfqpoint{1.907116in}{1.028742in}}%
\pgfpathlineto{\pgfqpoint{1.900999in}{1.017051in}}%
\pgfpathlineto{\pgfqpoint{1.894129in}{1.005444in}}%
\pgfpathlineto{\pgfqpoint{1.886509in}{0.993937in}}%
\pgfpathlineto{\pgfqpoint{1.878145in}{0.982540in}}%
\pgfpathlineto{\pgfqpoint{1.874007in}{0.968693in}}%
\pgfpathlineto{\pgfqpoint{1.869893in}{0.955347in}}%
\pgfpathlineto{\pgfqpoint{1.865801in}{0.942495in}}%
\pgfpathlineto{\pgfqpoint{1.861730in}{0.930128in}}%
\pgfpathlineto{\pgfqpoint{1.869924in}{0.941339in}}%
\pgfpathlineto{\pgfqpoint{1.877391in}{0.952660in}}%
\pgfpathlineto{\pgfqpoint{1.884127in}{0.964079in}}%
\pgfpathlineto{\pgfqpoint{1.890129in}{0.975583in}}%
\pgfpathclose%
\pgfusepath{fill}%
\end{pgfscope}%
\begin{pgfscope}%
\pgfpathrectangle{\pgfqpoint{0.041670in}{0.041670in}}{\pgfqpoint{2.216660in}{2.216660in}}%
\pgfusepath{clip}%
\pgfsetbuttcap%
\pgfsetroundjoin%
\definecolor{currentfill}{rgb}{0.268510,0.009605,0.335427}%
\pgfsetfillcolor{currentfill}%
\pgfsetlinewidth{0.000000pt}%
\definecolor{currentstroke}{rgb}{0.000000,0.000000,0.000000}%
\pgfsetstrokecolor{currentstroke}%
\pgfsetdash{}{0pt}%
\pgfpathmoveto{\pgfqpoint{1.664566in}{0.676826in}}%
\pgfpathlineto{\pgfqpoint{1.667712in}{0.676865in}}%
\pgfpathlineto{\pgfqpoint{1.670866in}{0.677184in}}%
\pgfpathlineto{\pgfqpoint{1.674029in}{0.677789in}}%
\pgfpathlineto{\pgfqpoint{1.677201in}{0.678685in}}%
\pgfpathlineto{\pgfqpoint{1.664991in}{0.670364in}}%
\pgfpathlineto{\pgfqpoint{1.652256in}{0.662245in}}%
\pgfpathlineto{\pgfqpoint{1.639007in}{0.654339in}}%
\pgfpathlineto{\pgfqpoint{1.625258in}{0.646653in}}%
\pgfpathlineto{\pgfqpoint{1.622406in}{0.645959in}}%
\pgfpathlineto{\pgfqpoint{1.619561in}{0.645556in}}%
\pgfpathlineto{\pgfqpoint{1.616725in}{0.645439in}}%
\pgfpathlineto{\pgfqpoint{1.613896in}{0.645604in}}%
\pgfpathlineto{\pgfqpoint{1.627306in}{0.653094in}}%
\pgfpathlineto{\pgfqpoint{1.640229in}{0.660801in}}%
\pgfpathlineto{\pgfqpoint{1.652653in}{0.668715in}}%
\pgfpathlineto{\pgfqpoint{1.664566in}{0.676826in}}%
\pgfpathclose%
\pgfusepath{fill}%
\end{pgfscope}%
\begin{pgfscope}%
\pgfpathrectangle{\pgfqpoint{0.041670in}{0.041670in}}{\pgfqpoint{2.216660in}{2.216660in}}%
\pgfusepath{clip}%
\pgfsetbuttcap%
\pgfsetroundjoin%
\definecolor{currentfill}{rgb}{0.762373,0.876424,0.137064}%
\pgfsetfillcolor{currentfill}%
\pgfsetlinewidth{0.000000pt}%
\definecolor{currentstroke}{rgb}{0.000000,0.000000,0.000000}%
\pgfsetstrokecolor{currentstroke}%
\pgfsetdash{}{0pt}%
\pgfpathmoveto{\pgfqpoint{1.061478in}{1.604830in}}%
\pgfpathlineto{\pgfqpoint{1.057810in}{1.601267in}}%
\pgfpathlineto{\pgfqpoint{1.054145in}{1.597590in}}%
\pgfpathlineto{\pgfqpoint{1.050482in}{1.593803in}}%
\pgfpathlineto{\pgfqpoint{1.046822in}{1.589905in}}%
\pgfpathlineto{\pgfqpoint{1.048273in}{1.591872in}}%
\pgfpathlineto{\pgfqpoint{1.049853in}{1.593815in}}%
\pgfpathlineto{\pgfqpoint{1.051562in}{1.595733in}}%
\pgfpathlineto{\pgfqpoint{1.053397in}{1.597624in}}%
\pgfpathlineto{\pgfqpoint{1.056872in}{1.601305in}}%
\pgfpathlineto{\pgfqpoint{1.060350in}{1.604877in}}%
\pgfpathlineto{\pgfqpoint{1.063831in}{1.608337in}}%
\pgfpathlineto{\pgfqpoint{1.067315in}{1.611685in}}%
\pgfpathlineto{\pgfqpoint{1.065685in}{1.610005in}}%
\pgfpathlineto{\pgfqpoint{1.064168in}{1.608302in}}%
\pgfpathlineto{\pgfqpoint{1.062765in}{1.606576in}}%
\pgfpathlineto{\pgfqpoint{1.061478in}{1.604830in}}%
\pgfpathclose%
\pgfusepath{fill}%
\end{pgfscope}%
\begin{pgfscope}%
\pgfpathrectangle{\pgfqpoint{0.041670in}{0.041670in}}{\pgfqpoint{2.216660in}{2.216660in}}%
\pgfusepath{clip}%
\pgfsetbuttcap%
\pgfsetroundjoin%
\definecolor{currentfill}{rgb}{0.814576,0.883393,0.110347}%
\pgfsetfillcolor{currentfill}%
\pgfsetlinewidth{0.000000pt}%
\definecolor{currentstroke}{rgb}{0.000000,0.000000,0.000000}%
\pgfsetstrokecolor{currentstroke}%
\pgfsetdash{}{0pt}%
\pgfpathmoveto{\pgfqpoint{1.270298in}{1.630764in}}%
\pgfpathlineto{\pgfqpoint{1.273494in}{1.628123in}}%
\pgfpathlineto{\pgfqpoint{1.276687in}{1.625365in}}%
\pgfpathlineto{\pgfqpoint{1.279877in}{1.622491in}}%
\pgfpathlineto{\pgfqpoint{1.283065in}{1.619502in}}%
\pgfpathlineto{\pgfqpoint{1.285219in}{1.617961in}}%
\pgfpathlineto{\pgfqpoint{1.287270in}{1.616389in}}%
\pgfpathlineto{\pgfqpoint{1.289215in}{1.614787in}}%
\pgfpathlineto{\pgfqpoint{1.291053in}{1.613157in}}%
\pgfpathlineto{\pgfqpoint{1.287615in}{1.616345in}}%
\pgfpathlineto{\pgfqpoint{1.284175in}{1.619418in}}%
\pgfpathlineto{\pgfqpoint{1.280731in}{1.622375in}}%
\pgfpathlineto{\pgfqpoint{1.277285in}{1.625215in}}%
\pgfpathlineto{\pgfqpoint{1.275678in}{1.626641in}}%
\pgfpathlineto{\pgfqpoint{1.273977in}{1.628042in}}%
\pgfpathlineto{\pgfqpoint{1.272183in}{1.629417in}}%
\pgfpathlineto{\pgfqpoint{1.270298in}{1.630764in}}%
\pgfpathclose%
\pgfusepath{fill}%
\end{pgfscope}%
\begin{pgfscope}%
\pgfpathrectangle{\pgfqpoint{0.041670in}{0.041670in}}{\pgfqpoint{2.216660in}{2.216660in}}%
\pgfusepath{clip}%
\pgfsetbuttcap%
\pgfsetroundjoin%
\definecolor{currentfill}{rgb}{0.412913,0.803041,0.357269}%
\pgfsetfillcolor{currentfill}%
\pgfsetlinewidth{0.000000pt}%
\definecolor{currentstroke}{rgb}{0.000000,0.000000,0.000000}%
\pgfsetstrokecolor{currentstroke}%
\pgfsetdash{}{0pt}%
\pgfpathmoveto{\pgfqpoint{1.381613in}{1.480545in}}%
\pgfpathlineto{\pgfqpoint{1.385434in}{1.474349in}}%
\pgfpathlineto{\pgfqpoint{1.389252in}{1.468071in}}%
\pgfpathlineto{\pgfqpoint{1.393066in}{1.461712in}}%
\pgfpathlineto{\pgfqpoint{1.396878in}{1.455276in}}%
\pgfpathlineto{\pgfqpoint{1.396595in}{1.451988in}}%
\pgfpathlineto{\pgfqpoint{1.396095in}{1.448706in}}%
\pgfpathlineto{\pgfqpoint{1.395378in}{1.445430in}}%
\pgfpathlineto{\pgfqpoint{1.394446in}{1.442166in}}%
\pgfpathlineto{\pgfqpoint{1.390670in}{1.448846in}}%
\pgfpathlineto{\pgfqpoint{1.386891in}{1.455447in}}%
\pgfpathlineto{\pgfqpoint{1.383109in}{1.461968in}}%
\pgfpathlineto{\pgfqpoint{1.379324in}{1.468406in}}%
\pgfpathlineto{\pgfqpoint{1.380198in}{1.471428in}}%
\pgfpathlineto{\pgfqpoint{1.380871in}{1.474461in}}%
\pgfpathlineto{\pgfqpoint{1.381343in}{1.477501in}}%
\pgfpathlineto{\pgfqpoint{1.381613in}{1.480545in}}%
\pgfpathclose%
\pgfusepath{fill}%
\end{pgfscope}%
\begin{pgfscope}%
\pgfpathrectangle{\pgfqpoint{0.041670in}{0.041670in}}{\pgfqpoint{2.216660in}{2.216660in}}%
\pgfusepath{clip}%
\pgfsetbuttcap%
\pgfsetroundjoin%
\definecolor{currentfill}{rgb}{0.699415,0.867117,0.175971}%
\pgfsetfillcolor{currentfill}%
\pgfsetlinewidth{0.000000pt}%
\definecolor{currentstroke}{rgb}{0.000000,0.000000,0.000000}%
\pgfsetstrokecolor{currentstroke}%
\pgfsetdash{}{0pt}%
\pgfpathmoveto{\pgfqpoint{1.042348in}{1.581848in}}%
\pgfpathlineto{\pgfqpoint{1.038563in}{1.577615in}}%
\pgfpathlineto{\pgfqpoint{1.034782in}{1.573274in}}%
\pgfpathlineto{\pgfqpoint{1.031002in}{1.568828in}}%
\pgfpathlineto{\pgfqpoint{1.027226in}{1.564276in}}%
\pgfpathlineto{\pgfqpoint{1.028249in}{1.566546in}}%
\pgfpathlineto{\pgfqpoint{1.029422in}{1.568798in}}%
\pgfpathlineto{\pgfqpoint{1.030743in}{1.571031in}}%
\pgfpathlineto{\pgfqpoint{1.032212in}{1.573242in}}%
\pgfpathlineto{\pgfqpoint{1.035860in}{1.577566in}}%
\pgfpathlineto{\pgfqpoint{1.039511in}{1.581785in}}%
\pgfpathlineto{\pgfqpoint{1.043165in}{1.585899in}}%
\pgfpathlineto{\pgfqpoint{1.046822in}{1.589905in}}%
\pgfpathlineto{\pgfqpoint{1.045503in}{1.587918in}}%
\pgfpathlineto{\pgfqpoint{1.044317in}{1.585911in}}%
\pgfpathlineto{\pgfqpoint{1.043265in}{1.583887in}}%
\pgfpathlineto{\pgfqpoint{1.042348in}{1.581848in}}%
\pgfpathclose%
\pgfusepath{fill}%
\end{pgfscope}%
\begin{pgfscope}%
\pgfpathrectangle{\pgfqpoint{0.041670in}{0.041670in}}{\pgfqpoint{2.216660in}{2.216660in}}%
\pgfusepath{clip}%
\pgfsetbuttcap%
\pgfsetroundjoin%
\definecolor{currentfill}{rgb}{0.166383,0.690856,0.496502}%
\pgfsetfillcolor{currentfill}%
\pgfsetlinewidth{0.000000pt}%
\definecolor{currentstroke}{rgb}{0.000000,0.000000,0.000000}%
\pgfsetstrokecolor{currentstroke}%
\pgfsetdash{}{0pt}%
\pgfpathmoveto{\pgfqpoint{1.432533in}{1.340802in}}%
\pgfpathlineto{\pgfqpoint{1.436180in}{1.333040in}}%
\pgfpathlineto{\pgfqpoint{1.439825in}{1.325229in}}%
\pgfpathlineto{\pgfqpoint{1.443466in}{1.317371in}}%
\pgfpathlineto{\pgfqpoint{1.447105in}{1.309468in}}%
\pgfpathlineto{\pgfqpoint{1.444581in}{1.305347in}}%
\pgfpathlineto{\pgfqpoint{1.441788in}{1.301265in}}%
\pgfpathlineto{\pgfqpoint{1.438728in}{1.297226in}}%
\pgfpathlineto{\pgfqpoint{1.435405in}{1.293235in}}%
\pgfpathlineto{\pgfqpoint{1.431917in}{1.301378in}}%
\pgfpathlineto{\pgfqpoint{1.428427in}{1.309476in}}%
\pgfpathlineto{\pgfqpoint{1.424935in}{1.317527in}}%
\pgfpathlineto{\pgfqpoint{1.421440in}{1.325528in}}%
\pgfpathlineto{\pgfqpoint{1.424590in}{1.329283in}}%
\pgfpathlineto{\pgfqpoint{1.427490in}{1.333084in}}%
\pgfpathlineto{\pgfqpoint{1.430139in}{1.336925in}}%
\pgfpathlineto{\pgfqpoint{1.432533in}{1.340802in}}%
\pgfpathclose%
\pgfusepath{fill}%
\end{pgfscope}%
\begin{pgfscope}%
\pgfpathrectangle{\pgfqpoint{0.041670in}{0.041670in}}{\pgfqpoint{2.216660in}{2.216660in}}%
\pgfusepath{clip}%
\pgfsetbuttcap%
\pgfsetroundjoin%
\definecolor{currentfill}{rgb}{0.163625,0.471133,0.558148}%
\pgfsetfillcolor{currentfill}%
\pgfsetlinewidth{0.000000pt}%
\definecolor{currentstroke}{rgb}{0.000000,0.000000,0.000000}%
\pgfsetstrokecolor{currentstroke}%
\pgfsetdash{}{0pt}%
\pgfpathmoveto{\pgfqpoint{0.906417in}{1.082072in}}%
\pgfpathlineto{\pgfqpoint{0.903497in}{1.073070in}}%
\pgfpathlineto{\pgfqpoint{0.900578in}{1.064081in}}%
\pgfpathlineto{\pgfqpoint{0.897660in}{1.055107in}}%
\pgfpathlineto{\pgfqpoint{0.894743in}{1.046153in}}%
\pgfpathlineto{\pgfqpoint{0.887483in}{1.050800in}}%
\pgfpathlineto{\pgfqpoint{0.880528in}{1.055558in}}%
\pgfpathlineto{\pgfqpoint{0.873886in}{1.060422in}}%
\pgfpathlineto{\pgfqpoint{0.867563in}{1.065387in}}%
\pgfpathlineto{\pgfqpoint{0.870749in}{1.074121in}}%
\pgfpathlineto{\pgfqpoint{0.873937in}{1.082874in}}%
\pgfpathlineto{\pgfqpoint{0.877126in}{1.091644in}}%
\pgfpathlineto{\pgfqpoint{0.880316in}{1.100427in}}%
\pgfpathlineto{\pgfqpoint{0.886389in}{1.095688in}}%
\pgfpathlineto{\pgfqpoint{0.892768in}{1.091047in}}%
\pgfpathlineto{\pgfqpoint{0.899446in}{1.086506in}}%
\pgfpathlineto{\pgfqpoint{0.906417in}{1.082072in}}%
\pgfpathclose%
\pgfusepath{fill}%
\end{pgfscope}%
\begin{pgfscope}%
\pgfpathrectangle{\pgfqpoint{0.041670in}{0.041670in}}{\pgfqpoint{2.216660in}{2.216660in}}%
\pgfusepath{clip}%
\pgfsetbuttcap%
\pgfsetroundjoin%
\definecolor{currentfill}{rgb}{0.122606,0.585371,0.546557}%
\pgfsetfillcolor{currentfill}%
\pgfsetlinewidth{0.000000pt}%
\definecolor{currentstroke}{rgb}{0.000000,0.000000,0.000000}%
\pgfsetstrokecolor{currentstroke}%
\pgfsetdash{}{0pt}%
\pgfpathmoveto{\pgfqpoint{0.918710in}{1.205816in}}%
\pgfpathlineto{\pgfqpoint{0.915502in}{1.197096in}}%
\pgfpathlineto{\pgfqpoint{0.912295in}{1.188355in}}%
\pgfpathlineto{\pgfqpoint{0.909090in}{1.179596in}}%
\pgfpathlineto{\pgfqpoint{0.905887in}{1.170823in}}%
\pgfpathlineto{\pgfqpoint{0.900600in}{1.175196in}}%
\pgfpathlineto{\pgfqpoint{0.895601in}{1.179647in}}%
\pgfpathlineto{\pgfqpoint{0.890895in}{1.184173in}}%
\pgfpathlineto{\pgfqpoint{0.886487in}{1.188767in}}%
\pgfpathlineto{\pgfqpoint{0.889908in}{1.197309in}}%
\pgfpathlineto{\pgfqpoint{0.893331in}{1.205836in}}%
\pgfpathlineto{\pgfqpoint{0.896756in}{1.214346in}}%
\pgfpathlineto{\pgfqpoint{0.900184in}{1.222836in}}%
\pgfpathlineto{\pgfqpoint{0.904395in}{1.218478in}}%
\pgfpathlineto{\pgfqpoint{0.908889in}{1.214186in}}%
\pgfpathlineto{\pgfqpoint{0.913662in}{1.209964in}}%
\pgfpathlineto{\pgfqpoint{0.918710in}{1.205816in}}%
\pgfpathclose%
\pgfusepath{fill}%
\end{pgfscope}%
\begin{pgfscope}%
\pgfpathrectangle{\pgfqpoint{0.041670in}{0.041670in}}{\pgfqpoint{2.216660in}{2.216660in}}%
\pgfusepath{clip}%
\pgfsetbuttcap%
\pgfsetroundjoin%
\definecolor{currentfill}{rgb}{0.896320,0.893616,0.096335}%
\pgfsetfillcolor{currentfill}%
\pgfsetlinewidth{0.000000pt}%
\definecolor{currentstroke}{rgb}{0.000000,0.000000,0.000000}%
\pgfsetstrokecolor{currentstroke}%
\pgfsetdash{}{0pt}%
\pgfpathmoveto{\pgfqpoint{1.215522in}{1.659611in}}%
\pgfpathlineto{\pgfqpoint{1.217287in}{1.658542in}}%
\pgfpathlineto{\pgfqpoint{1.219050in}{1.657349in}}%
\pgfpathlineto{\pgfqpoint{1.220811in}{1.656032in}}%
\pgfpathlineto{\pgfqpoint{1.222570in}{1.654593in}}%
\pgfpathlineto{\pgfqpoint{1.225167in}{1.653947in}}%
\pgfpathlineto{\pgfqpoint{1.227720in}{1.653263in}}%
\pgfpathlineto{\pgfqpoint{1.230226in}{1.652542in}}%
\pgfpathlineto{\pgfqpoint{1.232682in}{1.651784in}}%
\pgfpathlineto{\pgfqpoint{1.230504in}{1.653341in}}%
\pgfpathlineto{\pgfqpoint{1.228324in}{1.654774in}}%
\pgfpathlineto{\pgfqpoint{1.226143in}{1.656084in}}%
\pgfpathlineto{\pgfqpoint{1.223959in}{1.657269in}}%
\pgfpathlineto{\pgfqpoint{1.221910in}{1.657901in}}%
\pgfpathlineto{\pgfqpoint{1.219819in}{1.658502in}}%
\pgfpathlineto{\pgfqpoint{1.217689in}{1.659072in}}%
\pgfpathlineto{\pgfqpoint{1.215522in}{1.659611in}}%
\pgfpathclose%
\pgfusepath{fill}%
\end{pgfscope}%
\begin{pgfscope}%
\pgfpathrectangle{\pgfqpoint{0.041670in}{0.041670in}}{\pgfqpoint{2.216660in}{2.216660in}}%
\pgfusepath{clip}%
\pgfsetbuttcap%
\pgfsetroundjoin%
\definecolor{currentfill}{rgb}{0.814576,0.883393,0.110347}%
\pgfsetfillcolor{currentfill}%
\pgfsetlinewidth{0.000000pt}%
\definecolor{currentstroke}{rgb}{0.000000,0.000000,0.000000}%
\pgfsetstrokecolor{currentstroke}%
\pgfsetdash{}{0pt}%
\pgfpathmoveto{\pgfqpoint{1.081277in}{1.623928in}}%
\pgfpathlineto{\pgfqpoint{1.077782in}{1.621042in}}%
\pgfpathlineto{\pgfqpoint{1.074290in}{1.618039in}}%
\pgfpathlineto{\pgfqpoint{1.070801in}{1.614919in}}%
\pgfpathlineto{\pgfqpoint{1.067315in}{1.611685in}}%
\pgfpathlineto{\pgfqpoint{1.069055in}{1.613339in}}%
\pgfpathlineto{\pgfqpoint{1.070905in}{1.614966in}}%
\pgfpathlineto{\pgfqpoint{1.072862in}{1.616565in}}%
\pgfpathlineto{\pgfqpoint{1.074925in}{1.618134in}}%
\pgfpathlineto{\pgfqpoint{1.078173in}{1.621166in}}%
\pgfpathlineto{\pgfqpoint{1.081423in}{1.624083in}}%
\pgfpathlineto{\pgfqpoint{1.084676in}{1.626884in}}%
\pgfpathlineto{\pgfqpoint{1.087932in}{1.629568in}}%
\pgfpathlineto{\pgfqpoint{1.086128in}{1.628196in}}%
\pgfpathlineto{\pgfqpoint{1.084416in}{1.626798in}}%
\pgfpathlineto{\pgfqpoint{1.082798in}{1.625375in}}%
\pgfpathlineto{\pgfqpoint{1.081277in}{1.623928in}}%
\pgfpathclose%
\pgfusepath{fill}%
\end{pgfscope}%
\begin{pgfscope}%
\pgfpathrectangle{\pgfqpoint{0.041670in}{0.041670in}}{\pgfqpoint{2.216660in}{2.216660in}}%
\pgfusepath{clip}%
\pgfsetbuttcap%
\pgfsetroundjoin%
\definecolor{currentfill}{rgb}{0.282327,0.094955,0.417331}%
\pgfsetfillcolor{currentfill}%
\pgfsetlinewidth{0.000000pt}%
\definecolor{currentstroke}{rgb}{0.000000,0.000000,0.000000}%
\pgfsetstrokecolor{currentstroke}%
\pgfsetdash{}{0pt}%
\pgfpathmoveto{\pgfqpoint{1.536948in}{0.744291in}}%
\pgfpathlineto{\pgfqpoint{1.539643in}{0.738213in}}%
\pgfpathlineto{\pgfqpoint{1.542341in}{0.732288in}}%
\pgfpathlineto{\pgfqpoint{1.545041in}{0.726518in}}%
\pgfpathlineto{\pgfqpoint{1.547745in}{0.720909in}}%
\pgfpathlineto{\pgfqpoint{1.535925in}{0.714814in}}%
\pgfpathlineto{\pgfqpoint{1.523720in}{0.708916in}}%
\pgfpathlineto{\pgfqpoint{1.511144in}{0.703223in}}%
\pgfpathlineto{\pgfqpoint{1.498208in}{0.697741in}}%
\pgfpathlineto{\pgfqpoint{1.495862in}{0.703539in}}%
\pgfpathlineto{\pgfqpoint{1.493518in}{0.709496in}}%
\pgfpathlineto{\pgfqpoint{1.491176in}{0.715610in}}%
\pgfpathlineto{\pgfqpoint{1.488837in}{0.721876in}}%
\pgfpathlineto{\pgfqpoint{1.501399in}{0.727180in}}%
\pgfpathlineto{\pgfqpoint{1.513613in}{0.732687in}}%
\pgfpathlineto{\pgfqpoint{1.525466in}{0.738393in}}%
\pgfpathlineto{\pgfqpoint{1.536948in}{0.744291in}}%
\pgfpathclose%
\pgfusepath{fill}%
\end{pgfscope}%
\begin{pgfscope}%
\pgfpathrectangle{\pgfqpoint{0.041670in}{0.041670in}}{\pgfqpoint{2.216660in}{2.216660in}}%
\pgfusepath{clip}%
\pgfsetbuttcap%
\pgfsetroundjoin%
\definecolor{currentfill}{rgb}{0.636902,0.856542,0.216620}%
\pgfsetfillcolor{currentfill}%
\pgfsetlinewidth{0.000000pt}%
\definecolor{currentstroke}{rgb}{0.000000,0.000000,0.000000}%
\pgfsetstrokecolor{currentstroke}%
\pgfsetdash{}{0pt}%
\pgfpathmoveto{\pgfqpoint{1.024658in}{1.555072in}}%
\pgfpathlineto{\pgfqpoint{1.020816in}{1.550183in}}%
\pgfpathlineto{\pgfqpoint{1.016976in}{1.545192in}}%
\pgfpathlineto{\pgfqpoint{1.013139in}{1.540100in}}%
\pgfpathlineto{\pgfqpoint{1.009306in}{1.534909in}}%
\pgfpathlineto{\pgfqpoint{1.009764in}{1.537461in}}%
\pgfpathlineto{\pgfqpoint{1.010392in}{1.540004in}}%
\pgfpathlineto{\pgfqpoint{1.011188in}{1.542535in}}%
\pgfpathlineto{\pgfqpoint{1.012151in}{1.545052in}}%
\pgfpathlineto{\pgfqpoint{1.015915in}{1.550008in}}%
\pgfpathlineto{\pgfqpoint{1.019683in}{1.554865in}}%
\pgfpathlineto{\pgfqpoint{1.023453in}{1.559621in}}%
\pgfpathlineto{\pgfqpoint{1.027226in}{1.564276in}}%
\pgfpathlineto{\pgfqpoint{1.026355in}{1.561992in}}%
\pgfpathlineto{\pgfqpoint{1.025636in}{1.559695in}}%
\pgfpathlineto{\pgfqpoint{1.025070in}{1.557388in}}%
\pgfpathlineto{\pgfqpoint{1.024658in}{1.555072in}}%
\pgfpathclose%
\pgfusepath{fill}%
\end{pgfscope}%
\begin{pgfscope}%
\pgfpathrectangle{\pgfqpoint{0.041670in}{0.041670in}}{\pgfqpoint{2.216660in}{2.216660in}}%
\pgfusepath{clip}%
\pgfsetbuttcap%
\pgfsetroundjoin%
\definecolor{currentfill}{rgb}{0.279566,0.067836,0.391917}%
\pgfsetfillcolor{currentfill}%
\pgfsetlinewidth{0.000000pt}%
\definecolor{currentstroke}{rgb}{0.000000,0.000000,0.000000}%
\pgfsetstrokecolor{currentstroke}%
\pgfsetdash{}{0pt}%
\pgfpathmoveto{\pgfqpoint{1.547745in}{0.720909in}}%
\pgfpathlineto{\pgfqpoint{1.550451in}{0.715464in}}%
\pgfpathlineto{\pgfqpoint{1.553161in}{0.710187in}}%
\pgfpathlineto{\pgfqpoint{1.555874in}{0.705082in}}%
\pgfpathlineto{\pgfqpoint{1.558590in}{0.700155in}}%
\pgfpathlineto{\pgfqpoint{1.546430in}{0.693862in}}%
\pgfpathlineto{\pgfqpoint{1.533874in}{0.687773in}}%
\pgfpathlineto{\pgfqpoint{1.520934in}{0.681895in}}%
\pgfpathlineto{\pgfqpoint{1.507623in}{0.676234in}}%
\pgfpathlineto{\pgfqpoint{1.505265in}{0.681350in}}%
\pgfpathlineto{\pgfqpoint{1.502910in}{0.686642in}}%
\pgfpathlineto{\pgfqpoint{1.500558in}{0.692107in}}%
\pgfpathlineto{\pgfqpoint{1.498208in}{0.697741in}}%
\pgfpathlineto{\pgfqpoint{1.511144in}{0.703223in}}%
\pgfpathlineto{\pgfqpoint{1.523720in}{0.708916in}}%
\pgfpathlineto{\pgfqpoint{1.535925in}{0.714814in}}%
\pgfpathlineto{\pgfqpoint{1.547745in}{0.720909in}}%
\pgfpathclose%
\pgfusepath{fill}%
\end{pgfscope}%
\begin{pgfscope}%
\pgfpathrectangle{\pgfqpoint{0.041670in}{0.041670in}}{\pgfqpoint{2.216660in}{2.216660in}}%
\pgfusepath{clip}%
\pgfsetbuttcap%
\pgfsetroundjoin%
\definecolor{currentfill}{rgb}{0.896320,0.893616,0.096335}%
\pgfsetfillcolor{currentfill}%
\pgfsetlinewidth{0.000000pt}%
\definecolor{currentstroke}{rgb}{0.000000,0.000000,0.000000}%
\pgfsetstrokecolor{currentstroke}%
\pgfsetdash{}{0pt}%
\pgfpathmoveto{\pgfqpoint{1.134165in}{1.656683in}}%
\pgfpathlineto{\pgfqpoint{1.131893in}{1.655468in}}%
\pgfpathlineto{\pgfqpoint{1.129622in}{1.654129in}}%
\pgfpathlineto{\pgfqpoint{1.127354in}{1.652667in}}%
\pgfpathlineto{\pgfqpoint{1.125087in}{1.651081in}}%
\pgfpathlineto{\pgfqpoint{1.127498in}{1.651870in}}%
\pgfpathlineto{\pgfqpoint{1.129960in}{1.652624in}}%
\pgfpathlineto{\pgfqpoint{1.132471in}{1.653341in}}%
\pgfpathlineto{\pgfqpoint{1.135029in}{1.654020in}}%
\pgfpathlineto{\pgfqpoint{1.136884in}{1.655484in}}%
\pgfpathlineto{\pgfqpoint{1.138741in}{1.656825in}}%
\pgfpathlineto{\pgfqpoint{1.140599in}{1.658042in}}%
\pgfpathlineto{\pgfqpoint{1.142460in}{1.659134in}}%
\pgfpathlineto{\pgfqpoint{1.140326in}{1.658567in}}%
\pgfpathlineto{\pgfqpoint{1.138231in}{1.657969in}}%
\pgfpathlineto{\pgfqpoint{1.136176in}{1.657341in}}%
\pgfpathlineto{\pgfqpoint{1.134165in}{1.656683in}}%
\pgfpathclose%
\pgfusepath{fill}%
\end{pgfscope}%
\begin{pgfscope}%
\pgfpathrectangle{\pgfqpoint{0.041670in}{0.041670in}}{\pgfqpoint{2.216660in}{2.216660in}}%
\pgfusepath{clip}%
\pgfsetbuttcap%
\pgfsetroundjoin%
\definecolor{currentfill}{rgb}{0.283072,0.130895,0.449241}%
\pgfsetfillcolor{currentfill}%
\pgfsetlinewidth{0.000000pt}%
\definecolor{currentstroke}{rgb}{0.000000,0.000000,0.000000}%
\pgfsetstrokecolor{currentstroke}%
\pgfsetdash{}{0pt}%
\pgfpathmoveto{\pgfqpoint{1.526190in}{0.770039in}}%
\pgfpathlineto{\pgfqpoint{1.528876in}{0.763394in}}%
\pgfpathlineto{\pgfqpoint{1.531564in}{0.756885in}}%
\pgfpathlineto{\pgfqpoint{1.534255in}{0.750516in}}%
\pgfpathlineto{\pgfqpoint{1.536948in}{0.744291in}}%
\pgfpathlineto{\pgfqpoint{1.525466in}{0.738393in}}%
\pgfpathlineto{\pgfqpoint{1.513613in}{0.732687in}}%
\pgfpathlineto{\pgfqpoint{1.501399in}{0.727180in}}%
\pgfpathlineto{\pgfqpoint{1.488837in}{0.721876in}}%
\pgfpathlineto{\pgfqpoint{1.486500in}{0.728290in}}%
\pgfpathlineto{\pgfqpoint{1.484165in}{0.734847in}}%
\pgfpathlineto{\pgfqpoint{1.481832in}{0.741545in}}%
\pgfpathlineto{\pgfqpoint{1.479501in}{0.748378in}}%
\pgfpathlineto{\pgfqpoint{1.491690in}{0.753503in}}%
\pgfpathlineto{\pgfqpoint{1.503542in}{0.758826in}}%
\pgfpathlineto{\pgfqpoint{1.515046in}{0.764340in}}%
\pgfpathlineto{\pgfqpoint{1.526190in}{0.770039in}}%
\pgfpathclose%
\pgfusepath{fill}%
\end{pgfscope}%
\begin{pgfscope}%
\pgfpathrectangle{\pgfqpoint{0.041670in}{0.041670in}}{\pgfqpoint{2.216660in}{2.216660in}}%
\pgfusepath{clip}%
\pgfsetbuttcap%
\pgfsetroundjoin%
\definecolor{currentfill}{rgb}{0.147607,0.511733,0.557049}%
\pgfsetfillcolor{currentfill}%
\pgfsetlinewidth{0.000000pt}%
\definecolor{currentstroke}{rgb}{0.000000,0.000000,0.000000}%
\pgfsetstrokecolor{currentstroke}%
\pgfsetdash{}{0pt}%
\pgfpathmoveto{\pgfqpoint{1.471747in}{1.139719in}}%
\pgfpathlineto{\pgfqpoint{1.474995in}{1.130962in}}%
\pgfpathlineto{\pgfqpoint{1.478242in}{1.122208in}}%
\pgfpathlineto{\pgfqpoint{1.481488in}{1.113458in}}%
\pgfpathlineto{\pgfqpoint{1.484732in}{1.104715in}}%
\pgfpathlineto{\pgfqpoint{1.478934in}{1.099896in}}%
\pgfpathlineto{\pgfqpoint{1.472827in}{1.095168in}}%
\pgfpathlineto{\pgfqpoint{1.466414in}{1.090537in}}%
\pgfpathlineto{\pgfqpoint{1.459704in}{1.086008in}}%
\pgfpathlineto{\pgfqpoint{1.456718in}{1.094973in}}%
\pgfpathlineto{\pgfqpoint{1.453731in}{1.103945in}}%
\pgfpathlineto{\pgfqpoint{1.450743in}{1.112922in}}%
\pgfpathlineto{\pgfqpoint{1.447754in}{1.121900in}}%
\pgfpathlineto{\pgfqpoint{1.454186in}{1.126213in}}%
\pgfpathlineto{\pgfqpoint{1.460333in}{1.130624in}}%
\pgfpathlineto{\pgfqpoint{1.466188in}{1.135128in}}%
\pgfpathlineto{\pgfqpoint{1.471747in}{1.139719in}}%
\pgfpathclose%
\pgfusepath{fill}%
\end{pgfscope}%
\begin{pgfscope}%
\pgfpathrectangle{\pgfqpoint{0.041670in}{0.041670in}}{\pgfqpoint{2.216660in}{2.216660in}}%
\pgfusepath{clip}%
\pgfsetbuttcap%
\pgfsetroundjoin%
\definecolor{currentfill}{rgb}{0.281477,0.755203,0.432552}%
\pgfsetfillcolor{currentfill}%
\pgfsetlinewidth{0.000000pt}%
\definecolor{currentstroke}{rgb}{0.000000,0.000000,0.000000}%
\pgfsetstrokecolor{currentstroke}%
\pgfsetdash{}{0pt}%
\pgfpathmoveto{\pgfqpoint{0.958528in}{1.397829in}}%
\pgfpathlineto{\pgfqpoint{0.954890in}{1.390490in}}%
\pgfpathlineto{\pgfqpoint{0.951254in}{1.383083in}}%
\pgfpathlineto{\pgfqpoint{0.947621in}{1.375612in}}%
\pgfpathlineto{\pgfqpoint{0.943990in}{1.368077in}}%
\pgfpathlineto{\pgfqpoint{0.941753in}{1.371717in}}%
\pgfpathlineto{\pgfqpoint{0.939756in}{1.375388in}}%
\pgfpathlineto{\pgfqpoint{0.938002in}{1.379086in}}%
\pgfpathlineto{\pgfqpoint{0.936492in}{1.382806in}}%
\pgfpathlineto{\pgfqpoint{0.940230in}{1.390099in}}%
\pgfpathlineto{\pgfqpoint{0.943971in}{1.397328in}}%
\pgfpathlineto{\pgfqpoint{0.947714in}{1.404493in}}%
\pgfpathlineto{\pgfqpoint{0.951461in}{1.411592in}}%
\pgfpathlineto{\pgfqpoint{0.952886in}{1.408115in}}%
\pgfpathlineto{\pgfqpoint{0.954539in}{1.404660in}}%
\pgfpathlineto{\pgfqpoint{0.956420in}{1.401230in}}%
\pgfpathlineto{\pgfqpoint{0.958528in}{1.397829in}}%
\pgfpathclose%
\pgfusepath{fill}%
\end{pgfscope}%
\begin{pgfscope}%
\pgfpathrectangle{\pgfqpoint{0.041670in}{0.041670in}}{\pgfqpoint{2.216660in}{2.216660in}}%
\pgfusepath{clip}%
\pgfsetbuttcap%
\pgfsetroundjoin%
\definecolor{currentfill}{rgb}{0.565498,0.842430,0.262877}%
\pgfsetfillcolor{currentfill}%
\pgfsetlinewidth{0.000000pt}%
\definecolor{currentstroke}{rgb}{0.000000,0.000000,0.000000}%
\pgfsetstrokecolor{currentstroke}%
\pgfsetdash{}{0pt}%
\pgfpathmoveto{\pgfqpoint{1.350205in}{1.537178in}}%
\pgfpathlineto{\pgfqpoint{1.354025in}{1.531942in}}%
\pgfpathlineto{\pgfqpoint{1.357843in}{1.526610in}}%
\pgfpathlineto{\pgfqpoint{1.361658in}{1.521184in}}%
\pgfpathlineto{\pgfqpoint{1.365470in}{1.515666in}}%
\pgfpathlineto{\pgfqpoint{1.365955in}{1.512875in}}%
\pgfpathlineto{\pgfqpoint{1.366256in}{1.510077in}}%
\pgfpathlineto{\pgfqpoint{1.366372in}{1.507274in}}%
\pgfpathlineto{\pgfqpoint{1.366301in}{1.504470in}}%
\pgfpathlineto{\pgfqpoint{1.362466in}{1.510228in}}%
\pgfpathlineto{\pgfqpoint{1.358628in}{1.515893in}}%
\pgfpathlineto{\pgfqpoint{1.354787in}{1.521464in}}%
\pgfpathlineto{\pgfqpoint{1.350944in}{1.526938in}}%
\pgfpathlineto{\pgfqpoint{1.351015in}{1.529502in}}%
\pgfpathlineto{\pgfqpoint{1.350915in}{1.532065in}}%
\pgfpathlineto{\pgfqpoint{1.350645in}{1.534625in}}%
\pgfpathlineto{\pgfqpoint{1.350205in}{1.537178in}}%
\pgfpathclose%
\pgfusepath{fill}%
\end{pgfscope}%
\begin{pgfscope}%
\pgfpathrectangle{\pgfqpoint{0.041670in}{0.041670in}}{\pgfqpoint{2.216660in}{2.216660in}}%
\pgfusepath{clip}%
\pgfsetbuttcap%
\pgfsetroundjoin%
\definecolor{currentfill}{rgb}{0.412913,0.803041,0.357269}%
\pgfsetfillcolor{currentfill}%
\pgfsetlinewidth{0.000000pt}%
\definecolor{currentstroke}{rgb}{0.000000,0.000000,0.000000}%
\pgfsetstrokecolor{currentstroke}%
\pgfsetdash{}{0pt}%
\pgfpathmoveto{\pgfqpoint{0.981530in}{1.465730in}}%
\pgfpathlineto{\pgfqpoint{0.977762in}{1.459239in}}%
\pgfpathlineto{\pgfqpoint{0.973996in}{1.452665in}}%
\pgfpathlineto{\pgfqpoint{0.970234in}{1.446010in}}%
\pgfpathlineto{\pgfqpoint{0.966474in}{1.439276in}}%
\pgfpathlineto{\pgfqpoint{0.965350in}{1.442528in}}%
\pgfpathlineto{\pgfqpoint{0.964441in}{1.445794in}}%
\pgfpathlineto{\pgfqpoint{0.963749in}{1.449070in}}%
\pgfpathlineto{\pgfqpoint{0.963273in}{1.452354in}}%
\pgfpathlineto{\pgfqpoint{0.967082in}{1.458844in}}%
\pgfpathlineto{\pgfqpoint{0.970894in}{1.465257in}}%
\pgfpathlineto{\pgfqpoint{0.974709in}{1.471589in}}%
\pgfpathlineto{\pgfqpoint{0.978527in}{1.477839in}}%
\pgfpathlineto{\pgfqpoint{0.978976in}{1.474798in}}%
\pgfpathlineto{\pgfqpoint{0.979627in}{1.471765in}}%
\pgfpathlineto{\pgfqpoint{0.980478in}{1.468741in}}%
\pgfpathlineto{\pgfqpoint{0.981530in}{1.465730in}}%
\pgfpathclose%
\pgfusepath{fill}%
\end{pgfscope}%
\begin{pgfscope}%
\pgfpathrectangle{\pgfqpoint{0.041670in}{0.041670in}}{\pgfqpoint{2.216660in}{2.216660in}}%
\pgfusepath{clip}%
\pgfsetbuttcap%
\pgfsetroundjoin%
\definecolor{currentfill}{rgb}{0.855810,0.888601,0.097452}%
\pgfsetfillcolor{currentfill}%
\pgfsetlinewidth{0.000000pt}%
\definecolor{currentstroke}{rgb}{0.000000,0.000000,0.000000}%
\pgfsetstrokecolor{currentstroke}%
\pgfsetdash{}{0pt}%
\pgfpathmoveto{\pgfqpoint{1.250281in}{1.644502in}}%
\pgfpathlineto{\pgfqpoint{1.253187in}{1.642520in}}%
\pgfpathlineto{\pgfqpoint{1.256091in}{1.640418in}}%
\pgfpathlineto{\pgfqpoint{1.258992in}{1.638196in}}%
\pgfpathlineto{\pgfqpoint{1.261890in}{1.635854in}}%
\pgfpathlineto{\pgfqpoint{1.264118in}{1.634629in}}%
\pgfpathlineto{\pgfqpoint{1.266264in}{1.633372in}}%
\pgfpathlineto{\pgfqpoint{1.268324in}{1.632083in}}%
\pgfpathlineto{\pgfqpoint{1.270298in}{1.630764in}}%
\pgfpathlineto{\pgfqpoint{1.267099in}{1.633288in}}%
\pgfpathlineto{\pgfqpoint{1.263898in}{1.635692in}}%
\pgfpathlineto{\pgfqpoint{1.260695in}{1.637976in}}%
\pgfpathlineto{\pgfqpoint{1.257489in}{1.640139in}}%
\pgfpathlineto{\pgfqpoint{1.255797in}{1.641270in}}%
\pgfpathlineto{\pgfqpoint{1.254031in}{1.642374in}}%
\pgfpathlineto{\pgfqpoint{1.252192in}{1.643452in}}%
\pgfpathlineto{\pgfqpoint{1.250281in}{1.644502in}}%
\pgfpathclose%
\pgfusepath{fill}%
\end{pgfscope}%
\begin{pgfscope}%
\pgfpathrectangle{\pgfqpoint{0.041670in}{0.041670in}}{\pgfqpoint{2.216660in}{2.216660in}}%
\pgfusepath{clip}%
\pgfsetbuttcap%
\pgfsetroundjoin%
\definecolor{currentfill}{rgb}{0.274952,0.037752,0.364543}%
\pgfsetfillcolor{currentfill}%
\pgfsetlinewidth{0.000000pt}%
\definecolor{currentstroke}{rgb}{0.000000,0.000000,0.000000}%
\pgfsetstrokecolor{currentstroke}%
\pgfsetdash{}{0pt}%
\pgfpathmoveto{\pgfqpoint{1.558590in}{0.700155in}}%
\pgfpathlineto{\pgfqpoint{1.561310in}{0.695408in}}%
\pgfpathlineto{\pgfqpoint{1.564033in}{0.690847in}}%
\pgfpathlineto{\pgfqpoint{1.566761in}{0.686476in}}%
\pgfpathlineto{\pgfqpoint{1.569492in}{0.682299in}}%
\pgfpathlineto{\pgfqpoint{1.556992in}{0.675809in}}%
\pgfpathlineto{\pgfqpoint{1.544082in}{0.669529in}}%
\pgfpathlineto{\pgfqpoint{1.530777in}{0.663466in}}%
\pgfpathlineto{\pgfqpoint{1.517089in}{0.657628in}}%
\pgfpathlineto{\pgfqpoint{1.514717in}{0.661992in}}%
\pgfpathlineto{\pgfqpoint{1.512349in}{0.666551in}}%
\pgfpathlineto{\pgfqpoint{1.509984in}{0.671300in}}%
\pgfpathlineto{\pgfqpoint{1.507623in}{0.676234in}}%
\pgfpathlineto{\pgfqpoint{1.520934in}{0.681895in}}%
\pgfpathlineto{\pgfqpoint{1.533874in}{0.687773in}}%
\pgfpathlineto{\pgfqpoint{1.546430in}{0.693862in}}%
\pgfpathlineto{\pgfqpoint{1.558590in}{0.700155in}}%
\pgfpathclose%
\pgfusepath{fill}%
\end{pgfscope}%
\begin{pgfscope}%
\pgfpathrectangle{\pgfqpoint{0.041670in}{0.041670in}}{\pgfqpoint{2.216660in}{2.216660in}}%
\pgfusepath{clip}%
\pgfsetbuttcap%
\pgfsetroundjoin%
\definecolor{currentfill}{rgb}{0.280255,0.165693,0.476498}%
\pgfsetfillcolor{currentfill}%
\pgfsetlinewidth{0.000000pt}%
\definecolor{currentstroke}{rgb}{0.000000,0.000000,0.000000}%
\pgfsetstrokecolor{currentstroke}%
\pgfsetdash{}{0pt}%
\pgfpathmoveto{\pgfqpoint{1.515462in}{0.797902in}}%
\pgfpathlineto{\pgfqpoint{1.518142in}{0.790752in}}%
\pgfpathlineto{\pgfqpoint{1.520823in}{0.783722in}}%
\pgfpathlineto{\pgfqpoint{1.523505in}{0.776816in}}%
\pgfpathlineto{\pgfqpoint{1.526190in}{0.770039in}}%
\pgfpathlineto{\pgfqpoint{1.515046in}{0.764340in}}%
\pgfpathlineto{\pgfqpoint{1.503542in}{0.758826in}}%
\pgfpathlineto{\pgfqpoint{1.491690in}{0.753503in}}%
\pgfpathlineto{\pgfqpoint{1.479501in}{0.748378in}}%
\pgfpathlineto{\pgfqpoint{1.477172in}{0.755344in}}%
\pgfpathlineto{\pgfqpoint{1.474844in}{0.762437in}}%
\pgfpathlineto{\pgfqpoint{1.472518in}{0.769655in}}%
\pgfpathlineto{\pgfqpoint{1.470194in}{0.776994in}}%
\pgfpathlineto{\pgfqpoint{1.482011in}{0.781940in}}%
\pgfpathlineto{\pgfqpoint{1.493502in}{0.787078in}}%
\pgfpathlineto{\pgfqpoint{1.504656in}{0.792400in}}%
\pgfpathlineto{\pgfqpoint{1.515462in}{0.797902in}}%
\pgfpathclose%
\pgfusepath{fill}%
\end{pgfscope}%
\begin{pgfscope}%
\pgfpathrectangle{\pgfqpoint{0.041670in}{0.041670in}}{\pgfqpoint{2.216660in}{2.216660in}}%
\pgfusepath{clip}%
\pgfsetbuttcap%
\pgfsetroundjoin%
\definecolor{currentfill}{rgb}{0.260571,0.246922,0.522828}%
\pgfsetfillcolor{currentfill}%
\pgfsetlinewidth{0.000000pt}%
\definecolor{currentstroke}{rgb}{0.000000,0.000000,0.000000}%
\pgfsetstrokecolor{currentstroke}%
\pgfsetdash{}{0pt}%
\pgfpathmoveto{\pgfqpoint{1.829892in}{0.847724in}}%
\pgfpathlineto{\pgfqpoint{1.833806in}{0.856488in}}%
\pgfpathlineto{\pgfqpoint{1.837738in}{0.865676in}}%
\pgfpathlineto{\pgfqpoint{1.841688in}{0.875296in}}%
\pgfpathlineto{\pgfqpoint{1.845657in}{0.885354in}}%
\pgfpathlineto{\pgfqpoint{1.836928in}{0.874460in}}%
\pgfpathlineto{\pgfqpoint{1.827503in}{0.863699in}}%
\pgfpathlineto{\pgfqpoint{1.817387in}{0.853084in}}%
\pgfpathlineto{\pgfqpoint{1.806589in}{0.842626in}}%
\pgfpathlineto{\pgfqpoint{1.802831in}{0.832765in}}%
\pgfpathlineto{\pgfqpoint{1.799091in}{0.823344in}}%
\pgfpathlineto{\pgfqpoint{1.795369in}{0.814357in}}%
\pgfpathlineto{\pgfqpoint{1.791664in}{0.805796in}}%
\pgfpathlineto{\pgfqpoint{1.802226in}{0.816057in}}%
\pgfpathlineto{\pgfqpoint{1.812123in}{0.826473in}}%
\pgfpathlineto{\pgfqpoint{1.821348in}{0.837033in}}%
\pgfpathlineto{\pgfqpoint{1.829892in}{0.847724in}}%
\pgfpathclose%
\pgfusepath{fill}%
\end{pgfscope}%
\begin{pgfscope}%
\pgfpathrectangle{\pgfqpoint{0.041670in}{0.041670in}}{\pgfqpoint{2.216660in}{2.216660in}}%
\pgfusepath{clip}%
\pgfsetbuttcap%
\pgfsetroundjoin%
\definecolor{currentfill}{rgb}{0.120081,0.622161,0.534946}%
\pgfsetfillcolor{currentfill}%
\pgfsetlinewidth{0.000000pt}%
\definecolor{currentstroke}{rgb}{0.000000,0.000000,0.000000}%
\pgfsetstrokecolor{currentstroke}%
\pgfsetdash{}{0pt}%
\pgfpathmoveto{\pgfqpoint{1.449335in}{1.260260in}}%
\pgfpathlineto{\pgfqpoint{1.452811in}{1.251928in}}%
\pgfpathlineto{\pgfqpoint{1.456286in}{1.243566in}}%
\pgfpathlineto{\pgfqpoint{1.459759in}{1.235176in}}%
\pgfpathlineto{\pgfqpoint{1.463229in}{1.226762in}}%
\pgfpathlineto{\pgfqpoint{1.459272in}{1.222349in}}%
\pgfpathlineto{\pgfqpoint{1.455029in}{1.217998in}}%
\pgfpathlineto{\pgfqpoint{1.450504in}{1.213713in}}%
\pgfpathlineto{\pgfqpoint{1.445700in}{1.209499in}}%
\pgfpathlineto{\pgfqpoint{1.442436in}{1.218147in}}%
\pgfpathlineto{\pgfqpoint{1.439171in}{1.226769in}}%
\pgfpathlineto{\pgfqpoint{1.435903in}{1.235363in}}%
\pgfpathlineto{\pgfqpoint{1.432634in}{1.243926in}}%
\pgfpathlineto{\pgfqpoint{1.437210in}{1.247913in}}%
\pgfpathlineto{\pgfqpoint{1.441521in}{1.251967in}}%
\pgfpathlineto{\pgfqpoint{1.445564in}{1.256084in}}%
\pgfpathlineto{\pgfqpoint{1.449335in}{1.260260in}}%
\pgfpathclose%
\pgfusepath{fill}%
\end{pgfscope}%
\begin{pgfscope}%
\pgfpathrectangle{\pgfqpoint{0.041670in}{0.041670in}}{\pgfqpoint{2.216660in}{2.216660in}}%
\pgfusepath{clip}%
\pgfsetbuttcap%
\pgfsetroundjoin%
\definecolor{currentfill}{rgb}{0.212395,0.359683,0.551710}%
\pgfsetfillcolor{currentfill}%
\pgfsetlinewidth{0.000000pt}%
\definecolor{currentstroke}{rgb}{0.000000,0.000000,0.000000}%
\pgfsetstrokecolor{currentstroke}%
\pgfsetdash{}{0pt}%
\pgfpathmoveto{\pgfqpoint{0.905945in}{0.956582in}}%
\pgfpathlineto{\pgfqpoint{0.903352in}{0.947734in}}%
\pgfpathlineto{\pgfqpoint{0.900760in}{0.938937in}}%
\pgfpathlineto{\pgfqpoint{0.898167in}{0.930193in}}%
\pgfpathlineto{\pgfqpoint{0.895575in}{0.921505in}}%
\pgfpathlineto{\pgfqpoint{0.886152in}{0.926239in}}%
\pgfpathlineto{\pgfqpoint{0.877041in}{0.931120in}}%
\pgfpathlineto{\pgfqpoint{0.868251in}{0.936145in}}%
\pgfpathlineto{\pgfqpoint{0.859789in}{0.941307in}}%
\pgfpathlineto{\pgfqpoint{0.862700in}{0.949791in}}%
\pgfpathlineto{\pgfqpoint{0.865611in}{0.958331in}}%
\pgfpathlineto{\pgfqpoint{0.868522in}{0.966925in}}%
\pgfpathlineto{\pgfqpoint{0.871433in}{0.975570in}}%
\pgfpathlineto{\pgfqpoint{0.879595in}{0.970620in}}%
\pgfpathlineto{\pgfqpoint{0.888073in}{0.965801in}}%
\pgfpathlineto{\pgfqpoint{0.896859in}{0.961120in}}%
\pgfpathlineto{\pgfqpoint{0.905945in}{0.956582in}}%
\pgfpathclose%
\pgfusepath{fill}%
\end{pgfscope}%
\begin{pgfscope}%
\pgfpathrectangle{\pgfqpoint{0.041670in}{0.041670in}}{\pgfqpoint{2.216660in}{2.216660in}}%
\pgfusepath{clip}%
\pgfsetbuttcap%
\pgfsetroundjoin%
\definecolor{currentfill}{rgb}{0.166383,0.690856,0.496502}%
\pgfsetfillcolor{currentfill}%
\pgfsetlinewidth{0.000000pt}%
\definecolor{currentstroke}{rgb}{0.000000,0.000000,0.000000}%
\pgfsetstrokecolor{currentstroke}%
\pgfsetdash{}{0pt}%
\pgfpathmoveto{\pgfqpoint{0.941476in}{1.322230in}}%
\pgfpathlineto{\pgfqpoint{0.938023in}{1.314177in}}%
\pgfpathlineto{\pgfqpoint{0.934572in}{1.306075in}}%
\pgfpathlineto{\pgfqpoint{0.931124in}{1.297925in}}%
\pgfpathlineto{\pgfqpoint{0.927677in}{1.289729in}}%
\pgfpathlineto{\pgfqpoint{0.924122in}{1.293676in}}%
\pgfpathlineto{\pgfqpoint{0.920828in}{1.297673in}}%
\pgfpathlineto{\pgfqpoint{0.917798in}{1.301717in}}%
\pgfpathlineto{\pgfqpoint{0.915035in}{1.305803in}}%
\pgfpathlineto{\pgfqpoint{0.918646in}{1.313760in}}%
\pgfpathlineto{\pgfqpoint{0.922259in}{1.321672in}}%
\pgfpathlineto{\pgfqpoint{0.925874in}{1.329538in}}%
\pgfpathlineto{\pgfqpoint{0.929492in}{1.337354in}}%
\pgfpathlineto{\pgfqpoint{0.932113in}{1.333508in}}%
\pgfpathlineto{\pgfqpoint{0.934986in}{1.329703in}}%
\pgfpathlineto{\pgfqpoint{0.938107in}{1.325942in}}%
\pgfpathlineto{\pgfqpoint{0.941476in}{1.322230in}}%
\pgfpathclose%
\pgfusepath{fill}%
\end{pgfscope}%
\begin{pgfscope}%
\pgfpathrectangle{\pgfqpoint{0.041670in}{0.041670in}}{\pgfqpoint{2.216660in}{2.216660in}}%
\pgfusepath{clip}%
\pgfsetbuttcap%
\pgfsetroundjoin%
\definecolor{currentfill}{rgb}{0.896320,0.893616,0.096335}%
\pgfsetfillcolor{currentfill}%
\pgfsetlinewidth{0.000000pt}%
\definecolor{currentstroke}{rgb}{0.000000,0.000000,0.000000}%
\pgfsetstrokecolor{currentstroke}%
\pgfsetdash{}{0pt}%
\pgfpathmoveto{\pgfqpoint{1.223959in}{1.657269in}}%
\pgfpathlineto{\pgfqpoint{1.226143in}{1.656084in}}%
\pgfpathlineto{\pgfqpoint{1.228324in}{1.654774in}}%
\pgfpathlineto{\pgfqpoint{1.230504in}{1.653341in}}%
\pgfpathlineto{\pgfqpoint{1.232682in}{1.651784in}}%
\pgfpathlineto{\pgfqpoint{1.235087in}{1.650991in}}%
\pgfpathlineto{\pgfqpoint{1.237438in}{1.650163in}}%
\pgfpathlineto{\pgfqpoint{1.239733in}{1.649300in}}%
\pgfpathlineto{\pgfqpoint{1.241968in}{1.648403in}}%
\pgfpathlineto{\pgfqpoint{1.239405in}{1.650100in}}%
\pgfpathlineto{\pgfqpoint{1.236840in}{1.651674in}}%
\pgfpathlineto{\pgfqpoint{1.234273in}{1.653125in}}%
\pgfpathlineto{\pgfqpoint{1.231704in}{1.654450in}}%
\pgfpathlineto{\pgfqpoint{1.229839in}{1.655197in}}%
\pgfpathlineto{\pgfqpoint{1.227926in}{1.655917in}}%
\pgfpathlineto{\pgfqpoint{1.225965in}{1.656608in}}%
\pgfpathlineto{\pgfqpoint{1.223959in}{1.657269in}}%
\pgfpathclose%
\pgfusepath{fill}%
\end{pgfscope}%
\begin{pgfscope}%
\pgfpathrectangle{\pgfqpoint{0.041670in}{0.041670in}}{\pgfqpoint{2.216660in}{2.216660in}}%
\pgfusepath{clip}%
\pgfsetbuttcap%
\pgfsetroundjoin%
\definecolor{currentfill}{rgb}{0.855810,0.888601,0.097452}%
\pgfsetfillcolor{currentfill}%
\pgfsetlinewidth{0.000000pt}%
\definecolor{currentstroke}{rgb}{0.000000,0.000000,0.000000}%
\pgfsetstrokecolor{currentstroke}%
\pgfsetdash{}{0pt}%
\pgfpathmoveto{\pgfqpoint{1.100981in}{1.639114in}}%
\pgfpathlineto{\pgfqpoint{1.097715in}{1.636908in}}%
\pgfpathlineto{\pgfqpoint{1.094452in}{1.634581in}}%
\pgfpathlineto{\pgfqpoint{1.091191in}{1.632134in}}%
\pgfpathlineto{\pgfqpoint{1.087932in}{1.629568in}}%
\pgfpathlineto{\pgfqpoint{1.089827in}{1.630912in}}%
\pgfpathlineto{\pgfqpoint{1.091810in}{1.632228in}}%
\pgfpathlineto{\pgfqpoint{1.093880in}{1.633513in}}%
\pgfpathlineto{\pgfqpoint{1.096035in}{1.634767in}}%
\pgfpathlineto{\pgfqpoint{1.099004in}{1.637147in}}%
\pgfpathlineto{\pgfqpoint{1.101976in}{1.639408in}}%
\pgfpathlineto{\pgfqpoint{1.104950in}{1.641550in}}%
\pgfpathlineto{\pgfqpoint{1.107927in}{1.643570in}}%
\pgfpathlineto{\pgfqpoint{1.106079in}{1.642495in}}%
\pgfpathlineto{\pgfqpoint{1.104305in}{1.641394in}}%
\pgfpathlineto{\pgfqpoint{1.102605in}{1.640266in}}%
\pgfpathlineto{\pgfqpoint{1.100981in}{1.639114in}}%
\pgfpathclose%
\pgfusepath{fill}%
\end{pgfscope}%
\begin{pgfscope}%
\pgfpathrectangle{\pgfqpoint{0.041670in}{0.041670in}}{\pgfqpoint{2.216660in}{2.216660in}}%
\pgfusepath{clip}%
\pgfsetbuttcap%
\pgfsetroundjoin%
\definecolor{currentfill}{rgb}{0.195860,0.395433,0.555276}%
\pgfsetfillcolor{currentfill}%
\pgfsetlinewidth{0.000000pt}%
\definecolor{currentstroke}{rgb}{0.000000,0.000000,0.000000}%
\pgfsetstrokecolor{currentstroke}%
\pgfsetdash{}{0pt}%
\pgfpathmoveto{\pgfqpoint{1.483552in}{1.014902in}}%
\pgfpathlineto{\pgfqpoint{1.486530in}{1.006136in}}%
\pgfpathlineto{\pgfqpoint{1.489507in}{0.997408in}}%
\pgfpathlineto{\pgfqpoint{1.492483in}{0.988720in}}%
\pgfpathlineto{\pgfqpoint{1.495459in}{0.980077in}}%
\pgfpathlineto{\pgfqpoint{1.487586in}{0.975014in}}%
\pgfpathlineto{\pgfqpoint{1.479389in}{0.970078in}}%
\pgfpathlineto{\pgfqpoint{1.470876in}{0.965274in}}%
\pgfpathlineto{\pgfqpoint{1.462056in}{0.960609in}}%
\pgfpathlineto{\pgfqpoint{1.459388in}{0.969460in}}%
\pgfpathlineto{\pgfqpoint{1.456720in}{0.978355in}}%
\pgfpathlineto{\pgfqpoint{1.454051in}{0.987290in}}%
\pgfpathlineto{\pgfqpoint{1.451382in}{0.996262in}}%
\pgfpathlineto{\pgfqpoint{1.459875in}{1.000729in}}%
\pgfpathlineto{\pgfqpoint{1.468073in}{1.005328in}}%
\pgfpathlineto{\pgfqpoint{1.475968in}{1.010054in}}%
\pgfpathlineto{\pgfqpoint{1.483552in}{1.014902in}}%
\pgfpathclose%
\pgfusepath{fill}%
\end{pgfscope}%
\begin{pgfscope}%
\pgfpathrectangle{\pgfqpoint{0.041670in}{0.041670in}}{\pgfqpoint{2.216660in}{2.216660in}}%
\pgfusepath{clip}%
\pgfsetbuttcap%
\pgfsetroundjoin%
\definecolor{currentfill}{rgb}{0.282327,0.094955,0.417331}%
\pgfsetfillcolor{currentfill}%
\pgfsetlinewidth{0.000000pt}%
\definecolor{currentstroke}{rgb}{0.000000,0.000000,0.000000}%
\pgfsetstrokecolor{currentstroke}%
\pgfsetdash{}{0pt}%
\pgfpathmoveto{\pgfqpoint{0.668344in}{0.689214in}}%
\pgfpathlineto{\pgfqpoint{0.665148in}{0.692934in}}%
\pgfpathlineto{\pgfqpoint{0.661940in}{0.697003in}}%
\pgfpathlineto{\pgfqpoint{0.658720in}{0.701426in}}%
\pgfpathlineto{\pgfqpoint{0.655488in}{0.706211in}}%
\pgfpathlineto{\pgfqpoint{0.642434in}{0.715163in}}%
\pgfpathlineto{\pgfqpoint{0.629963in}{0.724319in}}%
\pgfpathlineto{\pgfqpoint{0.618088in}{0.733668in}}%
\pgfpathlineto{\pgfqpoint{0.606818in}{0.743200in}}%
\pgfpathlineto{\pgfqpoint{0.610333in}{0.738213in}}%
\pgfpathlineto{\pgfqpoint{0.613834in}{0.733585in}}%
\pgfpathlineto{\pgfqpoint{0.617323in}{0.729311in}}%
\pgfpathlineto{\pgfqpoint{0.620799in}{0.725384in}}%
\pgfpathlineto{\pgfqpoint{0.631812in}{0.716062in}}%
\pgfpathlineto{\pgfqpoint{0.643414in}{0.706920in}}%
\pgfpathlineto{\pgfqpoint{0.655595in}{0.697967in}}%
\pgfpathlineto{\pgfqpoint{0.668344in}{0.689214in}}%
\pgfpathclose%
\pgfusepath{fill}%
\end{pgfscope}%
\begin{pgfscope}%
\pgfpathrectangle{\pgfqpoint{0.041670in}{0.041670in}}{\pgfqpoint{2.216660in}{2.216660in}}%
\pgfusepath{clip}%
\pgfsetbuttcap%
\pgfsetroundjoin%
\definecolor{currentfill}{rgb}{0.565498,0.842430,0.262877}%
\pgfsetfillcolor{currentfill}%
\pgfsetlinewidth{0.000000pt}%
\definecolor{currentstroke}{rgb}{0.000000,0.000000,0.000000}%
\pgfsetstrokecolor{currentstroke}%
\pgfsetdash{}{0pt}%
\pgfpathmoveto{\pgfqpoint{1.009172in}{1.524660in}}%
\pgfpathlineto{\pgfqpoint{1.005332in}{1.519132in}}%
\pgfpathlineto{\pgfqpoint{1.001494in}{1.513509in}}%
\pgfpathlineto{\pgfqpoint{0.997659in}{1.507790in}}%
\pgfpathlineto{\pgfqpoint{0.993827in}{1.501979in}}%
\pgfpathlineto{\pgfqpoint{0.993591in}{1.504782in}}%
\pgfpathlineto{\pgfqpoint{0.993542in}{1.507586in}}%
\pgfpathlineto{\pgfqpoint{0.993678in}{1.510388in}}%
\pgfpathlineto{\pgfqpoint{0.993999in}{1.513186in}}%
\pgfpathlineto{\pgfqpoint{0.997821in}{1.518757in}}%
\pgfpathlineto{\pgfqpoint{1.001646in}{1.524236in}}%
\pgfpathlineto{\pgfqpoint{1.005475in}{1.529620in}}%
\pgfpathlineto{\pgfqpoint{1.009306in}{1.534909in}}%
\pgfpathlineto{\pgfqpoint{1.009017in}{1.532350in}}%
\pgfpathlineto{\pgfqpoint{1.008898in}{1.529787in}}%
\pgfpathlineto{\pgfqpoint{1.008949in}{1.527223in}}%
\pgfpathlineto{\pgfqpoint{1.009172in}{1.524660in}}%
\pgfpathclose%
\pgfusepath{fill}%
\end{pgfscope}%
\begin{pgfscope}%
\pgfpathrectangle{\pgfqpoint{0.041670in}{0.041670in}}{\pgfqpoint{2.216660in}{2.216660in}}%
\pgfusepath{clip}%
\pgfsetbuttcap%
\pgfsetroundjoin%
\definecolor{currentfill}{rgb}{0.896320,0.893616,0.096335}%
\pgfsetfillcolor{currentfill}%
\pgfsetlinewidth{0.000000pt}%
\definecolor{currentstroke}{rgb}{0.000000,0.000000,0.000000}%
\pgfsetstrokecolor{currentstroke}%
\pgfsetdash{}{0pt}%
\pgfpathmoveto{\pgfqpoint{1.126591in}{1.653763in}}%
\pgfpathlineto{\pgfqpoint{1.123941in}{1.652403in}}%
\pgfpathlineto{\pgfqpoint{1.121294in}{1.650919in}}%
\pgfpathlineto{\pgfqpoint{1.118648in}{1.649311in}}%
\pgfpathlineto{\pgfqpoint{1.116005in}{1.647579in}}%
\pgfpathlineto{\pgfqpoint{1.118187in}{1.648505in}}%
\pgfpathlineto{\pgfqpoint{1.120429in}{1.649397in}}%
\pgfpathlineto{\pgfqpoint{1.122730in}{1.650256in}}%
\pgfpathlineto{\pgfqpoint{1.125087in}{1.651081in}}%
\pgfpathlineto{\pgfqpoint{1.127354in}{1.652667in}}%
\pgfpathlineto{\pgfqpoint{1.129622in}{1.654129in}}%
\pgfpathlineto{\pgfqpoint{1.131893in}{1.655468in}}%
\pgfpathlineto{\pgfqpoint{1.134165in}{1.656683in}}%
\pgfpathlineto{\pgfqpoint{1.132199in}{1.655995in}}%
\pgfpathlineto{\pgfqpoint{1.130280in}{1.655279in}}%
\pgfpathlineto{\pgfqpoint{1.128410in}{1.654535in}}%
\pgfpathlineto{\pgfqpoint{1.126591in}{1.653763in}}%
\pgfpathclose%
\pgfusepath{fill}%
\end{pgfscope}%
\begin{pgfscope}%
\pgfpathrectangle{\pgfqpoint{0.041670in}{0.041670in}}{\pgfqpoint{2.216660in}{2.216660in}}%
\pgfusepath{clip}%
\pgfsetbuttcap%
\pgfsetroundjoin%
\definecolor{currentfill}{rgb}{0.274128,0.199721,0.498911}%
\pgfsetfillcolor{currentfill}%
\pgfsetlinewidth{0.000000pt}%
\definecolor{currentstroke}{rgb}{0.000000,0.000000,0.000000}%
\pgfsetstrokecolor{currentstroke}%
\pgfsetdash{}{0pt}%
\pgfpathmoveto{\pgfqpoint{1.504759in}{0.827634in}}%
\pgfpathlineto{\pgfqpoint{1.507433in}{0.820039in}}%
\pgfpathlineto{\pgfqpoint{1.510108in}{0.812549in}}%
\pgfpathlineto{\pgfqpoint{1.512785in}{0.805169in}}%
\pgfpathlineto{\pgfqpoint{1.515462in}{0.797902in}}%
\pgfpathlineto{\pgfqpoint{1.504656in}{0.792400in}}%
\pgfpathlineto{\pgfqpoint{1.493502in}{0.787078in}}%
\pgfpathlineto{\pgfqpoint{1.482011in}{0.781940in}}%
\pgfpathlineto{\pgfqpoint{1.470194in}{0.776994in}}%
\pgfpathlineto{\pgfqpoint{1.467871in}{0.784449in}}%
\pgfpathlineto{\pgfqpoint{1.465549in}{0.792017in}}%
\pgfpathlineto{\pgfqpoint{1.463228in}{0.799694in}}%
\pgfpathlineto{\pgfqpoint{1.460909in}{0.807477in}}%
\pgfpathlineto{\pgfqpoint{1.472354in}{0.812245in}}%
\pgfpathlineto{\pgfqpoint{1.483485in}{0.817198in}}%
\pgfpathlineto{\pgfqpoint{1.494290in}{0.822329in}}%
\pgfpathlineto{\pgfqpoint{1.504759in}{0.827634in}}%
\pgfpathclose%
\pgfusepath{fill}%
\end{pgfscope}%
\begin{pgfscope}%
\pgfpathrectangle{\pgfqpoint{0.041670in}{0.041670in}}{\pgfqpoint{2.216660in}{2.216660in}}%
\pgfusepath{clip}%
\pgfsetbuttcap%
\pgfsetroundjoin%
\definecolor{currentfill}{rgb}{0.271305,0.019942,0.347269}%
\pgfsetfillcolor{currentfill}%
\pgfsetlinewidth{0.000000pt}%
\definecolor{currentstroke}{rgb}{0.000000,0.000000,0.000000}%
\pgfsetstrokecolor{currentstroke}%
\pgfsetdash{}{0pt}%
\pgfpathmoveto{\pgfqpoint{1.569492in}{0.682299in}}%
\pgfpathlineto{\pgfqpoint{1.572228in}{0.678320in}}%
\pgfpathlineto{\pgfqpoint{1.574968in}{0.674544in}}%
\pgfpathlineto{\pgfqpoint{1.577713in}{0.670976in}}%
\pgfpathlineto{\pgfqpoint{1.580462in}{0.667620in}}%
\pgfpathlineto{\pgfqpoint{1.567619in}{0.660934in}}%
\pgfpathlineto{\pgfqpoint{1.554355in}{0.654464in}}%
\pgfpathlineto{\pgfqpoint{1.540682in}{0.648217in}}%
\pgfpathlineto{\pgfqpoint{1.526616in}{0.642201in}}%
\pgfpathlineto{\pgfqpoint{1.524228in}{0.645744in}}%
\pgfpathlineto{\pgfqpoint{1.521844in}{0.649499in}}%
\pgfpathlineto{\pgfqpoint{1.519465in}{0.653462in}}%
\pgfpathlineto{\pgfqpoint{1.517089in}{0.657628in}}%
\pgfpathlineto{\pgfqpoint{1.530777in}{0.663466in}}%
\pgfpathlineto{\pgfqpoint{1.544082in}{0.669529in}}%
\pgfpathlineto{\pgfqpoint{1.556992in}{0.675809in}}%
\pgfpathlineto{\pgfqpoint{1.569492in}{0.682299in}}%
\pgfpathclose%
\pgfusepath{fill}%
\end{pgfscope}%
\begin{pgfscope}%
\pgfpathrectangle{\pgfqpoint{0.041670in}{0.041670in}}{\pgfqpoint{2.216660in}{2.216660in}}%
\pgfusepath{clip}%
\pgfsetbuttcap%
\pgfsetroundjoin%
\definecolor{currentfill}{rgb}{0.762373,0.876424,0.137064}%
\pgfsetfillcolor{currentfill}%
\pgfsetlinewidth{0.000000pt}%
\definecolor{currentstroke}{rgb}{0.000000,0.000000,0.000000}%
\pgfsetstrokecolor{currentstroke}%
\pgfsetdash{}{0pt}%
\pgfpathmoveto{\pgfqpoint{1.297293in}{1.606383in}}%
\pgfpathlineto{\pgfqpoint{1.300925in}{1.602868in}}%
\pgfpathlineto{\pgfqpoint{1.304555in}{1.599241in}}%
\pgfpathlineto{\pgfqpoint{1.308181in}{1.595503in}}%
\pgfpathlineto{\pgfqpoint{1.311805in}{1.591654in}}%
\pgfpathlineto{\pgfqpoint{1.313241in}{1.589685in}}%
\pgfpathlineto{\pgfqpoint{1.314545in}{1.587696in}}%
\pgfpathlineto{\pgfqpoint{1.315716in}{1.585687in}}%
\pgfpathlineto{\pgfqpoint{1.316754in}{1.583661in}}%
\pgfpathlineto{\pgfqpoint{1.312989in}{1.587734in}}%
\pgfpathlineto{\pgfqpoint{1.309223in}{1.591697in}}%
\pgfpathlineto{\pgfqpoint{1.305453in}{1.595549in}}%
\pgfpathlineto{\pgfqpoint{1.301681in}{1.599287in}}%
\pgfpathlineto{\pgfqpoint{1.300762in}{1.601085in}}%
\pgfpathlineto{\pgfqpoint{1.299724in}{1.602869in}}%
\pgfpathlineto{\pgfqpoint{1.298567in}{1.604635in}}%
\pgfpathlineto{\pgfqpoint{1.297293in}{1.606383in}}%
\pgfpathclose%
\pgfusepath{fill}%
\end{pgfscope}%
\begin{pgfscope}%
\pgfpathrectangle{\pgfqpoint{0.041670in}{0.041670in}}{\pgfqpoint{2.216660in}{2.216660in}}%
\pgfusepath{clip}%
\pgfsetbuttcap%
\pgfsetroundjoin%
\definecolor{currentfill}{rgb}{0.282327,0.094955,0.417331}%
\pgfsetfillcolor{currentfill}%
\pgfsetlinewidth{0.000000pt}%
\definecolor{currentstroke}{rgb}{0.000000,0.000000,0.000000}%
\pgfsetstrokecolor{currentstroke}%
\pgfsetdash{}{0pt}%
\pgfpathmoveto{\pgfqpoint{0.882521in}{0.717339in}}%
\pgfpathlineto{\pgfqpoint{0.880267in}{0.711034in}}%
\pgfpathlineto{\pgfqpoint{0.878010in}{0.704882in}}%
\pgfpathlineto{\pgfqpoint{0.875752in}{0.698886in}}%
\pgfpathlineto{\pgfqpoint{0.873491in}{0.693050in}}%
\pgfpathlineto{\pgfqpoint{0.860247in}{0.698339in}}%
\pgfpathlineto{\pgfqpoint{0.847350in}{0.703845in}}%
\pgfpathlineto{\pgfqpoint{0.834815in}{0.709562in}}%
\pgfpathlineto{\pgfqpoint{0.822652in}{0.715481in}}%
\pgfpathlineto{\pgfqpoint{0.825281in}{0.721135in}}%
\pgfpathlineto{\pgfqpoint{0.827906in}{0.726948in}}%
\pgfpathlineto{\pgfqpoint{0.830529in}{0.732918in}}%
\pgfpathlineto{\pgfqpoint{0.833149in}{0.739039in}}%
\pgfpathlineto{\pgfqpoint{0.844962in}{0.733312in}}%
\pgfpathlineto{\pgfqpoint{0.857137in}{0.727782in}}%
\pgfpathlineto{\pgfqpoint{0.869660in}{0.722455in}}%
\pgfpathlineto{\pgfqpoint{0.882521in}{0.717339in}}%
\pgfpathclose%
\pgfusepath{fill}%
\end{pgfscope}%
\begin{pgfscope}%
\pgfpathrectangle{\pgfqpoint{0.041670in}{0.041670in}}{\pgfqpoint{2.216660in}{2.216660in}}%
\pgfusepath{clip}%
\pgfsetbuttcap%
\pgfsetroundjoin%
\definecolor{currentfill}{rgb}{0.268510,0.009605,0.335427}%
\pgfsetfillcolor{currentfill}%
\pgfsetlinewidth{0.000000pt}%
\definecolor{currentstroke}{rgb}{0.000000,0.000000,0.000000}%
\pgfsetstrokecolor{currentstroke}%
\pgfsetdash{}{0pt}%
\pgfpathmoveto{\pgfqpoint{0.758332in}{0.639134in}}%
\pgfpathlineto{\pgfqpoint{0.755581in}{0.638927in}}%
\pgfpathlineto{\pgfqpoint{0.752823in}{0.639002in}}%
\pgfpathlineto{\pgfqpoint{0.750056in}{0.639362in}}%
\pgfpathlineto{\pgfqpoint{0.747282in}{0.640015in}}%
\pgfpathlineto{\pgfqpoint{0.733100in}{0.647496in}}%
\pgfpathlineto{\pgfqpoint{0.719405in}{0.655206in}}%
\pgfpathlineto{\pgfqpoint{0.706213in}{0.663137in}}%
\pgfpathlineto{\pgfqpoint{0.693536in}{0.671279in}}%
\pgfpathlineto{\pgfqpoint{0.696641in}{0.670429in}}%
\pgfpathlineto{\pgfqpoint{0.699738in}{0.669871in}}%
\pgfpathlineto{\pgfqpoint{0.702827in}{0.669598in}}%
\pgfpathlineto{\pgfqpoint{0.705907in}{0.669606in}}%
\pgfpathlineto{\pgfqpoint{0.718275in}{0.661670in}}%
\pgfpathlineto{\pgfqpoint{0.731144in}{0.653940in}}%
\pgfpathlineto{\pgfqpoint{0.744501in}{0.646425in}}%
\pgfpathlineto{\pgfqpoint{0.758332in}{0.639134in}}%
\pgfpathclose%
\pgfusepath{fill}%
\end{pgfscope}%
\begin{pgfscope}%
\pgfpathrectangle{\pgfqpoint{0.041670in}{0.041670in}}{\pgfqpoint{2.216660in}{2.216660in}}%
\pgfusepath{clip}%
\pgfsetbuttcap%
\pgfsetroundjoin%
\definecolor{currentfill}{rgb}{0.344074,0.780029,0.397381}%
\pgfsetfillcolor{currentfill}%
\pgfsetlinewidth{0.000000pt}%
\definecolor{currentstroke}{rgb}{0.000000,0.000000,0.000000}%
\pgfsetstrokecolor{currentstroke}%
\pgfsetdash{}{0pt}%
\pgfpathmoveto{\pgfqpoint{1.394446in}{1.442166in}}%
\pgfpathlineto{\pgfqpoint{1.398219in}{1.435409in}}%
\pgfpathlineto{\pgfqpoint{1.401990in}{1.428577in}}%
\pgfpathlineto{\pgfqpoint{1.405758in}{1.421673in}}%
\pgfpathlineto{\pgfqpoint{1.409523in}{1.414697in}}%
\pgfpathlineto{\pgfqpoint{1.408302in}{1.411204in}}%
\pgfpathlineto{\pgfqpoint{1.406852in}{1.407730in}}%
\pgfpathlineto{\pgfqpoint{1.405173in}{1.404277in}}%
\pgfpathlineto{\pgfqpoint{1.403266in}{1.400851in}}%
\pgfpathlineto{\pgfqpoint{1.399596in}{1.408068in}}%
\pgfpathlineto{\pgfqpoint{1.395924in}{1.415214in}}%
\pgfpathlineto{\pgfqpoint{1.392249in}{1.422286in}}%
\pgfpathlineto{\pgfqpoint{1.388572in}{1.429283in}}%
\pgfpathlineto{\pgfqpoint{1.390360in}{1.432471in}}%
\pgfpathlineto{\pgfqpoint{1.391936in}{1.435683in}}%
\pgfpathlineto{\pgfqpoint{1.393298in}{1.438916in}}%
\pgfpathlineto{\pgfqpoint{1.394446in}{1.442166in}}%
\pgfpathclose%
\pgfusepath{fill}%
\end{pgfscope}%
\begin{pgfscope}%
\pgfpathrectangle{\pgfqpoint{0.041670in}{0.041670in}}{\pgfqpoint{2.216660in}{2.216660in}}%
\pgfusepath{clip}%
\pgfsetbuttcap%
\pgfsetroundjoin%
\definecolor{currentfill}{rgb}{0.147607,0.511733,0.557049}%
\pgfsetfillcolor{currentfill}%
\pgfsetlinewidth{0.000000pt}%
\definecolor{currentstroke}{rgb}{0.000000,0.000000,0.000000}%
\pgfsetstrokecolor{currentstroke}%
\pgfsetdash{}{0pt}%
\pgfpathmoveto{\pgfqpoint{0.918107in}{1.118151in}}%
\pgfpathlineto{\pgfqpoint{0.915183in}{1.109127in}}%
\pgfpathlineto{\pgfqpoint{0.912260in}{1.100103in}}%
\pgfpathlineto{\pgfqpoint{0.909338in}{1.091084in}}%
\pgfpathlineto{\pgfqpoint{0.906417in}{1.082072in}}%
\pgfpathlineto{\pgfqpoint{0.899446in}{1.086506in}}%
\pgfpathlineto{\pgfqpoint{0.892768in}{1.091047in}}%
\pgfpathlineto{\pgfqpoint{0.886389in}{1.095688in}}%
\pgfpathlineto{\pgfqpoint{0.880316in}{1.100427in}}%
\pgfpathlineto{\pgfqpoint{0.883507in}{1.109220in}}%
\pgfpathlineto{\pgfqpoint{0.886700in}{1.118021in}}%
\pgfpathlineto{\pgfqpoint{0.889894in}{1.126826in}}%
\pgfpathlineto{\pgfqpoint{0.893089in}{1.135633in}}%
\pgfpathlineto{\pgfqpoint{0.898912in}{1.131120in}}%
\pgfpathlineto{\pgfqpoint{0.905027in}{1.126699in}}%
\pgfpathlineto{\pgfqpoint{0.911427in}{1.122374in}}%
\pgfpathlineto{\pgfqpoint{0.918107in}{1.118151in}}%
\pgfpathclose%
\pgfusepath{fill}%
\end{pgfscope}%
\begin{pgfscope}%
\pgfpathrectangle{\pgfqpoint{0.041670in}{0.041670in}}{\pgfqpoint{2.216660in}{2.216660in}}%
\pgfusepath{clip}%
\pgfsetbuttcap%
\pgfsetroundjoin%
\definecolor{currentfill}{rgb}{0.935904,0.898570,0.108131}%
\pgfsetfillcolor{currentfill}%
\pgfsetlinewidth{0.000000pt}%
\definecolor{currentstroke}{rgb}{0.000000,0.000000,0.000000}%
\pgfsetstrokecolor{currentstroke}%
\pgfsetdash{}{0pt}%
\pgfpathmoveto{\pgfqpoint{1.172224in}{1.665607in}}%
\pgfpathlineto{\pgfqpoint{1.171744in}{1.665231in}}%
\pgfpathlineto{\pgfqpoint{1.171265in}{1.664727in}}%
\pgfpathlineto{\pgfqpoint{1.170785in}{1.664096in}}%
\pgfpathlineto{\pgfqpoint{1.170307in}{1.663338in}}%
\pgfpathlineto{\pgfqpoint{1.172745in}{1.663462in}}%
\pgfpathlineto{\pgfqpoint{1.175190in}{1.663550in}}%
\pgfpathlineto{\pgfqpoint{1.177639in}{1.663602in}}%
\pgfpathlineto{\pgfqpoint{1.180091in}{1.663618in}}%
\pgfpathlineto{\pgfqpoint{1.180084in}{1.664361in}}%
\pgfpathlineto{\pgfqpoint{1.180078in}{1.664979in}}%
\pgfpathlineto{\pgfqpoint{1.180071in}{1.665469in}}%
\pgfpathlineto{\pgfqpoint{1.180064in}{1.665831in}}%
\pgfpathlineto{\pgfqpoint{1.178099in}{1.665818in}}%
\pgfpathlineto{\pgfqpoint{1.176137in}{1.665776in}}%
\pgfpathlineto{\pgfqpoint{1.174178in}{1.665706in}}%
\pgfpathlineto{\pgfqpoint{1.172224in}{1.665607in}}%
\pgfpathclose%
\pgfusepath{fill}%
\end{pgfscope}%
\begin{pgfscope}%
\pgfpathrectangle{\pgfqpoint{0.041670in}{0.041670in}}{\pgfqpoint{2.216660in}{2.216660in}}%
\pgfusepath{clip}%
\pgfsetbuttcap%
\pgfsetroundjoin%
\definecolor{currentfill}{rgb}{0.814576,0.883393,0.110347}%
\pgfsetfillcolor{currentfill}%
\pgfsetlinewidth{0.000000pt}%
\definecolor{currentstroke}{rgb}{0.000000,0.000000,0.000000}%
\pgfsetstrokecolor{currentstroke}%
\pgfsetdash{}{0pt}%
\pgfpathmoveto{\pgfqpoint{1.277285in}{1.625215in}}%
\pgfpathlineto{\pgfqpoint{1.280731in}{1.622375in}}%
\pgfpathlineto{\pgfqpoint{1.284175in}{1.619418in}}%
\pgfpathlineto{\pgfqpoint{1.287615in}{1.616345in}}%
\pgfpathlineto{\pgfqpoint{1.291053in}{1.613157in}}%
\pgfpathlineto{\pgfqpoint{1.292782in}{1.611499in}}%
\pgfpathlineto{\pgfqpoint{1.294399in}{1.609817in}}%
\pgfpathlineto{\pgfqpoint{1.295903in}{1.608111in}}%
\pgfpathlineto{\pgfqpoint{1.297293in}{1.606383in}}%
\pgfpathlineto{\pgfqpoint{1.293659in}{1.609784in}}%
\pgfpathlineto{\pgfqpoint{1.290021in}{1.613070in}}%
\pgfpathlineto{\pgfqpoint{1.286381in}{1.616240in}}%
\pgfpathlineto{\pgfqpoint{1.282739in}{1.619293in}}%
\pgfpathlineto{\pgfqpoint{1.281524in}{1.620804in}}%
\pgfpathlineto{\pgfqpoint{1.280210in}{1.622295in}}%
\pgfpathlineto{\pgfqpoint{1.278796in}{1.623766in}}%
\pgfpathlineto{\pgfqpoint{1.277285in}{1.625215in}}%
\pgfpathclose%
\pgfusepath{fill}%
\end{pgfscope}%
\begin{pgfscope}%
\pgfpathrectangle{\pgfqpoint{0.041670in}{0.041670in}}{\pgfqpoint{2.216660in}{2.216660in}}%
\pgfusepath{clip}%
\pgfsetbuttcap%
\pgfsetroundjoin%
\definecolor{currentfill}{rgb}{0.935904,0.898570,0.108131}%
\pgfsetfillcolor{currentfill}%
\pgfsetlinewidth{0.000000pt}%
\definecolor{currentstroke}{rgb}{0.000000,0.000000,0.000000}%
\pgfsetstrokecolor{currentstroke}%
\pgfsetdash{}{0pt}%
\pgfpathmoveto{\pgfqpoint{1.180064in}{1.665831in}}%
\pgfpathlineto{\pgfqpoint{1.180071in}{1.665469in}}%
\pgfpathlineto{\pgfqpoint{1.180078in}{1.664979in}}%
\pgfpathlineto{\pgfqpoint{1.180084in}{1.664361in}}%
\pgfpathlineto{\pgfqpoint{1.180091in}{1.663618in}}%
\pgfpathlineto{\pgfqpoint{1.182543in}{1.663598in}}%
\pgfpathlineto{\pgfqpoint{1.184992in}{1.663542in}}%
\pgfpathlineto{\pgfqpoint{1.187436in}{1.663450in}}%
\pgfpathlineto{\pgfqpoint{1.189873in}{1.663322in}}%
\pgfpathlineto{\pgfqpoint{1.189381in}{1.664080in}}%
\pgfpathlineto{\pgfqpoint{1.188889in}{1.664712in}}%
\pgfpathlineto{\pgfqpoint{1.188396in}{1.665217in}}%
\pgfpathlineto{\pgfqpoint{1.187902in}{1.665594in}}%
\pgfpathlineto{\pgfqpoint{1.185950in}{1.665696in}}%
\pgfpathlineto{\pgfqpoint{1.183991in}{1.665770in}}%
\pgfpathlineto{\pgfqpoint{1.182029in}{1.665815in}}%
\pgfpathlineto{\pgfqpoint{1.180064in}{1.665831in}}%
\pgfpathclose%
\pgfusepath{fill}%
\end{pgfscope}%
\begin{pgfscope}%
\pgfpathrectangle{\pgfqpoint{0.041670in}{0.041670in}}{\pgfqpoint{2.216660in}{2.216660in}}%
\pgfusepath{clip}%
\pgfsetbuttcap%
\pgfsetroundjoin%
\definecolor{currentfill}{rgb}{0.283072,0.130895,0.449241}%
\pgfsetfillcolor{currentfill}%
\pgfsetlinewidth{0.000000pt}%
\definecolor{currentstroke}{rgb}{0.000000,0.000000,0.000000}%
\pgfsetstrokecolor{currentstroke}%
\pgfsetdash{}{0pt}%
\pgfpathmoveto{\pgfqpoint{0.891516in}{0.743994in}}%
\pgfpathlineto{\pgfqpoint{0.889270in}{0.737122in}}%
\pgfpathlineto{\pgfqpoint{0.887022in}{0.730386in}}%
\pgfpathlineto{\pgfqpoint{0.884772in}{0.723790in}}%
\pgfpathlineto{\pgfqpoint{0.882521in}{0.717339in}}%
\pgfpathlineto{\pgfqpoint{0.869660in}{0.722455in}}%
\pgfpathlineto{\pgfqpoint{0.857137in}{0.727782in}}%
\pgfpathlineto{\pgfqpoint{0.844962in}{0.733312in}}%
\pgfpathlineto{\pgfqpoint{0.833149in}{0.739039in}}%
\pgfpathlineto{\pgfqpoint{0.835767in}{0.745309in}}%
\pgfpathlineto{\pgfqpoint{0.838383in}{0.751722in}}%
\pgfpathlineto{\pgfqpoint{0.840996in}{0.758275in}}%
\pgfpathlineto{\pgfqpoint{0.843608in}{0.764964in}}%
\pgfpathlineto{\pgfqpoint{0.855072in}{0.759429in}}%
\pgfpathlineto{\pgfqpoint{0.866886in}{0.754085in}}%
\pgfpathlineto{\pgfqpoint{0.879038in}{0.748938in}}%
\pgfpathlineto{\pgfqpoint{0.891516in}{0.743994in}}%
\pgfpathclose%
\pgfusepath{fill}%
\end{pgfscope}%
\begin{pgfscope}%
\pgfpathrectangle{\pgfqpoint{0.041670in}{0.041670in}}{\pgfqpoint{2.216660in}{2.216660in}}%
\pgfusepath{clip}%
\pgfsetbuttcap%
\pgfsetroundjoin%
\definecolor{currentfill}{rgb}{0.279566,0.067836,0.391917}%
\pgfsetfillcolor{currentfill}%
\pgfsetlinewidth{0.000000pt}%
\definecolor{currentstroke}{rgb}{0.000000,0.000000,0.000000}%
\pgfsetstrokecolor{currentstroke}%
\pgfsetdash{}{0pt}%
\pgfpathmoveto{\pgfqpoint{0.873491in}{0.693050in}}%
\pgfpathlineto{\pgfqpoint{0.871227in}{0.687379in}}%
\pgfpathlineto{\pgfqpoint{0.868961in}{0.681875in}}%
\pgfpathlineto{\pgfqpoint{0.866691in}{0.676545in}}%
\pgfpathlineto{\pgfqpoint{0.864419in}{0.671391in}}%
\pgfpathlineto{\pgfqpoint{0.850790in}{0.676852in}}%
\pgfpathlineto{\pgfqpoint{0.837520in}{0.682538in}}%
\pgfpathlineto{\pgfqpoint{0.824622in}{0.688440in}}%
\pgfpathlineto{\pgfqpoint{0.812109in}{0.694552in}}%
\pgfpathlineto{\pgfqpoint{0.814750in}{0.699523in}}%
\pgfpathlineto{\pgfqpoint{0.817387in}{0.704671in}}%
\pgfpathlineto{\pgfqpoint{0.820021in}{0.709992in}}%
\pgfpathlineto{\pgfqpoint{0.822652in}{0.715481in}}%
\pgfpathlineto{\pgfqpoint{0.834815in}{0.709562in}}%
\pgfpathlineto{\pgfqpoint{0.847350in}{0.703845in}}%
\pgfpathlineto{\pgfqpoint{0.860247in}{0.698339in}}%
\pgfpathlineto{\pgfqpoint{0.873491in}{0.693050in}}%
\pgfpathclose%
\pgfusepath{fill}%
\end{pgfscope}%
\begin{pgfscope}%
\pgfpathrectangle{\pgfqpoint{0.041670in}{0.041670in}}{\pgfqpoint{2.216660in}{2.216660in}}%
\pgfusepath{clip}%
\pgfsetbuttcap%
\pgfsetroundjoin%
\definecolor{currentfill}{rgb}{0.935904,0.898570,0.108131}%
\pgfsetfillcolor{currentfill}%
\pgfsetlinewidth{0.000000pt}%
\definecolor{currentstroke}{rgb}{0.000000,0.000000,0.000000}%
\pgfsetstrokecolor{currentstroke}%
\pgfsetdash{}{0pt}%
\pgfpathmoveto{\pgfqpoint{1.164506in}{1.664925in}}%
\pgfpathlineto{\pgfqpoint{1.163546in}{1.664507in}}%
\pgfpathlineto{\pgfqpoint{1.162588in}{1.663961in}}%
\pgfpathlineto{\pgfqpoint{1.161630in}{1.663287in}}%
\pgfpathlineto{\pgfqpoint{1.160673in}{1.662487in}}%
\pgfpathlineto{\pgfqpoint{1.163059in}{1.662753in}}%
\pgfpathlineto{\pgfqpoint{1.165462in}{1.662984in}}%
\pgfpathlineto{\pgfqpoint{1.167878in}{1.663179in}}%
\pgfpathlineto{\pgfqpoint{1.170307in}{1.663338in}}%
\pgfpathlineto{\pgfqpoint{1.170785in}{1.664096in}}%
\pgfpathlineto{\pgfqpoint{1.171265in}{1.664727in}}%
\pgfpathlineto{\pgfqpoint{1.171744in}{1.665231in}}%
\pgfpathlineto{\pgfqpoint{1.172224in}{1.665607in}}%
\pgfpathlineto{\pgfqpoint{1.170278in}{1.665479in}}%
\pgfpathlineto{\pgfqpoint{1.168342in}{1.665323in}}%
\pgfpathlineto{\pgfqpoint{1.166417in}{1.665138in}}%
\pgfpathlineto{\pgfqpoint{1.164506in}{1.664925in}}%
\pgfpathclose%
\pgfusepath{fill}%
\end{pgfscope}%
\begin{pgfscope}%
\pgfpathrectangle{\pgfqpoint{0.041670in}{0.041670in}}{\pgfqpoint{2.216660in}{2.216660in}}%
\pgfusepath{clip}%
\pgfsetbuttcap%
\pgfsetroundjoin%
\definecolor{currentfill}{rgb}{0.935904,0.898570,0.108131}%
\pgfsetfillcolor{currentfill}%
\pgfsetlinewidth{0.000000pt}%
\definecolor{currentstroke}{rgb}{0.000000,0.000000,0.000000}%
\pgfsetstrokecolor{currentstroke}%
\pgfsetdash{}{0pt}%
\pgfpathmoveto{\pgfqpoint{1.187902in}{1.665594in}}%
\pgfpathlineto{\pgfqpoint{1.188396in}{1.665217in}}%
\pgfpathlineto{\pgfqpoint{1.188889in}{1.664712in}}%
\pgfpathlineto{\pgfqpoint{1.189381in}{1.664080in}}%
\pgfpathlineto{\pgfqpoint{1.189873in}{1.663322in}}%
\pgfpathlineto{\pgfqpoint{1.192301in}{1.663159in}}%
\pgfpathlineto{\pgfqpoint{1.194716in}{1.662960in}}%
\pgfpathlineto{\pgfqpoint{1.197117in}{1.662725in}}%
\pgfpathlineto{\pgfqpoint{1.199500in}{1.662456in}}%
\pgfpathlineto{\pgfqpoint{1.198530in}{1.663257in}}%
\pgfpathlineto{\pgfqpoint{1.197560in}{1.663932in}}%
\pgfpathlineto{\pgfqpoint{1.196588in}{1.664480in}}%
\pgfpathlineto{\pgfqpoint{1.195615in}{1.664900in}}%
\pgfpathlineto{\pgfqpoint{1.193705in}{1.665116in}}%
\pgfpathlineto{\pgfqpoint{1.191782in}{1.665304in}}%
\pgfpathlineto{\pgfqpoint{1.189847in}{1.665463in}}%
\pgfpathlineto{\pgfqpoint{1.187902in}{1.665594in}}%
\pgfpathclose%
\pgfusepath{fill}%
\end{pgfscope}%
\begin{pgfscope}%
\pgfpathrectangle{\pgfqpoint{0.041670in}{0.041670in}}{\pgfqpoint{2.216660in}{2.216660in}}%
\pgfusepath{clip}%
\pgfsetbuttcap%
\pgfsetroundjoin%
\definecolor{currentfill}{rgb}{0.699415,0.867117,0.175971}%
\pgfsetfillcolor{currentfill}%
\pgfsetlinewidth{0.000000pt}%
\definecolor{currentstroke}{rgb}{0.000000,0.000000,0.000000}%
\pgfsetstrokecolor{currentstroke}%
\pgfsetdash{}{0pt}%
\pgfpathmoveto{\pgfqpoint{1.316754in}{1.583661in}}%
\pgfpathlineto{\pgfqpoint{1.320515in}{1.579479in}}%
\pgfpathlineto{\pgfqpoint{1.324273in}{1.575190in}}%
\pgfpathlineto{\pgfqpoint{1.328029in}{1.570794in}}%
\pgfpathlineto{\pgfqpoint{1.331782in}{1.566294in}}%
\pgfpathlineto{\pgfqpoint{1.332788in}{1.564023in}}%
\pgfpathlineto{\pgfqpoint{1.333642in}{1.561737in}}%
\pgfpathlineto{\pgfqpoint{1.334344in}{1.559439in}}%
\pgfpathlineto{\pgfqpoint{1.334893in}{1.557131in}}%
\pgfpathlineto{\pgfqpoint{1.331058in}{1.561864in}}%
\pgfpathlineto{\pgfqpoint{1.327221in}{1.566492in}}%
\pgfpathlineto{\pgfqpoint{1.323380in}{1.571014in}}%
\pgfpathlineto{\pgfqpoint{1.319538in}{1.575428in}}%
\pgfpathlineto{\pgfqpoint{1.319048in}{1.577502in}}%
\pgfpathlineto{\pgfqpoint{1.318420in}{1.579567in}}%
\pgfpathlineto{\pgfqpoint{1.317655in}{1.581621in}}%
\pgfpathlineto{\pgfqpoint{1.316754in}{1.583661in}}%
\pgfpathclose%
\pgfusepath{fill}%
\end{pgfscope}%
\begin{pgfscope}%
\pgfpathrectangle{\pgfqpoint{0.041670in}{0.041670in}}{\pgfqpoint{2.216660in}{2.216660in}}%
\pgfusepath{clip}%
\pgfsetbuttcap%
\pgfsetroundjoin%
\definecolor{currentfill}{rgb}{0.487026,0.823929,0.312321}%
\pgfsetfillcolor{currentfill}%
\pgfsetlinewidth{0.000000pt}%
\definecolor{currentstroke}{rgb}{0.000000,0.000000,0.000000}%
\pgfsetstrokecolor{currentstroke}%
\pgfsetdash{}{0pt}%
\pgfpathmoveto{\pgfqpoint{1.366301in}{1.504470in}}%
\pgfpathlineto{\pgfqpoint{1.370134in}{1.498621in}}%
\pgfpathlineto{\pgfqpoint{1.373963in}{1.492683in}}%
\pgfpathlineto{\pgfqpoint{1.377790in}{1.486657in}}%
\pgfpathlineto{\pgfqpoint{1.381613in}{1.480545in}}%
\pgfpathlineto{\pgfqpoint{1.381343in}{1.477501in}}%
\pgfpathlineto{\pgfqpoint{1.380871in}{1.474461in}}%
\pgfpathlineto{\pgfqpoint{1.380198in}{1.471428in}}%
\pgfpathlineto{\pgfqpoint{1.379324in}{1.468406in}}%
\pgfpathlineto{\pgfqpoint{1.375537in}{1.474759in}}%
\pgfpathlineto{\pgfqpoint{1.371748in}{1.481027in}}%
\pgfpathlineto{\pgfqpoint{1.367955in}{1.487206in}}%
\pgfpathlineto{\pgfqpoint{1.364161in}{1.493295in}}%
\pgfpathlineto{\pgfqpoint{1.364974in}{1.496077in}}%
\pgfpathlineto{\pgfqpoint{1.365602in}{1.498869in}}%
\pgfpathlineto{\pgfqpoint{1.366045in}{1.501668in}}%
\pgfpathlineto{\pgfqpoint{1.366301in}{1.504470in}}%
\pgfpathclose%
\pgfusepath{fill}%
\end{pgfscope}%
\begin{pgfscope}%
\pgfpathrectangle{\pgfqpoint{0.041670in}{0.041670in}}{\pgfqpoint{2.216660in}{2.216660in}}%
\pgfusepath{clip}%
\pgfsetbuttcap%
\pgfsetroundjoin%
\definecolor{currentfill}{rgb}{0.120081,0.622161,0.534946}%
\pgfsetfillcolor{currentfill}%
\pgfsetlinewidth{0.000000pt}%
\definecolor{currentstroke}{rgb}{0.000000,0.000000,0.000000}%
\pgfsetstrokecolor{currentstroke}%
\pgfsetdash{}{0pt}%
\pgfpathmoveto{\pgfqpoint{0.931561in}{1.240442in}}%
\pgfpathlineto{\pgfqpoint{0.928346in}{1.231829in}}%
\pgfpathlineto{\pgfqpoint{0.925132in}{1.223185in}}%
\pgfpathlineto{\pgfqpoint{0.921920in}{1.214514in}}%
\pgfpathlineto{\pgfqpoint{0.918710in}{1.205816in}}%
\pgfpathlineto{\pgfqpoint{0.913662in}{1.209964in}}%
\pgfpathlineto{\pgfqpoint{0.908889in}{1.214186in}}%
\pgfpathlineto{\pgfqpoint{0.904395in}{1.218478in}}%
\pgfpathlineto{\pgfqpoint{0.900184in}{1.222836in}}%
\pgfpathlineto{\pgfqpoint{0.903613in}{1.231304in}}%
\pgfpathlineto{\pgfqpoint{0.907044in}{1.239747in}}%
\pgfpathlineto{\pgfqpoint{0.910478in}{1.248161in}}%
\pgfpathlineto{\pgfqpoint{0.913913in}{1.256546in}}%
\pgfpathlineto{\pgfqpoint{0.917926in}{1.252422in}}%
\pgfpathlineto{\pgfqpoint{0.922208in}{1.248360in}}%
\pgfpathlineto{\pgfqpoint{0.926754in}{1.244366in}}%
\pgfpathlineto{\pgfqpoint{0.931561in}{1.240442in}}%
\pgfpathclose%
\pgfusepath{fill}%
\end{pgfscope}%
\begin{pgfscope}%
\pgfpathrectangle{\pgfqpoint{0.041670in}{0.041670in}}{\pgfqpoint{2.216660in}{2.216660in}}%
\pgfusepath{clip}%
\pgfsetbuttcap%
\pgfsetroundjoin%
\definecolor{currentfill}{rgb}{0.220124,0.725509,0.466226}%
\pgfsetfillcolor{currentfill}%
\pgfsetlinewidth{0.000000pt}%
\definecolor{currentstroke}{rgb}{0.000000,0.000000,0.000000}%
\pgfsetstrokecolor{currentstroke}%
\pgfsetdash{}{0pt}%
\pgfpathmoveto{\pgfqpoint{1.417920in}{1.371311in}}%
\pgfpathlineto{\pgfqpoint{1.421578in}{1.363769in}}%
\pgfpathlineto{\pgfqpoint{1.425232in}{1.356169in}}%
\pgfpathlineto{\pgfqpoint{1.428884in}{1.348513in}}%
\pgfpathlineto{\pgfqpoint{1.432533in}{1.340802in}}%
\pgfpathlineto{\pgfqpoint{1.430139in}{1.336925in}}%
\pgfpathlineto{\pgfqpoint{1.427490in}{1.333084in}}%
\pgfpathlineto{\pgfqpoint{1.424590in}{1.329283in}}%
\pgfpathlineto{\pgfqpoint{1.421440in}{1.325528in}}%
\pgfpathlineto{\pgfqpoint{1.417943in}{1.333476in}}%
\pgfpathlineto{\pgfqpoint{1.414444in}{1.341371in}}%
\pgfpathlineto{\pgfqpoint{1.410943in}{1.349208in}}%
\pgfpathlineto{\pgfqpoint{1.407439in}{1.356987in}}%
\pgfpathlineto{\pgfqpoint{1.410413in}{1.360509in}}%
\pgfpathlineto{\pgfqpoint{1.413153in}{1.364072in}}%
\pgfpathlineto{\pgfqpoint{1.415656in}{1.367675in}}%
\pgfpathlineto{\pgfqpoint{1.417920in}{1.371311in}}%
\pgfpathclose%
\pgfusepath{fill}%
\end{pgfscope}%
\begin{pgfscope}%
\pgfpathrectangle{\pgfqpoint{0.041670in}{0.041670in}}{\pgfqpoint{2.216660in}{2.216660in}}%
\pgfusepath{clip}%
\pgfsetbuttcap%
\pgfsetroundjoin%
\definecolor{currentfill}{rgb}{0.280255,0.165693,0.476498}%
\pgfsetfillcolor{currentfill}%
\pgfsetlinewidth{0.000000pt}%
\definecolor{currentstroke}{rgb}{0.000000,0.000000,0.000000}%
\pgfsetstrokecolor{currentstroke}%
\pgfsetdash{}{0pt}%
\pgfpathmoveto{\pgfqpoint{0.900483in}{0.772762in}}%
\pgfpathlineto{\pgfqpoint{0.898243in}{0.765385in}}%
\pgfpathlineto{\pgfqpoint{0.896002in}{0.758129in}}%
\pgfpathlineto{\pgfqpoint{0.893760in}{0.750997in}}%
\pgfpathlineto{\pgfqpoint{0.891516in}{0.743994in}}%
\pgfpathlineto{\pgfqpoint{0.879038in}{0.748938in}}%
\pgfpathlineto{\pgfqpoint{0.866886in}{0.754085in}}%
\pgfpathlineto{\pgfqpoint{0.855072in}{0.759429in}}%
\pgfpathlineto{\pgfqpoint{0.843608in}{0.764964in}}%
\pgfpathlineto{\pgfqpoint{0.846217in}{0.771785in}}%
\pgfpathlineto{\pgfqpoint{0.848825in}{0.778735in}}%
\pgfpathlineto{\pgfqpoint{0.851431in}{0.785808in}}%
\pgfpathlineto{\pgfqpoint{0.854035in}{0.793003in}}%
\pgfpathlineto{\pgfqpoint{0.865152in}{0.787660in}}%
\pgfpathlineto{\pgfqpoint{0.876606in}{0.782502in}}%
\pgfpathlineto{\pgfqpoint{0.888387in}{0.777534in}}%
\pgfpathlineto{\pgfqpoint{0.900483in}{0.772762in}}%
\pgfpathclose%
\pgfusepath{fill}%
\end{pgfscope}%
\begin{pgfscope}%
\pgfpathrectangle{\pgfqpoint{0.041670in}{0.041670in}}{\pgfqpoint{2.216660in}{2.216660in}}%
\pgfusepath{clip}%
\pgfsetbuttcap%
\pgfsetroundjoin%
\definecolor{currentfill}{rgb}{0.263663,0.237631,0.518762}%
\pgfsetfillcolor{currentfill}%
\pgfsetlinewidth{0.000000pt}%
\definecolor{currentstroke}{rgb}{0.000000,0.000000,0.000000}%
\pgfsetstrokecolor{currentstroke}%
\pgfsetdash{}{0pt}%
\pgfpathmoveto{\pgfqpoint{1.494072in}{0.858997in}}%
\pgfpathlineto{\pgfqpoint{1.496743in}{0.851016in}}%
\pgfpathlineto{\pgfqpoint{1.499414in}{0.843126in}}%
\pgfpathlineto{\pgfqpoint{1.502086in}{0.835331in}}%
\pgfpathlineto{\pgfqpoint{1.504759in}{0.827634in}}%
\pgfpathlineto{\pgfqpoint{1.494290in}{0.822329in}}%
\pgfpathlineto{\pgfqpoint{1.483485in}{0.817198in}}%
\pgfpathlineto{\pgfqpoint{1.472354in}{0.812245in}}%
\pgfpathlineto{\pgfqpoint{1.460909in}{0.807477in}}%
\pgfpathlineto{\pgfqpoint{1.458590in}{0.815361in}}%
\pgfpathlineto{\pgfqpoint{1.456273in}{0.823344in}}%
\pgfpathlineto{\pgfqpoint{1.453956in}{0.831421in}}%
\pgfpathlineto{\pgfqpoint{1.451640in}{0.839590in}}%
\pgfpathlineto{\pgfqpoint{1.462714in}{0.844180in}}%
\pgfpathlineto{\pgfqpoint{1.473484in}{0.848949in}}%
\pgfpathlineto{\pgfqpoint{1.483940in}{0.853890in}}%
\pgfpathlineto{\pgfqpoint{1.494072in}{0.858997in}}%
\pgfpathclose%
\pgfusepath{fill}%
\end{pgfscope}%
\begin{pgfscope}%
\pgfpathrectangle{\pgfqpoint{0.041670in}{0.041670in}}{\pgfqpoint{2.216660in}{2.216660in}}%
\pgfusepath{clip}%
\pgfsetbuttcap%
\pgfsetroundjoin%
\definecolor{currentfill}{rgb}{0.762373,0.876424,0.137064}%
\pgfsetfillcolor{currentfill}%
\pgfsetlinewidth{0.000000pt}%
\definecolor{currentstroke}{rgb}{0.000000,0.000000,0.000000}%
\pgfsetstrokecolor{currentstroke}%
\pgfsetdash{}{0pt}%
\pgfpathmoveto{\pgfqpoint{1.057513in}{1.597677in}}%
\pgfpathlineto{\pgfqpoint{1.053718in}{1.593888in}}%
\pgfpathlineto{\pgfqpoint{1.049925in}{1.589986in}}%
\pgfpathlineto{\pgfqpoint{1.046135in}{1.585972in}}%
\pgfpathlineto{\pgfqpoint{1.042348in}{1.581848in}}%
\pgfpathlineto{\pgfqpoint{1.043265in}{1.583887in}}%
\pgfpathlineto{\pgfqpoint{1.044317in}{1.585911in}}%
\pgfpathlineto{\pgfqpoint{1.045503in}{1.587918in}}%
\pgfpathlineto{\pgfqpoint{1.046822in}{1.589905in}}%
\pgfpathlineto{\pgfqpoint{1.050482in}{1.593803in}}%
\pgfpathlineto{\pgfqpoint{1.054145in}{1.597590in}}%
\pgfpathlineto{\pgfqpoint{1.057810in}{1.601267in}}%
\pgfpathlineto{\pgfqpoint{1.061478in}{1.604830in}}%
\pgfpathlineto{\pgfqpoint{1.060308in}{1.603066in}}%
\pgfpathlineto{\pgfqpoint{1.059257in}{1.601284in}}%
\pgfpathlineto{\pgfqpoint{1.058325in}{1.599488in}}%
\pgfpathlineto{\pgfqpoint{1.057513in}{1.597677in}}%
\pgfpathclose%
\pgfusepath{fill}%
\end{pgfscope}%
\begin{pgfscope}%
\pgfpathrectangle{\pgfqpoint{0.041670in}{0.041670in}}{\pgfqpoint{2.216660in}{2.216660in}}%
\pgfusepath{clip}%
\pgfsetbuttcap%
\pgfsetroundjoin%
\definecolor{currentfill}{rgb}{0.896320,0.893616,0.096335}%
\pgfsetfillcolor{currentfill}%
\pgfsetlinewidth{0.000000pt}%
\definecolor{currentstroke}{rgb}{0.000000,0.000000,0.000000}%
\pgfsetstrokecolor{currentstroke}%
\pgfsetdash{}{0pt}%
\pgfpathmoveto{\pgfqpoint{1.231704in}{1.654450in}}%
\pgfpathlineto{\pgfqpoint{1.234273in}{1.653125in}}%
\pgfpathlineto{\pgfqpoint{1.236840in}{1.651674in}}%
\pgfpathlineto{\pgfqpoint{1.239405in}{1.650100in}}%
\pgfpathlineto{\pgfqpoint{1.241968in}{1.648403in}}%
\pgfpathlineto{\pgfqpoint{1.244143in}{1.647475in}}%
\pgfpathlineto{\pgfqpoint{1.246255in}{1.646514in}}%
\pgfpathlineto{\pgfqpoint{1.248302in}{1.645523in}}%
\pgfpathlineto{\pgfqpoint{1.250281in}{1.644502in}}%
\pgfpathlineto{\pgfqpoint{1.247373in}{1.646361in}}%
\pgfpathlineto{\pgfqpoint{1.244463in}{1.648097in}}%
\pgfpathlineto{\pgfqpoint{1.241550in}{1.649710in}}%
\pgfpathlineto{\pgfqpoint{1.238635in}{1.651198in}}%
\pgfpathlineto{\pgfqpoint{1.236985in}{1.652049in}}%
\pgfpathlineto{\pgfqpoint{1.235279in}{1.652875in}}%
\pgfpathlineto{\pgfqpoint{1.233518in}{1.653676in}}%
\pgfpathlineto{\pgfqpoint{1.231704in}{1.654450in}}%
\pgfpathclose%
\pgfusepath{fill}%
\end{pgfscope}%
\begin{pgfscope}%
\pgfpathrectangle{\pgfqpoint{0.041670in}{0.041670in}}{\pgfqpoint{2.216660in}{2.216660in}}%
\pgfusepath{clip}%
\pgfsetbuttcap%
\pgfsetroundjoin%
\definecolor{currentfill}{rgb}{0.935904,0.898570,0.108131}%
\pgfsetfillcolor{currentfill}%
\pgfsetlinewidth{0.000000pt}%
\definecolor{currentstroke}{rgb}{0.000000,0.000000,0.000000}%
\pgfsetstrokecolor{currentstroke}%
\pgfsetdash{}{0pt}%
\pgfpathmoveto{\pgfqpoint{1.195615in}{1.664900in}}%
\pgfpathlineto{\pgfqpoint{1.196588in}{1.664480in}}%
\pgfpathlineto{\pgfqpoint{1.197560in}{1.663932in}}%
\pgfpathlineto{\pgfqpoint{1.198530in}{1.663257in}}%
\pgfpathlineto{\pgfqpoint{1.199500in}{1.662456in}}%
\pgfpathlineto{\pgfqpoint{1.201865in}{1.662151in}}%
\pgfpathlineto{\pgfqpoint{1.204208in}{1.661812in}}%
\pgfpathlineto{\pgfqpoint{1.206528in}{1.661439in}}%
\pgfpathlineto{\pgfqpoint{1.205209in}{1.662291in}}%
\pgfpathlineto{\pgfqpoint{1.203889in}{1.663017in}}%
\pgfpathlineto{\pgfqpoint{1.202567in}{1.663615in}}%
\pgfpathlineto{\pgfqpoint{1.201245in}{1.664086in}}%
\pgfpathlineto{\pgfqpoint{1.199387in}{1.664385in}}%
\pgfpathlineto{\pgfqpoint{1.197510in}{1.664656in}}%
\pgfpathlineto{\pgfqpoint{1.195615in}{1.664900in}}%
\pgfpathclose%
\pgfusepath{fill}%
\end{pgfscope}%
\begin{pgfscope}%
\pgfpathrectangle{\pgfqpoint{0.041670in}{0.041670in}}{\pgfqpoint{2.216660in}{2.216660in}}%
\pgfusepath{clip}%
\pgfsetbuttcap%
\pgfsetroundjoin%
\definecolor{currentfill}{rgb}{0.814576,0.883393,0.110347}%
\pgfsetfillcolor{currentfill}%
\pgfsetlinewidth{0.000000pt}%
\definecolor{currentstroke}{rgb}{0.000000,0.000000,0.000000}%
\pgfsetstrokecolor{currentstroke}%
\pgfsetdash{}{0pt}%
\pgfpathmoveto{\pgfqpoint{1.076177in}{1.617936in}}%
\pgfpathlineto{\pgfqpoint{1.072499in}{1.614834in}}%
\pgfpathlineto{\pgfqpoint{1.068823in}{1.611615in}}%
\pgfpathlineto{\pgfqpoint{1.065149in}{1.608280in}}%
\pgfpathlineto{\pgfqpoint{1.061478in}{1.604830in}}%
\pgfpathlineto{\pgfqpoint{1.062765in}{1.606576in}}%
\pgfpathlineto{\pgfqpoint{1.064168in}{1.608302in}}%
\pgfpathlineto{\pgfqpoint{1.065685in}{1.610005in}}%
\pgfpathlineto{\pgfqpoint{1.067315in}{1.611685in}}%
\pgfpathlineto{\pgfqpoint{1.070801in}{1.614919in}}%
\pgfpathlineto{\pgfqpoint{1.074290in}{1.618039in}}%
\pgfpathlineto{\pgfqpoint{1.077782in}{1.621042in}}%
\pgfpathlineto{\pgfqpoint{1.081277in}{1.623928in}}%
\pgfpathlineto{\pgfqpoint{1.079852in}{1.622460in}}%
\pgfpathlineto{\pgfqpoint{1.078526in}{1.620970in}}%
\pgfpathlineto{\pgfqpoint{1.077301in}{1.619462in}}%
\pgfpathlineto{\pgfqpoint{1.076177in}{1.617936in}}%
\pgfpathclose%
\pgfusepath{fill}%
\end{pgfscope}%
\begin{pgfscope}%
\pgfpathrectangle{\pgfqpoint{0.041670in}{0.041670in}}{\pgfqpoint{2.216660in}{2.216660in}}%
\pgfusepath{clip}%
\pgfsetbuttcap%
\pgfsetroundjoin%
\definecolor{currentfill}{rgb}{0.935904,0.898570,0.108131}%
\pgfsetfillcolor{currentfill}%
\pgfsetlinewidth{0.000000pt}%
\definecolor{currentstroke}{rgb}{0.000000,0.000000,0.000000}%
\pgfsetstrokecolor{currentstroke}%
\pgfsetdash{}{0pt}%
\pgfpathmoveto{\pgfqpoint{1.157030in}{1.663797in}}%
\pgfpathlineto{\pgfqpoint{1.155607in}{1.663308in}}%
\pgfpathlineto{\pgfqpoint{1.154184in}{1.662692in}}%
\pgfpathlineto{\pgfqpoint{1.152762in}{1.661949in}}%
\pgfpathlineto{\pgfqpoint{1.151342in}{1.661079in}}%
\pgfpathlineto{\pgfqpoint{1.153638in}{1.661482in}}%
\pgfpathlineto{\pgfqpoint{1.155961in}{1.661852in}}%
\pgfpathlineto{\pgfqpoint{1.158306in}{1.662187in}}%
\pgfpathlineto{\pgfqpoint{1.160673in}{1.662487in}}%
\pgfpathlineto{\pgfqpoint{1.161630in}{1.663287in}}%
\pgfpathlineto{\pgfqpoint{1.162588in}{1.663961in}}%
\pgfpathlineto{\pgfqpoint{1.163546in}{1.664507in}}%
\pgfpathlineto{\pgfqpoint{1.164506in}{1.664925in}}%
\pgfpathlineto{\pgfqpoint{1.162610in}{1.664685in}}%
\pgfpathlineto{\pgfqpoint{1.160730in}{1.664416in}}%
\pgfpathlineto{\pgfqpoint{1.158870in}{1.664120in}}%
\pgfpathlineto{\pgfqpoint{1.157030in}{1.663797in}}%
\pgfpathclose%
\pgfusepath{fill}%
\end{pgfscope}%
\begin{pgfscope}%
\pgfpathrectangle{\pgfqpoint{0.041670in}{0.041670in}}{\pgfqpoint{2.216660in}{2.216660in}}%
\pgfusepath{clip}%
\pgfsetbuttcap%
\pgfsetroundjoin%
\definecolor{currentfill}{rgb}{0.274952,0.037752,0.364543}%
\pgfsetfillcolor{currentfill}%
\pgfsetlinewidth{0.000000pt}%
\definecolor{currentstroke}{rgb}{0.000000,0.000000,0.000000}%
\pgfsetstrokecolor{currentstroke}%
\pgfsetdash{}{0pt}%
\pgfpathmoveto{\pgfqpoint{0.864419in}{0.671391in}}%
\pgfpathlineto{\pgfqpoint{0.862144in}{0.666418in}}%
\pgfpathlineto{\pgfqpoint{0.859865in}{0.661631in}}%
\pgfpathlineto{\pgfqpoint{0.857583in}{0.657034in}}%
\pgfpathlineto{\pgfqpoint{0.855297in}{0.652632in}}%
\pgfpathlineto{\pgfqpoint{0.841281in}{0.658265in}}%
\pgfpathlineto{\pgfqpoint{0.827636in}{0.664129in}}%
\pgfpathlineto{\pgfqpoint{0.814373in}{0.670216in}}%
\pgfpathlineto{\pgfqpoint{0.801509in}{0.676520in}}%
\pgfpathlineto{\pgfqpoint{0.804165in}{0.680741in}}%
\pgfpathlineto{\pgfqpoint{0.806816in}{0.685156in}}%
\pgfpathlineto{\pgfqpoint{0.809464in}{0.689761in}}%
\pgfpathlineto{\pgfqpoint{0.812109in}{0.694552in}}%
\pgfpathlineto{\pgfqpoint{0.824622in}{0.688440in}}%
\pgfpathlineto{\pgfqpoint{0.837520in}{0.682538in}}%
\pgfpathlineto{\pgfqpoint{0.850790in}{0.676852in}}%
\pgfpathlineto{\pgfqpoint{0.864419in}{0.671391in}}%
\pgfpathclose%
\pgfusepath{fill}%
\end{pgfscope}%
\begin{pgfscope}%
\pgfpathrectangle{\pgfqpoint{0.041670in}{0.041670in}}{\pgfqpoint{2.216660in}{2.216660in}}%
\pgfusepath{clip}%
\pgfsetbuttcap%
\pgfsetroundjoin%
\definecolor{currentfill}{rgb}{0.272594,0.025563,0.353093}%
\pgfsetfillcolor{currentfill}%
\pgfsetlinewidth{0.000000pt}%
\definecolor{currentstroke}{rgb}{0.000000,0.000000,0.000000}%
\pgfsetstrokecolor{currentstroke}%
\pgfsetdash{}{0pt}%
\pgfpathmoveto{\pgfqpoint{1.677201in}{0.678685in}}%
\pgfpathlineto{\pgfqpoint{1.680382in}{0.679877in}}%
\pgfpathlineto{\pgfqpoint{1.683572in}{0.681370in}}%
\pgfpathlineto{\pgfqpoint{1.686772in}{0.683171in}}%
\pgfpathlineto{\pgfqpoint{1.689982in}{0.685285in}}%
\pgfpathlineto{\pgfqpoint{1.677472in}{0.676758in}}%
\pgfpathlineto{\pgfqpoint{1.664423in}{0.668437in}}%
\pgfpathlineto{\pgfqpoint{1.650846in}{0.660333in}}%
\pgfpathlineto{\pgfqpoint{1.636754in}{0.652456in}}%
\pgfpathlineto{\pgfqpoint{1.633867in}{0.650540in}}%
\pgfpathlineto{\pgfqpoint{1.630989in}{0.648938in}}%
\pgfpathlineto{\pgfqpoint{1.628119in}{0.647644in}}%
\pgfpathlineto{\pgfqpoint{1.625258in}{0.646653in}}%
\pgfpathlineto{\pgfqpoint{1.639007in}{0.654339in}}%
\pgfpathlineto{\pgfqpoint{1.652256in}{0.662245in}}%
\pgfpathlineto{\pgfqpoint{1.664991in}{0.670364in}}%
\pgfpathlineto{\pgfqpoint{1.677201in}{0.678685in}}%
\pgfpathclose%
\pgfusepath{fill}%
\end{pgfscope}%
\begin{pgfscope}%
\pgfpathrectangle{\pgfqpoint{0.041670in}{0.041670in}}{\pgfqpoint{2.216660in}{2.216660in}}%
\pgfusepath{clip}%
\pgfsetbuttcap%
\pgfsetroundjoin%
\definecolor{currentfill}{rgb}{0.133743,0.548535,0.553541}%
\pgfsetfillcolor{currentfill}%
\pgfsetlinewidth{0.000000pt}%
\definecolor{currentstroke}{rgb}{0.000000,0.000000,0.000000}%
\pgfsetstrokecolor{currentstroke}%
\pgfsetdash{}{0pt}%
\pgfpathmoveto{\pgfqpoint{1.458737in}{1.174706in}}%
\pgfpathlineto{\pgfqpoint{1.461992in}{1.165970in}}%
\pgfpathlineto{\pgfqpoint{1.465245in}{1.157226in}}%
\pgfpathlineto{\pgfqpoint{1.468497in}{1.148474in}}%
\pgfpathlineto{\pgfqpoint{1.471747in}{1.139719in}}%
\pgfpathlineto{\pgfqpoint{1.466188in}{1.135128in}}%
\pgfpathlineto{\pgfqpoint{1.460333in}{1.130624in}}%
\pgfpathlineto{\pgfqpoint{1.454186in}{1.126213in}}%
\pgfpathlineto{\pgfqpoint{1.447754in}{1.121900in}}%
\pgfpathlineto{\pgfqpoint{1.444764in}{1.130877in}}%
\pgfpathlineto{\pgfqpoint{1.441772in}{1.139849in}}%
\pgfpathlineto{\pgfqpoint{1.438779in}{1.148814in}}%
\pgfpathlineto{\pgfqpoint{1.435785in}{1.157769in}}%
\pgfpathlineto{\pgfqpoint{1.441936in}{1.161869in}}%
\pgfpathlineto{\pgfqpoint{1.447816in}{1.166061in}}%
\pgfpathlineto{\pgfqpoint{1.453418in}{1.170342in}}%
\pgfpathlineto{\pgfqpoint{1.458737in}{1.174706in}}%
\pgfpathclose%
\pgfusepath{fill}%
\end{pgfscope}%
\begin{pgfscope}%
\pgfpathrectangle{\pgfqpoint{0.041670in}{0.041670in}}{\pgfqpoint{2.216660in}{2.216660in}}%
\pgfusepath{clip}%
\pgfsetbuttcap%
\pgfsetroundjoin%
\definecolor{currentfill}{rgb}{0.855810,0.888601,0.097452}%
\pgfsetfillcolor{currentfill}%
\pgfsetlinewidth{0.000000pt}%
\definecolor{currentstroke}{rgb}{0.000000,0.000000,0.000000}%
\pgfsetstrokecolor{currentstroke}%
\pgfsetdash{}{0pt}%
\pgfpathmoveto{\pgfqpoint{1.257489in}{1.640139in}}%
\pgfpathlineto{\pgfqpoint{1.260695in}{1.637976in}}%
\pgfpathlineto{\pgfqpoint{1.263898in}{1.635692in}}%
\pgfpathlineto{\pgfqpoint{1.267099in}{1.633288in}}%
\pgfpathlineto{\pgfqpoint{1.270298in}{1.630764in}}%
\pgfpathlineto{\pgfqpoint{1.272183in}{1.629417in}}%
\pgfpathlineto{\pgfqpoint{1.273977in}{1.628042in}}%
\pgfpathlineto{\pgfqpoint{1.275678in}{1.626641in}}%
\pgfpathlineto{\pgfqpoint{1.277285in}{1.625215in}}%
\pgfpathlineto{\pgfqpoint{1.273837in}{1.627937in}}%
\pgfpathlineto{\pgfqpoint{1.270385in}{1.630540in}}%
\pgfpathlineto{\pgfqpoint{1.266932in}{1.633023in}}%
\pgfpathlineto{\pgfqpoint{1.263476in}{1.635385in}}%
\pgfpathlineto{\pgfqpoint{1.262099in}{1.636606in}}%
\pgfpathlineto{\pgfqpoint{1.260642in}{1.637807in}}%
\pgfpathlineto{\pgfqpoint{1.259104in}{1.638985in}}%
\pgfpathlineto{\pgfqpoint{1.257489in}{1.640139in}}%
\pgfpathclose%
\pgfusepath{fill}%
\end{pgfscope}%
\begin{pgfscope}%
\pgfpathrectangle{\pgfqpoint{0.041670in}{0.041670in}}{\pgfqpoint{2.216660in}{2.216660in}}%
\pgfusepath{clip}%
\pgfsetbuttcap%
\pgfsetroundjoin%
\definecolor{currentfill}{rgb}{0.195860,0.395433,0.555276}%
\pgfsetfillcolor{currentfill}%
\pgfsetlinewidth{0.000000pt}%
\definecolor{currentstroke}{rgb}{0.000000,0.000000,0.000000}%
\pgfsetstrokecolor{currentstroke}%
\pgfsetdash{}{0pt}%
\pgfpathmoveto{\pgfqpoint{0.916318in}{0.992407in}}%
\pgfpathlineto{\pgfqpoint{0.913724in}{0.983392in}}%
\pgfpathlineto{\pgfqpoint{0.911131in}{0.974414in}}%
\pgfpathlineto{\pgfqpoint{0.908538in}{0.965476in}}%
\pgfpathlineto{\pgfqpoint{0.905945in}{0.956582in}}%
\pgfpathlineto{\pgfqpoint{0.896859in}{0.961120in}}%
\pgfpathlineto{\pgfqpoint{0.888073in}{0.965801in}}%
\pgfpathlineto{\pgfqpoint{0.879595in}{0.970620in}}%
\pgfpathlineto{\pgfqpoint{0.871433in}{0.975570in}}%
\pgfpathlineto{\pgfqpoint{0.874345in}{0.984261in}}%
\pgfpathlineto{\pgfqpoint{0.877257in}{0.992997in}}%
\pgfpathlineto{\pgfqpoint{0.880170in}{1.001773in}}%
\pgfpathlineto{\pgfqpoint{0.883083in}{1.010587in}}%
\pgfpathlineto{\pgfqpoint{0.890944in}{1.005847in}}%
\pgfpathlineto{\pgfqpoint{0.899109in}{1.001233in}}%
\pgfpathlineto{\pgfqpoint{0.907570in}{0.996752in}}%
\pgfpathlineto{\pgfqpoint{0.916318in}{0.992407in}}%
\pgfpathclose%
\pgfusepath{fill}%
\end{pgfscope}%
\begin{pgfscope}%
\pgfpathrectangle{\pgfqpoint{0.041670in}{0.041670in}}{\pgfqpoint{2.216660in}{2.216660in}}%
\pgfusepath{clip}%
\pgfsetbuttcap%
\pgfsetroundjoin%
\definecolor{currentfill}{rgb}{0.896320,0.893616,0.096335}%
\pgfsetfillcolor{currentfill}%
\pgfsetlinewidth{0.000000pt}%
\definecolor{currentstroke}{rgb}{0.000000,0.000000,0.000000}%
\pgfsetstrokecolor{currentstroke}%
\pgfsetdash{}{0pt}%
\pgfpathmoveto{\pgfqpoint{1.119856in}{1.650421in}}%
\pgfpathlineto{\pgfqpoint{1.116870in}{1.648894in}}%
\pgfpathlineto{\pgfqpoint{1.113887in}{1.647243in}}%
\pgfpathlineto{\pgfqpoint{1.110906in}{1.645468in}}%
\pgfpathlineto{\pgfqpoint{1.107927in}{1.643570in}}%
\pgfpathlineto{\pgfqpoint{1.109845in}{1.644617in}}%
\pgfpathlineto{\pgfqpoint{1.111832in}{1.645634in}}%
\pgfpathlineto{\pgfqpoint{1.113886in}{1.646622in}}%
\pgfpathlineto{\pgfqpoint{1.116005in}{1.647579in}}%
\pgfpathlineto{\pgfqpoint{1.118648in}{1.649311in}}%
\pgfpathlineto{\pgfqpoint{1.121294in}{1.650919in}}%
\pgfpathlineto{\pgfqpoint{1.123941in}{1.652403in}}%
\pgfpathlineto{\pgfqpoint{1.126591in}{1.653763in}}%
\pgfpathlineto{\pgfqpoint{1.124824in}{1.652965in}}%
\pgfpathlineto{\pgfqpoint{1.123111in}{1.652142in}}%
\pgfpathlineto{\pgfqpoint{1.121455in}{1.651293in}}%
\pgfpathlineto{\pgfqpoint{1.119856in}{1.650421in}}%
\pgfpathclose%
\pgfusepath{fill}%
\end{pgfscope}%
\begin{pgfscope}%
\pgfpathrectangle{\pgfqpoint{0.041670in}{0.041670in}}{\pgfqpoint{2.216660in}{2.216660in}}%
\pgfusepath{clip}%
\pgfsetbuttcap%
\pgfsetroundjoin%
\definecolor{currentfill}{rgb}{0.268510,0.009605,0.335427}%
\pgfsetfillcolor{currentfill}%
\pgfsetlinewidth{0.000000pt}%
\definecolor{currentstroke}{rgb}{0.000000,0.000000,0.000000}%
\pgfsetstrokecolor{currentstroke}%
\pgfsetdash{}{0pt}%
\pgfpathmoveto{\pgfqpoint{1.580462in}{0.667620in}}%
\pgfpathlineto{\pgfqpoint{1.583217in}{0.664480in}}%
\pgfpathlineto{\pgfqpoint{1.585976in}{0.661562in}}%
\pgfpathlineto{\pgfqpoint{1.588741in}{0.658869in}}%
\pgfpathlineto{\pgfqpoint{1.591511in}{0.656408in}}%
\pgfpathlineto{\pgfqpoint{1.578324in}{0.649527in}}%
\pgfpathlineto{\pgfqpoint{1.564702in}{0.642868in}}%
\pgfpathlineto{\pgfqpoint{1.550661in}{0.636438in}}%
\pgfpathlineto{\pgfqpoint{1.536213in}{0.630246in}}%
\pgfpathlineto{\pgfqpoint{1.533806in}{0.632893in}}%
\pgfpathlineto{\pgfqpoint{1.531405in}{0.635771in}}%
\pgfpathlineto{\pgfqpoint{1.529008in}{0.638875in}}%
\pgfpathlineto{\pgfqpoint{1.526616in}{0.642201in}}%
\pgfpathlineto{\pgfqpoint{1.540682in}{0.648217in}}%
\pgfpathlineto{\pgfqpoint{1.554355in}{0.654464in}}%
\pgfpathlineto{\pgfqpoint{1.567619in}{0.660934in}}%
\pgfpathlineto{\pgfqpoint{1.580462in}{0.667620in}}%
\pgfpathclose%
\pgfusepath{fill}%
\end{pgfscope}%
\begin{pgfscope}%
\pgfpathrectangle{\pgfqpoint{0.041670in}{0.041670in}}{\pgfqpoint{2.216660in}{2.216660in}}%
\pgfusepath{clip}%
\pgfsetbuttcap%
\pgfsetroundjoin%
\definecolor{currentfill}{rgb}{0.699415,0.867117,0.175971}%
\pgfsetfillcolor{currentfill}%
\pgfsetlinewidth{0.000000pt}%
\definecolor{currentstroke}{rgb}{0.000000,0.000000,0.000000}%
\pgfsetstrokecolor{currentstroke}%
\pgfsetdash{}{0pt}%
\pgfpathmoveto{\pgfqpoint{1.040053in}{1.573579in}}%
\pgfpathlineto{\pgfqpoint{1.036201in}{1.569113in}}%
\pgfpathlineto{\pgfqpoint{1.032350in}{1.564539in}}%
\pgfpathlineto{\pgfqpoint{1.028503in}{1.559858in}}%
\pgfpathlineto{\pgfqpoint{1.024658in}{1.555072in}}%
\pgfpathlineto{\pgfqpoint{1.025070in}{1.557388in}}%
\pgfpathlineto{\pgfqpoint{1.025636in}{1.559695in}}%
\pgfpathlineto{\pgfqpoint{1.026355in}{1.561992in}}%
\pgfpathlineto{\pgfqpoint{1.027226in}{1.564276in}}%
\pgfpathlineto{\pgfqpoint{1.031002in}{1.568828in}}%
\pgfpathlineto{\pgfqpoint{1.034782in}{1.573274in}}%
\pgfpathlineto{\pgfqpoint{1.038563in}{1.577615in}}%
\pgfpathlineto{\pgfqpoint{1.042348in}{1.581848in}}%
\pgfpathlineto{\pgfqpoint{1.041568in}{1.579795in}}%
\pgfpathlineto{\pgfqpoint{1.040925in}{1.577732in}}%
\pgfpathlineto{\pgfqpoint{1.040420in}{1.575659in}}%
\pgfpathlineto{\pgfqpoint{1.040053in}{1.573579in}}%
\pgfpathclose%
\pgfusepath{fill}%
\end{pgfscope}%
\begin{pgfscope}%
\pgfpathrectangle{\pgfqpoint{0.041670in}{0.041670in}}{\pgfqpoint{2.216660in}{2.216660in}}%
\pgfusepath{clip}%
\pgfsetbuttcap%
\pgfsetroundjoin%
\definecolor{currentfill}{rgb}{0.935904,0.898570,0.108131}%
\pgfsetfillcolor{currentfill}%
\pgfsetlinewidth{0.000000pt}%
\definecolor{currentstroke}{rgb}{0.000000,0.000000,0.000000}%
\pgfsetstrokecolor{currentstroke}%
\pgfsetdash{}{0pt}%
\pgfpathmoveto{\pgfqpoint{1.201245in}{1.664086in}}%
\pgfpathlineto{\pgfqpoint{1.202567in}{1.663615in}}%
\pgfpathlineto{\pgfqpoint{1.203889in}{1.663017in}}%
\pgfpathlineto{\pgfqpoint{1.205209in}{1.662291in}}%
\pgfpathlineto{\pgfqpoint{1.206528in}{1.661439in}}%
\pgfpathlineto{\pgfqpoint{1.208821in}{1.661032in}}%
\pgfpathlineto{\pgfqpoint{1.211086in}{1.660591in}}%
\pgfpathlineto{\pgfqpoint{1.213321in}{1.660117in}}%
\pgfpathlineto{\pgfqpoint{1.215522in}{1.659611in}}%
\pgfpathlineto{\pgfqpoint{1.213757in}{1.660554in}}%
\pgfpathlineto{\pgfqpoint{1.211989in}{1.661370in}}%
\pgfpathlineto{\pgfqpoint{1.210220in}{1.662060in}}%
\pgfpathlineto{\pgfqpoint{1.208450in}{1.662622in}}%
\pgfpathlineto{\pgfqpoint{1.206686in}{1.663027in}}%
\pgfpathlineto{\pgfqpoint{1.204897in}{1.663407in}}%
\pgfpathlineto{\pgfqpoint{1.203082in}{1.663759in}}%
\pgfpathlineto{\pgfqpoint{1.201245in}{1.664086in}}%
\pgfpathclose%
\pgfusepath{fill}%
\end{pgfscope}%
\begin{pgfscope}%
\pgfpathrectangle{\pgfqpoint{0.041670in}{0.041670in}}{\pgfqpoint{2.216660in}{2.216660in}}%
\pgfusepath{clip}%
\pgfsetbuttcap%
\pgfsetroundjoin%
\definecolor{currentfill}{rgb}{0.344074,0.780029,0.397381}%
\pgfsetfillcolor{currentfill}%
\pgfsetlinewidth{0.000000pt}%
\definecolor{currentstroke}{rgb}{0.000000,0.000000,0.000000}%
\pgfsetstrokecolor{currentstroke}%
\pgfsetdash{}{0pt}%
\pgfpathmoveto{\pgfqpoint{0.973104in}{1.426473in}}%
\pgfpathlineto{\pgfqpoint{0.969456in}{1.419423in}}%
\pgfpathlineto{\pgfqpoint{0.965811in}{1.412298in}}%
\pgfpathlineto{\pgfqpoint{0.962168in}{1.405099in}}%
\pgfpathlineto{\pgfqpoint{0.958528in}{1.397829in}}%
\pgfpathlineto{\pgfqpoint{0.956420in}{1.401230in}}%
\pgfpathlineto{\pgfqpoint{0.954539in}{1.404660in}}%
\pgfpathlineto{\pgfqpoint{0.952886in}{1.408115in}}%
\pgfpathlineto{\pgfqpoint{0.951461in}{1.411592in}}%
\pgfpathlineto{\pgfqpoint{0.955210in}{1.418621in}}%
\pgfpathlineto{\pgfqpoint{0.958962in}{1.425580in}}%
\pgfpathlineto{\pgfqpoint{0.962716in}{1.432465in}}%
\pgfpathlineto{\pgfqpoint{0.966474in}{1.439276in}}%
\pgfpathlineto{\pgfqpoint{0.967812in}{1.436041in}}%
\pgfpathlineto{\pgfqpoint{0.969364in}{1.432827in}}%
\pgfpathlineto{\pgfqpoint{0.971129in}{1.429636in}}%
\pgfpathlineto{\pgfqpoint{0.973104in}{1.426473in}}%
\pgfpathclose%
\pgfusepath{fill}%
\end{pgfscope}%
\begin{pgfscope}%
\pgfpathrectangle{\pgfqpoint{0.041670in}{0.041670in}}{\pgfqpoint{2.216660in}{2.216660in}}%
\pgfusepath{clip}%
\pgfsetbuttcap%
\pgfsetroundjoin%
\definecolor{currentfill}{rgb}{0.274128,0.199721,0.498911}%
\pgfsetfillcolor{currentfill}%
\pgfsetlinewidth{0.000000pt}%
\definecolor{currentstroke}{rgb}{0.000000,0.000000,0.000000}%
\pgfsetstrokecolor{currentstroke}%
\pgfsetdash{}{0pt}%
\pgfpathmoveto{\pgfqpoint{0.909429in}{0.803398in}}%
\pgfpathlineto{\pgfqpoint{0.907194in}{0.795577in}}%
\pgfpathlineto{\pgfqpoint{0.904958in}{0.787862in}}%
\pgfpathlineto{\pgfqpoint{0.902721in}{0.780255in}}%
\pgfpathlineto{\pgfqpoint{0.900483in}{0.772762in}}%
\pgfpathlineto{\pgfqpoint{0.888387in}{0.777534in}}%
\pgfpathlineto{\pgfqpoint{0.876606in}{0.782502in}}%
\pgfpathlineto{\pgfqpoint{0.865152in}{0.787660in}}%
\pgfpathlineto{\pgfqpoint{0.854035in}{0.793003in}}%
\pgfpathlineto{\pgfqpoint{0.856638in}{0.800314in}}%
\pgfpathlineto{\pgfqpoint{0.859240in}{0.807738in}}%
\pgfpathlineto{\pgfqpoint{0.861841in}{0.815271in}}%
\pgfpathlineto{\pgfqpoint{0.864440in}{0.822910in}}%
\pgfpathlineto{\pgfqpoint{0.875208in}{0.817759in}}%
\pgfpathlineto{\pgfqpoint{0.886303in}{0.812786in}}%
\pgfpathlineto{\pgfqpoint{0.897714in}{0.807997in}}%
\pgfpathlineto{\pgfqpoint{0.909429in}{0.803398in}}%
\pgfpathclose%
\pgfusepath{fill}%
\end{pgfscope}%
\begin{pgfscope}%
\pgfpathrectangle{\pgfqpoint{0.041670in}{0.041670in}}{\pgfqpoint{2.216660in}{2.216660in}}%
\pgfusepath{clip}%
\pgfsetbuttcap%
\pgfsetroundjoin%
\definecolor{currentfill}{rgb}{0.179019,0.433756,0.557430}%
\pgfsetfillcolor{currentfill}%
\pgfsetlinewidth{0.000000pt}%
\definecolor{currentstroke}{rgb}{0.000000,0.000000,0.000000}%
\pgfsetstrokecolor{currentstroke}%
\pgfsetdash{}{0pt}%
\pgfpathmoveto{\pgfqpoint{1.471635in}{1.050278in}}%
\pgfpathlineto{\pgfqpoint{1.474616in}{1.041393in}}%
\pgfpathlineto{\pgfqpoint{1.477595in}{1.032534in}}%
\pgfpathlineto{\pgfqpoint{1.480574in}{1.023702in}}%
\pgfpathlineto{\pgfqpoint{1.483552in}{1.014902in}}%
\pgfpathlineto{\pgfqpoint{1.475968in}{1.010054in}}%
\pgfpathlineto{\pgfqpoint{1.468073in}{1.005328in}}%
\pgfpathlineto{\pgfqpoint{1.459875in}{1.000729in}}%
\pgfpathlineto{\pgfqpoint{1.451382in}{0.996262in}}%
\pgfpathlineto{\pgfqpoint{1.448712in}{1.005269in}}%
\pgfpathlineto{\pgfqpoint{1.446042in}{1.014306in}}%
\pgfpathlineto{\pgfqpoint{1.443372in}{1.023372in}}%
\pgfpathlineto{\pgfqpoint{1.440701in}{1.032462in}}%
\pgfpathlineto{\pgfqpoint{1.448866in}{1.036731in}}%
\pgfpathlineto{\pgfqpoint{1.456749in}{1.041126in}}%
\pgfpathlineto{\pgfqpoint{1.464341in}{1.045644in}}%
\pgfpathlineto{\pgfqpoint{1.471635in}{1.050278in}}%
\pgfpathclose%
\pgfusepath{fill}%
\end{pgfscope}%
\begin{pgfscope}%
\pgfpathrectangle{\pgfqpoint{0.041670in}{0.041670in}}{\pgfqpoint{2.216660in}{2.216660in}}%
\pgfusepath{clip}%
\pgfsetbuttcap%
\pgfsetroundjoin%
\definecolor{currentfill}{rgb}{0.201239,0.383670,0.554294}%
\pgfsetfillcolor{currentfill}%
\pgfsetlinewidth{0.000000pt}%
\definecolor{currentstroke}{rgb}{0.000000,0.000000,0.000000}%
\pgfsetstrokecolor{currentstroke}%
\pgfsetdash{}{0pt}%
\pgfpathmoveto{\pgfqpoint{0.506068in}{0.920265in}}%
\pgfpathlineto{\pgfqpoint{0.502039in}{0.932590in}}%
\pgfpathlineto{\pgfqpoint{0.497989in}{0.945401in}}%
\pgfpathlineto{\pgfqpoint{0.493916in}{0.958706in}}%
\pgfpathlineto{\pgfqpoint{0.489820in}{0.972513in}}%
\pgfpathlineto{\pgfqpoint{0.480799in}{0.983801in}}%
\pgfpathlineto{\pgfqpoint{0.472517in}{0.995210in}}%
\pgfpathlineto{\pgfqpoint{0.464981in}{1.006729in}}%
\pgfpathlineto{\pgfqpoint{0.458194in}{1.018346in}}%
\pgfpathlineto{\pgfqpoint{0.462450in}{1.004356in}}%
\pgfpathlineto{\pgfqpoint{0.466682in}{0.990866in}}%
\pgfpathlineto{\pgfqpoint{0.470892in}{0.977868in}}%
\pgfpathlineto{\pgfqpoint{0.475080in}{0.965353in}}%
\pgfpathlineto{\pgfqpoint{0.481734in}{0.953924in}}%
\pgfpathlineto{\pgfqpoint{0.489120in}{0.942592in}}%
\pgfpathlineto{\pgfqpoint{0.497233in}{0.931368in}}%
\pgfpathlineto{\pgfqpoint{0.506068in}{0.920265in}}%
\pgfpathclose%
\pgfusepath{fill}%
\end{pgfscope}%
\begin{pgfscope}%
\pgfpathrectangle{\pgfqpoint{0.041670in}{0.041670in}}{\pgfqpoint{2.216660in}{2.216660in}}%
\pgfusepath{clip}%
\pgfsetbuttcap%
\pgfsetroundjoin%
\definecolor{currentfill}{rgb}{0.855810,0.888601,0.097452}%
\pgfsetfillcolor{currentfill}%
\pgfsetlinewidth{0.000000pt}%
\definecolor{currentstroke}{rgb}{0.000000,0.000000,0.000000}%
\pgfsetstrokecolor{currentstroke}%
\pgfsetdash{}{0pt}%
\pgfpathmoveto{\pgfqpoint{1.095280in}{1.634282in}}%
\pgfpathlineto{\pgfqpoint{1.091775in}{1.631874in}}%
\pgfpathlineto{\pgfqpoint{1.088273in}{1.629346in}}%
\pgfpathlineto{\pgfqpoint{1.084774in}{1.626697in}}%
\pgfpathlineto{\pgfqpoint{1.081277in}{1.623928in}}%
\pgfpathlineto{\pgfqpoint{1.082798in}{1.625375in}}%
\pgfpathlineto{\pgfqpoint{1.084416in}{1.626798in}}%
\pgfpathlineto{\pgfqpoint{1.086128in}{1.628196in}}%
\pgfpathlineto{\pgfqpoint{1.087932in}{1.629568in}}%
\pgfpathlineto{\pgfqpoint{1.091191in}{1.632134in}}%
\pgfpathlineto{\pgfqpoint{1.094452in}{1.634581in}}%
\pgfpathlineto{\pgfqpoint{1.097715in}{1.636908in}}%
\pgfpathlineto{\pgfqpoint{1.100981in}{1.639114in}}%
\pgfpathlineto{\pgfqpoint{1.099435in}{1.637939in}}%
\pgfpathlineto{\pgfqpoint{1.097969in}{1.636741in}}%
\pgfpathlineto{\pgfqpoint{1.096583in}{1.635521in}}%
\pgfpathlineto{\pgfqpoint{1.095280in}{1.634282in}}%
\pgfpathclose%
\pgfusepath{fill}%
\end{pgfscope}%
\begin{pgfscope}%
\pgfpathrectangle{\pgfqpoint{0.041670in}{0.041670in}}{\pgfqpoint{2.216660in}{2.216660in}}%
\pgfusepath{clip}%
\pgfsetbuttcap%
\pgfsetroundjoin%
\definecolor{currentfill}{rgb}{0.636902,0.856542,0.216620}%
\pgfsetfillcolor{currentfill}%
\pgfsetlinewidth{0.000000pt}%
\definecolor{currentstroke}{rgb}{0.000000,0.000000,0.000000}%
\pgfsetstrokecolor{currentstroke}%
\pgfsetdash{}{0pt}%
\pgfpathmoveto{\pgfqpoint{1.334893in}{1.557131in}}%
\pgfpathlineto{\pgfqpoint{1.338725in}{1.552294in}}%
\pgfpathlineto{\pgfqpoint{1.342555in}{1.547355in}}%
\pgfpathlineto{\pgfqpoint{1.346381in}{1.542316in}}%
\pgfpathlineto{\pgfqpoint{1.350205in}{1.537178in}}%
\pgfpathlineto{\pgfqpoint{1.350645in}{1.534625in}}%
\pgfpathlineto{\pgfqpoint{1.350915in}{1.532065in}}%
\pgfpathlineto{\pgfqpoint{1.351015in}{1.529502in}}%
\pgfpathlineto{\pgfqpoint{1.350944in}{1.526938in}}%
\pgfpathlineto{\pgfqpoint{1.347098in}{1.532315in}}%
\pgfpathlineto{\pgfqpoint{1.343249in}{1.537592in}}%
\pgfpathlineto{\pgfqpoint{1.339398in}{1.542768in}}%
\pgfpathlineto{\pgfqpoint{1.335544in}{1.547842in}}%
\pgfpathlineto{\pgfqpoint{1.335614in}{1.550168in}}%
\pgfpathlineto{\pgfqpoint{1.335528in}{1.552493in}}%
\pgfpathlineto{\pgfqpoint{1.335288in}{1.554815in}}%
\pgfpathlineto{\pgfqpoint{1.334893in}{1.557131in}}%
\pgfpathclose%
\pgfusepath{fill}%
\end{pgfscope}%
\begin{pgfscope}%
\pgfpathrectangle{\pgfqpoint{0.041670in}{0.041670in}}{\pgfqpoint{2.216660in}{2.216660in}}%
\pgfusepath{clip}%
\pgfsetbuttcap%
\pgfsetroundjoin%
\definecolor{currentfill}{rgb}{0.935904,0.898570,0.108131}%
\pgfsetfillcolor{currentfill}%
\pgfsetlinewidth{0.000000pt}%
\definecolor{currentstroke}{rgb}{0.000000,0.000000,0.000000}%
\pgfsetstrokecolor{currentstroke}%
\pgfsetdash{}{0pt}%
\pgfpathmoveto{\pgfqpoint{1.149916in}{1.662239in}}%
\pgfpathlineto{\pgfqpoint{1.148049in}{1.661654in}}%
\pgfpathlineto{\pgfqpoint{1.146185in}{1.660941in}}%
\pgfpathlineto{\pgfqpoint{1.144321in}{1.660100in}}%
\pgfpathlineto{\pgfqpoint{1.142460in}{1.659134in}}%
\pgfpathlineto{\pgfqpoint{1.144630in}{1.659669in}}%
\pgfpathlineto{\pgfqpoint{1.146836in}{1.660172in}}%
\pgfpathlineto{\pgfqpoint{1.149074in}{1.660642in}}%
\pgfpathlineto{\pgfqpoint{1.151342in}{1.661079in}}%
\pgfpathlineto{\pgfqpoint{1.152762in}{1.661949in}}%
\pgfpathlineto{\pgfqpoint{1.154184in}{1.662692in}}%
\pgfpathlineto{\pgfqpoint{1.155607in}{1.663308in}}%
\pgfpathlineto{\pgfqpoint{1.157030in}{1.663797in}}%
\pgfpathlineto{\pgfqpoint{1.155213in}{1.663447in}}%
\pgfpathlineto{\pgfqpoint{1.153421in}{1.663071in}}%
\pgfpathlineto{\pgfqpoint{1.151654in}{1.662668in}}%
\pgfpathlineto{\pgfqpoint{1.149916in}{1.662239in}}%
\pgfpathclose%
\pgfusepath{fill}%
\end{pgfscope}%
\begin{pgfscope}%
\pgfpathrectangle{\pgfqpoint{0.041670in}{0.041670in}}{\pgfqpoint{2.216660in}{2.216660in}}%
\pgfusepath{clip}%
\pgfsetbuttcap%
\pgfsetroundjoin%
\definecolor{currentfill}{rgb}{0.487026,0.823929,0.312321}%
\pgfsetfillcolor{currentfill}%
\pgfsetlinewidth{0.000000pt}%
\definecolor{currentstroke}{rgb}{0.000000,0.000000,0.000000}%
\pgfsetstrokecolor{currentstroke}%
\pgfsetdash{}{0pt}%
\pgfpathmoveto{\pgfqpoint{0.996627in}{1.490833in}}%
\pgfpathlineto{\pgfqpoint{0.992849in}{1.484690in}}%
\pgfpathlineto{\pgfqpoint{0.989074in}{1.478458in}}%
\pgfpathlineto{\pgfqpoint{0.985300in}{1.472137in}}%
\pgfpathlineto{\pgfqpoint{0.981530in}{1.465730in}}%
\pgfpathlineto{\pgfqpoint{0.980478in}{1.468741in}}%
\pgfpathlineto{\pgfqpoint{0.979627in}{1.471765in}}%
\pgfpathlineto{\pgfqpoint{0.978976in}{1.474798in}}%
\pgfpathlineto{\pgfqpoint{0.978527in}{1.477839in}}%
\pgfpathlineto{\pgfqpoint{0.982348in}{1.484005in}}%
\pgfpathlineto{\pgfqpoint{0.986171in}{1.490084in}}%
\pgfpathlineto{\pgfqpoint{0.989998in}{1.496076in}}%
\pgfpathlineto{\pgfqpoint{0.993827in}{1.501979in}}%
\pgfpathlineto{\pgfqpoint{0.994249in}{1.499180in}}%
\pgfpathlineto{\pgfqpoint{0.994857in}{1.496387in}}%
\pgfpathlineto{\pgfqpoint{0.995650in}{1.493604in}}%
\pgfpathlineto{\pgfqpoint{0.996627in}{1.490833in}}%
\pgfpathclose%
\pgfusepath{fill}%
\end{pgfscope}%
\begin{pgfscope}%
\pgfpathrectangle{\pgfqpoint{0.041670in}{0.041670in}}{\pgfqpoint{2.216660in}{2.216660in}}%
\pgfusepath{clip}%
\pgfsetbuttcap%
\pgfsetroundjoin%
\definecolor{currentfill}{rgb}{0.134692,0.658636,0.517649}%
\pgfsetfillcolor{currentfill}%
\pgfsetlinewidth{0.000000pt}%
\definecolor{currentstroke}{rgb}{0.000000,0.000000,0.000000}%
\pgfsetstrokecolor{currentstroke}%
\pgfsetdash{}{0pt}%
\pgfpathmoveto{\pgfqpoint{1.435405in}{1.293235in}}%
\pgfpathlineto{\pgfqpoint{1.438891in}{1.285049in}}%
\pgfpathlineto{\pgfqpoint{1.442374in}{1.276823in}}%
\pgfpathlineto{\pgfqpoint{1.445856in}{1.268559in}}%
\pgfpathlineto{\pgfqpoint{1.449335in}{1.260260in}}%
\pgfpathlineto{\pgfqpoint{1.445564in}{1.256084in}}%
\pgfpathlineto{\pgfqpoint{1.441521in}{1.251967in}}%
\pgfpathlineto{\pgfqpoint{1.437210in}{1.247913in}}%
\pgfpathlineto{\pgfqpoint{1.432634in}{1.243926in}}%
\pgfpathlineto{\pgfqpoint{1.429363in}{1.252456in}}%
\pgfpathlineto{\pgfqpoint{1.426090in}{1.260951in}}%
\pgfpathlineto{\pgfqpoint{1.422816in}{1.269407in}}%
\pgfpathlineto{\pgfqpoint{1.419539in}{1.277822in}}%
\pgfpathlineto{\pgfqpoint{1.423885in}{1.281584in}}%
\pgfpathlineto{\pgfqpoint{1.427980in}{1.285409in}}%
\pgfpathlineto{\pgfqpoint{1.431822in}{1.289294in}}%
\pgfpathlineto{\pgfqpoint{1.435405in}{1.293235in}}%
\pgfpathclose%
\pgfusepath{fill}%
\end{pgfscope}%
\begin{pgfscope}%
\pgfpathrectangle{\pgfqpoint{0.041670in}{0.041670in}}{\pgfqpoint{2.216660in}{2.216660in}}%
\pgfusepath{clip}%
\pgfsetbuttcap%
\pgfsetroundjoin%
\definecolor{currentfill}{rgb}{0.282884,0.135920,0.453427}%
\pgfsetfillcolor{currentfill}%
\pgfsetlinewidth{0.000000pt}%
\definecolor{currentstroke}{rgb}{0.000000,0.000000,0.000000}%
\pgfsetstrokecolor{currentstroke}%
\pgfsetdash{}{0pt}%
\pgfpathmoveto{\pgfqpoint{1.762593in}{0.751817in}}%
\pgfpathlineto{\pgfqpoint{1.766175in}{0.757217in}}%
\pgfpathlineto{\pgfqpoint{1.769771in}{0.762988in}}%
\pgfpathlineto{\pgfqpoint{1.773382in}{0.769138in}}%
\pgfpathlineto{\pgfqpoint{1.777007in}{0.775673in}}%
\pgfpathlineto{\pgfqpoint{1.766038in}{0.765781in}}%
\pgfpathlineto{\pgfqpoint{1.754438in}{0.756064in}}%
\pgfpathlineto{\pgfqpoint{1.742217in}{0.746535in}}%
\pgfpathlineto{\pgfqpoint{1.729387in}{0.737203in}}%
\pgfpathlineto{\pgfqpoint{1.726033in}{0.730869in}}%
\pgfpathlineto{\pgfqpoint{1.722693in}{0.724921in}}%
\pgfpathlineto{\pgfqpoint{1.719367in}{0.719353in}}%
\pgfpathlineto{\pgfqpoint{1.716054in}{0.714158in}}%
\pgfpathlineto{\pgfqpoint{1.728590in}{0.723292in}}%
\pgfpathlineto{\pgfqpoint{1.740532in}{0.732620in}}%
\pgfpathlineto{\pgfqpoint{1.751870in}{0.742132in}}%
\pgfpathlineto{\pgfqpoint{1.762593in}{0.751817in}}%
\pgfpathclose%
\pgfusepath{fill}%
\end{pgfscope}%
\begin{pgfscope}%
\pgfpathrectangle{\pgfqpoint{0.041670in}{0.041670in}}{\pgfqpoint{2.216660in}{2.216660in}}%
\pgfusepath{clip}%
\pgfsetbuttcap%
\pgfsetroundjoin%
\definecolor{currentfill}{rgb}{0.220124,0.725509,0.466226}%
\pgfsetfillcolor{currentfill}%
\pgfsetlinewidth{0.000000pt}%
\definecolor{currentstroke}{rgb}{0.000000,0.000000,0.000000}%
\pgfsetstrokecolor{currentstroke}%
\pgfsetdash{}{0pt}%
\pgfpathmoveto{\pgfqpoint{0.955309in}{1.353895in}}%
\pgfpathlineto{\pgfqpoint{0.951847in}{1.346065in}}%
\pgfpathlineto{\pgfqpoint{0.948388in}{1.338176in}}%
\pgfpathlineto{\pgfqpoint{0.944931in}{1.330230in}}%
\pgfpathlineto{\pgfqpoint{0.941476in}{1.322230in}}%
\pgfpathlineto{\pgfqpoint{0.938107in}{1.325942in}}%
\pgfpathlineto{\pgfqpoint{0.934986in}{1.329703in}}%
\pgfpathlineto{\pgfqpoint{0.932113in}{1.333508in}}%
\pgfpathlineto{\pgfqpoint{0.929492in}{1.337354in}}%
\pgfpathlineto{\pgfqpoint{0.933113in}{1.345118in}}%
\pgfpathlineto{\pgfqpoint{0.936736in}{1.352828in}}%
\pgfpathlineto{\pgfqpoint{0.940362in}{1.360482in}}%
\pgfpathlineto{\pgfqpoint{0.943990in}{1.368077in}}%
\pgfpathlineto{\pgfqpoint{0.946467in}{1.364471in}}%
\pgfpathlineto{\pgfqpoint{0.949180in}{1.360903in}}%
\pgfpathlineto{\pgfqpoint{0.952128in}{1.357376in}}%
\pgfpathlineto{\pgfqpoint{0.955309in}{1.353895in}}%
\pgfpathclose%
\pgfusepath{fill}%
\end{pgfscope}%
\begin{pgfscope}%
\pgfpathrectangle{\pgfqpoint{0.041670in}{0.041670in}}{\pgfqpoint{2.216660in}{2.216660in}}%
\pgfusepath{clip}%
\pgfsetbuttcap%
\pgfsetroundjoin%
\definecolor{currentfill}{rgb}{0.248629,0.278775,0.534556}%
\pgfsetfillcolor{currentfill}%
\pgfsetlinewidth{0.000000pt}%
\definecolor{currentstroke}{rgb}{0.000000,0.000000,0.000000}%
\pgfsetstrokecolor{currentstroke}%
\pgfsetdash{}{0pt}%
\pgfpathmoveto{\pgfqpoint{1.483396in}{0.891763in}}%
\pgfpathlineto{\pgfqpoint{1.486065in}{0.883452in}}%
\pgfpathlineto{\pgfqpoint{1.488733in}{0.875219in}}%
\pgfpathlineto{\pgfqpoint{1.491402in}{0.867066in}}%
\pgfpathlineto{\pgfqpoint{1.494072in}{0.858997in}}%
\pgfpathlineto{\pgfqpoint{1.483940in}{0.853890in}}%
\pgfpathlineto{\pgfqpoint{1.473484in}{0.848949in}}%
\pgfpathlineto{\pgfqpoint{1.462714in}{0.844180in}}%
\pgfpathlineto{\pgfqpoint{1.451640in}{0.839590in}}%
\pgfpathlineto{\pgfqpoint{1.449325in}{0.847845in}}%
\pgfpathlineto{\pgfqpoint{1.447010in}{0.856185in}}%
\pgfpathlineto{\pgfqpoint{1.444696in}{0.864604in}}%
\pgfpathlineto{\pgfqpoint{1.442382in}{0.873101in}}%
\pgfpathlineto{\pgfqpoint{1.453084in}{0.877515in}}%
\pgfpathlineto{\pgfqpoint{1.463494in}{0.882100in}}%
\pgfpathlineto{\pgfqpoint{1.473602in}{0.886851in}}%
\pgfpathlineto{\pgfqpoint{1.483396in}{0.891763in}}%
\pgfpathclose%
\pgfusepath{fill}%
\end{pgfscope}%
\begin{pgfscope}%
\pgfpathrectangle{\pgfqpoint{0.041670in}{0.041670in}}{\pgfqpoint{2.216660in}{2.216660in}}%
\pgfusepath{clip}%
\pgfsetbuttcap%
\pgfsetroundjoin%
\definecolor{currentfill}{rgb}{0.271305,0.019942,0.347269}%
\pgfsetfillcolor{currentfill}%
\pgfsetlinewidth{0.000000pt}%
\definecolor{currentstroke}{rgb}{0.000000,0.000000,0.000000}%
\pgfsetstrokecolor{currentstroke}%
\pgfsetdash{}{0pt}%
\pgfpathmoveto{\pgfqpoint{0.855297in}{0.652632in}}%
\pgfpathlineto{\pgfqpoint{0.853008in}{0.648428in}}%
\pgfpathlineto{\pgfqpoint{0.850715in}{0.644427in}}%
\pgfpathlineto{\pgfqpoint{0.848418in}{0.640634in}}%
\pgfpathlineto{\pgfqpoint{0.846117in}{0.637053in}}%
\pgfpathlineto{\pgfqpoint{0.831712in}{0.642858in}}%
\pgfpathlineto{\pgfqpoint{0.817689in}{0.648900in}}%
\pgfpathlineto{\pgfqpoint{0.804061in}{0.655172in}}%
\pgfpathlineto{\pgfqpoint{0.790843in}{0.661666in}}%
\pgfpathlineto{\pgfqpoint{0.793516in}{0.665066in}}%
\pgfpathlineto{\pgfqpoint{0.796185in}{0.668678in}}%
\pgfpathlineto{\pgfqpoint{0.798849in}{0.672497in}}%
\pgfpathlineto{\pgfqpoint{0.801509in}{0.676520in}}%
\pgfpathlineto{\pgfqpoint{0.814373in}{0.670216in}}%
\pgfpathlineto{\pgfqpoint{0.827636in}{0.664129in}}%
\pgfpathlineto{\pgfqpoint{0.841281in}{0.658265in}}%
\pgfpathlineto{\pgfqpoint{0.855297in}{0.652632in}}%
\pgfpathclose%
\pgfusepath{fill}%
\end{pgfscope}%
\begin{pgfscope}%
\pgfpathrectangle{\pgfqpoint{0.041670in}{0.041670in}}{\pgfqpoint{2.216660in}{2.216660in}}%
\pgfusepath{clip}%
\pgfsetbuttcap%
\pgfsetroundjoin%
\definecolor{currentfill}{rgb}{0.935904,0.898570,0.108131}%
\pgfsetfillcolor{currentfill}%
\pgfsetlinewidth{0.000000pt}%
\definecolor{currentstroke}{rgb}{0.000000,0.000000,0.000000}%
\pgfsetstrokecolor{currentstroke}%
\pgfsetdash{}{0pt}%
\pgfpathmoveto{\pgfqpoint{1.208450in}{1.662622in}}%
\pgfpathlineto{\pgfqpoint{1.210220in}{1.662060in}}%
\pgfpathlineto{\pgfqpoint{1.211989in}{1.661370in}}%
\pgfpathlineto{\pgfqpoint{1.213757in}{1.660554in}}%
\pgfpathlineto{\pgfqpoint{1.215522in}{1.659611in}}%
\pgfpathlineto{\pgfqpoint{1.217689in}{1.659072in}}%
\pgfpathlineto{\pgfqpoint{1.219819in}{1.658502in}}%
\pgfpathlineto{\pgfqpoint{1.221910in}{1.657901in}}%
\pgfpathlineto{\pgfqpoint{1.223959in}{1.657269in}}%
\pgfpathlineto{\pgfqpoint{1.221773in}{1.658329in}}%
\pgfpathlineto{\pgfqpoint{1.219586in}{1.659262in}}%
\pgfpathlineto{\pgfqpoint{1.217397in}{1.660068in}}%
\pgfpathlineto{\pgfqpoint{1.215206in}{1.660747in}}%
\pgfpathlineto{\pgfqpoint{1.213565in}{1.661252in}}%
\pgfpathlineto{\pgfqpoint{1.211891in}{1.661734in}}%
\pgfpathlineto{\pgfqpoint{1.210185in}{1.662190in}}%
\pgfpathlineto{\pgfqpoint{1.208450in}{1.662622in}}%
\pgfpathclose%
\pgfusepath{fill}%
\end{pgfscope}%
\begin{pgfscope}%
\pgfpathrectangle{\pgfqpoint{0.041670in}{0.041670in}}{\pgfqpoint{2.216660in}{2.216660in}}%
\pgfusepath{clip}%
\pgfsetbuttcap%
\pgfsetroundjoin%
\definecolor{currentfill}{rgb}{0.263663,0.237631,0.518762}%
\pgfsetfillcolor{currentfill}%
\pgfsetlinewidth{0.000000pt}%
\definecolor{currentstroke}{rgb}{0.000000,0.000000,0.000000}%
\pgfsetstrokecolor{currentstroke}%
\pgfsetdash{}{0pt}%
\pgfpathmoveto{\pgfqpoint{0.918358in}{0.835663in}}%
\pgfpathlineto{\pgfqpoint{0.916127in}{0.827456in}}%
\pgfpathlineto{\pgfqpoint{0.913895in}{0.819341in}}%
\pgfpathlineto{\pgfqpoint{0.911662in}{0.811320in}}%
\pgfpathlineto{\pgfqpoint{0.909429in}{0.803398in}}%
\pgfpathlineto{\pgfqpoint{0.897714in}{0.807997in}}%
\pgfpathlineto{\pgfqpoint{0.886303in}{0.812786in}}%
\pgfpathlineto{\pgfqpoint{0.875208in}{0.817759in}}%
\pgfpathlineto{\pgfqpoint{0.864440in}{0.822910in}}%
\pgfpathlineto{\pgfqpoint{0.867038in}{0.830651in}}%
\pgfpathlineto{\pgfqpoint{0.869635in}{0.838490in}}%
\pgfpathlineto{\pgfqpoint{0.872232in}{0.846424in}}%
\pgfpathlineto{\pgfqpoint{0.874827in}{0.854449in}}%
\pgfpathlineto{\pgfqpoint{0.885248in}{0.849490in}}%
\pgfpathlineto{\pgfqpoint{0.895984in}{0.844702in}}%
\pgfpathlineto{\pgfqpoint{0.907025in}{0.840091in}}%
\pgfpathlineto{\pgfqpoint{0.918358in}{0.835663in}}%
\pgfpathclose%
\pgfusepath{fill}%
\end{pgfscope}%
\begin{pgfscope}%
\pgfpathrectangle{\pgfqpoint{0.041670in}{0.041670in}}{\pgfqpoint{2.216660in}{2.216660in}}%
\pgfusepath{clip}%
\pgfsetbuttcap%
\pgfsetroundjoin%
\definecolor{currentfill}{rgb}{0.935904,0.898570,0.108131}%
\pgfsetfillcolor{currentfill}%
\pgfsetlinewidth{0.000000pt}%
\definecolor{currentstroke}{rgb}{0.000000,0.000000,0.000000}%
\pgfsetstrokecolor{currentstroke}%
\pgfsetdash{}{0pt}%
\pgfpathmoveto{\pgfqpoint{1.143274in}{1.660277in}}%
\pgfpathlineto{\pgfqpoint{1.140994in}{1.659569in}}%
\pgfpathlineto{\pgfqpoint{1.138716in}{1.658734in}}%
\pgfpathlineto{\pgfqpoint{1.136440in}{1.657771in}}%
\pgfpathlineto{\pgfqpoint{1.134165in}{1.656683in}}%
\pgfpathlineto{\pgfqpoint{1.136176in}{1.657341in}}%
\pgfpathlineto{\pgfqpoint{1.138231in}{1.657969in}}%
\pgfpathlineto{\pgfqpoint{1.140326in}{1.658567in}}%
\pgfpathlineto{\pgfqpoint{1.142460in}{1.659134in}}%
\pgfpathlineto{\pgfqpoint{1.144321in}{1.660100in}}%
\pgfpathlineto{\pgfqpoint{1.146185in}{1.660941in}}%
\pgfpathlineto{\pgfqpoint{1.148049in}{1.661654in}}%
\pgfpathlineto{\pgfqpoint{1.149916in}{1.662239in}}%
\pgfpathlineto{\pgfqpoint{1.148207in}{1.661786in}}%
\pgfpathlineto{\pgfqpoint{1.146529in}{1.661307in}}%
\pgfpathlineto{\pgfqpoint{1.144884in}{1.660804in}}%
\pgfpathlineto{\pgfqpoint{1.143274in}{1.660277in}}%
\pgfpathclose%
\pgfusepath{fill}%
\end{pgfscope}%
\begin{pgfscope}%
\pgfpathrectangle{\pgfqpoint{0.041670in}{0.041670in}}{\pgfqpoint{2.216660in}{2.216660in}}%
\pgfusepath{clip}%
\pgfsetbuttcap%
\pgfsetroundjoin%
\definecolor{currentfill}{rgb}{0.896320,0.893616,0.096335}%
\pgfsetfillcolor{currentfill}%
\pgfsetlinewidth{0.000000pt}%
\definecolor{currentstroke}{rgb}{0.000000,0.000000,0.000000}%
\pgfsetstrokecolor{currentstroke}%
\pgfsetdash{}{0pt}%
\pgfpathmoveto{\pgfqpoint{1.238635in}{1.651198in}}%
\pgfpathlineto{\pgfqpoint{1.241550in}{1.649710in}}%
\pgfpathlineto{\pgfqpoint{1.244463in}{1.648097in}}%
\pgfpathlineto{\pgfqpoint{1.247373in}{1.646361in}}%
\pgfpathlineto{\pgfqpoint{1.250281in}{1.644502in}}%
\pgfpathlineto{\pgfqpoint{1.252192in}{1.643452in}}%
\pgfpathlineto{\pgfqpoint{1.254031in}{1.642374in}}%
\pgfpathlineto{\pgfqpoint{1.255797in}{1.641270in}}%
\pgfpathlineto{\pgfqpoint{1.257489in}{1.640139in}}%
\pgfpathlineto{\pgfqpoint{1.254281in}{1.642180in}}%
\pgfpathlineto{\pgfqpoint{1.251070in}{1.644098in}}%
\pgfpathlineto{\pgfqpoint{1.247857in}{1.645893in}}%
\pgfpathlineto{\pgfqpoint{1.244642in}{1.647562in}}%
\pgfpathlineto{\pgfqpoint{1.243233in}{1.648504in}}%
\pgfpathlineto{\pgfqpoint{1.241761in}{1.649424in}}%
\pgfpathlineto{\pgfqpoint{1.240228in}{1.650322in}}%
\pgfpathlineto{\pgfqpoint{1.238635in}{1.651198in}}%
\pgfpathclose%
\pgfusepath{fill}%
\end{pgfscope}%
\begin{pgfscope}%
\pgfpathrectangle{\pgfqpoint{0.041670in}{0.041670in}}{\pgfqpoint{2.216660in}{2.216660in}}%
\pgfusepath{clip}%
\pgfsetbuttcap%
\pgfsetroundjoin%
\definecolor{currentfill}{rgb}{0.636902,0.856542,0.216620}%
\pgfsetfillcolor{currentfill}%
\pgfsetlinewidth{0.000000pt}%
\definecolor{currentstroke}{rgb}{0.000000,0.000000,0.000000}%
\pgfsetstrokecolor{currentstroke}%
\pgfsetdash{}{0pt}%
\pgfpathmoveto{\pgfqpoint{1.024558in}{1.545776in}}%
\pgfpathlineto{\pgfqpoint{1.020708in}{1.540649in}}%
\pgfpathlineto{\pgfqpoint{1.016860in}{1.535420in}}%
\pgfpathlineto{\pgfqpoint{1.013014in}{1.530089in}}%
\pgfpathlineto{\pgfqpoint{1.009172in}{1.524660in}}%
\pgfpathlineto{\pgfqpoint{1.008949in}{1.527223in}}%
\pgfpathlineto{\pgfqpoint{1.008898in}{1.529787in}}%
\pgfpathlineto{\pgfqpoint{1.009017in}{1.532350in}}%
\pgfpathlineto{\pgfqpoint{1.009306in}{1.534909in}}%
\pgfpathlineto{\pgfqpoint{1.013139in}{1.540100in}}%
\pgfpathlineto{\pgfqpoint{1.016976in}{1.545192in}}%
\pgfpathlineto{\pgfqpoint{1.020816in}{1.550183in}}%
\pgfpathlineto{\pgfqpoint{1.024658in}{1.555072in}}%
\pgfpathlineto{\pgfqpoint{1.024400in}{1.552751in}}%
\pgfpathlineto{\pgfqpoint{1.024298in}{1.550426in}}%
\pgfpathlineto{\pgfqpoint{1.024350in}{1.548100in}}%
\pgfpathlineto{\pgfqpoint{1.024558in}{1.545776in}}%
\pgfpathclose%
\pgfusepath{fill}%
\end{pgfscope}%
\begin{pgfscope}%
\pgfpathrectangle{\pgfqpoint{0.041670in}{0.041670in}}{\pgfqpoint{2.216660in}{2.216660in}}%
\pgfusepath{clip}%
\pgfsetbuttcap%
\pgfsetroundjoin%
\definecolor{currentfill}{rgb}{0.133743,0.548535,0.553541}%
\pgfsetfillcolor{currentfill}%
\pgfsetlinewidth{0.000000pt}%
\definecolor{currentstroke}{rgb}{0.000000,0.000000,0.000000}%
\pgfsetstrokecolor{currentstroke}%
\pgfsetdash{}{0pt}%
\pgfpathmoveto{\pgfqpoint{0.929817in}{1.154207in}}%
\pgfpathlineto{\pgfqpoint{0.926887in}{1.145205in}}%
\pgfpathlineto{\pgfqpoint{0.923959in}{1.136193in}}%
\pgfpathlineto{\pgfqpoint{0.921033in}{1.127175in}}%
\pgfpathlineto{\pgfqpoint{0.918107in}{1.118151in}}%
\pgfpathlineto{\pgfqpoint{0.911427in}{1.122374in}}%
\pgfpathlineto{\pgfqpoint{0.905027in}{1.126699in}}%
\pgfpathlineto{\pgfqpoint{0.898912in}{1.131120in}}%
\pgfpathlineto{\pgfqpoint{0.893089in}{1.135633in}}%
\pgfpathlineto{\pgfqpoint{0.896286in}{1.144439in}}%
\pgfpathlineto{\pgfqpoint{0.899485in}{1.153242in}}%
\pgfpathlineto{\pgfqpoint{0.902685in}{1.162037in}}%
\pgfpathlineto{\pgfqpoint{0.905887in}{1.170823in}}%
\pgfpathlineto{\pgfqpoint{0.911458in}{1.166533in}}%
\pgfpathlineto{\pgfqpoint{0.917307in}{1.162330in}}%
\pgfpathlineto{\pgfqpoint{0.923428in}{1.158220in}}%
\pgfpathlineto{\pgfqpoint{0.929817in}{1.154207in}}%
\pgfpathclose%
\pgfusepath{fill}%
\end{pgfscope}%
\begin{pgfscope}%
\pgfpathrectangle{\pgfqpoint{0.041670in}{0.041670in}}{\pgfqpoint{2.216660in}{2.216660in}}%
\pgfusepath{clip}%
\pgfsetbuttcap%
\pgfsetroundjoin%
\definecolor{currentfill}{rgb}{0.260571,0.246922,0.522828}%
\pgfsetfillcolor{currentfill}%
\pgfsetlinewidth{0.000000pt}%
\definecolor{currentstroke}{rgb}{0.000000,0.000000,0.000000}%
\pgfsetstrokecolor{currentstroke}%
\pgfsetdash{}{0pt}%
\pgfpathmoveto{\pgfqpoint{0.578186in}{0.796815in}}%
\pgfpathlineto{\pgfqpoint{0.574537in}{0.805332in}}%
\pgfpathlineto{\pgfqpoint{0.570871in}{0.814276in}}%
\pgfpathlineto{\pgfqpoint{0.567187in}{0.823653in}}%
\pgfpathlineto{\pgfqpoint{0.563485in}{0.833472in}}%
\pgfpathlineto{\pgfqpoint{0.552088in}{0.843780in}}%
\pgfpathlineto{\pgfqpoint{0.541365in}{0.854256in}}%
\pgfpathlineto{\pgfqpoint{0.531326in}{0.864888in}}%
\pgfpathlineto{\pgfqpoint{0.521977in}{0.875664in}}%
\pgfpathlineto{\pgfqpoint{0.525905in}{0.865650in}}%
\pgfpathlineto{\pgfqpoint{0.529814in}{0.856075in}}%
\pgfpathlineto{\pgfqpoint{0.533706in}{0.846932in}}%
\pgfpathlineto{\pgfqpoint{0.537579in}{0.838214in}}%
\pgfpathlineto{\pgfqpoint{0.546728in}{0.827639in}}%
\pgfpathlineto{\pgfqpoint{0.556551in}{0.817207in}}%
\pgfpathlineto{\pgfqpoint{0.567040in}{0.806928in}}%
\pgfpathlineto{\pgfqpoint{0.578186in}{0.796815in}}%
\pgfpathclose%
\pgfusepath{fill}%
\end{pgfscope}%
\begin{pgfscope}%
\pgfpathrectangle{\pgfqpoint{0.041670in}{0.041670in}}{\pgfqpoint{2.216660in}{2.216660in}}%
\pgfusepath{clip}%
\pgfsetbuttcap%
\pgfsetroundjoin%
\definecolor{currentfill}{rgb}{0.896320,0.893616,0.096335}%
\pgfsetfillcolor{currentfill}%
\pgfsetlinewidth{0.000000pt}%
\definecolor{currentstroke}{rgb}{0.000000,0.000000,0.000000}%
\pgfsetstrokecolor{currentstroke}%
\pgfsetdash{}{0pt}%
\pgfpathmoveto{\pgfqpoint{1.114068in}{1.646708in}}%
\pgfpathlineto{\pgfqpoint{1.110793in}{1.644996in}}%
\pgfpathlineto{\pgfqpoint{1.107520in}{1.643159in}}%
\pgfpathlineto{\pgfqpoint{1.104249in}{1.641198in}}%
\pgfpathlineto{\pgfqpoint{1.100981in}{1.639114in}}%
\pgfpathlineto{\pgfqpoint{1.102605in}{1.640266in}}%
\pgfpathlineto{\pgfqpoint{1.104305in}{1.641394in}}%
\pgfpathlineto{\pgfqpoint{1.106079in}{1.642495in}}%
\pgfpathlineto{\pgfqpoint{1.107927in}{1.643570in}}%
\pgfpathlineto{\pgfqpoint{1.110906in}{1.645468in}}%
\pgfpathlineto{\pgfqpoint{1.113887in}{1.647243in}}%
\pgfpathlineto{\pgfqpoint{1.116870in}{1.648894in}}%
\pgfpathlineto{\pgfqpoint{1.119856in}{1.650421in}}%
\pgfpathlineto{\pgfqpoint{1.118316in}{1.649525in}}%
\pgfpathlineto{\pgfqpoint{1.116838in}{1.648607in}}%
\pgfpathlineto{\pgfqpoint{1.115421in}{1.647668in}}%
\pgfpathlineto{\pgfqpoint{1.114068in}{1.646708in}}%
\pgfpathclose%
\pgfusepath{fill}%
\end{pgfscope}%
\begin{pgfscope}%
\pgfpathrectangle{\pgfqpoint{0.041670in}{0.041670in}}{\pgfqpoint{2.216660in}{2.216660in}}%
\pgfusepath{clip}%
\pgfsetbuttcap%
\pgfsetroundjoin%
\definecolor{currentfill}{rgb}{0.412913,0.803041,0.357269}%
\pgfsetfillcolor{currentfill}%
\pgfsetlinewidth{0.000000pt}%
\definecolor{currentstroke}{rgb}{0.000000,0.000000,0.000000}%
\pgfsetstrokecolor{currentstroke}%
\pgfsetdash{}{0pt}%
\pgfpathmoveto{\pgfqpoint{1.379324in}{1.468406in}}%
\pgfpathlineto{\pgfqpoint{1.383109in}{1.461968in}}%
\pgfpathlineto{\pgfqpoint{1.386891in}{1.455447in}}%
\pgfpathlineto{\pgfqpoint{1.390670in}{1.448846in}}%
\pgfpathlineto{\pgfqpoint{1.394446in}{1.442166in}}%
\pgfpathlineto{\pgfqpoint{1.393298in}{1.438916in}}%
\pgfpathlineto{\pgfqpoint{1.391936in}{1.435683in}}%
\pgfpathlineto{\pgfqpoint{1.390360in}{1.432471in}}%
\pgfpathlineto{\pgfqpoint{1.388572in}{1.429283in}}%
\pgfpathlineto{\pgfqpoint{1.384892in}{1.436203in}}%
\pgfpathlineto{\pgfqpoint{1.381210in}{1.443044in}}%
\pgfpathlineto{\pgfqpoint{1.377525in}{1.449804in}}%
\pgfpathlineto{\pgfqpoint{1.373838in}{1.456480in}}%
\pgfpathlineto{\pgfqpoint{1.375507in}{1.459431in}}%
\pgfpathlineto{\pgfqpoint{1.376978in}{1.462404in}}%
\pgfpathlineto{\pgfqpoint{1.378251in}{1.465397in}}%
\pgfpathlineto{\pgfqpoint{1.379324in}{1.468406in}}%
\pgfpathclose%
\pgfusepath{fill}%
\end{pgfscope}%
\begin{pgfscope}%
\pgfpathrectangle{\pgfqpoint{0.041670in}{0.041670in}}{\pgfqpoint{2.216660in}{2.216660in}}%
\pgfusepath{clip}%
\pgfsetbuttcap%
\pgfsetroundjoin%
\definecolor{currentfill}{rgb}{0.179019,0.433756,0.557430}%
\pgfsetfillcolor{currentfill}%
\pgfsetlinewidth{0.000000pt}%
\definecolor{currentstroke}{rgb}{0.000000,0.000000,0.000000}%
\pgfsetstrokecolor{currentstroke}%
\pgfsetdash{}{0pt}%
\pgfpathmoveto{\pgfqpoint{0.926699in}{1.028777in}}%
\pgfpathlineto{\pgfqpoint{0.924103in}{1.019644in}}%
\pgfpathlineto{\pgfqpoint{0.921507in}{1.010537in}}%
\pgfpathlineto{\pgfqpoint{0.918913in}{1.001456in}}%
\pgfpathlineto{\pgfqpoint{0.916318in}{0.992407in}}%
\pgfpathlineto{\pgfqpoint{0.907570in}{0.996752in}}%
\pgfpathlineto{\pgfqpoint{0.899109in}{1.001233in}}%
\pgfpathlineto{\pgfqpoint{0.890944in}{1.005847in}}%
\pgfpathlineto{\pgfqpoint{0.883083in}{1.010587in}}%
\pgfpathlineto{\pgfqpoint{0.885997in}{1.019434in}}%
\pgfpathlineto{\pgfqpoint{0.888912in}{1.028314in}}%
\pgfpathlineto{\pgfqpoint{0.891827in}{1.037221in}}%
\pgfpathlineto{\pgfqpoint{0.894743in}{1.046153in}}%
\pgfpathlineto{\pgfqpoint{0.902303in}{1.041622in}}%
\pgfpathlineto{\pgfqpoint{0.910153in}{1.037213in}}%
\pgfpathlineto{\pgfqpoint{0.918288in}{1.032930in}}%
\pgfpathlineto{\pgfqpoint{0.926699in}{1.028777in}}%
\pgfpathclose%
\pgfusepath{fill}%
\end{pgfscope}%
\begin{pgfscope}%
\pgfpathrectangle{\pgfqpoint{0.041670in}{0.041670in}}{\pgfqpoint{2.216660in}{2.216660in}}%
\pgfusepath{clip}%
\pgfsetbuttcap%
\pgfsetroundjoin%
\definecolor{currentfill}{rgb}{0.814576,0.883393,0.110347}%
\pgfsetfillcolor{currentfill}%
\pgfsetlinewidth{0.000000pt}%
\definecolor{currentstroke}{rgb}{0.000000,0.000000,0.000000}%
\pgfsetstrokecolor{currentstroke}%
\pgfsetdash{}{0pt}%
\pgfpathmoveto{\pgfqpoint{1.282739in}{1.619293in}}%
\pgfpathlineto{\pgfqpoint{1.286381in}{1.616240in}}%
\pgfpathlineto{\pgfqpoint{1.290021in}{1.613070in}}%
\pgfpathlineto{\pgfqpoint{1.293659in}{1.609784in}}%
\pgfpathlineto{\pgfqpoint{1.297293in}{1.606383in}}%
\pgfpathlineto{\pgfqpoint{1.298567in}{1.604635in}}%
\pgfpathlineto{\pgfqpoint{1.299724in}{1.602869in}}%
\pgfpathlineto{\pgfqpoint{1.300762in}{1.601085in}}%
\pgfpathlineto{\pgfqpoint{1.301681in}{1.599287in}}%
\pgfpathlineto{\pgfqpoint{1.297906in}{1.602912in}}%
\pgfpathlineto{\pgfqpoint{1.294129in}{1.606421in}}%
\pgfpathlineto{\pgfqpoint{1.290350in}{1.609815in}}%
\pgfpathlineto{\pgfqpoint{1.286568in}{1.613090in}}%
\pgfpathlineto{\pgfqpoint{1.285767in}{1.614662in}}%
\pgfpathlineto{\pgfqpoint{1.284861in}{1.616221in}}%
\pgfpathlineto{\pgfqpoint{1.283851in}{1.617765in}}%
\pgfpathlineto{\pgfqpoint{1.282739in}{1.619293in}}%
\pgfpathclose%
\pgfusepath{fill}%
\end{pgfscope}%
\begin{pgfscope}%
\pgfpathrectangle{\pgfqpoint{0.041670in}{0.041670in}}{\pgfqpoint{2.216660in}{2.216660in}}%
\pgfusepath{clip}%
\pgfsetbuttcap%
\pgfsetroundjoin%
\definecolor{currentfill}{rgb}{0.134692,0.658636,0.517649}%
\pgfsetfillcolor{currentfill}%
\pgfsetlinewidth{0.000000pt}%
\definecolor{currentstroke}{rgb}{0.000000,0.000000,0.000000}%
\pgfsetstrokecolor{currentstroke}%
\pgfsetdash{}{0pt}%
\pgfpathmoveto{\pgfqpoint{0.944440in}{1.274535in}}%
\pgfpathlineto{\pgfqpoint{0.941218in}{1.266070in}}%
\pgfpathlineto{\pgfqpoint{0.937997in}{1.257565in}}%
\pgfpathlineto{\pgfqpoint{0.934778in}{1.249021in}}%
\pgfpathlineto{\pgfqpoint{0.931561in}{1.240442in}}%
\pgfpathlineto{\pgfqpoint{0.926754in}{1.244366in}}%
\pgfpathlineto{\pgfqpoint{0.922208in}{1.248360in}}%
\pgfpathlineto{\pgfqpoint{0.917926in}{1.252422in}}%
\pgfpathlineto{\pgfqpoint{0.913913in}{1.256546in}}%
\pgfpathlineto{\pgfqpoint{0.917351in}{1.264897in}}%
\pgfpathlineto{\pgfqpoint{0.920791in}{1.273213in}}%
\pgfpathlineto{\pgfqpoint{0.924233in}{1.281492in}}%
\pgfpathlineto{\pgfqpoint{0.927677in}{1.289729in}}%
\pgfpathlineto{\pgfqpoint{0.931490in}{1.285838in}}%
\pgfpathlineto{\pgfqpoint{0.935558in}{1.282006in}}%
\pgfpathlineto{\pgfqpoint{0.939876in}{1.278237in}}%
\pgfpathlineto{\pgfqpoint{0.944440in}{1.274535in}}%
\pgfpathclose%
\pgfusepath{fill}%
\end{pgfscope}%
\begin{pgfscope}%
\pgfpathrectangle{\pgfqpoint{0.041670in}{0.041670in}}{\pgfqpoint{2.216660in}{2.216660in}}%
\pgfusepath{clip}%
\pgfsetbuttcap%
\pgfsetroundjoin%
\definecolor{currentfill}{rgb}{0.281477,0.755203,0.432552}%
\pgfsetfillcolor{currentfill}%
\pgfsetlinewidth{0.000000pt}%
\definecolor{currentstroke}{rgb}{0.000000,0.000000,0.000000}%
\pgfsetstrokecolor{currentstroke}%
\pgfsetdash{}{0pt}%
\pgfpathmoveto{\pgfqpoint{1.403266in}{1.400851in}}%
\pgfpathlineto{\pgfqpoint{1.406934in}{1.393564in}}%
\pgfpathlineto{\pgfqpoint{1.410598in}{1.386211in}}%
\pgfpathlineto{\pgfqpoint{1.414261in}{1.378792in}}%
\pgfpathlineto{\pgfqpoint{1.417920in}{1.371311in}}%
\pgfpathlineto{\pgfqpoint{1.415656in}{1.367675in}}%
\pgfpathlineto{\pgfqpoint{1.413153in}{1.364072in}}%
\pgfpathlineto{\pgfqpoint{1.410413in}{1.360509in}}%
\pgfpathlineto{\pgfqpoint{1.407439in}{1.356987in}}%
\pgfpathlineto{\pgfqpoint{1.403933in}{1.364705in}}%
\pgfpathlineto{\pgfqpoint{1.400425in}{1.372359in}}%
\pgfpathlineto{\pgfqpoint{1.396915in}{1.379948in}}%
\pgfpathlineto{\pgfqpoint{1.393403in}{1.387469in}}%
\pgfpathlineto{\pgfqpoint{1.396200in}{1.390759in}}%
\pgfpathlineto{\pgfqpoint{1.398778in}{1.394088in}}%
\pgfpathlineto{\pgfqpoint{1.401134in}{1.397453in}}%
\pgfpathlineto{\pgfqpoint{1.403266in}{1.400851in}}%
\pgfpathclose%
\pgfusepath{fill}%
\end{pgfscope}%
\begin{pgfscope}%
\pgfpathrectangle{\pgfqpoint{0.041670in}{0.041670in}}{\pgfqpoint{2.216660in}{2.216660in}}%
\pgfusepath{clip}%
\pgfsetbuttcap%
\pgfsetroundjoin%
\definecolor{currentfill}{rgb}{0.565498,0.842430,0.262877}%
\pgfsetfillcolor{currentfill}%
\pgfsetlinewidth{0.000000pt}%
\definecolor{currentstroke}{rgb}{0.000000,0.000000,0.000000}%
\pgfsetstrokecolor{currentstroke}%
\pgfsetdash{}{0pt}%
\pgfpathmoveto{\pgfqpoint{1.350944in}{1.526938in}}%
\pgfpathlineto{\pgfqpoint{1.354787in}{1.521464in}}%
\pgfpathlineto{\pgfqpoint{1.358628in}{1.515893in}}%
\pgfpathlineto{\pgfqpoint{1.362466in}{1.510228in}}%
\pgfpathlineto{\pgfqpoint{1.366301in}{1.504470in}}%
\pgfpathlineto{\pgfqpoint{1.366045in}{1.501668in}}%
\pgfpathlineto{\pgfqpoint{1.365602in}{1.498869in}}%
\pgfpathlineto{\pgfqpoint{1.364974in}{1.496077in}}%
\pgfpathlineto{\pgfqpoint{1.364161in}{1.493295in}}%
\pgfpathlineto{\pgfqpoint{1.360363in}{1.499293in}}%
\pgfpathlineto{\pgfqpoint{1.356563in}{1.505198in}}%
\pgfpathlineto{\pgfqpoint{1.352761in}{1.511007in}}%
\pgfpathlineto{\pgfqpoint{1.348956in}{1.516720in}}%
\pgfpathlineto{\pgfqpoint{1.349709in}{1.519264in}}%
\pgfpathlineto{\pgfqpoint{1.350291in}{1.521816in}}%
\pgfpathlineto{\pgfqpoint{1.350703in}{1.524375in}}%
\pgfpathlineto{\pgfqpoint{1.350944in}{1.526938in}}%
\pgfpathclose%
\pgfusepath{fill}%
\end{pgfscope}%
\begin{pgfscope}%
\pgfpathrectangle{\pgfqpoint{0.041670in}{0.041670in}}{\pgfqpoint{2.216660in}{2.216660in}}%
\pgfusepath{clip}%
\pgfsetbuttcap%
\pgfsetroundjoin%
\definecolor{currentfill}{rgb}{0.267004,0.004874,0.329415}%
\pgfsetfillcolor{currentfill}%
\pgfsetlinewidth{0.000000pt}%
\definecolor{currentstroke}{rgb}{0.000000,0.000000,0.000000}%
\pgfsetstrokecolor{currentstroke}%
\pgfsetdash{}{0pt}%
\pgfpathmoveto{\pgfqpoint{1.591511in}{0.656408in}}%
\pgfpathlineto{\pgfqpoint{1.594287in}{0.654181in}}%
\pgfpathlineto{\pgfqpoint{1.597069in}{0.652196in}}%
\pgfpathlineto{\pgfqpoint{1.599857in}{0.650455in}}%
\pgfpathlineto{\pgfqpoint{1.602651in}{0.648964in}}%
\pgfpathlineto{\pgfqpoint{1.589117in}{0.641890in}}%
\pgfpathlineto{\pgfqpoint{1.575136in}{0.635044in}}%
\pgfpathlineto{\pgfqpoint{1.560723in}{0.628433in}}%
\pgfpathlineto{\pgfqpoint{1.545891in}{0.622065in}}%
\pgfpathlineto{\pgfqpoint{1.543463in}{0.623740in}}%
\pgfpathlineto{\pgfqpoint{1.541041in}{0.625665in}}%
\pgfpathlineto{\pgfqpoint{1.538624in}{0.627835in}}%
\pgfpathlineto{\pgfqpoint{1.536213in}{0.630246in}}%
\pgfpathlineto{\pgfqpoint{1.550661in}{0.636438in}}%
\pgfpathlineto{\pgfqpoint{1.564702in}{0.642868in}}%
\pgfpathlineto{\pgfqpoint{1.578324in}{0.649527in}}%
\pgfpathlineto{\pgfqpoint{1.591511in}{0.656408in}}%
\pgfpathclose%
\pgfusepath{fill}%
\end{pgfscope}%
\begin{pgfscope}%
\pgfpathrectangle{\pgfqpoint{0.041670in}{0.041670in}}{\pgfqpoint{2.216660in}{2.216660in}}%
\pgfusepath{clip}%
\pgfsetbuttcap%
\pgfsetroundjoin%
\definecolor{currentfill}{rgb}{0.935904,0.898570,0.108131}%
\pgfsetfillcolor{currentfill}%
\pgfsetlinewidth{0.000000pt}%
\definecolor{currentstroke}{rgb}{0.000000,0.000000,0.000000}%
\pgfsetstrokecolor{currentstroke}%
\pgfsetdash{}{0pt}%
\pgfpathmoveto{\pgfqpoint{1.215206in}{1.660747in}}%
\pgfpathlineto{\pgfqpoint{1.217397in}{1.660068in}}%
\pgfpathlineto{\pgfqpoint{1.219586in}{1.659262in}}%
\pgfpathlineto{\pgfqpoint{1.221773in}{1.658329in}}%
\pgfpathlineto{\pgfqpoint{1.223959in}{1.657269in}}%
\pgfpathlineto{\pgfqpoint{1.225965in}{1.656608in}}%
\pgfpathlineto{\pgfqpoint{1.227926in}{1.655917in}}%
\pgfpathlineto{\pgfqpoint{1.229839in}{1.655197in}}%
\pgfpathlineto{\pgfqpoint{1.231704in}{1.654450in}}%
\pgfpathlineto{\pgfqpoint{1.229133in}{1.655650in}}%
\pgfpathlineto{\pgfqpoint{1.226560in}{1.656724in}}%
\pgfpathlineto{\pgfqpoint{1.223985in}{1.657671in}}%
\pgfpathlineto{\pgfqpoint{1.221407in}{1.658490in}}%
\pgfpathlineto{\pgfqpoint{1.219915in}{1.659088in}}%
\pgfpathlineto{\pgfqpoint{1.218383in}{1.659664in}}%
\pgfpathlineto{\pgfqpoint{1.216813in}{1.660217in}}%
\pgfpathlineto{\pgfqpoint{1.215206in}{1.660747in}}%
\pgfpathclose%
\pgfusepath{fill}%
\end{pgfscope}%
\begin{pgfscope}%
\pgfpathrectangle{\pgfqpoint{0.041670in}{0.041670in}}{\pgfqpoint{2.216660in}{2.216660in}}%
\pgfusepath{clip}%
\pgfsetbuttcap%
\pgfsetroundjoin%
\definecolor{currentfill}{rgb}{0.231674,0.318106,0.544834}%
\pgfsetfillcolor{currentfill}%
\pgfsetlinewidth{0.000000pt}%
\definecolor{currentstroke}{rgb}{0.000000,0.000000,0.000000}%
\pgfsetstrokecolor{currentstroke}%
\pgfsetdash{}{0pt}%
\pgfpathmoveto{\pgfqpoint{1.472726in}{0.925705in}}%
\pgfpathlineto{\pgfqpoint{1.475393in}{0.917121in}}%
\pgfpathlineto{\pgfqpoint{1.478061in}{0.908600in}}%
\pgfpathlineto{\pgfqpoint{1.480728in}{0.900146in}}%
\pgfpathlineto{\pgfqpoint{1.483396in}{0.891763in}}%
\pgfpathlineto{\pgfqpoint{1.473602in}{0.886851in}}%
\pgfpathlineto{\pgfqpoint{1.463494in}{0.882100in}}%
\pgfpathlineto{\pgfqpoint{1.453084in}{0.877515in}}%
\pgfpathlineto{\pgfqpoint{1.442382in}{0.873101in}}%
\pgfpathlineto{\pgfqpoint{1.440069in}{0.881671in}}%
\pgfpathlineto{\pgfqpoint{1.437756in}{0.890311in}}%
\pgfpathlineto{\pgfqpoint{1.435444in}{0.899017in}}%
\pgfpathlineto{\pgfqpoint{1.433131in}{0.907786in}}%
\pgfpathlineto{\pgfqpoint{1.443462in}{0.912024in}}%
\pgfpathlineto{\pgfqpoint{1.453511in}{0.916427in}}%
\pgfpathlineto{\pgfqpoint{1.463269in}{0.920989in}}%
\pgfpathlineto{\pgfqpoint{1.472726in}{0.925705in}}%
\pgfpathclose%
\pgfusepath{fill}%
\end{pgfscope}%
\begin{pgfscope}%
\pgfpathrectangle{\pgfqpoint{0.041670in}{0.041670in}}{\pgfqpoint{2.216660in}{2.216660in}}%
\pgfusepath{clip}%
\pgfsetbuttcap%
\pgfsetroundjoin%
\definecolor{currentfill}{rgb}{0.122606,0.585371,0.546557}%
\pgfsetfillcolor{currentfill}%
\pgfsetlinewidth{0.000000pt}%
\definecolor{currentstroke}{rgb}{0.000000,0.000000,0.000000}%
\pgfsetstrokecolor{currentstroke}%
\pgfsetdash{}{0pt}%
\pgfpathmoveto{\pgfqpoint{1.445700in}{1.209499in}}%
\pgfpathlineto{\pgfqpoint{1.448962in}{1.200829in}}%
\pgfpathlineto{\pgfqpoint{1.452222in}{1.192138in}}%
\pgfpathlineto{\pgfqpoint{1.455480in}{1.183429in}}%
\pgfpathlineto{\pgfqpoint{1.458737in}{1.174706in}}%
\pgfpathlineto{\pgfqpoint{1.453418in}{1.170342in}}%
\pgfpathlineto{\pgfqpoint{1.447816in}{1.166061in}}%
\pgfpathlineto{\pgfqpoint{1.441936in}{1.161869in}}%
\pgfpathlineto{\pgfqpoint{1.435785in}{1.157769in}}%
\pgfpathlineto{\pgfqpoint{1.432789in}{1.166712in}}%
\pgfpathlineto{\pgfqpoint{1.429792in}{1.175639in}}%
\pgfpathlineto{\pgfqpoint{1.426793in}{1.184549in}}%
\pgfpathlineto{\pgfqpoint{1.423793in}{1.193437in}}%
\pgfpathlineto{\pgfqpoint{1.429663in}{1.197325in}}%
\pgfpathlineto{\pgfqpoint{1.435275in}{1.201301in}}%
\pgfpathlineto{\pgfqpoint{1.440622in}{1.205360in}}%
\pgfpathlineto{\pgfqpoint{1.445700in}{1.209499in}}%
\pgfpathclose%
\pgfusepath{fill}%
\end{pgfscope}%
\begin{pgfscope}%
\pgfpathrectangle{\pgfqpoint{0.041670in}{0.041670in}}{\pgfqpoint{2.216660in}{2.216660in}}%
\pgfusepath{clip}%
\pgfsetbuttcap%
\pgfsetroundjoin%
\definecolor{currentfill}{rgb}{0.762373,0.876424,0.137064}%
\pgfsetfillcolor{currentfill}%
\pgfsetlinewidth{0.000000pt}%
\definecolor{currentstroke}{rgb}{0.000000,0.000000,0.000000}%
\pgfsetstrokecolor{currentstroke}%
\pgfsetdash{}{0pt}%
\pgfpathmoveto{\pgfqpoint{1.301681in}{1.599287in}}%
\pgfpathlineto{\pgfqpoint{1.305453in}{1.595549in}}%
\pgfpathlineto{\pgfqpoint{1.309223in}{1.591697in}}%
\pgfpathlineto{\pgfqpoint{1.312989in}{1.587734in}}%
\pgfpathlineto{\pgfqpoint{1.316754in}{1.583661in}}%
\pgfpathlineto{\pgfqpoint{1.317655in}{1.581621in}}%
\pgfpathlineto{\pgfqpoint{1.318420in}{1.579567in}}%
\pgfpathlineto{\pgfqpoint{1.319048in}{1.577502in}}%
\pgfpathlineto{\pgfqpoint{1.319538in}{1.575428in}}%
\pgfpathlineto{\pgfqpoint{1.315692in}{1.579733in}}%
\pgfpathlineto{\pgfqpoint{1.311844in}{1.583927in}}%
\pgfpathlineto{\pgfqpoint{1.307994in}{1.588010in}}%
\pgfpathlineto{\pgfqpoint{1.304141in}{1.591980in}}%
\pgfpathlineto{\pgfqpoint{1.303710in}{1.593820in}}%
\pgfpathlineto{\pgfqpoint{1.303156in}{1.595653in}}%
\pgfpathlineto{\pgfqpoint{1.302479in}{1.597476in}}%
\pgfpathlineto{\pgfqpoint{1.301681in}{1.599287in}}%
\pgfpathclose%
\pgfusepath{fill}%
\end{pgfscope}%
\begin{pgfscope}%
\pgfpathrectangle{\pgfqpoint{0.041670in}{0.041670in}}{\pgfqpoint{2.216660in}{2.216660in}}%
\pgfusepath{clip}%
\pgfsetbuttcap%
\pgfsetroundjoin%
\definecolor{currentfill}{rgb}{0.855810,0.888601,0.097452}%
\pgfsetfillcolor{currentfill}%
\pgfsetlinewidth{0.000000pt}%
\definecolor{currentstroke}{rgb}{0.000000,0.000000,0.000000}%
\pgfsetstrokecolor{currentstroke}%
\pgfsetdash{}{0pt}%
\pgfpathmoveto{\pgfqpoint{1.263476in}{1.635385in}}%
\pgfpathlineto{\pgfqpoint{1.266932in}{1.633023in}}%
\pgfpathlineto{\pgfqpoint{1.270385in}{1.630540in}}%
\pgfpathlineto{\pgfqpoint{1.273837in}{1.627937in}}%
\pgfpathlineto{\pgfqpoint{1.277285in}{1.625215in}}%
\pgfpathlineto{\pgfqpoint{1.278796in}{1.623766in}}%
\pgfpathlineto{\pgfqpoint{1.280210in}{1.622295in}}%
\pgfpathlineto{\pgfqpoint{1.281524in}{1.620804in}}%
\pgfpathlineto{\pgfqpoint{1.282739in}{1.619293in}}%
\pgfpathlineto{\pgfqpoint{1.279093in}{1.622228in}}%
\pgfpathlineto{\pgfqpoint{1.275446in}{1.625043in}}%
\pgfpathlineto{\pgfqpoint{1.271796in}{1.627738in}}%
\pgfpathlineto{\pgfqpoint{1.268144in}{1.630311in}}%
\pgfpathlineto{\pgfqpoint{1.267105in}{1.631605in}}%
\pgfpathlineto{\pgfqpoint{1.265980in}{1.632883in}}%
\pgfpathlineto{\pgfqpoint{1.264770in}{1.634143in}}%
\pgfpathlineto{\pgfqpoint{1.263476in}{1.635385in}}%
\pgfpathclose%
\pgfusepath{fill}%
\end{pgfscope}%
\begin{pgfscope}%
\pgfpathrectangle{\pgfqpoint{0.041670in}{0.041670in}}{\pgfqpoint{2.216660in}{2.216660in}}%
\pgfusepath{clip}%
\pgfsetbuttcap%
\pgfsetroundjoin%
\definecolor{currentfill}{rgb}{0.268510,0.009605,0.335427}%
\pgfsetfillcolor{currentfill}%
\pgfsetlinewidth{0.000000pt}%
\definecolor{currentstroke}{rgb}{0.000000,0.000000,0.000000}%
\pgfsetstrokecolor{currentstroke}%
\pgfsetdash{}{0pt}%
\pgfpathmoveto{\pgfqpoint{0.846117in}{0.637053in}}%
\pgfpathlineto{\pgfqpoint{0.843811in}{0.633690in}}%
\pgfpathlineto{\pgfqpoint{0.841501in}{0.630547in}}%
\pgfpathlineto{\pgfqpoint{0.839187in}{0.627632in}}%
\pgfpathlineto{\pgfqpoint{0.836868in}{0.624947in}}%
\pgfpathlineto{\pgfqpoint{0.822072in}{0.630922in}}%
\pgfpathlineto{\pgfqpoint{0.807669in}{0.637141in}}%
\pgfpathlineto{\pgfqpoint{0.793673in}{0.643597in}}%
\pgfpathlineto{\pgfqpoint{0.780099in}{0.650281in}}%
\pgfpathlineto{\pgfqpoint{0.782793in}{0.652786in}}%
\pgfpathlineto{\pgfqpoint{0.785481in}{0.655521in}}%
\pgfpathlineto{\pgfqpoint{0.788164in}{0.658483in}}%
\pgfpathlineto{\pgfqpoint{0.790843in}{0.661666in}}%
\pgfpathlineto{\pgfqpoint{0.804061in}{0.655172in}}%
\pgfpathlineto{\pgfqpoint{0.817689in}{0.648900in}}%
\pgfpathlineto{\pgfqpoint{0.831712in}{0.642858in}}%
\pgfpathlineto{\pgfqpoint{0.846117in}{0.637053in}}%
\pgfpathclose%
\pgfusepath{fill}%
\end{pgfscope}%
\begin{pgfscope}%
\pgfpathrectangle{\pgfqpoint{0.041670in}{0.041670in}}{\pgfqpoint{2.216660in}{2.216660in}}%
\pgfusepath{clip}%
\pgfsetbuttcap%
\pgfsetroundjoin%
\definecolor{currentfill}{rgb}{0.935904,0.898570,0.108131}%
\pgfsetfillcolor{currentfill}%
\pgfsetlinewidth{0.000000pt}%
\definecolor{currentstroke}{rgb}{0.000000,0.000000,0.000000}%
\pgfsetstrokecolor{currentstroke}%
\pgfsetdash{}{0pt}%
\pgfpathmoveto{\pgfqpoint{1.137210in}{1.657940in}}%
\pgfpathlineto{\pgfqpoint{1.134552in}{1.657087in}}%
\pgfpathlineto{\pgfqpoint{1.131896in}{1.656106in}}%
\pgfpathlineto{\pgfqpoint{1.129243in}{1.654998in}}%
\pgfpathlineto{\pgfqpoint{1.126591in}{1.653763in}}%
\pgfpathlineto{\pgfqpoint{1.128410in}{1.654535in}}%
\pgfpathlineto{\pgfqpoint{1.130280in}{1.655279in}}%
\pgfpathlineto{\pgfqpoint{1.132199in}{1.655995in}}%
\pgfpathlineto{\pgfqpoint{1.134165in}{1.656683in}}%
\pgfpathlineto{\pgfqpoint{1.136440in}{1.657771in}}%
\pgfpathlineto{\pgfqpoint{1.138716in}{1.658734in}}%
\pgfpathlineto{\pgfqpoint{1.140994in}{1.659569in}}%
\pgfpathlineto{\pgfqpoint{1.143274in}{1.660277in}}%
\pgfpathlineto{\pgfqpoint{1.141700in}{1.659726in}}%
\pgfpathlineto{\pgfqpoint{1.140163in}{1.659153in}}%
\pgfpathlineto{\pgfqpoint{1.138666in}{1.658557in}}%
\pgfpathlineto{\pgfqpoint{1.137210in}{1.657940in}}%
\pgfpathclose%
\pgfusepath{fill}%
\end{pgfscope}%
\begin{pgfscope}%
\pgfpathrectangle{\pgfqpoint{0.041670in}{0.041670in}}{\pgfqpoint{2.216660in}{2.216660in}}%
\pgfusepath{clip}%
\pgfsetbuttcap%
\pgfsetroundjoin%
\definecolor{currentfill}{rgb}{0.163625,0.471133,0.558148}%
\pgfsetfillcolor{currentfill}%
\pgfsetlinewidth{0.000000pt}%
\definecolor{currentstroke}{rgb}{0.000000,0.000000,0.000000}%
\pgfsetstrokecolor{currentstroke}%
\pgfsetdash{}{0pt}%
\pgfpathmoveto{\pgfqpoint{1.459704in}{1.086008in}}%
\pgfpathlineto{\pgfqpoint{1.462688in}{1.077053in}}%
\pgfpathlineto{\pgfqpoint{1.465671in}{1.068111in}}%
\pgfpathlineto{\pgfqpoint{1.468654in}{1.059185in}}%
\pgfpathlineto{\pgfqpoint{1.471635in}{1.050278in}}%
\pgfpathlineto{\pgfqpoint{1.464341in}{1.045644in}}%
\pgfpathlineto{\pgfqpoint{1.456749in}{1.041126in}}%
\pgfpathlineto{\pgfqpoint{1.448866in}{1.036731in}}%
\pgfpathlineto{\pgfqpoint{1.440701in}{1.032462in}}%
\pgfpathlineto{\pgfqpoint{1.438029in}{1.041573in}}%
\pgfpathlineto{\pgfqpoint{1.435356in}{1.050704in}}%
\pgfpathlineto{\pgfqpoint{1.432683in}{1.059850in}}%
\pgfpathlineto{\pgfqpoint{1.430008in}{1.069009in}}%
\pgfpathlineto{\pgfqpoint{1.437846in}{1.073082in}}%
\pgfpathlineto{\pgfqpoint{1.445412in}{1.077275in}}%
\pgfpathlineto{\pgfqpoint{1.452701in}{1.081586in}}%
\pgfpathlineto{\pgfqpoint{1.459704in}{1.086008in}}%
\pgfpathclose%
\pgfusepath{fill}%
\end{pgfscope}%
\begin{pgfscope}%
\pgfpathrectangle{\pgfqpoint{0.041670in}{0.041670in}}{\pgfqpoint{2.216660in}{2.216660in}}%
\pgfusepath{clip}%
\pgfsetbuttcap%
\pgfsetroundjoin%
\definecolor{currentfill}{rgb}{0.248629,0.278775,0.534556}%
\pgfsetfillcolor{currentfill}%
\pgfsetlinewidth{0.000000pt}%
\definecolor{currentstroke}{rgb}{0.000000,0.000000,0.000000}%
\pgfsetstrokecolor{currentstroke}%
\pgfsetdash{}{0pt}%
\pgfpathmoveto{\pgfqpoint{0.927277in}{0.869325in}}%
\pgfpathlineto{\pgfqpoint{0.925048in}{0.860791in}}%
\pgfpathlineto{\pgfqpoint{0.922819in}{0.852333in}}%
\pgfpathlineto{\pgfqpoint{0.920589in}{0.843956in}}%
\pgfpathlineto{\pgfqpoint{0.918358in}{0.835663in}}%
\pgfpathlineto{\pgfqpoint{0.907025in}{0.840091in}}%
\pgfpathlineto{\pgfqpoint{0.895984in}{0.844702in}}%
\pgfpathlineto{\pgfqpoint{0.885248in}{0.849490in}}%
\pgfpathlineto{\pgfqpoint{0.874827in}{0.854449in}}%
\pgfpathlineto{\pgfqpoint{0.877422in}{0.862562in}}%
\pgfpathlineto{\pgfqpoint{0.880017in}{0.870758in}}%
\pgfpathlineto{\pgfqpoint{0.882611in}{0.879035in}}%
\pgfpathlineto{\pgfqpoint{0.885204in}{0.887389in}}%
\pgfpathlineto{\pgfqpoint{0.895277in}{0.882620in}}%
\pgfpathlineto{\pgfqpoint{0.905654in}{0.878016in}}%
\pgfpathlineto{\pgfqpoint{0.916324in}{0.873583in}}%
\pgfpathlineto{\pgfqpoint{0.927277in}{0.869325in}}%
\pgfpathclose%
\pgfusepath{fill}%
\end{pgfscope}%
\begin{pgfscope}%
\pgfpathrectangle{\pgfqpoint{0.041670in}{0.041670in}}{\pgfqpoint{2.216660in}{2.216660in}}%
\pgfusepath{clip}%
\pgfsetbuttcap%
\pgfsetroundjoin%
\definecolor{currentfill}{rgb}{0.272594,0.025563,0.353093}%
\pgfsetfillcolor{currentfill}%
\pgfsetlinewidth{0.000000pt}%
\definecolor{currentstroke}{rgb}{0.000000,0.000000,0.000000}%
\pgfsetstrokecolor{currentstroke}%
\pgfsetdash{}{0pt}%
\pgfpathmoveto{\pgfqpoint{0.747282in}{0.640015in}}%
\pgfpathlineto{\pgfqpoint{0.744500in}{0.640964in}}%
\pgfpathlineto{\pgfqpoint{0.741709in}{0.642216in}}%
\pgfpathlineto{\pgfqpoint{0.738910in}{0.643777in}}%
\pgfpathlineto{\pgfqpoint{0.736102in}{0.645651in}}%
\pgfpathlineto{\pgfqpoint{0.721565in}{0.653320in}}%
\pgfpathlineto{\pgfqpoint{0.707530in}{0.661223in}}%
\pgfpathlineto{\pgfqpoint{0.694011in}{0.669351in}}%
\pgfpathlineto{\pgfqpoint{0.681021in}{0.677695in}}%
\pgfpathlineto{\pgfqpoint{0.684164in}{0.675627in}}%
\pgfpathlineto{\pgfqpoint{0.687297in}{0.673872in}}%
\pgfpathlineto{\pgfqpoint{0.690421in}{0.672424in}}%
\pgfpathlineto{\pgfqpoint{0.693536in}{0.671279in}}%
\pgfpathlineto{\pgfqpoint{0.706213in}{0.663137in}}%
\pgfpathlineto{\pgfqpoint{0.719405in}{0.655206in}}%
\pgfpathlineto{\pgfqpoint{0.733100in}{0.647496in}}%
\pgfpathlineto{\pgfqpoint{0.747282in}{0.640015in}}%
\pgfpathclose%
\pgfusepath{fill}%
\end{pgfscope}%
\begin{pgfscope}%
\pgfpathrectangle{\pgfqpoint{0.041670in}{0.041670in}}{\pgfqpoint{2.216660in}{2.216660in}}%
\pgfusepath{clip}%
\pgfsetbuttcap%
\pgfsetroundjoin%
\definecolor{currentfill}{rgb}{0.814576,0.883393,0.110347}%
\pgfsetfillcolor{currentfill}%
\pgfsetlinewidth{0.000000pt}%
\definecolor{currentstroke}{rgb}{0.000000,0.000000,0.000000}%
\pgfsetstrokecolor{currentstroke}%
\pgfsetdash{}{0pt}%
\pgfpathmoveto{\pgfqpoint{1.072719in}{1.611684in}}%
\pgfpathlineto{\pgfqpoint{1.068914in}{1.608357in}}%
\pgfpathlineto{\pgfqpoint{1.065111in}{1.604913in}}%
\pgfpathlineto{\pgfqpoint{1.061311in}{1.601353in}}%
\pgfpathlineto{\pgfqpoint{1.057513in}{1.597677in}}%
\pgfpathlineto{\pgfqpoint{1.058325in}{1.599488in}}%
\pgfpathlineto{\pgfqpoint{1.059257in}{1.601284in}}%
\pgfpathlineto{\pgfqpoint{1.060308in}{1.603066in}}%
\pgfpathlineto{\pgfqpoint{1.061478in}{1.604830in}}%
\pgfpathlineto{\pgfqpoint{1.065149in}{1.608280in}}%
\pgfpathlineto{\pgfqpoint{1.068823in}{1.611615in}}%
\pgfpathlineto{\pgfqpoint{1.072499in}{1.614834in}}%
\pgfpathlineto{\pgfqpoint{1.076177in}{1.617936in}}%
\pgfpathlineto{\pgfqpoint{1.075156in}{1.616393in}}%
\pgfpathlineto{\pgfqpoint{1.074239in}{1.614836in}}%
\pgfpathlineto{\pgfqpoint{1.073426in}{1.613266in}}%
\pgfpathlineto{\pgfqpoint{1.072719in}{1.611684in}}%
\pgfpathclose%
\pgfusepath{fill}%
\end{pgfscope}%
\begin{pgfscope}%
\pgfpathrectangle{\pgfqpoint{0.041670in}{0.041670in}}{\pgfqpoint{2.216660in}{2.216660in}}%
\pgfusepath{clip}%
\pgfsetbuttcap%
\pgfsetroundjoin%
\definecolor{currentfill}{rgb}{0.855810,0.888601,0.097452}%
\pgfsetfillcolor{currentfill}%
\pgfsetlinewidth{0.000000pt}%
\definecolor{currentstroke}{rgb}{0.000000,0.000000,0.000000}%
\pgfsetstrokecolor{currentstroke}%
\pgfsetdash{}{0pt}%
\pgfpathmoveto{\pgfqpoint{1.090916in}{1.629149in}}%
\pgfpathlineto{\pgfqpoint{1.087228in}{1.626526in}}%
\pgfpathlineto{\pgfqpoint{1.083542in}{1.623783in}}%
\pgfpathlineto{\pgfqpoint{1.079858in}{1.620919in}}%
\pgfpathlineto{\pgfqpoint{1.076177in}{1.617936in}}%
\pgfpathlineto{\pgfqpoint{1.077301in}{1.619462in}}%
\pgfpathlineto{\pgfqpoint{1.078526in}{1.620970in}}%
\pgfpathlineto{\pgfqpoint{1.079852in}{1.622460in}}%
\pgfpathlineto{\pgfqpoint{1.081277in}{1.623928in}}%
\pgfpathlineto{\pgfqpoint{1.084774in}{1.626697in}}%
\pgfpathlineto{\pgfqpoint{1.088273in}{1.629346in}}%
\pgfpathlineto{\pgfqpoint{1.091775in}{1.631874in}}%
\pgfpathlineto{\pgfqpoint{1.095280in}{1.634282in}}%
\pgfpathlineto{\pgfqpoint{1.094060in}{1.633024in}}%
\pgfpathlineto{\pgfqpoint{1.092926in}{1.631748in}}%
\pgfpathlineto{\pgfqpoint{1.091877in}{1.630456in}}%
\pgfpathlineto{\pgfqpoint{1.090916in}{1.629149in}}%
\pgfpathclose%
\pgfusepath{fill}%
\end{pgfscope}%
\begin{pgfscope}%
\pgfpathrectangle{\pgfqpoint{0.041670in}{0.041670in}}{\pgfqpoint{2.216660in}{2.216660in}}%
\pgfusepath{clip}%
\pgfsetbuttcap%
\pgfsetroundjoin%
\definecolor{currentfill}{rgb}{0.412913,0.803041,0.357269}%
\pgfsetfillcolor{currentfill}%
\pgfsetlinewidth{0.000000pt}%
\definecolor{currentstroke}{rgb}{0.000000,0.000000,0.000000}%
\pgfsetstrokecolor{currentstroke}%
\pgfsetdash{}{0pt}%
\pgfpathmoveto{\pgfqpoint{0.987719in}{1.453879in}}%
\pgfpathlineto{\pgfqpoint{0.984062in}{1.447150in}}%
\pgfpathlineto{\pgfqpoint{0.980407in}{1.440338in}}%
\pgfpathlineto{\pgfqpoint{0.976754in}{1.433445in}}%
\pgfpathlineto{\pgfqpoint{0.973104in}{1.426473in}}%
\pgfpathlineto{\pgfqpoint{0.971129in}{1.429636in}}%
\pgfpathlineto{\pgfqpoint{0.969364in}{1.432827in}}%
\pgfpathlineto{\pgfqpoint{0.967812in}{1.436041in}}%
\pgfpathlineto{\pgfqpoint{0.966474in}{1.439276in}}%
\pgfpathlineto{\pgfqpoint{0.970234in}{1.446010in}}%
\pgfpathlineto{\pgfqpoint{0.973996in}{1.452665in}}%
\pgfpathlineto{\pgfqpoint{0.977762in}{1.459239in}}%
\pgfpathlineto{\pgfqpoint{0.981530in}{1.465730in}}%
\pgfpathlineto{\pgfqpoint{0.982781in}{1.462736in}}%
\pgfpathlineto{\pgfqpoint{0.984230in}{1.459760in}}%
\pgfpathlineto{\pgfqpoint{0.985877in}{1.456807in}}%
\pgfpathlineto{\pgfqpoint{0.987719in}{1.453879in}}%
\pgfpathclose%
\pgfusepath{fill}%
\end{pgfscope}%
\begin{pgfscope}%
\pgfpathrectangle{\pgfqpoint{0.041670in}{0.041670in}}{\pgfqpoint{2.216660in}{2.216660in}}%
\pgfusepath{clip}%
\pgfsetbuttcap%
\pgfsetroundjoin%
\definecolor{currentfill}{rgb}{0.762373,0.876424,0.137064}%
\pgfsetfillcolor{currentfill}%
\pgfsetlinewidth{0.000000pt}%
\definecolor{currentstroke}{rgb}{0.000000,0.000000,0.000000}%
\pgfsetstrokecolor{currentstroke}%
\pgfsetdash{}{0pt}%
\pgfpathmoveto{\pgfqpoint{1.055489in}{1.590339in}}%
\pgfpathlineto{\pgfqpoint{1.051627in}{1.586317in}}%
\pgfpathlineto{\pgfqpoint{1.047766in}{1.582182in}}%
\pgfpathlineto{\pgfqpoint{1.043909in}{1.577936in}}%
\pgfpathlineto{\pgfqpoint{1.040053in}{1.573579in}}%
\pgfpathlineto{\pgfqpoint{1.040420in}{1.575659in}}%
\pgfpathlineto{\pgfqpoint{1.040925in}{1.577732in}}%
\pgfpathlineto{\pgfqpoint{1.041568in}{1.579795in}}%
\pgfpathlineto{\pgfqpoint{1.042348in}{1.581848in}}%
\pgfpathlineto{\pgfqpoint{1.046135in}{1.585972in}}%
\pgfpathlineto{\pgfqpoint{1.049925in}{1.589986in}}%
\pgfpathlineto{\pgfqpoint{1.053718in}{1.593888in}}%
\pgfpathlineto{\pgfqpoint{1.057513in}{1.597677in}}%
\pgfpathlineto{\pgfqpoint{1.056823in}{1.595856in}}%
\pgfpathlineto{\pgfqpoint{1.056255in}{1.594024in}}%
\pgfpathlineto{\pgfqpoint{1.055811in}{1.592184in}}%
\pgfpathlineto{\pgfqpoint{1.055489in}{1.590339in}}%
\pgfpathclose%
\pgfusepath{fill}%
\end{pgfscope}%
\begin{pgfscope}%
\pgfpathrectangle{\pgfqpoint{0.041670in}{0.041670in}}{\pgfqpoint{2.216660in}{2.216660in}}%
\pgfusepath{clip}%
\pgfsetbuttcap%
\pgfsetroundjoin%
\definecolor{currentfill}{rgb}{0.166383,0.690856,0.496502}%
\pgfsetfillcolor{currentfill}%
\pgfsetlinewidth{0.000000pt}%
\definecolor{currentstroke}{rgb}{0.000000,0.000000,0.000000}%
\pgfsetstrokecolor{currentstroke}%
\pgfsetdash{}{0pt}%
\pgfpathmoveto{\pgfqpoint{1.421440in}{1.325528in}}%
\pgfpathlineto{\pgfqpoint{1.424935in}{1.317527in}}%
\pgfpathlineto{\pgfqpoint{1.428427in}{1.309476in}}%
\pgfpathlineto{\pgfqpoint{1.431917in}{1.301378in}}%
\pgfpathlineto{\pgfqpoint{1.435405in}{1.293235in}}%
\pgfpathlineto{\pgfqpoint{1.431822in}{1.289294in}}%
\pgfpathlineto{\pgfqpoint{1.427980in}{1.285409in}}%
\pgfpathlineto{\pgfqpoint{1.423885in}{1.281584in}}%
\pgfpathlineto{\pgfqpoint{1.419539in}{1.277822in}}%
\pgfpathlineto{\pgfqpoint{1.416260in}{1.286194in}}%
\pgfpathlineto{\pgfqpoint{1.412980in}{1.294521in}}%
\pgfpathlineto{\pgfqpoint{1.409698in}{1.302800in}}%
\pgfpathlineto{\pgfqpoint{1.406413in}{1.311028in}}%
\pgfpathlineto{\pgfqpoint{1.410528in}{1.314567in}}%
\pgfpathlineto{\pgfqpoint{1.414406in}{1.318166in}}%
\pgfpathlineto{\pgfqpoint{1.418045in}{1.321820in}}%
\pgfpathlineto{\pgfqpoint{1.421440in}{1.325528in}}%
\pgfpathclose%
\pgfusepath{fill}%
\end{pgfscope}%
\begin{pgfscope}%
\pgfpathrectangle{\pgfqpoint{0.041670in}{0.041670in}}{\pgfqpoint{2.216660in}{2.216660in}}%
\pgfusepath{clip}%
\pgfsetbuttcap%
\pgfsetroundjoin%
\definecolor{currentfill}{rgb}{0.565498,0.842430,0.262877}%
\pgfsetfillcolor{currentfill}%
\pgfsetlinewidth{0.000000pt}%
\definecolor{currentstroke}{rgb}{0.000000,0.000000,0.000000}%
\pgfsetstrokecolor{currentstroke}%
\pgfsetdash{}{0pt}%
\pgfpathmoveto{\pgfqpoint{1.011764in}{1.514469in}}%
\pgfpathlineto{\pgfqpoint{1.007977in}{1.508703in}}%
\pgfpathlineto{\pgfqpoint{1.004191in}{1.502841in}}%
\pgfpathlineto{\pgfqpoint{1.000408in}{1.496883in}}%
\pgfpathlineto{\pgfqpoint{0.996627in}{1.490833in}}%
\pgfpathlineto{\pgfqpoint{0.995650in}{1.493604in}}%
\pgfpathlineto{\pgfqpoint{0.994857in}{1.496387in}}%
\pgfpathlineto{\pgfqpoint{0.994249in}{1.499180in}}%
\pgfpathlineto{\pgfqpoint{0.993827in}{1.501979in}}%
\pgfpathlineto{\pgfqpoint{0.997659in}{1.507790in}}%
\pgfpathlineto{\pgfqpoint{1.001494in}{1.513509in}}%
\pgfpathlineto{\pgfqpoint{1.005332in}{1.519132in}}%
\pgfpathlineto{\pgfqpoint{1.009172in}{1.524660in}}%
\pgfpathlineto{\pgfqpoint{1.009565in}{1.522100in}}%
\pgfpathlineto{\pgfqpoint{1.010128in}{1.519547in}}%
\pgfpathlineto{\pgfqpoint{1.010861in}{1.517002in}}%
\pgfpathlineto{\pgfqpoint{1.011764in}{1.514469in}}%
\pgfpathclose%
\pgfusepath{fill}%
\end{pgfscope}%
\begin{pgfscope}%
\pgfpathrectangle{\pgfqpoint{0.041670in}{0.041670in}}{\pgfqpoint{2.216660in}{2.216660in}}%
\pgfusepath{clip}%
\pgfsetbuttcap%
\pgfsetroundjoin%
\definecolor{currentfill}{rgb}{0.699415,0.867117,0.175971}%
\pgfsetfillcolor{currentfill}%
\pgfsetlinewidth{0.000000pt}%
\definecolor{currentstroke}{rgb}{0.000000,0.000000,0.000000}%
\pgfsetstrokecolor{currentstroke}%
\pgfsetdash{}{0pt}%
\pgfpathmoveto{\pgfqpoint{1.319538in}{1.575428in}}%
\pgfpathlineto{\pgfqpoint{1.323380in}{1.571014in}}%
\pgfpathlineto{\pgfqpoint{1.327221in}{1.566492in}}%
\pgfpathlineto{\pgfqpoint{1.331058in}{1.561864in}}%
\pgfpathlineto{\pgfqpoint{1.334893in}{1.557131in}}%
\pgfpathlineto{\pgfqpoint{1.335288in}{1.554815in}}%
\pgfpathlineto{\pgfqpoint{1.335528in}{1.552493in}}%
\pgfpathlineto{\pgfqpoint{1.335614in}{1.550168in}}%
\pgfpathlineto{\pgfqpoint{1.335544in}{1.547842in}}%
\pgfpathlineto{\pgfqpoint{1.331688in}{1.552812in}}%
\pgfpathlineto{\pgfqpoint{1.327829in}{1.557677in}}%
\pgfpathlineto{\pgfqpoint{1.323968in}{1.562435in}}%
\pgfpathlineto{\pgfqpoint{1.320104in}{1.567084in}}%
\pgfpathlineto{\pgfqpoint{1.320172in}{1.569173in}}%
\pgfpathlineto{\pgfqpoint{1.320100in}{1.571262in}}%
\pgfpathlineto{\pgfqpoint{1.319888in}{1.573347in}}%
\pgfpathlineto{\pgfqpoint{1.319538in}{1.575428in}}%
\pgfpathclose%
\pgfusepath{fill}%
\end{pgfscope}%
\begin{pgfscope}%
\pgfpathrectangle{\pgfqpoint{0.041670in}{0.041670in}}{\pgfqpoint{2.216660in}{2.216660in}}%
\pgfusepath{clip}%
\pgfsetbuttcap%
\pgfsetroundjoin%
\definecolor{currentfill}{rgb}{0.896320,0.893616,0.096335}%
\pgfsetfillcolor{currentfill}%
\pgfsetlinewidth{0.000000pt}%
\definecolor{currentstroke}{rgb}{0.000000,0.000000,0.000000}%
\pgfsetstrokecolor{currentstroke}%
\pgfsetdash{}{0pt}%
\pgfpathmoveto{\pgfqpoint{1.244642in}{1.647562in}}%
\pgfpathlineto{\pgfqpoint{1.247857in}{1.645893in}}%
\pgfpathlineto{\pgfqpoint{1.251070in}{1.644098in}}%
\pgfpathlineto{\pgfqpoint{1.254281in}{1.642180in}}%
\pgfpathlineto{\pgfqpoint{1.257489in}{1.640139in}}%
\pgfpathlineto{\pgfqpoint{1.259104in}{1.638985in}}%
\pgfpathlineto{\pgfqpoint{1.260642in}{1.637807in}}%
\pgfpathlineto{\pgfqpoint{1.262099in}{1.636606in}}%
\pgfpathlineto{\pgfqpoint{1.263476in}{1.635385in}}%
\pgfpathlineto{\pgfqpoint{1.260017in}{1.637624in}}%
\pgfpathlineto{\pgfqpoint{1.256556in}{1.639740in}}%
\pgfpathlineto{\pgfqpoint{1.253093in}{1.641732in}}%
\pgfpathlineto{\pgfqpoint{1.249628in}{1.643600in}}%
\pgfpathlineto{\pgfqpoint{1.248482in}{1.644618in}}%
\pgfpathlineto{\pgfqpoint{1.247268in}{1.645618in}}%
\pgfpathlineto{\pgfqpoint{1.245988in}{1.646600in}}%
\pgfpathlineto{\pgfqpoint{1.244642in}{1.647562in}}%
\pgfpathclose%
\pgfusepath{fill}%
\end{pgfscope}%
\begin{pgfscope}%
\pgfpathrectangle{\pgfqpoint{0.041670in}{0.041670in}}{\pgfqpoint{2.216660in}{2.216660in}}%
\pgfusepath{clip}%
\pgfsetbuttcap%
\pgfsetroundjoin%
\definecolor{currentfill}{rgb}{0.935904,0.898570,0.108131}%
\pgfsetfillcolor{currentfill}%
\pgfsetlinewidth{0.000000pt}%
\definecolor{currentstroke}{rgb}{0.000000,0.000000,0.000000}%
\pgfsetstrokecolor{currentstroke}%
\pgfsetdash{}{0pt}%
\pgfpathmoveto{\pgfqpoint{1.221407in}{1.658490in}}%
\pgfpathlineto{\pgfqpoint{1.223985in}{1.657671in}}%
\pgfpathlineto{\pgfqpoint{1.226560in}{1.656724in}}%
\pgfpathlineto{\pgfqpoint{1.229133in}{1.655650in}}%
\pgfpathlineto{\pgfqpoint{1.231704in}{1.654450in}}%
\pgfpathlineto{\pgfqpoint{1.233518in}{1.653676in}}%
\pgfpathlineto{\pgfqpoint{1.235279in}{1.652875in}}%
\pgfpathlineto{\pgfqpoint{1.236985in}{1.652049in}}%
\pgfpathlineto{\pgfqpoint{1.238635in}{1.651198in}}%
\pgfpathlineto{\pgfqpoint{1.235718in}{1.652560in}}%
\pgfpathlineto{\pgfqpoint{1.232799in}{1.653796in}}%
\pgfpathlineto{\pgfqpoint{1.229878in}{1.654905in}}%
\pgfpathlineto{\pgfqpoint{1.226955in}{1.655886in}}%
\pgfpathlineto{\pgfqpoint{1.225634in}{1.656567in}}%
\pgfpathlineto{\pgfqpoint{1.224269in}{1.657229in}}%
\pgfpathlineto{\pgfqpoint{1.222859in}{1.657870in}}%
\pgfpathlineto{\pgfqpoint{1.221407in}{1.658490in}}%
\pgfpathclose%
\pgfusepath{fill}%
\end{pgfscope}%
\begin{pgfscope}%
\pgfpathrectangle{\pgfqpoint{0.041670in}{0.041670in}}{\pgfqpoint{2.216660in}{2.216660in}}%
\pgfusepath{clip}%
\pgfsetbuttcap%
\pgfsetroundjoin%
\definecolor{currentfill}{rgb}{0.281477,0.755203,0.432552}%
\pgfsetfillcolor{currentfill}%
\pgfsetlinewidth{0.000000pt}%
\definecolor{currentstroke}{rgb}{0.000000,0.000000,0.000000}%
\pgfsetstrokecolor{currentstroke}%
\pgfsetdash{}{0pt}%
\pgfpathmoveto{\pgfqpoint{0.969176in}{1.384581in}}%
\pgfpathlineto{\pgfqpoint{0.965706in}{1.377009in}}%
\pgfpathlineto{\pgfqpoint{0.962238in}{1.369369in}}%
\pgfpathlineto{\pgfqpoint{0.958772in}{1.361664in}}%
\pgfpathlineto{\pgfqpoint{0.955309in}{1.353895in}}%
\pgfpathlineto{\pgfqpoint{0.952128in}{1.357376in}}%
\pgfpathlineto{\pgfqpoint{0.949180in}{1.360903in}}%
\pgfpathlineto{\pgfqpoint{0.946467in}{1.364471in}}%
\pgfpathlineto{\pgfqpoint{0.943990in}{1.368077in}}%
\pgfpathlineto{\pgfqpoint{0.947621in}{1.375612in}}%
\pgfpathlineto{\pgfqpoint{0.951254in}{1.383083in}}%
\pgfpathlineto{\pgfqpoint{0.954890in}{1.390490in}}%
\pgfpathlineto{\pgfqpoint{0.958528in}{1.397829in}}%
\pgfpathlineto{\pgfqpoint{0.960859in}{1.394460in}}%
\pgfpathlineto{\pgfqpoint{0.963412in}{1.391127in}}%
\pgfpathlineto{\pgfqpoint{0.966186in}{1.387833in}}%
\pgfpathlineto{\pgfqpoint{0.969176in}{1.384581in}}%
\pgfpathclose%
\pgfusepath{fill}%
\end{pgfscope}%
\begin{pgfscope}%
\pgfpathrectangle{\pgfqpoint{0.041670in}{0.041670in}}{\pgfqpoint{2.216660in}{2.216660in}}%
\pgfusepath{clip}%
\pgfsetbuttcap%
\pgfsetroundjoin%
\definecolor{currentfill}{rgb}{0.955300,0.901065,0.118128}%
\pgfsetfillcolor{currentfill}%
\pgfsetlinewidth{0.000000pt}%
\definecolor{currentstroke}{rgb}{0.000000,0.000000,0.000000}%
\pgfsetstrokecolor{currentstroke}%
\pgfsetdash{}{0pt}%
\pgfpathmoveto{\pgfqpoint{1.174148in}{1.665819in}}%
\pgfpathlineto{\pgfqpoint{1.173667in}{1.665961in}}%
\pgfpathlineto{\pgfqpoint{1.173186in}{1.665972in}}%
\pgfpathlineto{\pgfqpoint{1.172705in}{1.665854in}}%
\pgfpathlineto{\pgfqpoint{1.172224in}{1.665607in}}%
\pgfpathlineto{\pgfqpoint{1.174178in}{1.665706in}}%
\pgfpathlineto{\pgfqpoint{1.176137in}{1.665776in}}%
\pgfpathlineto{\pgfqpoint{1.178099in}{1.665818in}}%
\pgfpathlineto{\pgfqpoint{1.180064in}{1.665831in}}%
\pgfpathlineto{\pgfqpoint{1.180057in}{1.666064in}}%
\pgfpathlineto{\pgfqpoint{1.180050in}{1.666169in}}%
\pgfpathlineto{\pgfqpoint{1.180044in}{1.666143in}}%
\pgfpathlineto{\pgfqpoint{1.180037in}{1.665987in}}%
\pgfpathlineto{\pgfqpoint{1.178561in}{1.665977in}}%
\pgfpathlineto{\pgfqpoint{1.177087in}{1.665946in}}%
\pgfpathlineto{\pgfqpoint{1.175616in}{1.665893in}}%
\pgfpathlineto{\pgfqpoint{1.174148in}{1.665819in}}%
\pgfpathclose%
\pgfusepath{fill}%
\end{pgfscope}%
\begin{pgfscope}%
\pgfpathrectangle{\pgfqpoint{0.041670in}{0.041670in}}{\pgfqpoint{2.216660in}{2.216660in}}%
\pgfusepath{clip}%
\pgfsetbuttcap%
\pgfsetroundjoin%
\definecolor{currentfill}{rgb}{0.955300,0.901065,0.118128}%
\pgfsetfillcolor{currentfill}%
\pgfsetlinewidth{0.000000pt}%
\definecolor{currentstroke}{rgb}{0.000000,0.000000,0.000000}%
\pgfsetstrokecolor{currentstroke}%
\pgfsetdash{}{0pt}%
\pgfpathmoveto{\pgfqpoint{1.180037in}{1.665987in}}%
\pgfpathlineto{\pgfqpoint{1.180044in}{1.666143in}}%
\pgfpathlineto{\pgfqpoint{1.180050in}{1.666169in}}%
\pgfpathlineto{\pgfqpoint{1.180057in}{1.666064in}}%
\pgfpathlineto{\pgfqpoint{1.180064in}{1.665831in}}%
\pgfpathlineto{\pgfqpoint{1.182029in}{1.665815in}}%
\pgfpathlineto{\pgfqpoint{1.183991in}{1.665770in}}%
\pgfpathlineto{\pgfqpoint{1.185950in}{1.665696in}}%
\pgfpathlineto{\pgfqpoint{1.187902in}{1.665594in}}%
\pgfpathlineto{\pgfqpoint{1.187408in}{1.665842in}}%
\pgfpathlineto{\pgfqpoint{1.186914in}{1.665961in}}%
\pgfpathlineto{\pgfqpoint{1.186419in}{1.665950in}}%
\pgfpathlineto{\pgfqpoint{1.185924in}{1.665809in}}%
\pgfpathlineto{\pgfqpoint{1.184457in}{1.665886in}}%
\pgfpathlineto{\pgfqpoint{1.182986in}{1.665941in}}%
\pgfpathlineto{\pgfqpoint{1.181512in}{1.665975in}}%
\pgfpathlineto{\pgfqpoint{1.180037in}{1.665987in}}%
\pgfpathclose%
\pgfusepath{fill}%
\end{pgfscope}%
\begin{pgfscope}%
\pgfpathrectangle{\pgfqpoint{0.041670in}{0.041670in}}{\pgfqpoint{2.216660in}{2.216660in}}%
\pgfusepath{clip}%
\pgfsetbuttcap%
\pgfsetroundjoin%
\definecolor{currentfill}{rgb}{0.955300,0.901065,0.118128}%
\pgfsetfillcolor{currentfill}%
\pgfsetlinewidth{0.000000pt}%
\definecolor{currentstroke}{rgb}{0.000000,0.000000,0.000000}%
\pgfsetstrokecolor{currentstroke}%
\pgfsetdash{}{0pt}%
\pgfpathmoveto{\pgfqpoint{1.168351in}{1.665307in}}%
\pgfpathlineto{\pgfqpoint{1.167389in}{1.665407in}}%
\pgfpathlineto{\pgfqpoint{1.166427in}{1.665376in}}%
\pgfpathlineto{\pgfqpoint{1.165466in}{1.665215in}}%
\pgfpathlineto{\pgfqpoint{1.164506in}{1.664925in}}%
\pgfpathlineto{\pgfqpoint{1.166417in}{1.665138in}}%
\pgfpathlineto{\pgfqpoint{1.168342in}{1.665323in}}%
\pgfpathlineto{\pgfqpoint{1.170278in}{1.665479in}}%
\pgfpathlineto{\pgfqpoint{1.172224in}{1.665607in}}%
\pgfpathlineto{\pgfqpoint{1.172705in}{1.665854in}}%
\pgfpathlineto{\pgfqpoint{1.173186in}{1.665972in}}%
\pgfpathlineto{\pgfqpoint{1.173667in}{1.665961in}}%
\pgfpathlineto{\pgfqpoint{1.174148in}{1.665819in}}%
\pgfpathlineto{\pgfqpoint{1.172687in}{1.665723in}}%
\pgfpathlineto{\pgfqpoint{1.171233in}{1.665605in}}%
\pgfpathlineto{\pgfqpoint{1.169787in}{1.665467in}}%
\pgfpathlineto{\pgfqpoint{1.168351in}{1.665307in}}%
\pgfpathclose%
\pgfusepath{fill}%
\end{pgfscope}%
\begin{pgfscope}%
\pgfpathrectangle{\pgfqpoint{0.041670in}{0.041670in}}{\pgfqpoint{2.216660in}{2.216660in}}%
\pgfusepath{clip}%
\pgfsetbuttcap%
\pgfsetroundjoin%
\definecolor{currentfill}{rgb}{0.955300,0.901065,0.118128}%
\pgfsetfillcolor{currentfill}%
\pgfsetlinewidth{0.000000pt}%
\definecolor{currentstroke}{rgb}{0.000000,0.000000,0.000000}%
\pgfsetstrokecolor{currentstroke}%
\pgfsetdash{}{0pt}%
\pgfpathmoveto{\pgfqpoint{1.185924in}{1.665809in}}%
\pgfpathlineto{\pgfqpoint{1.186419in}{1.665950in}}%
\pgfpathlineto{\pgfqpoint{1.186914in}{1.665961in}}%
\pgfpathlineto{\pgfqpoint{1.187408in}{1.665842in}}%
\pgfpathlineto{\pgfqpoint{1.187902in}{1.665594in}}%
\pgfpathlineto{\pgfqpoint{1.189847in}{1.665463in}}%
\pgfpathlineto{\pgfqpoint{1.191782in}{1.665304in}}%
\pgfpathlineto{\pgfqpoint{1.193705in}{1.665116in}}%
\pgfpathlineto{\pgfqpoint{1.195615in}{1.664900in}}%
\pgfpathlineto{\pgfqpoint{1.194642in}{1.665191in}}%
\pgfpathlineto{\pgfqpoint{1.193668in}{1.665354in}}%
\pgfpathlineto{\pgfqpoint{1.192693in}{1.665386in}}%
\pgfpathlineto{\pgfqpoint{1.191717in}{1.665288in}}%
\pgfpathlineto{\pgfqpoint{1.190283in}{1.665450in}}%
\pgfpathlineto{\pgfqpoint{1.188838in}{1.665591in}}%
\pgfpathlineto{\pgfqpoint{1.187385in}{1.665711in}}%
\pgfpathlineto{\pgfqpoint{1.185924in}{1.665809in}}%
\pgfpathclose%
\pgfusepath{fill}%
\end{pgfscope}%
\begin{pgfscope}%
\pgfpathrectangle{\pgfqpoint{0.041670in}{0.041670in}}{\pgfqpoint{2.216660in}{2.216660in}}%
\pgfusepath{clip}%
\pgfsetbuttcap%
\pgfsetroundjoin%
\definecolor{currentfill}{rgb}{0.935904,0.898570,0.108131}%
\pgfsetfillcolor{currentfill}%
\pgfsetlinewidth{0.000000pt}%
\definecolor{currentstroke}{rgb}{0.000000,0.000000,0.000000}%
\pgfsetstrokecolor{currentstroke}%
\pgfsetdash{}{0pt}%
\pgfpathmoveto{\pgfqpoint{1.131820in}{1.655264in}}%
\pgfpathlineto{\pgfqpoint{1.128826in}{1.654244in}}%
\pgfpathlineto{\pgfqpoint{1.125834in}{1.653097in}}%
\pgfpathlineto{\pgfqpoint{1.122844in}{1.651822in}}%
\pgfpathlineto{\pgfqpoint{1.119856in}{1.650421in}}%
\pgfpathlineto{\pgfqpoint{1.121455in}{1.651293in}}%
\pgfpathlineto{\pgfqpoint{1.123111in}{1.652142in}}%
\pgfpathlineto{\pgfqpoint{1.124824in}{1.652965in}}%
\pgfpathlineto{\pgfqpoint{1.126591in}{1.653763in}}%
\pgfpathlineto{\pgfqpoint{1.129243in}{1.654998in}}%
\pgfpathlineto{\pgfqpoint{1.131896in}{1.656106in}}%
\pgfpathlineto{\pgfqpoint{1.134552in}{1.657087in}}%
\pgfpathlineto{\pgfqpoint{1.137210in}{1.657940in}}%
\pgfpathlineto{\pgfqpoint{1.135796in}{1.657301in}}%
\pgfpathlineto{\pgfqpoint{1.134425in}{1.656642in}}%
\pgfpathlineto{\pgfqpoint{1.133099in}{1.655963in}}%
\pgfpathlineto{\pgfqpoint{1.131820in}{1.655264in}}%
\pgfpathclose%
\pgfusepath{fill}%
\end{pgfscope}%
\begin{pgfscope}%
\pgfpathrectangle{\pgfqpoint{0.041670in}{0.041670in}}{\pgfqpoint{2.216660in}{2.216660in}}%
\pgfusepath{clip}%
\pgfsetbuttcap%
\pgfsetroundjoin%
\definecolor{currentfill}{rgb}{0.896320,0.893616,0.096335}%
\pgfsetfillcolor{currentfill}%
\pgfsetlinewidth{0.000000pt}%
\definecolor{currentstroke}{rgb}{0.000000,0.000000,0.000000}%
\pgfsetstrokecolor{currentstroke}%
\pgfsetdash{}{0pt}%
\pgfpathmoveto{\pgfqpoint{1.109320in}{1.642681in}}%
\pgfpathlineto{\pgfqpoint{1.105807in}{1.640768in}}%
\pgfpathlineto{\pgfqpoint{1.102296in}{1.638730in}}%
\pgfpathlineto{\pgfqpoint{1.098787in}{1.636567in}}%
\pgfpathlineto{\pgfqpoint{1.095280in}{1.634282in}}%
\pgfpathlineto{\pgfqpoint{1.096583in}{1.635521in}}%
\pgfpathlineto{\pgfqpoint{1.097969in}{1.636741in}}%
\pgfpathlineto{\pgfqpoint{1.099435in}{1.637939in}}%
\pgfpathlineto{\pgfqpoint{1.100981in}{1.639114in}}%
\pgfpathlineto{\pgfqpoint{1.104249in}{1.641198in}}%
\pgfpathlineto{\pgfqpoint{1.107520in}{1.643159in}}%
\pgfpathlineto{\pgfqpoint{1.110793in}{1.644996in}}%
\pgfpathlineto{\pgfqpoint{1.114068in}{1.646708in}}%
\pgfpathlineto{\pgfqpoint{1.112780in}{1.645728in}}%
\pgfpathlineto{\pgfqpoint{1.111559in}{1.644730in}}%
\pgfpathlineto{\pgfqpoint{1.110405in}{1.643714in}}%
\pgfpathlineto{\pgfqpoint{1.109320in}{1.642681in}}%
\pgfpathclose%
\pgfusepath{fill}%
\end{pgfscope}%
\begin{pgfscope}%
\pgfpathrectangle{\pgfqpoint{0.041670in}{0.041670in}}{\pgfqpoint{2.216660in}{2.216660in}}%
\pgfusepath{clip}%
\pgfsetbuttcap%
\pgfsetroundjoin%
\definecolor{currentfill}{rgb}{0.122606,0.585371,0.546557}%
\pgfsetfillcolor{currentfill}%
\pgfsetlinewidth{0.000000pt}%
\definecolor{currentstroke}{rgb}{0.000000,0.000000,0.000000}%
\pgfsetstrokecolor{currentstroke}%
\pgfsetdash{}{0pt}%
\pgfpathmoveto{\pgfqpoint{0.941547in}{1.190059in}}%
\pgfpathlineto{\pgfqpoint{0.938612in}{1.181125in}}%
\pgfpathlineto{\pgfqpoint{0.935679in}{1.172169in}}%
\pgfpathlineto{\pgfqpoint{0.932747in}{1.163196in}}%
\pgfpathlineto{\pgfqpoint{0.929817in}{1.154207in}}%
\pgfpathlineto{\pgfqpoint{0.923428in}{1.158220in}}%
\pgfpathlineto{\pgfqpoint{0.917307in}{1.162330in}}%
\pgfpathlineto{\pgfqpoint{0.911458in}{1.166533in}}%
\pgfpathlineto{\pgfqpoint{0.905887in}{1.170823in}}%
\pgfpathlineto{\pgfqpoint{0.909090in}{1.179596in}}%
\pgfpathlineto{\pgfqpoint{0.912295in}{1.188355in}}%
\pgfpathlineto{\pgfqpoint{0.915502in}{1.197096in}}%
\pgfpathlineto{\pgfqpoint{0.918710in}{1.205816in}}%
\pgfpathlineto{\pgfqpoint{0.924028in}{1.201748in}}%
\pgfpathlineto{\pgfqpoint{0.929610in}{1.197762in}}%
\pgfpathlineto{\pgfqpoint{0.935452in}{1.193865in}}%
\pgfpathlineto{\pgfqpoint{0.941547in}{1.190059in}}%
\pgfpathclose%
\pgfusepath{fill}%
\end{pgfscope}%
\begin{pgfscope}%
\pgfpathrectangle{\pgfqpoint{0.041670in}{0.041670in}}{\pgfqpoint{2.216660in}{2.216660in}}%
\pgfusepath{clip}%
\pgfsetbuttcap%
\pgfsetroundjoin%
\definecolor{currentfill}{rgb}{0.282884,0.135920,0.453427}%
\pgfsetfillcolor{currentfill}%
\pgfsetlinewidth{0.000000pt}%
\definecolor{currentstroke}{rgb}{0.000000,0.000000,0.000000}%
\pgfsetstrokecolor{currentstroke}%
\pgfsetdash{}{0pt}%
\pgfpathmoveto{\pgfqpoint{0.655488in}{0.706211in}}%
\pgfpathlineto{\pgfqpoint{0.652243in}{0.711362in}}%
\pgfpathlineto{\pgfqpoint{0.648985in}{0.716887in}}%
\pgfpathlineto{\pgfqpoint{0.645714in}{0.722792in}}%
\pgfpathlineto{\pgfqpoint{0.642430in}{0.729083in}}%
\pgfpathlineto{\pgfqpoint{0.629068in}{0.738230in}}%
\pgfpathlineto{\pgfqpoint{0.616305in}{0.747584in}}%
\pgfpathlineto{\pgfqpoint{0.604152in}{0.757135in}}%
\pgfpathlineto{\pgfqpoint{0.592622in}{0.766872in}}%
\pgfpathlineto{\pgfqpoint{0.596192in}{0.760382in}}%
\pgfpathlineto{\pgfqpoint{0.599748in}{0.754278in}}%
\pgfpathlineto{\pgfqpoint{0.603290in}{0.748553in}}%
\pgfpathlineto{\pgfqpoint{0.606818in}{0.743200in}}%
\pgfpathlineto{\pgfqpoint{0.618088in}{0.733668in}}%
\pgfpathlineto{\pgfqpoint{0.629963in}{0.724319in}}%
\pgfpathlineto{\pgfqpoint{0.642434in}{0.715163in}}%
\pgfpathlineto{\pgfqpoint{0.655488in}{0.706211in}}%
\pgfpathclose%
\pgfusepath{fill}%
\end{pgfscope}%
\begin{pgfscope}%
\pgfpathrectangle{\pgfqpoint{0.041670in}{0.041670in}}{\pgfqpoint{2.216660in}{2.216660in}}%
\pgfusepath{clip}%
\pgfsetbuttcap%
\pgfsetroundjoin%
\definecolor{currentfill}{rgb}{0.212395,0.359683,0.551710}%
\pgfsetfillcolor{currentfill}%
\pgfsetlinewidth{0.000000pt}%
\definecolor{currentstroke}{rgb}{0.000000,0.000000,0.000000}%
\pgfsetstrokecolor{currentstroke}%
\pgfsetdash{}{0pt}%
\pgfpathmoveto{\pgfqpoint{1.462056in}{0.960609in}}%
\pgfpathlineto{\pgfqpoint{1.464723in}{0.951805in}}%
\pgfpathlineto{\pgfqpoint{1.467391in}{0.943050in}}%
\pgfpathlineto{\pgfqpoint{1.470058in}{0.934349in}}%
\pgfpathlineto{\pgfqpoint{1.472726in}{0.925705in}}%
\pgfpathlineto{\pgfqpoint{1.463269in}{0.920989in}}%
\pgfpathlineto{\pgfqpoint{1.453511in}{0.916427in}}%
\pgfpathlineto{\pgfqpoint{1.443462in}{0.912024in}}%
\pgfpathlineto{\pgfqpoint{1.433131in}{0.907786in}}%
\pgfpathlineto{\pgfqpoint{1.430819in}{0.916616in}}%
\pgfpathlineto{\pgfqpoint{1.428506in}{0.925501in}}%
\pgfpathlineto{\pgfqpoint{1.426194in}{0.934440in}}%
\pgfpathlineto{\pgfqpoint{1.423882in}{0.943429in}}%
\pgfpathlineto{\pgfqpoint{1.433840in}{0.947492in}}%
\pgfpathlineto{\pgfqpoint{1.443529in}{0.951712in}}%
\pgfpathlineto{\pgfqpoint{1.452937in}{0.956087in}}%
\pgfpathlineto{\pgfqpoint{1.462056in}{0.960609in}}%
\pgfpathclose%
\pgfusepath{fill}%
\end{pgfscope}%
\begin{pgfscope}%
\pgfpathrectangle{\pgfqpoint{0.041670in}{0.041670in}}{\pgfqpoint{2.216660in}{2.216660in}}%
\pgfusepath{clip}%
\pgfsetbuttcap%
\pgfsetroundjoin%
\definecolor{currentfill}{rgb}{0.231674,0.318106,0.544834}%
\pgfsetfillcolor{currentfill}%
\pgfsetlinewidth{0.000000pt}%
\definecolor{currentstroke}{rgb}{0.000000,0.000000,0.000000}%
\pgfsetstrokecolor{currentstroke}%
\pgfsetdash{}{0pt}%
\pgfpathmoveto{\pgfqpoint{0.936189in}{0.904161in}}%
\pgfpathlineto{\pgfqpoint{0.933961in}{0.895354in}}%
\pgfpathlineto{\pgfqpoint{0.931733in}{0.886610in}}%
\pgfpathlineto{\pgfqpoint{0.929505in}{0.877933in}}%
\pgfpathlineto{\pgfqpoint{0.927277in}{0.869325in}}%
\pgfpathlineto{\pgfqpoint{0.916324in}{0.873583in}}%
\pgfpathlineto{\pgfqpoint{0.905654in}{0.878016in}}%
\pgfpathlineto{\pgfqpoint{0.895277in}{0.882620in}}%
\pgfpathlineto{\pgfqpoint{0.885204in}{0.887389in}}%
\pgfpathlineto{\pgfqpoint{0.887797in}{0.895816in}}%
\pgfpathlineto{\pgfqpoint{0.890390in}{0.904314in}}%
\pgfpathlineto{\pgfqpoint{0.892983in}{0.912878in}}%
\pgfpathlineto{\pgfqpoint{0.895575in}{0.921505in}}%
\pgfpathlineto{\pgfqpoint{0.905300in}{0.916926in}}%
\pgfpathlineto{\pgfqpoint{0.915317in}{0.912506in}}%
\pgfpathlineto{\pgfqpoint{0.925617in}{0.908249in}}%
\pgfpathlineto{\pgfqpoint{0.936189in}{0.904161in}}%
\pgfpathclose%
\pgfusepath{fill}%
\end{pgfscope}%
\begin{pgfscope}%
\pgfpathrectangle{\pgfqpoint{0.041670in}{0.041670in}}{\pgfqpoint{2.216660in}{2.216660in}}%
\pgfusepath{clip}%
\pgfsetbuttcap%
\pgfsetroundjoin%
\definecolor{currentfill}{rgb}{0.955300,0.901065,0.118128}%
\pgfsetfillcolor{currentfill}%
\pgfsetlinewidth{0.000000pt}%
\definecolor{currentstroke}{rgb}{0.000000,0.000000,0.000000}%
\pgfsetstrokecolor{currentstroke}%
\pgfsetdash{}{0pt}%
\pgfpathmoveto{\pgfqpoint{1.191717in}{1.665288in}}%
\pgfpathlineto{\pgfqpoint{1.192693in}{1.665386in}}%
\pgfpathlineto{\pgfqpoint{1.193668in}{1.665354in}}%
\pgfpathlineto{\pgfqpoint{1.194642in}{1.665191in}}%
\pgfpathlineto{\pgfqpoint{1.195615in}{1.664900in}}%
\pgfpathlineto{\pgfqpoint{1.197510in}{1.664656in}}%
\pgfpathlineto{\pgfqpoint{1.199387in}{1.664385in}}%
\pgfpathlineto{\pgfqpoint{1.201245in}{1.664086in}}%
\pgfpathlineto{\pgfqpoint{1.199922in}{1.664428in}}%
\pgfpathlineto{\pgfqpoint{1.198597in}{1.664641in}}%
\pgfpathlineto{\pgfqpoint{1.197272in}{1.664724in}}%
\pgfpathlineto{\pgfqpoint{1.195945in}{1.664676in}}%
\pgfpathlineto{\pgfqpoint{1.194550in}{1.664901in}}%
\pgfpathlineto{\pgfqpoint{1.193140in}{1.665105in}}%
\pgfpathlineto{\pgfqpoint{1.191717in}{1.665288in}}%
\pgfpathclose%
\pgfusepath{fill}%
\end{pgfscope}%
\begin{pgfscope}%
\pgfpathrectangle{\pgfqpoint{0.041670in}{0.041670in}}{\pgfqpoint{2.216660in}{2.216660in}}%
\pgfusepath{clip}%
\pgfsetbuttcap%
\pgfsetroundjoin%
\definecolor{currentfill}{rgb}{0.163625,0.471133,0.558148}%
\pgfsetfillcolor{currentfill}%
\pgfsetlinewidth{0.000000pt}%
\definecolor{currentstroke}{rgb}{0.000000,0.000000,0.000000}%
\pgfsetstrokecolor{currentstroke}%
\pgfsetdash{}{0pt}%
\pgfpathmoveto{\pgfqpoint{0.937089in}{1.065494in}}%
\pgfpathlineto{\pgfqpoint{0.934490in}{1.056293in}}%
\pgfpathlineto{\pgfqpoint{0.931892in}{1.047104in}}%
\pgfpathlineto{\pgfqpoint{0.929295in}{1.037931in}}%
\pgfpathlineto{\pgfqpoint{0.926699in}{1.028777in}}%
\pgfpathlineto{\pgfqpoint{0.918288in}{1.032930in}}%
\pgfpathlineto{\pgfqpoint{0.910153in}{1.037213in}}%
\pgfpathlineto{\pgfqpoint{0.902303in}{1.041622in}}%
\pgfpathlineto{\pgfqpoint{0.894743in}{1.046153in}}%
\pgfpathlineto{\pgfqpoint{0.897660in}{1.055107in}}%
\pgfpathlineto{\pgfqpoint{0.900578in}{1.064081in}}%
\pgfpathlineto{\pgfqpoint{0.903497in}{1.073070in}}%
\pgfpathlineto{\pgfqpoint{0.906417in}{1.082072in}}%
\pgfpathlineto{\pgfqpoint{0.913674in}{1.077749in}}%
\pgfpathlineto{\pgfqpoint{0.921210in}{1.073542in}}%
\pgfpathlineto{\pgfqpoint{0.929017in}{1.069455in}}%
\pgfpathlineto{\pgfqpoint{0.937089in}{1.065494in}}%
\pgfpathclose%
\pgfusepath{fill}%
\end{pgfscope}%
\begin{pgfscope}%
\pgfpathrectangle{\pgfqpoint{0.041670in}{0.041670in}}{\pgfqpoint{2.216660in}{2.216660in}}%
\pgfusepath{clip}%
\pgfsetbuttcap%
\pgfsetroundjoin%
\definecolor{currentfill}{rgb}{0.955300,0.901065,0.118128}%
\pgfsetfillcolor{currentfill}%
\pgfsetlinewidth{0.000000pt}%
\definecolor{currentstroke}{rgb}{0.000000,0.000000,0.000000}%
\pgfsetstrokecolor{currentstroke}%
\pgfsetdash{}{0pt}%
\pgfpathmoveto{\pgfqpoint{1.162738in}{1.664460in}}%
\pgfpathlineto{\pgfqpoint{1.161309in}{1.664489in}}%
\pgfpathlineto{\pgfqpoint{1.159882in}{1.664388in}}%
\pgfpathlineto{\pgfqpoint{1.158456in}{1.664157in}}%
\pgfpathlineto{\pgfqpoint{1.157030in}{1.663797in}}%
\pgfpathlineto{\pgfqpoint{1.158870in}{1.664120in}}%
\pgfpathlineto{\pgfqpoint{1.160730in}{1.664416in}}%
\pgfpathlineto{\pgfqpoint{1.162610in}{1.664685in}}%
\pgfpathlineto{\pgfqpoint{1.164506in}{1.664925in}}%
\pgfpathlineto{\pgfqpoint{1.165466in}{1.665215in}}%
\pgfpathlineto{\pgfqpoint{1.166427in}{1.665376in}}%
\pgfpathlineto{\pgfqpoint{1.167389in}{1.665407in}}%
\pgfpathlineto{\pgfqpoint{1.168351in}{1.665307in}}%
\pgfpathlineto{\pgfqpoint{1.166927in}{1.665126in}}%
\pgfpathlineto{\pgfqpoint{1.165516in}{1.664924in}}%
\pgfpathlineto{\pgfqpoint{1.164119in}{1.664702in}}%
\pgfpathlineto{\pgfqpoint{1.162738in}{1.664460in}}%
\pgfpathclose%
\pgfusepath{fill}%
\end{pgfscope}%
\begin{pgfscope}%
\pgfpathrectangle{\pgfqpoint{0.041670in}{0.041670in}}{\pgfqpoint{2.216660in}{2.216660in}}%
\pgfusepath{clip}%
\pgfsetbuttcap%
\pgfsetroundjoin%
\definecolor{currentfill}{rgb}{0.277941,0.056324,0.381191}%
\pgfsetfillcolor{currentfill}%
\pgfsetlinewidth{0.000000pt}%
\definecolor{currentstroke}{rgb}{0.000000,0.000000,0.000000}%
\pgfsetstrokecolor{currentstroke}%
\pgfsetdash{}{0pt}%
\pgfpathmoveto{\pgfqpoint{1.689982in}{0.685285in}}%
\pgfpathlineto{\pgfqpoint{1.693202in}{0.687717in}}%
\pgfpathlineto{\pgfqpoint{1.696432in}{0.690474in}}%
\pgfpathlineto{\pgfqpoint{1.699674in}{0.693561in}}%
\pgfpathlineto{\pgfqpoint{1.702926in}{0.696984in}}%
\pgfpathlineto{\pgfqpoint{1.690114in}{0.688254in}}%
\pgfpathlineto{\pgfqpoint{1.676747in}{0.679735in}}%
\pgfpathlineto{\pgfqpoint{1.662838in}{0.671437in}}%
\pgfpathlineto{\pgfqpoint{1.648401in}{0.663371in}}%
\pgfpathlineto{\pgfqpoint{1.645474in}{0.660142in}}%
\pgfpathlineto{\pgfqpoint{1.642558in}{0.657251in}}%
\pgfpathlineto{\pgfqpoint{1.639651in}{0.654691in}}%
\pgfpathlineto{\pgfqpoint{1.636754in}{0.652456in}}%
\pgfpathlineto{\pgfqpoint{1.650846in}{0.660333in}}%
\pgfpathlineto{\pgfqpoint{1.664423in}{0.668437in}}%
\pgfpathlineto{\pgfqpoint{1.677472in}{0.676758in}}%
\pgfpathlineto{\pgfqpoint{1.689982in}{0.685285in}}%
\pgfpathclose%
\pgfusepath{fill}%
\end{pgfscope}%
\begin{pgfscope}%
\pgfpathrectangle{\pgfqpoint{0.041670in}{0.041670in}}{\pgfqpoint{2.216660in}{2.216660in}}%
\pgfusepath{clip}%
\pgfsetbuttcap%
\pgfsetroundjoin%
\definecolor{currentfill}{rgb}{0.699415,0.867117,0.175971}%
\pgfsetfillcolor{currentfill}%
\pgfsetlinewidth{0.000000pt}%
\definecolor{currentstroke}{rgb}{0.000000,0.000000,0.000000}%
\pgfsetstrokecolor{currentstroke}%
\pgfsetdash{}{0pt}%
\pgfpathmoveto{\pgfqpoint{1.039983in}{1.565229in}}%
\pgfpathlineto{\pgfqpoint{1.036123in}{1.560526in}}%
\pgfpathlineto{\pgfqpoint{1.032266in}{1.555716in}}%
\pgfpathlineto{\pgfqpoint{1.028411in}{1.550798in}}%
\pgfpathlineto{\pgfqpoint{1.024558in}{1.545776in}}%
\pgfpathlineto{\pgfqpoint{1.024350in}{1.548100in}}%
\pgfpathlineto{\pgfqpoint{1.024298in}{1.550426in}}%
\pgfpathlineto{\pgfqpoint{1.024400in}{1.552751in}}%
\pgfpathlineto{\pgfqpoint{1.024658in}{1.555072in}}%
\pgfpathlineto{\pgfqpoint{1.028503in}{1.559858in}}%
\pgfpathlineto{\pgfqpoint{1.032350in}{1.564539in}}%
\pgfpathlineto{\pgfqpoint{1.036201in}{1.569113in}}%
\pgfpathlineto{\pgfqpoint{1.040053in}{1.573579in}}%
\pgfpathlineto{\pgfqpoint{1.039826in}{1.571494in}}%
\pgfpathlineto{\pgfqpoint{1.039739in}{1.569406in}}%
\pgfpathlineto{\pgfqpoint{1.039791in}{1.567316in}}%
\pgfpathlineto{\pgfqpoint{1.039983in}{1.565229in}}%
\pgfpathclose%
\pgfusepath{fill}%
\end{pgfscope}%
\begin{pgfscope}%
\pgfpathrectangle{\pgfqpoint{0.041670in}{0.041670in}}{\pgfqpoint{2.216660in}{2.216660in}}%
\pgfusepath{clip}%
\pgfsetbuttcap%
\pgfsetroundjoin%
\definecolor{currentfill}{rgb}{0.487026,0.823929,0.312321}%
\pgfsetfillcolor{currentfill}%
\pgfsetlinewidth{0.000000pt}%
\definecolor{currentstroke}{rgb}{0.000000,0.000000,0.000000}%
\pgfsetstrokecolor{currentstroke}%
\pgfsetdash{}{0pt}%
\pgfpathmoveto{\pgfqpoint{1.364161in}{1.493295in}}%
\pgfpathlineto{\pgfqpoint{1.367955in}{1.487206in}}%
\pgfpathlineto{\pgfqpoint{1.371748in}{1.481027in}}%
\pgfpathlineto{\pgfqpoint{1.375537in}{1.474759in}}%
\pgfpathlineto{\pgfqpoint{1.379324in}{1.468406in}}%
\pgfpathlineto{\pgfqpoint{1.378251in}{1.465397in}}%
\pgfpathlineto{\pgfqpoint{1.376978in}{1.462404in}}%
\pgfpathlineto{\pgfqpoint{1.375507in}{1.459431in}}%
\pgfpathlineto{\pgfqpoint{1.373838in}{1.456480in}}%
\pgfpathlineto{\pgfqpoint{1.370149in}{1.463072in}}%
\pgfpathlineto{\pgfqpoint{1.366457in}{1.469577in}}%
\pgfpathlineto{\pgfqpoint{1.362763in}{1.475993in}}%
\pgfpathlineto{\pgfqpoint{1.359067in}{1.482320in}}%
\pgfpathlineto{\pgfqpoint{1.360614in}{1.485035in}}%
\pgfpathlineto{\pgfqpoint{1.361980in}{1.487771in}}%
\pgfpathlineto{\pgfqpoint{1.363162in}{1.490526in}}%
\pgfpathlineto{\pgfqpoint{1.364161in}{1.493295in}}%
\pgfpathclose%
\pgfusepath{fill}%
\end{pgfscope}%
\begin{pgfscope}%
\pgfpathrectangle{\pgfqpoint{0.041670in}{0.041670in}}{\pgfqpoint{2.216660in}{2.216660in}}%
\pgfusepath{clip}%
\pgfsetbuttcap%
\pgfsetroundjoin%
\definecolor{currentfill}{rgb}{0.955300,0.901065,0.118128}%
\pgfsetfillcolor{currentfill}%
\pgfsetlinewidth{0.000000pt}%
\definecolor{currentstroke}{rgb}{0.000000,0.000000,0.000000}%
\pgfsetstrokecolor{currentstroke}%
\pgfsetdash{}{0pt}%
\pgfpathmoveto{\pgfqpoint{1.195945in}{1.664676in}}%
\pgfpathlineto{\pgfqpoint{1.197272in}{1.664724in}}%
\pgfpathlineto{\pgfqpoint{1.198597in}{1.664641in}}%
\pgfpathlineto{\pgfqpoint{1.199922in}{1.664428in}}%
\pgfpathlineto{\pgfqpoint{1.201245in}{1.664086in}}%
\pgfpathlineto{\pgfqpoint{1.203082in}{1.663759in}}%
\pgfpathlineto{\pgfqpoint{1.204897in}{1.663407in}}%
\pgfpathlineto{\pgfqpoint{1.206686in}{1.663027in}}%
\pgfpathlineto{\pgfqpoint{1.208450in}{1.662622in}}%
\pgfpathlineto{\pgfqpoint{1.206678in}{1.663055in}}%
\pgfpathlineto{\pgfqpoint{1.204905in}{1.663359in}}%
\pgfpathlineto{\pgfqpoint{1.203131in}{1.663533in}}%
\pgfpathlineto{\pgfqpoint{1.201355in}{1.663577in}}%
\pgfpathlineto{\pgfqpoint{1.200031in}{1.663881in}}%
\pgfpathlineto{\pgfqpoint{1.198687in}{1.664166in}}%
\pgfpathlineto{\pgfqpoint{1.197325in}{1.664431in}}%
\pgfpathlineto{\pgfqpoint{1.195945in}{1.664676in}}%
\pgfpathclose%
\pgfusepath{fill}%
\end{pgfscope}%
\begin{pgfscope}%
\pgfpathrectangle{\pgfqpoint{0.041670in}{0.041670in}}{\pgfqpoint{2.216660in}{2.216660in}}%
\pgfusepath{clip}%
\pgfsetbuttcap%
\pgfsetroundjoin%
\definecolor{currentfill}{rgb}{0.267004,0.004874,0.329415}%
\pgfsetfillcolor{currentfill}%
\pgfsetlinewidth{0.000000pt}%
\definecolor{currentstroke}{rgb}{0.000000,0.000000,0.000000}%
\pgfsetstrokecolor{currentstroke}%
\pgfsetdash{}{0pt}%
\pgfpathmoveto{\pgfqpoint{0.836868in}{0.624947in}}%
\pgfpathlineto{\pgfqpoint{0.834544in}{0.622498in}}%
\pgfpathlineto{\pgfqpoint{0.832215in}{0.620290in}}%
\pgfpathlineto{\pgfqpoint{0.829881in}{0.618328in}}%
\pgfpathlineto{\pgfqpoint{0.827541in}{0.616616in}}%
\pgfpathlineto{\pgfqpoint{0.812351in}{0.622761in}}%
\pgfpathlineto{\pgfqpoint{0.797565in}{0.629155in}}%
\pgfpathlineto{\pgfqpoint{0.783199in}{0.635793in}}%
\pgfpathlineto{\pgfqpoint{0.769267in}{0.642665in}}%
\pgfpathlineto{\pgfqpoint{0.771984in}{0.644199in}}%
\pgfpathlineto{\pgfqpoint{0.774695in}{0.645982in}}%
\pgfpathlineto{\pgfqpoint{0.777400in}{0.648011in}}%
\pgfpathlineto{\pgfqpoint{0.780099in}{0.650281in}}%
\pgfpathlineto{\pgfqpoint{0.793673in}{0.643597in}}%
\pgfpathlineto{\pgfqpoint{0.807669in}{0.637141in}}%
\pgfpathlineto{\pgfqpoint{0.822072in}{0.630922in}}%
\pgfpathlineto{\pgfqpoint{0.836868in}{0.624947in}}%
\pgfpathclose%
\pgfusepath{fill}%
\end{pgfscope}%
\begin{pgfscope}%
\pgfpathrectangle{\pgfqpoint{0.041670in}{0.041670in}}{\pgfqpoint{2.216660in}{2.216660in}}%
\pgfusepath{clip}%
\pgfsetbuttcap%
\pgfsetroundjoin%
\definecolor{currentfill}{rgb}{0.172719,0.448791,0.557885}%
\pgfsetfillcolor{currentfill}%
\pgfsetlinewidth{0.000000pt}%
\definecolor{currentstroke}{rgb}{0.000000,0.000000,0.000000}%
\pgfsetstrokecolor{currentstroke}%
\pgfsetdash{}{0pt}%
\pgfpathmoveto{\pgfqpoint{1.907116in}{1.028742in}}%
\pgfpathlineto{\pgfqpoint{1.911422in}{1.043280in}}%
\pgfpathlineto{\pgfqpoint{1.915752in}{1.058334in}}%
\pgfpathlineto{\pgfqpoint{1.920107in}{1.073913in}}%
\pgfpathlineto{\pgfqpoint{1.913902in}{1.062088in}}%
\pgfpathlineto{\pgfqpoint{1.906930in}{1.050349in}}%
\pgfpathlineto{\pgfqpoint{1.899195in}{1.038709in}}%
\pgfpathlineto{\pgfqpoint{1.890701in}{1.027180in}}%
\pgfpathlineto{\pgfqpoint{1.886491in}{1.011775in}}%
\pgfpathlineto{\pgfqpoint{1.882306in}{0.996898in}}%
\pgfpathlineto{\pgfqpoint{1.878145in}{0.982540in}}%
\pgfpathlineto{\pgfqpoint{1.886509in}{0.993937in}}%
\pgfpathlineto{\pgfqpoint{1.894129in}{1.005444in}}%
\pgfpathlineto{\pgfqpoint{1.900999in}{1.017051in}}%
\pgfpathlineto{\pgfqpoint{1.907116in}{1.028742in}}%
\pgfpathclose%
\pgfusepath{fill}%
\end{pgfscope}%
\begin{pgfscope}%
\pgfpathrectangle{\pgfqpoint{0.041670in}{0.041670in}}{\pgfqpoint{2.216660in}{2.216660in}}%
\pgfusepath{clip}%
\pgfsetbuttcap%
\pgfsetroundjoin%
\definecolor{currentfill}{rgb}{0.166383,0.690856,0.496502}%
\pgfsetfillcolor{currentfill}%
\pgfsetlinewidth{0.000000pt}%
\definecolor{currentstroke}{rgb}{0.000000,0.000000,0.000000}%
\pgfsetstrokecolor{currentstroke}%
\pgfsetdash{}{0pt}%
\pgfpathmoveto{\pgfqpoint{0.957349in}{1.307936in}}%
\pgfpathlineto{\pgfqpoint{0.954119in}{1.299659in}}%
\pgfpathlineto{\pgfqpoint{0.950891in}{1.291332in}}%
\pgfpathlineto{\pgfqpoint{0.947665in}{1.282956in}}%
\pgfpathlineto{\pgfqpoint{0.944440in}{1.274535in}}%
\pgfpathlineto{\pgfqpoint{0.939876in}{1.278237in}}%
\pgfpathlineto{\pgfqpoint{0.935558in}{1.282006in}}%
\pgfpathlineto{\pgfqpoint{0.931490in}{1.285838in}}%
\pgfpathlineto{\pgfqpoint{0.927677in}{1.289729in}}%
\pgfpathlineto{\pgfqpoint{0.931124in}{1.297925in}}%
\pgfpathlineto{\pgfqpoint{0.934572in}{1.306075in}}%
\pgfpathlineto{\pgfqpoint{0.938023in}{1.314177in}}%
\pgfpathlineto{\pgfqpoint{0.941476in}{1.322230in}}%
\pgfpathlineto{\pgfqpoint{0.945087in}{1.318569in}}%
\pgfpathlineto{\pgfqpoint{0.948939in}{1.314964in}}%
\pgfpathlineto{\pgfqpoint{0.953028in}{1.311419in}}%
\pgfpathlineto{\pgfqpoint{0.957349in}{1.307936in}}%
\pgfpathclose%
\pgfusepath{fill}%
\end{pgfscope}%
\begin{pgfscope}%
\pgfpathrectangle{\pgfqpoint{0.041670in}{0.041670in}}{\pgfqpoint{2.216660in}{2.216660in}}%
\pgfusepath{clip}%
\pgfsetbuttcap%
\pgfsetroundjoin%
\definecolor{currentfill}{rgb}{0.955300,0.901065,0.118128}%
\pgfsetfillcolor{currentfill}%
\pgfsetlinewidth{0.000000pt}%
\definecolor{currentstroke}{rgb}{0.000000,0.000000,0.000000}%
\pgfsetstrokecolor{currentstroke}%
\pgfsetdash{}{0pt}%
\pgfpathmoveto{\pgfqpoint{1.157395in}{1.663290in}}%
\pgfpathlineto{\pgfqpoint{1.155523in}{1.663222in}}%
\pgfpathlineto{\pgfqpoint{1.153653in}{1.663024in}}%
\pgfpathlineto{\pgfqpoint{1.151783in}{1.662696in}}%
\pgfpathlineto{\pgfqpoint{1.149916in}{1.662239in}}%
\pgfpathlineto{\pgfqpoint{1.151654in}{1.662668in}}%
\pgfpathlineto{\pgfqpoint{1.153421in}{1.663071in}}%
\pgfpathlineto{\pgfqpoint{1.155213in}{1.663447in}}%
\pgfpathlineto{\pgfqpoint{1.157030in}{1.663797in}}%
\pgfpathlineto{\pgfqpoint{1.158456in}{1.664157in}}%
\pgfpathlineto{\pgfqpoint{1.159882in}{1.664388in}}%
\pgfpathlineto{\pgfqpoint{1.161309in}{1.664489in}}%
\pgfpathlineto{\pgfqpoint{1.162738in}{1.664460in}}%
\pgfpathlineto{\pgfqpoint{1.161373in}{1.664197in}}%
\pgfpathlineto{\pgfqpoint{1.160027in}{1.663914in}}%
\pgfpathlineto{\pgfqpoint{1.158701in}{1.663612in}}%
\pgfpathlineto{\pgfqpoint{1.157395in}{1.663290in}}%
\pgfpathclose%
\pgfusepath{fill}%
\end{pgfscope}%
\begin{pgfscope}%
\pgfpathrectangle{\pgfqpoint{0.041670in}{0.041670in}}{\pgfqpoint{2.216660in}{2.216660in}}%
\pgfusepath{clip}%
\pgfsetbuttcap%
\pgfsetroundjoin%
\definecolor{currentfill}{rgb}{0.147607,0.511733,0.557049}%
\pgfsetfillcolor{currentfill}%
\pgfsetlinewidth{0.000000pt}%
\definecolor{currentstroke}{rgb}{0.000000,0.000000,0.000000}%
\pgfsetstrokecolor{currentstroke}%
\pgfsetdash{}{0pt}%
\pgfpathmoveto{\pgfqpoint{1.447754in}{1.121900in}}%
\pgfpathlineto{\pgfqpoint{1.450743in}{1.112922in}}%
\pgfpathlineto{\pgfqpoint{1.453731in}{1.103945in}}%
\pgfpathlineto{\pgfqpoint{1.456718in}{1.094973in}}%
\pgfpathlineto{\pgfqpoint{1.459704in}{1.086008in}}%
\pgfpathlineto{\pgfqpoint{1.452701in}{1.081586in}}%
\pgfpathlineto{\pgfqpoint{1.445412in}{1.077275in}}%
\pgfpathlineto{\pgfqpoint{1.437846in}{1.073082in}}%
\pgfpathlineto{\pgfqpoint{1.430008in}{1.069009in}}%
\pgfpathlineto{\pgfqpoint{1.427333in}{1.078177in}}%
\pgfpathlineto{\pgfqpoint{1.424657in}{1.087352in}}%
\pgfpathlineto{\pgfqpoint{1.421981in}{1.096532in}}%
\pgfpathlineto{\pgfqpoint{1.419303in}{1.105712in}}%
\pgfpathlineto{\pgfqpoint{1.426811in}{1.109590in}}%
\pgfpathlineto{\pgfqpoint{1.434060in}{1.113584in}}%
\pgfpathlineto{\pgfqpoint{1.441043in}{1.117689in}}%
\pgfpathlineto{\pgfqpoint{1.447754in}{1.121900in}}%
\pgfpathclose%
\pgfusepath{fill}%
\end{pgfscope}%
\begin{pgfscope}%
\pgfpathrectangle{\pgfqpoint{0.041670in}{0.041670in}}{\pgfqpoint{2.216660in}{2.216660in}}%
\pgfusepath{clip}%
\pgfsetbuttcap%
\pgfsetroundjoin%
\definecolor{currentfill}{rgb}{0.120081,0.622161,0.534946}%
\pgfsetfillcolor{currentfill}%
\pgfsetlinewidth{0.000000pt}%
\definecolor{currentstroke}{rgb}{0.000000,0.000000,0.000000}%
\pgfsetstrokecolor{currentstroke}%
\pgfsetdash{}{0pt}%
\pgfpathmoveto{\pgfqpoint{1.432634in}{1.243926in}}%
\pgfpathlineto{\pgfqpoint{1.435903in}{1.235363in}}%
\pgfpathlineto{\pgfqpoint{1.439171in}{1.226769in}}%
\pgfpathlineto{\pgfqpoint{1.442436in}{1.218147in}}%
\pgfpathlineto{\pgfqpoint{1.445700in}{1.209499in}}%
\pgfpathlineto{\pgfqpoint{1.440622in}{1.205360in}}%
\pgfpathlineto{\pgfqpoint{1.435275in}{1.201301in}}%
\pgfpathlineto{\pgfqpoint{1.429663in}{1.197325in}}%
\pgfpathlineto{\pgfqpoint{1.423793in}{1.193437in}}%
\pgfpathlineto{\pgfqpoint{1.420792in}{1.202303in}}%
\pgfpathlineto{\pgfqpoint{1.417789in}{1.211142in}}%
\pgfpathlineto{\pgfqpoint{1.414784in}{1.219953in}}%
\pgfpathlineto{\pgfqpoint{1.411778in}{1.228732in}}%
\pgfpathlineto{\pgfqpoint{1.417366in}{1.232409in}}%
\pgfpathlineto{\pgfqpoint{1.422708in}{1.236170in}}%
\pgfpathlineto{\pgfqpoint{1.427799in}{1.240010in}}%
\pgfpathlineto{\pgfqpoint{1.432634in}{1.243926in}}%
\pgfpathclose%
\pgfusepath{fill}%
\end{pgfscope}%
\begin{pgfscope}%
\pgfpathrectangle{\pgfqpoint{0.041670in}{0.041670in}}{\pgfqpoint{2.216660in}{2.216660in}}%
\pgfusepath{clip}%
\pgfsetbuttcap%
\pgfsetroundjoin%
\definecolor{currentfill}{rgb}{0.344074,0.780029,0.397381}%
\pgfsetfillcolor{currentfill}%
\pgfsetlinewidth{0.000000pt}%
\definecolor{currentstroke}{rgb}{0.000000,0.000000,0.000000}%
\pgfsetstrokecolor{currentstroke}%
\pgfsetdash{}{0pt}%
\pgfpathmoveto{\pgfqpoint{1.388572in}{1.429283in}}%
\pgfpathlineto{\pgfqpoint{1.392249in}{1.422286in}}%
\pgfpathlineto{\pgfqpoint{1.395924in}{1.415214in}}%
\pgfpathlineto{\pgfqpoint{1.399596in}{1.408068in}}%
\pgfpathlineto{\pgfqpoint{1.403266in}{1.400851in}}%
\pgfpathlineto{\pgfqpoint{1.401134in}{1.397453in}}%
\pgfpathlineto{\pgfqpoint{1.398778in}{1.394088in}}%
\pgfpathlineto{\pgfqpoint{1.396200in}{1.390759in}}%
\pgfpathlineto{\pgfqpoint{1.393403in}{1.387469in}}%
\pgfpathlineto{\pgfqpoint{1.389888in}{1.394921in}}%
\pgfpathlineto{\pgfqpoint{1.386371in}{1.402301in}}%
\pgfpathlineto{\pgfqpoint{1.382852in}{1.409607in}}%
\pgfpathlineto{\pgfqpoint{1.379331in}{1.416837in}}%
\pgfpathlineto{\pgfqpoint{1.381951in}{1.419897in}}%
\pgfpathlineto{\pgfqpoint{1.384365in}{1.422993in}}%
\pgfpathlineto{\pgfqpoint{1.386573in}{1.426123in}}%
\pgfpathlineto{\pgfqpoint{1.388572in}{1.429283in}}%
\pgfpathclose%
\pgfusepath{fill}%
\end{pgfscope}%
\begin{pgfscope}%
\pgfpathrectangle{\pgfqpoint{0.041670in}{0.041670in}}{\pgfqpoint{2.216660in}{2.216660in}}%
\pgfusepath{clip}%
\pgfsetbuttcap%
\pgfsetroundjoin%
\definecolor{currentfill}{rgb}{0.267004,0.004874,0.329415}%
\pgfsetfillcolor{currentfill}%
\pgfsetlinewidth{0.000000pt}%
\definecolor{currentstroke}{rgb}{0.000000,0.000000,0.000000}%
\pgfsetstrokecolor{currentstroke}%
\pgfsetdash{}{0pt}%
\pgfpathmoveto{\pgfqpoint{1.602651in}{0.648964in}}%
\pgfpathlineto{\pgfqpoint{1.605452in}{0.647729in}}%
\pgfpathlineto{\pgfqpoint{1.608259in}{0.646754in}}%
\pgfpathlineto{\pgfqpoint{1.611074in}{0.646043in}}%
\pgfpathlineto{\pgfqpoint{1.613896in}{0.645604in}}%
\pgfpathlineto{\pgfqpoint{1.600012in}{0.638338in}}%
\pgfpathlineto{\pgfqpoint{1.585669in}{0.631305in}}%
\pgfpathlineto{\pgfqpoint{1.570880in}{0.624515in}}%
\pgfpathlineto{\pgfqpoint{1.555661in}{0.617974in}}%
\pgfpathlineto{\pgfqpoint{1.553210in}{0.618596in}}%
\pgfpathlineto{\pgfqpoint{1.550764in}{0.619488in}}%
\pgfpathlineto{\pgfqpoint{1.548324in}{0.620647in}}%
\pgfpathlineto{\pgfqpoint{1.545891in}{0.622065in}}%
\pgfpathlineto{\pgfqpoint{1.560723in}{0.628433in}}%
\pgfpathlineto{\pgfqpoint{1.575136in}{0.635044in}}%
\pgfpathlineto{\pgfqpoint{1.589117in}{0.641890in}}%
\pgfpathlineto{\pgfqpoint{1.602651in}{0.648964in}}%
\pgfpathclose%
\pgfusepath{fill}%
\end{pgfscope}%
\begin{pgfscope}%
\pgfpathrectangle{\pgfqpoint{0.041670in}{0.041670in}}{\pgfqpoint{2.216660in}{2.216660in}}%
\pgfusepath{clip}%
\pgfsetbuttcap%
\pgfsetroundjoin%
\definecolor{currentfill}{rgb}{0.636902,0.856542,0.216620}%
\pgfsetfillcolor{currentfill}%
\pgfsetlinewidth{0.000000pt}%
\definecolor{currentstroke}{rgb}{0.000000,0.000000,0.000000}%
\pgfsetstrokecolor{currentstroke}%
\pgfsetdash{}{0pt}%
\pgfpathmoveto{\pgfqpoint{1.335544in}{1.547842in}}%
\pgfpathlineto{\pgfqpoint{1.339398in}{1.542768in}}%
\pgfpathlineto{\pgfqpoint{1.343249in}{1.537592in}}%
\pgfpathlineto{\pgfqpoint{1.347098in}{1.532315in}}%
\pgfpathlineto{\pgfqpoint{1.350944in}{1.526938in}}%
\pgfpathlineto{\pgfqpoint{1.350703in}{1.524375in}}%
\pgfpathlineto{\pgfqpoint{1.350291in}{1.521816in}}%
\pgfpathlineto{\pgfqpoint{1.349709in}{1.519264in}}%
\pgfpathlineto{\pgfqpoint{1.348956in}{1.516720in}}%
\pgfpathlineto{\pgfqpoint{1.345149in}{1.522335in}}%
\pgfpathlineto{\pgfqpoint{1.341340in}{1.527850in}}%
\pgfpathlineto{\pgfqpoint{1.337528in}{1.533264in}}%
\pgfpathlineto{\pgfqpoint{1.333714in}{1.538575in}}%
\pgfpathlineto{\pgfqpoint{1.334404in}{1.540882in}}%
\pgfpathlineto{\pgfqpoint{1.334939in}{1.543197in}}%
\pgfpathlineto{\pgfqpoint{1.335319in}{1.545517in}}%
\pgfpathlineto{\pgfqpoint{1.335544in}{1.547842in}}%
\pgfpathclose%
\pgfusepath{fill}%
\end{pgfscope}%
\begin{pgfscope}%
\pgfpathrectangle{\pgfqpoint{0.041670in}{0.041670in}}{\pgfqpoint{2.216660in}{2.216660in}}%
\pgfusepath{clip}%
\pgfsetbuttcap%
\pgfsetroundjoin%
\definecolor{currentfill}{rgb}{0.935904,0.898570,0.108131}%
\pgfsetfillcolor{currentfill}%
\pgfsetlinewidth{0.000000pt}%
\definecolor{currentstroke}{rgb}{0.000000,0.000000,0.000000}%
\pgfsetstrokecolor{currentstroke}%
\pgfsetdash{}{0pt}%
\pgfpathmoveto{\pgfqpoint{1.226955in}{1.655886in}}%
\pgfpathlineto{\pgfqpoint{1.229878in}{1.654905in}}%
\pgfpathlineto{\pgfqpoint{1.232799in}{1.653796in}}%
\pgfpathlineto{\pgfqpoint{1.235718in}{1.652560in}}%
\pgfpathlineto{\pgfqpoint{1.238635in}{1.651198in}}%
\pgfpathlineto{\pgfqpoint{1.240228in}{1.650322in}}%
\pgfpathlineto{\pgfqpoint{1.241761in}{1.649424in}}%
\pgfpathlineto{\pgfqpoint{1.243233in}{1.648504in}}%
\pgfpathlineto{\pgfqpoint{1.244642in}{1.647562in}}%
\pgfpathlineto{\pgfqpoint{1.241425in}{1.649106in}}%
\pgfpathlineto{\pgfqpoint{1.238205in}{1.650523in}}%
\pgfpathlineto{\pgfqpoint{1.234984in}{1.651814in}}%
\pgfpathlineto{\pgfqpoint{1.231760in}{1.652977in}}%
\pgfpathlineto{\pgfqpoint{1.230633in}{1.653730in}}%
\pgfpathlineto{\pgfqpoint{1.229455in}{1.654467in}}%
\pgfpathlineto{\pgfqpoint{1.228229in}{1.655186in}}%
\pgfpathlineto{\pgfqpoint{1.226955in}{1.655886in}}%
\pgfpathclose%
\pgfusepath{fill}%
\end{pgfscope}%
\begin{pgfscope}%
\pgfpathrectangle{\pgfqpoint{0.041670in}{0.041670in}}{\pgfqpoint{2.216660in}{2.216660in}}%
\pgfusepath{clip}%
\pgfsetbuttcap%
\pgfsetroundjoin%
\definecolor{currentfill}{rgb}{0.855810,0.888601,0.097452}%
\pgfsetfillcolor{currentfill}%
\pgfsetlinewidth{0.000000pt}%
\definecolor{currentstroke}{rgb}{0.000000,0.000000,0.000000}%
\pgfsetstrokecolor{currentstroke}%
\pgfsetdash{}{0pt}%
\pgfpathmoveto{\pgfqpoint{1.268144in}{1.630311in}}%
\pgfpathlineto{\pgfqpoint{1.271796in}{1.627738in}}%
\pgfpathlineto{\pgfqpoint{1.275446in}{1.625043in}}%
\pgfpathlineto{\pgfqpoint{1.279093in}{1.622228in}}%
\pgfpathlineto{\pgfqpoint{1.282739in}{1.619293in}}%
\pgfpathlineto{\pgfqpoint{1.283851in}{1.617765in}}%
\pgfpathlineto{\pgfqpoint{1.284861in}{1.616221in}}%
\pgfpathlineto{\pgfqpoint{1.285767in}{1.614662in}}%
\pgfpathlineto{\pgfqpoint{1.286568in}{1.613090in}}%
\pgfpathlineto{\pgfqpoint{1.282783in}{1.616248in}}%
\pgfpathlineto{\pgfqpoint{1.278997in}{1.619286in}}%
\pgfpathlineto{\pgfqpoint{1.275208in}{1.622203in}}%
\pgfpathlineto{\pgfqpoint{1.271417in}{1.624999in}}%
\pgfpathlineto{\pgfqpoint{1.270733in}{1.626345in}}%
\pgfpathlineto{\pgfqpoint{1.269959in}{1.627680in}}%
\pgfpathlineto{\pgfqpoint{1.269095in}{1.629002in}}%
\pgfpathlineto{\pgfqpoint{1.268144in}{1.630311in}}%
\pgfpathclose%
\pgfusepath{fill}%
\end{pgfscope}%
\begin{pgfscope}%
\pgfpathrectangle{\pgfqpoint{0.041670in}{0.041670in}}{\pgfqpoint{2.216660in}{2.216660in}}%
\pgfusepath{clip}%
\pgfsetbuttcap%
\pgfsetroundjoin%
\definecolor{currentfill}{rgb}{0.814576,0.883393,0.110347}%
\pgfsetfillcolor{currentfill}%
\pgfsetlinewidth{0.000000pt}%
\definecolor{currentstroke}{rgb}{0.000000,0.000000,0.000000}%
\pgfsetstrokecolor{currentstroke}%
\pgfsetdash{}{0pt}%
\pgfpathmoveto{\pgfqpoint{1.286568in}{1.613090in}}%
\pgfpathlineto{\pgfqpoint{1.290350in}{1.609815in}}%
\pgfpathlineto{\pgfqpoint{1.294129in}{1.606421in}}%
\pgfpathlineto{\pgfqpoint{1.297906in}{1.602912in}}%
\pgfpathlineto{\pgfqpoint{1.301681in}{1.599287in}}%
\pgfpathlineto{\pgfqpoint{1.302479in}{1.597476in}}%
\pgfpathlineto{\pgfqpoint{1.303156in}{1.595653in}}%
\pgfpathlineto{\pgfqpoint{1.303710in}{1.593820in}}%
\pgfpathlineto{\pgfqpoint{1.304141in}{1.591980in}}%
\pgfpathlineto{\pgfqpoint{1.300286in}{1.595835in}}%
\pgfpathlineto{\pgfqpoint{1.296428in}{1.599575in}}%
\pgfpathlineto{\pgfqpoint{1.292569in}{1.603199in}}%
\pgfpathlineto{\pgfqpoint{1.288707in}{1.606705in}}%
\pgfpathlineto{\pgfqpoint{1.288333in}{1.608313in}}%
\pgfpathlineto{\pgfqpoint{1.287852in}{1.609914in}}%
\pgfpathlineto{\pgfqpoint{1.287263in}{1.611507in}}%
\pgfpathlineto{\pgfqpoint{1.286568in}{1.613090in}}%
\pgfpathclose%
\pgfusepath{fill}%
\end{pgfscope}%
\begin{pgfscope}%
\pgfpathrectangle{\pgfqpoint{0.041670in}{0.041670in}}{\pgfqpoint{2.216660in}{2.216660in}}%
\pgfusepath{clip}%
\pgfsetbuttcap%
\pgfsetroundjoin%
\definecolor{currentfill}{rgb}{0.955300,0.901065,0.118128}%
\pgfsetfillcolor{currentfill}%
\pgfsetlinewidth{0.000000pt}%
\definecolor{currentstroke}{rgb}{0.000000,0.000000,0.000000}%
\pgfsetstrokecolor{currentstroke}%
\pgfsetdash{}{0pt}%
\pgfpathmoveto{\pgfqpoint{1.201355in}{1.663577in}}%
\pgfpathlineto{\pgfqpoint{1.203131in}{1.663533in}}%
\pgfpathlineto{\pgfqpoint{1.204905in}{1.663359in}}%
\pgfpathlineto{\pgfqpoint{1.206678in}{1.663055in}}%
\pgfpathlineto{\pgfqpoint{1.208450in}{1.662622in}}%
\pgfpathlineto{\pgfqpoint{1.210185in}{1.662190in}}%
\pgfpathlineto{\pgfqpoint{1.211891in}{1.661734in}}%
\pgfpathlineto{\pgfqpoint{1.213565in}{1.661252in}}%
\pgfpathlineto{\pgfqpoint{1.215206in}{1.660747in}}%
\pgfpathlineto{\pgfqpoint{1.213014in}{1.661296in}}%
\pgfpathlineto{\pgfqpoint{1.210820in}{1.661717in}}%
\pgfpathlineto{\pgfqpoint{1.208624in}{1.662008in}}%
\pgfpathlineto{\pgfqpoint{1.206427in}{1.662169in}}%
\pgfpathlineto{\pgfqpoint{1.205195in}{1.662549in}}%
\pgfpathlineto{\pgfqpoint{1.203939in}{1.662910in}}%
\pgfpathlineto{\pgfqpoint{1.202658in}{1.663253in}}%
\pgfpathlineto{\pgfqpoint{1.201355in}{1.663577in}}%
\pgfpathclose%
\pgfusepath{fill}%
\end{pgfscope}%
\begin{pgfscope}%
\pgfpathrectangle{\pgfqpoint{0.041670in}{0.041670in}}{\pgfqpoint{2.216660in}{2.216660in}}%
\pgfusepath{clip}%
\pgfsetbuttcap%
\pgfsetroundjoin%
\definecolor{currentfill}{rgb}{0.935904,0.898570,0.108131}%
\pgfsetfillcolor{currentfill}%
\pgfsetlinewidth{0.000000pt}%
\definecolor{currentstroke}{rgb}{0.000000,0.000000,0.000000}%
\pgfsetstrokecolor{currentstroke}%
\pgfsetdash{}{0pt}%
\pgfpathmoveto{\pgfqpoint{1.127190in}{1.652293in}}%
\pgfpathlineto{\pgfqpoint{1.123907in}{1.651088in}}%
\pgfpathlineto{\pgfqpoint{1.120625in}{1.649754in}}%
\pgfpathlineto{\pgfqpoint{1.117346in}{1.648294in}}%
\pgfpathlineto{\pgfqpoint{1.114068in}{1.646708in}}%
\pgfpathlineto{\pgfqpoint{1.115421in}{1.647668in}}%
\pgfpathlineto{\pgfqpoint{1.116838in}{1.648607in}}%
\pgfpathlineto{\pgfqpoint{1.118316in}{1.649525in}}%
\pgfpathlineto{\pgfqpoint{1.119856in}{1.650421in}}%
\pgfpathlineto{\pgfqpoint{1.122844in}{1.651822in}}%
\pgfpathlineto{\pgfqpoint{1.125834in}{1.653097in}}%
\pgfpathlineto{\pgfqpoint{1.128826in}{1.654244in}}%
\pgfpathlineto{\pgfqpoint{1.131820in}{1.655264in}}%
\pgfpathlineto{\pgfqpoint{1.130588in}{1.654548in}}%
\pgfpathlineto{\pgfqpoint{1.129405in}{1.653813in}}%
\pgfpathlineto{\pgfqpoint{1.128272in}{1.653061in}}%
\pgfpathlineto{\pgfqpoint{1.127190in}{1.652293in}}%
\pgfpathclose%
\pgfusepath{fill}%
\end{pgfscope}%
\begin{pgfscope}%
\pgfpathrectangle{\pgfqpoint{0.041670in}{0.041670in}}{\pgfqpoint{2.216660in}{2.216660in}}%
\pgfusepath{clip}%
\pgfsetbuttcap%
\pgfsetroundjoin%
\definecolor{currentfill}{rgb}{0.955300,0.901065,0.118128}%
\pgfsetfillcolor{currentfill}%
\pgfsetlinewidth{0.000000pt}%
\definecolor{currentstroke}{rgb}{0.000000,0.000000,0.000000}%
\pgfsetstrokecolor{currentstroke}%
\pgfsetdash{}{0pt}%
\pgfpathmoveto{\pgfqpoint{1.152409in}{1.661816in}}%
\pgfpathlineto{\pgfqpoint{1.150123in}{1.661626in}}%
\pgfpathlineto{\pgfqpoint{1.147838in}{1.661306in}}%
\pgfpathlineto{\pgfqpoint{1.145555in}{1.660856in}}%
\pgfpathlineto{\pgfqpoint{1.143274in}{1.660277in}}%
\pgfpathlineto{\pgfqpoint{1.144884in}{1.660804in}}%
\pgfpathlineto{\pgfqpoint{1.146529in}{1.661307in}}%
\pgfpathlineto{\pgfqpoint{1.148207in}{1.661786in}}%
\pgfpathlineto{\pgfqpoint{1.149916in}{1.662239in}}%
\pgfpathlineto{\pgfqpoint{1.151783in}{1.662696in}}%
\pgfpathlineto{\pgfqpoint{1.153653in}{1.663024in}}%
\pgfpathlineto{\pgfqpoint{1.155523in}{1.663222in}}%
\pgfpathlineto{\pgfqpoint{1.157395in}{1.663290in}}%
\pgfpathlineto{\pgfqpoint{1.156112in}{1.662949in}}%
\pgfpathlineto{\pgfqpoint{1.154853in}{1.662590in}}%
\pgfpathlineto{\pgfqpoint{1.153618in}{1.662212in}}%
\pgfpathlineto{\pgfqpoint{1.152409in}{1.661816in}}%
\pgfpathclose%
\pgfusepath{fill}%
\end{pgfscope}%
\begin{pgfscope}%
\pgfpathrectangle{\pgfqpoint{0.041670in}{0.041670in}}{\pgfqpoint{2.216660in}{2.216660in}}%
\pgfusepath{clip}%
\pgfsetbuttcap%
\pgfsetroundjoin%
\definecolor{currentfill}{rgb}{0.283072,0.130895,0.449241}%
\pgfsetfillcolor{currentfill}%
\pgfsetlinewidth{0.000000pt}%
\definecolor{currentstroke}{rgb}{0.000000,0.000000,0.000000}%
\pgfsetstrokecolor{currentstroke}%
\pgfsetdash{}{0pt}%
\pgfpathmoveto{\pgfqpoint{1.479501in}{0.748378in}}%
\pgfpathlineto{\pgfqpoint{1.481832in}{0.741545in}}%
\pgfpathlineto{\pgfqpoint{1.484165in}{0.734847in}}%
\pgfpathlineto{\pgfqpoint{1.486500in}{0.728290in}}%
\pgfpathlineto{\pgfqpoint{1.488837in}{0.721876in}}%
\pgfpathlineto{\pgfqpoint{1.475940in}{0.716783in}}%
\pgfpathlineto{\pgfqpoint{1.462722in}{0.711907in}}%
\pgfpathlineto{\pgfqpoint{1.449195in}{0.707253in}}%
\pgfpathlineto{\pgfqpoint{1.435375in}{0.702827in}}%
\pgfpathlineto{\pgfqpoint{1.433438in}{0.709402in}}%
\pgfpathlineto{\pgfqpoint{1.431502in}{0.716120in}}%
\pgfpathlineto{\pgfqpoint{1.429568in}{0.722978in}}%
\pgfpathlineto{\pgfqpoint{1.427636in}{0.729973in}}%
\pgfpathlineto{\pgfqpoint{1.441042in}{0.734249in}}%
\pgfpathlineto{\pgfqpoint{1.454164in}{0.738746in}}%
\pgfpathlineto{\pgfqpoint{1.466988in}{0.743457in}}%
\pgfpathlineto{\pgfqpoint{1.479501in}{0.748378in}}%
\pgfpathclose%
\pgfusepath{fill}%
\end{pgfscope}%
\begin{pgfscope}%
\pgfpathrectangle{\pgfqpoint{0.041670in}{0.041670in}}{\pgfqpoint{2.216660in}{2.216660in}}%
\pgfusepath{clip}%
\pgfsetbuttcap%
\pgfsetroundjoin%
\definecolor{currentfill}{rgb}{0.282327,0.094955,0.417331}%
\pgfsetfillcolor{currentfill}%
\pgfsetlinewidth{0.000000pt}%
\definecolor{currentstroke}{rgb}{0.000000,0.000000,0.000000}%
\pgfsetstrokecolor{currentstroke}%
\pgfsetdash{}{0pt}%
\pgfpathmoveto{\pgfqpoint{1.488837in}{0.721876in}}%
\pgfpathlineto{\pgfqpoint{1.491176in}{0.715610in}}%
\pgfpathlineto{\pgfqpoint{1.493518in}{0.709496in}}%
\pgfpathlineto{\pgfqpoint{1.495862in}{0.703539in}}%
\pgfpathlineto{\pgfqpoint{1.498208in}{0.697741in}}%
\pgfpathlineto{\pgfqpoint{1.484927in}{0.692476in}}%
\pgfpathlineto{\pgfqpoint{1.471313in}{0.687435in}}%
\pgfpathlineto{\pgfqpoint{1.457381in}{0.682624in}}%
\pgfpathlineto{\pgfqpoint{1.443145in}{0.678048in}}%
\pgfpathlineto{\pgfqpoint{1.441200in}{0.684007in}}%
\pgfpathlineto{\pgfqpoint{1.439256in}{0.690125in}}%
\pgfpathlineto{\pgfqpoint{1.437315in}{0.696400in}}%
\pgfpathlineto{\pgfqpoint{1.435375in}{0.702827in}}%
\pgfpathlineto{\pgfqpoint{1.449195in}{0.707253in}}%
\pgfpathlineto{\pgfqpoint{1.462722in}{0.711907in}}%
\pgfpathlineto{\pgfqpoint{1.475940in}{0.716783in}}%
\pgfpathlineto{\pgfqpoint{1.488837in}{0.721876in}}%
\pgfpathclose%
\pgfusepath{fill}%
\end{pgfscope}%
\begin{pgfscope}%
\pgfpathrectangle{\pgfqpoint{0.041670in}{0.041670in}}{\pgfqpoint{2.216660in}{2.216660in}}%
\pgfusepath{clip}%
\pgfsetbuttcap%
\pgfsetroundjoin%
\definecolor{currentfill}{rgb}{0.896320,0.893616,0.096335}%
\pgfsetfillcolor{currentfill}%
\pgfsetlinewidth{0.000000pt}%
\definecolor{currentstroke}{rgb}{0.000000,0.000000,0.000000}%
\pgfsetstrokecolor{currentstroke}%
\pgfsetdash{}{0pt}%
\pgfpathmoveto{\pgfqpoint{1.249628in}{1.643600in}}%
\pgfpathlineto{\pgfqpoint{1.253093in}{1.641732in}}%
\pgfpathlineto{\pgfqpoint{1.256556in}{1.639740in}}%
\pgfpathlineto{\pgfqpoint{1.260017in}{1.637624in}}%
\pgfpathlineto{\pgfqpoint{1.263476in}{1.635385in}}%
\pgfpathlineto{\pgfqpoint{1.264770in}{1.634143in}}%
\pgfpathlineto{\pgfqpoint{1.265980in}{1.632883in}}%
\pgfpathlineto{\pgfqpoint{1.267105in}{1.631605in}}%
\pgfpathlineto{\pgfqpoint{1.268144in}{1.630311in}}%
\pgfpathlineto{\pgfqpoint{1.264489in}{1.632763in}}%
\pgfpathlineto{\pgfqpoint{1.260833in}{1.635091in}}%
\pgfpathlineto{\pgfqpoint{1.257174in}{1.637295in}}%
\pgfpathlineto{\pgfqpoint{1.253513in}{1.639374in}}%
\pgfpathlineto{\pgfqpoint{1.252649in}{1.640452in}}%
\pgfpathlineto{\pgfqpoint{1.251713in}{1.641516in}}%
\pgfpathlineto{\pgfqpoint{1.250705in}{1.642566in}}%
\pgfpathlineto{\pgfqpoint{1.249628in}{1.643600in}}%
\pgfpathclose%
\pgfusepath{fill}%
\end{pgfscope}%
\begin{pgfscope}%
\pgfpathrectangle{\pgfqpoint{0.041670in}{0.041670in}}{\pgfqpoint{2.216660in}{2.216660in}}%
\pgfusepath{clip}%
\pgfsetbuttcap%
\pgfsetroundjoin%
\definecolor{currentfill}{rgb}{0.212395,0.359683,0.551710}%
\pgfsetfillcolor{currentfill}%
\pgfsetlinewidth{0.000000pt}%
\definecolor{currentstroke}{rgb}{0.000000,0.000000,0.000000}%
\pgfsetstrokecolor{currentstroke}%
\pgfsetdash{}{0pt}%
\pgfpathmoveto{\pgfqpoint{0.945099in}{0.939953in}}%
\pgfpathlineto{\pgfqpoint{0.942871in}{0.930927in}}%
\pgfpathlineto{\pgfqpoint{0.940644in}{0.921951in}}%
\pgfpathlineto{\pgfqpoint{0.938416in}{0.913028in}}%
\pgfpathlineto{\pgfqpoint{0.936189in}{0.904161in}}%
\pgfpathlineto{\pgfqpoint{0.925617in}{0.908249in}}%
\pgfpathlineto{\pgfqpoint{0.915317in}{0.912506in}}%
\pgfpathlineto{\pgfqpoint{0.905300in}{0.916926in}}%
\pgfpathlineto{\pgfqpoint{0.895575in}{0.921505in}}%
\pgfpathlineto{\pgfqpoint{0.898167in}{0.930193in}}%
\pgfpathlineto{\pgfqpoint{0.900760in}{0.938937in}}%
\pgfpathlineto{\pgfqpoint{0.903352in}{0.947734in}}%
\pgfpathlineto{\pgfqpoint{0.905945in}{0.956582in}}%
\pgfpathlineto{\pgfqpoint{0.915322in}{0.952191in}}%
\pgfpathlineto{\pgfqpoint{0.924979in}{0.947953in}}%
\pgfpathlineto{\pgfqpoint{0.934908in}{0.943872in}}%
\pgfpathlineto{\pgfqpoint{0.945099in}{0.939953in}}%
\pgfpathclose%
\pgfusepath{fill}%
\end{pgfscope}%
\begin{pgfscope}%
\pgfpathrectangle{\pgfqpoint{0.041670in}{0.041670in}}{\pgfqpoint{2.216660in}{2.216660in}}%
\pgfusepath{clip}%
\pgfsetbuttcap%
\pgfsetroundjoin%
\definecolor{currentfill}{rgb}{0.855810,0.888601,0.097452}%
\pgfsetfillcolor{currentfill}%
\pgfsetlinewidth{0.000000pt}%
\definecolor{currentstroke}{rgb}{0.000000,0.000000,0.000000}%
\pgfsetstrokecolor{currentstroke}%
\pgfsetdash{}{0pt}%
\pgfpathmoveto{\pgfqpoint{1.087961in}{1.623795in}}%
\pgfpathlineto{\pgfqpoint{1.084147in}{1.620948in}}%
\pgfpathlineto{\pgfqpoint{1.080336in}{1.617980in}}%
\pgfpathlineto{\pgfqpoint{1.076526in}{1.614892in}}%
\pgfpathlineto{\pgfqpoint{1.072719in}{1.611684in}}%
\pgfpathlineto{\pgfqpoint{1.073426in}{1.613266in}}%
\pgfpathlineto{\pgfqpoint{1.074239in}{1.614836in}}%
\pgfpathlineto{\pgfqpoint{1.075156in}{1.616393in}}%
\pgfpathlineto{\pgfqpoint{1.076177in}{1.617936in}}%
\pgfpathlineto{\pgfqpoint{1.079858in}{1.620919in}}%
\pgfpathlineto{\pgfqpoint{1.083542in}{1.623783in}}%
\pgfpathlineto{\pgfqpoint{1.087228in}{1.626526in}}%
\pgfpathlineto{\pgfqpoint{1.090916in}{1.629149in}}%
\pgfpathlineto{\pgfqpoint{1.090043in}{1.627827in}}%
\pgfpathlineto{\pgfqpoint{1.089259in}{1.626494in}}%
\pgfpathlineto{\pgfqpoint{1.088564in}{1.625149in}}%
\pgfpathlineto{\pgfqpoint{1.087961in}{1.623795in}}%
\pgfpathclose%
\pgfusepath{fill}%
\end{pgfscope}%
\begin{pgfscope}%
\pgfpathrectangle{\pgfqpoint{0.041670in}{0.041670in}}{\pgfqpoint{2.216660in}{2.216660in}}%
\pgfusepath{clip}%
\pgfsetbuttcap%
\pgfsetroundjoin%
\definecolor{currentfill}{rgb}{0.280255,0.165693,0.476498}%
\pgfsetfillcolor{currentfill}%
\pgfsetlinewidth{0.000000pt}%
\definecolor{currentstroke}{rgb}{0.000000,0.000000,0.000000}%
\pgfsetstrokecolor{currentstroke}%
\pgfsetdash{}{0pt}%
\pgfpathmoveto{\pgfqpoint{1.470194in}{0.776994in}}%
\pgfpathlineto{\pgfqpoint{1.472518in}{0.769655in}}%
\pgfpathlineto{\pgfqpoint{1.474844in}{0.762437in}}%
\pgfpathlineto{\pgfqpoint{1.477172in}{0.755344in}}%
\pgfpathlineto{\pgfqpoint{1.479501in}{0.748378in}}%
\pgfpathlineto{\pgfqpoint{1.466988in}{0.743457in}}%
\pgfpathlineto{\pgfqpoint{1.454164in}{0.738746in}}%
\pgfpathlineto{\pgfqpoint{1.441042in}{0.734249in}}%
\pgfpathlineto{\pgfqpoint{1.427636in}{0.729973in}}%
\pgfpathlineto{\pgfqpoint{1.425705in}{0.737099in}}%
\pgfpathlineto{\pgfqpoint{1.423776in}{0.744353in}}%
\pgfpathlineto{\pgfqpoint{1.421848in}{0.751732in}}%
\pgfpathlineto{\pgfqpoint{1.419922in}{0.759231in}}%
\pgfpathlineto{\pgfqpoint{1.432915in}{0.763358in}}%
\pgfpathlineto{\pgfqpoint{1.445633in}{0.767697in}}%
\pgfpathlineto{\pgfqpoint{1.458064in}{0.772244in}}%
\pgfpathlineto{\pgfqpoint{1.470194in}{0.776994in}}%
\pgfpathclose%
\pgfusepath{fill}%
\end{pgfscope}%
\begin{pgfscope}%
\pgfpathrectangle{\pgfqpoint{0.041670in}{0.041670in}}{\pgfqpoint{2.216660in}{2.216660in}}%
\pgfusepath{clip}%
\pgfsetbuttcap%
\pgfsetroundjoin%
\definecolor{currentfill}{rgb}{0.487026,0.823929,0.312321}%
\pgfsetfillcolor{currentfill}%
\pgfsetlinewidth{0.000000pt}%
\definecolor{currentstroke}{rgb}{0.000000,0.000000,0.000000}%
\pgfsetstrokecolor{currentstroke}%
\pgfsetdash{}{0pt}%
\pgfpathmoveto{\pgfqpoint{1.002370in}{1.479926in}}%
\pgfpathlineto{\pgfqpoint{0.998704in}{1.473548in}}%
\pgfpathlineto{\pgfqpoint{0.995040in}{1.467079in}}%
\pgfpathlineto{\pgfqpoint{0.991378in}{1.460523in}}%
\pgfpathlineto{\pgfqpoint{0.987719in}{1.453879in}}%
\pgfpathlineto{\pgfqpoint{0.985877in}{1.456807in}}%
\pgfpathlineto{\pgfqpoint{0.984230in}{1.459760in}}%
\pgfpathlineto{\pgfqpoint{0.982781in}{1.462736in}}%
\pgfpathlineto{\pgfqpoint{0.981530in}{1.465730in}}%
\pgfpathlineto{\pgfqpoint{0.985300in}{1.472137in}}%
\pgfpathlineto{\pgfqpoint{0.989074in}{1.478458in}}%
\pgfpathlineto{\pgfqpoint{0.992849in}{1.484690in}}%
\pgfpathlineto{\pgfqpoint{0.996627in}{1.490833in}}%
\pgfpathlineto{\pgfqpoint{0.997789in}{1.488076in}}%
\pgfpathlineto{\pgfqpoint{0.999135in}{1.485338in}}%
\pgfpathlineto{\pgfqpoint{1.000662in}{1.482620in}}%
\pgfpathlineto{\pgfqpoint{1.002370in}{1.479926in}}%
\pgfpathclose%
\pgfusepath{fill}%
\end{pgfscope}%
\begin{pgfscope}%
\pgfpathrectangle{\pgfqpoint{0.041670in}{0.041670in}}{\pgfqpoint{2.216660in}{2.216660in}}%
\pgfusepath{clip}%
\pgfsetbuttcap%
\pgfsetroundjoin%
\definecolor{currentfill}{rgb}{0.195860,0.395433,0.555276}%
\pgfsetfillcolor{currentfill}%
\pgfsetlinewidth{0.000000pt}%
\definecolor{currentstroke}{rgb}{0.000000,0.000000,0.000000}%
\pgfsetstrokecolor{currentstroke}%
\pgfsetdash{}{0pt}%
\pgfpathmoveto{\pgfqpoint{1.451382in}{0.996262in}}%
\pgfpathlineto{\pgfqpoint{1.454051in}{0.987290in}}%
\pgfpathlineto{\pgfqpoint{1.456720in}{0.978355in}}%
\pgfpathlineto{\pgfqpoint{1.459388in}{0.969460in}}%
\pgfpathlineto{\pgfqpoint{1.462056in}{0.960609in}}%
\pgfpathlineto{\pgfqpoint{1.452937in}{0.956087in}}%
\pgfpathlineto{\pgfqpoint{1.443529in}{0.951712in}}%
\pgfpathlineto{\pgfqpoint{1.433840in}{0.947492in}}%
\pgfpathlineto{\pgfqpoint{1.423882in}{0.943429in}}%
\pgfpathlineto{\pgfqpoint{1.421569in}{0.952464in}}%
\pgfpathlineto{\pgfqpoint{1.419257in}{0.961542in}}%
\pgfpathlineto{\pgfqpoint{1.416944in}{0.970661in}}%
\pgfpathlineto{\pgfqpoint{1.414631in}{0.979816in}}%
\pgfpathlineto{\pgfqpoint{1.424217in}{0.983705in}}%
\pgfpathlineto{\pgfqpoint{1.433544in}{0.987746in}}%
\pgfpathlineto{\pgfqpoint{1.442602in}{0.991933in}}%
\pgfpathlineto{\pgfqpoint{1.451382in}{0.996262in}}%
\pgfpathclose%
\pgfusepath{fill}%
\end{pgfscope}%
\begin{pgfscope}%
\pgfpathrectangle{\pgfqpoint{0.041670in}{0.041670in}}{\pgfqpoint{2.216660in}{2.216660in}}%
\pgfusepath{clip}%
\pgfsetbuttcap%
\pgfsetroundjoin%
\definecolor{currentfill}{rgb}{0.233603,0.313828,0.543914}%
\pgfsetfillcolor{currentfill}%
\pgfsetlinewidth{0.000000pt}%
\definecolor{currentstroke}{rgb}{0.000000,0.000000,0.000000}%
\pgfsetstrokecolor{currentstroke}%
\pgfsetdash{}{0pt}%
\pgfpathmoveto{\pgfqpoint{1.845657in}{0.885354in}}%
\pgfpathlineto{\pgfqpoint{1.849645in}{0.895859in}}%
\pgfpathlineto{\pgfqpoint{1.853653in}{0.906818in}}%
\pgfpathlineto{\pgfqpoint{1.857681in}{0.918238in}}%
\pgfpathlineto{\pgfqpoint{1.861730in}{0.930128in}}%
\pgfpathlineto{\pgfqpoint{1.852816in}{0.919039in}}%
\pgfpathlineto{\pgfqpoint{1.843186in}{0.908086in}}%
\pgfpathlineto{\pgfqpoint{1.832848in}{0.897280in}}%
\pgfpathlineto{\pgfqpoint{1.821811in}{0.886633in}}%
\pgfpathlineto{\pgfqpoint{1.817976in}{0.874932in}}%
\pgfpathlineto{\pgfqpoint{1.814161in}{0.863702in}}%
\pgfpathlineto{\pgfqpoint{1.810365in}{0.852936in}}%
\pgfpathlineto{\pgfqpoint{1.806589in}{0.842626in}}%
\pgfpathlineto{\pgfqpoint{1.817387in}{0.853084in}}%
\pgfpathlineto{\pgfqpoint{1.827503in}{0.863699in}}%
\pgfpathlineto{\pgfqpoint{1.836928in}{0.874460in}}%
\pgfpathlineto{\pgfqpoint{1.845657in}{0.885354in}}%
\pgfpathclose%
\pgfusepath{fill}%
\end{pgfscope}%
\begin{pgfscope}%
\pgfpathrectangle{\pgfqpoint{0.041670in}{0.041670in}}{\pgfqpoint{2.216660in}{2.216660in}}%
\pgfusepath{clip}%
\pgfsetbuttcap%
\pgfsetroundjoin%
\definecolor{currentfill}{rgb}{0.220124,0.725509,0.466226}%
\pgfsetfillcolor{currentfill}%
\pgfsetlinewidth{0.000000pt}%
\definecolor{currentstroke}{rgb}{0.000000,0.000000,0.000000}%
\pgfsetstrokecolor{currentstroke}%
\pgfsetdash{}{0pt}%
\pgfpathmoveto{\pgfqpoint{1.407439in}{1.356987in}}%
\pgfpathlineto{\pgfqpoint{1.410943in}{1.349208in}}%
\pgfpathlineto{\pgfqpoint{1.414444in}{1.341371in}}%
\pgfpathlineto{\pgfqpoint{1.417943in}{1.333476in}}%
\pgfpathlineto{\pgfqpoint{1.421440in}{1.325528in}}%
\pgfpathlineto{\pgfqpoint{1.418045in}{1.321820in}}%
\pgfpathlineto{\pgfqpoint{1.414406in}{1.318166in}}%
\pgfpathlineto{\pgfqpoint{1.410528in}{1.314567in}}%
\pgfpathlineto{\pgfqpoint{1.406413in}{1.311028in}}%
\pgfpathlineto{\pgfqpoint{1.403127in}{1.319204in}}%
\pgfpathlineto{\pgfqpoint{1.399839in}{1.327325in}}%
\pgfpathlineto{\pgfqpoint{1.396549in}{1.335389in}}%
\pgfpathlineto{\pgfqpoint{1.393257in}{1.343394in}}%
\pgfpathlineto{\pgfqpoint{1.397139in}{1.346711in}}%
\pgfpathlineto{\pgfqpoint{1.400799in}{1.350085in}}%
\pgfpathlineto{\pgfqpoint{1.404234in}{1.353511in}}%
\pgfpathlineto{\pgfqpoint{1.407439in}{1.356987in}}%
\pgfpathclose%
\pgfusepath{fill}%
\end{pgfscope}%
\begin{pgfscope}%
\pgfpathrectangle{\pgfqpoint{0.041670in}{0.041670in}}{\pgfqpoint{2.216660in}{2.216660in}}%
\pgfusepath{clip}%
\pgfsetbuttcap%
\pgfsetroundjoin%
\definecolor{currentfill}{rgb}{0.814576,0.883393,0.110347}%
\pgfsetfillcolor{currentfill}%
\pgfsetlinewidth{0.000000pt}%
\definecolor{currentstroke}{rgb}{0.000000,0.000000,0.000000}%
\pgfsetstrokecolor{currentstroke}%
\pgfsetdash{}{0pt}%
\pgfpathmoveto{\pgfqpoint{1.070961in}{1.605271in}}%
\pgfpathlineto{\pgfqpoint{1.067090in}{1.601713in}}%
\pgfpathlineto{\pgfqpoint{1.063221in}{1.598038in}}%
\pgfpathlineto{\pgfqpoint{1.059354in}{1.594246in}}%
\pgfpathlineto{\pgfqpoint{1.055489in}{1.590339in}}%
\pgfpathlineto{\pgfqpoint{1.055811in}{1.592184in}}%
\pgfpathlineto{\pgfqpoint{1.056255in}{1.594024in}}%
\pgfpathlineto{\pgfqpoint{1.056823in}{1.595856in}}%
\pgfpathlineto{\pgfqpoint{1.057513in}{1.597677in}}%
\pgfpathlineto{\pgfqpoint{1.061311in}{1.601353in}}%
\pgfpathlineto{\pgfqpoint{1.065111in}{1.604913in}}%
\pgfpathlineto{\pgfqpoint{1.068914in}{1.608357in}}%
\pgfpathlineto{\pgfqpoint{1.072719in}{1.611684in}}%
\pgfpathlineto{\pgfqpoint{1.072118in}{1.610092in}}%
\pgfpathlineto{\pgfqpoint{1.071625in}{1.608491in}}%
\pgfpathlineto{\pgfqpoint{1.071239in}{1.606884in}}%
\pgfpathlineto{\pgfqpoint{1.070961in}{1.605271in}}%
\pgfpathclose%
\pgfusepath{fill}%
\end{pgfscope}%
\begin{pgfscope}%
\pgfpathrectangle{\pgfqpoint{0.041670in}{0.041670in}}{\pgfqpoint{2.216660in}{2.216660in}}%
\pgfusepath{clip}%
\pgfsetbuttcap%
\pgfsetroundjoin%
\definecolor{currentfill}{rgb}{0.279566,0.067836,0.391917}%
\pgfsetfillcolor{currentfill}%
\pgfsetlinewidth{0.000000pt}%
\definecolor{currentstroke}{rgb}{0.000000,0.000000,0.000000}%
\pgfsetstrokecolor{currentstroke}%
\pgfsetdash{}{0pt}%
\pgfpathmoveto{\pgfqpoint{1.498208in}{0.697741in}}%
\pgfpathlineto{\pgfqpoint{1.500558in}{0.692107in}}%
\pgfpathlineto{\pgfqpoint{1.502910in}{0.686642in}}%
\pgfpathlineto{\pgfqpoint{1.505265in}{0.681350in}}%
\pgfpathlineto{\pgfqpoint{1.507623in}{0.676234in}}%
\pgfpathlineto{\pgfqpoint{1.493955in}{0.670798in}}%
\pgfpathlineto{\pgfqpoint{1.479944in}{0.665593in}}%
\pgfpathlineto{\pgfqpoint{1.465605in}{0.660624in}}%
\pgfpathlineto{\pgfqpoint{1.450953in}{0.655898in}}%
\pgfpathlineto{\pgfqpoint{1.448997in}{0.661175in}}%
\pgfpathlineto{\pgfqpoint{1.447044in}{0.666628in}}%
\pgfpathlineto{\pgfqpoint{1.445094in}{0.672254in}}%
\pgfpathlineto{\pgfqpoint{1.443145in}{0.678048in}}%
\pgfpathlineto{\pgfqpoint{1.457381in}{0.682624in}}%
\pgfpathlineto{\pgfqpoint{1.471313in}{0.687435in}}%
\pgfpathlineto{\pgfqpoint{1.484927in}{0.692476in}}%
\pgfpathlineto{\pgfqpoint{1.498208in}{0.697741in}}%
\pgfpathclose%
\pgfusepath{fill}%
\end{pgfscope}%
\begin{pgfscope}%
\pgfpathrectangle{\pgfqpoint{0.041670in}{0.041670in}}{\pgfqpoint{2.216660in}{2.216660in}}%
\pgfusepath{clip}%
\pgfsetbuttcap%
\pgfsetroundjoin%
\definecolor{currentfill}{rgb}{0.636902,0.856542,0.216620}%
\pgfsetfillcolor{currentfill}%
\pgfsetlinewidth{0.000000pt}%
\definecolor{currentstroke}{rgb}{0.000000,0.000000,0.000000}%
\pgfsetstrokecolor{currentstroke}%
\pgfsetdash{}{0pt}%
\pgfpathmoveto{\pgfqpoint{1.026938in}{1.536534in}}%
\pgfpathlineto{\pgfqpoint{1.023142in}{1.531170in}}%
\pgfpathlineto{\pgfqpoint{1.019347in}{1.525704in}}%
\pgfpathlineto{\pgfqpoint{1.015555in}{1.520136in}}%
\pgfpathlineto{\pgfqpoint{1.011764in}{1.514469in}}%
\pgfpathlineto{\pgfqpoint{1.010861in}{1.517002in}}%
\pgfpathlineto{\pgfqpoint{1.010128in}{1.519547in}}%
\pgfpathlineto{\pgfqpoint{1.009565in}{1.522100in}}%
\pgfpathlineto{\pgfqpoint{1.009172in}{1.524660in}}%
\pgfpathlineto{\pgfqpoint{1.013014in}{1.530089in}}%
\pgfpathlineto{\pgfqpoint{1.016860in}{1.535420in}}%
\pgfpathlineto{\pgfqpoint{1.020708in}{1.540649in}}%
\pgfpathlineto{\pgfqpoint{1.024558in}{1.545776in}}%
\pgfpathlineto{\pgfqpoint{1.024921in}{1.543454in}}%
\pgfpathlineto{\pgfqpoint{1.025439in}{1.541139in}}%
\pgfpathlineto{\pgfqpoint{1.026112in}{1.538831in}}%
\pgfpathlineto{\pgfqpoint{1.026938in}{1.536534in}}%
\pgfpathclose%
\pgfusepath{fill}%
\end{pgfscope}%
\begin{pgfscope}%
\pgfpathrectangle{\pgfqpoint{0.041670in}{0.041670in}}{\pgfqpoint{2.216660in}{2.216660in}}%
\pgfusepath{clip}%
\pgfsetbuttcap%
\pgfsetroundjoin%
\definecolor{currentfill}{rgb}{0.896320,0.893616,0.096335}%
\pgfsetfillcolor{currentfill}%
\pgfsetlinewidth{0.000000pt}%
\definecolor{currentstroke}{rgb}{0.000000,0.000000,0.000000}%
\pgfsetstrokecolor{currentstroke}%
\pgfsetdash{}{0pt}%
\pgfpathmoveto{\pgfqpoint{1.105690in}{1.638405in}}%
\pgfpathlineto{\pgfqpoint{1.101993in}{1.636278in}}%
\pgfpathlineto{\pgfqpoint{1.098299in}{1.634025in}}%
\pgfpathlineto{\pgfqpoint{1.094606in}{1.631648in}}%
\pgfpathlineto{\pgfqpoint{1.090916in}{1.629149in}}%
\pgfpathlineto{\pgfqpoint{1.091877in}{1.630456in}}%
\pgfpathlineto{\pgfqpoint{1.092926in}{1.631748in}}%
\pgfpathlineto{\pgfqpoint{1.094060in}{1.633024in}}%
\pgfpathlineto{\pgfqpoint{1.095280in}{1.634282in}}%
\pgfpathlineto{\pgfqpoint{1.098787in}{1.636567in}}%
\pgfpathlineto{\pgfqpoint{1.102296in}{1.638730in}}%
\pgfpathlineto{\pgfqpoint{1.105807in}{1.640768in}}%
\pgfpathlineto{\pgfqpoint{1.109320in}{1.642681in}}%
\pgfpathlineto{\pgfqpoint{1.108305in}{1.641633in}}%
\pgfpathlineto{\pgfqpoint{1.107361in}{1.640570in}}%
\pgfpathlineto{\pgfqpoint{1.106489in}{1.639494in}}%
\pgfpathlineto{\pgfqpoint{1.105690in}{1.638405in}}%
\pgfpathclose%
\pgfusepath{fill}%
\end{pgfscope}%
\begin{pgfscope}%
\pgfpathrectangle{\pgfqpoint{0.041670in}{0.041670in}}{\pgfqpoint{2.216660in}{2.216660in}}%
\pgfusepath{clip}%
\pgfsetbuttcap%
\pgfsetroundjoin%
\definecolor{currentfill}{rgb}{0.762373,0.876424,0.137064}%
\pgfsetfillcolor{currentfill}%
\pgfsetlinewidth{0.000000pt}%
\definecolor{currentstroke}{rgb}{0.000000,0.000000,0.000000}%
\pgfsetstrokecolor{currentstroke}%
\pgfsetdash{}{0pt}%
\pgfpathmoveto{\pgfqpoint{1.304141in}{1.591980in}}%
\pgfpathlineto{\pgfqpoint{1.307994in}{1.588010in}}%
\pgfpathlineto{\pgfqpoint{1.311844in}{1.583927in}}%
\pgfpathlineto{\pgfqpoint{1.315692in}{1.579733in}}%
\pgfpathlineto{\pgfqpoint{1.319538in}{1.575428in}}%
\pgfpathlineto{\pgfqpoint{1.319888in}{1.573347in}}%
\pgfpathlineto{\pgfqpoint{1.320100in}{1.571262in}}%
\pgfpathlineto{\pgfqpoint{1.320172in}{1.569173in}}%
\pgfpathlineto{\pgfqpoint{1.320104in}{1.567084in}}%
\pgfpathlineto{\pgfqpoint{1.316238in}{1.571625in}}%
\pgfpathlineto{\pgfqpoint{1.312370in}{1.576054in}}%
\pgfpathlineto{\pgfqpoint{1.308500in}{1.580372in}}%
\pgfpathlineto{\pgfqpoint{1.304628in}{1.584576in}}%
\pgfpathlineto{\pgfqpoint{1.304692in}{1.586430in}}%
\pgfpathlineto{\pgfqpoint{1.304633in}{1.588283in}}%
\pgfpathlineto{\pgfqpoint{1.304449in}{1.590133in}}%
\pgfpathlineto{\pgfqpoint{1.304141in}{1.591980in}}%
\pgfpathclose%
\pgfusepath{fill}%
\end{pgfscope}%
\begin{pgfscope}%
\pgfpathrectangle{\pgfqpoint{0.041670in}{0.041670in}}{\pgfqpoint{2.216660in}{2.216660in}}%
\pgfusepath{clip}%
\pgfsetbuttcap%
\pgfsetroundjoin%
\definecolor{currentfill}{rgb}{0.955300,0.901065,0.118128}%
\pgfsetfillcolor{currentfill}%
\pgfsetlinewidth{0.000000pt}%
\definecolor{currentstroke}{rgb}{0.000000,0.000000,0.000000}%
\pgfsetstrokecolor{currentstroke}%
\pgfsetdash{}{0pt}%
\pgfpathmoveto{\pgfqpoint{1.206427in}{1.662169in}}%
\pgfpathlineto{\pgfqpoint{1.208624in}{1.662008in}}%
\pgfpathlineto{\pgfqpoint{1.210820in}{1.661717in}}%
\pgfpathlineto{\pgfqpoint{1.213014in}{1.661296in}}%
\pgfpathlineto{\pgfqpoint{1.215206in}{1.660747in}}%
\pgfpathlineto{\pgfqpoint{1.216813in}{1.660217in}}%
\pgfpathlineto{\pgfqpoint{1.218383in}{1.659664in}}%
\pgfpathlineto{\pgfqpoint{1.219915in}{1.659088in}}%
\pgfpathlineto{\pgfqpoint{1.221407in}{1.658490in}}%
\pgfpathlineto{\pgfqpoint{1.218829in}{1.659180in}}%
\pgfpathlineto{\pgfqpoint{1.216248in}{1.659742in}}%
\pgfpathlineto{\pgfqpoint{1.213665in}{1.660173in}}%
\pgfpathlineto{\pgfqpoint{1.211081in}{1.660475in}}%
\pgfpathlineto{\pgfqpoint{1.209961in}{1.660924in}}%
\pgfpathlineto{\pgfqpoint{1.208811in}{1.661356in}}%
\pgfpathlineto{\pgfqpoint{1.207633in}{1.661771in}}%
\pgfpathlineto{\pgfqpoint{1.206427in}{1.662169in}}%
\pgfpathclose%
\pgfusepath{fill}%
\end{pgfscope}%
\begin{pgfscope}%
\pgfpathrectangle{\pgfqpoint{0.041670in}{0.041670in}}{\pgfqpoint{2.216660in}{2.216660in}}%
\pgfusepath{clip}%
\pgfsetbuttcap%
\pgfsetroundjoin%
\definecolor{currentfill}{rgb}{0.344074,0.780029,0.397381}%
\pgfsetfillcolor{currentfill}%
\pgfsetlinewidth{0.000000pt}%
\definecolor{currentstroke}{rgb}{0.000000,0.000000,0.000000}%
\pgfsetstrokecolor{currentstroke}%
\pgfsetdash{}{0pt}%
\pgfpathmoveto{\pgfqpoint{0.983077in}{1.414152in}}%
\pgfpathlineto{\pgfqpoint{0.979599in}{1.406871in}}%
\pgfpathlineto{\pgfqpoint{0.976123in}{1.399514in}}%
\pgfpathlineto{\pgfqpoint{0.972648in}{1.392084in}}%
\pgfpathlineto{\pgfqpoint{0.969176in}{1.384581in}}%
\pgfpathlineto{\pgfqpoint{0.966186in}{1.387833in}}%
\pgfpathlineto{\pgfqpoint{0.963412in}{1.391127in}}%
\pgfpathlineto{\pgfqpoint{0.960859in}{1.394460in}}%
\pgfpathlineto{\pgfqpoint{0.958528in}{1.397829in}}%
\pgfpathlineto{\pgfqpoint{0.962168in}{1.405099in}}%
\pgfpathlineto{\pgfqpoint{0.965811in}{1.412298in}}%
\pgfpathlineto{\pgfqpoint{0.969456in}{1.419423in}}%
\pgfpathlineto{\pgfqpoint{0.973104in}{1.426473in}}%
\pgfpathlineto{\pgfqpoint{0.975289in}{1.423339in}}%
\pgfpathlineto{\pgfqpoint{0.977681in}{1.420239in}}%
\pgfpathlineto{\pgfqpoint{0.980278in}{1.417175in}}%
\pgfpathlineto{\pgfqpoint{0.983077in}{1.414152in}}%
\pgfpathclose%
\pgfusepath{fill}%
\end{pgfscope}%
\begin{pgfscope}%
\pgfpathrectangle{\pgfqpoint{0.041670in}{0.041670in}}{\pgfqpoint{2.216660in}{2.216660in}}%
\pgfusepath{clip}%
\pgfsetbuttcap%
\pgfsetroundjoin%
\definecolor{currentfill}{rgb}{0.274128,0.199721,0.498911}%
\pgfsetfillcolor{currentfill}%
\pgfsetlinewidth{0.000000pt}%
\definecolor{currentstroke}{rgb}{0.000000,0.000000,0.000000}%
\pgfsetstrokecolor{currentstroke}%
\pgfsetdash{}{0pt}%
\pgfpathmoveto{\pgfqpoint{1.460909in}{0.807477in}}%
\pgfpathlineto{\pgfqpoint{1.463228in}{0.799694in}}%
\pgfpathlineto{\pgfqpoint{1.465549in}{0.792017in}}%
\pgfpathlineto{\pgfqpoint{1.467871in}{0.784449in}}%
\pgfpathlineto{\pgfqpoint{1.470194in}{0.776994in}}%
\pgfpathlineto{\pgfqpoint{1.458064in}{0.772244in}}%
\pgfpathlineto{\pgfqpoint{1.445633in}{0.767697in}}%
\pgfpathlineto{\pgfqpoint{1.432915in}{0.763358in}}%
\pgfpathlineto{\pgfqpoint{1.419922in}{0.759231in}}%
\pgfpathlineto{\pgfqpoint{1.417996in}{0.766846in}}%
\pgfpathlineto{\pgfqpoint{1.416072in}{0.774575in}}%
\pgfpathlineto{\pgfqpoint{1.414149in}{0.782412in}}%
\pgfpathlineto{\pgfqpoint{1.412227in}{0.790355in}}%
\pgfpathlineto{\pgfqpoint{1.424808in}{0.794333in}}%
\pgfpathlineto{\pgfqpoint{1.437123in}{0.798515in}}%
\pgfpathlineto{\pgfqpoint{1.449161in}{0.802899in}}%
\pgfpathlineto{\pgfqpoint{1.460909in}{0.807477in}}%
\pgfpathclose%
\pgfusepath{fill}%
\end{pgfscope}%
\begin{pgfscope}%
\pgfpathrectangle{\pgfqpoint{0.041670in}{0.041670in}}{\pgfqpoint{2.216660in}{2.216660in}}%
\pgfusepath{clip}%
\pgfsetbuttcap%
\pgfsetroundjoin%
\definecolor{currentfill}{rgb}{0.955300,0.901065,0.118128}%
\pgfsetfillcolor{currentfill}%
\pgfsetlinewidth{0.000000pt}%
\definecolor{currentstroke}{rgb}{0.000000,0.000000,0.000000}%
\pgfsetstrokecolor{currentstroke}%
\pgfsetdash{}{0pt}%
\pgfpathmoveto{\pgfqpoint{1.147859in}{1.660062in}}%
\pgfpathlineto{\pgfqpoint{1.145194in}{1.659726in}}%
\pgfpathlineto{\pgfqpoint{1.142531in}{1.659260in}}%
\pgfpathlineto{\pgfqpoint{1.139869in}{1.658664in}}%
\pgfpathlineto{\pgfqpoint{1.137210in}{1.657940in}}%
\pgfpathlineto{\pgfqpoint{1.138666in}{1.658557in}}%
\pgfpathlineto{\pgfqpoint{1.140163in}{1.659153in}}%
\pgfpathlineto{\pgfqpoint{1.141700in}{1.659726in}}%
\pgfpathlineto{\pgfqpoint{1.143274in}{1.660277in}}%
\pgfpathlineto{\pgfqpoint{1.145555in}{1.660856in}}%
\pgfpathlineto{\pgfqpoint{1.147838in}{1.661306in}}%
\pgfpathlineto{\pgfqpoint{1.150123in}{1.661626in}}%
\pgfpathlineto{\pgfqpoint{1.152409in}{1.661816in}}%
\pgfpathlineto{\pgfqpoint{1.151228in}{1.661403in}}%
\pgfpathlineto{\pgfqpoint{1.150075in}{1.660973in}}%
\pgfpathlineto{\pgfqpoint{1.148951in}{1.660525in}}%
\pgfpathlineto{\pgfqpoint{1.147859in}{1.660062in}}%
\pgfpathclose%
\pgfusepath{fill}%
\end{pgfscope}%
\begin{pgfscope}%
\pgfpathrectangle{\pgfqpoint{0.041670in}{0.041670in}}{\pgfqpoint{2.216660in}{2.216660in}}%
\pgfusepath{clip}%
\pgfsetbuttcap%
\pgfsetroundjoin%
\definecolor{currentfill}{rgb}{0.147607,0.511733,0.557049}%
\pgfsetfillcolor{currentfill}%
\pgfsetlinewidth{0.000000pt}%
\definecolor{currentstroke}{rgb}{0.000000,0.000000,0.000000}%
\pgfsetstrokecolor{currentstroke}%
\pgfsetdash{}{0pt}%
\pgfpathmoveto{\pgfqpoint{0.947492in}{1.102365in}}%
\pgfpathlineto{\pgfqpoint{0.944890in}{1.093143in}}%
\pgfpathlineto{\pgfqpoint{0.942289in}{1.083922in}}%
\pgfpathlineto{\pgfqpoint{0.939688in}{1.074704in}}%
\pgfpathlineto{\pgfqpoint{0.937089in}{1.065494in}}%
\pgfpathlineto{\pgfqpoint{0.929017in}{1.069455in}}%
\pgfpathlineto{\pgfqpoint{0.921210in}{1.073542in}}%
\pgfpathlineto{\pgfqpoint{0.913674in}{1.077749in}}%
\pgfpathlineto{\pgfqpoint{0.906417in}{1.082072in}}%
\pgfpathlineto{\pgfqpoint{0.909338in}{1.091084in}}%
\pgfpathlineto{\pgfqpoint{0.912260in}{1.100103in}}%
\pgfpathlineto{\pgfqpoint{0.915183in}{1.109127in}}%
\pgfpathlineto{\pgfqpoint{0.918107in}{1.118151in}}%
\pgfpathlineto{\pgfqpoint{0.925061in}{1.114034in}}%
\pgfpathlineto{\pgfqpoint{0.932281in}{1.110028in}}%
\pgfpathlineto{\pgfqpoint{0.939760in}{1.106137in}}%
\pgfpathlineto{\pgfqpoint{0.947492in}{1.102365in}}%
\pgfpathclose%
\pgfusepath{fill}%
\end{pgfscope}%
\begin{pgfscope}%
\pgfpathrectangle{\pgfqpoint{0.041670in}{0.041670in}}{\pgfqpoint{2.216660in}{2.216660in}}%
\pgfusepath{clip}%
\pgfsetbuttcap%
\pgfsetroundjoin%
\definecolor{currentfill}{rgb}{0.120081,0.622161,0.534946}%
\pgfsetfillcolor{currentfill}%
\pgfsetlinewidth{0.000000pt}%
\definecolor{currentstroke}{rgb}{0.000000,0.000000,0.000000}%
\pgfsetstrokecolor{currentstroke}%
\pgfsetdash{}{0pt}%
\pgfpathmoveto{\pgfqpoint{0.953300in}{1.225537in}}%
\pgfpathlineto{\pgfqpoint{0.950360in}{1.216712in}}%
\pgfpathlineto{\pgfqpoint{0.947421in}{1.207856in}}%
\pgfpathlineto{\pgfqpoint{0.944483in}{1.198971in}}%
\pgfpathlineto{\pgfqpoint{0.941547in}{1.190059in}}%
\pgfpathlineto{\pgfqpoint{0.935452in}{1.193865in}}%
\pgfpathlineto{\pgfqpoint{0.929610in}{1.197762in}}%
\pgfpathlineto{\pgfqpoint{0.924028in}{1.201748in}}%
\pgfpathlineto{\pgfqpoint{0.918710in}{1.205816in}}%
\pgfpathlineto{\pgfqpoint{0.921920in}{1.214514in}}%
\pgfpathlineto{\pgfqpoint{0.925132in}{1.223185in}}%
\pgfpathlineto{\pgfqpoint{0.928346in}{1.231829in}}%
\pgfpathlineto{\pgfqpoint{0.931561in}{1.240442in}}%
\pgfpathlineto{\pgfqpoint{0.936624in}{1.236593in}}%
\pgfpathlineto{\pgfqpoint{0.941938in}{1.232823in}}%
\pgfpathlineto{\pgfqpoint{0.947499in}{1.229137in}}%
\pgfpathlineto{\pgfqpoint{0.953300in}{1.225537in}}%
\pgfpathclose%
\pgfusepath{fill}%
\end{pgfscope}%
\begin{pgfscope}%
\pgfpathrectangle{\pgfqpoint{0.041670in}{0.041670in}}{\pgfqpoint{2.216660in}{2.216660in}}%
\pgfusepath{clip}%
\pgfsetbuttcap%
\pgfsetroundjoin%
\definecolor{currentfill}{rgb}{0.935904,0.898570,0.108131}%
\pgfsetfillcolor{currentfill}%
\pgfsetlinewidth{0.000000pt}%
\definecolor{currentstroke}{rgb}{0.000000,0.000000,0.000000}%
\pgfsetstrokecolor{currentstroke}%
\pgfsetdash{}{0pt}%
\pgfpathmoveto{\pgfqpoint{1.231760in}{1.652977in}}%
\pgfpathlineto{\pgfqpoint{1.234984in}{1.651814in}}%
\pgfpathlineto{\pgfqpoint{1.238205in}{1.650523in}}%
\pgfpathlineto{\pgfqpoint{1.241425in}{1.649106in}}%
\pgfpathlineto{\pgfqpoint{1.244642in}{1.647562in}}%
\pgfpathlineto{\pgfqpoint{1.245988in}{1.646600in}}%
\pgfpathlineto{\pgfqpoint{1.247268in}{1.645618in}}%
\pgfpathlineto{\pgfqpoint{1.248482in}{1.644618in}}%
\pgfpathlineto{\pgfqpoint{1.249628in}{1.643600in}}%
\pgfpathlineto{\pgfqpoint{1.246161in}{1.645342in}}%
\pgfpathlineto{\pgfqpoint{1.242691in}{1.646958in}}%
\pgfpathlineto{\pgfqpoint{1.239220in}{1.648446in}}%
\pgfpathlineto{\pgfqpoint{1.235747in}{1.649807in}}%
\pgfpathlineto{\pgfqpoint{1.234831in}{1.650621in}}%
\pgfpathlineto{\pgfqpoint{1.233860in}{1.651421in}}%
\pgfpathlineto{\pgfqpoint{1.232836in}{1.652207in}}%
\pgfpathlineto{\pgfqpoint{1.231760in}{1.652977in}}%
\pgfpathclose%
\pgfusepath{fill}%
\end{pgfscope}%
\begin{pgfscope}%
\pgfpathrectangle{\pgfqpoint{0.041670in}{0.041670in}}{\pgfqpoint{2.216660in}{2.216660in}}%
\pgfusepath{clip}%
\pgfsetbuttcap%
\pgfsetroundjoin%
\definecolor{currentfill}{rgb}{0.274952,0.037752,0.364543}%
\pgfsetfillcolor{currentfill}%
\pgfsetlinewidth{0.000000pt}%
\definecolor{currentstroke}{rgb}{0.000000,0.000000,0.000000}%
\pgfsetstrokecolor{currentstroke}%
\pgfsetdash{}{0pt}%
\pgfpathmoveto{\pgfqpoint{1.507623in}{0.676234in}}%
\pgfpathlineto{\pgfqpoint{1.509984in}{0.671300in}}%
\pgfpathlineto{\pgfqpoint{1.512349in}{0.666551in}}%
\pgfpathlineto{\pgfqpoint{1.514717in}{0.661992in}}%
\pgfpathlineto{\pgfqpoint{1.517089in}{0.657628in}}%
\pgfpathlineto{\pgfqpoint{1.503033in}{0.652020in}}%
\pgfpathlineto{\pgfqpoint{1.488623in}{0.646651in}}%
\pgfpathlineto{\pgfqpoint{1.473875in}{0.641525in}}%
\pgfpathlineto{\pgfqpoint{1.458804in}{0.636650in}}%
\pgfpathlineto{\pgfqpoint{1.456836in}{0.641175in}}%
\pgfpathlineto{\pgfqpoint{1.454872in}{0.645894in}}%
\pgfpathlineto{\pgfqpoint{1.452911in}{0.650804in}}%
\pgfpathlineto{\pgfqpoint{1.450953in}{0.655898in}}%
\pgfpathlineto{\pgfqpoint{1.465605in}{0.660624in}}%
\pgfpathlineto{\pgfqpoint{1.479944in}{0.665593in}}%
\pgfpathlineto{\pgfqpoint{1.493955in}{0.670798in}}%
\pgfpathlineto{\pgfqpoint{1.507623in}{0.676234in}}%
\pgfpathclose%
\pgfusepath{fill}%
\end{pgfscope}%
\begin{pgfscope}%
\pgfpathrectangle{\pgfqpoint{0.041670in}{0.041670in}}{\pgfqpoint{2.216660in}{2.216660in}}%
\pgfusepath{clip}%
\pgfsetbuttcap%
\pgfsetroundjoin%
\definecolor{currentfill}{rgb}{0.762373,0.876424,0.137064}%
\pgfsetfillcolor{currentfill}%
\pgfsetlinewidth{0.000000pt}%
\definecolor{currentstroke}{rgb}{0.000000,0.000000,0.000000}%
\pgfsetstrokecolor{currentstroke}%
\pgfsetdash{}{0pt}%
\pgfpathmoveto{\pgfqpoint{1.055444in}{1.582930in}}%
\pgfpathlineto{\pgfqpoint{1.051576in}{1.578673in}}%
\pgfpathlineto{\pgfqpoint{1.047709in}{1.574303in}}%
\pgfpathlineto{\pgfqpoint{1.043845in}{1.569821in}}%
\pgfpathlineto{\pgfqpoint{1.039983in}{1.565229in}}%
\pgfpathlineto{\pgfqpoint{1.039791in}{1.567316in}}%
\pgfpathlineto{\pgfqpoint{1.039739in}{1.569406in}}%
\pgfpathlineto{\pgfqpoint{1.039826in}{1.571494in}}%
\pgfpathlineto{\pgfqpoint{1.040053in}{1.573579in}}%
\pgfpathlineto{\pgfqpoint{1.043909in}{1.577936in}}%
\pgfpathlineto{\pgfqpoint{1.047766in}{1.582182in}}%
\pgfpathlineto{\pgfqpoint{1.051627in}{1.586317in}}%
\pgfpathlineto{\pgfqpoint{1.055489in}{1.590339in}}%
\pgfpathlineto{\pgfqpoint{1.055291in}{1.588488in}}%
\pgfpathlineto{\pgfqpoint{1.055218in}{1.586636in}}%
\pgfpathlineto{\pgfqpoint{1.055269in}{1.584782in}}%
\pgfpathlineto{\pgfqpoint{1.055444in}{1.582930in}}%
\pgfpathclose%
\pgfusepath{fill}%
\end{pgfscope}%
\begin{pgfscope}%
\pgfpathrectangle{\pgfqpoint{0.041670in}{0.041670in}}{\pgfqpoint{2.216660in}{2.216660in}}%
\pgfusepath{clip}%
\pgfsetbuttcap%
\pgfsetroundjoin%
\definecolor{currentfill}{rgb}{0.935904,0.898570,0.108131}%
\pgfsetfillcolor{currentfill}%
\pgfsetlinewidth{0.000000pt}%
\definecolor{currentstroke}{rgb}{0.000000,0.000000,0.000000}%
\pgfsetstrokecolor{currentstroke}%
\pgfsetdash{}{0pt}%
\pgfpathmoveto{\pgfqpoint{1.123395in}{1.649072in}}%
\pgfpathlineto{\pgfqpoint{1.119874in}{1.647665in}}%
\pgfpathlineto{\pgfqpoint{1.116354in}{1.646131in}}%
\pgfpathlineto{\pgfqpoint{1.112836in}{1.644469in}}%
\pgfpathlineto{\pgfqpoint{1.109320in}{1.642681in}}%
\pgfpathlineto{\pgfqpoint{1.110405in}{1.643714in}}%
\pgfpathlineto{\pgfqpoint{1.111559in}{1.644730in}}%
\pgfpathlineto{\pgfqpoint{1.112780in}{1.645728in}}%
\pgfpathlineto{\pgfqpoint{1.114068in}{1.646708in}}%
\pgfpathlineto{\pgfqpoint{1.117346in}{1.648294in}}%
\pgfpathlineto{\pgfqpoint{1.120625in}{1.649754in}}%
\pgfpathlineto{\pgfqpoint{1.123907in}{1.651088in}}%
\pgfpathlineto{\pgfqpoint{1.127190in}{1.652293in}}%
\pgfpathlineto{\pgfqpoint{1.126161in}{1.651509in}}%
\pgfpathlineto{\pgfqpoint{1.125184in}{1.650711in}}%
\pgfpathlineto{\pgfqpoint{1.124262in}{1.649898in}}%
\pgfpathlineto{\pgfqpoint{1.123395in}{1.649072in}}%
\pgfpathclose%
\pgfusepath{fill}%
\end{pgfscope}%
\begin{pgfscope}%
\pgfpathrectangle{\pgfqpoint{0.041670in}{0.041670in}}{\pgfqpoint{2.216660in}{2.216660in}}%
\pgfusepath{clip}%
\pgfsetbuttcap%
\pgfsetroundjoin%
\definecolor{currentfill}{rgb}{0.133743,0.548535,0.553541}%
\pgfsetfillcolor{currentfill}%
\pgfsetlinewidth{0.000000pt}%
\definecolor{currentstroke}{rgb}{0.000000,0.000000,0.000000}%
\pgfsetstrokecolor{currentstroke}%
\pgfsetdash{}{0pt}%
\pgfpathmoveto{\pgfqpoint{1.435785in}{1.157769in}}%
\pgfpathlineto{\pgfqpoint{1.438779in}{1.148814in}}%
\pgfpathlineto{\pgfqpoint{1.441772in}{1.139849in}}%
\pgfpathlineto{\pgfqpoint{1.444764in}{1.130877in}}%
\pgfpathlineto{\pgfqpoint{1.447754in}{1.121900in}}%
\pgfpathlineto{\pgfqpoint{1.441043in}{1.117689in}}%
\pgfpathlineto{\pgfqpoint{1.434060in}{1.113584in}}%
\pgfpathlineto{\pgfqpoint{1.426811in}{1.109590in}}%
\pgfpathlineto{\pgfqpoint{1.419303in}{1.105712in}}%
\pgfpathlineto{\pgfqpoint{1.416624in}{1.114890in}}%
\pgfpathlineto{\pgfqpoint{1.413944in}{1.124064in}}%
\pgfpathlineto{\pgfqpoint{1.411263in}{1.133230in}}%
\pgfpathlineto{\pgfqpoint{1.408581in}{1.142386in}}%
\pgfpathlineto{\pgfqpoint{1.415759in}{1.146071in}}%
\pgfpathlineto{\pgfqpoint{1.422690in}{1.149866in}}%
\pgfpathlineto{\pgfqpoint{1.429367in}{1.153767in}}%
\pgfpathlineto{\pgfqpoint{1.435785in}{1.157769in}}%
\pgfpathclose%
\pgfusepath{fill}%
\end{pgfscope}%
\begin{pgfscope}%
\pgfpathrectangle{\pgfqpoint{0.041670in}{0.041670in}}{\pgfqpoint{2.216660in}{2.216660in}}%
\pgfusepath{clip}%
\pgfsetbuttcap%
\pgfsetroundjoin%
\definecolor{currentfill}{rgb}{0.565498,0.842430,0.262877}%
\pgfsetfillcolor{currentfill}%
\pgfsetlinewidth{0.000000pt}%
\definecolor{currentstroke}{rgb}{0.000000,0.000000,0.000000}%
\pgfsetstrokecolor{currentstroke}%
\pgfsetdash{}{0pt}%
\pgfpathmoveto{\pgfqpoint{1.348956in}{1.516720in}}%
\pgfpathlineto{\pgfqpoint{1.352761in}{1.511007in}}%
\pgfpathlineto{\pgfqpoint{1.356563in}{1.505198in}}%
\pgfpathlineto{\pgfqpoint{1.360363in}{1.499293in}}%
\pgfpathlineto{\pgfqpoint{1.364161in}{1.493295in}}%
\pgfpathlineto{\pgfqpoint{1.363162in}{1.490526in}}%
\pgfpathlineto{\pgfqpoint{1.361980in}{1.487771in}}%
\pgfpathlineto{\pgfqpoint{1.360614in}{1.485035in}}%
\pgfpathlineto{\pgfqpoint{1.359067in}{1.482320in}}%
\pgfpathlineto{\pgfqpoint{1.355369in}{1.488554in}}%
\pgfpathlineto{\pgfqpoint{1.351668in}{1.494694in}}%
\pgfpathlineto{\pgfqpoint{1.347965in}{1.500740in}}%
\pgfpathlineto{\pgfqpoint{1.344261in}{1.506688in}}%
\pgfpathlineto{\pgfqpoint{1.345686in}{1.509170in}}%
\pgfpathlineto{\pgfqpoint{1.346944in}{1.511671in}}%
\pgfpathlineto{\pgfqpoint{1.348034in}{1.514188in}}%
\pgfpathlineto{\pgfqpoint{1.348956in}{1.516720in}}%
\pgfpathclose%
\pgfusepath{fill}%
\end{pgfscope}%
\begin{pgfscope}%
\pgfpathrectangle{\pgfqpoint{0.041670in}{0.041670in}}{\pgfqpoint{2.216660in}{2.216660in}}%
\pgfusepath{clip}%
\pgfsetbuttcap%
\pgfsetroundjoin%
\definecolor{currentfill}{rgb}{0.283072,0.130895,0.449241}%
\pgfsetfillcolor{currentfill}%
\pgfsetlinewidth{0.000000pt}%
\definecolor{currentstroke}{rgb}{0.000000,0.000000,0.000000}%
\pgfsetstrokecolor{currentstroke}%
\pgfsetdash{}{0pt}%
\pgfpathmoveto{\pgfqpoint{0.944417in}{0.726360in}}%
\pgfpathlineto{\pgfqpoint{0.942579in}{0.719335in}}%
\pgfpathlineto{\pgfqpoint{0.940739in}{0.712445in}}%
\pgfpathlineto{\pgfqpoint{0.938897in}{0.705694in}}%
\pgfpathlineto{\pgfqpoint{0.937053in}{0.699088in}}%
\pgfpathlineto{\pgfqpoint{0.922985in}{0.703307in}}%
\pgfpathlineto{\pgfqpoint{0.909197in}{0.707759in}}%
\pgfpathlineto{\pgfqpoint{0.895704in}{0.712438in}}%
\pgfpathlineto{\pgfqpoint{0.882521in}{0.717339in}}%
\pgfpathlineto{\pgfqpoint{0.884772in}{0.723790in}}%
\pgfpathlineto{\pgfqpoint{0.887022in}{0.730386in}}%
\pgfpathlineto{\pgfqpoint{0.889270in}{0.737122in}}%
\pgfpathlineto{\pgfqpoint{0.891516in}{0.743994in}}%
\pgfpathlineto{\pgfqpoint{0.904306in}{0.739259in}}%
\pgfpathlineto{\pgfqpoint{0.917395in}{0.734738in}}%
\pgfpathlineto{\pgfqpoint{0.930770in}{0.730437in}}%
\pgfpathlineto{\pgfqpoint{0.944417in}{0.726360in}}%
\pgfpathclose%
\pgfusepath{fill}%
\end{pgfscope}%
\begin{pgfscope}%
\pgfpathrectangle{\pgfqpoint{0.041670in}{0.041670in}}{\pgfqpoint{2.216660in}{2.216660in}}%
\pgfusepath{clip}%
\pgfsetbuttcap%
\pgfsetroundjoin%
\definecolor{currentfill}{rgb}{0.955300,0.901065,0.118128}%
\pgfsetfillcolor{currentfill}%
\pgfsetlinewidth{0.000000pt}%
\definecolor{currentstroke}{rgb}{0.000000,0.000000,0.000000}%
\pgfsetstrokecolor{currentstroke}%
\pgfsetdash{}{0pt}%
\pgfpathmoveto{\pgfqpoint{1.211081in}{1.660475in}}%
\pgfpathlineto{\pgfqpoint{1.213665in}{1.660173in}}%
\pgfpathlineto{\pgfqpoint{1.216248in}{1.659742in}}%
\pgfpathlineto{\pgfqpoint{1.218829in}{1.659180in}}%
\pgfpathlineto{\pgfqpoint{1.221407in}{1.658490in}}%
\pgfpathlineto{\pgfqpoint{1.222859in}{1.657870in}}%
\pgfpathlineto{\pgfqpoint{1.224269in}{1.657229in}}%
\pgfpathlineto{\pgfqpoint{1.225634in}{1.656567in}}%
\pgfpathlineto{\pgfqpoint{1.226955in}{1.655886in}}%
\pgfpathlineto{\pgfqpoint{1.224030in}{1.656739in}}%
\pgfpathlineto{\pgfqpoint{1.221103in}{1.657463in}}%
\pgfpathlineto{\pgfqpoint{1.218174in}{1.658057in}}%
\pgfpathlineto{\pgfqpoint{1.215243in}{1.658521in}}%
\pgfpathlineto{\pgfqpoint{1.214253in}{1.659032in}}%
\pgfpathlineto{\pgfqpoint{1.213228in}{1.659528in}}%
\pgfpathlineto{\pgfqpoint{1.212171in}{1.660009in}}%
\pgfpathlineto{\pgfqpoint{1.211081in}{1.660475in}}%
\pgfpathclose%
\pgfusepath{fill}%
\end{pgfscope}%
\begin{pgfscope}%
\pgfpathrectangle{\pgfqpoint{0.041670in}{0.041670in}}{\pgfqpoint{2.216660in}{2.216660in}}%
\pgfusepath{clip}%
\pgfsetbuttcap%
\pgfsetroundjoin%
\definecolor{currentfill}{rgb}{0.263663,0.237631,0.518762}%
\pgfsetfillcolor{currentfill}%
\pgfsetlinewidth{0.000000pt}%
\definecolor{currentstroke}{rgb}{0.000000,0.000000,0.000000}%
\pgfsetstrokecolor{currentstroke}%
\pgfsetdash{}{0pt}%
\pgfpathmoveto{\pgfqpoint{1.451640in}{0.839590in}}%
\pgfpathlineto{\pgfqpoint{1.453956in}{0.831421in}}%
\pgfpathlineto{\pgfqpoint{1.456273in}{0.823344in}}%
\pgfpathlineto{\pgfqpoint{1.458590in}{0.815361in}}%
\pgfpathlineto{\pgfqpoint{1.460909in}{0.807477in}}%
\pgfpathlineto{\pgfqpoint{1.449161in}{0.802899in}}%
\pgfpathlineto{\pgfqpoint{1.437123in}{0.798515in}}%
\pgfpathlineto{\pgfqpoint{1.424808in}{0.794333in}}%
\pgfpathlineto{\pgfqpoint{1.412227in}{0.790355in}}%
\pgfpathlineto{\pgfqpoint{1.410306in}{0.798400in}}%
\pgfpathlineto{\pgfqpoint{1.408385in}{0.806542in}}%
\pgfpathlineto{\pgfqpoint{1.406466in}{0.814779in}}%
\pgfpathlineto{\pgfqpoint{1.404547in}{0.823107in}}%
\pgfpathlineto{\pgfqpoint{1.416716in}{0.826936in}}%
\pgfpathlineto{\pgfqpoint{1.428629in}{0.830962in}}%
\pgfpathlineto{\pgfqpoint{1.440275in}{0.835182in}}%
\pgfpathlineto{\pgfqpoint{1.451640in}{0.839590in}}%
\pgfpathclose%
\pgfusepath{fill}%
\end{pgfscope}%
\begin{pgfscope}%
\pgfpathrectangle{\pgfqpoint{0.041670in}{0.041670in}}{\pgfqpoint{2.216660in}{2.216660in}}%
\pgfusepath{clip}%
\pgfsetbuttcap%
\pgfsetroundjoin%
\definecolor{currentfill}{rgb}{0.276194,0.190074,0.493001}%
\pgfsetfillcolor{currentfill}%
\pgfsetlinewidth{0.000000pt}%
\definecolor{currentstroke}{rgb}{0.000000,0.000000,0.000000}%
\pgfsetstrokecolor{currentstroke}%
\pgfsetdash{}{0pt}%
\pgfpathmoveto{\pgfqpoint{1.777007in}{0.775673in}}%
\pgfpathlineto{\pgfqpoint{1.780648in}{0.782600in}}%
\pgfpathlineto{\pgfqpoint{1.784304in}{0.789924in}}%
\pgfpathlineto{\pgfqpoint{1.787975in}{0.797654in}}%
\pgfpathlineto{\pgfqpoint{1.791664in}{0.805796in}}%
\pgfpathlineto{\pgfqpoint{1.780445in}{0.795702in}}%
\pgfpathlineto{\pgfqpoint{1.768579in}{0.785787in}}%
\pgfpathlineto{\pgfqpoint{1.756076in}{0.776062in}}%
\pgfpathlineto{\pgfqpoint{1.742947in}{0.766538in}}%
\pgfpathlineto{\pgfqpoint{1.739534in}{0.758591in}}%
\pgfpathlineto{\pgfqpoint{1.736137in}{0.751057in}}%
\pgfpathlineto{\pgfqpoint{1.732755in}{0.743930in}}%
\pgfpathlineto{\pgfqpoint{1.729387in}{0.737203in}}%
\pgfpathlineto{\pgfqpoint{1.742217in}{0.746535in}}%
\pgfpathlineto{\pgfqpoint{1.754438in}{0.756064in}}%
\pgfpathlineto{\pgfqpoint{1.766038in}{0.765781in}}%
\pgfpathlineto{\pgfqpoint{1.777007in}{0.775673in}}%
\pgfpathclose%
\pgfusepath{fill}%
\end{pgfscope}%
\begin{pgfscope}%
\pgfpathrectangle{\pgfqpoint{0.041670in}{0.041670in}}{\pgfqpoint{2.216660in}{2.216660in}}%
\pgfusepath{clip}%
\pgfsetbuttcap%
\pgfsetroundjoin%
\definecolor{currentfill}{rgb}{0.277941,0.056324,0.381191}%
\pgfsetfillcolor{currentfill}%
\pgfsetlinewidth{0.000000pt}%
\definecolor{currentstroke}{rgb}{0.000000,0.000000,0.000000}%
\pgfsetstrokecolor{currentstroke}%
\pgfsetdash{}{0pt}%
\pgfpathmoveto{\pgfqpoint{0.736102in}{0.645651in}}%
\pgfpathlineto{\pgfqpoint{0.733285in}{0.647845in}}%
\pgfpathlineto{\pgfqpoint{0.730458in}{0.650364in}}%
\pgfpathlineto{\pgfqpoint{0.727622in}{0.653215in}}%
\pgfpathlineto{\pgfqpoint{0.724775in}{0.656403in}}%
\pgfpathlineto{\pgfqpoint{0.709879in}{0.664255in}}%
\pgfpathlineto{\pgfqpoint{0.695499in}{0.672348in}}%
\pgfpathlineto{\pgfqpoint{0.681650in}{0.680671in}}%
\pgfpathlineto{\pgfqpoint{0.668344in}{0.689214in}}%
\pgfpathlineto{\pgfqpoint{0.671529in}{0.685835in}}%
\pgfpathlineto{\pgfqpoint{0.674704in}{0.682793in}}%
\pgfpathlineto{\pgfqpoint{0.677867in}{0.680082in}}%
\pgfpathlineto{\pgfqpoint{0.681021in}{0.677695in}}%
\pgfpathlineto{\pgfqpoint{0.694011in}{0.669351in}}%
\pgfpathlineto{\pgfqpoint{0.707530in}{0.661223in}}%
\pgfpathlineto{\pgfqpoint{0.721565in}{0.653320in}}%
\pgfpathlineto{\pgfqpoint{0.736102in}{0.645651in}}%
\pgfpathclose%
\pgfusepath{fill}%
\end{pgfscope}%
\begin{pgfscope}%
\pgfpathrectangle{\pgfqpoint{0.041670in}{0.041670in}}{\pgfqpoint{2.216660in}{2.216660in}}%
\pgfusepath{clip}%
\pgfsetbuttcap%
\pgfsetroundjoin%
\definecolor{currentfill}{rgb}{0.282327,0.094955,0.417331}%
\pgfsetfillcolor{currentfill}%
\pgfsetlinewidth{0.000000pt}%
\definecolor{currentstroke}{rgb}{0.000000,0.000000,0.000000}%
\pgfsetstrokecolor{currentstroke}%
\pgfsetdash{}{0pt}%
\pgfpathmoveto{\pgfqpoint{0.937053in}{0.699088in}}%
\pgfpathlineto{\pgfqpoint{0.935208in}{0.692630in}}%
\pgfpathlineto{\pgfqpoint{0.933361in}{0.686323in}}%
\pgfpathlineto{\pgfqpoint{0.931512in}{0.680173in}}%
\pgfpathlineto{\pgfqpoint{0.929660in}{0.674182in}}%
\pgfpathlineto{\pgfqpoint{0.915168in}{0.678545in}}%
\pgfpathlineto{\pgfqpoint{0.900966in}{0.683147in}}%
\pgfpathlineto{\pgfqpoint{0.887068in}{0.687984in}}%
\pgfpathlineto{\pgfqpoint{0.873491in}{0.693050in}}%
\pgfpathlineto{\pgfqpoint{0.875752in}{0.698886in}}%
\pgfpathlineto{\pgfqpoint{0.878010in}{0.704882in}}%
\pgfpathlineto{\pgfqpoint{0.880267in}{0.711034in}}%
\pgfpathlineto{\pgfqpoint{0.882521in}{0.717339in}}%
\pgfpathlineto{\pgfqpoint{0.895704in}{0.712438in}}%
\pgfpathlineto{\pgfqpoint{0.909197in}{0.707759in}}%
\pgfpathlineto{\pgfqpoint{0.922985in}{0.703307in}}%
\pgfpathlineto{\pgfqpoint{0.937053in}{0.699088in}}%
\pgfpathclose%
\pgfusepath{fill}%
\end{pgfscope}%
\begin{pgfscope}%
\pgfpathrectangle{\pgfqpoint{0.041670in}{0.041670in}}{\pgfqpoint{2.216660in}{2.216660in}}%
\pgfusepath{clip}%
\pgfsetbuttcap%
\pgfsetroundjoin%
\definecolor{currentfill}{rgb}{0.220124,0.725509,0.466226}%
\pgfsetfillcolor{currentfill}%
\pgfsetlinewidth{0.000000pt}%
\definecolor{currentstroke}{rgb}{0.000000,0.000000,0.000000}%
\pgfsetstrokecolor{currentstroke}%
\pgfsetdash{}{0pt}%
\pgfpathmoveto{\pgfqpoint{0.970287in}{1.340495in}}%
\pgfpathlineto{\pgfqpoint{0.967050in}{1.332443in}}%
\pgfpathlineto{\pgfqpoint{0.963814in}{1.324330in}}%
\pgfpathlineto{\pgfqpoint{0.960581in}{1.316161in}}%
\pgfpathlineto{\pgfqpoint{0.957349in}{1.307936in}}%
\pgfpathlineto{\pgfqpoint{0.953028in}{1.311419in}}%
\pgfpathlineto{\pgfqpoint{0.948939in}{1.314964in}}%
\pgfpathlineto{\pgfqpoint{0.945087in}{1.318569in}}%
\pgfpathlineto{\pgfqpoint{0.941476in}{1.322230in}}%
\pgfpathlineto{\pgfqpoint{0.944931in}{1.330230in}}%
\pgfpathlineto{\pgfqpoint{0.948388in}{1.338176in}}%
\pgfpathlineto{\pgfqpoint{0.951847in}{1.346065in}}%
\pgfpathlineto{\pgfqpoint{0.955309in}{1.353895in}}%
\pgfpathlineto{\pgfqpoint{0.958718in}{1.350463in}}%
\pgfpathlineto{\pgfqpoint{0.962353in}{1.347083in}}%
\pgfpathlineto{\pgfqpoint{0.966210in}{1.343760in}}%
\pgfpathlineto{\pgfqpoint{0.970287in}{1.340495in}}%
\pgfpathclose%
\pgfusepath{fill}%
\end{pgfscope}%
\begin{pgfscope}%
\pgfpathrectangle{\pgfqpoint{0.041670in}{0.041670in}}{\pgfqpoint{2.216660in}{2.216660in}}%
\pgfusepath{clip}%
\pgfsetbuttcap%
\pgfsetroundjoin%
\definecolor{currentfill}{rgb}{0.134692,0.658636,0.517649}%
\pgfsetfillcolor{currentfill}%
\pgfsetlinewidth{0.000000pt}%
\definecolor{currentstroke}{rgb}{0.000000,0.000000,0.000000}%
\pgfsetstrokecolor{currentstroke}%
\pgfsetdash{}{0pt}%
\pgfpathmoveto{\pgfqpoint{1.419539in}{1.277822in}}%
\pgfpathlineto{\pgfqpoint{1.422816in}{1.269407in}}%
\pgfpathlineto{\pgfqpoint{1.426090in}{1.260951in}}%
\pgfpathlineto{\pgfqpoint{1.429363in}{1.252456in}}%
\pgfpathlineto{\pgfqpoint{1.432634in}{1.243926in}}%
\pgfpathlineto{\pgfqpoint{1.427799in}{1.240010in}}%
\pgfpathlineto{\pgfqpoint{1.422708in}{1.236170in}}%
\pgfpathlineto{\pgfqpoint{1.417366in}{1.232409in}}%
\pgfpathlineto{\pgfqpoint{1.411778in}{1.228732in}}%
\pgfpathlineto{\pgfqpoint{1.408771in}{1.237478in}}%
\pgfpathlineto{\pgfqpoint{1.405762in}{1.246188in}}%
\pgfpathlineto{\pgfqpoint{1.402751in}{1.254859in}}%
\pgfpathlineto{\pgfqpoint{1.399739in}{1.263489in}}%
\pgfpathlineto{\pgfqpoint{1.405042in}{1.266957in}}%
\pgfpathlineto{\pgfqpoint{1.410113in}{1.270505in}}%
\pgfpathlineto{\pgfqpoint{1.414947in}{1.274128in}}%
\pgfpathlineto{\pgfqpoint{1.419539in}{1.277822in}}%
\pgfpathclose%
\pgfusepath{fill}%
\end{pgfscope}%
\begin{pgfscope}%
\pgfpathrectangle{\pgfqpoint{0.041670in}{0.041670in}}{\pgfqpoint{2.216660in}{2.216660in}}%
\pgfusepath{clip}%
\pgfsetbuttcap%
\pgfsetroundjoin%
\definecolor{currentfill}{rgb}{0.267004,0.004874,0.329415}%
\pgfsetfillcolor{currentfill}%
\pgfsetlinewidth{0.000000pt}%
\definecolor{currentstroke}{rgb}{0.000000,0.000000,0.000000}%
\pgfsetstrokecolor{currentstroke}%
\pgfsetdash{}{0pt}%
\pgfpathmoveto{\pgfqpoint{0.827541in}{0.616616in}}%
\pgfpathlineto{\pgfqpoint{0.825196in}{0.615160in}}%
\pgfpathlineto{\pgfqpoint{0.822845in}{0.613965in}}%
\pgfpathlineto{\pgfqpoint{0.820488in}{0.613035in}}%
\pgfpathlineto{\pgfqpoint{0.818124in}{0.612376in}}%
\pgfpathlineto{\pgfqpoint{0.802537in}{0.618688in}}%
\pgfpathlineto{\pgfqpoint{0.787365in}{0.625257in}}%
\pgfpathlineto{\pgfqpoint{0.772625in}{0.632075in}}%
\pgfpathlineto{\pgfqpoint{0.758332in}{0.639134in}}%
\pgfpathlineto{\pgfqpoint{0.761076in}{0.639616in}}%
\pgfpathlineto{\pgfqpoint{0.763813in}{0.640369in}}%
\pgfpathlineto{\pgfqpoint{0.766543in}{0.641387in}}%
\pgfpathlineto{\pgfqpoint{0.769267in}{0.642665in}}%
\pgfpathlineto{\pgfqpoint{0.783199in}{0.635793in}}%
\pgfpathlineto{\pgfqpoint{0.797565in}{0.629155in}}%
\pgfpathlineto{\pgfqpoint{0.812351in}{0.622761in}}%
\pgfpathlineto{\pgfqpoint{0.827541in}{0.616616in}}%
\pgfpathclose%
\pgfusepath{fill}%
\end{pgfscope}%
\begin{pgfscope}%
\pgfpathrectangle{\pgfqpoint{0.041670in}{0.041670in}}{\pgfqpoint{2.216660in}{2.216660in}}%
\pgfusepath{clip}%
\pgfsetbuttcap%
\pgfsetroundjoin%
\definecolor{currentfill}{rgb}{0.195860,0.395433,0.555276}%
\pgfsetfillcolor{currentfill}%
\pgfsetlinewidth{0.000000pt}%
\definecolor{currentstroke}{rgb}{0.000000,0.000000,0.000000}%
\pgfsetstrokecolor{currentstroke}%
\pgfsetdash{}{0pt}%
\pgfpathmoveto{\pgfqpoint{0.954010in}{0.976490in}}%
\pgfpathlineto{\pgfqpoint{0.951781in}{0.967297in}}%
\pgfpathlineto{\pgfqpoint{0.949554in}{0.958141in}}%
\pgfpathlineto{\pgfqpoint{0.947326in}{0.949026in}}%
\pgfpathlineto{\pgfqpoint{0.945099in}{0.939953in}}%
\pgfpathlineto{\pgfqpoint{0.934908in}{0.943872in}}%
\pgfpathlineto{\pgfqpoint{0.924979in}{0.947953in}}%
\pgfpathlineto{\pgfqpoint{0.915322in}{0.952191in}}%
\pgfpathlineto{\pgfqpoint{0.905945in}{0.956582in}}%
\pgfpathlineto{\pgfqpoint{0.908538in}{0.965476in}}%
\pgfpathlineto{\pgfqpoint{0.911131in}{0.974414in}}%
\pgfpathlineto{\pgfqpoint{0.913724in}{0.983392in}}%
\pgfpathlineto{\pgfqpoint{0.916318in}{0.992407in}}%
\pgfpathlineto{\pgfqpoint{0.925346in}{0.988204in}}%
\pgfpathlineto{\pgfqpoint{0.934643in}{0.984147in}}%
\pgfpathlineto{\pgfqpoint{0.944201in}{0.980241in}}%
\pgfpathlineto{\pgfqpoint{0.954010in}{0.976490in}}%
\pgfpathclose%
\pgfusepath{fill}%
\end{pgfscope}%
\begin{pgfscope}%
\pgfpathrectangle{\pgfqpoint{0.041670in}{0.041670in}}{\pgfqpoint{2.216660in}{2.216660in}}%
\pgfusepath{clip}%
\pgfsetbuttcap%
\pgfsetroundjoin%
\definecolor{currentfill}{rgb}{0.280255,0.165693,0.476498}%
\pgfsetfillcolor{currentfill}%
\pgfsetlinewidth{0.000000pt}%
\definecolor{currentstroke}{rgb}{0.000000,0.000000,0.000000}%
\pgfsetstrokecolor{currentstroke}%
\pgfsetdash{}{0pt}%
\pgfpathmoveto{\pgfqpoint{0.951757in}{0.755745in}}%
\pgfpathlineto{\pgfqpoint{0.949924in}{0.748215in}}%
\pgfpathlineto{\pgfqpoint{0.948089in}{0.740804in}}%
\pgfpathlineto{\pgfqpoint{0.946254in}{0.733518in}}%
\pgfpathlineto{\pgfqpoint{0.944417in}{0.726360in}}%
\pgfpathlineto{\pgfqpoint{0.930770in}{0.730437in}}%
\pgfpathlineto{\pgfqpoint{0.917395in}{0.734738in}}%
\pgfpathlineto{\pgfqpoint{0.904306in}{0.739259in}}%
\pgfpathlineto{\pgfqpoint{0.891516in}{0.743994in}}%
\pgfpathlineto{\pgfqpoint{0.893760in}{0.750997in}}%
\pgfpathlineto{\pgfqpoint{0.896002in}{0.758129in}}%
\pgfpathlineto{\pgfqpoint{0.898243in}{0.765385in}}%
\pgfpathlineto{\pgfqpoint{0.900483in}{0.772762in}}%
\pgfpathlineto{\pgfqpoint{0.912881in}{0.768192in}}%
\pgfpathlineto{\pgfqpoint{0.925568in}{0.763829in}}%
\pgfpathlineto{\pgfqpoint{0.938531in}{0.759679in}}%
\pgfpathlineto{\pgfqpoint{0.951757in}{0.755745in}}%
\pgfpathclose%
\pgfusepath{fill}%
\end{pgfscope}%
\begin{pgfscope}%
\pgfpathrectangle{\pgfqpoint{0.041670in}{0.041670in}}{\pgfqpoint{2.216660in}{2.216660in}}%
\pgfusepath{clip}%
\pgfsetbuttcap%
\pgfsetroundjoin%
\definecolor{currentfill}{rgb}{0.955300,0.901065,0.118128}%
\pgfsetfillcolor{currentfill}%
\pgfsetlinewidth{0.000000pt}%
\definecolor{currentstroke}{rgb}{0.000000,0.000000,0.000000}%
\pgfsetstrokecolor{currentstroke}%
\pgfsetdash{}{0pt}%
\pgfpathmoveto{\pgfqpoint{1.143815in}{1.658054in}}%
\pgfpathlineto{\pgfqpoint{1.140814in}{1.657551in}}%
\pgfpathlineto{\pgfqpoint{1.137814in}{1.656919in}}%
\pgfpathlineto{\pgfqpoint{1.134816in}{1.656156in}}%
\pgfpathlineto{\pgfqpoint{1.131820in}{1.655264in}}%
\pgfpathlineto{\pgfqpoint{1.133099in}{1.655963in}}%
\pgfpathlineto{\pgfqpoint{1.134425in}{1.656642in}}%
\pgfpathlineto{\pgfqpoint{1.135796in}{1.657301in}}%
\pgfpathlineto{\pgfqpoint{1.137210in}{1.657940in}}%
\pgfpathlineto{\pgfqpoint{1.139869in}{1.658664in}}%
\pgfpathlineto{\pgfqpoint{1.142531in}{1.659260in}}%
\pgfpathlineto{\pgfqpoint{1.145194in}{1.659726in}}%
\pgfpathlineto{\pgfqpoint{1.147859in}{1.660062in}}%
\pgfpathlineto{\pgfqpoint{1.146797in}{1.659583in}}%
\pgfpathlineto{\pgfqpoint{1.145769in}{1.659088in}}%
\pgfpathlineto{\pgfqpoint{1.144775in}{1.658578in}}%
\pgfpathlineto{\pgfqpoint{1.143815in}{1.658054in}}%
\pgfpathclose%
\pgfusepath{fill}%
\end{pgfscope}%
\begin{pgfscope}%
\pgfpathrectangle{\pgfqpoint{0.041670in}{0.041670in}}{\pgfqpoint{2.216660in}{2.216660in}}%
\pgfusepath{clip}%
\pgfsetbuttcap%
\pgfsetroundjoin%
\definecolor{currentfill}{rgb}{0.412913,0.803041,0.357269}%
\pgfsetfillcolor{currentfill}%
\pgfsetlinewidth{0.000000pt}%
\definecolor{currentstroke}{rgb}{0.000000,0.000000,0.000000}%
\pgfsetstrokecolor{currentstroke}%
\pgfsetdash{}{0pt}%
\pgfpathmoveto{\pgfqpoint{1.373838in}{1.456480in}}%
\pgfpathlineto{\pgfqpoint{1.377525in}{1.449804in}}%
\pgfpathlineto{\pgfqpoint{1.381210in}{1.443044in}}%
\pgfpathlineto{\pgfqpoint{1.384892in}{1.436203in}}%
\pgfpathlineto{\pgfqpoint{1.388572in}{1.429283in}}%
\pgfpathlineto{\pgfqpoint{1.386573in}{1.426123in}}%
\pgfpathlineto{\pgfqpoint{1.384365in}{1.422993in}}%
\pgfpathlineto{\pgfqpoint{1.381951in}{1.419897in}}%
\pgfpathlineto{\pgfqpoint{1.379331in}{1.416837in}}%
\pgfpathlineto{\pgfqpoint{1.375808in}{1.423990in}}%
\pgfpathlineto{\pgfqpoint{1.372282in}{1.431063in}}%
\pgfpathlineto{\pgfqpoint{1.368755in}{1.438054in}}%
\pgfpathlineto{\pgfqpoint{1.365226in}{1.444962in}}%
\pgfpathlineto{\pgfqpoint{1.367666in}{1.447793in}}%
\pgfpathlineto{\pgfqpoint{1.369916in}{1.450658in}}%
\pgfpathlineto{\pgfqpoint{1.371974in}{1.453555in}}%
\pgfpathlineto{\pgfqpoint{1.373838in}{1.456480in}}%
\pgfpathclose%
\pgfusepath{fill}%
\end{pgfscope}%
\begin{pgfscope}%
\pgfpathrectangle{\pgfqpoint{0.041670in}{0.041670in}}{\pgfqpoint{2.216660in}{2.216660in}}%
\pgfusepath{clip}%
\pgfsetbuttcap%
\pgfsetroundjoin%
\definecolor{currentfill}{rgb}{0.699415,0.867117,0.175971}%
\pgfsetfillcolor{currentfill}%
\pgfsetlinewidth{0.000000pt}%
\definecolor{currentstroke}{rgb}{0.000000,0.000000,0.000000}%
\pgfsetstrokecolor{currentstroke}%
\pgfsetdash{}{0pt}%
\pgfpathmoveto{\pgfqpoint{1.320104in}{1.567084in}}%
\pgfpathlineto{\pgfqpoint{1.323968in}{1.562435in}}%
\pgfpathlineto{\pgfqpoint{1.327829in}{1.557677in}}%
\pgfpathlineto{\pgfqpoint{1.331688in}{1.552812in}}%
\pgfpathlineto{\pgfqpoint{1.335544in}{1.547842in}}%
\pgfpathlineto{\pgfqpoint{1.335319in}{1.545517in}}%
\pgfpathlineto{\pgfqpoint{1.334939in}{1.543197in}}%
\pgfpathlineto{\pgfqpoint{1.334404in}{1.540882in}}%
\pgfpathlineto{\pgfqpoint{1.333714in}{1.538575in}}%
\pgfpathlineto{\pgfqpoint{1.329898in}{1.543782in}}%
\pgfpathlineto{\pgfqpoint{1.326079in}{1.548883in}}%
\pgfpathlineto{\pgfqpoint{1.322259in}{1.553877in}}%
\pgfpathlineto{\pgfqpoint{1.318436in}{1.558763in}}%
\pgfpathlineto{\pgfqpoint{1.319062in}{1.560834in}}%
\pgfpathlineto{\pgfqpoint{1.319549in}{1.562913in}}%
\pgfpathlineto{\pgfqpoint{1.319897in}{1.564997in}}%
\pgfpathlineto{\pgfqpoint{1.320104in}{1.567084in}}%
\pgfpathclose%
\pgfusepath{fill}%
\end{pgfscope}%
\begin{pgfscope}%
\pgfpathrectangle{\pgfqpoint{0.041670in}{0.041670in}}{\pgfqpoint{2.216660in}{2.216660in}}%
\pgfusepath{clip}%
\pgfsetbuttcap%
\pgfsetroundjoin%
\definecolor{currentfill}{rgb}{0.179019,0.433756,0.557430}%
\pgfsetfillcolor{currentfill}%
\pgfsetlinewidth{0.000000pt}%
\definecolor{currentstroke}{rgb}{0.000000,0.000000,0.000000}%
\pgfsetstrokecolor{currentstroke}%
\pgfsetdash{}{0pt}%
\pgfpathmoveto{\pgfqpoint{1.440701in}{1.032462in}}%
\pgfpathlineto{\pgfqpoint{1.443372in}{1.023372in}}%
\pgfpathlineto{\pgfqpoint{1.446042in}{1.014306in}}%
\pgfpathlineto{\pgfqpoint{1.448712in}{1.005269in}}%
\pgfpathlineto{\pgfqpoint{1.451382in}{0.996262in}}%
\pgfpathlineto{\pgfqpoint{1.442602in}{0.991933in}}%
\pgfpathlineto{\pgfqpoint{1.433544in}{0.987746in}}%
\pgfpathlineto{\pgfqpoint{1.424217in}{0.983705in}}%
\pgfpathlineto{\pgfqpoint{1.414631in}{0.979816in}}%
\pgfpathlineto{\pgfqpoint{1.412318in}{0.989005in}}%
\pgfpathlineto{\pgfqpoint{1.410004in}{0.998225in}}%
\pgfpathlineto{\pgfqpoint{1.407690in}{1.007473in}}%
\pgfpathlineto{\pgfqpoint{1.405375in}{1.016745in}}%
\pgfpathlineto{\pgfqpoint{1.414589in}{1.020461in}}%
\pgfpathlineto{\pgfqpoint{1.423553in}{1.024322in}}%
\pgfpathlineto{\pgfqpoint{1.432260in}{1.028324in}}%
\pgfpathlineto{\pgfqpoint{1.440701in}{1.032462in}}%
\pgfpathclose%
\pgfusepath{fill}%
\end{pgfscope}%
\begin{pgfscope}%
\pgfpathrectangle{\pgfqpoint{0.041670in}{0.041670in}}{\pgfqpoint{2.216660in}{2.216660in}}%
\pgfusepath{clip}%
\pgfsetbuttcap%
\pgfsetroundjoin%
\definecolor{currentfill}{rgb}{0.279566,0.067836,0.391917}%
\pgfsetfillcolor{currentfill}%
\pgfsetlinewidth{0.000000pt}%
\definecolor{currentstroke}{rgb}{0.000000,0.000000,0.000000}%
\pgfsetstrokecolor{currentstroke}%
\pgfsetdash{}{0pt}%
\pgfpathmoveto{\pgfqpoint{0.929660in}{0.674182in}}%
\pgfpathlineto{\pgfqpoint{0.927807in}{0.668356in}}%
\pgfpathlineto{\pgfqpoint{0.925951in}{0.662699in}}%
\pgfpathlineto{\pgfqpoint{0.924093in}{0.657214in}}%
\pgfpathlineto{\pgfqpoint{0.922232in}{0.651906in}}%
\pgfpathlineto{\pgfqpoint{0.907314in}{0.656411in}}%
\pgfpathlineto{\pgfqpoint{0.892696in}{0.661164in}}%
\pgfpathlineto{\pgfqpoint{0.878392in}{0.666159in}}%
\pgfpathlineto{\pgfqpoint{0.864419in}{0.671391in}}%
\pgfpathlineto{\pgfqpoint{0.866691in}{0.676545in}}%
\pgfpathlineto{\pgfqpoint{0.868961in}{0.681875in}}%
\pgfpathlineto{\pgfqpoint{0.871227in}{0.687379in}}%
\pgfpathlineto{\pgfqpoint{0.873491in}{0.693050in}}%
\pgfpathlineto{\pgfqpoint{0.887068in}{0.687984in}}%
\pgfpathlineto{\pgfqpoint{0.900966in}{0.683147in}}%
\pgfpathlineto{\pgfqpoint{0.915168in}{0.678545in}}%
\pgfpathlineto{\pgfqpoint{0.929660in}{0.674182in}}%
\pgfpathclose%
\pgfusepath{fill}%
\end{pgfscope}%
\begin{pgfscope}%
\pgfpathrectangle{\pgfqpoint{0.041670in}{0.041670in}}{\pgfqpoint{2.216660in}{2.216660in}}%
\pgfusepath{clip}%
\pgfsetbuttcap%
\pgfsetroundjoin%
\definecolor{currentfill}{rgb}{0.896320,0.893616,0.096335}%
\pgfsetfillcolor{currentfill}%
\pgfsetlinewidth{0.000000pt}%
\definecolor{currentstroke}{rgb}{0.000000,0.000000,0.000000}%
\pgfsetstrokecolor{currentstroke}%
\pgfsetdash{}{0pt}%
\pgfpathmoveto{\pgfqpoint{1.253513in}{1.639374in}}%
\pgfpathlineto{\pgfqpoint{1.257174in}{1.637295in}}%
\pgfpathlineto{\pgfqpoint{1.260833in}{1.635091in}}%
\pgfpathlineto{\pgfqpoint{1.264489in}{1.632763in}}%
\pgfpathlineto{\pgfqpoint{1.268144in}{1.630311in}}%
\pgfpathlineto{\pgfqpoint{1.269095in}{1.629002in}}%
\pgfpathlineto{\pgfqpoint{1.269959in}{1.627680in}}%
\pgfpathlineto{\pgfqpoint{1.270733in}{1.626345in}}%
\pgfpathlineto{\pgfqpoint{1.271417in}{1.624999in}}%
\pgfpathlineto{\pgfqpoint{1.267624in}{1.627673in}}%
\pgfpathlineto{\pgfqpoint{1.263829in}{1.630223in}}%
\pgfpathlineto{\pgfqpoint{1.260031in}{1.632649in}}%
\pgfpathlineto{\pgfqpoint{1.256232in}{1.634950in}}%
\pgfpathlineto{\pgfqpoint{1.255665in}{1.636071in}}%
\pgfpathlineto{\pgfqpoint{1.255022in}{1.637182in}}%
\pgfpathlineto{\pgfqpoint{1.254304in}{1.638284in}}%
\pgfpathlineto{\pgfqpoint{1.253513in}{1.639374in}}%
\pgfpathclose%
\pgfusepath{fill}%
\end{pgfscope}%
\begin{pgfscope}%
\pgfpathrectangle{\pgfqpoint{0.041670in}{0.041670in}}{\pgfqpoint{2.216660in}{2.216660in}}%
\pgfusepath{clip}%
\pgfsetbuttcap%
\pgfsetroundjoin%
\definecolor{currentfill}{rgb}{0.855810,0.888601,0.097452}%
\pgfsetfillcolor{currentfill}%
\pgfsetlinewidth{0.000000pt}%
\definecolor{currentstroke}{rgb}{0.000000,0.000000,0.000000}%
\pgfsetstrokecolor{currentstroke}%
\pgfsetdash{}{0pt}%
\pgfpathmoveto{\pgfqpoint{1.271417in}{1.624999in}}%
\pgfpathlineto{\pgfqpoint{1.275208in}{1.622203in}}%
\pgfpathlineto{\pgfqpoint{1.278997in}{1.619286in}}%
\pgfpathlineto{\pgfqpoint{1.282783in}{1.616248in}}%
\pgfpathlineto{\pgfqpoint{1.286568in}{1.613090in}}%
\pgfpathlineto{\pgfqpoint{1.287263in}{1.611507in}}%
\pgfpathlineto{\pgfqpoint{1.287852in}{1.609914in}}%
\pgfpathlineto{\pgfqpoint{1.288333in}{1.608313in}}%
\pgfpathlineto{\pgfqpoint{1.288707in}{1.606705in}}%
\pgfpathlineto{\pgfqpoint{1.284843in}{1.610092in}}%
\pgfpathlineto{\pgfqpoint{1.280977in}{1.613360in}}%
\pgfpathlineto{\pgfqpoint{1.277109in}{1.616506in}}%
\pgfpathlineto{\pgfqpoint{1.273239in}{1.619532in}}%
\pgfpathlineto{\pgfqpoint{1.272922in}{1.620908in}}%
\pgfpathlineto{\pgfqpoint{1.272512in}{1.622279in}}%
\pgfpathlineto{\pgfqpoint{1.272010in}{1.623643in}}%
\pgfpathlineto{\pgfqpoint{1.271417in}{1.624999in}}%
\pgfpathclose%
\pgfusepath{fill}%
\end{pgfscope}%
\begin{pgfscope}%
\pgfpathrectangle{\pgfqpoint{0.041670in}{0.041670in}}{\pgfqpoint{2.216660in}{2.216660in}}%
\pgfusepath{clip}%
\pgfsetbuttcap%
\pgfsetroundjoin%
\definecolor{currentfill}{rgb}{0.274128,0.199721,0.498911}%
\pgfsetfillcolor{currentfill}%
\pgfsetlinewidth{0.000000pt}%
\definecolor{currentstroke}{rgb}{0.000000,0.000000,0.000000}%
\pgfsetstrokecolor{currentstroke}%
\pgfsetdash{}{0pt}%
\pgfpathmoveto{\pgfqpoint{0.959077in}{0.786996in}}%
\pgfpathlineto{\pgfqpoint{0.957249in}{0.779021in}}%
\pgfpathlineto{\pgfqpoint{0.955419in}{0.771152in}}%
\pgfpathlineto{\pgfqpoint{0.953588in}{0.763392in}}%
\pgfpathlineto{\pgfqpoint{0.951757in}{0.755745in}}%
\pgfpathlineto{\pgfqpoint{0.938531in}{0.759679in}}%
\pgfpathlineto{\pgfqpoint{0.925568in}{0.763829in}}%
\pgfpathlineto{\pgfqpoint{0.912881in}{0.768192in}}%
\pgfpathlineto{\pgfqpoint{0.900483in}{0.772762in}}%
\pgfpathlineto{\pgfqpoint{0.902721in}{0.780255in}}%
\pgfpathlineto{\pgfqpoint{0.904958in}{0.787862in}}%
\pgfpathlineto{\pgfqpoint{0.907194in}{0.795577in}}%
\pgfpathlineto{\pgfqpoint{0.909429in}{0.803398in}}%
\pgfpathlineto{\pgfqpoint{0.921435in}{0.798993in}}%
\pgfpathlineto{\pgfqpoint{0.933720in}{0.794787in}}%
\pgfpathlineto{\pgfqpoint{0.946272in}{0.790787in}}%
\pgfpathlineto{\pgfqpoint{0.959077in}{0.786996in}}%
\pgfpathclose%
\pgfusepath{fill}%
\end{pgfscope}%
\begin{pgfscope}%
\pgfpathrectangle{\pgfqpoint{0.041670in}{0.041670in}}{\pgfqpoint{2.216660in}{2.216660in}}%
\pgfusepath{clip}%
\pgfsetbuttcap%
\pgfsetroundjoin%
\definecolor{currentfill}{rgb}{0.974417,0.903590,0.130215}%
\pgfsetfillcolor{currentfill}%
\pgfsetlinewidth{0.000000pt}%
\definecolor{currentstroke}{rgb}{0.000000,0.000000,0.000000}%
\pgfsetstrokecolor{currentstroke}%
\pgfsetdash{}{0pt}%
\pgfpathmoveto{\pgfqpoint{1.176079in}{1.663935in}}%
\pgfpathlineto{\pgfqpoint{1.175595in}{1.664604in}}%
\pgfpathlineto{\pgfqpoint{1.175113in}{1.665141in}}%
\pgfpathlineto{\pgfqpoint{1.174630in}{1.665546in}}%
\pgfpathlineto{\pgfqpoint{1.174148in}{1.665819in}}%
\pgfpathlineto{\pgfqpoint{1.175616in}{1.665893in}}%
\pgfpathlineto{\pgfqpoint{1.177087in}{1.665946in}}%
\pgfpathlineto{\pgfqpoint{1.178561in}{1.665977in}}%
\pgfpathlineto{\pgfqpoint{1.180037in}{1.665987in}}%
\pgfpathlineto{\pgfqpoint{1.180030in}{1.665700in}}%
\pgfpathlineto{\pgfqpoint{1.180023in}{1.665281in}}%
\pgfpathlineto{\pgfqpoint{1.180016in}{1.664731in}}%
\pgfpathlineto{\pgfqpoint{1.180010in}{1.664048in}}%
\pgfpathlineto{\pgfqpoint{1.179024in}{1.664041in}}%
\pgfpathlineto{\pgfqpoint{1.178040in}{1.664020in}}%
\pgfpathlineto{\pgfqpoint{1.177058in}{1.663985in}}%
\pgfpathlineto{\pgfqpoint{1.176079in}{1.663935in}}%
\pgfpathclose%
\pgfusepath{fill}%
\end{pgfscope}%
\begin{pgfscope}%
\pgfpathrectangle{\pgfqpoint{0.041670in}{0.041670in}}{\pgfqpoint{2.216660in}{2.216660in}}%
\pgfusepath{clip}%
\pgfsetbuttcap%
\pgfsetroundjoin%
\definecolor{currentfill}{rgb}{0.974417,0.903590,0.130215}%
\pgfsetfillcolor{currentfill}%
\pgfsetlinewidth{0.000000pt}%
\definecolor{currentstroke}{rgb}{0.000000,0.000000,0.000000}%
\pgfsetstrokecolor{currentstroke}%
\pgfsetdash{}{0pt}%
\pgfpathmoveto{\pgfqpoint{1.180010in}{1.664048in}}%
\pgfpathlineto{\pgfqpoint{1.180016in}{1.664731in}}%
\pgfpathlineto{\pgfqpoint{1.180023in}{1.665281in}}%
\pgfpathlineto{\pgfqpoint{1.180030in}{1.665700in}}%
\pgfpathlineto{\pgfqpoint{1.180037in}{1.665987in}}%
\pgfpathlineto{\pgfqpoint{1.181512in}{1.665975in}}%
\pgfpathlineto{\pgfqpoint{1.182986in}{1.665941in}}%
\pgfpathlineto{\pgfqpoint{1.184457in}{1.665886in}}%
\pgfpathlineto{\pgfqpoint{1.185924in}{1.665809in}}%
\pgfpathlineto{\pgfqpoint{1.185429in}{1.665537in}}%
\pgfpathlineto{\pgfqpoint{1.184933in}{1.665133in}}%
\pgfpathlineto{\pgfqpoint{1.184436in}{1.664597in}}%
\pgfpathlineto{\pgfqpoint{1.183940in}{1.663929in}}%
\pgfpathlineto{\pgfqpoint{1.182961in}{1.663980in}}%
\pgfpathlineto{\pgfqpoint{1.181979in}{1.664017in}}%
\pgfpathlineto{\pgfqpoint{1.180995in}{1.664040in}}%
\pgfpathlineto{\pgfqpoint{1.180010in}{1.664048in}}%
\pgfpathclose%
\pgfusepath{fill}%
\end{pgfscope}%
\begin{pgfscope}%
\pgfpathrectangle{\pgfqpoint{0.041670in}{0.041670in}}{\pgfqpoint{2.216660in}{2.216660in}}%
\pgfusepath{clip}%
\pgfsetbuttcap%
\pgfsetroundjoin%
\definecolor{currentfill}{rgb}{0.974417,0.903590,0.130215}%
\pgfsetfillcolor{currentfill}%
\pgfsetlinewidth{0.000000pt}%
\definecolor{currentstroke}{rgb}{0.000000,0.000000,0.000000}%
\pgfsetstrokecolor{currentstroke}%
\pgfsetdash{}{0pt}%
\pgfpathmoveto{\pgfqpoint{1.172209in}{1.663594in}}%
\pgfpathlineto{\pgfqpoint{1.171243in}{1.664220in}}%
\pgfpathlineto{\pgfqpoint{1.170279in}{1.664714in}}%
\pgfpathlineto{\pgfqpoint{1.169315in}{1.665076in}}%
\pgfpathlineto{\pgfqpoint{1.168351in}{1.665307in}}%
\pgfpathlineto{\pgfqpoint{1.169787in}{1.665467in}}%
\pgfpathlineto{\pgfqpoint{1.171233in}{1.665605in}}%
\pgfpathlineto{\pgfqpoint{1.172687in}{1.665723in}}%
\pgfpathlineto{\pgfqpoint{1.174148in}{1.665819in}}%
\pgfpathlineto{\pgfqpoint{1.174630in}{1.665546in}}%
\pgfpathlineto{\pgfqpoint{1.175113in}{1.665141in}}%
\pgfpathlineto{\pgfqpoint{1.175595in}{1.664604in}}%
\pgfpathlineto{\pgfqpoint{1.176079in}{1.663935in}}%
\pgfpathlineto{\pgfqpoint{1.175103in}{1.663871in}}%
\pgfpathlineto{\pgfqpoint{1.174132in}{1.663793in}}%
\pgfpathlineto{\pgfqpoint{1.173167in}{1.663700in}}%
\pgfpathlineto{\pgfqpoint{1.172209in}{1.663594in}}%
\pgfpathclose%
\pgfusepath{fill}%
\end{pgfscope}%
\begin{pgfscope}%
\pgfpathrectangle{\pgfqpoint{0.041670in}{0.041670in}}{\pgfqpoint{2.216660in}{2.216660in}}%
\pgfusepath{clip}%
\pgfsetbuttcap%
\pgfsetroundjoin%
\definecolor{currentfill}{rgb}{0.974417,0.903590,0.130215}%
\pgfsetfillcolor{currentfill}%
\pgfsetlinewidth{0.000000pt}%
\definecolor{currentstroke}{rgb}{0.000000,0.000000,0.000000}%
\pgfsetstrokecolor{currentstroke}%
\pgfsetdash{}{0pt}%
\pgfpathmoveto{\pgfqpoint{1.183940in}{1.663929in}}%
\pgfpathlineto{\pgfqpoint{1.184436in}{1.664597in}}%
\pgfpathlineto{\pgfqpoint{1.184933in}{1.665133in}}%
\pgfpathlineto{\pgfqpoint{1.185429in}{1.665537in}}%
\pgfpathlineto{\pgfqpoint{1.185924in}{1.665809in}}%
\pgfpathlineto{\pgfqpoint{1.187385in}{1.665711in}}%
\pgfpathlineto{\pgfqpoint{1.188838in}{1.665591in}}%
\pgfpathlineto{\pgfqpoint{1.190283in}{1.665450in}}%
\pgfpathlineto{\pgfqpoint{1.191717in}{1.665288in}}%
\pgfpathlineto{\pgfqpoint{1.190741in}{1.665059in}}%
\pgfpathlineto{\pgfqpoint{1.189763in}{1.664698in}}%
\pgfpathlineto{\pgfqpoint{1.188785in}{1.664206in}}%
\pgfpathlineto{\pgfqpoint{1.187807in}{1.663581in}}%
\pgfpathlineto{\pgfqpoint{1.186849in}{1.663689in}}%
\pgfpathlineto{\pgfqpoint{1.185885in}{1.663783in}}%
\pgfpathlineto{\pgfqpoint{1.184915in}{1.663863in}}%
\pgfpathlineto{\pgfqpoint{1.183940in}{1.663929in}}%
\pgfpathclose%
\pgfusepath{fill}%
\end{pgfscope}%
\begin{pgfscope}%
\pgfpathrectangle{\pgfqpoint{0.041670in}{0.041670in}}{\pgfqpoint{2.216660in}{2.216660in}}%
\pgfusepath{clip}%
\pgfsetbuttcap%
\pgfsetroundjoin%
\definecolor{currentfill}{rgb}{0.271305,0.019942,0.347269}%
\pgfsetfillcolor{currentfill}%
\pgfsetlinewidth{0.000000pt}%
\definecolor{currentstroke}{rgb}{0.000000,0.000000,0.000000}%
\pgfsetstrokecolor{currentstroke}%
\pgfsetdash{}{0pt}%
\pgfpathmoveto{\pgfqpoint{1.517089in}{0.657628in}}%
\pgfpathlineto{\pgfqpoint{1.519465in}{0.653462in}}%
\pgfpathlineto{\pgfqpoint{1.521844in}{0.649499in}}%
\pgfpathlineto{\pgfqpoint{1.524228in}{0.645744in}}%
\pgfpathlineto{\pgfqpoint{1.526616in}{0.642201in}}%
\pgfpathlineto{\pgfqpoint{1.512170in}{0.636423in}}%
\pgfpathlineto{\pgfqpoint{1.497359in}{0.630890in}}%
\pgfpathlineto{\pgfqpoint{1.482199in}{0.625608in}}%
\pgfpathlineto{\pgfqpoint{1.466706in}{0.620584in}}%
\pgfpathlineto{\pgfqpoint{1.464725in}{0.624287in}}%
\pgfpathlineto{\pgfqpoint{1.462748in}{0.628201in}}%
\pgfpathlineto{\pgfqpoint{1.460774in}{0.632324in}}%
\pgfpathlineto{\pgfqpoint{1.458804in}{0.636650in}}%
\pgfpathlineto{\pgfqpoint{1.473875in}{0.641525in}}%
\pgfpathlineto{\pgfqpoint{1.488623in}{0.646651in}}%
\pgfpathlineto{\pgfqpoint{1.503033in}{0.652020in}}%
\pgfpathlineto{\pgfqpoint{1.517089in}{0.657628in}}%
\pgfpathclose%
\pgfusepath{fill}%
\end{pgfscope}%
\begin{pgfscope}%
\pgfpathrectangle{\pgfqpoint{0.041670in}{0.041670in}}{\pgfqpoint{2.216660in}{2.216660in}}%
\pgfusepath{clip}%
\pgfsetbuttcap%
\pgfsetroundjoin%
\definecolor{currentfill}{rgb}{0.565498,0.842430,0.262877}%
\pgfsetfillcolor{currentfill}%
\pgfsetlinewidth{0.000000pt}%
\definecolor{currentstroke}{rgb}{0.000000,0.000000,0.000000}%
\pgfsetstrokecolor{currentstroke}%
\pgfsetdash{}{0pt}%
\pgfpathmoveto{\pgfqpoint{1.017055in}{1.504500in}}%
\pgfpathlineto{\pgfqpoint{1.013380in}{1.498500in}}%
\pgfpathlineto{\pgfqpoint{1.009708in}{1.492404in}}%
\pgfpathlineto{\pgfqpoint{1.006038in}{1.486212in}}%
\pgfpathlineto{\pgfqpoint{1.002370in}{1.479926in}}%
\pgfpathlineto{\pgfqpoint{1.000662in}{1.482620in}}%
\pgfpathlineto{\pgfqpoint{0.999135in}{1.485338in}}%
\pgfpathlineto{\pgfqpoint{0.997789in}{1.488076in}}%
\pgfpathlineto{\pgfqpoint{0.996627in}{1.490833in}}%
\pgfpathlineto{\pgfqpoint{1.000408in}{1.496883in}}%
\pgfpathlineto{\pgfqpoint{1.004191in}{1.502841in}}%
\pgfpathlineto{\pgfqpoint{1.007977in}{1.508703in}}%
\pgfpathlineto{\pgfqpoint{1.011764in}{1.514469in}}%
\pgfpathlineto{\pgfqpoint{1.012836in}{1.511950in}}%
\pgfpathlineto{\pgfqpoint{1.014076in}{1.509447in}}%
\pgfpathlineto{\pgfqpoint{1.015483in}{1.506962in}}%
\pgfpathlineto{\pgfqpoint{1.017055in}{1.504500in}}%
\pgfpathclose%
\pgfusepath{fill}%
\end{pgfscope}%
\begin{pgfscope}%
\pgfpathrectangle{\pgfqpoint{0.041670in}{0.041670in}}{\pgfqpoint{2.216660in}{2.216660in}}%
\pgfusepath{clip}%
\pgfsetbuttcap%
\pgfsetroundjoin%
\definecolor{currentfill}{rgb}{0.248629,0.278775,0.534556}%
\pgfsetfillcolor{currentfill}%
\pgfsetlinewidth{0.000000pt}%
\definecolor{currentstroke}{rgb}{0.000000,0.000000,0.000000}%
\pgfsetstrokecolor{currentstroke}%
\pgfsetdash{}{0pt}%
\pgfpathmoveto{\pgfqpoint{1.442382in}{0.873101in}}%
\pgfpathlineto{\pgfqpoint{1.444696in}{0.864604in}}%
\pgfpathlineto{\pgfqpoint{1.447010in}{0.856185in}}%
\pgfpathlineto{\pgfqpoint{1.449325in}{0.847845in}}%
\pgfpathlineto{\pgfqpoint{1.451640in}{0.839590in}}%
\pgfpathlineto{\pgfqpoint{1.440275in}{0.835182in}}%
\pgfpathlineto{\pgfqpoint{1.428629in}{0.830962in}}%
\pgfpathlineto{\pgfqpoint{1.416716in}{0.826936in}}%
\pgfpathlineto{\pgfqpoint{1.404547in}{0.823107in}}%
\pgfpathlineto{\pgfqpoint{1.402629in}{0.831522in}}%
\pgfpathlineto{\pgfqpoint{1.400711in}{0.840021in}}%
\pgfpathlineto{\pgfqpoint{1.398794in}{0.848599in}}%
\pgfpathlineto{\pgfqpoint{1.396877in}{0.857255in}}%
\pgfpathlineto{\pgfqpoint{1.408635in}{0.860935in}}%
\pgfpathlineto{\pgfqpoint{1.420146in}{0.864806in}}%
\pgfpathlineto{\pgfqpoint{1.431399in}{0.868863in}}%
\pgfpathlineto{\pgfqpoint{1.442382in}{0.873101in}}%
\pgfpathclose%
\pgfusepath{fill}%
\end{pgfscope}%
\begin{pgfscope}%
\pgfpathrectangle{\pgfqpoint{0.041670in}{0.041670in}}{\pgfqpoint{2.216660in}{2.216660in}}%
\pgfusepath{clip}%
\pgfsetbuttcap%
\pgfsetroundjoin%
\definecolor{currentfill}{rgb}{0.268510,0.009605,0.335427}%
\pgfsetfillcolor{currentfill}%
\pgfsetlinewidth{0.000000pt}%
\definecolor{currentstroke}{rgb}{0.000000,0.000000,0.000000}%
\pgfsetstrokecolor{currentstroke}%
\pgfsetdash{}{0pt}%
\pgfpathmoveto{\pgfqpoint{1.613896in}{0.645604in}}%
\pgfpathlineto{\pgfqpoint{1.616725in}{0.645439in}}%
\pgfpathlineto{\pgfqpoint{1.619561in}{0.645556in}}%
\pgfpathlineto{\pgfqpoint{1.622406in}{0.645959in}}%
\pgfpathlineto{\pgfqpoint{1.625258in}{0.646653in}}%
\pgfpathlineto{\pgfqpoint{1.611022in}{0.639198in}}%
\pgfpathlineto{\pgfqpoint{1.596313in}{0.631982in}}%
\pgfpathlineto{\pgfqpoint{1.581146in}{0.625013in}}%
\pgfpathlineto{\pgfqpoint{1.565537in}{0.618301in}}%
\pgfpathlineto{\pgfqpoint{1.563057in}{0.617787in}}%
\pgfpathlineto{\pgfqpoint{1.560585in}{0.617564in}}%
\pgfpathlineto{\pgfqpoint{1.558120in}{0.617628in}}%
\pgfpathlineto{\pgfqpoint{1.555661in}{0.617974in}}%
\pgfpathlineto{\pgfqpoint{1.570880in}{0.624515in}}%
\pgfpathlineto{\pgfqpoint{1.585669in}{0.631305in}}%
\pgfpathlineto{\pgfqpoint{1.600012in}{0.638338in}}%
\pgfpathlineto{\pgfqpoint{1.613896in}{0.645604in}}%
\pgfpathclose%
\pgfusepath{fill}%
\end{pgfscope}%
\begin{pgfscope}%
\pgfpathrectangle{\pgfqpoint{0.041670in}{0.041670in}}{\pgfqpoint{2.216660in}{2.216660in}}%
\pgfusepath{clip}%
\pgfsetbuttcap%
\pgfsetroundjoin%
\definecolor{currentfill}{rgb}{0.896320,0.893616,0.096335}%
\pgfsetfillcolor{currentfill}%
\pgfsetlinewidth{0.000000pt}%
\definecolor{currentstroke}{rgb}{0.000000,0.000000,0.000000}%
\pgfsetstrokecolor{currentstroke}%
\pgfsetdash{}{0pt}%
\pgfpathmoveto{\pgfqpoint{1.103236in}{1.633947in}}%
\pgfpathlineto{\pgfqpoint{1.099415in}{1.631595in}}%
\pgfpathlineto{\pgfqpoint{1.095595in}{1.629119in}}%
\pgfpathlineto{\pgfqpoint{1.091777in}{1.626518in}}%
\pgfpathlineto{\pgfqpoint{1.087961in}{1.623795in}}%
\pgfpathlineto{\pgfqpoint{1.088564in}{1.625149in}}%
\pgfpathlineto{\pgfqpoint{1.089259in}{1.626494in}}%
\pgfpathlineto{\pgfqpoint{1.090043in}{1.627827in}}%
\pgfpathlineto{\pgfqpoint{1.090916in}{1.629149in}}%
\pgfpathlineto{\pgfqpoint{1.094606in}{1.631648in}}%
\pgfpathlineto{\pgfqpoint{1.098299in}{1.634025in}}%
\pgfpathlineto{\pgfqpoint{1.101993in}{1.636278in}}%
\pgfpathlineto{\pgfqpoint{1.105690in}{1.638405in}}%
\pgfpathlineto{\pgfqpoint{1.104964in}{1.637305in}}%
\pgfpathlineto{\pgfqpoint{1.104313in}{1.636195in}}%
\pgfpathlineto{\pgfqpoint{1.103737in}{1.635075in}}%
\pgfpathlineto{\pgfqpoint{1.103236in}{1.633947in}}%
\pgfpathclose%
\pgfusepath{fill}%
\end{pgfscope}%
\begin{pgfscope}%
\pgfpathrectangle{\pgfqpoint{0.041670in}{0.041670in}}{\pgfqpoint{2.216660in}{2.216660in}}%
\pgfusepath{clip}%
\pgfsetbuttcap%
\pgfsetroundjoin%
\definecolor{currentfill}{rgb}{0.955300,0.901065,0.118128}%
\pgfsetfillcolor{currentfill}%
\pgfsetlinewidth{0.000000pt}%
\definecolor{currentstroke}{rgb}{0.000000,0.000000,0.000000}%
\pgfsetstrokecolor{currentstroke}%
\pgfsetdash{}{0pt}%
\pgfpathmoveto{\pgfqpoint{1.215243in}{1.658521in}}%
\pgfpathlineto{\pgfqpoint{1.218174in}{1.658057in}}%
\pgfpathlineto{\pgfqpoint{1.221103in}{1.657463in}}%
\pgfpathlineto{\pgfqpoint{1.224030in}{1.656739in}}%
\pgfpathlineto{\pgfqpoint{1.226955in}{1.655886in}}%
\pgfpathlineto{\pgfqpoint{1.228229in}{1.655186in}}%
\pgfpathlineto{\pgfqpoint{1.229455in}{1.654467in}}%
\pgfpathlineto{\pgfqpoint{1.230633in}{1.653730in}}%
\pgfpathlineto{\pgfqpoint{1.231760in}{1.652977in}}%
\pgfpathlineto{\pgfqpoint{1.228535in}{1.654011in}}%
\pgfpathlineto{\pgfqpoint{1.225307in}{1.654916in}}%
\pgfpathlineto{\pgfqpoint{1.222078in}{1.655692in}}%
\pgfpathlineto{\pgfqpoint{1.218847in}{1.656337in}}%
\pgfpathlineto{\pgfqpoint{1.218002in}{1.656903in}}%
\pgfpathlineto{\pgfqpoint{1.217119in}{1.657455in}}%
\pgfpathlineto{\pgfqpoint{1.216199in}{1.657995in}}%
\pgfpathlineto{\pgfqpoint{1.215243in}{1.658521in}}%
\pgfpathclose%
\pgfusepath{fill}%
\end{pgfscope}%
\begin{pgfscope}%
\pgfpathrectangle{\pgfqpoint{0.041670in}{0.041670in}}{\pgfqpoint{2.216660in}{2.216660in}}%
\pgfusepath{clip}%
\pgfsetbuttcap%
\pgfsetroundjoin%
\definecolor{currentfill}{rgb}{0.974417,0.903590,0.130215}%
\pgfsetfillcolor{currentfill}%
\pgfsetlinewidth{0.000000pt}%
\definecolor{currentstroke}{rgb}{0.000000,0.000000,0.000000}%
\pgfsetstrokecolor{currentstroke}%
\pgfsetdash{}{0pt}%
\pgfpathmoveto{\pgfqpoint{1.187807in}{1.663581in}}%
\pgfpathlineto{\pgfqpoint{1.188785in}{1.664206in}}%
\pgfpathlineto{\pgfqpoint{1.189763in}{1.664698in}}%
\pgfpathlineto{\pgfqpoint{1.190741in}{1.665059in}}%
\pgfpathlineto{\pgfqpoint{1.191717in}{1.665288in}}%
\pgfpathlineto{\pgfqpoint{1.193140in}{1.665105in}}%
\pgfpathlineto{\pgfqpoint{1.194550in}{1.664901in}}%
\pgfpathlineto{\pgfqpoint{1.195945in}{1.664676in}}%
\pgfpathlineto{\pgfqpoint{1.194617in}{1.664498in}}%
\pgfpathlineto{\pgfqpoint{1.193289in}{1.664188in}}%
\pgfpathlineto{\pgfqpoint{1.191959in}{1.663747in}}%
\pgfpathlineto{\pgfqpoint{1.190629in}{1.663173in}}%
\pgfpathlineto{\pgfqpoint{1.189698in}{1.663322in}}%
\pgfpathlineto{\pgfqpoint{1.188757in}{1.663459in}}%
\pgfpathlineto{\pgfqpoint{1.187807in}{1.663581in}}%
\pgfpathclose%
\pgfusepath{fill}%
\end{pgfscope}%
\begin{pgfscope}%
\pgfpathrectangle{\pgfqpoint{0.041670in}{0.041670in}}{\pgfqpoint{2.216660in}{2.216660in}}%
\pgfusepath{clip}%
\pgfsetbuttcap%
\pgfsetroundjoin%
\definecolor{currentfill}{rgb}{0.281477,0.755203,0.432552}%
\pgfsetfillcolor{currentfill}%
\pgfsetlinewidth{0.000000pt}%
\definecolor{currentstroke}{rgb}{0.000000,0.000000,0.000000}%
\pgfsetstrokecolor{currentstroke}%
\pgfsetdash{}{0pt}%
\pgfpathmoveto{\pgfqpoint{1.393403in}{1.387469in}}%
\pgfpathlineto{\pgfqpoint{1.396915in}{1.379948in}}%
\pgfpathlineto{\pgfqpoint{1.400425in}{1.372359in}}%
\pgfpathlineto{\pgfqpoint{1.403933in}{1.364705in}}%
\pgfpathlineto{\pgfqpoint{1.407439in}{1.356987in}}%
\pgfpathlineto{\pgfqpoint{1.404234in}{1.353511in}}%
\pgfpathlineto{\pgfqpoint{1.400799in}{1.350085in}}%
\pgfpathlineto{\pgfqpoint{1.397139in}{1.346711in}}%
\pgfpathlineto{\pgfqpoint{1.393257in}{1.343394in}}%
\pgfpathlineto{\pgfqpoint{1.389963in}{1.351337in}}%
\pgfpathlineto{\pgfqpoint{1.386668in}{1.359216in}}%
\pgfpathlineto{\pgfqpoint{1.383370in}{1.367029in}}%
\pgfpathlineto{\pgfqpoint{1.380071in}{1.374774in}}%
\pgfpathlineto{\pgfqpoint{1.383719in}{1.377872in}}%
\pgfpathlineto{\pgfqpoint{1.387159in}{1.381023in}}%
\pgfpathlineto{\pgfqpoint{1.390388in}{1.384223in}}%
\pgfpathlineto{\pgfqpoint{1.393403in}{1.387469in}}%
\pgfpathclose%
\pgfusepath{fill}%
\end{pgfscope}%
\begin{pgfscope}%
\pgfpathrectangle{\pgfqpoint{0.041670in}{0.041670in}}{\pgfqpoint{2.216660in}{2.216660in}}%
\pgfusepath{clip}%
\pgfsetbuttcap%
\pgfsetroundjoin%
\definecolor{currentfill}{rgb}{0.974417,0.903590,0.130215}%
\pgfsetfillcolor{currentfill}%
\pgfsetlinewidth{0.000000pt}%
\definecolor{currentstroke}{rgb}{0.000000,0.000000,0.000000}%
\pgfsetstrokecolor{currentstroke}%
\pgfsetdash{}{0pt}%
\pgfpathmoveto{\pgfqpoint{1.168462in}{1.663028in}}%
\pgfpathlineto{\pgfqpoint{1.167029in}{1.663584in}}%
\pgfpathlineto{\pgfqpoint{1.165598in}{1.664008in}}%
\pgfpathlineto{\pgfqpoint{1.164167in}{1.664299in}}%
\pgfpathlineto{\pgfqpoint{1.162738in}{1.664460in}}%
\pgfpathlineto{\pgfqpoint{1.164119in}{1.664702in}}%
\pgfpathlineto{\pgfqpoint{1.165516in}{1.664924in}}%
\pgfpathlineto{\pgfqpoint{1.166927in}{1.665126in}}%
\pgfpathlineto{\pgfqpoint{1.168351in}{1.665307in}}%
\pgfpathlineto{\pgfqpoint{1.169315in}{1.665076in}}%
\pgfpathlineto{\pgfqpoint{1.170279in}{1.664714in}}%
\pgfpathlineto{\pgfqpoint{1.171243in}{1.664220in}}%
\pgfpathlineto{\pgfqpoint{1.172209in}{1.663594in}}%
\pgfpathlineto{\pgfqpoint{1.171258in}{1.663473in}}%
\pgfpathlineto{\pgfqpoint{1.170316in}{1.663338in}}%
\pgfpathlineto{\pgfqpoint{1.169384in}{1.663190in}}%
\pgfpathlineto{\pgfqpoint{1.168462in}{1.663028in}}%
\pgfpathclose%
\pgfusepath{fill}%
\end{pgfscope}%
\begin{pgfscope}%
\pgfpathrectangle{\pgfqpoint{0.041670in}{0.041670in}}{\pgfqpoint{2.216660in}{2.216660in}}%
\pgfusepath{clip}%
\pgfsetbuttcap%
\pgfsetroundjoin%
\definecolor{currentfill}{rgb}{0.935904,0.898570,0.108131}%
\pgfsetfillcolor{currentfill}%
\pgfsetlinewidth{0.000000pt}%
\definecolor{currentstroke}{rgb}{0.000000,0.000000,0.000000}%
\pgfsetstrokecolor{currentstroke}%
\pgfsetdash{}{0pt}%
\pgfpathmoveto{\pgfqpoint{1.235747in}{1.649807in}}%
\pgfpathlineto{\pgfqpoint{1.239220in}{1.648446in}}%
\pgfpathlineto{\pgfqpoint{1.242691in}{1.646958in}}%
\pgfpathlineto{\pgfqpoint{1.246161in}{1.645342in}}%
\pgfpathlineto{\pgfqpoint{1.249628in}{1.643600in}}%
\pgfpathlineto{\pgfqpoint{1.250705in}{1.642566in}}%
\pgfpathlineto{\pgfqpoint{1.251713in}{1.641516in}}%
\pgfpathlineto{\pgfqpoint{1.252649in}{1.640452in}}%
\pgfpathlineto{\pgfqpoint{1.253513in}{1.639374in}}%
\pgfpathlineto{\pgfqpoint{1.249850in}{1.641327in}}%
\pgfpathlineto{\pgfqpoint{1.246185in}{1.643154in}}%
\pgfpathlineto{\pgfqpoint{1.242518in}{1.644854in}}%
\pgfpathlineto{\pgfqpoint{1.238850in}{1.646426in}}%
\pgfpathlineto{\pgfqpoint{1.238160in}{1.647288in}}%
\pgfpathlineto{\pgfqpoint{1.237412in}{1.648139in}}%
\pgfpathlineto{\pgfqpoint{1.236607in}{1.648979in}}%
\pgfpathlineto{\pgfqpoint{1.235747in}{1.649807in}}%
\pgfpathclose%
\pgfusepath{fill}%
\end{pgfscope}%
\begin{pgfscope}%
\pgfpathrectangle{\pgfqpoint{0.041670in}{0.041670in}}{\pgfqpoint{2.216660in}{2.216660in}}%
\pgfusepath{clip}%
\pgfsetbuttcap%
\pgfsetroundjoin%
\definecolor{currentfill}{rgb}{0.699415,0.867117,0.175971}%
\pgfsetfillcolor{currentfill}%
\pgfsetlinewidth{0.000000pt}%
\definecolor{currentstroke}{rgb}{0.000000,0.000000,0.000000}%
\pgfsetstrokecolor{currentstroke}%
\pgfsetdash{}{0pt}%
\pgfpathmoveto{\pgfqpoint{1.042147in}{1.556930in}}%
\pgfpathlineto{\pgfqpoint{1.038342in}{1.551992in}}%
\pgfpathlineto{\pgfqpoint{1.034538in}{1.546946in}}%
\pgfpathlineto{\pgfqpoint{1.030737in}{1.541793in}}%
\pgfpathlineto{\pgfqpoint{1.026938in}{1.536534in}}%
\pgfpathlineto{\pgfqpoint{1.026112in}{1.538831in}}%
\pgfpathlineto{\pgfqpoint{1.025439in}{1.541139in}}%
\pgfpathlineto{\pgfqpoint{1.024921in}{1.543454in}}%
\pgfpathlineto{\pgfqpoint{1.024558in}{1.545776in}}%
\pgfpathlineto{\pgfqpoint{1.028411in}{1.550798in}}%
\pgfpathlineto{\pgfqpoint{1.032266in}{1.555716in}}%
\pgfpathlineto{\pgfqpoint{1.036123in}{1.560526in}}%
\pgfpathlineto{\pgfqpoint{1.039983in}{1.565229in}}%
\pgfpathlineto{\pgfqpoint{1.040315in}{1.563144in}}%
\pgfpathlineto{\pgfqpoint{1.040786in}{1.561065in}}%
\pgfpathlineto{\pgfqpoint{1.041397in}{1.558993in}}%
\pgfpathlineto{\pgfqpoint{1.042147in}{1.556930in}}%
\pgfpathclose%
\pgfusepath{fill}%
\end{pgfscope}%
\begin{pgfscope}%
\pgfpathrectangle{\pgfqpoint{0.041670in}{0.041670in}}{\pgfqpoint{2.216660in}{2.216660in}}%
\pgfusepath{clip}%
\pgfsetbuttcap%
\pgfsetroundjoin%
\definecolor{currentfill}{rgb}{0.855810,0.888601,0.097452}%
\pgfsetfillcolor{currentfill}%
\pgfsetlinewidth{0.000000pt}%
\definecolor{currentstroke}{rgb}{0.000000,0.000000,0.000000}%
\pgfsetstrokecolor{currentstroke}%
\pgfsetdash{}{0pt}%
\pgfpathmoveto{\pgfqpoint{1.086466in}{1.618304in}}%
\pgfpathlineto{\pgfqpoint{1.082587in}{1.615228in}}%
\pgfpathlineto{\pgfqpoint{1.078710in}{1.612029in}}%
\pgfpathlineto{\pgfqpoint{1.074835in}{1.608710in}}%
\pgfpathlineto{\pgfqpoint{1.070961in}{1.605271in}}%
\pgfpathlineto{\pgfqpoint{1.071239in}{1.606884in}}%
\pgfpathlineto{\pgfqpoint{1.071625in}{1.608491in}}%
\pgfpathlineto{\pgfqpoint{1.072118in}{1.610092in}}%
\pgfpathlineto{\pgfqpoint{1.072719in}{1.611684in}}%
\pgfpathlineto{\pgfqpoint{1.076526in}{1.614892in}}%
\pgfpathlineto{\pgfqpoint{1.080336in}{1.617980in}}%
\pgfpathlineto{\pgfqpoint{1.084147in}{1.620948in}}%
\pgfpathlineto{\pgfqpoint{1.087961in}{1.623795in}}%
\pgfpathlineto{\pgfqpoint{1.087449in}{1.622431in}}%
\pgfpathlineto{\pgfqpoint{1.087029in}{1.621061in}}%
\pgfpathlineto{\pgfqpoint{1.086701in}{1.619685in}}%
\pgfpathlineto{\pgfqpoint{1.086466in}{1.618304in}}%
\pgfpathclose%
\pgfusepath{fill}%
\end{pgfscope}%
\begin{pgfscope}%
\pgfpathrectangle{\pgfqpoint{0.041670in}{0.041670in}}{\pgfqpoint{2.216660in}{2.216660in}}%
\pgfusepath{clip}%
\pgfsetbuttcap%
\pgfsetroundjoin%
\definecolor{currentfill}{rgb}{0.814576,0.883393,0.110347}%
\pgfsetfillcolor{currentfill}%
\pgfsetlinewidth{0.000000pt}%
\definecolor{currentstroke}{rgb}{0.000000,0.000000,0.000000}%
\pgfsetstrokecolor{currentstroke}%
\pgfsetdash{}{0pt}%
\pgfpathmoveto{\pgfqpoint{1.288707in}{1.606705in}}%
\pgfpathlineto{\pgfqpoint{1.292569in}{1.603199in}}%
\pgfpathlineto{\pgfqpoint{1.296428in}{1.599575in}}%
\pgfpathlineto{\pgfqpoint{1.300286in}{1.595835in}}%
\pgfpathlineto{\pgfqpoint{1.304141in}{1.591980in}}%
\pgfpathlineto{\pgfqpoint{1.304449in}{1.590133in}}%
\pgfpathlineto{\pgfqpoint{1.304633in}{1.588283in}}%
\pgfpathlineto{\pgfqpoint{1.304692in}{1.586430in}}%
\pgfpathlineto{\pgfqpoint{1.304628in}{1.584576in}}%
\pgfpathlineto{\pgfqpoint{1.300753in}{1.588666in}}%
\pgfpathlineto{\pgfqpoint{1.296877in}{1.592640in}}%
\pgfpathlineto{\pgfqpoint{1.292998in}{1.596497in}}%
\pgfpathlineto{\pgfqpoint{1.289118in}{1.600237in}}%
\pgfpathlineto{\pgfqpoint{1.289178in}{1.601856in}}%
\pgfpathlineto{\pgfqpoint{1.289130in}{1.603475in}}%
\pgfpathlineto{\pgfqpoint{1.288973in}{1.605092in}}%
\pgfpathlineto{\pgfqpoint{1.288707in}{1.606705in}}%
\pgfpathclose%
\pgfusepath{fill}%
\end{pgfscope}%
\begin{pgfscope}%
\pgfpathrectangle{\pgfqpoint{0.041670in}{0.041670in}}{\pgfqpoint{2.216660in}{2.216660in}}%
\pgfusepath{clip}%
\pgfsetbuttcap%
\pgfsetroundjoin%
\definecolor{currentfill}{rgb}{0.955300,0.901065,0.118128}%
\pgfsetfillcolor{currentfill}%
\pgfsetlinewidth{0.000000pt}%
\definecolor{currentstroke}{rgb}{0.000000,0.000000,0.000000}%
\pgfsetstrokecolor{currentstroke}%
\pgfsetdash{}{0pt}%
\pgfpathmoveto{\pgfqpoint{1.140344in}{1.655824in}}%
\pgfpathlineto{\pgfqpoint{1.137053in}{1.655136in}}%
\pgfpathlineto{\pgfqpoint{1.133763in}{1.654318in}}%
\pgfpathlineto{\pgfqpoint{1.130476in}{1.653370in}}%
\pgfpathlineto{\pgfqpoint{1.127190in}{1.652293in}}%
\pgfpathlineto{\pgfqpoint{1.128272in}{1.653061in}}%
\pgfpathlineto{\pgfqpoint{1.129405in}{1.653813in}}%
\pgfpathlineto{\pgfqpoint{1.130588in}{1.654548in}}%
\pgfpathlineto{\pgfqpoint{1.131820in}{1.655264in}}%
\pgfpathlineto{\pgfqpoint{1.134816in}{1.656156in}}%
\pgfpathlineto{\pgfqpoint{1.137814in}{1.656919in}}%
\pgfpathlineto{\pgfqpoint{1.140814in}{1.657551in}}%
\pgfpathlineto{\pgfqpoint{1.143815in}{1.658054in}}%
\pgfpathlineto{\pgfqpoint{1.142891in}{1.657516in}}%
\pgfpathlineto{\pgfqpoint{1.142004in}{1.656965in}}%
\pgfpathlineto{\pgfqpoint{1.141155in}{1.656401in}}%
\pgfpathlineto{\pgfqpoint{1.140344in}{1.655824in}}%
\pgfpathclose%
\pgfusepath{fill}%
\end{pgfscope}%
\begin{pgfscope}%
\pgfpathrectangle{\pgfqpoint{0.041670in}{0.041670in}}{\pgfqpoint{2.216660in}{2.216660in}}%
\pgfusepath{clip}%
\pgfsetbuttcap%
\pgfsetroundjoin%
\definecolor{currentfill}{rgb}{0.133743,0.548535,0.553541}%
\pgfsetfillcolor{currentfill}%
\pgfsetlinewidth{0.000000pt}%
\definecolor{currentstroke}{rgb}{0.000000,0.000000,0.000000}%
\pgfsetstrokecolor{currentstroke}%
\pgfsetdash{}{0pt}%
\pgfpathmoveto{\pgfqpoint{0.957910in}{1.139206in}}%
\pgfpathlineto{\pgfqpoint{0.955304in}{1.130008in}}%
\pgfpathlineto{\pgfqpoint{0.952699in}{1.120800in}}%
\pgfpathlineto{\pgfqpoint{0.950095in}{1.111585in}}%
\pgfpathlineto{\pgfqpoint{0.947492in}{1.102365in}}%
\pgfpathlineto{\pgfqpoint{0.939760in}{1.106137in}}%
\pgfpathlineto{\pgfqpoint{0.932281in}{1.110028in}}%
\pgfpathlineto{\pgfqpoint{0.925061in}{1.114034in}}%
\pgfpathlineto{\pgfqpoint{0.918107in}{1.118151in}}%
\pgfpathlineto{\pgfqpoint{0.921033in}{1.127175in}}%
\pgfpathlineto{\pgfqpoint{0.923959in}{1.136193in}}%
\pgfpathlineto{\pgfqpoint{0.926887in}{1.145205in}}%
\pgfpathlineto{\pgfqpoint{0.929817in}{1.154207in}}%
\pgfpathlineto{\pgfqpoint{0.936466in}{1.150294in}}%
\pgfpathlineto{\pgfqpoint{0.943369in}{1.146487in}}%
\pgfpathlineto{\pgfqpoint{0.950519in}{1.142790in}}%
\pgfpathlineto{\pgfqpoint{0.957910in}{1.139206in}}%
\pgfpathclose%
\pgfusepath{fill}%
\end{pgfscope}%
\begin{pgfscope}%
\pgfpathrectangle{\pgfqpoint{0.041670in}{0.041670in}}{\pgfqpoint{2.216660in}{2.216660in}}%
\pgfusepath{clip}%
\pgfsetbuttcap%
\pgfsetroundjoin%
\definecolor{currentfill}{rgb}{0.274952,0.037752,0.364543}%
\pgfsetfillcolor{currentfill}%
\pgfsetlinewidth{0.000000pt}%
\definecolor{currentstroke}{rgb}{0.000000,0.000000,0.000000}%
\pgfsetstrokecolor{currentstroke}%
\pgfsetdash{}{0pt}%
\pgfpathmoveto{\pgfqpoint{0.922232in}{0.651906in}}%
\pgfpathlineto{\pgfqpoint{0.920368in}{0.646780in}}%
\pgfpathlineto{\pgfqpoint{0.918502in}{0.641839in}}%
\pgfpathlineto{\pgfqpoint{0.916633in}{0.637088in}}%
\pgfpathlineto{\pgfqpoint{0.914761in}{0.632532in}}%
\pgfpathlineto{\pgfqpoint{0.899416in}{0.637179in}}%
\pgfpathlineto{\pgfqpoint{0.884380in}{0.642082in}}%
\pgfpathlineto{\pgfqpoint{0.869668in}{0.647235in}}%
\pgfpathlineto{\pgfqpoint{0.855297in}{0.652632in}}%
\pgfpathlineto{\pgfqpoint{0.857583in}{0.657034in}}%
\pgfpathlineto{\pgfqpoint{0.859865in}{0.661631in}}%
\pgfpathlineto{\pgfqpoint{0.862144in}{0.666418in}}%
\pgfpathlineto{\pgfqpoint{0.864419in}{0.671391in}}%
\pgfpathlineto{\pgfqpoint{0.878392in}{0.666159in}}%
\pgfpathlineto{\pgfqpoint{0.892696in}{0.661164in}}%
\pgfpathlineto{\pgfqpoint{0.907314in}{0.656411in}}%
\pgfpathlineto{\pgfqpoint{0.922232in}{0.651906in}}%
\pgfpathclose%
\pgfusepath{fill}%
\end{pgfscope}%
\begin{pgfscope}%
\pgfpathrectangle{\pgfqpoint{0.041670in}{0.041670in}}{\pgfqpoint{2.216660in}{2.216660in}}%
\pgfusepath{clip}%
\pgfsetbuttcap%
\pgfsetroundjoin%
\definecolor{currentfill}{rgb}{0.974417,0.903590,0.130215}%
\pgfsetfillcolor{currentfill}%
\pgfsetlinewidth{0.000000pt}%
\definecolor{currentstroke}{rgb}{0.000000,0.000000,0.000000}%
\pgfsetstrokecolor{currentstroke}%
\pgfsetdash{}{0pt}%
\pgfpathmoveto{\pgfqpoint{1.190629in}{1.663173in}}%
\pgfpathlineto{\pgfqpoint{1.191959in}{1.663747in}}%
\pgfpathlineto{\pgfqpoint{1.193289in}{1.664188in}}%
\pgfpathlineto{\pgfqpoint{1.194617in}{1.664498in}}%
\pgfpathlineto{\pgfqpoint{1.195945in}{1.664676in}}%
\pgfpathlineto{\pgfqpoint{1.197325in}{1.664431in}}%
\pgfpathlineto{\pgfqpoint{1.198687in}{1.664166in}}%
\pgfpathlineto{\pgfqpoint{1.200031in}{1.663881in}}%
\pgfpathlineto{\pgfqpoint{1.201355in}{1.663577in}}%
\pgfpathlineto{\pgfqpoint{1.199578in}{1.663490in}}%
\pgfpathlineto{\pgfqpoint{1.197800in}{1.663271in}}%
\pgfpathlineto{\pgfqpoint{1.196020in}{1.662921in}}%
\pgfpathlineto{\pgfqpoint{1.194240in}{1.662439in}}%
\pgfpathlineto{\pgfqpoint{1.193356in}{1.662642in}}%
\pgfpathlineto{\pgfqpoint{1.192459in}{1.662832in}}%
\pgfpathlineto{\pgfqpoint{1.191550in}{1.663009in}}%
\pgfpathlineto{\pgfqpoint{1.190629in}{1.663173in}}%
\pgfpathclose%
\pgfusepath{fill}%
\end{pgfscope}%
\begin{pgfscope}%
\pgfpathrectangle{\pgfqpoint{0.041670in}{0.041670in}}{\pgfqpoint{2.216660in}{2.216660in}}%
\pgfusepath{clip}%
\pgfsetbuttcap%
\pgfsetroundjoin%
\definecolor{currentfill}{rgb}{0.412913,0.803041,0.357269}%
\pgfsetfillcolor{currentfill}%
\pgfsetlinewidth{0.000000pt}%
\definecolor{currentstroke}{rgb}{0.000000,0.000000,0.000000}%
\pgfsetstrokecolor{currentstroke}%
\pgfsetdash{}{0pt}%
\pgfpathmoveto{\pgfqpoint{0.997011in}{1.442477in}}%
\pgfpathlineto{\pgfqpoint{0.993525in}{1.435519in}}%
\pgfpathlineto{\pgfqpoint{0.990040in}{1.428478in}}%
\pgfpathlineto{\pgfqpoint{0.986558in}{1.421355in}}%
\pgfpathlineto{\pgfqpoint{0.983077in}{1.414152in}}%
\pgfpathlineto{\pgfqpoint{0.980278in}{1.417175in}}%
\pgfpathlineto{\pgfqpoint{0.977681in}{1.420239in}}%
\pgfpathlineto{\pgfqpoint{0.975289in}{1.423339in}}%
\pgfpathlineto{\pgfqpoint{0.973104in}{1.426473in}}%
\pgfpathlineto{\pgfqpoint{0.976754in}{1.433445in}}%
\pgfpathlineto{\pgfqpoint{0.980407in}{1.440338in}}%
\pgfpathlineto{\pgfqpoint{0.984062in}{1.447150in}}%
\pgfpathlineto{\pgfqpoint{0.987719in}{1.453879in}}%
\pgfpathlineto{\pgfqpoint{0.989755in}{1.450979in}}%
\pgfpathlineto{\pgfqpoint{0.991984in}{1.448110in}}%
\pgfpathlineto{\pgfqpoint{0.994404in}{1.445275in}}%
\pgfpathlineto{\pgfqpoint{0.997011in}{1.442477in}}%
\pgfpathclose%
\pgfusepath{fill}%
\end{pgfscope}%
\begin{pgfscope}%
\pgfpathrectangle{\pgfqpoint{0.041670in}{0.041670in}}{\pgfqpoint{2.216660in}{2.216660in}}%
\pgfusepath{clip}%
\pgfsetbuttcap%
\pgfsetroundjoin%
\definecolor{currentfill}{rgb}{0.263663,0.237631,0.518762}%
\pgfsetfillcolor{currentfill}%
\pgfsetlinewidth{0.000000pt}%
\definecolor{currentstroke}{rgb}{0.000000,0.000000,0.000000}%
\pgfsetstrokecolor{currentstroke}%
\pgfsetdash{}{0pt}%
\pgfpathmoveto{\pgfqpoint{0.966384in}{0.819873in}}%
\pgfpathlineto{\pgfqpoint{0.964558in}{0.811514in}}%
\pgfpathlineto{\pgfqpoint{0.962732in}{0.803246in}}%
\pgfpathlineto{\pgfqpoint{0.960905in}{0.795072in}}%
\pgfpathlineto{\pgfqpoint{0.959077in}{0.786996in}}%
\pgfpathlineto{\pgfqpoint{0.946272in}{0.790787in}}%
\pgfpathlineto{\pgfqpoint{0.933720in}{0.794787in}}%
\pgfpathlineto{\pgfqpoint{0.921435in}{0.798993in}}%
\pgfpathlineto{\pgfqpoint{0.909429in}{0.803398in}}%
\pgfpathlineto{\pgfqpoint{0.911662in}{0.811320in}}%
\pgfpathlineto{\pgfqpoint{0.913895in}{0.819341in}}%
\pgfpathlineto{\pgfqpoint{0.916127in}{0.827456in}}%
\pgfpathlineto{\pgfqpoint{0.918358in}{0.835663in}}%
\pgfpathlineto{\pgfqpoint{0.929973in}{0.831422in}}%
\pgfpathlineto{\pgfqpoint{0.941857in}{0.827374in}}%
\pgfpathlineto{\pgfqpoint{0.953999in}{0.823523in}}%
\pgfpathlineto{\pgfqpoint{0.966384in}{0.819873in}}%
\pgfpathclose%
\pgfusepath{fill}%
\end{pgfscope}%
\begin{pgfscope}%
\pgfpathrectangle{\pgfqpoint{0.041670in}{0.041670in}}{\pgfqpoint{2.216660in}{2.216660in}}%
\pgfusepath{clip}%
\pgfsetbuttcap%
\pgfsetroundjoin%
\definecolor{currentfill}{rgb}{0.974417,0.903590,0.130215}%
\pgfsetfillcolor{currentfill}%
\pgfsetlinewidth{0.000000pt}%
\definecolor{currentstroke}{rgb}{0.000000,0.000000,0.000000}%
\pgfsetstrokecolor{currentstroke}%
\pgfsetdash{}{0pt}%
\pgfpathmoveto{\pgfqpoint{1.164897in}{1.662247in}}%
\pgfpathlineto{\pgfqpoint{1.163019in}{1.662706in}}%
\pgfpathlineto{\pgfqpoint{1.161143in}{1.663032in}}%
\pgfpathlineto{\pgfqpoint{1.159269in}{1.663227in}}%
\pgfpathlineto{\pgfqpoint{1.157395in}{1.663290in}}%
\pgfpathlineto{\pgfqpoint{1.158701in}{1.663612in}}%
\pgfpathlineto{\pgfqpoint{1.160027in}{1.663914in}}%
\pgfpathlineto{\pgfqpoint{1.161373in}{1.664197in}}%
\pgfpathlineto{\pgfqpoint{1.162738in}{1.664460in}}%
\pgfpathlineto{\pgfqpoint{1.164167in}{1.664299in}}%
\pgfpathlineto{\pgfqpoint{1.165598in}{1.664008in}}%
\pgfpathlineto{\pgfqpoint{1.167029in}{1.663584in}}%
\pgfpathlineto{\pgfqpoint{1.168462in}{1.663028in}}%
\pgfpathlineto{\pgfqpoint{1.167551in}{1.662852in}}%
\pgfpathlineto{\pgfqpoint{1.166653in}{1.662664in}}%
\pgfpathlineto{\pgfqpoint{1.165768in}{1.662462in}}%
\pgfpathlineto{\pgfqpoint{1.164897in}{1.662247in}}%
\pgfpathclose%
\pgfusepath{fill}%
\end{pgfscope}%
\begin{pgfscope}%
\pgfpathrectangle{\pgfqpoint{0.041670in}{0.041670in}}{\pgfqpoint{2.216660in}{2.216660in}}%
\pgfusepath{clip}%
\pgfsetbuttcap%
\pgfsetroundjoin%
\definecolor{currentfill}{rgb}{0.935904,0.898570,0.108131}%
\pgfsetfillcolor{currentfill}%
\pgfsetlinewidth{0.000000pt}%
\definecolor{currentstroke}{rgb}{0.000000,0.000000,0.000000}%
\pgfsetstrokecolor{currentstroke}%
\pgfsetdash{}{0pt}%
\pgfpathmoveto{\pgfqpoint{1.120496in}{1.645652in}}%
\pgfpathlineto{\pgfqpoint{1.116792in}{1.644031in}}%
\pgfpathlineto{\pgfqpoint{1.113089in}{1.642283in}}%
\pgfpathlineto{\pgfqpoint{1.109389in}{1.640407in}}%
\pgfpathlineto{\pgfqpoint{1.105690in}{1.638405in}}%
\pgfpathlineto{\pgfqpoint{1.106489in}{1.639494in}}%
\pgfpathlineto{\pgfqpoint{1.107361in}{1.640570in}}%
\pgfpathlineto{\pgfqpoint{1.108305in}{1.641633in}}%
\pgfpathlineto{\pgfqpoint{1.109320in}{1.642681in}}%
\pgfpathlineto{\pgfqpoint{1.112836in}{1.644469in}}%
\pgfpathlineto{\pgfqpoint{1.116354in}{1.646131in}}%
\pgfpathlineto{\pgfqpoint{1.119874in}{1.647665in}}%
\pgfpathlineto{\pgfqpoint{1.123395in}{1.649072in}}%
\pgfpathlineto{\pgfqpoint{1.122584in}{1.648233in}}%
\pgfpathlineto{\pgfqpoint{1.121830in}{1.647383in}}%
\pgfpathlineto{\pgfqpoint{1.121134in}{1.646522in}}%
\pgfpathlineto{\pgfqpoint{1.120496in}{1.645652in}}%
\pgfpathclose%
\pgfusepath{fill}%
\end{pgfscope}%
\begin{pgfscope}%
\pgfpathrectangle{\pgfqpoint{0.041670in}{0.041670in}}{\pgfqpoint{2.216660in}{2.216660in}}%
\pgfusepath{clip}%
\pgfsetbuttcap%
\pgfsetroundjoin%
\definecolor{currentfill}{rgb}{0.134692,0.658636,0.517649}%
\pgfsetfillcolor{currentfill}%
\pgfsetlinewidth{0.000000pt}%
\definecolor{currentstroke}{rgb}{0.000000,0.000000,0.000000}%
\pgfsetstrokecolor{currentstroke}%
\pgfsetdash{}{0pt}%
\pgfpathmoveto{\pgfqpoint{0.965076in}{1.260475in}}%
\pgfpathlineto{\pgfqpoint{0.962130in}{1.251800in}}%
\pgfpathlineto{\pgfqpoint{0.959185in}{1.243084in}}%
\pgfpathlineto{\pgfqpoint{0.956242in}{1.234329in}}%
\pgfpathlineto{\pgfqpoint{0.953300in}{1.225537in}}%
\pgfpathlineto{\pgfqpoint{0.947499in}{1.229137in}}%
\pgfpathlineto{\pgfqpoint{0.941938in}{1.232823in}}%
\pgfpathlineto{\pgfqpoint{0.936624in}{1.236593in}}%
\pgfpathlineto{\pgfqpoint{0.931561in}{1.240442in}}%
\pgfpathlineto{\pgfqpoint{0.934778in}{1.249021in}}%
\pgfpathlineto{\pgfqpoint{0.937997in}{1.257565in}}%
\pgfpathlineto{\pgfqpoint{0.941218in}{1.266070in}}%
\pgfpathlineto{\pgfqpoint{0.944440in}{1.274535in}}%
\pgfpathlineto{\pgfqpoint{0.949248in}{1.270904in}}%
\pgfpathlineto{\pgfqpoint{0.954293in}{1.267347in}}%
\pgfpathlineto{\pgfqpoint{0.959571in}{1.263870in}}%
\pgfpathlineto{\pgfqpoint{0.965076in}{1.260475in}}%
\pgfpathclose%
\pgfusepath{fill}%
\end{pgfscope}%
\begin{pgfscope}%
\pgfpathrectangle{\pgfqpoint{0.041670in}{0.041670in}}{\pgfqpoint{2.216660in}{2.216660in}}%
\pgfusepath{clip}%
\pgfsetbuttcap%
\pgfsetroundjoin%
\definecolor{currentfill}{rgb}{0.172719,0.448791,0.557885}%
\pgfsetfillcolor{currentfill}%
\pgfsetlinewidth{0.000000pt}%
\definecolor{currentstroke}{rgb}{0.000000,0.000000,0.000000}%
\pgfsetstrokecolor{currentstroke}%
\pgfsetdash{}{0pt}%
\pgfpathmoveto{\pgfqpoint{0.489820in}{0.972513in}}%
\pgfpathlineto{\pgfqpoint{0.485702in}{0.986832in}}%
\pgfpathlineto{\pgfqpoint{0.481559in}{1.001670in}}%
\pgfpathlineto{\pgfqpoint{0.477391in}{1.017037in}}%
\pgfpathlineto{\pgfqpoint{0.468227in}{1.028455in}}%
\pgfpathlineto{\pgfqpoint{0.459818in}{1.039997in}}%
\pgfpathlineto{\pgfqpoint{0.452167in}{1.051648in}}%
\pgfpathlineto{\pgfqpoint{0.445281in}{1.063398in}}%
\pgfpathlineto{\pgfqpoint{0.449610in}{1.047858in}}%
\pgfpathlineto{\pgfqpoint{0.453914in}{1.032843in}}%
\pgfpathlineto{\pgfqpoint{0.458194in}{1.018346in}}%
\pgfpathlineto{\pgfqpoint{0.464981in}{1.006729in}}%
\pgfpathlineto{\pgfqpoint{0.472517in}{0.995210in}}%
\pgfpathlineto{\pgfqpoint{0.480799in}{0.983801in}}%
\pgfpathlineto{\pgfqpoint{0.489820in}{0.972513in}}%
\pgfpathclose%
\pgfusepath{fill}%
\end{pgfscope}%
\begin{pgfscope}%
\pgfpathrectangle{\pgfqpoint{0.041670in}{0.041670in}}{\pgfqpoint{2.216660in}{2.216660in}}%
\pgfusepath{clip}%
\pgfsetbuttcap%
\pgfsetroundjoin%
\definecolor{currentfill}{rgb}{0.974417,0.903590,0.130215}%
\pgfsetfillcolor{currentfill}%
\pgfsetlinewidth{0.000000pt}%
\definecolor{currentstroke}{rgb}{0.000000,0.000000,0.000000}%
\pgfsetstrokecolor{currentstroke}%
\pgfsetdash{}{0pt}%
\pgfpathmoveto{\pgfqpoint{1.194240in}{1.662439in}}%
\pgfpathlineto{\pgfqpoint{1.196020in}{1.662921in}}%
\pgfpathlineto{\pgfqpoint{1.197800in}{1.663271in}}%
\pgfpathlineto{\pgfqpoint{1.199578in}{1.663490in}}%
\pgfpathlineto{\pgfqpoint{1.201355in}{1.663577in}}%
\pgfpathlineto{\pgfqpoint{1.202658in}{1.663253in}}%
\pgfpathlineto{\pgfqpoint{1.203939in}{1.662910in}}%
\pgfpathlineto{\pgfqpoint{1.205195in}{1.662549in}}%
\pgfpathlineto{\pgfqpoint{1.206427in}{1.662169in}}%
\pgfpathlineto{\pgfqpoint{1.204229in}{1.662199in}}%
\pgfpathlineto{\pgfqpoint{1.202029in}{1.662097in}}%
\pgfpathlineto{\pgfqpoint{1.199827in}{1.661864in}}%
\pgfpathlineto{\pgfqpoint{1.197624in}{1.661499in}}%
\pgfpathlineto{\pgfqpoint{1.196802in}{1.661752in}}%
\pgfpathlineto{\pgfqpoint{1.195963in}{1.661993in}}%
\pgfpathlineto{\pgfqpoint{1.195109in}{1.662222in}}%
\pgfpathlineto{\pgfqpoint{1.194240in}{1.662439in}}%
\pgfpathclose%
\pgfusepath{fill}%
\end{pgfscope}%
\begin{pgfscope}%
\pgfpathrectangle{\pgfqpoint{0.041670in}{0.041670in}}{\pgfqpoint{2.216660in}{2.216660in}}%
\pgfusepath{clip}%
\pgfsetbuttcap%
\pgfsetroundjoin%
\definecolor{currentfill}{rgb}{0.814576,0.883393,0.110347}%
\pgfsetfillcolor{currentfill}%
\pgfsetlinewidth{0.000000pt}%
\definecolor{currentstroke}{rgb}{0.000000,0.000000,0.000000}%
\pgfsetstrokecolor{currentstroke}%
\pgfsetdash{}{0pt}%
\pgfpathmoveto{\pgfqpoint{1.070937in}{1.598799in}}%
\pgfpathlineto{\pgfqpoint{1.067061in}{1.595007in}}%
\pgfpathlineto{\pgfqpoint{1.063187in}{1.591098in}}%
\pgfpathlineto{\pgfqpoint{1.059314in}{1.587072in}}%
\pgfpathlineto{\pgfqpoint{1.055444in}{1.582930in}}%
\pgfpathlineto{\pgfqpoint{1.055269in}{1.584782in}}%
\pgfpathlineto{\pgfqpoint{1.055218in}{1.586636in}}%
\pgfpathlineto{\pgfqpoint{1.055291in}{1.588488in}}%
\pgfpathlineto{\pgfqpoint{1.055489in}{1.590339in}}%
\pgfpathlineto{\pgfqpoint{1.059354in}{1.594246in}}%
\pgfpathlineto{\pgfqpoint{1.063221in}{1.598038in}}%
\pgfpathlineto{\pgfqpoint{1.067090in}{1.601713in}}%
\pgfpathlineto{\pgfqpoint{1.070961in}{1.605271in}}%
\pgfpathlineto{\pgfqpoint{1.070792in}{1.603655in}}%
\pgfpathlineto{\pgfqpoint{1.070732in}{1.602036in}}%
\pgfpathlineto{\pgfqpoint{1.070780in}{1.600417in}}%
\pgfpathlineto{\pgfqpoint{1.070937in}{1.598799in}}%
\pgfpathclose%
\pgfusepath{fill}%
\end{pgfscope}%
\begin{pgfscope}%
\pgfpathrectangle{\pgfqpoint{0.041670in}{0.041670in}}{\pgfqpoint{2.216660in}{2.216660in}}%
\pgfusepath{clip}%
\pgfsetbuttcap%
\pgfsetroundjoin%
\definecolor{currentfill}{rgb}{0.179019,0.433756,0.557430}%
\pgfsetfillcolor{currentfill}%
\pgfsetlinewidth{0.000000pt}%
\definecolor{currentstroke}{rgb}{0.000000,0.000000,0.000000}%
\pgfsetstrokecolor{currentstroke}%
\pgfsetdash{}{0pt}%
\pgfpathmoveto{\pgfqpoint{0.962925in}{1.013566in}}%
\pgfpathlineto{\pgfqpoint{0.960695in}{1.004257in}}%
\pgfpathlineto{\pgfqpoint{0.958466in}{0.994973in}}%
\pgfpathlineto{\pgfqpoint{0.956238in}{0.985716in}}%
\pgfpathlineto{\pgfqpoint{0.954010in}{0.976490in}}%
\pgfpathlineto{\pgfqpoint{0.944201in}{0.980241in}}%
\pgfpathlineto{\pgfqpoint{0.934643in}{0.984147in}}%
\pgfpathlineto{\pgfqpoint{0.925346in}{0.988204in}}%
\pgfpathlineto{\pgfqpoint{0.916318in}{0.992407in}}%
\pgfpathlineto{\pgfqpoint{0.918913in}{1.001456in}}%
\pgfpathlineto{\pgfqpoint{0.921507in}{1.010537in}}%
\pgfpathlineto{\pgfqpoint{0.924103in}{1.019644in}}%
\pgfpathlineto{\pgfqpoint{0.926699in}{1.028777in}}%
\pgfpathlineto{\pgfqpoint{0.935376in}{1.024760in}}%
\pgfpathlineto{\pgfqpoint{0.944313in}{1.020883in}}%
\pgfpathlineto{\pgfqpoint{0.953499in}{1.017150in}}%
\pgfpathlineto{\pgfqpoint{0.962925in}{1.013566in}}%
\pgfpathclose%
\pgfusepath{fill}%
\end{pgfscope}%
\begin{pgfscope}%
\pgfpathrectangle{\pgfqpoint{0.041670in}{0.041670in}}{\pgfqpoint{2.216660in}{2.216660in}}%
\pgfusepath{clip}%
\pgfsetbuttcap%
\pgfsetroundjoin%
\definecolor{currentfill}{rgb}{0.974417,0.903590,0.130215}%
\pgfsetfillcolor{currentfill}%
\pgfsetlinewidth{0.000000pt}%
\definecolor{currentstroke}{rgb}{0.000000,0.000000,0.000000}%
\pgfsetstrokecolor{currentstroke}%
\pgfsetdash{}{0pt}%
\pgfpathmoveto{\pgfqpoint{1.161570in}{1.661263in}}%
\pgfpathlineto{\pgfqpoint{1.159278in}{1.661599in}}%
\pgfpathlineto{\pgfqpoint{1.156987in}{1.661803in}}%
\pgfpathlineto{\pgfqpoint{1.154697in}{1.661876in}}%
\pgfpathlineto{\pgfqpoint{1.152409in}{1.661816in}}%
\pgfpathlineto{\pgfqpoint{1.153618in}{1.662212in}}%
\pgfpathlineto{\pgfqpoint{1.154853in}{1.662590in}}%
\pgfpathlineto{\pgfqpoint{1.156112in}{1.662949in}}%
\pgfpathlineto{\pgfqpoint{1.157395in}{1.663290in}}%
\pgfpathlineto{\pgfqpoint{1.159269in}{1.663227in}}%
\pgfpathlineto{\pgfqpoint{1.161143in}{1.663032in}}%
\pgfpathlineto{\pgfqpoint{1.163019in}{1.662706in}}%
\pgfpathlineto{\pgfqpoint{1.164897in}{1.662247in}}%
\pgfpathlineto{\pgfqpoint{1.164040in}{1.662020in}}%
\pgfpathlineto{\pgfqpoint{1.163200in}{1.661780in}}%
\pgfpathlineto{\pgfqpoint{1.162376in}{1.661527in}}%
\pgfpathlineto{\pgfqpoint{1.161570in}{1.661263in}}%
\pgfpathclose%
\pgfusepath{fill}%
\end{pgfscope}%
\begin{pgfscope}%
\pgfpathrectangle{\pgfqpoint{0.041670in}{0.041670in}}{\pgfqpoint{2.216660in}{2.216660in}}%
\pgfusepath{clip}%
\pgfsetbuttcap%
\pgfsetroundjoin%
\definecolor{currentfill}{rgb}{0.122606,0.585371,0.546557}%
\pgfsetfillcolor{currentfill}%
\pgfsetlinewidth{0.000000pt}%
\definecolor{currentstroke}{rgb}{0.000000,0.000000,0.000000}%
\pgfsetstrokecolor{currentstroke}%
\pgfsetdash{}{0pt}%
\pgfpathmoveto{\pgfqpoint{1.423793in}{1.193437in}}%
\pgfpathlineto{\pgfqpoint{1.426793in}{1.184549in}}%
\pgfpathlineto{\pgfqpoint{1.429792in}{1.175639in}}%
\pgfpathlineto{\pgfqpoint{1.432789in}{1.166712in}}%
\pgfpathlineto{\pgfqpoint{1.435785in}{1.157769in}}%
\pgfpathlineto{\pgfqpoint{1.429367in}{1.153767in}}%
\pgfpathlineto{\pgfqpoint{1.422690in}{1.149866in}}%
\pgfpathlineto{\pgfqpoint{1.415759in}{1.146071in}}%
\pgfpathlineto{\pgfqpoint{1.408581in}{1.142386in}}%
\pgfpathlineto{\pgfqpoint{1.405898in}{1.151529in}}%
\pgfpathlineto{\pgfqpoint{1.403214in}{1.160656in}}%
\pgfpathlineto{\pgfqpoint{1.400529in}{1.169764in}}%
\pgfpathlineto{\pgfqpoint{1.397842in}{1.178852in}}%
\pgfpathlineto{\pgfqpoint{1.404688in}{1.182346in}}%
\pgfpathlineto{\pgfqpoint{1.411300in}{1.185944in}}%
\pgfpathlineto{\pgfqpoint{1.417670in}{1.189642in}}%
\pgfpathlineto{\pgfqpoint{1.423793in}{1.193437in}}%
\pgfpathclose%
\pgfusepath{fill}%
\end{pgfscope}%
\begin{pgfscope}%
\pgfpathrectangle{\pgfqpoint{0.041670in}{0.041670in}}{\pgfqpoint{2.216660in}{2.216660in}}%
\pgfusepath{clip}%
\pgfsetbuttcap%
\pgfsetroundjoin%
\definecolor{currentfill}{rgb}{0.636902,0.856542,0.216620}%
\pgfsetfillcolor{currentfill}%
\pgfsetlinewidth{0.000000pt}%
\definecolor{currentstroke}{rgb}{0.000000,0.000000,0.000000}%
\pgfsetstrokecolor{currentstroke}%
\pgfsetdash{}{0pt}%
\pgfpathmoveto{\pgfqpoint{1.333714in}{1.538575in}}%
\pgfpathlineto{\pgfqpoint{1.337528in}{1.533264in}}%
\pgfpathlineto{\pgfqpoint{1.341340in}{1.527850in}}%
\pgfpathlineto{\pgfqpoint{1.345149in}{1.522335in}}%
\pgfpathlineto{\pgfqpoint{1.348956in}{1.516720in}}%
\pgfpathlineto{\pgfqpoint{1.348034in}{1.514188in}}%
\pgfpathlineto{\pgfqpoint{1.346944in}{1.511671in}}%
\pgfpathlineto{\pgfqpoint{1.345686in}{1.509170in}}%
\pgfpathlineto{\pgfqpoint{1.344261in}{1.506688in}}%
\pgfpathlineto{\pgfqpoint{1.340554in}{1.512537in}}%
\pgfpathlineto{\pgfqpoint{1.336845in}{1.518287in}}%
\pgfpathlineto{\pgfqpoint{1.333134in}{1.523935in}}%
\pgfpathlineto{\pgfqpoint{1.329421in}{1.529479in}}%
\pgfpathlineto{\pgfqpoint{1.330723in}{1.531729in}}%
\pgfpathlineto{\pgfqpoint{1.331873in}{1.533997in}}%
\pgfpathlineto{\pgfqpoint{1.332870in}{1.536279in}}%
\pgfpathlineto{\pgfqpoint{1.333714in}{1.538575in}}%
\pgfpathclose%
\pgfusepath{fill}%
\end{pgfscope}%
\begin{pgfscope}%
\pgfpathrectangle{\pgfqpoint{0.041670in}{0.041670in}}{\pgfqpoint{2.216660in}{2.216660in}}%
\pgfusepath{clip}%
\pgfsetbuttcap%
\pgfsetroundjoin%
\definecolor{currentfill}{rgb}{0.282327,0.094955,0.417331}%
\pgfsetfillcolor{currentfill}%
\pgfsetlinewidth{0.000000pt}%
\definecolor{currentstroke}{rgb}{0.000000,0.000000,0.000000}%
\pgfsetstrokecolor{currentstroke}%
\pgfsetdash{}{0pt}%
\pgfpathmoveto{\pgfqpoint{1.702926in}{0.696984in}}%
\pgfpathlineto{\pgfqpoint{1.706190in}{0.700749in}}%
\pgfpathlineto{\pgfqpoint{1.709466in}{0.704862in}}%
\pgfpathlineto{\pgfqpoint{1.712754in}{0.709330in}}%
\pgfpathlineto{\pgfqpoint{1.716054in}{0.714158in}}%
\pgfpathlineto{\pgfqpoint{1.702936in}{0.705229in}}%
\pgfpathlineto{\pgfqpoint{1.689248in}{0.696516in}}%
\pgfpathlineto{\pgfqpoint{1.675003in}{0.688028in}}%
\pgfpathlineto{\pgfqpoint{1.660215in}{0.679777in}}%
\pgfpathlineto{\pgfqpoint{1.657244in}{0.675139in}}%
\pgfpathlineto{\pgfqpoint{1.654286in}{0.670863in}}%
\pgfpathlineto{\pgfqpoint{1.651338in}{0.666942in}}%
\pgfpathlineto{\pgfqpoint{1.648401in}{0.663371in}}%
\pgfpathlineto{\pgfqpoint{1.662838in}{0.671437in}}%
\pgfpathlineto{\pgfqpoint{1.676747in}{0.679735in}}%
\pgfpathlineto{\pgfqpoint{1.690114in}{0.688254in}}%
\pgfpathlineto{\pgfqpoint{1.702926in}{0.696984in}}%
\pgfpathclose%
\pgfusepath{fill}%
\end{pgfscope}%
\begin{pgfscope}%
\pgfpathrectangle{\pgfqpoint{0.041670in}{0.041670in}}{\pgfqpoint{2.216660in}{2.216660in}}%
\pgfusepath{clip}%
\pgfsetbuttcap%
\pgfsetroundjoin%
\definecolor{currentfill}{rgb}{0.231674,0.318106,0.544834}%
\pgfsetfillcolor{currentfill}%
\pgfsetlinewidth{0.000000pt}%
\definecolor{currentstroke}{rgb}{0.000000,0.000000,0.000000}%
\pgfsetstrokecolor{currentstroke}%
\pgfsetdash{}{0pt}%
\pgfpathmoveto{\pgfqpoint{1.433131in}{0.907786in}}%
\pgfpathlineto{\pgfqpoint{1.435444in}{0.899017in}}%
\pgfpathlineto{\pgfqpoint{1.437756in}{0.890311in}}%
\pgfpathlineto{\pgfqpoint{1.440069in}{0.881671in}}%
\pgfpathlineto{\pgfqpoint{1.442382in}{0.873101in}}%
\pgfpathlineto{\pgfqpoint{1.431399in}{0.868863in}}%
\pgfpathlineto{\pgfqpoint{1.420146in}{0.864806in}}%
\pgfpathlineto{\pgfqpoint{1.408635in}{0.860935in}}%
\pgfpathlineto{\pgfqpoint{1.396877in}{0.857255in}}%
\pgfpathlineto{\pgfqpoint{1.394961in}{0.865983in}}%
\pgfpathlineto{\pgfqpoint{1.393045in}{0.874781in}}%
\pgfpathlineto{\pgfqpoint{1.391129in}{0.883646in}}%
\pgfpathlineto{\pgfqpoint{1.389214in}{0.892573in}}%
\pgfpathlineto{\pgfqpoint{1.400560in}{0.896107in}}%
\pgfpathlineto{\pgfqpoint{1.411669in}{0.899823in}}%
\pgfpathlineto{\pgfqpoint{1.422530in}{0.903718in}}%
\pgfpathlineto{\pgfqpoint{1.433131in}{0.907786in}}%
\pgfpathclose%
\pgfusepath{fill}%
\end{pgfscope}%
\begin{pgfscope}%
\pgfpathrectangle{\pgfqpoint{0.041670in}{0.041670in}}{\pgfqpoint{2.216660in}{2.216660in}}%
\pgfusepath{clip}%
\pgfsetbuttcap%
\pgfsetroundjoin%
\definecolor{currentfill}{rgb}{0.955300,0.901065,0.118128}%
\pgfsetfillcolor{currentfill}%
\pgfsetlinewidth{0.000000pt}%
\definecolor{currentstroke}{rgb}{0.000000,0.000000,0.000000}%
\pgfsetstrokecolor{currentstroke}%
\pgfsetdash{}{0pt}%
\pgfpathmoveto{\pgfqpoint{1.218847in}{1.656337in}}%
\pgfpathlineto{\pgfqpoint{1.222078in}{1.655692in}}%
\pgfpathlineto{\pgfqpoint{1.225307in}{1.654916in}}%
\pgfpathlineto{\pgfqpoint{1.228535in}{1.654011in}}%
\pgfpathlineto{\pgfqpoint{1.231760in}{1.652977in}}%
\pgfpathlineto{\pgfqpoint{1.232836in}{1.652207in}}%
\pgfpathlineto{\pgfqpoint{1.233860in}{1.651421in}}%
\pgfpathlineto{\pgfqpoint{1.234831in}{1.650621in}}%
\pgfpathlineto{\pgfqpoint{1.235747in}{1.649807in}}%
\pgfpathlineto{\pgfqpoint{1.232271in}{1.651039in}}%
\pgfpathlineto{\pgfqpoint{1.228794in}{1.652142in}}%
\pgfpathlineto{\pgfqpoint{1.225315in}{1.653116in}}%
\pgfpathlineto{\pgfqpoint{1.221834in}{1.653959in}}%
\pgfpathlineto{\pgfqpoint{1.221148in}{1.654570in}}%
\pgfpathlineto{\pgfqpoint{1.220421in}{1.655170in}}%
\pgfpathlineto{\pgfqpoint{1.219654in}{1.655759in}}%
\pgfpathlineto{\pgfqpoint{1.218847in}{1.656337in}}%
\pgfpathclose%
\pgfusepath{fill}%
\end{pgfscope}%
\begin{pgfscope}%
\pgfpathrectangle{\pgfqpoint{0.041670in}{0.041670in}}{\pgfqpoint{2.216660in}{2.216660in}}%
\pgfusepath{clip}%
\pgfsetbuttcap%
\pgfsetroundjoin%
\definecolor{currentfill}{rgb}{0.163625,0.471133,0.558148}%
\pgfsetfillcolor{currentfill}%
\pgfsetlinewidth{0.000000pt}%
\definecolor{currentstroke}{rgb}{0.000000,0.000000,0.000000}%
\pgfsetstrokecolor{currentstroke}%
\pgfsetdash{}{0pt}%
\pgfpathmoveto{\pgfqpoint{1.430008in}{1.069009in}}%
\pgfpathlineto{\pgfqpoint{1.432683in}{1.059850in}}%
\pgfpathlineto{\pgfqpoint{1.435356in}{1.050704in}}%
\pgfpathlineto{\pgfqpoint{1.438029in}{1.041573in}}%
\pgfpathlineto{\pgfqpoint{1.440701in}{1.032462in}}%
\pgfpathlineto{\pgfqpoint{1.432260in}{1.028324in}}%
\pgfpathlineto{\pgfqpoint{1.423553in}{1.024322in}}%
\pgfpathlineto{\pgfqpoint{1.414589in}{1.020461in}}%
\pgfpathlineto{\pgfqpoint{1.405375in}{1.016745in}}%
\pgfpathlineto{\pgfqpoint{1.403060in}{1.026038in}}%
\pgfpathlineto{\pgfqpoint{1.400745in}{1.035349in}}%
\pgfpathlineto{\pgfqpoint{1.398429in}{1.044676in}}%
\pgfpathlineto{\pgfqpoint{1.396112in}{1.054015in}}%
\pgfpathlineto{\pgfqpoint{1.404951in}{1.057561in}}%
\pgfpathlineto{\pgfqpoint{1.413553in}{1.061244in}}%
\pgfpathlineto{\pgfqpoint{1.421908in}{1.065061in}}%
\pgfpathlineto{\pgfqpoint{1.430008in}{1.069009in}}%
\pgfpathclose%
\pgfusepath{fill}%
\end{pgfscope}%
\begin{pgfscope}%
\pgfpathrectangle{\pgfqpoint{0.041670in}{0.041670in}}{\pgfqpoint{2.216660in}{2.216660in}}%
\pgfusepath{clip}%
\pgfsetbuttcap%
\pgfsetroundjoin%
\definecolor{currentfill}{rgb}{0.281477,0.755203,0.432552}%
\pgfsetfillcolor{currentfill}%
\pgfsetlinewidth{0.000000pt}%
\definecolor{currentstroke}{rgb}{0.000000,0.000000,0.000000}%
\pgfsetstrokecolor{currentstroke}%
\pgfsetdash{}{0pt}%
\pgfpathmoveto{\pgfqpoint{0.983254in}{1.372068in}}%
\pgfpathlineto{\pgfqpoint{0.980010in}{1.364275in}}%
\pgfpathlineto{\pgfqpoint{0.976767in}{1.356414in}}%
\pgfpathlineto{\pgfqpoint{0.973526in}{1.348486in}}%
\pgfpathlineto{\pgfqpoint{0.970287in}{1.340495in}}%
\pgfpathlineto{\pgfqpoint{0.966210in}{1.343760in}}%
\pgfpathlineto{\pgfqpoint{0.962353in}{1.347083in}}%
\pgfpathlineto{\pgfqpoint{0.958718in}{1.350463in}}%
\pgfpathlineto{\pgfqpoint{0.955309in}{1.353895in}}%
\pgfpathlineto{\pgfqpoint{0.958772in}{1.361664in}}%
\pgfpathlineto{\pgfqpoint{0.962238in}{1.369369in}}%
\pgfpathlineto{\pgfqpoint{0.965706in}{1.377009in}}%
\pgfpathlineto{\pgfqpoint{0.969176in}{1.384581in}}%
\pgfpathlineto{\pgfqpoint{0.972381in}{1.381376in}}%
\pgfpathlineto{\pgfqpoint{0.975798in}{1.378219in}}%
\pgfpathlineto{\pgfqpoint{0.979424in}{1.375116in}}%
\pgfpathlineto{\pgfqpoint{0.983254in}{1.372068in}}%
\pgfpathclose%
\pgfusepath{fill}%
\end{pgfscope}%
\begin{pgfscope}%
\pgfpathrectangle{\pgfqpoint{0.041670in}{0.041670in}}{\pgfqpoint{2.216660in}{2.216660in}}%
\pgfusepath{clip}%
\pgfsetbuttcap%
\pgfsetroundjoin%
\definecolor{currentfill}{rgb}{0.762373,0.876424,0.137064}%
\pgfsetfillcolor{currentfill}%
\pgfsetlinewidth{0.000000pt}%
\definecolor{currentstroke}{rgb}{0.000000,0.000000,0.000000}%
\pgfsetstrokecolor{currentstroke}%
\pgfsetdash{}{0pt}%
\pgfpathmoveto{\pgfqpoint{1.304628in}{1.584576in}}%
\pgfpathlineto{\pgfqpoint{1.308500in}{1.580372in}}%
\pgfpathlineto{\pgfqpoint{1.312370in}{1.576054in}}%
\pgfpathlineto{\pgfqpoint{1.316238in}{1.571625in}}%
\pgfpathlineto{\pgfqpoint{1.320104in}{1.567084in}}%
\pgfpathlineto{\pgfqpoint{1.319897in}{1.564997in}}%
\pgfpathlineto{\pgfqpoint{1.319549in}{1.562913in}}%
\pgfpathlineto{\pgfqpoint{1.319062in}{1.560834in}}%
\pgfpathlineto{\pgfqpoint{1.318436in}{1.558763in}}%
\pgfpathlineto{\pgfqpoint{1.314612in}{1.563539in}}%
\pgfpathlineto{\pgfqpoint{1.310785in}{1.568203in}}%
\pgfpathlineto{\pgfqpoint{1.306957in}{1.572756in}}%
\pgfpathlineto{\pgfqpoint{1.303126in}{1.577194in}}%
\pgfpathlineto{\pgfqpoint{1.303688in}{1.579031in}}%
\pgfpathlineto{\pgfqpoint{1.304125in}{1.580875in}}%
\pgfpathlineto{\pgfqpoint{1.304439in}{1.582724in}}%
\pgfpathlineto{\pgfqpoint{1.304628in}{1.584576in}}%
\pgfpathclose%
\pgfusepath{fill}%
\end{pgfscope}%
\begin{pgfscope}%
\pgfpathrectangle{\pgfqpoint{0.041670in}{0.041670in}}{\pgfqpoint{2.216660in}{2.216660in}}%
\pgfusepath{clip}%
\pgfsetbuttcap%
\pgfsetroundjoin%
\definecolor{currentfill}{rgb}{0.166383,0.690856,0.496502}%
\pgfsetfillcolor{currentfill}%
\pgfsetlinewidth{0.000000pt}%
\definecolor{currentstroke}{rgb}{0.000000,0.000000,0.000000}%
\pgfsetstrokecolor{currentstroke}%
\pgfsetdash{}{0pt}%
\pgfpathmoveto{\pgfqpoint{1.406413in}{1.311028in}}%
\pgfpathlineto{\pgfqpoint{1.409698in}{1.302800in}}%
\pgfpathlineto{\pgfqpoint{1.412980in}{1.294521in}}%
\pgfpathlineto{\pgfqpoint{1.416260in}{1.286194in}}%
\pgfpathlineto{\pgfqpoint{1.419539in}{1.277822in}}%
\pgfpathlineto{\pgfqpoint{1.414947in}{1.274128in}}%
\pgfpathlineto{\pgfqpoint{1.410113in}{1.270505in}}%
\pgfpathlineto{\pgfqpoint{1.405042in}{1.266957in}}%
\pgfpathlineto{\pgfqpoint{1.399739in}{1.263489in}}%
\pgfpathlineto{\pgfqpoint{1.396725in}{1.272075in}}%
\pgfpathlineto{\pgfqpoint{1.393709in}{1.280615in}}%
\pgfpathlineto{\pgfqpoint{1.390692in}{1.289107in}}%
\pgfpathlineto{\pgfqpoint{1.387674in}{1.297548in}}%
\pgfpathlineto{\pgfqpoint{1.392692in}{1.300810in}}%
\pgfpathlineto{\pgfqpoint{1.397491in}{1.304146in}}%
\pgfpathlineto{\pgfqpoint{1.402066in}{1.307554in}}%
\pgfpathlineto{\pgfqpoint{1.406413in}{1.311028in}}%
\pgfpathclose%
\pgfusepath{fill}%
\end{pgfscope}%
\begin{pgfscope}%
\pgfpathrectangle{\pgfqpoint{0.041670in}{0.041670in}}{\pgfqpoint{2.216660in}{2.216660in}}%
\pgfusepath{clip}%
\pgfsetbuttcap%
\pgfsetroundjoin%
\definecolor{currentfill}{rgb}{0.248629,0.278775,0.534556}%
\pgfsetfillcolor{currentfill}%
\pgfsetlinewidth{0.000000pt}%
\definecolor{currentstroke}{rgb}{0.000000,0.000000,0.000000}%
\pgfsetstrokecolor{currentstroke}%
\pgfsetdash{}{0pt}%
\pgfpathmoveto{\pgfqpoint{0.973681in}{0.854146in}}%
\pgfpathlineto{\pgfqpoint{0.971857in}{0.845459in}}%
\pgfpathlineto{\pgfqpoint{0.970033in}{0.836849in}}%
\pgfpathlineto{\pgfqpoint{0.968209in}{0.828319in}}%
\pgfpathlineto{\pgfqpoint{0.966384in}{0.819873in}}%
\pgfpathlineto{\pgfqpoint{0.953999in}{0.823523in}}%
\pgfpathlineto{\pgfqpoint{0.941857in}{0.827374in}}%
\pgfpathlineto{\pgfqpoint{0.929973in}{0.831422in}}%
\pgfpathlineto{\pgfqpoint{0.918358in}{0.835663in}}%
\pgfpathlineto{\pgfqpoint{0.920589in}{0.843956in}}%
\pgfpathlineto{\pgfqpoint{0.922819in}{0.852333in}}%
\pgfpathlineto{\pgfqpoint{0.925048in}{0.860791in}}%
\pgfpathlineto{\pgfqpoint{0.927277in}{0.869325in}}%
\pgfpathlineto{\pgfqpoint{0.938500in}{0.865248in}}%
\pgfpathlineto{\pgfqpoint{0.949984in}{0.861356in}}%
\pgfpathlineto{\pgfqpoint{0.961714in}{0.857654in}}%
\pgfpathlineto{\pgfqpoint{0.973681in}{0.854146in}}%
\pgfpathclose%
\pgfusepath{fill}%
\end{pgfscope}%
\begin{pgfscope}%
\pgfpathrectangle{\pgfqpoint{0.041670in}{0.041670in}}{\pgfqpoint{2.216660in}{2.216660in}}%
\pgfusepath{clip}%
\pgfsetbuttcap%
\pgfsetroundjoin%
\definecolor{currentfill}{rgb}{0.974417,0.903590,0.130215}%
\pgfsetfillcolor{currentfill}%
\pgfsetlinewidth{0.000000pt}%
\definecolor{currentstroke}{rgb}{0.000000,0.000000,0.000000}%
\pgfsetstrokecolor{currentstroke}%
\pgfsetdash{}{0pt}%
\pgfpathmoveto{\pgfqpoint{1.197624in}{1.661499in}}%
\pgfpathlineto{\pgfqpoint{1.199827in}{1.661864in}}%
\pgfpathlineto{\pgfqpoint{1.202029in}{1.662097in}}%
\pgfpathlineto{\pgfqpoint{1.204229in}{1.662199in}}%
\pgfpathlineto{\pgfqpoint{1.206427in}{1.662169in}}%
\pgfpathlineto{\pgfqpoint{1.207633in}{1.661771in}}%
\pgfpathlineto{\pgfqpoint{1.208811in}{1.661356in}}%
\pgfpathlineto{\pgfqpoint{1.209961in}{1.660924in}}%
\pgfpathlineto{\pgfqpoint{1.211081in}{1.660475in}}%
\pgfpathlineto{\pgfqpoint{1.208496in}{1.660645in}}%
\pgfpathlineto{\pgfqpoint{1.205908in}{1.660685in}}%
\pgfpathlineto{\pgfqpoint{1.203319in}{1.660592in}}%
\pgfpathlineto{\pgfqpoint{1.200729in}{1.660368in}}%
\pgfpathlineto{\pgfqpoint{1.199981in}{1.660667in}}%
\pgfpathlineto{\pgfqpoint{1.199214in}{1.660956in}}%
\pgfpathlineto{\pgfqpoint{1.198428in}{1.661233in}}%
\pgfpathlineto{\pgfqpoint{1.197624in}{1.661499in}}%
\pgfpathclose%
\pgfusepath{fill}%
\end{pgfscope}%
\begin{pgfscope}%
\pgfpathrectangle{\pgfqpoint{0.041670in}{0.041670in}}{\pgfqpoint{2.216660in}{2.216660in}}%
\pgfusepath{clip}%
\pgfsetbuttcap%
\pgfsetroundjoin%
\definecolor{currentfill}{rgb}{0.487026,0.823929,0.312321}%
\pgfsetfillcolor{currentfill}%
\pgfsetlinewidth{0.000000pt}%
\definecolor{currentstroke}{rgb}{0.000000,0.000000,0.000000}%
\pgfsetstrokecolor{currentstroke}%
\pgfsetdash{}{0pt}%
\pgfpathmoveto{\pgfqpoint{1.359067in}{1.482320in}}%
\pgfpathlineto{\pgfqpoint{1.362763in}{1.475993in}}%
\pgfpathlineto{\pgfqpoint{1.366457in}{1.469577in}}%
\pgfpathlineto{\pgfqpoint{1.370149in}{1.463072in}}%
\pgfpathlineto{\pgfqpoint{1.373838in}{1.456480in}}%
\pgfpathlineto{\pgfqpoint{1.371974in}{1.453555in}}%
\pgfpathlineto{\pgfqpoint{1.369916in}{1.450658in}}%
\pgfpathlineto{\pgfqpoint{1.367666in}{1.447793in}}%
\pgfpathlineto{\pgfqpoint{1.365226in}{1.444962in}}%
\pgfpathlineto{\pgfqpoint{1.361694in}{1.451784in}}%
\pgfpathlineto{\pgfqpoint{1.358161in}{1.458520in}}%
\pgfpathlineto{\pgfqpoint{1.354625in}{1.465166in}}%
\pgfpathlineto{\pgfqpoint{1.351088in}{1.471722in}}%
\pgfpathlineto{\pgfqpoint{1.353348in}{1.474326in}}%
\pgfpathlineto{\pgfqpoint{1.355432in}{1.476962in}}%
\pgfpathlineto{\pgfqpoint{1.357339in}{1.479628in}}%
\pgfpathlineto{\pgfqpoint{1.359067in}{1.482320in}}%
\pgfpathclose%
\pgfusepath{fill}%
\end{pgfscope}%
\begin{pgfscope}%
\pgfpathrectangle{\pgfqpoint{0.041670in}{0.041670in}}{\pgfqpoint{2.216660in}{2.216660in}}%
\pgfusepath{clip}%
\pgfsetbuttcap%
\pgfsetroundjoin%
\definecolor{currentfill}{rgb}{0.955300,0.901065,0.118128}%
\pgfsetfillcolor{currentfill}%
\pgfsetlinewidth{0.000000pt}%
\definecolor{currentstroke}{rgb}{0.000000,0.000000,0.000000}%
\pgfsetstrokecolor{currentstroke}%
\pgfsetdash{}{0pt}%
\pgfpathmoveto{\pgfqpoint{1.137500in}{1.653408in}}%
\pgfpathlineto{\pgfqpoint{1.133971in}{1.652518in}}%
\pgfpathlineto{\pgfqpoint{1.130444in}{1.651499in}}%
\pgfpathlineto{\pgfqpoint{1.126919in}{1.650350in}}%
\pgfpathlineto{\pgfqpoint{1.123395in}{1.649072in}}%
\pgfpathlineto{\pgfqpoint{1.124262in}{1.649898in}}%
\pgfpathlineto{\pgfqpoint{1.125184in}{1.650711in}}%
\pgfpathlineto{\pgfqpoint{1.126161in}{1.651509in}}%
\pgfpathlineto{\pgfqpoint{1.127190in}{1.652293in}}%
\pgfpathlineto{\pgfqpoint{1.130476in}{1.653370in}}%
\pgfpathlineto{\pgfqpoint{1.133763in}{1.654318in}}%
\pgfpathlineto{\pgfqpoint{1.137053in}{1.655136in}}%
\pgfpathlineto{\pgfqpoint{1.140344in}{1.655824in}}%
\pgfpathlineto{\pgfqpoint{1.139572in}{1.655236in}}%
\pgfpathlineto{\pgfqpoint{1.138840in}{1.654637in}}%
\pgfpathlineto{\pgfqpoint{1.138150in}{1.654027in}}%
\pgfpathlineto{\pgfqpoint{1.137500in}{1.653408in}}%
\pgfpathclose%
\pgfusepath{fill}%
\end{pgfscope}%
\begin{pgfscope}%
\pgfpathrectangle{\pgfqpoint{0.041670in}{0.041670in}}{\pgfqpoint{2.216660in}{2.216660in}}%
\pgfusepath{clip}%
\pgfsetbuttcap%
\pgfsetroundjoin%
\definecolor{currentfill}{rgb}{0.233603,0.313828,0.543914}%
\pgfsetfillcolor{currentfill}%
\pgfsetlinewidth{0.000000pt}%
\definecolor{currentstroke}{rgb}{0.000000,0.000000,0.000000}%
\pgfsetstrokecolor{currentstroke}%
\pgfsetdash{}{0pt}%
\pgfpathmoveto{\pgfqpoint{0.563485in}{0.833472in}}%
\pgfpathlineto{\pgfqpoint{0.559765in}{0.843740in}}%
\pgfpathlineto{\pgfqpoint{0.556027in}{0.854465in}}%
\pgfpathlineto{\pgfqpoint{0.552269in}{0.865653in}}%
\pgfpathlineto{\pgfqpoint{0.548491in}{0.877314in}}%
\pgfpathlineto{\pgfqpoint{0.536838in}{0.887808in}}%
\pgfpathlineto{\pgfqpoint{0.525878in}{0.898473in}}%
\pgfpathlineto{\pgfqpoint{0.515619in}{0.909296in}}%
\pgfpathlineto{\pgfqpoint{0.506068in}{0.920265in}}%
\pgfpathlineto{\pgfqpoint{0.510076in}{0.908418in}}%
\pgfpathlineto{\pgfqpoint{0.514063in}{0.897041in}}%
\pgfpathlineto{\pgfqpoint{0.518030in}{0.886125in}}%
\pgfpathlineto{\pgfqpoint{0.521977in}{0.875664in}}%
\pgfpathlineto{\pgfqpoint{0.531326in}{0.864888in}}%
\pgfpathlineto{\pgfqpoint{0.541365in}{0.854256in}}%
\pgfpathlineto{\pgfqpoint{0.552088in}{0.843780in}}%
\pgfpathlineto{\pgfqpoint{0.563485in}{0.833472in}}%
\pgfpathclose%
\pgfusepath{fill}%
\end{pgfscope}%
\begin{pgfscope}%
\pgfpathrectangle{\pgfqpoint{0.041670in}{0.041670in}}{\pgfqpoint{2.216660in}{2.216660in}}%
\pgfusepath{clip}%
\pgfsetbuttcap%
\pgfsetroundjoin%
\definecolor{currentfill}{rgb}{0.268510,0.009605,0.335427}%
\pgfsetfillcolor{currentfill}%
\pgfsetlinewidth{0.000000pt}%
\definecolor{currentstroke}{rgb}{0.000000,0.000000,0.000000}%
\pgfsetstrokecolor{currentstroke}%
\pgfsetdash{}{0pt}%
\pgfpathmoveto{\pgfqpoint{1.526616in}{0.642201in}}%
\pgfpathlineto{\pgfqpoint{1.529008in}{0.638875in}}%
\pgfpathlineto{\pgfqpoint{1.531405in}{0.635771in}}%
\pgfpathlineto{\pgfqpoint{1.533806in}{0.632893in}}%
\pgfpathlineto{\pgfqpoint{1.536213in}{0.630246in}}%
\pgfpathlineto{\pgfqpoint{1.521374in}{0.624299in}}%
\pgfpathlineto{\pgfqpoint{1.506160in}{0.618603in}}%
\pgfpathlineto{\pgfqpoint{1.490586in}{0.613166in}}%
\pgfpathlineto{\pgfqpoint{1.474668in}{0.607994in}}%
\pgfpathlineto{\pgfqpoint{1.472672in}{0.610799in}}%
\pgfpathlineto{\pgfqpoint{1.470679in}{0.613836in}}%
\pgfpathlineto{\pgfqpoint{1.468691in}{0.617099in}}%
\pgfpathlineto{\pgfqpoint{1.466706in}{0.620584in}}%
\pgfpathlineto{\pgfqpoint{1.482199in}{0.625608in}}%
\pgfpathlineto{\pgfqpoint{1.497359in}{0.630890in}}%
\pgfpathlineto{\pgfqpoint{1.512170in}{0.636423in}}%
\pgfpathlineto{\pgfqpoint{1.526616in}{0.642201in}}%
\pgfpathclose%
\pgfusepath{fill}%
\end{pgfscope}%
\begin{pgfscope}%
\pgfpathrectangle{\pgfqpoint{0.041670in}{0.041670in}}{\pgfqpoint{2.216660in}{2.216660in}}%
\pgfusepath{clip}%
\pgfsetbuttcap%
\pgfsetroundjoin%
\definecolor{currentfill}{rgb}{0.974417,0.903590,0.130215}%
\pgfsetfillcolor{currentfill}%
\pgfsetlinewidth{0.000000pt}%
\definecolor{currentstroke}{rgb}{0.000000,0.000000,0.000000}%
\pgfsetstrokecolor{currentstroke}%
\pgfsetdash{}{0pt}%
\pgfpathmoveto{\pgfqpoint{1.158534in}{1.660092in}}%
\pgfpathlineto{\pgfqpoint{1.155863in}{1.660282in}}%
\pgfpathlineto{\pgfqpoint{1.153193in}{1.660341in}}%
\pgfpathlineto{\pgfqpoint{1.150525in}{1.660267in}}%
\pgfpathlineto{\pgfqpoint{1.147859in}{1.660062in}}%
\pgfpathlineto{\pgfqpoint{1.148951in}{1.660525in}}%
\pgfpathlineto{\pgfqpoint{1.150075in}{1.660973in}}%
\pgfpathlineto{\pgfqpoint{1.151228in}{1.661403in}}%
\pgfpathlineto{\pgfqpoint{1.152409in}{1.661816in}}%
\pgfpathlineto{\pgfqpoint{1.154697in}{1.661876in}}%
\pgfpathlineto{\pgfqpoint{1.156987in}{1.661803in}}%
\pgfpathlineto{\pgfqpoint{1.159278in}{1.661599in}}%
\pgfpathlineto{\pgfqpoint{1.161570in}{1.661263in}}%
\pgfpathlineto{\pgfqpoint{1.160782in}{1.660987in}}%
\pgfpathlineto{\pgfqpoint{1.160013in}{1.660700in}}%
\pgfpathlineto{\pgfqpoint{1.159263in}{1.660402in}}%
\pgfpathlineto{\pgfqpoint{1.158534in}{1.660092in}}%
\pgfpathclose%
\pgfusepath{fill}%
\end{pgfscope}%
\begin{pgfscope}%
\pgfpathrectangle{\pgfqpoint{0.041670in}{0.041670in}}{\pgfqpoint{2.216660in}{2.216660in}}%
\pgfusepath{clip}%
\pgfsetbuttcap%
\pgfsetroundjoin%
\definecolor{currentfill}{rgb}{0.271305,0.019942,0.347269}%
\pgfsetfillcolor{currentfill}%
\pgfsetlinewidth{0.000000pt}%
\definecolor{currentstroke}{rgb}{0.000000,0.000000,0.000000}%
\pgfsetstrokecolor{currentstroke}%
\pgfsetdash{}{0pt}%
\pgfpathmoveto{\pgfqpoint{0.914761in}{0.632532in}}%
\pgfpathlineto{\pgfqpoint{0.912886in}{0.628174in}}%
\pgfpathlineto{\pgfqpoint{0.911008in}{0.624020in}}%
\pgfpathlineto{\pgfqpoint{0.909126in}{0.620074in}}%
\pgfpathlineto{\pgfqpoint{0.907241in}{0.616340in}}%
\pgfpathlineto{\pgfqpoint{0.891466in}{0.621130in}}%
\pgfpathlineto{\pgfqpoint{0.876009in}{0.626183in}}%
\pgfpathlineto{\pgfqpoint{0.860888in}{0.631493in}}%
\pgfpathlineto{\pgfqpoint{0.846117in}{0.637053in}}%
\pgfpathlineto{\pgfqpoint{0.848418in}{0.640634in}}%
\pgfpathlineto{\pgfqpoint{0.850715in}{0.644427in}}%
\pgfpathlineto{\pgfqpoint{0.853008in}{0.648428in}}%
\pgfpathlineto{\pgfqpoint{0.855297in}{0.652632in}}%
\pgfpathlineto{\pgfqpoint{0.869668in}{0.647235in}}%
\pgfpathlineto{\pgfqpoint{0.884380in}{0.642082in}}%
\pgfpathlineto{\pgfqpoint{0.899416in}{0.637179in}}%
\pgfpathlineto{\pgfqpoint{0.914761in}{0.632532in}}%
\pgfpathclose%
\pgfusepath{fill}%
\end{pgfscope}%
\begin{pgfscope}%
\pgfpathrectangle{\pgfqpoint{0.041670in}{0.041670in}}{\pgfqpoint{2.216660in}{2.216660in}}%
\pgfusepath{clip}%
\pgfsetbuttcap%
\pgfsetroundjoin%
\definecolor{currentfill}{rgb}{0.896320,0.893616,0.096335}%
\pgfsetfillcolor{currentfill}%
\pgfsetlinewidth{0.000000pt}%
\definecolor{currentstroke}{rgb}{0.000000,0.000000,0.000000}%
\pgfsetstrokecolor{currentstroke}%
\pgfsetdash{}{0pt}%
\pgfpathmoveto{\pgfqpoint{1.256232in}{1.634950in}}%
\pgfpathlineto{\pgfqpoint{1.260031in}{1.632649in}}%
\pgfpathlineto{\pgfqpoint{1.263829in}{1.630223in}}%
\pgfpathlineto{\pgfqpoint{1.267624in}{1.627673in}}%
\pgfpathlineto{\pgfqpoint{1.271417in}{1.624999in}}%
\pgfpathlineto{\pgfqpoint{1.272010in}{1.623643in}}%
\pgfpathlineto{\pgfqpoint{1.272512in}{1.622279in}}%
\pgfpathlineto{\pgfqpoint{1.272922in}{1.620908in}}%
\pgfpathlineto{\pgfqpoint{1.273239in}{1.619532in}}%
\pgfpathlineto{\pgfqpoint{1.269367in}{1.622434in}}%
\pgfpathlineto{\pgfqpoint{1.265494in}{1.625213in}}%
\pgfpathlineto{\pgfqpoint{1.261618in}{1.627868in}}%
\pgfpathlineto{\pgfqpoint{1.257741in}{1.630398in}}%
\pgfpathlineto{\pgfqpoint{1.257479in}{1.631544in}}%
\pgfpathlineto{\pgfqpoint{1.257140in}{1.632685in}}%
\pgfpathlineto{\pgfqpoint{1.256725in}{1.633821in}}%
\pgfpathlineto{\pgfqpoint{1.256232in}{1.634950in}}%
\pgfpathclose%
\pgfusepath{fill}%
\end{pgfscope}%
\begin{pgfscope}%
\pgfpathrectangle{\pgfqpoint{0.041670in}{0.041670in}}{\pgfqpoint{2.216660in}{2.216660in}}%
\pgfusepath{clip}%
\pgfsetbuttcap%
\pgfsetroundjoin%
\definecolor{currentfill}{rgb}{0.935904,0.898570,0.108131}%
\pgfsetfillcolor{currentfill}%
\pgfsetlinewidth{0.000000pt}%
\definecolor{currentstroke}{rgb}{0.000000,0.000000,0.000000}%
\pgfsetstrokecolor{currentstroke}%
\pgfsetdash{}{0pt}%
\pgfpathmoveto{\pgfqpoint{1.238850in}{1.646426in}}%
\pgfpathlineto{\pgfqpoint{1.242518in}{1.644854in}}%
\pgfpathlineto{\pgfqpoint{1.246185in}{1.643154in}}%
\pgfpathlineto{\pgfqpoint{1.249850in}{1.641327in}}%
\pgfpathlineto{\pgfqpoint{1.253513in}{1.639374in}}%
\pgfpathlineto{\pgfqpoint{1.254304in}{1.638284in}}%
\pgfpathlineto{\pgfqpoint{1.255022in}{1.637182in}}%
\pgfpathlineto{\pgfqpoint{1.255665in}{1.636071in}}%
\pgfpathlineto{\pgfqpoint{1.256232in}{1.634950in}}%
\pgfpathlineto{\pgfqpoint{1.252432in}{1.637125in}}%
\pgfpathlineto{\pgfqpoint{1.248629in}{1.639174in}}%
\pgfpathlineto{\pgfqpoint{1.244825in}{1.641095in}}%
\pgfpathlineto{\pgfqpoint{1.241019in}{1.642888in}}%
\pgfpathlineto{\pgfqpoint{1.240566in}{1.643785in}}%
\pgfpathlineto{\pgfqpoint{1.240053in}{1.644673in}}%
\pgfpathlineto{\pgfqpoint{1.239481in}{1.645554in}}%
\pgfpathlineto{\pgfqpoint{1.238850in}{1.646426in}}%
\pgfpathclose%
\pgfusepath{fill}%
\end{pgfscope}%
\begin{pgfscope}%
\pgfpathrectangle{\pgfqpoint{0.041670in}{0.041670in}}{\pgfqpoint{2.216660in}{2.216660in}}%
\pgfusepath{clip}%
\pgfsetbuttcap%
\pgfsetroundjoin%
\definecolor{currentfill}{rgb}{0.636902,0.856542,0.216620}%
\pgfsetfillcolor{currentfill}%
\pgfsetlinewidth{0.000000pt}%
\definecolor{currentstroke}{rgb}{0.000000,0.000000,0.000000}%
\pgfsetstrokecolor{currentstroke}%
\pgfsetdash{}{0pt}%
\pgfpathmoveto{\pgfqpoint{1.031772in}{1.527496in}}%
\pgfpathlineto{\pgfqpoint{1.028090in}{1.521900in}}%
\pgfpathlineto{\pgfqpoint{1.024410in}{1.516201in}}%
\pgfpathlineto{\pgfqpoint{1.020731in}{1.510401in}}%
\pgfpathlineto{\pgfqpoint{1.017055in}{1.504500in}}%
\pgfpathlineto{\pgfqpoint{1.015483in}{1.506962in}}%
\pgfpathlineto{\pgfqpoint{1.014076in}{1.509447in}}%
\pgfpathlineto{\pgfqpoint{1.012836in}{1.511950in}}%
\pgfpathlineto{\pgfqpoint{1.011764in}{1.514469in}}%
\pgfpathlineto{\pgfqpoint{1.015555in}{1.520136in}}%
\pgfpathlineto{\pgfqpoint{1.019347in}{1.525704in}}%
\pgfpathlineto{\pgfqpoint{1.023142in}{1.531170in}}%
\pgfpathlineto{\pgfqpoint{1.026938in}{1.536534in}}%
\pgfpathlineto{\pgfqpoint{1.027919in}{1.534250in}}%
\pgfpathlineto{\pgfqpoint{1.029052in}{1.531980in}}%
\pgfpathlineto{\pgfqpoint{1.030337in}{1.529728in}}%
\pgfpathlineto{\pgfqpoint{1.031772in}{1.527496in}}%
\pgfpathclose%
\pgfusepath{fill}%
\end{pgfscope}%
\begin{pgfscope}%
\pgfpathrectangle{\pgfqpoint{0.041670in}{0.041670in}}{\pgfqpoint{2.216660in}{2.216660in}}%
\pgfusepath{clip}%
\pgfsetbuttcap%
\pgfsetroundjoin%
\definecolor{currentfill}{rgb}{0.974417,0.903590,0.130215}%
\pgfsetfillcolor{currentfill}%
\pgfsetlinewidth{0.000000pt}%
\definecolor{currentstroke}{rgb}{0.000000,0.000000,0.000000}%
\pgfsetstrokecolor{currentstroke}%
\pgfsetdash{}{0pt}%
\pgfpathmoveto{\pgfqpoint{1.200729in}{1.660368in}}%
\pgfpathlineto{\pgfqpoint{1.203319in}{1.660592in}}%
\pgfpathlineto{\pgfqpoint{1.205908in}{1.660685in}}%
\pgfpathlineto{\pgfqpoint{1.208496in}{1.660645in}}%
\pgfpathlineto{\pgfqpoint{1.211081in}{1.660475in}}%
\pgfpathlineto{\pgfqpoint{1.212171in}{1.660009in}}%
\pgfpathlineto{\pgfqpoint{1.213228in}{1.659528in}}%
\pgfpathlineto{\pgfqpoint{1.214253in}{1.659032in}}%
\pgfpathlineto{\pgfqpoint{1.215243in}{1.658521in}}%
\pgfpathlineto{\pgfqpoint{1.212311in}{1.658854in}}%
\pgfpathlineto{\pgfqpoint{1.209377in}{1.659055in}}%
\pgfpathlineto{\pgfqpoint{1.206441in}{1.659126in}}%
\pgfpathlineto{\pgfqpoint{1.203504in}{1.659064in}}%
\pgfpathlineto{\pgfqpoint{1.202843in}{1.659405in}}%
\pgfpathlineto{\pgfqpoint{1.202160in}{1.659736in}}%
\pgfpathlineto{\pgfqpoint{1.201455in}{1.660057in}}%
\pgfpathlineto{\pgfqpoint{1.200729in}{1.660368in}}%
\pgfpathclose%
\pgfusepath{fill}%
\end{pgfscope}%
\begin{pgfscope}%
\pgfpathrectangle{\pgfqpoint{0.041670in}{0.041670in}}{\pgfqpoint{2.216660in}{2.216660in}}%
\pgfusepath{clip}%
\pgfsetbuttcap%
\pgfsetroundjoin%
\definecolor{currentfill}{rgb}{0.762373,0.876424,0.137064}%
\pgfsetfillcolor{currentfill}%
\pgfsetlinewidth{0.000000pt}%
\definecolor{currentstroke}{rgb}{0.000000,0.000000,0.000000}%
\pgfsetstrokecolor{currentstroke}%
\pgfsetdash{}{0pt}%
\pgfpathmoveto{\pgfqpoint{1.057386in}{1.575569in}}%
\pgfpathlineto{\pgfqpoint{1.053573in}{1.571079in}}%
\pgfpathlineto{\pgfqpoint{1.049763in}{1.566474in}}%
\pgfpathlineto{\pgfqpoint{1.045954in}{1.561758in}}%
\pgfpathlineto{\pgfqpoint{1.042147in}{1.556930in}}%
\pgfpathlineto{\pgfqpoint{1.041397in}{1.558993in}}%
\pgfpathlineto{\pgfqpoint{1.040786in}{1.561065in}}%
\pgfpathlineto{\pgfqpoint{1.040315in}{1.563144in}}%
\pgfpathlineto{\pgfqpoint{1.039983in}{1.565229in}}%
\pgfpathlineto{\pgfqpoint{1.043845in}{1.569821in}}%
\pgfpathlineto{\pgfqpoint{1.047709in}{1.574303in}}%
\pgfpathlineto{\pgfqpoint{1.051576in}{1.578673in}}%
\pgfpathlineto{\pgfqpoint{1.055444in}{1.582930in}}%
\pgfpathlineto{\pgfqpoint{1.055744in}{1.581080in}}%
\pgfpathlineto{\pgfqpoint{1.056167in}{1.579236in}}%
\pgfpathlineto{\pgfqpoint{1.056715in}{1.577398in}}%
\pgfpathlineto{\pgfqpoint{1.057386in}{1.575569in}}%
\pgfpathclose%
\pgfusepath{fill}%
\end{pgfscope}%
\begin{pgfscope}%
\pgfpathrectangle{\pgfqpoint{0.041670in}{0.041670in}}{\pgfqpoint{2.216660in}{2.216660in}}%
\pgfusepath{clip}%
\pgfsetbuttcap%
\pgfsetroundjoin%
\definecolor{currentfill}{rgb}{0.855810,0.888601,0.097452}%
\pgfsetfillcolor{currentfill}%
\pgfsetlinewidth{0.000000pt}%
\definecolor{currentstroke}{rgb}{0.000000,0.000000,0.000000}%
\pgfsetstrokecolor{currentstroke}%
\pgfsetdash{}{0pt}%
\pgfpathmoveto{\pgfqpoint{1.273239in}{1.619532in}}%
\pgfpathlineto{\pgfqpoint{1.277109in}{1.616506in}}%
\pgfpathlineto{\pgfqpoint{1.280977in}{1.613360in}}%
\pgfpathlineto{\pgfqpoint{1.284843in}{1.610092in}}%
\pgfpathlineto{\pgfqpoint{1.288707in}{1.606705in}}%
\pgfpathlineto{\pgfqpoint{1.288973in}{1.605092in}}%
\pgfpathlineto{\pgfqpoint{1.289130in}{1.603475in}}%
\pgfpathlineto{\pgfqpoint{1.289178in}{1.601856in}}%
\pgfpathlineto{\pgfqpoint{1.289118in}{1.600237in}}%
\pgfpathlineto{\pgfqpoint{1.285235in}{1.603858in}}%
\pgfpathlineto{\pgfqpoint{1.281351in}{1.607358in}}%
\pgfpathlineto{\pgfqpoint{1.277466in}{1.610738in}}%
\pgfpathlineto{\pgfqpoint{1.273578in}{1.613996in}}%
\pgfpathlineto{\pgfqpoint{1.273633in}{1.615381in}}%
\pgfpathlineto{\pgfqpoint{1.273595in}{1.616767in}}%
\pgfpathlineto{\pgfqpoint{1.273464in}{1.618151in}}%
\pgfpathlineto{\pgfqpoint{1.273239in}{1.619532in}}%
\pgfpathclose%
\pgfusepath{fill}%
\end{pgfscope}%
\begin{pgfscope}%
\pgfpathrectangle{\pgfqpoint{0.041670in}{0.041670in}}{\pgfqpoint{2.216660in}{2.216660in}}%
\pgfusepath{clip}%
\pgfsetbuttcap%
\pgfsetroundjoin%
\definecolor{currentfill}{rgb}{0.896320,0.893616,0.096335}%
\pgfsetfillcolor{currentfill}%
\pgfsetlinewidth{0.000000pt}%
\definecolor{currentstroke}{rgb}{0.000000,0.000000,0.000000}%
\pgfsetstrokecolor{currentstroke}%
\pgfsetdash{}{0pt}%
\pgfpathmoveto{\pgfqpoint{1.102001in}{1.629376in}}%
\pgfpathlineto{\pgfqpoint{1.098115in}{1.626795in}}%
\pgfpathlineto{\pgfqpoint{1.094230in}{1.624089in}}%
\pgfpathlineto{\pgfqpoint{1.090347in}{1.621258in}}%
\pgfpathlineto{\pgfqpoint{1.086466in}{1.618304in}}%
\pgfpathlineto{\pgfqpoint{1.086701in}{1.619685in}}%
\pgfpathlineto{\pgfqpoint{1.087029in}{1.621061in}}%
\pgfpathlineto{\pgfqpoint{1.087449in}{1.622431in}}%
\pgfpathlineto{\pgfqpoint{1.087961in}{1.623795in}}%
\pgfpathlineto{\pgfqpoint{1.091777in}{1.626518in}}%
\pgfpathlineto{\pgfqpoint{1.095595in}{1.629119in}}%
\pgfpathlineto{\pgfqpoint{1.099415in}{1.631595in}}%
\pgfpathlineto{\pgfqpoint{1.103236in}{1.633947in}}%
\pgfpathlineto{\pgfqpoint{1.102812in}{1.632812in}}%
\pgfpathlineto{\pgfqpoint{1.102464in}{1.631671in}}%
\pgfpathlineto{\pgfqpoint{1.102194in}{1.630525in}}%
\pgfpathlineto{\pgfqpoint{1.102001in}{1.629376in}}%
\pgfpathclose%
\pgfusepath{fill}%
\end{pgfscope}%
\begin{pgfscope}%
\pgfpathrectangle{\pgfqpoint{0.041670in}{0.041670in}}{\pgfqpoint{2.216660in}{2.216660in}}%
\pgfusepath{clip}%
\pgfsetbuttcap%
\pgfsetroundjoin%
\definecolor{currentfill}{rgb}{0.935904,0.898570,0.108131}%
\pgfsetfillcolor{currentfill}%
\pgfsetlinewidth{0.000000pt}%
\definecolor{currentstroke}{rgb}{0.000000,0.000000,0.000000}%
\pgfsetstrokecolor{currentstroke}%
\pgfsetdash{}{0pt}%
\pgfpathmoveto{\pgfqpoint{1.118540in}{1.642086in}}%
\pgfpathlineto{\pgfqpoint{1.114711in}{1.640243in}}%
\pgfpathlineto{\pgfqpoint{1.110885in}{1.638271in}}%
\pgfpathlineto{\pgfqpoint{1.107060in}{1.636172in}}%
\pgfpathlineto{\pgfqpoint{1.103236in}{1.633947in}}%
\pgfpathlineto{\pgfqpoint{1.103737in}{1.635075in}}%
\pgfpathlineto{\pgfqpoint{1.104313in}{1.636195in}}%
\pgfpathlineto{\pgfqpoint{1.104964in}{1.637305in}}%
\pgfpathlineto{\pgfqpoint{1.105690in}{1.638405in}}%
\pgfpathlineto{\pgfqpoint{1.109389in}{1.640407in}}%
\pgfpathlineto{\pgfqpoint{1.113089in}{1.642283in}}%
\pgfpathlineto{\pgfqpoint{1.116792in}{1.644031in}}%
\pgfpathlineto{\pgfqpoint{1.120496in}{1.645652in}}%
\pgfpathlineto{\pgfqpoint{1.119917in}{1.644772in}}%
\pgfpathlineto{\pgfqpoint{1.119398in}{1.643884in}}%
\pgfpathlineto{\pgfqpoint{1.118938in}{1.642988in}}%
\pgfpathlineto{\pgfqpoint{1.118540in}{1.642086in}}%
\pgfpathclose%
\pgfusepath{fill}%
\end{pgfscope}%
\begin{pgfscope}%
\pgfpathrectangle{\pgfqpoint{0.041670in}{0.041670in}}{\pgfqpoint{2.216660in}{2.216660in}}%
\pgfusepath{clip}%
\pgfsetbuttcap%
\pgfsetroundjoin%
\definecolor{currentfill}{rgb}{0.344074,0.780029,0.397381}%
\pgfsetfillcolor{currentfill}%
\pgfsetlinewidth{0.000000pt}%
\definecolor{currentstroke}{rgb}{0.000000,0.000000,0.000000}%
\pgfsetstrokecolor{currentstroke}%
\pgfsetdash{}{0pt}%
\pgfpathmoveto{\pgfqpoint{1.379331in}{1.416837in}}%
\pgfpathlineto{\pgfqpoint{1.382852in}{1.409607in}}%
\pgfpathlineto{\pgfqpoint{1.386371in}{1.402301in}}%
\pgfpathlineto{\pgfqpoint{1.389888in}{1.394921in}}%
\pgfpathlineto{\pgfqpoint{1.393403in}{1.387469in}}%
\pgfpathlineto{\pgfqpoint{1.390388in}{1.384223in}}%
\pgfpathlineto{\pgfqpoint{1.387159in}{1.381023in}}%
\pgfpathlineto{\pgfqpoint{1.383719in}{1.377872in}}%
\pgfpathlineto{\pgfqpoint{1.380071in}{1.374774in}}%
\pgfpathlineto{\pgfqpoint{1.376769in}{1.382449in}}%
\pgfpathlineto{\pgfqpoint{1.373466in}{1.390052in}}%
\pgfpathlineto{\pgfqpoint{1.370161in}{1.397580in}}%
\pgfpathlineto{\pgfqpoint{1.366854in}{1.405032in}}%
\pgfpathlineto{\pgfqpoint{1.370268in}{1.407913in}}%
\pgfpathlineto{\pgfqpoint{1.373487in}{1.410842in}}%
\pgfpathlineto{\pgfqpoint{1.376509in}{1.413818in}}%
\pgfpathlineto{\pgfqpoint{1.379331in}{1.416837in}}%
\pgfpathclose%
\pgfusepath{fill}%
\end{pgfscope}%
\begin{pgfscope}%
\pgfpathrectangle{\pgfqpoint{0.041670in}{0.041670in}}{\pgfqpoint{2.216660in}{2.216660in}}%
\pgfusepath{clip}%
\pgfsetbuttcap%
\pgfsetroundjoin%
\definecolor{currentfill}{rgb}{0.974417,0.903590,0.130215}%
\pgfsetfillcolor{currentfill}%
\pgfsetlinewidth{0.000000pt}%
\definecolor{currentstroke}{rgb}{0.000000,0.000000,0.000000}%
\pgfsetstrokecolor{currentstroke}%
\pgfsetdash{}{0pt}%
\pgfpathmoveto{\pgfqpoint{1.155838in}{1.658752in}}%
\pgfpathlineto{\pgfqpoint{1.152830in}{1.658775in}}%
\pgfpathlineto{\pgfqpoint{1.149824in}{1.658667in}}%
\pgfpathlineto{\pgfqpoint{1.146819in}{1.658426in}}%
\pgfpathlineto{\pgfqpoint{1.143815in}{1.658054in}}%
\pgfpathlineto{\pgfqpoint{1.144775in}{1.658578in}}%
\pgfpathlineto{\pgfqpoint{1.145769in}{1.659088in}}%
\pgfpathlineto{\pgfqpoint{1.146797in}{1.659583in}}%
\pgfpathlineto{\pgfqpoint{1.147859in}{1.660062in}}%
\pgfpathlineto{\pgfqpoint{1.150525in}{1.660267in}}%
\pgfpathlineto{\pgfqpoint{1.153193in}{1.660341in}}%
\pgfpathlineto{\pgfqpoint{1.155863in}{1.660282in}}%
\pgfpathlineto{\pgfqpoint{1.158534in}{1.660092in}}%
\pgfpathlineto{\pgfqpoint{1.157827in}{1.659772in}}%
\pgfpathlineto{\pgfqpoint{1.157141in}{1.659442in}}%
\pgfpathlineto{\pgfqpoint{1.156478in}{1.659102in}}%
\pgfpathlineto{\pgfqpoint{1.155838in}{1.658752in}}%
\pgfpathclose%
\pgfusepath{fill}%
\end{pgfscope}%
\begin{pgfscope}%
\pgfpathrectangle{\pgfqpoint{0.041670in}{0.041670in}}{\pgfqpoint{2.216660in}{2.216660in}}%
\pgfusepath{clip}%
\pgfsetbuttcap%
\pgfsetroundjoin%
\definecolor{currentfill}{rgb}{0.276194,0.190074,0.493001}%
\pgfsetfillcolor{currentfill}%
\pgfsetlinewidth{0.000000pt}%
\definecolor{currentstroke}{rgb}{0.000000,0.000000,0.000000}%
\pgfsetstrokecolor{currentstroke}%
\pgfsetdash{}{0pt}%
\pgfpathmoveto{\pgfqpoint{0.642430in}{0.729083in}}%
\pgfpathlineto{\pgfqpoint{0.639131in}{0.735768in}}%
\pgfpathlineto{\pgfqpoint{0.635819in}{0.742853in}}%
\pgfpathlineto{\pgfqpoint{0.632491in}{0.750345in}}%
\pgfpathlineto{\pgfqpoint{0.629149in}{0.758251in}}%
\pgfpathlineto{\pgfqpoint{0.615474in}{0.767586in}}%
\pgfpathlineto{\pgfqpoint{0.602414in}{0.777133in}}%
\pgfpathlineto{\pgfqpoint{0.589981in}{0.786879in}}%
\pgfpathlineto{\pgfqpoint{0.578186in}{0.796815in}}%
\pgfpathlineto{\pgfqpoint{0.581819in}{0.788717in}}%
\pgfpathlineto{\pgfqpoint{0.585436in}{0.781032in}}%
\pgfpathlineto{\pgfqpoint{0.589036in}{0.773753in}}%
\pgfpathlineto{\pgfqpoint{0.592622in}{0.766872in}}%
\pgfpathlineto{\pgfqpoint{0.604152in}{0.757135in}}%
\pgfpathlineto{\pgfqpoint{0.616305in}{0.747584in}}%
\pgfpathlineto{\pgfqpoint{0.629068in}{0.738230in}}%
\pgfpathlineto{\pgfqpoint{0.642430in}{0.729083in}}%
\pgfpathclose%
\pgfusepath{fill}%
\end{pgfscope}%
\begin{pgfscope}%
\pgfpathrectangle{\pgfqpoint{0.041670in}{0.041670in}}{\pgfqpoint{2.216660in}{2.216660in}}%
\pgfusepath{clip}%
\pgfsetbuttcap%
\pgfsetroundjoin%
\definecolor{currentfill}{rgb}{0.122606,0.585371,0.546557}%
\pgfsetfillcolor{currentfill}%
\pgfsetlinewidth{0.000000pt}%
\definecolor{currentstroke}{rgb}{0.000000,0.000000,0.000000}%
\pgfsetstrokecolor{currentstroke}%
\pgfsetdash{}{0pt}%
\pgfpathmoveto{\pgfqpoint{0.968344in}{1.175837in}}%
\pgfpathlineto{\pgfqpoint{0.965734in}{1.166708in}}%
\pgfpathlineto{\pgfqpoint{0.963125in}{1.157558in}}%
\pgfpathlineto{\pgfqpoint{0.960517in}{1.148390in}}%
\pgfpathlineto{\pgfqpoint{0.957910in}{1.139206in}}%
\pgfpathlineto{\pgfqpoint{0.950519in}{1.142790in}}%
\pgfpathlineto{\pgfqpoint{0.943369in}{1.146487in}}%
\pgfpathlineto{\pgfqpoint{0.936466in}{1.150294in}}%
\pgfpathlineto{\pgfqpoint{0.929817in}{1.154207in}}%
\pgfpathlineto{\pgfqpoint{0.932747in}{1.163196in}}%
\pgfpathlineto{\pgfqpoint{0.935679in}{1.172169in}}%
\pgfpathlineto{\pgfqpoint{0.938612in}{1.181125in}}%
\pgfpathlineto{\pgfqpoint{0.941547in}{1.190059in}}%
\pgfpathlineto{\pgfqpoint{0.947890in}{1.186350in}}%
\pgfpathlineto{\pgfqpoint{0.954475in}{1.182740in}}%
\pgfpathlineto{\pgfqpoint{0.961295in}{1.179235in}}%
\pgfpathlineto{\pgfqpoint{0.968344in}{1.175837in}}%
\pgfpathclose%
\pgfusepath{fill}%
\end{pgfscope}%
\begin{pgfscope}%
\pgfpathrectangle{\pgfqpoint{0.041670in}{0.041670in}}{\pgfqpoint{2.216660in}{2.216660in}}%
\pgfusepath{clip}%
\pgfsetbuttcap%
\pgfsetroundjoin%
\definecolor{currentfill}{rgb}{0.268510,0.009605,0.335427}%
\pgfsetfillcolor{currentfill}%
\pgfsetlinewidth{0.000000pt}%
\definecolor{currentstroke}{rgb}{0.000000,0.000000,0.000000}%
\pgfsetstrokecolor{currentstroke}%
\pgfsetdash{}{0pt}%
\pgfpathmoveto{\pgfqpoint{0.818124in}{0.612376in}}%
\pgfpathlineto{\pgfqpoint{0.815755in}{0.611994in}}%
\pgfpathlineto{\pgfqpoint{0.813379in}{0.611893in}}%
\pgfpathlineto{\pgfqpoint{0.810996in}{0.612078in}}%
\pgfpathlineto{\pgfqpoint{0.808607in}{0.612556in}}%
\pgfpathlineto{\pgfqpoint{0.792617in}{0.619034in}}%
\pgfpathlineto{\pgfqpoint{0.777056in}{0.625775in}}%
\pgfpathlineto{\pgfqpoint{0.761939in}{0.632771in}}%
\pgfpathlineto{\pgfqpoint{0.747282in}{0.640015in}}%
\pgfpathlineto{\pgfqpoint{0.750056in}{0.639362in}}%
\pgfpathlineto{\pgfqpoint{0.752823in}{0.639002in}}%
\pgfpathlineto{\pgfqpoint{0.755581in}{0.638927in}}%
\pgfpathlineto{\pgfqpoint{0.758332in}{0.639134in}}%
\pgfpathlineto{\pgfqpoint{0.772625in}{0.632075in}}%
\pgfpathlineto{\pgfqpoint{0.787365in}{0.625257in}}%
\pgfpathlineto{\pgfqpoint{0.802537in}{0.618688in}}%
\pgfpathlineto{\pgfqpoint{0.818124in}{0.612376in}}%
\pgfpathclose%
\pgfusepath{fill}%
\end{pgfscope}%
\begin{pgfscope}%
\pgfpathrectangle{\pgfqpoint{0.041670in}{0.041670in}}{\pgfqpoint{2.216660in}{2.216660in}}%
\pgfusepath{clip}%
\pgfsetbuttcap%
\pgfsetroundjoin%
\definecolor{currentfill}{rgb}{0.487026,0.823929,0.312321}%
\pgfsetfillcolor{currentfill}%
\pgfsetlinewidth{0.000000pt}%
\definecolor{currentstroke}{rgb}{0.000000,0.000000,0.000000}%
\pgfsetstrokecolor{currentstroke}%
\pgfsetdash{}{0pt}%
\pgfpathmoveto{\pgfqpoint{1.010976in}{1.469435in}}%
\pgfpathlineto{\pgfqpoint{1.007482in}{1.462830in}}%
\pgfpathlineto{\pgfqpoint{1.003990in}{1.456134in}}%
\pgfpathlineto{\pgfqpoint{1.000500in}{1.449349in}}%
\pgfpathlineto{\pgfqpoint{0.997011in}{1.442477in}}%
\pgfpathlineto{\pgfqpoint{0.994404in}{1.445275in}}%
\pgfpathlineto{\pgfqpoint{0.991984in}{1.448110in}}%
\pgfpathlineto{\pgfqpoint{0.989755in}{1.450979in}}%
\pgfpathlineto{\pgfqpoint{0.987719in}{1.453879in}}%
\pgfpathlineto{\pgfqpoint{0.991378in}{1.460523in}}%
\pgfpathlineto{\pgfqpoint{0.995040in}{1.467079in}}%
\pgfpathlineto{\pgfqpoint{0.998704in}{1.473548in}}%
\pgfpathlineto{\pgfqpoint{1.002370in}{1.479926in}}%
\pgfpathlineto{\pgfqpoint{1.004257in}{1.477257in}}%
\pgfpathlineto{\pgfqpoint{1.006322in}{1.474618in}}%
\pgfpathlineto{\pgfqpoint{1.008562in}{1.472009in}}%
\pgfpathlineto{\pgfqpoint{1.010976in}{1.469435in}}%
\pgfpathclose%
\pgfusepath{fill}%
\end{pgfscope}%
\begin{pgfscope}%
\pgfpathrectangle{\pgfqpoint{0.041670in}{0.041670in}}{\pgfqpoint{2.216660in}{2.216660in}}%
\pgfusepath{clip}%
\pgfsetbuttcap%
\pgfsetroundjoin%
\definecolor{currentfill}{rgb}{0.212395,0.359683,0.551710}%
\pgfsetfillcolor{currentfill}%
\pgfsetlinewidth{0.000000pt}%
\definecolor{currentstroke}{rgb}{0.000000,0.000000,0.000000}%
\pgfsetstrokecolor{currentstroke}%
\pgfsetdash{}{0pt}%
\pgfpathmoveto{\pgfqpoint{1.423882in}{0.943429in}}%
\pgfpathlineto{\pgfqpoint{1.426194in}{0.934440in}}%
\pgfpathlineto{\pgfqpoint{1.428506in}{0.925501in}}%
\pgfpathlineto{\pgfqpoint{1.430819in}{0.916616in}}%
\pgfpathlineto{\pgfqpoint{1.433131in}{0.907786in}}%
\pgfpathlineto{\pgfqpoint{1.422530in}{0.903718in}}%
\pgfpathlineto{\pgfqpoint{1.411669in}{0.899823in}}%
\pgfpathlineto{\pgfqpoint{1.400560in}{0.896107in}}%
\pgfpathlineto{\pgfqpoint{1.389214in}{0.892573in}}%
\pgfpathlineto{\pgfqpoint{1.387299in}{0.901560in}}%
\pgfpathlineto{\pgfqpoint{1.385383in}{0.910603in}}%
\pgfpathlineto{\pgfqpoint{1.383468in}{0.919699in}}%
\pgfpathlineto{\pgfqpoint{1.381553in}{0.928845in}}%
\pgfpathlineto{\pgfqpoint{1.392488in}{0.932232in}}%
\pgfpathlineto{\pgfqpoint{1.403195in}{0.935794in}}%
\pgfpathlineto{\pgfqpoint{1.413663in}{0.939528in}}%
\pgfpathlineto{\pgfqpoint{1.423882in}{0.943429in}}%
\pgfpathclose%
\pgfusepath{fill}%
\end{pgfscope}%
\begin{pgfscope}%
\pgfpathrectangle{\pgfqpoint{0.041670in}{0.041670in}}{\pgfqpoint{2.216660in}{2.216660in}}%
\pgfusepath{clip}%
\pgfsetbuttcap%
\pgfsetroundjoin%
\definecolor{currentfill}{rgb}{0.955300,0.901065,0.118128}%
\pgfsetfillcolor{currentfill}%
\pgfsetlinewidth{0.000000pt}%
\definecolor{currentstroke}{rgb}{0.000000,0.000000,0.000000}%
\pgfsetstrokecolor{currentstroke}%
\pgfsetdash{}{0pt}%
\pgfpathmoveto{\pgfqpoint{1.221834in}{1.653959in}}%
\pgfpathlineto{\pgfqpoint{1.225315in}{1.653116in}}%
\pgfpathlineto{\pgfqpoint{1.228794in}{1.652142in}}%
\pgfpathlineto{\pgfqpoint{1.232271in}{1.651039in}}%
\pgfpathlineto{\pgfqpoint{1.235747in}{1.649807in}}%
\pgfpathlineto{\pgfqpoint{1.236607in}{1.648979in}}%
\pgfpathlineto{\pgfqpoint{1.237412in}{1.648139in}}%
\pgfpathlineto{\pgfqpoint{1.238160in}{1.647288in}}%
\pgfpathlineto{\pgfqpoint{1.238850in}{1.646426in}}%
\pgfpathlineto{\pgfqpoint{1.235179in}{1.647870in}}%
\pgfpathlineto{\pgfqpoint{1.231507in}{1.649184in}}%
\pgfpathlineto{\pgfqpoint{1.227833in}{1.650369in}}%
\pgfpathlineto{\pgfqpoint{1.224158in}{1.651423in}}%
\pgfpathlineto{\pgfqpoint{1.223641in}{1.652070in}}%
\pgfpathlineto{\pgfqpoint{1.223082in}{1.652708in}}%
\pgfpathlineto{\pgfqpoint{1.222479in}{1.653338in}}%
\pgfpathlineto{\pgfqpoint{1.221834in}{1.653959in}}%
\pgfpathclose%
\pgfusepath{fill}%
\end{pgfscope}%
\begin{pgfscope}%
\pgfpathrectangle{\pgfqpoint{0.041670in}{0.041670in}}{\pgfqpoint{2.216660in}{2.216660in}}%
\pgfusepath{clip}%
\pgfsetbuttcap%
\pgfsetroundjoin%
\definecolor{currentfill}{rgb}{0.231674,0.318106,0.544834}%
\pgfsetfillcolor{currentfill}%
\pgfsetlinewidth{0.000000pt}%
\definecolor{currentstroke}{rgb}{0.000000,0.000000,0.000000}%
\pgfsetstrokecolor{currentstroke}%
\pgfsetdash{}{0pt}%
\pgfpathmoveto{\pgfqpoint{0.980971in}{0.889589in}}%
\pgfpathlineto{\pgfqpoint{0.979149in}{0.880631in}}%
\pgfpathlineto{\pgfqpoint{0.977326in}{0.871735in}}%
\pgfpathlineto{\pgfqpoint{0.975504in}{0.862906in}}%
\pgfpathlineto{\pgfqpoint{0.973681in}{0.854146in}}%
\pgfpathlineto{\pgfqpoint{0.961714in}{0.857654in}}%
\pgfpathlineto{\pgfqpoint{0.949984in}{0.861356in}}%
\pgfpathlineto{\pgfqpoint{0.938500in}{0.865248in}}%
\pgfpathlineto{\pgfqpoint{0.927277in}{0.869325in}}%
\pgfpathlineto{\pgfqpoint{0.929505in}{0.877933in}}%
\pgfpathlineto{\pgfqpoint{0.931733in}{0.886610in}}%
\pgfpathlineto{\pgfqpoint{0.933961in}{0.895354in}}%
\pgfpathlineto{\pgfqpoint{0.936189in}{0.904161in}}%
\pgfpathlineto{\pgfqpoint{0.947021in}{0.900247in}}%
\pgfpathlineto{\pgfqpoint{0.958103in}{0.896511in}}%
\pgfpathlineto{\pgfqpoint{0.969424in}{0.892957in}}%
\pgfpathlineto{\pgfqpoint{0.980971in}{0.889589in}}%
\pgfpathclose%
\pgfusepath{fill}%
\end{pgfscope}%
\begin{pgfscope}%
\pgfpathrectangle{\pgfqpoint{0.041670in}{0.041670in}}{\pgfqpoint{2.216660in}{2.216660in}}%
\pgfusepath{clip}%
\pgfsetbuttcap%
\pgfsetroundjoin%
\definecolor{currentfill}{rgb}{0.163625,0.471133,0.558148}%
\pgfsetfillcolor{currentfill}%
\pgfsetlinewidth{0.000000pt}%
\definecolor{currentstroke}{rgb}{0.000000,0.000000,0.000000}%
\pgfsetstrokecolor{currentstroke}%
\pgfsetdash{}{0pt}%
\pgfpathmoveto{\pgfqpoint{0.971847in}{1.050984in}}%
\pgfpathlineto{\pgfqpoint{0.969616in}{1.041608in}}%
\pgfpathlineto{\pgfqpoint{0.967385in}{1.032244in}}%
\pgfpathlineto{\pgfqpoint{0.965155in}{1.022896in}}%
\pgfpathlineto{\pgfqpoint{0.962925in}{1.013566in}}%
\pgfpathlineto{\pgfqpoint{0.953499in}{1.017150in}}%
\pgfpathlineto{\pgfqpoint{0.944313in}{1.020883in}}%
\pgfpathlineto{\pgfqpoint{0.935376in}{1.024760in}}%
\pgfpathlineto{\pgfqpoint{0.926699in}{1.028777in}}%
\pgfpathlineto{\pgfqpoint{0.929295in}{1.037931in}}%
\pgfpathlineto{\pgfqpoint{0.931892in}{1.047104in}}%
\pgfpathlineto{\pgfqpoint{0.934490in}{1.056293in}}%
\pgfpathlineto{\pgfqpoint{0.937089in}{1.065494in}}%
\pgfpathlineto{\pgfqpoint{0.945416in}{1.061661in}}%
\pgfpathlineto{\pgfqpoint{0.953991in}{1.057963in}}%
\pgfpathlineto{\pgfqpoint{0.962804in}{1.054402in}}%
\pgfpathlineto{\pgfqpoint{0.971847in}{1.050984in}}%
\pgfpathclose%
\pgfusepath{fill}%
\end{pgfscope}%
\begin{pgfscope}%
\pgfpathrectangle{\pgfqpoint{0.041670in}{0.041670in}}{\pgfqpoint{2.216660in}{2.216660in}}%
\pgfusepath{clip}%
\pgfsetbuttcap%
\pgfsetroundjoin%
\definecolor{currentfill}{rgb}{0.166383,0.690856,0.496502}%
\pgfsetfillcolor{currentfill}%
\pgfsetlinewidth{0.000000pt}%
\definecolor{currentstroke}{rgb}{0.000000,0.000000,0.000000}%
\pgfsetstrokecolor{currentstroke}%
\pgfsetdash{}{0pt}%
\pgfpathmoveto{\pgfqpoint{0.976877in}{1.294714in}}%
\pgfpathlineto{\pgfqpoint{0.973925in}{1.286228in}}%
\pgfpathlineto{\pgfqpoint{0.970974in}{1.277691in}}%
\pgfpathlineto{\pgfqpoint{0.968024in}{1.269106in}}%
\pgfpathlineto{\pgfqpoint{0.965076in}{1.260475in}}%
\pgfpathlineto{\pgfqpoint{0.959571in}{1.263870in}}%
\pgfpathlineto{\pgfqpoint{0.954293in}{1.267347in}}%
\pgfpathlineto{\pgfqpoint{0.949248in}{1.270904in}}%
\pgfpathlineto{\pgfqpoint{0.944440in}{1.274535in}}%
\pgfpathlineto{\pgfqpoint{0.947665in}{1.282956in}}%
\pgfpathlineto{\pgfqpoint{0.950891in}{1.291332in}}%
\pgfpathlineto{\pgfqpoint{0.954119in}{1.299659in}}%
\pgfpathlineto{\pgfqpoint{0.957349in}{1.307936in}}%
\pgfpathlineto{\pgfqpoint{0.961899in}{1.304521in}}%
\pgfpathlineto{\pgfqpoint{0.966674in}{1.301177in}}%
\pgfpathlineto{\pgfqpoint{0.971668in}{1.297906in}}%
\pgfpathlineto{\pgfqpoint{0.976877in}{1.294714in}}%
\pgfpathclose%
\pgfusepath{fill}%
\end{pgfscope}%
\begin{pgfscope}%
\pgfpathrectangle{\pgfqpoint{0.041670in}{0.041670in}}{\pgfqpoint{2.216660in}{2.216660in}}%
\pgfusepath{clip}%
\pgfsetbuttcap%
\pgfsetroundjoin%
\definecolor{currentfill}{rgb}{0.955300,0.901065,0.118128}%
\pgfsetfillcolor{currentfill}%
\pgfsetlinewidth{0.000000pt}%
\definecolor{currentstroke}{rgb}{0.000000,0.000000,0.000000}%
\pgfsetstrokecolor{currentstroke}%
\pgfsetdash{}{0pt}%
\pgfpathmoveto{\pgfqpoint{1.135330in}{1.650843in}}%
\pgfpathlineto{\pgfqpoint{1.131619in}{1.649740in}}%
\pgfpathlineto{\pgfqpoint{1.127910in}{1.648506in}}%
\pgfpathlineto{\pgfqpoint{1.124202in}{1.647144in}}%
\pgfpathlineto{\pgfqpoint{1.120496in}{1.645652in}}%
\pgfpathlineto{\pgfqpoint{1.121134in}{1.646522in}}%
\pgfpathlineto{\pgfqpoint{1.121830in}{1.647383in}}%
\pgfpathlineto{\pgfqpoint{1.122584in}{1.648233in}}%
\pgfpathlineto{\pgfqpoint{1.123395in}{1.649072in}}%
\pgfpathlineto{\pgfqpoint{1.126919in}{1.650350in}}%
\pgfpathlineto{\pgfqpoint{1.130444in}{1.651499in}}%
\pgfpathlineto{\pgfqpoint{1.133971in}{1.652518in}}%
\pgfpathlineto{\pgfqpoint{1.137500in}{1.653408in}}%
\pgfpathlineto{\pgfqpoint{1.136893in}{1.652779in}}%
\pgfpathlineto{\pgfqpoint{1.136328in}{1.652141in}}%
\pgfpathlineto{\pgfqpoint{1.135807in}{1.651496in}}%
\pgfpathlineto{\pgfqpoint{1.135330in}{1.650843in}}%
\pgfpathclose%
\pgfusepath{fill}%
\end{pgfscope}%
\begin{pgfscope}%
\pgfpathrectangle{\pgfqpoint{0.041670in}{0.041670in}}{\pgfqpoint{2.216660in}{2.216660in}}%
\pgfusepath{clip}%
\pgfsetbuttcap%
\pgfsetroundjoin%
\definecolor{currentfill}{rgb}{0.855810,0.888601,0.097452}%
\pgfsetfillcolor{currentfill}%
\pgfsetlinewidth{0.000000pt}%
\definecolor{currentstroke}{rgb}{0.000000,0.000000,0.000000}%
\pgfsetstrokecolor{currentstroke}%
\pgfsetdash{}{0pt}%
\pgfpathmoveto{\pgfqpoint{1.086459in}{1.612765in}}%
\pgfpathlineto{\pgfqpoint{1.082576in}{1.609455in}}%
\pgfpathlineto{\pgfqpoint{1.078695in}{1.606024in}}%
\pgfpathlineto{\pgfqpoint{1.074815in}{1.602471in}}%
\pgfpathlineto{\pgfqpoint{1.070937in}{1.598799in}}%
\pgfpathlineto{\pgfqpoint{1.070780in}{1.600417in}}%
\pgfpathlineto{\pgfqpoint{1.070732in}{1.602036in}}%
\pgfpathlineto{\pgfqpoint{1.070792in}{1.603655in}}%
\pgfpathlineto{\pgfqpoint{1.070961in}{1.605271in}}%
\pgfpathlineto{\pgfqpoint{1.074835in}{1.608710in}}%
\pgfpathlineto{\pgfqpoint{1.078710in}{1.612029in}}%
\pgfpathlineto{\pgfqpoint{1.082587in}{1.615228in}}%
\pgfpathlineto{\pgfqpoint{1.086466in}{1.618304in}}%
\pgfpathlineto{\pgfqpoint{1.086325in}{1.616921in}}%
\pgfpathlineto{\pgfqpoint{1.086276in}{1.615535in}}%
\pgfpathlineto{\pgfqpoint{1.086321in}{1.614150in}}%
\pgfpathlineto{\pgfqpoint{1.086459in}{1.612765in}}%
\pgfpathclose%
\pgfusepath{fill}%
\end{pgfscope}%
\begin{pgfscope}%
\pgfpathrectangle{\pgfqpoint{0.041670in}{0.041670in}}{\pgfqpoint{2.216660in}{2.216660in}}%
\pgfusepath{clip}%
\pgfsetbuttcap%
\pgfsetroundjoin%
\definecolor{currentfill}{rgb}{0.974417,0.903590,0.130215}%
\pgfsetfillcolor{currentfill}%
\pgfsetlinewidth{0.000000pt}%
\definecolor{currentstroke}{rgb}{0.000000,0.000000,0.000000}%
\pgfsetstrokecolor{currentstroke}%
\pgfsetdash{}{0pt}%
\pgfpathmoveto{\pgfqpoint{1.203504in}{1.659064in}}%
\pgfpathlineto{\pgfqpoint{1.206441in}{1.659126in}}%
\pgfpathlineto{\pgfqpoint{1.209377in}{1.659055in}}%
\pgfpathlineto{\pgfqpoint{1.212311in}{1.658854in}}%
\pgfpathlineto{\pgfqpoint{1.215243in}{1.658521in}}%
\pgfpathlineto{\pgfqpoint{1.216199in}{1.657995in}}%
\pgfpathlineto{\pgfqpoint{1.217119in}{1.657455in}}%
\pgfpathlineto{\pgfqpoint{1.218002in}{1.656903in}}%
\pgfpathlineto{\pgfqpoint{1.218847in}{1.656337in}}%
\pgfpathlineto{\pgfqpoint{1.215614in}{1.656852in}}%
\pgfpathlineto{\pgfqpoint{1.212379in}{1.657235in}}%
\pgfpathlineto{\pgfqpoint{1.209143in}{1.657487in}}%
\pgfpathlineto{\pgfqpoint{1.205906in}{1.657607in}}%
\pgfpathlineto{\pgfqpoint{1.205342in}{1.657984in}}%
\pgfpathlineto{\pgfqpoint{1.204754in}{1.658353in}}%
\pgfpathlineto{\pgfqpoint{1.204141in}{1.658713in}}%
\pgfpathlineto{\pgfqpoint{1.203504in}{1.659064in}}%
\pgfpathclose%
\pgfusepath{fill}%
\end{pgfscope}%
\begin{pgfscope}%
\pgfpathrectangle{\pgfqpoint{0.041670in}{0.041670in}}{\pgfqpoint{2.216660in}{2.216660in}}%
\pgfusepath{clip}%
\pgfsetbuttcap%
\pgfsetroundjoin%
\definecolor{currentfill}{rgb}{0.974417,0.903590,0.130215}%
\pgfsetfillcolor{currentfill}%
\pgfsetlinewidth{0.000000pt}%
\definecolor{currentstroke}{rgb}{0.000000,0.000000,0.000000}%
\pgfsetstrokecolor{currentstroke}%
\pgfsetdash{}{0pt}%
\pgfpathmoveto{\pgfqpoint{1.153525in}{1.657265in}}%
\pgfpathlineto{\pgfqpoint{1.150227in}{1.657102in}}%
\pgfpathlineto{\pgfqpoint{1.146931in}{1.656808in}}%
\pgfpathlineto{\pgfqpoint{1.143637in}{1.656382in}}%
\pgfpathlineto{\pgfqpoint{1.140344in}{1.655824in}}%
\pgfpathlineto{\pgfqpoint{1.141155in}{1.656401in}}%
\pgfpathlineto{\pgfqpoint{1.142004in}{1.656965in}}%
\pgfpathlineto{\pgfqpoint{1.142891in}{1.657516in}}%
\pgfpathlineto{\pgfqpoint{1.143815in}{1.658054in}}%
\pgfpathlineto{\pgfqpoint{1.146819in}{1.658426in}}%
\pgfpathlineto{\pgfqpoint{1.149824in}{1.658667in}}%
\pgfpathlineto{\pgfqpoint{1.152830in}{1.658775in}}%
\pgfpathlineto{\pgfqpoint{1.155838in}{1.658752in}}%
\pgfpathlineto{\pgfqpoint{1.155223in}{1.658393in}}%
\pgfpathlineto{\pgfqpoint{1.154631in}{1.658026in}}%
\pgfpathlineto{\pgfqpoint{1.154065in}{1.657649in}}%
\pgfpathlineto{\pgfqpoint{1.153525in}{1.657265in}}%
\pgfpathclose%
\pgfusepath{fill}%
\end{pgfscope}%
\begin{pgfscope}%
\pgfpathrectangle{\pgfqpoint{0.041670in}{0.041670in}}{\pgfqpoint{2.216660in}{2.216660in}}%
\pgfusepath{clip}%
\pgfsetbuttcap%
\pgfsetroundjoin%
\definecolor{currentfill}{rgb}{0.120081,0.622161,0.534946}%
\pgfsetfillcolor{currentfill}%
\pgfsetlinewidth{0.000000pt}%
\definecolor{currentstroke}{rgb}{0.000000,0.000000,0.000000}%
\pgfsetstrokecolor{currentstroke}%
\pgfsetdash{}{0pt}%
\pgfpathmoveto{\pgfqpoint{1.411778in}{1.228732in}}%
\pgfpathlineto{\pgfqpoint{1.414784in}{1.219953in}}%
\pgfpathlineto{\pgfqpoint{1.417789in}{1.211142in}}%
\pgfpathlineto{\pgfqpoint{1.420792in}{1.202303in}}%
\pgfpathlineto{\pgfqpoint{1.423793in}{1.193437in}}%
\pgfpathlineto{\pgfqpoint{1.417670in}{1.189642in}}%
\pgfpathlineto{\pgfqpoint{1.411300in}{1.185944in}}%
\pgfpathlineto{\pgfqpoint{1.404688in}{1.182346in}}%
\pgfpathlineto{\pgfqpoint{1.397842in}{1.178852in}}%
\pgfpathlineto{\pgfqpoint{1.395155in}{1.187916in}}%
\pgfpathlineto{\pgfqpoint{1.392466in}{1.196953in}}%
\pgfpathlineto{\pgfqpoint{1.389776in}{1.205961in}}%
\pgfpathlineto{\pgfqpoint{1.387084in}{1.214938in}}%
\pgfpathlineto{\pgfqpoint{1.393598in}{1.218242in}}%
\pgfpathlineto{\pgfqpoint{1.399888in}{1.221645in}}%
\pgfpathlineto{\pgfqpoint{1.405951in}{1.225143in}}%
\pgfpathlineto{\pgfqpoint{1.411778in}{1.228732in}}%
\pgfpathclose%
\pgfusepath{fill}%
\end{pgfscope}%
\begin{pgfscope}%
\pgfpathrectangle{\pgfqpoint{0.041670in}{0.041670in}}{\pgfqpoint{2.216660in}{2.216660in}}%
\pgfusepath{clip}%
\pgfsetbuttcap%
\pgfsetroundjoin%
\definecolor{currentfill}{rgb}{0.699415,0.867117,0.175971}%
\pgfsetfillcolor{currentfill}%
\pgfsetlinewidth{0.000000pt}%
\definecolor{currentstroke}{rgb}{0.000000,0.000000,0.000000}%
\pgfsetstrokecolor{currentstroke}%
\pgfsetdash{}{0pt}%
\pgfpathmoveto{\pgfqpoint{1.318436in}{1.558763in}}%
\pgfpathlineto{\pgfqpoint{1.322259in}{1.553877in}}%
\pgfpathlineto{\pgfqpoint{1.326079in}{1.548883in}}%
\pgfpathlineto{\pgfqpoint{1.329898in}{1.543782in}}%
\pgfpathlineto{\pgfqpoint{1.333714in}{1.538575in}}%
\pgfpathlineto{\pgfqpoint{1.332870in}{1.536279in}}%
\pgfpathlineto{\pgfqpoint{1.331873in}{1.533997in}}%
\pgfpathlineto{\pgfqpoint{1.330723in}{1.531729in}}%
\pgfpathlineto{\pgfqpoint{1.329421in}{1.529479in}}%
\pgfpathlineto{\pgfqpoint{1.325706in}{1.534919in}}%
\pgfpathlineto{\pgfqpoint{1.321990in}{1.540253in}}%
\pgfpathlineto{\pgfqpoint{1.318272in}{1.545479in}}%
\pgfpathlineto{\pgfqpoint{1.314551in}{1.550597in}}%
\pgfpathlineto{\pgfqpoint{1.315728in}{1.552617in}}%
\pgfpathlineto{\pgfqpoint{1.316768in}{1.554652in}}%
\pgfpathlineto{\pgfqpoint{1.317671in}{1.556702in}}%
\pgfpathlineto{\pgfqpoint{1.318436in}{1.558763in}}%
\pgfpathclose%
\pgfusepath{fill}%
\end{pgfscope}%
\begin{pgfscope}%
\pgfpathrectangle{\pgfqpoint{0.041670in}{0.041670in}}{\pgfqpoint{2.216660in}{2.216660in}}%
\pgfusepath{clip}%
\pgfsetbuttcap%
\pgfsetroundjoin%
\definecolor{currentfill}{rgb}{0.147607,0.511733,0.557049}%
\pgfsetfillcolor{currentfill}%
\pgfsetlinewidth{0.000000pt}%
\definecolor{currentstroke}{rgb}{0.000000,0.000000,0.000000}%
\pgfsetstrokecolor{currentstroke}%
\pgfsetdash{}{0pt}%
\pgfpathmoveto{\pgfqpoint{1.419303in}{1.105712in}}%
\pgfpathlineto{\pgfqpoint{1.421981in}{1.096532in}}%
\pgfpathlineto{\pgfqpoint{1.424657in}{1.087352in}}%
\pgfpathlineto{\pgfqpoint{1.427333in}{1.078177in}}%
\pgfpathlineto{\pgfqpoint{1.430008in}{1.069009in}}%
\pgfpathlineto{\pgfqpoint{1.421908in}{1.065061in}}%
\pgfpathlineto{\pgfqpoint{1.413553in}{1.061244in}}%
\pgfpathlineto{\pgfqpoint{1.404951in}{1.057561in}}%
\pgfpathlineto{\pgfqpoint{1.396112in}{1.054015in}}%
\pgfpathlineto{\pgfqpoint{1.393795in}{1.063364in}}%
\pgfpathlineto{\pgfqpoint{1.391477in}{1.072719in}}%
\pgfpathlineto{\pgfqpoint{1.389158in}{1.082078in}}%
\pgfpathlineto{\pgfqpoint{1.386839in}{1.091437in}}%
\pgfpathlineto{\pgfqpoint{1.395304in}{1.094812in}}%
\pgfpathlineto{\pgfqpoint{1.403541in}{1.098319in}}%
\pgfpathlineto{\pgfqpoint{1.411544in}{1.101953in}}%
\pgfpathlineto{\pgfqpoint{1.419303in}{1.105712in}}%
\pgfpathclose%
\pgfusepath{fill}%
\end{pgfscope}%
\begin{pgfscope}%
\pgfpathrectangle{\pgfqpoint{0.041670in}{0.041670in}}{\pgfqpoint{2.216660in}{2.216660in}}%
\pgfusepath{clip}%
\pgfsetbuttcap%
\pgfsetroundjoin%
\definecolor{currentfill}{rgb}{0.814576,0.883393,0.110347}%
\pgfsetfillcolor{currentfill}%
\pgfsetlinewidth{0.000000pt}%
\definecolor{currentstroke}{rgb}{0.000000,0.000000,0.000000}%
\pgfsetstrokecolor{currentstroke}%
\pgfsetdash{}{0pt}%
\pgfpathmoveto{\pgfqpoint{1.289118in}{1.600237in}}%
\pgfpathlineto{\pgfqpoint{1.292998in}{1.596497in}}%
\pgfpathlineto{\pgfqpoint{1.296877in}{1.592640in}}%
\pgfpathlineto{\pgfqpoint{1.300753in}{1.588666in}}%
\pgfpathlineto{\pgfqpoint{1.304628in}{1.584576in}}%
\pgfpathlineto{\pgfqpoint{1.304439in}{1.582724in}}%
\pgfpathlineto{\pgfqpoint{1.304125in}{1.580875in}}%
\pgfpathlineto{\pgfqpoint{1.303688in}{1.579031in}}%
\pgfpathlineto{\pgfqpoint{1.303126in}{1.577194in}}%
\pgfpathlineto{\pgfqpoint{1.299294in}{1.581518in}}%
\pgfpathlineto{\pgfqpoint{1.295461in}{1.585726in}}%
\pgfpathlineto{\pgfqpoint{1.291625in}{1.589817in}}%
\pgfpathlineto{\pgfqpoint{1.287788in}{1.593790in}}%
\pgfpathlineto{\pgfqpoint{1.288283in}{1.595394in}}%
\pgfpathlineto{\pgfqpoint{1.288670in}{1.597004in}}%
\pgfpathlineto{\pgfqpoint{1.288948in}{1.598619in}}%
\pgfpathlineto{\pgfqpoint{1.289118in}{1.600237in}}%
\pgfpathclose%
\pgfusepath{fill}%
\end{pgfscope}%
\begin{pgfscope}%
\pgfpathrectangle{\pgfqpoint{0.041670in}{0.041670in}}{\pgfqpoint{2.216660in}{2.216660in}}%
\pgfusepath{clip}%
\pgfsetbuttcap%
\pgfsetroundjoin%
\definecolor{currentfill}{rgb}{0.268510,0.009605,0.335427}%
\pgfsetfillcolor{currentfill}%
\pgfsetlinewidth{0.000000pt}%
\definecolor{currentstroke}{rgb}{0.000000,0.000000,0.000000}%
\pgfsetstrokecolor{currentstroke}%
\pgfsetdash{}{0pt}%
\pgfpathmoveto{\pgfqpoint{0.907241in}{0.616340in}}%
\pgfpathlineto{\pgfqpoint{0.905353in}{0.612823in}}%
\pgfpathlineto{\pgfqpoint{0.903461in}{0.609529in}}%
\pgfpathlineto{\pgfqpoint{0.901565in}{0.606461in}}%
\pgfpathlineto{\pgfqpoint{0.899665in}{0.603624in}}%
\pgfpathlineto{\pgfqpoint{0.883456in}{0.608555in}}%
\pgfpathlineto{\pgfqpoint{0.867576in}{0.613757in}}%
\pgfpathlineto{\pgfqpoint{0.852041in}{0.619223in}}%
\pgfpathlineto{\pgfqpoint{0.836868in}{0.624947in}}%
\pgfpathlineto{\pgfqpoint{0.839187in}{0.627632in}}%
\pgfpathlineto{\pgfqpoint{0.841501in}{0.630547in}}%
\pgfpathlineto{\pgfqpoint{0.843811in}{0.633690in}}%
\pgfpathlineto{\pgfqpoint{0.846117in}{0.637053in}}%
\pgfpathlineto{\pgfqpoint{0.860888in}{0.631493in}}%
\pgfpathlineto{\pgfqpoint{0.876009in}{0.626183in}}%
\pgfpathlineto{\pgfqpoint{0.891466in}{0.621130in}}%
\pgfpathlineto{\pgfqpoint{0.907241in}{0.616340in}}%
\pgfpathclose%
\pgfusepath{fill}%
\end{pgfscope}%
\begin{pgfscope}%
\pgfpathrectangle{\pgfqpoint{0.041670in}{0.041670in}}{\pgfqpoint{2.216660in}{2.216660in}}%
\pgfusepath{clip}%
\pgfsetbuttcap%
\pgfsetroundjoin%
\definecolor{currentfill}{rgb}{0.344074,0.780029,0.397381}%
\pgfsetfillcolor{currentfill}%
\pgfsetlinewidth{0.000000pt}%
\definecolor{currentstroke}{rgb}{0.000000,0.000000,0.000000}%
\pgfsetstrokecolor{currentstroke}%
\pgfsetdash{}{0pt}%
\pgfpathmoveto{\pgfqpoint{0.996249in}{1.402516in}}%
\pgfpathlineto{\pgfqpoint{0.992998in}{1.395017in}}%
\pgfpathlineto{\pgfqpoint{0.989748in}{1.387441in}}%
\pgfpathlineto{\pgfqpoint{0.986500in}{1.379790in}}%
\pgfpathlineto{\pgfqpoint{0.983254in}{1.372068in}}%
\pgfpathlineto{\pgfqpoint{0.979424in}{1.375116in}}%
\pgfpathlineto{\pgfqpoint{0.975798in}{1.378219in}}%
\pgfpathlineto{\pgfqpoint{0.972381in}{1.381376in}}%
\pgfpathlineto{\pgfqpoint{0.969176in}{1.384581in}}%
\pgfpathlineto{\pgfqpoint{0.972648in}{1.392084in}}%
\pgfpathlineto{\pgfqpoint{0.976123in}{1.399514in}}%
\pgfpathlineto{\pgfqpoint{0.979599in}{1.406871in}}%
\pgfpathlineto{\pgfqpoint{0.983077in}{1.414152in}}%
\pgfpathlineto{\pgfqpoint{0.986077in}{1.411171in}}%
\pgfpathlineto{\pgfqpoint{0.989275in}{1.408236in}}%
\pgfpathlineto{\pgfqpoint{0.992667in}{1.405350in}}%
\pgfpathlineto{\pgfqpoint{0.996249in}{1.402516in}}%
\pgfpathclose%
\pgfusepath{fill}%
\end{pgfscope}%
\begin{pgfscope}%
\pgfpathrectangle{\pgfqpoint{0.041670in}{0.041670in}}{\pgfqpoint{2.216660in}{2.216660in}}%
\pgfusepath{clip}%
\pgfsetbuttcap%
\pgfsetroundjoin%
\definecolor{currentfill}{rgb}{0.565498,0.842430,0.262877}%
\pgfsetfillcolor{currentfill}%
\pgfsetlinewidth{0.000000pt}%
\definecolor{currentstroke}{rgb}{0.000000,0.000000,0.000000}%
\pgfsetstrokecolor{currentstroke}%
\pgfsetdash{}{0pt}%
\pgfpathmoveto{\pgfqpoint{1.344261in}{1.506688in}}%
\pgfpathlineto{\pgfqpoint{1.347965in}{1.500740in}}%
\pgfpathlineto{\pgfqpoint{1.351668in}{1.494694in}}%
\pgfpathlineto{\pgfqpoint{1.355369in}{1.488554in}}%
\pgfpathlineto{\pgfqpoint{1.359067in}{1.482320in}}%
\pgfpathlineto{\pgfqpoint{1.357339in}{1.479628in}}%
\pgfpathlineto{\pgfqpoint{1.355432in}{1.476962in}}%
\pgfpathlineto{\pgfqpoint{1.353348in}{1.474326in}}%
\pgfpathlineto{\pgfqpoint{1.351088in}{1.471722in}}%
\pgfpathlineto{\pgfqpoint{1.347549in}{1.478185in}}%
\pgfpathlineto{\pgfqpoint{1.344008in}{1.484554in}}%
\pgfpathlineto{\pgfqpoint{1.340465in}{1.490827in}}%
\pgfpathlineto{\pgfqpoint{1.336920in}{1.497003in}}%
\pgfpathlineto{\pgfqpoint{1.338998in}{1.499383in}}%
\pgfpathlineto{\pgfqpoint{1.340915in}{1.501792in}}%
\pgfpathlineto{\pgfqpoint{1.342670in}{1.504228in}}%
\pgfpathlineto{\pgfqpoint{1.344261in}{1.506688in}}%
\pgfpathclose%
\pgfusepath{fill}%
\end{pgfscope}%
\begin{pgfscope}%
\pgfpathrectangle{\pgfqpoint{0.041670in}{0.041670in}}{\pgfqpoint{2.216660in}{2.216660in}}%
\pgfusepath{clip}%
\pgfsetbuttcap%
\pgfsetroundjoin%
\definecolor{currentfill}{rgb}{0.220124,0.725509,0.466226}%
\pgfsetfillcolor{currentfill}%
\pgfsetlinewidth{0.000000pt}%
\definecolor{currentstroke}{rgb}{0.000000,0.000000,0.000000}%
\pgfsetstrokecolor{currentstroke}%
\pgfsetdash{}{0pt}%
\pgfpathmoveto{\pgfqpoint{1.393257in}{1.343394in}}%
\pgfpathlineto{\pgfqpoint{1.396549in}{1.335389in}}%
\pgfpathlineto{\pgfqpoint{1.399839in}{1.327325in}}%
\pgfpathlineto{\pgfqpoint{1.403127in}{1.319204in}}%
\pgfpathlineto{\pgfqpoint{1.406413in}{1.311028in}}%
\pgfpathlineto{\pgfqpoint{1.402066in}{1.307554in}}%
\pgfpathlineto{\pgfqpoint{1.397491in}{1.304146in}}%
\pgfpathlineto{\pgfqpoint{1.392692in}{1.300810in}}%
\pgfpathlineto{\pgfqpoint{1.387674in}{1.297548in}}%
\pgfpathlineto{\pgfqpoint{1.384654in}{1.305936in}}%
\pgfpathlineto{\pgfqpoint{1.381632in}{1.314268in}}%
\pgfpathlineto{\pgfqpoint{1.378608in}{1.322543in}}%
\pgfpathlineto{\pgfqpoint{1.375584in}{1.330758in}}%
\pgfpathlineto{\pgfqpoint{1.380315in}{1.333815in}}%
\pgfpathlineto{\pgfqpoint{1.384841in}{1.336943in}}%
\pgfpathlineto{\pgfqpoint{1.389156in}{1.340137in}}%
\pgfpathlineto{\pgfqpoint{1.393257in}{1.343394in}}%
\pgfpathclose%
\pgfusepath{fill}%
\end{pgfscope}%
\begin{pgfscope}%
\pgfpathrectangle{\pgfqpoint{0.041670in}{0.041670in}}{\pgfqpoint{2.216660in}{2.216660in}}%
\pgfusepath{clip}%
\pgfsetbuttcap%
\pgfsetroundjoin%
\definecolor{currentfill}{rgb}{0.993248,0.906157,0.143936}%
\pgfsetfillcolor{currentfill}%
\pgfsetlinewidth{0.000000pt}%
\definecolor{currentstroke}{rgb}{0.000000,0.000000,0.000000}%
\pgfsetstrokecolor{currentstroke}%
\pgfsetdash{}{0pt}%
\pgfpathmoveto{\pgfqpoint{1.178014in}{1.659927in}}%
\pgfpathlineto{\pgfqpoint{1.177530in}{1.661130in}}%
\pgfpathlineto{\pgfqpoint{1.177046in}{1.662198in}}%
\pgfpathlineto{\pgfqpoint{1.176562in}{1.663133in}}%
\pgfpathlineto{\pgfqpoint{1.176079in}{1.663935in}}%
\pgfpathlineto{\pgfqpoint{1.177058in}{1.663985in}}%
\pgfpathlineto{\pgfqpoint{1.178040in}{1.664020in}}%
\pgfpathlineto{\pgfqpoint{1.179024in}{1.664041in}}%
\pgfpathlineto{\pgfqpoint{1.180010in}{1.664048in}}%
\pgfpathlineto{\pgfqpoint{1.180003in}{1.663232in}}%
\pgfpathlineto{\pgfqpoint{1.179996in}{1.662283in}}%
\pgfpathlineto{\pgfqpoint{1.179989in}{1.661200in}}%
\pgfpathlineto{\pgfqpoint{1.179982in}{1.659984in}}%
\pgfpathlineto{\pgfqpoint{1.179489in}{1.659981in}}%
\pgfpathlineto{\pgfqpoint{1.178996in}{1.659970in}}%
\pgfpathlineto{\pgfqpoint{1.178505in}{1.659952in}}%
\pgfpathlineto{\pgfqpoint{1.178014in}{1.659927in}}%
\pgfpathclose%
\pgfusepath{fill}%
\end{pgfscope}%
\begin{pgfscope}%
\pgfpathrectangle{\pgfqpoint{0.041670in}{0.041670in}}{\pgfqpoint{2.216660in}{2.216660in}}%
\pgfusepath{clip}%
\pgfsetbuttcap%
\pgfsetroundjoin%
\definecolor{currentfill}{rgb}{0.993248,0.906157,0.143936}%
\pgfsetfillcolor{currentfill}%
\pgfsetlinewidth{0.000000pt}%
\definecolor{currentstroke}{rgb}{0.000000,0.000000,0.000000}%
\pgfsetstrokecolor{currentstroke}%
\pgfsetdash{}{0pt}%
\pgfpathmoveto{\pgfqpoint{1.179982in}{1.659984in}}%
\pgfpathlineto{\pgfqpoint{1.179989in}{1.661200in}}%
\pgfpathlineto{\pgfqpoint{1.179996in}{1.662283in}}%
\pgfpathlineto{\pgfqpoint{1.180003in}{1.663232in}}%
\pgfpathlineto{\pgfqpoint{1.180010in}{1.664048in}}%
\pgfpathlineto{\pgfqpoint{1.180995in}{1.664040in}}%
\pgfpathlineto{\pgfqpoint{1.181979in}{1.664017in}}%
\pgfpathlineto{\pgfqpoint{1.182961in}{1.663980in}}%
\pgfpathlineto{\pgfqpoint{1.183940in}{1.663929in}}%
\pgfpathlineto{\pgfqpoint{1.183443in}{1.663128in}}%
\pgfpathlineto{\pgfqpoint{1.182945in}{1.662194in}}%
\pgfpathlineto{\pgfqpoint{1.182448in}{1.661126in}}%
\pgfpathlineto{\pgfqpoint{1.181950in}{1.659924in}}%
\pgfpathlineto{\pgfqpoint{1.181460in}{1.659950in}}%
\pgfpathlineto{\pgfqpoint{1.180968in}{1.659968in}}%
\pgfpathlineto{\pgfqpoint{1.180475in}{1.659980in}}%
\pgfpathlineto{\pgfqpoint{1.179982in}{1.659984in}}%
\pgfpathclose%
\pgfusepath{fill}%
\end{pgfscope}%
\begin{pgfscope}%
\pgfpathrectangle{\pgfqpoint{0.041670in}{0.041670in}}{\pgfqpoint{2.216660in}{2.216660in}}%
\pgfusepath{clip}%
\pgfsetbuttcap%
\pgfsetroundjoin%
\definecolor{currentfill}{rgb}{0.993248,0.906157,0.143936}%
\pgfsetfillcolor{currentfill}%
\pgfsetlinewidth{0.000000pt}%
\definecolor{currentstroke}{rgb}{0.000000,0.000000,0.000000}%
\pgfsetstrokecolor{currentstroke}%
\pgfsetdash{}{0pt}%
\pgfpathmoveto{\pgfqpoint{1.176077in}{1.659756in}}%
\pgfpathlineto{\pgfqpoint{1.175109in}{1.660916in}}%
\pgfpathlineto{\pgfqpoint{1.174142in}{1.661942in}}%
\pgfpathlineto{\pgfqpoint{1.173175in}{1.662834in}}%
\pgfpathlineto{\pgfqpoint{1.172209in}{1.663594in}}%
\pgfpathlineto{\pgfqpoint{1.173167in}{1.663700in}}%
\pgfpathlineto{\pgfqpoint{1.174132in}{1.663793in}}%
\pgfpathlineto{\pgfqpoint{1.175103in}{1.663871in}}%
\pgfpathlineto{\pgfqpoint{1.176079in}{1.663935in}}%
\pgfpathlineto{\pgfqpoint{1.176562in}{1.663133in}}%
\pgfpathlineto{\pgfqpoint{1.177046in}{1.662198in}}%
\pgfpathlineto{\pgfqpoint{1.177530in}{1.661130in}}%
\pgfpathlineto{\pgfqpoint{1.178014in}{1.659927in}}%
\pgfpathlineto{\pgfqpoint{1.177526in}{1.659895in}}%
\pgfpathlineto{\pgfqpoint{1.177040in}{1.659856in}}%
\pgfpathlineto{\pgfqpoint{1.176557in}{1.659810in}}%
\pgfpathlineto{\pgfqpoint{1.176077in}{1.659756in}}%
\pgfpathclose%
\pgfusepath{fill}%
\end{pgfscope}%
\begin{pgfscope}%
\pgfpathrectangle{\pgfqpoint{0.041670in}{0.041670in}}{\pgfqpoint{2.216660in}{2.216660in}}%
\pgfusepath{clip}%
\pgfsetbuttcap%
\pgfsetroundjoin%
\definecolor{currentfill}{rgb}{0.993248,0.906157,0.143936}%
\pgfsetfillcolor{currentfill}%
\pgfsetlinewidth{0.000000pt}%
\definecolor{currentstroke}{rgb}{0.000000,0.000000,0.000000}%
\pgfsetstrokecolor{currentstroke}%
\pgfsetdash{}{0pt}%
\pgfpathmoveto{\pgfqpoint{1.181950in}{1.659924in}}%
\pgfpathlineto{\pgfqpoint{1.182448in}{1.661126in}}%
\pgfpathlineto{\pgfqpoint{1.182945in}{1.662194in}}%
\pgfpathlineto{\pgfqpoint{1.183443in}{1.663128in}}%
\pgfpathlineto{\pgfqpoint{1.183940in}{1.663929in}}%
\pgfpathlineto{\pgfqpoint{1.184915in}{1.663863in}}%
\pgfpathlineto{\pgfqpoint{1.185885in}{1.663783in}}%
\pgfpathlineto{\pgfqpoint{1.186849in}{1.663689in}}%
\pgfpathlineto{\pgfqpoint{1.187807in}{1.663581in}}%
\pgfpathlineto{\pgfqpoint{1.186828in}{1.662823in}}%
\pgfpathlineto{\pgfqpoint{1.185848in}{1.661932in}}%
\pgfpathlineto{\pgfqpoint{1.184867in}{1.660908in}}%
\pgfpathlineto{\pgfqpoint{1.183886in}{1.659750in}}%
\pgfpathlineto{\pgfqpoint{1.183406in}{1.659804in}}%
\pgfpathlineto{\pgfqpoint{1.182924in}{1.659851in}}%
\pgfpathlineto{\pgfqpoint{1.182438in}{1.659891in}}%
\pgfpathlineto{\pgfqpoint{1.181950in}{1.659924in}}%
\pgfpathclose%
\pgfusepath{fill}%
\end{pgfscope}%
\begin{pgfscope}%
\pgfpathrectangle{\pgfqpoint{0.041670in}{0.041670in}}{\pgfqpoint{2.216660in}{2.216660in}}%
\pgfusepath{clip}%
\pgfsetbuttcap%
\pgfsetroundjoin%
\definecolor{currentfill}{rgb}{0.935904,0.898570,0.108131}%
\pgfsetfillcolor{currentfill}%
\pgfsetlinewidth{0.000000pt}%
\definecolor{currentstroke}{rgb}{0.000000,0.000000,0.000000}%
\pgfsetstrokecolor{currentstroke}%
\pgfsetdash{}{0pt}%
\pgfpathmoveto{\pgfqpoint{1.241019in}{1.642888in}}%
\pgfpathlineto{\pgfqpoint{1.244825in}{1.641095in}}%
\pgfpathlineto{\pgfqpoint{1.248629in}{1.639174in}}%
\pgfpathlineto{\pgfqpoint{1.252432in}{1.637125in}}%
\pgfpathlineto{\pgfqpoint{1.256232in}{1.634950in}}%
\pgfpathlineto{\pgfqpoint{1.256725in}{1.633821in}}%
\pgfpathlineto{\pgfqpoint{1.257140in}{1.632685in}}%
\pgfpathlineto{\pgfqpoint{1.257479in}{1.631544in}}%
\pgfpathlineto{\pgfqpoint{1.257741in}{1.630398in}}%
\pgfpathlineto{\pgfqpoint{1.253863in}{1.632801in}}%
\pgfpathlineto{\pgfqpoint{1.249982in}{1.635078in}}%
\pgfpathlineto{\pgfqpoint{1.246101in}{1.637228in}}%
\pgfpathlineto{\pgfqpoint{1.242218in}{1.639249in}}%
\pgfpathlineto{\pgfqpoint{1.242010in}{1.640165in}}%
\pgfpathlineto{\pgfqpoint{1.241741in}{1.641078in}}%
\pgfpathlineto{\pgfqpoint{1.241410in}{1.641986in}}%
\pgfpathlineto{\pgfqpoint{1.241019in}{1.642888in}}%
\pgfpathclose%
\pgfusepath{fill}%
\end{pgfscope}%
\begin{pgfscope}%
\pgfpathrectangle{\pgfqpoint{0.041670in}{0.041670in}}{\pgfqpoint{2.216660in}{2.216660in}}%
\pgfusepath{clip}%
\pgfsetbuttcap%
\pgfsetroundjoin%
\definecolor{currentfill}{rgb}{0.974417,0.903590,0.130215}%
\pgfsetfillcolor{currentfill}%
\pgfsetlinewidth{0.000000pt}%
\definecolor{currentstroke}{rgb}{0.000000,0.000000,0.000000}%
\pgfsetstrokecolor{currentstroke}%
\pgfsetdash{}{0pt}%
\pgfpathmoveto{\pgfqpoint{1.205906in}{1.657607in}}%
\pgfpathlineto{\pgfqpoint{1.209143in}{1.657487in}}%
\pgfpathlineto{\pgfqpoint{1.212379in}{1.657235in}}%
\pgfpathlineto{\pgfqpoint{1.215614in}{1.656852in}}%
\pgfpathlineto{\pgfqpoint{1.218847in}{1.656337in}}%
\pgfpathlineto{\pgfqpoint{1.219654in}{1.655759in}}%
\pgfpathlineto{\pgfqpoint{1.220421in}{1.655170in}}%
\pgfpathlineto{\pgfqpoint{1.221148in}{1.654570in}}%
\pgfpathlineto{\pgfqpoint{1.221834in}{1.653959in}}%
\pgfpathlineto{\pgfqpoint{1.218352in}{1.654672in}}%
\pgfpathlineto{\pgfqpoint{1.214868in}{1.655253in}}%
\pgfpathlineto{\pgfqpoint{1.211383in}{1.655703in}}%
\pgfpathlineto{\pgfqpoint{1.207896in}{1.656021in}}%
\pgfpathlineto{\pgfqpoint{1.207439in}{1.656428in}}%
\pgfpathlineto{\pgfqpoint{1.206954in}{1.656828in}}%
\pgfpathlineto{\pgfqpoint{1.206443in}{1.657222in}}%
\pgfpathlineto{\pgfqpoint{1.205906in}{1.657607in}}%
\pgfpathclose%
\pgfusepath{fill}%
\end{pgfscope}%
\begin{pgfscope}%
\pgfpathrectangle{\pgfqpoint{0.041670in}{0.041670in}}{\pgfqpoint{2.216660in}{2.216660in}}%
\pgfusepath{clip}%
\pgfsetbuttcap%
\pgfsetroundjoin%
\definecolor{currentfill}{rgb}{0.993248,0.906157,0.143936}%
\pgfsetfillcolor{currentfill}%
\pgfsetlinewidth{0.000000pt}%
\definecolor{currentstroke}{rgb}{0.000000,0.000000,0.000000}%
\pgfsetstrokecolor{currentstroke}%
\pgfsetdash{}{0pt}%
\pgfpathmoveto{\pgfqpoint{1.183886in}{1.659750in}}%
\pgfpathlineto{\pgfqpoint{1.184867in}{1.660908in}}%
\pgfpathlineto{\pgfqpoint{1.185848in}{1.661932in}}%
\pgfpathlineto{\pgfqpoint{1.186828in}{1.662823in}}%
\pgfpathlineto{\pgfqpoint{1.187807in}{1.663581in}}%
\pgfpathlineto{\pgfqpoint{1.188757in}{1.663459in}}%
\pgfpathlineto{\pgfqpoint{1.189698in}{1.663322in}}%
\pgfpathlineto{\pgfqpoint{1.190629in}{1.663173in}}%
\pgfpathlineto{\pgfqpoint{1.189298in}{1.662466in}}%
\pgfpathlineto{\pgfqpoint{1.187965in}{1.661626in}}%
\pgfpathlineto{\pgfqpoint{1.186632in}{1.660652in}}%
\pgfpathlineto{\pgfqpoint{1.185298in}{1.659545in}}%
\pgfpathlineto{\pgfqpoint{1.184832in}{1.659620in}}%
\pgfpathlineto{\pgfqpoint{1.184361in}{1.659689in}}%
\pgfpathlineto{\pgfqpoint{1.183886in}{1.659750in}}%
\pgfpathclose%
\pgfusepath{fill}%
\end{pgfscope}%
\begin{pgfscope}%
\pgfpathrectangle{\pgfqpoint{0.041670in}{0.041670in}}{\pgfqpoint{2.216660in}{2.216660in}}%
\pgfusepath{clip}%
\pgfsetbuttcap%
\pgfsetroundjoin%
\definecolor{currentfill}{rgb}{0.993248,0.906157,0.143936}%
\pgfsetfillcolor{currentfill}%
\pgfsetlinewidth{0.000000pt}%
\definecolor{currentstroke}{rgb}{0.000000,0.000000,0.000000}%
\pgfsetstrokecolor{currentstroke}%
\pgfsetdash{}{0pt}%
\pgfpathmoveto{\pgfqpoint{1.174201in}{1.659473in}}%
\pgfpathlineto{\pgfqpoint{1.172765in}{1.660562in}}%
\pgfpathlineto{\pgfqpoint{1.171330in}{1.661517in}}%
\pgfpathlineto{\pgfqpoint{1.169895in}{1.662339in}}%
\pgfpathlineto{\pgfqpoint{1.168462in}{1.663028in}}%
\pgfpathlineto{\pgfqpoint{1.169384in}{1.663190in}}%
\pgfpathlineto{\pgfqpoint{1.170316in}{1.663338in}}%
\pgfpathlineto{\pgfqpoint{1.171258in}{1.663473in}}%
\pgfpathlineto{\pgfqpoint{1.172209in}{1.663594in}}%
\pgfpathlineto{\pgfqpoint{1.173175in}{1.662834in}}%
\pgfpathlineto{\pgfqpoint{1.174142in}{1.661942in}}%
\pgfpathlineto{\pgfqpoint{1.175109in}{1.660916in}}%
\pgfpathlineto{\pgfqpoint{1.176077in}{1.659756in}}%
\pgfpathlineto{\pgfqpoint{1.175601in}{1.659696in}}%
\pgfpathlineto{\pgfqpoint{1.175130in}{1.659628in}}%
\pgfpathlineto{\pgfqpoint{1.174663in}{1.659554in}}%
\pgfpathlineto{\pgfqpoint{1.174201in}{1.659473in}}%
\pgfpathclose%
\pgfusepath{fill}%
\end{pgfscope}%
\begin{pgfscope}%
\pgfpathrectangle{\pgfqpoint{0.041670in}{0.041670in}}{\pgfqpoint{2.216660in}{2.216660in}}%
\pgfusepath{clip}%
\pgfsetbuttcap%
\pgfsetroundjoin%
\definecolor{currentfill}{rgb}{0.267004,0.004874,0.329415}%
\pgfsetfillcolor{currentfill}%
\pgfsetlinewidth{0.000000pt}%
\definecolor{currentstroke}{rgb}{0.000000,0.000000,0.000000}%
\pgfsetstrokecolor{currentstroke}%
\pgfsetdash{}{0pt}%
\pgfpathmoveto{\pgfqpoint{1.536213in}{0.630246in}}%
\pgfpathlineto{\pgfqpoint{1.538624in}{0.627835in}}%
\pgfpathlineto{\pgfqpoint{1.541041in}{0.625665in}}%
\pgfpathlineto{\pgfqpoint{1.543463in}{0.623740in}}%
\pgfpathlineto{\pgfqpoint{1.545891in}{0.622065in}}%
\pgfpathlineto{\pgfqpoint{1.530657in}{0.615949in}}%
\pgfpathlineto{\pgfqpoint{1.515036in}{0.610092in}}%
\pgfpathlineto{\pgfqpoint{1.499044in}{0.604500in}}%
\pgfpathlineto{\pgfqpoint{1.482699in}{0.599181in}}%
\pgfpathlineto{\pgfqpoint{1.480685in}{0.601013in}}%
\pgfpathlineto{\pgfqpoint{1.478675in}{0.603096in}}%
\pgfpathlineto{\pgfqpoint{1.476669in}{0.605424in}}%
\pgfpathlineto{\pgfqpoint{1.474668in}{0.607994in}}%
\pgfpathlineto{\pgfqpoint{1.490586in}{0.613166in}}%
\pgfpathlineto{\pgfqpoint{1.506160in}{0.618603in}}%
\pgfpathlineto{\pgfqpoint{1.521374in}{0.624299in}}%
\pgfpathlineto{\pgfqpoint{1.536213in}{0.630246in}}%
\pgfpathclose%
\pgfusepath{fill}%
\end{pgfscope}%
\begin{pgfscope}%
\pgfpathrectangle{\pgfqpoint{0.041670in}{0.041670in}}{\pgfqpoint{2.216660in}{2.216660in}}%
\pgfusepath{clip}%
\pgfsetbuttcap%
\pgfsetroundjoin%
\definecolor{currentfill}{rgb}{0.212395,0.359683,0.551710}%
\pgfsetfillcolor{currentfill}%
\pgfsetlinewidth{0.000000pt}%
\definecolor{currentstroke}{rgb}{0.000000,0.000000,0.000000}%
\pgfsetstrokecolor{currentstroke}%
\pgfsetdash{}{0pt}%
\pgfpathmoveto{\pgfqpoint{0.988258in}{0.925985in}}%
\pgfpathlineto{\pgfqpoint{0.986436in}{0.916808in}}%
\pgfpathlineto{\pgfqpoint{0.984614in}{0.907681in}}%
\pgfpathlineto{\pgfqpoint{0.982793in}{0.898607in}}%
\pgfpathlineto{\pgfqpoint{0.980971in}{0.889589in}}%
\pgfpathlineto{\pgfqpoint{0.969424in}{0.892957in}}%
\pgfpathlineto{\pgfqpoint{0.958103in}{0.896511in}}%
\pgfpathlineto{\pgfqpoint{0.947021in}{0.900247in}}%
\pgfpathlineto{\pgfqpoint{0.936189in}{0.904161in}}%
\pgfpathlineto{\pgfqpoint{0.938416in}{0.913028in}}%
\pgfpathlineto{\pgfqpoint{0.940644in}{0.921951in}}%
\pgfpathlineto{\pgfqpoint{0.942871in}{0.930927in}}%
\pgfpathlineto{\pgfqpoint{0.945099in}{0.939953in}}%
\pgfpathlineto{\pgfqpoint{0.955540in}{0.936201in}}%
\pgfpathlineto{\pgfqpoint{0.966221in}{0.932619in}}%
\pgfpathlineto{\pgfqpoint{0.977130in}{0.929213in}}%
\pgfpathlineto{\pgfqpoint{0.988258in}{0.925985in}}%
\pgfpathclose%
\pgfusepath{fill}%
\end{pgfscope}%
\begin{pgfscope}%
\pgfpathrectangle{\pgfqpoint{0.041670in}{0.041670in}}{\pgfqpoint{2.216660in}{2.216660in}}%
\pgfusepath{clip}%
\pgfsetbuttcap%
\pgfsetroundjoin%
\definecolor{currentfill}{rgb}{0.272594,0.025563,0.353093}%
\pgfsetfillcolor{currentfill}%
\pgfsetlinewidth{0.000000pt}%
\definecolor{currentstroke}{rgb}{0.000000,0.000000,0.000000}%
\pgfsetstrokecolor{currentstroke}%
\pgfsetdash{}{0pt}%
\pgfpathmoveto{\pgfqpoint{1.625258in}{0.646653in}}%
\pgfpathlineto{\pgfqpoint{1.628119in}{0.647644in}}%
\pgfpathlineto{\pgfqpoint{1.630989in}{0.648938in}}%
\pgfpathlineto{\pgfqpoint{1.633867in}{0.650540in}}%
\pgfpathlineto{\pgfqpoint{1.636754in}{0.652456in}}%
\pgfpathlineto{\pgfqpoint{1.622162in}{0.644814in}}%
\pgfpathlineto{\pgfqpoint{1.607084in}{0.637416in}}%
\pgfpathlineto{\pgfqpoint{1.591534in}{0.630273in}}%
\pgfpathlineto{\pgfqpoint{1.575530in}{0.623392in}}%
\pgfpathlineto{\pgfqpoint{1.573020in}{0.621653in}}%
\pgfpathlineto{\pgfqpoint{1.570518in}{0.620229in}}%
\pgfpathlineto{\pgfqpoint{1.568023in}{0.619113in}}%
\pgfpathlineto{\pgfqpoint{1.565537in}{0.618301in}}%
\pgfpathlineto{\pgfqpoint{1.581146in}{0.625013in}}%
\pgfpathlineto{\pgfqpoint{1.596313in}{0.631982in}}%
\pgfpathlineto{\pgfqpoint{1.611022in}{0.639198in}}%
\pgfpathlineto{\pgfqpoint{1.625258in}{0.646653in}}%
\pgfpathclose%
\pgfusepath{fill}%
\end{pgfscope}%
\begin{pgfscope}%
\pgfpathrectangle{\pgfqpoint{0.041670in}{0.041670in}}{\pgfqpoint{2.216660in}{2.216660in}}%
\pgfusepath{clip}%
\pgfsetbuttcap%
\pgfsetroundjoin%
\definecolor{currentfill}{rgb}{0.955300,0.901065,0.118128}%
\pgfsetfillcolor{currentfill}%
\pgfsetlinewidth{0.000000pt}%
\definecolor{currentstroke}{rgb}{0.000000,0.000000,0.000000}%
\pgfsetstrokecolor{currentstroke}%
\pgfsetdash{}{0pt}%
\pgfpathmoveto{\pgfqpoint{1.224158in}{1.651423in}}%
\pgfpathlineto{\pgfqpoint{1.227833in}{1.650369in}}%
\pgfpathlineto{\pgfqpoint{1.231507in}{1.649184in}}%
\pgfpathlineto{\pgfqpoint{1.235179in}{1.647870in}}%
\pgfpathlineto{\pgfqpoint{1.238850in}{1.646426in}}%
\pgfpathlineto{\pgfqpoint{1.239481in}{1.645554in}}%
\pgfpathlineto{\pgfqpoint{1.240053in}{1.644673in}}%
\pgfpathlineto{\pgfqpoint{1.240566in}{1.643785in}}%
\pgfpathlineto{\pgfqpoint{1.241019in}{1.642888in}}%
\pgfpathlineto{\pgfqpoint{1.237211in}{1.644553in}}%
\pgfpathlineto{\pgfqpoint{1.233402in}{1.646089in}}%
\pgfpathlineto{\pgfqpoint{1.229591in}{1.647495in}}%
\pgfpathlineto{\pgfqpoint{1.225780in}{1.648770in}}%
\pgfpathlineto{\pgfqpoint{1.225442in}{1.649442in}}%
\pgfpathlineto{\pgfqpoint{1.225058in}{1.650109in}}%
\pgfpathlineto{\pgfqpoint{1.224630in}{1.650770in}}%
\pgfpathlineto{\pgfqpoint{1.224158in}{1.651423in}}%
\pgfpathclose%
\pgfusepath{fill}%
\end{pgfscope}%
\begin{pgfscope}%
\pgfpathrectangle{\pgfqpoint{0.041670in}{0.041670in}}{\pgfqpoint{2.216660in}{2.216660in}}%
\pgfusepath{clip}%
\pgfsetbuttcap%
\pgfsetroundjoin%
\definecolor{currentfill}{rgb}{0.699415,0.867117,0.175971}%
\pgfsetfillcolor{currentfill}%
\pgfsetlinewidth{0.000000pt}%
\definecolor{currentstroke}{rgb}{0.000000,0.000000,0.000000}%
\pgfsetstrokecolor{currentstroke}%
\pgfsetdash{}{0pt}%
\pgfpathmoveto{\pgfqpoint{1.046518in}{1.548817in}}%
\pgfpathlineto{\pgfqpoint{1.042829in}{1.543648in}}%
\pgfpathlineto{\pgfqpoint{1.039141in}{1.538371in}}%
\pgfpathlineto{\pgfqpoint{1.035456in}{1.532987in}}%
\pgfpathlineto{\pgfqpoint{1.031772in}{1.527496in}}%
\pgfpathlineto{\pgfqpoint{1.030337in}{1.529728in}}%
\pgfpathlineto{\pgfqpoint{1.029052in}{1.531980in}}%
\pgfpathlineto{\pgfqpoint{1.027919in}{1.534250in}}%
\pgfpathlineto{\pgfqpoint{1.026938in}{1.536534in}}%
\pgfpathlineto{\pgfqpoint{1.030737in}{1.541793in}}%
\pgfpathlineto{\pgfqpoint{1.034538in}{1.546946in}}%
\pgfpathlineto{\pgfqpoint{1.038342in}{1.551992in}}%
\pgfpathlineto{\pgfqpoint{1.042147in}{1.556930in}}%
\pgfpathlineto{\pgfqpoint{1.043034in}{1.554879in}}%
\pgfpathlineto{\pgfqpoint{1.044059in}{1.552842in}}%
\pgfpathlineto{\pgfqpoint{1.045221in}{1.550820in}}%
\pgfpathlineto{\pgfqpoint{1.046518in}{1.548817in}}%
\pgfpathclose%
\pgfusepath{fill}%
\end{pgfscope}%
\begin{pgfscope}%
\pgfpathrectangle{\pgfqpoint{0.041670in}{0.041670in}}{\pgfqpoint{2.216660in}{2.216660in}}%
\pgfusepath{clip}%
\pgfsetbuttcap%
\pgfsetroundjoin%
\definecolor{currentfill}{rgb}{0.974417,0.903590,0.130215}%
\pgfsetfillcolor{currentfill}%
\pgfsetlinewidth{0.000000pt}%
\definecolor{currentstroke}{rgb}{0.000000,0.000000,0.000000}%
\pgfsetstrokecolor{currentstroke}%
\pgfsetdash{}{0pt}%
\pgfpathmoveto{\pgfqpoint{1.151631in}{1.655653in}}%
\pgfpathlineto{\pgfqpoint{1.148096in}{1.655289in}}%
\pgfpathlineto{\pgfqpoint{1.144563in}{1.654794in}}%
\pgfpathlineto{\pgfqpoint{1.141031in}{1.654166in}}%
\pgfpathlineto{\pgfqpoint{1.137500in}{1.653408in}}%
\pgfpathlineto{\pgfqpoint{1.138150in}{1.654027in}}%
\pgfpathlineto{\pgfqpoint{1.138840in}{1.654637in}}%
\pgfpathlineto{\pgfqpoint{1.139572in}{1.655236in}}%
\pgfpathlineto{\pgfqpoint{1.140344in}{1.655824in}}%
\pgfpathlineto{\pgfqpoint{1.143637in}{1.656382in}}%
\pgfpathlineto{\pgfqpoint{1.146931in}{1.656808in}}%
\pgfpathlineto{\pgfqpoint{1.150227in}{1.657102in}}%
\pgfpathlineto{\pgfqpoint{1.153525in}{1.657265in}}%
\pgfpathlineto{\pgfqpoint{1.153011in}{1.656872in}}%
\pgfpathlineto{\pgfqpoint{1.152524in}{1.656473in}}%
\pgfpathlineto{\pgfqpoint{1.152064in}{1.656066in}}%
\pgfpathlineto{\pgfqpoint{1.151631in}{1.655653in}}%
\pgfpathclose%
\pgfusepath{fill}%
\end{pgfscope}%
\begin{pgfscope}%
\pgfpathrectangle{\pgfqpoint{0.041670in}{0.041670in}}{\pgfqpoint{2.216660in}{2.216660in}}%
\pgfusepath{clip}%
\pgfsetbuttcap%
\pgfsetroundjoin%
\definecolor{currentfill}{rgb}{0.993248,0.906157,0.143936}%
\pgfsetfillcolor{currentfill}%
\pgfsetlinewidth{0.000000pt}%
\definecolor{currentstroke}{rgb}{0.000000,0.000000,0.000000}%
\pgfsetstrokecolor{currentstroke}%
\pgfsetdash{}{0pt}%
\pgfpathmoveto{\pgfqpoint{1.185298in}{1.659545in}}%
\pgfpathlineto{\pgfqpoint{1.186632in}{1.660652in}}%
\pgfpathlineto{\pgfqpoint{1.187965in}{1.661626in}}%
\pgfpathlineto{\pgfqpoint{1.189298in}{1.662466in}}%
\pgfpathlineto{\pgfqpoint{1.190629in}{1.663173in}}%
\pgfpathlineto{\pgfqpoint{1.191550in}{1.663009in}}%
\pgfpathlineto{\pgfqpoint{1.192459in}{1.662832in}}%
\pgfpathlineto{\pgfqpoint{1.193356in}{1.662642in}}%
\pgfpathlineto{\pgfqpoint{1.194240in}{1.662439in}}%
\pgfpathlineto{\pgfqpoint{1.192458in}{1.661823in}}%
\pgfpathlineto{\pgfqpoint{1.190675in}{1.661075in}}%
\pgfpathlineto{\pgfqpoint{1.188891in}{1.660193in}}%
\pgfpathlineto{\pgfqpoint{1.187105in}{1.659178in}}%
\pgfpathlineto{\pgfqpoint{1.186663in}{1.659280in}}%
\pgfpathlineto{\pgfqpoint{1.186214in}{1.659375in}}%
\pgfpathlineto{\pgfqpoint{1.185759in}{1.659463in}}%
\pgfpathlineto{\pgfqpoint{1.185298in}{1.659545in}}%
\pgfpathclose%
\pgfusepath{fill}%
\end{pgfscope}%
\begin{pgfscope}%
\pgfpathrectangle{\pgfqpoint{0.041670in}{0.041670in}}{\pgfqpoint{2.216660in}{2.216660in}}%
\pgfusepath{clip}%
\pgfsetbuttcap%
\pgfsetroundjoin%
\definecolor{currentfill}{rgb}{0.814576,0.883393,0.110347}%
\pgfsetfillcolor{currentfill}%
\pgfsetlinewidth{0.000000pt}%
\definecolor{currentstroke}{rgb}{0.000000,0.000000,0.000000}%
\pgfsetstrokecolor{currentstroke}%
\pgfsetdash{}{0pt}%
\pgfpathmoveto{\pgfqpoint{1.072653in}{1.592371in}}%
\pgfpathlineto{\pgfqpoint{1.068833in}{1.588346in}}%
\pgfpathlineto{\pgfqpoint{1.065016in}{1.584204in}}%
\pgfpathlineto{\pgfqpoint{1.061200in}{1.579945in}}%
\pgfpathlineto{\pgfqpoint{1.057386in}{1.575569in}}%
\pgfpathlineto{\pgfqpoint{1.056715in}{1.577398in}}%
\pgfpathlineto{\pgfqpoint{1.056167in}{1.579236in}}%
\pgfpathlineto{\pgfqpoint{1.055744in}{1.581080in}}%
\pgfpathlineto{\pgfqpoint{1.055444in}{1.582930in}}%
\pgfpathlineto{\pgfqpoint{1.059314in}{1.587072in}}%
\pgfpathlineto{\pgfqpoint{1.063187in}{1.591098in}}%
\pgfpathlineto{\pgfqpoint{1.067061in}{1.595007in}}%
\pgfpathlineto{\pgfqpoint{1.070937in}{1.598799in}}%
\pgfpathlineto{\pgfqpoint{1.071203in}{1.597184in}}%
\pgfpathlineto{\pgfqpoint{1.071578in}{1.595573in}}%
\pgfpathlineto{\pgfqpoint{1.072061in}{1.593968in}}%
\pgfpathlineto{\pgfqpoint{1.072653in}{1.592371in}}%
\pgfpathclose%
\pgfusepath{fill}%
\end{pgfscope}%
\begin{pgfscope}%
\pgfpathrectangle{\pgfqpoint{0.041670in}{0.041670in}}{\pgfqpoint{2.216660in}{2.216660in}}%
\pgfusepath{clip}%
\pgfsetbuttcap%
\pgfsetroundjoin%
\definecolor{currentfill}{rgb}{0.195860,0.395433,0.555276}%
\pgfsetfillcolor{currentfill}%
\pgfsetlinewidth{0.000000pt}%
\definecolor{currentstroke}{rgb}{0.000000,0.000000,0.000000}%
\pgfsetstrokecolor{currentstroke}%
\pgfsetdash{}{0pt}%
\pgfpathmoveto{\pgfqpoint{1.414631in}{0.979816in}}%
\pgfpathlineto{\pgfqpoint{1.416944in}{0.970661in}}%
\pgfpathlineto{\pgfqpoint{1.419257in}{0.961542in}}%
\pgfpathlineto{\pgfqpoint{1.421569in}{0.952464in}}%
\pgfpathlineto{\pgfqpoint{1.423882in}{0.943429in}}%
\pgfpathlineto{\pgfqpoint{1.413663in}{0.939528in}}%
\pgfpathlineto{\pgfqpoint{1.403195in}{0.935794in}}%
\pgfpathlineto{\pgfqpoint{1.392488in}{0.932232in}}%
\pgfpathlineto{\pgfqpoint{1.381553in}{0.928845in}}%
\pgfpathlineto{\pgfqpoint{1.379638in}{0.938037in}}%
\pgfpathlineto{\pgfqpoint{1.377723in}{0.947272in}}%
\pgfpathlineto{\pgfqpoint{1.375808in}{0.956547in}}%
\pgfpathlineto{\pgfqpoint{1.373893in}{0.965858in}}%
\pgfpathlineto{\pgfqpoint{1.384416in}{0.969100in}}%
\pgfpathlineto{\pgfqpoint{1.394720in}{0.972509in}}%
\pgfpathlineto{\pgfqpoint{1.404795in}{0.976083in}}%
\pgfpathlineto{\pgfqpoint{1.414631in}{0.979816in}}%
\pgfpathclose%
\pgfusepath{fill}%
\end{pgfscope}%
\begin{pgfscope}%
\pgfpathrectangle{\pgfqpoint{0.041670in}{0.041670in}}{\pgfqpoint{2.216660in}{2.216660in}}%
\pgfusepath{clip}%
\pgfsetbuttcap%
\pgfsetroundjoin%
\definecolor{currentfill}{rgb}{0.283072,0.130895,0.449241}%
\pgfsetfillcolor{currentfill}%
\pgfsetlinewidth{0.000000pt}%
\definecolor{currentstroke}{rgb}{0.000000,0.000000,0.000000}%
\pgfsetstrokecolor{currentstroke}%
\pgfsetdash{}{0pt}%
\pgfpathmoveto{\pgfqpoint{1.427636in}{0.729973in}}%
\pgfpathlineto{\pgfqpoint{1.429568in}{0.722978in}}%
\pgfpathlineto{\pgfqpoint{1.431502in}{0.716120in}}%
\pgfpathlineto{\pgfqpoint{1.433438in}{0.709402in}}%
\pgfpathlineto{\pgfqpoint{1.435375in}{0.702827in}}%
\pgfpathlineto{\pgfqpoint{1.421276in}{0.698634in}}%
\pgfpathlineto{\pgfqpoint{1.406913in}{0.694679in}}%
\pgfpathlineto{\pgfqpoint{1.392301in}{0.690967in}}%
\pgfpathlineto{\pgfqpoint{1.377456in}{0.687502in}}%
\pgfpathlineto{\pgfqpoint{1.375955in}{0.694207in}}%
\pgfpathlineto{\pgfqpoint{1.374455in}{0.701055in}}%
\pgfpathlineto{\pgfqpoint{1.372957in}{0.708043in}}%
\pgfpathlineto{\pgfqpoint{1.371460in}{0.715167in}}%
\pgfpathlineto{\pgfqpoint{1.385857in}{0.718515in}}%
\pgfpathlineto{\pgfqpoint{1.400029in}{0.722101in}}%
\pgfpathlineto{\pgfqpoint{1.413960in}{0.725922in}}%
\pgfpathlineto{\pgfqpoint{1.427636in}{0.729973in}}%
\pgfpathclose%
\pgfusepath{fill}%
\end{pgfscope}%
\begin{pgfscope}%
\pgfpathrectangle{\pgfqpoint{0.041670in}{0.041670in}}{\pgfqpoint{2.216660in}{2.216660in}}%
\pgfusepath{clip}%
\pgfsetbuttcap%
\pgfsetroundjoin%
\definecolor{currentfill}{rgb}{0.993248,0.906157,0.143936}%
\pgfsetfillcolor{currentfill}%
\pgfsetlinewidth{0.000000pt}%
\definecolor{currentstroke}{rgb}{0.000000,0.000000,0.000000}%
\pgfsetstrokecolor{currentstroke}%
\pgfsetdash{}{0pt}%
\pgfpathmoveto{\pgfqpoint{1.172417in}{1.659082in}}%
\pgfpathlineto{\pgfqpoint{1.170535in}{1.660073in}}%
\pgfpathlineto{\pgfqpoint{1.168655in}{1.660931in}}%
\pgfpathlineto{\pgfqpoint{1.166775in}{1.661656in}}%
\pgfpathlineto{\pgfqpoint{1.164897in}{1.662247in}}%
\pgfpathlineto{\pgfqpoint{1.165768in}{1.662462in}}%
\pgfpathlineto{\pgfqpoint{1.166653in}{1.662664in}}%
\pgfpathlineto{\pgfqpoint{1.167551in}{1.662852in}}%
\pgfpathlineto{\pgfqpoint{1.168462in}{1.663028in}}%
\pgfpathlineto{\pgfqpoint{1.169895in}{1.662339in}}%
\pgfpathlineto{\pgfqpoint{1.171330in}{1.661517in}}%
\pgfpathlineto{\pgfqpoint{1.172765in}{1.660562in}}%
\pgfpathlineto{\pgfqpoint{1.174201in}{1.659473in}}%
\pgfpathlineto{\pgfqpoint{1.173746in}{1.659385in}}%
\pgfpathlineto{\pgfqpoint{1.173296in}{1.659290in}}%
\pgfpathlineto{\pgfqpoint{1.172853in}{1.659189in}}%
\pgfpathlineto{\pgfqpoint{1.172417in}{1.659082in}}%
\pgfpathclose%
\pgfusepath{fill}%
\end{pgfscope}%
\begin{pgfscope}%
\pgfpathrectangle{\pgfqpoint{0.041670in}{0.041670in}}{\pgfqpoint{2.216660in}{2.216660in}}%
\pgfusepath{clip}%
\pgfsetbuttcap%
\pgfsetroundjoin%
\definecolor{currentfill}{rgb}{0.896320,0.893616,0.096335}%
\pgfsetfillcolor{currentfill}%
\pgfsetlinewidth{0.000000pt}%
\definecolor{currentstroke}{rgb}{0.000000,0.000000,0.000000}%
\pgfsetstrokecolor{currentstroke}%
\pgfsetdash{}{0pt}%
\pgfpathmoveto{\pgfqpoint{1.257741in}{1.630398in}}%
\pgfpathlineto{\pgfqpoint{1.261618in}{1.627868in}}%
\pgfpathlineto{\pgfqpoint{1.265494in}{1.625213in}}%
\pgfpathlineto{\pgfqpoint{1.269367in}{1.622434in}}%
\pgfpathlineto{\pgfqpoint{1.273239in}{1.619532in}}%
\pgfpathlineto{\pgfqpoint{1.273464in}{1.618151in}}%
\pgfpathlineto{\pgfqpoint{1.273595in}{1.616767in}}%
\pgfpathlineto{\pgfqpoint{1.273633in}{1.615381in}}%
\pgfpathlineto{\pgfqpoint{1.273578in}{1.613996in}}%
\pgfpathlineto{\pgfqpoint{1.269689in}{1.617130in}}%
\pgfpathlineto{\pgfqpoint{1.265798in}{1.620142in}}%
\pgfpathlineto{\pgfqpoint{1.261906in}{1.623028in}}%
\pgfpathlineto{\pgfqpoint{1.258013in}{1.625790in}}%
\pgfpathlineto{\pgfqpoint{1.258062in}{1.626943in}}%
\pgfpathlineto{\pgfqpoint{1.258033in}{1.628096in}}%
\pgfpathlineto{\pgfqpoint{1.257926in}{1.629248in}}%
\pgfpathlineto{\pgfqpoint{1.257741in}{1.630398in}}%
\pgfpathclose%
\pgfusepath{fill}%
\end{pgfscope}%
\begin{pgfscope}%
\pgfpathrectangle{\pgfqpoint{0.041670in}{0.041670in}}{\pgfqpoint{2.216660in}{2.216660in}}%
\pgfusepath{clip}%
\pgfsetbuttcap%
\pgfsetroundjoin%
\definecolor{currentfill}{rgb}{0.280255,0.165693,0.476498}%
\pgfsetfillcolor{currentfill}%
\pgfsetlinewidth{0.000000pt}%
\definecolor{currentstroke}{rgb}{0.000000,0.000000,0.000000}%
\pgfsetstrokecolor{currentstroke}%
\pgfsetdash{}{0pt}%
\pgfpathmoveto{\pgfqpoint{1.419922in}{0.759231in}}%
\pgfpathlineto{\pgfqpoint{1.421848in}{0.751732in}}%
\pgfpathlineto{\pgfqpoint{1.423776in}{0.744353in}}%
\pgfpathlineto{\pgfqpoint{1.425705in}{0.737099in}}%
\pgfpathlineto{\pgfqpoint{1.427636in}{0.729973in}}%
\pgfpathlineto{\pgfqpoint{1.413960in}{0.725922in}}%
\pgfpathlineto{\pgfqpoint{1.400029in}{0.722101in}}%
\pgfpathlineto{\pgfqpoint{1.385857in}{0.718515in}}%
\pgfpathlineto{\pgfqpoint{1.371460in}{0.715167in}}%
\pgfpathlineto{\pgfqpoint{1.369964in}{0.722424in}}%
\pgfpathlineto{\pgfqpoint{1.368469in}{0.729808in}}%
\pgfpathlineto{\pgfqpoint{1.366976in}{0.737316in}}%
\pgfpathlineto{\pgfqpoint{1.365484in}{0.744945in}}%
\pgfpathlineto{\pgfqpoint{1.379434in}{0.748174in}}%
\pgfpathlineto{\pgfqpoint{1.393167in}{0.751635in}}%
\pgfpathlineto{\pgfqpoint{1.406668in}{0.755322in}}%
\pgfpathlineto{\pgfqpoint{1.419922in}{0.759231in}}%
\pgfpathclose%
\pgfusepath{fill}%
\end{pgfscope}%
\begin{pgfscope}%
\pgfpathrectangle{\pgfqpoint{0.041670in}{0.041670in}}{\pgfqpoint{2.216660in}{2.216660in}}%
\pgfusepath{clip}%
\pgfsetbuttcap%
\pgfsetroundjoin%
\definecolor{currentfill}{rgb}{0.935904,0.898570,0.108131}%
\pgfsetfillcolor{currentfill}%
\pgfsetlinewidth{0.000000pt}%
\definecolor{currentstroke}{rgb}{0.000000,0.000000,0.000000}%
\pgfsetstrokecolor{currentstroke}%
\pgfsetdash{}{0pt}%
\pgfpathmoveto{\pgfqpoint{1.117560in}{1.638433in}}%
\pgfpathlineto{\pgfqpoint{1.113668in}{1.636360in}}%
\pgfpathlineto{\pgfqpoint{1.109777in}{1.634159in}}%
\pgfpathlineto{\pgfqpoint{1.105888in}{1.631831in}}%
\pgfpathlineto{\pgfqpoint{1.102001in}{1.629376in}}%
\pgfpathlineto{\pgfqpoint{1.102194in}{1.630525in}}%
\pgfpathlineto{\pgfqpoint{1.102464in}{1.631671in}}%
\pgfpathlineto{\pgfqpoint{1.102812in}{1.632812in}}%
\pgfpathlineto{\pgfqpoint{1.103236in}{1.633947in}}%
\pgfpathlineto{\pgfqpoint{1.107060in}{1.636172in}}%
\pgfpathlineto{\pgfqpoint{1.110885in}{1.638271in}}%
\pgfpathlineto{\pgfqpoint{1.114711in}{1.640243in}}%
\pgfpathlineto{\pgfqpoint{1.118540in}{1.642086in}}%
\pgfpathlineto{\pgfqpoint{1.118202in}{1.641179in}}%
\pgfpathlineto{\pgfqpoint{1.117926in}{1.640267in}}%
\pgfpathlineto{\pgfqpoint{1.117712in}{1.639351in}}%
\pgfpathlineto{\pgfqpoint{1.117560in}{1.638433in}}%
\pgfpathclose%
\pgfusepath{fill}%
\end{pgfscope}%
\begin{pgfscope}%
\pgfpathrectangle{\pgfqpoint{0.041670in}{0.041670in}}{\pgfqpoint{2.216660in}{2.216660in}}%
\pgfusepath{clip}%
\pgfsetbuttcap%
\pgfsetroundjoin%
\definecolor{currentfill}{rgb}{0.282327,0.094955,0.417331}%
\pgfsetfillcolor{currentfill}%
\pgfsetlinewidth{0.000000pt}%
\definecolor{currentstroke}{rgb}{0.000000,0.000000,0.000000}%
\pgfsetstrokecolor{currentstroke}%
\pgfsetdash{}{0pt}%
\pgfpathmoveto{\pgfqpoint{0.724775in}{0.656403in}}%
\pgfpathlineto{\pgfqpoint{0.721919in}{0.659934in}}%
\pgfpathlineto{\pgfqpoint{0.719052in}{0.663814in}}%
\pgfpathlineto{\pgfqpoint{0.716174in}{0.668051in}}%
\pgfpathlineto{\pgfqpoint{0.713285in}{0.672649in}}%
\pgfpathlineto{\pgfqpoint{0.698026in}{0.680682in}}%
\pgfpathlineto{\pgfqpoint{0.683297in}{0.688960in}}%
\pgfpathlineto{\pgfqpoint{0.669113in}{0.697473in}}%
\pgfpathlineto{\pgfqpoint{0.655488in}{0.706211in}}%
\pgfpathlineto{\pgfqpoint{0.658720in}{0.701426in}}%
\pgfpathlineto{\pgfqpoint{0.661940in}{0.697003in}}%
\pgfpathlineto{\pgfqpoint{0.665148in}{0.692934in}}%
\pgfpathlineto{\pgfqpoint{0.668344in}{0.689214in}}%
\pgfpathlineto{\pgfqpoint{0.681650in}{0.680671in}}%
\pgfpathlineto{\pgfqpoint{0.695499in}{0.672348in}}%
\pgfpathlineto{\pgfqpoint{0.709879in}{0.664255in}}%
\pgfpathlineto{\pgfqpoint{0.724775in}{0.656403in}}%
\pgfpathclose%
\pgfusepath{fill}%
\end{pgfscope}%
\begin{pgfscope}%
\pgfpathrectangle{\pgfqpoint{0.041670in}{0.041670in}}{\pgfqpoint{2.216660in}{2.216660in}}%
\pgfusepath{clip}%
\pgfsetbuttcap%
\pgfsetroundjoin%
\definecolor{currentfill}{rgb}{0.955300,0.901065,0.118128}%
\pgfsetfillcolor{currentfill}%
\pgfsetlinewidth{0.000000pt}%
\definecolor{currentstroke}{rgb}{0.000000,0.000000,0.000000}%
\pgfsetstrokecolor{currentstroke}%
\pgfsetdash{}{0pt}%
\pgfpathmoveto{\pgfqpoint{1.133868in}{1.648169in}}%
\pgfpathlineto{\pgfqpoint{1.130034in}{1.646843in}}%
\pgfpathlineto{\pgfqpoint{1.126201in}{1.645387in}}%
\pgfpathlineto{\pgfqpoint{1.122370in}{1.643801in}}%
\pgfpathlineto{\pgfqpoint{1.118540in}{1.642086in}}%
\pgfpathlineto{\pgfqpoint{1.118938in}{1.642988in}}%
\pgfpathlineto{\pgfqpoint{1.119398in}{1.643884in}}%
\pgfpathlineto{\pgfqpoint{1.119917in}{1.644772in}}%
\pgfpathlineto{\pgfqpoint{1.120496in}{1.645652in}}%
\pgfpathlineto{\pgfqpoint{1.124202in}{1.647144in}}%
\pgfpathlineto{\pgfqpoint{1.127910in}{1.648506in}}%
\pgfpathlineto{\pgfqpoint{1.131619in}{1.649740in}}%
\pgfpathlineto{\pgfqpoint{1.135330in}{1.650843in}}%
\pgfpathlineto{\pgfqpoint{1.134897in}{1.650183in}}%
\pgfpathlineto{\pgfqpoint{1.134509in}{1.649517in}}%
\pgfpathlineto{\pgfqpoint{1.134166in}{1.648845in}}%
\pgfpathlineto{\pgfqpoint{1.133868in}{1.648169in}}%
\pgfpathclose%
\pgfusepath{fill}%
\end{pgfscope}%
\begin{pgfscope}%
\pgfpathrectangle{\pgfqpoint{0.041670in}{0.041670in}}{\pgfqpoint{2.216660in}{2.216660in}}%
\pgfusepath{clip}%
\pgfsetbuttcap%
\pgfsetroundjoin%
\definecolor{currentfill}{rgb}{0.993248,0.906157,0.143936}%
\pgfsetfillcolor{currentfill}%
\pgfsetlinewidth{0.000000pt}%
\definecolor{currentstroke}{rgb}{0.000000,0.000000,0.000000}%
\pgfsetstrokecolor{currentstroke}%
\pgfsetdash{}{0pt}%
\pgfpathmoveto{\pgfqpoint{1.187105in}{1.659178in}}%
\pgfpathlineto{\pgfqpoint{1.188891in}{1.660193in}}%
\pgfpathlineto{\pgfqpoint{1.190675in}{1.661075in}}%
\pgfpathlineto{\pgfqpoint{1.192458in}{1.661823in}}%
\pgfpathlineto{\pgfqpoint{1.194240in}{1.662439in}}%
\pgfpathlineto{\pgfqpoint{1.195109in}{1.662222in}}%
\pgfpathlineto{\pgfqpoint{1.195963in}{1.661993in}}%
\pgfpathlineto{\pgfqpoint{1.196802in}{1.661752in}}%
\pgfpathlineto{\pgfqpoint{1.197624in}{1.661499in}}%
\pgfpathlineto{\pgfqpoint{1.195420in}{1.661001in}}%
\pgfpathlineto{\pgfqpoint{1.193214in}{1.660369in}}%
\pgfpathlineto{\pgfqpoint{1.191007in}{1.659605in}}%
\pgfpathlineto{\pgfqpoint{1.188799in}{1.658707in}}%
\pgfpathlineto{\pgfqpoint{1.188388in}{1.658834in}}%
\pgfpathlineto{\pgfqpoint{1.187968in}{1.658955in}}%
\pgfpathlineto{\pgfqpoint{1.187541in}{1.659069in}}%
\pgfpathlineto{\pgfqpoint{1.187105in}{1.659178in}}%
\pgfpathclose%
\pgfusepath{fill}%
\end{pgfscope}%
\begin{pgfscope}%
\pgfpathrectangle{\pgfqpoint{0.041670in}{0.041670in}}{\pgfqpoint{2.216660in}{2.216660in}}%
\pgfusepath{clip}%
\pgfsetbuttcap%
\pgfsetroundjoin%
\definecolor{currentfill}{rgb}{0.282327,0.094955,0.417331}%
\pgfsetfillcolor{currentfill}%
\pgfsetlinewidth{0.000000pt}%
\definecolor{currentstroke}{rgb}{0.000000,0.000000,0.000000}%
\pgfsetstrokecolor{currentstroke}%
\pgfsetdash{}{0pt}%
\pgfpathmoveto{\pgfqpoint{1.435375in}{0.702827in}}%
\pgfpathlineto{\pgfqpoint{1.437315in}{0.696400in}}%
\pgfpathlineto{\pgfqpoint{1.439256in}{0.690125in}}%
\pgfpathlineto{\pgfqpoint{1.441200in}{0.684007in}}%
\pgfpathlineto{\pgfqpoint{1.443145in}{0.678048in}}%
\pgfpathlineto{\pgfqpoint{1.428622in}{0.673713in}}%
\pgfpathlineto{\pgfqpoint{1.413825in}{0.669624in}}%
\pgfpathlineto{\pgfqpoint{1.398772in}{0.665786in}}%
\pgfpathlineto{\pgfqpoint{1.383477in}{0.662203in}}%
\pgfpathlineto{\pgfqpoint{1.381969in}{0.668292in}}%
\pgfpathlineto{\pgfqpoint{1.380463in}{0.674540in}}%
\pgfpathlineto{\pgfqpoint{1.378959in}{0.680945in}}%
\pgfpathlineto{\pgfqpoint{1.377456in}{0.687502in}}%
\pgfpathlineto{\pgfqpoint{1.392301in}{0.690967in}}%
\pgfpathlineto{\pgfqpoint{1.406913in}{0.694679in}}%
\pgfpathlineto{\pgfqpoint{1.421276in}{0.698634in}}%
\pgfpathlineto{\pgfqpoint{1.435375in}{0.702827in}}%
\pgfpathclose%
\pgfusepath{fill}%
\end{pgfscope}%
\begin{pgfscope}%
\pgfpathrectangle{\pgfqpoint{0.041670in}{0.041670in}}{\pgfqpoint{2.216660in}{2.216660in}}%
\pgfusepath{clip}%
\pgfsetbuttcap%
\pgfsetroundjoin%
\definecolor{currentfill}{rgb}{0.412913,0.803041,0.357269}%
\pgfsetfillcolor{currentfill}%
\pgfsetlinewidth{0.000000pt}%
\definecolor{currentstroke}{rgb}{0.000000,0.000000,0.000000}%
\pgfsetstrokecolor{currentstroke}%
\pgfsetdash{}{0pt}%
\pgfpathmoveto{\pgfqpoint{1.365226in}{1.444962in}}%
\pgfpathlineto{\pgfqpoint{1.368755in}{1.438054in}}%
\pgfpathlineto{\pgfqpoint{1.372282in}{1.431063in}}%
\pgfpathlineto{\pgfqpoint{1.375808in}{1.423990in}}%
\pgfpathlineto{\pgfqpoint{1.379331in}{1.416837in}}%
\pgfpathlineto{\pgfqpoint{1.376509in}{1.413818in}}%
\pgfpathlineto{\pgfqpoint{1.373487in}{1.410842in}}%
\pgfpathlineto{\pgfqpoint{1.370268in}{1.407913in}}%
\pgfpathlineto{\pgfqpoint{1.366854in}{1.405032in}}%
\pgfpathlineto{\pgfqpoint{1.363546in}{1.412406in}}%
\pgfpathlineto{\pgfqpoint{1.360236in}{1.419700in}}%
\pgfpathlineto{\pgfqpoint{1.356924in}{1.426912in}}%
\pgfpathlineto{\pgfqpoint{1.353610in}{1.434040in}}%
\pgfpathlineto{\pgfqpoint{1.356787in}{1.436704in}}%
\pgfpathlineto{\pgfqpoint{1.359783in}{1.439415in}}%
\pgfpathlineto{\pgfqpoint{1.362597in}{1.442168in}}%
\pgfpathlineto{\pgfqpoint{1.365226in}{1.444962in}}%
\pgfpathclose%
\pgfusepath{fill}%
\end{pgfscope}%
\begin{pgfscope}%
\pgfpathrectangle{\pgfqpoint{0.041670in}{0.041670in}}{\pgfqpoint{2.216660in}{2.216660in}}%
\pgfusepath{clip}%
\pgfsetbuttcap%
\pgfsetroundjoin%
\definecolor{currentfill}{rgb}{0.993248,0.906157,0.143936}%
\pgfsetfillcolor{currentfill}%
\pgfsetlinewidth{0.000000pt}%
\definecolor{currentstroke}{rgb}{0.000000,0.000000,0.000000}%
\pgfsetstrokecolor{currentstroke}%
\pgfsetdash{}{0pt}%
\pgfpathmoveto{\pgfqpoint{1.170753in}{1.658589in}}%
\pgfpathlineto{\pgfqpoint{1.168455in}{1.659458in}}%
\pgfpathlineto{\pgfqpoint{1.166159in}{1.660193in}}%
\pgfpathlineto{\pgfqpoint{1.163864in}{1.660795in}}%
\pgfpathlineto{\pgfqpoint{1.161570in}{1.661263in}}%
\pgfpathlineto{\pgfqpoint{1.162376in}{1.661527in}}%
\pgfpathlineto{\pgfqpoint{1.163200in}{1.661780in}}%
\pgfpathlineto{\pgfqpoint{1.164040in}{1.662020in}}%
\pgfpathlineto{\pgfqpoint{1.164897in}{1.662247in}}%
\pgfpathlineto{\pgfqpoint{1.166775in}{1.661656in}}%
\pgfpathlineto{\pgfqpoint{1.168655in}{1.660931in}}%
\pgfpathlineto{\pgfqpoint{1.170535in}{1.660073in}}%
\pgfpathlineto{\pgfqpoint{1.172417in}{1.659082in}}%
\pgfpathlineto{\pgfqpoint{1.171989in}{1.658968in}}%
\pgfpathlineto{\pgfqpoint{1.171568in}{1.658848in}}%
\pgfpathlineto{\pgfqpoint{1.171156in}{1.658721in}}%
\pgfpathlineto{\pgfqpoint{1.170753in}{1.658589in}}%
\pgfpathclose%
\pgfusepath{fill}%
\end{pgfscope}%
\begin{pgfscope}%
\pgfpathrectangle{\pgfqpoint{0.041670in}{0.041670in}}{\pgfqpoint{2.216660in}{2.216660in}}%
\pgfusepath{clip}%
\pgfsetbuttcap%
\pgfsetroundjoin%
\definecolor{currentfill}{rgb}{0.120081,0.622161,0.534946}%
\pgfsetfillcolor{currentfill}%
\pgfsetlinewidth{0.000000pt}%
\definecolor{currentstroke}{rgb}{0.000000,0.000000,0.000000}%
\pgfsetstrokecolor{currentstroke}%
\pgfsetdash{}{0pt}%
\pgfpathmoveto{\pgfqpoint{0.978796in}{1.212087in}}%
\pgfpathlineto{\pgfqpoint{0.976181in}{1.203069in}}%
\pgfpathlineto{\pgfqpoint{0.973568in}{1.194020in}}%
\pgfpathlineto{\pgfqpoint{0.970955in}{1.184942in}}%
\pgfpathlineto{\pgfqpoint{0.968344in}{1.175837in}}%
\pgfpathlineto{\pgfqpoint{0.961295in}{1.179235in}}%
\pgfpathlineto{\pgfqpoint{0.954475in}{1.182740in}}%
\pgfpathlineto{\pgfqpoint{0.947890in}{1.186350in}}%
\pgfpathlineto{\pgfqpoint{0.941547in}{1.190059in}}%
\pgfpathlineto{\pgfqpoint{0.944483in}{1.198971in}}%
\pgfpathlineto{\pgfqpoint{0.947421in}{1.207856in}}%
\pgfpathlineto{\pgfqpoint{0.950360in}{1.216712in}}%
\pgfpathlineto{\pgfqpoint{0.953300in}{1.225537in}}%
\pgfpathlineto{\pgfqpoint{0.959336in}{1.222029in}}%
\pgfpathlineto{\pgfqpoint{0.965602in}{1.218615in}}%
\pgfpathlineto{\pgfqpoint{0.972091in}{1.215300in}}%
\pgfpathlineto{\pgfqpoint{0.978796in}{1.212087in}}%
\pgfpathclose%
\pgfusepath{fill}%
\end{pgfscope}%
\begin{pgfscope}%
\pgfpathrectangle{\pgfqpoint{0.041670in}{0.041670in}}{\pgfqpoint{2.216660in}{2.216660in}}%
\pgfusepath{clip}%
\pgfsetbuttcap%
\pgfsetroundjoin%
\definecolor{currentfill}{rgb}{0.274128,0.199721,0.498911}%
\pgfsetfillcolor{currentfill}%
\pgfsetlinewidth{0.000000pt}%
\definecolor{currentstroke}{rgb}{0.000000,0.000000,0.000000}%
\pgfsetstrokecolor{currentstroke}%
\pgfsetdash{}{0pt}%
\pgfpathmoveto{\pgfqpoint{1.412227in}{0.790355in}}%
\pgfpathlineto{\pgfqpoint{1.414149in}{0.782412in}}%
\pgfpathlineto{\pgfqpoint{1.416072in}{0.774575in}}%
\pgfpathlineto{\pgfqpoint{1.417996in}{0.766846in}}%
\pgfpathlineto{\pgfqpoint{1.419922in}{0.759231in}}%
\pgfpathlineto{\pgfqpoint{1.406668in}{0.755322in}}%
\pgfpathlineto{\pgfqpoint{1.393167in}{0.751635in}}%
\pgfpathlineto{\pgfqpoint{1.379434in}{0.748174in}}%
\pgfpathlineto{\pgfqpoint{1.365484in}{0.744945in}}%
\pgfpathlineto{\pgfqpoint{1.363992in}{0.752690in}}%
\pgfpathlineto{\pgfqpoint{1.362502in}{0.760548in}}%
\pgfpathlineto{\pgfqpoint{1.361012in}{0.768514in}}%
\pgfpathlineto{\pgfqpoint{1.359523in}{0.776587in}}%
\pgfpathlineto{\pgfqpoint{1.373028in}{0.779699in}}%
\pgfpathlineto{\pgfqpoint{1.386324in}{0.783034in}}%
\pgfpathlineto{\pgfqpoint{1.399394in}{0.786588in}}%
\pgfpathlineto{\pgfqpoint{1.412227in}{0.790355in}}%
\pgfpathclose%
\pgfusepath{fill}%
\end{pgfscope}%
\begin{pgfscope}%
\pgfpathrectangle{\pgfqpoint{0.041670in}{0.041670in}}{\pgfqpoint{2.216660in}{2.216660in}}%
\pgfusepath{clip}%
\pgfsetbuttcap%
\pgfsetroundjoin%
\definecolor{currentfill}{rgb}{0.565498,0.842430,0.262877}%
\pgfsetfillcolor{currentfill}%
\pgfsetlinewidth{0.000000pt}%
\definecolor{currentstroke}{rgb}{0.000000,0.000000,0.000000}%
\pgfsetstrokecolor{currentstroke}%
\pgfsetdash{}{0pt}%
\pgfpathmoveto{\pgfqpoint{1.024970in}{1.494914in}}%
\pgfpathlineto{\pgfqpoint{1.021469in}{1.488689in}}%
\pgfpathlineto{\pgfqpoint{1.017969in}{1.482366in}}%
\pgfpathlineto{\pgfqpoint{1.014472in}{1.475948in}}%
\pgfpathlineto{\pgfqpoint{1.010976in}{1.469435in}}%
\pgfpathlineto{\pgfqpoint{1.008562in}{1.472009in}}%
\pgfpathlineto{\pgfqpoint{1.006322in}{1.474618in}}%
\pgfpathlineto{\pgfqpoint{1.004257in}{1.477257in}}%
\pgfpathlineto{\pgfqpoint{1.002370in}{1.479926in}}%
\pgfpathlineto{\pgfqpoint{1.006038in}{1.486212in}}%
\pgfpathlineto{\pgfqpoint{1.009708in}{1.492404in}}%
\pgfpathlineto{\pgfqpoint{1.013380in}{1.498500in}}%
\pgfpathlineto{\pgfqpoint{1.017055in}{1.504500in}}%
\pgfpathlineto{\pgfqpoint{1.018791in}{1.502061in}}%
\pgfpathlineto{\pgfqpoint{1.020691in}{1.499649in}}%
\pgfpathlineto{\pgfqpoint{1.022751in}{1.497266in}}%
\pgfpathlineto{\pgfqpoint{1.024970in}{1.494914in}}%
\pgfpathclose%
\pgfusepath{fill}%
\end{pgfscope}%
\begin{pgfscope}%
\pgfpathrectangle{\pgfqpoint{0.041670in}{0.041670in}}{\pgfqpoint{2.216660in}{2.216660in}}%
\pgfusepath{clip}%
\pgfsetbuttcap%
\pgfsetroundjoin%
\definecolor{currentfill}{rgb}{0.147607,0.511733,0.557049}%
\pgfsetfillcolor{currentfill}%
\pgfsetlinewidth{0.000000pt}%
\definecolor{currentstroke}{rgb}{0.000000,0.000000,0.000000}%
\pgfsetstrokecolor{currentstroke}%
\pgfsetdash{}{0pt}%
\pgfpathmoveto{\pgfqpoint{0.980779in}{1.088550in}}%
\pgfpathlineto{\pgfqpoint{0.978545in}{1.079155in}}%
\pgfpathlineto{\pgfqpoint{0.976312in}{1.069760in}}%
\pgfpathlineto{\pgfqpoint{0.974079in}{1.060369in}}%
\pgfpathlineto{\pgfqpoint{0.971847in}{1.050984in}}%
\pgfpathlineto{\pgfqpoint{0.962804in}{1.054402in}}%
\pgfpathlineto{\pgfqpoint{0.953991in}{1.057963in}}%
\pgfpathlineto{\pgfqpoint{0.945416in}{1.061661in}}%
\pgfpathlineto{\pgfqpoint{0.937089in}{1.065494in}}%
\pgfpathlineto{\pgfqpoint{0.939688in}{1.074704in}}%
\pgfpathlineto{\pgfqpoint{0.942289in}{1.083922in}}%
\pgfpathlineto{\pgfqpoint{0.944890in}{1.093143in}}%
\pgfpathlineto{\pgfqpoint{0.947492in}{1.102365in}}%
\pgfpathlineto{\pgfqpoint{0.955467in}{1.098716in}}%
\pgfpathlineto{\pgfqpoint{0.963680in}{1.095195in}}%
\pgfpathlineto{\pgfqpoint{0.972120in}{1.091805in}}%
\pgfpathlineto{\pgfqpoint{0.980779in}{1.088550in}}%
\pgfpathclose%
\pgfusepath{fill}%
\end{pgfscope}%
\begin{pgfscope}%
\pgfpathrectangle{\pgfqpoint{0.041670in}{0.041670in}}{\pgfqpoint{2.216660in}{2.216660in}}%
\pgfusepath{clip}%
\pgfsetbuttcap%
\pgfsetroundjoin%
\definecolor{currentfill}{rgb}{0.896320,0.893616,0.096335}%
\pgfsetfillcolor{currentfill}%
\pgfsetlinewidth{0.000000pt}%
\definecolor{currentstroke}{rgb}{0.000000,0.000000,0.000000}%
\pgfsetstrokecolor{currentstroke}%
\pgfsetdash{}{0pt}%
\pgfpathmoveto{\pgfqpoint{1.102006in}{1.624765in}}%
\pgfpathlineto{\pgfqpoint{1.098117in}{1.621953in}}%
\pgfpathlineto{\pgfqpoint{1.094230in}{1.619014in}}%
\pgfpathlineto{\pgfqpoint{1.090344in}{1.615951in}}%
\pgfpathlineto{\pgfqpoint{1.086459in}{1.612765in}}%
\pgfpathlineto{\pgfqpoint{1.086321in}{1.614150in}}%
\pgfpathlineto{\pgfqpoint{1.086276in}{1.615535in}}%
\pgfpathlineto{\pgfqpoint{1.086325in}{1.616921in}}%
\pgfpathlineto{\pgfqpoint{1.086466in}{1.618304in}}%
\pgfpathlineto{\pgfqpoint{1.090347in}{1.621258in}}%
\pgfpathlineto{\pgfqpoint{1.094230in}{1.624089in}}%
\pgfpathlineto{\pgfqpoint{1.098115in}{1.626795in}}%
\pgfpathlineto{\pgfqpoint{1.102001in}{1.629376in}}%
\pgfpathlineto{\pgfqpoint{1.101885in}{1.628224in}}%
\pgfpathlineto{\pgfqpoint{1.101847in}{1.627071in}}%
\pgfpathlineto{\pgfqpoint{1.101888in}{1.625918in}}%
\pgfpathlineto{\pgfqpoint{1.102006in}{1.624765in}}%
\pgfpathclose%
\pgfusepath{fill}%
\end{pgfscope}%
\begin{pgfscope}%
\pgfpathrectangle{\pgfqpoint{0.041670in}{0.041670in}}{\pgfqpoint{2.216660in}{2.216660in}}%
\pgfusepath{clip}%
\pgfsetbuttcap%
\pgfsetroundjoin%
\definecolor{currentfill}{rgb}{0.993248,0.906157,0.143936}%
\pgfsetfillcolor{currentfill}%
\pgfsetlinewidth{0.000000pt}%
\definecolor{currentstroke}{rgb}{0.000000,0.000000,0.000000}%
\pgfsetstrokecolor{currentstroke}%
\pgfsetdash{}{0pt}%
\pgfpathmoveto{\pgfqpoint{1.188799in}{1.658707in}}%
\pgfpathlineto{\pgfqpoint{1.191007in}{1.659605in}}%
\pgfpathlineto{\pgfqpoint{1.193214in}{1.660369in}}%
\pgfpathlineto{\pgfqpoint{1.195420in}{1.661001in}}%
\pgfpathlineto{\pgfqpoint{1.197624in}{1.661499in}}%
\pgfpathlineto{\pgfqpoint{1.198428in}{1.661233in}}%
\pgfpathlineto{\pgfqpoint{1.199214in}{1.660956in}}%
\pgfpathlineto{\pgfqpoint{1.199981in}{1.660667in}}%
\pgfpathlineto{\pgfqpoint{1.200729in}{1.660368in}}%
\pgfpathlineto{\pgfqpoint{1.198136in}{1.660011in}}%
\pgfpathlineto{\pgfqpoint{1.195543in}{1.659521in}}%
\pgfpathlineto{\pgfqpoint{1.192948in}{1.658897in}}%
\pgfpathlineto{\pgfqpoint{1.190352in}{1.658141in}}%
\pgfpathlineto{\pgfqpoint{1.189978in}{1.658291in}}%
\pgfpathlineto{\pgfqpoint{1.189595in}{1.658435in}}%
\pgfpathlineto{\pgfqpoint{1.189201in}{1.658574in}}%
\pgfpathlineto{\pgfqpoint{1.188799in}{1.658707in}}%
\pgfpathclose%
\pgfusepath{fill}%
\end{pgfscope}%
\begin{pgfscope}%
\pgfpathrectangle{\pgfqpoint{0.041670in}{0.041670in}}{\pgfqpoint{2.216660in}{2.216660in}}%
\pgfusepath{clip}%
\pgfsetbuttcap%
\pgfsetroundjoin%
\definecolor{currentfill}{rgb}{0.993248,0.906157,0.143936}%
\pgfsetfillcolor{currentfill}%
\pgfsetlinewidth{0.000000pt}%
\definecolor{currentstroke}{rgb}{0.000000,0.000000,0.000000}%
\pgfsetstrokecolor{currentstroke}%
\pgfsetdash{}{0pt}%
\pgfpathmoveto{\pgfqpoint{1.169234in}{1.658003in}}%
\pgfpathlineto{\pgfqpoint{1.166557in}{1.658725in}}%
\pgfpathlineto{\pgfqpoint{1.163882in}{1.659314in}}%
\pgfpathlineto{\pgfqpoint{1.161207in}{1.659770in}}%
\pgfpathlineto{\pgfqpoint{1.158534in}{1.660092in}}%
\pgfpathlineto{\pgfqpoint{1.159263in}{1.660402in}}%
\pgfpathlineto{\pgfqpoint{1.160013in}{1.660700in}}%
\pgfpathlineto{\pgfqpoint{1.160782in}{1.660987in}}%
\pgfpathlineto{\pgfqpoint{1.161570in}{1.661263in}}%
\pgfpathlineto{\pgfqpoint{1.163864in}{1.660795in}}%
\pgfpathlineto{\pgfqpoint{1.166159in}{1.660193in}}%
\pgfpathlineto{\pgfqpoint{1.168455in}{1.659458in}}%
\pgfpathlineto{\pgfqpoint{1.170753in}{1.658589in}}%
\pgfpathlineto{\pgfqpoint{1.170358in}{1.658451in}}%
\pgfpathlineto{\pgfqpoint{1.169974in}{1.658307in}}%
\pgfpathlineto{\pgfqpoint{1.169599in}{1.658158in}}%
\pgfpathlineto{\pgfqpoint{1.169234in}{1.658003in}}%
\pgfpathclose%
\pgfusepath{fill}%
\end{pgfscope}%
\begin{pgfscope}%
\pgfpathrectangle{\pgfqpoint{0.041670in}{0.041670in}}{\pgfqpoint{2.216660in}{2.216660in}}%
\pgfusepath{clip}%
\pgfsetbuttcap%
\pgfsetroundjoin%
\definecolor{currentfill}{rgb}{0.974417,0.903590,0.130215}%
\pgfsetfillcolor{currentfill}%
\pgfsetlinewidth{0.000000pt}%
\definecolor{currentstroke}{rgb}{0.000000,0.000000,0.000000}%
\pgfsetstrokecolor{currentstroke}%
\pgfsetdash{}{0pt}%
\pgfpathmoveto{\pgfqpoint{1.207896in}{1.656021in}}%
\pgfpathlineto{\pgfqpoint{1.211383in}{1.655703in}}%
\pgfpathlineto{\pgfqpoint{1.214868in}{1.655253in}}%
\pgfpathlineto{\pgfqpoint{1.218352in}{1.654672in}}%
\pgfpathlineto{\pgfqpoint{1.221834in}{1.653959in}}%
\pgfpathlineto{\pgfqpoint{1.222479in}{1.653338in}}%
\pgfpathlineto{\pgfqpoint{1.223082in}{1.652708in}}%
\pgfpathlineto{\pgfqpoint{1.223641in}{1.652070in}}%
\pgfpathlineto{\pgfqpoint{1.224158in}{1.651423in}}%
\pgfpathlineto{\pgfqpoint{1.220481in}{1.652347in}}%
\pgfpathlineto{\pgfqpoint{1.216803in}{1.653140in}}%
\pgfpathlineto{\pgfqpoint{1.213123in}{1.653801in}}%
\pgfpathlineto{\pgfqpoint{1.209442in}{1.654330in}}%
\pgfpathlineto{\pgfqpoint{1.209098in}{1.654761in}}%
\pgfpathlineto{\pgfqpoint{1.208726in}{1.655187in}}%
\pgfpathlineto{\pgfqpoint{1.208325in}{1.655607in}}%
\pgfpathlineto{\pgfqpoint{1.207896in}{1.656021in}}%
\pgfpathclose%
\pgfusepath{fill}%
\end{pgfscope}%
\begin{pgfscope}%
\pgfpathrectangle{\pgfqpoint{0.041670in}{0.041670in}}{\pgfqpoint{2.216660in}{2.216660in}}%
\pgfusepath{clip}%
\pgfsetbuttcap%
\pgfsetroundjoin%
\definecolor{currentfill}{rgb}{0.220124,0.725509,0.466226}%
\pgfsetfillcolor{currentfill}%
\pgfsetlinewidth{0.000000pt}%
\definecolor{currentstroke}{rgb}{0.000000,0.000000,0.000000}%
\pgfsetstrokecolor{currentstroke}%
\pgfsetdash{}{0pt}%
\pgfpathmoveto{\pgfqpoint{0.988702in}{1.328102in}}%
\pgfpathlineto{\pgfqpoint{0.985743in}{1.319843in}}%
\pgfpathlineto{\pgfqpoint{0.982786in}{1.311523in}}%
\pgfpathlineto{\pgfqpoint{0.979831in}{1.303146in}}%
\pgfpathlineto{\pgfqpoint{0.976877in}{1.294714in}}%
\pgfpathlineto{\pgfqpoint{0.971668in}{1.297906in}}%
\pgfpathlineto{\pgfqpoint{0.966674in}{1.301177in}}%
\pgfpathlineto{\pgfqpoint{0.961899in}{1.304521in}}%
\pgfpathlineto{\pgfqpoint{0.957349in}{1.307936in}}%
\pgfpathlineto{\pgfqpoint{0.960581in}{1.316161in}}%
\pgfpathlineto{\pgfqpoint{0.963814in}{1.324330in}}%
\pgfpathlineto{\pgfqpoint{0.967050in}{1.332443in}}%
\pgfpathlineto{\pgfqpoint{0.970287in}{1.340495in}}%
\pgfpathlineto{\pgfqpoint{0.974579in}{1.337294in}}%
\pgfpathlineto{\pgfqpoint{0.979081in}{1.334159in}}%
\pgfpathlineto{\pgfqpoint{0.983790in}{1.331094in}}%
\pgfpathlineto{\pgfqpoint{0.988702in}{1.328102in}}%
\pgfpathclose%
\pgfusepath{fill}%
\end{pgfscope}%
\begin{pgfscope}%
\pgfpathrectangle{\pgfqpoint{0.041670in}{0.041670in}}{\pgfqpoint{2.216660in}{2.216660in}}%
\pgfusepath{clip}%
\pgfsetbuttcap%
\pgfsetroundjoin%
\definecolor{currentfill}{rgb}{0.762373,0.876424,0.137064}%
\pgfsetfillcolor{currentfill}%
\pgfsetlinewidth{0.000000pt}%
\definecolor{currentstroke}{rgb}{0.000000,0.000000,0.000000}%
\pgfsetstrokecolor{currentstroke}%
\pgfsetdash{}{0pt}%
\pgfpathmoveto{\pgfqpoint{1.303126in}{1.577194in}}%
\pgfpathlineto{\pgfqpoint{1.306957in}{1.572756in}}%
\pgfpathlineto{\pgfqpoint{1.310785in}{1.568203in}}%
\pgfpathlineto{\pgfqpoint{1.314612in}{1.563539in}}%
\pgfpathlineto{\pgfqpoint{1.318436in}{1.558763in}}%
\pgfpathlineto{\pgfqpoint{1.317671in}{1.556702in}}%
\pgfpathlineto{\pgfqpoint{1.316768in}{1.554652in}}%
\pgfpathlineto{\pgfqpoint{1.315728in}{1.552617in}}%
\pgfpathlineto{\pgfqpoint{1.314551in}{1.550597in}}%
\pgfpathlineto{\pgfqpoint{1.310830in}{1.555604in}}%
\pgfpathlineto{\pgfqpoint{1.307106in}{1.560500in}}%
\pgfpathlineto{\pgfqpoint{1.303381in}{1.565283in}}%
\pgfpathlineto{\pgfqpoint{1.299655in}{1.569953in}}%
\pgfpathlineto{\pgfqpoint{1.300705in}{1.571744in}}%
\pgfpathlineto{\pgfqpoint{1.301634in}{1.573549in}}%
\pgfpathlineto{\pgfqpoint{1.302442in}{1.575366in}}%
\pgfpathlineto{\pgfqpoint{1.303126in}{1.577194in}}%
\pgfpathclose%
\pgfusepath{fill}%
\end{pgfscope}%
\begin{pgfscope}%
\pgfpathrectangle{\pgfqpoint{0.041670in}{0.041670in}}{\pgfqpoint{2.216660in}{2.216660in}}%
\pgfusepath{clip}%
\pgfsetbuttcap%
\pgfsetroundjoin%
\definecolor{currentfill}{rgb}{0.279566,0.067836,0.391917}%
\pgfsetfillcolor{currentfill}%
\pgfsetlinewidth{0.000000pt}%
\definecolor{currentstroke}{rgb}{0.000000,0.000000,0.000000}%
\pgfsetstrokecolor{currentstroke}%
\pgfsetdash{}{0pt}%
\pgfpathmoveto{\pgfqpoint{1.443145in}{0.678048in}}%
\pgfpathlineto{\pgfqpoint{1.445094in}{0.672254in}}%
\pgfpathlineto{\pgfqpoint{1.447044in}{0.666628in}}%
\pgfpathlineto{\pgfqpoint{1.448997in}{0.661175in}}%
\pgfpathlineto{\pgfqpoint{1.450953in}{0.655898in}}%
\pgfpathlineto{\pgfqpoint{1.436003in}{0.651421in}}%
\pgfpathlineto{\pgfqpoint{1.420771in}{0.647198in}}%
\pgfpathlineto{\pgfqpoint{1.405273in}{0.643234in}}%
\pgfpathlineto{\pgfqpoint{1.389527in}{0.639534in}}%
\pgfpathlineto{\pgfqpoint{1.388012in}{0.644940in}}%
\pgfpathlineto{\pgfqpoint{1.386498in}{0.650523in}}%
\pgfpathlineto{\pgfqpoint{1.384987in}{0.656279in}}%
\pgfpathlineto{\pgfqpoint{1.383477in}{0.662203in}}%
\pgfpathlineto{\pgfqpoint{1.398772in}{0.665786in}}%
\pgfpathlineto{\pgfqpoint{1.413825in}{0.669624in}}%
\pgfpathlineto{\pgfqpoint{1.428622in}{0.673713in}}%
\pgfpathlineto{\pgfqpoint{1.443145in}{0.678048in}}%
\pgfpathclose%
\pgfusepath{fill}%
\end{pgfscope}%
\begin{pgfscope}%
\pgfpathrectangle{\pgfqpoint{0.041670in}{0.041670in}}{\pgfqpoint{2.216660in}{2.216660in}}%
\pgfusepath{clip}%
\pgfsetbuttcap%
\pgfsetroundjoin%
\definecolor{currentfill}{rgb}{0.974417,0.903590,0.130215}%
\pgfsetfillcolor{currentfill}%
\pgfsetlinewidth{0.000000pt}%
\definecolor{currentstroke}{rgb}{0.000000,0.000000,0.000000}%
\pgfsetstrokecolor{currentstroke}%
\pgfsetdash{}{0pt}%
\pgfpathmoveto{\pgfqpoint{1.150187in}{1.653942in}}%
\pgfpathlineto{\pgfqpoint{1.146471in}{1.653365in}}%
\pgfpathlineto{\pgfqpoint{1.142756in}{1.652656in}}%
\pgfpathlineto{\pgfqpoint{1.139042in}{1.651815in}}%
\pgfpathlineto{\pgfqpoint{1.135330in}{1.650843in}}%
\pgfpathlineto{\pgfqpoint{1.135807in}{1.651496in}}%
\pgfpathlineto{\pgfqpoint{1.136328in}{1.652141in}}%
\pgfpathlineto{\pgfqpoint{1.136893in}{1.652779in}}%
\pgfpathlineto{\pgfqpoint{1.137500in}{1.653408in}}%
\pgfpathlineto{\pgfqpoint{1.141031in}{1.654166in}}%
\pgfpathlineto{\pgfqpoint{1.144563in}{1.654794in}}%
\pgfpathlineto{\pgfqpoint{1.148096in}{1.655289in}}%
\pgfpathlineto{\pgfqpoint{1.151631in}{1.655653in}}%
\pgfpathlineto{\pgfqpoint{1.151227in}{1.655234in}}%
\pgfpathlineto{\pgfqpoint{1.150851in}{1.654808in}}%
\pgfpathlineto{\pgfqpoint{1.150505in}{1.654378in}}%
\pgfpathlineto{\pgfqpoint{1.150187in}{1.653942in}}%
\pgfpathclose%
\pgfusepath{fill}%
\end{pgfscope}%
\begin{pgfscope}%
\pgfpathrectangle{\pgfqpoint{0.041670in}{0.041670in}}{\pgfqpoint{2.216660in}{2.216660in}}%
\pgfusepath{clip}%
\pgfsetbuttcap%
\pgfsetroundjoin%
\definecolor{currentfill}{rgb}{0.855810,0.888601,0.097452}%
\pgfsetfillcolor{currentfill}%
\pgfsetlinewidth{0.000000pt}%
\definecolor{currentstroke}{rgb}{0.000000,0.000000,0.000000}%
\pgfsetstrokecolor{currentstroke}%
\pgfsetdash{}{0pt}%
\pgfpathmoveto{\pgfqpoint{1.273578in}{1.613996in}}%
\pgfpathlineto{\pgfqpoint{1.277466in}{1.610738in}}%
\pgfpathlineto{\pgfqpoint{1.281351in}{1.607358in}}%
\pgfpathlineto{\pgfqpoint{1.285235in}{1.603858in}}%
\pgfpathlineto{\pgfqpoint{1.289118in}{1.600237in}}%
\pgfpathlineto{\pgfqpoint{1.288948in}{1.598619in}}%
\pgfpathlineto{\pgfqpoint{1.288670in}{1.597004in}}%
\pgfpathlineto{\pgfqpoint{1.288283in}{1.595394in}}%
\pgfpathlineto{\pgfqpoint{1.287788in}{1.593790in}}%
\pgfpathlineto{\pgfqpoint{1.283949in}{1.597644in}}%
\pgfpathlineto{\pgfqpoint{1.280109in}{1.601377in}}%
\pgfpathlineto{\pgfqpoint{1.276268in}{1.604989in}}%
\pgfpathlineto{\pgfqpoint{1.272425in}{1.608479in}}%
\pgfpathlineto{\pgfqpoint{1.272852in}{1.609851in}}%
\pgfpathlineto{\pgfqpoint{1.273188in}{1.611229in}}%
\pgfpathlineto{\pgfqpoint{1.273429in}{1.612611in}}%
\pgfpathlineto{\pgfqpoint{1.273578in}{1.613996in}}%
\pgfpathclose%
\pgfusepath{fill}%
\end{pgfscope}%
\begin{pgfscope}%
\pgfpathrectangle{\pgfqpoint{0.041670in}{0.041670in}}{\pgfqpoint{2.216660in}{2.216660in}}%
\pgfusepath{clip}%
\pgfsetbuttcap%
\pgfsetroundjoin%
\definecolor{currentfill}{rgb}{0.993248,0.906157,0.143936}%
\pgfsetfillcolor{currentfill}%
\pgfsetlinewidth{0.000000pt}%
\definecolor{currentstroke}{rgb}{0.000000,0.000000,0.000000}%
\pgfsetstrokecolor{currentstroke}%
\pgfsetdash{}{0pt}%
\pgfpathmoveto{\pgfqpoint{1.190352in}{1.658141in}}%
\pgfpathlineto{\pgfqpoint{1.192948in}{1.658897in}}%
\pgfpathlineto{\pgfqpoint{1.195543in}{1.659521in}}%
\pgfpathlineto{\pgfqpoint{1.198136in}{1.660011in}}%
\pgfpathlineto{\pgfqpoint{1.200729in}{1.660368in}}%
\pgfpathlineto{\pgfqpoint{1.201455in}{1.660057in}}%
\pgfpathlineto{\pgfqpoint{1.202160in}{1.659736in}}%
\pgfpathlineto{\pgfqpoint{1.202843in}{1.659405in}}%
\pgfpathlineto{\pgfqpoint{1.203504in}{1.659064in}}%
\pgfpathlineto{\pgfqpoint{1.200565in}{1.658869in}}%
\pgfpathlineto{\pgfqpoint{1.197625in}{1.658542in}}%
\pgfpathlineto{\pgfqpoint{1.194683in}{1.658082in}}%
\pgfpathlineto{\pgfqpoint{1.191740in}{1.657488in}}%
\pgfpathlineto{\pgfqpoint{1.191410in}{1.657659in}}%
\pgfpathlineto{\pgfqpoint{1.191068in}{1.657825in}}%
\pgfpathlineto{\pgfqpoint{1.190715in}{1.657985in}}%
\pgfpathlineto{\pgfqpoint{1.190352in}{1.658141in}}%
\pgfpathclose%
\pgfusepath{fill}%
\end{pgfscope}%
\begin{pgfscope}%
\pgfpathrectangle{\pgfqpoint{0.041670in}{0.041670in}}{\pgfqpoint{2.216660in}{2.216660in}}%
\pgfusepath{clip}%
\pgfsetbuttcap%
\pgfsetroundjoin%
\definecolor{currentfill}{rgb}{0.263663,0.237631,0.518762}%
\pgfsetfillcolor{currentfill}%
\pgfsetlinewidth{0.000000pt}%
\definecolor{currentstroke}{rgb}{0.000000,0.000000,0.000000}%
\pgfsetstrokecolor{currentstroke}%
\pgfsetdash{}{0pt}%
\pgfpathmoveto{\pgfqpoint{1.404547in}{0.823107in}}%
\pgfpathlineto{\pgfqpoint{1.406466in}{0.814779in}}%
\pgfpathlineto{\pgfqpoint{1.408385in}{0.806542in}}%
\pgfpathlineto{\pgfqpoint{1.410306in}{0.798400in}}%
\pgfpathlineto{\pgfqpoint{1.412227in}{0.790355in}}%
\pgfpathlineto{\pgfqpoint{1.399394in}{0.786588in}}%
\pgfpathlineto{\pgfqpoint{1.386324in}{0.783034in}}%
\pgfpathlineto{\pgfqpoint{1.373028in}{0.779699in}}%
\pgfpathlineto{\pgfqpoint{1.359523in}{0.776587in}}%
\pgfpathlineto{\pgfqpoint{1.358035in}{0.784760in}}%
\pgfpathlineto{\pgfqpoint{1.356548in}{0.793032in}}%
\pgfpathlineto{\pgfqpoint{1.355061in}{0.801398in}}%
\pgfpathlineto{\pgfqpoint{1.353575in}{0.809854in}}%
\pgfpathlineto{\pgfqpoint{1.366635in}{0.812850in}}%
\pgfpathlineto{\pgfqpoint{1.379493in}{0.816060in}}%
\pgfpathlineto{\pgfqpoint{1.392135in}{0.819480in}}%
\pgfpathlineto{\pgfqpoint{1.404547in}{0.823107in}}%
\pgfpathclose%
\pgfusepath{fill}%
\end{pgfscope}%
\begin{pgfscope}%
\pgfpathrectangle{\pgfqpoint{0.041670in}{0.041670in}}{\pgfqpoint{2.216660in}{2.216660in}}%
\pgfusepath{clip}%
\pgfsetbuttcap%
\pgfsetroundjoin%
\definecolor{currentfill}{rgb}{0.133743,0.548535,0.553541}%
\pgfsetfillcolor{currentfill}%
\pgfsetlinewidth{0.000000pt}%
\definecolor{currentstroke}{rgb}{0.000000,0.000000,0.000000}%
\pgfsetstrokecolor{currentstroke}%
\pgfsetdash{}{0pt}%
\pgfpathmoveto{\pgfqpoint{1.408581in}{1.142386in}}%
\pgfpathlineto{\pgfqpoint{1.411263in}{1.133230in}}%
\pgfpathlineto{\pgfqpoint{1.413944in}{1.124064in}}%
\pgfpathlineto{\pgfqpoint{1.416624in}{1.114890in}}%
\pgfpathlineto{\pgfqpoint{1.419303in}{1.105712in}}%
\pgfpathlineto{\pgfqpoint{1.411544in}{1.101953in}}%
\pgfpathlineto{\pgfqpoint{1.403541in}{1.098319in}}%
\pgfpathlineto{\pgfqpoint{1.395304in}{1.094812in}}%
\pgfpathlineto{\pgfqpoint{1.386839in}{1.091437in}}%
\pgfpathlineto{\pgfqpoint{1.384519in}{1.100794in}}%
\pgfpathlineto{\pgfqpoint{1.382198in}{1.110146in}}%
\pgfpathlineto{\pgfqpoint{1.379876in}{1.119490in}}%
\pgfpathlineto{\pgfqpoint{1.377553in}{1.128823in}}%
\pgfpathlineto{\pgfqpoint{1.385643in}{1.132029in}}%
\pgfpathlineto{\pgfqpoint{1.393516in}{1.135361in}}%
\pgfpathlineto{\pgfqpoint{1.401165in}{1.138815in}}%
\pgfpathlineto{\pgfqpoint{1.408581in}{1.142386in}}%
\pgfpathclose%
\pgfusepath{fill}%
\end{pgfscope}%
\begin{pgfscope}%
\pgfpathrectangle{\pgfqpoint{0.041670in}{0.041670in}}{\pgfqpoint{2.216660in}{2.216660in}}%
\pgfusepath{clip}%
\pgfsetbuttcap%
\pgfsetroundjoin%
\definecolor{currentfill}{rgb}{0.993248,0.906157,0.143936}%
\pgfsetfillcolor{currentfill}%
\pgfsetlinewidth{0.000000pt}%
\definecolor{currentstroke}{rgb}{0.000000,0.000000,0.000000}%
\pgfsetstrokecolor{currentstroke}%
\pgfsetdash{}{0pt}%
\pgfpathmoveto{\pgfqpoint{1.167886in}{1.657332in}}%
\pgfpathlineto{\pgfqpoint{1.164872in}{1.657887in}}%
\pgfpathlineto{\pgfqpoint{1.161859in}{1.658308in}}%
\pgfpathlineto{\pgfqpoint{1.158848in}{1.658597in}}%
\pgfpathlineto{\pgfqpoint{1.155838in}{1.658752in}}%
\pgfpathlineto{\pgfqpoint{1.156478in}{1.659102in}}%
\pgfpathlineto{\pgfqpoint{1.157141in}{1.659442in}}%
\pgfpathlineto{\pgfqpoint{1.157827in}{1.659772in}}%
\pgfpathlineto{\pgfqpoint{1.158534in}{1.660092in}}%
\pgfpathlineto{\pgfqpoint{1.161207in}{1.659770in}}%
\pgfpathlineto{\pgfqpoint{1.163882in}{1.659314in}}%
\pgfpathlineto{\pgfqpoint{1.166557in}{1.658725in}}%
\pgfpathlineto{\pgfqpoint{1.169234in}{1.658003in}}%
\pgfpathlineto{\pgfqpoint{1.168880in}{1.657843in}}%
\pgfpathlineto{\pgfqpoint{1.168538in}{1.657677in}}%
\pgfpathlineto{\pgfqpoint{1.168206in}{1.657507in}}%
\pgfpathlineto{\pgfqpoint{1.167886in}{1.657332in}}%
\pgfpathclose%
\pgfusepath{fill}%
\end{pgfscope}%
\begin{pgfscope}%
\pgfpathrectangle{\pgfqpoint{0.041670in}{0.041670in}}{\pgfqpoint{2.216660in}{2.216660in}}%
\pgfusepath{clip}%
\pgfsetbuttcap%
\pgfsetroundjoin%
\definecolor{currentfill}{rgb}{0.134692,0.658636,0.517649}%
\pgfsetfillcolor{currentfill}%
\pgfsetlinewidth{0.000000pt}%
\definecolor{currentstroke}{rgb}{0.000000,0.000000,0.000000}%
\pgfsetstrokecolor{currentstroke}%
\pgfsetdash{}{0pt}%
\pgfpathmoveto{\pgfqpoint{1.399739in}{1.263489in}}%
\pgfpathlineto{\pgfqpoint{1.402751in}{1.254859in}}%
\pgfpathlineto{\pgfqpoint{1.405762in}{1.246188in}}%
\pgfpathlineto{\pgfqpoint{1.408771in}{1.237478in}}%
\pgfpathlineto{\pgfqpoint{1.411778in}{1.228732in}}%
\pgfpathlineto{\pgfqpoint{1.405951in}{1.225143in}}%
\pgfpathlineto{\pgfqpoint{1.399888in}{1.221645in}}%
\pgfpathlineto{\pgfqpoint{1.393598in}{1.218242in}}%
\pgfpathlineto{\pgfqpoint{1.387084in}{1.214938in}}%
\pgfpathlineto{\pgfqpoint{1.384392in}{1.223880in}}%
\pgfpathlineto{\pgfqpoint{1.381698in}{1.232786in}}%
\pgfpathlineto{\pgfqpoint{1.379003in}{1.241653in}}%
\pgfpathlineto{\pgfqpoint{1.376307in}{1.250478in}}%
\pgfpathlineto{\pgfqpoint{1.382486in}{1.253594in}}%
\pgfpathlineto{\pgfqpoint{1.388455in}{1.256803in}}%
\pgfpathlineto{\pgfqpoint{1.394208in}{1.260103in}}%
\pgfpathlineto{\pgfqpoint{1.399739in}{1.263489in}}%
\pgfpathclose%
\pgfusepath{fill}%
\end{pgfscope}%
\begin{pgfscope}%
\pgfpathrectangle{\pgfqpoint{0.041670in}{0.041670in}}{\pgfqpoint{2.216660in}{2.216660in}}%
\pgfusepath{clip}%
\pgfsetbuttcap%
\pgfsetroundjoin%
\definecolor{currentfill}{rgb}{0.195860,0.395433,0.555276}%
\pgfsetfillcolor{currentfill}%
\pgfsetlinewidth{0.000000pt}%
\definecolor{currentstroke}{rgb}{0.000000,0.000000,0.000000}%
\pgfsetstrokecolor{currentstroke}%
\pgfsetdash{}{0pt}%
\pgfpathmoveto{\pgfqpoint{0.995545in}{0.963121in}}%
\pgfpathlineto{\pgfqpoint{0.993723in}{0.953779in}}%
\pgfpathlineto{\pgfqpoint{0.991901in}{0.944473in}}%
\pgfpathlineto{\pgfqpoint{0.990080in}{0.935207in}}%
\pgfpathlineto{\pgfqpoint{0.988258in}{0.925985in}}%
\pgfpathlineto{\pgfqpoint{0.977130in}{0.929213in}}%
\pgfpathlineto{\pgfqpoint{0.966221in}{0.932619in}}%
\pgfpathlineto{\pgfqpoint{0.955540in}{0.936201in}}%
\pgfpathlineto{\pgfqpoint{0.945099in}{0.939953in}}%
\pgfpathlineto{\pgfqpoint{0.947326in}{0.949026in}}%
\pgfpathlineto{\pgfqpoint{0.949554in}{0.958141in}}%
\pgfpathlineto{\pgfqpoint{0.951781in}{0.967297in}}%
\pgfpathlineto{\pgfqpoint{0.954010in}{0.976490in}}%
\pgfpathlineto{\pgfqpoint{0.964059in}{0.972898in}}%
\pgfpathlineto{\pgfqpoint{0.974338in}{0.969470in}}%
\pgfpathlineto{\pgfqpoint{0.984837in}{0.966210in}}%
\pgfpathlineto{\pgfqpoint{0.995545in}{0.963121in}}%
\pgfpathclose%
\pgfusepath{fill}%
\end{pgfscope}%
\begin{pgfscope}%
\pgfpathrectangle{\pgfqpoint{0.041670in}{0.041670in}}{\pgfqpoint{2.216660in}{2.216660in}}%
\pgfusepath{clip}%
\pgfsetbuttcap%
\pgfsetroundjoin%
\definecolor{currentfill}{rgb}{0.283072,0.130895,0.449241}%
\pgfsetfillcolor{currentfill}%
\pgfsetlinewidth{0.000000pt}%
\definecolor{currentstroke}{rgb}{0.000000,0.000000,0.000000}%
\pgfsetstrokecolor{currentstroke}%
\pgfsetdash{}{0pt}%
\pgfpathmoveto{\pgfqpoint{1.001424in}{0.712396in}}%
\pgfpathlineto{\pgfqpoint{1.000028in}{0.705247in}}%
\pgfpathlineto{\pgfqpoint{0.998631in}{0.698235in}}%
\pgfpathlineto{\pgfqpoint{0.997232in}{0.691362in}}%
\pgfpathlineto{\pgfqpoint{0.995832in}{0.684633in}}%
\pgfpathlineto{\pgfqpoint{0.980793in}{0.687875in}}%
\pgfpathlineto{\pgfqpoint{0.965973in}{0.691367in}}%
\pgfpathlineto{\pgfqpoint{0.951388in}{0.695107in}}%
\pgfpathlineto{\pgfqpoint{0.937053in}{0.699088in}}%
\pgfpathlineto{\pgfqpoint{0.938897in}{0.705694in}}%
\pgfpathlineto{\pgfqpoint{0.940739in}{0.712445in}}%
\pgfpathlineto{\pgfqpoint{0.942579in}{0.719335in}}%
\pgfpathlineto{\pgfqpoint{0.944417in}{0.726360in}}%
\pgfpathlineto{\pgfqpoint{0.958321in}{0.722514in}}%
\pgfpathlineto{\pgfqpoint{0.972467in}{0.718901in}}%
\pgfpathlineto{\pgfqpoint{0.986840in}{0.715527in}}%
\pgfpathlineto{\pgfqpoint{1.001424in}{0.712396in}}%
\pgfpathclose%
\pgfusepath{fill}%
\end{pgfscope}%
\begin{pgfscope}%
\pgfpathrectangle{\pgfqpoint{0.041670in}{0.041670in}}{\pgfqpoint{2.216660in}{2.216660in}}%
\pgfusepath{clip}%
\pgfsetbuttcap%
\pgfsetroundjoin%
\definecolor{currentfill}{rgb}{0.636902,0.856542,0.216620}%
\pgfsetfillcolor{currentfill}%
\pgfsetlinewidth{0.000000pt}%
\definecolor{currentstroke}{rgb}{0.000000,0.000000,0.000000}%
\pgfsetstrokecolor{currentstroke}%
\pgfsetdash{}{0pt}%
\pgfpathmoveto{\pgfqpoint{1.329421in}{1.529479in}}%
\pgfpathlineto{\pgfqpoint{1.333134in}{1.523935in}}%
\pgfpathlineto{\pgfqpoint{1.336845in}{1.518287in}}%
\pgfpathlineto{\pgfqpoint{1.340554in}{1.512537in}}%
\pgfpathlineto{\pgfqpoint{1.344261in}{1.506688in}}%
\pgfpathlineto{\pgfqpoint{1.342670in}{1.504228in}}%
\pgfpathlineto{\pgfqpoint{1.340915in}{1.501792in}}%
\pgfpathlineto{\pgfqpoint{1.338998in}{1.499383in}}%
\pgfpathlineto{\pgfqpoint{1.336920in}{1.497003in}}%
\pgfpathlineto{\pgfqpoint{1.333374in}{1.503080in}}%
\pgfpathlineto{\pgfqpoint{1.329826in}{1.509056in}}%
\pgfpathlineto{\pgfqpoint{1.326276in}{1.514930in}}%
\pgfpathlineto{\pgfqpoint{1.322725in}{1.520701in}}%
\pgfpathlineto{\pgfqpoint{1.324620in}{1.522858in}}%
\pgfpathlineto{\pgfqpoint{1.326368in}{1.525041in}}%
\pgfpathlineto{\pgfqpoint{1.327969in}{1.527249in}}%
\pgfpathlineto{\pgfqpoint{1.329421in}{1.529479in}}%
\pgfpathclose%
\pgfusepath{fill}%
\end{pgfscope}%
\begin{pgfscope}%
\pgfpathrectangle{\pgfqpoint{0.041670in}{0.041670in}}{\pgfqpoint{2.216660in}{2.216660in}}%
\pgfusepath{clip}%
\pgfsetbuttcap%
\pgfsetroundjoin%
\definecolor{currentfill}{rgb}{0.280255,0.165693,0.476498}%
\pgfsetfillcolor{currentfill}%
\pgfsetlinewidth{0.000000pt}%
\definecolor{currentstroke}{rgb}{0.000000,0.000000,0.000000}%
\pgfsetstrokecolor{currentstroke}%
\pgfsetdash{}{0pt}%
\pgfpathmoveto{\pgfqpoint{1.006998in}{0.742270in}}%
\pgfpathlineto{\pgfqpoint{1.005606in}{0.734618in}}%
\pgfpathlineto{\pgfqpoint{1.004213in}{0.727085in}}%
\pgfpathlineto{\pgfqpoint{1.002819in}{0.719676in}}%
\pgfpathlineto{\pgfqpoint{1.001424in}{0.712396in}}%
\pgfpathlineto{\pgfqpoint{0.986840in}{0.715527in}}%
\pgfpathlineto{\pgfqpoint{0.972467in}{0.718901in}}%
\pgfpathlineto{\pgfqpoint{0.958321in}{0.722514in}}%
\pgfpathlineto{\pgfqpoint{0.944417in}{0.726360in}}%
\pgfpathlineto{\pgfqpoint{0.946254in}{0.733518in}}%
\pgfpathlineto{\pgfqpoint{0.948089in}{0.740804in}}%
\pgfpathlineto{\pgfqpoint{0.949924in}{0.748215in}}%
\pgfpathlineto{\pgfqpoint{0.951757in}{0.755745in}}%
\pgfpathlineto{\pgfqpoint{0.965231in}{0.752033in}}%
\pgfpathlineto{\pgfqpoint{0.978938in}{0.748547in}}%
\pgfpathlineto{\pgfqpoint{0.992866in}{0.745292in}}%
\pgfpathlineto{\pgfqpoint{1.006998in}{0.742270in}}%
\pgfpathclose%
\pgfusepath{fill}%
\end{pgfscope}%
\begin{pgfscope}%
\pgfpathrectangle{\pgfqpoint{0.041670in}{0.041670in}}{\pgfqpoint{2.216660in}{2.216660in}}%
\pgfusepath{clip}%
\pgfsetbuttcap%
\pgfsetroundjoin%
\definecolor{currentfill}{rgb}{0.955300,0.901065,0.118128}%
\pgfsetfillcolor{currentfill}%
\pgfsetlinewidth{0.000000pt}%
\definecolor{currentstroke}{rgb}{0.000000,0.000000,0.000000}%
\pgfsetstrokecolor{currentstroke}%
\pgfsetdash{}{0pt}%
\pgfpathmoveto{\pgfqpoint{1.225780in}{1.648770in}}%
\pgfpathlineto{\pgfqpoint{1.229591in}{1.647495in}}%
\pgfpathlineto{\pgfqpoint{1.233402in}{1.646089in}}%
\pgfpathlineto{\pgfqpoint{1.237211in}{1.644553in}}%
\pgfpathlineto{\pgfqpoint{1.241019in}{1.642888in}}%
\pgfpathlineto{\pgfqpoint{1.241410in}{1.641986in}}%
\pgfpathlineto{\pgfqpoint{1.241741in}{1.641078in}}%
\pgfpathlineto{\pgfqpoint{1.242010in}{1.640165in}}%
\pgfpathlineto{\pgfqpoint{1.242218in}{1.639249in}}%
\pgfpathlineto{\pgfqpoint{1.238333in}{1.641142in}}%
\pgfpathlineto{\pgfqpoint{1.234448in}{1.642905in}}%
\pgfpathlineto{\pgfqpoint{1.230561in}{1.644539in}}%
\pgfpathlineto{\pgfqpoint{1.226673in}{1.646042in}}%
\pgfpathlineto{\pgfqpoint{1.226519in}{1.646729in}}%
\pgfpathlineto{\pgfqpoint{1.226318in}{1.647413in}}%
\pgfpathlineto{\pgfqpoint{1.226072in}{1.648094in}}%
\pgfpathlineto{\pgfqpoint{1.225780in}{1.648770in}}%
\pgfpathclose%
\pgfusepath{fill}%
\end{pgfscope}%
\begin{pgfscope}%
\pgfpathrectangle{\pgfqpoint{0.041670in}{0.041670in}}{\pgfqpoint{2.216660in}{2.216660in}}%
\pgfusepath{clip}%
\pgfsetbuttcap%
\pgfsetroundjoin%
\definecolor{currentfill}{rgb}{0.412913,0.803041,0.357269}%
\pgfsetfillcolor{currentfill}%
\pgfsetlinewidth{0.000000pt}%
\definecolor{currentstroke}{rgb}{0.000000,0.000000,0.000000}%
\pgfsetstrokecolor{currentstroke}%
\pgfsetdash{}{0pt}%
\pgfpathmoveto{\pgfqpoint{1.009272in}{1.431712in}}%
\pgfpathlineto{\pgfqpoint{1.006014in}{1.424537in}}%
\pgfpathlineto{\pgfqpoint{1.002757in}{1.417278in}}%
\pgfpathlineto{\pgfqpoint{0.999503in}{1.409937in}}%
\pgfpathlineto{\pgfqpoint{0.996249in}{1.402516in}}%
\pgfpathlineto{\pgfqpoint{0.992667in}{1.405350in}}%
\pgfpathlineto{\pgfqpoint{0.989275in}{1.408236in}}%
\pgfpathlineto{\pgfqpoint{0.986077in}{1.411171in}}%
\pgfpathlineto{\pgfqpoint{0.983077in}{1.414152in}}%
\pgfpathlineto{\pgfqpoint{0.986558in}{1.421355in}}%
\pgfpathlineto{\pgfqpoint{0.990040in}{1.428478in}}%
\pgfpathlineto{\pgfqpoint{0.993525in}{1.435519in}}%
\pgfpathlineto{\pgfqpoint{0.997011in}{1.442477in}}%
\pgfpathlineto{\pgfqpoint{0.999805in}{1.439719in}}%
\pgfpathlineto{\pgfqpoint{1.002781in}{1.437003in}}%
\pgfpathlineto{\pgfqpoint{1.005938in}{1.434333in}}%
\pgfpathlineto{\pgfqpoint{1.009272in}{1.431712in}}%
\pgfpathclose%
\pgfusepath{fill}%
\end{pgfscope}%
\begin{pgfscope}%
\pgfpathrectangle{\pgfqpoint{0.041670in}{0.041670in}}{\pgfqpoint{2.216660in}{2.216660in}}%
\pgfusepath{clip}%
\pgfsetbuttcap%
\pgfsetroundjoin%
\definecolor{currentfill}{rgb}{0.993248,0.906157,0.143936}%
\pgfsetfillcolor{currentfill}%
\pgfsetlinewidth{0.000000pt}%
\definecolor{currentstroke}{rgb}{0.000000,0.000000,0.000000}%
\pgfsetstrokecolor{currentstroke}%
\pgfsetdash{}{0pt}%
\pgfpathmoveto{\pgfqpoint{1.191740in}{1.657488in}}%
\pgfpathlineto{\pgfqpoint{1.194683in}{1.658082in}}%
\pgfpathlineto{\pgfqpoint{1.197625in}{1.658542in}}%
\pgfpathlineto{\pgfqpoint{1.200565in}{1.658869in}}%
\pgfpathlineto{\pgfqpoint{1.203504in}{1.659064in}}%
\pgfpathlineto{\pgfqpoint{1.204141in}{1.658713in}}%
\pgfpathlineto{\pgfqpoint{1.204754in}{1.658353in}}%
\pgfpathlineto{\pgfqpoint{1.205342in}{1.657984in}}%
\pgfpathlineto{\pgfqpoint{1.205906in}{1.657607in}}%
\pgfpathlineto{\pgfqpoint{1.202666in}{1.657594in}}%
\pgfpathlineto{\pgfqpoint{1.199426in}{1.657449in}}%
\pgfpathlineto{\pgfqpoint{1.196184in}{1.657171in}}%
\pgfpathlineto{\pgfqpoint{1.192940in}{1.656759in}}%
\pgfpathlineto{\pgfqpoint{1.192659in}{1.656948in}}%
\pgfpathlineto{\pgfqpoint{1.192365in}{1.657132in}}%
\pgfpathlineto{\pgfqpoint{1.192058in}{1.657312in}}%
\pgfpathlineto{\pgfqpoint{1.191740in}{1.657488in}}%
\pgfpathclose%
\pgfusepath{fill}%
\end{pgfscope}%
\begin{pgfscope}%
\pgfpathrectangle{\pgfqpoint{0.041670in}{0.041670in}}{\pgfqpoint{2.216660in}{2.216660in}}%
\pgfusepath{clip}%
\pgfsetbuttcap%
\pgfsetroundjoin%
\definecolor{currentfill}{rgb}{0.855810,0.888601,0.097452}%
\pgfsetfillcolor{currentfill}%
\pgfsetlinewidth{0.000000pt}%
\definecolor{currentstroke}{rgb}{0.000000,0.000000,0.000000}%
\pgfsetstrokecolor{currentstroke}%
\pgfsetdash{}{0pt}%
\pgfpathmoveto{\pgfqpoint{1.087943in}{1.607264in}}%
\pgfpathlineto{\pgfqpoint{1.084118in}{1.603723in}}%
\pgfpathlineto{\pgfqpoint{1.080295in}{1.600060in}}%
\pgfpathlineto{\pgfqpoint{1.076473in}{1.596276in}}%
\pgfpathlineto{\pgfqpoint{1.072653in}{1.592371in}}%
\pgfpathlineto{\pgfqpoint{1.072061in}{1.593968in}}%
\pgfpathlineto{\pgfqpoint{1.071578in}{1.595573in}}%
\pgfpathlineto{\pgfqpoint{1.071203in}{1.597184in}}%
\pgfpathlineto{\pgfqpoint{1.070937in}{1.598799in}}%
\pgfpathlineto{\pgfqpoint{1.074815in}{1.602471in}}%
\pgfpathlineto{\pgfqpoint{1.078695in}{1.606024in}}%
\pgfpathlineto{\pgfqpoint{1.082576in}{1.609455in}}%
\pgfpathlineto{\pgfqpoint{1.086459in}{1.612765in}}%
\pgfpathlineto{\pgfqpoint{1.086691in}{1.611383in}}%
\pgfpathlineto{\pgfqpoint{1.087015in}{1.610004in}}%
\pgfpathlineto{\pgfqpoint{1.087433in}{1.608631in}}%
\pgfpathlineto{\pgfqpoint{1.087943in}{1.607264in}}%
\pgfpathclose%
\pgfusepath{fill}%
\end{pgfscope}%
\begin{pgfscope}%
\pgfpathrectangle{\pgfqpoint{0.041670in}{0.041670in}}{\pgfqpoint{2.216660in}{2.216660in}}%
\pgfusepath{clip}%
\pgfsetbuttcap%
\pgfsetroundjoin%
\definecolor{currentfill}{rgb}{0.935904,0.898570,0.108131}%
\pgfsetfillcolor{currentfill}%
\pgfsetlinewidth{0.000000pt}%
\definecolor{currentstroke}{rgb}{0.000000,0.000000,0.000000}%
\pgfsetstrokecolor{currentstroke}%
\pgfsetdash{}{0pt}%
\pgfpathmoveto{\pgfqpoint{1.242218in}{1.639249in}}%
\pgfpathlineto{\pgfqpoint{1.246101in}{1.637228in}}%
\pgfpathlineto{\pgfqpoint{1.249982in}{1.635078in}}%
\pgfpathlineto{\pgfqpoint{1.253863in}{1.632801in}}%
\pgfpathlineto{\pgfqpoint{1.257741in}{1.630398in}}%
\pgfpathlineto{\pgfqpoint{1.257926in}{1.629248in}}%
\pgfpathlineto{\pgfqpoint{1.258033in}{1.628096in}}%
\pgfpathlineto{\pgfqpoint{1.258062in}{1.626943in}}%
\pgfpathlineto{\pgfqpoint{1.258013in}{1.625790in}}%
\pgfpathlineto{\pgfqpoint{1.254118in}{1.628425in}}%
\pgfpathlineto{\pgfqpoint{1.250222in}{1.630933in}}%
\pgfpathlineto{\pgfqpoint{1.246325in}{1.633314in}}%
\pgfpathlineto{\pgfqpoint{1.242426in}{1.635566in}}%
\pgfpathlineto{\pgfqpoint{1.242468in}{1.636488in}}%
\pgfpathlineto{\pgfqpoint{1.242447in}{1.637410in}}%
\pgfpathlineto{\pgfqpoint{1.242363in}{1.638330in}}%
\pgfpathlineto{\pgfqpoint{1.242218in}{1.639249in}}%
\pgfpathclose%
\pgfusepath{fill}%
\end{pgfscope}%
\begin{pgfscope}%
\pgfpathrectangle{\pgfqpoint{0.041670in}{0.041670in}}{\pgfqpoint{2.216660in}{2.216660in}}%
\pgfusepath{clip}%
\pgfsetbuttcap%
\pgfsetroundjoin%
\definecolor{currentfill}{rgb}{0.179019,0.433756,0.557430}%
\pgfsetfillcolor{currentfill}%
\pgfsetlinewidth{0.000000pt}%
\definecolor{currentstroke}{rgb}{0.000000,0.000000,0.000000}%
\pgfsetstrokecolor{currentstroke}%
\pgfsetdash{}{0pt}%
\pgfpathmoveto{\pgfqpoint{1.405375in}{1.016745in}}%
\pgfpathlineto{\pgfqpoint{1.407690in}{1.007473in}}%
\pgfpathlineto{\pgfqpoint{1.410004in}{0.998225in}}%
\pgfpathlineto{\pgfqpoint{1.412318in}{0.989005in}}%
\pgfpathlineto{\pgfqpoint{1.414631in}{0.979816in}}%
\pgfpathlineto{\pgfqpoint{1.404795in}{0.976083in}}%
\pgfpathlineto{\pgfqpoint{1.394720in}{0.972509in}}%
\pgfpathlineto{\pgfqpoint{1.384416in}{0.969100in}}%
\pgfpathlineto{\pgfqpoint{1.373893in}{0.965858in}}%
\pgfpathlineto{\pgfqpoint{1.371977in}{0.975203in}}%
\pgfpathlineto{\pgfqpoint{1.370062in}{0.984578in}}%
\pgfpathlineto{\pgfqpoint{1.368146in}{0.993981in}}%
\pgfpathlineto{\pgfqpoint{1.366230in}{1.003408in}}%
\pgfpathlineto{\pgfqpoint{1.376340in}{1.006505in}}%
\pgfpathlineto{\pgfqpoint{1.386241in}{1.009763in}}%
\pgfpathlineto{\pgfqpoint{1.395923in}{1.013177in}}%
\pgfpathlineto{\pgfqpoint{1.405375in}{1.016745in}}%
\pgfpathclose%
\pgfusepath{fill}%
\end{pgfscope}%
\begin{pgfscope}%
\pgfpathrectangle{\pgfqpoint{0.041670in}{0.041670in}}{\pgfqpoint{2.216660in}{2.216660in}}%
\pgfusepath{clip}%
\pgfsetbuttcap%
\pgfsetroundjoin%
\definecolor{currentfill}{rgb}{0.281477,0.755203,0.432552}%
\pgfsetfillcolor{currentfill}%
\pgfsetlinewidth{0.000000pt}%
\definecolor{currentstroke}{rgb}{0.000000,0.000000,0.000000}%
\pgfsetstrokecolor{currentstroke}%
\pgfsetdash{}{0pt}%
\pgfpathmoveto{\pgfqpoint{1.380071in}{1.374774in}}%
\pgfpathlineto{\pgfqpoint{1.383370in}{1.367029in}}%
\pgfpathlineto{\pgfqpoint{1.386668in}{1.359216in}}%
\pgfpathlineto{\pgfqpoint{1.389963in}{1.351337in}}%
\pgfpathlineto{\pgfqpoint{1.393257in}{1.343394in}}%
\pgfpathlineto{\pgfqpoint{1.389156in}{1.340137in}}%
\pgfpathlineto{\pgfqpoint{1.384841in}{1.336943in}}%
\pgfpathlineto{\pgfqpoint{1.380315in}{1.333815in}}%
\pgfpathlineto{\pgfqpoint{1.375584in}{1.330758in}}%
\pgfpathlineto{\pgfqpoint{1.372557in}{1.338911in}}%
\pgfpathlineto{\pgfqpoint{1.369529in}{1.347000in}}%
\pgfpathlineto{\pgfqpoint{1.366500in}{1.355022in}}%
\pgfpathlineto{\pgfqpoint{1.363468in}{1.362976in}}%
\pgfpathlineto{\pgfqpoint{1.367912in}{1.365830in}}%
\pgfpathlineto{\pgfqpoint{1.372164in}{1.368750in}}%
\pgfpathlineto{\pgfqpoint{1.376218in}{1.371732in}}%
\pgfpathlineto{\pgfqpoint{1.380071in}{1.374774in}}%
\pgfpathclose%
\pgfusepath{fill}%
\end{pgfscope}%
\begin{pgfscope}%
\pgfpathrectangle{\pgfqpoint{0.041670in}{0.041670in}}{\pgfqpoint{2.216660in}{2.216660in}}%
\pgfusepath{clip}%
\pgfsetbuttcap%
\pgfsetroundjoin%
\definecolor{currentfill}{rgb}{0.282327,0.094955,0.417331}%
\pgfsetfillcolor{currentfill}%
\pgfsetlinewidth{0.000000pt}%
\definecolor{currentstroke}{rgb}{0.000000,0.000000,0.000000}%
\pgfsetstrokecolor{currentstroke}%
\pgfsetdash{}{0pt}%
\pgfpathmoveto{\pgfqpoint{0.995832in}{0.684633in}}%
\pgfpathlineto{\pgfqpoint{0.994430in}{0.678052in}}%
\pgfpathlineto{\pgfqpoint{0.993027in}{0.671623in}}%
\pgfpathlineto{\pgfqpoint{0.991623in}{0.665349in}}%
\pgfpathlineto{\pgfqpoint{0.990217in}{0.659236in}}%
\pgfpathlineto{\pgfqpoint{0.974722in}{0.662588in}}%
\pgfpathlineto{\pgfqpoint{0.959453in}{0.666199in}}%
\pgfpathlineto{\pgfqpoint{0.944427in}{0.670066in}}%
\pgfpathlineto{\pgfqpoint{0.929660in}{0.674182in}}%
\pgfpathlineto{\pgfqpoint{0.931512in}{0.680173in}}%
\pgfpathlineto{\pgfqpoint{0.933361in}{0.686323in}}%
\pgfpathlineto{\pgfqpoint{0.935208in}{0.692630in}}%
\pgfpathlineto{\pgfqpoint{0.937053in}{0.699088in}}%
\pgfpathlineto{\pgfqpoint{0.951388in}{0.695107in}}%
\pgfpathlineto{\pgfqpoint{0.965973in}{0.691367in}}%
\pgfpathlineto{\pgfqpoint{0.980793in}{0.687875in}}%
\pgfpathlineto{\pgfqpoint{0.995832in}{0.684633in}}%
\pgfpathclose%
\pgfusepath{fill}%
\end{pgfscope}%
\begin{pgfscope}%
\pgfpathrectangle{\pgfqpoint{0.041670in}{0.041670in}}{\pgfqpoint{2.216660in}{2.216660in}}%
\pgfusepath{clip}%
\pgfsetbuttcap%
\pgfsetroundjoin%
\definecolor{currentfill}{rgb}{0.993248,0.906157,0.143936}%
\pgfsetfillcolor{currentfill}%
\pgfsetlinewidth{0.000000pt}%
\definecolor{currentstroke}{rgb}{0.000000,0.000000,0.000000}%
\pgfsetstrokecolor{currentstroke}%
\pgfsetdash{}{0pt}%
\pgfpathmoveto{\pgfqpoint{1.166730in}{1.656588in}}%
\pgfpathlineto{\pgfqpoint{1.163427in}{1.656957in}}%
\pgfpathlineto{\pgfqpoint{1.160125in}{1.657192in}}%
\pgfpathlineto{\pgfqpoint{1.156824in}{1.657295in}}%
\pgfpathlineto{\pgfqpoint{1.153525in}{1.657265in}}%
\pgfpathlineto{\pgfqpoint{1.154065in}{1.657649in}}%
\pgfpathlineto{\pgfqpoint{1.154631in}{1.658026in}}%
\pgfpathlineto{\pgfqpoint{1.155223in}{1.658393in}}%
\pgfpathlineto{\pgfqpoint{1.155838in}{1.658752in}}%
\pgfpathlineto{\pgfqpoint{1.158848in}{1.658597in}}%
\pgfpathlineto{\pgfqpoint{1.161859in}{1.658308in}}%
\pgfpathlineto{\pgfqpoint{1.164872in}{1.657887in}}%
\pgfpathlineto{\pgfqpoint{1.167886in}{1.657332in}}%
\pgfpathlineto{\pgfqpoint{1.167578in}{1.657152in}}%
\pgfpathlineto{\pgfqpoint{1.167283in}{1.656968in}}%
\pgfpathlineto{\pgfqpoint{1.167000in}{1.656780in}}%
\pgfpathlineto{\pgfqpoint{1.166730in}{1.656588in}}%
\pgfpathclose%
\pgfusepath{fill}%
\end{pgfscope}%
\begin{pgfscope}%
\pgfpathrectangle{\pgfqpoint{0.041670in}{0.041670in}}{\pgfqpoint{2.216660in}{2.216660in}}%
\pgfusepath{clip}%
\pgfsetbuttcap%
\pgfsetroundjoin%
\definecolor{currentfill}{rgb}{0.762373,0.876424,0.137064}%
\pgfsetfillcolor{currentfill}%
\pgfsetlinewidth{0.000000pt}%
\definecolor{currentstroke}{rgb}{0.000000,0.000000,0.000000}%
\pgfsetstrokecolor{currentstroke}%
\pgfsetdash{}{0pt}%
\pgfpathmoveto{\pgfqpoint{1.061290in}{1.568374in}}%
\pgfpathlineto{\pgfqpoint{1.057595in}{1.563655in}}%
\pgfpathlineto{\pgfqpoint{1.053901in}{1.558821in}}%
\pgfpathlineto{\pgfqpoint{1.050209in}{1.553875in}}%
\pgfpathlineto{\pgfqpoint{1.046518in}{1.548817in}}%
\pgfpathlineto{\pgfqpoint{1.045221in}{1.550820in}}%
\pgfpathlineto{\pgfqpoint{1.044059in}{1.552842in}}%
\pgfpathlineto{\pgfqpoint{1.043034in}{1.554879in}}%
\pgfpathlineto{\pgfqpoint{1.042147in}{1.556930in}}%
\pgfpathlineto{\pgfqpoint{1.045954in}{1.561758in}}%
\pgfpathlineto{\pgfqpoint{1.049763in}{1.566474in}}%
\pgfpathlineto{\pgfqpoint{1.053573in}{1.571079in}}%
\pgfpathlineto{\pgfqpoint{1.057386in}{1.575569in}}%
\pgfpathlineto{\pgfqpoint{1.058180in}{1.573750in}}%
\pgfpathlineto{\pgfqpoint{1.059095in}{1.571944in}}%
\pgfpathlineto{\pgfqpoint{1.060132in}{1.570151in}}%
\pgfpathlineto{\pgfqpoint{1.061290in}{1.568374in}}%
\pgfpathclose%
\pgfusepath{fill}%
\end{pgfscope}%
\begin{pgfscope}%
\pgfpathrectangle{\pgfqpoint{0.041670in}{0.041670in}}{\pgfqpoint{2.216660in}{2.216660in}}%
\pgfusepath{clip}%
\pgfsetbuttcap%
\pgfsetroundjoin%
\definecolor{currentfill}{rgb}{0.274128,0.199721,0.498911}%
\pgfsetfillcolor{currentfill}%
\pgfsetlinewidth{0.000000pt}%
\definecolor{currentstroke}{rgb}{0.000000,0.000000,0.000000}%
\pgfsetstrokecolor{currentstroke}%
\pgfsetdash{}{0pt}%
\pgfpathmoveto{\pgfqpoint{1.012556in}{0.774010in}}%
\pgfpathlineto{\pgfqpoint{1.011168in}{0.765913in}}%
\pgfpathlineto{\pgfqpoint{1.009779in}{0.757922in}}%
\pgfpathlineto{\pgfqpoint{1.008389in}{0.750040in}}%
\pgfpathlineto{\pgfqpoint{1.006998in}{0.742270in}}%
\pgfpathlineto{\pgfqpoint{0.992866in}{0.745292in}}%
\pgfpathlineto{\pgfqpoint{0.978938in}{0.748547in}}%
\pgfpathlineto{\pgfqpoint{0.965231in}{0.752033in}}%
\pgfpathlineto{\pgfqpoint{0.951757in}{0.755745in}}%
\pgfpathlineto{\pgfqpoint{0.953588in}{0.763392in}}%
\pgfpathlineto{\pgfqpoint{0.955419in}{0.771152in}}%
\pgfpathlineto{\pgfqpoint{0.957249in}{0.779021in}}%
\pgfpathlineto{\pgfqpoint{0.959077in}{0.786996in}}%
\pgfpathlineto{\pgfqpoint{0.972122in}{0.783418in}}%
\pgfpathlineto{\pgfqpoint{0.985393in}{0.780059in}}%
\pgfpathlineto{\pgfqpoint{0.998876in}{0.776921in}}%
\pgfpathlineto{\pgfqpoint{1.012556in}{0.774010in}}%
\pgfpathclose%
\pgfusepath{fill}%
\end{pgfscope}%
\begin{pgfscope}%
\pgfpathrectangle{\pgfqpoint{0.041670in}{0.041670in}}{\pgfqpoint{2.216660in}{2.216660in}}%
\pgfusepath{clip}%
\pgfsetbuttcap%
\pgfsetroundjoin%
\definecolor{currentfill}{rgb}{0.955300,0.901065,0.118128}%
\pgfsetfillcolor{currentfill}%
\pgfsetlinewidth{0.000000pt}%
\definecolor{currentstroke}{rgb}{0.000000,0.000000,0.000000}%
\pgfsetstrokecolor{currentstroke}%
\pgfsetdash{}{0pt}%
\pgfpathmoveto{\pgfqpoint{1.133139in}{1.645430in}}%
\pgfpathlineto{\pgfqpoint{1.129243in}{1.643876in}}%
\pgfpathlineto{\pgfqpoint{1.125347in}{1.642191in}}%
\pgfpathlineto{\pgfqpoint{1.121453in}{1.640376in}}%
\pgfpathlineto{\pgfqpoint{1.117560in}{1.638433in}}%
\pgfpathlineto{\pgfqpoint{1.117712in}{1.639351in}}%
\pgfpathlineto{\pgfqpoint{1.117926in}{1.640267in}}%
\pgfpathlineto{\pgfqpoint{1.118202in}{1.641179in}}%
\pgfpathlineto{\pgfqpoint{1.118540in}{1.642086in}}%
\pgfpathlineto{\pgfqpoint{1.122370in}{1.643801in}}%
\pgfpathlineto{\pgfqpoint{1.126201in}{1.645387in}}%
\pgfpathlineto{\pgfqpoint{1.130034in}{1.646843in}}%
\pgfpathlineto{\pgfqpoint{1.133868in}{1.648169in}}%
\pgfpathlineto{\pgfqpoint{1.133616in}{1.647489in}}%
\pgfpathlineto{\pgfqpoint{1.133411in}{1.646805in}}%
\pgfpathlineto{\pgfqpoint{1.133252in}{1.646119in}}%
\pgfpathlineto{\pgfqpoint{1.133139in}{1.645430in}}%
\pgfpathclose%
\pgfusepath{fill}%
\end{pgfscope}%
\begin{pgfscope}%
\pgfpathrectangle{\pgfqpoint{0.041670in}{0.041670in}}{\pgfqpoint{2.216660in}{2.216660in}}%
\pgfusepath{clip}%
\pgfsetbuttcap%
\pgfsetroundjoin%
\definecolor{currentfill}{rgb}{0.974417,0.903590,0.130215}%
\pgfsetfillcolor{currentfill}%
\pgfsetlinewidth{0.000000pt}%
\definecolor{currentstroke}{rgb}{0.000000,0.000000,0.000000}%
\pgfsetstrokecolor{currentstroke}%
\pgfsetdash{}{0pt}%
\pgfpathmoveto{\pgfqpoint{1.209442in}{1.654330in}}%
\pgfpathlineto{\pgfqpoint{1.213123in}{1.653801in}}%
\pgfpathlineto{\pgfqpoint{1.216803in}{1.653140in}}%
\pgfpathlineto{\pgfqpoint{1.220481in}{1.652347in}}%
\pgfpathlineto{\pgfqpoint{1.224158in}{1.651423in}}%
\pgfpathlineto{\pgfqpoint{1.224630in}{1.650770in}}%
\pgfpathlineto{\pgfqpoint{1.225058in}{1.650109in}}%
\pgfpathlineto{\pgfqpoint{1.225442in}{1.649442in}}%
\pgfpathlineto{\pgfqpoint{1.225780in}{1.648770in}}%
\pgfpathlineto{\pgfqpoint{1.221966in}{1.649915in}}%
\pgfpathlineto{\pgfqpoint{1.218152in}{1.650929in}}%
\pgfpathlineto{\pgfqpoint{1.214336in}{1.651811in}}%
\pgfpathlineto{\pgfqpoint{1.210520in}{1.652561in}}%
\pgfpathlineto{\pgfqpoint{1.210295in}{1.653009in}}%
\pgfpathlineto{\pgfqpoint{1.210041in}{1.653453in}}%
\pgfpathlineto{\pgfqpoint{1.209756in}{1.653894in}}%
\pgfpathlineto{\pgfqpoint{1.209442in}{1.654330in}}%
\pgfpathclose%
\pgfusepath{fill}%
\end{pgfscope}%
\begin{pgfscope}%
\pgfpathrectangle{\pgfqpoint{0.041670in}{0.041670in}}{\pgfqpoint{2.216660in}{2.216660in}}%
\pgfusepath{clip}%
\pgfsetbuttcap%
\pgfsetroundjoin%
\definecolor{currentfill}{rgb}{0.267004,0.004874,0.329415}%
\pgfsetfillcolor{currentfill}%
\pgfsetlinewidth{0.000000pt}%
\definecolor{currentstroke}{rgb}{0.000000,0.000000,0.000000}%
\pgfsetstrokecolor{currentstroke}%
\pgfsetdash{}{0pt}%
\pgfpathmoveto{\pgfqpoint{0.899665in}{0.603624in}}%
\pgfpathlineto{\pgfqpoint{0.897760in}{0.601023in}}%
\pgfpathlineto{\pgfqpoint{0.895852in}{0.598664in}}%
\pgfpathlineto{\pgfqpoint{0.893940in}{0.596550in}}%
\pgfpathlineto{\pgfqpoint{0.892022in}{0.594687in}}%
\pgfpathlineto{\pgfqpoint{0.875378in}{0.599759in}}%
\pgfpathlineto{\pgfqpoint{0.859071in}{0.605108in}}%
\pgfpathlineto{\pgfqpoint{0.843120in}{0.610730in}}%
\pgfpathlineto{\pgfqpoint{0.827541in}{0.616616in}}%
\pgfpathlineto{\pgfqpoint{0.829881in}{0.618328in}}%
\pgfpathlineto{\pgfqpoint{0.832215in}{0.620290in}}%
\pgfpathlineto{\pgfqpoint{0.834544in}{0.622498in}}%
\pgfpathlineto{\pgfqpoint{0.836868in}{0.624947in}}%
\pgfpathlineto{\pgfqpoint{0.852041in}{0.619223in}}%
\pgfpathlineto{\pgfqpoint{0.867576in}{0.613757in}}%
\pgfpathlineto{\pgfqpoint{0.883456in}{0.608555in}}%
\pgfpathlineto{\pgfqpoint{0.899665in}{0.603624in}}%
\pgfpathclose%
\pgfusepath{fill}%
\end{pgfscope}%
\begin{pgfscope}%
\pgfpathrectangle{\pgfqpoint{0.041670in}{0.041670in}}{\pgfqpoint{2.216660in}{2.216660in}}%
\pgfusepath{clip}%
\pgfsetbuttcap%
\pgfsetroundjoin%
\definecolor{currentfill}{rgb}{0.248629,0.278775,0.534556}%
\pgfsetfillcolor{currentfill}%
\pgfsetlinewidth{0.000000pt}%
\definecolor{currentstroke}{rgb}{0.000000,0.000000,0.000000}%
\pgfsetstrokecolor{currentstroke}%
\pgfsetdash{}{0pt}%
\pgfpathmoveto{\pgfqpoint{1.396877in}{0.857255in}}%
\pgfpathlineto{\pgfqpoint{1.398794in}{0.848599in}}%
\pgfpathlineto{\pgfqpoint{1.400711in}{0.840021in}}%
\pgfpathlineto{\pgfqpoint{1.402629in}{0.831522in}}%
\pgfpathlineto{\pgfqpoint{1.404547in}{0.823107in}}%
\pgfpathlineto{\pgfqpoint{1.392135in}{0.819480in}}%
\pgfpathlineto{\pgfqpoint{1.379493in}{0.816060in}}%
\pgfpathlineto{\pgfqpoint{1.366635in}{0.812850in}}%
\pgfpathlineto{\pgfqpoint{1.353575in}{0.809854in}}%
\pgfpathlineto{\pgfqpoint{1.352089in}{0.818398in}}%
\pgfpathlineto{\pgfqpoint{1.350604in}{0.827025in}}%
\pgfpathlineto{\pgfqpoint{1.349120in}{0.835732in}}%
\pgfpathlineto{\pgfqpoint{1.347635in}{0.844515in}}%
\pgfpathlineto{\pgfqpoint{1.360252in}{0.847395in}}%
\pgfpathlineto{\pgfqpoint{1.372673in}{0.850480in}}%
\pgfpathlineto{\pgfqpoint{1.384886in}{0.853768in}}%
\pgfpathlineto{\pgfqpoint{1.396877in}{0.857255in}}%
\pgfpathclose%
\pgfusepath{fill}%
\end{pgfscope}%
\begin{pgfscope}%
\pgfpathrectangle{\pgfqpoint{0.041670in}{0.041670in}}{\pgfqpoint{2.216660in}{2.216660in}}%
\pgfusepath{clip}%
\pgfsetbuttcap%
\pgfsetroundjoin%
\definecolor{currentfill}{rgb}{0.274952,0.037752,0.364543}%
\pgfsetfillcolor{currentfill}%
\pgfsetlinewidth{0.000000pt}%
\definecolor{currentstroke}{rgb}{0.000000,0.000000,0.000000}%
\pgfsetstrokecolor{currentstroke}%
\pgfsetdash{}{0pt}%
\pgfpathmoveto{\pgfqpoint{1.450953in}{0.655898in}}%
\pgfpathlineto{\pgfqpoint{1.452911in}{0.650804in}}%
\pgfpathlineto{\pgfqpoint{1.454872in}{0.645894in}}%
\pgfpathlineto{\pgfqpoint{1.456836in}{0.641175in}}%
\pgfpathlineto{\pgfqpoint{1.458804in}{0.636650in}}%
\pgfpathlineto{\pgfqpoint{1.443425in}{0.632031in}}%
\pgfpathlineto{\pgfqpoint{1.427756in}{0.627674in}}%
\pgfpathlineto{\pgfqpoint{1.411813in}{0.623584in}}%
\pgfpathlineto{\pgfqpoint{1.395613in}{0.619766in}}%
\pgfpathlineto{\pgfqpoint{1.394088in}{0.624421in}}%
\pgfpathlineto{\pgfqpoint{1.392565in}{0.629270in}}%
\pgfpathlineto{\pgfqpoint{1.391045in}{0.634309in}}%
\pgfpathlineto{\pgfqpoint{1.389527in}{0.639534in}}%
\pgfpathlineto{\pgfqpoint{1.405273in}{0.643234in}}%
\pgfpathlineto{\pgfqpoint{1.420771in}{0.647198in}}%
\pgfpathlineto{\pgfqpoint{1.436003in}{0.651421in}}%
\pgfpathlineto{\pgfqpoint{1.450953in}{0.655898in}}%
\pgfpathclose%
\pgfusepath{fill}%
\end{pgfscope}%
\begin{pgfscope}%
\pgfpathrectangle{\pgfqpoint{0.041670in}{0.041670in}}{\pgfqpoint{2.216660in}{2.216660in}}%
\pgfusepath{clip}%
\pgfsetbuttcap%
\pgfsetroundjoin%
\definecolor{currentfill}{rgb}{0.974417,0.903590,0.130215}%
\pgfsetfillcolor{currentfill}%
\pgfsetlinewidth{0.000000pt}%
\definecolor{currentstroke}{rgb}{0.000000,0.000000,0.000000}%
\pgfsetstrokecolor{currentstroke}%
\pgfsetdash{}{0pt}%
\pgfpathmoveto{\pgfqpoint{1.149216in}{1.652160in}}%
\pgfpathlineto{\pgfqpoint{1.145378in}{1.651360in}}%
\pgfpathlineto{\pgfqpoint{1.141540in}{1.650428in}}%
\pgfpathlineto{\pgfqpoint{1.137703in}{1.649364in}}%
\pgfpathlineto{\pgfqpoint{1.133868in}{1.648169in}}%
\pgfpathlineto{\pgfqpoint{1.134166in}{1.648845in}}%
\pgfpathlineto{\pgfqpoint{1.134509in}{1.649517in}}%
\pgfpathlineto{\pgfqpoint{1.134897in}{1.650183in}}%
\pgfpathlineto{\pgfqpoint{1.135330in}{1.650843in}}%
\pgfpathlineto{\pgfqpoint{1.139042in}{1.651815in}}%
\pgfpathlineto{\pgfqpoint{1.142756in}{1.652656in}}%
\pgfpathlineto{\pgfqpoint{1.146471in}{1.653365in}}%
\pgfpathlineto{\pgfqpoint{1.150187in}{1.653942in}}%
\pgfpathlineto{\pgfqpoint{1.149899in}{1.653502in}}%
\pgfpathlineto{\pgfqpoint{1.149641in}{1.653059in}}%
\pgfpathlineto{\pgfqpoint{1.149414in}{1.652611in}}%
\pgfpathlineto{\pgfqpoint{1.149216in}{1.652160in}}%
\pgfpathclose%
\pgfusepath{fill}%
\end{pgfscope}%
\begin{pgfscope}%
\pgfpathrectangle{\pgfqpoint{0.041670in}{0.041670in}}{\pgfqpoint{2.216660in}{2.216660in}}%
\pgfusepath{clip}%
\pgfsetbuttcap%
\pgfsetroundjoin%
\definecolor{currentfill}{rgb}{0.935904,0.898570,0.108131}%
\pgfsetfillcolor{currentfill}%
\pgfsetlinewidth{0.000000pt}%
\definecolor{currentstroke}{rgb}{0.000000,0.000000,0.000000}%
\pgfsetstrokecolor{currentstroke}%
\pgfsetdash{}{0pt}%
\pgfpathmoveto{\pgfqpoint{1.117572in}{1.634748in}}%
\pgfpathlineto{\pgfqpoint{1.113679in}{1.632444in}}%
\pgfpathlineto{\pgfqpoint{1.109787in}{1.630012in}}%
\pgfpathlineto{\pgfqpoint{1.105896in}{1.627452in}}%
\pgfpathlineto{\pgfqpoint{1.102006in}{1.624765in}}%
\pgfpathlineto{\pgfqpoint{1.101888in}{1.625918in}}%
\pgfpathlineto{\pgfqpoint{1.101847in}{1.627071in}}%
\pgfpathlineto{\pgfqpoint{1.101885in}{1.628224in}}%
\pgfpathlineto{\pgfqpoint{1.102001in}{1.629376in}}%
\pgfpathlineto{\pgfqpoint{1.105888in}{1.631831in}}%
\pgfpathlineto{\pgfqpoint{1.109777in}{1.634159in}}%
\pgfpathlineto{\pgfqpoint{1.113668in}{1.636360in}}%
\pgfpathlineto{\pgfqpoint{1.117560in}{1.638433in}}%
\pgfpathlineto{\pgfqpoint{1.117469in}{1.637512in}}%
\pgfpathlineto{\pgfqpoint{1.117441in}{1.636590in}}%
\pgfpathlineto{\pgfqpoint{1.117476in}{1.635669in}}%
\pgfpathlineto{\pgfqpoint{1.117572in}{1.634748in}}%
\pgfpathclose%
\pgfusepath{fill}%
\end{pgfscope}%
\begin{pgfscope}%
\pgfpathrectangle{\pgfqpoint{0.041670in}{0.041670in}}{\pgfqpoint{2.216660in}{2.216660in}}%
\pgfusepath{clip}%
\pgfsetbuttcap%
\pgfsetroundjoin%
\definecolor{currentfill}{rgb}{0.993248,0.906157,0.143936}%
\pgfsetfillcolor{currentfill}%
\pgfsetlinewidth{0.000000pt}%
\definecolor{currentstroke}{rgb}{0.000000,0.000000,0.000000}%
\pgfsetstrokecolor{currentstroke}%
\pgfsetdash{}{0pt}%
\pgfpathmoveto{\pgfqpoint{1.192940in}{1.656759in}}%
\pgfpathlineto{\pgfqpoint{1.196184in}{1.657171in}}%
\pgfpathlineto{\pgfqpoint{1.199426in}{1.657449in}}%
\pgfpathlineto{\pgfqpoint{1.202666in}{1.657594in}}%
\pgfpathlineto{\pgfqpoint{1.205906in}{1.657607in}}%
\pgfpathlineto{\pgfqpoint{1.206443in}{1.657222in}}%
\pgfpathlineto{\pgfqpoint{1.206954in}{1.656828in}}%
\pgfpathlineto{\pgfqpoint{1.207439in}{1.656428in}}%
\pgfpathlineto{\pgfqpoint{1.207896in}{1.656021in}}%
\pgfpathlineto{\pgfqpoint{1.204407in}{1.656206in}}%
\pgfpathlineto{\pgfqpoint{1.200918in}{1.656259in}}%
\pgfpathlineto{\pgfqpoint{1.197427in}{1.656178in}}%
\pgfpathlineto{\pgfqpoint{1.193934in}{1.655965in}}%
\pgfpathlineto{\pgfqpoint{1.193706in}{1.656169in}}%
\pgfpathlineto{\pgfqpoint{1.193464in}{1.656369in}}%
\pgfpathlineto{\pgfqpoint{1.193209in}{1.656566in}}%
\pgfpathlineto{\pgfqpoint{1.192940in}{1.656759in}}%
\pgfpathclose%
\pgfusepath{fill}%
\end{pgfscope}%
\begin{pgfscope}%
\pgfpathrectangle{\pgfqpoint{0.041670in}{0.041670in}}{\pgfqpoint{2.216660in}{2.216660in}}%
\pgfusepath{clip}%
\pgfsetbuttcap%
\pgfsetroundjoin%
\definecolor{currentfill}{rgb}{0.993248,0.906157,0.143936}%
\pgfsetfillcolor{currentfill}%
\pgfsetlinewidth{0.000000pt}%
\definecolor{currentstroke}{rgb}{0.000000,0.000000,0.000000}%
\pgfsetstrokecolor{currentstroke}%
\pgfsetdash{}{0pt}%
\pgfpathmoveto{\pgfqpoint{1.165784in}{1.655781in}}%
\pgfpathlineto{\pgfqpoint{1.162244in}{1.655949in}}%
\pgfpathlineto{\pgfqpoint{1.158705in}{1.655983in}}%
\pgfpathlineto{\pgfqpoint{1.155167in}{1.655884in}}%
\pgfpathlineto{\pgfqpoint{1.151631in}{1.655653in}}%
\pgfpathlineto{\pgfqpoint{1.152064in}{1.656066in}}%
\pgfpathlineto{\pgfqpoint{1.152524in}{1.656473in}}%
\pgfpathlineto{\pgfqpoint{1.153011in}{1.656872in}}%
\pgfpathlineto{\pgfqpoint{1.153525in}{1.657265in}}%
\pgfpathlineto{\pgfqpoint{1.156824in}{1.657295in}}%
\pgfpathlineto{\pgfqpoint{1.160125in}{1.657192in}}%
\pgfpathlineto{\pgfqpoint{1.163427in}{1.656957in}}%
\pgfpathlineto{\pgfqpoint{1.166730in}{1.656588in}}%
\pgfpathlineto{\pgfqpoint{1.166473in}{1.656391in}}%
\pgfpathlineto{\pgfqpoint{1.166230in}{1.656191in}}%
\pgfpathlineto{\pgfqpoint{1.166000in}{1.655988in}}%
\pgfpathlineto{\pgfqpoint{1.165784in}{1.655781in}}%
\pgfpathclose%
\pgfusepath{fill}%
\end{pgfscope}%
\begin{pgfscope}%
\pgfpathrectangle{\pgfqpoint{0.041670in}{0.041670in}}{\pgfqpoint{2.216660in}{2.216660in}}%
\pgfusepath{clip}%
\pgfsetbuttcap%
\pgfsetroundjoin%
\definecolor{currentfill}{rgb}{0.487026,0.823929,0.312321}%
\pgfsetfillcolor{currentfill}%
\pgfsetlinewidth{0.000000pt}%
\definecolor{currentstroke}{rgb}{0.000000,0.000000,0.000000}%
\pgfsetstrokecolor{currentstroke}%
\pgfsetdash{}{0pt}%
\pgfpathmoveto{\pgfqpoint{1.351088in}{1.471722in}}%
\pgfpathlineto{\pgfqpoint{1.354625in}{1.465166in}}%
\pgfpathlineto{\pgfqpoint{1.358161in}{1.458520in}}%
\pgfpathlineto{\pgfqpoint{1.361694in}{1.451784in}}%
\pgfpathlineto{\pgfqpoint{1.365226in}{1.444962in}}%
\pgfpathlineto{\pgfqpoint{1.362597in}{1.442168in}}%
\pgfpathlineto{\pgfqpoint{1.359783in}{1.439415in}}%
\pgfpathlineto{\pgfqpoint{1.356787in}{1.436704in}}%
\pgfpathlineto{\pgfqpoint{1.353610in}{1.434040in}}%
\pgfpathlineto{\pgfqpoint{1.350295in}{1.441082in}}%
\pgfpathlineto{\pgfqpoint{1.346978in}{1.448036in}}%
\pgfpathlineto{\pgfqpoint{1.343659in}{1.454901in}}%
\pgfpathlineto{\pgfqpoint{1.340339in}{1.461675in}}%
\pgfpathlineto{\pgfqpoint{1.343278in}{1.464126in}}%
\pgfpathlineto{\pgfqpoint{1.346051in}{1.466619in}}%
\pgfpathlineto{\pgfqpoint{1.348655in}{1.469152in}}%
\pgfpathlineto{\pgfqpoint{1.351088in}{1.471722in}}%
\pgfpathclose%
\pgfusepath{fill}%
\end{pgfscope}%
\begin{pgfscope}%
\pgfpathrectangle{\pgfqpoint{0.041670in}{0.041670in}}{\pgfqpoint{2.216660in}{2.216660in}}%
\pgfusepath{clip}%
\pgfsetbuttcap%
\pgfsetroundjoin%
\definecolor{currentfill}{rgb}{0.263663,0.237631,0.518762}%
\pgfsetfillcolor{currentfill}%
\pgfsetlinewidth{0.000000pt}%
\definecolor{currentstroke}{rgb}{0.000000,0.000000,0.000000}%
\pgfsetstrokecolor{currentstroke}%
\pgfsetdash{}{0pt}%
\pgfpathmoveto{\pgfqpoint{1.018103in}{0.807374in}}%
\pgfpathlineto{\pgfqpoint{1.016717in}{0.798893in}}%
\pgfpathlineto{\pgfqpoint{1.015331in}{0.790503in}}%
\pgfpathlineto{\pgfqpoint{1.013944in}{0.782207in}}%
\pgfpathlineto{\pgfqpoint{1.012556in}{0.774010in}}%
\pgfpathlineto{\pgfqpoint{0.998876in}{0.776921in}}%
\pgfpathlineto{\pgfqpoint{0.985393in}{0.780059in}}%
\pgfpathlineto{\pgfqpoint{0.972122in}{0.783418in}}%
\pgfpathlineto{\pgfqpoint{0.959077in}{0.786996in}}%
\pgfpathlineto{\pgfqpoint{0.960905in}{0.795072in}}%
\pgfpathlineto{\pgfqpoint{0.962732in}{0.803246in}}%
\pgfpathlineto{\pgfqpoint{0.964558in}{0.811514in}}%
\pgfpathlineto{\pgfqpoint{0.966384in}{0.819873in}}%
\pgfpathlineto{\pgfqpoint{0.979001in}{0.816430in}}%
\pgfpathlineto{\pgfqpoint{0.991835in}{0.813196in}}%
\pgfpathlineto{\pgfqpoint{1.004874in}{0.810176in}}%
\pgfpathlineto{\pgfqpoint{1.018103in}{0.807374in}}%
\pgfpathclose%
\pgfusepath{fill}%
\end{pgfscope}%
\begin{pgfscope}%
\pgfpathrectangle{\pgfqpoint{0.041670in}{0.041670in}}{\pgfqpoint{2.216660in}{2.216660in}}%
\pgfusepath{clip}%
\pgfsetbuttcap%
\pgfsetroundjoin%
\definecolor{currentfill}{rgb}{0.636902,0.856542,0.216620}%
\pgfsetfillcolor{currentfill}%
\pgfsetlinewidth{0.000000pt}%
\definecolor{currentstroke}{rgb}{0.000000,0.000000,0.000000}%
\pgfsetstrokecolor{currentstroke}%
\pgfsetdash{}{0pt}%
\pgfpathmoveto{\pgfqpoint{1.038990in}{1.518808in}}%
\pgfpathlineto{\pgfqpoint{1.035483in}{1.512988in}}%
\pgfpathlineto{\pgfqpoint{1.031977in}{1.507065in}}%
\pgfpathlineto{\pgfqpoint{1.028472in}{1.501040in}}%
\pgfpathlineto{\pgfqpoint{1.024970in}{1.494914in}}%
\pgfpathlineto{\pgfqpoint{1.022751in}{1.497266in}}%
\pgfpathlineto{\pgfqpoint{1.020691in}{1.499649in}}%
\pgfpathlineto{\pgfqpoint{1.018791in}{1.502061in}}%
\pgfpathlineto{\pgfqpoint{1.017055in}{1.504500in}}%
\pgfpathlineto{\pgfqpoint{1.020731in}{1.510401in}}%
\pgfpathlineto{\pgfqpoint{1.024410in}{1.516201in}}%
\pgfpathlineto{\pgfqpoint{1.028090in}{1.521900in}}%
\pgfpathlineto{\pgfqpoint{1.031772in}{1.527496in}}%
\pgfpathlineto{\pgfqpoint{1.033356in}{1.525285in}}%
\pgfpathlineto{\pgfqpoint{1.035089in}{1.523099in}}%
\pgfpathlineto{\pgfqpoint{1.036967in}{1.520939in}}%
\pgfpathlineto{\pgfqpoint{1.038990in}{1.518808in}}%
\pgfpathclose%
\pgfusepath{fill}%
\end{pgfscope}%
\begin{pgfscope}%
\pgfpathrectangle{\pgfqpoint{0.041670in}{0.041670in}}{\pgfqpoint{2.216660in}{2.216660in}}%
\pgfusepath{clip}%
\pgfsetbuttcap%
\pgfsetroundjoin%
\definecolor{currentfill}{rgb}{0.279566,0.067836,0.391917}%
\pgfsetfillcolor{currentfill}%
\pgfsetlinewidth{0.000000pt}%
\definecolor{currentstroke}{rgb}{0.000000,0.000000,0.000000}%
\pgfsetstrokecolor{currentstroke}%
\pgfsetdash{}{0pt}%
\pgfpathmoveto{\pgfqpoint{0.990217in}{0.659236in}}%
\pgfpathlineto{\pgfqpoint{0.988809in}{0.653288in}}%
\pgfpathlineto{\pgfqpoint{0.987399in}{0.647507in}}%
\pgfpathlineto{\pgfqpoint{0.985987in}{0.641900in}}%
\pgfpathlineto{\pgfqpoint{0.984574in}{0.636469in}}%
\pgfpathlineto{\pgfqpoint{0.968621in}{0.639932in}}%
\pgfpathlineto{\pgfqpoint{0.952902in}{0.643661in}}%
\pgfpathlineto{\pgfqpoint{0.937433in}{0.647655in}}%
\pgfpathlineto{\pgfqpoint{0.922232in}{0.651906in}}%
\pgfpathlineto{\pgfqpoint{0.924093in}{0.657214in}}%
\pgfpathlineto{\pgfqpoint{0.925951in}{0.662699in}}%
\pgfpathlineto{\pgfqpoint{0.927807in}{0.668356in}}%
\pgfpathlineto{\pgfqpoint{0.929660in}{0.674182in}}%
\pgfpathlineto{\pgfqpoint{0.944427in}{0.670066in}}%
\pgfpathlineto{\pgfqpoint{0.959453in}{0.666199in}}%
\pgfpathlineto{\pgfqpoint{0.974722in}{0.662588in}}%
\pgfpathlineto{\pgfqpoint{0.990217in}{0.659236in}}%
\pgfpathclose%
\pgfusepath{fill}%
\end{pgfscope}%
\begin{pgfscope}%
\pgfpathrectangle{\pgfqpoint{0.041670in}{0.041670in}}{\pgfqpoint{2.216660in}{2.216660in}}%
\pgfusepath{clip}%
\pgfsetbuttcap%
\pgfsetroundjoin%
\definecolor{currentfill}{rgb}{0.260571,0.246922,0.522828}%
\pgfsetfillcolor{currentfill}%
\pgfsetlinewidth{0.000000pt}%
\definecolor{currentstroke}{rgb}{0.000000,0.000000,0.000000}%
\pgfsetstrokecolor{currentstroke}%
\pgfsetdash{}{0pt}%
\pgfpathmoveto{\pgfqpoint{1.791664in}{0.805796in}}%
\pgfpathlineto{\pgfqpoint{1.795369in}{0.814357in}}%
\pgfpathlineto{\pgfqpoint{1.799091in}{0.823344in}}%
\pgfpathlineto{\pgfqpoint{1.802831in}{0.832765in}}%
\pgfpathlineto{\pgfqpoint{1.806589in}{0.842626in}}%
\pgfpathlineto{\pgfqpoint{1.795117in}{0.832338in}}%
\pgfpathlineto{\pgfqpoint{1.782981in}{0.822231in}}%
\pgfpathlineto{\pgfqpoint{1.770191in}{0.812318in}}%
\pgfpathlineto{\pgfqpoint{1.756759in}{0.802609in}}%
\pgfpathlineto{\pgfqpoint{1.753281in}{0.792934in}}%
\pgfpathlineto{\pgfqpoint{1.749820in}{0.783703in}}%
\pgfpathlineto{\pgfqpoint{1.746375in}{0.774907in}}%
\pgfpathlineto{\pgfqpoint{1.742947in}{0.766538in}}%
\pgfpathlineto{\pgfqpoint{1.756076in}{0.776062in}}%
\pgfpathlineto{\pgfqpoint{1.768579in}{0.785787in}}%
\pgfpathlineto{\pgfqpoint{1.780445in}{0.795702in}}%
\pgfpathlineto{\pgfqpoint{1.791664in}{0.805796in}}%
\pgfpathclose%
\pgfusepath{fill}%
\end{pgfscope}%
\begin{pgfscope}%
\pgfpathrectangle{\pgfqpoint{0.041670in}{0.041670in}}{\pgfqpoint{2.216660in}{2.216660in}}%
\pgfusepath{clip}%
\pgfsetbuttcap%
\pgfsetroundjoin%
\definecolor{currentfill}{rgb}{0.896320,0.893616,0.096335}%
\pgfsetfillcolor{currentfill}%
\pgfsetlinewidth{0.000000pt}%
\definecolor{currentstroke}{rgb}{0.000000,0.000000,0.000000}%
\pgfsetstrokecolor{currentstroke}%
\pgfsetdash{}{0pt}%
\pgfpathmoveto{\pgfqpoint{1.258013in}{1.625790in}}%
\pgfpathlineto{\pgfqpoint{1.261906in}{1.623028in}}%
\pgfpathlineto{\pgfqpoint{1.265798in}{1.620142in}}%
\pgfpathlineto{\pgfqpoint{1.269689in}{1.617130in}}%
\pgfpathlineto{\pgfqpoint{1.273578in}{1.613996in}}%
\pgfpathlineto{\pgfqpoint{1.273429in}{1.612611in}}%
\pgfpathlineto{\pgfqpoint{1.273188in}{1.611229in}}%
\pgfpathlineto{\pgfqpoint{1.272852in}{1.609851in}}%
\pgfpathlineto{\pgfqpoint{1.272425in}{1.608479in}}%
\pgfpathlineto{\pgfqpoint{1.268580in}{1.611846in}}%
\pgfpathlineto{\pgfqpoint{1.264735in}{1.615089in}}%
\pgfpathlineto{\pgfqpoint{1.260888in}{1.618207in}}%
\pgfpathlineto{\pgfqpoint{1.257040in}{1.621199in}}%
\pgfpathlineto{\pgfqpoint{1.257400in}{1.622341in}}%
\pgfpathlineto{\pgfqpoint{1.257682in}{1.623488in}}%
\pgfpathlineto{\pgfqpoint{1.257886in}{1.624638in}}%
\pgfpathlineto{\pgfqpoint{1.258013in}{1.625790in}}%
\pgfpathclose%
\pgfusepath{fill}%
\end{pgfscope}%
\begin{pgfscope}%
\pgfpathrectangle{\pgfqpoint{0.041670in}{0.041670in}}{\pgfqpoint{2.216660in}{2.216660in}}%
\pgfusepath{clip}%
\pgfsetbuttcap%
\pgfsetroundjoin%
\definecolor{currentfill}{rgb}{0.272594,0.025563,0.353093}%
\pgfsetfillcolor{currentfill}%
\pgfsetlinewidth{0.000000pt}%
\definecolor{currentstroke}{rgb}{0.000000,0.000000,0.000000}%
\pgfsetstrokecolor{currentstroke}%
\pgfsetdash{}{0pt}%
\pgfpathmoveto{\pgfqpoint{0.808607in}{0.612556in}}%
\pgfpathlineto{\pgfqpoint{0.806210in}{0.613332in}}%
\pgfpathlineto{\pgfqpoint{0.803806in}{0.614411in}}%
\pgfpathlineto{\pgfqpoint{0.801394in}{0.615799in}}%
\pgfpathlineto{\pgfqpoint{0.798975in}{0.617502in}}%
\pgfpathlineto{\pgfqpoint{0.782580in}{0.624143in}}%
\pgfpathlineto{\pgfqpoint{0.766625in}{0.631054in}}%
\pgfpathlineto{\pgfqpoint{0.751127in}{0.638226in}}%
\pgfpathlineto{\pgfqpoint{0.736102in}{0.645651in}}%
\pgfpathlineto{\pgfqpoint{0.738910in}{0.643777in}}%
\pgfpathlineto{\pgfqpoint{0.741709in}{0.642216in}}%
\pgfpathlineto{\pgfqpoint{0.744500in}{0.640964in}}%
\pgfpathlineto{\pgfqpoint{0.747282in}{0.640015in}}%
\pgfpathlineto{\pgfqpoint{0.761939in}{0.632771in}}%
\pgfpathlineto{\pgfqpoint{0.777056in}{0.625775in}}%
\pgfpathlineto{\pgfqpoint{0.792617in}{0.619034in}}%
\pgfpathlineto{\pgfqpoint{0.808607in}{0.612556in}}%
\pgfpathclose%
\pgfusepath{fill}%
\end{pgfscope}%
\begin{pgfscope}%
\pgfpathrectangle{\pgfqpoint{0.041670in}{0.041670in}}{\pgfqpoint{2.216660in}{2.216660in}}%
\pgfusepath{clip}%
\pgfsetbuttcap%
\pgfsetroundjoin%
\definecolor{currentfill}{rgb}{0.133743,0.548535,0.553541}%
\pgfsetfillcolor{currentfill}%
\pgfsetlinewidth{0.000000pt}%
\definecolor{currentstroke}{rgb}{0.000000,0.000000,0.000000}%
\pgfsetstrokecolor{currentstroke}%
\pgfsetdash{}{0pt}%
\pgfpathmoveto{\pgfqpoint{0.989722in}{1.126081in}}%
\pgfpathlineto{\pgfqpoint{0.987485in}{1.116712in}}%
\pgfpathlineto{\pgfqpoint{0.985249in}{1.107332in}}%
\pgfpathlineto{\pgfqpoint{0.983014in}{1.097944in}}%
\pgfpathlineto{\pgfqpoint{0.980779in}{1.088550in}}%
\pgfpathlineto{\pgfqpoint{0.972120in}{1.091805in}}%
\pgfpathlineto{\pgfqpoint{0.963680in}{1.095195in}}%
\pgfpathlineto{\pgfqpoint{0.955467in}{1.098716in}}%
\pgfpathlineto{\pgfqpoint{0.947492in}{1.102365in}}%
\pgfpathlineto{\pgfqpoint{0.950095in}{1.111585in}}%
\pgfpathlineto{\pgfqpoint{0.952699in}{1.120800in}}%
\pgfpathlineto{\pgfqpoint{0.955304in}{1.130008in}}%
\pgfpathlineto{\pgfqpoint{0.957910in}{1.139206in}}%
\pgfpathlineto{\pgfqpoint{0.965533in}{1.135739in}}%
\pgfpathlineto{\pgfqpoint{0.973381in}{1.132393in}}%
\pgfpathlineto{\pgfqpoint{0.981447in}{1.129173in}}%
\pgfpathlineto{\pgfqpoint{0.989722in}{1.126081in}}%
\pgfpathclose%
\pgfusepath{fill}%
\end{pgfscope}%
\begin{pgfscope}%
\pgfpathrectangle{\pgfqpoint{0.041670in}{0.041670in}}{\pgfqpoint{2.216660in}{2.216660in}}%
\pgfusepath{clip}%
\pgfsetbuttcap%
\pgfsetroundjoin%
\definecolor{currentfill}{rgb}{0.814576,0.883393,0.110347}%
\pgfsetfillcolor{currentfill}%
\pgfsetlinewidth{0.000000pt}%
\definecolor{currentstroke}{rgb}{0.000000,0.000000,0.000000}%
\pgfsetstrokecolor{currentstroke}%
\pgfsetdash{}{0pt}%
\pgfpathmoveto{\pgfqpoint{1.287788in}{1.593790in}}%
\pgfpathlineto{\pgfqpoint{1.291625in}{1.589817in}}%
\pgfpathlineto{\pgfqpoint{1.295461in}{1.585726in}}%
\pgfpathlineto{\pgfqpoint{1.299294in}{1.581518in}}%
\pgfpathlineto{\pgfqpoint{1.303126in}{1.577194in}}%
\pgfpathlineto{\pgfqpoint{1.302442in}{1.575366in}}%
\pgfpathlineto{\pgfqpoint{1.301634in}{1.573549in}}%
\pgfpathlineto{\pgfqpoint{1.300705in}{1.571744in}}%
\pgfpathlineto{\pgfqpoint{1.299655in}{1.569953in}}%
\pgfpathlineto{\pgfqpoint{1.295927in}{1.574507in}}%
\pgfpathlineto{\pgfqpoint{1.292197in}{1.578945in}}%
\pgfpathlineto{\pgfqpoint{1.288466in}{1.583265in}}%
\pgfpathlineto{\pgfqpoint{1.284734in}{1.587467in}}%
\pgfpathlineto{\pgfqpoint{1.285657in}{1.589031in}}%
\pgfpathlineto{\pgfqpoint{1.286474in}{1.590607in}}%
\pgfpathlineto{\pgfqpoint{1.287185in}{1.592194in}}%
\pgfpathlineto{\pgfqpoint{1.287788in}{1.593790in}}%
\pgfpathclose%
\pgfusepath{fill}%
\end{pgfscope}%
\begin{pgfscope}%
\pgfpathrectangle{\pgfqpoint{0.041670in}{0.041670in}}{\pgfqpoint{2.216660in}{2.216660in}}%
\pgfusepath{clip}%
\pgfsetbuttcap%
\pgfsetroundjoin%
\definecolor{currentfill}{rgb}{0.134692,0.658636,0.517649}%
\pgfsetfillcolor{currentfill}%
\pgfsetlinewidth{0.000000pt}%
\definecolor{currentstroke}{rgb}{0.000000,0.000000,0.000000}%
\pgfsetstrokecolor{currentstroke}%
\pgfsetdash{}{0pt}%
\pgfpathmoveto{\pgfqpoint{0.989267in}{1.247789in}}%
\pgfpathlineto{\pgfqpoint{0.986648in}{1.238924in}}%
\pgfpathlineto{\pgfqpoint{0.984029in}{1.230016in}}%
\pgfpathlineto{\pgfqpoint{0.981412in}{1.221070in}}%
\pgfpathlineto{\pgfqpoint{0.978796in}{1.212087in}}%
\pgfpathlineto{\pgfqpoint{0.972091in}{1.215300in}}%
\pgfpathlineto{\pgfqpoint{0.965602in}{1.218615in}}%
\pgfpathlineto{\pgfqpoint{0.959336in}{1.222029in}}%
\pgfpathlineto{\pgfqpoint{0.953300in}{1.225537in}}%
\pgfpathlineto{\pgfqpoint{0.956242in}{1.234329in}}%
\pgfpathlineto{\pgfqpoint{0.959185in}{1.243084in}}%
\pgfpathlineto{\pgfqpoint{0.962130in}{1.251800in}}%
\pgfpathlineto{\pgfqpoint{0.965076in}{1.260475in}}%
\pgfpathlineto{\pgfqpoint{0.970805in}{1.257166in}}%
\pgfpathlineto{\pgfqpoint{0.976750in}{1.253946in}}%
\pgfpathlineto{\pgfqpoint{0.982906in}{1.250819in}}%
\pgfpathlineto{\pgfqpoint{0.989267in}{1.247789in}}%
\pgfpathclose%
\pgfusepath{fill}%
\end{pgfscope}%
\begin{pgfscope}%
\pgfpathrectangle{\pgfqpoint{0.041670in}{0.041670in}}{\pgfqpoint{2.216660in}{2.216660in}}%
\pgfusepath{clip}%
\pgfsetbuttcap%
\pgfsetroundjoin%
\definecolor{currentfill}{rgb}{0.993248,0.906157,0.143936}%
\pgfsetfillcolor{currentfill}%
\pgfsetlinewidth{0.000000pt}%
\definecolor{currentstroke}{rgb}{0.000000,0.000000,0.000000}%
\pgfsetstrokecolor{currentstroke}%
\pgfsetdash{}{0pt}%
\pgfpathmoveto{\pgfqpoint{1.193934in}{1.655965in}}%
\pgfpathlineto{\pgfqpoint{1.197427in}{1.656178in}}%
\pgfpathlineto{\pgfqpoint{1.200918in}{1.656259in}}%
\pgfpathlineto{\pgfqpoint{1.204407in}{1.656206in}}%
\pgfpathlineto{\pgfqpoint{1.207896in}{1.656021in}}%
\pgfpathlineto{\pgfqpoint{1.208325in}{1.655607in}}%
\pgfpathlineto{\pgfqpoint{1.208726in}{1.655187in}}%
\pgfpathlineto{\pgfqpoint{1.209098in}{1.654761in}}%
\pgfpathlineto{\pgfqpoint{1.209442in}{1.654330in}}%
\pgfpathlineto{\pgfqpoint{1.205760in}{1.654726in}}%
\pgfpathlineto{\pgfqpoint{1.202076in}{1.654990in}}%
\pgfpathlineto{\pgfqpoint{1.198392in}{1.655121in}}%
\pgfpathlineto{\pgfqpoint{1.194706in}{1.655119in}}%
\pgfpathlineto{\pgfqpoint{1.194535in}{1.655335in}}%
\pgfpathlineto{\pgfqpoint{1.194349in}{1.655548in}}%
\pgfpathlineto{\pgfqpoint{1.194149in}{1.655758in}}%
\pgfpathlineto{\pgfqpoint{1.193934in}{1.655965in}}%
\pgfpathclose%
\pgfusepath{fill}%
\end{pgfscope}%
\begin{pgfscope}%
\pgfpathrectangle{\pgfqpoint{0.041670in}{0.041670in}}{\pgfqpoint{2.216660in}{2.216660in}}%
\pgfusepath{clip}%
\pgfsetbuttcap%
\pgfsetroundjoin%
\definecolor{currentfill}{rgb}{0.201239,0.383670,0.554294}%
\pgfsetfillcolor{currentfill}%
\pgfsetlinewidth{0.000000pt}%
\definecolor{currentstroke}{rgb}{0.000000,0.000000,0.000000}%
\pgfsetstrokecolor{currentstroke}%
\pgfsetdash{}{0pt}%
\pgfpathmoveto{\pgfqpoint{1.861730in}{0.930128in}}%
\pgfpathlineto{\pgfqpoint{1.865801in}{0.942495in}}%
\pgfpathlineto{\pgfqpoint{1.869893in}{0.955347in}}%
\pgfpathlineto{\pgfqpoint{1.874007in}{0.968693in}}%
\pgfpathlineto{\pgfqpoint{1.878145in}{0.982540in}}%
\pgfpathlineto{\pgfqpoint{1.869041in}{0.971267in}}%
\pgfpathlineto{\pgfqpoint{1.859204in}{0.960131in}}%
\pgfpathlineto{\pgfqpoint{1.848641in}{0.949144in}}%
\pgfpathlineto{\pgfqpoint{1.837360in}{0.938318in}}%
\pgfpathlineto{\pgfqpoint{1.833440in}{0.924648in}}%
\pgfpathlineto{\pgfqpoint{1.829542in}{0.911483in}}%
\pgfpathlineto{\pgfqpoint{1.825666in}{0.898814in}}%
\pgfpathlineto{\pgfqpoint{1.821811in}{0.886633in}}%
\pgfpathlineto{\pgfqpoint{1.832848in}{0.897280in}}%
\pgfpathlineto{\pgfqpoint{1.843186in}{0.908086in}}%
\pgfpathlineto{\pgfqpoint{1.852816in}{0.919039in}}%
\pgfpathlineto{\pgfqpoint{1.861730in}{0.930128in}}%
\pgfpathclose%
\pgfusepath{fill}%
\end{pgfscope}%
\begin{pgfscope}%
\pgfpathrectangle{\pgfqpoint{0.041670in}{0.041670in}}{\pgfqpoint{2.216660in}{2.216660in}}%
\pgfusepath{clip}%
\pgfsetbuttcap%
\pgfsetroundjoin%
\definecolor{currentfill}{rgb}{0.267004,0.004874,0.329415}%
\pgfsetfillcolor{currentfill}%
\pgfsetlinewidth{0.000000pt}%
\definecolor{currentstroke}{rgb}{0.000000,0.000000,0.000000}%
\pgfsetstrokecolor{currentstroke}%
\pgfsetdash{}{0pt}%
\pgfpathmoveto{\pgfqpoint{1.545891in}{0.622065in}}%
\pgfpathlineto{\pgfqpoint{1.548324in}{0.620647in}}%
\pgfpathlineto{\pgfqpoint{1.550764in}{0.619488in}}%
\pgfpathlineto{\pgfqpoint{1.553210in}{0.618596in}}%
\pgfpathlineto{\pgfqpoint{1.555661in}{0.617974in}}%
\pgfpathlineto{\pgfqpoint{1.540028in}{0.611691in}}%
\pgfpathlineto{\pgfqpoint{1.523997in}{0.605674in}}%
\pgfpathlineto{\pgfqpoint{1.507584in}{0.599929in}}%
\pgfpathlineto{\pgfqpoint{1.490808in}{0.594464in}}%
\pgfpathlineto{\pgfqpoint{1.488773in}{0.595242in}}%
\pgfpathlineto{\pgfqpoint{1.486743in}{0.596290in}}%
\pgfpathlineto{\pgfqpoint{1.484719in}{0.597605in}}%
\pgfpathlineto{\pgfqpoint{1.482699in}{0.599181in}}%
\pgfpathlineto{\pgfqpoint{1.499044in}{0.604500in}}%
\pgfpathlineto{\pgfqpoint{1.515036in}{0.610092in}}%
\pgfpathlineto{\pgfqpoint{1.530657in}{0.615949in}}%
\pgfpathlineto{\pgfqpoint{1.545891in}{0.622065in}}%
\pgfpathclose%
\pgfusepath{fill}%
\end{pgfscope}%
\begin{pgfscope}%
\pgfpathrectangle{\pgfqpoint{0.041670in}{0.041670in}}{\pgfqpoint{2.216660in}{2.216660in}}%
\pgfusepath{clip}%
\pgfsetbuttcap%
\pgfsetroundjoin%
\definecolor{currentfill}{rgb}{0.974417,0.903590,0.130215}%
\pgfsetfillcolor{currentfill}%
\pgfsetlinewidth{0.000000pt}%
\definecolor{currentstroke}{rgb}{0.000000,0.000000,0.000000}%
\pgfsetstrokecolor{currentstroke}%
\pgfsetdash{}{0pt}%
\pgfpathmoveto{\pgfqpoint{1.210520in}{1.652561in}}%
\pgfpathlineto{\pgfqpoint{1.214336in}{1.651811in}}%
\pgfpathlineto{\pgfqpoint{1.218152in}{1.650929in}}%
\pgfpathlineto{\pgfqpoint{1.221966in}{1.649915in}}%
\pgfpathlineto{\pgfqpoint{1.225780in}{1.648770in}}%
\pgfpathlineto{\pgfqpoint{1.226072in}{1.648094in}}%
\pgfpathlineto{\pgfqpoint{1.226318in}{1.647413in}}%
\pgfpathlineto{\pgfqpoint{1.226519in}{1.646729in}}%
\pgfpathlineto{\pgfqpoint{1.226673in}{1.646042in}}%
\pgfpathlineto{\pgfqpoint{1.222784in}{1.647415in}}%
\pgfpathlineto{\pgfqpoint{1.218894in}{1.648656in}}%
\pgfpathlineto{\pgfqpoint{1.215003in}{1.649765in}}%
\pgfpathlineto{\pgfqpoint{1.211111in}{1.650743in}}%
\pgfpathlineto{\pgfqpoint{1.211009in}{1.651200in}}%
\pgfpathlineto{\pgfqpoint{1.210877in}{1.651656in}}%
\pgfpathlineto{\pgfqpoint{1.210713in}{1.652110in}}%
\pgfpathlineto{\pgfqpoint{1.210520in}{1.652561in}}%
\pgfpathclose%
\pgfusepath{fill}%
\end{pgfscope}%
\begin{pgfscope}%
\pgfpathrectangle{\pgfqpoint{0.041670in}{0.041670in}}{\pgfqpoint{2.216660in}{2.216660in}}%
\pgfusepath{clip}%
\pgfsetbuttcap%
\pgfsetroundjoin%
\definecolor{currentfill}{rgb}{0.993248,0.906157,0.143936}%
\pgfsetfillcolor{currentfill}%
\pgfsetlinewidth{0.000000pt}%
\definecolor{currentstroke}{rgb}{0.000000,0.000000,0.000000}%
\pgfsetstrokecolor{currentstroke}%
\pgfsetdash{}{0pt}%
\pgfpathmoveto{\pgfqpoint{1.165064in}{1.654925in}}%
\pgfpathlineto{\pgfqpoint{1.161343in}{1.654879in}}%
\pgfpathlineto{\pgfqpoint{1.157623in}{1.654700in}}%
\pgfpathlineto{\pgfqpoint{1.153905in}{1.654387in}}%
\pgfpathlineto{\pgfqpoint{1.150187in}{1.653942in}}%
\pgfpathlineto{\pgfqpoint{1.150505in}{1.654378in}}%
\pgfpathlineto{\pgfqpoint{1.150851in}{1.654808in}}%
\pgfpathlineto{\pgfqpoint{1.151227in}{1.655234in}}%
\pgfpathlineto{\pgfqpoint{1.151631in}{1.655653in}}%
\pgfpathlineto{\pgfqpoint{1.155167in}{1.655884in}}%
\pgfpathlineto{\pgfqpoint{1.158705in}{1.655983in}}%
\pgfpathlineto{\pgfqpoint{1.162244in}{1.655949in}}%
\pgfpathlineto{\pgfqpoint{1.165784in}{1.655781in}}%
\pgfpathlineto{\pgfqpoint{1.165582in}{1.655571in}}%
\pgfpathlineto{\pgfqpoint{1.165395in}{1.655359in}}%
\pgfpathlineto{\pgfqpoint{1.165222in}{1.655143in}}%
\pgfpathlineto{\pgfqpoint{1.165064in}{1.654925in}}%
\pgfpathclose%
\pgfusepath{fill}%
\end{pgfscope}%
\begin{pgfscope}%
\pgfpathrectangle{\pgfqpoint{0.041670in}{0.041670in}}{\pgfqpoint{2.216660in}{2.216660in}}%
\pgfusepath{clip}%
\pgfsetbuttcap%
\pgfsetroundjoin%
\definecolor{currentfill}{rgb}{0.281477,0.755203,0.432552}%
\pgfsetfillcolor{currentfill}%
\pgfsetlinewidth{0.000000pt}%
\definecolor{currentstroke}{rgb}{0.000000,0.000000,0.000000}%
\pgfsetstrokecolor{currentstroke}%
\pgfsetdash{}{0pt}%
\pgfpathmoveto{\pgfqpoint{1.000550in}{1.360497in}}%
\pgfpathlineto{\pgfqpoint{0.997586in}{1.352499in}}%
\pgfpathlineto{\pgfqpoint{0.994623in}{1.344432in}}%
\pgfpathlineto{\pgfqpoint{0.991661in}{1.336299in}}%
\pgfpathlineto{\pgfqpoint{0.988702in}{1.328102in}}%
\pgfpathlineto{\pgfqpoint{0.983790in}{1.331094in}}%
\pgfpathlineto{\pgfqpoint{0.979081in}{1.334159in}}%
\pgfpathlineto{\pgfqpoint{0.974579in}{1.337294in}}%
\pgfpathlineto{\pgfqpoint{0.970287in}{1.340495in}}%
\pgfpathlineto{\pgfqpoint{0.973526in}{1.348486in}}%
\pgfpathlineto{\pgfqpoint{0.976767in}{1.356414in}}%
\pgfpathlineto{\pgfqpoint{0.980010in}{1.364275in}}%
\pgfpathlineto{\pgfqpoint{0.983254in}{1.372068in}}%
\pgfpathlineto{\pgfqpoint{0.987286in}{1.369079in}}%
\pgfpathlineto{\pgfqpoint{0.991515in}{1.366152in}}%
\pgfpathlineto{\pgfqpoint{0.995938in}{1.363290in}}%
\pgfpathlineto{\pgfqpoint{1.000550in}{1.360497in}}%
\pgfpathclose%
\pgfusepath{fill}%
\end{pgfscope}%
\begin{pgfscope}%
\pgfpathrectangle{\pgfqpoint{0.041670in}{0.041670in}}{\pgfqpoint{2.216660in}{2.216660in}}%
\pgfusepath{clip}%
\pgfsetbuttcap%
\pgfsetroundjoin%
\definecolor{currentfill}{rgb}{0.179019,0.433756,0.557430}%
\pgfsetfillcolor{currentfill}%
\pgfsetlinewidth{0.000000pt}%
\definecolor{currentstroke}{rgb}{0.000000,0.000000,0.000000}%
\pgfsetstrokecolor{currentstroke}%
\pgfsetdash{}{0pt}%
\pgfpathmoveto{\pgfqpoint{1.002834in}{1.000792in}}%
\pgfpathlineto{\pgfqpoint{1.001012in}{0.991335in}}%
\pgfpathlineto{\pgfqpoint{0.999189in}{0.981902in}}%
\pgfpathlineto{\pgfqpoint{0.997367in}{0.972496in}}%
\pgfpathlineto{\pgfqpoint{0.995545in}{0.963121in}}%
\pgfpathlineto{\pgfqpoint{0.984837in}{0.966210in}}%
\pgfpathlineto{\pgfqpoint{0.974338in}{0.969470in}}%
\pgfpathlineto{\pgfqpoint{0.964059in}{0.972898in}}%
\pgfpathlineto{\pgfqpoint{0.954010in}{0.976490in}}%
\pgfpathlineto{\pgfqpoint{0.956238in}{0.985716in}}%
\pgfpathlineto{\pgfqpoint{0.958466in}{0.994973in}}%
\pgfpathlineto{\pgfqpoint{0.960695in}{1.004257in}}%
\pgfpathlineto{\pgfqpoint{0.962925in}{1.013566in}}%
\pgfpathlineto{\pgfqpoint{0.972582in}{1.010134in}}%
\pgfpathlineto{\pgfqpoint{0.982459in}{1.006859in}}%
\pgfpathlineto{\pgfqpoint{0.992547in}{1.003744in}}%
\pgfpathlineto{\pgfqpoint{1.002834in}{1.000792in}}%
\pgfpathclose%
\pgfusepath{fill}%
\end{pgfscope}%
\begin{pgfscope}%
\pgfpathrectangle{\pgfqpoint{0.041670in}{0.041670in}}{\pgfqpoint{2.216660in}{2.216660in}}%
\pgfusepath{clip}%
\pgfsetbuttcap%
\pgfsetroundjoin%
\definecolor{currentfill}{rgb}{0.955300,0.901065,0.118128}%
\pgfsetfillcolor{currentfill}%
\pgfsetlinewidth{0.000000pt}%
\definecolor{currentstroke}{rgb}{0.000000,0.000000,0.000000}%
\pgfsetstrokecolor{currentstroke}%
\pgfsetdash{}{0pt}%
\pgfpathmoveto{\pgfqpoint{1.226673in}{1.646042in}}%
\pgfpathlineto{\pgfqpoint{1.230561in}{1.644539in}}%
\pgfpathlineto{\pgfqpoint{1.234448in}{1.642905in}}%
\pgfpathlineto{\pgfqpoint{1.238333in}{1.641142in}}%
\pgfpathlineto{\pgfqpoint{1.242218in}{1.639249in}}%
\pgfpathlineto{\pgfqpoint{1.242363in}{1.638330in}}%
\pgfpathlineto{\pgfqpoint{1.242447in}{1.637410in}}%
\pgfpathlineto{\pgfqpoint{1.242468in}{1.636488in}}%
\pgfpathlineto{\pgfqpoint{1.242426in}{1.635566in}}%
\pgfpathlineto{\pgfqpoint{1.238527in}{1.637690in}}%
\pgfpathlineto{\pgfqpoint{1.234626in}{1.639684in}}%
\pgfpathlineto{\pgfqpoint{1.230725in}{1.641549in}}%
\pgfpathlineto{\pgfqpoint{1.226823in}{1.643282in}}%
\pgfpathlineto{\pgfqpoint{1.226855in}{1.643973in}}%
\pgfpathlineto{\pgfqpoint{1.226841in}{1.644664in}}%
\pgfpathlineto{\pgfqpoint{1.226780in}{1.645354in}}%
\pgfpathlineto{\pgfqpoint{1.226673in}{1.646042in}}%
\pgfpathclose%
\pgfusepath{fill}%
\end{pgfscope}%
\begin{pgfscope}%
\pgfpathrectangle{\pgfqpoint{0.041670in}{0.041670in}}{\pgfqpoint{2.216660in}{2.216660in}}%
\pgfusepath{clip}%
\pgfsetbuttcap%
\pgfsetroundjoin%
\definecolor{currentfill}{rgb}{0.896320,0.893616,0.096335}%
\pgfsetfillcolor{currentfill}%
\pgfsetlinewidth{0.000000pt}%
\definecolor{currentstroke}{rgb}{0.000000,0.000000,0.000000}%
\pgfsetstrokecolor{currentstroke}%
\pgfsetdash{}{0pt}%
\pgfpathmoveto{\pgfqpoint{1.103254in}{1.620189in}}%
\pgfpathlineto{\pgfqpoint{1.099424in}{1.617145in}}%
\pgfpathlineto{\pgfqpoint{1.095596in}{1.613976in}}%
\pgfpathlineto{\pgfqpoint{1.091769in}{1.610682in}}%
\pgfpathlineto{\pgfqpoint{1.087943in}{1.607264in}}%
\pgfpathlineto{\pgfqpoint{1.087433in}{1.608631in}}%
\pgfpathlineto{\pgfqpoint{1.087015in}{1.610004in}}%
\pgfpathlineto{\pgfqpoint{1.086691in}{1.611383in}}%
\pgfpathlineto{\pgfqpoint{1.086459in}{1.612765in}}%
\pgfpathlineto{\pgfqpoint{1.090344in}{1.615951in}}%
\pgfpathlineto{\pgfqpoint{1.094230in}{1.619014in}}%
\pgfpathlineto{\pgfqpoint{1.098117in}{1.621953in}}%
\pgfpathlineto{\pgfqpoint{1.102006in}{1.624765in}}%
\pgfpathlineto{\pgfqpoint{1.102201in}{1.623615in}}%
\pgfpathlineto{\pgfqpoint{1.102475in}{1.622468in}}%
\pgfpathlineto{\pgfqpoint{1.102826in}{1.621326in}}%
\pgfpathlineto{\pgfqpoint{1.103254in}{1.620189in}}%
\pgfpathclose%
\pgfusepath{fill}%
\end{pgfscope}%
\begin{pgfscope}%
\pgfpathrectangle{\pgfqpoint{0.041670in}{0.041670in}}{\pgfqpoint{2.216660in}{2.216660in}}%
\pgfusepath{clip}%
\pgfsetbuttcap%
\pgfsetroundjoin%
\definecolor{currentfill}{rgb}{0.231674,0.318106,0.544834}%
\pgfsetfillcolor{currentfill}%
\pgfsetlinewidth{0.000000pt}%
\definecolor{currentstroke}{rgb}{0.000000,0.000000,0.000000}%
\pgfsetstrokecolor{currentstroke}%
\pgfsetdash{}{0pt}%
\pgfpathmoveto{\pgfqpoint{1.389214in}{0.892573in}}%
\pgfpathlineto{\pgfqpoint{1.391129in}{0.883646in}}%
\pgfpathlineto{\pgfqpoint{1.393045in}{0.874781in}}%
\pgfpathlineto{\pgfqpoint{1.394961in}{0.865983in}}%
\pgfpathlineto{\pgfqpoint{1.396877in}{0.857255in}}%
\pgfpathlineto{\pgfqpoint{1.384886in}{0.853768in}}%
\pgfpathlineto{\pgfqpoint{1.372673in}{0.850480in}}%
\pgfpathlineto{\pgfqpoint{1.360252in}{0.847395in}}%
\pgfpathlineto{\pgfqpoint{1.347635in}{0.844515in}}%
\pgfpathlineto{\pgfqpoint{1.346151in}{0.853372in}}%
\pgfpathlineto{\pgfqpoint{1.344668in}{0.862298in}}%
\pgfpathlineto{\pgfqpoint{1.343185in}{0.871290in}}%
\pgfpathlineto{\pgfqpoint{1.341702in}{0.880345in}}%
\pgfpathlineto{\pgfqpoint{1.353874in}{0.883109in}}%
\pgfpathlineto{\pgfqpoint{1.365858in}{0.886071in}}%
\pgfpathlineto{\pgfqpoint{1.377643in}{0.889227in}}%
\pgfpathlineto{\pgfqpoint{1.389214in}{0.892573in}}%
\pgfpathclose%
\pgfusepath{fill}%
\end{pgfscope}%
\begin{pgfscope}%
\pgfpathrectangle{\pgfqpoint{0.041670in}{0.041670in}}{\pgfqpoint{2.216660in}{2.216660in}}%
\pgfusepath{clip}%
\pgfsetbuttcap%
\pgfsetroundjoin%
\definecolor{currentfill}{rgb}{0.699415,0.867117,0.175971}%
\pgfsetfillcolor{currentfill}%
\pgfsetlinewidth{0.000000pt}%
\definecolor{currentstroke}{rgb}{0.000000,0.000000,0.000000}%
\pgfsetstrokecolor{currentstroke}%
\pgfsetdash{}{0pt}%
\pgfpathmoveto{\pgfqpoint{1.314551in}{1.550597in}}%
\pgfpathlineto{\pgfqpoint{1.318272in}{1.545479in}}%
\pgfpathlineto{\pgfqpoint{1.321990in}{1.540253in}}%
\pgfpathlineto{\pgfqpoint{1.325706in}{1.534919in}}%
\pgfpathlineto{\pgfqpoint{1.329421in}{1.529479in}}%
\pgfpathlineto{\pgfqpoint{1.327969in}{1.527249in}}%
\pgfpathlineto{\pgfqpoint{1.326368in}{1.525041in}}%
\pgfpathlineto{\pgfqpoint{1.324620in}{1.522858in}}%
\pgfpathlineto{\pgfqpoint{1.322725in}{1.520701in}}%
\pgfpathlineto{\pgfqpoint{1.319172in}{1.526366in}}%
\pgfpathlineto{\pgfqpoint{1.315618in}{1.531925in}}%
\pgfpathlineto{\pgfqpoint{1.312062in}{1.537376in}}%
\pgfpathlineto{\pgfqpoint{1.308504in}{1.542718in}}%
\pgfpathlineto{\pgfqpoint{1.310214in}{1.544654in}}%
\pgfpathlineto{\pgfqpoint{1.311793in}{1.546614in}}%
\pgfpathlineto{\pgfqpoint{1.313239in}{1.548595in}}%
\pgfpathlineto{\pgfqpoint{1.314551in}{1.550597in}}%
\pgfpathclose%
\pgfusepath{fill}%
\end{pgfscope}%
\begin{pgfscope}%
\pgfpathrectangle{\pgfqpoint{0.041670in}{0.041670in}}{\pgfqpoint{2.216660in}{2.216660in}}%
\pgfusepath{clip}%
\pgfsetbuttcap%
\pgfsetroundjoin%
\definecolor{currentfill}{rgb}{0.974417,0.903590,0.130215}%
\pgfsetfillcolor{currentfill}%
\pgfsetlinewidth{0.000000pt}%
\definecolor{currentstroke}{rgb}{0.000000,0.000000,0.000000}%
\pgfsetstrokecolor{currentstroke}%
\pgfsetdash{}{0pt}%
\pgfpathmoveto{\pgfqpoint{1.148734in}{1.650335in}}%
\pgfpathlineto{\pgfqpoint{1.144834in}{1.649306in}}%
\pgfpathlineto{\pgfqpoint{1.140935in}{1.648146in}}%
\pgfpathlineto{\pgfqpoint{1.137037in}{1.646854in}}%
\pgfpathlineto{\pgfqpoint{1.133139in}{1.645430in}}%
\pgfpathlineto{\pgfqpoint{1.133252in}{1.646119in}}%
\pgfpathlineto{\pgfqpoint{1.133411in}{1.646805in}}%
\pgfpathlineto{\pgfqpoint{1.133616in}{1.647489in}}%
\pgfpathlineto{\pgfqpoint{1.133868in}{1.648169in}}%
\pgfpathlineto{\pgfqpoint{1.137703in}{1.649364in}}%
\pgfpathlineto{\pgfqpoint{1.141540in}{1.650428in}}%
\pgfpathlineto{\pgfqpoint{1.145378in}{1.651360in}}%
\pgfpathlineto{\pgfqpoint{1.149216in}{1.652160in}}%
\pgfpathlineto{\pgfqpoint{1.149050in}{1.651707in}}%
\pgfpathlineto{\pgfqpoint{1.148914in}{1.651251in}}%
\pgfpathlineto{\pgfqpoint{1.148809in}{1.650794in}}%
\pgfpathlineto{\pgfqpoint{1.148734in}{1.650335in}}%
\pgfpathclose%
\pgfusepath{fill}%
\end{pgfscope}%
\begin{pgfscope}%
\pgfpathrectangle{\pgfqpoint{0.041670in}{0.041670in}}{\pgfqpoint{2.216660in}{2.216660in}}%
\pgfusepath{clip}%
\pgfsetbuttcap%
\pgfsetroundjoin%
\definecolor{currentfill}{rgb}{0.122606,0.585371,0.546557}%
\pgfsetfillcolor{currentfill}%
\pgfsetlinewidth{0.000000pt}%
\definecolor{currentstroke}{rgb}{0.000000,0.000000,0.000000}%
\pgfsetstrokecolor{currentstroke}%
\pgfsetdash{}{0pt}%
\pgfpathmoveto{\pgfqpoint{1.397842in}{1.178852in}}%
\pgfpathlineto{\pgfqpoint{1.400529in}{1.169764in}}%
\pgfpathlineto{\pgfqpoint{1.403214in}{1.160656in}}%
\pgfpathlineto{\pgfqpoint{1.405898in}{1.151529in}}%
\pgfpathlineto{\pgfqpoint{1.408581in}{1.142386in}}%
\pgfpathlineto{\pgfqpoint{1.401165in}{1.138815in}}%
\pgfpathlineto{\pgfqpoint{1.393516in}{1.135361in}}%
\pgfpathlineto{\pgfqpoint{1.385643in}{1.132029in}}%
\pgfpathlineto{\pgfqpoint{1.377553in}{1.128823in}}%
\pgfpathlineto{\pgfqpoint{1.375230in}{1.138143in}}%
\pgfpathlineto{\pgfqpoint{1.372906in}{1.147447in}}%
\pgfpathlineto{\pgfqpoint{1.370581in}{1.156732in}}%
\pgfpathlineto{\pgfqpoint{1.368255in}{1.165995in}}%
\pgfpathlineto{\pgfqpoint{1.375967in}{1.169034in}}%
\pgfpathlineto{\pgfqpoint{1.383475in}{1.172193in}}%
\pgfpathlineto{\pgfqpoint{1.390769in}{1.175466in}}%
\pgfpathlineto{\pgfqpoint{1.397842in}{1.178852in}}%
\pgfpathclose%
\pgfusepath{fill}%
\end{pgfscope}%
\begin{pgfscope}%
\pgfpathrectangle{\pgfqpoint{0.041670in}{0.041670in}}{\pgfqpoint{2.216660in}{2.216660in}}%
\pgfusepath{clip}%
\pgfsetbuttcap%
\pgfsetroundjoin%
\definecolor{currentfill}{rgb}{0.248629,0.278775,0.534556}%
\pgfsetfillcolor{currentfill}%
\pgfsetlinewidth{0.000000pt}%
\definecolor{currentstroke}{rgb}{0.000000,0.000000,0.000000}%
\pgfsetstrokecolor{currentstroke}%
\pgfsetdash{}{0pt}%
\pgfpathmoveto{\pgfqpoint{1.023642in}{0.842131in}}%
\pgfpathlineto{\pgfqpoint{1.022258in}{0.833324in}}%
\pgfpathlineto{\pgfqpoint{1.020873in}{0.824593in}}%
\pgfpathlineto{\pgfqpoint{1.019489in}{0.815941in}}%
\pgfpathlineto{\pgfqpoint{1.018103in}{0.807374in}}%
\pgfpathlineto{\pgfqpoint{1.004874in}{0.810176in}}%
\pgfpathlineto{\pgfqpoint{0.991835in}{0.813196in}}%
\pgfpathlineto{\pgfqpoint{0.979001in}{0.816430in}}%
\pgfpathlineto{\pgfqpoint{0.966384in}{0.819873in}}%
\pgfpathlineto{\pgfqpoint{0.968209in}{0.828319in}}%
\pgfpathlineto{\pgfqpoint{0.970033in}{0.836849in}}%
\pgfpathlineto{\pgfqpoint{0.971857in}{0.845459in}}%
\pgfpathlineto{\pgfqpoint{0.973681in}{0.854146in}}%
\pgfpathlineto{\pgfqpoint{0.985869in}{0.850836in}}%
\pgfpathlineto{\pgfqpoint{0.998268in}{0.847728in}}%
\pgfpathlineto{\pgfqpoint{1.010863in}{0.844825in}}%
\pgfpathlineto{\pgfqpoint{1.023642in}{0.842131in}}%
\pgfpathclose%
\pgfusepath{fill}%
\end{pgfscope}%
\begin{pgfscope}%
\pgfpathrectangle{\pgfqpoint{0.041670in}{0.041670in}}{\pgfqpoint{2.216660in}{2.216660in}}%
\pgfusepath{clip}%
\pgfsetbuttcap%
\pgfsetroundjoin%
\definecolor{currentfill}{rgb}{0.282884,0.135920,0.453427}%
\pgfsetfillcolor{currentfill}%
\pgfsetlinewidth{0.000000pt}%
\definecolor{currentstroke}{rgb}{0.000000,0.000000,0.000000}%
\pgfsetstrokecolor{currentstroke}%
\pgfsetdash{}{0pt}%
\pgfpathmoveto{\pgfqpoint{1.716054in}{0.714158in}}%
\pgfpathlineto{\pgfqpoint{1.719367in}{0.719353in}}%
\pgfpathlineto{\pgfqpoint{1.722693in}{0.724921in}}%
\pgfpathlineto{\pgfqpoint{1.726033in}{0.730869in}}%
\pgfpathlineto{\pgfqpoint{1.729387in}{0.737203in}}%
\pgfpathlineto{\pgfqpoint{1.715959in}{0.728080in}}%
\pgfpathlineto{\pgfqpoint{1.701945in}{0.719177in}}%
\pgfpathlineto{\pgfqpoint{1.687359in}{0.710505in}}%
\pgfpathlineto{\pgfqpoint{1.672216in}{0.702073in}}%
\pgfpathlineto{\pgfqpoint{1.669197in}{0.695924in}}%
\pgfpathlineto{\pgfqpoint{1.666191in}{0.690163in}}%
\pgfpathlineto{\pgfqpoint{1.663197in}{0.684783in}}%
\pgfpathlineto{\pgfqpoint{1.660215in}{0.679777in}}%
\pgfpathlineto{\pgfqpoint{1.675003in}{0.688028in}}%
\pgfpathlineto{\pgfqpoint{1.689248in}{0.696516in}}%
\pgfpathlineto{\pgfqpoint{1.702936in}{0.705229in}}%
\pgfpathlineto{\pgfqpoint{1.716054in}{0.714158in}}%
\pgfpathclose%
\pgfusepath{fill}%
\end{pgfscope}%
\begin{pgfscope}%
\pgfpathrectangle{\pgfqpoint{0.041670in}{0.041670in}}{\pgfqpoint{2.216660in}{2.216660in}}%
\pgfusepath{clip}%
\pgfsetbuttcap%
\pgfsetroundjoin%
\definecolor{currentfill}{rgb}{0.814576,0.883393,0.110347}%
\pgfsetfillcolor{currentfill}%
\pgfsetlinewidth{0.000000pt}%
\definecolor{currentstroke}{rgb}{0.000000,0.000000,0.000000}%
\pgfsetstrokecolor{currentstroke}%
\pgfsetdash{}{0pt}%
\pgfpathmoveto{\pgfqpoint{1.076085in}{1.586089in}}%
\pgfpathlineto{\pgfqpoint{1.072384in}{1.581837in}}%
\pgfpathlineto{\pgfqpoint{1.068685in}{1.577467in}}%
\pgfpathlineto{\pgfqpoint{1.064987in}{1.572979in}}%
\pgfpathlineto{\pgfqpoint{1.061290in}{1.568374in}}%
\pgfpathlineto{\pgfqpoint{1.060132in}{1.570151in}}%
\pgfpathlineto{\pgfqpoint{1.059095in}{1.571944in}}%
\pgfpathlineto{\pgfqpoint{1.058180in}{1.573750in}}%
\pgfpathlineto{\pgfqpoint{1.057386in}{1.575569in}}%
\pgfpathlineto{\pgfqpoint{1.061200in}{1.579945in}}%
\pgfpathlineto{\pgfqpoint{1.065016in}{1.584204in}}%
\pgfpathlineto{\pgfqpoint{1.068833in}{1.588346in}}%
\pgfpathlineto{\pgfqpoint{1.072653in}{1.592371in}}%
\pgfpathlineto{\pgfqpoint{1.073351in}{1.590783in}}%
\pgfpathlineto{\pgfqpoint{1.074156in}{1.589205in}}%
\pgfpathlineto{\pgfqpoint{1.075068in}{1.587640in}}%
\pgfpathlineto{\pgfqpoint{1.076085in}{1.586089in}}%
\pgfpathclose%
\pgfusepath{fill}%
\end{pgfscope}%
\begin{pgfscope}%
\pgfpathrectangle{\pgfqpoint{0.041670in}{0.041670in}}{\pgfqpoint{2.216660in}{2.216660in}}%
\pgfusepath{clip}%
\pgfsetbuttcap%
\pgfsetroundjoin%
\definecolor{currentfill}{rgb}{0.166383,0.690856,0.496502}%
\pgfsetfillcolor{currentfill}%
\pgfsetlinewidth{0.000000pt}%
\definecolor{currentstroke}{rgb}{0.000000,0.000000,0.000000}%
\pgfsetstrokecolor{currentstroke}%
\pgfsetdash{}{0pt}%
\pgfpathmoveto{\pgfqpoint{1.387674in}{1.297548in}}%
\pgfpathlineto{\pgfqpoint{1.390692in}{1.289107in}}%
\pgfpathlineto{\pgfqpoint{1.393709in}{1.280615in}}%
\pgfpathlineto{\pgfqpoint{1.396725in}{1.272075in}}%
\pgfpathlineto{\pgfqpoint{1.399739in}{1.263489in}}%
\pgfpathlineto{\pgfqpoint{1.394208in}{1.260103in}}%
\pgfpathlineto{\pgfqpoint{1.388455in}{1.256803in}}%
\pgfpathlineto{\pgfqpoint{1.382486in}{1.253594in}}%
\pgfpathlineto{\pgfqpoint{1.376307in}{1.250478in}}%
\pgfpathlineto{\pgfqpoint{1.373609in}{1.259259in}}%
\pgfpathlineto{\pgfqpoint{1.370910in}{1.267993in}}%
\pgfpathlineto{\pgfqpoint{1.368210in}{1.276679in}}%
\pgfpathlineto{\pgfqpoint{1.365509in}{1.285314in}}%
\pgfpathlineto{\pgfqpoint{1.371353in}{1.288244in}}%
\pgfpathlineto{\pgfqpoint{1.376999in}{1.291261in}}%
\pgfpathlineto{\pgfqpoint{1.382441in}{1.294364in}}%
\pgfpathlineto{\pgfqpoint{1.387674in}{1.297548in}}%
\pgfpathclose%
\pgfusepath{fill}%
\end{pgfscope}%
\begin{pgfscope}%
\pgfpathrectangle{\pgfqpoint{0.041670in}{0.041670in}}{\pgfqpoint{2.216660in}{2.216660in}}%
\pgfusepath{clip}%
\pgfsetbuttcap%
\pgfsetroundjoin%
\definecolor{currentfill}{rgb}{0.955300,0.901065,0.118128}%
\pgfsetfillcolor{currentfill}%
\pgfsetlinewidth{0.000000pt}%
\definecolor{currentstroke}{rgb}{0.000000,0.000000,0.000000}%
\pgfsetstrokecolor{currentstroke}%
\pgfsetdash{}{0pt}%
\pgfpathmoveto{\pgfqpoint{1.133155in}{1.642669in}}%
\pgfpathlineto{\pgfqpoint{1.129258in}{1.640884in}}%
\pgfpathlineto{\pgfqpoint{1.125362in}{1.638969in}}%
\pgfpathlineto{\pgfqpoint{1.121467in}{1.636923in}}%
\pgfpathlineto{\pgfqpoint{1.117572in}{1.634748in}}%
\pgfpathlineto{\pgfqpoint{1.117476in}{1.635669in}}%
\pgfpathlineto{\pgfqpoint{1.117441in}{1.636590in}}%
\pgfpathlineto{\pgfqpoint{1.117469in}{1.637512in}}%
\pgfpathlineto{\pgfqpoint{1.117560in}{1.638433in}}%
\pgfpathlineto{\pgfqpoint{1.121453in}{1.640376in}}%
\pgfpathlineto{\pgfqpoint{1.125347in}{1.642191in}}%
\pgfpathlineto{\pgfqpoint{1.129243in}{1.643876in}}%
\pgfpathlineto{\pgfqpoint{1.133139in}{1.645430in}}%
\pgfpathlineto{\pgfqpoint{1.133073in}{1.644740in}}%
\pgfpathlineto{\pgfqpoint{1.133054in}{1.644050in}}%
\pgfpathlineto{\pgfqpoint{1.133081in}{1.643359in}}%
\pgfpathlineto{\pgfqpoint{1.133155in}{1.642669in}}%
\pgfpathclose%
\pgfusepath{fill}%
\end{pgfscope}%
\begin{pgfscope}%
\pgfpathrectangle{\pgfqpoint{0.041670in}{0.041670in}}{\pgfqpoint{2.216660in}{2.216660in}}%
\pgfusepath{clip}%
\pgfsetbuttcap%
\pgfsetroundjoin%
\definecolor{currentfill}{rgb}{0.993248,0.906157,0.143936}%
\pgfsetfillcolor{currentfill}%
\pgfsetlinewidth{0.000000pt}%
\definecolor{currentstroke}{rgb}{0.000000,0.000000,0.000000}%
\pgfsetstrokecolor{currentstroke}%
\pgfsetdash{}{0pt}%
\pgfpathmoveto{\pgfqpoint{1.179955in}{1.653775in}}%
\pgfpathlineto{\pgfqpoint{1.179469in}{1.655515in}}%
\pgfpathlineto{\pgfqpoint{1.178984in}{1.657120in}}%
\pgfpathlineto{\pgfqpoint{1.178499in}{1.658591in}}%
\pgfpathlineto{\pgfqpoint{1.178014in}{1.659927in}}%
\pgfpathlineto{\pgfqpoint{1.178505in}{1.659952in}}%
\pgfpathlineto{\pgfqpoint{1.178996in}{1.659970in}}%
\pgfpathlineto{\pgfqpoint{1.179489in}{1.659981in}}%
\pgfpathlineto{\pgfqpoint{1.179982in}{1.659984in}}%
\pgfpathlineto{\pgfqpoint{1.179975in}{1.658633in}}%
\pgfpathlineto{\pgfqpoint{1.179969in}{1.657148in}}%
\pgfpathlineto{\pgfqpoint{1.179962in}{1.655529in}}%
\pgfpathlineto{\pgfqpoint{1.179955in}{1.653775in}}%
\pgfpathlineto{\pgfqpoint{1.179955in}{1.653775in}}%
\pgfpathlineto{\pgfqpoint{1.179955in}{1.653775in}}%
\pgfpathlineto{\pgfqpoint{1.179955in}{1.653775in}}%
\pgfpathlineto{\pgfqpoint{1.179955in}{1.653775in}}%
\pgfpathclose%
\pgfusepath{fill}%
\end{pgfscope}%
\begin{pgfscope}%
\pgfpathrectangle{\pgfqpoint{0.041670in}{0.041670in}}{\pgfqpoint{2.216660in}{2.216660in}}%
\pgfusepath{clip}%
\pgfsetbuttcap%
\pgfsetroundjoin%
\definecolor{currentfill}{rgb}{0.993248,0.906157,0.143936}%
\pgfsetfillcolor{currentfill}%
\pgfsetlinewidth{0.000000pt}%
\definecolor{currentstroke}{rgb}{0.000000,0.000000,0.000000}%
\pgfsetstrokecolor{currentstroke}%
\pgfsetdash{}{0pt}%
\pgfpathmoveto{\pgfqpoint{1.179955in}{1.653775in}}%
\pgfpathlineto{\pgfqpoint{1.179962in}{1.655529in}}%
\pgfpathlineto{\pgfqpoint{1.179969in}{1.657148in}}%
\pgfpathlineto{\pgfqpoint{1.179975in}{1.658633in}}%
\pgfpathlineto{\pgfqpoint{1.179982in}{1.659984in}}%
\pgfpathlineto{\pgfqpoint{1.180475in}{1.659980in}}%
\pgfpathlineto{\pgfqpoint{1.180968in}{1.659968in}}%
\pgfpathlineto{\pgfqpoint{1.181460in}{1.659950in}}%
\pgfpathlineto{\pgfqpoint{1.181950in}{1.659924in}}%
\pgfpathlineto{\pgfqpoint{1.181452in}{1.658589in}}%
\pgfpathlineto{\pgfqpoint{1.180953in}{1.657119in}}%
\pgfpathlineto{\pgfqpoint{1.180454in}{1.655514in}}%
\pgfpathlineto{\pgfqpoint{1.179955in}{1.653775in}}%
\pgfpathlineto{\pgfqpoint{1.179955in}{1.653775in}}%
\pgfpathlineto{\pgfqpoint{1.179955in}{1.653775in}}%
\pgfpathlineto{\pgfqpoint{1.179955in}{1.653775in}}%
\pgfpathlineto{\pgfqpoint{1.179955in}{1.653775in}}%
\pgfpathclose%
\pgfusepath{fill}%
\end{pgfscope}%
\begin{pgfscope}%
\pgfpathrectangle{\pgfqpoint{0.041670in}{0.041670in}}{\pgfqpoint{2.216660in}{2.216660in}}%
\pgfusepath{clip}%
\pgfsetbuttcap%
\pgfsetroundjoin%
\definecolor{currentfill}{rgb}{0.993248,0.906157,0.143936}%
\pgfsetfillcolor{currentfill}%
\pgfsetlinewidth{0.000000pt}%
\definecolor{currentstroke}{rgb}{0.000000,0.000000,0.000000}%
\pgfsetstrokecolor{currentstroke}%
\pgfsetdash{}{0pt}%
\pgfpathmoveto{\pgfqpoint{1.194706in}{1.655119in}}%
\pgfpathlineto{\pgfqpoint{1.198392in}{1.655121in}}%
\pgfpathlineto{\pgfqpoint{1.202076in}{1.654990in}}%
\pgfpathlineto{\pgfqpoint{1.205760in}{1.654726in}}%
\pgfpathlineto{\pgfqpoint{1.209442in}{1.654330in}}%
\pgfpathlineto{\pgfqpoint{1.209756in}{1.653894in}}%
\pgfpathlineto{\pgfqpoint{1.210041in}{1.653453in}}%
\pgfpathlineto{\pgfqpoint{1.210295in}{1.653009in}}%
\pgfpathlineto{\pgfqpoint{1.210520in}{1.652561in}}%
\pgfpathlineto{\pgfqpoint{1.206702in}{1.653179in}}%
\pgfpathlineto{\pgfqpoint{1.202883in}{1.653664in}}%
\pgfpathlineto{\pgfqpoint{1.199064in}{1.654016in}}%
\pgfpathlineto{\pgfqpoint{1.195243in}{1.654235in}}%
\pgfpathlineto{\pgfqpoint{1.195131in}{1.654459in}}%
\pgfpathlineto{\pgfqpoint{1.195005in}{1.654681in}}%
\pgfpathlineto{\pgfqpoint{1.194863in}{1.654901in}}%
\pgfpathlineto{\pgfqpoint{1.194706in}{1.655119in}}%
\pgfpathclose%
\pgfusepath{fill}%
\end{pgfscope}%
\begin{pgfscope}%
\pgfpathrectangle{\pgfqpoint{0.041670in}{0.041670in}}{\pgfqpoint{2.216660in}{2.216660in}}%
\pgfusepath{clip}%
\pgfsetbuttcap%
\pgfsetroundjoin%
\definecolor{currentfill}{rgb}{0.163625,0.471133,0.558148}%
\pgfsetfillcolor{currentfill}%
\pgfsetlinewidth{0.000000pt}%
\definecolor{currentstroke}{rgb}{0.000000,0.000000,0.000000}%
\pgfsetstrokecolor{currentstroke}%
\pgfsetdash{}{0pt}%
\pgfpathmoveto{\pgfqpoint{1.396112in}{1.054015in}}%
\pgfpathlineto{\pgfqpoint{1.398429in}{1.044676in}}%
\pgfpathlineto{\pgfqpoint{1.400745in}{1.035349in}}%
\pgfpathlineto{\pgfqpoint{1.403060in}{1.026038in}}%
\pgfpathlineto{\pgfqpoint{1.405375in}{1.016745in}}%
\pgfpathlineto{\pgfqpoint{1.395923in}{1.013177in}}%
\pgfpathlineto{\pgfqpoint{1.386241in}{1.009763in}}%
\pgfpathlineto{\pgfqpoint{1.376340in}{1.006505in}}%
\pgfpathlineto{\pgfqpoint{1.366230in}{1.003408in}}%
\pgfpathlineto{\pgfqpoint{1.364313in}{1.012855in}}%
\pgfpathlineto{\pgfqpoint{1.362396in}{1.022321in}}%
\pgfpathlineto{\pgfqpoint{1.360479in}{1.031802in}}%
\pgfpathlineto{\pgfqpoint{1.358561in}{1.041295in}}%
\pgfpathlineto{\pgfqpoint{1.368259in}{1.044249in}}%
\pgfpathlineto{\pgfqpoint{1.377756in}{1.047356in}}%
\pgfpathlineto{\pgfqpoint{1.387044in}{1.050613in}}%
\pgfpathlineto{\pgfqpoint{1.396112in}{1.054015in}}%
\pgfpathclose%
\pgfusepath{fill}%
\end{pgfscope}%
\begin{pgfscope}%
\pgfpathrectangle{\pgfqpoint{0.041670in}{0.041670in}}{\pgfqpoint{2.216660in}{2.216660in}}%
\pgfusepath{clip}%
\pgfsetbuttcap%
\pgfsetroundjoin%
\definecolor{currentfill}{rgb}{0.993248,0.906157,0.143936}%
\pgfsetfillcolor{currentfill}%
\pgfsetlinewidth{0.000000pt}%
\definecolor{currentstroke}{rgb}{0.000000,0.000000,0.000000}%
\pgfsetstrokecolor{currentstroke}%
\pgfsetdash{}{0pt}%
\pgfpathmoveto{\pgfqpoint{1.179955in}{1.653775in}}%
\pgfpathlineto{\pgfqpoint{1.178985in}{1.655472in}}%
\pgfpathlineto{\pgfqpoint{1.178015in}{1.657035in}}%
\pgfpathlineto{\pgfqpoint{1.177046in}{1.658463in}}%
\pgfpathlineto{\pgfqpoint{1.176077in}{1.659756in}}%
\pgfpathlineto{\pgfqpoint{1.176557in}{1.659810in}}%
\pgfpathlineto{\pgfqpoint{1.177040in}{1.659856in}}%
\pgfpathlineto{\pgfqpoint{1.177526in}{1.659895in}}%
\pgfpathlineto{\pgfqpoint{1.178014in}{1.659927in}}%
\pgfpathlineto{\pgfqpoint{1.178499in}{1.658591in}}%
\pgfpathlineto{\pgfqpoint{1.178984in}{1.657120in}}%
\pgfpathlineto{\pgfqpoint{1.179469in}{1.655515in}}%
\pgfpathlineto{\pgfqpoint{1.179955in}{1.653775in}}%
\pgfpathlineto{\pgfqpoint{1.179955in}{1.653775in}}%
\pgfpathlineto{\pgfqpoint{1.179955in}{1.653775in}}%
\pgfpathlineto{\pgfqpoint{1.179955in}{1.653775in}}%
\pgfpathlineto{\pgfqpoint{1.179955in}{1.653775in}}%
\pgfpathclose%
\pgfusepath{fill}%
\end{pgfscope}%
\begin{pgfscope}%
\pgfpathrectangle{\pgfqpoint{0.041670in}{0.041670in}}{\pgfqpoint{2.216660in}{2.216660in}}%
\pgfusepath{clip}%
\pgfsetbuttcap%
\pgfsetroundjoin%
\definecolor{currentfill}{rgb}{0.993248,0.906157,0.143936}%
\pgfsetfillcolor{currentfill}%
\pgfsetlinewidth{0.000000pt}%
\definecolor{currentstroke}{rgb}{0.000000,0.000000,0.000000}%
\pgfsetstrokecolor{currentstroke}%
\pgfsetdash{}{0pt}%
\pgfpathmoveto{\pgfqpoint{1.179955in}{1.653775in}}%
\pgfpathlineto{\pgfqpoint{1.180454in}{1.655514in}}%
\pgfpathlineto{\pgfqpoint{1.180953in}{1.657119in}}%
\pgfpathlineto{\pgfqpoint{1.181452in}{1.658589in}}%
\pgfpathlineto{\pgfqpoint{1.181950in}{1.659924in}}%
\pgfpathlineto{\pgfqpoint{1.182438in}{1.659891in}}%
\pgfpathlineto{\pgfqpoint{1.182924in}{1.659851in}}%
\pgfpathlineto{\pgfqpoint{1.183406in}{1.659804in}}%
\pgfpathlineto{\pgfqpoint{1.183886in}{1.659750in}}%
\pgfpathlineto{\pgfqpoint{1.182904in}{1.658458in}}%
\pgfpathlineto{\pgfqpoint{1.181921in}{1.657031in}}%
\pgfpathlineto{\pgfqpoint{1.180938in}{1.655471in}}%
\pgfpathlineto{\pgfqpoint{1.179955in}{1.653775in}}%
\pgfpathlineto{\pgfqpoint{1.179955in}{1.653775in}}%
\pgfpathlineto{\pgfqpoint{1.179955in}{1.653775in}}%
\pgfpathlineto{\pgfqpoint{1.179955in}{1.653775in}}%
\pgfpathlineto{\pgfqpoint{1.179955in}{1.653775in}}%
\pgfpathclose%
\pgfusepath{fill}%
\end{pgfscope}%
\begin{pgfscope}%
\pgfpathrectangle{\pgfqpoint{0.041670in}{0.041670in}}{\pgfqpoint{2.216660in}{2.216660in}}%
\pgfusepath{clip}%
\pgfsetbuttcap%
\pgfsetroundjoin%
\definecolor{currentfill}{rgb}{0.487026,0.823929,0.312321}%
\pgfsetfillcolor{currentfill}%
\pgfsetlinewidth{0.000000pt}%
\definecolor{currentstroke}{rgb}{0.000000,0.000000,0.000000}%
\pgfsetstrokecolor{currentstroke}%
\pgfsetdash{}{0pt}%
\pgfpathmoveto{\pgfqpoint{1.022320in}{1.459534in}}%
\pgfpathlineto{\pgfqpoint{1.019056in}{1.452713in}}%
\pgfpathlineto{\pgfqpoint{1.015793in}{1.445802in}}%
\pgfpathlineto{\pgfqpoint{1.012532in}{1.438801in}}%
\pgfpathlineto{\pgfqpoint{1.009272in}{1.431712in}}%
\pgfpathlineto{\pgfqpoint{1.005938in}{1.434333in}}%
\pgfpathlineto{\pgfqpoint{1.002781in}{1.437003in}}%
\pgfpathlineto{\pgfqpoint{0.999805in}{1.439719in}}%
\pgfpathlineto{\pgfqpoint{0.997011in}{1.442477in}}%
\pgfpathlineto{\pgfqpoint{1.000500in}{1.449349in}}%
\pgfpathlineto{\pgfqpoint{1.003990in}{1.456134in}}%
\pgfpathlineto{\pgfqpoint{1.007482in}{1.462830in}}%
\pgfpathlineto{\pgfqpoint{1.010976in}{1.469435in}}%
\pgfpathlineto{\pgfqpoint{1.013561in}{1.466898in}}%
\pgfpathlineto{\pgfqpoint{1.016316in}{1.464400in}}%
\pgfpathlineto{\pgfqpoint{1.019236in}{1.461945in}}%
\pgfpathlineto{\pgfqpoint{1.022320in}{1.459534in}}%
\pgfpathclose%
\pgfusepath{fill}%
\end{pgfscope}%
\begin{pgfscope}%
\pgfpathrectangle{\pgfqpoint{0.041670in}{0.041670in}}{\pgfqpoint{2.216660in}{2.216660in}}%
\pgfusepath{clip}%
\pgfsetbuttcap%
\pgfsetroundjoin%
\definecolor{currentfill}{rgb}{0.993248,0.906157,0.143936}%
\pgfsetfillcolor{currentfill}%
\pgfsetlinewidth{0.000000pt}%
\definecolor{currentstroke}{rgb}{0.000000,0.000000,0.000000}%
\pgfsetstrokecolor{currentstroke}%
\pgfsetdash{}{0pt}%
\pgfpathmoveto{\pgfqpoint{1.179955in}{1.653775in}}%
\pgfpathlineto{\pgfqpoint{1.180938in}{1.655471in}}%
\pgfpathlineto{\pgfqpoint{1.181921in}{1.657031in}}%
\pgfpathlineto{\pgfqpoint{1.182904in}{1.658458in}}%
\pgfpathlineto{\pgfqpoint{1.183886in}{1.659750in}}%
\pgfpathlineto{\pgfqpoint{1.184361in}{1.659689in}}%
\pgfpathlineto{\pgfqpoint{1.184832in}{1.659620in}}%
\pgfpathlineto{\pgfqpoint{1.185298in}{1.659545in}}%
\pgfpathlineto{\pgfqpoint{1.183964in}{1.658304in}}%
\pgfpathlineto{\pgfqpoint{1.182628in}{1.656929in}}%
\pgfpathlineto{\pgfqpoint{1.181292in}{1.655419in}}%
\pgfpathlineto{\pgfqpoint{1.179955in}{1.653775in}}%
\pgfpathlineto{\pgfqpoint{1.179955in}{1.653775in}}%
\pgfpathlineto{\pgfqpoint{1.179955in}{1.653775in}}%
\pgfpathlineto{\pgfqpoint{1.179955in}{1.653775in}}%
\pgfpathclose%
\pgfusepath{fill}%
\end{pgfscope}%
\begin{pgfscope}%
\pgfpathrectangle{\pgfqpoint{0.041670in}{0.041670in}}{\pgfqpoint{2.216660in}{2.216660in}}%
\pgfusepath{clip}%
\pgfsetbuttcap%
\pgfsetroundjoin%
\definecolor{currentfill}{rgb}{0.993248,0.906157,0.143936}%
\pgfsetfillcolor{currentfill}%
\pgfsetlinewidth{0.000000pt}%
\definecolor{currentstroke}{rgb}{0.000000,0.000000,0.000000}%
\pgfsetstrokecolor{currentstroke}%
\pgfsetdash{}{0pt}%
\pgfpathmoveto{\pgfqpoint{1.179955in}{1.653775in}}%
\pgfpathlineto{\pgfqpoint{1.178515in}{1.655401in}}%
\pgfpathlineto{\pgfqpoint{1.177077in}{1.656893in}}%
\pgfpathlineto{\pgfqpoint{1.175639in}{1.658250in}}%
\pgfpathlineto{\pgfqpoint{1.174201in}{1.659473in}}%
\pgfpathlineto{\pgfqpoint{1.174663in}{1.659554in}}%
\pgfpathlineto{\pgfqpoint{1.175130in}{1.659628in}}%
\pgfpathlineto{\pgfqpoint{1.175601in}{1.659696in}}%
\pgfpathlineto{\pgfqpoint{1.176077in}{1.659756in}}%
\pgfpathlineto{\pgfqpoint{1.177046in}{1.658463in}}%
\pgfpathlineto{\pgfqpoint{1.178015in}{1.657035in}}%
\pgfpathlineto{\pgfqpoint{1.178985in}{1.655472in}}%
\pgfpathlineto{\pgfqpoint{1.179955in}{1.653775in}}%
\pgfpathlineto{\pgfqpoint{1.179955in}{1.653775in}}%
\pgfpathlineto{\pgfqpoint{1.179955in}{1.653775in}}%
\pgfpathlineto{\pgfqpoint{1.179955in}{1.653775in}}%
\pgfpathlineto{\pgfqpoint{1.179955in}{1.653775in}}%
\pgfpathclose%
\pgfusepath{fill}%
\end{pgfscope}%
\begin{pgfscope}%
\pgfpathrectangle{\pgfqpoint{0.041670in}{0.041670in}}{\pgfqpoint{2.216660in}{2.216660in}}%
\pgfusepath{clip}%
\pgfsetbuttcap%
\pgfsetroundjoin%
\definecolor{currentfill}{rgb}{0.993248,0.906157,0.143936}%
\pgfsetfillcolor{currentfill}%
\pgfsetlinewidth{0.000000pt}%
\definecolor{currentstroke}{rgb}{0.000000,0.000000,0.000000}%
\pgfsetstrokecolor{currentstroke}%
\pgfsetdash{}{0pt}%
\pgfpathmoveto{\pgfqpoint{1.179955in}{1.653775in}}%
\pgfpathlineto{\pgfqpoint{1.181292in}{1.655419in}}%
\pgfpathlineto{\pgfqpoint{1.182628in}{1.656929in}}%
\pgfpathlineto{\pgfqpoint{1.183964in}{1.658304in}}%
\pgfpathlineto{\pgfqpoint{1.185298in}{1.659545in}}%
\pgfpathlineto{\pgfqpoint{1.185759in}{1.659463in}}%
\pgfpathlineto{\pgfqpoint{1.186214in}{1.659375in}}%
\pgfpathlineto{\pgfqpoint{1.186663in}{1.659280in}}%
\pgfpathlineto{\pgfqpoint{1.187105in}{1.659178in}}%
\pgfpathlineto{\pgfqpoint{1.185319in}{1.658028in}}%
\pgfpathlineto{\pgfqpoint{1.183532in}{1.656745in}}%
\pgfpathlineto{\pgfqpoint{1.181744in}{1.655327in}}%
\pgfpathlineto{\pgfqpoint{1.179955in}{1.653775in}}%
\pgfpathlineto{\pgfqpoint{1.179955in}{1.653775in}}%
\pgfpathlineto{\pgfqpoint{1.179955in}{1.653775in}}%
\pgfpathlineto{\pgfqpoint{1.179955in}{1.653775in}}%
\pgfpathlineto{\pgfqpoint{1.179955in}{1.653775in}}%
\pgfpathclose%
\pgfusepath{fill}%
\end{pgfscope}%
\begin{pgfscope}%
\pgfpathrectangle{\pgfqpoint{0.041670in}{0.041670in}}{\pgfqpoint{2.216660in}{2.216660in}}%
\pgfusepath{clip}%
\pgfsetbuttcap%
\pgfsetroundjoin%
\definecolor{currentfill}{rgb}{0.993248,0.906157,0.143936}%
\pgfsetfillcolor{currentfill}%
\pgfsetlinewidth{0.000000pt}%
\definecolor{currentstroke}{rgb}{0.000000,0.000000,0.000000}%
\pgfsetstrokecolor{currentstroke}%
\pgfsetdash{}{0pt}%
\pgfpathmoveto{\pgfqpoint{1.164580in}{1.654034in}}%
\pgfpathlineto{\pgfqpoint{1.160738in}{1.653765in}}%
\pgfpathlineto{\pgfqpoint{1.156897in}{1.653363in}}%
\pgfpathlineto{\pgfqpoint{1.153056in}{1.652828in}}%
\pgfpathlineto{\pgfqpoint{1.149216in}{1.652160in}}%
\pgfpathlineto{\pgfqpoint{1.149414in}{1.652611in}}%
\pgfpathlineto{\pgfqpoint{1.149641in}{1.653059in}}%
\pgfpathlineto{\pgfqpoint{1.149899in}{1.653502in}}%
\pgfpathlineto{\pgfqpoint{1.150187in}{1.653942in}}%
\pgfpathlineto{\pgfqpoint{1.153905in}{1.654387in}}%
\pgfpathlineto{\pgfqpoint{1.157623in}{1.654700in}}%
\pgfpathlineto{\pgfqpoint{1.161343in}{1.654879in}}%
\pgfpathlineto{\pgfqpoint{1.165064in}{1.654925in}}%
\pgfpathlineto{\pgfqpoint{1.164920in}{1.654705in}}%
\pgfpathlineto{\pgfqpoint{1.164792in}{1.654483in}}%
\pgfpathlineto{\pgfqpoint{1.164678in}{1.654260in}}%
\pgfpathlineto{\pgfqpoint{1.164580in}{1.654034in}}%
\pgfpathclose%
\pgfusepath{fill}%
\end{pgfscope}%
\begin{pgfscope}%
\pgfpathrectangle{\pgfqpoint{0.041670in}{0.041670in}}{\pgfqpoint{2.216660in}{2.216660in}}%
\pgfusepath{clip}%
\pgfsetbuttcap%
\pgfsetroundjoin%
\definecolor{currentfill}{rgb}{0.993248,0.906157,0.143936}%
\pgfsetfillcolor{currentfill}%
\pgfsetlinewidth{0.000000pt}%
\definecolor{currentstroke}{rgb}{0.000000,0.000000,0.000000}%
\pgfsetstrokecolor{currentstroke}%
\pgfsetdash{}{0pt}%
\pgfpathmoveto{\pgfqpoint{1.179955in}{1.653775in}}%
\pgfpathlineto{\pgfqpoint{1.178069in}{1.655303in}}%
\pgfpathlineto{\pgfqpoint{1.176184in}{1.656697in}}%
\pgfpathlineto{\pgfqpoint{1.174300in}{1.657956in}}%
\pgfpathlineto{\pgfqpoint{1.172417in}{1.659082in}}%
\pgfpathlineto{\pgfqpoint{1.172853in}{1.659189in}}%
\pgfpathlineto{\pgfqpoint{1.173296in}{1.659290in}}%
\pgfpathlineto{\pgfqpoint{1.173746in}{1.659385in}}%
\pgfpathlineto{\pgfqpoint{1.174201in}{1.659473in}}%
\pgfpathlineto{\pgfqpoint{1.175639in}{1.658250in}}%
\pgfpathlineto{\pgfqpoint{1.177077in}{1.656893in}}%
\pgfpathlineto{\pgfqpoint{1.178515in}{1.655401in}}%
\pgfpathlineto{\pgfqpoint{1.179955in}{1.653775in}}%
\pgfpathlineto{\pgfqpoint{1.179955in}{1.653775in}}%
\pgfpathlineto{\pgfqpoint{1.179955in}{1.653775in}}%
\pgfpathlineto{\pgfqpoint{1.179955in}{1.653775in}}%
\pgfpathlineto{\pgfqpoint{1.179955in}{1.653775in}}%
\pgfpathclose%
\pgfusepath{fill}%
\end{pgfscope}%
\begin{pgfscope}%
\pgfpathrectangle{\pgfqpoint{0.041670in}{0.041670in}}{\pgfqpoint{2.216660in}{2.216660in}}%
\pgfusepath{clip}%
\pgfsetbuttcap%
\pgfsetroundjoin%
\definecolor{currentfill}{rgb}{0.274952,0.037752,0.364543}%
\pgfsetfillcolor{currentfill}%
\pgfsetlinewidth{0.000000pt}%
\definecolor{currentstroke}{rgb}{0.000000,0.000000,0.000000}%
\pgfsetstrokecolor{currentstroke}%
\pgfsetdash{}{0pt}%
\pgfpathmoveto{\pgfqpoint{0.984574in}{0.636469in}}%
\pgfpathlineto{\pgfqpoint{0.983158in}{0.631220in}}%
\pgfpathlineto{\pgfqpoint{0.981740in}{0.626157in}}%
\pgfpathlineto{\pgfqpoint{0.980320in}{0.621284in}}%
\pgfpathlineto{\pgfqpoint{0.978898in}{0.616605in}}%
\pgfpathlineto{\pgfqpoint{0.962485in}{0.620177in}}%
\pgfpathlineto{\pgfqpoint{0.946313in}{0.624025in}}%
\pgfpathlineto{\pgfqpoint{0.930399in}{0.628145in}}%
\pgfpathlineto{\pgfqpoint{0.914761in}{0.632532in}}%
\pgfpathlineto{\pgfqpoint{0.916633in}{0.637088in}}%
\pgfpathlineto{\pgfqpoint{0.918502in}{0.641839in}}%
\pgfpathlineto{\pgfqpoint{0.920368in}{0.646780in}}%
\pgfpathlineto{\pgfqpoint{0.922232in}{0.651906in}}%
\pgfpathlineto{\pgfqpoint{0.937433in}{0.647655in}}%
\pgfpathlineto{\pgfqpoint{0.952902in}{0.643661in}}%
\pgfpathlineto{\pgfqpoint{0.968621in}{0.639932in}}%
\pgfpathlineto{\pgfqpoint{0.984574in}{0.636469in}}%
\pgfpathclose%
\pgfusepath{fill}%
\end{pgfscope}%
\begin{pgfscope}%
\pgfpathrectangle{\pgfqpoint{0.041670in}{0.041670in}}{\pgfqpoint{2.216660in}{2.216660in}}%
\pgfusepath{clip}%
\pgfsetbuttcap%
\pgfsetroundjoin%
\definecolor{currentfill}{rgb}{0.271305,0.019942,0.347269}%
\pgfsetfillcolor{currentfill}%
\pgfsetlinewidth{0.000000pt}%
\definecolor{currentstroke}{rgb}{0.000000,0.000000,0.000000}%
\pgfsetstrokecolor{currentstroke}%
\pgfsetdash{}{0pt}%
\pgfpathmoveto{\pgfqpoint{1.458804in}{0.636650in}}%
\pgfpathlineto{\pgfqpoint{1.460774in}{0.632324in}}%
\pgfpathlineto{\pgfqpoint{1.462748in}{0.628201in}}%
\pgfpathlineto{\pgfqpoint{1.464725in}{0.624287in}}%
\pgfpathlineto{\pgfqpoint{1.466706in}{0.620584in}}%
\pgfpathlineto{\pgfqpoint{1.450897in}{0.615824in}}%
\pgfpathlineto{\pgfqpoint{1.434787in}{0.611334in}}%
\pgfpathlineto{\pgfqpoint{1.418395in}{0.607119in}}%
\pgfpathlineto{\pgfqpoint{1.401739in}{0.603184in}}%
\pgfpathlineto{\pgfqpoint{1.400203in}{0.607015in}}%
\pgfpathlineto{\pgfqpoint{1.398670in}{0.611059in}}%
\pgfpathlineto{\pgfqpoint{1.397140in}{0.615311in}}%
\pgfpathlineto{\pgfqpoint{1.395613in}{0.619766in}}%
\pgfpathlineto{\pgfqpoint{1.411813in}{0.623584in}}%
\pgfpathlineto{\pgfqpoint{1.427756in}{0.627674in}}%
\pgfpathlineto{\pgfqpoint{1.443425in}{0.632031in}}%
\pgfpathlineto{\pgfqpoint{1.458804in}{0.636650in}}%
\pgfpathclose%
\pgfusepath{fill}%
\end{pgfscope}%
\begin{pgfscope}%
\pgfpathrectangle{\pgfqpoint{0.041670in}{0.041670in}}{\pgfqpoint{2.216660in}{2.216660in}}%
\pgfusepath{clip}%
\pgfsetbuttcap%
\pgfsetroundjoin%
\definecolor{currentfill}{rgb}{0.993248,0.906157,0.143936}%
\pgfsetfillcolor{currentfill}%
\pgfsetlinewidth{0.000000pt}%
\definecolor{currentstroke}{rgb}{0.000000,0.000000,0.000000}%
\pgfsetstrokecolor{currentstroke}%
\pgfsetdash{}{0pt}%
\pgfpathmoveto{\pgfqpoint{1.179955in}{1.653775in}}%
\pgfpathlineto{\pgfqpoint{1.181744in}{1.655327in}}%
\pgfpathlineto{\pgfqpoint{1.183532in}{1.656745in}}%
\pgfpathlineto{\pgfqpoint{1.185319in}{1.658028in}}%
\pgfpathlineto{\pgfqpoint{1.187105in}{1.659178in}}%
\pgfpathlineto{\pgfqpoint{1.187541in}{1.659069in}}%
\pgfpathlineto{\pgfqpoint{1.187968in}{1.658955in}}%
\pgfpathlineto{\pgfqpoint{1.188388in}{1.658834in}}%
\pgfpathlineto{\pgfqpoint{1.188799in}{1.658707in}}%
\pgfpathlineto{\pgfqpoint{1.186590in}{1.657675in}}%
\pgfpathlineto{\pgfqpoint{1.184379in}{1.656510in}}%
\pgfpathlineto{\pgfqpoint{1.182168in}{1.655210in}}%
\pgfpathlineto{\pgfqpoint{1.179955in}{1.653775in}}%
\pgfpathlineto{\pgfqpoint{1.179955in}{1.653775in}}%
\pgfpathlineto{\pgfqpoint{1.179955in}{1.653775in}}%
\pgfpathlineto{\pgfqpoint{1.179955in}{1.653775in}}%
\pgfpathlineto{\pgfqpoint{1.179955in}{1.653775in}}%
\pgfpathclose%
\pgfusepath{fill}%
\end{pgfscope}%
\begin{pgfscope}%
\pgfpathrectangle{\pgfqpoint{0.041670in}{0.041670in}}{\pgfqpoint{2.216660in}{2.216660in}}%
\pgfusepath{clip}%
\pgfsetbuttcap%
\pgfsetroundjoin%
\definecolor{currentfill}{rgb}{0.993248,0.906157,0.143936}%
\pgfsetfillcolor{currentfill}%
\pgfsetlinewidth{0.000000pt}%
\definecolor{currentstroke}{rgb}{0.000000,0.000000,0.000000}%
\pgfsetstrokecolor{currentstroke}%
\pgfsetdash{}{0pt}%
\pgfpathmoveto{\pgfqpoint{1.179955in}{1.653775in}}%
\pgfpathlineto{\pgfqpoint{1.177653in}{1.655180in}}%
\pgfpathlineto{\pgfqpoint{1.175351in}{1.656451in}}%
\pgfpathlineto{\pgfqpoint{1.173051in}{1.657587in}}%
\pgfpathlineto{\pgfqpoint{1.170753in}{1.658589in}}%
\pgfpathlineto{\pgfqpoint{1.171156in}{1.658721in}}%
\pgfpathlineto{\pgfqpoint{1.171568in}{1.658848in}}%
\pgfpathlineto{\pgfqpoint{1.171989in}{1.658968in}}%
\pgfpathlineto{\pgfqpoint{1.172417in}{1.659082in}}%
\pgfpathlineto{\pgfqpoint{1.174300in}{1.657956in}}%
\pgfpathlineto{\pgfqpoint{1.176184in}{1.656697in}}%
\pgfpathlineto{\pgfqpoint{1.178069in}{1.655303in}}%
\pgfpathlineto{\pgfqpoint{1.179955in}{1.653775in}}%
\pgfpathlineto{\pgfqpoint{1.179955in}{1.653775in}}%
\pgfpathlineto{\pgfqpoint{1.179955in}{1.653775in}}%
\pgfpathlineto{\pgfqpoint{1.179955in}{1.653775in}}%
\pgfpathlineto{\pgfqpoint{1.179955in}{1.653775in}}%
\pgfpathclose%
\pgfusepath{fill}%
\end{pgfscope}%
\begin{pgfscope}%
\pgfpathrectangle{\pgfqpoint{0.041670in}{0.041670in}}{\pgfqpoint{2.216660in}{2.216660in}}%
\pgfusepath{clip}%
\pgfsetbuttcap%
\pgfsetroundjoin%
\definecolor{currentfill}{rgb}{0.935904,0.898570,0.108131}%
\pgfsetfillcolor{currentfill}%
\pgfsetlinewidth{0.000000pt}%
\definecolor{currentstroke}{rgb}{0.000000,0.000000,0.000000}%
\pgfsetstrokecolor{currentstroke}%
\pgfsetdash{}{0pt}%
\pgfpathmoveto{\pgfqpoint{1.242426in}{1.635566in}}%
\pgfpathlineto{\pgfqpoint{1.246325in}{1.633314in}}%
\pgfpathlineto{\pgfqpoint{1.250222in}{1.630933in}}%
\pgfpathlineto{\pgfqpoint{1.254118in}{1.628425in}}%
\pgfpathlineto{\pgfqpoint{1.258013in}{1.625790in}}%
\pgfpathlineto{\pgfqpoint{1.257886in}{1.624638in}}%
\pgfpathlineto{\pgfqpoint{1.257682in}{1.623488in}}%
\pgfpathlineto{\pgfqpoint{1.257400in}{1.622341in}}%
\pgfpathlineto{\pgfqpoint{1.257040in}{1.621199in}}%
\pgfpathlineto{\pgfqpoint{1.253191in}{1.624065in}}%
\pgfpathlineto{\pgfqpoint{1.249342in}{1.626804in}}%
\pgfpathlineto{\pgfqpoint{1.245491in}{1.629416in}}%
\pgfpathlineto{\pgfqpoint{1.241639in}{1.631899in}}%
\pgfpathlineto{\pgfqpoint{1.241929in}{1.632811in}}%
\pgfpathlineto{\pgfqpoint{1.242157in}{1.633727in}}%
\pgfpathlineto{\pgfqpoint{1.242323in}{1.634646in}}%
\pgfpathlineto{\pgfqpoint{1.242426in}{1.635566in}}%
\pgfpathclose%
\pgfusepath{fill}%
\end{pgfscope}%
\begin{pgfscope}%
\pgfpathrectangle{\pgfqpoint{0.041670in}{0.041670in}}{\pgfqpoint{2.216660in}{2.216660in}}%
\pgfusepath{clip}%
\pgfsetbuttcap%
\pgfsetroundjoin%
\definecolor{currentfill}{rgb}{0.344074,0.780029,0.397381}%
\pgfsetfillcolor{currentfill}%
\pgfsetlinewidth{0.000000pt}%
\definecolor{currentstroke}{rgb}{0.000000,0.000000,0.000000}%
\pgfsetstrokecolor{currentstroke}%
\pgfsetdash{}{0pt}%
\pgfpathmoveto{\pgfqpoint{1.366854in}{1.405032in}}%
\pgfpathlineto{\pgfqpoint{1.370161in}{1.397580in}}%
\pgfpathlineto{\pgfqpoint{1.373466in}{1.390052in}}%
\pgfpathlineto{\pgfqpoint{1.376769in}{1.382449in}}%
\pgfpathlineto{\pgfqpoint{1.380071in}{1.374774in}}%
\pgfpathlineto{\pgfqpoint{1.376218in}{1.371732in}}%
\pgfpathlineto{\pgfqpoint{1.372164in}{1.368750in}}%
\pgfpathlineto{\pgfqpoint{1.367912in}{1.365830in}}%
\pgfpathlineto{\pgfqpoint{1.363468in}{1.362976in}}%
\pgfpathlineto{\pgfqpoint{1.360436in}{1.370859in}}%
\pgfpathlineto{\pgfqpoint{1.357402in}{1.378670in}}%
\pgfpathlineto{\pgfqpoint{1.354366in}{1.386406in}}%
\pgfpathlineto{\pgfqpoint{1.351329in}{1.394065in}}%
\pgfpathlineto{\pgfqpoint{1.355484in}{1.396718in}}%
\pgfpathlineto{\pgfqpoint{1.359459in}{1.399432in}}%
\pgfpathlineto{\pgfqpoint{1.363250in}{1.402205in}}%
\pgfpathlineto{\pgfqpoint{1.366854in}{1.405032in}}%
\pgfpathclose%
\pgfusepath{fill}%
\end{pgfscope}%
\begin{pgfscope}%
\pgfpathrectangle{\pgfqpoint{0.041670in}{0.041670in}}{\pgfqpoint{2.216660in}{2.216660in}}%
\pgfusepath{clip}%
\pgfsetbuttcap%
\pgfsetroundjoin%
\definecolor{currentfill}{rgb}{0.993248,0.906157,0.143936}%
\pgfsetfillcolor{currentfill}%
\pgfsetlinewidth{0.000000pt}%
\definecolor{currentstroke}{rgb}{0.000000,0.000000,0.000000}%
\pgfsetstrokecolor{currentstroke}%
\pgfsetdash{}{0pt}%
\pgfpathmoveto{\pgfqpoint{1.179955in}{1.653775in}}%
\pgfpathlineto{\pgfqpoint{1.182168in}{1.655210in}}%
\pgfpathlineto{\pgfqpoint{1.184379in}{1.656510in}}%
\pgfpathlineto{\pgfqpoint{1.186590in}{1.657675in}}%
\pgfpathlineto{\pgfqpoint{1.188799in}{1.658707in}}%
\pgfpathlineto{\pgfqpoint{1.189201in}{1.658574in}}%
\pgfpathlineto{\pgfqpoint{1.189595in}{1.658435in}}%
\pgfpathlineto{\pgfqpoint{1.189978in}{1.658291in}}%
\pgfpathlineto{\pgfqpoint{1.190352in}{1.658141in}}%
\pgfpathlineto{\pgfqpoint{1.187755in}{1.657250in}}%
\pgfpathlineto{\pgfqpoint{1.185156in}{1.656226in}}%
\pgfpathlineto{\pgfqpoint{1.182556in}{1.655068in}}%
\pgfpathlineto{\pgfqpoint{1.179955in}{1.653775in}}%
\pgfpathlineto{\pgfqpoint{1.179955in}{1.653775in}}%
\pgfpathlineto{\pgfqpoint{1.179955in}{1.653775in}}%
\pgfpathlineto{\pgfqpoint{1.179955in}{1.653775in}}%
\pgfpathlineto{\pgfqpoint{1.179955in}{1.653775in}}%
\pgfpathclose%
\pgfusepath{fill}%
\end{pgfscope}%
\begin{pgfscope}%
\pgfpathrectangle{\pgfqpoint{0.041670in}{0.041670in}}{\pgfqpoint{2.216660in}{2.216660in}}%
\pgfusepath{clip}%
\pgfsetbuttcap%
\pgfsetroundjoin%
\definecolor{currentfill}{rgb}{0.993248,0.906157,0.143936}%
\pgfsetfillcolor{currentfill}%
\pgfsetlinewidth{0.000000pt}%
\definecolor{currentstroke}{rgb}{0.000000,0.000000,0.000000}%
\pgfsetstrokecolor{currentstroke}%
\pgfsetdash{}{0pt}%
\pgfpathmoveto{\pgfqpoint{1.179955in}{1.653775in}}%
\pgfpathlineto{\pgfqpoint{1.177273in}{1.655033in}}%
\pgfpathlineto{\pgfqpoint{1.174592in}{1.656157in}}%
\pgfpathlineto{\pgfqpoint{1.171913in}{1.657147in}}%
\pgfpathlineto{\pgfqpoint{1.169234in}{1.658003in}}%
\pgfpathlineto{\pgfqpoint{1.169599in}{1.658158in}}%
\pgfpathlineto{\pgfqpoint{1.169974in}{1.658307in}}%
\pgfpathlineto{\pgfqpoint{1.170358in}{1.658451in}}%
\pgfpathlineto{\pgfqpoint{1.170753in}{1.658589in}}%
\pgfpathlineto{\pgfqpoint{1.173051in}{1.657587in}}%
\pgfpathlineto{\pgfqpoint{1.175351in}{1.656451in}}%
\pgfpathlineto{\pgfqpoint{1.177653in}{1.655180in}}%
\pgfpathlineto{\pgfqpoint{1.179955in}{1.653775in}}%
\pgfpathlineto{\pgfqpoint{1.179955in}{1.653775in}}%
\pgfpathlineto{\pgfqpoint{1.179955in}{1.653775in}}%
\pgfpathlineto{\pgfqpoint{1.179955in}{1.653775in}}%
\pgfpathlineto{\pgfqpoint{1.179955in}{1.653775in}}%
\pgfpathclose%
\pgfusepath{fill}%
\end{pgfscope}%
\begin{pgfscope}%
\pgfpathrectangle{\pgfqpoint{0.041670in}{0.041670in}}{\pgfqpoint{2.216660in}{2.216660in}}%
\pgfusepath{clip}%
\pgfsetbuttcap%
\pgfsetroundjoin%
\definecolor{currentfill}{rgb}{0.855810,0.888601,0.097452}%
\pgfsetfillcolor{currentfill}%
\pgfsetlinewidth{0.000000pt}%
\definecolor{currentstroke}{rgb}{0.000000,0.000000,0.000000}%
\pgfsetstrokecolor{currentstroke}%
\pgfsetdash{}{0pt}%
\pgfpathmoveto{\pgfqpoint{1.272425in}{1.608479in}}%
\pgfpathlineto{\pgfqpoint{1.276268in}{1.604989in}}%
\pgfpathlineto{\pgfqpoint{1.280109in}{1.601377in}}%
\pgfpathlineto{\pgfqpoint{1.283949in}{1.597644in}}%
\pgfpathlineto{\pgfqpoint{1.287788in}{1.593790in}}%
\pgfpathlineto{\pgfqpoint{1.287185in}{1.592194in}}%
\pgfpathlineto{\pgfqpoint{1.286474in}{1.590607in}}%
\pgfpathlineto{\pgfqpoint{1.285657in}{1.589031in}}%
\pgfpathlineto{\pgfqpoint{1.284734in}{1.587467in}}%
\pgfpathlineto{\pgfqpoint{1.281001in}{1.591550in}}%
\pgfpathlineto{\pgfqpoint{1.277266in}{1.595512in}}%
\pgfpathlineto{\pgfqpoint{1.273530in}{1.599352in}}%
\pgfpathlineto{\pgfqpoint{1.269793in}{1.603070in}}%
\pgfpathlineto{\pgfqpoint{1.270588in}{1.604407in}}%
\pgfpathlineto{\pgfqpoint{1.271292in}{1.605755in}}%
\pgfpathlineto{\pgfqpoint{1.271904in}{1.607113in}}%
\pgfpathlineto{\pgfqpoint{1.272425in}{1.608479in}}%
\pgfpathclose%
\pgfusepath{fill}%
\end{pgfscope}%
\begin{pgfscope}%
\pgfpathrectangle{\pgfqpoint{0.041670in}{0.041670in}}{\pgfqpoint{2.216660in}{2.216660in}}%
\pgfusepath{clip}%
\pgfsetbuttcap%
\pgfsetroundjoin%
\definecolor{currentfill}{rgb}{0.993248,0.906157,0.143936}%
\pgfsetfillcolor{currentfill}%
\pgfsetlinewidth{0.000000pt}%
\definecolor{currentstroke}{rgb}{0.000000,0.000000,0.000000}%
\pgfsetstrokecolor{currentstroke}%
\pgfsetdash{}{0pt}%
\pgfpathmoveto{\pgfqpoint{1.179955in}{1.653775in}}%
\pgfpathlineto{\pgfqpoint{1.182556in}{1.655068in}}%
\pgfpathlineto{\pgfqpoint{1.185156in}{1.656226in}}%
\pgfpathlineto{\pgfqpoint{1.187755in}{1.657250in}}%
\pgfpathlineto{\pgfqpoint{1.190352in}{1.658141in}}%
\pgfpathlineto{\pgfqpoint{1.190715in}{1.657985in}}%
\pgfpathlineto{\pgfqpoint{1.191068in}{1.657825in}}%
\pgfpathlineto{\pgfqpoint{1.191410in}{1.657659in}}%
\pgfpathlineto{\pgfqpoint{1.191740in}{1.657488in}}%
\pgfpathlineto{\pgfqpoint{1.188795in}{1.656761in}}%
\pgfpathlineto{\pgfqpoint{1.185850in}{1.655900in}}%
\pgfpathlineto{\pgfqpoint{1.182903in}{1.654904in}}%
\pgfpathlineto{\pgfqpoint{1.179955in}{1.653775in}}%
\pgfpathlineto{\pgfqpoint{1.179955in}{1.653775in}}%
\pgfpathlineto{\pgfqpoint{1.179955in}{1.653775in}}%
\pgfpathlineto{\pgfqpoint{1.179955in}{1.653775in}}%
\pgfpathlineto{\pgfqpoint{1.179955in}{1.653775in}}%
\pgfpathclose%
\pgfusepath{fill}%
\end{pgfscope}%
\begin{pgfscope}%
\pgfpathrectangle{\pgfqpoint{0.041670in}{0.041670in}}{\pgfqpoint{2.216660in}{2.216660in}}%
\pgfusepath{clip}%
\pgfsetbuttcap%
\pgfsetroundjoin%
\definecolor{currentfill}{rgb}{0.993248,0.906157,0.143936}%
\pgfsetfillcolor{currentfill}%
\pgfsetlinewidth{0.000000pt}%
\definecolor{currentstroke}{rgb}{0.000000,0.000000,0.000000}%
\pgfsetstrokecolor{currentstroke}%
\pgfsetdash{}{0pt}%
\pgfpathmoveto{\pgfqpoint{1.179955in}{1.653775in}}%
\pgfpathlineto{\pgfqpoint{1.176936in}{1.654865in}}%
\pgfpathlineto{\pgfqpoint{1.173918in}{1.655822in}}%
\pgfpathlineto{\pgfqpoint{1.170902in}{1.656644in}}%
\pgfpathlineto{\pgfqpoint{1.167886in}{1.657332in}}%
\pgfpathlineto{\pgfqpoint{1.168206in}{1.657507in}}%
\pgfpathlineto{\pgfqpoint{1.168538in}{1.657677in}}%
\pgfpathlineto{\pgfqpoint{1.168880in}{1.657843in}}%
\pgfpathlineto{\pgfqpoint{1.169234in}{1.658003in}}%
\pgfpathlineto{\pgfqpoint{1.171913in}{1.657147in}}%
\pgfpathlineto{\pgfqpoint{1.174592in}{1.656157in}}%
\pgfpathlineto{\pgfqpoint{1.177273in}{1.655033in}}%
\pgfpathlineto{\pgfqpoint{1.179955in}{1.653775in}}%
\pgfpathlineto{\pgfqpoint{1.179955in}{1.653775in}}%
\pgfpathlineto{\pgfqpoint{1.179955in}{1.653775in}}%
\pgfpathlineto{\pgfqpoint{1.179955in}{1.653775in}}%
\pgfpathlineto{\pgfqpoint{1.179955in}{1.653775in}}%
\pgfpathclose%
\pgfusepath{fill}%
\end{pgfscope}%
\begin{pgfscope}%
\pgfpathrectangle{\pgfqpoint{0.041670in}{0.041670in}}{\pgfqpoint{2.216660in}{2.216660in}}%
\pgfusepath{clip}%
\pgfsetbuttcap%
\pgfsetroundjoin%
\definecolor{currentfill}{rgb}{0.699415,0.867117,0.175971}%
\pgfsetfillcolor{currentfill}%
\pgfsetlinewidth{0.000000pt}%
\definecolor{currentstroke}{rgb}{0.000000,0.000000,0.000000}%
\pgfsetstrokecolor{currentstroke}%
\pgfsetdash{}{0pt}%
\pgfpathmoveto{\pgfqpoint{1.053035in}{1.541019in}}%
\pgfpathlineto{\pgfqpoint{1.049521in}{1.535629in}}%
\pgfpathlineto{\pgfqpoint{1.046010in}{1.530129in}}%
\pgfpathlineto{\pgfqpoint{1.042499in}{1.524522in}}%
\pgfpathlineto{\pgfqpoint{1.038990in}{1.518808in}}%
\pgfpathlineto{\pgfqpoint{1.036967in}{1.520939in}}%
\pgfpathlineto{\pgfqpoint{1.035089in}{1.523099in}}%
\pgfpathlineto{\pgfqpoint{1.033356in}{1.525285in}}%
\pgfpathlineto{\pgfqpoint{1.031772in}{1.527496in}}%
\pgfpathlineto{\pgfqpoint{1.035456in}{1.532987in}}%
\pgfpathlineto{\pgfqpoint{1.039141in}{1.538371in}}%
\pgfpathlineto{\pgfqpoint{1.042829in}{1.543648in}}%
\pgfpathlineto{\pgfqpoint{1.046518in}{1.548817in}}%
\pgfpathlineto{\pgfqpoint{1.047949in}{1.546833in}}%
\pgfpathlineto{\pgfqpoint{1.049513in}{1.544870in}}%
\pgfpathlineto{\pgfqpoint{1.051209in}{1.542932in}}%
\pgfpathlineto{\pgfqpoint{1.053035in}{1.541019in}}%
\pgfpathclose%
\pgfusepath{fill}%
\end{pgfscope}%
\begin{pgfscope}%
\pgfpathrectangle{\pgfqpoint{0.041670in}{0.041670in}}{\pgfqpoint{2.216660in}{2.216660in}}%
\pgfusepath{clip}%
\pgfsetbuttcap%
\pgfsetroundjoin%
\definecolor{currentfill}{rgb}{0.974417,0.903590,0.130215}%
\pgfsetfillcolor{currentfill}%
\pgfsetlinewidth{0.000000pt}%
\definecolor{currentstroke}{rgb}{0.000000,0.000000,0.000000}%
\pgfsetstrokecolor{currentstroke}%
\pgfsetdash{}{0pt}%
\pgfpathmoveto{\pgfqpoint{1.211111in}{1.650743in}}%
\pgfpathlineto{\pgfqpoint{1.215003in}{1.649765in}}%
\pgfpathlineto{\pgfqpoint{1.218894in}{1.648656in}}%
\pgfpathlineto{\pgfqpoint{1.222784in}{1.647415in}}%
\pgfpathlineto{\pgfqpoint{1.226673in}{1.646042in}}%
\pgfpathlineto{\pgfqpoint{1.226780in}{1.645354in}}%
\pgfpathlineto{\pgfqpoint{1.226841in}{1.644664in}}%
\pgfpathlineto{\pgfqpoint{1.226855in}{1.643973in}}%
\pgfpathlineto{\pgfqpoint{1.226823in}{1.643282in}}%
\pgfpathlineto{\pgfqpoint{1.222920in}{1.644885in}}%
\pgfpathlineto{\pgfqpoint{1.219016in}{1.646357in}}%
\pgfpathlineto{\pgfqpoint{1.215111in}{1.647696in}}%
\pgfpathlineto{\pgfqpoint{1.211206in}{1.648904in}}%
\pgfpathlineto{\pgfqpoint{1.211229in}{1.649364in}}%
\pgfpathlineto{\pgfqpoint{1.211221in}{1.649824in}}%
\pgfpathlineto{\pgfqpoint{1.211182in}{1.650284in}}%
\pgfpathlineto{\pgfqpoint{1.211111in}{1.650743in}}%
\pgfpathclose%
\pgfusepath{fill}%
\end{pgfscope}%
\begin{pgfscope}%
\pgfpathrectangle{\pgfqpoint{0.041670in}{0.041670in}}{\pgfqpoint{2.216660in}{2.216660in}}%
\pgfusepath{clip}%
\pgfsetbuttcap%
\pgfsetroundjoin%
\definecolor{currentfill}{rgb}{0.993248,0.906157,0.143936}%
\pgfsetfillcolor{currentfill}%
\pgfsetlinewidth{0.000000pt}%
\definecolor{currentstroke}{rgb}{0.000000,0.000000,0.000000}%
\pgfsetstrokecolor{currentstroke}%
\pgfsetdash{}{0pt}%
\pgfpathmoveto{\pgfqpoint{1.195243in}{1.654235in}}%
\pgfpathlineto{\pgfqpoint{1.199064in}{1.654016in}}%
\pgfpathlineto{\pgfqpoint{1.202883in}{1.653664in}}%
\pgfpathlineto{\pgfqpoint{1.206702in}{1.653179in}}%
\pgfpathlineto{\pgfqpoint{1.210520in}{1.652561in}}%
\pgfpathlineto{\pgfqpoint{1.210713in}{1.652110in}}%
\pgfpathlineto{\pgfqpoint{1.210877in}{1.651656in}}%
\pgfpathlineto{\pgfqpoint{1.211009in}{1.651200in}}%
\pgfpathlineto{\pgfqpoint{1.211111in}{1.650743in}}%
\pgfpathlineto{\pgfqpoint{1.207218in}{1.651588in}}%
\pgfpathlineto{\pgfqpoint{1.203325in}{1.652300in}}%
\pgfpathlineto{\pgfqpoint{1.199431in}{1.652879in}}%
\pgfpathlineto{\pgfqpoint{1.195537in}{1.653325in}}%
\pgfpathlineto{\pgfqpoint{1.195487in}{1.653554in}}%
\pgfpathlineto{\pgfqpoint{1.195421in}{1.653782in}}%
\pgfpathlineto{\pgfqpoint{1.195340in}{1.654009in}}%
\pgfpathlineto{\pgfqpoint{1.195243in}{1.654235in}}%
\pgfpathclose%
\pgfusepath{fill}%
\end{pgfscope}%
\begin{pgfscope}%
\pgfpathrectangle{\pgfqpoint{0.041670in}{0.041670in}}{\pgfqpoint{2.216660in}{2.216660in}}%
\pgfusepath{clip}%
\pgfsetbuttcap%
\pgfsetroundjoin%
\definecolor{currentfill}{rgb}{0.993248,0.906157,0.143936}%
\pgfsetfillcolor{currentfill}%
\pgfsetlinewidth{0.000000pt}%
\definecolor{currentstroke}{rgb}{0.000000,0.000000,0.000000}%
\pgfsetstrokecolor{currentstroke}%
\pgfsetdash{}{0pt}%
\pgfpathmoveto{\pgfqpoint{1.179955in}{1.653775in}}%
\pgfpathlineto{\pgfqpoint{1.182903in}{1.654904in}}%
\pgfpathlineto{\pgfqpoint{1.185850in}{1.655900in}}%
\pgfpathlineto{\pgfqpoint{1.188795in}{1.656761in}}%
\pgfpathlineto{\pgfqpoint{1.191740in}{1.657488in}}%
\pgfpathlineto{\pgfqpoint{1.192058in}{1.657312in}}%
\pgfpathlineto{\pgfqpoint{1.192365in}{1.657132in}}%
\pgfpathlineto{\pgfqpoint{1.192659in}{1.656948in}}%
\pgfpathlineto{\pgfqpoint{1.192940in}{1.656759in}}%
\pgfpathlineto{\pgfqpoint{1.189696in}{1.656214in}}%
\pgfpathlineto{\pgfqpoint{1.186450in}{1.655535in}}%
\pgfpathlineto{\pgfqpoint{1.183203in}{1.654722in}}%
\pgfpathlineto{\pgfqpoint{1.179955in}{1.653775in}}%
\pgfpathlineto{\pgfqpoint{1.179955in}{1.653775in}}%
\pgfpathlineto{\pgfqpoint{1.179955in}{1.653775in}}%
\pgfpathlineto{\pgfqpoint{1.179955in}{1.653775in}}%
\pgfpathlineto{\pgfqpoint{1.179955in}{1.653775in}}%
\pgfpathclose%
\pgfusepath{fill}%
\end{pgfscope}%
\begin{pgfscope}%
\pgfpathrectangle{\pgfqpoint{0.041670in}{0.041670in}}{\pgfqpoint{2.216660in}{2.216660in}}%
\pgfusepath{clip}%
\pgfsetbuttcap%
\pgfsetroundjoin%
\definecolor{currentfill}{rgb}{0.993248,0.906157,0.143936}%
\pgfsetfillcolor{currentfill}%
\pgfsetlinewidth{0.000000pt}%
\definecolor{currentstroke}{rgb}{0.000000,0.000000,0.000000}%
\pgfsetstrokecolor{currentstroke}%
\pgfsetdash{}{0pt}%
\pgfpathmoveto{\pgfqpoint{1.179955in}{1.653775in}}%
\pgfpathlineto{\pgfqpoint{1.176647in}{1.654679in}}%
\pgfpathlineto{\pgfqpoint{1.173340in}{1.655449in}}%
\pgfpathlineto{\pgfqpoint{1.170035in}{1.656085in}}%
\pgfpathlineto{\pgfqpoint{1.166730in}{1.656588in}}%
\pgfpathlineto{\pgfqpoint{1.167000in}{1.656780in}}%
\pgfpathlineto{\pgfqpoint{1.167283in}{1.656968in}}%
\pgfpathlineto{\pgfqpoint{1.167578in}{1.657152in}}%
\pgfpathlineto{\pgfqpoint{1.167886in}{1.657332in}}%
\pgfpathlineto{\pgfqpoint{1.170902in}{1.656644in}}%
\pgfpathlineto{\pgfqpoint{1.173918in}{1.655822in}}%
\pgfpathlineto{\pgfqpoint{1.176936in}{1.654865in}}%
\pgfpathlineto{\pgfqpoint{1.179955in}{1.653775in}}%
\pgfpathlineto{\pgfqpoint{1.179955in}{1.653775in}}%
\pgfpathlineto{\pgfqpoint{1.179955in}{1.653775in}}%
\pgfpathlineto{\pgfqpoint{1.179955in}{1.653775in}}%
\pgfpathlineto{\pgfqpoint{1.179955in}{1.653775in}}%
\pgfpathclose%
\pgfusepath{fill}%
\end{pgfscope}%
\begin{pgfscope}%
\pgfpathrectangle{\pgfqpoint{0.041670in}{0.041670in}}{\pgfqpoint{2.216660in}{2.216660in}}%
\pgfusepath{clip}%
\pgfsetbuttcap%
\pgfsetroundjoin%
\definecolor{currentfill}{rgb}{0.935904,0.898570,0.108131}%
\pgfsetfillcolor{currentfill}%
\pgfsetlinewidth{0.000000pt}%
\definecolor{currentstroke}{rgb}{0.000000,0.000000,0.000000}%
\pgfsetstrokecolor{currentstroke}%
\pgfsetdash{}{0pt}%
\pgfpathmoveto{\pgfqpoint{1.118580in}{1.631091in}}%
\pgfpathlineto{\pgfqpoint{1.114747in}{1.628558in}}%
\pgfpathlineto{\pgfqpoint{1.110915in}{1.625896in}}%
\pgfpathlineto{\pgfqpoint{1.107084in}{1.623106in}}%
\pgfpathlineto{\pgfqpoint{1.103254in}{1.620189in}}%
\pgfpathlineto{\pgfqpoint{1.102826in}{1.621326in}}%
\pgfpathlineto{\pgfqpoint{1.102475in}{1.622468in}}%
\pgfpathlineto{\pgfqpoint{1.102201in}{1.623615in}}%
\pgfpathlineto{\pgfqpoint{1.102006in}{1.624765in}}%
\pgfpathlineto{\pgfqpoint{1.105896in}{1.627452in}}%
\pgfpathlineto{\pgfqpoint{1.109787in}{1.630012in}}%
\pgfpathlineto{\pgfqpoint{1.113679in}{1.632444in}}%
\pgfpathlineto{\pgfqpoint{1.117572in}{1.634748in}}%
\pgfpathlineto{\pgfqpoint{1.117731in}{1.633829in}}%
\pgfpathlineto{\pgfqpoint{1.117952in}{1.632913in}}%
\pgfpathlineto{\pgfqpoint{1.118235in}{1.632000in}}%
\pgfpathlineto{\pgfqpoint{1.118580in}{1.631091in}}%
\pgfpathclose%
\pgfusepath{fill}%
\end{pgfscope}%
\begin{pgfscope}%
\pgfpathrectangle{\pgfqpoint{0.041670in}{0.041670in}}{\pgfqpoint{2.216660in}{2.216660in}}%
\pgfusepath{clip}%
\pgfsetbuttcap%
\pgfsetroundjoin%
\definecolor{currentfill}{rgb}{0.565498,0.842430,0.262877}%
\pgfsetfillcolor{currentfill}%
\pgfsetlinewidth{0.000000pt}%
\definecolor{currentstroke}{rgb}{0.000000,0.000000,0.000000}%
\pgfsetstrokecolor{currentstroke}%
\pgfsetdash{}{0pt}%
\pgfpathmoveto{\pgfqpoint{1.336920in}{1.497003in}}%
\pgfpathlineto{\pgfqpoint{1.340465in}{1.490827in}}%
\pgfpathlineto{\pgfqpoint{1.344008in}{1.484554in}}%
\pgfpathlineto{\pgfqpoint{1.347549in}{1.478185in}}%
\pgfpathlineto{\pgfqpoint{1.351088in}{1.471722in}}%
\pgfpathlineto{\pgfqpoint{1.348655in}{1.469152in}}%
\pgfpathlineto{\pgfqpoint{1.346051in}{1.466619in}}%
\pgfpathlineto{\pgfqpoint{1.343278in}{1.464126in}}%
\pgfpathlineto{\pgfqpoint{1.340339in}{1.461675in}}%
\pgfpathlineto{\pgfqpoint{1.337017in}{1.468356in}}%
\pgfpathlineto{\pgfqpoint{1.333694in}{1.474942in}}%
\pgfpathlineto{\pgfqpoint{1.330369in}{1.481432in}}%
\pgfpathlineto{\pgfqpoint{1.327043in}{1.487824in}}%
\pgfpathlineto{\pgfqpoint{1.329742in}{1.490063in}}%
\pgfpathlineto{\pgfqpoint{1.332290in}{1.492341in}}%
\pgfpathlineto{\pgfqpoint{1.334684in}{1.494655in}}%
\pgfpathlineto{\pgfqpoint{1.336920in}{1.497003in}}%
\pgfpathclose%
\pgfusepath{fill}%
\end{pgfscope}%
\begin{pgfscope}%
\pgfpathrectangle{\pgfqpoint{0.041670in}{0.041670in}}{\pgfqpoint{2.216660in}{2.216660in}}%
\pgfusepath{clip}%
\pgfsetbuttcap%
\pgfsetroundjoin%
\definecolor{currentfill}{rgb}{0.993248,0.906157,0.143936}%
\pgfsetfillcolor{currentfill}%
\pgfsetlinewidth{0.000000pt}%
\definecolor{currentstroke}{rgb}{0.000000,0.000000,0.000000}%
\pgfsetstrokecolor{currentstroke}%
\pgfsetdash{}{0pt}%
\pgfpathmoveto{\pgfqpoint{1.164341in}{1.653122in}}%
\pgfpathlineto{\pgfqpoint{1.160439in}{1.652624in}}%
\pgfpathlineto{\pgfqpoint{1.156537in}{1.651994in}}%
\pgfpathlineto{\pgfqpoint{1.152635in}{1.651231in}}%
\pgfpathlineto{\pgfqpoint{1.148734in}{1.650335in}}%
\pgfpathlineto{\pgfqpoint{1.148809in}{1.650794in}}%
\pgfpathlineto{\pgfqpoint{1.148914in}{1.651251in}}%
\pgfpathlineto{\pgfqpoint{1.149050in}{1.651707in}}%
\pgfpathlineto{\pgfqpoint{1.149216in}{1.652160in}}%
\pgfpathlineto{\pgfqpoint{1.153056in}{1.652828in}}%
\pgfpathlineto{\pgfqpoint{1.156897in}{1.653363in}}%
\pgfpathlineto{\pgfqpoint{1.160738in}{1.653765in}}%
\pgfpathlineto{\pgfqpoint{1.164580in}{1.654034in}}%
\pgfpathlineto{\pgfqpoint{1.164497in}{1.653807in}}%
\pgfpathlineto{\pgfqpoint{1.164430in}{1.653580in}}%
\pgfpathlineto{\pgfqpoint{1.164378in}{1.653351in}}%
\pgfpathlineto{\pgfqpoint{1.164341in}{1.653122in}}%
\pgfpathclose%
\pgfusepath{fill}%
\end{pgfscope}%
\begin{pgfscope}%
\pgfpathrectangle{\pgfqpoint{0.041670in}{0.041670in}}{\pgfqpoint{2.216660in}{2.216660in}}%
\pgfusepath{clip}%
\pgfsetbuttcap%
\pgfsetroundjoin%
\definecolor{currentfill}{rgb}{0.231674,0.318106,0.544834}%
\pgfsetfillcolor{currentfill}%
\pgfsetlinewidth{0.000000pt}%
\definecolor{currentstroke}{rgb}{0.000000,0.000000,0.000000}%
\pgfsetstrokecolor{currentstroke}%
\pgfsetdash{}{0pt}%
\pgfpathmoveto{\pgfqpoint{1.029175in}{0.878056in}}%
\pgfpathlineto{\pgfqpoint{1.027792in}{0.868978in}}%
\pgfpathlineto{\pgfqpoint{1.026409in}{0.859962in}}%
\pgfpathlineto{\pgfqpoint{1.025026in}{0.851012in}}%
\pgfpathlineto{\pgfqpoint{1.023642in}{0.842131in}}%
\pgfpathlineto{\pgfqpoint{1.010863in}{0.844825in}}%
\pgfpathlineto{\pgfqpoint{0.998268in}{0.847728in}}%
\pgfpathlineto{\pgfqpoint{0.985869in}{0.850836in}}%
\pgfpathlineto{\pgfqpoint{0.973681in}{0.854146in}}%
\pgfpathlineto{\pgfqpoint{0.975504in}{0.862906in}}%
\pgfpathlineto{\pgfqpoint{0.977326in}{0.871735in}}%
\pgfpathlineto{\pgfqpoint{0.979149in}{0.880631in}}%
\pgfpathlineto{\pgfqpoint{0.980971in}{0.889589in}}%
\pgfpathlineto{\pgfqpoint{0.992732in}{0.886412in}}%
\pgfpathlineto{\pgfqpoint{1.004695in}{0.883428in}}%
\pgfpathlineto{\pgfqpoint{1.016847in}{0.880642in}}%
\pgfpathlineto{\pgfqpoint{1.029175in}{0.878056in}}%
\pgfpathclose%
\pgfusepath{fill}%
\end{pgfscope}%
\begin{pgfscope}%
\pgfpathrectangle{\pgfqpoint{0.041670in}{0.041670in}}{\pgfqpoint{2.216660in}{2.216660in}}%
\pgfusepath{clip}%
\pgfsetbuttcap%
\pgfsetroundjoin%
\definecolor{currentfill}{rgb}{0.974417,0.903590,0.130215}%
\pgfsetfillcolor{currentfill}%
\pgfsetlinewidth{0.000000pt}%
\definecolor{currentstroke}{rgb}{0.000000,0.000000,0.000000}%
\pgfsetstrokecolor{currentstroke}%
\pgfsetdash{}{0pt}%
\pgfpathmoveto{\pgfqpoint{1.148750in}{1.648495in}}%
\pgfpathlineto{\pgfqpoint{1.144850in}{1.647237in}}%
\pgfpathlineto{\pgfqpoint{1.140951in}{1.645846in}}%
\pgfpathlineto{\pgfqpoint{1.137053in}{1.644323in}}%
\pgfpathlineto{\pgfqpoint{1.133155in}{1.642669in}}%
\pgfpathlineto{\pgfqpoint{1.133081in}{1.643359in}}%
\pgfpathlineto{\pgfqpoint{1.133054in}{1.644050in}}%
\pgfpathlineto{\pgfqpoint{1.133073in}{1.644740in}}%
\pgfpathlineto{\pgfqpoint{1.133139in}{1.645430in}}%
\pgfpathlineto{\pgfqpoint{1.137037in}{1.646854in}}%
\pgfpathlineto{\pgfqpoint{1.140935in}{1.648146in}}%
\pgfpathlineto{\pgfqpoint{1.144834in}{1.649306in}}%
\pgfpathlineto{\pgfqpoint{1.148734in}{1.650335in}}%
\pgfpathlineto{\pgfqpoint{1.148692in}{1.649875in}}%
\pgfpathlineto{\pgfqpoint{1.148680in}{1.649415in}}%
\pgfpathlineto{\pgfqpoint{1.148699in}{1.648955in}}%
\pgfpathlineto{\pgfqpoint{1.148750in}{1.648495in}}%
\pgfpathclose%
\pgfusepath{fill}%
\end{pgfscope}%
\begin{pgfscope}%
\pgfpathrectangle{\pgfqpoint{0.041670in}{0.041670in}}{\pgfqpoint{2.216660in}{2.216660in}}%
\pgfusepath{clip}%
\pgfsetbuttcap%
\pgfsetroundjoin%
\definecolor{currentfill}{rgb}{0.993248,0.906157,0.143936}%
\pgfsetfillcolor{currentfill}%
\pgfsetlinewidth{0.000000pt}%
\definecolor{currentstroke}{rgb}{0.000000,0.000000,0.000000}%
\pgfsetstrokecolor{currentstroke}%
\pgfsetdash{}{0pt}%
\pgfpathmoveto{\pgfqpoint{1.179955in}{1.653775in}}%
\pgfpathlineto{\pgfqpoint{1.183203in}{1.654722in}}%
\pgfpathlineto{\pgfqpoint{1.186450in}{1.655535in}}%
\pgfpathlineto{\pgfqpoint{1.189696in}{1.656214in}}%
\pgfpathlineto{\pgfqpoint{1.192940in}{1.656759in}}%
\pgfpathlineto{\pgfqpoint{1.193209in}{1.656566in}}%
\pgfpathlineto{\pgfqpoint{1.193464in}{1.656369in}}%
\pgfpathlineto{\pgfqpoint{1.193706in}{1.656169in}}%
\pgfpathlineto{\pgfqpoint{1.193934in}{1.655965in}}%
\pgfpathlineto{\pgfqpoint{1.190441in}{1.655618in}}%
\pgfpathlineto{\pgfqpoint{1.186947in}{1.655138in}}%
\pgfpathlineto{\pgfqpoint{1.183451in}{1.654523in}}%
\pgfpathlineto{\pgfqpoint{1.179955in}{1.653775in}}%
\pgfpathlineto{\pgfqpoint{1.179955in}{1.653775in}}%
\pgfpathlineto{\pgfqpoint{1.179955in}{1.653775in}}%
\pgfpathlineto{\pgfqpoint{1.179955in}{1.653775in}}%
\pgfpathlineto{\pgfqpoint{1.179955in}{1.653775in}}%
\pgfpathclose%
\pgfusepath{fill}%
\end{pgfscope}%
\begin{pgfscope}%
\pgfpathrectangle{\pgfqpoint{0.041670in}{0.041670in}}{\pgfqpoint{2.216660in}{2.216660in}}%
\pgfusepath{clip}%
\pgfsetbuttcap%
\pgfsetroundjoin%
\definecolor{currentfill}{rgb}{0.212395,0.359683,0.551710}%
\pgfsetfillcolor{currentfill}%
\pgfsetlinewidth{0.000000pt}%
\definecolor{currentstroke}{rgb}{0.000000,0.000000,0.000000}%
\pgfsetstrokecolor{currentstroke}%
\pgfsetdash{}{0pt}%
\pgfpathmoveto{\pgfqpoint{1.381553in}{0.928845in}}%
\pgfpathlineto{\pgfqpoint{1.383468in}{0.919699in}}%
\pgfpathlineto{\pgfqpoint{1.385383in}{0.910603in}}%
\pgfpathlineto{\pgfqpoint{1.387299in}{0.901560in}}%
\pgfpathlineto{\pgfqpoint{1.389214in}{0.892573in}}%
\pgfpathlineto{\pgfqpoint{1.377643in}{0.889227in}}%
\pgfpathlineto{\pgfqpoint{1.365858in}{0.886071in}}%
\pgfpathlineto{\pgfqpoint{1.353874in}{0.883109in}}%
\pgfpathlineto{\pgfqpoint{1.341702in}{0.880345in}}%
\pgfpathlineto{\pgfqpoint{1.340219in}{0.889459in}}%
\pgfpathlineto{\pgfqpoint{1.338736in}{0.898629in}}%
\pgfpathlineto{\pgfqpoint{1.337253in}{0.907852in}}%
\pgfpathlineto{\pgfqpoint{1.335771in}{0.917124in}}%
\pgfpathlineto{\pgfqpoint{1.347499in}{0.919773in}}%
\pgfpathlineto{\pgfqpoint{1.359047in}{0.922612in}}%
\pgfpathlineto{\pgfqpoint{1.370403in}{0.925637in}}%
\pgfpathlineto{\pgfqpoint{1.381553in}{0.928845in}}%
\pgfpathclose%
\pgfusepath{fill}%
\end{pgfscope}%
\begin{pgfscope}%
\pgfpathrectangle{\pgfqpoint{0.041670in}{0.041670in}}{\pgfqpoint{2.216660in}{2.216660in}}%
\pgfusepath{clip}%
\pgfsetbuttcap%
\pgfsetroundjoin%
\definecolor{currentfill}{rgb}{0.993248,0.906157,0.143936}%
\pgfsetfillcolor{currentfill}%
\pgfsetlinewidth{0.000000pt}%
\definecolor{currentstroke}{rgb}{0.000000,0.000000,0.000000}%
\pgfsetstrokecolor{currentstroke}%
\pgfsetdash{}{0pt}%
\pgfpathmoveto{\pgfqpoint{1.179955in}{1.653775in}}%
\pgfpathlineto{\pgfqpoint{1.176411in}{1.654477in}}%
\pgfpathlineto{\pgfqpoint{1.172868in}{1.655046in}}%
\pgfpathlineto{\pgfqpoint{1.169325in}{1.655480in}}%
\pgfpathlineto{\pgfqpoint{1.165784in}{1.655781in}}%
\pgfpathlineto{\pgfqpoint{1.166000in}{1.655988in}}%
\pgfpathlineto{\pgfqpoint{1.166230in}{1.656191in}}%
\pgfpathlineto{\pgfqpoint{1.166473in}{1.656391in}}%
\pgfpathlineto{\pgfqpoint{1.166730in}{1.656588in}}%
\pgfpathlineto{\pgfqpoint{1.170035in}{1.656085in}}%
\pgfpathlineto{\pgfqpoint{1.173340in}{1.655449in}}%
\pgfpathlineto{\pgfqpoint{1.176647in}{1.654679in}}%
\pgfpathlineto{\pgfqpoint{1.179955in}{1.653775in}}%
\pgfpathlineto{\pgfqpoint{1.179955in}{1.653775in}}%
\pgfpathlineto{\pgfqpoint{1.179955in}{1.653775in}}%
\pgfpathlineto{\pgfqpoint{1.179955in}{1.653775in}}%
\pgfpathlineto{\pgfqpoint{1.179955in}{1.653775in}}%
\pgfpathclose%
\pgfusepath{fill}%
\end{pgfscope}%
\begin{pgfscope}%
\pgfpathrectangle{\pgfqpoint{0.041670in}{0.041670in}}{\pgfqpoint{2.216660in}{2.216660in}}%
\pgfusepath{clip}%
\pgfsetbuttcap%
\pgfsetroundjoin%
\definecolor{currentfill}{rgb}{0.993248,0.906157,0.143936}%
\pgfsetfillcolor{currentfill}%
\pgfsetlinewidth{0.000000pt}%
\definecolor{currentstroke}{rgb}{0.000000,0.000000,0.000000}%
\pgfsetstrokecolor{currentstroke}%
\pgfsetdash{}{0pt}%
\pgfpathmoveto{\pgfqpoint{1.179955in}{1.653775in}}%
\pgfpathlineto{\pgfqpoint{1.183451in}{1.654523in}}%
\pgfpathlineto{\pgfqpoint{1.186947in}{1.655138in}}%
\pgfpathlineto{\pgfqpoint{1.190441in}{1.655618in}}%
\pgfpathlineto{\pgfqpoint{1.193934in}{1.655965in}}%
\pgfpathlineto{\pgfqpoint{1.194149in}{1.655758in}}%
\pgfpathlineto{\pgfqpoint{1.194349in}{1.655548in}}%
\pgfpathlineto{\pgfqpoint{1.194535in}{1.655335in}}%
\pgfpathlineto{\pgfqpoint{1.194706in}{1.655119in}}%
\pgfpathlineto{\pgfqpoint{1.191020in}{1.654984in}}%
\pgfpathlineto{\pgfqpoint{1.187332in}{1.654715in}}%
\pgfpathlineto{\pgfqpoint{1.183644in}{1.654312in}}%
\pgfpathlineto{\pgfqpoint{1.179955in}{1.653775in}}%
\pgfpathlineto{\pgfqpoint{1.179955in}{1.653775in}}%
\pgfpathlineto{\pgfqpoint{1.179955in}{1.653775in}}%
\pgfpathlineto{\pgfqpoint{1.179955in}{1.653775in}}%
\pgfpathlineto{\pgfqpoint{1.179955in}{1.653775in}}%
\pgfpathclose%
\pgfusepath{fill}%
\end{pgfscope}%
\begin{pgfscope}%
\pgfpathrectangle{\pgfqpoint{0.041670in}{0.041670in}}{\pgfqpoint{2.216660in}{2.216660in}}%
\pgfusepath{clip}%
\pgfsetbuttcap%
\pgfsetroundjoin%
\definecolor{currentfill}{rgb}{0.122606,0.585371,0.546557}%
\pgfsetfillcolor{currentfill}%
\pgfsetlinewidth{0.000000pt}%
\definecolor{currentstroke}{rgb}{0.000000,0.000000,0.000000}%
\pgfsetstrokecolor{currentstroke}%
\pgfsetdash{}{0pt}%
\pgfpathmoveto{\pgfqpoint{0.998677in}{1.163396in}}%
\pgfpathlineto{\pgfqpoint{0.996437in}{1.154097in}}%
\pgfpathlineto{\pgfqpoint{0.994198in}{1.144776in}}%
\pgfpathlineto{\pgfqpoint{0.991959in}{1.135437in}}%
\pgfpathlineto{\pgfqpoint{0.989722in}{1.126081in}}%
\pgfpathlineto{\pgfqpoint{0.981447in}{1.129173in}}%
\pgfpathlineto{\pgfqpoint{0.973381in}{1.132393in}}%
\pgfpathlineto{\pgfqpoint{0.965533in}{1.135739in}}%
\pgfpathlineto{\pgfqpoint{0.957910in}{1.139206in}}%
\pgfpathlineto{\pgfqpoint{0.960517in}{1.148390in}}%
\pgfpathlineto{\pgfqpoint{0.963125in}{1.157558in}}%
\pgfpathlineto{\pgfqpoint{0.965734in}{1.166708in}}%
\pgfpathlineto{\pgfqpoint{0.968344in}{1.175837in}}%
\pgfpathlineto{\pgfqpoint{0.975614in}{1.172551in}}%
\pgfpathlineto{\pgfqpoint{0.983098in}{1.169379in}}%
\pgfpathlineto{\pgfqpoint{0.990788in}{1.166327in}}%
\pgfpathlineto{\pgfqpoint{0.998677in}{1.163396in}}%
\pgfpathclose%
\pgfusepath{fill}%
\end{pgfscope}%
\begin{pgfscope}%
\pgfpathrectangle{\pgfqpoint{0.041670in}{0.041670in}}{\pgfqpoint{2.216660in}{2.216660in}}%
\pgfusepath{clip}%
\pgfsetbuttcap%
\pgfsetroundjoin%
\definecolor{currentfill}{rgb}{0.855810,0.888601,0.097452}%
\pgfsetfillcolor{currentfill}%
\pgfsetlinewidth{0.000000pt}%
\definecolor{currentstroke}{rgb}{0.000000,0.000000,0.000000}%
\pgfsetstrokecolor{currentstroke}%
\pgfsetdash{}{0pt}%
\pgfpathmoveto{\pgfqpoint{1.090898in}{1.601891in}}%
\pgfpathlineto{\pgfqpoint{1.087193in}{1.598124in}}%
\pgfpathlineto{\pgfqpoint{1.083489in}{1.594233in}}%
\pgfpathlineto{\pgfqpoint{1.079786in}{1.590222in}}%
\pgfpathlineto{\pgfqpoint{1.076085in}{1.586089in}}%
\pgfpathlineto{\pgfqpoint{1.075068in}{1.587640in}}%
\pgfpathlineto{\pgfqpoint{1.074156in}{1.589205in}}%
\pgfpathlineto{\pgfqpoint{1.073351in}{1.590783in}}%
\pgfpathlineto{\pgfqpoint{1.072653in}{1.592371in}}%
\pgfpathlineto{\pgfqpoint{1.076473in}{1.596276in}}%
\pgfpathlineto{\pgfqpoint{1.080295in}{1.600060in}}%
\pgfpathlineto{\pgfqpoint{1.084118in}{1.603723in}}%
\pgfpathlineto{\pgfqpoint{1.087943in}{1.607264in}}%
\pgfpathlineto{\pgfqpoint{1.088545in}{1.605906in}}%
\pgfpathlineto{\pgfqpoint{1.089239in}{1.604556in}}%
\pgfpathlineto{\pgfqpoint{1.090024in}{1.603218in}}%
\pgfpathlineto{\pgfqpoint{1.090898in}{1.601891in}}%
\pgfpathclose%
\pgfusepath{fill}%
\end{pgfscope}%
\begin{pgfscope}%
\pgfpathrectangle{\pgfqpoint{0.041670in}{0.041670in}}{\pgfqpoint{2.216660in}{2.216660in}}%
\pgfusepath{clip}%
\pgfsetbuttcap%
\pgfsetroundjoin%
\definecolor{currentfill}{rgb}{0.993248,0.906157,0.143936}%
\pgfsetfillcolor{currentfill}%
\pgfsetlinewidth{0.000000pt}%
\definecolor{currentstroke}{rgb}{0.000000,0.000000,0.000000}%
\pgfsetstrokecolor{currentstroke}%
\pgfsetdash{}{0pt}%
\pgfpathmoveto{\pgfqpoint{1.179955in}{1.653775in}}%
\pgfpathlineto{\pgfqpoint{1.176231in}{1.654263in}}%
\pgfpathlineto{\pgfqpoint{1.172508in}{1.654618in}}%
\pgfpathlineto{\pgfqpoint{1.168785in}{1.654838in}}%
\pgfpathlineto{\pgfqpoint{1.165064in}{1.654925in}}%
\pgfpathlineto{\pgfqpoint{1.165222in}{1.655143in}}%
\pgfpathlineto{\pgfqpoint{1.165395in}{1.655359in}}%
\pgfpathlineto{\pgfqpoint{1.165582in}{1.655571in}}%
\pgfpathlineto{\pgfqpoint{1.165784in}{1.655781in}}%
\pgfpathlineto{\pgfqpoint{1.169325in}{1.655480in}}%
\pgfpathlineto{\pgfqpoint{1.172868in}{1.655046in}}%
\pgfpathlineto{\pgfqpoint{1.176411in}{1.654477in}}%
\pgfpathlineto{\pgfqpoint{1.179955in}{1.653775in}}%
\pgfpathlineto{\pgfqpoint{1.179955in}{1.653775in}}%
\pgfpathlineto{\pgfqpoint{1.179955in}{1.653775in}}%
\pgfpathlineto{\pgfqpoint{1.179955in}{1.653775in}}%
\pgfpathlineto{\pgfqpoint{1.179955in}{1.653775in}}%
\pgfpathclose%
\pgfusepath{fill}%
\end{pgfscope}%
\begin{pgfscope}%
\pgfpathrectangle{\pgfqpoint{0.041670in}{0.041670in}}{\pgfqpoint{2.216660in}{2.216660in}}%
\pgfusepath{clip}%
\pgfsetbuttcap%
\pgfsetroundjoin%
\definecolor{currentfill}{rgb}{0.955300,0.901065,0.118128}%
\pgfsetfillcolor{currentfill}%
\pgfsetlinewidth{0.000000pt}%
\definecolor{currentstroke}{rgb}{0.000000,0.000000,0.000000}%
\pgfsetstrokecolor{currentstroke}%
\pgfsetdash{}{0pt}%
\pgfpathmoveto{\pgfqpoint{1.226823in}{1.643282in}}%
\pgfpathlineto{\pgfqpoint{1.230725in}{1.641549in}}%
\pgfpathlineto{\pgfqpoint{1.234626in}{1.639684in}}%
\pgfpathlineto{\pgfqpoint{1.238527in}{1.637690in}}%
\pgfpathlineto{\pgfqpoint{1.242426in}{1.635566in}}%
\pgfpathlineto{\pgfqpoint{1.242323in}{1.634646in}}%
\pgfpathlineto{\pgfqpoint{1.242157in}{1.633727in}}%
\pgfpathlineto{\pgfqpoint{1.241929in}{1.632811in}}%
\pgfpathlineto{\pgfqpoint{1.241639in}{1.631899in}}%
\pgfpathlineto{\pgfqpoint{1.237787in}{1.634252in}}%
\pgfpathlineto{\pgfqpoint{1.233934in}{1.636477in}}%
\pgfpathlineto{\pgfqpoint{1.230080in}{1.638571in}}%
\pgfpathlineto{\pgfqpoint{1.226226in}{1.640534in}}%
\pgfpathlineto{\pgfqpoint{1.226445in}{1.641218in}}%
\pgfpathlineto{\pgfqpoint{1.226617in}{1.641904in}}%
\pgfpathlineto{\pgfqpoint{1.226743in}{1.642593in}}%
\pgfpathlineto{\pgfqpoint{1.226823in}{1.643282in}}%
\pgfpathclose%
\pgfusepath{fill}%
\end{pgfscope}%
\begin{pgfscope}%
\pgfpathrectangle{\pgfqpoint{0.041670in}{0.041670in}}{\pgfqpoint{2.216660in}{2.216660in}}%
\pgfusepath{clip}%
\pgfsetbuttcap%
\pgfsetroundjoin%
\definecolor{currentfill}{rgb}{0.762373,0.876424,0.137064}%
\pgfsetfillcolor{currentfill}%
\pgfsetlinewidth{0.000000pt}%
\definecolor{currentstroke}{rgb}{0.000000,0.000000,0.000000}%
\pgfsetstrokecolor{currentstroke}%
\pgfsetdash{}{0pt}%
\pgfpathmoveto{\pgfqpoint{1.299655in}{1.569953in}}%
\pgfpathlineto{\pgfqpoint{1.303381in}{1.565283in}}%
\pgfpathlineto{\pgfqpoint{1.307106in}{1.560500in}}%
\pgfpathlineto{\pgfqpoint{1.310830in}{1.555604in}}%
\pgfpathlineto{\pgfqpoint{1.314551in}{1.550597in}}%
\pgfpathlineto{\pgfqpoint{1.313239in}{1.548595in}}%
\pgfpathlineto{\pgfqpoint{1.311793in}{1.546614in}}%
\pgfpathlineto{\pgfqpoint{1.310214in}{1.544654in}}%
\pgfpathlineto{\pgfqpoint{1.308504in}{1.542718in}}%
\pgfpathlineto{\pgfqpoint{1.304946in}{1.547950in}}%
\pgfpathlineto{\pgfqpoint{1.301385in}{1.553069in}}%
\pgfpathlineto{\pgfqpoint{1.297824in}{1.558076in}}%
\pgfpathlineto{\pgfqpoint{1.294261in}{1.562968in}}%
\pgfpathlineto{\pgfqpoint{1.295786in}{1.564684in}}%
\pgfpathlineto{\pgfqpoint{1.297194in}{1.566421in}}%
\pgfpathlineto{\pgfqpoint{1.298484in}{1.568178in}}%
\pgfpathlineto{\pgfqpoint{1.299655in}{1.569953in}}%
\pgfpathclose%
\pgfusepath{fill}%
\end{pgfscope}%
\begin{pgfscope}%
\pgfpathrectangle{\pgfqpoint{0.041670in}{0.041670in}}{\pgfqpoint{2.216660in}{2.216660in}}%
\pgfusepath{clip}%
\pgfsetbuttcap%
\pgfsetroundjoin%
\definecolor{currentfill}{rgb}{0.166383,0.690856,0.496502}%
\pgfsetfillcolor{currentfill}%
\pgfsetlinewidth{0.000000pt}%
\definecolor{currentstroke}{rgb}{0.000000,0.000000,0.000000}%
\pgfsetstrokecolor{currentstroke}%
\pgfsetdash{}{0pt}%
\pgfpathmoveto{\pgfqpoint{0.999758in}{1.282786in}}%
\pgfpathlineto{\pgfqpoint{0.997133in}{1.274111in}}%
\pgfpathlineto{\pgfqpoint{0.994510in}{1.265385in}}%
\pgfpathlineto{\pgfqpoint{0.991888in}{1.256611in}}%
\pgfpathlineto{\pgfqpoint{0.989267in}{1.247789in}}%
\pgfpathlineto{\pgfqpoint{0.982906in}{1.250819in}}%
\pgfpathlineto{\pgfqpoint{0.976750in}{1.253946in}}%
\pgfpathlineto{\pgfqpoint{0.970805in}{1.257166in}}%
\pgfpathlineto{\pgfqpoint{0.965076in}{1.260475in}}%
\pgfpathlineto{\pgfqpoint{0.968024in}{1.269106in}}%
\pgfpathlineto{\pgfqpoint{0.970974in}{1.277691in}}%
\pgfpathlineto{\pgfqpoint{0.973925in}{1.286228in}}%
\pgfpathlineto{\pgfqpoint{0.976877in}{1.294714in}}%
\pgfpathlineto{\pgfqpoint{0.982296in}{1.291602in}}%
\pgfpathlineto{\pgfqpoint{0.987919in}{1.288575in}}%
\pgfpathlineto{\pgfqpoint{0.993742in}{1.285635in}}%
\pgfpathlineto{\pgfqpoint{0.999758in}{1.282786in}}%
\pgfpathclose%
\pgfusepath{fill}%
\end{pgfscope}%
\begin{pgfscope}%
\pgfpathrectangle{\pgfqpoint{0.041670in}{0.041670in}}{\pgfqpoint{2.216660in}{2.216660in}}%
\pgfusepath{clip}%
\pgfsetbuttcap%
\pgfsetroundjoin%
\definecolor{currentfill}{rgb}{0.163625,0.471133,0.558148}%
\pgfsetfillcolor{currentfill}%
\pgfsetlinewidth{0.000000pt}%
\definecolor{currentstroke}{rgb}{0.000000,0.000000,0.000000}%
\pgfsetstrokecolor{currentstroke}%
\pgfsetdash{}{0pt}%
\pgfpathmoveto{\pgfqpoint{1.010129in}{1.038801in}}%
\pgfpathlineto{\pgfqpoint{1.008305in}{1.029278in}}%
\pgfpathlineto{\pgfqpoint{1.006481in}{1.019766in}}%
\pgfpathlineto{\pgfqpoint{1.004657in}{1.010270in}}%
\pgfpathlineto{\pgfqpoint{1.002834in}{1.000792in}}%
\pgfpathlineto{\pgfqpoint{0.992547in}{1.003744in}}%
\pgfpathlineto{\pgfqpoint{0.982459in}{1.006859in}}%
\pgfpathlineto{\pgfqpoint{0.972582in}{1.010134in}}%
\pgfpathlineto{\pgfqpoint{0.962925in}{1.013566in}}%
\pgfpathlineto{\pgfqpoint{0.965155in}{1.022896in}}%
\pgfpathlineto{\pgfqpoint{0.967385in}{1.032244in}}%
\pgfpathlineto{\pgfqpoint{0.969616in}{1.041608in}}%
\pgfpathlineto{\pgfqpoint{0.971847in}{1.050984in}}%
\pgfpathlineto{\pgfqpoint{0.981111in}{1.047710in}}%
\pgfpathlineto{\pgfqpoint{0.990586in}{1.044587in}}%
\pgfpathlineto{\pgfqpoint{1.000262in}{1.041615in}}%
\pgfpathlineto{\pgfqpoint{1.010129in}{1.038801in}}%
\pgfpathclose%
\pgfusepath{fill}%
\end{pgfscope}%
\begin{pgfscope}%
\pgfpathrectangle{\pgfqpoint{0.041670in}{0.041670in}}{\pgfqpoint{2.216660in}{2.216660in}}%
\pgfusepath{clip}%
\pgfsetbuttcap%
\pgfsetroundjoin%
\definecolor{currentfill}{rgb}{0.993248,0.906157,0.143936}%
\pgfsetfillcolor{currentfill}%
\pgfsetlinewidth{0.000000pt}%
\definecolor{currentstroke}{rgb}{0.000000,0.000000,0.000000}%
\pgfsetstrokecolor{currentstroke}%
\pgfsetdash{}{0pt}%
\pgfpathmoveto{\pgfqpoint{1.179955in}{1.653775in}}%
\pgfpathlineto{\pgfqpoint{1.183644in}{1.654312in}}%
\pgfpathlineto{\pgfqpoint{1.187332in}{1.654715in}}%
\pgfpathlineto{\pgfqpoint{1.191020in}{1.654984in}}%
\pgfpathlineto{\pgfqpoint{1.194706in}{1.655119in}}%
\pgfpathlineto{\pgfqpoint{1.194863in}{1.654901in}}%
\pgfpathlineto{\pgfqpoint{1.195005in}{1.654681in}}%
\pgfpathlineto{\pgfqpoint{1.195131in}{1.654459in}}%
\pgfpathlineto{\pgfqpoint{1.195243in}{1.654235in}}%
\pgfpathlineto{\pgfqpoint{1.191422in}{1.654320in}}%
\pgfpathlineto{\pgfqpoint{1.187600in}{1.654272in}}%
\pgfpathlineto{\pgfqpoint{1.183778in}{1.654091in}}%
\pgfpathlineto{\pgfqpoint{1.179955in}{1.653775in}}%
\pgfpathlineto{\pgfqpoint{1.179955in}{1.653775in}}%
\pgfpathlineto{\pgfqpoint{1.179955in}{1.653775in}}%
\pgfpathlineto{\pgfqpoint{1.179955in}{1.653775in}}%
\pgfpathlineto{\pgfqpoint{1.179955in}{1.653775in}}%
\pgfpathclose%
\pgfusepath{fill}%
\end{pgfscope}%
\begin{pgfscope}%
\pgfpathrectangle{\pgfqpoint{0.041670in}{0.041670in}}{\pgfqpoint{2.216660in}{2.216660in}}%
\pgfusepath{clip}%
\pgfsetbuttcap%
\pgfsetroundjoin%
\definecolor{currentfill}{rgb}{0.993248,0.906157,0.143936}%
\pgfsetfillcolor{currentfill}%
\pgfsetlinewidth{0.000000pt}%
\definecolor{currentstroke}{rgb}{0.000000,0.000000,0.000000}%
\pgfsetstrokecolor{currentstroke}%
\pgfsetdash{}{0pt}%
\pgfpathmoveto{\pgfqpoint{1.195537in}{1.653325in}}%
\pgfpathlineto{\pgfqpoint{1.199431in}{1.652879in}}%
\pgfpathlineto{\pgfqpoint{1.203325in}{1.652300in}}%
\pgfpathlineto{\pgfqpoint{1.207218in}{1.651588in}}%
\pgfpathlineto{\pgfqpoint{1.211111in}{1.650743in}}%
\pgfpathlineto{\pgfqpoint{1.211182in}{1.650284in}}%
\pgfpathlineto{\pgfqpoint{1.211221in}{1.649824in}}%
\pgfpathlineto{\pgfqpoint{1.211229in}{1.649364in}}%
\pgfpathlineto{\pgfqpoint{1.211206in}{1.648904in}}%
\pgfpathlineto{\pgfqpoint{1.207301in}{1.649979in}}%
\pgfpathlineto{\pgfqpoint{1.203395in}{1.650921in}}%
\pgfpathlineto{\pgfqpoint{1.199489in}{1.651730in}}%
\pgfpathlineto{\pgfqpoint{1.195582in}{1.652406in}}%
\pgfpathlineto{\pgfqpoint{1.195594in}{1.652636in}}%
\pgfpathlineto{\pgfqpoint{1.195591in}{1.652866in}}%
\pgfpathlineto{\pgfqpoint{1.195572in}{1.653096in}}%
\pgfpathlineto{\pgfqpoint{1.195537in}{1.653325in}}%
\pgfpathclose%
\pgfusepath{fill}%
\end{pgfscope}%
\begin{pgfscope}%
\pgfpathrectangle{\pgfqpoint{0.041670in}{0.041670in}}{\pgfqpoint{2.216660in}{2.216660in}}%
\pgfusepath{clip}%
\pgfsetbuttcap%
\pgfsetroundjoin%
\definecolor{currentfill}{rgb}{0.993248,0.906157,0.143936}%
\pgfsetfillcolor{currentfill}%
\pgfsetlinewidth{0.000000pt}%
\definecolor{currentstroke}{rgb}{0.000000,0.000000,0.000000}%
\pgfsetstrokecolor{currentstroke}%
\pgfsetdash{}{0pt}%
\pgfpathmoveto{\pgfqpoint{1.179955in}{1.653775in}}%
\pgfpathlineto{\pgfqpoint{1.176110in}{1.654041in}}%
\pgfpathlineto{\pgfqpoint{1.172266in}{1.654172in}}%
\pgfpathlineto{\pgfqpoint{1.168423in}{1.654170in}}%
\pgfpathlineto{\pgfqpoint{1.164580in}{1.654034in}}%
\pgfpathlineto{\pgfqpoint{1.164678in}{1.654260in}}%
\pgfpathlineto{\pgfqpoint{1.164792in}{1.654483in}}%
\pgfpathlineto{\pgfqpoint{1.164920in}{1.654705in}}%
\pgfpathlineto{\pgfqpoint{1.165064in}{1.654925in}}%
\pgfpathlineto{\pgfqpoint{1.168785in}{1.654838in}}%
\pgfpathlineto{\pgfqpoint{1.172508in}{1.654618in}}%
\pgfpathlineto{\pgfqpoint{1.176231in}{1.654263in}}%
\pgfpathlineto{\pgfqpoint{1.179955in}{1.653775in}}%
\pgfpathlineto{\pgfqpoint{1.179955in}{1.653775in}}%
\pgfpathlineto{\pgfqpoint{1.179955in}{1.653775in}}%
\pgfpathlineto{\pgfqpoint{1.179955in}{1.653775in}}%
\pgfpathlineto{\pgfqpoint{1.179955in}{1.653775in}}%
\pgfpathclose%
\pgfusepath{fill}%
\end{pgfscope}%
\begin{pgfscope}%
\pgfpathrectangle{\pgfqpoint{0.041670in}{0.041670in}}{\pgfqpoint{2.216660in}{2.216660in}}%
\pgfusepath{clip}%
\pgfsetbuttcap%
\pgfsetroundjoin%
\definecolor{currentfill}{rgb}{0.277941,0.056324,0.381191}%
\pgfsetfillcolor{currentfill}%
\pgfsetlinewidth{0.000000pt}%
\definecolor{currentstroke}{rgb}{0.000000,0.000000,0.000000}%
\pgfsetstrokecolor{currentstroke}%
\pgfsetdash{}{0pt}%
\pgfpathmoveto{\pgfqpoint{1.636754in}{0.652456in}}%
\pgfpathlineto{\pgfqpoint{1.639651in}{0.654691in}}%
\pgfpathlineto{\pgfqpoint{1.642558in}{0.657251in}}%
\pgfpathlineto{\pgfqpoint{1.645474in}{0.660142in}}%
\pgfpathlineto{\pgfqpoint{1.648401in}{0.663371in}}%
\pgfpathlineto{\pgfqpoint{1.633448in}{0.655545in}}%
\pgfpathlineto{\pgfqpoint{1.617996in}{0.647970in}}%
\pgfpathlineto{\pgfqpoint{1.602060in}{0.640654in}}%
\pgfpathlineto{\pgfqpoint{1.585656in}{0.633607in}}%
\pgfpathlineto{\pgfqpoint{1.583111in}{0.630552in}}%
\pgfpathlineto{\pgfqpoint{1.580575in}{0.627835in}}%
\pgfpathlineto{\pgfqpoint{1.578048in}{0.625450in}}%
\pgfpathlineto{\pgfqpoint{1.575530in}{0.623392in}}%
\pgfpathlineto{\pgfqpoint{1.591534in}{0.630273in}}%
\pgfpathlineto{\pgfqpoint{1.607084in}{0.637416in}}%
\pgfpathlineto{\pgfqpoint{1.622162in}{0.644814in}}%
\pgfpathlineto{\pgfqpoint{1.636754in}{0.652456in}}%
\pgfpathclose%
\pgfusepath{fill}%
\end{pgfscope}%
\begin{pgfscope}%
\pgfpathrectangle{\pgfqpoint{0.041670in}{0.041670in}}{\pgfqpoint{2.216660in}{2.216660in}}%
\pgfusepath{clip}%
\pgfsetbuttcap%
\pgfsetroundjoin%
\definecolor{currentfill}{rgb}{0.344074,0.780029,0.397381}%
\pgfsetfillcolor{currentfill}%
\pgfsetlinewidth{0.000000pt}%
\definecolor{currentstroke}{rgb}{0.000000,0.000000,0.000000}%
\pgfsetstrokecolor{currentstroke}%
\pgfsetdash{}{0pt}%
\pgfpathmoveto{\pgfqpoint{1.012422in}{1.391760in}}%
\pgfpathlineto{\pgfqpoint{1.009452in}{1.384057in}}%
\pgfpathlineto{\pgfqpoint{1.006483in}{1.376278in}}%
\pgfpathlineto{\pgfqpoint{1.003516in}{1.368424in}}%
\pgfpathlineto{\pgfqpoint{1.000550in}{1.360497in}}%
\pgfpathlineto{\pgfqpoint{0.995938in}{1.363290in}}%
\pgfpathlineto{\pgfqpoint{0.991515in}{1.366152in}}%
\pgfpathlineto{\pgfqpoint{0.987286in}{1.369079in}}%
\pgfpathlineto{\pgfqpoint{0.983254in}{1.372068in}}%
\pgfpathlineto{\pgfqpoint{0.986500in}{1.379790in}}%
\pgfpathlineto{\pgfqpoint{0.989748in}{1.387441in}}%
\pgfpathlineto{\pgfqpoint{0.992998in}{1.395017in}}%
\pgfpathlineto{\pgfqpoint{0.996249in}{1.402516in}}%
\pgfpathlineto{\pgfqpoint{1.000020in}{1.399737in}}%
\pgfpathlineto{\pgfqpoint{1.003975in}{1.397016in}}%
\pgfpathlineto{\pgfqpoint{1.008110in}{1.394356in}}%
\pgfpathlineto{\pgfqpoint{1.012422in}{1.391760in}}%
\pgfpathclose%
\pgfusepath{fill}%
\end{pgfscope}%
\begin{pgfscope}%
\pgfpathrectangle{\pgfqpoint{0.041670in}{0.041670in}}{\pgfqpoint{2.216660in}{2.216660in}}%
\pgfusepath{clip}%
\pgfsetbuttcap%
\pgfsetroundjoin%
\definecolor{currentfill}{rgb}{0.993248,0.906157,0.143936}%
\pgfsetfillcolor{currentfill}%
\pgfsetlinewidth{0.000000pt}%
\definecolor{currentstroke}{rgb}{0.000000,0.000000,0.000000}%
\pgfsetstrokecolor{currentstroke}%
\pgfsetdash{}{0pt}%
\pgfpathmoveto{\pgfqpoint{1.164351in}{1.652202in}}%
\pgfpathlineto{\pgfqpoint{1.160450in}{1.651475in}}%
\pgfpathlineto{\pgfqpoint{1.156550in}{1.650615in}}%
\pgfpathlineto{\pgfqpoint{1.152650in}{1.649621in}}%
\pgfpathlineto{\pgfqpoint{1.148750in}{1.648495in}}%
\pgfpathlineto{\pgfqpoint{1.148699in}{1.648955in}}%
\pgfpathlineto{\pgfqpoint{1.148680in}{1.649415in}}%
\pgfpathlineto{\pgfqpoint{1.148692in}{1.649875in}}%
\pgfpathlineto{\pgfqpoint{1.148734in}{1.650335in}}%
\pgfpathlineto{\pgfqpoint{1.152635in}{1.651231in}}%
\pgfpathlineto{\pgfqpoint{1.156537in}{1.651994in}}%
\pgfpathlineto{\pgfqpoint{1.160439in}{1.652624in}}%
\pgfpathlineto{\pgfqpoint{1.164341in}{1.653122in}}%
\pgfpathlineto{\pgfqpoint{1.164320in}{1.652892in}}%
\pgfpathlineto{\pgfqpoint{1.164315in}{1.652662in}}%
\pgfpathlineto{\pgfqpoint{1.164325in}{1.652432in}}%
\pgfpathlineto{\pgfqpoint{1.164351in}{1.652202in}}%
\pgfpathclose%
\pgfusepath{fill}%
\end{pgfscope}%
\begin{pgfscope}%
\pgfpathrectangle{\pgfqpoint{0.041670in}{0.041670in}}{\pgfqpoint{2.216660in}{2.216660in}}%
\pgfusepath{clip}%
\pgfsetbuttcap%
\pgfsetroundjoin%
\definecolor{currentfill}{rgb}{0.993248,0.906157,0.143936}%
\pgfsetfillcolor{currentfill}%
\pgfsetlinewidth{0.000000pt}%
\definecolor{currentstroke}{rgb}{0.000000,0.000000,0.000000}%
\pgfsetstrokecolor{currentstroke}%
\pgfsetdash{}{0pt}%
\pgfpathmoveto{\pgfqpoint{1.179955in}{1.653775in}}%
\pgfpathlineto{\pgfqpoint{1.183778in}{1.654091in}}%
\pgfpathlineto{\pgfqpoint{1.187600in}{1.654272in}}%
\pgfpathlineto{\pgfqpoint{1.191422in}{1.654320in}}%
\pgfpathlineto{\pgfqpoint{1.195243in}{1.654235in}}%
\pgfpathlineto{\pgfqpoint{1.195340in}{1.654009in}}%
\pgfpathlineto{\pgfqpoint{1.195421in}{1.653782in}}%
\pgfpathlineto{\pgfqpoint{1.195487in}{1.653554in}}%
\pgfpathlineto{\pgfqpoint{1.195537in}{1.653325in}}%
\pgfpathlineto{\pgfqpoint{1.191642in}{1.653638in}}%
\pgfpathlineto{\pgfqpoint{1.187746in}{1.653818in}}%
\pgfpathlineto{\pgfqpoint{1.183851in}{1.653863in}}%
\pgfpathlineto{\pgfqpoint{1.179955in}{1.653775in}}%
\pgfpathlineto{\pgfqpoint{1.179955in}{1.653775in}}%
\pgfpathlineto{\pgfqpoint{1.179955in}{1.653775in}}%
\pgfpathlineto{\pgfqpoint{1.179955in}{1.653775in}}%
\pgfpathlineto{\pgfqpoint{1.179955in}{1.653775in}}%
\pgfpathclose%
\pgfusepath{fill}%
\end{pgfscope}%
\begin{pgfscope}%
\pgfpathrectangle{\pgfqpoint{0.041670in}{0.041670in}}{\pgfqpoint{2.216660in}{2.216660in}}%
\pgfusepath{clip}%
\pgfsetbuttcap%
\pgfsetroundjoin%
\definecolor{currentfill}{rgb}{0.955300,0.901065,0.118128}%
\pgfsetfillcolor{currentfill}%
\pgfsetlinewidth{0.000000pt}%
\definecolor{currentstroke}{rgb}{0.000000,0.000000,0.000000}%
\pgfsetstrokecolor{currentstroke}%
\pgfsetdash{}{0pt}%
\pgfpathmoveto{\pgfqpoint{1.133918in}{1.639930in}}%
\pgfpathlineto{\pgfqpoint{1.130082in}{1.637916in}}%
\pgfpathlineto{\pgfqpoint{1.126248in}{1.635771in}}%
\pgfpathlineto{\pgfqpoint{1.122413in}{1.633496in}}%
\pgfpathlineto{\pgfqpoint{1.118580in}{1.631091in}}%
\pgfpathlineto{\pgfqpoint{1.118235in}{1.632000in}}%
\pgfpathlineto{\pgfqpoint{1.117952in}{1.632913in}}%
\pgfpathlineto{\pgfqpoint{1.117731in}{1.633829in}}%
\pgfpathlineto{\pgfqpoint{1.117572in}{1.634748in}}%
\pgfpathlineto{\pgfqpoint{1.121467in}{1.636923in}}%
\pgfpathlineto{\pgfqpoint{1.125362in}{1.638969in}}%
\pgfpathlineto{\pgfqpoint{1.129258in}{1.640884in}}%
\pgfpathlineto{\pgfqpoint{1.133155in}{1.642669in}}%
\pgfpathlineto{\pgfqpoint{1.133276in}{1.641981in}}%
\pgfpathlineto{\pgfqpoint{1.133444in}{1.641294in}}%
\pgfpathlineto{\pgfqpoint{1.133658in}{1.640610in}}%
\pgfpathlineto{\pgfqpoint{1.133918in}{1.639930in}}%
\pgfpathclose%
\pgfusepath{fill}%
\end{pgfscope}%
\begin{pgfscope}%
\pgfpathrectangle{\pgfqpoint{0.041670in}{0.041670in}}{\pgfqpoint{2.216660in}{2.216660in}}%
\pgfusepath{clip}%
\pgfsetbuttcap%
\pgfsetroundjoin%
\definecolor{currentfill}{rgb}{0.896320,0.893616,0.096335}%
\pgfsetfillcolor{currentfill}%
\pgfsetlinewidth{0.000000pt}%
\definecolor{currentstroke}{rgb}{0.000000,0.000000,0.000000}%
\pgfsetstrokecolor{currentstroke}%
\pgfsetdash{}{0pt}%
\pgfpathmoveto{\pgfqpoint{1.257040in}{1.621199in}}%
\pgfpathlineto{\pgfqpoint{1.260888in}{1.618207in}}%
\pgfpathlineto{\pgfqpoint{1.264735in}{1.615089in}}%
\pgfpathlineto{\pgfqpoint{1.268580in}{1.611846in}}%
\pgfpathlineto{\pgfqpoint{1.272425in}{1.608479in}}%
\pgfpathlineto{\pgfqpoint{1.271904in}{1.607113in}}%
\pgfpathlineto{\pgfqpoint{1.271292in}{1.605755in}}%
\pgfpathlineto{\pgfqpoint{1.270588in}{1.604407in}}%
\pgfpathlineto{\pgfqpoint{1.269793in}{1.603070in}}%
\pgfpathlineto{\pgfqpoint{1.266055in}{1.606664in}}%
\pgfpathlineto{\pgfqpoint{1.262317in}{1.610135in}}%
\pgfpathlineto{\pgfqpoint{1.258577in}{1.613480in}}%
\pgfpathlineto{\pgfqpoint{1.254836in}{1.616699in}}%
\pgfpathlineto{\pgfqpoint{1.255501in}{1.617812in}}%
\pgfpathlineto{\pgfqpoint{1.256091in}{1.618933in}}%
\pgfpathlineto{\pgfqpoint{1.256604in}{1.620063in}}%
\pgfpathlineto{\pgfqpoint{1.257040in}{1.621199in}}%
\pgfpathclose%
\pgfusepath{fill}%
\end{pgfscope}%
\begin{pgfscope}%
\pgfpathrectangle{\pgfqpoint{0.041670in}{0.041670in}}{\pgfqpoint{2.216660in}{2.216660in}}%
\pgfusepath{clip}%
\pgfsetbuttcap%
\pgfsetroundjoin%
\definecolor{currentfill}{rgb}{0.993248,0.906157,0.143936}%
\pgfsetfillcolor{currentfill}%
\pgfsetlinewidth{0.000000pt}%
\definecolor{currentstroke}{rgb}{0.000000,0.000000,0.000000}%
\pgfsetstrokecolor{currentstroke}%
\pgfsetdash{}{0pt}%
\pgfpathmoveto{\pgfqpoint{1.179955in}{1.653775in}}%
\pgfpathlineto{\pgfqpoint{1.176051in}{1.653812in}}%
\pgfpathlineto{\pgfqpoint{1.172148in}{1.653716in}}%
\pgfpathlineto{\pgfqpoint{1.168244in}{1.653485in}}%
\pgfpathlineto{\pgfqpoint{1.164341in}{1.653122in}}%
\pgfpathlineto{\pgfqpoint{1.164378in}{1.653351in}}%
\pgfpathlineto{\pgfqpoint{1.164430in}{1.653580in}}%
\pgfpathlineto{\pgfqpoint{1.164497in}{1.653807in}}%
\pgfpathlineto{\pgfqpoint{1.164580in}{1.654034in}}%
\pgfpathlineto{\pgfqpoint{1.168423in}{1.654170in}}%
\pgfpathlineto{\pgfqpoint{1.172266in}{1.654172in}}%
\pgfpathlineto{\pgfqpoint{1.176110in}{1.654041in}}%
\pgfpathlineto{\pgfqpoint{1.179955in}{1.653775in}}%
\pgfpathlineto{\pgfqpoint{1.179955in}{1.653775in}}%
\pgfpathlineto{\pgfqpoint{1.179955in}{1.653775in}}%
\pgfpathlineto{\pgfqpoint{1.179955in}{1.653775in}}%
\pgfpathlineto{\pgfqpoint{1.179955in}{1.653775in}}%
\pgfpathclose%
\pgfusepath{fill}%
\end{pgfscope}%
\begin{pgfscope}%
\pgfpathrectangle{\pgfqpoint{0.041670in}{0.041670in}}{\pgfqpoint{2.216660in}{2.216660in}}%
\pgfusepath{clip}%
\pgfsetbuttcap%
\pgfsetroundjoin%
\definecolor{currentfill}{rgb}{0.267004,0.004874,0.329415}%
\pgfsetfillcolor{currentfill}%
\pgfsetlinewidth{0.000000pt}%
\definecolor{currentstroke}{rgb}{0.000000,0.000000,0.000000}%
\pgfsetstrokecolor{currentstroke}%
\pgfsetdash{}{0pt}%
\pgfpathmoveto{\pgfqpoint{0.892022in}{0.594687in}}%
\pgfpathlineto{\pgfqpoint{0.890101in}{0.593080in}}%
\pgfpathlineto{\pgfqpoint{0.888174in}{0.591735in}}%
\pgfpathlineto{\pgfqpoint{0.886242in}{0.590655in}}%
\pgfpathlineto{\pgfqpoint{0.884306in}{0.589847in}}%
\pgfpathlineto{\pgfqpoint{0.867220in}{0.595057in}}%
\pgfpathlineto{\pgfqpoint{0.850483in}{0.600554in}}%
\pgfpathlineto{\pgfqpoint{0.834112in}{0.606329in}}%
\pgfpathlineto{\pgfqpoint{0.818124in}{0.612376in}}%
\pgfpathlineto{\pgfqpoint{0.820488in}{0.613035in}}%
\pgfpathlineto{\pgfqpoint{0.822845in}{0.613965in}}%
\pgfpathlineto{\pgfqpoint{0.825196in}{0.615160in}}%
\pgfpathlineto{\pgfqpoint{0.827541in}{0.616616in}}%
\pgfpathlineto{\pgfqpoint{0.843120in}{0.610730in}}%
\pgfpathlineto{\pgfqpoint{0.859071in}{0.605108in}}%
\pgfpathlineto{\pgfqpoint{0.875378in}{0.599759in}}%
\pgfpathlineto{\pgfqpoint{0.892022in}{0.594687in}}%
\pgfpathclose%
\pgfusepath{fill}%
\end{pgfscope}%
\begin{pgfscope}%
\pgfpathrectangle{\pgfqpoint{0.041670in}{0.041670in}}{\pgfqpoint{2.216660in}{2.216660in}}%
\pgfusepath{clip}%
\pgfsetbuttcap%
\pgfsetroundjoin%
\definecolor{currentfill}{rgb}{0.993248,0.906157,0.143936}%
\pgfsetfillcolor{currentfill}%
\pgfsetlinewidth{0.000000pt}%
\definecolor{currentstroke}{rgb}{0.000000,0.000000,0.000000}%
\pgfsetstrokecolor{currentstroke}%
\pgfsetdash{}{0pt}%
\pgfpathmoveto{\pgfqpoint{1.179955in}{1.653775in}}%
\pgfpathlineto{\pgfqpoint{1.183851in}{1.653863in}}%
\pgfpathlineto{\pgfqpoint{1.187746in}{1.653818in}}%
\pgfpathlineto{\pgfqpoint{1.191642in}{1.653638in}}%
\pgfpathlineto{\pgfqpoint{1.195537in}{1.653325in}}%
\pgfpathlineto{\pgfqpoint{1.195572in}{1.653096in}}%
\pgfpathlineto{\pgfqpoint{1.195591in}{1.652866in}}%
\pgfpathlineto{\pgfqpoint{1.195594in}{1.652636in}}%
\pgfpathlineto{\pgfqpoint{1.195582in}{1.652406in}}%
\pgfpathlineto{\pgfqpoint{1.191676in}{1.652949in}}%
\pgfpathlineto{\pgfqpoint{1.187769in}{1.653358in}}%
\pgfpathlineto{\pgfqpoint{1.183862in}{1.653634in}}%
\pgfpathlineto{\pgfqpoint{1.179955in}{1.653775in}}%
\pgfpathlineto{\pgfqpoint{1.179955in}{1.653775in}}%
\pgfpathlineto{\pgfqpoint{1.179955in}{1.653775in}}%
\pgfpathlineto{\pgfqpoint{1.179955in}{1.653775in}}%
\pgfpathlineto{\pgfqpoint{1.179955in}{1.653775in}}%
\pgfpathclose%
\pgfusepath{fill}%
\end{pgfscope}%
\begin{pgfscope}%
\pgfpathrectangle{\pgfqpoint{0.041670in}{0.041670in}}{\pgfqpoint{2.216660in}{2.216660in}}%
\pgfusepath{clip}%
\pgfsetbuttcap%
\pgfsetroundjoin%
\definecolor{currentfill}{rgb}{0.120081,0.622161,0.534946}%
\pgfsetfillcolor{currentfill}%
\pgfsetlinewidth{0.000000pt}%
\definecolor{currentstroke}{rgb}{0.000000,0.000000,0.000000}%
\pgfsetstrokecolor{currentstroke}%
\pgfsetdash{}{0pt}%
\pgfpathmoveto{\pgfqpoint{1.387084in}{1.214938in}}%
\pgfpathlineto{\pgfqpoint{1.389776in}{1.205961in}}%
\pgfpathlineto{\pgfqpoint{1.392466in}{1.196953in}}%
\pgfpathlineto{\pgfqpoint{1.395155in}{1.187916in}}%
\pgfpathlineto{\pgfqpoint{1.397842in}{1.178852in}}%
\pgfpathlineto{\pgfqpoint{1.390769in}{1.175466in}}%
\pgfpathlineto{\pgfqpoint{1.383475in}{1.172193in}}%
\pgfpathlineto{\pgfqpoint{1.375967in}{1.169034in}}%
\pgfpathlineto{\pgfqpoint{1.368255in}{1.165995in}}%
\pgfpathlineto{\pgfqpoint{1.365928in}{1.175234in}}%
\pgfpathlineto{\pgfqpoint{1.363600in}{1.184447in}}%
\pgfpathlineto{\pgfqpoint{1.361271in}{1.193630in}}%
\pgfpathlineto{\pgfqpoint{1.358941in}{1.202781in}}%
\pgfpathlineto{\pgfqpoint{1.366277in}{1.205654in}}%
\pgfpathlineto{\pgfqpoint{1.373417in}{1.208640in}}%
\pgfpathlineto{\pgfqpoint{1.380355in}{1.211736in}}%
\pgfpathlineto{\pgfqpoint{1.387084in}{1.214938in}}%
\pgfpathclose%
\pgfusepath{fill}%
\end{pgfscope}%
\begin{pgfscope}%
\pgfpathrectangle{\pgfqpoint{0.041670in}{0.041670in}}{\pgfqpoint{2.216660in}{2.216660in}}%
\pgfusepath{clip}%
\pgfsetbuttcap%
\pgfsetroundjoin%
\definecolor{currentfill}{rgb}{0.974417,0.903590,0.130215}%
\pgfsetfillcolor{currentfill}%
\pgfsetlinewidth{0.000000pt}%
\definecolor{currentstroke}{rgb}{0.000000,0.000000,0.000000}%
\pgfsetstrokecolor{currentstroke}%
\pgfsetdash{}{0pt}%
\pgfpathmoveto{\pgfqpoint{1.211206in}{1.648904in}}%
\pgfpathlineto{\pgfqpoint{1.215111in}{1.647696in}}%
\pgfpathlineto{\pgfqpoint{1.219016in}{1.646357in}}%
\pgfpathlineto{\pgfqpoint{1.222920in}{1.644885in}}%
\pgfpathlineto{\pgfqpoint{1.226823in}{1.643282in}}%
\pgfpathlineto{\pgfqpoint{1.226743in}{1.642593in}}%
\pgfpathlineto{\pgfqpoint{1.226617in}{1.641904in}}%
\pgfpathlineto{\pgfqpoint{1.226445in}{1.641218in}}%
\pgfpathlineto{\pgfqpoint{1.226226in}{1.640534in}}%
\pgfpathlineto{\pgfqpoint{1.222371in}{1.642367in}}%
\pgfpathlineto{\pgfqpoint{1.218515in}{1.644068in}}%
\pgfpathlineto{\pgfqpoint{1.214660in}{1.645636in}}%
\pgfpathlineto{\pgfqpoint{1.210804in}{1.647073in}}%
\pgfpathlineto{\pgfqpoint{1.210951in}{1.647528in}}%
\pgfpathlineto{\pgfqpoint{1.211067in}{1.647986in}}%
\pgfpathlineto{\pgfqpoint{1.211152in}{1.648444in}}%
\pgfpathlineto{\pgfqpoint{1.211206in}{1.648904in}}%
\pgfpathclose%
\pgfusepath{fill}%
\end{pgfscope}%
\begin{pgfscope}%
\pgfpathrectangle{\pgfqpoint{0.041670in}{0.041670in}}{\pgfqpoint{2.216660in}{2.216660in}}%
\pgfusepath{clip}%
\pgfsetbuttcap%
\pgfsetroundjoin%
\definecolor{currentfill}{rgb}{0.565498,0.842430,0.262877}%
\pgfsetfillcolor{currentfill}%
\pgfsetlinewidth{0.000000pt}%
\definecolor{currentstroke}{rgb}{0.000000,0.000000,0.000000}%
\pgfsetstrokecolor{currentstroke}%
\pgfsetdash{}{0pt}%
\pgfpathmoveto{\pgfqpoint{1.035393in}{1.485868in}}%
\pgfpathlineto{\pgfqpoint{1.032122in}{1.479430in}}%
\pgfpathlineto{\pgfqpoint{1.028854in}{1.472894in}}%
\pgfpathlineto{\pgfqpoint{1.025586in}{1.466261in}}%
\pgfpathlineto{\pgfqpoint{1.022320in}{1.459534in}}%
\pgfpathlineto{\pgfqpoint{1.019236in}{1.461945in}}%
\pgfpathlineto{\pgfqpoint{1.016316in}{1.464400in}}%
\pgfpathlineto{\pgfqpoint{1.013561in}{1.466898in}}%
\pgfpathlineto{\pgfqpoint{1.010976in}{1.469435in}}%
\pgfpathlineto{\pgfqpoint{1.014472in}{1.475948in}}%
\pgfpathlineto{\pgfqpoint{1.017969in}{1.482366in}}%
\pgfpathlineto{\pgfqpoint{1.021469in}{1.488689in}}%
\pgfpathlineto{\pgfqpoint{1.024970in}{1.494914in}}%
\pgfpathlineto{\pgfqpoint{1.027346in}{1.492596in}}%
\pgfpathlineto{\pgfqpoint{1.029877in}{1.490314in}}%
\pgfpathlineto{\pgfqpoint{1.032560in}{1.488071in}}%
\pgfpathlineto{\pgfqpoint{1.035393in}{1.485868in}}%
\pgfpathclose%
\pgfusepath{fill}%
\end{pgfscope}%
\begin{pgfscope}%
\pgfpathrectangle{\pgfqpoint{0.041670in}{0.041670in}}{\pgfqpoint{2.216660in}{2.216660in}}%
\pgfusepath{clip}%
\pgfsetbuttcap%
\pgfsetroundjoin%
\definecolor{currentfill}{rgb}{0.993248,0.906157,0.143936}%
\pgfsetfillcolor{currentfill}%
\pgfsetlinewidth{0.000000pt}%
\definecolor{currentstroke}{rgb}{0.000000,0.000000,0.000000}%
\pgfsetstrokecolor{currentstroke}%
\pgfsetdash{}{0pt}%
\pgfpathmoveto{\pgfqpoint{1.179955in}{1.653775in}}%
\pgfpathlineto{\pgfqpoint{1.176054in}{1.653582in}}%
\pgfpathlineto{\pgfqpoint{1.172153in}{1.653256in}}%
\pgfpathlineto{\pgfqpoint{1.168252in}{1.652796in}}%
\pgfpathlineto{\pgfqpoint{1.164351in}{1.652202in}}%
\pgfpathlineto{\pgfqpoint{1.164325in}{1.652432in}}%
\pgfpathlineto{\pgfqpoint{1.164315in}{1.652662in}}%
\pgfpathlineto{\pgfqpoint{1.164320in}{1.652892in}}%
\pgfpathlineto{\pgfqpoint{1.164341in}{1.653122in}}%
\pgfpathlineto{\pgfqpoint{1.168244in}{1.653485in}}%
\pgfpathlineto{\pgfqpoint{1.172148in}{1.653716in}}%
\pgfpathlineto{\pgfqpoint{1.176051in}{1.653812in}}%
\pgfpathlineto{\pgfqpoint{1.179955in}{1.653775in}}%
\pgfpathlineto{\pgfqpoint{1.179955in}{1.653775in}}%
\pgfpathlineto{\pgfqpoint{1.179955in}{1.653775in}}%
\pgfpathlineto{\pgfqpoint{1.179955in}{1.653775in}}%
\pgfpathlineto{\pgfqpoint{1.179955in}{1.653775in}}%
\pgfpathclose%
\pgfusepath{fill}%
\end{pgfscope}%
\begin{pgfscope}%
\pgfpathrectangle{\pgfqpoint{0.041670in}{0.041670in}}{\pgfqpoint{2.216660in}{2.216660in}}%
\pgfusepath{clip}%
\pgfsetbuttcap%
\pgfsetroundjoin%
\definecolor{currentfill}{rgb}{0.271305,0.019942,0.347269}%
\pgfsetfillcolor{currentfill}%
\pgfsetlinewidth{0.000000pt}%
\definecolor{currentstroke}{rgb}{0.000000,0.000000,0.000000}%
\pgfsetstrokecolor{currentstroke}%
\pgfsetdash{}{0pt}%
\pgfpathmoveto{\pgfqpoint{0.978898in}{0.616605in}}%
\pgfpathlineto{\pgfqpoint{0.977474in}{0.612125in}}%
\pgfpathlineto{\pgfqpoint{0.976046in}{0.607849in}}%
\pgfpathlineto{\pgfqpoint{0.974617in}{0.603781in}}%
\pgfpathlineto{\pgfqpoint{0.973184in}{0.599925in}}%
\pgfpathlineto{\pgfqpoint{0.956308in}{0.603607in}}%
\pgfpathlineto{\pgfqpoint{0.939680in}{0.607573in}}%
\pgfpathlineto{\pgfqpoint{0.923318in}{0.611819in}}%
\pgfpathlineto{\pgfqpoint{0.907241in}{0.616340in}}%
\pgfpathlineto{\pgfqpoint{0.909126in}{0.620074in}}%
\pgfpathlineto{\pgfqpoint{0.911008in}{0.624020in}}%
\pgfpathlineto{\pgfqpoint{0.912886in}{0.628174in}}%
\pgfpathlineto{\pgfqpoint{0.914761in}{0.632532in}}%
\pgfpathlineto{\pgfqpoint{0.930399in}{0.628145in}}%
\pgfpathlineto{\pgfqpoint{0.946313in}{0.624025in}}%
\pgfpathlineto{\pgfqpoint{0.962485in}{0.620177in}}%
\pgfpathlineto{\pgfqpoint{0.978898in}{0.616605in}}%
\pgfpathclose%
\pgfusepath{fill}%
\end{pgfscope}%
\begin{pgfscope}%
\pgfpathrectangle{\pgfqpoint{0.041670in}{0.041670in}}{\pgfqpoint{2.216660in}{2.216660in}}%
\pgfusepath{clip}%
\pgfsetbuttcap%
\pgfsetroundjoin%
\definecolor{currentfill}{rgb}{0.147607,0.511733,0.557049}%
\pgfsetfillcolor{currentfill}%
\pgfsetlinewidth{0.000000pt}%
\definecolor{currentstroke}{rgb}{0.000000,0.000000,0.000000}%
\pgfsetstrokecolor{currentstroke}%
\pgfsetdash{}{0pt}%
\pgfpathmoveto{\pgfqpoint{1.386839in}{1.091437in}}%
\pgfpathlineto{\pgfqpoint{1.389158in}{1.082078in}}%
\pgfpathlineto{\pgfqpoint{1.391477in}{1.072719in}}%
\pgfpathlineto{\pgfqpoint{1.393795in}{1.063364in}}%
\pgfpathlineto{\pgfqpoint{1.396112in}{1.054015in}}%
\pgfpathlineto{\pgfqpoint{1.387044in}{1.050613in}}%
\pgfpathlineto{\pgfqpoint{1.377756in}{1.047356in}}%
\pgfpathlineto{\pgfqpoint{1.368259in}{1.044249in}}%
\pgfpathlineto{\pgfqpoint{1.358561in}{1.041295in}}%
\pgfpathlineto{\pgfqpoint{1.356643in}{1.050797in}}%
\pgfpathlineto{\pgfqpoint{1.354724in}{1.060305in}}%
\pgfpathlineto{\pgfqpoint{1.352805in}{1.069816in}}%
\pgfpathlineto{\pgfqpoint{1.350885in}{1.079328in}}%
\pgfpathlineto{\pgfqpoint{1.360170in}{1.082139in}}%
\pgfpathlineto{\pgfqpoint{1.369263in}{1.085097in}}%
\pgfpathlineto{\pgfqpoint{1.378156in}{1.088197in}}%
\pgfpathlineto{\pgfqpoint{1.386839in}{1.091437in}}%
\pgfpathclose%
\pgfusepath{fill}%
\end{pgfscope}%
\begin{pgfscope}%
\pgfpathrectangle{\pgfqpoint{0.041670in}{0.041670in}}{\pgfqpoint{2.216660in}{2.216660in}}%
\pgfusepath{clip}%
\pgfsetbuttcap%
\pgfsetroundjoin%
\definecolor{currentfill}{rgb}{0.220124,0.725509,0.466226}%
\pgfsetfillcolor{currentfill}%
\pgfsetlinewidth{0.000000pt}%
\definecolor{currentstroke}{rgb}{0.000000,0.000000,0.000000}%
\pgfsetstrokecolor{currentstroke}%
\pgfsetdash{}{0pt}%
\pgfpathmoveto{\pgfqpoint{1.375584in}{1.330758in}}%
\pgfpathlineto{\pgfqpoint{1.378608in}{1.322543in}}%
\pgfpathlineto{\pgfqpoint{1.381632in}{1.314268in}}%
\pgfpathlineto{\pgfqpoint{1.384654in}{1.305936in}}%
\pgfpathlineto{\pgfqpoint{1.387674in}{1.297548in}}%
\pgfpathlineto{\pgfqpoint{1.382441in}{1.294364in}}%
\pgfpathlineto{\pgfqpoint{1.376999in}{1.291261in}}%
\pgfpathlineto{\pgfqpoint{1.371353in}{1.288244in}}%
\pgfpathlineto{\pgfqpoint{1.365509in}{1.285314in}}%
\pgfpathlineto{\pgfqpoint{1.362806in}{1.293895in}}%
\pgfpathlineto{\pgfqpoint{1.360102in}{1.302420in}}%
\pgfpathlineto{\pgfqpoint{1.357397in}{1.310887in}}%
\pgfpathlineto{\pgfqpoint{1.354690in}{1.319294in}}%
\pgfpathlineto{\pgfqpoint{1.360198in}{1.322039in}}%
\pgfpathlineto{\pgfqpoint{1.365520in}{1.324867in}}%
\pgfpathlineto{\pgfqpoint{1.370650in}{1.327774in}}%
\pgfpathlineto{\pgfqpoint{1.375584in}{1.330758in}}%
\pgfpathclose%
\pgfusepath{fill}%
\end{pgfscope}%
\begin{pgfscope}%
\pgfpathrectangle{\pgfqpoint{0.041670in}{0.041670in}}{\pgfqpoint{2.216660in}{2.216660in}}%
\pgfusepath{clip}%
\pgfsetbuttcap%
\pgfsetroundjoin%
\definecolor{currentfill}{rgb}{0.993248,0.906157,0.143936}%
\pgfsetfillcolor{currentfill}%
\pgfsetlinewidth{0.000000pt}%
\definecolor{currentstroke}{rgb}{0.000000,0.000000,0.000000}%
\pgfsetstrokecolor{currentstroke}%
\pgfsetdash{}{0pt}%
\pgfpathmoveto{\pgfqpoint{1.179955in}{1.653775in}}%
\pgfpathlineto{\pgfqpoint{1.183862in}{1.653634in}}%
\pgfpathlineto{\pgfqpoint{1.187769in}{1.653358in}}%
\pgfpathlineto{\pgfqpoint{1.191676in}{1.652949in}}%
\pgfpathlineto{\pgfqpoint{1.195582in}{1.652406in}}%
\pgfpathlineto{\pgfqpoint{1.195555in}{1.652176in}}%
\pgfpathlineto{\pgfqpoint{1.195512in}{1.651947in}}%
\pgfpathlineto{\pgfqpoint{1.195453in}{1.651719in}}%
\pgfpathlineto{\pgfqpoint{1.195379in}{1.651491in}}%
\pgfpathlineto{\pgfqpoint{1.191523in}{1.652263in}}%
\pgfpathlineto{\pgfqpoint{1.187666in}{1.652901in}}%
\pgfpathlineto{\pgfqpoint{1.183810in}{1.653405in}}%
\pgfpathlineto{\pgfqpoint{1.179955in}{1.653775in}}%
\pgfpathlineto{\pgfqpoint{1.179955in}{1.653775in}}%
\pgfpathlineto{\pgfqpoint{1.179955in}{1.653775in}}%
\pgfpathlineto{\pgfqpoint{1.179955in}{1.653775in}}%
\pgfpathlineto{\pgfqpoint{1.179955in}{1.653775in}}%
\pgfpathclose%
\pgfusepath{fill}%
\end{pgfscope}%
\begin{pgfscope}%
\pgfpathrectangle{\pgfqpoint{0.041670in}{0.041670in}}{\pgfqpoint{2.216660in}{2.216660in}}%
\pgfusepath{clip}%
\pgfsetbuttcap%
\pgfsetroundjoin%
\definecolor{currentfill}{rgb}{0.762373,0.876424,0.137064}%
\pgfsetfillcolor{currentfill}%
\pgfsetlinewidth{0.000000pt}%
\definecolor{currentstroke}{rgb}{0.000000,0.000000,0.000000}%
\pgfsetstrokecolor{currentstroke}%
\pgfsetdash{}{0pt}%
\pgfpathmoveto{\pgfqpoint{1.067100in}{1.561462in}}%
\pgfpathlineto{\pgfqpoint{1.063582in}{1.556522in}}%
\pgfpathlineto{\pgfqpoint{1.060065in}{1.551467in}}%
\pgfpathlineto{\pgfqpoint{1.056549in}{1.546299in}}%
\pgfpathlineto{\pgfqpoint{1.053035in}{1.541019in}}%
\pgfpathlineto{\pgfqpoint{1.051209in}{1.542932in}}%
\pgfpathlineto{\pgfqpoint{1.049513in}{1.544870in}}%
\pgfpathlineto{\pgfqpoint{1.047949in}{1.546833in}}%
\pgfpathlineto{\pgfqpoint{1.046518in}{1.548817in}}%
\pgfpathlineto{\pgfqpoint{1.050209in}{1.553875in}}%
\pgfpathlineto{\pgfqpoint{1.053901in}{1.558821in}}%
\pgfpathlineto{\pgfqpoint{1.057595in}{1.563655in}}%
\pgfpathlineto{\pgfqpoint{1.061290in}{1.568374in}}%
\pgfpathlineto{\pgfqpoint{1.062567in}{1.566615in}}%
\pgfpathlineto{\pgfqpoint{1.063962in}{1.564876in}}%
\pgfpathlineto{\pgfqpoint{1.065473in}{1.563157in}}%
\pgfpathlineto{\pgfqpoint{1.067100in}{1.561462in}}%
\pgfpathclose%
\pgfusepath{fill}%
\end{pgfscope}%
\begin{pgfscope}%
\pgfpathrectangle{\pgfqpoint{0.041670in}{0.041670in}}{\pgfqpoint{2.216660in}{2.216660in}}%
\pgfusepath{clip}%
\pgfsetbuttcap%
\pgfsetroundjoin%
\definecolor{currentfill}{rgb}{0.974417,0.903590,0.130215}%
\pgfsetfillcolor{currentfill}%
\pgfsetlinewidth{0.000000pt}%
\definecolor{currentstroke}{rgb}{0.000000,0.000000,0.000000}%
\pgfsetstrokecolor{currentstroke}%
\pgfsetdash{}{0pt}%
\pgfpathmoveto{\pgfqpoint{1.149262in}{1.646670in}}%
\pgfpathlineto{\pgfqpoint{1.145426in}{1.645183in}}%
\pgfpathlineto{\pgfqpoint{1.141589in}{1.643564in}}%
\pgfpathlineto{\pgfqpoint{1.137753in}{1.641813in}}%
\pgfpathlineto{\pgfqpoint{1.133918in}{1.639930in}}%
\pgfpathlineto{\pgfqpoint{1.133658in}{1.640610in}}%
\pgfpathlineto{\pgfqpoint{1.133444in}{1.641294in}}%
\pgfpathlineto{\pgfqpoint{1.133276in}{1.641981in}}%
\pgfpathlineto{\pgfqpoint{1.133155in}{1.642669in}}%
\pgfpathlineto{\pgfqpoint{1.137053in}{1.644323in}}%
\pgfpathlineto{\pgfqpoint{1.140951in}{1.645846in}}%
\pgfpathlineto{\pgfqpoint{1.144850in}{1.647237in}}%
\pgfpathlineto{\pgfqpoint{1.148750in}{1.648495in}}%
\pgfpathlineto{\pgfqpoint{1.148832in}{1.648036in}}%
\pgfpathlineto{\pgfqpoint{1.148944in}{1.647579in}}%
\pgfpathlineto{\pgfqpoint{1.149088in}{1.647124in}}%
\pgfpathlineto{\pgfqpoint{1.149262in}{1.646670in}}%
\pgfpathclose%
\pgfusepath{fill}%
\end{pgfscope}%
\begin{pgfscope}%
\pgfpathrectangle{\pgfqpoint{0.041670in}{0.041670in}}{\pgfqpoint{2.216660in}{2.216660in}}%
\pgfusepath{clip}%
\pgfsetbuttcap%
\pgfsetroundjoin%
\definecolor{currentfill}{rgb}{0.993248,0.906157,0.143936}%
\pgfsetfillcolor{currentfill}%
\pgfsetlinewidth{0.000000pt}%
\definecolor{currentstroke}{rgb}{0.000000,0.000000,0.000000}%
\pgfsetstrokecolor{currentstroke}%
\pgfsetdash{}{0pt}%
\pgfpathmoveto{\pgfqpoint{1.179955in}{1.653775in}}%
\pgfpathlineto{\pgfqpoint{1.176119in}{1.653355in}}%
\pgfpathlineto{\pgfqpoint{1.172283in}{1.652800in}}%
\pgfpathlineto{\pgfqpoint{1.168446in}{1.652112in}}%
\pgfpathlineto{\pgfqpoint{1.164610in}{1.651290in}}%
\pgfpathlineto{\pgfqpoint{1.164522in}{1.651517in}}%
\pgfpathlineto{\pgfqpoint{1.164450in}{1.651744in}}%
\pgfpathlineto{\pgfqpoint{1.164393in}{1.651973in}}%
\pgfpathlineto{\pgfqpoint{1.164351in}{1.652202in}}%
\pgfpathlineto{\pgfqpoint{1.168252in}{1.652796in}}%
\pgfpathlineto{\pgfqpoint{1.172153in}{1.653256in}}%
\pgfpathlineto{\pgfqpoint{1.176054in}{1.653582in}}%
\pgfpathlineto{\pgfqpoint{1.179955in}{1.653775in}}%
\pgfpathlineto{\pgfqpoint{1.179955in}{1.653775in}}%
\pgfpathlineto{\pgfqpoint{1.179955in}{1.653775in}}%
\pgfpathlineto{\pgfqpoint{1.179955in}{1.653775in}}%
\pgfpathlineto{\pgfqpoint{1.179955in}{1.653775in}}%
\pgfpathclose%
\pgfusepath{fill}%
\end{pgfscope}%
\begin{pgfscope}%
\pgfpathrectangle{\pgfqpoint{0.041670in}{0.041670in}}{\pgfqpoint{2.216660in}{2.216660in}}%
\pgfusepath{clip}%
\pgfsetbuttcap%
\pgfsetroundjoin%
\definecolor{currentfill}{rgb}{0.993248,0.906157,0.143936}%
\pgfsetfillcolor{currentfill}%
\pgfsetlinewidth{0.000000pt}%
\definecolor{currentstroke}{rgb}{0.000000,0.000000,0.000000}%
\pgfsetstrokecolor{currentstroke}%
\pgfsetdash{}{0pt}%
\pgfpathmoveto{\pgfqpoint{1.195582in}{1.652406in}}%
\pgfpathlineto{\pgfqpoint{1.199489in}{1.651730in}}%
\pgfpathlineto{\pgfqpoint{1.203395in}{1.650921in}}%
\pgfpathlineto{\pgfqpoint{1.207301in}{1.649979in}}%
\pgfpathlineto{\pgfqpoint{1.211206in}{1.648904in}}%
\pgfpathlineto{\pgfqpoint{1.211152in}{1.648444in}}%
\pgfpathlineto{\pgfqpoint{1.211067in}{1.647986in}}%
\pgfpathlineto{\pgfqpoint{1.210951in}{1.647528in}}%
\pgfpathlineto{\pgfqpoint{1.210804in}{1.647073in}}%
\pgfpathlineto{\pgfqpoint{1.206948in}{1.648377in}}%
\pgfpathlineto{\pgfqpoint{1.203092in}{1.649548in}}%
\pgfpathlineto{\pgfqpoint{1.199235in}{1.650587in}}%
\pgfpathlineto{\pgfqpoint{1.195379in}{1.651491in}}%
\pgfpathlineto{\pgfqpoint{1.195453in}{1.651719in}}%
\pgfpathlineto{\pgfqpoint{1.195512in}{1.651947in}}%
\pgfpathlineto{\pgfqpoint{1.195555in}{1.652176in}}%
\pgfpathlineto{\pgfqpoint{1.195582in}{1.652406in}}%
\pgfpathclose%
\pgfusepath{fill}%
\end{pgfscope}%
\begin{pgfscope}%
\pgfpathrectangle{\pgfqpoint{0.041670in}{0.041670in}}{\pgfqpoint{2.216660in}{2.216660in}}%
\pgfusepath{clip}%
\pgfsetbuttcap%
\pgfsetroundjoin%
\definecolor{currentfill}{rgb}{0.212395,0.359683,0.551710}%
\pgfsetfillcolor{currentfill}%
\pgfsetlinewidth{0.000000pt}%
\definecolor{currentstroke}{rgb}{0.000000,0.000000,0.000000}%
\pgfsetstrokecolor{currentstroke}%
\pgfsetdash{}{0pt}%
\pgfpathmoveto{\pgfqpoint{1.034706in}{0.914931in}}%
\pgfpathlineto{\pgfqpoint{1.033323in}{0.905635in}}%
\pgfpathlineto{\pgfqpoint{1.031941in}{0.896388in}}%
\pgfpathlineto{\pgfqpoint{1.030558in}{0.887194in}}%
\pgfpathlineto{\pgfqpoint{1.029175in}{0.878056in}}%
\pgfpathlineto{\pgfqpoint{1.016847in}{0.880642in}}%
\pgfpathlineto{\pgfqpoint{1.004695in}{0.883428in}}%
\pgfpathlineto{\pgfqpoint{0.992732in}{0.886412in}}%
\pgfpathlineto{\pgfqpoint{0.980971in}{0.889589in}}%
\pgfpathlineto{\pgfqpoint{0.982793in}{0.898607in}}%
\pgfpathlineto{\pgfqpoint{0.984614in}{0.907681in}}%
\pgfpathlineto{\pgfqpoint{0.986436in}{0.916808in}}%
\pgfpathlineto{\pgfqpoint{0.988258in}{0.925985in}}%
\pgfpathlineto{\pgfqpoint{0.999591in}{0.922939in}}%
\pgfpathlineto{\pgfqpoint{1.011118in}{0.920079in}}%
\pgfpathlineto{\pgfqpoint{1.022827in}{0.917409in}}%
\pgfpathlineto{\pgfqpoint{1.034706in}{0.914931in}}%
\pgfpathclose%
\pgfusepath{fill}%
\end{pgfscope}%
\begin{pgfscope}%
\pgfpathrectangle{\pgfqpoint{0.041670in}{0.041670in}}{\pgfqpoint{2.216660in}{2.216660in}}%
\pgfusepath{clip}%
\pgfsetbuttcap%
\pgfsetroundjoin%
\definecolor{currentfill}{rgb}{0.896320,0.893616,0.096335}%
\pgfsetfillcolor{currentfill}%
\pgfsetlinewidth{0.000000pt}%
\definecolor{currentstroke}{rgb}{0.000000,0.000000,0.000000}%
\pgfsetstrokecolor{currentstroke}%
\pgfsetdash{}{0pt}%
\pgfpathmoveto{\pgfqpoint{1.105727in}{1.615719in}}%
\pgfpathlineto{\pgfqpoint{1.102019in}{1.612450in}}%
\pgfpathlineto{\pgfqpoint{1.098311in}{1.609055in}}%
\pgfpathlineto{\pgfqpoint{1.094604in}{1.605535in}}%
\pgfpathlineto{\pgfqpoint{1.090898in}{1.601891in}}%
\pgfpathlineto{\pgfqpoint{1.090024in}{1.603218in}}%
\pgfpathlineto{\pgfqpoint{1.089239in}{1.604556in}}%
\pgfpathlineto{\pgfqpoint{1.088545in}{1.605906in}}%
\pgfpathlineto{\pgfqpoint{1.087943in}{1.607264in}}%
\pgfpathlineto{\pgfqpoint{1.091769in}{1.610682in}}%
\pgfpathlineto{\pgfqpoint{1.095596in}{1.613976in}}%
\pgfpathlineto{\pgfqpoint{1.099424in}{1.617145in}}%
\pgfpathlineto{\pgfqpoint{1.103254in}{1.620189in}}%
\pgfpathlineto{\pgfqpoint{1.103758in}{1.619059in}}%
\pgfpathlineto{\pgfqpoint{1.104339in}{1.617936in}}%
\pgfpathlineto{\pgfqpoint{1.104996in}{1.616823in}}%
\pgfpathlineto{\pgfqpoint{1.105727in}{1.615719in}}%
\pgfpathclose%
\pgfusepath{fill}%
\end{pgfscope}%
\begin{pgfscope}%
\pgfpathrectangle{\pgfqpoint{0.041670in}{0.041670in}}{\pgfqpoint{2.216660in}{2.216660in}}%
\pgfusepath{clip}%
\pgfsetbuttcap%
\pgfsetroundjoin%
\definecolor{currentfill}{rgb}{0.280255,0.165693,0.476498}%
\pgfsetfillcolor{currentfill}%
\pgfsetlinewidth{0.000000pt}%
\definecolor{currentstroke}{rgb}{0.000000,0.000000,0.000000}%
\pgfsetstrokecolor{currentstroke}%
\pgfsetdash{}{0pt}%
\pgfpathmoveto{\pgfqpoint{1.365484in}{0.744945in}}%
\pgfpathlineto{\pgfqpoint{1.366976in}{0.737316in}}%
\pgfpathlineto{\pgfqpoint{1.368469in}{0.729808in}}%
\pgfpathlineto{\pgfqpoint{1.369964in}{0.722424in}}%
\pgfpathlineto{\pgfqpoint{1.371460in}{0.715167in}}%
\pgfpathlineto{\pgfqpoint{1.356853in}{0.712063in}}%
\pgfpathlineto{\pgfqpoint{1.342051in}{0.709206in}}%
\pgfpathlineto{\pgfqpoint{1.327072in}{0.706598in}}%
\pgfpathlineto{\pgfqpoint{1.311930in}{0.704244in}}%
\pgfpathlineto{\pgfqpoint{1.310898in}{0.711596in}}%
\pgfpathlineto{\pgfqpoint{1.309866in}{0.719076in}}%
\pgfpathlineto{\pgfqpoint{1.308836in}{0.726680in}}%
\pgfpathlineto{\pgfqpoint{1.307806in}{0.734405in}}%
\pgfpathlineto{\pgfqpoint{1.322475in}{0.736676in}}%
\pgfpathlineto{\pgfqpoint{1.336989in}{0.739192in}}%
\pgfpathlineto{\pgfqpoint{1.351330in}{0.741949in}}%
\pgfpathlineto{\pgfqpoint{1.365484in}{0.744945in}}%
\pgfpathclose%
\pgfusepath{fill}%
\end{pgfscope}%
\begin{pgfscope}%
\pgfpathrectangle{\pgfqpoint{0.041670in}{0.041670in}}{\pgfqpoint{2.216660in}{2.216660in}}%
\pgfusepath{clip}%
\pgfsetbuttcap%
\pgfsetroundjoin%
\definecolor{currentfill}{rgb}{0.412913,0.803041,0.357269}%
\pgfsetfillcolor{currentfill}%
\pgfsetlinewidth{0.000000pt}%
\definecolor{currentstroke}{rgb}{0.000000,0.000000,0.000000}%
\pgfsetstrokecolor{currentstroke}%
\pgfsetdash{}{0pt}%
\pgfpathmoveto{\pgfqpoint{1.353610in}{1.434040in}}%
\pgfpathlineto{\pgfqpoint{1.356924in}{1.426912in}}%
\pgfpathlineto{\pgfqpoint{1.360236in}{1.419700in}}%
\pgfpathlineto{\pgfqpoint{1.363546in}{1.412406in}}%
\pgfpathlineto{\pgfqpoint{1.366854in}{1.405032in}}%
\pgfpathlineto{\pgfqpoint{1.363250in}{1.402205in}}%
\pgfpathlineto{\pgfqpoint{1.359459in}{1.399432in}}%
\pgfpathlineto{\pgfqpoint{1.355484in}{1.396718in}}%
\pgfpathlineto{\pgfqpoint{1.351329in}{1.394065in}}%
\pgfpathlineto{\pgfqpoint{1.348290in}{1.401645in}}%
\pgfpathlineto{\pgfqpoint{1.345250in}{1.409145in}}%
\pgfpathlineto{\pgfqpoint{1.342209in}{1.416562in}}%
\pgfpathlineto{\pgfqpoint{1.339166in}{1.423894in}}%
\pgfpathlineto{\pgfqpoint{1.343031in}{1.426348in}}%
\pgfpathlineto{\pgfqpoint{1.346729in}{1.428859in}}%
\pgfpathlineto{\pgfqpoint{1.350256in}{1.431424in}}%
\pgfpathlineto{\pgfqpoint{1.353610in}{1.434040in}}%
\pgfpathclose%
\pgfusepath{fill}%
\end{pgfscope}%
\begin{pgfscope}%
\pgfpathrectangle{\pgfqpoint{0.041670in}{0.041670in}}{\pgfqpoint{2.216660in}{2.216660in}}%
\pgfusepath{clip}%
\pgfsetbuttcap%
\pgfsetroundjoin%
\definecolor{currentfill}{rgb}{0.993248,0.906157,0.143936}%
\pgfsetfillcolor{currentfill}%
\pgfsetlinewidth{0.000000pt}%
\definecolor{currentstroke}{rgb}{0.000000,0.000000,0.000000}%
\pgfsetstrokecolor{currentstroke}%
\pgfsetdash{}{0pt}%
\pgfpathmoveto{\pgfqpoint{1.164610in}{1.651290in}}%
\pgfpathlineto{\pgfqpoint{1.160773in}{1.650335in}}%
\pgfpathlineto{\pgfqpoint{1.156936in}{1.649247in}}%
\pgfpathlineto{\pgfqpoint{1.153099in}{1.648025in}}%
\pgfpathlineto{\pgfqpoint{1.149262in}{1.646670in}}%
\pgfpathlineto{\pgfqpoint{1.149088in}{1.647124in}}%
\pgfpathlineto{\pgfqpoint{1.148944in}{1.647579in}}%
\pgfpathlineto{\pgfqpoint{1.148832in}{1.648036in}}%
\pgfpathlineto{\pgfqpoint{1.148750in}{1.648495in}}%
\pgfpathlineto{\pgfqpoint{1.152650in}{1.649621in}}%
\pgfpathlineto{\pgfqpoint{1.156550in}{1.650615in}}%
\pgfpathlineto{\pgfqpoint{1.160450in}{1.651475in}}%
\pgfpathlineto{\pgfqpoint{1.164351in}{1.652202in}}%
\pgfpathlineto{\pgfqpoint{1.164393in}{1.651973in}}%
\pgfpathlineto{\pgfqpoint{1.164450in}{1.651744in}}%
\pgfpathlineto{\pgfqpoint{1.164522in}{1.651517in}}%
\pgfpathlineto{\pgfqpoint{1.164610in}{1.651290in}}%
\pgfpathclose%
\pgfusepath{fill}%
\end{pgfscope}%
\begin{pgfscope}%
\pgfpathrectangle{\pgfqpoint{0.041670in}{0.041670in}}{\pgfqpoint{2.216660in}{2.216660in}}%
\pgfusepath{clip}%
\pgfsetbuttcap%
\pgfsetroundjoin%
\definecolor{currentfill}{rgb}{0.993248,0.906157,0.143936}%
\pgfsetfillcolor{currentfill}%
\pgfsetlinewidth{0.000000pt}%
\definecolor{currentstroke}{rgb}{0.000000,0.000000,0.000000}%
\pgfsetstrokecolor{currentstroke}%
\pgfsetdash{}{0pt}%
\pgfpathmoveto{\pgfqpoint{1.179955in}{1.653775in}}%
\pgfpathlineto{\pgfqpoint{1.183810in}{1.653405in}}%
\pgfpathlineto{\pgfqpoint{1.187666in}{1.652901in}}%
\pgfpathlineto{\pgfqpoint{1.191523in}{1.652263in}}%
\pgfpathlineto{\pgfqpoint{1.195379in}{1.651491in}}%
\pgfpathlineto{\pgfqpoint{1.195289in}{1.651265in}}%
\pgfpathlineto{\pgfqpoint{1.195185in}{1.651040in}}%
\pgfpathlineto{\pgfqpoint{1.195065in}{1.650817in}}%
\pgfpathlineto{\pgfqpoint{1.194930in}{1.650596in}}%
\pgfpathlineto{\pgfqpoint{1.191185in}{1.651591in}}%
\pgfpathlineto{\pgfqpoint{1.187441in}{1.652453in}}%
\pgfpathlineto{\pgfqpoint{1.183698in}{1.653181in}}%
\pgfpathlineto{\pgfqpoint{1.179955in}{1.653775in}}%
\pgfpathlineto{\pgfqpoint{1.179955in}{1.653775in}}%
\pgfpathlineto{\pgfqpoint{1.179955in}{1.653775in}}%
\pgfpathlineto{\pgfqpoint{1.179955in}{1.653775in}}%
\pgfpathlineto{\pgfqpoint{1.179955in}{1.653775in}}%
\pgfpathclose%
\pgfusepath{fill}%
\end{pgfscope}%
\begin{pgfscope}%
\pgfpathrectangle{\pgfqpoint{0.041670in}{0.041670in}}{\pgfqpoint{2.216660in}{2.216660in}}%
\pgfusepath{clip}%
\pgfsetbuttcap%
\pgfsetroundjoin%
\definecolor{currentfill}{rgb}{0.283072,0.130895,0.449241}%
\pgfsetfillcolor{currentfill}%
\pgfsetlinewidth{0.000000pt}%
\definecolor{currentstroke}{rgb}{0.000000,0.000000,0.000000}%
\pgfsetstrokecolor{currentstroke}%
\pgfsetdash{}{0pt}%
\pgfpathmoveto{\pgfqpoint{1.371460in}{0.715167in}}%
\pgfpathlineto{\pgfqpoint{1.372957in}{0.708043in}}%
\pgfpathlineto{\pgfqpoint{1.374455in}{0.701055in}}%
\pgfpathlineto{\pgfqpoint{1.375955in}{0.694207in}}%
\pgfpathlineto{\pgfqpoint{1.377456in}{0.687502in}}%
\pgfpathlineto{\pgfqpoint{1.362394in}{0.684288in}}%
\pgfpathlineto{\pgfqpoint{1.347131in}{0.681330in}}%
\pgfpathlineto{\pgfqpoint{1.331684in}{0.678631in}}%
\pgfpathlineto{\pgfqpoint{1.316069in}{0.676193in}}%
\pgfpathlineto{\pgfqpoint{1.315033in}{0.682994in}}%
\pgfpathlineto{\pgfqpoint{1.313998in}{0.689939in}}%
\pgfpathlineto{\pgfqpoint{1.312963in}{0.697023in}}%
\pgfpathlineto{\pgfqpoint{1.311930in}{0.704244in}}%
\pgfpathlineto{\pgfqpoint{1.327072in}{0.706598in}}%
\pgfpathlineto{\pgfqpoint{1.342051in}{0.709206in}}%
\pgfpathlineto{\pgfqpoint{1.356853in}{0.712063in}}%
\pgfpathlineto{\pgfqpoint{1.371460in}{0.715167in}}%
\pgfpathclose%
\pgfusepath{fill}%
\end{pgfscope}%
\begin{pgfscope}%
\pgfpathrectangle{\pgfqpoint{0.041670in}{0.041670in}}{\pgfqpoint{2.216660in}{2.216660in}}%
\pgfusepath{clip}%
\pgfsetbuttcap%
\pgfsetroundjoin%
\definecolor{currentfill}{rgb}{0.993248,0.906157,0.143936}%
\pgfsetfillcolor{currentfill}%
\pgfsetlinewidth{0.000000pt}%
\definecolor{currentstroke}{rgb}{0.000000,0.000000,0.000000}%
\pgfsetstrokecolor{currentstroke}%
\pgfsetdash{}{0pt}%
\pgfpathmoveto{\pgfqpoint{1.179955in}{1.653775in}}%
\pgfpathlineto{\pgfqpoint{1.176245in}{1.653132in}}%
\pgfpathlineto{\pgfqpoint{1.172535in}{1.652356in}}%
\pgfpathlineto{\pgfqpoint{1.168824in}{1.651445in}}%
\pgfpathlineto{\pgfqpoint{1.165113in}{1.650401in}}%
\pgfpathlineto{\pgfqpoint{1.164964in}{1.650620in}}%
\pgfpathlineto{\pgfqpoint{1.164831in}{1.650842in}}%
\pgfpathlineto{\pgfqpoint{1.164713in}{1.651065in}}%
\pgfpathlineto{\pgfqpoint{1.164610in}{1.651290in}}%
\pgfpathlineto{\pgfqpoint{1.168446in}{1.652112in}}%
\pgfpathlineto{\pgfqpoint{1.172283in}{1.652800in}}%
\pgfpathlineto{\pgfqpoint{1.176119in}{1.653355in}}%
\pgfpathlineto{\pgfqpoint{1.179955in}{1.653775in}}%
\pgfpathlineto{\pgfqpoint{1.179955in}{1.653775in}}%
\pgfpathlineto{\pgfqpoint{1.179955in}{1.653775in}}%
\pgfpathlineto{\pgfqpoint{1.179955in}{1.653775in}}%
\pgfpathlineto{\pgfqpoint{1.179955in}{1.653775in}}%
\pgfpathclose%
\pgfusepath{fill}%
\end{pgfscope}%
\begin{pgfscope}%
\pgfpathrectangle{\pgfqpoint{0.041670in}{0.041670in}}{\pgfqpoint{2.216660in}{2.216660in}}%
\pgfusepath{clip}%
\pgfsetbuttcap%
\pgfsetroundjoin%
\definecolor{currentfill}{rgb}{0.935904,0.898570,0.108131}%
\pgfsetfillcolor{currentfill}%
\pgfsetlinewidth{0.000000pt}%
\definecolor{currentstroke}{rgb}{0.000000,0.000000,0.000000}%
\pgfsetstrokecolor{currentstroke}%
\pgfsetdash{}{0pt}%
\pgfpathmoveto{\pgfqpoint{1.241639in}{1.631899in}}%
\pgfpathlineto{\pgfqpoint{1.245491in}{1.629416in}}%
\pgfpathlineto{\pgfqpoint{1.249342in}{1.626804in}}%
\pgfpathlineto{\pgfqpoint{1.253191in}{1.624065in}}%
\pgfpathlineto{\pgfqpoint{1.257040in}{1.621199in}}%
\pgfpathlineto{\pgfqpoint{1.256604in}{1.620063in}}%
\pgfpathlineto{\pgfqpoint{1.256091in}{1.618933in}}%
\pgfpathlineto{\pgfqpoint{1.255501in}{1.617812in}}%
\pgfpathlineto{\pgfqpoint{1.254836in}{1.616699in}}%
\pgfpathlineto{\pgfqpoint{1.251095in}{1.619792in}}%
\pgfpathlineto{\pgfqpoint{1.247353in}{1.622758in}}%
\pgfpathlineto{\pgfqpoint{1.243610in}{1.625595in}}%
\pgfpathlineto{\pgfqpoint{1.239867in}{1.628304in}}%
\pgfpathlineto{\pgfqpoint{1.240401in}{1.629193in}}%
\pgfpathlineto{\pgfqpoint{1.240875in}{1.630089in}}%
\pgfpathlineto{\pgfqpoint{1.241288in}{1.630991in}}%
\pgfpathlineto{\pgfqpoint{1.241639in}{1.631899in}}%
\pgfpathclose%
\pgfusepath{fill}%
\end{pgfscope}%
\begin{pgfscope}%
\pgfpathrectangle{\pgfqpoint{0.041670in}{0.041670in}}{\pgfqpoint{2.216660in}{2.216660in}}%
\pgfusepath{clip}%
\pgfsetbuttcap%
\pgfsetroundjoin%
\definecolor{currentfill}{rgb}{0.195860,0.395433,0.555276}%
\pgfsetfillcolor{currentfill}%
\pgfsetlinewidth{0.000000pt}%
\definecolor{currentstroke}{rgb}{0.000000,0.000000,0.000000}%
\pgfsetstrokecolor{currentstroke}%
\pgfsetdash{}{0pt}%
\pgfpathmoveto{\pgfqpoint{1.373893in}{0.965858in}}%
\pgfpathlineto{\pgfqpoint{1.375808in}{0.956547in}}%
\pgfpathlineto{\pgfqpoint{1.377723in}{0.947272in}}%
\pgfpathlineto{\pgfqpoint{1.379638in}{0.938037in}}%
\pgfpathlineto{\pgfqpoint{1.381553in}{0.928845in}}%
\pgfpathlineto{\pgfqpoint{1.370403in}{0.925637in}}%
\pgfpathlineto{\pgfqpoint{1.359047in}{0.922612in}}%
\pgfpathlineto{\pgfqpoint{1.347499in}{0.919773in}}%
\pgfpathlineto{\pgfqpoint{1.335771in}{0.917124in}}%
\pgfpathlineto{\pgfqpoint{1.334288in}{0.926442in}}%
\pgfpathlineto{\pgfqpoint{1.332806in}{0.935804in}}%
\pgfpathlineto{\pgfqpoint{1.331323in}{0.945204in}}%
\pgfpathlineto{\pgfqpoint{1.329841in}{0.954642in}}%
\pgfpathlineto{\pgfqpoint{1.341125in}{0.957177in}}%
\pgfpathlineto{\pgfqpoint{1.352236in}{0.959893in}}%
\pgfpathlineto{\pgfqpoint{1.363163in}{0.962788in}}%
\pgfpathlineto{\pgfqpoint{1.373893in}{0.965858in}}%
\pgfpathclose%
\pgfusepath{fill}%
\end{pgfscope}%
\begin{pgfscope}%
\pgfpathrectangle{\pgfqpoint{0.041670in}{0.041670in}}{\pgfqpoint{2.216660in}{2.216660in}}%
\pgfusepath{clip}%
\pgfsetbuttcap%
\pgfsetroundjoin%
\definecolor{currentfill}{rgb}{0.274128,0.199721,0.498911}%
\pgfsetfillcolor{currentfill}%
\pgfsetlinewidth{0.000000pt}%
\definecolor{currentstroke}{rgb}{0.000000,0.000000,0.000000}%
\pgfsetstrokecolor{currentstroke}%
\pgfsetdash{}{0pt}%
\pgfpathmoveto{\pgfqpoint{1.359523in}{0.776587in}}%
\pgfpathlineto{\pgfqpoint{1.361012in}{0.768514in}}%
\pgfpathlineto{\pgfqpoint{1.362502in}{0.760548in}}%
\pgfpathlineto{\pgfqpoint{1.363992in}{0.752690in}}%
\pgfpathlineto{\pgfqpoint{1.365484in}{0.744945in}}%
\pgfpathlineto{\pgfqpoint{1.351330in}{0.741949in}}%
\pgfpathlineto{\pgfqpoint{1.336989in}{0.739192in}}%
\pgfpathlineto{\pgfqpoint{1.322475in}{0.736676in}}%
\pgfpathlineto{\pgfqpoint{1.307806in}{0.734405in}}%
\pgfpathlineto{\pgfqpoint{1.306776in}{0.742246in}}%
\pgfpathlineto{\pgfqpoint{1.305748in}{0.750200in}}%
\pgfpathlineto{\pgfqpoint{1.304720in}{0.758262in}}%
\pgfpathlineto{\pgfqpoint{1.303693in}{0.766430in}}%
\pgfpathlineto{\pgfqpoint{1.317892in}{0.768619in}}%
\pgfpathlineto{\pgfqpoint{1.331940in}{0.771043in}}%
\pgfpathlineto{\pgfqpoint{1.345822in}{0.773700in}}%
\pgfpathlineto{\pgfqpoint{1.359523in}{0.776587in}}%
\pgfpathclose%
\pgfusepath{fill}%
\end{pgfscope}%
\begin{pgfscope}%
\pgfpathrectangle{\pgfqpoint{0.041670in}{0.041670in}}{\pgfqpoint{2.216660in}{2.216660in}}%
\pgfusepath{clip}%
\pgfsetbuttcap%
\pgfsetroundjoin%
\definecolor{currentfill}{rgb}{0.636902,0.856542,0.216620}%
\pgfsetfillcolor{currentfill}%
\pgfsetlinewidth{0.000000pt}%
\definecolor{currentstroke}{rgb}{0.000000,0.000000,0.000000}%
\pgfsetstrokecolor{currentstroke}%
\pgfsetdash{}{0pt}%
\pgfpathmoveto{\pgfqpoint{1.322725in}{1.520701in}}%
\pgfpathlineto{\pgfqpoint{1.326276in}{1.514930in}}%
\pgfpathlineto{\pgfqpoint{1.329826in}{1.509056in}}%
\pgfpathlineto{\pgfqpoint{1.333374in}{1.503080in}}%
\pgfpathlineto{\pgfqpoint{1.336920in}{1.497003in}}%
\pgfpathlineto{\pgfqpoint{1.334684in}{1.494655in}}%
\pgfpathlineto{\pgfqpoint{1.332290in}{1.492341in}}%
\pgfpathlineto{\pgfqpoint{1.329742in}{1.490063in}}%
\pgfpathlineto{\pgfqpoint{1.327043in}{1.487824in}}%
\pgfpathlineto{\pgfqpoint{1.323715in}{1.494117in}}%
\pgfpathlineto{\pgfqpoint{1.320386in}{1.500308in}}%
\pgfpathlineto{\pgfqpoint{1.317055in}{1.506398in}}%
\pgfpathlineto{\pgfqpoint{1.313724in}{1.512383in}}%
\pgfpathlineto{\pgfqpoint{1.316183in}{1.514412in}}%
\pgfpathlineto{\pgfqpoint{1.318505in}{1.516475in}}%
\pgfpathlineto{\pgfqpoint{1.320686in}{1.518573in}}%
\pgfpathlineto{\pgfqpoint{1.322725in}{1.520701in}}%
\pgfpathclose%
\pgfusepath{fill}%
\end{pgfscope}%
\begin{pgfscope}%
\pgfpathrectangle{\pgfqpoint{0.041670in}{0.041670in}}{\pgfqpoint{2.216660in}{2.216660in}}%
\pgfusepath{clip}%
\pgfsetbuttcap%
\pgfsetroundjoin%
\definecolor{currentfill}{rgb}{0.268510,0.009605,0.335427}%
\pgfsetfillcolor{currentfill}%
\pgfsetlinewidth{0.000000pt}%
\definecolor{currentstroke}{rgb}{0.000000,0.000000,0.000000}%
\pgfsetstrokecolor{currentstroke}%
\pgfsetdash{}{0pt}%
\pgfpathmoveto{\pgfqpoint{1.466706in}{0.620584in}}%
\pgfpathlineto{\pgfqpoint{1.468691in}{0.617099in}}%
\pgfpathlineto{\pgfqpoint{1.470679in}{0.613836in}}%
\pgfpathlineto{\pgfqpoint{1.472672in}{0.610799in}}%
\pgfpathlineto{\pgfqpoint{1.474668in}{0.607994in}}%
\pgfpathlineto{\pgfqpoint{1.458425in}{0.603093in}}%
\pgfpathlineto{\pgfqpoint{1.441872in}{0.598470in}}%
\pgfpathlineto{\pgfqpoint{1.425028in}{0.594130in}}%
\pgfpathlineto{\pgfqpoint{1.407912in}{0.590078in}}%
\pgfpathlineto{\pgfqpoint{1.406364in}{0.593012in}}%
\pgfpathlineto{\pgfqpoint{1.404819in}{0.596178in}}%
\pgfpathlineto{\pgfqpoint{1.403277in}{0.599570in}}%
\pgfpathlineto{\pgfqpoint{1.401739in}{0.603184in}}%
\pgfpathlineto{\pgfqpoint{1.418395in}{0.607119in}}%
\pgfpathlineto{\pgfqpoint{1.434787in}{0.611334in}}%
\pgfpathlineto{\pgfqpoint{1.450897in}{0.615824in}}%
\pgfpathlineto{\pgfqpoint{1.466706in}{0.620584in}}%
\pgfpathclose%
\pgfusepath{fill}%
\end{pgfscope}%
\begin{pgfscope}%
\pgfpathrectangle{\pgfqpoint{0.041670in}{0.041670in}}{\pgfqpoint{2.216660in}{2.216660in}}%
\pgfusepath{clip}%
\pgfsetbuttcap%
\pgfsetroundjoin%
\definecolor{currentfill}{rgb}{0.814576,0.883393,0.110347}%
\pgfsetfillcolor{currentfill}%
\pgfsetlinewidth{0.000000pt}%
\definecolor{currentstroke}{rgb}{0.000000,0.000000,0.000000}%
\pgfsetstrokecolor{currentstroke}%
\pgfsetdash{}{0pt}%
\pgfpathmoveto{\pgfqpoint{1.284734in}{1.587467in}}%
\pgfpathlineto{\pgfqpoint{1.288466in}{1.583265in}}%
\pgfpathlineto{\pgfqpoint{1.292197in}{1.578945in}}%
\pgfpathlineto{\pgfqpoint{1.295927in}{1.574507in}}%
\pgfpathlineto{\pgfqpoint{1.299655in}{1.569953in}}%
\pgfpathlineto{\pgfqpoint{1.298484in}{1.568178in}}%
\pgfpathlineto{\pgfqpoint{1.297194in}{1.566421in}}%
\pgfpathlineto{\pgfqpoint{1.295786in}{1.564684in}}%
\pgfpathlineto{\pgfqpoint{1.294261in}{1.562968in}}%
\pgfpathlineto{\pgfqpoint{1.290698in}{1.567744in}}%
\pgfpathlineto{\pgfqpoint{1.287133in}{1.572404in}}%
\pgfpathlineto{\pgfqpoint{1.283567in}{1.576947in}}%
\pgfpathlineto{\pgfqpoint{1.280000in}{1.581370in}}%
\pgfpathlineto{\pgfqpoint{1.281337in}{1.582868in}}%
\pgfpathlineto{\pgfqpoint{1.282573in}{1.584384in}}%
\pgfpathlineto{\pgfqpoint{1.283706in}{1.585918in}}%
\pgfpathlineto{\pgfqpoint{1.284734in}{1.587467in}}%
\pgfpathclose%
\pgfusepath{fill}%
\end{pgfscope}%
\begin{pgfscope}%
\pgfpathrectangle{\pgfqpoint{0.041670in}{0.041670in}}{\pgfqpoint{2.216660in}{2.216660in}}%
\pgfusepath{clip}%
\pgfsetbuttcap%
\pgfsetroundjoin%
\definecolor{currentfill}{rgb}{0.993248,0.906157,0.143936}%
\pgfsetfillcolor{currentfill}%
\pgfsetlinewidth{0.000000pt}%
\definecolor{currentstroke}{rgb}{0.000000,0.000000,0.000000}%
\pgfsetstrokecolor{currentstroke}%
\pgfsetdash{}{0pt}%
\pgfpathmoveto{\pgfqpoint{1.179955in}{1.653775in}}%
\pgfpathlineto{\pgfqpoint{1.183698in}{1.653181in}}%
\pgfpathlineto{\pgfqpoint{1.187441in}{1.652453in}}%
\pgfpathlineto{\pgfqpoint{1.191185in}{1.651591in}}%
\pgfpathlineto{\pgfqpoint{1.194930in}{1.650596in}}%
\pgfpathlineto{\pgfqpoint{1.194780in}{1.650377in}}%
\pgfpathlineto{\pgfqpoint{1.194615in}{1.650160in}}%
\pgfpathlineto{\pgfqpoint{1.194435in}{1.649945in}}%
\pgfpathlineto{\pgfqpoint{1.194241in}{1.649734in}}%
\pgfpathlineto{\pgfqpoint{1.190669in}{1.650945in}}%
\pgfpathlineto{\pgfqpoint{1.187097in}{1.652022in}}%
\pgfpathlineto{\pgfqpoint{1.183525in}{1.652966in}}%
\pgfpathlineto{\pgfqpoint{1.179955in}{1.653775in}}%
\pgfpathlineto{\pgfqpoint{1.179955in}{1.653775in}}%
\pgfpathlineto{\pgfqpoint{1.179955in}{1.653775in}}%
\pgfpathlineto{\pgfqpoint{1.179955in}{1.653775in}}%
\pgfpathlineto{\pgfqpoint{1.179955in}{1.653775in}}%
\pgfpathclose%
\pgfusepath{fill}%
\end{pgfscope}%
\begin{pgfscope}%
\pgfpathrectangle{\pgfqpoint{0.041670in}{0.041670in}}{\pgfqpoint{2.216660in}{2.216660in}}%
\pgfusepath{clip}%
\pgfsetbuttcap%
\pgfsetroundjoin%
\definecolor{currentfill}{rgb}{0.993248,0.906157,0.143936}%
\pgfsetfillcolor{currentfill}%
\pgfsetlinewidth{0.000000pt}%
\definecolor{currentstroke}{rgb}{0.000000,0.000000,0.000000}%
\pgfsetstrokecolor{currentstroke}%
\pgfsetdash{}{0pt}%
\pgfpathmoveto{\pgfqpoint{1.179955in}{1.653775in}}%
\pgfpathlineto{\pgfqpoint{1.176430in}{1.652919in}}%
\pgfpathlineto{\pgfqpoint{1.172905in}{1.651929in}}%
\pgfpathlineto{\pgfqpoint{1.169379in}{1.650806in}}%
\pgfpathlineto{\pgfqpoint{1.165853in}{1.649548in}}%
\pgfpathlineto{\pgfqpoint{1.165646in}{1.649757in}}%
\pgfpathlineto{\pgfqpoint{1.165454in}{1.649969in}}%
\pgfpathlineto{\pgfqpoint{1.165276in}{1.650184in}}%
\pgfpathlineto{\pgfqpoint{1.165113in}{1.650401in}}%
\pgfpathlineto{\pgfqpoint{1.168824in}{1.651445in}}%
\pgfpathlineto{\pgfqpoint{1.172535in}{1.652356in}}%
\pgfpathlineto{\pgfqpoint{1.176245in}{1.653132in}}%
\pgfpathlineto{\pgfqpoint{1.179955in}{1.653775in}}%
\pgfpathlineto{\pgfqpoint{1.179955in}{1.653775in}}%
\pgfpathlineto{\pgfqpoint{1.179955in}{1.653775in}}%
\pgfpathlineto{\pgfqpoint{1.179955in}{1.653775in}}%
\pgfpathlineto{\pgfqpoint{1.179955in}{1.653775in}}%
\pgfpathclose%
\pgfusepath{fill}%
\end{pgfscope}%
\begin{pgfscope}%
\pgfpathrectangle{\pgfqpoint{0.041670in}{0.041670in}}{\pgfqpoint{2.216660in}{2.216660in}}%
\pgfusepath{clip}%
\pgfsetbuttcap%
\pgfsetroundjoin%
\definecolor{currentfill}{rgb}{0.993248,0.906157,0.143936}%
\pgfsetfillcolor{currentfill}%
\pgfsetlinewidth{0.000000pt}%
\definecolor{currentstroke}{rgb}{0.000000,0.000000,0.000000}%
\pgfsetstrokecolor{currentstroke}%
\pgfsetdash{}{0pt}%
\pgfpathmoveto{\pgfqpoint{1.179955in}{1.653775in}}%
\pgfpathlineto{\pgfqpoint{1.183525in}{1.652966in}}%
\pgfpathlineto{\pgfqpoint{1.187097in}{1.652022in}}%
\pgfpathlineto{\pgfqpoint{1.190669in}{1.650945in}}%
\pgfpathlineto{\pgfqpoint{1.194241in}{1.649734in}}%
\pgfpathlineto{\pgfqpoint{1.194033in}{1.649525in}}%
\pgfpathlineto{\pgfqpoint{1.193811in}{1.649319in}}%
\pgfpathlineto{\pgfqpoint{1.193575in}{1.649117in}}%
\pgfpathlineto{\pgfqpoint{1.193325in}{1.648918in}}%
\pgfpathlineto{\pgfqpoint{1.189982in}{1.650334in}}%
\pgfpathlineto{\pgfqpoint{1.186638in}{1.651615in}}%
\pgfpathlineto{\pgfqpoint{1.183296in}{1.652762in}}%
\pgfpathlineto{\pgfqpoint{1.179955in}{1.653775in}}%
\pgfpathlineto{\pgfqpoint{1.179955in}{1.653775in}}%
\pgfpathlineto{\pgfqpoint{1.179955in}{1.653775in}}%
\pgfpathlineto{\pgfqpoint{1.179955in}{1.653775in}}%
\pgfpathlineto{\pgfqpoint{1.179955in}{1.653775in}}%
\pgfpathclose%
\pgfusepath{fill}%
\end{pgfscope}%
\begin{pgfscope}%
\pgfpathrectangle{\pgfqpoint{0.041670in}{0.041670in}}{\pgfqpoint{2.216660in}{2.216660in}}%
\pgfusepath{clip}%
\pgfsetbuttcap%
\pgfsetroundjoin%
\definecolor{currentfill}{rgb}{0.993248,0.906157,0.143936}%
\pgfsetfillcolor{currentfill}%
\pgfsetlinewidth{0.000000pt}%
\definecolor{currentstroke}{rgb}{0.000000,0.000000,0.000000}%
\pgfsetstrokecolor{currentstroke}%
\pgfsetdash{}{0pt}%
\pgfpathmoveto{\pgfqpoint{1.179955in}{1.653775in}}%
\pgfpathlineto{\pgfqpoint{1.176672in}{1.652719in}}%
\pgfpathlineto{\pgfqpoint{1.173388in}{1.651528in}}%
\pgfpathlineto{\pgfqpoint{1.170103in}{1.650204in}}%
\pgfpathlineto{\pgfqpoint{1.166817in}{1.648745in}}%
\pgfpathlineto{\pgfqpoint{1.166556in}{1.648940in}}%
\pgfpathlineto{\pgfqpoint{1.166308in}{1.649139in}}%
\pgfpathlineto{\pgfqpoint{1.166073in}{1.649342in}}%
\pgfpathlineto{\pgfqpoint{1.165853in}{1.649548in}}%
\pgfpathlineto{\pgfqpoint{1.169379in}{1.650806in}}%
\pgfpathlineto{\pgfqpoint{1.172905in}{1.651929in}}%
\pgfpathlineto{\pgfqpoint{1.176430in}{1.652919in}}%
\pgfpathlineto{\pgfqpoint{1.179955in}{1.653775in}}%
\pgfpathlineto{\pgfqpoint{1.179955in}{1.653775in}}%
\pgfpathlineto{\pgfqpoint{1.179955in}{1.653775in}}%
\pgfpathlineto{\pgfqpoint{1.179955in}{1.653775in}}%
\pgfpathlineto{\pgfqpoint{1.179955in}{1.653775in}}%
\pgfpathclose%
\pgfusepath{fill}%
\end{pgfscope}%
\begin{pgfscope}%
\pgfpathrectangle{\pgfqpoint{0.041670in}{0.041670in}}{\pgfqpoint{2.216660in}{2.216660in}}%
\pgfusepath{clip}%
\pgfsetbuttcap%
\pgfsetroundjoin%
\definecolor{currentfill}{rgb}{0.282327,0.094955,0.417331}%
\pgfsetfillcolor{currentfill}%
\pgfsetlinewidth{0.000000pt}%
\definecolor{currentstroke}{rgb}{0.000000,0.000000,0.000000}%
\pgfsetstrokecolor{currentstroke}%
\pgfsetdash{}{0pt}%
\pgfpathmoveto{\pgfqpoint{1.377456in}{0.687502in}}%
\pgfpathlineto{\pgfqpoint{1.378959in}{0.680945in}}%
\pgfpathlineto{\pgfqpoint{1.380463in}{0.674540in}}%
\pgfpathlineto{\pgfqpoint{1.381969in}{0.668292in}}%
\pgfpathlineto{\pgfqpoint{1.383477in}{0.662203in}}%
\pgfpathlineto{\pgfqpoint{1.367958in}{0.658880in}}%
\pgfpathlineto{\pgfqpoint{1.352232in}{0.655821in}}%
\pgfpathlineto{\pgfqpoint{1.336315in}{0.653030in}}%
\pgfpathlineto{\pgfqpoint{1.320225in}{0.650509in}}%
\pgfpathlineto{\pgfqpoint{1.319184in}{0.656694in}}%
\pgfpathlineto{\pgfqpoint{1.318145in}{0.663039in}}%
\pgfpathlineto{\pgfqpoint{1.317106in}{0.669540in}}%
\pgfpathlineto{\pgfqpoint{1.316069in}{0.676193in}}%
\pgfpathlineto{\pgfqpoint{1.331684in}{0.678631in}}%
\pgfpathlineto{\pgfqpoint{1.347131in}{0.681330in}}%
\pgfpathlineto{\pgfqpoint{1.362394in}{0.684288in}}%
\pgfpathlineto{\pgfqpoint{1.377456in}{0.687502in}}%
\pgfpathclose%
\pgfusepath{fill}%
\end{pgfscope}%
\begin{pgfscope}%
\pgfpathrectangle{\pgfqpoint{0.041670in}{0.041670in}}{\pgfqpoint{2.216660in}{2.216660in}}%
\pgfusepath{clip}%
\pgfsetbuttcap%
\pgfsetroundjoin%
\definecolor{currentfill}{rgb}{0.993248,0.906157,0.143936}%
\pgfsetfillcolor{currentfill}%
\pgfsetlinewidth{0.000000pt}%
\definecolor{currentstroke}{rgb}{0.000000,0.000000,0.000000}%
\pgfsetstrokecolor{currentstroke}%
\pgfsetdash{}{0pt}%
\pgfpathmoveto{\pgfqpoint{1.195379in}{1.651491in}}%
\pgfpathlineto{\pgfqpoint{1.199235in}{1.650587in}}%
\pgfpathlineto{\pgfqpoint{1.203092in}{1.649548in}}%
\pgfpathlineto{\pgfqpoint{1.206948in}{1.648377in}}%
\pgfpathlineto{\pgfqpoint{1.210804in}{1.647073in}}%
\pgfpathlineto{\pgfqpoint{1.210626in}{1.646620in}}%
\pgfpathlineto{\pgfqpoint{1.210417in}{1.646170in}}%
\pgfpathlineto{\pgfqpoint{1.210179in}{1.645723in}}%
\pgfpathlineto{\pgfqpoint{1.209909in}{1.645280in}}%
\pgfpathlineto{\pgfqpoint{1.206164in}{1.646809in}}%
\pgfpathlineto{\pgfqpoint{1.202419in}{1.648204in}}%
\pgfpathlineto{\pgfqpoint{1.198674in}{1.649467in}}%
\pgfpathlineto{\pgfqpoint{1.194930in}{1.650596in}}%
\pgfpathlineto{\pgfqpoint{1.195065in}{1.650817in}}%
\pgfpathlineto{\pgfqpoint{1.195185in}{1.651040in}}%
\pgfpathlineto{\pgfqpoint{1.195289in}{1.651265in}}%
\pgfpathlineto{\pgfqpoint{1.195379in}{1.651491in}}%
\pgfpathclose%
\pgfusepath{fill}%
\end{pgfscope}%
\begin{pgfscope}%
\pgfpathrectangle{\pgfqpoint{0.041670in}{0.041670in}}{\pgfqpoint{2.216660in}{2.216660in}}%
\pgfusepath{clip}%
\pgfsetbuttcap%
\pgfsetroundjoin%
\definecolor{currentfill}{rgb}{0.935904,0.898570,0.108131}%
\pgfsetfillcolor{currentfill}%
\pgfsetlinewidth{0.000000pt}%
\definecolor{currentstroke}{rgb}{0.000000,0.000000,0.000000}%
\pgfsetstrokecolor{currentstroke}%
\pgfsetdash{}{0pt}%
\pgfpathmoveto{\pgfqpoint{1.120568in}{1.627521in}}%
\pgfpathlineto{\pgfqpoint{1.116857in}{1.624763in}}%
\pgfpathlineto{\pgfqpoint{1.113146in}{1.621876in}}%
\pgfpathlineto{\pgfqpoint{1.109437in}{1.618861in}}%
\pgfpathlineto{\pgfqpoint{1.105727in}{1.615719in}}%
\pgfpathlineto{\pgfqpoint{1.104996in}{1.616823in}}%
\pgfpathlineto{\pgfqpoint{1.104339in}{1.617936in}}%
\pgfpathlineto{\pgfqpoint{1.103758in}{1.619059in}}%
\pgfpathlineto{\pgfqpoint{1.103254in}{1.620189in}}%
\pgfpathlineto{\pgfqpoint{1.107084in}{1.623106in}}%
\pgfpathlineto{\pgfqpoint{1.110915in}{1.625896in}}%
\pgfpathlineto{\pgfqpoint{1.114747in}{1.628558in}}%
\pgfpathlineto{\pgfqpoint{1.118580in}{1.631091in}}%
\pgfpathlineto{\pgfqpoint{1.118986in}{1.630189in}}%
\pgfpathlineto{\pgfqpoint{1.119453in}{1.629292in}}%
\pgfpathlineto{\pgfqpoint{1.119980in}{1.628403in}}%
\pgfpathlineto{\pgfqpoint{1.120568in}{1.627521in}}%
\pgfpathclose%
\pgfusepath{fill}%
\end{pgfscope}%
\begin{pgfscope}%
\pgfpathrectangle{\pgfqpoint{0.041670in}{0.041670in}}{\pgfqpoint{2.216660in}{2.216660in}}%
\pgfusepath{clip}%
\pgfsetbuttcap%
\pgfsetroundjoin%
\definecolor{currentfill}{rgb}{0.955300,0.901065,0.118128}%
\pgfsetfillcolor{currentfill}%
\pgfsetlinewidth{0.000000pt}%
\definecolor{currentstroke}{rgb}{0.000000,0.000000,0.000000}%
\pgfsetstrokecolor{currentstroke}%
\pgfsetdash{}{0pt}%
\pgfpathmoveto{\pgfqpoint{1.226226in}{1.640534in}}%
\pgfpathlineto{\pgfqpoint{1.230080in}{1.638571in}}%
\pgfpathlineto{\pgfqpoint{1.233934in}{1.636477in}}%
\pgfpathlineto{\pgfqpoint{1.237787in}{1.634252in}}%
\pgfpathlineto{\pgfqpoint{1.241639in}{1.631899in}}%
\pgfpathlineto{\pgfqpoint{1.241288in}{1.630991in}}%
\pgfpathlineto{\pgfqpoint{1.240875in}{1.630089in}}%
\pgfpathlineto{\pgfqpoint{1.240401in}{1.629193in}}%
\pgfpathlineto{\pgfqpoint{1.239867in}{1.628304in}}%
\pgfpathlineto{\pgfqpoint{1.236124in}{1.630884in}}%
\pgfpathlineto{\pgfqpoint{1.232379in}{1.633334in}}%
\pgfpathlineto{\pgfqpoint{1.228635in}{1.635653in}}%
\pgfpathlineto{\pgfqpoint{1.224890in}{1.637842in}}%
\pgfpathlineto{\pgfqpoint{1.225292in}{1.638508in}}%
\pgfpathlineto{\pgfqpoint{1.225649in}{1.639179in}}%
\pgfpathlineto{\pgfqpoint{1.225960in}{1.639854in}}%
\pgfpathlineto{\pgfqpoint{1.226226in}{1.640534in}}%
\pgfpathclose%
\pgfusepath{fill}%
\end{pgfscope}%
\begin{pgfscope}%
\pgfpathrectangle{\pgfqpoint{0.041670in}{0.041670in}}{\pgfqpoint{2.216660in}{2.216660in}}%
\pgfusepath{clip}%
\pgfsetbuttcap%
\pgfsetroundjoin%
\definecolor{currentfill}{rgb}{0.993248,0.906157,0.143936}%
\pgfsetfillcolor{currentfill}%
\pgfsetlinewidth{0.000000pt}%
\definecolor{currentstroke}{rgb}{0.000000,0.000000,0.000000}%
\pgfsetstrokecolor{currentstroke}%
\pgfsetdash{}{0pt}%
\pgfpathmoveto{\pgfqpoint{1.179955in}{1.653775in}}%
\pgfpathlineto{\pgfqpoint{1.183296in}{1.652762in}}%
\pgfpathlineto{\pgfqpoint{1.186638in}{1.651615in}}%
\pgfpathlineto{\pgfqpoint{1.189982in}{1.650334in}}%
\pgfpathlineto{\pgfqpoint{1.193325in}{1.648918in}}%
\pgfpathlineto{\pgfqpoint{1.193063in}{1.648724in}}%
\pgfpathlineto{\pgfqpoint{1.192786in}{1.648533in}}%
\pgfpathlineto{\pgfqpoint{1.192498in}{1.648346in}}%
\pgfpathlineto{\pgfqpoint{1.192196in}{1.648164in}}%
\pgfpathlineto{\pgfqpoint{1.189134in}{1.649768in}}%
\pgfpathlineto{\pgfqpoint{1.186074in}{1.651238in}}%
\pgfpathlineto{\pgfqpoint{1.183014in}{1.652574in}}%
\pgfpathlineto{\pgfqpoint{1.179955in}{1.653775in}}%
\pgfpathlineto{\pgfqpoint{1.179955in}{1.653775in}}%
\pgfpathlineto{\pgfqpoint{1.179955in}{1.653775in}}%
\pgfpathlineto{\pgfqpoint{1.179955in}{1.653775in}}%
\pgfpathlineto{\pgfqpoint{1.179955in}{1.653775in}}%
\pgfpathclose%
\pgfusepath{fill}%
\end{pgfscope}%
\begin{pgfscope}%
\pgfpathrectangle{\pgfqpoint{0.041670in}{0.041670in}}{\pgfqpoint{2.216660in}{2.216660in}}%
\pgfusepath{clip}%
\pgfsetbuttcap%
\pgfsetroundjoin%
\definecolor{currentfill}{rgb}{0.263663,0.237631,0.518762}%
\pgfsetfillcolor{currentfill}%
\pgfsetlinewidth{0.000000pt}%
\definecolor{currentstroke}{rgb}{0.000000,0.000000,0.000000}%
\pgfsetstrokecolor{currentstroke}%
\pgfsetdash{}{0pt}%
\pgfpathmoveto{\pgfqpoint{1.353575in}{0.809854in}}%
\pgfpathlineto{\pgfqpoint{1.355061in}{0.801398in}}%
\pgfpathlineto{\pgfqpoint{1.356548in}{0.793032in}}%
\pgfpathlineto{\pgfqpoint{1.358035in}{0.784760in}}%
\pgfpathlineto{\pgfqpoint{1.359523in}{0.776587in}}%
\pgfpathlineto{\pgfqpoint{1.345822in}{0.773700in}}%
\pgfpathlineto{\pgfqpoint{1.331940in}{0.771043in}}%
\pgfpathlineto{\pgfqpoint{1.317892in}{0.768619in}}%
\pgfpathlineto{\pgfqpoint{1.303693in}{0.766430in}}%
\pgfpathlineto{\pgfqpoint{1.302666in}{0.774699in}}%
\pgfpathlineto{\pgfqpoint{1.301639in}{0.783066in}}%
\pgfpathlineto{\pgfqpoint{1.300614in}{0.791527in}}%
\pgfpathlineto{\pgfqpoint{1.299588in}{0.800079in}}%
\pgfpathlineto{\pgfqpoint{1.313318in}{0.802186in}}%
\pgfpathlineto{\pgfqpoint{1.326902in}{0.804519in}}%
\pgfpathlineto{\pgfqpoint{1.340325in}{0.807076in}}%
\pgfpathlineto{\pgfqpoint{1.353575in}{0.809854in}}%
\pgfpathclose%
\pgfusepath{fill}%
\end{pgfscope}%
\begin{pgfscope}%
\pgfpathrectangle{\pgfqpoint{0.041670in}{0.041670in}}{\pgfqpoint{2.216660in}{2.216660in}}%
\pgfusepath{clip}%
\pgfsetbuttcap%
\pgfsetroundjoin%
\definecolor{currentfill}{rgb}{0.993248,0.906157,0.143936}%
\pgfsetfillcolor{currentfill}%
\pgfsetlinewidth{0.000000pt}%
\definecolor{currentstroke}{rgb}{0.000000,0.000000,0.000000}%
\pgfsetstrokecolor{currentstroke}%
\pgfsetdash{}{0pt}%
\pgfpathmoveto{\pgfqpoint{1.165113in}{1.650401in}}%
\pgfpathlineto{\pgfqpoint{1.161401in}{1.649223in}}%
\pgfpathlineto{\pgfqpoint{1.157689in}{1.647912in}}%
\pgfpathlineto{\pgfqpoint{1.153977in}{1.646467in}}%
\pgfpathlineto{\pgfqpoint{1.150265in}{1.644890in}}%
\pgfpathlineto{\pgfqpoint{1.149969in}{1.645329in}}%
\pgfpathlineto{\pgfqpoint{1.149703in}{1.645773in}}%
\pgfpathlineto{\pgfqpoint{1.149468in}{1.646220in}}%
\pgfpathlineto{\pgfqpoint{1.149262in}{1.646670in}}%
\pgfpathlineto{\pgfqpoint{1.153099in}{1.648025in}}%
\pgfpathlineto{\pgfqpoint{1.156936in}{1.649247in}}%
\pgfpathlineto{\pgfqpoint{1.160773in}{1.650335in}}%
\pgfpathlineto{\pgfqpoint{1.164610in}{1.651290in}}%
\pgfpathlineto{\pgfqpoint{1.164713in}{1.651065in}}%
\pgfpathlineto{\pgfqpoint{1.164831in}{1.650842in}}%
\pgfpathlineto{\pgfqpoint{1.164964in}{1.650620in}}%
\pgfpathlineto{\pgfqpoint{1.165113in}{1.650401in}}%
\pgfpathclose%
\pgfusepath{fill}%
\end{pgfscope}%
\begin{pgfscope}%
\pgfpathrectangle{\pgfqpoint{0.041670in}{0.041670in}}{\pgfqpoint{2.216660in}{2.216660in}}%
\pgfusepath{clip}%
\pgfsetbuttcap%
\pgfsetroundjoin%
\definecolor{currentfill}{rgb}{0.260571,0.246922,0.522828}%
\pgfsetfillcolor{currentfill}%
\pgfsetlinewidth{0.000000pt}%
\definecolor{currentstroke}{rgb}{0.000000,0.000000,0.000000}%
\pgfsetstrokecolor{currentstroke}%
\pgfsetdash{}{0pt}%
\pgfpathmoveto{\pgfqpoint{0.629149in}{0.758251in}}%
\pgfpathlineto{\pgfqpoint{0.625791in}{0.766578in}}%
\pgfpathlineto{\pgfqpoint{0.622417in}{0.775334in}}%
\pgfpathlineto{\pgfqpoint{0.619027in}{0.784525in}}%
\pgfpathlineto{\pgfqpoint{0.615620in}{0.794160in}}%
\pgfpathlineto{\pgfqpoint{0.601627in}{0.803677in}}%
\pgfpathlineto{\pgfqpoint{0.588266in}{0.813409in}}%
\pgfpathlineto{\pgfqpoint{0.575548in}{0.823345in}}%
\pgfpathlineto{\pgfqpoint{0.563485in}{0.833472in}}%
\pgfpathlineto{\pgfqpoint{0.567187in}{0.823653in}}%
\pgfpathlineto{\pgfqpoint{0.570871in}{0.814276in}}%
\pgfpathlineto{\pgfqpoint{0.574537in}{0.805332in}}%
\pgfpathlineto{\pgfqpoint{0.578186in}{0.796815in}}%
\pgfpathlineto{\pgfqpoint{0.589981in}{0.786879in}}%
\pgfpathlineto{\pgfqpoint{0.602414in}{0.777133in}}%
\pgfpathlineto{\pgfqpoint{0.615474in}{0.767586in}}%
\pgfpathlineto{\pgfqpoint{0.629149in}{0.758251in}}%
\pgfpathclose%
\pgfusepath{fill}%
\end{pgfscope}%
\begin{pgfscope}%
\pgfpathrectangle{\pgfqpoint{0.041670in}{0.041670in}}{\pgfqpoint{2.216660in}{2.216660in}}%
\pgfusepath{clip}%
\pgfsetbuttcap%
\pgfsetroundjoin%
\definecolor{currentfill}{rgb}{0.993248,0.906157,0.143936}%
\pgfsetfillcolor{currentfill}%
\pgfsetlinewidth{0.000000pt}%
\definecolor{currentstroke}{rgb}{0.000000,0.000000,0.000000}%
\pgfsetstrokecolor{currentstroke}%
\pgfsetdash{}{0pt}%
\pgfpathmoveto{\pgfqpoint{1.179955in}{1.653775in}}%
\pgfpathlineto{\pgfqpoint{1.176966in}{1.652534in}}%
\pgfpathlineto{\pgfqpoint{1.173975in}{1.651159in}}%
\pgfpathlineto{\pgfqpoint{1.170984in}{1.649649in}}%
\pgfpathlineto{\pgfqpoint{1.167992in}{1.648005in}}%
\pgfpathlineto{\pgfqpoint{1.167679in}{1.648184in}}%
\pgfpathlineto{\pgfqpoint{1.167379in}{1.648367in}}%
\pgfpathlineto{\pgfqpoint{1.167092in}{1.648554in}}%
\pgfpathlineto{\pgfqpoint{1.166817in}{1.648745in}}%
\pgfpathlineto{\pgfqpoint{1.170103in}{1.650204in}}%
\pgfpathlineto{\pgfqpoint{1.173388in}{1.651528in}}%
\pgfpathlineto{\pgfqpoint{1.176672in}{1.652719in}}%
\pgfpathlineto{\pgfqpoint{1.179955in}{1.653775in}}%
\pgfpathlineto{\pgfqpoint{1.179955in}{1.653775in}}%
\pgfpathlineto{\pgfqpoint{1.179955in}{1.653775in}}%
\pgfpathlineto{\pgfqpoint{1.179955in}{1.653775in}}%
\pgfpathlineto{\pgfqpoint{1.179955in}{1.653775in}}%
\pgfpathclose%
\pgfusepath{fill}%
\end{pgfscope}%
\begin{pgfscope}%
\pgfpathrectangle{\pgfqpoint{0.041670in}{0.041670in}}{\pgfqpoint{2.216660in}{2.216660in}}%
\pgfusepath{clip}%
\pgfsetbuttcap%
\pgfsetroundjoin%
\definecolor{currentfill}{rgb}{0.974417,0.903590,0.130215}%
\pgfsetfillcolor{currentfill}%
\pgfsetlinewidth{0.000000pt}%
\definecolor{currentstroke}{rgb}{0.000000,0.000000,0.000000}%
\pgfsetstrokecolor{currentstroke}%
\pgfsetdash{}{0pt}%
\pgfpathmoveto{\pgfqpoint{1.210804in}{1.647073in}}%
\pgfpathlineto{\pgfqpoint{1.214660in}{1.645636in}}%
\pgfpathlineto{\pgfqpoint{1.218515in}{1.644068in}}%
\pgfpathlineto{\pgfqpoint{1.222371in}{1.642367in}}%
\pgfpathlineto{\pgfqpoint{1.226226in}{1.640534in}}%
\pgfpathlineto{\pgfqpoint{1.225960in}{1.639854in}}%
\pgfpathlineto{\pgfqpoint{1.225649in}{1.639179in}}%
\pgfpathlineto{\pgfqpoint{1.225292in}{1.638508in}}%
\pgfpathlineto{\pgfqpoint{1.224890in}{1.637842in}}%
\pgfpathlineto{\pgfqpoint{1.221145in}{1.639900in}}%
\pgfpathlineto{\pgfqpoint{1.217400in}{1.641825in}}%
\pgfpathlineto{\pgfqpoint{1.213655in}{1.643619in}}%
\pgfpathlineto{\pgfqpoint{1.209909in}{1.645280in}}%
\pgfpathlineto{\pgfqpoint{1.210179in}{1.645723in}}%
\pgfpathlineto{\pgfqpoint{1.210417in}{1.646170in}}%
\pgfpathlineto{\pgfqpoint{1.210626in}{1.646620in}}%
\pgfpathlineto{\pgfqpoint{1.210804in}{1.647073in}}%
\pgfpathclose%
\pgfusepath{fill}%
\end{pgfscope}%
\begin{pgfscope}%
\pgfpathrectangle{\pgfqpoint{0.041670in}{0.041670in}}{\pgfqpoint{2.216660in}{2.216660in}}%
\pgfusepath{clip}%
\pgfsetbuttcap%
\pgfsetroundjoin%
\definecolor{currentfill}{rgb}{0.993248,0.906157,0.143936}%
\pgfsetfillcolor{currentfill}%
\pgfsetlinewidth{0.000000pt}%
\definecolor{currentstroke}{rgb}{0.000000,0.000000,0.000000}%
\pgfsetstrokecolor{currentstroke}%
\pgfsetdash{}{0pt}%
\pgfpathmoveto{\pgfqpoint{1.179955in}{1.653775in}}%
\pgfpathlineto{\pgfqpoint{1.183014in}{1.652574in}}%
\pgfpathlineto{\pgfqpoint{1.186074in}{1.651238in}}%
\pgfpathlineto{\pgfqpoint{1.189134in}{1.649768in}}%
\pgfpathlineto{\pgfqpoint{1.192196in}{1.648164in}}%
\pgfpathlineto{\pgfqpoint{1.191883in}{1.647986in}}%
\pgfpathlineto{\pgfqpoint{1.191557in}{1.647813in}}%
\pgfpathlineto{\pgfqpoint{1.191220in}{1.647644in}}%
\pgfpathlineto{\pgfqpoint{1.190871in}{1.647481in}}%
\pgfpathlineto{\pgfqpoint{1.188141in}{1.649256in}}%
\pgfpathlineto{\pgfqpoint{1.185411in}{1.650897in}}%
\pgfpathlineto{\pgfqpoint{1.182682in}{1.652403in}}%
\pgfpathlineto{\pgfqpoint{1.179955in}{1.653775in}}%
\pgfpathlineto{\pgfqpoint{1.179955in}{1.653775in}}%
\pgfpathlineto{\pgfqpoint{1.179955in}{1.653775in}}%
\pgfpathlineto{\pgfqpoint{1.179955in}{1.653775in}}%
\pgfpathlineto{\pgfqpoint{1.179955in}{1.653775in}}%
\pgfpathclose%
\pgfusepath{fill}%
\end{pgfscope}%
\begin{pgfscope}%
\pgfpathrectangle{\pgfqpoint{0.041670in}{0.041670in}}{\pgfqpoint{2.216660in}{2.216660in}}%
\pgfusepath{clip}%
\pgfsetbuttcap%
\pgfsetroundjoin%
\definecolor{currentfill}{rgb}{0.120081,0.622161,0.534946}%
\pgfsetfillcolor{currentfill}%
\pgfsetlinewidth{0.000000pt}%
\definecolor{currentstroke}{rgb}{0.000000,0.000000,0.000000}%
\pgfsetstrokecolor{currentstroke}%
\pgfsetdash{}{0pt}%
\pgfpathmoveto{\pgfqpoint{1.007647in}{1.200323in}}%
\pgfpathlineto{\pgfqpoint{1.005403in}{1.191137in}}%
\pgfpathlineto{\pgfqpoint{1.003160in}{1.181919in}}%
\pgfpathlineto{\pgfqpoint{1.000918in}{1.172671in}}%
\pgfpathlineto{\pgfqpoint{0.998677in}{1.163396in}}%
\pgfpathlineto{\pgfqpoint{0.990788in}{1.166327in}}%
\pgfpathlineto{\pgfqpoint{0.983098in}{1.169379in}}%
\pgfpathlineto{\pgfqpoint{0.975614in}{1.172551in}}%
\pgfpathlineto{\pgfqpoint{0.968344in}{1.175837in}}%
\pgfpathlineto{\pgfqpoint{0.970955in}{1.184942in}}%
\pgfpathlineto{\pgfqpoint{0.973568in}{1.194020in}}%
\pgfpathlineto{\pgfqpoint{0.976181in}{1.203069in}}%
\pgfpathlineto{\pgfqpoint{0.978796in}{1.212087in}}%
\pgfpathlineto{\pgfqpoint{0.985712in}{1.208979in}}%
\pgfpathlineto{\pgfqpoint{0.992830in}{1.205981in}}%
\pgfpathlineto{\pgfqpoint{1.000144in}{1.203094in}}%
\pgfpathlineto{\pgfqpoint{1.007647in}{1.200323in}}%
\pgfpathclose%
\pgfusepath{fill}%
\end{pgfscope}%
\begin{pgfscope}%
\pgfpathrectangle{\pgfqpoint{0.041670in}{0.041670in}}{\pgfqpoint{2.216660in}{2.216660in}}%
\pgfusepath{clip}%
\pgfsetbuttcap%
\pgfsetroundjoin%
\definecolor{currentfill}{rgb}{0.147607,0.511733,0.557049}%
\pgfsetfillcolor{currentfill}%
\pgfsetlinewidth{0.000000pt}%
\definecolor{currentstroke}{rgb}{0.000000,0.000000,0.000000}%
\pgfsetstrokecolor{currentstroke}%
\pgfsetdash{}{0pt}%
\pgfpathmoveto{\pgfqpoint{1.017429in}{1.076953in}}%
\pgfpathlineto{\pgfqpoint{1.015604in}{1.067412in}}%
\pgfpathlineto{\pgfqpoint{1.013778in}{1.057871in}}%
\pgfpathlineto{\pgfqpoint{1.011953in}{1.048333in}}%
\pgfpathlineto{\pgfqpoint{1.010129in}{1.038801in}}%
\pgfpathlineto{\pgfqpoint{1.000262in}{1.041615in}}%
\pgfpathlineto{\pgfqpoint{0.990586in}{1.044587in}}%
\pgfpathlineto{\pgfqpoint{0.981111in}{1.047710in}}%
\pgfpathlineto{\pgfqpoint{0.971847in}{1.050984in}}%
\pgfpathlineto{\pgfqpoint{0.974079in}{1.060369in}}%
\pgfpathlineto{\pgfqpoint{0.976312in}{1.069760in}}%
\pgfpathlineto{\pgfqpoint{0.978545in}{1.079155in}}%
\pgfpathlineto{\pgfqpoint{0.980779in}{1.088550in}}%
\pgfpathlineto{\pgfqpoint{0.989649in}{1.085434in}}%
\pgfpathlineto{\pgfqpoint{0.998720in}{1.082461in}}%
\pgfpathlineto{\pgfqpoint{1.007984in}{1.079633in}}%
\pgfpathlineto{\pgfqpoint{1.017429in}{1.076953in}}%
\pgfpathclose%
\pgfusepath{fill}%
\end{pgfscope}%
\begin{pgfscope}%
\pgfpathrectangle{\pgfqpoint{0.041670in}{0.041670in}}{\pgfqpoint{2.216660in}{2.216660in}}%
\pgfusepath{clip}%
\pgfsetbuttcap%
\pgfsetroundjoin%
\definecolor{currentfill}{rgb}{0.993248,0.906157,0.143936}%
\pgfsetfillcolor{currentfill}%
\pgfsetlinewidth{0.000000pt}%
\definecolor{currentstroke}{rgb}{0.000000,0.000000,0.000000}%
\pgfsetstrokecolor{currentstroke}%
\pgfsetdash{}{0pt}%
\pgfpathmoveto{\pgfqpoint{1.179955in}{1.653775in}}%
\pgfpathlineto{\pgfqpoint{1.177307in}{1.652368in}}%
\pgfpathlineto{\pgfqpoint{1.174658in}{1.650826in}}%
\pgfpathlineto{\pgfqpoint{1.172008in}{1.649150in}}%
\pgfpathlineto{\pgfqpoint{1.169357in}{1.647340in}}%
\pgfpathlineto{\pgfqpoint{1.168999in}{1.647499in}}%
\pgfpathlineto{\pgfqpoint{1.168652in}{1.647663in}}%
\pgfpathlineto{\pgfqpoint{1.168316in}{1.647832in}}%
\pgfpathlineto{\pgfqpoint{1.167992in}{1.648005in}}%
\pgfpathlineto{\pgfqpoint{1.170984in}{1.649649in}}%
\pgfpathlineto{\pgfqpoint{1.173975in}{1.651159in}}%
\pgfpathlineto{\pgfqpoint{1.176966in}{1.652534in}}%
\pgfpathlineto{\pgfqpoint{1.179955in}{1.653775in}}%
\pgfpathlineto{\pgfqpoint{1.179955in}{1.653775in}}%
\pgfpathlineto{\pgfqpoint{1.179955in}{1.653775in}}%
\pgfpathlineto{\pgfqpoint{1.179955in}{1.653775in}}%
\pgfpathlineto{\pgfqpoint{1.179955in}{1.653775in}}%
\pgfpathclose%
\pgfusepath{fill}%
\end{pgfscope}%
\begin{pgfscope}%
\pgfpathrectangle{\pgfqpoint{0.041670in}{0.041670in}}{\pgfqpoint{2.216660in}{2.216660in}}%
\pgfusepath{clip}%
\pgfsetbuttcap%
\pgfsetroundjoin%
\definecolor{currentfill}{rgb}{0.955300,0.901065,0.118128}%
\pgfsetfillcolor{currentfill}%
\pgfsetlinewidth{0.000000pt}%
\definecolor{currentstroke}{rgb}{0.000000,0.000000,0.000000}%
\pgfsetstrokecolor{currentstroke}%
\pgfsetdash{}{0pt}%
\pgfpathmoveto{\pgfqpoint{1.135415in}{1.637256in}}%
\pgfpathlineto{\pgfqpoint{1.131703in}{1.635018in}}%
\pgfpathlineto{\pgfqpoint{1.127991in}{1.632649in}}%
\pgfpathlineto{\pgfqpoint{1.124279in}{1.630150in}}%
\pgfpathlineto{\pgfqpoint{1.120568in}{1.627521in}}%
\pgfpathlineto{\pgfqpoint{1.119980in}{1.628403in}}%
\pgfpathlineto{\pgfqpoint{1.119453in}{1.629292in}}%
\pgfpathlineto{\pgfqpoint{1.118986in}{1.630189in}}%
\pgfpathlineto{\pgfqpoint{1.118580in}{1.631091in}}%
\pgfpathlineto{\pgfqpoint{1.122413in}{1.633496in}}%
\pgfpathlineto{\pgfqpoint{1.126248in}{1.635771in}}%
\pgfpathlineto{\pgfqpoint{1.130082in}{1.637916in}}%
\pgfpathlineto{\pgfqpoint{1.133918in}{1.639930in}}%
\pgfpathlineto{\pgfqpoint{1.134224in}{1.639253in}}%
\pgfpathlineto{\pgfqpoint{1.134576in}{1.638582in}}%
\pgfpathlineto{\pgfqpoint{1.134973in}{1.637916in}}%
\pgfpathlineto{\pgfqpoint{1.135415in}{1.637256in}}%
\pgfpathclose%
\pgfusepath{fill}%
\end{pgfscope}%
\begin{pgfscope}%
\pgfpathrectangle{\pgfqpoint{0.041670in}{0.041670in}}{\pgfqpoint{2.216660in}{2.216660in}}%
\pgfusepath{clip}%
\pgfsetbuttcap%
\pgfsetroundjoin%
\definecolor{currentfill}{rgb}{0.814576,0.883393,0.110347}%
\pgfsetfillcolor{currentfill}%
\pgfsetlinewidth{0.000000pt}%
\definecolor{currentstroke}{rgb}{0.000000,0.000000,0.000000}%
\pgfsetstrokecolor{currentstroke}%
\pgfsetdash{}{0pt}%
\pgfpathmoveto{\pgfqpoint{1.081183in}{1.580056in}}%
\pgfpathlineto{\pgfqpoint{1.077661in}{1.575585in}}%
\pgfpathlineto{\pgfqpoint{1.074140in}{1.570994in}}%
\pgfpathlineto{\pgfqpoint{1.070619in}{1.566286in}}%
\pgfpathlineto{\pgfqpoint{1.067100in}{1.561462in}}%
\pgfpathlineto{\pgfqpoint{1.065473in}{1.563157in}}%
\pgfpathlineto{\pgfqpoint{1.063962in}{1.564876in}}%
\pgfpathlineto{\pgfqpoint{1.062567in}{1.566615in}}%
\pgfpathlineto{\pgfqpoint{1.061290in}{1.568374in}}%
\pgfpathlineto{\pgfqpoint{1.064987in}{1.572979in}}%
\pgfpathlineto{\pgfqpoint{1.068685in}{1.577467in}}%
\pgfpathlineto{\pgfqpoint{1.072384in}{1.581837in}}%
\pgfpathlineto{\pgfqpoint{1.076085in}{1.586089in}}%
\pgfpathlineto{\pgfqpoint{1.077206in}{1.584554in}}%
\pgfpathlineto{\pgfqpoint{1.078430in}{1.583035in}}%
\pgfpathlineto{\pgfqpoint{1.079756in}{1.581535in}}%
\pgfpathlineto{\pgfqpoint{1.081183in}{1.580056in}}%
\pgfpathclose%
\pgfusepath{fill}%
\end{pgfscope}%
\begin{pgfscope}%
\pgfpathrectangle{\pgfqpoint{0.041670in}{0.041670in}}{\pgfqpoint{2.216660in}{2.216660in}}%
\pgfusepath{clip}%
\pgfsetbuttcap%
\pgfsetroundjoin%
\definecolor{currentfill}{rgb}{0.280255,0.165693,0.476498}%
\pgfsetfillcolor{currentfill}%
\pgfsetlinewidth{0.000000pt}%
\definecolor{currentstroke}{rgb}{0.000000,0.000000,0.000000}%
\pgfsetstrokecolor{currentstroke}%
\pgfsetdash{}{0pt}%
\pgfpathmoveto{\pgfqpoint{1.065262in}{0.732593in}}%
\pgfpathlineto{\pgfqpoint{1.064338in}{0.724852in}}%
\pgfpathlineto{\pgfqpoint{1.063413in}{0.717231in}}%
\pgfpathlineto{\pgfqpoint{1.062487in}{0.709734in}}%
\pgfpathlineto{\pgfqpoint{1.061561in}{0.702366in}}%
\pgfpathlineto{\pgfqpoint{1.046290in}{0.704493in}}%
\pgfpathlineto{\pgfqpoint{1.031165in}{0.706875in}}%
\pgfpathlineto{\pgfqpoint{1.016205in}{0.709511in}}%
\pgfpathlineto{\pgfqpoint{1.001424in}{0.712396in}}%
\pgfpathlineto{\pgfqpoint{1.002819in}{0.719676in}}%
\pgfpathlineto{\pgfqpoint{1.004213in}{0.727085in}}%
\pgfpathlineto{\pgfqpoint{1.005606in}{0.734618in}}%
\pgfpathlineto{\pgfqpoint{1.006998in}{0.742270in}}%
\pgfpathlineto{\pgfqpoint{1.021319in}{0.739487in}}%
\pgfpathlineto{\pgfqpoint{1.035814in}{0.736944in}}%
\pgfpathlineto{\pgfqpoint{1.050467in}{0.734645in}}%
\pgfpathlineto{\pgfqpoint{1.065262in}{0.732593in}}%
\pgfpathclose%
\pgfusepath{fill}%
\end{pgfscope}%
\begin{pgfscope}%
\pgfpathrectangle{\pgfqpoint{0.041670in}{0.041670in}}{\pgfqpoint{2.216660in}{2.216660in}}%
\pgfusepath{clip}%
\pgfsetbuttcap%
\pgfsetroundjoin%
\definecolor{currentfill}{rgb}{0.974417,0.903590,0.130215}%
\pgfsetfillcolor{currentfill}%
\pgfsetlinewidth{0.000000pt}%
\definecolor{currentstroke}{rgb}{0.000000,0.000000,0.000000}%
\pgfsetstrokecolor{currentstroke}%
\pgfsetdash{}{0pt}%
\pgfpathmoveto{\pgfqpoint{1.150265in}{1.644890in}}%
\pgfpathlineto{\pgfqpoint{1.146552in}{1.643180in}}%
\pgfpathlineto{\pgfqpoint{1.142840in}{1.641337in}}%
\pgfpathlineto{\pgfqpoint{1.139127in}{1.639362in}}%
\pgfpathlineto{\pgfqpoint{1.135415in}{1.637256in}}%
\pgfpathlineto{\pgfqpoint{1.134973in}{1.637916in}}%
\pgfpathlineto{\pgfqpoint{1.134576in}{1.638582in}}%
\pgfpathlineto{\pgfqpoint{1.134224in}{1.639253in}}%
\pgfpathlineto{\pgfqpoint{1.133918in}{1.639930in}}%
\pgfpathlineto{\pgfqpoint{1.137753in}{1.641813in}}%
\pgfpathlineto{\pgfqpoint{1.141589in}{1.643564in}}%
\pgfpathlineto{\pgfqpoint{1.145426in}{1.645183in}}%
\pgfpathlineto{\pgfqpoint{1.149262in}{1.646670in}}%
\pgfpathlineto{\pgfqpoint{1.149468in}{1.646220in}}%
\pgfpathlineto{\pgfqpoint{1.149703in}{1.645773in}}%
\pgfpathlineto{\pgfqpoint{1.149969in}{1.645329in}}%
\pgfpathlineto{\pgfqpoint{1.150265in}{1.644890in}}%
\pgfpathclose%
\pgfusepath{fill}%
\end{pgfscope}%
\begin{pgfscope}%
\pgfpathrectangle{\pgfqpoint{0.041670in}{0.041670in}}{\pgfqpoint{2.216660in}{2.216660in}}%
\pgfusepath{clip}%
\pgfsetbuttcap%
\pgfsetroundjoin%
\definecolor{currentfill}{rgb}{0.220124,0.725509,0.466226}%
\pgfsetfillcolor{currentfill}%
\pgfsetlinewidth{0.000000pt}%
\definecolor{currentstroke}{rgb}{0.000000,0.000000,0.000000}%
\pgfsetstrokecolor{currentstroke}%
\pgfsetdash{}{0pt}%
\pgfpathmoveto{\pgfqpoint{1.010267in}{1.316925in}}%
\pgfpathlineto{\pgfqpoint{1.007638in}{1.308479in}}%
\pgfpathlineto{\pgfqpoint{1.005010in}{1.299972in}}%
\pgfpathlineto{\pgfqpoint{1.002383in}{1.291407in}}%
\pgfpathlineto{\pgfqpoint{0.999758in}{1.282786in}}%
\pgfpathlineto{\pgfqpoint{0.993742in}{1.285635in}}%
\pgfpathlineto{\pgfqpoint{0.987919in}{1.288575in}}%
\pgfpathlineto{\pgfqpoint{0.982296in}{1.291602in}}%
\pgfpathlineto{\pgfqpoint{0.976877in}{1.294714in}}%
\pgfpathlineto{\pgfqpoint{0.979831in}{1.303146in}}%
\pgfpathlineto{\pgfqpoint{0.982786in}{1.311523in}}%
\pgfpathlineto{\pgfqpoint{0.985743in}{1.319843in}}%
\pgfpathlineto{\pgfqpoint{0.988702in}{1.328102in}}%
\pgfpathlineto{\pgfqpoint{0.993810in}{1.325186in}}%
\pgfpathlineto{\pgfqpoint{0.999111in}{1.322349in}}%
\pgfpathlineto{\pgfqpoint{1.004598in}{1.319595in}}%
\pgfpathlineto{\pgfqpoint{1.010267in}{1.316925in}}%
\pgfpathclose%
\pgfusepath{fill}%
\end{pgfscope}%
\begin{pgfscope}%
\pgfpathrectangle{\pgfqpoint{0.041670in}{0.041670in}}{\pgfqpoint{2.216660in}{2.216660in}}%
\pgfusepath{clip}%
\pgfsetbuttcap%
\pgfsetroundjoin%
\definecolor{currentfill}{rgb}{0.282884,0.135920,0.453427}%
\pgfsetfillcolor{currentfill}%
\pgfsetlinewidth{0.000000pt}%
\definecolor{currentstroke}{rgb}{0.000000,0.000000,0.000000}%
\pgfsetstrokecolor{currentstroke}%
\pgfsetdash{}{0pt}%
\pgfpathmoveto{\pgfqpoint{0.713285in}{0.672649in}}%
\pgfpathlineto{\pgfqpoint{0.710384in}{0.677615in}}%
\pgfpathlineto{\pgfqpoint{0.707472in}{0.682956in}}%
\pgfpathlineto{\pgfqpoint{0.704548in}{0.688678in}}%
\pgfpathlineto{\pgfqpoint{0.701612in}{0.694788in}}%
\pgfpathlineto{\pgfqpoint{0.685984in}{0.702998in}}%
\pgfpathlineto{\pgfqpoint{0.670902in}{0.711457in}}%
\pgfpathlineto{\pgfqpoint{0.656379in}{0.720155in}}%
\pgfpathlineto{\pgfqpoint{0.642430in}{0.729083in}}%
\pgfpathlineto{\pgfqpoint{0.645714in}{0.722792in}}%
\pgfpathlineto{\pgfqpoint{0.648985in}{0.716887in}}%
\pgfpathlineto{\pgfqpoint{0.652243in}{0.711362in}}%
\pgfpathlineto{\pgfqpoint{0.655488in}{0.706211in}}%
\pgfpathlineto{\pgfqpoint{0.669113in}{0.697473in}}%
\pgfpathlineto{\pgfqpoint{0.683297in}{0.688960in}}%
\pgfpathlineto{\pgfqpoint{0.698026in}{0.680682in}}%
\pgfpathlineto{\pgfqpoint{0.713285in}{0.672649in}}%
\pgfpathclose%
\pgfusepath{fill}%
\end{pgfscope}%
\begin{pgfscope}%
\pgfpathrectangle{\pgfqpoint{0.041670in}{0.041670in}}{\pgfqpoint{2.216660in}{2.216660in}}%
\pgfusepath{clip}%
\pgfsetbuttcap%
\pgfsetroundjoin%
\definecolor{currentfill}{rgb}{0.993248,0.906157,0.143936}%
\pgfsetfillcolor{currentfill}%
\pgfsetlinewidth{0.000000pt}%
\definecolor{currentstroke}{rgb}{0.000000,0.000000,0.000000}%
\pgfsetstrokecolor{currentstroke}%
\pgfsetdash{}{0pt}%
\pgfpathmoveto{\pgfqpoint{1.179955in}{1.653775in}}%
\pgfpathlineto{\pgfqpoint{1.182682in}{1.652403in}}%
\pgfpathlineto{\pgfqpoint{1.185411in}{1.650897in}}%
\pgfpathlineto{\pgfqpoint{1.188141in}{1.649256in}}%
\pgfpathlineto{\pgfqpoint{1.190871in}{1.647481in}}%
\pgfpathlineto{\pgfqpoint{1.190512in}{1.647323in}}%
\pgfpathlineto{\pgfqpoint{1.190142in}{1.647170in}}%
\pgfpathlineto{\pgfqpoint{1.189762in}{1.647023in}}%
\pgfpathlineto{\pgfqpoint{1.189372in}{1.646881in}}%
\pgfpathlineto{\pgfqpoint{1.187017in}{1.648807in}}%
\pgfpathlineto{\pgfqpoint{1.184662in}{1.650597in}}%
\pgfpathlineto{\pgfqpoint{1.182308in}{1.652253in}}%
\pgfpathlineto{\pgfqpoint{1.179955in}{1.653775in}}%
\pgfpathlineto{\pgfqpoint{1.179955in}{1.653775in}}%
\pgfpathlineto{\pgfqpoint{1.179955in}{1.653775in}}%
\pgfpathlineto{\pgfqpoint{1.179955in}{1.653775in}}%
\pgfpathlineto{\pgfqpoint{1.179955in}{1.653775in}}%
\pgfpathclose%
\pgfusepath{fill}%
\end{pgfscope}%
\begin{pgfscope}%
\pgfpathrectangle{\pgfqpoint{0.041670in}{0.041670in}}{\pgfqpoint{2.216660in}{2.216660in}}%
\pgfusepath{clip}%
\pgfsetbuttcap%
\pgfsetroundjoin%
\definecolor{currentfill}{rgb}{0.993248,0.906157,0.143936}%
\pgfsetfillcolor{currentfill}%
\pgfsetlinewidth{0.000000pt}%
\definecolor{currentstroke}{rgb}{0.000000,0.000000,0.000000}%
\pgfsetstrokecolor{currentstroke}%
\pgfsetdash{}{0pt}%
\pgfpathmoveto{\pgfqpoint{1.179955in}{1.653775in}}%
\pgfpathlineto{\pgfqpoint{1.177691in}{1.652223in}}%
\pgfpathlineto{\pgfqpoint{1.175425in}{1.650537in}}%
\pgfpathlineto{\pgfqpoint{1.173159in}{1.648716in}}%
\pgfpathlineto{\pgfqpoint{1.170892in}{1.646761in}}%
\pgfpathlineto{\pgfqpoint{1.170493in}{1.646897in}}%
\pgfpathlineto{\pgfqpoint{1.170105in}{1.647039in}}%
\pgfpathlineto{\pgfqpoint{1.169726in}{1.647187in}}%
\pgfpathlineto{\pgfqpoint{1.169357in}{1.647340in}}%
\pgfpathlineto{\pgfqpoint{1.172008in}{1.649150in}}%
\pgfpathlineto{\pgfqpoint{1.174658in}{1.650826in}}%
\pgfpathlineto{\pgfqpoint{1.177307in}{1.652368in}}%
\pgfpathlineto{\pgfqpoint{1.179955in}{1.653775in}}%
\pgfpathlineto{\pgfqpoint{1.179955in}{1.653775in}}%
\pgfpathlineto{\pgfqpoint{1.179955in}{1.653775in}}%
\pgfpathlineto{\pgfqpoint{1.179955in}{1.653775in}}%
\pgfpathlineto{\pgfqpoint{1.179955in}{1.653775in}}%
\pgfpathclose%
\pgfusepath{fill}%
\end{pgfscope}%
\begin{pgfscope}%
\pgfpathrectangle{\pgfqpoint{0.041670in}{0.041670in}}{\pgfqpoint{2.216660in}{2.216660in}}%
\pgfusepath{clip}%
\pgfsetbuttcap%
\pgfsetroundjoin%
\definecolor{currentfill}{rgb}{0.283072,0.130895,0.449241}%
\pgfsetfillcolor{currentfill}%
\pgfsetlinewidth{0.000000pt}%
\definecolor{currentstroke}{rgb}{0.000000,0.000000,0.000000}%
\pgfsetstrokecolor{currentstroke}%
\pgfsetdash{}{0pt}%
\pgfpathmoveto{\pgfqpoint{1.061561in}{0.702366in}}%
\pgfpathlineto{\pgfqpoint{1.060634in}{0.695129in}}%
\pgfpathlineto{\pgfqpoint{1.059706in}{0.688028in}}%
\pgfpathlineto{\pgfqpoint{1.058777in}{0.681067in}}%
\pgfpathlineto{\pgfqpoint{1.057847in}{0.674249in}}%
\pgfpathlineto{\pgfqpoint{1.042098in}{0.676451in}}%
\pgfpathlineto{\pgfqpoint{1.026501in}{0.678918in}}%
\pgfpathlineto{\pgfqpoint{1.011074in}{0.681646in}}%
\pgfpathlineto{\pgfqpoint{0.995832in}{0.684633in}}%
\pgfpathlineto{\pgfqpoint{0.997232in}{0.691362in}}%
\pgfpathlineto{\pgfqpoint{0.998631in}{0.698235in}}%
\pgfpathlineto{\pgfqpoint{1.000028in}{0.705247in}}%
\pgfpathlineto{\pgfqpoint{1.001424in}{0.712396in}}%
\pgfpathlineto{\pgfqpoint{1.016205in}{0.709511in}}%
\pgfpathlineto{\pgfqpoint{1.031165in}{0.706875in}}%
\pgfpathlineto{\pgfqpoint{1.046290in}{0.704493in}}%
\pgfpathlineto{\pgfqpoint{1.061561in}{0.702366in}}%
\pgfpathclose%
\pgfusepath{fill}%
\end{pgfscope}%
\begin{pgfscope}%
\pgfpathrectangle{\pgfqpoint{0.041670in}{0.041670in}}{\pgfqpoint{2.216660in}{2.216660in}}%
\pgfusepath{clip}%
\pgfsetbuttcap%
\pgfsetroundjoin%
\definecolor{currentfill}{rgb}{0.268510,0.009605,0.335427}%
\pgfsetfillcolor{currentfill}%
\pgfsetlinewidth{0.000000pt}%
\definecolor{currentstroke}{rgb}{0.000000,0.000000,0.000000}%
\pgfsetstrokecolor{currentstroke}%
\pgfsetdash{}{0pt}%
\pgfpathmoveto{\pgfqpoint{1.555661in}{0.617974in}}%
\pgfpathlineto{\pgfqpoint{1.558120in}{0.617628in}}%
\pgfpathlineto{\pgfqpoint{1.560585in}{0.617564in}}%
\pgfpathlineto{\pgfqpoint{1.563057in}{0.617787in}}%
\pgfpathlineto{\pgfqpoint{1.565537in}{0.618301in}}%
\pgfpathlineto{\pgfqpoint{1.549501in}{0.611853in}}%
\pgfpathlineto{\pgfqpoint{1.533055in}{0.605678in}}%
\pgfpathlineto{\pgfqpoint{1.516217in}{0.599782in}}%
\pgfpathlineto{\pgfqpoint{1.499005in}{0.594172in}}%
\pgfpathlineto{\pgfqpoint{1.496947in}{0.593812in}}%
\pgfpathlineto{\pgfqpoint{1.494895in}{0.593744in}}%
\pgfpathlineto{\pgfqpoint{1.492848in}{0.593963in}}%
\pgfpathlineto{\pgfqpoint{1.490808in}{0.594464in}}%
\pgfpathlineto{\pgfqpoint{1.507584in}{0.599929in}}%
\pgfpathlineto{\pgfqpoint{1.523997in}{0.605674in}}%
\pgfpathlineto{\pgfqpoint{1.540028in}{0.611691in}}%
\pgfpathlineto{\pgfqpoint{1.555661in}{0.617974in}}%
\pgfpathclose%
\pgfusepath{fill}%
\end{pgfscope}%
\begin{pgfscope}%
\pgfpathrectangle{\pgfqpoint{0.041670in}{0.041670in}}{\pgfqpoint{2.216660in}{2.216660in}}%
\pgfusepath{clip}%
\pgfsetbuttcap%
\pgfsetroundjoin%
\definecolor{currentfill}{rgb}{0.412913,0.803041,0.357269}%
\pgfsetfillcolor{currentfill}%
\pgfsetlinewidth{0.000000pt}%
\definecolor{currentstroke}{rgb}{0.000000,0.000000,0.000000}%
\pgfsetstrokecolor{currentstroke}%
\pgfsetdash{}{0pt}%
\pgfpathmoveto{\pgfqpoint{1.024316in}{1.421763in}}%
\pgfpathlineto{\pgfqpoint{1.021340in}{1.414387in}}%
\pgfpathlineto{\pgfqpoint{1.018366in}{1.406926in}}%
\pgfpathlineto{\pgfqpoint{1.015393in}{1.399383in}}%
\pgfpathlineto{\pgfqpoint{1.012422in}{1.391760in}}%
\pgfpathlineto{\pgfqpoint{1.008110in}{1.394356in}}%
\pgfpathlineto{\pgfqpoint{1.003975in}{1.397016in}}%
\pgfpathlineto{\pgfqpoint{1.000020in}{1.399737in}}%
\pgfpathlineto{\pgfqpoint{0.996249in}{1.402516in}}%
\pgfpathlineto{\pgfqpoint{0.999503in}{1.409937in}}%
\pgfpathlineto{\pgfqpoint{1.002757in}{1.417278in}}%
\pgfpathlineto{\pgfqpoint{1.006014in}{1.424537in}}%
\pgfpathlineto{\pgfqpoint{1.009272in}{1.431712in}}%
\pgfpathlineto{\pgfqpoint{1.012780in}{1.429141in}}%
\pgfpathlineto{\pgfqpoint{1.016460in}{1.426625in}}%
\pgfpathlineto{\pgfqpoint{1.020306in}{1.424164in}}%
\pgfpathlineto{\pgfqpoint{1.024316in}{1.421763in}}%
\pgfpathclose%
\pgfusepath{fill}%
\end{pgfscope}%
\begin{pgfscope}%
\pgfpathrectangle{\pgfqpoint{0.041670in}{0.041670in}}{\pgfqpoint{2.216660in}{2.216660in}}%
\pgfusepath{clip}%
\pgfsetbuttcap%
\pgfsetroundjoin%
\definecolor{currentfill}{rgb}{0.636902,0.856542,0.216620}%
\pgfsetfillcolor{currentfill}%
\pgfsetlinewidth{0.000000pt}%
\definecolor{currentstroke}{rgb}{0.000000,0.000000,0.000000}%
\pgfsetstrokecolor{currentstroke}%
\pgfsetdash{}{0pt}%
\pgfpathmoveto{\pgfqpoint{1.048486in}{1.510611in}}%
\pgfpathlineto{\pgfqpoint{1.045211in}{1.504580in}}%
\pgfpathlineto{\pgfqpoint{1.041937in}{1.498445in}}%
\pgfpathlineto{\pgfqpoint{1.038664in}{1.492207in}}%
\pgfpathlineto{\pgfqpoint{1.035393in}{1.485868in}}%
\pgfpathlineto{\pgfqpoint{1.032560in}{1.488071in}}%
\pgfpathlineto{\pgfqpoint{1.029877in}{1.490314in}}%
\pgfpathlineto{\pgfqpoint{1.027346in}{1.492596in}}%
\pgfpathlineto{\pgfqpoint{1.024970in}{1.494914in}}%
\pgfpathlineto{\pgfqpoint{1.028472in}{1.501040in}}%
\pgfpathlineto{\pgfqpoint{1.031977in}{1.507065in}}%
\pgfpathlineto{\pgfqpoint{1.035483in}{1.512988in}}%
\pgfpathlineto{\pgfqpoint{1.038990in}{1.518808in}}%
\pgfpathlineto{\pgfqpoint{1.041156in}{1.516707in}}%
\pgfpathlineto{\pgfqpoint{1.043462in}{1.514639in}}%
\pgfpathlineto{\pgfqpoint{1.045906in}{1.512606in}}%
\pgfpathlineto{\pgfqpoint{1.048486in}{1.510611in}}%
\pgfpathclose%
\pgfusepath{fill}%
\end{pgfscope}%
\begin{pgfscope}%
\pgfpathrectangle{\pgfqpoint{0.041670in}{0.041670in}}{\pgfqpoint{2.216660in}{2.216660in}}%
\pgfusepath{clip}%
\pgfsetbuttcap%
\pgfsetroundjoin%
\definecolor{currentfill}{rgb}{0.274128,0.199721,0.498911}%
\pgfsetfillcolor{currentfill}%
\pgfsetlinewidth{0.000000pt}%
\definecolor{currentstroke}{rgb}{0.000000,0.000000,0.000000}%
\pgfsetstrokecolor{currentstroke}%
\pgfsetdash{}{0pt}%
\pgfpathmoveto{\pgfqpoint{1.068953in}{0.764684in}}%
\pgfpathlineto{\pgfqpoint{1.068031in}{0.756500in}}%
\pgfpathlineto{\pgfqpoint{1.067108in}{0.748421in}}%
\pgfpathlineto{\pgfqpoint{1.066186in}{0.740451in}}%
\pgfpathlineto{\pgfqpoint{1.065262in}{0.732593in}}%
\pgfpathlineto{\pgfqpoint{1.050467in}{0.734645in}}%
\pgfpathlineto{\pgfqpoint{1.035814in}{0.736944in}}%
\pgfpathlineto{\pgfqpoint{1.021319in}{0.739487in}}%
\pgfpathlineto{\pgfqpoint{1.006998in}{0.742270in}}%
\pgfpathlineto{\pgfqpoint{1.008389in}{0.750040in}}%
\pgfpathlineto{\pgfqpoint{1.009779in}{0.757922in}}%
\pgfpathlineto{\pgfqpoint{1.011168in}{0.765913in}}%
\pgfpathlineto{\pgfqpoint{1.012556in}{0.774010in}}%
\pgfpathlineto{\pgfqpoint{1.026419in}{0.771327in}}%
\pgfpathlineto{\pgfqpoint{1.040449in}{0.768876in}}%
\pgfpathlineto{\pgfqpoint{1.054632in}{0.766661in}}%
\pgfpathlineto{\pgfqpoint{1.068953in}{0.764684in}}%
\pgfpathclose%
\pgfusepath{fill}%
\end{pgfscope}%
\begin{pgfscope}%
\pgfpathrectangle{\pgfqpoint{0.041670in}{0.041670in}}{\pgfqpoint{2.216660in}{2.216660in}}%
\pgfusepath{clip}%
\pgfsetbuttcap%
\pgfsetroundjoin%
\definecolor{currentfill}{rgb}{0.993248,0.906157,0.143936}%
\pgfsetfillcolor{currentfill}%
\pgfsetlinewidth{0.000000pt}%
\definecolor{currentstroke}{rgb}{0.000000,0.000000,0.000000}%
\pgfsetstrokecolor{currentstroke}%
\pgfsetdash{}{0pt}%
\pgfpathmoveto{\pgfqpoint{1.179955in}{1.653775in}}%
\pgfpathlineto{\pgfqpoint{1.182308in}{1.652253in}}%
\pgfpathlineto{\pgfqpoint{1.184662in}{1.650597in}}%
\pgfpathlineto{\pgfqpoint{1.187017in}{1.648807in}}%
\pgfpathlineto{\pgfqpoint{1.189372in}{1.646881in}}%
\pgfpathlineto{\pgfqpoint{1.188973in}{1.646746in}}%
\pgfpathlineto{\pgfqpoint{1.188565in}{1.646616in}}%
\pgfpathlineto{\pgfqpoint{1.188148in}{1.646492in}}%
\pgfpathlineto{\pgfqpoint{1.187723in}{1.646375in}}%
\pgfpathlineto{\pgfqpoint{1.185780in}{1.648427in}}%
\pgfpathlineto{\pgfqpoint{1.183837in}{1.650344in}}%
\pgfpathlineto{\pgfqpoint{1.181896in}{1.652127in}}%
\pgfpathlineto{\pgfqpoint{1.179955in}{1.653775in}}%
\pgfpathlineto{\pgfqpoint{1.179955in}{1.653775in}}%
\pgfpathlineto{\pgfqpoint{1.179955in}{1.653775in}}%
\pgfpathlineto{\pgfqpoint{1.179955in}{1.653775in}}%
\pgfpathlineto{\pgfqpoint{1.179955in}{1.653775in}}%
\pgfpathclose%
\pgfusepath{fill}%
\end{pgfscope}%
\begin{pgfscope}%
\pgfpathrectangle{\pgfqpoint{0.041670in}{0.041670in}}{\pgfqpoint{2.216660in}{2.216660in}}%
\pgfusepath{clip}%
\pgfsetbuttcap%
\pgfsetroundjoin%
\definecolor{currentfill}{rgb}{0.855810,0.888601,0.097452}%
\pgfsetfillcolor{currentfill}%
\pgfsetlinewidth{0.000000pt}%
\definecolor{currentstroke}{rgb}{0.000000,0.000000,0.000000}%
\pgfsetstrokecolor{currentstroke}%
\pgfsetdash{}{0pt}%
\pgfpathmoveto{\pgfqpoint{1.269793in}{1.603070in}}%
\pgfpathlineto{\pgfqpoint{1.273530in}{1.599352in}}%
\pgfpathlineto{\pgfqpoint{1.277266in}{1.595512in}}%
\pgfpathlineto{\pgfqpoint{1.281001in}{1.591550in}}%
\pgfpathlineto{\pgfqpoint{1.284734in}{1.587467in}}%
\pgfpathlineto{\pgfqpoint{1.283706in}{1.585918in}}%
\pgfpathlineto{\pgfqpoint{1.282573in}{1.584384in}}%
\pgfpathlineto{\pgfqpoint{1.281337in}{1.582868in}}%
\pgfpathlineto{\pgfqpoint{1.280000in}{1.581370in}}%
\pgfpathlineto{\pgfqpoint{1.276432in}{1.585674in}}%
\pgfpathlineto{\pgfqpoint{1.272863in}{1.589856in}}%
\pgfpathlineto{\pgfqpoint{1.269293in}{1.593917in}}%
\pgfpathlineto{\pgfqpoint{1.265723in}{1.597855in}}%
\pgfpathlineto{\pgfqpoint{1.266872in}{1.599136in}}%
\pgfpathlineto{\pgfqpoint{1.267935in}{1.600433in}}%
\pgfpathlineto{\pgfqpoint{1.268909in}{1.601745in}}%
\pgfpathlineto{\pgfqpoint{1.269793in}{1.603070in}}%
\pgfpathclose%
\pgfusepath{fill}%
\end{pgfscope}%
\begin{pgfscope}%
\pgfpathrectangle{\pgfqpoint{0.041670in}{0.041670in}}{\pgfqpoint{2.216660in}{2.216660in}}%
\pgfusepath{clip}%
\pgfsetbuttcap%
\pgfsetroundjoin%
\definecolor{currentfill}{rgb}{0.993248,0.906157,0.143936}%
\pgfsetfillcolor{currentfill}%
\pgfsetlinewidth{0.000000pt}%
\definecolor{currentstroke}{rgb}{0.000000,0.000000,0.000000}%
\pgfsetstrokecolor{currentstroke}%
\pgfsetdash{}{0pt}%
\pgfpathmoveto{\pgfqpoint{1.179955in}{1.653775in}}%
\pgfpathlineto{\pgfqpoint{1.178110in}{1.652102in}}%
\pgfpathlineto{\pgfqpoint{1.176265in}{1.650295in}}%
\pgfpathlineto{\pgfqpoint{1.174418in}{1.648352in}}%
\pgfpathlineto{\pgfqpoint{1.172571in}{1.646276in}}%
\pgfpathlineto{\pgfqpoint{1.172139in}{1.646388in}}%
\pgfpathlineto{\pgfqpoint{1.171715in}{1.646506in}}%
\pgfpathlineto{\pgfqpoint{1.171299in}{1.646630in}}%
\pgfpathlineto{\pgfqpoint{1.170892in}{1.646761in}}%
\pgfpathlineto{\pgfqpoint{1.173159in}{1.648716in}}%
\pgfpathlineto{\pgfqpoint{1.175425in}{1.650537in}}%
\pgfpathlineto{\pgfqpoint{1.177691in}{1.652223in}}%
\pgfpathlineto{\pgfqpoint{1.179955in}{1.653775in}}%
\pgfpathlineto{\pgfqpoint{1.179955in}{1.653775in}}%
\pgfpathlineto{\pgfqpoint{1.179955in}{1.653775in}}%
\pgfpathlineto{\pgfqpoint{1.179955in}{1.653775in}}%
\pgfpathlineto{\pgfqpoint{1.179955in}{1.653775in}}%
\pgfpathclose%
\pgfusepath{fill}%
\end{pgfscope}%
\begin{pgfscope}%
\pgfpathrectangle{\pgfqpoint{0.041670in}{0.041670in}}{\pgfqpoint{2.216660in}{2.216660in}}%
\pgfusepath{clip}%
\pgfsetbuttcap%
\pgfsetroundjoin%
\definecolor{currentfill}{rgb}{0.993248,0.906157,0.143936}%
\pgfsetfillcolor{currentfill}%
\pgfsetlinewidth{0.000000pt}%
\definecolor{currentstroke}{rgb}{0.000000,0.000000,0.000000}%
\pgfsetstrokecolor{currentstroke}%
\pgfsetdash{}{0pt}%
\pgfpathmoveto{\pgfqpoint{1.194930in}{1.650596in}}%
\pgfpathlineto{\pgfqpoint{1.198674in}{1.649467in}}%
\pgfpathlineto{\pgfqpoint{1.202419in}{1.648204in}}%
\pgfpathlineto{\pgfqpoint{1.206164in}{1.646809in}}%
\pgfpathlineto{\pgfqpoint{1.209909in}{1.645280in}}%
\pgfpathlineto{\pgfqpoint{1.209610in}{1.644841in}}%
\pgfpathlineto{\pgfqpoint{1.209282in}{1.644407in}}%
\pgfpathlineto{\pgfqpoint{1.208924in}{1.643978in}}%
\pgfpathlineto{\pgfqpoint{1.208537in}{1.643554in}}%
\pgfpathlineto{\pgfqpoint{1.204962in}{1.645299in}}%
\pgfpathlineto{\pgfqpoint{1.201388in}{1.646910in}}%
\pgfpathlineto{\pgfqpoint{1.197815in}{1.648389in}}%
\pgfpathlineto{\pgfqpoint{1.194241in}{1.649734in}}%
\pgfpathlineto{\pgfqpoint{1.194435in}{1.649945in}}%
\pgfpathlineto{\pgfqpoint{1.194615in}{1.650160in}}%
\pgfpathlineto{\pgfqpoint{1.194780in}{1.650377in}}%
\pgfpathlineto{\pgfqpoint{1.194930in}{1.650596in}}%
\pgfpathclose%
\pgfusepath{fill}%
\end{pgfscope}%
\begin{pgfscope}%
\pgfpathrectangle{\pgfqpoint{0.041670in}{0.041670in}}{\pgfqpoint{2.216660in}{2.216660in}}%
\pgfusepath{clip}%
\pgfsetbuttcap%
\pgfsetroundjoin%
\definecolor{currentfill}{rgb}{0.993248,0.906157,0.143936}%
\pgfsetfillcolor{currentfill}%
\pgfsetlinewidth{0.000000pt}%
\definecolor{currentstroke}{rgb}{0.000000,0.000000,0.000000}%
\pgfsetstrokecolor{currentstroke}%
\pgfsetdash{}{0pt}%
\pgfpathmoveto{\pgfqpoint{1.179955in}{1.653775in}}%
\pgfpathlineto{\pgfqpoint{1.181896in}{1.652127in}}%
\pgfpathlineto{\pgfqpoint{1.183837in}{1.650344in}}%
\pgfpathlineto{\pgfqpoint{1.185780in}{1.648427in}}%
\pgfpathlineto{\pgfqpoint{1.187723in}{1.646375in}}%
\pgfpathlineto{\pgfqpoint{1.187290in}{1.646264in}}%
\pgfpathlineto{\pgfqpoint{1.186850in}{1.646159in}}%
\pgfpathlineto{\pgfqpoint{1.186403in}{1.646061in}}%
\pgfpathlineto{\pgfqpoint{1.185949in}{1.645969in}}%
\pgfpathlineto{\pgfqpoint{1.184450in}{1.648123in}}%
\pgfpathlineto{\pgfqpoint{1.182951in}{1.650141in}}%
\pgfpathlineto{\pgfqpoint{1.181452in}{1.652026in}}%
\pgfpathlineto{\pgfqpoint{1.179955in}{1.653775in}}%
\pgfpathlineto{\pgfqpoint{1.179955in}{1.653775in}}%
\pgfpathlineto{\pgfqpoint{1.179955in}{1.653775in}}%
\pgfpathlineto{\pgfqpoint{1.179955in}{1.653775in}}%
\pgfpathlineto{\pgfqpoint{1.179955in}{1.653775in}}%
\pgfpathclose%
\pgfusepath{fill}%
\end{pgfscope}%
\begin{pgfscope}%
\pgfpathrectangle{\pgfqpoint{0.041670in}{0.041670in}}{\pgfqpoint{2.216660in}{2.216660in}}%
\pgfusepath{clip}%
\pgfsetbuttcap%
\pgfsetroundjoin%
\definecolor{currentfill}{rgb}{0.195860,0.395433,0.555276}%
\pgfsetfillcolor{currentfill}%
\pgfsetlinewidth{0.000000pt}%
\definecolor{currentstroke}{rgb}{0.000000,0.000000,0.000000}%
\pgfsetstrokecolor{currentstroke}%
\pgfsetdash{}{0pt}%
\pgfpathmoveto{\pgfqpoint{1.040235in}{0.952543in}}%
\pgfpathlineto{\pgfqpoint{1.038853in}{0.943082in}}%
\pgfpathlineto{\pgfqpoint{1.037470in}{0.933658in}}%
\pgfpathlineto{\pgfqpoint{1.036088in}{0.924273in}}%
\pgfpathlineto{\pgfqpoint{1.034706in}{0.914931in}}%
\pgfpathlineto{\pgfqpoint{1.022827in}{0.917409in}}%
\pgfpathlineto{\pgfqpoint{1.011118in}{0.920079in}}%
\pgfpathlineto{\pgfqpoint{0.999591in}{0.922939in}}%
\pgfpathlineto{\pgfqpoint{0.988258in}{0.925985in}}%
\pgfpathlineto{\pgfqpoint{0.990080in}{0.935207in}}%
\pgfpathlineto{\pgfqpoint{0.991901in}{0.944473in}}%
\pgfpathlineto{\pgfqpoint{0.993723in}{0.953779in}}%
\pgfpathlineto{\pgfqpoint{0.995545in}{0.963121in}}%
\pgfpathlineto{\pgfqpoint{1.006450in}{0.960206in}}%
\pgfpathlineto{\pgfqpoint{1.017541in}{0.957470in}}%
\pgfpathlineto{\pgfqpoint{1.028807in}{0.954914in}}%
\pgfpathlineto{\pgfqpoint{1.040235in}{0.952543in}}%
\pgfpathclose%
\pgfusepath{fill}%
\end{pgfscope}%
\begin{pgfscope}%
\pgfpathrectangle{\pgfqpoint{0.041670in}{0.041670in}}{\pgfqpoint{2.216660in}{2.216660in}}%
\pgfusepath{clip}%
\pgfsetbuttcap%
\pgfsetroundjoin%
\definecolor{currentfill}{rgb}{0.993248,0.906157,0.143936}%
\pgfsetfillcolor{currentfill}%
\pgfsetlinewidth{0.000000pt}%
\definecolor{currentstroke}{rgb}{0.000000,0.000000,0.000000}%
\pgfsetstrokecolor{currentstroke}%
\pgfsetdash{}{0pt}%
\pgfpathmoveto{\pgfqpoint{1.179955in}{1.653775in}}%
\pgfpathlineto{\pgfqpoint{1.178559in}{1.652007in}}%
\pgfpathlineto{\pgfqpoint{1.177163in}{1.650104in}}%
\pgfpathlineto{\pgfqpoint{1.175766in}{1.648066in}}%
\pgfpathlineto{\pgfqpoint{1.174368in}{1.645893in}}%
\pgfpathlineto{\pgfqpoint{1.173910in}{1.645979in}}%
\pgfpathlineto{\pgfqpoint{1.173457in}{1.646071in}}%
\pgfpathlineto{\pgfqpoint{1.173010in}{1.646170in}}%
\pgfpathlineto{\pgfqpoint{1.172571in}{1.646276in}}%
\pgfpathlineto{\pgfqpoint{1.174418in}{1.648352in}}%
\pgfpathlineto{\pgfqpoint{1.176265in}{1.650295in}}%
\pgfpathlineto{\pgfqpoint{1.178110in}{1.652102in}}%
\pgfpathlineto{\pgfqpoint{1.179955in}{1.653775in}}%
\pgfpathlineto{\pgfqpoint{1.179955in}{1.653775in}}%
\pgfpathlineto{\pgfqpoint{1.179955in}{1.653775in}}%
\pgfpathlineto{\pgfqpoint{1.179955in}{1.653775in}}%
\pgfpathlineto{\pgfqpoint{1.179955in}{1.653775in}}%
\pgfpathclose%
\pgfusepath{fill}%
\end{pgfscope}%
\begin{pgfscope}%
\pgfpathrectangle{\pgfqpoint{0.041670in}{0.041670in}}{\pgfqpoint{2.216660in}{2.216660in}}%
\pgfusepath{clip}%
\pgfsetbuttcap%
\pgfsetroundjoin%
\definecolor{currentfill}{rgb}{0.993248,0.906157,0.143936}%
\pgfsetfillcolor{currentfill}%
\pgfsetlinewidth{0.000000pt}%
\definecolor{currentstroke}{rgb}{0.000000,0.000000,0.000000}%
\pgfsetstrokecolor{currentstroke}%
\pgfsetdash{}{0pt}%
\pgfpathmoveto{\pgfqpoint{1.179955in}{1.653775in}}%
\pgfpathlineto{\pgfqpoint{1.181452in}{1.652026in}}%
\pgfpathlineto{\pgfqpoint{1.182951in}{1.650141in}}%
\pgfpathlineto{\pgfqpoint{1.184450in}{1.648123in}}%
\pgfpathlineto{\pgfqpoint{1.185949in}{1.645969in}}%
\pgfpathlineto{\pgfqpoint{1.185490in}{1.645884in}}%
\pgfpathlineto{\pgfqpoint{1.185025in}{1.645806in}}%
\pgfpathlineto{\pgfqpoint{1.184555in}{1.645735in}}%
\pgfpathlineto{\pgfqpoint{1.184080in}{1.645671in}}%
\pgfpathlineto{\pgfqpoint{1.183048in}{1.647899in}}%
\pgfpathlineto{\pgfqpoint{1.182016in}{1.649992in}}%
\pgfpathlineto{\pgfqpoint{1.180985in}{1.651951in}}%
\pgfpathlineto{\pgfqpoint{1.179955in}{1.653775in}}%
\pgfpathlineto{\pgfqpoint{1.179955in}{1.653775in}}%
\pgfpathlineto{\pgfqpoint{1.179955in}{1.653775in}}%
\pgfpathlineto{\pgfqpoint{1.179955in}{1.653775in}}%
\pgfpathlineto{\pgfqpoint{1.179955in}{1.653775in}}%
\pgfpathclose%
\pgfusepath{fill}%
\end{pgfscope}%
\begin{pgfscope}%
\pgfpathrectangle{\pgfqpoint{0.041670in}{0.041670in}}{\pgfqpoint{2.216660in}{2.216660in}}%
\pgfusepath{clip}%
\pgfsetbuttcap%
\pgfsetroundjoin%
\definecolor{currentfill}{rgb}{0.993248,0.906157,0.143936}%
\pgfsetfillcolor{currentfill}%
\pgfsetlinewidth{0.000000pt}%
\definecolor{currentstroke}{rgb}{0.000000,0.000000,0.000000}%
\pgfsetstrokecolor{currentstroke}%
\pgfsetdash{}{0pt}%
\pgfpathmoveto{\pgfqpoint{1.165853in}{1.649548in}}%
\pgfpathlineto{\pgfqpoint{1.162325in}{1.648157in}}%
\pgfpathlineto{\pgfqpoint{1.158798in}{1.646632in}}%
\pgfpathlineto{\pgfqpoint{1.155269in}{1.644973in}}%
\pgfpathlineto{\pgfqpoint{1.151741in}{1.643182in}}%
\pgfpathlineto{\pgfqpoint{1.151328in}{1.643601in}}%
\pgfpathlineto{\pgfqpoint{1.150945in}{1.644025in}}%
\pgfpathlineto{\pgfqpoint{1.150590in}{1.644455in}}%
\pgfpathlineto{\pgfqpoint{1.150265in}{1.644890in}}%
\pgfpathlineto{\pgfqpoint{1.153977in}{1.646467in}}%
\pgfpathlineto{\pgfqpoint{1.157689in}{1.647912in}}%
\pgfpathlineto{\pgfqpoint{1.161401in}{1.649223in}}%
\pgfpathlineto{\pgfqpoint{1.165113in}{1.650401in}}%
\pgfpathlineto{\pgfqpoint{1.165276in}{1.650184in}}%
\pgfpathlineto{\pgfqpoint{1.165454in}{1.649969in}}%
\pgfpathlineto{\pgfqpoint{1.165646in}{1.649757in}}%
\pgfpathlineto{\pgfqpoint{1.165853in}{1.649548in}}%
\pgfpathclose%
\pgfusepath{fill}%
\end{pgfscope}%
\begin{pgfscope}%
\pgfpathrectangle{\pgfqpoint{0.041670in}{0.041670in}}{\pgfqpoint{2.216660in}{2.216660in}}%
\pgfusepath{clip}%
\pgfsetbuttcap%
\pgfsetroundjoin%
\definecolor{currentfill}{rgb}{0.993248,0.906157,0.143936}%
\pgfsetfillcolor{currentfill}%
\pgfsetlinewidth{0.000000pt}%
\definecolor{currentstroke}{rgb}{0.000000,0.000000,0.000000}%
\pgfsetstrokecolor{currentstroke}%
\pgfsetdash{}{0pt}%
\pgfpathmoveto{\pgfqpoint{1.179955in}{1.653775in}}%
\pgfpathlineto{\pgfqpoint{1.179031in}{1.651938in}}%
\pgfpathlineto{\pgfqpoint{1.178106in}{1.649967in}}%
\pgfpathlineto{\pgfqpoint{1.177181in}{1.647860in}}%
\pgfpathlineto{\pgfqpoint{1.176255in}{1.645619in}}%
\pgfpathlineto{\pgfqpoint{1.175777in}{1.645677in}}%
\pgfpathlineto{\pgfqpoint{1.175302in}{1.645742in}}%
\pgfpathlineto{\pgfqpoint{1.174833in}{1.645814in}}%
\pgfpathlineto{\pgfqpoint{1.174368in}{1.645893in}}%
\pgfpathlineto{\pgfqpoint{1.175766in}{1.648066in}}%
\pgfpathlineto{\pgfqpoint{1.177163in}{1.650104in}}%
\pgfpathlineto{\pgfqpoint{1.178559in}{1.652007in}}%
\pgfpathlineto{\pgfqpoint{1.179955in}{1.653775in}}%
\pgfpathlineto{\pgfqpoint{1.179955in}{1.653775in}}%
\pgfpathlineto{\pgfqpoint{1.179955in}{1.653775in}}%
\pgfpathlineto{\pgfqpoint{1.179955in}{1.653775in}}%
\pgfpathlineto{\pgfqpoint{1.179955in}{1.653775in}}%
\pgfpathclose%
\pgfusepath{fill}%
\end{pgfscope}%
\begin{pgfscope}%
\pgfpathrectangle{\pgfqpoint{0.041670in}{0.041670in}}{\pgfqpoint{2.216660in}{2.216660in}}%
\pgfusepath{clip}%
\pgfsetbuttcap%
\pgfsetroundjoin%
\definecolor{currentfill}{rgb}{0.279566,0.067836,0.391917}%
\pgfsetfillcolor{currentfill}%
\pgfsetlinewidth{0.000000pt}%
\definecolor{currentstroke}{rgb}{0.000000,0.000000,0.000000}%
\pgfsetstrokecolor{currentstroke}%
\pgfsetdash{}{0pt}%
\pgfpathmoveto{\pgfqpoint{1.383477in}{0.662203in}}%
\pgfpathlineto{\pgfqpoint{1.384987in}{0.656279in}}%
\pgfpathlineto{\pgfqpoint{1.386498in}{0.650523in}}%
\pgfpathlineto{\pgfqpoint{1.388012in}{0.644940in}}%
\pgfpathlineto{\pgfqpoint{1.389527in}{0.639534in}}%
\pgfpathlineto{\pgfqpoint{1.373550in}{0.636101in}}%
\pgfpathlineto{\pgfqpoint{1.357358in}{0.632942in}}%
\pgfpathlineto{\pgfqpoint{1.340969in}{0.630059in}}%
\pgfpathlineto{\pgfqpoint{1.324402in}{0.627455in}}%
\pgfpathlineto{\pgfqpoint{1.323355in}{0.632957in}}%
\pgfpathlineto{\pgfqpoint{1.322310in}{0.638637in}}%
\pgfpathlineto{\pgfqpoint{1.321267in}{0.644489in}}%
\pgfpathlineto{\pgfqpoint{1.320225in}{0.650509in}}%
\pgfpathlineto{\pgfqpoint{1.336315in}{0.653030in}}%
\pgfpathlineto{\pgfqpoint{1.352232in}{0.655821in}}%
\pgfpathlineto{\pgfqpoint{1.367958in}{0.658880in}}%
\pgfpathlineto{\pgfqpoint{1.383477in}{0.662203in}}%
\pgfpathclose%
\pgfusepath{fill}%
\end{pgfscope}%
\begin{pgfscope}%
\pgfpathrectangle{\pgfqpoint{0.041670in}{0.041670in}}{\pgfqpoint{2.216660in}{2.216660in}}%
\pgfusepath{clip}%
\pgfsetbuttcap%
\pgfsetroundjoin%
\definecolor{currentfill}{rgb}{0.248629,0.278775,0.534556}%
\pgfsetfillcolor{currentfill}%
\pgfsetlinewidth{0.000000pt}%
\definecolor{currentstroke}{rgb}{0.000000,0.000000,0.000000}%
\pgfsetstrokecolor{currentstroke}%
\pgfsetdash{}{0pt}%
\pgfpathmoveto{\pgfqpoint{1.347635in}{0.844515in}}%
\pgfpathlineto{\pgfqpoint{1.349120in}{0.835732in}}%
\pgfpathlineto{\pgfqpoint{1.350604in}{0.827025in}}%
\pgfpathlineto{\pgfqpoint{1.352089in}{0.818398in}}%
\pgfpathlineto{\pgfqpoint{1.353575in}{0.809854in}}%
\pgfpathlineto{\pgfqpoint{1.340325in}{0.807076in}}%
\pgfpathlineto{\pgfqpoint{1.326902in}{0.804519in}}%
\pgfpathlineto{\pgfqpoint{1.313318in}{0.802186in}}%
\pgfpathlineto{\pgfqpoint{1.299588in}{0.800079in}}%
\pgfpathlineto{\pgfqpoint{1.298563in}{0.808718in}}%
\pgfpathlineto{\pgfqpoint{1.297539in}{0.817440in}}%
\pgfpathlineto{\pgfqpoint{1.296514in}{0.826242in}}%
\pgfpathlineto{\pgfqpoint{1.295490in}{0.835120in}}%
\pgfpathlineto{\pgfqpoint{1.308751in}{0.837144in}}%
\pgfpathlineto{\pgfqpoint{1.321871in}{0.839387in}}%
\pgfpathlineto{\pgfqpoint{1.334837in}{0.841845in}}%
\pgfpathlineto{\pgfqpoint{1.347635in}{0.844515in}}%
\pgfpathclose%
\pgfusepath{fill}%
\end{pgfscope}%
\begin{pgfscope}%
\pgfpathrectangle{\pgfqpoint{0.041670in}{0.041670in}}{\pgfqpoint{2.216660in}{2.216660in}}%
\pgfusepath{clip}%
\pgfsetbuttcap%
\pgfsetroundjoin%
\definecolor{currentfill}{rgb}{0.993248,0.906157,0.143936}%
\pgfsetfillcolor{currentfill}%
\pgfsetlinewidth{0.000000pt}%
\definecolor{currentstroke}{rgb}{0.000000,0.000000,0.000000}%
\pgfsetstrokecolor{currentstroke}%
\pgfsetdash{}{0pt}%
\pgfpathmoveto{\pgfqpoint{1.179955in}{1.653775in}}%
\pgfpathlineto{\pgfqpoint{1.180985in}{1.651951in}}%
\pgfpathlineto{\pgfqpoint{1.182016in}{1.649992in}}%
\pgfpathlineto{\pgfqpoint{1.183048in}{1.647899in}}%
\pgfpathlineto{\pgfqpoint{1.184080in}{1.645671in}}%
\pgfpathlineto{\pgfqpoint{1.183601in}{1.645613in}}%
\pgfpathlineto{\pgfqpoint{1.183119in}{1.645563in}}%
\pgfpathlineto{\pgfqpoint{1.182633in}{1.645520in}}%
\pgfpathlineto{\pgfqpoint{1.182145in}{1.645484in}}%
\pgfpathlineto{\pgfqpoint{1.181597in}{1.647759in}}%
\pgfpathlineto{\pgfqpoint{1.181049in}{1.649899in}}%
\pgfpathlineto{\pgfqpoint{1.180502in}{1.651905in}}%
\pgfpathlineto{\pgfqpoint{1.179955in}{1.653775in}}%
\pgfpathlineto{\pgfqpoint{1.179955in}{1.653775in}}%
\pgfpathlineto{\pgfqpoint{1.179955in}{1.653775in}}%
\pgfpathlineto{\pgfqpoint{1.179955in}{1.653775in}}%
\pgfpathlineto{\pgfqpoint{1.179955in}{1.653775in}}%
\pgfpathclose%
\pgfusepath{fill}%
\end{pgfscope}%
\begin{pgfscope}%
\pgfpathrectangle{\pgfqpoint{0.041670in}{0.041670in}}{\pgfqpoint{2.216660in}{2.216660in}}%
\pgfusepath{clip}%
\pgfsetbuttcap%
\pgfsetroundjoin%
\definecolor{currentfill}{rgb}{0.993248,0.906157,0.143936}%
\pgfsetfillcolor{currentfill}%
\pgfsetlinewidth{0.000000pt}%
\definecolor{currentstroke}{rgb}{0.000000,0.000000,0.000000}%
\pgfsetstrokecolor{currentstroke}%
\pgfsetdash{}{0pt}%
\pgfpathmoveto{\pgfqpoint{1.179955in}{1.653775in}}%
\pgfpathlineto{\pgfqpoint{1.179517in}{1.651898in}}%
\pgfpathlineto{\pgfqpoint{1.179078in}{1.649886in}}%
\pgfpathlineto{\pgfqpoint{1.178640in}{1.647740in}}%
\pgfpathlineto{\pgfqpoint{1.178201in}{1.645459in}}%
\pgfpathlineto{\pgfqpoint{1.177711in}{1.645488in}}%
\pgfpathlineto{\pgfqpoint{1.177222in}{1.645525in}}%
\pgfpathlineto{\pgfqpoint{1.176737in}{1.645568in}}%
\pgfpathlineto{\pgfqpoint{1.176255in}{1.645619in}}%
\pgfpathlineto{\pgfqpoint{1.177181in}{1.647860in}}%
\pgfpathlineto{\pgfqpoint{1.178106in}{1.649967in}}%
\pgfpathlineto{\pgfqpoint{1.179031in}{1.651938in}}%
\pgfpathlineto{\pgfqpoint{1.179955in}{1.653775in}}%
\pgfpathlineto{\pgfqpoint{1.179955in}{1.653775in}}%
\pgfpathlineto{\pgfqpoint{1.179955in}{1.653775in}}%
\pgfpathlineto{\pgfqpoint{1.179955in}{1.653775in}}%
\pgfpathlineto{\pgfqpoint{1.179955in}{1.653775in}}%
\pgfpathclose%
\pgfusepath{fill}%
\end{pgfscope}%
\begin{pgfscope}%
\pgfpathrectangle{\pgfqpoint{0.041670in}{0.041670in}}{\pgfqpoint{2.216660in}{2.216660in}}%
\pgfusepath{clip}%
\pgfsetbuttcap%
\pgfsetroundjoin%
\definecolor{currentfill}{rgb}{0.993248,0.906157,0.143936}%
\pgfsetfillcolor{currentfill}%
\pgfsetlinewidth{0.000000pt}%
\definecolor{currentstroke}{rgb}{0.000000,0.000000,0.000000}%
\pgfsetstrokecolor{currentstroke}%
\pgfsetdash{}{0pt}%
\pgfpathmoveto{\pgfqpoint{1.179955in}{1.653775in}}%
\pgfpathlineto{\pgfqpoint{1.180502in}{1.651905in}}%
\pgfpathlineto{\pgfqpoint{1.181049in}{1.649899in}}%
\pgfpathlineto{\pgfqpoint{1.181597in}{1.647759in}}%
\pgfpathlineto{\pgfqpoint{1.182145in}{1.645484in}}%
\pgfpathlineto{\pgfqpoint{1.181654in}{1.645456in}}%
\pgfpathlineto{\pgfqpoint{1.181162in}{1.645434in}}%
\pgfpathlineto{\pgfqpoint{1.180669in}{1.645420in}}%
\pgfpathlineto{\pgfqpoint{1.180175in}{1.645413in}}%
\pgfpathlineto{\pgfqpoint{1.180120in}{1.647706in}}%
\pgfpathlineto{\pgfqpoint{1.180065in}{1.649864in}}%
\pgfpathlineto{\pgfqpoint{1.180010in}{1.651887in}}%
\pgfpathlineto{\pgfqpoint{1.179955in}{1.653775in}}%
\pgfpathlineto{\pgfqpoint{1.179955in}{1.653775in}}%
\pgfpathlineto{\pgfqpoint{1.179955in}{1.653775in}}%
\pgfpathlineto{\pgfqpoint{1.179955in}{1.653775in}}%
\pgfpathlineto{\pgfqpoint{1.179955in}{1.653775in}}%
\pgfpathclose%
\pgfusepath{fill}%
\end{pgfscope}%
\begin{pgfscope}%
\pgfpathrectangle{\pgfqpoint{0.041670in}{0.041670in}}{\pgfqpoint{2.216660in}{2.216660in}}%
\pgfusepath{clip}%
\pgfsetbuttcap%
\pgfsetroundjoin%
\definecolor{currentfill}{rgb}{0.993248,0.906157,0.143936}%
\pgfsetfillcolor{currentfill}%
\pgfsetlinewidth{0.000000pt}%
\definecolor{currentstroke}{rgb}{0.000000,0.000000,0.000000}%
\pgfsetstrokecolor{currentstroke}%
\pgfsetdash{}{0pt}%
\pgfpathmoveto{\pgfqpoint{1.179955in}{1.653775in}}%
\pgfpathlineto{\pgfqpoint{1.180010in}{1.651887in}}%
\pgfpathlineto{\pgfqpoint{1.180065in}{1.649864in}}%
\pgfpathlineto{\pgfqpoint{1.180120in}{1.647706in}}%
\pgfpathlineto{\pgfqpoint{1.180175in}{1.645413in}}%
\pgfpathlineto{\pgfqpoint{1.179680in}{1.645414in}}%
\pgfpathlineto{\pgfqpoint{1.179186in}{1.645421in}}%
\pgfpathlineto{\pgfqpoint{1.178693in}{1.645436in}}%
\pgfpathlineto{\pgfqpoint{1.178201in}{1.645459in}}%
\pgfpathlineto{\pgfqpoint{1.178640in}{1.647740in}}%
\pgfpathlineto{\pgfqpoint{1.179078in}{1.649886in}}%
\pgfpathlineto{\pgfqpoint{1.179517in}{1.651898in}}%
\pgfpathlineto{\pgfqpoint{1.179955in}{1.653775in}}%
\pgfpathlineto{\pgfqpoint{1.179955in}{1.653775in}}%
\pgfpathlineto{\pgfqpoint{1.179955in}{1.653775in}}%
\pgfpathlineto{\pgfqpoint{1.179955in}{1.653775in}}%
\pgfpathlineto{\pgfqpoint{1.179955in}{1.653775in}}%
\pgfpathclose%
\pgfusepath{fill}%
\end{pgfscope}%
\begin{pgfscope}%
\pgfpathrectangle{\pgfqpoint{0.041670in}{0.041670in}}{\pgfqpoint{2.216660in}{2.216660in}}%
\pgfusepath{clip}%
\pgfsetbuttcap%
\pgfsetroundjoin%
\definecolor{currentfill}{rgb}{0.133743,0.548535,0.553541}%
\pgfsetfillcolor{currentfill}%
\pgfsetlinewidth{0.000000pt}%
\definecolor{currentstroke}{rgb}{0.000000,0.000000,0.000000}%
\pgfsetstrokecolor{currentstroke}%
\pgfsetdash{}{0pt}%
\pgfpathmoveto{\pgfqpoint{1.377553in}{1.128823in}}%
\pgfpathlineto{\pgfqpoint{1.379876in}{1.119490in}}%
\pgfpathlineto{\pgfqpoint{1.382198in}{1.110146in}}%
\pgfpathlineto{\pgfqpoint{1.384519in}{1.100794in}}%
\pgfpathlineto{\pgfqpoint{1.386839in}{1.091437in}}%
\pgfpathlineto{\pgfqpoint{1.378156in}{1.088197in}}%
\pgfpathlineto{\pgfqpoint{1.369263in}{1.085097in}}%
\pgfpathlineto{\pgfqpoint{1.360170in}{1.082139in}}%
\pgfpathlineto{\pgfqpoint{1.350885in}{1.079328in}}%
\pgfpathlineto{\pgfqpoint{1.348965in}{1.088837in}}%
\pgfpathlineto{\pgfqpoint{1.347044in}{1.098340in}}%
\pgfpathlineto{\pgfqpoint{1.345123in}{1.107835in}}%
\pgfpathlineto{\pgfqpoint{1.343201in}{1.117320in}}%
\pgfpathlineto{\pgfqpoint{1.352071in}{1.119991in}}%
\pgfpathlineto{\pgfqpoint{1.360759in}{1.122800in}}%
\pgfpathlineto{\pgfqpoint{1.369256in}{1.125745in}}%
\pgfpathlineto{\pgfqpoint{1.377553in}{1.128823in}}%
\pgfpathclose%
\pgfusepath{fill}%
\end{pgfscope}%
\begin{pgfscope}%
\pgfpathrectangle{\pgfqpoint{0.041670in}{0.041670in}}{\pgfqpoint{2.216660in}{2.216660in}}%
\pgfusepath{clip}%
\pgfsetbuttcap%
\pgfsetroundjoin%
\definecolor{currentfill}{rgb}{0.699415,0.867117,0.175971}%
\pgfsetfillcolor{currentfill}%
\pgfsetlinewidth{0.000000pt}%
\definecolor{currentstroke}{rgb}{0.000000,0.000000,0.000000}%
\pgfsetstrokecolor{currentstroke}%
\pgfsetdash{}{0pt}%
\pgfpathmoveto{\pgfqpoint{1.308504in}{1.542718in}}%
\pgfpathlineto{\pgfqpoint{1.312062in}{1.537376in}}%
\pgfpathlineto{\pgfqpoint{1.315618in}{1.531925in}}%
\pgfpathlineto{\pgfqpoint{1.319172in}{1.526366in}}%
\pgfpathlineto{\pgfqpoint{1.322725in}{1.520701in}}%
\pgfpathlineto{\pgfqpoint{1.320686in}{1.518573in}}%
\pgfpathlineto{\pgfqpoint{1.318505in}{1.516475in}}%
\pgfpathlineto{\pgfqpoint{1.316183in}{1.514412in}}%
\pgfpathlineto{\pgfqpoint{1.313724in}{1.512383in}}%
\pgfpathlineto{\pgfqpoint{1.310391in}{1.518263in}}%
\pgfpathlineto{\pgfqpoint{1.307056in}{1.524035in}}%
\pgfpathlineto{\pgfqpoint{1.303721in}{1.529700in}}%
\pgfpathlineto{\pgfqpoint{1.300384in}{1.535255in}}%
\pgfpathlineto{\pgfqpoint{1.302602in}{1.537075in}}%
\pgfpathlineto{\pgfqpoint{1.304696in}{1.538927in}}%
\pgfpathlineto{\pgfqpoint{1.306664in}{1.540809in}}%
\pgfpathlineto{\pgfqpoint{1.308504in}{1.542718in}}%
\pgfpathclose%
\pgfusepath{fill}%
\end{pgfscope}%
\begin{pgfscope}%
\pgfpathrectangle{\pgfqpoint{0.041670in}{0.041670in}}{\pgfqpoint{2.216660in}{2.216660in}}%
\pgfusepath{clip}%
\pgfsetbuttcap%
\pgfsetroundjoin%
\definecolor{currentfill}{rgb}{0.134692,0.658636,0.517649}%
\pgfsetfillcolor{currentfill}%
\pgfsetlinewidth{0.000000pt}%
\definecolor{currentstroke}{rgb}{0.000000,0.000000,0.000000}%
\pgfsetstrokecolor{currentstroke}%
\pgfsetdash{}{0pt}%
\pgfpathmoveto{\pgfqpoint{1.376307in}{1.250478in}}%
\pgfpathlineto{\pgfqpoint{1.379003in}{1.241653in}}%
\pgfpathlineto{\pgfqpoint{1.381698in}{1.232786in}}%
\pgfpathlineto{\pgfqpoint{1.384392in}{1.223880in}}%
\pgfpathlineto{\pgfqpoint{1.387084in}{1.214938in}}%
\pgfpathlineto{\pgfqpoint{1.380355in}{1.211736in}}%
\pgfpathlineto{\pgfqpoint{1.373417in}{1.208640in}}%
\pgfpathlineto{\pgfqpoint{1.366277in}{1.205654in}}%
\pgfpathlineto{\pgfqpoint{1.358941in}{1.202781in}}%
\pgfpathlineto{\pgfqpoint{1.356610in}{1.211897in}}%
\pgfpathlineto{\pgfqpoint{1.354279in}{1.220976in}}%
\pgfpathlineto{\pgfqpoint{1.351946in}{1.230016in}}%
\pgfpathlineto{\pgfqpoint{1.349612in}{1.239014in}}%
\pgfpathlineto{\pgfqpoint{1.356569in}{1.241723in}}%
\pgfpathlineto{\pgfqpoint{1.363342in}{1.244539in}}%
\pgfpathlineto{\pgfqpoint{1.369923in}{1.247459in}}%
\pgfpathlineto{\pgfqpoint{1.376307in}{1.250478in}}%
\pgfpathclose%
\pgfusepath{fill}%
\end{pgfscope}%
\begin{pgfscope}%
\pgfpathrectangle{\pgfqpoint{0.041670in}{0.041670in}}{\pgfqpoint{2.216660in}{2.216660in}}%
\pgfusepath{clip}%
\pgfsetbuttcap%
\pgfsetroundjoin%
\definecolor{currentfill}{rgb}{0.282327,0.094955,0.417331}%
\pgfsetfillcolor{currentfill}%
\pgfsetlinewidth{0.000000pt}%
\definecolor{currentstroke}{rgb}{0.000000,0.000000,0.000000}%
\pgfsetstrokecolor{currentstroke}%
\pgfsetdash{}{0pt}%
\pgfpathmoveto{\pgfqpoint{1.057847in}{0.674249in}}%
\pgfpathlineto{\pgfqpoint{1.056917in}{0.667580in}}%
\pgfpathlineto{\pgfqpoint{1.055985in}{0.661062in}}%
\pgfpathlineto{\pgfqpoint{1.055052in}{0.654700in}}%
\pgfpathlineto{\pgfqpoint{1.054118in}{0.648499in}}%
\pgfpathlineto{\pgfqpoint{1.037889in}{0.650776in}}%
\pgfpathlineto{\pgfqpoint{1.021818in}{0.653326in}}%
\pgfpathlineto{\pgfqpoint{1.005921in}{0.656148in}}%
\pgfpathlineto{\pgfqpoint{0.990217in}{0.659236in}}%
\pgfpathlineto{\pgfqpoint{0.991623in}{0.665349in}}%
\pgfpathlineto{\pgfqpoint{0.993027in}{0.671623in}}%
\pgfpathlineto{\pgfqpoint{0.994430in}{0.678052in}}%
\pgfpathlineto{\pgfqpoint{0.995832in}{0.684633in}}%
\pgfpathlineto{\pgfqpoint{1.011074in}{0.681646in}}%
\pgfpathlineto{\pgfqpoint{1.026501in}{0.678918in}}%
\pgfpathlineto{\pgfqpoint{1.042098in}{0.676451in}}%
\pgfpathlineto{\pgfqpoint{1.057847in}{0.674249in}}%
\pgfpathclose%
\pgfusepath{fill}%
\end{pgfscope}%
\begin{pgfscope}%
\pgfpathrectangle{\pgfqpoint{0.041670in}{0.041670in}}{\pgfqpoint{2.216660in}{2.216660in}}%
\pgfusepath{clip}%
\pgfsetbuttcap%
\pgfsetroundjoin%
\definecolor{currentfill}{rgb}{0.201239,0.383670,0.554294}%
\pgfsetfillcolor{currentfill}%
\pgfsetlinewidth{0.000000pt}%
\definecolor{currentstroke}{rgb}{0.000000,0.000000,0.000000}%
\pgfsetstrokecolor{currentstroke}%
\pgfsetdash{}{0pt}%
\pgfpathmoveto{\pgfqpoint{0.548491in}{0.877314in}}%
\pgfpathlineto{\pgfqpoint{0.544693in}{0.889455in}}%
\pgfpathlineto{\pgfqpoint{0.540874in}{0.902084in}}%
\pgfpathlineto{\pgfqpoint{0.537034in}{0.915210in}}%
\pgfpathlineto{\pgfqpoint{0.533173in}{0.928840in}}%
\pgfpathlineto{\pgfqpoint{0.521261in}{0.939512in}}%
\pgfpathlineto{\pgfqpoint{0.510059in}{0.950357in}}%
\pgfpathlineto{\pgfqpoint{0.499576in}{0.961361in}}%
\pgfpathlineto{\pgfqpoint{0.489820in}{0.972513in}}%
\pgfpathlineto{\pgfqpoint{0.493916in}{0.958706in}}%
\pgfpathlineto{\pgfqpoint{0.497989in}{0.945401in}}%
\pgfpathlineto{\pgfqpoint{0.502039in}{0.932590in}}%
\pgfpathlineto{\pgfqpoint{0.506068in}{0.920265in}}%
\pgfpathlineto{\pgfqpoint{0.515619in}{0.909296in}}%
\pgfpathlineto{\pgfqpoint{0.525878in}{0.898473in}}%
\pgfpathlineto{\pgfqpoint{0.536838in}{0.887808in}}%
\pgfpathlineto{\pgfqpoint{0.548491in}{0.877314in}}%
\pgfpathclose%
\pgfusepath{fill}%
\end{pgfscope}%
\begin{pgfscope}%
\pgfpathrectangle{\pgfqpoint{0.041670in}{0.041670in}}{\pgfqpoint{2.216660in}{2.216660in}}%
\pgfusepath{clip}%
\pgfsetbuttcap%
\pgfsetroundjoin%
\definecolor{currentfill}{rgb}{0.263663,0.237631,0.518762}%
\pgfsetfillcolor{currentfill}%
\pgfsetlinewidth{0.000000pt}%
\definecolor{currentstroke}{rgb}{0.000000,0.000000,0.000000}%
\pgfsetstrokecolor{currentstroke}%
\pgfsetdash{}{0pt}%
\pgfpathmoveto{\pgfqpoint{1.072635in}{0.798399in}}%
\pgfpathlineto{\pgfqpoint{1.071715in}{0.789831in}}%
\pgfpathlineto{\pgfqpoint{1.070795in}{0.781353in}}%
\pgfpathlineto{\pgfqpoint{1.069874in}{0.772970in}}%
\pgfpathlineto{\pgfqpoint{1.068953in}{0.764684in}}%
\pgfpathlineto{\pgfqpoint{1.054632in}{0.766661in}}%
\pgfpathlineto{\pgfqpoint{1.040449in}{0.768876in}}%
\pgfpathlineto{\pgfqpoint{1.026419in}{0.771327in}}%
\pgfpathlineto{\pgfqpoint{1.012556in}{0.774010in}}%
\pgfpathlineto{\pgfqpoint{1.013944in}{0.782207in}}%
\pgfpathlineto{\pgfqpoint{1.015331in}{0.790503in}}%
\pgfpathlineto{\pgfqpoint{1.016717in}{0.798893in}}%
\pgfpathlineto{\pgfqpoint{1.018103in}{0.807374in}}%
\pgfpathlineto{\pgfqpoint{1.031508in}{0.804792in}}%
\pgfpathlineto{\pgfqpoint{1.045075in}{0.802434in}}%
\pgfpathlineto{\pgfqpoint{1.058789in}{0.800302in}}%
\pgfpathlineto{\pgfqpoint{1.072635in}{0.798399in}}%
\pgfpathclose%
\pgfusepath{fill}%
\end{pgfscope}%
\begin{pgfscope}%
\pgfpathrectangle{\pgfqpoint{0.041670in}{0.041670in}}{\pgfqpoint{2.216660in}{2.216660in}}%
\pgfusepath{clip}%
\pgfsetbuttcap%
\pgfsetroundjoin%
\definecolor{currentfill}{rgb}{0.281477,0.755203,0.432552}%
\pgfsetfillcolor{currentfill}%
\pgfsetlinewidth{0.000000pt}%
\definecolor{currentstroke}{rgb}{0.000000,0.000000,0.000000}%
\pgfsetstrokecolor{currentstroke}%
\pgfsetdash{}{0pt}%
\pgfpathmoveto{\pgfqpoint{1.363468in}{1.362976in}}%
\pgfpathlineto{\pgfqpoint{1.366500in}{1.355022in}}%
\pgfpathlineto{\pgfqpoint{1.369529in}{1.347000in}}%
\pgfpathlineto{\pgfqpoint{1.372557in}{1.338911in}}%
\pgfpathlineto{\pgfqpoint{1.375584in}{1.330758in}}%
\pgfpathlineto{\pgfqpoint{1.370650in}{1.327774in}}%
\pgfpathlineto{\pgfqpoint{1.365520in}{1.324867in}}%
\pgfpathlineto{\pgfqpoint{1.360198in}{1.322039in}}%
\pgfpathlineto{\pgfqpoint{1.354690in}{1.319294in}}%
\pgfpathlineto{\pgfqpoint{1.351983in}{1.327638in}}%
\pgfpathlineto{\pgfqpoint{1.349274in}{1.335918in}}%
\pgfpathlineto{\pgfqpoint{1.346563in}{1.344130in}}%
\pgfpathlineto{\pgfqpoint{1.343852in}{1.352274in}}%
\pgfpathlineto{\pgfqpoint{1.349023in}{1.354837in}}%
\pgfpathlineto{\pgfqpoint{1.354019in}{1.357476in}}%
\pgfpathlineto{\pgfqpoint{1.358836in}{1.360190in}}%
\pgfpathlineto{\pgfqpoint{1.363468in}{1.362976in}}%
\pgfpathclose%
\pgfusepath{fill}%
\end{pgfscope}%
\begin{pgfscope}%
\pgfpathrectangle{\pgfqpoint{0.041670in}{0.041670in}}{\pgfqpoint{2.216660in}{2.216660in}}%
\pgfusepath{clip}%
\pgfsetbuttcap%
\pgfsetroundjoin%
\definecolor{currentfill}{rgb}{0.179019,0.433756,0.557430}%
\pgfsetfillcolor{currentfill}%
\pgfsetlinewidth{0.000000pt}%
\definecolor{currentstroke}{rgb}{0.000000,0.000000,0.000000}%
\pgfsetstrokecolor{currentstroke}%
\pgfsetdash{}{0pt}%
\pgfpathmoveto{\pgfqpoint{1.366230in}{1.003408in}}%
\pgfpathlineto{\pgfqpoint{1.368146in}{0.993981in}}%
\pgfpathlineto{\pgfqpoint{1.370062in}{0.984578in}}%
\pgfpathlineto{\pgfqpoint{1.371977in}{0.975203in}}%
\pgfpathlineto{\pgfqpoint{1.373893in}{0.965858in}}%
\pgfpathlineto{\pgfqpoint{1.363163in}{0.962788in}}%
\pgfpathlineto{\pgfqpoint{1.352236in}{0.959893in}}%
\pgfpathlineto{\pgfqpoint{1.341125in}{0.957177in}}%
\pgfpathlineto{\pgfqpoint{1.329841in}{0.954642in}}%
\pgfpathlineto{\pgfqpoint{1.328358in}{0.964112in}}%
\pgfpathlineto{\pgfqpoint{1.326875in}{0.973613in}}%
\pgfpathlineto{\pgfqpoint{1.325392in}{0.983140in}}%
\pgfpathlineto{\pgfqpoint{1.323909in}{0.992692in}}%
\pgfpathlineto{\pgfqpoint{1.334749in}{0.995113in}}%
\pgfpathlineto{\pgfqpoint{1.345423in}{0.997709in}}%
\pgfpathlineto{\pgfqpoint{1.355920in}{1.000475in}}%
\pgfpathlineto{\pgfqpoint{1.366230in}{1.003408in}}%
\pgfpathclose%
\pgfusepath{fill}%
\end{pgfscope}%
\begin{pgfscope}%
\pgfpathrectangle{\pgfqpoint{0.041670in}{0.041670in}}{\pgfqpoint{2.216660in}{2.216660in}}%
\pgfusepath{clip}%
\pgfsetbuttcap%
\pgfsetroundjoin%
\definecolor{currentfill}{rgb}{0.487026,0.823929,0.312321}%
\pgfsetfillcolor{currentfill}%
\pgfsetlinewidth{0.000000pt}%
\definecolor{currentstroke}{rgb}{0.000000,0.000000,0.000000}%
\pgfsetstrokecolor{currentstroke}%
\pgfsetdash{}{0pt}%
\pgfpathmoveto{\pgfqpoint{1.340339in}{1.461675in}}%
\pgfpathlineto{\pgfqpoint{1.343659in}{1.454901in}}%
\pgfpathlineto{\pgfqpoint{1.346978in}{1.448036in}}%
\pgfpathlineto{\pgfqpoint{1.350295in}{1.441082in}}%
\pgfpathlineto{\pgfqpoint{1.353610in}{1.434040in}}%
\pgfpathlineto{\pgfqpoint{1.350256in}{1.431424in}}%
\pgfpathlineto{\pgfqpoint{1.346729in}{1.428859in}}%
\pgfpathlineto{\pgfqpoint{1.343031in}{1.426348in}}%
\pgfpathlineto{\pgfqpoint{1.339166in}{1.423894in}}%
\pgfpathlineto{\pgfqpoint{1.336122in}{1.431141in}}%
\pgfpathlineto{\pgfqpoint{1.333077in}{1.438299in}}%
\pgfpathlineto{\pgfqpoint{1.330030in}{1.445368in}}%
\pgfpathlineto{\pgfqpoint{1.326982in}{1.452345in}}%
\pgfpathlineto{\pgfqpoint{1.330555in}{1.454601in}}%
\pgfpathlineto{\pgfqpoint{1.333974in}{1.456910in}}%
\pgfpathlineto{\pgfqpoint{1.337237in}{1.459269in}}%
\pgfpathlineto{\pgfqpoint{1.340339in}{1.461675in}}%
\pgfpathclose%
\pgfusepath{fill}%
\end{pgfscope}%
\begin{pgfscope}%
\pgfpathrectangle{\pgfqpoint{0.041670in}{0.041670in}}{\pgfqpoint{2.216660in}{2.216660in}}%
\pgfusepath{clip}%
\pgfsetbuttcap%
\pgfsetroundjoin%
\definecolor{currentfill}{rgb}{0.855810,0.888601,0.097452}%
\pgfsetfillcolor{currentfill}%
\pgfsetlinewidth{0.000000pt}%
\definecolor{currentstroke}{rgb}{0.000000,0.000000,0.000000}%
\pgfsetstrokecolor{currentstroke}%
\pgfsetdash{}{0pt}%
\pgfpathmoveto{\pgfqpoint{1.095281in}{1.596731in}}%
\pgfpathlineto{\pgfqpoint{1.091756in}{1.592746in}}%
\pgfpathlineto{\pgfqpoint{1.088231in}{1.588637in}}%
\pgfpathlineto{\pgfqpoint{1.084707in}{1.584407in}}%
\pgfpathlineto{\pgfqpoint{1.081183in}{1.580056in}}%
\pgfpathlineto{\pgfqpoint{1.079756in}{1.581535in}}%
\pgfpathlineto{\pgfqpoint{1.078430in}{1.583035in}}%
\pgfpathlineto{\pgfqpoint{1.077206in}{1.584554in}}%
\pgfpathlineto{\pgfqpoint{1.076085in}{1.586089in}}%
\pgfpathlineto{\pgfqpoint{1.079786in}{1.590222in}}%
\pgfpathlineto{\pgfqpoint{1.083489in}{1.594233in}}%
\pgfpathlineto{\pgfqpoint{1.087193in}{1.598124in}}%
\pgfpathlineto{\pgfqpoint{1.090898in}{1.601891in}}%
\pgfpathlineto{\pgfqpoint{1.091863in}{1.600578in}}%
\pgfpathlineto{\pgfqpoint{1.092915in}{1.599279in}}%
\pgfpathlineto{\pgfqpoint{1.094055in}{1.597997in}}%
\pgfpathlineto{\pgfqpoint{1.095281in}{1.596731in}}%
\pgfpathclose%
\pgfusepath{fill}%
\end{pgfscope}%
\begin{pgfscope}%
\pgfpathrectangle{\pgfqpoint{0.041670in}{0.041670in}}{\pgfqpoint{2.216660in}{2.216660in}}%
\pgfusepath{clip}%
\pgfsetbuttcap%
\pgfsetroundjoin%
\definecolor{currentfill}{rgb}{0.896320,0.893616,0.096335}%
\pgfsetfillcolor{currentfill}%
\pgfsetlinewidth{0.000000pt}%
\definecolor{currentstroke}{rgb}{0.000000,0.000000,0.000000}%
\pgfsetstrokecolor{currentstroke}%
\pgfsetdash{}{0pt}%
\pgfpathmoveto{\pgfqpoint{1.254836in}{1.616699in}}%
\pgfpathlineto{\pgfqpoint{1.258577in}{1.613480in}}%
\pgfpathlineto{\pgfqpoint{1.262317in}{1.610135in}}%
\pgfpathlineto{\pgfqpoint{1.266055in}{1.606664in}}%
\pgfpathlineto{\pgfqpoint{1.269793in}{1.603070in}}%
\pgfpathlineto{\pgfqpoint{1.268909in}{1.601745in}}%
\pgfpathlineto{\pgfqpoint{1.267935in}{1.600433in}}%
\pgfpathlineto{\pgfqpoint{1.266872in}{1.599136in}}%
\pgfpathlineto{\pgfqpoint{1.265723in}{1.597855in}}%
\pgfpathlineto{\pgfqpoint{1.262151in}{1.601670in}}%
\pgfpathlineto{\pgfqpoint{1.258580in}{1.605360in}}%
\pgfpathlineto{\pgfqpoint{1.255007in}{1.608924in}}%
\pgfpathlineto{\pgfqpoint{1.251434in}{1.612363in}}%
\pgfpathlineto{\pgfqpoint{1.252394in}{1.613428in}}%
\pgfpathlineto{\pgfqpoint{1.253282in}{1.614506in}}%
\pgfpathlineto{\pgfqpoint{1.254096in}{1.615597in}}%
\pgfpathlineto{\pgfqpoint{1.254836in}{1.616699in}}%
\pgfpathclose%
\pgfusepath{fill}%
\end{pgfscope}%
\begin{pgfscope}%
\pgfpathrectangle{\pgfqpoint{0.041670in}{0.041670in}}{\pgfqpoint{2.216660in}{2.216660in}}%
\pgfusepath{clip}%
\pgfsetbuttcap%
\pgfsetroundjoin%
\definecolor{currentfill}{rgb}{0.268510,0.009605,0.335427}%
\pgfsetfillcolor{currentfill}%
\pgfsetlinewidth{0.000000pt}%
\definecolor{currentstroke}{rgb}{0.000000,0.000000,0.000000}%
\pgfsetstrokecolor{currentstroke}%
\pgfsetdash{}{0pt}%
\pgfpathmoveto{\pgfqpoint{0.973184in}{0.599925in}}%
\pgfpathlineto{\pgfqpoint{0.971749in}{0.596287in}}%
\pgfpathlineto{\pgfqpoint{0.970312in}{0.592871in}}%
\pgfpathlineto{\pgfqpoint{0.968871in}{0.589681in}}%
\pgfpathlineto{\pgfqpoint{0.967427in}{0.586723in}}%
\pgfpathlineto{\pgfqpoint{0.950083in}{0.590514in}}%
\pgfpathlineto{\pgfqpoint{0.932996in}{0.594598in}}%
\pgfpathlineto{\pgfqpoint{0.916184in}{0.598970in}}%
\pgfpathlineto{\pgfqpoint{0.899665in}{0.603624in}}%
\pgfpathlineto{\pgfqpoint{0.901565in}{0.606461in}}%
\pgfpathlineto{\pgfqpoint{0.903461in}{0.609529in}}%
\pgfpathlineto{\pgfqpoint{0.905353in}{0.612823in}}%
\pgfpathlineto{\pgfqpoint{0.907241in}{0.616340in}}%
\pgfpathlineto{\pgfqpoint{0.923318in}{0.611819in}}%
\pgfpathlineto{\pgfqpoint{0.939680in}{0.607573in}}%
\pgfpathlineto{\pgfqpoint{0.956308in}{0.603607in}}%
\pgfpathlineto{\pgfqpoint{0.973184in}{0.599925in}}%
\pgfpathclose%
\pgfusepath{fill}%
\end{pgfscope}%
\begin{pgfscope}%
\pgfpathrectangle{\pgfqpoint{0.041670in}{0.041670in}}{\pgfqpoint{2.216660in}{2.216660in}}%
\pgfusepath{clip}%
\pgfsetbuttcap%
\pgfsetroundjoin%
\definecolor{currentfill}{rgb}{0.974417,0.903590,0.130215}%
\pgfsetfillcolor{currentfill}%
\pgfsetlinewidth{0.000000pt}%
\definecolor{currentstroke}{rgb}{0.000000,0.000000,0.000000}%
\pgfsetstrokecolor{currentstroke}%
\pgfsetdash{}{0pt}%
\pgfpathmoveto{\pgfqpoint{1.209909in}{1.645280in}}%
\pgfpathlineto{\pgfqpoint{1.213655in}{1.643619in}}%
\pgfpathlineto{\pgfqpoint{1.217400in}{1.641825in}}%
\pgfpathlineto{\pgfqpoint{1.221145in}{1.639900in}}%
\pgfpathlineto{\pgfqpoint{1.224890in}{1.637842in}}%
\pgfpathlineto{\pgfqpoint{1.224443in}{1.637183in}}%
\pgfpathlineto{\pgfqpoint{1.223952in}{1.636531in}}%
\pgfpathlineto{\pgfqpoint{1.223416in}{1.635886in}}%
\pgfpathlineto{\pgfqpoint{1.222837in}{1.635249in}}%
\pgfpathlineto{\pgfqpoint{1.219262in}{1.637523in}}%
\pgfpathlineto{\pgfqpoint{1.215687in}{1.639666in}}%
\pgfpathlineto{\pgfqpoint{1.212112in}{1.641676in}}%
\pgfpathlineto{\pgfqpoint{1.208537in}{1.643554in}}%
\pgfpathlineto{\pgfqpoint{1.208924in}{1.643978in}}%
\pgfpathlineto{\pgfqpoint{1.209282in}{1.644407in}}%
\pgfpathlineto{\pgfqpoint{1.209610in}{1.644841in}}%
\pgfpathlineto{\pgfqpoint{1.209909in}{1.645280in}}%
\pgfpathclose%
\pgfusepath{fill}%
\end{pgfscope}%
\begin{pgfscope}%
\pgfpathrectangle{\pgfqpoint{0.041670in}{0.041670in}}{\pgfqpoint{2.216660in}{2.216660in}}%
\pgfusepath{clip}%
\pgfsetbuttcap%
\pgfsetroundjoin%
\definecolor{currentfill}{rgb}{0.993248,0.906157,0.143936}%
\pgfsetfillcolor{currentfill}%
\pgfsetlinewidth{0.000000pt}%
\definecolor{currentstroke}{rgb}{0.000000,0.000000,0.000000}%
\pgfsetstrokecolor{currentstroke}%
\pgfsetdash{}{0pt}%
\pgfpathmoveto{\pgfqpoint{1.194241in}{1.649734in}}%
\pgfpathlineto{\pgfqpoint{1.197815in}{1.648389in}}%
\pgfpathlineto{\pgfqpoint{1.201388in}{1.646910in}}%
\pgfpathlineto{\pgfqpoint{1.204962in}{1.645299in}}%
\pgfpathlineto{\pgfqpoint{1.208537in}{1.643554in}}%
\pgfpathlineto{\pgfqpoint{1.208121in}{1.643136in}}%
\pgfpathlineto{\pgfqpoint{1.207678in}{1.642724in}}%
\pgfpathlineto{\pgfqpoint{1.207206in}{1.642319in}}%
\pgfpathlineto{\pgfqpoint{1.206708in}{1.641921in}}%
\pgfpathlineto{\pgfqpoint{1.203361in}{1.643871in}}%
\pgfpathlineto{\pgfqpoint{1.200015in}{1.645687in}}%
\pgfpathlineto{\pgfqpoint{1.196670in}{1.647370in}}%
\pgfpathlineto{\pgfqpoint{1.193325in}{1.648918in}}%
\pgfpathlineto{\pgfqpoint{1.193575in}{1.649117in}}%
\pgfpathlineto{\pgfqpoint{1.193811in}{1.649319in}}%
\pgfpathlineto{\pgfqpoint{1.194033in}{1.649525in}}%
\pgfpathlineto{\pgfqpoint{1.194241in}{1.649734in}}%
\pgfpathclose%
\pgfusepath{fill}%
\end{pgfscope}%
\begin{pgfscope}%
\pgfpathrectangle{\pgfqpoint{0.041670in}{0.041670in}}{\pgfqpoint{2.216660in}{2.216660in}}%
\pgfusepath{clip}%
\pgfsetbuttcap%
\pgfsetroundjoin%
\definecolor{currentfill}{rgb}{0.993248,0.906157,0.143936}%
\pgfsetfillcolor{currentfill}%
\pgfsetlinewidth{0.000000pt}%
\definecolor{currentstroke}{rgb}{0.000000,0.000000,0.000000}%
\pgfsetstrokecolor{currentstroke}%
\pgfsetdash{}{0pt}%
\pgfpathmoveto{\pgfqpoint{1.166817in}{1.648745in}}%
\pgfpathlineto{\pgfqpoint{1.163531in}{1.647153in}}%
\pgfpathlineto{\pgfqpoint{1.160244in}{1.645427in}}%
\pgfpathlineto{\pgfqpoint{1.156956in}{1.643567in}}%
\pgfpathlineto{\pgfqpoint{1.153668in}{1.641574in}}%
\pgfpathlineto{\pgfqpoint{1.153145in}{1.641965in}}%
\pgfpathlineto{\pgfqpoint{1.152650in}{1.642364in}}%
\pgfpathlineto{\pgfqpoint{1.152181in}{1.642770in}}%
\pgfpathlineto{\pgfqpoint{1.151741in}{1.643182in}}%
\pgfpathlineto{\pgfqpoint{1.155269in}{1.644973in}}%
\pgfpathlineto{\pgfqpoint{1.158798in}{1.646632in}}%
\pgfpathlineto{\pgfqpoint{1.162325in}{1.648157in}}%
\pgfpathlineto{\pgfqpoint{1.165853in}{1.649548in}}%
\pgfpathlineto{\pgfqpoint{1.166073in}{1.649342in}}%
\pgfpathlineto{\pgfqpoint{1.166308in}{1.649139in}}%
\pgfpathlineto{\pgfqpoint{1.166556in}{1.648940in}}%
\pgfpathlineto{\pgfqpoint{1.166817in}{1.648745in}}%
\pgfpathclose%
\pgfusepath{fill}%
\end{pgfscope}%
\begin{pgfscope}%
\pgfpathrectangle{\pgfqpoint{0.041670in}{0.041670in}}{\pgfqpoint{2.216660in}{2.216660in}}%
\pgfusepath{clip}%
\pgfsetbuttcap%
\pgfsetroundjoin%
\definecolor{currentfill}{rgb}{0.277941,0.056324,0.381191}%
\pgfsetfillcolor{currentfill}%
\pgfsetlinewidth{0.000000pt}%
\definecolor{currentstroke}{rgb}{0.000000,0.000000,0.000000}%
\pgfsetstrokecolor{currentstroke}%
\pgfsetdash{}{0pt}%
\pgfpathmoveto{\pgfqpoint{0.798975in}{0.617502in}}%
\pgfpathlineto{\pgfqpoint{0.796547in}{0.619525in}}%
\pgfpathlineto{\pgfqpoint{0.794112in}{0.621874in}}%
\pgfpathlineto{\pgfqpoint{0.791667in}{0.624555in}}%
\pgfpathlineto{\pgfqpoint{0.789215in}{0.627574in}}%
\pgfpathlineto{\pgfqpoint{0.772409in}{0.634376in}}%
\pgfpathlineto{\pgfqpoint{0.756056in}{0.641454in}}%
\pgfpathlineto{\pgfqpoint{0.740173in}{0.648799in}}%
\pgfpathlineto{\pgfqpoint{0.724775in}{0.656403in}}%
\pgfpathlineto{\pgfqpoint{0.727622in}{0.653215in}}%
\pgfpathlineto{\pgfqpoint{0.730458in}{0.650364in}}%
\pgfpathlineto{\pgfqpoint{0.733285in}{0.647845in}}%
\pgfpathlineto{\pgfqpoint{0.736102in}{0.645651in}}%
\pgfpathlineto{\pgfqpoint{0.751127in}{0.638226in}}%
\pgfpathlineto{\pgfqpoint{0.766625in}{0.631054in}}%
\pgfpathlineto{\pgfqpoint{0.782580in}{0.624143in}}%
\pgfpathlineto{\pgfqpoint{0.798975in}{0.617502in}}%
\pgfpathclose%
\pgfusepath{fill}%
\end{pgfscope}%
\begin{pgfscope}%
\pgfpathrectangle{\pgfqpoint{0.041670in}{0.041670in}}{\pgfqpoint{2.216660in}{2.216660in}}%
\pgfusepath{clip}%
\pgfsetbuttcap%
\pgfsetroundjoin%
\definecolor{currentfill}{rgb}{0.974417,0.903590,0.130215}%
\pgfsetfillcolor{currentfill}%
\pgfsetlinewidth{0.000000pt}%
\definecolor{currentstroke}{rgb}{0.000000,0.000000,0.000000}%
\pgfsetstrokecolor{currentstroke}%
\pgfsetdash{}{0pt}%
\pgfpathmoveto{\pgfqpoint{1.151741in}{1.643182in}}%
\pgfpathlineto{\pgfqpoint{1.148212in}{1.641258in}}%
\pgfpathlineto{\pgfqpoint{1.144683in}{1.639201in}}%
\pgfpathlineto{\pgfqpoint{1.141153in}{1.637012in}}%
\pgfpathlineto{\pgfqpoint{1.137624in}{1.634690in}}%
\pgfpathlineto{\pgfqpoint{1.137006in}{1.635319in}}%
\pgfpathlineto{\pgfqpoint{1.136432in}{1.635957in}}%
\pgfpathlineto{\pgfqpoint{1.135901in}{1.636603in}}%
\pgfpathlineto{\pgfqpoint{1.135415in}{1.637256in}}%
\pgfpathlineto{\pgfqpoint{1.139127in}{1.639362in}}%
\pgfpathlineto{\pgfqpoint{1.142840in}{1.641337in}}%
\pgfpathlineto{\pgfqpoint{1.146552in}{1.643180in}}%
\pgfpathlineto{\pgfqpoint{1.150265in}{1.644890in}}%
\pgfpathlineto{\pgfqpoint{1.150590in}{1.644455in}}%
\pgfpathlineto{\pgfqpoint{1.150945in}{1.644025in}}%
\pgfpathlineto{\pgfqpoint{1.151328in}{1.643601in}}%
\pgfpathlineto{\pgfqpoint{1.151741in}{1.643182in}}%
\pgfpathclose%
\pgfusepath{fill}%
\end{pgfscope}%
\begin{pgfscope}%
\pgfpathrectangle{\pgfqpoint{0.041670in}{0.041670in}}{\pgfqpoint{2.216660in}{2.216660in}}%
\pgfusepath{clip}%
\pgfsetbuttcap%
\pgfsetroundjoin%
\definecolor{currentfill}{rgb}{0.935904,0.898570,0.108131}%
\pgfsetfillcolor{currentfill}%
\pgfsetlinewidth{0.000000pt}%
\definecolor{currentstroke}{rgb}{0.000000,0.000000,0.000000}%
\pgfsetstrokecolor{currentstroke}%
\pgfsetdash{}{0pt}%
\pgfpathmoveto{\pgfqpoint{1.239867in}{1.628304in}}%
\pgfpathlineto{\pgfqpoint{1.243610in}{1.625595in}}%
\pgfpathlineto{\pgfqpoint{1.247353in}{1.622758in}}%
\pgfpathlineto{\pgfqpoint{1.251095in}{1.619792in}}%
\pgfpathlineto{\pgfqpoint{1.254836in}{1.616699in}}%
\pgfpathlineto{\pgfqpoint{1.254096in}{1.615597in}}%
\pgfpathlineto{\pgfqpoint{1.253282in}{1.614506in}}%
\pgfpathlineto{\pgfqpoint{1.252394in}{1.613428in}}%
\pgfpathlineto{\pgfqpoint{1.251434in}{1.612363in}}%
\pgfpathlineto{\pgfqpoint{1.247860in}{1.615674in}}%
\pgfpathlineto{\pgfqpoint{1.244286in}{1.618859in}}%
\pgfpathlineto{\pgfqpoint{1.240712in}{1.621914in}}%
\pgfpathlineto{\pgfqpoint{1.237137in}{1.624841in}}%
\pgfpathlineto{\pgfqpoint{1.237908in}{1.625692in}}%
\pgfpathlineto{\pgfqpoint{1.238620in}{1.626553in}}%
\pgfpathlineto{\pgfqpoint{1.239273in}{1.627424in}}%
\pgfpathlineto{\pgfqpoint{1.239867in}{1.628304in}}%
\pgfpathclose%
\pgfusepath{fill}%
\end{pgfscope}%
\begin{pgfscope}%
\pgfpathrectangle{\pgfqpoint{0.041670in}{0.041670in}}{\pgfqpoint{2.216660in}{2.216660in}}%
\pgfusepath{clip}%
\pgfsetbuttcap%
\pgfsetroundjoin%
\definecolor{currentfill}{rgb}{0.955300,0.901065,0.118128}%
\pgfsetfillcolor{currentfill}%
\pgfsetlinewidth{0.000000pt}%
\definecolor{currentstroke}{rgb}{0.000000,0.000000,0.000000}%
\pgfsetstrokecolor{currentstroke}%
\pgfsetdash{}{0pt}%
\pgfpathmoveto{\pgfqpoint{1.224890in}{1.637842in}}%
\pgfpathlineto{\pgfqpoint{1.228635in}{1.635653in}}%
\pgfpathlineto{\pgfqpoint{1.232379in}{1.633334in}}%
\pgfpathlineto{\pgfqpoint{1.236124in}{1.630884in}}%
\pgfpathlineto{\pgfqpoint{1.239867in}{1.628304in}}%
\pgfpathlineto{\pgfqpoint{1.239273in}{1.627424in}}%
\pgfpathlineto{\pgfqpoint{1.238620in}{1.626553in}}%
\pgfpathlineto{\pgfqpoint{1.237908in}{1.625692in}}%
\pgfpathlineto{\pgfqpoint{1.237137in}{1.624841in}}%
\pgfpathlineto{\pgfqpoint{1.233562in}{1.627639in}}%
\pgfpathlineto{\pgfqpoint{1.229987in}{1.630306in}}%
\pgfpathlineto{\pgfqpoint{1.226412in}{1.632843in}}%
\pgfpathlineto{\pgfqpoint{1.222837in}{1.635249in}}%
\pgfpathlineto{\pgfqpoint{1.223416in}{1.635886in}}%
\pgfpathlineto{\pgfqpoint{1.223952in}{1.636531in}}%
\pgfpathlineto{\pgfqpoint{1.224443in}{1.637183in}}%
\pgfpathlineto{\pgfqpoint{1.224890in}{1.637842in}}%
\pgfpathclose%
\pgfusepath{fill}%
\end{pgfscope}%
\begin{pgfscope}%
\pgfpathrectangle{\pgfqpoint{0.041670in}{0.041670in}}{\pgfqpoint{2.216660in}{2.216660in}}%
\pgfusepath{clip}%
\pgfsetbuttcap%
\pgfsetroundjoin%
\definecolor{currentfill}{rgb}{0.248629,0.278775,0.534556}%
\pgfsetfillcolor{currentfill}%
\pgfsetlinewidth{0.000000pt}%
\definecolor{currentstroke}{rgb}{0.000000,0.000000,0.000000}%
\pgfsetstrokecolor{currentstroke}%
\pgfsetdash{}{0pt}%
\pgfpathmoveto{\pgfqpoint{1.076312in}{0.833505in}}%
\pgfpathlineto{\pgfqpoint{1.075394in}{0.824610in}}%
\pgfpathlineto{\pgfqpoint{1.074475in}{0.815792in}}%
\pgfpathlineto{\pgfqpoint{1.073555in}{0.807054in}}%
\pgfpathlineto{\pgfqpoint{1.072635in}{0.798399in}}%
\pgfpathlineto{\pgfqpoint{1.058789in}{0.800302in}}%
\pgfpathlineto{\pgfqpoint{1.045075in}{0.802434in}}%
\pgfpathlineto{\pgfqpoint{1.031508in}{0.804792in}}%
\pgfpathlineto{\pgfqpoint{1.018103in}{0.807374in}}%
\pgfpathlineto{\pgfqpoint{1.019489in}{0.815941in}}%
\pgfpathlineto{\pgfqpoint{1.020873in}{0.824593in}}%
\pgfpathlineto{\pgfqpoint{1.022258in}{0.833324in}}%
\pgfpathlineto{\pgfqpoint{1.023642in}{0.842131in}}%
\pgfpathlineto{\pgfqpoint{1.036590in}{0.839650in}}%
\pgfpathlineto{\pgfqpoint{1.049694in}{0.837383in}}%
\pgfpathlineto{\pgfqpoint{1.062940in}{0.835334in}}%
\pgfpathlineto{\pgfqpoint{1.076312in}{0.833505in}}%
\pgfpathclose%
\pgfusepath{fill}%
\end{pgfscope}%
\begin{pgfscope}%
\pgfpathrectangle{\pgfqpoint{0.041670in}{0.041670in}}{\pgfqpoint{2.216660in}{2.216660in}}%
\pgfusepath{clip}%
\pgfsetbuttcap%
\pgfsetroundjoin%
\definecolor{currentfill}{rgb}{0.231674,0.318106,0.544834}%
\pgfsetfillcolor{currentfill}%
\pgfsetlinewidth{0.000000pt}%
\definecolor{currentstroke}{rgb}{0.000000,0.000000,0.000000}%
\pgfsetstrokecolor{currentstroke}%
\pgfsetdash{}{0pt}%
\pgfpathmoveto{\pgfqpoint{1.341702in}{0.880345in}}%
\pgfpathlineto{\pgfqpoint{1.343185in}{0.871290in}}%
\pgfpathlineto{\pgfqpoint{1.344668in}{0.862298in}}%
\pgfpathlineto{\pgfqpoint{1.346151in}{0.853372in}}%
\pgfpathlineto{\pgfqpoint{1.347635in}{0.844515in}}%
\pgfpathlineto{\pgfqpoint{1.334837in}{0.841845in}}%
\pgfpathlineto{\pgfqpoint{1.321871in}{0.839387in}}%
\pgfpathlineto{\pgfqpoint{1.308751in}{0.837144in}}%
\pgfpathlineto{\pgfqpoint{1.295490in}{0.835120in}}%
\pgfpathlineto{\pgfqpoint{1.294467in}{0.844071in}}%
\pgfpathlineto{\pgfqpoint{1.293443in}{0.853091in}}%
\pgfpathlineto{\pgfqpoint{1.292420in}{0.862178in}}%
\pgfpathlineto{\pgfqpoint{1.291397in}{0.871327in}}%
\pgfpathlineto{\pgfqpoint{1.304188in}{0.873270in}}%
\pgfpathlineto{\pgfqpoint{1.316846in}{0.875423in}}%
\pgfpathlineto{\pgfqpoint{1.329354in}{0.877782in}}%
\pgfpathlineto{\pgfqpoint{1.341702in}{0.880345in}}%
\pgfpathclose%
\pgfusepath{fill}%
\end{pgfscope}%
\begin{pgfscope}%
\pgfpathrectangle{\pgfqpoint{0.041670in}{0.041670in}}{\pgfqpoint{2.216660in}{2.216660in}}%
\pgfusepath{clip}%
\pgfsetbuttcap%
\pgfsetroundjoin%
\definecolor{currentfill}{rgb}{0.896320,0.893616,0.096335}%
\pgfsetfillcolor{currentfill}%
\pgfsetlinewidth{0.000000pt}%
\definecolor{currentstroke}{rgb}{0.000000,0.000000,0.000000}%
\pgfsetstrokecolor{currentstroke}%
\pgfsetdash{}{0pt}%
\pgfpathmoveto{\pgfqpoint{1.109390in}{1.611428in}}%
\pgfpathlineto{\pgfqpoint{1.105862in}{1.607943in}}%
\pgfpathlineto{\pgfqpoint{1.102334in}{1.604331in}}%
\pgfpathlineto{\pgfqpoint{1.098808in}{1.600593in}}%
\pgfpathlineto{\pgfqpoint{1.095281in}{1.596731in}}%
\pgfpathlineto{\pgfqpoint{1.094055in}{1.597997in}}%
\pgfpathlineto{\pgfqpoint{1.092915in}{1.599279in}}%
\pgfpathlineto{\pgfqpoint{1.091863in}{1.600578in}}%
\pgfpathlineto{\pgfqpoint{1.090898in}{1.601891in}}%
\pgfpathlineto{\pgfqpoint{1.094604in}{1.605535in}}%
\pgfpathlineto{\pgfqpoint{1.098311in}{1.609055in}}%
\pgfpathlineto{\pgfqpoint{1.102019in}{1.612450in}}%
\pgfpathlineto{\pgfqpoint{1.105727in}{1.615719in}}%
\pgfpathlineto{\pgfqpoint{1.106533in}{1.614627in}}%
\pgfpathlineto{\pgfqpoint{1.107413in}{1.613547in}}%
\pgfpathlineto{\pgfqpoint{1.108366in}{1.612480in}}%
\pgfpathlineto{\pgfqpoint{1.109390in}{1.611428in}}%
\pgfpathclose%
\pgfusepath{fill}%
\end{pgfscope}%
\begin{pgfscope}%
\pgfpathrectangle{\pgfqpoint{0.041670in}{0.041670in}}{\pgfqpoint{2.216660in}{2.216660in}}%
\pgfusepath{clip}%
\pgfsetbuttcap%
\pgfsetroundjoin%
\definecolor{currentfill}{rgb}{0.699415,0.867117,0.175971}%
\pgfsetfillcolor{currentfill}%
\pgfsetlinewidth{0.000000pt}%
\definecolor{currentstroke}{rgb}{0.000000,0.000000,0.000000}%
\pgfsetstrokecolor{currentstroke}%
\pgfsetdash{}{0pt}%
\pgfpathmoveto{\pgfqpoint{1.061599in}{1.533665in}}%
\pgfpathlineto{\pgfqpoint{1.058320in}{1.528065in}}%
\pgfpathlineto{\pgfqpoint{1.055041in}{1.522355in}}%
\pgfpathlineto{\pgfqpoint{1.051763in}{1.516536in}}%
\pgfpathlineto{\pgfqpoint{1.048486in}{1.510611in}}%
\pgfpathlineto{\pgfqpoint{1.045906in}{1.512606in}}%
\pgfpathlineto{\pgfqpoint{1.043462in}{1.514639in}}%
\pgfpathlineto{\pgfqpoint{1.041156in}{1.516707in}}%
\pgfpathlineto{\pgfqpoint{1.038990in}{1.518808in}}%
\pgfpathlineto{\pgfqpoint{1.042499in}{1.524522in}}%
\pgfpathlineto{\pgfqpoint{1.046010in}{1.530129in}}%
\pgfpathlineto{\pgfqpoint{1.049521in}{1.535629in}}%
\pgfpathlineto{\pgfqpoint{1.053035in}{1.541019in}}%
\pgfpathlineto{\pgfqpoint{1.054988in}{1.539134in}}%
\pgfpathlineto{\pgfqpoint{1.057068in}{1.537279in}}%
\pgfpathlineto{\pgfqpoint{1.059273in}{1.535456in}}%
\pgfpathlineto{\pgfqpoint{1.061599in}{1.533665in}}%
\pgfpathclose%
\pgfusepath{fill}%
\end{pgfscope}%
\begin{pgfscope}%
\pgfpathrectangle{\pgfqpoint{0.041670in}{0.041670in}}{\pgfqpoint{2.216660in}{2.216660in}}%
\pgfusepath{clip}%
\pgfsetbuttcap%
\pgfsetroundjoin%
\definecolor{currentfill}{rgb}{0.993248,0.906157,0.143936}%
\pgfsetfillcolor{currentfill}%
\pgfsetlinewidth{0.000000pt}%
\definecolor{currentstroke}{rgb}{0.000000,0.000000,0.000000}%
\pgfsetstrokecolor{currentstroke}%
\pgfsetdash{}{0pt}%
\pgfpathmoveto{\pgfqpoint{1.193325in}{1.648918in}}%
\pgfpathlineto{\pgfqpoint{1.196670in}{1.647370in}}%
\pgfpathlineto{\pgfqpoint{1.200015in}{1.645687in}}%
\pgfpathlineto{\pgfqpoint{1.203361in}{1.643871in}}%
\pgfpathlineto{\pgfqpoint{1.206708in}{1.641921in}}%
\pgfpathlineto{\pgfqpoint{1.206182in}{1.641531in}}%
\pgfpathlineto{\pgfqpoint{1.205631in}{1.641149in}}%
\pgfpathlineto{\pgfqpoint{1.205054in}{1.640775in}}%
\pgfpathlineto{\pgfqpoint{1.204451in}{1.640409in}}%
\pgfpathlineto{\pgfqpoint{1.201386in}{1.642548in}}%
\pgfpathlineto{\pgfqpoint{1.198322in}{1.644554in}}%
\pgfpathlineto{\pgfqpoint{1.195259in}{1.646426in}}%
\pgfpathlineto{\pgfqpoint{1.192196in}{1.648164in}}%
\pgfpathlineto{\pgfqpoint{1.192498in}{1.648346in}}%
\pgfpathlineto{\pgfqpoint{1.192786in}{1.648533in}}%
\pgfpathlineto{\pgfqpoint{1.193063in}{1.648724in}}%
\pgfpathlineto{\pgfqpoint{1.193325in}{1.648918in}}%
\pgfpathclose%
\pgfusepath{fill}%
\end{pgfscope}%
\begin{pgfscope}%
\pgfpathrectangle{\pgfqpoint{0.041670in}{0.041670in}}{\pgfqpoint{2.216660in}{2.216660in}}%
\pgfusepath{clip}%
\pgfsetbuttcap%
\pgfsetroundjoin%
\definecolor{currentfill}{rgb}{0.955300,0.901065,0.118128}%
\pgfsetfillcolor{currentfill}%
\pgfsetlinewidth{0.000000pt}%
\definecolor{currentstroke}{rgb}{0.000000,0.000000,0.000000}%
\pgfsetstrokecolor{currentstroke}%
\pgfsetdash{}{0pt}%
\pgfpathmoveto{\pgfqpoint{1.137624in}{1.634690in}}%
\pgfpathlineto{\pgfqpoint{1.134094in}{1.632238in}}%
\pgfpathlineto{\pgfqpoint{1.130564in}{1.629654in}}%
\pgfpathlineto{\pgfqpoint{1.127035in}{1.626940in}}%
\pgfpathlineto{\pgfqpoint{1.123505in}{1.624095in}}%
\pgfpathlineto{\pgfqpoint{1.122684in}{1.624935in}}%
\pgfpathlineto{\pgfqpoint{1.121920in}{1.625787in}}%
\pgfpathlineto{\pgfqpoint{1.121214in}{1.626649in}}%
\pgfpathlineto{\pgfqpoint{1.120568in}{1.627521in}}%
\pgfpathlineto{\pgfqpoint{1.124279in}{1.630150in}}%
\pgfpathlineto{\pgfqpoint{1.127991in}{1.632649in}}%
\pgfpathlineto{\pgfqpoint{1.131703in}{1.635018in}}%
\pgfpathlineto{\pgfqpoint{1.135415in}{1.637256in}}%
\pgfpathlineto{\pgfqpoint{1.135901in}{1.636603in}}%
\pgfpathlineto{\pgfqpoint{1.136432in}{1.635957in}}%
\pgfpathlineto{\pgfqpoint{1.137006in}{1.635319in}}%
\pgfpathlineto{\pgfqpoint{1.137624in}{1.634690in}}%
\pgfpathclose%
\pgfusepath{fill}%
\end{pgfscope}%
\begin{pgfscope}%
\pgfpathrectangle{\pgfqpoint{0.041670in}{0.041670in}}{\pgfqpoint{2.216660in}{2.216660in}}%
\pgfusepath{clip}%
\pgfsetbuttcap%
\pgfsetroundjoin%
\definecolor{currentfill}{rgb}{0.279566,0.067836,0.391917}%
\pgfsetfillcolor{currentfill}%
\pgfsetlinewidth{0.000000pt}%
\definecolor{currentstroke}{rgb}{0.000000,0.000000,0.000000}%
\pgfsetstrokecolor{currentstroke}%
\pgfsetdash{}{0pt}%
\pgfpathmoveto{\pgfqpoint{1.054118in}{0.648499in}}%
\pgfpathlineto{\pgfqpoint{1.053183in}{0.642462in}}%
\pgfpathlineto{\pgfqpoint{1.052246in}{0.636593in}}%
\pgfpathlineto{\pgfqpoint{1.051309in}{0.630897in}}%
\pgfpathlineto{\pgfqpoint{1.050370in}{0.625378in}}%
\pgfpathlineto{\pgfqpoint{1.033659in}{0.627730in}}%
\pgfpathlineto{\pgfqpoint{1.017111in}{0.630365in}}%
\pgfpathlineto{\pgfqpoint{1.000743in}{0.633279in}}%
\pgfpathlineto{\pgfqpoint{0.984574in}{0.636469in}}%
\pgfpathlineto{\pgfqpoint{0.985987in}{0.641900in}}%
\pgfpathlineto{\pgfqpoint{0.987399in}{0.647507in}}%
\pgfpathlineto{\pgfqpoint{0.988809in}{0.653288in}}%
\pgfpathlineto{\pgfqpoint{0.990217in}{0.659236in}}%
\pgfpathlineto{\pgfqpoint{1.005921in}{0.656148in}}%
\pgfpathlineto{\pgfqpoint{1.021818in}{0.653326in}}%
\pgfpathlineto{\pgfqpoint{1.037889in}{0.650776in}}%
\pgfpathlineto{\pgfqpoint{1.054118in}{0.648499in}}%
\pgfpathclose%
\pgfusepath{fill}%
\end{pgfscope}%
\begin{pgfscope}%
\pgfpathrectangle{\pgfqpoint{0.041670in}{0.041670in}}{\pgfqpoint{2.216660in}{2.216660in}}%
\pgfusepath{clip}%
\pgfsetbuttcap%
\pgfsetroundjoin%
\definecolor{currentfill}{rgb}{0.993248,0.906157,0.143936}%
\pgfsetfillcolor{currentfill}%
\pgfsetlinewidth{0.000000pt}%
\definecolor{currentstroke}{rgb}{0.000000,0.000000,0.000000}%
\pgfsetstrokecolor{currentstroke}%
\pgfsetdash{}{0pt}%
\pgfpathmoveto{\pgfqpoint{1.167992in}{1.648005in}}%
\pgfpathlineto{\pgfqpoint{1.164999in}{1.646228in}}%
\pgfpathlineto{\pgfqpoint{1.162005in}{1.644316in}}%
\pgfpathlineto{\pgfqpoint{1.159010in}{1.642271in}}%
\pgfpathlineto{\pgfqpoint{1.156015in}{1.640092in}}%
\pgfpathlineto{\pgfqpoint{1.155390in}{1.640449in}}%
\pgfpathlineto{\pgfqpoint{1.154791in}{1.640816in}}%
\pgfpathlineto{\pgfqpoint{1.154216in}{1.641191in}}%
\pgfpathlineto{\pgfqpoint{1.153668in}{1.641574in}}%
\pgfpathlineto{\pgfqpoint{1.156956in}{1.643567in}}%
\pgfpathlineto{\pgfqpoint{1.160244in}{1.645427in}}%
\pgfpathlineto{\pgfqpoint{1.163531in}{1.647153in}}%
\pgfpathlineto{\pgfqpoint{1.166817in}{1.648745in}}%
\pgfpathlineto{\pgfqpoint{1.167092in}{1.648554in}}%
\pgfpathlineto{\pgfqpoint{1.167379in}{1.648367in}}%
\pgfpathlineto{\pgfqpoint{1.167679in}{1.648184in}}%
\pgfpathlineto{\pgfqpoint{1.167992in}{1.648005in}}%
\pgfpathclose%
\pgfusepath{fill}%
\end{pgfscope}%
\begin{pgfscope}%
\pgfpathrectangle{\pgfqpoint{0.041670in}{0.041670in}}{\pgfqpoint{2.216660in}{2.216660in}}%
\pgfusepath{clip}%
\pgfsetbuttcap%
\pgfsetroundjoin%
\definecolor{currentfill}{rgb}{0.935904,0.898570,0.108131}%
\pgfsetfillcolor{currentfill}%
\pgfsetlinewidth{0.000000pt}%
\definecolor{currentstroke}{rgb}{0.000000,0.000000,0.000000}%
\pgfsetstrokecolor{currentstroke}%
\pgfsetdash{}{0pt}%
\pgfpathmoveto{\pgfqpoint{1.123505in}{1.624095in}}%
\pgfpathlineto{\pgfqpoint{1.119976in}{1.621121in}}%
\pgfpathlineto{\pgfqpoint{1.116447in}{1.618018in}}%
\pgfpathlineto{\pgfqpoint{1.112918in}{1.614787in}}%
\pgfpathlineto{\pgfqpoint{1.109390in}{1.611428in}}%
\pgfpathlineto{\pgfqpoint{1.108366in}{1.612480in}}%
\pgfpathlineto{\pgfqpoint{1.107413in}{1.613547in}}%
\pgfpathlineto{\pgfqpoint{1.106533in}{1.614627in}}%
\pgfpathlineto{\pgfqpoint{1.105727in}{1.615719in}}%
\pgfpathlineto{\pgfqpoint{1.109437in}{1.618861in}}%
\pgfpathlineto{\pgfqpoint{1.113146in}{1.621876in}}%
\pgfpathlineto{\pgfqpoint{1.116857in}{1.624763in}}%
\pgfpathlineto{\pgfqpoint{1.120568in}{1.627521in}}%
\pgfpathlineto{\pgfqpoint{1.121214in}{1.626649in}}%
\pgfpathlineto{\pgfqpoint{1.121920in}{1.625787in}}%
\pgfpathlineto{\pgfqpoint{1.122684in}{1.624935in}}%
\pgfpathlineto{\pgfqpoint{1.123505in}{1.624095in}}%
\pgfpathclose%
\pgfusepath{fill}%
\end{pgfscope}%
\begin{pgfscope}%
\pgfpathrectangle{\pgfqpoint{0.041670in}{0.041670in}}{\pgfqpoint{2.216660in}{2.216660in}}%
\pgfusepath{clip}%
\pgfsetbuttcap%
\pgfsetroundjoin%
\definecolor{currentfill}{rgb}{0.762373,0.876424,0.137064}%
\pgfsetfillcolor{currentfill}%
\pgfsetlinewidth{0.000000pt}%
\definecolor{currentstroke}{rgb}{0.000000,0.000000,0.000000}%
\pgfsetstrokecolor{currentstroke}%
\pgfsetdash{}{0pt}%
\pgfpathmoveto{\pgfqpoint{1.294261in}{1.562968in}}%
\pgfpathlineto{\pgfqpoint{1.297824in}{1.558076in}}%
\pgfpathlineto{\pgfqpoint{1.301385in}{1.553069in}}%
\pgfpathlineto{\pgfqpoint{1.304946in}{1.547950in}}%
\pgfpathlineto{\pgfqpoint{1.308504in}{1.542718in}}%
\pgfpathlineto{\pgfqpoint{1.306664in}{1.540809in}}%
\pgfpathlineto{\pgfqpoint{1.304696in}{1.538927in}}%
\pgfpathlineto{\pgfqpoint{1.302602in}{1.537075in}}%
\pgfpathlineto{\pgfqpoint{1.300384in}{1.535255in}}%
\pgfpathlineto{\pgfqpoint{1.297047in}{1.540699in}}%
\pgfpathlineto{\pgfqpoint{1.293708in}{1.546031in}}%
\pgfpathlineto{\pgfqpoint{1.290368in}{1.551249in}}%
\pgfpathlineto{\pgfqpoint{1.287028in}{1.556353in}}%
\pgfpathlineto{\pgfqpoint{1.289003in}{1.557966in}}%
\pgfpathlineto{\pgfqpoint{1.290868in}{1.559607in}}%
\pgfpathlineto{\pgfqpoint{1.292622in}{1.561275in}}%
\pgfpathlineto{\pgfqpoint{1.294261in}{1.562968in}}%
\pgfpathclose%
\pgfusepath{fill}%
\end{pgfscope}%
\begin{pgfscope}%
\pgfpathrectangle{\pgfqpoint{0.041670in}{0.041670in}}{\pgfqpoint{2.216660in}{2.216660in}}%
\pgfusepath{clip}%
\pgfsetbuttcap%
\pgfsetroundjoin%
\definecolor{currentfill}{rgb}{0.133743,0.548535,0.553541}%
\pgfsetfillcolor{currentfill}%
\pgfsetlinewidth{0.000000pt}%
\definecolor{currentstroke}{rgb}{0.000000,0.000000,0.000000}%
\pgfsetstrokecolor{currentstroke}%
\pgfsetdash{}{0pt}%
\pgfpathmoveto{\pgfqpoint{1.024738in}{1.115065in}}%
\pgfpathlineto{\pgfqpoint{1.022910in}{1.105551in}}%
\pgfpathlineto{\pgfqpoint{1.021083in}{1.096026in}}%
\pgfpathlineto{\pgfqpoint{1.019256in}{1.086492in}}%
\pgfpathlineto{\pgfqpoint{1.017429in}{1.076953in}}%
\pgfpathlineto{\pgfqpoint{1.007984in}{1.079633in}}%
\pgfpathlineto{\pgfqpoint{0.998720in}{1.082461in}}%
\pgfpathlineto{\pgfqpoint{0.989649in}{1.085434in}}%
\pgfpathlineto{\pgfqpoint{0.980779in}{1.088550in}}%
\pgfpathlineto{\pgfqpoint{0.983014in}{1.097944in}}%
\pgfpathlineto{\pgfqpoint{0.985249in}{1.107332in}}%
\pgfpathlineto{\pgfqpoint{0.987485in}{1.116712in}}%
\pgfpathlineto{\pgfqpoint{0.989722in}{1.126081in}}%
\pgfpathlineto{\pgfqpoint{0.998197in}{1.123121in}}%
\pgfpathlineto{\pgfqpoint{1.006864in}{1.120296in}}%
\pgfpathlineto{\pgfqpoint{1.015714in}{1.117610in}}%
\pgfpathlineto{\pgfqpoint{1.024738in}{1.115065in}}%
\pgfpathclose%
\pgfusepath{fill}%
\end{pgfscope}%
\begin{pgfscope}%
\pgfpathrectangle{\pgfqpoint{0.041670in}{0.041670in}}{\pgfqpoint{2.216660in}{2.216660in}}%
\pgfusepath{clip}%
\pgfsetbuttcap%
\pgfsetroundjoin%
\definecolor{currentfill}{rgb}{0.274952,0.037752,0.364543}%
\pgfsetfillcolor{currentfill}%
\pgfsetlinewidth{0.000000pt}%
\definecolor{currentstroke}{rgb}{0.000000,0.000000,0.000000}%
\pgfsetstrokecolor{currentstroke}%
\pgfsetdash{}{0pt}%
\pgfpathmoveto{\pgfqpoint{1.389527in}{0.639534in}}%
\pgfpathlineto{\pgfqpoint{1.391045in}{0.634309in}}%
\pgfpathlineto{\pgfqpoint{1.392565in}{0.629270in}}%
\pgfpathlineto{\pgfqpoint{1.394088in}{0.624421in}}%
\pgfpathlineto{\pgfqpoint{1.395613in}{0.619766in}}%
\pgfpathlineto{\pgfqpoint{1.379174in}{0.616225in}}%
\pgfpathlineto{\pgfqpoint{1.362514in}{0.612965in}}%
\pgfpathlineto{\pgfqpoint{1.345650in}{0.609990in}}%
\pgfpathlineto{\pgfqpoint{1.328603in}{0.607304in}}%
\pgfpathlineto{\pgfqpoint{1.327550in}{0.612054in}}%
\pgfpathlineto{\pgfqpoint{1.326499in}{0.616999in}}%
\pgfpathlineto{\pgfqpoint{1.325450in}{0.622134in}}%
\pgfpathlineto{\pgfqpoint{1.324402in}{0.627455in}}%
\pgfpathlineto{\pgfqpoint{1.340969in}{0.630059in}}%
\pgfpathlineto{\pgfqpoint{1.357358in}{0.632942in}}%
\pgfpathlineto{\pgfqpoint{1.373550in}{0.636101in}}%
\pgfpathlineto{\pgfqpoint{1.389527in}{0.639534in}}%
\pgfpathclose%
\pgfusepath{fill}%
\end{pgfscope}%
\begin{pgfscope}%
\pgfpathrectangle{\pgfqpoint{0.041670in}{0.041670in}}{\pgfqpoint{2.216660in}{2.216660in}}%
\pgfusepath{clip}%
\pgfsetbuttcap%
\pgfsetroundjoin%
\definecolor{currentfill}{rgb}{0.487026,0.823929,0.312321}%
\pgfsetfillcolor{currentfill}%
\pgfsetlinewidth{0.000000pt}%
\definecolor{currentstroke}{rgb}{0.000000,0.000000,0.000000}%
\pgfsetstrokecolor{currentstroke}%
\pgfsetdash{}{0pt}%
\pgfpathmoveto{\pgfqpoint{1.036230in}{1.450384in}}%
\pgfpathlineto{\pgfqpoint{1.033250in}{1.443365in}}%
\pgfpathlineto{\pgfqpoint{1.030271in}{1.436253in}}%
\pgfpathlineto{\pgfqpoint{1.027293in}{1.429052in}}%
\pgfpathlineto{\pgfqpoint{1.024316in}{1.421763in}}%
\pgfpathlineto{\pgfqpoint{1.020306in}{1.424164in}}%
\pgfpathlineto{\pgfqpoint{1.016460in}{1.426625in}}%
\pgfpathlineto{\pgfqpoint{1.012780in}{1.429141in}}%
\pgfpathlineto{\pgfqpoint{1.009272in}{1.431712in}}%
\pgfpathlineto{\pgfqpoint{1.012532in}{1.438801in}}%
\pgfpathlineto{\pgfqpoint{1.015793in}{1.445802in}}%
\pgfpathlineto{\pgfqpoint{1.019056in}{1.452713in}}%
\pgfpathlineto{\pgfqpoint{1.022320in}{1.459534in}}%
\pgfpathlineto{\pgfqpoint{1.025565in}{1.457170in}}%
\pgfpathlineto{\pgfqpoint{1.028967in}{1.454855in}}%
\pgfpathlineto{\pgfqpoint{1.032524in}{1.452593in}}%
\pgfpathlineto{\pgfqpoint{1.036230in}{1.450384in}}%
\pgfpathclose%
\pgfusepath{fill}%
\end{pgfscope}%
\begin{pgfscope}%
\pgfpathrectangle{\pgfqpoint{0.041670in}{0.041670in}}{\pgfqpoint{2.216660in}{2.216660in}}%
\pgfusepath{clip}%
\pgfsetbuttcap%
\pgfsetroundjoin%
\definecolor{currentfill}{rgb}{0.179019,0.433756,0.557430}%
\pgfsetfillcolor{currentfill}%
\pgfsetlinewidth{0.000000pt}%
\definecolor{currentstroke}{rgb}{0.000000,0.000000,0.000000}%
\pgfsetstrokecolor{currentstroke}%
\pgfsetdash{}{0pt}%
\pgfpathmoveto{\pgfqpoint{1.045766in}{0.990687in}}%
\pgfpathlineto{\pgfqpoint{1.044383in}{0.981112in}}%
\pgfpathlineto{\pgfqpoint{1.043000in}{0.971561in}}%
\pgfpathlineto{\pgfqpoint{1.041618in}{0.962037in}}%
\pgfpathlineto{\pgfqpoint{1.040235in}{0.952543in}}%
\pgfpathlineto{\pgfqpoint{1.028807in}{0.954914in}}%
\pgfpathlineto{\pgfqpoint{1.017541in}{0.957470in}}%
\pgfpathlineto{\pgfqpoint{1.006450in}{0.960206in}}%
\pgfpathlineto{\pgfqpoint{0.995545in}{0.963121in}}%
\pgfpathlineto{\pgfqpoint{0.997367in}{0.972496in}}%
\pgfpathlineto{\pgfqpoint{0.999189in}{0.981902in}}%
\pgfpathlineto{\pgfqpoint{1.001012in}{0.991335in}}%
\pgfpathlineto{\pgfqpoint{1.002834in}{1.000792in}}%
\pgfpathlineto{\pgfqpoint{1.013311in}{0.998008in}}%
\pgfpathlineto{\pgfqpoint{1.023966in}{0.995393in}}%
\pgfpathlineto{\pgfqpoint{1.034788in}{0.992952in}}%
\pgfpathlineto{\pgfqpoint{1.045766in}{0.990687in}}%
\pgfpathclose%
\pgfusepath{fill}%
\end{pgfscope}%
\begin{pgfscope}%
\pgfpathrectangle{\pgfqpoint{0.041670in}{0.041670in}}{\pgfqpoint{2.216660in}{2.216660in}}%
\pgfusepath{clip}%
\pgfsetbuttcap%
\pgfsetroundjoin%
\definecolor{currentfill}{rgb}{0.134692,0.658636,0.517649}%
\pgfsetfillcolor{currentfill}%
\pgfsetlinewidth{0.000000pt}%
\definecolor{currentstroke}{rgb}{0.000000,0.000000,0.000000}%
\pgfsetstrokecolor{currentstroke}%
\pgfsetdash{}{0pt}%
\pgfpathmoveto{\pgfqpoint{1.016631in}{1.236697in}}%
\pgfpathlineto{\pgfqpoint{1.014383in}{1.227664in}}%
\pgfpathlineto{\pgfqpoint{1.012137in}{1.218589in}}%
\pgfpathlineto{\pgfqpoint{1.009891in}{1.209475in}}%
\pgfpathlineto{\pgfqpoint{1.007647in}{1.200323in}}%
\pgfpathlineto{\pgfqpoint{1.000144in}{1.203094in}}%
\pgfpathlineto{\pgfqpoint{0.992830in}{1.205981in}}%
\pgfpathlineto{\pgfqpoint{0.985712in}{1.208979in}}%
\pgfpathlineto{\pgfqpoint{0.978796in}{1.212087in}}%
\pgfpathlineto{\pgfqpoint{0.981412in}{1.221070in}}%
\pgfpathlineto{\pgfqpoint{0.984029in}{1.230016in}}%
\pgfpathlineto{\pgfqpoint{0.986648in}{1.238924in}}%
\pgfpathlineto{\pgfqpoint{0.989267in}{1.247789in}}%
\pgfpathlineto{\pgfqpoint{0.995827in}{1.244859in}}%
\pgfpathlineto{\pgfqpoint{1.002579in}{1.242031in}}%
\pgfpathlineto{\pgfqpoint{1.009516in}{1.239310in}}%
\pgfpathlineto{\pgfqpoint{1.016631in}{1.236697in}}%
\pgfpathclose%
\pgfusepath{fill}%
\end{pgfscope}%
\begin{pgfscope}%
\pgfpathrectangle{\pgfqpoint{0.041670in}{0.041670in}}{\pgfqpoint{2.216660in}{2.216660in}}%
\pgfusepath{clip}%
\pgfsetbuttcap%
\pgfsetroundjoin%
\definecolor{currentfill}{rgb}{0.281477,0.755203,0.432552}%
\pgfsetfillcolor{currentfill}%
\pgfsetlinewidth{0.000000pt}%
\definecolor{currentstroke}{rgb}{0.000000,0.000000,0.000000}%
\pgfsetstrokecolor{currentstroke}%
\pgfsetdash{}{0pt}%
\pgfpathmoveto{\pgfqpoint{1.020796in}{1.350063in}}%
\pgfpathlineto{\pgfqpoint{1.018162in}{1.341880in}}%
\pgfpathlineto{\pgfqpoint{1.015530in}{1.333628in}}%
\pgfpathlineto{\pgfqpoint{1.012898in}{1.325309in}}%
\pgfpathlineto{\pgfqpoint{1.010267in}{1.316925in}}%
\pgfpathlineto{\pgfqpoint{1.004598in}{1.319595in}}%
\pgfpathlineto{\pgfqpoint{0.999111in}{1.322349in}}%
\pgfpathlineto{\pgfqpoint{0.993810in}{1.325186in}}%
\pgfpathlineto{\pgfqpoint{0.988702in}{1.328102in}}%
\pgfpathlineto{\pgfqpoint{0.991661in}{1.336299in}}%
\pgfpathlineto{\pgfqpoint{0.994623in}{1.344432in}}%
\pgfpathlineto{\pgfqpoint{0.997586in}{1.352499in}}%
\pgfpathlineto{\pgfqpoint{1.000550in}{1.360497in}}%
\pgfpathlineto{\pgfqpoint{1.005347in}{1.357774in}}%
\pgfpathlineto{\pgfqpoint{1.010323in}{1.355126in}}%
\pgfpathlineto{\pgfqpoint{1.015475in}{1.352555in}}%
\pgfpathlineto{\pgfqpoint{1.020796in}{1.350063in}}%
\pgfpathclose%
\pgfusepath{fill}%
\end{pgfscope}%
\begin{pgfscope}%
\pgfpathrectangle{\pgfqpoint{0.041670in}{0.041670in}}{\pgfqpoint{2.216660in}{2.216660in}}%
\pgfusepath{clip}%
\pgfsetbuttcap%
\pgfsetroundjoin%
\definecolor{currentfill}{rgb}{0.974417,0.903590,0.130215}%
\pgfsetfillcolor{currentfill}%
\pgfsetlinewidth{0.000000pt}%
\definecolor{currentstroke}{rgb}{0.000000,0.000000,0.000000}%
\pgfsetstrokecolor{currentstroke}%
\pgfsetdash{}{0pt}%
\pgfpathmoveto{\pgfqpoint{1.208537in}{1.643554in}}%
\pgfpathlineto{\pgfqpoint{1.212112in}{1.641676in}}%
\pgfpathlineto{\pgfqpoint{1.215687in}{1.639666in}}%
\pgfpathlineto{\pgfqpoint{1.219262in}{1.637523in}}%
\pgfpathlineto{\pgfqpoint{1.222837in}{1.635249in}}%
\pgfpathlineto{\pgfqpoint{1.222215in}{1.634621in}}%
\pgfpathlineto{\pgfqpoint{1.221551in}{1.634003in}}%
\pgfpathlineto{\pgfqpoint{1.220845in}{1.633394in}}%
\pgfpathlineto{\pgfqpoint{1.220098in}{1.632796in}}%
\pgfpathlineto{\pgfqpoint{1.216750in}{1.635276in}}%
\pgfpathlineto{\pgfqpoint{1.213402in}{1.637624in}}%
\pgfpathlineto{\pgfqpoint{1.210055in}{1.639839in}}%
\pgfpathlineto{\pgfqpoint{1.206708in}{1.641921in}}%
\pgfpathlineto{\pgfqpoint{1.207206in}{1.642319in}}%
\pgfpathlineto{\pgfqpoint{1.207678in}{1.642724in}}%
\pgfpathlineto{\pgfqpoint{1.208121in}{1.643136in}}%
\pgfpathlineto{\pgfqpoint{1.208537in}{1.643554in}}%
\pgfpathclose%
\pgfusepath{fill}%
\end{pgfscope}%
\begin{pgfscope}%
\pgfpathrectangle{\pgfqpoint{0.041670in}{0.041670in}}{\pgfqpoint{2.216660in}{2.216660in}}%
\pgfusepath{clip}%
\pgfsetbuttcap%
\pgfsetroundjoin%
\definecolor{currentfill}{rgb}{0.267004,0.004874,0.329415}%
\pgfsetfillcolor{currentfill}%
\pgfsetlinewidth{0.000000pt}%
\definecolor{currentstroke}{rgb}{0.000000,0.000000,0.000000}%
\pgfsetstrokecolor{currentstroke}%
\pgfsetdash{}{0pt}%
\pgfpathmoveto{\pgfqpoint{1.474668in}{0.607994in}}%
\pgfpathlineto{\pgfqpoint{1.476669in}{0.605424in}}%
\pgfpathlineto{\pgfqpoint{1.478675in}{0.603096in}}%
\pgfpathlineto{\pgfqpoint{1.480685in}{0.601013in}}%
\pgfpathlineto{\pgfqpoint{1.482699in}{0.599181in}}%
\pgfpathlineto{\pgfqpoint{1.466018in}{0.594141in}}%
\pgfpathlineto{\pgfqpoint{1.449018in}{0.589386in}}%
\pgfpathlineto{\pgfqpoint{1.431719in}{0.584922in}}%
\pgfpathlineto{\pgfqpoint{1.414139in}{0.580755in}}%
\pgfpathlineto{\pgfqpoint{1.412576in}{0.582714in}}%
\pgfpathlineto{\pgfqpoint{1.411018in}{0.584925in}}%
\pgfpathlineto{\pgfqpoint{1.409463in}{0.587381in}}%
\pgfpathlineto{\pgfqpoint{1.407912in}{0.590078in}}%
\pgfpathlineto{\pgfqpoint{1.425028in}{0.594130in}}%
\pgfpathlineto{\pgfqpoint{1.441872in}{0.598470in}}%
\pgfpathlineto{\pgfqpoint{1.458425in}{0.603093in}}%
\pgfpathlineto{\pgfqpoint{1.474668in}{0.607994in}}%
\pgfpathclose%
\pgfusepath{fill}%
\end{pgfscope}%
\begin{pgfscope}%
\pgfpathrectangle{\pgfqpoint{0.041670in}{0.041670in}}{\pgfqpoint{2.216660in}{2.216660in}}%
\pgfusepath{clip}%
\pgfsetbuttcap%
\pgfsetroundjoin%
\definecolor{currentfill}{rgb}{0.993248,0.906157,0.143936}%
\pgfsetfillcolor{currentfill}%
\pgfsetlinewidth{0.000000pt}%
\definecolor{currentstroke}{rgb}{0.000000,0.000000,0.000000}%
\pgfsetstrokecolor{currentstroke}%
\pgfsetdash{}{0pt}%
\pgfpathmoveto{\pgfqpoint{1.192196in}{1.648164in}}%
\pgfpathlineto{\pgfqpoint{1.195259in}{1.646426in}}%
\pgfpathlineto{\pgfqpoint{1.198322in}{1.644554in}}%
\pgfpathlineto{\pgfqpoint{1.201386in}{1.642548in}}%
\pgfpathlineto{\pgfqpoint{1.204451in}{1.640409in}}%
\pgfpathlineto{\pgfqpoint{1.203824in}{1.640053in}}%
\pgfpathlineto{\pgfqpoint{1.203173in}{1.639706in}}%
\pgfpathlineto{\pgfqpoint{1.202499in}{1.639369in}}%
\pgfpathlineto{\pgfqpoint{1.201803in}{1.639041in}}%
\pgfpathlineto{\pgfqpoint{1.199069in}{1.641352in}}%
\pgfpathlineto{\pgfqpoint{1.196335in}{1.643529in}}%
\pgfpathlineto{\pgfqpoint{1.193603in}{1.645572in}}%
\pgfpathlineto{\pgfqpoint{1.190871in}{1.647481in}}%
\pgfpathlineto{\pgfqpoint{1.191220in}{1.647644in}}%
\pgfpathlineto{\pgfqpoint{1.191557in}{1.647813in}}%
\pgfpathlineto{\pgfqpoint{1.191883in}{1.647986in}}%
\pgfpathlineto{\pgfqpoint{1.192196in}{1.648164in}}%
\pgfpathclose%
\pgfusepath{fill}%
\end{pgfscope}%
\begin{pgfscope}%
\pgfpathrectangle{\pgfqpoint{0.041670in}{0.041670in}}{\pgfqpoint{2.216660in}{2.216660in}}%
\pgfusepath{clip}%
\pgfsetbuttcap%
\pgfsetroundjoin%
\definecolor{currentfill}{rgb}{0.993248,0.906157,0.143936}%
\pgfsetfillcolor{currentfill}%
\pgfsetlinewidth{0.000000pt}%
\definecolor{currentstroke}{rgb}{0.000000,0.000000,0.000000}%
\pgfsetstrokecolor{currentstroke}%
\pgfsetdash{}{0pt}%
\pgfpathmoveto{\pgfqpoint{1.169357in}{1.647340in}}%
\pgfpathlineto{\pgfqpoint{1.166705in}{1.645396in}}%
\pgfpathlineto{\pgfqpoint{1.164053in}{1.643318in}}%
\pgfpathlineto{\pgfqpoint{1.161399in}{1.641105in}}%
\pgfpathlineto{\pgfqpoint{1.158745in}{1.638759in}}%
\pgfpathlineto{\pgfqpoint{1.158029in}{1.639077in}}%
\pgfpathlineto{\pgfqpoint{1.157335in}{1.639406in}}%
\pgfpathlineto{\pgfqpoint{1.156663in}{1.639744in}}%
\pgfpathlineto{\pgfqpoint{1.156015in}{1.640092in}}%
\pgfpathlineto{\pgfqpoint{1.159010in}{1.642271in}}%
\pgfpathlineto{\pgfqpoint{1.162005in}{1.644316in}}%
\pgfpathlineto{\pgfqpoint{1.164999in}{1.646228in}}%
\pgfpathlineto{\pgfqpoint{1.167992in}{1.648005in}}%
\pgfpathlineto{\pgfqpoint{1.168316in}{1.647832in}}%
\pgfpathlineto{\pgfqpoint{1.168652in}{1.647663in}}%
\pgfpathlineto{\pgfqpoint{1.168999in}{1.647499in}}%
\pgfpathlineto{\pgfqpoint{1.169357in}{1.647340in}}%
\pgfpathclose%
\pgfusepath{fill}%
\end{pgfscope}%
\begin{pgfscope}%
\pgfpathrectangle{\pgfqpoint{0.041670in}{0.041670in}}{\pgfqpoint{2.216660in}{2.216660in}}%
\pgfusepath{clip}%
\pgfsetbuttcap%
\pgfsetroundjoin%
\definecolor{currentfill}{rgb}{0.565498,0.842430,0.262877}%
\pgfsetfillcolor{currentfill}%
\pgfsetlinewidth{0.000000pt}%
\definecolor{currentstroke}{rgb}{0.000000,0.000000,0.000000}%
\pgfsetstrokecolor{currentstroke}%
\pgfsetdash{}{0pt}%
\pgfpathmoveto{\pgfqpoint{1.327043in}{1.487824in}}%
\pgfpathlineto{\pgfqpoint{1.330369in}{1.481432in}}%
\pgfpathlineto{\pgfqpoint{1.333694in}{1.474942in}}%
\pgfpathlineto{\pgfqpoint{1.337017in}{1.468356in}}%
\pgfpathlineto{\pgfqpoint{1.340339in}{1.461675in}}%
\pgfpathlineto{\pgfqpoint{1.337237in}{1.459269in}}%
\pgfpathlineto{\pgfqpoint{1.333974in}{1.456910in}}%
\pgfpathlineto{\pgfqpoint{1.330555in}{1.454601in}}%
\pgfpathlineto{\pgfqpoint{1.326982in}{1.452345in}}%
\pgfpathlineto{\pgfqpoint{1.323932in}{1.459228in}}%
\pgfpathlineto{\pgfqpoint{1.320882in}{1.466017in}}%
\pgfpathlineto{\pgfqpoint{1.317830in}{1.472709in}}%
\pgfpathlineto{\pgfqpoint{1.314777in}{1.479302in}}%
\pgfpathlineto{\pgfqpoint{1.318058in}{1.481363in}}%
\pgfpathlineto{\pgfqpoint{1.321197in}{1.483472in}}%
\pgfpathlineto{\pgfqpoint{1.324193in}{1.485626in}}%
\pgfpathlineto{\pgfqpoint{1.327043in}{1.487824in}}%
\pgfpathclose%
\pgfusepath{fill}%
\end{pgfscope}%
\begin{pgfscope}%
\pgfpathrectangle{\pgfqpoint{0.041670in}{0.041670in}}{\pgfqpoint{2.216660in}{2.216660in}}%
\pgfusepath{clip}%
\pgfsetbuttcap%
\pgfsetroundjoin%
\definecolor{currentfill}{rgb}{0.974417,0.903590,0.130215}%
\pgfsetfillcolor{currentfill}%
\pgfsetlinewidth{0.000000pt}%
\definecolor{currentstroke}{rgb}{0.000000,0.000000,0.000000}%
\pgfsetstrokecolor{currentstroke}%
\pgfsetdash{}{0pt}%
\pgfpathmoveto{\pgfqpoint{1.153668in}{1.641574in}}%
\pgfpathlineto{\pgfqpoint{1.150379in}{1.639448in}}%
\pgfpathlineto{\pgfqpoint{1.147089in}{1.637189in}}%
\pgfpathlineto{\pgfqpoint{1.143800in}{1.634798in}}%
\pgfpathlineto{\pgfqpoint{1.140510in}{1.632274in}}%
\pgfpathlineto{\pgfqpoint{1.139727in}{1.632862in}}%
\pgfpathlineto{\pgfqpoint{1.138985in}{1.633461in}}%
\pgfpathlineto{\pgfqpoint{1.138283in}{1.634071in}}%
\pgfpathlineto{\pgfqpoint{1.137624in}{1.634690in}}%
\pgfpathlineto{\pgfqpoint{1.141153in}{1.637012in}}%
\pgfpathlineto{\pgfqpoint{1.144683in}{1.639201in}}%
\pgfpathlineto{\pgfqpoint{1.148212in}{1.641258in}}%
\pgfpathlineto{\pgfqpoint{1.151741in}{1.643182in}}%
\pgfpathlineto{\pgfqpoint{1.152181in}{1.642770in}}%
\pgfpathlineto{\pgfqpoint{1.152650in}{1.642364in}}%
\pgfpathlineto{\pgfqpoint{1.153145in}{1.641965in}}%
\pgfpathlineto{\pgfqpoint{1.153668in}{1.641574in}}%
\pgfpathclose%
\pgfusepath{fill}%
\end{pgfscope}%
\begin{pgfscope}%
\pgfpathrectangle{\pgfqpoint{0.041670in}{0.041670in}}{\pgfqpoint{2.216660in}{2.216660in}}%
\pgfusepath{clip}%
\pgfsetbuttcap%
\pgfsetroundjoin%
\definecolor{currentfill}{rgb}{0.163625,0.471133,0.558148}%
\pgfsetfillcolor{currentfill}%
\pgfsetlinewidth{0.000000pt}%
\definecolor{currentstroke}{rgb}{0.000000,0.000000,0.000000}%
\pgfsetstrokecolor{currentstroke}%
\pgfsetdash{}{0pt}%
\pgfpathmoveto{\pgfqpoint{1.358561in}{1.041295in}}%
\pgfpathlineto{\pgfqpoint{1.360479in}{1.031802in}}%
\pgfpathlineto{\pgfqpoint{1.362396in}{1.022321in}}%
\pgfpathlineto{\pgfqpoint{1.364313in}{1.012855in}}%
\pgfpathlineto{\pgfqpoint{1.366230in}{1.003408in}}%
\pgfpathlineto{\pgfqpoint{1.355920in}{1.000475in}}%
\pgfpathlineto{\pgfqpoint{1.345423in}{0.997709in}}%
\pgfpathlineto{\pgfqpoint{1.334749in}{0.995113in}}%
\pgfpathlineto{\pgfqpoint{1.323909in}{0.992692in}}%
\pgfpathlineto{\pgfqpoint{1.322426in}{1.002264in}}%
\pgfpathlineto{\pgfqpoint{1.320942in}{1.011854in}}%
\pgfpathlineto{\pgfqpoint{1.319458in}{1.021459in}}%
\pgfpathlineto{\pgfqpoint{1.317974in}{1.031076in}}%
\pgfpathlineto{\pgfqpoint{1.328369in}{1.033385in}}%
\pgfpathlineto{\pgfqpoint{1.338606in}{1.035860in}}%
\pgfpathlineto{\pgfqpoint{1.348673in}{1.038498in}}%
\pgfpathlineto{\pgfqpoint{1.358561in}{1.041295in}}%
\pgfpathclose%
\pgfusepath{fill}%
\end{pgfscope}%
\begin{pgfscope}%
\pgfpathrectangle{\pgfqpoint{0.041670in}{0.041670in}}{\pgfqpoint{2.216660in}{2.216660in}}%
\pgfusepath{clip}%
\pgfsetbuttcap%
\pgfsetroundjoin%
\definecolor{currentfill}{rgb}{0.231674,0.318106,0.544834}%
\pgfsetfillcolor{currentfill}%
\pgfsetlinewidth{0.000000pt}%
\definecolor{currentstroke}{rgb}{0.000000,0.000000,0.000000}%
\pgfsetstrokecolor{currentstroke}%
\pgfsetdash{}{0pt}%
\pgfpathmoveto{\pgfqpoint{1.079985in}{0.869777in}}%
\pgfpathlineto{\pgfqpoint{1.079067in}{0.860612in}}%
\pgfpathlineto{\pgfqpoint{1.078149in}{0.851509in}}%
\pgfpathlineto{\pgfqpoint{1.077231in}{0.842472in}}%
\pgfpathlineto{\pgfqpoint{1.076312in}{0.833505in}}%
\pgfpathlineto{\pgfqpoint{1.062940in}{0.835334in}}%
\pgfpathlineto{\pgfqpoint{1.049694in}{0.837383in}}%
\pgfpathlineto{\pgfqpoint{1.036590in}{0.839650in}}%
\pgfpathlineto{\pgfqpoint{1.023642in}{0.842131in}}%
\pgfpathlineto{\pgfqpoint{1.025026in}{0.851012in}}%
\pgfpathlineto{\pgfqpoint{1.026409in}{0.859962in}}%
\pgfpathlineto{\pgfqpoint{1.027792in}{0.868978in}}%
\pgfpathlineto{\pgfqpoint{1.029175in}{0.878056in}}%
\pgfpathlineto{\pgfqpoint{1.041667in}{0.875674in}}%
\pgfpathlineto{\pgfqpoint{1.054308in}{0.873499in}}%
\pgfpathlineto{\pgfqpoint{1.067085in}{0.871532in}}%
\pgfpathlineto{\pgfqpoint{1.079985in}{0.869777in}}%
\pgfpathclose%
\pgfusepath{fill}%
\end{pgfscope}%
\begin{pgfscope}%
\pgfpathrectangle{\pgfqpoint{0.041670in}{0.041670in}}{\pgfqpoint{2.216660in}{2.216660in}}%
\pgfusepath{clip}%
\pgfsetbuttcap%
\pgfsetroundjoin%
\definecolor{currentfill}{rgb}{0.122606,0.585371,0.546557}%
\pgfsetfillcolor{currentfill}%
\pgfsetlinewidth{0.000000pt}%
\definecolor{currentstroke}{rgb}{0.000000,0.000000,0.000000}%
\pgfsetstrokecolor{currentstroke}%
\pgfsetdash{}{0pt}%
\pgfpathmoveto{\pgfqpoint{1.368255in}{1.165995in}}%
\pgfpathlineto{\pgfqpoint{1.370581in}{1.156732in}}%
\pgfpathlineto{\pgfqpoint{1.372906in}{1.147447in}}%
\pgfpathlineto{\pgfqpoint{1.375230in}{1.138143in}}%
\pgfpathlineto{\pgfqpoint{1.377553in}{1.128823in}}%
\pgfpathlineto{\pgfqpoint{1.369256in}{1.125745in}}%
\pgfpathlineto{\pgfqpoint{1.360759in}{1.122800in}}%
\pgfpathlineto{\pgfqpoint{1.352071in}{1.119991in}}%
\pgfpathlineto{\pgfqpoint{1.343201in}{1.117320in}}%
\pgfpathlineto{\pgfqpoint{1.341279in}{1.126790in}}%
\pgfpathlineto{\pgfqpoint{1.339355in}{1.136245in}}%
\pgfpathlineto{\pgfqpoint{1.337431in}{1.145680in}}%
\pgfpathlineto{\pgfqpoint{1.335507in}{1.155093in}}%
\pgfpathlineto{\pgfqpoint{1.343962in}{1.157624in}}%
\pgfpathlineto{\pgfqpoint{1.352243in}{1.160287in}}%
\pgfpathlineto{\pgfqpoint{1.360344in}{1.163078in}}%
\pgfpathlineto{\pgfqpoint{1.368255in}{1.165995in}}%
\pgfpathclose%
\pgfusepath{fill}%
\end{pgfscope}%
\begin{pgfscope}%
\pgfpathrectangle{\pgfqpoint{0.041670in}{0.041670in}}{\pgfqpoint{2.216660in}{2.216660in}}%
\pgfusepath{clip}%
\pgfsetbuttcap%
\pgfsetroundjoin%
\definecolor{currentfill}{rgb}{0.268510,0.009605,0.335427}%
\pgfsetfillcolor{currentfill}%
\pgfsetlinewidth{0.000000pt}%
\definecolor{currentstroke}{rgb}{0.000000,0.000000,0.000000}%
\pgfsetstrokecolor{currentstroke}%
\pgfsetdash{}{0pt}%
\pgfpathmoveto{\pgfqpoint{0.884306in}{0.589847in}}%
\pgfpathlineto{\pgfqpoint{0.882364in}{0.589315in}}%
\pgfpathlineto{\pgfqpoint{0.880416in}{0.589065in}}%
\pgfpathlineto{\pgfqpoint{0.878463in}{0.589103in}}%
\pgfpathlineto{\pgfqpoint{0.876505in}{0.589433in}}%
\pgfpathlineto{\pgfqpoint{0.858975in}{0.594781in}}%
\pgfpathlineto{\pgfqpoint{0.841803in}{0.600423in}}%
\pgfpathlineto{\pgfqpoint{0.825007in}{0.606350in}}%
\pgfpathlineto{\pgfqpoint{0.808607in}{0.612556in}}%
\pgfpathlineto{\pgfqpoint{0.810996in}{0.612078in}}%
\pgfpathlineto{\pgfqpoint{0.813379in}{0.611893in}}%
\pgfpathlineto{\pgfqpoint{0.815755in}{0.611994in}}%
\pgfpathlineto{\pgfqpoint{0.818124in}{0.612376in}}%
\pgfpathlineto{\pgfqpoint{0.834112in}{0.606329in}}%
\pgfpathlineto{\pgfqpoint{0.850483in}{0.600554in}}%
\pgfpathlineto{\pgfqpoint{0.867220in}{0.595057in}}%
\pgfpathlineto{\pgfqpoint{0.884306in}{0.589847in}}%
\pgfpathclose%
\pgfusepath{fill}%
\end{pgfscope}%
\begin{pgfscope}%
\pgfpathrectangle{\pgfqpoint{0.041670in}{0.041670in}}{\pgfqpoint{2.216660in}{2.216660in}}%
\pgfusepath{clip}%
\pgfsetbuttcap%
\pgfsetroundjoin%
\definecolor{currentfill}{rgb}{0.344074,0.780029,0.397381}%
\pgfsetfillcolor{currentfill}%
\pgfsetlinewidth{0.000000pt}%
\definecolor{currentstroke}{rgb}{0.000000,0.000000,0.000000}%
\pgfsetstrokecolor{currentstroke}%
\pgfsetdash{}{0pt}%
\pgfpathmoveto{\pgfqpoint{1.351329in}{1.394065in}}%
\pgfpathlineto{\pgfqpoint{1.354366in}{1.386406in}}%
\pgfpathlineto{\pgfqpoint{1.357402in}{1.378670in}}%
\pgfpathlineto{\pgfqpoint{1.360436in}{1.370859in}}%
\pgfpathlineto{\pgfqpoint{1.363468in}{1.362976in}}%
\pgfpathlineto{\pgfqpoint{1.358836in}{1.360190in}}%
\pgfpathlineto{\pgfqpoint{1.354019in}{1.357476in}}%
\pgfpathlineto{\pgfqpoint{1.349023in}{1.354837in}}%
\pgfpathlineto{\pgfqpoint{1.343852in}{1.352274in}}%
\pgfpathlineto{\pgfqpoint{1.341139in}{1.360347in}}%
\pgfpathlineto{\pgfqpoint{1.338425in}{1.368346in}}%
\pgfpathlineto{\pgfqpoint{1.335710in}{1.376270in}}%
\pgfpathlineto{\pgfqpoint{1.332994in}{1.384118in}}%
\pgfpathlineto{\pgfqpoint{1.337826in}{1.386499in}}%
\pgfpathlineto{\pgfqpoint{1.342496in}{1.388953in}}%
\pgfpathlineto{\pgfqpoint{1.346998in}{1.391475in}}%
\pgfpathlineto{\pgfqpoint{1.351329in}{1.394065in}}%
\pgfpathclose%
\pgfusepath{fill}%
\end{pgfscope}%
\begin{pgfscope}%
\pgfpathrectangle{\pgfqpoint{0.041670in}{0.041670in}}{\pgfqpoint{2.216660in}{2.216660in}}%
\pgfusepath{clip}%
\pgfsetbuttcap%
\pgfsetroundjoin%
\definecolor{currentfill}{rgb}{0.166383,0.690856,0.496502}%
\pgfsetfillcolor{currentfill}%
\pgfsetlinewidth{0.000000pt}%
\definecolor{currentstroke}{rgb}{0.000000,0.000000,0.000000}%
\pgfsetstrokecolor{currentstroke}%
\pgfsetdash{}{0pt}%
\pgfpathmoveto{\pgfqpoint{1.365509in}{1.285314in}}%
\pgfpathlineto{\pgfqpoint{1.368210in}{1.276679in}}%
\pgfpathlineto{\pgfqpoint{1.370910in}{1.267993in}}%
\pgfpathlineto{\pgfqpoint{1.373609in}{1.259259in}}%
\pgfpathlineto{\pgfqpoint{1.376307in}{1.250478in}}%
\pgfpathlineto{\pgfqpoint{1.369923in}{1.247459in}}%
\pgfpathlineto{\pgfqpoint{1.363342in}{1.244539in}}%
\pgfpathlineto{\pgfqpoint{1.356569in}{1.241723in}}%
\pgfpathlineto{\pgfqpoint{1.349612in}{1.239014in}}%
\pgfpathlineto{\pgfqpoint{1.347278in}{1.247967in}}%
\pgfpathlineto{\pgfqpoint{1.344942in}{1.256874in}}%
\pgfpathlineto{\pgfqpoint{1.342605in}{1.265731in}}%
\pgfpathlineto{\pgfqpoint{1.340268in}{1.274536in}}%
\pgfpathlineto{\pgfqpoint{1.346845in}{1.277084in}}%
\pgfpathlineto{\pgfqpoint{1.353249in}{1.279731in}}%
\pgfpathlineto{\pgfqpoint{1.359472in}{1.282475in}}%
\pgfpathlineto{\pgfqpoint{1.365509in}{1.285314in}}%
\pgfpathclose%
\pgfusepath{fill}%
\end{pgfscope}%
\begin{pgfscope}%
\pgfpathrectangle{\pgfqpoint{0.041670in}{0.041670in}}{\pgfqpoint{2.216660in}{2.216660in}}%
\pgfusepath{clip}%
\pgfsetbuttcap%
\pgfsetroundjoin%
\definecolor{currentfill}{rgb}{0.762373,0.876424,0.137064}%
\pgfsetfillcolor{currentfill}%
\pgfsetlinewidth{0.000000pt}%
\definecolor{currentstroke}{rgb}{0.000000,0.000000,0.000000}%
\pgfsetstrokecolor{currentstroke}%
\pgfsetdash{}{0pt}%
\pgfpathmoveto{\pgfqpoint{1.074729in}{1.554944in}}%
\pgfpathlineto{\pgfqpoint{1.071445in}{1.549795in}}%
\pgfpathlineto{\pgfqpoint{1.068162in}{1.544532in}}%
\pgfpathlineto{\pgfqpoint{1.064880in}{1.539155in}}%
\pgfpathlineto{\pgfqpoint{1.061599in}{1.533665in}}%
\pgfpathlineto{\pgfqpoint{1.059273in}{1.535456in}}%
\pgfpathlineto{\pgfqpoint{1.057068in}{1.537279in}}%
\pgfpathlineto{\pgfqpoint{1.054988in}{1.539134in}}%
\pgfpathlineto{\pgfqpoint{1.053035in}{1.541019in}}%
\pgfpathlineto{\pgfqpoint{1.056549in}{1.546299in}}%
\pgfpathlineto{\pgfqpoint{1.060065in}{1.551467in}}%
\pgfpathlineto{\pgfqpoint{1.063582in}{1.556522in}}%
\pgfpathlineto{\pgfqpoint{1.067100in}{1.561462in}}%
\pgfpathlineto{\pgfqpoint{1.068841in}{1.559791in}}%
\pgfpathlineto{\pgfqpoint{1.070694in}{1.558147in}}%
\pgfpathlineto{\pgfqpoint{1.072657in}{1.556530in}}%
\pgfpathlineto{\pgfqpoint{1.074729in}{1.554944in}}%
\pgfpathclose%
\pgfusepath{fill}%
\end{pgfscope}%
\begin{pgfscope}%
\pgfpathrectangle{\pgfqpoint{0.041670in}{0.041670in}}{\pgfqpoint{2.216660in}{2.216660in}}%
\pgfusepath{clip}%
\pgfsetbuttcap%
\pgfsetroundjoin%
\definecolor{currentfill}{rgb}{0.814576,0.883393,0.110347}%
\pgfsetfillcolor{currentfill}%
\pgfsetlinewidth{0.000000pt}%
\definecolor{currentstroke}{rgb}{0.000000,0.000000,0.000000}%
\pgfsetstrokecolor{currentstroke}%
\pgfsetdash{}{0pt}%
\pgfpathmoveto{\pgfqpoint{1.280000in}{1.581370in}}%
\pgfpathlineto{\pgfqpoint{1.283567in}{1.576947in}}%
\pgfpathlineto{\pgfqpoint{1.287133in}{1.572404in}}%
\pgfpathlineto{\pgfqpoint{1.290698in}{1.567744in}}%
\pgfpathlineto{\pgfqpoint{1.294261in}{1.562968in}}%
\pgfpathlineto{\pgfqpoint{1.292622in}{1.561275in}}%
\pgfpathlineto{\pgfqpoint{1.290868in}{1.559607in}}%
\pgfpathlineto{\pgfqpoint{1.289003in}{1.557966in}}%
\pgfpathlineto{\pgfqpoint{1.287028in}{1.556353in}}%
\pgfpathlineto{\pgfqpoint{1.283686in}{1.561340in}}%
\pgfpathlineto{\pgfqpoint{1.280344in}{1.566211in}}%
\pgfpathlineto{\pgfqpoint{1.277001in}{1.570964in}}%
\pgfpathlineto{\pgfqpoint{1.273657in}{1.575597in}}%
\pgfpathlineto{\pgfqpoint{1.275388in}{1.577005in}}%
\pgfpathlineto{\pgfqpoint{1.277024in}{1.578437in}}%
\pgfpathlineto{\pgfqpoint{1.278561in}{1.579893in}}%
\pgfpathlineto{\pgfqpoint{1.280000in}{1.581370in}}%
\pgfpathclose%
\pgfusepath{fill}%
\end{pgfscope}%
\begin{pgfscope}%
\pgfpathrectangle{\pgfqpoint{0.041670in}{0.041670in}}{\pgfqpoint{2.216660in}{2.216660in}}%
\pgfusepath{clip}%
\pgfsetbuttcap%
\pgfsetroundjoin%
\definecolor{currentfill}{rgb}{0.212395,0.359683,0.551710}%
\pgfsetfillcolor{currentfill}%
\pgfsetlinewidth{0.000000pt}%
\definecolor{currentstroke}{rgb}{0.000000,0.000000,0.000000}%
\pgfsetstrokecolor{currentstroke}%
\pgfsetdash{}{0pt}%
\pgfpathmoveto{\pgfqpoint{1.335771in}{0.917124in}}%
\pgfpathlineto{\pgfqpoint{1.337253in}{0.907852in}}%
\pgfpathlineto{\pgfqpoint{1.338736in}{0.898629in}}%
\pgfpathlineto{\pgfqpoint{1.340219in}{0.889459in}}%
\pgfpathlineto{\pgfqpoint{1.341702in}{0.880345in}}%
\pgfpathlineto{\pgfqpoint{1.329354in}{0.877782in}}%
\pgfpathlineto{\pgfqpoint{1.316846in}{0.875423in}}%
\pgfpathlineto{\pgfqpoint{1.304188in}{0.873270in}}%
\pgfpathlineto{\pgfqpoint{1.291397in}{0.871327in}}%
\pgfpathlineto{\pgfqpoint{1.290374in}{0.880535in}}%
\pgfpathlineto{\pgfqpoint{1.289351in}{0.889799in}}%
\pgfpathlineto{\pgfqpoint{1.288328in}{0.899116in}}%
\pgfpathlineto{\pgfqpoint{1.287306in}{0.908482in}}%
\pgfpathlineto{\pgfqpoint{1.299629in}{0.910344in}}%
\pgfpathlineto{\pgfqpoint{1.311823in}{0.912407in}}%
\pgfpathlineto{\pgfqpoint{1.323874in}{0.914668in}}%
\pgfpathlineto{\pgfqpoint{1.335771in}{0.917124in}}%
\pgfpathclose%
\pgfusepath{fill}%
\end{pgfscope}%
\begin{pgfscope}%
\pgfpathrectangle{\pgfqpoint{0.041670in}{0.041670in}}{\pgfqpoint{2.216660in}{2.216660in}}%
\pgfusepath{clip}%
\pgfsetbuttcap%
\pgfsetroundjoin%
\definecolor{currentfill}{rgb}{0.993248,0.906157,0.143936}%
\pgfsetfillcolor{currentfill}%
\pgfsetlinewidth{0.000000pt}%
\definecolor{currentstroke}{rgb}{0.000000,0.000000,0.000000}%
\pgfsetstrokecolor{currentstroke}%
\pgfsetdash{}{0pt}%
\pgfpathmoveto{\pgfqpoint{1.190871in}{1.647481in}}%
\pgfpathlineto{\pgfqpoint{1.193603in}{1.645572in}}%
\pgfpathlineto{\pgfqpoint{1.196335in}{1.643529in}}%
\pgfpathlineto{\pgfqpoint{1.199069in}{1.641352in}}%
\pgfpathlineto{\pgfqpoint{1.201803in}{1.639041in}}%
\pgfpathlineto{\pgfqpoint{1.201084in}{1.638725in}}%
\pgfpathlineto{\pgfqpoint{1.200344in}{1.638419in}}%
\pgfpathlineto{\pgfqpoint{1.199584in}{1.638124in}}%
\pgfpathlineto{\pgfqpoint{1.198804in}{1.637840in}}%
\pgfpathlineto{\pgfqpoint{1.196445in}{1.640301in}}%
\pgfpathlineto{\pgfqpoint{1.194087in}{1.642629in}}%
\pgfpathlineto{\pgfqpoint{1.191729in}{1.644822in}}%
\pgfpathlineto{\pgfqpoint{1.189372in}{1.646881in}}%
\pgfpathlineto{\pgfqpoint{1.189762in}{1.647023in}}%
\pgfpathlineto{\pgfqpoint{1.190142in}{1.647170in}}%
\pgfpathlineto{\pgfqpoint{1.190512in}{1.647323in}}%
\pgfpathlineto{\pgfqpoint{1.190871in}{1.647481in}}%
\pgfpathclose%
\pgfusepath{fill}%
\end{pgfscope}%
\begin{pgfscope}%
\pgfpathrectangle{\pgfqpoint{0.041670in}{0.041670in}}{\pgfqpoint{2.216660in}{2.216660in}}%
\pgfusepath{clip}%
\pgfsetbuttcap%
\pgfsetroundjoin%
\definecolor{currentfill}{rgb}{0.993248,0.906157,0.143936}%
\pgfsetfillcolor{currentfill}%
\pgfsetlinewidth{0.000000pt}%
\definecolor{currentstroke}{rgb}{0.000000,0.000000,0.000000}%
\pgfsetstrokecolor{currentstroke}%
\pgfsetdash{}{0pt}%
\pgfpathmoveto{\pgfqpoint{1.170892in}{1.646761in}}%
\pgfpathlineto{\pgfqpoint{1.168624in}{1.644671in}}%
\pgfpathlineto{\pgfqpoint{1.166355in}{1.642447in}}%
\pgfpathlineto{\pgfqpoint{1.164085in}{1.640089in}}%
\pgfpathlineto{\pgfqpoint{1.161815in}{1.637598in}}%
\pgfpathlineto{\pgfqpoint{1.161018in}{1.637871in}}%
\pgfpathlineto{\pgfqpoint{1.160240in}{1.638156in}}%
\pgfpathlineto{\pgfqpoint{1.159482in}{1.638452in}}%
\pgfpathlineto{\pgfqpoint{1.158745in}{1.638759in}}%
\pgfpathlineto{\pgfqpoint{1.161399in}{1.641105in}}%
\pgfpathlineto{\pgfqpoint{1.164053in}{1.643318in}}%
\pgfpathlineto{\pgfqpoint{1.166705in}{1.645396in}}%
\pgfpathlineto{\pgfqpoint{1.169357in}{1.647340in}}%
\pgfpathlineto{\pgfqpoint{1.169726in}{1.647187in}}%
\pgfpathlineto{\pgfqpoint{1.170105in}{1.647039in}}%
\pgfpathlineto{\pgfqpoint{1.170493in}{1.646897in}}%
\pgfpathlineto{\pgfqpoint{1.170892in}{1.646761in}}%
\pgfpathclose%
\pgfusepath{fill}%
\end{pgfscope}%
\begin{pgfscope}%
\pgfpathrectangle{\pgfqpoint{0.041670in}{0.041670in}}{\pgfqpoint{2.216660in}{2.216660in}}%
\pgfusepath{clip}%
\pgfsetbuttcap%
\pgfsetroundjoin%
\definecolor{currentfill}{rgb}{0.955300,0.901065,0.118128}%
\pgfsetfillcolor{currentfill}%
\pgfsetlinewidth{0.000000pt}%
\definecolor{currentstroke}{rgb}{0.000000,0.000000,0.000000}%
\pgfsetstrokecolor{currentstroke}%
\pgfsetdash{}{0pt}%
\pgfpathmoveto{\pgfqpoint{1.222837in}{1.635249in}}%
\pgfpathlineto{\pgfqpoint{1.226412in}{1.632843in}}%
\pgfpathlineto{\pgfqpoint{1.229987in}{1.630306in}}%
\pgfpathlineto{\pgfqpoint{1.233562in}{1.627639in}}%
\pgfpathlineto{\pgfqpoint{1.237137in}{1.624841in}}%
\pgfpathlineto{\pgfqpoint{1.236310in}{1.624003in}}%
\pgfpathlineto{\pgfqpoint{1.235426in}{1.623176in}}%
\pgfpathlineto{\pgfqpoint{1.234486in}{1.622364in}}%
\pgfpathlineto{\pgfqpoint{1.233492in}{1.621565in}}%
\pgfpathlineto{\pgfqpoint{1.230143in}{1.624569in}}%
\pgfpathlineto{\pgfqpoint{1.226794in}{1.627442in}}%
\pgfpathlineto{\pgfqpoint{1.223446in}{1.630185in}}%
\pgfpathlineto{\pgfqpoint{1.220098in}{1.632796in}}%
\pgfpathlineto{\pgfqpoint{1.220845in}{1.633394in}}%
\pgfpathlineto{\pgfqpoint{1.221551in}{1.634003in}}%
\pgfpathlineto{\pgfqpoint{1.222215in}{1.634621in}}%
\pgfpathlineto{\pgfqpoint{1.222837in}{1.635249in}}%
\pgfpathclose%
\pgfusepath{fill}%
\end{pgfscope}%
\begin{pgfscope}%
\pgfpathrectangle{\pgfqpoint{0.041670in}{0.041670in}}{\pgfqpoint{2.216660in}{2.216660in}}%
\pgfusepath{clip}%
\pgfsetbuttcap%
\pgfsetroundjoin%
\definecolor{currentfill}{rgb}{0.280255,0.165693,0.476498}%
\pgfsetfillcolor{currentfill}%
\pgfsetlinewidth{0.000000pt}%
\definecolor{currentstroke}{rgb}{0.000000,0.000000,0.000000}%
\pgfsetstrokecolor{currentstroke}%
\pgfsetdash{}{0pt}%
\pgfpathmoveto{\pgfqpoint{1.307806in}{0.734405in}}%
\pgfpathlineto{\pgfqpoint{1.308836in}{0.726680in}}%
\pgfpathlineto{\pgfqpoint{1.309866in}{0.719076in}}%
\pgfpathlineto{\pgfqpoint{1.310898in}{0.711596in}}%
\pgfpathlineto{\pgfqpoint{1.311930in}{0.704244in}}%
\pgfpathlineto{\pgfqpoint{1.296644in}{0.702145in}}%
\pgfpathlineto{\pgfqpoint{1.281228in}{0.700306in}}%
\pgfpathlineto{\pgfqpoint{1.265701in}{0.698727in}}%
\pgfpathlineto{\pgfqpoint{1.250080in}{0.697411in}}%
\pgfpathlineto{\pgfqpoint{1.249531in}{0.704823in}}%
\pgfpathlineto{\pgfqpoint{1.248982in}{0.712363in}}%
\pgfpathlineto{\pgfqpoint{1.248434in}{0.720028in}}%
\pgfpathlineto{\pgfqpoint{1.247886in}{0.727813in}}%
\pgfpathlineto{\pgfqpoint{1.263020in}{0.729082in}}%
\pgfpathlineto{\pgfqpoint{1.278062in}{0.730606in}}%
\pgfpathlineto{\pgfqpoint{1.292996in}{0.732380in}}%
\pgfpathlineto{\pgfqpoint{1.307806in}{0.734405in}}%
\pgfpathclose%
\pgfusepath{fill}%
\end{pgfscope}%
\begin{pgfscope}%
\pgfpathrectangle{\pgfqpoint{0.041670in}{0.041670in}}{\pgfqpoint{2.216660in}{2.216660in}}%
\pgfusepath{clip}%
\pgfsetbuttcap%
\pgfsetroundjoin%
\definecolor{currentfill}{rgb}{0.274952,0.037752,0.364543}%
\pgfsetfillcolor{currentfill}%
\pgfsetlinewidth{0.000000pt}%
\definecolor{currentstroke}{rgb}{0.000000,0.000000,0.000000}%
\pgfsetstrokecolor{currentstroke}%
\pgfsetdash{}{0pt}%
\pgfpathmoveto{\pgfqpoint{1.050370in}{0.625378in}}%
\pgfpathlineto{\pgfqpoint{1.049430in}{0.620041in}}%
\pgfpathlineto{\pgfqpoint{1.048488in}{0.614889in}}%
\pgfpathlineto{\pgfqpoint{1.047545in}{0.609928in}}%
\pgfpathlineto{\pgfqpoint{1.046600in}{0.605161in}}%
\pgfpathlineto{\pgfqpoint{1.029404in}{0.607588in}}%
\pgfpathlineto{\pgfqpoint{1.012376in}{0.610306in}}%
\pgfpathlineto{\pgfqpoint{0.995535in}{0.613313in}}%
\pgfpathlineto{\pgfqpoint{0.978898in}{0.616605in}}%
\pgfpathlineto{\pgfqpoint{0.980320in}{0.621284in}}%
\pgfpathlineto{\pgfqpoint{0.981740in}{0.626157in}}%
\pgfpathlineto{\pgfqpoint{0.983158in}{0.631220in}}%
\pgfpathlineto{\pgfqpoint{0.984574in}{0.636469in}}%
\pgfpathlineto{\pgfqpoint{1.000743in}{0.633279in}}%
\pgfpathlineto{\pgfqpoint{1.017111in}{0.630365in}}%
\pgfpathlineto{\pgfqpoint{1.033659in}{0.627730in}}%
\pgfpathlineto{\pgfqpoint{1.050370in}{0.625378in}}%
\pgfpathclose%
\pgfusepath{fill}%
\end{pgfscope}%
\begin{pgfscope}%
\pgfpathrectangle{\pgfqpoint{0.041670in}{0.041670in}}{\pgfqpoint{2.216660in}{2.216660in}}%
\pgfusepath{clip}%
\pgfsetbuttcap%
\pgfsetroundjoin%
\definecolor{currentfill}{rgb}{0.993248,0.906157,0.143936}%
\pgfsetfillcolor{currentfill}%
\pgfsetlinewidth{0.000000pt}%
\definecolor{currentstroke}{rgb}{0.000000,0.000000,0.000000}%
\pgfsetstrokecolor{currentstroke}%
\pgfsetdash{}{0pt}%
\pgfpathmoveto{\pgfqpoint{1.189372in}{1.646881in}}%
\pgfpathlineto{\pgfqpoint{1.191729in}{1.644822in}}%
\pgfpathlineto{\pgfqpoint{1.194087in}{1.642629in}}%
\pgfpathlineto{\pgfqpoint{1.196445in}{1.640301in}}%
\pgfpathlineto{\pgfqpoint{1.198804in}{1.637840in}}%
\pgfpathlineto{\pgfqpoint{1.198005in}{1.637568in}}%
\pgfpathlineto{\pgfqpoint{1.197189in}{1.637308in}}%
\pgfpathlineto{\pgfqpoint{1.196355in}{1.637060in}}%
\pgfpathlineto{\pgfqpoint{1.195504in}{1.636824in}}%
\pgfpathlineto{\pgfqpoint{1.193558in}{1.639413in}}%
\pgfpathlineto{\pgfqpoint{1.191612in}{1.641868in}}%
\pgfpathlineto{\pgfqpoint{1.189667in}{1.644189in}}%
\pgfpathlineto{\pgfqpoint{1.187723in}{1.646375in}}%
\pgfpathlineto{\pgfqpoint{1.188148in}{1.646492in}}%
\pgfpathlineto{\pgfqpoint{1.188565in}{1.646616in}}%
\pgfpathlineto{\pgfqpoint{1.188973in}{1.646746in}}%
\pgfpathlineto{\pgfqpoint{1.189372in}{1.646881in}}%
\pgfpathclose%
\pgfusepath{fill}%
\end{pgfscope}%
\begin{pgfscope}%
\pgfpathrectangle{\pgfqpoint{0.041670in}{0.041670in}}{\pgfqpoint{2.216660in}{2.216660in}}%
\pgfusepath{clip}%
\pgfsetbuttcap%
\pgfsetroundjoin%
\definecolor{currentfill}{rgb}{0.974417,0.903590,0.130215}%
\pgfsetfillcolor{currentfill}%
\pgfsetlinewidth{0.000000pt}%
\definecolor{currentstroke}{rgb}{0.000000,0.000000,0.000000}%
\pgfsetstrokecolor{currentstroke}%
\pgfsetdash{}{0pt}%
\pgfpathmoveto{\pgfqpoint{1.206708in}{1.641921in}}%
\pgfpathlineto{\pgfqpoint{1.210055in}{1.639839in}}%
\pgfpathlineto{\pgfqpoint{1.213402in}{1.637624in}}%
\pgfpathlineto{\pgfqpoint{1.216750in}{1.635276in}}%
\pgfpathlineto{\pgfqpoint{1.220098in}{1.632796in}}%
\pgfpathlineto{\pgfqpoint{1.219311in}{1.632210in}}%
\pgfpathlineto{\pgfqpoint{1.218484in}{1.631635in}}%
\pgfpathlineto{\pgfqpoint{1.217619in}{1.631073in}}%
\pgfpathlineto{\pgfqpoint{1.216716in}{1.630524in}}%
\pgfpathlineto{\pgfqpoint{1.213649in}{1.633194in}}%
\pgfpathlineto{\pgfqpoint{1.210582in}{1.635732in}}%
\pgfpathlineto{\pgfqpoint{1.207516in}{1.638137in}}%
\pgfpathlineto{\pgfqpoint{1.204451in}{1.640409in}}%
\pgfpathlineto{\pgfqpoint{1.205054in}{1.640775in}}%
\pgfpathlineto{\pgfqpoint{1.205631in}{1.641149in}}%
\pgfpathlineto{\pgfqpoint{1.206182in}{1.641531in}}%
\pgfpathlineto{\pgfqpoint{1.206708in}{1.641921in}}%
\pgfpathclose%
\pgfusepath{fill}%
\end{pgfscope}%
\begin{pgfscope}%
\pgfpathrectangle{\pgfqpoint{0.041670in}{0.041670in}}{\pgfqpoint{2.216660in}{2.216660in}}%
\pgfusepath{clip}%
\pgfsetbuttcap%
\pgfsetroundjoin%
\definecolor{currentfill}{rgb}{0.274128,0.199721,0.498911}%
\pgfsetfillcolor{currentfill}%
\pgfsetlinewidth{0.000000pt}%
\definecolor{currentstroke}{rgb}{0.000000,0.000000,0.000000}%
\pgfsetstrokecolor{currentstroke}%
\pgfsetdash{}{0pt}%
\pgfpathmoveto{\pgfqpoint{1.303693in}{0.766430in}}%
\pgfpathlineto{\pgfqpoint{1.304720in}{0.758262in}}%
\pgfpathlineto{\pgfqpoint{1.305748in}{0.750200in}}%
\pgfpathlineto{\pgfqpoint{1.306776in}{0.742246in}}%
\pgfpathlineto{\pgfqpoint{1.307806in}{0.734405in}}%
\pgfpathlineto{\pgfqpoint{1.292996in}{0.732380in}}%
\pgfpathlineto{\pgfqpoint{1.278062in}{0.730606in}}%
\pgfpathlineto{\pgfqpoint{1.263020in}{0.729082in}}%
\pgfpathlineto{\pgfqpoint{1.247886in}{0.727813in}}%
\pgfpathlineto{\pgfqpoint{1.247339in}{0.735714in}}%
\pgfpathlineto{\pgfqpoint{1.246792in}{0.743727in}}%
\pgfpathlineto{\pgfqpoint{1.246245in}{0.751850in}}%
\pgfpathlineto{\pgfqpoint{1.245699in}{0.760078in}}%
\pgfpathlineto{\pgfqpoint{1.260345in}{0.761301in}}%
\pgfpathlineto{\pgfqpoint{1.274904in}{0.762769in}}%
\pgfpathlineto{\pgfqpoint{1.289358in}{0.764479in}}%
\pgfpathlineto{\pgfqpoint{1.303693in}{0.766430in}}%
\pgfpathclose%
\pgfusepath{fill}%
\end{pgfscope}%
\begin{pgfscope}%
\pgfpathrectangle{\pgfqpoint{0.041670in}{0.041670in}}{\pgfqpoint{2.216660in}{2.216660in}}%
\pgfusepath{clip}%
\pgfsetbuttcap%
\pgfsetroundjoin%
\definecolor{currentfill}{rgb}{0.855810,0.888601,0.097452}%
\pgfsetfillcolor{currentfill}%
\pgfsetlinewidth{0.000000pt}%
\definecolor{currentstroke}{rgb}{0.000000,0.000000,0.000000}%
\pgfsetstrokecolor{currentstroke}%
\pgfsetdash{}{0pt}%
\pgfpathmoveto{\pgfqpoint{1.265723in}{1.597855in}}%
\pgfpathlineto{\pgfqpoint{1.269293in}{1.593917in}}%
\pgfpathlineto{\pgfqpoint{1.272863in}{1.589856in}}%
\pgfpathlineto{\pgfqpoint{1.276432in}{1.585674in}}%
\pgfpathlineto{\pgfqpoint{1.280000in}{1.581370in}}%
\pgfpathlineto{\pgfqpoint{1.278561in}{1.579893in}}%
\pgfpathlineto{\pgfqpoint{1.277024in}{1.578437in}}%
\pgfpathlineto{\pgfqpoint{1.275388in}{1.577005in}}%
\pgfpathlineto{\pgfqpoint{1.273657in}{1.575597in}}%
\pgfpathlineto{\pgfqpoint{1.270312in}{1.580111in}}%
\pgfpathlineto{\pgfqpoint{1.266967in}{1.584503in}}%
\pgfpathlineto{\pgfqpoint{1.263621in}{1.588773in}}%
\pgfpathlineto{\pgfqpoint{1.260275in}{1.592920in}}%
\pgfpathlineto{\pgfqpoint{1.261762in}{1.594123in}}%
\pgfpathlineto{\pgfqpoint{1.263166in}{1.595347in}}%
\pgfpathlineto{\pgfqpoint{1.264487in}{1.596592in}}%
\pgfpathlineto{\pgfqpoint{1.265723in}{1.597855in}}%
\pgfpathclose%
\pgfusepath{fill}%
\end{pgfscope}%
\begin{pgfscope}%
\pgfpathrectangle{\pgfqpoint{0.041670in}{0.041670in}}{\pgfqpoint{2.216660in}{2.216660in}}%
\pgfusepath{clip}%
\pgfsetbuttcap%
\pgfsetroundjoin%
\definecolor{currentfill}{rgb}{0.955300,0.901065,0.118128}%
\pgfsetfillcolor{currentfill}%
\pgfsetlinewidth{0.000000pt}%
\definecolor{currentstroke}{rgb}{0.000000,0.000000,0.000000}%
\pgfsetstrokecolor{currentstroke}%
\pgfsetdash{}{0pt}%
\pgfpathmoveto{\pgfqpoint{1.140510in}{1.632274in}}%
\pgfpathlineto{\pgfqpoint{1.137219in}{1.629619in}}%
\pgfpathlineto{\pgfqpoint{1.133929in}{1.626833in}}%
\pgfpathlineto{\pgfqpoint{1.130638in}{1.623915in}}%
\pgfpathlineto{\pgfqpoint{1.127347in}{1.620868in}}%
\pgfpathlineto{\pgfqpoint{1.126305in}{1.621653in}}%
\pgfpathlineto{\pgfqpoint{1.125317in}{1.622453in}}%
\pgfpathlineto{\pgfqpoint{1.124383in}{1.623268in}}%
\pgfpathlineto{\pgfqpoint{1.123505in}{1.624095in}}%
\pgfpathlineto{\pgfqpoint{1.127035in}{1.626940in}}%
\pgfpathlineto{\pgfqpoint{1.130564in}{1.629654in}}%
\pgfpathlineto{\pgfqpoint{1.134094in}{1.632238in}}%
\pgfpathlineto{\pgfqpoint{1.137624in}{1.634690in}}%
\pgfpathlineto{\pgfqpoint{1.138283in}{1.634071in}}%
\pgfpathlineto{\pgfqpoint{1.138985in}{1.633461in}}%
\pgfpathlineto{\pgfqpoint{1.139727in}{1.632862in}}%
\pgfpathlineto{\pgfqpoint{1.140510in}{1.632274in}}%
\pgfpathclose%
\pgfusepath{fill}%
\end{pgfscope}%
\begin{pgfscope}%
\pgfpathrectangle{\pgfqpoint{0.041670in}{0.041670in}}{\pgfqpoint{2.216660in}{2.216660in}}%
\pgfusepath{clip}%
\pgfsetbuttcap%
\pgfsetroundjoin%
\definecolor{currentfill}{rgb}{0.993248,0.906157,0.143936}%
\pgfsetfillcolor{currentfill}%
\pgfsetlinewidth{0.000000pt}%
\definecolor{currentstroke}{rgb}{0.000000,0.000000,0.000000}%
\pgfsetstrokecolor{currentstroke}%
\pgfsetdash{}{0pt}%
\pgfpathmoveto{\pgfqpoint{1.172571in}{1.646276in}}%
\pgfpathlineto{\pgfqpoint{1.170723in}{1.644064in}}%
\pgfpathlineto{\pgfqpoint{1.168874in}{1.641719in}}%
\pgfpathlineto{\pgfqpoint{1.167025in}{1.639239in}}%
\pgfpathlineto{\pgfqpoint{1.165175in}{1.636626in}}%
\pgfpathlineto{\pgfqpoint{1.164310in}{1.636850in}}%
\pgfpathlineto{\pgfqpoint{1.163462in}{1.637087in}}%
\pgfpathlineto{\pgfqpoint{1.162629in}{1.637336in}}%
\pgfpathlineto{\pgfqpoint{1.161815in}{1.637598in}}%
\pgfpathlineto{\pgfqpoint{1.164085in}{1.640089in}}%
\pgfpathlineto{\pgfqpoint{1.166355in}{1.642447in}}%
\pgfpathlineto{\pgfqpoint{1.168624in}{1.644671in}}%
\pgfpathlineto{\pgfqpoint{1.170892in}{1.646761in}}%
\pgfpathlineto{\pgfqpoint{1.171299in}{1.646630in}}%
\pgfpathlineto{\pgfqpoint{1.171715in}{1.646506in}}%
\pgfpathlineto{\pgfqpoint{1.172139in}{1.646388in}}%
\pgfpathlineto{\pgfqpoint{1.172571in}{1.646276in}}%
\pgfpathclose%
\pgfusepath{fill}%
\end{pgfscope}%
\begin{pgfscope}%
\pgfpathrectangle{\pgfqpoint{0.041670in}{0.041670in}}{\pgfqpoint{2.216660in}{2.216660in}}%
\pgfusepath{clip}%
\pgfsetbuttcap%
\pgfsetroundjoin%
\definecolor{currentfill}{rgb}{0.283072,0.130895,0.449241}%
\pgfsetfillcolor{currentfill}%
\pgfsetlinewidth{0.000000pt}%
\definecolor{currentstroke}{rgb}{0.000000,0.000000,0.000000}%
\pgfsetstrokecolor{currentstroke}%
\pgfsetdash{}{0pt}%
\pgfpathmoveto{\pgfqpoint{1.311930in}{0.704244in}}%
\pgfpathlineto{\pgfqpoint{1.312963in}{0.697023in}}%
\pgfpathlineto{\pgfqpoint{1.313998in}{0.689939in}}%
\pgfpathlineto{\pgfqpoint{1.315033in}{0.682994in}}%
\pgfpathlineto{\pgfqpoint{1.316069in}{0.676193in}}%
\pgfpathlineto{\pgfqpoint{1.300304in}{0.674021in}}%
\pgfpathlineto{\pgfqpoint{1.284406in}{0.672116in}}%
\pgfpathlineto{\pgfqpoint{1.268392in}{0.670482in}}%
\pgfpathlineto{\pgfqpoint{1.252281in}{0.669119in}}%
\pgfpathlineto{\pgfqpoint{1.251730in}{0.675980in}}%
\pgfpathlineto{\pgfqpoint{1.251179in}{0.682985in}}%
\pgfpathlineto{\pgfqpoint{1.250629in}{0.690130in}}%
\pgfpathlineto{\pgfqpoint{1.250080in}{0.697411in}}%
\pgfpathlineto{\pgfqpoint{1.265701in}{0.698727in}}%
\pgfpathlineto{\pgfqpoint{1.281228in}{0.700306in}}%
\pgfpathlineto{\pgfqpoint{1.296644in}{0.702145in}}%
\pgfpathlineto{\pgfqpoint{1.311930in}{0.704244in}}%
\pgfpathclose%
\pgfusepath{fill}%
\end{pgfscope}%
\begin{pgfscope}%
\pgfpathrectangle{\pgfqpoint{0.041670in}{0.041670in}}{\pgfqpoint{2.216660in}{2.216660in}}%
\pgfusepath{clip}%
\pgfsetbuttcap%
\pgfsetroundjoin%
\definecolor{currentfill}{rgb}{0.935904,0.898570,0.108131}%
\pgfsetfillcolor{currentfill}%
\pgfsetlinewidth{0.000000pt}%
\definecolor{currentstroke}{rgb}{0.000000,0.000000,0.000000}%
\pgfsetstrokecolor{currentstroke}%
\pgfsetdash{}{0pt}%
\pgfpathmoveto{\pgfqpoint{1.237137in}{1.624841in}}%
\pgfpathlineto{\pgfqpoint{1.240712in}{1.621914in}}%
\pgfpathlineto{\pgfqpoint{1.244286in}{1.618859in}}%
\pgfpathlineto{\pgfqpoint{1.247860in}{1.615674in}}%
\pgfpathlineto{\pgfqpoint{1.251434in}{1.612363in}}%
\pgfpathlineto{\pgfqpoint{1.250402in}{1.611312in}}%
\pgfpathlineto{\pgfqpoint{1.249299in}{1.610278in}}%
\pgfpathlineto{\pgfqpoint{1.248126in}{1.609259in}}%
\pgfpathlineto{\pgfqpoint{1.246885in}{1.608259in}}%
\pgfpathlineto{\pgfqpoint{1.243537in}{1.611778in}}%
\pgfpathlineto{\pgfqpoint{1.240189in}{1.615169in}}%
\pgfpathlineto{\pgfqpoint{1.236840in}{1.618432in}}%
\pgfpathlineto{\pgfqpoint{1.233492in}{1.621565in}}%
\pgfpathlineto{\pgfqpoint{1.234486in}{1.622364in}}%
\pgfpathlineto{\pgfqpoint{1.235426in}{1.623176in}}%
\pgfpathlineto{\pgfqpoint{1.236310in}{1.624003in}}%
\pgfpathlineto{\pgfqpoint{1.237137in}{1.624841in}}%
\pgfpathclose%
\pgfusepath{fill}%
\end{pgfscope}%
\begin{pgfscope}%
\pgfpathrectangle{\pgfqpoint{0.041670in}{0.041670in}}{\pgfqpoint{2.216660in}{2.216660in}}%
\pgfusepath{clip}%
\pgfsetbuttcap%
\pgfsetroundjoin%
\definecolor{currentfill}{rgb}{0.565498,0.842430,0.262877}%
\pgfsetfillcolor{currentfill}%
\pgfsetlinewidth{0.000000pt}%
\definecolor{currentstroke}{rgb}{0.000000,0.000000,0.000000}%
\pgfsetstrokecolor{currentstroke}%
\pgfsetdash{}{0pt}%
\pgfpathmoveto{\pgfqpoint{1.048164in}{1.477512in}}%
\pgfpathlineto{\pgfqpoint{1.045179in}{1.470876in}}%
\pgfpathlineto{\pgfqpoint{1.042195in}{1.464142in}}%
\pgfpathlineto{\pgfqpoint{1.039212in}{1.457311in}}%
\pgfpathlineto{\pgfqpoint{1.036230in}{1.450384in}}%
\pgfpathlineto{\pgfqpoint{1.032524in}{1.452593in}}%
\pgfpathlineto{\pgfqpoint{1.028967in}{1.454855in}}%
\pgfpathlineto{\pgfqpoint{1.025565in}{1.457170in}}%
\pgfpathlineto{\pgfqpoint{1.022320in}{1.459534in}}%
\pgfpathlineto{\pgfqpoint{1.025586in}{1.466261in}}%
\pgfpathlineto{\pgfqpoint{1.028854in}{1.472894in}}%
\pgfpathlineto{\pgfqpoint{1.032122in}{1.479430in}}%
\pgfpathlineto{\pgfqpoint{1.035393in}{1.485868in}}%
\pgfpathlineto{\pgfqpoint{1.038372in}{1.483709in}}%
\pgfpathlineto{\pgfqpoint{1.041496in}{1.481595in}}%
\pgfpathlineto{\pgfqpoint{1.044761in}{1.479529in}}%
\pgfpathlineto{\pgfqpoint{1.048164in}{1.477512in}}%
\pgfpathclose%
\pgfusepath{fill}%
\end{pgfscope}%
\begin{pgfscope}%
\pgfpathrectangle{\pgfqpoint{0.041670in}{0.041670in}}{\pgfqpoint{2.216660in}{2.216660in}}%
\pgfusepath{clip}%
\pgfsetbuttcap%
\pgfsetroundjoin%
\definecolor{currentfill}{rgb}{0.814576,0.883393,0.110347}%
\pgfsetfillcolor{currentfill}%
\pgfsetlinewidth{0.000000pt}%
\definecolor{currentstroke}{rgb}{0.000000,0.000000,0.000000}%
\pgfsetstrokecolor{currentstroke}%
\pgfsetdash{}{0pt}%
\pgfpathmoveto{\pgfqpoint{1.087872in}{1.574368in}}%
\pgfpathlineto{\pgfqpoint{1.084585in}{1.569690in}}%
\pgfpathlineto{\pgfqpoint{1.081299in}{1.564892in}}%
\pgfpathlineto{\pgfqpoint{1.078013in}{1.559977in}}%
\pgfpathlineto{\pgfqpoint{1.074729in}{1.554944in}}%
\pgfpathlineto{\pgfqpoint{1.072657in}{1.556530in}}%
\pgfpathlineto{\pgfqpoint{1.070694in}{1.558147in}}%
\pgfpathlineto{\pgfqpoint{1.068841in}{1.559791in}}%
\pgfpathlineto{\pgfqpoint{1.067100in}{1.561462in}}%
\pgfpathlineto{\pgfqpoint{1.070619in}{1.566286in}}%
\pgfpathlineto{\pgfqpoint{1.074140in}{1.570994in}}%
\pgfpathlineto{\pgfqpoint{1.077661in}{1.575585in}}%
\pgfpathlineto{\pgfqpoint{1.081183in}{1.580056in}}%
\pgfpathlineto{\pgfqpoint{1.082710in}{1.578598in}}%
\pgfpathlineto{\pgfqpoint{1.084335in}{1.577163in}}%
\pgfpathlineto{\pgfqpoint{1.086056in}{1.575752in}}%
\pgfpathlineto{\pgfqpoint{1.087872in}{1.574368in}}%
\pgfpathclose%
\pgfusepath{fill}%
\end{pgfscope}%
\begin{pgfscope}%
\pgfpathrectangle{\pgfqpoint{0.041670in}{0.041670in}}{\pgfqpoint{2.216660in}{2.216660in}}%
\pgfusepath{clip}%
\pgfsetbuttcap%
\pgfsetroundjoin%
\definecolor{currentfill}{rgb}{0.974417,0.903590,0.130215}%
\pgfsetfillcolor{currentfill}%
\pgfsetlinewidth{0.000000pt}%
\definecolor{currentstroke}{rgb}{0.000000,0.000000,0.000000}%
\pgfsetstrokecolor{currentstroke}%
\pgfsetdash{}{0pt}%
\pgfpathmoveto{\pgfqpoint{1.156015in}{1.640092in}}%
\pgfpathlineto{\pgfqpoint{1.153019in}{1.637780in}}%
\pgfpathlineto{\pgfqpoint{1.150022in}{1.635335in}}%
\pgfpathlineto{\pgfqpoint{1.147025in}{1.632757in}}%
\pgfpathlineto{\pgfqpoint{1.144027in}{1.630047in}}%
\pgfpathlineto{\pgfqpoint{1.143092in}{1.630584in}}%
\pgfpathlineto{\pgfqpoint{1.142193in}{1.631135in}}%
\pgfpathlineto{\pgfqpoint{1.141332in}{1.631698in}}%
\pgfpathlineto{\pgfqpoint{1.140510in}{1.632274in}}%
\pgfpathlineto{\pgfqpoint{1.143800in}{1.634798in}}%
\pgfpathlineto{\pgfqpoint{1.147089in}{1.637189in}}%
\pgfpathlineto{\pgfqpoint{1.150379in}{1.639448in}}%
\pgfpathlineto{\pgfqpoint{1.153668in}{1.641574in}}%
\pgfpathlineto{\pgfqpoint{1.154216in}{1.641191in}}%
\pgfpathlineto{\pgfqpoint{1.154791in}{1.640816in}}%
\pgfpathlineto{\pgfqpoint{1.155390in}{1.640449in}}%
\pgfpathlineto{\pgfqpoint{1.156015in}{1.640092in}}%
\pgfpathclose%
\pgfusepath{fill}%
\end{pgfscope}%
\begin{pgfscope}%
\pgfpathrectangle{\pgfqpoint{0.041670in}{0.041670in}}{\pgfqpoint{2.216660in}{2.216660in}}%
\pgfusepath{clip}%
\pgfsetbuttcap%
\pgfsetroundjoin%
\definecolor{currentfill}{rgb}{0.896320,0.893616,0.096335}%
\pgfsetfillcolor{currentfill}%
\pgfsetlinewidth{0.000000pt}%
\definecolor{currentstroke}{rgb}{0.000000,0.000000,0.000000}%
\pgfsetstrokecolor{currentstroke}%
\pgfsetdash{}{0pt}%
\pgfpathmoveto{\pgfqpoint{1.251434in}{1.612363in}}%
\pgfpathlineto{\pgfqpoint{1.255007in}{1.608924in}}%
\pgfpathlineto{\pgfqpoint{1.258580in}{1.605360in}}%
\pgfpathlineto{\pgfqpoint{1.262151in}{1.601670in}}%
\pgfpathlineto{\pgfqpoint{1.265723in}{1.597855in}}%
\pgfpathlineto{\pgfqpoint{1.264487in}{1.596592in}}%
\pgfpathlineto{\pgfqpoint{1.263166in}{1.595347in}}%
\pgfpathlineto{\pgfqpoint{1.261762in}{1.594123in}}%
\pgfpathlineto{\pgfqpoint{1.260275in}{1.592920in}}%
\pgfpathlineto{\pgfqpoint{1.256928in}{1.596942in}}%
\pgfpathlineto{\pgfqpoint{1.253581in}{1.600841in}}%
\pgfpathlineto{\pgfqpoint{1.250233in}{1.604613in}}%
\pgfpathlineto{\pgfqpoint{1.246885in}{1.608259in}}%
\pgfpathlineto{\pgfqpoint{1.248126in}{1.609259in}}%
\pgfpathlineto{\pgfqpoint{1.249299in}{1.610278in}}%
\pgfpathlineto{\pgfqpoint{1.250402in}{1.611312in}}%
\pgfpathlineto{\pgfqpoint{1.251434in}{1.612363in}}%
\pgfpathclose%
\pgfusepath{fill}%
\end{pgfscope}%
\begin{pgfscope}%
\pgfpathrectangle{\pgfqpoint{0.041670in}{0.041670in}}{\pgfqpoint{2.216660in}{2.216660in}}%
\pgfusepath{clip}%
\pgfsetbuttcap%
\pgfsetroundjoin%
\definecolor{currentfill}{rgb}{0.993248,0.906157,0.143936}%
\pgfsetfillcolor{currentfill}%
\pgfsetlinewidth{0.000000pt}%
\definecolor{currentstroke}{rgb}{0.000000,0.000000,0.000000}%
\pgfsetstrokecolor{currentstroke}%
\pgfsetdash{}{0pt}%
\pgfpathmoveto{\pgfqpoint{1.187723in}{1.646375in}}%
\pgfpathlineto{\pgfqpoint{1.189667in}{1.644189in}}%
\pgfpathlineto{\pgfqpoint{1.191612in}{1.641868in}}%
\pgfpathlineto{\pgfqpoint{1.193558in}{1.639413in}}%
\pgfpathlineto{\pgfqpoint{1.195504in}{1.636824in}}%
\pgfpathlineto{\pgfqpoint{1.194638in}{1.636601in}}%
\pgfpathlineto{\pgfqpoint{1.193757in}{1.636391in}}%
\pgfpathlineto{\pgfqpoint{1.192862in}{1.636194in}}%
\pgfpathlineto{\pgfqpoint{1.191955in}{1.636011in}}%
\pgfpathlineto{\pgfqpoint{1.190453in}{1.638702in}}%
\pgfpathlineto{\pgfqpoint{1.188951in}{1.641258in}}%
\pgfpathlineto{\pgfqpoint{1.187450in}{1.643681in}}%
\pgfpathlineto{\pgfqpoint{1.185949in}{1.645969in}}%
\pgfpathlineto{\pgfqpoint{1.186403in}{1.646061in}}%
\pgfpathlineto{\pgfqpoint{1.186850in}{1.646159in}}%
\pgfpathlineto{\pgfqpoint{1.187290in}{1.646264in}}%
\pgfpathlineto{\pgfqpoint{1.187723in}{1.646375in}}%
\pgfpathclose%
\pgfusepath{fill}%
\end{pgfscope}%
\begin{pgfscope}%
\pgfpathrectangle{\pgfqpoint{0.041670in}{0.041670in}}{\pgfqpoint{2.216660in}{2.216660in}}%
\pgfusepath{clip}%
\pgfsetbuttcap%
\pgfsetroundjoin%
\definecolor{currentfill}{rgb}{0.163625,0.471133,0.558148}%
\pgfsetfillcolor{currentfill}%
\pgfsetlinewidth{0.000000pt}%
\definecolor{currentstroke}{rgb}{0.000000,0.000000,0.000000}%
\pgfsetstrokecolor{currentstroke}%
\pgfsetdash{}{0pt}%
\pgfpathmoveto{\pgfqpoint{1.051300in}{1.029164in}}%
\pgfpathlineto{\pgfqpoint{1.049916in}{1.019524in}}%
\pgfpathlineto{\pgfqpoint{1.048532in}{1.009896in}}%
\pgfpathlineto{\pgfqpoint{1.047149in}{1.000282in}}%
\pgfpathlineto{\pgfqpoint{1.045766in}{0.990687in}}%
\pgfpathlineto{\pgfqpoint{1.034788in}{0.992952in}}%
\pgfpathlineto{\pgfqpoint{1.023966in}{0.995393in}}%
\pgfpathlineto{\pgfqpoint{1.013311in}{0.998008in}}%
\pgfpathlineto{\pgfqpoint{1.002834in}{1.000792in}}%
\pgfpathlineto{\pgfqpoint{1.004657in}{1.010270in}}%
\pgfpathlineto{\pgfqpoint{1.006481in}{1.019766in}}%
\pgfpathlineto{\pgfqpoint{1.008305in}{1.029278in}}%
\pgfpathlineto{\pgfqpoint{1.010129in}{1.038801in}}%
\pgfpathlineto{\pgfqpoint{1.020177in}{1.036145in}}%
\pgfpathlineto{\pgfqpoint{1.030395in}{1.033652in}}%
\pgfpathlineto{\pgfqpoint{1.040773in}{1.031324in}}%
\pgfpathlineto{\pgfqpoint{1.051300in}{1.029164in}}%
\pgfpathclose%
\pgfusepath{fill}%
\end{pgfscope}%
\begin{pgfscope}%
\pgfpathrectangle{\pgfqpoint{0.041670in}{0.041670in}}{\pgfqpoint{2.216660in}{2.216660in}}%
\pgfusepath{clip}%
\pgfsetbuttcap%
\pgfsetroundjoin%
\definecolor{currentfill}{rgb}{0.263663,0.237631,0.518762}%
\pgfsetfillcolor{currentfill}%
\pgfsetlinewidth{0.000000pt}%
\definecolor{currentstroke}{rgb}{0.000000,0.000000,0.000000}%
\pgfsetstrokecolor{currentstroke}%
\pgfsetdash{}{0pt}%
\pgfpathmoveto{\pgfqpoint{1.299588in}{0.800079in}}%
\pgfpathlineto{\pgfqpoint{1.300614in}{0.791527in}}%
\pgfpathlineto{\pgfqpoint{1.301639in}{0.783066in}}%
\pgfpathlineto{\pgfqpoint{1.302666in}{0.774699in}}%
\pgfpathlineto{\pgfqpoint{1.303693in}{0.766430in}}%
\pgfpathlineto{\pgfqpoint{1.289358in}{0.764479in}}%
\pgfpathlineto{\pgfqpoint{1.274904in}{0.762769in}}%
\pgfpathlineto{\pgfqpoint{1.260345in}{0.761301in}}%
\pgfpathlineto{\pgfqpoint{1.245699in}{0.760078in}}%
\pgfpathlineto{\pgfqpoint{1.245153in}{0.768407in}}%
\pgfpathlineto{\pgfqpoint{1.244607in}{0.776833in}}%
\pgfpathlineto{\pgfqpoint{1.244062in}{0.785354in}}%
\pgfpathlineto{\pgfqpoint{1.243516in}{0.793966in}}%
\pgfpathlineto{\pgfqpoint{1.257677in}{0.795143in}}%
\pgfpathlineto{\pgfqpoint{1.271753in}{0.796556in}}%
\pgfpathlineto{\pgfqpoint{1.285728in}{0.798202in}}%
\pgfpathlineto{\pgfqpoint{1.299588in}{0.800079in}}%
\pgfpathclose%
\pgfusepath{fill}%
\end{pgfscope}%
\begin{pgfscope}%
\pgfpathrectangle{\pgfqpoint{0.041670in}{0.041670in}}{\pgfqpoint{2.216660in}{2.216660in}}%
\pgfusepath{clip}%
\pgfsetbuttcap%
\pgfsetroundjoin%
\definecolor{currentfill}{rgb}{0.212395,0.359683,0.551710}%
\pgfsetfillcolor{currentfill}%
\pgfsetlinewidth{0.000000pt}%
\definecolor{currentstroke}{rgb}{0.000000,0.000000,0.000000}%
\pgfsetstrokecolor{currentstroke}%
\pgfsetdash{}{0pt}%
\pgfpathmoveto{\pgfqpoint{1.083656in}{0.906996in}}%
\pgfpathlineto{\pgfqpoint{1.082739in}{0.897614in}}%
\pgfpathlineto{\pgfqpoint{1.081821in}{0.888282in}}%
\pgfpathlineto{\pgfqpoint{1.080903in}{0.879002in}}%
\pgfpathlineto{\pgfqpoint{1.079985in}{0.869777in}}%
\pgfpathlineto{\pgfqpoint{1.067085in}{0.871532in}}%
\pgfpathlineto{\pgfqpoint{1.054308in}{0.873499in}}%
\pgfpathlineto{\pgfqpoint{1.041667in}{0.875674in}}%
\pgfpathlineto{\pgfqpoint{1.029175in}{0.878056in}}%
\pgfpathlineto{\pgfqpoint{1.030558in}{0.887194in}}%
\pgfpathlineto{\pgfqpoint{1.031941in}{0.896388in}}%
\pgfpathlineto{\pgfqpoint{1.033323in}{0.905635in}}%
\pgfpathlineto{\pgfqpoint{1.034706in}{0.914931in}}%
\pgfpathlineto{\pgfqpoint{1.046740in}{0.912648in}}%
\pgfpathlineto{\pgfqpoint{1.058919in}{0.910563in}}%
\pgfpathlineto{\pgfqpoint{1.071229in}{0.908679in}}%
\pgfpathlineto{\pgfqpoint{1.083656in}{0.906996in}}%
\pgfpathclose%
\pgfusepath{fill}%
\end{pgfscope}%
\begin{pgfscope}%
\pgfpathrectangle{\pgfqpoint{0.041670in}{0.041670in}}{\pgfqpoint{2.216660in}{2.216660in}}%
\pgfusepath{clip}%
\pgfsetbuttcap%
\pgfsetroundjoin%
\definecolor{currentfill}{rgb}{0.993248,0.906157,0.143936}%
\pgfsetfillcolor{currentfill}%
\pgfsetlinewidth{0.000000pt}%
\definecolor{currentstroke}{rgb}{0.000000,0.000000,0.000000}%
\pgfsetstrokecolor{currentstroke}%
\pgfsetdash{}{0pt}%
\pgfpathmoveto{\pgfqpoint{1.174368in}{1.645893in}}%
\pgfpathlineto{\pgfqpoint{1.172970in}{1.643586in}}%
\pgfpathlineto{\pgfqpoint{1.171571in}{1.641145in}}%
\pgfpathlineto{\pgfqpoint{1.170172in}{1.638569in}}%
\pgfpathlineto{\pgfqpoint{1.168772in}{1.635859in}}%
\pgfpathlineto{\pgfqpoint{1.167854in}{1.636031in}}%
\pgfpathlineto{\pgfqpoint{1.166947in}{1.636216in}}%
\pgfpathlineto{\pgfqpoint{1.166054in}{1.636414in}}%
\pgfpathlineto{\pgfqpoint{1.165175in}{1.636626in}}%
\pgfpathlineto{\pgfqpoint{1.167025in}{1.639239in}}%
\pgfpathlineto{\pgfqpoint{1.168874in}{1.641719in}}%
\pgfpathlineto{\pgfqpoint{1.170723in}{1.644064in}}%
\pgfpathlineto{\pgfqpoint{1.172571in}{1.646276in}}%
\pgfpathlineto{\pgfqpoint{1.173010in}{1.646170in}}%
\pgfpathlineto{\pgfqpoint{1.173457in}{1.646071in}}%
\pgfpathlineto{\pgfqpoint{1.173910in}{1.645979in}}%
\pgfpathlineto{\pgfqpoint{1.174368in}{1.645893in}}%
\pgfpathclose%
\pgfusepath{fill}%
\end{pgfscope}%
\begin{pgfscope}%
\pgfpathrectangle{\pgfqpoint{0.041670in}{0.041670in}}{\pgfqpoint{2.216660in}{2.216660in}}%
\pgfusepath{clip}%
\pgfsetbuttcap%
\pgfsetroundjoin%
\definecolor{currentfill}{rgb}{0.636902,0.856542,0.216620}%
\pgfsetfillcolor{currentfill}%
\pgfsetlinewidth{0.000000pt}%
\definecolor{currentstroke}{rgb}{0.000000,0.000000,0.000000}%
\pgfsetstrokecolor{currentstroke}%
\pgfsetdash{}{0pt}%
\pgfpathmoveto{\pgfqpoint{1.313724in}{1.512383in}}%
\pgfpathlineto{\pgfqpoint{1.317055in}{1.506398in}}%
\pgfpathlineto{\pgfqpoint{1.320386in}{1.500308in}}%
\pgfpathlineto{\pgfqpoint{1.323715in}{1.494117in}}%
\pgfpathlineto{\pgfqpoint{1.327043in}{1.487824in}}%
\pgfpathlineto{\pgfqpoint{1.324193in}{1.485626in}}%
\pgfpathlineto{\pgfqpoint{1.321197in}{1.483472in}}%
\pgfpathlineto{\pgfqpoint{1.318058in}{1.481363in}}%
\pgfpathlineto{\pgfqpoint{1.314777in}{1.479302in}}%
\pgfpathlineto{\pgfqpoint{1.311723in}{1.485796in}}%
\pgfpathlineto{\pgfqpoint{1.308668in}{1.492188in}}%
\pgfpathlineto{\pgfqpoint{1.305612in}{1.498477in}}%
\pgfpathlineto{\pgfqpoint{1.302554in}{1.504662in}}%
\pgfpathlineto{\pgfqpoint{1.305541in}{1.506529in}}%
\pgfpathlineto{\pgfqpoint{1.308400in}{1.508440in}}%
\pgfpathlineto{\pgfqpoint{1.311128in}{1.510392in}}%
\pgfpathlineto{\pgfqpoint{1.313724in}{1.512383in}}%
\pgfpathclose%
\pgfusepath{fill}%
\end{pgfscope}%
\begin{pgfscope}%
\pgfpathrectangle{\pgfqpoint{0.041670in}{0.041670in}}{\pgfqpoint{2.216660in}{2.216660in}}%
\pgfusepath{clip}%
\pgfsetbuttcap%
\pgfsetroundjoin%
\definecolor{currentfill}{rgb}{0.122606,0.585371,0.546557}%
\pgfsetfillcolor{currentfill}%
\pgfsetlinewidth{0.000000pt}%
\definecolor{currentstroke}{rgb}{0.000000,0.000000,0.000000}%
\pgfsetstrokecolor{currentstroke}%
\pgfsetdash{}{0pt}%
\pgfpathmoveto{\pgfqpoint{1.032056in}{1.152955in}}%
\pgfpathlineto{\pgfqpoint{1.030226in}{1.143513in}}%
\pgfpathlineto{\pgfqpoint{1.028396in}{1.134048in}}%
\pgfpathlineto{\pgfqpoint{1.026567in}{1.124565in}}%
\pgfpathlineto{\pgfqpoint{1.024738in}{1.115065in}}%
\pgfpathlineto{\pgfqpoint{1.015714in}{1.117610in}}%
\pgfpathlineto{\pgfqpoint{1.006864in}{1.120296in}}%
\pgfpathlineto{\pgfqpoint{0.998197in}{1.123121in}}%
\pgfpathlineto{\pgfqpoint{0.989722in}{1.126081in}}%
\pgfpathlineto{\pgfqpoint{0.991959in}{1.135437in}}%
\pgfpathlineto{\pgfqpoint{0.994198in}{1.144776in}}%
\pgfpathlineto{\pgfqpoint{0.996437in}{1.154097in}}%
\pgfpathlineto{\pgfqpoint{0.998677in}{1.163396in}}%
\pgfpathlineto{\pgfqpoint{1.006757in}{1.160590in}}%
\pgfpathlineto{\pgfqpoint{1.015019in}{1.157913in}}%
\pgfpathlineto{\pgfqpoint{1.023455in}{1.155367in}}%
\pgfpathlineto{\pgfqpoint{1.032056in}{1.152955in}}%
\pgfpathclose%
\pgfusepath{fill}%
\end{pgfscope}%
\begin{pgfscope}%
\pgfpathrectangle{\pgfqpoint{0.041670in}{0.041670in}}{\pgfqpoint{2.216660in}{2.216660in}}%
\pgfusepath{clip}%
\pgfsetbuttcap%
\pgfsetroundjoin%
\definecolor{currentfill}{rgb}{0.280255,0.165693,0.476498}%
\pgfsetfillcolor{currentfill}%
\pgfsetlinewidth{0.000000pt}%
\definecolor{currentstroke}{rgb}{0.000000,0.000000,0.000000}%
\pgfsetstrokecolor{currentstroke}%
\pgfsetdash{}{0pt}%
\pgfpathmoveto{\pgfqpoint{1.125539in}{0.726898in}}%
\pgfpathlineto{\pgfqpoint{1.125100in}{0.719105in}}%
\pgfpathlineto{\pgfqpoint{1.124661in}{0.711432in}}%
\pgfpathlineto{\pgfqpoint{1.124221in}{0.703883in}}%
\pgfpathlineto{\pgfqpoint{1.123781in}{0.696462in}}%
\pgfpathlineto{\pgfqpoint{1.108090in}{0.697544in}}%
\pgfpathlineto{\pgfqpoint{1.092478in}{0.698889in}}%
\pgfpathlineto{\pgfqpoint{1.076963in}{0.700497in}}%
\pgfpathlineto{\pgfqpoint{1.061561in}{0.702366in}}%
\pgfpathlineto{\pgfqpoint{1.062487in}{0.709734in}}%
\pgfpathlineto{\pgfqpoint{1.063413in}{0.717231in}}%
\pgfpathlineto{\pgfqpoint{1.064338in}{0.724852in}}%
\pgfpathlineto{\pgfqpoint{1.065262in}{0.732593in}}%
\pgfpathlineto{\pgfqpoint{1.080183in}{0.730790in}}%
\pgfpathlineto{\pgfqpoint{1.095214in}{0.729239in}}%
\pgfpathlineto{\pgfqpoint{1.110338in}{0.727941in}}%
\pgfpathlineto{\pgfqpoint{1.125539in}{0.726898in}}%
\pgfpathclose%
\pgfusepath{fill}%
\end{pgfscope}%
\begin{pgfscope}%
\pgfpathrectangle{\pgfqpoint{0.041670in}{0.041670in}}{\pgfqpoint{2.216660in}{2.216660in}}%
\pgfusepath{clip}%
\pgfsetbuttcap%
\pgfsetroundjoin%
\definecolor{currentfill}{rgb}{0.344074,0.780029,0.397381}%
\pgfsetfillcolor{currentfill}%
\pgfsetlinewidth{0.000000pt}%
\definecolor{currentstroke}{rgb}{0.000000,0.000000,0.000000}%
\pgfsetstrokecolor{currentstroke}%
\pgfsetdash{}{0pt}%
\pgfpathmoveto{\pgfqpoint{1.031343in}{1.382063in}}%
\pgfpathlineto{\pgfqpoint{1.028705in}{1.374177in}}%
\pgfpathlineto{\pgfqpoint{1.026068in}{1.366214in}}%
\pgfpathlineto{\pgfqpoint{1.023431in}{1.358175in}}%
\pgfpathlineto{\pgfqpoint{1.020796in}{1.350063in}}%
\pgfpathlineto{\pgfqpoint{1.015475in}{1.352555in}}%
\pgfpathlineto{\pgfqpoint{1.010323in}{1.355126in}}%
\pgfpathlineto{\pgfqpoint{1.005347in}{1.357774in}}%
\pgfpathlineto{\pgfqpoint{1.000550in}{1.360497in}}%
\pgfpathlineto{\pgfqpoint{1.003516in}{1.368424in}}%
\pgfpathlineto{\pgfqpoint{1.006483in}{1.376278in}}%
\pgfpathlineto{\pgfqpoint{1.009452in}{1.384057in}}%
\pgfpathlineto{\pgfqpoint{1.012422in}{1.391760in}}%
\pgfpathlineto{\pgfqpoint{1.016905in}{1.389229in}}%
\pgfpathlineto{\pgfqpoint{1.021557in}{1.386768in}}%
\pgfpathlineto{\pgfqpoint{1.026371in}{1.384379in}}%
\pgfpathlineto{\pgfqpoint{1.031343in}{1.382063in}}%
\pgfpathclose%
\pgfusepath{fill}%
\end{pgfscope}%
\begin{pgfscope}%
\pgfpathrectangle{\pgfqpoint{0.041670in}{0.041670in}}{\pgfqpoint{2.216660in}{2.216660in}}%
\pgfusepath{clip}%
\pgfsetbuttcap%
\pgfsetroundjoin%
\definecolor{currentfill}{rgb}{0.935904,0.898570,0.108131}%
\pgfsetfillcolor{currentfill}%
\pgfsetlinewidth{0.000000pt}%
\definecolor{currentstroke}{rgb}{0.000000,0.000000,0.000000}%
\pgfsetstrokecolor{currentstroke}%
\pgfsetdash{}{0pt}%
\pgfpathmoveto{\pgfqpoint{1.127347in}{1.620868in}}%
\pgfpathlineto{\pgfqpoint{1.124056in}{1.617690in}}%
\pgfpathlineto{\pgfqpoint{1.120766in}{1.614384in}}%
\pgfpathlineto{\pgfqpoint{1.117475in}{1.610949in}}%
\pgfpathlineto{\pgfqpoint{1.114184in}{1.607386in}}%
\pgfpathlineto{\pgfqpoint{1.112883in}{1.608369in}}%
\pgfpathlineto{\pgfqpoint{1.111650in}{1.609372in}}%
\pgfpathlineto{\pgfqpoint{1.110485in}{1.610392in}}%
\pgfpathlineto{\pgfqpoint{1.109390in}{1.611428in}}%
\pgfpathlineto{\pgfqpoint{1.112918in}{1.614787in}}%
\pgfpathlineto{\pgfqpoint{1.116447in}{1.618018in}}%
\pgfpathlineto{\pgfqpoint{1.119976in}{1.621121in}}%
\pgfpathlineto{\pgfqpoint{1.123505in}{1.624095in}}%
\pgfpathlineto{\pgfqpoint{1.124383in}{1.623268in}}%
\pgfpathlineto{\pgfqpoint{1.125317in}{1.622453in}}%
\pgfpathlineto{\pgfqpoint{1.126305in}{1.621653in}}%
\pgfpathlineto{\pgfqpoint{1.127347in}{1.620868in}}%
\pgfpathclose%
\pgfusepath{fill}%
\end{pgfscope}%
\begin{pgfscope}%
\pgfpathrectangle{\pgfqpoint{0.041670in}{0.041670in}}{\pgfqpoint{2.216660in}{2.216660in}}%
\pgfusepath{clip}%
\pgfsetbuttcap%
\pgfsetroundjoin%
\definecolor{currentfill}{rgb}{0.274128,0.199721,0.498911}%
\pgfsetfillcolor{currentfill}%
\pgfsetlinewidth{0.000000pt}%
\definecolor{currentstroke}{rgb}{0.000000,0.000000,0.000000}%
\pgfsetstrokecolor{currentstroke}%
\pgfsetdash{}{0pt}%
\pgfpathmoveto{\pgfqpoint{1.127291in}{0.759196in}}%
\pgfpathlineto{\pgfqpoint{1.126853in}{0.750960in}}%
\pgfpathlineto{\pgfqpoint{1.126415in}{0.742829in}}%
\pgfpathlineto{\pgfqpoint{1.125977in}{0.734807in}}%
\pgfpathlineto{\pgfqpoint{1.125539in}{0.726898in}}%
\pgfpathlineto{\pgfqpoint{1.110338in}{0.727941in}}%
\pgfpathlineto{\pgfqpoint{1.095214in}{0.729239in}}%
\pgfpathlineto{\pgfqpoint{1.080183in}{0.730790in}}%
\pgfpathlineto{\pgfqpoint{1.065262in}{0.732593in}}%
\pgfpathlineto{\pgfqpoint{1.066186in}{0.740451in}}%
\pgfpathlineto{\pgfqpoint{1.067108in}{0.748421in}}%
\pgfpathlineto{\pgfqpoint{1.068031in}{0.756500in}}%
\pgfpathlineto{\pgfqpoint{1.068953in}{0.764684in}}%
\pgfpathlineto{\pgfqpoint{1.083394in}{0.762947in}}%
\pgfpathlineto{\pgfqpoint{1.097942in}{0.761452in}}%
\pgfpathlineto{\pgfqpoint{1.112580in}{0.760201in}}%
\pgfpathlineto{\pgfqpoint{1.127291in}{0.759196in}}%
\pgfpathclose%
\pgfusepath{fill}%
\end{pgfscope}%
\begin{pgfscope}%
\pgfpathrectangle{\pgfqpoint{0.041670in}{0.041670in}}{\pgfqpoint{2.216660in}{2.216660in}}%
\pgfusepath{clip}%
\pgfsetbuttcap%
\pgfsetroundjoin%
\definecolor{currentfill}{rgb}{0.271305,0.019942,0.347269}%
\pgfsetfillcolor{currentfill}%
\pgfsetlinewidth{0.000000pt}%
\definecolor{currentstroke}{rgb}{0.000000,0.000000,0.000000}%
\pgfsetstrokecolor{currentstroke}%
\pgfsetdash{}{0pt}%
\pgfpathmoveto{\pgfqpoint{1.395613in}{0.619766in}}%
\pgfpathlineto{\pgfqpoint{1.397140in}{0.615311in}}%
\pgfpathlineto{\pgfqpoint{1.398670in}{0.611059in}}%
\pgfpathlineto{\pgfqpoint{1.400203in}{0.607015in}}%
\pgfpathlineto{\pgfqpoint{1.401739in}{0.603184in}}%
\pgfpathlineto{\pgfqpoint{1.384836in}{0.599534in}}%
\pgfpathlineto{\pgfqpoint{1.367704in}{0.596174in}}%
\pgfpathlineto{\pgfqpoint{1.350364in}{0.593107in}}%
\pgfpathlineto{\pgfqpoint{1.332833in}{0.590338in}}%
\pgfpathlineto{\pgfqpoint{1.331772in}{0.594265in}}%
\pgfpathlineto{\pgfqpoint{1.330714in}{0.598404in}}%
\pgfpathlineto{\pgfqpoint{1.329657in}{0.602752in}}%
\pgfpathlineto{\pgfqpoint{1.328603in}{0.607304in}}%
\pgfpathlineto{\pgfqpoint{1.345650in}{0.609990in}}%
\pgfpathlineto{\pgfqpoint{1.362514in}{0.612965in}}%
\pgfpathlineto{\pgfqpoint{1.379174in}{0.616225in}}%
\pgfpathlineto{\pgfqpoint{1.395613in}{0.619766in}}%
\pgfpathclose%
\pgfusepath{fill}%
\end{pgfscope}%
\begin{pgfscope}%
\pgfpathrectangle{\pgfqpoint{0.041670in}{0.041670in}}{\pgfqpoint{2.216660in}{2.216660in}}%
\pgfusepath{clip}%
\pgfsetbuttcap%
\pgfsetroundjoin%
\definecolor{currentfill}{rgb}{0.993248,0.906157,0.143936}%
\pgfsetfillcolor{currentfill}%
\pgfsetlinewidth{0.000000pt}%
\definecolor{currentstroke}{rgb}{0.000000,0.000000,0.000000}%
\pgfsetstrokecolor{currentstroke}%
\pgfsetdash{}{0pt}%
\pgfpathmoveto{\pgfqpoint{1.185949in}{1.645969in}}%
\pgfpathlineto{\pgfqpoint{1.187450in}{1.643681in}}%
\pgfpathlineto{\pgfqpoint{1.188951in}{1.641258in}}%
\pgfpathlineto{\pgfqpoint{1.190453in}{1.638702in}}%
\pgfpathlineto{\pgfqpoint{1.191955in}{1.636011in}}%
\pgfpathlineto{\pgfqpoint{1.191035in}{1.635841in}}%
\pgfpathlineto{\pgfqpoint{1.190104in}{1.635684in}}%
\pgfpathlineto{\pgfqpoint{1.189163in}{1.635541in}}%
\pgfpathlineto{\pgfqpoint{1.188213in}{1.635413in}}%
\pgfpathlineto{\pgfqpoint{1.187179in}{1.638179in}}%
\pgfpathlineto{\pgfqpoint{1.186146in}{1.640810in}}%
\pgfpathlineto{\pgfqpoint{1.185113in}{1.643308in}}%
\pgfpathlineto{\pgfqpoint{1.184080in}{1.645671in}}%
\pgfpathlineto{\pgfqpoint{1.184555in}{1.645735in}}%
\pgfpathlineto{\pgfqpoint{1.185025in}{1.645806in}}%
\pgfpathlineto{\pgfqpoint{1.185490in}{1.645884in}}%
\pgfpathlineto{\pgfqpoint{1.185949in}{1.645969in}}%
\pgfpathclose%
\pgfusepath{fill}%
\end{pgfscope}%
\begin{pgfscope}%
\pgfpathrectangle{\pgfqpoint{0.041670in}{0.041670in}}{\pgfqpoint{2.216660in}{2.216660in}}%
\pgfusepath{clip}%
\pgfsetbuttcap%
\pgfsetroundjoin%
\definecolor{currentfill}{rgb}{0.855810,0.888601,0.097452}%
\pgfsetfillcolor{currentfill}%
\pgfsetlinewidth{0.000000pt}%
\definecolor{currentstroke}{rgb}{0.000000,0.000000,0.000000}%
\pgfsetstrokecolor{currentstroke}%
\pgfsetdash{}{0pt}%
\pgfpathmoveto{\pgfqpoint{1.101025in}{1.591869in}}%
\pgfpathlineto{\pgfqpoint{1.097736in}{1.587677in}}%
\pgfpathlineto{\pgfqpoint{1.094447in}{1.583363in}}%
\pgfpathlineto{\pgfqpoint{1.091159in}{1.578926in}}%
\pgfpathlineto{\pgfqpoint{1.087872in}{1.574368in}}%
\pgfpathlineto{\pgfqpoint{1.086056in}{1.575752in}}%
\pgfpathlineto{\pgfqpoint{1.084335in}{1.577163in}}%
\pgfpathlineto{\pgfqpoint{1.082710in}{1.578598in}}%
\pgfpathlineto{\pgfqpoint{1.081183in}{1.580056in}}%
\pgfpathlineto{\pgfqpoint{1.084707in}{1.584407in}}%
\pgfpathlineto{\pgfqpoint{1.088231in}{1.588637in}}%
\pgfpathlineto{\pgfqpoint{1.091756in}{1.592746in}}%
\pgfpathlineto{\pgfqpoint{1.095281in}{1.596731in}}%
\pgfpathlineto{\pgfqpoint{1.096593in}{1.595485in}}%
\pgfpathlineto{\pgfqpoint{1.097988in}{1.594258in}}%
\pgfpathlineto{\pgfqpoint{1.099466in}{1.593052in}}%
\pgfpathlineto{\pgfqpoint{1.101025in}{1.591869in}}%
\pgfpathclose%
\pgfusepath{fill}%
\end{pgfscope}%
\begin{pgfscope}%
\pgfpathrectangle{\pgfqpoint{0.041670in}{0.041670in}}{\pgfqpoint{2.216660in}{2.216660in}}%
\pgfusepath{clip}%
\pgfsetbuttcap%
\pgfsetroundjoin%
\definecolor{currentfill}{rgb}{0.166383,0.690856,0.496502}%
\pgfsetfillcolor{currentfill}%
\pgfsetlinewidth{0.000000pt}%
\definecolor{currentstroke}{rgb}{0.000000,0.000000,0.000000}%
\pgfsetstrokecolor{currentstroke}%
\pgfsetdash{}{0pt}%
\pgfpathmoveto{\pgfqpoint{1.025629in}{1.272358in}}%
\pgfpathlineto{\pgfqpoint{1.023378in}{1.263518in}}%
\pgfpathlineto{\pgfqpoint{1.021128in}{1.254626in}}%
\pgfpathlineto{\pgfqpoint{1.018879in}{1.245685in}}%
\pgfpathlineto{\pgfqpoint{1.016631in}{1.236697in}}%
\pgfpathlineto{\pgfqpoint{1.009516in}{1.239310in}}%
\pgfpathlineto{\pgfqpoint{1.002579in}{1.242031in}}%
\pgfpathlineto{\pgfqpoint{0.995827in}{1.244859in}}%
\pgfpathlineto{\pgfqpoint{0.989267in}{1.247789in}}%
\pgfpathlineto{\pgfqpoint{0.991888in}{1.256611in}}%
\pgfpathlineto{\pgfqpoint{0.994510in}{1.265385in}}%
\pgfpathlineto{\pgfqpoint{0.997133in}{1.274111in}}%
\pgfpathlineto{\pgfqpoint{0.999758in}{1.282786in}}%
\pgfpathlineto{\pgfqpoint{1.005961in}{1.280031in}}%
\pgfpathlineto{\pgfqpoint{1.012345in}{1.277373in}}%
\pgfpathlineto{\pgfqpoint{1.018903in}{1.274814in}}%
\pgfpathlineto{\pgfqpoint{1.025629in}{1.272358in}}%
\pgfpathclose%
\pgfusepath{fill}%
\end{pgfscope}%
\begin{pgfscope}%
\pgfpathrectangle{\pgfqpoint{0.041670in}{0.041670in}}{\pgfqpoint{2.216660in}{2.216660in}}%
\pgfusepath{clip}%
\pgfsetbuttcap%
\pgfsetroundjoin%
\definecolor{currentfill}{rgb}{0.993248,0.906157,0.143936}%
\pgfsetfillcolor{currentfill}%
\pgfsetlinewidth{0.000000pt}%
\definecolor{currentstroke}{rgb}{0.000000,0.000000,0.000000}%
\pgfsetstrokecolor{currentstroke}%
\pgfsetdash{}{0pt}%
\pgfpathmoveto{\pgfqpoint{1.176255in}{1.645619in}}%
\pgfpathlineto{\pgfqpoint{1.175329in}{1.643244in}}%
\pgfpathlineto{\pgfqpoint{1.174402in}{1.640733in}}%
\pgfpathlineto{\pgfqpoint{1.173476in}{1.638089in}}%
\pgfpathlineto{\pgfqpoint{1.172548in}{1.635310in}}%
\pgfpathlineto{\pgfqpoint{1.171591in}{1.635426in}}%
\pgfpathlineto{\pgfqpoint{1.170641in}{1.635557in}}%
\pgfpathlineto{\pgfqpoint{1.169701in}{1.635701in}}%
\pgfpathlineto{\pgfqpoint{1.168772in}{1.635859in}}%
\pgfpathlineto{\pgfqpoint{1.170172in}{1.638569in}}%
\pgfpathlineto{\pgfqpoint{1.171571in}{1.641145in}}%
\pgfpathlineto{\pgfqpoint{1.172970in}{1.643586in}}%
\pgfpathlineto{\pgfqpoint{1.174368in}{1.645893in}}%
\pgfpathlineto{\pgfqpoint{1.174833in}{1.645814in}}%
\pgfpathlineto{\pgfqpoint{1.175302in}{1.645742in}}%
\pgfpathlineto{\pgfqpoint{1.175777in}{1.645677in}}%
\pgfpathlineto{\pgfqpoint{1.176255in}{1.645619in}}%
\pgfpathclose%
\pgfusepath{fill}%
\end{pgfscope}%
\begin{pgfscope}%
\pgfpathrectangle{\pgfqpoint{0.041670in}{0.041670in}}{\pgfqpoint{2.216660in}{2.216660in}}%
\pgfusepath{clip}%
\pgfsetbuttcap%
\pgfsetroundjoin%
\definecolor{currentfill}{rgb}{0.282327,0.094955,0.417331}%
\pgfsetfillcolor{currentfill}%
\pgfsetlinewidth{0.000000pt}%
\definecolor{currentstroke}{rgb}{0.000000,0.000000,0.000000}%
\pgfsetstrokecolor{currentstroke}%
\pgfsetdash{}{0pt}%
\pgfpathmoveto{\pgfqpoint{1.316069in}{0.676193in}}%
\pgfpathlineto{\pgfqpoint{1.317106in}{0.669540in}}%
\pgfpathlineto{\pgfqpoint{1.318145in}{0.663039in}}%
\pgfpathlineto{\pgfqpoint{1.319184in}{0.656694in}}%
\pgfpathlineto{\pgfqpoint{1.320225in}{0.650509in}}%
\pgfpathlineto{\pgfqpoint{1.303980in}{0.648263in}}%
\pgfpathlineto{\pgfqpoint{1.287597in}{0.646293in}}%
\pgfpathlineto{\pgfqpoint{1.271095in}{0.644603in}}%
\pgfpathlineto{\pgfqpoint{1.254491in}{0.643193in}}%
\pgfpathlineto{\pgfqpoint{1.253938in}{0.649439in}}%
\pgfpathlineto{\pgfqpoint{1.253385in}{0.655844in}}%
\pgfpathlineto{\pgfqpoint{1.252833in}{0.662406in}}%
\pgfpathlineto{\pgfqpoint{1.252281in}{0.669119in}}%
\pgfpathlineto{\pgfqpoint{1.268392in}{0.670482in}}%
\pgfpathlineto{\pgfqpoint{1.284406in}{0.672116in}}%
\pgfpathlineto{\pgfqpoint{1.300304in}{0.674021in}}%
\pgfpathlineto{\pgfqpoint{1.316069in}{0.676193in}}%
\pgfpathclose%
\pgfusepath{fill}%
\end{pgfscope}%
\begin{pgfscope}%
\pgfpathrectangle{\pgfqpoint{0.041670in}{0.041670in}}{\pgfqpoint{2.216660in}{2.216660in}}%
\pgfusepath{clip}%
\pgfsetbuttcap%
\pgfsetroundjoin%
\definecolor{currentfill}{rgb}{0.896320,0.893616,0.096335}%
\pgfsetfillcolor{currentfill}%
\pgfsetlinewidth{0.000000pt}%
\definecolor{currentstroke}{rgb}{0.000000,0.000000,0.000000}%
\pgfsetstrokecolor{currentstroke}%
\pgfsetdash{}{0pt}%
\pgfpathmoveto{\pgfqpoint{1.114184in}{1.607386in}}%
\pgfpathlineto{\pgfqpoint{1.110894in}{1.603695in}}%
\pgfpathlineto{\pgfqpoint{1.107604in}{1.599879in}}%
\pgfpathlineto{\pgfqpoint{1.104314in}{1.595936in}}%
\pgfpathlineto{\pgfqpoint{1.101025in}{1.591869in}}%
\pgfpathlineto{\pgfqpoint{1.099466in}{1.593052in}}%
\pgfpathlineto{\pgfqpoint{1.097988in}{1.594258in}}%
\pgfpathlineto{\pgfqpoint{1.096593in}{1.595485in}}%
\pgfpathlineto{\pgfqpoint{1.095281in}{1.596731in}}%
\pgfpathlineto{\pgfqpoint{1.098808in}{1.600593in}}%
\pgfpathlineto{\pgfqpoint{1.102334in}{1.604331in}}%
\pgfpathlineto{\pgfqpoint{1.105862in}{1.607943in}}%
\pgfpathlineto{\pgfqpoint{1.109390in}{1.611428in}}%
\pgfpathlineto{\pgfqpoint{1.110485in}{1.610392in}}%
\pgfpathlineto{\pgfqpoint{1.111650in}{1.609372in}}%
\pgfpathlineto{\pgfqpoint{1.112883in}{1.608369in}}%
\pgfpathlineto{\pgfqpoint{1.114184in}{1.607386in}}%
\pgfpathclose%
\pgfusepath{fill}%
\end{pgfscope}%
\begin{pgfscope}%
\pgfpathrectangle{\pgfqpoint{0.041670in}{0.041670in}}{\pgfqpoint{2.216660in}{2.216660in}}%
\pgfusepath{clip}%
\pgfsetbuttcap%
\pgfsetroundjoin%
\definecolor{currentfill}{rgb}{0.267004,0.004874,0.329415}%
\pgfsetfillcolor{currentfill}%
\pgfsetlinewidth{0.000000pt}%
\definecolor{currentstroke}{rgb}{0.000000,0.000000,0.000000}%
\pgfsetstrokecolor{currentstroke}%
\pgfsetdash{}{0pt}%
\pgfpathmoveto{\pgfqpoint{0.967427in}{0.586723in}}%
\pgfpathlineto{\pgfqpoint{0.965980in}{0.584002in}}%
\pgfpathlineto{\pgfqpoint{0.964529in}{0.581521in}}%
\pgfpathlineto{\pgfqpoint{0.963076in}{0.579287in}}%
\pgfpathlineto{\pgfqpoint{0.961619in}{0.577304in}}%
\pgfpathlineto{\pgfqpoint{0.943804in}{0.581203in}}%
\pgfpathlineto{\pgfqpoint{0.926254in}{0.585404in}}%
\pgfpathlineto{\pgfqpoint{0.908987in}{0.589900in}}%
\pgfpathlineto{\pgfqpoint{0.892022in}{0.594687in}}%
\pgfpathlineto{\pgfqpoint{0.893940in}{0.596550in}}%
\pgfpathlineto{\pgfqpoint{0.895852in}{0.598664in}}%
\pgfpathlineto{\pgfqpoint{0.897760in}{0.601023in}}%
\pgfpathlineto{\pgfqpoint{0.899665in}{0.603624in}}%
\pgfpathlineto{\pgfqpoint{0.916184in}{0.598970in}}%
\pgfpathlineto{\pgfqpoint{0.932996in}{0.594598in}}%
\pgfpathlineto{\pgfqpoint{0.950083in}{0.590514in}}%
\pgfpathlineto{\pgfqpoint{0.967427in}{0.586723in}}%
\pgfpathclose%
\pgfusepath{fill}%
\end{pgfscope}%
\begin{pgfscope}%
\pgfpathrectangle{\pgfqpoint{0.041670in}{0.041670in}}{\pgfqpoint{2.216660in}{2.216660in}}%
\pgfusepath{clip}%
\pgfsetbuttcap%
\pgfsetroundjoin%
\definecolor{currentfill}{rgb}{0.283072,0.130895,0.449241}%
\pgfsetfillcolor{currentfill}%
\pgfsetlinewidth{0.000000pt}%
\definecolor{currentstroke}{rgb}{0.000000,0.000000,0.000000}%
\pgfsetstrokecolor{currentstroke}%
\pgfsetdash{}{0pt}%
\pgfpathmoveto{\pgfqpoint{1.123781in}{0.696462in}}%
\pgfpathlineto{\pgfqpoint{1.123341in}{0.689174in}}%
\pgfpathlineto{\pgfqpoint{1.122901in}{0.682020in}}%
\pgfpathlineto{\pgfqpoint{1.122459in}{0.675007in}}%
\pgfpathlineto{\pgfqpoint{1.122018in}{0.668137in}}%
\pgfpathlineto{\pgfqpoint{1.105834in}{0.669257in}}%
\pgfpathlineto{\pgfqpoint{1.089733in}{0.670650in}}%
\pgfpathlineto{\pgfqpoint{1.073731in}{0.672315in}}%
\pgfpathlineto{\pgfqpoint{1.057847in}{0.674249in}}%
\pgfpathlineto{\pgfqpoint{1.058777in}{0.681067in}}%
\pgfpathlineto{\pgfqpoint{1.059706in}{0.688028in}}%
\pgfpathlineto{\pgfqpoint{1.060634in}{0.695129in}}%
\pgfpathlineto{\pgfqpoint{1.061561in}{0.702366in}}%
\pgfpathlineto{\pgfqpoint{1.076963in}{0.700497in}}%
\pgfpathlineto{\pgfqpoint{1.092478in}{0.698889in}}%
\pgfpathlineto{\pgfqpoint{1.108090in}{0.697544in}}%
\pgfpathlineto{\pgfqpoint{1.123781in}{0.696462in}}%
\pgfpathclose%
\pgfusepath{fill}%
\end{pgfscope}%
\begin{pgfscope}%
\pgfpathrectangle{\pgfqpoint{0.041670in}{0.041670in}}{\pgfqpoint{2.216660in}{2.216660in}}%
\pgfusepath{clip}%
\pgfsetbuttcap%
\pgfsetroundjoin%
\definecolor{currentfill}{rgb}{0.195860,0.395433,0.555276}%
\pgfsetfillcolor{currentfill}%
\pgfsetlinewidth{0.000000pt}%
\definecolor{currentstroke}{rgb}{0.000000,0.000000,0.000000}%
\pgfsetstrokecolor{currentstroke}%
\pgfsetdash{}{0pt}%
\pgfpathmoveto{\pgfqpoint{1.329841in}{0.954642in}}%
\pgfpathlineto{\pgfqpoint{1.331323in}{0.945204in}}%
\pgfpathlineto{\pgfqpoint{1.332806in}{0.935804in}}%
\pgfpathlineto{\pgfqpoint{1.334288in}{0.926442in}}%
\pgfpathlineto{\pgfqpoint{1.335771in}{0.917124in}}%
\pgfpathlineto{\pgfqpoint{1.323874in}{0.914668in}}%
\pgfpathlineto{\pgfqpoint{1.311823in}{0.912407in}}%
\pgfpathlineto{\pgfqpoint{1.299629in}{0.910344in}}%
\pgfpathlineto{\pgfqpoint{1.287306in}{0.908482in}}%
\pgfpathlineto{\pgfqpoint{1.286283in}{0.917893in}}%
\pgfpathlineto{\pgfqpoint{1.285260in}{0.927348in}}%
\pgfpathlineto{\pgfqpoint{1.284238in}{0.936842in}}%
\pgfpathlineto{\pgfqpoint{1.283215in}{0.946372in}}%
\pgfpathlineto{\pgfqpoint{1.295070in}{0.948154in}}%
\pgfpathlineto{\pgfqpoint{1.306801in}{0.950127in}}%
\pgfpathlineto{\pgfqpoint{1.318395in}{0.952291in}}%
\pgfpathlineto{\pgfqpoint{1.329841in}{0.954642in}}%
\pgfpathclose%
\pgfusepath{fill}%
\end{pgfscope}%
\begin{pgfscope}%
\pgfpathrectangle{\pgfqpoint{0.041670in}{0.041670in}}{\pgfqpoint{2.216660in}{2.216660in}}%
\pgfusepath{clip}%
\pgfsetbuttcap%
\pgfsetroundjoin%
\definecolor{currentfill}{rgb}{0.993248,0.906157,0.143936}%
\pgfsetfillcolor{currentfill}%
\pgfsetlinewidth{0.000000pt}%
\definecolor{currentstroke}{rgb}{0.000000,0.000000,0.000000}%
\pgfsetstrokecolor{currentstroke}%
\pgfsetdash{}{0pt}%
\pgfpathmoveto{\pgfqpoint{1.184080in}{1.645671in}}%
\pgfpathlineto{\pgfqpoint{1.185113in}{1.643308in}}%
\pgfpathlineto{\pgfqpoint{1.186146in}{1.640810in}}%
\pgfpathlineto{\pgfqpoint{1.187179in}{1.638179in}}%
\pgfpathlineto{\pgfqpoint{1.188213in}{1.635413in}}%
\pgfpathlineto{\pgfqpoint{1.187255in}{1.635298in}}%
\pgfpathlineto{\pgfqpoint{1.186289in}{1.635197in}}%
\pgfpathlineto{\pgfqpoint{1.185317in}{1.635111in}}%
\pgfpathlineto{\pgfqpoint{1.184339in}{1.635039in}}%
\pgfpathlineto{\pgfqpoint{1.183790in}{1.637852in}}%
\pgfpathlineto{\pgfqpoint{1.183241in}{1.640531in}}%
\pgfpathlineto{\pgfqpoint{1.182693in}{1.643075in}}%
\pgfpathlineto{\pgfqpoint{1.182145in}{1.645484in}}%
\pgfpathlineto{\pgfqpoint{1.182633in}{1.645520in}}%
\pgfpathlineto{\pgfqpoint{1.183119in}{1.645563in}}%
\pgfpathlineto{\pgfqpoint{1.183601in}{1.645613in}}%
\pgfpathlineto{\pgfqpoint{1.184080in}{1.645671in}}%
\pgfpathclose%
\pgfusepath{fill}%
\end{pgfscope}%
\begin{pgfscope}%
\pgfpathrectangle{\pgfqpoint{0.041670in}{0.041670in}}{\pgfqpoint{2.216660in}{2.216660in}}%
\pgfusepath{clip}%
\pgfsetbuttcap%
\pgfsetroundjoin%
\definecolor{currentfill}{rgb}{0.993248,0.906157,0.143936}%
\pgfsetfillcolor{currentfill}%
\pgfsetlinewidth{0.000000pt}%
\definecolor{currentstroke}{rgb}{0.000000,0.000000,0.000000}%
\pgfsetstrokecolor{currentstroke}%
\pgfsetdash{}{0pt}%
\pgfpathmoveto{\pgfqpoint{1.178201in}{1.645459in}}%
\pgfpathlineto{\pgfqpoint{1.177762in}{1.643043in}}%
\pgfpathlineto{\pgfqpoint{1.177322in}{1.640492in}}%
\pgfpathlineto{\pgfqpoint{1.176883in}{1.637807in}}%
\pgfpathlineto{\pgfqpoint{1.176443in}{1.634987in}}%
\pgfpathlineto{\pgfqpoint{1.175462in}{1.635046in}}%
\pgfpathlineto{\pgfqpoint{1.174485in}{1.635120in}}%
\pgfpathlineto{\pgfqpoint{1.173513in}{1.635208in}}%
\pgfpathlineto{\pgfqpoint{1.172548in}{1.635310in}}%
\pgfpathlineto{\pgfqpoint{1.173476in}{1.638089in}}%
\pgfpathlineto{\pgfqpoint{1.174402in}{1.640733in}}%
\pgfpathlineto{\pgfqpoint{1.175329in}{1.643244in}}%
\pgfpathlineto{\pgfqpoint{1.176255in}{1.645619in}}%
\pgfpathlineto{\pgfqpoint{1.176737in}{1.645568in}}%
\pgfpathlineto{\pgfqpoint{1.177222in}{1.645525in}}%
\pgfpathlineto{\pgfqpoint{1.177711in}{1.645488in}}%
\pgfpathlineto{\pgfqpoint{1.178201in}{1.645459in}}%
\pgfpathclose%
\pgfusepath{fill}%
\end{pgfscope}%
\begin{pgfscope}%
\pgfpathrectangle{\pgfqpoint{0.041670in}{0.041670in}}{\pgfqpoint{2.216660in}{2.216660in}}%
\pgfusepath{clip}%
\pgfsetbuttcap%
\pgfsetroundjoin%
\definecolor{currentfill}{rgb}{0.993248,0.906157,0.143936}%
\pgfsetfillcolor{currentfill}%
\pgfsetlinewidth{0.000000pt}%
\definecolor{currentstroke}{rgb}{0.000000,0.000000,0.000000}%
\pgfsetstrokecolor{currentstroke}%
\pgfsetdash{}{0pt}%
\pgfpathmoveto{\pgfqpoint{1.182145in}{1.645484in}}%
\pgfpathlineto{\pgfqpoint{1.182693in}{1.643075in}}%
\pgfpathlineto{\pgfqpoint{1.183241in}{1.640531in}}%
\pgfpathlineto{\pgfqpoint{1.183790in}{1.637852in}}%
\pgfpathlineto{\pgfqpoint{1.184339in}{1.635039in}}%
\pgfpathlineto{\pgfqpoint{1.183357in}{1.634982in}}%
\pgfpathlineto{\pgfqpoint{1.182372in}{1.634939in}}%
\pgfpathlineto{\pgfqpoint{1.181384in}{1.634910in}}%
\pgfpathlineto{\pgfqpoint{1.180395in}{1.634897in}}%
\pgfpathlineto{\pgfqpoint{1.180340in}{1.637727in}}%
\pgfpathlineto{\pgfqpoint{1.180285in}{1.640424in}}%
\pgfpathlineto{\pgfqpoint{1.180230in}{1.642986in}}%
\pgfpathlineto{\pgfqpoint{1.180175in}{1.645413in}}%
\pgfpathlineto{\pgfqpoint{1.180669in}{1.645420in}}%
\pgfpathlineto{\pgfqpoint{1.181162in}{1.645434in}}%
\pgfpathlineto{\pgfqpoint{1.181654in}{1.645456in}}%
\pgfpathlineto{\pgfqpoint{1.182145in}{1.645484in}}%
\pgfpathclose%
\pgfusepath{fill}%
\end{pgfscope}%
\begin{pgfscope}%
\pgfpathrectangle{\pgfqpoint{0.041670in}{0.041670in}}{\pgfqpoint{2.216660in}{2.216660in}}%
\pgfusepath{clip}%
\pgfsetbuttcap%
\pgfsetroundjoin%
\definecolor{currentfill}{rgb}{0.993248,0.906157,0.143936}%
\pgfsetfillcolor{currentfill}%
\pgfsetlinewidth{0.000000pt}%
\definecolor{currentstroke}{rgb}{0.000000,0.000000,0.000000}%
\pgfsetstrokecolor{currentstroke}%
\pgfsetdash{}{0pt}%
\pgfpathmoveto{\pgfqpoint{1.180175in}{1.645413in}}%
\pgfpathlineto{\pgfqpoint{1.180230in}{1.642986in}}%
\pgfpathlineto{\pgfqpoint{1.180285in}{1.640424in}}%
\pgfpathlineto{\pgfqpoint{1.180340in}{1.637727in}}%
\pgfpathlineto{\pgfqpoint{1.180395in}{1.634897in}}%
\pgfpathlineto{\pgfqpoint{1.179405in}{1.634897in}}%
\pgfpathlineto{\pgfqpoint{1.178416in}{1.634913in}}%
\pgfpathlineto{\pgfqpoint{1.177428in}{1.634943in}}%
\pgfpathlineto{\pgfqpoint{1.176443in}{1.634987in}}%
\pgfpathlineto{\pgfqpoint{1.176883in}{1.637807in}}%
\pgfpathlineto{\pgfqpoint{1.177322in}{1.640492in}}%
\pgfpathlineto{\pgfqpoint{1.177762in}{1.643043in}}%
\pgfpathlineto{\pgfqpoint{1.178201in}{1.645459in}}%
\pgfpathlineto{\pgfqpoint{1.178693in}{1.645436in}}%
\pgfpathlineto{\pgfqpoint{1.179186in}{1.645421in}}%
\pgfpathlineto{\pgfqpoint{1.179680in}{1.645414in}}%
\pgfpathlineto{\pgfqpoint{1.180175in}{1.645413in}}%
\pgfpathclose%
\pgfusepath{fill}%
\end{pgfscope}%
\begin{pgfscope}%
\pgfpathrectangle{\pgfqpoint{0.041670in}{0.041670in}}{\pgfqpoint{2.216660in}{2.216660in}}%
\pgfusepath{clip}%
\pgfsetbuttcap%
\pgfsetroundjoin%
\definecolor{currentfill}{rgb}{0.974417,0.903590,0.130215}%
\pgfsetfillcolor{currentfill}%
\pgfsetlinewidth{0.000000pt}%
\definecolor{currentstroke}{rgb}{0.000000,0.000000,0.000000}%
\pgfsetstrokecolor{currentstroke}%
\pgfsetdash{}{0pt}%
\pgfpathmoveto{\pgfqpoint{1.204451in}{1.640409in}}%
\pgfpathlineto{\pgfqpoint{1.207516in}{1.638137in}}%
\pgfpathlineto{\pgfqpoint{1.210582in}{1.635732in}}%
\pgfpathlineto{\pgfqpoint{1.213649in}{1.633194in}}%
\pgfpathlineto{\pgfqpoint{1.216716in}{1.630524in}}%
\pgfpathlineto{\pgfqpoint{1.215776in}{1.629988in}}%
\pgfpathlineto{\pgfqpoint{1.214800in}{1.629467in}}%
\pgfpathlineto{\pgfqpoint{1.213789in}{1.628960in}}%
\pgfpathlineto{\pgfqpoint{1.212745in}{1.628468in}}%
\pgfpathlineto{\pgfqpoint{1.210008in}{1.631310in}}%
\pgfpathlineto{\pgfqpoint{1.207272in}{1.634020in}}%
\pgfpathlineto{\pgfqpoint{1.204537in}{1.636597in}}%
\pgfpathlineto{\pgfqpoint{1.201803in}{1.639041in}}%
\pgfpathlineto{\pgfqpoint{1.202499in}{1.639369in}}%
\pgfpathlineto{\pgfqpoint{1.203173in}{1.639706in}}%
\pgfpathlineto{\pgfqpoint{1.203824in}{1.640053in}}%
\pgfpathlineto{\pgfqpoint{1.204451in}{1.640409in}}%
\pgfpathclose%
\pgfusepath{fill}%
\end{pgfscope}%
\begin{pgfscope}%
\pgfpathrectangle{\pgfqpoint{0.041670in}{0.041670in}}{\pgfqpoint{2.216660in}{2.216660in}}%
\pgfusepath{clip}%
\pgfsetbuttcap%
\pgfsetroundjoin%
\definecolor{currentfill}{rgb}{0.147607,0.511733,0.557049}%
\pgfsetfillcolor{currentfill}%
\pgfsetlinewidth{0.000000pt}%
\definecolor{currentstroke}{rgb}{0.000000,0.000000,0.000000}%
\pgfsetstrokecolor{currentstroke}%
\pgfsetdash{}{0pt}%
\pgfpathmoveto{\pgfqpoint{1.350885in}{1.079328in}}%
\pgfpathlineto{\pgfqpoint{1.352805in}{1.069816in}}%
\pgfpathlineto{\pgfqpoint{1.354724in}{1.060305in}}%
\pgfpathlineto{\pgfqpoint{1.356643in}{1.050797in}}%
\pgfpathlineto{\pgfqpoint{1.358561in}{1.041295in}}%
\pgfpathlineto{\pgfqpoint{1.348673in}{1.038498in}}%
\pgfpathlineto{\pgfqpoint{1.338606in}{1.035860in}}%
\pgfpathlineto{\pgfqpoint{1.328369in}{1.033385in}}%
\pgfpathlineto{\pgfqpoint{1.317974in}{1.031076in}}%
\pgfpathlineto{\pgfqpoint{1.316490in}{1.040701in}}%
\pgfpathlineto{\pgfqpoint{1.315005in}{1.050332in}}%
\pgfpathlineto{\pgfqpoint{1.313520in}{1.059966in}}%
\pgfpathlineto{\pgfqpoint{1.312035in}{1.069601in}}%
\pgfpathlineto{\pgfqpoint{1.321984in}{1.071799in}}%
\pgfpathlineto{\pgfqpoint{1.331783in}{1.074154in}}%
\pgfpathlineto{\pgfqpoint{1.341420in}{1.076665in}}%
\pgfpathlineto{\pgfqpoint{1.350885in}{1.079328in}}%
\pgfpathclose%
\pgfusepath{fill}%
\end{pgfscope}%
\begin{pgfscope}%
\pgfpathrectangle{\pgfqpoint{0.041670in}{0.041670in}}{\pgfqpoint{2.216660in}{2.216660in}}%
\pgfusepath{clip}%
\pgfsetbuttcap%
\pgfsetroundjoin%
\definecolor{currentfill}{rgb}{0.263663,0.237631,0.518762}%
\pgfsetfillcolor{currentfill}%
\pgfsetlinewidth{0.000000pt}%
\definecolor{currentstroke}{rgb}{0.000000,0.000000,0.000000}%
\pgfsetstrokecolor{currentstroke}%
\pgfsetdash{}{0pt}%
\pgfpathmoveto{\pgfqpoint{1.129040in}{0.793118in}}%
\pgfpathlineto{\pgfqpoint{1.128603in}{0.784498in}}%
\pgfpathlineto{\pgfqpoint{1.128166in}{0.775969in}}%
\pgfpathlineto{\pgfqpoint{1.127729in}{0.767534in}}%
\pgfpathlineto{\pgfqpoint{1.127291in}{0.759196in}}%
\pgfpathlineto{\pgfqpoint{1.112580in}{0.760201in}}%
\pgfpathlineto{\pgfqpoint{1.097942in}{0.761452in}}%
\pgfpathlineto{\pgfqpoint{1.083394in}{0.762947in}}%
\pgfpathlineto{\pgfqpoint{1.068953in}{0.764684in}}%
\pgfpathlineto{\pgfqpoint{1.069874in}{0.772970in}}%
\pgfpathlineto{\pgfqpoint{1.070795in}{0.781353in}}%
\pgfpathlineto{\pgfqpoint{1.071715in}{0.789831in}}%
\pgfpathlineto{\pgfqpoint{1.072635in}{0.798399in}}%
\pgfpathlineto{\pgfqpoint{1.086599in}{0.796727in}}%
\pgfpathlineto{\pgfqpoint{1.100664in}{0.795289in}}%
\pgfpathlineto{\pgfqpoint{1.114816in}{0.794085in}}%
\pgfpathlineto{\pgfqpoint{1.129040in}{0.793118in}}%
\pgfpathclose%
\pgfusepath{fill}%
\end{pgfscope}%
\begin{pgfscope}%
\pgfpathrectangle{\pgfqpoint{0.041670in}{0.041670in}}{\pgfqpoint{2.216660in}{2.216660in}}%
\pgfusepath{clip}%
\pgfsetbuttcap%
\pgfsetroundjoin%
\definecolor{currentfill}{rgb}{0.412913,0.803041,0.357269}%
\pgfsetfillcolor{currentfill}%
\pgfsetlinewidth{0.000000pt}%
\definecolor{currentstroke}{rgb}{0.000000,0.000000,0.000000}%
\pgfsetstrokecolor{currentstroke}%
\pgfsetdash{}{0pt}%
\pgfpathmoveto{\pgfqpoint{1.339166in}{1.423894in}}%
\pgfpathlineto{\pgfqpoint{1.342209in}{1.416562in}}%
\pgfpathlineto{\pgfqpoint{1.345250in}{1.409145in}}%
\pgfpathlineto{\pgfqpoint{1.348290in}{1.401645in}}%
\pgfpathlineto{\pgfqpoint{1.351329in}{1.394065in}}%
\pgfpathlineto{\pgfqpoint{1.346998in}{1.391475in}}%
\pgfpathlineto{\pgfqpoint{1.342496in}{1.388953in}}%
\pgfpathlineto{\pgfqpoint{1.337826in}{1.386499in}}%
\pgfpathlineto{\pgfqpoint{1.332994in}{1.384118in}}%
\pgfpathlineto{\pgfqpoint{1.330277in}{1.391886in}}%
\pgfpathlineto{\pgfqpoint{1.327558in}{1.399572in}}%
\pgfpathlineto{\pgfqpoint{1.324838in}{1.407176in}}%
\pgfpathlineto{\pgfqpoint{1.322118in}{1.414695in}}%
\pgfpathlineto{\pgfqpoint{1.326610in}{1.416898in}}%
\pgfpathlineto{\pgfqpoint{1.330952in}{1.419167in}}%
\pgfpathlineto{\pgfqpoint{1.335139in}{1.421500in}}%
\pgfpathlineto{\pgfqpoint{1.339166in}{1.423894in}}%
\pgfpathclose%
\pgfusepath{fill}%
\end{pgfscope}%
\begin{pgfscope}%
\pgfpathrectangle{\pgfqpoint{0.041670in}{0.041670in}}{\pgfqpoint{2.216660in}{2.216660in}}%
\pgfusepath{clip}%
\pgfsetbuttcap%
\pgfsetroundjoin%
\definecolor{currentfill}{rgb}{0.248629,0.278775,0.534556}%
\pgfsetfillcolor{currentfill}%
\pgfsetlinewidth{0.000000pt}%
\definecolor{currentstroke}{rgb}{0.000000,0.000000,0.000000}%
\pgfsetstrokecolor{currentstroke}%
\pgfsetdash{}{0pt}%
\pgfpathmoveto{\pgfqpoint{1.295490in}{0.835120in}}%
\pgfpathlineto{\pgfqpoint{1.296514in}{0.826242in}}%
\pgfpathlineto{\pgfqpoint{1.297539in}{0.817440in}}%
\pgfpathlineto{\pgfqpoint{1.298563in}{0.808718in}}%
\pgfpathlineto{\pgfqpoint{1.299588in}{0.800079in}}%
\pgfpathlineto{\pgfqpoint{1.285728in}{0.798202in}}%
\pgfpathlineto{\pgfqpoint{1.271753in}{0.796556in}}%
\pgfpathlineto{\pgfqpoint{1.257677in}{0.795143in}}%
\pgfpathlineto{\pgfqpoint{1.243516in}{0.793966in}}%
\pgfpathlineto{\pgfqpoint{1.242971in}{0.802664in}}%
\pgfpathlineto{\pgfqpoint{1.242426in}{0.811446in}}%
\pgfpathlineto{\pgfqpoint{1.241882in}{0.820307in}}%
\pgfpathlineto{\pgfqpoint{1.241337in}{0.829244in}}%
\pgfpathlineto{\pgfqpoint{1.255013in}{0.830376in}}%
\pgfpathlineto{\pgfqpoint{1.268607in}{0.831734in}}%
\pgfpathlineto{\pgfqpoint{1.282104in}{0.833315in}}%
\pgfpathlineto{\pgfqpoint{1.295490in}{0.835120in}}%
\pgfpathclose%
\pgfusepath{fill}%
\end{pgfscope}%
\begin{pgfscope}%
\pgfpathrectangle{\pgfqpoint{0.041670in}{0.041670in}}{\pgfqpoint{2.216660in}{2.216660in}}%
\pgfusepath{clip}%
\pgfsetbuttcap%
\pgfsetroundjoin%
\definecolor{currentfill}{rgb}{0.974417,0.903590,0.130215}%
\pgfsetfillcolor{currentfill}%
\pgfsetlinewidth{0.000000pt}%
\definecolor{currentstroke}{rgb}{0.000000,0.000000,0.000000}%
\pgfsetstrokecolor{currentstroke}%
\pgfsetdash{}{0pt}%
\pgfpathmoveto{\pgfqpoint{1.158745in}{1.638759in}}%
\pgfpathlineto{\pgfqpoint{1.156090in}{1.636280in}}%
\pgfpathlineto{\pgfqpoint{1.153434in}{1.633667in}}%
\pgfpathlineto{\pgfqpoint{1.150778in}{1.630922in}}%
\pgfpathlineto{\pgfqpoint{1.148121in}{1.628044in}}%
\pgfpathlineto{\pgfqpoint{1.147047in}{1.628522in}}%
\pgfpathlineto{\pgfqpoint{1.146006in}{1.629015in}}%
\pgfpathlineto{\pgfqpoint{1.144999in}{1.629524in}}%
\pgfpathlineto{\pgfqpoint{1.144027in}{1.630047in}}%
\pgfpathlineto{\pgfqpoint{1.147025in}{1.632757in}}%
\pgfpathlineto{\pgfqpoint{1.150022in}{1.635335in}}%
\pgfpathlineto{\pgfqpoint{1.153019in}{1.637780in}}%
\pgfpathlineto{\pgfqpoint{1.156015in}{1.640092in}}%
\pgfpathlineto{\pgfqpoint{1.156663in}{1.639744in}}%
\pgfpathlineto{\pgfqpoint{1.157335in}{1.639406in}}%
\pgfpathlineto{\pgfqpoint{1.158029in}{1.639077in}}%
\pgfpathlineto{\pgfqpoint{1.158745in}{1.638759in}}%
\pgfpathclose%
\pgfusepath{fill}%
\end{pgfscope}%
\begin{pgfscope}%
\pgfpathrectangle{\pgfqpoint{0.041670in}{0.041670in}}{\pgfqpoint{2.216660in}{2.216660in}}%
\pgfusepath{clip}%
\pgfsetbuttcap%
\pgfsetroundjoin%
\definecolor{currentfill}{rgb}{0.120081,0.622161,0.534946}%
\pgfsetfillcolor{currentfill}%
\pgfsetlinewidth{0.000000pt}%
\definecolor{currentstroke}{rgb}{0.000000,0.000000,0.000000}%
\pgfsetstrokecolor{currentstroke}%
\pgfsetdash{}{0pt}%
\pgfpathmoveto{\pgfqpoint{1.358941in}{1.202781in}}%
\pgfpathlineto{\pgfqpoint{1.361271in}{1.193630in}}%
\pgfpathlineto{\pgfqpoint{1.363600in}{1.184447in}}%
\pgfpathlineto{\pgfqpoint{1.365928in}{1.175234in}}%
\pgfpathlineto{\pgfqpoint{1.368255in}{1.165995in}}%
\pgfpathlineto{\pgfqpoint{1.360344in}{1.163078in}}%
\pgfpathlineto{\pgfqpoint{1.352243in}{1.160287in}}%
\pgfpathlineto{\pgfqpoint{1.343962in}{1.157624in}}%
\pgfpathlineto{\pgfqpoint{1.335507in}{1.155093in}}%
\pgfpathlineto{\pgfqpoint{1.333582in}{1.164481in}}%
\pgfpathlineto{\pgfqpoint{1.331656in}{1.173842in}}%
\pgfpathlineto{\pgfqpoint{1.329729in}{1.183174in}}%
\pgfpathlineto{\pgfqpoint{1.327802in}{1.192473in}}%
\pgfpathlineto{\pgfqpoint{1.335840in}{1.194866in}}%
\pgfpathlineto{\pgfqpoint{1.343715in}{1.197383in}}%
\pgfpathlineto{\pgfqpoint{1.351418in}{1.200023in}}%
\pgfpathlineto{\pgfqpoint{1.358941in}{1.202781in}}%
\pgfpathclose%
\pgfusepath{fill}%
\end{pgfscope}%
\begin{pgfscope}%
\pgfpathrectangle{\pgfqpoint{0.041670in}{0.041670in}}{\pgfqpoint{2.216660in}{2.216660in}}%
\pgfusepath{clip}%
\pgfsetbuttcap%
\pgfsetroundjoin%
\definecolor{currentfill}{rgb}{0.955300,0.901065,0.118128}%
\pgfsetfillcolor{currentfill}%
\pgfsetlinewidth{0.000000pt}%
\definecolor{currentstroke}{rgb}{0.000000,0.000000,0.000000}%
\pgfsetstrokecolor{currentstroke}%
\pgfsetdash{}{0pt}%
\pgfpathmoveto{\pgfqpoint{1.220098in}{1.632796in}}%
\pgfpathlineto{\pgfqpoint{1.223446in}{1.630185in}}%
\pgfpathlineto{\pgfqpoint{1.226794in}{1.627442in}}%
\pgfpathlineto{\pgfqpoint{1.230143in}{1.624569in}}%
\pgfpathlineto{\pgfqpoint{1.233492in}{1.621565in}}%
\pgfpathlineto{\pgfqpoint{1.232443in}{1.620781in}}%
\pgfpathlineto{\pgfqpoint{1.231342in}{1.620014in}}%
\pgfpathlineto{\pgfqpoint{1.230190in}{1.619262in}}%
\pgfpathlineto{\pgfqpoint{1.228987in}{1.618529in}}%
\pgfpathlineto{\pgfqpoint{1.225919in}{1.621724in}}%
\pgfpathlineto{\pgfqpoint{1.222851in}{1.624788in}}%
\pgfpathlineto{\pgfqpoint{1.219783in}{1.627722in}}%
\pgfpathlineto{\pgfqpoint{1.216716in}{1.630524in}}%
\pgfpathlineto{\pgfqpoint{1.217619in}{1.631073in}}%
\pgfpathlineto{\pgfqpoint{1.218484in}{1.631635in}}%
\pgfpathlineto{\pgfqpoint{1.219311in}{1.632210in}}%
\pgfpathlineto{\pgfqpoint{1.220098in}{1.632796in}}%
\pgfpathclose%
\pgfusepath{fill}%
\end{pgfscope}%
\begin{pgfscope}%
\pgfpathrectangle{\pgfqpoint{0.041670in}{0.041670in}}{\pgfqpoint{2.216660in}{2.216660in}}%
\pgfusepath{clip}%
\pgfsetbuttcap%
\pgfsetroundjoin%
\definecolor{currentfill}{rgb}{0.220124,0.725509,0.466226}%
\pgfsetfillcolor{currentfill}%
\pgfsetlinewidth{0.000000pt}%
\definecolor{currentstroke}{rgb}{0.000000,0.000000,0.000000}%
\pgfsetstrokecolor{currentstroke}%
\pgfsetdash{}{0pt}%
\pgfpathmoveto{\pgfqpoint{1.354690in}{1.319294in}}%
\pgfpathlineto{\pgfqpoint{1.357397in}{1.310887in}}%
\pgfpathlineto{\pgfqpoint{1.360102in}{1.302420in}}%
\pgfpathlineto{\pgfqpoint{1.362806in}{1.293895in}}%
\pgfpathlineto{\pgfqpoint{1.365509in}{1.285314in}}%
\pgfpathlineto{\pgfqpoint{1.359472in}{1.282475in}}%
\pgfpathlineto{\pgfqpoint{1.353249in}{1.279731in}}%
\pgfpathlineto{\pgfqpoint{1.346845in}{1.277084in}}%
\pgfpathlineto{\pgfqpoint{1.340268in}{1.274536in}}%
\pgfpathlineto{\pgfqpoint{1.337929in}{1.283288in}}%
\pgfpathlineto{\pgfqpoint{1.335589in}{1.291983in}}%
\pgfpathlineto{\pgfqpoint{1.333249in}{1.300620in}}%
\pgfpathlineto{\pgfqpoint{1.330907in}{1.309196in}}%
\pgfpathlineto{\pgfqpoint{1.337104in}{1.311583in}}%
\pgfpathlineto{\pgfqpoint{1.343137in}{1.314063in}}%
\pgfpathlineto{\pgfqpoint{1.349001in}{1.316634in}}%
\pgfpathlineto{\pgfqpoint{1.354690in}{1.319294in}}%
\pgfpathclose%
\pgfusepath{fill}%
\end{pgfscope}%
\begin{pgfscope}%
\pgfpathrectangle{\pgfqpoint{0.041670in}{0.041670in}}{\pgfqpoint{2.216660in}{2.216660in}}%
\pgfusepath{clip}%
\pgfsetbuttcap%
\pgfsetroundjoin%
\definecolor{currentfill}{rgb}{0.276194,0.190074,0.493001}%
\pgfsetfillcolor{currentfill}%
\pgfsetlinewidth{0.000000pt}%
\definecolor{currentstroke}{rgb}{0.000000,0.000000,0.000000}%
\pgfsetstrokecolor{currentstroke}%
\pgfsetdash{}{0pt}%
\pgfpathmoveto{\pgfqpoint{1.729387in}{0.737203in}}%
\pgfpathlineto{\pgfqpoint{1.732755in}{0.743930in}}%
\pgfpathlineto{\pgfqpoint{1.736137in}{0.751057in}}%
\pgfpathlineto{\pgfqpoint{1.739534in}{0.758591in}}%
\pgfpathlineto{\pgfqpoint{1.742947in}{0.766538in}}%
\pgfpathlineto{\pgfqpoint{1.729204in}{0.757227in}}%
\pgfpathlineto{\pgfqpoint{1.714860in}{0.748140in}}%
\pgfpathlineto{\pgfqpoint{1.699929in}{0.739288in}}%
\pgfpathlineto{\pgfqpoint{1.684424in}{0.730680in}}%
\pgfpathlineto{\pgfqpoint{1.681351in}{0.722913in}}%
\pgfpathlineto{\pgfqpoint{1.678293in}{0.715560in}}%
\pgfpathlineto{\pgfqpoint{1.675247in}{0.708616in}}%
\pgfpathlineto{\pgfqpoint{1.672216in}{0.702073in}}%
\pgfpathlineto{\pgfqpoint{1.687359in}{0.710505in}}%
\pgfpathlineto{\pgfqpoint{1.701945in}{0.719177in}}%
\pgfpathlineto{\pgfqpoint{1.715959in}{0.728080in}}%
\pgfpathlineto{\pgfqpoint{1.729387in}{0.737203in}}%
\pgfpathclose%
\pgfusepath{fill}%
\end{pgfscope}%
\begin{pgfscope}%
\pgfpathrectangle{\pgfqpoint{0.041670in}{0.041670in}}{\pgfqpoint{2.216660in}{2.216660in}}%
\pgfusepath{clip}%
\pgfsetbuttcap%
\pgfsetroundjoin%
\definecolor{currentfill}{rgb}{0.282327,0.094955,0.417331}%
\pgfsetfillcolor{currentfill}%
\pgfsetlinewidth{0.000000pt}%
\definecolor{currentstroke}{rgb}{0.000000,0.000000,0.000000}%
\pgfsetstrokecolor{currentstroke}%
\pgfsetdash{}{0pt}%
\pgfpathmoveto{\pgfqpoint{1.648401in}{0.663371in}}%
\pgfpathlineto{\pgfqpoint{1.651338in}{0.666942in}}%
\pgfpathlineto{\pgfqpoint{1.654286in}{0.670863in}}%
\pgfpathlineto{\pgfqpoint{1.657244in}{0.675139in}}%
\pgfpathlineto{\pgfqpoint{1.660215in}{0.679777in}}%
\pgfpathlineto{\pgfqpoint{1.644897in}{0.671772in}}%
\pgfpathlineto{\pgfqpoint{1.629067in}{0.664022in}}%
\pgfpathlineto{\pgfqpoint{1.612738in}{0.656537in}}%
\pgfpathlineto{\pgfqpoint{1.595929in}{0.649326in}}%
\pgfpathlineto{\pgfqpoint{1.593346in}{0.644859in}}%
\pgfpathlineto{\pgfqpoint{1.590773in}{0.640754in}}%
\pgfpathlineto{\pgfqpoint{1.588210in}{0.637005in}}%
\pgfpathlineto{\pgfqpoint{1.585656in}{0.633607in}}%
\pgfpathlineto{\pgfqpoint{1.602060in}{0.640654in}}%
\pgfpathlineto{\pgfqpoint{1.617996in}{0.647970in}}%
\pgfpathlineto{\pgfqpoint{1.633448in}{0.655545in}}%
\pgfpathlineto{\pgfqpoint{1.648401in}{0.663371in}}%
\pgfpathclose%
\pgfusepath{fill}%
\end{pgfscope}%
\begin{pgfscope}%
\pgfpathrectangle{\pgfqpoint{0.041670in}{0.041670in}}{\pgfqpoint{2.216660in}{2.216660in}}%
\pgfusepath{clip}%
\pgfsetbuttcap%
\pgfsetroundjoin%
\definecolor{currentfill}{rgb}{0.636902,0.856542,0.216620}%
\pgfsetfillcolor{currentfill}%
\pgfsetlinewidth{0.000000pt}%
\definecolor{currentstroke}{rgb}{0.000000,0.000000,0.000000}%
\pgfsetstrokecolor{currentstroke}%
\pgfsetdash{}{0pt}%
\pgfpathmoveto{\pgfqpoint{1.060115in}{1.503041in}}%
\pgfpathlineto{\pgfqpoint{1.057126in}{1.496814in}}%
\pgfpathlineto{\pgfqpoint{1.054138in}{1.490482in}}%
\pgfpathlineto{\pgfqpoint{1.051150in}{1.484048in}}%
\pgfpathlineto{\pgfqpoint{1.048164in}{1.477512in}}%
\pgfpathlineto{\pgfqpoint{1.044761in}{1.479529in}}%
\pgfpathlineto{\pgfqpoint{1.041496in}{1.481595in}}%
\pgfpathlineto{\pgfqpoint{1.038372in}{1.483709in}}%
\pgfpathlineto{\pgfqpoint{1.035393in}{1.485868in}}%
\pgfpathlineto{\pgfqpoint{1.038664in}{1.492207in}}%
\pgfpathlineto{\pgfqpoint{1.041937in}{1.498445in}}%
\pgfpathlineto{\pgfqpoint{1.045211in}{1.504580in}}%
\pgfpathlineto{\pgfqpoint{1.048486in}{1.510611in}}%
\pgfpathlineto{\pgfqpoint{1.051200in}{1.508655in}}%
\pgfpathlineto{\pgfqpoint{1.054045in}{1.506739in}}%
\pgfpathlineto{\pgfqpoint{1.057017in}{1.504868in}}%
\pgfpathlineto{\pgfqpoint{1.060115in}{1.503041in}}%
\pgfpathclose%
\pgfusepath{fill}%
\end{pgfscope}%
\begin{pgfscope}%
\pgfpathrectangle{\pgfqpoint{0.041670in}{0.041670in}}{\pgfqpoint{2.216660in}{2.216660in}}%
\pgfusepath{clip}%
\pgfsetbuttcap%
\pgfsetroundjoin%
\definecolor{currentfill}{rgb}{0.282327,0.094955,0.417331}%
\pgfsetfillcolor{currentfill}%
\pgfsetlinewidth{0.000000pt}%
\definecolor{currentstroke}{rgb}{0.000000,0.000000,0.000000}%
\pgfsetstrokecolor{currentstroke}%
\pgfsetdash{}{0pt}%
\pgfpathmoveto{\pgfqpoint{1.122018in}{0.668137in}}%
\pgfpathlineto{\pgfqpoint{1.121576in}{0.661416in}}%
\pgfpathlineto{\pgfqpoint{1.121133in}{0.654846in}}%
\pgfpathlineto{\pgfqpoint{1.120690in}{0.648432in}}%
\pgfpathlineto{\pgfqpoint{1.120247in}{0.642178in}}%
\pgfpathlineto{\pgfqpoint{1.103569in}{0.643336in}}%
\pgfpathlineto{\pgfqpoint{1.086976in}{0.644777in}}%
\pgfpathlineto{\pgfqpoint{1.070486in}{0.646498in}}%
\pgfpathlineto{\pgfqpoint{1.054118in}{0.648499in}}%
\pgfpathlineto{\pgfqpoint{1.055052in}{0.654700in}}%
\pgfpathlineto{\pgfqpoint{1.055985in}{0.661062in}}%
\pgfpathlineto{\pgfqpoint{1.056917in}{0.667580in}}%
\pgfpathlineto{\pgfqpoint{1.057847in}{0.674249in}}%
\pgfpathlineto{\pgfqpoint{1.073731in}{0.672315in}}%
\pgfpathlineto{\pgfqpoint{1.089733in}{0.670650in}}%
\pgfpathlineto{\pgfqpoint{1.105834in}{0.669257in}}%
\pgfpathlineto{\pgfqpoint{1.122018in}{0.668137in}}%
\pgfpathclose%
\pgfusepath{fill}%
\end{pgfscope}%
\begin{pgfscope}%
\pgfpathrectangle{\pgfqpoint{0.041670in}{0.041670in}}{\pgfqpoint{2.216660in}{2.216660in}}%
\pgfusepath{clip}%
\pgfsetbuttcap%
\pgfsetroundjoin%
\definecolor{currentfill}{rgb}{0.699415,0.867117,0.175971}%
\pgfsetfillcolor{currentfill}%
\pgfsetlinewidth{0.000000pt}%
\definecolor{currentstroke}{rgb}{0.000000,0.000000,0.000000}%
\pgfsetstrokecolor{currentstroke}%
\pgfsetdash{}{0pt}%
\pgfpathmoveto{\pgfqpoint{1.300384in}{1.535255in}}%
\pgfpathlineto{\pgfqpoint{1.303721in}{1.529700in}}%
\pgfpathlineto{\pgfqpoint{1.307056in}{1.524035in}}%
\pgfpathlineto{\pgfqpoint{1.310391in}{1.518263in}}%
\pgfpathlineto{\pgfqpoint{1.313724in}{1.512383in}}%
\pgfpathlineto{\pgfqpoint{1.311128in}{1.510392in}}%
\pgfpathlineto{\pgfqpoint{1.308400in}{1.508440in}}%
\pgfpathlineto{\pgfqpoint{1.305541in}{1.506529in}}%
\pgfpathlineto{\pgfqpoint{1.302554in}{1.504662in}}%
\pgfpathlineto{\pgfqpoint{1.299496in}{1.510741in}}%
\pgfpathlineto{\pgfqpoint{1.296437in}{1.516713in}}%
\pgfpathlineto{\pgfqpoint{1.293377in}{1.522576in}}%
\pgfpathlineto{\pgfqpoint{1.290316in}{1.528329in}}%
\pgfpathlineto{\pgfqpoint{1.293008in}{1.530004in}}%
\pgfpathlineto{\pgfqpoint{1.295585in}{1.531718in}}%
\pgfpathlineto{\pgfqpoint{1.298044in}{1.533469in}}%
\pgfpathlineto{\pgfqpoint{1.300384in}{1.535255in}}%
\pgfpathclose%
\pgfusepath{fill}%
\end{pgfscope}%
\begin{pgfscope}%
\pgfpathrectangle{\pgfqpoint{0.041670in}{0.041670in}}{\pgfqpoint{2.216660in}{2.216660in}}%
\pgfusepath{clip}%
\pgfsetbuttcap%
\pgfsetroundjoin%
\definecolor{currentfill}{rgb}{0.955300,0.901065,0.118128}%
\pgfsetfillcolor{currentfill}%
\pgfsetlinewidth{0.000000pt}%
\definecolor{currentstroke}{rgb}{0.000000,0.000000,0.000000}%
\pgfsetstrokecolor{currentstroke}%
\pgfsetdash{}{0pt}%
\pgfpathmoveto{\pgfqpoint{1.144027in}{1.630047in}}%
\pgfpathlineto{\pgfqpoint{1.141029in}{1.627205in}}%
\pgfpathlineto{\pgfqpoint{1.138031in}{1.624231in}}%
\pgfpathlineto{\pgfqpoint{1.135032in}{1.621127in}}%
\pgfpathlineto{\pgfqpoint{1.132033in}{1.617891in}}%
\pgfpathlineto{\pgfqpoint{1.130787in}{1.618609in}}%
\pgfpathlineto{\pgfqpoint{1.129589in}{1.619345in}}%
\pgfpathlineto{\pgfqpoint{1.128442in}{1.620098in}}%
\pgfpathlineto{\pgfqpoint{1.127347in}{1.620868in}}%
\pgfpathlineto{\pgfqpoint{1.130638in}{1.623915in}}%
\pgfpathlineto{\pgfqpoint{1.133929in}{1.626833in}}%
\pgfpathlineto{\pgfqpoint{1.137219in}{1.629619in}}%
\pgfpathlineto{\pgfqpoint{1.140510in}{1.632274in}}%
\pgfpathlineto{\pgfqpoint{1.141332in}{1.631698in}}%
\pgfpathlineto{\pgfqpoint{1.142193in}{1.631135in}}%
\pgfpathlineto{\pgfqpoint{1.143092in}{1.630584in}}%
\pgfpathlineto{\pgfqpoint{1.144027in}{1.630047in}}%
\pgfpathclose%
\pgfusepath{fill}%
\end{pgfscope}%
\begin{pgfscope}%
\pgfpathrectangle{\pgfqpoint{0.041670in}{0.041670in}}{\pgfqpoint{2.216660in}{2.216660in}}%
\pgfusepath{clip}%
\pgfsetbuttcap%
\pgfsetroundjoin%
\definecolor{currentfill}{rgb}{0.248629,0.278775,0.534556}%
\pgfsetfillcolor{currentfill}%
\pgfsetlinewidth{0.000000pt}%
\definecolor{currentstroke}{rgb}{0.000000,0.000000,0.000000}%
\pgfsetstrokecolor{currentstroke}%
\pgfsetdash{}{0pt}%
\pgfpathmoveto{\pgfqpoint{1.130785in}{0.828429in}}%
\pgfpathlineto{\pgfqpoint{1.130349in}{0.819484in}}%
\pgfpathlineto{\pgfqpoint{1.129913in}{0.810614in}}%
\pgfpathlineto{\pgfqpoint{1.129476in}{0.801824in}}%
\pgfpathlineto{\pgfqpoint{1.129040in}{0.793118in}}%
\pgfpathlineto{\pgfqpoint{1.114816in}{0.794085in}}%
\pgfpathlineto{\pgfqpoint{1.100664in}{0.795289in}}%
\pgfpathlineto{\pgfqpoint{1.086599in}{0.796727in}}%
\pgfpathlineto{\pgfqpoint{1.072635in}{0.798399in}}%
\pgfpathlineto{\pgfqpoint{1.073555in}{0.807054in}}%
\pgfpathlineto{\pgfqpoint{1.074475in}{0.815792in}}%
\pgfpathlineto{\pgfqpoint{1.075394in}{0.824610in}}%
\pgfpathlineto{\pgfqpoint{1.076312in}{0.833505in}}%
\pgfpathlineto{\pgfqpoint{1.089798in}{0.831898in}}%
\pgfpathlineto{\pgfqpoint{1.103382in}{0.830516in}}%
\pgfpathlineto{\pgfqpoint{1.117049in}{0.829359in}}%
\pgfpathlineto{\pgfqpoint{1.130785in}{0.828429in}}%
\pgfpathclose%
\pgfusepath{fill}%
\end{pgfscope}%
\begin{pgfscope}%
\pgfpathrectangle{\pgfqpoint{0.041670in}{0.041670in}}{\pgfqpoint{2.216660in}{2.216660in}}%
\pgfusepath{clip}%
\pgfsetbuttcap%
\pgfsetroundjoin%
\definecolor{currentfill}{rgb}{0.195860,0.395433,0.555276}%
\pgfsetfillcolor{currentfill}%
\pgfsetlinewidth{0.000000pt}%
\definecolor{currentstroke}{rgb}{0.000000,0.000000,0.000000}%
\pgfsetstrokecolor{currentstroke}%
\pgfsetdash{}{0pt}%
\pgfpathmoveto{\pgfqpoint{1.087326in}{0.944951in}}%
\pgfpathlineto{\pgfqpoint{1.086409in}{0.935405in}}%
\pgfpathlineto{\pgfqpoint{1.085491in}{0.925895in}}%
\pgfpathlineto{\pgfqpoint{1.084574in}{0.916424in}}%
\pgfpathlineto{\pgfqpoint{1.083656in}{0.906996in}}%
\pgfpathlineto{\pgfqpoint{1.071229in}{0.908679in}}%
\pgfpathlineto{\pgfqpoint{1.058919in}{0.910563in}}%
\pgfpathlineto{\pgfqpoint{1.046740in}{0.912648in}}%
\pgfpathlineto{\pgfqpoint{1.034706in}{0.914931in}}%
\pgfpathlineto{\pgfqpoint{1.036088in}{0.924273in}}%
\pgfpathlineto{\pgfqpoint{1.037470in}{0.933658in}}%
\pgfpathlineto{\pgfqpoint{1.038853in}{0.943082in}}%
\pgfpathlineto{\pgfqpoint{1.040235in}{0.952543in}}%
\pgfpathlineto{\pgfqpoint{1.051813in}{0.950359in}}%
\pgfpathlineto{\pgfqpoint{1.063530in}{0.948364in}}%
\pgfpathlineto{\pgfqpoint{1.075372in}{0.946560in}}%
\pgfpathlineto{\pgfqpoint{1.087326in}{0.944951in}}%
\pgfpathclose%
\pgfusepath{fill}%
\end{pgfscope}%
\begin{pgfscope}%
\pgfpathrectangle{\pgfqpoint{0.041670in}{0.041670in}}{\pgfqpoint{2.216660in}{2.216660in}}%
\pgfusepath{clip}%
\pgfsetbuttcap%
\pgfsetroundjoin%
\definecolor{currentfill}{rgb}{0.279566,0.067836,0.391917}%
\pgfsetfillcolor{currentfill}%
\pgfsetlinewidth{0.000000pt}%
\definecolor{currentstroke}{rgb}{0.000000,0.000000,0.000000}%
\pgfsetstrokecolor{currentstroke}%
\pgfsetdash{}{0pt}%
\pgfpathmoveto{\pgfqpoint{1.320225in}{0.650509in}}%
\pgfpathlineto{\pgfqpoint{1.321267in}{0.644489in}}%
\pgfpathlineto{\pgfqpoint{1.322310in}{0.638637in}}%
\pgfpathlineto{\pgfqpoint{1.323355in}{0.632957in}}%
\pgfpathlineto{\pgfqpoint{1.324402in}{0.627455in}}%
\pgfpathlineto{\pgfqpoint{1.307674in}{0.625135in}}%
\pgfpathlineto{\pgfqpoint{1.290804in}{0.623100in}}%
\pgfpathlineto{\pgfqpoint{1.273811in}{0.621354in}}%
\pgfpathlineto{\pgfqpoint{1.256713in}{0.619898in}}%
\pgfpathlineto{\pgfqpoint{1.256157in}{0.625460in}}%
\pgfpathlineto{\pgfqpoint{1.255601in}{0.631200in}}%
\pgfpathlineto{\pgfqpoint{1.255046in}{0.637113in}}%
\pgfpathlineto{\pgfqpoint{1.254491in}{0.643193in}}%
\pgfpathlineto{\pgfqpoint{1.271095in}{0.644603in}}%
\pgfpathlineto{\pgfqpoint{1.287597in}{0.646293in}}%
\pgfpathlineto{\pgfqpoint{1.303980in}{0.648263in}}%
\pgfpathlineto{\pgfqpoint{1.320225in}{0.650509in}}%
\pgfpathclose%
\pgfusepath{fill}%
\end{pgfscope}%
\begin{pgfscope}%
\pgfpathrectangle{\pgfqpoint{0.041670in}{0.041670in}}{\pgfqpoint{2.216660in}{2.216660in}}%
\pgfusepath{clip}%
\pgfsetbuttcap%
\pgfsetroundjoin%
\definecolor{currentfill}{rgb}{0.974417,0.903590,0.130215}%
\pgfsetfillcolor{currentfill}%
\pgfsetlinewidth{0.000000pt}%
\definecolor{currentstroke}{rgb}{0.000000,0.000000,0.000000}%
\pgfsetstrokecolor{currentstroke}%
\pgfsetdash{}{0pt}%
\pgfpathmoveto{\pgfqpoint{1.201803in}{1.639041in}}%
\pgfpathlineto{\pgfqpoint{1.204537in}{1.636597in}}%
\pgfpathlineto{\pgfqpoint{1.207272in}{1.634020in}}%
\pgfpathlineto{\pgfqpoint{1.210008in}{1.631310in}}%
\pgfpathlineto{\pgfqpoint{1.212745in}{1.628468in}}%
\pgfpathlineto{\pgfqpoint{1.211667in}{1.627992in}}%
\pgfpathlineto{\pgfqpoint{1.210557in}{1.627531in}}%
\pgfpathlineto{\pgfqpoint{1.209417in}{1.627088in}}%
\pgfpathlineto{\pgfqpoint{1.208247in}{1.626661in}}%
\pgfpathlineto{\pgfqpoint{1.205886in}{1.629656in}}%
\pgfpathlineto{\pgfqpoint{1.203525in}{1.632517in}}%
\pgfpathlineto{\pgfqpoint{1.201164in}{1.635245in}}%
\pgfpathlineto{\pgfqpoint{1.198804in}{1.637840in}}%
\pgfpathlineto{\pgfqpoint{1.199584in}{1.638124in}}%
\pgfpathlineto{\pgfqpoint{1.200344in}{1.638419in}}%
\pgfpathlineto{\pgfqpoint{1.201084in}{1.638725in}}%
\pgfpathlineto{\pgfqpoint{1.201803in}{1.639041in}}%
\pgfpathclose%
\pgfusepath{fill}%
\end{pgfscope}%
\begin{pgfscope}%
\pgfpathrectangle{\pgfqpoint{0.041670in}{0.041670in}}{\pgfqpoint{2.216660in}{2.216660in}}%
\pgfusepath{clip}%
\pgfsetbuttcap%
\pgfsetroundjoin%
\definecolor{currentfill}{rgb}{0.271305,0.019942,0.347269}%
\pgfsetfillcolor{currentfill}%
\pgfsetlinewidth{0.000000pt}%
\definecolor{currentstroke}{rgb}{0.000000,0.000000,0.000000}%
\pgfsetstrokecolor{currentstroke}%
\pgfsetdash{}{0pt}%
\pgfpathmoveto{\pgfqpoint{1.046600in}{0.605161in}}%
\pgfpathlineto{\pgfqpoint{1.045653in}{0.600593in}}%
\pgfpathlineto{\pgfqpoint{1.044705in}{0.596229in}}%
\pgfpathlineto{\pgfqpoint{1.043756in}{0.592072in}}%
\pgfpathlineto{\pgfqpoint{1.042804in}{0.588129in}}%
\pgfpathlineto{\pgfqpoint{1.025120in}{0.590631in}}%
\pgfpathlineto{\pgfqpoint{1.007610in}{0.593433in}}%
\pgfpathlineto{\pgfqpoint{0.990291in}{0.596533in}}%
\pgfpathlineto{\pgfqpoint{0.973184in}{0.599925in}}%
\pgfpathlineto{\pgfqpoint{0.974617in}{0.603781in}}%
\pgfpathlineto{\pgfqpoint{0.976046in}{0.607849in}}%
\pgfpathlineto{\pgfqpoint{0.977474in}{0.612125in}}%
\pgfpathlineto{\pgfqpoint{0.978898in}{0.616605in}}%
\pgfpathlineto{\pgfqpoint{0.995535in}{0.613313in}}%
\pgfpathlineto{\pgfqpoint{1.012376in}{0.610306in}}%
\pgfpathlineto{\pgfqpoint{1.029404in}{0.607588in}}%
\pgfpathlineto{\pgfqpoint{1.046600in}{0.605161in}}%
\pgfpathclose%
\pgfusepath{fill}%
\end{pgfscope}%
\begin{pgfscope}%
\pgfpathrectangle{\pgfqpoint{0.041670in}{0.041670in}}{\pgfqpoint{2.216660in}{2.216660in}}%
\pgfusepath{clip}%
\pgfsetbuttcap%
\pgfsetroundjoin%
\definecolor{currentfill}{rgb}{0.272594,0.025563,0.353093}%
\pgfsetfillcolor{currentfill}%
\pgfsetlinewidth{0.000000pt}%
\definecolor{currentstroke}{rgb}{0.000000,0.000000,0.000000}%
\pgfsetstrokecolor{currentstroke}%
\pgfsetdash{}{0pt}%
\pgfpathmoveto{\pgfqpoint{1.565537in}{0.618301in}}%
\pgfpathlineto{\pgfqpoint{1.568023in}{0.619113in}}%
\pgfpathlineto{\pgfqpoint{1.570518in}{0.620229in}}%
\pgfpathlineto{\pgfqpoint{1.573020in}{0.621653in}}%
\pgfpathlineto{\pgfqpoint{1.575530in}{0.623392in}}%
\pgfpathlineto{\pgfqpoint{1.559087in}{0.616781in}}%
\pgfpathlineto{\pgfqpoint{1.542223in}{0.610449in}}%
\pgfpathlineto{\pgfqpoint{1.524954in}{0.604404in}}%
\pgfpathlineto{\pgfqpoint{1.507301in}{0.598652in}}%
\pgfpathlineto{\pgfqpoint{1.505217in}{0.597065in}}%
\pgfpathlineto{\pgfqpoint{1.503140in}{0.595794in}}%
\pgfpathlineto{\pgfqpoint{1.501069in}{0.594831in}}%
\pgfpathlineto{\pgfqpoint{1.499005in}{0.594172in}}%
\pgfpathlineto{\pgfqpoint{1.516217in}{0.599782in}}%
\pgfpathlineto{\pgfqpoint{1.533055in}{0.605678in}}%
\pgfpathlineto{\pgfqpoint{1.549501in}{0.611853in}}%
\pgfpathlineto{\pgfqpoint{1.565537in}{0.618301in}}%
\pgfpathclose%
\pgfusepath{fill}%
\end{pgfscope}%
\begin{pgfscope}%
\pgfpathrectangle{\pgfqpoint{0.041670in}{0.041670in}}{\pgfqpoint{2.216660in}{2.216660in}}%
\pgfusepath{clip}%
\pgfsetbuttcap%
\pgfsetroundjoin%
\definecolor{currentfill}{rgb}{0.231674,0.318106,0.544834}%
\pgfsetfillcolor{currentfill}%
\pgfsetlinewidth{0.000000pt}%
\definecolor{currentstroke}{rgb}{0.000000,0.000000,0.000000}%
\pgfsetstrokecolor{currentstroke}%
\pgfsetdash{}{0pt}%
\pgfpathmoveto{\pgfqpoint{1.291397in}{0.871327in}}%
\pgfpathlineto{\pgfqpoint{1.292420in}{0.862178in}}%
\pgfpathlineto{\pgfqpoint{1.293443in}{0.853091in}}%
\pgfpathlineto{\pgfqpoint{1.294467in}{0.844071in}}%
\pgfpathlineto{\pgfqpoint{1.295490in}{0.835120in}}%
\pgfpathlineto{\pgfqpoint{1.282104in}{0.833315in}}%
\pgfpathlineto{\pgfqpoint{1.268607in}{0.831734in}}%
\pgfpathlineto{\pgfqpoint{1.255013in}{0.830376in}}%
\pgfpathlineto{\pgfqpoint{1.241337in}{0.829244in}}%
\pgfpathlineto{\pgfqpoint{1.240793in}{0.838255in}}%
\pgfpathlineto{\pgfqpoint{1.240249in}{0.847335in}}%
\pgfpathlineto{\pgfqpoint{1.239705in}{0.856480in}}%
\pgfpathlineto{\pgfqpoint{1.239161in}{0.865688in}}%
\pgfpathlineto{\pgfqpoint{1.252352in}{0.866774in}}%
\pgfpathlineto{\pgfqpoint{1.265464in}{0.868077in}}%
\pgfpathlineto{\pgfqpoint{1.278484in}{0.869595in}}%
\pgfpathlineto{\pgfqpoint{1.291397in}{0.871327in}}%
\pgfpathclose%
\pgfusepath{fill}%
\end{pgfscope}%
\begin{pgfscope}%
\pgfpathrectangle{\pgfqpoint{0.041670in}{0.041670in}}{\pgfqpoint{2.216660in}{2.216660in}}%
\pgfusepath{clip}%
\pgfsetbuttcap%
\pgfsetroundjoin%
\definecolor{currentfill}{rgb}{0.974417,0.903590,0.130215}%
\pgfsetfillcolor{currentfill}%
\pgfsetlinewidth{0.000000pt}%
\definecolor{currentstroke}{rgb}{0.000000,0.000000,0.000000}%
\pgfsetstrokecolor{currentstroke}%
\pgfsetdash{}{0pt}%
\pgfpathmoveto{\pgfqpoint{1.161815in}{1.637598in}}%
\pgfpathlineto{\pgfqpoint{1.159543in}{1.634972in}}%
\pgfpathlineto{\pgfqpoint{1.157272in}{1.632213in}}%
\pgfpathlineto{\pgfqpoint{1.154999in}{1.629322in}}%
\pgfpathlineto{\pgfqpoint{1.152726in}{1.626297in}}%
\pgfpathlineto{\pgfqpoint{1.151531in}{1.626708in}}%
\pgfpathlineto{\pgfqpoint{1.150364in}{1.627136in}}%
\pgfpathlineto{\pgfqpoint{1.149227in}{1.627582in}}%
\pgfpathlineto{\pgfqpoint{1.148121in}{1.628044in}}%
\pgfpathlineto{\pgfqpoint{1.150778in}{1.630922in}}%
\pgfpathlineto{\pgfqpoint{1.153434in}{1.633667in}}%
\pgfpathlineto{\pgfqpoint{1.156090in}{1.636280in}}%
\pgfpathlineto{\pgfqpoint{1.158745in}{1.638759in}}%
\pgfpathlineto{\pgfqpoint{1.159482in}{1.638452in}}%
\pgfpathlineto{\pgfqpoint{1.160240in}{1.638156in}}%
\pgfpathlineto{\pgfqpoint{1.161018in}{1.637871in}}%
\pgfpathlineto{\pgfqpoint{1.161815in}{1.637598in}}%
\pgfpathclose%
\pgfusepath{fill}%
\end{pgfscope}%
\begin{pgfscope}%
\pgfpathrectangle{\pgfqpoint{0.041670in}{0.041670in}}{\pgfqpoint{2.216660in}{2.216660in}}%
\pgfusepath{clip}%
\pgfsetbuttcap%
\pgfsetroundjoin%
\definecolor{currentfill}{rgb}{0.147607,0.511733,0.557049}%
\pgfsetfillcolor{currentfill}%
\pgfsetlinewidth{0.000000pt}%
\definecolor{currentstroke}{rgb}{0.000000,0.000000,0.000000}%
\pgfsetstrokecolor{currentstroke}%
\pgfsetdash{}{0pt}%
\pgfpathmoveto{\pgfqpoint{1.056838in}{1.067781in}}%
\pgfpathlineto{\pgfqpoint{1.055453in}{1.058124in}}%
\pgfpathlineto{\pgfqpoint{1.054068in}{1.048467in}}%
\pgfpathlineto{\pgfqpoint{1.052684in}{1.038812in}}%
\pgfpathlineto{\pgfqpoint{1.051300in}{1.029164in}}%
\pgfpathlineto{\pgfqpoint{1.040773in}{1.031324in}}%
\pgfpathlineto{\pgfqpoint{1.030395in}{1.033652in}}%
\pgfpathlineto{\pgfqpoint{1.020177in}{1.036145in}}%
\pgfpathlineto{\pgfqpoint{1.010129in}{1.038801in}}%
\pgfpathlineto{\pgfqpoint{1.011953in}{1.048333in}}%
\pgfpathlineto{\pgfqpoint{1.013778in}{1.057871in}}%
\pgfpathlineto{\pgfqpoint{1.015604in}{1.067412in}}%
\pgfpathlineto{\pgfqpoint{1.017429in}{1.076953in}}%
\pgfpathlineto{\pgfqpoint{1.027048in}{1.074426in}}%
\pgfpathlineto{\pgfqpoint{1.036829in}{1.072053in}}%
\pgfpathlineto{\pgfqpoint{1.046762in}{1.069837in}}%
\pgfpathlineto{\pgfqpoint{1.056838in}{1.067781in}}%
\pgfpathclose%
\pgfusepath{fill}%
\end{pgfscope}%
\begin{pgfscope}%
\pgfpathrectangle{\pgfqpoint{0.041670in}{0.041670in}}{\pgfqpoint{2.216660in}{2.216660in}}%
\pgfusepath{clip}%
\pgfsetbuttcap%
\pgfsetroundjoin%
\definecolor{currentfill}{rgb}{0.935904,0.898570,0.108131}%
\pgfsetfillcolor{currentfill}%
\pgfsetlinewidth{0.000000pt}%
\definecolor{currentstroke}{rgb}{0.000000,0.000000,0.000000}%
\pgfsetstrokecolor{currentstroke}%
\pgfsetdash{}{0pt}%
\pgfpathmoveto{\pgfqpoint{1.233492in}{1.621565in}}%
\pgfpathlineto{\pgfqpoint{1.236840in}{1.618432in}}%
\pgfpathlineto{\pgfqpoint{1.240189in}{1.615169in}}%
\pgfpathlineto{\pgfqpoint{1.243537in}{1.611778in}}%
\pgfpathlineto{\pgfqpoint{1.246885in}{1.608259in}}%
\pgfpathlineto{\pgfqpoint{1.245577in}{1.607277in}}%
\pgfpathlineto{\pgfqpoint{1.244202in}{1.606315in}}%
\pgfpathlineto{\pgfqpoint{1.242763in}{1.605374in}}%
\pgfpathlineto{\pgfqpoint{1.241261in}{1.604455in}}%
\pgfpathlineto{\pgfqpoint{1.238192in}{1.608166in}}%
\pgfpathlineto{\pgfqpoint{1.235124in}{1.611749in}}%
\pgfpathlineto{\pgfqpoint{1.232055in}{1.615204in}}%
\pgfpathlineto{\pgfqpoint{1.228987in}{1.618529in}}%
\pgfpathlineto{\pgfqpoint{1.230190in}{1.619262in}}%
\pgfpathlineto{\pgfqpoint{1.231342in}{1.620014in}}%
\pgfpathlineto{\pgfqpoint{1.232443in}{1.620781in}}%
\pgfpathlineto{\pgfqpoint{1.233492in}{1.621565in}}%
\pgfpathclose%
\pgfusepath{fill}%
\end{pgfscope}%
\begin{pgfscope}%
\pgfpathrectangle{\pgfqpoint{0.041670in}{0.041670in}}{\pgfqpoint{2.216660in}{2.216660in}}%
\pgfusepath{clip}%
\pgfsetbuttcap%
\pgfsetroundjoin%
\definecolor{currentfill}{rgb}{0.280255,0.165693,0.476498}%
\pgfsetfillcolor{currentfill}%
\pgfsetlinewidth{0.000000pt}%
\definecolor{currentstroke}{rgb}{0.000000,0.000000,0.000000}%
\pgfsetstrokecolor{currentstroke}%
\pgfsetdash{}{0pt}%
\pgfpathmoveto{\pgfqpoint{1.247886in}{0.727813in}}%
\pgfpathlineto{\pgfqpoint{1.248434in}{0.720028in}}%
\pgfpathlineto{\pgfqpoint{1.248982in}{0.712363in}}%
\pgfpathlineto{\pgfqpoint{1.249531in}{0.704823in}}%
\pgfpathlineto{\pgfqpoint{1.250080in}{0.697411in}}%
\pgfpathlineto{\pgfqpoint{1.234381in}{0.696359in}}%
\pgfpathlineto{\pgfqpoint{1.218622in}{0.695572in}}%
\pgfpathlineto{\pgfqpoint{1.202820in}{0.695053in}}%
\pgfpathlineto{\pgfqpoint{1.186993in}{0.694800in}}%
\pgfpathlineto{\pgfqpoint{1.186938in}{0.702235in}}%
\pgfpathlineto{\pgfqpoint{1.186882in}{0.709799in}}%
\pgfpathlineto{\pgfqpoint{1.186827in}{0.717486in}}%
\pgfpathlineto{\pgfqpoint{1.186772in}{0.725294in}}%
\pgfpathlineto{\pgfqpoint{1.202104in}{0.725538in}}%
\pgfpathlineto{\pgfqpoint{1.217412in}{0.726039in}}%
\pgfpathlineto{\pgfqpoint{1.232678in}{0.726798in}}%
\pgfpathlineto{\pgfqpoint{1.247886in}{0.727813in}}%
\pgfpathclose%
\pgfusepath{fill}%
\end{pgfscope}%
\begin{pgfscope}%
\pgfpathrectangle{\pgfqpoint{0.041670in}{0.041670in}}{\pgfqpoint{2.216660in}{2.216660in}}%
\pgfusepath{clip}%
\pgfsetbuttcap%
\pgfsetroundjoin%
\definecolor{currentfill}{rgb}{0.274128,0.199721,0.498911}%
\pgfsetfillcolor{currentfill}%
\pgfsetlinewidth{0.000000pt}%
\definecolor{currentstroke}{rgb}{0.000000,0.000000,0.000000}%
\pgfsetstrokecolor{currentstroke}%
\pgfsetdash{}{0pt}%
\pgfpathmoveto{\pgfqpoint{1.245699in}{0.760078in}}%
\pgfpathlineto{\pgfqpoint{1.246245in}{0.751850in}}%
\pgfpathlineto{\pgfqpoint{1.246792in}{0.743727in}}%
\pgfpathlineto{\pgfqpoint{1.247339in}{0.735714in}}%
\pgfpathlineto{\pgfqpoint{1.247886in}{0.727813in}}%
\pgfpathlineto{\pgfqpoint{1.232678in}{0.726798in}}%
\pgfpathlineto{\pgfqpoint{1.217412in}{0.726039in}}%
\pgfpathlineto{\pgfqpoint{1.202104in}{0.725538in}}%
\pgfpathlineto{\pgfqpoint{1.186772in}{0.725294in}}%
\pgfpathlineto{\pgfqpoint{1.186717in}{0.733218in}}%
\pgfpathlineto{\pgfqpoint{1.186663in}{0.741255in}}%
\pgfpathlineto{\pgfqpoint{1.186608in}{0.749400in}}%
\pgfpathlineto{\pgfqpoint{1.186553in}{0.757651in}}%
\pgfpathlineto{\pgfqpoint{1.201391in}{0.757886in}}%
\pgfpathlineto{\pgfqpoint{1.216205in}{0.758369in}}%
\pgfpathlineto{\pgfqpoint{1.230980in}{0.759100in}}%
\pgfpathlineto{\pgfqpoint{1.245699in}{0.760078in}}%
\pgfpathclose%
\pgfusepath{fill}%
\end{pgfscope}%
\begin{pgfscope}%
\pgfpathrectangle{\pgfqpoint{0.041670in}{0.041670in}}{\pgfqpoint{2.216660in}{2.216660in}}%
\pgfusepath{clip}%
\pgfsetbuttcap%
\pgfsetroundjoin%
\definecolor{currentfill}{rgb}{0.412913,0.803041,0.357269}%
\pgfsetfillcolor{currentfill}%
\pgfsetlinewidth{0.000000pt}%
\definecolor{currentstroke}{rgb}{0.000000,0.000000,0.000000}%
\pgfsetstrokecolor{currentstroke}%
\pgfsetdash{}{0pt}%
\pgfpathmoveto{\pgfqpoint{1.041908in}{1.412796in}}%
\pgfpathlineto{\pgfqpoint{1.039265in}{1.405238in}}%
\pgfpathlineto{\pgfqpoint{1.036624in}{1.397596in}}%
\pgfpathlineto{\pgfqpoint{1.033983in}{1.389870in}}%
\pgfpathlineto{\pgfqpoint{1.031343in}{1.382063in}}%
\pgfpathlineto{\pgfqpoint{1.026371in}{1.384379in}}%
\pgfpathlineto{\pgfqpoint{1.021557in}{1.386768in}}%
\pgfpathlineto{\pgfqpoint{1.016905in}{1.389229in}}%
\pgfpathlineto{\pgfqpoint{1.012422in}{1.391760in}}%
\pgfpathlineto{\pgfqpoint{1.015393in}{1.399383in}}%
\pgfpathlineto{\pgfqpoint{1.018366in}{1.406926in}}%
\pgfpathlineto{\pgfqpoint{1.021340in}{1.414387in}}%
\pgfpathlineto{\pgfqpoint{1.024316in}{1.421763in}}%
\pgfpathlineto{\pgfqpoint{1.028485in}{1.419423in}}%
\pgfpathlineto{\pgfqpoint{1.032810in}{1.417147in}}%
\pgfpathlineto{\pgfqpoint{1.037286in}{1.414937in}}%
\pgfpathlineto{\pgfqpoint{1.041908in}{1.412796in}}%
\pgfpathclose%
\pgfusepath{fill}%
\end{pgfscope}%
\begin{pgfscope}%
\pgfpathrectangle{\pgfqpoint{0.041670in}{0.041670in}}{\pgfqpoint{2.216660in}{2.216660in}}%
\pgfusepath{clip}%
\pgfsetbuttcap%
\pgfsetroundjoin%
\definecolor{currentfill}{rgb}{0.762373,0.876424,0.137064}%
\pgfsetfillcolor{currentfill}%
\pgfsetlinewidth{0.000000pt}%
\definecolor{currentstroke}{rgb}{0.000000,0.000000,0.000000}%
\pgfsetstrokecolor{currentstroke}%
\pgfsetdash{}{0pt}%
\pgfpathmoveto{\pgfqpoint{1.287028in}{1.556353in}}%
\pgfpathlineto{\pgfqpoint{1.290368in}{1.551249in}}%
\pgfpathlineto{\pgfqpoint{1.293708in}{1.546031in}}%
\pgfpathlineto{\pgfqpoint{1.297047in}{1.540699in}}%
\pgfpathlineto{\pgfqpoint{1.300384in}{1.535255in}}%
\pgfpathlineto{\pgfqpoint{1.298044in}{1.533469in}}%
\pgfpathlineto{\pgfqpoint{1.295585in}{1.531718in}}%
\pgfpathlineto{\pgfqpoint{1.293008in}{1.530004in}}%
\pgfpathlineto{\pgfqpoint{1.290316in}{1.528329in}}%
\pgfpathlineto{\pgfqpoint{1.287254in}{1.533971in}}%
\pgfpathlineto{\pgfqpoint{1.284192in}{1.539500in}}%
\pgfpathlineto{\pgfqpoint{1.281129in}{1.544915in}}%
\pgfpathlineto{\pgfqpoint{1.278065in}{1.550216in}}%
\pgfpathlineto{\pgfqpoint{1.280460in}{1.551699in}}%
\pgfpathlineto{\pgfqpoint{1.282754in}{1.553218in}}%
\pgfpathlineto{\pgfqpoint{1.284944in}{1.554770in}}%
\pgfpathlineto{\pgfqpoint{1.287028in}{1.556353in}}%
\pgfpathclose%
\pgfusepath{fill}%
\end{pgfscope}%
\begin{pgfscope}%
\pgfpathrectangle{\pgfqpoint{0.041670in}{0.041670in}}{\pgfqpoint{2.216660in}{2.216660in}}%
\pgfusepath{clip}%
\pgfsetbuttcap%
\pgfsetroundjoin%
\definecolor{currentfill}{rgb}{0.179019,0.433756,0.557430}%
\pgfsetfillcolor{currentfill}%
\pgfsetlinewidth{0.000000pt}%
\definecolor{currentstroke}{rgb}{0.000000,0.000000,0.000000}%
\pgfsetstrokecolor{currentstroke}%
\pgfsetdash{}{0pt}%
\pgfpathmoveto{\pgfqpoint{1.323909in}{0.992692in}}%
\pgfpathlineto{\pgfqpoint{1.325392in}{0.983140in}}%
\pgfpathlineto{\pgfqpoint{1.326875in}{0.973613in}}%
\pgfpathlineto{\pgfqpoint{1.328358in}{0.964112in}}%
\pgfpathlineto{\pgfqpoint{1.329841in}{0.954642in}}%
\pgfpathlineto{\pgfqpoint{1.318395in}{0.952291in}}%
\pgfpathlineto{\pgfqpoint{1.306801in}{0.950127in}}%
\pgfpathlineto{\pgfqpoint{1.295070in}{0.948154in}}%
\pgfpathlineto{\pgfqpoint{1.283215in}{0.946372in}}%
\pgfpathlineto{\pgfqpoint{1.282193in}{0.955935in}}%
\pgfpathlineto{\pgfqpoint{1.281170in}{0.965528in}}%
\pgfpathlineto{\pgfqpoint{1.280147in}{0.975148in}}%
\pgfpathlineto{\pgfqpoint{1.279124in}{0.984792in}}%
\pgfpathlineto{\pgfqpoint{1.290511in}{0.986494in}}%
\pgfpathlineto{\pgfqpoint{1.301778in}{0.988379in}}%
\pgfpathlineto{\pgfqpoint{1.312915in}{0.990446in}}%
\pgfpathlineto{\pgfqpoint{1.323909in}{0.992692in}}%
\pgfpathclose%
\pgfusepath{fill}%
\end{pgfscope}%
\begin{pgfscope}%
\pgfpathrectangle{\pgfqpoint{0.041670in}{0.041670in}}{\pgfqpoint{2.216660in}{2.216660in}}%
\pgfusepath{clip}%
\pgfsetbuttcap%
\pgfsetroundjoin%
\definecolor{currentfill}{rgb}{0.233603,0.313828,0.543914}%
\pgfsetfillcolor{currentfill}%
\pgfsetlinewidth{0.000000pt}%
\definecolor{currentstroke}{rgb}{0.000000,0.000000,0.000000}%
\pgfsetstrokecolor{currentstroke}%
\pgfsetdash{}{0pt}%
\pgfpathmoveto{\pgfqpoint{1.806589in}{0.842626in}}%
\pgfpathlineto{\pgfqpoint{1.810365in}{0.852936in}}%
\pgfpathlineto{\pgfqpoint{1.814161in}{0.863702in}}%
\pgfpathlineto{\pgfqpoint{1.817976in}{0.874932in}}%
\pgfpathlineto{\pgfqpoint{1.821811in}{0.886633in}}%
\pgfpathlineto{\pgfqpoint{1.810082in}{0.876159in}}%
\pgfpathlineto{\pgfqpoint{1.797671in}{0.865868in}}%
\pgfpathlineto{\pgfqpoint{1.784590in}{0.855774in}}%
\pgfpathlineto{\pgfqpoint{1.770850in}{0.845887in}}%
\pgfpathlineto{\pgfqpoint{1.767299in}{0.834365in}}%
\pgfpathlineto{\pgfqpoint{1.763768in}{0.823316in}}%
\pgfpathlineto{\pgfqpoint{1.760254in}{0.812733in}}%
\pgfpathlineto{\pgfqpoint{1.756759in}{0.802609in}}%
\pgfpathlineto{\pgfqpoint{1.770191in}{0.812318in}}%
\pgfpathlineto{\pgfqpoint{1.782981in}{0.822231in}}%
\pgfpathlineto{\pgfqpoint{1.795117in}{0.832338in}}%
\pgfpathlineto{\pgfqpoint{1.806589in}{0.842626in}}%
\pgfpathclose%
\pgfusepath{fill}%
\end{pgfscope}%
\begin{pgfscope}%
\pgfpathrectangle{\pgfqpoint{0.041670in}{0.041670in}}{\pgfqpoint{2.216660in}{2.216660in}}%
\pgfusepath{clip}%
\pgfsetbuttcap%
\pgfsetroundjoin%
\definecolor{currentfill}{rgb}{0.280255,0.165693,0.476498}%
\pgfsetfillcolor{currentfill}%
\pgfsetlinewidth{0.000000pt}%
\definecolor{currentstroke}{rgb}{0.000000,0.000000,0.000000}%
\pgfsetstrokecolor{currentstroke}%
\pgfsetdash{}{0pt}%
\pgfpathmoveto{\pgfqpoint{1.186772in}{0.725294in}}%
\pgfpathlineto{\pgfqpoint{1.186827in}{0.717486in}}%
\pgfpathlineto{\pgfqpoint{1.186882in}{0.709799in}}%
\pgfpathlineto{\pgfqpoint{1.186938in}{0.702235in}}%
\pgfpathlineto{\pgfqpoint{1.186993in}{0.694800in}}%
\pgfpathlineto{\pgfqpoint{1.171158in}{0.694815in}}%
\pgfpathlineto{\pgfqpoint{1.155333in}{0.695097in}}%
\pgfpathlineto{\pgfqpoint{1.139535in}{0.695647in}}%
\pgfpathlineto{\pgfqpoint{1.123781in}{0.696462in}}%
\pgfpathlineto{\pgfqpoint{1.124221in}{0.703883in}}%
\pgfpathlineto{\pgfqpoint{1.124661in}{0.711432in}}%
\pgfpathlineto{\pgfqpoint{1.125100in}{0.719105in}}%
\pgfpathlineto{\pgfqpoint{1.125539in}{0.726898in}}%
\pgfpathlineto{\pgfqpoint{1.140799in}{0.726111in}}%
\pgfpathlineto{\pgfqpoint{1.156103in}{0.725581in}}%
\pgfpathlineto{\pgfqpoint{1.171433in}{0.725308in}}%
\pgfpathlineto{\pgfqpoint{1.186772in}{0.725294in}}%
\pgfpathclose%
\pgfusepath{fill}%
\end{pgfscope}%
\begin{pgfscope}%
\pgfpathrectangle{\pgfqpoint{0.041670in}{0.041670in}}{\pgfqpoint{2.216660in}{2.216660in}}%
\pgfusepath{clip}%
\pgfsetbuttcap%
\pgfsetroundjoin%
\definecolor{currentfill}{rgb}{0.274128,0.199721,0.498911}%
\pgfsetfillcolor{currentfill}%
\pgfsetlinewidth{0.000000pt}%
\definecolor{currentstroke}{rgb}{0.000000,0.000000,0.000000}%
\pgfsetstrokecolor{currentstroke}%
\pgfsetdash{}{0pt}%
\pgfpathmoveto{\pgfqpoint{1.186553in}{0.757651in}}%
\pgfpathlineto{\pgfqpoint{1.186608in}{0.749400in}}%
\pgfpathlineto{\pgfqpoint{1.186663in}{0.741255in}}%
\pgfpathlineto{\pgfqpoint{1.186717in}{0.733218in}}%
\pgfpathlineto{\pgfqpoint{1.186772in}{0.725294in}}%
\pgfpathlineto{\pgfqpoint{1.171433in}{0.725308in}}%
\pgfpathlineto{\pgfqpoint{1.156103in}{0.725581in}}%
\pgfpathlineto{\pgfqpoint{1.140799in}{0.726111in}}%
\pgfpathlineto{\pgfqpoint{1.125539in}{0.726898in}}%
\pgfpathlineto{\pgfqpoint{1.125977in}{0.734807in}}%
\pgfpathlineto{\pgfqpoint{1.126415in}{0.742829in}}%
\pgfpathlineto{\pgfqpoint{1.126853in}{0.750960in}}%
\pgfpathlineto{\pgfqpoint{1.127291in}{0.759196in}}%
\pgfpathlineto{\pgfqpoint{1.142060in}{0.758438in}}%
\pgfpathlineto{\pgfqpoint{1.156871in}{0.757927in}}%
\pgfpathlineto{\pgfqpoint{1.171708in}{0.757665in}}%
\pgfpathlineto{\pgfqpoint{1.186553in}{0.757651in}}%
\pgfpathclose%
\pgfusepath{fill}%
\end{pgfscope}%
\begin{pgfscope}%
\pgfpathrectangle{\pgfqpoint{0.041670in}{0.041670in}}{\pgfqpoint{2.216660in}{2.216660in}}%
\pgfusepath{clip}%
\pgfsetbuttcap%
\pgfsetroundjoin%
\definecolor{currentfill}{rgb}{0.699415,0.867117,0.175971}%
\pgfsetfillcolor{currentfill}%
\pgfsetlinewidth{0.000000pt}%
\definecolor{currentstroke}{rgb}{0.000000,0.000000,0.000000}%
\pgfsetstrokecolor{currentstroke}%
\pgfsetdash{}{0pt}%
\pgfpathmoveto{\pgfqpoint{1.072080in}{1.526875in}}%
\pgfpathlineto{\pgfqpoint{1.069088in}{1.521080in}}%
\pgfpathlineto{\pgfqpoint{1.066096in}{1.515175in}}%
\pgfpathlineto{\pgfqpoint{1.063105in}{1.509162in}}%
\pgfpathlineto{\pgfqpoint{1.060115in}{1.503041in}}%
\pgfpathlineto{\pgfqpoint{1.057017in}{1.504868in}}%
\pgfpathlineto{\pgfqpoint{1.054045in}{1.506739in}}%
\pgfpathlineto{\pgfqpoint{1.051200in}{1.508655in}}%
\pgfpathlineto{\pgfqpoint{1.048486in}{1.510611in}}%
\pgfpathlineto{\pgfqpoint{1.051763in}{1.516536in}}%
\pgfpathlineto{\pgfqpoint{1.055041in}{1.522355in}}%
\pgfpathlineto{\pgfqpoint{1.058320in}{1.528065in}}%
\pgfpathlineto{\pgfqpoint{1.061599in}{1.533665in}}%
\pgfpathlineto{\pgfqpoint{1.064046in}{1.531910in}}%
\pgfpathlineto{\pgfqpoint{1.066610in}{1.530192in}}%
\pgfpathlineto{\pgfqpoint{1.069289in}{1.528513in}}%
\pgfpathlineto{\pgfqpoint{1.072080in}{1.526875in}}%
\pgfpathclose%
\pgfusepath{fill}%
\end{pgfscope}%
\begin{pgfscope}%
\pgfpathrectangle{\pgfqpoint{0.041670in}{0.041670in}}{\pgfqpoint{2.216660in}{2.216660in}}%
\pgfusepath{clip}%
\pgfsetbuttcap%
\pgfsetroundjoin%
\definecolor{currentfill}{rgb}{0.120081,0.622161,0.534946}%
\pgfsetfillcolor{currentfill}%
\pgfsetlinewidth{0.000000pt}%
\definecolor{currentstroke}{rgb}{0.000000,0.000000,0.000000}%
\pgfsetstrokecolor{currentstroke}%
\pgfsetdash{}{0pt}%
\pgfpathmoveto{\pgfqpoint{1.039385in}{1.190453in}}%
\pgfpathlineto{\pgfqpoint{1.037552in}{1.181124in}}%
\pgfpathlineto{\pgfqpoint{1.035719in}{1.171764in}}%
\pgfpathlineto{\pgfqpoint{1.033887in}{1.162373in}}%
\pgfpathlineto{\pgfqpoint{1.032056in}{1.152955in}}%
\pgfpathlineto{\pgfqpoint{1.023455in}{1.155367in}}%
\pgfpathlineto{\pgfqpoint{1.015019in}{1.157913in}}%
\pgfpathlineto{\pgfqpoint{1.006757in}{1.160590in}}%
\pgfpathlineto{\pgfqpoint{0.998677in}{1.163396in}}%
\pgfpathlineto{\pgfqpoint{1.000918in}{1.172671in}}%
\pgfpathlineto{\pgfqpoint{1.003160in}{1.181919in}}%
\pgfpathlineto{\pgfqpoint{1.005403in}{1.191137in}}%
\pgfpathlineto{\pgfqpoint{1.007647in}{1.200323in}}%
\pgfpathlineto{\pgfqpoint{1.015330in}{1.197671in}}%
\pgfpathlineto{\pgfqpoint{1.023186in}{1.195140in}}%
\pgfpathlineto{\pgfqpoint{1.031207in}{1.192733in}}%
\pgfpathlineto{\pgfqpoint{1.039385in}{1.190453in}}%
\pgfpathclose%
\pgfusepath{fill}%
\end{pgfscope}%
\begin{pgfscope}%
\pgfpathrectangle{\pgfqpoint{0.041670in}{0.041670in}}{\pgfqpoint{2.216660in}{2.216660in}}%
\pgfusepath{clip}%
\pgfsetbuttcap%
\pgfsetroundjoin%
\definecolor{currentfill}{rgb}{0.283072,0.130895,0.449241}%
\pgfsetfillcolor{currentfill}%
\pgfsetlinewidth{0.000000pt}%
\definecolor{currentstroke}{rgb}{0.000000,0.000000,0.000000}%
\pgfsetstrokecolor{currentstroke}%
\pgfsetdash{}{0pt}%
\pgfpathmoveto{\pgfqpoint{1.250080in}{0.697411in}}%
\pgfpathlineto{\pgfqpoint{1.250629in}{0.690130in}}%
\pgfpathlineto{\pgfqpoint{1.251179in}{0.682985in}}%
\pgfpathlineto{\pgfqpoint{1.251730in}{0.675980in}}%
\pgfpathlineto{\pgfqpoint{1.252281in}{0.669119in}}%
\pgfpathlineto{\pgfqpoint{1.236089in}{0.668030in}}%
\pgfpathlineto{\pgfqpoint{1.219836in}{0.667216in}}%
\pgfpathlineto{\pgfqpoint{1.203538in}{0.666678in}}%
\pgfpathlineto{\pgfqpoint{1.187214in}{0.666416in}}%
\pgfpathlineto{\pgfqpoint{1.187158in}{0.673300in}}%
\pgfpathlineto{\pgfqpoint{1.187103in}{0.680329in}}%
\pgfpathlineto{\pgfqpoint{1.187048in}{0.687496in}}%
\pgfpathlineto{\pgfqpoint{1.186993in}{0.694800in}}%
\pgfpathlineto{\pgfqpoint{1.202820in}{0.695053in}}%
\pgfpathlineto{\pgfqpoint{1.218622in}{0.695572in}}%
\pgfpathlineto{\pgfqpoint{1.234381in}{0.696359in}}%
\pgfpathlineto{\pgfqpoint{1.250080in}{0.697411in}}%
\pgfpathclose%
\pgfusepath{fill}%
\end{pgfscope}%
\begin{pgfscope}%
\pgfpathrectangle{\pgfqpoint{0.041670in}{0.041670in}}{\pgfqpoint{2.216660in}{2.216660in}}%
\pgfusepath{clip}%
\pgfsetbuttcap%
\pgfsetroundjoin%
\definecolor{currentfill}{rgb}{0.220124,0.725509,0.466226}%
\pgfsetfillcolor{currentfill}%
\pgfsetlinewidth{0.000000pt}%
\definecolor{currentstroke}{rgb}{0.000000,0.000000,0.000000}%
\pgfsetstrokecolor{currentstroke}%
\pgfsetdash{}{0pt}%
\pgfpathmoveto{\pgfqpoint{1.034643in}{1.307156in}}%
\pgfpathlineto{\pgfqpoint{1.032388in}{1.298546in}}%
\pgfpathlineto{\pgfqpoint{1.030134in}{1.289874in}}%
\pgfpathlineto{\pgfqpoint{1.027881in}{1.281145in}}%
\pgfpathlineto{\pgfqpoint{1.025629in}{1.272358in}}%
\pgfpathlineto{\pgfqpoint{1.018903in}{1.274814in}}%
\pgfpathlineto{\pgfqpoint{1.012345in}{1.277373in}}%
\pgfpathlineto{\pgfqpoint{1.005961in}{1.280031in}}%
\pgfpathlineto{\pgfqpoint{0.999758in}{1.282786in}}%
\pgfpathlineto{\pgfqpoint{1.002383in}{1.291407in}}%
\pgfpathlineto{\pgfqpoint{1.005010in}{1.299972in}}%
\pgfpathlineto{\pgfqpoint{1.007638in}{1.308479in}}%
\pgfpathlineto{\pgfqpoint{1.010267in}{1.316925in}}%
\pgfpathlineto{\pgfqpoint{1.016112in}{1.314344in}}%
\pgfpathlineto{\pgfqpoint{1.022128in}{1.311854in}}%
\pgfpathlineto{\pgfqpoint{1.028306in}{1.309457in}}%
\pgfpathlineto{\pgfqpoint{1.034643in}{1.307156in}}%
\pgfpathclose%
\pgfusepath{fill}%
\end{pgfscope}%
\begin{pgfscope}%
\pgfpathrectangle{\pgfqpoint{0.041670in}{0.041670in}}{\pgfqpoint{2.216660in}{2.216660in}}%
\pgfusepath{clip}%
\pgfsetbuttcap%
\pgfsetroundjoin%
\definecolor{currentfill}{rgb}{0.263663,0.237631,0.518762}%
\pgfsetfillcolor{currentfill}%
\pgfsetlinewidth{0.000000pt}%
\definecolor{currentstroke}{rgb}{0.000000,0.000000,0.000000}%
\pgfsetstrokecolor{currentstroke}%
\pgfsetdash{}{0pt}%
\pgfpathmoveto{\pgfqpoint{1.243516in}{0.793966in}}%
\pgfpathlineto{\pgfqpoint{1.244062in}{0.785354in}}%
\pgfpathlineto{\pgfqpoint{1.244607in}{0.776833in}}%
\pgfpathlineto{\pgfqpoint{1.245153in}{0.768407in}}%
\pgfpathlineto{\pgfqpoint{1.245699in}{0.760078in}}%
\pgfpathlineto{\pgfqpoint{1.230980in}{0.759100in}}%
\pgfpathlineto{\pgfqpoint{1.216205in}{0.758369in}}%
\pgfpathlineto{\pgfqpoint{1.201391in}{0.757886in}}%
\pgfpathlineto{\pgfqpoint{1.186553in}{0.757651in}}%
\pgfpathlineto{\pgfqpoint{1.186498in}{0.766003in}}%
\pgfpathlineto{\pgfqpoint{1.186443in}{0.774452in}}%
\pgfpathlineto{\pgfqpoint{1.186388in}{0.782996in}}%
\pgfpathlineto{\pgfqpoint{1.186334in}{0.791630in}}%
\pgfpathlineto{\pgfqpoint{1.200679in}{0.791856in}}%
\pgfpathlineto{\pgfqpoint{1.215002in}{0.792321in}}%
\pgfpathlineto{\pgfqpoint{1.229286in}{0.793025in}}%
\pgfpathlineto{\pgfqpoint{1.243516in}{0.793966in}}%
\pgfpathclose%
\pgfusepath{fill}%
\end{pgfscope}%
\begin{pgfscope}%
\pgfpathrectangle{\pgfqpoint{0.041670in}{0.041670in}}{\pgfqpoint{2.216660in}{2.216660in}}%
\pgfusepath{clip}%
\pgfsetbuttcap%
\pgfsetroundjoin%
\definecolor{currentfill}{rgb}{0.487026,0.823929,0.312321}%
\pgfsetfillcolor{currentfill}%
\pgfsetlinewidth{0.000000pt}%
\definecolor{currentstroke}{rgb}{0.000000,0.000000,0.000000}%
\pgfsetstrokecolor{currentstroke}%
\pgfsetdash{}{0pt}%
\pgfpathmoveto{\pgfqpoint{1.326982in}{1.452345in}}%
\pgfpathlineto{\pgfqpoint{1.330030in}{1.445368in}}%
\pgfpathlineto{\pgfqpoint{1.333077in}{1.438299in}}%
\pgfpathlineto{\pgfqpoint{1.336122in}{1.431141in}}%
\pgfpathlineto{\pgfqpoint{1.339166in}{1.423894in}}%
\pgfpathlineto{\pgfqpoint{1.335139in}{1.421500in}}%
\pgfpathlineto{\pgfqpoint{1.330952in}{1.419167in}}%
\pgfpathlineto{\pgfqpoint{1.326610in}{1.416898in}}%
\pgfpathlineto{\pgfqpoint{1.322118in}{1.414695in}}%
\pgfpathlineto{\pgfqpoint{1.319396in}{1.422128in}}%
\pgfpathlineto{\pgfqpoint{1.316673in}{1.429472in}}%
\pgfpathlineto{\pgfqpoint{1.313949in}{1.436725in}}%
\pgfpathlineto{\pgfqpoint{1.311224in}{1.443887in}}%
\pgfpathlineto{\pgfqpoint{1.315376in}{1.445911in}}%
\pgfpathlineto{\pgfqpoint{1.319388in}{1.447997in}}%
\pgfpathlineto{\pgfqpoint{1.323258in}{1.450143in}}%
\pgfpathlineto{\pgfqpoint{1.326982in}{1.452345in}}%
\pgfpathclose%
\pgfusepath{fill}%
\end{pgfscope}%
\begin{pgfscope}%
\pgfpathrectangle{\pgfqpoint{0.041670in}{0.041670in}}{\pgfqpoint{2.216660in}{2.216660in}}%
\pgfusepath{clip}%
\pgfsetbuttcap%
\pgfsetroundjoin%
\definecolor{currentfill}{rgb}{0.935904,0.898570,0.108131}%
\pgfsetfillcolor{currentfill}%
\pgfsetlinewidth{0.000000pt}%
\definecolor{currentstroke}{rgb}{0.000000,0.000000,0.000000}%
\pgfsetstrokecolor{currentstroke}%
\pgfsetdash{}{0pt}%
\pgfpathmoveto{\pgfqpoint{1.132033in}{1.617891in}}%
\pgfpathlineto{\pgfqpoint{1.129034in}{1.614526in}}%
\pgfpathlineto{\pgfqpoint{1.126035in}{1.611031in}}%
\pgfpathlineto{\pgfqpoint{1.123036in}{1.607408in}}%
\pgfpathlineto{\pgfqpoint{1.120036in}{1.603656in}}%
\pgfpathlineto{\pgfqpoint{1.118479in}{1.604556in}}%
\pgfpathlineto{\pgfqpoint{1.116984in}{1.605478in}}%
\pgfpathlineto{\pgfqpoint{1.115552in}{1.606421in}}%
\pgfpathlineto{\pgfqpoint{1.114184in}{1.607386in}}%
\pgfpathlineto{\pgfqpoint{1.117475in}{1.610949in}}%
\pgfpathlineto{\pgfqpoint{1.120766in}{1.614384in}}%
\pgfpathlineto{\pgfqpoint{1.124056in}{1.617690in}}%
\pgfpathlineto{\pgfqpoint{1.127347in}{1.620868in}}%
\pgfpathlineto{\pgfqpoint{1.128442in}{1.620098in}}%
\pgfpathlineto{\pgfqpoint{1.129589in}{1.619345in}}%
\pgfpathlineto{\pgfqpoint{1.130787in}{1.618609in}}%
\pgfpathlineto{\pgfqpoint{1.132033in}{1.617891in}}%
\pgfpathclose%
\pgfusepath{fill}%
\end{pgfscope}%
\begin{pgfscope}%
\pgfpathrectangle{\pgfqpoint{0.041670in}{0.041670in}}{\pgfqpoint{2.216660in}{2.216660in}}%
\pgfusepath{clip}%
\pgfsetbuttcap%
\pgfsetroundjoin%
\definecolor{currentfill}{rgb}{0.231674,0.318106,0.544834}%
\pgfsetfillcolor{currentfill}%
\pgfsetlinewidth{0.000000pt}%
\definecolor{currentstroke}{rgb}{0.000000,0.000000,0.000000}%
\pgfsetstrokecolor{currentstroke}%
\pgfsetdash{}{0pt}%
\pgfpathmoveto{\pgfqpoint{1.132529in}{0.864906in}}%
\pgfpathlineto{\pgfqpoint{1.132093in}{0.855690in}}%
\pgfpathlineto{\pgfqpoint{1.131657in}{0.846536in}}%
\pgfpathlineto{\pgfqpoint{1.131221in}{0.837448in}}%
\pgfpathlineto{\pgfqpoint{1.130785in}{0.828429in}}%
\pgfpathlineto{\pgfqpoint{1.117049in}{0.829359in}}%
\pgfpathlineto{\pgfqpoint{1.103382in}{0.830516in}}%
\pgfpathlineto{\pgfqpoint{1.089798in}{0.831898in}}%
\pgfpathlineto{\pgfqpoint{1.076312in}{0.833505in}}%
\pgfpathlineto{\pgfqpoint{1.077231in}{0.842472in}}%
\pgfpathlineto{\pgfqpoint{1.078149in}{0.851509in}}%
\pgfpathlineto{\pgfqpoint{1.079067in}{0.860612in}}%
\pgfpathlineto{\pgfqpoint{1.079985in}{0.869777in}}%
\pgfpathlineto{\pgfqpoint{1.092994in}{0.868235in}}%
\pgfpathlineto{\pgfqpoint{1.106097in}{0.866908in}}%
\pgfpathlineto{\pgfqpoint{1.119280in}{0.865798in}}%
\pgfpathlineto{\pgfqpoint{1.132529in}{0.864906in}}%
\pgfpathclose%
\pgfusepath{fill}%
\end{pgfscope}%
\begin{pgfscope}%
\pgfpathrectangle{\pgfqpoint{0.041670in}{0.041670in}}{\pgfqpoint{2.216660in}{2.216660in}}%
\pgfusepath{clip}%
\pgfsetbuttcap%
\pgfsetroundjoin%
\definecolor{currentfill}{rgb}{0.279566,0.067836,0.391917}%
\pgfsetfillcolor{currentfill}%
\pgfsetlinewidth{0.000000pt}%
\definecolor{currentstroke}{rgb}{0.000000,0.000000,0.000000}%
\pgfsetstrokecolor{currentstroke}%
\pgfsetdash{}{0pt}%
\pgfpathmoveto{\pgfqpoint{1.120247in}{0.642178in}}%
\pgfpathlineto{\pgfqpoint{1.119803in}{0.636089in}}%
\pgfpathlineto{\pgfqpoint{1.119358in}{0.630168in}}%
\pgfpathlineto{\pgfqpoint{1.118913in}{0.624420in}}%
\pgfpathlineto{\pgfqpoint{1.118467in}{0.618849in}}%
\pgfpathlineto{\pgfqpoint{1.101292in}{0.620045in}}%
\pgfpathlineto{\pgfqpoint{1.084205in}{0.621533in}}%
\pgfpathlineto{\pgfqpoint{1.067225in}{0.623312in}}%
\pgfpathlineto{\pgfqpoint{1.050370in}{0.625378in}}%
\pgfpathlineto{\pgfqpoint{1.051309in}{0.630897in}}%
\pgfpathlineto{\pgfqpoint{1.052246in}{0.636593in}}%
\pgfpathlineto{\pgfqpoint{1.053183in}{0.642462in}}%
\pgfpathlineto{\pgfqpoint{1.054118in}{0.648499in}}%
\pgfpathlineto{\pgfqpoint{1.070486in}{0.646498in}}%
\pgfpathlineto{\pgfqpoint{1.086976in}{0.644777in}}%
\pgfpathlineto{\pgfqpoint{1.103569in}{0.643336in}}%
\pgfpathlineto{\pgfqpoint{1.120247in}{0.642178in}}%
\pgfpathclose%
\pgfusepath{fill}%
\end{pgfscope}%
\begin{pgfscope}%
\pgfpathrectangle{\pgfqpoint{0.041670in}{0.041670in}}{\pgfqpoint{2.216660in}{2.216660in}}%
\pgfusepath{clip}%
\pgfsetbuttcap%
\pgfsetroundjoin%
\definecolor{currentfill}{rgb}{0.955300,0.901065,0.118128}%
\pgfsetfillcolor{currentfill}%
\pgfsetlinewidth{0.000000pt}%
\definecolor{currentstroke}{rgb}{0.000000,0.000000,0.000000}%
\pgfsetstrokecolor{currentstroke}%
\pgfsetdash{}{0pt}%
\pgfpathmoveto{\pgfqpoint{1.216716in}{1.630524in}}%
\pgfpathlineto{\pgfqpoint{1.219783in}{1.627722in}}%
\pgfpathlineto{\pgfqpoint{1.222851in}{1.624788in}}%
\pgfpathlineto{\pgfqpoint{1.225919in}{1.621724in}}%
\pgfpathlineto{\pgfqpoint{1.228987in}{1.618529in}}%
\pgfpathlineto{\pgfqpoint{1.227735in}{1.617813in}}%
\pgfpathlineto{\pgfqpoint{1.226434in}{1.617116in}}%
\pgfpathlineto{\pgfqpoint{1.225087in}{1.616438in}}%
\pgfpathlineto{\pgfqpoint{1.223695in}{1.615781in}}%
\pgfpathlineto{\pgfqpoint{1.220957in}{1.619149in}}%
\pgfpathlineto{\pgfqpoint{1.218219in}{1.622387in}}%
\pgfpathlineto{\pgfqpoint{1.215482in}{1.625493in}}%
\pgfpathlineto{\pgfqpoint{1.212745in}{1.628468in}}%
\pgfpathlineto{\pgfqpoint{1.213789in}{1.628960in}}%
\pgfpathlineto{\pgfqpoint{1.214800in}{1.629467in}}%
\pgfpathlineto{\pgfqpoint{1.215776in}{1.629988in}}%
\pgfpathlineto{\pgfqpoint{1.216716in}{1.630524in}}%
\pgfpathclose%
\pgfusepath{fill}%
\end{pgfscope}%
\begin{pgfscope}%
\pgfpathrectangle{\pgfqpoint{0.041670in}{0.041670in}}{\pgfqpoint{2.216660in}{2.216660in}}%
\pgfusepath{clip}%
\pgfsetbuttcap%
\pgfsetroundjoin%
\definecolor{currentfill}{rgb}{0.974417,0.903590,0.130215}%
\pgfsetfillcolor{currentfill}%
\pgfsetlinewidth{0.000000pt}%
\definecolor{currentstroke}{rgb}{0.000000,0.000000,0.000000}%
\pgfsetstrokecolor{currentstroke}%
\pgfsetdash{}{0pt}%
\pgfpathmoveto{\pgfqpoint{1.198804in}{1.637840in}}%
\pgfpathlineto{\pgfqpoint{1.201164in}{1.635245in}}%
\pgfpathlineto{\pgfqpoint{1.203525in}{1.632517in}}%
\pgfpathlineto{\pgfqpoint{1.205886in}{1.629656in}}%
\pgfpathlineto{\pgfqpoint{1.208247in}{1.626661in}}%
\pgfpathlineto{\pgfqpoint{1.207049in}{1.626252in}}%
\pgfpathlineto{\pgfqpoint{1.205824in}{1.625861in}}%
\pgfpathlineto{\pgfqpoint{1.204572in}{1.625488in}}%
\pgfpathlineto{\pgfqpoint{1.203296in}{1.625134in}}%
\pgfpathlineto{\pgfqpoint{1.201347in}{1.628256in}}%
\pgfpathlineto{\pgfqpoint{1.199399in}{1.631246in}}%
\pgfpathlineto{\pgfqpoint{1.197451in}{1.634102in}}%
\pgfpathlineto{\pgfqpoint{1.195504in}{1.636824in}}%
\pgfpathlineto{\pgfqpoint{1.196355in}{1.637060in}}%
\pgfpathlineto{\pgfqpoint{1.197189in}{1.637308in}}%
\pgfpathlineto{\pgfqpoint{1.198005in}{1.637568in}}%
\pgfpathlineto{\pgfqpoint{1.198804in}{1.637840in}}%
\pgfpathclose%
\pgfusepath{fill}%
\end{pgfscope}%
\begin{pgfscope}%
\pgfpathrectangle{\pgfqpoint{0.041670in}{0.041670in}}{\pgfqpoint{2.216660in}{2.216660in}}%
\pgfusepath{clip}%
\pgfsetbuttcap%
\pgfsetroundjoin%
\definecolor{currentfill}{rgb}{0.896320,0.893616,0.096335}%
\pgfsetfillcolor{currentfill}%
\pgfsetlinewidth{0.000000pt}%
\definecolor{currentstroke}{rgb}{0.000000,0.000000,0.000000}%
\pgfsetstrokecolor{currentstroke}%
\pgfsetdash{}{0pt}%
\pgfpathmoveto{\pgfqpoint{1.246885in}{1.608259in}}%
\pgfpathlineto{\pgfqpoint{1.250233in}{1.604613in}}%
\pgfpathlineto{\pgfqpoint{1.253581in}{1.600841in}}%
\pgfpathlineto{\pgfqpoint{1.256928in}{1.596942in}}%
\pgfpathlineto{\pgfqpoint{1.260275in}{1.592920in}}%
\pgfpathlineto{\pgfqpoint{1.258707in}{1.591739in}}%
\pgfpathlineto{\pgfqpoint{1.257060in}{1.590581in}}%
\pgfpathlineto{\pgfqpoint{1.255335in}{1.589449in}}%
\pgfpathlineto{\pgfqpoint{1.253534in}{1.588343in}}%
\pgfpathlineto{\pgfqpoint{1.250466in}{1.592559in}}%
\pgfpathlineto{\pgfqpoint{1.247398in}{1.596651in}}%
\pgfpathlineto{\pgfqpoint{1.244329in}{1.600616in}}%
\pgfpathlineto{\pgfqpoint{1.241261in}{1.604455in}}%
\pgfpathlineto{\pgfqpoint{1.242763in}{1.605374in}}%
\pgfpathlineto{\pgfqpoint{1.244202in}{1.606315in}}%
\pgfpathlineto{\pgfqpoint{1.245577in}{1.607277in}}%
\pgfpathlineto{\pgfqpoint{1.246885in}{1.608259in}}%
\pgfpathclose%
\pgfusepath{fill}%
\end{pgfscope}%
\begin{pgfscope}%
\pgfpathrectangle{\pgfqpoint{0.041670in}{0.041670in}}{\pgfqpoint{2.216660in}{2.216660in}}%
\pgfusepath{clip}%
\pgfsetbuttcap%
\pgfsetroundjoin%
\definecolor{currentfill}{rgb}{0.267004,0.004874,0.329415}%
\pgfsetfillcolor{currentfill}%
\pgfsetlinewidth{0.000000pt}%
\definecolor{currentstroke}{rgb}{0.000000,0.000000,0.000000}%
\pgfsetstrokecolor{currentstroke}%
\pgfsetdash{}{0pt}%
\pgfpathmoveto{\pgfqpoint{1.482699in}{0.599181in}}%
\pgfpathlineto{\pgfqpoint{1.484719in}{0.597605in}}%
\pgfpathlineto{\pgfqpoint{1.486743in}{0.596290in}}%
\pgfpathlineto{\pgfqpoint{1.488773in}{0.595242in}}%
\pgfpathlineto{\pgfqpoint{1.490808in}{0.594464in}}%
\pgfpathlineto{\pgfqpoint{1.473685in}{0.589286in}}%
\pgfpathlineto{\pgfqpoint{1.456235in}{0.584400in}}%
\pgfpathlineto{\pgfqpoint{1.438475in}{0.579813in}}%
\pgfpathlineto{\pgfqpoint{1.420427in}{0.575531in}}%
\pgfpathlineto{\pgfqpoint{1.418849in}{0.576435in}}%
\pgfpathlineto{\pgfqpoint{1.417275in}{0.577610in}}%
\pgfpathlineto{\pgfqpoint{1.415705in}{0.579052in}}%
\pgfpathlineto{\pgfqpoint{1.414139in}{0.580755in}}%
\pgfpathlineto{\pgfqpoint{1.431719in}{0.584922in}}%
\pgfpathlineto{\pgfqpoint{1.449018in}{0.589386in}}%
\pgfpathlineto{\pgfqpoint{1.466018in}{0.594141in}}%
\pgfpathlineto{\pgfqpoint{1.482699in}{0.599181in}}%
\pgfpathclose%
\pgfusepath{fill}%
\end{pgfscope}%
\begin{pgfscope}%
\pgfpathrectangle{\pgfqpoint{0.041670in}{0.041670in}}{\pgfqpoint{2.216660in}{2.216660in}}%
\pgfusepath{clip}%
\pgfsetbuttcap%
\pgfsetroundjoin%
\definecolor{currentfill}{rgb}{0.283072,0.130895,0.449241}%
\pgfsetfillcolor{currentfill}%
\pgfsetlinewidth{0.000000pt}%
\definecolor{currentstroke}{rgb}{0.000000,0.000000,0.000000}%
\pgfsetstrokecolor{currentstroke}%
\pgfsetdash{}{0pt}%
\pgfpathmoveto{\pgfqpoint{1.186993in}{0.694800in}}%
\pgfpathlineto{\pgfqpoint{1.187048in}{0.687496in}}%
\pgfpathlineto{\pgfqpoint{1.187103in}{0.680329in}}%
\pgfpathlineto{\pgfqpoint{1.187158in}{0.673300in}}%
\pgfpathlineto{\pgfqpoint{1.187214in}{0.666416in}}%
\pgfpathlineto{\pgfqpoint{1.170882in}{0.666432in}}%
\pgfpathlineto{\pgfqpoint{1.154560in}{0.666724in}}%
\pgfpathlineto{\pgfqpoint{1.138266in}{0.667293in}}%
\pgfpathlineto{\pgfqpoint{1.122018in}{0.668137in}}%
\pgfpathlineto{\pgfqpoint{1.122459in}{0.675007in}}%
\pgfpathlineto{\pgfqpoint{1.122901in}{0.682020in}}%
\pgfpathlineto{\pgfqpoint{1.123341in}{0.689174in}}%
\pgfpathlineto{\pgfqpoint{1.123781in}{0.696462in}}%
\pgfpathlineto{\pgfqpoint{1.139535in}{0.695647in}}%
\pgfpathlineto{\pgfqpoint{1.155333in}{0.695097in}}%
\pgfpathlineto{\pgfqpoint{1.171158in}{0.694815in}}%
\pgfpathlineto{\pgfqpoint{1.186993in}{0.694800in}}%
\pgfpathclose%
\pgfusepath{fill}%
\end{pgfscope}%
\begin{pgfscope}%
\pgfpathrectangle{\pgfqpoint{0.041670in}{0.041670in}}{\pgfqpoint{2.216660in}{2.216660in}}%
\pgfusepath{clip}%
\pgfsetbuttcap%
\pgfsetroundjoin%
\definecolor{currentfill}{rgb}{0.263663,0.237631,0.518762}%
\pgfsetfillcolor{currentfill}%
\pgfsetlinewidth{0.000000pt}%
\definecolor{currentstroke}{rgb}{0.000000,0.000000,0.000000}%
\pgfsetstrokecolor{currentstroke}%
\pgfsetdash{}{0pt}%
\pgfpathmoveto{\pgfqpoint{1.186334in}{0.791630in}}%
\pgfpathlineto{\pgfqpoint{1.186388in}{0.782996in}}%
\pgfpathlineto{\pgfqpoint{1.186443in}{0.774452in}}%
\pgfpathlineto{\pgfqpoint{1.186498in}{0.766003in}}%
\pgfpathlineto{\pgfqpoint{1.186553in}{0.757651in}}%
\pgfpathlineto{\pgfqpoint{1.171708in}{0.757665in}}%
\pgfpathlineto{\pgfqpoint{1.156871in}{0.757927in}}%
\pgfpathlineto{\pgfqpoint{1.142060in}{0.758438in}}%
\pgfpathlineto{\pgfqpoint{1.127291in}{0.759196in}}%
\pgfpathlineto{\pgfqpoint{1.127729in}{0.767534in}}%
\pgfpathlineto{\pgfqpoint{1.128166in}{0.775969in}}%
\pgfpathlineto{\pgfqpoint{1.128603in}{0.784498in}}%
\pgfpathlineto{\pgfqpoint{1.129040in}{0.793118in}}%
\pgfpathlineto{\pgfqpoint{1.143319in}{0.792388in}}%
\pgfpathlineto{\pgfqpoint{1.157638in}{0.791896in}}%
\pgfpathlineto{\pgfqpoint{1.171981in}{0.791644in}}%
\pgfpathlineto{\pgfqpoint{1.186334in}{0.791630in}}%
\pgfpathclose%
\pgfusepath{fill}%
\end{pgfscope}%
\begin{pgfscope}%
\pgfpathrectangle{\pgfqpoint{0.041670in}{0.041670in}}{\pgfqpoint{2.216660in}{2.216660in}}%
\pgfusepath{clip}%
\pgfsetbuttcap%
\pgfsetroundjoin%
\definecolor{currentfill}{rgb}{0.814576,0.883393,0.110347}%
\pgfsetfillcolor{currentfill}%
\pgfsetlinewidth{0.000000pt}%
\definecolor{currentstroke}{rgb}{0.000000,0.000000,0.000000}%
\pgfsetstrokecolor{currentstroke}%
\pgfsetdash{}{0pt}%
\pgfpathmoveto{\pgfqpoint{1.273657in}{1.575597in}}%
\pgfpathlineto{\pgfqpoint{1.277001in}{1.570964in}}%
\pgfpathlineto{\pgfqpoint{1.280344in}{1.566211in}}%
\pgfpathlineto{\pgfqpoint{1.283686in}{1.561340in}}%
\pgfpathlineto{\pgfqpoint{1.287028in}{1.556353in}}%
\pgfpathlineto{\pgfqpoint{1.284944in}{1.554770in}}%
\pgfpathlineto{\pgfqpoint{1.282754in}{1.553218in}}%
\pgfpathlineto{\pgfqpoint{1.280460in}{1.551699in}}%
\pgfpathlineto{\pgfqpoint{1.278065in}{1.550216in}}%
\pgfpathlineto{\pgfqpoint{1.275000in}{1.555400in}}%
\pgfpathlineto{\pgfqpoint{1.271935in}{1.560466in}}%
\pgfpathlineto{\pgfqpoint{1.268869in}{1.565414in}}%
\pgfpathlineto{\pgfqpoint{1.265803in}{1.570243in}}%
\pgfpathlineto{\pgfqpoint{1.267902in}{1.571538in}}%
\pgfpathlineto{\pgfqpoint{1.269912in}{1.572862in}}%
\pgfpathlineto{\pgfqpoint{1.271831in}{1.574216in}}%
\pgfpathlineto{\pgfqpoint{1.273657in}{1.575597in}}%
\pgfpathclose%
\pgfusepath{fill}%
\end{pgfscope}%
\begin{pgfscope}%
\pgfpathrectangle{\pgfqpoint{0.041670in}{0.041670in}}{\pgfqpoint{2.216660in}{2.216660in}}%
\pgfusepath{clip}%
\pgfsetbuttcap%
\pgfsetroundjoin%
\definecolor{currentfill}{rgb}{0.133743,0.548535,0.553541}%
\pgfsetfillcolor{currentfill}%
\pgfsetlinewidth{0.000000pt}%
\definecolor{currentstroke}{rgb}{0.000000,0.000000,0.000000}%
\pgfsetstrokecolor{currentstroke}%
\pgfsetdash{}{0pt}%
\pgfpathmoveto{\pgfqpoint{1.343201in}{1.117320in}}%
\pgfpathlineto{\pgfqpoint{1.345123in}{1.107835in}}%
\pgfpathlineto{\pgfqpoint{1.347044in}{1.098340in}}%
\pgfpathlineto{\pgfqpoint{1.348965in}{1.088837in}}%
\pgfpathlineto{\pgfqpoint{1.350885in}{1.079328in}}%
\pgfpathlineto{\pgfqpoint{1.341420in}{1.076665in}}%
\pgfpathlineto{\pgfqpoint{1.331783in}{1.074154in}}%
\pgfpathlineto{\pgfqpoint{1.321984in}{1.071799in}}%
\pgfpathlineto{\pgfqpoint{1.312035in}{1.069601in}}%
\pgfpathlineto{\pgfqpoint{1.310549in}{1.079232in}}%
\pgfpathlineto{\pgfqpoint{1.309063in}{1.088858in}}%
\pgfpathlineto{\pgfqpoint{1.307576in}{1.098475in}}%
\pgfpathlineto{\pgfqpoint{1.306089in}{1.108081in}}%
\pgfpathlineto{\pgfqpoint{1.315593in}{1.110169in}}%
\pgfpathlineto{\pgfqpoint{1.324952in}{1.112406in}}%
\pgfpathlineto{\pgfqpoint{1.334158in}{1.114791in}}%
\pgfpathlineto{\pgfqpoint{1.343201in}{1.117320in}}%
\pgfpathclose%
\pgfusepath{fill}%
\end{pgfscope}%
\begin{pgfscope}%
\pgfpathrectangle{\pgfqpoint{0.041670in}{0.041670in}}{\pgfqpoint{2.216660in}{2.216660in}}%
\pgfusepath{clip}%
\pgfsetbuttcap%
\pgfsetroundjoin%
\definecolor{currentfill}{rgb}{0.974417,0.903590,0.130215}%
\pgfsetfillcolor{currentfill}%
\pgfsetlinewidth{0.000000pt}%
\definecolor{currentstroke}{rgb}{0.000000,0.000000,0.000000}%
\pgfsetstrokecolor{currentstroke}%
\pgfsetdash{}{0pt}%
\pgfpathmoveto{\pgfqpoint{1.165175in}{1.636626in}}%
\pgfpathlineto{\pgfqpoint{1.163324in}{1.633878in}}%
\pgfpathlineto{\pgfqpoint{1.161473in}{1.630997in}}%
\pgfpathlineto{\pgfqpoint{1.159621in}{1.627983in}}%
\pgfpathlineto{\pgfqpoint{1.157768in}{1.624835in}}%
\pgfpathlineto{\pgfqpoint{1.156471in}{1.625173in}}%
\pgfpathlineto{\pgfqpoint{1.155197in}{1.625529in}}%
\pgfpathlineto{\pgfqpoint{1.153949in}{1.625904in}}%
\pgfpathlineto{\pgfqpoint{1.152726in}{1.626297in}}%
\pgfpathlineto{\pgfqpoint{1.154999in}{1.629322in}}%
\pgfpathlineto{\pgfqpoint{1.157272in}{1.632213in}}%
\pgfpathlineto{\pgfqpoint{1.159543in}{1.634972in}}%
\pgfpathlineto{\pgfqpoint{1.161815in}{1.637598in}}%
\pgfpathlineto{\pgfqpoint{1.162629in}{1.637336in}}%
\pgfpathlineto{\pgfqpoint{1.163462in}{1.637087in}}%
\pgfpathlineto{\pgfqpoint{1.164310in}{1.636850in}}%
\pgfpathlineto{\pgfqpoint{1.165175in}{1.636626in}}%
\pgfpathclose%
\pgfusepath{fill}%
\end{pgfscope}%
\begin{pgfscope}%
\pgfpathrectangle{\pgfqpoint{0.041670in}{0.041670in}}{\pgfqpoint{2.216660in}{2.216660in}}%
\pgfusepath{clip}%
\pgfsetbuttcap%
\pgfsetroundjoin%
\definecolor{currentfill}{rgb}{0.855810,0.888601,0.097452}%
\pgfsetfillcolor{currentfill}%
\pgfsetlinewidth{0.000000pt}%
\definecolor{currentstroke}{rgb}{0.000000,0.000000,0.000000}%
\pgfsetstrokecolor{currentstroke}%
\pgfsetdash{}{0pt}%
\pgfpathmoveto{\pgfqpoint{1.260275in}{1.592920in}}%
\pgfpathlineto{\pgfqpoint{1.263621in}{1.588773in}}%
\pgfpathlineto{\pgfqpoint{1.266967in}{1.584503in}}%
\pgfpathlineto{\pgfqpoint{1.270312in}{1.580111in}}%
\pgfpathlineto{\pgfqpoint{1.273657in}{1.575597in}}%
\pgfpathlineto{\pgfqpoint{1.271831in}{1.574216in}}%
\pgfpathlineto{\pgfqpoint{1.269912in}{1.572862in}}%
\pgfpathlineto{\pgfqpoint{1.267902in}{1.571538in}}%
\pgfpathlineto{\pgfqpoint{1.265803in}{1.570243in}}%
\pgfpathlineto{\pgfqpoint{1.262736in}{1.574951in}}%
\pgfpathlineto{\pgfqpoint{1.259669in}{1.579538in}}%
\pgfpathlineto{\pgfqpoint{1.256602in}{1.584002in}}%
\pgfpathlineto{\pgfqpoint{1.253534in}{1.588343in}}%
\pgfpathlineto{\pgfqpoint{1.255335in}{1.589449in}}%
\pgfpathlineto{\pgfqpoint{1.257060in}{1.590581in}}%
\pgfpathlineto{\pgfqpoint{1.258707in}{1.591739in}}%
\pgfpathlineto{\pgfqpoint{1.260275in}{1.592920in}}%
\pgfpathclose%
\pgfusepath{fill}%
\end{pgfscope}%
\begin{pgfscope}%
\pgfpathrectangle{\pgfqpoint{0.041670in}{0.041670in}}{\pgfqpoint{2.216660in}{2.216660in}}%
\pgfusepath{clip}%
\pgfsetbuttcap%
\pgfsetroundjoin%
\definecolor{currentfill}{rgb}{0.955300,0.901065,0.118128}%
\pgfsetfillcolor{currentfill}%
\pgfsetlinewidth{0.000000pt}%
\definecolor{currentstroke}{rgb}{0.000000,0.000000,0.000000}%
\pgfsetstrokecolor{currentstroke}%
\pgfsetdash{}{0pt}%
\pgfpathmoveto{\pgfqpoint{1.148121in}{1.628044in}}%
\pgfpathlineto{\pgfqpoint{1.145464in}{1.625033in}}%
\pgfpathlineto{\pgfqpoint{1.142806in}{1.621891in}}%
\pgfpathlineto{\pgfqpoint{1.140148in}{1.618618in}}%
\pgfpathlineto{\pgfqpoint{1.137490in}{1.615214in}}%
\pgfpathlineto{\pgfqpoint{1.136058in}{1.615853in}}%
\pgfpathlineto{\pgfqpoint{1.134670in}{1.616512in}}%
\pgfpathlineto{\pgfqpoint{1.133329in}{1.617192in}}%
\pgfpathlineto{\pgfqpoint{1.132033in}{1.617891in}}%
\pgfpathlineto{\pgfqpoint{1.135032in}{1.621127in}}%
\pgfpathlineto{\pgfqpoint{1.138031in}{1.624231in}}%
\pgfpathlineto{\pgfqpoint{1.141029in}{1.627205in}}%
\pgfpathlineto{\pgfqpoint{1.144027in}{1.630047in}}%
\pgfpathlineto{\pgfqpoint{1.144999in}{1.629524in}}%
\pgfpathlineto{\pgfqpoint{1.146006in}{1.629015in}}%
\pgfpathlineto{\pgfqpoint{1.147047in}{1.628522in}}%
\pgfpathlineto{\pgfqpoint{1.148121in}{1.628044in}}%
\pgfpathclose%
\pgfusepath{fill}%
\end{pgfscope}%
\begin{pgfscope}%
\pgfpathrectangle{\pgfqpoint{0.041670in}{0.041670in}}{\pgfqpoint{2.216660in}{2.216660in}}%
\pgfusepath{clip}%
\pgfsetbuttcap%
\pgfsetroundjoin%
\definecolor{currentfill}{rgb}{0.281477,0.755203,0.432552}%
\pgfsetfillcolor{currentfill}%
\pgfsetlinewidth{0.000000pt}%
\definecolor{currentstroke}{rgb}{0.000000,0.000000,0.000000}%
\pgfsetstrokecolor{currentstroke}%
\pgfsetdash{}{0pt}%
\pgfpathmoveto{\pgfqpoint{1.343852in}{1.352274in}}%
\pgfpathlineto{\pgfqpoint{1.346563in}{1.344130in}}%
\pgfpathlineto{\pgfqpoint{1.349274in}{1.335918in}}%
\pgfpathlineto{\pgfqpoint{1.351983in}{1.327638in}}%
\pgfpathlineto{\pgfqpoint{1.354690in}{1.319294in}}%
\pgfpathlineto{\pgfqpoint{1.349001in}{1.316634in}}%
\pgfpathlineto{\pgfqpoint{1.343137in}{1.314063in}}%
\pgfpathlineto{\pgfqpoint{1.337104in}{1.311583in}}%
\pgfpathlineto{\pgfqpoint{1.330907in}{1.309196in}}%
\pgfpathlineto{\pgfqpoint{1.328564in}{1.317710in}}%
\pgfpathlineto{\pgfqpoint{1.326221in}{1.326158in}}%
\pgfpathlineto{\pgfqpoint{1.323876in}{1.334538in}}%
\pgfpathlineto{\pgfqpoint{1.321531in}{1.342850in}}%
\pgfpathlineto{\pgfqpoint{1.327346in}{1.345077in}}%
\pgfpathlineto{\pgfqpoint{1.333008in}{1.347392in}}%
\pgfpathlineto{\pgfqpoint{1.338512in}{1.349792in}}%
\pgfpathlineto{\pgfqpoint{1.343852in}{1.352274in}}%
\pgfpathclose%
\pgfusepath{fill}%
\end{pgfscope}%
\begin{pgfscope}%
\pgfpathrectangle{\pgfqpoint{0.041670in}{0.041670in}}{\pgfqpoint{2.216660in}{2.216660in}}%
\pgfusepath{clip}%
\pgfsetbuttcap%
\pgfsetroundjoin%
\definecolor{currentfill}{rgb}{0.212395,0.359683,0.551710}%
\pgfsetfillcolor{currentfill}%
\pgfsetlinewidth{0.000000pt}%
\definecolor{currentstroke}{rgb}{0.000000,0.000000,0.000000}%
\pgfsetstrokecolor{currentstroke}%
\pgfsetdash{}{0pt}%
\pgfpathmoveto{\pgfqpoint{1.287306in}{0.908482in}}%
\pgfpathlineto{\pgfqpoint{1.288328in}{0.899116in}}%
\pgfpathlineto{\pgfqpoint{1.289351in}{0.889799in}}%
\pgfpathlineto{\pgfqpoint{1.290374in}{0.880535in}}%
\pgfpathlineto{\pgfqpoint{1.291397in}{0.871327in}}%
\pgfpathlineto{\pgfqpoint{1.278484in}{0.869595in}}%
\pgfpathlineto{\pgfqpoint{1.265464in}{0.868077in}}%
\pgfpathlineto{\pgfqpoint{1.252352in}{0.866774in}}%
\pgfpathlineto{\pgfqpoint{1.239161in}{0.865688in}}%
\pgfpathlineto{\pgfqpoint{1.238617in}{0.874955in}}%
\pgfpathlineto{\pgfqpoint{1.238073in}{0.884278in}}%
\pgfpathlineto{\pgfqpoint{1.237529in}{0.893654in}}%
\pgfpathlineto{\pgfqpoint{1.236986in}{0.903078in}}%
\pgfpathlineto{\pgfqpoint{1.249693in}{0.904119in}}%
\pgfpathlineto{\pgfqpoint{1.262324in}{0.905367in}}%
\pgfpathlineto{\pgfqpoint{1.274866in}{0.906822in}}%
\pgfpathlineto{\pgfqpoint{1.287306in}{0.908482in}}%
\pgfpathclose%
\pgfusepath{fill}%
\end{pgfscope}%
\begin{pgfscope}%
\pgfpathrectangle{\pgfqpoint{0.041670in}{0.041670in}}{\pgfqpoint{2.216660in}{2.216660in}}%
\pgfusepath{clip}%
\pgfsetbuttcap%
\pgfsetroundjoin%
\definecolor{currentfill}{rgb}{0.134692,0.658636,0.517649}%
\pgfsetfillcolor{currentfill}%
\pgfsetlinewidth{0.000000pt}%
\definecolor{currentstroke}{rgb}{0.000000,0.000000,0.000000}%
\pgfsetstrokecolor{currentstroke}%
\pgfsetdash{}{0pt}%
\pgfpathmoveto{\pgfqpoint{1.349612in}{1.239014in}}%
\pgfpathlineto{\pgfqpoint{1.351946in}{1.230016in}}%
\pgfpathlineto{\pgfqpoint{1.354279in}{1.220976in}}%
\pgfpathlineto{\pgfqpoint{1.356610in}{1.211897in}}%
\pgfpathlineto{\pgfqpoint{1.358941in}{1.202781in}}%
\pgfpathlineto{\pgfqpoint{1.351418in}{1.200023in}}%
\pgfpathlineto{\pgfqpoint{1.343715in}{1.197383in}}%
\pgfpathlineto{\pgfqpoint{1.335840in}{1.194866in}}%
\pgfpathlineto{\pgfqpoint{1.327802in}{1.192473in}}%
\pgfpathlineto{\pgfqpoint{1.325873in}{1.201737in}}%
\pgfpathlineto{\pgfqpoint{1.323945in}{1.210964in}}%
\pgfpathlineto{\pgfqpoint{1.322015in}{1.220151in}}%
\pgfpathlineto{\pgfqpoint{1.320085in}{1.229296in}}%
\pgfpathlineto{\pgfqpoint{1.327707in}{1.231552in}}%
\pgfpathlineto{\pgfqpoint{1.335174in}{1.233925in}}%
\pgfpathlineto{\pgfqpoint{1.342478in}{1.236413in}}%
\pgfpathlineto{\pgfqpoint{1.349612in}{1.239014in}}%
\pgfpathclose%
\pgfusepath{fill}%
\end{pgfscope}%
\begin{pgfscope}%
\pgfpathrectangle{\pgfqpoint{0.041670in}{0.041670in}}{\pgfqpoint{2.216660in}{2.216660in}}%
\pgfusepath{clip}%
\pgfsetbuttcap%
\pgfsetroundjoin%
\definecolor{currentfill}{rgb}{0.762373,0.876424,0.137064}%
\pgfsetfillcolor{currentfill}%
\pgfsetlinewidth{0.000000pt}%
\definecolor{currentstroke}{rgb}{0.000000,0.000000,0.000000}%
\pgfsetstrokecolor{currentstroke}%
\pgfsetdash{}{0pt}%
\pgfpathmoveto{\pgfqpoint{1.084058in}{1.548927in}}%
\pgfpathlineto{\pgfqpoint{1.081063in}{1.543585in}}%
\pgfpathlineto{\pgfqpoint{1.078068in}{1.538129in}}%
\pgfpathlineto{\pgfqpoint{1.075074in}{1.532558in}}%
\pgfpathlineto{\pgfqpoint{1.072080in}{1.526875in}}%
\pgfpathlineto{\pgfqpoint{1.069289in}{1.528513in}}%
\pgfpathlineto{\pgfqpoint{1.066610in}{1.530192in}}%
\pgfpathlineto{\pgfqpoint{1.064046in}{1.531910in}}%
\pgfpathlineto{\pgfqpoint{1.061599in}{1.533665in}}%
\pgfpathlineto{\pgfqpoint{1.064880in}{1.539155in}}%
\pgfpathlineto{\pgfqpoint{1.068162in}{1.544532in}}%
\pgfpathlineto{\pgfqpoint{1.071445in}{1.549795in}}%
\pgfpathlineto{\pgfqpoint{1.074729in}{1.554944in}}%
\pgfpathlineto{\pgfqpoint{1.076907in}{1.553389in}}%
\pgfpathlineto{\pgfqpoint{1.079189in}{1.551866in}}%
\pgfpathlineto{\pgfqpoint{1.081574in}{1.550379in}}%
\pgfpathlineto{\pgfqpoint{1.084058in}{1.548927in}}%
\pgfpathclose%
\pgfusepath{fill}%
\end{pgfscope}%
\begin{pgfscope}%
\pgfpathrectangle{\pgfqpoint{0.041670in}{0.041670in}}{\pgfqpoint{2.216660in}{2.216660in}}%
\pgfusepath{clip}%
\pgfsetbuttcap%
\pgfsetroundjoin%
\definecolor{currentfill}{rgb}{0.248629,0.278775,0.534556}%
\pgfsetfillcolor{currentfill}%
\pgfsetlinewidth{0.000000pt}%
\definecolor{currentstroke}{rgb}{0.000000,0.000000,0.000000}%
\pgfsetstrokecolor{currentstroke}%
\pgfsetdash{}{0pt}%
\pgfpathmoveto{\pgfqpoint{1.241337in}{0.829244in}}%
\pgfpathlineto{\pgfqpoint{1.241882in}{0.820307in}}%
\pgfpathlineto{\pgfqpoint{1.242426in}{0.811446in}}%
\pgfpathlineto{\pgfqpoint{1.242971in}{0.802664in}}%
\pgfpathlineto{\pgfqpoint{1.243516in}{0.793966in}}%
\pgfpathlineto{\pgfqpoint{1.229286in}{0.793025in}}%
\pgfpathlineto{\pgfqpoint{1.215002in}{0.792321in}}%
\pgfpathlineto{\pgfqpoint{1.200679in}{0.791856in}}%
\pgfpathlineto{\pgfqpoint{1.186334in}{0.791630in}}%
\pgfpathlineto{\pgfqpoint{1.186279in}{0.800351in}}%
\pgfpathlineto{\pgfqpoint{1.186224in}{0.809156in}}%
\pgfpathlineto{\pgfqpoint{1.186170in}{0.818040in}}%
\pgfpathlineto{\pgfqpoint{1.186115in}{0.827000in}}%
\pgfpathlineto{\pgfqpoint{1.199968in}{0.827217in}}%
\pgfpathlineto{\pgfqpoint{1.213800in}{0.827664in}}%
\pgfpathlineto{\pgfqpoint{1.227595in}{0.828340in}}%
\pgfpathlineto{\pgfqpoint{1.241337in}{0.829244in}}%
\pgfpathclose%
\pgfusepath{fill}%
\end{pgfscope}%
\begin{pgfscope}%
\pgfpathrectangle{\pgfqpoint{0.041670in}{0.041670in}}{\pgfqpoint{2.216660in}{2.216660in}}%
\pgfusepath{clip}%
\pgfsetbuttcap%
\pgfsetroundjoin%
\definecolor{currentfill}{rgb}{0.896320,0.893616,0.096335}%
\pgfsetfillcolor{currentfill}%
\pgfsetlinewidth{0.000000pt}%
\definecolor{currentstroke}{rgb}{0.000000,0.000000,0.000000}%
\pgfsetstrokecolor{currentstroke}%
\pgfsetdash{}{0pt}%
\pgfpathmoveto{\pgfqpoint{1.120036in}{1.603656in}}%
\pgfpathlineto{\pgfqpoint{1.117037in}{1.599777in}}%
\pgfpathlineto{\pgfqpoint{1.114037in}{1.595771in}}%
\pgfpathlineto{\pgfqpoint{1.111038in}{1.591639in}}%
\pgfpathlineto{\pgfqpoint{1.108039in}{1.587382in}}%
\pgfpathlineto{\pgfqpoint{1.106172in}{1.588464in}}%
\pgfpathlineto{\pgfqpoint{1.104379in}{1.589574in}}%
\pgfpathlineto{\pgfqpoint{1.102663in}{1.590709in}}%
\pgfpathlineto{\pgfqpoint{1.101025in}{1.591869in}}%
\pgfpathlineto{\pgfqpoint{1.104314in}{1.595936in}}%
\pgfpathlineto{\pgfqpoint{1.107604in}{1.599879in}}%
\pgfpathlineto{\pgfqpoint{1.110894in}{1.603695in}}%
\pgfpathlineto{\pgfqpoint{1.114184in}{1.607386in}}%
\pgfpathlineto{\pgfqpoint{1.115552in}{1.606421in}}%
\pgfpathlineto{\pgfqpoint{1.116984in}{1.605478in}}%
\pgfpathlineto{\pgfqpoint{1.118479in}{1.604556in}}%
\pgfpathlineto{\pgfqpoint{1.120036in}{1.603656in}}%
\pgfpathclose%
\pgfusepath{fill}%
\end{pgfscope}%
\begin{pgfscope}%
\pgfpathrectangle{\pgfqpoint{0.041670in}{0.041670in}}{\pgfqpoint{2.216660in}{2.216660in}}%
\pgfusepath{clip}%
\pgfsetbuttcap%
\pgfsetroundjoin%
\definecolor{currentfill}{rgb}{0.179019,0.433756,0.557430}%
\pgfsetfillcolor{currentfill}%
\pgfsetlinewidth{0.000000pt}%
\definecolor{currentstroke}{rgb}{0.000000,0.000000,0.000000}%
\pgfsetstrokecolor{currentstroke}%
\pgfsetdash{}{0pt}%
\pgfpathmoveto{\pgfqpoint{1.090997in}{0.983434in}}%
\pgfpathlineto{\pgfqpoint{1.090079in}{0.973775in}}%
\pgfpathlineto{\pgfqpoint{1.089161in}{0.964139in}}%
\pgfpathlineto{\pgfqpoint{1.088244in}{0.954530in}}%
\pgfpathlineto{\pgfqpoint{1.087326in}{0.944951in}}%
\pgfpathlineto{\pgfqpoint{1.075372in}{0.946560in}}%
\pgfpathlineto{\pgfqpoint{1.063530in}{0.948364in}}%
\pgfpathlineto{\pgfqpoint{1.051813in}{0.950359in}}%
\pgfpathlineto{\pgfqpoint{1.040235in}{0.952543in}}%
\pgfpathlineto{\pgfqpoint{1.041618in}{0.962037in}}%
\pgfpathlineto{\pgfqpoint{1.043000in}{0.971561in}}%
\pgfpathlineto{\pgfqpoint{1.044383in}{0.981112in}}%
\pgfpathlineto{\pgfqpoint{1.045766in}{0.990687in}}%
\pgfpathlineto{\pgfqpoint{1.056887in}{0.988600in}}%
\pgfpathlineto{\pgfqpoint{1.068141in}{0.986694in}}%
\pgfpathlineto{\pgfqpoint{1.079515in}{0.984972in}}%
\pgfpathlineto{\pgfqpoint{1.090997in}{0.983434in}}%
\pgfpathclose%
\pgfusepath{fill}%
\end{pgfscope}%
\begin{pgfscope}%
\pgfpathrectangle{\pgfqpoint{0.041670in}{0.041670in}}{\pgfqpoint{2.216660in}{2.216660in}}%
\pgfusepath{clip}%
\pgfsetbuttcap%
\pgfsetroundjoin%
\definecolor{currentfill}{rgb}{0.282327,0.094955,0.417331}%
\pgfsetfillcolor{currentfill}%
\pgfsetlinewidth{0.000000pt}%
\definecolor{currentstroke}{rgb}{0.000000,0.000000,0.000000}%
\pgfsetstrokecolor{currentstroke}%
\pgfsetdash{}{0pt}%
\pgfpathmoveto{\pgfqpoint{1.252281in}{0.669119in}}%
\pgfpathlineto{\pgfqpoint{1.252833in}{0.662406in}}%
\pgfpathlineto{\pgfqpoint{1.253385in}{0.655844in}}%
\pgfpathlineto{\pgfqpoint{1.253938in}{0.649439in}}%
\pgfpathlineto{\pgfqpoint{1.254491in}{0.643193in}}%
\pgfpathlineto{\pgfqpoint{1.237805in}{0.642067in}}%
\pgfpathlineto{\pgfqpoint{1.221055in}{0.641225in}}%
\pgfpathlineto{\pgfqpoint{1.204259in}{0.640669in}}%
\pgfpathlineto{\pgfqpoint{1.187436in}{0.640398in}}%
\pgfpathlineto{\pgfqpoint{1.187380in}{0.646667in}}%
\pgfpathlineto{\pgfqpoint{1.187324in}{0.653095in}}%
\pgfpathlineto{\pgfqpoint{1.187269in}{0.659680in}}%
\pgfpathlineto{\pgfqpoint{1.187214in}{0.666416in}}%
\pgfpathlineto{\pgfqpoint{1.203538in}{0.666678in}}%
\pgfpathlineto{\pgfqpoint{1.219836in}{0.667216in}}%
\pgfpathlineto{\pgfqpoint{1.236089in}{0.668030in}}%
\pgfpathlineto{\pgfqpoint{1.252281in}{0.669119in}}%
\pgfpathclose%
\pgfusepath{fill}%
\end{pgfscope}%
\begin{pgfscope}%
\pgfpathrectangle{\pgfqpoint{0.041670in}{0.041670in}}{\pgfqpoint{2.216660in}{2.216660in}}%
\pgfusepath{clip}%
\pgfsetbuttcap%
\pgfsetroundjoin%
\definecolor{currentfill}{rgb}{0.268510,0.009605,0.335427}%
\pgfsetfillcolor{currentfill}%
\pgfsetlinewidth{0.000000pt}%
\definecolor{currentstroke}{rgb}{0.000000,0.000000,0.000000}%
\pgfsetstrokecolor{currentstroke}%
\pgfsetdash{}{0pt}%
\pgfpathmoveto{\pgfqpoint{1.401739in}{0.603184in}}%
\pgfpathlineto{\pgfqpoint{1.403277in}{0.599570in}}%
\pgfpathlineto{\pgfqpoint{1.404819in}{0.596178in}}%
\pgfpathlineto{\pgfqpoint{1.406364in}{0.593012in}}%
\pgfpathlineto{\pgfqpoint{1.407912in}{0.590078in}}%
\pgfpathlineto{\pgfqpoint{1.390541in}{0.586320in}}%
\pgfpathlineto{\pgfqpoint{1.372935in}{0.582860in}}%
\pgfpathlineto{\pgfqpoint{1.355113in}{0.579703in}}%
\pgfpathlineto{\pgfqpoint{1.337095in}{0.576851in}}%
\pgfpathlineto{\pgfqpoint{1.336026in}{0.579880in}}%
\pgfpathlineto{\pgfqpoint{1.334960in}{0.583141in}}%
\pgfpathlineto{\pgfqpoint{1.333895in}{0.586628in}}%
\pgfpathlineto{\pgfqpoint{1.332833in}{0.590338in}}%
\pgfpathlineto{\pgfqpoint{1.350364in}{0.593107in}}%
\pgfpathlineto{\pgfqpoint{1.367704in}{0.596174in}}%
\pgfpathlineto{\pgfqpoint{1.384836in}{0.599534in}}%
\pgfpathlineto{\pgfqpoint{1.401739in}{0.603184in}}%
\pgfpathclose%
\pgfusepath{fill}%
\end{pgfscope}%
\begin{pgfscope}%
\pgfpathrectangle{\pgfqpoint{0.041670in}{0.041670in}}{\pgfqpoint{2.216660in}{2.216660in}}%
\pgfusepath{clip}%
\pgfsetbuttcap%
\pgfsetroundjoin%
\definecolor{currentfill}{rgb}{0.974417,0.903590,0.130215}%
\pgfsetfillcolor{currentfill}%
\pgfsetlinewidth{0.000000pt}%
\definecolor{currentstroke}{rgb}{0.000000,0.000000,0.000000}%
\pgfsetstrokecolor{currentstroke}%
\pgfsetdash{}{0pt}%
\pgfpathmoveto{\pgfqpoint{1.195504in}{1.636824in}}%
\pgfpathlineto{\pgfqpoint{1.197451in}{1.634102in}}%
\pgfpathlineto{\pgfqpoint{1.199399in}{1.631246in}}%
\pgfpathlineto{\pgfqpoint{1.201347in}{1.628256in}}%
\pgfpathlineto{\pgfqpoint{1.203296in}{1.625134in}}%
\pgfpathlineto{\pgfqpoint{1.201996in}{1.624799in}}%
\pgfpathlineto{\pgfqpoint{1.200674in}{1.624483in}}%
\pgfpathlineto{\pgfqpoint{1.199331in}{1.624187in}}%
\pgfpathlineto{\pgfqpoint{1.197969in}{1.623911in}}%
\pgfpathlineto{\pgfqpoint{1.196465in}{1.627136in}}%
\pgfpathlineto{\pgfqpoint{1.194961in}{1.630228in}}%
\pgfpathlineto{\pgfqpoint{1.193458in}{1.633186in}}%
\pgfpathlineto{\pgfqpoint{1.191955in}{1.636011in}}%
\pgfpathlineto{\pgfqpoint{1.192862in}{1.636194in}}%
\pgfpathlineto{\pgfqpoint{1.193757in}{1.636391in}}%
\pgfpathlineto{\pgfqpoint{1.194638in}{1.636601in}}%
\pgfpathlineto{\pgfqpoint{1.195504in}{1.636824in}}%
\pgfpathclose%
\pgfusepath{fill}%
\end{pgfscope}%
\begin{pgfscope}%
\pgfpathrectangle{\pgfqpoint{0.041670in}{0.041670in}}{\pgfqpoint{2.216660in}{2.216660in}}%
\pgfusepath{clip}%
\pgfsetbuttcap%
\pgfsetroundjoin%
\definecolor{currentfill}{rgb}{0.172719,0.448791,0.557885}%
\pgfsetfillcolor{currentfill}%
\pgfsetlinewidth{0.000000pt}%
\definecolor{currentstroke}{rgb}{0.000000,0.000000,0.000000}%
\pgfsetstrokecolor{currentstroke}%
\pgfsetdash{}{0pt}%
\pgfpathmoveto{\pgfqpoint{1.878145in}{0.982540in}}%
\pgfpathlineto{\pgfqpoint{1.882306in}{0.996898in}}%
\pgfpathlineto{\pgfqpoint{1.886491in}{1.011775in}}%
\pgfpathlineto{\pgfqpoint{1.890701in}{1.027180in}}%
\pgfpathlineto{\pgfqpoint{1.881454in}{1.015776in}}%
\pgfpathlineto{\pgfqpoint{1.871459in}{1.004510in}}%
\pgfpathlineto{\pgfqpoint{1.860724in}{0.993393in}}%
\pgfpathlineto{\pgfqpoint{1.849258in}{0.982440in}}%
\pgfpathlineto{\pgfqpoint{1.845268in}{0.967205in}}%
\pgfpathlineto{\pgfqpoint{1.841303in}{0.952500in}}%
\pgfpathlineto{\pgfqpoint{1.837360in}{0.938318in}}%
\pgfpathlineto{\pgfqpoint{1.848641in}{0.949144in}}%
\pgfpathlineto{\pgfqpoint{1.859204in}{0.960131in}}%
\pgfpathlineto{\pgfqpoint{1.869041in}{0.971267in}}%
\pgfpathlineto{\pgfqpoint{1.878145in}{0.982540in}}%
\pgfpathclose%
\pgfusepath{fill}%
\end{pgfscope}%
\begin{pgfscope}%
\pgfpathrectangle{\pgfqpoint{0.041670in}{0.041670in}}{\pgfqpoint{2.216660in}{2.216660in}}%
\pgfusepath{clip}%
\pgfsetbuttcap%
\pgfsetroundjoin%
\definecolor{currentfill}{rgb}{0.248629,0.278775,0.534556}%
\pgfsetfillcolor{currentfill}%
\pgfsetlinewidth{0.000000pt}%
\definecolor{currentstroke}{rgb}{0.000000,0.000000,0.000000}%
\pgfsetstrokecolor{currentstroke}%
\pgfsetdash{}{0pt}%
\pgfpathmoveto{\pgfqpoint{1.186115in}{0.827000in}}%
\pgfpathlineto{\pgfqpoint{1.186170in}{0.818040in}}%
\pgfpathlineto{\pgfqpoint{1.186224in}{0.809156in}}%
\pgfpathlineto{\pgfqpoint{1.186279in}{0.800351in}}%
\pgfpathlineto{\pgfqpoint{1.186334in}{0.791630in}}%
\pgfpathlineto{\pgfqpoint{1.171981in}{0.791644in}}%
\pgfpathlineto{\pgfqpoint{1.157638in}{0.791896in}}%
\pgfpathlineto{\pgfqpoint{1.143319in}{0.792388in}}%
\pgfpathlineto{\pgfqpoint{1.129040in}{0.793118in}}%
\pgfpathlineto{\pgfqpoint{1.129476in}{0.801824in}}%
\pgfpathlineto{\pgfqpoint{1.129913in}{0.810614in}}%
\pgfpathlineto{\pgfqpoint{1.130349in}{0.819484in}}%
\pgfpathlineto{\pgfqpoint{1.130785in}{0.828429in}}%
\pgfpathlineto{\pgfqpoint{1.144575in}{0.827728in}}%
\pgfpathlineto{\pgfqpoint{1.158403in}{0.827256in}}%
\pgfpathlineto{\pgfqpoint{1.172255in}{0.827013in}}%
\pgfpathlineto{\pgfqpoint{1.186115in}{0.827000in}}%
\pgfpathclose%
\pgfusepath{fill}%
\end{pgfscope}%
\begin{pgfscope}%
\pgfpathrectangle{\pgfqpoint{0.041670in}{0.041670in}}{\pgfqpoint{2.216660in}{2.216660in}}%
\pgfusepath{clip}%
\pgfsetbuttcap%
\pgfsetroundjoin%
\definecolor{currentfill}{rgb}{0.974417,0.903590,0.130215}%
\pgfsetfillcolor{currentfill}%
\pgfsetlinewidth{0.000000pt}%
\definecolor{currentstroke}{rgb}{0.000000,0.000000,0.000000}%
\pgfsetstrokecolor{currentstroke}%
\pgfsetdash{}{0pt}%
\pgfpathmoveto{\pgfqpoint{1.168772in}{1.635859in}}%
\pgfpathlineto{\pgfqpoint{1.167371in}{1.633015in}}%
\pgfpathlineto{\pgfqpoint{1.165970in}{1.630037in}}%
\pgfpathlineto{\pgfqpoint{1.164569in}{1.626926in}}%
\pgfpathlineto{\pgfqpoint{1.163167in}{1.623682in}}%
\pgfpathlineto{\pgfqpoint{1.161788in}{1.623941in}}%
\pgfpathlineto{\pgfqpoint{1.160428in}{1.624219in}}%
\pgfpathlineto{\pgfqpoint{1.159088in}{1.624517in}}%
\pgfpathlineto{\pgfqpoint{1.157768in}{1.624835in}}%
\pgfpathlineto{\pgfqpoint{1.159621in}{1.627983in}}%
\pgfpathlineto{\pgfqpoint{1.161473in}{1.630997in}}%
\pgfpathlineto{\pgfqpoint{1.163324in}{1.633878in}}%
\pgfpathlineto{\pgfqpoint{1.165175in}{1.636626in}}%
\pgfpathlineto{\pgfqpoint{1.166054in}{1.636414in}}%
\pgfpathlineto{\pgfqpoint{1.166947in}{1.636216in}}%
\pgfpathlineto{\pgfqpoint{1.167854in}{1.636031in}}%
\pgfpathlineto{\pgfqpoint{1.168772in}{1.635859in}}%
\pgfpathclose%
\pgfusepath{fill}%
\end{pgfscope}%
\begin{pgfscope}%
\pgfpathrectangle{\pgfqpoint{0.041670in}{0.041670in}}{\pgfqpoint{2.216660in}{2.216660in}}%
\pgfusepath{clip}%
\pgfsetbuttcap%
\pgfsetroundjoin%
\definecolor{currentfill}{rgb}{0.282327,0.094955,0.417331}%
\pgfsetfillcolor{currentfill}%
\pgfsetlinewidth{0.000000pt}%
\definecolor{currentstroke}{rgb}{0.000000,0.000000,0.000000}%
\pgfsetstrokecolor{currentstroke}%
\pgfsetdash{}{0pt}%
\pgfpathmoveto{\pgfqpoint{1.187214in}{0.666416in}}%
\pgfpathlineto{\pgfqpoint{1.187269in}{0.659680in}}%
\pgfpathlineto{\pgfqpoint{1.187324in}{0.653095in}}%
\pgfpathlineto{\pgfqpoint{1.187380in}{0.646667in}}%
\pgfpathlineto{\pgfqpoint{1.187436in}{0.640398in}}%
\pgfpathlineto{\pgfqpoint{1.170604in}{0.640414in}}%
\pgfpathlineto{\pgfqpoint{1.153783in}{0.640716in}}%
\pgfpathlineto{\pgfqpoint{1.136991in}{0.641305in}}%
\pgfpathlineto{\pgfqpoint{1.120247in}{0.642178in}}%
\pgfpathlineto{\pgfqpoint{1.120690in}{0.648432in}}%
\pgfpathlineto{\pgfqpoint{1.121133in}{0.654846in}}%
\pgfpathlineto{\pgfqpoint{1.121576in}{0.661416in}}%
\pgfpathlineto{\pgfqpoint{1.122018in}{0.668137in}}%
\pgfpathlineto{\pgfqpoint{1.138266in}{0.667293in}}%
\pgfpathlineto{\pgfqpoint{1.154560in}{0.666724in}}%
\pgfpathlineto{\pgfqpoint{1.170882in}{0.666432in}}%
\pgfpathlineto{\pgfqpoint{1.187214in}{0.666416in}}%
\pgfpathclose%
\pgfusepath{fill}%
\end{pgfscope}%
\begin{pgfscope}%
\pgfpathrectangle{\pgfqpoint{0.041670in}{0.041670in}}{\pgfqpoint{2.216660in}{2.216660in}}%
\pgfusepath{clip}%
\pgfsetbuttcap%
\pgfsetroundjoin%
\definecolor{currentfill}{rgb}{0.814576,0.883393,0.110347}%
\pgfsetfillcolor{currentfill}%
\pgfsetlinewidth{0.000000pt}%
\definecolor{currentstroke}{rgb}{0.000000,0.000000,0.000000}%
\pgfsetstrokecolor{currentstroke}%
\pgfsetdash{}{0pt}%
\pgfpathmoveto{\pgfqpoint{1.096045in}{1.569119in}}%
\pgfpathlineto{\pgfqpoint{1.093048in}{1.564249in}}%
\pgfpathlineto{\pgfqpoint{1.090051in}{1.559260in}}%
\pgfpathlineto{\pgfqpoint{1.087054in}{1.554152in}}%
\pgfpathlineto{\pgfqpoint{1.084058in}{1.548927in}}%
\pgfpathlineto{\pgfqpoint{1.081574in}{1.550379in}}%
\pgfpathlineto{\pgfqpoint{1.079189in}{1.551866in}}%
\pgfpathlineto{\pgfqpoint{1.076907in}{1.553389in}}%
\pgfpathlineto{\pgfqpoint{1.074729in}{1.554944in}}%
\pgfpathlineto{\pgfqpoint{1.078013in}{1.559977in}}%
\pgfpathlineto{\pgfqpoint{1.081299in}{1.564892in}}%
\pgfpathlineto{\pgfqpoint{1.084585in}{1.569690in}}%
\pgfpathlineto{\pgfqpoint{1.087872in}{1.574368in}}%
\pgfpathlineto{\pgfqpoint{1.089780in}{1.573011in}}%
\pgfpathlineto{\pgfqpoint{1.091780in}{1.571683in}}%
\pgfpathlineto{\pgfqpoint{1.093869in}{1.570385in}}%
\pgfpathlineto{\pgfqpoint{1.096045in}{1.569119in}}%
\pgfpathclose%
\pgfusepath{fill}%
\end{pgfscope}%
\begin{pgfscope}%
\pgfpathrectangle{\pgfqpoint{0.041670in}{0.041670in}}{\pgfqpoint{2.216660in}{2.216660in}}%
\pgfusepath{clip}%
\pgfsetbuttcap%
\pgfsetroundjoin%
\definecolor{currentfill}{rgb}{0.274952,0.037752,0.364543}%
\pgfsetfillcolor{currentfill}%
\pgfsetlinewidth{0.000000pt}%
\definecolor{currentstroke}{rgb}{0.000000,0.000000,0.000000}%
\pgfsetstrokecolor{currentstroke}%
\pgfsetdash{}{0pt}%
\pgfpathmoveto{\pgfqpoint{1.324402in}{0.627455in}}%
\pgfpathlineto{\pgfqpoint{1.325450in}{0.622134in}}%
\pgfpathlineto{\pgfqpoint{1.326499in}{0.616999in}}%
\pgfpathlineto{\pgfqpoint{1.327550in}{0.612054in}}%
\pgfpathlineto{\pgfqpoint{1.328603in}{0.607304in}}%
\pgfpathlineto{\pgfqpoint{1.311390in}{0.604909in}}%
\pgfpathlineto{\pgfqpoint{1.294030in}{0.602810in}}%
\pgfpathlineto{\pgfqpoint{1.276543in}{0.601008in}}%
\pgfpathlineto{\pgfqpoint{1.258948in}{0.599506in}}%
\pgfpathlineto{\pgfqpoint{1.258388in}{0.604316in}}%
\pgfpathlineto{\pgfqpoint{1.257829in}{0.609321in}}%
\pgfpathlineto{\pgfqpoint{1.257270in}{0.614517in}}%
\pgfpathlineto{\pgfqpoint{1.256713in}{0.619898in}}%
\pgfpathlineto{\pgfqpoint{1.273811in}{0.621354in}}%
\pgfpathlineto{\pgfqpoint{1.290804in}{0.623100in}}%
\pgfpathlineto{\pgfqpoint{1.307674in}{0.625135in}}%
\pgfpathlineto{\pgfqpoint{1.324402in}{0.627455in}}%
\pgfpathclose%
\pgfusepath{fill}%
\end{pgfscope}%
\begin{pgfscope}%
\pgfpathrectangle{\pgfqpoint{0.041670in}{0.041670in}}{\pgfqpoint{2.216660in}{2.216660in}}%
\pgfusepath{clip}%
\pgfsetbuttcap%
\pgfsetroundjoin%
\definecolor{currentfill}{rgb}{0.855810,0.888601,0.097452}%
\pgfsetfillcolor{currentfill}%
\pgfsetlinewidth{0.000000pt}%
\definecolor{currentstroke}{rgb}{0.000000,0.000000,0.000000}%
\pgfsetstrokecolor{currentstroke}%
\pgfsetdash{}{0pt}%
\pgfpathmoveto{\pgfqpoint{1.108039in}{1.587382in}}%
\pgfpathlineto{\pgfqpoint{1.105040in}{1.583001in}}%
\pgfpathlineto{\pgfqpoint{1.102042in}{1.578496in}}%
\pgfpathlineto{\pgfqpoint{1.099043in}{1.573868in}}%
\pgfpathlineto{\pgfqpoint{1.096045in}{1.569119in}}%
\pgfpathlineto{\pgfqpoint{1.093869in}{1.570385in}}%
\pgfpathlineto{\pgfqpoint{1.091780in}{1.571683in}}%
\pgfpathlineto{\pgfqpoint{1.089780in}{1.573011in}}%
\pgfpathlineto{\pgfqpoint{1.087872in}{1.574368in}}%
\pgfpathlineto{\pgfqpoint{1.091159in}{1.578926in}}%
\pgfpathlineto{\pgfqpoint{1.094447in}{1.583363in}}%
\pgfpathlineto{\pgfqpoint{1.097736in}{1.587677in}}%
\pgfpathlineto{\pgfqpoint{1.101025in}{1.591869in}}%
\pgfpathlineto{\pgfqpoint{1.102663in}{1.590709in}}%
\pgfpathlineto{\pgfqpoint{1.104379in}{1.589574in}}%
\pgfpathlineto{\pgfqpoint{1.106172in}{1.588464in}}%
\pgfpathlineto{\pgfqpoint{1.108039in}{1.587382in}}%
\pgfpathclose%
\pgfusepath{fill}%
\end{pgfscope}%
\begin{pgfscope}%
\pgfpathrectangle{\pgfqpoint{0.041670in}{0.041670in}}{\pgfqpoint{2.216660in}{2.216660in}}%
\pgfusepath{clip}%
\pgfsetbuttcap%
\pgfsetroundjoin%
\definecolor{currentfill}{rgb}{0.487026,0.823929,0.312321}%
\pgfsetfillcolor{currentfill}%
\pgfsetlinewidth{0.000000pt}%
\definecolor{currentstroke}{rgb}{0.000000,0.000000,0.000000}%
\pgfsetstrokecolor{currentstroke}%
\pgfsetdash{}{0pt}%
\pgfpathmoveto{\pgfqpoint{1.052489in}{1.442140in}}%
\pgfpathlineto{\pgfqpoint{1.049843in}{1.434941in}}%
\pgfpathlineto{\pgfqpoint{1.047197in}{1.427649in}}%
\pgfpathlineto{\pgfqpoint{1.044552in}{1.420267in}}%
\pgfpathlineto{\pgfqpoint{1.041908in}{1.412796in}}%
\pgfpathlineto{\pgfqpoint{1.037286in}{1.414937in}}%
\pgfpathlineto{\pgfqpoint{1.032810in}{1.417147in}}%
\pgfpathlineto{\pgfqpoint{1.028485in}{1.419423in}}%
\pgfpathlineto{\pgfqpoint{1.024316in}{1.421763in}}%
\pgfpathlineto{\pgfqpoint{1.027293in}{1.429052in}}%
\pgfpathlineto{\pgfqpoint{1.030271in}{1.436253in}}%
\pgfpathlineto{\pgfqpoint{1.033250in}{1.443365in}}%
\pgfpathlineto{\pgfqpoint{1.036230in}{1.450384in}}%
\pgfpathlineto{\pgfqpoint{1.040084in}{1.448233in}}%
\pgfpathlineto{\pgfqpoint{1.044081in}{1.446140in}}%
\pgfpathlineto{\pgfqpoint{1.048218in}{1.444109in}}%
\pgfpathlineto{\pgfqpoint{1.052489in}{1.442140in}}%
\pgfpathclose%
\pgfusepath{fill}%
\end{pgfscope}%
\begin{pgfscope}%
\pgfpathrectangle{\pgfqpoint{0.041670in}{0.041670in}}{\pgfqpoint{2.216660in}{2.216660in}}%
\pgfusepath{clip}%
\pgfsetbuttcap%
\pgfsetroundjoin%
\definecolor{currentfill}{rgb}{0.212395,0.359683,0.551710}%
\pgfsetfillcolor{currentfill}%
\pgfsetlinewidth{0.000000pt}%
\definecolor{currentstroke}{rgb}{0.000000,0.000000,0.000000}%
\pgfsetstrokecolor{currentstroke}%
\pgfsetdash{}{0pt}%
\pgfpathmoveto{\pgfqpoint{1.134272in}{0.902328in}}%
\pgfpathlineto{\pgfqpoint{1.133836in}{0.892896in}}%
\pgfpathlineto{\pgfqpoint{1.133400in}{0.883512in}}%
\pgfpathlineto{\pgfqpoint{1.132965in}{0.874181in}}%
\pgfpathlineto{\pgfqpoint{1.132529in}{0.864906in}}%
\pgfpathlineto{\pgfqpoint{1.119280in}{0.865798in}}%
\pgfpathlineto{\pgfqpoint{1.106097in}{0.866908in}}%
\pgfpathlineto{\pgfqpoint{1.092994in}{0.868235in}}%
\pgfpathlineto{\pgfqpoint{1.079985in}{0.869777in}}%
\pgfpathlineto{\pgfqpoint{1.080903in}{0.879002in}}%
\pgfpathlineto{\pgfqpoint{1.081821in}{0.888282in}}%
\pgfpathlineto{\pgfqpoint{1.082739in}{0.897614in}}%
\pgfpathlineto{\pgfqpoint{1.083656in}{0.906996in}}%
\pgfpathlineto{\pgfqpoint{1.096188in}{0.905519in}}%
\pgfpathlineto{\pgfqpoint{1.108810in}{0.904247in}}%
\pgfpathlineto{\pgfqpoint{1.121509in}{0.903183in}}%
\pgfpathlineto{\pgfqpoint{1.134272in}{0.902328in}}%
\pgfpathclose%
\pgfusepath{fill}%
\end{pgfscope}%
\begin{pgfscope}%
\pgfpathrectangle{\pgfqpoint{0.041670in}{0.041670in}}{\pgfqpoint{2.216660in}{2.216660in}}%
\pgfusepath{clip}%
\pgfsetbuttcap%
\pgfsetroundjoin%
\definecolor{currentfill}{rgb}{0.565498,0.842430,0.262877}%
\pgfsetfillcolor{currentfill}%
\pgfsetlinewidth{0.000000pt}%
\definecolor{currentstroke}{rgb}{0.000000,0.000000,0.000000}%
\pgfsetstrokecolor{currentstroke}%
\pgfsetdash{}{0pt}%
\pgfpathmoveto{\pgfqpoint{1.314777in}{1.479302in}}%
\pgfpathlineto{\pgfqpoint{1.317830in}{1.472709in}}%
\pgfpathlineto{\pgfqpoint{1.320882in}{1.466017in}}%
\pgfpathlineto{\pgfqpoint{1.323932in}{1.459228in}}%
\pgfpathlineto{\pgfqpoint{1.326982in}{1.452345in}}%
\pgfpathlineto{\pgfqpoint{1.323258in}{1.450143in}}%
\pgfpathlineto{\pgfqpoint{1.319388in}{1.447997in}}%
\pgfpathlineto{\pgfqpoint{1.315376in}{1.445911in}}%
\pgfpathlineto{\pgfqpoint{1.311224in}{1.443887in}}%
\pgfpathlineto{\pgfqpoint{1.308498in}{1.450955in}}%
\pgfpathlineto{\pgfqpoint{1.305771in}{1.457927in}}%
\pgfpathlineto{\pgfqpoint{1.303043in}{1.464802in}}%
\pgfpathlineto{\pgfqpoint{1.300315in}{1.471579in}}%
\pgfpathlineto{\pgfqpoint{1.304124in}{1.473427in}}%
\pgfpathlineto{\pgfqpoint{1.307807in}{1.475332in}}%
\pgfpathlineto{\pgfqpoint{1.311359in}{1.477291in}}%
\pgfpathlineto{\pgfqpoint{1.314777in}{1.479302in}}%
\pgfpathclose%
\pgfusepath{fill}%
\end{pgfscope}%
\begin{pgfscope}%
\pgfpathrectangle{\pgfqpoint{0.041670in}{0.041670in}}{\pgfqpoint{2.216660in}{2.216660in}}%
\pgfusepath{clip}%
\pgfsetbuttcap%
\pgfsetroundjoin%
\definecolor{currentfill}{rgb}{0.974417,0.903590,0.130215}%
\pgfsetfillcolor{currentfill}%
\pgfsetlinewidth{0.000000pt}%
\definecolor{currentstroke}{rgb}{0.000000,0.000000,0.000000}%
\pgfsetstrokecolor{currentstroke}%
\pgfsetdash{}{0pt}%
\pgfpathmoveto{\pgfqpoint{1.191955in}{1.636011in}}%
\pgfpathlineto{\pgfqpoint{1.193458in}{1.633186in}}%
\pgfpathlineto{\pgfqpoint{1.194961in}{1.630228in}}%
\pgfpathlineto{\pgfqpoint{1.196465in}{1.627136in}}%
\pgfpathlineto{\pgfqpoint{1.197969in}{1.623911in}}%
\pgfpathlineto{\pgfqpoint{1.196589in}{1.623655in}}%
\pgfpathlineto{\pgfqpoint{1.195192in}{1.623419in}}%
\pgfpathlineto{\pgfqpoint{1.193779in}{1.623205in}}%
\pgfpathlineto{\pgfqpoint{1.192353in}{1.623011in}}%
\pgfpathlineto{\pgfqpoint{1.191317in}{1.626311in}}%
\pgfpathlineto{\pgfqpoint{1.190282in}{1.629479in}}%
\pgfpathlineto{\pgfqpoint{1.189247in}{1.632512in}}%
\pgfpathlineto{\pgfqpoint{1.188213in}{1.635413in}}%
\pgfpathlineto{\pgfqpoint{1.189163in}{1.635541in}}%
\pgfpathlineto{\pgfqpoint{1.190104in}{1.635684in}}%
\pgfpathlineto{\pgfqpoint{1.191035in}{1.635841in}}%
\pgfpathlineto{\pgfqpoint{1.191955in}{1.636011in}}%
\pgfpathclose%
\pgfusepath{fill}%
\end{pgfscope}%
\begin{pgfscope}%
\pgfpathrectangle{\pgfqpoint{0.041670in}{0.041670in}}{\pgfqpoint{2.216660in}{2.216660in}}%
\pgfusepath{clip}%
\pgfsetbuttcap%
\pgfsetroundjoin%
\definecolor{currentfill}{rgb}{0.133743,0.548535,0.553541}%
\pgfsetfillcolor{currentfill}%
\pgfsetlinewidth{0.000000pt}%
\definecolor{currentstroke}{rgb}{0.000000,0.000000,0.000000}%
\pgfsetstrokecolor{currentstroke}%
\pgfsetdash{}{0pt}%
\pgfpathmoveto{\pgfqpoint{1.062381in}{1.106353in}}%
\pgfpathlineto{\pgfqpoint{1.060995in}{1.096724in}}%
\pgfpathlineto{\pgfqpoint{1.059609in}{1.087084in}}%
\pgfpathlineto{\pgfqpoint{1.058223in}{1.077436in}}%
\pgfpathlineto{\pgfqpoint{1.056838in}{1.067781in}}%
\pgfpathlineto{\pgfqpoint{1.046762in}{1.069837in}}%
\pgfpathlineto{\pgfqpoint{1.036829in}{1.072053in}}%
\pgfpathlineto{\pgfqpoint{1.027048in}{1.074426in}}%
\pgfpathlineto{\pgfqpoint{1.017429in}{1.076953in}}%
\pgfpathlineto{\pgfqpoint{1.019256in}{1.086492in}}%
\pgfpathlineto{\pgfqpoint{1.021083in}{1.096026in}}%
\pgfpathlineto{\pgfqpoint{1.022910in}{1.105551in}}%
\pgfpathlineto{\pgfqpoint{1.024738in}{1.115065in}}%
\pgfpathlineto{\pgfqpoint{1.033927in}{1.112664in}}%
\pgfpathlineto{\pgfqpoint{1.043270in}{1.110410in}}%
\pgfpathlineto{\pgfqpoint{1.052758in}{1.108306in}}%
\pgfpathlineto{\pgfqpoint{1.062381in}{1.106353in}}%
\pgfpathclose%
\pgfusepath{fill}%
\end{pgfscope}%
\begin{pgfscope}%
\pgfpathrectangle{\pgfqpoint{0.041670in}{0.041670in}}{\pgfqpoint{2.216660in}{2.216660in}}%
\pgfusepath{clip}%
\pgfsetbuttcap%
\pgfsetroundjoin%
\definecolor{currentfill}{rgb}{0.163625,0.471133,0.558148}%
\pgfsetfillcolor{currentfill}%
\pgfsetlinewidth{0.000000pt}%
\definecolor{currentstroke}{rgb}{0.000000,0.000000,0.000000}%
\pgfsetstrokecolor{currentstroke}%
\pgfsetdash{}{0pt}%
\pgfpathmoveto{\pgfqpoint{1.317974in}{1.031076in}}%
\pgfpathlineto{\pgfqpoint{1.319458in}{1.021459in}}%
\pgfpathlineto{\pgfqpoint{1.320942in}{1.011854in}}%
\pgfpathlineto{\pgfqpoint{1.322426in}{1.002264in}}%
\pgfpathlineto{\pgfqpoint{1.323909in}{0.992692in}}%
\pgfpathlineto{\pgfqpoint{1.312915in}{0.990446in}}%
\pgfpathlineto{\pgfqpoint{1.301778in}{0.988379in}}%
\pgfpathlineto{\pgfqpoint{1.290511in}{0.986494in}}%
\pgfpathlineto{\pgfqpoint{1.279124in}{0.984792in}}%
\pgfpathlineto{\pgfqpoint{1.278101in}{0.994456in}}%
\pgfpathlineto{\pgfqpoint{1.277078in}{1.004138in}}%
\pgfpathlineto{\pgfqpoint{1.276055in}{1.013835in}}%
\pgfpathlineto{\pgfqpoint{1.275031in}{1.023543in}}%
\pgfpathlineto{\pgfqpoint{1.285949in}{1.025166in}}%
\pgfpathlineto{\pgfqpoint{1.296753in}{1.026964in}}%
\pgfpathlineto{\pgfqpoint{1.307432in}{1.028934in}}%
\pgfpathlineto{\pgfqpoint{1.317974in}{1.031076in}}%
\pgfpathclose%
\pgfusepath{fill}%
\end{pgfscope}%
\begin{pgfscope}%
\pgfpathrectangle{\pgfqpoint{0.041670in}{0.041670in}}{\pgfqpoint{2.216660in}{2.216660in}}%
\pgfusepath{clip}%
\pgfsetbuttcap%
\pgfsetroundjoin%
\definecolor{currentfill}{rgb}{0.974417,0.903590,0.130215}%
\pgfsetfillcolor{currentfill}%
\pgfsetlinewidth{0.000000pt}%
\definecolor{currentstroke}{rgb}{0.000000,0.000000,0.000000}%
\pgfsetstrokecolor{currentstroke}%
\pgfsetdash{}{0pt}%
\pgfpathmoveto{\pgfqpoint{1.172548in}{1.635310in}}%
\pgfpathlineto{\pgfqpoint{1.171621in}{1.632397in}}%
\pgfpathlineto{\pgfqpoint{1.170693in}{1.629350in}}%
\pgfpathlineto{\pgfqpoint{1.169764in}{1.626170in}}%
\pgfpathlineto{\pgfqpoint{1.168835in}{1.622856in}}%
\pgfpathlineto{\pgfqpoint{1.167398in}{1.623031in}}%
\pgfpathlineto{\pgfqpoint{1.165973in}{1.623228in}}%
\pgfpathlineto{\pgfqpoint{1.164562in}{1.623445in}}%
\pgfpathlineto{\pgfqpoint{1.163167in}{1.623682in}}%
\pgfpathlineto{\pgfqpoint{1.164569in}{1.626926in}}%
\pgfpathlineto{\pgfqpoint{1.165970in}{1.630037in}}%
\pgfpathlineto{\pgfqpoint{1.167371in}{1.633015in}}%
\pgfpathlineto{\pgfqpoint{1.168772in}{1.635859in}}%
\pgfpathlineto{\pgfqpoint{1.169701in}{1.635701in}}%
\pgfpathlineto{\pgfqpoint{1.170641in}{1.635557in}}%
\pgfpathlineto{\pgfqpoint{1.171591in}{1.635426in}}%
\pgfpathlineto{\pgfqpoint{1.172548in}{1.635310in}}%
\pgfpathclose%
\pgfusepath{fill}%
\end{pgfscope}%
\begin{pgfscope}%
\pgfpathrectangle{\pgfqpoint{0.041670in}{0.041670in}}{\pgfqpoint{2.216660in}{2.216660in}}%
\pgfusepath{clip}%
\pgfsetbuttcap%
\pgfsetroundjoin%
\definecolor{currentfill}{rgb}{0.955300,0.901065,0.118128}%
\pgfsetfillcolor{currentfill}%
\pgfsetlinewidth{0.000000pt}%
\definecolor{currentstroke}{rgb}{0.000000,0.000000,0.000000}%
\pgfsetstrokecolor{currentstroke}%
\pgfsetdash{}{0pt}%
\pgfpathmoveto{\pgfqpoint{1.212745in}{1.628468in}}%
\pgfpathlineto{\pgfqpoint{1.215482in}{1.625493in}}%
\pgfpathlineto{\pgfqpoint{1.218219in}{1.622387in}}%
\pgfpathlineto{\pgfqpoint{1.220957in}{1.619149in}}%
\pgfpathlineto{\pgfqpoint{1.223695in}{1.615781in}}%
\pgfpathlineto{\pgfqpoint{1.222258in}{1.615144in}}%
\pgfpathlineto{\pgfqpoint{1.220779in}{1.614529in}}%
\pgfpathlineto{\pgfqpoint{1.219259in}{1.613937in}}%
\pgfpathlineto{\pgfqpoint{1.217699in}{1.613366in}}%
\pgfpathlineto{\pgfqpoint{1.215336in}{1.616887in}}%
\pgfpathlineto{\pgfqpoint{1.212972in}{1.620277in}}%
\pgfpathlineto{\pgfqpoint{1.210610in}{1.623535in}}%
\pgfpathlineto{\pgfqpoint{1.208247in}{1.626661in}}%
\pgfpathlineto{\pgfqpoint{1.209417in}{1.627088in}}%
\pgfpathlineto{\pgfqpoint{1.210557in}{1.627531in}}%
\pgfpathlineto{\pgfqpoint{1.211667in}{1.627992in}}%
\pgfpathlineto{\pgfqpoint{1.212745in}{1.628468in}}%
\pgfpathclose%
\pgfusepath{fill}%
\end{pgfscope}%
\begin{pgfscope}%
\pgfpathrectangle{\pgfqpoint{0.041670in}{0.041670in}}{\pgfqpoint{2.216660in}{2.216660in}}%
\pgfusepath{clip}%
\pgfsetbuttcap%
\pgfsetroundjoin%
\definecolor{currentfill}{rgb}{0.935904,0.898570,0.108131}%
\pgfsetfillcolor{currentfill}%
\pgfsetlinewidth{0.000000pt}%
\definecolor{currentstroke}{rgb}{0.000000,0.000000,0.000000}%
\pgfsetstrokecolor{currentstroke}%
\pgfsetdash{}{0pt}%
\pgfpathmoveto{\pgfqpoint{1.228987in}{1.618529in}}%
\pgfpathlineto{\pgfqpoint{1.232055in}{1.615204in}}%
\pgfpathlineto{\pgfqpoint{1.235124in}{1.611749in}}%
\pgfpathlineto{\pgfqpoint{1.238192in}{1.608166in}}%
\pgfpathlineto{\pgfqpoint{1.241261in}{1.604455in}}%
\pgfpathlineto{\pgfqpoint{1.239697in}{1.603558in}}%
\pgfpathlineto{\pgfqpoint{1.238072in}{1.602684in}}%
\pgfpathlineto{\pgfqpoint{1.236390in}{1.601835in}}%
\pgfpathlineto{\pgfqpoint{1.234650in}{1.601011in}}%
\pgfpathlineto{\pgfqpoint{1.231911in}{1.604897in}}%
\pgfpathlineto{\pgfqpoint{1.229172in}{1.608654in}}%
\pgfpathlineto{\pgfqpoint{1.226433in}{1.612282in}}%
\pgfpathlineto{\pgfqpoint{1.223695in}{1.615781in}}%
\pgfpathlineto{\pgfqpoint{1.225087in}{1.616438in}}%
\pgfpathlineto{\pgfqpoint{1.226434in}{1.617116in}}%
\pgfpathlineto{\pgfqpoint{1.227735in}{1.617813in}}%
\pgfpathlineto{\pgfqpoint{1.228987in}{1.618529in}}%
\pgfpathclose%
\pgfusepath{fill}%
\end{pgfscope}%
\begin{pgfscope}%
\pgfpathrectangle{\pgfqpoint{0.041670in}{0.041670in}}{\pgfqpoint{2.216660in}{2.216660in}}%
\pgfusepath{clip}%
\pgfsetbuttcap%
\pgfsetroundjoin%
\definecolor{currentfill}{rgb}{0.231674,0.318106,0.544834}%
\pgfsetfillcolor{currentfill}%
\pgfsetlinewidth{0.000000pt}%
\definecolor{currentstroke}{rgb}{0.000000,0.000000,0.000000}%
\pgfsetstrokecolor{currentstroke}%
\pgfsetdash{}{0pt}%
\pgfpathmoveto{\pgfqpoint{1.239161in}{0.865688in}}%
\pgfpathlineto{\pgfqpoint{1.239705in}{0.856480in}}%
\pgfpathlineto{\pgfqpoint{1.240249in}{0.847335in}}%
\pgfpathlineto{\pgfqpoint{1.240793in}{0.838255in}}%
\pgfpathlineto{\pgfqpoint{1.241337in}{0.829244in}}%
\pgfpathlineto{\pgfqpoint{1.227595in}{0.828340in}}%
\pgfpathlineto{\pgfqpoint{1.213800in}{0.827664in}}%
\pgfpathlineto{\pgfqpoint{1.199968in}{0.827217in}}%
\pgfpathlineto{\pgfqpoint{1.186115in}{0.827000in}}%
\pgfpathlineto{\pgfqpoint{1.186060in}{0.836033in}}%
\pgfpathlineto{\pgfqpoint{1.186006in}{0.845136in}}%
\pgfpathlineto{\pgfqpoint{1.185951in}{0.854304in}}%
\pgfpathlineto{\pgfqpoint{1.185897in}{0.863534in}}%
\pgfpathlineto{\pgfqpoint{1.199259in}{0.863743in}}%
\pgfpathlineto{\pgfqpoint{1.212600in}{0.864172in}}%
\pgfpathlineto{\pgfqpoint{1.225905in}{0.864820in}}%
\pgfpathlineto{\pgfqpoint{1.239161in}{0.865688in}}%
\pgfpathclose%
\pgfusepath{fill}%
\end{pgfscope}%
\begin{pgfscope}%
\pgfpathrectangle{\pgfqpoint{0.041670in}{0.041670in}}{\pgfqpoint{2.216660in}{2.216660in}}%
\pgfusepath{clip}%
\pgfsetbuttcap%
\pgfsetroundjoin%
\definecolor{currentfill}{rgb}{0.974417,0.903590,0.130215}%
\pgfsetfillcolor{currentfill}%
\pgfsetlinewidth{0.000000pt}%
\definecolor{currentstroke}{rgb}{0.000000,0.000000,0.000000}%
\pgfsetstrokecolor{currentstroke}%
\pgfsetdash{}{0pt}%
\pgfpathmoveto{\pgfqpoint{1.188213in}{1.635413in}}%
\pgfpathlineto{\pgfqpoint{1.189247in}{1.632512in}}%
\pgfpathlineto{\pgfqpoint{1.190282in}{1.629479in}}%
\pgfpathlineto{\pgfqpoint{1.191317in}{1.626311in}}%
\pgfpathlineto{\pgfqpoint{1.192353in}{1.623011in}}%
\pgfpathlineto{\pgfqpoint{1.190914in}{1.622838in}}%
\pgfpathlineto{\pgfqpoint{1.189464in}{1.622687in}}%
\pgfpathlineto{\pgfqpoint{1.188004in}{1.622557in}}%
\pgfpathlineto{\pgfqpoint{1.186537in}{1.622449in}}%
\pgfpathlineto{\pgfqpoint{1.185987in}{1.625797in}}%
\pgfpathlineto{\pgfqpoint{1.185438in}{1.629011in}}%
\pgfpathlineto{\pgfqpoint{1.184888in}{1.632092in}}%
\pgfpathlineto{\pgfqpoint{1.184339in}{1.635039in}}%
\pgfpathlineto{\pgfqpoint{1.185317in}{1.635111in}}%
\pgfpathlineto{\pgfqpoint{1.186289in}{1.635197in}}%
\pgfpathlineto{\pgfqpoint{1.187255in}{1.635298in}}%
\pgfpathlineto{\pgfqpoint{1.188213in}{1.635413in}}%
\pgfpathclose%
\pgfusepath{fill}%
\end{pgfscope}%
\begin{pgfscope}%
\pgfpathrectangle{\pgfqpoint{0.041670in}{0.041670in}}{\pgfqpoint{2.216660in}{2.216660in}}%
\pgfusepath{clip}%
\pgfsetbuttcap%
\pgfsetroundjoin%
\definecolor{currentfill}{rgb}{0.955300,0.901065,0.118128}%
\pgfsetfillcolor{currentfill}%
\pgfsetlinewidth{0.000000pt}%
\definecolor{currentstroke}{rgb}{0.000000,0.000000,0.000000}%
\pgfsetstrokecolor{currentstroke}%
\pgfsetdash{}{0pt}%
\pgfpathmoveto{\pgfqpoint{1.152726in}{1.626297in}}%
\pgfpathlineto{\pgfqpoint{1.150453in}{1.623140in}}%
\pgfpathlineto{\pgfqpoint{1.148179in}{1.619851in}}%
\pgfpathlineto{\pgfqpoint{1.145904in}{1.616431in}}%
\pgfpathlineto{\pgfqpoint{1.143629in}{1.612879in}}%
\pgfpathlineto{\pgfqpoint{1.142035in}{1.613429in}}%
\pgfpathlineto{\pgfqpoint{1.140480in}{1.614001in}}%
\pgfpathlineto{\pgfqpoint{1.138964in}{1.614597in}}%
\pgfpathlineto{\pgfqpoint{1.137490in}{1.615214in}}%
\pgfpathlineto{\pgfqpoint{1.140148in}{1.618618in}}%
\pgfpathlineto{\pgfqpoint{1.142806in}{1.621891in}}%
\pgfpathlineto{\pgfqpoint{1.145464in}{1.625033in}}%
\pgfpathlineto{\pgfqpoint{1.148121in}{1.628044in}}%
\pgfpathlineto{\pgfqpoint{1.149227in}{1.627582in}}%
\pgfpathlineto{\pgfqpoint{1.150364in}{1.627136in}}%
\pgfpathlineto{\pgfqpoint{1.151531in}{1.626708in}}%
\pgfpathlineto{\pgfqpoint{1.152726in}{1.626297in}}%
\pgfpathclose%
\pgfusepath{fill}%
\end{pgfscope}%
\begin{pgfscope}%
\pgfpathrectangle{\pgfqpoint{0.041670in}{0.041670in}}{\pgfqpoint{2.216660in}{2.216660in}}%
\pgfusepath{clip}%
\pgfsetbuttcap%
\pgfsetroundjoin%
\definecolor{currentfill}{rgb}{0.231674,0.318106,0.544834}%
\pgfsetfillcolor{currentfill}%
\pgfsetlinewidth{0.000000pt}%
\definecolor{currentstroke}{rgb}{0.000000,0.000000,0.000000}%
\pgfsetstrokecolor{currentstroke}%
\pgfsetdash{}{0pt}%
\pgfpathmoveto{\pgfqpoint{1.185897in}{0.863534in}}%
\pgfpathlineto{\pgfqpoint{1.185951in}{0.854304in}}%
\pgfpathlineto{\pgfqpoint{1.186006in}{0.845136in}}%
\pgfpathlineto{\pgfqpoint{1.186060in}{0.836033in}}%
\pgfpathlineto{\pgfqpoint{1.186115in}{0.827000in}}%
\pgfpathlineto{\pgfqpoint{1.172255in}{0.827013in}}%
\pgfpathlineto{\pgfqpoint{1.158403in}{0.827256in}}%
\pgfpathlineto{\pgfqpoint{1.144575in}{0.827728in}}%
\pgfpathlineto{\pgfqpoint{1.130785in}{0.828429in}}%
\pgfpathlineto{\pgfqpoint{1.131221in}{0.837448in}}%
\pgfpathlineto{\pgfqpoint{1.131657in}{0.846536in}}%
\pgfpathlineto{\pgfqpoint{1.132093in}{0.855690in}}%
\pgfpathlineto{\pgfqpoint{1.132529in}{0.864906in}}%
\pgfpathlineto{\pgfqpoint{1.145830in}{0.864233in}}%
\pgfpathlineto{\pgfqpoint{1.159168in}{0.863779in}}%
\pgfpathlineto{\pgfqpoint{1.172528in}{0.863547in}}%
\pgfpathlineto{\pgfqpoint{1.185897in}{0.863534in}}%
\pgfpathclose%
\pgfusepath{fill}%
\end{pgfscope}%
\begin{pgfscope}%
\pgfpathrectangle{\pgfqpoint{0.041670in}{0.041670in}}{\pgfqpoint{2.216660in}{2.216660in}}%
\pgfusepath{clip}%
\pgfsetbuttcap%
\pgfsetroundjoin%
\definecolor{currentfill}{rgb}{0.281477,0.755203,0.432552}%
\pgfsetfillcolor{currentfill}%
\pgfsetlinewidth{0.000000pt}%
\definecolor{currentstroke}{rgb}{0.000000,0.000000,0.000000}%
\pgfsetstrokecolor{currentstroke}%
\pgfsetdash{}{0pt}%
\pgfpathmoveto{\pgfqpoint{1.043672in}{1.340946in}}%
\pgfpathlineto{\pgfqpoint{1.041413in}{1.332600in}}%
\pgfpathlineto{\pgfqpoint{1.039156in}{1.324186in}}%
\pgfpathlineto{\pgfqpoint{1.036899in}{1.315703in}}%
\pgfpathlineto{\pgfqpoint{1.034643in}{1.307156in}}%
\pgfpathlineto{\pgfqpoint{1.028306in}{1.309457in}}%
\pgfpathlineto{\pgfqpoint{1.022128in}{1.311854in}}%
\pgfpathlineto{\pgfqpoint{1.016112in}{1.314344in}}%
\pgfpathlineto{\pgfqpoint{1.010267in}{1.316925in}}%
\pgfpathlineto{\pgfqpoint{1.012898in}{1.325309in}}%
\pgfpathlineto{\pgfqpoint{1.015530in}{1.333628in}}%
\pgfpathlineto{\pgfqpoint{1.018162in}{1.341880in}}%
\pgfpathlineto{\pgfqpoint{1.020796in}{1.350063in}}%
\pgfpathlineto{\pgfqpoint{1.026282in}{1.347654in}}%
\pgfpathlineto{\pgfqpoint{1.031927in}{1.345330in}}%
\pgfpathlineto{\pgfqpoint{1.037726in}{1.343093in}}%
\pgfpathlineto{\pgfqpoint{1.043672in}{1.340946in}}%
\pgfpathclose%
\pgfusepath{fill}%
\end{pgfscope}%
\begin{pgfscope}%
\pgfpathrectangle{\pgfqpoint{0.041670in}{0.041670in}}{\pgfqpoint{2.216660in}{2.216660in}}%
\pgfusepath{clip}%
\pgfsetbuttcap%
\pgfsetroundjoin%
\definecolor{currentfill}{rgb}{0.974417,0.903590,0.130215}%
\pgfsetfillcolor{currentfill}%
\pgfsetlinewidth{0.000000pt}%
\definecolor{currentstroke}{rgb}{0.000000,0.000000,0.000000}%
\pgfsetstrokecolor{currentstroke}%
\pgfsetdash{}{0pt}%
\pgfpathmoveto{\pgfqpoint{1.176443in}{1.634987in}}%
\pgfpathlineto{\pgfqpoint{1.176003in}{1.632034in}}%
\pgfpathlineto{\pgfqpoint{1.175563in}{1.628947in}}%
\pgfpathlineto{\pgfqpoint{1.175123in}{1.625726in}}%
\pgfpathlineto{\pgfqpoint{1.174683in}{1.622371in}}%
\pgfpathlineto{\pgfqpoint{1.173209in}{1.622460in}}%
\pgfpathlineto{\pgfqpoint{1.171743in}{1.622571in}}%
\pgfpathlineto{\pgfqpoint{1.170284in}{1.622703in}}%
\pgfpathlineto{\pgfqpoint{1.168835in}{1.622856in}}%
\pgfpathlineto{\pgfqpoint{1.169764in}{1.626170in}}%
\pgfpathlineto{\pgfqpoint{1.170693in}{1.629350in}}%
\pgfpathlineto{\pgfqpoint{1.171621in}{1.632397in}}%
\pgfpathlineto{\pgfqpoint{1.172548in}{1.635310in}}%
\pgfpathlineto{\pgfqpoint{1.173513in}{1.635208in}}%
\pgfpathlineto{\pgfqpoint{1.174485in}{1.635120in}}%
\pgfpathlineto{\pgfqpoint{1.175462in}{1.635046in}}%
\pgfpathlineto{\pgfqpoint{1.176443in}{1.634987in}}%
\pgfpathclose%
\pgfusepath{fill}%
\end{pgfscope}%
\begin{pgfscope}%
\pgfpathrectangle{\pgfqpoint{0.041670in}{0.041670in}}{\pgfqpoint{2.216660in}{2.216660in}}%
\pgfusepath{clip}%
\pgfsetbuttcap%
\pgfsetroundjoin%
\definecolor{currentfill}{rgb}{0.134692,0.658636,0.517649}%
\pgfsetfillcolor{currentfill}%
\pgfsetlinewidth{0.000000pt}%
\definecolor{currentstroke}{rgb}{0.000000,0.000000,0.000000}%
\pgfsetstrokecolor{currentstroke}%
\pgfsetdash{}{0pt}%
\pgfpathmoveto{\pgfqpoint{1.046724in}{1.227391in}}%
\pgfpathlineto{\pgfqpoint{1.044888in}{1.218218in}}%
\pgfpathlineto{\pgfqpoint{1.043053in}{1.209002in}}%
\pgfpathlineto{\pgfqpoint{1.041218in}{1.199746in}}%
\pgfpathlineto{\pgfqpoint{1.039385in}{1.190453in}}%
\pgfpathlineto{\pgfqpoint{1.031207in}{1.192733in}}%
\pgfpathlineto{\pgfqpoint{1.023186in}{1.195140in}}%
\pgfpathlineto{\pgfqpoint{1.015330in}{1.197671in}}%
\pgfpathlineto{\pgfqpoint{1.007647in}{1.200323in}}%
\pgfpathlineto{\pgfqpoint{1.009891in}{1.209475in}}%
\pgfpathlineto{\pgfqpoint{1.012137in}{1.218589in}}%
\pgfpathlineto{\pgfqpoint{1.014383in}{1.227664in}}%
\pgfpathlineto{\pgfqpoint{1.016631in}{1.236697in}}%
\pgfpathlineto{\pgfqpoint{1.023916in}{1.234196in}}%
\pgfpathlineto{\pgfqpoint{1.031366in}{1.231810in}}%
\pgfpathlineto{\pgfqpoint{1.038971in}{1.229541in}}%
\pgfpathlineto{\pgfqpoint{1.046724in}{1.227391in}}%
\pgfpathclose%
\pgfusepath{fill}%
\end{pgfscope}%
\begin{pgfscope}%
\pgfpathrectangle{\pgfqpoint{0.041670in}{0.041670in}}{\pgfqpoint{2.216660in}{2.216660in}}%
\pgfusepath{clip}%
\pgfsetbuttcap%
\pgfsetroundjoin%
\definecolor{currentfill}{rgb}{0.274952,0.037752,0.364543}%
\pgfsetfillcolor{currentfill}%
\pgfsetlinewidth{0.000000pt}%
\definecolor{currentstroke}{rgb}{0.000000,0.000000,0.000000}%
\pgfsetstrokecolor{currentstroke}%
\pgfsetdash{}{0pt}%
\pgfpathmoveto{\pgfqpoint{1.118467in}{0.618849in}}%
\pgfpathlineto{\pgfqpoint{1.118020in}{0.613460in}}%
\pgfpathlineto{\pgfqpoint{1.117573in}{0.608256in}}%
\pgfpathlineto{\pgfqpoint{1.117125in}{0.603242in}}%
\pgfpathlineto{\pgfqpoint{1.116677in}{0.598423in}}%
\pgfpathlineto{\pgfqpoint{1.099002in}{0.599658in}}%
\pgfpathlineto{\pgfqpoint{1.081418in}{0.601193in}}%
\pgfpathlineto{\pgfqpoint{1.063944in}{0.603028in}}%
\pgfpathlineto{\pgfqpoint{1.046600in}{0.605161in}}%
\pgfpathlineto{\pgfqpoint{1.047545in}{0.609928in}}%
\pgfpathlineto{\pgfqpoint{1.048488in}{0.614889in}}%
\pgfpathlineto{\pgfqpoint{1.049430in}{0.620041in}}%
\pgfpathlineto{\pgfqpoint{1.050370in}{0.625378in}}%
\pgfpathlineto{\pgfqpoint{1.067225in}{0.623312in}}%
\pgfpathlineto{\pgfqpoint{1.084205in}{0.621533in}}%
\pgfpathlineto{\pgfqpoint{1.101292in}{0.620045in}}%
\pgfpathlineto{\pgfqpoint{1.118467in}{0.618849in}}%
\pgfpathclose%
\pgfusepath{fill}%
\end{pgfscope}%
\begin{pgfscope}%
\pgfpathrectangle{\pgfqpoint{0.041670in}{0.041670in}}{\pgfqpoint{2.216660in}{2.216660in}}%
\pgfusepath{clip}%
\pgfsetbuttcap%
\pgfsetroundjoin%
\definecolor{currentfill}{rgb}{0.195860,0.395433,0.555276}%
\pgfsetfillcolor{currentfill}%
\pgfsetlinewidth{0.000000pt}%
\definecolor{currentstroke}{rgb}{0.000000,0.000000,0.000000}%
\pgfsetstrokecolor{currentstroke}%
\pgfsetdash{}{0pt}%
\pgfpathmoveto{\pgfqpoint{1.283215in}{0.946372in}}%
\pgfpathlineto{\pgfqpoint{1.284238in}{0.936842in}}%
\pgfpathlineto{\pgfqpoint{1.285260in}{0.927348in}}%
\pgfpathlineto{\pgfqpoint{1.286283in}{0.917893in}}%
\pgfpathlineto{\pgfqpoint{1.287306in}{0.908482in}}%
\pgfpathlineto{\pgfqpoint{1.274866in}{0.906822in}}%
\pgfpathlineto{\pgfqpoint{1.262324in}{0.905367in}}%
\pgfpathlineto{\pgfqpoint{1.249693in}{0.904119in}}%
\pgfpathlineto{\pgfqpoint{1.236986in}{0.903078in}}%
\pgfpathlineto{\pgfqpoint{1.236442in}{0.912548in}}%
\pgfpathlineto{\pgfqpoint{1.235898in}{0.922061in}}%
\pgfpathlineto{\pgfqpoint{1.235355in}{0.931613in}}%
\pgfpathlineto{\pgfqpoint{1.234811in}{0.941202in}}%
\pgfpathlineto{\pgfqpoint{1.247034in}{0.942197in}}%
\pgfpathlineto{\pgfqpoint{1.259184in}{0.943392in}}%
\pgfpathlineto{\pgfqpoint{1.271249in}{0.944784in}}%
\pgfpathlineto{\pgfqpoint{1.283215in}{0.946372in}}%
\pgfpathclose%
\pgfusepath{fill}%
\end{pgfscope}%
\begin{pgfscope}%
\pgfpathrectangle{\pgfqpoint{0.041670in}{0.041670in}}{\pgfqpoint{2.216660in}{2.216660in}}%
\pgfusepath{clip}%
\pgfsetbuttcap%
\pgfsetroundjoin%
\definecolor{currentfill}{rgb}{0.974417,0.903590,0.130215}%
\pgfsetfillcolor{currentfill}%
\pgfsetlinewidth{0.000000pt}%
\definecolor{currentstroke}{rgb}{0.000000,0.000000,0.000000}%
\pgfsetstrokecolor{currentstroke}%
\pgfsetdash{}{0pt}%
\pgfpathmoveto{\pgfqpoint{1.184339in}{1.635039in}}%
\pgfpathlineto{\pgfqpoint{1.184888in}{1.632092in}}%
\pgfpathlineto{\pgfqpoint{1.185438in}{1.629011in}}%
\pgfpathlineto{\pgfqpoint{1.185987in}{1.625797in}}%
\pgfpathlineto{\pgfqpoint{1.186537in}{1.622449in}}%
\pgfpathlineto{\pgfqpoint{1.185063in}{1.622363in}}%
\pgfpathlineto{\pgfqpoint{1.183583in}{1.622298in}}%
\pgfpathlineto{\pgfqpoint{1.182100in}{1.622256in}}%
\pgfpathlineto{\pgfqpoint{1.180615in}{1.622235in}}%
\pgfpathlineto{\pgfqpoint{1.180560in}{1.625601in}}%
\pgfpathlineto{\pgfqpoint{1.180505in}{1.628833in}}%
\pgfpathlineto{\pgfqpoint{1.180450in}{1.631932in}}%
\pgfpathlineto{\pgfqpoint{1.180395in}{1.634897in}}%
\pgfpathlineto{\pgfqpoint{1.181384in}{1.634910in}}%
\pgfpathlineto{\pgfqpoint{1.182372in}{1.634939in}}%
\pgfpathlineto{\pgfqpoint{1.183357in}{1.634982in}}%
\pgfpathlineto{\pgfqpoint{1.184339in}{1.635039in}}%
\pgfpathclose%
\pgfusepath{fill}%
\end{pgfscope}%
\begin{pgfscope}%
\pgfpathrectangle{\pgfqpoint{0.041670in}{0.041670in}}{\pgfqpoint{2.216660in}{2.216660in}}%
\pgfusepath{clip}%
\pgfsetbuttcap%
\pgfsetroundjoin%
\definecolor{currentfill}{rgb}{0.935904,0.898570,0.108131}%
\pgfsetfillcolor{currentfill}%
\pgfsetlinewidth{0.000000pt}%
\definecolor{currentstroke}{rgb}{0.000000,0.000000,0.000000}%
\pgfsetstrokecolor{currentstroke}%
\pgfsetdash{}{0pt}%
\pgfpathmoveto{\pgfqpoint{1.137490in}{1.615214in}}%
\pgfpathlineto{\pgfqpoint{1.134831in}{1.611680in}}%
\pgfpathlineto{\pgfqpoint{1.132172in}{1.608016in}}%
\pgfpathlineto{\pgfqpoint{1.129512in}{1.604222in}}%
\pgfpathlineto{\pgfqpoint{1.126853in}{1.600301in}}%
\pgfpathlineto{\pgfqpoint{1.125064in}{1.601102in}}%
\pgfpathlineto{\pgfqpoint{1.123330in}{1.601928in}}%
\pgfpathlineto{\pgfqpoint{1.121654in}{1.602780in}}%
\pgfpathlineto{\pgfqpoint{1.120036in}{1.603656in}}%
\pgfpathlineto{\pgfqpoint{1.123036in}{1.607408in}}%
\pgfpathlineto{\pgfqpoint{1.126035in}{1.611031in}}%
\pgfpathlineto{\pgfqpoint{1.129034in}{1.614526in}}%
\pgfpathlineto{\pgfqpoint{1.132033in}{1.617891in}}%
\pgfpathlineto{\pgfqpoint{1.133329in}{1.617192in}}%
\pgfpathlineto{\pgfqpoint{1.134670in}{1.616512in}}%
\pgfpathlineto{\pgfqpoint{1.136058in}{1.615853in}}%
\pgfpathlineto{\pgfqpoint{1.137490in}{1.615214in}}%
\pgfpathclose%
\pgfusepath{fill}%
\end{pgfscope}%
\begin{pgfscope}%
\pgfpathrectangle{\pgfqpoint{0.041670in}{0.041670in}}{\pgfqpoint{2.216660in}{2.216660in}}%
\pgfusepath{clip}%
\pgfsetbuttcap%
\pgfsetroundjoin%
\definecolor{currentfill}{rgb}{0.974417,0.903590,0.130215}%
\pgfsetfillcolor{currentfill}%
\pgfsetlinewidth{0.000000pt}%
\definecolor{currentstroke}{rgb}{0.000000,0.000000,0.000000}%
\pgfsetstrokecolor{currentstroke}%
\pgfsetdash{}{0pt}%
\pgfpathmoveto{\pgfqpoint{1.180395in}{1.634897in}}%
\pgfpathlineto{\pgfqpoint{1.180450in}{1.631932in}}%
\pgfpathlineto{\pgfqpoint{1.180505in}{1.628833in}}%
\pgfpathlineto{\pgfqpoint{1.180560in}{1.625601in}}%
\pgfpathlineto{\pgfqpoint{1.180615in}{1.622235in}}%
\pgfpathlineto{\pgfqpoint{1.179129in}{1.622236in}}%
\pgfpathlineto{\pgfqpoint{1.177644in}{1.622259in}}%
\pgfpathlineto{\pgfqpoint{1.176162in}{1.622304in}}%
\pgfpathlineto{\pgfqpoint{1.174683in}{1.622371in}}%
\pgfpathlineto{\pgfqpoint{1.175123in}{1.625726in}}%
\pgfpathlineto{\pgfqpoint{1.175563in}{1.628947in}}%
\pgfpathlineto{\pgfqpoint{1.176003in}{1.632034in}}%
\pgfpathlineto{\pgfqpoint{1.176443in}{1.634987in}}%
\pgfpathlineto{\pgfqpoint{1.177428in}{1.634943in}}%
\pgfpathlineto{\pgfqpoint{1.178416in}{1.634913in}}%
\pgfpathlineto{\pgfqpoint{1.179405in}{1.634897in}}%
\pgfpathlineto{\pgfqpoint{1.180395in}{1.634897in}}%
\pgfpathclose%
\pgfusepath{fill}%
\end{pgfscope}%
\begin{pgfscope}%
\pgfpathrectangle{\pgfqpoint{0.041670in}{0.041670in}}{\pgfqpoint{2.216660in}{2.216660in}}%
\pgfusepath{clip}%
\pgfsetbuttcap%
\pgfsetroundjoin%
\definecolor{currentfill}{rgb}{0.279566,0.067836,0.391917}%
\pgfsetfillcolor{currentfill}%
\pgfsetlinewidth{0.000000pt}%
\definecolor{currentstroke}{rgb}{0.000000,0.000000,0.000000}%
\pgfsetstrokecolor{currentstroke}%
\pgfsetdash{}{0pt}%
\pgfpathmoveto{\pgfqpoint{1.254491in}{0.643193in}}%
\pgfpathlineto{\pgfqpoint{1.255046in}{0.637113in}}%
\pgfpathlineto{\pgfqpoint{1.255601in}{0.631200in}}%
\pgfpathlineto{\pgfqpoint{1.256157in}{0.625460in}}%
\pgfpathlineto{\pgfqpoint{1.256713in}{0.619898in}}%
\pgfpathlineto{\pgfqpoint{1.239530in}{0.618734in}}%
\pgfpathlineto{\pgfqpoint{1.222280in}{0.617865in}}%
\pgfpathlineto{\pgfqpoint{1.204983in}{0.617290in}}%
\pgfpathlineto{\pgfqpoint{1.187659in}{0.617010in}}%
\pgfpathlineto{\pgfqpoint{1.187603in}{0.622596in}}%
\pgfpathlineto{\pgfqpoint{1.187547in}{0.628359in}}%
\pgfpathlineto{\pgfqpoint{1.187491in}{0.634294in}}%
\pgfpathlineto{\pgfqpoint{1.187436in}{0.640398in}}%
\pgfpathlineto{\pgfqpoint{1.204259in}{0.640669in}}%
\pgfpathlineto{\pgfqpoint{1.221055in}{0.641225in}}%
\pgfpathlineto{\pgfqpoint{1.237805in}{0.642067in}}%
\pgfpathlineto{\pgfqpoint{1.254491in}{0.643193in}}%
\pgfpathclose%
\pgfusepath{fill}%
\end{pgfscope}%
\begin{pgfscope}%
\pgfpathrectangle{\pgfqpoint{0.041670in}{0.041670in}}{\pgfqpoint{2.216660in}{2.216660in}}%
\pgfusepath{clip}%
\pgfsetbuttcap%
\pgfsetroundjoin%
\definecolor{currentfill}{rgb}{0.122606,0.585371,0.546557}%
\pgfsetfillcolor{currentfill}%
\pgfsetlinewidth{0.000000pt}%
\definecolor{currentstroke}{rgb}{0.000000,0.000000,0.000000}%
\pgfsetstrokecolor{currentstroke}%
\pgfsetdash{}{0pt}%
\pgfpathmoveto{\pgfqpoint{1.335507in}{1.155093in}}%
\pgfpathlineto{\pgfqpoint{1.337431in}{1.145680in}}%
\pgfpathlineto{\pgfqpoint{1.339355in}{1.136245in}}%
\pgfpathlineto{\pgfqpoint{1.341279in}{1.126790in}}%
\pgfpathlineto{\pgfqpoint{1.343201in}{1.117320in}}%
\pgfpathlineto{\pgfqpoint{1.334158in}{1.114791in}}%
\pgfpathlineto{\pgfqpoint{1.324952in}{1.112406in}}%
\pgfpathlineto{\pgfqpoint{1.315593in}{1.110169in}}%
\pgfpathlineto{\pgfqpoint{1.306089in}{1.108081in}}%
\pgfpathlineto{\pgfqpoint{1.304602in}{1.117673in}}%
\pgfpathlineto{\pgfqpoint{1.303114in}{1.127248in}}%
\pgfpathlineto{\pgfqpoint{1.301625in}{1.136804in}}%
\pgfpathlineto{\pgfqpoint{1.300136in}{1.146337in}}%
\pgfpathlineto{\pgfqpoint{1.309193in}{1.148316in}}%
\pgfpathlineto{\pgfqpoint{1.318114in}{1.150436in}}%
\pgfpathlineto{\pgfqpoint{1.326888in}{1.152696in}}%
\pgfpathlineto{\pgfqpoint{1.335507in}{1.155093in}}%
\pgfpathclose%
\pgfusepath{fill}%
\end{pgfscope}%
\begin{pgfscope}%
\pgfpathrectangle{\pgfqpoint{0.041670in}{0.041670in}}{\pgfqpoint{2.216660in}{2.216660in}}%
\pgfusepath{clip}%
\pgfsetbuttcap%
\pgfsetroundjoin%
\definecolor{currentfill}{rgb}{0.272594,0.025563,0.353093}%
\pgfsetfillcolor{currentfill}%
\pgfsetlinewidth{0.000000pt}%
\definecolor{currentstroke}{rgb}{0.000000,0.000000,0.000000}%
\pgfsetstrokecolor{currentstroke}%
\pgfsetdash{}{0pt}%
\pgfpathmoveto{\pgfqpoint{0.876505in}{0.589433in}}%
\pgfpathlineto{\pgfqpoint{0.874540in}{0.590062in}}%
\pgfpathlineto{\pgfqpoint{0.872569in}{0.590994in}}%
\pgfpathlineto{\pgfqpoint{0.870592in}{0.592236in}}%
\pgfpathlineto{\pgfqpoint{0.868609in}{0.593792in}}%
\pgfpathlineto{\pgfqpoint{0.850629in}{0.599277in}}%
\pgfpathlineto{\pgfqpoint{0.833017in}{0.605061in}}%
\pgfpathlineto{\pgfqpoint{0.815793in}{0.611139in}}%
\pgfpathlineto{\pgfqpoint{0.798975in}{0.617502in}}%
\pgfpathlineto{\pgfqpoint{0.801394in}{0.615799in}}%
\pgfpathlineto{\pgfqpoint{0.803806in}{0.614411in}}%
\pgfpathlineto{\pgfqpoint{0.806210in}{0.613332in}}%
\pgfpathlineto{\pgfqpoint{0.808607in}{0.612556in}}%
\pgfpathlineto{\pgfqpoint{0.825007in}{0.606350in}}%
\pgfpathlineto{\pgfqpoint{0.841803in}{0.600423in}}%
\pgfpathlineto{\pgfqpoint{0.858975in}{0.594781in}}%
\pgfpathlineto{\pgfqpoint{0.876505in}{0.589433in}}%
\pgfpathclose%
\pgfusepath{fill}%
\end{pgfscope}%
\begin{pgfscope}%
\pgfpathrectangle{\pgfqpoint{0.041670in}{0.041670in}}{\pgfqpoint{2.216660in}{2.216660in}}%
\pgfusepath{clip}%
\pgfsetbuttcap%
\pgfsetroundjoin%
\definecolor{currentfill}{rgb}{0.268510,0.009605,0.335427}%
\pgfsetfillcolor{currentfill}%
\pgfsetlinewidth{0.000000pt}%
\definecolor{currentstroke}{rgb}{0.000000,0.000000,0.000000}%
\pgfsetstrokecolor{currentstroke}%
\pgfsetdash{}{0pt}%
\pgfpathmoveto{\pgfqpoint{1.042804in}{0.588129in}}%
\pgfpathlineto{\pgfqpoint{1.041851in}{0.584403in}}%
\pgfpathlineto{\pgfqpoint{1.040895in}{0.580899in}}%
\pgfpathlineto{\pgfqpoint{1.039938in}{0.577622in}}%
\pgfpathlineto{\pgfqpoint{1.038979in}{0.574577in}}%
\pgfpathlineto{\pgfqpoint{1.020803in}{0.577153in}}%
\pgfpathlineto{\pgfqpoint{1.002806in}{0.580039in}}%
\pgfpathlineto{\pgfqpoint{0.985008in}{0.583230in}}%
\pgfpathlineto{\pgfqpoint{0.967427in}{0.586723in}}%
\pgfpathlineto{\pgfqpoint{0.968871in}{0.589681in}}%
\pgfpathlineto{\pgfqpoint{0.970312in}{0.592871in}}%
\pgfpathlineto{\pgfqpoint{0.971749in}{0.596287in}}%
\pgfpathlineto{\pgfqpoint{0.973184in}{0.599925in}}%
\pgfpathlineto{\pgfqpoint{0.990291in}{0.596533in}}%
\pgfpathlineto{\pgfqpoint{1.007610in}{0.593433in}}%
\pgfpathlineto{\pgfqpoint{1.025120in}{0.590631in}}%
\pgfpathlineto{\pgfqpoint{1.042804in}{0.588129in}}%
\pgfpathclose%
\pgfusepath{fill}%
\end{pgfscope}%
\begin{pgfscope}%
\pgfpathrectangle{\pgfqpoint{0.041670in}{0.041670in}}{\pgfqpoint{2.216660in}{2.216660in}}%
\pgfusepath{clip}%
\pgfsetbuttcap%
\pgfsetroundjoin%
\definecolor{currentfill}{rgb}{0.344074,0.780029,0.397381}%
\pgfsetfillcolor{currentfill}%
\pgfsetlinewidth{0.000000pt}%
\definecolor{currentstroke}{rgb}{0.000000,0.000000,0.000000}%
\pgfsetstrokecolor{currentstroke}%
\pgfsetdash{}{0pt}%
\pgfpathmoveto{\pgfqpoint{1.332994in}{1.384118in}}%
\pgfpathlineto{\pgfqpoint{1.335710in}{1.376270in}}%
\pgfpathlineto{\pgfqpoint{1.338425in}{1.368346in}}%
\pgfpathlineto{\pgfqpoint{1.341139in}{1.360347in}}%
\pgfpathlineto{\pgfqpoint{1.343852in}{1.352274in}}%
\pgfpathlineto{\pgfqpoint{1.338512in}{1.349792in}}%
\pgfpathlineto{\pgfqpoint{1.333008in}{1.347392in}}%
\pgfpathlineto{\pgfqpoint{1.327346in}{1.345077in}}%
\pgfpathlineto{\pgfqpoint{1.321531in}{1.342850in}}%
\pgfpathlineto{\pgfqpoint{1.319184in}{1.351090in}}%
\pgfpathlineto{\pgfqpoint{1.316837in}{1.359256in}}%
\pgfpathlineto{\pgfqpoint{1.314488in}{1.367347in}}%
\pgfpathlineto{\pgfqpoint{1.312139in}{1.375360in}}%
\pgfpathlineto{\pgfqpoint{1.317572in}{1.377429in}}%
\pgfpathlineto{\pgfqpoint{1.322862in}{1.379580in}}%
\pgfpathlineto{\pgfqpoint{1.328004in}{1.381811in}}%
\pgfpathlineto{\pgfqpoint{1.332994in}{1.384118in}}%
\pgfpathclose%
\pgfusepath{fill}%
\end{pgfscope}%
\begin{pgfscope}%
\pgfpathrectangle{\pgfqpoint{0.041670in}{0.041670in}}{\pgfqpoint{2.216660in}{2.216660in}}%
\pgfusepath{clip}%
\pgfsetbuttcap%
\pgfsetroundjoin%
\definecolor{currentfill}{rgb}{0.282327,0.094955,0.417331}%
\pgfsetfillcolor{currentfill}%
\pgfsetlinewidth{0.000000pt}%
\definecolor{currentstroke}{rgb}{0.000000,0.000000,0.000000}%
\pgfsetstrokecolor{currentstroke}%
\pgfsetdash{}{0pt}%
\pgfpathmoveto{\pgfqpoint{0.789215in}{0.627574in}}%
\pgfpathlineto{\pgfqpoint{0.786753in}{0.630937in}}%
\pgfpathlineto{\pgfqpoint{0.784282in}{0.634651in}}%
\pgfpathlineto{\pgfqpoint{0.781802in}{0.638721in}}%
\pgfpathlineto{\pgfqpoint{0.779312in}{0.643154in}}%
\pgfpathlineto{\pgfqpoint{0.762090in}{0.650114in}}%
\pgfpathlineto{\pgfqpoint{0.745333in}{0.657355in}}%
\pgfpathlineto{\pgfqpoint{0.729059in}{0.664870in}}%
\pgfpathlineto{\pgfqpoint{0.713285in}{0.672649in}}%
\pgfpathlineto{\pgfqpoint{0.716174in}{0.668051in}}%
\pgfpathlineto{\pgfqpoint{0.719052in}{0.663814in}}%
\pgfpathlineto{\pgfqpoint{0.721919in}{0.659934in}}%
\pgfpathlineto{\pgfqpoint{0.724775in}{0.656403in}}%
\pgfpathlineto{\pgfqpoint{0.740173in}{0.648799in}}%
\pgfpathlineto{\pgfqpoint{0.756056in}{0.641454in}}%
\pgfpathlineto{\pgfqpoint{0.772409in}{0.634376in}}%
\pgfpathlineto{\pgfqpoint{0.789215in}{0.627574in}}%
\pgfpathclose%
\pgfusepath{fill}%
\end{pgfscope}%
\begin{pgfscope}%
\pgfpathrectangle{\pgfqpoint{0.041670in}{0.041670in}}{\pgfqpoint{2.216660in}{2.216660in}}%
\pgfusepath{clip}%
\pgfsetbuttcap%
\pgfsetroundjoin%
\definecolor{currentfill}{rgb}{0.279566,0.067836,0.391917}%
\pgfsetfillcolor{currentfill}%
\pgfsetlinewidth{0.000000pt}%
\definecolor{currentstroke}{rgb}{0.000000,0.000000,0.000000}%
\pgfsetstrokecolor{currentstroke}%
\pgfsetdash{}{0pt}%
\pgfpathmoveto{\pgfqpoint{1.187436in}{0.640398in}}%
\pgfpathlineto{\pgfqpoint{1.187491in}{0.634294in}}%
\pgfpathlineto{\pgfqpoint{1.187547in}{0.628359in}}%
\pgfpathlineto{\pgfqpoint{1.187603in}{0.622596in}}%
\pgfpathlineto{\pgfqpoint{1.187659in}{0.617010in}}%
\pgfpathlineto{\pgfqpoint{1.170325in}{0.617027in}}%
\pgfpathlineto{\pgfqpoint{1.153003in}{0.617339in}}%
\pgfpathlineto{\pgfqpoint{1.135710in}{0.617947in}}%
\pgfpathlineto{\pgfqpoint{1.118467in}{0.618849in}}%
\pgfpathlineto{\pgfqpoint{1.118913in}{0.624420in}}%
\pgfpathlineto{\pgfqpoint{1.119358in}{0.630168in}}%
\pgfpathlineto{\pgfqpoint{1.119803in}{0.636089in}}%
\pgfpathlineto{\pgfqpoint{1.120247in}{0.642178in}}%
\pgfpathlineto{\pgfqpoint{1.136991in}{0.641305in}}%
\pgfpathlineto{\pgfqpoint{1.153783in}{0.640716in}}%
\pgfpathlineto{\pgfqpoint{1.170604in}{0.640414in}}%
\pgfpathlineto{\pgfqpoint{1.187436in}{0.640398in}}%
\pgfpathclose%
\pgfusepath{fill}%
\end{pgfscope}%
\begin{pgfscope}%
\pgfpathrectangle{\pgfqpoint{0.041670in}{0.041670in}}{\pgfqpoint{2.216660in}{2.216660in}}%
\pgfusepath{clip}%
\pgfsetbuttcap%
\pgfsetroundjoin%
\definecolor{currentfill}{rgb}{0.267004,0.004874,0.329415}%
\pgfsetfillcolor{currentfill}%
\pgfsetlinewidth{0.000000pt}%
\definecolor{currentstroke}{rgb}{0.000000,0.000000,0.000000}%
\pgfsetstrokecolor{currentstroke}%
\pgfsetdash{}{0pt}%
\pgfpathmoveto{\pgfqpoint{0.961619in}{0.577304in}}%
\pgfpathlineto{\pgfqpoint{0.960158in}{0.575577in}}%
\pgfpathlineto{\pgfqpoint{0.958694in}{0.574112in}}%
\pgfpathlineto{\pgfqpoint{0.957225in}{0.572913in}}%
\pgfpathlineto{\pgfqpoint{0.955753in}{0.571985in}}%
\pgfpathlineto{\pgfqpoint{0.937464in}{0.575992in}}%
\pgfpathlineto{\pgfqpoint{0.919446in}{0.580308in}}%
\pgfpathlineto{\pgfqpoint{0.901721in}{0.584928in}}%
\pgfpathlineto{\pgfqpoint{0.884306in}{0.589847in}}%
\pgfpathlineto{\pgfqpoint{0.886242in}{0.590655in}}%
\pgfpathlineto{\pgfqpoint{0.888174in}{0.591735in}}%
\pgfpathlineto{\pgfqpoint{0.890101in}{0.593080in}}%
\pgfpathlineto{\pgfqpoint{0.892022in}{0.594687in}}%
\pgfpathlineto{\pgfqpoint{0.908987in}{0.589900in}}%
\pgfpathlineto{\pgfqpoint{0.926254in}{0.585404in}}%
\pgfpathlineto{\pgfqpoint{0.943804in}{0.581203in}}%
\pgfpathlineto{\pgfqpoint{0.961619in}{0.577304in}}%
\pgfpathclose%
\pgfusepath{fill}%
\end{pgfscope}%
\begin{pgfscope}%
\pgfpathrectangle{\pgfqpoint{0.041670in}{0.041670in}}{\pgfqpoint{2.216660in}{2.216660in}}%
\pgfusepath{clip}%
\pgfsetbuttcap%
\pgfsetroundjoin%
\definecolor{currentfill}{rgb}{0.163625,0.471133,0.558148}%
\pgfsetfillcolor{currentfill}%
\pgfsetlinewidth{0.000000pt}%
\definecolor{currentstroke}{rgb}{0.000000,0.000000,0.000000}%
\pgfsetstrokecolor{currentstroke}%
\pgfsetdash{}{0pt}%
\pgfpathmoveto{\pgfqpoint{1.094669in}{1.022248in}}%
\pgfpathlineto{\pgfqpoint{1.093750in}{1.012525in}}%
\pgfpathlineto{\pgfqpoint{1.092832in}{1.002812in}}%
\pgfpathlineto{\pgfqpoint{1.091914in}{0.993115in}}%
\pgfpathlineto{\pgfqpoint{1.090997in}{0.983434in}}%
\pgfpathlineto{\pgfqpoint{1.079515in}{0.984972in}}%
\pgfpathlineto{\pgfqpoint{1.068141in}{0.986694in}}%
\pgfpathlineto{\pgfqpoint{1.056887in}{0.988600in}}%
\pgfpathlineto{\pgfqpoint{1.045766in}{0.990687in}}%
\pgfpathlineto{\pgfqpoint{1.047149in}{1.000282in}}%
\pgfpathlineto{\pgfqpoint{1.048532in}{1.009896in}}%
\pgfpathlineto{\pgfqpoint{1.049916in}{1.019524in}}%
\pgfpathlineto{\pgfqpoint{1.051300in}{1.029164in}}%
\pgfpathlineto{\pgfqpoint{1.061964in}{1.027174in}}%
\pgfpathlineto{\pgfqpoint{1.072755in}{1.025357in}}%
\pgfpathlineto{\pgfqpoint{1.083660in}{1.023714in}}%
\pgfpathlineto{\pgfqpoint{1.094669in}{1.022248in}}%
\pgfpathclose%
\pgfusepath{fill}%
\end{pgfscope}%
\begin{pgfscope}%
\pgfpathrectangle{\pgfqpoint{0.041670in}{0.041670in}}{\pgfqpoint{2.216660in}{2.216660in}}%
\pgfusepath{clip}%
\pgfsetbuttcap%
\pgfsetroundjoin%
\definecolor{currentfill}{rgb}{0.636902,0.856542,0.216620}%
\pgfsetfillcolor{currentfill}%
\pgfsetlinewidth{0.000000pt}%
\definecolor{currentstroke}{rgb}{0.000000,0.000000,0.000000}%
\pgfsetstrokecolor{currentstroke}%
\pgfsetdash{}{0pt}%
\pgfpathmoveto{\pgfqpoint{1.302554in}{1.504662in}}%
\pgfpathlineto{\pgfqpoint{1.305612in}{1.498477in}}%
\pgfpathlineto{\pgfqpoint{1.308668in}{1.492188in}}%
\pgfpathlineto{\pgfqpoint{1.311723in}{1.485796in}}%
\pgfpathlineto{\pgfqpoint{1.314777in}{1.479302in}}%
\pgfpathlineto{\pgfqpoint{1.311359in}{1.477291in}}%
\pgfpathlineto{\pgfqpoint{1.307807in}{1.475332in}}%
\pgfpathlineto{\pgfqpoint{1.304124in}{1.473427in}}%
\pgfpathlineto{\pgfqpoint{1.300315in}{1.471579in}}%
\pgfpathlineto{\pgfqpoint{1.297585in}{1.478255in}}%
\pgfpathlineto{\pgfqpoint{1.294855in}{1.484829in}}%
\pgfpathlineto{\pgfqpoint{1.292124in}{1.491300in}}%
\pgfpathlineto{\pgfqpoint{1.289392in}{1.497667in}}%
\pgfpathlineto{\pgfqpoint{1.292858in}{1.499341in}}%
\pgfpathlineto{\pgfqpoint{1.296210in}{1.501066in}}%
\pgfpathlineto{\pgfqpoint{1.299443in}{1.502841in}}%
\pgfpathlineto{\pgfqpoint{1.302554in}{1.504662in}}%
\pgfpathclose%
\pgfusepath{fill}%
\end{pgfscope}%
\begin{pgfscope}%
\pgfpathrectangle{\pgfqpoint{0.041670in}{0.041670in}}{\pgfqpoint{2.216660in}{2.216660in}}%
\pgfusepath{clip}%
\pgfsetbuttcap%
\pgfsetroundjoin%
\definecolor{currentfill}{rgb}{0.565498,0.842430,0.262877}%
\pgfsetfillcolor{currentfill}%
\pgfsetlinewidth{0.000000pt}%
\definecolor{currentstroke}{rgb}{0.000000,0.000000,0.000000}%
\pgfsetstrokecolor{currentstroke}%
\pgfsetdash{}{0pt}%
\pgfpathmoveto{\pgfqpoint{1.063085in}{1.469984in}}%
\pgfpathlineto{\pgfqpoint{1.060435in}{1.463170in}}%
\pgfpathlineto{\pgfqpoint{1.057785in}{1.456256in}}%
\pgfpathlineto{\pgfqpoint{1.055137in}{1.449246in}}%
\pgfpathlineto{\pgfqpoint{1.052489in}{1.442140in}}%
\pgfpathlineto{\pgfqpoint{1.048218in}{1.444109in}}%
\pgfpathlineto{\pgfqpoint{1.044081in}{1.446140in}}%
\pgfpathlineto{\pgfqpoint{1.040084in}{1.448233in}}%
\pgfpathlineto{\pgfqpoint{1.036230in}{1.450384in}}%
\pgfpathlineto{\pgfqpoint{1.039212in}{1.457311in}}%
\pgfpathlineto{\pgfqpoint{1.042195in}{1.464142in}}%
\pgfpathlineto{\pgfqpoint{1.045179in}{1.470876in}}%
\pgfpathlineto{\pgfqpoint{1.048164in}{1.477512in}}%
\pgfpathlineto{\pgfqpoint{1.051701in}{1.475547in}}%
\pgfpathlineto{\pgfqpoint{1.055370in}{1.473636in}}%
\pgfpathlineto{\pgfqpoint{1.059166in}{1.471781in}}%
\pgfpathlineto{\pgfqpoint{1.063085in}{1.469984in}}%
\pgfpathclose%
\pgfusepath{fill}%
\end{pgfscope}%
\begin{pgfscope}%
\pgfpathrectangle{\pgfqpoint{0.041670in}{0.041670in}}{\pgfqpoint{2.216660in}{2.216660in}}%
\pgfusepath{clip}%
\pgfsetbuttcap%
\pgfsetroundjoin%
\definecolor{currentfill}{rgb}{0.166383,0.690856,0.496502}%
\pgfsetfillcolor{currentfill}%
\pgfsetlinewidth{0.000000pt}%
\definecolor{currentstroke}{rgb}{0.000000,0.000000,0.000000}%
\pgfsetstrokecolor{currentstroke}%
\pgfsetdash{}{0pt}%
\pgfpathmoveto{\pgfqpoint{1.340268in}{1.274536in}}%
\pgfpathlineto{\pgfqpoint{1.342605in}{1.265731in}}%
\pgfpathlineto{\pgfqpoint{1.344942in}{1.256874in}}%
\pgfpathlineto{\pgfqpoint{1.347278in}{1.247967in}}%
\pgfpathlineto{\pgfqpoint{1.349612in}{1.239014in}}%
\pgfpathlineto{\pgfqpoint{1.342478in}{1.236413in}}%
\pgfpathlineto{\pgfqpoint{1.335174in}{1.233925in}}%
\pgfpathlineto{\pgfqpoint{1.327707in}{1.231552in}}%
\pgfpathlineto{\pgfqpoint{1.320085in}{1.229296in}}%
\pgfpathlineto{\pgfqpoint{1.318154in}{1.238395in}}%
\pgfpathlineto{\pgfqpoint{1.316222in}{1.247448in}}%
\pgfpathlineto{\pgfqpoint{1.314289in}{1.256451in}}%
\pgfpathlineto{\pgfqpoint{1.312356in}{1.265402in}}%
\pgfpathlineto{\pgfqpoint{1.319560in}{1.267522in}}%
\pgfpathlineto{\pgfqpoint{1.326618in}{1.269753in}}%
\pgfpathlineto{\pgfqpoint{1.333523in}{1.272092in}}%
\pgfpathlineto{\pgfqpoint{1.340268in}{1.274536in}}%
\pgfpathclose%
\pgfusepath{fill}%
\end{pgfscope}%
\begin{pgfscope}%
\pgfpathrectangle{\pgfqpoint{0.041670in}{0.041670in}}{\pgfqpoint{2.216660in}{2.216660in}}%
\pgfusepath{clip}%
\pgfsetbuttcap%
\pgfsetroundjoin%
\definecolor{currentfill}{rgb}{0.195860,0.395433,0.555276}%
\pgfsetfillcolor{currentfill}%
\pgfsetlinewidth{0.000000pt}%
\definecolor{currentstroke}{rgb}{0.000000,0.000000,0.000000}%
\pgfsetstrokecolor{currentstroke}%
\pgfsetdash{}{0pt}%
\pgfpathmoveto{\pgfqpoint{1.136014in}{0.940484in}}%
\pgfpathlineto{\pgfqpoint{1.135578in}{0.930888in}}%
\pgfpathlineto{\pgfqpoint{1.135143in}{0.921328in}}%
\pgfpathlineto{\pgfqpoint{1.134707in}{0.911807in}}%
\pgfpathlineto{\pgfqpoint{1.134272in}{0.902328in}}%
\pgfpathlineto{\pgfqpoint{1.121509in}{0.903183in}}%
\pgfpathlineto{\pgfqpoint{1.108810in}{0.904247in}}%
\pgfpathlineto{\pgfqpoint{1.096188in}{0.905519in}}%
\pgfpathlineto{\pgfqpoint{1.083656in}{0.906996in}}%
\pgfpathlineto{\pgfqpoint{1.084574in}{0.916424in}}%
\pgfpathlineto{\pgfqpoint{1.085491in}{0.925895in}}%
\pgfpathlineto{\pgfqpoint{1.086409in}{0.935405in}}%
\pgfpathlineto{\pgfqpoint{1.087326in}{0.944951in}}%
\pgfpathlineto{\pgfqpoint{1.099381in}{0.943537in}}%
\pgfpathlineto{\pgfqpoint{1.111522in}{0.942320in}}%
\pgfpathlineto{\pgfqpoint{1.123737in}{0.941302in}}%
\pgfpathlineto{\pgfqpoint{1.136014in}{0.940484in}}%
\pgfpathclose%
\pgfusepath{fill}%
\end{pgfscope}%
\begin{pgfscope}%
\pgfpathrectangle{\pgfqpoint{0.041670in}{0.041670in}}{\pgfqpoint{2.216660in}{2.216660in}}%
\pgfusepath{clip}%
\pgfsetbuttcap%
\pgfsetroundjoin%
\definecolor{currentfill}{rgb}{0.896320,0.893616,0.096335}%
\pgfsetfillcolor{currentfill}%
\pgfsetlinewidth{0.000000pt}%
\definecolor{currentstroke}{rgb}{0.000000,0.000000,0.000000}%
\pgfsetstrokecolor{currentstroke}%
\pgfsetdash{}{0pt}%
\pgfpathmoveto{\pgfqpoint{1.241261in}{1.604455in}}%
\pgfpathlineto{\pgfqpoint{1.244329in}{1.600616in}}%
\pgfpathlineto{\pgfqpoint{1.247398in}{1.596651in}}%
\pgfpathlineto{\pgfqpoint{1.250466in}{1.592559in}}%
\pgfpathlineto{\pgfqpoint{1.253534in}{1.588343in}}%
\pgfpathlineto{\pgfqpoint{1.251659in}{1.587263in}}%
\pgfpathlineto{\pgfqpoint{1.249711in}{1.586212in}}%
\pgfpathlineto{\pgfqpoint{1.247693in}{1.585191in}}%
\pgfpathlineto{\pgfqpoint{1.245606in}{1.584199in}}%
\pgfpathlineto{\pgfqpoint{1.242867in}{1.588591in}}%
\pgfpathlineto{\pgfqpoint{1.240128in}{1.592858in}}%
\pgfpathlineto{\pgfqpoint{1.237389in}{1.596998in}}%
\pgfpathlineto{\pgfqpoint{1.234650in}{1.601011in}}%
\pgfpathlineto{\pgfqpoint{1.236390in}{1.601835in}}%
\pgfpathlineto{\pgfqpoint{1.238072in}{1.602684in}}%
\pgfpathlineto{\pgfqpoint{1.239697in}{1.603558in}}%
\pgfpathlineto{\pgfqpoint{1.241261in}{1.604455in}}%
\pgfpathclose%
\pgfusepath{fill}%
\end{pgfscope}%
\begin{pgfscope}%
\pgfpathrectangle{\pgfqpoint{0.041670in}{0.041670in}}{\pgfqpoint{2.216660in}{2.216660in}}%
\pgfusepath{clip}%
\pgfsetbuttcap%
\pgfsetroundjoin%
\definecolor{currentfill}{rgb}{0.212395,0.359683,0.551710}%
\pgfsetfillcolor{currentfill}%
\pgfsetlinewidth{0.000000pt}%
\definecolor{currentstroke}{rgb}{0.000000,0.000000,0.000000}%
\pgfsetstrokecolor{currentstroke}%
\pgfsetdash{}{0pt}%
\pgfpathmoveto{\pgfqpoint{1.236986in}{0.903078in}}%
\pgfpathlineto{\pgfqpoint{1.237529in}{0.893654in}}%
\pgfpathlineto{\pgfqpoint{1.238073in}{0.884278in}}%
\pgfpathlineto{\pgfqpoint{1.238617in}{0.874955in}}%
\pgfpathlineto{\pgfqpoint{1.239161in}{0.865688in}}%
\pgfpathlineto{\pgfqpoint{1.225905in}{0.864820in}}%
\pgfpathlineto{\pgfqpoint{1.212600in}{0.864172in}}%
\pgfpathlineto{\pgfqpoint{1.199259in}{0.863743in}}%
\pgfpathlineto{\pgfqpoint{1.185897in}{0.863534in}}%
\pgfpathlineto{\pgfqpoint{1.185842in}{0.872824in}}%
\pgfpathlineto{\pgfqpoint{1.185787in}{0.882169in}}%
\pgfpathlineto{\pgfqpoint{1.185733in}{0.891567in}}%
\pgfpathlineto{\pgfqpoint{1.185678in}{0.901014in}}%
\pgfpathlineto{\pgfqpoint{1.198549in}{0.901214in}}%
\pgfpathlineto{\pgfqpoint{1.211400in}{0.901625in}}%
\pgfpathlineto{\pgfqpoint{1.224217in}{0.902246in}}%
\pgfpathlineto{\pgfqpoint{1.236986in}{0.903078in}}%
\pgfpathclose%
\pgfusepath{fill}%
\end{pgfscope}%
\begin{pgfscope}%
\pgfpathrectangle{\pgfqpoint{0.041670in}{0.041670in}}{\pgfqpoint{2.216660in}{2.216660in}}%
\pgfusepath{clip}%
\pgfsetbuttcap%
\pgfsetroundjoin%
\definecolor{currentfill}{rgb}{0.955300,0.901065,0.118128}%
\pgfsetfillcolor{currentfill}%
\pgfsetlinewidth{0.000000pt}%
\definecolor{currentstroke}{rgb}{0.000000,0.000000,0.000000}%
\pgfsetstrokecolor{currentstroke}%
\pgfsetdash{}{0pt}%
\pgfpathmoveto{\pgfqpoint{1.208247in}{1.626661in}}%
\pgfpathlineto{\pgfqpoint{1.210610in}{1.623535in}}%
\pgfpathlineto{\pgfqpoint{1.212972in}{1.620277in}}%
\pgfpathlineto{\pgfqpoint{1.215336in}{1.616887in}}%
\pgfpathlineto{\pgfqpoint{1.217699in}{1.613366in}}%
\pgfpathlineto{\pgfqpoint{1.216101in}{1.612820in}}%
\pgfpathlineto{\pgfqpoint{1.214467in}{1.612297in}}%
\pgfpathlineto{\pgfqpoint{1.212798in}{1.611798in}}%
\pgfpathlineto{\pgfqpoint{1.211096in}{1.611325in}}%
\pgfpathlineto{\pgfqpoint{1.209145in}{1.614974in}}%
\pgfpathlineto{\pgfqpoint{1.207195in}{1.618493in}}%
\pgfpathlineto{\pgfqpoint{1.205245in}{1.621880in}}%
\pgfpathlineto{\pgfqpoint{1.203296in}{1.625134in}}%
\pgfpathlineto{\pgfqpoint{1.204572in}{1.625488in}}%
\pgfpathlineto{\pgfqpoint{1.205824in}{1.625861in}}%
\pgfpathlineto{\pgfqpoint{1.207049in}{1.626252in}}%
\pgfpathlineto{\pgfqpoint{1.208247in}{1.626661in}}%
\pgfpathclose%
\pgfusepath{fill}%
\end{pgfscope}%
\begin{pgfscope}%
\pgfpathrectangle{\pgfqpoint{0.041670in}{0.041670in}}{\pgfqpoint{2.216660in}{2.216660in}}%
\pgfusepath{clip}%
\pgfsetbuttcap%
\pgfsetroundjoin%
\definecolor{currentfill}{rgb}{0.212395,0.359683,0.551710}%
\pgfsetfillcolor{currentfill}%
\pgfsetlinewidth{0.000000pt}%
\definecolor{currentstroke}{rgb}{0.000000,0.000000,0.000000}%
\pgfsetstrokecolor{currentstroke}%
\pgfsetdash{}{0pt}%
\pgfpathmoveto{\pgfqpoint{1.185678in}{0.901014in}}%
\pgfpathlineto{\pgfqpoint{1.185733in}{0.891567in}}%
\pgfpathlineto{\pgfqpoint{1.185787in}{0.882169in}}%
\pgfpathlineto{\pgfqpoint{1.185842in}{0.872824in}}%
\pgfpathlineto{\pgfqpoint{1.185897in}{0.863534in}}%
\pgfpathlineto{\pgfqpoint{1.172528in}{0.863547in}}%
\pgfpathlineto{\pgfqpoint{1.159168in}{0.863779in}}%
\pgfpathlineto{\pgfqpoint{1.145830in}{0.864233in}}%
\pgfpathlineto{\pgfqpoint{1.132529in}{0.864906in}}%
\pgfpathlineto{\pgfqpoint{1.132965in}{0.874181in}}%
\pgfpathlineto{\pgfqpoint{1.133400in}{0.883512in}}%
\pgfpathlineto{\pgfqpoint{1.133836in}{0.892896in}}%
\pgfpathlineto{\pgfqpoint{1.134272in}{0.902328in}}%
\pgfpathlineto{\pgfqpoint{1.147084in}{0.901683in}}%
\pgfpathlineto{\pgfqpoint{1.159931in}{0.901249in}}%
\pgfpathlineto{\pgfqpoint{1.172801in}{0.901026in}}%
\pgfpathlineto{\pgfqpoint{1.185678in}{0.901014in}}%
\pgfpathclose%
\pgfusepath{fill}%
\end{pgfscope}%
\begin{pgfscope}%
\pgfpathrectangle{\pgfqpoint{0.041670in}{0.041670in}}{\pgfqpoint{2.216660in}{2.216660in}}%
\pgfusepath{clip}%
\pgfsetbuttcap%
\pgfsetroundjoin%
\definecolor{currentfill}{rgb}{0.276194,0.190074,0.493001}%
\pgfsetfillcolor{currentfill}%
\pgfsetlinewidth{0.000000pt}%
\definecolor{currentstroke}{rgb}{0.000000,0.000000,0.000000}%
\pgfsetstrokecolor{currentstroke}%
\pgfsetdash{}{0pt}%
\pgfpathmoveto{\pgfqpoint{0.701612in}{0.694788in}}%
\pgfpathlineto{\pgfqpoint{0.698663in}{0.701293in}}%
\pgfpathlineto{\pgfqpoint{0.695701in}{0.708199in}}%
\pgfpathlineto{\pgfqpoint{0.692726in}{0.715514in}}%
\pgfpathlineto{\pgfqpoint{0.689737in}{0.723244in}}%
\pgfpathlineto{\pgfqpoint{0.673735in}{0.731624in}}%
\pgfpathlineto{\pgfqpoint{0.658293in}{0.740260in}}%
\pgfpathlineto{\pgfqpoint{0.643426in}{0.749139in}}%
\pgfpathlineto{\pgfqpoint{0.629149in}{0.758251in}}%
\pgfpathlineto{\pgfqpoint{0.632491in}{0.750345in}}%
\pgfpathlineto{\pgfqpoint{0.635819in}{0.742853in}}%
\pgfpathlineto{\pgfqpoint{0.639131in}{0.735768in}}%
\pgfpathlineto{\pgfqpoint{0.642430in}{0.729083in}}%
\pgfpathlineto{\pgfqpoint{0.656379in}{0.720155in}}%
\pgfpathlineto{\pgfqpoint{0.670902in}{0.711457in}}%
\pgfpathlineto{\pgfqpoint{0.685984in}{0.702998in}}%
\pgfpathlineto{\pgfqpoint{0.701612in}{0.694788in}}%
\pgfpathclose%
\pgfusepath{fill}%
\end{pgfscope}%
\begin{pgfscope}%
\pgfpathrectangle{\pgfqpoint{0.041670in}{0.041670in}}{\pgfqpoint{2.216660in}{2.216660in}}%
\pgfusepath{clip}%
\pgfsetbuttcap%
\pgfsetroundjoin%
\definecolor{currentfill}{rgb}{0.955300,0.901065,0.118128}%
\pgfsetfillcolor{currentfill}%
\pgfsetlinewidth{0.000000pt}%
\definecolor{currentstroke}{rgb}{0.000000,0.000000,0.000000}%
\pgfsetstrokecolor{currentstroke}%
\pgfsetdash{}{0pt}%
\pgfpathmoveto{\pgfqpoint{1.157768in}{1.624835in}}%
\pgfpathlineto{\pgfqpoint{1.155915in}{1.621556in}}%
\pgfpathlineto{\pgfqpoint{1.154062in}{1.618144in}}%
\pgfpathlineto{\pgfqpoint{1.152208in}{1.614600in}}%
\pgfpathlineto{\pgfqpoint{1.150353in}{1.610925in}}%
\pgfpathlineto{\pgfqpoint{1.148623in}{1.611376in}}%
\pgfpathlineto{\pgfqpoint{1.146925in}{1.611852in}}%
\pgfpathlineto{\pgfqpoint{1.145259in}{1.612354in}}%
\pgfpathlineto{\pgfqpoint{1.143629in}{1.612879in}}%
\pgfpathlineto{\pgfqpoint{1.145904in}{1.616431in}}%
\pgfpathlineto{\pgfqpoint{1.148179in}{1.619851in}}%
\pgfpathlineto{\pgfqpoint{1.150453in}{1.623140in}}%
\pgfpathlineto{\pgfqpoint{1.152726in}{1.626297in}}%
\pgfpathlineto{\pgfqpoint{1.153949in}{1.625904in}}%
\pgfpathlineto{\pgfqpoint{1.155197in}{1.625529in}}%
\pgfpathlineto{\pgfqpoint{1.156471in}{1.625173in}}%
\pgfpathlineto{\pgfqpoint{1.157768in}{1.624835in}}%
\pgfpathclose%
\pgfusepath{fill}%
\end{pgfscope}%
\begin{pgfscope}%
\pgfpathrectangle{\pgfqpoint{0.041670in}{0.041670in}}{\pgfqpoint{2.216660in}{2.216660in}}%
\pgfusepath{clip}%
\pgfsetbuttcap%
\pgfsetroundjoin%
\definecolor{currentfill}{rgb}{0.896320,0.893616,0.096335}%
\pgfsetfillcolor{currentfill}%
\pgfsetlinewidth{0.000000pt}%
\definecolor{currentstroke}{rgb}{0.000000,0.000000,0.000000}%
\pgfsetstrokecolor{currentstroke}%
\pgfsetdash{}{0pt}%
\pgfpathmoveto{\pgfqpoint{1.126853in}{1.600301in}}%
\pgfpathlineto{\pgfqpoint{1.124193in}{1.596252in}}%
\pgfpathlineto{\pgfqpoint{1.121534in}{1.592075in}}%
\pgfpathlineto{\pgfqpoint{1.118874in}{1.587772in}}%
\pgfpathlineto{\pgfqpoint{1.116214in}{1.583344in}}%
\pgfpathlineto{\pgfqpoint{1.114068in}{1.584308in}}%
\pgfpathlineto{\pgfqpoint{1.111989in}{1.585303in}}%
\pgfpathlineto{\pgfqpoint{1.109979in}{1.586328in}}%
\pgfpathlineto{\pgfqpoint{1.108039in}{1.587382in}}%
\pgfpathlineto{\pgfqpoint{1.111038in}{1.591639in}}%
\pgfpathlineto{\pgfqpoint{1.114037in}{1.595771in}}%
\pgfpathlineto{\pgfqpoint{1.117037in}{1.599777in}}%
\pgfpathlineto{\pgfqpoint{1.120036in}{1.603656in}}%
\pgfpathlineto{\pgfqpoint{1.121654in}{1.602780in}}%
\pgfpathlineto{\pgfqpoint{1.123330in}{1.601928in}}%
\pgfpathlineto{\pgfqpoint{1.125064in}{1.601102in}}%
\pgfpathlineto{\pgfqpoint{1.126853in}{1.600301in}}%
\pgfpathclose%
\pgfusepath{fill}%
\end{pgfscope}%
\begin{pgfscope}%
\pgfpathrectangle{\pgfqpoint{0.041670in}{0.041670in}}{\pgfqpoint{2.216660in}{2.216660in}}%
\pgfusepath{clip}%
\pgfsetbuttcap%
\pgfsetroundjoin%
\definecolor{currentfill}{rgb}{0.699415,0.867117,0.175971}%
\pgfsetfillcolor{currentfill}%
\pgfsetlinewidth{0.000000pt}%
\definecolor{currentstroke}{rgb}{0.000000,0.000000,0.000000}%
\pgfsetstrokecolor{currentstroke}%
\pgfsetdash{}{0pt}%
\pgfpathmoveto{\pgfqpoint{1.290316in}{1.528329in}}%
\pgfpathlineto{\pgfqpoint{1.293377in}{1.522576in}}%
\pgfpathlineto{\pgfqpoint{1.296437in}{1.516713in}}%
\pgfpathlineto{\pgfqpoint{1.299496in}{1.510741in}}%
\pgfpathlineto{\pgfqpoint{1.302554in}{1.504662in}}%
\pgfpathlineto{\pgfqpoint{1.299443in}{1.502841in}}%
\pgfpathlineto{\pgfqpoint{1.296210in}{1.501066in}}%
\pgfpathlineto{\pgfqpoint{1.292858in}{1.499341in}}%
\pgfpathlineto{\pgfqpoint{1.289392in}{1.497667in}}%
\pgfpathlineto{\pgfqpoint{1.286659in}{1.503927in}}%
\pgfpathlineto{\pgfqpoint{1.283925in}{1.510079in}}%
\pgfpathlineto{\pgfqpoint{1.281191in}{1.516122in}}%
\pgfpathlineto{\pgfqpoint{1.278457in}{1.522055in}}%
\pgfpathlineto{\pgfqpoint{1.281580in}{1.523557in}}%
\pgfpathlineto{\pgfqpoint{1.284599in}{1.525104in}}%
\pgfpathlineto{\pgfqpoint{1.287512in}{1.526695in}}%
\pgfpathlineto{\pgfqpoint{1.290316in}{1.528329in}}%
\pgfpathclose%
\pgfusepath{fill}%
\end{pgfscope}%
\begin{pgfscope}%
\pgfpathrectangle{\pgfqpoint{0.041670in}{0.041670in}}{\pgfqpoint{2.216660in}{2.216660in}}%
\pgfusepath{clip}%
\pgfsetbuttcap%
\pgfsetroundjoin%
\definecolor{currentfill}{rgb}{0.147607,0.511733,0.557049}%
\pgfsetfillcolor{currentfill}%
\pgfsetlinewidth{0.000000pt}%
\definecolor{currentstroke}{rgb}{0.000000,0.000000,0.000000}%
\pgfsetstrokecolor{currentstroke}%
\pgfsetdash{}{0pt}%
\pgfpathmoveto{\pgfqpoint{1.312035in}{1.069601in}}%
\pgfpathlineto{\pgfqpoint{1.313520in}{1.059966in}}%
\pgfpathlineto{\pgfqpoint{1.315005in}{1.050332in}}%
\pgfpathlineto{\pgfqpoint{1.316490in}{1.040701in}}%
\pgfpathlineto{\pgfqpoint{1.317974in}{1.031076in}}%
\pgfpathlineto{\pgfqpoint{1.307432in}{1.028934in}}%
\pgfpathlineto{\pgfqpoint{1.296753in}{1.026964in}}%
\pgfpathlineto{\pgfqpoint{1.285949in}{1.025166in}}%
\pgfpathlineto{\pgfqpoint{1.275031in}{1.023543in}}%
\pgfpathlineto{\pgfqpoint{1.274008in}{1.033259in}}%
\pgfpathlineto{\pgfqpoint{1.272984in}{1.042982in}}%
\pgfpathlineto{\pgfqpoint{1.271960in}{1.052707in}}%
\pgfpathlineto{\pgfqpoint{1.270936in}{1.062432in}}%
\pgfpathlineto{\pgfqpoint{1.281384in}{1.063976in}}%
\pgfpathlineto{\pgfqpoint{1.291724in}{1.065687in}}%
\pgfpathlineto{\pgfqpoint{1.301944in}{1.067563in}}%
\pgfpathlineto{\pgfqpoint{1.312035in}{1.069601in}}%
\pgfpathclose%
\pgfusepath{fill}%
\end{pgfscope}%
\begin{pgfscope}%
\pgfpathrectangle{\pgfqpoint{0.041670in}{0.041670in}}{\pgfqpoint{2.216660in}{2.216660in}}%
\pgfusepath{clip}%
\pgfsetbuttcap%
\pgfsetroundjoin%
\definecolor{currentfill}{rgb}{0.855810,0.888601,0.097452}%
\pgfsetfillcolor{currentfill}%
\pgfsetlinewidth{0.000000pt}%
\definecolor{currentstroke}{rgb}{0.000000,0.000000,0.000000}%
\pgfsetstrokecolor{currentstroke}%
\pgfsetdash{}{0pt}%
\pgfpathmoveto{\pgfqpoint{1.253534in}{1.588343in}}%
\pgfpathlineto{\pgfqpoint{1.256602in}{1.584002in}}%
\pgfpathlineto{\pgfqpoint{1.259669in}{1.579538in}}%
\pgfpathlineto{\pgfqpoint{1.262736in}{1.574951in}}%
\pgfpathlineto{\pgfqpoint{1.265803in}{1.570243in}}%
\pgfpathlineto{\pgfqpoint{1.263617in}{1.568980in}}%
\pgfpathlineto{\pgfqpoint{1.261347in}{1.567751in}}%
\pgfpathlineto{\pgfqpoint{1.258994in}{1.566555in}}%
\pgfpathlineto{\pgfqpoint{1.256561in}{1.565395in}}%
\pgfpathlineto{\pgfqpoint{1.253823in}{1.570280in}}%
\pgfpathlineto{\pgfqpoint{1.251084in}{1.575043in}}%
\pgfpathlineto{\pgfqpoint{1.248345in}{1.579683in}}%
\pgfpathlineto{\pgfqpoint{1.245606in}{1.584199in}}%
\pgfpathlineto{\pgfqpoint{1.247693in}{1.585191in}}%
\pgfpathlineto{\pgfqpoint{1.249711in}{1.586212in}}%
\pgfpathlineto{\pgfqpoint{1.251659in}{1.587263in}}%
\pgfpathlineto{\pgfqpoint{1.253534in}{1.588343in}}%
\pgfpathclose%
\pgfusepath{fill}%
\end{pgfscope}%
\begin{pgfscope}%
\pgfpathrectangle{\pgfqpoint{0.041670in}{0.041670in}}{\pgfqpoint{2.216660in}{2.216660in}}%
\pgfusepath{clip}%
\pgfsetbuttcap%
\pgfsetroundjoin%
\definecolor{currentfill}{rgb}{0.122606,0.585371,0.546557}%
\pgfsetfillcolor{currentfill}%
\pgfsetlinewidth{0.000000pt}%
\definecolor{currentstroke}{rgb}{0.000000,0.000000,0.000000}%
\pgfsetstrokecolor{currentstroke}%
\pgfsetdash{}{0pt}%
\pgfpathmoveto{\pgfqpoint{1.067931in}{1.144700in}}%
\pgfpathlineto{\pgfqpoint{1.066543in}{1.135144in}}%
\pgfpathlineto{\pgfqpoint{1.065155in}{1.125566in}}%
\pgfpathlineto{\pgfqpoint{1.063768in}{1.115968in}}%
\pgfpathlineto{\pgfqpoint{1.062381in}{1.106353in}}%
\pgfpathlineto{\pgfqpoint{1.052758in}{1.108306in}}%
\pgfpathlineto{\pgfqpoint{1.043270in}{1.110410in}}%
\pgfpathlineto{\pgfqpoint{1.033927in}{1.112664in}}%
\pgfpathlineto{\pgfqpoint{1.024738in}{1.115065in}}%
\pgfpathlineto{\pgfqpoint{1.026567in}{1.124565in}}%
\pgfpathlineto{\pgfqpoint{1.028396in}{1.134048in}}%
\pgfpathlineto{\pgfqpoint{1.030226in}{1.143513in}}%
\pgfpathlineto{\pgfqpoint{1.032056in}{1.152955in}}%
\pgfpathlineto{\pgfqpoint{1.040814in}{1.150680in}}%
\pgfpathlineto{\pgfqpoint{1.049718in}{1.148544in}}%
\pgfpathlineto{\pgfqpoint{1.058761in}{1.146550in}}%
\pgfpathlineto{\pgfqpoint{1.067931in}{1.144700in}}%
\pgfpathclose%
\pgfusepath{fill}%
\end{pgfscope}%
\begin{pgfscope}%
\pgfpathrectangle{\pgfqpoint{0.041670in}{0.041670in}}{\pgfqpoint{2.216660in}{2.216660in}}%
\pgfusepath{clip}%
\pgfsetbuttcap%
\pgfsetroundjoin%
\definecolor{currentfill}{rgb}{0.935904,0.898570,0.108131}%
\pgfsetfillcolor{currentfill}%
\pgfsetlinewidth{0.000000pt}%
\definecolor{currentstroke}{rgb}{0.000000,0.000000,0.000000}%
\pgfsetstrokecolor{currentstroke}%
\pgfsetdash{}{0pt}%
\pgfpathmoveto{\pgfqpoint{1.223695in}{1.615781in}}%
\pgfpathlineto{\pgfqpoint{1.226433in}{1.612282in}}%
\pgfpathlineto{\pgfqpoint{1.229172in}{1.608654in}}%
\pgfpathlineto{\pgfqpoint{1.231911in}{1.604897in}}%
\pgfpathlineto{\pgfqpoint{1.234650in}{1.601011in}}%
\pgfpathlineto{\pgfqpoint{1.232855in}{1.600214in}}%
\pgfpathlineto{\pgfqpoint{1.231006in}{1.599443in}}%
\pgfpathlineto{\pgfqpoint{1.229106in}{1.598700in}}%
\pgfpathlineto{\pgfqpoint{1.227157in}{1.597985in}}%
\pgfpathlineto{\pgfqpoint{1.224792in}{1.602024in}}%
\pgfpathlineto{\pgfqpoint{1.222427in}{1.605934in}}%
\pgfpathlineto{\pgfqpoint{1.220063in}{1.609715in}}%
\pgfpathlineto{\pgfqpoint{1.217699in}{1.613366in}}%
\pgfpathlineto{\pgfqpoint{1.219259in}{1.613937in}}%
\pgfpathlineto{\pgfqpoint{1.220779in}{1.614529in}}%
\pgfpathlineto{\pgfqpoint{1.222258in}{1.615144in}}%
\pgfpathlineto{\pgfqpoint{1.223695in}{1.615781in}}%
\pgfpathclose%
\pgfusepath{fill}%
\end{pgfscope}%
\begin{pgfscope}%
\pgfpathrectangle{\pgfqpoint{0.041670in}{0.041670in}}{\pgfqpoint{2.216660in}{2.216660in}}%
\pgfusepath{clip}%
\pgfsetbuttcap%
\pgfsetroundjoin%
\definecolor{currentfill}{rgb}{0.636902,0.856542,0.216620}%
\pgfsetfillcolor{currentfill}%
\pgfsetlinewidth{0.000000pt}%
\definecolor{currentstroke}{rgb}{0.000000,0.000000,0.000000}%
\pgfsetstrokecolor{currentstroke}%
\pgfsetdash{}{0pt}%
\pgfpathmoveto{\pgfqpoint{1.073694in}{1.496223in}}%
\pgfpathlineto{\pgfqpoint{1.071040in}{1.489819in}}%
\pgfpathlineto{\pgfqpoint{1.068388in}{1.483310in}}%
\pgfpathlineto{\pgfqpoint{1.065736in}{1.476698in}}%
\pgfpathlineto{\pgfqpoint{1.063085in}{1.469984in}}%
\pgfpathlineto{\pgfqpoint{1.059166in}{1.471781in}}%
\pgfpathlineto{\pgfqpoint{1.055370in}{1.473636in}}%
\pgfpathlineto{\pgfqpoint{1.051701in}{1.475547in}}%
\pgfpathlineto{\pgfqpoint{1.048164in}{1.477512in}}%
\pgfpathlineto{\pgfqpoint{1.051150in}{1.484048in}}%
\pgfpathlineto{\pgfqpoint{1.054138in}{1.490482in}}%
\pgfpathlineto{\pgfqpoint{1.057126in}{1.496814in}}%
\pgfpathlineto{\pgfqpoint{1.060115in}{1.503041in}}%
\pgfpathlineto{\pgfqpoint{1.063335in}{1.501261in}}%
\pgfpathlineto{\pgfqpoint{1.066673in}{1.499530in}}%
\pgfpathlineto{\pgfqpoint{1.070127in}{1.497850in}}%
\pgfpathlineto{\pgfqpoint{1.073694in}{1.496223in}}%
\pgfpathclose%
\pgfusepath{fill}%
\end{pgfscope}%
\begin{pgfscope}%
\pgfpathrectangle{\pgfqpoint{0.041670in}{0.041670in}}{\pgfqpoint{2.216660in}{2.216660in}}%
\pgfusepath{clip}%
\pgfsetbuttcap%
\pgfsetroundjoin%
\definecolor{currentfill}{rgb}{0.271305,0.019942,0.347269}%
\pgfsetfillcolor{currentfill}%
\pgfsetlinewidth{0.000000pt}%
\definecolor{currentstroke}{rgb}{0.000000,0.000000,0.000000}%
\pgfsetstrokecolor{currentstroke}%
\pgfsetdash{}{0pt}%
\pgfpathmoveto{\pgfqpoint{1.328603in}{0.607304in}}%
\pgfpathlineto{\pgfqpoint{1.329657in}{0.602752in}}%
\pgfpathlineto{\pgfqpoint{1.330714in}{0.598404in}}%
\pgfpathlineto{\pgfqpoint{1.331772in}{0.594265in}}%
\pgfpathlineto{\pgfqpoint{1.332833in}{0.590338in}}%
\pgfpathlineto{\pgfqpoint{1.315131in}{0.587870in}}%
\pgfpathlineto{\pgfqpoint{1.297278in}{0.585706in}}%
\pgfpathlineto{\pgfqpoint{1.279294in}{0.583848in}}%
\pgfpathlineto{\pgfqpoint{1.261198in}{0.582299in}}%
\pgfpathlineto{\pgfqpoint{1.260634in}{0.586286in}}%
\pgfpathlineto{\pgfqpoint{1.260071in}{0.590486in}}%
\pgfpathlineto{\pgfqpoint{1.259509in}{0.594894in}}%
\pgfpathlineto{\pgfqpoint{1.258948in}{0.599506in}}%
\pgfpathlineto{\pgfqpoint{1.276543in}{0.601008in}}%
\pgfpathlineto{\pgfqpoint{1.294030in}{0.602810in}}%
\pgfpathlineto{\pgfqpoint{1.311390in}{0.604909in}}%
\pgfpathlineto{\pgfqpoint{1.328603in}{0.607304in}}%
\pgfpathclose%
\pgfusepath{fill}%
\end{pgfscope}%
\begin{pgfscope}%
\pgfpathrectangle{\pgfqpoint{0.041670in}{0.041670in}}{\pgfqpoint{2.216660in}{2.216660in}}%
\pgfusepath{clip}%
\pgfsetbuttcap%
\pgfsetroundjoin%
\definecolor{currentfill}{rgb}{0.179019,0.433756,0.557430}%
\pgfsetfillcolor{currentfill}%
\pgfsetlinewidth{0.000000pt}%
\definecolor{currentstroke}{rgb}{0.000000,0.000000,0.000000}%
\pgfsetstrokecolor{currentstroke}%
\pgfsetdash{}{0pt}%
\pgfpathmoveto{\pgfqpoint{1.279124in}{0.984792in}}%
\pgfpathlineto{\pgfqpoint{1.280147in}{0.975148in}}%
\pgfpathlineto{\pgfqpoint{1.281170in}{0.965528in}}%
\pgfpathlineto{\pgfqpoint{1.282193in}{0.955935in}}%
\pgfpathlineto{\pgfqpoint{1.283215in}{0.946372in}}%
\pgfpathlineto{\pgfqpoint{1.271249in}{0.944784in}}%
\pgfpathlineto{\pgfqpoint{1.259184in}{0.943392in}}%
\pgfpathlineto{\pgfqpoint{1.247034in}{0.942197in}}%
\pgfpathlineto{\pgfqpoint{1.234811in}{0.941202in}}%
\pgfpathlineto{\pgfqpoint{1.234267in}{0.950823in}}%
\pgfpathlineto{\pgfqpoint{1.233724in}{0.960474in}}%
\pgfpathlineto{\pgfqpoint{1.233180in}{0.970152in}}%
\pgfpathlineto{\pgfqpoint{1.232636in}{0.979853in}}%
\pgfpathlineto{\pgfqpoint{1.244375in}{0.980804in}}%
\pgfpathlineto{\pgfqpoint{1.256044in}{0.981945in}}%
\pgfpathlineto{\pgfqpoint{1.267631in}{0.983275in}}%
\pgfpathlineto{\pgfqpoint{1.279124in}{0.984792in}}%
\pgfpathclose%
\pgfusepath{fill}%
\end{pgfscope}%
\begin{pgfscope}%
\pgfpathrectangle{\pgfqpoint{0.041670in}{0.041670in}}{\pgfqpoint{2.216660in}{2.216660in}}%
\pgfusepath{clip}%
\pgfsetbuttcap%
\pgfsetroundjoin%
\definecolor{currentfill}{rgb}{0.344074,0.780029,0.397381}%
\pgfsetfillcolor{currentfill}%
\pgfsetlinewidth{0.000000pt}%
\definecolor{currentstroke}{rgb}{0.000000,0.000000,0.000000}%
\pgfsetstrokecolor{currentstroke}%
\pgfsetdash{}{0pt}%
\pgfpathmoveto{\pgfqpoint{1.052714in}{1.373591in}}%
\pgfpathlineto{\pgfqpoint{1.050452in}{1.365544in}}%
\pgfpathlineto{\pgfqpoint{1.048191in}{1.357420in}}%
\pgfpathlineto{\pgfqpoint{1.045931in}{1.349220in}}%
\pgfpathlineto{\pgfqpoint{1.043672in}{1.340946in}}%
\pgfpathlineto{\pgfqpoint{1.037726in}{1.343093in}}%
\pgfpathlineto{\pgfqpoint{1.031927in}{1.345330in}}%
\pgfpathlineto{\pgfqpoint{1.026282in}{1.347654in}}%
\pgfpathlineto{\pgfqpoint{1.020796in}{1.350063in}}%
\pgfpathlineto{\pgfqpoint{1.023431in}{1.358175in}}%
\pgfpathlineto{\pgfqpoint{1.026068in}{1.366214in}}%
\pgfpathlineto{\pgfqpoint{1.028705in}{1.374177in}}%
\pgfpathlineto{\pgfqpoint{1.031343in}{1.382063in}}%
\pgfpathlineto{\pgfqpoint{1.036469in}{1.379824in}}%
\pgfpathlineto{\pgfqpoint{1.041743in}{1.377664in}}%
\pgfpathlineto{\pgfqpoint{1.047160in}{1.375586in}}%
\pgfpathlineto{\pgfqpoint{1.052714in}{1.373591in}}%
\pgfpathclose%
\pgfusepath{fill}%
\end{pgfscope}%
\begin{pgfscope}%
\pgfpathrectangle{\pgfqpoint{0.041670in}{0.041670in}}{\pgfqpoint{2.216660in}{2.216660in}}%
\pgfusepath{clip}%
\pgfsetbuttcap%
\pgfsetroundjoin%
\definecolor{currentfill}{rgb}{0.762373,0.876424,0.137064}%
\pgfsetfillcolor{currentfill}%
\pgfsetlinewidth{0.000000pt}%
\definecolor{currentstroke}{rgb}{0.000000,0.000000,0.000000}%
\pgfsetstrokecolor{currentstroke}%
\pgfsetdash{}{0pt}%
\pgfpathmoveto{\pgfqpoint{1.278065in}{1.550216in}}%
\pgfpathlineto{\pgfqpoint{1.281129in}{1.544915in}}%
\pgfpathlineto{\pgfqpoint{1.284192in}{1.539500in}}%
\pgfpathlineto{\pgfqpoint{1.287254in}{1.533971in}}%
\pgfpathlineto{\pgfqpoint{1.290316in}{1.528329in}}%
\pgfpathlineto{\pgfqpoint{1.287512in}{1.526695in}}%
\pgfpathlineto{\pgfqpoint{1.284599in}{1.525104in}}%
\pgfpathlineto{\pgfqpoint{1.281580in}{1.523557in}}%
\pgfpathlineto{\pgfqpoint{1.278457in}{1.522055in}}%
\pgfpathlineto{\pgfqpoint{1.275721in}{1.527876in}}%
\pgfpathlineto{\pgfqpoint{1.272986in}{1.533585in}}%
\pgfpathlineto{\pgfqpoint{1.270249in}{1.539179in}}%
\pgfpathlineto{\pgfqpoint{1.267513in}{1.544657in}}%
\pgfpathlineto{\pgfqpoint{1.270291in}{1.545987in}}%
\pgfpathlineto{\pgfqpoint{1.272977in}{1.547358in}}%
\pgfpathlineto{\pgfqpoint{1.275570in}{1.548768in}}%
\pgfpathlineto{\pgfqpoint{1.278065in}{1.550216in}}%
\pgfpathclose%
\pgfusepath{fill}%
\end{pgfscope}%
\begin{pgfscope}%
\pgfpathrectangle{\pgfqpoint{0.041670in}{0.041670in}}{\pgfqpoint{2.216660in}{2.216660in}}%
\pgfusepath{clip}%
\pgfsetbuttcap%
\pgfsetroundjoin%
\definecolor{currentfill}{rgb}{0.814576,0.883393,0.110347}%
\pgfsetfillcolor{currentfill}%
\pgfsetlinewidth{0.000000pt}%
\definecolor{currentstroke}{rgb}{0.000000,0.000000,0.000000}%
\pgfsetstrokecolor{currentstroke}%
\pgfsetdash{}{0pt}%
\pgfpathmoveto{\pgfqpoint{1.265803in}{1.570243in}}%
\pgfpathlineto{\pgfqpoint{1.268869in}{1.565414in}}%
\pgfpathlineto{\pgfqpoint{1.271935in}{1.560466in}}%
\pgfpathlineto{\pgfqpoint{1.275000in}{1.555400in}}%
\pgfpathlineto{\pgfqpoint{1.278065in}{1.550216in}}%
\pgfpathlineto{\pgfqpoint{1.275570in}{1.548768in}}%
\pgfpathlineto{\pgfqpoint{1.272977in}{1.547358in}}%
\pgfpathlineto{\pgfqpoint{1.270291in}{1.545987in}}%
\pgfpathlineto{\pgfqpoint{1.267513in}{1.544657in}}%
\pgfpathlineto{\pgfqpoint{1.264775in}{1.550019in}}%
\pgfpathlineto{\pgfqpoint{1.262038in}{1.555264in}}%
\pgfpathlineto{\pgfqpoint{1.259300in}{1.560389in}}%
\pgfpathlineto{\pgfqpoint{1.256561in}{1.565395in}}%
\pgfpathlineto{\pgfqpoint{1.258994in}{1.566555in}}%
\pgfpathlineto{\pgfqpoint{1.261347in}{1.567751in}}%
\pgfpathlineto{\pgfqpoint{1.263617in}{1.568980in}}%
\pgfpathlineto{\pgfqpoint{1.265803in}{1.570243in}}%
\pgfpathclose%
\pgfusepath{fill}%
\end{pgfscope}%
\begin{pgfscope}%
\pgfpathrectangle{\pgfqpoint{0.041670in}{0.041670in}}{\pgfqpoint{2.216660in}{2.216660in}}%
\pgfusepath{clip}%
\pgfsetbuttcap%
\pgfsetroundjoin%
\definecolor{currentfill}{rgb}{0.955300,0.901065,0.118128}%
\pgfsetfillcolor{currentfill}%
\pgfsetlinewidth{0.000000pt}%
\definecolor{currentstroke}{rgb}{0.000000,0.000000,0.000000}%
\pgfsetstrokecolor{currentstroke}%
\pgfsetdash{}{0pt}%
\pgfpathmoveto{\pgfqpoint{1.203296in}{1.625134in}}%
\pgfpathlineto{\pgfqpoint{1.205245in}{1.621880in}}%
\pgfpathlineto{\pgfqpoint{1.207195in}{1.618493in}}%
\pgfpathlineto{\pgfqpoint{1.209145in}{1.614974in}}%
\pgfpathlineto{\pgfqpoint{1.211096in}{1.611325in}}%
\pgfpathlineto{\pgfqpoint{1.209362in}{1.610876in}}%
\pgfpathlineto{\pgfqpoint{1.207599in}{1.610454in}}%
\pgfpathlineto{\pgfqpoint{1.205808in}{1.610058in}}%
\pgfpathlineto{\pgfqpoint{1.203990in}{1.609689in}}%
\pgfpathlineto{\pgfqpoint{1.202485in}{1.613442in}}%
\pgfpathlineto{\pgfqpoint{1.200979in}{1.617063in}}%
\pgfpathlineto{\pgfqpoint{1.199474in}{1.620553in}}%
\pgfpathlineto{\pgfqpoint{1.197969in}{1.623911in}}%
\pgfpathlineto{\pgfqpoint{1.199331in}{1.624187in}}%
\pgfpathlineto{\pgfqpoint{1.200674in}{1.624483in}}%
\pgfpathlineto{\pgfqpoint{1.201996in}{1.624799in}}%
\pgfpathlineto{\pgfqpoint{1.203296in}{1.625134in}}%
\pgfpathclose%
\pgfusepath{fill}%
\end{pgfscope}%
\begin{pgfscope}%
\pgfpathrectangle{\pgfqpoint{0.041670in}{0.041670in}}{\pgfqpoint{2.216660in}{2.216660in}}%
\pgfusepath{clip}%
\pgfsetbuttcap%
\pgfsetroundjoin%
\definecolor{currentfill}{rgb}{0.935904,0.898570,0.108131}%
\pgfsetfillcolor{currentfill}%
\pgfsetlinewidth{0.000000pt}%
\definecolor{currentstroke}{rgb}{0.000000,0.000000,0.000000}%
\pgfsetstrokecolor{currentstroke}%
\pgfsetdash{}{0pt}%
\pgfpathmoveto{\pgfqpoint{1.143629in}{1.612879in}}%
\pgfpathlineto{\pgfqpoint{1.141354in}{1.609197in}}%
\pgfpathlineto{\pgfqpoint{1.139078in}{1.605385in}}%
\pgfpathlineto{\pgfqpoint{1.136802in}{1.601444in}}%
\pgfpathlineto{\pgfqpoint{1.134526in}{1.597374in}}%
\pgfpathlineto{\pgfqpoint{1.132534in}{1.598063in}}%
\pgfpathlineto{\pgfqpoint{1.130590in}{1.598781in}}%
\pgfpathlineto{\pgfqpoint{1.128695in}{1.599527in}}%
\pgfpathlineto{\pgfqpoint{1.126853in}{1.600301in}}%
\pgfpathlineto{\pgfqpoint{1.129512in}{1.604222in}}%
\pgfpathlineto{\pgfqpoint{1.132172in}{1.608016in}}%
\pgfpathlineto{\pgfqpoint{1.134831in}{1.611680in}}%
\pgfpathlineto{\pgfqpoint{1.137490in}{1.615214in}}%
\pgfpathlineto{\pgfqpoint{1.138964in}{1.614597in}}%
\pgfpathlineto{\pgfqpoint{1.140480in}{1.614001in}}%
\pgfpathlineto{\pgfqpoint{1.142035in}{1.613429in}}%
\pgfpathlineto{\pgfqpoint{1.143629in}{1.612879in}}%
\pgfpathclose%
\pgfusepath{fill}%
\end{pgfscope}%
\begin{pgfscope}%
\pgfpathrectangle{\pgfqpoint{0.041670in}{0.041670in}}{\pgfqpoint{2.216660in}{2.216660in}}%
\pgfusepath{clip}%
\pgfsetbuttcap%
\pgfsetroundjoin%
\definecolor{currentfill}{rgb}{0.166383,0.690856,0.496502}%
\pgfsetfillcolor{currentfill}%
\pgfsetlinewidth{0.000000pt}%
\definecolor{currentstroke}{rgb}{0.000000,0.000000,0.000000}%
\pgfsetstrokecolor{currentstroke}%
\pgfsetdash{}{0pt}%
\pgfpathmoveto{\pgfqpoint{1.054074in}{1.263612in}}%
\pgfpathlineto{\pgfqpoint{1.052235in}{1.254632in}}%
\pgfpathlineto{\pgfqpoint{1.050397in}{1.245601in}}%
\pgfpathlineto{\pgfqpoint{1.048560in}{1.236519in}}%
\pgfpathlineto{\pgfqpoint{1.046724in}{1.227391in}}%
\pgfpathlineto{\pgfqpoint{1.038971in}{1.229541in}}%
\pgfpathlineto{\pgfqpoint{1.031366in}{1.231810in}}%
\pgfpathlineto{\pgfqpoint{1.023916in}{1.234196in}}%
\pgfpathlineto{\pgfqpoint{1.016631in}{1.236697in}}%
\pgfpathlineto{\pgfqpoint{1.018879in}{1.245685in}}%
\pgfpathlineto{\pgfqpoint{1.021128in}{1.254626in}}%
\pgfpathlineto{\pgfqpoint{1.023378in}{1.263518in}}%
\pgfpathlineto{\pgfqpoint{1.025629in}{1.272358in}}%
\pgfpathlineto{\pgfqpoint{1.032517in}{1.270008in}}%
\pgfpathlineto{\pgfqpoint{1.039558in}{1.267765in}}%
\pgfpathlineto{\pgfqpoint{1.046746in}{1.265632in}}%
\pgfpathlineto{\pgfqpoint{1.054074in}{1.263612in}}%
\pgfpathclose%
\pgfusepath{fill}%
\end{pgfscope}%
\begin{pgfscope}%
\pgfpathrectangle{\pgfqpoint{0.041670in}{0.041670in}}{\pgfqpoint{2.216660in}{2.216660in}}%
\pgfusepath{clip}%
\pgfsetbuttcap%
\pgfsetroundjoin%
\definecolor{currentfill}{rgb}{0.412913,0.803041,0.357269}%
\pgfsetfillcolor{currentfill}%
\pgfsetlinewidth{0.000000pt}%
\definecolor{currentstroke}{rgb}{0.000000,0.000000,0.000000}%
\pgfsetstrokecolor{currentstroke}%
\pgfsetdash{}{0pt}%
\pgfpathmoveto{\pgfqpoint{1.322118in}{1.414695in}}%
\pgfpathlineto{\pgfqpoint{1.324838in}{1.407176in}}%
\pgfpathlineto{\pgfqpoint{1.327558in}{1.399572in}}%
\pgfpathlineto{\pgfqpoint{1.330277in}{1.391886in}}%
\pgfpathlineto{\pgfqpoint{1.332994in}{1.384118in}}%
\pgfpathlineto{\pgfqpoint{1.328004in}{1.381811in}}%
\pgfpathlineto{\pgfqpoint{1.322862in}{1.379580in}}%
\pgfpathlineto{\pgfqpoint{1.317572in}{1.377429in}}%
\pgfpathlineto{\pgfqpoint{1.312139in}{1.375360in}}%
\pgfpathlineto{\pgfqpoint{1.309789in}{1.383294in}}%
\pgfpathlineto{\pgfqpoint{1.307438in}{1.391146in}}%
\pgfpathlineto{\pgfqpoint{1.305086in}{1.398915in}}%
\pgfpathlineto{\pgfqpoint{1.302734in}{1.406598in}}%
\pgfpathlineto{\pgfqpoint{1.307782in}{1.408511in}}%
\pgfpathlineto{\pgfqpoint{1.312699in}{1.410500in}}%
\pgfpathlineto{\pgfqpoint{1.317479in}{1.412562in}}%
\pgfpathlineto{\pgfqpoint{1.322118in}{1.414695in}}%
\pgfpathclose%
\pgfusepath{fill}%
\end{pgfscope}%
\begin{pgfscope}%
\pgfpathrectangle{\pgfqpoint{0.041670in}{0.041670in}}{\pgfqpoint{2.216660in}{2.216660in}}%
\pgfusepath{clip}%
\pgfsetbuttcap%
\pgfsetroundjoin%
\definecolor{currentfill}{rgb}{0.274952,0.037752,0.364543}%
\pgfsetfillcolor{currentfill}%
\pgfsetlinewidth{0.000000pt}%
\definecolor{currentstroke}{rgb}{0.000000,0.000000,0.000000}%
\pgfsetstrokecolor{currentstroke}%
\pgfsetdash{}{0pt}%
\pgfpathmoveto{\pgfqpoint{1.256713in}{0.619898in}}%
\pgfpathlineto{\pgfqpoint{1.257270in}{0.614517in}}%
\pgfpathlineto{\pgfqpoint{1.257829in}{0.609321in}}%
\pgfpathlineto{\pgfqpoint{1.258388in}{0.604316in}}%
\pgfpathlineto{\pgfqpoint{1.258948in}{0.599506in}}%
\pgfpathlineto{\pgfqpoint{1.241265in}{0.598305in}}%
\pgfpathlineto{\pgfqpoint{1.223513in}{0.597407in}}%
\pgfpathlineto{\pgfqpoint{1.205712in}{0.596814in}}%
\pgfpathlineto{\pgfqpoint{1.187883in}{0.596526in}}%
\pgfpathlineto{\pgfqpoint{1.187827in}{0.601359in}}%
\pgfpathlineto{\pgfqpoint{1.187771in}{0.606388in}}%
\pgfpathlineto{\pgfqpoint{1.187715in}{0.611606in}}%
\pgfpathlineto{\pgfqpoint{1.187659in}{0.617010in}}%
\pgfpathlineto{\pgfqpoint{1.204983in}{0.617290in}}%
\pgfpathlineto{\pgfqpoint{1.222280in}{0.617865in}}%
\pgfpathlineto{\pgfqpoint{1.239530in}{0.618734in}}%
\pgfpathlineto{\pgfqpoint{1.256713in}{0.619898in}}%
\pgfpathclose%
\pgfusepath{fill}%
\end{pgfscope}%
\begin{pgfscope}%
\pgfpathrectangle{\pgfqpoint{0.041670in}{0.041670in}}{\pgfqpoint{2.216660in}{2.216660in}}%
\pgfusepath{clip}%
\pgfsetbuttcap%
\pgfsetroundjoin%
\definecolor{currentfill}{rgb}{0.855810,0.888601,0.097452}%
\pgfsetfillcolor{currentfill}%
\pgfsetlinewidth{0.000000pt}%
\definecolor{currentstroke}{rgb}{0.000000,0.000000,0.000000}%
\pgfsetstrokecolor{currentstroke}%
\pgfsetdash{}{0pt}%
\pgfpathmoveto{\pgfqpoint{1.116214in}{1.583344in}}%
\pgfpathlineto{\pgfqpoint{1.113554in}{1.578792in}}%
\pgfpathlineto{\pgfqpoint{1.110895in}{1.574115in}}%
\pgfpathlineto{\pgfqpoint{1.108235in}{1.569316in}}%
\pgfpathlineto{\pgfqpoint{1.105576in}{1.564394in}}%
\pgfpathlineto{\pgfqpoint{1.103074in}{1.565522in}}%
\pgfpathlineto{\pgfqpoint{1.100650in}{1.566686in}}%
\pgfpathlineto{\pgfqpoint{1.098306in}{1.567886in}}%
\pgfpathlineto{\pgfqpoint{1.096045in}{1.569119in}}%
\pgfpathlineto{\pgfqpoint{1.099043in}{1.573868in}}%
\pgfpathlineto{\pgfqpoint{1.102042in}{1.578496in}}%
\pgfpathlineto{\pgfqpoint{1.105040in}{1.583001in}}%
\pgfpathlineto{\pgfqpoint{1.108039in}{1.587382in}}%
\pgfpathlineto{\pgfqpoint{1.109979in}{1.586328in}}%
\pgfpathlineto{\pgfqpoint{1.111989in}{1.585303in}}%
\pgfpathlineto{\pgfqpoint{1.114068in}{1.584308in}}%
\pgfpathlineto{\pgfqpoint{1.116214in}{1.583344in}}%
\pgfpathclose%
\pgfusepath{fill}%
\end{pgfscope}%
\begin{pgfscope}%
\pgfpathrectangle{\pgfqpoint{0.041670in}{0.041670in}}{\pgfqpoint{2.216660in}{2.216660in}}%
\pgfusepath{clip}%
\pgfsetbuttcap%
\pgfsetroundjoin%
\definecolor{currentfill}{rgb}{0.955300,0.901065,0.118128}%
\pgfsetfillcolor{currentfill}%
\pgfsetlinewidth{0.000000pt}%
\definecolor{currentstroke}{rgb}{0.000000,0.000000,0.000000}%
\pgfsetstrokecolor{currentstroke}%
\pgfsetdash{}{0pt}%
\pgfpathmoveto{\pgfqpoint{1.163167in}{1.623682in}}%
\pgfpathlineto{\pgfqpoint{1.161764in}{1.620306in}}%
\pgfpathlineto{\pgfqpoint{1.160362in}{1.616796in}}%
\pgfpathlineto{\pgfqpoint{1.158959in}{1.613155in}}%
\pgfpathlineto{\pgfqpoint{1.157555in}{1.609383in}}%
\pgfpathlineto{\pgfqpoint{1.155716in}{1.609728in}}%
\pgfpathlineto{\pgfqpoint{1.153902in}{1.610101in}}%
\pgfpathlineto{\pgfqpoint{1.152114in}{1.610500in}}%
\pgfpathlineto{\pgfqpoint{1.150353in}{1.610925in}}%
\pgfpathlineto{\pgfqpoint{1.152208in}{1.614600in}}%
\pgfpathlineto{\pgfqpoint{1.154062in}{1.618144in}}%
\pgfpathlineto{\pgfqpoint{1.155915in}{1.621556in}}%
\pgfpathlineto{\pgfqpoint{1.157768in}{1.624835in}}%
\pgfpathlineto{\pgfqpoint{1.159088in}{1.624517in}}%
\pgfpathlineto{\pgfqpoint{1.160428in}{1.624219in}}%
\pgfpathlineto{\pgfqpoint{1.161788in}{1.623941in}}%
\pgfpathlineto{\pgfqpoint{1.163167in}{1.623682in}}%
\pgfpathclose%
\pgfusepath{fill}%
\end{pgfscope}%
\begin{pgfscope}%
\pgfpathrectangle{\pgfqpoint{0.041670in}{0.041670in}}{\pgfqpoint{2.216660in}{2.216660in}}%
\pgfusepath{clip}%
\pgfsetbuttcap%
\pgfsetroundjoin%
\definecolor{currentfill}{rgb}{0.195860,0.395433,0.555276}%
\pgfsetfillcolor{currentfill}%
\pgfsetlinewidth{0.000000pt}%
\definecolor{currentstroke}{rgb}{0.000000,0.000000,0.000000}%
\pgfsetstrokecolor{currentstroke}%
\pgfsetdash{}{0pt}%
\pgfpathmoveto{\pgfqpoint{1.234811in}{0.941202in}}%
\pgfpathlineto{\pgfqpoint{1.235355in}{0.931613in}}%
\pgfpathlineto{\pgfqpoint{1.235898in}{0.922061in}}%
\pgfpathlineto{\pgfqpoint{1.236442in}{0.912548in}}%
\pgfpathlineto{\pgfqpoint{1.236986in}{0.903078in}}%
\pgfpathlineto{\pgfqpoint{1.224217in}{0.902246in}}%
\pgfpathlineto{\pgfqpoint{1.211400in}{0.901625in}}%
\pgfpathlineto{\pgfqpoint{1.198549in}{0.901214in}}%
\pgfpathlineto{\pgfqpoint{1.185678in}{0.901014in}}%
\pgfpathlineto{\pgfqpoint{1.185624in}{0.910507in}}%
\pgfpathlineto{\pgfqpoint{1.185569in}{0.920042in}}%
\pgfpathlineto{\pgfqpoint{1.185514in}{0.929616in}}%
\pgfpathlineto{\pgfqpoint{1.185460in}{0.939227in}}%
\pgfpathlineto{\pgfqpoint{1.197840in}{0.939418in}}%
\pgfpathlineto{\pgfqpoint{1.210201in}{0.939811in}}%
\pgfpathlineto{\pgfqpoint{1.222529in}{0.940406in}}%
\pgfpathlineto{\pgfqpoint{1.234811in}{0.941202in}}%
\pgfpathclose%
\pgfusepath{fill}%
\end{pgfscope}%
\begin{pgfscope}%
\pgfpathrectangle{\pgfqpoint{0.041670in}{0.041670in}}{\pgfqpoint{2.216660in}{2.216660in}}%
\pgfusepath{clip}%
\pgfsetbuttcap%
\pgfsetroundjoin%
\definecolor{currentfill}{rgb}{0.699415,0.867117,0.175971}%
\pgfsetfillcolor{currentfill}%
\pgfsetlinewidth{0.000000pt}%
\definecolor{currentstroke}{rgb}{0.000000,0.000000,0.000000}%
\pgfsetstrokecolor{currentstroke}%
\pgfsetdash{}{0pt}%
\pgfpathmoveto{\pgfqpoint{1.084313in}{1.520760in}}%
\pgfpathlineto{\pgfqpoint{1.081657in}{1.514790in}}%
\pgfpathlineto{\pgfqpoint{1.079002in}{1.508710in}}%
\pgfpathlineto{\pgfqpoint{1.076348in}{1.502520in}}%
\pgfpathlineto{\pgfqpoint{1.073694in}{1.496223in}}%
\pgfpathlineto{\pgfqpoint{1.070127in}{1.497850in}}%
\pgfpathlineto{\pgfqpoint{1.066673in}{1.499530in}}%
\pgfpathlineto{\pgfqpoint{1.063335in}{1.501261in}}%
\pgfpathlineto{\pgfqpoint{1.060115in}{1.503041in}}%
\pgfpathlineto{\pgfqpoint{1.063105in}{1.509162in}}%
\pgfpathlineto{\pgfqpoint{1.066096in}{1.515175in}}%
\pgfpathlineto{\pgfqpoint{1.069088in}{1.521080in}}%
\pgfpathlineto{\pgfqpoint{1.072080in}{1.526875in}}%
\pgfpathlineto{\pgfqpoint{1.074981in}{1.525279in}}%
\pgfpathlineto{\pgfqpoint{1.077989in}{1.523726in}}%
\pgfpathlineto{\pgfqpoint{1.081101in}{1.522220in}}%
\pgfpathlineto{\pgfqpoint{1.084313in}{1.520760in}}%
\pgfpathclose%
\pgfusepath{fill}%
\end{pgfscope}%
\begin{pgfscope}%
\pgfpathrectangle{\pgfqpoint{0.041670in}{0.041670in}}{\pgfqpoint{2.216660in}{2.216660in}}%
\pgfusepath{clip}%
\pgfsetbuttcap%
\pgfsetroundjoin%
\definecolor{currentfill}{rgb}{0.179019,0.433756,0.557430}%
\pgfsetfillcolor{currentfill}%
\pgfsetlinewidth{0.000000pt}%
\definecolor{currentstroke}{rgb}{0.000000,0.000000,0.000000}%
\pgfsetstrokecolor{currentstroke}%
\pgfsetdash{}{0pt}%
\pgfpathmoveto{\pgfqpoint{1.137756in}{0.979168in}}%
\pgfpathlineto{\pgfqpoint{1.137320in}{0.969459in}}%
\pgfpathlineto{\pgfqpoint{1.136885in}{0.959773in}}%
\pgfpathlineto{\pgfqpoint{1.136449in}{0.950114in}}%
\pgfpathlineto{\pgfqpoint{1.136014in}{0.940484in}}%
\pgfpathlineto{\pgfqpoint{1.123737in}{0.941302in}}%
\pgfpathlineto{\pgfqpoint{1.111522in}{0.942320in}}%
\pgfpathlineto{\pgfqpoint{1.099381in}{0.943537in}}%
\pgfpathlineto{\pgfqpoint{1.087326in}{0.944951in}}%
\pgfpathlineto{\pgfqpoint{1.088244in}{0.954530in}}%
\pgfpathlineto{\pgfqpoint{1.089161in}{0.964139in}}%
\pgfpathlineto{\pgfqpoint{1.090079in}{0.973775in}}%
\pgfpathlineto{\pgfqpoint{1.090997in}{0.983434in}}%
\pgfpathlineto{\pgfqpoint{1.102574in}{0.982084in}}%
\pgfpathlineto{\pgfqpoint{1.114235in}{0.980922in}}%
\pgfpathlineto{\pgfqpoint{1.125966in}{0.979950in}}%
\pgfpathlineto{\pgfqpoint{1.137756in}{0.979168in}}%
\pgfpathclose%
\pgfusepath{fill}%
\end{pgfscope}%
\begin{pgfscope}%
\pgfpathrectangle{\pgfqpoint{0.041670in}{0.041670in}}{\pgfqpoint{2.216660in}{2.216660in}}%
\pgfusepath{clip}%
\pgfsetbuttcap%
\pgfsetroundjoin%
\definecolor{currentfill}{rgb}{0.120081,0.622161,0.534946}%
\pgfsetfillcolor{currentfill}%
\pgfsetlinewidth{0.000000pt}%
\definecolor{currentstroke}{rgb}{0.000000,0.000000,0.000000}%
\pgfsetstrokecolor{currentstroke}%
\pgfsetdash{}{0pt}%
\pgfpathmoveto{\pgfqpoint{1.327802in}{1.192473in}}%
\pgfpathlineto{\pgfqpoint{1.329729in}{1.183174in}}%
\pgfpathlineto{\pgfqpoint{1.331656in}{1.173842in}}%
\pgfpathlineto{\pgfqpoint{1.333582in}{1.164481in}}%
\pgfpathlineto{\pgfqpoint{1.335507in}{1.155093in}}%
\pgfpathlineto{\pgfqpoint{1.326888in}{1.152696in}}%
\pgfpathlineto{\pgfqpoint{1.318114in}{1.150436in}}%
\pgfpathlineto{\pgfqpoint{1.309193in}{1.148316in}}%
\pgfpathlineto{\pgfqpoint{1.300136in}{1.146337in}}%
\pgfpathlineto{\pgfqpoint{1.298647in}{1.155846in}}%
\pgfpathlineto{\pgfqpoint{1.297157in}{1.165327in}}%
\pgfpathlineto{\pgfqpoint{1.295667in}{1.174779in}}%
\pgfpathlineto{\pgfqpoint{1.294176in}{1.184197in}}%
\pgfpathlineto{\pgfqpoint{1.302786in}{1.186067in}}%
\pgfpathlineto{\pgfqpoint{1.311266in}{1.188071in}}%
\pgfpathlineto{\pgfqpoint{1.319607in}{1.190207in}}%
\pgfpathlineto{\pgfqpoint{1.327802in}{1.192473in}}%
\pgfpathclose%
\pgfusepath{fill}%
\end{pgfscope}%
\begin{pgfscope}%
\pgfpathrectangle{\pgfqpoint{0.041670in}{0.041670in}}{\pgfqpoint{2.216660in}{2.216660in}}%
\pgfusepath{clip}%
\pgfsetbuttcap%
\pgfsetroundjoin%
\definecolor{currentfill}{rgb}{0.147607,0.511733,0.557049}%
\pgfsetfillcolor{currentfill}%
\pgfsetlinewidth{0.000000pt}%
\definecolor{currentstroke}{rgb}{0.000000,0.000000,0.000000}%
\pgfsetstrokecolor{currentstroke}%
\pgfsetdash{}{0pt}%
\pgfpathmoveto{\pgfqpoint{1.098343in}{1.061200in}}%
\pgfpathlineto{\pgfqpoint{1.097424in}{1.051459in}}%
\pgfpathlineto{\pgfqpoint{1.096506in}{1.041719in}}%
\pgfpathlineto{\pgfqpoint{1.095587in}{1.031981in}}%
\pgfpathlineto{\pgfqpoint{1.094669in}{1.022248in}}%
\pgfpathlineto{\pgfqpoint{1.083660in}{1.023714in}}%
\pgfpathlineto{\pgfqpoint{1.072755in}{1.025357in}}%
\pgfpathlineto{\pgfqpoint{1.061964in}{1.027174in}}%
\pgfpathlineto{\pgfqpoint{1.051300in}{1.029164in}}%
\pgfpathlineto{\pgfqpoint{1.052684in}{1.038812in}}%
\pgfpathlineto{\pgfqpoint{1.054068in}{1.048467in}}%
\pgfpathlineto{\pgfqpoint{1.055453in}{1.058124in}}%
\pgfpathlineto{\pgfqpoint{1.056838in}{1.067781in}}%
\pgfpathlineto{\pgfqpoint{1.067044in}{1.065887in}}%
\pgfpathlineto{\pgfqpoint{1.077372in}{1.064158in}}%
\pgfpathlineto{\pgfqpoint{1.087808in}{1.062595in}}%
\pgfpathlineto{\pgfqpoint{1.098343in}{1.061200in}}%
\pgfpathclose%
\pgfusepath{fill}%
\end{pgfscope}%
\begin{pgfscope}%
\pgfpathrectangle{\pgfqpoint{0.041670in}{0.041670in}}{\pgfqpoint{2.216660in}{2.216660in}}%
\pgfusepath{clip}%
\pgfsetbuttcap%
\pgfsetroundjoin%
\definecolor{currentfill}{rgb}{0.274952,0.037752,0.364543}%
\pgfsetfillcolor{currentfill}%
\pgfsetlinewidth{0.000000pt}%
\definecolor{currentstroke}{rgb}{0.000000,0.000000,0.000000}%
\pgfsetstrokecolor{currentstroke}%
\pgfsetdash{}{0pt}%
\pgfpathmoveto{\pgfqpoint{1.187659in}{0.617010in}}%
\pgfpathlineto{\pgfqpoint{1.187715in}{0.611606in}}%
\pgfpathlineto{\pgfqpoint{1.187771in}{0.606388in}}%
\pgfpathlineto{\pgfqpoint{1.187827in}{0.601359in}}%
\pgfpathlineto{\pgfqpoint{1.187883in}{0.596526in}}%
\pgfpathlineto{\pgfqpoint{1.170045in}{0.596543in}}%
\pgfpathlineto{\pgfqpoint{1.152218in}{0.596865in}}%
\pgfpathlineto{\pgfqpoint{1.134422in}{0.597492in}}%
\pgfpathlineto{\pgfqpoint{1.116677in}{0.598423in}}%
\pgfpathlineto{\pgfqpoint{1.117125in}{0.603242in}}%
\pgfpathlineto{\pgfqpoint{1.117573in}{0.608256in}}%
\pgfpathlineto{\pgfqpoint{1.118020in}{0.613460in}}%
\pgfpathlineto{\pgfqpoint{1.118467in}{0.618849in}}%
\pgfpathlineto{\pgfqpoint{1.135710in}{0.617947in}}%
\pgfpathlineto{\pgfqpoint{1.153003in}{0.617339in}}%
\pgfpathlineto{\pgfqpoint{1.170325in}{0.617027in}}%
\pgfpathlineto{\pgfqpoint{1.187659in}{0.617010in}}%
\pgfpathclose%
\pgfusepath{fill}%
\end{pgfscope}%
\begin{pgfscope}%
\pgfpathrectangle{\pgfqpoint{0.041670in}{0.041670in}}{\pgfqpoint{2.216660in}{2.216660in}}%
\pgfusepath{clip}%
\pgfsetbuttcap%
\pgfsetroundjoin%
\definecolor{currentfill}{rgb}{0.195860,0.395433,0.555276}%
\pgfsetfillcolor{currentfill}%
\pgfsetlinewidth{0.000000pt}%
\definecolor{currentstroke}{rgb}{0.000000,0.000000,0.000000}%
\pgfsetstrokecolor{currentstroke}%
\pgfsetdash{}{0pt}%
\pgfpathmoveto{\pgfqpoint{1.185460in}{0.939227in}}%
\pgfpathlineto{\pgfqpoint{1.185514in}{0.929616in}}%
\pgfpathlineto{\pgfqpoint{1.185569in}{0.920042in}}%
\pgfpathlineto{\pgfqpoint{1.185624in}{0.910507in}}%
\pgfpathlineto{\pgfqpoint{1.185678in}{0.901014in}}%
\pgfpathlineto{\pgfqpoint{1.172801in}{0.901026in}}%
\pgfpathlineto{\pgfqpoint{1.159931in}{0.901249in}}%
\pgfpathlineto{\pgfqpoint{1.147084in}{0.901683in}}%
\pgfpathlineto{\pgfqpoint{1.134272in}{0.902328in}}%
\pgfpathlineto{\pgfqpoint{1.134707in}{0.911807in}}%
\pgfpathlineto{\pgfqpoint{1.135143in}{0.921328in}}%
\pgfpathlineto{\pgfqpoint{1.135578in}{0.930888in}}%
\pgfpathlineto{\pgfqpoint{1.136014in}{0.940484in}}%
\pgfpathlineto{\pgfqpoint{1.148337in}{0.939867in}}%
\pgfpathlineto{\pgfqpoint{1.160695in}{0.939452in}}%
\pgfpathlineto{\pgfqpoint{1.173074in}{0.939238in}}%
\pgfpathlineto{\pgfqpoint{1.185460in}{0.939227in}}%
\pgfpathclose%
\pgfusepath{fill}%
\end{pgfscope}%
\begin{pgfscope}%
\pgfpathrectangle{\pgfqpoint{0.041670in}{0.041670in}}{\pgfqpoint{2.216660in}{2.216660in}}%
\pgfusepath{clip}%
\pgfsetbuttcap%
\pgfsetroundjoin%
\definecolor{currentfill}{rgb}{0.271305,0.019942,0.347269}%
\pgfsetfillcolor{currentfill}%
\pgfsetlinewidth{0.000000pt}%
\definecolor{currentstroke}{rgb}{0.000000,0.000000,0.000000}%
\pgfsetstrokecolor{currentstroke}%
\pgfsetdash{}{0pt}%
\pgfpathmoveto{\pgfqpoint{1.116677in}{0.598423in}}%
\pgfpathlineto{\pgfqpoint{1.116227in}{0.593803in}}%
\pgfpathlineto{\pgfqpoint{1.115777in}{0.589387in}}%
\pgfpathlineto{\pgfqpoint{1.115326in}{0.585179in}}%
\pgfpathlineto{\pgfqpoint{1.114874in}{0.581184in}}%
\pgfpathlineto{\pgfqpoint{1.096696in}{0.582456in}}%
\pgfpathlineto{\pgfqpoint{1.078612in}{0.584039in}}%
\pgfpathlineto{\pgfqpoint{1.060641in}{0.585931in}}%
\pgfpathlineto{\pgfqpoint{1.042804in}{0.588129in}}%
\pgfpathlineto{\pgfqpoint{1.043756in}{0.592072in}}%
\pgfpathlineto{\pgfqpoint{1.044705in}{0.596229in}}%
\pgfpathlineto{\pgfqpoint{1.045653in}{0.600593in}}%
\pgfpathlineto{\pgfqpoint{1.046600in}{0.605161in}}%
\pgfpathlineto{\pgfqpoint{1.063944in}{0.603028in}}%
\pgfpathlineto{\pgfqpoint{1.081418in}{0.601193in}}%
\pgfpathlineto{\pgfqpoint{1.099002in}{0.599658in}}%
\pgfpathlineto{\pgfqpoint{1.116677in}{0.598423in}}%
\pgfpathclose%
\pgfusepath{fill}%
\end{pgfscope}%
\begin{pgfscope}%
\pgfpathrectangle{\pgfqpoint{0.041670in}{0.041670in}}{\pgfqpoint{2.216660in}{2.216660in}}%
\pgfusepath{clip}%
\pgfsetbuttcap%
\pgfsetroundjoin%
\definecolor{currentfill}{rgb}{0.814576,0.883393,0.110347}%
\pgfsetfillcolor{currentfill}%
\pgfsetlinewidth{0.000000pt}%
\definecolor{currentstroke}{rgb}{0.000000,0.000000,0.000000}%
\pgfsetstrokecolor{currentstroke}%
\pgfsetdash{}{0pt}%
\pgfpathmoveto{\pgfqpoint{1.105576in}{1.564394in}}%
\pgfpathlineto{\pgfqpoint{1.102917in}{1.559352in}}%
\pgfpathlineto{\pgfqpoint{1.100258in}{1.554190in}}%
\pgfpathlineto{\pgfqpoint{1.097600in}{1.548909in}}%
\pgfpathlineto{\pgfqpoint{1.094942in}{1.543510in}}%
\pgfpathlineto{\pgfqpoint{1.092084in}{1.544803in}}%
\pgfpathlineto{\pgfqpoint{1.089316in}{1.546138in}}%
\pgfpathlineto{\pgfqpoint{1.086640in}{1.547513in}}%
\pgfpathlineto{\pgfqpoint{1.084058in}{1.548927in}}%
\pgfpathlineto{\pgfqpoint{1.087054in}{1.554152in}}%
\pgfpathlineto{\pgfqpoint{1.090051in}{1.559260in}}%
\pgfpathlineto{\pgfqpoint{1.093048in}{1.564249in}}%
\pgfpathlineto{\pgfqpoint{1.096045in}{1.569119in}}%
\pgfpathlineto{\pgfqpoint{1.098306in}{1.567886in}}%
\pgfpathlineto{\pgfqpoint{1.100650in}{1.566686in}}%
\pgfpathlineto{\pgfqpoint{1.103074in}{1.565522in}}%
\pgfpathlineto{\pgfqpoint{1.105576in}{1.564394in}}%
\pgfpathclose%
\pgfusepath{fill}%
\end{pgfscope}%
\begin{pgfscope}%
\pgfpathrectangle{\pgfqpoint{0.041670in}{0.041670in}}{\pgfqpoint{2.216660in}{2.216660in}}%
\pgfusepath{clip}%
\pgfsetbuttcap%
\pgfsetroundjoin%
\definecolor{currentfill}{rgb}{0.220124,0.725509,0.466226}%
\pgfsetfillcolor{currentfill}%
\pgfsetlinewidth{0.000000pt}%
\definecolor{currentstroke}{rgb}{0.000000,0.000000,0.000000}%
\pgfsetstrokecolor{currentstroke}%
\pgfsetdash{}{0pt}%
\pgfpathmoveto{\pgfqpoint{1.330907in}{1.309196in}}%
\pgfpathlineto{\pgfqpoint{1.333249in}{1.300620in}}%
\pgfpathlineto{\pgfqpoint{1.335589in}{1.291983in}}%
\pgfpathlineto{\pgfqpoint{1.337929in}{1.283288in}}%
\pgfpathlineto{\pgfqpoint{1.340268in}{1.274536in}}%
\pgfpathlineto{\pgfqpoint{1.333523in}{1.272092in}}%
\pgfpathlineto{\pgfqpoint{1.326618in}{1.269753in}}%
\pgfpathlineto{\pgfqpoint{1.319560in}{1.267522in}}%
\pgfpathlineto{\pgfqpoint{1.312356in}{1.265402in}}%
\pgfpathlineto{\pgfqpoint{1.310422in}{1.274298in}}%
\pgfpathlineto{\pgfqpoint{1.308487in}{1.283138in}}%
\pgfpathlineto{\pgfqpoint{1.306552in}{1.291919in}}%
\pgfpathlineto{\pgfqpoint{1.304615in}{1.300640in}}%
\pgfpathlineto{\pgfqpoint{1.311401in}{1.302626in}}%
\pgfpathlineto{\pgfqpoint{1.318049in}{1.304715in}}%
\pgfpathlineto{\pgfqpoint{1.324553in}{1.306906in}}%
\pgfpathlineto{\pgfqpoint{1.330907in}{1.309196in}}%
\pgfpathclose%
\pgfusepath{fill}%
\end{pgfscope}%
\begin{pgfscope}%
\pgfpathrectangle{\pgfqpoint{0.041670in}{0.041670in}}{\pgfqpoint{2.216660in}{2.216660in}}%
\pgfusepath{clip}%
\pgfsetbuttcap%
\pgfsetroundjoin%
\definecolor{currentfill}{rgb}{0.762373,0.876424,0.137064}%
\pgfsetfillcolor{currentfill}%
\pgfsetlinewidth{0.000000pt}%
\definecolor{currentstroke}{rgb}{0.000000,0.000000,0.000000}%
\pgfsetstrokecolor{currentstroke}%
\pgfsetdash{}{0pt}%
\pgfpathmoveto{\pgfqpoint{1.094942in}{1.543510in}}%
\pgfpathlineto{\pgfqpoint{1.092284in}{1.537995in}}%
\pgfpathlineto{\pgfqpoint{1.089626in}{1.532364in}}%
\pgfpathlineto{\pgfqpoint{1.086970in}{1.526618in}}%
\pgfpathlineto{\pgfqpoint{1.084313in}{1.520760in}}%
\pgfpathlineto{\pgfqpoint{1.081101in}{1.522220in}}%
\pgfpathlineto{\pgfqpoint{1.077989in}{1.523726in}}%
\pgfpathlineto{\pgfqpoint{1.074981in}{1.525279in}}%
\pgfpathlineto{\pgfqpoint{1.072080in}{1.526875in}}%
\pgfpathlineto{\pgfqpoint{1.075074in}{1.532558in}}%
\pgfpathlineto{\pgfqpoint{1.078068in}{1.538129in}}%
\pgfpathlineto{\pgfqpoint{1.081063in}{1.543585in}}%
\pgfpathlineto{\pgfqpoint{1.084058in}{1.548927in}}%
\pgfpathlineto{\pgfqpoint{1.086640in}{1.547513in}}%
\pgfpathlineto{\pgfqpoint{1.089316in}{1.546138in}}%
\pgfpathlineto{\pgfqpoint{1.092084in}{1.544803in}}%
\pgfpathlineto{\pgfqpoint{1.094942in}{1.543510in}}%
\pgfpathclose%
\pgfusepath{fill}%
\end{pgfscope}%
\begin{pgfscope}%
\pgfpathrectangle{\pgfqpoint{0.041670in}{0.041670in}}{\pgfqpoint{2.216660in}{2.216660in}}%
\pgfusepath{clip}%
\pgfsetbuttcap%
\pgfsetroundjoin%
\definecolor{currentfill}{rgb}{0.267004,0.004874,0.329415}%
\pgfsetfillcolor{currentfill}%
\pgfsetlinewidth{0.000000pt}%
\definecolor{currentstroke}{rgb}{0.000000,0.000000,0.000000}%
\pgfsetstrokecolor{currentstroke}%
\pgfsetdash{}{0pt}%
\pgfpathmoveto{\pgfqpoint{1.407912in}{0.590078in}}%
\pgfpathlineto{\pgfqpoint{1.409463in}{0.587381in}}%
\pgfpathlineto{\pgfqpoint{1.411018in}{0.584925in}}%
\pgfpathlineto{\pgfqpoint{1.412576in}{0.582714in}}%
\pgfpathlineto{\pgfqpoint{1.414139in}{0.580755in}}%
\pgfpathlineto{\pgfqpoint{1.396296in}{0.576890in}}%
\pgfpathlineto{\pgfqpoint{1.378212in}{0.573331in}}%
\pgfpathlineto{\pgfqpoint{1.359905in}{0.570082in}}%
\pgfpathlineto{\pgfqpoint{1.341395in}{0.567149in}}%
\pgfpathlineto{\pgfqpoint{1.340317in}{0.569203in}}%
\pgfpathlineto{\pgfqpoint{1.339240in}{0.571508in}}%
\pgfpathlineto{\pgfqpoint{1.338167in}{0.574059in}}%
\pgfpathlineto{\pgfqpoint{1.337095in}{0.576851in}}%
\pgfpathlineto{\pgfqpoint{1.355113in}{0.579703in}}%
\pgfpathlineto{\pgfqpoint{1.372935in}{0.582860in}}%
\pgfpathlineto{\pgfqpoint{1.390541in}{0.586320in}}%
\pgfpathlineto{\pgfqpoint{1.407912in}{0.590078in}}%
\pgfpathclose%
\pgfusepath{fill}%
\end{pgfscope}%
\begin{pgfscope}%
\pgfpathrectangle{\pgfqpoint{0.041670in}{0.041670in}}{\pgfqpoint{2.216660in}{2.216660in}}%
\pgfusepath{clip}%
\pgfsetbuttcap%
\pgfsetroundjoin%
\definecolor{currentfill}{rgb}{0.955300,0.901065,0.118128}%
\pgfsetfillcolor{currentfill}%
\pgfsetlinewidth{0.000000pt}%
\definecolor{currentstroke}{rgb}{0.000000,0.000000,0.000000}%
\pgfsetstrokecolor{currentstroke}%
\pgfsetdash{}{0pt}%
\pgfpathmoveto{\pgfqpoint{1.197969in}{1.623911in}}%
\pgfpathlineto{\pgfqpoint{1.199474in}{1.620553in}}%
\pgfpathlineto{\pgfqpoint{1.200979in}{1.617063in}}%
\pgfpathlineto{\pgfqpoint{1.202485in}{1.613442in}}%
\pgfpathlineto{\pgfqpoint{1.203990in}{1.609689in}}%
\pgfpathlineto{\pgfqpoint{1.202149in}{1.609346in}}%
\pgfpathlineto{\pgfqpoint{1.200285in}{1.609032in}}%
\pgfpathlineto{\pgfqpoint{1.198401in}{1.608745in}}%
\pgfpathlineto{\pgfqpoint{1.196498in}{1.608485in}}%
\pgfpathlineto{\pgfqpoint{1.195461in}{1.612314in}}%
\pgfpathlineto{\pgfqpoint{1.194425in}{1.616012in}}%
\pgfpathlineto{\pgfqpoint{1.193388in}{1.619578in}}%
\pgfpathlineto{\pgfqpoint{1.192353in}{1.623011in}}%
\pgfpathlineto{\pgfqpoint{1.193779in}{1.623205in}}%
\pgfpathlineto{\pgfqpoint{1.195192in}{1.623419in}}%
\pgfpathlineto{\pgfqpoint{1.196589in}{1.623655in}}%
\pgfpathlineto{\pgfqpoint{1.197969in}{1.623911in}}%
\pgfpathclose%
\pgfusepath{fill}%
\end{pgfscope}%
\begin{pgfscope}%
\pgfpathrectangle{\pgfqpoint{0.041670in}{0.041670in}}{\pgfqpoint{2.216660in}{2.216660in}}%
\pgfusepath{clip}%
\pgfsetbuttcap%
\pgfsetroundjoin%
\definecolor{currentfill}{rgb}{0.955300,0.901065,0.118128}%
\pgfsetfillcolor{currentfill}%
\pgfsetlinewidth{0.000000pt}%
\definecolor{currentstroke}{rgb}{0.000000,0.000000,0.000000}%
\pgfsetstrokecolor{currentstroke}%
\pgfsetdash{}{0pt}%
\pgfpathmoveto{\pgfqpoint{1.168835in}{1.622856in}}%
\pgfpathlineto{\pgfqpoint{1.167906in}{1.619410in}}%
\pgfpathlineto{\pgfqpoint{1.166977in}{1.615832in}}%
\pgfpathlineto{\pgfqpoint{1.166048in}{1.612121in}}%
\pgfpathlineto{\pgfqpoint{1.165118in}{1.608279in}}%
\pgfpathlineto{\pgfqpoint{1.163200in}{1.608513in}}%
\pgfpathlineto{\pgfqpoint{1.161299in}{1.608775in}}%
\pgfpathlineto{\pgfqpoint{1.159417in}{1.609065in}}%
\pgfpathlineto{\pgfqpoint{1.157555in}{1.609383in}}%
\pgfpathlineto{\pgfqpoint{1.158959in}{1.613155in}}%
\pgfpathlineto{\pgfqpoint{1.160362in}{1.616796in}}%
\pgfpathlineto{\pgfqpoint{1.161764in}{1.620306in}}%
\pgfpathlineto{\pgfqpoint{1.163167in}{1.623682in}}%
\pgfpathlineto{\pgfqpoint{1.164562in}{1.623445in}}%
\pgfpathlineto{\pgfqpoint{1.165973in}{1.623228in}}%
\pgfpathlineto{\pgfqpoint{1.167398in}{1.623031in}}%
\pgfpathlineto{\pgfqpoint{1.168835in}{1.622856in}}%
\pgfpathclose%
\pgfusepath{fill}%
\end{pgfscope}%
\begin{pgfscope}%
\pgfpathrectangle{\pgfqpoint{0.041670in}{0.041670in}}{\pgfqpoint{2.216660in}{2.216660in}}%
\pgfusepath{clip}%
\pgfsetbuttcap%
\pgfsetroundjoin%
\definecolor{currentfill}{rgb}{0.233603,0.313828,0.543914}%
\pgfsetfillcolor{currentfill}%
\pgfsetlinewidth{0.000000pt}%
\definecolor{currentstroke}{rgb}{0.000000,0.000000,0.000000}%
\pgfsetstrokecolor{currentstroke}%
\pgfsetdash{}{0pt}%
\pgfpathmoveto{\pgfqpoint{0.615620in}{0.794160in}}%
\pgfpathlineto{\pgfqpoint{0.612196in}{0.804245in}}%
\pgfpathlineto{\pgfqpoint{0.608755in}{0.814789in}}%
\pgfpathlineto{\pgfqpoint{0.605295in}{0.825798in}}%
\pgfpathlineto{\pgfqpoint{0.601818in}{0.837283in}}%
\pgfpathlineto{\pgfqpoint{0.587501in}{0.846975in}}%
\pgfpathlineto{\pgfqpoint{0.573833in}{0.856885in}}%
\pgfpathlineto{\pgfqpoint{0.560826in}{0.867002in}}%
\pgfpathlineto{\pgfqpoint{0.548491in}{0.877314in}}%
\pgfpathlineto{\pgfqpoint{0.552269in}{0.865653in}}%
\pgfpathlineto{\pgfqpoint{0.556027in}{0.854465in}}%
\pgfpathlineto{\pgfqpoint{0.559765in}{0.843740in}}%
\pgfpathlineto{\pgfqpoint{0.563485in}{0.833472in}}%
\pgfpathlineto{\pgfqpoint{0.575548in}{0.823345in}}%
\pgfpathlineto{\pgfqpoint{0.588266in}{0.813409in}}%
\pgfpathlineto{\pgfqpoint{0.601627in}{0.803677in}}%
\pgfpathlineto{\pgfqpoint{0.615620in}{0.794160in}}%
\pgfpathclose%
\pgfusepath{fill}%
\end{pgfscope}%
\begin{pgfscope}%
\pgfpathrectangle{\pgfqpoint{0.041670in}{0.041670in}}{\pgfqpoint{2.216660in}{2.216660in}}%
\pgfusepath{clip}%
\pgfsetbuttcap%
\pgfsetroundjoin%
\definecolor{currentfill}{rgb}{0.896320,0.893616,0.096335}%
\pgfsetfillcolor{currentfill}%
\pgfsetlinewidth{0.000000pt}%
\definecolor{currentstroke}{rgb}{0.000000,0.000000,0.000000}%
\pgfsetstrokecolor{currentstroke}%
\pgfsetdash{}{0pt}%
\pgfpathmoveto{\pgfqpoint{1.234650in}{1.601011in}}%
\pgfpathlineto{\pgfqpoint{1.237389in}{1.596998in}}%
\pgfpathlineto{\pgfqpoint{1.240128in}{1.592858in}}%
\pgfpathlineto{\pgfqpoint{1.242867in}{1.588591in}}%
\pgfpathlineto{\pgfqpoint{1.245606in}{1.584199in}}%
\pgfpathlineto{\pgfqpoint{1.243453in}{1.583239in}}%
\pgfpathlineto{\pgfqpoint{1.241236in}{1.582312in}}%
\pgfpathlineto{\pgfqpoint{1.238957in}{1.581417in}}%
\pgfpathlineto{\pgfqpoint{1.236618in}{1.580557in}}%
\pgfpathlineto{\pgfqpoint{1.234252in}{1.585103in}}%
\pgfpathlineto{\pgfqpoint{1.231887in}{1.589524in}}%
\pgfpathlineto{\pgfqpoint{1.229522in}{1.593818in}}%
\pgfpathlineto{\pgfqpoint{1.227157in}{1.597985in}}%
\pgfpathlineto{\pgfqpoint{1.229106in}{1.598700in}}%
\pgfpathlineto{\pgfqpoint{1.231006in}{1.599443in}}%
\pgfpathlineto{\pgfqpoint{1.232855in}{1.600214in}}%
\pgfpathlineto{\pgfqpoint{1.234650in}{1.601011in}}%
\pgfpathclose%
\pgfusepath{fill}%
\end{pgfscope}%
\begin{pgfscope}%
\pgfpathrectangle{\pgfqpoint{0.041670in}{0.041670in}}{\pgfqpoint{2.216660in}{2.216660in}}%
\pgfusepath{clip}%
\pgfsetbuttcap%
\pgfsetroundjoin%
\definecolor{currentfill}{rgb}{0.412913,0.803041,0.357269}%
\pgfsetfillcolor{currentfill}%
\pgfsetlinewidth{0.000000pt}%
\definecolor{currentstroke}{rgb}{0.000000,0.000000,0.000000}%
\pgfsetstrokecolor{currentstroke}%
\pgfsetdash{}{0pt}%
\pgfpathmoveto{\pgfqpoint{1.061770in}{1.404963in}}%
\pgfpathlineto{\pgfqpoint{1.059505in}{1.397246in}}%
\pgfpathlineto{\pgfqpoint{1.057241in}{1.389444in}}%
\pgfpathlineto{\pgfqpoint{1.054977in}{1.381558in}}%
\pgfpathlineto{\pgfqpoint{1.052714in}{1.373591in}}%
\pgfpathlineto{\pgfqpoint{1.047160in}{1.375586in}}%
\pgfpathlineto{\pgfqpoint{1.041743in}{1.377664in}}%
\pgfpathlineto{\pgfqpoint{1.036469in}{1.379824in}}%
\pgfpathlineto{\pgfqpoint{1.031343in}{1.382063in}}%
\pgfpathlineto{\pgfqpoint{1.033983in}{1.389870in}}%
\pgfpathlineto{\pgfqpoint{1.036624in}{1.397596in}}%
\pgfpathlineto{\pgfqpoint{1.039265in}{1.405238in}}%
\pgfpathlineto{\pgfqpoint{1.041908in}{1.412796in}}%
\pgfpathlineto{\pgfqpoint{1.046673in}{1.410726in}}%
\pgfpathlineto{\pgfqpoint{1.051575in}{1.408729in}}%
\pgfpathlineto{\pgfqpoint{1.056609in}{1.406807in}}%
\pgfpathlineto{\pgfqpoint{1.061770in}{1.404963in}}%
\pgfpathclose%
\pgfusepath{fill}%
\end{pgfscope}%
\begin{pgfscope}%
\pgfpathrectangle{\pgfqpoint{0.041670in}{0.041670in}}{\pgfqpoint{2.216660in}{2.216660in}}%
\pgfusepath{clip}%
\pgfsetbuttcap%
\pgfsetroundjoin%
\definecolor{currentfill}{rgb}{0.935904,0.898570,0.108131}%
\pgfsetfillcolor{currentfill}%
\pgfsetlinewidth{0.000000pt}%
\definecolor{currentstroke}{rgb}{0.000000,0.000000,0.000000}%
\pgfsetstrokecolor{currentstroke}%
\pgfsetdash{}{0pt}%
\pgfpathmoveto{\pgfqpoint{1.217699in}{1.613366in}}%
\pgfpathlineto{\pgfqpoint{1.220063in}{1.609715in}}%
\pgfpathlineto{\pgfqpoint{1.222427in}{1.605934in}}%
\pgfpathlineto{\pgfqpoint{1.224792in}{1.602024in}}%
\pgfpathlineto{\pgfqpoint{1.227157in}{1.597985in}}%
\pgfpathlineto{\pgfqpoint{1.225160in}{1.597300in}}%
\pgfpathlineto{\pgfqpoint{1.223117in}{1.596644in}}%
\pgfpathlineto{\pgfqpoint{1.221030in}{1.596019in}}%
\pgfpathlineto{\pgfqpoint{1.218902in}{1.595425in}}%
\pgfpathlineto{\pgfqpoint{1.216950in}{1.599594in}}%
\pgfpathlineto{\pgfqpoint{1.214998in}{1.603634in}}%
\pgfpathlineto{\pgfqpoint{1.213047in}{1.607544in}}%
\pgfpathlineto{\pgfqpoint{1.211096in}{1.611325in}}%
\pgfpathlineto{\pgfqpoint{1.212798in}{1.611798in}}%
\pgfpathlineto{\pgfqpoint{1.214467in}{1.612297in}}%
\pgfpathlineto{\pgfqpoint{1.216101in}{1.612820in}}%
\pgfpathlineto{\pgfqpoint{1.217699in}{1.613366in}}%
\pgfpathclose%
\pgfusepath{fill}%
\end{pgfscope}%
\begin{pgfscope}%
\pgfpathrectangle{\pgfqpoint{0.041670in}{0.041670in}}{\pgfqpoint{2.216660in}{2.216660in}}%
\pgfusepath{clip}%
\pgfsetbuttcap%
\pgfsetroundjoin%
\definecolor{currentfill}{rgb}{0.133743,0.548535,0.553541}%
\pgfsetfillcolor{currentfill}%
\pgfsetlinewidth{0.000000pt}%
\definecolor{currentstroke}{rgb}{0.000000,0.000000,0.000000}%
\pgfsetstrokecolor{currentstroke}%
\pgfsetdash{}{0pt}%
\pgfpathmoveto{\pgfqpoint{1.306089in}{1.108081in}}%
\pgfpathlineto{\pgfqpoint{1.307576in}{1.098475in}}%
\pgfpathlineto{\pgfqpoint{1.309063in}{1.088858in}}%
\pgfpathlineto{\pgfqpoint{1.310549in}{1.079232in}}%
\pgfpathlineto{\pgfqpoint{1.312035in}{1.069601in}}%
\pgfpathlineto{\pgfqpoint{1.301944in}{1.067563in}}%
\pgfpathlineto{\pgfqpoint{1.291724in}{1.065687in}}%
\pgfpathlineto{\pgfqpoint{1.281384in}{1.063976in}}%
\pgfpathlineto{\pgfqpoint{1.270936in}{1.062432in}}%
\pgfpathlineto{\pgfqpoint{1.269911in}{1.072153in}}%
\pgfpathlineto{\pgfqpoint{1.268886in}{1.081869in}}%
\pgfpathlineto{\pgfqpoint{1.267861in}{1.091577in}}%
\pgfpathlineto{\pgfqpoint{1.266836in}{1.101273in}}%
\pgfpathlineto{\pgfqpoint{1.276814in}{1.102739in}}%
\pgfpathlineto{\pgfqpoint{1.286690in}{1.104364in}}%
\pgfpathlineto{\pgfqpoint{1.296451in}{1.106146in}}%
\pgfpathlineto{\pgfqpoint{1.306089in}{1.108081in}}%
\pgfpathclose%
\pgfusepath{fill}%
\end{pgfscope}%
\begin{pgfscope}%
\pgfpathrectangle{\pgfqpoint{0.041670in}{0.041670in}}{\pgfqpoint{2.216660in}{2.216660in}}%
\pgfusepath{clip}%
\pgfsetbuttcap%
\pgfsetroundjoin%
\definecolor{currentfill}{rgb}{0.487026,0.823929,0.312321}%
\pgfsetfillcolor{currentfill}%
\pgfsetlinewidth{0.000000pt}%
\definecolor{currentstroke}{rgb}{0.000000,0.000000,0.000000}%
\pgfsetstrokecolor{currentstroke}%
\pgfsetdash{}{0pt}%
\pgfpathmoveto{\pgfqpoint{1.311224in}{1.443887in}}%
\pgfpathlineto{\pgfqpoint{1.313949in}{1.436725in}}%
\pgfpathlineto{\pgfqpoint{1.316673in}{1.429472in}}%
\pgfpathlineto{\pgfqpoint{1.319396in}{1.422128in}}%
\pgfpathlineto{\pgfqpoint{1.322118in}{1.414695in}}%
\pgfpathlineto{\pgfqpoint{1.317479in}{1.412562in}}%
\pgfpathlineto{\pgfqpoint{1.312699in}{1.410500in}}%
\pgfpathlineto{\pgfqpoint{1.307782in}{1.408511in}}%
\pgfpathlineto{\pgfqpoint{1.302734in}{1.406598in}}%
\pgfpathlineto{\pgfqpoint{1.300380in}{1.414195in}}%
\pgfpathlineto{\pgfqpoint{1.298026in}{1.421702in}}%
\pgfpathlineto{\pgfqpoint{1.295671in}{1.429119in}}%
\pgfpathlineto{\pgfqpoint{1.293315in}{1.436443in}}%
\pgfpathlineto{\pgfqpoint{1.297979in}{1.438202in}}%
\pgfpathlineto{\pgfqpoint{1.302521in}{1.440030in}}%
\pgfpathlineto{\pgfqpoint{1.306938in}{1.441926in}}%
\pgfpathlineto{\pgfqpoint{1.311224in}{1.443887in}}%
\pgfpathclose%
\pgfusepath{fill}%
\end{pgfscope}%
\begin{pgfscope}%
\pgfpathrectangle{\pgfqpoint{0.041670in}{0.041670in}}{\pgfqpoint{2.216660in}{2.216660in}}%
\pgfusepath{clip}%
\pgfsetbuttcap%
\pgfsetroundjoin%
\definecolor{currentfill}{rgb}{0.955300,0.901065,0.118128}%
\pgfsetfillcolor{currentfill}%
\pgfsetlinewidth{0.000000pt}%
\definecolor{currentstroke}{rgb}{0.000000,0.000000,0.000000}%
\pgfsetstrokecolor{currentstroke}%
\pgfsetdash{}{0pt}%
\pgfpathmoveto{\pgfqpoint{1.192353in}{1.623011in}}%
\pgfpathlineto{\pgfqpoint{1.193388in}{1.619578in}}%
\pgfpathlineto{\pgfqpoint{1.194425in}{1.616012in}}%
\pgfpathlineto{\pgfqpoint{1.195461in}{1.612314in}}%
\pgfpathlineto{\pgfqpoint{1.196498in}{1.608485in}}%
\pgfpathlineto{\pgfqpoint{1.194578in}{1.608255in}}%
\pgfpathlineto{\pgfqpoint{1.192643in}{1.608052in}}%
\pgfpathlineto{\pgfqpoint{1.190696in}{1.607879in}}%
\pgfpathlineto{\pgfqpoint{1.188738in}{1.607734in}}%
\pgfpathlineto{\pgfqpoint{1.188187in}{1.611611in}}%
\pgfpathlineto{\pgfqpoint{1.187637in}{1.615356in}}%
\pgfpathlineto{\pgfqpoint{1.187087in}{1.618969in}}%
\pgfpathlineto{\pgfqpoint{1.186537in}{1.622449in}}%
\pgfpathlineto{\pgfqpoint{1.188004in}{1.622557in}}%
\pgfpathlineto{\pgfqpoint{1.189464in}{1.622687in}}%
\pgfpathlineto{\pgfqpoint{1.190914in}{1.622838in}}%
\pgfpathlineto{\pgfqpoint{1.192353in}{1.623011in}}%
\pgfpathclose%
\pgfusepath{fill}%
\end{pgfscope}%
\begin{pgfscope}%
\pgfpathrectangle{\pgfqpoint{0.041670in}{0.041670in}}{\pgfqpoint{2.216660in}{2.216660in}}%
\pgfusepath{clip}%
\pgfsetbuttcap%
\pgfsetroundjoin%
\definecolor{currentfill}{rgb}{0.163625,0.471133,0.558148}%
\pgfsetfillcolor{currentfill}%
\pgfsetlinewidth{0.000000pt}%
\definecolor{currentstroke}{rgb}{0.000000,0.000000,0.000000}%
\pgfsetstrokecolor{currentstroke}%
\pgfsetdash{}{0pt}%
\pgfpathmoveto{\pgfqpoint{1.275031in}{1.023543in}}%
\pgfpathlineto{\pgfqpoint{1.276055in}{1.013835in}}%
\pgfpathlineto{\pgfqpoint{1.277078in}{1.004138in}}%
\pgfpathlineto{\pgfqpoint{1.278101in}{0.994456in}}%
\pgfpathlineto{\pgfqpoint{1.279124in}{0.984792in}}%
\pgfpathlineto{\pgfqpoint{1.267631in}{0.983275in}}%
\pgfpathlineto{\pgfqpoint{1.256044in}{0.981945in}}%
\pgfpathlineto{\pgfqpoint{1.244375in}{0.980804in}}%
\pgfpathlineto{\pgfqpoint{1.232636in}{0.979853in}}%
\pgfpathlineto{\pgfqpoint{1.232092in}{0.989575in}}%
\pgfpathlineto{\pgfqpoint{1.231549in}{0.999315in}}%
\pgfpathlineto{\pgfqpoint{1.231005in}{1.009069in}}%
\pgfpathlineto{\pgfqpoint{1.230461in}{1.018834in}}%
\pgfpathlineto{\pgfqpoint{1.241715in}{1.019741in}}%
\pgfpathlineto{\pgfqpoint{1.252902in}{1.020829in}}%
\pgfpathlineto{\pgfqpoint{1.264012in}{1.022096in}}%
\pgfpathlineto{\pgfqpoint{1.275031in}{1.023543in}}%
\pgfpathclose%
\pgfusepath{fill}%
\end{pgfscope}%
\begin{pgfscope}%
\pgfpathrectangle{\pgfqpoint{0.041670in}{0.041670in}}{\pgfqpoint{2.216660in}{2.216660in}}%
\pgfusepath{clip}%
\pgfsetbuttcap%
\pgfsetroundjoin%
\definecolor{currentfill}{rgb}{0.955300,0.901065,0.118128}%
\pgfsetfillcolor{currentfill}%
\pgfsetlinewidth{0.000000pt}%
\definecolor{currentstroke}{rgb}{0.000000,0.000000,0.000000}%
\pgfsetstrokecolor{currentstroke}%
\pgfsetdash{}{0pt}%
\pgfpathmoveto{\pgfqpoint{1.174683in}{1.622371in}}%
\pgfpathlineto{\pgfqpoint{1.174242in}{1.618884in}}%
\pgfpathlineto{\pgfqpoint{1.173802in}{1.615265in}}%
\pgfpathlineto{\pgfqpoint{1.173361in}{1.611513in}}%
\pgfpathlineto{\pgfqpoint{1.172920in}{1.607630in}}%
\pgfpathlineto{\pgfqpoint{1.170954in}{1.607749in}}%
\pgfpathlineto{\pgfqpoint{1.168997in}{1.607897in}}%
\pgfpathlineto{\pgfqpoint{1.167051in}{1.608073in}}%
\pgfpathlineto{\pgfqpoint{1.165118in}{1.608279in}}%
\pgfpathlineto{\pgfqpoint{1.166048in}{1.612121in}}%
\pgfpathlineto{\pgfqpoint{1.166977in}{1.615832in}}%
\pgfpathlineto{\pgfqpoint{1.167906in}{1.619410in}}%
\pgfpathlineto{\pgfqpoint{1.168835in}{1.622856in}}%
\pgfpathlineto{\pgfqpoint{1.170284in}{1.622703in}}%
\pgfpathlineto{\pgfqpoint{1.171743in}{1.622571in}}%
\pgfpathlineto{\pgfqpoint{1.173209in}{1.622460in}}%
\pgfpathlineto{\pgfqpoint{1.174683in}{1.622371in}}%
\pgfpathclose%
\pgfusepath{fill}%
\end{pgfscope}%
\begin{pgfscope}%
\pgfpathrectangle{\pgfqpoint{0.041670in}{0.041670in}}{\pgfqpoint{2.216660in}{2.216660in}}%
\pgfusepath{clip}%
\pgfsetbuttcap%
\pgfsetroundjoin%
\definecolor{currentfill}{rgb}{0.120081,0.622161,0.534946}%
\pgfsetfillcolor{currentfill}%
\pgfsetlinewidth{0.000000pt}%
\definecolor{currentstroke}{rgb}{0.000000,0.000000,0.000000}%
\pgfsetstrokecolor{currentstroke}%
\pgfsetdash{}{0pt}%
\pgfpathmoveto{\pgfqpoint{1.073488in}{1.182649in}}%
\pgfpathlineto{\pgfqpoint{1.072098in}{1.173208in}}%
\pgfpathlineto{\pgfqpoint{1.070709in}{1.163735in}}%
\pgfpathlineto{\pgfqpoint{1.069320in}{1.154231in}}%
\pgfpathlineto{\pgfqpoint{1.067931in}{1.144700in}}%
\pgfpathlineto{\pgfqpoint{1.058761in}{1.146550in}}%
\pgfpathlineto{\pgfqpoint{1.049718in}{1.148544in}}%
\pgfpathlineto{\pgfqpoint{1.040814in}{1.150680in}}%
\pgfpathlineto{\pgfqpoint{1.032056in}{1.152955in}}%
\pgfpathlineto{\pgfqpoint{1.033887in}{1.162373in}}%
\pgfpathlineto{\pgfqpoint{1.035719in}{1.171764in}}%
\pgfpathlineto{\pgfqpoint{1.037552in}{1.181124in}}%
\pgfpathlineto{\pgfqpoint{1.039385in}{1.190453in}}%
\pgfpathlineto{\pgfqpoint{1.047710in}{1.188302in}}%
\pgfpathlineto{\pgfqpoint{1.056175in}{1.186283in}}%
\pgfpathlineto{\pgfqpoint{1.064771in}{1.184398in}}%
\pgfpathlineto{\pgfqpoint{1.073488in}{1.182649in}}%
\pgfpathclose%
\pgfusepath{fill}%
\end{pgfscope}%
\begin{pgfscope}%
\pgfpathrectangle{\pgfqpoint{0.041670in}{0.041670in}}{\pgfqpoint{2.216660in}{2.216660in}}%
\pgfusepath{clip}%
\pgfsetbuttcap%
\pgfsetroundjoin%
\definecolor{currentfill}{rgb}{0.268510,0.009605,0.335427}%
\pgfsetfillcolor{currentfill}%
\pgfsetlinewidth{0.000000pt}%
\definecolor{currentstroke}{rgb}{0.000000,0.000000,0.000000}%
\pgfsetstrokecolor{currentstroke}%
\pgfsetdash{}{0pt}%
\pgfpathmoveto{\pgfqpoint{1.490808in}{0.594464in}}%
\pgfpathlineto{\pgfqpoint{1.492848in}{0.593963in}}%
\pgfpathlineto{\pgfqpoint{1.494895in}{0.593744in}}%
\pgfpathlineto{\pgfqpoint{1.496947in}{0.593812in}}%
\pgfpathlineto{\pgfqpoint{1.499005in}{0.594172in}}%
\pgfpathlineto{\pgfqpoint{1.481436in}{0.588857in}}%
\pgfpathlineto{\pgfqpoint{1.463530in}{0.583842in}}%
\pgfpathlineto{\pgfqpoint{1.445306in}{0.579134in}}%
\pgfpathlineto{\pgfqpoint{1.426784in}{0.574738in}}%
\pgfpathlineto{\pgfqpoint{1.425188in}{0.574503in}}%
\pgfpathlineto{\pgfqpoint{1.423596in}{0.574560in}}%
\pgfpathlineto{\pgfqpoint{1.422009in}{0.574904in}}%
\pgfpathlineto{\pgfqpoint{1.420427in}{0.575531in}}%
\pgfpathlineto{\pgfqpoint{1.438475in}{0.579813in}}%
\pgfpathlineto{\pgfqpoint{1.456235in}{0.584400in}}%
\pgfpathlineto{\pgfqpoint{1.473685in}{0.589286in}}%
\pgfpathlineto{\pgfqpoint{1.490808in}{0.594464in}}%
\pgfpathclose%
\pgfusepath{fill}%
\end{pgfscope}%
\begin{pgfscope}%
\pgfpathrectangle{\pgfqpoint{0.041670in}{0.041670in}}{\pgfqpoint{2.216660in}{2.216660in}}%
\pgfusepath{clip}%
\pgfsetbuttcap%
\pgfsetroundjoin%
\definecolor{currentfill}{rgb}{0.935904,0.898570,0.108131}%
\pgfsetfillcolor{currentfill}%
\pgfsetlinewidth{0.000000pt}%
\definecolor{currentstroke}{rgb}{0.000000,0.000000,0.000000}%
\pgfsetstrokecolor{currentstroke}%
\pgfsetdash{}{0pt}%
\pgfpathmoveto{\pgfqpoint{1.150353in}{1.610925in}}%
\pgfpathlineto{\pgfqpoint{1.148499in}{1.607119in}}%
\pgfpathlineto{\pgfqpoint{1.146644in}{1.603184in}}%
\pgfpathlineto{\pgfqpoint{1.144788in}{1.599118in}}%
\pgfpathlineto{\pgfqpoint{1.142933in}{1.594924in}}%
\pgfpathlineto{\pgfqpoint{1.140769in}{1.595490in}}%
\pgfpathlineto{\pgfqpoint{1.138646in}{1.596087in}}%
\pgfpathlineto{\pgfqpoint{1.136564in}{1.596715in}}%
\pgfpathlineto{\pgfqpoint{1.134526in}{1.597374in}}%
\pgfpathlineto{\pgfqpoint{1.136802in}{1.601444in}}%
\pgfpathlineto{\pgfqpoint{1.139078in}{1.605385in}}%
\pgfpathlineto{\pgfqpoint{1.141354in}{1.609197in}}%
\pgfpathlineto{\pgfqpoint{1.143629in}{1.612879in}}%
\pgfpathlineto{\pgfqpoint{1.145259in}{1.612354in}}%
\pgfpathlineto{\pgfqpoint{1.146925in}{1.611852in}}%
\pgfpathlineto{\pgfqpoint{1.148623in}{1.611376in}}%
\pgfpathlineto{\pgfqpoint{1.150353in}{1.610925in}}%
\pgfpathclose%
\pgfusepath{fill}%
\end{pgfscope}%
\begin{pgfscope}%
\pgfpathrectangle{\pgfqpoint{0.041670in}{0.041670in}}{\pgfqpoint{2.216660in}{2.216660in}}%
\pgfusepath{clip}%
\pgfsetbuttcap%
\pgfsetroundjoin%
\definecolor{currentfill}{rgb}{0.896320,0.893616,0.096335}%
\pgfsetfillcolor{currentfill}%
\pgfsetlinewidth{0.000000pt}%
\definecolor{currentstroke}{rgb}{0.000000,0.000000,0.000000}%
\pgfsetstrokecolor{currentstroke}%
\pgfsetdash{}{0pt}%
\pgfpathmoveto{\pgfqpoint{1.134526in}{1.597374in}}%
\pgfpathlineto{\pgfqpoint{1.132250in}{1.593176in}}%
\pgfpathlineto{\pgfqpoint{1.129973in}{1.588851in}}%
\pgfpathlineto{\pgfqpoint{1.127696in}{1.584399in}}%
\pgfpathlineto{\pgfqpoint{1.125419in}{1.579822in}}%
\pgfpathlineto{\pgfqpoint{1.123029in}{1.580651in}}%
\pgfpathlineto{\pgfqpoint{1.120697in}{1.581515in}}%
\pgfpathlineto{\pgfqpoint{1.118424in}{1.582413in}}%
\pgfpathlineto{\pgfqpoint{1.116214in}{1.583344in}}%
\pgfpathlineto{\pgfqpoint{1.118874in}{1.587772in}}%
\pgfpathlineto{\pgfqpoint{1.121534in}{1.592075in}}%
\pgfpathlineto{\pgfqpoint{1.124193in}{1.596252in}}%
\pgfpathlineto{\pgfqpoint{1.126853in}{1.600301in}}%
\pgfpathlineto{\pgfqpoint{1.128695in}{1.599527in}}%
\pgfpathlineto{\pgfqpoint{1.130590in}{1.598781in}}%
\pgfpathlineto{\pgfqpoint{1.132534in}{1.598063in}}%
\pgfpathlineto{\pgfqpoint{1.134526in}{1.597374in}}%
\pgfpathclose%
\pgfusepath{fill}%
\end{pgfscope}%
\begin{pgfscope}%
\pgfpathrectangle{\pgfqpoint{0.041670in}{0.041670in}}{\pgfqpoint{2.216660in}{2.216660in}}%
\pgfusepath{clip}%
\pgfsetbuttcap%
\pgfsetroundjoin%
\definecolor{currentfill}{rgb}{0.955300,0.901065,0.118128}%
\pgfsetfillcolor{currentfill}%
\pgfsetlinewidth{0.000000pt}%
\definecolor{currentstroke}{rgb}{0.000000,0.000000,0.000000}%
\pgfsetstrokecolor{currentstroke}%
\pgfsetdash{}{0pt}%
\pgfpathmoveto{\pgfqpoint{1.186537in}{1.622449in}}%
\pgfpathlineto{\pgfqpoint{1.187087in}{1.618969in}}%
\pgfpathlineto{\pgfqpoint{1.187637in}{1.615356in}}%
\pgfpathlineto{\pgfqpoint{1.188187in}{1.611611in}}%
\pgfpathlineto{\pgfqpoint{1.188738in}{1.607734in}}%
\pgfpathlineto{\pgfqpoint{1.186771in}{1.607619in}}%
\pgfpathlineto{\pgfqpoint{1.184797in}{1.607532in}}%
\pgfpathlineto{\pgfqpoint{1.182818in}{1.607475in}}%
\pgfpathlineto{\pgfqpoint{1.180836in}{1.607448in}}%
\pgfpathlineto{\pgfqpoint{1.180781in}{1.611342in}}%
\pgfpathlineto{\pgfqpoint{1.180726in}{1.615105in}}%
\pgfpathlineto{\pgfqpoint{1.180670in}{1.618736in}}%
\pgfpathlineto{\pgfqpoint{1.180615in}{1.622235in}}%
\pgfpathlineto{\pgfqpoint{1.182100in}{1.622256in}}%
\pgfpathlineto{\pgfqpoint{1.183583in}{1.622298in}}%
\pgfpathlineto{\pgfqpoint{1.185063in}{1.622363in}}%
\pgfpathlineto{\pgfqpoint{1.186537in}{1.622449in}}%
\pgfpathclose%
\pgfusepath{fill}%
\end{pgfscope}%
\begin{pgfscope}%
\pgfpathrectangle{\pgfqpoint{0.041670in}{0.041670in}}{\pgfqpoint{2.216660in}{2.216660in}}%
\pgfusepath{clip}%
\pgfsetbuttcap%
\pgfsetroundjoin%
\definecolor{currentfill}{rgb}{0.955300,0.901065,0.118128}%
\pgfsetfillcolor{currentfill}%
\pgfsetlinewidth{0.000000pt}%
\definecolor{currentstroke}{rgb}{0.000000,0.000000,0.000000}%
\pgfsetstrokecolor{currentstroke}%
\pgfsetdash{}{0pt}%
\pgfpathmoveto{\pgfqpoint{1.180615in}{1.622235in}}%
\pgfpathlineto{\pgfqpoint{1.180670in}{1.618736in}}%
\pgfpathlineto{\pgfqpoint{1.180726in}{1.615105in}}%
\pgfpathlineto{\pgfqpoint{1.180781in}{1.611342in}}%
\pgfpathlineto{\pgfqpoint{1.180836in}{1.607448in}}%
\pgfpathlineto{\pgfqpoint{1.178853in}{1.607449in}}%
\pgfpathlineto{\pgfqpoint{1.176872in}{1.607480in}}%
\pgfpathlineto{\pgfqpoint{1.174893in}{1.607541in}}%
\pgfpathlineto{\pgfqpoint{1.172920in}{1.607630in}}%
\pgfpathlineto{\pgfqpoint{1.173361in}{1.611513in}}%
\pgfpathlineto{\pgfqpoint{1.173802in}{1.615265in}}%
\pgfpathlineto{\pgfqpoint{1.174242in}{1.618884in}}%
\pgfpathlineto{\pgfqpoint{1.174683in}{1.622371in}}%
\pgfpathlineto{\pgfqpoint{1.176162in}{1.622304in}}%
\pgfpathlineto{\pgfqpoint{1.177644in}{1.622259in}}%
\pgfpathlineto{\pgfqpoint{1.179129in}{1.622236in}}%
\pgfpathlineto{\pgfqpoint{1.180615in}{1.622235in}}%
\pgfpathclose%
\pgfusepath{fill}%
\end{pgfscope}%
\begin{pgfscope}%
\pgfpathrectangle{\pgfqpoint{0.041670in}{0.041670in}}{\pgfqpoint{2.216660in}{2.216660in}}%
\pgfusepath{clip}%
\pgfsetbuttcap%
\pgfsetroundjoin%
\definecolor{currentfill}{rgb}{0.179019,0.433756,0.557430}%
\pgfsetfillcolor{currentfill}%
\pgfsetlinewidth{0.000000pt}%
\definecolor{currentstroke}{rgb}{0.000000,0.000000,0.000000}%
\pgfsetstrokecolor{currentstroke}%
\pgfsetdash{}{0pt}%
\pgfpathmoveto{\pgfqpoint{1.232636in}{0.979853in}}%
\pgfpathlineto{\pgfqpoint{1.233180in}{0.970152in}}%
\pgfpathlineto{\pgfqpoint{1.233724in}{0.960474in}}%
\pgfpathlineto{\pgfqpoint{1.234267in}{0.950823in}}%
\pgfpathlineto{\pgfqpoint{1.234811in}{0.941202in}}%
\pgfpathlineto{\pgfqpoint{1.222529in}{0.940406in}}%
\pgfpathlineto{\pgfqpoint{1.210201in}{0.939811in}}%
\pgfpathlineto{\pgfqpoint{1.197840in}{0.939418in}}%
\pgfpathlineto{\pgfqpoint{1.185460in}{0.939227in}}%
\pgfpathlineto{\pgfqpoint{1.185405in}{0.948870in}}%
\pgfpathlineto{\pgfqpoint{1.185351in}{0.958544in}}%
\pgfpathlineto{\pgfqpoint{1.185296in}{0.968244in}}%
\pgfpathlineto{\pgfqpoint{1.185242in}{0.977967in}}%
\pgfpathlineto{\pgfqpoint{1.197131in}{0.978150in}}%
\pgfpathlineto{\pgfqpoint{1.209001in}{0.978525in}}%
\pgfpathlineto{\pgfqpoint{1.220841in}{0.979093in}}%
\pgfpathlineto{\pgfqpoint{1.232636in}{0.979853in}}%
\pgfpathclose%
\pgfusepath{fill}%
\end{pgfscope}%
\begin{pgfscope}%
\pgfpathrectangle{\pgfqpoint{0.041670in}{0.041670in}}{\pgfqpoint{2.216660in}{2.216660in}}%
\pgfusepath{clip}%
\pgfsetbuttcap%
\pgfsetroundjoin%
\definecolor{currentfill}{rgb}{0.220124,0.725509,0.466226}%
\pgfsetfillcolor{currentfill}%
\pgfsetlinewidth{0.000000pt}%
\definecolor{currentstroke}{rgb}{0.000000,0.000000,0.000000}%
\pgfsetstrokecolor{currentstroke}%
\pgfsetdash{}{0pt}%
\pgfpathmoveto{\pgfqpoint{1.061435in}{1.298963in}}%
\pgfpathlineto{\pgfqpoint{1.059594in}{1.290214in}}%
\pgfpathlineto{\pgfqpoint{1.057753in}{1.281405in}}%
\pgfpathlineto{\pgfqpoint{1.055913in}{1.272537in}}%
\pgfpathlineto{\pgfqpoint{1.054074in}{1.263612in}}%
\pgfpathlineto{\pgfqpoint{1.046746in}{1.265632in}}%
\pgfpathlineto{\pgfqpoint{1.039558in}{1.267765in}}%
\pgfpathlineto{\pgfqpoint{1.032517in}{1.270008in}}%
\pgfpathlineto{\pgfqpoint{1.025629in}{1.272358in}}%
\pgfpathlineto{\pgfqpoint{1.027881in}{1.281145in}}%
\pgfpathlineto{\pgfqpoint{1.030134in}{1.289874in}}%
\pgfpathlineto{\pgfqpoint{1.032388in}{1.298546in}}%
\pgfpathlineto{\pgfqpoint{1.034643in}{1.307156in}}%
\pgfpathlineto{\pgfqpoint{1.041131in}{1.304954in}}%
\pgfpathlineto{\pgfqpoint{1.047763in}{1.302853in}}%
\pgfpathlineto{\pgfqpoint{1.054534in}{1.300855in}}%
\pgfpathlineto{\pgfqpoint{1.061435in}{1.298963in}}%
\pgfpathclose%
\pgfusepath{fill}%
\end{pgfscope}%
\begin{pgfscope}%
\pgfpathrectangle{\pgfqpoint{0.041670in}{0.041670in}}{\pgfqpoint{2.216660in}{2.216660in}}%
\pgfusepath{clip}%
\pgfsetbuttcap%
\pgfsetroundjoin%
\definecolor{currentfill}{rgb}{0.163625,0.471133,0.558148}%
\pgfsetfillcolor{currentfill}%
\pgfsetlinewidth{0.000000pt}%
\definecolor{currentstroke}{rgb}{0.000000,0.000000,0.000000}%
\pgfsetstrokecolor{currentstroke}%
\pgfsetdash{}{0pt}%
\pgfpathmoveto{\pgfqpoint{1.139499in}{1.018181in}}%
\pgfpathlineto{\pgfqpoint{1.139063in}{1.008407in}}%
\pgfpathlineto{\pgfqpoint{1.138627in}{0.998646in}}%
\pgfpathlineto{\pgfqpoint{1.138191in}{0.988898in}}%
\pgfpathlineto{\pgfqpoint{1.137756in}{0.979168in}}%
\pgfpathlineto{\pgfqpoint{1.125966in}{0.979950in}}%
\pgfpathlineto{\pgfqpoint{1.114235in}{0.980922in}}%
\pgfpathlineto{\pgfqpoint{1.102574in}{0.982084in}}%
\pgfpathlineto{\pgfqpoint{1.090997in}{0.983434in}}%
\pgfpathlineto{\pgfqpoint{1.091914in}{0.993115in}}%
\pgfpathlineto{\pgfqpoint{1.092832in}{1.002812in}}%
\pgfpathlineto{\pgfqpoint{1.093750in}{1.012525in}}%
\pgfpathlineto{\pgfqpoint{1.094669in}{1.022248in}}%
\pgfpathlineto{\pgfqpoint{1.105769in}{1.020961in}}%
\pgfpathlineto{\pgfqpoint{1.116948in}{1.019853in}}%
\pgfpathlineto{\pgfqpoint{1.128196in}{1.018926in}}%
\pgfpathlineto{\pgfqpoint{1.139499in}{1.018181in}}%
\pgfpathclose%
\pgfusepath{fill}%
\end{pgfscope}%
\begin{pgfscope}%
\pgfpathrectangle{\pgfqpoint{0.041670in}{0.041670in}}{\pgfqpoint{2.216660in}{2.216660in}}%
\pgfusepath{clip}%
\pgfsetbuttcap%
\pgfsetroundjoin%
\definecolor{currentfill}{rgb}{0.179019,0.433756,0.557430}%
\pgfsetfillcolor{currentfill}%
\pgfsetlinewidth{0.000000pt}%
\definecolor{currentstroke}{rgb}{0.000000,0.000000,0.000000}%
\pgfsetstrokecolor{currentstroke}%
\pgfsetdash{}{0pt}%
\pgfpathmoveto{\pgfqpoint{1.185242in}{0.977967in}}%
\pgfpathlineto{\pgfqpoint{1.185296in}{0.968244in}}%
\pgfpathlineto{\pgfqpoint{1.185351in}{0.958544in}}%
\pgfpathlineto{\pgfqpoint{1.185405in}{0.948870in}}%
\pgfpathlineto{\pgfqpoint{1.185460in}{0.939227in}}%
\pgfpathlineto{\pgfqpoint{1.173074in}{0.939238in}}%
\pgfpathlineto{\pgfqpoint{1.160695in}{0.939452in}}%
\pgfpathlineto{\pgfqpoint{1.148337in}{0.939867in}}%
\pgfpathlineto{\pgfqpoint{1.136014in}{0.940484in}}%
\pgfpathlineto{\pgfqpoint{1.136449in}{0.950114in}}%
\pgfpathlineto{\pgfqpoint{1.136885in}{0.959773in}}%
\pgfpathlineto{\pgfqpoint{1.137320in}{0.969459in}}%
\pgfpathlineto{\pgfqpoint{1.137756in}{0.979168in}}%
\pgfpathlineto{\pgfqpoint{1.149591in}{0.978579in}}%
\pgfpathlineto{\pgfqpoint{1.161459in}{0.978182in}}%
\pgfpathlineto{\pgfqpoint{1.173347in}{0.977978in}}%
\pgfpathlineto{\pgfqpoint{1.185242in}{0.977967in}}%
\pgfpathclose%
\pgfusepath{fill}%
\end{pgfscope}%
\begin{pgfscope}%
\pgfpathrectangle{\pgfqpoint{0.041670in}{0.041670in}}{\pgfqpoint{2.216660in}{2.216660in}}%
\pgfusepath{clip}%
\pgfsetbuttcap%
\pgfsetroundjoin%
\definecolor{currentfill}{rgb}{0.133743,0.548535,0.553541}%
\pgfsetfillcolor{currentfill}%
\pgfsetlinewidth{0.000000pt}%
\definecolor{currentstroke}{rgb}{0.000000,0.000000,0.000000}%
\pgfsetstrokecolor{currentstroke}%
\pgfsetdash{}{0pt}%
\pgfpathmoveto{\pgfqpoint{1.102021in}{1.100103in}}%
\pgfpathlineto{\pgfqpoint{1.101101in}{1.090392in}}%
\pgfpathlineto{\pgfqpoint{1.100182in}{1.080669in}}%
\pgfpathlineto{\pgfqpoint{1.099262in}{1.070937in}}%
\pgfpathlineto{\pgfqpoint{1.098343in}{1.061200in}}%
\pgfpathlineto{\pgfqpoint{1.087808in}{1.062595in}}%
\pgfpathlineto{\pgfqpoint{1.077372in}{1.064158in}}%
\pgfpathlineto{\pgfqpoint{1.067044in}{1.065887in}}%
\pgfpathlineto{\pgfqpoint{1.056838in}{1.067781in}}%
\pgfpathlineto{\pgfqpoint{1.058223in}{1.077436in}}%
\pgfpathlineto{\pgfqpoint{1.059609in}{1.087084in}}%
\pgfpathlineto{\pgfqpoint{1.060995in}{1.096724in}}%
\pgfpathlineto{\pgfqpoint{1.062381in}{1.106353in}}%
\pgfpathlineto{\pgfqpoint{1.072130in}{1.104555in}}%
\pgfpathlineto{\pgfqpoint{1.081993in}{1.102912in}}%
\pgfpathlineto{\pgfqpoint{1.091960in}{1.101428in}}%
\pgfpathlineto{\pgfqpoint{1.102021in}{1.100103in}}%
\pgfpathclose%
\pgfusepath{fill}%
\end{pgfscope}%
\begin{pgfscope}%
\pgfpathrectangle{\pgfqpoint{0.041670in}{0.041670in}}{\pgfqpoint{2.216660in}{2.216660in}}%
\pgfusepath{clip}%
\pgfsetbuttcap%
\pgfsetroundjoin%
\definecolor{currentfill}{rgb}{0.134692,0.658636,0.517649}%
\pgfsetfillcolor{currentfill}%
\pgfsetlinewidth{0.000000pt}%
\definecolor{currentstroke}{rgb}{0.000000,0.000000,0.000000}%
\pgfsetstrokecolor{currentstroke}%
\pgfsetdash{}{0pt}%
\pgfpathmoveto{\pgfqpoint{1.320085in}{1.229296in}}%
\pgfpathlineto{\pgfqpoint{1.322015in}{1.220151in}}%
\pgfpathlineto{\pgfqpoint{1.323945in}{1.210964in}}%
\pgfpathlineto{\pgfqpoint{1.325873in}{1.201737in}}%
\pgfpathlineto{\pgfqpoint{1.327802in}{1.192473in}}%
\pgfpathlineto{\pgfqpoint{1.319607in}{1.190207in}}%
\pgfpathlineto{\pgfqpoint{1.311266in}{1.188071in}}%
\pgfpathlineto{\pgfqpoint{1.302786in}{1.186067in}}%
\pgfpathlineto{\pgfqpoint{1.294176in}{1.184197in}}%
\pgfpathlineto{\pgfqpoint{1.292685in}{1.193580in}}%
\pgfpathlineto{\pgfqpoint{1.291193in}{1.202926in}}%
\pgfpathlineto{\pgfqpoint{1.289700in}{1.212231in}}%
\pgfpathlineto{\pgfqpoint{1.288208in}{1.221494in}}%
\pgfpathlineto{\pgfqpoint{1.296369in}{1.223257in}}%
\pgfpathlineto{\pgfqpoint{1.304408in}{1.225146in}}%
\pgfpathlineto{\pgfqpoint{1.312316in}{1.227160in}}%
\pgfpathlineto{\pgfqpoint{1.320085in}{1.229296in}}%
\pgfpathclose%
\pgfusepath{fill}%
\end{pgfscope}%
\begin{pgfscope}%
\pgfpathrectangle{\pgfqpoint{0.041670in}{0.041670in}}{\pgfqpoint{2.216660in}{2.216660in}}%
\pgfusepath{clip}%
\pgfsetbuttcap%
\pgfsetroundjoin%
\definecolor{currentfill}{rgb}{0.855810,0.888601,0.097452}%
\pgfsetfillcolor{currentfill}%
\pgfsetlinewidth{0.000000pt}%
\definecolor{currentstroke}{rgb}{0.000000,0.000000,0.000000}%
\pgfsetstrokecolor{currentstroke}%
\pgfsetdash{}{0pt}%
\pgfpathmoveto{\pgfqpoint{1.245606in}{1.584199in}}%
\pgfpathlineto{\pgfqpoint{1.248345in}{1.579683in}}%
\pgfpathlineto{\pgfqpoint{1.251084in}{1.575043in}}%
\pgfpathlineto{\pgfqpoint{1.253823in}{1.570280in}}%
\pgfpathlineto{\pgfqpoint{1.256561in}{1.565395in}}%
\pgfpathlineto{\pgfqpoint{1.254051in}{1.564271in}}%
\pgfpathlineto{\pgfqpoint{1.251465in}{1.563186in}}%
\pgfpathlineto{\pgfqpoint{1.248807in}{1.562139in}}%
\pgfpathlineto{\pgfqpoint{1.246079in}{1.561132in}}%
\pgfpathlineto{\pgfqpoint{1.243714in}{1.566173in}}%
\pgfpathlineto{\pgfqpoint{1.241348in}{1.571091in}}%
\pgfpathlineto{\pgfqpoint{1.238983in}{1.575886in}}%
\pgfpathlineto{\pgfqpoint{1.236618in}{1.580557in}}%
\pgfpathlineto{\pgfqpoint{1.238957in}{1.581417in}}%
\pgfpathlineto{\pgfqpoint{1.241236in}{1.582312in}}%
\pgfpathlineto{\pgfqpoint{1.243453in}{1.583239in}}%
\pgfpathlineto{\pgfqpoint{1.245606in}{1.584199in}}%
\pgfpathclose%
\pgfusepath{fill}%
\end{pgfscope}%
\begin{pgfscope}%
\pgfpathrectangle{\pgfqpoint{0.041670in}{0.041670in}}{\pgfqpoint{2.216660in}{2.216660in}}%
\pgfusepath{clip}%
\pgfsetbuttcap%
\pgfsetroundjoin%
\definecolor{currentfill}{rgb}{0.281477,0.755203,0.432552}%
\pgfsetfillcolor{currentfill}%
\pgfsetlinewidth{0.000000pt}%
\definecolor{currentstroke}{rgb}{0.000000,0.000000,0.000000}%
\pgfsetstrokecolor{currentstroke}%
\pgfsetdash{}{0pt}%
\pgfpathmoveto{\pgfqpoint{1.321531in}{1.342850in}}%
\pgfpathlineto{\pgfqpoint{1.323876in}{1.334538in}}%
\pgfpathlineto{\pgfqpoint{1.326221in}{1.326158in}}%
\pgfpathlineto{\pgfqpoint{1.328564in}{1.317710in}}%
\pgfpathlineto{\pgfqpoint{1.330907in}{1.309196in}}%
\pgfpathlineto{\pgfqpoint{1.324553in}{1.306906in}}%
\pgfpathlineto{\pgfqpoint{1.318049in}{1.304715in}}%
\pgfpathlineto{\pgfqpoint{1.311401in}{1.302626in}}%
\pgfpathlineto{\pgfqpoint{1.304615in}{1.300640in}}%
\pgfpathlineto{\pgfqpoint{1.302678in}{1.309296in}}%
\pgfpathlineto{\pgfqpoint{1.300741in}{1.317887in}}%
\pgfpathlineto{\pgfqpoint{1.298802in}{1.326411in}}%
\pgfpathlineto{\pgfqpoint{1.296863in}{1.334865in}}%
\pgfpathlineto{\pgfqpoint{1.303229in}{1.336718in}}%
\pgfpathlineto{\pgfqpoint{1.309466in}{1.338668in}}%
\pgfpathlineto{\pgfqpoint{1.315569in}{1.340713in}}%
\pgfpathlineto{\pgfqpoint{1.321531in}{1.342850in}}%
\pgfpathclose%
\pgfusepath{fill}%
\end{pgfscope}%
\begin{pgfscope}%
\pgfpathrectangle{\pgfqpoint{0.041670in}{0.041670in}}{\pgfqpoint{2.216660in}{2.216660in}}%
\pgfusepath{clip}%
\pgfsetbuttcap%
\pgfsetroundjoin%
\definecolor{currentfill}{rgb}{0.267004,0.004874,0.329415}%
\pgfsetfillcolor{currentfill}%
\pgfsetlinewidth{0.000000pt}%
\definecolor{currentstroke}{rgb}{0.000000,0.000000,0.000000}%
\pgfsetstrokecolor{currentstroke}%
\pgfsetdash{}{0pt}%
\pgfpathmoveto{\pgfqpoint{1.038979in}{0.574577in}}%
\pgfpathlineto{\pgfqpoint{1.038017in}{0.571768in}}%
\pgfpathlineto{\pgfqpoint{1.037054in}{0.569200in}}%
\pgfpathlineto{\pgfqpoint{1.036088in}{0.566879in}}%
\pgfpathlineto{\pgfqpoint{1.035120in}{0.564809in}}%
\pgfpathlineto{\pgfqpoint{1.016448in}{0.567459in}}%
\pgfpathlineto{\pgfqpoint{0.997961in}{0.570428in}}%
\pgfpathlineto{\pgfqpoint{0.979677in}{0.573711in}}%
\pgfpathlineto{\pgfqpoint{0.961619in}{0.577304in}}%
\pgfpathlineto{\pgfqpoint{0.963076in}{0.579287in}}%
\pgfpathlineto{\pgfqpoint{0.964529in}{0.581521in}}%
\pgfpathlineto{\pgfqpoint{0.965980in}{0.584002in}}%
\pgfpathlineto{\pgfqpoint{0.967427in}{0.586723in}}%
\pgfpathlineto{\pgfqpoint{0.985008in}{0.583230in}}%
\pgfpathlineto{\pgfqpoint{1.002806in}{0.580039in}}%
\pgfpathlineto{\pgfqpoint{1.020803in}{0.577153in}}%
\pgfpathlineto{\pgfqpoint{1.038979in}{0.574577in}}%
\pgfpathclose%
\pgfusepath{fill}%
\end{pgfscope}%
\begin{pgfscope}%
\pgfpathrectangle{\pgfqpoint{0.041670in}{0.041670in}}{\pgfqpoint{2.216660in}{2.216660in}}%
\pgfusepath{clip}%
\pgfsetbuttcap%
\pgfsetroundjoin%
\definecolor{currentfill}{rgb}{0.565498,0.842430,0.262877}%
\pgfsetfillcolor{currentfill}%
\pgfsetlinewidth{0.000000pt}%
\definecolor{currentstroke}{rgb}{0.000000,0.000000,0.000000}%
\pgfsetstrokecolor{currentstroke}%
\pgfsetdash{}{0pt}%
\pgfpathmoveto{\pgfqpoint{1.300315in}{1.471579in}}%
\pgfpathlineto{\pgfqpoint{1.303043in}{1.464802in}}%
\pgfpathlineto{\pgfqpoint{1.305771in}{1.457927in}}%
\pgfpathlineto{\pgfqpoint{1.308498in}{1.450955in}}%
\pgfpathlineto{\pgfqpoint{1.311224in}{1.443887in}}%
\pgfpathlineto{\pgfqpoint{1.306938in}{1.441926in}}%
\pgfpathlineto{\pgfqpoint{1.302521in}{1.440030in}}%
\pgfpathlineto{\pgfqpoint{1.297979in}{1.438202in}}%
\pgfpathlineto{\pgfqpoint{1.293315in}{1.436443in}}%
\pgfpathlineto{\pgfqpoint{1.290958in}{1.443673in}}%
\pgfpathlineto{\pgfqpoint{1.288601in}{1.450808in}}%
\pgfpathlineto{\pgfqpoint{1.286243in}{1.457845in}}%
\pgfpathlineto{\pgfqpoint{1.283884in}{1.464783in}}%
\pgfpathlineto{\pgfqpoint{1.288162in}{1.466388in}}%
\pgfpathlineto{\pgfqpoint{1.292330in}{1.468057in}}%
\pgfpathlineto{\pgfqpoint{1.296382in}{1.469788in}}%
\pgfpathlineto{\pgfqpoint{1.300315in}{1.471579in}}%
\pgfpathclose%
\pgfusepath{fill}%
\end{pgfscope}%
\begin{pgfscope}%
\pgfpathrectangle{\pgfqpoint{0.041670in}{0.041670in}}{\pgfqpoint{2.216660in}{2.216660in}}%
\pgfusepath{clip}%
\pgfsetbuttcap%
\pgfsetroundjoin%
\definecolor{currentfill}{rgb}{0.487026,0.823929,0.312321}%
\pgfsetfillcolor{currentfill}%
\pgfsetlinewidth{0.000000pt}%
\definecolor{currentstroke}{rgb}{0.000000,0.000000,0.000000}%
\pgfsetstrokecolor{currentstroke}%
\pgfsetdash{}{0pt}%
\pgfpathmoveto{\pgfqpoint{1.070839in}{1.434940in}}%
\pgfpathlineto{\pgfqpoint{1.068571in}{1.427583in}}%
\pgfpathlineto{\pgfqpoint{1.066303in}{1.420133in}}%
\pgfpathlineto{\pgfqpoint{1.064036in}{1.412592in}}%
\pgfpathlineto{\pgfqpoint{1.061770in}{1.404963in}}%
\pgfpathlineto{\pgfqpoint{1.056609in}{1.406807in}}%
\pgfpathlineto{\pgfqpoint{1.051575in}{1.408729in}}%
\pgfpathlineto{\pgfqpoint{1.046673in}{1.410726in}}%
\pgfpathlineto{\pgfqpoint{1.041908in}{1.412796in}}%
\pgfpathlineto{\pgfqpoint{1.044552in}{1.420267in}}%
\pgfpathlineto{\pgfqpoint{1.047197in}{1.427649in}}%
\pgfpathlineto{\pgfqpoint{1.049843in}{1.434941in}}%
\pgfpathlineto{\pgfqpoint{1.052489in}{1.442140in}}%
\pgfpathlineto{\pgfqpoint{1.056892in}{1.440237in}}%
\pgfpathlineto{\pgfqpoint{1.061420in}{1.438402in}}%
\pgfpathlineto{\pgfqpoint{1.066071in}{1.436635in}}%
\pgfpathlineto{\pgfqpoint{1.070839in}{1.434940in}}%
\pgfpathclose%
\pgfusepath{fill}%
\end{pgfscope}%
\begin{pgfscope}%
\pgfpathrectangle{\pgfqpoint{0.041670in}{0.041670in}}{\pgfqpoint{2.216660in}{2.216660in}}%
\pgfusepath{clip}%
\pgfsetbuttcap%
\pgfsetroundjoin%
\definecolor{currentfill}{rgb}{0.277941,0.056324,0.381191}%
\pgfsetfillcolor{currentfill}%
\pgfsetlinewidth{0.000000pt}%
\definecolor{currentstroke}{rgb}{0.000000,0.000000,0.000000}%
\pgfsetstrokecolor{currentstroke}%
\pgfsetdash{}{0pt}%
\pgfpathmoveto{\pgfqpoint{1.575530in}{0.623392in}}%
\pgfpathlineto{\pgfqpoint{1.578048in}{0.625450in}}%
\pgfpathlineto{\pgfqpoint{1.580575in}{0.627835in}}%
\pgfpathlineto{\pgfqpoint{1.583111in}{0.630552in}}%
\pgfpathlineto{\pgfqpoint{1.585656in}{0.633607in}}%
\pgfpathlineto{\pgfqpoint{1.568801in}{0.626836in}}%
\pgfpathlineto{\pgfqpoint{1.551512in}{0.620350in}}%
\pgfpathlineto{\pgfqpoint{1.533809in}{0.614158in}}%
\pgfpathlineto{\pgfqpoint{1.515709in}{0.608266in}}%
\pgfpathlineto{\pgfqpoint{1.513596in}{0.605361in}}%
\pgfpathlineto{\pgfqpoint{1.511490in}{0.602794in}}%
\pgfpathlineto{\pgfqpoint{1.509392in}{0.600560in}}%
\pgfpathlineto{\pgfqpoint{1.507301in}{0.598652in}}%
\pgfpathlineto{\pgfqpoint{1.524954in}{0.604404in}}%
\pgfpathlineto{\pgfqpoint{1.542223in}{0.610449in}}%
\pgfpathlineto{\pgfqpoint{1.559087in}{0.616781in}}%
\pgfpathlineto{\pgfqpoint{1.575530in}{0.623392in}}%
\pgfpathclose%
\pgfusepath{fill}%
\end{pgfscope}%
\begin{pgfscope}%
\pgfpathrectangle{\pgfqpoint{0.041670in}{0.041670in}}{\pgfqpoint{2.216660in}{2.216660in}}%
\pgfusepath{clip}%
\pgfsetbuttcap%
\pgfsetroundjoin%
\definecolor{currentfill}{rgb}{0.935904,0.898570,0.108131}%
\pgfsetfillcolor{currentfill}%
\pgfsetlinewidth{0.000000pt}%
\definecolor{currentstroke}{rgb}{0.000000,0.000000,0.000000}%
\pgfsetstrokecolor{currentstroke}%
\pgfsetdash{}{0pt}%
\pgfpathmoveto{\pgfqpoint{1.211096in}{1.611325in}}%
\pgfpathlineto{\pgfqpoint{1.213047in}{1.607544in}}%
\pgfpathlineto{\pgfqpoint{1.214998in}{1.603634in}}%
\pgfpathlineto{\pgfqpoint{1.216950in}{1.599594in}}%
\pgfpathlineto{\pgfqpoint{1.218902in}{1.595425in}}%
\pgfpathlineto{\pgfqpoint{1.216734in}{1.594863in}}%
\pgfpathlineto{\pgfqpoint{1.214530in}{1.594334in}}%
\pgfpathlineto{\pgfqpoint{1.212290in}{1.593837in}}%
\pgfpathlineto{\pgfqpoint{1.210017in}{1.593374in}}%
\pgfpathlineto{\pgfqpoint{1.208510in}{1.597647in}}%
\pgfpathlineto{\pgfqpoint{1.207003in}{1.601791in}}%
\pgfpathlineto{\pgfqpoint{1.205497in}{1.605805in}}%
\pgfpathlineto{\pgfqpoint{1.203990in}{1.609689in}}%
\pgfpathlineto{\pgfqpoint{1.205808in}{1.610058in}}%
\pgfpathlineto{\pgfqpoint{1.207599in}{1.610454in}}%
\pgfpathlineto{\pgfqpoint{1.209362in}{1.610876in}}%
\pgfpathlineto{\pgfqpoint{1.211096in}{1.611325in}}%
\pgfpathclose%
\pgfusepath{fill}%
\end{pgfscope}%
\begin{pgfscope}%
\pgfpathrectangle{\pgfqpoint{0.041670in}{0.041670in}}{\pgfqpoint{2.216660in}{2.216660in}}%
\pgfusepath{clip}%
\pgfsetbuttcap%
\pgfsetroundjoin%
\definecolor{currentfill}{rgb}{0.271305,0.019942,0.347269}%
\pgfsetfillcolor{currentfill}%
\pgfsetlinewidth{0.000000pt}%
\definecolor{currentstroke}{rgb}{0.000000,0.000000,0.000000}%
\pgfsetstrokecolor{currentstroke}%
\pgfsetdash{}{0pt}%
\pgfpathmoveto{\pgfqpoint{1.258948in}{0.599506in}}%
\pgfpathlineto{\pgfqpoint{1.259509in}{0.594894in}}%
\pgfpathlineto{\pgfqpoint{1.260071in}{0.590486in}}%
\pgfpathlineto{\pgfqpoint{1.260634in}{0.586286in}}%
\pgfpathlineto{\pgfqpoint{1.261198in}{0.582299in}}%
\pgfpathlineto{\pgfqpoint{1.243011in}{0.581062in}}%
\pgfpathlineto{\pgfqpoint{1.224754in}{0.580137in}}%
\pgfpathlineto{\pgfqpoint{1.206446in}{0.579525in}}%
\pgfpathlineto{\pgfqpoint{1.188109in}{0.579228in}}%
\pgfpathlineto{\pgfqpoint{1.188052in}{0.583238in}}%
\pgfpathlineto{\pgfqpoint{1.187996in}{0.587460in}}%
\pgfpathlineto{\pgfqpoint{1.187939in}{0.591891in}}%
\pgfpathlineto{\pgfqpoint{1.187883in}{0.596526in}}%
\pgfpathlineto{\pgfqpoint{1.205712in}{0.596814in}}%
\pgfpathlineto{\pgfqpoint{1.223513in}{0.597407in}}%
\pgfpathlineto{\pgfqpoint{1.241265in}{0.598305in}}%
\pgfpathlineto{\pgfqpoint{1.258948in}{0.599506in}}%
\pgfpathclose%
\pgfusepath{fill}%
\end{pgfscope}%
\begin{pgfscope}%
\pgfpathrectangle{\pgfqpoint{0.041670in}{0.041670in}}{\pgfqpoint{2.216660in}{2.216660in}}%
\pgfusepath{clip}%
\pgfsetbuttcap%
\pgfsetroundjoin%
\definecolor{currentfill}{rgb}{0.172719,0.448791,0.557885}%
\pgfsetfillcolor{currentfill}%
\pgfsetlinewidth{0.000000pt}%
\definecolor{currentstroke}{rgb}{0.000000,0.000000,0.000000}%
\pgfsetstrokecolor{currentstroke}%
\pgfsetdash{}{0pt}%
\pgfpathmoveto{\pgfqpoint{0.533173in}{0.928840in}}%
\pgfpathlineto{\pgfqpoint{0.529289in}{0.942985in}}%
\pgfpathlineto{\pgfqpoint{0.525382in}{0.957652in}}%
\pgfpathlineto{\pgfqpoint{0.521452in}{0.972850in}}%
\pgfpathlineto{\pgfqpoint{0.509342in}{0.983649in}}%
\pgfpathlineto{\pgfqpoint{0.497956in}{0.994621in}}%
\pgfpathlineto{\pgfqpoint{0.487303in}{1.005754in}}%
\pgfpathlineto{\pgfqpoint{0.477391in}{1.017037in}}%
\pgfpathlineto{\pgfqpoint{0.481559in}{1.001670in}}%
\pgfpathlineto{\pgfqpoint{0.485702in}{0.986832in}}%
\pgfpathlineto{\pgfqpoint{0.489820in}{0.972513in}}%
\pgfpathlineto{\pgfqpoint{0.499576in}{0.961361in}}%
\pgfpathlineto{\pgfqpoint{0.510059in}{0.950357in}}%
\pgfpathlineto{\pgfqpoint{0.521261in}{0.939512in}}%
\pgfpathlineto{\pgfqpoint{0.533173in}{0.928840in}}%
\pgfpathclose%
\pgfusepath{fill}%
\end{pgfscope}%
\begin{pgfscope}%
\pgfpathrectangle{\pgfqpoint{0.041670in}{0.041670in}}{\pgfqpoint{2.216660in}{2.216660in}}%
\pgfusepath{clip}%
\pgfsetbuttcap%
\pgfsetroundjoin%
\definecolor{currentfill}{rgb}{0.855810,0.888601,0.097452}%
\pgfsetfillcolor{currentfill}%
\pgfsetlinewidth{0.000000pt}%
\definecolor{currentstroke}{rgb}{0.000000,0.000000,0.000000}%
\pgfsetstrokecolor{currentstroke}%
\pgfsetdash{}{0pt}%
\pgfpathmoveto{\pgfqpoint{1.125419in}{1.579822in}}%
\pgfpathlineto{\pgfqpoint{1.123143in}{1.575120in}}%
\pgfpathlineto{\pgfqpoint{1.120866in}{1.570293in}}%
\pgfpathlineto{\pgfqpoint{1.118589in}{1.565344in}}%
\pgfpathlineto{\pgfqpoint{1.116312in}{1.560272in}}%
\pgfpathlineto{\pgfqpoint{1.113524in}{1.561242in}}%
\pgfpathlineto{\pgfqpoint{1.110804in}{1.562253in}}%
\pgfpathlineto{\pgfqpoint{1.108153in}{1.563304in}}%
\pgfpathlineto{\pgfqpoint{1.105576in}{1.564394in}}%
\pgfpathlineto{\pgfqpoint{1.108235in}{1.569316in}}%
\pgfpathlineto{\pgfqpoint{1.110895in}{1.574115in}}%
\pgfpathlineto{\pgfqpoint{1.113554in}{1.578792in}}%
\pgfpathlineto{\pgfqpoint{1.116214in}{1.583344in}}%
\pgfpathlineto{\pgfqpoint{1.118424in}{1.582413in}}%
\pgfpathlineto{\pgfqpoint{1.120697in}{1.581515in}}%
\pgfpathlineto{\pgfqpoint{1.123029in}{1.580651in}}%
\pgfpathlineto{\pgfqpoint{1.125419in}{1.579822in}}%
\pgfpathclose%
\pgfusepath{fill}%
\end{pgfscope}%
\begin{pgfscope}%
\pgfpathrectangle{\pgfqpoint{0.041670in}{0.041670in}}{\pgfqpoint{2.216660in}{2.216660in}}%
\pgfusepath{clip}%
\pgfsetbuttcap%
\pgfsetroundjoin%
\definecolor{currentfill}{rgb}{0.935904,0.898570,0.108131}%
\pgfsetfillcolor{currentfill}%
\pgfsetlinewidth{0.000000pt}%
\definecolor{currentstroke}{rgb}{0.000000,0.000000,0.000000}%
\pgfsetstrokecolor{currentstroke}%
\pgfsetdash{}{0pt}%
\pgfpathmoveto{\pgfqpoint{1.157555in}{1.609383in}}%
\pgfpathlineto{\pgfqpoint{1.156151in}{1.605480in}}%
\pgfpathlineto{\pgfqpoint{1.154747in}{1.601447in}}%
\pgfpathlineto{\pgfqpoint{1.153343in}{1.597283in}}%
\pgfpathlineto{\pgfqpoint{1.151938in}{1.592991in}}%
\pgfpathlineto{\pgfqpoint{1.149638in}{1.593424in}}%
\pgfpathlineto{\pgfqpoint{1.147369in}{1.593891in}}%
\pgfpathlineto{\pgfqpoint{1.145133in}{1.594391in}}%
\pgfpathlineto{\pgfqpoint{1.142933in}{1.594924in}}%
\pgfpathlineto{\pgfqpoint{1.144788in}{1.599118in}}%
\pgfpathlineto{\pgfqpoint{1.146644in}{1.603184in}}%
\pgfpathlineto{\pgfqpoint{1.148499in}{1.607119in}}%
\pgfpathlineto{\pgfqpoint{1.150353in}{1.610925in}}%
\pgfpathlineto{\pgfqpoint{1.152114in}{1.610500in}}%
\pgfpathlineto{\pgfqpoint{1.153902in}{1.610101in}}%
\pgfpathlineto{\pgfqpoint{1.155716in}{1.609728in}}%
\pgfpathlineto{\pgfqpoint{1.157555in}{1.609383in}}%
\pgfpathclose%
\pgfusepath{fill}%
\end{pgfscope}%
\begin{pgfscope}%
\pgfpathrectangle{\pgfqpoint{0.041670in}{0.041670in}}{\pgfqpoint{2.216660in}{2.216660in}}%
\pgfusepath{clip}%
\pgfsetbuttcap%
\pgfsetroundjoin%
\definecolor{currentfill}{rgb}{0.268510,0.009605,0.335427}%
\pgfsetfillcolor{currentfill}%
\pgfsetlinewidth{0.000000pt}%
\definecolor{currentstroke}{rgb}{0.000000,0.000000,0.000000}%
\pgfsetstrokecolor{currentstroke}%
\pgfsetdash{}{0pt}%
\pgfpathmoveto{\pgfqpoint{1.332833in}{0.590338in}}%
\pgfpathlineto{\pgfqpoint{1.333895in}{0.586628in}}%
\pgfpathlineto{\pgfqpoint{1.334960in}{0.583141in}}%
\pgfpathlineto{\pgfqpoint{1.336026in}{0.579880in}}%
\pgfpathlineto{\pgfqpoint{1.337095in}{0.576851in}}%
\pgfpathlineto{\pgfqpoint{1.318901in}{0.574310in}}%
\pgfpathlineto{\pgfqpoint{1.300551in}{0.572081in}}%
\pgfpathlineto{\pgfqpoint{1.282066in}{0.570168in}}%
\pgfpathlineto{\pgfqpoint{1.263466in}{0.568574in}}%
\pgfpathlineto{\pgfqpoint{1.262897in}{0.571662in}}%
\pgfpathlineto{\pgfqpoint{1.262330in}{0.574983in}}%
\pgfpathlineto{\pgfqpoint{1.261763in}{0.578530in}}%
\pgfpathlineto{\pgfqpoint{1.261198in}{0.582299in}}%
\pgfpathlineto{\pgfqpoint{1.279294in}{0.583848in}}%
\pgfpathlineto{\pgfqpoint{1.297278in}{0.585706in}}%
\pgfpathlineto{\pgfqpoint{1.315131in}{0.587870in}}%
\pgfpathlineto{\pgfqpoint{1.332833in}{0.590338in}}%
\pgfpathclose%
\pgfusepath{fill}%
\end{pgfscope}%
\begin{pgfscope}%
\pgfpathrectangle{\pgfqpoint{0.041670in}{0.041670in}}{\pgfqpoint{2.216660in}{2.216660in}}%
\pgfusepath{clip}%
\pgfsetbuttcap%
\pgfsetroundjoin%
\definecolor{currentfill}{rgb}{0.814576,0.883393,0.110347}%
\pgfsetfillcolor{currentfill}%
\pgfsetlinewidth{0.000000pt}%
\definecolor{currentstroke}{rgb}{0.000000,0.000000,0.000000}%
\pgfsetstrokecolor{currentstroke}%
\pgfsetdash{}{0pt}%
\pgfpathmoveto{\pgfqpoint{1.256561in}{1.565395in}}%
\pgfpathlineto{\pgfqpoint{1.259300in}{1.560389in}}%
\pgfpathlineto{\pgfqpoint{1.262038in}{1.555264in}}%
\pgfpathlineto{\pgfqpoint{1.264775in}{1.550019in}}%
\pgfpathlineto{\pgfqpoint{1.267513in}{1.544657in}}%
\pgfpathlineto{\pgfqpoint{1.264645in}{1.543369in}}%
\pgfpathlineto{\pgfqpoint{1.261692in}{1.542124in}}%
\pgfpathlineto{\pgfqpoint{1.258655in}{1.540924in}}%
\pgfpathlineto{\pgfqpoint{1.255539in}{1.539769in}}%
\pgfpathlineto{\pgfqpoint{1.253174in}{1.545288in}}%
\pgfpathlineto{\pgfqpoint{1.250809in}{1.550689in}}%
\pgfpathlineto{\pgfqpoint{1.248444in}{1.555971in}}%
\pgfpathlineto{\pgfqpoint{1.246079in}{1.561132in}}%
\pgfpathlineto{\pgfqpoint{1.248807in}{1.562139in}}%
\pgfpathlineto{\pgfqpoint{1.251465in}{1.563186in}}%
\pgfpathlineto{\pgfqpoint{1.254051in}{1.564271in}}%
\pgfpathlineto{\pgfqpoint{1.256561in}{1.565395in}}%
\pgfpathclose%
\pgfusepath{fill}%
\end{pgfscope}%
\begin{pgfscope}%
\pgfpathrectangle{\pgfqpoint{0.041670in}{0.041670in}}{\pgfqpoint{2.216660in}{2.216660in}}%
\pgfusepath{clip}%
\pgfsetbuttcap%
\pgfsetroundjoin%
\definecolor{currentfill}{rgb}{0.636902,0.856542,0.216620}%
\pgfsetfillcolor{currentfill}%
\pgfsetlinewidth{0.000000pt}%
\definecolor{currentstroke}{rgb}{0.000000,0.000000,0.000000}%
\pgfsetstrokecolor{currentstroke}%
\pgfsetdash{}{0pt}%
\pgfpathmoveto{\pgfqpoint{1.289392in}{1.497667in}}%
\pgfpathlineto{\pgfqpoint{1.292124in}{1.491300in}}%
\pgfpathlineto{\pgfqpoint{1.294855in}{1.484829in}}%
\pgfpathlineto{\pgfqpoint{1.297585in}{1.478255in}}%
\pgfpathlineto{\pgfqpoint{1.300315in}{1.471579in}}%
\pgfpathlineto{\pgfqpoint{1.296382in}{1.469788in}}%
\pgfpathlineto{\pgfqpoint{1.292330in}{1.468057in}}%
\pgfpathlineto{\pgfqpoint{1.288162in}{1.466388in}}%
\pgfpathlineto{\pgfqpoint{1.283884in}{1.464783in}}%
\pgfpathlineto{\pgfqpoint{1.281525in}{1.471620in}}%
\pgfpathlineto{\pgfqpoint{1.279165in}{1.478355in}}%
\pgfpathlineto{\pgfqpoint{1.276804in}{1.484986in}}%
\pgfpathlineto{\pgfqpoint{1.274443in}{1.491512in}}%
\pgfpathlineto{\pgfqpoint{1.278335in}{1.492966in}}%
\pgfpathlineto{\pgfqpoint{1.282126in}{1.494478in}}%
\pgfpathlineto{\pgfqpoint{1.285813in}{1.496045in}}%
\pgfpathlineto{\pgfqpoint{1.289392in}{1.497667in}}%
\pgfpathclose%
\pgfusepath{fill}%
\end{pgfscope}%
\begin{pgfscope}%
\pgfpathrectangle{\pgfqpoint{0.041670in}{0.041670in}}{\pgfqpoint{2.216660in}{2.216660in}}%
\pgfusepath{clip}%
\pgfsetbuttcap%
\pgfsetroundjoin%
\definecolor{currentfill}{rgb}{0.147607,0.511733,0.557049}%
\pgfsetfillcolor{currentfill}%
\pgfsetlinewidth{0.000000pt}%
\definecolor{currentstroke}{rgb}{0.000000,0.000000,0.000000}%
\pgfsetstrokecolor{currentstroke}%
\pgfsetdash{}{0pt}%
\pgfpathmoveto{\pgfqpoint{1.270936in}{1.062432in}}%
\pgfpathlineto{\pgfqpoint{1.271960in}{1.052707in}}%
\pgfpathlineto{\pgfqpoint{1.272984in}{1.042982in}}%
\pgfpathlineto{\pgfqpoint{1.274008in}{1.033259in}}%
\pgfpathlineto{\pgfqpoint{1.275031in}{1.023543in}}%
\pgfpathlineto{\pgfqpoint{1.264012in}{1.022096in}}%
\pgfpathlineto{\pgfqpoint{1.252902in}{1.020829in}}%
\pgfpathlineto{\pgfqpoint{1.241715in}{1.019741in}}%
\pgfpathlineto{\pgfqpoint{1.230461in}{1.018834in}}%
\pgfpathlineto{\pgfqpoint{1.229916in}{1.028608in}}%
\pgfpathlineto{\pgfqpoint{1.229372in}{1.038387in}}%
\pgfpathlineto{\pgfqpoint{1.228828in}{1.048169in}}%
\pgfpathlineto{\pgfqpoint{1.228283in}{1.057951in}}%
\pgfpathlineto{\pgfqpoint{1.239053in}{1.058813in}}%
\pgfpathlineto{\pgfqpoint{1.249759in}{1.059849in}}%
\pgfpathlineto{\pgfqpoint{1.260390in}{1.061055in}}%
\pgfpathlineto{\pgfqpoint{1.270936in}{1.062432in}}%
\pgfpathclose%
\pgfusepath{fill}%
\end{pgfscope}%
\begin{pgfscope}%
\pgfpathrectangle{\pgfqpoint{0.041670in}{0.041670in}}{\pgfqpoint{2.216660in}{2.216660in}}%
\pgfusepath{clip}%
\pgfsetbuttcap%
\pgfsetroundjoin%
\definecolor{currentfill}{rgb}{0.271305,0.019942,0.347269}%
\pgfsetfillcolor{currentfill}%
\pgfsetlinewidth{0.000000pt}%
\definecolor{currentstroke}{rgb}{0.000000,0.000000,0.000000}%
\pgfsetstrokecolor{currentstroke}%
\pgfsetdash{}{0pt}%
\pgfpathmoveto{\pgfqpoint{1.187883in}{0.596526in}}%
\pgfpathlineto{\pgfqpoint{1.187939in}{0.591891in}}%
\pgfpathlineto{\pgfqpoint{1.187996in}{0.587460in}}%
\pgfpathlineto{\pgfqpoint{1.188052in}{0.583238in}}%
\pgfpathlineto{\pgfqpoint{1.188109in}{0.579228in}}%
\pgfpathlineto{\pgfqpoint{1.169762in}{0.579245in}}%
\pgfpathlineto{\pgfqpoint{1.151427in}{0.579577in}}%
\pgfpathlineto{\pgfqpoint{1.133124in}{0.580224in}}%
\pgfpathlineto{\pgfqpoint{1.114874in}{0.581184in}}%
\pgfpathlineto{\pgfqpoint{1.115326in}{0.585179in}}%
\pgfpathlineto{\pgfqpoint{1.115777in}{0.589387in}}%
\pgfpathlineto{\pgfqpoint{1.116227in}{0.593803in}}%
\pgfpathlineto{\pgfqpoint{1.116677in}{0.598423in}}%
\pgfpathlineto{\pgfqpoint{1.134422in}{0.597492in}}%
\pgfpathlineto{\pgfqpoint{1.152218in}{0.596865in}}%
\pgfpathlineto{\pgfqpoint{1.170045in}{0.596543in}}%
\pgfpathlineto{\pgfqpoint{1.187883in}{0.596526in}}%
\pgfpathclose%
\pgfusepath{fill}%
\end{pgfscope}%
\begin{pgfscope}%
\pgfpathrectangle{\pgfqpoint{0.041670in}{0.041670in}}{\pgfqpoint{2.216660in}{2.216660in}}%
\pgfusepath{clip}%
\pgfsetbuttcap%
\pgfsetroundjoin%
\definecolor{currentfill}{rgb}{0.122606,0.585371,0.546557}%
\pgfsetfillcolor{currentfill}%
\pgfsetlinewidth{0.000000pt}%
\definecolor{currentstroke}{rgb}{0.000000,0.000000,0.000000}%
\pgfsetstrokecolor{currentstroke}%
\pgfsetdash{}{0pt}%
\pgfpathmoveto{\pgfqpoint{1.300136in}{1.146337in}}%
\pgfpathlineto{\pgfqpoint{1.301625in}{1.136804in}}%
\pgfpathlineto{\pgfqpoint{1.303114in}{1.127248in}}%
\pgfpathlineto{\pgfqpoint{1.304602in}{1.117673in}}%
\pgfpathlineto{\pgfqpoint{1.306089in}{1.108081in}}%
\pgfpathlineto{\pgfqpoint{1.296451in}{1.106146in}}%
\pgfpathlineto{\pgfqpoint{1.286690in}{1.104364in}}%
\pgfpathlineto{\pgfqpoint{1.276814in}{1.102739in}}%
\pgfpathlineto{\pgfqpoint{1.266836in}{1.101273in}}%
\pgfpathlineto{\pgfqpoint{1.265811in}{1.110954in}}%
\pgfpathlineto{\pgfqpoint{1.264785in}{1.120619in}}%
\pgfpathlineto{\pgfqpoint{1.263759in}{1.130264in}}%
\pgfpathlineto{\pgfqpoint{1.262732in}{1.139886in}}%
\pgfpathlineto{\pgfqpoint{1.272240in}{1.141276in}}%
\pgfpathlineto{\pgfqpoint{1.281650in}{1.142816in}}%
\pgfpathlineto{\pgfqpoint{1.290952in}{1.144503in}}%
\pgfpathlineto{\pgfqpoint{1.300136in}{1.146337in}}%
\pgfpathclose%
\pgfusepath{fill}%
\end{pgfscope}%
\begin{pgfscope}%
\pgfpathrectangle{\pgfqpoint{0.041670in}{0.041670in}}{\pgfqpoint{2.216660in}{2.216660in}}%
\pgfusepath{clip}%
\pgfsetbuttcap%
\pgfsetroundjoin%
\definecolor{currentfill}{rgb}{0.896320,0.893616,0.096335}%
\pgfsetfillcolor{currentfill}%
\pgfsetlinewidth{0.000000pt}%
\definecolor{currentstroke}{rgb}{0.000000,0.000000,0.000000}%
\pgfsetstrokecolor{currentstroke}%
\pgfsetdash{}{0pt}%
\pgfpathmoveto{\pgfqpoint{1.227157in}{1.597985in}}%
\pgfpathlineto{\pgfqpoint{1.229522in}{1.593818in}}%
\pgfpathlineto{\pgfqpoint{1.231887in}{1.589524in}}%
\pgfpathlineto{\pgfqpoint{1.234252in}{1.585103in}}%
\pgfpathlineto{\pgfqpoint{1.236618in}{1.580557in}}%
\pgfpathlineto{\pgfqpoint{1.234221in}{1.579732in}}%
\pgfpathlineto{\pgfqpoint{1.231770in}{1.578943in}}%
\pgfpathlineto{\pgfqpoint{1.229266in}{1.578191in}}%
\pgfpathlineto{\pgfqpoint{1.226712in}{1.577476in}}%
\pgfpathlineto{\pgfqpoint{1.224759in}{1.582153in}}%
\pgfpathlineto{\pgfqpoint{1.222806in}{1.586704in}}%
\pgfpathlineto{\pgfqpoint{1.220854in}{1.591128in}}%
\pgfpathlineto{\pgfqpoint{1.218902in}{1.595425in}}%
\pgfpathlineto{\pgfqpoint{1.221030in}{1.596019in}}%
\pgfpathlineto{\pgfqpoint{1.223117in}{1.596644in}}%
\pgfpathlineto{\pgfqpoint{1.225160in}{1.597300in}}%
\pgfpathlineto{\pgfqpoint{1.227157in}{1.597985in}}%
\pgfpathclose%
\pgfusepath{fill}%
\end{pgfscope}%
\begin{pgfscope}%
\pgfpathrectangle{\pgfqpoint{0.041670in}{0.041670in}}{\pgfqpoint{2.216660in}{2.216660in}}%
\pgfusepath{clip}%
\pgfsetbuttcap%
\pgfsetroundjoin%
\definecolor{currentfill}{rgb}{0.134692,0.658636,0.517649}%
\pgfsetfillcolor{currentfill}%
\pgfsetlinewidth{0.000000pt}%
\definecolor{currentstroke}{rgb}{0.000000,0.000000,0.000000}%
\pgfsetstrokecolor{currentstroke}%
\pgfsetdash{}{0pt}%
\pgfpathmoveto{\pgfqpoint{1.079053in}{1.220035in}}%
\pgfpathlineto{\pgfqpoint{1.077661in}{1.210750in}}%
\pgfpathlineto{\pgfqpoint{1.076269in}{1.201423in}}%
\pgfpathlineto{\pgfqpoint{1.074879in}{1.192055in}}%
\pgfpathlineto{\pgfqpoint{1.073488in}{1.182649in}}%
\pgfpathlineto{\pgfqpoint{1.064771in}{1.184398in}}%
\pgfpathlineto{\pgfqpoint{1.056175in}{1.186283in}}%
\pgfpathlineto{\pgfqpoint{1.047710in}{1.188302in}}%
\pgfpathlineto{\pgfqpoint{1.039385in}{1.190453in}}%
\pgfpathlineto{\pgfqpoint{1.041218in}{1.199746in}}%
\pgfpathlineto{\pgfqpoint{1.043053in}{1.209002in}}%
\pgfpathlineto{\pgfqpoint{1.044888in}{1.218218in}}%
\pgfpathlineto{\pgfqpoint{1.046724in}{1.227391in}}%
\pgfpathlineto{\pgfqpoint{1.054617in}{1.225364in}}%
\pgfpathlineto{\pgfqpoint{1.062641in}{1.223460in}}%
\pgfpathlineto{\pgfqpoint{1.070790in}{1.221683in}}%
\pgfpathlineto{\pgfqpoint{1.079053in}{1.220035in}}%
\pgfpathclose%
\pgfusepath{fill}%
\end{pgfscope}%
\begin{pgfscope}%
\pgfpathrectangle{\pgfqpoint{0.041670in}{0.041670in}}{\pgfqpoint{2.216660in}{2.216660in}}%
\pgfusepath{clip}%
\pgfsetbuttcap%
\pgfsetroundjoin%
\definecolor{currentfill}{rgb}{0.565498,0.842430,0.262877}%
\pgfsetfillcolor{currentfill}%
\pgfsetlinewidth{0.000000pt}%
\definecolor{currentstroke}{rgb}{0.000000,0.000000,0.000000}%
\pgfsetstrokecolor{currentstroke}%
\pgfsetdash{}{0pt}%
\pgfpathmoveto{\pgfqpoint{1.079918in}{1.463410in}}%
\pgfpathlineto{\pgfqpoint{1.077647in}{1.456440in}}%
\pgfpathlineto{\pgfqpoint{1.075377in}{1.449370in}}%
\pgfpathlineto{\pgfqpoint{1.073108in}{1.442203in}}%
\pgfpathlineto{\pgfqpoint{1.070839in}{1.434940in}}%
\pgfpathlineto{\pgfqpoint{1.066071in}{1.436635in}}%
\pgfpathlineto{\pgfqpoint{1.061420in}{1.438402in}}%
\pgfpathlineto{\pgfqpoint{1.056892in}{1.440237in}}%
\pgfpathlineto{\pgfqpoint{1.052489in}{1.442140in}}%
\pgfpathlineto{\pgfqpoint{1.055137in}{1.449246in}}%
\pgfpathlineto{\pgfqpoint{1.057785in}{1.456256in}}%
\pgfpathlineto{\pgfqpoint{1.060435in}{1.463170in}}%
\pgfpathlineto{\pgfqpoint{1.063085in}{1.469984in}}%
\pgfpathlineto{\pgfqpoint{1.067124in}{1.468247in}}%
\pgfpathlineto{\pgfqpoint{1.071279in}{1.466571in}}%
\pgfpathlineto{\pgfqpoint{1.075545in}{1.464958in}}%
\pgfpathlineto{\pgfqpoint{1.079918in}{1.463410in}}%
\pgfpathclose%
\pgfusepath{fill}%
\end{pgfscope}%
\begin{pgfscope}%
\pgfpathrectangle{\pgfqpoint{0.041670in}{0.041670in}}{\pgfqpoint{2.216660in}{2.216660in}}%
\pgfusepath{clip}%
\pgfsetbuttcap%
\pgfsetroundjoin%
\definecolor{currentfill}{rgb}{0.163625,0.471133,0.558148}%
\pgfsetfillcolor{currentfill}%
\pgfsetlinewidth{0.000000pt}%
\definecolor{currentstroke}{rgb}{0.000000,0.000000,0.000000}%
\pgfsetstrokecolor{currentstroke}%
\pgfsetdash{}{0pt}%
\pgfpathmoveto{\pgfqpoint{1.230461in}{1.018834in}}%
\pgfpathlineto{\pgfqpoint{1.231005in}{1.009069in}}%
\pgfpathlineto{\pgfqpoint{1.231549in}{0.999315in}}%
\pgfpathlineto{\pgfqpoint{1.232092in}{0.989575in}}%
\pgfpathlineto{\pgfqpoint{1.232636in}{0.979853in}}%
\pgfpathlineto{\pgfqpoint{1.220841in}{0.979093in}}%
\pgfpathlineto{\pgfqpoint{1.209001in}{0.978525in}}%
\pgfpathlineto{\pgfqpoint{1.197131in}{0.978150in}}%
\pgfpathlineto{\pgfqpoint{1.185242in}{0.977967in}}%
\pgfpathlineto{\pgfqpoint{1.185187in}{0.987711in}}%
\pgfpathlineto{\pgfqpoint{1.185132in}{0.997472in}}%
\pgfpathlineto{\pgfqpoint{1.185078in}{1.007248in}}%
\pgfpathlineto{\pgfqpoint{1.185023in}{1.017035in}}%
\pgfpathlineto{\pgfqpoint{1.196421in}{1.017209in}}%
\pgfpathlineto{\pgfqpoint{1.207802in}{1.017568in}}%
\pgfpathlineto{\pgfqpoint{1.219152in}{1.018109in}}%
\pgfpathlineto{\pgfqpoint{1.230461in}{1.018834in}}%
\pgfpathclose%
\pgfusepath{fill}%
\end{pgfscope}%
\begin{pgfscope}%
\pgfpathrectangle{\pgfqpoint{0.041670in}{0.041670in}}{\pgfqpoint{2.216660in}{2.216660in}}%
\pgfusepath{clip}%
\pgfsetbuttcap%
\pgfsetroundjoin%
\definecolor{currentfill}{rgb}{0.699415,0.867117,0.175971}%
\pgfsetfillcolor{currentfill}%
\pgfsetlinewidth{0.000000pt}%
\definecolor{currentstroke}{rgb}{0.000000,0.000000,0.000000}%
\pgfsetstrokecolor{currentstroke}%
\pgfsetdash{}{0pt}%
\pgfpathmoveto{\pgfqpoint{1.278457in}{1.522055in}}%
\pgfpathlineto{\pgfqpoint{1.281191in}{1.516122in}}%
\pgfpathlineto{\pgfqpoint{1.283925in}{1.510079in}}%
\pgfpathlineto{\pgfqpoint{1.286659in}{1.503927in}}%
\pgfpathlineto{\pgfqpoint{1.289392in}{1.497667in}}%
\pgfpathlineto{\pgfqpoint{1.285813in}{1.496045in}}%
\pgfpathlineto{\pgfqpoint{1.282126in}{1.494478in}}%
\pgfpathlineto{\pgfqpoint{1.278335in}{1.492966in}}%
\pgfpathlineto{\pgfqpoint{1.274443in}{1.491512in}}%
\pgfpathlineto{\pgfqpoint{1.272082in}{1.497932in}}%
\pgfpathlineto{\pgfqpoint{1.269720in}{1.504243in}}%
\pgfpathlineto{\pgfqpoint{1.267357in}{1.510446in}}%
\pgfpathlineto{\pgfqpoint{1.264994in}{1.516537in}}%
\pgfpathlineto{\pgfqpoint{1.268499in}{1.517840in}}%
\pgfpathlineto{\pgfqpoint{1.271913in}{1.519196in}}%
\pgfpathlineto{\pgfqpoint{1.275233in}{1.520601in}}%
\pgfpathlineto{\pgfqpoint{1.278457in}{1.522055in}}%
\pgfpathclose%
\pgfusepath{fill}%
\end{pgfscope}%
\begin{pgfscope}%
\pgfpathrectangle{\pgfqpoint{0.041670in}{0.041670in}}{\pgfqpoint{2.216660in}{2.216660in}}%
\pgfusepath{clip}%
\pgfsetbuttcap%
\pgfsetroundjoin%
\definecolor{currentfill}{rgb}{0.762373,0.876424,0.137064}%
\pgfsetfillcolor{currentfill}%
\pgfsetlinewidth{0.000000pt}%
\definecolor{currentstroke}{rgb}{0.000000,0.000000,0.000000}%
\pgfsetstrokecolor{currentstroke}%
\pgfsetdash{}{0pt}%
\pgfpathmoveto{\pgfqpoint{1.267513in}{1.544657in}}%
\pgfpathlineto{\pgfqpoint{1.270249in}{1.539179in}}%
\pgfpathlineto{\pgfqpoint{1.272986in}{1.533585in}}%
\pgfpathlineto{\pgfqpoint{1.275721in}{1.527876in}}%
\pgfpathlineto{\pgfqpoint{1.278457in}{1.522055in}}%
\pgfpathlineto{\pgfqpoint{1.275233in}{1.520601in}}%
\pgfpathlineto{\pgfqpoint{1.271913in}{1.519196in}}%
\pgfpathlineto{\pgfqpoint{1.268499in}{1.517840in}}%
\pgfpathlineto{\pgfqpoint{1.264994in}{1.516537in}}%
\pgfpathlineto{\pgfqpoint{1.262631in}{1.522516in}}%
\pgfpathlineto{\pgfqpoint{1.260267in}{1.528382in}}%
\pgfpathlineto{\pgfqpoint{1.257903in}{1.534134in}}%
\pgfpathlineto{\pgfqpoint{1.255539in}{1.539769in}}%
\pgfpathlineto{\pgfqpoint{1.258655in}{1.540924in}}%
\pgfpathlineto{\pgfqpoint{1.261692in}{1.542124in}}%
\pgfpathlineto{\pgfqpoint{1.264645in}{1.543369in}}%
\pgfpathlineto{\pgfqpoint{1.267513in}{1.544657in}}%
\pgfpathclose%
\pgfusepath{fill}%
\end{pgfscope}%
\begin{pgfscope}%
\pgfpathrectangle{\pgfqpoint{0.041670in}{0.041670in}}{\pgfqpoint{2.216660in}{2.216660in}}%
\pgfusepath{clip}%
\pgfsetbuttcap%
\pgfsetroundjoin%
\definecolor{currentfill}{rgb}{0.281477,0.755203,0.432552}%
\pgfsetfillcolor{currentfill}%
\pgfsetlinewidth{0.000000pt}%
\definecolor{currentstroke}{rgb}{0.000000,0.000000,0.000000}%
\pgfsetstrokecolor{currentstroke}%
\pgfsetdash{}{0pt}%
\pgfpathmoveto{\pgfqpoint{1.068807in}{1.333300in}}%
\pgfpathlineto{\pgfqpoint{1.066963in}{1.324819in}}%
\pgfpathlineto{\pgfqpoint{1.065120in}{1.316267in}}%
\pgfpathlineto{\pgfqpoint{1.063277in}{1.307648in}}%
\pgfpathlineto{\pgfqpoint{1.061435in}{1.298963in}}%
\pgfpathlineto{\pgfqpoint{1.054534in}{1.300855in}}%
\pgfpathlineto{\pgfqpoint{1.047763in}{1.302853in}}%
\pgfpathlineto{\pgfqpoint{1.041131in}{1.304954in}}%
\pgfpathlineto{\pgfqpoint{1.034643in}{1.307156in}}%
\pgfpathlineto{\pgfqpoint{1.036899in}{1.315703in}}%
\pgfpathlineto{\pgfqpoint{1.039156in}{1.324186in}}%
\pgfpathlineto{\pgfqpoint{1.041413in}{1.332600in}}%
\pgfpathlineto{\pgfqpoint{1.043672in}{1.340946in}}%
\pgfpathlineto{\pgfqpoint{1.049759in}{1.338891in}}%
\pgfpathlineto{\pgfqpoint{1.055982in}{1.336930in}}%
\pgfpathlineto{\pgfqpoint{1.062333in}{1.335066in}}%
\pgfpathlineto{\pgfqpoint{1.068807in}{1.333300in}}%
\pgfpathclose%
\pgfusepath{fill}%
\end{pgfscope}%
\begin{pgfscope}%
\pgfpathrectangle{\pgfqpoint{0.041670in}{0.041670in}}{\pgfqpoint{2.216660in}{2.216660in}}%
\pgfusepath{clip}%
\pgfsetbuttcap%
\pgfsetroundjoin%
\definecolor{currentfill}{rgb}{0.896320,0.893616,0.096335}%
\pgfsetfillcolor{currentfill}%
\pgfsetlinewidth{0.000000pt}%
\definecolor{currentstroke}{rgb}{0.000000,0.000000,0.000000}%
\pgfsetstrokecolor{currentstroke}%
\pgfsetdash{}{0pt}%
\pgfpathmoveto{\pgfqpoint{1.142933in}{1.594924in}}%
\pgfpathlineto{\pgfqpoint{1.141077in}{1.590602in}}%
\pgfpathlineto{\pgfqpoint{1.139221in}{1.586152in}}%
\pgfpathlineto{\pgfqpoint{1.137364in}{1.581575in}}%
\pgfpathlineto{\pgfqpoint{1.135508in}{1.576873in}}%
\pgfpathlineto{\pgfqpoint{1.132912in}{1.577554in}}%
\pgfpathlineto{\pgfqpoint{1.130363in}{1.578272in}}%
\pgfpathlineto{\pgfqpoint{1.127865in}{1.579029in}}%
\pgfpathlineto{\pgfqpoint{1.125419in}{1.579822in}}%
\pgfpathlineto{\pgfqpoint{1.127696in}{1.584399in}}%
\pgfpathlineto{\pgfqpoint{1.129973in}{1.588851in}}%
\pgfpathlineto{\pgfqpoint{1.132250in}{1.593176in}}%
\pgfpathlineto{\pgfqpoint{1.134526in}{1.597374in}}%
\pgfpathlineto{\pgfqpoint{1.136564in}{1.596715in}}%
\pgfpathlineto{\pgfqpoint{1.138646in}{1.596087in}}%
\pgfpathlineto{\pgfqpoint{1.140769in}{1.595490in}}%
\pgfpathlineto{\pgfqpoint{1.142933in}{1.594924in}}%
\pgfpathclose%
\pgfusepath{fill}%
\end{pgfscope}%
\begin{pgfscope}%
\pgfpathrectangle{\pgfqpoint{0.041670in}{0.041670in}}{\pgfqpoint{2.216660in}{2.216660in}}%
\pgfusepath{clip}%
\pgfsetbuttcap%
\pgfsetroundjoin%
\definecolor{currentfill}{rgb}{0.163625,0.471133,0.558148}%
\pgfsetfillcolor{currentfill}%
\pgfsetlinewidth{0.000000pt}%
\definecolor{currentstroke}{rgb}{0.000000,0.000000,0.000000}%
\pgfsetstrokecolor{currentstroke}%
\pgfsetdash{}{0pt}%
\pgfpathmoveto{\pgfqpoint{1.185023in}{1.017035in}}%
\pgfpathlineto{\pgfqpoint{1.185078in}{1.007248in}}%
\pgfpathlineto{\pgfqpoint{1.185132in}{0.997472in}}%
\pgfpathlineto{\pgfqpoint{1.185187in}{0.987711in}}%
\pgfpathlineto{\pgfqpoint{1.185242in}{0.977967in}}%
\pgfpathlineto{\pgfqpoint{1.173347in}{0.977978in}}%
\pgfpathlineto{\pgfqpoint{1.161459in}{0.978182in}}%
\pgfpathlineto{\pgfqpoint{1.149591in}{0.978579in}}%
\pgfpathlineto{\pgfqpoint{1.137756in}{0.979168in}}%
\pgfpathlineto{\pgfqpoint{1.138191in}{0.988898in}}%
\pgfpathlineto{\pgfqpoint{1.138627in}{0.998646in}}%
\pgfpathlineto{\pgfqpoint{1.139063in}{1.008407in}}%
\pgfpathlineto{\pgfqpoint{1.139499in}{1.018181in}}%
\pgfpathlineto{\pgfqpoint{1.150845in}{1.017619in}}%
\pgfpathlineto{\pgfqpoint{1.162223in}{1.017240in}}%
\pgfpathlineto{\pgfqpoint{1.173620in}{1.017046in}}%
\pgfpathlineto{\pgfqpoint{1.185023in}{1.017035in}}%
\pgfpathclose%
\pgfusepath{fill}%
\end{pgfscope}%
\begin{pgfscope}%
\pgfpathrectangle{\pgfqpoint{0.041670in}{0.041670in}}{\pgfqpoint{2.216660in}{2.216660in}}%
\pgfusepath{clip}%
\pgfsetbuttcap%
\pgfsetroundjoin%
\definecolor{currentfill}{rgb}{0.147607,0.511733,0.557049}%
\pgfsetfillcolor{currentfill}%
\pgfsetlinewidth{0.000000pt}%
\definecolor{currentstroke}{rgb}{0.000000,0.000000,0.000000}%
\pgfsetstrokecolor{currentstroke}%
\pgfsetdash{}{0pt}%
\pgfpathmoveto{\pgfqpoint{1.141243in}{1.057329in}}%
\pgfpathlineto{\pgfqpoint{1.140807in}{1.047539in}}%
\pgfpathlineto{\pgfqpoint{1.140371in}{1.037750in}}%
\pgfpathlineto{\pgfqpoint{1.139935in}{1.027962in}}%
\pgfpathlineto{\pgfqpoint{1.139499in}{1.018181in}}%
\pgfpathlineto{\pgfqpoint{1.128196in}{1.018926in}}%
\pgfpathlineto{\pgfqpoint{1.116948in}{1.019853in}}%
\pgfpathlineto{\pgfqpoint{1.105769in}{1.020961in}}%
\pgfpathlineto{\pgfqpoint{1.094669in}{1.022248in}}%
\pgfpathlineto{\pgfqpoint{1.095587in}{1.031981in}}%
\pgfpathlineto{\pgfqpoint{1.096506in}{1.041719in}}%
\pgfpathlineto{\pgfqpoint{1.097424in}{1.051459in}}%
\pgfpathlineto{\pgfqpoint{1.098343in}{1.061200in}}%
\pgfpathlineto{\pgfqpoint{1.108966in}{1.059974in}}%
\pgfpathlineto{\pgfqpoint{1.119664in}{1.058920in}}%
\pgfpathlineto{\pgfqpoint{1.130427in}{1.058038in}}%
\pgfpathlineto{\pgfqpoint{1.141243in}{1.057329in}}%
\pgfpathclose%
\pgfusepath{fill}%
\end{pgfscope}%
\begin{pgfscope}%
\pgfpathrectangle{\pgfqpoint{0.041670in}{0.041670in}}{\pgfqpoint{2.216660in}{2.216660in}}%
\pgfusepath{clip}%
\pgfsetbuttcap%
\pgfsetroundjoin%
\definecolor{currentfill}{rgb}{0.935904,0.898570,0.108131}%
\pgfsetfillcolor{currentfill}%
\pgfsetlinewidth{0.000000pt}%
\definecolor{currentstroke}{rgb}{0.000000,0.000000,0.000000}%
\pgfsetstrokecolor{currentstroke}%
\pgfsetdash{}{0pt}%
\pgfpathmoveto{\pgfqpoint{1.203990in}{1.609689in}}%
\pgfpathlineto{\pgfqpoint{1.205497in}{1.605805in}}%
\pgfpathlineto{\pgfqpoint{1.207003in}{1.601791in}}%
\pgfpathlineto{\pgfqpoint{1.208510in}{1.597647in}}%
\pgfpathlineto{\pgfqpoint{1.210017in}{1.593374in}}%
\pgfpathlineto{\pgfqpoint{1.207715in}{1.592945in}}%
\pgfpathlineto{\pgfqpoint{1.205384in}{1.592550in}}%
\pgfpathlineto{\pgfqpoint{1.203027in}{1.592190in}}%
\pgfpathlineto{\pgfqpoint{1.200647in}{1.591865in}}%
\pgfpathlineto{\pgfqpoint{1.199609in}{1.596215in}}%
\pgfpathlineto{\pgfqpoint{1.198572in}{1.600435in}}%
\pgfpathlineto{\pgfqpoint{1.197534in}{1.604525in}}%
\pgfpathlineto{\pgfqpoint{1.196498in}{1.608485in}}%
\pgfpathlineto{\pgfqpoint{1.198401in}{1.608745in}}%
\pgfpathlineto{\pgfqpoint{1.200285in}{1.609032in}}%
\pgfpathlineto{\pgfqpoint{1.202149in}{1.609346in}}%
\pgfpathlineto{\pgfqpoint{1.203990in}{1.609689in}}%
\pgfpathclose%
\pgfusepath{fill}%
\end{pgfscope}%
\begin{pgfscope}%
\pgfpathrectangle{\pgfqpoint{0.041670in}{0.041670in}}{\pgfqpoint{2.216660in}{2.216660in}}%
\pgfusepath{clip}%
\pgfsetbuttcap%
\pgfsetroundjoin%
\definecolor{currentfill}{rgb}{0.814576,0.883393,0.110347}%
\pgfsetfillcolor{currentfill}%
\pgfsetlinewidth{0.000000pt}%
\definecolor{currentstroke}{rgb}{0.000000,0.000000,0.000000}%
\pgfsetstrokecolor{currentstroke}%
\pgfsetdash{}{0pt}%
\pgfpathmoveto{\pgfqpoint{1.116312in}{1.560272in}}%
\pgfpathlineto{\pgfqpoint{1.114035in}{1.555079in}}%
\pgfpathlineto{\pgfqpoint{1.111759in}{1.549766in}}%
\pgfpathlineto{\pgfqpoint{1.109482in}{1.544333in}}%
\pgfpathlineto{\pgfqpoint{1.107206in}{1.538783in}}%
\pgfpathlineto{\pgfqpoint{1.104021in}{1.539895in}}%
\pgfpathlineto{\pgfqpoint{1.100913in}{1.541055in}}%
\pgfpathlineto{\pgfqpoint{1.097885in}{1.542260in}}%
\pgfpathlineto{\pgfqpoint{1.094942in}{1.543510in}}%
\pgfpathlineto{\pgfqpoint{1.097600in}{1.548909in}}%
\pgfpathlineto{\pgfqpoint{1.100258in}{1.554190in}}%
\pgfpathlineto{\pgfqpoint{1.102917in}{1.559352in}}%
\pgfpathlineto{\pgfqpoint{1.105576in}{1.564394in}}%
\pgfpathlineto{\pgfqpoint{1.108153in}{1.563304in}}%
\pgfpathlineto{\pgfqpoint{1.110804in}{1.562253in}}%
\pgfpathlineto{\pgfqpoint{1.113524in}{1.561242in}}%
\pgfpathlineto{\pgfqpoint{1.116312in}{1.560272in}}%
\pgfpathclose%
\pgfusepath{fill}%
\end{pgfscope}%
\begin{pgfscope}%
\pgfpathrectangle{\pgfqpoint{0.041670in}{0.041670in}}{\pgfqpoint{2.216660in}{2.216660in}}%
\pgfusepath{clip}%
\pgfsetbuttcap%
\pgfsetroundjoin%
\definecolor{currentfill}{rgb}{0.282884,0.135920,0.453427}%
\pgfsetfillcolor{currentfill}%
\pgfsetlinewidth{0.000000pt}%
\definecolor{currentstroke}{rgb}{0.000000,0.000000,0.000000}%
\pgfsetstrokecolor{currentstroke}%
\pgfsetdash{}{0pt}%
\pgfpathmoveto{\pgfqpoint{1.660215in}{0.679777in}}%
\pgfpathlineto{\pgfqpoint{1.663197in}{0.684783in}}%
\pgfpathlineto{\pgfqpoint{1.666191in}{0.690163in}}%
\pgfpathlineto{\pgfqpoint{1.669197in}{0.695924in}}%
\pgfpathlineto{\pgfqpoint{1.672216in}{0.702073in}}%
\pgfpathlineto{\pgfqpoint{1.656529in}{0.693892in}}%
\pgfpathlineto{\pgfqpoint{1.640314in}{0.685972in}}%
\pgfpathlineto{\pgfqpoint{1.623588in}{0.678322in}}%
\pgfpathlineto{\pgfqpoint{1.606368in}{0.670952in}}%
\pgfpathlineto{\pgfqpoint{1.603742in}{0.664969in}}%
\pgfpathlineto{\pgfqpoint{1.601127in}{0.659375in}}%
\pgfpathlineto{\pgfqpoint{1.598523in}{0.654163in}}%
\pgfpathlineto{\pgfqpoint{1.595929in}{0.649326in}}%
\pgfpathlineto{\pgfqpoint{1.612738in}{0.656537in}}%
\pgfpathlineto{\pgfqpoint{1.629067in}{0.664022in}}%
\pgfpathlineto{\pgfqpoint{1.644897in}{0.671772in}}%
\pgfpathlineto{\pgfqpoint{1.660215in}{0.679777in}}%
\pgfpathclose%
\pgfusepath{fill}%
\end{pgfscope}%
\begin{pgfscope}%
\pgfpathrectangle{\pgfqpoint{0.041670in}{0.041670in}}{\pgfqpoint{2.216660in}{2.216660in}}%
\pgfusepath{clip}%
\pgfsetbuttcap%
\pgfsetroundjoin%
\definecolor{currentfill}{rgb}{0.344074,0.780029,0.397381}%
\pgfsetfillcolor{currentfill}%
\pgfsetlinewidth{0.000000pt}%
\definecolor{currentstroke}{rgb}{0.000000,0.000000,0.000000}%
\pgfsetstrokecolor{currentstroke}%
\pgfsetdash{}{0pt}%
\pgfpathmoveto{\pgfqpoint{1.312139in}{1.375360in}}%
\pgfpathlineto{\pgfqpoint{1.314488in}{1.367347in}}%
\pgfpathlineto{\pgfqpoint{1.316837in}{1.359256in}}%
\pgfpathlineto{\pgfqpoint{1.319184in}{1.351090in}}%
\pgfpathlineto{\pgfqpoint{1.321531in}{1.342850in}}%
\pgfpathlineto{\pgfqpoint{1.315569in}{1.340713in}}%
\pgfpathlineto{\pgfqpoint{1.309466in}{1.338668in}}%
\pgfpathlineto{\pgfqpoint{1.303229in}{1.336718in}}%
\pgfpathlineto{\pgfqpoint{1.296863in}{1.334865in}}%
\pgfpathlineto{\pgfqpoint{1.294923in}{1.343247in}}%
\pgfpathlineto{\pgfqpoint{1.292983in}{1.351555in}}%
\pgfpathlineto{\pgfqpoint{1.291041in}{1.359787in}}%
\pgfpathlineto{\pgfqpoint{1.289099in}{1.367941in}}%
\pgfpathlineto{\pgfqpoint{1.295044in}{1.369663in}}%
\pgfpathlineto{\pgfqpoint{1.300870in}{1.371475in}}%
\pgfpathlineto{\pgfqpoint{1.306570in}{1.373374in}}%
\pgfpathlineto{\pgfqpoint{1.312139in}{1.375360in}}%
\pgfpathclose%
\pgfusepath{fill}%
\end{pgfscope}%
\begin{pgfscope}%
\pgfpathrectangle{\pgfqpoint{0.041670in}{0.041670in}}{\pgfqpoint{2.216660in}{2.216660in}}%
\pgfusepath{clip}%
\pgfsetbuttcap%
\pgfsetroundjoin%
\definecolor{currentfill}{rgb}{0.935904,0.898570,0.108131}%
\pgfsetfillcolor{currentfill}%
\pgfsetlinewidth{0.000000pt}%
\definecolor{currentstroke}{rgb}{0.000000,0.000000,0.000000}%
\pgfsetstrokecolor{currentstroke}%
\pgfsetdash{}{0pt}%
\pgfpathmoveto{\pgfqpoint{1.165118in}{1.608279in}}%
\pgfpathlineto{\pgfqpoint{1.164188in}{1.604306in}}%
\pgfpathlineto{\pgfqpoint{1.163258in}{1.600202in}}%
\pgfpathlineto{\pgfqpoint{1.162327in}{1.595969in}}%
\pgfpathlineto{\pgfqpoint{1.161396in}{1.591606in}}%
\pgfpathlineto{\pgfqpoint{1.158998in}{1.591900in}}%
\pgfpathlineto{\pgfqpoint{1.156620in}{1.592228in}}%
\pgfpathlineto{\pgfqpoint{1.154266in}{1.592592in}}%
\pgfpathlineto{\pgfqpoint{1.151938in}{1.592991in}}%
\pgfpathlineto{\pgfqpoint{1.153343in}{1.597283in}}%
\pgfpathlineto{\pgfqpoint{1.154747in}{1.601447in}}%
\pgfpathlineto{\pgfqpoint{1.156151in}{1.605480in}}%
\pgfpathlineto{\pgfqpoint{1.157555in}{1.609383in}}%
\pgfpathlineto{\pgfqpoint{1.159417in}{1.609065in}}%
\pgfpathlineto{\pgfqpoint{1.161299in}{1.608775in}}%
\pgfpathlineto{\pgfqpoint{1.163200in}{1.608513in}}%
\pgfpathlineto{\pgfqpoint{1.165118in}{1.608279in}}%
\pgfpathclose%
\pgfusepath{fill}%
\end{pgfscope}%
\begin{pgfscope}%
\pgfpathrectangle{\pgfqpoint{0.041670in}{0.041670in}}{\pgfqpoint{2.216660in}{2.216660in}}%
\pgfusepath{clip}%
\pgfsetbuttcap%
\pgfsetroundjoin%
\definecolor{currentfill}{rgb}{0.268510,0.009605,0.335427}%
\pgfsetfillcolor{currentfill}%
\pgfsetlinewidth{0.000000pt}%
\definecolor{currentstroke}{rgb}{0.000000,0.000000,0.000000}%
\pgfsetstrokecolor{currentstroke}%
\pgfsetdash{}{0pt}%
\pgfpathmoveto{\pgfqpoint{1.114874in}{0.581184in}}%
\pgfpathlineto{\pgfqpoint{1.114421in}{0.577406in}}%
\pgfpathlineto{\pgfqpoint{1.113967in}{0.573851in}}%
\pgfpathlineto{\pgfqpoint{1.113513in}{0.570522in}}%
\pgfpathlineto{\pgfqpoint{1.113057in}{0.567425in}}%
\pgfpathlineto{\pgfqpoint{1.094372in}{0.568735in}}%
\pgfpathlineto{\pgfqpoint{1.075784in}{0.570365in}}%
\pgfpathlineto{\pgfqpoint{1.057313in}{0.572313in}}%
\pgfpathlineto{\pgfqpoint{1.038979in}{0.574577in}}%
\pgfpathlineto{\pgfqpoint{1.039938in}{0.577622in}}%
\pgfpathlineto{\pgfqpoint{1.040895in}{0.580899in}}%
\pgfpathlineto{\pgfqpoint{1.041851in}{0.584403in}}%
\pgfpathlineto{\pgfqpoint{1.042804in}{0.588129in}}%
\pgfpathlineto{\pgfqpoint{1.060641in}{0.585931in}}%
\pgfpathlineto{\pgfqpoint{1.078612in}{0.584039in}}%
\pgfpathlineto{\pgfqpoint{1.096696in}{0.582456in}}%
\pgfpathlineto{\pgfqpoint{1.114874in}{0.581184in}}%
\pgfpathclose%
\pgfusepath{fill}%
\end{pgfscope}%
\begin{pgfscope}%
\pgfpathrectangle{\pgfqpoint{0.041670in}{0.041670in}}{\pgfqpoint{2.216660in}{2.216660in}}%
\pgfusepath{clip}%
\pgfsetbuttcap%
\pgfsetroundjoin%
\definecolor{currentfill}{rgb}{0.268510,0.009605,0.335427}%
\pgfsetfillcolor{currentfill}%
\pgfsetlinewidth{0.000000pt}%
\definecolor{currentstroke}{rgb}{0.000000,0.000000,0.000000}%
\pgfsetstrokecolor{currentstroke}%
\pgfsetdash{}{0pt}%
\pgfpathmoveto{\pgfqpoint{0.955753in}{0.571985in}}%
\pgfpathlineto{\pgfqpoint{0.954277in}{0.571335in}}%
\pgfpathlineto{\pgfqpoint{0.952797in}{0.570966in}}%
\pgfpathlineto{\pgfqpoint{0.951312in}{0.570886in}}%
\pgfpathlineto{\pgfqpoint{0.949823in}{0.571098in}}%
\pgfpathlineto{\pgfqpoint{0.931053in}{0.575211in}}%
\pgfpathlineto{\pgfqpoint{0.912564in}{0.579642in}}%
\pgfpathlineto{\pgfqpoint{0.894374in}{0.584385in}}%
\pgfpathlineto{\pgfqpoint{0.876505in}{0.589433in}}%
\pgfpathlineto{\pgfqpoint{0.878463in}{0.589103in}}%
\pgfpathlineto{\pgfqpoint{0.880416in}{0.589065in}}%
\pgfpathlineto{\pgfqpoint{0.882364in}{0.589315in}}%
\pgfpathlineto{\pgfqpoint{0.884306in}{0.589847in}}%
\pgfpathlineto{\pgfqpoint{0.901721in}{0.584928in}}%
\pgfpathlineto{\pgfqpoint{0.919446in}{0.580308in}}%
\pgfpathlineto{\pgfqpoint{0.937464in}{0.575992in}}%
\pgfpathlineto{\pgfqpoint{0.955753in}{0.571985in}}%
\pgfpathclose%
\pgfusepath{fill}%
\end{pgfscope}%
\begin{pgfscope}%
\pgfpathrectangle{\pgfqpoint{0.041670in}{0.041670in}}{\pgfqpoint{2.216660in}{2.216660in}}%
\pgfusepath{clip}%
\pgfsetbuttcap%
\pgfsetroundjoin%
\definecolor{currentfill}{rgb}{0.636902,0.856542,0.216620}%
\pgfsetfillcolor{currentfill}%
\pgfsetlinewidth{0.000000pt}%
\definecolor{currentstroke}{rgb}{0.000000,0.000000,0.000000}%
\pgfsetstrokecolor{currentstroke}%
\pgfsetdash{}{0pt}%
\pgfpathmoveto{\pgfqpoint{1.089007in}{1.490270in}}%
\pgfpathlineto{\pgfqpoint{1.086734in}{1.483711in}}%
\pgfpathlineto{\pgfqpoint{1.084462in}{1.477048in}}%
\pgfpathlineto{\pgfqpoint{1.082190in}{1.470280in}}%
\pgfpathlineto{\pgfqpoint{1.079918in}{1.463410in}}%
\pgfpathlineto{\pgfqpoint{1.075545in}{1.464958in}}%
\pgfpathlineto{\pgfqpoint{1.071279in}{1.466571in}}%
\pgfpathlineto{\pgfqpoint{1.067124in}{1.468247in}}%
\pgfpathlineto{\pgfqpoint{1.063085in}{1.469984in}}%
\pgfpathlineto{\pgfqpoint{1.065736in}{1.476698in}}%
\pgfpathlineto{\pgfqpoint{1.068388in}{1.483310in}}%
\pgfpathlineto{\pgfqpoint{1.071040in}{1.489819in}}%
\pgfpathlineto{\pgfqpoint{1.073694in}{1.496223in}}%
\pgfpathlineto{\pgfqpoint{1.077369in}{1.494649in}}%
\pgfpathlineto{\pgfqpoint{1.081148in}{1.493131in}}%
\pgfpathlineto{\pgfqpoint{1.085029in}{1.491671in}}%
\pgfpathlineto{\pgfqpoint{1.089007in}{1.490270in}}%
\pgfpathclose%
\pgfusepath{fill}%
\end{pgfscope}%
\begin{pgfscope}%
\pgfpathrectangle{\pgfqpoint{0.041670in}{0.041670in}}{\pgfqpoint{2.216660in}{2.216660in}}%
\pgfusepath{clip}%
\pgfsetbuttcap%
\pgfsetroundjoin%
\definecolor{currentfill}{rgb}{0.166383,0.690856,0.496502}%
\pgfsetfillcolor{currentfill}%
\pgfsetlinewidth{0.000000pt}%
\definecolor{currentstroke}{rgb}{0.000000,0.000000,0.000000}%
\pgfsetstrokecolor{currentstroke}%
\pgfsetdash{}{0pt}%
\pgfpathmoveto{\pgfqpoint{1.312356in}{1.265402in}}%
\pgfpathlineto{\pgfqpoint{1.314289in}{1.256451in}}%
\pgfpathlineto{\pgfqpoint{1.316222in}{1.247448in}}%
\pgfpathlineto{\pgfqpoint{1.318154in}{1.238395in}}%
\pgfpathlineto{\pgfqpoint{1.320085in}{1.229296in}}%
\pgfpathlineto{\pgfqpoint{1.312316in}{1.227160in}}%
\pgfpathlineto{\pgfqpoint{1.304408in}{1.225146in}}%
\pgfpathlineto{\pgfqpoint{1.296369in}{1.223257in}}%
\pgfpathlineto{\pgfqpoint{1.288208in}{1.221494in}}%
\pgfpathlineto{\pgfqpoint{1.286714in}{1.230711in}}%
\pgfpathlineto{\pgfqpoint{1.285220in}{1.239882in}}%
\pgfpathlineto{\pgfqpoint{1.283726in}{1.249002in}}%
\pgfpathlineto{\pgfqpoint{1.282231in}{1.258069in}}%
\pgfpathlineto{\pgfqpoint{1.289943in}{1.259726in}}%
\pgfpathlineto{\pgfqpoint{1.297540in}{1.261502in}}%
\pgfpathlineto{\pgfqpoint{1.305013in}{1.263394in}}%
\pgfpathlineto{\pgfqpoint{1.312356in}{1.265402in}}%
\pgfpathclose%
\pgfusepath{fill}%
\end{pgfscope}%
\begin{pgfscope}%
\pgfpathrectangle{\pgfqpoint{0.041670in}{0.041670in}}{\pgfqpoint{2.216660in}{2.216660in}}%
\pgfusepath{clip}%
\pgfsetbuttcap%
\pgfsetroundjoin%
\definecolor{currentfill}{rgb}{0.122606,0.585371,0.546557}%
\pgfsetfillcolor{currentfill}%
\pgfsetlinewidth{0.000000pt}%
\definecolor{currentstroke}{rgb}{0.000000,0.000000,0.000000}%
\pgfsetstrokecolor{currentstroke}%
\pgfsetdash{}{0pt}%
\pgfpathmoveto{\pgfqpoint{1.105703in}{1.138778in}}%
\pgfpathlineto{\pgfqpoint{1.104782in}{1.129140in}}%
\pgfpathlineto{\pgfqpoint{1.103862in}{1.119480in}}%
\pgfpathlineto{\pgfqpoint{1.102941in}{1.109800in}}%
\pgfpathlineto{\pgfqpoint{1.102021in}{1.100103in}}%
\pgfpathlineto{\pgfqpoint{1.091960in}{1.101428in}}%
\pgfpathlineto{\pgfqpoint{1.081993in}{1.102912in}}%
\pgfpathlineto{\pgfqpoint{1.072130in}{1.104555in}}%
\pgfpathlineto{\pgfqpoint{1.062381in}{1.106353in}}%
\pgfpathlineto{\pgfqpoint{1.063768in}{1.115968in}}%
\pgfpathlineto{\pgfqpoint{1.065155in}{1.125566in}}%
\pgfpathlineto{\pgfqpoint{1.066543in}{1.135144in}}%
\pgfpathlineto{\pgfqpoint{1.067931in}{1.144700in}}%
\pgfpathlineto{\pgfqpoint{1.077221in}{1.142996in}}%
\pgfpathlineto{\pgfqpoint{1.086619in}{1.141440in}}%
\pgfpathlineto{\pgfqpoint{1.096117in}{1.140033in}}%
\pgfpathlineto{\pgfqpoint{1.105703in}{1.138778in}}%
\pgfpathclose%
\pgfusepath{fill}%
\end{pgfscope}%
\begin{pgfscope}%
\pgfpathrectangle{\pgfqpoint{0.041670in}{0.041670in}}{\pgfqpoint{2.216660in}{2.216660in}}%
\pgfusepath{clip}%
\pgfsetbuttcap%
\pgfsetroundjoin%
\definecolor{currentfill}{rgb}{0.762373,0.876424,0.137064}%
\pgfsetfillcolor{currentfill}%
\pgfsetlinewidth{0.000000pt}%
\definecolor{currentstroke}{rgb}{0.000000,0.000000,0.000000}%
\pgfsetstrokecolor{currentstroke}%
\pgfsetdash{}{0pt}%
\pgfpathmoveto{\pgfqpoint{1.107206in}{1.538783in}}%
\pgfpathlineto{\pgfqpoint{1.104930in}{1.533115in}}%
\pgfpathlineto{\pgfqpoint{1.102654in}{1.527332in}}%
\pgfpathlineto{\pgfqpoint{1.100379in}{1.521434in}}%
\pgfpathlineto{\pgfqpoint{1.098104in}{1.515423in}}%
\pgfpathlineto{\pgfqpoint{1.094522in}{1.516679in}}%
\pgfpathlineto{\pgfqpoint{1.091027in}{1.517988in}}%
\pgfpathlineto{\pgfqpoint{1.087623in}{1.519349in}}%
\pgfpathlineto{\pgfqpoint{1.084313in}{1.520760in}}%
\pgfpathlineto{\pgfqpoint{1.086970in}{1.526618in}}%
\pgfpathlineto{\pgfqpoint{1.089626in}{1.532364in}}%
\pgfpathlineto{\pgfqpoint{1.092284in}{1.537995in}}%
\pgfpathlineto{\pgfqpoint{1.094942in}{1.543510in}}%
\pgfpathlineto{\pgfqpoint{1.097885in}{1.542260in}}%
\pgfpathlineto{\pgfqpoint{1.100913in}{1.541055in}}%
\pgfpathlineto{\pgfqpoint{1.104021in}{1.539895in}}%
\pgfpathlineto{\pgfqpoint{1.107206in}{1.538783in}}%
\pgfpathclose%
\pgfusepath{fill}%
\end{pgfscope}%
\begin{pgfscope}%
\pgfpathrectangle{\pgfqpoint{0.041670in}{0.041670in}}{\pgfqpoint{2.216660in}{2.216660in}}%
\pgfusepath{clip}%
\pgfsetbuttcap%
\pgfsetroundjoin%
\definecolor{currentfill}{rgb}{0.699415,0.867117,0.175971}%
\pgfsetfillcolor{currentfill}%
\pgfsetlinewidth{0.000000pt}%
\definecolor{currentstroke}{rgb}{0.000000,0.000000,0.000000}%
\pgfsetstrokecolor{currentstroke}%
\pgfsetdash{}{0pt}%
\pgfpathmoveto{\pgfqpoint{1.098104in}{1.515423in}}%
\pgfpathlineto{\pgfqpoint{1.095829in}{1.509299in}}%
\pgfpathlineto{\pgfqpoint{1.093555in}{1.503065in}}%
\pgfpathlineto{\pgfqpoint{1.091281in}{1.496722in}}%
\pgfpathlineto{\pgfqpoint{1.089007in}{1.490270in}}%
\pgfpathlineto{\pgfqpoint{1.085029in}{1.491671in}}%
\pgfpathlineto{\pgfqpoint{1.081148in}{1.493131in}}%
\pgfpathlineto{\pgfqpoint{1.077369in}{1.494649in}}%
\pgfpathlineto{\pgfqpoint{1.073694in}{1.496223in}}%
\pgfpathlineto{\pgfqpoint{1.076348in}{1.502520in}}%
\pgfpathlineto{\pgfqpoint{1.079002in}{1.508710in}}%
\pgfpathlineto{\pgfqpoint{1.081657in}{1.514790in}}%
\pgfpathlineto{\pgfqpoint{1.084313in}{1.520760in}}%
\pgfpathlineto{\pgfqpoint{1.087623in}{1.519349in}}%
\pgfpathlineto{\pgfqpoint{1.091027in}{1.517988in}}%
\pgfpathlineto{\pgfqpoint{1.094522in}{1.516679in}}%
\pgfpathlineto{\pgfqpoint{1.098104in}{1.515423in}}%
\pgfpathclose%
\pgfusepath{fill}%
\end{pgfscope}%
\begin{pgfscope}%
\pgfpathrectangle{\pgfqpoint{0.041670in}{0.041670in}}{\pgfqpoint{2.216660in}{2.216660in}}%
\pgfusepath{clip}%
\pgfsetbuttcap%
\pgfsetroundjoin%
\definecolor{currentfill}{rgb}{0.935904,0.898570,0.108131}%
\pgfsetfillcolor{currentfill}%
\pgfsetlinewidth{0.000000pt}%
\definecolor{currentstroke}{rgb}{0.000000,0.000000,0.000000}%
\pgfsetstrokecolor{currentstroke}%
\pgfsetdash{}{0pt}%
\pgfpathmoveto{\pgfqpoint{1.196498in}{1.608485in}}%
\pgfpathlineto{\pgfqpoint{1.197534in}{1.604525in}}%
\pgfpathlineto{\pgfqpoint{1.198572in}{1.600435in}}%
\pgfpathlineto{\pgfqpoint{1.199609in}{1.596215in}}%
\pgfpathlineto{\pgfqpoint{1.200647in}{1.591865in}}%
\pgfpathlineto{\pgfqpoint{1.198246in}{1.591576in}}%
\pgfpathlineto{\pgfqpoint{1.195826in}{1.591322in}}%
\pgfpathlineto{\pgfqpoint{1.193390in}{1.591105in}}%
\pgfpathlineto{\pgfqpoint{1.190941in}{1.590923in}}%
\pgfpathlineto{\pgfqpoint{1.190390in}{1.595321in}}%
\pgfpathlineto{\pgfqpoint{1.189839in}{1.599589in}}%
\pgfpathlineto{\pgfqpoint{1.189288in}{1.603727in}}%
\pgfpathlineto{\pgfqpoint{1.188738in}{1.607734in}}%
\pgfpathlineto{\pgfqpoint{1.190696in}{1.607879in}}%
\pgfpathlineto{\pgfqpoint{1.192643in}{1.608052in}}%
\pgfpathlineto{\pgfqpoint{1.194578in}{1.608255in}}%
\pgfpathlineto{\pgfqpoint{1.196498in}{1.608485in}}%
\pgfpathclose%
\pgfusepath{fill}%
\end{pgfscope}%
\begin{pgfscope}%
\pgfpathrectangle{\pgfqpoint{0.041670in}{0.041670in}}{\pgfqpoint{2.216660in}{2.216660in}}%
\pgfusepath{clip}%
\pgfsetbuttcap%
\pgfsetroundjoin%
\definecolor{currentfill}{rgb}{0.935904,0.898570,0.108131}%
\pgfsetfillcolor{currentfill}%
\pgfsetlinewidth{0.000000pt}%
\definecolor{currentstroke}{rgb}{0.000000,0.000000,0.000000}%
\pgfsetstrokecolor{currentstroke}%
\pgfsetdash{}{0pt}%
\pgfpathmoveto{\pgfqpoint{1.172920in}{1.607630in}}%
\pgfpathlineto{\pgfqpoint{1.172479in}{1.603616in}}%
\pgfpathlineto{\pgfqpoint{1.172038in}{1.599471in}}%
\pgfpathlineto{\pgfqpoint{1.171597in}{1.595197in}}%
\pgfpathlineto{\pgfqpoint{1.171156in}{1.590793in}}%
\pgfpathlineto{\pgfqpoint{1.168696in}{1.590942in}}%
\pgfpathlineto{\pgfqpoint{1.166248in}{1.591127in}}%
\pgfpathlineto{\pgfqpoint{1.163814in}{1.591349in}}%
\pgfpathlineto{\pgfqpoint{1.161396in}{1.591606in}}%
\pgfpathlineto{\pgfqpoint{1.162327in}{1.595969in}}%
\pgfpathlineto{\pgfqpoint{1.163258in}{1.600202in}}%
\pgfpathlineto{\pgfqpoint{1.164188in}{1.604306in}}%
\pgfpathlineto{\pgfqpoint{1.165118in}{1.608279in}}%
\pgfpathlineto{\pgfqpoint{1.167051in}{1.608073in}}%
\pgfpathlineto{\pgfqpoint{1.168997in}{1.607897in}}%
\pgfpathlineto{\pgfqpoint{1.170954in}{1.607749in}}%
\pgfpathlineto{\pgfqpoint{1.172920in}{1.607630in}}%
\pgfpathclose%
\pgfusepath{fill}%
\end{pgfscope}%
\begin{pgfscope}%
\pgfpathrectangle{\pgfqpoint{0.041670in}{0.041670in}}{\pgfqpoint{2.216660in}{2.216660in}}%
\pgfusepath{clip}%
\pgfsetbuttcap%
\pgfsetroundjoin%
\definecolor{currentfill}{rgb}{0.855810,0.888601,0.097452}%
\pgfsetfillcolor{currentfill}%
\pgfsetlinewidth{0.000000pt}%
\definecolor{currentstroke}{rgb}{0.000000,0.000000,0.000000}%
\pgfsetstrokecolor{currentstroke}%
\pgfsetdash{}{0pt}%
\pgfpathmoveto{\pgfqpoint{1.236618in}{1.580557in}}%
\pgfpathlineto{\pgfqpoint{1.238983in}{1.575886in}}%
\pgfpathlineto{\pgfqpoint{1.241348in}{1.571091in}}%
\pgfpathlineto{\pgfqpoint{1.243714in}{1.566173in}}%
\pgfpathlineto{\pgfqpoint{1.246079in}{1.561132in}}%
\pgfpathlineto{\pgfqpoint{1.243284in}{1.560167in}}%
\pgfpathlineto{\pgfqpoint{1.240424in}{1.559243in}}%
\pgfpathlineto{\pgfqpoint{1.237503in}{1.558362in}}%
\pgfpathlineto{\pgfqpoint{1.234524in}{1.557526in}}%
\pgfpathlineto{\pgfqpoint{1.232571in}{1.562698in}}%
\pgfpathlineto{\pgfqpoint{1.230618in}{1.567747in}}%
\pgfpathlineto{\pgfqpoint{1.228665in}{1.572674in}}%
\pgfpathlineto{\pgfqpoint{1.226712in}{1.577476in}}%
\pgfpathlineto{\pgfqpoint{1.229266in}{1.578191in}}%
\pgfpathlineto{\pgfqpoint{1.231770in}{1.578943in}}%
\pgfpathlineto{\pgfqpoint{1.234221in}{1.579732in}}%
\pgfpathlineto{\pgfqpoint{1.236618in}{1.580557in}}%
\pgfpathclose%
\pgfusepath{fill}%
\end{pgfscope}%
\begin{pgfscope}%
\pgfpathrectangle{\pgfqpoint{0.041670in}{0.041670in}}{\pgfqpoint{2.216660in}{2.216660in}}%
\pgfusepath{clip}%
\pgfsetbuttcap%
\pgfsetroundjoin%
\definecolor{currentfill}{rgb}{0.935904,0.898570,0.108131}%
\pgfsetfillcolor{currentfill}%
\pgfsetlinewidth{0.000000pt}%
\definecolor{currentstroke}{rgb}{0.000000,0.000000,0.000000}%
\pgfsetstrokecolor{currentstroke}%
\pgfsetdash{}{0pt}%
\pgfpathmoveto{\pgfqpoint{1.188738in}{1.607734in}}%
\pgfpathlineto{\pgfqpoint{1.189288in}{1.603727in}}%
\pgfpathlineto{\pgfqpoint{1.189839in}{1.599589in}}%
\pgfpathlineto{\pgfqpoint{1.190390in}{1.595321in}}%
\pgfpathlineto{\pgfqpoint{1.190941in}{1.590923in}}%
\pgfpathlineto{\pgfqpoint{1.188480in}{1.590778in}}%
\pgfpathlineto{\pgfqpoint{1.186011in}{1.590670in}}%
\pgfpathlineto{\pgfqpoint{1.183536in}{1.590599in}}%
\pgfpathlineto{\pgfqpoint{1.181057in}{1.590564in}}%
\pgfpathlineto{\pgfqpoint{1.181002in}{1.594979in}}%
\pgfpathlineto{\pgfqpoint{1.180947in}{1.599266in}}%
\pgfpathlineto{\pgfqpoint{1.180891in}{1.603422in}}%
\pgfpathlineto{\pgfqpoint{1.180836in}{1.607448in}}%
\pgfpathlineto{\pgfqpoint{1.182818in}{1.607475in}}%
\pgfpathlineto{\pgfqpoint{1.184797in}{1.607532in}}%
\pgfpathlineto{\pgfqpoint{1.186771in}{1.607619in}}%
\pgfpathlineto{\pgfqpoint{1.188738in}{1.607734in}}%
\pgfpathclose%
\pgfusepath{fill}%
\end{pgfscope}%
\begin{pgfscope}%
\pgfpathrectangle{\pgfqpoint{0.041670in}{0.041670in}}{\pgfqpoint{2.216660in}{2.216660in}}%
\pgfusepath{clip}%
\pgfsetbuttcap%
\pgfsetroundjoin%
\definecolor{currentfill}{rgb}{0.133743,0.548535,0.553541}%
\pgfsetfillcolor{currentfill}%
\pgfsetlinewidth{0.000000pt}%
\definecolor{currentstroke}{rgb}{0.000000,0.000000,0.000000}%
\pgfsetstrokecolor{currentstroke}%
\pgfsetdash{}{0pt}%
\pgfpathmoveto{\pgfqpoint{1.266836in}{1.101273in}}%
\pgfpathlineto{\pgfqpoint{1.267861in}{1.091577in}}%
\pgfpathlineto{\pgfqpoint{1.268886in}{1.081869in}}%
\pgfpathlineto{\pgfqpoint{1.269911in}{1.072153in}}%
\pgfpathlineto{\pgfqpoint{1.270936in}{1.062432in}}%
\pgfpathlineto{\pgfqpoint{1.260390in}{1.061055in}}%
\pgfpathlineto{\pgfqpoint{1.249759in}{1.059849in}}%
\pgfpathlineto{\pgfqpoint{1.239053in}{1.058813in}}%
\pgfpathlineto{\pgfqpoint{1.228283in}{1.057951in}}%
\pgfpathlineto{\pgfqpoint{1.227739in}{1.067729in}}%
\pgfpathlineto{\pgfqpoint{1.227194in}{1.077501in}}%
\pgfpathlineto{\pgfqpoint{1.226649in}{1.087265in}}%
\pgfpathlineto{\pgfqpoint{1.226104in}{1.097017in}}%
\pgfpathlineto{\pgfqpoint{1.236389in}{1.097837in}}%
\pgfpathlineto{\pgfqpoint{1.246612in}{1.098820in}}%
\pgfpathlineto{\pgfqpoint{1.256765in}{1.099966in}}%
\pgfpathlineto{\pgfqpoint{1.266836in}{1.101273in}}%
\pgfpathclose%
\pgfusepath{fill}%
\end{pgfscope}%
\begin{pgfscope}%
\pgfpathrectangle{\pgfqpoint{0.041670in}{0.041670in}}{\pgfqpoint{2.216660in}{2.216660in}}%
\pgfusepath{clip}%
\pgfsetbuttcap%
\pgfsetroundjoin%
\definecolor{currentfill}{rgb}{0.935904,0.898570,0.108131}%
\pgfsetfillcolor{currentfill}%
\pgfsetlinewidth{0.000000pt}%
\definecolor{currentstroke}{rgb}{0.000000,0.000000,0.000000}%
\pgfsetstrokecolor{currentstroke}%
\pgfsetdash{}{0pt}%
\pgfpathmoveto{\pgfqpoint{1.180836in}{1.607448in}}%
\pgfpathlineto{\pgfqpoint{1.180891in}{1.603422in}}%
\pgfpathlineto{\pgfqpoint{1.180947in}{1.599266in}}%
\pgfpathlineto{\pgfqpoint{1.181002in}{1.594979in}}%
\pgfpathlineto{\pgfqpoint{1.181057in}{1.590564in}}%
\pgfpathlineto{\pgfqpoint{1.178577in}{1.590566in}}%
\pgfpathlineto{\pgfqpoint{1.176098in}{1.590605in}}%
\pgfpathlineto{\pgfqpoint{1.173624in}{1.590680in}}%
\pgfpathlineto{\pgfqpoint{1.171156in}{1.590793in}}%
\pgfpathlineto{\pgfqpoint{1.171597in}{1.595197in}}%
\pgfpathlineto{\pgfqpoint{1.172038in}{1.599471in}}%
\pgfpathlineto{\pgfqpoint{1.172479in}{1.603616in}}%
\pgfpathlineto{\pgfqpoint{1.172920in}{1.607630in}}%
\pgfpathlineto{\pgfqpoint{1.174893in}{1.607541in}}%
\pgfpathlineto{\pgfqpoint{1.176872in}{1.607480in}}%
\pgfpathlineto{\pgfqpoint{1.178853in}{1.607449in}}%
\pgfpathlineto{\pgfqpoint{1.180836in}{1.607448in}}%
\pgfpathclose%
\pgfusepath{fill}%
\end{pgfscope}%
\begin{pgfscope}%
\pgfpathrectangle{\pgfqpoint{0.041670in}{0.041670in}}{\pgfqpoint{2.216660in}{2.216660in}}%
\pgfusepath{clip}%
\pgfsetbuttcap%
\pgfsetroundjoin%
\definecolor{currentfill}{rgb}{0.896320,0.893616,0.096335}%
\pgfsetfillcolor{currentfill}%
\pgfsetlinewidth{0.000000pt}%
\definecolor{currentstroke}{rgb}{0.000000,0.000000,0.000000}%
\pgfsetstrokecolor{currentstroke}%
\pgfsetdash{}{0pt}%
\pgfpathmoveto{\pgfqpoint{1.218902in}{1.595425in}}%
\pgfpathlineto{\pgfqpoint{1.220854in}{1.591128in}}%
\pgfpathlineto{\pgfqpoint{1.222806in}{1.586704in}}%
\pgfpathlineto{\pgfqpoint{1.224759in}{1.582153in}}%
\pgfpathlineto{\pgfqpoint{1.226712in}{1.577476in}}%
\pgfpathlineto{\pgfqpoint{1.224110in}{1.576799in}}%
\pgfpathlineto{\pgfqpoint{1.221464in}{1.576162in}}%
\pgfpathlineto{\pgfqpoint{1.218776in}{1.575564in}}%
\pgfpathlineto{\pgfqpoint{1.216048in}{1.575007in}}%
\pgfpathlineto{\pgfqpoint{1.214540in}{1.579788in}}%
\pgfpathlineto{\pgfqpoint{1.213032in}{1.584444in}}%
\pgfpathlineto{\pgfqpoint{1.211525in}{1.588973in}}%
\pgfpathlineto{\pgfqpoint{1.210017in}{1.593374in}}%
\pgfpathlineto{\pgfqpoint{1.212290in}{1.593837in}}%
\pgfpathlineto{\pgfqpoint{1.214530in}{1.594334in}}%
\pgfpathlineto{\pgfqpoint{1.216734in}{1.594863in}}%
\pgfpathlineto{\pgfqpoint{1.218902in}{1.595425in}}%
\pgfpathclose%
\pgfusepath{fill}%
\end{pgfscope}%
\begin{pgfscope}%
\pgfpathrectangle{\pgfqpoint{0.041670in}{0.041670in}}{\pgfqpoint{2.216660in}{2.216660in}}%
\pgfusepath{clip}%
\pgfsetbuttcap%
\pgfsetroundjoin%
\definecolor{currentfill}{rgb}{0.344074,0.780029,0.397381}%
\pgfsetfillcolor{currentfill}%
\pgfsetlinewidth{0.000000pt}%
\definecolor{currentstroke}{rgb}{0.000000,0.000000,0.000000}%
\pgfsetstrokecolor{currentstroke}%
\pgfsetdash{}{0pt}%
\pgfpathmoveto{\pgfqpoint{1.076190in}{1.366488in}}%
\pgfpathlineto{\pgfqpoint{1.074343in}{1.358306in}}%
\pgfpathlineto{\pgfqpoint{1.072497in}{1.350046in}}%
\pgfpathlineto{\pgfqpoint{1.070652in}{1.341710in}}%
\pgfpathlineto{\pgfqpoint{1.068807in}{1.333300in}}%
\pgfpathlineto{\pgfqpoint{1.062333in}{1.335066in}}%
\pgfpathlineto{\pgfqpoint{1.055982in}{1.336930in}}%
\pgfpathlineto{\pgfqpoint{1.049759in}{1.338891in}}%
\pgfpathlineto{\pgfqpoint{1.043672in}{1.340946in}}%
\pgfpathlineto{\pgfqpoint{1.045931in}{1.349220in}}%
\pgfpathlineto{\pgfqpoint{1.048191in}{1.357420in}}%
\pgfpathlineto{\pgfqpoint{1.050452in}{1.365544in}}%
\pgfpathlineto{\pgfqpoint{1.052714in}{1.373591in}}%
\pgfpathlineto{\pgfqpoint{1.058400in}{1.371682in}}%
\pgfpathlineto{\pgfqpoint{1.064212in}{1.369860in}}%
\pgfpathlineto{\pgfqpoint{1.070144in}{1.368128in}}%
\pgfpathlineto{\pgfqpoint{1.076190in}{1.366488in}}%
\pgfpathclose%
\pgfusepath{fill}%
\end{pgfscope}%
\begin{pgfscope}%
\pgfpathrectangle{\pgfqpoint{0.041670in}{0.041670in}}{\pgfqpoint{2.216660in}{2.216660in}}%
\pgfusepath{clip}%
\pgfsetbuttcap%
\pgfsetroundjoin%
\definecolor{currentfill}{rgb}{0.120081,0.622161,0.534946}%
\pgfsetfillcolor{currentfill}%
\pgfsetlinewidth{0.000000pt}%
\definecolor{currentstroke}{rgb}{0.000000,0.000000,0.000000}%
\pgfsetstrokecolor{currentstroke}%
\pgfsetdash{}{0pt}%
\pgfpathmoveto{\pgfqpoint{1.294176in}{1.184197in}}%
\pgfpathlineto{\pgfqpoint{1.295667in}{1.174779in}}%
\pgfpathlineto{\pgfqpoint{1.297157in}{1.165327in}}%
\pgfpathlineto{\pgfqpoint{1.298647in}{1.155846in}}%
\pgfpathlineto{\pgfqpoint{1.300136in}{1.146337in}}%
\pgfpathlineto{\pgfqpoint{1.290952in}{1.144503in}}%
\pgfpathlineto{\pgfqpoint{1.281650in}{1.142816in}}%
\pgfpathlineto{\pgfqpoint{1.272240in}{1.141276in}}%
\pgfpathlineto{\pgfqpoint{1.262732in}{1.139886in}}%
\pgfpathlineto{\pgfqpoint{1.261705in}{1.149483in}}%
\pgfpathlineto{\pgfqpoint{1.260678in}{1.159053in}}%
\pgfpathlineto{\pgfqpoint{1.259651in}{1.168593in}}%
\pgfpathlineto{\pgfqpoint{1.258623in}{1.178099in}}%
\pgfpathlineto{\pgfqpoint{1.267660in}{1.179413in}}%
\pgfpathlineto{\pgfqpoint{1.276604in}{1.180868in}}%
\pgfpathlineto{\pgfqpoint{1.285446in}{1.182463in}}%
\pgfpathlineto{\pgfqpoint{1.294176in}{1.184197in}}%
\pgfpathclose%
\pgfusepath{fill}%
\end{pgfscope}%
\begin{pgfscope}%
\pgfpathrectangle{\pgfqpoint{0.041670in}{0.041670in}}{\pgfqpoint{2.216660in}{2.216660in}}%
\pgfusepath{clip}%
\pgfsetbuttcap%
\pgfsetroundjoin%
\definecolor{currentfill}{rgb}{0.147607,0.511733,0.557049}%
\pgfsetfillcolor{currentfill}%
\pgfsetlinewidth{0.000000pt}%
\definecolor{currentstroke}{rgb}{0.000000,0.000000,0.000000}%
\pgfsetstrokecolor{currentstroke}%
\pgfsetdash{}{0pt}%
\pgfpathmoveto{\pgfqpoint{1.228283in}{1.057951in}}%
\pgfpathlineto{\pgfqpoint{1.228828in}{1.048169in}}%
\pgfpathlineto{\pgfqpoint{1.229372in}{1.038387in}}%
\pgfpathlineto{\pgfqpoint{1.229916in}{1.028608in}}%
\pgfpathlineto{\pgfqpoint{1.230461in}{1.018834in}}%
\pgfpathlineto{\pgfqpoint{1.219152in}{1.018109in}}%
\pgfpathlineto{\pgfqpoint{1.207802in}{1.017568in}}%
\pgfpathlineto{\pgfqpoint{1.196421in}{1.017209in}}%
\pgfpathlineto{\pgfqpoint{1.185023in}{1.017035in}}%
\pgfpathlineto{\pgfqpoint{1.184969in}{1.026831in}}%
\pgfpathlineto{\pgfqpoint{1.184914in}{1.036632in}}%
\pgfpathlineto{\pgfqpoint{1.184859in}{1.046436in}}%
\pgfpathlineto{\pgfqpoint{1.184805in}{1.056239in}}%
\pgfpathlineto{\pgfqpoint{1.195711in}{1.056405in}}%
\pgfpathlineto{\pgfqpoint{1.206601in}{1.056746in}}%
\pgfpathlineto{\pgfqpoint{1.217462in}{1.057261in}}%
\pgfpathlineto{\pgfqpoint{1.228283in}{1.057951in}}%
\pgfpathclose%
\pgfusepath{fill}%
\end{pgfscope}%
\begin{pgfscope}%
\pgfpathrectangle{\pgfqpoint{0.041670in}{0.041670in}}{\pgfqpoint{2.216660in}{2.216660in}}%
\pgfusepath{clip}%
\pgfsetbuttcap%
\pgfsetroundjoin%
\definecolor{currentfill}{rgb}{0.412913,0.803041,0.357269}%
\pgfsetfillcolor{currentfill}%
\pgfsetlinewidth{0.000000pt}%
\definecolor{currentstroke}{rgb}{0.000000,0.000000,0.000000}%
\pgfsetstrokecolor{currentstroke}%
\pgfsetdash{}{0pt}%
\pgfpathmoveto{\pgfqpoint{1.302734in}{1.406598in}}%
\pgfpathlineto{\pgfqpoint{1.305086in}{1.398915in}}%
\pgfpathlineto{\pgfqpoint{1.307438in}{1.391146in}}%
\pgfpathlineto{\pgfqpoint{1.309789in}{1.383294in}}%
\pgfpathlineto{\pgfqpoint{1.312139in}{1.375360in}}%
\pgfpathlineto{\pgfqpoint{1.306570in}{1.373374in}}%
\pgfpathlineto{\pgfqpoint{1.300870in}{1.371475in}}%
\pgfpathlineto{\pgfqpoint{1.295044in}{1.369663in}}%
\pgfpathlineto{\pgfqpoint{1.289099in}{1.367941in}}%
\pgfpathlineto{\pgfqpoint{1.287157in}{1.376016in}}%
\pgfpathlineto{\pgfqpoint{1.285214in}{1.384008in}}%
\pgfpathlineto{\pgfqpoint{1.283270in}{1.391917in}}%
\pgfpathlineto{\pgfqpoint{1.281325in}{1.399740in}}%
\pgfpathlineto{\pgfqpoint{1.286849in}{1.401331in}}%
\pgfpathlineto{\pgfqpoint{1.292261in}{1.403006in}}%
\pgfpathlineto{\pgfqpoint{1.297558in}{1.404762in}}%
\pgfpathlineto{\pgfqpoint{1.302734in}{1.406598in}}%
\pgfpathclose%
\pgfusepath{fill}%
\end{pgfscope}%
\begin{pgfscope}%
\pgfpathrectangle{\pgfqpoint{0.041670in}{0.041670in}}{\pgfqpoint{2.216660in}{2.216660in}}%
\pgfusepath{clip}%
\pgfsetbuttcap%
\pgfsetroundjoin%
\definecolor{currentfill}{rgb}{0.166383,0.690856,0.496502}%
\pgfsetfillcolor{currentfill}%
\pgfsetlinewidth{0.000000pt}%
\definecolor{currentstroke}{rgb}{0.000000,0.000000,0.000000}%
\pgfsetstrokecolor{currentstroke}%
\pgfsetdash{}{0pt}%
\pgfpathmoveto{\pgfqpoint{1.084625in}{1.256698in}}%
\pgfpathlineto{\pgfqpoint{1.083231in}{1.247609in}}%
\pgfpathlineto{\pgfqpoint{1.081838in}{1.238467in}}%
\pgfpathlineto{\pgfqpoint{1.080445in}{1.229275in}}%
\pgfpathlineto{\pgfqpoint{1.079053in}{1.220035in}}%
\pgfpathlineto{\pgfqpoint{1.070790in}{1.221683in}}%
\pgfpathlineto{\pgfqpoint{1.062641in}{1.223460in}}%
\pgfpathlineto{\pgfqpoint{1.054617in}{1.225364in}}%
\pgfpathlineto{\pgfqpoint{1.046724in}{1.227391in}}%
\pgfpathlineto{\pgfqpoint{1.048560in}{1.236519in}}%
\pgfpathlineto{\pgfqpoint{1.050397in}{1.245601in}}%
\pgfpathlineto{\pgfqpoint{1.052235in}{1.254632in}}%
\pgfpathlineto{\pgfqpoint{1.054074in}{1.263612in}}%
\pgfpathlineto{\pgfqpoint{1.061533in}{1.261706in}}%
\pgfpathlineto{\pgfqpoint{1.069117in}{1.259917in}}%
\pgfpathlineto{\pgfqpoint{1.076817in}{1.258248in}}%
\pgfpathlineto{\pgfqpoint{1.084625in}{1.256698in}}%
\pgfpathclose%
\pgfusepath{fill}%
\end{pgfscope}%
\begin{pgfscope}%
\pgfpathrectangle{\pgfqpoint{0.041670in}{0.041670in}}{\pgfqpoint{2.216660in}{2.216660in}}%
\pgfusepath{clip}%
\pgfsetbuttcap%
\pgfsetroundjoin%
\definecolor{currentfill}{rgb}{0.896320,0.893616,0.096335}%
\pgfsetfillcolor{currentfill}%
\pgfsetlinewidth{0.000000pt}%
\definecolor{currentstroke}{rgb}{0.000000,0.000000,0.000000}%
\pgfsetstrokecolor{currentstroke}%
\pgfsetdash{}{0pt}%
\pgfpathmoveto{\pgfqpoint{1.151938in}{1.592991in}}%
\pgfpathlineto{\pgfqpoint{1.150533in}{1.588570in}}%
\pgfpathlineto{\pgfqpoint{1.149128in}{1.584022in}}%
\pgfpathlineto{\pgfqpoint{1.147723in}{1.579347in}}%
\pgfpathlineto{\pgfqpoint{1.146317in}{1.574545in}}%
\pgfpathlineto{\pgfqpoint{1.143557in}{1.575067in}}%
\pgfpathlineto{\pgfqpoint{1.140833in}{1.575629in}}%
\pgfpathlineto{\pgfqpoint{1.138149in}{1.576231in}}%
\pgfpathlineto{\pgfqpoint{1.135508in}{1.576873in}}%
\pgfpathlineto{\pgfqpoint{1.137364in}{1.581575in}}%
\pgfpathlineto{\pgfqpoint{1.139221in}{1.586152in}}%
\pgfpathlineto{\pgfqpoint{1.141077in}{1.590602in}}%
\pgfpathlineto{\pgfqpoint{1.142933in}{1.594924in}}%
\pgfpathlineto{\pgfqpoint{1.145133in}{1.594391in}}%
\pgfpathlineto{\pgfqpoint{1.147369in}{1.593891in}}%
\pgfpathlineto{\pgfqpoint{1.149638in}{1.593424in}}%
\pgfpathlineto{\pgfqpoint{1.151938in}{1.592991in}}%
\pgfpathclose%
\pgfusepath{fill}%
\end{pgfscope}%
\begin{pgfscope}%
\pgfpathrectangle{\pgfqpoint{0.041670in}{0.041670in}}{\pgfqpoint{2.216660in}{2.216660in}}%
\pgfusepath{clip}%
\pgfsetbuttcap%
\pgfsetroundjoin%
\definecolor{currentfill}{rgb}{0.855810,0.888601,0.097452}%
\pgfsetfillcolor{currentfill}%
\pgfsetlinewidth{0.000000pt}%
\definecolor{currentstroke}{rgb}{0.000000,0.000000,0.000000}%
\pgfsetstrokecolor{currentstroke}%
\pgfsetdash{}{0pt}%
\pgfpathmoveto{\pgfqpoint{1.135508in}{1.576873in}}%
\pgfpathlineto{\pgfqpoint{1.133651in}{1.572045in}}%
\pgfpathlineto{\pgfqpoint{1.131795in}{1.567093in}}%
\pgfpathlineto{\pgfqpoint{1.129938in}{1.562017in}}%
\pgfpathlineto{\pgfqpoint{1.128081in}{1.556819in}}%
\pgfpathlineto{\pgfqpoint{1.125052in}{1.557616in}}%
\pgfpathlineto{\pgfqpoint{1.122079in}{1.558458in}}%
\pgfpathlineto{\pgfqpoint{1.119165in}{1.559343in}}%
\pgfpathlineto{\pgfqpoint{1.116312in}{1.560272in}}%
\pgfpathlineto{\pgfqpoint{1.118589in}{1.565344in}}%
\pgfpathlineto{\pgfqpoint{1.120866in}{1.570293in}}%
\pgfpathlineto{\pgfqpoint{1.123143in}{1.575120in}}%
\pgfpathlineto{\pgfqpoint{1.125419in}{1.579822in}}%
\pgfpathlineto{\pgfqpoint{1.127865in}{1.579029in}}%
\pgfpathlineto{\pgfqpoint{1.130363in}{1.578272in}}%
\pgfpathlineto{\pgfqpoint{1.132912in}{1.577554in}}%
\pgfpathlineto{\pgfqpoint{1.135508in}{1.576873in}}%
\pgfpathclose%
\pgfusepath{fill}%
\end{pgfscope}%
\begin{pgfscope}%
\pgfpathrectangle{\pgfqpoint{0.041670in}{0.041670in}}{\pgfqpoint{2.216660in}{2.216660in}}%
\pgfusepath{clip}%
\pgfsetbuttcap%
\pgfsetroundjoin%
\definecolor{currentfill}{rgb}{0.147607,0.511733,0.557049}%
\pgfsetfillcolor{currentfill}%
\pgfsetlinewidth{0.000000pt}%
\definecolor{currentstroke}{rgb}{0.000000,0.000000,0.000000}%
\pgfsetstrokecolor{currentstroke}%
\pgfsetdash{}{0pt}%
\pgfpathmoveto{\pgfqpoint{1.184805in}{1.056239in}}%
\pgfpathlineto{\pgfqpoint{1.184859in}{1.046436in}}%
\pgfpathlineto{\pgfqpoint{1.184914in}{1.036632in}}%
\pgfpathlineto{\pgfqpoint{1.184969in}{1.026831in}}%
\pgfpathlineto{\pgfqpoint{1.185023in}{1.017035in}}%
\pgfpathlineto{\pgfqpoint{1.173620in}{1.017046in}}%
\pgfpathlineto{\pgfqpoint{1.162223in}{1.017240in}}%
\pgfpathlineto{\pgfqpoint{1.150845in}{1.017619in}}%
\pgfpathlineto{\pgfqpoint{1.139499in}{1.018181in}}%
\pgfpathlineto{\pgfqpoint{1.139935in}{1.027962in}}%
\pgfpathlineto{\pgfqpoint{1.140371in}{1.037750in}}%
\pgfpathlineto{\pgfqpoint{1.140807in}{1.047539in}}%
\pgfpathlineto{\pgfqpoint{1.141243in}{1.057329in}}%
\pgfpathlineto{\pgfqpoint{1.152100in}{1.056794in}}%
\pgfpathlineto{\pgfqpoint{1.162987in}{1.056434in}}%
\pgfpathlineto{\pgfqpoint{1.173893in}{1.056249in}}%
\pgfpathlineto{\pgfqpoint{1.184805in}{1.056239in}}%
\pgfpathclose%
\pgfusepath{fill}%
\end{pgfscope}%
\begin{pgfscope}%
\pgfpathrectangle{\pgfqpoint{0.041670in}{0.041670in}}{\pgfqpoint{2.216660in}{2.216660in}}%
\pgfusepath{clip}%
\pgfsetbuttcap%
\pgfsetroundjoin%
\definecolor{currentfill}{rgb}{0.133743,0.548535,0.553541}%
\pgfsetfillcolor{currentfill}%
\pgfsetlinewidth{0.000000pt}%
\definecolor{currentstroke}{rgb}{0.000000,0.000000,0.000000}%
\pgfsetstrokecolor{currentstroke}%
\pgfsetdash{}{0pt}%
\pgfpathmoveto{\pgfqpoint{1.142988in}{1.096427in}}%
\pgfpathlineto{\pgfqpoint{1.142552in}{1.086667in}}%
\pgfpathlineto{\pgfqpoint{1.142115in}{1.076896in}}%
\pgfpathlineto{\pgfqpoint{1.141679in}{1.067115in}}%
\pgfpathlineto{\pgfqpoint{1.141243in}{1.057329in}}%
\pgfpathlineto{\pgfqpoint{1.130427in}{1.058038in}}%
\pgfpathlineto{\pgfqpoint{1.119664in}{1.058920in}}%
\pgfpathlineto{\pgfqpoint{1.108966in}{1.059974in}}%
\pgfpathlineto{\pgfqpoint{1.098343in}{1.061200in}}%
\pgfpathlineto{\pgfqpoint{1.099262in}{1.070937in}}%
\pgfpathlineto{\pgfqpoint{1.100182in}{1.080669in}}%
\pgfpathlineto{\pgfqpoint{1.101101in}{1.090392in}}%
\pgfpathlineto{\pgfqpoint{1.102021in}{1.100103in}}%
\pgfpathlineto{\pgfqpoint{1.112166in}{1.098939in}}%
\pgfpathlineto{\pgfqpoint{1.122382in}{1.097938in}}%
\pgfpathlineto{\pgfqpoint{1.132660in}{1.097100in}}%
\pgfpathlineto{\pgfqpoint{1.142988in}{1.096427in}}%
\pgfpathclose%
\pgfusepath{fill}%
\end{pgfscope}%
\begin{pgfscope}%
\pgfpathrectangle{\pgfqpoint{0.041670in}{0.041670in}}{\pgfqpoint{2.216660in}{2.216660in}}%
\pgfusepath{clip}%
\pgfsetbuttcap%
\pgfsetroundjoin%
\definecolor{currentfill}{rgb}{0.267004,0.004874,0.329415}%
\pgfsetfillcolor{currentfill}%
\pgfsetlinewidth{0.000000pt}%
\definecolor{currentstroke}{rgb}{0.000000,0.000000,0.000000}%
\pgfsetstrokecolor{currentstroke}%
\pgfsetdash{}{0pt}%
\pgfpathmoveto{\pgfqpoint{1.414139in}{0.580755in}}%
\pgfpathlineto{\pgfqpoint{1.415705in}{0.579052in}}%
\pgfpathlineto{\pgfqpoint{1.417275in}{0.577610in}}%
\pgfpathlineto{\pgfqpoint{1.418849in}{0.576435in}}%
\pgfpathlineto{\pgfqpoint{1.420427in}{0.575531in}}%
\pgfpathlineto{\pgfqpoint{1.402108in}{0.571559in}}%
\pgfpathlineto{\pgfqpoint{1.383540in}{0.567902in}}%
\pgfpathlineto{\pgfqpoint{1.364744in}{0.564564in}}%
\pgfpathlineto{\pgfqpoint{1.345738in}{0.561550in}}%
\pgfpathlineto{\pgfqpoint{1.344648in}{0.562547in}}%
\pgfpathlineto{\pgfqpoint{1.343561in}{0.563816in}}%
\pgfpathlineto{\pgfqpoint{1.342477in}{0.565352in}}%
\pgfpathlineto{\pgfqpoint{1.341395in}{0.567149in}}%
\pgfpathlineto{\pgfqpoint{1.359905in}{0.570082in}}%
\pgfpathlineto{\pgfqpoint{1.378212in}{0.573331in}}%
\pgfpathlineto{\pgfqpoint{1.396296in}{0.576890in}}%
\pgfpathlineto{\pgfqpoint{1.414139in}{0.580755in}}%
\pgfpathclose%
\pgfusepath{fill}%
\end{pgfscope}%
\begin{pgfscope}%
\pgfpathrectangle{\pgfqpoint{0.041670in}{0.041670in}}{\pgfqpoint{2.216660in}{2.216660in}}%
\pgfusepath{clip}%
\pgfsetbuttcap%
\pgfsetroundjoin%
\definecolor{currentfill}{rgb}{0.277941,0.056324,0.381191}%
\pgfsetfillcolor{currentfill}%
\pgfsetlinewidth{0.000000pt}%
\definecolor{currentstroke}{rgb}{0.000000,0.000000,0.000000}%
\pgfsetstrokecolor{currentstroke}%
\pgfsetdash{}{0pt}%
\pgfpathmoveto{\pgfqpoint{0.868609in}{0.593792in}}%
\pgfpathlineto{\pgfqpoint{0.866619in}{0.595670in}}%
\pgfpathlineto{\pgfqpoint{0.864622in}{0.597875in}}%
\pgfpathlineto{\pgfqpoint{0.862618in}{0.600412in}}%
\pgfpathlineto{\pgfqpoint{0.860607in}{0.603288in}}%
\pgfpathlineto{\pgfqpoint{0.842171in}{0.608906in}}%
\pgfpathlineto{\pgfqpoint{0.824114in}{0.614831in}}%
\pgfpathlineto{\pgfqpoint{0.806456in}{0.621057in}}%
\pgfpathlineto{\pgfqpoint{0.789215in}{0.627574in}}%
\pgfpathlineto{\pgfqpoint{0.791667in}{0.624555in}}%
\pgfpathlineto{\pgfqpoint{0.794112in}{0.621874in}}%
\pgfpathlineto{\pgfqpoint{0.796547in}{0.619525in}}%
\pgfpathlineto{\pgfqpoint{0.798975in}{0.617502in}}%
\pgfpathlineto{\pgfqpoint{0.815793in}{0.611139in}}%
\pgfpathlineto{\pgfqpoint{0.833017in}{0.605061in}}%
\pgfpathlineto{\pgfqpoint{0.850629in}{0.599277in}}%
\pgfpathlineto{\pgfqpoint{0.868609in}{0.593792in}}%
\pgfpathclose%
\pgfusepath{fill}%
\end{pgfscope}%
\begin{pgfscope}%
\pgfpathrectangle{\pgfqpoint{0.041670in}{0.041670in}}{\pgfqpoint{2.216660in}{2.216660in}}%
\pgfusepath{clip}%
\pgfsetbuttcap%
\pgfsetroundjoin%
\definecolor{currentfill}{rgb}{0.220124,0.725509,0.466226}%
\pgfsetfillcolor{currentfill}%
\pgfsetlinewidth{0.000000pt}%
\definecolor{currentstroke}{rgb}{0.000000,0.000000,0.000000}%
\pgfsetstrokecolor{currentstroke}%
\pgfsetdash{}{0pt}%
\pgfpathmoveto{\pgfqpoint{1.304615in}{1.300640in}}%
\pgfpathlineto{\pgfqpoint{1.306552in}{1.291919in}}%
\pgfpathlineto{\pgfqpoint{1.308487in}{1.283138in}}%
\pgfpathlineto{\pgfqpoint{1.310422in}{1.274298in}}%
\pgfpathlineto{\pgfqpoint{1.312356in}{1.265402in}}%
\pgfpathlineto{\pgfqpoint{1.305013in}{1.263394in}}%
\pgfpathlineto{\pgfqpoint{1.297540in}{1.261502in}}%
\pgfpathlineto{\pgfqpoint{1.289943in}{1.259726in}}%
\pgfpathlineto{\pgfqpoint{1.282231in}{1.258069in}}%
\pgfpathlineto{\pgfqpoint{1.280735in}{1.267083in}}%
\pgfpathlineto{\pgfqpoint{1.279239in}{1.276039in}}%
\pgfpathlineto{\pgfqpoint{1.277742in}{1.284936in}}%
\pgfpathlineto{\pgfqpoint{1.276245in}{1.293772in}}%
\pgfpathlineto{\pgfqpoint{1.283508in}{1.295323in}}%
\pgfpathlineto{\pgfqpoint{1.290662in}{1.296986in}}%
\pgfpathlineto{\pgfqpoint{1.297700in}{1.298759in}}%
\pgfpathlineto{\pgfqpoint{1.304615in}{1.300640in}}%
\pgfpathclose%
\pgfusepath{fill}%
\end{pgfscope}%
\begin{pgfscope}%
\pgfpathrectangle{\pgfqpoint{0.041670in}{0.041670in}}{\pgfqpoint{2.216660in}{2.216660in}}%
\pgfusepath{clip}%
\pgfsetbuttcap%
\pgfsetroundjoin%
\definecolor{currentfill}{rgb}{0.120081,0.622161,0.534946}%
\pgfsetfillcolor{currentfill}%
\pgfsetlinewidth{0.000000pt}%
\definecolor{currentstroke}{rgb}{0.000000,0.000000,0.000000}%
\pgfsetstrokecolor{currentstroke}%
\pgfsetdash{}{0pt}%
\pgfpathmoveto{\pgfqpoint{1.109389in}{1.177052in}}%
\pgfpathlineto{\pgfqpoint{1.108467in}{1.167530in}}%
\pgfpathlineto{\pgfqpoint{1.107546in}{1.157975in}}%
\pgfpathlineto{\pgfqpoint{1.106624in}{1.148390in}}%
\pgfpathlineto{\pgfqpoint{1.105703in}{1.138778in}}%
\pgfpathlineto{\pgfqpoint{1.096117in}{1.140033in}}%
\pgfpathlineto{\pgfqpoint{1.086619in}{1.141440in}}%
\pgfpathlineto{\pgfqpoint{1.077221in}{1.142996in}}%
\pgfpathlineto{\pgfqpoint{1.067931in}{1.144700in}}%
\pgfpathlineto{\pgfqpoint{1.069320in}{1.154231in}}%
\pgfpathlineto{\pgfqpoint{1.070709in}{1.163735in}}%
\pgfpathlineto{\pgfqpoint{1.072098in}{1.173208in}}%
\pgfpathlineto{\pgfqpoint{1.073488in}{1.182649in}}%
\pgfpathlineto{\pgfqpoint{1.082318in}{1.181038in}}%
\pgfpathlineto{\pgfqpoint{1.091251in}{1.179568in}}%
\pgfpathlineto{\pgfqpoint{1.100278in}{1.178238in}}%
\pgfpathlineto{\pgfqpoint{1.109389in}{1.177052in}}%
\pgfpathclose%
\pgfusepath{fill}%
\end{pgfscope}%
\begin{pgfscope}%
\pgfpathrectangle{\pgfqpoint{0.041670in}{0.041670in}}{\pgfqpoint{2.216660in}{2.216660in}}%
\pgfusepath{clip}%
\pgfsetbuttcap%
\pgfsetroundjoin%
\definecolor{currentfill}{rgb}{0.814576,0.883393,0.110347}%
\pgfsetfillcolor{currentfill}%
\pgfsetlinewidth{0.000000pt}%
\definecolor{currentstroke}{rgb}{0.000000,0.000000,0.000000}%
\pgfsetstrokecolor{currentstroke}%
\pgfsetdash{}{0pt}%
\pgfpathmoveto{\pgfqpoint{1.246079in}{1.561132in}}%
\pgfpathlineto{\pgfqpoint{1.248444in}{1.555971in}}%
\pgfpathlineto{\pgfqpoint{1.250809in}{1.550689in}}%
\pgfpathlineto{\pgfqpoint{1.253174in}{1.545288in}}%
\pgfpathlineto{\pgfqpoint{1.255539in}{1.539769in}}%
\pgfpathlineto{\pgfqpoint{1.252345in}{1.538662in}}%
\pgfpathlineto{\pgfqpoint{1.249078in}{1.537603in}}%
\pgfpathlineto{\pgfqpoint{1.245740in}{1.536593in}}%
\pgfpathlineto{\pgfqpoint{1.242335in}{1.535633in}}%
\pgfpathlineto{\pgfqpoint{1.240382in}{1.541285in}}%
\pgfpathlineto{\pgfqpoint{1.238429in}{1.546818in}}%
\pgfpathlineto{\pgfqpoint{1.236477in}{1.552232in}}%
\pgfpathlineto{\pgfqpoint{1.234524in}{1.557526in}}%
\pgfpathlineto{\pgfqpoint{1.237503in}{1.558362in}}%
\pgfpathlineto{\pgfqpoint{1.240424in}{1.559243in}}%
\pgfpathlineto{\pgfqpoint{1.243284in}{1.560167in}}%
\pgfpathlineto{\pgfqpoint{1.246079in}{1.561132in}}%
\pgfpathclose%
\pgfusepath{fill}%
\end{pgfscope}%
\begin{pgfscope}%
\pgfpathrectangle{\pgfqpoint{0.041670in}{0.041670in}}{\pgfqpoint{2.216660in}{2.216660in}}%
\pgfusepath{clip}%
\pgfsetbuttcap%
\pgfsetroundjoin%
\definecolor{currentfill}{rgb}{0.260571,0.246922,0.522828}%
\pgfsetfillcolor{currentfill}%
\pgfsetlinewidth{0.000000pt}%
\definecolor{currentstroke}{rgb}{0.000000,0.000000,0.000000}%
\pgfsetstrokecolor{currentstroke}%
\pgfsetdash{}{0pt}%
\pgfpathmoveto{\pgfqpoint{1.742947in}{0.766538in}}%
\pgfpathlineto{\pgfqpoint{1.746375in}{0.774907in}}%
\pgfpathlineto{\pgfqpoint{1.749820in}{0.783703in}}%
\pgfpathlineto{\pgfqpoint{1.753281in}{0.792934in}}%
\pgfpathlineto{\pgfqpoint{1.756759in}{0.802609in}}%
\pgfpathlineto{\pgfqpoint{1.742697in}{0.793116in}}%
\pgfpathlineto{\pgfqpoint{1.728017in}{0.783851in}}%
\pgfpathlineto{\pgfqpoint{1.712734in}{0.774824in}}%
\pgfpathlineto{\pgfqpoint{1.696862in}{0.766048in}}%
\pgfpathlineto{\pgfqpoint{1.693730in}{0.756547in}}%
\pgfpathlineto{\pgfqpoint{1.690613in}{0.747490in}}%
\pgfpathlineto{\pgfqpoint{1.687511in}{0.738870in}}%
\pgfpathlineto{\pgfqpoint{1.684424in}{0.730680in}}%
\pgfpathlineto{\pgfqpoint{1.699929in}{0.739288in}}%
\pgfpathlineto{\pgfqpoint{1.714860in}{0.748140in}}%
\pgfpathlineto{\pgfqpoint{1.729204in}{0.757227in}}%
\pgfpathlineto{\pgfqpoint{1.742947in}{0.766538in}}%
\pgfpathclose%
\pgfusepath{fill}%
\end{pgfscope}%
\begin{pgfscope}%
\pgfpathrectangle{\pgfqpoint{0.041670in}{0.041670in}}{\pgfqpoint{2.216660in}{2.216660in}}%
\pgfusepath{clip}%
\pgfsetbuttcap%
\pgfsetroundjoin%
\definecolor{currentfill}{rgb}{0.487026,0.823929,0.312321}%
\pgfsetfillcolor{currentfill}%
\pgfsetlinewidth{0.000000pt}%
\definecolor{currentstroke}{rgb}{0.000000,0.000000,0.000000}%
\pgfsetstrokecolor{currentstroke}%
\pgfsetdash{}{0pt}%
\pgfpathmoveto{\pgfqpoint{1.293315in}{1.436443in}}%
\pgfpathlineto{\pgfqpoint{1.295671in}{1.429119in}}%
\pgfpathlineto{\pgfqpoint{1.298026in}{1.421702in}}%
\pgfpathlineto{\pgfqpoint{1.300380in}{1.414195in}}%
\pgfpathlineto{\pgfqpoint{1.302734in}{1.406598in}}%
\pgfpathlineto{\pgfqpoint{1.297558in}{1.404762in}}%
\pgfpathlineto{\pgfqpoint{1.292261in}{1.403006in}}%
\pgfpathlineto{\pgfqpoint{1.286849in}{1.401331in}}%
\pgfpathlineto{\pgfqpoint{1.281325in}{1.399740in}}%
\pgfpathlineto{\pgfqpoint{1.279380in}{1.407475in}}%
\pgfpathlineto{\pgfqpoint{1.277434in}{1.415122in}}%
\pgfpathlineto{\pgfqpoint{1.275488in}{1.422677in}}%
\pgfpathlineto{\pgfqpoint{1.273541in}{1.430140in}}%
\pgfpathlineto{\pgfqpoint{1.278642in}{1.431603in}}%
\pgfpathlineto{\pgfqpoint{1.283642in}{1.433142in}}%
\pgfpathlineto{\pgfqpoint{1.288534in}{1.434756in}}%
\pgfpathlineto{\pgfqpoint{1.293315in}{1.436443in}}%
\pgfpathclose%
\pgfusepath{fill}%
\end{pgfscope}%
\begin{pgfscope}%
\pgfpathrectangle{\pgfqpoint{0.041670in}{0.041670in}}{\pgfqpoint{2.216660in}{2.216660in}}%
\pgfusepath{clip}%
\pgfsetbuttcap%
\pgfsetroundjoin%
\definecolor{currentfill}{rgb}{0.412913,0.803041,0.357269}%
\pgfsetfillcolor{currentfill}%
\pgfsetlinewidth{0.000000pt}%
\definecolor{currentstroke}{rgb}{0.000000,0.000000,0.000000}%
\pgfsetstrokecolor{currentstroke}%
\pgfsetdash{}{0pt}%
\pgfpathmoveto{\pgfqpoint{1.083582in}{1.398396in}}%
\pgfpathlineto{\pgfqpoint{1.081733in}{1.390546in}}%
\pgfpathlineto{\pgfqpoint{1.079885in}{1.382610in}}%
\pgfpathlineto{\pgfqpoint{1.078037in}{1.374590in}}%
\pgfpathlineto{\pgfqpoint{1.076190in}{1.366488in}}%
\pgfpathlineto{\pgfqpoint{1.070144in}{1.368128in}}%
\pgfpathlineto{\pgfqpoint{1.064212in}{1.369860in}}%
\pgfpathlineto{\pgfqpoint{1.058400in}{1.371682in}}%
\pgfpathlineto{\pgfqpoint{1.052714in}{1.373591in}}%
\pgfpathlineto{\pgfqpoint{1.054977in}{1.381558in}}%
\pgfpathlineto{\pgfqpoint{1.057241in}{1.389444in}}%
\pgfpathlineto{\pgfqpoint{1.059505in}{1.397246in}}%
\pgfpathlineto{\pgfqpoint{1.061770in}{1.404963in}}%
\pgfpathlineto{\pgfqpoint{1.067054in}{1.403197in}}%
\pgfpathlineto{\pgfqpoint{1.072454in}{1.401513in}}%
\pgfpathlineto{\pgfqpoint{1.077966in}{1.399912in}}%
\pgfpathlineto{\pgfqpoint{1.083582in}{1.398396in}}%
\pgfpathclose%
\pgfusepath{fill}%
\end{pgfscope}%
\begin{pgfscope}%
\pgfpathrectangle{\pgfqpoint{0.041670in}{0.041670in}}{\pgfqpoint{2.216660in}{2.216660in}}%
\pgfusepath{clip}%
\pgfsetbuttcap%
\pgfsetroundjoin%
\definecolor{currentfill}{rgb}{0.268510,0.009605,0.335427}%
\pgfsetfillcolor{currentfill}%
\pgfsetlinewidth{0.000000pt}%
\definecolor{currentstroke}{rgb}{0.000000,0.000000,0.000000}%
\pgfsetstrokecolor{currentstroke}%
\pgfsetdash{}{0pt}%
\pgfpathmoveto{\pgfqpoint{1.261198in}{0.582299in}}%
\pgfpathlineto{\pgfqpoint{1.261763in}{0.578530in}}%
\pgfpathlineto{\pgfqpoint{1.262330in}{0.574983in}}%
\pgfpathlineto{\pgfqpoint{1.262897in}{0.571662in}}%
\pgfpathlineto{\pgfqpoint{1.263466in}{0.568574in}}%
\pgfpathlineto{\pgfqpoint{1.244772in}{0.567299in}}%
\pgfpathlineto{\pgfqpoint{1.226005in}{0.566346in}}%
\pgfpathlineto{\pgfqpoint{1.207186in}{0.565717in}}%
\pgfpathlineto{\pgfqpoint{1.188337in}{0.565410in}}%
\pgfpathlineto{\pgfqpoint{1.188280in}{0.568522in}}%
\pgfpathlineto{\pgfqpoint{1.188223in}{0.571865in}}%
\pgfpathlineto{\pgfqpoint{1.188166in}{0.575435in}}%
\pgfpathlineto{\pgfqpoint{1.188109in}{0.579228in}}%
\pgfpathlineto{\pgfqpoint{1.206446in}{0.579525in}}%
\pgfpathlineto{\pgfqpoint{1.224754in}{0.580137in}}%
\pgfpathlineto{\pgfqpoint{1.243011in}{0.581062in}}%
\pgfpathlineto{\pgfqpoint{1.261198in}{0.582299in}}%
\pgfpathclose%
\pgfusepath{fill}%
\end{pgfscope}%
\begin{pgfscope}%
\pgfpathrectangle{\pgfqpoint{0.041670in}{0.041670in}}{\pgfqpoint{2.216660in}{2.216660in}}%
\pgfusepath{clip}%
\pgfsetbuttcap%
\pgfsetroundjoin%
\definecolor{currentfill}{rgb}{0.896320,0.893616,0.096335}%
\pgfsetfillcolor{currentfill}%
\pgfsetlinewidth{0.000000pt}%
\definecolor{currentstroke}{rgb}{0.000000,0.000000,0.000000}%
\pgfsetstrokecolor{currentstroke}%
\pgfsetdash{}{0pt}%
\pgfpathmoveto{\pgfqpoint{1.210017in}{1.593374in}}%
\pgfpathlineto{\pgfqpoint{1.211525in}{1.588973in}}%
\pgfpathlineto{\pgfqpoint{1.213032in}{1.584444in}}%
\pgfpathlineto{\pgfqpoint{1.214540in}{1.579788in}}%
\pgfpathlineto{\pgfqpoint{1.216048in}{1.575007in}}%
\pgfpathlineto{\pgfqpoint{1.213284in}{1.574490in}}%
\pgfpathlineto{\pgfqpoint{1.210485in}{1.574014in}}%
\pgfpathlineto{\pgfqpoint{1.207656in}{1.573581in}}%
\pgfpathlineto{\pgfqpoint{1.204799in}{1.573190in}}%
\pgfpathlineto{\pgfqpoint{1.203760in}{1.578049in}}%
\pgfpathlineto{\pgfqpoint{1.202722in}{1.582782in}}%
\pgfpathlineto{\pgfqpoint{1.201684in}{1.587387in}}%
\pgfpathlineto{\pgfqpoint{1.200647in}{1.591865in}}%
\pgfpathlineto{\pgfqpoint{1.203027in}{1.592190in}}%
\pgfpathlineto{\pgfqpoint{1.205384in}{1.592550in}}%
\pgfpathlineto{\pgfqpoint{1.207715in}{1.592945in}}%
\pgfpathlineto{\pgfqpoint{1.210017in}{1.593374in}}%
\pgfpathclose%
\pgfusepath{fill}%
\end{pgfscope}%
\begin{pgfscope}%
\pgfpathrectangle{\pgfqpoint{0.041670in}{0.041670in}}{\pgfqpoint{2.216660in}{2.216660in}}%
\pgfusepath{clip}%
\pgfsetbuttcap%
\pgfsetroundjoin%
\definecolor{currentfill}{rgb}{0.814576,0.883393,0.110347}%
\pgfsetfillcolor{currentfill}%
\pgfsetlinewidth{0.000000pt}%
\definecolor{currentstroke}{rgb}{0.000000,0.000000,0.000000}%
\pgfsetstrokecolor{currentstroke}%
\pgfsetdash{}{0pt}%
\pgfpathmoveto{\pgfqpoint{1.128081in}{1.556819in}}%
\pgfpathlineto{\pgfqpoint{1.126225in}{1.551500in}}%
\pgfpathlineto{\pgfqpoint{1.124368in}{1.546060in}}%
\pgfpathlineto{\pgfqpoint{1.122511in}{1.540501in}}%
\pgfpathlineto{\pgfqpoint{1.120655in}{1.534823in}}%
\pgfpathlineto{\pgfqpoint{1.117193in}{1.535737in}}%
\pgfpathlineto{\pgfqpoint{1.113796in}{1.536702in}}%
\pgfpathlineto{\pgfqpoint{1.110465in}{1.537718in}}%
\pgfpathlineto{\pgfqpoint{1.107206in}{1.538783in}}%
\pgfpathlineto{\pgfqpoint{1.109482in}{1.544333in}}%
\pgfpathlineto{\pgfqpoint{1.111759in}{1.549766in}}%
\pgfpathlineto{\pgfqpoint{1.114035in}{1.555079in}}%
\pgfpathlineto{\pgfqpoint{1.116312in}{1.560272in}}%
\pgfpathlineto{\pgfqpoint{1.119165in}{1.559343in}}%
\pgfpathlineto{\pgfqpoint{1.122079in}{1.558458in}}%
\pgfpathlineto{\pgfqpoint{1.125052in}{1.557616in}}%
\pgfpathlineto{\pgfqpoint{1.128081in}{1.556819in}}%
\pgfpathclose%
\pgfusepath{fill}%
\end{pgfscope}%
\begin{pgfscope}%
\pgfpathrectangle{\pgfqpoint{0.041670in}{0.041670in}}{\pgfqpoint{2.216660in}{2.216660in}}%
\pgfusepath{clip}%
\pgfsetbuttcap%
\pgfsetroundjoin%
\definecolor{currentfill}{rgb}{0.896320,0.893616,0.096335}%
\pgfsetfillcolor{currentfill}%
\pgfsetlinewidth{0.000000pt}%
\definecolor{currentstroke}{rgb}{0.000000,0.000000,0.000000}%
\pgfsetstrokecolor{currentstroke}%
\pgfsetdash{}{0pt}%
\pgfpathmoveto{\pgfqpoint{1.161396in}{1.591606in}}%
\pgfpathlineto{\pgfqpoint{1.160466in}{1.587115in}}%
\pgfpathlineto{\pgfqpoint{1.159535in}{1.582496in}}%
\pgfpathlineto{\pgfqpoint{1.158604in}{1.577750in}}%
\pgfpathlineto{\pgfqpoint{1.157672in}{1.572878in}}%
\pgfpathlineto{\pgfqpoint{1.154792in}{1.573231in}}%
\pgfpathlineto{\pgfqpoint{1.151938in}{1.573627in}}%
\pgfpathlineto{\pgfqpoint{1.149112in}{1.574065in}}%
\pgfpathlineto{\pgfqpoint{1.146317in}{1.574545in}}%
\pgfpathlineto{\pgfqpoint{1.147723in}{1.579347in}}%
\pgfpathlineto{\pgfqpoint{1.149128in}{1.584022in}}%
\pgfpathlineto{\pgfqpoint{1.150533in}{1.588570in}}%
\pgfpathlineto{\pgfqpoint{1.151938in}{1.592991in}}%
\pgfpathlineto{\pgfqpoint{1.154266in}{1.592592in}}%
\pgfpathlineto{\pgfqpoint{1.156620in}{1.592228in}}%
\pgfpathlineto{\pgfqpoint{1.158998in}{1.591900in}}%
\pgfpathlineto{\pgfqpoint{1.161396in}{1.591606in}}%
\pgfpathclose%
\pgfusepath{fill}%
\end{pgfscope}%
\begin{pgfscope}%
\pgfpathrectangle{\pgfqpoint{0.041670in}{0.041670in}}{\pgfqpoint{2.216660in}{2.216660in}}%
\pgfusepath{clip}%
\pgfsetbuttcap%
\pgfsetroundjoin%
\definecolor{currentfill}{rgb}{0.268510,0.009605,0.335427}%
\pgfsetfillcolor{currentfill}%
\pgfsetlinewidth{0.000000pt}%
\definecolor{currentstroke}{rgb}{0.000000,0.000000,0.000000}%
\pgfsetstrokecolor{currentstroke}%
\pgfsetdash{}{0pt}%
\pgfpathmoveto{\pgfqpoint{1.188109in}{0.579228in}}%
\pgfpathlineto{\pgfqpoint{1.188166in}{0.575435in}}%
\pgfpathlineto{\pgfqpoint{1.188223in}{0.571865in}}%
\pgfpathlineto{\pgfqpoint{1.188280in}{0.568522in}}%
\pgfpathlineto{\pgfqpoint{1.188337in}{0.565410in}}%
\pgfpathlineto{\pgfqpoint{1.169478in}{0.565428in}}%
\pgfpathlineto{\pgfqpoint{1.150631in}{0.565771in}}%
\pgfpathlineto{\pgfqpoint{1.131817in}{0.566436in}}%
\pgfpathlineto{\pgfqpoint{1.113057in}{0.567425in}}%
\pgfpathlineto{\pgfqpoint{1.113513in}{0.570522in}}%
\pgfpathlineto{\pgfqpoint{1.113967in}{0.573851in}}%
\pgfpathlineto{\pgfqpoint{1.114421in}{0.577406in}}%
\pgfpathlineto{\pgfqpoint{1.114874in}{0.581184in}}%
\pgfpathlineto{\pgfqpoint{1.133124in}{0.580224in}}%
\pgfpathlineto{\pgfqpoint{1.151427in}{0.579577in}}%
\pgfpathlineto{\pgfqpoint{1.169762in}{0.579245in}}%
\pgfpathlineto{\pgfqpoint{1.188109in}{0.579228in}}%
\pgfpathclose%
\pgfusepath{fill}%
\end{pgfscope}%
\begin{pgfscope}%
\pgfpathrectangle{\pgfqpoint{0.041670in}{0.041670in}}{\pgfqpoint{2.216660in}{2.216660in}}%
\pgfusepath{clip}%
\pgfsetbuttcap%
\pgfsetroundjoin%
\definecolor{currentfill}{rgb}{0.122606,0.585371,0.546557}%
\pgfsetfillcolor{currentfill}%
\pgfsetlinewidth{0.000000pt}%
\definecolor{currentstroke}{rgb}{0.000000,0.000000,0.000000}%
\pgfsetstrokecolor{currentstroke}%
\pgfsetdash{}{0pt}%
\pgfpathmoveto{\pgfqpoint{1.262732in}{1.139886in}}%
\pgfpathlineto{\pgfqpoint{1.263759in}{1.130264in}}%
\pgfpathlineto{\pgfqpoint{1.264785in}{1.120619in}}%
\pgfpathlineto{\pgfqpoint{1.265811in}{1.110954in}}%
\pgfpathlineto{\pgfqpoint{1.266836in}{1.101273in}}%
\pgfpathlineto{\pgfqpoint{1.256765in}{1.099966in}}%
\pgfpathlineto{\pgfqpoint{1.246612in}{1.098820in}}%
\pgfpathlineto{\pgfqpoint{1.236389in}{1.097837in}}%
\pgfpathlineto{\pgfqpoint{1.226104in}{1.097017in}}%
\pgfpathlineto{\pgfqpoint{1.225559in}{1.106755in}}%
\pgfpathlineto{\pgfqpoint{1.225014in}{1.116475in}}%
\pgfpathlineto{\pgfqpoint{1.224469in}{1.126176in}}%
\pgfpathlineto{\pgfqpoint{1.223923in}{1.135854in}}%
\pgfpathlineto{\pgfqpoint{1.233722in}{1.136631in}}%
\pgfpathlineto{\pgfqpoint{1.243463in}{1.137562in}}%
\pgfpathlineto{\pgfqpoint{1.253136in}{1.138648in}}%
\pgfpathlineto{\pgfqpoint{1.262732in}{1.139886in}}%
\pgfpathclose%
\pgfusepath{fill}%
\end{pgfscope}%
\begin{pgfscope}%
\pgfpathrectangle{\pgfqpoint{0.041670in}{0.041670in}}{\pgfqpoint{2.216660in}{2.216660in}}%
\pgfusepath{clip}%
\pgfsetbuttcap%
\pgfsetroundjoin%
\definecolor{currentfill}{rgb}{0.762373,0.876424,0.137064}%
\pgfsetfillcolor{currentfill}%
\pgfsetlinewidth{0.000000pt}%
\definecolor{currentstroke}{rgb}{0.000000,0.000000,0.000000}%
\pgfsetstrokecolor{currentstroke}%
\pgfsetdash{}{0pt}%
\pgfpathmoveto{\pgfqpoint{1.255539in}{1.539769in}}%
\pgfpathlineto{\pgfqpoint{1.257903in}{1.534134in}}%
\pgfpathlineto{\pgfqpoint{1.260267in}{1.528382in}}%
\pgfpathlineto{\pgfqpoint{1.262631in}{1.522516in}}%
\pgfpathlineto{\pgfqpoint{1.264994in}{1.516537in}}%
\pgfpathlineto{\pgfqpoint{1.261403in}{1.515286in}}%
\pgfpathlineto{\pgfqpoint{1.257728in}{1.514090in}}%
\pgfpathlineto{\pgfqpoint{1.253974in}{1.512950in}}%
\pgfpathlineto{\pgfqpoint{1.250144in}{1.511866in}}%
\pgfpathlineto{\pgfqpoint{1.248192in}{1.517980in}}%
\pgfpathlineto{\pgfqpoint{1.246240in}{1.523979in}}%
\pgfpathlineto{\pgfqpoint{1.244288in}{1.529864in}}%
\pgfpathlineto{\pgfqpoint{1.242335in}{1.535633in}}%
\pgfpathlineto{\pgfqpoint{1.245740in}{1.536593in}}%
\pgfpathlineto{\pgfqpoint{1.249078in}{1.537603in}}%
\pgfpathlineto{\pgfqpoint{1.252345in}{1.538662in}}%
\pgfpathlineto{\pgfqpoint{1.255539in}{1.539769in}}%
\pgfpathclose%
\pgfusepath{fill}%
\end{pgfscope}%
\begin{pgfscope}%
\pgfpathrectangle{\pgfqpoint{0.041670in}{0.041670in}}{\pgfqpoint{2.216660in}{2.216660in}}%
\pgfusepath{clip}%
\pgfsetbuttcap%
\pgfsetroundjoin%
\definecolor{currentfill}{rgb}{0.220124,0.725509,0.466226}%
\pgfsetfillcolor{currentfill}%
\pgfsetlinewidth{0.000000pt}%
\definecolor{currentstroke}{rgb}{0.000000,0.000000,0.000000}%
\pgfsetstrokecolor{currentstroke}%
\pgfsetdash{}{0pt}%
\pgfpathmoveto{\pgfqpoint{1.090205in}{1.292488in}}%
\pgfpathlineto{\pgfqpoint{1.088809in}{1.283630in}}%
\pgfpathlineto{\pgfqpoint{1.087414in}{1.274712in}}%
\pgfpathlineto{\pgfqpoint{1.086019in}{1.265733in}}%
\pgfpathlineto{\pgfqpoint{1.084625in}{1.256698in}}%
\pgfpathlineto{\pgfqpoint{1.076817in}{1.258248in}}%
\pgfpathlineto{\pgfqpoint{1.069117in}{1.259917in}}%
\pgfpathlineto{\pgfqpoint{1.061533in}{1.261706in}}%
\pgfpathlineto{\pgfqpoint{1.054074in}{1.263612in}}%
\pgfpathlineto{\pgfqpoint{1.055913in}{1.272537in}}%
\pgfpathlineto{\pgfqpoint{1.057753in}{1.281405in}}%
\pgfpathlineto{\pgfqpoint{1.059594in}{1.290214in}}%
\pgfpathlineto{\pgfqpoint{1.061435in}{1.298963in}}%
\pgfpathlineto{\pgfqpoint{1.068460in}{1.297178in}}%
\pgfpathlineto{\pgfqpoint{1.075602in}{1.295503in}}%
\pgfpathlineto{\pgfqpoint{1.082852in}{1.293939in}}%
\pgfpathlineto{\pgfqpoint{1.090205in}{1.292488in}}%
\pgfpathclose%
\pgfusepath{fill}%
\end{pgfscope}%
\begin{pgfscope}%
\pgfpathrectangle{\pgfqpoint{0.041670in}{0.041670in}}{\pgfqpoint{2.216660in}{2.216660in}}%
\pgfusepath{clip}%
\pgfsetbuttcap%
\pgfsetroundjoin%
\definecolor{currentfill}{rgb}{0.565498,0.842430,0.262877}%
\pgfsetfillcolor{currentfill}%
\pgfsetlinewidth{0.000000pt}%
\definecolor{currentstroke}{rgb}{0.000000,0.000000,0.000000}%
\pgfsetstrokecolor{currentstroke}%
\pgfsetdash{}{0pt}%
\pgfpathmoveto{\pgfqpoint{1.283884in}{1.464783in}}%
\pgfpathlineto{\pgfqpoint{1.286243in}{1.457845in}}%
\pgfpathlineto{\pgfqpoint{1.288601in}{1.450808in}}%
\pgfpathlineto{\pgfqpoint{1.290958in}{1.443673in}}%
\pgfpathlineto{\pgfqpoint{1.293315in}{1.436443in}}%
\pgfpathlineto{\pgfqpoint{1.288534in}{1.434756in}}%
\pgfpathlineto{\pgfqpoint{1.283642in}{1.433142in}}%
\pgfpathlineto{\pgfqpoint{1.278642in}{1.431603in}}%
\pgfpathlineto{\pgfqpoint{1.273541in}{1.430140in}}%
\pgfpathlineto{\pgfqpoint{1.271594in}{1.437508in}}%
\pgfpathlineto{\pgfqpoint{1.269646in}{1.444780in}}%
\pgfpathlineto{\pgfqpoint{1.267698in}{1.451954in}}%
\pgfpathlineto{\pgfqpoint{1.265749in}{1.459029in}}%
\pgfpathlineto{\pgfqpoint{1.270427in}{1.460364in}}%
\pgfpathlineto{\pgfqpoint{1.275012in}{1.461769in}}%
\pgfpathlineto{\pgfqpoint{1.279499in}{1.463242in}}%
\pgfpathlineto{\pgfqpoint{1.283884in}{1.464783in}}%
\pgfpathclose%
\pgfusepath{fill}%
\end{pgfscope}%
\begin{pgfscope}%
\pgfpathrectangle{\pgfqpoint{0.041670in}{0.041670in}}{\pgfqpoint{2.216660in}{2.216660in}}%
\pgfusepath{clip}%
\pgfsetbuttcap%
\pgfsetroundjoin%
\definecolor{currentfill}{rgb}{0.134692,0.658636,0.517649}%
\pgfsetfillcolor{currentfill}%
\pgfsetlinewidth{0.000000pt}%
\definecolor{currentstroke}{rgb}{0.000000,0.000000,0.000000}%
\pgfsetstrokecolor{currentstroke}%
\pgfsetdash{}{0pt}%
\pgfpathmoveto{\pgfqpoint{1.288208in}{1.221494in}}%
\pgfpathlineto{\pgfqpoint{1.289700in}{1.212231in}}%
\pgfpathlineto{\pgfqpoint{1.291193in}{1.202926in}}%
\pgfpathlineto{\pgfqpoint{1.292685in}{1.193580in}}%
\pgfpathlineto{\pgfqpoint{1.294176in}{1.184197in}}%
\pgfpathlineto{\pgfqpoint{1.285446in}{1.182463in}}%
\pgfpathlineto{\pgfqpoint{1.276604in}{1.180868in}}%
\pgfpathlineto{\pgfqpoint{1.267660in}{1.179413in}}%
\pgfpathlineto{\pgfqpoint{1.258623in}{1.178099in}}%
\pgfpathlineto{\pgfqpoint{1.257595in}{1.187570in}}%
\pgfpathlineto{\pgfqpoint{1.256567in}{1.197004in}}%
\pgfpathlineto{\pgfqpoint{1.255538in}{1.206396in}}%
\pgfpathlineto{\pgfqpoint{1.254509in}{1.215746in}}%
\pgfpathlineto{\pgfqpoint{1.263074in}{1.216984in}}%
\pgfpathlineto{\pgfqpoint{1.271551in}{1.218356in}}%
\pgfpathlineto{\pgfqpoint{1.279932in}{1.219860in}}%
\pgfpathlineto{\pgfqpoint{1.288208in}{1.221494in}}%
\pgfpathclose%
\pgfusepath{fill}%
\end{pgfscope}%
\begin{pgfscope}%
\pgfpathrectangle{\pgfqpoint{0.041670in}{0.041670in}}{\pgfqpoint{2.216660in}{2.216660in}}%
\pgfusepath{clip}%
\pgfsetbuttcap%
\pgfsetroundjoin%
\definecolor{currentfill}{rgb}{0.133743,0.548535,0.553541}%
\pgfsetfillcolor{currentfill}%
\pgfsetlinewidth{0.000000pt}%
\definecolor{currentstroke}{rgb}{0.000000,0.000000,0.000000}%
\pgfsetstrokecolor{currentstroke}%
\pgfsetdash{}{0pt}%
\pgfpathmoveto{\pgfqpoint{1.226104in}{1.097017in}}%
\pgfpathlineto{\pgfqpoint{1.226649in}{1.087265in}}%
\pgfpathlineto{\pgfqpoint{1.227194in}{1.077501in}}%
\pgfpathlineto{\pgfqpoint{1.227739in}{1.067729in}}%
\pgfpathlineto{\pgfqpoint{1.228283in}{1.057951in}}%
\pgfpathlineto{\pgfqpoint{1.217462in}{1.057261in}}%
\pgfpathlineto{\pgfqpoint{1.206601in}{1.056746in}}%
\pgfpathlineto{\pgfqpoint{1.195711in}{1.056405in}}%
\pgfpathlineto{\pgfqpoint{1.184805in}{1.056239in}}%
\pgfpathlineto{\pgfqpoint{1.184750in}{1.066039in}}%
\pgfpathlineto{\pgfqpoint{1.184695in}{1.075833in}}%
\pgfpathlineto{\pgfqpoint{1.184641in}{1.085619in}}%
\pgfpathlineto{\pgfqpoint{1.184586in}{1.095392in}}%
\pgfpathlineto{\pgfqpoint{1.195001in}{1.095549in}}%
\pgfpathlineto{\pgfqpoint{1.205400in}{1.095873in}}%
\pgfpathlineto{\pgfqpoint{1.215771in}{1.096362in}}%
\pgfpathlineto{\pgfqpoint{1.226104in}{1.097017in}}%
\pgfpathclose%
\pgfusepath{fill}%
\end{pgfscope}%
\begin{pgfscope}%
\pgfpathrectangle{\pgfqpoint{0.041670in}{0.041670in}}{\pgfqpoint{2.216660in}{2.216660in}}%
\pgfusepath{clip}%
\pgfsetbuttcap%
\pgfsetroundjoin%
\definecolor{currentfill}{rgb}{0.855810,0.888601,0.097452}%
\pgfsetfillcolor{currentfill}%
\pgfsetlinewidth{0.000000pt}%
\definecolor{currentstroke}{rgb}{0.000000,0.000000,0.000000}%
\pgfsetstrokecolor{currentstroke}%
\pgfsetdash{}{0pt}%
\pgfpathmoveto{\pgfqpoint{1.226712in}{1.577476in}}%
\pgfpathlineto{\pgfqpoint{1.228665in}{1.572674in}}%
\pgfpathlineto{\pgfqpoint{1.230618in}{1.567747in}}%
\pgfpathlineto{\pgfqpoint{1.232571in}{1.562698in}}%
\pgfpathlineto{\pgfqpoint{1.234524in}{1.557526in}}%
\pgfpathlineto{\pgfqpoint{1.231488in}{1.556734in}}%
\pgfpathlineto{\pgfqpoint{1.228401in}{1.555987in}}%
\pgfpathlineto{\pgfqpoint{1.225264in}{1.555287in}}%
\pgfpathlineto{\pgfqpoint{1.222081in}{1.554635in}}%
\pgfpathlineto{\pgfqpoint{1.220573in}{1.559912in}}%
\pgfpathlineto{\pgfqpoint{1.219064in}{1.565068in}}%
\pgfpathlineto{\pgfqpoint{1.217556in}{1.570099in}}%
\pgfpathlineto{\pgfqpoint{1.216048in}{1.575007in}}%
\pgfpathlineto{\pgfqpoint{1.218776in}{1.575564in}}%
\pgfpathlineto{\pgfqpoint{1.221464in}{1.576162in}}%
\pgfpathlineto{\pgfqpoint{1.224110in}{1.576799in}}%
\pgfpathlineto{\pgfqpoint{1.226712in}{1.577476in}}%
\pgfpathclose%
\pgfusepath{fill}%
\end{pgfscope}%
\begin{pgfscope}%
\pgfpathrectangle{\pgfqpoint{0.041670in}{0.041670in}}{\pgfqpoint{2.216660in}{2.216660in}}%
\pgfusepath{clip}%
\pgfsetbuttcap%
\pgfsetroundjoin%
\definecolor{currentfill}{rgb}{0.133743,0.548535,0.553541}%
\pgfsetfillcolor{currentfill}%
\pgfsetlinewidth{0.000000pt}%
\definecolor{currentstroke}{rgb}{0.000000,0.000000,0.000000}%
\pgfsetstrokecolor{currentstroke}%
\pgfsetdash{}{0pt}%
\pgfpathmoveto{\pgfqpoint{1.184586in}{1.095392in}}%
\pgfpathlineto{\pgfqpoint{1.184641in}{1.085619in}}%
\pgfpathlineto{\pgfqpoint{1.184695in}{1.075833in}}%
\pgfpathlineto{\pgfqpoint{1.184750in}{1.066039in}}%
\pgfpathlineto{\pgfqpoint{1.184805in}{1.056239in}}%
\pgfpathlineto{\pgfqpoint{1.173893in}{1.056249in}}%
\pgfpathlineto{\pgfqpoint{1.162987in}{1.056434in}}%
\pgfpathlineto{\pgfqpoint{1.152100in}{1.056794in}}%
\pgfpathlineto{\pgfqpoint{1.141243in}{1.057329in}}%
\pgfpathlineto{\pgfqpoint{1.141679in}{1.067115in}}%
\pgfpathlineto{\pgfqpoint{1.142115in}{1.076896in}}%
\pgfpathlineto{\pgfqpoint{1.142552in}{1.086667in}}%
\pgfpathlineto{\pgfqpoint{1.142988in}{1.096427in}}%
\pgfpathlineto{\pgfqpoint{1.153356in}{1.095919in}}%
\pgfpathlineto{\pgfqpoint{1.163753in}{1.095577in}}%
\pgfpathlineto{\pgfqpoint{1.174166in}{1.095401in}}%
\pgfpathlineto{\pgfqpoint{1.184586in}{1.095392in}}%
\pgfpathclose%
\pgfusepath{fill}%
\end{pgfscope}%
\begin{pgfscope}%
\pgfpathrectangle{\pgfqpoint{0.041670in}{0.041670in}}{\pgfqpoint{2.216660in}{2.216660in}}%
\pgfusepath{clip}%
\pgfsetbuttcap%
\pgfsetroundjoin%
\definecolor{currentfill}{rgb}{0.487026,0.823929,0.312321}%
\pgfsetfillcolor{currentfill}%
\pgfsetlinewidth{0.000000pt}%
\definecolor{currentstroke}{rgb}{0.000000,0.000000,0.000000}%
\pgfsetstrokecolor{currentstroke}%
\pgfsetdash{}{0pt}%
\pgfpathmoveto{\pgfqpoint{1.090984in}{1.428905in}}%
\pgfpathlineto{\pgfqpoint{1.089133in}{1.421415in}}%
\pgfpathlineto{\pgfqpoint{1.087282in}{1.413833in}}%
\pgfpathlineto{\pgfqpoint{1.085432in}{1.406159in}}%
\pgfpathlineto{\pgfqpoint{1.083582in}{1.398396in}}%
\pgfpathlineto{\pgfqpoint{1.077966in}{1.399912in}}%
\pgfpathlineto{\pgfqpoint{1.072454in}{1.401513in}}%
\pgfpathlineto{\pgfqpoint{1.067054in}{1.403197in}}%
\pgfpathlineto{\pgfqpoint{1.061770in}{1.404963in}}%
\pgfpathlineto{\pgfqpoint{1.064036in}{1.412592in}}%
\pgfpathlineto{\pgfqpoint{1.066303in}{1.420133in}}%
\pgfpathlineto{\pgfqpoint{1.068571in}{1.427583in}}%
\pgfpathlineto{\pgfqpoint{1.070839in}{1.434940in}}%
\pgfpathlineto{\pgfqpoint{1.075719in}{1.433318in}}%
\pgfpathlineto{\pgfqpoint{1.080707in}{1.431770in}}%
\pgfpathlineto{\pgfqpoint{1.085797in}{1.430298in}}%
\pgfpathlineto{\pgfqpoint{1.090984in}{1.428905in}}%
\pgfpathclose%
\pgfusepath{fill}%
\end{pgfscope}%
\begin{pgfscope}%
\pgfpathrectangle{\pgfqpoint{0.041670in}{0.041670in}}{\pgfqpoint{2.216660in}{2.216660in}}%
\pgfusepath{clip}%
\pgfsetbuttcap%
\pgfsetroundjoin%
\definecolor{currentfill}{rgb}{0.699415,0.867117,0.175971}%
\pgfsetfillcolor{currentfill}%
\pgfsetlinewidth{0.000000pt}%
\definecolor{currentstroke}{rgb}{0.000000,0.000000,0.000000}%
\pgfsetstrokecolor{currentstroke}%
\pgfsetdash{}{0pt}%
\pgfpathmoveto{\pgfqpoint{1.264994in}{1.516537in}}%
\pgfpathlineto{\pgfqpoint{1.267357in}{1.510446in}}%
\pgfpathlineto{\pgfqpoint{1.269720in}{1.504243in}}%
\pgfpathlineto{\pgfqpoint{1.272082in}{1.497932in}}%
\pgfpathlineto{\pgfqpoint{1.274443in}{1.491512in}}%
\pgfpathlineto{\pgfqpoint{1.270455in}{1.490118in}}%
\pgfpathlineto{\pgfqpoint{1.266374in}{1.488784in}}%
\pgfpathlineto{\pgfqpoint{1.262204in}{1.487511in}}%
\pgfpathlineto{\pgfqpoint{1.257949in}{1.486303in}}%
\pgfpathlineto{\pgfqpoint{1.255999in}{1.492857in}}%
\pgfpathlineto{\pgfqpoint{1.254047in}{1.499304in}}%
\pgfpathlineto{\pgfqpoint{1.252096in}{1.505641in}}%
\pgfpathlineto{\pgfqpoint{1.250144in}{1.511866in}}%
\pgfpathlineto{\pgfqpoint{1.253974in}{1.512950in}}%
\pgfpathlineto{\pgfqpoint{1.257728in}{1.514090in}}%
\pgfpathlineto{\pgfqpoint{1.261403in}{1.515286in}}%
\pgfpathlineto{\pgfqpoint{1.264994in}{1.516537in}}%
\pgfpathclose%
\pgfusepath{fill}%
\end{pgfscope}%
\begin{pgfscope}%
\pgfpathrectangle{\pgfqpoint{0.041670in}{0.041670in}}{\pgfqpoint{2.216660in}{2.216660in}}%
\pgfusepath{clip}%
\pgfsetbuttcap%
\pgfsetroundjoin%
\definecolor{currentfill}{rgb}{0.636902,0.856542,0.216620}%
\pgfsetfillcolor{currentfill}%
\pgfsetlinewidth{0.000000pt}%
\definecolor{currentstroke}{rgb}{0.000000,0.000000,0.000000}%
\pgfsetstrokecolor{currentstroke}%
\pgfsetdash{}{0pt}%
\pgfpathmoveto{\pgfqpoint{1.274443in}{1.491512in}}%
\pgfpathlineto{\pgfqpoint{1.276804in}{1.484986in}}%
\pgfpathlineto{\pgfqpoint{1.279165in}{1.478355in}}%
\pgfpathlineto{\pgfqpoint{1.281525in}{1.471620in}}%
\pgfpathlineto{\pgfqpoint{1.283884in}{1.464783in}}%
\pgfpathlineto{\pgfqpoint{1.279499in}{1.463242in}}%
\pgfpathlineto{\pgfqpoint{1.275012in}{1.461769in}}%
\pgfpathlineto{\pgfqpoint{1.270427in}{1.460364in}}%
\pgfpathlineto{\pgfqpoint{1.265749in}{1.459029in}}%
\pgfpathlineto{\pgfqpoint{1.263800in}{1.466002in}}%
\pgfpathlineto{\pgfqpoint{1.261850in}{1.472874in}}%
\pgfpathlineto{\pgfqpoint{1.259900in}{1.479641in}}%
\pgfpathlineto{\pgfqpoint{1.257949in}{1.486303in}}%
\pgfpathlineto{\pgfqpoint{1.262204in}{1.487511in}}%
\pgfpathlineto{\pgfqpoint{1.266374in}{1.488784in}}%
\pgfpathlineto{\pgfqpoint{1.270455in}{1.490118in}}%
\pgfpathlineto{\pgfqpoint{1.274443in}{1.491512in}}%
\pgfpathclose%
\pgfusepath{fill}%
\end{pgfscope}%
\begin{pgfscope}%
\pgfpathrectangle{\pgfqpoint{0.041670in}{0.041670in}}{\pgfqpoint{2.216660in}{2.216660in}}%
\pgfusepath{clip}%
\pgfsetbuttcap%
\pgfsetroundjoin%
\definecolor{currentfill}{rgb}{0.267004,0.004874,0.329415}%
\pgfsetfillcolor{currentfill}%
\pgfsetlinewidth{0.000000pt}%
\definecolor{currentstroke}{rgb}{0.000000,0.000000,0.000000}%
\pgfsetstrokecolor{currentstroke}%
\pgfsetdash{}{0pt}%
\pgfpathmoveto{\pgfqpoint{1.337095in}{0.576851in}}%
\pgfpathlineto{\pgfqpoint{1.338167in}{0.574059in}}%
\pgfpathlineto{\pgfqpoint{1.339240in}{0.571508in}}%
\pgfpathlineto{\pgfqpoint{1.340317in}{0.569203in}}%
\pgfpathlineto{\pgfqpoint{1.341395in}{0.567149in}}%
\pgfpathlineto{\pgfqpoint{1.322705in}{0.564535in}}%
\pgfpathlineto{\pgfqpoint{1.303854in}{0.562242in}}%
\pgfpathlineto{\pgfqpoint{1.284863in}{0.560274in}}%
\pgfpathlineto{\pgfqpoint{1.265754in}{0.558634in}}%
\pgfpathlineto{\pgfqpoint{1.265180in}{0.560747in}}%
\pgfpathlineto{\pgfqpoint{1.264607in}{0.563111in}}%
\pgfpathlineto{\pgfqpoint{1.264036in}{0.565722in}}%
\pgfpathlineto{\pgfqpoint{1.263466in}{0.568574in}}%
\pgfpathlineto{\pgfqpoint{1.282066in}{0.570168in}}%
\pgfpathlineto{\pgfqpoint{1.300551in}{0.572081in}}%
\pgfpathlineto{\pgfqpoint{1.318901in}{0.574310in}}%
\pgfpathlineto{\pgfqpoint{1.337095in}{0.576851in}}%
\pgfpathclose%
\pgfusepath{fill}%
\end{pgfscope}%
\begin{pgfscope}%
\pgfpathrectangle{\pgfqpoint{0.041670in}{0.041670in}}{\pgfqpoint{2.216660in}{2.216660in}}%
\pgfusepath{clip}%
\pgfsetbuttcap%
\pgfsetroundjoin%
\definecolor{currentfill}{rgb}{0.122606,0.585371,0.546557}%
\pgfsetfillcolor{currentfill}%
\pgfsetlinewidth{0.000000pt}%
\definecolor{currentstroke}{rgb}{0.000000,0.000000,0.000000}%
\pgfsetstrokecolor{currentstroke}%
\pgfsetdash{}{0pt}%
\pgfpathmoveto{\pgfqpoint{1.144736in}{1.135295in}}%
\pgfpathlineto{\pgfqpoint{1.144299in}{1.125609in}}%
\pgfpathlineto{\pgfqpoint{1.143862in}{1.115901in}}%
\pgfpathlineto{\pgfqpoint{1.143425in}{1.106172in}}%
\pgfpathlineto{\pgfqpoint{1.142988in}{1.096427in}}%
\pgfpathlineto{\pgfqpoint{1.132660in}{1.097100in}}%
\pgfpathlineto{\pgfqpoint{1.122382in}{1.097938in}}%
\pgfpathlineto{\pgfqpoint{1.112166in}{1.098939in}}%
\pgfpathlineto{\pgfqpoint{1.102021in}{1.100103in}}%
\pgfpathlineto{\pgfqpoint{1.102941in}{1.109800in}}%
\pgfpathlineto{\pgfqpoint{1.103862in}{1.119480in}}%
\pgfpathlineto{\pgfqpoint{1.104782in}{1.129140in}}%
\pgfpathlineto{\pgfqpoint{1.105703in}{1.138778in}}%
\pgfpathlineto{\pgfqpoint{1.115369in}{1.137675in}}%
\pgfpathlineto{\pgfqpoint{1.125103in}{1.136726in}}%
\pgfpathlineto{\pgfqpoint{1.134895in}{1.135933in}}%
\pgfpathlineto{\pgfqpoint{1.144736in}{1.135295in}}%
\pgfpathclose%
\pgfusepath{fill}%
\end{pgfscope}%
\begin{pgfscope}%
\pgfpathrectangle{\pgfqpoint{0.041670in}{0.041670in}}{\pgfqpoint{2.216660in}{2.216660in}}%
\pgfusepath{clip}%
\pgfsetbuttcap%
\pgfsetroundjoin%
\definecolor{currentfill}{rgb}{0.896320,0.893616,0.096335}%
\pgfsetfillcolor{currentfill}%
\pgfsetlinewidth{0.000000pt}%
\definecolor{currentstroke}{rgb}{0.000000,0.000000,0.000000}%
\pgfsetstrokecolor{currentstroke}%
\pgfsetdash{}{0pt}%
\pgfpathmoveto{\pgfqpoint{1.200647in}{1.591865in}}%
\pgfpathlineto{\pgfqpoint{1.201684in}{1.587387in}}%
\pgfpathlineto{\pgfqpoint{1.202722in}{1.582782in}}%
\pgfpathlineto{\pgfqpoint{1.203760in}{1.578049in}}%
\pgfpathlineto{\pgfqpoint{1.204799in}{1.573190in}}%
\pgfpathlineto{\pgfqpoint{1.201916in}{1.572841in}}%
\pgfpathlineto{\pgfqpoint{1.199011in}{1.572536in}}%
\pgfpathlineto{\pgfqpoint{1.196086in}{1.572274in}}%
\pgfpathlineto{\pgfqpoint{1.193146in}{1.572055in}}%
\pgfpathlineto{\pgfqpoint{1.192594in}{1.576963in}}%
\pgfpathlineto{\pgfqpoint{1.192043in}{1.581744in}}%
\pgfpathlineto{\pgfqpoint{1.191492in}{1.586397in}}%
\pgfpathlineto{\pgfqpoint{1.190941in}{1.590923in}}%
\pgfpathlineto{\pgfqpoint{1.193390in}{1.591105in}}%
\pgfpathlineto{\pgfqpoint{1.195826in}{1.591322in}}%
\pgfpathlineto{\pgfqpoint{1.198246in}{1.591576in}}%
\pgfpathlineto{\pgfqpoint{1.200647in}{1.591865in}}%
\pgfpathclose%
\pgfusepath{fill}%
\end{pgfscope}%
\begin{pgfscope}%
\pgfpathrectangle{\pgfqpoint{0.041670in}{0.041670in}}{\pgfqpoint{2.216660in}{2.216660in}}%
\pgfusepath{clip}%
\pgfsetbuttcap%
\pgfsetroundjoin%
\definecolor{currentfill}{rgb}{0.281477,0.755203,0.432552}%
\pgfsetfillcolor{currentfill}%
\pgfsetlinewidth{0.000000pt}%
\definecolor{currentstroke}{rgb}{0.000000,0.000000,0.000000}%
\pgfsetstrokecolor{currentstroke}%
\pgfsetdash{}{0pt}%
\pgfpathmoveto{\pgfqpoint{1.296863in}{1.334865in}}%
\pgfpathlineto{\pgfqpoint{1.298802in}{1.326411in}}%
\pgfpathlineto{\pgfqpoint{1.300741in}{1.317887in}}%
\pgfpathlineto{\pgfqpoint{1.302678in}{1.309296in}}%
\pgfpathlineto{\pgfqpoint{1.304615in}{1.300640in}}%
\pgfpathlineto{\pgfqpoint{1.297700in}{1.298759in}}%
\pgfpathlineto{\pgfqpoint{1.290662in}{1.296986in}}%
\pgfpathlineto{\pgfqpoint{1.283508in}{1.295323in}}%
\pgfpathlineto{\pgfqpoint{1.276245in}{1.293772in}}%
\pgfpathlineto{\pgfqpoint{1.274748in}{1.302544in}}%
\pgfpathlineto{\pgfqpoint{1.273249in}{1.311250in}}%
\pgfpathlineto{\pgfqpoint{1.271751in}{1.319889in}}%
\pgfpathlineto{\pgfqpoint{1.270251in}{1.328457in}}%
\pgfpathlineto{\pgfqpoint{1.277063in}{1.329905in}}%
\pgfpathlineto{\pgfqpoint{1.283774in}{1.331456in}}%
\pgfpathlineto{\pgfqpoint{1.290376in}{1.333110in}}%
\pgfpathlineto{\pgfqpoint{1.296863in}{1.334865in}}%
\pgfpathclose%
\pgfusepath{fill}%
\end{pgfscope}%
\begin{pgfscope}%
\pgfpathrectangle{\pgfqpoint{0.041670in}{0.041670in}}{\pgfqpoint{2.216660in}{2.216660in}}%
\pgfusepath{clip}%
\pgfsetbuttcap%
\pgfsetroundjoin%
\definecolor{currentfill}{rgb}{0.855810,0.888601,0.097452}%
\pgfsetfillcolor{currentfill}%
\pgfsetlinewidth{0.000000pt}%
\definecolor{currentstroke}{rgb}{0.000000,0.000000,0.000000}%
\pgfsetstrokecolor{currentstroke}%
\pgfsetdash{}{0pt}%
\pgfpathmoveto{\pgfqpoint{1.146317in}{1.574545in}}%
\pgfpathlineto{\pgfqpoint{1.144912in}{1.569618in}}%
\pgfpathlineto{\pgfqpoint{1.143506in}{1.564567in}}%
\pgfpathlineto{\pgfqpoint{1.142100in}{1.559392in}}%
\pgfpathlineto{\pgfqpoint{1.140694in}{1.554094in}}%
\pgfpathlineto{\pgfqpoint{1.137473in}{1.554705in}}%
\pgfpathlineto{\pgfqpoint{1.134295in}{1.555363in}}%
\pgfpathlineto{\pgfqpoint{1.131163in}{1.556068in}}%
\pgfpathlineto{\pgfqpoint{1.128081in}{1.556819in}}%
\pgfpathlineto{\pgfqpoint{1.129938in}{1.562017in}}%
\pgfpathlineto{\pgfqpoint{1.131795in}{1.567093in}}%
\pgfpathlineto{\pgfqpoint{1.133651in}{1.572045in}}%
\pgfpathlineto{\pgfqpoint{1.135508in}{1.576873in}}%
\pgfpathlineto{\pgfqpoint{1.138149in}{1.576231in}}%
\pgfpathlineto{\pgfqpoint{1.140833in}{1.575629in}}%
\pgfpathlineto{\pgfqpoint{1.143557in}{1.575067in}}%
\pgfpathlineto{\pgfqpoint{1.146317in}{1.574545in}}%
\pgfpathclose%
\pgfusepath{fill}%
\end{pgfscope}%
\begin{pgfscope}%
\pgfpathrectangle{\pgfqpoint{0.041670in}{0.041670in}}{\pgfqpoint{2.216660in}{2.216660in}}%
\pgfusepath{clip}%
\pgfsetbuttcap%
\pgfsetroundjoin%
\definecolor{currentfill}{rgb}{0.267004,0.004874,0.329415}%
\pgfsetfillcolor{currentfill}%
\pgfsetlinewidth{0.000000pt}%
\definecolor{currentstroke}{rgb}{0.000000,0.000000,0.000000}%
\pgfsetstrokecolor{currentstroke}%
\pgfsetdash{}{0pt}%
\pgfpathmoveto{\pgfqpoint{1.035120in}{0.564809in}}%
\pgfpathlineto{\pgfqpoint{1.034149in}{0.562996in}}%
\pgfpathlineto{\pgfqpoint{1.033176in}{0.561444in}}%
\pgfpathlineto{\pgfqpoint{1.032201in}{0.560159in}}%
\pgfpathlineto{\pgfqpoint{1.031222in}{0.559145in}}%
\pgfpathlineto{\pgfqpoint{1.012050in}{0.561868in}}%
\pgfpathlineto{\pgfqpoint{0.993067in}{0.564919in}}%
\pgfpathlineto{\pgfqpoint{0.974294in}{0.568293in}}%
\pgfpathlineto{\pgfqpoint{0.955753in}{0.571985in}}%
\pgfpathlineto{\pgfqpoint{0.957225in}{0.572913in}}%
\pgfpathlineto{\pgfqpoint{0.958694in}{0.574112in}}%
\pgfpathlineto{\pgfqpoint{0.960158in}{0.575577in}}%
\pgfpathlineto{\pgfqpoint{0.961619in}{0.577304in}}%
\pgfpathlineto{\pgfqpoint{0.979677in}{0.573711in}}%
\pgfpathlineto{\pgfqpoint{0.997961in}{0.570428in}}%
\pgfpathlineto{\pgfqpoint{1.016448in}{0.567459in}}%
\pgfpathlineto{\pgfqpoint{1.035120in}{0.564809in}}%
\pgfpathclose%
\pgfusepath{fill}%
\end{pgfscope}%
\begin{pgfscope}%
\pgfpathrectangle{\pgfqpoint{0.041670in}{0.041670in}}{\pgfqpoint{2.216660in}{2.216660in}}%
\pgfusepath{clip}%
\pgfsetbuttcap%
\pgfsetroundjoin%
\definecolor{currentfill}{rgb}{0.896320,0.893616,0.096335}%
\pgfsetfillcolor{currentfill}%
\pgfsetlinewidth{0.000000pt}%
\definecolor{currentstroke}{rgb}{0.000000,0.000000,0.000000}%
\pgfsetstrokecolor{currentstroke}%
\pgfsetdash{}{0pt}%
\pgfpathmoveto{\pgfqpoint{1.171156in}{1.590793in}}%
\pgfpathlineto{\pgfqpoint{1.170714in}{1.586260in}}%
\pgfpathlineto{\pgfqpoint{1.170273in}{1.581600in}}%
\pgfpathlineto{\pgfqpoint{1.169831in}{1.576812in}}%
\pgfpathlineto{\pgfqpoint{1.169389in}{1.571898in}}%
\pgfpathlineto{\pgfqpoint{1.166437in}{1.572078in}}%
\pgfpathlineto{\pgfqpoint{1.163497in}{1.572301in}}%
\pgfpathlineto{\pgfqpoint{1.160575in}{1.572568in}}%
\pgfpathlineto{\pgfqpoint{1.157672in}{1.572878in}}%
\pgfpathlineto{\pgfqpoint{1.158604in}{1.577750in}}%
\pgfpathlineto{\pgfqpoint{1.159535in}{1.582496in}}%
\pgfpathlineto{\pgfqpoint{1.160466in}{1.587115in}}%
\pgfpathlineto{\pgfqpoint{1.161396in}{1.591606in}}%
\pgfpathlineto{\pgfqpoint{1.163814in}{1.591349in}}%
\pgfpathlineto{\pgfqpoint{1.166248in}{1.591127in}}%
\pgfpathlineto{\pgfqpoint{1.168696in}{1.590942in}}%
\pgfpathlineto{\pgfqpoint{1.171156in}{1.590793in}}%
\pgfpathclose%
\pgfusepath{fill}%
\end{pgfscope}%
\begin{pgfscope}%
\pgfpathrectangle{\pgfqpoint{0.041670in}{0.041670in}}{\pgfqpoint{2.216660in}{2.216660in}}%
\pgfusepath{clip}%
\pgfsetbuttcap%
\pgfsetroundjoin%
\definecolor{currentfill}{rgb}{0.762373,0.876424,0.137064}%
\pgfsetfillcolor{currentfill}%
\pgfsetlinewidth{0.000000pt}%
\definecolor{currentstroke}{rgb}{0.000000,0.000000,0.000000}%
\pgfsetstrokecolor{currentstroke}%
\pgfsetdash{}{0pt}%
\pgfpathmoveto{\pgfqpoint{1.120655in}{1.534823in}}%
\pgfpathlineto{\pgfqpoint{1.118798in}{1.529028in}}%
\pgfpathlineto{\pgfqpoint{1.116942in}{1.523117in}}%
\pgfpathlineto{\pgfqpoint{1.115086in}{1.517091in}}%
\pgfpathlineto{\pgfqpoint{1.113230in}{1.510952in}}%
\pgfpathlineto{\pgfqpoint{1.109336in}{1.511984in}}%
\pgfpathlineto{\pgfqpoint{1.105515in}{1.513074in}}%
\pgfpathlineto{\pgfqpoint{1.101769in}{1.514221in}}%
\pgfpathlineto{\pgfqpoint{1.098104in}{1.515423in}}%
\pgfpathlineto{\pgfqpoint{1.100379in}{1.521434in}}%
\pgfpathlineto{\pgfqpoint{1.102654in}{1.527332in}}%
\pgfpathlineto{\pgfqpoint{1.104930in}{1.533115in}}%
\pgfpathlineto{\pgfqpoint{1.107206in}{1.538783in}}%
\pgfpathlineto{\pgfqpoint{1.110465in}{1.537718in}}%
\pgfpathlineto{\pgfqpoint{1.113796in}{1.536702in}}%
\pgfpathlineto{\pgfqpoint{1.117193in}{1.535737in}}%
\pgfpathlineto{\pgfqpoint{1.120655in}{1.534823in}}%
\pgfpathclose%
\pgfusepath{fill}%
\end{pgfscope}%
\begin{pgfscope}%
\pgfpathrectangle{\pgfqpoint{0.041670in}{0.041670in}}{\pgfqpoint{2.216660in}{2.216660in}}%
\pgfusepath{clip}%
\pgfsetbuttcap%
\pgfsetroundjoin%
\definecolor{currentfill}{rgb}{0.134692,0.658636,0.517649}%
\pgfsetfillcolor{currentfill}%
\pgfsetlinewidth{0.000000pt}%
\definecolor{currentstroke}{rgb}{0.000000,0.000000,0.000000}%
\pgfsetstrokecolor{currentstroke}%
\pgfsetdash{}{0pt}%
\pgfpathmoveto{\pgfqpoint{1.113080in}{1.214759in}}%
\pgfpathlineto{\pgfqpoint{1.112157in}{1.205394in}}%
\pgfpathlineto{\pgfqpoint{1.111234in}{1.195986in}}%
\pgfpathlineto{\pgfqpoint{1.110312in}{1.186538in}}%
\pgfpathlineto{\pgfqpoint{1.109389in}{1.177052in}}%
\pgfpathlineto{\pgfqpoint{1.100278in}{1.178238in}}%
\pgfpathlineto{\pgfqpoint{1.091251in}{1.179568in}}%
\pgfpathlineto{\pgfqpoint{1.082318in}{1.181038in}}%
\pgfpathlineto{\pgfqpoint{1.073488in}{1.182649in}}%
\pgfpathlineto{\pgfqpoint{1.074879in}{1.192055in}}%
\pgfpathlineto{\pgfqpoint{1.076269in}{1.201423in}}%
\pgfpathlineto{\pgfqpoint{1.077661in}{1.210750in}}%
\pgfpathlineto{\pgfqpoint{1.079053in}{1.220035in}}%
\pgfpathlineto{\pgfqpoint{1.087422in}{1.218517in}}%
\pgfpathlineto{\pgfqpoint{1.095889in}{1.217130in}}%
\pgfpathlineto{\pgfqpoint{1.104445in}{1.215877in}}%
\pgfpathlineto{\pgfqpoint{1.113080in}{1.214759in}}%
\pgfpathclose%
\pgfusepath{fill}%
\end{pgfscope}%
\begin{pgfscope}%
\pgfpathrectangle{\pgfqpoint{0.041670in}{0.041670in}}{\pgfqpoint{2.216660in}{2.216660in}}%
\pgfusepath{clip}%
\pgfsetbuttcap%
\pgfsetroundjoin%
\definecolor{currentfill}{rgb}{0.565498,0.842430,0.262877}%
\pgfsetfillcolor{currentfill}%
\pgfsetlinewidth{0.000000pt}%
\definecolor{currentstroke}{rgb}{0.000000,0.000000,0.000000}%
\pgfsetstrokecolor{currentstroke}%
\pgfsetdash{}{0pt}%
\pgfpathmoveto{\pgfqpoint{1.098393in}{1.457902in}}%
\pgfpathlineto{\pgfqpoint{1.096540in}{1.450800in}}%
\pgfpathlineto{\pgfqpoint{1.094688in}{1.443599in}}%
\pgfpathlineto{\pgfqpoint{1.092836in}{1.436300in}}%
\pgfpathlineto{\pgfqpoint{1.090984in}{1.428905in}}%
\pgfpathlineto{\pgfqpoint{1.085797in}{1.430298in}}%
\pgfpathlineto{\pgfqpoint{1.080707in}{1.431770in}}%
\pgfpathlineto{\pgfqpoint{1.075719in}{1.433318in}}%
\pgfpathlineto{\pgfqpoint{1.070839in}{1.434940in}}%
\pgfpathlineto{\pgfqpoint{1.073108in}{1.442203in}}%
\pgfpathlineto{\pgfqpoint{1.075377in}{1.449370in}}%
\pgfpathlineto{\pgfqpoint{1.077647in}{1.456440in}}%
\pgfpathlineto{\pgfqpoint{1.079918in}{1.463410in}}%
\pgfpathlineto{\pgfqpoint{1.084394in}{1.461929in}}%
\pgfpathlineto{\pgfqpoint{1.088969in}{1.460517in}}%
\pgfpathlineto{\pgfqpoint{1.093636in}{1.459174in}}%
\pgfpathlineto{\pgfqpoint{1.098393in}{1.457902in}}%
\pgfpathclose%
\pgfusepath{fill}%
\end{pgfscope}%
\begin{pgfscope}%
\pgfpathrectangle{\pgfqpoint{0.041670in}{0.041670in}}{\pgfqpoint{2.216660in}{2.216660in}}%
\pgfusepath{clip}%
\pgfsetbuttcap%
\pgfsetroundjoin%
\definecolor{currentfill}{rgb}{0.896320,0.893616,0.096335}%
\pgfsetfillcolor{currentfill}%
\pgfsetlinewidth{0.000000pt}%
\definecolor{currentstroke}{rgb}{0.000000,0.000000,0.000000}%
\pgfsetstrokecolor{currentstroke}%
\pgfsetdash{}{0pt}%
\pgfpathmoveto{\pgfqpoint{1.190941in}{1.590923in}}%
\pgfpathlineto{\pgfqpoint{1.191492in}{1.586397in}}%
\pgfpathlineto{\pgfqpoint{1.192043in}{1.581744in}}%
\pgfpathlineto{\pgfqpoint{1.192594in}{1.576963in}}%
\pgfpathlineto{\pgfqpoint{1.193146in}{1.572055in}}%
\pgfpathlineto{\pgfqpoint{1.190191in}{1.571881in}}%
\pgfpathlineto{\pgfqpoint{1.187227in}{1.571751in}}%
\pgfpathlineto{\pgfqpoint{1.184255in}{1.571664in}}%
\pgfpathlineto{\pgfqpoint{1.181278in}{1.571623in}}%
\pgfpathlineto{\pgfqpoint{1.181223in}{1.576548in}}%
\pgfpathlineto{\pgfqpoint{1.181168in}{1.581348in}}%
\pgfpathlineto{\pgfqpoint{1.181112in}{1.586020in}}%
\pgfpathlineto{\pgfqpoint{1.181057in}{1.590564in}}%
\pgfpathlineto{\pgfqpoint{1.183536in}{1.590599in}}%
\pgfpathlineto{\pgfqpoint{1.186011in}{1.590670in}}%
\pgfpathlineto{\pgfqpoint{1.188480in}{1.590778in}}%
\pgfpathlineto{\pgfqpoint{1.190941in}{1.590923in}}%
\pgfpathclose%
\pgfusepath{fill}%
\end{pgfscope}%
\begin{pgfscope}%
\pgfpathrectangle{\pgfqpoint{0.041670in}{0.041670in}}{\pgfqpoint{2.216660in}{2.216660in}}%
\pgfusepath{clip}%
\pgfsetbuttcap%
\pgfsetroundjoin%
\definecolor{currentfill}{rgb}{0.896320,0.893616,0.096335}%
\pgfsetfillcolor{currentfill}%
\pgfsetlinewidth{0.000000pt}%
\definecolor{currentstroke}{rgb}{0.000000,0.000000,0.000000}%
\pgfsetstrokecolor{currentstroke}%
\pgfsetdash{}{0pt}%
\pgfpathmoveto{\pgfqpoint{1.181057in}{1.590564in}}%
\pgfpathlineto{\pgfqpoint{1.181112in}{1.586020in}}%
\pgfpathlineto{\pgfqpoint{1.181168in}{1.581348in}}%
\pgfpathlineto{\pgfqpoint{1.181223in}{1.576548in}}%
\pgfpathlineto{\pgfqpoint{1.181278in}{1.571623in}}%
\pgfpathlineto{\pgfqpoint{1.178301in}{1.571625in}}%
\pgfpathlineto{\pgfqpoint{1.175324in}{1.571672in}}%
\pgfpathlineto{\pgfqpoint{1.172353in}{1.571763in}}%
\pgfpathlineto{\pgfqpoint{1.169389in}{1.571898in}}%
\pgfpathlineto{\pgfqpoint{1.169831in}{1.576812in}}%
\pgfpathlineto{\pgfqpoint{1.170273in}{1.581600in}}%
\pgfpathlineto{\pgfqpoint{1.170714in}{1.586260in}}%
\pgfpathlineto{\pgfqpoint{1.171156in}{1.590793in}}%
\pgfpathlineto{\pgfqpoint{1.173624in}{1.590680in}}%
\pgfpathlineto{\pgfqpoint{1.176098in}{1.590605in}}%
\pgfpathlineto{\pgfqpoint{1.178577in}{1.590566in}}%
\pgfpathlineto{\pgfqpoint{1.181057in}{1.590564in}}%
\pgfpathclose%
\pgfusepath{fill}%
\end{pgfscope}%
\begin{pgfscope}%
\pgfpathrectangle{\pgfqpoint{0.041670in}{0.041670in}}{\pgfqpoint{2.216660in}{2.216660in}}%
\pgfusepath{clip}%
\pgfsetbuttcap%
\pgfsetroundjoin%
\definecolor{currentfill}{rgb}{0.282884,0.135920,0.453427}%
\pgfsetfillcolor{currentfill}%
\pgfsetlinewidth{0.000000pt}%
\definecolor{currentstroke}{rgb}{0.000000,0.000000,0.000000}%
\pgfsetstrokecolor{currentstroke}%
\pgfsetdash{}{0pt}%
\pgfpathmoveto{\pgfqpoint{0.779312in}{0.643154in}}%
\pgfpathlineto{\pgfqpoint{0.776812in}{0.647956in}}%
\pgfpathlineto{\pgfqpoint{0.774302in}{0.653134in}}%
\pgfpathlineto{\pgfqpoint{0.771781in}{0.658694in}}%
\pgfpathlineto{\pgfqpoint{0.769250in}{0.664643in}}%
\pgfpathlineto{\pgfqpoint{0.751605in}{0.671757in}}%
\pgfpathlineto{\pgfqpoint{0.734439in}{0.679158in}}%
\pgfpathlineto{\pgfqpoint{0.717769in}{0.686838in}}%
\pgfpathlineto{\pgfqpoint{0.701612in}{0.694788in}}%
\pgfpathlineto{\pgfqpoint{0.704548in}{0.688678in}}%
\pgfpathlineto{\pgfqpoint{0.707472in}{0.682956in}}%
\pgfpathlineto{\pgfqpoint{0.710384in}{0.677615in}}%
\pgfpathlineto{\pgfqpoint{0.713285in}{0.672649in}}%
\pgfpathlineto{\pgfqpoint{0.729059in}{0.664870in}}%
\pgfpathlineto{\pgfqpoint{0.745333in}{0.657355in}}%
\pgfpathlineto{\pgfqpoint{0.762090in}{0.650114in}}%
\pgfpathlineto{\pgfqpoint{0.779312in}{0.643154in}}%
\pgfpathclose%
\pgfusepath{fill}%
\end{pgfscope}%
\begin{pgfscope}%
\pgfpathrectangle{\pgfqpoint{0.041670in}{0.041670in}}{\pgfqpoint{2.216660in}{2.216660in}}%
\pgfusepath{clip}%
\pgfsetbuttcap%
\pgfsetroundjoin%
\definecolor{currentfill}{rgb}{0.699415,0.867117,0.175971}%
\pgfsetfillcolor{currentfill}%
\pgfsetlinewidth{0.000000pt}%
\definecolor{currentstroke}{rgb}{0.000000,0.000000,0.000000}%
\pgfsetstrokecolor{currentstroke}%
\pgfsetdash{}{0pt}%
\pgfpathmoveto{\pgfqpoint{1.113230in}{1.510952in}}%
\pgfpathlineto{\pgfqpoint{1.111375in}{1.504700in}}%
\pgfpathlineto{\pgfqpoint{1.109519in}{1.498336in}}%
\pgfpathlineto{\pgfqpoint{1.107664in}{1.491864in}}%
\pgfpathlineto{\pgfqpoint{1.105809in}{1.485282in}}%
\pgfpathlineto{\pgfqpoint{1.101484in}{1.486434in}}%
\pgfpathlineto{\pgfqpoint{1.097238in}{1.487650in}}%
\pgfpathlineto{\pgfqpoint{1.093078in}{1.488929in}}%
\pgfpathlineto{\pgfqpoint{1.089007in}{1.490270in}}%
\pgfpathlineto{\pgfqpoint{1.091281in}{1.496722in}}%
\pgfpathlineto{\pgfqpoint{1.093555in}{1.503065in}}%
\pgfpathlineto{\pgfqpoint{1.095829in}{1.509299in}}%
\pgfpathlineto{\pgfqpoint{1.098104in}{1.515423in}}%
\pgfpathlineto{\pgfqpoint{1.101769in}{1.514221in}}%
\pgfpathlineto{\pgfqpoint{1.105515in}{1.513074in}}%
\pgfpathlineto{\pgfqpoint{1.109336in}{1.511984in}}%
\pgfpathlineto{\pgfqpoint{1.113230in}{1.510952in}}%
\pgfpathclose%
\pgfusepath{fill}%
\end{pgfscope}%
\begin{pgfscope}%
\pgfpathrectangle{\pgfqpoint{0.041670in}{0.041670in}}{\pgfqpoint{2.216660in}{2.216660in}}%
\pgfusepath{clip}%
\pgfsetbuttcap%
\pgfsetroundjoin%
\definecolor{currentfill}{rgb}{0.636902,0.856542,0.216620}%
\pgfsetfillcolor{currentfill}%
\pgfsetlinewidth{0.000000pt}%
\definecolor{currentstroke}{rgb}{0.000000,0.000000,0.000000}%
\pgfsetstrokecolor{currentstroke}%
\pgfsetdash{}{0pt}%
\pgfpathmoveto{\pgfqpoint{1.105809in}{1.485282in}}%
\pgfpathlineto{\pgfqpoint{1.103955in}{1.478594in}}%
\pgfpathlineto{\pgfqpoint{1.102100in}{1.471800in}}%
\pgfpathlineto{\pgfqpoint{1.100247in}{1.464902in}}%
\pgfpathlineto{\pgfqpoint{1.098393in}{1.457902in}}%
\pgfpathlineto{\pgfqpoint{1.093636in}{1.459174in}}%
\pgfpathlineto{\pgfqpoint{1.088969in}{1.460517in}}%
\pgfpathlineto{\pgfqpoint{1.084394in}{1.461929in}}%
\pgfpathlineto{\pgfqpoint{1.079918in}{1.463410in}}%
\pgfpathlineto{\pgfqpoint{1.082190in}{1.470280in}}%
\pgfpathlineto{\pgfqpoint{1.084462in}{1.477048in}}%
\pgfpathlineto{\pgfqpoint{1.086734in}{1.483711in}}%
\pgfpathlineto{\pgfqpoint{1.089007in}{1.490270in}}%
\pgfpathlineto{\pgfqpoint{1.093078in}{1.488929in}}%
\pgfpathlineto{\pgfqpoint{1.097238in}{1.487650in}}%
\pgfpathlineto{\pgfqpoint{1.101484in}{1.486434in}}%
\pgfpathlineto{\pgfqpoint{1.105809in}{1.485282in}}%
\pgfpathclose%
\pgfusepath{fill}%
\end{pgfscope}%
\begin{pgfscope}%
\pgfpathrectangle{\pgfqpoint{0.041670in}{0.041670in}}{\pgfqpoint{2.216660in}{2.216660in}}%
\pgfusepath{clip}%
\pgfsetbuttcap%
\pgfsetroundjoin%
\definecolor{currentfill}{rgb}{0.281477,0.755203,0.432552}%
\pgfsetfillcolor{currentfill}%
\pgfsetlinewidth{0.000000pt}%
\definecolor{currentstroke}{rgb}{0.000000,0.000000,0.000000}%
\pgfsetstrokecolor{currentstroke}%
\pgfsetdash{}{0pt}%
\pgfpathmoveto{\pgfqpoint{1.095793in}{1.327259in}}%
\pgfpathlineto{\pgfqpoint{1.094395in}{1.318669in}}%
\pgfpathlineto{\pgfqpoint{1.092998in}{1.310010in}}%
\pgfpathlineto{\pgfqpoint{1.091601in}{1.301282in}}%
\pgfpathlineto{\pgfqpoint{1.090205in}{1.292488in}}%
\pgfpathlineto{\pgfqpoint{1.082852in}{1.293939in}}%
\pgfpathlineto{\pgfqpoint{1.075602in}{1.295503in}}%
\pgfpathlineto{\pgfqpoint{1.068460in}{1.297178in}}%
\pgfpathlineto{\pgfqpoint{1.061435in}{1.298963in}}%
\pgfpathlineto{\pgfqpoint{1.063277in}{1.307648in}}%
\pgfpathlineto{\pgfqpoint{1.065120in}{1.316267in}}%
\pgfpathlineto{\pgfqpoint{1.066963in}{1.324819in}}%
\pgfpathlineto{\pgfqpoint{1.068807in}{1.333300in}}%
\pgfpathlineto{\pgfqpoint{1.075397in}{1.331635in}}%
\pgfpathlineto{\pgfqpoint{1.082096in}{1.330072in}}%
\pgfpathlineto{\pgfqpoint{1.088897in}{1.328613in}}%
\pgfpathlineto{\pgfqpoint{1.095793in}{1.327259in}}%
\pgfpathclose%
\pgfusepath{fill}%
\end{pgfscope}%
\begin{pgfscope}%
\pgfpathrectangle{\pgfqpoint{0.041670in}{0.041670in}}{\pgfqpoint{2.216660in}{2.216660in}}%
\pgfusepath{clip}%
\pgfsetbuttcap%
\pgfsetroundjoin%
\definecolor{currentfill}{rgb}{0.267004,0.004874,0.329415}%
\pgfsetfillcolor{currentfill}%
\pgfsetlinewidth{0.000000pt}%
\definecolor{currentstroke}{rgb}{0.000000,0.000000,0.000000}%
\pgfsetstrokecolor{currentstroke}%
\pgfsetdash{}{0pt}%
\pgfpathmoveto{\pgfqpoint{1.113057in}{0.567425in}}%
\pgfpathlineto{\pgfqpoint{1.112600in}{0.564565in}}%
\pgfpathlineto{\pgfqpoint{1.112142in}{0.561946in}}%
\pgfpathlineto{\pgfqpoint{1.111684in}{0.559573in}}%
\pgfpathlineto{\pgfqpoint{1.111224in}{0.557452in}}%
\pgfpathlineto{\pgfqpoint{1.092028in}{0.558800in}}%
\pgfpathlineto{\pgfqpoint{1.072931in}{0.560477in}}%
\pgfpathlineto{\pgfqpoint{1.053954in}{0.562481in}}%
\pgfpathlineto{\pgfqpoint{1.035120in}{0.564809in}}%
\pgfpathlineto{\pgfqpoint{1.036088in}{0.566879in}}%
\pgfpathlineto{\pgfqpoint{1.037054in}{0.569200in}}%
\pgfpathlineto{\pgfqpoint{1.038017in}{0.571768in}}%
\pgfpathlineto{\pgfqpoint{1.038979in}{0.574577in}}%
\pgfpathlineto{\pgfqpoint{1.057313in}{0.572313in}}%
\pgfpathlineto{\pgfqpoint{1.075784in}{0.570365in}}%
\pgfpathlineto{\pgfqpoint{1.094372in}{0.568735in}}%
\pgfpathlineto{\pgfqpoint{1.113057in}{0.567425in}}%
\pgfpathclose%
\pgfusepath{fill}%
\end{pgfscope}%
\begin{pgfscope}%
\pgfpathrectangle{\pgfqpoint{0.041670in}{0.041670in}}{\pgfqpoint{2.216660in}{2.216660in}}%
\pgfusepath{clip}%
\pgfsetbuttcap%
\pgfsetroundjoin%
\definecolor{currentfill}{rgb}{0.272594,0.025563,0.353093}%
\pgfsetfillcolor{currentfill}%
\pgfsetlinewidth{0.000000pt}%
\definecolor{currentstroke}{rgb}{0.000000,0.000000,0.000000}%
\pgfsetstrokecolor{currentstroke}%
\pgfsetdash{}{0pt}%
\pgfpathmoveto{\pgfqpoint{1.499005in}{0.594172in}}%
\pgfpathlineto{\pgfqpoint{1.501069in}{0.594831in}}%
\pgfpathlineto{\pgfqpoint{1.503140in}{0.595794in}}%
\pgfpathlineto{\pgfqpoint{1.505217in}{0.597065in}}%
\pgfpathlineto{\pgfqpoint{1.507301in}{0.598652in}}%
\pgfpathlineto{\pgfqpoint{1.489281in}{0.593202in}}%
\pgfpathlineto{\pgfqpoint{1.470914in}{0.588059in}}%
\pgfpathlineto{\pgfqpoint{1.452220in}{0.583231in}}%
\pgfpathlineto{\pgfqpoint{1.433220in}{0.578723in}}%
\pgfpathlineto{\pgfqpoint{1.431603in}{0.577260in}}%
\pgfpathlineto{\pgfqpoint{1.429992in}{0.576111in}}%
\pgfpathlineto{\pgfqpoint{1.428385in}{0.575273in}}%
\pgfpathlineto{\pgfqpoint{1.426784in}{0.574738in}}%
\pgfpathlineto{\pgfqpoint{1.445306in}{0.579134in}}%
\pgfpathlineto{\pgfqpoint{1.463530in}{0.583842in}}%
\pgfpathlineto{\pgfqpoint{1.481436in}{0.588857in}}%
\pgfpathlineto{\pgfqpoint{1.499005in}{0.594172in}}%
\pgfpathclose%
\pgfusepath{fill}%
\end{pgfscope}%
\begin{pgfscope}%
\pgfpathrectangle{\pgfqpoint{0.041670in}{0.041670in}}{\pgfqpoint{2.216660in}{2.216660in}}%
\pgfusepath{clip}%
\pgfsetbuttcap%
\pgfsetroundjoin%
\definecolor{currentfill}{rgb}{0.120081,0.622161,0.534946}%
\pgfsetfillcolor{currentfill}%
\pgfsetlinewidth{0.000000pt}%
\definecolor{currentstroke}{rgb}{0.000000,0.000000,0.000000}%
\pgfsetstrokecolor{currentstroke}%
\pgfsetdash{}{0pt}%
\pgfpathmoveto{\pgfqpoint{1.258623in}{1.178099in}}%
\pgfpathlineto{\pgfqpoint{1.259651in}{1.168593in}}%
\pgfpathlineto{\pgfqpoint{1.260678in}{1.159053in}}%
\pgfpathlineto{\pgfqpoint{1.261705in}{1.149483in}}%
\pgfpathlineto{\pgfqpoint{1.262732in}{1.139886in}}%
\pgfpathlineto{\pgfqpoint{1.253136in}{1.138648in}}%
\pgfpathlineto{\pgfqpoint{1.243463in}{1.137562in}}%
\pgfpathlineto{\pgfqpoint{1.233722in}{1.136631in}}%
\pgfpathlineto{\pgfqpoint{1.223923in}{1.135854in}}%
\pgfpathlineto{\pgfqpoint{1.223378in}{1.145507in}}%
\pgfpathlineto{\pgfqpoint{1.222832in}{1.155132in}}%
\pgfpathlineto{\pgfqpoint{1.222286in}{1.164727in}}%
\pgfpathlineto{\pgfqpoint{1.221740in}{1.174289in}}%
\pgfpathlineto{\pgfqpoint{1.231052in}{1.175022in}}%
\pgfpathlineto{\pgfqpoint{1.240309in}{1.175903in}}%
\pgfpathlineto{\pgfqpoint{1.249503in}{1.176929in}}%
\pgfpathlineto{\pgfqpoint{1.258623in}{1.178099in}}%
\pgfpathclose%
\pgfusepath{fill}%
\end{pgfscope}%
\begin{pgfscope}%
\pgfpathrectangle{\pgfqpoint{0.041670in}{0.041670in}}{\pgfqpoint{2.216660in}{2.216660in}}%
\pgfusepath{clip}%
\pgfsetbuttcap%
\pgfsetroundjoin%
\definecolor{currentfill}{rgb}{0.814576,0.883393,0.110347}%
\pgfsetfillcolor{currentfill}%
\pgfsetlinewidth{0.000000pt}%
\definecolor{currentstroke}{rgb}{0.000000,0.000000,0.000000}%
\pgfsetstrokecolor{currentstroke}%
\pgfsetdash{}{0pt}%
\pgfpathmoveto{\pgfqpoint{1.234524in}{1.557526in}}%
\pgfpathlineto{\pgfqpoint{1.236477in}{1.552232in}}%
\pgfpathlineto{\pgfqpoint{1.238429in}{1.546818in}}%
\pgfpathlineto{\pgfqpoint{1.240382in}{1.541285in}}%
\pgfpathlineto{\pgfqpoint{1.242335in}{1.535633in}}%
\pgfpathlineto{\pgfqpoint{1.238866in}{1.534725in}}%
\pgfpathlineto{\pgfqpoint{1.235338in}{1.533869in}}%
\pgfpathlineto{\pgfqpoint{1.231752in}{1.533066in}}%
\pgfpathlineto{\pgfqpoint{1.228114in}{1.532317in}}%
\pgfpathlineto{\pgfqpoint{1.226606in}{1.538075in}}%
\pgfpathlineto{\pgfqpoint{1.225098in}{1.543715in}}%
\pgfpathlineto{\pgfqpoint{1.223589in}{1.549235in}}%
\pgfpathlineto{\pgfqpoint{1.222081in}{1.554635in}}%
\pgfpathlineto{\pgfqpoint{1.225264in}{1.555287in}}%
\pgfpathlineto{\pgfqpoint{1.228401in}{1.555987in}}%
\pgfpathlineto{\pgfqpoint{1.231488in}{1.556734in}}%
\pgfpathlineto{\pgfqpoint{1.234524in}{1.557526in}}%
\pgfpathclose%
\pgfusepath{fill}%
\end{pgfscope}%
\begin{pgfscope}%
\pgfpathrectangle{\pgfqpoint{0.041670in}{0.041670in}}{\pgfqpoint{2.216660in}{2.216660in}}%
\pgfusepath{clip}%
\pgfsetbuttcap%
\pgfsetroundjoin%
\definecolor{currentfill}{rgb}{0.166383,0.690856,0.496502}%
\pgfsetfillcolor{currentfill}%
\pgfsetlinewidth{0.000000pt}%
\definecolor{currentstroke}{rgb}{0.000000,0.000000,0.000000}%
\pgfsetstrokecolor{currentstroke}%
\pgfsetdash{}{0pt}%
\pgfpathmoveto{\pgfqpoint{1.282231in}{1.258069in}}%
\pgfpathlineto{\pgfqpoint{1.283726in}{1.249002in}}%
\pgfpathlineto{\pgfqpoint{1.285220in}{1.239882in}}%
\pgfpathlineto{\pgfqpoint{1.286714in}{1.230711in}}%
\pgfpathlineto{\pgfqpoint{1.288208in}{1.221494in}}%
\pgfpathlineto{\pgfqpoint{1.279932in}{1.219860in}}%
\pgfpathlineto{\pgfqpoint{1.271551in}{1.218356in}}%
\pgfpathlineto{\pgfqpoint{1.263074in}{1.216984in}}%
\pgfpathlineto{\pgfqpoint{1.254509in}{1.215746in}}%
\pgfpathlineto{\pgfqpoint{1.253480in}{1.225051in}}%
\pgfpathlineto{\pgfqpoint{1.252450in}{1.234307in}}%
\pgfpathlineto{\pgfqpoint{1.251420in}{1.243514in}}%
\pgfpathlineto{\pgfqpoint{1.250389in}{1.252668in}}%
\pgfpathlineto{\pgfqpoint{1.258482in}{1.253832in}}%
\pgfpathlineto{\pgfqpoint{1.266492in}{1.255121in}}%
\pgfpathlineto{\pgfqpoint{1.274411in}{1.256534in}}%
\pgfpathlineto{\pgfqpoint{1.282231in}{1.258069in}}%
\pgfpathclose%
\pgfusepath{fill}%
\end{pgfscope}%
\begin{pgfscope}%
\pgfpathrectangle{\pgfqpoint{0.041670in}{0.041670in}}{\pgfqpoint{2.216660in}{2.216660in}}%
\pgfusepath{clip}%
\pgfsetbuttcap%
\pgfsetroundjoin%
\definecolor{currentfill}{rgb}{0.855810,0.888601,0.097452}%
\pgfsetfillcolor{currentfill}%
\pgfsetlinewidth{0.000000pt}%
\definecolor{currentstroke}{rgb}{0.000000,0.000000,0.000000}%
\pgfsetstrokecolor{currentstroke}%
\pgfsetdash{}{0pt}%
\pgfpathmoveto{\pgfqpoint{1.216048in}{1.575007in}}%
\pgfpathlineto{\pgfqpoint{1.217556in}{1.570099in}}%
\pgfpathlineto{\pgfqpoint{1.219064in}{1.565068in}}%
\pgfpathlineto{\pgfqpoint{1.220573in}{1.559912in}}%
\pgfpathlineto{\pgfqpoint{1.222081in}{1.554635in}}%
\pgfpathlineto{\pgfqpoint{1.218855in}{1.554030in}}%
\pgfpathlineto{\pgfqpoint{1.215589in}{1.553473in}}%
\pgfpathlineto{\pgfqpoint{1.212287in}{1.552965in}}%
\pgfpathlineto{\pgfqpoint{1.208952in}{1.552507in}}%
\pgfpathlineto{\pgfqpoint{1.207914in}{1.557863in}}%
\pgfpathlineto{\pgfqpoint{1.206875in}{1.563096in}}%
\pgfpathlineto{\pgfqpoint{1.205837in}{1.568205in}}%
\pgfpathlineto{\pgfqpoint{1.204799in}{1.573190in}}%
\pgfpathlineto{\pgfqpoint{1.207656in}{1.573581in}}%
\pgfpathlineto{\pgfqpoint{1.210485in}{1.574014in}}%
\pgfpathlineto{\pgfqpoint{1.213284in}{1.574490in}}%
\pgfpathlineto{\pgfqpoint{1.216048in}{1.575007in}}%
\pgfpathclose%
\pgfusepath{fill}%
\end{pgfscope}%
\begin{pgfscope}%
\pgfpathrectangle{\pgfqpoint{0.041670in}{0.041670in}}{\pgfqpoint{2.216660in}{2.216660in}}%
\pgfusepath{clip}%
\pgfsetbuttcap%
\pgfsetroundjoin%
\definecolor{currentfill}{rgb}{0.344074,0.780029,0.397381}%
\pgfsetfillcolor{currentfill}%
\pgfsetlinewidth{0.000000pt}%
\definecolor{currentstroke}{rgb}{0.000000,0.000000,0.000000}%
\pgfsetstrokecolor{currentstroke}%
\pgfsetdash{}{0pt}%
\pgfpathmoveto{\pgfqpoint{1.289099in}{1.367941in}}%
\pgfpathlineto{\pgfqpoint{1.291041in}{1.359787in}}%
\pgfpathlineto{\pgfqpoint{1.292983in}{1.351555in}}%
\pgfpathlineto{\pgfqpoint{1.294923in}{1.343247in}}%
\pgfpathlineto{\pgfqpoint{1.296863in}{1.334865in}}%
\pgfpathlineto{\pgfqpoint{1.290376in}{1.333110in}}%
\pgfpathlineto{\pgfqpoint{1.283774in}{1.331456in}}%
\pgfpathlineto{\pgfqpoint{1.277063in}{1.329905in}}%
\pgfpathlineto{\pgfqpoint{1.270251in}{1.328457in}}%
\pgfpathlineto{\pgfqpoint{1.268752in}{1.336953in}}%
\pgfpathlineto{\pgfqpoint{1.267252in}{1.345375in}}%
\pgfpathlineto{\pgfqpoint{1.265751in}{1.353721in}}%
\pgfpathlineto{\pgfqpoint{1.264250in}{1.361989in}}%
\pgfpathlineto{\pgfqpoint{1.270610in}{1.363333in}}%
\pgfpathlineto{\pgfqpoint{1.276876in}{1.364775in}}%
\pgfpathlineto{\pgfqpoint{1.283041in}{1.366311in}}%
\pgfpathlineto{\pgfqpoint{1.289099in}{1.367941in}}%
\pgfpathclose%
\pgfusepath{fill}%
\end{pgfscope}%
\begin{pgfscope}%
\pgfpathrectangle{\pgfqpoint{0.041670in}{0.041670in}}{\pgfqpoint{2.216660in}{2.216660in}}%
\pgfusepath{clip}%
\pgfsetbuttcap%
\pgfsetroundjoin%
\definecolor{currentfill}{rgb}{0.122606,0.585371,0.546557}%
\pgfsetfillcolor{currentfill}%
\pgfsetlinewidth{0.000000pt}%
\definecolor{currentstroke}{rgb}{0.000000,0.000000,0.000000}%
\pgfsetstrokecolor{currentstroke}%
\pgfsetdash{}{0pt}%
\pgfpathmoveto{\pgfqpoint{1.223923in}{1.135854in}}%
\pgfpathlineto{\pgfqpoint{1.224469in}{1.126176in}}%
\pgfpathlineto{\pgfqpoint{1.225014in}{1.116475in}}%
\pgfpathlineto{\pgfqpoint{1.225559in}{1.106755in}}%
\pgfpathlineto{\pgfqpoint{1.226104in}{1.097017in}}%
\pgfpathlineto{\pgfqpoint{1.215771in}{1.096362in}}%
\pgfpathlineto{\pgfqpoint{1.205400in}{1.095873in}}%
\pgfpathlineto{\pgfqpoint{1.195001in}{1.095549in}}%
\pgfpathlineto{\pgfqpoint{1.184586in}{1.095392in}}%
\pgfpathlineto{\pgfqpoint{1.184531in}{1.105151in}}%
\pgfpathlineto{\pgfqpoint{1.184476in}{1.114893in}}%
\pgfpathlineto{\pgfqpoint{1.184422in}{1.124615in}}%
\pgfpathlineto{\pgfqpoint{1.184367in}{1.134314in}}%
\pgfpathlineto{\pgfqpoint{1.194289in}{1.134463in}}%
\pgfpathlineto{\pgfqpoint{1.204197in}{1.134770in}}%
\pgfpathlineto{\pgfqpoint{1.214078in}{1.135234in}}%
\pgfpathlineto{\pgfqpoint{1.223923in}{1.135854in}}%
\pgfpathclose%
\pgfusepath{fill}%
\end{pgfscope}%
\begin{pgfscope}%
\pgfpathrectangle{\pgfqpoint{0.041670in}{0.041670in}}{\pgfqpoint{2.216660in}{2.216660in}}%
\pgfusepath{clip}%
\pgfsetbuttcap%
\pgfsetroundjoin%
\definecolor{currentfill}{rgb}{0.122606,0.585371,0.546557}%
\pgfsetfillcolor{currentfill}%
\pgfsetlinewidth{0.000000pt}%
\definecolor{currentstroke}{rgb}{0.000000,0.000000,0.000000}%
\pgfsetstrokecolor{currentstroke}%
\pgfsetdash{}{0pt}%
\pgfpathmoveto{\pgfqpoint{1.184367in}{1.134314in}}%
\pgfpathlineto{\pgfqpoint{1.184422in}{1.124615in}}%
\pgfpathlineto{\pgfqpoint{1.184476in}{1.114893in}}%
\pgfpathlineto{\pgfqpoint{1.184531in}{1.105151in}}%
\pgfpathlineto{\pgfqpoint{1.184586in}{1.095392in}}%
\pgfpathlineto{\pgfqpoint{1.174166in}{1.095401in}}%
\pgfpathlineto{\pgfqpoint{1.163753in}{1.095577in}}%
\pgfpathlineto{\pgfqpoint{1.153356in}{1.095919in}}%
\pgfpathlineto{\pgfqpoint{1.142988in}{1.096427in}}%
\pgfpathlineto{\pgfqpoint{1.143425in}{1.106172in}}%
\pgfpathlineto{\pgfqpoint{1.143862in}{1.115901in}}%
\pgfpathlineto{\pgfqpoint{1.144299in}{1.125609in}}%
\pgfpathlineto{\pgfqpoint{1.144736in}{1.135295in}}%
\pgfpathlineto{\pgfqpoint{1.154614in}{1.134814in}}%
\pgfpathlineto{\pgfqpoint{1.164518in}{1.134490in}}%
\pgfpathlineto{\pgfqpoint{1.174440in}{1.134323in}}%
\pgfpathlineto{\pgfqpoint{1.184367in}{1.134314in}}%
\pgfpathclose%
\pgfusepath{fill}%
\end{pgfscope}%
\begin{pgfscope}%
\pgfpathrectangle{\pgfqpoint{0.041670in}{0.041670in}}{\pgfqpoint{2.216660in}{2.216660in}}%
\pgfusepath{clip}%
\pgfsetbuttcap%
\pgfsetroundjoin%
\definecolor{currentfill}{rgb}{0.855810,0.888601,0.097452}%
\pgfsetfillcolor{currentfill}%
\pgfsetlinewidth{0.000000pt}%
\definecolor{currentstroke}{rgb}{0.000000,0.000000,0.000000}%
\pgfsetstrokecolor{currentstroke}%
\pgfsetdash{}{0pt}%
\pgfpathmoveto{\pgfqpoint{1.157672in}{1.572878in}}%
\pgfpathlineto{\pgfqpoint{1.156741in}{1.567880in}}%
\pgfpathlineto{\pgfqpoint{1.155810in}{1.562757in}}%
\pgfpathlineto{\pgfqpoint{1.154878in}{1.557511in}}%
\pgfpathlineto{\pgfqpoint{1.153947in}{1.552142in}}%
\pgfpathlineto{\pgfqpoint{1.150585in}{1.552556in}}%
\pgfpathlineto{\pgfqpoint{1.147254in}{1.553019in}}%
\pgfpathlineto{\pgfqpoint{1.143956in}{1.553532in}}%
\pgfpathlineto{\pgfqpoint{1.140694in}{1.554094in}}%
\pgfpathlineto{\pgfqpoint{1.142100in}{1.559392in}}%
\pgfpathlineto{\pgfqpoint{1.143506in}{1.564567in}}%
\pgfpathlineto{\pgfqpoint{1.144912in}{1.569618in}}%
\pgfpathlineto{\pgfqpoint{1.146317in}{1.574545in}}%
\pgfpathlineto{\pgfqpoint{1.149112in}{1.574065in}}%
\pgfpathlineto{\pgfqpoint{1.151938in}{1.573627in}}%
\pgfpathlineto{\pgfqpoint{1.154792in}{1.573231in}}%
\pgfpathlineto{\pgfqpoint{1.157672in}{1.572878in}}%
\pgfpathclose%
\pgfusepath{fill}%
\end{pgfscope}%
\begin{pgfscope}%
\pgfpathrectangle{\pgfqpoint{0.041670in}{0.041670in}}{\pgfqpoint{2.216660in}{2.216660in}}%
\pgfusepath{clip}%
\pgfsetbuttcap%
\pgfsetroundjoin%
\definecolor{currentfill}{rgb}{0.814576,0.883393,0.110347}%
\pgfsetfillcolor{currentfill}%
\pgfsetlinewidth{0.000000pt}%
\definecolor{currentstroke}{rgb}{0.000000,0.000000,0.000000}%
\pgfsetstrokecolor{currentstroke}%
\pgfsetdash{}{0pt}%
\pgfpathmoveto{\pgfqpoint{1.140694in}{1.554094in}}%
\pgfpathlineto{\pgfqpoint{1.139288in}{1.548675in}}%
\pgfpathlineto{\pgfqpoint{1.137883in}{1.543135in}}%
\pgfpathlineto{\pgfqpoint{1.136477in}{1.537475in}}%
\pgfpathlineto{\pgfqpoint{1.135071in}{1.531697in}}%
\pgfpathlineto{\pgfqpoint{1.131389in}{1.532397in}}%
\pgfpathlineto{\pgfqpoint{1.127756in}{1.533152in}}%
\pgfpathlineto{\pgfqpoint{1.124177in}{1.533961in}}%
\pgfpathlineto{\pgfqpoint{1.120655in}{1.534823in}}%
\pgfpathlineto{\pgfqpoint{1.122511in}{1.540501in}}%
\pgfpathlineto{\pgfqpoint{1.124368in}{1.546060in}}%
\pgfpathlineto{\pgfqpoint{1.126225in}{1.551500in}}%
\pgfpathlineto{\pgfqpoint{1.128081in}{1.556819in}}%
\pgfpathlineto{\pgfqpoint{1.131163in}{1.556068in}}%
\pgfpathlineto{\pgfqpoint{1.134295in}{1.555363in}}%
\pgfpathlineto{\pgfqpoint{1.137473in}{1.554705in}}%
\pgfpathlineto{\pgfqpoint{1.140694in}{1.554094in}}%
\pgfpathclose%
\pgfusepath{fill}%
\end{pgfscope}%
\begin{pgfscope}%
\pgfpathrectangle{\pgfqpoint{0.041670in}{0.041670in}}{\pgfqpoint{2.216660in}{2.216660in}}%
\pgfusepath{clip}%
\pgfsetbuttcap%
\pgfsetroundjoin%
\definecolor{currentfill}{rgb}{0.120081,0.622161,0.534946}%
\pgfsetfillcolor{currentfill}%
\pgfsetlinewidth{0.000000pt}%
\definecolor{currentstroke}{rgb}{0.000000,0.000000,0.000000}%
\pgfsetstrokecolor{currentstroke}%
\pgfsetdash{}{0pt}%
\pgfpathmoveto{\pgfqpoint{1.146485in}{1.173760in}}%
\pgfpathlineto{\pgfqpoint{1.146047in}{1.164191in}}%
\pgfpathlineto{\pgfqpoint{1.145610in}{1.154588in}}%
\pgfpathlineto{\pgfqpoint{1.145173in}{1.144956in}}%
\pgfpathlineto{\pgfqpoint{1.144736in}{1.135295in}}%
\pgfpathlineto{\pgfqpoint{1.134895in}{1.135933in}}%
\pgfpathlineto{\pgfqpoint{1.125103in}{1.136726in}}%
\pgfpathlineto{\pgfqpoint{1.115369in}{1.137675in}}%
\pgfpathlineto{\pgfqpoint{1.105703in}{1.138778in}}%
\pgfpathlineto{\pgfqpoint{1.106624in}{1.148390in}}%
\pgfpathlineto{\pgfqpoint{1.107546in}{1.157975in}}%
\pgfpathlineto{\pgfqpoint{1.108467in}{1.167530in}}%
\pgfpathlineto{\pgfqpoint{1.109389in}{1.177052in}}%
\pgfpathlineto{\pgfqpoint{1.118576in}{1.176010in}}%
\pgfpathlineto{\pgfqpoint{1.127827in}{1.175113in}}%
\pgfpathlineto{\pgfqpoint{1.137133in}{1.174363in}}%
\pgfpathlineto{\pgfqpoint{1.146485in}{1.173760in}}%
\pgfpathclose%
\pgfusepath{fill}%
\end{pgfscope}%
\begin{pgfscope}%
\pgfpathrectangle{\pgfqpoint{0.041670in}{0.041670in}}{\pgfqpoint{2.216660in}{2.216660in}}%
\pgfusepath{clip}%
\pgfsetbuttcap%
\pgfsetroundjoin%
\definecolor{currentfill}{rgb}{0.166383,0.690856,0.496502}%
\pgfsetfillcolor{currentfill}%
\pgfsetlinewidth{0.000000pt}%
\definecolor{currentstroke}{rgb}{0.000000,0.000000,0.000000}%
\pgfsetstrokecolor{currentstroke}%
\pgfsetdash{}{0pt}%
\pgfpathmoveto{\pgfqpoint{1.116776in}{1.251740in}}%
\pgfpathlineto{\pgfqpoint{1.115852in}{1.242572in}}%
\pgfpathlineto{\pgfqpoint{1.114928in}{1.233350in}}%
\pgfpathlineto{\pgfqpoint{1.114004in}{1.224078in}}%
\pgfpathlineto{\pgfqpoint{1.113080in}{1.214759in}}%
\pgfpathlineto{\pgfqpoint{1.104445in}{1.215877in}}%
\pgfpathlineto{\pgfqpoint{1.095889in}{1.217130in}}%
\pgfpathlineto{\pgfqpoint{1.087422in}{1.218517in}}%
\pgfpathlineto{\pgfqpoint{1.079053in}{1.220035in}}%
\pgfpathlineto{\pgfqpoint{1.080445in}{1.229275in}}%
\pgfpathlineto{\pgfqpoint{1.081838in}{1.238467in}}%
\pgfpathlineto{\pgfqpoint{1.083231in}{1.247609in}}%
\pgfpathlineto{\pgfqpoint{1.084625in}{1.256698in}}%
\pgfpathlineto{\pgfqpoint{1.092533in}{1.255272in}}%
\pgfpathlineto{\pgfqpoint{1.100533in}{1.253969in}}%
\pgfpathlineto{\pgfqpoint{1.108617in}{1.252791in}}%
\pgfpathlineto{\pgfqpoint{1.116776in}{1.251740in}}%
\pgfpathclose%
\pgfusepath{fill}%
\end{pgfscope}%
\begin{pgfscope}%
\pgfpathrectangle{\pgfqpoint{0.041670in}{0.041670in}}{\pgfqpoint{2.216660in}{2.216660in}}%
\pgfusepath{clip}%
\pgfsetbuttcap%
\pgfsetroundjoin%
\definecolor{currentfill}{rgb}{0.344074,0.780029,0.397381}%
\pgfsetfillcolor{currentfill}%
\pgfsetlinewidth{0.000000pt}%
\definecolor{currentstroke}{rgb}{0.000000,0.000000,0.000000}%
\pgfsetstrokecolor{currentstroke}%
\pgfsetdash{}{0pt}%
\pgfpathmoveto{\pgfqpoint{1.101387in}{1.360876in}}%
\pgfpathlineto{\pgfqpoint{1.099988in}{1.352587in}}%
\pgfpathlineto{\pgfqpoint{1.098589in}{1.344220in}}%
\pgfpathlineto{\pgfqpoint{1.097191in}{1.335777in}}%
\pgfpathlineto{\pgfqpoint{1.095793in}{1.327259in}}%
\pgfpathlineto{\pgfqpoint{1.088897in}{1.328613in}}%
\pgfpathlineto{\pgfqpoint{1.082096in}{1.330072in}}%
\pgfpathlineto{\pgfqpoint{1.075397in}{1.331635in}}%
\pgfpathlineto{\pgfqpoint{1.068807in}{1.333300in}}%
\pgfpathlineto{\pgfqpoint{1.070652in}{1.341710in}}%
\pgfpathlineto{\pgfqpoint{1.072497in}{1.350046in}}%
\pgfpathlineto{\pgfqpoint{1.074343in}{1.358306in}}%
\pgfpathlineto{\pgfqpoint{1.076190in}{1.366488in}}%
\pgfpathlineto{\pgfqpoint{1.082344in}{1.364941in}}%
\pgfpathlineto{\pgfqpoint{1.088599in}{1.363489in}}%
\pgfpathlineto{\pgfqpoint{1.094949in}{1.362133in}}%
\pgfpathlineto{\pgfqpoint{1.101387in}{1.360876in}}%
\pgfpathclose%
\pgfusepath{fill}%
\end{pgfscope}%
\begin{pgfscope}%
\pgfpathrectangle{\pgfqpoint{0.041670in}{0.041670in}}{\pgfqpoint{2.216660in}{2.216660in}}%
\pgfusepath{clip}%
\pgfsetbuttcap%
\pgfsetroundjoin%
\definecolor{currentfill}{rgb}{0.412913,0.803041,0.357269}%
\pgfsetfillcolor{currentfill}%
\pgfsetlinewidth{0.000000pt}%
\definecolor{currentstroke}{rgb}{0.000000,0.000000,0.000000}%
\pgfsetstrokecolor{currentstroke}%
\pgfsetdash{}{0pt}%
\pgfpathmoveto{\pgfqpoint{1.281325in}{1.399740in}}%
\pgfpathlineto{\pgfqpoint{1.283270in}{1.391917in}}%
\pgfpathlineto{\pgfqpoint{1.285214in}{1.384008in}}%
\pgfpathlineto{\pgfqpoint{1.287157in}{1.376016in}}%
\pgfpathlineto{\pgfqpoint{1.289099in}{1.367941in}}%
\pgfpathlineto{\pgfqpoint{1.283041in}{1.366311in}}%
\pgfpathlineto{\pgfqpoint{1.276876in}{1.364775in}}%
\pgfpathlineto{\pgfqpoint{1.270610in}{1.363333in}}%
\pgfpathlineto{\pgfqpoint{1.264250in}{1.361989in}}%
\pgfpathlineto{\pgfqpoint{1.262748in}{1.370176in}}%
\pgfpathlineto{\pgfqpoint{1.261246in}{1.378281in}}%
\pgfpathlineto{\pgfqpoint{1.259744in}{1.386302in}}%
\pgfpathlineto{\pgfqpoint{1.258241in}{1.394238in}}%
\pgfpathlineto{\pgfqpoint{1.264149in}{1.395481in}}%
\pgfpathlineto{\pgfqpoint{1.269969in}{1.396813in}}%
\pgfpathlineto{\pgfqpoint{1.275697in}{1.398233in}}%
\pgfpathlineto{\pgfqpoint{1.281325in}{1.399740in}}%
\pgfpathclose%
\pgfusepath{fill}%
\end{pgfscope}%
\begin{pgfscope}%
\pgfpathrectangle{\pgfqpoint{0.041670in}{0.041670in}}{\pgfqpoint{2.216660in}{2.216660in}}%
\pgfusepath{clip}%
\pgfsetbuttcap%
\pgfsetroundjoin%
\definecolor{currentfill}{rgb}{0.762373,0.876424,0.137064}%
\pgfsetfillcolor{currentfill}%
\pgfsetlinewidth{0.000000pt}%
\definecolor{currentstroke}{rgb}{0.000000,0.000000,0.000000}%
\pgfsetstrokecolor{currentstroke}%
\pgfsetdash{}{0pt}%
\pgfpathmoveto{\pgfqpoint{1.242335in}{1.535633in}}%
\pgfpathlineto{\pgfqpoint{1.244288in}{1.529864in}}%
\pgfpathlineto{\pgfqpoint{1.246240in}{1.523979in}}%
\pgfpathlineto{\pgfqpoint{1.248192in}{1.517980in}}%
\pgfpathlineto{\pgfqpoint{1.250144in}{1.511866in}}%
\pgfpathlineto{\pgfqpoint{1.246243in}{1.510840in}}%
\pgfpathlineto{\pgfqpoint{1.242273in}{1.509874in}}%
\pgfpathlineto{\pgfqpoint{1.238240in}{1.508967in}}%
\pgfpathlineto{\pgfqpoint{1.234147in}{1.508121in}}%
\pgfpathlineto{\pgfqpoint{1.232639in}{1.514342in}}%
\pgfpathlineto{\pgfqpoint{1.231131in}{1.520449in}}%
\pgfpathlineto{\pgfqpoint{1.229622in}{1.526441in}}%
\pgfpathlineto{\pgfqpoint{1.228114in}{1.532317in}}%
\pgfpathlineto{\pgfqpoint{1.231752in}{1.533066in}}%
\pgfpathlineto{\pgfqpoint{1.235338in}{1.533869in}}%
\pgfpathlineto{\pgfqpoint{1.238866in}{1.534725in}}%
\pgfpathlineto{\pgfqpoint{1.242335in}{1.535633in}}%
\pgfpathclose%
\pgfusepath{fill}%
\end{pgfscope}%
\begin{pgfscope}%
\pgfpathrectangle{\pgfqpoint{0.041670in}{0.041670in}}{\pgfqpoint{2.216660in}{2.216660in}}%
\pgfusepath{clip}%
\pgfsetbuttcap%
\pgfsetroundjoin%
\definecolor{currentfill}{rgb}{0.855810,0.888601,0.097452}%
\pgfsetfillcolor{currentfill}%
\pgfsetlinewidth{0.000000pt}%
\definecolor{currentstroke}{rgb}{0.000000,0.000000,0.000000}%
\pgfsetstrokecolor{currentstroke}%
\pgfsetdash{}{0pt}%
\pgfpathmoveto{\pgfqpoint{1.204799in}{1.573190in}}%
\pgfpathlineto{\pgfqpoint{1.205837in}{1.568205in}}%
\pgfpathlineto{\pgfqpoint{1.206875in}{1.563096in}}%
\pgfpathlineto{\pgfqpoint{1.207914in}{1.557863in}}%
\pgfpathlineto{\pgfqpoint{1.208952in}{1.552507in}}%
\pgfpathlineto{\pgfqpoint{1.205588in}{1.552099in}}%
\pgfpathlineto{\pgfqpoint{1.202197in}{1.551742in}}%
\pgfpathlineto{\pgfqpoint{1.198784in}{1.551435in}}%
\pgfpathlineto{\pgfqpoint{1.195352in}{1.551179in}}%
\pgfpathlineto{\pgfqpoint{1.194800in}{1.556584in}}%
\pgfpathlineto{\pgfqpoint{1.194249in}{1.561865in}}%
\pgfpathlineto{\pgfqpoint{1.193697in}{1.567023in}}%
\pgfpathlineto{\pgfqpoint{1.193146in}{1.572055in}}%
\pgfpathlineto{\pgfqpoint{1.196086in}{1.572274in}}%
\pgfpathlineto{\pgfqpoint{1.199011in}{1.572536in}}%
\pgfpathlineto{\pgfqpoint{1.201916in}{1.572841in}}%
\pgfpathlineto{\pgfqpoint{1.204799in}{1.573190in}}%
\pgfpathclose%
\pgfusepath{fill}%
\end{pgfscope}%
\begin{pgfscope}%
\pgfpathrectangle{\pgfqpoint{0.041670in}{0.041670in}}{\pgfqpoint{2.216660in}{2.216660in}}%
\pgfusepath{clip}%
\pgfsetbuttcap%
\pgfsetroundjoin%
\definecolor{currentfill}{rgb}{0.855810,0.888601,0.097452}%
\pgfsetfillcolor{currentfill}%
\pgfsetlinewidth{0.000000pt}%
\definecolor{currentstroke}{rgb}{0.000000,0.000000,0.000000}%
\pgfsetstrokecolor{currentstroke}%
\pgfsetdash{}{0pt}%
\pgfpathmoveto{\pgfqpoint{1.169389in}{1.571898in}}%
\pgfpathlineto{\pgfqpoint{1.168948in}{1.566859in}}%
\pgfpathlineto{\pgfqpoint{1.168506in}{1.561694in}}%
\pgfpathlineto{\pgfqpoint{1.168064in}{1.556406in}}%
\pgfpathlineto{\pgfqpoint{1.167623in}{1.550995in}}%
\pgfpathlineto{\pgfqpoint{1.164176in}{1.551205in}}%
\pgfpathlineto{\pgfqpoint{1.160745in}{1.551466in}}%
\pgfpathlineto{\pgfqpoint{1.157334in}{1.551779in}}%
\pgfpathlineto{\pgfqpoint{1.153947in}{1.552142in}}%
\pgfpathlineto{\pgfqpoint{1.154878in}{1.557511in}}%
\pgfpathlineto{\pgfqpoint{1.155810in}{1.562757in}}%
\pgfpathlineto{\pgfqpoint{1.156741in}{1.567880in}}%
\pgfpathlineto{\pgfqpoint{1.157672in}{1.572878in}}%
\pgfpathlineto{\pgfqpoint{1.160575in}{1.572568in}}%
\pgfpathlineto{\pgfqpoint{1.163497in}{1.572301in}}%
\pgfpathlineto{\pgfqpoint{1.166437in}{1.572078in}}%
\pgfpathlineto{\pgfqpoint{1.169389in}{1.571898in}}%
\pgfpathclose%
\pgfusepath{fill}%
\end{pgfscope}%
\begin{pgfscope}%
\pgfpathrectangle{\pgfqpoint{0.041670in}{0.041670in}}{\pgfqpoint{2.216660in}{2.216660in}}%
\pgfusepath{clip}%
\pgfsetbuttcap%
\pgfsetroundjoin%
\definecolor{currentfill}{rgb}{0.220124,0.725509,0.466226}%
\pgfsetfillcolor{currentfill}%
\pgfsetlinewidth{0.000000pt}%
\definecolor{currentstroke}{rgb}{0.000000,0.000000,0.000000}%
\pgfsetstrokecolor{currentstroke}%
\pgfsetdash{}{0pt}%
\pgfpathmoveto{\pgfqpoint{1.276245in}{1.293772in}}%
\pgfpathlineto{\pgfqpoint{1.277742in}{1.284936in}}%
\pgfpathlineto{\pgfqpoint{1.279239in}{1.276039in}}%
\pgfpathlineto{\pgfqpoint{1.280735in}{1.267083in}}%
\pgfpathlineto{\pgfqpoint{1.282231in}{1.258069in}}%
\pgfpathlineto{\pgfqpoint{1.274411in}{1.256534in}}%
\pgfpathlineto{\pgfqpoint{1.266492in}{1.255121in}}%
\pgfpathlineto{\pgfqpoint{1.258482in}{1.253832in}}%
\pgfpathlineto{\pgfqpoint{1.250389in}{1.252668in}}%
\pgfpathlineto{\pgfqpoint{1.249359in}{1.261767in}}%
\pgfpathlineto{\pgfqpoint{1.248328in}{1.270810in}}%
\pgfpathlineto{\pgfqpoint{1.247296in}{1.279792in}}%
\pgfpathlineto{\pgfqpoint{1.246264in}{1.288714in}}%
\pgfpathlineto{\pgfqpoint{1.253884in}{1.289803in}}%
\pgfpathlineto{\pgfqpoint{1.261426in}{1.291010in}}%
\pgfpathlineto{\pgfqpoint{1.268882in}{1.292334in}}%
\pgfpathlineto{\pgfqpoint{1.276245in}{1.293772in}}%
\pgfpathclose%
\pgfusepath{fill}%
\end{pgfscope}%
\begin{pgfscope}%
\pgfpathrectangle{\pgfqpoint{0.041670in}{0.041670in}}{\pgfqpoint{2.216660in}{2.216660in}}%
\pgfusepath{clip}%
\pgfsetbuttcap%
\pgfsetroundjoin%
\definecolor{currentfill}{rgb}{0.201239,0.383670,0.554294}%
\pgfsetfillcolor{currentfill}%
\pgfsetlinewidth{0.000000pt}%
\definecolor{currentstroke}{rgb}{0.000000,0.000000,0.000000}%
\pgfsetstrokecolor{currentstroke}%
\pgfsetdash{}{0pt}%
\pgfpathmoveto{\pgfqpoint{1.821811in}{0.886633in}}%
\pgfpathlineto{\pgfqpoint{1.825666in}{0.898814in}}%
\pgfpathlineto{\pgfqpoint{1.829542in}{0.911483in}}%
\pgfpathlineto{\pgfqpoint{1.833440in}{0.924648in}}%
\pgfpathlineto{\pgfqpoint{1.837360in}{0.938318in}}%
\pgfpathlineto{\pgfqpoint{1.825370in}{0.927666in}}%
\pgfpathlineto{\pgfqpoint{1.812680in}{0.917200in}}%
\pgfpathlineto{\pgfqpoint{1.799302in}{0.906934in}}%
\pgfpathlineto{\pgfqpoint{1.785248in}{0.896878in}}%
\pgfpathlineto{\pgfqpoint{1.781618in}{0.883378in}}%
\pgfpathlineto{\pgfqpoint{1.778008in}{0.870385in}}%
\pgfpathlineto{\pgfqpoint{1.774419in}{0.857891in}}%
\pgfpathlineto{\pgfqpoint{1.770850in}{0.845887in}}%
\pgfpathlineto{\pgfqpoint{1.784590in}{0.855774in}}%
\pgfpathlineto{\pgfqpoint{1.797671in}{0.865868in}}%
\pgfpathlineto{\pgfqpoint{1.810082in}{0.876159in}}%
\pgfpathlineto{\pgfqpoint{1.821811in}{0.886633in}}%
\pgfpathclose%
\pgfusepath{fill}%
\end{pgfscope}%
\begin{pgfscope}%
\pgfpathrectangle{\pgfqpoint{0.041670in}{0.041670in}}{\pgfqpoint{2.216660in}{2.216660in}}%
\pgfusepath{clip}%
\pgfsetbuttcap%
\pgfsetroundjoin%
\definecolor{currentfill}{rgb}{0.134692,0.658636,0.517649}%
\pgfsetfillcolor{currentfill}%
\pgfsetlinewidth{0.000000pt}%
\definecolor{currentstroke}{rgb}{0.000000,0.000000,0.000000}%
\pgfsetstrokecolor{currentstroke}%
\pgfsetdash{}{0pt}%
\pgfpathmoveto{\pgfqpoint{1.254509in}{1.215746in}}%
\pgfpathlineto{\pgfqpoint{1.255538in}{1.206396in}}%
\pgfpathlineto{\pgfqpoint{1.256567in}{1.197004in}}%
\pgfpathlineto{\pgfqpoint{1.257595in}{1.187570in}}%
\pgfpathlineto{\pgfqpoint{1.258623in}{1.178099in}}%
\pgfpathlineto{\pgfqpoint{1.249503in}{1.176929in}}%
\pgfpathlineto{\pgfqpoint{1.240309in}{1.175903in}}%
\pgfpathlineto{\pgfqpoint{1.231052in}{1.175022in}}%
\pgfpathlineto{\pgfqpoint{1.221740in}{1.174289in}}%
\pgfpathlineto{\pgfqpoint{1.221193in}{1.183815in}}%
\pgfpathlineto{\pgfqpoint{1.220647in}{1.193303in}}%
\pgfpathlineto{\pgfqpoint{1.220100in}{1.202750in}}%
\pgfpathlineto{\pgfqpoint{1.219553in}{1.212155in}}%
\pgfpathlineto{\pgfqpoint{1.228378in}{1.212846in}}%
\pgfpathlineto{\pgfqpoint{1.237152in}{1.213676in}}%
\pgfpathlineto{\pgfqpoint{1.245865in}{1.214643in}}%
\pgfpathlineto{\pgfqpoint{1.254509in}{1.215746in}}%
\pgfpathclose%
\pgfusepath{fill}%
\end{pgfscope}%
\begin{pgfscope}%
\pgfpathrectangle{\pgfqpoint{0.041670in}{0.041670in}}{\pgfqpoint{2.216660in}{2.216660in}}%
\pgfusepath{clip}%
\pgfsetbuttcap%
\pgfsetroundjoin%
\definecolor{currentfill}{rgb}{0.762373,0.876424,0.137064}%
\pgfsetfillcolor{currentfill}%
\pgfsetlinewidth{0.000000pt}%
\definecolor{currentstroke}{rgb}{0.000000,0.000000,0.000000}%
\pgfsetstrokecolor{currentstroke}%
\pgfsetdash{}{0pt}%
\pgfpathmoveto{\pgfqpoint{1.135071in}{1.531697in}}%
\pgfpathlineto{\pgfqpoint{1.133665in}{1.525802in}}%
\pgfpathlineto{\pgfqpoint{1.132259in}{1.519790in}}%
\pgfpathlineto{\pgfqpoint{1.130854in}{1.513662in}}%
\pgfpathlineto{\pgfqpoint{1.129448in}{1.507421in}}%
\pgfpathlineto{\pgfqpoint{1.125305in}{1.508212in}}%
\pgfpathlineto{\pgfqpoint{1.121219in}{1.509065in}}%
\pgfpathlineto{\pgfqpoint{1.117192in}{1.509978in}}%
\pgfpathlineto{\pgfqpoint{1.113230in}{1.510952in}}%
\pgfpathlineto{\pgfqpoint{1.115086in}{1.517091in}}%
\pgfpathlineto{\pgfqpoint{1.116942in}{1.523117in}}%
\pgfpathlineto{\pgfqpoint{1.118798in}{1.529028in}}%
\pgfpathlineto{\pgfqpoint{1.120655in}{1.534823in}}%
\pgfpathlineto{\pgfqpoint{1.124177in}{1.533961in}}%
\pgfpathlineto{\pgfqpoint{1.127756in}{1.533152in}}%
\pgfpathlineto{\pgfqpoint{1.131389in}{1.532397in}}%
\pgfpathlineto{\pgfqpoint{1.135071in}{1.531697in}}%
\pgfpathclose%
\pgfusepath{fill}%
\end{pgfscope}%
\begin{pgfscope}%
\pgfpathrectangle{\pgfqpoint{0.041670in}{0.041670in}}{\pgfqpoint{2.216660in}{2.216660in}}%
\pgfusepath{clip}%
\pgfsetbuttcap%
\pgfsetroundjoin%
\definecolor{currentfill}{rgb}{0.487026,0.823929,0.312321}%
\pgfsetfillcolor{currentfill}%
\pgfsetlinewidth{0.000000pt}%
\definecolor{currentstroke}{rgb}{0.000000,0.000000,0.000000}%
\pgfsetstrokecolor{currentstroke}%
\pgfsetdash{}{0pt}%
\pgfpathmoveto{\pgfqpoint{1.273541in}{1.430140in}}%
\pgfpathlineto{\pgfqpoint{1.275488in}{1.422677in}}%
\pgfpathlineto{\pgfqpoint{1.277434in}{1.415122in}}%
\pgfpathlineto{\pgfqpoint{1.279380in}{1.407475in}}%
\pgfpathlineto{\pgfqpoint{1.281325in}{1.399740in}}%
\pgfpathlineto{\pgfqpoint{1.275697in}{1.398233in}}%
\pgfpathlineto{\pgfqpoint{1.269969in}{1.396813in}}%
\pgfpathlineto{\pgfqpoint{1.264149in}{1.395481in}}%
\pgfpathlineto{\pgfqpoint{1.258241in}{1.394238in}}%
\pgfpathlineto{\pgfqpoint{1.256737in}{1.402085in}}%
\pgfpathlineto{\pgfqpoint{1.255233in}{1.409843in}}%
\pgfpathlineto{\pgfqpoint{1.253729in}{1.417510in}}%
\pgfpathlineto{\pgfqpoint{1.252225in}{1.425084in}}%
\pgfpathlineto{\pgfqpoint{1.257680in}{1.426226in}}%
\pgfpathlineto{\pgfqpoint{1.263055in}{1.427450in}}%
\pgfpathlineto{\pgfqpoint{1.268344in}{1.428755in}}%
\pgfpathlineto{\pgfqpoint{1.273541in}{1.430140in}}%
\pgfpathclose%
\pgfusepath{fill}%
\end{pgfscope}%
\begin{pgfscope}%
\pgfpathrectangle{\pgfqpoint{0.041670in}{0.041670in}}{\pgfqpoint{2.216660in}{2.216660in}}%
\pgfusepath{clip}%
\pgfsetbuttcap%
\pgfsetroundjoin%
\definecolor{currentfill}{rgb}{0.120081,0.622161,0.534946}%
\pgfsetfillcolor{currentfill}%
\pgfsetlinewidth{0.000000pt}%
\definecolor{currentstroke}{rgb}{0.000000,0.000000,0.000000}%
\pgfsetstrokecolor{currentstroke}%
\pgfsetdash{}{0pt}%
\pgfpathmoveto{\pgfqpoint{1.221740in}{1.174289in}}%
\pgfpathlineto{\pgfqpoint{1.222286in}{1.164727in}}%
\pgfpathlineto{\pgfqpoint{1.222832in}{1.155132in}}%
\pgfpathlineto{\pgfqpoint{1.223378in}{1.145507in}}%
\pgfpathlineto{\pgfqpoint{1.223923in}{1.135854in}}%
\pgfpathlineto{\pgfqpoint{1.214078in}{1.135234in}}%
\pgfpathlineto{\pgfqpoint{1.204197in}{1.134770in}}%
\pgfpathlineto{\pgfqpoint{1.194289in}{1.134463in}}%
\pgfpathlineto{\pgfqpoint{1.184367in}{1.134314in}}%
\pgfpathlineto{\pgfqpoint{1.184312in}{1.143989in}}%
\pgfpathlineto{\pgfqpoint{1.184257in}{1.153635in}}%
\pgfpathlineto{\pgfqpoint{1.184203in}{1.163251in}}%
\pgfpathlineto{\pgfqpoint{1.184148in}{1.172833in}}%
\pgfpathlineto{\pgfqpoint{1.193577in}{1.172974in}}%
\pgfpathlineto{\pgfqpoint{1.202993in}{1.173264in}}%
\pgfpathlineto{\pgfqpoint{1.212383in}{1.173702in}}%
\pgfpathlineto{\pgfqpoint{1.221740in}{1.174289in}}%
\pgfpathclose%
\pgfusepath{fill}%
\end{pgfscope}%
\begin{pgfscope}%
\pgfpathrectangle{\pgfqpoint{0.041670in}{0.041670in}}{\pgfqpoint{2.216660in}{2.216660in}}%
\pgfusepath{clip}%
\pgfsetbuttcap%
\pgfsetroundjoin%
\definecolor{currentfill}{rgb}{0.855810,0.888601,0.097452}%
\pgfsetfillcolor{currentfill}%
\pgfsetlinewidth{0.000000pt}%
\definecolor{currentstroke}{rgb}{0.000000,0.000000,0.000000}%
\pgfsetstrokecolor{currentstroke}%
\pgfsetdash{}{0pt}%
\pgfpathmoveto{\pgfqpoint{1.193146in}{1.572055in}}%
\pgfpathlineto{\pgfqpoint{1.193697in}{1.567023in}}%
\pgfpathlineto{\pgfqpoint{1.194249in}{1.561865in}}%
\pgfpathlineto{\pgfqpoint{1.194800in}{1.556584in}}%
\pgfpathlineto{\pgfqpoint{1.195352in}{1.551179in}}%
\pgfpathlineto{\pgfqpoint{1.191903in}{1.550975in}}%
\pgfpathlineto{\pgfqpoint{1.188443in}{1.550822in}}%
\pgfpathlineto{\pgfqpoint{1.184974in}{1.550721in}}%
\pgfpathlineto{\pgfqpoint{1.181500in}{1.550672in}}%
\pgfpathlineto{\pgfqpoint{1.181444in}{1.556095in}}%
\pgfpathlineto{\pgfqpoint{1.181389in}{1.561395in}}%
\pgfpathlineto{\pgfqpoint{1.181334in}{1.566571in}}%
\pgfpathlineto{\pgfqpoint{1.181278in}{1.571623in}}%
\pgfpathlineto{\pgfqpoint{1.184255in}{1.571664in}}%
\pgfpathlineto{\pgfqpoint{1.187227in}{1.571751in}}%
\pgfpathlineto{\pgfqpoint{1.190191in}{1.571881in}}%
\pgfpathlineto{\pgfqpoint{1.193146in}{1.572055in}}%
\pgfpathclose%
\pgfusepath{fill}%
\end{pgfscope}%
\begin{pgfscope}%
\pgfpathrectangle{\pgfqpoint{0.041670in}{0.041670in}}{\pgfqpoint{2.216660in}{2.216660in}}%
\pgfusepath{clip}%
\pgfsetbuttcap%
\pgfsetroundjoin%
\definecolor{currentfill}{rgb}{0.699415,0.867117,0.175971}%
\pgfsetfillcolor{currentfill}%
\pgfsetlinewidth{0.000000pt}%
\definecolor{currentstroke}{rgb}{0.000000,0.000000,0.000000}%
\pgfsetstrokecolor{currentstroke}%
\pgfsetdash{}{0pt}%
\pgfpathmoveto{\pgfqpoint{1.250144in}{1.511866in}}%
\pgfpathlineto{\pgfqpoint{1.252096in}{1.505641in}}%
\pgfpathlineto{\pgfqpoint{1.254047in}{1.499304in}}%
\pgfpathlineto{\pgfqpoint{1.255999in}{1.492857in}}%
\pgfpathlineto{\pgfqpoint{1.257949in}{1.486303in}}%
\pgfpathlineto{\pgfqpoint{1.253615in}{1.485158in}}%
\pgfpathlineto{\pgfqpoint{1.249205in}{1.484080in}}%
\pgfpathlineto{\pgfqpoint{1.244724in}{1.483068in}}%
\pgfpathlineto{\pgfqpoint{1.240177in}{1.482125in}}%
\pgfpathlineto{\pgfqpoint{1.238669in}{1.488788in}}%
\pgfpathlineto{\pgfqpoint{1.237162in}{1.495343in}}%
\pgfpathlineto{\pgfqpoint{1.235654in}{1.501788in}}%
\pgfpathlineto{\pgfqpoint{1.234147in}{1.508121in}}%
\pgfpathlineto{\pgfqpoint{1.238240in}{1.508967in}}%
\pgfpathlineto{\pgfqpoint{1.242273in}{1.509874in}}%
\pgfpathlineto{\pgfqpoint{1.246243in}{1.510840in}}%
\pgfpathlineto{\pgfqpoint{1.250144in}{1.511866in}}%
\pgfpathclose%
\pgfusepath{fill}%
\end{pgfscope}%
\begin{pgfscope}%
\pgfpathrectangle{\pgfqpoint{0.041670in}{0.041670in}}{\pgfqpoint{2.216660in}{2.216660in}}%
\pgfusepath{clip}%
\pgfsetbuttcap%
\pgfsetroundjoin%
\definecolor{currentfill}{rgb}{0.814576,0.883393,0.110347}%
\pgfsetfillcolor{currentfill}%
\pgfsetlinewidth{0.000000pt}%
\definecolor{currentstroke}{rgb}{0.000000,0.000000,0.000000}%
\pgfsetstrokecolor{currentstroke}%
\pgfsetdash{}{0pt}%
\pgfpathmoveto{\pgfqpoint{1.222081in}{1.554635in}}%
\pgfpathlineto{\pgfqpoint{1.223589in}{1.549235in}}%
\pgfpathlineto{\pgfqpoint{1.225098in}{1.543715in}}%
\pgfpathlineto{\pgfqpoint{1.226606in}{1.538075in}}%
\pgfpathlineto{\pgfqpoint{1.228114in}{1.532317in}}%
\pgfpathlineto{\pgfqpoint{1.224427in}{1.531623in}}%
\pgfpathlineto{\pgfqpoint{1.220694in}{1.530984in}}%
\pgfpathlineto{\pgfqpoint{1.216919in}{1.530402in}}%
\pgfpathlineto{\pgfqpoint{1.213107in}{1.529877in}}%
\pgfpathlineto{\pgfqpoint{1.212068in}{1.535713in}}%
\pgfpathlineto{\pgfqpoint{1.211030in}{1.541432in}}%
\pgfpathlineto{\pgfqpoint{1.209991in}{1.547030in}}%
\pgfpathlineto{\pgfqpoint{1.208952in}{1.552507in}}%
\pgfpathlineto{\pgfqpoint{1.212287in}{1.552965in}}%
\pgfpathlineto{\pgfqpoint{1.215589in}{1.553473in}}%
\pgfpathlineto{\pgfqpoint{1.218855in}{1.554030in}}%
\pgfpathlineto{\pgfqpoint{1.222081in}{1.554635in}}%
\pgfpathclose%
\pgfusepath{fill}%
\end{pgfscope}%
\begin{pgfscope}%
\pgfpathrectangle{\pgfqpoint{0.041670in}{0.041670in}}{\pgfqpoint{2.216660in}{2.216660in}}%
\pgfusepath{clip}%
\pgfsetbuttcap%
\pgfsetroundjoin%
\definecolor{currentfill}{rgb}{0.855810,0.888601,0.097452}%
\pgfsetfillcolor{currentfill}%
\pgfsetlinewidth{0.000000pt}%
\definecolor{currentstroke}{rgb}{0.000000,0.000000,0.000000}%
\pgfsetstrokecolor{currentstroke}%
\pgfsetdash{}{0pt}%
\pgfpathmoveto{\pgfqpoint{1.181278in}{1.571623in}}%
\pgfpathlineto{\pgfqpoint{1.181334in}{1.566571in}}%
\pgfpathlineto{\pgfqpoint{1.181389in}{1.561395in}}%
\pgfpathlineto{\pgfqpoint{1.181444in}{1.556095in}}%
\pgfpathlineto{\pgfqpoint{1.181500in}{1.550672in}}%
\pgfpathlineto{\pgfqpoint{1.178024in}{1.550675in}}%
\pgfpathlineto{\pgfqpoint{1.174550in}{1.550730in}}%
\pgfpathlineto{\pgfqpoint{1.171082in}{1.550837in}}%
\pgfpathlineto{\pgfqpoint{1.167623in}{1.550995in}}%
\pgfpathlineto{\pgfqpoint{1.168064in}{1.556406in}}%
\pgfpathlineto{\pgfqpoint{1.168506in}{1.561694in}}%
\pgfpathlineto{\pgfqpoint{1.168948in}{1.566859in}}%
\pgfpathlineto{\pgfqpoint{1.169389in}{1.571898in}}%
\pgfpathlineto{\pgfqpoint{1.172353in}{1.571763in}}%
\pgfpathlineto{\pgfqpoint{1.175324in}{1.571672in}}%
\pgfpathlineto{\pgfqpoint{1.178301in}{1.571625in}}%
\pgfpathlineto{\pgfqpoint{1.181278in}{1.571623in}}%
\pgfpathclose%
\pgfusepath{fill}%
\end{pgfscope}%
\begin{pgfscope}%
\pgfpathrectangle{\pgfqpoint{0.041670in}{0.041670in}}{\pgfqpoint{2.216660in}{2.216660in}}%
\pgfusepath{clip}%
\pgfsetbuttcap%
\pgfsetroundjoin%
\definecolor{currentfill}{rgb}{0.412913,0.803041,0.357269}%
\pgfsetfillcolor{currentfill}%
\pgfsetlinewidth{0.000000pt}%
\definecolor{currentstroke}{rgb}{0.000000,0.000000,0.000000}%
\pgfsetstrokecolor{currentstroke}%
\pgfsetdash{}{0pt}%
\pgfpathmoveto{\pgfqpoint{1.106989in}{1.393209in}}%
\pgfpathlineto{\pgfqpoint{1.105588in}{1.385253in}}%
\pgfpathlineto{\pgfqpoint{1.104187in}{1.377211in}}%
\pgfpathlineto{\pgfqpoint{1.102787in}{1.369084in}}%
\pgfpathlineto{\pgfqpoint{1.101387in}{1.360876in}}%
\pgfpathlineto{\pgfqpoint{1.094949in}{1.362133in}}%
\pgfpathlineto{\pgfqpoint{1.088599in}{1.363489in}}%
\pgfpathlineto{\pgfqpoint{1.082344in}{1.364941in}}%
\pgfpathlineto{\pgfqpoint{1.076190in}{1.366488in}}%
\pgfpathlineto{\pgfqpoint{1.078037in}{1.374590in}}%
\pgfpathlineto{\pgfqpoint{1.079885in}{1.382610in}}%
\pgfpathlineto{\pgfqpoint{1.081733in}{1.390546in}}%
\pgfpathlineto{\pgfqpoint{1.083582in}{1.398396in}}%
\pgfpathlineto{\pgfqpoint{1.089299in}{1.396966in}}%
\pgfpathlineto{\pgfqpoint{1.095110in}{1.395624in}}%
\pgfpathlineto{\pgfqpoint{1.101009in}{1.394371in}}%
\pgfpathlineto{\pgfqpoint{1.106989in}{1.393209in}}%
\pgfpathclose%
\pgfusepath{fill}%
\end{pgfscope}%
\begin{pgfscope}%
\pgfpathrectangle{\pgfqpoint{0.041670in}{0.041670in}}{\pgfqpoint{2.216660in}{2.216660in}}%
\pgfusepath{clip}%
\pgfsetbuttcap%
\pgfsetroundjoin%
\definecolor{currentfill}{rgb}{0.120081,0.622161,0.534946}%
\pgfsetfillcolor{currentfill}%
\pgfsetlinewidth{0.000000pt}%
\definecolor{currentstroke}{rgb}{0.000000,0.000000,0.000000}%
\pgfsetstrokecolor{currentstroke}%
\pgfsetdash{}{0pt}%
\pgfpathmoveto{\pgfqpoint{1.184148in}{1.172833in}}%
\pgfpathlineto{\pgfqpoint{1.184203in}{1.163251in}}%
\pgfpathlineto{\pgfqpoint{1.184257in}{1.153635in}}%
\pgfpathlineto{\pgfqpoint{1.184312in}{1.143989in}}%
\pgfpathlineto{\pgfqpoint{1.184367in}{1.134314in}}%
\pgfpathlineto{\pgfqpoint{1.174440in}{1.134323in}}%
\pgfpathlineto{\pgfqpoint{1.164518in}{1.134490in}}%
\pgfpathlineto{\pgfqpoint{1.154614in}{1.134814in}}%
\pgfpathlineto{\pgfqpoint{1.144736in}{1.135295in}}%
\pgfpathlineto{\pgfqpoint{1.145173in}{1.144956in}}%
\pgfpathlineto{\pgfqpoint{1.145610in}{1.154588in}}%
\pgfpathlineto{\pgfqpoint{1.146047in}{1.164191in}}%
\pgfpathlineto{\pgfqpoint{1.146485in}{1.173760in}}%
\pgfpathlineto{\pgfqpoint{1.155872in}{1.173305in}}%
\pgfpathlineto{\pgfqpoint{1.165285in}{1.172999in}}%
\pgfpathlineto{\pgfqpoint{1.174714in}{1.172842in}}%
\pgfpathlineto{\pgfqpoint{1.184148in}{1.172833in}}%
\pgfpathclose%
\pgfusepath{fill}%
\end{pgfscope}%
\begin{pgfscope}%
\pgfpathrectangle{\pgfqpoint{0.041670in}{0.041670in}}{\pgfqpoint{2.216660in}{2.216660in}}%
\pgfusepath{clip}%
\pgfsetbuttcap%
\pgfsetroundjoin%
\definecolor{currentfill}{rgb}{0.134692,0.658636,0.517649}%
\pgfsetfillcolor{currentfill}%
\pgfsetlinewidth{0.000000pt}%
\definecolor{currentstroke}{rgb}{0.000000,0.000000,0.000000}%
\pgfsetstrokecolor{currentstroke}%
\pgfsetdash{}{0pt}%
\pgfpathmoveto{\pgfqpoint{1.148236in}{1.211657in}}%
\pgfpathlineto{\pgfqpoint{1.147798in}{1.202245in}}%
\pgfpathlineto{\pgfqpoint{1.147360in}{1.192789in}}%
\pgfpathlineto{\pgfqpoint{1.146923in}{1.183294in}}%
\pgfpathlineto{\pgfqpoint{1.146485in}{1.173760in}}%
\pgfpathlineto{\pgfqpoint{1.137133in}{1.174363in}}%
\pgfpathlineto{\pgfqpoint{1.127827in}{1.175113in}}%
\pgfpathlineto{\pgfqpoint{1.118576in}{1.176010in}}%
\pgfpathlineto{\pgfqpoint{1.109389in}{1.177052in}}%
\pgfpathlineto{\pgfqpoint{1.110312in}{1.186538in}}%
\pgfpathlineto{\pgfqpoint{1.111234in}{1.195986in}}%
\pgfpathlineto{\pgfqpoint{1.112157in}{1.205394in}}%
\pgfpathlineto{\pgfqpoint{1.113080in}{1.214759in}}%
\pgfpathlineto{\pgfqpoint{1.121786in}{1.213777in}}%
\pgfpathlineto{\pgfqpoint{1.130554in}{1.212932in}}%
\pgfpathlineto{\pgfqpoint{1.139374in}{1.212225in}}%
\pgfpathlineto{\pgfqpoint{1.148236in}{1.211657in}}%
\pgfpathclose%
\pgfusepath{fill}%
\end{pgfscope}%
\begin{pgfscope}%
\pgfpathrectangle{\pgfqpoint{0.041670in}{0.041670in}}{\pgfqpoint{2.216660in}{2.216660in}}%
\pgfusepath{clip}%
\pgfsetbuttcap%
\pgfsetroundjoin%
\definecolor{currentfill}{rgb}{0.565498,0.842430,0.262877}%
\pgfsetfillcolor{currentfill}%
\pgfsetlinewidth{0.000000pt}%
\definecolor{currentstroke}{rgb}{0.000000,0.000000,0.000000}%
\pgfsetstrokecolor{currentstroke}%
\pgfsetdash{}{0pt}%
\pgfpathmoveto{\pgfqpoint{1.265749in}{1.459029in}}%
\pgfpathlineto{\pgfqpoint{1.267698in}{1.451954in}}%
\pgfpathlineto{\pgfqpoint{1.269646in}{1.444780in}}%
\pgfpathlineto{\pgfqpoint{1.271594in}{1.437508in}}%
\pgfpathlineto{\pgfqpoint{1.273541in}{1.430140in}}%
\pgfpathlineto{\pgfqpoint{1.268344in}{1.428755in}}%
\pgfpathlineto{\pgfqpoint{1.263055in}{1.427450in}}%
\pgfpathlineto{\pgfqpoint{1.257680in}{1.426226in}}%
\pgfpathlineto{\pgfqpoint{1.252225in}{1.425084in}}%
\pgfpathlineto{\pgfqpoint{1.250720in}{1.432562in}}%
\pgfpathlineto{\pgfqpoint{1.249215in}{1.439945in}}%
\pgfpathlineto{\pgfqpoint{1.247709in}{1.447229in}}%
\pgfpathlineto{\pgfqpoint{1.246203in}{1.454414in}}%
\pgfpathlineto{\pgfqpoint{1.251205in}{1.455456in}}%
\pgfpathlineto{\pgfqpoint{1.256133in}{1.456574in}}%
\pgfpathlineto{\pgfqpoint{1.260983in}{1.457765in}}%
\pgfpathlineto{\pgfqpoint{1.265749in}{1.459029in}}%
\pgfpathclose%
\pgfusepath{fill}%
\end{pgfscope}%
\begin{pgfscope}%
\pgfpathrectangle{\pgfqpoint{0.041670in}{0.041670in}}{\pgfqpoint{2.216660in}{2.216660in}}%
\pgfusepath{clip}%
\pgfsetbuttcap%
\pgfsetroundjoin%
\definecolor{currentfill}{rgb}{0.636902,0.856542,0.216620}%
\pgfsetfillcolor{currentfill}%
\pgfsetlinewidth{0.000000pt}%
\definecolor{currentstroke}{rgb}{0.000000,0.000000,0.000000}%
\pgfsetstrokecolor{currentstroke}%
\pgfsetdash{}{0pt}%
\pgfpathmoveto{\pgfqpoint{1.257949in}{1.486303in}}%
\pgfpathlineto{\pgfqpoint{1.259900in}{1.479641in}}%
\pgfpathlineto{\pgfqpoint{1.261850in}{1.472874in}}%
\pgfpathlineto{\pgfqpoint{1.263800in}{1.466002in}}%
\pgfpathlineto{\pgfqpoint{1.265749in}{1.459029in}}%
\pgfpathlineto{\pgfqpoint{1.260983in}{1.457765in}}%
\pgfpathlineto{\pgfqpoint{1.256133in}{1.456574in}}%
\pgfpathlineto{\pgfqpoint{1.251205in}{1.455456in}}%
\pgfpathlineto{\pgfqpoint{1.246203in}{1.454414in}}%
\pgfpathlineto{\pgfqpoint{1.244697in}{1.461497in}}%
\pgfpathlineto{\pgfqpoint{1.243190in}{1.468478in}}%
\pgfpathlineto{\pgfqpoint{1.241684in}{1.475354in}}%
\pgfpathlineto{\pgfqpoint{1.240177in}{1.482125in}}%
\pgfpathlineto{\pgfqpoint{1.244724in}{1.483068in}}%
\pgfpathlineto{\pgfqpoint{1.249205in}{1.484080in}}%
\pgfpathlineto{\pgfqpoint{1.253615in}{1.485158in}}%
\pgfpathlineto{\pgfqpoint{1.257949in}{1.486303in}}%
\pgfpathclose%
\pgfusepath{fill}%
\end{pgfscope}%
\begin{pgfscope}%
\pgfpathrectangle{\pgfqpoint{0.041670in}{0.041670in}}{\pgfqpoint{2.216660in}{2.216660in}}%
\pgfusepath{clip}%
\pgfsetbuttcap%
\pgfsetroundjoin%
\definecolor{currentfill}{rgb}{0.814576,0.883393,0.110347}%
\pgfsetfillcolor{currentfill}%
\pgfsetlinewidth{0.000000pt}%
\definecolor{currentstroke}{rgb}{0.000000,0.000000,0.000000}%
\pgfsetstrokecolor{currentstroke}%
\pgfsetdash{}{0pt}%
\pgfpathmoveto{\pgfqpoint{1.153947in}{1.552142in}}%
\pgfpathlineto{\pgfqpoint{1.153015in}{1.546651in}}%
\pgfpathlineto{\pgfqpoint{1.152083in}{1.541040in}}%
\pgfpathlineto{\pgfqpoint{1.151152in}{1.535308in}}%
\pgfpathlineto{\pgfqpoint{1.150220in}{1.529458in}}%
\pgfpathlineto{\pgfqpoint{1.146377in}{1.529932in}}%
\pgfpathlineto{\pgfqpoint{1.142569in}{1.530464in}}%
\pgfpathlineto{\pgfqpoint{1.138799in}{1.531053in}}%
\pgfpathlineto{\pgfqpoint{1.135071in}{1.531697in}}%
\pgfpathlineto{\pgfqpoint{1.136477in}{1.537475in}}%
\pgfpathlineto{\pgfqpoint{1.137883in}{1.543135in}}%
\pgfpathlineto{\pgfqpoint{1.139288in}{1.548675in}}%
\pgfpathlineto{\pgfqpoint{1.140694in}{1.554094in}}%
\pgfpathlineto{\pgfqpoint{1.143956in}{1.553532in}}%
\pgfpathlineto{\pgfqpoint{1.147254in}{1.553019in}}%
\pgfpathlineto{\pgfqpoint{1.150585in}{1.552556in}}%
\pgfpathlineto{\pgfqpoint{1.153947in}{1.552142in}}%
\pgfpathclose%
\pgfusepath{fill}%
\end{pgfscope}%
\begin{pgfscope}%
\pgfpathrectangle{\pgfqpoint{0.041670in}{0.041670in}}{\pgfqpoint{2.216660in}{2.216660in}}%
\pgfusepath{clip}%
\pgfsetbuttcap%
\pgfsetroundjoin%
\definecolor{currentfill}{rgb}{0.220124,0.725509,0.466226}%
\pgfsetfillcolor{currentfill}%
\pgfsetlinewidth{0.000000pt}%
\definecolor{currentstroke}{rgb}{0.000000,0.000000,0.000000}%
\pgfsetstrokecolor{currentstroke}%
\pgfsetdash{}{0pt}%
\pgfpathmoveto{\pgfqpoint{1.120477in}{1.287845in}}%
\pgfpathlineto{\pgfqpoint{1.119551in}{1.278909in}}%
\pgfpathlineto{\pgfqpoint{1.118626in}{1.269911in}}%
\pgfpathlineto{\pgfqpoint{1.117701in}{1.260855in}}%
\pgfpathlineto{\pgfqpoint{1.116776in}{1.251740in}}%
\pgfpathlineto{\pgfqpoint{1.108617in}{1.252791in}}%
\pgfpathlineto{\pgfqpoint{1.100533in}{1.253969in}}%
\pgfpathlineto{\pgfqpoint{1.092533in}{1.255272in}}%
\pgfpathlineto{\pgfqpoint{1.084625in}{1.256698in}}%
\pgfpathlineto{\pgfqpoint{1.086019in}{1.265733in}}%
\pgfpathlineto{\pgfqpoint{1.087414in}{1.274712in}}%
\pgfpathlineto{\pgfqpoint{1.088809in}{1.283630in}}%
\pgfpathlineto{\pgfqpoint{1.090205in}{1.292488in}}%
\pgfpathlineto{\pgfqpoint{1.097651in}{1.291152in}}%
\pgfpathlineto{\pgfqpoint{1.105184in}{1.289931in}}%
\pgfpathlineto{\pgfqpoint{1.112795in}{1.288829in}}%
\pgfpathlineto{\pgfqpoint{1.120477in}{1.287845in}}%
\pgfpathclose%
\pgfusepath{fill}%
\end{pgfscope}%
\begin{pgfscope}%
\pgfpathrectangle{\pgfqpoint{0.041670in}{0.041670in}}{\pgfqpoint{2.216660in}{2.216660in}}%
\pgfusepath{clip}%
\pgfsetbuttcap%
\pgfsetroundjoin%
\definecolor{currentfill}{rgb}{0.267004,0.004874,0.329415}%
\pgfsetfillcolor{currentfill}%
\pgfsetlinewidth{0.000000pt}%
\definecolor{currentstroke}{rgb}{0.000000,0.000000,0.000000}%
\pgfsetstrokecolor{currentstroke}%
\pgfsetdash{}{0pt}%
\pgfpathmoveto{\pgfqpoint{1.263466in}{0.568574in}}%
\pgfpathlineto{\pgfqpoint{1.264036in}{0.565722in}}%
\pgfpathlineto{\pgfqpoint{1.264607in}{0.563111in}}%
\pgfpathlineto{\pgfqpoint{1.265180in}{0.560747in}}%
\pgfpathlineto{\pgfqpoint{1.265754in}{0.558634in}}%
\pgfpathlineto{\pgfqpoint{1.246548in}{0.557323in}}%
\pgfpathlineto{\pgfqpoint{1.227267in}{0.556342in}}%
\pgfpathlineto{\pgfqpoint{1.207932in}{0.555695in}}%
\pgfpathlineto{\pgfqpoint{1.188566in}{0.555380in}}%
\pgfpathlineto{\pgfqpoint{1.188509in}{0.557515in}}%
\pgfpathlineto{\pgfqpoint{1.188451in}{0.559902in}}%
\pgfpathlineto{\pgfqpoint{1.188394in}{0.562536in}}%
\pgfpathlineto{\pgfqpoint{1.188337in}{0.565410in}}%
\pgfpathlineto{\pgfqpoint{1.207186in}{0.565717in}}%
\pgfpathlineto{\pgfqpoint{1.226005in}{0.566346in}}%
\pgfpathlineto{\pgfqpoint{1.244772in}{0.567299in}}%
\pgfpathlineto{\pgfqpoint{1.263466in}{0.568574in}}%
\pgfpathclose%
\pgfusepath{fill}%
\end{pgfscope}%
\begin{pgfscope}%
\pgfpathrectangle{\pgfqpoint{0.041670in}{0.041670in}}{\pgfqpoint{2.216660in}{2.216660in}}%
\pgfusepath{clip}%
\pgfsetbuttcap%
\pgfsetroundjoin%
\definecolor{currentfill}{rgb}{0.699415,0.867117,0.175971}%
\pgfsetfillcolor{currentfill}%
\pgfsetlinewidth{0.000000pt}%
\definecolor{currentstroke}{rgb}{0.000000,0.000000,0.000000}%
\pgfsetstrokecolor{currentstroke}%
\pgfsetdash{}{0pt}%
\pgfpathmoveto{\pgfqpoint{1.129448in}{1.507421in}}%
\pgfpathlineto{\pgfqpoint{1.128043in}{1.501068in}}%
\pgfpathlineto{\pgfqpoint{1.126637in}{1.494603in}}%
\pgfpathlineto{\pgfqpoint{1.125232in}{1.488028in}}%
\pgfpathlineto{\pgfqpoint{1.123827in}{1.481344in}}%
\pgfpathlineto{\pgfqpoint{1.119225in}{1.482226in}}%
\pgfpathlineto{\pgfqpoint{1.114684in}{1.483177in}}%
\pgfpathlineto{\pgfqpoint{1.110211in}{1.484196in}}%
\pgfpathlineto{\pgfqpoint{1.105809in}{1.485282in}}%
\pgfpathlineto{\pgfqpoint{1.107664in}{1.491864in}}%
\pgfpathlineto{\pgfqpoint{1.109519in}{1.498336in}}%
\pgfpathlineto{\pgfqpoint{1.111375in}{1.504700in}}%
\pgfpathlineto{\pgfqpoint{1.113230in}{1.510952in}}%
\pgfpathlineto{\pgfqpoint{1.117192in}{1.509978in}}%
\pgfpathlineto{\pgfqpoint{1.121219in}{1.509065in}}%
\pgfpathlineto{\pgfqpoint{1.125305in}{1.508212in}}%
\pgfpathlineto{\pgfqpoint{1.129448in}{1.507421in}}%
\pgfpathclose%
\pgfusepath{fill}%
\end{pgfscope}%
\begin{pgfscope}%
\pgfpathrectangle{\pgfqpoint{0.041670in}{0.041670in}}{\pgfqpoint{2.216660in}{2.216660in}}%
\pgfusepath{clip}%
\pgfsetbuttcap%
\pgfsetroundjoin%
\definecolor{currentfill}{rgb}{0.272594,0.025563,0.353093}%
\pgfsetfillcolor{currentfill}%
\pgfsetlinewidth{0.000000pt}%
\definecolor{currentstroke}{rgb}{0.000000,0.000000,0.000000}%
\pgfsetstrokecolor{currentstroke}%
\pgfsetdash{}{0pt}%
\pgfpathmoveto{\pgfqpoint{0.949823in}{0.571098in}}%
\pgfpathlineto{\pgfqpoint{0.948329in}{0.571609in}}%
\pgfpathlineto{\pgfqpoint{0.946831in}{0.572424in}}%
\pgfpathlineto{\pgfqpoint{0.945328in}{0.573549in}}%
\pgfpathlineto{\pgfqpoint{0.943820in}{0.574990in}}%
\pgfpathlineto{\pgfqpoint{0.924564in}{0.579208in}}%
\pgfpathlineto{\pgfqpoint{0.905597in}{0.583752in}}%
\pgfpathlineto{\pgfqpoint{0.886938in}{0.588615in}}%
\pgfpathlineto{\pgfqpoint{0.868609in}{0.593792in}}%
\pgfpathlineto{\pgfqpoint{0.870592in}{0.592236in}}%
\pgfpathlineto{\pgfqpoint{0.872569in}{0.590994in}}%
\pgfpathlineto{\pgfqpoint{0.874540in}{0.590062in}}%
\pgfpathlineto{\pgfqpoint{0.876505in}{0.589433in}}%
\pgfpathlineto{\pgfqpoint{0.894374in}{0.584385in}}%
\pgfpathlineto{\pgfqpoint{0.912564in}{0.579642in}}%
\pgfpathlineto{\pgfqpoint{0.931053in}{0.575211in}}%
\pgfpathlineto{\pgfqpoint{0.949823in}{0.571098in}}%
\pgfpathclose%
\pgfusepath{fill}%
\end{pgfscope}%
\begin{pgfscope}%
\pgfpathrectangle{\pgfqpoint{0.041670in}{0.041670in}}{\pgfqpoint{2.216660in}{2.216660in}}%
\pgfusepath{clip}%
\pgfsetbuttcap%
\pgfsetroundjoin%
\definecolor{currentfill}{rgb}{0.487026,0.823929,0.312321}%
\pgfsetfillcolor{currentfill}%
\pgfsetlinewidth{0.000000pt}%
\definecolor{currentstroke}{rgb}{0.000000,0.000000,0.000000}%
\pgfsetstrokecolor{currentstroke}%
\pgfsetdash{}{0pt}%
\pgfpathmoveto{\pgfqpoint{1.112597in}{1.424138in}}%
\pgfpathlineto{\pgfqpoint{1.111194in}{1.416544in}}%
\pgfpathlineto{\pgfqpoint{1.109792in}{1.408857in}}%
\pgfpathlineto{\pgfqpoint{1.108391in}{1.401078in}}%
\pgfpathlineto{\pgfqpoint{1.106989in}{1.393209in}}%
\pgfpathlineto{\pgfqpoint{1.101009in}{1.394371in}}%
\pgfpathlineto{\pgfqpoint{1.095110in}{1.395624in}}%
\pgfpathlineto{\pgfqpoint{1.089299in}{1.396966in}}%
\pgfpathlineto{\pgfqpoint{1.083582in}{1.398396in}}%
\pgfpathlineto{\pgfqpoint{1.085432in}{1.406159in}}%
\pgfpathlineto{\pgfqpoint{1.087282in}{1.413833in}}%
\pgfpathlineto{\pgfqpoint{1.089133in}{1.421415in}}%
\pgfpathlineto{\pgfqpoint{1.090984in}{1.428905in}}%
\pgfpathlineto{\pgfqpoint{1.096263in}{1.427591in}}%
\pgfpathlineto{\pgfqpoint{1.101629in}{1.426358in}}%
\pgfpathlineto{\pgfqpoint{1.107075in}{1.425206in}}%
\pgfpathlineto{\pgfqpoint{1.112597in}{1.424138in}}%
\pgfpathclose%
\pgfusepath{fill}%
\end{pgfscope}%
\begin{pgfscope}%
\pgfpathrectangle{\pgfqpoint{0.041670in}{0.041670in}}{\pgfqpoint{2.216660in}{2.216660in}}%
\pgfusepath{clip}%
\pgfsetbuttcap%
\pgfsetroundjoin%
\definecolor{currentfill}{rgb}{0.260571,0.246922,0.522828}%
\pgfsetfillcolor{currentfill}%
\pgfsetlinewidth{0.000000pt}%
\definecolor{currentstroke}{rgb}{0.000000,0.000000,0.000000}%
\pgfsetstrokecolor{currentstroke}%
\pgfsetdash{}{0pt}%
\pgfpathmoveto{\pgfqpoint{0.689737in}{0.723244in}}%
\pgfpathlineto{\pgfqpoint{0.686734in}{0.731397in}}%
\pgfpathlineto{\pgfqpoint{0.683716in}{0.739979in}}%
\pgfpathlineto{\pgfqpoint{0.680684in}{0.748999in}}%
\pgfpathlineto{\pgfqpoint{0.677637in}{0.758464in}}%
\pgfpathlineto{\pgfqpoint{0.661255in}{0.767010in}}%
\pgfpathlineto{\pgfqpoint{0.645448in}{0.775815in}}%
\pgfpathlineto{\pgfqpoint{0.630232in}{0.784869in}}%
\pgfpathlineto{\pgfqpoint{0.615620in}{0.794160in}}%
\pgfpathlineto{\pgfqpoint{0.619027in}{0.784525in}}%
\pgfpathlineto{\pgfqpoint{0.622417in}{0.775334in}}%
\pgfpathlineto{\pgfqpoint{0.625791in}{0.766578in}}%
\pgfpathlineto{\pgfqpoint{0.629149in}{0.758251in}}%
\pgfpathlineto{\pgfqpoint{0.643426in}{0.749139in}}%
\pgfpathlineto{\pgfqpoint{0.658293in}{0.740260in}}%
\pgfpathlineto{\pgfqpoint{0.673735in}{0.731624in}}%
\pgfpathlineto{\pgfqpoint{0.689737in}{0.723244in}}%
\pgfpathclose%
\pgfusepath{fill}%
\end{pgfscope}%
\begin{pgfscope}%
\pgfpathrectangle{\pgfqpoint{0.041670in}{0.041670in}}{\pgfqpoint{2.216660in}{2.216660in}}%
\pgfusepath{clip}%
\pgfsetbuttcap%
\pgfsetroundjoin%
\definecolor{currentfill}{rgb}{0.267004,0.004874,0.329415}%
\pgfsetfillcolor{currentfill}%
\pgfsetlinewidth{0.000000pt}%
\definecolor{currentstroke}{rgb}{0.000000,0.000000,0.000000}%
\pgfsetstrokecolor{currentstroke}%
\pgfsetdash{}{0pt}%
\pgfpathmoveto{\pgfqpoint{1.188337in}{0.565410in}}%
\pgfpathlineto{\pgfqpoint{1.188394in}{0.562536in}}%
\pgfpathlineto{\pgfqpoint{1.188451in}{0.559902in}}%
\pgfpathlineto{\pgfqpoint{1.188509in}{0.557515in}}%
\pgfpathlineto{\pgfqpoint{1.188566in}{0.555380in}}%
\pgfpathlineto{\pgfqpoint{1.169191in}{0.555398in}}%
\pgfpathlineto{\pgfqpoint{1.149827in}{0.555750in}}%
\pgfpathlineto{\pgfqpoint{1.130498in}{0.556435in}}%
\pgfpathlineto{\pgfqpoint{1.111224in}{0.557452in}}%
\pgfpathlineto{\pgfqpoint{1.111684in}{0.559573in}}%
\pgfpathlineto{\pgfqpoint{1.112142in}{0.561946in}}%
\pgfpathlineto{\pgfqpoint{1.112600in}{0.564565in}}%
\pgfpathlineto{\pgfqpoint{1.113057in}{0.567425in}}%
\pgfpathlineto{\pgfqpoint{1.131817in}{0.566436in}}%
\pgfpathlineto{\pgfqpoint{1.150631in}{0.565771in}}%
\pgfpathlineto{\pgfqpoint{1.169478in}{0.565428in}}%
\pgfpathlineto{\pgfqpoint{1.188337in}{0.565410in}}%
\pgfpathclose%
\pgfusepath{fill}%
\end{pgfscope}%
\begin{pgfscope}%
\pgfpathrectangle{\pgfqpoint{0.041670in}{0.041670in}}{\pgfqpoint{2.216660in}{2.216660in}}%
\pgfusepath{clip}%
\pgfsetbuttcap%
\pgfsetroundjoin%
\definecolor{currentfill}{rgb}{0.281477,0.755203,0.432552}%
\pgfsetfillcolor{currentfill}%
\pgfsetlinewidth{0.000000pt}%
\definecolor{currentstroke}{rgb}{0.000000,0.000000,0.000000}%
\pgfsetstrokecolor{currentstroke}%
\pgfsetdash{}{0pt}%
\pgfpathmoveto{\pgfqpoint{1.270251in}{1.328457in}}%
\pgfpathlineto{\pgfqpoint{1.271751in}{1.319889in}}%
\pgfpathlineto{\pgfqpoint{1.273249in}{1.311250in}}%
\pgfpathlineto{\pgfqpoint{1.274748in}{1.302544in}}%
\pgfpathlineto{\pgfqpoint{1.276245in}{1.293772in}}%
\pgfpathlineto{\pgfqpoint{1.268882in}{1.292334in}}%
\pgfpathlineto{\pgfqpoint{1.261426in}{1.291010in}}%
\pgfpathlineto{\pgfqpoint{1.253884in}{1.289803in}}%
\pgfpathlineto{\pgfqpoint{1.246264in}{1.288714in}}%
\pgfpathlineto{\pgfqpoint{1.245232in}{1.297571in}}%
\pgfpathlineto{\pgfqpoint{1.244200in}{1.306362in}}%
\pgfpathlineto{\pgfqpoint{1.243167in}{1.315085in}}%
\pgfpathlineto{\pgfqpoint{1.242134in}{1.323738in}}%
\pgfpathlineto{\pgfqpoint{1.249279in}{1.324755in}}%
\pgfpathlineto{\pgfqpoint{1.256352in}{1.325881in}}%
\pgfpathlineto{\pgfqpoint{1.263345in}{1.327115in}}%
\pgfpathlineto{\pgfqpoint{1.270251in}{1.328457in}}%
\pgfpathclose%
\pgfusepath{fill}%
\end{pgfscope}%
\begin{pgfscope}%
\pgfpathrectangle{\pgfqpoint{0.041670in}{0.041670in}}{\pgfqpoint{2.216660in}{2.216660in}}%
\pgfusepath{clip}%
\pgfsetbuttcap%
\pgfsetroundjoin%
\definecolor{currentfill}{rgb}{0.282327,0.094955,0.417331}%
\pgfsetfillcolor{currentfill}%
\pgfsetlinewidth{0.000000pt}%
\definecolor{currentstroke}{rgb}{0.000000,0.000000,0.000000}%
\pgfsetstrokecolor{currentstroke}%
\pgfsetdash{}{0pt}%
\pgfpathmoveto{\pgfqpoint{1.585656in}{0.633607in}}%
\pgfpathlineto{\pgfqpoint{1.588210in}{0.637005in}}%
\pgfpathlineto{\pgfqpoint{1.590773in}{0.640754in}}%
\pgfpathlineto{\pgfqpoint{1.593346in}{0.644859in}}%
\pgfpathlineto{\pgfqpoint{1.595929in}{0.649326in}}%
\pgfpathlineto{\pgfqpoint{1.578657in}{0.642398in}}%
\pgfpathlineto{\pgfqpoint{1.560938in}{0.635762in}}%
\pgfpathlineto{\pgfqpoint{1.542793in}{0.629426in}}%
\pgfpathlineto{\pgfqpoint{1.524241in}{0.623397in}}%
\pgfpathlineto{\pgfqpoint{1.522095in}{0.619076in}}%
\pgfpathlineto{\pgfqpoint{1.519958in}{0.615118in}}%
\pgfpathlineto{\pgfqpoint{1.517829in}{0.611517in}}%
\pgfpathlineto{\pgfqpoint{1.515709in}{0.608266in}}%
\pgfpathlineto{\pgfqpoint{1.533809in}{0.614158in}}%
\pgfpathlineto{\pgfqpoint{1.551512in}{0.620350in}}%
\pgfpathlineto{\pgfqpoint{1.568801in}{0.626836in}}%
\pgfpathlineto{\pgfqpoint{1.585656in}{0.633607in}}%
\pgfpathclose%
\pgfusepath{fill}%
\end{pgfscope}%
\begin{pgfscope}%
\pgfpathrectangle{\pgfqpoint{0.041670in}{0.041670in}}{\pgfqpoint{2.216660in}{2.216660in}}%
\pgfusepath{clip}%
\pgfsetbuttcap%
\pgfsetroundjoin%
\definecolor{currentfill}{rgb}{0.636902,0.856542,0.216620}%
\pgfsetfillcolor{currentfill}%
\pgfsetlinewidth{0.000000pt}%
\definecolor{currentstroke}{rgb}{0.000000,0.000000,0.000000}%
\pgfsetstrokecolor{currentstroke}%
\pgfsetdash{}{0pt}%
\pgfpathmoveto{\pgfqpoint{1.123827in}{1.481344in}}%
\pgfpathlineto{\pgfqpoint{1.122423in}{1.474553in}}%
\pgfpathlineto{\pgfqpoint{1.121018in}{1.467656in}}%
\pgfpathlineto{\pgfqpoint{1.119614in}{1.460655in}}%
\pgfpathlineto{\pgfqpoint{1.118210in}{1.453551in}}%
\pgfpathlineto{\pgfqpoint{1.113147in}{1.454526in}}%
\pgfpathlineto{\pgfqpoint{1.108154in}{1.455577in}}%
\pgfpathlineto{\pgfqpoint{1.103234in}{1.456702in}}%
\pgfpathlineto{\pgfqpoint{1.098393in}{1.457902in}}%
\pgfpathlineto{\pgfqpoint{1.100247in}{1.464902in}}%
\pgfpathlineto{\pgfqpoint{1.102100in}{1.471800in}}%
\pgfpathlineto{\pgfqpoint{1.103955in}{1.478594in}}%
\pgfpathlineto{\pgfqpoint{1.105809in}{1.485282in}}%
\pgfpathlineto{\pgfqpoint{1.110211in}{1.484196in}}%
\pgfpathlineto{\pgfqpoint{1.114684in}{1.483177in}}%
\pgfpathlineto{\pgfqpoint{1.119225in}{1.482226in}}%
\pgfpathlineto{\pgfqpoint{1.123827in}{1.481344in}}%
\pgfpathclose%
\pgfusepath{fill}%
\end{pgfscope}%
\begin{pgfscope}%
\pgfpathrectangle{\pgfqpoint{0.041670in}{0.041670in}}{\pgfqpoint{2.216660in}{2.216660in}}%
\pgfusepath{clip}%
\pgfsetbuttcap%
\pgfsetroundjoin%
\definecolor{currentfill}{rgb}{0.565498,0.842430,0.262877}%
\pgfsetfillcolor{currentfill}%
\pgfsetlinewidth{0.000000pt}%
\definecolor{currentstroke}{rgb}{0.000000,0.000000,0.000000}%
\pgfsetstrokecolor{currentstroke}%
\pgfsetdash{}{0pt}%
\pgfpathmoveto{\pgfqpoint{1.118210in}{1.453551in}}%
\pgfpathlineto{\pgfqpoint{1.116806in}{1.446346in}}%
\pgfpathlineto{\pgfqpoint{1.115403in}{1.439041in}}%
\pgfpathlineto{\pgfqpoint{1.114000in}{1.431638in}}%
\pgfpathlineto{\pgfqpoint{1.112597in}{1.424138in}}%
\pgfpathlineto{\pgfqpoint{1.107075in}{1.425206in}}%
\pgfpathlineto{\pgfqpoint{1.101629in}{1.426358in}}%
\pgfpathlineto{\pgfqpoint{1.096263in}{1.427591in}}%
\pgfpathlineto{\pgfqpoint{1.090984in}{1.428905in}}%
\pgfpathlineto{\pgfqpoint{1.092836in}{1.436300in}}%
\pgfpathlineto{\pgfqpoint{1.094688in}{1.443599in}}%
\pgfpathlineto{\pgfqpoint{1.096540in}{1.450800in}}%
\pgfpathlineto{\pgfqpoint{1.098393in}{1.457902in}}%
\pgfpathlineto{\pgfqpoint{1.103234in}{1.456702in}}%
\pgfpathlineto{\pgfqpoint{1.108154in}{1.455577in}}%
\pgfpathlineto{\pgfqpoint{1.113147in}{1.454526in}}%
\pgfpathlineto{\pgfqpoint{1.118210in}{1.453551in}}%
\pgfpathclose%
\pgfusepath{fill}%
\end{pgfscope}%
\begin{pgfscope}%
\pgfpathrectangle{\pgfqpoint{0.041670in}{0.041670in}}{\pgfqpoint{2.216660in}{2.216660in}}%
\pgfusepath{clip}%
\pgfsetbuttcap%
\pgfsetroundjoin%
\definecolor{currentfill}{rgb}{0.166383,0.690856,0.496502}%
\pgfsetfillcolor{currentfill}%
\pgfsetlinewidth{0.000000pt}%
\definecolor{currentstroke}{rgb}{0.000000,0.000000,0.000000}%
\pgfsetstrokecolor{currentstroke}%
\pgfsetdash{}{0pt}%
\pgfpathmoveto{\pgfqpoint{1.250389in}{1.252668in}}%
\pgfpathlineto{\pgfqpoint{1.251420in}{1.243514in}}%
\pgfpathlineto{\pgfqpoint{1.252450in}{1.234307in}}%
\pgfpathlineto{\pgfqpoint{1.253480in}{1.225051in}}%
\pgfpathlineto{\pgfqpoint{1.254509in}{1.215746in}}%
\pgfpathlineto{\pgfqpoint{1.245865in}{1.214643in}}%
\pgfpathlineto{\pgfqpoint{1.237152in}{1.213676in}}%
\pgfpathlineto{\pgfqpoint{1.228378in}{1.212846in}}%
\pgfpathlineto{\pgfqpoint{1.219553in}{1.212155in}}%
\pgfpathlineto{\pgfqpoint{1.219006in}{1.221514in}}%
\pgfpathlineto{\pgfqpoint{1.218459in}{1.230825in}}%
\pgfpathlineto{\pgfqpoint{1.217912in}{1.240085in}}%
\pgfpathlineto{\pgfqpoint{1.217364in}{1.249293in}}%
\pgfpathlineto{\pgfqpoint{1.225701in}{1.249943in}}%
\pgfpathlineto{\pgfqpoint{1.233991in}{1.250723in}}%
\pgfpathlineto{\pgfqpoint{1.242223in}{1.251632in}}%
\pgfpathlineto{\pgfqpoint{1.250389in}{1.252668in}}%
\pgfpathclose%
\pgfusepath{fill}%
\end{pgfscope}%
\begin{pgfscope}%
\pgfpathrectangle{\pgfqpoint{0.041670in}{0.041670in}}{\pgfqpoint{2.216660in}{2.216660in}}%
\pgfusepath{clip}%
\pgfsetbuttcap%
\pgfsetroundjoin%
\definecolor{currentfill}{rgb}{0.814576,0.883393,0.110347}%
\pgfsetfillcolor{currentfill}%
\pgfsetlinewidth{0.000000pt}%
\definecolor{currentstroke}{rgb}{0.000000,0.000000,0.000000}%
\pgfsetstrokecolor{currentstroke}%
\pgfsetdash{}{0pt}%
\pgfpathmoveto{\pgfqpoint{1.208952in}{1.552507in}}%
\pgfpathlineto{\pgfqpoint{1.209991in}{1.547030in}}%
\pgfpathlineto{\pgfqpoint{1.211030in}{1.541432in}}%
\pgfpathlineto{\pgfqpoint{1.212068in}{1.535713in}}%
\pgfpathlineto{\pgfqpoint{1.213107in}{1.529877in}}%
\pgfpathlineto{\pgfqpoint{1.209261in}{1.529408in}}%
\pgfpathlineto{\pgfqpoint{1.205385in}{1.528998in}}%
\pgfpathlineto{\pgfqpoint{1.201482in}{1.528646in}}%
\pgfpathlineto{\pgfqpoint{1.197558in}{1.528353in}}%
\pgfpathlineto{\pgfqpoint{1.197006in}{1.534239in}}%
\pgfpathlineto{\pgfqpoint{1.196455in}{1.540006in}}%
\pgfpathlineto{\pgfqpoint{1.195903in}{1.545653in}}%
\pgfpathlineto{\pgfqpoint{1.195352in}{1.551179in}}%
\pgfpathlineto{\pgfqpoint{1.198784in}{1.551435in}}%
\pgfpathlineto{\pgfqpoint{1.202197in}{1.551742in}}%
\pgfpathlineto{\pgfqpoint{1.205588in}{1.552099in}}%
\pgfpathlineto{\pgfqpoint{1.208952in}{1.552507in}}%
\pgfpathclose%
\pgfusepath{fill}%
\end{pgfscope}%
\begin{pgfscope}%
\pgfpathrectangle{\pgfqpoint{0.041670in}{0.041670in}}{\pgfqpoint{2.216660in}{2.216660in}}%
\pgfusepath{clip}%
\pgfsetbuttcap%
\pgfsetroundjoin%
\definecolor{currentfill}{rgb}{0.268510,0.009605,0.335427}%
\pgfsetfillcolor{currentfill}%
\pgfsetlinewidth{0.000000pt}%
\definecolor{currentstroke}{rgb}{0.000000,0.000000,0.000000}%
\pgfsetstrokecolor{currentstroke}%
\pgfsetdash{}{0pt}%
\pgfpathmoveto{\pgfqpoint{1.420427in}{0.575531in}}%
\pgfpathlineto{\pgfqpoint{1.422009in}{0.574904in}}%
\pgfpathlineto{\pgfqpoint{1.423596in}{0.574560in}}%
\pgfpathlineto{\pgfqpoint{1.425188in}{0.574503in}}%
\pgfpathlineto{\pgfqpoint{1.426784in}{0.574738in}}%
\pgfpathlineto{\pgfqpoint{1.407985in}{0.570661in}}%
\pgfpathlineto{\pgfqpoint{1.388928in}{0.566906in}}%
\pgfpathlineto{\pgfqpoint{1.369636in}{0.563479in}}%
\pgfpathlineto{\pgfqpoint{1.350130in}{0.560385in}}%
\pgfpathlineto{\pgfqpoint{1.349027in}{0.560242in}}%
\pgfpathlineto{\pgfqpoint{1.347928in}{0.560392in}}%
\pgfpathlineto{\pgfqpoint{1.346831in}{0.560829in}}%
\pgfpathlineto{\pgfqpoint{1.345738in}{0.561550in}}%
\pgfpathlineto{\pgfqpoint{1.364744in}{0.564564in}}%
\pgfpathlineto{\pgfqpoint{1.383540in}{0.567902in}}%
\pgfpathlineto{\pgfqpoint{1.402108in}{0.571559in}}%
\pgfpathlineto{\pgfqpoint{1.420427in}{0.575531in}}%
\pgfpathclose%
\pgfusepath{fill}%
\end{pgfscope}%
\begin{pgfscope}%
\pgfpathrectangle{\pgfqpoint{0.041670in}{0.041670in}}{\pgfqpoint{2.216660in}{2.216660in}}%
\pgfusepath{clip}%
\pgfsetbuttcap%
\pgfsetroundjoin%
\definecolor{currentfill}{rgb}{0.762373,0.876424,0.137064}%
\pgfsetfillcolor{currentfill}%
\pgfsetlinewidth{0.000000pt}%
\definecolor{currentstroke}{rgb}{0.000000,0.000000,0.000000}%
\pgfsetstrokecolor{currentstroke}%
\pgfsetdash{}{0pt}%
\pgfpathmoveto{\pgfqpoint{1.228114in}{1.532317in}}%
\pgfpathlineto{\pgfqpoint{1.229622in}{1.526441in}}%
\pgfpathlineto{\pgfqpoint{1.231131in}{1.520449in}}%
\pgfpathlineto{\pgfqpoint{1.232639in}{1.514342in}}%
\pgfpathlineto{\pgfqpoint{1.234147in}{1.508121in}}%
\pgfpathlineto{\pgfqpoint{1.229998in}{1.507337in}}%
\pgfpathlineto{\pgfqpoint{1.225798in}{1.506616in}}%
\pgfpathlineto{\pgfqpoint{1.221551in}{1.505959in}}%
\pgfpathlineto{\pgfqpoint{1.217261in}{1.505365in}}%
\pgfpathlineto{\pgfqpoint{1.216223in}{1.511665in}}%
\pgfpathlineto{\pgfqpoint{1.215184in}{1.517851in}}%
\pgfpathlineto{\pgfqpoint{1.214146in}{1.523922in}}%
\pgfpathlineto{\pgfqpoint{1.213107in}{1.529877in}}%
\pgfpathlineto{\pgfqpoint{1.216919in}{1.530402in}}%
\pgfpathlineto{\pgfqpoint{1.220694in}{1.530984in}}%
\pgfpathlineto{\pgfqpoint{1.224427in}{1.531623in}}%
\pgfpathlineto{\pgfqpoint{1.228114in}{1.532317in}}%
\pgfpathclose%
\pgfusepath{fill}%
\end{pgfscope}%
\begin{pgfscope}%
\pgfpathrectangle{\pgfqpoint{0.041670in}{0.041670in}}{\pgfqpoint{2.216660in}{2.216660in}}%
\pgfusepath{clip}%
\pgfsetbuttcap%
\pgfsetroundjoin%
\definecolor{currentfill}{rgb}{0.134692,0.658636,0.517649}%
\pgfsetfillcolor{currentfill}%
\pgfsetlinewidth{0.000000pt}%
\definecolor{currentstroke}{rgb}{0.000000,0.000000,0.000000}%
\pgfsetstrokecolor{currentstroke}%
\pgfsetdash{}{0pt}%
\pgfpathmoveto{\pgfqpoint{1.219553in}{1.212155in}}%
\pgfpathlineto{\pgfqpoint{1.220100in}{1.202750in}}%
\pgfpathlineto{\pgfqpoint{1.220647in}{1.193303in}}%
\pgfpathlineto{\pgfqpoint{1.221193in}{1.183815in}}%
\pgfpathlineto{\pgfqpoint{1.221740in}{1.174289in}}%
\pgfpathlineto{\pgfqpoint{1.212383in}{1.173702in}}%
\pgfpathlineto{\pgfqpoint{1.202993in}{1.173264in}}%
\pgfpathlineto{\pgfqpoint{1.193577in}{1.172974in}}%
\pgfpathlineto{\pgfqpoint{1.184148in}{1.172833in}}%
\pgfpathlineto{\pgfqpoint{1.184093in}{1.182381in}}%
\pgfpathlineto{\pgfqpoint{1.184038in}{1.191890in}}%
\pgfpathlineto{\pgfqpoint{1.183983in}{1.201358in}}%
\pgfpathlineto{\pgfqpoint{1.183928in}{1.210783in}}%
\pgfpathlineto{\pgfqpoint{1.192864in}{1.210916in}}%
\pgfpathlineto{\pgfqpoint{1.201787in}{1.211189in}}%
\pgfpathlineto{\pgfqpoint{1.210686in}{1.211602in}}%
\pgfpathlineto{\pgfqpoint{1.219553in}{1.212155in}}%
\pgfpathclose%
\pgfusepath{fill}%
\end{pgfscope}%
\begin{pgfscope}%
\pgfpathrectangle{\pgfqpoint{0.041670in}{0.041670in}}{\pgfqpoint{2.216660in}{2.216660in}}%
\pgfusepath{clip}%
\pgfsetbuttcap%
\pgfsetroundjoin%
\definecolor{currentfill}{rgb}{0.814576,0.883393,0.110347}%
\pgfsetfillcolor{currentfill}%
\pgfsetlinewidth{0.000000pt}%
\definecolor{currentstroke}{rgb}{0.000000,0.000000,0.000000}%
\pgfsetstrokecolor{currentstroke}%
\pgfsetdash{}{0pt}%
\pgfpathmoveto{\pgfqpoint{1.167623in}{1.550995in}}%
\pgfpathlineto{\pgfqpoint{1.167181in}{1.545462in}}%
\pgfpathlineto{\pgfqpoint{1.166739in}{1.539808in}}%
\pgfpathlineto{\pgfqpoint{1.166297in}{1.534034in}}%
\pgfpathlineto{\pgfqpoint{1.165855in}{1.528142in}}%
\pgfpathlineto{\pgfqpoint{1.161915in}{1.528383in}}%
\pgfpathlineto{\pgfqpoint{1.157993in}{1.528682in}}%
\pgfpathlineto{\pgfqpoint{1.154093in}{1.529041in}}%
\pgfpathlineto{\pgfqpoint{1.150220in}{1.529458in}}%
\pgfpathlineto{\pgfqpoint{1.151152in}{1.535308in}}%
\pgfpathlineto{\pgfqpoint{1.152083in}{1.541040in}}%
\pgfpathlineto{\pgfqpoint{1.153015in}{1.546651in}}%
\pgfpathlineto{\pgfqpoint{1.153947in}{1.552142in}}%
\pgfpathlineto{\pgfqpoint{1.157334in}{1.551779in}}%
\pgfpathlineto{\pgfqpoint{1.160745in}{1.551466in}}%
\pgfpathlineto{\pgfqpoint{1.164176in}{1.551205in}}%
\pgfpathlineto{\pgfqpoint{1.167623in}{1.550995in}}%
\pgfpathclose%
\pgfusepath{fill}%
\end{pgfscope}%
\begin{pgfscope}%
\pgfpathrectangle{\pgfqpoint{0.041670in}{0.041670in}}{\pgfqpoint{2.216660in}{2.216660in}}%
\pgfusepath{clip}%
\pgfsetbuttcap%
\pgfsetroundjoin%
\definecolor{currentfill}{rgb}{0.281477,0.755203,0.432552}%
\pgfsetfillcolor{currentfill}%
\pgfsetlinewidth{0.000000pt}%
\definecolor{currentstroke}{rgb}{0.000000,0.000000,0.000000}%
\pgfsetstrokecolor{currentstroke}%
\pgfsetdash{}{0pt}%
\pgfpathmoveto{\pgfqpoint{1.124182in}{1.322928in}}%
\pgfpathlineto{\pgfqpoint{1.123255in}{1.314260in}}%
\pgfpathlineto{\pgfqpoint{1.122329in}{1.305523in}}%
\pgfpathlineto{\pgfqpoint{1.121403in}{1.296717in}}%
\pgfpathlineto{\pgfqpoint{1.120477in}{1.287845in}}%
\pgfpathlineto{\pgfqpoint{1.112795in}{1.288829in}}%
\pgfpathlineto{\pgfqpoint{1.105184in}{1.289931in}}%
\pgfpathlineto{\pgfqpoint{1.097651in}{1.291152in}}%
\pgfpathlineto{\pgfqpoint{1.090205in}{1.292488in}}%
\pgfpathlineto{\pgfqpoint{1.091601in}{1.301282in}}%
\pgfpathlineto{\pgfqpoint{1.092998in}{1.310010in}}%
\pgfpathlineto{\pgfqpoint{1.094395in}{1.318669in}}%
\pgfpathlineto{\pgfqpoint{1.095793in}{1.327259in}}%
\pgfpathlineto{\pgfqpoint{1.102776in}{1.326013in}}%
\pgfpathlineto{\pgfqpoint{1.109841in}{1.324874in}}%
\pgfpathlineto{\pgfqpoint{1.116978in}{1.323846in}}%
\pgfpathlineto{\pgfqpoint{1.124182in}{1.322928in}}%
\pgfpathclose%
\pgfusepath{fill}%
\end{pgfscope}%
\begin{pgfscope}%
\pgfpathrectangle{\pgfqpoint{0.041670in}{0.041670in}}{\pgfqpoint{2.216660in}{2.216660in}}%
\pgfusepath{clip}%
\pgfsetbuttcap%
\pgfsetroundjoin%
\definecolor{currentfill}{rgb}{0.134692,0.658636,0.517649}%
\pgfsetfillcolor{currentfill}%
\pgfsetlinewidth{0.000000pt}%
\definecolor{currentstroke}{rgb}{0.000000,0.000000,0.000000}%
\pgfsetstrokecolor{currentstroke}%
\pgfsetdash{}{0pt}%
\pgfpathmoveto{\pgfqpoint{1.183928in}{1.210783in}}%
\pgfpathlineto{\pgfqpoint{1.183983in}{1.201358in}}%
\pgfpathlineto{\pgfqpoint{1.184038in}{1.191890in}}%
\pgfpathlineto{\pgfqpoint{1.184093in}{1.182381in}}%
\pgfpathlineto{\pgfqpoint{1.184148in}{1.172833in}}%
\pgfpathlineto{\pgfqpoint{1.174714in}{1.172842in}}%
\pgfpathlineto{\pgfqpoint{1.165285in}{1.172999in}}%
\pgfpathlineto{\pgfqpoint{1.155872in}{1.173305in}}%
\pgfpathlineto{\pgfqpoint{1.146485in}{1.173760in}}%
\pgfpathlineto{\pgfqpoint{1.146923in}{1.183294in}}%
\pgfpathlineto{\pgfqpoint{1.147360in}{1.192789in}}%
\pgfpathlineto{\pgfqpoint{1.147798in}{1.202245in}}%
\pgfpathlineto{\pgfqpoint{1.148236in}{1.211657in}}%
\pgfpathlineto{\pgfqpoint{1.157133in}{1.211228in}}%
\pgfpathlineto{\pgfqpoint{1.166053in}{1.210939in}}%
\pgfpathlineto{\pgfqpoint{1.174988in}{1.210791in}}%
\pgfpathlineto{\pgfqpoint{1.183928in}{1.210783in}}%
\pgfpathclose%
\pgfusepath{fill}%
\end{pgfscope}%
\begin{pgfscope}%
\pgfpathrectangle{\pgfqpoint{0.041670in}{0.041670in}}{\pgfqpoint{2.216660in}{2.216660in}}%
\pgfusepath{clip}%
\pgfsetbuttcap%
\pgfsetroundjoin%
\definecolor{currentfill}{rgb}{0.166383,0.690856,0.496502}%
\pgfsetfillcolor{currentfill}%
\pgfsetlinewidth{0.000000pt}%
\definecolor{currentstroke}{rgb}{0.000000,0.000000,0.000000}%
\pgfsetstrokecolor{currentstroke}%
\pgfsetdash{}{0pt}%
\pgfpathmoveto{\pgfqpoint{1.149990in}{1.248825in}}%
\pgfpathlineto{\pgfqpoint{1.149551in}{1.239610in}}%
\pgfpathlineto{\pgfqpoint{1.149113in}{1.230342in}}%
\pgfpathlineto{\pgfqpoint{1.148675in}{1.221023in}}%
\pgfpathlineto{\pgfqpoint{1.148236in}{1.211657in}}%
\pgfpathlineto{\pgfqpoint{1.139374in}{1.212225in}}%
\pgfpathlineto{\pgfqpoint{1.130554in}{1.212932in}}%
\pgfpathlineto{\pgfqpoint{1.121786in}{1.213777in}}%
\pgfpathlineto{\pgfqpoint{1.113080in}{1.214759in}}%
\pgfpathlineto{\pgfqpoint{1.114004in}{1.224078in}}%
\pgfpathlineto{\pgfqpoint{1.114928in}{1.233350in}}%
\pgfpathlineto{\pgfqpoint{1.115852in}{1.242572in}}%
\pgfpathlineto{\pgfqpoint{1.116776in}{1.251740in}}%
\pgfpathlineto{\pgfqpoint{1.125001in}{1.250818in}}%
\pgfpathlineto{\pgfqpoint{1.133285in}{1.250023in}}%
\pgfpathlineto{\pgfqpoint{1.141617in}{1.249359in}}%
\pgfpathlineto{\pgfqpoint{1.149990in}{1.248825in}}%
\pgfpathclose%
\pgfusepath{fill}%
\end{pgfscope}%
\begin{pgfscope}%
\pgfpathrectangle{\pgfqpoint{0.041670in}{0.041670in}}{\pgfqpoint{2.216660in}{2.216660in}}%
\pgfusepath{clip}%
\pgfsetbuttcap%
\pgfsetroundjoin%
\definecolor{currentfill}{rgb}{0.762373,0.876424,0.137064}%
\pgfsetfillcolor{currentfill}%
\pgfsetlinewidth{0.000000pt}%
\definecolor{currentstroke}{rgb}{0.000000,0.000000,0.000000}%
\pgfsetstrokecolor{currentstroke}%
\pgfsetdash{}{0pt}%
\pgfpathmoveto{\pgfqpoint{1.150220in}{1.529458in}}%
\pgfpathlineto{\pgfqpoint{1.149288in}{1.523490in}}%
\pgfpathlineto{\pgfqpoint{1.148357in}{1.517405in}}%
\pgfpathlineto{\pgfqpoint{1.147425in}{1.511206in}}%
\pgfpathlineto{\pgfqpoint{1.146494in}{1.504892in}}%
\pgfpathlineto{\pgfqpoint{1.142170in}{1.505428in}}%
\pgfpathlineto{\pgfqpoint{1.137885in}{1.506028in}}%
\pgfpathlineto{\pgfqpoint{1.133643in}{1.506693in}}%
\pgfpathlineto{\pgfqpoint{1.129448in}{1.507421in}}%
\pgfpathlineto{\pgfqpoint{1.130854in}{1.513662in}}%
\pgfpathlineto{\pgfqpoint{1.132259in}{1.519790in}}%
\pgfpathlineto{\pgfqpoint{1.133665in}{1.525802in}}%
\pgfpathlineto{\pgfqpoint{1.135071in}{1.531697in}}%
\pgfpathlineto{\pgfqpoint{1.138799in}{1.531053in}}%
\pgfpathlineto{\pgfqpoint{1.142569in}{1.530464in}}%
\pgfpathlineto{\pgfqpoint{1.146377in}{1.529932in}}%
\pgfpathlineto{\pgfqpoint{1.150220in}{1.529458in}}%
\pgfpathclose%
\pgfusepath{fill}%
\end{pgfscope}%
\begin{pgfscope}%
\pgfpathrectangle{\pgfqpoint{0.041670in}{0.041670in}}{\pgfqpoint{2.216660in}{2.216660in}}%
\pgfusepath{clip}%
\pgfsetbuttcap%
\pgfsetroundjoin%
\definecolor{currentfill}{rgb}{0.344074,0.780029,0.397381}%
\pgfsetfillcolor{currentfill}%
\pgfsetlinewidth{0.000000pt}%
\definecolor{currentstroke}{rgb}{0.000000,0.000000,0.000000}%
\pgfsetstrokecolor{currentstroke}%
\pgfsetdash{}{0pt}%
\pgfpathmoveto{\pgfqpoint{1.264250in}{1.361989in}}%
\pgfpathlineto{\pgfqpoint{1.265751in}{1.353721in}}%
\pgfpathlineto{\pgfqpoint{1.267252in}{1.345375in}}%
\pgfpathlineto{\pgfqpoint{1.268752in}{1.336953in}}%
\pgfpathlineto{\pgfqpoint{1.270251in}{1.328457in}}%
\pgfpathlineto{\pgfqpoint{1.263345in}{1.327115in}}%
\pgfpathlineto{\pgfqpoint{1.256352in}{1.325881in}}%
\pgfpathlineto{\pgfqpoint{1.249279in}{1.324755in}}%
\pgfpathlineto{\pgfqpoint{1.242134in}{1.323738in}}%
\pgfpathlineto{\pgfqpoint{1.241101in}{1.332319in}}%
\pgfpathlineto{\pgfqpoint{1.240067in}{1.340825in}}%
\pgfpathlineto{\pgfqpoint{1.239033in}{1.349254in}}%
\pgfpathlineto{\pgfqpoint{1.237998in}{1.357605in}}%
\pgfpathlineto{\pgfqpoint{1.244669in}{1.358549in}}%
\pgfpathlineto{\pgfqpoint{1.251273in}{1.359595in}}%
\pgfpathlineto{\pgfqpoint{1.257802in}{1.360742in}}%
\pgfpathlineto{\pgfqpoint{1.264250in}{1.361989in}}%
\pgfpathclose%
\pgfusepath{fill}%
\end{pgfscope}%
\begin{pgfscope}%
\pgfpathrectangle{\pgfqpoint{0.041670in}{0.041670in}}{\pgfqpoint{2.216660in}{2.216660in}}%
\pgfusepath{clip}%
\pgfsetbuttcap%
\pgfsetroundjoin%
\definecolor{currentfill}{rgb}{0.814576,0.883393,0.110347}%
\pgfsetfillcolor{currentfill}%
\pgfsetlinewidth{0.000000pt}%
\definecolor{currentstroke}{rgb}{0.000000,0.000000,0.000000}%
\pgfsetstrokecolor{currentstroke}%
\pgfsetdash{}{0pt}%
\pgfpathmoveto{\pgfqpoint{1.195352in}{1.551179in}}%
\pgfpathlineto{\pgfqpoint{1.195903in}{1.545653in}}%
\pgfpathlineto{\pgfqpoint{1.196455in}{1.540006in}}%
\pgfpathlineto{\pgfqpoint{1.197006in}{1.534239in}}%
\pgfpathlineto{\pgfqpoint{1.197558in}{1.528353in}}%
\pgfpathlineto{\pgfqpoint{1.193616in}{1.528118in}}%
\pgfpathlineto{\pgfqpoint{1.189659in}{1.527943in}}%
\pgfpathlineto{\pgfqpoint{1.185693in}{1.527828in}}%
\pgfpathlineto{\pgfqpoint{1.181721in}{1.527771in}}%
\pgfpathlineto{\pgfqpoint{1.181666in}{1.533676in}}%
\pgfpathlineto{\pgfqpoint{1.181610in}{1.539462in}}%
\pgfpathlineto{\pgfqpoint{1.181555in}{1.545127in}}%
\pgfpathlineto{\pgfqpoint{1.181500in}{1.550672in}}%
\pgfpathlineto{\pgfqpoint{1.184974in}{1.550721in}}%
\pgfpathlineto{\pgfqpoint{1.188443in}{1.550822in}}%
\pgfpathlineto{\pgfqpoint{1.191903in}{1.550975in}}%
\pgfpathlineto{\pgfqpoint{1.195352in}{1.551179in}}%
\pgfpathclose%
\pgfusepath{fill}%
\end{pgfscope}%
\begin{pgfscope}%
\pgfpathrectangle{\pgfqpoint{0.041670in}{0.041670in}}{\pgfqpoint{2.216660in}{2.216660in}}%
\pgfusepath{clip}%
\pgfsetbuttcap%
\pgfsetroundjoin%
\definecolor{currentfill}{rgb}{0.814576,0.883393,0.110347}%
\pgfsetfillcolor{currentfill}%
\pgfsetlinewidth{0.000000pt}%
\definecolor{currentstroke}{rgb}{0.000000,0.000000,0.000000}%
\pgfsetstrokecolor{currentstroke}%
\pgfsetdash{}{0pt}%
\pgfpathmoveto{\pgfqpoint{1.181500in}{1.550672in}}%
\pgfpathlineto{\pgfqpoint{1.181555in}{1.545127in}}%
\pgfpathlineto{\pgfqpoint{1.181610in}{1.539462in}}%
\pgfpathlineto{\pgfqpoint{1.181666in}{1.533676in}}%
\pgfpathlineto{\pgfqpoint{1.181721in}{1.527771in}}%
\pgfpathlineto{\pgfqpoint{1.177747in}{1.527775in}}%
\pgfpathlineto{\pgfqpoint{1.173775in}{1.527837in}}%
\pgfpathlineto{\pgfqpoint{1.169810in}{1.527960in}}%
\pgfpathlineto{\pgfqpoint{1.165855in}{1.528142in}}%
\pgfpathlineto{\pgfqpoint{1.166297in}{1.534034in}}%
\pgfpathlineto{\pgfqpoint{1.166739in}{1.539808in}}%
\pgfpathlineto{\pgfqpoint{1.167181in}{1.545462in}}%
\pgfpathlineto{\pgfqpoint{1.167623in}{1.550995in}}%
\pgfpathlineto{\pgfqpoint{1.171082in}{1.550837in}}%
\pgfpathlineto{\pgfqpoint{1.174550in}{1.550730in}}%
\pgfpathlineto{\pgfqpoint{1.178024in}{1.550675in}}%
\pgfpathlineto{\pgfqpoint{1.181500in}{1.550672in}}%
\pgfpathclose%
\pgfusepath{fill}%
\end{pgfscope}%
\begin{pgfscope}%
\pgfpathrectangle{\pgfqpoint{0.041670in}{0.041670in}}{\pgfqpoint{2.216660in}{2.216660in}}%
\pgfusepath{clip}%
\pgfsetbuttcap%
\pgfsetroundjoin%
\definecolor{currentfill}{rgb}{0.267004,0.004874,0.329415}%
\pgfsetfillcolor{currentfill}%
\pgfsetlinewidth{0.000000pt}%
\definecolor{currentstroke}{rgb}{0.000000,0.000000,0.000000}%
\pgfsetstrokecolor{currentstroke}%
\pgfsetdash{}{0pt}%
\pgfpathmoveto{\pgfqpoint{1.341395in}{0.567149in}}%
\pgfpathlineto{\pgfqpoint{1.342477in}{0.565352in}}%
\pgfpathlineto{\pgfqpoint{1.343561in}{0.563816in}}%
\pgfpathlineto{\pgfqpoint{1.344648in}{0.562547in}}%
\pgfpathlineto{\pgfqpoint{1.345738in}{0.561550in}}%
\pgfpathlineto{\pgfqpoint{1.326546in}{0.558863in}}%
\pgfpathlineto{\pgfqpoint{1.307189in}{0.556507in}}%
\pgfpathlineto{\pgfqpoint{1.287688in}{0.554484in}}%
\pgfpathlineto{\pgfqpoint{1.268064in}{0.552798in}}%
\pgfpathlineto{\pgfqpoint{1.267484in}{0.553855in}}%
\pgfpathlineto{\pgfqpoint{1.266906in}{0.555183in}}%
\pgfpathlineto{\pgfqpoint{1.266329in}{0.556777in}}%
\pgfpathlineto{\pgfqpoint{1.265754in}{0.558634in}}%
\pgfpathlineto{\pgfqpoint{1.284863in}{0.560274in}}%
\pgfpathlineto{\pgfqpoint{1.303854in}{0.562242in}}%
\pgfpathlineto{\pgfqpoint{1.322705in}{0.564535in}}%
\pgfpathlineto{\pgfqpoint{1.341395in}{0.567149in}}%
\pgfpathclose%
\pgfusepath{fill}%
\end{pgfscope}%
\begin{pgfscope}%
\pgfpathrectangle{\pgfqpoint{0.041670in}{0.041670in}}{\pgfqpoint{2.216660in}{2.216660in}}%
\pgfusepath{clip}%
\pgfsetbuttcap%
\pgfsetroundjoin%
\definecolor{currentfill}{rgb}{0.699415,0.867117,0.175971}%
\pgfsetfillcolor{currentfill}%
\pgfsetlinewidth{0.000000pt}%
\definecolor{currentstroke}{rgb}{0.000000,0.000000,0.000000}%
\pgfsetstrokecolor{currentstroke}%
\pgfsetdash{}{0pt}%
\pgfpathmoveto{\pgfqpoint{1.234147in}{1.508121in}}%
\pgfpathlineto{\pgfqpoint{1.235654in}{1.501788in}}%
\pgfpathlineto{\pgfqpoint{1.237162in}{1.495343in}}%
\pgfpathlineto{\pgfqpoint{1.238669in}{1.488788in}}%
\pgfpathlineto{\pgfqpoint{1.240177in}{1.482125in}}%
\pgfpathlineto{\pgfqpoint{1.235567in}{1.481250in}}%
\pgfpathlineto{\pgfqpoint{1.230900in}{1.480446in}}%
\pgfpathlineto{\pgfqpoint{1.226181in}{1.479712in}}%
\pgfpathlineto{\pgfqpoint{1.221415in}{1.479050in}}%
\pgfpathlineto{\pgfqpoint{1.220376in}{1.485793in}}%
\pgfpathlineto{\pgfqpoint{1.219338in}{1.492428in}}%
\pgfpathlineto{\pgfqpoint{1.218300in}{1.498952in}}%
\pgfpathlineto{\pgfqpoint{1.217261in}{1.505365in}}%
\pgfpathlineto{\pgfqpoint{1.221551in}{1.505959in}}%
\pgfpathlineto{\pgfqpoint{1.225798in}{1.506616in}}%
\pgfpathlineto{\pgfqpoint{1.229998in}{1.507337in}}%
\pgfpathlineto{\pgfqpoint{1.234147in}{1.508121in}}%
\pgfpathclose%
\pgfusepath{fill}%
\end{pgfscope}%
\begin{pgfscope}%
\pgfpathrectangle{\pgfqpoint{0.041670in}{0.041670in}}{\pgfqpoint{2.216660in}{2.216660in}}%
\pgfusepath{clip}%
\pgfsetbuttcap%
\pgfsetroundjoin%
\definecolor{currentfill}{rgb}{0.220124,0.725509,0.466226}%
\pgfsetfillcolor{currentfill}%
\pgfsetlinewidth{0.000000pt}%
\definecolor{currentstroke}{rgb}{0.000000,0.000000,0.000000}%
\pgfsetstrokecolor{currentstroke}%
\pgfsetdash{}{0pt}%
\pgfpathmoveto{\pgfqpoint{1.246264in}{1.288714in}}%
\pgfpathlineto{\pgfqpoint{1.247296in}{1.279792in}}%
\pgfpathlineto{\pgfqpoint{1.248328in}{1.270810in}}%
\pgfpathlineto{\pgfqpoint{1.249359in}{1.261767in}}%
\pgfpathlineto{\pgfqpoint{1.250389in}{1.252668in}}%
\pgfpathlineto{\pgfqpoint{1.242223in}{1.251632in}}%
\pgfpathlineto{\pgfqpoint{1.233991in}{1.250723in}}%
\pgfpathlineto{\pgfqpoint{1.225701in}{1.249943in}}%
\pgfpathlineto{\pgfqpoint{1.217364in}{1.249293in}}%
\pgfpathlineto{\pgfqpoint{1.216816in}{1.258447in}}%
\pgfpathlineto{\pgfqpoint{1.216268in}{1.267542in}}%
\pgfpathlineto{\pgfqpoint{1.215720in}{1.276579in}}%
\pgfpathlineto{\pgfqpoint{1.215172in}{1.285553in}}%
\pgfpathlineto{\pgfqpoint{1.223021in}{1.286162in}}%
\pgfpathlineto{\pgfqpoint{1.230825in}{1.286892in}}%
\pgfpathlineto{\pgfqpoint{1.238576in}{1.287743in}}%
\pgfpathlineto{\pgfqpoint{1.246264in}{1.288714in}}%
\pgfpathclose%
\pgfusepath{fill}%
\end{pgfscope}%
\begin{pgfscope}%
\pgfpathrectangle{\pgfqpoint{0.041670in}{0.041670in}}{\pgfqpoint{2.216660in}{2.216660in}}%
\pgfusepath{clip}%
\pgfsetbuttcap%
\pgfsetroundjoin%
\definecolor{currentfill}{rgb}{0.344074,0.780029,0.397381}%
\pgfsetfillcolor{currentfill}%
\pgfsetlinewidth{0.000000pt}%
\definecolor{currentstroke}{rgb}{0.000000,0.000000,0.000000}%
\pgfsetstrokecolor{currentstroke}%
\pgfsetdash{}{0pt}%
\pgfpathmoveto{\pgfqpoint{1.127892in}{1.356852in}}%
\pgfpathlineto{\pgfqpoint{1.126964in}{1.348487in}}%
\pgfpathlineto{\pgfqpoint{1.126036in}{1.340043in}}%
\pgfpathlineto{\pgfqpoint{1.125109in}{1.331523in}}%
\pgfpathlineto{\pgfqpoint{1.124182in}{1.322928in}}%
\pgfpathlineto{\pgfqpoint{1.116978in}{1.323846in}}%
\pgfpathlineto{\pgfqpoint{1.109841in}{1.324874in}}%
\pgfpathlineto{\pgfqpoint{1.102776in}{1.326013in}}%
\pgfpathlineto{\pgfqpoint{1.095793in}{1.327259in}}%
\pgfpathlineto{\pgfqpoint{1.097191in}{1.335777in}}%
\pgfpathlineto{\pgfqpoint{1.098589in}{1.344220in}}%
\pgfpathlineto{\pgfqpoint{1.099988in}{1.352587in}}%
\pgfpathlineto{\pgfqpoint{1.101387in}{1.360876in}}%
\pgfpathlineto{\pgfqpoint{1.107908in}{1.359718in}}%
\pgfpathlineto{\pgfqpoint{1.114503in}{1.358661in}}%
\pgfpathlineto{\pgfqpoint{1.121167in}{1.357705in}}%
\pgfpathlineto{\pgfqpoint{1.127892in}{1.356852in}}%
\pgfpathclose%
\pgfusepath{fill}%
\end{pgfscope}%
\begin{pgfscope}%
\pgfpathrectangle{\pgfqpoint{0.041670in}{0.041670in}}{\pgfqpoint{2.216660in}{2.216660in}}%
\pgfusepath{clip}%
\pgfsetbuttcap%
\pgfsetroundjoin%
\definecolor{currentfill}{rgb}{0.412913,0.803041,0.357269}%
\pgfsetfillcolor{currentfill}%
\pgfsetlinewidth{0.000000pt}%
\definecolor{currentstroke}{rgb}{0.000000,0.000000,0.000000}%
\pgfsetstrokecolor{currentstroke}%
\pgfsetdash{}{0pt}%
\pgfpathmoveto{\pgfqpoint{1.258241in}{1.394238in}}%
\pgfpathlineto{\pgfqpoint{1.259744in}{1.386302in}}%
\pgfpathlineto{\pgfqpoint{1.261246in}{1.378281in}}%
\pgfpathlineto{\pgfqpoint{1.262748in}{1.370176in}}%
\pgfpathlineto{\pgfqpoint{1.264250in}{1.361989in}}%
\pgfpathlineto{\pgfqpoint{1.257802in}{1.360742in}}%
\pgfpathlineto{\pgfqpoint{1.251273in}{1.359595in}}%
\pgfpathlineto{\pgfqpoint{1.244669in}{1.358549in}}%
\pgfpathlineto{\pgfqpoint{1.237998in}{1.357605in}}%
\pgfpathlineto{\pgfqpoint{1.236964in}{1.365876in}}%
\pgfpathlineto{\pgfqpoint{1.235929in}{1.374064in}}%
\pgfpathlineto{\pgfqpoint{1.234894in}{1.382168in}}%
\pgfpathlineto{\pgfqpoint{1.233858in}{1.390186in}}%
\pgfpathlineto{\pgfqpoint{1.240054in}{1.391059in}}%
\pgfpathlineto{\pgfqpoint{1.246187in}{1.392026in}}%
\pgfpathlineto{\pgfqpoint{1.252251in}{1.393086in}}%
\pgfpathlineto{\pgfqpoint{1.258241in}{1.394238in}}%
\pgfpathclose%
\pgfusepath{fill}%
\end{pgfscope}%
\begin{pgfscope}%
\pgfpathrectangle{\pgfqpoint{0.041670in}{0.041670in}}{\pgfqpoint{2.216660in}{2.216660in}}%
\pgfusepath{clip}%
\pgfsetbuttcap%
\pgfsetroundjoin%
\definecolor{currentfill}{rgb}{0.699415,0.867117,0.175971}%
\pgfsetfillcolor{currentfill}%
\pgfsetlinewidth{0.000000pt}%
\definecolor{currentstroke}{rgb}{0.000000,0.000000,0.000000}%
\pgfsetstrokecolor{currentstroke}%
\pgfsetdash{}{0pt}%
\pgfpathmoveto{\pgfqpoint{1.146494in}{1.504892in}}%
\pgfpathlineto{\pgfqpoint{1.145562in}{1.498465in}}%
\pgfpathlineto{\pgfqpoint{1.144631in}{1.491927in}}%
\pgfpathlineto{\pgfqpoint{1.143699in}{1.485279in}}%
\pgfpathlineto{\pgfqpoint{1.142768in}{1.478522in}}%
\pgfpathlineto{\pgfqpoint{1.137963in}{1.479120in}}%
\pgfpathlineto{\pgfqpoint{1.133202in}{1.479790in}}%
\pgfpathlineto{\pgfqpoint{1.128488in}{1.480532in}}%
\pgfpathlineto{\pgfqpoint{1.123827in}{1.481344in}}%
\pgfpathlineto{\pgfqpoint{1.125232in}{1.488028in}}%
\pgfpathlineto{\pgfqpoint{1.126637in}{1.494603in}}%
\pgfpathlineto{\pgfqpoint{1.128043in}{1.501068in}}%
\pgfpathlineto{\pgfqpoint{1.129448in}{1.507421in}}%
\pgfpathlineto{\pgfqpoint{1.133643in}{1.506693in}}%
\pgfpathlineto{\pgfqpoint{1.137885in}{1.506028in}}%
\pgfpathlineto{\pgfqpoint{1.142170in}{1.505428in}}%
\pgfpathlineto{\pgfqpoint{1.146494in}{1.504892in}}%
\pgfpathclose%
\pgfusepath{fill}%
\end{pgfscope}%
\begin{pgfscope}%
\pgfpathrectangle{\pgfqpoint{0.041670in}{0.041670in}}{\pgfqpoint{2.216660in}{2.216660in}}%
\pgfusepath{clip}%
\pgfsetbuttcap%
\pgfsetroundjoin%
\definecolor{currentfill}{rgb}{0.166383,0.690856,0.496502}%
\pgfsetfillcolor{currentfill}%
\pgfsetlinewidth{0.000000pt}%
\definecolor{currentstroke}{rgb}{0.000000,0.000000,0.000000}%
\pgfsetstrokecolor{currentstroke}%
\pgfsetdash{}{0pt}%
\pgfpathmoveto{\pgfqpoint{1.217364in}{1.249293in}}%
\pgfpathlineto{\pgfqpoint{1.217912in}{1.240085in}}%
\pgfpathlineto{\pgfqpoint{1.218459in}{1.230825in}}%
\pgfpathlineto{\pgfqpoint{1.219006in}{1.221514in}}%
\pgfpathlineto{\pgfqpoint{1.219553in}{1.212155in}}%
\pgfpathlineto{\pgfqpoint{1.210686in}{1.211602in}}%
\pgfpathlineto{\pgfqpoint{1.201787in}{1.211189in}}%
\pgfpathlineto{\pgfqpoint{1.192864in}{1.210916in}}%
\pgfpathlineto{\pgfqpoint{1.183928in}{1.210783in}}%
\pgfpathlineto{\pgfqpoint{1.183873in}{1.220163in}}%
\pgfpathlineto{\pgfqpoint{1.183819in}{1.229495in}}%
\pgfpathlineto{\pgfqpoint{1.183764in}{1.238776in}}%
\pgfpathlineto{\pgfqpoint{1.183709in}{1.248005in}}%
\pgfpathlineto{\pgfqpoint{1.192151in}{1.248129in}}%
\pgfpathlineto{\pgfqpoint{1.200580in}{1.248386in}}%
\pgfpathlineto{\pgfqpoint{1.208987in}{1.248774in}}%
\pgfpathlineto{\pgfqpoint{1.217364in}{1.249293in}}%
\pgfpathclose%
\pgfusepath{fill}%
\end{pgfscope}%
\begin{pgfscope}%
\pgfpathrectangle{\pgfqpoint{0.041670in}{0.041670in}}{\pgfqpoint{2.216660in}{2.216660in}}%
\pgfusepath{clip}%
\pgfsetbuttcap%
\pgfsetroundjoin%
\definecolor{currentfill}{rgb}{0.762373,0.876424,0.137064}%
\pgfsetfillcolor{currentfill}%
\pgfsetlinewidth{0.000000pt}%
\definecolor{currentstroke}{rgb}{0.000000,0.000000,0.000000}%
\pgfsetstrokecolor{currentstroke}%
\pgfsetdash{}{0pt}%
\pgfpathmoveto{\pgfqpoint{1.213107in}{1.529877in}}%
\pgfpathlineto{\pgfqpoint{1.214146in}{1.523922in}}%
\pgfpathlineto{\pgfqpoint{1.215184in}{1.517851in}}%
\pgfpathlineto{\pgfqpoint{1.216223in}{1.511665in}}%
\pgfpathlineto{\pgfqpoint{1.217261in}{1.505365in}}%
\pgfpathlineto{\pgfqpoint{1.212934in}{1.504836in}}%
\pgfpathlineto{\pgfqpoint{1.208572in}{1.504373in}}%
\pgfpathlineto{\pgfqpoint{1.204181in}{1.503975in}}%
\pgfpathlineto{\pgfqpoint{1.199765in}{1.503644in}}%
\pgfpathlineto{\pgfqpoint{1.199213in}{1.509994in}}%
\pgfpathlineto{\pgfqpoint{1.198661in}{1.516229in}}%
\pgfpathlineto{\pgfqpoint{1.198110in}{1.522349in}}%
\pgfpathlineto{\pgfqpoint{1.197558in}{1.528353in}}%
\pgfpathlineto{\pgfqpoint{1.201482in}{1.528646in}}%
\pgfpathlineto{\pgfqpoint{1.205385in}{1.528998in}}%
\pgfpathlineto{\pgfqpoint{1.209261in}{1.529408in}}%
\pgfpathlineto{\pgfqpoint{1.213107in}{1.529877in}}%
\pgfpathclose%
\pgfusepath{fill}%
\end{pgfscope}%
\begin{pgfscope}%
\pgfpathrectangle{\pgfqpoint{0.041670in}{0.041670in}}{\pgfqpoint{2.216660in}{2.216660in}}%
\pgfusepath{clip}%
\pgfsetbuttcap%
\pgfsetroundjoin%
\definecolor{currentfill}{rgb}{0.636902,0.856542,0.216620}%
\pgfsetfillcolor{currentfill}%
\pgfsetlinewidth{0.000000pt}%
\definecolor{currentstroke}{rgb}{0.000000,0.000000,0.000000}%
\pgfsetstrokecolor{currentstroke}%
\pgfsetdash{}{0pt}%
\pgfpathmoveto{\pgfqpoint{1.240177in}{1.482125in}}%
\pgfpathlineto{\pgfqpoint{1.241684in}{1.475354in}}%
\pgfpathlineto{\pgfqpoint{1.243190in}{1.468478in}}%
\pgfpathlineto{\pgfqpoint{1.244697in}{1.461497in}}%
\pgfpathlineto{\pgfqpoint{1.246203in}{1.454414in}}%
\pgfpathlineto{\pgfqpoint{1.241133in}{1.453448in}}%
\pgfpathlineto{\pgfqpoint{1.236000in}{1.452559in}}%
\pgfpathlineto{\pgfqpoint{1.230809in}{1.451748in}}%
\pgfpathlineto{\pgfqpoint{1.225566in}{1.451017in}}%
\pgfpathlineto{\pgfqpoint{1.224528in}{1.458181in}}%
\pgfpathlineto{\pgfqpoint{1.223490in}{1.465242in}}%
\pgfpathlineto{\pgfqpoint{1.222453in}{1.472199in}}%
\pgfpathlineto{\pgfqpoint{1.221415in}{1.479050in}}%
\pgfpathlineto{\pgfqpoint{1.226181in}{1.479712in}}%
\pgfpathlineto{\pgfqpoint{1.230900in}{1.480446in}}%
\pgfpathlineto{\pgfqpoint{1.235567in}{1.481250in}}%
\pgfpathlineto{\pgfqpoint{1.240177in}{1.482125in}}%
\pgfpathclose%
\pgfusepath{fill}%
\end{pgfscope}%
\begin{pgfscope}%
\pgfpathrectangle{\pgfqpoint{0.041670in}{0.041670in}}{\pgfqpoint{2.216660in}{2.216660in}}%
\pgfusepath{clip}%
\pgfsetbuttcap%
\pgfsetroundjoin%
\definecolor{currentfill}{rgb}{0.220124,0.725509,0.466226}%
\pgfsetfillcolor{currentfill}%
\pgfsetlinewidth{0.000000pt}%
\definecolor{currentstroke}{rgb}{0.000000,0.000000,0.000000}%
\pgfsetstrokecolor{currentstroke}%
\pgfsetdash{}{0pt}%
\pgfpathmoveto{\pgfqpoint{1.151746in}{1.285115in}}%
\pgfpathlineto{\pgfqpoint{1.151307in}{1.276133in}}%
\pgfpathlineto{\pgfqpoint{1.150868in}{1.267089in}}%
\pgfpathlineto{\pgfqpoint{1.150429in}{1.257986in}}%
\pgfpathlineto{\pgfqpoint{1.149990in}{1.248825in}}%
\pgfpathlineto{\pgfqpoint{1.141617in}{1.249359in}}%
\pgfpathlineto{\pgfqpoint{1.133285in}{1.250023in}}%
\pgfpathlineto{\pgfqpoint{1.125001in}{1.250818in}}%
\pgfpathlineto{\pgfqpoint{1.116776in}{1.251740in}}%
\pgfpathlineto{\pgfqpoint{1.117701in}{1.260855in}}%
\pgfpathlineto{\pgfqpoint{1.118626in}{1.269911in}}%
\pgfpathlineto{\pgfqpoint{1.119551in}{1.278909in}}%
\pgfpathlineto{\pgfqpoint{1.120477in}{1.287845in}}%
\pgfpathlineto{\pgfqpoint{1.128221in}{1.286980in}}%
\pgfpathlineto{\pgfqpoint{1.136019in}{1.286237in}}%
\pgfpathlineto{\pgfqpoint{1.143864in}{1.285615in}}%
\pgfpathlineto{\pgfqpoint{1.151746in}{1.285115in}}%
\pgfpathclose%
\pgfusepath{fill}%
\end{pgfscope}%
\begin{pgfscope}%
\pgfpathrectangle{\pgfqpoint{0.041670in}{0.041670in}}{\pgfqpoint{2.216660in}{2.216660in}}%
\pgfusepath{clip}%
\pgfsetbuttcap%
\pgfsetroundjoin%
\definecolor{currentfill}{rgb}{0.166383,0.690856,0.496502}%
\pgfsetfillcolor{currentfill}%
\pgfsetlinewidth{0.000000pt}%
\definecolor{currentstroke}{rgb}{0.000000,0.000000,0.000000}%
\pgfsetstrokecolor{currentstroke}%
\pgfsetdash{}{0pt}%
\pgfpathmoveto{\pgfqpoint{1.183709in}{1.248005in}}%
\pgfpathlineto{\pgfqpoint{1.183764in}{1.238776in}}%
\pgfpathlineto{\pgfqpoint{1.183819in}{1.229495in}}%
\pgfpathlineto{\pgfqpoint{1.183873in}{1.220163in}}%
\pgfpathlineto{\pgfqpoint{1.183928in}{1.210783in}}%
\pgfpathlineto{\pgfqpoint{1.174988in}{1.210791in}}%
\pgfpathlineto{\pgfqpoint{1.166053in}{1.210939in}}%
\pgfpathlineto{\pgfqpoint{1.157133in}{1.211228in}}%
\pgfpathlineto{\pgfqpoint{1.148236in}{1.211657in}}%
\pgfpathlineto{\pgfqpoint{1.148675in}{1.221023in}}%
\pgfpathlineto{\pgfqpoint{1.149113in}{1.230342in}}%
\pgfpathlineto{\pgfqpoint{1.149551in}{1.239610in}}%
\pgfpathlineto{\pgfqpoint{1.149990in}{1.248825in}}%
\pgfpathlineto{\pgfqpoint{1.158394in}{1.248423in}}%
\pgfpathlineto{\pgfqpoint{1.166822in}{1.248151in}}%
\pgfpathlineto{\pgfqpoint{1.175263in}{1.248012in}}%
\pgfpathlineto{\pgfqpoint{1.183709in}{1.248005in}}%
\pgfpathclose%
\pgfusepath{fill}%
\end{pgfscope}%
\begin{pgfscope}%
\pgfpathrectangle{\pgfqpoint{0.041670in}{0.041670in}}{\pgfqpoint{2.216660in}{2.216660in}}%
\pgfusepath{clip}%
\pgfsetbuttcap%
\pgfsetroundjoin%
\definecolor{currentfill}{rgb}{0.487026,0.823929,0.312321}%
\pgfsetfillcolor{currentfill}%
\pgfsetlinewidth{0.000000pt}%
\definecolor{currentstroke}{rgb}{0.000000,0.000000,0.000000}%
\pgfsetstrokecolor{currentstroke}%
\pgfsetdash{}{0pt}%
\pgfpathmoveto{\pgfqpoint{1.252225in}{1.425084in}}%
\pgfpathlineto{\pgfqpoint{1.253729in}{1.417510in}}%
\pgfpathlineto{\pgfqpoint{1.255233in}{1.409843in}}%
\pgfpathlineto{\pgfqpoint{1.256737in}{1.402085in}}%
\pgfpathlineto{\pgfqpoint{1.258241in}{1.394238in}}%
\pgfpathlineto{\pgfqpoint{1.252251in}{1.393086in}}%
\pgfpathlineto{\pgfqpoint{1.246187in}{1.392026in}}%
\pgfpathlineto{\pgfqpoint{1.240054in}{1.391059in}}%
\pgfpathlineto{\pgfqpoint{1.233858in}{1.390186in}}%
\pgfpathlineto{\pgfqpoint{1.232822in}{1.398116in}}%
\pgfpathlineto{\pgfqpoint{1.231786in}{1.405957in}}%
\pgfpathlineto{\pgfqpoint{1.230750in}{1.413706in}}%
\pgfpathlineto{\pgfqpoint{1.229714in}{1.421361in}}%
\pgfpathlineto{\pgfqpoint{1.235433in}{1.422163in}}%
\pgfpathlineto{\pgfqpoint{1.241096in}{1.423051in}}%
\pgfpathlineto{\pgfqpoint{1.246695in}{1.424025in}}%
\pgfpathlineto{\pgfqpoint{1.252225in}{1.425084in}}%
\pgfpathclose%
\pgfusepath{fill}%
\end{pgfscope}%
\begin{pgfscope}%
\pgfpathrectangle{\pgfqpoint{0.041670in}{0.041670in}}{\pgfqpoint{2.216660in}{2.216660in}}%
\pgfusepath{clip}%
\pgfsetbuttcap%
\pgfsetroundjoin%
\definecolor{currentfill}{rgb}{0.762373,0.876424,0.137064}%
\pgfsetfillcolor{currentfill}%
\pgfsetlinewidth{0.000000pt}%
\definecolor{currentstroke}{rgb}{0.000000,0.000000,0.000000}%
\pgfsetstrokecolor{currentstroke}%
\pgfsetdash{}{0pt}%
\pgfpathmoveto{\pgfqpoint{1.165855in}{1.528142in}}%
\pgfpathlineto{\pgfqpoint{1.165413in}{1.522131in}}%
\pgfpathlineto{\pgfqpoint{1.164971in}{1.516004in}}%
\pgfpathlineto{\pgfqpoint{1.164530in}{1.509762in}}%
\pgfpathlineto{\pgfqpoint{1.164088in}{1.503405in}}%
\pgfpathlineto{\pgfqpoint{1.159653in}{1.503677in}}%
\pgfpathlineto{\pgfqpoint{1.155240in}{1.504016in}}%
\pgfpathlineto{\pgfqpoint{1.150852in}{1.504421in}}%
\pgfpathlineto{\pgfqpoint{1.146494in}{1.504892in}}%
\pgfpathlineto{\pgfqpoint{1.147425in}{1.511206in}}%
\pgfpathlineto{\pgfqpoint{1.148357in}{1.517405in}}%
\pgfpathlineto{\pgfqpoint{1.149288in}{1.523490in}}%
\pgfpathlineto{\pgfqpoint{1.150220in}{1.529458in}}%
\pgfpathlineto{\pgfqpoint{1.154093in}{1.529041in}}%
\pgfpathlineto{\pgfqpoint{1.157993in}{1.528682in}}%
\pgfpathlineto{\pgfqpoint{1.161915in}{1.528383in}}%
\pgfpathlineto{\pgfqpoint{1.165855in}{1.528142in}}%
\pgfpathclose%
\pgfusepath{fill}%
\end{pgfscope}%
\begin{pgfscope}%
\pgfpathrectangle{\pgfqpoint{0.041670in}{0.041670in}}{\pgfqpoint{2.216660in}{2.216660in}}%
\pgfusepath{clip}%
\pgfsetbuttcap%
\pgfsetroundjoin%
\definecolor{currentfill}{rgb}{0.268510,0.009605,0.335427}%
\pgfsetfillcolor{currentfill}%
\pgfsetlinewidth{0.000000pt}%
\definecolor{currentstroke}{rgb}{0.000000,0.000000,0.000000}%
\pgfsetstrokecolor{currentstroke}%
\pgfsetdash{}{0pt}%
\pgfpathmoveto{\pgfqpoint{1.031222in}{0.559145in}}%
\pgfpathlineto{\pgfqpoint{1.030241in}{0.558409in}}%
\pgfpathlineto{\pgfqpoint{1.029258in}{0.557955in}}%
\pgfpathlineto{\pgfqpoint{1.028271in}{0.557789in}}%
\pgfpathlineto{\pgfqpoint{1.027281in}{0.557916in}}%
\pgfpathlineto{\pgfqpoint{1.007603in}{0.560712in}}%
\pgfpathlineto{\pgfqpoint{0.988119in}{0.563844in}}%
\pgfpathlineto{\pgfqpoint{0.968852in}{0.567307in}}%
\pgfpathlineto{\pgfqpoint{0.949823in}{0.571098in}}%
\pgfpathlineto{\pgfqpoint{0.951312in}{0.570886in}}%
\pgfpathlineto{\pgfqpoint{0.952797in}{0.570966in}}%
\pgfpathlineto{\pgfqpoint{0.954277in}{0.571335in}}%
\pgfpathlineto{\pgfqpoint{0.955753in}{0.571985in}}%
\pgfpathlineto{\pgfqpoint{0.974294in}{0.568293in}}%
\pgfpathlineto{\pgfqpoint{0.993067in}{0.564919in}}%
\pgfpathlineto{\pgfqpoint{1.012050in}{0.561868in}}%
\pgfpathlineto{\pgfqpoint{1.031222in}{0.559145in}}%
\pgfpathclose%
\pgfusepath{fill}%
\end{pgfscope}%
\begin{pgfscope}%
\pgfpathrectangle{\pgfqpoint{0.041670in}{0.041670in}}{\pgfqpoint{2.216660in}{2.216660in}}%
\pgfusepath{clip}%
\pgfsetbuttcap%
\pgfsetroundjoin%
\definecolor{currentfill}{rgb}{0.565498,0.842430,0.262877}%
\pgfsetfillcolor{currentfill}%
\pgfsetlinewidth{0.000000pt}%
\definecolor{currentstroke}{rgb}{0.000000,0.000000,0.000000}%
\pgfsetstrokecolor{currentstroke}%
\pgfsetdash{}{0pt}%
\pgfpathmoveto{\pgfqpoint{1.246203in}{1.454414in}}%
\pgfpathlineto{\pgfqpoint{1.247709in}{1.447229in}}%
\pgfpathlineto{\pgfqpoint{1.249215in}{1.439945in}}%
\pgfpathlineto{\pgfqpoint{1.250720in}{1.432562in}}%
\pgfpathlineto{\pgfqpoint{1.252225in}{1.425084in}}%
\pgfpathlineto{\pgfqpoint{1.246695in}{1.424025in}}%
\pgfpathlineto{\pgfqpoint{1.241096in}{1.423051in}}%
\pgfpathlineto{\pgfqpoint{1.235433in}{1.422163in}}%
\pgfpathlineto{\pgfqpoint{1.229714in}{1.421361in}}%
\pgfpathlineto{\pgfqpoint{1.228677in}{1.428922in}}%
\pgfpathlineto{\pgfqpoint{1.227640in}{1.436385in}}%
\pgfpathlineto{\pgfqpoint{1.226603in}{1.443751in}}%
\pgfpathlineto{\pgfqpoint{1.225566in}{1.451017in}}%
\pgfpathlineto{\pgfqpoint{1.230809in}{1.451748in}}%
\pgfpathlineto{\pgfqpoint{1.236000in}{1.452559in}}%
\pgfpathlineto{\pgfqpoint{1.241133in}{1.453448in}}%
\pgfpathlineto{\pgfqpoint{1.246203in}{1.454414in}}%
\pgfpathclose%
\pgfusepath{fill}%
\end{pgfscope}%
\begin{pgfscope}%
\pgfpathrectangle{\pgfqpoint{0.041670in}{0.041670in}}{\pgfqpoint{2.216660in}{2.216660in}}%
\pgfusepath{clip}%
\pgfsetbuttcap%
\pgfsetroundjoin%
\definecolor{currentfill}{rgb}{0.267004,0.004874,0.329415}%
\pgfsetfillcolor{currentfill}%
\pgfsetlinewidth{0.000000pt}%
\definecolor{currentstroke}{rgb}{0.000000,0.000000,0.000000}%
\pgfsetstrokecolor{currentstroke}%
\pgfsetdash{}{0pt}%
\pgfpathmoveto{\pgfqpoint{1.111224in}{0.557452in}}%
\pgfpathlineto{\pgfqpoint{1.110763in}{0.555587in}}%
\pgfpathlineto{\pgfqpoint{1.110301in}{0.553984in}}%
\pgfpathlineto{\pgfqpoint{1.109837in}{0.552648in}}%
\pgfpathlineto{\pgfqpoint{1.109372in}{0.551584in}}%
\pgfpathlineto{\pgfqpoint{1.089660in}{0.552969in}}%
\pgfpathlineto{\pgfqpoint{1.070049in}{0.554693in}}%
\pgfpathlineto{\pgfqpoint{1.050563in}{0.556752in}}%
\pgfpathlineto{\pgfqpoint{1.031222in}{0.559145in}}%
\pgfpathlineto{\pgfqpoint{1.032201in}{0.560159in}}%
\pgfpathlineto{\pgfqpoint{1.033176in}{0.561444in}}%
\pgfpathlineto{\pgfqpoint{1.034149in}{0.562996in}}%
\pgfpathlineto{\pgfqpoint{1.035120in}{0.564809in}}%
\pgfpathlineto{\pgfqpoint{1.053954in}{0.562481in}}%
\pgfpathlineto{\pgfqpoint{1.072931in}{0.560477in}}%
\pgfpathlineto{\pgfqpoint{1.092028in}{0.558800in}}%
\pgfpathlineto{\pgfqpoint{1.111224in}{0.557452in}}%
\pgfpathclose%
\pgfusepath{fill}%
\end{pgfscope}%
\begin{pgfscope}%
\pgfpathrectangle{\pgfqpoint{0.041670in}{0.041670in}}{\pgfqpoint{2.216660in}{2.216660in}}%
\pgfusepath{clip}%
\pgfsetbuttcap%
\pgfsetroundjoin%
\definecolor{currentfill}{rgb}{0.412913,0.803041,0.357269}%
\pgfsetfillcolor{currentfill}%
\pgfsetlinewidth{0.000000pt}%
\definecolor{currentstroke}{rgb}{0.000000,0.000000,0.000000}%
\pgfsetstrokecolor{currentstroke}%
\pgfsetdash{}{0pt}%
\pgfpathmoveto{\pgfqpoint{1.131606in}{1.389491in}}%
\pgfpathlineto{\pgfqpoint{1.130677in}{1.381459in}}%
\pgfpathlineto{\pgfqpoint{1.129748in}{1.373340in}}%
\pgfpathlineto{\pgfqpoint{1.128820in}{1.365138in}}%
\pgfpathlineto{\pgfqpoint{1.127892in}{1.356852in}}%
\pgfpathlineto{\pgfqpoint{1.121167in}{1.357705in}}%
\pgfpathlineto{\pgfqpoint{1.114503in}{1.358661in}}%
\pgfpathlineto{\pgfqpoint{1.107908in}{1.359718in}}%
\pgfpathlineto{\pgfqpoint{1.101387in}{1.360876in}}%
\pgfpathlineto{\pgfqpoint{1.102787in}{1.369084in}}%
\pgfpathlineto{\pgfqpoint{1.104187in}{1.377211in}}%
\pgfpathlineto{\pgfqpoint{1.105588in}{1.385253in}}%
\pgfpathlineto{\pgfqpoint{1.106989in}{1.393209in}}%
\pgfpathlineto{\pgfqpoint{1.113045in}{1.392139in}}%
\pgfpathlineto{\pgfqpoint{1.119171in}{1.391162in}}%
\pgfpathlineto{\pgfqpoint{1.125360in}{1.390279in}}%
\pgfpathlineto{\pgfqpoint{1.131606in}{1.389491in}}%
\pgfpathclose%
\pgfusepath{fill}%
\end{pgfscope}%
\begin{pgfscope}%
\pgfpathrectangle{\pgfqpoint{0.041670in}{0.041670in}}{\pgfqpoint{2.216660in}{2.216660in}}%
\pgfusepath{clip}%
\pgfsetbuttcap%
\pgfsetroundjoin%
\definecolor{currentfill}{rgb}{0.636902,0.856542,0.216620}%
\pgfsetfillcolor{currentfill}%
\pgfsetlinewidth{0.000000pt}%
\definecolor{currentstroke}{rgb}{0.000000,0.000000,0.000000}%
\pgfsetstrokecolor{currentstroke}%
\pgfsetdash{}{0pt}%
\pgfpathmoveto{\pgfqpoint{1.142768in}{1.478522in}}%
\pgfpathlineto{\pgfqpoint{1.141837in}{1.471657in}}%
\pgfpathlineto{\pgfqpoint{1.140906in}{1.464687in}}%
\pgfpathlineto{\pgfqpoint{1.139975in}{1.457611in}}%
\pgfpathlineto{\pgfqpoint{1.139045in}{1.450433in}}%
\pgfpathlineto{\pgfqpoint{1.133759in}{1.451094in}}%
\pgfpathlineto{\pgfqpoint{1.128521in}{1.451834in}}%
\pgfpathlineto{\pgfqpoint{1.123336in}{1.452654in}}%
\pgfpathlineto{\pgfqpoint{1.118210in}{1.453551in}}%
\pgfpathlineto{\pgfqpoint{1.119614in}{1.460655in}}%
\pgfpathlineto{\pgfqpoint{1.121018in}{1.467656in}}%
\pgfpathlineto{\pgfqpoint{1.122423in}{1.474553in}}%
\pgfpathlineto{\pgfqpoint{1.123827in}{1.481344in}}%
\pgfpathlineto{\pgfqpoint{1.128488in}{1.480532in}}%
\pgfpathlineto{\pgfqpoint{1.133202in}{1.479790in}}%
\pgfpathlineto{\pgfqpoint{1.137963in}{1.479120in}}%
\pgfpathlineto{\pgfqpoint{1.142768in}{1.478522in}}%
\pgfpathclose%
\pgfusepath{fill}%
\end{pgfscope}%
\begin{pgfscope}%
\pgfpathrectangle{\pgfqpoint{0.041670in}{0.041670in}}{\pgfqpoint{2.216660in}{2.216660in}}%
\pgfusepath{clip}%
\pgfsetbuttcap%
\pgfsetroundjoin%
\definecolor{currentfill}{rgb}{0.281477,0.755203,0.432552}%
\pgfsetfillcolor{currentfill}%
\pgfsetlinewidth{0.000000pt}%
\definecolor{currentstroke}{rgb}{0.000000,0.000000,0.000000}%
\pgfsetstrokecolor{currentstroke}%
\pgfsetdash{}{0pt}%
\pgfpathmoveto{\pgfqpoint{1.242134in}{1.323738in}}%
\pgfpathlineto{\pgfqpoint{1.243167in}{1.315085in}}%
\pgfpathlineto{\pgfqpoint{1.244200in}{1.306362in}}%
\pgfpathlineto{\pgfqpoint{1.245232in}{1.297571in}}%
\pgfpathlineto{\pgfqpoint{1.246264in}{1.288714in}}%
\pgfpathlineto{\pgfqpoint{1.238576in}{1.287743in}}%
\pgfpathlineto{\pgfqpoint{1.230825in}{1.286892in}}%
\pgfpathlineto{\pgfqpoint{1.223021in}{1.286162in}}%
\pgfpathlineto{\pgfqpoint{1.215172in}{1.285553in}}%
\pgfpathlineto{\pgfqpoint{1.214624in}{1.294464in}}%
\pgfpathlineto{\pgfqpoint{1.214075in}{1.303308in}}%
\pgfpathlineto{\pgfqpoint{1.213526in}{1.312084in}}%
\pgfpathlineto{\pgfqpoint{1.212977in}{1.320790in}}%
\pgfpathlineto{\pgfqpoint{1.220338in}{1.321358in}}%
\pgfpathlineto{\pgfqpoint{1.227656in}{1.322039in}}%
\pgfpathlineto{\pgfqpoint{1.234924in}{1.322832in}}%
\pgfpathlineto{\pgfqpoint{1.242134in}{1.323738in}}%
\pgfpathclose%
\pgfusepath{fill}%
\end{pgfscope}%
\begin{pgfscope}%
\pgfpathrectangle{\pgfqpoint{0.041670in}{0.041670in}}{\pgfqpoint{2.216660in}{2.216660in}}%
\pgfusepath{clip}%
\pgfsetbuttcap%
\pgfsetroundjoin%
\definecolor{currentfill}{rgb}{0.487026,0.823929,0.312321}%
\pgfsetfillcolor{currentfill}%
\pgfsetlinewidth{0.000000pt}%
\definecolor{currentstroke}{rgb}{0.000000,0.000000,0.000000}%
\pgfsetstrokecolor{currentstroke}%
\pgfsetdash{}{0pt}%
\pgfpathmoveto{\pgfqpoint{1.135324in}{1.420722in}}%
\pgfpathlineto{\pgfqpoint{1.134394in}{1.413052in}}%
\pgfpathlineto{\pgfqpoint{1.133464in}{1.405289in}}%
\pgfpathlineto{\pgfqpoint{1.132535in}{1.397435in}}%
\pgfpathlineto{\pgfqpoint{1.131606in}{1.389491in}}%
\pgfpathlineto{\pgfqpoint{1.125360in}{1.390279in}}%
\pgfpathlineto{\pgfqpoint{1.119171in}{1.391162in}}%
\pgfpathlineto{\pgfqpoint{1.113045in}{1.392139in}}%
\pgfpathlineto{\pgfqpoint{1.106989in}{1.393209in}}%
\pgfpathlineto{\pgfqpoint{1.108391in}{1.401078in}}%
\pgfpathlineto{\pgfqpoint{1.109792in}{1.408857in}}%
\pgfpathlineto{\pgfqpoint{1.111194in}{1.416544in}}%
\pgfpathlineto{\pgfqpoint{1.112597in}{1.424138in}}%
\pgfpathlineto{\pgfqpoint{1.118189in}{1.423155in}}%
\pgfpathlineto{\pgfqpoint{1.123844in}{1.422257in}}%
\pgfpathlineto{\pgfqpoint{1.129558in}{1.421446in}}%
\pgfpathlineto{\pgfqpoint{1.135324in}{1.420722in}}%
\pgfpathclose%
\pgfusepath{fill}%
\end{pgfscope}%
\begin{pgfscope}%
\pgfpathrectangle{\pgfqpoint{0.041670in}{0.041670in}}{\pgfqpoint{2.216660in}{2.216660in}}%
\pgfusepath{clip}%
\pgfsetbuttcap%
\pgfsetroundjoin%
\definecolor{currentfill}{rgb}{0.762373,0.876424,0.137064}%
\pgfsetfillcolor{currentfill}%
\pgfsetlinewidth{0.000000pt}%
\definecolor{currentstroke}{rgb}{0.000000,0.000000,0.000000}%
\pgfsetstrokecolor{currentstroke}%
\pgfsetdash{}{0pt}%
\pgfpathmoveto{\pgfqpoint{1.197558in}{1.528353in}}%
\pgfpathlineto{\pgfqpoint{1.198110in}{1.522349in}}%
\pgfpathlineto{\pgfqpoint{1.198661in}{1.516229in}}%
\pgfpathlineto{\pgfqpoint{1.199213in}{1.509994in}}%
\pgfpathlineto{\pgfqpoint{1.199765in}{1.503644in}}%
\pgfpathlineto{\pgfqpoint{1.195328in}{1.503379in}}%
\pgfpathlineto{\pgfqpoint{1.190876in}{1.503181in}}%
\pgfpathlineto{\pgfqpoint{1.186413in}{1.503050in}}%
\pgfpathlineto{\pgfqpoint{1.181942in}{1.502987in}}%
\pgfpathlineto{\pgfqpoint{1.181887in}{1.509356in}}%
\pgfpathlineto{\pgfqpoint{1.181832in}{1.515610in}}%
\pgfpathlineto{\pgfqpoint{1.181776in}{1.521749in}}%
\pgfpathlineto{\pgfqpoint{1.181721in}{1.527771in}}%
\pgfpathlineto{\pgfqpoint{1.185693in}{1.527828in}}%
\pgfpathlineto{\pgfqpoint{1.189659in}{1.527943in}}%
\pgfpathlineto{\pgfqpoint{1.193616in}{1.528118in}}%
\pgfpathlineto{\pgfqpoint{1.197558in}{1.528353in}}%
\pgfpathclose%
\pgfusepath{fill}%
\end{pgfscope}%
\begin{pgfscope}%
\pgfpathrectangle{\pgfqpoint{0.041670in}{0.041670in}}{\pgfqpoint{2.216660in}{2.216660in}}%
\pgfusepath{clip}%
\pgfsetbuttcap%
\pgfsetroundjoin%
\definecolor{currentfill}{rgb}{0.565498,0.842430,0.262877}%
\pgfsetfillcolor{currentfill}%
\pgfsetlinewidth{0.000000pt}%
\definecolor{currentstroke}{rgb}{0.000000,0.000000,0.000000}%
\pgfsetstrokecolor{currentstroke}%
\pgfsetdash{}{0pt}%
\pgfpathmoveto{\pgfqpoint{1.139045in}{1.450433in}}%
\pgfpathlineto{\pgfqpoint{1.138114in}{1.443154in}}%
\pgfpathlineto{\pgfqpoint{1.137184in}{1.435774in}}%
\pgfpathlineto{\pgfqpoint{1.136254in}{1.428296in}}%
\pgfpathlineto{\pgfqpoint{1.135324in}{1.420722in}}%
\pgfpathlineto{\pgfqpoint{1.129558in}{1.421446in}}%
\pgfpathlineto{\pgfqpoint{1.123844in}{1.422257in}}%
\pgfpathlineto{\pgfqpoint{1.118189in}{1.423155in}}%
\pgfpathlineto{\pgfqpoint{1.112597in}{1.424138in}}%
\pgfpathlineto{\pgfqpoint{1.114000in}{1.431638in}}%
\pgfpathlineto{\pgfqpoint{1.115403in}{1.439041in}}%
\pgfpathlineto{\pgfqpoint{1.116806in}{1.446346in}}%
\pgfpathlineto{\pgfqpoint{1.118210in}{1.453551in}}%
\pgfpathlineto{\pgfqpoint{1.123336in}{1.452654in}}%
\pgfpathlineto{\pgfqpoint{1.128521in}{1.451834in}}%
\pgfpathlineto{\pgfqpoint{1.133759in}{1.451094in}}%
\pgfpathlineto{\pgfqpoint{1.139045in}{1.450433in}}%
\pgfpathclose%
\pgfusepath{fill}%
\end{pgfscope}%
\begin{pgfscope}%
\pgfpathrectangle{\pgfqpoint{0.041670in}{0.041670in}}{\pgfqpoint{2.216660in}{2.216660in}}%
\pgfusepath{clip}%
\pgfsetbuttcap%
\pgfsetroundjoin%
\definecolor{currentfill}{rgb}{0.762373,0.876424,0.137064}%
\pgfsetfillcolor{currentfill}%
\pgfsetlinewidth{0.000000pt}%
\definecolor{currentstroke}{rgb}{0.000000,0.000000,0.000000}%
\pgfsetstrokecolor{currentstroke}%
\pgfsetdash{}{0pt}%
\pgfpathmoveto{\pgfqpoint{1.181721in}{1.527771in}}%
\pgfpathlineto{\pgfqpoint{1.181776in}{1.521749in}}%
\pgfpathlineto{\pgfqpoint{1.181832in}{1.515610in}}%
\pgfpathlineto{\pgfqpoint{1.181887in}{1.509356in}}%
\pgfpathlineto{\pgfqpoint{1.181942in}{1.502987in}}%
\pgfpathlineto{\pgfqpoint{1.177470in}{1.502991in}}%
\pgfpathlineto{\pgfqpoint{1.173001in}{1.503062in}}%
\pgfpathlineto{\pgfqpoint{1.168538in}{1.503200in}}%
\pgfpathlineto{\pgfqpoint{1.164088in}{1.503405in}}%
\pgfpathlineto{\pgfqpoint{1.164530in}{1.509762in}}%
\pgfpathlineto{\pgfqpoint{1.164971in}{1.516004in}}%
\pgfpathlineto{\pgfqpoint{1.165413in}{1.522131in}}%
\pgfpathlineto{\pgfqpoint{1.165855in}{1.528142in}}%
\pgfpathlineto{\pgfqpoint{1.169810in}{1.527960in}}%
\pgfpathlineto{\pgfqpoint{1.173775in}{1.527837in}}%
\pgfpathlineto{\pgfqpoint{1.177747in}{1.527775in}}%
\pgfpathlineto{\pgfqpoint{1.181721in}{1.527771in}}%
\pgfpathclose%
\pgfusepath{fill}%
\end{pgfscope}%
\begin{pgfscope}%
\pgfpathrectangle{\pgfqpoint{0.041670in}{0.041670in}}{\pgfqpoint{2.216660in}{2.216660in}}%
\pgfusepath{clip}%
\pgfsetbuttcap%
\pgfsetroundjoin%
\definecolor{currentfill}{rgb}{0.281477,0.755203,0.432552}%
\pgfsetfillcolor{currentfill}%
\pgfsetlinewidth{0.000000pt}%
\definecolor{currentstroke}{rgb}{0.000000,0.000000,0.000000}%
\pgfsetstrokecolor{currentstroke}%
\pgfsetdash{}{0pt}%
\pgfpathmoveto{\pgfqpoint{1.153504in}{1.320381in}}%
\pgfpathlineto{\pgfqpoint{1.153064in}{1.311668in}}%
\pgfpathlineto{\pgfqpoint{1.152625in}{1.302885in}}%
\pgfpathlineto{\pgfqpoint{1.152185in}{1.294033in}}%
\pgfpathlineto{\pgfqpoint{1.151746in}{1.285115in}}%
\pgfpathlineto{\pgfqpoint{1.143864in}{1.285615in}}%
\pgfpathlineto{\pgfqpoint{1.136019in}{1.286237in}}%
\pgfpathlineto{\pgfqpoint{1.128221in}{1.286980in}}%
\pgfpathlineto{\pgfqpoint{1.120477in}{1.287845in}}%
\pgfpathlineto{\pgfqpoint{1.121403in}{1.296717in}}%
\pgfpathlineto{\pgfqpoint{1.122329in}{1.305523in}}%
\pgfpathlineto{\pgfqpoint{1.123255in}{1.314260in}}%
\pgfpathlineto{\pgfqpoint{1.124182in}{1.322928in}}%
\pgfpathlineto{\pgfqpoint{1.131444in}{1.322121in}}%
\pgfpathlineto{\pgfqpoint{1.138757in}{1.321428in}}%
\pgfpathlineto{\pgfqpoint{1.146113in}{1.320847in}}%
\pgfpathlineto{\pgfqpoint{1.153504in}{1.320381in}}%
\pgfpathclose%
\pgfusepath{fill}%
\end{pgfscope}%
\begin{pgfscope}%
\pgfpathrectangle{\pgfqpoint{0.041670in}{0.041670in}}{\pgfqpoint{2.216660in}{2.216660in}}%
\pgfusepath{clip}%
\pgfsetbuttcap%
\pgfsetroundjoin%
\definecolor{currentfill}{rgb}{0.699415,0.867117,0.175971}%
\pgfsetfillcolor{currentfill}%
\pgfsetlinewidth{0.000000pt}%
\definecolor{currentstroke}{rgb}{0.000000,0.000000,0.000000}%
\pgfsetstrokecolor{currentstroke}%
\pgfsetdash{}{0pt}%
\pgfpathmoveto{\pgfqpoint{1.217261in}{1.505365in}}%
\pgfpathlineto{\pgfqpoint{1.218300in}{1.498952in}}%
\pgfpathlineto{\pgfqpoint{1.219338in}{1.492428in}}%
\pgfpathlineto{\pgfqpoint{1.220376in}{1.485793in}}%
\pgfpathlineto{\pgfqpoint{1.221415in}{1.479050in}}%
\pgfpathlineto{\pgfqpoint{1.216605in}{1.478460in}}%
\pgfpathlineto{\pgfqpoint{1.211758in}{1.477943in}}%
\pgfpathlineto{\pgfqpoint{1.206878in}{1.477499in}}%
\pgfpathlineto{\pgfqpoint{1.201970in}{1.477129in}}%
\pgfpathlineto{\pgfqpoint{1.201419in}{1.483923in}}%
\pgfpathlineto{\pgfqpoint{1.200868in}{1.490607in}}%
\pgfpathlineto{\pgfqpoint{1.200316in}{1.497181in}}%
\pgfpathlineto{\pgfqpoint{1.199765in}{1.503644in}}%
\pgfpathlineto{\pgfqpoint{1.204181in}{1.503975in}}%
\pgfpathlineto{\pgfqpoint{1.208572in}{1.504373in}}%
\pgfpathlineto{\pgfqpoint{1.212934in}{1.504836in}}%
\pgfpathlineto{\pgfqpoint{1.217261in}{1.505365in}}%
\pgfpathclose%
\pgfusepath{fill}%
\end{pgfscope}%
\begin{pgfscope}%
\pgfpathrectangle{\pgfqpoint{0.041670in}{0.041670in}}{\pgfqpoint{2.216660in}{2.216660in}}%
\pgfusepath{clip}%
\pgfsetbuttcap%
\pgfsetroundjoin%
\definecolor{currentfill}{rgb}{0.220124,0.725509,0.466226}%
\pgfsetfillcolor{currentfill}%
\pgfsetlinewidth{0.000000pt}%
\definecolor{currentstroke}{rgb}{0.000000,0.000000,0.000000}%
\pgfsetstrokecolor{currentstroke}%
\pgfsetdash{}{0pt}%
\pgfpathmoveto{\pgfqpoint{1.215172in}{1.285553in}}%
\pgfpathlineto{\pgfqpoint{1.215720in}{1.276579in}}%
\pgfpathlineto{\pgfqpoint{1.216268in}{1.267542in}}%
\pgfpathlineto{\pgfqpoint{1.216816in}{1.258447in}}%
\pgfpathlineto{\pgfqpoint{1.217364in}{1.249293in}}%
\pgfpathlineto{\pgfqpoint{1.208987in}{1.248774in}}%
\pgfpathlineto{\pgfqpoint{1.200580in}{1.248386in}}%
\pgfpathlineto{\pgfqpoint{1.192151in}{1.248129in}}%
\pgfpathlineto{\pgfqpoint{1.183709in}{1.248005in}}%
\pgfpathlineto{\pgfqpoint{1.183654in}{1.257178in}}%
\pgfpathlineto{\pgfqpoint{1.183599in}{1.266295in}}%
\pgfpathlineto{\pgfqpoint{1.183544in}{1.275352in}}%
\pgfpathlineto{\pgfqpoint{1.183489in}{1.284347in}}%
\pgfpathlineto{\pgfqpoint{1.191436in}{1.284463in}}%
\pgfpathlineto{\pgfqpoint{1.199371in}{1.284704in}}%
\pgfpathlineto{\pgfqpoint{1.207286in}{1.285067in}}%
\pgfpathlineto{\pgfqpoint{1.215172in}{1.285553in}}%
\pgfpathclose%
\pgfusepath{fill}%
\end{pgfscope}%
\begin{pgfscope}%
\pgfpathrectangle{\pgfqpoint{0.041670in}{0.041670in}}{\pgfqpoint{2.216660in}{2.216660in}}%
\pgfusepath{clip}%
\pgfsetbuttcap%
\pgfsetroundjoin%
\definecolor{currentfill}{rgb}{0.220124,0.725509,0.466226}%
\pgfsetfillcolor{currentfill}%
\pgfsetlinewidth{0.000000pt}%
\definecolor{currentstroke}{rgb}{0.000000,0.000000,0.000000}%
\pgfsetstrokecolor{currentstroke}%
\pgfsetdash{}{0pt}%
\pgfpathmoveto{\pgfqpoint{1.183489in}{1.284347in}}%
\pgfpathlineto{\pgfqpoint{1.183544in}{1.275352in}}%
\pgfpathlineto{\pgfqpoint{1.183599in}{1.266295in}}%
\pgfpathlineto{\pgfqpoint{1.183654in}{1.257178in}}%
\pgfpathlineto{\pgfqpoint{1.183709in}{1.248005in}}%
\pgfpathlineto{\pgfqpoint{1.175263in}{1.248012in}}%
\pgfpathlineto{\pgfqpoint{1.166822in}{1.248151in}}%
\pgfpathlineto{\pgfqpoint{1.158394in}{1.248423in}}%
\pgfpathlineto{\pgfqpoint{1.149990in}{1.248825in}}%
\pgfpathlineto{\pgfqpoint{1.150429in}{1.257986in}}%
\pgfpathlineto{\pgfqpoint{1.150868in}{1.267089in}}%
\pgfpathlineto{\pgfqpoint{1.151307in}{1.276133in}}%
\pgfpathlineto{\pgfqpoint{1.151746in}{1.285115in}}%
\pgfpathlineto{\pgfqpoint{1.159658in}{1.284738in}}%
\pgfpathlineto{\pgfqpoint{1.167591in}{1.284484in}}%
\pgfpathlineto{\pgfqpoint{1.175538in}{1.284353in}}%
\pgfpathlineto{\pgfqpoint{1.183489in}{1.284347in}}%
\pgfpathclose%
\pgfusepath{fill}%
\end{pgfscope}%
\begin{pgfscope}%
\pgfpathrectangle{\pgfqpoint{0.041670in}{0.041670in}}{\pgfqpoint{2.216660in}{2.216660in}}%
\pgfusepath{clip}%
\pgfsetbuttcap%
\pgfsetroundjoin%
\definecolor{currentfill}{rgb}{0.699415,0.867117,0.175971}%
\pgfsetfillcolor{currentfill}%
\pgfsetlinewidth{0.000000pt}%
\definecolor{currentstroke}{rgb}{0.000000,0.000000,0.000000}%
\pgfsetstrokecolor{currentstroke}%
\pgfsetdash{}{0pt}%
\pgfpathmoveto{\pgfqpoint{1.164088in}{1.503405in}}%
\pgfpathlineto{\pgfqpoint{1.163646in}{1.496936in}}%
\pgfpathlineto{\pgfqpoint{1.163204in}{1.490355in}}%
\pgfpathlineto{\pgfqpoint{1.162763in}{1.483663in}}%
\pgfpathlineto{\pgfqpoint{1.162321in}{1.476863in}}%
\pgfpathlineto{\pgfqpoint{1.157393in}{1.477167in}}%
\pgfpathlineto{\pgfqpoint{1.152488in}{1.477545in}}%
\pgfpathlineto{\pgfqpoint{1.147611in}{1.477996in}}%
\pgfpathlineto{\pgfqpoint{1.142768in}{1.478522in}}%
\pgfpathlineto{\pgfqpoint{1.143699in}{1.485279in}}%
\pgfpathlineto{\pgfqpoint{1.144631in}{1.491927in}}%
\pgfpathlineto{\pgfqpoint{1.145562in}{1.498465in}}%
\pgfpathlineto{\pgfqpoint{1.146494in}{1.504892in}}%
\pgfpathlineto{\pgfqpoint{1.150852in}{1.504421in}}%
\pgfpathlineto{\pgfqpoint{1.155240in}{1.504016in}}%
\pgfpathlineto{\pgfqpoint{1.159653in}{1.503677in}}%
\pgfpathlineto{\pgfqpoint{1.164088in}{1.503405in}}%
\pgfpathclose%
\pgfusepath{fill}%
\end{pgfscope}%
\begin{pgfscope}%
\pgfpathrectangle{\pgfqpoint{0.041670in}{0.041670in}}{\pgfqpoint{2.216660in}{2.216660in}}%
\pgfusepath{clip}%
\pgfsetbuttcap%
\pgfsetroundjoin%
\definecolor{currentfill}{rgb}{0.282327,0.094955,0.417331}%
\pgfsetfillcolor{currentfill}%
\pgfsetlinewidth{0.000000pt}%
\definecolor{currentstroke}{rgb}{0.000000,0.000000,0.000000}%
\pgfsetstrokecolor{currentstroke}%
\pgfsetdash{}{0pt}%
\pgfpathmoveto{\pgfqpoint{0.860607in}{0.603288in}}%
\pgfpathlineto{\pgfqpoint{0.858588in}{0.606509in}}%
\pgfpathlineto{\pgfqpoint{0.856562in}{0.610081in}}%
\pgfpathlineto{\pgfqpoint{0.854528in}{0.614010in}}%
\pgfpathlineto{\pgfqpoint{0.852486in}{0.618302in}}%
\pgfpathlineto{\pgfqpoint{0.833588in}{0.624051in}}%
\pgfpathlineto{\pgfqpoint{0.815080in}{0.630115in}}%
\pgfpathlineto{\pgfqpoint{0.796981in}{0.636485in}}%
\pgfpathlineto{\pgfqpoint{0.779312in}{0.643154in}}%
\pgfpathlineto{\pgfqpoint{0.781802in}{0.638721in}}%
\pgfpathlineto{\pgfqpoint{0.784282in}{0.634651in}}%
\pgfpathlineto{\pgfqpoint{0.786753in}{0.630937in}}%
\pgfpathlineto{\pgfqpoint{0.789215in}{0.627574in}}%
\pgfpathlineto{\pgfqpoint{0.806456in}{0.621057in}}%
\pgfpathlineto{\pgfqpoint{0.824114in}{0.614831in}}%
\pgfpathlineto{\pgfqpoint{0.842171in}{0.608906in}}%
\pgfpathlineto{\pgfqpoint{0.860607in}{0.603288in}}%
\pgfpathclose%
\pgfusepath{fill}%
\end{pgfscope}%
\begin{pgfscope}%
\pgfpathrectangle{\pgfqpoint{0.041670in}{0.041670in}}{\pgfqpoint{2.216660in}{2.216660in}}%
\pgfusepath{clip}%
\pgfsetbuttcap%
\pgfsetroundjoin%
\definecolor{currentfill}{rgb}{0.344074,0.780029,0.397381}%
\pgfsetfillcolor{currentfill}%
\pgfsetlinewidth{0.000000pt}%
\definecolor{currentstroke}{rgb}{0.000000,0.000000,0.000000}%
\pgfsetstrokecolor{currentstroke}%
\pgfsetdash{}{0pt}%
\pgfpathmoveto{\pgfqpoint{1.237998in}{1.357605in}}%
\pgfpathlineto{\pgfqpoint{1.239033in}{1.349254in}}%
\pgfpathlineto{\pgfqpoint{1.240067in}{1.340825in}}%
\pgfpathlineto{\pgfqpoint{1.241101in}{1.332319in}}%
\pgfpathlineto{\pgfqpoint{1.242134in}{1.323738in}}%
\pgfpathlineto{\pgfqpoint{1.234924in}{1.322832in}}%
\pgfpathlineto{\pgfqpoint{1.227656in}{1.322039in}}%
\pgfpathlineto{\pgfqpoint{1.220338in}{1.321358in}}%
\pgfpathlineto{\pgfqpoint{1.212977in}{1.320790in}}%
\pgfpathlineto{\pgfqpoint{1.212428in}{1.329423in}}%
\pgfpathlineto{\pgfqpoint{1.211879in}{1.337982in}}%
\pgfpathlineto{\pgfqpoint{1.211330in}{1.346464in}}%
\pgfpathlineto{\pgfqpoint{1.210780in}{1.354867in}}%
\pgfpathlineto{\pgfqpoint{1.217651in}{1.355394in}}%
\pgfpathlineto{\pgfqpoint{1.224482in}{1.356027in}}%
\pgfpathlineto{\pgfqpoint{1.231267in}{1.356764in}}%
\pgfpathlineto{\pgfqpoint{1.237998in}{1.357605in}}%
\pgfpathclose%
\pgfusepath{fill}%
\end{pgfscope}%
\begin{pgfscope}%
\pgfpathrectangle{\pgfqpoint{0.041670in}{0.041670in}}{\pgfqpoint{2.216660in}{2.216660in}}%
\pgfusepath{clip}%
\pgfsetbuttcap%
\pgfsetroundjoin%
\definecolor{currentfill}{rgb}{0.276194,0.190074,0.493001}%
\pgfsetfillcolor{currentfill}%
\pgfsetlinewidth{0.000000pt}%
\definecolor{currentstroke}{rgb}{0.000000,0.000000,0.000000}%
\pgfsetstrokecolor{currentstroke}%
\pgfsetdash{}{0pt}%
\pgfpathmoveto{\pgfqpoint{1.672216in}{0.702073in}}%
\pgfpathlineto{\pgfqpoint{1.675247in}{0.708616in}}%
\pgfpathlineto{\pgfqpoint{1.678293in}{0.715560in}}%
\pgfpathlineto{\pgfqpoint{1.681351in}{0.722913in}}%
\pgfpathlineto{\pgfqpoint{1.684424in}{0.730680in}}%
\pgfpathlineto{\pgfqpoint{1.668361in}{0.722329in}}%
\pgfpathlineto{\pgfqpoint{1.651757in}{0.714243in}}%
\pgfpathlineto{\pgfqpoint{1.634626in}{0.706433in}}%
\pgfpathlineto{\pgfqpoint{1.616989in}{0.698908in}}%
\pgfpathlineto{\pgfqpoint{1.614315in}{0.691301in}}%
\pgfpathlineto{\pgfqpoint{1.611654in}{0.684111in}}%
\pgfpathlineto{\pgfqpoint{1.609005in}{0.677330in}}%
\pgfpathlineto{\pgfqpoint{1.606368in}{0.670952in}}%
\pgfpathlineto{\pgfqpoint{1.623588in}{0.678322in}}%
\pgfpathlineto{\pgfqpoint{1.640314in}{0.685972in}}%
\pgfpathlineto{\pgfqpoint{1.656529in}{0.693892in}}%
\pgfpathlineto{\pgfqpoint{1.672216in}{0.702073in}}%
\pgfpathclose%
\pgfusepath{fill}%
\end{pgfscope}%
\begin{pgfscope}%
\pgfpathrectangle{\pgfqpoint{0.041670in}{0.041670in}}{\pgfqpoint{2.216660in}{2.216660in}}%
\pgfusepath{clip}%
\pgfsetbuttcap%
\pgfsetroundjoin%
\definecolor{currentfill}{rgb}{0.636902,0.856542,0.216620}%
\pgfsetfillcolor{currentfill}%
\pgfsetlinewidth{0.000000pt}%
\definecolor{currentstroke}{rgb}{0.000000,0.000000,0.000000}%
\pgfsetstrokecolor{currentstroke}%
\pgfsetdash{}{0pt}%
\pgfpathmoveto{\pgfqpoint{1.221415in}{1.479050in}}%
\pgfpathlineto{\pgfqpoint{1.222453in}{1.472199in}}%
\pgfpathlineto{\pgfqpoint{1.223490in}{1.465242in}}%
\pgfpathlineto{\pgfqpoint{1.224528in}{1.458181in}}%
\pgfpathlineto{\pgfqpoint{1.225566in}{1.451017in}}%
\pgfpathlineto{\pgfqpoint{1.220275in}{1.450365in}}%
\pgfpathlineto{\pgfqpoint{1.214943in}{1.449794in}}%
\pgfpathlineto{\pgfqpoint{1.209574in}{1.449303in}}%
\pgfpathlineto{\pgfqpoint{1.204175in}{1.448895in}}%
\pgfpathlineto{\pgfqpoint{1.203624in}{1.456110in}}%
\pgfpathlineto{\pgfqpoint{1.203073in}{1.463221in}}%
\pgfpathlineto{\pgfqpoint{1.202522in}{1.470228in}}%
\pgfpathlineto{\pgfqpoint{1.201970in}{1.477129in}}%
\pgfpathlineto{\pgfqpoint{1.206878in}{1.477499in}}%
\pgfpathlineto{\pgfqpoint{1.211758in}{1.477943in}}%
\pgfpathlineto{\pgfqpoint{1.216605in}{1.478460in}}%
\pgfpathlineto{\pgfqpoint{1.221415in}{1.479050in}}%
\pgfpathclose%
\pgfusepath{fill}%
\end{pgfscope}%
\begin{pgfscope}%
\pgfpathrectangle{\pgfqpoint{0.041670in}{0.041670in}}{\pgfqpoint{2.216660in}{2.216660in}}%
\pgfusepath{clip}%
\pgfsetbuttcap%
\pgfsetroundjoin%
\definecolor{currentfill}{rgb}{0.344074,0.780029,0.397381}%
\pgfsetfillcolor{currentfill}%
\pgfsetlinewidth{0.000000pt}%
\definecolor{currentstroke}{rgb}{0.000000,0.000000,0.000000}%
\pgfsetstrokecolor{currentstroke}%
\pgfsetdash{}{0pt}%
\pgfpathmoveto{\pgfqpoint{1.155264in}{1.354487in}}%
\pgfpathlineto{\pgfqpoint{1.154824in}{1.346077in}}%
\pgfpathlineto{\pgfqpoint{1.154384in}{1.337587in}}%
\pgfpathlineto{\pgfqpoint{1.153944in}{1.329022in}}%
\pgfpathlineto{\pgfqpoint{1.153504in}{1.320381in}}%
\pgfpathlineto{\pgfqpoint{1.146113in}{1.320847in}}%
\pgfpathlineto{\pgfqpoint{1.138757in}{1.321428in}}%
\pgfpathlineto{\pgfqpoint{1.131444in}{1.322121in}}%
\pgfpathlineto{\pgfqpoint{1.124182in}{1.322928in}}%
\pgfpathlineto{\pgfqpoint{1.125109in}{1.331523in}}%
\pgfpathlineto{\pgfqpoint{1.126036in}{1.340043in}}%
\pgfpathlineto{\pgfqpoint{1.126964in}{1.348487in}}%
\pgfpathlineto{\pgfqpoint{1.127892in}{1.356852in}}%
\pgfpathlineto{\pgfqpoint{1.134671in}{1.356104in}}%
\pgfpathlineto{\pgfqpoint{1.141498in}{1.355459in}}%
\pgfpathlineto{\pgfqpoint{1.148364in}{1.354920in}}%
\pgfpathlineto{\pgfqpoint{1.155264in}{1.354487in}}%
\pgfpathclose%
\pgfusepath{fill}%
\end{pgfscope}%
\begin{pgfscope}%
\pgfpathrectangle{\pgfqpoint{0.041670in}{0.041670in}}{\pgfqpoint{2.216660in}{2.216660in}}%
\pgfusepath{clip}%
\pgfsetbuttcap%
\pgfsetroundjoin%
\definecolor{currentfill}{rgb}{0.699415,0.867117,0.175971}%
\pgfsetfillcolor{currentfill}%
\pgfsetlinewidth{0.000000pt}%
\definecolor{currentstroke}{rgb}{0.000000,0.000000,0.000000}%
\pgfsetstrokecolor{currentstroke}%
\pgfsetdash{}{0pt}%
\pgfpathmoveto{\pgfqpoint{1.199765in}{1.503644in}}%
\pgfpathlineto{\pgfqpoint{1.200316in}{1.497181in}}%
\pgfpathlineto{\pgfqpoint{1.200868in}{1.490607in}}%
\pgfpathlineto{\pgfqpoint{1.201419in}{1.483923in}}%
\pgfpathlineto{\pgfqpoint{1.201970in}{1.477129in}}%
\pgfpathlineto{\pgfqpoint{1.197040in}{1.476834in}}%
\pgfpathlineto{\pgfqpoint{1.192092in}{1.476613in}}%
\pgfpathlineto{\pgfqpoint{1.187132in}{1.476467in}}%
\pgfpathlineto{\pgfqpoint{1.182164in}{1.476396in}}%
\pgfpathlineto{\pgfqpoint{1.182109in}{1.483209in}}%
\pgfpathlineto{\pgfqpoint{1.182053in}{1.489912in}}%
\pgfpathlineto{\pgfqpoint{1.181998in}{1.496505in}}%
\pgfpathlineto{\pgfqpoint{1.181942in}{1.502987in}}%
\pgfpathlineto{\pgfqpoint{1.186413in}{1.503050in}}%
\pgfpathlineto{\pgfqpoint{1.190876in}{1.503181in}}%
\pgfpathlineto{\pgfqpoint{1.195328in}{1.503379in}}%
\pgfpathlineto{\pgfqpoint{1.199765in}{1.503644in}}%
\pgfpathclose%
\pgfusepath{fill}%
\end{pgfscope}%
\begin{pgfscope}%
\pgfpathrectangle{\pgfqpoint{0.041670in}{0.041670in}}{\pgfqpoint{2.216660in}{2.216660in}}%
\pgfusepath{clip}%
\pgfsetbuttcap%
\pgfsetroundjoin%
\definecolor{currentfill}{rgb}{0.412913,0.803041,0.357269}%
\pgfsetfillcolor{currentfill}%
\pgfsetlinewidth{0.000000pt}%
\definecolor{currentstroke}{rgb}{0.000000,0.000000,0.000000}%
\pgfsetstrokecolor{currentstroke}%
\pgfsetdash{}{0pt}%
\pgfpathmoveto{\pgfqpoint{1.233858in}{1.390186in}}%
\pgfpathlineto{\pgfqpoint{1.234894in}{1.382168in}}%
\pgfpathlineto{\pgfqpoint{1.235929in}{1.374064in}}%
\pgfpathlineto{\pgfqpoint{1.236964in}{1.365876in}}%
\pgfpathlineto{\pgfqpoint{1.237998in}{1.357605in}}%
\pgfpathlineto{\pgfqpoint{1.231267in}{1.356764in}}%
\pgfpathlineto{\pgfqpoint{1.224482in}{1.356027in}}%
\pgfpathlineto{\pgfqpoint{1.217651in}{1.355394in}}%
\pgfpathlineto{\pgfqpoint{1.210780in}{1.354867in}}%
\pgfpathlineto{\pgfqpoint{1.210231in}{1.363190in}}%
\pgfpathlineto{\pgfqpoint{1.209681in}{1.371430in}}%
\pgfpathlineto{\pgfqpoint{1.209131in}{1.379586in}}%
\pgfpathlineto{\pgfqpoint{1.208581in}{1.387656in}}%
\pgfpathlineto{\pgfqpoint{1.214961in}{1.388143in}}%
\pgfpathlineto{\pgfqpoint{1.221305in}{1.388728in}}%
\pgfpathlineto{\pgfqpoint{1.227607in}{1.389409in}}%
\pgfpathlineto{\pgfqpoint{1.233858in}{1.390186in}}%
\pgfpathclose%
\pgfusepath{fill}%
\end{pgfscope}%
\begin{pgfscope}%
\pgfpathrectangle{\pgfqpoint{0.041670in}{0.041670in}}{\pgfqpoint{2.216660in}{2.216660in}}%
\pgfusepath{clip}%
\pgfsetbuttcap%
\pgfsetroundjoin%
\definecolor{currentfill}{rgb}{0.699415,0.867117,0.175971}%
\pgfsetfillcolor{currentfill}%
\pgfsetlinewidth{0.000000pt}%
\definecolor{currentstroke}{rgb}{0.000000,0.000000,0.000000}%
\pgfsetstrokecolor{currentstroke}%
\pgfsetdash{}{0pt}%
\pgfpathmoveto{\pgfqpoint{1.181942in}{1.502987in}}%
\pgfpathlineto{\pgfqpoint{1.181998in}{1.496505in}}%
\pgfpathlineto{\pgfqpoint{1.182053in}{1.489912in}}%
\pgfpathlineto{\pgfqpoint{1.182109in}{1.483209in}}%
\pgfpathlineto{\pgfqpoint{1.182164in}{1.476396in}}%
\pgfpathlineto{\pgfqpoint{1.177194in}{1.476400in}}%
\pgfpathlineto{\pgfqpoint{1.172226in}{1.476480in}}%
\pgfpathlineto{\pgfqpoint{1.167267in}{1.476634in}}%
\pgfpathlineto{\pgfqpoint{1.162321in}{1.476863in}}%
\pgfpathlineto{\pgfqpoint{1.162763in}{1.483663in}}%
\pgfpathlineto{\pgfqpoint{1.163204in}{1.490355in}}%
\pgfpathlineto{\pgfqpoint{1.163646in}{1.496936in}}%
\pgfpathlineto{\pgfqpoint{1.164088in}{1.503405in}}%
\pgfpathlineto{\pgfqpoint{1.168538in}{1.503200in}}%
\pgfpathlineto{\pgfqpoint{1.173001in}{1.503062in}}%
\pgfpathlineto{\pgfqpoint{1.177470in}{1.502991in}}%
\pgfpathlineto{\pgfqpoint{1.181942in}{1.502987in}}%
\pgfpathclose%
\pgfusepath{fill}%
\end{pgfscope}%
\begin{pgfscope}%
\pgfpathrectangle{\pgfqpoint{0.041670in}{0.041670in}}{\pgfqpoint{2.216660in}{2.216660in}}%
\pgfusepath{clip}%
\pgfsetbuttcap%
\pgfsetroundjoin%
\definecolor{currentfill}{rgb}{0.281477,0.755203,0.432552}%
\pgfsetfillcolor{currentfill}%
\pgfsetlinewidth{0.000000pt}%
\definecolor{currentstroke}{rgb}{0.000000,0.000000,0.000000}%
\pgfsetstrokecolor{currentstroke}%
\pgfsetdash{}{0pt}%
\pgfpathmoveto{\pgfqpoint{1.212977in}{1.320790in}}%
\pgfpathlineto{\pgfqpoint{1.213526in}{1.312084in}}%
\pgfpathlineto{\pgfqpoint{1.214075in}{1.303308in}}%
\pgfpathlineto{\pgfqpoint{1.214624in}{1.294464in}}%
\pgfpathlineto{\pgfqpoint{1.215172in}{1.285553in}}%
\pgfpathlineto{\pgfqpoint{1.207286in}{1.285067in}}%
\pgfpathlineto{\pgfqpoint{1.199371in}{1.284704in}}%
\pgfpathlineto{\pgfqpoint{1.191436in}{1.284463in}}%
\pgfpathlineto{\pgfqpoint{1.183489in}{1.284347in}}%
\pgfpathlineto{\pgfqpoint{1.183434in}{1.293277in}}%
\pgfpathlineto{\pgfqpoint{1.183379in}{1.302142in}}%
\pgfpathlineto{\pgfqpoint{1.183324in}{1.310939in}}%
\pgfpathlineto{\pgfqpoint{1.183268in}{1.319664in}}%
\pgfpathlineto{\pgfqpoint{1.190720in}{1.319773in}}%
\pgfpathlineto{\pgfqpoint{1.198161in}{1.319997in}}%
\pgfpathlineto{\pgfqpoint{1.205583in}{1.320336in}}%
\pgfpathlineto{\pgfqpoint{1.212977in}{1.320790in}}%
\pgfpathclose%
\pgfusepath{fill}%
\end{pgfscope}%
\begin{pgfscope}%
\pgfpathrectangle{\pgfqpoint{0.041670in}{0.041670in}}{\pgfqpoint{2.216660in}{2.216660in}}%
\pgfusepath{clip}%
\pgfsetbuttcap%
\pgfsetroundjoin%
\definecolor{currentfill}{rgb}{0.636902,0.856542,0.216620}%
\pgfsetfillcolor{currentfill}%
\pgfsetlinewidth{0.000000pt}%
\definecolor{currentstroke}{rgb}{0.000000,0.000000,0.000000}%
\pgfsetstrokecolor{currentstroke}%
\pgfsetdash{}{0pt}%
\pgfpathmoveto{\pgfqpoint{1.162321in}{1.476863in}}%
\pgfpathlineto{\pgfqpoint{1.161879in}{1.469955in}}%
\pgfpathlineto{\pgfqpoint{1.161438in}{1.462941in}}%
\pgfpathlineto{\pgfqpoint{1.160996in}{1.455823in}}%
\pgfpathlineto{\pgfqpoint{1.160555in}{1.448601in}}%
\pgfpathlineto{\pgfqpoint{1.155133in}{1.448936in}}%
\pgfpathlineto{\pgfqpoint{1.149737in}{1.449354in}}%
\pgfpathlineto{\pgfqpoint{1.144373in}{1.449853in}}%
\pgfpathlineto{\pgfqpoint{1.139045in}{1.450433in}}%
\pgfpathlineto{\pgfqpoint{1.139975in}{1.457611in}}%
\pgfpathlineto{\pgfqpoint{1.140906in}{1.464687in}}%
\pgfpathlineto{\pgfqpoint{1.141837in}{1.471657in}}%
\pgfpathlineto{\pgfqpoint{1.142768in}{1.478522in}}%
\pgfpathlineto{\pgfqpoint{1.147611in}{1.477996in}}%
\pgfpathlineto{\pgfqpoint{1.152488in}{1.477545in}}%
\pgfpathlineto{\pgfqpoint{1.157393in}{1.477167in}}%
\pgfpathlineto{\pgfqpoint{1.162321in}{1.476863in}}%
\pgfpathclose%
\pgfusepath{fill}%
\end{pgfscope}%
\begin{pgfscope}%
\pgfpathrectangle{\pgfqpoint{0.041670in}{0.041670in}}{\pgfqpoint{2.216660in}{2.216660in}}%
\pgfusepath{clip}%
\pgfsetbuttcap%
\pgfsetroundjoin%
\definecolor{currentfill}{rgb}{0.281477,0.755203,0.432552}%
\pgfsetfillcolor{currentfill}%
\pgfsetlinewidth{0.000000pt}%
\definecolor{currentstroke}{rgb}{0.000000,0.000000,0.000000}%
\pgfsetstrokecolor{currentstroke}%
\pgfsetdash{}{0pt}%
\pgfpathmoveto{\pgfqpoint{1.183268in}{1.319664in}}%
\pgfpathlineto{\pgfqpoint{1.183324in}{1.310939in}}%
\pgfpathlineto{\pgfqpoint{1.183379in}{1.302142in}}%
\pgfpathlineto{\pgfqpoint{1.183434in}{1.293277in}}%
\pgfpathlineto{\pgfqpoint{1.183489in}{1.284347in}}%
\pgfpathlineto{\pgfqpoint{1.175538in}{1.284353in}}%
\pgfpathlineto{\pgfqpoint{1.167591in}{1.284484in}}%
\pgfpathlineto{\pgfqpoint{1.159658in}{1.284738in}}%
\pgfpathlineto{\pgfqpoint{1.151746in}{1.285115in}}%
\pgfpathlineto{\pgfqpoint{1.152185in}{1.294033in}}%
\pgfpathlineto{\pgfqpoint{1.152625in}{1.302885in}}%
\pgfpathlineto{\pgfqpoint{1.153064in}{1.311668in}}%
\pgfpathlineto{\pgfqpoint{1.153504in}{1.320381in}}%
\pgfpathlineto{\pgfqpoint{1.160923in}{1.320029in}}%
\pgfpathlineto{\pgfqpoint{1.168362in}{1.319792in}}%
\pgfpathlineto{\pgfqpoint{1.175813in}{1.319671in}}%
\pgfpathlineto{\pgfqpoint{1.183268in}{1.319664in}}%
\pgfpathclose%
\pgfusepath{fill}%
\end{pgfscope}%
\begin{pgfscope}%
\pgfpathrectangle{\pgfqpoint{0.041670in}{0.041670in}}{\pgfqpoint{2.216660in}{2.216660in}}%
\pgfusepath{clip}%
\pgfsetbuttcap%
\pgfsetroundjoin%
\definecolor{currentfill}{rgb}{0.565498,0.842430,0.262877}%
\pgfsetfillcolor{currentfill}%
\pgfsetlinewidth{0.000000pt}%
\definecolor{currentstroke}{rgb}{0.000000,0.000000,0.000000}%
\pgfsetstrokecolor{currentstroke}%
\pgfsetdash{}{0pt}%
\pgfpathmoveto{\pgfqpoint{1.225566in}{1.451017in}}%
\pgfpathlineto{\pgfqpoint{1.226603in}{1.443751in}}%
\pgfpathlineto{\pgfqpoint{1.227640in}{1.436385in}}%
\pgfpathlineto{\pgfqpoint{1.228677in}{1.428922in}}%
\pgfpathlineto{\pgfqpoint{1.229714in}{1.421361in}}%
\pgfpathlineto{\pgfqpoint{1.223942in}{1.420647in}}%
\pgfpathlineto{\pgfqpoint{1.218125in}{1.420021in}}%
\pgfpathlineto{\pgfqpoint{1.212269in}{1.419484in}}%
\pgfpathlineto{\pgfqpoint{1.206379in}{1.419036in}}%
\pgfpathlineto{\pgfqpoint{1.205828in}{1.426647in}}%
\pgfpathlineto{\pgfqpoint{1.205277in}{1.434162in}}%
\pgfpathlineto{\pgfqpoint{1.204726in}{1.441579in}}%
\pgfpathlineto{\pgfqpoint{1.204175in}{1.448895in}}%
\pgfpathlineto{\pgfqpoint{1.209574in}{1.449303in}}%
\pgfpathlineto{\pgfqpoint{1.214943in}{1.449794in}}%
\pgfpathlineto{\pgfqpoint{1.220275in}{1.450365in}}%
\pgfpathlineto{\pgfqpoint{1.225566in}{1.451017in}}%
\pgfpathclose%
\pgfusepath{fill}%
\end{pgfscope}%
\begin{pgfscope}%
\pgfpathrectangle{\pgfqpoint{0.041670in}{0.041670in}}{\pgfqpoint{2.216660in}{2.216660in}}%
\pgfusepath{clip}%
\pgfsetbuttcap%
\pgfsetroundjoin%
\definecolor{currentfill}{rgb}{0.487026,0.823929,0.312321}%
\pgfsetfillcolor{currentfill}%
\pgfsetlinewidth{0.000000pt}%
\definecolor{currentstroke}{rgb}{0.000000,0.000000,0.000000}%
\pgfsetstrokecolor{currentstroke}%
\pgfsetdash{}{0pt}%
\pgfpathmoveto{\pgfqpoint{1.229714in}{1.421361in}}%
\pgfpathlineto{\pgfqpoint{1.230750in}{1.413706in}}%
\pgfpathlineto{\pgfqpoint{1.231786in}{1.405957in}}%
\pgfpathlineto{\pgfqpoint{1.232822in}{1.398116in}}%
\pgfpathlineto{\pgfqpoint{1.233858in}{1.390186in}}%
\pgfpathlineto{\pgfqpoint{1.227607in}{1.389409in}}%
\pgfpathlineto{\pgfqpoint{1.221305in}{1.388728in}}%
\pgfpathlineto{\pgfqpoint{1.214961in}{1.388143in}}%
\pgfpathlineto{\pgfqpoint{1.208581in}{1.387656in}}%
\pgfpathlineto{\pgfqpoint{1.208030in}{1.395638in}}%
\pgfpathlineto{\pgfqpoint{1.207480in}{1.403529in}}%
\pgfpathlineto{\pgfqpoint{1.206930in}{1.411329in}}%
\pgfpathlineto{\pgfqpoint{1.206379in}{1.419036in}}%
\pgfpathlineto{\pgfqpoint{1.212269in}{1.419484in}}%
\pgfpathlineto{\pgfqpoint{1.218125in}{1.420021in}}%
\pgfpathlineto{\pgfqpoint{1.223942in}{1.420647in}}%
\pgfpathlineto{\pgfqpoint{1.229714in}{1.421361in}}%
\pgfpathclose%
\pgfusepath{fill}%
\end{pgfscope}%
\begin{pgfscope}%
\pgfpathrectangle{\pgfqpoint{0.041670in}{0.041670in}}{\pgfqpoint{2.216660in}{2.216660in}}%
\pgfusepath{clip}%
\pgfsetbuttcap%
\pgfsetroundjoin%
\definecolor{currentfill}{rgb}{0.412913,0.803041,0.357269}%
\pgfsetfillcolor{currentfill}%
\pgfsetlinewidth{0.000000pt}%
\definecolor{currentstroke}{rgb}{0.000000,0.000000,0.000000}%
\pgfsetstrokecolor{currentstroke}%
\pgfsetdash{}{0pt}%
\pgfpathmoveto{\pgfqpoint{1.157026in}{1.387305in}}%
\pgfpathlineto{\pgfqpoint{1.156585in}{1.379228in}}%
\pgfpathlineto{\pgfqpoint{1.156145in}{1.371065in}}%
\pgfpathlineto{\pgfqpoint{1.155704in}{1.362817in}}%
\pgfpathlineto{\pgfqpoint{1.155264in}{1.354487in}}%
\pgfpathlineto{\pgfqpoint{1.148364in}{1.354920in}}%
\pgfpathlineto{\pgfqpoint{1.141498in}{1.355459in}}%
\pgfpathlineto{\pgfqpoint{1.134671in}{1.356104in}}%
\pgfpathlineto{\pgfqpoint{1.127892in}{1.356852in}}%
\pgfpathlineto{\pgfqpoint{1.128820in}{1.365138in}}%
\pgfpathlineto{\pgfqpoint{1.129748in}{1.373340in}}%
\pgfpathlineto{\pgfqpoint{1.130677in}{1.381459in}}%
\pgfpathlineto{\pgfqpoint{1.131606in}{1.389491in}}%
\pgfpathlineto{\pgfqpoint{1.137902in}{1.388799in}}%
\pgfpathlineto{\pgfqpoint{1.144242in}{1.388203in}}%
\pgfpathlineto{\pgfqpoint{1.150619in}{1.387705in}}%
\pgfpathlineto{\pgfqpoint{1.157026in}{1.387305in}}%
\pgfpathclose%
\pgfusepath{fill}%
\end{pgfscope}%
\begin{pgfscope}%
\pgfpathrectangle{\pgfqpoint{0.041670in}{0.041670in}}{\pgfqpoint{2.216660in}{2.216660in}}%
\pgfusepath{clip}%
\pgfsetbuttcap%
\pgfsetroundjoin%
\definecolor{currentfill}{rgb}{0.201239,0.383670,0.554294}%
\pgfsetfillcolor{currentfill}%
\pgfsetlinewidth{0.000000pt}%
\definecolor{currentstroke}{rgb}{0.000000,0.000000,0.000000}%
\pgfsetstrokecolor{currentstroke}%
\pgfsetdash{}{0pt}%
\pgfpathmoveto{\pgfqpoint{0.601818in}{0.837283in}}%
\pgfpathlineto{\pgfqpoint{0.598321in}{0.849249in}}%
\pgfpathlineto{\pgfqpoint{0.594805in}{0.861706in}}%
\pgfpathlineto{\pgfqpoint{0.591269in}{0.874662in}}%
\pgfpathlineto{\pgfqpoint{0.587713in}{0.888125in}}%
\pgfpathlineto{\pgfqpoint{0.573067in}{0.897984in}}%
\pgfpathlineto{\pgfqpoint{0.559087in}{0.908064in}}%
\pgfpathlineto{\pgfqpoint{0.545785in}{0.918354in}}%
\pgfpathlineto{\pgfqpoint{0.533173in}{0.928840in}}%
\pgfpathlineto{\pgfqpoint{0.537034in}{0.915210in}}%
\pgfpathlineto{\pgfqpoint{0.540874in}{0.902084in}}%
\pgfpathlineto{\pgfqpoint{0.544693in}{0.889455in}}%
\pgfpathlineto{\pgfqpoint{0.548491in}{0.877314in}}%
\pgfpathlineto{\pgfqpoint{0.560826in}{0.867002in}}%
\pgfpathlineto{\pgfqpoint{0.573833in}{0.856885in}}%
\pgfpathlineto{\pgfqpoint{0.587501in}{0.846975in}}%
\pgfpathlineto{\pgfqpoint{0.601818in}{0.837283in}}%
\pgfpathclose%
\pgfusepath{fill}%
\end{pgfscope}%
\begin{pgfscope}%
\pgfpathrectangle{\pgfqpoint{0.041670in}{0.041670in}}{\pgfqpoint{2.216660in}{2.216660in}}%
\pgfusepath{clip}%
\pgfsetbuttcap%
\pgfsetroundjoin%
\definecolor{currentfill}{rgb}{0.565498,0.842430,0.262877}%
\pgfsetfillcolor{currentfill}%
\pgfsetlinewidth{0.000000pt}%
\definecolor{currentstroke}{rgb}{0.000000,0.000000,0.000000}%
\pgfsetstrokecolor{currentstroke}%
\pgfsetdash{}{0pt}%
\pgfpathmoveto{\pgfqpoint{1.160555in}{1.448601in}}%
\pgfpathlineto{\pgfqpoint{1.160113in}{1.441277in}}%
\pgfpathlineto{\pgfqpoint{1.159672in}{1.433854in}}%
\pgfpathlineto{\pgfqpoint{1.159231in}{1.426332in}}%
\pgfpathlineto{\pgfqpoint{1.158790in}{1.418714in}}%
\pgfpathlineto{\pgfqpoint{1.152875in}{1.419081in}}%
\pgfpathlineto{\pgfqpoint{1.146988in}{1.419539in}}%
\pgfpathlineto{\pgfqpoint{1.141136in}{1.420086in}}%
\pgfpathlineto{\pgfqpoint{1.135324in}{1.420722in}}%
\pgfpathlineto{\pgfqpoint{1.136254in}{1.428296in}}%
\pgfpathlineto{\pgfqpoint{1.137184in}{1.435774in}}%
\pgfpathlineto{\pgfqpoint{1.138114in}{1.443154in}}%
\pgfpathlineto{\pgfqpoint{1.139045in}{1.450433in}}%
\pgfpathlineto{\pgfqpoint{1.144373in}{1.449853in}}%
\pgfpathlineto{\pgfqpoint{1.149737in}{1.449354in}}%
\pgfpathlineto{\pgfqpoint{1.155133in}{1.448936in}}%
\pgfpathlineto{\pgfqpoint{1.160555in}{1.448601in}}%
\pgfpathclose%
\pgfusepath{fill}%
\end{pgfscope}%
\begin{pgfscope}%
\pgfpathrectangle{\pgfqpoint{0.041670in}{0.041670in}}{\pgfqpoint{2.216660in}{2.216660in}}%
\pgfusepath{clip}%
\pgfsetbuttcap%
\pgfsetroundjoin%
\definecolor{currentfill}{rgb}{0.487026,0.823929,0.312321}%
\pgfsetfillcolor{currentfill}%
\pgfsetlinewidth{0.000000pt}%
\definecolor{currentstroke}{rgb}{0.000000,0.000000,0.000000}%
\pgfsetstrokecolor{currentstroke}%
\pgfsetdash{}{0pt}%
\pgfpathmoveto{\pgfqpoint{1.158790in}{1.418714in}}%
\pgfpathlineto{\pgfqpoint{1.158349in}{1.411000in}}%
\pgfpathlineto{\pgfqpoint{1.157908in}{1.403193in}}%
\pgfpathlineto{\pgfqpoint{1.157467in}{1.395294in}}%
\pgfpathlineto{\pgfqpoint{1.157026in}{1.387305in}}%
\pgfpathlineto{\pgfqpoint{1.150619in}{1.387705in}}%
\pgfpathlineto{\pgfqpoint{1.144242in}{1.388203in}}%
\pgfpathlineto{\pgfqpoint{1.137902in}{1.388799in}}%
\pgfpathlineto{\pgfqpoint{1.131606in}{1.389491in}}%
\pgfpathlineto{\pgfqpoint{1.132535in}{1.397435in}}%
\pgfpathlineto{\pgfqpoint{1.133464in}{1.405289in}}%
\pgfpathlineto{\pgfqpoint{1.134394in}{1.413052in}}%
\pgfpathlineto{\pgfqpoint{1.135324in}{1.420722in}}%
\pgfpathlineto{\pgfqpoint{1.141136in}{1.420086in}}%
\pgfpathlineto{\pgfqpoint{1.146988in}{1.419539in}}%
\pgfpathlineto{\pgfqpoint{1.152875in}{1.419081in}}%
\pgfpathlineto{\pgfqpoint{1.158790in}{1.418714in}}%
\pgfpathclose%
\pgfusepath{fill}%
\end{pgfscope}%
\begin{pgfscope}%
\pgfpathrectangle{\pgfqpoint{0.041670in}{0.041670in}}{\pgfqpoint{2.216660in}{2.216660in}}%
\pgfusepath{clip}%
\pgfsetbuttcap%
\pgfsetroundjoin%
\definecolor{currentfill}{rgb}{0.636902,0.856542,0.216620}%
\pgfsetfillcolor{currentfill}%
\pgfsetlinewidth{0.000000pt}%
\definecolor{currentstroke}{rgb}{0.000000,0.000000,0.000000}%
\pgfsetstrokecolor{currentstroke}%
\pgfsetdash{}{0pt}%
\pgfpathmoveto{\pgfqpoint{1.201970in}{1.477129in}}%
\pgfpathlineto{\pgfqpoint{1.202522in}{1.470228in}}%
\pgfpathlineto{\pgfqpoint{1.203073in}{1.463221in}}%
\pgfpathlineto{\pgfqpoint{1.203624in}{1.456110in}}%
\pgfpathlineto{\pgfqpoint{1.204175in}{1.448895in}}%
\pgfpathlineto{\pgfqpoint{1.198751in}{1.448569in}}%
\pgfpathlineto{\pgfqpoint{1.193308in}{1.448325in}}%
\pgfpathlineto{\pgfqpoint{1.187851in}{1.448163in}}%
\pgfpathlineto{\pgfqpoint{1.182385in}{1.448085in}}%
\pgfpathlineto{\pgfqpoint{1.182330in}{1.455319in}}%
\pgfpathlineto{\pgfqpoint{1.182274in}{1.462450in}}%
\pgfpathlineto{\pgfqpoint{1.182219in}{1.469476in}}%
\pgfpathlineto{\pgfqpoint{1.182164in}{1.476396in}}%
\pgfpathlineto{\pgfqpoint{1.187132in}{1.476467in}}%
\pgfpathlineto{\pgfqpoint{1.192092in}{1.476613in}}%
\pgfpathlineto{\pgfqpoint{1.197040in}{1.476834in}}%
\pgfpathlineto{\pgfqpoint{1.201970in}{1.477129in}}%
\pgfpathclose%
\pgfusepath{fill}%
\end{pgfscope}%
\begin{pgfscope}%
\pgfpathrectangle{\pgfqpoint{0.041670in}{0.041670in}}{\pgfqpoint{2.216660in}{2.216660in}}%
\pgfusepath{clip}%
\pgfsetbuttcap%
\pgfsetroundjoin%
\definecolor{currentfill}{rgb}{0.344074,0.780029,0.397381}%
\pgfsetfillcolor{currentfill}%
\pgfsetlinewidth{0.000000pt}%
\definecolor{currentstroke}{rgb}{0.000000,0.000000,0.000000}%
\pgfsetstrokecolor{currentstroke}%
\pgfsetdash{}{0pt}%
\pgfpathmoveto{\pgfqpoint{1.210780in}{1.354867in}}%
\pgfpathlineto{\pgfqpoint{1.211330in}{1.346464in}}%
\pgfpathlineto{\pgfqpoint{1.211879in}{1.337982in}}%
\pgfpathlineto{\pgfqpoint{1.212428in}{1.329423in}}%
\pgfpathlineto{\pgfqpoint{1.212977in}{1.320790in}}%
\pgfpathlineto{\pgfqpoint{1.205583in}{1.320336in}}%
\pgfpathlineto{\pgfqpoint{1.198161in}{1.319997in}}%
\pgfpathlineto{\pgfqpoint{1.190720in}{1.319773in}}%
\pgfpathlineto{\pgfqpoint{1.183268in}{1.319664in}}%
\pgfpathlineto{\pgfqpoint{1.183213in}{1.328318in}}%
\pgfpathlineto{\pgfqpoint{1.183158in}{1.336896in}}%
\pgfpathlineto{\pgfqpoint{1.183103in}{1.345398in}}%
\pgfpathlineto{\pgfqpoint{1.183048in}{1.353822in}}%
\pgfpathlineto{\pgfqpoint{1.190004in}{1.353923in}}%
\pgfpathlineto{\pgfqpoint{1.196950in}{1.354131in}}%
\pgfpathlineto{\pgfqpoint{1.203877in}{1.354446in}}%
\pgfpathlineto{\pgfqpoint{1.210780in}{1.354867in}}%
\pgfpathclose%
\pgfusepath{fill}%
\end{pgfscope}%
\begin{pgfscope}%
\pgfpathrectangle{\pgfqpoint{0.041670in}{0.041670in}}{\pgfqpoint{2.216660in}{2.216660in}}%
\pgfusepath{clip}%
\pgfsetbuttcap%
\pgfsetroundjoin%
\definecolor{currentfill}{rgb}{0.267004,0.004874,0.329415}%
\pgfsetfillcolor{currentfill}%
\pgfsetlinewidth{0.000000pt}%
\definecolor{currentstroke}{rgb}{0.000000,0.000000,0.000000}%
\pgfsetstrokecolor{currentstroke}%
\pgfsetdash{}{0pt}%
\pgfpathmoveto{\pgfqpoint{1.265754in}{0.558634in}}%
\pgfpathlineto{\pgfqpoint{1.266329in}{0.556777in}}%
\pgfpathlineto{\pgfqpoint{1.266906in}{0.555183in}}%
\pgfpathlineto{\pgfqpoint{1.267484in}{0.553855in}}%
\pgfpathlineto{\pgfqpoint{1.268064in}{0.552798in}}%
\pgfpathlineto{\pgfqpoint{1.248342in}{0.551451in}}%
\pgfpathlineto{\pgfqpoint{1.228541in}{0.550444in}}%
\pgfpathlineto{\pgfqpoint{1.208686in}{0.549778in}}%
\pgfpathlineto{\pgfqpoint{1.188798in}{0.549454in}}%
\pgfpathlineto{\pgfqpoint{1.188740in}{0.550533in}}%
\pgfpathlineto{\pgfqpoint{1.188682in}{0.551883in}}%
\pgfpathlineto{\pgfqpoint{1.188624in}{0.553501in}}%
\pgfpathlineto{\pgfqpoint{1.188566in}{0.555380in}}%
\pgfpathlineto{\pgfqpoint{1.207932in}{0.555695in}}%
\pgfpathlineto{\pgfqpoint{1.227267in}{0.556342in}}%
\pgfpathlineto{\pgfqpoint{1.246548in}{0.557323in}}%
\pgfpathlineto{\pgfqpoint{1.265754in}{0.558634in}}%
\pgfpathclose%
\pgfusepath{fill}%
\end{pgfscope}%
\begin{pgfscope}%
\pgfpathrectangle{\pgfqpoint{0.041670in}{0.041670in}}{\pgfqpoint{2.216660in}{2.216660in}}%
\pgfusepath{clip}%
\pgfsetbuttcap%
\pgfsetroundjoin%
\definecolor{currentfill}{rgb}{0.636902,0.856542,0.216620}%
\pgfsetfillcolor{currentfill}%
\pgfsetlinewidth{0.000000pt}%
\definecolor{currentstroke}{rgb}{0.000000,0.000000,0.000000}%
\pgfsetstrokecolor{currentstroke}%
\pgfsetdash{}{0pt}%
\pgfpathmoveto{\pgfqpoint{1.182164in}{1.476396in}}%
\pgfpathlineto{\pgfqpoint{1.182219in}{1.469476in}}%
\pgfpathlineto{\pgfqpoint{1.182274in}{1.462450in}}%
\pgfpathlineto{\pgfqpoint{1.182330in}{1.455319in}}%
\pgfpathlineto{\pgfqpoint{1.182385in}{1.448085in}}%
\pgfpathlineto{\pgfqpoint{1.176917in}{1.448090in}}%
\pgfpathlineto{\pgfqpoint{1.171452in}{1.448177in}}%
\pgfpathlineto{\pgfqpoint{1.165996in}{1.448348in}}%
\pgfpathlineto{\pgfqpoint{1.160555in}{1.448601in}}%
\pgfpathlineto{\pgfqpoint{1.160996in}{1.455823in}}%
\pgfpathlineto{\pgfqpoint{1.161438in}{1.462941in}}%
\pgfpathlineto{\pgfqpoint{1.161879in}{1.469955in}}%
\pgfpathlineto{\pgfqpoint{1.162321in}{1.476863in}}%
\pgfpathlineto{\pgfqpoint{1.167267in}{1.476634in}}%
\pgfpathlineto{\pgfqpoint{1.172226in}{1.476480in}}%
\pgfpathlineto{\pgfqpoint{1.177194in}{1.476400in}}%
\pgfpathlineto{\pgfqpoint{1.182164in}{1.476396in}}%
\pgfpathclose%
\pgfusepath{fill}%
\end{pgfscope}%
\begin{pgfscope}%
\pgfpathrectangle{\pgfqpoint{0.041670in}{0.041670in}}{\pgfqpoint{2.216660in}{2.216660in}}%
\pgfusepath{clip}%
\pgfsetbuttcap%
\pgfsetroundjoin%
\definecolor{currentfill}{rgb}{0.344074,0.780029,0.397381}%
\pgfsetfillcolor{currentfill}%
\pgfsetlinewidth{0.000000pt}%
\definecolor{currentstroke}{rgb}{0.000000,0.000000,0.000000}%
\pgfsetstrokecolor{currentstroke}%
\pgfsetdash{}{0pt}%
\pgfpathmoveto{\pgfqpoint{1.183048in}{1.353822in}}%
\pgfpathlineto{\pgfqpoint{1.183103in}{1.345398in}}%
\pgfpathlineto{\pgfqpoint{1.183158in}{1.336896in}}%
\pgfpathlineto{\pgfqpoint{1.183213in}{1.328318in}}%
\pgfpathlineto{\pgfqpoint{1.183268in}{1.319664in}}%
\pgfpathlineto{\pgfqpoint{1.175813in}{1.319671in}}%
\pgfpathlineto{\pgfqpoint{1.168362in}{1.319792in}}%
\pgfpathlineto{\pgfqpoint{1.160923in}{1.320029in}}%
\pgfpathlineto{\pgfqpoint{1.153504in}{1.320381in}}%
\pgfpathlineto{\pgfqpoint{1.153944in}{1.329022in}}%
\pgfpathlineto{\pgfqpoint{1.154384in}{1.337587in}}%
\pgfpathlineto{\pgfqpoint{1.154824in}{1.346077in}}%
\pgfpathlineto{\pgfqpoint{1.155264in}{1.354487in}}%
\pgfpathlineto{\pgfqpoint{1.162189in}{1.354161in}}%
\pgfpathlineto{\pgfqpoint{1.169133in}{1.353941in}}%
\pgfpathlineto{\pgfqpoint{1.176089in}{1.353828in}}%
\pgfpathlineto{\pgfqpoint{1.183048in}{1.353822in}}%
\pgfpathclose%
\pgfusepath{fill}%
\end{pgfscope}%
\begin{pgfscope}%
\pgfpathrectangle{\pgfqpoint{0.041670in}{0.041670in}}{\pgfqpoint{2.216660in}{2.216660in}}%
\pgfusepath{clip}%
\pgfsetbuttcap%
\pgfsetroundjoin%
\definecolor{currentfill}{rgb}{0.277941,0.056324,0.381191}%
\pgfsetfillcolor{currentfill}%
\pgfsetlinewidth{0.000000pt}%
\definecolor{currentstroke}{rgb}{0.000000,0.000000,0.000000}%
\pgfsetstrokecolor{currentstroke}%
\pgfsetdash{}{0pt}%
\pgfpathmoveto{\pgfqpoint{1.507301in}{0.598652in}}%
\pgfpathlineto{\pgfqpoint{1.509392in}{0.600560in}}%
\pgfpathlineto{\pgfqpoint{1.511490in}{0.602794in}}%
\pgfpathlineto{\pgfqpoint{1.513596in}{0.605361in}}%
\pgfpathlineto{\pgfqpoint{1.515709in}{0.608266in}}%
\pgfpathlineto{\pgfqpoint{1.497232in}{0.602683in}}%
\pgfpathlineto{\pgfqpoint{1.478398in}{0.597415in}}%
\pgfpathlineto{\pgfqpoint{1.459228in}{0.592469in}}%
\pgfpathlineto{\pgfqpoint{1.439742in}{0.587851in}}%
\pgfpathlineto{\pgfqpoint{1.438103in}{0.585066in}}%
\pgfpathlineto{\pgfqpoint{1.436469in}{0.582621in}}%
\pgfpathlineto{\pgfqpoint{1.434842in}{0.580509in}}%
\pgfpathlineto{\pgfqpoint{1.433220in}{0.578723in}}%
\pgfpathlineto{\pgfqpoint{1.452220in}{0.583231in}}%
\pgfpathlineto{\pgfqpoint{1.470914in}{0.588059in}}%
\pgfpathlineto{\pgfqpoint{1.489281in}{0.593202in}}%
\pgfpathlineto{\pgfqpoint{1.507301in}{0.598652in}}%
\pgfpathclose%
\pgfusepath{fill}%
\end{pgfscope}%
\begin{pgfscope}%
\pgfpathrectangle{\pgfqpoint{0.041670in}{0.041670in}}{\pgfqpoint{2.216660in}{2.216660in}}%
\pgfusepath{clip}%
\pgfsetbuttcap%
\pgfsetroundjoin%
\definecolor{currentfill}{rgb}{0.267004,0.004874,0.329415}%
\pgfsetfillcolor{currentfill}%
\pgfsetlinewidth{0.000000pt}%
\definecolor{currentstroke}{rgb}{0.000000,0.000000,0.000000}%
\pgfsetstrokecolor{currentstroke}%
\pgfsetdash{}{0pt}%
\pgfpathmoveto{\pgfqpoint{1.188566in}{0.555380in}}%
\pgfpathlineto{\pgfqpoint{1.188624in}{0.553501in}}%
\pgfpathlineto{\pgfqpoint{1.188682in}{0.551883in}}%
\pgfpathlineto{\pgfqpoint{1.188740in}{0.550533in}}%
\pgfpathlineto{\pgfqpoint{1.188798in}{0.549454in}}%
\pgfpathlineto{\pgfqpoint{1.168901in}{0.549473in}}%
\pgfpathlineto{\pgfqpoint{1.149015in}{0.549835in}}%
\pgfpathlineto{\pgfqpoint{1.129165in}{0.550539in}}%
\pgfpathlineto{\pgfqpoint{1.109372in}{0.551584in}}%
\pgfpathlineto{\pgfqpoint{1.109837in}{0.552648in}}%
\pgfpathlineto{\pgfqpoint{1.110301in}{0.553984in}}%
\pgfpathlineto{\pgfqpoint{1.110763in}{0.555587in}}%
\pgfpathlineto{\pgfqpoint{1.111224in}{0.557452in}}%
\pgfpathlineto{\pgfqpoint{1.130498in}{0.556435in}}%
\pgfpathlineto{\pgfqpoint{1.149827in}{0.555750in}}%
\pgfpathlineto{\pgfqpoint{1.169191in}{0.555398in}}%
\pgfpathlineto{\pgfqpoint{1.188566in}{0.555380in}}%
\pgfpathclose%
\pgfusepath{fill}%
\end{pgfscope}%
\begin{pgfscope}%
\pgfpathrectangle{\pgfqpoint{0.041670in}{0.041670in}}{\pgfqpoint{2.216660in}{2.216660in}}%
\pgfusepath{clip}%
\pgfsetbuttcap%
\pgfsetroundjoin%
\definecolor{currentfill}{rgb}{0.412913,0.803041,0.357269}%
\pgfsetfillcolor{currentfill}%
\pgfsetlinewidth{0.000000pt}%
\definecolor{currentstroke}{rgb}{0.000000,0.000000,0.000000}%
\pgfsetstrokecolor{currentstroke}%
\pgfsetdash{}{0pt}%
\pgfpathmoveto{\pgfqpoint{1.208581in}{1.387656in}}%
\pgfpathlineto{\pgfqpoint{1.209131in}{1.379586in}}%
\pgfpathlineto{\pgfqpoint{1.209681in}{1.371430in}}%
\pgfpathlineto{\pgfqpoint{1.210231in}{1.363190in}}%
\pgfpathlineto{\pgfqpoint{1.210780in}{1.354867in}}%
\pgfpathlineto{\pgfqpoint{1.203877in}{1.354446in}}%
\pgfpathlineto{\pgfqpoint{1.196950in}{1.354131in}}%
\pgfpathlineto{\pgfqpoint{1.190004in}{1.353923in}}%
\pgfpathlineto{\pgfqpoint{1.183048in}{1.353822in}}%
\pgfpathlineto{\pgfqpoint{1.182993in}{1.362164in}}%
\pgfpathlineto{\pgfqpoint{1.182938in}{1.370425in}}%
\pgfpathlineto{\pgfqpoint{1.182882in}{1.378600in}}%
\pgfpathlineto{\pgfqpoint{1.182827in}{1.386690in}}%
\pgfpathlineto{\pgfqpoint{1.189287in}{1.386783in}}%
\pgfpathlineto{\pgfqpoint{1.195737in}{1.386976in}}%
\pgfpathlineto{\pgfqpoint{1.202170in}{1.387267in}}%
\pgfpathlineto{\pgfqpoint{1.208581in}{1.387656in}}%
\pgfpathclose%
\pgfusepath{fill}%
\end{pgfscope}%
\begin{pgfscope}%
\pgfpathrectangle{\pgfqpoint{0.041670in}{0.041670in}}{\pgfqpoint{2.216660in}{2.216660in}}%
\pgfusepath{clip}%
\pgfsetbuttcap%
\pgfsetroundjoin%
\definecolor{currentfill}{rgb}{0.565498,0.842430,0.262877}%
\pgfsetfillcolor{currentfill}%
\pgfsetlinewidth{0.000000pt}%
\definecolor{currentstroke}{rgb}{0.000000,0.000000,0.000000}%
\pgfsetstrokecolor{currentstroke}%
\pgfsetdash{}{0pt}%
\pgfpathmoveto{\pgfqpoint{1.204175in}{1.448895in}}%
\pgfpathlineto{\pgfqpoint{1.204726in}{1.441579in}}%
\pgfpathlineto{\pgfqpoint{1.205277in}{1.434162in}}%
\pgfpathlineto{\pgfqpoint{1.205828in}{1.426647in}}%
\pgfpathlineto{\pgfqpoint{1.206379in}{1.419036in}}%
\pgfpathlineto{\pgfqpoint{1.200461in}{1.418678in}}%
\pgfpathlineto{\pgfqpoint{1.194523in}{1.418411in}}%
\pgfpathlineto{\pgfqpoint{1.188569in}{1.418234in}}%
\pgfpathlineto{\pgfqpoint{1.182606in}{1.418148in}}%
\pgfpathlineto{\pgfqpoint{1.182551in}{1.425779in}}%
\pgfpathlineto{\pgfqpoint{1.182496in}{1.433314in}}%
\pgfpathlineto{\pgfqpoint{1.182440in}{1.440749in}}%
\pgfpathlineto{\pgfqpoint{1.182385in}{1.448085in}}%
\pgfpathlineto{\pgfqpoint{1.187851in}{1.448163in}}%
\pgfpathlineto{\pgfqpoint{1.193308in}{1.448325in}}%
\pgfpathlineto{\pgfqpoint{1.198751in}{1.448569in}}%
\pgfpathlineto{\pgfqpoint{1.204175in}{1.448895in}}%
\pgfpathclose%
\pgfusepath{fill}%
\end{pgfscope}%
\begin{pgfscope}%
\pgfpathrectangle{\pgfqpoint{0.041670in}{0.041670in}}{\pgfqpoint{2.216660in}{2.216660in}}%
\pgfusepath{clip}%
\pgfsetbuttcap%
\pgfsetroundjoin%
\definecolor{currentfill}{rgb}{0.412913,0.803041,0.357269}%
\pgfsetfillcolor{currentfill}%
\pgfsetlinewidth{0.000000pt}%
\definecolor{currentstroke}{rgb}{0.000000,0.000000,0.000000}%
\pgfsetstrokecolor{currentstroke}%
\pgfsetdash{}{0pt}%
\pgfpathmoveto{\pgfqpoint{1.182827in}{1.386690in}}%
\pgfpathlineto{\pgfqpoint{1.182882in}{1.378600in}}%
\pgfpathlineto{\pgfqpoint{1.182938in}{1.370425in}}%
\pgfpathlineto{\pgfqpoint{1.182993in}{1.362164in}}%
\pgfpathlineto{\pgfqpoint{1.183048in}{1.353822in}}%
\pgfpathlineto{\pgfqpoint{1.176089in}{1.353828in}}%
\pgfpathlineto{\pgfqpoint{1.169133in}{1.353941in}}%
\pgfpathlineto{\pgfqpoint{1.162189in}{1.354161in}}%
\pgfpathlineto{\pgfqpoint{1.155264in}{1.354487in}}%
\pgfpathlineto{\pgfqpoint{1.155704in}{1.362817in}}%
\pgfpathlineto{\pgfqpoint{1.156145in}{1.371065in}}%
\pgfpathlineto{\pgfqpoint{1.156585in}{1.379228in}}%
\pgfpathlineto{\pgfqpoint{1.157026in}{1.387305in}}%
\pgfpathlineto{\pgfqpoint{1.163457in}{1.387003in}}%
\pgfpathlineto{\pgfqpoint{1.169906in}{1.386800in}}%
\pgfpathlineto{\pgfqpoint{1.176365in}{1.386695in}}%
\pgfpathlineto{\pgfqpoint{1.182827in}{1.386690in}}%
\pgfpathclose%
\pgfusepath{fill}%
\end{pgfscope}%
\begin{pgfscope}%
\pgfpathrectangle{\pgfqpoint{0.041670in}{0.041670in}}{\pgfqpoint{2.216660in}{2.216660in}}%
\pgfusepath{clip}%
\pgfsetbuttcap%
\pgfsetroundjoin%
\definecolor{currentfill}{rgb}{0.565498,0.842430,0.262877}%
\pgfsetfillcolor{currentfill}%
\pgfsetlinewidth{0.000000pt}%
\definecolor{currentstroke}{rgb}{0.000000,0.000000,0.000000}%
\pgfsetstrokecolor{currentstroke}%
\pgfsetdash{}{0pt}%
\pgfpathmoveto{\pgfqpoint{1.182385in}{1.448085in}}%
\pgfpathlineto{\pgfqpoint{1.182440in}{1.440749in}}%
\pgfpathlineto{\pgfqpoint{1.182496in}{1.433314in}}%
\pgfpathlineto{\pgfqpoint{1.182551in}{1.425779in}}%
\pgfpathlineto{\pgfqpoint{1.182606in}{1.418148in}}%
\pgfpathlineto{\pgfqpoint{1.176641in}{1.418153in}}%
\pgfpathlineto{\pgfqpoint{1.170679in}{1.418249in}}%
\pgfpathlineto{\pgfqpoint{1.164726in}{1.418436in}}%
\pgfpathlineto{\pgfqpoint{1.158790in}{1.418714in}}%
\pgfpathlineto{\pgfqpoint{1.159231in}{1.426332in}}%
\pgfpathlineto{\pgfqpoint{1.159672in}{1.433854in}}%
\pgfpathlineto{\pgfqpoint{1.160113in}{1.441277in}}%
\pgfpathlineto{\pgfqpoint{1.160555in}{1.448601in}}%
\pgfpathlineto{\pgfqpoint{1.165996in}{1.448348in}}%
\pgfpathlineto{\pgfqpoint{1.171452in}{1.448177in}}%
\pgfpathlineto{\pgfqpoint{1.176917in}{1.448090in}}%
\pgfpathlineto{\pgfqpoint{1.182385in}{1.448085in}}%
\pgfpathclose%
\pgfusepath{fill}%
\end{pgfscope}%
\begin{pgfscope}%
\pgfpathrectangle{\pgfqpoint{0.041670in}{0.041670in}}{\pgfqpoint{2.216660in}{2.216660in}}%
\pgfusepath{clip}%
\pgfsetbuttcap%
\pgfsetroundjoin%
\definecolor{currentfill}{rgb}{0.487026,0.823929,0.312321}%
\pgfsetfillcolor{currentfill}%
\pgfsetlinewidth{0.000000pt}%
\definecolor{currentstroke}{rgb}{0.000000,0.000000,0.000000}%
\pgfsetstrokecolor{currentstroke}%
\pgfsetdash{}{0pt}%
\pgfpathmoveto{\pgfqpoint{1.206379in}{1.419036in}}%
\pgfpathlineto{\pgfqpoint{1.206930in}{1.411329in}}%
\pgfpathlineto{\pgfqpoint{1.207480in}{1.403529in}}%
\pgfpathlineto{\pgfqpoint{1.208030in}{1.395638in}}%
\pgfpathlineto{\pgfqpoint{1.208581in}{1.387656in}}%
\pgfpathlineto{\pgfqpoint{1.202170in}{1.387267in}}%
\pgfpathlineto{\pgfqpoint{1.195737in}{1.386976in}}%
\pgfpathlineto{\pgfqpoint{1.189287in}{1.386783in}}%
\pgfpathlineto{\pgfqpoint{1.182827in}{1.386690in}}%
\pgfpathlineto{\pgfqpoint{1.182772in}{1.394691in}}%
\pgfpathlineto{\pgfqpoint{1.182717in}{1.402603in}}%
\pgfpathlineto{\pgfqpoint{1.182661in}{1.410422in}}%
\pgfpathlineto{\pgfqpoint{1.182606in}{1.418148in}}%
\pgfpathlineto{\pgfqpoint{1.188569in}{1.418234in}}%
\pgfpathlineto{\pgfqpoint{1.194523in}{1.418411in}}%
\pgfpathlineto{\pgfqpoint{1.200461in}{1.418678in}}%
\pgfpathlineto{\pgfqpoint{1.206379in}{1.419036in}}%
\pgfpathclose%
\pgfusepath{fill}%
\end{pgfscope}%
\begin{pgfscope}%
\pgfpathrectangle{\pgfqpoint{0.041670in}{0.041670in}}{\pgfqpoint{2.216660in}{2.216660in}}%
\pgfusepath{clip}%
\pgfsetbuttcap%
\pgfsetroundjoin%
\definecolor{currentfill}{rgb}{0.487026,0.823929,0.312321}%
\pgfsetfillcolor{currentfill}%
\pgfsetlinewidth{0.000000pt}%
\definecolor{currentstroke}{rgb}{0.000000,0.000000,0.000000}%
\pgfsetstrokecolor{currentstroke}%
\pgfsetdash{}{0pt}%
\pgfpathmoveto{\pgfqpoint{1.182606in}{1.418148in}}%
\pgfpathlineto{\pgfqpoint{1.182661in}{1.410422in}}%
\pgfpathlineto{\pgfqpoint{1.182717in}{1.402603in}}%
\pgfpathlineto{\pgfqpoint{1.182772in}{1.394691in}}%
\pgfpathlineto{\pgfqpoint{1.182827in}{1.386690in}}%
\pgfpathlineto{\pgfqpoint{1.176365in}{1.386695in}}%
\pgfpathlineto{\pgfqpoint{1.169906in}{1.386800in}}%
\pgfpathlineto{\pgfqpoint{1.163457in}{1.387003in}}%
\pgfpathlineto{\pgfqpoint{1.157026in}{1.387305in}}%
\pgfpathlineto{\pgfqpoint{1.157467in}{1.395294in}}%
\pgfpathlineto{\pgfqpoint{1.157908in}{1.403193in}}%
\pgfpathlineto{\pgfqpoint{1.158349in}{1.411000in}}%
\pgfpathlineto{\pgfqpoint{1.158790in}{1.418714in}}%
\pgfpathlineto{\pgfqpoint{1.164726in}{1.418436in}}%
\pgfpathlineto{\pgfqpoint{1.170679in}{1.418249in}}%
\pgfpathlineto{\pgfqpoint{1.176641in}{1.418153in}}%
\pgfpathlineto{\pgfqpoint{1.182606in}{1.418148in}}%
\pgfpathclose%
\pgfusepath{fill}%
\end{pgfscope}%
\begin{pgfscope}%
\pgfpathrectangle{\pgfqpoint{0.041670in}{0.041670in}}{\pgfqpoint{2.216660in}{2.216660in}}%
\pgfusepath{clip}%
\pgfsetbuttcap%
\pgfsetroundjoin%
\definecolor{currentfill}{rgb}{0.268510,0.009605,0.335427}%
\pgfsetfillcolor{currentfill}%
\pgfsetlinewidth{0.000000pt}%
\definecolor{currentstroke}{rgb}{0.000000,0.000000,0.000000}%
\pgfsetstrokecolor{currentstroke}%
\pgfsetdash{}{0pt}%
\pgfpathmoveto{\pgfqpoint{1.345738in}{0.561550in}}%
\pgfpathlineto{\pgfqpoint{1.346831in}{0.560829in}}%
\pgfpathlineto{\pgfqpoint{1.347928in}{0.560392in}}%
\pgfpathlineto{\pgfqpoint{1.349027in}{0.560242in}}%
\pgfpathlineto{\pgfqpoint{1.350130in}{0.560385in}}%
\pgfpathlineto{\pgfqpoint{1.330431in}{0.557626in}}%
\pgfpathlineto{\pgfqpoint{1.310561in}{0.555207in}}%
\pgfpathlineto{\pgfqpoint{1.290544in}{0.553131in}}%
\pgfpathlineto{\pgfqpoint{1.270401in}{0.551400in}}%
\pgfpathlineto{\pgfqpoint{1.269814in}{0.551315in}}%
\pgfpathlineto{\pgfqpoint{1.269229in}{0.551523in}}%
\pgfpathlineto{\pgfqpoint{1.268646in}{0.552020in}}%
\pgfpathlineto{\pgfqpoint{1.268064in}{0.552798in}}%
\pgfpathlineto{\pgfqpoint{1.287688in}{0.554484in}}%
\pgfpathlineto{\pgfqpoint{1.307189in}{0.556507in}}%
\pgfpathlineto{\pgfqpoint{1.326546in}{0.558863in}}%
\pgfpathlineto{\pgfqpoint{1.345738in}{0.561550in}}%
\pgfpathclose%
\pgfusepath{fill}%
\end{pgfscope}%
\begin{pgfscope}%
\pgfpathrectangle{\pgfqpoint{0.041670in}{0.041670in}}{\pgfqpoint{2.216660in}{2.216660in}}%
\pgfusepath{clip}%
\pgfsetbuttcap%
\pgfsetroundjoin%
\definecolor{currentfill}{rgb}{0.272594,0.025563,0.353093}%
\pgfsetfillcolor{currentfill}%
\pgfsetlinewidth{0.000000pt}%
\definecolor{currentstroke}{rgb}{0.000000,0.000000,0.000000}%
\pgfsetstrokecolor{currentstroke}%
\pgfsetdash{}{0pt}%
\pgfpathmoveto{\pgfqpoint{1.426784in}{0.574738in}}%
\pgfpathlineto{\pgfqpoint{1.428385in}{0.575273in}}%
\pgfpathlineto{\pgfqpoint{1.429992in}{0.576111in}}%
\pgfpathlineto{\pgfqpoint{1.431603in}{0.577260in}}%
\pgfpathlineto{\pgfqpoint{1.433220in}{0.578723in}}%
\pgfpathlineto{\pgfqpoint{1.413933in}{0.574542in}}%
\pgfpathlineto{\pgfqpoint{1.394383in}{0.570691in}}%
\pgfpathlineto{\pgfqpoint{1.374589in}{0.567177in}}%
\pgfpathlineto{\pgfqpoint{1.354575in}{0.564003in}}%
\pgfpathlineto{\pgfqpoint{1.353458in}{0.562630in}}%
\pgfpathlineto{\pgfqpoint{1.352345in}{0.561574in}}%
\pgfpathlineto{\pgfqpoint{1.351236in}{0.560827in}}%
\pgfpathlineto{\pgfqpoint{1.350130in}{0.560385in}}%
\pgfpathlineto{\pgfqpoint{1.369636in}{0.563479in}}%
\pgfpathlineto{\pgfqpoint{1.388928in}{0.566906in}}%
\pgfpathlineto{\pgfqpoint{1.407985in}{0.570661in}}%
\pgfpathlineto{\pgfqpoint{1.426784in}{0.574738in}}%
\pgfpathclose%
\pgfusepath{fill}%
\end{pgfscope}%
\begin{pgfscope}%
\pgfpathrectangle{\pgfqpoint{0.041670in}{0.041670in}}{\pgfqpoint{2.216660in}{2.216660in}}%
\pgfusepath{clip}%
\pgfsetbuttcap%
\pgfsetroundjoin%
\definecolor{currentfill}{rgb}{0.277941,0.056324,0.381191}%
\pgfsetfillcolor{currentfill}%
\pgfsetlinewidth{0.000000pt}%
\definecolor{currentstroke}{rgb}{0.000000,0.000000,0.000000}%
\pgfsetstrokecolor{currentstroke}%
\pgfsetdash{}{0pt}%
\pgfpathmoveto{\pgfqpoint{0.943820in}{0.574990in}}%
\pgfpathlineto{\pgfqpoint{0.942307in}{0.576752in}}%
\pgfpathlineto{\pgfqpoint{0.940789in}{0.578841in}}%
\pgfpathlineto{\pgfqpoint{0.939265in}{0.581264in}}%
\pgfpathlineto{\pgfqpoint{0.937735in}{0.584026in}}%
\pgfpathlineto{\pgfqpoint{0.917987in}{0.588347in}}%
\pgfpathlineto{\pgfqpoint{0.898536in}{0.593002in}}%
\pgfpathlineto{\pgfqpoint{0.879402in}{0.597985in}}%
\pgfpathlineto{\pgfqpoint{0.860607in}{0.603288in}}%
\pgfpathlineto{\pgfqpoint{0.862618in}{0.600412in}}%
\pgfpathlineto{\pgfqpoint{0.864622in}{0.597875in}}%
\pgfpathlineto{\pgfqpoint{0.866619in}{0.595670in}}%
\pgfpathlineto{\pgfqpoint{0.868609in}{0.593792in}}%
\pgfpathlineto{\pgfqpoint{0.886938in}{0.588615in}}%
\pgfpathlineto{\pgfqpoint{0.905597in}{0.583752in}}%
\pgfpathlineto{\pgfqpoint{0.924564in}{0.579208in}}%
\pgfpathlineto{\pgfqpoint{0.943820in}{0.574990in}}%
\pgfpathclose%
\pgfusepath{fill}%
\end{pgfscope}%
\begin{pgfscope}%
\pgfpathrectangle{\pgfqpoint{0.041670in}{0.041670in}}{\pgfqpoint{2.216660in}{2.216660in}}%
\pgfusepath{clip}%
\pgfsetbuttcap%
\pgfsetroundjoin%
\definecolor{currentfill}{rgb}{0.276194,0.190074,0.493001}%
\pgfsetfillcolor{currentfill}%
\pgfsetlinewidth{0.000000pt}%
\definecolor{currentstroke}{rgb}{0.000000,0.000000,0.000000}%
\pgfsetstrokecolor{currentstroke}%
\pgfsetdash{}{0pt}%
\pgfpathmoveto{\pgfqpoint{0.769250in}{0.664643in}}%
\pgfpathlineto{\pgfqpoint{0.766707in}{0.670988in}}%
\pgfpathlineto{\pgfqpoint{0.764153in}{0.677735in}}%
\pgfpathlineto{\pgfqpoint{0.761588in}{0.684892in}}%
\pgfpathlineto{\pgfqpoint{0.759011in}{0.692466in}}%
\pgfpathlineto{\pgfqpoint{0.740937in}{0.699729in}}%
\pgfpathlineto{\pgfqpoint{0.723355in}{0.707287in}}%
\pgfpathlineto{\pgfqpoint{0.706282in}{0.715128in}}%
\pgfpathlineto{\pgfqpoint{0.689737in}{0.723244in}}%
\pgfpathlineto{\pgfqpoint{0.692726in}{0.715514in}}%
\pgfpathlineto{\pgfqpoint{0.695701in}{0.708199in}}%
\pgfpathlineto{\pgfqpoint{0.698663in}{0.701293in}}%
\pgfpathlineto{\pgfqpoint{0.701612in}{0.694788in}}%
\pgfpathlineto{\pgfqpoint{0.717769in}{0.686838in}}%
\pgfpathlineto{\pgfqpoint{0.734439in}{0.679158in}}%
\pgfpathlineto{\pgfqpoint{0.751605in}{0.671757in}}%
\pgfpathlineto{\pgfqpoint{0.769250in}{0.664643in}}%
\pgfpathclose%
\pgfusepath{fill}%
\end{pgfscope}%
\begin{pgfscope}%
\pgfpathrectangle{\pgfqpoint{0.041670in}{0.041670in}}{\pgfqpoint{2.216660in}{2.216660in}}%
\pgfusepath{clip}%
\pgfsetbuttcap%
\pgfsetroundjoin%
\definecolor{currentfill}{rgb}{0.233603,0.313828,0.543914}%
\pgfsetfillcolor{currentfill}%
\pgfsetlinewidth{0.000000pt}%
\definecolor{currentstroke}{rgb}{0.000000,0.000000,0.000000}%
\pgfsetstrokecolor{currentstroke}%
\pgfsetdash{}{0pt}%
\pgfpathmoveto{\pgfqpoint{1.756759in}{0.802609in}}%
\pgfpathlineto{\pgfqpoint{1.760254in}{0.812733in}}%
\pgfpathlineto{\pgfqpoint{1.763768in}{0.823316in}}%
\pgfpathlineto{\pgfqpoint{1.767299in}{0.834365in}}%
\pgfpathlineto{\pgfqpoint{1.770850in}{0.845887in}}%
\pgfpathlineto{\pgfqpoint{1.756462in}{0.836220in}}%
\pgfpathlineto{\pgfqpoint{1.741441in}{0.826784in}}%
\pgfpathlineto{\pgfqpoint{1.725800in}{0.817590in}}%
\pgfpathlineto{\pgfqpoint{1.709555in}{0.808650in}}%
\pgfpathlineto{\pgfqpoint{1.706356in}{0.797294in}}%
\pgfpathlineto{\pgfqpoint{1.703175in}{0.786413in}}%
\pgfpathlineto{\pgfqpoint{1.700011in}{0.776001in}}%
\pgfpathlineto{\pgfqpoint{1.696862in}{0.766048in}}%
\pgfpathlineto{\pgfqpoint{1.712734in}{0.774824in}}%
\pgfpathlineto{\pgfqpoint{1.728017in}{0.783851in}}%
\pgfpathlineto{\pgfqpoint{1.742697in}{0.793116in}}%
\pgfpathlineto{\pgfqpoint{1.756759in}{0.802609in}}%
\pgfpathclose%
\pgfusepath{fill}%
\end{pgfscope}%
\begin{pgfscope}%
\pgfpathrectangle{\pgfqpoint{0.041670in}{0.041670in}}{\pgfqpoint{2.216660in}{2.216660in}}%
\pgfusepath{clip}%
\pgfsetbuttcap%
\pgfsetroundjoin%
\definecolor{currentfill}{rgb}{0.268510,0.009605,0.335427}%
\pgfsetfillcolor{currentfill}%
\pgfsetlinewidth{0.000000pt}%
\definecolor{currentstroke}{rgb}{0.000000,0.000000,0.000000}%
\pgfsetstrokecolor{currentstroke}%
\pgfsetdash{}{0pt}%
\pgfpathmoveto{\pgfqpoint{1.109372in}{0.551584in}}%
\pgfpathlineto{\pgfqpoint{1.108906in}{0.550797in}}%
\pgfpathlineto{\pgfqpoint{1.108439in}{0.550292in}}%
\pgfpathlineto{\pgfqpoint{1.107970in}{0.550076in}}%
\pgfpathlineto{\pgfqpoint{1.107500in}{0.550153in}}%
\pgfpathlineto{\pgfqpoint{1.087265in}{0.551575in}}%
\pgfpathlineto{\pgfqpoint{1.067135in}{0.553345in}}%
\pgfpathlineto{\pgfqpoint{1.047133in}{0.555459in}}%
\pgfpathlineto{\pgfqpoint{1.027281in}{0.557916in}}%
\pgfpathlineto{\pgfqpoint{1.028271in}{0.557789in}}%
\pgfpathlineto{\pgfqpoint{1.029258in}{0.557955in}}%
\pgfpathlineto{\pgfqpoint{1.030241in}{0.558409in}}%
\pgfpathlineto{\pgfqpoint{1.031222in}{0.559145in}}%
\pgfpathlineto{\pgfqpoint{1.050563in}{0.556752in}}%
\pgfpathlineto{\pgfqpoint{1.070049in}{0.554693in}}%
\pgfpathlineto{\pgfqpoint{1.089660in}{0.552969in}}%
\pgfpathlineto{\pgfqpoint{1.109372in}{0.551584in}}%
\pgfpathclose%
\pgfusepath{fill}%
\end{pgfscope}%
\begin{pgfscope}%
\pgfpathrectangle{\pgfqpoint{0.041670in}{0.041670in}}{\pgfqpoint{2.216660in}{2.216660in}}%
\pgfusepath{clip}%
\pgfsetbuttcap%
\pgfsetroundjoin%
\definecolor{currentfill}{rgb}{0.272594,0.025563,0.353093}%
\pgfsetfillcolor{currentfill}%
\pgfsetlinewidth{0.000000pt}%
\definecolor{currentstroke}{rgb}{0.000000,0.000000,0.000000}%
\pgfsetstrokecolor{currentstroke}%
\pgfsetdash{}{0pt}%
\pgfpathmoveto{\pgfqpoint{1.027281in}{0.557916in}}%
\pgfpathlineto{\pgfqpoint{1.026289in}{0.558342in}}%
\pgfpathlineto{\pgfqpoint{1.025293in}{0.559073in}}%
\pgfpathlineto{\pgfqpoint{1.024294in}{0.560114in}}%
\pgfpathlineto{\pgfqpoint{1.023292in}{0.561471in}}%
\pgfpathlineto{\pgfqpoint{1.003101in}{0.564338in}}%
\pgfpathlineto{\pgfqpoint{0.983110in}{0.567550in}}%
\pgfpathlineto{\pgfqpoint{0.963342in}{0.571102in}}%
\pgfpathlineto{\pgfqpoint{0.943820in}{0.574990in}}%
\pgfpathlineto{\pgfqpoint{0.945328in}{0.573549in}}%
\pgfpathlineto{\pgfqpoint{0.946831in}{0.572424in}}%
\pgfpathlineto{\pgfqpoint{0.948329in}{0.571609in}}%
\pgfpathlineto{\pgfqpoint{0.949823in}{0.571098in}}%
\pgfpathlineto{\pgfqpoint{0.968852in}{0.567307in}}%
\pgfpathlineto{\pgfqpoint{0.988119in}{0.563844in}}%
\pgfpathlineto{\pgfqpoint{1.007603in}{0.560712in}}%
\pgfpathlineto{\pgfqpoint{1.027281in}{0.557916in}}%
\pgfpathclose%
\pgfusepath{fill}%
\end{pgfscope}%
\begin{pgfscope}%
\pgfpathrectangle{\pgfqpoint{0.041670in}{0.041670in}}{\pgfqpoint{2.216660in}{2.216660in}}%
\pgfusepath{clip}%
\pgfsetbuttcap%
\pgfsetroundjoin%
\definecolor{currentfill}{rgb}{0.282884,0.135920,0.453427}%
\pgfsetfillcolor{currentfill}%
\pgfsetlinewidth{0.000000pt}%
\definecolor{currentstroke}{rgb}{0.000000,0.000000,0.000000}%
\pgfsetstrokecolor{currentstroke}%
\pgfsetdash{}{0pt}%
\pgfpathmoveto{\pgfqpoint{1.595929in}{0.649326in}}%
\pgfpathlineto{\pgfqpoint{1.598523in}{0.654163in}}%
\pgfpathlineto{\pgfqpoint{1.601127in}{0.659375in}}%
\pgfpathlineto{\pgfqpoint{1.603742in}{0.664969in}}%
\pgfpathlineto{\pgfqpoint{1.606368in}{0.670952in}}%
\pgfpathlineto{\pgfqpoint{1.588671in}{0.663871in}}%
\pgfpathlineto{\pgfqpoint{1.570516in}{0.657087in}}%
\pgfpathlineto{\pgfqpoint{1.551923in}{0.650610in}}%
\pgfpathlineto{\pgfqpoint{1.532911in}{0.644447in}}%
\pgfpathlineto{\pgfqpoint{1.530729in}{0.638606in}}%
\pgfpathlineto{\pgfqpoint{1.528558in}{0.633156in}}%
\pgfpathlineto{\pgfqpoint{1.526395in}{0.628088in}}%
\pgfpathlineto{\pgfqpoint{1.524241in}{0.623397in}}%
\pgfpathlineto{\pgfqpoint{1.542793in}{0.629426in}}%
\pgfpathlineto{\pgfqpoint{1.560938in}{0.635762in}}%
\pgfpathlineto{\pgfqpoint{1.578657in}{0.642398in}}%
\pgfpathlineto{\pgfqpoint{1.595929in}{0.649326in}}%
\pgfpathclose%
\pgfusepath{fill}%
\end{pgfscope}%
\begin{pgfscope}%
\pgfpathrectangle{\pgfqpoint{0.041670in}{0.041670in}}{\pgfqpoint{2.216660in}{2.216660in}}%
\pgfusepath{clip}%
\pgfsetbuttcap%
\pgfsetroundjoin%
\definecolor{currentfill}{rgb}{0.268510,0.009605,0.335427}%
\pgfsetfillcolor{currentfill}%
\pgfsetlinewidth{0.000000pt}%
\definecolor{currentstroke}{rgb}{0.000000,0.000000,0.000000}%
\pgfsetstrokecolor{currentstroke}%
\pgfsetdash{}{0pt}%
\pgfpathmoveto{\pgfqpoint{1.268064in}{0.552798in}}%
\pgfpathlineto{\pgfqpoint{1.268646in}{0.552020in}}%
\pgfpathlineto{\pgfqpoint{1.269229in}{0.551523in}}%
\pgfpathlineto{\pgfqpoint{1.269814in}{0.551315in}}%
\pgfpathlineto{\pgfqpoint{1.270401in}{0.551400in}}%
\pgfpathlineto{\pgfqpoint{1.250156in}{0.550017in}}%
\pgfpathlineto{\pgfqpoint{1.229830in}{0.548982in}}%
\pgfpathlineto{\pgfqpoint{1.209448in}{0.548299in}}%
\pgfpathlineto{\pgfqpoint{1.189033in}{0.547966in}}%
\pgfpathlineto{\pgfqpoint{1.188974in}{0.547903in}}%
\pgfpathlineto{\pgfqpoint{1.188915in}{0.548134in}}%
\pgfpathlineto{\pgfqpoint{1.188857in}{0.548653in}}%
\pgfpathlineto{\pgfqpoint{1.188798in}{0.549454in}}%
\pgfpathlineto{\pgfqpoint{1.208686in}{0.549778in}}%
\pgfpathlineto{\pgfqpoint{1.228541in}{0.550444in}}%
\pgfpathlineto{\pgfqpoint{1.248342in}{0.551451in}}%
\pgfpathlineto{\pgfqpoint{1.268064in}{0.552798in}}%
\pgfpathclose%
\pgfusepath{fill}%
\end{pgfscope}%
\begin{pgfscope}%
\pgfpathrectangle{\pgfqpoint{0.041670in}{0.041670in}}{\pgfqpoint{2.216660in}{2.216660in}}%
\pgfusepath{clip}%
\pgfsetbuttcap%
\pgfsetroundjoin%
\definecolor{currentfill}{rgb}{0.268510,0.009605,0.335427}%
\pgfsetfillcolor{currentfill}%
\pgfsetlinewidth{0.000000pt}%
\definecolor{currentstroke}{rgb}{0.000000,0.000000,0.000000}%
\pgfsetstrokecolor{currentstroke}%
\pgfsetdash{}{0pt}%
\pgfpathmoveto{\pgfqpoint{1.188798in}{0.549454in}}%
\pgfpathlineto{\pgfqpoint{1.188857in}{0.548653in}}%
\pgfpathlineto{\pgfqpoint{1.188915in}{0.548134in}}%
\pgfpathlineto{\pgfqpoint{1.188974in}{0.547903in}}%
\pgfpathlineto{\pgfqpoint{1.189033in}{0.547966in}}%
\pgfpathlineto{\pgfqpoint{1.168607in}{0.547986in}}%
\pgfpathlineto{\pgfqpoint{1.148195in}{0.548357in}}%
\pgfpathlineto{\pgfqpoint{1.127818in}{0.549080in}}%
\pgfpathlineto{\pgfqpoint{1.107500in}{0.550153in}}%
\pgfpathlineto{\pgfqpoint{1.107970in}{0.550076in}}%
\pgfpathlineto{\pgfqpoint{1.108439in}{0.550292in}}%
\pgfpathlineto{\pgfqpoint{1.108906in}{0.550797in}}%
\pgfpathlineto{\pgfqpoint{1.109372in}{0.551584in}}%
\pgfpathlineto{\pgfqpoint{1.129165in}{0.550539in}}%
\pgfpathlineto{\pgfqpoint{1.149015in}{0.549835in}}%
\pgfpathlineto{\pgfqpoint{1.168901in}{0.549473in}}%
\pgfpathlineto{\pgfqpoint{1.188798in}{0.549454in}}%
\pgfpathclose%
\pgfusepath{fill}%
\end{pgfscope}%
\begin{pgfscope}%
\pgfpathrectangle{\pgfqpoint{0.041670in}{0.041670in}}{\pgfqpoint{2.216660in}{2.216660in}}%
\pgfusepath{clip}%
\pgfsetbuttcap%
\pgfsetroundjoin%
\definecolor{currentfill}{rgb}{0.172719,0.448791,0.557885}%
\pgfsetfillcolor{currentfill}%
\pgfsetlinewidth{0.000000pt}%
\definecolor{currentstroke}{rgb}{0.000000,0.000000,0.000000}%
\pgfsetstrokecolor{currentstroke}%
\pgfsetdash{}{0pt}%
\pgfpathmoveto{\pgfqpoint{1.837360in}{0.938318in}}%
\pgfpathlineto{\pgfqpoint{1.841303in}{0.952500in}}%
\pgfpathlineto{\pgfqpoint{1.845268in}{0.967205in}}%
\pgfpathlineto{\pgfqpoint{1.849258in}{0.982440in}}%
\pgfpathlineto{\pgfqpoint{1.837068in}{0.971662in}}%
\pgfpathlineto{\pgfqpoint{1.824166in}{0.961072in}}%
\pgfpathlineto{\pgfqpoint{1.810561in}{0.950682in}}%
\pgfpathlineto{\pgfqpoint{1.796266in}{0.940505in}}%
\pgfpathlineto{\pgfqpoint{1.792572in}{0.925432in}}%
\pgfpathlineto{\pgfqpoint{1.788899in}{0.910893in}}%
\pgfpathlineto{\pgfqpoint{1.785248in}{0.896878in}}%
\pgfpathlineto{\pgfqpoint{1.799302in}{0.906934in}}%
\pgfpathlineto{\pgfqpoint{1.812680in}{0.917200in}}%
\pgfpathlineto{\pgfqpoint{1.825370in}{0.927666in}}%
\pgfpathlineto{\pgfqpoint{1.837360in}{0.938318in}}%
\pgfpathclose%
\pgfusepath{fill}%
\end{pgfscope}%
\begin{pgfscope}%
\pgfpathrectangle{\pgfqpoint{0.041670in}{0.041670in}}{\pgfqpoint{2.216660in}{2.216660in}}%
\pgfusepath{clip}%
\pgfsetbuttcap%
\pgfsetroundjoin%
\definecolor{currentfill}{rgb}{0.233603,0.313828,0.543914}%
\pgfsetfillcolor{currentfill}%
\pgfsetlinewidth{0.000000pt}%
\definecolor{currentstroke}{rgb}{0.000000,0.000000,0.000000}%
\pgfsetstrokecolor{currentstroke}%
\pgfsetdash{}{0pt}%
\pgfpathmoveto{\pgfqpoint{0.677637in}{0.758464in}}%
\pgfpathlineto{\pgfqpoint{0.674575in}{0.768381in}}%
\pgfpathlineto{\pgfqpoint{0.671496in}{0.778758in}}%
\pgfpathlineto{\pgfqpoint{0.668401in}{0.789604in}}%
\pgfpathlineto{\pgfqpoint{0.665290in}{0.800925in}}%
\pgfpathlineto{\pgfqpoint{0.648521in}{0.809631in}}%
\pgfpathlineto{\pgfqpoint{0.632342in}{0.818599in}}%
\pgfpathlineto{\pgfqpoint{0.616769in}{0.827820in}}%
\pgfpathlineto{\pgfqpoint{0.601818in}{0.837283in}}%
\pgfpathlineto{\pgfqpoint{0.605295in}{0.825798in}}%
\pgfpathlineto{\pgfqpoint{0.608755in}{0.814789in}}%
\pgfpathlineto{\pgfqpoint{0.612196in}{0.804245in}}%
\pgfpathlineto{\pgfqpoint{0.615620in}{0.794160in}}%
\pgfpathlineto{\pgfqpoint{0.630232in}{0.784869in}}%
\pgfpathlineto{\pgfqpoint{0.645448in}{0.775815in}}%
\pgfpathlineto{\pgfqpoint{0.661255in}{0.767010in}}%
\pgfpathlineto{\pgfqpoint{0.677637in}{0.758464in}}%
\pgfpathclose%
\pgfusepath{fill}%
\end{pgfscope}%
\begin{pgfscope}%
\pgfpathrectangle{\pgfqpoint{0.041670in}{0.041670in}}{\pgfqpoint{2.216660in}{2.216660in}}%
\pgfusepath{clip}%
\pgfsetbuttcap%
\pgfsetroundjoin%
\definecolor{currentfill}{rgb}{0.282884,0.135920,0.453427}%
\pgfsetfillcolor{currentfill}%
\pgfsetlinewidth{0.000000pt}%
\definecolor{currentstroke}{rgb}{0.000000,0.000000,0.000000}%
\pgfsetstrokecolor{currentstroke}%
\pgfsetdash{}{0pt}%
\pgfpathmoveto{\pgfqpoint{0.852486in}{0.618302in}}%
\pgfpathlineto{\pgfqpoint{0.850436in}{0.622965in}}%
\pgfpathlineto{\pgfqpoint{0.848377in}{0.628004in}}%
\pgfpathlineto{\pgfqpoint{0.846310in}{0.633426in}}%
\pgfpathlineto{\pgfqpoint{0.844234in}{0.639238in}}%
\pgfpathlineto{\pgfqpoint{0.824866in}{0.645116in}}%
\pgfpathlineto{\pgfqpoint{0.805900in}{0.651314in}}%
\pgfpathlineto{\pgfqpoint{0.787354in}{0.657826in}}%
\pgfpathlineto{\pgfqpoint{0.769250in}{0.664643in}}%
\pgfpathlineto{\pgfqpoint{0.771781in}{0.658694in}}%
\pgfpathlineto{\pgfqpoint{0.774302in}{0.653134in}}%
\pgfpathlineto{\pgfqpoint{0.776812in}{0.647956in}}%
\pgfpathlineto{\pgfqpoint{0.779312in}{0.643154in}}%
\pgfpathlineto{\pgfqpoint{0.796981in}{0.636485in}}%
\pgfpathlineto{\pgfqpoint{0.815080in}{0.630115in}}%
\pgfpathlineto{\pgfqpoint{0.833588in}{0.624051in}}%
\pgfpathlineto{\pgfqpoint{0.852486in}{0.618302in}}%
\pgfpathclose%
\pgfusepath{fill}%
\end{pgfscope}%
\begin{pgfscope}%
\pgfpathrectangle{\pgfqpoint{0.041670in}{0.041670in}}{\pgfqpoint{2.216660in}{2.216660in}}%
\pgfusepath{clip}%
\pgfsetbuttcap%
\pgfsetroundjoin%
\definecolor{currentfill}{rgb}{0.282327,0.094955,0.417331}%
\pgfsetfillcolor{currentfill}%
\pgfsetlinewidth{0.000000pt}%
\definecolor{currentstroke}{rgb}{0.000000,0.000000,0.000000}%
\pgfsetstrokecolor{currentstroke}%
\pgfsetdash{}{0pt}%
\pgfpathmoveto{\pgfqpoint{1.515709in}{0.608266in}}%
\pgfpathlineto{\pgfqpoint{1.517829in}{0.611517in}}%
\pgfpathlineto{\pgfqpoint{1.519958in}{0.615118in}}%
\pgfpathlineto{\pgfqpoint{1.522095in}{0.619076in}}%
\pgfpathlineto{\pgfqpoint{1.524241in}{0.623397in}}%
\pgfpathlineto{\pgfqpoint{1.505300in}{0.617683in}}%
\pgfpathlineto{\pgfqpoint{1.485993in}{0.612292in}}%
\pgfpathlineto{\pgfqpoint{1.466340in}{0.607230in}}%
\pgfpathlineto{\pgfqpoint{1.446362in}{0.602503in}}%
\pgfpathlineto{\pgfqpoint{1.444698in}{0.598300in}}%
\pgfpathlineto{\pgfqpoint{1.443039in}{0.594461in}}%
\pgfpathlineto{\pgfqpoint{1.441388in}{0.590980in}}%
\pgfpathlineto{\pgfqpoint{1.439742in}{0.587851in}}%
\pgfpathlineto{\pgfqpoint{1.459228in}{0.592469in}}%
\pgfpathlineto{\pgfqpoint{1.478398in}{0.597415in}}%
\pgfpathlineto{\pgfqpoint{1.497232in}{0.602683in}}%
\pgfpathlineto{\pgfqpoint{1.515709in}{0.608266in}}%
\pgfpathclose%
\pgfusepath{fill}%
\end{pgfscope}%
\begin{pgfscope}%
\pgfpathrectangle{\pgfqpoint{0.041670in}{0.041670in}}{\pgfqpoint{2.216660in}{2.216660in}}%
\pgfusepath{clip}%
\pgfsetbuttcap%
\pgfsetroundjoin%
\definecolor{currentfill}{rgb}{0.272594,0.025563,0.353093}%
\pgfsetfillcolor{currentfill}%
\pgfsetlinewidth{0.000000pt}%
\definecolor{currentstroke}{rgb}{0.000000,0.000000,0.000000}%
\pgfsetstrokecolor{currentstroke}%
\pgfsetdash{}{0pt}%
\pgfpathmoveto{\pgfqpoint{1.350130in}{0.560385in}}%
\pgfpathlineto{\pgfqpoint{1.351236in}{0.560827in}}%
\pgfpathlineto{\pgfqpoint{1.352345in}{0.561574in}}%
\pgfpathlineto{\pgfqpoint{1.353458in}{0.562630in}}%
\pgfpathlineto{\pgfqpoint{1.354575in}{0.564003in}}%
\pgfpathlineto{\pgfqpoint{1.334363in}{0.561174in}}%
\pgfpathlineto{\pgfqpoint{1.313976in}{0.558693in}}%
\pgfpathlineto{\pgfqpoint{1.293436in}{0.556563in}}%
\pgfpathlineto{\pgfqpoint{1.272767in}{0.554787in}}%
\pgfpathlineto{\pgfqpoint{1.272172in}{0.553472in}}%
\pgfpathlineto{\pgfqpoint{1.271580in}{0.552473in}}%
\pgfpathlineto{\pgfqpoint{1.270990in}{0.551784in}}%
\pgfpathlineto{\pgfqpoint{1.270401in}{0.551400in}}%
\pgfpathlineto{\pgfqpoint{1.290544in}{0.553131in}}%
\pgfpathlineto{\pgfqpoint{1.310561in}{0.555207in}}%
\pgfpathlineto{\pgfqpoint{1.330431in}{0.557626in}}%
\pgfpathlineto{\pgfqpoint{1.350130in}{0.560385in}}%
\pgfpathclose%
\pgfusepath{fill}%
\end{pgfscope}%
\begin{pgfscope}%
\pgfpathrectangle{\pgfqpoint{0.041670in}{0.041670in}}{\pgfqpoint{2.216660in}{2.216660in}}%
\pgfusepath{clip}%
\pgfsetbuttcap%
\pgfsetroundjoin%
\definecolor{currentfill}{rgb}{0.277941,0.056324,0.381191}%
\pgfsetfillcolor{currentfill}%
\pgfsetlinewidth{0.000000pt}%
\definecolor{currentstroke}{rgb}{0.000000,0.000000,0.000000}%
\pgfsetstrokecolor{currentstroke}%
\pgfsetdash{}{0pt}%
\pgfpathmoveto{\pgfqpoint{1.433220in}{0.578723in}}%
\pgfpathlineto{\pgfqpoint{1.434842in}{0.580509in}}%
\pgfpathlineto{\pgfqpoint{1.436469in}{0.582621in}}%
\pgfpathlineto{\pgfqpoint{1.438103in}{0.585066in}}%
\pgfpathlineto{\pgfqpoint{1.439742in}{0.587851in}}%
\pgfpathlineto{\pgfqpoint{1.419963in}{0.583566in}}%
\pgfpathlineto{\pgfqpoint{1.399911in}{0.579621in}}%
\pgfpathlineto{\pgfqpoint{1.379610in}{0.576020in}}%
\pgfpathlineto{\pgfqpoint{1.359081in}{0.572768in}}%
\pgfpathlineto{\pgfqpoint{1.357949in}{0.570074in}}%
\pgfpathlineto{\pgfqpoint{1.356820in}{0.567719in}}%
\pgfpathlineto{\pgfqpoint{1.355696in}{0.565697in}}%
\pgfpathlineto{\pgfqpoint{1.354575in}{0.564003in}}%
\pgfpathlineto{\pgfqpoint{1.374589in}{0.567177in}}%
\pgfpathlineto{\pgfqpoint{1.394383in}{0.570691in}}%
\pgfpathlineto{\pgfqpoint{1.413933in}{0.574542in}}%
\pgfpathlineto{\pgfqpoint{1.433220in}{0.578723in}}%
\pgfpathclose%
\pgfusepath{fill}%
\end{pgfscope}%
\begin{pgfscope}%
\pgfpathrectangle{\pgfqpoint{0.041670in}{0.041670in}}{\pgfqpoint{2.216660in}{2.216660in}}%
\pgfusepath{clip}%
\pgfsetbuttcap%
\pgfsetroundjoin%
\definecolor{currentfill}{rgb}{0.272594,0.025563,0.353093}%
\pgfsetfillcolor{currentfill}%
\pgfsetlinewidth{0.000000pt}%
\definecolor{currentstroke}{rgb}{0.000000,0.000000,0.000000}%
\pgfsetstrokecolor{currentstroke}%
\pgfsetdash{}{0pt}%
\pgfpathmoveto{\pgfqpoint{1.107500in}{0.550153in}}%
\pgfpathlineto{\pgfqpoint{1.107029in}{0.550529in}}%
\pgfpathlineto{\pgfqpoint{1.106556in}{0.551210in}}%
\pgfpathlineto{\pgfqpoint{1.106081in}{0.552201in}}%
\pgfpathlineto{\pgfqpoint{1.105605in}{0.553508in}}%
\pgfpathlineto{\pgfqpoint{1.084841in}{0.554967in}}%
\pgfpathlineto{\pgfqpoint{1.064185in}{0.556782in}}%
\pgfpathlineto{\pgfqpoint{1.043661in}{0.558951in}}%
\pgfpathlineto{\pgfqpoint{1.023292in}{0.561471in}}%
\pgfpathlineto{\pgfqpoint{1.024294in}{0.560114in}}%
\pgfpathlineto{\pgfqpoint{1.025293in}{0.559073in}}%
\pgfpathlineto{\pgfqpoint{1.026289in}{0.558342in}}%
\pgfpathlineto{\pgfqpoint{1.027281in}{0.557916in}}%
\pgfpathlineto{\pgfqpoint{1.047133in}{0.555459in}}%
\pgfpathlineto{\pgfqpoint{1.067135in}{0.553345in}}%
\pgfpathlineto{\pgfqpoint{1.087265in}{0.551575in}}%
\pgfpathlineto{\pgfqpoint{1.107500in}{0.550153in}}%
\pgfpathclose%
\pgfusepath{fill}%
\end{pgfscope}%
\begin{pgfscope}%
\pgfpathrectangle{\pgfqpoint{0.041670in}{0.041670in}}{\pgfqpoint{2.216660in}{2.216660in}}%
\pgfusepath{clip}%
\pgfsetbuttcap%
\pgfsetroundjoin%
\definecolor{currentfill}{rgb}{0.260571,0.246922,0.522828}%
\pgfsetfillcolor{currentfill}%
\pgfsetlinewidth{0.000000pt}%
\definecolor{currentstroke}{rgb}{0.000000,0.000000,0.000000}%
\pgfsetstrokecolor{currentstroke}%
\pgfsetdash{}{0pt}%
\pgfpathmoveto{\pgfqpoint{1.684424in}{0.730680in}}%
\pgfpathlineto{\pgfqpoint{1.687511in}{0.738870in}}%
\pgfpathlineto{\pgfqpoint{1.690613in}{0.747490in}}%
\pgfpathlineto{\pgfqpoint{1.693730in}{0.756547in}}%
\pgfpathlineto{\pgfqpoint{1.696862in}{0.766048in}}%
\pgfpathlineto{\pgfqpoint{1.680418in}{0.757531in}}%
\pgfpathlineto{\pgfqpoint{1.663416in}{0.749285in}}%
\pgfpathlineto{\pgfqpoint{1.645875in}{0.741319in}}%
\pgfpathlineto{\pgfqpoint{1.627812in}{0.733644in}}%
\pgfpathlineto{\pgfqpoint{1.625086in}{0.724299in}}%
\pgfpathlineto{\pgfqpoint{1.622374in}{0.715399in}}%
\pgfpathlineto{\pgfqpoint{1.619675in}{0.706938in}}%
\pgfpathlineto{\pgfqpoint{1.616989in}{0.698908in}}%
\pgfpathlineto{\pgfqpoint{1.634626in}{0.706433in}}%
\pgfpathlineto{\pgfqpoint{1.651757in}{0.714243in}}%
\pgfpathlineto{\pgfqpoint{1.668361in}{0.722329in}}%
\pgfpathlineto{\pgfqpoint{1.684424in}{0.730680in}}%
\pgfpathclose%
\pgfusepath{fill}%
\end{pgfscope}%
\begin{pgfscope}%
\pgfpathrectangle{\pgfqpoint{0.041670in}{0.041670in}}{\pgfqpoint{2.216660in}{2.216660in}}%
\pgfusepath{clip}%
\pgfsetbuttcap%
\pgfsetroundjoin%
\definecolor{currentfill}{rgb}{0.282327,0.094955,0.417331}%
\pgfsetfillcolor{currentfill}%
\pgfsetlinewidth{0.000000pt}%
\definecolor{currentstroke}{rgb}{0.000000,0.000000,0.000000}%
\pgfsetstrokecolor{currentstroke}%
\pgfsetdash{}{0pt}%
\pgfpathmoveto{\pgfqpoint{0.937735in}{0.584026in}}%
\pgfpathlineto{\pgfqpoint{0.936200in}{0.587133in}}%
\pgfpathlineto{\pgfqpoint{0.934660in}{0.590591in}}%
\pgfpathlineto{\pgfqpoint{0.933113in}{0.594408in}}%
\pgfpathlineto{\pgfqpoint{0.931560in}{0.598588in}}%
\pgfpathlineto{\pgfqpoint{0.911312in}{0.603012in}}%
\pgfpathlineto{\pgfqpoint{0.891370in}{0.607776in}}%
\pgfpathlineto{\pgfqpoint{0.871754in}{0.612875in}}%
\pgfpathlineto{\pgfqpoint{0.852486in}{0.618302in}}%
\pgfpathlineto{\pgfqpoint{0.854528in}{0.614010in}}%
\pgfpathlineto{\pgfqpoint{0.856562in}{0.610081in}}%
\pgfpathlineto{\pgfqpoint{0.858588in}{0.606509in}}%
\pgfpathlineto{\pgfqpoint{0.860607in}{0.603288in}}%
\pgfpathlineto{\pgfqpoint{0.879402in}{0.597985in}}%
\pgfpathlineto{\pgfqpoint{0.898536in}{0.593002in}}%
\pgfpathlineto{\pgfqpoint{0.917987in}{0.588347in}}%
\pgfpathlineto{\pgfqpoint{0.937735in}{0.584026in}}%
\pgfpathclose%
\pgfusepath{fill}%
\end{pgfscope}%
\begin{pgfscope}%
\pgfpathrectangle{\pgfqpoint{0.041670in}{0.041670in}}{\pgfqpoint{2.216660in}{2.216660in}}%
\pgfusepath{clip}%
\pgfsetbuttcap%
\pgfsetroundjoin%
\definecolor{currentfill}{rgb}{0.277941,0.056324,0.381191}%
\pgfsetfillcolor{currentfill}%
\pgfsetlinewidth{0.000000pt}%
\definecolor{currentstroke}{rgb}{0.000000,0.000000,0.000000}%
\pgfsetstrokecolor{currentstroke}%
\pgfsetdash{}{0pt}%
\pgfpathmoveto{\pgfqpoint{1.023292in}{0.561471in}}%
\pgfpathlineto{\pgfqpoint{1.022286in}{0.563149in}}%
\pgfpathlineto{\pgfqpoint{1.021277in}{0.565155in}}%
\pgfpathlineto{\pgfqpoint{1.020264in}{0.567495in}}%
\pgfpathlineto{\pgfqpoint{1.019248in}{0.570174in}}%
\pgfpathlineto{\pgfqpoint{0.998537in}{0.573112in}}%
\pgfpathlineto{\pgfqpoint{0.978033in}{0.576403in}}%
\pgfpathlineto{\pgfqpoint{0.957758in}{0.580043in}}%
\pgfpathlineto{\pgfqpoint{0.937735in}{0.584026in}}%
\pgfpathlineto{\pgfqpoint{0.939265in}{0.581264in}}%
\pgfpathlineto{\pgfqpoint{0.940789in}{0.578841in}}%
\pgfpathlineto{\pgfqpoint{0.942307in}{0.576752in}}%
\pgfpathlineto{\pgfqpoint{0.943820in}{0.574990in}}%
\pgfpathlineto{\pgfqpoint{0.963342in}{0.571102in}}%
\pgfpathlineto{\pgfqpoint{0.983110in}{0.567550in}}%
\pgfpathlineto{\pgfqpoint{1.003101in}{0.564338in}}%
\pgfpathlineto{\pgfqpoint{1.023292in}{0.561471in}}%
\pgfpathclose%
\pgfusepath{fill}%
\end{pgfscope}%
\begin{pgfscope}%
\pgfpathrectangle{\pgfqpoint{0.041670in}{0.041670in}}{\pgfqpoint{2.216660in}{2.216660in}}%
\pgfusepath{clip}%
\pgfsetbuttcap%
\pgfsetroundjoin%
\definecolor{currentfill}{rgb}{0.272594,0.025563,0.353093}%
\pgfsetfillcolor{currentfill}%
\pgfsetlinewidth{0.000000pt}%
\definecolor{currentstroke}{rgb}{0.000000,0.000000,0.000000}%
\pgfsetstrokecolor{currentstroke}%
\pgfsetdash{}{0pt}%
\pgfpathmoveto{\pgfqpoint{1.270401in}{0.551400in}}%
\pgfpathlineto{\pgfqpoint{1.270990in}{0.551784in}}%
\pgfpathlineto{\pgfqpoint{1.271580in}{0.552473in}}%
\pgfpathlineto{\pgfqpoint{1.272172in}{0.553472in}}%
\pgfpathlineto{\pgfqpoint{1.272767in}{0.554787in}}%
\pgfpathlineto{\pgfqpoint{1.251992in}{0.553368in}}%
\pgfpathlineto{\pgfqpoint{1.231135in}{0.552307in}}%
\pgfpathlineto{\pgfqpoint{1.210220in}{0.551606in}}%
\pgfpathlineto{\pgfqpoint{1.189271in}{0.551265in}}%
\pgfpathlineto{\pgfqpoint{1.189211in}{0.549972in}}%
\pgfpathlineto{\pgfqpoint{1.189151in}{0.548995in}}%
\pgfpathlineto{\pgfqpoint{1.189092in}{0.548328in}}%
\pgfpathlineto{\pgfqpoint{1.189033in}{0.547966in}}%
\pgfpathlineto{\pgfqpoint{1.209448in}{0.548299in}}%
\pgfpathlineto{\pgfqpoint{1.229830in}{0.548982in}}%
\pgfpathlineto{\pgfqpoint{1.250156in}{0.550017in}}%
\pgfpathlineto{\pgfqpoint{1.270401in}{0.551400in}}%
\pgfpathclose%
\pgfusepath{fill}%
\end{pgfscope}%
\begin{pgfscope}%
\pgfpathrectangle{\pgfqpoint{0.041670in}{0.041670in}}{\pgfqpoint{2.216660in}{2.216660in}}%
\pgfusepath{clip}%
\pgfsetbuttcap%
\pgfsetroundjoin%
\definecolor{currentfill}{rgb}{0.172719,0.448791,0.557885}%
\pgfsetfillcolor{currentfill}%
\pgfsetlinewidth{0.000000pt}%
\definecolor{currentstroke}{rgb}{0.000000,0.000000,0.000000}%
\pgfsetstrokecolor{currentstroke}%
\pgfsetdash{}{0pt}%
\pgfpathmoveto{\pgfqpoint{0.587713in}{0.888125in}}%
\pgfpathlineto{\pgfqpoint{0.584136in}{0.902105in}}%
\pgfpathlineto{\pgfqpoint{0.580538in}{0.916609in}}%
\pgfpathlineto{\pgfqpoint{0.576919in}{0.931648in}}%
\pgfpathlineto{\pgfqpoint{0.562021in}{0.941625in}}%
\pgfpathlineto{\pgfqpoint{0.547803in}{0.951826in}}%
\pgfpathlineto{\pgfqpoint{0.534276in}{0.962239in}}%
\pgfpathlineto{\pgfqpoint{0.521452in}{0.972850in}}%
\pgfpathlineto{\pgfqpoint{0.525382in}{0.957652in}}%
\pgfpathlineto{\pgfqpoint{0.529289in}{0.942985in}}%
\pgfpathlineto{\pgfqpoint{0.533173in}{0.928840in}}%
\pgfpathlineto{\pgfqpoint{0.545785in}{0.918354in}}%
\pgfpathlineto{\pgfqpoint{0.559087in}{0.908064in}}%
\pgfpathlineto{\pgfqpoint{0.573067in}{0.897984in}}%
\pgfpathlineto{\pgfqpoint{0.587713in}{0.888125in}}%
\pgfpathclose%
\pgfusepath{fill}%
\end{pgfscope}%
\begin{pgfscope}%
\pgfpathrectangle{\pgfqpoint{0.041670in}{0.041670in}}{\pgfqpoint{2.216660in}{2.216660in}}%
\pgfusepath{clip}%
\pgfsetbuttcap%
\pgfsetroundjoin%
\definecolor{currentfill}{rgb}{0.272594,0.025563,0.353093}%
\pgfsetfillcolor{currentfill}%
\pgfsetlinewidth{0.000000pt}%
\definecolor{currentstroke}{rgb}{0.000000,0.000000,0.000000}%
\pgfsetstrokecolor{currentstroke}%
\pgfsetdash{}{0pt}%
\pgfpathmoveto{\pgfqpoint{1.189033in}{0.547966in}}%
\pgfpathlineto{\pgfqpoint{1.189092in}{0.548328in}}%
\pgfpathlineto{\pgfqpoint{1.189151in}{0.548995in}}%
\pgfpathlineto{\pgfqpoint{1.189211in}{0.549972in}}%
\pgfpathlineto{\pgfqpoint{1.189271in}{0.551265in}}%
\pgfpathlineto{\pgfqpoint{1.168310in}{0.551285in}}%
\pgfpathlineto{\pgfqpoint{1.147363in}{0.551666in}}%
\pgfpathlineto{\pgfqpoint{1.126454in}{0.552408in}}%
\pgfpathlineto{\pgfqpoint{1.105605in}{0.553508in}}%
\pgfpathlineto{\pgfqpoint{1.106081in}{0.552201in}}%
\pgfpathlineto{\pgfqpoint{1.106556in}{0.551210in}}%
\pgfpathlineto{\pgfqpoint{1.107029in}{0.550529in}}%
\pgfpathlineto{\pgfqpoint{1.107500in}{0.550153in}}%
\pgfpathlineto{\pgfqpoint{1.127818in}{0.549080in}}%
\pgfpathlineto{\pgfqpoint{1.148195in}{0.548357in}}%
\pgfpathlineto{\pgfqpoint{1.168607in}{0.547986in}}%
\pgfpathlineto{\pgfqpoint{1.189033in}{0.547966in}}%
\pgfpathclose%
\pgfusepath{fill}%
\end{pgfscope}%
\begin{pgfscope}%
\pgfpathrectangle{\pgfqpoint{0.041670in}{0.041670in}}{\pgfqpoint{2.216660in}{2.216660in}}%
\pgfusepath{clip}%
\pgfsetbuttcap%
\pgfsetroundjoin%
\definecolor{currentfill}{rgb}{0.260571,0.246922,0.522828}%
\pgfsetfillcolor{currentfill}%
\pgfsetlinewidth{0.000000pt}%
\definecolor{currentstroke}{rgb}{0.000000,0.000000,0.000000}%
\pgfsetstrokecolor{currentstroke}%
\pgfsetdash{}{0pt}%
\pgfpathmoveto{\pgfqpoint{0.759011in}{0.692466in}}%
\pgfpathlineto{\pgfqpoint{0.756421in}{0.700464in}}%
\pgfpathlineto{\pgfqpoint{0.753820in}{0.708892in}}%
\pgfpathlineto{\pgfqpoint{0.751205in}{0.717760in}}%
\pgfpathlineto{\pgfqpoint{0.748577in}{0.727074in}}%
\pgfpathlineto{\pgfqpoint{0.730066in}{0.734482in}}%
\pgfpathlineto{\pgfqpoint{0.712060in}{0.742190in}}%
\pgfpathlineto{\pgfqpoint{0.694578in}{0.750187in}}%
\pgfpathlineto{\pgfqpoint{0.677637in}{0.758464in}}%
\pgfpathlineto{\pgfqpoint{0.680684in}{0.748999in}}%
\pgfpathlineto{\pgfqpoint{0.683716in}{0.739979in}}%
\pgfpathlineto{\pgfqpoint{0.686734in}{0.731397in}}%
\pgfpathlineto{\pgfqpoint{0.689737in}{0.723244in}}%
\pgfpathlineto{\pgfqpoint{0.706282in}{0.715128in}}%
\pgfpathlineto{\pgfqpoint{0.723355in}{0.707287in}}%
\pgfpathlineto{\pgfqpoint{0.740937in}{0.699729in}}%
\pgfpathlineto{\pgfqpoint{0.759011in}{0.692466in}}%
\pgfpathclose%
\pgfusepath{fill}%
\end{pgfscope}%
\begin{pgfscope}%
\pgfpathrectangle{\pgfqpoint{0.041670in}{0.041670in}}{\pgfqpoint{2.216660in}{2.216660in}}%
\pgfusepath{clip}%
\pgfsetbuttcap%
\pgfsetroundjoin%
\definecolor{currentfill}{rgb}{0.276194,0.190074,0.493001}%
\pgfsetfillcolor{currentfill}%
\pgfsetlinewidth{0.000000pt}%
\definecolor{currentstroke}{rgb}{0.000000,0.000000,0.000000}%
\pgfsetstrokecolor{currentstroke}%
\pgfsetdash{}{0pt}%
\pgfpathmoveto{\pgfqpoint{1.606368in}{0.670952in}}%
\pgfpathlineto{\pgfqpoint{1.609005in}{0.677330in}}%
\pgfpathlineto{\pgfqpoint{1.611654in}{0.684111in}}%
\pgfpathlineto{\pgfqpoint{1.614315in}{0.691301in}}%
\pgfpathlineto{\pgfqpoint{1.616989in}{0.698908in}}%
\pgfpathlineto{\pgfqpoint{1.598861in}{0.691677in}}%
\pgfpathlineto{\pgfqpoint{1.580263in}{0.684751in}}%
\pgfpathlineto{\pgfqpoint{1.561214in}{0.678136in}}%
\pgfpathlineto{\pgfqpoint{1.541734in}{0.671842in}}%
\pgfpathlineto{\pgfqpoint{1.539513in}{0.664374in}}%
\pgfpathlineto{\pgfqpoint{1.537302in}{0.657324in}}%
\pgfpathlineto{\pgfqpoint{1.535102in}{0.650684in}}%
\pgfpathlineto{\pgfqpoint{1.532911in}{0.644447in}}%
\pgfpathlineto{\pgfqpoint{1.551923in}{0.650610in}}%
\pgfpathlineto{\pgfqpoint{1.570516in}{0.657087in}}%
\pgfpathlineto{\pgfqpoint{1.588671in}{0.663871in}}%
\pgfpathlineto{\pgfqpoint{1.606368in}{0.670952in}}%
\pgfpathclose%
\pgfusepath{fill}%
\end{pgfscope}%
\begin{pgfscope}%
\pgfpathrectangle{\pgfqpoint{0.041670in}{0.041670in}}{\pgfqpoint{2.216660in}{2.216660in}}%
\pgfusepath{clip}%
\pgfsetbuttcap%
\pgfsetroundjoin%
\definecolor{currentfill}{rgb}{0.277941,0.056324,0.381191}%
\pgfsetfillcolor{currentfill}%
\pgfsetlinewidth{0.000000pt}%
\definecolor{currentstroke}{rgb}{0.000000,0.000000,0.000000}%
\pgfsetstrokecolor{currentstroke}%
\pgfsetdash{}{0pt}%
\pgfpathmoveto{\pgfqpoint{1.354575in}{0.564003in}}%
\pgfpathlineto{\pgfqpoint{1.355696in}{0.565697in}}%
\pgfpathlineto{\pgfqpoint{1.356820in}{0.567719in}}%
\pgfpathlineto{\pgfqpoint{1.357949in}{0.570074in}}%
\pgfpathlineto{\pgfqpoint{1.359081in}{0.572768in}}%
\pgfpathlineto{\pgfqpoint{1.338349in}{0.569869in}}%
\pgfpathlineto{\pgfqpoint{1.317437in}{0.567327in}}%
\pgfpathlineto{\pgfqpoint{1.296367in}{0.565145in}}%
\pgfpathlineto{\pgfqpoint{1.275165in}{0.563326in}}%
\pgfpathlineto{\pgfqpoint{1.274562in}{0.560688in}}%
\pgfpathlineto{\pgfqpoint{1.273962in}{0.558389in}}%
\pgfpathlineto{\pgfqpoint{1.273363in}{0.556424in}}%
\pgfpathlineto{\pgfqpoint{1.272767in}{0.554787in}}%
\pgfpathlineto{\pgfqpoint{1.293436in}{0.556563in}}%
\pgfpathlineto{\pgfqpoint{1.313976in}{0.558693in}}%
\pgfpathlineto{\pgfqpoint{1.334363in}{0.561174in}}%
\pgfpathlineto{\pgfqpoint{1.354575in}{0.564003in}}%
\pgfpathclose%
\pgfusepath{fill}%
\end{pgfscope}%
\begin{pgfscope}%
\pgfpathrectangle{\pgfqpoint{0.041670in}{0.041670in}}{\pgfqpoint{2.216660in}{2.216660in}}%
\pgfusepath{clip}%
\pgfsetbuttcap%
\pgfsetroundjoin%
\definecolor{currentfill}{rgb}{0.201239,0.383670,0.554294}%
\pgfsetfillcolor{currentfill}%
\pgfsetlinewidth{0.000000pt}%
\definecolor{currentstroke}{rgb}{0.000000,0.000000,0.000000}%
\pgfsetstrokecolor{currentstroke}%
\pgfsetdash{}{0pt}%
\pgfpathmoveto{\pgfqpoint{1.770850in}{0.845887in}}%
\pgfpathlineto{\pgfqpoint{1.774419in}{0.857891in}}%
\pgfpathlineto{\pgfqpoint{1.778008in}{0.870385in}}%
\pgfpathlineto{\pgfqpoint{1.781618in}{0.883378in}}%
\pgfpathlineto{\pgfqpoint{1.785248in}{0.896878in}}%
\pgfpathlineto{\pgfqpoint{1.770529in}{0.887044in}}%
\pgfpathlineto{\pgfqpoint{1.755159in}{0.877445in}}%
\pgfpathlineto{\pgfqpoint{1.739153in}{0.868092in}}%
\pgfpathlineto{\pgfqpoint{1.722527in}{0.858997in}}%
\pgfpathlineto{\pgfqpoint{1.719256in}{0.845655in}}%
\pgfpathlineto{\pgfqpoint{1.716004in}{0.832822in}}%
\pgfpathlineto{\pgfqpoint{1.712770in}{0.820490in}}%
\pgfpathlineto{\pgfqpoint{1.709555in}{0.808650in}}%
\pgfpathlineto{\pgfqpoint{1.725800in}{0.817590in}}%
\pgfpathlineto{\pgfqpoint{1.741441in}{0.826784in}}%
\pgfpathlineto{\pgfqpoint{1.756462in}{0.836220in}}%
\pgfpathlineto{\pgfqpoint{1.770850in}{0.845887in}}%
\pgfpathclose%
\pgfusepath{fill}%
\end{pgfscope}%
\begin{pgfscope}%
\pgfpathrectangle{\pgfqpoint{0.041670in}{0.041670in}}{\pgfqpoint{2.216660in}{2.216660in}}%
\pgfusepath{clip}%
\pgfsetbuttcap%
\pgfsetroundjoin%
\definecolor{currentfill}{rgb}{0.277941,0.056324,0.381191}%
\pgfsetfillcolor{currentfill}%
\pgfsetlinewidth{0.000000pt}%
\definecolor{currentstroke}{rgb}{0.000000,0.000000,0.000000}%
\pgfsetstrokecolor{currentstroke}%
\pgfsetdash{}{0pt}%
\pgfpathmoveto{\pgfqpoint{1.105605in}{0.553508in}}%
\pgfpathlineto{\pgfqpoint{1.105127in}{0.555137in}}%
\pgfpathlineto{\pgfqpoint{1.104648in}{0.557094in}}%
\pgfpathlineto{\pgfqpoint{1.104166in}{0.559385in}}%
\pgfpathlineto{\pgfqpoint{1.103683in}{0.562015in}}%
\pgfpathlineto{\pgfqpoint{1.082383in}{0.563510in}}%
\pgfpathlineto{\pgfqpoint{1.061194in}{0.565370in}}%
\pgfpathlineto{\pgfqpoint{1.040141in}{0.567592in}}%
\pgfpathlineto{\pgfqpoint{1.019248in}{0.570174in}}%
\pgfpathlineto{\pgfqpoint{1.020264in}{0.567495in}}%
\pgfpathlineto{\pgfqpoint{1.021277in}{0.565155in}}%
\pgfpathlineto{\pgfqpoint{1.022286in}{0.563149in}}%
\pgfpathlineto{\pgfqpoint{1.023292in}{0.561471in}}%
\pgfpathlineto{\pgfqpoint{1.043661in}{0.558951in}}%
\pgfpathlineto{\pgfqpoint{1.064185in}{0.556782in}}%
\pgfpathlineto{\pgfqpoint{1.084841in}{0.554967in}}%
\pgfpathlineto{\pgfqpoint{1.105605in}{0.553508in}}%
\pgfpathclose%
\pgfusepath{fill}%
\end{pgfscope}%
\begin{pgfscope}%
\pgfpathrectangle{\pgfqpoint{0.041670in}{0.041670in}}{\pgfqpoint{2.216660in}{2.216660in}}%
\pgfusepath{clip}%
\pgfsetbuttcap%
\pgfsetroundjoin%
\definecolor{currentfill}{rgb}{0.282884,0.135920,0.453427}%
\pgfsetfillcolor{currentfill}%
\pgfsetlinewidth{0.000000pt}%
\definecolor{currentstroke}{rgb}{0.000000,0.000000,0.000000}%
\pgfsetstrokecolor{currentstroke}%
\pgfsetdash{}{0pt}%
\pgfpathmoveto{\pgfqpoint{1.524241in}{0.623397in}}%
\pgfpathlineto{\pgfqpoint{1.526395in}{0.628088in}}%
\pgfpathlineto{\pgfqpoint{1.528558in}{0.633156in}}%
\pgfpathlineto{\pgfqpoint{1.530729in}{0.638606in}}%
\pgfpathlineto{\pgfqpoint{1.532911in}{0.644447in}}%
\pgfpathlineto{\pgfqpoint{1.513500in}{0.638606in}}%
\pgfpathlineto{\pgfqpoint{1.493712in}{0.633094in}}%
\pgfpathlineto{\pgfqpoint{1.473568in}{0.627918in}}%
\pgfpathlineto{\pgfqpoint{1.453090in}{0.623086in}}%
\pgfpathlineto{\pgfqpoint{1.451398in}{0.617361in}}%
\pgfpathlineto{\pgfqpoint{1.449712in}{0.612027in}}%
\pgfpathlineto{\pgfqpoint{1.448034in}{0.607077in}}%
\pgfpathlineto{\pgfqpoint{1.446362in}{0.602503in}}%
\pgfpathlineto{\pgfqpoint{1.466340in}{0.607230in}}%
\pgfpathlineto{\pgfqpoint{1.485993in}{0.612292in}}%
\pgfpathlineto{\pgfqpoint{1.505300in}{0.617683in}}%
\pgfpathlineto{\pgfqpoint{1.524241in}{0.623397in}}%
\pgfpathclose%
\pgfusepath{fill}%
\end{pgfscope}%
\begin{pgfscope}%
\pgfpathrectangle{\pgfqpoint{0.041670in}{0.041670in}}{\pgfqpoint{2.216660in}{2.216660in}}%
\pgfusepath{clip}%
\pgfsetbuttcap%
\pgfsetroundjoin%
\definecolor{currentfill}{rgb}{0.282327,0.094955,0.417331}%
\pgfsetfillcolor{currentfill}%
\pgfsetlinewidth{0.000000pt}%
\definecolor{currentstroke}{rgb}{0.000000,0.000000,0.000000}%
\pgfsetstrokecolor{currentstroke}%
\pgfsetdash{}{0pt}%
\pgfpathmoveto{\pgfqpoint{1.439742in}{0.587851in}}%
\pgfpathlineto{\pgfqpoint{1.441388in}{0.590980in}}%
\pgfpathlineto{\pgfqpoint{1.443039in}{0.594461in}}%
\pgfpathlineto{\pgfqpoint{1.444698in}{0.598300in}}%
\pgfpathlineto{\pgfqpoint{1.446362in}{0.602503in}}%
\pgfpathlineto{\pgfqpoint{1.426082in}{0.598118in}}%
\pgfpathlineto{\pgfqpoint{1.405522in}{0.594080in}}%
\pgfpathlineto{\pgfqpoint{1.384706in}{0.590395in}}%
\pgfpathlineto{\pgfqpoint{1.363655in}{0.587066in}}%
\pgfpathlineto{\pgfqpoint{1.362505in}{0.582951in}}%
\pgfpathlineto{\pgfqpoint{1.361360in}{0.579201in}}%
\pgfpathlineto{\pgfqpoint{1.360218in}{0.575809in}}%
\pgfpathlineto{\pgfqpoint{1.359081in}{0.572768in}}%
\pgfpathlineto{\pgfqpoint{1.379610in}{0.576020in}}%
\pgfpathlineto{\pgfqpoint{1.399911in}{0.579621in}}%
\pgfpathlineto{\pgfqpoint{1.419963in}{0.583566in}}%
\pgfpathlineto{\pgfqpoint{1.439742in}{0.587851in}}%
\pgfpathclose%
\pgfusepath{fill}%
\end{pgfscope}%
\begin{pgfscope}%
\pgfpathrectangle{\pgfqpoint{0.041670in}{0.041670in}}{\pgfqpoint{2.216660in}{2.216660in}}%
\pgfusepath{clip}%
\pgfsetbuttcap%
\pgfsetroundjoin%
\definecolor{currentfill}{rgb}{0.276194,0.190074,0.493001}%
\pgfsetfillcolor{currentfill}%
\pgfsetlinewidth{0.000000pt}%
\definecolor{currentstroke}{rgb}{0.000000,0.000000,0.000000}%
\pgfsetstrokecolor{currentstroke}%
\pgfsetdash{}{0pt}%
\pgfpathmoveto{\pgfqpoint{0.844234in}{0.639238in}}%
\pgfpathlineto{\pgfqpoint{0.842148in}{0.645447in}}%
\pgfpathlineto{\pgfqpoint{0.840054in}{0.652060in}}%
\pgfpathlineto{\pgfqpoint{0.837949in}{0.659083in}}%
\pgfpathlineto{\pgfqpoint{0.835835in}{0.666523in}}%
\pgfpathlineto{\pgfqpoint{0.815990in}{0.672526in}}%
\pgfpathlineto{\pgfqpoint{0.796558in}{0.678855in}}%
\pgfpathlineto{\pgfqpoint{0.777558in}{0.685505in}}%
\pgfpathlineto{\pgfqpoint{0.759011in}{0.692466in}}%
\pgfpathlineto{\pgfqpoint{0.761588in}{0.684892in}}%
\pgfpathlineto{\pgfqpoint{0.764153in}{0.677735in}}%
\pgfpathlineto{\pgfqpoint{0.766707in}{0.670988in}}%
\pgfpathlineto{\pgfqpoint{0.769250in}{0.664643in}}%
\pgfpathlineto{\pgfqpoint{0.787354in}{0.657826in}}%
\pgfpathlineto{\pgfqpoint{0.805900in}{0.651314in}}%
\pgfpathlineto{\pgfqpoint{0.824866in}{0.645116in}}%
\pgfpathlineto{\pgfqpoint{0.844234in}{0.639238in}}%
\pgfpathclose%
\pgfusepath{fill}%
\end{pgfscope}%
\begin{pgfscope}%
\pgfpathrectangle{\pgfqpoint{0.041670in}{0.041670in}}{\pgfqpoint{2.216660in}{2.216660in}}%
\pgfusepath{clip}%
\pgfsetbuttcap%
\pgfsetroundjoin%
\definecolor{currentfill}{rgb}{0.277941,0.056324,0.381191}%
\pgfsetfillcolor{currentfill}%
\pgfsetlinewidth{0.000000pt}%
\definecolor{currentstroke}{rgb}{0.000000,0.000000,0.000000}%
\pgfsetstrokecolor{currentstroke}%
\pgfsetdash{}{0pt}%
\pgfpathmoveto{\pgfqpoint{1.272767in}{0.554787in}}%
\pgfpathlineto{\pgfqpoint{1.273363in}{0.556424in}}%
\pgfpathlineto{\pgfqpoint{1.273962in}{0.558389in}}%
\pgfpathlineto{\pgfqpoint{1.274562in}{0.560688in}}%
\pgfpathlineto{\pgfqpoint{1.275165in}{0.563326in}}%
\pgfpathlineto{\pgfqpoint{1.253854in}{0.561872in}}%
\pgfpathlineto{\pgfqpoint{1.232458in}{0.560785in}}%
\pgfpathlineto{\pgfqpoint{1.211003in}{0.560066in}}%
\pgfpathlineto{\pgfqpoint{1.189511in}{0.559717in}}%
\pgfpathlineto{\pgfqpoint{1.189451in}{0.557100in}}%
\pgfpathlineto{\pgfqpoint{1.189391in}{0.554823in}}%
\pgfpathlineto{\pgfqpoint{1.189331in}{0.552880in}}%
\pgfpathlineto{\pgfqpoint{1.189271in}{0.551265in}}%
\pgfpathlineto{\pgfqpoint{1.210220in}{0.551606in}}%
\pgfpathlineto{\pgfqpoint{1.231135in}{0.552307in}}%
\pgfpathlineto{\pgfqpoint{1.251992in}{0.553368in}}%
\pgfpathlineto{\pgfqpoint{1.272767in}{0.554787in}}%
\pgfpathclose%
\pgfusepath{fill}%
\end{pgfscope}%
\begin{pgfscope}%
\pgfpathrectangle{\pgfqpoint{0.041670in}{0.041670in}}{\pgfqpoint{2.216660in}{2.216660in}}%
\pgfusepath{clip}%
\pgfsetbuttcap%
\pgfsetroundjoin%
\definecolor{currentfill}{rgb}{0.282327,0.094955,0.417331}%
\pgfsetfillcolor{currentfill}%
\pgfsetlinewidth{0.000000pt}%
\definecolor{currentstroke}{rgb}{0.000000,0.000000,0.000000}%
\pgfsetstrokecolor{currentstroke}%
\pgfsetdash{}{0pt}%
\pgfpathmoveto{\pgfqpoint{1.019248in}{0.570174in}}%
\pgfpathlineto{\pgfqpoint{1.018227in}{0.573199in}}%
\pgfpathlineto{\pgfqpoint{1.017203in}{0.576576in}}%
\pgfpathlineto{\pgfqpoint{1.016175in}{0.580311in}}%
\pgfpathlineto{\pgfqpoint{1.015142in}{0.584410in}}%
\pgfpathlineto{\pgfqpoint{0.993905in}{0.587418in}}%
\pgfpathlineto{\pgfqpoint{0.972879in}{0.590787in}}%
\pgfpathlineto{\pgfqpoint{0.952090in}{0.594512in}}%
\pgfpathlineto{\pgfqpoint{0.931560in}{0.598588in}}%
\pgfpathlineto{\pgfqpoint{0.933113in}{0.594408in}}%
\pgfpathlineto{\pgfqpoint{0.934660in}{0.590591in}}%
\pgfpathlineto{\pgfqpoint{0.936200in}{0.587133in}}%
\pgfpathlineto{\pgfqpoint{0.937735in}{0.584026in}}%
\pgfpathlineto{\pgfqpoint{0.957758in}{0.580043in}}%
\pgfpathlineto{\pgfqpoint{0.978033in}{0.576403in}}%
\pgfpathlineto{\pgfqpoint{0.998537in}{0.573112in}}%
\pgfpathlineto{\pgfqpoint{1.019248in}{0.570174in}}%
\pgfpathclose%
\pgfusepath{fill}%
\end{pgfscope}%
\begin{pgfscope}%
\pgfpathrectangle{\pgfqpoint{0.041670in}{0.041670in}}{\pgfqpoint{2.216660in}{2.216660in}}%
\pgfusepath{clip}%
\pgfsetbuttcap%
\pgfsetroundjoin%
\definecolor{currentfill}{rgb}{0.277941,0.056324,0.381191}%
\pgfsetfillcolor{currentfill}%
\pgfsetlinewidth{0.000000pt}%
\definecolor{currentstroke}{rgb}{0.000000,0.000000,0.000000}%
\pgfsetstrokecolor{currentstroke}%
\pgfsetdash{}{0pt}%
\pgfpathmoveto{\pgfqpoint{1.189271in}{0.551265in}}%
\pgfpathlineto{\pgfqpoint{1.189331in}{0.552880in}}%
\pgfpathlineto{\pgfqpoint{1.189391in}{0.554823in}}%
\pgfpathlineto{\pgfqpoint{1.189451in}{0.557100in}}%
\pgfpathlineto{\pgfqpoint{1.189511in}{0.559717in}}%
\pgfpathlineto{\pgfqpoint{1.168009in}{0.559737in}}%
\pgfpathlineto{\pgfqpoint{1.146521in}{0.560128in}}%
\pgfpathlineto{\pgfqpoint{1.125071in}{0.560887in}}%
\pgfpathlineto{\pgfqpoint{1.103683in}{0.562015in}}%
\pgfpathlineto{\pgfqpoint{1.104166in}{0.559385in}}%
\pgfpathlineto{\pgfqpoint{1.104648in}{0.557094in}}%
\pgfpathlineto{\pgfqpoint{1.105127in}{0.555137in}}%
\pgfpathlineto{\pgfqpoint{1.105605in}{0.553508in}}%
\pgfpathlineto{\pgfqpoint{1.126454in}{0.552408in}}%
\pgfpathlineto{\pgfqpoint{1.147363in}{0.551666in}}%
\pgfpathlineto{\pgfqpoint{1.168310in}{0.551285in}}%
\pgfpathlineto{\pgfqpoint{1.189271in}{0.551265in}}%
\pgfpathclose%
\pgfusepath{fill}%
\end{pgfscope}%
\begin{pgfscope}%
\pgfpathrectangle{\pgfqpoint{0.041670in}{0.041670in}}{\pgfqpoint{2.216660in}{2.216660in}}%
\pgfusepath{clip}%
\pgfsetbuttcap%
\pgfsetroundjoin%
\definecolor{currentfill}{rgb}{0.282884,0.135920,0.453427}%
\pgfsetfillcolor{currentfill}%
\pgfsetlinewidth{0.000000pt}%
\definecolor{currentstroke}{rgb}{0.000000,0.000000,0.000000}%
\pgfsetstrokecolor{currentstroke}%
\pgfsetdash{}{0pt}%
\pgfpathmoveto{\pgfqpoint{0.931560in}{0.598588in}}%
\pgfpathlineto{\pgfqpoint{0.930001in}{0.603139in}}%
\pgfpathlineto{\pgfqpoint{0.928435in}{0.608068in}}%
\pgfpathlineto{\pgfqpoint{0.926863in}{0.613380in}}%
\pgfpathlineto{\pgfqpoint{0.925283in}{0.619083in}}%
\pgfpathlineto{\pgfqpoint{0.904528in}{0.623605in}}%
\pgfpathlineto{\pgfqpoint{0.884087in}{0.628476in}}%
\pgfpathlineto{\pgfqpoint{0.863981in}{0.633690in}}%
\pgfpathlineto{\pgfqpoint{0.844234in}{0.639238in}}%
\pgfpathlineto{\pgfqpoint{0.846310in}{0.633426in}}%
\pgfpathlineto{\pgfqpoint{0.848377in}{0.628004in}}%
\pgfpathlineto{\pgfqpoint{0.850436in}{0.622965in}}%
\pgfpathlineto{\pgfqpoint{0.852486in}{0.618302in}}%
\pgfpathlineto{\pgfqpoint{0.871754in}{0.612875in}}%
\pgfpathlineto{\pgfqpoint{0.891370in}{0.607776in}}%
\pgfpathlineto{\pgfqpoint{0.911312in}{0.603012in}}%
\pgfpathlineto{\pgfqpoint{0.931560in}{0.598588in}}%
\pgfpathclose%
\pgfusepath{fill}%
\end{pgfscope}%
\begin{pgfscope}%
\pgfpathrectangle{\pgfqpoint{0.041670in}{0.041670in}}{\pgfqpoint{2.216660in}{2.216660in}}%
\pgfusepath{clip}%
\pgfsetbuttcap%
\pgfsetroundjoin%
\definecolor{currentfill}{rgb}{0.201239,0.383670,0.554294}%
\pgfsetfillcolor{currentfill}%
\pgfsetlinewidth{0.000000pt}%
\definecolor{currentstroke}{rgb}{0.000000,0.000000,0.000000}%
\pgfsetstrokecolor{currentstroke}%
\pgfsetdash{}{0pt}%
\pgfpathmoveto{\pgfqpoint{0.665290in}{0.800925in}}%
\pgfpathlineto{\pgfqpoint{0.662162in}{0.812731in}}%
\pgfpathlineto{\pgfqpoint{0.659016in}{0.825029in}}%
\pgfpathlineto{\pgfqpoint{0.655852in}{0.837829in}}%
\pgfpathlineto{\pgfqpoint{0.652670in}{0.851137in}}%
\pgfpathlineto{\pgfqpoint{0.635505in}{0.859995in}}%
\pgfpathlineto{\pgfqpoint{0.618947in}{0.869119in}}%
\pgfpathlineto{\pgfqpoint{0.603011in}{0.878500in}}%
\pgfpathlineto{\pgfqpoint{0.587713in}{0.888125in}}%
\pgfpathlineto{\pgfqpoint{0.591269in}{0.874662in}}%
\pgfpathlineto{\pgfqpoint{0.594805in}{0.861706in}}%
\pgfpathlineto{\pgfqpoint{0.598321in}{0.849249in}}%
\pgfpathlineto{\pgfqpoint{0.601818in}{0.837283in}}%
\pgfpathlineto{\pgfqpoint{0.616769in}{0.827820in}}%
\pgfpathlineto{\pgfqpoint{0.632342in}{0.818599in}}%
\pgfpathlineto{\pgfqpoint{0.648521in}{0.809631in}}%
\pgfpathlineto{\pgfqpoint{0.665290in}{0.800925in}}%
\pgfpathclose%
\pgfusepath{fill}%
\end{pgfscope}%
\begin{pgfscope}%
\pgfpathrectangle{\pgfqpoint{0.041670in}{0.041670in}}{\pgfqpoint{2.216660in}{2.216660in}}%
\pgfusepath{clip}%
\pgfsetbuttcap%
\pgfsetroundjoin%
\definecolor{currentfill}{rgb}{0.233603,0.313828,0.543914}%
\pgfsetfillcolor{currentfill}%
\pgfsetlinewidth{0.000000pt}%
\definecolor{currentstroke}{rgb}{0.000000,0.000000,0.000000}%
\pgfsetstrokecolor{currentstroke}%
\pgfsetdash{}{0pt}%
\pgfpathmoveto{\pgfqpoint{1.696862in}{0.766048in}}%
\pgfpathlineto{\pgfqpoint{1.700011in}{0.776001in}}%
\pgfpathlineto{\pgfqpoint{1.703175in}{0.786413in}}%
\pgfpathlineto{\pgfqpoint{1.706356in}{0.797294in}}%
\pgfpathlineto{\pgfqpoint{1.709555in}{0.808650in}}%
\pgfpathlineto{\pgfqpoint{1.692721in}{0.799975in}}%
\pgfpathlineto{\pgfqpoint{1.675315in}{0.791574in}}%
\pgfpathlineto{\pgfqpoint{1.657355in}{0.783460in}}%
\pgfpathlineto{\pgfqpoint{1.638858in}{0.775641in}}%
\pgfpathlineto{\pgfqpoint{1.636075in}{0.764433in}}%
\pgfpathlineto{\pgfqpoint{1.633306in}{0.753704in}}%
\pgfpathlineto{\pgfqpoint{1.630552in}{0.743443in}}%
\pgfpathlineto{\pgfqpoint{1.627812in}{0.733644in}}%
\pgfpathlineto{\pgfqpoint{1.645875in}{0.741319in}}%
\pgfpathlineto{\pgfqpoint{1.663416in}{0.749285in}}%
\pgfpathlineto{\pgfqpoint{1.680418in}{0.757531in}}%
\pgfpathlineto{\pgfqpoint{1.696862in}{0.766048in}}%
\pgfpathclose%
\pgfusepath{fill}%
\end{pgfscope}%
\begin{pgfscope}%
\pgfpathrectangle{\pgfqpoint{0.041670in}{0.041670in}}{\pgfqpoint{2.216660in}{2.216660in}}%
\pgfusepath{clip}%
\pgfsetbuttcap%
\pgfsetroundjoin%
\definecolor{currentfill}{rgb}{0.282327,0.094955,0.417331}%
\pgfsetfillcolor{currentfill}%
\pgfsetlinewidth{0.000000pt}%
\definecolor{currentstroke}{rgb}{0.000000,0.000000,0.000000}%
\pgfsetstrokecolor{currentstroke}%
\pgfsetdash{}{0pt}%
\pgfpathmoveto{\pgfqpoint{1.359081in}{0.572768in}}%
\pgfpathlineto{\pgfqpoint{1.360218in}{0.575809in}}%
\pgfpathlineto{\pgfqpoint{1.361360in}{0.579201in}}%
\pgfpathlineto{\pgfqpoint{1.362505in}{0.582951in}}%
\pgfpathlineto{\pgfqpoint{1.363655in}{0.587066in}}%
\pgfpathlineto{\pgfqpoint{1.342395in}{0.584099in}}%
\pgfpathlineto{\pgfqpoint{1.320950in}{0.581497in}}%
\pgfpathlineto{\pgfqpoint{1.299343in}{0.579263in}}%
\pgfpathlineto{\pgfqpoint{1.277599in}{0.577401in}}%
\pgfpathlineto{\pgfqpoint{1.276987in}{0.573341in}}%
\pgfpathlineto{\pgfqpoint{1.276377in}{0.569646in}}%
\pgfpathlineto{\pgfqpoint{1.275770in}{0.566310in}}%
\pgfpathlineto{\pgfqpoint{1.275165in}{0.563326in}}%
\pgfpathlineto{\pgfqpoint{1.296367in}{0.565145in}}%
\pgfpathlineto{\pgfqpoint{1.317437in}{0.567327in}}%
\pgfpathlineto{\pgfqpoint{1.338349in}{0.569869in}}%
\pgfpathlineto{\pgfqpoint{1.359081in}{0.572768in}}%
\pgfpathclose%
\pgfusepath{fill}%
\end{pgfscope}%
\begin{pgfscope}%
\pgfpathrectangle{\pgfqpoint{0.041670in}{0.041670in}}{\pgfqpoint{2.216660in}{2.216660in}}%
\pgfusepath{clip}%
\pgfsetbuttcap%
\pgfsetroundjoin%
\definecolor{currentfill}{rgb}{0.282327,0.094955,0.417331}%
\pgfsetfillcolor{currentfill}%
\pgfsetlinewidth{0.000000pt}%
\definecolor{currentstroke}{rgb}{0.000000,0.000000,0.000000}%
\pgfsetstrokecolor{currentstroke}%
\pgfsetdash{}{0pt}%
\pgfpathmoveto{\pgfqpoint{1.103683in}{0.562015in}}%
\pgfpathlineto{\pgfqpoint{1.103199in}{0.564992in}}%
\pgfpathlineto{\pgfqpoint{1.102712in}{0.568320in}}%
\pgfpathlineto{\pgfqpoint{1.102224in}{0.572007in}}%
\pgfpathlineto{\pgfqpoint{1.101733in}{0.576059in}}%
\pgfpathlineto{\pgfqpoint{1.079889in}{0.577589in}}%
\pgfpathlineto{\pgfqpoint{1.058159in}{0.579493in}}%
\pgfpathlineto{\pgfqpoint{1.036569in}{0.581768in}}%
\pgfpathlineto{\pgfqpoint{1.015142in}{0.584410in}}%
\pgfpathlineto{\pgfqpoint{1.016175in}{0.580311in}}%
\pgfpathlineto{\pgfqpoint{1.017203in}{0.576576in}}%
\pgfpathlineto{\pgfqpoint{1.018227in}{0.573199in}}%
\pgfpathlineto{\pgfqpoint{1.019248in}{0.570174in}}%
\pgfpathlineto{\pgfqpoint{1.040141in}{0.567592in}}%
\pgfpathlineto{\pgfqpoint{1.061194in}{0.565370in}}%
\pgfpathlineto{\pgfqpoint{1.082383in}{0.563510in}}%
\pgfpathlineto{\pgfqpoint{1.103683in}{0.562015in}}%
\pgfpathclose%
\pgfusepath{fill}%
\end{pgfscope}%
\begin{pgfscope}%
\pgfpathrectangle{\pgfqpoint{0.041670in}{0.041670in}}{\pgfqpoint{2.216660in}{2.216660in}}%
\pgfusepath{clip}%
\pgfsetbuttcap%
\pgfsetroundjoin%
\definecolor{currentfill}{rgb}{0.260571,0.246922,0.522828}%
\pgfsetfillcolor{currentfill}%
\pgfsetlinewidth{0.000000pt}%
\definecolor{currentstroke}{rgb}{0.000000,0.000000,0.000000}%
\pgfsetstrokecolor{currentstroke}%
\pgfsetdash{}{0pt}%
\pgfpathmoveto{\pgfqpoint{1.616989in}{0.698908in}}%
\pgfpathlineto{\pgfqpoint{1.619675in}{0.706938in}}%
\pgfpathlineto{\pgfqpoint{1.622374in}{0.715399in}}%
\pgfpathlineto{\pgfqpoint{1.625086in}{0.724299in}}%
\pgfpathlineto{\pgfqpoint{1.627812in}{0.733644in}}%
\pgfpathlineto{\pgfqpoint{1.609246in}{0.726269in}}%
\pgfpathlineto{\pgfqpoint{1.590196in}{0.719204in}}%
\pgfpathlineto{\pgfqpoint{1.570683in}{0.712457in}}%
\pgfpathlineto{\pgfqpoint{1.550727in}{0.706036in}}%
\pgfpathlineto{\pgfqpoint{1.548462in}{0.696825in}}%
\pgfpathlineto{\pgfqpoint{1.546209in}{0.688060in}}%
\pgfpathlineto{\pgfqpoint{1.543966in}{0.679735in}}%
\pgfpathlineto{\pgfqpoint{1.541734in}{0.671842in}}%
\pgfpathlineto{\pgfqpoint{1.561214in}{0.678136in}}%
\pgfpathlineto{\pgfqpoint{1.580263in}{0.684751in}}%
\pgfpathlineto{\pgfqpoint{1.598861in}{0.691677in}}%
\pgfpathlineto{\pgfqpoint{1.616989in}{0.698908in}}%
\pgfpathclose%
\pgfusepath{fill}%
\end{pgfscope}%
\begin{pgfscope}%
\pgfpathrectangle{\pgfqpoint{0.041670in}{0.041670in}}{\pgfqpoint{2.216660in}{2.216660in}}%
\pgfusepath{clip}%
\pgfsetbuttcap%
\pgfsetroundjoin%
\definecolor{currentfill}{rgb}{0.233603,0.313828,0.543914}%
\pgfsetfillcolor{currentfill}%
\pgfsetlinewidth{0.000000pt}%
\definecolor{currentstroke}{rgb}{0.000000,0.000000,0.000000}%
\pgfsetstrokecolor{currentstroke}%
\pgfsetdash{}{0pt}%
\pgfpathmoveto{\pgfqpoint{0.748577in}{0.727074in}}%
\pgfpathlineto{\pgfqpoint{0.745935in}{0.736841in}}%
\pgfpathlineto{\pgfqpoint{0.743280in}{0.747071in}}%
\pgfpathlineto{\pgfqpoint{0.740611in}{0.757770in}}%
\pgfpathlineto{\pgfqpoint{0.737927in}{0.768946in}}%
\pgfpathlineto{\pgfqpoint{0.718970in}{0.776494in}}%
\pgfpathlineto{\pgfqpoint{0.700533in}{0.784347in}}%
\pgfpathlineto{\pgfqpoint{0.682633in}{0.792494in}}%
\pgfpathlineto{\pgfqpoint{0.665290in}{0.800925in}}%
\pgfpathlineto{\pgfqpoint{0.668401in}{0.789604in}}%
\pgfpathlineto{\pgfqpoint{0.671496in}{0.778758in}}%
\pgfpathlineto{\pgfqpoint{0.674575in}{0.768381in}}%
\pgfpathlineto{\pgfqpoint{0.677637in}{0.758464in}}%
\pgfpathlineto{\pgfqpoint{0.694578in}{0.750187in}}%
\pgfpathlineto{\pgfqpoint{0.712060in}{0.742190in}}%
\pgfpathlineto{\pgfqpoint{0.730066in}{0.734482in}}%
\pgfpathlineto{\pgfqpoint{0.748577in}{0.727074in}}%
\pgfpathclose%
\pgfusepath{fill}%
\end{pgfscope}%
\begin{pgfscope}%
\pgfpathrectangle{\pgfqpoint{0.041670in}{0.041670in}}{\pgfqpoint{2.216660in}{2.216660in}}%
\pgfusepath{clip}%
\pgfsetbuttcap%
\pgfsetroundjoin%
\definecolor{currentfill}{rgb}{0.282884,0.135920,0.453427}%
\pgfsetfillcolor{currentfill}%
\pgfsetlinewidth{0.000000pt}%
\definecolor{currentstroke}{rgb}{0.000000,0.000000,0.000000}%
\pgfsetstrokecolor{currentstroke}%
\pgfsetdash{}{0pt}%
\pgfpathmoveto{\pgfqpoint{1.446362in}{0.602503in}}%
\pgfpathlineto{\pgfqpoint{1.448034in}{0.607077in}}%
\pgfpathlineto{\pgfqpoint{1.449712in}{0.612027in}}%
\pgfpathlineto{\pgfqpoint{1.451398in}{0.617361in}}%
\pgfpathlineto{\pgfqpoint{1.453090in}{0.623086in}}%
\pgfpathlineto{\pgfqpoint{1.432302in}{0.618602in}}%
\pgfpathlineto{\pgfqpoint{1.411226in}{0.614473in}}%
\pgfpathlineto{\pgfqpoint{1.389885in}{0.610705in}}%
\pgfpathlineto{\pgfqpoint{1.368305in}{0.607301in}}%
\pgfpathlineto{\pgfqpoint{1.367135in}{0.601663in}}%
\pgfpathlineto{\pgfqpoint{1.365970in}{0.596415in}}%
\pgfpathlineto{\pgfqpoint{1.364810in}{0.591552in}}%
\pgfpathlineto{\pgfqpoint{1.363655in}{0.587066in}}%
\pgfpathlineto{\pgfqpoint{1.384706in}{0.590395in}}%
\pgfpathlineto{\pgfqpoint{1.405522in}{0.594080in}}%
\pgfpathlineto{\pgfqpoint{1.426082in}{0.598118in}}%
\pgfpathlineto{\pgfqpoint{1.446362in}{0.602503in}}%
\pgfpathclose%
\pgfusepath{fill}%
\end{pgfscope}%
\begin{pgfscope}%
\pgfpathrectangle{\pgfqpoint{0.041670in}{0.041670in}}{\pgfqpoint{2.216660in}{2.216660in}}%
\pgfusepath{clip}%
\pgfsetbuttcap%
\pgfsetroundjoin%
\definecolor{currentfill}{rgb}{0.276194,0.190074,0.493001}%
\pgfsetfillcolor{currentfill}%
\pgfsetlinewidth{0.000000pt}%
\definecolor{currentstroke}{rgb}{0.000000,0.000000,0.000000}%
\pgfsetstrokecolor{currentstroke}%
\pgfsetdash{}{0pt}%
\pgfpathmoveto{\pgfqpoint{1.532911in}{0.644447in}}%
\pgfpathlineto{\pgfqpoint{1.535102in}{0.650684in}}%
\pgfpathlineto{\pgfqpoint{1.537302in}{0.657324in}}%
\pgfpathlineto{\pgfqpoint{1.539513in}{0.664374in}}%
\pgfpathlineto{\pgfqpoint{1.541734in}{0.671842in}}%
\pgfpathlineto{\pgfqpoint{1.521845in}{0.665877in}}%
\pgfpathlineto{\pgfqpoint{1.501568in}{0.660248in}}%
\pgfpathlineto{\pgfqpoint{1.480925in}{0.654962in}}%
\pgfpathlineto{\pgfqpoint{1.459939in}{0.650026in}}%
\pgfpathlineto{\pgfqpoint{1.458215in}{0.642670in}}%
\pgfpathlineto{\pgfqpoint{1.456499in}{0.635733in}}%
\pgfpathlineto{\pgfqpoint{1.454791in}{0.629207in}}%
\pgfpathlineto{\pgfqpoint{1.453090in}{0.623086in}}%
\pgfpathlineto{\pgfqpoint{1.473568in}{0.627918in}}%
\pgfpathlineto{\pgfqpoint{1.493712in}{0.633094in}}%
\pgfpathlineto{\pgfqpoint{1.513500in}{0.638606in}}%
\pgfpathlineto{\pgfqpoint{1.532911in}{0.644447in}}%
\pgfpathclose%
\pgfusepath{fill}%
\end{pgfscope}%
\begin{pgfscope}%
\pgfpathrectangle{\pgfqpoint{0.041670in}{0.041670in}}{\pgfqpoint{2.216660in}{2.216660in}}%
\pgfusepath{clip}%
\pgfsetbuttcap%
\pgfsetroundjoin%
\definecolor{currentfill}{rgb}{0.282327,0.094955,0.417331}%
\pgfsetfillcolor{currentfill}%
\pgfsetlinewidth{0.000000pt}%
\definecolor{currentstroke}{rgb}{0.000000,0.000000,0.000000}%
\pgfsetstrokecolor{currentstroke}%
\pgfsetdash{}{0pt}%
\pgfpathmoveto{\pgfqpoint{1.275165in}{0.563326in}}%
\pgfpathlineto{\pgfqpoint{1.275770in}{0.566310in}}%
\pgfpathlineto{\pgfqpoint{1.276377in}{0.569646in}}%
\pgfpathlineto{\pgfqpoint{1.276987in}{0.573341in}}%
\pgfpathlineto{\pgfqpoint{1.277599in}{0.577401in}}%
\pgfpathlineto{\pgfqpoint{1.255744in}{0.575912in}}%
\pgfpathlineto{\pgfqpoint{1.233801in}{0.574799in}}%
\pgfpathlineto{\pgfqpoint{1.211797in}{0.574064in}}%
\pgfpathlineto{\pgfqpoint{1.189756in}{0.573706in}}%
\pgfpathlineto{\pgfqpoint{1.189694in}{0.569668in}}%
\pgfpathlineto{\pgfqpoint{1.189633in}{0.565994in}}%
\pgfpathlineto{\pgfqpoint{1.189572in}{0.562679in}}%
\pgfpathlineto{\pgfqpoint{1.189511in}{0.559717in}}%
\pgfpathlineto{\pgfqpoint{1.211003in}{0.560066in}}%
\pgfpathlineto{\pgfqpoint{1.232458in}{0.560785in}}%
\pgfpathlineto{\pgfqpoint{1.253854in}{0.561872in}}%
\pgfpathlineto{\pgfqpoint{1.275165in}{0.563326in}}%
\pgfpathclose%
\pgfusepath{fill}%
\end{pgfscope}%
\begin{pgfscope}%
\pgfpathrectangle{\pgfqpoint{0.041670in}{0.041670in}}{\pgfqpoint{2.216660in}{2.216660in}}%
\pgfusepath{clip}%
\pgfsetbuttcap%
\pgfsetroundjoin%
\definecolor{currentfill}{rgb}{0.282327,0.094955,0.417331}%
\pgfsetfillcolor{currentfill}%
\pgfsetlinewidth{0.000000pt}%
\definecolor{currentstroke}{rgb}{0.000000,0.000000,0.000000}%
\pgfsetstrokecolor{currentstroke}%
\pgfsetdash{}{0pt}%
\pgfpathmoveto{\pgfqpoint{1.189511in}{0.559717in}}%
\pgfpathlineto{\pgfqpoint{1.189572in}{0.562679in}}%
\pgfpathlineto{\pgfqpoint{1.189633in}{0.565994in}}%
\pgfpathlineto{\pgfqpoint{1.189694in}{0.569668in}}%
\pgfpathlineto{\pgfqpoint{1.189756in}{0.573706in}}%
\pgfpathlineto{\pgfqpoint{1.167704in}{0.573727in}}%
\pgfpathlineto{\pgfqpoint{1.145666in}{0.574127in}}%
\pgfpathlineto{\pgfqpoint{1.123667in}{0.574904in}}%
\pgfpathlineto{\pgfqpoint{1.101733in}{0.576059in}}%
\pgfpathlineto{\pgfqpoint{1.102224in}{0.572007in}}%
\pgfpathlineto{\pgfqpoint{1.102712in}{0.568320in}}%
\pgfpathlineto{\pgfqpoint{1.103199in}{0.564992in}}%
\pgfpathlineto{\pgfqpoint{1.103683in}{0.562015in}}%
\pgfpathlineto{\pgfqpoint{1.125071in}{0.560887in}}%
\pgfpathlineto{\pgfqpoint{1.146521in}{0.560128in}}%
\pgfpathlineto{\pgfqpoint{1.168009in}{0.559737in}}%
\pgfpathlineto{\pgfqpoint{1.189511in}{0.559717in}}%
\pgfpathclose%
\pgfusepath{fill}%
\end{pgfscope}%
\begin{pgfscope}%
\pgfpathrectangle{\pgfqpoint{0.041670in}{0.041670in}}{\pgfqpoint{2.216660in}{2.216660in}}%
\pgfusepath{clip}%
\pgfsetbuttcap%
\pgfsetroundjoin%
\definecolor{currentfill}{rgb}{0.282884,0.135920,0.453427}%
\pgfsetfillcolor{currentfill}%
\pgfsetlinewidth{0.000000pt}%
\definecolor{currentstroke}{rgb}{0.000000,0.000000,0.000000}%
\pgfsetstrokecolor{currentstroke}%
\pgfsetdash{}{0pt}%
\pgfpathmoveto{\pgfqpoint{1.015142in}{0.584410in}}%
\pgfpathlineto{\pgfqpoint{1.014106in}{0.588881in}}%
\pgfpathlineto{\pgfqpoint{1.013065in}{0.593729in}}%
\pgfpathlineto{\pgfqpoint{1.012020in}{0.598962in}}%
\pgfpathlineto{\pgfqpoint{1.010970in}{0.604586in}}%
\pgfpathlineto{\pgfqpoint{0.989196in}{0.607661in}}%
\pgfpathlineto{\pgfqpoint{0.967641in}{0.611106in}}%
\pgfpathlineto{\pgfqpoint{0.946329in}{0.614914in}}%
\pgfpathlineto{\pgfqpoint{0.925283in}{0.619083in}}%
\pgfpathlineto{\pgfqpoint{0.926863in}{0.613380in}}%
\pgfpathlineto{\pgfqpoint{0.928435in}{0.608068in}}%
\pgfpathlineto{\pgfqpoint{0.930001in}{0.603139in}}%
\pgfpathlineto{\pgfqpoint{0.931560in}{0.598588in}}%
\pgfpathlineto{\pgfqpoint{0.952090in}{0.594512in}}%
\pgfpathlineto{\pgfqpoint{0.972879in}{0.590787in}}%
\pgfpathlineto{\pgfqpoint{0.993905in}{0.587418in}}%
\pgfpathlineto{\pgfqpoint{1.015142in}{0.584410in}}%
\pgfpathclose%
\pgfusepath{fill}%
\end{pgfscope}%
\begin{pgfscope}%
\pgfpathrectangle{\pgfqpoint{0.041670in}{0.041670in}}{\pgfqpoint{2.216660in}{2.216660in}}%
\pgfusepath{clip}%
\pgfsetbuttcap%
\pgfsetroundjoin%
\definecolor{currentfill}{rgb}{0.260571,0.246922,0.522828}%
\pgfsetfillcolor{currentfill}%
\pgfsetlinewidth{0.000000pt}%
\definecolor{currentstroke}{rgb}{0.000000,0.000000,0.000000}%
\pgfsetstrokecolor{currentstroke}%
\pgfsetdash{}{0pt}%
\pgfpathmoveto{\pgfqpoint{0.835835in}{0.666523in}}%
\pgfpathlineto{\pgfqpoint{0.833711in}{0.674389in}}%
\pgfpathlineto{\pgfqpoint{0.831576in}{0.682687in}}%
\pgfpathlineto{\pgfqpoint{0.829431in}{0.691425in}}%
\pgfpathlineto{\pgfqpoint{0.827275in}{0.700610in}}%
\pgfpathlineto{\pgfqpoint{0.806944in}{0.706733in}}%
\pgfpathlineto{\pgfqpoint{0.787036in}{0.713190in}}%
\pgfpathlineto{\pgfqpoint{0.767574in}{0.719973in}}%
\pgfpathlineto{\pgfqpoint{0.748577in}{0.727074in}}%
\pgfpathlineto{\pgfqpoint{0.751205in}{0.717760in}}%
\pgfpathlineto{\pgfqpoint{0.753820in}{0.708892in}}%
\pgfpathlineto{\pgfqpoint{0.756421in}{0.700464in}}%
\pgfpathlineto{\pgfqpoint{0.759011in}{0.692466in}}%
\pgfpathlineto{\pgfqpoint{0.777558in}{0.685505in}}%
\pgfpathlineto{\pgfqpoint{0.796558in}{0.678855in}}%
\pgfpathlineto{\pgfqpoint{0.815990in}{0.672526in}}%
\pgfpathlineto{\pgfqpoint{0.835835in}{0.666523in}}%
\pgfpathclose%
\pgfusepath{fill}%
\end{pgfscope}%
\begin{pgfscope}%
\pgfpathrectangle{\pgfqpoint{0.041670in}{0.041670in}}{\pgfqpoint{2.216660in}{2.216660in}}%
\pgfusepath{clip}%
\pgfsetbuttcap%
\pgfsetroundjoin%
\definecolor{currentfill}{rgb}{0.172719,0.448791,0.557885}%
\pgfsetfillcolor{currentfill}%
\pgfsetlinewidth{0.000000pt}%
\definecolor{currentstroke}{rgb}{0.000000,0.000000,0.000000}%
\pgfsetstrokecolor{currentstroke}%
\pgfsetdash{}{0pt}%
\pgfpathmoveto{\pgfqpoint{1.785248in}{0.896878in}}%
\pgfpathlineto{\pgfqpoint{1.788899in}{0.910893in}}%
\pgfpathlineto{\pgfqpoint{1.792572in}{0.925432in}}%
\pgfpathlineto{\pgfqpoint{1.796266in}{0.940505in}}%
\pgfpathlineto{\pgfqpoint{1.781295in}{0.930553in}}%
\pgfpathlineto{\pgfqpoint{1.765659in}{0.920838in}}%
\pgfpathlineto{\pgfqpoint{1.749375in}{0.911372in}}%
\pgfpathlineto{\pgfqpoint{1.732457in}{0.902165in}}%
\pgfpathlineto{\pgfqpoint{1.729127in}{0.887243in}}%
\pgfpathlineto{\pgfqpoint{1.725817in}{0.872857in}}%
\pgfpathlineto{\pgfqpoint{1.722527in}{0.858997in}}%
\pgfpathlineto{\pgfqpoint{1.739153in}{0.868092in}}%
\pgfpathlineto{\pgfqpoint{1.755159in}{0.877445in}}%
\pgfpathlineto{\pgfqpoint{1.770529in}{0.887044in}}%
\pgfpathlineto{\pgfqpoint{1.785248in}{0.896878in}}%
\pgfpathclose%
\pgfusepath{fill}%
\end{pgfscope}%
\begin{pgfscope}%
\pgfpathrectangle{\pgfqpoint{0.041670in}{0.041670in}}{\pgfqpoint{2.216660in}{2.216660in}}%
\pgfusepath{clip}%
\pgfsetbuttcap%
\pgfsetroundjoin%
\definecolor{currentfill}{rgb}{0.276194,0.190074,0.493001}%
\pgfsetfillcolor{currentfill}%
\pgfsetlinewidth{0.000000pt}%
\definecolor{currentstroke}{rgb}{0.000000,0.000000,0.000000}%
\pgfsetstrokecolor{currentstroke}%
\pgfsetdash{}{0pt}%
\pgfpathmoveto{\pgfqpoint{0.925283in}{0.619083in}}%
\pgfpathlineto{\pgfqpoint{0.923697in}{0.625183in}}%
\pgfpathlineto{\pgfqpoint{0.922104in}{0.631687in}}%
\pgfpathlineto{\pgfqpoint{0.920503in}{0.638603in}}%
\pgfpathlineto{\pgfqpoint{0.918895in}{0.645938in}}%
\pgfpathlineto{\pgfqpoint{0.897623in}{0.650557in}}%
\pgfpathlineto{\pgfqpoint{0.876674in}{0.655532in}}%
\pgfpathlineto{\pgfqpoint{0.856070in}{0.660856in}}%
\pgfpathlineto{\pgfqpoint{0.835835in}{0.666523in}}%
\pgfpathlineto{\pgfqpoint{0.837949in}{0.659083in}}%
\pgfpathlineto{\pgfqpoint{0.840054in}{0.652060in}}%
\pgfpathlineto{\pgfqpoint{0.842148in}{0.645447in}}%
\pgfpathlineto{\pgfqpoint{0.844234in}{0.639238in}}%
\pgfpathlineto{\pgfqpoint{0.863981in}{0.633690in}}%
\pgfpathlineto{\pgfqpoint{0.884087in}{0.628476in}}%
\pgfpathlineto{\pgfqpoint{0.904528in}{0.623605in}}%
\pgfpathlineto{\pgfqpoint{0.925283in}{0.619083in}}%
\pgfpathclose%
\pgfusepath{fill}%
\end{pgfscope}%
\begin{pgfscope}%
\pgfpathrectangle{\pgfqpoint{0.041670in}{0.041670in}}{\pgfqpoint{2.216660in}{2.216660in}}%
\pgfusepath{clip}%
\pgfsetbuttcap%
\pgfsetroundjoin%
\definecolor{currentfill}{rgb}{0.282884,0.135920,0.453427}%
\pgfsetfillcolor{currentfill}%
\pgfsetlinewidth{0.000000pt}%
\definecolor{currentstroke}{rgb}{0.000000,0.000000,0.000000}%
\pgfsetstrokecolor{currentstroke}%
\pgfsetdash{}{0pt}%
\pgfpathmoveto{\pgfqpoint{1.363655in}{0.587066in}}%
\pgfpathlineto{\pgfqpoint{1.364810in}{0.591552in}}%
\pgfpathlineto{\pgfqpoint{1.365970in}{0.596415in}}%
\pgfpathlineto{\pgfqpoint{1.367135in}{0.601663in}}%
\pgfpathlineto{\pgfqpoint{1.368305in}{0.607301in}}%
\pgfpathlineto{\pgfqpoint{1.346508in}{0.604267in}}%
\pgfpathlineto{\pgfqpoint{1.324521in}{0.601606in}}%
\pgfpathlineto{\pgfqpoint{1.302368in}{0.599322in}}%
\pgfpathlineto{\pgfqpoint{1.280074in}{0.597418in}}%
\pgfpathlineto{\pgfqpoint{1.279451in}{0.591833in}}%
\pgfpathlineto{\pgfqpoint{1.278831in}{0.586640in}}%
\pgfpathlineto{\pgfqpoint{1.278214in}{0.581831in}}%
\pgfpathlineto{\pgfqpoint{1.277599in}{0.577401in}}%
\pgfpathlineto{\pgfqpoint{1.299343in}{0.579263in}}%
\pgfpathlineto{\pgfqpoint{1.320950in}{0.581497in}}%
\pgfpathlineto{\pgfqpoint{1.342395in}{0.584099in}}%
\pgfpathlineto{\pgfqpoint{1.363655in}{0.587066in}}%
\pgfpathclose%
\pgfusepath{fill}%
\end{pgfscope}%
\begin{pgfscope}%
\pgfpathrectangle{\pgfqpoint{0.041670in}{0.041670in}}{\pgfqpoint{2.216660in}{2.216660in}}%
\pgfusepath{clip}%
\pgfsetbuttcap%
\pgfsetroundjoin%
\definecolor{currentfill}{rgb}{0.172719,0.448791,0.557885}%
\pgfsetfillcolor{currentfill}%
\pgfsetlinewidth{0.000000pt}%
\definecolor{currentstroke}{rgb}{0.000000,0.000000,0.000000}%
\pgfsetstrokecolor{currentstroke}%
\pgfsetdash{}{0pt}%
\pgfpathmoveto{\pgfqpoint{0.652670in}{0.851137in}}%
\pgfpathlineto{\pgfqpoint{0.649469in}{0.864964in}}%
\pgfpathlineto{\pgfqpoint{0.646249in}{0.879319in}}%
\pgfpathlineto{\pgfqpoint{0.643009in}{0.894209in}}%
\pgfpathlineto{\pgfqpoint{0.625542in}{0.903175in}}%
\pgfpathlineto{\pgfqpoint{0.608694in}{0.912411in}}%
\pgfpathlineto{\pgfqpoint{0.592481in}{0.921906in}}%
\pgfpathlineto{\pgfqpoint{0.576919in}{0.931648in}}%
\pgfpathlineto{\pgfqpoint{0.580538in}{0.916609in}}%
\pgfpathlineto{\pgfqpoint{0.584136in}{0.902105in}}%
\pgfpathlineto{\pgfqpoint{0.587713in}{0.888125in}}%
\pgfpathlineto{\pgfqpoint{0.603011in}{0.878500in}}%
\pgfpathlineto{\pgfqpoint{0.618947in}{0.869119in}}%
\pgfpathlineto{\pgfqpoint{0.635505in}{0.859995in}}%
\pgfpathlineto{\pgfqpoint{0.652670in}{0.851137in}}%
\pgfpathclose%
\pgfusepath{fill}%
\end{pgfscope}%
\begin{pgfscope}%
\pgfpathrectangle{\pgfqpoint{0.041670in}{0.041670in}}{\pgfqpoint{2.216660in}{2.216660in}}%
\pgfusepath{clip}%
\pgfsetbuttcap%
\pgfsetroundjoin%
\definecolor{currentfill}{rgb}{0.282884,0.135920,0.453427}%
\pgfsetfillcolor{currentfill}%
\pgfsetlinewidth{0.000000pt}%
\definecolor{currentstroke}{rgb}{0.000000,0.000000,0.000000}%
\pgfsetstrokecolor{currentstroke}%
\pgfsetdash{}{0pt}%
\pgfpathmoveto{\pgfqpoint{1.101733in}{0.576059in}}%
\pgfpathlineto{\pgfqpoint{1.101241in}{0.580482in}}%
\pgfpathlineto{\pgfqpoint{1.100746in}{0.585283in}}%
\pgfpathlineto{\pgfqpoint{1.100249in}{0.590469in}}%
\pgfpathlineto{\pgfqpoint{1.099750in}{0.596046in}}%
\pgfpathlineto{\pgfqpoint{1.077353in}{0.597610in}}%
\pgfpathlineto{\pgfqpoint{1.055073in}{0.599557in}}%
\pgfpathlineto{\pgfqpoint{1.032937in}{0.601883in}}%
\pgfpathlineto{\pgfqpoint{1.010970in}{0.604586in}}%
\pgfpathlineto{\pgfqpoint{1.012020in}{0.598962in}}%
\pgfpathlineto{\pgfqpoint{1.013065in}{0.593729in}}%
\pgfpathlineto{\pgfqpoint{1.014106in}{0.588881in}}%
\pgfpathlineto{\pgfqpoint{1.015142in}{0.584410in}}%
\pgfpathlineto{\pgfqpoint{1.036569in}{0.581768in}}%
\pgfpathlineto{\pgfqpoint{1.058159in}{0.579493in}}%
\pgfpathlineto{\pgfqpoint{1.079889in}{0.577589in}}%
\pgfpathlineto{\pgfqpoint{1.101733in}{0.576059in}}%
\pgfpathclose%
\pgfusepath{fill}%
\end{pgfscope}%
\begin{pgfscope}%
\pgfpathrectangle{\pgfqpoint{0.041670in}{0.041670in}}{\pgfqpoint{2.216660in}{2.216660in}}%
\pgfusepath{clip}%
\pgfsetbuttcap%
\pgfsetroundjoin%
\definecolor{currentfill}{rgb}{0.201239,0.383670,0.554294}%
\pgfsetfillcolor{currentfill}%
\pgfsetlinewidth{0.000000pt}%
\definecolor{currentstroke}{rgb}{0.000000,0.000000,0.000000}%
\pgfsetstrokecolor{currentstroke}%
\pgfsetdash{}{0pt}%
\pgfpathmoveto{\pgfqpoint{1.709555in}{0.808650in}}%
\pgfpathlineto{\pgfqpoint{1.712770in}{0.820490in}}%
\pgfpathlineto{\pgfqpoint{1.716004in}{0.832822in}}%
\pgfpathlineto{\pgfqpoint{1.719256in}{0.845655in}}%
\pgfpathlineto{\pgfqpoint{1.722527in}{0.858997in}}%
\pgfpathlineto{\pgfqpoint{1.705296in}{0.850170in}}%
\pgfpathlineto{\pgfqpoint{1.687478in}{0.841623in}}%
\pgfpathlineto{\pgfqpoint{1.669090in}{0.833366in}}%
\pgfpathlineto{\pgfqpoint{1.650151in}{0.825409in}}%
\pgfpathlineto{\pgfqpoint{1.647304in}{0.812209in}}%
\pgfpathlineto{\pgfqpoint{1.644473in}{0.799520in}}%
\pgfpathlineto{\pgfqpoint{1.641658in}{0.787333in}}%
\pgfpathlineto{\pgfqpoint{1.638858in}{0.775641in}}%
\pgfpathlineto{\pgfqpoint{1.657355in}{0.783460in}}%
\pgfpathlineto{\pgfqpoint{1.675315in}{0.791574in}}%
\pgfpathlineto{\pgfqpoint{1.692721in}{0.799975in}}%
\pgfpathlineto{\pgfqpoint{1.709555in}{0.808650in}}%
\pgfpathclose%
\pgfusepath{fill}%
\end{pgfscope}%
\begin{pgfscope}%
\pgfpathrectangle{\pgfqpoint{0.041670in}{0.041670in}}{\pgfqpoint{2.216660in}{2.216660in}}%
\pgfusepath{clip}%
\pgfsetbuttcap%
\pgfsetroundjoin%
\definecolor{currentfill}{rgb}{0.282884,0.135920,0.453427}%
\pgfsetfillcolor{currentfill}%
\pgfsetlinewidth{0.000000pt}%
\definecolor{currentstroke}{rgb}{0.000000,0.000000,0.000000}%
\pgfsetstrokecolor{currentstroke}%
\pgfsetdash{}{0pt}%
\pgfpathmoveto{\pgfqpoint{1.277599in}{0.577401in}}%
\pgfpathlineto{\pgfqpoint{1.278214in}{0.581831in}}%
\pgfpathlineto{\pgfqpoint{1.278831in}{0.586640in}}%
\pgfpathlineto{\pgfqpoint{1.279451in}{0.591833in}}%
\pgfpathlineto{\pgfqpoint{1.280074in}{0.597418in}}%
\pgfpathlineto{\pgfqpoint{1.257665in}{0.595895in}}%
\pgfpathlineto{\pgfqpoint{1.235166in}{0.594757in}}%
\pgfpathlineto{\pgfqpoint{1.212604in}{0.594005in}}%
\pgfpathlineto{\pgfqpoint{1.190004in}{0.593639in}}%
\pgfpathlineto{\pgfqpoint{1.189942in}{0.588076in}}%
\pgfpathlineto{\pgfqpoint{1.189880in}{0.582903in}}%
\pgfpathlineto{\pgfqpoint{1.189818in}{0.578116in}}%
\pgfpathlineto{\pgfqpoint{1.189756in}{0.573706in}}%
\pgfpathlineto{\pgfqpoint{1.211797in}{0.574064in}}%
\pgfpathlineto{\pgfqpoint{1.233801in}{0.574799in}}%
\pgfpathlineto{\pgfqpoint{1.255744in}{0.575912in}}%
\pgfpathlineto{\pgfqpoint{1.277599in}{0.577401in}}%
\pgfpathclose%
\pgfusepath{fill}%
\end{pgfscope}%
\begin{pgfscope}%
\pgfpathrectangle{\pgfqpoint{0.041670in}{0.041670in}}{\pgfqpoint{2.216660in}{2.216660in}}%
\pgfusepath{clip}%
\pgfsetbuttcap%
\pgfsetroundjoin%
\definecolor{currentfill}{rgb}{0.276194,0.190074,0.493001}%
\pgfsetfillcolor{currentfill}%
\pgfsetlinewidth{0.000000pt}%
\definecolor{currentstroke}{rgb}{0.000000,0.000000,0.000000}%
\pgfsetstrokecolor{currentstroke}%
\pgfsetdash{}{0pt}%
\pgfpathmoveto{\pgfqpoint{1.453090in}{0.623086in}}%
\pgfpathlineto{\pgfqpoint{1.454791in}{0.629207in}}%
\pgfpathlineto{\pgfqpoint{1.456499in}{0.635733in}}%
\pgfpathlineto{\pgfqpoint{1.458215in}{0.642670in}}%
\pgfpathlineto{\pgfqpoint{1.459939in}{0.650026in}}%
\pgfpathlineto{\pgfqpoint{1.438633in}{0.645447in}}%
\pgfpathlineto{\pgfqpoint{1.417031in}{0.641229in}}%
\pgfpathlineto{\pgfqpoint{1.395158in}{0.637380in}}%
\pgfpathlineto{\pgfqpoint{1.373037in}{0.633903in}}%
\pgfpathlineto{\pgfqpoint{1.371846in}{0.626631in}}%
\pgfpathlineto{\pgfqpoint{1.370660in}{0.619779in}}%
\pgfpathlineto{\pgfqpoint{1.369480in}{0.613337in}}%
\pgfpathlineto{\pgfqpoint{1.368305in}{0.607301in}}%
\pgfpathlineto{\pgfqpoint{1.389885in}{0.610705in}}%
\pgfpathlineto{\pgfqpoint{1.411226in}{0.614473in}}%
\pgfpathlineto{\pgfqpoint{1.432302in}{0.618602in}}%
\pgfpathlineto{\pgfqpoint{1.453090in}{0.623086in}}%
\pgfpathclose%
\pgfusepath{fill}%
\end{pgfscope}%
\begin{pgfscope}%
\pgfpathrectangle{\pgfqpoint{0.041670in}{0.041670in}}{\pgfqpoint{2.216660in}{2.216660in}}%
\pgfusepath{clip}%
\pgfsetbuttcap%
\pgfsetroundjoin%
\definecolor{currentfill}{rgb}{0.282884,0.135920,0.453427}%
\pgfsetfillcolor{currentfill}%
\pgfsetlinewidth{0.000000pt}%
\definecolor{currentstroke}{rgb}{0.000000,0.000000,0.000000}%
\pgfsetstrokecolor{currentstroke}%
\pgfsetdash{}{0pt}%
\pgfpathmoveto{\pgfqpoint{1.189756in}{0.573706in}}%
\pgfpathlineto{\pgfqpoint{1.189818in}{0.578116in}}%
\pgfpathlineto{\pgfqpoint{1.189880in}{0.582903in}}%
\pgfpathlineto{\pgfqpoint{1.189942in}{0.588076in}}%
\pgfpathlineto{\pgfqpoint{1.190004in}{0.593639in}}%
\pgfpathlineto{\pgfqpoint{1.167393in}{0.593661in}}%
\pgfpathlineto{\pgfqpoint{1.144796in}{0.594070in}}%
\pgfpathlineto{\pgfqpoint{1.122240in}{0.594865in}}%
\pgfpathlineto{\pgfqpoint{1.099750in}{0.596046in}}%
\pgfpathlineto{\pgfqpoint{1.100249in}{0.590469in}}%
\pgfpathlineto{\pgfqpoint{1.100746in}{0.585283in}}%
\pgfpathlineto{\pgfqpoint{1.101241in}{0.580482in}}%
\pgfpathlineto{\pgfqpoint{1.101733in}{0.576059in}}%
\pgfpathlineto{\pgfqpoint{1.123667in}{0.574904in}}%
\pgfpathlineto{\pgfqpoint{1.145666in}{0.574127in}}%
\pgfpathlineto{\pgfqpoint{1.167704in}{0.573727in}}%
\pgfpathlineto{\pgfqpoint{1.189756in}{0.573706in}}%
\pgfpathclose%
\pgfusepath{fill}%
\end{pgfscope}%
\begin{pgfscope}%
\pgfpathrectangle{\pgfqpoint{0.041670in}{0.041670in}}{\pgfqpoint{2.216660in}{2.216660in}}%
\pgfusepath{clip}%
\pgfsetbuttcap%
\pgfsetroundjoin%
\definecolor{currentfill}{rgb}{0.233603,0.313828,0.543914}%
\pgfsetfillcolor{currentfill}%
\pgfsetlinewidth{0.000000pt}%
\definecolor{currentstroke}{rgb}{0.000000,0.000000,0.000000}%
\pgfsetstrokecolor{currentstroke}%
\pgfsetdash{}{0pt}%
\pgfpathmoveto{\pgfqpoint{1.627812in}{0.733644in}}%
\pgfpathlineto{\pgfqpoint{1.630552in}{0.743443in}}%
\pgfpathlineto{\pgfqpoint{1.633306in}{0.753704in}}%
\pgfpathlineto{\pgfqpoint{1.636075in}{0.764433in}}%
\pgfpathlineto{\pgfqpoint{1.638858in}{0.775641in}}%
\pgfpathlineto{\pgfqpoint{1.619845in}{0.768127in}}%
\pgfpathlineto{\pgfqpoint{1.600335in}{0.760928in}}%
\pgfpathlineto{\pgfqpoint{1.580349in}{0.754053in}}%
\pgfpathlineto{\pgfqpoint{1.559908in}{0.747510in}}%
\pgfpathlineto{\pgfqpoint{1.557594in}{0.736432in}}%
\pgfpathlineto{\pgfqpoint{1.555293in}{0.725832in}}%
\pgfpathlineto{\pgfqpoint{1.553004in}{0.715703in}}%
\pgfpathlineto{\pgfqpoint{1.550727in}{0.706036in}}%
\pgfpathlineto{\pgfqpoint{1.570683in}{0.712457in}}%
\pgfpathlineto{\pgfqpoint{1.590196in}{0.719204in}}%
\pgfpathlineto{\pgfqpoint{1.609246in}{0.726269in}}%
\pgfpathlineto{\pgfqpoint{1.627812in}{0.733644in}}%
\pgfpathclose%
\pgfusepath{fill}%
\end{pgfscope}%
\begin{pgfscope}%
\pgfpathrectangle{\pgfqpoint{0.041670in}{0.041670in}}{\pgfqpoint{2.216660in}{2.216660in}}%
\pgfusepath{clip}%
\pgfsetbuttcap%
\pgfsetroundjoin%
\definecolor{currentfill}{rgb}{0.260571,0.246922,0.522828}%
\pgfsetfillcolor{currentfill}%
\pgfsetlinewidth{0.000000pt}%
\definecolor{currentstroke}{rgb}{0.000000,0.000000,0.000000}%
\pgfsetstrokecolor{currentstroke}%
\pgfsetdash{}{0pt}%
\pgfpathmoveto{\pgfqpoint{1.541734in}{0.671842in}}%
\pgfpathlineto{\pgfqpoint{1.543966in}{0.679735in}}%
\pgfpathlineto{\pgfqpoint{1.546209in}{0.688060in}}%
\pgfpathlineto{\pgfqpoint{1.548462in}{0.696825in}}%
\pgfpathlineto{\pgfqpoint{1.550727in}{0.706036in}}%
\pgfpathlineto{\pgfqpoint{1.530351in}{0.699951in}}%
\pgfpathlineto{\pgfqpoint{1.509575in}{0.694208in}}%
\pgfpathlineto{\pgfqpoint{1.488424in}{0.688815in}}%
\pgfpathlineto{\pgfqpoint{1.466919in}{0.683779in}}%
\pgfpathlineto{\pgfqpoint{1.465161in}{0.674676in}}%
\pgfpathlineto{\pgfqpoint{1.463412in}{0.666021in}}%
\pgfpathlineto{\pgfqpoint{1.461671in}{0.657807in}}%
\pgfpathlineto{\pgfqpoint{1.459939in}{0.650026in}}%
\pgfpathlineto{\pgfqpoint{1.480925in}{0.654962in}}%
\pgfpathlineto{\pgfqpoint{1.501568in}{0.660248in}}%
\pgfpathlineto{\pgfqpoint{1.521845in}{0.665877in}}%
\pgfpathlineto{\pgfqpoint{1.541734in}{0.671842in}}%
\pgfpathclose%
\pgfusepath{fill}%
\end{pgfscope}%
\begin{pgfscope}%
\pgfpathrectangle{\pgfqpoint{0.041670in}{0.041670in}}{\pgfqpoint{2.216660in}{2.216660in}}%
\pgfusepath{clip}%
\pgfsetbuttcap%
\pgfsetroundjoin%
\definecolor{currentfill}{rgb}{0.276194,0.190074,0.493001}%
\pgfsetfillcolor{currentfill}%
\pgfsetlinewidth{0.000000pt}%
\definecolor{currentstroke}{rgb}{0.000000,0.000000,0.000000}%
\pgfsetstrokecolor{currentstroke}%
\pgfsetdash{}{0pt}%
\pgfpathmoveto{\pgfqpoint{1.010970in}{0.604586in}}%
\pgfpathlineto{\pgfqpoint{1.009915in}{0.610607in}}%
\pgfpathlineto{\pgfqpoint{1.008856in}{0.617034in}}%
\pgfpathlineto{\pgfqpoint{1.007791in}{0.623872in}}%
\pgfpathlineto{\pgfqpoint{1.006722in}{0.631129in}}%
\pgfpathlineto{\pgfqpoint{0.984403in}{0.634271in}}%
\pgfpathlineto{\pgfqpoint{0.962309in}{0.637789in}}%
\pgfpathlineto{\pgfqpoint{0.940465in}{0.641680in}}%
\pgfpathlineto{\pgfqpoint{0.918895in}{0.645938in}}%
\pgfpathlineto{\pgfqpoint{0.920503in}{0.638603in}}%
\pgfpathlineto{\pgfqpoint{0.922104in}{0.631687in}}%
\pgfpathlineto{\pgfqpoint{0.923697in}{0.625183in}}%
\pgfpathlineto{\pgfqpoint{0.925283in}{0.619083in}}%
\pgfpathlineto{\pgfqpoint{0.946329in}{0.614914in}}%
\pgfpathlineto{\pgfqpoint{0.967641in}{0.611106in}}%
\pgfpathlineto{\pgfqpoint{0.989196in}{0.607661in}}%
\pgfpathlineto{\pgfqpoint{1.010970in}{0.604586in}}%
\pgfpathclose%
\pgfusepath{fill}%
\end{pgfscope}%
\begin{pgfscope}%
\pgfpathrectangle{\pgfqpoint{0.041670in}{0.041670in}}{\pgfqpoint{2.216660in}{2.216660in}}%
\pgfusepath{clip}%
\pgfsetbuttcap%
\pgfsetroundjoin%
\definecolor{currentfill}{rgb}{0.201239,0.383670,0.554294}%
\pgfsetfillcolor{currentfill}%
\pgfsetlinewidth{0.000000pt}%
\definecolor{currentstroke}{rgb}{0.000000,0.000000,0.000000}%
\pgfsetstrokecolor{currentstroke}%
\pgfsetdash{}{0pt}%
\pgfpathmoveto{\pgfqpoint{0.737927in}{0.768946in}}%
\pgfpathlineto{\pgfqpoint{0.735228in}{0.780609in}}%
\pgfpathlineto{\pgfqpoint{0.732514in}{0.792766in}}%
\pgfpathlineto{\pgfqpoint{0.729785in}{0.805425in}}%
\pgfpathlineto{\pgfqpoint{0.727039in}{0.818596in}}%
\pgfpathlineto{\pgfqpoint{0.707628in}{0.826278in}}%
\pgfpathlineto{\pgfqpoint{0.688749in}{0.834268in}}%
\pgfpathlineto{\pgfqpoint{0.670424in}{0.842558in}}%
\pgfpathlineto{\pgfqpoint{0.652670in}{0.851137in}}%
\pgfpathlineto{\pgfqpoint{0.655852in}{0.837829in}}%
\pgfpathlineto{\pgfqpoint{0.659016in}{0.825029in}}%
\pgfpathlineto{\pgfqpoint{0.662162in}{0.812731in}}%
\pgfpathlineto{\pgfqpoint{0.665290in}{0.800925in}}%
\pgfpathlineto{\pgfqpoint{0.682633in}{0.792494in}}%
\pgfpathlineto{\pgfqpoint{0.700533in}{0.784347in}}%
\pgfpathlineto{\pgfqpoint{0.718970in}{0.776494in}}%
\pgfpathlineto{\pgfqpoint{0.737927in}{0.768946in}}%
\pgfpathclose%
\pgfusepath{fill}%
\end{pgfscope}%
\begin{pgfscope}%
\pgfpathrectangle{\pgfqpoint{0.041670in}{0.041670in}}{\pgfqpoint{2.216660in}{2.216660in}}%
\pgfusepath{clip}%
\pgfsetbuttcap%
\pgfsetroundjoin%
\definecolor{currentfill}{rgb}{0.260571,0.246922,0.522828}%
\pgfsetfillcolor{currentfill}%
\pgfsetlinewidth{0.000000pt}%
\definecolor{currentstroke}{rgb}{0.000000,0.000000,0.000000}%
\pgfsetstrokecolor{currentstroke}%
\pgfsetdash{}{0pt}%
\pgfpathmoveto{\pgfqpoint{0.918895in}{0.645938in}}%
\pgfpathlineto{\pgfqpoint{0.917278in}{0.653698in}}%
\pgfpathlineto{\pgfqpoint{0.915655in}{0.661891in}}%
\pgfpathlineto{\pgfqpoint{0.914022in}{0.670525in}}%
\pgfpathlineto{\pgfqpoint{0.912382in}{0.679608in}}%
\pgfpathlineto{\pgfqpoint{0.890584in}{0.684321in}}%
\pgfpathlineto{\pgfqpoint{0.869118in}{0.689397in}}%
\pgfpathlineto{\pgfqpoint{0.848007in}{0.694829in}}%
\pgfpathlineto{\pgfqpoint{0.827275in}{0.700610in}}%
\pgfpathlineto{\pgfqpoint{0.829431in}{0.691425in}}%
\pgfpathlineto{\pgfqpoint{0.831576in}{0.682687in}}%
\pgfpathlineto{\pgfqpoint{0.833711in}{0.674389in}}%
\pgfpathlineto{\pgfqpoint{0.835835in}{0.666523in}}%
\pgfpathlineto{\pgfqpoint{0.856070in}{0.660856in}}%
\pgfpathlineto{\pgfqpoint{0.876674in}{0.655532in}}%
\pgfpathlineto{\pgfqpoint{0.897623in}{0.650557in}}%
\pgfpathlineto{\pgfqpoint{0.918895in}{0.645938in}}%
\pgfpathclose%
\pgfusepath{fill}%
\end{pgfscope}%
\begin{pgfscope}%
\pgfpathrectangle{\pgfqpoint{0.041670in}{0.041670in}}{\pgfqpoint{2.216660in}{2.216660in}}%
\pgfusepath{clip}%
\pgfsetbuttcap%
\pgfsetroundjoin%
\definecolor{currentfill}{rgb}{0.233603,0.313828,0.543914}%
\pgfsetfillcolor{currentfill}%
\pgfsetlinewidth{0.000000pt}%
\definecolor{currentstroke}{rgb}{0.000000,0.000000,0.000000}%
\pgfsetstrokecolor{currentstroke}%
\pgfsetdash{}{0pt}%
\pgfpathmoveto{\pgfqpoint{0.827275in}{0.700610in}}%
\pgfpathlineto{\pgfqpoint{0.825107in}{0.710250in}}%
\pgfpathlineto{\pgfqpoint{0.822929in}{0.720354in}}%
\pgfpathlineto{\pgfqpoint{0.820738in}{0.730928in}}%
\pgfpathlineto{\pgfqpoint{0.818536in}{0.741981in}}%
\pgfpathlineto{\pgfqpoint{0.797709in}{0.748221in}}%
\pgfpathlineto{\pgfqpoint{0.777317in}{0.754800in}}%
\pgfpathlineto{\pgfqpoint{0.757383in}{0.761712in}}%
\pgfpathlineto{\pgfqpoint{0.737927in}{0.768946in}}%
\pgfpathlineto{\pgfqpoint{0.740611in}{0.757770in}}%
\pgfpathlineto{\pgfqpoint{0.743280in}{0.747071in}}%
\pgfpathlineto{\pgfqpoint{0.745935in}{0.736841in}}%
\pgfpathlineto{\pgfqpoint{0.748577in}{0.727074in}}%
\pgfpathlineto{\pgfqpoint{0.767574in}{0.719973in}}%
\pgfpathlineto{\pgfqpoint{0.787036in}{0.713190in}}%
\pgfpathlineto{\pgfqpoint{0.806944in}{0.706733in}}%
\pgfpathlineto{\pgfqpoint{0.827275in}{0.700610in}}%
\pgfpathclose%
\pgfusepath{fill}%
\end{pgfscope}%
\begin{pgfscope}%
\pgfpathrectangle{\pgfqpoint{0.041670in}{0.041670in}}{\pgfqpoint{2.216660in}{2.216660in}}%
\pgfusepath{clip}%
\pgfsetbuttcap%
\pgfsetroundjoin%
\definecolor{currentfill}{rgb}{0.276194,0.190074,0.493001}%
\pgfsetfillcolor{currentfill}%
\pgfsetlinewidth{0.000000pt}%
\definecolor{currentstroke}{rgb}{0.000000,0.000000,0.000000}%
\pgfsetstrokecolor{currentstroke}%
\pgfsetdash{}{0pt}%
\pgfpathmoveto{\pgfqpoint{1.368305in}{0.607301in}}%
\pgfpathlineto{\pgfqpoint{1.369480in}{0.613337in}}%
\pgfpathlineto{\pgfqpoint{1.370660in}{0.619779in}}%
\pgfpathlineto{\pgfqpoint{1.371846in}{0.626631in}}%
\pgfpathlineto{\pgfqpoint{1.373037in}{0.633903in}}%
\pgfpathlineto{\pgfqpoint{1.350695in}{0.630804in}}%
\pgfpathlineto{\pgfqpoint{1.328156in}{0.628086in}}%
\pgfpathlineto{\pgfqpoint{1.305447in}{0.625752in}}%
\pgfpathlineto{\pgfqpoint{1.282593in}{0.623807in}}%
\pgfpathlineto{\pgfqpoint{1.281959in}{0.616588in}}%
\pgfpathlineto{\pgfqpoint{1.281327in}{0.609788in}}%
\pgfpathlineto{\pgfqpoint{1.280699in}{0.603400in}}%
\pgfpathlineto{\pgfqpoint{1.280074in}{0.597418in}}%
\pgfpathlineto{\pgfqpoint{1.302368in}{0.599322in}}%
\pgfpathlineto{\pgfqpoint{1.324521in}{0.601606in}}%
\pgfpathlineto{\pgfqpoint{1.346508in}{0.604267in}}%
\pgfpathlineto{\pgfqpoint{1.368305in}{0.607301in}}%
\pgfpathclose%
\pgfusepath{fill}%
\end{pgfscope}%
\begin{pgfscope}%
\pgfpathrectangle{\pgfqpoint{0.041670in}{0.041670in}}{\pgfqpoint{2.216660in}{2.216660in}}%
\pgfusepath{clip}%
\pgfsetbuttcap%
\pgfsetroundjoin%
\definecolor{currentfill}{rgb}{0.276194,0.190074,0.493001}%
\pgfsetfillcolor{currentfill}%
\pgfsetlinewidth{0.000000pt}%
\definecolor{currentstroke}{rgb}{0.000000,0.000000,0.000000}%
\pgfsetstrokecolor{currentstroke}%
\pgfsetdash{}{0pt}%
\pgfpathmoveto{\pgfqpoint{1.099750in}{0.596046in}}%
\pgfpathlineto{\pgfqpoint{1.099249in}{0.602021in}}%
\pgfpathlineto{\pgfqpoint{1.098746in}{0.608401in}}%
\pgfpathlineto{\pgfqpoint{1.098240in}{0.615193in}}%
\pgfpathlineto{\pgfqpoint{1.097732in}{0.622405in}}%
\pgfpathlineto{\pgfqpoint{1.074771in}{0.624004in}}%
\pgfpathlineto{\pgfqpoint{1.051932in}{0.625993in}}%
\pgfpathlineto{\pgfqpoint{1.029240in}{0.628369in}}%
\pgfpathlineto{\pgfqpoint{1.006722in}{0.631129in}}%
\pgfpathlineto{\pgfqpoint{1.007791in}{0.623872in}}%
\pgfpathlineto{\pgfqpoint{1.008856in}{0.617034in}}%
\pgfpathlineto{\pgfqpoint{1.009915in}{0.610607in}}%
\pgfpathlineto{\pgfqpoint{1.010970in}{0.604586in}}%
\pgfpathlineto{\pgfqpoint{1.032937in}{0.601883in}}%
\pgfpathlineto{\pgfqpoint{1.055073in}{0.599557in}}%
\pgfpathlineto{\pgfqpoint{1.077353in}{0.597610in}}%
\pgfpathlineto{\pgfqpoint{1.099750in}{0.596046in}}%
\pgfpathclose%
\pgfusepath{fill}%
\end{pgfscope}%
\begin{pgfscope}%
\pgfpathrectangle{\pgfqpoint{0.041670in}{0.041670in}}{\pgfqpoint{2.216660in}{2.216660in}}%
\pgfusepath{clip}%
\pgfsetbuttcap%
\pgfsetroundjoin%
\definecolor{currentfill}{rgb}{0.276194,0.190074,0.493001}%
\pgfsetfillcolor{currentfill}%
\pgfsetlinewidth{0.000000pt}%
\definecolor{currentstroke}{rgb}{0.000000,0.000000,0.000000}%
\pgfsetstrokecolor{currentstroke}%
\pgfsetdash{}{0pt}%
\pgfpathmoveto{\pgfqpoint{1.280074in}{0.597418in}}%
\pgfpathlineto{\pgfqpoint{1.280699in}{0.603400in}}%
\pgfpathlineto{\pgfqpoint{1.281327in}{0.609788in}}%
\pgfpathlineto{\pgfqpoint{1.281959in}{0.616588in}}%
\pgfpathlineto{\pgfqpoint{1.282593in}{0.623807in}}%
\pgfpathlineto{\pgfqpoint{1.259620in}{0.622252in}}%
\pgfpathlineto{\pgfqpoint{1.236556in}{0.621089in}}%
\pgfpathlineto{\pgfqpoint{1.213426in}{0.620321in}}%
\pgfpathlineto{\pgfqpoint{1.190257in}{0.619947in}}%
\pgfpathlineto{\pgfqpoint{1.190194in}{0.612748in}}%
\pgfpathlineto{\pgfqpoint{1.190130in}{0.605969in}}%
\pgfpathlineto{\pgfqpoint{1.190067in}{0.599601in}}%
\pgfpathlineto{\pgfqpoint{1.190004in}{0.593639in}}%
\pgfpathlineto{\pgfqpoint{1.212604in}{0.594005in}}%
\pgfpathlineto{\pgfqpoint{1.235166in}{0.594757in}}%
\pgfpathlineto{\pgfqpoint{1.257665in}{0.595895in}}%
\pgfpathlineto{\pgfqpoint{1.280074in}{0.597418in}}%
\pgfpathclose%
\pgfusepath{fill}%
\end{pgfscope}%
\begin{pgfscope}%
\pgfpathrectangle{\pgfqpoint{0.041670in}{0.041670in}}{\pgfqpoint{2.216660in}{2.216660in}}%
\pgfusepath{clip}%
\pgfsetbuttcap%
\pgfsetroundjoin%
\definecolor{currentfill}{rgb}{0.276194,0.190074,0.493001}%
\pgfsetfillcolor{currentfill}%
\pgfsetlinewidth{0.000000pt}%
\definecolor{currentstroke}{rgb}{0.000000,0.000000,0.000000}%
\pgfsetstrokecolor{currentstroke}%
\pgfsetdash{}{0pt}%
\pgfpathmoveto{\pgfqpoint{1.190004in}{0.593639in}}%
\pgfpathlineto{\pgfqpoint{1.190067in}{0.599601in}}%
\pgfpathlineto{\pgfqpoint{1.190130in}{0.605969in}}%
\pgfpathlineto{\pgfqpoint{1.190194in}{0.612748in}}%
\pgfpathlineto{\pgfqpoint{1.190257in}{0.619947in}}%
\pgfpathlineto{\pgfqpoint{1.167077in}{0.619969in}}%
\pgfpathlineto{\pgfqpoint{1.143911in}{0.620387in}}%
\pgfpathlineto{\pgfqpoint{1.120788in}{0.621199in}}%
\pgfpathlineto{\pgfqpoint{1.097732in}{0.622405in}}%
\pgfpathlineto{\pgfqpoint{1.098240in}{0.615193in}}%
\pgfpathlineto{\pgfqpoint{1.098746in}{0.608401in}}%
\pgfpathlineto{\pgfqpoint{1.099249in}{0.602021in}}%
\pgfpathlineto{\pgfqpoint{1.099750in}{0.596046in}}%
\pgfpathlineto{\pgfqpoint{1.122240in}{0.594865in}}%
\pgfpathlineto{\pgfqpoint{1.144796in}{0.594070in}}%
\pgfpathlineto{\pgfqpoint{1.167393in}{0.593661in}}%
\pgfpathlineto{\pgfqpoint{1.190004in}{0.593639in}}%
\pgfpathclose%
\pgfusepath{fill}%
\end{pgfscope}%
\begin{pgfscope}%
\pgfpathrectangle{\pgfqpoint{0.041670in}{0.041670in}}{\pgfqpoint{2.216660in}{2.216660in}}%
\pgfusepath{clip}%
\pgfsetbuttcap%
\pgfsetroundjoin%
\definecolor{currentfill}{rgb}{0.172719,0.448791,0.557885}%
\pgfsetfillcolor{currentfill}%
\pgfsetlinewidth{0.000000pt}%
\definecolor{currentstroke}{rgb}{0.000000,0.000000,0.000000}%
\pgfsetstrokecolor{currentstroke}%
\pgfsetdash{}{0pt}%
\pgfpathmoveto{\pgfqpoint{1.722527in}{0.858997in}}%
\pgfpathlineto{\pgfqpoint{1.725817in}{0.872857in}}%
\pgfpathlineto{\pgfqpoint{1.729127in}{0.887243in}}%
\pgfpathlineto{\pgfqpoint{1.732457in}{0.902165in}}%
\pgfpathlineto{\pgfqpoint{1.714923in}{0.893230in}}%
\pgfpathlineto{\pgfqpoint{1.696789in}{0.884578in}}%
\pgfpathlineto{\pgfqpoint{1.678074in}{0.876219in}}%
\pgfpathlineto{\pgfqpoint{1.658797in}{0.868163in}}%
\pgfpathlineto{\pgfqpoint{1.655898in}{0.853377in}}%
\pgfpathlineto{\pgfqpoint{1.653016in}{0.839128in}}%
\pgfpathlineto{\pgfqpoint{1.650151in}{0.825409in}}%
\pgfpathlineto{\pgfqpoint{1.669090in}{0.833366in}}%
\pgfpathlineto{\pgfqpoint{1.687478in}{0.841623in}}%
\pgfpathlineto{\pgfqpoint{1.705296in}{0.850170in}}%
\pgfpathlineto{\pgfqpoint{1.722527in}{0.858997in}}%
\pgfpathclose%
\pgfusepath{fill}%
\end{pgfscope}%
\begin{pgfscope}%
\pgfpathrectangle{\pgfqpoint{0.041670in}{0.041670in}}{\pgfqpoint{2.216660in}{2.216660in}}%
\pgfusepath{clip}%
\pgfsetbuttcap%
\pgfsetroundjoin%
\definecolor{currentfill}{rgb}{0.260571,0.246922,0.522828}%
\pgfsetfillcolor{currentfill}%
\pgfsetlinewidth{0.000000pt}%
\definecolor{currentstroke}{rgb}{0.000000,0.000000,0.000000}%
\pgfsetstrokecolor{currentstroke}%
\pgfsetdash{}{0pt}%
\pgfpathmoveto{\pgfqpoint{1.459939in}{0.650026in}}%
\pgfpathlineto{\pgfqpoint{1.461671in}{0.657807in}}%
\pgfpathlineto{\pgfqpoint{1.463412in}{0.666021in}}%
\pgfpathlineto{\pgfqpoint{1.465161in}{0.674676in}}%
\pgfpathlineto{\pgfqpoint{1.466919in}{0.683779in}}%
\pgfpathlineto{\pgfqpoint{1.445086in}{0.679107in}}%
\pgfpathlineto{\pgfqpoint{1.422949in}{0.674804in}}%
\pgfpathlineto{\pgfqpoint{1.400533in}{0.670876in}}%
\pgfpathlineto{\pgfqpoint{1.377862in}{0.667328in}}%
\pgfpathlineto{\pgfqpoint{1.376647in}{0.658306in}}%
\pgfpathlineto{\pgfqpoint{1.375438in}{0.649733in}}%
\pgfpathlineto{\pgfqpoint{1.374234in}{0.641601in}}%
\pgfpathlineto{\pgfqpoint{1.373037in}{0.633903in}}%
\pgfpathlineto{\pgfqpoint{1.395158in}{0.637380in}}%
\pgfpathlineto{\pgfqpoint{1.417031in}{0.641229in}}%
\pgfpathlineto{\pgfqpoint{1.438633in}{0.645447in}}%
\pgfpathlineto{\pgfqpoint{1.459939in}{0.650026in}}%
\pgfpathclose%
\pgfusepath{fill}%
\end{pgfscope}%
\begin{pgfscope}%
\pgfpathrectangle{\pgfqpoint{0.041670in}{0.041670in}}{\pgfqpoint{2.216660in}{2.216660in}}%
\pgfusepath{clip}%
\pgfsetbuttcap%
\pgfsetroundjoin%
\definecolor{currentfill}{rgb}{0.260571,0.246922,0.522828}%
\pgfsetfillcolor{currentfill}%
\pgfsetlinewidth{0.000000pt}%
\definecolor{currentstroke}{rgb}{0.000000,0.000000,0.000000}%
\pgfsetstrokecolor{currentstroke}%
\pgfsetdash{}{0pt}%
\pgfpathmoveto{\pgfqpoint{1.006722in}{0.631129in}}%
\pgfpathlineto{\pgfqpoint{1.005648in}{0.638813in}}%
\pgfpathlineto{\pgfqpoint{1.004568in}{0.646931in}}%
\pgfpathlineto{\pgfqpoint{1.003483in}{0.655490in}}%
\pgfpathlineto{\pgfqpoint{1.002392in}{0.664498in}}%
\pgfpathlineto{\pgfqpoint{0.979517in}{0.667703in}}%
\pgfpathlineto{\pgfqpoint{0.956873in}{0.671294in}}%
\pgfpathlineto{\pgfqpoint{0.934487in}{0.675263in}}%
\pgfpathlineto{\pgfqpoint{0.912382in}{0.679608in}}%
\pgfpathlineto{\pgfqpoint{0.914022in}{0.670525in}}%
\pgfpathlineto{\pgfqpoint{0.915655in}{0.661891in}}%
\pgfpathlineto{\pgfqpoint{0.917278in}{0.653698in}}%
\pgfpathlineto{\pgfqpoint{0.918895in}{0.645938in}}%
\pgfpathlineto{\pgfqpoint{0.940465in}{0.641680in}}%
\pgfpathlineto{\pgfqpoint{0.962309in}{0.637789in}}%
\pgfpathlineto{\pgfqpoint{0.984403in}{0.634271in}}%
\pgfpathlineto{\pgfqpoint{1.006722in}{0.631129in}}%
\pgfpathclose%
\pgfusepath{fill}%
\end{pgfscope}%
\begin{pgfscope}%
\pgfpathrectangle{\pgfqpoint{0.041670in}{0.041670in}}{\pgfqpoint{2.216660in}{2.216660in}}%
\pgfusepath{clip}%
\pgfsetbuttcap%
\pgfsetroundjoin%
\definecolor{currentfill}{rgb}{0.233603,0.313828,0.543914}%
\pgfsetfillcolor{currentfill}%
\pgfsetlinewidth{0.000000pt}%
\definecolor{currentstroke}{rgb}{0.000000,0.000000,0.000000}%
\pgfsetstrokecolor{currentstroke}%
\pgfsetdash{}{0pt}%
\pgfpathmoveto{\pgfqpoint{1.550727in}{0.706036in}}%
\pgfpathlineto{\pgfqpoint{1.553004in}{0.715703in}}%
\pgfpathlineto{\pgfqpoint{1.555293in}{0.725832in}}%
\pgfpathlineto{\pgfqpoint{1.557594in}{0.736432in}}%
\pgfpathlineto{\pgfqpoint{1.559908in}{0.747510in}}%
\pgfpathlineto{\pgfqpoint{1.539034in}{0.741309in}}%
\pgfpathlineto{\pgfqpoint{1.517750in}{0.735457in}}%
\pgfpathlineto{\pgfqpoint{1.496080in}{0.729961in}}%
\pgfpathlineto{\pgfqpoint{1.474047in}{0.724828in}}%
\pgfpathlineto{\pgfqpoint{1.472250in}{0.713854in}}%
\pgfpathlineto{\pgfqpoint{1.470464in}{0.703360in}}%
\pgfpathlineto{\pgfqpoint{1.468687in}{0.693338in}}%
\pgfpathlineto{\pgfqpoint{1.466919in}{0.683779in}}%
\pgfpathlineto{\pgfqpoint{1.488424in}{0.688815in}}%
\pgfpathlineto{\pgfqpoint{1.509575in}{0.694208in}}%
\pgfpathlineto{\pgfqpoint{1.530351in}{0.699951in}}%
\pgfpathlineto{\pgfqpoint{1.550727in}{0.706036in}}%
\pgfpathclose%
\pgfusepath{fill}%
\end{pgfscope}%
\begin{pgfscope}%
\pgfpathrectangle{\pgfqpoint{0.041670in}{0.041670in}}{\pgfqpoint{2.216660in}{2.216660in}}%
\pgfusepath{clip}%
\pgfsetbuttcap%
\pgfsetroundjoin%
\definecolor{currentfill}{rgb}{0.201239,0.383670,0.554294}%
\pgfsetfillcolor{currentfill}%
\pgfsetlinewidth{0.000000pt}%
\definecolor{currentstroke}{rgb}{0.000000,0.000000,0.000000}%
\pgfsetstrokecolor{currentstroke}%
\pgfsetdash{}{0pt}%
\pgfpathmoveto{\pgfqpoint{1.638858in}{0.775641in}}%
\pgfpathlineto{\pgfqpoint{1.641658in}{0.787333in}}%
\pgfpathlineto{\pgfqpoint{1.644473in}{0.799520in}}%
\pgfpathlineto{\pgfqpoint{1.647304in}{0.812209in}}%
\pgfpathlineto{\pgfqpoint{1.650151in}{0.825409in}}%
\pgfpathlineto{\pgfqpoint{1.630682in}{0.817762in}}%
\pgfpathlineto{\pgfqpoint{1.610701in}{0.810436in}}%
\pgfpathlineto{\pgfqpoint{1.590232in}{0.803438in}}%
\pgfpathlineto{\pgfqpoint{1.569295in}{0.796779in}}%
\pgfpathlineto{\pgfqpoint{1.566928in}{0.783702in}}%
\pgfpathlineto{\pgfqpoint{1.564574in}{0.771137in}}%
\pgfpathlineto{\pgfqpoint{1.562234in}{0.759076in}}%
\pgfpathlineto{\pgfqpoint{1.559908in}{0.747510in}}%
\pgfpathlineto{\pgfqpoint{1.580349in}{0.754053in}}%
\pgfpathlineto{\pgfqpoint{1.600335in}{0.760928in}}%
\pgfpathlineto{\pgfqpoint{1.619845in}{0.768127in}}%
\pgfpathlineto{\pgfqpoint{1.638858in}{0.775641in}}%
\pgfpathclose%
\pgfusepath{fill}%
\end{pgfscope}%
\begin{pgfscope}%
\pgfpathrectangle{\pgfqpoint{0.041670in}{0.041670in}}{\pgfqpoint{2.216660in}{2.216660in}}%
\pgfusepath{clip}%
\pgfsetbuttcap%
\pgfsetroundjoin%
\definecolor{currentfill}{rgb}{0.172719,0.448791,0.557885}%
\pgfsetfillcolor{currentfill}%
\pgfsetlinewidth{0.000000pt}%
\definecolor{currentstroke}{rgb}{0.000000,0.000000,0.000000}%
\pgfsetstrokecolor{currentstroke}%
\pgfsetdash{}{0pt}%
\pgfpathmoveto{\pgfqpoint{0.727039in}{0.818596in}}%
\pgfpathlineto{\pgfqpoint{0.724277in}{0.832287in}}%
\pgfpathlineto{\pgfqpoint{0.721499in}{0.846507in}}%
\pgfpathlineto{\pgfqpoint{0.718703in}{0.861266in}}%
\pgfpathlineto{\pgfqpoint{0.698943in}{0.869043in}}%
\pgfpathlineto{\pgfqpoint{0.679728in}{0.877133in}}%
\pgfpathlineto{\pgfqpoint{0.661077in}{0.885525in}}%
\pgfpathlineto{\pgfqpoint{0.643009in}{0.894209in}}%
\pgfpathlineto{\pgfqpoint{0.646249in}{0.879319in}}%
\pgfpathlineto{\pgfqpoint{0.649469in}{0.864964in}}%
\pgfpathlineto{\pgfqpoint{0.652670in}{0.851137in}}%
\pgfpathlineto{\pgfqpoint{0.670424in}{0.842558in}}%
\pgfpathlineto{\pgfqpoint{0.688749in}{0.834268in}}%
\pgfpathlineto{\pgfqpoint{0.707628in}{0.826278in}}%
\pgfpathlineto{\pgfqpoint{0.727039in}{0.818596in}}%
\pgfpathclose%
\pgfusepath{fill}%
\end{pgfscope}%
\begin{pgfscope}%
\pgfpathrectangle{\pgfqpoint{0.041670in}{0.041670in}}{\pgfqpoint{2.216660in}{2.216660in}}%
\pgfusepath{clip}%
\pgfsetbuttcap%
\pgfsetroundjoin%
\definecolor{currentfill}{rgb}{0.233603,0.313828,0.543914}%
\pgfsetfillcolor{currentfill}%
\pgfsetlinewidth{0.000000pt}%
\definecolor{currentstroke}{rgb}{0.000000,0.000000,0.000000}%
\pgfsetstrokecolor{currentstroke}%
\pgfsetdash{}{0pt}%
\pgfpathmoveto{\pgfqpoint{0.912382in}{0.679608in}}%
\pgfpathlineto{\pgfqpoint{0.910733in}{0.689146in}}%
\pgfpathlineto{\pgfqpoint{0.909075in}{0.699148in}}%
\pgfpathlineto{\pgfqpoint{0.907408in}{0.709623in}}%
\pgfpathlineto{\pgfqpoint{0.905732in}{0.720577in}}%
\pgfpathlineto{\pgfqpoint{0.883398in}{0.725380in}}%
\pgfpathlineto{\pgfqpoint{0.861404in}{0.730553in}}%
\pgfpathlineto{\pgfqpoint{0.839775in}{0.736089in}}%
\pgfpathlineto{\pgfqpoint{0.818536in}{0.741981in}}%
\pgfpathlineto{\pgfqpoint{0.820738in}{0.730928in}}%
\pgfpathlineto{\pgfqpoint{0.822929in}{0.720354in}}%
\pgfpathlineto{\pgfqpoint{0.825107in}{0.710250in}}%
\pgfpathlineto{\pgfqpoint{0.827275in}{0.700610in}}%
\pgfpathlineto{\pgfqpoint{0.848007in}{0.694829in}}%
\pgfpathlineto{\pgfqpoint{0.869118in}{0.689397in}}%
\pgfpathlineto{\pgfqpoint{0.890584in}{0.684321in}}%
\pgfpathlineto{\pgfqpoint{0.912382in}{0.679608in}}%
\pgfpathclose%
\pgfusepath{fill}%
\end{pgfscope}%
\begin{pgfscope}%
\pgfpathrectangle{\pgfqpoint{0.041670in}{0.041670in}}{\pgfqpoint{2.216660in}{2.216660in}}%
\pgfusepath{clip}%
\pgfsetbuttcap%
\pgfsetroundjoin%
\definecolor{currentfill}{rgb}{0.201239,0.383670,0.554294}%
\pgfsetfillcolor{currentfill}%
\pgfsetlinewidth{0.000000pt}%
\definecolor{currentstroke}{rgb}{0.000000,0.000000,0.000000}%
\pgfsetstrokecolor{currentstroke}%
\pgfsetdash{}{0pt}%
\pgfpathmoveto{\pgfqpoint{0.818536in}{0.741981in}}%
\pgfpathlineto{\pgfqpoint{0.816321in}{0.753521in}}%
\pgfpathlineto{\pgfqpoint{0.814094in}{0.765558in}}%
\pgfpathlineto{\pgfqpoint{0.811853in}{0.778098in}}%
\pgfpathlineto{\pgfqpoint{0.809600in}{0.791151in}}%
\pgfpathlineto{\pgfqpoint{0.788266in}{0.797502in}}%
\pgfpathlineto{\pgfqpoint{0.767380in}{0.804199in}}%
\pgfpathlineto{\pgfqpoint{0.746964in}{0.811234in}}%
\pgfpathlineto{\pgfqpoint{0.727039in}{0.818596in}}%
\pgfpathlineto{\pgfqpoint{0.729785in}{0.805425in}}%
\pgfpathlineto{\pgfqpoint{0.732514in}{0.792766in}}%
\pgfpathlineto{\pgfqpoint{0.735228in}{0.780609in}}%
\pgfpathlineto{\pgfqpoint{0.737927in}{0.768946in}}%
\pgfpathlineto{\pgfqpoint{0.757383in}{0.761712in}}%
\pgfpathlineto{\pgfqpoint{0.777317in}{0.754800in}}%
\pgfpathlineto{\pgfqpoint{0.797709in}{0.748221in}}%
\pgfpathlineto{\pgfqpoint{0.818536in}{0.741981in}}%
\pgfpathclose%
\pgfusepath{fill}%
\end{pgfscope}%
\begin{pgfscope}%
\pgfpathrectangle{\pgfqpoint{0.041670in}{0.041670in}}{\pgfqpoint{2.216660in}{2.216660in}}%
\pgfusepath{clip}%
\pgfsetbuttcap%
\pgfsetroundjoin%
\definecolor{currentfill}{rgb}{0.260571,0.246922,0.522828}%
\pgfsetfillcolor{currentfill}%
\pgfsetlinewidth{0.000000pt}%
\definecolor{currentstroke}{rgb}{0.000000,0.000000,0.000000}%
\pgfsetstrokecolor{currentstroke}%
\pgfsetdash{}{0pt}%
\pgfpathmoveto{\pgfqpoint{1.373037in}{0.633903in}}%
\pgfpathlineto{\pgfqpoint{1.374234in}{0.641601in}}%
\pgfpathlineto{\pgfqpoint{1.375438in}{0.649733in}}%
\pgfpathlineto{\pgfqpoint{1.376647in}{0.658306in}}%
\pgfpathlineto{\pgfqpoint{1.377862in}{0.667328in}}%
\pgfpathlineto{\pgfqpoint{1.354963in}{0.664166in}}%
\pgfpathlineto{\pgfqpoint{1.331862in}{0.661392in}}%
\pgfpathlineto{\pgfqpoint{1.308586in}{0.659011in}}%
\pgfpathlineto{\pgfqpoint{1.285161in}{0.657026in}}%
\pgfpathlineto{\pgfqpoint{1.284514in}{0.648055in}}%
\pgfpathlineto{\pgfqpoint{1.283870in}{0.639533in}}%
\pgfpathlineto{\pgfqpoint{1.283230in}{0.631453in}}%
\pgfpathlineto{\pgfqpoint{1.282593in}{0.623807in}}%
\pgfpathlineto{\pgfqpoint{1.305447in}{0.625752in}}%
\pgfpathlineto{\pgfqpoint{1.328156in}{0.628086in}}%
\pgfpathlineto{\pgfqpoint{1.350695in}{0.630804in}}%
\pgfpathlineto{\pgfqpoint{1.373037in}{0.633903in}}%
\pgfpathclose%
\pgfusepath{fill}%
\end{pgfscope}%
\begin{pgfscope}%
\pgfpathrectangle{\pgfqpoint{0.041670in}{0.041670in}}{\pgfqpoint{2.216660in}{2.216660in}}%
\pgfusepath{clip}%
\pgfsetbuttcap%
\pgfsetroundjoin%
\definecolor{currentfill}{rgb}{0.260571,0.246922,0.522828}%
\pgfsetfillcolor{currentfill}%
\pgfsetlinewidth{0.000000pt}%
\definecolor{currentstroke}{rgb}{0.000000,0.000000,0.000000}%
\pgfsetstrokecolor{currentstroke}%
\pgfsetdash{}{0pt}%
\pgfpathmoveto{\pgfqpoint{1.097732in}{0.622405in}}%
\pgfpathlineto{\pgfqpoint{1.097221in}{0.630044in}}%
\pgfpathlineto{\pgfqpoint{1.096708in}{0.638117in}}%
\pgfpathlineto{\pgfqpoint{1.096193in}{0.646631in}}%
\pgfpathlineto{\pgfqpoint{1.095674in}{0.655595in}}%
\pgfpathlineto{\pgfqpoint{1.072139in}{0.657227in}}%
\pgfpathlineto{\pgfqpoint{1.048729in}{0.659256in}}%
\pgfpathlineto{\pgfqpoint{1.025471in}{0.661681in}}%
\pgfpathlineto{\pgfqpoint{1.002392in}{0.664498in}}%
\pgfpathlineto{\pgfqpoint{1.003483in}{0.655490in}}%
\pgfpathlineto{\pgfqpoint{1.004568in}{0.646931in}}%
\pgfpathlineto{\pgfqpoint{1.005648in}{0.638813in}}%
\pgfpathlineto{\pgfqpoint{1.006722in}{0.631129in}}%
\pgfpathlineto{\pgfqpoint{1.029240in}{0.628369in}}%
\pgfpathlineto{\pgfqpoint{1.051932in}{0.625993in}}%
\pgfpathlineto{\pgfqpoint{1.074771in}{0.624004in}}%
\pgfpathlineto{\pgfqpoint{1.097732in}{0.622405in}}%
\pgfpathclose%
\pgfusepath{fill}%
\end{pgfscope}%
\begin{pgfscope}%
\pgfpathrectangle{\pgfqpoint{0.041670in}{0.041670in}}{\pgfqpoint{2.216660in}{2.216660in}}%
\pgfusepath{clip}%
\pgfsetbuttcap%
\pgfsetroundjoin%
\definecolor{currentfill}{rgb}{0.260571,0.246922,0.522828}%
\pgfsetfillcolor{currentfill}%
\pgfsetlinewidth{0.000000pt}%
\definecolor{currentstroke}{rgb}{0.000000,0.000000,0.000000}%
\pgfsetstrokecolor{currentstroke}%
\pgfsetdash{}{0pt}%
\pgfpathmoveto{\pgfqpoint{1.282593in}{0.623807in}}%
\pgfpathlineto{\pgfqpoint{1.283230in}{0.631453in}}%
\pgfpathlineto{\pgfqpoint{1.283870in}{0.639533in}}%
\pgfpathlineto{\pgfqpoint{1.284514in}{0.648055in}}%
\pgfpathlineto{\pgfqpoint{1.285161in}{0.657026in}}%
\pgfpathlineto{\pgfqpoint{1.261614in}{0.655439in}}%
\pgfpathlineto{\pgfqpoint{1.237973in}{0.654252in}}%
\pgfpathlineto{\pgfqpoint{1.214264in}{0.653468in}}%
\pgfpathlineto{\pgfqpoint{1.190515in}{0.653087in}}%
\pgfpathlineto{\pgfqpoint{1.190450in}{0.644136in}}%
\pgfpathlineto{\pgfqpoint{1.190386in}{0.635634in}}%
\pgfpathlineto{\pgfqpoint{1.190321in}{0.627573in}}%
\pgfpathlineto{\pgfqpoint{1.190257in}{0.619947in}}%
\pgfpathlineto{\pgfqpoint{1.213426in}{0.620321in}}%
\pgfpathlineto{\pgfqpoint{1.236556in}{0.621089in}}%
\pgfpathlineto{\pgfqpoint{1.259620in}{0.622252in}}%
\pgfpathlineto{\pgfqpoint{1.282593in}{0.623807in}}%
\pgfpathclose%
\pgfusepath{fill}%
\end{pgfscope}%
\begin{pgfscope}%
\pgfpathrectangle{\pgfqpoint{0.041670in}{0.041670in}}{\pgfqpoint{2.216660in}{2.216660in}}%
\pgfusepath{clip}%
\pgfsetbuttcap%
\pgfsetroundjoin%
\definecolor{currentfill}{rgb}{0.260571,0.246922,0.522828}%
\pgfsetfillcolor{currentfill}%
\pgfsetlinewidth{0.000000pt}%
\definecolor{currentstroke}{rgb}{0.000000,0.000000,0.000000}%
\pgfsetstrokecolor{currentstroke}%
\pgfsetdash{}{0pt}%
\pgfpathmoveto{\pgfqpoint{1.190257in}{0.619947in}}%
\pgfpathlineto{\pgfqpoint{1.190321in}{0.627573in}}%
\pgfpathlineto{\pgfqpoint{1.190386in}{0.635634in}}%
\pgfpathlineto{\pgfqpoint{1.190450in}{0.644136in}}%
\pgfpathlineto{\pgfqpoint{1.190515in}{0.653087in}}%
\pgfpathlineto{\pgfqpoint{1.166755in}{0.653109in}}%
\pgfpathlineto{\pgfqpoint{1.143009in}{0.653535in}}%
\pgfpathlineto{\pgfqpoint{1.119306in}{0.654364in}}%
\pgfpathlineto{\pgfqpoint{1.095674in}{0.655595in}}%
\pgfpathlineto{\pgfqpoint{1.096193in}{0.646631in}}%
\pgfpathlineto{\pgfqpoint{1.096708in}{0.638117in}}%
\pgfpathlineto{\pgfqpoint{1.097221in}{0.630044in}}%
\pgfpathlineto{\pgfqpoint{1.097732in}{0.622405in}}%
\pgfpathlineto{\pgfqpoint{1.120788in}{0.621199in}}%
\pgfpathlineto{\pgfqpoint{1.143911in}{0.620387in}}%
\pgfpathlineto{\pgfqpoint{1.167077in}{0.619969in}}%
\pgfpathlineto{\pgfqpoint{1.190257in}{0.619947in}}%
\pgfpathclose%
\pgfusepath{fill}%
\end{pgfscope}%
\begin{pgfscope}%
\pgfpathrectangle{\pgfqpoint{0.041670in}{0.041670in}}{\pgfqpoint{2.216660in}{2.216660in}}%
\pgfusepath{clip}%
\pgfsetbuttcap%
\pgfsetroundjoin%
\definecolor{currentfill}{rgb}{0.233603,0.313828,0.543914}%
\pgfsetfillcolor{currentfill}%
\pgfsetlinewidth{0.000000pt}%
\definecolor{currentstroke}{rgb}{0.000000,0.000000,0.000000}%
\pgfsetstrokecolor{currentstroke}%
\pgfsetdash{}{0pt}%
\pgfpathmoveto{\pgfqpoint{1.466919in}{0.683779in}}%
\pgfpathlineto{\pgfqpoint{1.468687in}{0.693338in}}%
\pgfpathlineto{\pgfqpoint{1.470464in}{0.703360in}}%
\pgfpathlineto{\pgfqpoint{1.472250in}{0.713854in}}%
\pgfpathlineto{\pgfqpoint{1.474047in}{0.724828in}}%
\pgfpathlineto{\pgfqpoint{1.451676in}{0.720066in}}%
\pgfpathlineto{\pgfqpoint{1.428992in}{0.715681in}}%
\pgfpathlineto{\pgfqpoint{1.406021in}{0.711677in}}%
\pgfpathlineto{\pgfqpoint{1.382789in}{0.708061in}}%
\pgfpathlineto{\pgfqpoint{1.381547in}{0.697165in}}%
\pgfpathlineto{\pgfqpoint{1.380312in}{0.686750in}}%
\pgfpathlineto{\pgfqpoint{1.379084in}{0.676807in}}%
\pgfpathlineto{\pgfqpoint{1.377862in}{0.667328in}}%
\pgfpathlineto{\pgfqpoint{1.400533in}{0.670876in}}%
\pgfpathlineto{\pgfqpoint{1.422949in}{0.674804in}}%
\pgfpathlineto{\pgfqpoint{1.445086in}{0.679107in}}%
\pgfpathlineto{\pgfqpoint{1.466919in}{0.683779in}}%
\pgfpathclose%
\pgfusepath{fill}%
\end{pgfscope}%
\begin{pgfscope}%
\pgfpathrectangle{\pgfqpoint{0.041670in}{0.041670in}}{\pgfqpoint{2.216660in}{2.216660in}}%
\pgfusepath{clip}%
\pgfsetbuttcap%
\pgfsetroundjoin%
\definecolor{currentfill}{rgb}{0.233603,0.313828,0.543914}%
\pgfsetfillcolor{currentfill}%
\pgfsetlinewidth{0.000000pt}%
\definecolor{currentstroke}{rgb}{0.000000,0.000000,0.000000}%
\pgfsetstrokecolor{currentstroke}%
\pgfsetdash{}{0pt}%
\pgfpathmoveto{\pgfqpoint{1.002392in}{0.664498in}}%
\pgfpathlineto{\pgfqpoint{1.001295in}{0.673962in}}%
\pgfpathlineto{\pgfqpoint{1.000193in}{0.683892in}}%
\pgfpathlineto{\pgfqpoint{0.999084in}{0.694294in}}%
\pgfpathlineto{\pgfqpoint{0.997970in}{0.705176in}}%
\pgfpathlineto{\pgfqpoint{0.974528in}{0.708444in}}%
\pgfpathlineto{\pgfqpoint{0.951323in}{0.712103in}}%
\pgfpathlineto{\pgfqpoint{0.928383in}{0.716149in}}%
\pgfpathlineto{\pgfqpoint{0.905732in}{0.720577in}}%
\pgfpathlineto{\pgfqpoint{0.907408in}{0.709623in}}%
\pgfpathlineto{\pgfqpoint{0.909075in}{0.699148in}}%
\pgfpathlineto{\pgfqpoint{0.910733in}{0.689146in}}%
\pgfpathlineto{\pgfqpoint{0.912382in}{0.679608in}}%
\pgfpathlineto{\pgfqpoint{0.934487in}{0.675263in}}%
\pgfpathlineto{\pgfqpoint{0.956873in}{0.671294in}}%
\pgfpathlineto{\pgfqpoint{0.979517in}{0.667703in}}%
\pgfpathlineto{\pgfqpoint{1.002392in}{0.664498in}}%
\pgfpathclose%
\pgfusepath{fill}%
\end{pgfscope}%
\begin{pgfscope}%
\pgfpathrectangle{\pgfqpoint{0.041670in}{0.041670in}}{\pgfqpoint{2.216660in}{2.216660in}}%
\pgfusepath{clip}%
\pgfsetbuttcap%
\pgfsetroundjoin%
\definecolor{currentfill}{rgb}{0.172719,0.448791,0.557885}%
\pgfsetfillcolor{currentfill}%
\pgfsetlinewidth{0.000000pt}%
\definecolor{currentstroke}{rgb}{0.000000,0.000000,0.000000}%
\pgfsetstrokecolor{currentstroke}%
\pgfsetdash{}{0pt}%
\pgfpathmoveto{\pgfqpoint{1.650151in}{0.825409in}}%
\pgfpathlineto{\pgfqpoint{1.653016in}{0.839128in}}%
\pgfpathlineto{\pgfqpoint{1.655898in}{0.853377in}}%
\pgfpathlineto{\pgfqpoint{1.658797in}{0.868163in}}%
\pgfpathlineto{\pgfqpoint{1.638979in}{0.860422in}}%
\pgfpathlineto{\pgfqpoint{1.618639in}{0.853004in}}%
\pgfpathlineto{\pgfqpoint{1.597799in}{0.845919in}}%
\pgfpathlineto{\pgfqpoint{1.576483in}{0.839177in}}%
\pgfpathlineto{\pgfqpoint{1.574072in}{0.824508in}}%
\pgfpathlineto{\pgfqpoint{1.571676in}{0.810378in}}%
\pgfpathlineto{\pgfqpoint{1.569295in}{0.796779in}}%
\pgfpathlineto{\pgfqpoint{1.590232in}{0.803438in}}%
\pgfpathlineto{\pgfqpoint{1.610701in}{0.810436in}}%
\pgfpathlineto{\pgfqpoint{1.630682in}{0.817762in}}%
\pgfpathlineto{\pgfqpoint{1.650151in}{0.825409in}}%
\pgfpathclose%
\pgfusepath{fill}%
\end{pgfscope}%
\begin{pgfscope}%
\pgfpathrectangle{\pgfqpoint{0.041670in}{0.041670in}}{\pgfqpoint{2.216660in}{2.216660in}}%
\pgfusepath{clip}%
\pgfsetbuttcap%
\pgfsetroundjoin%
\definecolor{currentfill}{rgb}{0.201239,0.383670,0.554294}%
\pgfsetfillcolor{currentfill}%
\pgfsetlinewidth{0.000000pt}%
\definecolor{currentstroke}{rgb}{0.000000,0.000000,0.000000}%
\pgfsetstrokecolor{currentstroke}%
\pgfsetdash{}{0pt}%
\pgfpathmoveto{\pgfqpoint{1.559908in}{0.747510in}}%
\pgfpathlineto{\pgfqpoint{1.562234in}{0.759076in}}%
\pgfpathlineto{\pgfqpoint{1.564574in}{0.771137in}}%
\pgfpathlineto{\pgfqpoint{1.566928in}{0.783702in}}%
\pgfpathlineto{\pgfqpoint{1.569295in}{0.796779in}}%
\pgfpathlineto{\pgfqpoint{1.547913in}{0.790467in}}%
\pgfpathlineto{\pgfqpoint{1.526109in}{0.784510in}}%
\pgfpathlineto{\pgfqpoint{1.503909in}{0.778916in}}%
\pgfpathlineto{\pgfqpoint{1.481335in}{0.773691in}}%
\pgfpathlineto{\pgfqpoint{1.479497in}{0.760713in}}%
\pgfpathlineto{\pgfqpoint{1.477670in}{0.748249in}}%
\pgfpathlineto{\pgfqpoint{1.475853in}{0.736290in}}%
\pgfpathlineto{\pgfqpoint{1.474047in}{0.724828in}}%
\pgfpathlineto{\pgfqpoint{1.496080in}{0.729961in}}%
\pgfpathlineto{\pgfqpoint{1.517750in}{0.735457in}}%
\pgfpathlineto{\pgfqpoint{1.539034in}{0.741309in}}%
\pgfpathlineto{\pgfqpoint{1.559908in}{0.747510in}}%
\pgfpathclose%
\pgfusepath{fill}%
\end{pgfscope}%
\begin{pgfscope}%
\pgfpathrectangle{\pgfqpoint{0.041670in}{0.041670in}}{\pgfqpoint{2.216660in}{2.216660in}}%
\pgfusepath{clip}%
\pgfsetbuttcap%
\pgfsetroundjoin%
\definecolor{currentfill}{rgb}{0.201239,0.383670,0.554294}%
\pgfsetfillcolor{currentfill}%
\pgfsetlinewidth{0.000000pt}%
\definecolor{currentstroke}{rgb}{0.000000,0.000000,0.000000}%
\pgfsetstrokecolor{currentstroke}%
\pgfsetdash{}{0pt}%
\pgfpathmoveto{\pgfqpoint{0.905732in}{0.720577in}}%
\pgfpathlineto{\pgfqpoint{0.904047in}{0.732020in}}%
\pgfpathlineto{\pgfqpoint{0.902352in}{0.743959in}}%
\pgfpathlineto{\pgfqpoint{0.900647in}{0.756404in}}%
\pgfpathlineto{\pgfqpoint{0.898932in}{0.769363in}}%
\pgfpathlineto{\pgfqpoint{0.876049in}{0.774253in}}%
\pgfpathlineto{\pgfqpoint{0.853515in}{0.779519in}}%
\pgfpathlineto{\pgfqpoint{0.831358in}{0.785154in}}%
\pgfpathlineto{\pgfqpoint{0.809600in}{0.791151in}}%
\pgfpathlineto{\pgfqpoint{0.811853in}{0.778098in}}%
\pgfpathlineto{\pgfqpoint{0.814094in}{0.765558in}}%
\pgfpathlineto{\pgfqpoint{0.816321in}{0.753521in}}%
\pgfpathlineto{\pgfqpoint{0.818536in}{0.741981in}}%
\pgfpathlineto{\pgfqpoint{0.839775in}{0.736089in}}%
\pgfpathlineto{\pgfqpoint{0.861404in}{0.730553in}}%
\pgfpathlineto{\pgfqpoint{0.883398in}{0.725380in}}%
\pgfpathlineto{\pgfqpoint{0.905732in}{0.720577in}}%
\pgfpathclose%
\pgfusepath{fill}%
\end{pgfscope}%
\begin{pgfscope}%
\pgfpathrectangle{\pgfqpoint{0.041670in}{0.041670in}}{\pgfqpoint{2.216660in}{2.216660in}}%
\pgfusepath{clip}%
\pgfsetbuttcap%
\pgfsetroundjoin%
\definecolor{currentfill}{rgb}{0.172719,0.448791,0.557885}%
\pgfsetfillcolor{currentfill}%
\pgfsetlinewidth{0.000000pt}%
\definecolor{currentstroke}{rgb}{0.000000,0.000000,0.000000}%
\pgfsetstrokecolor{currentstroke}%
\pgfsetdash{}{0pt}%
\pgfpathmoveto{\pgfqpoint{0.809600in}{0.791151in}}%
\pgfpathlineto{\pgfqpoint{0.807333in}{0.804726in}}%
\pgfpathlineto{\pgfqpoint{0.805052in}{0.818832in}}%
\pgfpathlineto{\pgfqpoint{0.802757in}{0.833478in}}%
\pgfpathlineto{\pgfqpoint{0.781036in}{0.839909in}}%
\pgfpathlineto{\pgfqpoint{0.759771in}{0.846690in}}%
\pgfpathlineto{\pgfqpoint{0.738986in}{0.853812in}}%
\pgfpathlineto{\pgfqpoint{0.718703in}{0.861266in}}%
\pgfpathlineto{\pgfqpoint{0.721499in}{0.846507in}}%
\pgfpathlineto{\pgfqpoint{0.724277in}{0.832287in}}%
\pgfpathlineto{\pgfqpoint{0.727039in}{0.818596in}}%
\pgfpathlineto{\pgfqpoint{0.746964in}{0.811234in}}%
\pgfpathlineto{\pgfqpoint{0.767380in}{0.804199in}}%
\pgfpathlineto{\pgfqpoint{0.788266in}{0.797502in}}%
\pgfpathlineto{\pgfqpoint{0.809600in}{0.791151in}}%
\pgfpathclose%
\pgfusepath{fill}%
\end{pgfscope}%
\begin{pgfscope}%
\pgfpathrectangle{\pgfqpoint{0.041670in}{0.041670in}}{\pgfqpoint{2.216660in}{2.216660in}}%
\pgfusepath{clip}%
\pgfsetbuttcap%
\pgfsetroundjoin%
\definecolor{currentfill}{rgb}{0.233603,0.313828,0.543914}%
\pgfsetfillcolor{currentfill}%
\pgfsetlinewidth{0.000000pt}%
\definecolor{currentstroke}{rgb}{0.000000,0.000000,0.000000}%
\pgfsetstrokecolor{currentstroke}%
\pgfsetdash{}{0pt}%
\pgfpathmoveto{\pgfqpoint{1.377862in}{0.667328in}}%
\pgfpathlineto{\pgfqpoint{1.379084in}{0.676807in}}%
\pgfpathlineto{\pgfqpoint{1.380312in}{0.686750in}}%
\pgfpathlineto{\pgfqpoint{1.381547in}{0.697165in}}%
\pgfpathlineto{\pgfqpoint{1.382789in}{0.708061in}}%
\pgfpathlineto{\pgfqpoint{1.359322in}{0.704838in}}%
\pgfpathlineto{\pgfqpoint{1.335647in}{0.702010in}}%
\pgfpathlineto{\pgfqpoint{1.311792in}{0.699583in}}%
\pgfpathlineto{\pgfqpoint{1.287784in}{0.697560in}}%
\pgfpathlineto{\pgfqpoint{1.287123in}{0.686713in}}%
\pgfpathlineto{\pgfqpoint{1.286465in}{0.676347in}}%
\pgfpathlineto{\pgfqpoint{1.285811in}{0.666454in}}%
\pgfpathlineto{\pgfqpoint{1.285161in}{0.657026in}}%
\pgfpathlineto{\pgfqpoint{1.308586in}{0.659011in}}%
\pgfpathlineto{\pgfqpoint{1.331862in}{0.661392in}}%
\pgfpathlineto{\pgfqpoint{1.354963in}{0.664166in}}%
\pgfpathlineto{\pgfqpoint{1.377862in}{0.667328in}}%
\pgfpathclose%
\pgfusepath{fill}%
\end{pgfscope}%
\begin{pgfscope}%
\pgfpathrectangle{\pgfqpoint{0.041670in}{0.041670in}}{\pgfqpoint{2.216660in}{2.216660in}}%
\pgfusepath{clip}%
\pgfsetbuttcap%
\pgfsetroundjoin%
\definecolor{currentfill}{rgb}{0.233603,0.313828,0.543914}%
\pgfsetfillcolor{currentfill}%
\pgfsetlinewidth{0.000000pt}%
\definecolor{currentstroke}{rgb}{0.000000,0.000000,0.000000}%
\pgfsetstrokecolor{currentstroke}%
\pgfsetdash{}{0pt}%
\pgfpathmoveto{\pgfqpoint{1.095674in}{0.655595in}}%
\pgfpathlineto{\pgfqpoint{1.095153in}{0.665016in}}%
\pgfpathlineto{\pgfqpoint{1.094629in}{0.674902in}}%
\pgfpathlineto{\pgfqpoint{1.094102in}{0.685261in}}%
\pgfpathlineto{\pgfqpoint{1.093573in}{0.696102in}}%
\pgfpathlineto{\pgfqpoint{1.069452in}{0.697765in}}%
\pgfpathlineto{\pgfqpoint{1.045459in}{0.699833in}}%
\pgfpathlineto{\pgfqpoint{1.021623in}{0.702305in}}%
\pgfpathlineto{\pgfqpoint{0.997970in}{0.705176in}}%
\pgfpathlineto{\pgfqpoint{0.999084in}{0.694294in}}%
\pgfpathlineto{\pgfqpoint{1.000193in}{0.683892in}}%
\pgfpathlineto{\pgfqpoint{1.001295in}{0.673962in}}%
\pgfpathlineto{\pgfqpoint{1.002392in}{0.664498in}}%
\pgfpathlineto{\pgfqpoint{1.025471in}{0.661681in}}%
\pgfpathlineto{\pgfqpoint{1.048729in}{0.659256in}}%
\pgfpathlineto{\pgfqpoint{1.072139in}{0.657227in}}%
\pgfpathlineto{\pgfqpoint{1.095674in}{0.655595in}}%
\pgfpathclose%
\pgfusepath{fill}%
\end{pgfscope}%
\begin{pgfscope}%
\pgfpathrectangle{\pgfqpoint{0.041670in}{0.041670in}}{\pgfqpoint{2.216660in}{2.216660in}}%
\pgfusepath{clip}%
\pgfsetbuttcap%
\pgfsetroundjoin%
\definecolor{currentfill}{rgb}{0.233603,0.313828,0.543914}%
\pgfsetfillcolor{currentfill}%
\pgfsetlinewidth{0.000000pt}%
\definecolor{currentstroke}{rgb}{0.000000,0.000000,0.000000}%
\pgfsetstrokecolor{currentstroke}%
\pgfsetdash{}{0pt}%
\pgfpathmoveto{\pgfqpoint{1.285161in}{0.657026in}}%
\pgfpathlineto{\pgfqpoint{1.285811in}{0.666454in}}%
\pgfpathlineto{\pgfqpoint{1.286465in}{0.676347in}}%
\pgfpathlineto{\pgfqpoint{1.287123in}{0.686713in}}%
\pgfpathlineto{\pgfqpoint{1.287784in}{0.697560in}}%
\pgfpathlineto{\pgfqpoint{1.263650in}{0.695942in}}%
\pgfpathlineto{\pgfqpoint{1.239419in}{0.694733in}}%
\pgfpathlineto{\pgfqpoint{1.215120in}{0.693934in}}%
\pgfpathlineto{\pgfqpoint{1.190779in}{0.693545in}}%
\pgfpathlineto{\pgfqpoint{1.190712in}{0.682717in}}%
\pgfpathlineto{\pgfqpoint{1.190646in}{0.672369in}}%
\pgfpathlineto{\pgfqpoint{1.190581in}{0.662496in}}%
\pgfpathlineto{\pgfqpoint{1.190515in}{0.653087in}}%
\pgfpathlineto{\pgfqpoint{1.214264in}{0.653468in}}%
\pgfpathlineto{\pgfqpoint{1.237973in}{0.654252in}}%
\pgfpathlineto{\pgfqpoint{1.261614in}{0.655439in}}%
\pgfpathlineto{\pgfqpoint{1.285161in}{0.657026in}}%
\pgfpathclose%
\pgfusepath{fill}%
\end{pgfscope}%
\begin{pgfscope}%
\pgfpathrectangle{\pgfqpoint{0.041670in}{0.041670in}}{\pgfqpoint{2.216660in}{2.216660in}}%
\pgfusepath{clip}%
\pgfsetbuttcap%
\pgfsetroundjoin%
\definecolor{currentfill}{rgb}{0.233603,0.313828,0.543914}%
\pgfsetfillcolor{currentfill}%
\pgfsetlinewidth{0.000000pt}%
\definecolor{currentstroke}{rgb}{0.000000,0.000000,0.000000}%
\pgfsetstrokecolor{currentstroke}%
\pgfsetdash{}{0pt}%
\pgfpathmoveto{\pgfqpoint{1.190515in}{0.653087in}}%
\pgfpathlineto{\pgfqpoint{1.190581in}{0.662496in}}%
\pgfpathlineto{\pgfqpoint{1.190646in}{0.672369in}}%
\pgfpathlineto{\pgfqpoint{1.190712in}{0.682717in}}%
\pgfpathlineto{\pgfqpoint{1.190779in}{0.693545in}}%
\pgfpathlineto{\pgfqpoint{1.166425in}{0.693568in}}%
\pgfpathlineto{\pgfqpoint{1.142088in}{0.694002in}}%
\pgfpathlineto{\pgfqpoint{1.117794in}{0.694847in}}%
\pgfpathlineto{\pgfqpoint{1.093573in}{0.696102in}}%
\pgfpathlineto{\pgfqpoint{1.094102in}{0.685261in}}%
\pgfpathlineto{\pgfqpoint{1.094629in}{0.674902in}}%
\pgfpathlineto{\pgfqpoint{1.095153in}{0.665016in}}%
\pgfpathlineto{\pgfqpoint{1.095674in}{0.655595in}}%
\pgfpathlineto{\pgfqpoint{1.119306in}{0.654364in}}%
\pgfpathlineto{\pgfqpoint{1.143009in}{0.653535in}}%
\pgfpathlineto{\pgfqpoint{1.166755in}{0.653109in}}%
\pgfpathlineto{\pgfqpoint{1.190515in}{0.653087in}}%
\pgfpathclose%
\pgfusepath{fill}%
\end{pgfscope}%
\begin{pgfscope}%
\pgfpathrectangle{\pgfqpoint{0.041670in}{0.041670in}}{\pgfqpoint{2.216660in}{2.216660in}}%
\pgfusepath{clip}%
\pgfsetbuttcap%
\pgfsetroundjoin%
\definecolor{currentfill}{rgb}{0.201239,0.383670,0.554294}%
\pgfsetfillcolor{currentfill}%
\pgfsetlinewidth{0.000000pt}%
\definecolor{currentstroke}{rgb}{0.000000,0.000000,0.000000}%
\pgfsetstrokecolor{currentstroke}%
\pgfsetdash{}{0pt}%
\pgfpathmoveto{\pgfqpoint{1.474047in}{0.724828in}}%
\pgfpathlineto{\pgfqpoint{1.475853in}{0.736290in}}%
\pgfpathlineto{\pgfqpoint{1.477670in}{0.748249in}}%
\pgfpathlineto{\pgfqpoint{1.479497in}{0.760713in}}%
\pgfpathlineto{\pgfqpoint{1.481335in}{0.773691in}}%
\pgfpathlineto{\pgfqpoint{1.458414in}{0.768843in}}%
\pgfpathlineto{\pgfqpoint{1.435172in}{0.764379in}}%
\pgfpathlineto{\pgfqpoint{1.411634in}{0.760303in}}%
\pgfpathlineto{\pgfqpoint{1.387827in}{0.756622in}}%
\pgfpathlineto{\pgfqpoint{1.386556in}{0.743718in}}%
\pgfpathlineto{\pgfqpoint{1.385293in}{0.731329in}}%
\pgfpathlineto{\pgfqpoint{1.384037in}{0.719447in}}%
\pgfpathlineto{\pgfqpoint{1.382789in}{0.708061in}}%
\pgfpathlineto{\pgfqpoint{1.406021in}{0.711677in}}%
\pgfpathlineto{\pgfqpoint{1.428992in}{0.715681in}}%
\pgfpathlineto{\pgfqpoint{1.451676in}{0.720066in}}%
\pgfpathlineto{\pgfqpoint{1.474047in}{0.724828in}}%
\pgfpathclose%
\pgfusepath{fill}%
\end{pgfscope}%
\begin{pgfscope}%
\pgfpathrectangle{\pgfqpoint{0.041670in}{0.041670in}}{\pgfqpoint{2.216660in}{2.216660in}}%
\pgfusepath{clip}%
\pgfsetbuttcap%
\pgfsetroundjoin%
\definecolor{currentfill}{rgb}{0.201239,0.383670,0.554294}%
\pgfsetfillcolor{currentfill}%
\pgfsetlinewidth{0.000000pt}%
\definecolor{currentstroke}{rgb}{0.000000,0.000000,0.000000}%
\pgfsetstrokecolor{currentstroke}%
\pgfsetdash{}{0pt}%
\pgfpathmoveto{\pgfqpoint{0.997970in}{0.705176in}}%
\pgfpathlineto{\pgfqpoint{0.996849in}{0.716548in}}%
\pgfpathlineto{\pgfqpoint{0.995722in}{0.728418in}}%
\pgfpathlineto{\pgfqpoint{0.994588in}{0.740794in}}%
\pgfpathlineto{\pgfqpoint{0.993447in}{0.753684in}}%
\pgfpathlineto{\pgfqpoint{0.969425in}{0.757011in}}%
\pgfpathlineto{\pgfqpoint{0.945647in}{0.760736in}}%
\pgfpathlineto{\pgfqpoint{0.922140in}{0.764856in}}%
\pgfpathlineto{\pgfqpoint{0.898932in}{0.769363in}}%
\pgfpathlineto{\pgfqpoint{0.900647in}{0.756404in}}%
\pgfpathlineto{\pgfqpoint{0.902352in}{0.743959in}}%
\pgfpathlineto{\pgfqpoint{0.904047in}{0.732020in}}%
\pgfpathlineto{\pgfqpoint{0.905732in}{0.720577in}}%
\pgfpathlineto{\pgfqpoint{0.928383in}{0.716149in}}%
\pgfpathlineto{\pgfqpoint{0.951323in}{0.712103in}}%
\pgfpathlineto{\pgfqpoint{0.974528in}{0.708444in}}%
\pgfpathlineto{\pgfqpoint{0.997970in}{0.705176in}}%
\pgfpathclose%
\pgfusepath{fill}%
\end{pgfscope}%
\begin{pgfscope}%
\pgfpathrectangle{\pgfqpoint{0.041670in}{0.041670in}}{\pgfqpoint{2.216660in}{2.216660in}}%
\pgfusepath{clip}%
\pgfsetbuttcap%
\pgfsetroundjoin%
\definecolor{currentfill}{rgb}{0.172719,0.448791,0.557885}%
\pgfsetfillcolor{currentfill}%
\pgfsetlinewidth{0.000000pt}%
\definecolor{currentstroke}{rgb}{0.000000,0.000000,0.000000}%
\pgfsetstrokecolor{currentstroke}%
\pgfsetdash{}{0pt}%
\pgfpathmoveto{\pgfqpoint{1.569295in}{0.796779in}}%
\pgfpathlineto{\pgfqpoint{1.571676in}{0.810378in}}%
\pgfpathlineto{\pgfqpoint{1.574072in}{0.824508in}}%
\pgfpathlineto{\pgfqpoint{1.576483in}{0.839177in}}%
\pgfpathlineto{\pgfqpoint{1.554712in}{0.832785in}}%
\pgfpathlineto{\pgfqpoint{1.532511in}{0.826753in}}%
\pgfpathlineto{\pgfqpoint{1.509904in}{0.821088in}}%
\pgfpathlineto{\pgfqpoint{1.486917in}{0.815797in}}%
\pgfpathlineto{\pgfqpoint{1.485045in}{0.801224in}}%
\pgfpathlineto{\pgfqpoint{1.483184in}{0.787192in}}%
\pgfpathlineto{\pgfqpoint{1.481335in}{0.773691in}}%
\pgfpathlineto{\pgfqpoint{1.503909in}{0.778916in}}%
\pgfpathlineto{\pgfqpoint{1.526109in}{0.784510in}}%
\pgfpathlineto{\pgfqpoint{1.547913in}{0.790467in}}%
\pgfpathlineto{\pgfqpoint{1.569295in}{0.796779in}}%
\pgfpathclose%
\pgfusepath{fill}%
\end{pgfscope}%
\begin{pgfscope}%
\pgfpathrectangle{\pgfqpoint{0.041670in}{0.041670in}}{\pgfqpoint{2.216660in}{2.216660in}}%
\pgfusepath{clip}%
\pgfsetbuttcap%
\pgfsetroundjoin%
\definecolor{currentfill}{rgb}{0.172719,0.448791,0.557885}%
\pgfsetfillcolor{currentfill}%
\pgfsetlinewidth{0.000000pt}%
\definecolor{currentstroke}{rgb}{0.000000,0.000000,0.000000}%
\pgfsetstrokecolor{currentstroke}%
\pgfsetdash{}{0pt}%
\pgfpathmoveto{\pgfqpoint{0.898932in}{0.769363in}}%
\pgfpathlineto{\pgfqpoint{0.897207in}{0.782845in}}%
\pgfpathlineto{\pgfqpoint{0.895471in}{0.796859in}}%
\pgfpathlineto{\pgfqpoint{0.893724in}{0.811415in}}%
\pgfpathlineto{\pgfqpoint{0.870420in}{0.816367in}}%
\pgfpathlineto{\pgfqpoint{0.847474in}{0.821699in}}%
\pgfpathlineto{\pgfqpoint{0.824911in}{0.827405in}}%
\pgfpathlineto{\pgfqpoint{0.802757in}{0.833478in}}%
\pgfpathlineto{\pgfqpoint{0.805052in}{0.818832in}}%
\pgfpathlineto{\pgfqpoint{0.807333in}{0.804726in}}%
\pgfpathlineto{\pgfqpoint{0.809600in}{0.791151in}}%
\pgfpathlineto{\pgfqpoint{0.831358in}{0.785154in}}%
\pgfpathlineto{\pgfqpoint{0.853515in}{0.779519in}}%
\pgfpathlineto{\pgfqpoint{0.876049in}{0.774253in}}%
\pgfpathlineto{\pgfqpoint{0.898932in}{0.769363in}}%
\pgfpathclose%
\pgfusepath{fill}%
\end{pgfscope}%
\begin{pgfscope}%
\pgfpathrectangle{\pgfqpoint{0.041670in}{0.041670in}}{\pgfqpoint{2.216660in}{2.216660in}}%
\pgfusepath{clip}%
\pgfsetbuttcap%
\pgfsetroundjoin%
\definecolor{currentfill}{rgb}{0.201239,0.383670,0.554294}%
\pgfsetfillcolor{currentfill}%
\pgfsetlinewidth{0.000000pt}%
\definecolor{currentstroke}{rgb}{0.000000,0.000000,0.000000}%
\pgfsetstrokecolor{currentstroke}%
\pgfsetdash{}{0pt}%
\pgfpathmoveto{\pgfqpoint{1.382789in}{0.708061in}}%
\pgfpathlineto{\pgfqpoint{1.384037in}{0.719447in}}%
\pgfpathlineto{\pgfqpoint{1.385293in}{0.731329in}}%
\pgfpathlineto{\pgfqpoint{1.386556in}{0.743718in}}%
\pgfpathlineto{\pgfqpoint{1.387827in}{0.756622in}}%
\pgfpathlineto{\pgfqpoint{1.363779in}{0.753339in}}%
\pgfpathlineto{\pgfqpoint{1.339518in}{0.750461in}}%
\pgfpathlineto{\pgfqpoint{1.315071in}{0.747990in}}%
\pgfpathlineto{\pgfqpoint{1.290466in}{0.745929in}}%
\pgfpathlineto{\pgfqpoint{1.289790in}{0.733073in}}%
\pgfpathlineto{\pgfqpoint{1.289117in}{0.720731in}}%
\pgfpathlineto{\pgfqpoint{1.288448in}{0.708897in}}%
\pgfpathlineto{\pgfqpoint{1.287784in}{0.697560in}}%
\pgfpathlineto{\pgfqpoint{1.311792in}{0.699583in}}%
\pgfpathlineto{\pgfqpoint{1.335647in}{0.702010in}}%
\pgfpathlineto{\pgfqpoint{1.359322in}{0.704838in}}%
\pgfpathlineto{\pgfqpoint{1.382789in}{0.708061in}}%
\pgfpathclose%
\pgfusepath{fill}%
\end{pgfscope}%
\begin{pgfscope}%
\pgfpathrectangle{\pgfqpoint{0.041670in}{0.041670in}}{\pgfqpoint{2.216660in}{2.216660in}}%
\pgfusepath{clip}%
\pgfsetbuttcap%
\pgfsetroundjoin%
\definecolor{currentfill}{rgb}{0.201239,0.383670,0.554294}%
\pgfsetfillcolor{currentfill}%
\pgfsetlinewidth{0.000000pt}%
\definecolor{currentstroke}{rgb}{0.000000,0.000000,0.000000}%
\pgfsetstrokecolor{currentstroke}%
\pgfsetdash{}{0pt}%
\pgfpathmoveto{\pgfqpoint{1.093573in}{0.696102in}}%
\pgfpathlineto{\pgfqpoint{1.093040in}{0.707432in}}%
\pgfpathlineto{\pgfqpoint{1.092504in}{0.719260in}}%
\pgfpathlineto{\pgfqpoint{1.091965in}{0.731595in}}%
\pgfpathlineto{\pgfqpoint{1.091423in}{0.744445in}}%
\pgfpathlineto{\pgfqpoint{1.066703in}{0.746138in}}%
\pgfpathlineto{\pgfqpoint{1.042114in}{0.748244in}}%
\pgfpathlineto{\pgfqpoint{1.017686in}{0.750761in}}%
\pgfpathlineto{\pgfqpoint{0.993447in}{0.753684in}}%
\pgfpathlineto{\pgfqpoint{0.994588in}{0.740794in}}%
\pgfpathlineto{\pgfqpoint{0.995722in}{0.728418in}}%
\pgfpathlineto{\pgfqpoint{0.996849in}{0.716548in}}%
\pgfpathlineto{\pgfqpoint{0.997970in}{0.705176in}}%
\pgfpathlineto{\pgfqpoint{1.021623in}{0.702305in}}%
\pgfpathlineto{\pgfqpoint{1.045459in}{0.699833in}}%
\pgfpathlineto{\pgfqpoint{1.069452in}{0.697765in}}%
\pgfpathlineto{\pgfqpoint{1.093573in}{0.696102in}}%
\pgfpathclose%
\pgfusepath{fill}%
\end{pgfscope}%
\begin{pgfscope}%
\pgfpathrectangle{\pgfqpoint{0.041670in}{0.041670in}}{\pgfqpoint{2.216660in}{2.216660in}}%
\pgfusepath{clip}%
\pgfsetbuttcap%
\pgfsetroundjoin%
\definecolor{currentfill}{rgb}{0.201239,0.383670,0.554294}%
\pgfsetfillcolor{currentfill}%
\pgfsetlinewidth{0.000000pt}%
\definecolor{currentstroke}{rgb}{0.000000,0.000000,0.000000}%
\pgfsetstrokecolor{currentstroke}%
\pgfsetdash{}{0pt}%
\pgfpathmoveto{\pgfqpoint{1.287784in}{0.697560in}}%
\pgfpathlineto{\pgfqpoint{1.288448in}{0.708897in}}%
\pgfpathlineto{\pgfqpoint{1.289117in}{0.720731in}}%
\pgfpathlineto{\pgfqpoint{1.289790in}{0.733073in}}%
\pgfpathlineto{\pgfqpoint{1.290466in}{0.745929in}}%
\pgfpathlineto{\pgfqpoint{1.265733in}{0.744282in}}%
\pgfpathlineto{\pgfqpoint{1.240899in}{0.743051in}}%
\pgfpathlineto{\pgfqpoint{1.215995in}{0.742237in}}%
\pgfpathlineto{\pgfqpoint{1.191048in}{0.741842in}}%
\pgfpathlineto{\pgfqpoint{1.190980in}{0.729003in}}%
\pgfpathlineto{\pgfqpoint{1.190913in}{0.716680in}}%
\pgfpathlineto{\pgfqpoint{1.190845in}{0.704863in}}%
\pgfpathlineto{\pgfqpoint{1.190779in}{0.693545in}}%
\pgfpathlineto{\pgfqpoint{1.215120in}{0.693934in}}%
\pgfpathlineto{\pgfqpoint{1.239419in}{0.694733in}}%
\pgfpathlineto{\pgfqpoint{1.263650in}{0.695942in}}%
\pgfpathlineto{\pgfqpoint{1.287784in}{0.697560in}}%
\pgfpathclose%
\pgfusepath{fill}%
\end{pgfscope}%
\begin{pgfscope}%
\pgfpathrectangle{\pgfqpoint{0.041670in}{0.041670in}}{\pgfqpoint{2.216660in}{2.216660in}}%
\pgfusepath{clip}%
\pgfsetbuttcap%
\pgfsetroundjoin%
\definecolor{currentfill}{rgb}{0.201239,0.383670,0.554294}%
\pgfsetfillcolor{currentfill}%
\pgfsetlinewidth{0.000000pt}%
\definecolor{currentstroke}{rgb}{0.000000,0.000000,0.000000}%
\pgfsetstrokecolor{currentstroke}%
\pgfsetdash{}{0pt}%
\pgfpathmoveto{\pgfqpoint{1.190779in}{0.693545in}}%
\pgfpathlineto{\pgfqpoint{1.190845in}{0.704863in}}%
\pgfpathlineto{\pgfqpoint{1.190913in}{0.716680in}}%
\pgfpathlineto{\pgfqpoint{1.190980in}{0.729003in}}%
\pgfpathlineto{\pgfqpoint{1.191048in}{0.741842in}}%
\pgfpathlineto{\pgfqpoint{1.166089in}{0.741865in}}%
\pgfpathlineto{\pgfqpoint{1.141145in}{0.742307in}}%
\pgfpathlineto{\pgfqpoint{1.116247in}{0.743167in}}%
\pgfpathlineto{\pgfqpoint{1.091423in}{0.744445in}}%
\pgfpathlineto{\pgfqpoint{1.091965in}{0.731595in}}%
\pgfpathlineto{\pgfqpoint{1.092504in}{0.719260in}}%
\pgfpathlineto{\pgfqpoint{1.093040in}{0.707432in}}%
\pgfpathlineto{\pgfqpoint{1.093573in}{0.696102in}}%
\pgfpathlineto{\pgfqpoint{1.117794in}{0.694847in}}%
\pgfpathlineto{\pgfqpoint{1.142088in}{0.694002in}}%
\pgfpathlineto{\pgfqpoint{1.166425in}{0.693568in}}%
\pgfpathlineto{\pgfqpoint{1.190779in}{0.693545in}}%
\pgfpathclose%
\pgfusepath{fill}%
\end{pgfscope}%
\begin{pgfscope}%
\pgfpathrectangle{\pgfqpoint{0.041670in}{0.041670in}}{\pgfqpoint{2.216660in}{2.216660in}}%
\pgfusepath{clip}%
\pgfsetbuttcap%
\pgfsetroundjoin%
\definecolor{currentfill}{rgb}{0.172719,0.448791,0.557885}%
\pgfsetfillcolor{currentfill}%
\pgfsetlinewidth{0.000000pt}%
\definecolor{currentstroke}{rgb}{0.000000,0.000000,0.000000}%
\pgfsetstrokecolor{currentstroke}%
\pgfsetdash{}{0pt}%
\pgfpathmoveto{\pgfqpoint{1.481335in}{0.773691in}}%
\pgfpathlineto{\pgfqpoint{1.483184in}{0.787192in}}%
\pgfpathlineto{\pgfqpoint{1.485045in}{0.801224in}}%
\pgfpathlineto{\pgfqpoint{1.486917in}{0.815797in}}%
\pgfpathlineto{\pgfqpoint{1.463575in}{0.810888in}}%
\pgfpathlineto{\pgfqpoint{1.439905in}{0.806367in}}%
\pgfpathlineto{\pgfqpoint{1.415933in}{0.802239in}}%
\pgfpathlineto{\pgfqpoint{1.391686in}{0.798511in}}%
\pgfpathlineto{\pgfqpoint{1.390392in}{0.784009in}}%
\pgfpathlineto{\pgfqpoint{1.389106in}{0.770049in}}%
\pgfpathlineto{\pgfqpoint{1.387827in}{0.756622in}}%
\pgfpathlineto{\pgfqpoint{1.411634in}{0.760303in}}%
\pgfpathlineto{\pgfqpoint{1.435172in}{0.764379in}}%
\pgfpathlineto{\pgfqpoint{1.458414in}{0.768843in}}%
\pgfpathlineto{\pgfqpoint{1.481335in}{0.773691in}}%
\pgfpathclose%
\pgfusepath{fill}%
\end{pgfscope}%
\begin{pgfscope}%
\pgfpathrectangle{\pgfqpoint{0.041670in}{0.041670in}}{\pgfqpoint{2.216660in}{2.216660in}}%
\pgfusepath{clip}%
\pgfsetbuttcap%
\pgfsetroundjoin%
\definecolor{currentfill}{rgb}{0.172719,0.448791,0.557885}%
\pgfsetfillcolor{currentfill}%
\pgfsetlinewidth{0.000000pt}%
\definecolor{currentstroke}{rgb}{0.000000,0.000000,0.000000}%
\pgfsetstrokecolor{currentstroke}%
\pgfsetdash{}{0pt}%
\pgfpathmoveto{\pgfqpoint{0.993447in}{0.753684in}}%
\pgfpathlineto{\pgfqpoint{0.992300in}{0.767099in}}%
\pgfpathlineto{\pgfqpoint{0.991145in}{0.781046in}}%
\pgfpathlineto{\pgfqpoint{0.989983in}{0.795536in}}%
\pgfpathlineto{\pgfqpoint{0.965517in}{0.798905in}}%
\pgfpathlineto{\pgfqpoint{0.941299in}{0.802678in}}%
\pgfpathlineto{\pgfqpoint{0.917359in}{0.806850in}}%
\pgfpathlineto{\pgfqpoint{0.893724in}{0.811415in}}%
\pgfpathlineto{\pgfqpoint{0.895471in}{0.796859in}}%
\pgfpathlineto{\pgfqpoint{0.897207in}{0.782845in}}%
\pgfpathlineto{\pgfqpoint{0.898932in}{0.769363in}}%
\pgfpathlineto{\pgfqpoint{0.922140in}{0.764856in}}%
\pgfpathlineto{\pgfqpoint{0.945647in}{0.760736in}}%
\pgfpathlineto{\pgfqpoint{0.969425in}{0.757011in}}%
\pgfpathlineto{\pgfqpoint{0.993447in}{0.753684in}}%
\pgfpathclose%
\pgfusepath{fill}%
\end{pgfscope}%
\begin{pgfscope}%
\pgfpathrectangle{\pgfqpoint{0.041670in}{0.041670in}}{\pgfqpoint{2.216660in}{2.216660in}}%
\pgfusepath{clip}%
\pgfsetbuttcap%
\pgfsetroundjoin%
\definecolor{currentfill}{rgb}{0.172719,0.448791,0.557885}%
\pgfsetfillcolor{currentfill}%
\pgfsetlinewidth{0.000000pt}%
\definecolor{currentstroke}{rgb}{0.000000,0.000000,0.000000}%
\pgfsetstrokecolor{currentstroke}%
\pgfsetdash{}{0pt}%
\pgfpathmoveto{\pgfqpoint{1.387827in}{0.756622in}}%
\pgfpathlineto{\pgfqpoint{1.389106in}{0.770049in}}%
\pgfpathlineto{\pgfqpoint{1.390392in}{0.784009in}}%
\pgfpathlineto{\pgfqpoint{1.391686in}{0.798511in}}%
\pgfpathlineto{\pgfqpoint{1.367194in}{0.795187in}}%
\pgfpathlineto{\pgfqpoint{1.342483in}{0.792272in}}%
\pgfpathlineto{\pgfqpoint{1.317582in}{0.789769in}}%
\pgfpathlineto{\pgfqpoint{1.292521in}{0.787682in}}%
\pgfpathlineto{\pgfqpoint{1.291832in}{0.773225in}}%
\pgfpathlineto{\pgfqpoint{1.291147in}{0.759311in}}%
\pgfpathlineto{\pgfqpoint{1.290466in}{0.745929in}}%
\pgfpathlineto{\pgfqpoint{1.315071in}{0.747990in}}%
\pgfpathlineto{\pgfqpoint{1.339518in}{0.750461in}}%
\pgfpathlineto{\pgfqpoint{1.363779in}{0.753339in}}%
\pgfpathlineto{\pgfqpoint{1.387827in}{0.756622in}}%
\pgfpathclose%
\pgfusepath{fill}%
\end{pgfscope}%
\begin{pgfscope}%
\pgfpathrectangle{\pgfqpoint{0.041670in}{0.041670in}}{\pgfqpoint{2.216660in}{2.216660in}}%
\pgfusepath{clip}%
\pgfsetbuttcap%
\pgfsetroundjoin%
\definecolor{currentfill}{rgb}{0.172719,0.448791,0.557885}%
\pgfsetfillcolor{currentfill}%
\pgfsetlinewidth{0.000000pt}%
\definecolor{currentstroke}{rgb}{0.000000,0.000000,0.000000}%
\pgfsetstrokecolor{currentstroke}%
\pgfsetdash{}{0pt}%
\pgfpathmoveto{\pgfqpoint{1.091423in}{0.744445in}}%
\pgfpathlineto{\pgfqpoint{1.090878in}{0.757820in}}%
\pgfpathlineto{\pgfqpoint{1.090329in}{0.771728in}}%
\pgfpathlineto{\pgfqpoint{1.089777in}{0.786179in}}%
\pgfpathlineto{\pgfqpoint{1.064597in}{0.787893in}}%
\pgfpathlineto{\pgfqpoint{1.039552in}{0.790027in}}%
\pgfpathlineto{\pgfqpoint{1.014671in}{0.792575in}}%
\pgfpathlineto{\pgfqpoint{0.989983in}{0.795536in}}%
\pgfpathlineto{\pgfqpoint{0.991145in}{0.781046in}}%
\pgfpathlineto{\pgfqpoint{0.992300in}{0.767099in}}%
\pgfpathlineto{\pgfqpoint{0.993447in}{0.753684in}}%
\pgfpathlineto{\pgfqpoint{1.017686in}{0.750761in}}%
\pgfpathlineto{\pgfqpoint{1.042114in}{0.748244in}}%
\pgfpathlineto{\pgfqpoint{1.066703in}{0.746138in}}%
\pgfpathlineto{\pgfqpoint{1.091423in}{0.744445in}}%
\pgfpathclose%
\pgfusepath{fill}%
\end{pgfscope}%
\begin{pgfscope}%
\pgfpathrectangle{\pgfqpoint{0.041670in}{0.041670in}}{\pgfqpoint{2.216660in}{2.216660in}}%
\pgfusepath{clip}%
\pgfsetbuttcap%
\pgfsetroundjoin%
\definecolor{currentfill}{rgb}{0.172719,0.448791,0.557885}%
\pgfsetfillcolor{currentfill}%
\pgfsetlinewidth{0.000000pt}%
\definecolor{currentstroke}{rgb}{0.000000,0.000000,0.000000}%
\pgfsetstrokecolor{currentstroke}%
\pgfsetdash{}{0pt}%
\pgfpathmoveto{\pgfqpoint{1.290466in}{0.745929in}}%
\pgfpathlineto{\pgfqpoint{1.291147in}{0.759311in}}%
\pgfpathlineto{\pgfqpoint{1.291832in}{0.773225in}}%
\pgfpathlineto{\pgfqpoint{1.292521in}{0.787682in}}%
\pgfpathlineto{\pgfqpoint{1.267328in}{0.786014in}}%
\pgfpathlineto{\pgfqpoint{1.242033in}{0.784767in}}%
\pgfpathlineto{\pgfqpoint{1.216665in}{0.783943in}}%
\pgfpathlineto{\pgfqpoint{1.191255in}{0.783542in}}%
\pgfpathlineto{\pgfqpoint{1.191185in}{0.769102in}}%
\pgfpathlineto{\pgfqpoint{1.191117in}{0.755205in}}%
\pgfpathlineto{\pgfqpoint{1.191048in}{0.741842in}}%
\pgfpathlineto{\pgfqpoint{1.215995in}{0.742237in}}%
\pgfpathlineto{\pgfqpoint{1.240899in}{0.743051in}}%
\pgfpathlineto{\pgfqpoint{1.265733in}{0.744282in}}%
\pgfpathlineto{\pgfqpoint{1.290466in}{0.745929in}}%
\pgfpathclose%
\pgfusepath{fill}%
\end{pgfscope}%
\begin{pgfscope}%
\pgfpathrectangle{\pgfqpoint{0.041670in}{0.041670in}}{\pgfqpoint{2.216660in}{2.216660in}}%
\pgfusepath{clip}%
\pgfsetbuttcap%
\pgfsetroundjoin%
\definecolor{currentfill}{rgb}{0.172719,0.448791,0.557885}%
\pgfsetfillcolor{currentfill}%
\pgfsetlinewidth{0.000000pt}%
\definecolor{currentstroke}{rgb}{0.000000,0.000000,0.000000}%
\pgfsetstrokecolor{currentstroke}%
\pgfsetdash{}{0pt}%
\pgfpathmoveto{\pgfqpoint{1.191048in}{0.741842in}}%
\pgfpathlineto{\pgfqpoint{1.191117in}{0.755205in}}%
\pgfpathlineto{\pgfqpoint{1.191185in}{0.769102in}}%
\pgfpathlineto{\pgfqpoint{1.191255in}{0.783542in}}%
\pgfpathlineto{\pgfqpoint{1.165831in}{0.783565in}}%
\pgfpathlineto{\pgfqpoint{1.140423in}{0.784013in}}%
\pgfpathlineto{\pgfqpoint{1.115062in}{0.784885in}}%
\pgfpathlineto{\pgfqpoint{1.089777in}{0.786179in}}%
\pgfpathlineto{\pgfqpoint{1.090329in}{0.771728in}}%
\pgfpathlineto{\pgfqpoint{1.090878in}{0.757820in}}%
\pgfpathlineto{\pgfqpoint{1.091423in}{0.744445in}}%
\pgfpathlineto{\pgfqpoint{1.116247in}{0.743167in}}%
\pgfpathlineto{\pgfqpoint{1.141145in}{0.742307in}}%
\pgfpathlineto{\pgfqpoint{1.166089in}{0.741865in}}%
\pgfpathlineto{\pgfqpoint{1.191048in}{0.741842in}}%
\pgfpathclose%
\pgfusepath{fill}%
\end{pgfscope}%
\begin{pgfscope}%
\pgfpathrectangle{\pgfqpoint{0.041670in}{0.041670in}}{\pgfqpoint{2.216660in}{2.216660in}}%
\pgfusepath{clip}%
\pgfsetbuttcap%
\pgfsetroundjoin%
\pgfsetlinewidth{1.505625pt}%
\definecolor{currentstroke}{rgb}{0.000000,0.000000,0.000000}%
\pgfsetstrokecolor{currentstroke}%
\pgfsetdash{}{0pt}%
\pgfpathmoveto{\pgfqpoint{0.729297in}{0.393877in}}%
\pgfpathlineto{\pgfqpoint{1.179955in}{1.026681in}}%
\pgfusepath{stroke}%
\end{pgfscope}%
\begin{pgfscope}%
\pgfpathrectangle{\pgfqpoint{0.041670in}{0.041670in}}{\pgfqpoint{2.216660in}{2.216660in}}%
\pgfusepath{clip}%
\pgfsetbuttcap%
\pgfsetroundjoin%
\pgfsetlinewidth{1.505625pt}%
\definecolor{currentstroke}{rgb}{0.000000,0.000000,0.000000}%
\pgfsetstrokecolor{currentstroke}%
\pgfsetdash{}{0pt}%
\pgfpathmoveto{\pgfqpoint{0.729297in}{0.393877in}}%
\pgfpathlineto{\pgfqpoint{0.754715in}{0.342371in}}%
\pgfusepath{stroke}%
\end{pgfscope}%
\begin{pgfscope}%
\pgfpathrectangle{\pgfqpoint{0.041670in}{0.041670in}}{\pgfqpoint{2.216660in}{2.216660in}}%
\pgfusepath{clip}%
\pgfsetbuttcap%
\pgfsetroundjoin%
\pgfsetlinewidth{1.505625pt}%
\definecolor{currentstroke}{rgb}{0.000000,0.000000,0.000000}%
\pgfsetstrokecolor{currentstroke}%
\pgfsetdash{}{0pt}%
\pgfpathmoveto{\pgfqpoint{0.729297in}{0.393877in}}%
\pgfpathlineto{\pgfqpoint{0.751085in}{0.512414in}}%
\pgfusepath{stroke}%
\end{pgfscope}%
\end{pgfpicture}%
\makeatother%
\endgroup%

	\end{subfigure}
	\caption{(\subref{sfig:Landau free energy}) Landau free energy and (\subref{sfig:Ginzburg Landau free energy}) Mexican hat potential} 
	\label{fig:Landau free energy and Ginzburg-Landau free energy}
\end{figure}

\Cref{sfig:Landau free energy} shows the free energy as a function of a single-component, real order parameter \(\Psi\) and it illustrates the essence of Landau theory: there are two cases for the minima of the free energy \(f\)
\begin{equation}
	\Psi = \begin{cases}
		0 & T \geq T_C \\
		\pm \sqrt{\frac{a (T_C - T)}{u}} & T < T_C
	\end{cases} \;,
\end{equation}
so there is a for \(T < T_C\) there are two minima corresponding to ground states with broken symmetry.
When the order parameter can be calculated from some microscopic theory, the critical temperature \(T_C\) can be extracted from the behavior of the order parameter near \(T_C\) via a linear fit of
\begin{equation}
	\vert \Psi \vert^2 \propto T_C - T \;.
\end{equation}

Generalizing this from a one to an \(n\)-component order parameters is straightforward.
One example is the complex or two component order parameter that will become important for 
\begin{equation}
	\Psi = \Psi_1 + \iu \Psi_2 = \vert \Psi \vert e^{\iu \phi} \,.
\end{equation}
The Landau free energy then takes the form
\begin{equation}
	f_L [\Psi] = r \Psi^* \Psi + \frac{u}{2} (\Psi^* \Psi)^2 = r \vert \Psi \vert^2 + \frac{u}{2} \vert \Psi \vert^4
\end{equation}
with again
\begin{equation}
	r = a(T_C - T) \;.
\end{equation}
Instead of the two minima, the free energy here is rotational symmetry, because it is independent of the phase of the order parameter: 
\begin{equation}
	f_L [\Psi] = f_L [e^{\iu \phi} \Psi] \;.
\end{equation}
This gives the so called `Mexican hat' potential shown in \cref{sfig:Ginzburg Landau free energy}.
In this potential, the order parameter can be rotated continuously from one symmetry-broken state to another.

In 1950, Ginzburg and Landau published their theory of superconductivity, based on Landaus theory of phase transitions \cite{ginzburgTheorySuperconductivity1950}.
Where Landau theory as described above has an uniform order parameters, Ginzburg-Landau theory accounts for it being inhomogeneous, so an order parameter with spatially varying amplitude or direction.
This in turn leads to the order parameter developing a fixed phase, which is the underlying mechanism of the superflow in superconductors.

Ginzburg-Landau theory can be developed for a general \(n\)-component order parameter, but in superfluids and superconductors the order parameter is complex, i.e. two-component.
The Ginzburg-Landau free energy for a complex order parameter is
\begin{equation}
	f_{\mathrm{GL}} [\Psi, \adif{\Psi}] = \frac{\hbar^2}{2m^*} \vert \adif{\Psi} \vert^2 + r \vert \Psi \vert^2 + \frac{u}{2} \vert \Psi \vert^4 \;,
	\label{eq:free energy ginzburg-landau theory}
\end{equation}
where the gradient term \(\adif{\Psi}\) is added in comparison to the Landau free energy.
The prefactor \(\frac{\hbar^2}{2m^*}\) is chosen to illustrate the interpretation of the Ginzburg-Landau free energy as the energy of a condensate of bosons, where the gradient term \(\vert \adif{\Psi} \vert^2\) is the kinetic energy.
The free energy in \cref{eq:free energy ginzburg-landau theory} is sensitive to a twist of the phase of the order parameter.
Substituting the expression \(\Psi = \vert \Psi \vert e^{\iu \phi}\), the gradient term reads
\begin{equation}
	\Delta \Psi = (\adif{\vert \Psi \vert} + \iu \adif{\phi} \vert \Psi \vert) e^{\iu \phi} \;.
\end{equation}
With that, \cref{eq:free energy ginzburg-landau theory} becomes
\begin{equation}
	f_{GL}  = \frac{\hbar^2}{2m^*} \vert \Psi \vert^2 (\adif{\phi})^2 + \left[ \frac{\hbar^2}{2m^*} (\adif{\vert \Psi \vert})^2 + r \vert \Psi \vert^2 + \frac{u}{2} \vert \Psi \vert^4 \right] \;.
	\label{eq:free energy ginzburg-landau theory with phase}
\end{equation}
Now the contributions of phase and amplitude variations are split up: the first term describes energy cost of variations in the phase of the order parameter and the second term describes energy cost of variations in the magnitude of the order parameter.
The dominating fluctuation is determined by the ratio of the factors \(\frac{\hbar^2}{2m^*}\) and \(r\), which has the dimension \(\mathrm{Length}^2\), from which define the \gls{correlation length}
\begin{equation}
	\xi = \sqrt{\frac{\hbar^2}{2m^* \vert r \vert}} = \xi_0 \left(1 - \frac{T}{T_C}\right)^{-\frac{1}{2}}
	\label{eq:correlation length GL theory}
\end{equation}
where we define the \gls{coherence length} \(\xi_0 = \xi(T=0) = \sqrt{\frac{\hbar^2}{2 m a T_C}}\).
On length scales above \(\xi\), the physics is entirely controlled by the phase degrees of freedom, i.e.
\begin{equation}
	f_{\mathrm{GL}} = \frac{\hbar^2}{2 m^*} \vert \Psi \vert^2 (\adif{\phi})^2 + \mathrm{const.}
\end{equation}
Take case of frozen amplitude fluctuations, i.e. \(\adif{\vert \Psi (\vb{r}) \vert} = 0\).
Stationary point condition for \cref{eq:free energy ginzburg-landau theory with phase} gives:
\begin{equation}
	\vert \Psi \vert = \vert \Psi_0 \vert \sqrt{1 - \xi^2 \vert \adif{\phi} (\vb{r}) \vert^2}
	\label{eq:breakdown of OP with phase fluctuations}
\end{equation}
\todo{Work over paragraph}
This shows that the superconducting order gets suppressed and eventually destroyed by short-ranged (below \(\xi\)) phase fluctuations.
By introducing a particular form of phase fluctuations \(\phi = \vb{q} \cdot \vb{r}\) into a microscopic model, it is possible to probe this breakdown of superconductivity and thus gain insight into the nature of superconductivity, in particular this gives access to \(\xi\).

Superconductors: charged superfluids, coupling to electromagnetic fields.
Free energy with minimal coupling to an electromagnetic field:
\begin{equation}
	f_{\mathrm{GL}} [\Psi, \vb{A}] = \frac{\hbar^2}{2 m^*} \left\vert \left( \Delta - \frac{\iu e^*}{\hbar} \vb{A} \right) \Psi \right\vert^2 + r \vert \Psi \vert^2 + \frac{u}{2} \vert \Psi \vert^4
	\label{eq:GL energy with minimal coupling to EM field}
\end{equation}
Describes really two intertwined Ginzburg-Landau theories for \(\Psi\) and \(\vb{A}\) respectively.
This mean there are two length scales, the coherence length \(\xi\) governing amplitude fluctuations of \(\Psi\) and the London penetration depth \(\lambda_L\), which determines the distance magnetic fields penetrate into the superconductor.
Can get the current density from the stationary point condition of the free energy for the vector potential \(\vb{A}\):
\begin{equation}
	\fdv{f_{\mathrm{GL}}}{\vb{A}} = 0 = -\vb{j} + \frac{1}{\mu_0} \nabla \times \vb{B}
\end{equation}
with the supercurrent density
\begin{equation}
	\vb{j} = -\iu \frac{e\hbar}{m^*} \left(\Psi^* \adif{\Psi} - \Psi \adif{\Psi^*} - \frac{4 e^2}{m^*} \vert \Psi \vert^2 \vb{A} \right) \;.
\end{equation}
Introducing the OP with phase \(\Psi = \vert \Psi \vert e^{\iu \phi}\):
\begin{equation}
	\vb{j} = 
\end{equation}
\todo{Explanation supercurrent: introduce complex OP with phase, rewrite equation}

\todo{Explanation superfluid weight}

\subsection{Superconducting Length Scales}

One of the challenges in achieving high-temperature superconductivity is the fact, that the two intrinsic energy scales of superconductors i.e. the pairing amplitude and the phase coherence often compete.
Can be seen in the phenomenon of BCS-BEC crossover physics \cite{chenWhenSuperconductivityCrosses2024}.
This describes the fact that in many recently discovered
\todo{Description BCS-BEC crossover}

The picture of this crossover is the following: for small attractive interaction, pairs of electrons are

\todo{Competing energy scales via interaction strength: higher U gives more tightly bound pairs}

\todo{Critical surface of a superconductor}

\cite{hazraBoundsSuperconductingTransition2019}

These two energy scales are equivalently defined via the coherence length \(\xi_0\) and the London penetration depth introduced in \cref{sub:Landau and Ginzburg-Landau Theory}, so having access to these scales from microscopic models and ab initio approaches is very important in the search for high \(T_C\)-superconductors.

\citeauthor{wittBypassingLatticeBCS2024} introduced a framework for doing this \cite{wittBypassingLatticeBCS2024}.
As already discussed in the context of \cref{eq:breakdown of OP with phase fluctuations}, strong phase fluctuations destroy superconducting order.

\todo{Better introduction}

\todo{work over paragraph}

In most materials: Cooper pairs do not carry finite center-of-mass momentum.
In presence of e.g.\ external fields or magnetism: SC states with FMP might arise \cite{chenFiniteMomentumCooper2018, wanOrbitalFuldeFerrell2023, yuanSupercurrentDiodeEffect2022}

\todo{Write better: connection of the section above and this theory}

Procedure: enforce FMP states via constraints on pair-center-of-mass momentum \(\vb{q}\), access characteristic lenght scales \(\xi_0, \lambda_L\) through analysis of the momentum and temperature-dependent OP\@.
FF-type pairing with Cooper pairs carrying finite momentum:
\begin{equation}
	\Psi_{\vb{q}} (\vb{r}) = \vert \Psi_{\vb{q}} \vert e^{\iu \vb{q} \vb{r}}
\end{equation}
Then the free energy density \cref{eq:free energy ginzburg-landau theory} is
\begin{equation}
	f_{GL} [\Psi_{\vb{q}}] = r \vert \Psi_{\vb{q}} \vert^2 + \frac{u}{2} \vert \Psi_{\vb{q}} \vert^4 + \frac{\hbar^2 q^2}{2 m^*} \vert \Psi_{\vb{q}} \vert^2
\end{equation}
Stationary point of the system:
\begin{equation}
	\fdv{f_{GL}}{\Psi_{\vb{q}}^*} = 2 \Psi_{\vb{q}} \left[r (1 - \xi^2 q^2) + u \vert \Psi_{\vb{q}} \vert^2\right] = 0
\end{equation}
which results in the \(\vb{q}\)-dependence of the OP
\begin{equation}
	\vert \Psi_{\vb{q}} \vert^2 = \vert \Psi_{0} \vert^2 \left(1 - \xi(T)^2 q^2\right)
\end{equation}
For some value, SC order breaks down, \(\psi_{\vb{q}_c} = 0\), because the kinetic energy from phase modulation exceeds the gain in energy from pairing.
In GL theory: \(q_c = \xi(T)^{-1}\).
The temperature dependence of the OP and extracted \(\xi(T)\) gives access to the coherence length via \cref{eq:correlation length GL theory}
\begin{equation}
	\xi(T) = \xi_0 \left(1 - \frac{T}{T_C}\right)^{-\frac{1}{2}}
\end{equation}
The Cooper pair

\todo{Depairing current from FMP}

The momentum of the Cooper pairs entails a charge supercurrent \(\vb{j}_{\vb{q}}\).
For small \(q\)

\begin{figure}[t]
	\centering
	\begin{subfigure}[b]{0.5\textwidth}
		\caption{\hfill\null}\label{sfig:Ginzburg Landau OP vs q}
		\centering
		%% Creator: Matplotlib, PGF backend
%%
%% To include the figure in your LaTeX document, write
%%   \input{<filename>.pgf}
%%
%% Make sure the required packages are loaded in your preamble
%%   \usepackage{pgf}
%%
%% Also ensure that all the required font packages are loaded; for instance,
%% the lmodern package is sometimes necessary when using math font.
%%   \usepackage{lmodern}
%%
%% Figures using additional raster images can only be included by \input if
%% they are in the same directory as the main LaTeX file. For loading figures
%% from other directories you can use the `import` package
%%   \usepackage{import}
%%
%% and then include the figures with
%%   \import{<path to file>}{<filename>.pgf}
%%
%% Matplotlib used the following preamble
%%   \def\mathdefault#1{#1}
%%   \everymath=\expandafter{\the\everymath\displaystyle}
%%   \IfFileExists{scrextend.sty}{
%%     \usepackage[fontsize=11.000000pt]{scrextend}
%%   }{
%%     \renewcommand{\normalsize}{\fontsize{11.000000}{13.200000}\selectfont}
%%     \normalsize
%%   }
%%   \usepackage{fontspec}\usepackage{unicode-math}\setmathfont{texgyrepagella-math.otf}\setmainfont{texgyrepagella-math}\usepackage{nicefrac}
%%   \makeatletter\@ifpackageloaded{underscore}{}{\usepackage[strings]{underscore}}\makeatother
%%
\begingroup%
\makeatletter%
\begin{pgfpicture}%
\pgfpathrectangle{\pgfpointorigin}{\pgfqpoint{2.300000in}{1.150000in}}%
\pgfusepath{use as bounding box, clip}%
\begin{pgfscope}%
\pgfsetbuttcap%
\pgfsetmiterjoin%
\definecolor{currentfill}{rgb}{1.000000,1.000000,1.000000}%
\pgfsetfillcolor{currentfill}%
\pgfsetlinewidth{0.000000pt}%
\definecolor{currentstroke}{rgb}{1.000000,1.000000,1.000000}%
\pgfsetstrokecolor{currentstroke}%
\pgfsetdash{}{0pt}%
\pgfpathmoveto{\pgfqpoint{0.000000in}{0.000000in}}%
\pgfpathlineto{\pgfqpoint{2.300000in}{0.000000in}}%
\pgfpathlineto{\pgfqpoint{2.300000in}{1.150000in}}%
\pgfpathlineto{\pgfqpoint{0.000000in}{1.150000in}}%
\pgfpathlineto{\pgfqpoint{0.000000in}{0.000000in}}%
\pgfpathclose%
\pgfusepath{fill}%
\end{pgfscope}%
\begin{pgfscope}%
\pgfsetbuttcap%
\pgfsetmiterjoin%
\definecolor{currentfill}{rgb}{1.000000,1.000000,1.000000}%
\pgfsetfillcolor{currentfill}%
\pgfsetlinewidth{0.000000pt}%
\definecolor{currentstroke}{rgb}{0.000000,0.000000,0.000000}%
\pgfsetstrokecolor{currentstroke}%
\pgfsetstrokeopacity{0.000000}%
\pgfsetdash{}{0pt}%
\pgfpathmoveto{\pgfqpoint{0.388607in}{0.090281in}}%
\pgfpathlineto{\pgfqpoint{2.216663in}{0.090281in}}%
\pgfpathlineto{\pgfqpoint{2.216663in}{1.066663in}}%
\pgfpathlineto{\pgfqpoint{0.388607in}{1.066663in}}%
\pgfpathlineto{\pgfqpoint{0.388607in}{0.090281in}}%
\pgfpathclose%
\pgfusepath{fill}%
\end{pgfscope}%
\begin{pgfscope}%
\pgfsetbuttcap%
\pgfsetroundjoin%
\definecolor{currentfill}{rgb}{0.000000,0.000000,0.000000}%
\pgfsetfillcolor{currentfill}%
\pgfsetlinewidth{0.803000pt}%
\definecolor{currentstroke}{rgb}{0.000000,0.000000,0.000000}%
\pgfsetstrokecolor{currentstroke}%
\pgfsetdash{}{0pt}%
\pgfsys@defobject{currentmarker}{\pgfqpoint{0.000000in}{-0.048611in}}{\pgfqpoint{0.000000in}{0.000000in}}{%
\pgfpathmoveto{\pgfqpoint{0.000000in}{0.000000in}}%
\pgfpathlineto{\pgfqpoint{0.000000in}{-0.048611in}}%
\pgfusepath{stroke,fill}%
}%
\begin{pgfscope}%
\pgfsys@transformshift{2.129613in}{0.090281in}%
\pgfsys@useobject{currentmarker}{}%
\end{pgfscope}%
\end{pgfscope}%
\begin{pgfscope}%
\definecolor{textcolor}{rgb}{0.000000,0.000000,0.000000}%
\pgfsetstrokecolor{textcolor}%
\pgfsetfillcolor{textcolor}%
\pgftext[x=2.253224in,y=0.256266in,,top]{\color{textcolor}{\sffamily\fontsize{11.000000}{13.200000}\selectfont\catcode`\^=\active\def^{\ifmmode\sp\else\^{}\fi}\catcode`\%=\active\def%{\%}$q$}}%
\end{pgfscope}%
\begin{pgfscope}%
\pgfsetbuttcap%
\pgfsetroundjoin%
\definecolor{currentfill}{rgb}{0.000000,0.000000,0.000000}%
\pgfsetfillcolor{currentfill}%
\pgfsetlinewidth{0.803000pt}%
\definecolor{currentstroke}{rgb}{0.000000,0.000000,0.000000}%
\pgfsetstrokecolor{currentstroke}%
\pgfsetdash{}{0pt}%
\pgfsys@defobject{currentmarker}{\pgfqpoint{-0.048611in}{0.000000in}}{\pgfqpoint{-0.000000in}{0.000000in}}{%
\pgfpathmoveto{\pgfqpoint{-0.000000in}{0.000000in}}%
\pgfpathlineto{\pgfqpoint{-0.048611in}{0.000000in}}%
\pgfusepath{stroke,fill}%
}%
\begin{pgfscope}%
\pgfsys@transformshift{0.388607in}{0.813527in}%
\pgfsys@useobject{currentmarker}{}%
\end{pgfscope}%
\end{pgfscope}%
\begin{pgfscope}%
\definecolor{textcolor}{rgb}{0.000000,0.000000,0.000000}%
\pgfsetstrokecolor{textcolor}%
\pgfsetfillcolor{textcolor}%
\pgftext[x=0.041670in, y=0.760666in, left, base]{\color{textcolor}{\sffamily\fontsize{11.000000}{13.200000}\selectfont\catcode`\^=\active\def^{\ifmmode\sp\else\^{}\fi}\catcode`\%=\active\def%{\%}$\vert \Psi_0 \vert$}}%
\end{pgfscope}%
\begin{pgfscope}%
\definecolor{textcolor}{rgb}{0.000000,0.000000,0.000000}%
\pgfsetstrokecolor{textcolor}%
\pgfsetfillcolor{textcolor}%
\pgftext[x=0.580553in,y=0.959261in,,bottom]{\color{textcolor}{\sffamily\fontsize{11.000000}{13.200000}\selectfont\catcode`\^=\active\def^{\ifmmode\sp\else\^{}\fi}\catcode`\%=\active\def%{\%}$\vert \Psi_q \vert$}}%
\end{pgfscope}%
\begin{pgfscope}%
\pgfpathrectangle{\pgfqpoint{0.388607in}{0.090281in}}{\pgfqpoint{1.828056in}{0.976382in}}%
\pgfusepath{clip}%
\pgfsetrectcap%
\pgfsetroundjoin%
\pgfsetlinewidth{1.505625pt}%
\definecolor{currentstroke}{rgb}{0.247059,0.564706,0.854902}%
\pgfsetstrokecolor{currentstroke}%
\pgfsetdash{}{0pt}%
\pgfpathmoveto{\pgfqpoint{0.388607in}{0.813527in}}%
\pgfpathlineto{\pgfqpoint{0.406193in}{0.813490in}}%
\pgfpathlineto{\pgfqpoint{0.423779in}{0.813380in}}%
\pgfpathlineto{\pgfqpoint{0.441365in}{0.813195in}}%
\pgfpathlineto{\pgfqpoint{0.458951in}{0.812937in}}%
\pgfpathlineto{\pgfqpoint{0.476537in}{0.812604in}}%
\pgfpathlineto{\pgfqpoint{0.494123in}{0.812198in}}%
\pgfpathlineto{\pgfqpoint{0.511709in}{0.811717in}}%
\pgfpathlineto{\pgfqpoint{0.529295in}{0.811162in}}%
\pgfpathlineto{\pgfqpoint{0.546880in}{0.810532in}}%
\pgfpathlineto{\pgfqpoint{0.564466in}{0.809828in}}%
\pgfpathlineto{\pgfqpoint{0.582052in}{0.809049in}}%
\pgfpathlineto{\pgfqpoint{0.599638in}{0.808194in}}%
\pgfpathlineto{\pgfqpoint{0.617224in}{0.807265in}}%
\pgfpathlineto{\pgfqpoint{0.634810in}{0.806259in}}%
\pgfpathlineto{\pgfqpoint{0.652396in}{0.805177in}}%
\pgfpathlineto{\pgfqpoint{0.669982in}{0.804019in}}%
\pgfpathlineto{\pgfqpoint{0.687568in}{0.802784in}}%
\pgfpathlineto{\pgfqpoint{0.705154in}{0.801472in}}%
\pgfpathlineto{\pgfqpoint{0.722740in}{0.800083in}}%
\pgfpathlineto{\pgfqpoint{0.740326in}{0.798615in}}%
\pgfpathlineto{\pgfqpoint{0.757911in}{0.797069in}}%
\pgfpathlineto{\pgfqpoint{0.775497in}{0.795443in}}%
\pgfpathlineto{\pgfqpoint{0.793083in}{0.793738in}}%
\pgfpathlineto{\pgfqpoint{0.810669in}{0.791953in}}%
\pgfpathlineto{\pgfqpoint{0.828255in}{0.790087in}}%
\pgfpathlineto{\pgfqpoint{0.845841in}{0.788140in}}%
\pgfpathlineto{\pgfqpoint{0.863427in}{0.786110in}}%
\pgfpathlineto{\pgfqpoint{0.881013in}{0.783997in}}%
\pgfpathlineto{\pgfqpoint{0.898599in}{0.781801in}}%
\pgfpathlineto{\pgfqpoint{0.916185in}{0.779521in}}%
\pgfpathlineto{\pgfqpoint{0.933771in}{0.777155in}}%
\pgfpathlineto{\pgfqpoint{0.951357in}{0.774703in}}%
\pgfpathlineto{\pgfqpoint{0.968942in}{0.772164in}}%
\pgfpathlineto{\pgfqpoint{0.986528in}{0.769537in}}%
\pgfpathlineto{\pgfqpoint{1.004114in}{0.766821in}}%
\pgfpathlineto{\pgfqpoint{1.021700in}{0.764014in}}%
\pgfpathlineto{\pgfqpoint{1.039286in}{0.761117in}}%
\pgfpathlineto{\pgfqpoint{1.056872in}{0.758127in}}%
\pgfpathlineto{\pgfqpoint{1.074458in}{0.755043in}}%
\pgfpathlineto{\pgfqpoint{1.092044in}{0.751864in}}%
\pgfpathlineto{\pgfqpoint{1.109630in}{0.748589in}}%
\pgfpathlineto{\pgfqpoint{1.127216in}{0.745216in}}%
\pgfpathlineto{\pgfqpoint{1.144802in}{0.741743in}}%
\pgfpathlineto{\pgfqpoint{1.162388in}{0.738170in}}%
\pgfpathlineto{\pgfqpoint{1.179974in}{0.734493in}}%
\pgfpathlineto{\pgfqpoint{1.197559in}{0.730713in}}%
\pgfpathlineto{\pgfqpoint{1.215145in}{0.726826in}}%
\pgfpathlineto{\pgfqpoint{1.232731in}{0.722831in}}%
\pgfpathlineto{\pgfqpoint{1.250317in}{0.718725in}}%
\pgfpathlineto{\pgfqpoint{1.267903in}{0.714507in}}%
\pgfpathlineto{\pgfqpoint{1.285489in}{0.710175in}}%
\pgfpathlineto{\pgfqpoint{1.303075in}{0.705725in}}%
\pgfpathlineto{\pgfqpoint{1.320661in}{0.701155in}}%
\pgfpathlineto{\pgfqpoint{1.338247in}{0.696463in}}%
\pgfpathlineto{\pgfqpoint{1.355833in}{0.691645in}}%
\pgfpathlineto{\pgfqpoint{1.373419in}{0.686699in}}%
\pgfpathlineto{\pgfqpoint{1.391005in}{0.681622in}}%
\pgfpathlineto{\pgfqpoint{1.408590in}{0.676409in}}%
\pgfpathlineto{\pgfqpoint{1.426176in}{0.671058in}}%
\pgfpathlineto{\pgfqpoint{1.443762in}{0.665564in}}%
\pgfpathlineto{\pgfqpoint{1.461348in}{0.659924in}}%
\pgfpathlineto{\pgfqpoint{1.478934in}{0.654133in}}%
\pgfpathlineto{\pgfqpoint{1.496520in}{0.648185in}}%
\pgfpathlineto{\pgfqpoint{1.514106in}{0.642077in}}%
\pgfpathlineto{\pgfqpoint{1.531692in}{0.635803in}}%
\pgfpathlineto{\pgfqpoint{1.549278in}{0.629357in}}%
\pgfpathlineto{\pgfqpoint{1.566864in}{0.622732in}}%
\pgfpathlineto{\pgfqpoint{1.584450in}{0.615923in}}%
\pgfpathlineto{\pgfqpoint{1.602036in}{0.608921in}}%
\pgfpathlineto{\pgfqpoint{1.619621in}{0.601719in}}%
\pgfpathlineto{\pgfqpoint{1.637207in}{0.594309in}}%
\pgfpathlineto{\pgfqpoint{1.654793in}{0.586680in}}%
\pgfpathlineto{\pgfqpoint{1.672379in}{0.578823in}}%
\pgfpathlineto{\pgfqpoint{1.689965in}{0.570726in}}%
\pgfpathlineto{\pgfqpoint{1.707551in}{0.562378in}}%
\pgfpathlineto{\pgfqpoint{1.725137in}{0.553764in}}%
\pgfpathlineto{\pgfqpoint{1.742723in}{0.544870in}}%
\pgfpathlineto{\pgfqpoint{1.760309in}{0.535678in}}%
\pgfpathlineto{\pgfqpoint{1.777895in}{0.526170in}}%
\pgfpathlineto{\pgfqpoint{1.795481in}{0.516325in}}%
\pgfpathlineto{\pgfqpoint{1.813067in}{0.506118in}}%
\pgfpathlineto{\pgfqpoint{1.830652in}{0.495523in}}%
\pgfpathlineto{\pgfqpoint{1.848238in}{0.484508in}}%
\pgfpathlineto{\pgfqpoint{1.865824in}{0.473037in}}%
\pgfpathlineto{\pgfqpoint{1.883410in}{0.461068in}}%
\pgfpathlineto{\pgfqpoint{1.900996in}{0.448549in}}%
\pgfpathlineto{\pgfqpoint{1.918582in}{0.435423in}}%
\pgfpathlineto{\pgfqpoint{1.936168in}{0.421617in}}%
\pgfpathlineto{\pgfqpoint{1.953754in}{0.407041in}}%
\pgfpathlineto{\pgfqpoint{1.971340in}{0.391584in}}%
\pgfpathlineto{\pgfqpoint{1.988926in}{0.375103in}}%
\pgfpathlineto{\pgfqpoint{2.006512in}{0.357407in}}%
\pgfpathlineto{\pgfqpoint{2.024098in}{0.338239in}}%
\pgfpathlineto{\pgfqpoint{2.041683in}{0.317223in}}%
\pgfpathlineto{\pgfqpoint{2.059269in}{0.293789in}}%
\pgfpathlineto{\pgfqpoint{2.076855in}{0.266978in}}%
\pgfpathlineto{\pgfqpoint{2.094441in}{0.234923in}}%
\pgfpathlineto{\pgfqpoint{2.112027in}{0.192819in}}%
\pgfpathlineto{\pgfqpoint{2.129613in}{0.090281in}}%
\pgfusepath{stroke}%
\end{pgfscope}%
\begin{pgfscope}%
\pgfsetbuttcap%
\pgfsetmiterjoin%
\definecolor{currentfill}{rgb}{0.000000,0.000000,0.000000}%
\pgfsetfillcolor{currentfill}%
\pgfsetlinewidth{1.003750pt}%
\definecolor{currentstroke}{rgb}{0.000000,0.000000,0.000000}%
\pgfsetstrokecolor{currentstroke}%
\pgfsetdash{}{0pt}%
\pgfsys@defobject{currentmarker}{\pgfqpoint{-0.041667in}{-0.041667in}}{\pgfqpoint{0.041667in}{0.041667in}}{%
\pgfpathmoveto{\pgfqpoint{0.041667in}{-0.000000in}}%
\pgfpathlineto{\pgfqpoint{-0.041667in}{0.041667in}}%
\pgfpathlineto{\pgfqpoint{-0.041667in}{-0.041667in}}%
\pgfpathlineto{\pgfqpoint{0.041667in}{-0.000000in}}%
\pgfpathclose%
\pgfusepath{stroke,fill}%
}%
\begin{pgfscope}%
\pgfsys@transformshift{2.216663in}{0.090281in}%
\pgfsys@useobject{currentmarker}{}%
\end{pgfscope}%
\end{pgfscope}%
\begin{pgfscope}%
\pgfsetbuttcap%
\pgfsetmiterjoin%
\definecolor{currentfill}{rgb}{0.000000,0.000000,0.000000}%
\pgfsetfillcolor{currentfill}%
\pgfsetlinewidth{1.003750pt}%
\definecolor{currentstroke}{rgb}{0.000000,0.000000,0.000000}%
\pgfsetstrokecolor{currentstroke}%
\pgfsetdash{}{0pt}%
\pgfsys@defobject{currentmarker}{\pgfqpoint{-0.041667in}{-0.041667in}}{\pgfqpoint{0.041667in}{0.041667in}}{%
\pgfpathmoveto{\pgfqpoint{0.000000in}{0.041667in}}%
\pgfpathlineto{\pgfqpoint{-0.041667in}{-0.041667in}}%
\pgfpathlineto{\pgfqpoint{0.041667in}{-0.041667in}}%
\pgfpathlineto{\pgfqpoint{0.000000in}{0.041667in}}%
\pgfpathclose%
\pgfusepath{stroke,fill}%
}%
\begin{pgfscope}%
\pgfsys@transformshift{0.388607in}{1.066663in}%
\pgfsys@useobject{currentmarker}{}%
\end{pgfscope}%
\end{pgfscope}%
\begin{pgfscope}%
\pgfsetrectcap%
\pgfsetmiterjoin%
\pgfsetlinewidth{0.803000pt}%
\definecolor{currentstroke}{rgb}{0.000000,0.000000,0.000000}%
\pgfsetstrokecolor{currentstroke}%
\pgfsetdash{}{0pt}%
\pgfpathmoveto{\pgfqpoint{0.388607in}{0.090281in}}%
\pgfpathlineto{\pgfqpoint{0.388607in}{1.066663in}}%
\pgfusepath{stroke}%
\end{pgfscope}%
\begin{pgfscope}%
\pgfsetrectcap%
\pgfsetmiterjoin%
\pgfsetlinewidth{0.803000pt}%
\definecolor{currentstroke}{rgb}{0.000000,0.000000,0.000000}%
\pgfsetstrokecolor{currentstroke}%
\pgfsetdash{}{0pt}%
\pgfpathmoveto{\pgfqpoint{0.388607in}{0.090281in}}%
\pgfpathlineto{\pgfqpoint{2.216663in}{0.090281in}}%
\pgfusepath{stroke}%
\end{pgfscope}%
\end{pgfpicture}%
\makeatother%
\endgroup%

	\end{subfigure}%
	\hfill
	\begin{subfigure}[b]{0.5\textwidth}
		\centering
		\caption{\hfill\null}\label{sfig:Ginzburg Landau current vs q}
		%% Creator: Matplotlib, PGF backend
%%
%% To include the figure in your LaTeX document, write
%%   \input{<filename>.pgf}
%%
%% Make sure the required packages are loaded in your preamble
%%   \usepackage{pgf}
%%
%% Also ensure that all the required font packages are loaded; for instance,
%% the lmodern package is sometimes necessary when using math font.
%%   \usepackage{lmodern}
%%
%% Figures using additional raster images can only be included by \input if
%% they are in the same directory as the main LaTeX file. For loading figures
%% from other directories you can use the `import` package
%%   \usepackage{import}
%%
%% and then include the figures with
%%   \import{<path to file>}{<filename>.pgf}
%%
%% Matplotlib used the following preamble
%%   \def\mathdefault#1{#1}
%%   \everymath=\expandafter{\the\everymath\displaystyle}
%%   \IfFileExists{scrextend.sty}{
%%     \usepackage[fontsize=11.000000pt]{scrextend}
%%   }{
%%     \renewcommand{\normalsize}{\fontsize{11.000000}{13.200000}\selectfont}
%%     \normalsize
%%   }
%%   \usepackage{fontspec}\usepackage{unicode-math}\setmathfont{texgyrepagella-math.otf}\setmainfont{texgyrepagella-math}\usepackage{nicefrac}
%%   \makeatletter\@ifpackageloaded{underscore}{}{\usepackage[strings]{underscore}}\makeatother
%%
\begingroup%
\makeatletter%
\begin{pgfpicture}%
\pgfpathrectangle{\pgfpointorigin}{\pgfqpoint{2.300000in}{1.150000in}}%
\pgfusepath{use as bounding box, clip}%
\begin{pgfscope}%
\pgfsetbuttcap%
\pgfsetmiterjoin%
\definecolor{currentfill}{rgb}{1.000000,1.000000,1.000000}%
\pgfsetfillcolor{currentfill}%
\pgfsetlinewidth{0.000000pt}%
\definecolor{currentstroke}{rgb}{1.000000,1.000000,1.000000}%
\pgfsetstrokecolor{currentstroke}%
\pgfsetdash{}{0pt}%
\pgfpathmoveto{\pgfqpoint{0.000000in}{0.000000in}}%
\pgfpathlineto{\pgfqpoint{2.300000in}{0.000000in}}%
\pgfpathlineto{\pgfqpoint{2.300000in}{1.150000in}}%
\pgfpathlineto{\pgfqpoint{0.000000in}{1.150000in}}%
\pgfpathlineto{\pgfqpoint{0.000000in}{0.000000in}}%
\pgfpathclose%
\pgfusepath{fill}%
\end{pgfscope}%
\begin{pgfscope}%
\pgfsetbuttcap%
\pgfsetmiterjoin%
\definecolor{currentfill}{rgb}{1.000000,1.000000,1.000000}%
\pgfsetfillcolor{currentfill}%
\pgfsetlinewidth{0.000000pt}%
\definecolor{currentstroke}{rgb}{0.000000,0.000000,0.000000}%
\pgfsetstrokecolor{currentstroke}%
\pgfsetstrokeopacity{0.000000}%
\pgfsetdash{}{0pt}%
\pgfpathmoveto{\pgfqpoint{0.317092in}{0.090281in}}%
\pgfpathlineto{\pgfqpoint{2.215339in}{0.090281in}}%
\pgfpathlineto{\pgfqpoint{2.215339in}{1.066663in}}%
\pgfpathlineto{\pgfqpoint{0.317092in}{1.066663in}}%
\pgfpathlineto{\pgfqpoint{0.317092in}{0.090281in}}%
\pgfpathclose%
\pgfusepath{fill}%
\end{pgfscope}%
\begin{pgfscope}%
\pgfsetbuttcap%
\pgfsetroundjoin%
\definecolor{currentfill}{rgb}{0.000000,0.000000,0.000000}%
\pgfsetfillcolor{currentfill}%
\pgfsetlinewidth{0.803000pt}%
\definecolor{currentstroke}{rgb}{0.000000,0.000000,0.000000}%
\pgfsetstrokecolor{currentstroke}%
\pgfsetdash{}{0pt}%
\pgfsys@defobject{currentmarker}{\pgfqpoint{0.000000in}{-0.048611in}}{\pgfqpoint{0.000000in}{0.000000in}}{%
\pgfpathmoveto{\pgfqpoint{0.000000in}{0.000000in}}%
\pgfpathlineto{\pgfqpoint{0.000000in}{-0.048611in}}%
\pgfusepath{stroke,fill}%
}%
\begin{pgfscope}%
\pgfsys@transformshift{1.360902in}{0.090281in}%
\pgfsys@useobject{currentmarker}{}%
\end{pgfscope}%
\end{pgfscope}%
\begin{pgfscope}%
\pgfsetbuttcap%
\pgfsetroundjoin%
\definecolor{currentfill}{rgb}{0.000000,0.000000,0.000000}%
\pgfsetfillcolor{currentfill}%
\pgfsetlinewidth{0.803000pt}%
\definecolor{currentstroke}{rgb}{0.000000,0.000000,0.000000}%
\pgfsetstrokecolor{currentstroke}%
\pgfsetdash{}{0pt}%
\pgfsys@defobject{currentmarker}{\pgfqpoint{0.000000in}{-0.048611in}}{\pgfqpoint{0.000000in}{0.000000in}}{%
\pgfpathmoveto{\pgfqpoint{0.000000in}{0.000000in}}%
\pgfpathlineto{\pgfqpoint{0.000000in}{-0.048611in}}%
\pgfusepath{stroke,fill}%
}%
\begin{pgfscope}%
\pgfsys@transformshift{2.124946in}{0.090281in}%
\pgfsys@useobject{currentmarker}{}%
\end{pgfscope}%
\end{pgfscope}%
\begin{pgfscope}%
\definecolor{textcolor}{rgb}{0.000000,0.000000,0.000000}%
\pgfsetstrokecolor{textcolor}%
\pgfsetfillcolor{textcolor}%
\pgftext[x=2.253304in,y=0.256266in,,top]{\color{textcolor}{\rmfamily\fontsize{11.000000}{13.200000}\selectfont\catcode`\^=\active\def^{\ifmmode\sp\else\^{}\fi}\catcode`\%=\active\def%{\%}$q$}}%
\end{pgfscope}%
\begin{pgfscope}%
\pgfsetbuttcap%
\pgfsetroundjoin%
\definecolor{currentfill}{rgb}{0.000000,0.000000,0.000000}%
\pgfsetfillcolor{currentfill}%
\pgfsetlinewidth{0.803000pt}%
\definecolor{currentstroke}{rgb}{0.000000,0.000000,0.000000}%
\pgfsetstrokecolor{currentstroke}%
\pgfsetdash{}{0pt}%
\pgfsys@defobject{currentmarker}{\pgfqpoint{-0.048611in}{0.000000in}}{\pgfqpoint{-0.000000in}{0.000000in}}{%
\pgfpathmoveto{\pgfqpoint{-0.000000in}{0.000000in}}%
\pgfpathlineto{\pgfqpoint{-0.048611in}{0.000000in}}%
\pgfusepath{stroke,fill}%
}%
\begin{pgfscope}%
\pgfsys@transformshift{0.317092in}{0.903933in}%
\pgfsys@useobject{currentmarker}{}%
\end{pgfscope}%
\end{pgfscope}%
\begin{pgfscope}%
\definecolor{textcolor}{rgb}{0.000000,0.000000,0.000000}%
\pgfsetstrokecolor{textcolor}%
\pgfsetfillcolor{textcolor}%
\pgftext[x=0.041670in, y=0.852752in, left, base]{\color{textcolor}{\rmfamily\fontsize{11.000000}{13.200000}\selectfont\catcode`\^=\active\def^{\ifmmode\sp\else\^{}\fi}\catcode`\%=\active\def%{\%}$j_{\mathrm{dp}}$}}%
\end{pgfscope}%
\begin{pgfscope}%
\definecolor{textcolor}{rgb}{0.000000,0.000000,0.000000}%
\pgfsetstrokecolor{textcolor}%
\pgfsetfillcolor{textcolor}%
\pgftext[x=0.468952in,y=0.959261in,,bottom]{\color{textcolor}{\rmfamily\fontsize{11.000000}{13.200000}\selectfont\catcode`\^=\active\def^{\ifmmode\sp\else\^{}\fi}\catcode`\%=\active\def%{\%}$j_q$}}%
\end{pgfscope}%
\begin{pgfscope}%
\pgfpathrectangle{\pgfqpoint{0.317092in}{0.090281in}}{\pgfqpoint{1.898247in}{0.976382in}}%
\pgfusepath{clip}%
\pgfsetrectcap%
\pgfsetroundjoin%
\pgfsetlinewidth{1.505625pt}%
\definecolor{currentstroke}{rgb}{0.247059,0.564706,0.854902}%
\pgfsetstrokecolor{currentstroke}%
\pgfsetdash{}{0pt}%
\pgfpathmoveto{\pgfqpoint{0.317092in}{0.090281in}}%
\pgfpathlineto{\pgfqpoint{0.335353in}{0.111632in}}%
\pgfpathlineto{\pgfqpoint{0.353614in}{0.132969in}}%
\pgfpathlineto{\pgfqpoint{0.371875in}{0.154281in}}%
\pgfpathlineto{\pgfqpoint{0.390136in}{0.175553in}}%
\pgfpathlineto{\pgfqpoint{0.408398in}{0.196773in}}%
\pgfpathlineto{\pgfqpoint{0.426659in}{0.217927in}}%
\pgfpathlineto{\pgfqpoint{0.444920in}{0.239004in}}%
\pgfpathlineto{\pgfqpoint{0.463181in}{0.259988in}}%
\pgfpathlineto{\pgfqpoint{0.481442in}{0.280868in}}%
\pgfpathlineto{\pgfqpoint{0.499703in}{0.301631in}}%
\pgfpathlineto{\pgfqpoint{0.517964in}{0.322262in}}%
\pgfpathlineto{\pgfqpoint{0.536226in}{0.342750in}}%
\pgfpathlineto{\pgfqpoint{0.554487in}{0.363081in}}%
\pgfpathlineto{\pgfqpoint{0.572748in}{0.383242in}}%
\pgfpathlineto{\pgfqpoint{0.591009in}{0.403221in}}%
\pgfpathlineto{\pgfqpoint{0.609270in}{0.423003in}}%
\pgfpathlineto{\pgfqpoint{0.627531in}{0.442575in}}%
\pgfpathlineto{\pgfqpoint{0.645793in}{0.461926in}}%
\pgfpathlineto{\pgfqpoint{0.664054in}{0.481041in}}%
\pgfpathlineto{\pgfqpoint{0.682315in}{0.499908in}}%
\pgfpathlineto{\pgfqpoint{0.700576in}{0.518514in}}%
\pgfpathlineto{\pgfqpoint{0.718837in}{0.536845in}}%
\pgfpathlineto{\pgfqpoint{0.737098in}{0.554889in}}%
\pgfpathlineto{\pgfqpoint{0.755359in}{0.572631in}}%
\pgfpathlineto{\pgfqpoint{0.773621in}{0.590060in}}%
\pgfpathlineto{\pgfqpoint{0.791882in}{0.607163in}}%
\pgfpathlineto{\pgfqpoint{0.810143in}{0.623925in}}%
\pgfpathlineto{\pgfqpoint{0.828404in}{0.640335in}}%
\pgfpathlineto{\pgfqpoint{0.846665in}{0.656378in}}%
\pgfpathlineto{\pgfqpoint{0.864926in}{0.672043in}}%
\pgfpathlineto{\pgfqpoint{0.883188in}{0.687315in}}%
\pgfpathlineto{\pgfqpoint{0.901449in}{0.702182in}}%
\pgfpathlineto{\pgfqpoint{0.919710in}{0.716631in}}%
\pgfpathlineto{\pgfqpoint{0.937971in}{0.730648in}}%
\pgfpathlineto{\pgfqpoint{0.956232in}{0.744221in}}%
\pgfpathlineto{\pgfqpoint{0.974493in}{0.757336in}}%
\pgfpathlineto{\pgfqpoint{0.992754in}{0.769981in}}%
\pgfpathlineto{\pgfqpoint{1.011016in}{0.782142in}}%
\pgfpathlineto{\pgfqpoint{1.029277in}{0.793807in}}%
\pgfpathlineto{\pgfqpoint{1.047538in}{0.804961in}}%
\pgfpathlineto{\pgfqpoint{1.065799in}{0.815593in}}%
\pgfpathlineto{\pgfqpoint{1.084060in}{0.825689in}}%
\pgfpathlineto{\pgfqpoint{1.102321in}{0.835236in}}%
\pgfpathlineto{\pgfqpoint{1.120582in}{0.844220in}}%
\pgfpathlineto{\pgfqpoint{1.138844in}{0.852630in}}%
\pgfpathlineto{\pgfqpoint{1.157105in}{0.860451in}}%
\pgfpathlineto{\pgfqpoint{1.175366in}{0.867671in}}%
\pgfpathlineto{\pgfqpoint{1.193627in}{0.874277in}}%
\pgfpathlineto{\pgfqpoint{1.211888in}{0.880255in}}%
\pgfpathlineto{\pgfqpoint{1.230149in}{0.885593in}}%
\pgfpathlineto{\pgfqpoint{1.248411in}{0.890277in}}%
\pgfpathlineto{\pgfqpoint{1.266672in}{0.894294in}}%
\pgfpathlineto{\pgfqpoint{1.284933in}{0.897632in}}%
\pgfpathlineto{\pgfqpoint{1.303194in}{0.900277in}}%
\pgfpathlineto{\pgfqpoint{1.321455in}{0.902216in}}%
\pgfpathlineto{\pgfqpoint{1.339716in}{0.903436in}}%
\pgfpathlineto{\pgfqpoint{1.357977in}{0.903924in}}%
\pgfpathlineto{\pgfqpoint{1.376239in}{0.903667in}}%
\pgfpathlineto{\pgfqpoint{1.394500in}{0.902651in}}%
\pgfpathlineto{\pgfqpoint{1.412761in}{0.900865in}}%
\pgfpathlineto{\pgfqpoint{1.431022in}{0.898294in}}%
\pgfpathlineto{\pgfqpoint{1.449283in}{0.894926in}}%
\pgfpathlineto{\pgfqpoint{1.467544in}{0.890747in}}%
\pgfpathlineto{\pgfqpoint{1.485806in}{0.885745in}}%
\pgfpathlineto{\pgfqpoint{1.504067in}{0.879906in}}%
\pgfpathlineto{\pgfqpoint{1.522328in}{0.873218in}}%
\pgfpathlineto{\pgfqpoint{1.540589in}{0.865667in}}%
\pgfpathlineto{\pgfqpoint{1.558850in}{0.857240in}}%
\pgfpathlineto{\pgfqpoint{1.577111in}{0.847924in}}%
\pgfpathlineto{\pgfqpoint{1.595372in}{0.837706in}}%
\pgfpathlineto{\pgfqpoint{1.613634in}{0.826573in}}%
\pgfpathlineto{\pgfqpoint{1.631895in}{0.814512in}}%
\pgfpathlineto{\pgfqpoint{1.650156in}{0.801510in}}%
\pgfpathlineto{\pgfqpoint{1.668417in}{0.787554in}}%
\pgfpathlineto{\pgfqpoint{1.686678in}{0.772630in}}%
\pgfpathlineto{\pgfqpoint{1.704939in}{0.756726in}}%
\pgfpathlineto{\pgfqpoint{1.723201in}{0.739829in}}%
\pgfpathlineto{\pgfqpoint{1.741462in}{0.721925in}}%
\pgfpathlineto{\pgfqpoint{1.759723in}{0.703001in}}%
\pgfpathlineto{\pgfqpoint{1.777984in}{0.683045in}}%
\pgfpathlineto{\pgfqpoint{1.796245in}{0.662043in}}%
\pgfpathlineto{\pgfqpoint{1.814506in}{0.639982in}}%
\pgfpathlineto{\pgfqpoint{1.832767in}{0.616849in}}%
\pgfpathlineto{\pgfqpoint{1.851029in}{0.592631in}}%
\pgfpathlineto{\pgfqpoint{1.869290in}{0.567316in}}%
\pgfpathlineto{\pgfqpoint{1.887551in}{0.540889in}}%
\pgfpathlineto{\pgfqpoint{1.905812in}{0.513338in}}%
\pgfpathlineto{\pgfqpoint{1.924073in}{0.484649in}}%
\pgfpathlineto{\pgfqpoint{1.942334in}{0.454811in}}%
\pgfpathlineto{\pgfqpoint{1.960595in}{0.423809in}}%
\pgfpathlineto{\pgfqpoint{1.978857in}{0.391630in}}%
\pgfpathlineto{\pgfqpoint{1.997118in}{0.358262in}}%
\pgfpathlineto{\pgfqpoint{2.015379in}{0.323692in}}%
\pgfpathlineto{\pgfqpoint{2.033640in}{0.287905in}}%
\pgfpathlineto{\pgfqpoint{2.051901in}{0.250890in}}%
\pgfpathlineto{\pgfqpoint{2.070162in}{0.212633in}}%
\pgfpathlineto{\pgfqpoint{2.088424in}{0.173122in}}%
\pgfpathlineto{\pgfqpoint{2.106685in}{0.132342in}}%
\pgfpathlineto{\pgfqpoint{2.124946in}{0.090281in}}%
\pgfusepath{stroke}%
\end{pgfscope}%
\begin{pgfscope}%
\pgfsetbuttcap%
\pgfsetmiterjoin%
\definecolor{currentfill}{rgb}{0.000000,0.000000,0.000000}%
\pgfsetfillcolor{currentfill}%
\pgfsetlinewidth{1.003750pt}%
\definecolor{currentstroke}{rgb}{0.000000,0.000000,0.000000}%
\pgfsetstrokecolor{currentstroke}%
\pgfsetdash{}{0pt}%
\pgfsys@defobject{currentmarker}{\pgfqpoint{-0.041667in}{-0.041667in}}{\pgfqpoint{0.041667in}{0.041667in}}{%
\pgfpathmoveto{\pgfqpoint{0.041667in}{-0.000000in}}%
\pgfpathlineto{\pgfqpoint{-0.041667in}{0.041667in}}%
\pgfpathlineto{\pgfqpoint{-0.041667in}{-0.041667in}}%
\pgfpathlineto{\pgfqpoint{0.041667in}{-0.000000in}}%
\pgfpathclose%
\pgfusepath{stroke,fill}%
}%
\begin{pgfscope}%
\pgfsys@transformshift{2.215339in}{0.090281in}%
\pgfsys@useobject{currentmarker}{}%
\end{pgfscope}%
\end{pgfscope}%
\begin{pgfscope}%
\pgfsetbuttcap%
\pgfsetmiterjoin%
\definecolor{currentfill}{rgb}{0.000000,0.000000,0.000000}%
\pgfsetfillcolor{currentfill}%
\pgfsetlinewidth{1.003750pt}%
\definecolor{currentstroke}{rgb}{0.000000,0.000000,0.000000}%
\pgfsetstrokecolor{currentstroke}%
\pgfsetdash{}{0pt}%
\pgfsys@defobject{currentmarker}{\pgfqpoint{-0.041667in}{-0.041667in}}{\pgfqpoint{0.041667in}{0.041667in}}{%
\pgfpathmoveto{\pgfqpoint{0.000000in}{0.041667in}}%
\pgfpathlineto{\pgfqpoint{-0.041667in}{-0.041667in}}%
\pgfpathlineto{\pgfqpoint{0.041667in}{-0.041667in}}%
\pgfpathlineto{\pgfqpoint{0.000000in}{0.041667in}}%
\pgfpathclose%
\pgfusepath{stroke,fill}%
}%
\begin{pgfscope}%
\pgfsys@transformshift{0.317092in}{1.066663in}%
\pgfsys@useobject{currentmarker}{}%
\end{pgfscope}%
\end{pgfscope}%
\begin{pgfscope}%
\pgfsetrectcap%
\pgfsetmiterjoin%
\pgfsetlinewidth{0.803000pt}%
\definecolor{currentstroke}{rgb}{0.000000,0.000000,0.000000}%
\pgfsetstrokecolor{currentstroke}%
\pgfsetdash{}{0pt}%
\pgfpathmoveto{\pgfqpoint{0.317092in}{0.090281in}}%
\pgfpathlineto{\pgfqpoint{0.317092in}{1.066663in}}%
\pgfusepath{stroke}%
\end{pgfscope}%
\begin{pgfscope}%
\pgfsetrectcap%
\pgfsetmiterjoin%
\pgfsetlinewidth{0.803000pt}%
\definecolor{currentstroke}{rgb}{0.000000,0.000000,0.000000}%
\pgfsetstrokecolor{currentstroke}%
\pgfsetdash{}{0pt}%
\pgfpathmoveto{\pgfqpoint{0.317092in}{0.090281in}}%
\pgfpathlineto{\pgfqpoint{2.215339in}{0.090281in}}%
\pgfusepath{stroke}%
\end{pgfscope}%
\end{pgfpicture}%
\makeatother%
\endgroup%

	\end{subfigure}
	\caption{\subref{sfig:Ginzburg Landau OP vs q} and \subref{sfig:Ginzburg Landau current vs q}} 
	\label{fig:Ginzburg Landau OP and current vs q}
\end{figure}

The depairing current is an upper boundary for the maximal current that can flow through a material, also called the critical current \(\vb{j}_c\).
The value of \(\vb{j}_c\) is strongly dependent on the geometry of the sample \cite{bardeenCriticalFieldsCurrents1962, xuAchievingTheoreticalDepairing2010}, so \(\vb{j}_{dp}\) is not necessarily experimentally available, but it can be used to calculate the London penetration depth:
\begin{equation}
	\lambda_L
\end{equation}
\todo{Where does this formula come from?}

\todo{\(D_S\)}

\todo{What else can be done with the FMP method?}

\todo{Connection of the FMP method to linear response techniques}

\section{Bardeen-Coooper-Schrieffer Theory}\label{sec:bcs-theory}

It took nearly 50 years after the first discovery of superconductivity in mercury by Heike Kamerlingh Onnes in 1911 \cite{onnesFurtherExperimentsLiquid1991} for the first microscopic description of this phenomenon to be published in 1957 by John Bardeen, Leon Cooper and J. Robert Schrieffer \cite{bardeenTheorySuperconductivity1957}.

The \glsxtrfull{bcs} description of superconductivity is based on the fact that the Fermi sea is unstable towards development of bound pairs under arbitrarily small attraction \cite{cooperBoundElectronPairs1956}.
The origin of the attractive interaction \(V_{{\vb{k}, \vb{k}^{\prime}}}\), which Bardeen, Cooper and Schrieffer identified as a retarded electron-phonon interaction \cite{bardeenTheorySuperconductivity1957}.

There exist many textbooks tackling BCS theory from different angles, such as refs. \cite{colemanIntroductionManyBodyPhysics2015, tinkhamIntroductionSuperconductivity1996}.
This section gives an introduction to the relevant physics of \glsxtrshort{bcs} theory as originally proposed, then derives the 

\todo{Better introduction}

\subsection{BCS Hamiltonian}

\todo{Work over paragraph}

BCS-Hamiltonian:
\begin{equation}\label{eq:BCS Hamiltonian}
	H_{\text{BCS}} = \sum_{\vb{k}\sigma} \epsilon_{\vb{k}\sigma} c_{\vb{k}\sigma}^{\dagger} c_{\vb{k}\sigma} + \sum_{\vb{k}, \vb{k}^{\prime}} V_{\vb{k}, \vb{k}^{\prime}} c_{\vb{k}\uparrow}^{\dagger} c_{-\vb{k}\downarrow}^{\dagger} c_{-\vb{k}^{\prime}\downarrow} c_{\vb{k}^{\prime}\uparrow}
\end{equation}
This Hamiltonian can be solved exactly using a mean field approach, because it involves an interaction at zero momentum and thus infinite range.
Order parameter in mean field BCS theory is the pairing amplitude
\begin{equation}
	\Delta = - \frac{U}{N_{\vb{k}}} \sum_{\vb{k}} \braket{c_{-\vb{k} \downarrow} c_{\vb{k} \uparrow}} = - U \braket{c_{-\vb{r}=0 \downarrow} c_{\vb{r=0} \uparrow}} \simeq U \Psi \;.
\end{equation}
More about mean field theory in \cref{ssec:Multiband BCS Mean Field Theory}

A finite \(\Delta\) corresponds to the pairing introduced above: there is a finite expectation value for a coherent creation/annihilation of a pair of electrons with opposite momentum and spin.
A finite \(\Delta\) also introduces a band gap into the spectrum.
BCS theory brings multiple aspects together: concept of paired electrons with the pairing amplitude being the order parameter in SC, an explanation for the attractive interaction overcoming Coulomb repulsion and a model Hamiltonian that very elegantly captures the essential physics.

It is very successful in two ways: on the one hand it could quantitatively predict effects in the SCs known at the time, for example the Hebel-Slichter peak that was measured in 1957 \cite{hebelNuclearRelaxationSuperconducting1957, hebelNuclearSpinRelaxation1959} and the band gap measured by Giaever in 1960 \cite{giaeverStudySuperconductorsElectron1961}.  
On the other hand, it established electronic pairing, i.e. the picture of a quantum-mechanical wave function with a defined phase as already described by Fritz London in 1937 \cite{londonNewConceptionSupraconductivity1937} as the microscopic mechanism behind SC.
This picture still holds today even for high \(T_C\)/unconventional superconductors, so SCs that cannot be described by BCS theory \cite{zhouHightemperatureSuperconductivity2021}.

\todo{Other pairing interactions can be taken, gives explanations for a lot of different SCs}

\subsection{Multiband BCS Mean Field Theory}\label{ssec:Multiband BCS Mean Field Theory}

The Hubbard model is the simplest model for interacting electron systems.
It goes back to works by Hubbard \cite{hubbardElectronCorrelationsNarrow1963}, Kanamori \cite{kanamoriElectronCorrelationFerromagnetism1963} and Gutzweiler \cite{gutzwillerEffectCorrelationFerromagnetism1963}.
\begin{equation}
	H_{\mathrm{int}} = U \sum_{i} c_{i, \uparrow}^{\dagger} c_{i, \downarrow}^{\dagger} c_{i, \downarrow} c_{i, \uparrow}
\end{equation}
where \(U > 0\).

Besides 

\cite{qinHubbardModelComputational2022}

\todo{Some relevance of the repulsive Hubbard model}

This simple Hubbard model can be extended in a multitude of ways to model a variety of physical system.
In this work: extension to multiple orbitals (i.e. atoms in the unit cell for lattice systems) and an attractive interaction, i.e. a negative \(U\).
Physical motivation for taking a negative-U Hubbard model: electrons can experience a local attraction interaction, for example through electrons coupling with phononic degrees of freedom or with electronic excitations that can be described as bosons \cite{micnasSuperconductivityNarrowbandSystems1990}.
\todo{There are some more specific papers to the specific mechanisms (and also some more mechanism), could cite these here and say some more things}
The form of the interaction term is then: \todo{Order of operators? -> also in all other equations!}
\begin{equation}
	H_{\mathrm{int}} = -\sum_{i, \alpha} U_{\alpha} c_{i, \alpha, \uparrow}^{\dagger} c_{i, \alpha, \downarrow}^{\dagger} c_{i, \alpha, \downarrow} c_{i, \alpha, \uparrow}
	\label{eq:Hubbard interaction multiband}
\end{equation}
where \(\alpha\) counts orbitals and the minus sign in front is taken so that \(U > 0\) now corresponds to an attractive interaction (this is purely convention).

There are a multitude of ways to derive a mean field description of a given interacting Hamiltonian.
Very rigorous in path integral formulations as saddle points, given for example in ref. \cite{colemanIntroductionManyBodyPhysics2015}.
The review follows \cite{huhtinenSuperconductivityNormalState2023}.
A more intuitive way based on ref. \cite{bruusManyBodyQuantumTheory2004} discussed here looks at the operators and which one are small. 

Look at interaction term \cref{eq:Hubbard interaction multiband}.
Mean-field approximation (here specifically for superconductivity i.e. pairing): operators do not deviate much from their average value, i.e. the deviation operators \todo{there are other combinations, talk about that}
\todo{deviations with small deltas}
\begin{align}
	d_{i, \alpha} = c_{i, \alpha, \uparrow}^{\dagger} c_{i, \alpha, \downarrow}^{\dagger} - \braket{c_{i, \alpha, \uparrow}^{\dagger} c_{i, \alpha, \downarrow}^{\dagger}} \\
	e_{i, \alpha} = c_{i, \alpha, \downarrow} c_{i, \alpha, \uparrow} - \braket{c_{i, \alpha, \downarrow} c_{i, \alpha, \uparrow}}
\end{align}
are small (dont contribute much to expectation values and correlation functions), so that in the interaction part of the Hamiltonian
\begin{align}
	H_{\mathrm{int}} &= -\sum_{i, \alpha} U_{\alpha} c_{i, \alpha, \uparrow}^{\dagger} c_{i, \alpha, \downarrow}^{\dagger} c_{i, \alpha, \downarrow} c_{i, \alpha, \uparrow} \\
	&= -\sum_{i, \alpha} U_{\alpha} 
	\left( d_{i, \alpha}^{\dagger} + \braket{c_{i, \alpha, \uparrow}^{\dagger} c_{i, \alpha, \downarrow}^{\dagger}} \right)
	\left( e_{i, \alpha} + \braket{c_{i, \alpha, \downarrow} c_{i, \alpha, \uparrow}} \right) \\
	&= -\sum_{i, \alpha} U_{\alpha} (
		d_{i, \alpha} e_{i, \alpha}
		+ d_{i, \alpha} \braket{c_{i, \alpha, \downarrow} c_{i, \alpha, \uparrow}}
		+ e_{i, \alpha} \braket{c_{i, \alpha, \uparrow}^{\dagger} c_{i, \alpha, \downarrow}^{\dagger}} \\
		&+ \braket{c_{i, \alpha, \uparrow}^{\dagger} c_{i, \alpha, \downarrow}^{\dagger}} \braket{c_{i, \alpha, \downarrow} c_{i, \alpha, \uparrow}} )
\end{align}
the first term is quadratic in the deviation and can be neglected.
Thus arrive at the approximation
\begin{align}
	H_{\mathrm{int}} &\approx -\sum_{i, \alpha} U_{\alpha} \left(
	d_{i, \alpha} \braket{c_{i, \alpha, \downarrow} c_{i, \alpha, \uparrow}}
	+ e_{i, \alpha} \braket{c_{i, \alpha, \uparrow}^{\dagger} c_{i, \alpha, \downarrow}^{\dagger}}
	+ \braket{c_{i, \alpha, \uparrow}^{\dagger} c_{i, \alpha, \downarrow}^{\dagger}} \braket{c_{i, \alpha, \downarrow} c_{i, \alpha, \uparrow}}
	\right) \\
	&= -\sum_{i, \alpha} U_{\alpha} (
		c_{i, \alpha, \uparrow}^{\dagger} c_{i, \alpha, \downarrow}^{\dagger} \braket{c_{i, \alpha, \downarrow} c_{i, \alpha, \uparrow}}
		+ c_{i, \alpha, \downarrow} c_{i, \alpha, \uparrow} \braket{c_{i, \alpha, \uparrow}^{\dagger} c_{i, \alpha, \downarrow}^{\dagger}} \\	
	&- \braket{c_{i, \alpha, \uparrow}^{\dagger} c_{i, \alpha, \downarrow}^{\dagger}} \braket{c_{i, \alpha, \downarrow} c_{i, \alpha, \uparrow}} ) \\
	&= \sum_{i, \alpha} (\Delta_{i, \alpha} c_{i, \alpha, \uparrow}^{\dagger} c_{i, \alpha, \downarrow}^{\dagger} + \Delta_{i, \alpha}^{*} c_{i, \alpha, \downarrow} c_{i, \alpha, \uparrow} - \frac{\vert \Delta_{i, \alpha} \vert^2}{U_{\alpha}})
\end{align}
with the expectation value
\begin{align}
	\Delta_{i, \alpha} = -U_{\alpha} \braket{c_{i, \alpha, \downarrow} c_{i, \alpha, \uparrow}}
\end{align}
which is called the superconducting gap.
Using the Fourier transform
\begin{equation}
	c_{i \alpha \sigma} = \frac{1}{\sqrt{N}} \sum_{\vb{k}} e^{\iu \vb{k} \vb{r}_{i \alpha}} c_{\vb{k} \alpha \sigma}
\end{equation}
can write \todo{Look at that Hamiltonian again, is that correct and can I write it better?}
\begin{align}
	H_{\mathrm{MF}} = \sum_{\vb{k} \alpha \beta \sigma} [H_{0, \sigma} (\vb{k})]_{\alpha \beta} c_{\vb{k} \alpha \sigma}^{\dagger} c_{\vb{k} \beta \sigma}
	-\mu \sum_{\vb{k} \alpha \sigma} n_{\vb{k} \alpha \sigma}
	+ \sum_{\alpha, \vb{k}} (\Delta_{\alpha} c_{\vb{k} \alpha \uparrow}^{\dagger} c_{-\vb{k} \alpha \downarrow}^{\dagger} + \Delta_{\alpha}^* c_{-\vb{k} \alpha \downarrow} c_{\vb{k} \alpha \uparrow})
	\label{eq:Mean Field Hamiltonian fourier transform}
\end{align}
To include finite momentum, take the ansatz of a Fulde-Ferrel (FF) type pairing \cite{kinnunenFuldeFerrellLarkin2018}: \todo{How to include finite momentum, rewrite equations} 
\begin{equation}
	\Delta
\end{equation}
The Hamiltonian in \cref{eq:Mean Field Hamiltonian fourier transform} can be writen as \todo{Get the remaining terms here}
\begin{align}
	H_{\mathrm{MF}} &= \sum_{\vb{k}} \vb{C}_{\vb{k}}^{\dagger} H_{\mathrm{BdG}} (\vb{k}) \vb{C}_{\vb{k}} \\
	C_{\vb{k}} &= 
		\begin{pmatrix}
			c_{\vb{k} 1 \uparrow} & 
			c_{\vb{k} 2 \uparrow} &
			\ldots &
			c_{\vb{k} n_{\mathrm{orb}} \uparrow} &
			c_{-\vb{k} 1 \downarrow}^{\dagger} &
			c_{-\vb{k} 2 \downarrow}^{\dagger} &
			\ldots &
			c_{-\vb{k} n_{\mathrm{orb}} \downarrow}^{\dagger}
		\end{pmatrix}^{T}
\end{align}
with the so-called Bogoliubov-de-Gennes matrix
\begin{equation}
	H_{\mathrm{BdG}} (\vb{k}) =
	\begin{pmatrix}
		H_{0, \uparrow} (\vb{k}) - \mu & \Delta \\
		\Delta^{\dagger} & - H_{0, \downarrow}^* (-\vb{k}) + \mu
	\end{pmatrix}
\end{equation}
with \(H_{0, \sigma}\) being the F.T. of the kinetic term and \(\Delta = \diag(\Delta_1, \Delta_2, \ldots, \Delta_{n_{\mathrm{orb}}})\).
Problem is  now reduced to diagonalization of the BdG matrix.
Write
\begin{equation}
	H_{\mathrm{BdG}} = U_{\vb{k}} \epsilon_{\vb{k}} U_{\vb{k}}^{\dagger}
\end{equation}
and 
\begin{equation}
	H_{\mathrm{MF}} = \sum_{\vb{k}} \gamma_{\vb{k}} \epsilon_{\vb{k}} \gamma_{\vb{k}}^{\dagger}
\end{equation}
with quasi-particle operators
\begin{equation}
	\gamma_{\vb{k}} = U_{\vb{k}}^{\dagger} c_{\vb{k}}
\end{equation}

\todo{Write indeces everywhere without comma}

Using the gap equation
\todo{gap equation}
\begin{equation}
	\Delta_{\alpha} = -U
\end{equation}
the order parameter can be determined self-consistently, i.e. starting from an initial value, the BdG matrix needs to be set up, diagonalized and then used to determine \(\Delta_{\alpha}\) again, until a converged value is found.

\todo{SC current in BCS}

\section{Dynamical Mean-Field Theory (DMFT)}\label{sec:Dynamical Mean-Field Theory}

\todo{Introduction DMFT, citing what has been achieved with it so far, what is the basic idea etc.}

\subsection{Green's Function Formalism}

\todo{Give an introduction }

Green's functions: method to encode influence of many-body effects on propagation of particles in a system.

Following~\cite{bruusManyBodyQuantumTheory2004}

\todo{Work over the paragraph}

\todo{Slim down to relevant information}

Have different kinds of Green's functions, for example the retarded Green's function (here with the \(k\) for lattice systems):
\begin{equation}
	G^R (\vb{k}, \sigma, \sigma^{\prime} t) = -\iu \Theta(t- t^{\prime}) \braket{ \{c_{\vb{k} \sigma} (t), c_{\vb{k} \sigma^{\prime}}^{\dagger} (0)\}}
\end{equation}
They give the amplitude of a particle inserted at momentum \(\vb{k}^{\prime}\) at time \(t^{\prime}\) to propagate to position \(\vb{k}\) at time \(t\).
Define Fourier-transform:
\begin{equation}
	G^R (\vb{k}, \sigma, \sigma^{\prime}, \omega) = \int_{-\infty}^{\infty} \odif{t} G^R (\vb{k}, \sigma, \sigma^{\prime} t)
\end{equation}
Can define the spectral function from this:
\begin{equation}
	A(\vb{k} \sigma, \omega) = -2 \Im G^R (\vb{k} \sigma, \omega)
\end{equation}
Looking at the diagonal elements of \(G^R\) here.
The spectral function can be thought of as the energy resolution of a particle with energy \(\omega\).
This mean, for non-interacting systems, the spectral function is a delta-function around the single-particle energies:
\begin{equation}
	A_0 (\vb{k} \sigma, \omega) = 2\pi \delta (\omega - \epsilon_{\vb{k} \sigma})
\end{equation}
For interacting systems this is not true, but \(A\) can still be peaked.

\todo{Show GFs can be related to observables}

A mathematical trick to calculate GFs in praxis is to introduce the imaginary time variable \(\tau\)
\begin{equation}
	t \to -\iu \tau
\end{equation}
where \(\tau\) is real and has the dimension time.
This enables the simultaneous expansion of exponential \(e^{-\beta H}\) coming from the thermodynamic average and \(e^{-\iu H t}\) coming from the time evolution of operators.
Define Matsubara GF \gls{matsubara correlation function}:
\todo{Introduce Greens functions instead of correlation functions}
\begin{equation}
	\mathcal{C}_{A B} (\tau, 0) = - \Braket{T_{\tau} (A(\tau) B(0))}
\end{equation}
with time-ordering operator in imaginary time:
\begin{equation}
	T_{\tau} (A(\tau) B(\tau^{\prime})) = \Theta(\tau - \tau^{\prime}) A(\tau) B(\tau^{\prime}) \pm \Theta(\tau^{\prime} - \tau) B(\tau^{\prime}) A(\tau)
\end{equation}
so that operators with later `times' go to the left.

Can prove from properties of Matsubara GF, that they are only defined for
\begin{equation}
	-\beta < \tau < \beta
\end{equation}
Due to this, the Fourier transform of the Matsubara GF is defined on discrete values:
\begin{equation}
	\mathcal{C}_{A B} (\iu \omega_n) = \int_{0}^{\beta} \odif{\tau}
\end{equation}
with fermionic/bosonic Matsubara frequencies
\begin{equation}
	\omega_n =
	\begin{cases}
		\frac{2n \pi}{\beta} \, \text{for bosons} \\
		\frac{(2n + 1)\pi}{\beta} \, \text{for fermions}
	\end{cases}
\end{equation}

It turns out that Matsubara GFs and retarded GFs can be generated from a common function \(\mathcal{C}_{AB} (z)\) that is defined on the entire complex plane except for the real axis.
So we can get the retarded GF \(\mathcal{C}_{AB}^R (\omega)\) by analytic continuation:
\begin{equation}
	\mathcal{C}_{AB}^R (\omega) = \mathcal{C}_{AB} (\iu \omega_n \to \omega + \iu \eta)
\end{equation}
\todo{What is the eta there -> need to define it in retarded GF}
So in particular the extrapolation of the Matsubara GF to zero is proportional to the density of states at the chemical potential.
Gapped: density is zero (Matsubara GF goes to 0), metal: density is finite (Matsubara GF goes to finite value).

\todo{single-particle Matsubara GF}

\subsection{Self Energy}

\todo{Short introduction to diagrams}

\todo{Self energy}

\todo{Dyson equation}

Dyson equation:
\begin{equation}
	\mathcal{G}_{\sigma} (\vb{k}, \iu \omega_n) = \frac{\mathcal{G}_{\sigma}^0 (\vb{k}, \iu \omega_n)}{1 - \mathcal{G}_{\sigma}^0 (\vb{k}, \iu \omega_n) \Sigma_{\sigma} (\vb{k}, \iu \omega_n)} = \frac{1}{\iu \omega_n - \xi_{\vb{k} - \Sigma_{\sigma} (\vb{k}, \iu \omega_n)}}
\end{equation}


\subsection{Nambu-Gorkov GF}

Introduction following~\cite[ch. 14.7]{colemanIntroductionManyBodyPhysics2015}

\todo{More general introduction into NG GFs, how they look like, what they describe etc.}

Order parameter can be chosen as the anomalous GF:
\begin{equation}
	\Psi = F^{\mathrm{loc}} (\tau = 0^-)
\end{equation}
or the superconducting gap
\begin{equation}
	\Delta = Z \Sigma^{\mathrm{AN}}
\end{equation}
that can be calculated from the anomalous self-energy \(\Sigma^{\mathrm{AN}}\) and quasiparticle weight \(Z\)
\todo{Sources for these?}

\todo{How to get quasiparticle weight?}

\subsection{DMFT}

Following \cite{georgesDynamicalMeanfieldTheory1996}.

Most general non-interacting electronic Hamiltonian in second quantization:
\begin{equation}
	H_0 = \sum_{i, j, \sigma}
\end{equation}
with lattice coordinates \(i, j\) and spin \(\sigma\).

One particle Green's function (many-body object, coming from the Hubbard model):
\begin{equation}
	G(\vb{k}, \iu \omega_n) = \frac{1}{\iu \omega_n + \mu - \epsilon_{\vb{k}} - \Sigma(\vb{k}, \iu \omega_n)}
\end{equation}
with the self energy \(\Sigma(\iu \omega_n)\) coming from the solution of the effect on-site problem:

The Dyson equation
\begin{equation}
	G(\vb{k}, \iu \omega_n) = \left( G_0 (\vb{k}, \iu \omega_n) - \Sigma(\vb{k}, \iu \omega_n)\right)^{-1}
\end{equation}
relates the non-interacting Greens function \(G_0 (\vb{k}, \iu \omega_n)\) and the fully-interacting Greens function \(G (\vb{k}, \iu \omega_n)\) (inversion of a matrix!).

\section{Quantum Metric}

First formulated in \cite{provostRiemannianStructureManifolds1980}

Following Cheng - a pedagogical Introduction \todo{See what is specific to this paper, see that I can derive that myself}

Parameter dependent Hamiltonian \(\{H(\lambda)\}\), smooth dependence on parameter \(\lambda = (\lambda_1, \lambda_2, \ldots) \in \mathcal{M}\) (base manifold)

Hamiltonian acts on parametrized Hilbert space \(\mathcal{H} (\lambda)\)

Eigenenergies \(E_n (\lambda)\), eigenstates \(\ket{\phi_n (\lambda)}\)

System state \(\ket{\psi (\lambda)}\) is linear combination of \(\ket{\psi_n (\lambda)}\) at every point in \(\mathcal{M}\)

Infinitesimal variation of the parameter \(\odif{\lambda}\) \todo{Dont get it here}:
\begin{equation}
	\odif{s^2} = \vert\vert \psi (\lambda + \odif{\lambda}) - \si (\lambda) \vert \vert^2 = \braket{\fdif{\psi} | \fdif{\psi}} = \braket{\pdif{\mu} \psi | \pdif{\nu} \psi} \odif{\lambda^{\mu}} \odif{\lambda^{\nu}} = (\gamma_{\mu \nu} + \iu \sigma_{\mu \nu}) \odif{\lambda^{\mu}} \odif{\lambda^{\nu}}
\end{equation}
Last part is splitting up into real and imaginary part

\subsection{Quantum Metric and superfluid weight}

\todo{Write up notes about quantum metric and superfluid weight}

\cite{peottaSuperfluidityTopologicallyNontrivial2015}

\end{document}
