\chapter{Green's Function Formalism}\label{ch:green's-function-formalism}

Source: \cite{Bruus_Flensberg_2004}

\todo{Definitions of GFs}

\todo{Imaginary time and frequency}

For doing calculations for finite temperatures: introduce imaginary time \gls{imaginary time}.

Most general non-interacting electronic Hamiltonian in second quantization:
\begin{equation}
    H_0 = \sum_{i, j, \sigma}
\end{equation}
with lattice coordinates \(i, j\) and spin \(\sigma\).

\todo{Show how Green's functions can be used to describe many-body effects -> Spectral function, self-energy}

From TRIQS tutorial:
As you can see, the behavior of the imaginary part is very different for the two values of $U$. When
$U$ is small, the system is a metal and the imaginary part extrapolated to zero goes to a finite value.
Instead, for large $U$, the system is a Mott insulator and the imaginary part goes to zero. The reason
is that the extrapolation to zero is directly proportional to the density of states at the chemical
potential. If the system is gapped, the density is zero; if the system is a metal, there is spectral
weight and the density is finite. Therefore, even on the Matsubara axis, one has a way to decide if the
system is metallic or not. \todo{Source for that connection between self energy and DOS}

One particle Green's function (many-body object, coming from the Hubbard model):
\begin{equation}
    G(\vb{k}, \iu \omega_n) = \frac{1}{\iu \omega_n + \mu - \epsilon_{\vb{k}} - \Sigma(\vb{k}, \iu \omega_n)}
\end{equation}
with the self energy \(\Sigma(\iu \omega_n)\) coming from the solution of the effect on-site problem:

The Dyson equation
\begin{equation}
    G(\vb{k}, \iu \omega_n) = \left( G_0 (\vb{k}, \iu \omega_n) - \Sigma(\vb{k}, \iu \omega_n)\right)^{-1}
\end{equation}
relates the non-interacting Greens function \(G_0 (\vb{k}, \iu \omega_n)\) and the fully-interacting Greens function \(G (\vb{k}, \iu \omega_n)\) (inversion of a matrix!).

\section{Nambu-Gorkov GF}

Order parameter can be chosen as the anomalous GF:
\begin{equation}
    \Psi = F^{\mathrm{loc}} (\tau = 0^-)
\end{equation}
or the superconducting gap
\begin{equation}
    \Delta = Z \Sigma^{\mathrm{AN}}
\end{equation}
that can be calculated from the anomalous self-energy \(\Sigma^{\mathrm{AN}}\) and quasiparticle weight \(Z\)
\todo{Sources for these?}

\todo{How to get quasiparticle weight?}

