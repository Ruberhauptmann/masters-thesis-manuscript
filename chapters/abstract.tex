\documentclass[../main.tex]{subfiles}
\graphicspath{{\subfix{../images/}}, {\subfix{../}}}

\begin{document}
	
\clearpage
\section*{Abstract}

This thesis investigates superconductivity in a system with flat electronic bands.
Systems like this potentially host superconductivity with high \(T_{\mathrm{C}}\).
To fully classify the superconducting state, the coherence length \(\xi_0\) and the London penetration depth \(\lambda_{\mathrm{L}, 0}\) are calculated by extracting information on the length scales from the breakdown of the superconducting order parameter when introducing a finite momentum.

A class of systems attracting significant recent interest are graphene-based system that host flat band due to specific structural configurations.
The model investigated in this thesis is a conceptually simple model of a flat band hybridized with graphene that can be realized in two-dimensional adatom heterostructures.

The \gls{fmp} method is applied to both this decorated graphene model and a one-band Hubbard model with a local attractive interaction to calculate the superconducting length scales.

\glsresetall
\section*{Kurzzusammenfassung}

In dieser Arbeit wird die Supraleitung in einem System mit flachen elektronischen Bändern untersucht.
Solche Systeme zeigen potenziell Supraleitfähigkeit mit hoher \(T_{\mathrm{C}}\).
Um den supraleitenden Zustand vollständig zu klassifizieren werden die Kohärenzlänge \(\xi_0\) und die London-Eindringtiefe \(\lambda_{\mathrm{L}, 0}\) berechnet.
Diese werden extrahiert aus der Unterdrückung des Ordnungsparameters bei Einführung eines endlichen Impulses extrahiert werden.

Eine Klasse von Systemen, die in letzter Zeit auf großes Interesse stoßen, Graphen-basierte Systeme die aufgrund spezifischer struktureller Konfigurationen flache Bänder zeigen.
Das in dieser Arbeit untersuchte Modell ist ein konzeptionell einfaches Modell eines flachen Bandes das mit den dispersiven Graphen-Bändern hybridisiert, welches in zweidimensionalen Adatom-Heterostrukturen realisiert werden kann.

Die \gls{fmp}-Methode wird sowohl auf dieses dekorierte Graphen-Modell als auch auf ein Ein-Band-Hubbard-Modell mit einer lokalen anziehenden Wechselwirkung angewandt, um die supraleitenden Längenskalen zu berechnen.

%Diese Arbeit untersucht Supraleitung mit einem besonderen Fokus auf die Kohärenzlänge (die die Ausdehnung von Cooper-Paaren beschreibt) und die London’sche Eindringtiefe (die angibt, wie weit Magnetfelder in ein Material eindringen können). Diese Längenskalen sind eng mit den Materialeigenschaften im supraleitenden Zustand verknüpft.

%Die Methode basiert auf der Ginzburg-Landau-Theorie, wobei dem Ordnungsparameter ein endlicher Impuls zugewiesen wird. Um diese Längenskalen in mikroskopischen Theorien zu berechnen, muss dieser endliche Impuls in das jeweilige theoretische Rahmenwerk eingeführt werden. In der theoretischen Einleitung wird dieser Zugang sowohl in \gls{bcs}-Theorie als auch in 
%Dynamische Molekularfeld-Theorie demonstriert.

%In jüngster Zeit haben Systeme mit flachen elektronischen Bändern an Bedeutung gewohnen, insbesondere Graphen-basierte Systeme, in denen flache Bänder aufgrund spezieller struktureller Eigenschaften auftreten. Ein in dieser Arbeit behandeltes Beispiel besteht aus einer Schicht von Gruppe-IV-Atomen zwischen einer Graphenlage und einem \ce{SiC}-Substrat. Ein Minimalmodell für dieses System ist eine modifizierte Graphenstruktur mit einem zusätzlichen Atom in der Einheitszelle, das das flache Band stellt.

%Die Methode der Paarbildung mit endlichem Impuls wird sowohl auf dieses dekorierte Graphenmodell als auch auf ein Ein-Band-Hubbard-Modell mit einer einfachen lokalen attraktiven Wechselwirkung angewendet. Die supraleitenden Längenskalen werden berechnet und zwischen den \gls{bcs}- und \gls{dmft}-Ansätzen verglichen.

%Die Arbeit schließt mit einer Diskussion über mögliche Anwendungen der Methode auf realistischere Versionen des dekorierten Graphenmodells sowie über die Perspektiven einer vollständigen \gls{dmft}-Behandlung in diesem Kontext.


\end{document}
