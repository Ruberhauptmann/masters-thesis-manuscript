\documentclass[../main.tex]{subfiles}
\graphicspath{{\subfix{../images/}}, {\subfix{../}}}

\begin{document}

\chapter*{Abstract}

This thesis investigates superconductivity in system with flat electronic bands, which support high \(T_{\mathrm{C}}\) superconductivity.


With a particular focus on the coherence length (describing the size of Cooper pairs) and the London penetration depth (describing the distance magnetic fields can penetrate into the material).
These length scales are connected to the properties of the material when in the superconducting state.



The method to calculate the superconducting length scales works by extracting information on the length scales from the breakdown of the order parameter when introducing a finite momentum.

A class of systems attracting significant recent interest are graphene based system that host flat band due to specific structural configurations.
The model investigated in this thesis is a conceptually simple model of a flat band hybridized with graphene that can be realized in two-dimensional adatom hetero structures.

The finite-momentum pairing method is applied to both this decorated graphene model and a one-band Hubbard model with a simple local attractive interaction.
The superconducting length scales are calculated and compared between the \gls{bcs} and \gls{dmft} frameworks.

\glsresetall
\chapter*{Kurzzusammenfassung}

\todo{Check translation}

Diese Arbeit untersucht Supraleitung mit einem besonderen Fokus auf die Kohärenzlänge (die die Ausdehnung von Cooper-Paaren beschreibt) und die London’sche Eindringtiefe (die angibt, wie weit Magnetfelder in ein Material eindringen können). Diese Längenskalen sind eng mit den Materialeigenschaften im supraleitenden Zustand verknüpft.

Die Methode basiert auf der Ginzburg-Landau-Theorie, wobei dem Ordnungsparameter ein endlicher Impuls zugewiesen wird. Um diese Längenskalen in mikroskopischen Theorien zu berechnen, muss dieser endliche Impuls in das jeweilige theoretische Rahmenwerk eingeführt werden. In der theoretischen Einleitung wird dieser Zugang sowohl in \gls{bcs}-Theorie als auch in 
Dynamische Molekularfeld-Theorie demonstriert.

In jüngster Zeit haben Systeme mit flachen elektronischen Bändern an Bedeutung gewohnen, insbesondere Graphen-basierte Systeme, in denen flache Bänder aufgrund spezieller struktureller Eigenschaften auftreten. Ein in dieser Arbeit behandeltes Beispiel besteht aus einer Schicht von Gruppe-IV-Atomen zwischen einer Graphenlage und einem \ce{SiC}-Substrat. Ein Minimalmodell für dieses System ist eine modifizierte Graphenstruktur mit einem zusätzlichen Atom in der Einheitszelle, das das flache Band stellt.

Die Methode der Paarbildung mit endlichem Impuls wird sowohl auf dieses dekorierte Graphenmodell als auch auf ein Ein-Band-Hubbard-Modell mit einer einfachen lokalen attraktiven Wechselwirkung angewendet. Die supraleitenden Längenskalen werden berechnet und zwischen den \gls{bcs}- und \gls{dmft}-Ansätzen verglichen.

Die Arbeit schließt mit einer Diskussion über mögliche Anwendungen der Methode auf realistischere Versionen des dekorierten Graphenmodells sowie über die Perspektiven einer vollständigen \gls{dmft}-Behandlung in diesem Kontext.


\end{document}
