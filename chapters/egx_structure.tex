\documentclass[../main.tex]{subfiles}
\graphicspath{{\subfix{../images/}}, {\subfix{../}}}

\begin{document}
\chapter{EG-X Hamiltonian in Reciprocal Space}\label{ch:EG-X Hamiltonian in Reciprocal Space}

In the following chapter, the model Hamiltonian
\begin{align}
	H_0 = &-t_{\mathrm{X}} \sum_{\langle ij \rangle, \sigma} d_{i, \sigma}^{\dagger} d_{j, \sigma}
	-t_{\mathrm{Gr}} \sum_{\langle ij \rangle, \sigma} \left(
	c_{i, \sigma}^{(A), \dagger} c_{j, \sigma}^{(B)} +
	c_{j, \sigma^{\prime}}^{(B), \dagger} c_{i, \sigma}^{(A)} \right) \notag \\
	&+ V \sum_{i, \sigma} d_{i, \sigma}^{\dagger} c_{i, \sigma}^{(A)} + \mathrm{h.c.} \label{eq:EG-X model Hamiltonian non-interacting appendix}
\end{align}
will be treated via Fourier transform.

The first step is to write out the sums over nearest neighbors \(\langle i, j \rangle\) explicitly, writing \(\vb{\delta}_{\mathrm{X}}, \vb{\delta}_{\epsilon}\) (\(\epsilon = A, B\)) for the vectors to the nearest neighbors of the \(\mathrm{X}\) atoms and Graphene \(A, B\) sites.
Doing the calculation for example of the \(\mathrm{X}\) atoms:
\begin{align}
	-t_{\mathrm{X}} \sum_{\langle ij \rangle, \sigma} (d_{i, \sigma}^{\dagger} d_{j, \sigma} + d_{j, \sigma}^{\dagger} d_{i, \sigma})
	&= -\frac{t_X}{2} \sum_{i,\sigma} \sum_{\vb{\delta}_{\mathrm{X}}} d_{i, \sigma}^{\dagger} d_{i + \vb{\delta}_{\mathrm{X}}, \sigma}
	-\frac{t_X}{2} \sum_{j,\sigma} \sum_{\vb{\delta}_{\mathrm{X}}} d_{j, \sigma}^{\dagger} d_{j + \vb{\delta}_{\mathrm{X}}, \sigma} \label{eq:EG-X model X atoms nearest neighbor sum double counting} \\
	&= -t_X \sum_{i,\sigma} \sum_{\vb{\delta}_{\mathrm{X}}} d_{i, \sigma}^{\dagger} d_{i + \vb{\delta}_{\mathrm{X}}, \sigma}
	\label{eq:EG-X model X atoms nearest neighbours written out}
\end{align}
The factor \(\nicefrac{1}{2}\) in \cref{eq:EG-X model X atoms nearest neighbor sum double counting} is to account for double counting when going to the sum over all lattice sites \(i\).
By relabeling \(j \to i\) in the second sum, we then get \cref{eq:EG-X model X atoms nearest neighbours written out}.
Using now the discrete Fourier transform
\begin{align}
	c_{i} &= \frac{1}{\sqrt{N}} \sum_{\vb{k}} e^{\iu \vb{k} \vb{r}_{i}} c_{\vb{k}} \notag\\
	c_{i}^{\dagger} &= \frac{1}{\sqrt{N}} \sum_{\vb{k}} e^{-\iu \vb{k} \vb{r}_{i}} c_{\vb{k}}^{\dagger}
\end{align}
with the completeness relation:
\begin{equation}
	\sum_{i} e^{\iu \vb{k} \vb{r}_{i}} e^{-\iu \vb{k}^{\prime} \vb{r}_{i}} = N \delta_{\vb{k}, \vb{k}^{\prime}}
	\;,
\end{equation}
\cref{eq:EG-X model X atoms nearest neighbours written out} reads:
\begin{align}
	-t_X \frac{1}{N} \sum_{i,\sigma} \sum_{\vb{\delta}_{\mathrm{X}}} d_{i, \sigma}^{\dagger} d_{i + \vb{\delta}_{\mathrm{X}}, \sigma^{\prime}}
	&= -t_X \frac{1}{N} \sum_{i,\sigma, \sigma^{\prime}} \sum_{\vb{\delta}_{\mathrm{X}}} \sum_{\vb{k}, \vb{k}^{\prime}} e^{-\iu \vb{k} \vb{r}_i} d_{\vb{k}, \sigma}^{\dagger} e^{\iu \vb{k}^{\prime} \vb{r}_i} e^{\iu \vb{k}^{\prime} \vb{\delta}_{\mathrm{X}}} d_{\vb{k}^{\prime}, \sigma^{\prime}} \\
	&= -t_X \frac{1}{N} \sum_{\vb{k}, \vb{k^{\prime}}, \sigma, \sigma^{\prime}} \sum_{\vb{\delta}_{\mathrm{X}}} d_{\vb{k}, \sigma}^{\dagger}  e^{\iu \vb{k}^{\prime} \vb{\delta}_{\mathrm{X}}} d_{\vb{k}^{\prime}, \sigma^{\prime}} \sum_{i} e^{-\iu \vb{k} \vb{r}_i} e^{\iu \vb{k}^{\prime} \vb{r}_i} \\
	&= -t_X \frac{1}{N} \sum_{\vb{k}, \vb{k^{\prime}}, \sigma, \sigma^{\prime}} \sum_{\vb{\delta}_{\mathrm{X}}} d_{\vb{k}, \sigma}^{\dagger}  e^{\iu \vb{k}^{\prime} \vb{\delta}_{\mathrm{X}}} d_{\vb{k}^{\prime}, \sigma^{\prime}} N \delta_{\vb{k}, \vb{k}^{\prime}} \\
	&= -t_X \sum_{\vb{k}, \sigma, \sigma^{\prime}}  d_{\vb{k}, \sigma}^{\dagger}d_{\vb{k}, \sigma^{\prime}} \sum_{\vb{\delta}_{\mathrm{X}}} e^{\iu \vb{k} \vb{\delta}_{\mathrm{X}}}
\end{align}


The nearest neighbours for \(\mathrm{X}\) atoms are the vectors \(\vb{\delta}_{AA, i}\) from section~\ref{sec:lattice-structure-of-graphene}.
With that, we can calculate:
\begin{align}
	f_{\mathrm{X}} (\vb{k}) &= -t_X \sum_{\vb{\delta}_{\mathrm{X}}} e^{\iu \vb{k} \vb{\delta}_{\mathrm{X}}} \\
	&= -t_X \left( e^{\iu a (\frac{k_x}{2} + \frac{\sqrt{3} k_y}{2})}
	+ e^{\iu a k_x}
	+ e^{\iu a (\frac{k_x}{2} - \frac{\sqrt{3} k_y}{2})}
	\right. \\
	&+ \left. e^{\iu a (-\frac{k_x}{2} - \frac{\sqrt{3} k_y}{2})}
	+ e^{-\iu a k_x}
	+ e^{\iu a (-\frac{k_x}{2} + \frac{\sqrt{3} k_y}{2})} \right) \\
	&= -t_X \left( 2 \cos{(a k_x)} + 2 e^{\iu a \frac{\sqrt{3} k_y}{2}} \cos{(\frac{a}{2} k_x)} + 2 e^{-\iu a \frac{\sqrt{3} k_y}{2}} \cos{(\frac{a}{2} k_x)} \right) \\
	&= -2t_X \left( \cos{(a k_x)} + 2 \cos{(\frac{a}{2} k_x)} \cos{(\sqrt{3} \frac{ a}{2} k_y)} \right)
\end{align}
We can do the same for the hopping between Graphene sites, for example :
\begin{align}
	-t_{\mathrm{Gr}} \sum_{\langle ij \rangle, \sigma \sigma^{\prime}} c_{i, \sigma}^{(A), \dagger} c_{j, \sigma^{\prime}}^{(B)}
	&= -t_{\mathrm{Gr}} \sum_{i, \sigma \sigma^{\prime}} \sum_{\vb{\delta}_{AB}} c_{i, \sigma}^{(A), \dagger} c_{i + \vb{\delta}_{AB} , \sigma^{\prime}}^{(B)} \\
	&= -t_{\mathrm{Gr}} \sum_{\vb{k}, \sigma, \sigma^{\prime}}  c_{\vb{k}, \sigma}^{(A) \dagger} c_{\vb{k}, \sigma^{\prime}}^{(B)} \sum_{\vb{\delta}_{AB}} e^{\iu \vb{k} \vb{\delta}_{AB}}
\end{align}
We note
\begin{align}
	\sum_{\vb{\delta}_{AB}} e^{\iu \vb{k} \vb{\delta}_{AB}} = \left( \sum_{\vb{\delta}_{BA}} e^{\iu \vb{k} \vb{\delta}_{BA}} \right)^* = \sum_{\vb{\delta}_{BA}} e^{-\iu \vb{k} \vb{\delta}_{BA}}
\end{align}
and calculate
\begin{align}
	f_{Gr} &= -t_{Gr} \sum_{\vb{\delta}_{AB}} e^{\iu \vb{k} \vb{\delta}_{AB}} \\
	&= -t_{Gr} \left(
	e^{\iu \frac{a}{\sqrt{3}} k_y} +
	e^{\iu \frac{a}{2\sqrt{3}} (\sqrt{3} k_x - k_y)} +
	e^{\iu \frac{a}{2\sqrt{3}} (-\sqrt{3} k_x - k_y)} \right) \\
	&= -t_{Gr} \left(
	e^{\iu \frac{a}{\sqrt{3}} k_y} +
	e^{-\iu \frac{a}{2\sqrt{3}} k_y} \left(
	e^{\iu \frac{a}{2} k_x} + e^{-\iu \frac{a}{2} k_x}
	\right) \right) \\
	&= -t_{Gr} \left(
	e^{\iu \frac{a}{\sqrt{3}} k_y} +
	2 e^{-\iu \frac{a}{2\sqrt{3}} k_y}
	\cos{(\frac{a}{2} k_x)} \right)
\end{align}
All together, we get:
\begin{align}
	H_0 &= \sum_{\vb{k}, \sigma, \sigma^{\prime}} \begin{pmatrix} c_{k, \sigma}^{A, \dagger} & c_{k, \sigma}^{B, \dagger} & d_{k, \sigma}^{\dagger} \end{pmatrix}
	\begin{pmatrix}
		0 & f_{Gr} & V \\
		f_{Gr}^* & 0 & 0 \\
		V & 0 & f_{X}
	\end{pmatrix} \begin{pmatrix} c_{k, \sigma}^{A} \\ c_{k, \sigma}^{B} \\ d_{k, \sigma} \end{pmatrix}
	\label{eq:EG-X Hamiltonian non-interacting matrix appendix}
\end{align}


\end{document}