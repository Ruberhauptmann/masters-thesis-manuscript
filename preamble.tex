% For fixing errors with the scrbook class
\usepackage{scrhack}

%%%%%%%%%%%%%%%%%%%%%%%%%%%%%%
% Language and font settings
%%%%%%%%%%%%%%%%%%%%%%%%%%%%%%
\usepackage[english]{babel} % set document language to English

% If the document is not compiled with XeLaTeX, we need to enable input and output for special characters
\usepackage{ifxetex}
\ifxetex
    \usepackage{fontspec} % enables selection of fonts in XeLaTeX
\else
    \usepackage[T1]{fontenc} % to encode glyphs like ä,ö,ü in the output font
    \usepackage{lmodern} % use latin modern font (its prettier!)
\fi

%%%%%%%%%%%%%%%%%%%%%%%%%%%%%%
% Setup various aspects of the layout
%%%%%%%%%%%%%%%%%%%%%%%%%%%%%%
\renewcommand{\thechapter}{\Roman{chapter}} % chapters with roman numerals (I, II, III, IV, ...)

\setlength{\parindent}{0em} % no indentation at the start of paragraphs
\setlength{\parskip}{0.9ex} % create a little distance between paragraphs
\setlength{\fboxsep}{0.6em} % create a bit more distance between an fbox and the text in it

\usepackage{uni-titlepage} % create titlepages in KOMA

%%%%%%%%%%%%%%%%%%%%%%%%%%%%%%
% Define header and footer
%%%%%%%%%%%%%%%%%%%%%%%%%%%%%%
\usepackage{scrlayer-scrpage}
\clearpairofpagestyles % clear default page style
\ihead[]{\headmark} % chapter/section title on the inner side
\ohead[]{\pagemark} % page number on the outer side
\cfoot[\pagemark]{} % pagenumber in the center footer for the 'plain' pagestyle (per default, this gets used for pages on the beginning of chapters)

%%%%%%%%%%%%%%%%%%%%%%%%%%%%%%
% Some automatic enhancements for the documents
%%%%%%%%%%%%%%%%%%%%%%%%%%%%%%
\usepackage{microtype} % enables microtype refinements, makes the text body look prettier overall
\usepackage[defaultlines=2, all]{nowidow} % avoids widows (single lines at the top of a page) and orphans (single lines at the bottom of a page)

%%%%%%%%%%%%%%%%%%%%%%%%%%%%%%
% Add commands to fine tune some aspects of the document
%%%%%%%%%%%%%%%%%%%%%%%%%%%%%%
\usepackage{ragged2e} % gives commands \Centering, \RaggedRight, \RaggedLeft (and corresponding environments), which support hyphenation and thus look prettier
\usepackage{enumitem} % provides user control over the layout of the three basic list environments

\usepackage{perpage} % provides a mechanism to reset counters per page and/or keep their occurences sorted in order of appearance on the page

%%%%%%%%%%%%%%%%%%%%%%%%%%%%%%
% Setup maths and physics things
%%%%%%%%%%%%%%%%%%%%%%%%%%%%%%
\usepackage{amssymb} % provides a number of symbols
\usepackage{amsmath} % provides enhancements for documents with mathematical formulas, e.g. the align environment
\usepackage{amsfonts} % provides a font with all kinds of mathematical symbols
\usepackage{amsthm} % provides a possibility to define theorem environments

\usepackage{mathtools} % provide several tools for math typesetting as an extension to amsmath
\usepackage{nicefrac} % provides a tool for typesetting inline fractions as a/b
\usepackage{bm} % provides the \bm command to typeset any symbol in bold
\usepackage{upgreek} % provides non-slanted greek letters

\usepackage{tensor} % provides typesetting for tensors
\usepackage{braket} % provides typesetting for braket notation
\usepackage{siunitx} % support to typeset units
\sisetup{
    range-phrase=-,
    range-units=single
} % options for typesetting things like 1-5cm
\usepackage{derivative} % for typesetting derivatives and differentials

\numberwithin{equation}{chapter} % number equations as <chapter>.<equationnumber>

%%%%%%%%%%%%%%%%%%%%%%%%%%%%%%
% Setup inclusion of graphics
%%%%%%%%%%%%%%%%%%%%%%%%%%%%%%
\usepackage{tikz} % general graphics package for LaTeX

\usepackage{pgfplots} % package to draw axes and labeled plots
\pgfplotsset{compat=1.18} % recommended by pgfplots: set compatibility to the newest version

\usepackage{graphicx} % gives \includegraphics command
\usepackage{epsfig} % use eps files in figures
\graphicspath{images/} % setup path for inclusion of images

\usepackage[export]{adjustbox} % give options to \includegraphics to align images (for the two logos in the titlepage)

%%%%%%%%%%%%%%%%%%%%%%%%%%%%%%
% Bibliography
%%%%%%%%%%%%%%%%%%%%%%%%%%%%%%
\usepackage[
    style=numeric-comp,
    sorting=none,
    giveninits=true
]{biblatex} % use biblatex with a numeric (compact) style, given in the order of citation and abbreviation of given names
\addbibresource{bibliography.bib} % include the bibliography (compatible to subfiles!)
\renewbibmacro*{doi+eprint+url}{
    \printfield{doi}
    \newunit\newblock
    \iffieldundef{doi}{
        \usebibmacro{eprint}
        \iffieldundef{eprint}{\usebibmacro{url+urldate}}{}%
    }{}
}% only print URL if there is no DOI

%%%%%%%%%%%%%%%%%%%%%%%%%%%%%%
% Color definitions
%%%%%%%%%%%%%%%%%%%%%%%%%%%%%%
\usepackage{xcolor} % to define and use colors

% Define some custom colors
\definecolor{myred}{HTML}{A3061E}
\definecolor{myblue}{HTML}{003F77}
\definecolor{myyellow}{HTML}{FFBC42}
\definecolor{mygreen}{HTML}{0B6E4F}
\colorlet{myorange}{myyellow!60!myred}
\colorlet{myviolett}{myred!50!myblue!80}

% Define colors from the UHH branding
\definecolor{UHHred}{HTML}{E2001A}
\definecolor{UHHblue}{HTML}{0271BB}
\definecolor{UHHblack}{HTML}{000000}
\definecolor{UHHgray}{HTML}{3B515B}

%%%%%%%%%%%%%%%%%%%%%%%%%%%%%%
% Define environments for code listings
%%%%%%%%%%%%%%%%%%%%%%%%%%%%%%
\usepackage{listings} % to use and define code listings
\lstdefinestyle{python}{
    language=Python,
	basicstyle=\ttfamily,
	keywordstyle=\color{myred},
	identifierstyle=\color{myblue},
	stringstyle=\color{mygreen},
	commentstyle=\color{black!50},
	numberstyle=\color{black!50}\tiny,
	numbers=left,
	belowcaptionskip=\baselineskip,
} % code environment for python code

%%%%%%%%%%%%%%%%%%%%%%%%%%%%%%
% Setup of floats and captions
%%%%%%%%%%%%%%%%%%%%%%%%%%%%%%
\usepackage{caption} % allows control of caption for float environments
\captionsetup{
    font=small,
    format=plain,
    labelfont=bf,
    labelsep=colon,
    margin=10pt,
    textfont=sl,
    singlelinecheck=true,
} % setup the caption for floats

\usepackage{booktabs} % package to typeset prettier tables

%%%%%%%%%%%%%%%%%%%%%%%%%%%%%%
% Things for referencing
%%%%%%%%%%%%%%%%%%%%%%%%%%%%%%
\usepackage{hyperref} % enables use of hyperlinks
\hypersetup{
    linkcolor = UHHblue,
    citecolor  = purple,
    urlcolor   = myblue,
    colorlinks = true,
} % define colors for links

\usepackage{csquotes} % provides \enquote command to do quotes automatically
\usepackage{cleveref} % provides \cref (and \Cref) command, to automatically write out references, depending on the type of reference (figure, equation, etc.)
% must be loaded after hyperref!


%%%%%%%%%%%%%%%%%%%%%%%%%%%%%%
% Setup of epigraphs
%%%%%%%%%%%%%%%%%%%%%%%%%%%%%%
\usepackage{epigraph} % package to use the \epigraph command
\renewcommand{\epigraphsize}{\small} % set up font size for epigraphs
\setlength{\epigraphwidth}{0.35\textwidth} % set up width of epigraphs
\renewcommand{\textflush}{flushright} % flush epigraph text to the right
\renewcommand{\sourceflush}{flushright} % flush sources of epigraphs to the right

%%%%%%%%%%%%%%%%%%%%%%%%%%%%%%
% Setup of the glossary
%%%%%%%%%%%%%%%%%%%%%%%%%%%%%%
\usepackage[symbols,nogroupskip,sort=use,indexonlyfirst]{glossaries-extra}
\makenoidxglossaries
\newglossarystyle{symbolstyle}{%
 \setglossarystyle{long3col}% base it on the long3col style
  \renewenvironment{theglossary}%
    {\begin{longtable}{@{\extracolsep{\fill}}llc}}%
    {\end{longtable}}%
 \renewcommand*{\glossaryheader}{%
  \bfseries Symbol&\bfseries Meaning&
  \bfseries Definition\tabularnewline\endhead}%
  \renewcommand{\glossentry}[2]{%
    \glsentryitem{##1}\glstarget{##1}{\glossentryname{##1}} &
    \glossentrydesc{##1} \glsentryuseri{##1} & ##2\tabularnewline
  }%
}

%%%%%%%%%%%%%%%%%%%%%%%%%%%%%%
% To put in todos
%%%%%%%%%%%%%%%%%%%%%%%%%%%%%%
\usepackage[
    linecolor=myyellow,
    backgroundcolor=myyellow,
    %disable
]{todonotes}
